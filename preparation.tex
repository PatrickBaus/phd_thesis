% FIXME: Change variable names according to Noise Sources in Bulk CMOS
\chapter{Preparation}
\begin{chapquote}{Lewis Carroll, \textit{Alice in Wonderland}}
``Begin at the beginning,'' the King said gravely, ``and go on till you come to the end: then stop.''
\end{chapquote}
\section{Grounding and Shielding}
Add parts from "references\\Grounding and Shielding.pdf"

\section{Laser System}
%Explanations exist; they have existed for all time; there is always a well-known solution to every human problem — neat, plausible, and wrong.
\subsection{Requirements Laser System}
One purpose of the laser system is to be used for the spectroscopy of highly charged ions in a Penning trap. For example, an interesting transition of \ce{Ar^13+} can be found at $\lambda = \qty{441.25575(17)}{\nm}$ \cite{ar13+_wavelength} with a lifetime of \qty{9.573(6)}{\ms} \cite{ar13+_lifetime}, which corresponds to natural linewidth of $\Gamma \approx 2 \pi \times \qty{16.63(1)}{\Hz}$. While this linewidth is fairly small, there is substantial doppler broadening at \qty{4}{\K} of
\begin{equation}
    \Delta \nu (\lambda = \qty{441}{\nm}, T=\qty{4}{\K}, m=\qty{39.948}{u}) = \frac{2}{\lambda}\sqrt{2 \ln 2 \frac{k_B T}{m}} \approx 2 \pi \times \qty{150}{\MHz} \, . \label{eqn:doppler_broadening}
\end{equation}

\clearpage
\section{Laser Current Driver}
\label{sec:laser_current_driver}
% Include Emission wavelength dependence of characteristic temperature of InGaN laser diodes
% Check Diode Laser Characteristics
% I-lamda in Wavelength Dependence of InGaN Laser Diode Characteristics
% also Determination of piezoelectric fields in strained GaInN quantum wells using the quantum-confined Stark effect
Laser diodes are current driven devices, because
\begin{equation}
    P_{out} \propto I\,, \nonumber
\end{equation}

and the diode current $I$ approximately follows the Shockley equation \cite{shockley_diode}
\begin{equation}
    I = I_0 \left( e^{\frac{qV_d}{k_B T}} - 1\right) \, .
\end{equation}

$k_B$ is Boltzmann constant, $T$ the temperature, $q$ the electron charge and $V_d$ the diode voltage. The exponential dependence of the current on the supply voltage calls for a current source to drive a laser diode safely, without risking thermal damage.

The primary function of a laser driver is to provide a stable, but user adjustable, current. The user adjustable current can typically be modulated at frequencies up to serveral \unit{\MHz} to shape the frequency and ampltitude of the laser beam. Additional features like current and voltage limits aid in protecting the expensive laser diodes. This section deals with the design challenges of such a device used for high precission laser spectroscopy. First the design requirements are derived and then technical specifications are developed.

The focus of this work many lies on two types of laser diodes, indium gallium nitride (InGaN) and aluminium gallium arsenide (AlGaAs), but is not limited to those two types. The former material is used for blue laser didoes at around \qty{450}{\nm}, but also cover up to green wavelengths, and the latter for near-infrared laser diodes at \qty{780}{\nm}, both wavelengths used for experiments in this group.

The design requirements are split into four parts, that need to be discussed. The ambient environment, the diode voltage and current requirements, the modulation bandwith and finally the noise specifications.

\clearpage
\subsection{Design Goals: Ambient Environment}
The laser driver is to be used in a clean laboratory environment. Typical lab temperatures are in the range of \qtyrange{20}{25}{\celsius} and ware mostly met in our labs, before improvements were implemented as part of this work. Humidity is only controlled with dehumidifiers and therefore in the range of \qtyrange{15}{60}{\percent rH}. The air is typically filtered using H14 HEPA filters. Figure \ref{fig:lab_temperature_start_of_project} shows a typical \qty{1}{d} span of the lab temperature as it was found at the start of this project.

\begin{figure}[ht]
    \centering
    %% Creator: Matplotlib, PGF backend
%%
%% To include the figure in your LaTeX document, write
%%   \input{<filename>.pgf}
%%
%% Make sure the required packages are loaded in your preamble
%%   \usepackage{pgf}
%%
%% Also ensure that all the required font packages are loaded; for instance,
%% the lmodern package is sometimes necessary when using math font.
%%   \usepackage{lmodern}
%%
%% Figures using additional raster images can only be included by \input if
%% they are in the same directory as the main LaTeX file. For loading figures
%% from other directories you can use the `import` package
%%   \usepackage{import}
%%
%% and then include the figures with
%%   \import{<path to file>}{<filename>.pgf}
%%
%% Matplotlib used the following preamble
%%   \usepackage{siunitx}
%%   \usepackage{fontspec}
%%
\begingroup%
\makeatletter%
\begin{pgfpicture}%
\pgfpathrectangle{\pgfpointorigin}{\pgfqpoint{5.208662in}{3.219130in}}%
\pgfusepath{use as bounding box, clip}%
\begin{pgfscope}%
\pgfsetbuttcap%
\pgfsetmiterjoin%
\definecolor{currentfill}{rgb}{1.000000,1.000000,1.000000}%
\pgfsetfillcolor{currentfill}%
\pgfsetlinewidth{0.000000pt}%
\definecolor{currentstroke}{rgb}{1.000000,1.000000,1.000000}%
\pgfsetstrokecolor{currentstroke}%
\pgfsetdash{}{0pt}%
\pgfpathmoveto{\pgfqpoint{0.000000in}{0.000000in}}%
\pgfpathlineto{\pgfqpoint{5.208662in}{0.000000in}}%
\pgfpathlineto{\pgfqpoint{5.208662in}{3.219130in}}%
\pgfpathlineto{\pgfqpoint{0.000000in}{3.219130in}}%
\pgfpathlineto{\pgfqpoint{0.000000in}{0.000000in}}%
\pgfpathclose%
\pgfusepath{fill}%
\end{pgfscope}%
\begin{pgfscope}%
\pgfsetbuttcap%
\pgfsetmiterjoin%
\definecolor{currentfill}{rgb}{1.000000,1.000000,1.000000}%
\pgfsetfillcolor{currentfill}%
\pgfsetlinewidth{0.000000pt}%
\definecolor{currentstroke}{rgb}{0.000000,0.000000,0.000000}%
\pgfsetstrokecolor{currentstroke}%
\pgfsetstrokeopacity{0.000000}%
\pgfsetdash{}{0pt}%
\pgfpathmoveto{\pgfqpoint{0.693677in}{0.539544in}}%
\pgfpathlineto{\pgfqpoint{5.058662in}{0.539544in}}%
\pgfpathlineto{\pgfqpoint{5.058662in}{3.053228in}}%
\pgfpathlineto{\pgfqpoint{0.693677in}{3.053228in}}%
\pgfpathlineto{\pgfqpoint{0.693677in}{0.539544in}}%
\pgfpathclose%
\pgfusepath{fill}%
\end{pgfscope}%
\begin{pgfscope}%
\pgfsetbuttcap%
\pgfsetroundjoin%
\definecolor{currentfill}{rgb}{0.000000,0.000000,0.000000}%
\pgfsetfillcolor{currentfill}%
\pgfsetlinewidth{0.803000pt}%
\definecolor{currentstroke}{rgb}{0.000000,0.000000,0.000000}%
\pgfsetstrokecolor{currentstroke}%
\pgfsetdash{}{0pt}%
\pgfsys@defobject{currentmarker}{\pgfqpoint{0.000000in}{-0.048611in}}{\pgfqpoint{0.000000in}{0.000000in}}{%
\pgfpathmoveto{\pgfqpoint{0.000000in}{0.000000in}}%
\pgfpathlineto{\pgfqpoint{0.000000in}{-0.048611in}}%
\pgfusepath{stroke,fill}%
}%
\begin{pgfscope}%
\pgfsys@transformshift{0.892085in}{0.539544in}%
\pgfsys@useobject{currentmarker}{}%
\end{pgfscope}%
\end{pgfscope}%
\begin{pgfscope}%
\definecolor{textcolor}{rgb}{0.000000,0.000000,0.000000}%
\pgfsetstrokecolor{textcolor}%
\pgfsetfillcolor{textcolor}%
\pgftext[x=0.892085in,y=0.442322in,,top]{\color{textcolor}\rmfamily\fontsize{8.000000}{9.600000}\selectfont \(\displaystyle {00{:}00}\)}%
\end{pgfscope}%
\begin{pgfscope}%
\pgfsetbuttcap%
\pgfsetroundjoin%
\definecolor{currentfill}{rgb}{0.000000,0.000000,0.000000}%
\pgfsetfillcolor{currentfill}%
\pgfsetlinewidth{0.803000pt}%
\definecolor{currentstroke}{rgb}{0.000000,0.000000,0.000000}%
\pgfsetstrokecolor{currentstroke}%
\pgfsetdash{}{0pt}%
\pgfsys@defobject{currentmarker}{\pgfqpoint{0.000000in}{-0.048611in}}{\pgfqpoint{0.000000in}{0.000000in}}{%
\pgfpathmoveto{\pgfqpoint{0.000000in}{0.000000in}}%
\pgfpathlineto{\pgfqpoint{0.000000in}{-0.048611in}}%
\pgfusepath{stroke,fill}%
}%
\begin{pgfscope}%
\pgfsys@transformshift{1.388451in}{0.539544in}%
\pgfsys@useobject{currentmarker}{}%
\end{pgfscope}%
\end{pgfscope}%
\begin{pgfscope}%
\definecolor{textcolor}{rgb}{0.000000,0.000000,0.000000}%
\pgfsetstrokecolor{textcolor}%
\pgfsetfillcolor{textcolor}%
\pgftext[x=1.388451in,y=0.442322in,,top]{\color{textcolor}\rmfamily\fontsize{8.000000}{9.600000}\selectfont \(\displaystyle {03{:}00}\)}%
\end{pgfscope}%
\begin{pgfscope}%
\pgfsetbuttcap%
\pgfsetroundjoin%
\definecolor{currentfill}{rgb}{0.000000,0.000000,0.000000}%
\pgfsetfillcolor{currentfill}%
\pgfsetlinewidth{0.803000pt}%
\definecolor{currentstroke}{rgb}{0.000000,0.000000,0.000000}%
\pgfsetstrokecolor{currentstroke}%
\pgfsetdash{}{0pt}%
\pgfsys@defobject{currentmarker}{\pgfqpoint{0.000000in}{-0.048611in}}{\pgfqpoint{0.000000in}{0.000000in}}{%
\pgfpathmoveto{\pgfqpoint{0.000000in}{0.000000in}}%
\pgfpathlineto{\pgfqpoint{0.000000in}{-0.048611in}}%
\pgfusepath{stroke,fill}%
}%
\begin{pgfscope}%
\pgfsys@transformshift{1.884817in}{0.539544in}%
\pgfsys@useobject{currentmarker}{}%
\end{pgfscope}%
\end{pgfscope}%
\begin{pgfscope}%
\definecolor{textcolor}{rgb}{0.000000,0.000000,0.000000}%
\pgfsetstrokecolor{textcolor}%
\pgfsetfillcolor{textcolor}%
\pgftext[x=1.884817in,y=0.442322in,,top]{\color{textcolor}\rmfamily\fontsize{8.000000}{9.600000}\selectfont \(\displaystyle {06{:}00}\)}%
\end{pgfscope}%
\begin{pgfscope}%
\pgfsetbuttcap%
\pgfsetroundjoin%
\definecolor{currentfill}{rgb}{0.000000,0.000000,0.000000}%
\pgfsetfillcolor{currentfill}%
\pgfsetlinewidth{0.803000pt}%
\definecolor{currentstroke}{rgb}{0.000000,0.000000,0.000000}%
\pgfsetstrokecolor{currentstroke}%
\pgfsetdash{}{0pt}%
\pgfsys@defobject{currentmarker}{\pgfqpoint{0.000000in}{-0.048611in}}{\pgfqpoint{0.000000in}{0.000000in}}{%
\pgfpathmoveto{\pgfqpoint{0.000000in}{0.000000in}}%
\pgfpathlineto{\pgfqpoint{0.000000in}{-0.048611in}}%
\pgfusepath{stroke,fill}%
}%
\begin{pgfscope}%
\pgfsys@transformshift{2.381182in}{0.539544in}%
\pgfsys@useobject{currentmarker}{}%
\end{pgfscope}%
\end{pgfscope}%
\begin{pgfscope}%
\definecolor{textcolor}{rgb}{0.000000,0.000000,0.000000}%
\pgfsetstrokecolor{textcolor}%
\pgfsetfillcolor{textcolor}%
\pgftext[x=2.381182in,y=0.442322in,,top]{\color{textcolor}\rmfamily\fontsize{8.000000}{9.600000}\selectfont \(\displaystyle {09{:}00}\)}%
\end{pgfscope}%
\begin{pgfscope}%
\pgfsetbuttcap%
\pgfsetroundjoin%
\definecolor{currentfill}{rgb}{0.000000,0.000000,0.000000}%
\pgfsetfillcolor{currentfill}%
\pgfsetlinewidth{0.803000pt}%
\definecolor{currentstroke}{rgb}{0.000000,0.000000,0.000000}%
\pgfsetstrokecolor{currentstroke}%
\pgfsetdash{}{0pt}%
\pgfsys@defobject{currentmarker}{\pgfqpoint{0.000000in}{-0.048611in}}{\pgfqpoint{0.000000in}{0.000000in}}{%
\pgfpathmoveto{\pgfqpoint{0.000000in}{0.000000in}}%
\pgfpathlineto{\pgfqpoint{0.000000in}{-0.048611in}}%
\pgfusepath{stroke,fill}%
}%
\begin{pgfscope}%
\pgfsys@transformshift{2.877548in}{0.539544in}%
\pgfsys@useobject{currentmarker}{}%
\end{pgfscope}%
\end{pgfscope}%
\begin{pgfscope}%
\definecolor{textcolor}{rgb}{0.000000,0.000000,0.000000}%
\pgfsetstrokecolor{textcolor}%
\pgfsetfillcolor{textcolor}%
\pgftext[x=2.877548in,y=0.442322in,,top]{\color{textcolor}\rmfamily\fontsize{8.000000}{9.600000}\selectfont \(\displaystyle {12{:}00}\)}%
\end{pgfscope}%
\begin{pgfscope}%
\pgfsetbuttcap%
\pgfsetroundjoin%
\definecolor{currentfill}{rgb}{0.000000,0.000000,0.000000}%
\pgfsetfillcolor{currentfill}%
\pgfsetlinewidth{0.803000pt}%
\definecolor{currentstroke}{rgb}{0.000000,0.000000,0.000000}%
\pgfsetstrokecolor{currentstroke}%
\pgfsetdash{}{0pt}%
\pgfsys@defobject{currentmarker}{\pgfqpoint{0.000000in}{-0.048611in}}{\pgfqpoint{0.000000in}{0.000000in}}{%
\pgfpathmoveto{\pgfqpoint{0.000000in}{0.000000in}}%
\pgfpathlineto{\pgfqpoint{0.000000in}{-0.048611in}}%
\pgfusepath{stroke,fill}%
}%
\begin{pgfscope}%
\pgfsys@transformshift{3.373914in}{0.539544in}%
\pgfsys@useobject{currentmarker}{}%
\end{pgfscope}%
\end{pgfscope}%
\begin{pgfscope}%
\definecolor{textcolor}{rgb}{0.000000,0.000000,0.000000}%
\pgfsetstrokecolor{textcolor}%
\pgfsetfillcolor{textcolor}%
\pgftext[x=3.373914in,y=0.442322in,,top]{\color{textcolor}\rmfamily\fontsize{8.000000}{9.600000}\selectfont \(\displaystyle {15{:}00}\)}%
\end{pgfscope}%
\begin{pgfscope}%
\pgfsetbuttcap%
\pgfsetroundjoin%
\definecolor{currentfill}{rgb}{0.000000,0.000000,0.000000}%
\pgfsetfillcolor{currentfill}%
\pgfsetlinewidth{0.803000pt}%
\definecolor{currentstroke}{rgb}{0.000000,0.000000,0.000000}%
\pgfsetstrokecolor{currentstroke}%
\pgfsetdash{}{0pt}%
\pgfsys@defobject{currentmarker}{\pgfqpoint{0.000000in}{-0.048611in}}{\pgfqpoint{0.000000in}{0.000000in}}{%
\pgfpathmoveto{\pgfqpoint{0.000000in}{0.000000in}}%
\pgfpathlineto{\pgfqpoint{0.000000in}{-0.048611in}}%
\pgfusepath{stroke,fill}%
}%
\begin{pgfscope}%
\pgfsys@transformshift{3.870280in}{0.539544in}%
\pgfsys@useobject{currentmarker}{}%
\end{pgfscope}%
\end{pgfscope}%
\begin{pgfscope}%
\definecolor{textcolor}{rgb}{0.000000,0.000000,0.000000}%
\pgfsetstrokecolor{textcolor}%
\pgfsetfillcolor{textcolor}%
\pgftext[x=3.870280in,y=0.442322in,,top]{\color{textcolor}\rmfamily\fontsize{8.000000}{9.600000}\selectfont \(\displaystyle {18{:}00}\)}%
\end{pgfscope}%
\begin{pgfscope}%
\pgfsetbuttcap%
\pgfsetroundjoin%
\definecolor{currentfill}{rgb}{0.000000,0.000000,0.000000}%
\pgfsetfillcolor{currentfill}%
\pgfsetlinewidth{0.803000pt}%
\definecolor{currentstroke}{rgb}{0.000000,0.000000,0.000000}%
\pgfsetstrokecolor{currentstroke}%
\pgfsetdash{}{0pt}%
\pgfsys@defobject{currentmarker}{\pgfqpoint{0.000000in}{-0.048611in}}{\pgfqpoint{0.000000in}{0.000000in}}{%
\pgfpathmoveto{\pgfqpoint{0.000000in}{0.000000in}}%
\pgfpathlineto{\pgfqpoint{0.000000in}{-0.048611in}}%
\pgfusepath{stroke,fill}%
}%
\begin{pgfscope}%
\pgfsys@transformshift{4.366645in}{0.539544in}%
\pgfsys@useobject{currentmarker}{}%
\end{pgfscope}%
\end{pgfscope}%
\begin{pgfscope}%
\definecolor{textcolor}{rgb}{0.000000,0.000000,0.000000}%
\pgfsetstrokecolor{textcolor}%
\pgfsetfillcolor{textcolor}%
\pgftext[x=4.366645in,y=0.442322in,,top]{\color{textcolor}\rmfamily\fontsize{8.000000}{9.600000}\selectfont \(\displaystyle {21{:}00}\)}%
\end{pgfscope}%
\begin{pgfscope}%
\pgfsetbuttcap%
\pgfsetroundjoin%
\definecolor{currentfill}{rgb}{0.000000,0.000000,0.000000}%
\pgfsetfillcolor{currentfill}%
\pgfsetlinewidth{0.803000pt}%
\definecolor{currentstroke}{rgb}{0.000000,0.000000,0.000000}%
\pgfsetstrokecolor{currentstroke}%
\pgfsetdash{}{0pt}%
\pgfsys@defobject{currentmarker}{\pgfqpoint{0.000000in}{-0.048611in}}{\pgfqpoint{0.000000in}{0.000000in}}{%
\pgfpathmoveto{\pgfqpoint{0.000000in}{0.000000in}}%
\pgfpathlineto{\pgfqpoint{0.000000in}{-0.048611in}}%
\pgfusepath{stroke,fill}%
}%
\begin{pgfscope}%
\pgfsys@transformshift{4.863011in}{0.539544in}%
\pgfsys@useobject{currentmarker}{}%
\end{pgfscope}%
\end{pgfscope}%
\begin{pgfscope}%
\definecolor{textcolor}{rgb}{0.000000,0.000000,0.000000}%
\pgfsetstrokecolor{textcolor}%
\pgfsetfillcolor{textcolor}%
\pgftext[x=4.863011in,y=0.442322in,,top]{\color{textcolor}\rmfamily\fontsize{8.000000}{9.600000}\selectfont \(\displaystyle {00{:}00}\)}%
\end{pgfscope}%
\begin{pgfscope}%
\definecolor{textcolor}{rgb}{0.000000,0.000000,0.000000}%
\pgfsetstrokecolor{textcolor}%
\pgfsetfillcolor{textcolor}%
\pgftext[x=2.876169in,y=0.288100in,,top]{\color{textcolor}\rmfamily\fontsize{10.000000}{12.000000}\selectfont Time (UTC)}%
\end{pgfscope}%
\begin{pgfscope}%
\pgfsetbuttcap%
\pgfsetroundjoin%
\definecolor{currentfill}{rgb}{0.000000,0.000000,0.000000}%
\pgfsetfillcolor{currentfill}%
\pgfsetlinewidth{0.803000pt}%
\definecolor{currentstroke}{rgb}{0.000000,0.000000,0.000000}%
\pgfsetstrokecolor{currentstroke}%
\pgfsetdash{}{0pt}%
\pgfsys@defobject{currentmarker}{\pgfqpoint{-0.048611in}{0.000000in}}{\pgfqpoint{-0.000000in}{0.000000in}}{%
\pgfpathmoveto{\pgfqpoint{-0.000000in}{0.000000in}}%
\pgfpathlineto{\pgfqpoint{-0.048611in}{0.000000in}}%
\pgfusepath{stroke,fill}%
}%
\begin{pgfscope}%
\pgfsys@transformshift{0.693677in}{0.745209in}%
\pgfsys@useobject{currentmarker}{}%
\end{pgfscope}%
\end{pgfscope}%
\begin{pgfscope}%
\definecolor{textcolor}{rgb}{0.000000,0.000000,0.000000}%
\pgfsetstrokecolor{textcolor}%
\pgfsetfillcolor{textcolor}%
\pgftext[x=0.327546in, y=0.706654in, left, base]{\color{textcolor}\rmfamily\fontsize{8.000000}{9.600000}\selectfont \(\displaystyle {20.00}\)}%
\end{pgfscope}%
\begin{pgfscope}%
\pgfsetbuttcap%
\pgfsetroundjoin%
\definecolor{currentfill}{rgb}{0.000000,0.000000,0.000000}%
\pgfsetfillcolor{currentfill}%
\pgfsetlinewidth{0.803000pt}%
\definecolor{currentstroke}{rgb}{0.000000,0.000000,0.000000}%
\pgfsetstrokecolor{currentstroke}%
\pgfsetdash{}{0pt}%
\pgfsys@defobject{currentmarker}{\pgfqpoint{-0.048611in}{0.000000in}}{\pgfqpoint{-0.000000in}{0.000000in}}{%
\pgfpathmoveto{\pgfqpoint{-0.000000in}{0.000000in}}%
\pgfpathlineto{\pgfqpoint{-0.048611in}{0.000000in}}%
\pgfusepath{stroke,fill}%
}%
\begin{pgfscope}%
\pgfsys@transformshift{0.693677in}{1.071662in}%
\pgfsys@useobject{currentmarker}{}%
\end{pgfscope}%
\end{pgfscope}%
\begin{pgfscope}%
\definecolor{textcolor}{rgb}{0.000000,0.000000,0.000000}%
\pgfsetstrokecolor{textcolor}%
\pgfsetfillcolor{textcolor}%
\pgftext[x=0.327546in, y=1.033106in, left, base]{\color{textcolor}\rmfamily\fontsize{8.000000}{9.600000}\selectfont \(\displaystyle {20.25}\)}%
\end{pgfscope}%
\begin{pgfscope}%
\pgfsetbuttcap%
\pgfsetroundjoin%
\definecolor{currentfill}{rgb}{0.000000,0.000000,0.000000}%
\pgfsetfillcolor{currentfill}%
\pgfsetlinewidth{0.803000pt}%
\definecolor{currentstroke}{rgb}{0.000000,0.000000,0.000000}%
\pgfsetstrokecolor{currentstroke}%
\pgfsetdash{}{0pt}%
\pgfsys@defobject{currentmarker}{\pgfqpoint{-0.048611in}{0.000000in}}{\pgfqpoint{-0.000000in}{0.000000in}}{%
\pgfpathmoveto{\pgfqpoint{-0.000000in}{0.000000in}}%
\pgfpathlineto{\pgfqpoint{-0.048611in}{0.000000in}}%
\pgfusepath{stroke,fill}%
}%
\begin{pgfscope}%
\pgfsys@transformshift{0.693677in}{1.398114in}%
\pgfsys@useobject{currentmarker}{}%
\end{pgfscope}%
\end{pgfscope}%
\begin{pgfscope}%
\definecolor{textcolor}{rgb}{0.000000,0.000000,0.000000}%
\pgfsetstrokecolor{textcolor}%
\pgfsetfillcolor{textcolor}%
\pgftext[x=0.327546in, y=1.359559in, left, base]{\color{textcolor}\rmfamily\fontsize{8.000000}{9.600000}\selectfont \(\displaystyle {20.50}\)}%
\end{pgfscope}%
\begin{pgfscope}%
\pgfsetbuttcap%
\pgfsetroundjoin%
\definecolor{currentfill}{rgb}{0.000000,0.000000,0.000000}%
\pgfsetfillcolor{currentfill}%
\pgfsetlinewidth{0.803000pt}%
\definecolor{currentstroke}{rgb}{0.000000,0.000000,0.000000}%
\pgfsetstrokecolor{currentstroke}%
\pgfsetdash{}{0pt}%
\pgfsys@defobject{currentmarker}{\pgfqpoint{-0.048611in}{0.000000in}}{\pgfqpoint{-0.000000in}{0.000000in}}{%
\pgfpathmoveto{\pgfqpoint{-0.000000in}{0.000000in}}%
\pgfpathlineto{\pgfqpoint{-0.048611in}{0.000000in}}%
\pgfusepath{stroke,fill}%
}%
\begin{pgfscope}%
\pgfsys@transformshift{0.693677in}{1.724567in}%
\pgfsys@useobject{currentmarker}{}%
\end{pgfscope}%
\end{pgfscope}%
\begin{pgfscope}%
\definecolor{textcolor}{rgb}{0.000000,0.000000,0.000000}%
\pgfsetstrokecolor{textcolor}%
\pgfsetfillcolor{textcolor}%
\pgftext[x=0.327546in, y=1.686011in, left, base]{\color{textcolor}\rmfamily\fontsize{8.000000}{9.600000}\selectfont \(\displaystyle {20.75}\)}%
\end{pgfscope}%
\begin{pgfscope}%
\pgfsetbuttcap%
\pgfsetroundjoin%
\definecolor{currentfill}{rgb}{0.000000,0.000000,0.000000}%
\pgfsetfillcolor{currentfill}%
\pgfsetlinewidth{0.803000pt}%
\definecolor{currentstroke}{rgb}{0.000000,0.000000,0.000000}%
\pgfsetstrokecolor{currentstroke}%
\pgfsetdash{}{0pt}%
\pgfsys@defobject{currentmarker}{\pgfqpoint{-0.048611in}{0.000000in}}{\pgfqpoint{-0.000000in}{0.000000in}}{%
\pgfpathmoveto{\pgfqpoint{-0.000000in}{0.000000in}}%
\pgfpathlineto{\pgfqpoint{-0.048611in}{0.000000in}}%
\pgfusepath{stroke,fill}%
}%
\begin{pgfscope}%
\pgfsys@transformshift{0.693677in}{2.051019in}%
\pgfsys@useobject{currentmarker}{}%
\end{pgfscope}%
\end{pgfscope}%
\begin{pgfscope}%
\definecolor{textcolor}{rgb}{0.000000,0.000000,0.000000}%
\pgfsetstrokecolor{textcolor}%
\pgfsetfillcolor{textcolor}%
\pgftext[x=0.327546in, y=2.012463in, left, base]{\color{textcolor}\rmfamily\fontsize{8.000000}{9.600000}\selectfont \(\displaystyle {21.00}\)}%
\end{pgfscope}%
\begin{pgfscope}%
\pgfsetbuttcap%
\pgfsetroundjoin%
\definecolor{currentfill}{rgb}{0.000000,0.000000,0.000000}%
\pgfsetfillcolor{currentfill}%
\pgfsetlinewidth{0.803000pt}%
\definecolor{currentstroke}{rgb}{0.000000,0.000000,0.000000}%
\pgfsetstrokecolor{currentstroke}%
\pgfsetdash{}{0pt}%
\pgfsys@defobject{currentmarker}{\pgfqpoint{-0.048611in}{0.000000in}}{\pgfqpoint{-0.000000in}{0.000000in}}{%
\pgfpathmoveto{\pgfqpoint{-0.000000in}{0.000000in}}%
\pgfpathlineto{\pgfqpoint{-0.048611in}{0.000000in}}%
\pgfusepath{stroke,fill}%
}%
\begin{pgfscope}%
\pgfsys@transformshift{0.693677in}{2.377471in}%
\pgfsys@useobject{currentmarker}{}%
\end{pgfscope}%
\end{pgfscope}%
\begin{pgfscope}%
\definecolor{textcolor}{rgb}{0.000000,0.000000,0.000000}%
\pgfsetstrokecolor{textcolor}%
\pgfsetfillcolor{textcolor}%
\pgftext[x=0.327546in, y=2.338916in, left, base]{\color{textcolor}\rmfamily\fontsize{8.000000}{9.600000}\selectfont \(\displaystyle {21.25}\)}%
\end{pgfscope}%
\begin{pgfscope}%
\pgfsetbuttcap%
\pgfsetroundjoin%
\definecolor{currentfill}{rgb}{0.000000,0.000000,0.000000}%
\pgfsetfillcolor{currentfill}%
\pgfsetlinewidth{0.803000pt}%
\definecolor{currentstroke}{rgb}{0.000000,0.000000,0.000000}%
\pgfsetstrokecolor{currentstroke}%
\pgfsetdash{}{0pt}%
\pgfsys@defobject{currentmarker}{\pgfqpoint{-0.048611in}{0.000000in}}{\pgfqpoint{-0.000000in}{0.000000in}}{%
\pgfpathmoveto{\pgfqpoint{-0.000000in}{0.000000in}}%
\pgfpathlineto{\pgfqpoint{-0.048611in}{0.000000in}}%
\pgfusepath{stroke,fill}%
}%
\begin{pgfscope}%
\pgfsys@transformshift{0.693677in}{2.703924in}%
\pgfsys@useobject{currentmarker}{}%
\end{pgfscope}%
\end{pgfscope}%
\begin{pgfscope}%
\definecolor{textcolor}{rgb}{0.000000,0.000000,0.000000}%
\pgfsetstrokecolor{textcolor}%
\pgfsetfillcolor{textcolor}%
\pgftext[x=0.327546in, y=2.665368in, left, base]{\color{textcolor}\rmfamily\fontsize{8.000000}{9.600000}\selectfont \(\displaystyle {21.50}\)}%
\end{pgfscope}%
\begin{pgfscope}%
\pgfsetbuttcap%
\pgfsetroundjoin%
\definecolor{currentfill}{rgb}{0.000000,0.000000,0.000000}%
\pgfsetfillcolor{currentfill}%
\pgfsetlinewidth{0.803000pt}%
\definecolor{currentstroke}{rgb}{0.000000,0.000000,0.000000}%
\pgfsetstrokecolor{currentstroke}%
\pgfsetdash{}{0pt}%
\pgfsys@defobject{currentmarker}{\pgfqpoint{-0.048611in}{0.000000in}}{\pgfqpoint{-0.000000in}{0.000000in}}{%
\pgfpathmoveto{\pgfqpoint{-0.000000in}{0.000000in}}%
\pgfpathlineto{\pgfqpoint{-0.048611in}{0.000000in}}%
\pgfusepath{stroke,fill}%
}%
\begin{pgfscope}%
\pgfsys@transformshift{0.693677in}{3.030376in}%
\pgfsys@useobject{currentmarker}{}%
\end{pgfscope}%
\end{pgfscope}%
\begin{pgfscope}%
\definecolor{textcolor}{rgb}{0.000000,0.000000,0.000000}%
\pgfsetstrokecolor{textcolor}%
\pgfsetfillcolor{textcolor}%
\pgftext[x=0.327546in, y=2.991821in, left, base]{\color{textcolor}\rmfamily\fontsize{8.000000}{9.600000}\selectfont \(\displaystyle {21.75}\)}%
\end{pgfscope}%
\begin{pgfscope}%
\definecolor{textcolor}{rgb}{0.000000,0.000000,0.000000}%
\pgfsetstrokecolor{textcolor}%
\pgfsetfillcolor{textcolor}%
\pgftext[x=0.271991in,y=1.796386in,,bottom,rotate=90.000000]{\color{textcolor}\rmfamily\fontsize{10.000000}{12.000000}\selectfont Temperature in \unit{\celsius}}%
\end{pgfscope}%
\begin{pgfscope}%
\pgfpathrectangle{\pgfqpoint{0.693677in}{0.539544in}}{\pgfqpoint{4.364985in}{2.513684in}}%
\pgfusepath{clip}%
\pgfsetrectcap%
\pgfsetroundjoin%
\pgfsetlinewidth{0.501875pt}%
\definecolor{currentstroke}{rgb}{0.121569,0.466667,0.705882}%
\pgfsetstrokecolor{currentstroke}%
\pgfsetstrokeopacity{0.700000}%
\pgfsetdash{}{0pt}%
\pgfpathmoveto{\pgfqpoint{0.892085in}{2.612517in}}%
\pgfpathlineto{\pgfqpoint{0.894843in}{2.703924in}}%
\pgfpathlineto{\pgfqpoint{0.930691in}{2.782272in}}%
\pgfpathlineto{\pgfqpoint{0.933449in}{2.703924in}}%
\pgfpathlineto{\pgfqpoint{0.936207in}{2.782272in}}%
\pgfpathlineto{\pgfqpoint{0.938964in}{2.703924in}}%
\pgfpathlineto{\pgfqpoint{0.941722in}{2.782272in}}%
\pgfpathlineto{\pgfqpoint{0.944479in}{2.703924in}}%
\pgfpathlineto{\pgfqpoint{0.947237in}{2.782272in}}%
\pgfpathlineto{\pgfqpoint{0.949995in}{2.703924in}}%
\pgfpathlineto{\pgfqpoint{0.952752in}{2.782272in}}%
\pgfpathlineto{\pgfqpoint{0.955510in}{2.703924in}}%
\pgfpathlineto{\pgfqpoint{0.958267in}{2.782272in}}%
\pgfpathlineto{\pgfqpoint{0.961025in}{2.703924in}}%
\pgfpathlineto{\pgfqpoint{0.963783in}{2.782272in}}%
\pgfpathlineto{\pgfqpoint{0.966540in}{2.703924in}}%
\pgfpathlineto{\pgfqpoint{0.969298in}{2.782272in}}%
\pgfpathlineto{\pgfqpoint{0.972055in}{2.703924in}}%
\pgfpathlineto{\pgfqpoint{0.974813in}{2.782272in}}%
\pgfpathlineto{\pgfqpoint{0.977570in}{2.703924in}}%
\pgfpathlineto{\pgfqpoint{0.980328in}{2.782272in}}%
\pgfpathlineto{\pgfqpoint{0.985843in}{2.703924in}}%
\pgfpathlineto{\pgfqpoint{0.988601in}{2.782272in}}%
\pgfpathlineto{\pgfqpoint{1.032722in}{2.860621in}}%
\pgfpathlineto{\pgfqpoint{1.035480in}{2.782272in}}%
\pgfpathlineto{\pgfqpoint{1.040995in}{2.860621in}}%
\pgfpathlineto{\pgfqpoint{1.043753in}{2.782272in}}%
\pgfpathlineto{\pgfqpoint{1.046510in}{2.860621in}}%
\pgfpathlineto{\pgfqpoint{1.049268in}{2.782272in}}%
\pgfpathlineto{\pgfqpoint{1.054783in}{2.860621in}}%
\pgfpathlineto{\pgfqpoint{1.057540in}{2.782272in}}%
\pgfpathlineto{\pgfqpoint{1.060298in}{2.860621in}}%
\pgfpathlineto{\pgfqpoint{1.063056in}{2.782272in}}%
\pgfpathlineto{\pgfqpoint{1.065813in}{2.860621in}}%
\pgfpathlineto{\pgfqpoint{1.068571in}{2.782272in}}%
\pgfpathlineto{\pgfqpoint{1.071328in}{2.860621in}}%
\pgfpathlineto{\pgfqpoint{1.074086in}{2.782272in}}%
\pgfpathlineto{\pgfqpoint{1.076844in}{2.860621in}}%
\pgfpathlineto{\pgfqpoint{1.079601in}{2.782272in}}%
\pgfpathlineto{\pgfqpoint{1.082359in}{2.860621in}}%
\pgfpathlineto{\pgfqpoint{1.085116in}{2.782272in}}%
\pgfpathlineto{\pgfqpoint{1.087874in}{2.860621in}}%
\pgfpathlineto{\pgfqpoint{1.090632in}{2.782272in}}%
\pgfpathlineto{\pgfqpoint{1.093389in}{2.860621in}}%
\pgfpathlineto{\pgfqpoint{1.096147in}{2.782272in}}%
\pgfpathlineto{\pgfqpoint{1.098904in}{2.860621in}}%
\pgfpathlineto{\pgfqpoint{1.109935in}{2.782272in}}%
\pgfpathlineto{\pgfqpoint{1.112692in}{2.860621in}}%
\pgfpathlineto{\pgfqpoint{1.123723in}{2.782272in}}%
\pgfpathlineto{\pgfqpoint{1.126480in}{2.860621in}}%
\pgfpathlineto{\pgfqpoint{1.178874in}{2.938970in}}%
\pgfpathlineto{\pgfqpoint{1.181632in}{2.860621in}}%
\pgfpathlineto{\pgfqpoint{1.184390in}{2.938970in}}%
\pgfpathlineto{\pgfqpoint{1.187147in}{2.860621in}}%
\pgfpathlineto{\pgfqpoint{1.189905in}{2.938970in}}%
\pgfpathlineto{\pgfqpoint{1.220238in}{2.051019in}}%
\pgfpathlineto{\pgfqpoint{1.225753in}{1.881264in}}%
\pgfpathlineto{\pgfqpoint{1.231268in}{1.802915in}}%
\pgfpathlineto{\pgfqpoint{1.234026in}{1.724567in}}%
\pgfpathlineto{\pgfqpoint{1.239541in}{1.633160in}}%
\pgfpathlineto{\pgfqpoint{1.242299in}{1.554811in}}%
\pgfpathlineto{\pgfqpoint{1.250572in}{1.476463in}}%
\pgfpathlineto{\pgfqpoint{1.256087in}{1.398114in}}%
\pgfpathlineto{\pgfqpoint{1.258844in}{1.476463in}}%
\pgfpathlineto{\pgfqpoint{1.264360in}{1.306707in}}%
\pgfpathlineto{\pgfqpoint{1.272632in}{1.228359in}}%
\pgfpathlineto{\pgfqpoint{1.275390in}{1.306707in}}%
\pgfpathlineto{\pgfqpoint{1.280905in}{1.150010in}}%
\pgfpathlineto{\pgfqpoint{1.283663in}{1.228359in}}%
\pgfpathlineto{\pgfqpoint{1.286420in}{1.150010in}}%
\pgfpathlineto{\pgfqpoint{1.291935in}{1.071662in}}%
\pgfpathlineto{\pgfqpoint{1.294693in}{1.150010in}}%
\pgfpathlineto{\pgfqpoint{1.297451in}{1.071662in}}%
\pgfpathlineto{\pgfqpoint{1.300208in}{1.150010in}}%
\pgfpathlineto{\pgfqpoint{1.302966in}{1.071662in}}%
\pgfpathlineto{\pgfqpoint{1.313996in}{0.980255in}}%
\pgfpathlineto{\pgfqpoint{1.316754in}{1.071662in}}%
\pgfpathlineto{\pgfqpoint{1.319511in}{0.980255in}}%
\pgfpathlineto{\pgfqpoint{1.322269in}{1.071662in}}%
\pgfpathlineto{\pgfqpoint{1.325026in}{0.980255in}}%
\pgfpathlineto{\pgfqpoint{1.330542in}{0.901906in}}%
\pgfpathlineto{\pgfqpoint{1.333299in}{0.980255in}}%
\pgfpathlineto{\pgfqpoint{1.336057in}{0.901906in}}%
\pgfpathlineto{\pgfqpoint{1.347087in}{0.823558in}}%
\pgfpathlineto{\pgfqpoint{1.349845in}{0.901906in}}%
\pgfpathlineto{\pgfqpoint{1.352602in}{0.823558in}}%
\pgfpathlineto{\pgfqpoint{1.355360in}{0.901906in}}%
\pgfpathlineto{\pgfqpoint{1.358118in}{0.823558in}}%
\pgfpathlineto{\pgfqpoint{1.369148in}{0.745209in}}%
\pgfpathlineto{\pgfqpoint{1.371905in}{0.823558in}}%
\pgfpathlineto{\pgfqpoint{1.374663in}{0.745209in}}%
\pgfpathlineto{\pgfqpoint{1.377421in}{0.823558in}}%
\pgfpathlineto{\pgfqpoint{1.382936in}{0.901906in}}%
\pgfpathlineto{\pgfqpoint{1.388451in}{1.071662in}}%
\pgfpathlineto{\pgfqpoint{1.393966in}{1.150010in}}%
\pgfpathlineto{\pgfqpoint{1.396724in}{1.228359in}}%
\pgfpathlineto{\pgfqpoint{1.402239in}{1.306707in}}%
\pgfpathlineto{\pgfqpoint{1.404996in}{1.398114in}}%
\pgfpathlineto{\pgfqpoint{1.432572in}{1.802915in}}%
\pgfpathlineto{\pgfqpoint{1.438088in}{1.724567in}}%
\pgfpathlineto{\pgfqpoint{1.443603in}{1.881264in}}%
\pgfpathlineto{\pgfqpoint{1.449118in}{1.959612in}}%
\pgfpathlineto{\pgfqpoint{1.451875in}{1.881264in}}%
\pgfpathlineto{\pgfqpoint{1.454633in}{1.959612in}}%
\pgfpathlineto{\pgfqpoint{1.460148in}{2.051019in}}%
\pgfpathlineto{\pgfqpoint{1.462906in}{1.959612in}}%
\pgfpathlineto{\pgfqpoint{1.468421in}{2.129368in}}%
\pgfpathlineto{\pgfqpoint{1.471179in}{2.051019in}}%
\pgfpathlineto{\pgfqpoint{1.473936in}{2.129368in}}%
\pgfpathlineto{\pgfqpoint{1.482209in}{2.207716in}}%
\pgfpathlineto{\pgfqpoint{1.484967in}{2.129368in}}%
\pgfpathlineto{\pgfqpoint{1.487724in}{2.207716in}}%
\pgfpathlineto{\pgfqpoint{1.490482in}{2.129368in}}%
\pgfpathlineto{\pgfqpoint{1.493239in}{2.207716in}}%
\pgfpathlineto{\pgfqpoint{1.501512in}{2.286065in}}%
\pgfpathlineto{\pgfqpoint{1.504270in}{2.207716in}}%
\pgfpathlineto{\pgfqpoint{1.507027in}{2.286065in}}%
\pgfpathlineto{\pgfqpoint{1.509785in}{2.207716in}}%
\pgfpathlineto{\pgfqpoint{1.512542in}{2.286065in}}%
\pgfpathlineto{\pgfqpoint{1.520815in}{2.377471in}}%
\pgfpathlineto{\pgfqpoint{1.523573in}{2.286065in}}%
\pgfpathlineto{\pgfqpoint{1.526330in}{2.377471in}}%
\pgfpathlineto{\pgfqpoint{1.529088in}{2.286065in}}%
\pgfpathlineto{\pgfqpoint{1.531846in}{2.377471in}}%
\pgfpathlineto{\pgfqpoint{1.545633in}{2.455820in}}%
\pgfpathlineto{\pgfqpoint{1.548391in}{2.377471in}}%
\pgfpathlineto{\pgfqpoint{1.551149in}{2.455820in}}%
\pgfpathlineto{\pgfqpoint{1.553906in}{2.377471in}}%
\pgfpathlineto{\pgfqpoint{1.556664in}{2.455820in}}%
\pgfpathlineto{\pgfqpoint{1.562179in}{2.377471in}}%
\pgfpathlineto{\pgfqpoint{1.564937in}{2.455820in}}%
\pgfpathlineto{\pgfqpoint{1.581482in}{2.534169in}}%
\pgfpathlineto{\pgfqpoint{1.584240in}{2.455820in}}%
\pgfpathlineto{\pgfqpoint{1.586997in}{2.534169in}}%
\pgfpathlineto{\pgfqpoint{1.589755in}{2.455820in}}%
\pgfpathlineto{\pgfqpoint{1.592512in}{2.534169in}}%
\pgfpathlineto{\pgfqpoint{1.595270in}{2.455820in}}%
\pgfpathlineto{\pgfqpoint{1.598028in}{2.534169in}}%
\pgfpathlineto{\pgfqpoint{1.620088in}{2.612517in}}%
\pgfpathlineto{\pgfqpoint{1.622846in}{2.534169in}}%
\pgfpathlineto{\pgfqpoint{1.628361in}{2.612517in}}%
\pgfpathlineto{\pgfqpoint{1.631119in}{2.534169in}}%
\pgfpathlineto{\pgfqpoint{1.633876in}{2.612517in}}%
\pgfpathlineto{\pgfqpoint{1.636634in}{2.534169in}}%
\pgfpathlineto{\pgfqpoint{1.639391in}{2.612517in}}%
\pgfpathlineto{\pgfqpoint{1.642149in}{2.534169in}}%
\pgfpathlineto{\pgfqpoint{1.644907in}{2.612517in}}%
\pgfpathlineto{\pgfqpoint{1.647664in}{2.534169in}}%
\pgfpathlineto{\pgfqpoint{1.650422in}{2.612517in}}%
\pgfpathlineto{\pgfqpoint{1.686270in}{2.703924in}}%
\pgfpathlineto{\pgfqpoint{1.689028in}{2.612517in}}%
\pgfpathlineto{\pgfqpoint{1.691786in}{2.703924in}}%
\pgfpathlineto{\pgfqpoint{1.694543in}{2.612517in}}%
\pgfpathlineto{\pgfqpoint{1.697301in}{2.703924in}}%
\pgfpathlineto{\pgfqpoint{1.700058in}{2.612517in}}%
\pgfpathlineto{\pgfqpoint{1.702816in}{2.703924in}}%
\pgfpathlineto{\pgfqpoint{1.705574in}{2.612517in}}%
\pgfpathlineto{\pgfqpoint{1.708331in}{2.703924in}}%
\pgfpathlineto{\pgfqpoint{1.711089in}{2.612517in}}%
\pgfpathlineto{\pgfqpoint{1.713846in}{2.703924in}}%
\pgfpathlineto{\pgfqpoint{1.716604in}{2.612517in}}%
\pgfpathlineto{\pgfqpoint{1.719361in}{2.703924in}}%
\pgfpathlineto{\pgfqpoint{1.724877in}{2.612517in}}%
\pgfpathlineto{\pgfqpoint{1.727634in}{2.703924in}}%
\pgfpathlineto{\pgfqpoint{1.733149in}{2.612517in}}%
\pgfpathlineto{\pgfqpoint{1.735907in}{2.703924in}}%
\pgfpathlineto{\pgfqpoint{1.755210in}{2.782272in}}%
\pgfpathlineto{\pgfqpoint{1.757968in}{2.703924in}}%
\pgfpathlineto{\pgfqpoint{1.766240in}{2.782272in}}%
\pgfpathlineto{\pgfqpoint{1.768998in}{2.703924in}}%
\pgfpathlineto{\pgfqpoint{1.771756in}{2.782272in}}%
\pgfpathlineto{\pgfqpoint{1.774513in}{2.703924in}}%
\pgfpathlineto{\pgfqpoint{1.777271in}{2.782272in}}%
\pgfpathlineto{\pgfqpoint{1.780028in}{2.703924in}}%
\pgfpathlineto{\pgfqpoint{1.782786in}{2.782272in}}%
\pgfpathlineto{\pgfqpoint{1.785544in}{2.703924in}}%
\pgfpathlineto{\pgfqpoint{1.788301in}{2.782272in}}%
\pgfpathlineto{\pgfqpoint{1.791059in}{2.703924in}}%
\pgfpathlineto{\pgfqpoint{1.793816in}{2.782272in}}%
\pgfpathlineto{\pgfqpoint{1.799331in}{2.703924in}}%
\pgfpathlineto{\pgfqpoint{1.802089in}{2.782272in}}%
\pgfpathlineto{\pgfqpoint{1.804847in}{2.703924in}}%
\pgfpathlineto{\pgfqpoint{1.807604in}{2.782272in}}%
\pgfpathlineto{\pgfqpoint{1.810362in}{2.703924in}}%
\pgfpathlineto{\pgfqpoint{1.813119in}{2.782272in}}%
\pgfpathlineto{\pgfqpoint{1.884817in}{2.860621in}}%
\pgfpathlineto{\pgfqpoint{1.887574in}{2.782272in}}%
\pgfpathlineto{\pgfqpoint{1.890332in}{2.860621in}}%
\pgfpathlineto{\pgfqpoint{1.893089in}{2.782272in}}%
\pgfpathlineto{\pgfqpoint{1.895847in}{2.860621in}}%
\pgfpathlineto{\pgfqpoint{1.898605in}{2.782272in}}%
\pgfpathlineto{\pgfqpoint{1.901362in}{2.860621in}}%
\pgfpathlineto{\pgfqpoint{1.904120in}{2.782272in}}%
\pgfpathlineto{\pgfqpoint{1.906877in}{2.860621in}}%
\pgfpathlineto{\pgfqpoint{1.909635in}{2.782272in}}%
\pgfpathlineto{\pgfqpoint{1.912393in}{2.860621in}}%
\pgfpathlineto{\pgfqpoint{1.915150in}{2.782272in}}%
\pgfpathlineto{\pgfqpoint{1.917908in}{2.860621in}}%
\pgfpathlineto{\pgfqpoint{1.920665in}{2.782272in}}%
\pgfpathlineto{\pgfqpoint{1.923423in}{2.860621in}}%
\pgfpathlineto{\pgfqpoint{1.926181in}{2.782272in}}%
\pgfpathlineto{\pgfqpoint{1.928938in}{2.860621in}}%
\pgfpathlineto{\pgfqpoint{1.939968in}{2.782272in}}%
\pgfpathlineto{\pgfqpoint{1.942726in}{2.860621in}}%
\pgfpathlineto{\pgfqpoint{1.992363in}{2.782272in}}%
\pgfpathlineto{\pgfqpoint{2.017181in}{2.051019in}}%
\pgfpathlineto{\pgfqpoint{2.022696in}{1.881264in}}%
\pgfpathlineto{\pgfqpoint{2.033726in}{1.724567in}}%
\pgfpathlineto{\pgfqpoint{2.036484in}{1.633160in}}%
\pgfpathlineto{\pgfqpoint{2.041999in}{1.554811in}}%
\pgfpathlineto{\pgfqpoint{2.050272in}{1.476463in}}%
\pgfpathlineto{\pgfqpoint{2.066817in}{1.228359in}}%
\pgfpathlineto{\pgfqpoint{2.069575in}{1.306707in}}%
\pgfpathlineto{\pgfqpoint{2.072333in}{1.228359in}}%
\pgfpathlineto{\pgfqpoint{2.077848in}{1.150010in}}%
\pgfpathlineto{\pgfqpoint{2.080605in}{1.228359in}}%
\pgfpathlineto{\pgfqpoint{2.083363in}{1.150010in}}%
\pgfpathlineto{\pgfqpoint{2.088878in}{1.071662in}}%
\pgfpathlineto{\pgfqpoint{2.091636in}{1.150010in}}%
\pgfpathlineto{\pgfqpoint{2.094393in}{1.071662in}}%
\pgfpathlineto{\pgfqpoint{2.102666in}{0.980255in}}%
\pgfpathlineto{\pgfqpoint{2.105424in}{1.071662in}}%
\pgfpathlineto{\pgfqpoint{2.108181in}{0.980255in}}%
\pgfpathlineto{\pgfqpoint{2.110939in}{1.071662in}}%
\pgfpathlineto{\pgfqpoint{2.116454in}{0.901906in}}%
\pgfpathlineto{\pgfqpoint{2.119212in}{0.980255in}}%
\pgfpathlineto{\pgfqpoint{2.121969in}{0.901906in}}%
\pgfpathlineto{\pgfqpoint{2.124727in}{0.980255in}}%
\pgfpathlineto{\pgfqpoint{2.127484in}{0.901906in}}%
\pgfpathlineto{\pgfqpoint{2.133000in}{0.823558in}}%
\pgfpathlineto{\pgfqpoint{2.135757in}{0.901906in}}%
\pgfpathlineto{\pgfqpoint{2.138515in}{0.823558in}}%
\pgfpathlineto{\pgfqpoint{2.152303in}{0.745209in}}%
\pgfpathlineto{\pgfqpoint{2.155060in}{0.823558in}}%
\pgfpathlineto{\pgfqpoint{2.157818in}{0.745209in}}%
\pgfpathlineto{\pgfqpoint{2.160575in}{0.823558in}}%
\pgfpathlineto{\pgfqpoint{2.163333in}{0.745209in}}%
\pgfpathlineto{\pgfqpoint{2.182636in}{0.823558in}}%
\pgfpathlineto{\pgfqpoint{2.188151in}{0.980255in}}%
\pgfpathlineto{\pgfqpoint{2.193666in}{1.071662in}}%
\pgfpathlineto{\pgfqpoint{2.196424in}{1.150010in}}%
\pgfpathlineto{\pgfqpoint{2.201939in}{1.228359in}}%
\pgfpathlineto{\pgfqpoint{2.204697in}{1.306707in}}%
\pgfpathlineto{\pgfqpoint{2.210212in}{1.398114in}}%
\pgfpathlineto{\pgfqpoint{2.212970in}{1.476463in}}%
\pgfpathlineto{\pgfqpoint{2.218485in}{1.554811in}}%
\pgfpathlineto{\pgfqpoint{2.221242in}{1.476463in}}%
\pgfpathlineto{\pgfqpoint{2.226758in}{1.633160in}}%
\pgfpathlineto{\pgfqpoint{2.237788in}{1.802915in}}%
\pgfpathlineto{\pgfqpoint{2.246061in}{1.881264in}}%
\pgfpathlineto{\pgfqpoint{2.248818in}{1.802915in}}%
\pgfpathlineto{\pgfqpoint{2.254333in}{1.959612in}}%
\pgfpathlineto{\pgfqpoint{2.257091in}{1.881264in}}%
\pgfpathlineto{\pgfqpoint{2.259849in}{1.959612in}}%
\pgfpathlineto{\pgfqpoint{2.265364in}{2.051019in}}%
\pgfpathlineto{\pgfqpoint{2.268121in}{1.959612in}}%
\pgfpathlineto{\pgfqpoint{2.270879in}{2.051019in}}%
\pgfpathlineto{\pgfqpoint{2.276394in}{2.129368in}}%
\pgfpathlineto{\pgfqpoint{2.279152in}{2.051019in}}%
\pgfpathlineto{\pgfqpoint{2.281909in}{2.129368in}}%
\pgfpathlineto{\pgfqpoint{2.292940in}{2.207716in}}%
\pgfpathlineto{\pgfqpoint{2.295697in}{2.129368in}}%
\pgfpathlineto{\pgfqpoint{2.298455in}{2.207716in}}%
\pgfpathlineto{\pgfqpoint{2.309485in}{2.286065in}}%
\pgfpathlineto{\pgfqpoint{2.312243in}{2.207716in}}%
\pgfpathlineto{\pgfqpoint{2.315000in}{2.286065in}}%
\pgfpathlineto{\pgfqpoint{2.317758in}{2.207716in}}%
\pgfpathlineto{\pgfqpoint{2.320516in}{2.286065in}}%
\pgfpathlineto{\pgfqpoint{2.334303in}{2.377471in}}%
\pgfpathlineto{\pgfqpoint{2.337061in}{2.286065in}}%
\pgfpathlineto{\pgfqpoint{2.339819in}{2.377471in}}%
\pgfpathlineto{\pgfqpoint{2.342576in}{2.286065in}}%
\pgfpathlineto{\pgfqpoint{2.345334in}{2.377471in}}%
\pgfpathlineto{\pgfqpoint{2.348091in}{2.286065in}}%
\pgfpathlineto{\pgfqpoint{2.350849in}{2.377471in}}%
\pgfpathlineto{\pgfqpoint{2.367394in}{2.455820in}}%
\pgfpathlineto{\pgfqpoint{2.370152in}{2.377471in}}%
\pgfpathlineto{\pgfqpoint{2.372910in}{2.455820in}}%
\pgfpathlineto{\pgfqpoint{2.375667in}{2.377471in}}%
\pgfpathlineto{\pgfqpoint{2.378425in}{2.455820in}}%
\pgfpathlineto{\pgfqpoint{2.381182in}{2.377471in}}%
\pgfpathlineto{\pgfqpoint{2.383940in}{2.455820in}}%
\pgfpathlineto{\pgfqpoint{2.397728in}{2.534169in}}%
\pgfpathlineto{\pgfqpoint{2.400486in}{2.455820in}}%
\pgfpathlineto{\pgfqpoint{2.403243in}{2.534169in}}%
\pgfpathlineto{\pgfqpoint{2.406001in}{2.455820in}}%
\pgfpathlineto{\pgfqpoint{2.408758in}{2.534169in}}%
\pgfpathlineto{\pgfqpoint{2.411516in}{2.455820in}}%
\pgfpathlineto{\pgfqpoint{2.414273in}{2.534169in}}%
\pgfpathlineto{\pgfqpoint{2.417031in}{2.455820in}}%
\pgfpathlineto{\pgfqpoint{2.419789in}{2.534169in}}%
\pgfpathlineto{\pgfqpoint{2.422546in}{2.455820in}}%
\pgfpathlineto{\pgfqpoint{2.425304in}{2.534169in}}%
\pgfpathlineto{\pgfqpoint{2.447365in}{2.612517in}}%
\pgfpathlineto{\pgfqpoint{2.450122in}{2.534169in}}%
\pgfpathlineto{\pgfqpoint{2.452880in}{2.612517in}}%
\pgfpathlineto{\pgfqpoint{2.455637in}{2.534169in}}%
\pgfpathlineto{\pgfqpoint{2.458395in}{2.612517in}}%
\pgfpathlineto{\pgfqpoint{2.461152in}{2.534169in}}%
\pgfpathlineto{\pgfqpoint{2.463910in}{2.612517in}}%
\pgfpathlineto{\pgfqpoint{2.466668in}{2.534169in}}%
\pgfpathlineto{\pgfqpoint{2.469425in}{2.612517in}}%
\pgfpathlineto{\pgfqpoint{2.472183in}{2.534169in}}%
\pgfpathlineto{\pgfqpoint{2.474940in}{2.612517in}}%
\pgfpathlineto{\pgfqpoint{2.477698in}{2.534169in}}%
\pgfpathlineto{\pgfqpoint{2.480456in}{2.612517in}}%
\pgfpathlineto{\pgfqpoint{2.516304in}{2.703924in}}%
\pgfpathlineto{\pgfqpoint{2.519062in}{2.612517in}}%
\pgfpathlineto{\pgfqpoint{2.521819in}{2.703924in}}%
\pgfpathlineto{\pgfqpoint{2.524577in}{2.612517in}}%
\pgfpathlineto{\pgfqpoint{2.527335in}{2.703924in}}%
\pgfpathlineto{\pgfqpoint{2.530092in}{2.612517in}}%
\pgfpathlineto{\pgfqpoint{2.532850in}{2.703924in}}%
\pgfpathlineto{\pgfqpoint{2.535607in}{2.612517in}}%
\pgfpathlineto{\pgfqpoint{2.538365in}{2.703924in}}%
\pgfpathlineto{\pgfqpoint{2.541122in}{2.612517in}}%
\pgfpathlineto{\pgfqpoint{2.543880in}{2.703924in}}%
\pgfpathlineto{\pgfqpoint{2.546638in}{2.612517in}}%
\pgfpathlineto{\pgfqpoint{2.549395in}{2.703924in}}%
\pgfpathlineto{\pgfqpoint{2.552153in}{2.612517in}}%
\pgfpathlineto{\pgfqpoint{2.554910in}{2.703924in}}%
\pgfpathlineto{\pgfqpoint{2.571456in}{2.782272in}}%
\pgfpathlineto{\pgfqpoint{2.574214in}{2.703924in}}%
\pgfpathlineto{\pgfqpoint{2.579729in}{2.782272in}}%
\pgfpathlineto{\pgfqpoint{2.582486in}{2.703924in}}%
\pgfpathlineto{\pgfqpoint{2.585244in}{2.782272in}}%
\pgfpathlineto{\pgfqpoint{2.588001in}{2.703924in}}%
\pgfpathlineto{\pgfqpoint{2.590759in}{2.782272in}}%
\pgfpathlineto{\pgfqpoint{2.593517in}{2.703924in}}%
\pgfpathlineto{\pgfqpoint{2.596274in}{2.782272in}}%
\pgfpathlineto{\pgfqpoint{2.599032in}{2.703924in}}%
\pgfpathlineto{\pgfqpoint{2.601789in}{2.782272in}}%
\pgfpathlineto{\pgfqpoint{2.604547in}{2.703924in}}%
\pgfpathlineto{\pgfqpoint{2.607305in}{2.782272in}}%
\pgfpathlineto{\pgfqpoint{2.610062in}{2.703924in}}%
\pgfpathlineto{\pgfqpoint{2.612820in}{2.782272in}}%
\pgfpathlineto{\pgfqpoint{2.615577in}{2.703924in}}%
\pgfpathlineto{\pgfqpoint{2.618335in}{2.782272in}}%
\pgfpathlineto{\pgfqpoint{2.621093in}{2.703924in}}%
\pgfpathlineto{\pgfqpoint{2.623850in}{2.782272in}}%
\pgfpathlineto{\pgfqpoint{2.626608in}{2.703924in}}%
\pgfpathlineto{\pgfqpoint{2.629365in}{2.782272in}}%
\pgfpathlineto{\pgfqpoint{2.634880in}{2.703924in}}%
\pgfpathlineto{\pgfqpoint{2.637638in}{2.782272in}}%
\pgfpathlineto{\pgfqpoint{2.640396in}{2.703924in}}%
\pgfpathlineto{\pgfqpoint{2.643153in}{2.782272in}}%
\pgfpathlineto{\pgfqpoint{2.645911in}{2.703924in}}%
\pgfpathlineto{\pgfqpoint{2.648668in}{2.782272in}}%
\pgfpathlineto{\pgfqpoint{2.703820in}{2.860621in}}%
\pgfpathlineto{\pgfqpoint{2.706578in}{2.782272in}}%
\pgfpathlineto{\pgfqpoint{2.709335in}{2.860621in}}%
\pgfpathlineto{\pgfqpoint{2.712093in}{2.782272in}}%
\pgfpathlineto{\pgfqpoint{2.714851in}{2.860621in}}%
\pgfpathlineto{\pgfqpoint{2.717608in}{2.782272in}}%
\pgfpathlineto{\pgfqpoint{2.720366in}{2.860621in}}%
\pgfpathlineto{\pgfqpoint{2.723123in}{2.782272in}}%
\pgfpathlineto{\pgfqpoint{2.725881in}{2.860621in}}%
\pgfpathlineto{\pgfqpoint{2.728638in}{2.782272in}}%
\pgfpathlineto{\pgfqpoint{2.731396in}{2.860621in}}%
\pgfpathlineto{\pgfqpoint{2.734154in}{2.782272in}}%
\pgfpathlineto{\pgfqpoint{2.736911in}{2.860621in}}%
\pgfpathlineto{\pgfqpoint{2.739669in}{2.782272in}}%
\pgfpathlineto{\pgfqpoint{2.742426in}{2.860621in}}%
\pgfpathlineto{\pgfqpoint{2.745184in}{2.782272in}}%
\pgfpathlineto{\pgfqpoint{2.747942in}{2.860621in}}%
\pgfpathlineto{\pgfqpoint{2.750699in}{2.782272in}}%
\pgfpathlineto{\pgfqpoint{2.753457in}{2.860621in}}%
\pgfpathlineto{\pgfqpoint{2.758972in}{2.782272in}}%
\pgfpathlineto{\pgfqpoint{2.761729in}{2.860621in}}%
\pgfpathlineto{\pgfqpoint{2.764487in}{2.782272in}}%
\pgfpathlineto{\pgfqpoint{2.767245in}{2.860621in}}%
\pgfpathlineto{\pgfqpoint{2.770002in}{2.782272in}}%
\pgfpathlineto{\pgfqpoint{2.772760in}{2.860621in}}%
\pgfpathlineto{\pgfqpoint{2.827912in}{2.782272in}}%
\pgfpathlineto{\pgfqpoint{2.852730in}{2.051019in}}%
\pgfpathlineto{\pgfqpoint{2.861003in}{1.802915in}}%
\pgfpathlineto{\pgfqpoint{2.863760in}{1.724567in}}%
\pgfpathlineto{\pgfqpoint{2.869275in}{1.633160in}}%
\pgfpathlineto{\pgfqpoint{2.872033in}{1.554811in}}%
\pgfpathlineto{\pgfqpoint{2.874791in}{1.633160in}}%
\pgfpathlineto{\pgfqpoint{2.880306in}{1.476463in}}%
\pgfpathlineto{\pgfqpoint{2.888579in}{1.398114in}}%
\pgfpathlineto{\pgfqpoint{2.894094in}{1.306707in}}%
\pgfpathlineto{\pgfqpoint{2.910639in}{1.150010in}}%
\pgfpathlineto{\pgfqpoint{2.913397in}{1.228359in}}%
\pgfpathlineto{\pgfqpoint{2.916154in}{1.150010in}}%
\pgfpathlineto{\pgfqpoint{2.921670in}{1.071662in}}%
\pgfpathlineto{\pgfqpoint{2.924427in}{1.150010in}}%
\pgfpathlineto{\pgfqpoint{2.927185in}{1.071662in}}%
\pgfpathlineto{\pgfqpoint{2.932700in}{0.980255in}}%
\pgfpathlineto{\pgfqpoint{2.935457in}{1.071662in}}%
\pgfpathlineto{\pgfqpoint{2.938215in}{0.980255in}}%
\pgfpathlineto{\pgfqpoint{2.949245in}{0.901906in}}%
\pgfpathlineto{\pgfqpoint{2.952003in}{0.980255in}}%
\pgfpathlineto{\pgfqpoint{2.954761in}{0.901906in}}%
\pgfpathlineto{\pgfqpoint{2.968549in}{0.823558in}}%
\pgfpathlineto{\pgfqpoint{2.971306in}{0.901906in}}%
\pgfpathlineto{\pgfqpoint{2.974064in}{0.823558in}}%
\pgfpathlineto{\pgfqpoint{2.976821in}{0.901906in}}%
\pgfpathlineto{\pgfqpoint{2.979579in}{0.823558in}}%
\pgfpathlineto{\pgfqpoint{2.996124in}{0.745209in}}%
\pgfpathlineto{\pgfqpoint{2.998882in}{0.823558in}}%
\pgfpathlineto{\pgfqpoint{3.001640in}{0.745209in}}%
\pgfpathlineto{\pgfqpoint{3.004397in}{0.823558in}}%
\pgfpathlineto{\pgfqpoint{3.007155in}{0.745209in}}%
\pgfpathlineto{\pgfqpoint{3.012670in}{0.901906in}}%
\pgfpathlineto{\pgfqpoint{3.018185in}{0.980255in}}%
\pgfpathlineto{\pgfqpoint{3.023700in}{1.150010in}}%
\pgfpathlineto{\pgfqpoint{3.029215in}{1.228359in}}%
\pgfpathlineto{\pgfqpoint{3.031973in}{1.306707in}}%
\pgfpathlineto{\pgfqpoint{3.043003in}{1.476463in}}%
\pgfpathlineto{\pgfqpoint{3.048519in}{1.554811in}}%
\pgfpathlineto{\pgfqpoint{3.051276in}{1.476463in}}%
\pgfpathlineto{\pgfqpoint{3.059549in}{1.724567in}}%
\pgfpathlineto{\pgfqpoint{3.062307in}{1.633160in}}%
\pgfpathlineto{\pgfqpoint{3.067822in}{1.802915in}}%
\pgfpathlineto{\pgfqpoint{3.073337in}{1.881264in}}%
\pgfpathlineto{\pgfqpoint{3.078852in}{1.802915in}}%
\pgfpathlineto{\pgfqpoint{3.084367in}{1.959612in}}%
\pgfpathlineto{\pgfqpoint{3.087125in}{1.881264in}}%
\pgfpathlineto{\pgfqpoint{3.092640in}{2.051019in}}%
\pgfpathlineto{\pgfqpoint{3.095398in}{1.959612in}}%
\pgfpathlineto{\pgfqpoint{3.098155in}{2.051019in}}%
\pgfpathlineto{\pgfqpoint{3.103670in}{2.129368in}}%
\pgfpathlineto{\pgfqpoint{3.106428in}{2.051019in}}%
\pgfpathlineto{\pgfqpoint{3.109185in}{2.129368in}}%
\pgfpathlineto{\pgfqpoint{3.120216in}{2.207716in}}%
\pgfpathlineto{\pgfqpoint{3.122973in}{2.129368in}}%
\pgfpathlineto{\pgfqpoint{3.125731in}{2.207716in}}%
\pgfpathlineto{\pgfqpoint{3.128489in}{2.129368in}}%
\pgfpathlineto{\pgfqpoint{3.131246in}{2.207716in}}%
\pgfpathlineto{\pgfqpoint{3.136761in}{2.286065in}}%
\pgfpathlineto{\pgfqpoint{3.139519in}{2.207716in}}%
\pgfpathlineto{\pgfqpoint{3.142277in}{2.286065in}}%
\pgfpathlineto{\pgfqpoint{3.147792in}{2.207716in}}%
\pgfpathlineto{\pgfqpoint{3.150549in}{2.286065in}}%
\pgfpathlineto{\pgfqpoint{3.158822in}{2.377471in}}%
\pgfpathlineto{\pgfqpoint{3.161580in}{2.286065in}}%
\pgfpathlineto{\pgfqpoint{3.164337in}{2.377471in}}%
\pgfpathlineto{\pgfqpoint{3.167095in}{2.286065in}}%
\pgfpathlineto{\pgfqpoint{3.169852in}{2.377471in}}%
\pgfpathlineto{\pgfqpoint{3.186398in}{2.455820in}}%
\pgfpathlineto{\pgfqpoint{3.189156in}{2.377471in}}%
\pgfpathlineto{\pgfqpoint{3.191913in}{2.455820in}}%
\pgfpathlineto{\pgfqpoint{3.194671in}{2.377471in}}%
\pgfpathlineto{\pgfqpoint{3.197428in}{2.455820in}}%
\pgfpathlineto{\pgfqpoint{3.200186in}{2.377471in}}%
\pgfpathlineto{\pgfqpoint{3.202943in}{2.455820in}}%
\pgfpathlineto{\pgfqpoint{3.216731in}{2.534169in}}%
\pgfpathlineto{\pgfqpoint{3.219489in}{2.455820in}}%
\pgfpathlineto{\pgfqpoint{3.222247in}{2.534169in}}%
\pgfpathlineto{\pgfqpoint{3.225004in}{2.455820in}}%
\pgfpathlineto{\pgfqpoint{3.227762in}{2.534169in}}%
\pgfpathlineto{\pgfqpoint{3.230519in}{2.455820in}}%
\pgfpathlineto{\pgfqpoint{3.233277in}{2.534169in}}%
\pgfpathlineto{\pgfqpoint{3.236035in}{2.455820in}}%
\pgfpathlineto{\pgfqpoint{3.238792in}{2.534169in}}%
\pgfpathlineto{\pgfqpoint{3.244307in}{2.455820in}}%
\pgfpathlineto{\pgfqpoint{3.247065in}{2.534169in}}%
\pgfpathlineto{\pgfqpoint{3.266368in}{2.612517in}}%
\pgfpathlineto{\pgfqpoint{3.269126in}{2.534169in}}%
\pgfpathlineto{\pgfqpoint{3.274641in}{2.612517in}}%
\pgfpathlineto{\pgfqpoint{3.277398in}{2.534169in}}%
\pgfpathlineto{\pgfqpoint{3.280156in}{2.612517in}}%
\pgfpathlineto{\pgfqpoint{3.282914in}{2.534169in}}%
\pgfpathlineto{\pgfqpoint{3.285671in}{2.612517in}}%
\pgfpathlineto{\pgfqpoint{3.288429in}{2.534169in}}%
\pgfpathlineto{\pgfqpoint{3.291186in}{2.612517in}}%
\pgfpathlineto{\pgfqpoint{3.293944in}{2.534169in}}%
\pgfpathlineto{\pgfqpoint{3.296701in}{2.612517in}}%
\pgfpathlineto{\pgfqpoint{3.299459in}{2.534169in}}%
\pgfpathlineto{\pgfqpoint{3.302217in}{2.612517in}}%
\pgfpathlineto{\pgfqpoint{3.332550in}{2.703924in}}%
\pgfpathlineto{\pgfqpoint{3.335308in}{2.612517in}}%
\pgfpathlineto{\pgfqpoint{3.338065in}{2.703924in}}%
\pgfpathlineto{\pgfqpoint{3.340823in}{2.612517in}}%
\pgfpathlineto{\pgfqpoint{3.343580in}{2.703924in}}%
\pgfpathlineto{\pgfqpoint{3.346338in}{2.612517in}}%
\pgfpathlineto{\pgfqpoint{3.349096in}{2.703924in}}%
\pgfpathlineto{\pgfqpoint{3.351853in}{2.612517in}}%
\pgfpathlineto{\pgfqpoint{3.354611in}{2.703924in}}%
\pgfpathlineto{\pgfqpoint{3.357368in}{2.612517in}}%
\pgfpathlineto{\pgfqpoint{3.360126in}{2.703924in}}%
\pgfpathlineto{\pgfqpoint{3.362884in}{2.612517in}}%
\pgfpathlineto{\pgfqpoint{3.365641in}{2.703924in}}%
\pgfpathlineto{\pgfqpoint{3.368399in}{2.612517in}}%
\pgfpathlineto{\pgfqpoint{3.371156in}{2.703924in}}%
\pgfpathlineto{\pgfqpoint{3.373914in}{2.612517in}}%
\pgfpathlineto{\pgfqpoint{3.376671in}{2.703924in}}%
\pgfpathlineto{\pgfqpoint{3.382187in}{2.612517in}}%
\pgfpathlineto{\pgfqpoint{3.384944in}{2.703924in}}%
\pgfpathlineto{\pgfqpoint{3.415278in}{2.782272in}}%
\pgfpathlineto{\pgfqpoint{3.418035in}{2.703924in}}%
\pgfpathlineto{\pgfqpoint{3.420793in}{2.782272in}}%
\pgfpathlineto{\pgfqpoint{3.423550in}{2.703924in}}%
\pgfpathlineto{\pgfqpoint{3.426308in}{2.782272in}}%
\pgfpathlineto{\pgfqpoint{3.429066in}{2.703924in}}%
\pgfpathlineto{\pgfqpoint{3.431823in}{2.782272in}}%
\pgfpathlineto{\pgfqpoint{3.434581in}{2.703924in}}%
\pgfpathlineto{\pgfqpoint{3.437338in}{2.782272in}}%
\pgfpathlineto{\pgfqpoint{3.440096in}{2.703924in}}%
\pgfpathlineto{\pgfqpoint{3.442854in}{2.782272in}}%
\pgfpathlineto{\pgfqpoint{3.445611in}{2.703924in}}%
\pgfpathlineto{\pgfqpoint{3.448369in}{2.782272in}}%
\pgfpathlineto{\pgfqpoint{3.451126in}{2.703924in}}%
\pgfpathlineto{\pgfqpoint{3.453884in}{2.782272in}}%
\pgfpathlineto{\pgfqpoint{3.456642in}{2.703924in}}%
\pgfpathlineto{\pgfqpoint{3.459399in}{2.782272in}}%
\pgfpathlineto{\pgfqpoint{3.462157in}{2.703924in}}%
\pgfpathlineto{\pgfqpoint{3.464914in}{2.782272in}}%
\pgfpathlineto{\pgfqpoint{3.467672in}{2.703924in}}%
\pgfpathlineto{\pgfqpoint{3.470429in}{2.782272in}}%
\pgfpathlineto{\pgfqpoint{3.536612in}{2.860621in}}%
\pgfpathlineto{\pgfqpoint{3.539369in}{2.782272in}}%
\pgfpathlineto{\pgfqpoint{3.544884in}{2.860621in}}%
\pgfpathlineto{\pgfqpoint{3.547642in}{2.782272in}}%
\pgfpathlineto{\pgfqpoint{3.550399in}{2.860621in}}%
\pgfpathlineto{\pgfqpoint{3.553157in}{2.782272in}}%
\pgfpathlineto{\pgfqpoint{3.555915in}{2.860621in}}%
\pgfpathlineto{\pgfqpoint{3.558672in}{2.782272in}}%
\pgfpathlineto{\pgfqpoint{3.561430in}{2.860621in}}%
\pgfpathlineto{\pgfqpoint{3.564187in}{2.782272in}}%
\pgfpathlineto{\pgfqpoint{3.566945in}{2.860621in}}%
\pgfpathlineto{\pgfqpoint{3.569703in}{2.782272in}}%
\pgfpathlineto{\pgfqpoint{3.572460in}{2.860621in}}%
\pgfpathlineto{\pgfqpoint{3.575218in}{2.782272in}}%
\pgfpathlineto{\pgfqpoint{3.577975in}{2.860621in}}%
\pgfpathlineto{\pgfqpoint{3.580733in}{2.782272in}}%
\pgfpathlineto{\pgfqpoint{3.583491in}{2.860621in}}%
\pgfpathlineto{\pgfqpoint{3.586248in}{2.782272in}}%
\pgfpathlineto{\pgfqpoint{3.589006in}{2.860621in}}%
\pgfpathlineto{\pgfqpoint{3.622097in}{2.782272in}}%
\pgfpathlineto{\pgfqpoint{3.627612in}{2.703924in}}%
\pgfpathlineto{\pgfqpoint{3.635885in}{2.455820in}}%
\pgfpathlineto{\pgfqpoint{3.660703in}{1.724567in}}%
\pgfpathlineto{\pgfqpoint{3.663461in}{1.633160in}}%
\pgfpathlineto{\pgfqpoint{3.674491in}{1.476463in}}%
\pgfpathlineto{\pgfqpoint{3.677248in}{1.554811in}}%
\pgfpathlineto{\pgfqpoint{3.682764in}{1.398114in}}%
\pgfpathlineto{\pgfqpoint{3.685521in}{1.476463in}}%
\pgfpathlineto{\pgfqpoint{3.691036in}{1.306707in}}%
\pgfpathlineto{\pgfqpoint{3.696552in}{1.228359in}}%
\pgfpathlineto{\pgfqpoint{3.699309in}{1.306707in}}%
\pgfpathlineto{\pgfqpoint{3.704824in}{1.150010in}}%
\pgfpathlineto{\pgfqpoint{3.707582in}{1.228359in}}%
\pgfpathlineto{\pgfqpoint{3.710340in}{1.150010in}}%
\pgfpathlineto{\pgfqpoint{3.715855in}{1.071662in}}%
\pgfpathlineto{\pgfqpoint{3.718612in}{1.150010in}}%
\pgfpathlineto{\pgfqpoint{3.721370in}{1.071662in}}%
\pgfpathlineto{\pgfqpoint{3.726885in}{0.980255in}}%
\pgfpathlineto{\pgfqpoint{3.729643in}{1.071662in}}%
\pgfpathlineto{\pgfqpoint{3.732400in}{0.980255in}}%
\pgfpathlineto{\pgfqpoint{3.740673in}{0.901906in}}%
\pgfpathlineto{\pgfqpoint{3.743431in}{0.980255in}}%
\pgfpathlineto{\pgfqpoint{3.746188in}{0.901906in}}%
\pgfpathlineto{\pgfqpoint{3.751703in}{0.823558in}}%
\pgfpathlineto{\pgfqpoint{3.754461in}{0.901906in}}%
\pgfpathlineto{\pgfqpoint{3.757219in}{0.823558in}}%
\pgfpathlineto{\pgfqpoint{3.759976in}{0.901906in}}%
\pgfpathlineto{\pgfqpoint{3.762734in}{0.823558in}}%
\pgfpathlineto{\pgfqpoint{3.773764in}{0.745209in}}%
\pgfpathlineto{\pgfqpoint{3.776522in}{0.823558in}}%
\pgfpathlineto{\pgfqpoint{3.779279in}{0.745209in}}%
\pgfpathlineto{\pgfqpoint{3.795825in}{0.653803in}}%
\pgfpathlineto{\pgfqpoint{3.798582in}{0.745209in}}%
\pgfpathlineto{\pgfqpoint{3.804098in}{0.823558in}}%
\pgfpathlineto{\pgfqpoint{3.806855in}{0.901906in}}%
\pgfpathlineto{\pgfqpoint{3.812370in}{0.980255in}}%
\pgfpathlineto{\pgfqpoint{3.815128in}{1.071662in}}%
\pgfpathlineto{\pgfqpoint{3.820643in}{1.150010in}}%
\pgfpathlineto{\pgfqpoint{3.826158in}{1.306707in}}%
\pgfpathlineto{\pgfqpoint{3.831673in}{1.398114in}}%
\pgfpathlineto{\pgfqpoint{3.834431in}{1.476463in}}%
\pgfpathlineto{\pgfqpoint{3.839946in}{1.554811in}}%
\pgfpathlineto{\pgfqpoint{3.848219in}{1.633160in}}%
\pgfpathlineto{\pgfqpoint{3.853734in}{1.724567in}}%
\pgfpathlineto{\pgfqpoint{3.856492in}{1.633160in}}%
\pgfpathlineto{\pgfqpoint{3.862007in}{1.802915in}}%
\pgfpathlineto{\pgfqpoint{3.867522in}{1.881264in}}%
\pgfpathlineto{\pgfqpoint{3.870280in}{1.802915in}}%
\pgfpathlineto{\pgfqpoint{3.875795in}{1.959612in}}%
\pgfpathlineto{\pgfqpoint{3.878552in}{1.881264in}}%
\pgfpathlineto{\pgfqpoint{3.881310in}{1.959612in}}%
\pgfpathlineto{\pgfqpoint{3.886825in}{2.051019in}}%
\pgfpathlineto{\pgfqpoint{3.889583in}{1.959612in}}%
\pgfpathlineto{\pgfqpoint{3.892340in}{2.051019in}}%
\pgfpathlineto{\pgfqpoint{3.897855in}{2.129368in}}%
\pgfpathlineto{\pgfqpoint{3.900613in}{2.051019in}}%
\pgfpathlineto{\pgfqpoint{3.903371in}{2.129368in}}%
\pgfpathlineto{\pgfqpoint{3.911643in}{2.207716in}}%
\pgfpathlineto{\pgfqpoint{3.914401in}{2.129368in}}%
\pgfpathlineto{\pgfqpoint{3.917159in}{2.207716in}}%
\pgfpathlineto{\pgfqpoint{3.930947in}{2.286065in}}%
\pgfpathlineto{\pgfqpoint{3.933704in}{2.207716in}}%
\pgfpathlineto{\pgfqpoint{3.936462in}{2.286065in}}%
\pgfpathlineto{\pgfqpoint{3.939219in}{2.207716in}}%
\pgfpathlineto{\pgfqpoint{3.941977in}{2.286065in}}%
\pgfpathlineto{\pgfqpoint{3.953007in}{2.377471in}}%
\pgfpathlineto{\pgfqpoint{3.955765in}{2.286065in}}%
\pgfpathlineto{\pgfqpoint{3.958522in}{2.377471in}}%
\pgfpathlineto{\pgfqpoint{3.961280in}{2.286065in}}%
\pgfpathlineto{\pgfqpoint{3.964038in}{2.377471in}}%
\pgfpathlineto{\pgfqpoint{3.980583in}{2.455820in}}%
\pgfpathlineto{\pgfqpoint{3.983341in}{2.377471in}}%
\pgfpathlineto{\pgfqpoint{3.986098in}{2.455820in}}%
\pgfpathlineto{\pgfqpoint{3.988856in}{2.377471in}}%
\pgfpathlineto{\pgfqpoint{3.991613in}{2.455820in}}%
\pgfpathlineto{\pgfqpoint{3.997129in}{2.377471in}}%
\pgfpathlineto{\pgfqpoint{3.999886in}{2.455820in}}%
\pgfpathlineto{\pgfqpoint{4.013674in}{2.534169in}}%
\pgfpathlineto{\pgfqpoint{4.016432in}{2.455820in}}%
\pgfpathlineto{\pgfqpoint{4.019189in}{2.534169in}}%
\pgfpathlineto{\pgfqpoint{4.021947in}{2.455820in}}%
\pgfpathlineto{\pgfqpoint{4.024705in}{2.534169in}}%
\pgfpathlineto{\pgfqpoint{4.027462in}{2.455820in}}%
\pgfpathlineto{\pgfqpoint{4.030220in}{2.534169in}}%
\pgfpathlineto{\pgfqpoint{4.032977in}{2.455820in}}%
\pgfpathlineto{\pgfqpoint{4.035735in}{2.534169in}}%
\pgfpathlineto{\pgfqpoint{4.063311in}{2.612517in}}%
\pgfpathlineto{\pgfqpoint{4.066068in}{2.534169in}}%
\pgfpathlineto{\pgfqpoint{4.068826in}{2.612517in}}%
\pgfpathlineto{\pgfqpoint{4.071583in}{2.534169in}}%
\pgfpathlineto{\pgfqpoint{4.074341in}{2.612517in}}%
\pgfpathlineto{\pgfqpoint{4.077099in}{2.534169in}}%
\pgfpathlineto{\pgfqpoint{4.079856in}{2.612517in}}%
\pgfpathlineto{\pgfqpoint{4.082614in}{2.534169in}}%
\pgfpathlineto{\pgfqpoint{4.085371in}{2.612517in}}%
\pgfpathlineto{\pgfqpoint{4.088129in}{2.534169in}}%
\pgfpathlineto{\pgfqpoint{4.090887in}{2.612517in}}%
\pgfpathlineto{\pgfqpoint{4.129493in}{2.703924in}}%
\pgfpathlineto{\pgfqpoint{4.132250in}{2.612517in}}%
\pgfpathlineto{\pgfqpoint{4.135008in}{2.703924in}}%
\pgfpathlineto{\pgfqpoint{4.137766in}{2.612517in}}%
\pgfpathlineto{\pgfqpoint{4.140523in}{2.703924in}}%
\pgfpathlineto{\pgfqpoint{4.143281in}{2.612517in}}%
\pgfpathlineto{\pgfqpoint{4.146038in}{2.703924in}}%
\pgfpathlineto{\pgfqpoint{4.148796in}{2.612517in}}%
\pgfpathlineto{\pgfqpoint{4.151554in}{2.703924in}}%
\pgfpathlineto{\pgfqpoint{4.154311in}{2.612517in}}%
\pgfpathlineto{\pgfqpoint{4.157069in}{2.703924in}}%
\pgfpathlineto{\pgfqpoint{4.159826in}{2.612517in}}%
\pgfpathlineto{\pgfqpoint{4.162584in}{2.703924in}}%
\pgfpathlineto{\pgfqpoint{4.165341in}{2.612517in}}%
\pgfpathlineto{\pgfqpoint{4.168099in}{2.703924in}}%
\pgfpathlineto{\pgfqpoint{4.170857in}{2.612517in}}%
\pgfpathlineto{\pgfqpoint{4.173614in}{2.703924in}}%
\pgfpathlineto{\pgfqpoint{4.176372in}{2.612517in}}%
\pgfpathlineto{\pgfqpoint{4.179129in}{2.703924in}}%
\pgfpathlineto{\pgfqpoint{4.206705in}{2.782272in}}%
\pgfpathlineto{\pgfqpoint{4.209463in}{2.703924in}}%
\pgfpathlineto{\pgfqpoint{4.212220in}{2.782272in}}%
\pgfpathlineto{\pgfqpoint{4.214978in}{2.703924in}}%
\pgfpathlineto{\pgfqpoint{4.217736in}{2.782272in}}%
\pgfpathlineto{\pgfqpoint{4.220493in}{2.703924in}}%
\pgfpathlineto{\pgfqpoint{4.223251in}{2.782272in}}%
\pgfpathlineto{\pgfqpoint{4.226008in}{2.703924in}}%
\pgfpathlineto{\pgfqpoint{4.228766in}{2.782272in}}%
\pgfpathlineto{\pgfqpoint{4.231524in}{2.703924in}}%
\pgfpathlineto{\pgfqpoint{4.234281in}{2.782272in}}%
\pgfpathlineto{\pgfqpoint{4.237039in}{2.703924in}}%
\pgfpathlineto{\pgfqpoint{4.239796in}{2.782272in}}%
\pgfpathlineto{\pgfqpoint{4.242554in}{2.703924in}}%
\pgfpathlineto{\pgfqpoint{4.245311in}{2.782272in}}%
\pgfpathlineto{\pgfqpoint{4.248069in}{2.703924in}}%
\pgfpathlineto{\pgfqpoint{4.250827in}{2.782272in}}%
\pgfpathlineto{\pgfqpoint{4.256342in}{2.703924in}}%
\pgfpathlineto{\pgfqpoint{4.259099in}{2.782272in}}%
\pgfpathlineto{\pgfqpoint{4.264615in}{2.703924in}}%
\pgfpathlineto{\pgfqpoint{4.267372in}{2.782272in}}%
\pgfpathlineto{\pgfqpoint{4.319766in}{2.860621in}}%
\pgfpathlineto{\pgfqpoint{4.322524in}{2.782272in}}%
\pgfpathlineto{\pgfqpoint{4.328039in}{2.860621in}}%
\pgfpathlineto{\pgfqpoint{4.330797in}{2.782272in}}%
\pgfpathlineto{\pgfqpoint{4.333554in}{2.860621in}}%
\pgfpathlineto{\pgfqpoint{4.336312in}{2.782272in}}%
\pgfpathlineto{\pgfqpoint{4.339069in}{2.860621in}}%
\pgfpathlineto{\pgfqpoint{4.341827in}{2.782272in}}%
\pgfpathlineto{\pgfqpoint{4.344585in}{2.860621in}}%
\pgfpathlineto{\pgfqpoint{4.347342in}{2.782272in}}%
\pgfpathlineto{\pgfqpoint{4.350100in}{2.860621in}}%
\pgfpathlineto{\pgfqpoint{4.352857in}{2.782272in}}%
\pgfpathlineto{\pgfqpoint{4.355615in}{2.860621in}}%
\pgfpathlineto{\pgfqpoint{4.358373in}{2.782272in}}%
\pgfpathlineto{\pgfqpoint{4.361130in}{2.860621in}}%
\pgfpathlineto{\pgfqpoint{4.363888in}{2.782272in}}%
\pgfpathlineto{\pgfqpoint{4.366645in}{2.860621in}}%
\pgfpathlineto{\pgfqpoint{4.369403in}{2.782272in}}%
\pgfpathlineto{\pgfqpoint{4.372161in}{2.860621in}}%
\pgfpathlineto{\pgfqpoint{4.374918in}{2.782272in}}%
\pgfpathlineto{\pgfqpoint{4.377676in}{2.860621in}}%
\pgfpathlineto{\pgfqpoint{4.380433in}{2.782272in}}%
\pgfpathlineto{\pgfqpoint{4.383191in}{2.860621in}}%
\pgfpathlineto{\pgfqpoint{4.385948in}{2.782272in}}%
\pgfpathlineto{\pgfqpoint{4.388706in}{2.860621in}}%
\pgfpathlineto{\pgfqpoint{4.391464in}{2.782272in}}%
\pgfpathlineto{\pgfqpoint{4.394221in}{2.860621in}}%
\pgfpathlineto{\pgfqpoint{4.424555in}{2.782272in}}%
\pgfpathlineto{\pgfqpoint{4.449373in}{2.051019in}}%
\pgfpathlineto{\pgfqpoint{4.457646in}{1.802915in}}%
\pgfpathlineto{\pgfqpoint{4.463161in}{1.633160in}}%
\pgfpathlineto{\pgfqpoint{4.479706in}{1.398114in}}%
\pgfpathlineto{\pgfqpoint{4.487979in}{1.306707in}}%
\pgfpathlineto{\pgfqpoint{4.490737in}{1.398114in}}%
\pgfpathlineto{\pgfqpoint{4.493494in}{1.306707in}}%
\pgfpathlineto{\pgfqpoint{4.504525in}{1.150010in}}%
\pgfpathlineto{\pgfqpoint{4.507282in}{1.228359in}}%
\pgfpathlineto{\pgfqpoint{4.510040in}{1.150010in}}%
\pgfpathlineto{\pgfqpoint{4.515555in}{1.071662in}}%
\pgfpathlineto{\pgfqpoint{4.518313in}{1.150010in}}%
\pgfpathlineto{\pgfqpoint{4.521070in}{1.071662in}}%
\pgfpathlineto{\pgfqpoint{4.526585in}{0.980255in}}%
\pgfpathlineto{\pgfqpoint{4.529343in}{1.071662in}}%
\pgfpathlineto{\pgfqpoint{4.532101in}{0.980255in}}%
\pgfpathlineto{\pgfqpoint{4.540373in}{0.901906in}}%
\pgfpathlineto{\pgfqpoint{4.543131in}{0.980255in}}%
\pgfpathlineto{\pgfqpoint{4.545889in}{0.901906in}}%
\pgfpathlineto{\pgfqpoint{4.556919in}{0.823558in}}%
\pgfpathlineto{\pgfqpoint{4.559676in}{0.901906in}}%
\pgfpathlineto{\pgfqpoint{4.562434in}{0.823558in}}%
\pgfpathlineto{\pgfqpoint{4.565192in}{0.901906in}}%
\pgfpathlineto{\pgfqpoint{4.567949in}{0.823558in}}%
\pgfpathlineto{\pgfqpoint{4.576222in}{0.745209in}}%
\pgfpathlineto{\pgfqpoint{4.578980in}{0.823558in}}%
\pgfpathlineto{\pgfqpoint{4.581737in}{0.745209in}}%
\pgfpathlineto{\pgfqpoint{4.595525in}{0.653803in}}%
\pgfpathlineto{\pgfqpoint{4.603798in}{0.901906in}}%
\pgfpathlineto{\pgfqpoint{4.609313in}{0.980255in}}%
\pgfpathlineto{\pgfqpoint{4.614828in}{1.150010in}}%
\pgfpathlineto{\pgfqpoint{4.620343in}{1.228359in}}%
\pgfpathlineto{\pgfqpoint{4.623101in}{1.306707in}}%
\pgfpathlineto{\pgfqpoint{4.634131in}{1.476463in}}%
\pgfpathlineto{\pgfqpoint{4.650677in}{1.724567in}}%
\pgfpathlineto{\pgfqpoint{4.658950in}{1.802915in}}%
\pgfpathlineto{\pgfqpoint{4.661707in}{1.724567in}}%
\pgfpathlineto{\pgfqpoint{4.667222in}{1.881264in}}%
\pgfpathlineto{\pgfqpoint{4.672738in}{1.959612in}}%
\pgfpathlineto{\pgfqpoint{4.675495in}{1.881264in}}%
\pgfpathlineto{\pgfqpoint{4.678253in}{1.959612in}}%
\pgfpathlineto{\pgfqpoint{4.683768in}{2.051019in}}%
\pgfpathlineto{\pgfqpoint{4.689283in}{1.959612in}}%
\pgfpathlineto{\pgfqpoint{4.694798in}{2.129368in}}%
\pgfpathlineto{\pgfqpoint{4.697556in}{2.051019in}}%
\pgfpathlineto{\pgfqpoint{4.700313in}{2.129368in}}%
\pgfpathlineto{\pgfqpoint{4.711344in}{2.207716in}}%
\pgfpathlineto{\pgfqpoint{4.714101in}{2.129368in}}%
\pgfpathlineto{\pgfqpoint{4.716859in}{2.207716in}}%
\pgfpathlineto{\pgfqpoint{4.727889in}{2.286065in}}%
\pgfpathlineto{\pgfqpoint{4.730647in}{2.207716in}}%
\pgfpathlineto{\pgfqpoint{4.733404in}{2.286065in}}%
\pgfpathlineto{\pgfqpoint{4.738920in}{2.207716in}}%
\pgfpathlineto{\pgfqpoint{4.741677in}{2.286065in}}%
\pgfpathlineto{\pgfqpoint{4.747192in}{2.377471in}}%
\pgfpathlineto{\pgfqpoint{4.749950in}{2.286065in}}%
\pgfpathlineto{\pgfqpoint{4.752708in}{2.377471in}}%
\pgfpathlineto{\pgfqpoint{4.755465in}{2.286065in}}%
\pgfpathlineto{\pgfqpoint{4.758223in}{2.377471in}}%
\pgfpathlineto{\pgfqpoint{4.760980in}{2.286065in}}%
\pgfpathlineto{\pgfqpoint{4.763738in}{2.377471in}}%
\pgfpathlineto{\pgfqpoint{4.783041in}{2.455820in}}%
\pgfpathlineto{\pgfqpoint{4.785799in}{2.377471in}}%
\pgfpathlineto{\pgfqpoint{4.788556in}{2.455820in}}%
\pgfpathlineto{\pgfqpoint{4.791314in}{2.377471in}}%
\pgfpathlineto{\pgfqpoint{4.794071in}{2.455820in}}%
\pgfpathlineto{\pgfqpoint{4.796829in}{2.377471in}}%
\pgfpathlineto{\pgfqpoint{4.799587in}{2.455820in}}%
\pgfpathlineto{\pgfqpoint{4.802344in}{2.377471in}}%
\pgfpathlineto{\pgfqpoint{4.805102in}{2.455820in}}%
\pgfpathlineto{\pgfqpoint{4.821647in}{2.534169in}}%
\pgfpathlineto{\pgfqpoint{4.824405in}{2.455820in}}%
\pgfpathlineto{\pgfqpoint{4.827162in}{2.534169in}}%
\pgfpathlineto{\pgfqpoint{4.829920in}{2.455820in}}%
\pgfpathlineto{\pgfqpoint{4.832678in}{2.534169in}}%
\pgfpathlineto{\pgfqpoint{4.835435in}{2.455820in}}%
\pgfpathlineto{\pgfqpoint{4.838193in}{2.534169in}}%
\pgfpathlineto{\pgfqpoint{4.840950in}{2.455820in}}%
\pgfpathlineto{\pgfqpoint{4.843708in}{2.534169in}}%
\pgfpathlineto{\pgfqpoint{4.860253in}{2.612517in}}%
\pgfpathlineto{\pgfqpoint{4.860253in}{2.612517in}}%
\pgfusepath{stroke}%
\end{pgfscope}%
\begin{pgfscope}%
\pgfsetrectcap%
\pgfsetmiterjoin%
\pgfsetlinewidth{0.803000pt}%
\definecolor{currentstroke}{rgb}{0.000000,0.000000,0.000000}%
\pgfsetstrokecolor{currentstroke}%
\pgfsetdash{}{0pt}%
\pgfpathmoveto{\pgfqpoint{0.693677in}{0.539544in}}%
\pgfpathlineto{\pgfqpoint{0.693677in}{3.053228in}}%
\pgfusepath{stroke}%
\end{pgfscope}%
\begin{pgfscope}%
\pgfsetrectcap%
\pgfsetmiterjoin%
\pgfsetlinewidth{0.803000pt}%
\definecolor{currentstroke}{rgb}{0.000000,0.000000,0.000000}%
\pgfsetstrokecolor{currentstroke}%
\pgfsetdash{}{0pt}%
\pgfpathmoveto{\pgfqpoint{5.058662in}{0.539544in}}%
\pgfpathlineto{\pgfqpoint{5.058662in}{3.053228in}}%
\pgfusepath{stroke}%
\end{pgfscope}%
\begin{pgfscope}%
\pgfsetrectcap%
\pgfsetmiterjoin%
\pgfsetlinewidth{0.803000pt}%
\definecolor{currentstroke}{rgb}{0.000000,0.000000,0.000000}%
\pgfsetstrokecolor{currentstroke}%
\pgfsetdash{}{0pt}%
\pgfpathmoveto{\pgfqpoint{0.693677in}{0.539544in}}%
\pgfpathlineto{\pgfqpoint{5.058662in}{0.539544in}}%
\pgfusepath{stroke}%
\end{pgfscope}%
\begin{pgfscope}%
\pgfsetrectcap%
\pgfsetmiterjoin%
\pgfsetlinewidth{0.803000pt}%
\definecolor{currentstroke}{rgb}{0.000000,0.000000,0.000000}%
\pgfsetstrokecolor{currentstroke}%
\pgfsetdash{}{0pt}%
\pgfpathmoveto{\pgfqpoint{0.693677in}{3.053228in}}%
\pgfpathlineto{\pgfqpoint{5.058662in}{3.053228in}}%
\pgfusepath{stroke}%
\end{pgfscope}%
\begin{pgfscope}%
\pgfsetbuttcap%
\pgfsetmiterjoin%
\definecolor{currentfill}{rgb}{1.000000,1.000000,1.000000}%
\pgfsetfillcolor{currentfill}%
\pgfsetfillopacity{0.800000}%
\pgfsetlinewidth{1.003750pt}%
\definecolor{currentstroke}{rgb}{0.800000,0.800000,0.800000}%
\pgfsetstrokecolor{currentstroke}%
\pgfsetstrokeopacity{0.800000}%
\pgfsetdash{}{0pt}%
\pgfpathmoveto{\pgfqpoint{0.771455in}{2.809450in}}%
\pgfpathlineto{\pgfqpoint{2.105788in}{2.809450in}}%
\pgfpathquadraticcurveto{\pgfqpoint{2.128010in}{2.809450in}}{\pgfqpoint{2.128010in}{2.831672in}}%
\pgfpathlineto{\pgfqpoint{2.128010in}{2.975450in}}%
\pgfpathquadraticcurveto{\pgfqpoint{2.128010in}{2.997672in}}{\pgfqpoint{2.105788in}{2.997672in}}%
\pgfpathlineto{\pgfqpoint{0.771455in}{2.997672in}}%
\pgfpathquadraticcurveto{\pgfqpoint{0.749232in}{2.997672in}}{\pgfqpoint{0.749232in}{2.975450in}}%
\pgfpathlineto{\pgfqpoint{0.749232in}{2.831672in}}%
\pgfpathquadraticcurveto{\pgfqpoint{0.749232in}{2.809450in}}{\pgfqpoint{0.771455in}{2.809450in}}%
\pgfpathlineto{\pgfqpoint{0.771455in}{2.809450in}}%
\pgfpathclose%
\pgfusepath{stroke,fill}%
\end{pgfscope}%
\begin{pgfscope}%
\pgfsetrectcap%
\pgfsetroundjoin%
\pgfsetlinewidth{0.501875pt}%
\definecolor{currentstroke}{rgb}{0.121569,0.466667,0.705882}%
\pgfsetstrokecolor{currentstroke}%
\pgfsetstrokeopacity{0.700000}%
\pgfsetdash{}{0pt}%
\pgfpathmoveto{\pgfqpoint{0.793677in}{2.914339in}}%
\pgfpathlineto{\pgfqpoint{0.904788in}{2.914339in}}%
\pgfpathlineto{\pgfqpoint{1.015899in}{2.914339in}}%
\pgfusepath{stroke}%
\end{pgfscope}%
\begin{pgfscope}%
\definecolor{textcolor}{rgb}{0.000000,0.000000,0.000000}%
\pgfsetstrokecolor{textcolor}%
\pgfsetfillcolor{textcolor}%
\pgftext[x=1.104788in,y=2.875450in,left,base]{\color{textcolor}\rmfamily\fontsize{8.000000}{9.600000}\selectfont Room temperature}%
\end{pgfscope}%
\end{pgfpicture}%
\makeatother%
\endgroup%

    \caption{Temperature in Lab 011 on 2016-11-26.}
    \label{fig:lab_temperature_start_of_project}
\end{figure}

As it can be seen there are strong oszillations of the temperature as a result of the on–off air conditioning temperature controller. The commercial controller used back then was realized using an IMI Heimeier EMO T Valve \cite{datasheet_heimeier_emo_t}, which is a 2 step valve. Altough this solution was later replaced by a custom design described in section \ref{}, these type controllers are found in many other labs and temperature swings of \qty{2}{\kelvin} must therefore be expected.

These expected environmental parameters can now be used to estimate the design requirements for the laser driver. The more demanding laser system is the \qty{450}{\nm} system \cite{thesis_baus} required for the spectroscopy of highly charged ions \cite{thesis_alex} at GSI. This system was found to be more susceptible to changes of the drive current since the wavelength selective filter element was far broader in comparison to a \qty{780}{\nm} system \cite{two_filter_paper}. This laser is stable over regions of tens of \unit{\uA} and requires a maximum drive current of \qty{145}{\mA} \cite{datasheet_osram_pl450b}.

From these thoughts, the requirements for the driver can be inferred. It should be able to supply at least \qty{150}{\mA} and stay well within \qty{10}{\uA} over the whole environmental range. For a worst-case assumption a tolerance of $3\sigma$ (\qty{99.7}{\percent}) must be met \cite{worst_case_design}.

The environmental parameters that mostly affect current sources are temperature and humidity. Air pressure is typically a matter of concern for high voltage systems \cite{IPC-2221B} and secondary to consider for this design as it is a low voltage system (\qty{<= 48}{\V}). Air pressure effects are also the most expensive to test for, as a pressure chamber is required. While humidity does affect electronics due to corrosion and also indirectly because the epoxy resin used in the FR-4 PCBs and component moulding is hygroscopic and the absorbed humidty leads to swelling and mechanical stress. This effect is very slow at ambient temperature and can easily take days to show \cite{epoxy_humidity}. This parameter is therefore handled via the long-term stability and not specified separately.

Given environmental conditions, the relative coefficients can be calculated. This estimation assumes a minimum setpoint resolution of 2 steps within the mode-hop-free region of the laser and calculates the \qty{99.7}{\percent} confidence interval. The steps are given in table \ref{tab:dgdrive_tempco}:

\begin{table}[hb]
    \centering
    \begin{tabular}{llr}
        Property& Value& Result \\
        \midrule
        Stable range & \qty{10}{\uA}& \qty{10}{\uA}\\
        2 steps of resolution  & $\div 2$& \qty{5}{\uA} \\
        $1 \sigma$  & $\div 2$& \qty{2.5}{\uA} \\
        Maximum output& \qty{150}{\mA}& \qty{17}{\uA \per \A}\\
        Temperature range& \qty{5}{\K}& \qty{3}{\uA \per \A \per \K}\\
        Worst case ($3 \sigma$)& $\div 3$& \qty{1}{\uA \per \A \per \K}\\
    \end{tabular}
    \caption{Estimated requirement for the temperature coefficient of the laser driver.}
    \label{tab:dgdrive_tempco}
\end{table}

While the requirements look moderate at first sight, doing a quick estimation leeds to a temperature coeffiction of\qty[per-mode = symbol]{1}{\uA \per \A \per \K}, preferably better than that when using a higher output driver -- a rather formidable specification for a current source.

Regarding the long-term stability, a \qty{30}{\day} figure can be estimated. One may be inclined to call for a drift, that is smaller than the stable range, but this would be short sighted, as there are other factors to consider. The laser including the external resonator has its own figure of merit regarding the spectral drift rate. \citeauthor{ecdl_stability} \cite{ecdl_stability} reported a drift of \qty{2.9}{\MHz \per \hour}, which was attributed either to the external resonator itself, the piezo or the collimation lens. It is most likely, that this drift was caused by mechanical changes of the external resonator as it defines the output mode of the laser. The mechanical drift limits the required stability of the current source considerably, as a typical frequency change of the internal resontor with the current of \qty[per-mode=symbol]{3}{\MHz \per \micro \A} \cite{diodelaser_modulation} can be assumed. The (linear) ageing drift of the external resonator over \qty{30}{\day} is equivalent to a \qty{720}{\uA} drift over the same period. For the electronics, the drift is assumed to follow an Arrhenius-like equation resulting from stress induced during manufacturing. This may eventually change to a slow linear drift after several months of relaxation. The coefficient may either be a positive or negative and leads to

\begin{table}[hb]
    \centering
    \begin{tabular}{llr}
        Property& Value& Result \\
        \midrule
        Ageing drift limit & \qty{720}{\uA}& \qty{720}{\uA}\\
        $1 \sigma$  & $\div 2$& \qty{360}{\uA} \\
        Maximum output& \qty{150}{\mA}& \qty{2400}{\uA \per \A}\\
        Worst case ($3 \sigma$)& $\div 3$& \qty{800}{\uA \per \A}\\
    \end{tabular}
    \caption{Estimated requirement for the long-term stability of the laser driver.}
    \label{tab:dgdrive_stability}
\end{table}

From these number, it straightforward to see, that the long-term stability of a laser driver is less important than the short-term temperature coefficient since the limiting factor is the mechanical construction of the laser. This necessitates an atomic reference for long-term stability and to compensate for accoustic resonances of the external resonator. Regarding the choice of suitable devices, the tight specification of the temperature coefficient most likely leads to a choice of components, that will pass these long-term criteria as well, alleviating the burden of proof a bit as long-term drift specifications are hard to come by since they literally need solid time to come by and cannot be extrapolated from high temperature burn-in tests \cite{voltage_reference_drift}.

%While \qty{800}{\uA \per \A} over a \qty{30}{\day} period may seem large at first, it is actually very hard to accurately produce a current. To put this number in perspective, a commercial high-end current source like the Keithley \device{2600B} is specified for the \qty{100}{\mA} range at about \qty{170}{\uA \per \A} for a \qty{30}{\day} period when calcuated from the 1-year specification \cite{datasheet_keithley2600}, again assuming an Arrhenius-like equation as the basis.

This leads to the following design specifications regarding the stabilty of the current driver:

\begin{center}
    \begin{tcolorbox}[
        new/auto counter,
        new/number within=chapter,
        colback=red!5!white,
        colframe=red!75!black,
        title=Current controller stability specifications,
        width=0.8\linewidth,
        label={lst:dgDrive_specs_environment}
    ]
    \begin{itemize}
        \item Temperature range \qtyrange[text-series-to-math, reset-text-series = false, reset-math-version = false]{20}{25}{\celsius}
        \item \textbf{Temperature coefficient \qty[text-series-to-math, reset-text-series = false, reset-math-version = false]{<= 1}{\uA \per \A \per \K}}
        \item Humidty (non-condensing) \qty{<= 75}{\percent rH}
        \item Humidty coefficient not specified, but included in the long-term drift
        \item Maximium altitude not specified
        \item Long-term drift over \qty{30}{\day} \qty{<= 800}{\uA \per \A}
    \end{itemize}
    \end{tcolorbox}
\end{center}

%A basic laser current driver design, that has some of the  can be found in the work of \citeauthor{libbrecht_hall} \cite{libbrecht_hall}. While this design contains all the basic features, like a current source, a modulation input and a voltage limit, there are several shortcomings that have emerged over the years with new generations of laser didoes. The laser driver used by legacy applications in this group is based on the aforementioned paper and has been successfully employed in several projects over the years, but several limitations have come up in recent years. In order to derive the design requirements of a new generation of laser drivers the important design elements need to be identified first. The essential design elements are the bulk current source, the modulation current source, the reference element and output programming. The next 4 sections will deal with each element and outline the design goals that were identified while employing the legacy generation of diode drivers in several experiments.

\clearpage
\subsection{Design Goals: Current Source}
% https://www.laserdiodecontrol.com/laser-diode-parameter-overview
% Diode Lasers and Photonic Integrated Circuits (Characteristic temperature)
% Near Threshold Operation of Semiconductor Lasers and Resonant-Type Laser Amplifiers
The change in output current caused by load impedance should be an order of magnitude less than the drift specification to ensure a negligible effect compared to the drift over time. The load resistance presented by the laser diodes most commonly used in our experiments ranges from \qty{50}{\ohm} \cite{datasheet_osram_pl450b} to \qty{30}{\ohm} \cite{datasheet_adl_785} and \qtyrange{10}{15}{\ohm} for \qty{780}{\nm} laser diode \cite{datasheet_sharp_780nm,datasheet_thorlabs_780nm}. It can therefore be estimated as
\begin{align}
    \frac{R_{load}}{R_{out}} &= \frac{I_{set}}{I_{out}} - 1 \leq \qty[per-mode = symbol]{6.7}{\uA \per \A} \nonumber\\
    R_{out} &\geq \frac{\qty{50}{\ohm}}{\qty[per-mode = symbol]{6.7}{\uA \per \A}} = \qty{7.5}{\mega \ohm}
\end{align}

An output impedance of more than \qty{7.5}{\mega \ohm} for slowly changing loads is a fairly moderate requirement and can typically be realised using a high precision control loop with an operational amplifier, but another important aspect is the compliance voltage of the current source.

The compliance voltage is the maximum voltage the current source can apply to the load and is another non-ideal component of a real current source. The required voltage strongly depends on the type of laser diode used. The near-inrared laser diodes discussed above have an operating voltage of \qtyrange{1.5}{3}{\V}, while the Osram \device{PL 450B} blue laser diode is specified for \qtyrange{5.5}{7}{\V}. The \qty{7}{\V} required by the Osram laser diode is fairly high for a Fabry–Perot laser diode and has proven difficult in the past \cite{thesis_baus} as most laser current driver available are designed for the much forward voltage of the near infrad laser diodes. Even higher voltages of around \qtyrange{12}{15}{\V} are required for quantum cascade lasers, but these are currently neither used nor is their use planned in our experiments.

The maximum output current of the laser driver currently required for laser diodes used in our group is \qty{250}{\mA} for the Thorlabs \device{L785H1} \cite{datasheet_thorlabs_780nm}. Therefore a maximum output current of \qty{300}{\mA} is considered sufficient.

The current noise of the laser driver can be estimated from the laser linewidth sought as the laser frequency is sensitive to the injection current. At low frequencies, about \qty[per-mode=symbol]{-3}{\MHz \per \micro \A} can be attributed to the thermal expansion of the internal resonator of the diode due resistive heating \cite{diodelaser_modulation}. Above \qty{1}{\MHz} this effect starts declining and exposes the change of the refractive index due the presence of charge carriers. This second effect is an order of magnitude weaker. Since the frequency sensitivity to current variations of the laser diode drops with higher frequencies the most important range is from DC to \qty{100}{\kHz}.

To estimate the linewidth requirement, it is important to look at the experimental setup. While the spectroscopy of \ce{Ar^13+} at \qty{4}{\K} is limited to around \qty{150}{\MHz}  as shown on page \pageref{eqn:doppler_broadening}, the quantum computing experiments in our group have more stringent needs. It was shown in \cite{ecdl_stability, ecdl_silicone_housing,ecdl_linewidth_scholten}, that with reasonable expense a passive linewidth of less than \qty{100}{\kHz} can be achieved. Using the frequency sensitivity to a current modulation of laser diodes \qty{100}{\kHz} translates to a current noise of \qty{30}{\nA_{rms}} from \qty{1}{\Hz} to \qty{100}{\kHz}. The lower \qty{1}{\Hz} limit is chosen fairly arbitrary, but the presence of $\frac 1 f$-noise inhibits a definition down to DC. There should be negligible amounts noise below \qty{1}{\Hz} compared to the upper \qty{100}{\kHz} though.

The final aspect of the current source, that needs to be specified, is the bandwidth of the current steering input. The bandwidth in these terms define a reasonably flat (\qty{\leq 3}{\dB}) response. As it was discussed above, beyond a frequency of \qty{1}{\MHz}, the frequency sensitivity of the laser diode to current modulation drops by an order of magnitude, altering the transfer function and introducing new challenges for control loops. Therefore a bandwidth of \qty{1}{\MHz} or more is considered sufficient.

Above \qty{1}{\MHz} it is recommended to either use more dedicated solutions like the direct modulation at the laser head presented in \cite{current_mod_paper} or switch to acousto-optic modulators (AOMs) or electro-optic modulators (EOMs).

This leads to the following requirements regarding the current source of the laser driver:

\begin{center}
    \begin{tcolorbox}[
        new/auto counter,
        new/number within=chapter,
        colback=red!5!white,
        colframe=red!75!black,
        title=Current source specifications,
        width=0.8\linewidth,
        label={lst:dgDrive_specs_electrical}
        ]
    \begin{itemize}
        \item Maximum output current \qty{300}{\mA}, optionally \qty{500}{\mA}
        \item \textbf{Compliance voltage \qty[text-series-to-math, reset-text-series = false, reset-math-version = false]{\geq 8}{\V}}
        \item Output impedance \qty{\geq 7.5}{\mega \ohm} at low frequencies (close to DC)
        \item \textbf{Current noise \qty[text-series-to-math, reset-text-series = false, reset-math-version = false]{\leq 30}{\nA_{rms}} from DC to \qty[text-series-to-math, reset-text-series = false, reset-math-version = false]{100}{\kHz}}
        \item \qty{3}{\dB}-bandwidth of the modulation source \qty{\geq 1}{\MHz}
    \end{itemize}
    \end{tcolorbox}
\end{center}

\clearpage
\subsection{Design Goals: User Interface and Form Factor}
The user interface must be remote controllable, as the Penning trap and the laser system is spacially separated with the laser system being located in a special laser lab for environmental and safety reasons. The spatial separation is about \qty{30}{\meter}. Ideally this remote interface is computer controlled to give full access to all features of the laser system. USB or Ethernet is preferred as this does not require extra hardware in the lab.

Regarding the application programming interface (API) support for both Labview and Python, with a strong tendency to Pyhton is favoured. The reason is, that most of the group has switched from Labview to labscript suite \cite{labscript_2013} on Python to run the experiments.

The local interface must be accessible without a computer to allow simple adjustment of the parameters while on the bench.

The form factor should allow integration into standard 19-inch racks to allow simple transportation from the experiment site at GSI to the university for testing and calibration.

\begin{center}
    \begin{tcolorbox}[
        new/auto counter,
        new/number within=chapter,
        colback=red!5!white,
        colframe=red!75!black,
        title=Current source user interface and form factor,
        width=0.8\linewidth,
        label={lst:dgDrive_form_factor}
        ]
    \begin{itemize}
        \item Remote computer interface required
        \item Python and optionally Labview drivers
        \item Rack mountable form factor preferred
    \end{itemize}
    \end{tcolorbox}
\end{center}

\clearpage
\section{LabKraken}
\subsection{Design Goals}
LabKraken is a designed to be a asynchronous, resilient data aquisition suite, that scales to thousands of sensors and accross different networks.
\subsection{Hardware}
\subsection{Software Architecture}
LabKraken needs to scale to thousands of sensors, which need to be served concurrently. This problem is commonly referered to as the C10K problem as dubbed by Dan Kegel back in 1999 \cite{10kProblem} and refers to serving \num{10000} concurrent connections via network sockets. While today millions of concurrent connections can be handled by servers, handling \num{10000} can still be challenging, especially, if the data sources are heterogeneous as is typical for sensor networks of different sensors from different manufacturers.

In order to meet the design goals, an asynchronous architecture was chosen and several different architectures were implemented over time. All in all four complete rewrites of the software were made to arrive at the architecture presented here. The reason for the rewrites is mostly historic and can be explained by the history of the programming language Python, which was used to write the code. The first first version was written for Python 2.6 and exclusively supported sensors made Tinkerforge. In 2015, Python 3.5 was released, which supported a new syntax for asynchronous coroutines. The software was rewritten from scratch to support this new syntax, because it made the code a lot more verbose and easier to follow. With the release of Python 3.7 in 2018 asynchronous generator expressions where mature enough to be used in productions and the programm was again rewritten to use the new syntax. In 2021 a new approach was taken and the programm was once more rewritten with a functional programming style. I will discuss each approach in the next sections to highlight the improvements, that were made over time. Each of these sections discusses the same programm, but written in different styles to show the differences.

\subsubsection{Threaded Design}
The first version of LabKraken used a threaded design approach, because the original libraries of the Tinkerforge sensors are built around threads. The following simplified example shows some code to connect to a temperature sensor over the network and read its data.

\inputpython{source/lab_kraken_threads.py}{1}{26}

\subsubsection{Device Identifiers}
Every sensor network needs device identifiers. Preferably those identifiers should be unique. Typically a device has some kind of internal indetifier. Here are a few examples of the sensors used in our network:

\begin{table}[ht]
\centering
\begin{tabularx}{0.95\textwidth}{|l|p{6.5cm}|X|}
    \hline
    Device Type& Identifiers& Example\\
    \hline
    GPIB (SCPI)& \textit{*IDN?} returns \newline \$manufacturer,\$name,\$serial,\$revision& \\
    \hline
    Tinkerforge& Each sensor has a base58 encoded integer device id& QE9 (163684)\\
    \hline
    Labnode& Universal Unique Identifier (UUID) & cc2f2159-e2fb-4ed9-\newline8021-7771890b37ad\\
    \hline
\end{tabularx}
\end{table}

As it can be seen above, these identifiers do not guarantee to uniquely identify a device within a network. The Tinkerforge id is the weakest, as it is a \qty{32}{\bit} integer (4.294.967.295 options), which might easily collide with another id from a different manufacturer. The tinkerforge id is presented as a base58 encoded string. An encoder/decoder example can be found in the TinkerforgeAsync library \cite{TinkerforgeAsync}.

The id string returned by a SCPI device is slightly better, but again does not guarantee uniqueness. As it is shown in the example the same device might return a different id defpending on its settings. This typically done by manufacturers for compatibility reasons.

The only reasonably unique id is the universal unique identifier (UUID) or globally unique identifier (GUID), as dubbed by Microsoft, used in the Labnodes. Their id can be used for networks with participant numbers going into the millions.

Calculating the probability of a collision between two random UUIDs is called the birthday problem \cite{BirthdayProblem} in probability theory. A randomly generated version 4 UUID of variant 1 as defined in RFC 4122 \cite{RFC-UUID} has \qty{122}{\bit} of entropy, that is out of \qty{128}{\bit}, \qty{4}{\bit} are reserved for the UUID version and \qty{2}{\bit} for the variant. This gives the probability of at least one collision in $n$ devices out of $M = 2^{122}$ possibilities:
\begin{align}
    p(n) &= 1 - 1 \cdot \left(1 - \frac{1}{M}\right) \cdot \left(1 - \frac{2}{M}\right) \dots \left(1 - \frac{n-1}{M}\right) \nonumber\\
    &= 1 - \prod_{k=1}^{n-1} \left(1 - \frac{k}{M} \right)
\end{align}
Using the Taylor series $e^x = 1+x \dots$, assuming $n \ll M$ and approximating we can simplify this to:
\begin{align}
    p(n) &\approx 1 - \left(e^\frac{-1}{M} \cdot e^\frac{-2}{M} \dots e^\frac{-(n-1)}{M} \right) \nonumber\\
    &\approx 1 - \left(e^\frac{-n(n-1)/2}{M} \right) \nonumber\\
    &\approx 1 - \left(1 - \frac{n^2}{2 M} \right) = \frac{n^2}{2 M}
\end{align}
For one million devices, this gives a probability of about \num{2e-25}, which is negligible.

In the Kraken implementation, all devices, except for the Labnodes, will be mapped to UUIDs using the underlying configuration database. It is up to the user to ensure the uniqueness of the non-UUID ids reported by the devices to ensure proper mapping.


\subsubsection{Limitations} % FIXME: Different title
There is one inherent limitation to the ethernet bus for instrumentation. The ethernet bus is inherently asynchronous and multiple controllers can talk to the device at the same time. Not only that, but different processes within the same controller can talk to the same device. This makes deterministic statements about the device state challenging.

While it is impossible to rule out the possibility of multiple controllers on a network, care was taken to synchronize the workers within Kraken.
\subsection{Databases}
\subsubsection{Cardinality}
\begin{itemize}
 \item TimescaleDB vs Influx
 \item Example Sensors vs. Experiment
\end{itemize}

\clearpage
\section{Short Introduction to Control Theory}
This section will give a very brief introduction into some basic concepts of control theory. Many systems require control over one or more process variables. For example, temperature control of a room or a device, or creating a programmable current from a voltage. All of this requires control over a process and is established trough feedback, which allows a controller to sense the state of the system.

The focus of this section is narrowed down to the concept of feedback and control with regard of developing and understanding of PID controllers for temperature control. In the following sections, first, a model for the system and its controller will be develop, and then using the model, tuning of the control parameters using different tuning algorithms will be discussed.

\subsection{Introduction to the Transfer Function and the Laplace Domain}
\label{sec:transfer_function}
There are two types of systems: open- and closed-loop systems. A system is called open loop, if the output of a system does not feed back to its input as in figure \ref{fig:open_loop}. On the other hand, if the output influces the input of the system via feedback it is called a closed-loop system, as shown in figure \ref{fig:closed_loop}. Although feedback can be treated in static systems, it is more useful to treat it in dynamic systems in either time-domain or frequency-domain. $G(s)$ is called the transfer function of the system, while $U(s)$ is the input, $Y(s)$ is the output, $\beta$ is the feedback parameter or feedback fraction. In this section, upper case letters  denote functions in the Laplace domain, while lower case letters are referring to functions in the time domain.

\begin{figure}[ht]
    \centering
    \begin{subfigure}{0.4\linewidth}
        \centering
        \import{figures/}{open_loop.tex}
        \caption{Open-loop system.}
        \label{fig:open_loop}
    \end{subfigure}
    \begin{subfigure}{0.4\linewidth}
        \centering
        \import{figures/}{closed_loop.tex}
        \caption{Closed-loop system.}
        \label{fig:closed_loop}
    \end{subfigure}
    \caption{Block diagram of closed- and open-loop systems.}
\end{figure}

It is convenient to express the transfer function as its Laplace transform for a number of reasons shown below. The unilateral Laplace transform is definded as:
\begin{equation}
    \mathscr{L}\left( f(t) \right) = F(s) = \int_0^\infty f(t) e^{-st}\,dt.
\end{equation}

with $f: \mathbb{R}^+ \to \mathbb{R}$, that is integrable and grows no faster than $e^{s_0t}$ for $s_0 \in \mathbb{R}$. The latter attribute is important for deriving the rules of differentiation and integration.

To understand the benefits of using the Laplace representation of the transfer function, a few useful properties should be discussed. First of all, the Laplace transform is linear:
\begin{align}
    \mathscr{L}\left(a \cdot f(t) + b \cdot g(t) \right) &= \int_0^\infty (a \cdot f(t) + b \cdot g(t)) e^{-st}\,dt \nonumber\\
    &= a \int_0^\infty f(t) e^{-st}\,dt + b \int_0^\infty g(t) e^{-st}\,dt \nonumber\\
    &= a \mathscr{L}\left(f(t)\right) + b \mathscr{L}\left(g(t)\right)
\end{align}

Another interesting property is the derivative and integral of a function $f$:

\begin{align}
    \mathscr{L}\left(\frac{df}{dt}\right) &= \int_0^\infty \underbracket{f'(t)}_{v'(t)} \underbracket{\vphantom{f'(t)}e^{-st}}_{u(t)}\,dt \nonumber\\
    &= \left[e^{-st} f(t) \right]_0^\infty - \int_0^\infty (-s)f'(t)\,dt \nonumber\\
    &= -f(0) + s \int_0^\infty f'(t)\,dt \nonumber\\
    &= s F(s) - f(0)
\end{align}

\begin{align}
    \mathscr{L} \left( \int_0^t f(\tau)\,d\tau \right) &= \int_0^\infty \left(\int_0^t f(\tau)\,d\tau e^{-st} \right)\,dt \nonumber\\
    &= \int_0^\infty \underbracket{e^{-st}\vphantom{\int_0^t}}_{v'(t)} \underbracket{\int_0^t f(t)\,d\tau}_{u(t)}\,dt \nonumber\\
    &= \left[\frac{-1}{s} e^{-st} \int_0^t f(t)\,d\tau \right]_0^\infty - \int_0^\infty \frac{-1}{s} e^{-s\tau} f(\tau)\,d\tau \nonumber\\
    &= 0 + \frac{1}{s} \int_0^\infty e^{-s\tau} f(\tau)\,d\tau \nonumber\\
    &= \frac{1}{s} F(s) \label{eqn:lapace_integration}
\end{align}

If the initial state $f(0)$ can be chosen to be $0$, the differentiation becomes a simple multiplication by $s$, while the integration becomes a division by $s$. Finally, the most important aspect is, that it is possible to give a simple relation between the input $u(t)$ and the ouput $y(t)$ of a system. The relationship between input and the ouput of a system as shown in figure \ref{fig:open_loop} is given by the convolution, see e.g. \cite{pid_basics}. Assuming the system has an initial state of $0$ for $t<0$, hence $u(t<0) = 0$ and $g(t<0) = 0$, one can calculate:

\begin{equation}
    y(t) = (u \ast g)(t) = \int_0^\infty u(\tau) g(t-\tau)\,d\tau
    \label{eqn:convolution}
\end{equation}

Applying the Laplace transform, greatly simplifies this:
\begin{align}
    Y(s) &= \int_0^\infty e^{-st} y(t)\,dt \nonumber\\
    \overset{\ref{eqn:convolution}}&{=} \int_0^\infty \underbrace{e^{-st}}_{e^{-s(t-\tau)}e^{-s\tau}} \int_0^\infty u(\tau) g(t-\tau)\,d\tau\,dt \nonumber\\
    &= \int_0^\infty \int_0^t e^{-s(t-\tau)} e^{-s\tau} g(t-\tau) u(\tau)\,d\tau\,dt \nonumber\\
    &= \int_0^\infty e^{-s\tau} u(\tau)\,d\tau \int_0^\infty e^{-st} g(t)\,dt \nonumber\\
    &= U(s) \cdot G(s)
\end{align}

This formula is a lot simpler than the convolution of $u(t)$ and $g(t)$, therefore the use of the Laplace transform has become very popular in control theory.

Another property, that is heavily used in control theory, is the time delay of functions. To demonstrate this property, let $f(t-\theta)$ be
\begin{equation}
    g(t) \coloneqq \begin{cases} f(t-\theta), & t \geq \theta \\ 0, & t < \theta \end{cases} \,. \label{eqn:delayed_f}
\end{equation}

The reason for this definition is, that the system must be causal. This means, it is impossible to get data from the future ($t<\theta$). To satisfy this requirement, any constant other than \num{0} may be chosen as well, as is done in later in section \ref{sec:pid_tuning_rules}, when determining tuning parameters and fitting experimental data to a model. An example of such a time delayed function $g(t)$ is shown in figure \ref{fig:heaviside_delayed}.

\begin{figure}[ht]
    \centering
    \begin{subfigure}{0.4\linewidth}
        \centering
        \scalebox{0.75}{%
            \import{figures/}{laplace_no_delay.tex}
        } % scalebox
        \caption{Original signal $f(t)$.}
        \label{fig:heaviside}
    \end{subfigure}
    \begin{subfigure}{0.4\linewidth}
        \centering
        \scalebox{0.75}{%
            \import{figures/}{laplace_time_delay.tex}
        } % scalebox
        \caption{Delayed signal $f(t-2)$.}
        \label{fig:heaviside_delayed}
    \end{subfigure}
\end{figure}

The Laplace transform of a delayed signal $g(t)$ can be calculated as follows:

\begin{align}
    \mathscr{L}\left( g(t) \right) &= \int_0^\infty f(t-\theta) e^{-st}\,dt \nonumber\\
    \overset{\ref{eqn:delayed_f}}&{=} \int_\theta^\infty f(t-\theta) e^{-st}\,dt \nonumber\\
    \overset{\tau \coloneqq t-\theta}&{=} \int_0^\infty f(\tau) e^{-s(\tau+\theta)}\,d\tau \nonumber\\
    &= e^{-s\theta} \int_0^\infty f(\tau) e^{-s\tau} \,d\tau \nonumber\\
    &= e^{-s\theta} F(s) \label{eqn:laplace_delayed}
\end{align}

To satisfy the causaulity requirement in the time domain, the Heaviside function $H(t)$ can be used to give a more concise representation of $g(t)$:
\begin{align}
    \mathscr{L}\left( f(t-\theta) H(t-\theta) \right) = e^{-s\theta} F(s) \label{eqn:laplace_causality}
\end{align}

Lastly, the Laplace transform of $e^{at}$ is given, which is commonly used in differential equations:
\begin{align}
    \mathscr{L}\left(e^{at} \right) &= \int_0^\infty e^{(a-s)t}\,dt = \frac{1}{a-s} \left[e^{(a-s)t} \right]_0^\infty = \frac{1}{s-a} \label{eqn:laplace_exponential}
\end{align}


Using these tools, it is possible calculate the transfer function of a closed-loop temperature controller. This is done in the next section.

\clearpage
\subsection{A Model for Temperature Control}
\label{sec:temperature_control_model}
\begin{figure}[ht]
    \centering
    \scalebox{1}{%
        \import{figures/}{first_order_model.tex}
    } % scalebox
    \caption{Simple temperature model of a generic system.}
    \label{fig:first_order_model_room}
\end{figure}

In order to describe a closed-loop system using a transfer function $G(s)$, one has to first create a model for the process and the controller involved. This section will derive the simple, but very useful first-order model with dead-time. This model can be derived from the idea, that the system at temperature $T_{system}$ has a thermal capacitance $C_{system}$, an influx of heat $\dot Q_{load}$ from a thermal load and a controller removing heat from the system through a heat exchanger with a resistance of $R_{force}$. Additionally, there is some leakage through the walls of the system to the ambient environment via $R_{leakage}$. This analogy of thermodynamics with electrondynamics allows to create the model shown in figure \ref{fig:first_order_model_room}. Since this model is to be used for a temperature controller, more simplifications can be made and a so-called small-signal model can be developed as opposed to the large signal model shown above. The small-signal model is an approximation around a working point, that is valid for small excursions around it, similar to a Taylor approximation. The small signal model can be used calculate the system response to small changes of the controller output in order to estimate the controll parameters.

Using the small signal approach, the system response can be split into a constant and a dynamic part -- the 0\textsuperscript{th} and 1\textsuperscript{st} order of the Taylor approximation. In order to simplify the system shown in figure \ref{fig:first_order_model_room} an assumption can made, that the system load $\dot Q_{load}$ and the flux through $R_{leakage}$ is \textit{reasonably stable}. \textit{Reasonably stable} means that it can be treated as small deviations and additionally any changes are within the bandwidth of the controller and well suppressed. This allows to treat them as (almost) constant effects, which result in an offset applied to the output of the controller. This allows to solely treat the room and its heat capacity in the dynamic model shown in figure \ref{fig:first_order_model}. Here $T_{force}$ and $T_{system}$ were replaced by $T_{in}$ and $T_{out}$ for better readability:

\begin{figure}[hb]
    \centering
    \scalebox{1}{%
        \import{figures/}{first_order_model_kirchhoff.tex}
    } % scalebox
    \caption{First order model.}
    \label{fig:first_order_model}
\end{figure}

This is the classic $RC$ circuit and exploiting the analogy of thermodynamics and electrodynamics again, using Kirchhoff's second law, one finds:

\begin{alignat}{1}
    \sum T_i &= 0 \nonumber\\
    T_{in}(t) - \dot{Q}(t) R - \frac 1 C \int \dot{Q}(t)\,dt &= 0 \label{eqn:first_order_model_kirchhoff}
\end{alignat}

Taking the Laplace transform, applying equation \ref{eqn:lapace_integration}, solving for $ \dot Q(s)$ and using $T_{out} = \frac{1}{sC} \dot Q(s)$ to replace $\dot Q$, equation \ref{eqn:first_order_model_kirchhoff} can be written as:
\begin{align*}
    T_{in}(s) - \dot{Q}(s) R - \frac{1}{sC} \dot{Q}(s) &= 0\\
    \dot{Q}(s) = \frac{T_{in}(s)}{R-\frac{1}{sC}} &= \frac{T_{out}}{\frac{1}{sC}}
\end{align*}

This allows to calculate the transfer function of the process $P$ using:
\begin{align}
    P(s) &= \frac{T_{out}}{T_{in}} = \frac{\frac{1}{sC}}{R-\frac{1}{sC}} \nonumber\\
    &= \frac{1}{sRC + 1} \nonumber\\
    &= \frac{1}{1 + s\tau} = \frac{K}{1 + s\tau} \label{eqn:first_order_model}
\end{align}
with the system gain $K$ and the time constant $\tau$. In case of the $RC$ circuit, the gain is $1$, but other systems may have a gain factor of $K \neq 1$, so it is included here for the sake of generality.

Equation \ref{eqn:first_order_model} is called the transfer function of a first-order model, because its origin is a differential equation of first order. This model describes homogeneous systems, like a room, very well, as can be seen in section \ref{}, but in order to derive the transfer function including the controller and the sensor some more work is required derive the sensor transfer function.

Expanding on figure \ref{fig:open_loop} and equation \ref{eqn:convolution} the open-loop transfer function becomes:
\begin{equation}
    G(s) = P(s) \cdot S(s)
\end{equation}

and the block diagram becomes
\begin{figure}[ht]
    \centering
    \scalebox{1}{%
        \import{figures/}{open_loop_full.tex}
    }% scalebox
    \caption{Open-loop system with sensor.}
\end{figure}

The transfer function of the sensor, given an ideal linear transducer, can be modeled as a delay line with delay $\theta$ and $f(t-\theta) = H(t-\theta)$. A gain of $1$ is assumed here, because any system gain can already be included in the parameter $K$. Using equation \ref{eqn:laplace_delayed} $S(s)$ can be written as
\begin{equation}
    S(s) = e^{-\theta s} .
\end{equation}

The full process model including the time delay is:
\begin{equation}
    G(s) = \frac{K}{1 + s\tau} e^{-\theta s} \label{eqn:first_order_plus_dead_time_model}
\end{equation}

This is called a first-order plus dead-time model (FOPDT) or first-order plus time-delay model (FOPTD). To fit experimental data to this model it is more convenient to transform the transfer function \ref{eqn:first_order_plus_dead_time_model} into the time domain. To calculate the output response an input $U(s)$ is required. In principal any function can do, but a step function is typically used, for example by \citeauthor{ziegler_nichols} \cite{ziegler_nichols} and many others \cite{tuning_rules,pessen_integral,simc,simc_paper,pid_controllers_for_time_delay_systems,pi_stabilization_of_fopdt_systems, pid_basics}. It is both simple to calculate and to apply to a real system. Using equations \ref{eqn:laplace_delayed} and \ref{eqn:laplace_exponential}, the Heaviside $H(t)$ step function transforms as
\begin{equation}
    \mathscr{L} \left(u(t) \right) = U(s) = \mathscr{L} \left( \Delta u H(t) \right) = \frac{\Delta u}{s}
\end{equation}

with the step size $\Delta u$. The output $Y(s)$ can then be calculated analytically.
\begin{align}
    Y(s) &= U(s) \cdot G(s)\nonumber\\
    &= \frac{\Delta u}{s} \frac{K}{1 + s\tau} e^{-\theta s} \nonumber\\
    &=  K \Delta u \frac{1}{s (1 + s\tau)} e^{-\theta s} \nonumber\\
    &= K \Delta u \left(\frac{1}{s} - \frac{\tau}{s\tau+1} \right) e^{-\theta s} \nonumber\\
    &= K \Delta u \left(\frac{1}{s} - \frac{1}{s+\frac{1}{\tau}} \right) e^{-\theta s}
\end{align}

To derive $y(t)$, the inverse Laplace transform of $Y(s)$ is required. Unfortunately, this is not as simple as the Laplace transform. Fortunately, the required equations were already derived in equations \ref{eqn:lapace_integration} and \ref{eqn:laplace_exponential}. Now, making sure causaulity is guaranteed as shown in \ref{eqn:laplace_causality}, the simple first order model can be transformed back into the time domain.
\begin{align}
     y(t) &= \mathscr{L}^{-1} \left(Y(s)\right) \nonumber\\
     &= K \Delta u \mathscr{L}^{-1} \left(\frac{1}{s} e^{-\theta s} \right)  - K \mathscr{L}^{-1} \left( \frac{1}{s+\frac{1}{\tau}} e^{-\theta s} \right) \nonumber\\
    \overset{\ref{eqn:laplace_exponential}}&{=} K \Delta u \cdot 1 \cdot H(t-\theta) - \left(e^{-\frac{t-\theta}{\tau}} \right) H(t-\theta) \nonumber\\
    &= K \Delta u \left(1-e^{-\frac{t-\theta}{\tau}} \right) H(t-\theta) \label{eqn:first_order_plus_dead_time_model_time-domain}
\end{align}

The time domain solution of the FOPDT model can now be used extract the parameters $\tau$, $\theta$ and $K$ from a real physical system.

The procedure can be summarized from the above as follows. The controller must be set to a constant output and the room must be given time reach equlibrium. Once the temperature has settled, an output step of $\Delta u$ is applied. The system will respond after a time delay and then follow an exponential function. A simulation of the step response applied to a first-order model with time delay is shown in figure \ref{fig:fopdt}. The gain is $K=1$. The solid black line showns the response of the transfer function, including the system and the sensor. The dashed lines show the individual components, the Heaviside function and the exponential term. The controller output step $\Delta u = 1$ is applied at $t=0$ and not shown explicitely. Now, it can be clearly seen, that the sensor does not register a change until the time delay $\theta$ has passed and the Heaviside function changes from $0$ to $1$. Then the system responds with an exponential decay towards \num{1}.

\begin{figure}[ht]
    \centering
    %% Creator: Matplotlib, PGF backend
%%
%% To include the figure in your LaTeX document, write
%%   \input{<filename>.pgf}
%%
%% Make sure the required packages are loaded in your preamble
%%   \usepackage{pgf}
%%
%% Also ensure that all the required font packages are loaded; for instance,
%% the lmodern package is sometimes necessary when using math font.
%%   \usepackage{lmodern}
%%
%% Figures using additional raster images can only be included by \input if
%% they are in the same directory as the main LaTeX file. For loading figures
%% from other directories you can use the `import` package
%%   \usepackage{import}
%%
%% and then include the figures with
%%   \import{<path to file>}{<filename>.pgf}
%%
%% Matplotlib used the following preamble
%%   \usepackage{fontspec}
%%   \setmainfont{DejaVuSerif.ttf}[Path=\detokenize{/home/maat/Documents/Uni/Physik/Phd/Thesis/data/env/lib/python3.10/site-packages/matplotlib/mpl-data/fonts/ttf/}]
%%   \setsansfont{DejaVuSans.ttf}[Path=\detokenize{/home/maat/Documents/Uni/Physik/Phd/Thesis/data/env/lib/python3.10/site-packages/matplotlib/mpl-data/fonts/ttf/}]
%%   \setmonofont{DejaVuSansMono.ttf}[Path=\detokenize{/home/maat/Documents/Uni/Physik/Phd/Thesis/data/env/lib/python3.10/site-packages/matplotlib/mpl-data/fonts/ttf/}]
%%
\begingroup%
\makeatletter%
\begin{pgfpicture}%
\pgfpathrectangle{\pgfpointorigin}{\pgfqpoint{5.208662in}{3.219130in}}%
\pgfusepath{use as bounding box, clip}%
\begin{pgfscope}%
\pgfsetbuttcap%
\pgfsetmiterjoin%
\definecolor{currentfill}{rgb}{1.000000,1.000000,1.000000}%
\pgfsetfillcolor{currentfill}%
\pgfsetlinewidth{0.000000pt}%
\definecolor{currentstroke}{rgb}{1.000000,1.000000,1.000000}%
\pgfsetstrokecolor{currentstroke}%
\pgfsetdash{}{0pt}%
\pgfpathmoveto{\pgfqpoint{0.000000in}{0.000000in}}%
\pgfpathlineto{\pgfqpoint{5.208662in}{0.000000in}}%
\pgfpathlineto{\pgfqpoint{5.208662in}{3.219130in}}%
\pgfpathlineto{\pgfqpoint{0.000000in}{3.219130in}}%
\pgfpathlineto{\pgfqpoint{0.000000in}{0.000000in}}%
\pgfpathclose%
\pgfusepath{fill}%
\end{pgfscope}%
\begin{pgfscope}%
\pgfsetbuttcap%
\pgfsetmiterjoin%
\definecolor{currentfill}{rgb}{1.000000,1.000000,1.000000}%
\pgfsetfillcolor{currentfill}%
\pgfsetlinewidth{0.000000pt}%
\definecolor{currentstroke}{rgb}{0.000000,0.000000,0.000000}%
\pgfsetstrokecolor{currentstroke}%
\pgfsetstrokeopacity{0.000000}%
\pgfsetdash{}{0pt}%
\pgfpathmoveto{\pgfqpoint{0.779028in}{0.582778in}}%
\pgfpathlineto{\pgfqpoint{4.970537in}{0.582778in}}%
\pgfpathlineto{\pgfqpoint{4.970537in}{3.014130in}}%
\pgfpathlineto{\pgfqpoint{0.779028in}{3.014130in}}%
\pgfpathlineto{\pgfqpoint{0.779028in}{0.582778in}}%
\pgfpathclose%
\pgfusepath{fill}%
\end{pgfscope}%
\begin{pgfscope}%
\pgfsetbuttcap%
\pgfsetroundjoin%
\definecolor{currentfill}{rgb}{0.000000,0.000000,0.000000}%
\pgfsetfillcolor{currentfill}%
\pgfsetlinewidth{0.803000pt}%
\definecolor{currentstroke}{rgb}{0.000000,0.000000,0.000000}%
\pgfsetstrokecolor{currentstroke}%
\pgfsetdash{}{0pt}%
\pgfsys@defobject{currentmarker}{\pgfqpoint{0.000000in}{-0.048611in}}{\pgfqpoint{0.000000in}{0.000000in}}{%
\pgfpathmoveto{\pgfqpoint{0.000000in}{0.000000in}}%
\pgfpathlineto{\pgfqpoint{0.000000in}{-0.048611in}}%
\pgfusepath{stroke,fill}%
}%
\begin{pgfscope}%
\pgfsys@transformshift{0.779028in}{0.582778in}%
\pgfsys@useobject{currentmarker}{}%
\end{pgfscope}%
\end{pgfscope}%
\begin{pgfscope}%
\definecolor{textcolor}{rgb}{0.000000,0.000000,0.000000}%
\pgfsetstrokecolor{textcolor}%
\pgfsetfillcolor{textcolor}%
\pgftext[x=0.779028in,y=0.485556in,,top]{\color{textcolor}\sffamily\fontsize{10.000000}{12.000000}\selectfont 0}%
\end{pgfscope}%
\begin{pgfscope}%
\pgfsetbuttcap%
\pgfsetroundjoin%
\definecolor{currentfill}{rgb}{0.000000,0.000000,0.000000}%
\pgfsetfillcolor{currentfill}%
\pgfsetlinewidth{0.803000pt}%
\definecolor{currentstroke}{rgb}{0.000000,0.000000,0.000000}%
\pgfsetstrokecolor{currentstroke}%
\pgfsetdash{}{0pt}%
\pgfsys@defobject{currentmarker}{\pgfqpoint{0.000000in}{-0.048611in}}{\pgfqpoint{0.000000in}{0.000000in}}{%
\pgfpathmoveto{\pgfqpoint{0.000000in}{0.000000in}}%
\pgfpathlineto{\pgfqpoint{0.000000in}{-0.048611in}}%
\pgfusepath{stroke,fill}%
}%
\begin{pgfscope}%
\pgfsys@transformshift{1.617330in}{0.582778in}%
\pgfsys@useobject{currentmarker}{}%
\end{pgfscope}%
\end{pgfscope}%
\begin{pgfscope}%
\definecolor{textcolor}{rgb}{0.000000,0.000000,0.000000}%
\pgfsetstrokecolor{textcolor}%
\pgfsetfillcolor{textcolor}%
\pgftext[x=1.617330in,y=0.485556in,,top]{\color{textcolor}\sffamily\fontsize{10.000000}{12.000000}\selectfont 2}%
\end{pgfscope}%
\begin{pgfscope}%
\pgfsetbuttcap%
\pgfsetroundjoin%
\definecolor{currentfill}{rgb}{0.000000,0.000000,0.000000}%
\pgfsetfillcolor{currentfill}%
\pgfsetlinewidth{0.803000pt}%
\definecolor{currentstroke}{rgb}{0.000000,0.000000,0.000000}%
\pgfsetstrokecolor{currentstroke}%
\pgfsetdash{}{0pt}%
\pgfsys@defobject{currentmarker}{\pgfqpoint{0.000000in}{-0.048611in}}{\pgfqpoint{0.000000in}{0.000000in}}{%
\pgfpathmoveto{\pgfqpoint{0.000000in}{0.000000in}}%
\pgfpathlineto{\pgfqpoint{0.000000in}{-0.048611in}}%
\pgfusepath{stroke,fill}%
}%
\begin{pgfscope}%
\pgfsys@transformshift{2.455631in}{0.582778in}%
\pgfsys@useobject{currentmarker}{}%
\end{pgfscope}%
\end{pgfscope}%
\begin{pgfscope}%
\definecolor{textcolor}{rgb}{0.000000,0.000000,0.000000}%
\pgfsetstrokecolor{textcolor}%
\pgfsetfillcolor{textcolor}%
\pgftext[x=2.455631in,y=0.485556in,,top]{\color{textcolor}\sffamily\fontsize{10.000000}{12.000000}\selectfont 4}%
\end{pgfscope}%
\begin{pgfscope}%
\pgfsetbuttcap%
\pgfsetroundjoin%
\definecolor{currentfill}{rgb}{0.000000,0.000000,0.000000}%
\pgfsetfillcolor{currentfill}%
\pgfsetlinewidth{0.803000pt}%
\definecolor{currentstroke}{rgb}{0.000000,0.000000,0.000000}%
\pgfsetstrokecolor{currentstroke}%
\pgfsetdash{}{0pt}%
\pgfsys@defobject{currentmarker}{\pgfqpoint{0.000000in}{-0.048611in}}{\pgfqpoint{0.000000in}{0.000000in}}{%
\pgfpathmoveto{\pgfqpoint{0.000000in}{0.000000in}}%
\pgfpathlineto{\pgfqpoint{0.000000in}{-0.048611in}}%
\pgfusepath{stroke,fill}%
}%
\begin{pgfscope}%
\pgfsys@transformshift{3.293933in}{0.582778in}%
\pgfsys@useobject{currentmarker}{}%
\end{pgfscope}%
\end{pgfscope}%
\begin{pgfscope}%
\definecolor{textcolor}{rgb}{0.000000,0.000000,0.000000}%
\pgfsetstrokecolor{textcolor}%
\pgfsetfillcolor{textcolor}%
\pgftext[x=3.293933in,y=0.485556in,,top]{\color{textcolor}\sffamily\fontsize{10.000000}{12.000000}\selectfont 6}%
\end{pgfscope}%
\begin{pgfscope}%
\pgfsetbuttcap%
\pgfsetroundjoin%
\definecolor{currentfill}{rgb}{0.000000,0.000000,0.000000}%
\pgfsetfillcolor{currentfill}%
\pgfsetlinewidth{0.803000pt}%
\definecolor{currentstroke}{rgb}{0.000000,0.000000,0.000000}%
\pgfsetstrokecolor{currentstroke}%
\pgfsetdash{}{0pt}%
\pgfsys@defobject{currentmarker}{\pgfqpoint{0.000000in}{-0.048611in}}{\pgfqpoint{0.000000in}{0.000000in}}{%
\pgfpathmoveto{\pgfqpoint{0.000000in}{0.000000in}}%
\pgfpathlineto{\pgfqpoint{0.000000in}{-0.048611in}}%
\pgfusepath{stroke,fill}%
}%
\begin{pgfscope}%
\pgfsys@transformshift{4.132235in}{0.582778in}%
\pgfsys@useobject{currentmarker}{}%
\end{pgfscope}%
\end{pgfscope}%
\begin{pgfscope}%
\definecolor{textcolor}{rgb}{0.000000,0.000000,0.000000}%
\pgfsetstrokecolor{textcolor}%
\pgfsetfillcolor{textcolor}%
\pgftext[x=4.132235in,y=0.485556in,,top]{\color{textcolor}\sffamily\fontsize{10.000000}{12.000000}\selectfont 8}%
\end{pgfscope}%
\begin{pgfscope}%
\pgfsetbuttcap%
\pgfsetroundjoin%
\definecolor{currentfill}{rgb}{0.000000,0.000000,0.000000}%
\pgfsetfillcolor{currentfill}%
\pgfsetlinewidth{0.803000pt}%
\definecolor{currentstroke}{rgb}{0.000000,0.000000,0.000000}%
\pgfsetstrokecolor{currentstroke}%
\pgfsetdash{}{0pt}%
\pgfsys@defobject{currentmarker}{\pgfqpoint{0.000000in}{-0.048611in}}{\pgfqpoint{0.000000in}{0.000000in}}{%
\pgfpathmoveto{\pgfqpoint{0.000000in}{0.000000in}}%
\pgfpathlineto{\pgfqpoint{0.000000in}{-0.048611in}}%
\pgfusepath{stroke,fill}%
}%
\begin{pgfscope}%
\pgfsys@transformshift{4.970537in}{0.582778in}%
\pgfsys@useobject{currentmarker}{}%
\end{pgfscope}%
\end{pgfscope}%
\begin{pgfscope}%
\definecolor{textcolor}{rgb}{0.000000,0.000000,0.000000}%
\pgfsetstrokecolor{textcolor}%
\pgfsetfillcolor{textcolor}%
\pgftext[x=4.970537in,y=0.485556in,,top]{\color{textcolor}\sffamily\fontsize{10.000000}{12.000000}\selectfont 10}%
\end{pgfscope}%
\begin{pgfscope}%
\definecolor{textcolor}{rgb}{0.000000,0.000000,0.000000}%
\pgfsetstrokecolor{textcolor}%
\pgfsetfillcolor{textcolor}%
\pgftext[x=2.874782in,y=0.295587in,,top]{\color{textcolor}\sffamily\fontsize{10.000000}{12.000000}\selectfont Time}%
\end{pgfscope}%
\begin{pgfscope}%
\pgfsetbuttcap%
\pgfsetroundjoin%
\definecolor{currentfill}{rgb}{0.000000,0.000000,0.000000}%
\pgfsetfillcolor{currentfill}%
\pgfsetlinewidth{0.803000pt}%
\definecolor{currentstroke}{rgb}{0.000000,0.000000,0.000000}%
\pgfsetstrokecolor{currentstroke}%
\pgfsetdash{}{0pt}%
\pgfsys@defobject{currentmarker}{\pgfqpoint{-0.048611in}{0.000000in}}{\pgfqpoint{-0.000000in}{0.000000in}}{%
\pgfpathmoveto{\pgfqpoint{-0.000000in}{0.000000in}}%
\pgfpathlineto{\pgfqpoint{-0.048611in}{0.000000in}}%
\pgfusepath{stroke,fill}%
}%
\begin{pgfscope}%
\pgfsys@transformshift{0.779028in}{0.582778in}%
\pgfsys@useobject{currentmarker}{}%
\end{pgfscope}%
\end{pgfscope}%
\begin{pgfscope}%
\definecolor{textcolor}{rgb}{0.000000,0.000000,0.000000}%
\pgfsetstrokecolor{textcolor}%
\pgfsetfillcolor{textcolor}%
\pgftext[x=0.352901in, y=0.530016in, left, base]{\color{textcolor}\sffamily\fontsize{10.000000}{12.000000}\selectfont \ensuremath{-}1.0}%
\end{pgfscope}%
\begin{pgfscope}%
\pgfsetbuttcap%
\pgfsetroundjoin%
\definecolor{currentfill}{rgb}{0.000000,0.000000,0.000000}%
\pgfsetfillcolor{currentfill}%
\pgfsetlinewidth{0.803000pt}%
\definecolor{currentstroke}{rgb}{0.000000,0.000000,0.000000}%
\pgfsetstrokecolor{currentstroke}%
\pgfsetdash{}{0pt}%
\pgfsys@defobject{currentmarker}{\pgfqpoint{-0.048611in}{0.000000in}}{\pgfqpoint{-0.000000in}{0.000000in}}{%
\pgfpathmoveto{\pgfqpoint{-0.000000in}{0.000000in}}%
\pgfpathlineto{\pgfqpoint{-0.048611in}{0.000000in}}%
\pgfusepath{stroke,fill}%
}%
\begin{pgfscope}%
\pgfsys@transformshift{0.779028in}{1.069048in}%
\pgfsys@useobject{currentmarker}{}%
\end{pgfscope}%
\end{pgfscope}%
\begin{pgfscope}%
\definecolor{textcolor}{rgb}{0.000000,0.000000,0.000000}%
\pgfsetstrokecolor{textcolor}%
\pgfsetfillcolor{textcolor}%
\pgftext[x=0.352901in, y=1.016287in, left, base]{\color{textcolor}\sffamily\fontsize{10.000000}{12.000000}\selectfont \ensuremath{-}0.5}%
\end{pgfscope}%
\begin{pgfscope}%
\pgfsetbuttcap%
\pgfsetroundjoin%
\definecolor{currentfill}{rgb}{0.000000,0.000000,0.000000}%
\pgfsetfillcolor{currentfill}%
\pgfsetlinewidth{0.803000pt}%
\definecolor{currentstroke}{rgb}{0.000000,0.000000,0.000000}%
\pgfsetstrokecolor{currentstroke}%
\pgfsetdash{}{0pt}%
\pgfsys@defobject{currentmarker}{\pgfqpoint{-0.048611in}{0.000000in}}{\pgfqpoint{-0.000000in}{0.000000in}}{%
\pgfpathmoveto{\pgfqpoint{-0.000000in}{0.000000in}}%
\pgfpathlineto{\pgfqpoint{-0.048611in}{0.000000in}}%
\pgfusepath{stroke,fill}%
}%
\begin{pgfscope}%
\pgfsys@transformshift{0.779028in}{1.555319in}%
\pgfsys@useobject{currentmarker}{}%
\end{pgfscope}%
\end{pgfscope}%
\begin{pgfscope}%
\definecolor{textcolor}{rgb}{0.000000,0.000000,0.000000}%
\pgfsetstrokecolor{textcolor}%
\pgfsetfillcolor{textcolor}%
\pgftext[x=0.460926in, y=1.502557in, left, base]{\color{textcolor}\sffamily\fontsize{10.000000}{12.000000}\selectfont 0.0}%
\end{pgfscope}%
\begin{pgfscope}%
\pgfsetbuttcap%
\pgfsetroundjoin%
\definecolor{currentfill}{rgb}{0.000000,0.000000,0.000000}%
\pgfsetfillcolor{currentfill}%
\pgfsetlinewidth{0.803000pt}%
\definecolor{currentstroke}{rgb}{0.000000,0.000000,0.000000}%
\pgfsetstrokecolor{currentstroke}%
\pgfsetdash{}{0pt}%
\pgfsys@defobject{currentmarker}{\pgfqpoint{-0.048611in}{0.000000in}}{\pgfqpoint{-0.000000in}{0.000000in}}{%
\pgfpathmoveto{\pgfqpoint{-0.000000in}{0.000000in}}%
\pgfpathlineto{\pgfqpoint{-0.048611in}{0.000000in}}%
\pgfusepath{stroke,fill}%
}%
\begin{pgfscope}%
\pgfsys@transformshift{0.779028in}{2.041589in}%
\pgfsys@useobject{currentmarker}{}%
\end{pgfscope}%
\end{pgfscope}%
\begin{pgfscope}%
\definecolor{textcolor}{rgb}{0.000000,0.000000,0.000000}%
\pgfsetstrokecolor{textcolor}%
\pgfsetfillcolor{textcolor}%
\pgftext[x=0.460926in, y=1.988828in, left, base]{\color{textcolor}\sffamily\fontsize{10.000000}{12.000000}\selectfont 0.5}%
\end{pgfscope}%
\begin{pgfscope}%
\pgfsetbuttcap%
\pgfsetroundjoin%
\definecolor{currentfill}{rgb}{0.000000,0.000000,0.000000}%
\pgfsetfillcolor{currentfill}%
\pgfsetlinewidth{0.803000pt}%
\definecolor{currentstroke}{rgb}{0.000000,0.000000,0.000000}%
\pgfsetstrokecolor{currentstroke}%
\pgfsetdash{}{0pt}%
\pgfsys@defobject{currentmarker}{\pgfqpoint{-0.048611in}{0.000000in}}{\pgfqpoint{-0.000000in}{0.000000in}}{%
\pgfpathmoveto{\pgfqpoint{-0.000000in}{0.000000in}}%
\pgfpathlineto{\pgfqpoint{-0.048611in}{0.000000in}}%
\pgfusepath{stroke,fill}%
}%
\begin{pgfscope}%
\pgfsys@transformshift{0.779028in}{2.527860in}%
\pgfsys@useobject{currentmarker}{}%
\end{pgfscope}%
\end{pgfscope}%
\begin{pgfscope}%
\definecolor{textcolor}{rgb}{0.000000,0.000000,0.000000}%
\pgfsetstrokecolor{textcolor}%
\pgfsetfillcolor{textcolor}%
\pgftext[x=0.460926in, y=2.475098in, left, base]{\color{textcolor}\sffamily\fontsize{10.000000}{12.000000}\selectfont 1.0}%
\end{pgfscope}%
\begin{pgfscope}%
\pgfsetbuttcap%
\pgfsetroundjoin%
\definecolor{currentfill}{rgb}{0.000000,0.000000,0.000000}%
\pgfsetfillcolor{currentfill}%
\pgfsetlinewidth{0.803000pt}%
\definecolor{currentstroke}{rgb}{0.000000,0.000000,0.000000}%
\pgfsetstrokecolor{currentstroke}%
\pgfsetdash{}{0pt}%
\pgfsys@defobject{currentmarker}{\pgfqpoint{-0.048611in}{0.000000in}}{\pgfqpoint{-0.000000in}{0.000000in}}{%
\pgfpathmoveto{\pgfqpoint{-0.000000in}{0.000000in}}%
\pgfpathlineto{\pgfqpoint{-0.048611in}{0.000000in}}%
\pgfusepath{stroke,fill}%
}%
\begin{pgfscope}%
\pgfsys@transformshift{0.779028in}{3.014130in}%
\pgfsys@useobject{currentmarker}{}%
\end{pgfscope}%
\end{pgfscope}%
\begin{pgfscope}%
\definecolor{textcolor}{rgb}{0.000000,0.000000,0.000000}%
\pgfsetstrokecolor{textcolor}%
\pgfsetfillcolor{textcolor}%
\pgftext[x=0.460926in, y=2.961369in, left, base]{\color{textcolor}\sffamily\fontsize{10.000000}{12.000000}\selectfont 1.5}%
\end{pgfscope}%
\begin{pgfscope}%
\definecolor{textcolor}{rgb}{0.000000,0.000000,0.000000}%
\pgfsetstrokecolor{textcolor}%
\pgfsetfillcolor{textcolor}%
\pgftext[x=0.297346in,y=1.798454in,,bottom,rotate=90.000000]{\color{textcolor}\sffamily\fontsize{10.000000}{12.000000}\selectfont Process Output}%
\end{pgfscope}%
\begin{pgfscope}%
\pgfpathrectangle{\pgfqpoint{0.779028in}{0.582778in}}{\pgfqpoint{4.191509in}{2.431352in}}%
\pgfusepath{clip}%
\pgfsetbuttcap%
\pgfsetroundjoin%
\pgfsetlinewidth{2.007500pt}%
\definecolor{currentstroke}{rgb}{0.000000,0.419608,0.643137}%
\pgfsetstrokecolor{currentstroke}%
\pgfsetstrokeopacity{0.700000}%
\pgfsetdash{{7.400000pt}{3.200000pt}}{0.000000pt}%
\pgfpathmoveto{\pgfqpoint{1.870301in}{0.572778in}}%
\pgfpathlineto{\pgfqpoint{1.910735in}{0.664918in}}%
\pgfpathlineto{\pgfqpoint{1.952650in}{0.755775in}}%
\pgfpathlineto{\pgfqpoint{1.994565in}{0.842200in}}%
\pgfpathlineto{\pgfqpoint{2.036481in}{0.924411in}}%
\pgfpathlineto{\pgfqpoint{2.078396in}{1.002612in}}%
\pgfpathlineto{\pgfqpoint{2.120311in}{1.076999in}}%
\pgfpathlineto{\pgfqpoint{2.162226in}{1.147758in}}%
\pgfpathlineto{\pgfqpoint{2.204141in}{1.215067in}}%
\pgfpathlineto{\pgfqpoint{2.246056in}{1.279092in}}%
\pgfpathlineto{\pgfqpoint{2.287971in}{1.339995in}}%
\pgfpathlineto{\pgfqpoint{2.329886in}{1.397928in}}%
\pgfpathlineto{\pgfqpoint{2.371801in}{1.453036in}}%
\pgfpathlineto{\pgfqpoint{2.413716in}{1.505455in}}%
\pgfpathlineto{\pgfqpoint{2.455631in}{1.555319in}}%
\pgfpathlineto{\pgfqpoint{2.497547in}{1.602750in}}%
\pgfpathlineto{\pgfqpoint{2.539462in}{1.647868in}}%
\pgfpathlineto{\pgfqpoint{2.581377in}{1.690786in}}%
\pgfpathlineto{\pgfqpoint{2.623292in}{1.731610in}}%
\pgfpathlineto{\pgfqpoint{2.665207in}{1.770444in}}%
\pgfpathlineto{\pgfqpoint{2.707122in}{1.807384in}}%
\pgfpathlineto{\pgfqpoint{2.749037in}{1.842522in}}%
\pgfpathlineto{\pgfqpoint{2.790952in}{1.875946in}}%
\pgfpathlineto{\pgfqpoint{2.832867in}{1.907740in}}%
\pgfpathlineto{\pgfqpoint{2.874782in}{1.937984in}}%
\pgfpathlineto{\pgfqpoint{2.916697in}{1.966752in}}%
\pgfpathlineto{\pgfqpoint{2.958613in}{1.994118in}}%
\pgfpathlineto{\pgfqpoint{3.000528in}{2.020149in}}%
\pgfpathlineto{\pgfqpoint{3.042443in}{2.044910in}}%
\pgfpathlineto{\pgfqpoint{3.084358in}{2.068464in}}%
\pgfpathlineto{\pgfqpoint{3.126273in}{2.090869in}}%
\pgfpathlineto{\pgfqpoint{3.168188in}{2.112181in}}%
\pgfpathlineto{\pgfqpoint{3.210103in}{2.132454in}}%
\pgfpathlineto{\pgfqpoint{3.252018in}{2.151738in}}%
\pgfpathlineto{\pgfqpoint{3.293933in}{2.170082in}}%
\pgfpathlineto{\pgfqpoint{3.335848in}{2.187531in}}%
\pgfpathlineto{\pgfqpoint{3.377763in}{2.204129in}}%
\pgfpathlineto{\pgfqpoint{3.419679in}{2.219917in}}%
\pgfpathlineto{\pgfqpoint{3.461594in}{2.234936in}}%
\pgfpathlineto{\pgfqpoint{3.503509in}{2.249222in}}%
\pgfpathlineto{\pgfqpoint{3.545424in}{2.262811in}}%
\pgfpathlineto{\pgfqpoint{3.587339in}{2.275738in}}%
\pgfpathlineto{\pgfqpoint{3.629254in}{2.288034in}}%
\pgfpathlineto{\pgfqpoint{3.671169in}{2.299730in}}%
\pgfpathlineto{\pgfqpoint{3.713084in}{2.310856in}}%
\pgfpathlineto{\pgfqpoint{3.754999in}{2.321440in}}%
\pgfpathlineto{\pgfqpoint{3.796914in}{2.331507in}}%
\pgfpathlineto{\pgfqpoint{3.838829in}{2.341083in}}%
\pgfpathlineto{\pgfqpoint{3.880745in}{2.350192in}}%
\pgfpathlineto{\pgfqpoint{3.922660in}{2.358857in}}%
\pgfpathlineto{\pgfqpoint{3.964575in}{2.367100in}}%
\pgfpathlineto{\pgfqpoint{4.006490in}{2.374940in}}%
\pgfpathlineto{\pgfqpoint{4.048405in}{2.382398in}}%
\pgfpathlineto{\pgfqpoint{4.090320in}{2.389492in}}%
\pgfpathlineto{\pgfqpoint{4.132235in}{2.396241in}}%
\pgfpathlineto{\pgfqpoint{4.174150in}{2.402660in}}%
\pgfpathlineto{\pgfqpoint{4.216065in}{2.408766in}}%
\pgfpathlineto{\pgfqpoint{4.257980in}{2.414574in}}%
\pgfpathlineto{\pgfqpoint{4.299895in}{2.420099in}}%
\pgfpathlineto{\pgfqpoint{4.341811in}{2.425355in}}%
\pgfpathlineto{\pgfqpoint{4.383726in}{2.430354in}}%
\pgfpathlineto{\pgfqpoint{4.425641in}{2.435109in}}%
\pgfpathlineto{\pgfqpoint{4.467556in}{2.439633in}}%
\pgfpathlineto{\pgfqpoint{4.509471in}{2.443936in}}%
\pgfpathlineto{\pgfqpoint{4.551386in}{2.448029in}}%
\pgfpathlineto{\pgfqpoint{4.593301in}{2.451922in}}%
\pgfpathlineto{\pgfqpoint{4.635216in}{2.455626in}}%
\pgfpathlineto{\pgfqpoint{4.677131in}{2.459148in}}%
\pgfpathlineto{\pgfqpoint{4.719046in}{2.462500in}}%
\pgfpathlineto{\pgfqpoint{4.760961in}{2.465687in}}%
\pgfpathlineto{\pgfqpoint{4.802877in}{2.468719in}}%
\pgfpathlineto{\pgfqpoint{4.844792in}{2.471604in}}%
\pgfpathlineto{\pgfqpoint{4.886707in}{2.474347in}}%
\pgfpathlineto{\pgfqpoint{4.928622in}{2.476957in}}%
\pgfpathlineto{\pgfqpoint{4.970537in}{2.479440in}}%
\pgfusepath{stroke}%
\end{pgfscope}%
\begin{pgfscope}%
\pgfpathrectangle{\pgfqpoint{0.779028in}{0.582778in}}{\pgfqpoint{4.191509in}{2.431352in}}%
\pgfusepath{clip}%
\pgfsetbuttcap%
\pgfsetroundjoin%
\pgfsetlinewidth{2.007500pt}%
\definecolor{currentstroke}{rgb}{1.000000,0.501961,0.054902}%
\pgfsetstrokecolor{currentstroke}%
\pgfsetstrokeopacity{0.700000}%
\pgfsetdash{{2.000000pt}{3.300000pt}}{0.000000pt}%
\pgfpathmoveto{\pgfqpoint{0.779028in}{1.555319in}}%
\pgfpathlineto{\pgfqpoint{2.455631in}{1.555319in}}%
\pgfpathlineto{\pgfqpoint{2.455673in}{2.527860in}}%
\pgfpathlineto{\pgfqpoint{4.970537in}{2.527860in}}%
\pgfusepath{stroke}%
\end{pgfscope}%
\begin{pgfscope}%
\pgfpathrectangle{\pgfqpoint{0.779028in}{0.582778in}}{\pgfqpoint{4.191509in}{2.431352in}}%
\pgfusepath{clip}%
\pgfsetrectcap%
\pgfsetroundjoin%
\pgfsetlinewidth{3.011250pt}%
\definecolor{currentstroke}{rgb}{0.349020,0.349020,0.349020}%
\pgfsetstrokecolor{currentstroke}%
\pgfsetdash{}{0pt}%
\pgfpathmoveto{\pgfqpoint{0.779028in}{1.555319in}}%
\pgfpathlineto{\pgfqpoint{0.820943in}{1.555319in}}%
\pgfpathlineto{\pgfqpoint{0.862858in}{1.555319in}}%
\pgfpathlineto{\pgfqpoint{0.904773in}{1.555319in}}%
\pgfpathlineto{\pgfqpoint{0.946688in}{1.555319in}}%
\pgfpathlineto{\pgfqpoint{0.988603in}{1.555319in}}%
\pgfpathlineto{\pgfqpoint{1.030518in}{1.555319in}}%
\pgfpathlineto{\pgfqpoint{1.072433in}{1.555319in}}%
\pgfpathlineto{\pgfqpoint{1.114349in}{1.555319in}}%
\pgfpathlineto{\pgfqpoint{1.156264in}{1.555319in}}%
\pgfpathlineto{\pgfqpoint{1.198179in}{1.555319in}}%
\pgfpathlineto{\pgfqpoint{1.240094in}{1.555319in}}%
\pgfpathlineto{\pgfqpoint{1.282009in}{1.555319in}}%
\pgfpathlineto{\pgfqpoint{1.323924in}{1.555319in}}%
\pgfpathlineto{\pgfqpoint{1.365839in}{1.555319in}}%
\pgfpathlineto{\pgfqpoint{1.407754in}{1.555319in}}%
\pgfpathlineto{\pgfqpoint{1.449669in}{1.555319in}}%
\pgfpathlineto{\pgfqpoint{1.491584in}{1.555319in}}%
\pgfpathlineto{\pgfqpoint{1.533499in}{1.555319in}}%
\pgfpathlineto{\pgfqpoint{1.575415in}{1.555319in}}%
\pgfpathlineto{\pgfqpoint{1.617330in}{1.555319in}}%
\pgfpathlineto{\pgfqpoint{1.659245in}{1.555319in}}%
\pgfpathlineto{\pgfqpoint{1.701160in}{1.555319in}}%
\pgfpathlineto{\pgfqpoint{1.743075in}{1.555319in}}%
\pgfpathlineto{\pgfqpoint{1.784990in}{1.555319in}}%
\pgfpathlineto{\pgfqpoint{1.826905in}{1.555319in}}%
\pgfpathlineto{\pgfqpoint{1.868820in}{1.555319in}}%
\pgfpathlineto{\pgfqpoint{1.910735in}{1.555319in}}%
\pgfpathlineto{\pgfqpoint{1.952650in}{1.555319in}}%
\pgfpathlineto{\pgfqpoint{1.994565in}{1.555319in}}%
\pgfpathlineto{\pgfqpoint{2.036481in}{1.555319in}}%
\pgfpathlineto{\pgfqpoint{2.078396in}{1.555319in}}%
\pgfpathlineto{\pgfqpoint{2.120311in}{1.555319in}}%
\pgfpathlineto{\pgfqpoint{2.162226in}{1.555319in}}%
\pgfpathlineto{\pgfqpoint{2.204141in}{1.555319in}}%
\pgfpathlineto{\pgfqpoint{2.246056in}{1.555319in}}%
\pgfpathlineto{\pgfqpoint{2.287971in}{1.555319in}}%
\pgfpathlineto{\pgfqpoint{2.329886in}{1.555319in}}%
\pgfpathlineto{\pgfqpoint{2.371801in}{1.555319in}}%
\pgfpathlineto{\pgfqpoint{2.413716in}{1.555319in}}%
\pgfpathlineto{\pgfqpoint{2.455631in}{1.555319in}}%
\pgfpathlineto{\pgfqpoint{2.497547in}{1.602750in}}%
\pgfpathlineto{\pgfqpoint{2.539462in}{1.647868in}}%
\pgfpathlineto{\pgfqpoint{2.581377in}{1.690786in}}%
\pgfpathlineto{\pgfqpoint{2.623292in}{1.731610in}}%
\pgfpathlineto{\pgfqpoint{2.665207in}{1.770444in}}%
\pgfpathlineto{\pgfqpoint{2.707122in}{1.807384in}}%
\pgfpathlineto{\pgfqpoint{2.749037in}{1.842522in}}%
\pgfpathlineto{\pgfqpoint{2.790952in}{1.875946in}}%
\pgfpathlineto{\pgfqpoint{2.832867in}{1.907740in}}%
\pgfpathlineto{\pgfqpoint{2.874782in}{1.937984in}}%
\pgfpathlineto{\pgfqpoint{2.916697in}{1.966752in}}%
\pgfpathlineto{\pgfqpoint{2.958613in}{1.994118in}}%
\pgfpathlineto{\pgfqpoint{3.000528in}{2.020149in}}%
\pgfpathlineto{\pgfqpoint{3.042443in}{2.044910in}}%
\pgfpathlineto{\pgfqpoint{3.084358in}{2.068464in}}%
\pgfpathlineto{\pgfqpoint{3.126273in}{2.090869in}}%
\pgfpathlineto{\pgfqpoint{3.168188in}{2.112181in}}%
\pgfpathlineto{\pgfqpoint{3.210103in}{2.132454in}}%
\pgfpathlineto{\pgfqpoint{3.252018in}{2.151738in}}%
\pgfpathlineto{\pgfqpoint{3.293933in}{2.170082in}}%
\pgfpathlineto{\pgfqpoint{3.335848in}{2.187531in}}%
\pgfpathlineto{\pgfqpoint{3.377763in}{2.204129in}}%
\pgfpathlineto{\pgfqpoint{3.419679in}{2.219917in}}%
\pgfpathlineto{\pgfqpoint{3.461594in}{2.234936in}}%
\pgfpathlineto{\pgfqpoint{3.503509in}{2.249222in}}%
\pgfpathlineto{\pgfqpoint{3.545424in}{2.262811in}}%
\pgfpathlineto{\pgfqpoint{3.587339in}{2.275738in}}%
\pgfpathlineto{\pgfqpoint{3.629254in}{2.288034in}}%
\pgfpathlineto{\pgfqpoint{3.671169in}{2.299730in}}%
\pgfpathlineto{\pgfqpoint{3.713084in}{2.310856in}}%
\pgfpathlineto{\pgfqpoint{3.754999in}{2.321440in}}%
\pgfpathlineto{\pgfqpoint{3.796914in}{2.331507in}}%
\pgfpathlineto{\pgfqpoint{3.838829in}{2.341083in}}%
\pgfpathlineto{\pgfqpoint{3.880745in}{2.350192in}}%
\pgfpathlineto{\pgfqpoint{3.922660in}{2.358857in}}%
\pgfpathlineto{\pgfqpoint{3.964575in}{2.367099in}}%
\pgfpathlineto{\pgfqpoint{4.006490in}{2.374940in}}%
\pgfpathlineto{\pgfqpoint{4.048405in}{2.382398in}}%
\pgfpathlineto{\pgfqpoint{4.090320in}{2.389492in}}%
\pgfpathlineto{\pgfqpoint{4.132235in}{2.396240in}}%
\pgfpathlineto{\pgfqpoint{4.174150in}{2.402659in}}%
\pgfpathlineto{\pgfqpoint{4.216065in}{2.408765in}}%
\pgfpathlineto{\pgfqpoint{4.257980in}{2.414574in}}%
\pgfpathlineto{\pgfqpoint{4.299895in}{2.420099in}}%
\pgfpathlineto{\pgfqpoint{4.341811in}{2.425354in}}%
\pgfpathlineto{\pgfqpoint{4.383726in}{2.430354in}}%
\pgfpathlineto{\pgfqpoint{4.425641in}{2.435109in}}%
\pgfpathlineto{\pgfqpoint{4.467556in}{2.439632in}}%
\pgfpathlineto{\pgfqpoint{4.509471in}{2.443935in}}%
\pgfpathlineto{\pgfqpoint{4.551386in}{2.448028in}}%
\pgfpathlineto{\pgfqpoint{4.593301in}{2.451922in}}%
\pgfpathlineto{\pgfqpoint{4.635216in}{2.455625in}}%
\pgfpathlineto{\pgfqpoint{4.677131in}{2.459148in}}%
\pgfpathlineto{\pgfqpoint{4.719046in}{2.462499in}}%
\pgfpathlineto{\pgfqpoint{4.760961in}{2.465687in}}%
\pgfpathlineto{\pgfqpoint{4.802877in}{2.468719in}}%
\pgfpathlineto{\pgfqpoint{4.844792in}{2.471604in}}%
\pgfpathlineto{\pgfqpoint{4.886707in}{2.474347in}}%
\pgfpathlineto{\pgfqpoint{4.928622in}{2.476957in}}%
\pgfpathlineto{\pgfqpoint{4.970537in}{2.479440in}}%
\pgfusepath{stroke}%
\end{pgfscope}%
\begin{pgfscope}%
\pgfsetrectcap%
\pgfsetmiterjoin%
\pgfsetlinewidth{0.803000pt}%
\definecolor{currentstroke}{rgb}{0.000000,0.000000,0.000000}%
\pgfsetstrokecolor{currentstroke}%
\pgfsetdash{}{0pt}%
\pgfpathmoveto{\pgfqpoint{0.779028in}{0.582778in}}%
\pgfpathlineto{\pgfqpoint{0.779028in}{3.014130in}}%
\pgfusepath{stroke}%
\end{pgfscope}%
\begin{pgfscope}%
\pgfsetrectcap%
\pgfsetmiterjoin%
\pgfsetlinewidth{0.803000pt}%
\definecolor{currentstroke}{rgb}{0.000000,0.000000,0.000000}%
\pgfsetstrokecolor{currentstroke}%
\pgfsetdash{}{0pt}%
\pgfpathmoveto{\pgfqpoint{4.970537in}{0.582778in}}%
\pgfpathlineto{\pgfqpoint{4.970537in}{3.014130in}}%
\pgfusepath{stroke}%
\end{pgfscope}%
\begin{pgfscope}%
\pgfsetrectcap%
\pgfsetmiterjoin%
\pgfsetlinewidth{0.803000pt}%
\definecolor{currentstroke}{rgb}{0.000000,0.000000,0.000000}%
\pgfsetstrokecolor{currentstroke}%
\pgfsetdash{}{0pt}%
\pgfpathmoveto{\pgfqpoint{0.779028in}{0.582778in}}%
\pgfpathlineto{\pgfqpoint{4.970537in}{0.582778in}}%
\pgfusepath{stroke}%
\end{pgfscope}%
\begin{pgfscope}%
\pgfsetrectcap%
\pgfsetmiterjoin%
\pgfsetlinewidth{0.803000pt}%
\definecolor{currentstroke}{rgb}{0.000000,0.000000,0.000000}%
\pgfsetstrokecolor{currentstroke}%
\pgfsetdash{}{0pt}%
\pgfpathmoveto{\pgfqpoint{0.779028in}{3.014130in}}%
\pgfpathlineto{\pgfqpoint{4.970537in}{3.014130in}}%
\pgfusepath{stroke}%
\end{pgfscope}%
\begin{pgfscope}%
\pgfsetbuttcap%
\pgfsetmiterjoin%
\definecolor{currentfill}{rgb}{1.000000,1.000000,1.000000}%
\pgfsetfillcolor{currentfill}%
\pgfsetfillopacity{0.800000}%
\pgfsetlinewidth{1.003750pt}%
\definecolor{currentstroke}{rgb}{0.800000,0.800000,0.800000}%
\pgfsetstrokecolor{currentstroke}%
\pgfsetstrokeopacity{0.800000}%
\pgfsetdash{}{0pt}%
\pgfpathmoveto{\pgfqpoint{0.876250in}{2.252837in}}%
\pgfpathlineto{\pgfqpoint{1.915452in}{2.252837in}}%
\pgfpathquadraticcurveto{\pgfqpoint{1.943230in}{2.252837in}}{\pgfqpoint{1.943230in}{2.280615in}}%
\pgfpathlineto{\pgfqpoint{1.943230in}{2.916908in}}%
\pgfpathquadraticcurveto{\pgfqpoint{1.943230in}{2.944686in}}{\pgfqpoint{1.915452in}{2.944686in}}%
\pgfpathlineto{\pgfqpoint{0.876250in}{2.944686in}}%
\pgfpathquadraticcurveto{\pgfqpoint{0.848472in}{2.944686in}}{\pgfqpoint{0.848472in}{2.916908in}}%
\pgfpathlineto{\pgfqpoint{0.848472in}{2.280615in}}%
\pgfpathquadraticcurveto{\pgfqpoint{0.848472in}{2.252837in}}{\pgfqpoint{0.876250in}{2.252837in}}%
\pgfpathlineto{\pgfqpoint{0.876250in}{2.252837in}}%
\pgfpathclose%
\pgfusepath{stroke,fill}%
\end{pgfscope}%
\begin{pgfscope}%
\pgfsetbuttcap%
\pgfsetroundjoin%
\pgfsetlinewidth{2.007500pt}%
\definecolor{currentstroke}{rgb}{0.000000,0.419608,0.643137}%
\pgfsetstrokecolor{currentstroke}%
\pgfsetstrokeopacity{0.700000}%
\pgfsetdash{{7.400000pt}{3.200000pt}}{0.000000pt}%
\pgfpathmoveto{\pgfqpoint{0.904028in}{2.805273in}}%
\pgfpathlineto{\pgfqpoint{1.042917in}{2.805273in}}%
\pgfpathlineto{\pgfqpoint{1.181806in}{2.805273in}}%
\pgfusepath{stroke}%
\end{pgfscope}%
\begin{pgfscope}%
\definecolor{textcolor}{rgb}{0.000000,0.000000,0.000000}%
\pgfsetstrokecolor{textcolor}%
\pgfsetfillcolor{textcolor}%
\pgftext[x=1.292917in,y=2.756661in,left,base]{\color{textcolor}\sffamily\fontsize{10.000000}{12.000000}\selectfont \(\displaystyle 1-e^{-\frac{t-\theta}{\tau} }\)}%
\end{pgfscope}%
\begin{pgfscope}%
\pgfsetbuttcap%
\pgfsetroundjoin%
\pgfsetlinewidth{2.007500pt}%
\definecolor{currentstroke}{rgb}{1.000000,0.501961,0.054902}%
\pgfsetstrokecolor{currentstroke}%
\pgfsetstrokeopacity{0.700000}%
\pgfsetdash{{2.000000pt}{3.300000pt}}{0.000000pt}%
\pgfpathmoveto{\pgfqpoint{0.904028in}{2.601415in}}%
\pgfpathlineto{\pgfqpoint{1.042917in}{2.601415in}}%
\pgfpathlineto{\pgfqpoint{1.181806in}{2.601415in}}%
\pgfusepath{stroke}%
\end{pgfscope}%
\begin{pgfscope}%
\definecolor{textcolor}{rgb}{0.000000,0.000000,0.000000}%
\pgfsetstrokecolor{textcolor}%
\pgfsetfillcolor{textcolor}%
\pgftext[x=1.292917in,y=2.552804in,left,base]{\color{textcolor}\sffamily\fontsize{10.000000}{12.000000}\selectfont \(\displaystyle H(t- \theta)\)}%
\end{pgfscope}%
\begin{pgfscope}%
\pgfsetrectcap%
\pgfsetroundjoin%
\pgfsetlinewidth{3.011250pt}%
\definecolor{currentstroke}{rgb}{0.349020,0.349020,0.349020}%
\pgfsetstrokecolor{currentstroke}%
\pgfsetdash{}{0pt}%
\pgfpathmoveto{\pgfqpoint{0.904028in}{2.391726in}}%
\pgfpathlineto{\pgfqpoint{1.042917in}{2.391726in}}%
\pgfpathlineto{\pgfqpoint{1.181806in}{2.391726in}}%
\pgfusepath{stroke}%
\end{pgfscope}%
\begin{pgfscope}%
\definecolor{textcolor}{rgb}{0.000000,0.000000,0.000000}%
\pgfsetstrokecolor{textcolor}%
\pgfsetfillcolor{textcolor}%
\pgftext[x=1.292917in,y=2.343115in,left,base]{\color{textcolor}\sffamily\fontsize{10.000000}{12.000000}\selectfont \(\displaystyle y(t)\)}%
\end{pgfscope}%
\end{pgfpicture}%
\makeatother%
\endgroup%

    \caption{Time domain plot of a first-order plus dead time model showing individual components of the model and the composite function $y(t)$. Model parameters used: $K= \Delta u = 1$, $\tau=2$, $\theta=4$.}
    \label{fig:fopdt}
\end{figure}

So far, only open-loop systems were discussed. With the system parameters in hand, it is now possible to design a controller around the system and close the loop to achieve a stable system. This is shown in the next section.

\clearpage
\subsection{PID Controller Basics}
\label{sec:pid_tuning_rules}
While there are many different types controllers, like the bang–bang controller utilized in the original lab temperature controller, which turns on at a certain threshold and turns off at another threshold, that resulted in the saw-tooth shaped room temperature curve shown in figure \ref{fig:lab_temperature_start_of_project}, a continuous control system is desired to keep fluctions to a minimum. The most commonly used controller type for non-integrating systems is the proportional–integral–derivative (PID) controller \cite{pid_in_industry}. A non-integrating system is a system without memory, that does not depend on previous inputs. Given the same input, a non-integrating will always return to the same steady state. The advantage of applying a PID controller is, that the controller does not need any special knowledge of system model. A universal PID is simple to implement and can be tuned to control a wide range of systems. While there are many different variations of the PID algorithm \cite{pid_controller}, this section only introduces the basic, parallel, PID controller and deals with some of the shortcomings in a practical application.

In order to extend the FOPDT system, derived in the previous section \ref{sec:temperature_control_model}, with the PID controller, one must move to a closed-loop system. Extending \ref{fig:closed_loop} and inserting a new control block into the transfer function yields figure \ref{fig:closed_loop_pid}.
\begin{figure}[ht]
    \centering
    \scalebox{1}{%
        \import{figures/}{closed_loop_pid.tex}
    }% scalebox
    \caption{Closed-loop system with a PID controller.}
    \label{fig:closed_loop_pid}
\end{figure}

The error signal $E(s)$ used by the PID controller is the difference between the setpoint and the control parameter, in this case the room temperature. The transfer function of the PID controller can be split into three parts. A proportional part, that is proportional to the error representing the present, an integral part, that is proportional to the accumulated error, representing the past, and a derivative part, that is proportional to the change in the the error, extrapolating into the future.
\begin{align}
    c(t) &= k_p e(t) + k_i \int_0^t e(\tau) \,d\tau + k_d \frac{\mathrm{d}e(t)}{\mathrm{d}t} \label{eqn:pid_controller}\\
    C(s) &= k_p + k_i \frac{1}{s} + k_d s \label{eqn:pid_controller_laplace}
\end{align}

The following discussion will mostly focus on equation \ref{eqn:pid_controller}, because, the time-domain equation is the one, that can be implemented in software. As hinted above, there are a few shortcommings with the classic PID equation, when used in a real system, when there are dynamic changes of the PID parameters, e.g. the setpoint or $k_i$.

The first problem to be addressed is occuring, when changing the PID parameter $k_i$. Assuming a settled system without external disturbances, the output is fully determined by the integrator value. Now, when $k_i$ is changed, the output immediate changes, due to the change of the integral term. This is unintended. To fix this, the integral term must changed to
\begin{equation}
    k_i \int_0^t e(\tau) \,d\tau \Rightarrow \int_0^t k_i(\tau) e(\tau) \,d\tau \,.
\end{equation}

This way, when adjusting $k_i$, its new value is applied to future error values only and there is no sudden kick.

The next issue is called \text{derivative kick}. When looking at the derivative part of equation \ref{eqn:pid_controller}, it can be seen that when instantantly changing the setpoint, as in a step function, $\frac{\mathrm{d}e(t)}{\mathrm{d}t} \to \infty$. This behaviour is not intended and to fix this, the derivative part can be modified as follows.
\begin{align}
    \frac{\mathrm{d}e(t)}{\mathrm{d}t} &= \frac{\mathrm{d}\left(u(t) - y(t)\right)}{\mathrm{d}t} \nonumber\\
    &= \underbrace{\cancel{\frac{\mathrm{d}u(t)}{\mathrm{d}t}}}_{\to \infty} - \frac{\mathrm{d}y(t)}{\mathrm{d}t} \nonumber\\
    &=- \frac{\mathrm{d}y(t)}{\mathrm{d}t}
\end{align}

The new derivative term is equal to the unmodfied one, except in case of setpoint changes. Removing the setpoint from the equation, the controller behaves as intended. This solution is sometimes called \textit{derivative on measurement} as opposed to \textit{derivative on error}.

The derivative term is also the cause the final problem to be discussed. Assuming a noisy input and by chance there is a very short input spike due to noise. The differential in the derivative term will again be sent to very high values, pushing the output await from the optimal value forcing the controller to rebalance.
\begin{figure}[hb]
    \centering
    %% Creator: Matplotlib, PGF backend
%%
%% To include the figure in your LaTeX document, write
%%   \input{<filename>.pgf}
%%
%% Make sure the required packages are loaded in your preamble
%%   \usepackage{pgf}
%%
%% Also ensure that all the required font packages are loaded; for instance,
%% the lmodern package is sometimes necessary when using math font.
%%   \usepackage{lmodern}
%%
%% Figures using additional raster images can only be included by \input if
%% they are in the same directory as the main LaTeX file. For loading figures
%% from other directories you can use the `import` package
%%   \usepackage{import}
%%
%% and then include the figures with
%%   \import{<path to file>}{<filename>.pgf}
%%
%% Matplotlib used the following preamble
%%   \usepackage{siunitx}
%%   \usepackage{fontspec}
%%
\begingroup%
\makeatletter%
\begin{pgfpicture}%
\pgfpathrectangle{\pgfpointorigin}{\pgfqpoint{5.431103in}{3.356606in}}%
\pgfusepath{use as bounding box, clip}%
\begin{pgfscope}%
\pgfsetbuttcap%
\pgfsetmiterjoin%
\definecolor{currentfill}{rgb}{1.000000,1.000000,1.000000}%
\pgfsetfillcolor{currentfill}%
\pgfsetlinewidth{0.000000pt}%
\definecolor{currentstroke}{rgb}{1.000000,1.000000,1.000000}%
\pgfsetstrokecolor{currentstroke}%
\pgfsetdash{}{0pt}%
\pgfpathmoveto{\pgfqpoint{0.000000in}{0.000000in}}%
\pgfpathlineto{\pgfqpoint{5.431103in}{0.000000in}}%
\pgfpathlineto{\pgfqpoint{5.431103in}{3.356606in}}%
\pgfpathlineto{\pgfqpoint{0.000000in}{3.356606in}}%
\pgfpathlineto{\pgfqpoint{0.000000in}{0.000000in}}%
\pgfpathclose%
\pgfusepath{fill}%
\end{pgfscope}%
\begin{pgfscope}%
\pgfsetbuttcap%
\pgfsetmiterjoin%
\definecolor{currentfill}{rgb}{1.000000,1.000000,1.000000}%
\pgfsetfillcolor{currentfill}%
\pgfsetlinewidth{0.000000pt}%
\definecolor{currentstroke}{rgb}{0.000000,0.000000,0.000000}%
\pgfsetstrokecolor{currentstroke}%
\pgfsetstrokeopacity{0.000000}%
\pgfsetdash{}{0pt}%
\pgfpathmoveto{\pgfqpoint{0.594124in}{0.540713in}}%
\pgfpathlineto{\pgfqpoint{5.281103in}{0.540713in}}%
\pgfpathlineto{\pgfqpoint{5.281103in}{3.206606in}}%
\pgfpathlineto{\pgfqpoint{0.594124in}{3.206606in}}%
\pgfpathlineto{\pgfqpoint{0.594124in}{0.540713in}}%
\pgfpathclose%
\pgfusepath{fill}%
\end{pgfscope}%
\begin{pgfscope}%
\pgfsetbuttcap%
\pgfsetroundjoin%
\definecolor{currentfill}{rgb}{0.000000,0.000000,0.000000}%
\pgfsetfillcolor{currentfill}%
\pgfsetlinewidth{0.803000pt}%
\definecolor{currentstroke}{rgb}{0.000000,0.000000,0.000000}%
\pgfsetstrokecolor{currentstroke}%
\pgfsetdash{}{0pt}%
\pgfsys@defobject{currentmarker}{\pgfqpoint{0.000000in}{-0.048611in}}{\pgfqpoint{0.000000in}{0.000000in}}{%
\pgfpathmoveto{\pgfqpoint{0.000000in}{0.000000in}}%
\pgfpathlineto{\pgfqpoint{0.000000in}{-0.048611in}}%
\pgfusepath{stroke,fill}%
}%
\begin{pgfscope}%
\pgfsys@transformshift{0.807169in}{0.540713in}%
\pgfsys@useobject{currentmarker}{}%
\end{pgfscope}%
\end{pgfscope}%
\begin{pgfscope}%
\definecolor{textcolor}{rgb}{0.000000,0.000000,0.000000}%
\pgfsetstrokecolor{textcolor}%
\pgfsetfillcolor{textcolor}%
\pgftext[x=0.807169in,y=0.443491in,,top]{\color{textcolor}\rmfamily\fontsize{8.000000}{9.600000}\selectfont \(\displaystyle {10^{-2}}\)}%
\end{pgfscope}%
\begin{pgfscope}%
\pgfsetbuttcap%
\pgfsetroundjoin%
\definecolor{currentfill}{rgb}{0.000000,0.000000,0.000000}%
\pgfsetfillcolor{currentfill}%
\pgfsetlinewidth{0.803000pt}%
\definecolor{currentstroke}{rgb}{0.000000,0.000000,0.000000}%
\pgfsetstrokecolor{currentstroke}%
\pgfsetdash{}{0pt}%
\pgfsys@defobject{currentmarker}{\pgfqpoint{0.000000in}{-0.048611in}}{\pgfqpoint{0.000000in}{0.000000in}}{%
\pgfpathmoveto{\pgfqpoint{0.000000in}{0.000000in}}%
\pgfpathlineto{\pgfqpoint{0.000000in}{-0.048611in}}%
\pgfusepath{stroke,fill}%
}%
\begin{pgfscope}%
\pgfsys@transformshift{1.517317in}{0.540713in}%
\pgfsys@useobject{currentmarker}{}%
\end{pgfscope}%
\end{pgfscope}%
\begin{pgfscope}%
\definecolor{textcolor}{rgb}{0.000000,0.000000,0.000000}%
\pgfsetstrokecolor{textcolor}%
\pgfsetfillcolor{textcolor}%
\pgftext[x=1.517317in,y=0.443491in,,top]{\color{textcolor}\rmfamily\fontsize{8.000000}{9.600000}\selectfont \(\displaystyle {10^{-1}}\)}%
\end{pgfscope}%
\begin{pgfscope}%
\pgfsetbuttcap%
\pgfsetroundjoin%
\definecolor{currentfill}{rgb}{0.000000,0.000000,0.000000}%
\pgfsetfillcolor{currentfill}%
\pgfsetlinewidth{0.803000pt}%
\definecolor{currentstroke}{rgb}{0.000000,0.000000,0.000000}%
\pgfsetstrokecolor{currentstroke}%
\pgfsetdash{}{0pt}%
\pgfsys@defobject{currentmarker}{\pgfqpoint{0.000000in}{-0.048611in}}{\pgfqpoint{0.000000in}{0.000000in}}{%
\pgfpathmoveto{\pgfqpoint{0.000000in}{0.000000in}}%
\pgfpathlineto{\pgfqpoint{0.000000in}{-0.048611in}}%
\pgfusepath{stroke,fill}%
}%
\begin{pgfscope}%
\pgfsys@transformshift{2.227465in}{0.540713in}%
\pgfsys@useobject{currentmarker}{}%
\end{pgfscope}%
\end{pgfscope}%
\begin{pgfscope}%
\definecolor{textcolor}{rgb}{0.000000,0.000000,0.000000}%
\pgfsetstrokecolor{textcolor}%
\pgfsetfillcolor{textcolor}%
\pgftext[x=2.227465in,y=0.443491in,,top]{\color{textcolor}\rmfamily\fontsize{8.000000}{9.600000}\selectfont \(\displaystyle {10^{0}}\)}%
\end{pgfscope}%
\begin{pgfscope}%
\pgfsetbuttcap%
\pgfsetroundjoin%
\definecolor{currentfill}{rgb}{0.000000,0.000000,0.000000}%
\pgfsetfillcolor{currentfill}%
\pgfsetlinewidth{0.803000pt}%
\definecolor{currentstroke}{rgb}{0.000000,0.000000,0.000000}%
\pgfsetstrokecolor{currentstroke}%
\pgfsetdash{}{0pt}%
\pgfsys@defobject{currentmarker}{\pgfqpoint{0.000000in}{-0.048611in}}{\pgfqpoint{0.000000in}{0.000000in}}{%
\pgfpathmoveto{\pgfqpoint{0.000000in}{0.000000in}}%
\pgfpathlineto{\pgfqpoint{0.000000in}{-0.048611in}}%
\pgfusepath{stroke,fill}%
}%
\begin{pgfscope}%
\pgfsys@transformshift{2.937613in}{0.540713in}%
\pgfsys@useobject{currentmarker}{}%
\end{pgfscope}%
\end{pgfscope}%
\begin{pgfscope}%
\definecolor{textcolor}{rgb}{0.000000,0.000000,0.000000}%
\pgfsetstrokecolor{textcolor}%
\pgfsetfillcolor{textcolor}%
\pgftext[x=2.937613in,y=0.443491in,,top]{\color{textcolor}\rmfamily\fontsize{8.000000}{9.600000}\selectfont \(\displaystyle {10^{1}}\)}%
\end{pgfscope}%
\begin{pgfscope}%
\pgfsetbuttcap%
\pgfsetroundjoin%
\definecolor{currentfill}{rgb}{0.000000,0.000000,0.000000}%
\pgfsetfillcolor{currentfill}%
\pgfsetlinewidth{0.803000pt}%
\definecolor{currentstroke}{rgb}{0.000000,0.000000,0.000000}%
\pgfsetstrokecolor{currentstroke}%
\pgfsetdash{}{0pt}%
\pgfsys@defobject{currentmarker}{\pgfqpoint{0.000000in}{-0.048611in}}{\pgfqpoint{0.000000in}{0.000000in}}{%
\pgfpathmoveto{\pgfqpoint{0.000000in}{0.000000in}}%
\pgfpathlineto{\pgfqpoint{0.000000in}{-0.048611in}}%
\pgfusepath{stroke,fill}%
}%
\begin{pgfscope}%
\pgfsys@transformshift{3.647762in}{0.540713in}%
\pgfsys@useobject{currentmarker}{}%
\end{pgfscope}%
\end{pgfscope}%
\begin{pgfscope}%
\definecolor{textcolor}{rgb}{0.000000,0.000000,0.000000}%
\pgfsetstrokecolor{textcolor}%
\pgfsetfillcolor{textcolor}%
\pgftext[x=3.647762in,y=0.443491in,,top]{\color{textcolor}\rmfamily\fontsize{8.000000}{9.600000}\selectfont \(\displaystyle {10^{2}}\)}%
\end{pgfscope}%
\begin{pgfscope}%
\pgfsetbuttcap%
\pgfsetroundjoin%
\definecolor{currentfill}{rgb}{0.000000,0.000000,0.000000}%
\pgfsetfillcolor{currentfill}%
\pgfsetlinewidth{0.803000pt}%
\definecolor{currentstroke}{rgb}{0.000000,0.000000,0.000000}%
\pgfsetstrokecolor{currentstroke}%
\pgfsetdash{}{0pt}%
\pgfsys@defobject{currentmarker}{\pgfqpoint{0.000000in}{-0.048611in}}{\pgfqpoint{0.000000in}{0.000000in}}{%
\pgfpathmoveto{\pgfqpoint{0.000000in}{0.000000in}}%
\pgfpathlineto{\pgfqpoint{0.000000in}{-0.048611in}}%
\pgfusepath{stroke,fill}%
}%
\begin{pgfscope}%
\pgfsys@transformshift{4.357910in}{0.540713in}%
\pgfsys@useobject{currentmarker}{}%
\end{pgfscope}%
\end{pgfscope}%
\begin{pgfscope}%
\definecolor{textcolor}{rgb}{0.000000,0.000000,0.000000}%
\pgfsetstrokecolor{textcolor}%
\pgfsetfillcolor{textcolor}%
\pgftext[x=4.357910in,y=0.443491in,,top]{\color{textcolor}\rmfamily\fontsize{8.000000}{9.600000}\selectfont \(\displaystyle {10^{3}}\)}%
\end{pgfscope}%
\begin{pgfscope}%
\pgfsetbuttcap%
\pgfsetroundjoin%
\definecolor{currentfill}{rgb}{0.000000,0.000000,0.000000}%
\pgfsetfillcolor{currentfill}%
\pgfsetlinewidth{0.803000pt}%
\definecolor{currentstroke}{rgb}{0.000000,0.000000,0.000000}%
\pgfsetstrokecolor{currentstroke}%
\pgfsetdash{}{0pt}%
\pgfsys@defobject{currentmarker}{\pgfqpoint{0.000000in}{-0.048611in}}{\pgfqpoint{0.000000in}{0.000000in}}{%
\pgfpathmoveto{\pgfqpoint{0.000000in}{0.000000in}}%
\pgfpathlineto{\pgfqpoint{0.000000in}{-0.048611in}}%
\pgfusepath{stroke,fill}%
}%
\begin{pgfscope}%
\pgfsys@transformshift{5.068058in}{0.540713in}%
\pgfsys@useobject{currentmarker}{}%
\end{pgfscope}%
\end{pgfscope}%
\begin{pgfscope}%
\definecolor{textcolor}{rgb}{0.000000,0.000000,0.000000}%
\pgfsetstrokecolor{textcolor}%
\pgfsetfillcolor{textcolor}%
\pgftext[x=5.068058in,y=0.443491in,,top]{\color{textcolor}\rmfamily\fontsize{8.000000}{9.600000}\selectfont \(\displaystyle {10^{4}}\)}%
\end{pgfscope}%
\begin{pgfscope}%
\pgfsetbuttcap%
\pgfsetroundjoin%
\definecolor{currentfill}{rgb}{0.000000,0.000000,0.000000}%
\pgfsetfillcolor{currentfill}%
\pgfsetlinewidth{0.602250pt}%
\definecolor{currentstroke}{rgb}{0.000000,0.000000,0.000000}%
\pgfsetstrokecolor{currentstroke}%
\pgfsetdash{}{0pt}%
\pgfsys@defobject{currentmarker}{\pgfqpoint{0.000000in}{-0.027778in}}{\pgfqpoint{0.000000in}{0.000000in}}{%
\pgfpathmoveto{\pgfqpoint{0.000000in}{0.000000in}}%
\pgfpathlineto{\pgfqpoint{0.000000in}{-0.027778in}}%
\pgfusepath{stroke,fill}%
}%
\begin{pgfscope}%
\pgfsys@transformshift{0.649623in}{0.540713in}%
\pgfsys@useobject{currentmarker}{}%
\end{pgfscope}%
\end{pgfscope}%
\begin{pgfscope}%
\pgfsetbuttcap%
\pgfsetroundjoin%
\definecolor{currentfill}{rgb}{0.000000,0.000000,0.000000}%
\pgfsetfillcolor{currentfill}%
\pgfsetlinewidth{0.602250pt}%
\definecolor{currentstroke}{rgb}{0.000000,0.000000,0.000000}%
\pgfsetstrokecolor{currentstroke}%
\pgfsetdash{}{0pt}%
\pgfsys@defobject{currentmarker}{\pgfqpoint{0.000000in}{-0.027778in}}{\pgfqpoint{0.000000in}{0.000000in}}{%
\pgfpathmoveto{\pgfqpoint{0.000000in}{0.000000in}}%
\pgfpathlineto{\pgfqpoint{0.000000in}{-0.027778in}}%
\pgfusepath{stroke,fill}%
}%
\begin{pgfscope}%
\pgfsys@transformshift{0.697165in}{0.540713in}%
\pgfsys@useobject{currentmarker}{}%
\end{pgfscope}%
\end{pgfscope}%
\begin{pgfscope}%
\pgfsetbuttcap%
\pgfsetroundjoin%
\definecolor{currentfill}{rgb}{0.000000,0.000000,0.000000}%
\pgfsetfillcolor{currentfill}%
\pgfsetlinewidth{0.602250pt}%
\definecolor{currentstroke}{rgb}{0.000000,0.000000,0.000000}%
\pgfsetstrokecolor{currentstroke}%
\pgfsetdash{}{0pt}%
\pgfsys@defobject{currentmarker}{\pgfqpoint{0.000000in}{-0.027778in}}{\pgfqpoint{0.000000in}{0.000000in}}{%
\pgfpathmoveto{\pgfqpoint{0.000000in}{0.000000in}}%
\pgfpathlineto{\pgfqpoint{0.000000in}{-0.027778in}}%
\pgfusepath{stroke,fill}%
}%
\begin{pgfscope}%
\pgfsys@transformshift{0.738348in}{0.540713in}%
\pgfsys@useobject{currentmarker}{}%
\end{pgfscope}%
\end{pgfscope}%
\begin{pgfscope}%
\pgfsetbuttcap%
\pgfsetroundjoin%
\definecolor{currentfill}{rgb}{0.000000,0.000000,0.000000}%
\pgfsetfillcolor{currentfill}%
\pgfsetlinewidth{0.602250pt}%
\definecolor{currentstroke}{rgb}{0.000000,0.000000,0.000000}%
\pgfsetstrokecolor{currentstroke}%
\pgfsetdash{}{0pt}%
\pgfsys@defobject{currentmarker}{\pgfqpoint{0.000000in}{-0.027778in}}{\pgfqpoint{0.000000in}{0.000000in}}{%
\pgfpathmoveto{\pgfqpoint{0.000000in}{0.000000in}}%
\pgfpathlineto{\pgfqpoint{0.000000in}{-0.027778in}}%
\pgfusepath{stroke,fill}%
}%
\begin{pgfscope}%
\pgfsys@transformshift{0.774674in}{0.540713in}%
\pgfsys@useobject{currentmarker}{}%
\end{pgfscope}%
\end{pgfscope}%
\begin{pgfscope}%
\pgfsetbuttcap%
\pgfsetroundjoin%
\definecolor{currentfill}{rgb}{0.000000,0.000000,0.000000}%
\pgfsetfillcolor{currentfill}%
\pgfsetlinewidth{0.602250pt}%
\definecolor{currentstroke}{rgb}{0.000000,0.000000,0.000000}%
\pgfsetstrokecolor{currentstroke}%
\pgfsetdash{}{0pt}%
\pgfsys@defobject{currentmarker}{\pgfqpoint{0.000000in}{-0.027778in}}{\pgfqpoint{0.000000in}{0.000000in}}{%
\pgfpathmoveto{\pgfqpoint{0.000000in}{0.000000in}}%
\pgfpathlineto{\pgfqpoint{0.000000in}{-0.027778in}}%
\pgfusepath{stroke,fill}%
}%
\begin{pgfscope}%
\pgfsys@transformshift{1.020945in}{0.540713in}%
\pgfsys@useobject{currentmarker}{}%
\end{pgfscope}%
\end{pgfscope}%
\begin{pgfscope}%
\pgfsetbuttcap%
\pgfsetroundjoin%
\definecolor{currentfill}{rgb}{0.000000,0.000000,0.000000}%
\pgfsetfillcolor{currentfill}%
\pgfsetlinewidth{0.602250pt}%
\definecolor{currentstroke}{rgb}{0.000000,0.000000,0.000000}%
\pgfsetstrokecolor{currentstroke}%
\pgfsetdash{}{0pt}%
\pgfsys@defobject{currentmarker}{\pgfqpoint{0.000000in}{-0.027778in}}{\pgfqpoint{0.000000in}{0.000000in}}{%
\pgfpathmoveto{\pgfqpoint{0.000000in}{0.000000in}}%
\pgfpathlineto{\pgfqpoint{0.000000in}{-0.027778in}}%
\pgfusepath{stroke,fill}%
}%
\begin{pgfscope}%
\pgfsys@transformshift{1.145996in}{0.540713in}%
\pgfsys@useobject{currentmarker}{}%
\end{pgfscope}%
\end{pgfscope}%
\begin{pgfscope}%
\pgfsetbuttcap%
\pgfsetroundjoin%
\definecolor{currentfill}{rgb}{0.000000,0.000000,0.000000}%
\pgfsetfillcolor{currentfill}%
\pgfsetlinewidth{0.602250pt}%
\definecolor{currentstroke}{rgb}{0.000000,0.000000,0.000000}%
\pgfsetstrokecolor{currentstroke}%
\pgfsetdash{}{0pt}%
\pgfsys@defobject{currentmarker}{\pgfqpoint{0.000000in}{-0.027778in}}{\pgfqpoint{0.000000in}{0.000000in}}{%
\pgfpathmoveto{\pgfqpoint{0.000000in}{0.000000in}}%
\pgfpathlineto{\pgfqpoint{0.000000in}{-0.027778in}}%
\pgfusepath{stroke,fill}%
}%
\begin{pgfscope}%
\pgfsys@transformshift{1.234721in}{0.540713in}%
\pgfsys@useobject{currentmarker}{}%
\end{pgfscope}%
\end{pgfscope}%
\begin{pgfscope}%
\pgfsetbuttcap%
\pgfsetroundjoin%
\definecolor{currentfill}{rgb}{0.000000,0.000000,0.000000}%
\pgfsetfillcolor{currentfill}%
\pgfsetlinewidth{0.602250pt}%
\definecolor{currentstroke}{rgb}{0.000000,0.000000,0.000000}%
\pgfsetstrokecolor{currentstroke}%
\pgfsetdash{}{0pt}%
\pgfsys@defobject{currentmarker}{\pgfqpoint{0.000000in}{-0.027778in}}{\pgfqpoint{0.000000in}{0.000000in}}{%
\pgfpathmoveto{\pgfqpoint{0.000000in}{0.000000in}}%
\pgfpathlineto{\pgfqpoint{0.000000in}{-0.027778in}}%
\pgfusepath{stroke,fill}%
}%
\begin{pgfscope}%
\pgfsys@transformshift{1.303541in}{0.540713in}%
\pgfsys@useobject{currentmarker}{}%
\end{pgfscope}%
\end{pgfscope}%
\begin{pgfscope}%
\pgfsetbuttcap%
\pgfsetroundjoin%
\definecolor{currentfill}{rgb}{0.000000,0.000000,0.000000}%
\pgfsetfillcolor{currentfill}%
\pgfsetlinewidth{0.602250pt}%
\definecolor{currentstroke}{rgb}{0.000000,0.000000,0.000000}%
\pgfsetstrokecolor{currentstroke}%
\pgfsetdash{}{0pt}%
\pgfsys@defobject{currentmarker}{\pgfqpoint{0.000000in}{-0.027778in}}{\pgfqpoint{0.000000in}{0.000000in}}{%
\pgfpathmoveto{\pgfqpoint{0.000000in}{0.000000in}}%
\pgfpathlineto{\pgfqpoint{0.000000in}{-0.027778in}}%
\pgfusepath{stroke,fill}%
}%
\begin{pgfscope}%
\pgfsys@transformshift{1.359772in}{0.540713in}%
\pgfsys@useobject{currentmarker}{}%
\end{pgfscope}%
\end{pgfscope}%
\begin{pgfscope}%
\pgfsetbuttcap%
\pgfsetroundjoin%
\definecolor{currentfill}{rgb}{0.000000,0.000000,0.000000}%
\pgfsetfillcolor{currentfill}%
\pgfsetlinewidth{0.602250pt}%
\definecolor{currentstroke}{rgb}{0.000000,0.000000,0.000000}%
\pgfsetstrokecolor{currentstroke}%
\pgfsetdash{}{0pt}%
\pgfsys@defobject{currentmarker}{\pgfqpoint{0.000000in}{-0.027778in}}{\pgfqpoint{0.000000in}{0.000000in}}{%
\pgfpathmoveto{\pgfqpoint{0.000000in}{0.000000in}}%
\pgfpathlineto{\pgfqpoint{0.000000in}{-0.027778in}}%
\pgfusepath{stroke,fill}%
}%
\begin{pgfscope}%
\pgfsys@transformshift{1.407314in}{0.540713in}%
\pgfsys@useobject{currentmarker}{}%
\end{pgfscope}%
\end{pgfscope}%
\begin{pgfscope}%
\pgfsetbuttcap%
\pgfsetroundjoin%
\definecolor{currentfill}{rgb}{0.000000,0.000000,0.000000}%
\pgfsetfillcolor{currentfill}%
\pgfsetlinewidth{0.602250pt}%
\definecolor{currentstroke}{rgb}{0.000000,0.000000,0.000000}%
\pgfsetstrokecolor{currentstroke}%
\pgfsetdash{}{0pt}%
\pgfsys@defobject{currentmarker}{\pgfqpoint{0.000000in}{-0.027778in}}{\pgfqpoint{0.000000in}{0.000000in}}{%
\pgfpathmoveto{\pgfqpoint{0.000000in}{0.000000in}}%
\pgfpathlineto{\pgfqpoint{0.000000in}{-0.027778in}}%
\pgfusepath{stroke,fill}%
}%
\begin{pgfscope}%
\pgfsys@transformshift{1.448497in}{0.540713in}%
\pgfsys@useobject{currentmarker}{}%
\end{pgfscope}%
\end{pgfscope}%
\begin{pgfscope}%
\pgfsetbuttcap%
\pgfsetroundjoin%
\definecolor{currentfill}{rgb}{0.000000,0.000000,0.000000}%
\pgfsetfillcolor{currentfill}%
\pgfsetlinewidth{0.602250pt}%
\definecolor{currentstroke}{rgb}{0.000000,0.000000,0.000000}%
\pgfsetstrokecolor{currentstroke}%
\pgfsetdash{}{0pt}%
\pgfsys@defobject{currentmarker}{\pgfqpoint{0.000000in}{-0.027778in}}{\pgfqpoint{0.000000in}{0.000000in}}{%
\pgfpathmoveto{\pgfqpoint{0.000000in}{0.000000in}}%
\pgfpathlineto{\pgfqpoint{0.000000in}{-0.027778in}}%
\pgfusepath{stroke,fill}%
}%
\begin{pgfscope}%
\pgfsys@transformshift{1.484822in}{0.540713in}%
\pgfsys@useobject{currentmarker}{}%
\end{pgfscope}%
\end{pgfscope}%
\begin{pgfscope}%
\pgfsetbuttcap%
\pgfsetroundjoin%
\definecolor{currentfill}{rgb}{0.000000,0.000000,0.000000}%
\pgfsetfillcolor{currentfill}%
\pgfsetlinewidth{0.602250pt}%
\definecolor{currentstroke}{rgb}{0.000000,0.000000,0.000000}%
\pgfsetstrokecolor{currentstroke}%
\pgfsetdash{}{0pt}%
\pgfsys@defobject{currentmarker}{\pgfqpoint{0.000000in}{-0.027778in}}{\pgfqpoint{0.000000in}{0.000000in}}{%
\pgfpathmoveto{\pgfqpoint{0.000000in}{0.000000in}}%
\pgfpathlineto{\pgfqpoint{0.000000in}{-0.027778in}}%
\pgfusepath{stroke,fill}%
}%
\begin{pgfscope}%
\pgfsys@transformshift{1.731093in}{0.540713in}%
\pgfsys@useobject{currentmarker}{}%
\end{pgfscope}%
\end{pgfscope}%
\begin{pgfscope}%
\pgfsetbuttcap%
\pgfsetroundjoin%
\definecolor{currentfill}{rgb}{0.000000,0.000000,0.000000}%
\pgfsetfillcolor{currentfill}%
\pgfsetlinewidth{0.602250pt}%
\definecolor{currentstroke}{rgb}{0.000000,0.000000,0.000000}%
\pgfsetstrokecolor{currentstroke}%
\pgfsetdash{}{0pt}%
\pgfsys@defobject{currentmarker}{\pgfqpoint{0.000000in}{-0.027778in}}{\pgfqpoint{0.000000in}{0.000000in}}{%
\pgfpathmoveto{\pgfqpoint{0.000000in}{0.000000in}}%
\pgfpathlineto{\pgfqpoint{0.000000in}{-0.027778in}}%
\pgfusepath{stroke,fill}%
}%
\begin{pgfscope}%
\pgfsys@transformshift{1.856144in}{0.540713in}%
\pgfsys@useobject{currentmarker}{}%
\end{pgfscope}%
\end{pgfscope}%
\begin{pgfscope}%
\pgfsetbuttcap%
\pgfsetroundjoin%
\definecolor{currentfill}{rgb}{0.000000,0.000000,0.000000}%
\pgfsetfillcolor{currentfill}%
\pgfsetlinewidth{0.602250pt}%
\definecolor{currentstroke}{rgb}{0.000000,0.000000,0.000000}%
\pgfsetstrokecolor{currentstroke}%
\pgfsetdash{}{0pt}%
\pgfsys@defobject{currentmarker}{\pgfqpoint{0.000000in}{-0.027778in}}{\pgfqpoint{0.000000in}{0.000000in}}{%
\pgfpathmoveto{\pgfqpoint{0.000000in}{0.000000in}}%
\pgfpathlineto{\pgfqpoint{0.000000in}{-0.027778in}}%
\pgfusepath{stroke,fill}%
}%
\begin{pgfscope}%
\pgfsys@transformshift{1.944869in}{0.540713in}%
\pgfsys@useobject{currentmarker}{}%
\end{pgfscope}%
\end{pgfscope}%
\begin{pgfscope}%
\pgfsetbuttcap%
\pgfsetroundjoin%
\definecolor{currentfill}{rgb}{0.000000,0.000000,0.000000}%
\pgfsetfillcolor{currentfill}%
\pgfsetlinewidth{0.602250pt}%
\definecolor{currentstroke}{rgb}{0.000000,0.000000,0.000000}%
\pgfsetstrokecolor{currentstroke}%
\pgfsetdash{}{0pt}%
\pgfsys@defobject{currentmarker}{\pgfqpoint{0.000000in}{-0.027778in}}{\pgfqpoint{0.000000in}{0.000000in}}{%
\pgfpathmoveto{\pgfqpoint{0.000000in}{0.000000in}}%
\pgfpathlineto{\pgfqpoint{0.000000in}{-0.027778in}}%
\pgfusepath{stroke,fill}%
}%
\begin{pgfscope}%
\pgfsys@transformshift{2.013689in}{0.540713in}%
\pgfsys@useobject{currentmarker}{}%
\end{pgfscope}%
\end{pgfscope}%
\begin{pgfscope}%
\pgfsetbuttcap%
\pgfsetroundjoin%
\definecolor{currentfill}{rgb}{0.000000,0.000000,0.000000}%
\pgfsetfillcolor{currentfill}%
\pgfsetlinewidth{0.602250pt}%
\definecolor{currentstroke}{rgb}{0.000000,0.000000,0.000000}%
\pgfsetstrokecolor{currentstroke}%
\pgfsetdash{}{0pt}%
\pgfsys@defobject{currentmarker}{\pgfqpoint{0.000000in}{-0.027778in}}{\pgfqpoint{0.000000in}{0.000000in}}{%
\pgfpathmoveto{\pgfqpoint{0.000000in}{0.000000in}}%
\pgfpathlineto{\pgfqpoint{0.000000in}{-0.027778in}}%
\pgfusepath{stroke,fill}%
}%
\begin{pgfscope}%
\pgfsys@transformshift{2.069920in}{0.540713in}%
\pgfsys@useobject{currentmarker}{}%
\end{pgfscope}%
\end{pgfscope}%
\begin{pgfscope}%
\pgfsetbuttcap%
\pgfsetroundjoin%
\definecolor{currentfill}{rgb}{0.000000,0.000000,0.000000}%
\pgfsetfillcolor{currentfill}%
\pgfsetlinewidth{0.602250pt}%
\definecolor{currentstroke}{rgb}{0.000000,0.000000,0.000000}%
\pgfsetstrokecolor{currentstroke}%
\pgfsetdash{}{0pt}%
\pgfsys@defobject{currentmarker}{\pgfqpoint{0.000000in}{-0.027778in}}{\pgfqpoint{0.000000in}{0.000000in}}{%
\pgfpathmoveto{\pgfqpoint{0.000000in}{0.000000in}}%
\pgfpathlineto{\pgfqpoint{0.000000in}{-0.027778in}}%
\pgfusepath{stroke,fill}%
}%
\begin{pgfscope}%
\pgfsys@transformshift{2.117462in}{0.540713in}%
\pgfsys@useobject{currentmarker}{}%
\end{pgfscope}%
\end{pgfscope}%
\begin{pgfscope}%
\pgfsetbuttcap%
\pgfsetroundjoin%
\definecolor{currentfill}{rgb}{0.000000,0.000000,0.000000}%
\pgfsetfillcolor{currentfill}%
\pgfsetlinewidth{0.602250pt}%
\definecolor{currentstroke}{rgb}{0.000000,0.000000,0.000000}%
\pgfsetstrokecolor{currentstroke}%
\pgfsetdash{}{0pt}%
\pgfsys@defobject{currentmarker}{\pgfqpoint{0.000000in}{-0.027778in}}{\pgfqpoint{0.000000in}{0.000000in}}{%
\pgfpathmoveto{\pgfqpoint{0.000000in}{0.000000in}}%
\pgfpathlineto{\pgfqpoint{0.000000in}{-0.027778in}}%
\pgfusepath{stroke,fill}%
}%
\begin{pgfscope}%
\pgfsys@transformshift{2.158645in}{0.540713in}%
\pgfsys@useobject{currentmarker}{}%
\end{pgfscope}%
\end{pgfscope}%
\begin{pgfscope}%
\pgfsetbuttcap%
\pgfsetroundjoin%
\definecolor{currentfill}{rgb}{0.000000,0.000000,0.000000}%
\pgfsetfillcolor{currentfill}%
\pgfsetlinewidth{0.602250pt}%
\definecolor{currentstroke}{rgb}{0.000000,0.000000,0.000000}%
\pgfsetstrokecolor{currentstroke}%
\pgfsetdash{}{0pt}%
\pgfsys@defobject{currentmarker}{\pgfqpoint{0.000000in}{-0.027778in}}{\pgfqpoint{0.000000in}{0.000000in}}{%
\pgfpathmoveto{\pgfqpoint{0.000000in}{0.000000in}}%
\pgfpathlineto{\pgfqpoint{0.000000in}{-0.027778in}}%
\pgfusepath{stroke,fill}%
}%
\begin{pgfscope}%
\pgfsys@transformshift{2.194971in}{0.540713in}%
\pgfsys@useobject{currentmarker}{}%
\end{pgfscope}%
\end{pgfscope}%
\begin{pgfscope}%
\pgfsetbuttcap%
\pgfsetroundjoin%
\definecolor{currentfill}{rgb}{0.000000,0.000000,0.000000}%
\pgfsetfillcolor{currentfill}%
\pgfsetlinewidth{0.602250pt}%
\definecolor{currentstroke}{rgb}{0.000000,0.000000,0.000000}%
\pgfsetstrokecolor{currentstroke}%
\pgfsetdash{}{0pt}%
\pgfsys@defobject{currentmarker}{\pgfqpoint{0.000000in}{-0.027778in}}{\pgfqpoint{0.000000in}{0.000000in}}{%
\pgfpathmoveto{\pgfqpoint{0.000000in}{0.000000in}}%
\pgfpathlineto{\pgfqpoint{0.000000in}{-0.027778in}}%
\pgfusepath{stroke,fill}%
}%
\begin{pgfscope}%
\pgfsys@transformshift{2.441241in}{0.540713in}%
\pgfsys@useobject{currentmarker}{}%
\end{pgfscope}%
\end{pgfscope}%
\begin{pgfscope}%
\pgfsetbuttcap%
\pgfsetroundjoin%
\definecolor{currentfill}{rgb}{0.000000,0.000000,0.000000}%
\pgfsetfillcolor{currentfill}%
\pgfsetlinewidth{0.602250pt}%
\definecolor{currentstroke}{rgb}{0.000000,0.000000,0.000000}%
\pgfsetstrokecolor{currentstroke}%
\pgfsetdash{}{0pt}%
\pgfsys@defobject{currentmarker}{\pgfqpoint{0.000000in}{-0.027778in}}{\pgfqpoint{0.000000in}{0.000000in}}{%
\pgfpathmoveto{\pgfqpoint{0.000000in}{0.000000in}}%
\pgfpathlineto{\pgfqpoint{0.000000in}{-0.027778in}}%
\pgfusepath{stroke,fill}%
}%
\begin{pgfscope}%
\pgfsys@transformshift{2.566292in}{0.540713in}%
\pgfsys@useobject{currentmarker}{}%
\end{pgfscope}%
\end{pgfscope}%
\begin{pgfscope}%
\pgfsetbuttcap%
\pgfsetroundjoin%
\definecolor{currentfill}{rgb}{0.000000,0.000000,0.000000}%
\pgfsetfillcolor{currentfill}%
\pgfsetlinewidth{0.602250pt}%
\definecolor{currentstroke}{rgb}{0.000000,0.000000,0.000000}%
\pgfsetstrokecolor{currentstroke}%
\pgfsetdash{}{0pt}%
\pgfsys@defobject{currentmarker}{\pgfqpoint{0.000000in}{-0.027778in}}{\pgfqpoint{0.000000in}{0.000000in}}{%
\pgfpathmoveto{\pgfqpoint{0.000000in}{0.000000in}}%
\pgfpathlineto{\pgfqpoint{0.000000in}{-0.027778in}}%
\pgfusepath{stroke,fill}%
}%
\begin{pgfscope}%
\pgfsys@transformshift{2.655017in}{0.540713in}%
\pgfsys@useobject{currentmarker}{}%
\end{pgfscope}%
\end{pgfscope}%
\begin{pgfscope}%
\pgfsetbuttcap%
\pgfsetroundjoin%
\definecolor{currentfill}{rgb}{0.000000,0.000000,0.000000}%
\pgfsetfillcolor{currentfill}%
\pgfsetlinewidth{0.602250pt}%
\definecolor{currentstroke}{rgb}{0.000000,0.000000,0.000000}%
\pgfsetstrokecolor{currentstroke}%
\pgfsetdash{}{0pt}%
\pgfsys@defobject{currentmarker}{\pgfqpoint{0.000000in}{-0.027778in}}{\pgfqpoint{0.000000in}{0.000000in}}{%
\pgfpathmoveto{\pgfqpoint{0.000000in}{0.000000in}}%
\pgfpathlineto{\pgfqpoint{0.000000in}{-0.027778in}}%
\pgfusepath{stroke,fill}%
}%
\begin{pgfscope}%
\pgfsys@transformshift{2.723838in}{0.540713in}%
\pgfsys@useobject{currentmarker}{}%
\end{pgfscope}%
\end{pgfscope}%
\begin{pgfscope}%
\pgfsetbuttcap%
\pgfsetroundjoin%
\definecolor{currentfill}{rgb}{0.000000,0.000000,0.000000}%
\pgfsetfillcolor{currentfill}%
\pgfsetlinewidth{0.602250pt}%
\definecolor{currentstroke}{rgb}{0.000000,0.000000,0.000000}%
\pgfsetstrokecolor{currentstroke}%
\pgfsetdash{}{0pt}%
\pgfsys@defobject{currentmarker}{\pgfqpoint{0.000000in}{-0.027778in}}{\pgfqpoint{0.000000in}{0.000000in}}{%
\pgfpathmoveto{\pgfqpoint{0.000000in}{0.000000in}}%
\pgfpathlineto{\pgfqpoint{0.000000in}{-0.027778in}}%
\pgfusepath{stroke,fill}%
}%
\begin{pgfscope}%
\pgfsys@transformshift{2.780068in}{0.540713in}%
\pgfsys@useobject{currentmarker}{}%
\end{pgfscope}%
\end{pgfscope}%
\begin{pgfscope}%
\pgfsetbuttcap%
\pgfsetroundjoin%
\definecolor{currentfill}{rgb}{0.000000,0.000000,0.000000}%
\pgfsetfillcolor{currentfill}%
\pgfsetlinewidth{0.602250pt}%
\definecolor{currentstroke}{rgb}{0.000000,0.000000,0.000000}%
\pgfsetstrokecolor{currentstroke}%
\pgfsetdash{}{0pt}%
\pgfsys@defobject{currentmarker}{\pgfqpoint{0.000000in}{-0.027778in}}{\pgfqpoint{0.000000in}{0.000000in}}{%
\pgfpathmoveto{\pgfqpoint{0.000000in}{0.000000in}}%
\pgfpathlineto{\pgfqpoint{0.000000in}{-0.027778in}}%
\pgfusepath{stroke,fill}%
}%
\begin{pgfscope}%
\pgfsys@transformshift{2.827610in}{0.540713in}%
\pgfsys@useobject{currentmarker}{}%
\end{pgfscope}%
\end{pgfscope}%
\begin{pgfscope}%
\pgfsetbuttcap%
\pgfsetroundjoin%
\definecolor{currentfill}{rgb}{0.000000,0.000000,0.000000}%
\pgfsetfillcolor{currentfill}%
\pgfsetlinewidth{0.602250pt}%
\definecolor{currentstroke}{rgb}{0.000000,0.000000,0.000000}%
\pgfsetstrokecolor{currentstroke}%
\pgfsetdash{}{0pt}%
\pgfsys@defobject{currentmarker}{\pgfqpoint{0.000000in}{-0.027778in}}{\pgfqpoint{0.000000in}{0.000000in}}{%
\pgfpathmoveto{\pgfqpoint{0.000000in}{0.000000in}}%
\pgfpathlineto{\pgfqpoint{0.000000in}{-0.027778in}}%
\pgfusepath{stroke,fill}%
}%
\begin{pgfscope}%
\pgfsys@transformshift{2.868793in}{0.540713in}%
\pgfsys@useobject{currentmarker}{}%
\end{pgfscope}%
\end{pgfscope}%
\begin{pgfscope}%
\pgfsetbuttcap%
\pgfsetroundjoin%
\definecolor{currentfill}{rgb}{0.000000,0.000000,0.000000}%
\pgfsetfillcolor{currentfill}%
\pgfsetlinewidth{0.602250pt}%
\definecolor{currentstroke}{rgb}{0.000000,0.000000,0.000000}%
\pgfsetstrokecolor{currentstroke}%
\pgfsetdash{}{0pt}%
\pgfsys@defobject{currentmarker}{\pgfqpoint{0.000000in}{-0.027778in}}{\pgfqpoint{0.000000in}{0.000000in}}{%
\pgfpathmoveto{\pgfqpoint{0.000000in}{0.000000in}}%
\pgfpathlineto{\pgfqpoint{0.000000in}{-0.027778in}}%
\pgfusepath{stroke,fill}%
}%
\begin{pgfscope}%
\pgfsys@transformshift{2.905119in}{0.540713in}%
\pgfsys@useobject{currentmarker}{}%
\end{pgfscope}%
\end{pgfscope}%
\begin{pgfscope}%
\pgfsetbuttcap%
\pgfsetroundjoin%
\definecolor{currentfill}{rgb}{0.000000,0.000000,0.000000}%
\pgfsetfillcolor{currentfill}%
\pgfsetlinewidth{0.602250pt}%
\definecolor{currentstroke}{rgb}{0.000000,0.000000,0.000000}%
\pgfsetstrokecolor{currentstroke}%
\pgfsetdash{}{0pt}%
\pgfsys@defobject{currentmarker}{\pgfqpoint{0.000000in}{-0.027778in}}{\pgfqpoint{0.000000in}{0.000000in}}{%
\pgfpathmoveto{\pgfqpoint{0.000000in}{0.000000in}}%
\pgfpathlineto{\pgfqpoint{0.000000in}{-0.027778in}}%
\pgfusepath{stroke,fill}%
}%
\begin{pgfscope}%
\pgfsys@transformshift{3.151389in}{0.540713in}%
\pgfsys@useobject{currentmarker}{}%
\end{pgfscope}%
\end{pgfscope}%
\begin{pgfscope}%
\pgfsetbuttcap%
\pgfsetroundjoin%
\definecolor{currentfill}{rgb}{0.000000,0.000000,0.000000}%
\pgfsetfillcolor{currentfill}%
\pgfsetlinewidth{0.602250pt}%
\definecolor{currentstroke}{rgb}{0.000000,0.000000,0.000000}%
\pgfsetstrokecolor{currentstroke}%
\pgfsetdash{}{0pt}%
\pgfsys@defobject{currentmarker}{\pgfqpoint{0.000000in}{-0.027778in}}{\pgfqpoint{0.000000in}{0.000000in}}{%
\pgfpathmoveto{\pgfqpoint{0.000000in}{0.000000in}}%
\pgfpathlineto{\pgfqpoint{0.000000in}{-0.027778in}}%
\pgfusepath{stroke,fill}%
}%
\begin{pgfscope}%
\pgfsys@transformshift{3.276440in}{0.540713in}%
\pgfsys@useobject{currentmarker}{}%
\end{pgfscope}%
\end{pgfscope}%
\begin{pgfscope}%
\pgfsetbuttcap%
\pgfsetroundjoin%
\definecolor{currentfill}{rgb}{0.000000,0.000000,0.000000}%
\pgfsetfillcolor{currentfill}%
\pgfsetlinewidth{0.602250pt}%
\definecolor{currentstroke}{rgb}{0.000000,0.000000,0.000000}%
\pgfsetstrokecolor{currentstroke}%
\pgfsetdash{}{0pt}%
\pgfsys@defobject{currentmarker}{\pgfqpoint{0.000000in}{-0.027778in}}{\pgfqpoint{0.000000in}{0.000000in}}{%
\pgfpathmoveto{\pgfqpoint{0.000000in}{0.000000in}}%
\pgfpathlineto{\pgfqpoint{0.000000in}{-0.027778in}}%
\pgfusepath{stroke,fill}%
}%
\begin{pgfscope}%
\pgfsys@transformshift{3.365165in}{0.540713in}%
\pgfsys@useobject{currentmarker}{}%
\end{pgfscope}%
\end{pgfscope}%
\begin{pgfscope}%
\pgfsetbuttcap%
\pgfsetroundjoin%
\definecolor{currentfill}{rgb}{0.000000,0.000000,0.000000}%
\pgfsetfillcolor{currentfill}%
\pgfsetlinewidth{0.602250pt}%
\definecolor{currentstroke}{rgb}{0.000000,0.000000,0.000000}%
\pgfsetstrokecolor{currentstroke}%
\pgfsetdash{}{0pt}%
\pgfsys@defobject{currentmarker}{\pgfqpoint{0.000000in}{-0.027778in}}{\pgfqpoint{0.000000in}{0.000000in}}{%
\pgfpathmoveto{\pgfqpoint{0.000000in}{0.000000in}}%
\pgfpathlineto{\pgfqpoint{0.000000in}{-0.027778in}}%
\pgfusepath{stroke,fill}%
}%
\begin{pgfscope}%
\pgfsys@transformshift{3.433986in}{0.540713in}%
\pgfsys@useobject{currentmarker}{}%
\end{pgfscope}%
\end{pgfscope}%
\begin{pgfscope}%
\pgfsetbuttcap%
\pgfsetroundjoin%
\definecolor{currentfill}{rgb}{0.000000,0.000000,0.000000}%
\pgfsetfillcolor{currentfill}%
\pgfsetlinewidth{0.602250pt}%
\definecolor{currentstroke}{rgb}{0.000000,0.000000,0.000000}%
\pgfsetstrokecolor{currentstroke}%
\pgfsetdash{}{0pt}%
\pgfsys@defobject{currentmarker}{\pgfqpoint{0.000000in}{-0.027778in}}{\pgfqpoint{0.000000in}{0.000000in}}{%
\pgfpathmoveto{\pgfqpoint{0.000000in}{0.000000in}}%
\pgfpathlineto{\pgfqpoint{0.000000in}{-0.027778in}}%
\pgfusepath{stroke,fill}%
}%
\begin{pgfscope}%
\pgfsys@transformshift{3.490216in}{0.540713in}%
\pgfsys@useobject{currentmarker}{}%
\end{pgfscope}%
\end{pgfscope}%
\begin{pgfscope}%
\pgfsetbuttcap%
\pgfsetroundjoin%
\definecolor{currentfill}{rgb}{0.000000,0.000000,0.000000}%
\pgfsetfillcolor{currentfill}%
\pgfsetlinewidth{0.602250pt}%
\definecolor{currentstroke}{rgb}{0.000000,0.000000,0.000000}%
\pgfsetstrokecolor{currentstroke}%
\pgfsetdash{}{0pt}%
\pgfsys@defobject{currentmarker}{\pgfqpoint{0.000000in}{-0.027778in}}{\pgfqpoint{0.000000in}{0.000000in}}{%
\pgfpathmoveto{\pgfqpoint{0.000000in}{0.000000in}}%
\pgfpathlineto{\pgfqpoint{0.000000in}{-0.027778in}}%
\pgfusepath{stroke,fill}%
}%
\begin{pgfscope}%
\pgfsys@transformshift{3.537758in}{0.540713in}%
\pgfsys@useobject{currentmarker}{}%
\end{pgfscope}%
\end{pgfscope}%
\begin{pgfscope}%
\pgfsetbuttcap%
\pgfsetroundjoin%
\definecolor{currentfill}{rgb}{0.000000,0.000000,0.000000}%
\pgfsetfillcolor{currentfill}%
\pgfsetlinewidth{0.602250pt}%
\definecolor{currentstroke}{rgb}{0.000000,0.000000,0.000000}%
\pgfsetstrokecolor{currentstroke}%
\pgfsetdash{}{0pt}%
\pgfsys@defobject{currentmarker}{\pgfqpoint{0.000000in}{-0.027778in}}{\pgfqpoint{0.000000in}{0.000000in}}{%
\pgfpathmoveto{\pgfqpoint{0.000000in}{0.000000in}}%
\pgfpathlineto{\pgfqpoint{0.000000in}{-0.027778in}}%
\pgfusepath{stroke,fill}%
}%
\begin{pgfscope}%
\pgfsys@transformshift{3.578941in}{0.540713in}%
\pgfsys@useobject{currentmarker}{}%
\end{pgfscope}%
\end{pgfscope}%
\begin{pgfscope}%
\pgfsetbuttcap%
\pgfsetroundjoin%
\definecolor{currentfill}{rgb}{0.000000,0.000000,0.000000}%
\pgfsetfillcolor{currentfill}%
\pgfsetlinewidth{0.602250pt}%
\definecolor{currentstroke}{rgb}{0.000000,0.000000,0.000000}%
\pgfsetstrokecolor{currentstroke}%
\pgfsetdash{}{0pt}%
\pgfsys@defobject{currentmarker}{\pgfqpoint{0.000000in}{-0.027778in}}{\pgfqpoint{0.000000in}{0.000000in}}{%
\pgfpathmoveto{\pgfqpoint{0.000000in}{0.000000in}}%
\pgfpathlineto{\pgfqpoint{0.000000in}{-0.027778in}}%
\pgfusepath{stroke,fill}%
}%
\begin{pgfscope}%
\pgfsys@transformshift{3.615267in}{0.540713in}%
\pgfsys@useobject{currentmarker}{}%
\end{pgfscope}%
\end{pgfscope}%
\begin{pgfscope}%
\pgfsetbuttcap%
\pgfsetroundjoin%
\definecolor{currentfill}{rgb}{0.000000,0.000000,0.000000}%
\pgfsetfillcolor{currentfill}%
\pgfsetlinewidth{0.602250pt}%
\definecolor{currentstroke}{rgb}{0.000000,0.000000,0.000000}%
\pgfsetstrokecolor{currentstroke}%
\pgfsetdash{}{0pt}%
\pgfsys@defobject{currentmarker}{\pgfqpoint{0.000000in}{-0.027778in}}{\pgfqpoint{0.000000in}{0.000000in}}{%
\pgfpathmoveto{\pgfqpoint{0.000000in}{0.000000in}}%
\pgfpathlineto{\pgfqpoint{0.000000in}{-0.027778in}}%
\pgfusepath{stroke,fill}%
}%
\begin{pgfscope}%
\pgfsys@transformshift{3.861538in}{0.540713in}%
\pgfsys@useobject{currentmarker}{}%
\end{pgfscope}%
\end{pgfscope}%
\begin{pgfscope}%
\pgfsetbuttcap%
\pgfsetroundjoin%
\definecolor{currentfill}{rgb}{0.000000,0.000000,0.000000}%
\pgfsetfillcolor{currentfill}%
\pgfsetlinewidth{0.602250pt}%
\definecolor{currentstroke}{rgb}{0.000000,0.000000,0.000000}%
\pgfsetstrokecolor{currentstroke}%
\pgfsetdash{}{0pt}%
\pgfsys@defobject{currentmarker}{\pgfqpoint{0.000000in}{-0.027778in}}{\pgfqpoint{0.000000in}{0.000000in}}{%
\pgfpathmoveto{\pgfqpoint{0.000000in}{0.000000in}}%
\pgfpathlineto{\pgfqpoint{0.000000in}{-0.027778in}}%
\pgfusepath{stroke,fill}%
}%
\begin{pgfscope}%
\pgfsys@transformshift{3.986588in}{0.540713in}%
\pgfsys@useobject{currentmarker}{}%
\end{pgfscope}%
\end{pgfscope}%
\begin{pgfscope}%
\pgfsetbuttcap%
\pgfsetroundjoin%
\definecolor{currentfill}{rgb}{0.000000,0.000000,0.000000}%
\pgfsetfillcolor{currentfill}%
\pgfsetlinewidth{0.602250pt}%
\definecolor{currentstroke}{rgb}{0.000000,0.000000,0.000000}%
\pgfsetstrokecolor{currentstroke}%
\pgfsetdash{}{0pt}%
\pgfsys@defobject{currentmarker}{\pgfqpoint{0.000000in}{-0.027778in}}{\pgfqpoint{0.000000in}{0.000000in}}{%
\pgfpathmoveto{\pgfqpoint{0.000000in}{0.000000in}}%
\pgfpathlineto{\pgfqpoint{0.000000in}{-0.027778in}}%
\pgfusepath{stroke,fill}%
}%
\begin{pgfscope}%
\pgfsys@transformshift{4.075313in}{0.540713in}%
\pgfsys@useobject{currentmarker}{}%
\end{pgfscope}%
\end{pgfscope}%
\begin{pgfscope}%
\pgfsetbuttcap%
\pgfsetroundjoin%
\definecolor{currentfill}{rgb}{0.000000,0.000000,0.000000}%
\pgfsetfillcolor{currentfill}%
\pgfsetlinewidth{0.602250pt}%
\definecolor{currentstroke}{rgb}{0.000000,0.000000,0.000000}%
\pgfsetstrokecolor{currentstroke}%
\pgfsetdash{}{0pt}%
\pgfsys@defobject{currentmarker}{\pgfqpoint{0.000000in}{-0.027778in}}{\pgfqpoint{0.000000in}{0.000000in}}{%
\pgfpathmoveto{\pgfqpoint{0.000000in}{0.000000in}}%
\pgfpathlineto{\pgfqpoint{0.000000in}{-0.027778in}}%
\pgfusepath{stroke,fill}%
}%
\begin{pgfscope}%
\pgfsys@transformshift{4.144134in}{0.540713in}%
\pgfsys@useobject{currentmarker}{}%
\end{pgfscope}%
\end{pgfscope}%
\begin{pgfscope}%
\pgfsetbuttcap%
\pgfsetroundjoin%
\definecolor{currentfill}{rgb}{0.000000,0.000000,0.000000}%
\pgfsetfillcolor{currentfill}%
\pgfsetlinewidth{0.602250pt}%
\definecolor{currentstroke}{rgb}{0.000000,0.000000,0.000000}%
\pgfsetstrokecolor{currentstroke}%
\pgfsetdash{}{0pt}%
\pgfsys@defobject{currentmarker}{\pgfqpoint{0.000000in}{-0.027778in}}{\pgfqpoint{0.000000in}{0.000000in}}{%
\pgfpathmoveto{\pgfqpoint{0.000000in}{0.000000in}}%
\pgfpathlineto{\pgfqpoint{0.000000in}{-0.027778in}}%
\pgfusepath{stroke,fill}%
}%
\begin{pgfscope}%
\pgfsys@transformshift{4.200364in}{0.540713in}%
\pgfsys@useobject{currentmarker}{}%
\end{pgfscope}%
\end{pgfscope}%
\begin{pgfscope}%
\pgfsetbuttcap%
\pgfsetroundjoin%
\definecolor{currentfill}{rgb}{0.000000,0.000000,0.000000}%
\pgfsetfillcolor{currentfill}%
\pgfsetlinewidth{0.602250pt}%
\definecolor{currentstroke}{rgb}{0.000000,0.000000,0.000000}%
\pgfsetstrokecolor{currentstroke}%
\pgfsetdash{}{0pt}%
\pgfsys@defobject{currentmarker}{\pgfqpoint{0.000000in}{-0.027778in}}{\pgfqpoint{0.000000in}{0.000000in}}{%
\pgfpathmoveto{\pgfqpoint{0.000000in}{0.000000in}}%
\pgfpathlineto{\pgfqpoint{0.000000in}{-0.027778in}}%
\pgfusepath{stroke,fill}%
}%
\begin{pgfscope}%
\pgfsys@transformshift{4.247907in}{0.540713in}%
\pgfsys@useobject{currentmarker}{}%
\end{pgfscope}%
\end{pgfscope}%
\begin{pgfscope}%
\pgfsetbuttcap%
\pgfsetroundjoin%
\definecolor{currentfill}{rgb}{0.000000,0.000000,0.000000}%
\pgfsetfillcolor{currentfill}%
\pgfsetlinewidth{0.602250pt}%
\definecolor{currentstroke}{rgb}{0.000000,0.000000,0.000000}%
\pgfsetstrokecolor{currentstroke}%
\pgfsetdash{}{0pt}%
\pgfsys@defobject{currentmarker}{\pgfqpoint{0.000000in}{-0.027778in}}{\pgfqpoint{0.000000in}{0.000000in}}{%
\pgfpathmoveto{\pgfqpoint{0.000000in}{0.000000in}}%
\pgfpathlineto{\pgfqpoint{0.000000in}{-0.027778in}}%
\pgfusepath{stroke,fill}%
}%
\begin{pgfscope}%
\pgfsys@transformshift{4.289089in}{0.540713in}%
\pgfsys@useobject{currentmarker}{}%
\end{pgfscope}%
\end{pgfscope}%
\begin{pgfscope}%
\pgfsetbuttcap%
\pgfsetroundjoin%
\definecolor{currentfill}{rgb}{0.000000,0.000000,0.000000}%
\pgfsetfillcolor{currentfill}%
\pgfsetlinewidth{0.602250pt}%
\definecolor{currentstroke}{rgb}{0.000000,0.000000,0.000000}%
\pgfsetstrokecolor{currentstroke}%
\pgfsetdash{}{0pt}%
\pgfsys@defobject{currentmarker}{\pgfqpoint{0.000000in}{-0.027778in}}{\pgfqpoint{0.000000in}{0.000000in}}{%
\pgfpathmoveto{\pgfqpoint{0.000000in}{0.000000in}}%
\pgfpathlineto{\pgfqpoint{0.000000in}{-0.027778in}}%
\pgfusepath{stroke,fill}%
}%
\begin{pgfscope}%
\pgfsys@transformshift{4.325415in}{0.540713in}%
\pgfsys@useobject{currentmarker}{}%
\end{pgfscope}%
\end{pgfscope}%
\begin{pgfscope}%
\pgfsetbuttcap%
\pgfsetroundjoin%
\definecolor{currentfill}{rgb}{0.000000,0.000000,0.000000}%
\pgfsetfillcolor{currentfill}%
\pgfsetlinewidth{0.602250pt}%
\definecolor{currentstroke}{rgb}{0.000000,0.000000,0.000000}%
\pgfsetstrokecolor{currentstroke}%
\pgfsetdash{}{0pt}%
\pgfsys@defobject{currentmarker}{\pgfqpoint{0.000000in}{-0.027778in}}{\pgfqpoint{0.000000in}{0.000000in}}{%
\pgfpathmoveto{\pgfqpoint{0.000000in}{0.000000in}}%
\pgfpathlineto{\pgfqpoint{0.000000in}{-0.027778in}}%
\pgfusepath{stroke,fill}%
}%
\begin{pgfscope}%
\pgfsys@transformshift{4.571686in}{0.540713in}%
\pgfsys@useobject{currentmarker}{}%
\end{pgfscope}%
\end{pgfscope}%
\begin{pgfscope}%
\pgfsetbuttcap%
\pgfsetroundjoin%
\definecolor{currentfill}{rgb}{0.000000,0.000000,0.000000}%
\pgfsetfillcolor{currentfill}%
\pgfsetlinewidth{0.602250pt}%
\definecolor{currentstroke}{rgb}{0.000000,0.000000,0.000000}%
\pgfsetstrokecolor{currentstroke}%
\pgfsetdash{}{0pt}%
\pgfsys@defobject{currentmarker}{\pgfqpoint{0.000000in}{-0.027778in}}{\pgfqpoint{0.000000in}{0.000000in}}{%
\pgfpathmoveto{\pgfqpoint{0.000000in}{0.000000in}}%
\pgfpathlineto{\pgfqpoint{0.000000in}{-0.027778in}}%
\pgfusepath{stroke,fill}%
}%
\begin{pgfscope}%
\pgfsys@transformshift{4.696737in}{0.540713in}%
\pgfsys@useobject{currentmarker}{}%
\end{pgfscope}%
\end{pgfscope}%
\begin{pgfscope}%
\pgfsetbuttcap%
\pgfsetroundjoin%
\definecolor{currentfill}{rgb}{0.000000,0.000000,0.000000}%
\pgfsetfillcolor{currentfill}%
\pgfsetlinewidth{0.602250pt}%
\definecolor{currentstroke}{rgb}{0.000000,0.000000,0.000000}%
\pgfsetstrokecolor{currentstroke}%
\pgfsetdash{}{0pt}%
\pgfsys@defobject{currentmarker}{\pgfqpoint{0.000000in}{-0.027778in}}{\pgfqpoint{0.000000in}{0.000000in}}{%
\pgfpathmoveto{\pgfqpoint{0.000000in}{0.000000in}}%
\pgfpathlineto{\pgfqpoint{0.000000in}{-0.027778in}}%
\pgfusepath{stroke,fill}%
}%
\begin{pgfscope}%
\pgfsys@transformshift{4.785462in}{0.540713in}%
\pgfsys@useobject{currentmarker}{}%
\end{pgfscope}%
\end{pgfscope}%
\begin{pgfscope}%
\pgfsetbuttcap%
\pgfsetroundjoin%
\definecolor{currentfill}{rgb}{0.000000,0.000000,0.000000}%
\pgfsetfillcolor{currentfill}%
\pgfsetlinewidth{0.602250pt}%
\definecolor{currentstroke}{rgb}{0.000000,0.000000,0.000000}%
\pgfsetstrokecolor{currentstroke}%
\pgfsetdash{}{0pt}%
\pgfsys@defobject{currentmarker}{\pgfqpoint{0.000000in}{-0.027778in}}{\pgfqpoint{0.000000in}{0.000000in}}{%
\pgfpathmoveto{\pgfqpoint{0.000000in}{0.000000in}}%
\pgfpathlineto{\pgfqpoint{0.000000in}{-0.027778in}}%
\pgfusepath{stroke,fill}%
}%
\begin{pgfscope}%
\pgfsys@transformshift{4.854282in}{0.540713in}%
\pgfsys@useobject{currentmarker}{}%
\end{pgfscope}%
\end{pgfscope}%
\begin{pgfscope}%
\pgfsetbuttcap%
\pgfsetroundjoin%
\definecolor{currentfill}{rgb}{0.000000,0.000000,0.000000}%
\pgfsetfillcolor{currentfill}%
\pgfsetlinewidth{0.602250pt}%
\definecolor{currentstroke}{rgb}{0.000000,0.000000,0.000000}%
\pgfsetstrokecolor{currentstroke}%
\pgfsetdash{}{0pt}%
\pgfsys@defobject{currentmarker}{\pgfqpoint{0.000000in}{-0.027778in}}{\pgfqpoint{0.000000in}{0.000000in}}{%
\pgfpathmoveto{\pgfqpoint{0.000000in}{0.000000in}}%
\pgfpathlineto{\pgfqpoint{0.000000in}{-0.027778in}}%
\pgfusepath{stroke,fill}%
}%
\begin{pgfscope}%
\pgfsys@transformshift{4.910513in}{0.540713in}%
\pgfsys@useobject{currentmarker}{}%
\end{pgfscope}%
\end{pgfscope}%
\begin{pgfscope}%
\pgfsetbuttcap%
\pgfsetroundjoin%
\definecolor{currentfill}{rgb}{0.000000,0.000000,0.000000}%
\pgfsetfillcolor{currentfill}%
\pgfsetlinewidth{0.602250pt}%
\definecolor{currentstroke}{rgb}{0.000000,0.000000,0.000000}%
\pgfsetstrokecolor{currentstroke}%
\pgfsetdash{}{0pt}%
\pgfsys@defobject{currentmarker}{\pgfqpoint{0.000000in}{-0.027778in}}{\pgfqpoint{0.000000in}{0.000000in}}{%
\pgfpathmoveto{\pgfqpoint{0.000000in}{0.000000in}}%
\pgfpathlineto{\pgfqpoint{0.000000in}{-0.027778in}}%
\pgfusepath{stroke,fill}%
}%
\begin{pgfscope}%
\pgfsys@transformshift{4.958055in}{0.540713in}%
\pgfsys@useobject{currentmarker}{}%
\end{pgfscope}%
\end{pgfscope}%
\begin{pgfscope}%
\pgfsetbuttcap%
\pgfsetroundjoin%
\definecolor{currentfill}{rgb}{0.000000,0.000000,0.000000}%
\pgfsetfillcolor{currentfill}%
\pgfsetlinewidth{0.602250pt}%
\definecolor{currentstroke}{rgb}{0.000000,0.000000,0.000000}%
\pgfsetstrokecolor{currentstroke}%
\pgfsetdash{}{0pt}%
\pgfsys@defobject{currentmarker}{\pgfqpoint{0.000000in}{-0.027778in}}{\pgfqpoint{0.000000in}{0.000000in}}{%
\pgfpathmoveto{\pgfqpoint{0.000000in}{0.000000in}}%
\pgfpathlineto{\pgfqpoint{0.000000in}{-0.027778in}}%
\pgfusepath{stroke,fill}%
}%
\begin{pgfscope}%
\pgfsys@transformshift{4.999238in}{0.540713in}%
\pgfsys@useobject{currentmarker}{}%
\end{pgfscope}%
\end{pgfscope}%
\begin{pgfscope}%
\pgfsetbuttcap%
\pgfsetroundjoin%
\definecolor{currentfill}{rgb}{0.000000,0.000000,0.000000}%
\pgfsetfillcolor{currentfill}%
\pgfsetlinewidth{0.602250pt}%
\definecolor{currentstroke}{rgb}{0.000000,0.000000,0.000000}%
\pgfsetstrokecolor{currentstroke}%
\pgfsetdash{}{0pt}%
\pgfsys@defobject{currentmarker}{\pgfqpoint{0.000000in}{-0.027778in}}{\pgfqpoint{0.000000in}{0.000000in}}{%
\pgfpathmoveto{\pgfqpoint{0.000000in}{0.000000in}}%
\pgfpathlineto{\pgfqpoint{0.000000in}{-0.027778in}}%
\pgfusepath{stroke,fill}%
}%
\begin{pgfscope}%
\pgfsys@transformshift{5.035563in}{0.540713in}%
\pgfsys@useobject{currentmarker}{}%
\end{pgfscope}%
\end{pgfscope}%
\begin{pgfscope}%
\definecolor{textcolor}{rgb}{0.000000,0.000000,0.000000}%
\pgfsetstrokecolor{textcolor}%
\pgfsetfillcolor{textcolor}%
\pgftext[x=2.937613in,y=0.288074in,,top]{\color{textcolor}\rmfamily\fontsize{10.000000}{12.000000}\selectfont Frequency in \unit{\radian \per \s}}%
\end{pgfscope}%
\begin{pgfscope}%
\pgfsetbuttcap%
\pgfsetroundjoin%
\definecolor{currentfill}{rgb}{0.000000,0.000000,0.000000}%
\pgfsetfillcolor{currentfill}%
\pgfsetlinewidth{0.803000pt}%
\definecolor{currentstroke}{rgb}{0.000000,0.000000,0.000000}%
\pgfsetstrokecolor{currentstroke}%
\pgfsetdash{}{0pt}%
\pgfsys@defobject{currentmarker}{\pgfqpoint{-0.048611in}{0.000000in}}{\pgfqpoint{-0.000000in}{0.000000in}}{%
\pgfpathmoveto{\pgfqpoint{-0.000000in}{0.000000in}}%
\pgfpathlineto{\pgfqpoint{-0.048611in}{0.000000in}}%
\pgfusepath{stroke,fill}%
}%
\begin{pgfscope}%
\pgfsys@transformshift{0.594124in}{0.661875in}%
\pgfsys@useobject{currentmarker}{}%
\end{pgfscope}%
\end{pgfscope}%
\begin{pgfscope}%
\definecolor{textcolor}{rgb}{0.000000,0.000000,0.000000}%
\pgfsetstrokecolor{textcolor}%
\pgfsetfillcolor{textcolor}%
\pgftext[x=0.320975in, y=0.622723in, left, base]{\color{textcolor}\rmfamily\fontsize{8.000000}{9.600000}\selectfont \(\displaystyle {10^{0}}\)}%
\end{pgfscope}%
\begin{pgfscope}%
\pgfsetbuttcap%
\pgfsetroundjoin%
\definecolor{currentfill}{rgb}{0.000000,0.000000,0.000000}%
\pgfsetfillcolor{currentfill}%
\pgfsetlinewidth{0.803000pt}%
\definecolor{currentstroke}{rgb}{0.000000,0.000000,0.000000}%
\pgfsetstrokecolor{currentstroke}%
\pgfsetdash{}{0pt}%
\pgfsys@defobject{currentmarker}{\pgfqpoint{-0.048611in}{0.000000in}}{\pgfqpoint{-0.000000in}{0.000000in}}{%
\pgfpathmoveto{\pgfqpoint{-0.000000in}{0.000000in}}%
\pgfpathlineto{\pgfqpoint{-0.048611in}{0.000000in}}%
\pgfusepath{stroke,fill}%
}%
\begin{pgfscope}%
\pgfsys@transformshift{0.594124in}{1.873639in}%
\pgfsys@useobject{currentmarker}{}%
\end{pgfscope}%
\end{pgfscope}%
\begin{pgfscope}%
\definecolor{textcolor}{rgb}{0.000000,0.000000,0.000000}%
\pgfsetstrokecolor{textcolor}%
\pgfsetfillcolor{textcolor}%
\pgftext[x=0.320975in, y=1.834486in, left, base]{\color{textcolor}\rmfamily\fontsize{8.000000}{9.600000}\selectfont \(\displaystyle {10^{1}}\)}%
\end{pgfscope}%
\begin{pgfscope}%
\pgfsetbuttcap%
\pgfsetroundjoin%
\definecolor{currentfill}{rgb}{0.000000,0.000000,0.000000}%
\pgfsetfillcolor{currentfill}%
\pgfsetlinewidth{0.803000pt}%
\definecolor{currentstroke}{rgb}{0.000000,0.000000,0.000000}%
\pgfsetstrokecolor{currentstroke}%
\pgfsetdash{}{0pt}%
\pgfsys@defobject{currentmarker}{\pgfqpoint{-0.048611in}{0.000000in}}{\pgfqpoint{-0.000000in}{0.000000in}}{%
\pgfpathmoveto{\pgfqpoint{-0.000000in}{0.000000in}}%
\pgfpathlineto{\pgfqpoint{-0.048611in}{0.000000in}}%
\pgfusepath{stroke,fill}%
}%
\begin{pgfscope}%
\pgfsys@transformshift{0.594124in}{3.085403in}%
\pgfsys@useobject{currentmarker}{}%
\end{pgfscope}%
\end{pgfscope}%
\begin{pgfscope}%
\definecolor{textcolor}{rgb}{0.000000,0.000000,0.000000}%
\pgfsetstrokecolor{textcolor}%
\pgfsetfillcolor{textcolor}%
\pgftext[x=0.320975in, y=3.046250in, left, base]{\color{textcolor}\rmfamily\fontsize{8.000000}{9.600000}\selectfont \(\displaystyle {10^{2}}\)}%
\end{pgfscope}%
\begin{pgfscope}%
\pgfsetbuttcap%
\pgfsetroundjoin%
\definecolor{currentfill}{rgb}{0.000000,0.000000,0.000000}%
\pgfsetfillcolor{currentfill}%
\pgfsetlinewidth{0.602250pt}%
\definecolor{currentstroke}{rgb}{0.000000,0.000000,0.000000}%
\pgfsetstrokecolor{currentstroke}%
\pgfsetdash{}{0pt}%
\pgfsys@defobject{currentmarker}{\pgfqpoint{-0.027778in}{0.000000in}}{\pgfqpoint{-0.000000in}{0.000000in}}{%
\pgfpathmoveto{\pgfqpoint{-0.000000in}{0.000000in}}%
\pgfpathlineto{\pgfqpoint{-0.027778in}{0.000000in}}%
\pgfusepath{stroke,fill}%
}%
\begin{pgfscope}%
\pgfsys@transformshift{0.594124in}{0.544443in}%
\pgfsys@useobject{currentmarker}{}%
\end{pgfscope}%
\end{pgfscope}%
\begin{pgfscope}%
\pgfsetbuttcap%
\pgfsetroundjoin%
\definecolor{currentfill}{rgb}{0.000000,0.000000,0.000000}%
\pgfsetfillcolor{currentfill}%
\pgfsetlinewidth{0.602250pt}%
\definecolor{currentstroke}{rgb}{0.000000,0.000000,0.000000}%
\pgfsetstrokecolor{currentstroke}%
\pgfsetdash{}{0pt}%
\pgfsys@defobject{currentmarker}{\pgfqpoint{-0.027778in}{0.000000in}}{\pgfqpoint{-0.000000in}{0.000000in}}{%
\pgfpathmoveto{\pgfqpoint{-0.000000in}{0.000000in}}%
\pgfpathlineto{\pgfqpoint{-0.027778in}{0.000000in}}%
\pgfusepath{stroke,fill}%
}%
\begin{pgfscope}%
\pgfsys@transformshift{0.594124in}{0.606428in}%
\pgfsys@useobject{currentmarker}{}%
\end{pgfscope}%
\end{pgfscope}%
\begin{pgfscope}%
\pgfsetbuttcap%
\pgfsetroundjoin%
\definecolor{currentfill}{rgb}{0.000000,0.000000,0.000000}%
\pgfsetfillcolor{currentfill}%
\pgfsetlinewidth{0.602250pt}%
\definecolor{currentstroke}{rgb}{0.000000,0.000000,0.000000}%
\pgfsetstrokecolor{currentstroke}%
\pgfsetdash{}{0pt}%
\pgfsys@defobject{currentmarker}{\pgfqpoint{-0.027778in}{0.000000in}}{\pgfqpoint{-0.000000in}{0.000000in}}{%
\pgfpathmoveto{\pgfqpoint{-0.000000in}{0.000000in}}%
\pgfpathlineto{\pgfqpoint{-0.027778in}{0.000000in}}%
\pgfusepath{stroke,fill}%
}%
\begin{pgfscope}%
\pgfsys@transformshift{0.594124in}{1.026653in}%
\pgfsys@useobject{currentmarker}{}%
\end{pgfscope}%
\end{pgfscope}%
\begin{pgfscope}%
\pgfsetbuttcap%
\pgfsetroundjoin%
\definecolor{currentfill}{rgb}{0.000000,0.000000,0.000000}%
\pgfsetfillcolor{currentfill}%
\pgfsetlinewidth{0.602250pt}%
\definecolor{currentstroke}{rgb}{0.000000,0.000000,0.000000}%
\pgfsetstrokecolor{currentstroke}%
\pgfsetdash{}{0pt}%
\pgfsys@defobject{currentmarker}{\pgfqpoint{-0.027778in}{0.000000in}}{\pgfqpoint{-0.000000in}{0.000000in}}{%
\pgfpathmoveto{\pgfqpoint{-0.000000in}{0.000000in}}%
\pgfpathlineto{\pgfqpoint{-0.027778in}{0.000000in}}%
\pgfusepath{stroke,fill}%
}%
\begin{pgfscope}%
\pgfsys@transformshift{0.594124in}{1.240034in}%
\pgfsys@useobject{currentmarker}{}%
\end{pgfscope}%
\end{pgfscope}%
\begin{pgfscope}%
\pgfsetbuttcap%
\pgfsetroundjoin%
\definecolor{currentfill}{rgb}{0.000000,0.000000,0.000000}%
\pgfsetfillcolor{currentfill}%
\pgfsetlinewidth{0.602250pt}%
\definecolor{currentstroke}{rgb}{0.000000,0.000000,0.000000}%
\pgfsetstrokecolor{currentstroke}%
\pgfsetdash{}{0pt}%
\pgfsys@defobject{currentmarker}{\pgfqpoint{-0.027778in}{0.000000in}}{\pgfqpoint{-0.000000in}{0.000000in}}{%
\pgfpathmoveto{\pgfqpoint{-0.000000in}{0.000000in}}%
\pgfpathlineto{\pgfqpoint{-0.027778in}{0.000000in}}%
\pgfusepath{stroke,fill}%
}%
\begin{pgfscope}%
\pgfsys@transformshift{0.594124in}{1.391430in}%
\pgfsys@useobject{currentmarker}{}%
\end{pgfscope}%
\end{pgfscope}%
\begin{pgfscope}%
\pgfsetbuttcap%
\pgfsetroundjoin%
\definecolor{currentfill}{rgb}{0.000000,0.000000,0.000000}%
\pgfsetfillcolor{currentfill}%
\pgfsetlinewidth{0.602250pt}%
\definecolor{currentstroke}{rgb}{0.000000,0.000000,0.000000}%
\pgfsetstrokecolor{currentstroke}%
\pgfsetdash{}{0pt}%
\pgfsys@defobject{currentmarker}{\pgfqpoint{-0.027778in}{0.000000in}}{\pgfqpoint{-0.000000in}{0.000000in}}{%
\pgfpathmoveto{\pgfqpoint{-0.000000in}{0.000000in}}%
\pgfpathlineto{\pgfqpoint{-0.027778in}{0.000000in}}%
\pgfusepath{stroke,fill}%
}%
\begin{pgfscope}%
\pgfsys@transformshift{0.594124in}{1.508862in}%
\pgfsys@useobject{currentmarker}{}%
\end{pgfscope}%
\end{pgfscope}%
\begin{pgfscope}%
\pgfsetbuttcap%
\pgfsetroundjoin%
\definecolor{currentfill}{rgb}{0.000000,0.000000,0.000000}%
\pgfsetfillcolor{currentfill}%
\pgfsetlinewidth{0.602250pt}%
\definecolor{currentstroke}{rgb}{0.000000,0.000000,0.000000}%
\pgfsetstrokecolor{currentstroke}%
\pgfsetdash{}{0pt}%
\pgfsys@defobject{currentmarker}{\pgfqpoint{-0.027778in}{0.000000in}}{\pgfqpoint{-0.000000in}{0.000000in}}{%
\pgfpathmoveto{\pgfqpoint{-0.000000in}{0.000000in}}%
\pgfpathlineto{\pgfqpoint{-0.027778in}{0.000000in}}%
\pgfusepath{stroke,fill}%
}%
\begin{pgfscope}%
\pgfsys@transformshift{0.594124in}{1.604811in}%
\pgfsys@useobject{currentmarker}{}%
\end{pgfscope}%
\end{pgfscope}%
\begin{pgfscope}%
\pgfsetbuttcap%
\pgfsetroundjoin%
\definecolor{currentfill}{rgb}{0.000000,0.000000,0.000000}%
\pgfsetfillcolor{currentfill}%
\pgfsetlinewidth{0.602250pt}%
\definecolor{currentstroke}{rgb}{0.000000,0.000000,0.000000}%
\pgfsetstrokecolor{currentstroke}%
\pgfsetdash{}{0pt}%
\pgfsys@defobject{currentmarker}{\pgfqpoint{-0.027778in}{0.000000in}}{\pgfqpoint{-0.000000in}{0.000000in}}{%
\pgfpathmoveto{\pgfqpoint{-0.000000in}{0.000000in}}%
\pgfpathlineto{\pgfqpoint{-0.027778in}{0.000000in}}%
\pgfusepath{stroke,fill}%
}%
\begin{pgfscope}%
\pgfsys@transformshift{0.594124in}{1.685935in}%
\pgfsys@useobject{currentmarker}{}%
\end{pgfscope}%
\end{pgfscope}%
\begin{pgfscope}%
\pgfsetbuttcap%
\pgfsetroundjoin%
\definecolor{currentfill}{rgb}{0.000000,0.000000,0.000000}%
\pgfsetfillcolor{currentfill}%
\pgfsetlinewidth{0.602250pt}%
\definecolor{currentstroke}{rgb}{0.000000,0.000000,0.000000}%
\pgfsetstrokecolor{currentstroke}%
\pgfsetdash{}{0pt}%
\pgfsys@defobject{currentmarker}{\pgfqpoint{-0.027778in}{0.000000in}}{\pgfqpoint{-0.000000in}{0.000000in}}{%
\pgfpathmoveto{\pgfqpoint{-0.000000in}{0.000000in}}%
\pgfpathlineto{\pgfqpoint{-0.027778in}{0.000000in}}%
\pgfusepath{stroke,fill}%
}%
\begin{pgfscope}%
\pgfsys@transformshift{0.594124in}{1.756207in}%
\pgfsys@useobject{currentmarker}{}%
\end{pgfscope}%
\end{pgfscope}%
\begin{pgfscope}%
\pgfsetbuttcap%
\pgfsetroundjoin%
\definecolor{currentfill}{rgb}{0.000000,0.000000,0.000000}%
\pgfsetfillcolor{currentfill}%
\pgfsetlinewidth{0.602250pt}%
\definecolor{currentstroke}{rgb}{0.000000,0.000000,0.000000}%
\pgfsetstrokecolor{currentstroke}%
\pgfsetdash{}{0pt}%
\pgfsys@defobject{currentmarker}{\pgfqpoint{-0.027778in}{0.000000in}}{\pgfqpoint{-0.000000in}{0.000000in}}{%
\pgfpathmoveto{\pgfqpoint{-0.000000in}{0.000000in}}%
\pgfpathlineto{\pgfqpoint{-0.027778in}{0.000000in}}%
\pgfusepath{stroke,fill}%
}%
\begin{pgfscope}%
\pgfsys@transformshift{0.594124in}{1.818192in}%
\pgfsys@useobject{currentmarker}{}%
\end{pgfscope}%
\end{pgfscope}%
\begin{pgfscope}%
\pgfsetbuttcap%
\pgfsetroundjoin%
\definecolor{currentfill}{rgb}{0.000000,0.000000,0.000000}%
\pgfsetfillcolor{currentfill}%
\pgfsetlinewidth{0.602250pt}%
\definecolor{currentstroke}{rgb}{0.000000,0.000000,0.000000}%
\pgfsetstrokecolor{currentstroke}%
\pgfsetdash{}{0pt}%
\pgfsys@defobject{currentmarker}{\pgfqpoint{-0.027778in}{0.000000in}}{\pgfqpoint{-0.000000in}{0.000000in}}{%
\pgfpathmoveto{\pgfqpoint{-0.000000in}{0.000000in}}%
\pgfpathlineto{\pgfqpoint{-0.027778in}{0.000000in}}%
\pgfusepath{stroke,fill}%
}%
\begin{pgfscope}%
\pgfsys@transformshift{0.594124in}{2.238416in}%
\pgfsys@useobject{currentmarker}{}%
\end{pgfscope}%
\end{pgfscope}%
\begin{pgfscope}%
\pgfsetbuttcap%
\pgfsetroundjoin%
\definecolor{currentfill}{rgb}{0.000000,0.000000,0.000000}%
\pgfsetfillcolor{currentfill}%
\pgfsetlinewidth{0.602250pt}%
\definecolor{currentstroke}{rgb}{0.000000,0.000000,0.000000}%
\pgfsetstrokecolor{currentstroke}%
\pgfsetdash{}{0pt}%
\pgfsys@defobject{currentmarker}{\pgfqpoint{-0.027778in}{0.000000in}}{\pgfqpoint{-0.000000in}{0.000000in}}{%
\pgfpathmoveto{\pgfqpoint{-0.000000in}{0.000000in}}%
\pgfpathlineto{\pgfqpoint{-0.027778in}{0.000000in}}%
\pgfusepath{stroke,fill}%
}%
\begin{pgfscope}%
\pgfsys@transformshift{0.594124in}{2.451797in}%
\pgfsys@useobject{currentmarker}{}%
\end{pgfscope}%
\end{pgfscope}%
\begin{pgfscope}%
\pgfsetbuttcap%
\pgfsetroundjoin%
\definecolor{currentfill}{rgb}{0.000000,0.000000,0.000000}%
\pgfsetfillcolor{currentfill}%
\pgfsetlinewidth{0.602250pt}%
\definecolor{currentstroke}{rgb}{0.000000,0.000000,0.000000}%
\pgfsetstrokecolor{currentstroke}%
\pgfsetdash{}{0pt}%
\pgfsys@defobject{currentmarker}{\pgfqpoint{-0.027778in}{0.000000in}}{\pgfqpoint{-0.000000in}{0.000000in}}{%
\pgfpathmoveto{\pgfqpoint{-0.000000in}{0.000000in}}%
\pgfpathlineto{\pgfqpoint{-0.027778in}{0.000000in}}%
\pgfusepath{stroke,fill}%
}%
\begin{pgfscope}%
\pgfsys@transformshift{0.594124in}{2.603194in}%
\pgfsys@useobject{currentmarker}{}%
\end{pgfscope}%
\end{pgfscope}%
\begin{pgfscope}%
\pgfsetbuttcap%
\pgfsetroundjoin%
\definecolor{currentfill}{rgb}{0.000000,0.000000,0.000000}%
\pgfsetfillcolor{currentfill}%
\pgfsetlinewidth{0.602250pt}%
\definecolor{currentstroke}{rgb}{0.000000,0.000000,0.000000}%
\pgfsetstrokecolor{currentstroke}%
\pgfsetdash{}{0pt}%
\pgfsys@defobject{currentmarker}{\pgfqpoint{-0.027778in}{0.000000in}}{\pgfqpoint{-0.000000in}{0.000000in}}{%
\pgfpathmoveto{\pgfqpoint{-0.000000in}{0.000000in}}%
\pgfpathlineto{\pgfqpoint{-0.027778in}{0.000000in}}%
\pgfusepath{stroke,fill}%
}%
\begin{pgfscope}%
\pgfsys@transformshift{0.594124in}{2.720626in}%
\pgfsys@useobject{currentmarker}{}%
\end{pgfscope}%
\end{pgfscope}%
\begin{pgfscope}%
\pgfsetbuttcap%
\pgfsetroundjoin%
\definecolor{currentfill}{rgb}{0.000000,0.000000,0.000000}%
\pgfsetfillcolor{currentfill}%
\pgfsetlinewidth{0.602250pt}%
\definecolor{currentstroke}{rgb}{0.000000,0.000000,0.000000}%
\pgfsetstrokecolor{currentstroke}%
\pgfsetdash{}{0pt}%
\pgfsys@defobject{currentmarker}{\pgfqpoint{-0.027778in}{0.000000in}}{\pgfqpoint{-0.000000in}{0.000000in}}{%
\pgfpathmoveto{\pgfqpoint{-0.000000in}{0.000000in}}%
\pgfpathlineto{\pgfqpoint{-0.027778in}{0.000000in}}%
\pgfusepath{stroke,fill}%
}%
\begin{pgfscope}%
\pgfsys@transformshift{0.594124in}{2.816575in}%
\pgfsys@useobject{currentmarker}{}%
\end{pgfscope}%
\end{pgfscope}%
\begin{pgfscope}%
\pgfsetbuttcap%
\pgfsetroundjoin%
\definecolor{currentfill}{rgb}{0.000000,0.000000,0.000000}%
\pgfsetfillcolor{currentfill}%
\pgfsetlinewidth{0.602250pt}%
\definecolor{currentstroke}{rgb}{0.000000,0.000000,0.000000}%
\pgfsetstrokecolor{currentstroke}%
\pgfsetdash{}{0pt}%
\pgfsys@defobject{currentmarker}{\pgfqpoint{-0.027778in}{0.000000in}}{\pgfqpoint{-0.000000in}{0.000000in}}{%
\pgfpathmoveto{\pgfqpoint{-0.000000in}{0.000000in}}%
\pgfpathlineto{\pgfqpoint{-0.027778in}{0.000000in}}%
\pgfusepath{stroke,fill}%
}%
\begin{pgfscope}%
\pgfsys@transformshift{0.594124in}{2.897698in}%
\pgfsys@useobject{currentmarker}{}%
\end{pgfscope}%
\end{pgfscope}%
\begin{pgfscope}%
\pgfsetbuttcap%
\pgfsetroundjoin%
\definecolor{currentfill}{rgb}{0.000000,0.000000,0.000000}%
\pgfsetfillcolor{currentfill}%
\pgfsetlinewidth{0.602250pt}%
\definecolor{currentstroke}{rgb}{0.000000,0.000000,0.000000}%
\pgfsetstrokecolor{currentstroke}%
\pgfsetdash{}{0pt}%
\pgfsys@defobject{currentmarker}{\pgfqpoint{-0.027778in}{0.000000in}}{\pgfqpoint{-0.000000in}{0.000000in}}{%
\pgfpathmoveto{\pgfqpoint{-0.000000in}{0.000000in}}%
\pgfpathlineto{\pgfqpoint{-0.027778in}{0.000000in}}%
\pgfusepath{stroke,fill}%
}%
\begin{pgfscope}%
\pgfsys@transformshift{0.594124in}{2.967971in}%
\pgfsys@useobject{currentmarker}{}%
\end{pgfscope}%
\end{pgfscope}%
\begin{pgfscope}%
\pgfsetbuttcap%
\pgfsetroundjoin%
\definecolor{currentfill}{rgb}{0.000000,0.000000,0.000000}%
\pgfsetfillcolor{currentfill}%
\pgfsetlinewidth{0.602250pt}%
\definecolor{currentstroke}{rgb}{0.000000,0.000000,0.000000}%
\pgfsetstrokecolor{currentstroke}%
\pgfsetdash{}{0pt}%
\pgfsys@defobject{currentmarker}{\pgfqpoint{-0.027778in}{0.000000in}}{\pgfqpoint{-0.000000in}{0.000000in}}{%
\pgfpathmoveto{\pgfqpoint{-0.000000in}{0.000000in}}%
\pgfpathlineto{\pgfqpoint{-0.027778in}{0.000000in}}%
\pgfusepath{stroke,fill}%
}%
\begin{pgfscope}%
\pgfsys@transformshift{0.594124in}{3.029955in}%
\pgfsys@useobject{currentmarker}{}%
\end{pgfscope}%
\end{pgfscope}%
\begin{pgfscope}%
\definecolor{textcolor}{rgb}{0.000000,0.000000,0.000000}%
\pgfsetstrokecolor{textcolor}%
\pgfsetfillcolor{textcolor}%
\pgftext[x=0.265420in,y=1.873660in,,bottom,rotate=90.000000]{\color{textcolor}\rmfamily\fontsize{10.000000}{12.000000}\selectfont Controller Output}%
\end{pgfscope}%
\begin{pgfscope}%
\pgfpathrectangle{\pgfqpoint{0.594124in}{0.540713in}}{\pgfqpoint{4.686978in}{2.665893in}}%
\pgfusepath{clip}%
\pgfsetrectcap%
\pgfsetroundjoin%
\pgfsetlinewidth{1.003750pt}%
\definecolor{currentstroke}{rgb}{0.003922,0.450980,0.698039}%
\pgfsetstrokecolor{currentstroke}%
\pgfsetstrokeopacity{0.700000}%
\pgfsetdash{}{0pt}%
\pgfpathmoveto{\pgfqpoint{0.807169in}{1.876252in}}%
\pgfpathlineto{\pgfqpoint{0.894126in}{1.729835in}}%
\pgfpathlineto{\pgfqpoint{0.981083in}{1.584870in}}%
\pgfpathlineto{\pgfqpoint{1.068040in}{1.442385in}}%
\pgfpathlineto{\pgfqpoint{1.154997in}{1.304055in}}%
\pgfpathlineto{\pgfqpoint{1.241953in}{1.172458in}}%
\pgfpathlineto{\pgfqpoint{1.328910in}{1.051206in}}%
\pgfpathlineto{\pgfqpoint{1.415867in}{0.944625in}}%
\pgfpathlineto{\pgfqpoint{1.502824in}{0.856669in}}%
\pgfpathlineto{\pgfqpoint{1.589781in}{0.789313in}}%
\pgfpathlineto{\pgfqpoint{1.676738in}{0.741553in}}%
\pgfpathlineto{\pgfqpoint{1.763695in}{0.709930in}}%
\pgfpathlineto{\pgfqpoint{1.850652in}{0.690081in}}%
\pgfpathlineto{\pgfqpoint{1.937609in}{0.678082in}}%
\pgfpathlineto{\pgfqpoint{2.024566in}{0.671005in}}%
\pgfpathlineto{\pgfqpoint{2.111523in}{0.666894in}}%
\pgfpathlineto{\pgfqpoint{2.198480in}{0.664533in}}%
\pgfpathlineto{\pgfqpoint{2.285437in}{0.663191in}}%
\pgfpathlineto{\pgfqpoint{2.372393in}{0.662444in}}%
\pgfpathlineto{\pgfqpoint{2.459350in}{0.662052in}}%
\pgfpathlineto{\pgfqpoint{2.546307in}{0.661890in}}%
\pgfpathlineto{\pgfqpoint{2.633264in}{0.661904in}}%
\pgfpathlineto{\pgfqpoint{2.720221in}{0.662099in}}%
\pgfpathlineto{\pgfqpoint{2.807178in}{0.662539in}}%
\pgfpathlineto{\pgfqpoint{2.894135in}{0.663365in}}%
\pgfpathlineto{\pgfqpoint{2.981092in}{0.664841in}}%
\pgfpathlineto{\pgfqpoint{3.068049in}{0.667432in}}%
\pgfpathlineto{\pgfqpoint{3.155006in}{0.671936in}}%
\pgfpathlineto{\pgfqpoint{3.241963in}{0.679674in}}%
\pgfpathlineto{\pgfqpoint{3.328920in}{0.692749in}}%
\pgfpathlineto{\pgfqpoint{3.415877in}{0.714265in}}%
\pgfpathlineto{\pgfqpoint{3.502833in}{0.748279in}}%
\pgfpathlineto{\pgfqpoint{3.589790in}{0.799114in}}%
\pgfpathlineto{\pgfqpoint{3.676747in}{0.869916in}}%
\pgfpathlineto{\pgfqpoint{3.763704in}{0.961186in}}%
\pgfpathlineto{\pgfqpoint{3.850661in}{1.070516in}}%
\pgfpathlineto{\pgfqpoint{3.937618in}{1.193783in}}%
\pgfpathlineto{\pgfqpoint{4.024575in}{1.326725in}}%
\pgfpathlineto{\pgfqpoint{4.111532in}{1.465899in}}%
\pgfpathlineto{\pgfqpoint{4.198489in}{1.608893in}}%
\pgfpathlineto{\pgfqpoint{4.285446in}{1.754158in}}%
\pgfpathlineto{\pgfqpoint{4.372403in}{1.900749in}}%
\pgfpathlineto{\pgfqpoint{4.459360in}{2.048105in}}%
\pgfpathlineto{\pgfqpoint{4.546317in}{2.195900in}}%
\pgfpathlineto{\pgfqpoint{4.633273in}{2.343947in}}%
\pgfpathlineto{\pgfqpoint{4.720230in}{2.492136in}}%
\pgfpathlineto{\pgfqpoint{4.807187in}{2.640408in}}%
\pgfpathlineto{\pgfqpoint{4.894144in}{2.788725in}}%
\pgfpathlineto{\pgfqpoint{4.981101in}{2.937070in}}%
\pgfpathlineto{\pgfqpoint{5.068058in}{3.085429in}}%
\pgfusepath{stroke}%
\end{pgfscope}%
\begin{pgfscope}%
\pgfpathrectangle{\pgfqpoint{0.594124in}{0.540713in}}{\pgfqpoint{4.686978in}{2.665893in}}%
\pgfusepath{clip}%
\pgfsetrectcap%
\pgfsetroundjoin%
\pgfsetlinewidth{1.003750pt}%
\definecolor{currentstroke}{rgb}{0.870588,0.560784,0.019608}%
\pgfsetstrokecolor{currentstroke}%
\pgfsetstrokeopacity{0.700000}%
\pgfsetdash{}{0pt}%
\pgfpathmoveto{\pgfqpoint{0.807169in}{1.876252in}}%
\pgfpathlineto{\pgfqpoint{0.894126in}{1.729835in}}%
\pgfpathlineto{\pgfqpoint{0.981083in}{1.584870in}}%
\pgfpathlineto{\pgfqpoint{1.068040in}{1.442385in}}%
\pgfpathlineto{\pgfqpoint{1.154997in}{1.304055in}}%
\pgfpathlineto{\pgfqpoint{1.241953in}{1.172458in}}%
\pgfpathlineto{\pgfqpoint{1.328910in}{1.051206in}}%
\pgfpathlineto{\pgfqpoint{1.415867in}{0.944625in}}%
\pgfpathlineto{\pgfqpoint{1.502824in}{0.856669in}}%
\pgfpathlineto{\pgfqpoint{1.589781in}{0.789313in}}%
\pgfpathlineto{\pgfqpoint{1.676738in}{0.741553in}}%
\pgfpathlineto{\pgfqpoint{1.763695in}{0.709930in}}%
\pgfpathlineto{\pgfqpoint{1.850652in}{0.690082in}}%
\pgfpathlineto{\pgfqpoint{1.937609in}{0.678083in}}%
\pgfpathlineto{\pgfqpoint{2.024566in}{0.671006in}}%
\pgfpathlineto{\pgfqpoint{2.111523in}{0.666897in}}%
\pgfpathlineto{\pgfqpoint{2.198480in}{0.664537in}}%
\pgfpathlineto{\pgfqpoint{2.285437in}{0.663199in}}%
\pgfpathlineto{\pgfqpoint{2.372393in}{0.662457in}}%
\pgfpathlineto{\pgfqpoint{2.459350in}{0.662076in}}%
\pgfpathlineto{\pgfqpoint{2.546307in}{0.661932in}}%
\pgfpathlineto{\pgfqpoint{2.633264in}{0.661977in}}%
\pgfpathlineto{\pgfqpoint{2.720221in}{0.662228in}}%
\pgfpathlineto{\pgfqpoint{2.807178in}{0.662764in}}%
\pgfpathlineto{\pgfqpoint{2.894135in}{0.663759in}}%
\pgfpathlineto{\pgfqpoint{2.981092in}{0.665529in}}%
\pgfpathlineto{\pgfqpoint{3.068049in}{0.668629in}}%
\pgfpathlineto{\pgfqpoint{3.155006in}{0.673997in}}%
\pgfpathlineto{\pgfqpoint{3.241963in}{0.683176in}}%
\pgfpathlineto{\pgfqpoint{3.328920in}{0.698560in}}%
\pgfpathlineto{\pgfqpoint{3.415877in}{0.723559in}}%
\pgfpathlineto{\pgfqpoint{3.502833in}{0.762357in}}%
\pgfpathlineto{\pgfqpoint{3.589790in}{0.818923in}}%
\pgfpathlineto{\pgfqpoint{3.676747in}{0.895389in}}%
\pgfpathlineto{\pgfqpoint{3.763704in}{0.990788in}}%
\pgfpathlineto{\pgfqpoint{3.850661in}{1.101219in}}%
\pgfpathlineto{\pgfqpoint{3.937618in}{1.221136in}}%
\pgfpathlineto{\pgfqpoint{4.024575in}{1.344539in}}%
\pgfpathlineto{\pgfqpoint{4.111532in}{1.465410in}}%
\pgfpathlineto{\pgfqpoint{4.198489in}{1.577736in}}%
\pgfpathlineto{\pgfqpoint{4.285446in}{1.675836in}}%
\pgfpathlineto{\pgfqpoint{4.372403in}{1.755451in}}%
\pgfpathlineto{\pgfqpoint{4.459360in}{1.815099in}}%
\pgfpathlineto{\pgfqpoint{4.546317in}{1.856474in}}%
\pgfpathlineto{\pgfqpoint{4.633273in}{1.883370in}}%
\pgfpathlineto{\pgfqpoint{4.720230in}{1.900026in}}%
\pgfpathlineto{\pgfqpoint{4.807187in}{1.910005in}}%
\pgfpathlineto{\pgfqpoint{4.894144in}{1.915860in}}%
\pgfpathlineto{\pgfqpoint{4.981101in}{1.919251in}}%
\pgfpathlineto{\pgfqpoint{5.068058in}{1.921201in}}%
\pgfusepath{stroke}%
\end{pgfscope}%
\begin{pgfscope}%
\pgfsetrectcap%
\pgfsetmiterjoin%
\pgfsetlinewidth{0.803000pt}%
\definecolor{currentstroke}{rgb}{0.000000,0.000000,0.000000}%
\pgfsetstrokecolor{currentstroke}%
\pgfsetdash{}{0pt}%
\pgfpathmoveto{\pgfqpoint{0.594124in}{0.540713in}}%
\pgfpathlineto{\pgfqpoint{0.594124in}{3.206606in}}%
\pgfusepath{stroke}%
\end{pgfscope}%
\begin{pgfscope}%
\pgfsetrectcap%
\pgfsetmiterjoin%
\pgfsetlinewidth{0.803000pt}%
\definecolor{currentstroke}{rgb}{0.000000,0.000000,0.000000}%
\pgfsetstrokecolor{currentstroke}%
\pgfsetdash{}{0pt}%
\pgfpathmoveto{\pgfqpoint{5.281103in}{0.540713in}}%
\pgfpathlineto{\pgfqpoint{5.281103in}{3.206606in}}%
\pgfusepath{stroke}%
\end{pgfscope}%
\begin{pgfscope}%
\pgfsetrectcap%
\pgfsetmiterjoin%
\pgfsetlinewidth{0.803000pt}%
\definecolor{currentstroke}{rgb}{0.000000,0.000000,0.000000}%
\pgfsetstrokecolor{currentstroke}%
\pgfsetdash{}{0pt}%
\pgfpathmoveto{\pgfqpoint{0.594124in}{0.540713in}}%
\pgfpathlineto{\pgfqpoint{5.281103in}{0.540713in}}%
\pgfusepath{stroke}%
\end{pgfscope}%
\begin{pgfscope}%
\pgfsetrectcap%
\pgfsetmiterjoin%
\pgfsetlinewidth{0.803000pt}%
\definecolor{currentstroke}{rgb}{0.000000,0.000000,0.000000}%
\pgfsetstrokecolor{currentstroke}%
\pgfsetdash{}{0pt}%
\pgfpathmoveto{\pgfqpoint{0.594124in}{3.206606in}}%
\pgfpathlineto{\pgfqpoint{5.281103in}{3.206606in}}%
\pgfusepath{stroke}%
\end{pgfscope}%
\begin{pgfscope}%
\pgfsetbuttcap%
\pgfsetmiterjoin%
\definecolor{currentfill}{rgb}{1.000000,1.000000,1.000000}%
\pgfsetfillcolor{currentfill}%
\pgfsetfillopacity{0.800000}%
\pgfsetlinewidth{1.003750pt}%
\definecolor{currentstroke}{rgb}{0.800000,0.800000,0.800000}%
\pgfsetstrokecolor{currentstroke}%
\pgfsetstrokeopacity{0.800000}%
\pgfsetdash{}{0pt}%
\pgfpathmoveto{\pgfqpoint{0.671902in}{2.746699in}}%
\pgfpathlineto{\pgfqpoint{3.834384in}{2.746699in}}%
\pgfpathquadraticcurveto{\pgfqpoint{3.856607in}{2.746699in}}{\pgfqpoint{3.856607in}{2.768921in}}%
\pgfpathlineto{\pgfqpoint{3.856607in}{3.128828in}}%
\pgfpathquadraticcurveto{\pgfqpoint{3.856607in}{3.151050in}}{\pgfqpoint{3.834384in}{3.151050in}}%
\pgfpathlineto{\pgfqpoint{0.671902in}{3.151050in}}%
\pgfpathquadraticcurveto{\pgfqpoint{0.649680in}{3.151050in}}{\pgfqpoint{0.649680in}{3.128828in}}%
\pgfpathlineto{\pgfqpoint{0.649680in}{2.768921in}}%
\pgfpathquadraticcurveto{\pgfqpoint{0.649680in}{2.746699in}}{\pgfqpoint{0.671902in}{2.746699in}}%
\pgfpathlineto{\pgfqpoint{0.671902in}{2.746699in}}%
\pgfpathclose%
\pgfusepath{stroke,fill}%
\end{pgfscope}%
\begin{pgfscope}%
\pgfsetrectcap%
\pgfsetroundjoin%
\pgfsetlinewidth{1.003750pt}%
\definecolor{currentstroke}{rgb}{0.003922,0.450980,0.698039}%
\pgfsetstrokecolor{currentstroke}%
\pgfsetstrokeopacity{0.700000}%
\pgfsetdash{}{0pt}%
\pgfpathmoveto{\pgfqpoint{0.694124in}{3.045634in}}%
\pgfpathlineto{\pgfqpoint{0.805235in}{3.045634in}}%
\pgfpathlineto{\pgfqpoint{0.916347in}{3.045634in}}%
\pgfusepath{stroke}%
\end{pgfscope}%
\begin{pgfscope}%
\definecolor{textcolor}{rgb}{0.000000,0.000000,0.000000}%
\pgfsetstrokecolor{textcolor}%
\pgfsetfillcolor{textcolor}%
\pgftext[x=1.005235in,y=3.006745in,left,base]{\color{textcolor}\rmfamily\fontsize{8.000000}{9.600000}\selectfont PID, \(\displaystyle K_p=1\), \(\displaystyle K_i=\qty{0.1}{\per \s}\), \(\displaystyle K_d=\qty{0.01}{\s}\)}%
\end{pgfscope}%
\begin{pgfscope}%
\pgfsetrectcap%
\pgfsetroundjoin%
\pgfsetlinewidth{1.003750pt}%
\definecolor{currentstroke}{rgb}{0.870588,0.560784,0.019608}%
\pgfsetstrokecolor{currentstroke}%
\pgfsetstrokeopacity{0.700000}%
\pgfsetdash{}{0pt}%
\pgfpathmoveto{\pgfqpoint{0.694124in}{2.860125in}}%
\pgfpathlineto{\pgfqpoint{0.805235in}{2.860125in}}%
\pgfpathlineto{\pgfqpoint{0.916347in}{2.860125in}}%
\pgfusepath{stroke}%
\end{pgfscope}%
\begin{pgfscope}%
\definecolor{textcolor}{rgb}{0.000000,0.000000,0.000000}%
\pgfsetstrokecolor{textcolor}%
\pgfsetfillcolor{textcolor}%
\pgftext[x=1.005235in,y=2.821236in,left,base]{\color{textcolor}\rmfamily\fontsize{8.000000}{9.600000}\selectfont PID+filter,\(\displaystyle K_p=1\), \(\displaystyle K_i=\qty{0.1}{\per \s}\), \(\displaystyle K_d=\qty{0.01}{\s}\), \(\displaystyle \alpha=\num0.1\)}%
\end{pgfscope}%
\end{pgfpicture}%
\makeatother%
\endgroup%

    \caption{Magnitude plot over frequency of the PID controller transfer function. Both the ideal PID controller and the PID controller with a filtered derivative are shown.}
    \label{fig:sim_pid_controller}
\end{figure}

To further discuss the problem and the solution it is best to visit the frequency domain and visualize the transfer function as in figure \ref{fig:sim_pid_controller}. The ideal PID controller without filtering of the derivative can be seen to show a very strong response to frequecy inputs. This is due to the integral action, which removes any (constant) offset. It needs to have infinite gain at DC to push the offset to zero. In reality this is limited by the input noise. Then follows a plateau, whith a magnitude of $k_p$ for the proportinal term and finally the differential gain start growing in magnitude and is ever growing with rising frequency, just as expected.

With some knowledge about the process or the sensor it is possible to define an upper frequecy, above which inputs become unrealistic and must therefore be unwanted noise. By filtering the derivative term with a first order filter causes it to roll off and its gain becomes constant. The tranfer function then changes to

\begin{equation}
    C(s) = k_p + k_i \frac{1}{s} + \frac{k_d s}{1 + s \alpha k_d} \,. \label{eqn:pid_controller_filtered}
\end{equation}

Typically $\alpha$ is in the range of \numrange{0.05}{0.2} \citep[p. 129]{pid_controller}.

Another alternative is to filter the whole input. Depending on the filter cutoff, there is not much difference to equation \ref{eqn:pid_controller_filtered}, because the filter will not touch the proportional and integral part of the transfer function if it is well within its passband.

From figure \ref{fig:sim_pid_controller} it also be seen, why in some publications, the gain $k_p$ is applied to all three terms and $k_i$ and $k_d$ are replaced with $T_i$ and $T_d$ to accomodate for that.
\begin{equation}
    C(s) = k_p \left(1 \frac{1}{T_i s} + \frac{T_d s}{1 + s \alpha T_d} \right) \label{eqn:pid_controller_series}
\end{equation}

Using this form allows to shift the curve up and down keeping its shape instead of just the $k_p$ part, thus chaging the corner frequencies. The alternative form is only given here for the sake of completeness. The authors only uses the ideal form shown in equation \ref{eqn:pid_controller_laplace} with the parameters $k_p$, $k_i$ and $k_d$.

This concludes the discussion of the PID controller and begs the question of how to derive the optimal PID parameters from a given system or model. The next section discusses these tuning rules.

\clearpage
\subsection{PID Tuning Rules}
\label{sec:pid_tuning_rules}
While there are many PID tuning rules to be found in literature, their application depends on the underlying system and the desired system response. This section will discuss several of the proposed solutions and compare them to the authors use case. It aims to give a simple method to determine decent PI(D) parameters for the applications found in the lab. Among the methods discussed are the most classic set of tuning rules developed by \citeauthor{ziegler_nichols} \cite{ziegler_nichols}, an improved version of \citeauthor{simc_paper} \cite{simc_paper}, that promises better performance for non-integrating systems. These rules all include simple instruction to extract the neccesary parameters using pen and paper. Using a computer and fitting algorithms, the bar for \textit{simple} has been raised considerably, so more complex approaches can be undertaken, that extract more parameters from the system. Using these additional parameter more precise control is promised by \citeauthor{pid_basics} \cite{pid_basics, advanced_pid_control} with a method called AMIGO. Finally, it is possible to shape the control loop to result in a desired transfer function. This technique is mostly used in motor control \cite{pid_controller,advanced_pid_control} and also requires the model parameters.

All of these rules will be compared against a demo model of a room to explain the details. It is a first order model with delay, that was derived in equation \ref{eqn:first_order_plus_dead_time_model}. The discussion is limited to the FOPDT model only, because the systems treated in this work could be modelled very well using these equation. Higher models are discussed in more details for example in \cite{advanced_pid_control,pid_controller} should the reader encounter such system and require the parameters.

\begin{equation}
    G(s) = \frac{K e^{-\theta s}}{1 + s \tau} \label{eqn:demo_process_model}
\end{equation}

The following parameters were extracted from Lab 011, using the techniques shown in section \ref{sec:temperature_control_model} using equation \ref{eqn:first_order_plus_dead_time_model_time-domain}. The details are discussed in section \ref{}. The gain has be scaled to the full scale output of the controller.
\begin{table}[hb]
    \centering
    \begin{tabular}{ccc}
        \toprule
        Gain K& Lag $\tau$& Delay $\theta$ \\
        \midrule
        \qty{13.07}{\K}& \qty{395}{\s}& \qty{187}{\s}\\
        \bottomrule
    \end{tabular}
\end{table}

Before detailing the tuning parameters, the loop shaping method shall be explained first, because it used to derive the SIMC rules proposed by \citeauthor{simc_paper} and can also be used to derive custom rules. The aim of this method is to derive a controller, that shapes the model in such a way, that a desired system response to setpoint changes is achieved. A general closed-loop system with a controller $C$ and a system $G$ is shown in figure \ref{fig:closed_loop_controller}. This will be used a basis to find the required controller for a desired transfer function $\frac{Y(s)}{U(s)}$.
\begin{figure}[ht]
    \centering
    \scalebox{1}{%
        \import{figures/}{closed_loop_controller.tex}
    }% scalebox
    \caption{Closed-loop system $G$ with a controller $C$.}
    \label{fig:closed_loop_controller}
\end{figure}

Starting with the transfer function of the controlled system including the controller and the system, most experimenters would, at least in a feaverish dream, prefer a transfer function of the following divine form
\begin{equation*}
    \frac{Y(s)}{U(s)} = 1 \,,
\end{equation*}
but unfortunately life is more profane and there is no controller, that will always (and with warp speed) force a system to a certain setpoint. One may therefore settle for the second-best choice, a first-order low pass with a slow roll-off, also a small delay must be added, to ensure causality. One therefore arrives at
\begin{equation}
    \frac{Y(s)}{U(s)} = \frac{e^{-\theta s}}{1 + s \tau_c}\,, \label{eqn:desired_transfer_function}
\end{equation}
where $\tau_c$ is the closed-loop time constant and a measure for the aggressiveness of the controller. A small $\tau_c$ results in a more aggressive controller.

From \ref{fig:closed_loop_controller} the closed-loop transfer function is
\begin{align*}
    \frac{Y(s)}{U(s)} &= \frac{C(s) G(s)}{C(s) G(s) + 1} \\
    \Rightarrow C(s) &= \frac{1}{G(s)} \frac{1}{\frac{Y(s)}{U(s)} -1}
\end{align*}

Using the desired tranfer function \ref{eqn:desired_transfer_function} yields
\begin{align}
    C(s) &= \frac{1}{G(s)} \frac{e^{-\theta s}}{s \tau_c +1 - \underbrace{e^{-\theta s}}_{\approx 1 - \theta s}}\\
    &\approx \frac{1}{G(s)} \frac{e^{-\theta s}}{s (\tau_c + \theta)} \,.
\end{align}

$e^{-\theta s}$ was approximated using a first-order Taylor expansion. Finally, substituting equation \ref{eqn:demo_process_model} results in
\begin{align}
    C(s) &= \frac{1}{K} \frac{s \tau + 1}{(\tau_c + \theta) s} \nonumber\\
    &= \underbrace{\frac{1}{K} \frac{\tau}{\tau_c + \theta}}_{k_p} + \underbrace{\frac{1}{K} \frac{1}{\tau_c + \theta}}_{k_i} \frac{1}{s}\,.
\end{align}

This is a PI controller with $k_p = \frac{1}{K} \frac{\tau}{\tau_c + \theta}$ and $k_i = \frac{1}{K} \frac{1}{\tau_c + \theta}$. From these calculations, it can be seen, that a first-order model can typically be treated using a PI controller. Second-order (and higher order) models typically neccesitate a PID or more sophisticated controller for optimal control. The problems discussed in this work mainly focus in temperature control of (mostly) homogeneous objects, so the focus lies on the PI controller for most the remaining section, but the ideas and simulations can similarily be applied to the PID controller as well. Any caveats to be expected when treating a PID instead of a PI controller will be mentioned.

Using the loop shaping technique, it is fairly easy to derive custom rules in case the model parameters can be extracted. As it was said above, one such loop-shaped tuning rule is the SIMC ruleset and the authors give rules for an ample variety of different models and also investigate the parameter choice regarding stability, load and setpoint disturbances. It is therefore recommended to check \cite{simc_paper} for an appropriate set of rules for more complex models in order to save time and effort before attempting a custom approach.

\begin{table}
    \centering
    \begin{tabular}{lcccc}
        \toprule
        Tuning Rule& $k_p$& $T_i$ & $T_d$ & Source \\
        \midrule
        Z-N PI & $\frac{0.9 \tau}{K \theta}$ & $\frac{\theta}{0.3}$ & -- & \cite{ziegler_nichols}\\
        Z-N PID & $\frac{1.2 \tau}{K \theta}$ & $2 \theta$ & $\frac{\theta}{2}$ & \cite{ziegler_nichols}\\
        SIMC PI & $\frac{\tau}{K (\tau_c + \theta)}$ & $\min\left(\tau, 4 (\tau_c+\theta)\right)$ & -- & \cite{simc_paper}\\
        SIMC PID & $\frac{\tau_1}{K (\tau_c + \theta)}$ & $\min\left(\tau_1, 4 (\tau_c+\theta)\right)$ & $\tau_2$ & \cite{simc_paper}\\
        AMIGO PI & $\frac{0.15}{K} + \left(0.35 - \frac{\tau \theta}{\left(\tau + \theta\right)^2}\right) \frac{\tau}{K \theta}$ & $0.35 \theta + \frac{13 \tau^2 \theta}{\tau^2 + 12 \tau \theta + 7 \theta^2}$ & -- & \citep[p. 228]{advanced_pid_control}\\
        AMIGO PID & $\frac{1}{K} \left(0.2 + 0.45 \frac{\tau}{\theta}\right)$ & $\frac{0.4 \theta + 0.8 \tau}{\theta + 0.1 \tau} \theta$ & $\frac{0.5 \tau \theta}{0.3 \theta + \tau}$ & \citep[p. 233]{advanced_pid_control}\\
        \bottomrule
    \end{tabular}
    \caption{PI/PID parameters for different tuning rules. The PI controllers assume a first-order model, the PID rules are required when dealing with a second-order model.}
    \label{tab:pid_tuning_parameters}
\end{table}

For reasons of brevity, in table \ref{tab:pid_tuning_parameters}, the PID parameters are given as $k_p$, $T_i$ and $T_d$ as introduced in equation \ref{eqn:pid_controller_series}. $k_i$ and $k_d$ can be calculated from
\begin{align*}
    k_i &= \frac{k_p}{T_i}\\
    k_d &= k_p T_d\,.
\end{align*}

Regarding the SIMC PI/PID algrorithm, \citeauthor{simc_paper} \cite{simc_paper} and \citep[ch. 5]{simc} suggests using $\tau_c = \theta$ for “\textit{tightest possible subject to maintaining smooth control}“. Following this recommendation, the minimum can be calculated from the parameters of this example as $\min\left(\tau, 4 (\tau_c+\theta)\right) = \min\left(\tau, 8 \theta\right) = \tau$.

Using the rules above, the full system can be simulated now. This was done using Python. The simulation source code can be found in \external{data/simulations/sim\_pid\_controller.py} as part of the online suplemental material \cite{supplemental_material}. The simulation can be used to model arbitrary PI(D) controller and arbitrary models can be used as well. It allows to compare different settings before applying them to a real system. It also considereably shortens deployment times, because especially for systems with long timescales, it becomes difficult to test several parameter sets on the fly, so using a simulation can reduce this time to a few minutes instead of hours.

The simulation emulates the PID controller developed for the lab temperature controller. By default is has a sampling rate of \qty{1}{\Hz}. The simulation  will apply a setpoint change of \qty{+1}{\K} \qty{10}{\s} into the simulation. After the simulation it will plot the time domain response of the controlled system. The setpoint change in this scenario is very similar to the load disturbances, that are expected. Typically a noise source is used instead, but in contrast to the statistical noise, that is used to test for disturbance rejection, the situation in the labs are different and cannot be moddeled with stationary noise. While there is some noise coming from the sensor, the major disturbances are usually caused by experimenters instead of the lab itself. These are event like a device being switched on or off for an extended period of time, longer, than the controller needs to settle. This is equivalent to a setpoint change in terms of the error term in equation \ref{eqn:pid_controller}, since there is no difference in the error term between a setpoint and a process variable change. Do note, this is not true for the PID controller, whose derivative term directly works on the measurement (or process variable) as this was explicitely implemented above. For PID controllers, there is therefore a difference between setpoint change behaviour and noise rejection. This must be kept in mind and tested accordingly.

Simulating the model above and using the PI parameters derived from table \ref{tab:pid_tuning_parameters}, gives the plot shown in figure \ref{fig:pid_controller_comparison}.

\begin{figure}[ht]
    \centering
    %% Creator: Matplotlib, PGF backend
%%
%% To include the figure in your LaTeX document, write
%%   \input{<filename>.pgf}
%%
%% Make sure the required packages are loaded in your preamble
%%   \usepackage{pgf}
%%
%% Also ensure that all the required font packages are loaded; for instance,
%% the lmodern package is sometimes necessary when using math font.
%%   \usepackage{lmodern}
%%
%% Figures using additional raster images can only be included by \input if
%% they are in the same directory as the main LaTeX file. For loading figures
%% from other directories you can use the `import` package
%%   \usepackage{import}
%%
%% and then include the figures with
%%   \import{<path to file>}{<filename>.pgf}
%%
%% Matplotlib used the following preamble
%%   \usepackage{siunitx}
%%   \sisetup{per-mode = symbol}%
%%   \usepackage{fontspec}
%%   \makeatletter\@ifpackageloaded{underscore}{}{\usepackage[strings]{underscore}}\makeatother
%%
\begingroup%
\makeatletter%
\begin{pgfpicture}%
\pgfpathrectangle{\pgfpointorigin}{\pgfqpoint{5.492126in}{3.394321in}}%
\pgfusepath{use as bounding box, clip}%
\begin{pgfscope}%
\pgfsetbuttcap%
\pgfsetmiterjoin%
\definecolor{currentfill}{rgb}{1.000000,1.000000,1.000000}%
\pgfsetfillcolor{currentfill}%
\pgfsetlinewidth{0.000000pt}%
\definecolor{currentstroke}{rgb}{1.000000,1.000000,1.000000}%
\pgfsetstrokecolor{currentstroke}%
\pgfsetdash{}{0pt}%
\pgfpathmoveto{\pgfqpoint{0.000000in}{0.000000in}}%
\pgfpathlineto{\pgfqpoint{5.492126in}{0.000000in}}%
\pgfpathlineto{\pgfqpoint{5.492126in}{3.394321in}}%
\pgfpathlineto{\pgfqpoint{0.000000in}{3.394321in}}%
\pgfpathlineto{\pgfqpoint{0.000000in}{0.000000in}}%
\pgfpathclose%
\pgfusepath{fill}%
\end{pgfscope}%
\begin{pgfscope}%
\pgfsetbuttcap%
\pgfsetmiterjoin%
\definecolor{currentfill}{rgb}{1.000000,1.000000,1.000000}%
\pgfsetfillcolor{currentfill}%
\pgfsetlinewidth{0.000000pt}%
\definecolor{currentstroke}{rgb}{0.000000,0.000000,0.000000}%
\pgfsetstrokecolor{currentstroke}%
\pgfsetstrokeopacity{0.000000}%
\pgfsetdash{}{0pt}%
\pgfpathmoveto{\pgfqpoint{0.576061in}{0.524170in}}%
\pgfpathlineto{\pgfqpoint{5.342126in}{0.524170in}}%
\pgfpathlineto{\pgfqpoint{5.342126in}{3.244321in}}%
\pgfpathlineto{\pgfqpoint{0.576061in}{3.244321in}}%
\pgfpathlineto{\pgfqpoint{0.576061in}{0.524170in}}%
\pgfpathclose%
\pgfusepath{fill}%
\end{pgfscope}%
\begin{pgfscope}%
\pgfsetbuttcap%
\pgfsetroundjoin%
\definecolor{currentfill}{rgb}{0.000000,0.000000,0.000000}%
\pgfsetfillcolor{currentfill}%
\pgfsetlinewidth{0.803000pt}%
\definecolor{currentstroke}{rgb}{0.000000,0.000000,0.000000}%
\pgfsetstrokecolor{currentstroke}%
\pgfsetdash{}{0pt}%
\pgfsys@defobject{currentmarker}{\pgfqpoint{0.000000in}{-0.048611in}}{\pgfqpoint{0.000000in}{0.000000in}}{%
\pgfpathmoveto{\pgfqpoint{0.000000in}{0.000000in}}%
\pgfpathlineto{\pgfqpoint{0.000000in}{-0.048611in}}%
\pgfusepath{stroke,fill}%
}%
\begin{pgfscope}%
\pgfsys@transformshift{0.792700in}{0.524170in}%
\pgfsys@useobject{currentmarker}{}%
\end{pgfscope}%
\end{pgfscope}%
\begin{pgfscope}%
\definecolor{textcolor}{rgb}{0.000000,0.000000,0.000000}%
\pgfsetstrokecolor{textcolor}%
\pgfsetfillcolor{textcolor}%
\pgftext[x=0.792700in,y=0.426948in,,top]{\color{textcolor}\rmfamily\fontsize{8.000000}{9.600000}\selectfont \(\displaystyle {0}\)}%
\end{pgfscope}%
\begin{pgfscope}%
\pgfsetbuttcap%
\pgfsetroundjoin%
\definecolor{currentfill}{rgb}{0.000000,0.000000,0.000000}%
\pgfsetfillcolor{currentfill}%
\pgfsetlinewidth{0.803000pt}%
\definecolor{currentstroke}{rgb}{0.000000,0.000000,0.000000}%
\pgfsetstrokecolor{currentstroke}%
\pgfsetdash{}{0pt}%
\pgfsys@defobject{currentmarker}{\pgfqpoint{0.000000in}{-0.048611in}}{\pgfqpoint{0.000000in}{0.000000in}}{%
\pgfpathmoveto{\pgfqpoint{0.000000in}{0.000000in}}%
\pgfpathlineto{\pgfqpoint{0.000000in}{-0.048611in}}%
\pgfusepath{stroke,fill}%
}%
\begin{pgfscope}%
\pgfsys@transformshift{1.515032in}{0.524170in}%
\pgfsys@useobject{currentmarker}{}%
\end{pgfscope}%
\end{pgfscope}%
\begin{pgfscope}%
\definecolor{textcolor}{rgb}{0.000000,0.000000,0.000000}%
\pgfsetstrokecolor{textcolor}%
\pgfsetfillcolor{textcolor}%
\pgftext[x=1.515032in,y=0.426948in,,top]{\color{textcolor}\rmfamily\fontsize{8.000000}{9.600000}\selectfont \(\displaystyle {10}\)}%
\end{pgfscope}%
\begin{pgfscope}%
\pgfsetbuttcap%
\pgfsetroundjoin%
\definecolor{currentfill}{rgb}{0.000000,0.000000,0.000000}%
\pgfsetfillcolor{currentfill}%
\pgfsetlinewidth{0.803000pt}%
\definecolor{currentstroke}{rgb}{0.000000,0.000000,0.000000}%
\pgfsetstrokecolor{currentstroke}%
\pgfsetdash{}{0pt}%
\pgfsys@defobject{currentmarker}{\pgfqpoint{0.000000in}{-0.048611in}}{\pgfqpoint{0.000000in}{0.000000in}}{%
\pgfpathmoveto{\pgfqpoint{0.000000in}{0.000000in}}%
\pgfpathlineto{\pgfqpoint{0.000000in}{-0.048611in}}%
\pgfusepath{stroke,fill}%
}%
\begin{pgfscope}%
\pgfsys@transformshift{2.237364in}{0.524170in}%
\pgfsys@useobject{currentmarker}{}%
\end{pgfscope}%
\end{pgfscope}%
\begin{pgfscope}%
\definecolor{textcolor}{rgb}{0.000000,0.000000,0.000000}%
\pgfsetstrokecolor{textcolor}%
\pgfsetfillcolor{textcolor}%
\pgftext[x=2.237364in,y=0.426948in,,top]{\color{textcolor}\rmfamily\fontsize{8.000000}{9.600000}\selectfont \(\displaystyle {20}\)}%
\end{pgfscope}%
\begin{pgfscope}%
\pgfsetbuttcap%
\pgfsetroundjoin%
\definecolor{currentfill}{rgb}{0.000000,0.000000,0.000000}%
\pgfsetfillcolor{currentfill}%
\pgfsetlinewidth{0.803000pt}%
\definecolor{currentstroke}{rgb}{0.000000,0.000000,0.000000}%
\pgfsetstrokecolor{currentstroke}%
\pgfsetdash{}{0pt}%
\pgfsys@defobject{currentmarker}{\pgfqpoint{0.000000in}{-0.048611in}}{\pgfqpoint{0.000000in}{0.000000in}}{%
\pgfpathmoveto{\pgfqpoint{0.000000in}{0.000000in}}%
\pgfpathlineto{\pgfqpoint{0.000000in}{-0.048611in}}%
\pgfusepath{stroke,fill}%
}%
\begin{pgfscope}%
\pgfsys@transformshift{2.959695in}{0.524170in}%
\pgfsys@useobject{currentmarker}{}%
\end{pgfscope}%
\end{pgfscope}%
\begin{pgfscope}%
\definecolor{textcolor}{rgb}{0.000000,0.000000,0.000000}%
\pgfsetstrokecolor{textcolor}%
\pgfsetfillcolor{textcolor}%
\pgftext[x=2.959695in,y=0.426948in,,top]{\color{textcolor}\rmfamily\fontsize{8.000000}{9.600000}\selectfont \(\displaystyle {30}\)}%
\end{pgfscope}%
\begin{pgfscope}%
\pgfsetbuttcap%
\pgfsetroundjoin%
\definecolor{currentfill}{rgb}{0.000000,0.000000,0.000000}%
\pgfsetfillcolor{currentfill}%
\pgfsetlinewidth{0.803000pt}%
\definecolor{currentstroke}{rgb}{0.000000,0.000000,0.000000}%
\pgfsetstrokecolor{currentstroke}%
\pgfsetdash{}{0pt}%
\pgfsys@defobject{currentmarker}{\pgfqpoint{0.000000in}{-0.048611in}}{\pgfqpoint{0.000000in}{0.000000in}}{%
\pgfpathmoveto{\pgfqpoint{0.000000in}{0.000000in}}%
\pgfpathlineto{\pgfqpoint{0.000000in}{-0.048611in}}%
\pgfusepath{stroke,fill}%
}%
\begin{pgfscope}%
\pgfsys@transformshift{3.682027in}{0.524170in}%
\pgfsys@useobject{currentmarker}{}%
\end{pgfscope}%
\end{pgfscope}%
\begin{pgfscope}%
\definecolor{textcolor}{rgb}{0.000000,0.000000,0.000000}%
\pgfsetstrokecolor{textcolor}%
\pgfsetfillcolor{textcolor}%
\pgftext[x=3.682027in,y=0.426948in,,top]{\color{textcolor}\rmfamily\fontsize{8.000000}{9.600000}\selectfont \(\displaystyle {40}\)}%
\end{pgfscope}%
\begin{pgfscope}%
\pgfsetbuttcap%
\pgfsetroundjoin%
\definecolor{currentfill}{rgb}{0.000000,0.000000,0.000000}%
\pgfsetfillcolor{currentfill}%
\pgfsetlinewidth{0.803000pt}%
\definecolor{currentstroke}{rgb}{0.000000,0.000000,0.000000}%
\pgfsetstrokecolor{currentstroke}%
\pgfsetdash{}{0pt}%
\pgfsys@defobject{currentmarker}{\pgfqpoint{0.000000in}{-0.048611in}}{\pgfqpoint{0.000000in}{0.000000in}}{%
\pgfpathmoveto{\pgfqpoint{0.000000in}{0.000000in}}%
\pgfpathlineto{\pgfqpoint{0.000000in}{-0.048611in}}%
\pgfusepath{stroke,fill}%
}%
\begin{pgfscope}%
\pgfsys@transformshift{4.404359in}{0.524170in}%
\pgfsys@useobject{currentmarker}{}%
\end{pgfscope}%
\end{pgfscope}%
\begin{pgfscope}%
\definecolor{textcolor}{rgb}{0.000000,0.000000,0.000000}%
\pgfsetstrokecolor{textcolor}%
\pgfsetfillcolor{textcolor}%
\pgftext[x=4.404359in,y=0.426948in,,top]{\color{textcolor}\rmfamily\fontsize{8.000000}{9.600000}\selectfont \(\displaystyle {50}\)}%
\end{pgfscope}%
\begin{pgfscope}%
\pgfsetbuttcap%
\pgfsetroundjoin%
\definecolor{currentfill}{rgb}{0.000000,0.000000,0.000000}%
\pgfsetfillcolor{currentfill}%
\pgfsetlinewidth{0.803000pt}%
\definecolor{currentstroke}{rgb}{0.000000,0.000000,0.000000}%
\pgfsetstrokecolor{currentstroke}%
\pgfsetdash{}{0pt}%
\pgfsys@defobject{currentmarker}{\pgfqpoint{0.000000in}{-0.048611in}}{\pgfqpoint{0.000000in}{0.000000in}}{%
\pgfpathmoveto{\pgfqpoint{0.000000in}{0.000000in}}%
\pgfpathlineto{\pgfqpoint{0.000000in}{-0.048611in}}%
\pgfusepath{stroke,fill}%
}%
\begin{pgfscope}%
\pgfsys@transformshift{5.126691in}{0.524170in}%
\pgfsys@useobject{currentmarker}{}%
\end{pgfscope}%
\end{pgfscope}%
\begin{pgfscope}%
\definecolor{textcolor}{rgb}{0.000000,0.000000,0.000000}%
\pgfsetstrokecolor{textcolor}%
\pgfsetfillcolor{textcolor}%
\pgftext[x=5.126691in,y=0.426948in,,top]{\color{textcolor}\rmfamily\fontsize{8.000000}{9.600000}\selectfont \(\displaystyle {60}\)}%
\end{pgfscope}%
\begin{pgfscope}%
\definecolor{textcolor}{rgb}{0.000000,0.000000,0.000000}%
\pgfsetstrokecolor{textcolor}%
\pgfsetfillcolor{textcolor}%
\pgftext[x=2.959093in,y=0.272725in,,top]{\color{textcolor}\rmfamily\fontsize{10.000000}{12.000000}\selectfont Time in \unit{\minute}}%
\end{pgfscope}%
\begin{pgfscope}%
\pgfsetbuttcap%
\pgfsetroundjoin%
\definecolor{currentfill}{rgb}{0.000000,0.000000,0.000000}%
\pgfsetfillcolor{currentfill}%
\pgfsetlinewidth{0.803000pt}%
\definecolor{currentstroke}{rgb}{0.000000,0.000000,0.000000}%
\pgfsetstrokecolor{currentstroke}%
\pgfsetdash{}{0pt}%
\pgfsys@defobject{currentmarker}{\pgfqpoint{-0.048611in}{0.000000in}}{\pgfqpoint{-0.000000in}{0.000000in}}{%
\pgfpathmoveto{\pgfqpoint{-0.000000in}{0.000000in}}%
\pgfpathlineto{\pgfqpoint{-0.048611in}{0.000000in}}%
\pgfusepath{stroke,fill}%
}%
\begin{pgfscope}%
\pgfsys@transformshift{0.576061in}{0.647813in}%
\pgfsys@useobject{currentmarker}{}%
\end{pgfscope}%
\end{pgfscope}%
\begin{pgfscope}%
\definecolor{textcolor}{rgb}{0.000000,0.000000,0.000000}%
\pgfsetstrokecolor{textcolor}%
\pgfsetfillcolor{textcolor}%
\pgftext[x=0.327987in, y=0.609257in, left, base]{\color{textcolor}\rmfamily\fontsize{8.000000}{9.600000}\selectfont \(\displaystyle {0.0}\)}%
\end{pgfscope}%
\begin{pgfscope}%
\pgfsetbuttcap%
\pgfsetroundjoin%
\definecolor{currentfill}{rgb}{0.000000,0.000000,0.000000}%
\pgfsetfillcolor{currentfill}%
\pgfsetlinewidth{0.803000pt}%
\definecolor{currentstroke}{rgb}{0.000000,0.000000,0.000000}%
\pgfsetstrokecolor{currentstroke}%
\pgfsetdash{}{0pt}%
\pgfsys@defobject{currentmarker}{\pgfqpoint{-0.048611in}{0.000000in}}{\pgfqpoint{-0.000000in}{0.000000in}}{%
\pgfpathmoveto{\pgfqpoint{-0.000000in}{0.000000in}}%
\pgfpathlineto{\pgfqpoint{-0.048611in}{0.000000in}}%
\pgfusepath{stroke,fill}%
}%
\begin{pgfscope}%
\pgfsys@transformshift{0.576061in}{1.045131in}%
\pgfsys@useobject{currentmarker}{}%
\end{pgfscope}%
\end{pgfscope}%
\begin{pgfscope}%
\definecolor{textcolor}{rgb}{0.000000,0.000000,0.000000}%
\pgfsetstrokecolor{textcolor}%
\pgfsetfillcolor{textcolor}%
\pgftext[x=0.327987in, y=1.006576in, left, base]{\color{textcolor}\rmfamily\fontsize{8.000000}{9.600000}\selectfont \(\displaystyle {0.2}\)}%
\end{pgfscope}%
\begin{pgfscope}%
\pgfsetbuttcap%
\pgfsetroundjoin%
\definecolor{currentfill}{rgb}{0.000000,0.000000,0.000000}%
\pgfsetfillcolor{currentfill}%
\pgfsetlinewidth{0.803000pt}%
\definecolor{currentstroke}{rgb}{0.000000,0.000000,0.000000}%
\pgfsetstrokecolor{currentstroke}%
\pgfsetdash{}{0pt}%
\pgfsys@defobject{currentmarker}{\pgfqpoint{-0.048611in}{0.000000in}}{\pgfqpoint{-0.000000in}{0.000000in}}{%
\pgfpathmoveto{\pgfqpoint{-0.000000in}{0.000000in}}%
\pgfpathlineto{\pgfqpoint{-0.048611in}{0.000000in}}%
\pgfusepath{stroke,fill}%
}%
\begin{pgfscope}%
\pgfsys@transformshift{0.576061in}{1.442450in}%
\pgfsys@useobject{currentmarker}{}%
\end{pgfscope}%
\end{pgfscope}%
\begin{pgfscope}%
\definecolor{textcolor}{rgb}{0.000000,0.000000,0.000000}%
\pgfsetstrokecolor{textcolor}%
\pgfsetfillcolor{textcolor}%
\pgftext[x=0.327987in, y=1.403894in, left, base]{\color{textcolor}\rmfamily\fontsize{8.000000}{9.600000}\selectfont \(\displaystyle {0.4}\)}%
\end{pgfscope}%
\begin{pgfscope}%
\pgfsetbuttcap%
\pgfsetroundjoin%
\definecolor{currentfill}{rgb}{0.000000,0.000000,0.000000}%
\pgfsetfillcolor{currentfill}%
\pgfsetlinewidth{0.803000pt}%
\definecolor{currentstroke}{rgb}{0.000000,0.000000,0.000000}%
\pgfsetstrokecolor{currentstroke}%
\pgfsetdash{}{0pt}%
\pgfsys@defobject{currentmarker}{\pgfqpoint{-0.048611in}{0.000000in}}{\pgfqpoint{-0.000000in}{0.000000in}}{%
\pgfpathmoveto{\pgfqpoint{-0.000000in}{0.000000in}}%
\pgfpathlineto{\pgfqpoint{-0.048611in}{0.000000in}}%
\pgfusepath{stroke,fill}%
}%
\begin{pgfscope}%
\pgfsys@transformshift{0.576061in}{1.839768in}%
\pgfsys@useobject{currentmarker}{}%
\end{pgfscope}%
\end{pgfscope}%
\begin{pgfscope}%
\definecolor{textcolor}{rgb}{0.000000,0.000000,0.000000}%
\pgfsetstrokecolor{textcolor}%
\pgfsetfillcolor{textcolor}%
\pgftext[x=0.327987in, y=1.801213in, left, base]{\color{textcolor}\rmfamily\fontsize{8.000000}{9.600000}\selectfont \(\displaystyle {0.6}\)}%
\end{pgfscope}%
\begin{pgfscope}%
\pgfsetbuttcap%
\pgfsetroundjoin%
\definecolor{currentfill}{rgb}{0.000000,0.000000,0.000000}%
\pgfsetfillcolor{currentfill}%
\pgfsetlinewidth{0.803000pt}%
\definecolor{currentstroke}{rgb}{0.000000,0.000000,0.000000}%
\pgfsetstrokecolor{currentstroke}%
\pgfsetdash{}{0pt}%
\pgfsys@defobject{currentmarker}{\pgfqpoint{-0.048611in}{0.000000in}}{\pgfqpoint{-0.000000in}{0.000000in}}{%
\pgfpathmoveto{\pgfqpoint{-0.000000in}{0.000000in}}%
\pgfpathlineto{\pgfqpoint{-0.048611in}{0.000000in}}%
\pgfusepath{stroke,fill}%
}%
\begin{pgfscope}%
\pgfsys@transformshift{0.576061in}{2.237087in}%
\pgfsys@useobject{currentmarker}{}%
\end{pgfscope}%
\end{pgfscope}%
\begin{pgfscope}%
\definecolor{textcolor}{rgb}{0.000000,0.000000,0.000000}%
\pgfsetstrokecolor{textcolor}%
\pgfsetfillcolor{textcolor}%
\pgftext[x=0.327987in, y=2.198531in, left, base]{\color{textcolor}\rmfamily\fontsize{8.000000}{9.600000}\selectfont \(\displaystyle {0.8}\)}%
\end{pgfscope}%
\begin{pgfscope}%
\pgfsetbuttcap%
\pgfsetroundjoin%
\definecolor{currentfill}{rgb}{0.000000,0.000000,0.000000}%
\pgfsetfillcolor{currentfill}%
\pgfsetlinewidth{0.803000pt}%
\definecolor{currentstroke}{rgb}{0.000000,0.000000,0.000000}%
\pgfsetstrokecolor{currentstroke}%
\pgfsetdash{}{0pt}%
\pgfsys@defobject{currentmarker}{\pgfqpoint{-0.048611in}{0.000000in}}{\pgfqpoint{-0.000000in}{0.000000in}}{%
\pgfpathmoveto{\pgfqpoint{-0.000000in}{0.000000in}}%
\pgfpathlineto{\pgfqpoint{-0.048611in}{0.000000in}}%
\pgfusepath{stroke,fill}%
}%
\begin{pgfscope}%
\pgfsys@transformshift{0.576061in}{2.634405in}%
\pgfsys@useobject{currentmarker}{}%
\end{pgfscope}%
\end{pgfscope}%
\begin{pgfscope}%
\definecolor{textcolor}{rgb}{0.000000,0.000000,0.000000}%
\pgfsetstrokecolor{textcolor}%
\pgfsetfillcolor{textcolor}%
\pgftext[x=0.327987in, y=2.595849in, left, base]{\color{textcolor}\rmfamily\fontsize{8.000000}{9.600000}\selectfont \(\displaystyle {1.0}\)}%
\end{pgfscope}%
\begin{pgfscope}%
\pgfsetbuttcap%
\pgfsetroundjoin%
\definecolor{currentfill}{rgb}{0.000000,0.000000,0.000000}%
\pgfsetfillcolor{currentfill}%
\pgfsetlinewidth{0.803000pt}%
\definecolor{currentstroke}{rgb}{0.000000,0.000000,0.000000}%
\pgfsetstrokecolor{currentstroke}%
\pgfsetdash{}{0pt}%
\pgfsys@defobject{currentmarker}{\pgfqpoint{-0.048611in}{0.000000in}}{\pgfqpoint{-0.000000in}{0.000000in}}{%
\pgfpathmoveto{\pgfqpoint{-0.000000in}{0.000000in}}%
\pgfpathlineto{\pgfqpoint{-0.048611in}{0.000000in}}%
\pgfusepath{stroke,fill}%
}%
\begin{pgfscope}%
\pgfsys@transformshift{0.576061in}{3.031723in}%
\pgfsys@useobject{currentmarker}{}%
\end{pgfscope}%
\end{pgfscope}%
\begin{pgfscope}%
\definecolor{textcolor}{rgb}{0.000000,0.000000,0.000000}%
\pgfsetstrokecolor{textcolor}%
\pgfsetfillcolor{textcolor}%
\pgftext[x=0.327987in, y=2.993168in, left, base]{\color{textcolor}\rmfamily\fontsize{8.000000}{9.600000}\selectfont \(\displaystyle {1.2}\)}%
\end{pgfscope}%
\begin{pgfscope}%
\definecolor{textcolor}{rgb}{0.000000,0.000000,0.000000}%
\pgfsetstrokecolor{textcolor}%
\pgfsetfillcolor{textcolor}%
\pgftext[x=0.272432in,y=1.884245in,,bottom,rotate=90.000000]{\color{textcolor}\rmfamily\fontsize{10.000000}{12.000000}\selectfont Temperature deviation in \unit{\K}}%
\end{pgfscope}%
\begin{pgfscope}%
\pgfpathrectangle{\pgfqpoint{0.576061in}{0.524170in}}{\pgfqpoint{4.766066in}{2.720151in}}%
\pgfusepath{clip}%
\pgfsetrectcap%
\pgfsetroundjoin%
\pgfsetlinewidth{1.003750pt}%
\definecolor{currentstroke}{rgb}{0.003922,0.450980,0.698039}%
\pgfsetstrokecolor{currentstroke}%
\pgfsetstrokeopacity{0.700000}%
\pgfsetdash{}{0pt}%
\pgfpathmoveto{\pgfqpoint{0.792700in}{0.647813in}}%
\pgfpathlineto{\pgfqpoint{1.029866in}{0.647813in}}%
\pgfpathlineto{\pgfqpoint{1.091264in}{1.128686in}}%
\pgfpathlineto{\pgfqpoint{1.155070in}{1.610834in}}%
\pgfpathlineto{\pgfqpoint{1.222487in}{2.103262in}}%
\pgfpathlineto{\pgfqpoint{1.271847in}{2.449827in}}%
\pgfpathlineto{\pgfqpoint{1.292313in}{2.575992in}}%
\pgfpathlineto{\pgfqpoint{1.311575in}{2.681901in}}%
\pgfpathlineto{\pgfqpoint{1.330837in}{2.775672in}}%
\pgfpathlineto{\pgfqpoint{1.348895in}{2.852812in}}%
\pgfpathlineto{\pgfqpoint{1.365750in}{2.915610in}}%
\pgfpathlineto{\pgfqpoint{1.381400in}{2.966127in}}%
\pgfpathlineto{\pgfqpoint{1.397051in}{3.009276in}}%
\pgfpathlineto{\pgfqpoint{1.411498in}{3.042680in}}%
\pgfpathlineto{\pgfqpoint{1.424740in}{3.067968in}}%
\pgfpathlineto{\pgfqpoint{1.436779in}{3.086595in}}%
\pgfpathlineto{\pgfqpoint{1.448818in}{3.101124in}}%
\pgfpathlineto{\pgfqpoint{1.459653in}{3.110738in}}%
\pgfpathlineto{\pgfqpoint{1.469284in}{3.116564in}}%
\pgfpathlineto{\pgfqpoint{1.478915in}{3.119854in}}%
\pgfpathlineto{\pgfqpoint{1.488546in}{3.120642in}}%
\pgfpathlineto{\pgfqpoint{1.498177in}{3.119030in}}%
\pgfpathlineto{\pgfqpoint{1.507808in}{3.115154in}}%
\pgfpathlineto{\pgfqpoint{1.518643in}{3.108254in}}%
\pgfpathlineto{\pgfqpoint{1.530682in}{3.097652in}}%
\pgfpathlineto{\pgfqpoint{1.543925in}{3.082709in}}%
\pgfpathlineto{\pgfqpoint{1.558372in}{3.062859in}}%
\pgfpathlineto{\pgfqpoint{1.575226in}{3.035554in}}%
\pgfpathlineto{\pgfqpoint{1.594488in}{2.999657in}}%
\pgfpathlineto{\pgfqpoint{1.616158in}{2.954382in}}%
\pgfpathlineto{\pgfqpoint{1.643848in}{2.890922in}}%
\pgfpathlineto{\pgfqpoint{1.683576in}{2.793545in}}%
\pgfpathlineto{\pgfqpoint{1.746178in}{2.640211in}}%
\pgfpathlineto{\pgfqpoint{1.776275in}{2.572503in}}%
\pgfpathlineto{\pgfqpoint{1.801557in}{2.520711in}}%
\pgfpathlineto{\pgfqpoint{1.824431in}{2.478631in}}%
\pgfpathlineto{\pgfqpoint{1.846101in}{2.443410in}}%
\pgfpathlineto{\pgfqpoint{1.865363in}{2.416143in}}%
\pgfpathlineto{\pgfqpoint{1.883421in}{2.394161in}}%
\pgfpathlineto{\pgfqpoint{1.901479in}{2.375706in}}%
\pgfpathlineto{\pgfqpoint{1.918334in}{2.361676in}}%
\pgfpathlineto{\pgfqpoint{1.933984in}{2.351388in}}%
\pgfpathlineto{\pgfqpoint{1.949635in}{2.343692in}}%
\pgfpathlineto{\pgfqpoint{1.965285in}{2.338518in}}%
\pgfpathlineto{\pgfqpoint{1.980936in}{2.335777in}}%
\pgfpathlineto{\pgfqpoint{1.996586in}{2.335361in}}%
\pgfpathlineto{\pgfqpoint{2.012237in}{2.337145in}}%
\pgfpathlineto{\pgfqpoint{2.029091in}{2.341369in}}%
\pgfpathlineto{\pgfqpoint{2.047150in}{2.348336in}}%
\pgfpathlineto{\pgfqpoint{2.066412in}{2.358283in}}%
\pgfpathlineto{\pgfqpoint{2.086878in}{2.371355in}}%
\pgfpathlineto{\pgfqpoint{2.110955in}{2.389521in}}%
\pgfpathlineto{\pgfqpoint{2.138645in}{2.413350in}}%
\pgfpathlineto{\pgfqpoint{2.174761in}{2.447631in}}%
\pgfpathlineto{\pgfqpoint{2.296354in}{2.565626in}}%
\pgfpathlineto{\pgfqpoint{2.328859in}{2.592627in}}%
\pgfpathlineto{\pgfqpoint{2.357752in}{2.613761in}}%
\pgfpathlineto{\pgfqpoint{2.384238in}{2.630474in}}%
\pgfpathlineto{\pgfqpoint{2.409519in}{2.643912in}}%
\pgfpathlineto{\pgfqpoint{2.434801in}{2.654832in}}%
\pgfpathlineto{\pgfqpoint{2.460083in}{2.663243in}}%
\pgfpathlineto{\pgfqpoint{2.485364in}{2.669213in}}%
\pgfpathlineto{\pgfqpoint{2.510646in}{2.672861in}}%
\pgfpathlineto{\pgfqpoint{2.537131in}{2.674369in}}%
\pgfpathlineto{\pgfqpoint{2.564821in}{2.673663in}}%
\pgfpathlineto{\pgfqpoint{2.594918in}{2.670594in}}%
\pgfpathlineto{\pgfqpoint{2.628627in}{2.664817in}}%
\pgfpathlineto{\pgfqpoint{2.668355in}{2.655646in}}%
\pgfpathlineto{\pgfqpoint{2.724938in}{2.640051in}}%
\pgfpathlineto{\pgfqpoint{2.823656in}{2.612716in}}%
\pgfpathlineto{\pgfqpoint{2.871812in}{2.601925in}}%
\pgfpathlineto{\pgfqpoint{2.915152in}{2.594509in}}%
\pgfpathlineto{\pgfqpoint{2.956084in}{2.589739in}}%
\pgfpathlineto{\pgfqpoint{2.997016in}{2.587176in}}%
\pgfpathlineto{\pgfqpoint{3.040356in}{2.586742in}}%
\pgfpathlineto{\pgfqpoint{3.086103in}{2.588529in}}%
\pgfpathlineto{\pgfqpoint{3.139074in}{2.592903in}}%
\pgfpathlineto{\pgfqpoint{3.207696in}{2.600979in}}%
\pgfpathlineto{\pgfqpoint{3.395502in}{2.624287in}}%
\pgfpathlineto{\pgfqpoint{3.464124in}{2.629834in}}%
\pgfpathlineto{\pgfqpoint{3.531541in}{2.633028in}}%
\pgfpathlineto{\pgfqpoint{3.603775in}{2.634160in}}%
\pgfpathlineto{\pgfqpoint{3.690454in}{2.633166in}}%
\pgfpathlineto{\pgfqpoint{3.863814in}{2.628240in}}%
\pgfpathlineto{\pgfqpoint{3.995038in}{2.625842in}}%
\pgfpathlineto{\pgfqpoint{4.116630in}{2.625933in}}%
\pgfpathlineto{\pgfqpoint{4.282766in}{2.628553in}}%
\pgfpathlineto{\pgfqpoint{4.544010in}{2.632512in}}%
\pgfpathlineto{\pgfqpoint{4.749874in}{2.633025in}}%
\pgfpathlineto{\pgfqpoint{5.125487in}{2.632587in}}%
\pgfpathlineto{\pgfqpoint{5.125487in}{2.632587in}}%
\pgfusepath{stroke}%
\end{pgfscope}%
\begin{pgfscope}%
\pgfpathrectangle{\pgfqpoint{0.576061in}{0.524170in}}{\pgfqpoint{4.766066in}{2.720151in}}%
\pgfusepath{clip}%
\pgfsetrectcap%
\pgfsetroundjoin%
\pgfsetlinewidth{1.003750pt}%
\definecolor{currentstroke}{rgb}{0.870588,0.560784,0.019608}%
\pgfsetstrokecolor{currentstroke}%
\pgfsetstrokeopacity{0.700000}%
\pgfsetdash{}{0pt}%
\pgfpathmoveto{\pgfqpoint{0.792700in}{0.647813in}}%
\pgfpathlineto{\pgfqpoint{1.029866in}{0.647813in}}%
\pgfpathlineto{\pgfqpoint{1.287497in}{1.785341in}}%
\pgfpathlineto{\pgfqpoint{1.318798in}{1.909427in}}%
\pgfpathlineto{\pgfqpoint{1.348895in}{2.019595in}}%
\pgfpathlineto{\pgfqpoint{1.377789in}{2.116920in}}%
\pgfpathlineto{\pgfqpoint{1.405478in}{2.202435in}}%
\pgfpathlineto{\pgfqpoint{1.431964in}{2.277129in}}%
\pgfpathlineto{\pgfqpoint{1.457245in}{2.341950in}}%
\pgfpathlineto{\pgfqpoint{1.481323in}{2.397801in}}%
\pgfpathlineto{\pgfqpoint{1.505401in}{2.447965in}}%
\pgfpathlineto{\pgfqpoint{1.528275in}{2.490584in}}%
\pgfpathlineto{\pgfqpoint{1.551148in}{2.528566in}}%
\pgfpathlineto{\pgfqpoint{1.574022in}{2.562176in}}%
\pgfpathlineto{\pgfqpoint{1.595692in}{2.590222in}}%
\pgfpathlineto{\pgfqpoint{1.617362in}{2.614807in}}%
\pgfpathlineto{\pgfqpoint{1.639032in}{2.636153in}}%
\pgfpathlineto{\pgfqpoint{1.660702in}{2.654487in}}%
\pgfpathlineto{\pgfqpoint{1.682372in}{2.670033in}}%
\pgfpathlineto{\pgfqpoint{1.705246in}{2.683667in}}%
\pgfpathlineto{\pgfqpoint{1.728120in}{2.694709in}}%
\pgfpathlineto{\pgfqpoint{1.752197in}{2.703816in}}%
\pgfpathlineto{\pgfqpoint{1.777479in}{2.710907in}}%
\pgfpathlineto{\pgfqpoint{1.803964in}{2.715956in}}%
\pgfpathlineto{\pgfqpoint{1.831654in}{2.718984in}}%
\pgfpathlineto{\pgfqpoint{1.861751in}{2.720074in}}%
\pgfpathlineto{\pgfqpoint{1.895460in}{2.719067in}}%
\pgfpathlineto{\pgfqpoint{1.933984in}{2.715679in}}%
\pgfpathlineto{\pgfqpoint{1.980936in}{2.709252in}}%
\pgfpathlineto{\pgfqpoint{2.045946in}{2.697926in}}%
\pgfpathlineto{\pgfqpoint{2.237364in}{2.663351in}}%
\pgfpathlineto{\pgfqpoint{2.312005in}{2.652877in}}%
\pgfpathlineto{\pgfqpoint{2.384238in}{2.644977in}}%
\pgfpathlineto{\pgfqpoint{2.460083in}{2.638937in}}%
\pgfpathlineto{\pgfqpoint{2.543151in}{2.634609in}}%
\pgfpathlineto{\pgfqpoint{2.639462in}{2.631912in}}%
\pgfpathlineto{\pgfqpoint{2.761054in}{2.630877in}}%
\pgfpathlineto{\pgfqpoint{2.954880in}{2.631773in}}%
\pgfpathlineto{\pgfqpoint{3.436434in}{2.634374in}}%
\pgfpathlineto{\pgfqpoint{4.085329in}{2.634449in}}%
\pgfpathlineto{\pgfqpoint{5.125487in}{2.634404in}}%
\pgfpathlineto{\pgfqpoint{5.125487in}{2.634404in}}%
\pgfusepath{stroke}%
\end{pgfscope}%
\begin{pgfscope}%
\pgfpathrectangle{\pgfqpoint{0.576061in}{0.524170in}}{\pgfqpoint{4.766066in}{2.720151in}}%
\pgfusepath{clip}%
\pgfsetrectcap%
\pgfsetroundjoin%
\pgfsetlinewidth{1.003750pt}%
\definecolor{currentstroke}{rgb}{0.800000,0.470588,0.737255}%
\pgfsetstrokecolor{currentstroke}%
\pgfsetstrokeopacity{0.700000}%
\pgfsetdash{}{0pt}%
\pgfpathmoveto{\pgfqpoint{0.792700in}{0.647813in}}%
\pgfpathlineto{\pgfqpoint{1.029866in}{0.647813in}}%
\pgfpathlineto{\pgfqpoint{1.153866in}{0.873323in}}%
\pgfpathlineto{\pgfqpoint{1.341672in}{1.215797in}}%
\pgfpathlineto{\pgfqpoint{1.395847in}{1.307368in}}%
\pgfpathlineto{\pgfqpoint{1.447614in}{1.390613in}}%
\pgfpathlineto{\pgfqpoint{1.498177in}{1.467810in}}%
\pgfpathlineto{\pgfqpoint{1.548741in}{1.540932in}}%
\pgfpathlineto{\pgfqpoint{1.599304in}{1.610089in}}%
\pgfpathlineto{\pgfqpoint{1.648663in}{1.673902in}}%
\pgfpathlineto{\pgfqpoint{1.699226in}{1.735623in}}%
\pgfpathlineto{\pgfqpoint{1.749790in}{1.793802in}}%
\pgfpathlineto{\pgfqpoint{1.800353in}{1.848598in}}%
\pgfpathlineto{\pgfqpoint{1.850916in}{1.900166in}}%
\pgfpathlineto{\pgfqpoint{1.902683in}{1.949782in}}%
\pgfpathlineto{\pgfqpoint{1.954450in}{1.996342in}}%
\pgfpathlineto{\pgfqpoint{2.007421in}{2.040986in}}%
\pgfpathlineto{\pgfqpoint{2.060392in}{2.082760in}}%
\pgfpathlineto{\pgfqpoint{2.114567in}{2.122683in}}%
\pgfpathlineto{\pgfqpoint{2.169946in}{2.160734in}}%
\pgfpathlineto{\pgfqpoint{2.226529in}{2.196903in}}%
\pgfpathlineto{\pgfqpoint{2.284315in}{2.231193in}}%
\pgfpathlineto{\pgfqpoint{2.343306in}{2.263613in}}%
\pgfpathlineto{\pgfqpoint{2.403500in}{2.294186in}}%
\pgfpathlineto{\pgfqpoint{2.466102in}{2.323478in}}%
\pgfpathlineto{\pgfqpoint{2.529908in}{2.350897in}}%
\pgfpathlineto{\pgfqpoint{2.596122in}{2.376947in}}%
\pgfpathlineto{\pgfqpoint{2.664743in}{2.401565in}}%
\pgfpathlineto{\pgfqpoint{2.735772in}{2.424706in}}%
\pgfpathlineto{\pgfqpoint{2.810413in}{2.446679in}}%
\pgfpathlineto{\pgfqpoint{2.888666in}{2.467374in}}%
\pgfpathlineto{\pgfqpoint{2.970530in}{2.486706in}}%
\pgfpathlineto{\pgfqpoint{3.057210in}{2.504858in}}%
\pgfpathlineto{\pgfqpoint{3.148705in}{2.521713in}}%
\pgfpathlineto{\pgfqpoint{3.246220in}{2.537377in}}%
\pgfpathlineto{\pgfqpoint{3.349755in}{2.551732in}}%
\pgfpathlineto{\pgfqpoint{3.461716in}{2.564972in}}%
\pgfpathlineto{\pgfqpoint{3.582105in}{2.576945in}}%
\pgfpathlineto{\pgfqpoint{3.713328in}{2.587744in}}%
\pgfpathlineto{\pgfqpoint{3.858998in}{2.597458in}}%
\pgfpathlineto{\pgfqpoint{4.021523in}{2.606009in}}%
\pgfpathlineto{\pgfqpoint{4.205718in}{2.613411in}}%
\pgfpathlineto{\pgfqpoint{4.417602in}{2.619645in}}%
\pgfpathlineto{\pgfqpoint{4.669214in}{2.624759in}}%
\pgfpathlineto{\pgfqpoint{4.979817in}{2.628762in}}%
\pgfpathlineto{\pgfqpoint{5.125487in}{2.630036in}}%
\pgfpathlineto{\pgfqpoint{5.125487in}{2.630036in}}%
\pgfusepath{stroke}%
\end{pgfscope}%
\begin{pgfscope}%
\pgfpathrectangle{\pgfqpoint{0.576061in}{0.524170in}}{\pgfqpoint{4.766066in}{2.720151in}}%
\pgfusepath{clip}%
\pgfsetbuttcap%
\pgfsetroundjoin%
\pgfsetlinewidth{1.003750pt}%
\definecolor{currentstroke}{rgb}{0.007843,0.619608,0.450980}%
\pgfsetstrokecolor{currentstroke}%
\pgfsetstrokeopacity{0.700000}%
\pgfsetdash{{3.700000pt}{1.600000pt}}{0.000000pt}%
\pgfpathmoveto{\pgfqpoint{0.792700in}{0.647813in}}%
\pgfpathlineto{\pgfqpoint{0.803535in}{0.647813in}}%
\pgfpathlineto{\pgfqpoint{0.804739in}{2.634405in}}%
\pgfpathlineto{\pgfqpoint{5.125487in}{2.634405in}}%
\pgfpathlineto{\pgfqpoint{5.125487in}{2.634405in}}%
\pgfusepath{stroke}%
\end{pgfscope}%
\begin{pgfscope}%
\pgfsetrectcap%
\pgfsetmiterjoin%
\pgfsetlinewidth{0.803000pt}%
\definecolor{currentstroke}{rgb}{0.000000,0.000000,0.000000}%
\pgfsetstrokecolor{currentstroke}%
\pgfsetdash{}{0pt}%
\pgfpathmoveto{\pgfqpoint{0.576061in}{0.524170in}}%
\pgfpathlineto{\pgfqpoint{0.576061in}{3.244321in}}%
\pgfusepath{stroke}%
\end{pgfscope}%
\begin{pgfscope}%
\pgfsetrectcap%
\pgfsetmiterjoin%
\pgfsetlinewidth{0.803000pt}%
\definecolor{currentstroke}{rgb}{0.000000,0.000000,0.000000}%
\pgfsetstrokecolor{currentstroke}%
\pgfsetdash{}{0pt}%
\pgfpathmoveto{\pgfqpoint{5.342126in}{0.524170in}}%
\pgfpathlineto{\pgfqpoint{5.342126in}{3.244321in}}%
\pgfusepath{stroke}%
\end{pgfscope}%
\begin{pgfscope}%
\pgfsetrectcap%
\pgfsetmiterjoin%
\pgfsetlinewidth{0.803000pt}%
\definecolor{currentstroke}{rgb}{0.000000,0.000000,0.000000}%
\pgfsetstrokecolor{currentstroke}%
\pgfsetdash{}{0pt}%
\pgfpathmoveto{\pgfqpoint{0.576061in}{0.524170in}}%
\pgfpathlineto{\pgfqpoint{5.342126in}{0.524170in}}%
\pgfusepath{stroke}%
\end{pgfscope}%
\begin{pgfscope}%
\pgfsetrectcap%
\pgfsetmiterjoin%
\pgfsetlinewidth{0.803000pt}%
\definecolor{currentstroke}{rgb}{0.000000,0.000000,0.000000}%
\pgfsetstrokecolor{currentstroke}%
\pgfsetdash{}{0pt}%
\pgfpathmoveto{\pgfqpoint{0.576061in}{3.244321in}}%
\pgfpathlineto{\pgfqpoint{5.342126in}{3.244321in}}%
\pgfusepath{stroke}%
\end{pgfscope}%
\begin{pgfscope}%
\pgfsetbuttcap%
\pgfsetmiterjoin%
\definecolor{currentfill}{rgb}{1.000000,1.000000,1.000000}%
\pgfsetfillcolor{currentfill}%
\pgfsetfillopacity{0.800000}%
\pgfsetlinewidth{1.003750pt}%
\definecolor{currentstroke}{rgb}{0.800000,0.800000,0.800000}%
\pgfsetstrokecolor{currentstroke}%
\pgfsetstrokeopacity{0.800000}%
\pgfsetdash{}{0pt}%
\pgfpathmoveto{\pgfqpoint{4.147682in}{0.579725in}}%
\pgfpathlineto{\pgfqpoint{5.264348in}{0.579725in}}%
\pgfpathquadraticcurveto{\pgfqpoint{5.286571in}{0.579725in}}{\pgfqpoint{5.286571in}{0.601948in}}%
\pgfpathlineto{\pgfqpoint{5.286571in}{1.212058in}}%
\pgfpathquadraticcurveto{\pgfqpoint{5.286571in}{1.234280in}}{\pgfqpoint{5.264348in}{1.234280in}}%
\pgfpathlineto{\pgfqpoint{4.147682in}{1.234280in}}%
\pgfpathquadraticcurveto{\pgfqpoint{4.125459in}{1.234280in}}{\pgfqpoint{4.125459in}{1.212058in}}%
\pgfpathlineto{\pgfqpoint{4.125459in}{0.601948in}}%
\pgfpathquadraticcurveto{\pgfqpoint{4.125459in}{0.579725in}}{\pgfqpoint{4.147682in}{0.579725in}}%
\pgfpathlineto{\pgfqpoint{4.147682in}{0.579725in}}%
\pgfpathclose%
\pgfusepath{stroke,fill}%
\end{pgfscope}%
\begin{pgfscope}%
\pgfsetrectcap%
\pgfsetroundjoin%
\pgfsetlinewidth{1.003750pt}%
\definecolor{currentstroke}{rgb}{0.003922,0.450980,0.698039}%
\pgfsetstrokecolor{currentstroke}%
\pgfsetstrokeopacity{0.700000}%
\pgfsetdash{}{0pt}%
\pgfpathmoveto{\pgfqpoint{4.169904in}{1.150947in}}%
\pgfpathlineto{\pgfqpoint{4.281015in}{1.150947in}}%
\pgfpathlineto{\pgfqpoint{4.392126in}{1.150947in}}%
\pgfusepath{stroke}%
\end{pgfscope}%
\begin{pgfscope}%
\definecolor{textcolor}{rgb}{0.000000,0.000000,0.000000}%
\pgfsetstrokecolor{textcolor}%
\pgfsetfillcolor{textcolor}%
\pgftext[x=4.481015in,y=1.112058in,left,base]{\color{textcolor}\rmfamily\fontsize{8.000000}{9.600000}\selectfont Ziegler-Nichols}%
\end{pgfscope}%
\begin{pgfscope}%
\pgfsetrectcap%
\pgfsetroundjoin%
\pgfsetlinewidth{1.003750pt}%
\definecolor{currentstroke}{rgb}{0.870588,0.560784,0.019608}%
\pgfsetstrokecolor{currentstroke}%
\pgfsetstrokeopacity{0.700000}%
\pgfsetdash{}{0pt}%
\pgfpathmoveto{\pgfqpoint{4.169904in}{0.994836in}}%
\pgfpathlineto{\pgfqpoint{4.281015in}{0.994836in}}%
\pgfpathlineto{\pgfqpoint{4.392126in}{0.994836in}}%
\pgfusepath{stroke}%
\end{pgfscope}%
\begin{pgfscope}%
\definecolor{textcolor}{rgb}{0.000000,0.000000,0.000000}%
\pgfsetstrokecolor{textcolor}%
\pgfsetfillcolor{textcolor}%
\pgftext[x=4.481015in,y=0.955947in,left,base]{\color{textcolor}\rmfamily\fontsize{8.000000}{9.600000}\selectfont SIMC}%
\end{pgfscope}%
\begin{pgfscope}%
\pgfsetrectcap%
\pgfsetroundjoin%
\pgfsetlinewidth{1.003750pt}%
\definecolor{currentstroke}{rgb}{0.800000,0.470588,0.737255}%
\pgfsetstrokecolor{currentstroke}%
\pgfsetstrokeopacity{0.700000}%
\pgfsetdash{}{0pt}%
\pgfpathmoveto{\pgfqpoint{4.169904in}{0.839947in}}%
\pgfpathlineto{\pgfqpoint{4.281015in}{0.839947in}}%
\pgfpathlineto{\pgfqpoint{4.392126in}{0.839947in}}%
\pgfusepath{stroke}%
\end{pgfscope}%
\begin{pgfscope}%
\definecolor{textcolor}{rgb}{0.000000,0.000000,0.000000}%
\pgfsetstrokecolor{textcolor}%
\pgfsetfillcolor{textcolor}%
\pgftext[x=4.481015in,y=0.801058in,left,base]{\color{textcolor}\rmfamily\fontsize{8.000000}{9.600000}\selectfont AMIGO}%
\end{pgfscope}%
\begin{pgfscope}%
\pgfsetbuttcap%
\pgfsetroundjoin%
\pgfsetlinewidth{1.003750pt}%
\definecolor{currentstroke}{rgb}{0.007843,0.619608,0.450980}%
\pgfsetstrokecolor{currentstroke}%
\pgfsetstrokeopacity{0.700000}%
\pgfsetdash{{3.700000pt}{1.600000pt}}{0.000000pt}%
\pgfpathmoveto{\pgfqpoint{4.169904in}{0.684614in}}%
\pgfpathlineto{\pgfqpoint{4.281015in}{0.684614in}}%
\pgfpathlineto{\pgfqpoint{4.392126in}{0.684614in}}%
\pgfusepath{stroke}%
\end{pgfscope}%
\begin{pgfscope}%
\definecolor{textcolor}{rgb}{0.000000,0.000000,0.000000}%
\pgfsetstrokecolor{textcolor}%
\pgfsetfillcolor{textcolor}%
\pgftext[x=4.481015in,y=0.645725in,left,base]{\color{textcolor}\rmfamily\fontsize{8.000000}{9.600000}\selectfont Setpoint}%
\end{pgfscope}%
\end{pgfpicture}%
\makeatother%
\endgroup%

    \caption{Different PI Controllers tuned with parameter derived using the following methods: Ziegler-Nichols, SIMC and AMIGO. The system model is the FOPTD model for room 011.}
    \label{fig:pid_controller_comparison}
\end{figure}

As it can be seen in figure \ref{fig:pid_controller_comparison}, the Ziegler-Nichols tuning rule produces a very aggressive PI controller, that shows quite a bit ringing, which is undesired for this application. The AMIGO rules are rather conservative, but do not produce any overshoot. The SIMC rules have proven the most useful for this application so far. This experience is in line with the results from \citeauthor{liebmann_thesis} \cite{liebmann_thesis}, who tested different PID tuning algorithms for their viability for temperature control in the labs discussed here.

To conclude, several PID tuning rules were presented and using a Python simulation tool it is possible test a set of PID parameters beforehand. Using an example, the different tuning rules were applied to a model for a real lab and the SIMC tuning rules were found to give the best results for this application.

\clearpage
\section{Noise and Allan Deviation}
\label{sec:allan_deviation}
The Allan variance \cite{adev} $\sigma_A^2(\tau)$ is a two-sample variance and used as a measure of stability. The Allan deviation $\sigma_A(\tau)$ is the square root of the variance. Originally, the Allan variance was used to quantify the performance of oscillators, namely the frequency stability, but it can be used evaluate any quantity. In order to define the Allan variance, a few terms need to be defined first. A single measurement value of the time series $y(t)$ can be written as
\begin{equation}
    \bar y_k(t) = \frac{1}{\tau} \int_{t_{k}}^{t_{k}+\tau} y(t)\,dt . \label{eqn:allan_variance_measurement}
\end{equation}
This is the $k$-th measurement with a measurement time or integration time $\tau$. The latter term is frequently used for DMMs. $t_k$ is the start of the $k$-th sampling inverval including the dead time $\theta$
\begin{equation}
    t_{k+1} = t_k + T
\end{equation}
with
\begin{equation}
    T \coloneqq \tau + \theta .
\end{equation}

\begin{figure}[hb]
    \centering
    \scalebox{1}{%
        \import{figures/}{allan_variance_definitions.tex}
    }% scalebox
    \caption{Measurement interval according to equation \ref{eqn:allan_variance_measurement}}
    \label{fig:allan_variance_definitions}
\end{figure}

Using this, the deviation over $N$ samples is defined as \cite{adev,psd_to_adev}
\begin{equation}
    \sigma_y^2(N,T,\tau) = \left\langle \frac{1}{N-1} \left(\sum _{k=0}^{N-1}\bar y_k^2(t)-\frac{1}{N}\left(\sum _{k=0}^{N-1} \bar y_k(t)\right)^2\right)\right\rangle
\end{equation}
The $\langle \; \rangle$ denotes the (infinite time) average over all measurands $y_k$ or, simply put, the expected value.

The Allan variance is a special case of this definition with zero dead-time ($\theta=0$) and only 2 samples:
\begin{align}
    \sigma_A^2(\tau) &= \sigma_A^2(N=2,T=\tau,\tau) \label{eqn:allan_coefficients}\\
    &= \left\langle \frac{\left(\bar y_{k+1} - \bar y_k \right)^2}{2} \right\rangle
\end{align}
It can be shown \cite{psd_to_adev}, that \ref{eqn:adev_estimator} is indeed more useful than $\sigma_A^2(N\to\infty,T=\tau,\tau)$, because $\sigma_A^2(N=2,T=\tau,\tau)$ converges for processes, that do not have a convergent $\sigma_A^2(N\to\infty,T=\tau,\tau)$.

In practice, no experiment can take an infinite number of samples, so typically the Allan variance is estimated using a number of samples $m$:
\begin{equation}
    \sigma_A^2(\tau) \approx \frac1 m \sum_{k=1}^m \frac{\left(\bar y_{k+1} - \bar y_{k} \right)^2}{2} \label{eqn:adev_estimator}
\end{equation}
This esitmation can lead to artifacts in the results as discussed later. In order to derive the Allan variance from a set of data points, the different values of $\tau$ are usually obtained by averaging over a number of samples since there is no dead time.

Additionally, the Allan variance is mathematically related to the two-sided power spectral density $S_y(f)$ \cite{psd_to_adev}:
\begin{equation}
    \sigma_A^2(\tau) = 2 \int_0^\infty S_y(f) \frac{\sin^4\left( \pi f \tau \right)}{(\pi f \tau)^2}\,df \label{eqn:psd_to_adev}
\end{equation}

and therefore all processes, that can be observed in the power spectral density can also be seen in the allan deviation. The inverse transform, however, is not always possible as shown by \citeauthor{inverse_adev} \cite{inverse_adev}.

Distinguishing different noise processes using the Allan deviation will be elaborated in the next section.

\subsection{Identifying Noise in Allan Deviation Plots}
It was already mentioned by \citeauthor{adev} in \cite{adev}, that types of noise, whose spectral density follows a power law
\begin{equation}
    S(f) = h_{\alpha} \cdot f^\alpha \label{eqn:power_law}
\end{equation}
can be easily identified in the Allan deviation plot. The constant $h_\alpha$ is called the power (intensity) coefficient. The most common types of noise encountered in experimental data and their representations can be found in table \ref{tab:adev_alpha}, which serves as a summary of this section. Since those types of noise is present in any measurement or electronic device, it warants a further discussion to understand their root causes and ideas to minimize them. While not a type of noise, linear drift can also be easily identified in the Allan deviation plot. It is therefore included in table \ref{tab:adev_alpha} as well.

\begin{table}[ht]
    \centering
    \begin{tabular}{lcc}
        \toprule
        Amplitude noise type& Power-law coefficient $\alpha$& Allan variance $\sigma_A^2$\\
        \midrule
            White noise & $0$& $\frac 1 2 h_0 \tau^{-1}$ \cite{adev_noise_types}\\
            Flicker noise& $-1$& $2 \ln 2 \, h_{-1} \tau^0$ \cite{adev_noise_types}\\
            Random walk noise& $-2$& $\frac 3 2 \pi^2 h_{-2} \tau^{1}$ \cite{adev_noise_types}\\
            Burst noise& $0 \textrm{ and } -\!2$& $y_{RMS}^2\frac{\bar \tau^2}{\tau^2} \left(4 e^{-\frac{\tau}{\bar \tau}} - e^{-\frac{2 \tau}{\bar \tau}} + 2 \frac{\tau}{\bar \tau} - 3 \right)$\\
            Drift & --& $\frac 1 2 D^2 \tau^2$ \cite{adev_drift}\\
        \bottomrule
    \end{tabular}
    \caption{Power law representations using the Allan variance.}
    \label{tab:adev_alpha}
\end{table}

In order to arrive at a good understanding of the features seen in an Allan deviation plot, this section will provide the reader with examples of each type of noise and the corresponding time domain, power spectral density and Allan deviation plot. Since a complete overview is not available in current literature, all required mathematical descriptions and simulation tools will be discussed here. The simulations were done using Python and the source code is linked to in the discussions.

\clearpage
\subsubsection{White Noise}
White noise is probably the most common type of noise found in measurement data. Johnson noise found in resistors, caused by the random fluctuation of the charge carriers, is one example of mostly white noise up to bandwidth of \qty{100}{\MHz}, from where on quantum corrections are required \cite{nist_johnson_noise}. Amplifiers also tend to have a white noise spectrum at higher frequencies.

For this reason, white noise typically makes up for a considerabe amount of noise in a measurement, unless one works at very low frequencies. White noise is a series of uncorrelated random events and therefore characterised by a uniform power spectral density, which means there is the same power in a given bandwidth at all frequencies up to infinity. White noise therefore has infinite power (variance). In reality a measurement is always limited in bandwidth and hence the above property of a constant power spectral density only holds within that bandwidth. Those bandlimited samples of white noise thus have a finite variance.
Since white noise is so common, a few properties should be mentioned. One such property is, that the variance $\sigma_{x+y}^2$ of two uncorrelated variables $x$ and $y$ adds as:
\begin{equation}
    \sigma_{x+y}^2  = \sigma_x^2 + \sigma_y^2 + \underbrace{2\,\mathrm{Cov}(x,y)}_{\text{uncorrelated}\, =\, 0}\ = \sigma_x^2 + \sigma_y^2 \label{eqn:adding_white_noise}
\end{equation}

This allows simple addition rules of variances from different sources, but it must be stressed here, that this property is only valid for uncorrelated sources like white noise, although it is usually incorrectly applied to all measurements in disreagard of the dominant noise present, which unfortunately obscures rather than clarifies the uncertainties involved.

In order to demonstrate the effect of white noise in Allan deviation plots, it was simulated using the excellent \textit{AllanTools} library \cite{allantools}. The noise generator chosen in the AllanTools library is based on the work of \citeauthor{noise_generation} \cite{noise_generation}. The full Python program code is published online \cite{}. For better comparison, all noise densities are normalized to give an Allan deviation of $\sigma_A(\tau_0)=1$, with $\tau_0$ being the smallest time interval between measurements.

Figure \ref{fig:white_noise_simulated} shows a sample of white noise in three different forms. Figure \ref{fig:white_noise_time} is the time series representation. From this sample, the power spectral density was calculated and is shown in figure \ref{fig:white_noise_psd}. The dashed line shows the expectation value of the power spectral density and the Allan deviation.

\begin{figure}[ht]
    \centering
    \begin{subfigure}{0.32\linewidth}
        \centering
        \scalebox{0.75}{%
            %% Creator: Matplotlib, PGF backend
%%
%% To include the figure in your LaTeX document, write
%%   \input{<filename>.pgf}
%%
%% Make sure the required packages are loaded in your preamble
%%   \usepackage{pgf}
%%
%% Also ensure that all the required font packages are loaded; for instance,
%% the lmodern package is sometimes necessary when using math font.
%%   \usepackage{lmodern}
%%
%% Figures using additional raster images can only be included by \input if
%% they are in the same directory as the main LaTeX file. For loading figures
%% from other directories you can use the `import` package
%%   \usepackage{import}
%%
%% and then include the figures with
%%   \import{<path to file>}{<filename>.pgf}
%%
%% Matplotlib used the following preamble
%%   \usepackage{siunitx}
%%   \usepackage{fontspec}
%%
\begingroup%
\makeatletter%
\begin{pgfpicture}%
\pgfpathrectangle{\pgfpointorigin}{\pgfqpoint{2.390000in}{1.830000in}}%
\pgfusepath{use as bounding box, clip}%
\begin{pgfscope}%
\pgfsetbuttcap%
\pgfsetmiterjoin%
\definecolor{currentfill}{rgb}{1.000000,1.000000,1.000000}%
\pgfsetfillcolor{currentfill}%
\pgfsetlinewidth{0.000000pt}%
\definecolor{currentstroke}{rgb}{1.000000,1.000000,1.000000}%
\pgfsetstrokecolor{currentstroke}%
\pgfsetdash{}{0pt}%
\pgfpathmoveto{\pgfqpoint{0.000000in}{0.000000in}}%
\pgfpathlineto{\pgfqpoint{2.390000in}{0.000000in}}%
\pgfpathlineto{\pgfqpoint{2.390000in}{1.830000in}}%
\pgfpathlineto{\pgfqpoint{0.000000in}{1.830000in}}%
\pgfpathlineto{\pgfqpoint{0.000000in}{0.000000in}}%
\pgfpathclose%
\pgfusepath{fill}%
\end{pgfscope}%
\begin{pgfscope}%
\pgfsetbuttcap%
\pgfsetmiterjoin%
\definecolor{currentfill}{rgb}{1.000000,1.000000,1.000000}%
\pgfsetfillcolor{currentfill}%
\pgfsetlinewidth{0.000000pt}%
\definecolor{currentstroke}{rgb}{0.000000,0.000000,0.000000}%
\pgfsetstrokecolor{currentstroke}%
\pgfsetstrokeopacity{0.000000}%
\pgfsetdash{}{0pt}%
\pgfpathmoveto{\pgfqpoint{0.471688in}{0.416447in}}%
\pgfpathlineto{\pgfqpoint{2.348330in}{0.416447in}}%
\pgfpathlineto{\pgfqpoint{2.348330in}{1.773646in}}%
\pgfpathlineto{\pgfqpoint{0.471688in}{1.773646in}}%
\pgfpathlineto{\pgfqpoint{0.471688in}{0.416447in}}%
\pgfpathclose%
\pgfusepath{fill}%
\end{pgfscope}%
\begin{pgfscope}%
\pgfpathrectangle{\pgfqpoint{0.471688in}{0.416447in}}{\pgfqpoint{1.876642in}{1.357199in}}%
\pgfusepath{clip}%
\pgfsetrectcap%
\pgfsetroundjoin%
\pgfsetlinewidth{0.803000pt}%
\definecolor{currentstroke}{rgb}{0.450000,0.450000,0.450000}%
\pgfsetstrokecolor{currentstroke}%
\pgfsetdash{}{0pt}%
\pgfpathmoveto{\pgfqpoint{0.556989in}{0.416447in}}%
\pgfpathlineto{\pgfqpoint{0.556989in}{1.773646in}}%
\pgfusepath{stroke}%
\end{pgfscope}%
\begin{pgfscope}%
\pgfsetbuttcap%
\pgfsetroundjoin%
\definecolor{currentfill}{rgb}{0.000000,0.000000,0.000000}%
\pgfsetfillcolor{currentfill}%
\pgfsetlinewidth{0.803000pt}%
\definecolor{currentstroke}{rgb}{0.000000,0.000000,0.000000}%
\pgfsetstrokecolor{currentstroke}%
\pgfsetdash{}{0pt}%
\pgfsys@defobject{currentmarker}{\pgfqpoint{0.000000in}{-0.048611in}}{\pgfqpoint{0.000000in}{0.000000in}}{%
\pgfpathmoveto{\pgfqpoint{0.000000in}{0.000000in}}%
\pgfpathlineto{\pgfqpoint{0.000000in}{-0.048611in}}%
\pgfusepath{stroke,fill}%
}%
\begin{pgfscope}%
\pgfsys@transformshift{0.556989in}{0.416447in}%
\pgfsys@useobject{currentmarker}{}%
\end{pgfscope}%
\end{pgfscope}%
\begin{pgfscope}%
\definecolor{textcolor}{rgb}{0.000000,0.000000,0.000000}%
\pgfsetstrokecolor{textcolor}%
\pgfsetfillcolor{textcolor}%
\pgftext[x=0.556989in,y=0.319225in,,top]{\color{textcolor}\rmfamily\fontsize{8.000000}{9.600000}\selectfont \(\displaystyle {0}\)}%
\end{pgfscope}%
\begin{pgfscope}%
\pgfpathrectangle{\pgfqpoint{0.471688in}{0.416447in}}{\pgfqpoint{1.876642in}{1.357199in}}%
\pgfusepath{clip}%
\pgfsetrectcap%
\pgfsetroundjoin%
\pgfsetlinewidth{0.803000pt}%
\definecolor{currentstroke}{rgb}{0.450000,0.450000,0.450000}%
\pgfsetstrokecolor{currentstroke}%
\pgfsetdash{}{0pt}%
\pgfpathmoveto{\pgfqpoint{1.077695in}{0.416447in}}%
\pgfpathlineto{\pgfqpoint{1.077695in}{1.773646in}}%
\pgfusepath{stroke}%
\end{pgfscope}%
\begin{pgfscope}%
\pgfsetbuttcap%
\pgfsetroundjoin%
\definecolor{currentfill}{rgb}{0.000000,0.000000,0.000000}%
\pgfsetfillcolor{currentfill}%
\pgfsetlinewidth{0.803000pt}%
\definecolor{currentstroke}{rgb}{0.000000,0.000000,0.000000}%
\pgfsetstrokecolor{currentstroke}%
\pgfsetdash{}{0pt}%
\pgfsys@defobject{currentmarker}{\pgfqpoint{0.000000in}{-0.048611in}}{\pgfqpoint{0.000000in}{0.000000in}}{%
\pgfpathmoveto{\pgfqpoint{0.000000in}{0.000000in}}%
\pgfpathlineto{\pgfqpoint{0.000000in}{-0.048611in}}%
\pgfusepath{stroke,fill}%
}%
\begin{pgfscope}%
\pgfsys@transformshift{1.077695in}{0.416447in}%
\pgfsys@useobject{currentmarker}{}%
\end{pgfscope}%
\end{pgfscope}%
\begin{pgfscope}%
\definecolor{textcolor}{rgb}{0.000000,0.000000,0.000000}%
\pgfsetstrokecolor{textcolor}%
\pgfsetfillcolor{textcolor}%
\pgftext[x=1.077695in,y=0.319225in,,top]{\color{textcolor}\rmfamily\fontsize{8.000000}{9.600000}\selectfont \(\displaystyle {5000}\)}%
\end{pgfscope}%
\begin{pgfscope}%
\pgfpathrectangle{\pgfqpoint{0.471688in}{0.416447in}}{\pgfqpoint{1.876642in}{1.357199in}}%
\pgfusepath{clip}%
\pgfsetrectcap%
\pgfsetroundjoin%
\pgfsetlinewidth{0.803000pt}%
\definecolor{currentstroke}{rgb}{0.450000,0.450000,0.450000}%
\pgfsetstrokecolor{currentstroke}%
\pgfsetdash{}{0pt}%
\pgfpathmoveto{\pgfqpoint{1.598400in}{0.416447in}}%
\pgfpathlineto{\pgfqpoint{1.598400in}{1.773646in}}%
\pgfusepath{stroke}%
\end{pgfscope}%
\begin{pgfscope}%
\pgfsetbuttcap%
\pgfsetroundjoin%
\definecolor{currentfill}{rgb}{0.000000,0.000000,0.000000}%
\pgfsetfillcolor{currentfill}%
\pgfsetlinewidth{0.803000pt}%
\definecolor{currentstroke}{rgb}{0.000000,0.000000,0.000000}%
\pgfsetstrokecolor{currentstroke}%
\pgfsetdash{}{0pt}%
\pgfsys@defobject{currentmarker}{\pgfqpoint{0.000000in}{-0.048611in}}{\pgfqpoint{0.000000in}{0.000000in}}{%
\pgfpathmoveto{\pgfqpoint{0.000000in}{0.000000in}}%
\pgfpathlineto{\pgfqpoint{0.000000in}{-0.048611in}}%
\pgfusepath{stroke,fill}%
}%
\begin{pgfscope}%
\pgfsys@transformshift{1.598400in}{0.416447in}%
\pgfsys@useobject{currentmarker}{}%
\end{pgfscope}%
\end{pgfscope}%
\begin{pgfscope}%
\definecolor{textcolor}{rgb}{0.000000,0.000000,0.000000}%
\pgfsetstrokecolor{textcolor}%
\pgfsetfillcolor{textcolor}%
\pgftext[x=1.598400in,y=0.319225in,,top]{\color{textcolor}\rmfamily\fontsize{8.000000}{9.600000}\selectfont \(\displaystyle {10000}\)}%
\end{pgfscope}%
\begin{pgfscope}%
\pgfpathrectangle{\pgfqpoint{0.471688in}{0.416447in}}{\pgfqpoint{1.876642in}{1.357199in}}%
\pgfusepath{clip}%
\pgfsetrectcap%
\pgfsetroundjoin%
\pgfsetlinewidth{0.803000pt}%
\definecolor{currentstroke}{rgb}{0.450000,0.450000,0.450000}%
\pgfsetstrokecolor{currentstroke}%
\pgfsetdash{}{0pt}%
\pgfpathmoveto{\pgfqpoint{2.119105in}{0.416447in}}%
\pgfpathlineto{\pgfqpoint{2.119105in}{1.773646in}}%
\pgfusepath{stroke}%
\end{pgfscope}%
\begin{pgfscope}%
\pgfsetbuttcap%
\pgfsetroundjoin%
\definecolor{currentfill}{rgb}{0.000000,0.000000,0.000000}%
\pgfsetfillcolor{currentfill}%
\pgfsetlinewidth{0.803000pt}%
\definecolor{currentstroke}{rgb}{0.000000,0.000000,0.000000}%
\pgfsetstrokecolor{currentstroke}%
\pgfsetdash{}{0pt}%
\pgfsys@defobject{currentmarker}{\pgfqpoint{0.000000in}{-0.048611in}}{\pgfqpoint{0.000000in}{0.000000in}}{%
\pgfpathmoveto{\pgfqpoint{0.000000in}{0.000000in}}%
\pgfpathlineto{\pgfqpoint{0.000000in}{-0.048611in}}%
\pgfusepath{stroke,fill}%
}%
\begin{pgfscope}%
\pgfsys@transformshift{2.119105in}{0.416447in}%
\pgfsys@useobject{currentmarker}{}%
\end{pgfscope}%
\end{pgfscope}%
\begin{pgfscope}%
\definecolor{textcolor}{rgb}{0.000000,0.000000,0.000000}%
\pgfsetstrokecolor{textcolor}%
\pgfsetfillcolor{textcolor}%
\pgftext[x=2.119105in,y=0.319225in,,top]{\color{textcolor}\rmfamily\fontsize{8.000000}{9.600000}\selectfont \(\displaystyle {15000}\)}%
\end{pgfscope}%
\begin{pgfscope}%
\definecolor{textcolor}{rgb}{0.000000,0.000000,0.000000}%
\pgfsetstrokecolor{textcolor}%
\pgfsetfillcolor{textcolor}%
\pgftext[x=1.410009in,y=0.165003in,,top]{\color{textcolor}\rmfamily\fontsize{10.000000}{12.000000}\selectfont Time in \unit{s}}%
\end{pgfscope}%
\begin{pgfscope}%
\pgfpathrectangle{\pgfqpoint{0.471688in}{0.416447in}}{\pgfqpoint{1.876642in}{1.357199in}}%
\pgfusepath{clip}%
\pgfsetrectcap%
\pgfsetroundjoin%
\pgfsetlinewidth{0.803000pt}%
\definecolor{currentstroke}{rgb}{0.450000,0.450000,0.450000}%
\pgfsetstrokecolor{currentstroke}%
\pgfsetdash{}{0pt}%
\pgfpathmoveto{\pgfqpoint{0.471688in}{0.466742in}}%
\pgfpathlineto{\pgfqpoint{2.348330in}{0.466742in}}%
\pgfusepath{stroke}%
\end{pgfscope}%
\begin{pgfscope}%
\pgfsetbuttcap%
\pgfsetroundjoin%
\definecolor{currentfill}{rgb}{0.000000,0.000000,0.000000}%
\pgfsetfillcolor{currentfill}%
\pgfsetlinewidth{0.803000pt}%
\definecolor{currentstroke}{rgb}{0.000000,0.000000,0.000000}%
\pgfsetstrokecolor{currentstroke}%
\pgfsetdash{}{0pt}%
\pgfsys@defobject{currentmarker}{\pgfqpoint{-0.048611in}{0.000000in}}{\pgfqpoint{-0.000000in}{0.000000in}}{%
\pgfpathmoveto{\pgfqpoint{-0.000000in}{0.000000in}}%
\pgfpathlineto{\pgfqpoint{-0.048611in}{0.000000in}}%
\pgfusepath{stroke,fill}%
}%
\begin{pgfscope}%
\pgfsys@transformshift{0.471688in}{0.466742in}%
\pgfsys@useobject{currentmarker}{}%
\end{pgfscope}%
\end{pgfscope}%
\begin{pgfscope}%
\definecolor{textcolor}{rgb}{0.000000,0.000000,0.000000}%
\pgfsetstrokecolor{textcolor}%
\pgfsetfillcolor{textcolor}%
\pgftext[x=0.223614in, y=0.428187in, left, base]{\color{textcolor}\rmfamily\fontsize{8.000000}{9.600000}\selectfont \(\displaystyle {\ensuremath{-}4}\)}%
\end{pgfscope}%
\begin{pgfscope}%
\pgfpathrectangle{\pgfqpoint{0.471688in}{0.416447in}}{\pgfqpoint{1.876642in}{1.357199in}}%
\pgfusepath{clip}%
\pgfsetrectcap%
\pgfsetroundjoin%
\pgfsetlinewidth{0.803000pt}%
\definecolor{currentstroke}{rgb}{0.450000,0.450000,0.450000}%
\pgfsetstrokecolor{currentstroke}%
\pgfsetdash{}{0pt}%
\pgfpathmoveto{\pgfqpoint{0.471688in}{0.760456in}}%
\pgfpathlineto{\pgfqpoint{2.348330in}{0.760456in}}%
\pgfusepath{stroke}%
\end{pgfscope}%
\begin{pgfscope}%
\pgfsetbuttcap%
\pgfsetroundjoin%
\definecolor{currentfill}{rgb}{0.000000,0.000000,0.000000}%
\pgfsetfillcolor{currentfill}%
\pgfsetlinewidth{0.803000pt}%
\definecolor{currentstroke}{rgb}{0.000000,0.000000,0.000000}%
\pgfsetstrokecolor{currentstroke}%
\pgfsetdash{}{0pt}%
\pgfsys@defobject{currentmarker}{\pgfqpoint{-0.048611in}{0.000000in}}{\pgfqpoint{-0.000000in}{0.000000in}}{%
\pgfpathmoveto{\pgfqpoint{-0.000000in}{0.000000in}}%
\pgfpathlineto{\pgfqpoint{-0.048611in}{0.000000in}}%
\pgfusepath{stroke,fill}%
}%
\begin{pgfscope}%
\pgfsys@transformshift{0.471688in}{0.760456in}%
\pgfsys@useobject{currentmarker}{}%
\end{pgfscope}%
\end{pgfscope}%
\begin{pgfscope}%
\definecolor{textcolor}{rgb}{0.000000,0.000000,0.000000}%
\pgfsetstrokecolor{textcolor}%
\pgfsetfillcolor{textcolor}%
\pgftext[x=0.223614in, y=0.721901in, left, base]{\color{textcolor}\rmfamily\fontsize{8.000000}{9.600000}\selectfont \(\displaystyle {\ensuremath{-}2}\)}%
\end{pgfscope}%
\begin{pgfscope}%
\pgfpathrectangle{\pgfqpoint{0.471688in}{0.416447in}}{\pgfqpoint{1.876642in}{1.357199in}}%
\pgfusepath{clip}%
\pgfsetrectcap%
\pgfsetroundjoin%
\pgfsetlinewidth{0.803000pt}%
\definecolor{currentstroke}{rgb}{0.450000,0.450000,0.450000}%
\pgfsetstrokecolor{currentstroke}%
\pgfsetdash{}{0pt}%
\pgfpathmoveto{\pgfqpoint{0.471688in}{1.054170in}}%
\pgfpathlineto{\pgfqpoint{2.348330in}{1.054170in}}%
\pgfusepath{stroke}%
\end{pgfscope}%
\begin{pgfscope}%
\pgfsetbuttcap%
\pgfsetroundjoin%
\definecolor{currentfill}{rgb}{0.000000,0.000000,0.000000}%
\pgfsetfillcolor{currentfill}%
\pgfsetlinewidth{0.803000pt}%
\definecolor{currentstroke}{rgb}{0.000000,0.000000,0.000000}%
\pgfsetstrokecolor{currentstroke}%
\pgfsetdash{}{0pt}%
\pgfsys@defobject{currentmarker}{\pgfqpoint{-0.048611in}{0.000000in}}{\pgfqpoint{-0.000000in}{0.000000in}}{%
\pgfpathmoveto{\pgfqpoint{-0.000000in}{0.000000in}}%
\pgfpathlineto{\pgfqpoint{-0.048611in}{0.000000in}}%
\pgfusepath{stroke,fill}%
}%
\begin{pgfscope}%
\pgfsys@transformshift{0.471688in}{1.054170in}%
\pgfsys@useobject{currentmarker}{}%
\end{pgfscope}%
\end{pgfscope}%
\begin{pgfscope}%
\definecolor{textcolor}{rgb}{0.000000,0.000000,0.000000}%
\pgfsetstrokecolor{textcolor}%
\pgfsetfillcolor{textcolor}%
\pgftext[x=0.315437in, y=1.015615in, left, base]{\color{textcolor}\rmfamily\fontsize{8.000000}{9.600000}\selectfont \(\displaystyle {0}\)}%
\end{pgfscope}%
\begin{pgfscope}%
\pgfpathrectangle{\pgfqpoint{0.471688in}{0.416447in}}{\pgfqpoint{1.876642in}{1.357199in}}%
\pgfusepath{clip}%
\pgfsetrectcap%
\pgfsetroundjoin%
\pgfsetlinewidth{0.803000pt}%
\definecolor{currentstroke}{rgb}{0.450000,0.450000,0.450000}%
\pgfsetstrokecolor{currentstroke}%
\pgfsetdash{}{0pt}%
\pgfpathmoveto{\pgfqpoint{0.471688in}{1.347884in}}%
\pgfpathlineto{\pgfqpoint{2.348330in}{1.347884in}}%
\pgfusepath{stroke}%
\end{pgfscope}%
\begin{pgfscope}%
\pgfsetbuttcap%
\pgfsetroundjoin%
\definecolor{currentfill}{rgb}{0.000000,0.000000,0.000000}%
\pgfsetfillcolor{currentfill}%
\pgfsetlinewidth{0.803000pt}%
\definecolor{currentstroke}{rgb}{0.000000,0.000000,0.000000}%
\pgfsetstrokecolor{currentstroke}%
\pgfsetdash{}{0pt}%
\pgfsys@defobject{currentmarker}{\pgfqpoint{-0.048611in}{0.000000in}}{\pgfqpoint{-0.000000in}{0.000000in}}{%
\pgfpathmoveto{\pgfqpoint{-0.000000in}{0.000000in}}%
\pgfpathlineto{\pgfqpoint{-0.048611in}{0.000000in}}%
\pgfusepath{stroke,fill}%
}%
\begin{pgfscope}%
\pgfsys@transformshift{0.471688in}{1.347884in}%
\pgfsys@useobject{currentmarker}{}%
\end{pgfscope}%
\end{pgfscope}%
\begin{pgfscope}%
\definecolor{textcolor}{rgb}{0.000000,0.000000,0.000000}%
\pgfsetstrokecolor{textcolor}%
\pgfsetfillcolor{textcolor}%
\pgftext[x=0.315437in, y=1.309329in, left, base]{\color{textcolor}\rmfamily\fontsize{8.000000}{9.600000}\selectfont \(\displaystyle {2}\)}%
\end{pgfscope}%
\begin{pgfscope}%
\pgfpathrectangle{\pgfqpoint{0.471688in}{0.416447in}}{\pgfqpoint{1.876642in}{1.357199in}}%
\pgfusepath{clip}%
\pgfsetrectcap%
\pgfsetroundjoin%
\pgfsetlinewidth{0.803000pt}%
\definecolor{currentstroke}{rgb}{0.450000,0.450000,0.450000}%
\pgfsetstrokecolor{currentstroke}%
\pgfsetdash{}{0pt}%
\pgfpathmoveto{\pgfqpoint{0.471688in}{1.641598in}}%
\pgfpathlineto{\pgfqpoint{2.348330in}{1.641598in}}%
\pgfusepath{stroke}%
\end{pgfscope}%
\begin{pgfscope}%
\pgfsetbuttcap%
\pgfsetroundjoin%
\definecolor{currentfill}{rgb}{0.000000,0.000000,0.000000}%
\pgfsetfillcolor{currentfill}%
\pgfsetlinewidth{0.803000pt}%
\definecolor{currentstroke}{rgb}{0.000000,0.000000,0.000000}%
\pgfsetstrokecolor{currentstroke}%
\pgfsetdash{}{0pt}%
\pgfsys@defobject{currentmarker}{\pgfqpoint{-0.048611in}{0.000000in}}{\pgfqpoint{-0.000000in}{0.000000in}}{%
\pgfpathmoveto{\pgfqpoint{-0.000000in}{0.000000in}}%
\pgfpathlineto{\pgfqpoint{-0.048611in}{0.000000in}}%
\pgfusepath{stroke,fill}%
}%
\begin{pgfscope}%
\pgfsys@transformshift{0.471688in}{1.641598in}%
\pgfsys@useobject{currentmarker}{}%
\end{pgfscope}%
\end{pgfscope}%
\begin{pgfscope}%
\definecolor{textcolor}{rgb}{0.000000,0.000000,0.000000}%
\pgfsetstrokecolor{textcolor}%
\pgfsetfillcolor{textcolor}%
\pgftext[x=0.315437in, y=1.603043in, left, base]{\color{textcolor}\rmfamily\fontsize{8.000000}{9.600000}\selectfont \(\displaystyle {4}\)}%
\end{pgfscope}%
\begin{pgfscope}%
\definecolor{textcolor}{rgb}{0.000000,0.000000,0.000000}%
\pgfsetstrokecolor{textcolor}%
\pgfsetfillcolor{textcolor}%
\pgftext[x=0.168059in,y=1.095047in,,bottom,rotate=90.000000]{\color{textcolor}\rmfamily\fontsize{10.000000}{12.000000}\selectfont Amplitude in arb. unit}%
\end{pgfscope}%
\begin{pgfscope}%
\pgfpathrectangle{\pgfqpoint{0.471688in}{0.416447in}}{\pgfqpoint{1.876642in}{1.357199in}}%
\pgfusepath{clip}%
\pgfsetrectcap%
\pgfsetroundjoin%
\pgfsetlinewidth{1.505625pt}%
\definecolor{currentstroke}{rgb}{0.000000,0.447059,0.698039}%
\pgfsetstrokecolor{currentstroke}%
\pgfsetdash{}{0pt}%
\pgfpathmoveto{\pgfqpoint{0.556989in}{1.033865in}}%
\pgfpathlineto{\pgfqpoint{0.557510in}{1.286089in}}%
\pgfpathlineto{\pgfqpoint{0.557718in}{0.985225in}}%
\pgfpathlineto{\pgfqpoint{0.558135in}{1.089704in}}%
\pgfpathlineto{\pgfqpoint{0.558239in}{0.773192in}}%
\pgfpathlineto{\pgfqpoint{0.558968in}{1.269411in}}%
\pgfpathlineto{\pgfqpoint{0.559281in}{0.844936in}}%
\pgfpathlineto{\pgfqpoint{0.560114in}{1.326190in}}%
\pgfpathlineto{\pgfqpoint{0.560426in}{1.174967in}}%
\pgfpathlineto{\pgfqpoint{0.560739in}{0.766379in}}%
\pgfpathlineto{\pgfqpoint{0.561572in}{0.948456in}}%
\pgfpathlineto{\pgfqpoint{0.561780in}{1.209416in}}%
\pgfpathlineto{\pgfqpoint{0.561988in}{0.795256in}}%
\pgfpathlineto{\pgfqpoint{0.562613in}{1.190935in}}%
\pgfpathlineto{\pgfqpoint{0.563446in}{0.878499in}}%
\pgfpathlineto{\pgfqpoint{0.563654in}{1.253344in}}%
\pgfpathlineto{\pgfqpoint{0.563759in}{1.043595in}}%
\pgfpathlineto{\pgfqpoint{0.564488in}{1.283949in}}%
\pgfpathlineto{\pgfqpoint{0.564592in}{0.669442in}}%
\pgfpathlineto{\pgfqpoint{0.564800in}{1.066954in}}%
\pgfpathlineto{\pgfqpoint{0.565112in}{0.762282in}}%
\pgfpathlineto{\pgfqpoint{0.565425in}{1.271209in}}%
\pgfpathlineto{\pgfqpoint{0.565737in}{0.980484in}}%
\pgfpathlineto{\pgfqpoint{0.566362in}{1.196423in}}%
\pgfpathlineto{\pgfqpoint{0.566466in}{0.951069in}}%
\pgfpathlineto{\pgfqpoint{0.566675in}{0.996587in}}%
\pgfpathlineto{\pgfqpoint{0.566779in}{0.839243in}}%
\pgfpathlineto{\pgfqpoint{0.566883in}{1.097658in}}%
\pgfpathlineto{\pgfqpoint{0.567716in}{1.030484in}}%
\pgfpathlineto{\pgfqpoint{0.568653in}{1.415915in}}%
\pgfpathlineto{\pgfqpoint{0.568341in}{0.772385in}}%
\pgfpathlineto{\pgfqpoint{0.568862in}{1.098455in}}%
\pgfpathlineto{\pgfqpoint{0.569695in}{0.848299in}}%
\pgfpathlineto{\pgfqpoint{0.569590in}{1.260181in}}%
\pgfpathlineto{\pgfqpoint{0.569799in}{1.140354in}}%
\pgfpathlineto{\pgfqpoint{0.569903in}{1.375854in}}%
\pgfpathlineto{\pgfqpoint{0.570424in}{0.826445in}}%
\pgfpathlineto{\pgfqpoint{0.570736in}{1.123721in}}%
\pgfpathlineto{\pgfqpoint{0.571673in}{0.818100in}}%
\pgfpathlineto{\pgfqpoint{0.570944in}{1.281789in}}%
\pgfpathlineto{\pgfqpoint{0.571777in}{1.081285in}}%
\pgfpathlineto{\pgfqpoint{0.571986in}{1.168987in}}%
\pgfpathlineto{\pgfqpoint{0.572194in}{0.860252in}}%
\pgfpathlineto{\pgfqpoint{0.573027in}{0.949263in}}%
\pgfpathlineto{\pgfqpoint{0.573131in}{1.328172in}}%
\pgfpathlineto{\pgfqpoint{0.573340in}{0.879219in}}%
\pgfpathlineto{\pgfqpoint{0.574173in}{1.174896in}}%
\pgfpathlineto{\pgfqpoint{0.574277in}{1.332728in}}%
\pgfpathlineto{\pgfqpoint{0.574589in}{0.923539in}}%
\pgfpathlineto{\pgfqpoint{0.575110in}{1.175648in}}%
\pgfpathlineto{\pgfqpoint{0.575839in}{0.896902in}}%
\pgfpathlineto{\pgfqpoint{0.575527in}{1.453646in}}%
\pgfpathlineto{\pgfqpoint{0.576256in}{1.123669in}}%
\pgfpathlineto{\pgfqpoint{0.576568in}{0.831704in}}%
\pgfpathlineto{\pgfqpoint{0.576776in}{1.179939in}}%
\pgfpathlineto{\pgfqpoint{0.577401in}{1.076746in}}%
\pgfpathlineto{\pgfqpoint{0.577505in}{1.062719in}}%
\pgfpathlineto{\pgfqpoint{0.578130in}{0.851850in}}%
\pgfpathlineto{\pgfqpoint{0.577922in}{1.213224in}}%
\pgfpathlineto{\pgfqpoint{0.578547in}{1.129809in}}%
\pgfpathlineto{\pgfqpoint{0.578651in}{1.619971in}}%
\pgfpathlineto{\pgfqpoint{0.579380in}{0.940676in}}%
\pgfpathlineto{\pgfqpoint{0.579588in}{0.982891in}}%
\pgfpathlineto{\pgfqpoint{0.579796in}{1.394094in}}%
\pgfpathlineto{\pgfqpoint{0.579900in}{0.779949in}}%
\pgfpathlineto{\pgfqpoint{0.580734in}{1.153974in}}%
\pgfpathlineto{\pgfqpoint{0.581463in}{0.756764in}}%
\pgfpathlineto{\pgfqpoint{0.581254in}{1.369024in}}%
\pgfpathlineto{\pgfqpoint{0.581879in}{0.937783in}}%
\pgfpathlineto{\pgfqpoint{0.582712in}{1.313440in}}%
\pgfpathlineto{\pgfqpoint{0.582296in}{0.877898in}}%
\pgfpathlineto{\pgfqpoint{0.582816in}{1.113645in}}%
\pgfpathlineto{\pgfqpoint{0.583337in}{0.831040in}}%
\pgfpathlineto{\pgfqpoint{0.583129in}{1.365824in}}%
\pgfpathlineto{\pgfqpoint{0.583962in}{0.918044in}}%
\pgfpathlineto{\pgfqpoint{0.584587in}{1.293901in}}%
\pgfpathlineto{\pgfqpoint{0.584170in}{0.578167in}}%
\pgfpathlineto{\pgfqpoint{0.585003in}{1.265831in}}%
\pgfpathlineto{\pgfqpoint{0.585108in}{0.843304in}}%
\pgfpathlineto{\pgfqpoint{0.586045in}{1.070841in}}%
\pgfpathlineto{\pgfqpoint{0.586461in}{1.367421in}}%
\pgfpathlineto{\pgfqpoint{0.586565in}{0.767493in}}%
\pgfpathlineto{\pgfqpoint{0.587086in}{1.023606in}}%
\pgfpathlineto{\pgfqpoint{0.587607in}{0.952412in}}%
\pgfpathlineto{\pgfqpoint{0.587711in}{1.186283in}}%
\pgfpathlineto{\pgfqpoint{0.587815in}{1.099299in}}%
\pgfpathlineto{\pgfqpoint{0.588128in}{0.932427in}}%
\pgfpathlineto{\pgfqpoint{0.588752in}{1.241804in}}%
\pgfpathlineto{\pgfqpoint{0.588857in}{0.967294in}}%
\pgfpathlineto{\pgfqpoint{0.589586in}{1.245889in}}%
\pgfpathlineto{\pgfqpoint{0.589898in}{1.008606in}}%
\pgfpathlineto{\pgfqpoint{0.590523in}{1.361452in}}%
\pgfpathlineto{\pgfqpoint{0.590731in}{0.875858in}}%
\pgfpathlineto{\pgfqpoint{0.591148in}{1.146447in}}%
\pgfpathlineto{\pgfqpoint{0.591356in}{0.922402in}}%
\pgfpathlineto{\pgfqpoint{0.591668in}{1.197374in}}%
\pgfpathlineto{\pgfqpoint{0.592293in}{0.933422in}}%
\pgfpathlineto{\pgfqpoint{0.593439in}{1.270837in}}%
\pgfpathlineto{\pgfqpoint{0.592918in}{0.841509in}}%
\pgfpathlineto{\pgfqpoint{0.593543in}{1.180124in}}%
\pgfpathlineto{\pgfqpoint{0.593855in}{0.906942in}}%
\pgfpathlineto{\pgfqpoint{0.594480in}{1.279264in}}%
\pgfpathlineto{\pgfqpoint{0.594688in}{1.113165in}}%
\pgfpathlineto{\pgfqpoint{0.595313in}{0.940648in}}%
\pgfpathlineto{\pgfqpoint{0.595834in}{1.370380in}}%
\pgfpathlineto{\pgfqpoint{0.596667in}{0.742261in}}%
\pgfpathlineto{\pgfqpoint{0.596251in}{1.375758in}}%
\pgfpathlineto{\pgfqpoint{0.596980in}{1.076257in}}%
\pgfpathlineto{\pgfqpoint{0.597188in}{1.329699in}}%
\pgfpathlineto{\pgfqpoint{0.597500in}{0.922232in}}%
\pgfpathlineto{\pgfqpoint{0.597604in}{1.126412in}}%
\pgfpathlineto{\pgfqpoint{0.598125in}{0.802585in}}%
\pgfpathlineto{\pgfqpoint{0.597813in}{1.323133in}}%
\pgfpathlineto{\pgfqpoint{0.598646in}{0.966148in}}%
\pgfpathlineto{\pgfqpoint{0.599167in}{0.897383in}}%
\pgfpathlineto{\pgfqpoint{0.599583in}{1.158676in}}%
\pgfpathlineto{\pgfqpoint{0.599791in}{0.828875in}}%
\pgfpathlineto{\pgfqpoint{0.600208in}{1.281968in}}%
\pgfpathlineto{\pgfqpoint{0.600520in}{1.064086in}}%
\pgfpathlineto{\pgfqpoint{0.600625in}{1.356806in}}%
\pgfpathlineto{\pgfqpoint{0.601249in}{0.912465in}}%
\pgfpathlineto{\pgfqpoint{0.601458in}{1.209607in}}%
\pgfpathlineto{\pgfqpoint{0.601770in}{0.754695in}}%
\pgfpathlineto{\pgfqpoint{0.602291in}{1.293344in}}%
\pgfpathlineto{\pgfqpoint{0.602603in}{1.046013in}}%
\pgfpathlineto{\pgfqpoint{0.603228in}{1.152516in}}%
\pgfpathlineto{\pgfqpoint{0.602916in}{0.750552in}}%
\pgfpathlineto{\pgfqpoint{0.603540in}{0.978705in}}%
\pgfpathlineto{\pgfqpoint{0.603853in}{1.194440in}}%
\pgfpathlineto{\pgfqpoint{0.604686in}{0.878254in}}%
\pgfpathlineto{\pgfqpoint{0.604790in}{1.342704in}}%
\pgfpathlineto{\pgfqpoint{0.605727in}{0.969607in}}%
\pgfpathlineto{\pgfqpoint{0.606665in}{1.506326in}}%
\pgfpathlineto{\pgfqpoint{0.605936in}{0.716117in}}%
\pgfpathlineto{\pgfqpoint{0.606873in}{1.035385in}}%
\pgfpathlineto{\pgfqpoint{0.607081in}{0.818252in}}%
\pgfpathlineto{\pgfqpoint{0.607706in}{1.301939in}}%
\pgfpathlineto{\pgfqpoint{0.607914in}{1.053000in}}%
\pgfpathlineto{\pgfqpoint{0.608852in}{0.851097in}}%
\pgfpathlineto{\pgfqpoint{0.609060in}{1.334582in}}%
\pgfpathlineto{\pgfqpoint{0.609164in}{0.848781in}}%
\pgfpathlineto{\pgfqpoint{0.610206in}{1.019079in}}%
\pgfpathlineto{\pgfqpoint{0.611039in}{0.809006in}}%
\pgfpathlineto{\pgfqpoint{0.610518in}{1.165105in}}%
\pgfpathlineto{\pgfqpoint{0.611247in}{0.956859in}}%
\pgfpathlineto{\pgfqpoint{0.612080in}{1.147952in}}%
\pgfpathlineto{\pgfqpoint{0.611559in}{0.789111in}}%
\pgfpathlineto{\pgfqpoint{0.612288in}{1.044466in}}%
\pgfpathlineto{\pgfqpoint{0.612392in}{0.876324in}}%
\pgfpathlineto{\pgfqpoint{0.612913in}{1.201952in}}%
\pgfpathlineto{\pgfqpoint{0.613330in}{1.131976in}}%
\pgfpathlineto{\pgfqpoint{0.613434in}{1.265875in}}%
\pgfpathlineto{\pgfqpoint{0.613538in}{0.691192in}}%
\pgfpathlineto{\pgfqpoint{0.614371in}{1.225667in}}%
\pgfpathlineto{\pgfqpoint{0.614788in}{0.982562in}}%
\pgfpathlineto{\pgfqpoint{0.615308in}{1.358958in}}%
\pgfpathlineto{\pgfqpoint{0.615517in}{1.006292in}}%
\pgfpathlineto{\pgfqpoint{0.616350in}{1.300367in}}%
\pgfpathlineto{\pgfqpoint{0.615829in}{0.754857in}}%
\pgfpathlineto{\pgfqpoint{0.616662in}{1.175974in}}%
\pgfpathlineto{\pgfqpoint{0.616766in}{0.729450in}}%
\pgfpathlineto{\pgfqpoint{0.617600in}{1.387638in}}%
\pgfpathlineto{\pgfqpoint{0.617704in}{1.080879in}}%
\pgfpathlineto{\pgfqpoint{0.617808in}{1.090623in}}%
\pgfpathlineto{\pgfqpoint{0.618745in}{1.206540in}}%
\pgfpathlineto{\pgfqpoint{0.619057in}{0.910432in}}%
\pgfpathlineto{\pgfqpoint{0.619891in}{1.329791in}}%
\pgfpathlineto{\pgfqpoint{0.620099in}{0.871384in}}%
\pgfpathlineto{\pgfqpoint{0.620203in}{0.792953in}}%
\pgfpathlineto{\pgfqpoint{0.620307in}{1.273875in}}%
\pgfpathlineto{\pgfqpoint{0.620724in}{0.888884in}}%
\pgfpathlineto{\pgfqpoint{0.620828in}{1.413346in}}%
\pgfpathlineto{\pgfqpoint{0.621869in}{1.288141in}}%
\pgfpathlineto{\pgfqpoint{0.622494in}{0.805620in}}%
\pgfpathlineto{\pgfqpoint{0.623015in}{0.811143in}}%
\pgfpathlineto{\pgfqpoint{0.623640in}{1.299278in}}%
\pgfpathlineto{\pgfqpoint{0.624056in}{1.014119in}}%
\pgfpathlineto{\pgfqpoint{0.624160in}{0.658114in}}%
\pgfpathlineto{\pgfqpoint{0.624994in}{1.432086in}}%
\pgfpathlineto{\pgfqpoint{0.625098in}{1.062867in}}%
\pgfpathlineto{\pgfqpoint{0.625410in}{1.083261in}}%
\pgfpathlineto{\pgfqpoint{0.626035in}{0.949484in}}%
\pgfpathlineto{\pgfqpoint{0.626347in}{1.275042in}}%
\pgfpathlineto{\pgfqpoint{0.626452in}{0.664857in}}%
\pgfpathlineto{\pgfqpoint{0.626972in}{0.999622in}}%
\pgfpathlineto{\pgfqpoint{0.627076in}{0.847467in}}%
\pgfpathlineto{\pgfqpoint{0.627389in}{1.311504in}}%
\pgfpathlineto{\pgfqpoint{0.627909in}{0.977141in}}%
\pgfpathlineto{\pgfqpoint{0.628847in}{1.224426in}}%
\pgfpathlineto{\pgfqpoint{0.628118in}{0.874692in}}%
\pgfpathlineto{\pgfqpoint{0.629055in}{1.144625in}}%
\pgfpathlineto{\pgfqpoint{0.629159in}{1.141271in}}%
\pgfpathlineto{\pgfqpoint{0.629888in}{1.208225in}}%
\pgfpathlineto{\pgfqpoint{0.630201in}{0.825568in}}%
\pgfpathlineto{\pgfqpoint{0.630513in}{1.311874in}}%
\pgfpathlineto{\pgfqpoint{0.630617in}{0.748424in}}%
\pgfpathlineto{\pgfqpoint{0.631346in}{1.216198in}}%
\pgfpathlineto{\pgfqpoint{0.631971in}{0.803020in}}%
\pgfpathlineto{\pgfqpoint{0.632492in}{1.056878in}}%
\pgfpathlineto{\pgfqpoint{0.633012in}{0.910489in}}%
\pgfpathlineto{\pgfqpoint{0.633117in}{1.114125in}}%
\pgfpathlineto{\pgfqpoint{0.633950in}{0.783825in}}%
\pgfpathlineto{\pgfqpoint{0.633741in}{1.298757in}}%
\pgfpathlineto{\pgfqpoint{0.634158in}{0.962411in}}%
\pgfpathlineto{\pgfqpoint{0.634887in}{0.847592in}}%
\pgfpathlineto{\pgfqpoint{0.635303in}{1.208905in}}%
\pgfpathlineto{\pgfqpoint{0.635408in}{0.914744in}}%
\pgfpathlineto{\pgfqpoint{0.635512in}{1.440754in}}%
\pgfpathlineto{\pgfqpoint{0.636345in}{1.040063in}}%
\pgfpathlineto{\pgfqpoint{0.636970in}{1.320341in}}%
\pgfpathlineto{\pgfqpoint{0.636553in}{0.950904in}}%
\pgfpathlineto{\pgfqpoint{0.637490in}{1.145488in}}%
\pgfpathlineto{\pgfqpoint{0.637595in}{0.823572in}}%
\pgfpathlineto{\pgfqpoint{0.638532in}{1.136380in}}%
\pgfpathlineto{\pgfqpoint{0.638636in}{1.212891in}}%
\pgfpathlineto{\pgfqpoint{0.639053in}{0.810247in}}%
\pgfpathlineto{\pgfqpoint{0.639365in}{1.094054in}}%
\pgfpathlineto{\pgfqpoint{0.639469in}{0.866671in}}%
\pgfpathlineto{\pgfqpoint{0.639677in}{1.208833in}}%
\pgfpathlineto{\pgfqpoint{0.640511in}{0.986282in}}%
\pgfpathlineto{\pgfqpoint{0.640719in}{1.008766in}}%
\pgfpathlineto{\pgfqpoint{0.641448in}{1.266669in}}%
\pgfpathlineto{\pgfqpoint{0.641135in}{0.922791in}}%
\pgfpathlineto{\pgfqpoint{0.641656in}{1.205704in}}%
\pgfpathlineto{\pgfqpoint{0.641760in}{0.836005in}}%
\pgfpathlineto{\pgfqpoint{0.642489in}{1.391780in}}%
\pgfpathlineto{\pgfqpoint{0.642698in}{0.988745in}}%
\pgfpathlineto{\pgfqpoint{0.642906in}{1.286142in}}%
\pgfpathlineto{\pgfqpoint{0.643322in}{0.856728in}}%
\pgfpathlineto{\pgfqpoint{0.643843in}{1.088566in}}%
\pgfpathlineto{\pgfqpoint{0.643947in}{1.281872in}}%
\pgfpathlineto{\pgfqpoint{0.644572in}{0.889291in}}%
\pgfpathlineto{\pgfqpoint{0.644884in}{1.126490in}}%
\pgfpathlineto{\pgfqpoint{0.644989in}{1.096637in}}%
\pgfpathlineto{\pgfqpoint{0.645093in}{1.414748in}}%
\pgfpathlineto{\pgfqpoint{0.646030in}{0.931933in}}%
\pgfpathlineto{\pgfqpoint{0.646759in}{1.317908in}}%
\pgfpathlineto{\pgfqpoint{0.646551in}{0.831573in}}%
\pgfpathlineto{\pgfqpoint{0.647176in}{1.103288in}}%
\pgfpathlineto{\pgfqpoint{0.647384in}{1.349383in}}%
\pgfpathlineto{\pgfqpoint{0.647696in}{0.851608in}}%
\pgfpathlineto{\pgfqpoint{0.648009in}{1.317714in}}%
\pgfpathlineto{\pgfqpoint{0.648634in}{0.976209in}}%
\pgfpathlineto{\pgfqpoint{0.648529in}{1.425268in}}%
\pgfpathlineto{\pgfqpoint{0.649154in}{1.139924in}}%
\pgfpathlineto{\pgfqpoint{0.649258in}{1.001406in}}%
\pgfpathlineto{\pgfqpoint{0.649987in}{1.257150in}}%
\pgfpathlineto{\pgfqpoint{0.650300in}{1.075716in}}%
\pgfpathlineto{\pgfqpoint{0.651237in}{0.854466in}}%
\pgfpathlineto{\pgfqpoint{0.651550in}{1.371860in}}%
\pgfpathlineto{\pgfqpoint{0.652487in}{0.988608in}}%
\pgfpathlineto{\pgfqpoint{0.652695in}{1.211841in}}%
\pgfpathlineto{\pgfqpoint{0.653216in}{0.830480in}}%
\pgfpathlineto{\pgfqpoint{0.653528in}{1.284079in}}%
\pgfpathlineto{\pgfqpoint{0.653736in}{0.972635in}}%
\pgfpathlineto{\pgfqpoint{0.653841in}{1.330431in}}%
\pgfpathlineto{\pgfqpoint{0.654049in}{0.731260in}}%
\pgfpathlineto{\pgfqpoint{0.654882in}{1.182297in}}%
\pgfpathlineto{\pgfqpoint{0.655403in}{0.830136in}}%
\pgfpathlineto{\pgfqpoint{0.655194in}{1.325362in}}%
\pgfpathlineto{\pgfqpoint{0.656028in}{0.870373in}}%
\pgfpathlineto{\pgfqpoint{0.656132in}{1.266197in}}%
\pgfpathlineto{\pgfqpoint{0.657173in}{1.158293in}}%
\pgfpathlineto{\pgfqpoint{0.657590in}{1.318651in}}%
\pgfpathlineto{\pgfqpoint{0.658423in}{0.776678in}}%
\pgfpathlineto{\pgfqpoint{0.659568in}{1.274955in}}%
\pgfpathlineto{\pgfqpoint{0.658631in}{0.698207in}}%
\pgfpathlineto{\pgfqpoint{0.659673in}{1.183017in}}%
\pgfpathlineto{\pgfqpoint{0.660089in}{0.754327in}}%
\pgfpathlineto{\pgfqpoint{0.660610in}{1.318173in}}%
\pgfpathlineto{\pgfqpoint{0.660818in}{0.970289in}}%
\pgfpathlineto{\pgfqpoint{0.661026in}{1.259676in}}%
\pgfpathlineto{\pgfqpoint{0.661339in}{0.959163in}}%
\pgfpathlineto{\pgfqpoint{0.661859in}{1.208305in}}%
\pgfpathlineto{\pgfqpoint{0.662380in}{0.952958in}}%
\pgfpathlineto{\pgfqpoint{0.662276in}{1.358937in}}%
\pgfpathlineto{\pgfqpoint{0.663005in}{1.116462in}}%
\pgfpathlineto{\pgfqpoint{0.663109in}{1.130952in}}%
\pgfpathlineto{\pgfqpoint{0.663422in}{1.368778in}}%
\pgfpathlineto{\pgfqpoint{0.664255in}{0.809127in}}%
\pgfpathlineto{\pgfqpoint{0.665192in}{1.301467in}}%
\pgfpathlineto{\pgfqpoint{0.665400in}{1.212582in}}%
\pgfpathlineto{\pgfqpoint{0.665921in}{0.822666in}}%
\pgfpathlineto{\pgfqpoint{0.666025in}{1.271011in}}%
\pgfpathlineto{\pgfqpoint{0.666650in}{0.973354in}}%
\pgfpathlineto{\pgfqpoint{0.666754in}{0.955538in}}%
\pgfpathlineto{\pgfqpoint{0.666858in}{1.050418in}}%
\pgfpathlineto{\pgfqpoint{0.666962in}{1.226394in}}%
\pgfpathlineto{\pgfqpoint{0.667379in}{0.635842in}}%
\pgfpathlineto{\pgfqpoint{0.667796in}{0.889739in}}%
\pgfpathlineto{\pgfqpoint{0.667900in}{0.864037in}}%
\pgfpathlineto{\pgfqpoint{0.668004in}{1.224646in}}%
\pgfpathlineto{\pgfqpoint{0.668941in}{0.825575in}}%
\pgfpathlineto{\pgfqpoint{0.669045in}{0.991299in}}%
\pgfpathlineto{\pgfqpoint{0.669670in}{1.359155in}}%
\pgfpathlineto{\pgfqpoint{0.670087in}{1.089238in}}%
\pgfpathlineto{\pgfqpoint{0.670816in}{1.161764in}}%
\pgfpathlineto{\pgfqpoint{0.671128in}{0.760721in}}%
\pgfpathlineto{\pgfqpoint{0.671649in}{1.360863in}}%
\pgfpathlineto{\pgfqpoint{0.671545in}{0.628835in}}%
\pgfpathlineto{\pgfqpoint{0.672378in}{1.335259in}}%
\pgfpathlineto{\pgfqpoint{0.672482in}{1.026167in}}%
\pgfpathlineto{\pgfqpoint{0.673107in}{1.407129in}}%
\pgfpathlineto{\pgfqpoint{0.673523in}{1.216525in}}%
\pgfpathlineto{\pgfqpoint{0.673627in}{1.228494in}}%
\pgfpathlineto{\pgfqpoint{0.673732in}{1.147972in}}%
\pgfpathlineto{\pgfqpoint{0.674044in}{0.885821in}}%
\pgfpathlineto{\pgfqpoint{0.673940in}{1.294051in}}%
\pgfpathlineto{\pgfqpoint{0.674773in}{1.291155in}}%
\pgfpathlineto{\pgfqpoint{0.675294in}{0.873888in}}%
\pgfpathlineto{\pgfqpoint{0.676023in}{1.028754in}}%
\pgfpathlineto{\pgfqpoint{0.676439in}{1.288335in}}%
\pgfpathlineto{\pgfqpoint{0.676335in}{0.842689in}}%
\pgfpathlineto{\pgfqpoint{0.676647in}{0.908577in}}%
\pgfpathlineto{\pgfqpoint{0.677376in}{1.384956in}}%
\pgfpathlineto{\pgfqpoint{0.677689in}{0.687115in}}%
\pgfpathlineto{\pgfqpoint{0.678210in}{1.412465in}}%
\pgfpathlineto{\pgfqpoint{0.678834in}{1.095465in}}%
\pgfpathlineto{\pgfqpoint{0.679355in}{0.805548in}}%
\pgfpathlineto{\pgfqpoint{0.679772in}{1.080589in}}%
\pgfpathlineto{\pgfqpoint{0.679876in}{1.258596in}}%
\pgfpathlineto{\pgfqpoint{0.680084in}{0.823531in}}%
\pgfpathlineto{\pgfqpoint{0.680813in}{1.069645in}}%
\pgfpathlineto{\pgfqpoint{0.681230in}{0.820228in}}%
\pgfpathlineto{\pgfqpoint{0.681750in}{1.254504in}}%
\pgfpathlineto{\pgfqpoint{0.681959in}{0.991109in}}%
\pgfpathlineto{\pgfqpoint{0.682792in}{1.311108in}}%
\pgfpathlineto{\pgfqpoint{0.682688in}{0.848188in}}%
\pgfpathlineto{\pgfqpoint{0.683000in}{0.952412in}}%
\pgfpathlineto{\pgfqpoint{0.683104in}{0.948667in}}%
\pgfpathlineto{\pgfqpoint{0.683833in}{1.369201in}}%
\pgfpathlineto{\pgfqpoint{0.683937in}{0.879763in}}%
\pgfpathlineto{\pgfqpoint{0.684250in}{1.114940in}}%
\pgfpathlineto{\pgfqpoint{0.685083in}{0.853005in}}%
\pgfpathlineto{\pgfqpoint{0.684875in}{1.378801in}}%
\pgfpathlineto{\pgfqpoint{0.685187in}{1.041206in}}%
\pgfpathlineto{\pgfqpoint{0.685291in}{1.433019in}}%
\pgfpathlineto{\pgfqpoint{0.685395in}{0.936145in}}%
\pgfpathlineto{\pgfqpoint{0.686228in}{0.968746in}}%
\pgfpathlineto{\pgfqpoint{0.686957in}{0.825676in}}%
\pgfpathlineto{\pgfqpoint{0.687062in}{1.102762in}}%
\pgfpathlineto{\pgfqpoint{0.687686in}{1.299799in}}%
\pgfpathlineto{\pgfqpoint{0.687270in}{0.761376in}}%
\pgfpathlineto{\pgfqpoint{0.688103in}{1.136165in}}%
\pgfpathlineto{\pgfqpoint{0.688624in}{0.846556in}}%
\pgfpathlineto{\pgfqpoint{0.689040in}{1.235699in}}%
\pgfpathlineto{\pgfqpoint{0.689249in}{1.143627in}}%
\pgfpathlineto{\pgfqpoint{0.689353in}{0.893757in}}%
\pgfpathlineto{\pgfqpoint{0.689769in}{1.394780in}}%
\pgfpathlineto{\pgfqpoint{0.690290in}{1.031779in}}%
\pgfpathlineto{\pgfqpoint{0.691123in}{0.738003in}}%
\pgfpathlineto{\pgfqpoint{0.691331in}{1.420285in}}%
\pgfpathlineto{\pgfqpoint{0.692269in}{0.956057in}}%
\pgfpathlineto{\pgfqpoint{0.692477in}{1.028476in}}%
\pgfpathlineto{\pgfqpoint{0.692789in}{0.922224in}}%
\pgfpathlineto{\pgfqpoint{0.693102in}{1.213642in}}%
\pgfpathlineto{\pgfqpoint{0.693310in}{1.050495in}}%
\pgfpathlineto{\pgfqpoint{0.693518in}{1.258332in}}%
\pgfpathlineto{\pgfqpoint{0.693831in}{0.861518in}}%
\pgfpathlineto{\pgfqpoint{0.694456in}{1.131959in}}%
\pgfpathlineto{\pgfqpoint{0.694872in}{0.764059in}}%
\pgfpathlineto{\pgfqpoint{0.695393in}{1.278972in}}%
\pgfpathlineto{\pgfqpoint{0.696018in}{0.964088in}}%
\pgfpathlineto{\pgfqpoint{0.696643in}{1.118338in}}%
\pgfpathlineto{\pgfqpoint{0.696747in}{1.229025in}}%
\pgfpathlineto{\pgfqpoint{0.697059in}{0.679109in}}%
\pgfpathlineto{\pgfqpoint{0.697476in}{0.882331in}}%
\pgfpathlineto{\pgfqpoint{0.697996in}{0.673658in}}%
\pgfpathlineto{\pgfqpoint{0.697788in}{1.165658in}}%
\pgfpathlineto{\pgfqpoint{0.698309in}{0.836982in}}%
\pgfpathlineto{\pgfqpoint{0.698934in}{1.229895in}}%
\pgfpathlineto{\pgfqpoint{0.699454in}{0.924231in}}%
\pgfpathlineto{\pgfqpoint{0.700496in}{1.322333in}}%
\pgfpathlineto{\pgfqpoint{0.699871in}{0.817208in}}%
\pgfpathlineto{\pgfqpoint{0.700600in}{1.153729in}}%
\pgfpathlineto{\pgfqpoint{0.701537in}{0.943526in}}%
\pgfpathlineto{\pgfqpoint{0.700808in}{1.370986in}}%
\pgfpathlineto{\pgfqpoint{0.701745in}{0.954643in}}%
\pgfpathlineto{\pgfqpoint{0.701850in}{1.349850in}}%
\pgfpathlineto{\pgfqpoint{0.702266in}{0.856335in}}%
\pgfpathlineto{\pgfqpoint{0.702891in}{1.228299in}}%
\pgfpathlineto{\pgfqpoint{0.703308in}{0.845367in}}%
\pgfpathlineto{\pgfqpoint{0.703412in}{1.274652in}}%
\pgfpathlineto{\pgfqpoint{0.703932in}{1.121121in}}%
\pgfpathlineto{\pgfqpoint{0.704037in}{1.372116in}}%
\pgfpathlineto{\pgfqpoint{0.704141in}{0.959665in}}%
\pgfpathlineto{\pgfqpoint{0.704974in}{1.000574in}}%
\pgfpathlineto{\pgfqpoint{0.705286in}{0.765397in}}%
\pgfpathlineto{\pgfqpoint{0.705390in}{1.356139in}}%
\pgfpathlineto{\pgfqpoint{0.705807in}{1.099315in}}%
\pgfpathlineto{\pgfqpoint{0.706536in}{1.290220in}}%
\pgfpathlineto{\pgfqpoint{0.706119in}{0.968541in}}%
\pgfpathlineto{\pgfqpoint{0.706953in}{1.226503in}}%
\pgfpathlineto{\pgfqpoint{0.707161in}{0.863721in}}%
\pgfpathlineto{\pgfqpoint{0.707994in}{1.304153in}}%
\pgfpathlineto{\pgfqpoint{0.708098in}{1.089527in}}%
\pgfpathlineto{\pgfqpoint{0.708202in}{1.436246in}}%
\pgfpathlineto{\pgfqpoint{0.708411in}{0.795590in}}%
\pgfpathlineto{\pgfqpoint{0.709140in}{1.023474in}}%
\pgfpathlineto{\pgfqpoint{0.709764in}{0.878547in}}%
\pgfpathlineto{\pgfqpoint{0.709556in}{1.299408in}}%
\pgfpathlineto{\pgfqpoint{0.710181in}{1.034489in}}%
\pgfpathlineto{\pgfqpoint{0.710493in}{1.149375in}}%
\pgfpathlineto{\pgfqpoint{0.710389in}{0.938903in}}%
\pgfpathlineto{\pgfqpoint{0.710702in}{1.115782in}}%
\pgfpathlineto{\pgfqpoint{0.711743in}{0.795583in}}%
\pgfpathlineto{\pgfqpoint{0.711222in}{1.307787in}}%
\pgfpathlineto{\pgfqpoint{0.711847in}{0.830117in}}%
\pgfpathlineto{\pgfqpoint{0.712160in}{1.429860in}}%
\pgfpathlineto{\pgfqpoint{0.712993in}{1.141187in}}%
\pgfpathlineto{\pgfqpoint{0.713097in}{1.168478in}}%
\pgfpathlineto{\pgfqpoint{0.713305in}{0.934012in}}%
\pgfpathlineto{\pgfqpoint{0.713513in}{1.029178in}}%
\pgfpathlineto{\pgfqpoint{0.713618in}{0.987611in}}%
\pgfpathlineto{\pgfqpoint{0.713826in}{1.194464in}}%
\pgfpathlineto{\pgfqpoint{0.713930in}{1.067154in}}%
\pgfpathlineto{\pgfqpoint{0.714763in}{1.276431in}}%
\pgfpathlineto{\pgfqpoint{0.714451in}{0.780056in}}%
\pgfpathlineto{\pgfqpoint{0.714867in}{1.146809in}}%
\pgfpathlineto{\pgfqpoint{0.714971in}{0.903761in}}%
\pgfpathlineto{\pgfqpoint{0.715805in}{1.330935in}}%
\pgfpathlineto{\pgfqpoint{0.716013in}{0.982379in}}%
\pgfpathlineto{\pgfqpoint{0.716846in}{1.129596in}}%
\pgfpathlineto{\pgfqpoint{0.717158in}{0.632358in}}%
\pgfpathlineto{\pgfqpoint{0.717367in}{1.314438in}}%
\pgfpathlineto{\pgfqpoint{0.718304in}{1.039462in}}%
\pgfpathlineto{\pgfqpoint{0.718512in}{0.925232in}}%
\pgfpathlineto{\pgfqpoint{0.718825in}{1.108121in}}%
\pgfpathlineto{\pgfqpoint{0.719137in}{1.273382in}}%
\pgfpathlineto{\pgfqpoint{0.719033in}{0.936218in}}%
\pgfpathlineto{\pgfqpoint{0.719449in}{0.944437in}}%
\pgfpathlineto{\pgfqpoint{0.719554in}{0.698154in}}%
\pgfpathlineto{\pgfqpoint{0.719658in}{1.183999in}}%
\pgfpathlineto{\pgfqpoint{0.720491in}{0.869969in}}%
\pgfpathlineto{\pgfqpoint{0.721116in}{1.225221in}}%
\pgfpathlineto{\pgfqpoint{0.721012in}{0.816977in}}%
\pgfpathlineto{\pgfqpoint{0.721532in}{1.222856in}}%
\pgfpathlineto{\pgfqpoint{0.722574in}{0.625150in}}%
\pgfpathlineto{\pgfqpoint{0.723199in}{1.356328in}}%
\pgfpathlineto{\pgfqpoint{0.723719in}{0.959306in}}%
\pgfpathlineto{\pgfqpoint{0.724032in}{0.914167in}}%
\pgfpathlineto{\pgfqpoint{0.724136in}{1.072185in}}%
\pgfpathlineto{\pgfqpoint{0.725073in}{1.523101in}}%
\pgfpathlineto{\pgfqpoint{0.724657in}{0.824605in}}%
\pgfpathlineto{\pgfqpoint{0.725177in}{1.098044in}}%
\pgfpathlineto{\pgfqpoint{0.725698in}{0.842949in}}%
\pgfpathlineto{\pgfqpoint{0.726010in}{1.329277in}}%
\pgfpathlineto{\pgfqpoint{0.726219in}{0.988478in}}%
\pgfpathlineto{\pgfqpoint{0.727260in}{0.913904in}}%
\pgfpathlineto{\pgfqpoint{0.727468in}{1.357067in}}%
\pgfpathlineto{\pgfqpoint{0.727572in}{0.897396in}}%
\pgfpathlineto{\pgfqpoint{0.728510in}{1.228546in}}%
\pgfpathlineto{\pgfqpoint{0.728614in}{1.434466in}}%
\pgfpathlineto{\pgfqpoint{0.729030in}{1.002305in}}%
\pgfpathlineto{\pgfqpoint{0.729447in}{1.247669in}}%
\pgfpathlineto{\pgfqpoint{0.729968in}{0.742561in}}%
\pgfpathlineto{\pgfqpoint{0.730488in}{0.980041in}}%
\pgfpathlineto{\pgfqpoint{0.731217in}{1.291400in}}%
\pgfpathlineto{\pgfqpoint{0.731113in}{0.918708in}}%
\pgfpathlineto{\pgfqpoint{0.731634in}{1.200779in}}%
\pgfpathlineto{\pgfqpoint{0.732259in}{0.881020in}}%
\pgfpathlineto{\pgfqpoint{0.732363in}{1.228589in}}%
\pgfpathlineto{\pgfqpoint{0.732780in}{1.016148in}}%
\pgfpathlineto{\pgfqpoint{0.733404in}{1.338916in}}%
\pgfpathlineto{\pgfqpoint{0.733196in}{0.856504in}}%
\pgfpathlineto{\pgfqpoint{0.733925in}{1.105663in}}%
\pgfpathlineto{\pgfqpoint{0.734446in}{1.018740in}}%
\pgfpathlineto{\pgfqpoint{0.734654in}{1.333181in}}%
\pgfpathlineto{\pgfqpoint{0.735175in}{0.735099in}}%
\pgfpathlineto{\pgfqpoint{0.735800in}{1.158358in}}%
\pgfpathlineto{\pgfqpoint{0.736424in}{0.916795in}}%
\pgfpathlineto{\pgfqpoint{0.736529in}{1.213654in}}%
\pgfpathlineto{\pgfqpoint{0.736841in}{1.108109in}}%
\pgfpathlineto{\pgfqpoint{0.736945in}{1.324121in}}%
\pgfpathlineto{\pgfqpoint{0.737570in}{0.816804in}}%
\pgfpathlineto{\pgfqpoint{0.737987in}{1.305386in}}%
\pgfpathlineto{\pgfqpoint{0.739132in}{0.657182in}}%
\pgfpathlineto{\pgfqpoint{0.740174in}{1.295011in}}%
\pgfpathlineto{\pgfqpoint{0.740278in}{1.163157in}}%
\pgfpathlineto{\pgfqpoint{0.740486in}{0.818910in}}%
\pgfpathlineto{\pgfqpoint{0.740798in}{1.372989in}}%
\pgfpathlineto{\pgfqpoint{0.741215in}{1.060574in}}%
\pgfpathlineto{\pgfqpoint{0.741319in}{1.303101in}}%
\pgfpathlineto{\pgfqpoint{0.741736in}{0.860316in}}%
\pgfpathlineto{\pgfqpoint{0.742152in}{0.952483in}}%
\pgfpathlineto{\pgfqpoint{0.742465in}{0.833192in}}%
\pgfpathlineto{\pgfqpoint{0.742569in}{1.165790in}}%
\pgfpathlineto{\pgfqpoint{0.743089in}{1.101043in}}%
\pgfpathlineto{\pgfqpoint{0.743194in}{1.251025in}}%
\pgfpathlineto{\pgfqpoint{0.743298in}{0.778788in}}%
\pgfpathlineto{\pgfqpoint{0.744131in}{1.212190in}}%
\pgfpathlineto{\pgfqpoint{0.744756in}{0.789367in}}%
\pgfpathlineto{\pgfqpoint{0.744339in}{1.333374in}}%
\pgfpathlineto{\pgfqpoint{0.745172in}{0.821961in}}%
\pgfpathlineto{\pgfqpoint{0.745589in}{1.290936in}}%
\pgfpathlineto{\pgfqpoint{0.746005in}{0.738161in}}%
\pgfpathlineto{\pgfqpoint{0.746214in}{1.023164in}}%
\pgfpathlineto{\pgfqpoint{0.746943in}{1.119072in}}%
\pgfpathlineto{\pgfqpoint{0.747047in}{0.894038in}}%
\pgfpathlineto{\pgfqpoint{0.747151in}{1.261376in}}%
\pgfpathlineto{\pgfqpoint{0.748088in}{0.886206in}}%
\pgfpathlineto{\pgfqpoint{0.748192in}{1.070113in}}%
\pgfpathlineto{\pgfqpoint{0.748609in}{0.777808in}}%
\pgfpathlineto{\pgfqpoint{0.748921in}{1.077646in}}%
\pgfpathlineto{\pgfqpoint{0.749026in}{0.903249in}}%
\pgfpathlineto{\pgfqpoint{0.749650in}{1.314887in}}%
\pgfpathlineto{\pgfqpoint{0.749755in}{0.878876in}}%
\pgfpathlineto{\pgfqpoint{0.750171in}{1.258664in}}%
\pgfpathlineto{\pgfqpoint{0.750900in}{0.797386in}}%
\pgfpathlineto{\pgfqpoint{0.751212in}{1.205807in}}%
\pgfpathlineto{\pgfqpoint{0.751317in}{1.219635in}}%
\pgfpathlineto{\pgfqpoint{0.752150in}{1.259136in}}%
\pgfpathlineto{\pgfqpoint{0.752462in}{0.806946in}}%
\pgfpathlineto{\pgfqpoint{0.752775in}{1.291430in}}%
\pgfpathlineto{\pgfqpoint{0.753712in}{1.134396in}}%
\pgfpathlineto{\pgfqpoint{0.753920in}{0.990084in}}%
\pgfpathlineto{\pgfqpoint{0.754441in}{1.288943in}}%
\pgfpathlineto{\pgfqpoint{0.754857in}{1.005147in}}%
\pgfpathlineto{\pgfqpoint{0.755899in}{1.381852in}}%
\pgfpathlineto{\pgfqpoint{0.755378in}{0.823476in}}%
\pgfpathlineto{\pgfqpoint{0.756003in}{1.147873in}}%
\pgfpathlineto{\pgfqpoint{0.756107in}{1.127723in}}%
\pgfpathlineto{\pgfqpoint{0.756836in}{0.763496in}}%
\pgfpathlineto{\pgfqpoint{0.756940in}{1.164006in}}%
\pgfpathlineto{\pgfqpoint{0.757149in}{1.089305in}}%
\pgfpathlineto{\pgfqpoint{0.757461in}{0.682580in}}%
\pgfpathlineto{\pgfqpoint{0.758294in}{1.442511in}}%
\pgfpathlineto{\pgfqpoint{0.758398in}{0.833449in}}%
\pgfpathlineto{\pgfqpoint{0.759440in}{0.999890in}}%
\pgfpathlineto{\pgfqpoint{0.759960in}{0.793788in}}%
\pgfpathlineto{\pgfqpoint{0.760481in}{1.256016in}}%
\pgfpathlineto{\pgfqpoint{0.760585in}{0.740532in}}%
\pgfpathlineto{\pgfqpoint{0.760689in}{1.514971in}}%
\pgfpathlineto{\pgfqpoint{0.761627in}{1.025190in}}%
\pgfpathlineto{\pgfqpoint{0.761731in}{1.031955in}}%
\pgfpathlineto{\pgfqpoint{0.761835in}{1.025547in}}%
\pgfpathlineto{\pgfqpoint{0.761939in}{1.220672in}}%
\pgfpathlineto{\pgfqpoint{0.762147in}{0.622354in}}%
\pgfpathlineto{\pgfqpoint{0.762980in}{1.113949in}}%
\pgfpathlineto{\pgfqpoint{0.763189in}{0.851564in}}%
\pgfpathlineto{\pgfqpoint{0.763501in}{1.194143in}}%
\pgfpathlineto{\pgfqpoint{0.764022in}{1.117601in}}%
\pgfpathlineto{\pgfqpoint{0.764334in}{1.228771in}}%
\pgfpathlineto{\pgfqpoint{0.764751in}{1.050276in}}%
\pgfpathlineto{\pgfqpoint{0.765584in}{0.776080in}}%
\pgfpathlineto{\pgfqpoint{0.765688in}{1.085494in}}%
\pgfpathlineto{\pgfqpoint{0.765896in}{0.934175in}}%
\pgfpathlineto{\pgfqpoint{0.766105in}{1.191859in}}%
\pgfpathlineto{\pgfqpoint{0.766209in}{0.818089in}}%
\pgfpathlineto{\pgfqpoint{0.766834in}{1.015325in}}%
\pgfpathlineto{\pgfqpoint{0.767667in}{0.610734in}}%
\pgfpathlineto{\pgfqpoint{0.767458in}{1.253993in}}%
\pgfpathlineto{\pgfqpoint{0.767771in}{1.081170in}}%
\pgfpathlineto{\pgfqpoint{0.768396in}{0.911198in}}%
\pgfpathlineto{\pgfqpoint{0.768812in}{1.459446in}}%
\pgfpathlineto{\pgfqpoint{0.769229in}{0.794369in}}%
\pgfpathlineto{\pgfqpoint{0.769958in}{1.054201in}}%
\pgfpathlineto{\pgfqpoint{0.770062in}{1.052805in}}%
\pgfpathlineto{\pgfqpoint{0.770791in}{0.856520in}}%
\pgfpathlineto{\pgfqpoint{0.770374in}{1.175342in}}%
\pgfpathlineto{\pgfqpoint{0.770999in}{0.984234in}}%
\pgfpathlineto{\pgfqpoint{0.771103in}{1.182581in}}%
\pgfpathlineto{\pgfqpoint{0.771937in}{0.968588in}}%
\pgfpathlineto{\pgfqpoint{0.772041in}{0.696013in}}%
\pgfpathlineto{\pgfqpoint{0.772249in}{1.263111in}}%
\pgfpathlineto{\pgfqpoint{0.772978in}{1.007753in}}%
\pgfpathlineto{\pgfqpoint{0.773186in}{1.230969in}}%
\pgfpathlineto{\pgfqpoint{0.773811in}{0.716960in}}%
\pgfpathlineto{\pgfqpoint{0.773915in}{0.973320in}}%
\pgfpathlineto{\pgfqpoint{0.774019in}{0.874900in}}%
\pgfpathlineto{\pgfqpoint{0.774436in}{1.267475in}}%
\pgfpathlineto{\pgfqpoint{0.774853in}{0.886925in}}%
\pgfpathlineto{\pgfqpoint{0.775373in}{0.825990in}}%
\pgfpathlineto{\pgfqpoint{0.775894in}{1.219448in}}%
\pgfpathlineto{\pgfqpoint{0.775998in}{0.879497in}}%
\pgfpathlineto{\pgfqpoint{0.776102in}{1.295408in}}%
\pgfpathlineto{\pgfqpoint{0.776935in}{0.906992in}}%
\pgfpathlineto{\pgfqpoint{0.777664in}{1.408480in}}%
\pgfpathlineto{\pgfqpoint{0.777456in}{0.677874in}}%
\pgfpathlineto{\pgfqpoint{0.778081in}{1.057470in}}%
\pgfpathlineto{\pgfqpoint{0.778393in}{1.217816in}}%
\pgfpathlineto{\pgfqpoint{0.778289in}{0.880760in}}%
\pgfpathlineto{\pgfqpoint{0.779018in}{1.141082in}}%
\pgfpathlineto{\pgfqpoint{0.779643in}{0.838654in}}%
\pgfpathlineto{\pgfqpoint{0.779435in}{1.281655in}}%
\pgfpathlineto{\pgfqpoint{0.780164in}{1.021392in}}%
\pgfpathlineto{\pgfqpoint{0.780893in}{1.298169in}}%
\pgfpathlineto{\pgfqpoint{0.780684in}{0.812857in}}%
\pgfpathlineto{\pgfqpoint{0.781205in}{0.869106in}}%
\pgfpathlineto{\pgfqpoint{0.782247in}{1.286284in}}%
\pgfpathlineto{\pgfqpoint{0.782351in}{0.905097in}}%
\pgfpathlineto{\pgfqpoint{0.782559in}{0.869486in}}%
\pgfpathlineto{\pgfqpoint{0.783392in}{1.361034in}}%
\pgfpathlineto{\pgfqpoint{0.784329in}{0.729343in}}%
\pgfpathlineto{\pgfqpoint{0.784538in}{0.845052in}}%
\pgfpathlineto{\pgfqpoint{0.785162in}{1.351454in}}%
\pgfpathlineto{\pgfqpoint{0.784954in}{0.819475in}}%
\pgfpathlineto{\pgfqpoint{0.785683in}{1.056982in}}%
\pgfpathlineto{\pgfqpoint{0.786308in}{0.864244in}}%
\pgfpathlineto{\pgfqpoint{0.785996in}{1.319152in}}%
\pgfpathlineto{\pgfqpoint{0.786516in}{1.123547in}}%
\pgfpathlineto{\pgfqpoint{0.786620in}{1.272123in}}%
\pgfpathlineto{\pgfqpoint{0.787037in}{0.870539in}}%
\pgfpathlineto{\pgfqpoint{0.787558in}{1.093216in}}%
\pgfpathlineto{\pgfqpoint{0.787870in}{1.362529in}}%
\pgfpathlineto{\pgfqpoint{0.788495in}{0.958805in}}%
\pgfpathlineto{\pgfqpoint{0.788703in}{1.262649in}}%
\pgfpathlineto{\pgfqpoint{0.789328in}{0.764847in}}%
\pgfpathlineto{\pgfqpoint{0.789641in}{1.161488in}}%
\pgfpathlineto{\pgfqpoint{0.790474in}{0.824451in}}%
\pgfpathlineto{\pgfqpoint{0.790370in}{1.344425in}}%
\pgfpathlineto{\pgfqpoint{0.790890in}{0.858804in}}%
\pgfpathlineto{\pgfqpoint{0.791307in}{1.276094in}}%
\pgfpathlineto{\pgfqpoint{0.791411in}{0.841055in}}%
\pgfpathlineto{\pgfqpoint{0.791932in}{0.992379in}}%
\pgfpathlineto{\pgfqpoint{0.792036in}{0.671723in}}%
\pgfpathlineto{\pgfqpoint{0.792452in}{1.235654in}}%
\pgfpathlineto{\pgfqpoint{0.792973in}{1.101113in}}%
\pgfpathlineto{\pgfqpoint{0.793910in}{1.153941in}}%
\pgfpathlineto{\pgfqpoint{0.794014in}{0.885462in}}%
\pgfpathlineto{\pgfqpoint{0.794535in}{1.229534in}}%
\pgfpathlineto{\pgfqpoint{0.794327in}{0.800724in}}%
\pgfpathlineto{\pgfqpoint{0.795056in}{1.017230in}}%
\pgfpathlineto{\pgfqpoint{0.795785in}{0.932214in}}%
\pgfpathlineto{\pgfqpoint{0.795368in}{1.309540in}}%
\pgfpathlineto{\pgfqpoint{0.795889in}{1.130905in}}%
\pgfpathlineto{\pgfqpoint{0.795993in}{1.115615in}}%
\pgfpathlineto{\pgfqpoint{0.796097in}{1.260005in}}%
\pgfpathlineto{\pgfqpoint{0.796201in}{1.149694in}}%
\pgfpathlineto{\pgfqpoint{0.796306in}{0.833432in}}%
\pgfpathlineto{\pgfqpoint{0.796930in}{1.510884in}}%
\pgfpathlineto{\pgfqpoint{0.797243in}{1.132666in}}%
\pgfpathlineto{\pgfqpoint{0.797555in}{1.317810in}}%
\pgfpathlineto{\pgfqpoint{0.797451in}{1.007078in}}%
\pgfpathlineto{\pgfqpoint{0.798076in}{1.177375in}}%
\pgfpathlineto{\pgfqpoint{0.798388in}{0.892605in}}%
\pgfpathlineto{\pgfqpoint{0.798805in}{1.300081in}}%
\pgfpathlineto{\pgfqpoint{0.799222in}{0.997322in}}%
\pgfpathlineto{\pgfqpoint{0.799950in}{1.184929in}}%
\pgfpathlineto{\pgfqpoint{0.799742in}{0.831260in}}%
\pgfpathlineto{\pgfqpoint{0.800367in}{1.055208in}}%
\pgfpathlineto{\pgfqpoint{0.800992in}{0.737937in}}%
\pgfpathlineto{\pgfqpoint{0.801096in}{1.215295in}}%
\pgfpathlineto{\pgfqpoint{0.801304in}{0.927466in}}%
\pgfpathlineto{\pgfqpoint{0.802033in}{1.232153in}}%
\pgfpathlineto{\pgfqpoint{0.801929in}{0.853570in}}%
\pgfpathlineto{\pgfqpoint{0.802346in}{1.075228in}}%
\pgfpathlineto{\pgfqpoint{0.802450in}{0.793318in}}%
\pgfpathlineto{\pgfqpoint{0.803179in}{1.279409in}}%
\pgfpathlineto{\pgfqpoint{0.803491in}{0.890201in}}%
\pgfpathlineto{\pgfqpoint{0.804220in}{1.266893in}}%
\pgfpathlineto{\pgfqpoint{0.804429in}{0.859431in}}%
\pgfpathlineto{\pgfqpoint{0.804637in}{1.092394in}}%
\pgfpathlineto{\pgfqpoint{0.805678in}{1.206204in}}%
\pgfpathlineto{\pgfqpoint{0.805782in}{0.783045in}}%
\pgfpathlineto{\pgfqpoint{0.806616in}{1.084609in}}%
\pgfpathlineto{\pgfqpoint{0.807032in}{0.973887in}}%
\pgfpathlineto{\pgfqpoint{0.807240in}{1.299929in}}%
\pgfpathlineto{\pgfqpoint{0.807449in}{0.863318in}}%
\pgfpathlineto{\pgfqpoint{0.807969in}{1.280026in}}%
\pgfpathlineto{\pgfqpoint{0.808802in}{0.859369in}}%
\pgfpathlineto{\pgfqpoint{0.809115in}{1.020483in}}%
\pgfpathlineto{\pgfqpoint{0.809740in}{1.219290in}}%
\pgfpathlineto{\pgfqpoint{0.809323in}{0.829316in}}%
\pgfpathlineto{\pgfqpoint{0.809948in}{1.137087in}}%
\pgfpathlineto{\pgfqpoint{0.810677in}{1.175481in}}%
\pgfpathlineto{\pgfqpoint{0.810885in}{0.818233in}}%
\pgfpathlineto{\pgfqpoint{0.810989in}{1.311109in}}%
\pgfpathlineto{\pgfqpoint{0.811927in}{0.952760in}}%
\pgfpathlineto{\pgfqpoint{0.812031in}{0.919621in}}%
\pgfpathlineto{\pgfqpoint{0.812239in}{1.044305in}}%
\pgfpathlineto{\pgfqpoint{0.812343in}{0.949056in}}%
\pgfpathlineto{\pgfqpoint{0.812552in}{1.272583in}}%
\pgfpathlineto{\pgfqpoint{0.813385in}{0.872411in}}%
\pgfpathlineto{\pgfqpoint{0.813489in}{1.181812in}}%
\pgfpathlineto{\pgfqpoint{0.813593in}{1.184475in}}%
\pgfpathlineto{\pgfqpoint{0.814426in}{0.851354in}}%
\pgfpathlineto{\pgfqpoint{0.814218in}{1.401322in}}%
\pgfpathlineto{\pgfqpoint{0.814739in}{1.020266in}}%
\pgfpathlineto{\pgfqpoint{0.815363in}{1.346785in}}%
\pgfpathlineto{\pgfqpoint{0.814947in}{0.857587in}}%
\pgfpathlineto{\pgfqpoint{0.815780in}{0.997318in}}%
\pgfpathlineto{\pgfqpoint{0.816509in}{1.266178in}}%
\pgfpathlineto{\pgfqpoint{0.816092in}{0.772253in}}%
\pgfpathlineto{\pgfqpoint{0.816821in}{1.013203in}}%
\pgfpathlineto{\pgfqpoint{0.816925in}{1.013086in}}%
\pgfpathlineto{\pgfqpoint{0.817863in}{1.487266in}}%
\pgfpathlineto{\pgfqpoint{0.817654in}{0.894808in}}%
\pgfpathlineto{\pgfqpoint{0.817967in}{1.236960in}}%
\pgfpathlineto{\pgfqpoint{0.819008in}{0.728251in}}%
\pgfpathlineto{\pgfqpoint{0.819112in}{1.020423in}}%
\pgfpathlineto{\pgfqpoint{0.819217in}{0.929135in}}%
\pgfpathlineto{\pgfqpoint{0.819425in}{1.492577in}}%
\pgfpathlineto{\pgfqpoint{0.820050in}{1.143291in}}%
\pgfpathlineto{\pgfqpoint{0.820154in}{1.356188in}}%
\pgfpathlineto{\pgfqpoint{0.820258in}{0.888091in}}%
\pgfpathlineto{\pgfqpoint{0.821091in}{1.189052in}}%
\pgfpathlineto{\pgfqpoint{0.822028in}{0.851858in}}%
\pgfpathlineto{\pgfqpoint{0.821820in}{1.410551in}}%
\pgfpathlineto{\pgfqpoint{0.822133in}{0.971229in}}%
\pgfpathlineto{\pgfqpoint{0.822341in}{1.360873in}}%
\pgfpathlineto{\pgfqpoint{0.823070in}{0.667092in}}%
\pgfpathlineto{\pgfqpoint{0.823278in}{1.102313in}}%
\pgfpathlineto{\pgfqpoint{0.823382in}{1.189906in}}%
\pgfpathlineto{\pgfqpoint{0.823486in}{0.905273in}}%
\pgfpathlineto{\pgfqpoint{0.824320in}{1.180240in}}%
\pgfpathlineto{\pgfqpoint{0.824632in}{0.841435in}}%
\pgfpathlineto{\pgfqpoint{0.824944in}{1.439052in}}%
\pgfpathlineto{\pgfqpoint{0.825361in}{1.095384in}}%
\pgfpathlineto{\pgfqpoint{0.825465in}{1.312436in}}%
\pgfpathlineto{\pgfqpoint{0.826194in}{0.860430in}}%
\pgfpathlineto{\pgfqpoint{0.826298in}{1.024361in}}%
\pgfpathlineto{\pgfqpoint{0.827131in}{0.724199in}}%
\pgfpathlineto{\pgfqpoint{0.826611in}{1.235897in}}%
\pgfpathlineto{\pgfqpoint{0.827235in}{0.965072in}}%
\pgfpathlineto{\pgfqpoint{0.827444in}{1.230410in}}%
\pgfpathlineto{\pgfqpoint{0.827652in}{0.778585in}}%
\pgfpathlineto{\pgfqpoint{0.828277in}{0.965359in}}%
\pgfpathlineto{\pgfqpoint{0.828381in}{0.730268in}}%
\pgfpathlineto{\pgfqpoint{0.828589in}{1.241395in}}%
\pgfpathlineto{\pgfqpoint{0.829318in}{0.946352in}}%
\pgfpathlineto{\pgfqpoint{0.829631in}{1.267968in}}%
\pgfpathlineto{\pgfqpoint{0.830256in}{0.927714in}}%
\pgfpathlineto{\pgfqpoint{0.830464in}{1.156662in}}%
\pgfpathlineto{\pgfqpoint{0.831193in}{1.303821in}}%
\pgfpathlineto{\pgfqpoint{0.831609in}{0.902537in}}%
\pgfpathlineto{\pgfqpoint{0.831818in}{1.319880in}}%
\pgfpathlineto{\pgfqpoint{0.832547in}{0.847753in}}%
\pgfpathlineto{\pgfqpoint{0.832755in}{1.186676in}}%
\pgfpathlineto{\pgfqpoint{0.832963in}{1.277288in}}%
\pgfpathlineto{\pgfqpoint{0.833900in}{0.881526in}}%
\pgfpathlineto{\pgfqpoint{0.834525in}{1.178731in}}%
\pgfpathlineto{\pgfqpoint{0.834421in}{0.853397in}}%
\pgfpathlineto{\pgfqpoint{0.835150in}{1.092418in}}%
\pgfpathlineto{\pgfqpoint{0.835463in}{0.765945in}}%
\pgfpathlineto{\pgfqpoint{0.835983in}{1.153907in}}%
\pgfpathlineto{\pgfqpoint{0.836192in}{1.005789in}}%
\pgfpathlineto{\pgfqpoint{0.836816in}{1.165310in}}%
\pgfpathlineto{\pgfqpoint{0.836712in}{0.828661in}}%
\pgfpathlineto{\pgfqpoint{0.837233in}{1.048398in}}%
\pgfpathlineto{\pgfqpoint{0.837337in}{1.034419in}}%
\pgfpathlineto{\pgfqpoint{0.837441in}{1.103200in}}%
\pgfpathlineto{\pgfqpoint{0.838274in}{1.349686in}}%
\pgfpathlineto{\pgfqpoint{0.837754in}{0.861462in}}%
\pgfpathlineto{\pgfqpoint{0.838587in}{1.207649in}}%
\pgfpathlineto{\pgfqpoint{0.839628in}{0.745767in}}%
\pgfpathlineto{\pgfqpoint{0.839003in}{1.283769in}}%
\pgfpathlineto{\pgfqpoint{0.839732in}{1.146031in}}%
\pgfpathlineto{\pgfqpoint{0.840149in}{1.341152in}}%
\pgfpathlineto{\pgfqpoint{0.839941in}{0.967179in}}%
\pgfpathlineto{\pgfqpoint{0.840253in}{0.979898in}}%
\pgfpathlineto{\pgfqpoint{0.840357in}{0.898662in}}%
\pgfpathlineto{\pgfqpoint{0.840566in}{1.342390in}}%
\pgfpathlineto{\pgfqpoint{0.841086in}{0.982900in}}%
\pgfpathlineto{\pgfqpoint{0.841607in}{1.345984in}}%
\pgfpathlineto{\pgfqpoint{0.841503in}{0.924963in}}%
\pgfpathlineto{\pgfqpoint{0.842232in}{1.115096in}}%
\pgfpathlineto{\pgfqpoint{0.842440in}{1.060561in}}%
\pgfpathlineto{\pgfqpoint{0.842544in}{1.136137in}}%
\pgfpathlineto{\pgfqpoint{0.842648in}{0.682687in}}%
\pgfpathlineto{\pgfqpoint{0.843273in}{1.311974in}}%
\pgfpathlineto{\pgfqpoint{0.843586in}{1.032019in}}%
\pgfpathlineto{\pgfqpoint{0.843794in}{0.987841in}}%
\pgfpathlineto{\pgfqpoint{0.843898in}{1.082811in}}%
\pgfpathlineto{\pgfqpoint{0.844106in}{0.888132in}}%
\pgfpathlineto{\pgfqpoint{0.844523in}{1.185660in}}%
\pgfpathlineto{\pgfqpoint{0.844939in}{1.049234in}}%
\pgfpathlineto{\pgfqpoint{0.845044in}{1.120952in}}%
\pgfpathlineto{\pgfqpoint{0.845668in}{0.826745in}}%
\pgfpathlineto{\pgfqpoint{0.845981in}{1.016549in}}%
\pgfpathlineto{\pgfqpoint{0.846085in}{1.018705in}}%
\pgfpathlineto{\pgfqpoint{0.846397in}{1.182599in}}%
\pgfpathlineto{\pgfqpoint{0.846606in}{0.877472in}}%
\pgfpathlineto{\pgfqpoint{0.846918in}{1.002488in}}%
\pgfpathlineto{\pgfqpoint{0.847126in}{0.863827in}}%
\pgfpathlineto{\pgfqpoint{0.847439in}{1.141555in}}%
\pgfpathlineto{\pgfqpoint{0.847960in}{0.929573in}}%
\pgfpathlineto{\pgfqpoint{0.848584in}{1.449661in}}%
\pgfpathlineto{\pgfqpoint{0.848168in}{0.791903in}}%
\pgfpathlineto{\pgfqpoint{0.849105in}{1.042334in}}%
\pgfpathlineto{\pgfqpoint{0.849209in}{0.931830in}}%
\pgfpathlineto{\pgfqpoint{0.849938in}{1.323294in}}%
\pgfpathlineto{\pgfqpoint{0.850042in}{1.173734in}}%
\pgfpathlineto{\pgfqpoint{0.850459in}{0.771078in}}%
\pgfpathlineto{\pgfqpoint{0.850667in}{1.178288in}}%
\pgfpathlineto{\pgfqpoint{0.850980in}{1.046474in}}%
\pgfpathlineto{\pgfqpoint{0.851084in}{1.304430in}}%
\pgfpathlineto{\pgfqpoint{0.851604in}{0.803945in}}%
\pgfpathlineto{\pgfqpoint{0.852021in}{1.018428in}}%
\pgfpathlineto{\pgfqpoint{0.852125in}{1.010137in}}%
\pgfpathlineto{\pgfqpoint{0.852438in}{1.380014in}}%
\pgfpathlineto{\pgfqpoint{0.852750in}{0.831441in}}%
\pgfpathlineto{\pgfqpoint{0.853167in}{0.893024in}}%
\pgfpathlineto{\pgfqpoint{0.854104in}{1.233036in}}%
\pgfpathlineto{\pgfqpoint{0.853896in}{0.891914in}}%
\pgfpathlineto{\pgfqpoint{0.854520in}{1.227421in}}%
\pgfpathlineto{\pgfqpoint{0.854937in}{0.892291in}}%
\pgfpathlineto{\pgfqpoint{0.855562in}{1.132712in}}%
\pgfpathlineto{\pgfqpoint{0.855666in}{1.234655in}}%
\pgfpathlineto{\pgfqpoint{0.855978in}{0.747676in}}%
\pgfpathlineto{\pgfqpoint{0.856395in}{0.916638in}}%
\pgfpathlineto{\pgfqpoint{0.856812in}{0.725470in}}%
\pgfpathlineto{\pgfqpoint{0.857124in}{1.121385in}}%
\pgfpathlineto{\pgfqpoint{0.857228in}{0.650680in}}%
\pgfpathlineto{\pgfqpoint{0.857540in}{1.258058in}}%
\pgfpathlineto{\pgfqpoint{0.858165in}{1.085803in}}%
\pgfpathlineto{\pgfqpoint{0.858269in}{1.128813in}}%
\pgfpathlineto{\pgfqpoint{0.858374in}{1.630766in}}%
\pgfpathlineto{\pgfqpoint{0.858478in}{0.748104in}}%
\pgfpathlineto{\pgfqpoint{0.859311in}{0.831363in}}%
\pgfpathlineto{\pgfqpoint{0.860456in}{1.325652in}}%
\pgfpathlineto{\pgfqpoint{0.860977in}{0.948605in}}%
\pgfpathlineto{\pgfqpoint{0.861602in}{1.037101in}}%
\pgfpathlineto{\pgfqpoint{0.861810in}{1.285928in}}%
\pgfpathlineto{\pgfqpoint{0.862227in}{0.923579in}}%
\pgfpathlineto{\pgfqpoint{0.862539in}{1.270442in}}%
\pgfpathlineto{\pgfqpoint{0.863789in}{0.614901in}}%
\pgfpathlineto{\pgfqpoint{0.864622in}{1.310783in}}%
\pgfpathlineto{\pgfqpoint{0.864935in}{1.220944in}}%
\pgfpathlineto{\pgfqpoint{0.865559in}{1.233789in}}%
\pgfpathlineto{\pgfqpoint{0.866184in}{0.773342in}}%
\pgfpathlineto{\pgfqpoint{0.866497in}{1.346016in}}%
\pgfpathlineto{\pgfqpoint{0.867434in}{1.161742in}}%
\pgfpathlineto{\pgfqpoint{0.867642in}{0.811139in}}%
\pgfpathlineto{\pgfqpoint{0.868059in}{0.888192in}}%
\pgfpathlineto{\pgfqpoint{0.868163in}{1.408342in}}%
\pgfpathlineto{\pgfqpoint{0.868996in}{0.871809in}}%
\pgfpathlineto{\pgfqpoint{0.869100in}{1.103247in}}%
\pgfpathlineto{\pgfqpoint{0.869308in}{0.773995in}}%
\pgfpathlineto{\pgfqpoint{0.869621in}{1.331391in}}%
\pgfpathlineto{\pgfqpoint{0.870037in}{0.826938in}}%
\pgfpathlineto{\pgfqpoint{0.870454in}{1.248873in}}%
\pgfpathlineto{\pgfqpoint{0.871079in}{0.638248in}}%
\pgfpathlineto{\pgfqpoint{0.871183in}{0.987915in}}%
\pgfpathlineto{\pgfqpoint{0.871391in}{1.230437in}}%
\pgfpathlineto{\pgfqpoint{0.871808in}{0.779586in}}%
\pgfpathlineto{\pgfqpoint{0.872329in}{1.059228in}}%
\pgfpathlineto{\pgfqpoint{0.872433in}{0.688742in}}%
\pgfpathlineto{\pgfqpoint{0.872641in}{1.120771in}}%
\pgfpathlineto{\pgfqpoint{0.873370in}{1.023828in}}%
\pgfpathlineto{\pgfqpoint{0.873995in}{0.753122in}}%
\pgfpathlineto{\pgfqpoint{0.874411in}{1.229994in}}%
\pgfpathlineto{\pgfqpoint{0.874515in}{0.852730in}}%
\pgfpathlineto{\pgfqpoint{0.874932in}{1.308958in}}%
\pgfpathlineto{\pgfqpoint{0.875557in}{1.100791in}}%
\pgfpathlineto{\pgfqpoint{0.875973in}{0.831760in}}%
\pgfpathlineto{\pgfqpoint{0.876182in}{1.311895in}}%
\pgfpathlineto{\pgfqpoint{0.876598in}{0.867760in}}%
\pgfpathlineto{\pgfqpoint{0.877119in}{1.321166in}}%
\pgfpathlineto{\pgfqpoint{0.877744in}{1.074783in}}%
\pgfpathlineto{\pgfqpoint{0.877848in}{0.765292in}}%
\pgfpathlineto{\pgfqpoint{0.878369in}{1.233185in}}%
\pgfpathlineto{\pgfqpoint{0.878785in}{1.165739in}}%
\pgfpathlineto{\pgfqpoint{0.879098in}{0.927303in}}%
\pgfpathlineto{\pgfqpoint{0.879202in}{1.242069in}}%
\pgfpathlineto{\pgfqpoint{0.879827in}{1.179815in}}%
\pgfpathlineto{\pgfqpoint{0.879931in}{1.296408in}}%
\pgfpathlineto{\pgfqpoint{0.880347in}{0.864263in}}%
\pgfpathlineto{\pgfqpoint{0.880556in}{1.120345in}}%
\pgfpathlineto{\pgfqpoint{0.880660in}{0.779575in}}%
\pgfpathlineto{\pgfqpoint{0.881597in}{1.208180in}}%
\pgfpathlineto{\pgfqpoint{0.881701in}{1.297279in}}%
\pgfpathlineto{\pgfqpoint{0.881909in}{0.964269in}}%
\pgfpathlineto{\pgfqpoint{0.882326in}{1.162794in}}%
\pgfpathlineto{\pgfqpoint{0.882430in}{0.773768in}}%
\pgfpathlineto{\pgfqpoint{0.883159in}{1.273713in}}%
\pgfpathlineto{\pgfqpoint{0.883367in}{1.119228in}}%
\pgfpathlineto{\pgfqpoint{0.883472in}{1.193912in}}%
\pgfpathlineto{\pgfqpoint{0.883576in}{0.904205in}}%
\pgfpathlineto{\pgfqpoint{0.883888in}{1.178525in}}%
\pgfpathlineto{\pgfqpoint{0.883992in}{0.741772in}}%
\pgfpathlineto{\pgfqpoint{0.885034in}{0.999950in}}%
\pgfpathlineto{\pgfqpoint{0.885138in}{0.627259in}}%
\pgfpathlineto{\pgfqpoint{0.885763in}{1.249291in}}%
\pgfpathlineto{\pgfqpoint{0.886075in}{1.121123in}}%
\pgfpathlineto{\pgfqpoint{0.886492in}{0.796902in}}%
\pgfpathlineto{\pgfqpoint{0.887117in}{1.143496in}}%
\pgfpathlineto{\pgfqpoint{0.887221in}{0.860740in}}%
\pgfpathlineto{\pgfqpoint{0.888054in}{1.197859in}}%
\pgfpathlineto{\pgfqpoint{0.887637in}{0.838763in}}%
\pgfpathlineto{\pgfqpoint{0.888366in}{1.005592in}}%
\pgfpathlineto{\pgfqpoint{0.888679in}{0.878414in}}%
\pgfpathlineto{\pgfqpoint{0.888575in}{1.249575in}}%
\pgfpathlineto{\pgfqpoint{0.888887in}{1.185307in}}%
\pgfpathlineto{\pgfqpoint{0.889304in}{1.389247in}}%
\pgfpathlineto{\pgfqpoint{0.889616in}{0.778078in}}%
\pgfpathlineto{\pgfqpoint{0.889720in}{1.158837in}}%
\pgfpathlineto{\pgfqpoint{0.889824in}{0.777617in}}%
\pgfpathlineto{\pgfqpoint{0.890449in}{1.241981in}}%
\pgfpathlineto{\pgfqpoint{0.890761in}{0.940700in}}%
\pgfpathlineto{\pgfqpoint{0.891699in}{1.315840in}}%
\pgfpathlineto{\pgfqpoint{0.891803in}{0.810884in}}%
\pgfpathlineto{\pgfqpoint{0.892428in}{1.255443in}}%
\pgfpathlineto{\pgfqpoint{0.893053in}{1.168623in}}%
\pgfpathlineto{\pgfqpoint{0.893365in}{0.756666in}}%
\pgfpathlineto{\pgfqpoint{0.893573in}{1.334876in}}%
\pgfpathlineto{\pgfqpoint{0.894094in}{1.081296in}}%
\pgfpathlineto{\pgfqpoint{0.894406in}{1.475415in}}%
\pgfpathlineto{\pgfqpoint{0.894511in}{0.809125in}}%
\pgfpathlineto{\pgfqpoint{0.895135in}{1.087800in}}%
\pgfpathlineto{\pgfqpoint{0.895240in}{0.781324in}}%
\pgfpathlineto{\pgfqpoint{0.896177in}{1.228451in}}%
\pgfpathlineto{\pgfqpoint{0.896281in}{1.265106in}}%
\pgfpathlineto{\pgfqpoint{0.896385in}{1.019202in}}%
\pgfpathlineto{\pgfqpoint{0.896489in}{1.060926in}}%
\pgfpathlineto{\pgfqpoint{0.896593in}{0.921321in}}%
\pgfpathlineto{\pgfqpoint{0.896906in}{1.339252in}}%
\pgfpathlineto{\pgfqpoint{0.897531in}{0.925970in}}%
\pgfpathlineto{\pgfqpoint{0.897947in}{1.253227in}}%
\pgfpathlineto{\pgfqpoint{0.897843in}{0.919609in}}%
\pgfpathlineto{\pgfqpoint{0.898572in}{1.058581in}}%
\pgfpathlineto{\pgfqpoint{0.898989in}{1.274958in}}%
\pgfpathlineto{\pgfqpoint{0.899613in}{0.688818in}}%
\pgfpathlineto{\pgfqpoint{0.900134in}{1.300360in}}%
\pgfpathlineto{\pgfqpoint{0.900759in}{1.085983in}}%
\pgfpathlineto{\pgfqpoint{0.901592in}{0.880604in}}%
\pgfpathlineto{\pgfqpoint{0.901384in}{1.312813in}}%
\pgfpathlineto{\pgfqpoint{0.901696in}{1.120179in}}%
\pgfpathlineto{\pgfqpoint{0.902529in}{1.394885in}}%
\pgfpathlineto{\pgfqpoint{0.901905in}{0.905146in}}%
\pgfpathlineto{\pgfqpoint{0.902634in}{1.082974in}}%
\pgfpathlineto{\pgfqpoint{0.902842in}{0.826695in}}%
\pgfpathlineto{\pgfqpoint{0.903050in}{1.216283in}}%
\pgfpathlineto{\pgfqpoint{0.903779in}{0.827169in}}%
\pgfpathlineto{\pgfqpoint{0.904612in}{1.317979in}}%
\pgfpathlineto{\pgfqpoint{0.904821in}{1.018531in}}%
\pgfpathlineto{\pgfqpoint{0.904925in}{0.903441in}}%
\pgfpathlineto{\pgfqpoint{0.905237in}{1.293322in}}%
\pgfpathlineto{\pgfqpoint{0.905862in}{1.004106in}}%
\pgfpathlineto{\pgfqpoint{0.906591in}{1.121456in}}%
\pgfpathlineto{\pgfqpoint{0.906487in}{0.741551in}}%
\pgfpathlineto{\pgfqpoint{0.906799in}{0.884795in}}%
\pgfpathlineto{\pgfqpoint{0.907007in}{0.790725in}}%
\pgfpathlineto{\pgfqpoint{0.907216in}{0.950046in}}%
\pgfpathlineto{\pgfqpoint{0.907320in}{1.341020in}}%
\pgfpathlineto{\pgfqpoint{0.908257in}{0.807060in}}%
\pgfpathlineto{\pgfqpoint{0.908882in}{1.372124in}}%
\pgfpathlineto{\pgfqpoint{0.909403in}{0.971351in}}%
\pgfpathlineto{\pgfqpoint{0.909923in}{0.822577in}}%
\pgfpathlineto{\pgfqpoint{0.909715in}{1.180708in}}%
\pgfpathlineto{\pgfqpoint{0.910236in}{1.043058in}}%
\pgfpathlineto{\pgfqpoint{0.910757in}{1.387749in}}%
\pgfpathlineto{\pgfqpoint{0.910444in}{0.850628in}}%
\pgfpathlineto{\pgfqpoint{0.911277in}{1.039654in}}%
\pgfpathlineto{\pgfqpoint{0.911902in}{1.354766in}}%
\pgfpathlineto{\pgfqpoint{0.911486in}{0.682292in}}%
\pgfpathlineto{\pgfqpoint{0.912215in}{1.346644in}}%
\pgfpathlineto{\pgfqpoint{0.913464in}{0.793351in}}%
\pgfpathlineto{\pgfqpoint{0.914506in}{1.387893in}}%
\pgfpathlineto{\pgfqpoint{0.914610in}{1.184732in}}%
\pgfpathlineto{\pgfqpoint{0.914922in}{0.864973in}}%
\pgfpathlineto{\pgfqpoint{0.915443in}{1.287471in}}%
\pgfpathlineto{\pgfqpoint{0.916276in}{0.869845in}}%
\pgfpathlineto{\pgfqpoint{0.916693in}{1.355664in}}%
\pgfpathlineto{\pgfqpoint{0.916588in}{0.623960in}}%
\pgfpathlineto{\pgfqpoint{0.917526in}{1.204483in}}%
\pgfpathlineto{\pgfqpoint{0.917734in}{0.857863in}}%
\pgfpathlineto{\pgfqpoint{0.918775in}{0.865520in}}%
\pgfpathlineto{\pgfqpoint{0.919400in}{0.671881in}}%
\pgfpathlineto{\pgfqpoint{0.919921in}{1.278890in}}%
\pgfpathlineto{\pgfqpoint{0.920233in}{0.819552in}}%
\pgfpathlineto{\pgfqpoint{0.920858in}{1.338361in}}%
\pgfpathlineto{\pgfqpoint{0.921171in}{1.027273in}}%
\pgfpathlineto{\pgfqpoint{0.921379in}{1.227556in}}%
\pgfpathlineto{\pgfqpoint{0.921900in}{0.852614in}}%
\pgfpathlineto{\pgfqpoint{0.922629in}{1.305595in}}%
\pgfpathlineto{\pgfqpoint{0.922733in}{0.797518in}}%
\pgfpathlineto{\pgfqpoint{0.923045in}{0.937337in}}%
\pgfpathlineto{\pgfqpoint{0.923566in}{0.688016in}}%
\pgfpathlineto{\pgfqpoint{0.924191in}{1.357857in}}%
\pgfpathlineto{\pgfqpoint{0.924295in}{0.785175in}}%
\pgfpathlineto{\pgfqpoint{0.924920in}{1.361960in}}%
\pgfpathlineto{\pgfqpoint{0.925336in}{1.046970in}}%
\pgfpathlineto{\pgfqpoint{0.925440in}{1.030620in}}%
\pgfpathlineto{\pgfqpoint{0.925545in}{1.102758in}}%
\pgfpathlineto{\pgfqpoint{0.926274in}{1.269542in}}%
\pgfpathlineto{\pgfqpoint{0.926378in}{0.890927in}}%
\pgfpathlineto{\pgfqpoint{0.926482in}{1.048909in}}%
\pgfpathlineto{\pgfqpoint{0.927003in}{0.783942in}}%
\pgfpathlineto{\pgfqpoint{0.926898in}{1.423010in}}%
\pgfpathlineto{\pgfqpoint{0.927523in}{0.863371in}}%
\pgfpathlineto{\pgfqpoint{0.927836in}{0.732565in}}%
\pgfpathlineto{\pgfqpoint{0.928773in}{1.246844in}}%
\pgfpathlineto{\pgfqpoint{0.929085in}{0.867653in}}%
\pgfpathlineto{\pgfqpoint{0.929710in}{1.366208in}}%
\pgfpathlineto{\pgfqpoint{0.929814in}{1.249687in}}%
\pgfpathlineto{\pgfqpoint{0.929919in}{1.336190in}}%
\pgfpathlineto{\pgfqpoint{0.930231in}{0.864210in}}%
\pgfpathlineto{\pgfqpoint{0.930543in}{1.236212in}}%
\pgfpathlineto{\pgfqpoint{0.931064in}{0.836904in}}%
\pgfpathlineto{\pgfqpoint{0.930960in}{1.392476in}}%
\pgfpathlineto{\pgfqpoint{0.931689in}{1.020764in}}%
\pgfpathlineto{\pgfqpoint{0.932626in}{0.855976in}}%
\pgfpathlineto{\pgfqpoint{0.932001in}{1.355200in}}%
\pgfpathlineto{\pgfqpoint{0.932730in}{0.993193in}}%
\pgfpathlineto{\pgfqpoint{0.933668in}{1.229630in}}%
\pgfpathlineto{\pgfqpoint{0.933459in}{0.909398in}}%
\pgfpathlineto{\pgfqpoint{0.933772in}{1.185556in}}%
\pgfpathlineto{\pgfqpoint{0.934084in}{0.763654in}}%
\pgfpathlineto{\pgfqpoint{0.934292in}{1.332871in}}%
\pgfpathlineto{\pgfqpoint{0.934813in}{0.961246in}}%
\pgfpathlineto{\pgfqpoint{0.935230in}{1.163459in}}%
\pgfpathlineto{\pgfqpoint{0.935021in}{0.961021in}}%
\pgfpathlineto{\pgfqpoint{0.935855in}{1.121402in}}%
\pgfpathlineto{\pgfqpoint{0.936479in}{0.890565in}}%
\pgfpathlineto{\pgfqpoint{0.936792in}{1.169141in}}%
\pgfpathlineto{\pgfqpoint{0.936896in}{0.990857in}}%
\pgfpathlineto{\pgfqpoint{0.937729in}{0.747756in}}%
\pgfpathlineto{\pgfqpoint{0.937937in}{1.379096in}}%
\pgfpathlineto{\pgfqpoint{0.938562in}{0.869444in}}%
\pgfpathlineto{\pgfqpoint{0.939083in}{1.205353in}}%
\pgfpathlineto{\pgfqpoint{0.939187in}{1.182276in}}%
\pgfpathlineto{\pgfqpoint{0.939291in}{0.886310in}}%
\pgfpathlineto{\pgfqpoint{0.939916in}{1.229181in}}%
\pgfpathlineto{\pgfqpoint{0.940333in}{1.080413in}}%
\pgfpathlineto{\pgfqpoint{0.940749in}{1.419138in}}%
\pgfpathlineto{\pgfqpoint{0.940645in}{0.929263in}}%
\pgfpathlineto{\pgfqpoint{0.941478in}{1.209452in}}%
\pgfpathlineto{\pgfqpoint{0.942415in}{0.919602in}}%
\pgfpathlineto{\pgfqpoint{0.942311in}{1.267497in}}%
\pgfpathlineto{\pgfqpoint{0.942832in}{0.977134in}}%
\pgfpathlineto{\pgfqpoint{0.943873in}{1.530441in}}%
\pgfpathlineto{\pgfqpoint{0.943561in}{0.740984in}}%
\pgfpathlineto{\pgfqpoint{0.943978in}{1.393104in}}%
\pgfpathlineto{\pgfqpoint{0.945227in}{0.714028in}}%
\pgfpathlineto{\pgfqpoint{0.945331in}{1.180176in}}%
\pgfpathlineto{\pgfqpoint{0.946373in}{0.949214in}}%
\pgfpathlineto{\pgfqpoint{0.947206in}{1.264008in}}%
\pgfpathlineto{\pgfqpoint{0.946998in}{0.734453in}}%
\pgfpathlineto{\pgfqpoint{0.947623in}{1.238159in}}%
\pgfpathlineto{\pgfqpoint{0.947935in}{0.765020in}}%
\pgfpathlineto{\pgfqpoint{0.948143in}{1.354064in}}%
\pgfpathlineto{\pgfqpoint{0.948664in}{0.934773in}}%
\pgfpathlineto{\pgfqpoint{0.949185in}{1.245600in}}%
\pgfpathlineto{\pgfqpoint{0.949080in}{0.910931in}}%
\pgfpathlineto{\pgfqpoint{0.949705in}{0.939328in}}%
\pgfpathlineto{\pgfqpoint{0.949809in}{0.942603in}}%
\pgfpathlineto{\pgfqpoint{0.950330in}{0.749356in}}%
\pgfpathlineto{\pgfqpoint{0.950018in}{1.123420in}}%
\pgfpathlineto{\pgfqpoint{0.950851in}{0.882020in}}%
\pgfpathlineto{\pgfqpoint{0.951476in}{1.245566in}}%
\pgfpathlineto{\pgfqpoint{0.951996in}{1.055328in}}%
\pgfpathlineto{\pgfqpoint{0.952725in}{0.819159in}}%
\pgfpathlineto{\pgfqpoint{0.952309in}{1.085318in}}%
\pgfpathlineto{\pgfqpoint{0.953038in}{0.923058in}}%
\pgfpathlineto{\pgfqpoint{0.953559in}{1.173242in}}%
\pgfpathlineto{\pgfqpoint{0.953350in}{0.765082in}}%
\pgfpathlineto{\pgfqpoint{0.954079in}{0.868470in}}%
\pgfpathlineto{\pgfqpoint{0.954183in}{0.790840in}}%
\pgfpathlineto{\pgfqpoint{0.954808in}{1.093718in}}%
\pgfpathlineto{\pgfqpoint{0.954912in}{1.047578in}}%
\pgfpathlineto{\pgfqpoint{0.955537in}{1.227739in}}%
\pgfpathlineto{\pgfqpoint{0.955329in}{0.881441in}}%
\pgfpathlineto{\pgfqpoint{0.955746in}{1.024428in}}%
\pgfpathlineto{\pgfqpoint{0.956058in}{0.845197in}}%
\pgfpathlineto{\pgfqpoint{0.956370in}{1.235929in}}%
\pgfpathlineto{\pgfqpoint{0.956891in}{0.949933in}}%
\pgfpathlineto{\pgfqpoint{0.957203in}{1.290431in}}%
\pgfpathlineto{\pgfqpoint{0.957724in}{0.843702in}}%
\pgfpathlineto{\pgfqpoint{0.958037in}{1.244506in}}%
\pgfpathlineto{\pgfqpoint{0.958453in}{0.849189in}}%
\pgfpathlineto{\pgfqpoint{0.959182in}{1.022507in}}%
\pgfpathlineto{\pgfqpoint{0.959390in}{0.947191in}}%
\pgfpathlineto{\pgfqpoint{0.959495in}{1.101095in}}%
\pgfpathlineto{\pgfqpoint{0.959599in}{1.297016in}}%
\pgfpathlineto{\pgfqpoint{0.960224in}{0.837274in}}%
\pgfpathlineto{\pgfqpoint{0.960536in}{1.193687in}}%
\pgfpathlineto{\pgfqpoint{0.961577in}{0.731654in}}%
\pgfpathlineto{\pgfqpoint{0.961682in}{1.030361in}}%
\pgfpathlineto{\pgfqpoint{0.961890in}{1.178942in}}%
\pgfpathlineto{\pgfqpoint{0.962202in}{0.936693in}}%
\pgfpathlineto{\pgfqpoint{0.962306in}{0.773967in}}%
\pgfpathlineto{\pgfqpoint{0.962619in}{1.258809in}}%
\pgfpathlineto{\pgfqpoint{0.963140in}{1.035070in}}%
\pgfpathlineto{\pgfqpoint{0.963244in}{1.299285in}}%
\pgfpathlineto{\pgfqpoint{0.964077in}{0.846328in}}%
\pgfpathlineto{\pgfqpoint{0.964285in}{1.202418in}}%
\pgfpathlineto{\pgfqpoint{0.965118in}{1.326664in}}%
\pgfpathlineto{\pgfqpoint{0.965431in}{0.817500in}}%
\pgfpathlineto{\pgfqpoint{0.966264in}{1.278572in}}%
\pgfpathlineto{\pgfqpoint{0.966680in}{1.111197in}}%
\pgfpathlineto{\pgfqpoint{0.966784in}{1.237669in}}%
\pgfpathlineto{\pgfqpoint{0.966889in}{0.858443in}}%
\pgfpathlineto{\pgfqpoint{0.967409in}{0.966500in}}%
\pgfpathlineto{\pgfqpoint{0.967513in}{0.703137in}}%
\pgfpathlineto{\pgfqpoint{0.967722in}{1.188320in}}%
\pgfpathlineto{\pgfqpoint{0.968347in}{0.931650in}}%
\pgfpathlineto{\pgfqpoint{0.968451in}{1.374920in}}%
\pgfpathlineto{\pgfqpoint{0.969388in}{1.331795in}}%
\pgfpathlineto{\pgfqpoint{0.970013in}{0.776662in}}%
\pgfpathlineto{\pgfqpoint{0.970117in}{1.350128in}}%
\pgfpathlineto{\pgfqpoint{0.970534in}{1.105212in}}%
\pgfpathlineto{\pgfqpoint{0.971054in}{1.266717in}}%
\pgfpathlineto{\pgfqpoint{0.971263in}{0.900021in}}%
\pgfpathlineto{\pgfqpoint{0.971575in}{1.517072in}}%
\pgfpathlineto{\pgfqpoint{0.971887in}{0.748400in}}%
\pgfpathlineto{\pgfqpoint{0.972304in}{0.940413in}}%
\pgfpathlineto{\pgfqpoint{0.973033in}{0.749029in}}%
\pgfpathlineto{\pgfqpoint{0.972720in}{1.164475in}}%
\pgfpathlineto{\pgfqpoint{0.973137in}{1.007132in}}%
\pgfpathlineto{\pgfqpoint{0.973241in}{1.295512in}}%
\pgfpathlineto{\pgfqpoint{0.974074in}{0.657011in}}%
\pgfpathlineto{\pgfqpoint{0.974178in}{0.961667in}}%
\pgfpathlineto{\pgfqpoint{0.974595in}{0.847754in}}%
\pgfpathlineto{\pgfqpoint{0.975532in}{1.372841in}}%
\pgfpathlineto{\pgfqpoint{0.975741in}{0.782214in}}%
\pgfpathlineto{\pgfqpoint{0.976678in}{0.846187in}}%
\pgfpathlineto{\pgfqpoint{0.977094in}{1.194669in}}%
\pgfpathlineto{\pgfqpoint{0.976886in}{0.637232in}}%
\pgfpathlineto{\pgfqpoint{0.977719in}{1.166645in}}%
\pgfpathlineto{\pgfqpoint{0.977823in}{0.816866in}}%
\pgfpathlineto{\pgfqpoint{0.978344in}{1.467522in}}%
\pgfpathlineto{\pgfqpoint{0.978761in}{1.080788in}}%
\pgfpathlineto{\pgfqpoint{0.979281in}{1.268904in}}%
\pgfpathlineto{\pgfqpoint{0.979073in}{0.904421in}}%
\pgfpathlineto{\pgfqpoint{0.979802in}{1.095183in}}%
\pgfpathlineto{\pgfqpoint{0.980635in}{1.208956in}}%
\pgfpathlineto{\pgfqpoint{0.980948in}{0.828365in}}%
\pgfpathlineto{\pgfqpoint{0.981468in}{0.759361in}}%
\pgfpathlineto{\pgfqpoint{0.981989in}{1.300696in}}%
\pgfpathlineto{\pgfqpoint{0.982093in}{0.773433in}}%
\pgfpathlineto{\pgfqpoint{0.983135in}{1.086708in}}%
\pgfpathlineto{\pgfqpoint{0.984072in}{1.252585in}}%
\pgfpathlineto{\pgfqpoint{0.983864in}{0.833852in}}%
\pgfpathlineto{\pgfqpoint{0.984176in}{1.184314in}}%
\pgfpathlineto{\pgfqpoint{0.985113in}{1.283089in}}%
\pgfpathlineto{\pgfqpoint{0.985217in}{0.829411in}}%
\pgfpathlineto{\pgfqpoint{0.985322in}{1.293495in}}%
\pgfpathlineto{\pgfqpoint{0.985634in}{0.776609in}}%
\pgfpathlineto{\pgfqpoint{0.986363in}{1.099355in}}%
\pgfpathlineto{\pgfqpoint{0.986467in}{0.893286in}}%
\pgfpathlineto{\pgfqpoint{0.986675in}{1.318326in}}%
\pgfpathlineto{\pgfqpoint{0.987509in}{0.997154in}}%
\pgfpathlineto{\pgfqpoint{0.988133in}{1.294775in}}%
\pgfpathlineto{\pgfqpoint{0.988342in}{1.099939in}}%
\pgfpathlineto{\pgfqpoint{0.988550in}{0.667829in}}%
\pgfpathlineto{\pgfqpoint{0.989175in}{1.219793in}}%
\pgfpathlineto{\pgfqpoint{0.989383in}{0.944479in}}%
\pgfpathlineto{\pgfqpoint{0.989487in}{1.199583in}}%
\pgfpathlineto{\pgfqpoint{0.990424in}{0.784488in}}%
\pgfpathlineto{\pgfqpoint{0.990945in}{1.199027in}}%
\pgfpathlineto{\pgfqpoint{0.991674in}{1.126351in}}%
\pgfpathlineto{\pgfqpoint{0.992403in}{0.929876in}}%
\pgfpathlineto{\pgfqpoint{0.991987in}{1.320482in}}%
\pgfpathlineto{\pgfqpoint{0.992716in}{1.106962in}}%
\pgfpathlineto{\pgfqpoint{0.992820in}{1.399149in}}%
\pgfpathlineto{\pgfqpoint{0.993549in}{0.697838in}}%
\pgfpathlineto{\pgfqpoint{0.993861in}{1.215317in}}%
\pgfpathlineto{\pgfqpoint{0.994069in}{0.612478in}}%
\pgfpathlineto{\pgfqpoint{0.995007in}{0.634948in}}%
\pgfpathlineto{\pgfqpoint{0.995944in}{1.326832in}}%
\pgfpathlineto{\pgfqpoint{0.996152in}{1.101819in}}%
\pgfpathlineto{\pgfqpoint{0.996361in}{0.702895in}}%
\pgfpathlineto{\pgfqpoint{0.996985in}{1.328619in}}%
\pgfpathlineto{\pgfqpoint{0.997194in}{1.094788in}}%
\pgfpathlineto{\pgfqpoint{0.997402in}{0.823475in}}%
\pgfpathlineto{\pgfqpoint{0.997506in}{1.253246in}}%
\pgfpathlineto{\pgfqpoint{0.997923in}{0.880958in}}%
\pgfpathlineto{\pgfqpoint{0.998235in}{1.267594in}}%
\pgfpathlineto{\pgfqpoint{0.998443in}{0.783675in}}%
\pgfpathlineto{\pgfqpoint{0.999068in}{1.078298in}}%
\pgfpathlineto{\pgfqpoint{0.999172in}{1.376701in}}%
\pgfpathlineto{\pgfqpoint{0.999797in}{0.893858in}}%
\pgfpathlineto{\pgfqpoint{1.000214in}{1.263368in}}%
\pgfpathlineto{\pgfqpoint{1.001255in}{0.761113in}}%
\pgfpathlineto{\pgfqpoint{1.001359in}{0.889508in}}%
\pgfpathlineto{\pgfqpoint{1.001984in}{0.757255in}}%
\pgfpathlineto{\pgfqpoint{1.002609in}{1.361571in}}%
\pgfpathlineto{\pgfqpoint{1.002713in}{0.832140in}}%
\pgfpathlineto{\pgfqpoint{1.003755in}{0.990976in}}%
\pgfpathlineto{\pgfqpoint{1.004171in}{0.829895in}}%
\pgfpathlineto{\pgfqpoint{1.004067in}{1.231369in}}%
\pgfpathlineto{\pgfqpoint{1.004900in}{0.911114in}}%
\pgfpathlineto{\pgfqpoint{1.005421in}{1.161663in}}%
\pgfpathlineto{\pgfqpoint{1.005837in}{0.797801in}}%
\pgfpathlineto{\pgfqpoint{1.006046in}{1.110480in}}%
\pgfpathlineto{\pgfqpoint{1.006358in}{0.888590in}}%
\pgfpathlineto{\pgfqpoint{1.006983in}{1.233377in}}%
\pgfpathlineto{\pgfqpoint{1.007191in}{0.942189in}}%
\pgfpathlineto{\pgfqpoint{1.007608in}{1.303744in}}%
\pgfpathlineto{\pgfqpoint{1.007920in}{0.884743in}}%
\pgfpathlineto{\pgfqpoint{1.008233in}{1.123639in}}%
\pgfpathlineto{\pgfqpoint{1.008649in}{0.751145in}}%
\pgfpathlineto{\pgfqpoint{1.009066in}{1.340298in}}%
\pgfpathlineto{\pgfqpoint{1.009378in}{1.058182in}}%
\pgfpathlineto{\pgfqpoint{1.009482in}{1.047214in}}%
\pgfpathlineto{\pgfqpoint{1.009899in}{0.755137in}}%
\pgfpathlineto{\pgfqpoint{1.010420in}{1.159301in}}%
\pgfpathlineto{\pgfqpoint{1.010524in}{1.087512in}}%
\pgfpathlineto{\pgfqpoint{1.011044in}{0.933070in}}%
\pgfpathlineto{\pgfqpoint{1.011253in}{1.229735in}}%
\pgfpathlineto{\pgfqpoint{1.011357in}{0.966551in}}%
\pgfpathlineto{\pgfqpoint{1.011461in}{1.362267in}}%
\pgfpathlineto{\pgfqpoint{1.011982in}{0.798313in}}%
\pgfpathlineto{\pgfqpoint{1.012502in}{1.149041in}}%
\pgfpathlineto{\pgfqpoint{1.012607in}{0.705876in}}%
\pgfpathlineto{\pgfqpoint{1.012711in}{1.354063in}}%
\pgfpathlineto{\pgfqpoint{1.013648in}{0.953385in}}%
\pgfpathlineto{\pgfqpoint{1.013752in}{0.843773in}}%
\pgfpathlineto{\pgfqpoint{1.014481in}{1.177481in}}%
\pgfpathlineto{\pgfqpoint{1.014793in}{0.912571in}}%
\pgfpathlineto{\pgfqpoint{1.014898in}{1.212784in}}%
\pgfpathlineto{\pgfqpoint{1.015939in}{1.133607in}}%
\pgfpathlineto{\pgfqpoint{1.016043in}{1.182842in}}%
\pgfpathlineto{\pgfqpoint{1.016356in}{0.898731in}}%
\pgfpathlineto{\pgfqpoint{1.016460in}{0.687313in}}%
\pgfpathlineto{\pgfqpoint{1.017189in}{1.255321in}}%
\pgfpathlineto{\pgfqpoint{1.017293in}{0.894676in}}%
\pgfpathlineto{\pgfqpoint{1.017397in}{1.238535in}}%
\pgfpathlineto{\pgfqpoint{1.018334in}{0.726610in}}%
\pgfpathlineto{\pgfqpoint{1.018438in}{1.087461in}}%
\pgfpathlineto{\pgfqpoint{1.019272in}{0.854895in}}%
\pgfpathlineto{\pgfqpoint{1.018855in}{1.226499in}}%
\pgfpathlineto{\pgfqpoint{1.019480in}{1.135219in}}%
\pgfpathlineto{\pgfqpoint{1.019584in}{1.154067in}}%
\pgfpathlineto{\pgfqpoint{1.019688in}{0.769745in}}%
\pgfpathlineto{\pgfqpoint{1.020209in}{1.273973in}}%
\pgfpathlineto{\pgfqpoint{1.020625in}{1.097725in}}%
\pgfpathlineto{\pgfqpoint{1.021459in}{1.325762in}}%
\pgfpathlineto{\pgfqpoint{1.021146in}{0.826320in}}%
\pgfpathlineto{\pgfqpoint{1.021667in}{1.224941in}}%
\pgfpathlineto{\pgfqpoint{1.021875in}{0.884343in}}%
\pgfpathlineto{\pgfqpoint{1.022604in}{1.244348in}}%
\pgfpathlineto{\pgfqpoint{1.022708in}{1.134093in}}%
\pgfpathlineto{\pgfqpoint{1.022812in}{1.178382in}}%
\pgfpathlineto{\pgfqpoint{1.023021in}{0.986444in}}%
\pgfpathlineto{\pgfqpoint{1.023125in}{1.165277in}}%
\pgfpathlineto{\pgfqpoint{1.023958in}{0.819969in}}%
\pgfpathlineto{\pgfqpoint{1.023333in}{1.208942in}}%
\pgfpathlineto{\pgfqpoint{1.024166in}{0.969299in}}%
\pgfpathlineto{\pgfqpoint{1.024999in}{1.300505in}}%
\pgfpathlineto{\pgfqpoint{1.024583in}{0.913545in}}%
\pgfpathlineto{\pgfqpoint{1.025208in}{1.212313in}}%
\pgfpathlineto{\pgfqpoint{1.025312in}{0.735538in}}%
\pgfpathlineto{\pgfqpoint{1.026249in}{1.093178in}}%
\pgfpathlineto{\pgfqpoint{1.026666in}{1.209518in}}%
\pgfpathlineto{\pgfqpoint{1.026561in}{0.879852in}}%
\pgfpathlineto{\pgfqpoint{1.026978in}{1.028042in}}%
\pgfpathlineto{\pgfqpoint{1.027082in}{0.920118in}}%
\pgfpathlineto{\pgfqpoint{1.027707in}{1.211944in}}%
\pgfpathlineto{\pgfqpoint{1.027811in}{1.127070in}}%
\pgfpathlineto{\pgfqpoint{1.027915in}{1.358908in}}%
\pgfpathlineto{\pgfqpoint{1.028228in}{0.850352in}}%
\pgfpathlineto{\pgfqpoint{1.028853in}{1.127721in}}%
\pgfpathlineto{\pgfqpoint{1.029894in}{0.786731in}}%
\pgfpathlineto{\pgfqpoint{1.029790in}{1.339836in}}%
\pgfpathlineto{\pgfqpoint{1.029998in}{0.966399in}}%
\pgfpathlineto{\pgfqpoint{1.030727in}{1.258769in}}%
\pgfpathlineto{\pgfqpoint{1.031039in}{0.912283in}}%
\pgfpathlineto{\pgfqpoint{1.031144in}{1.042512in}}%
\pgfpathlineto{\pgfqpoint{1.031664in}{1.150421in}}%
\pgfpathlineto{\pgfqpoint{1.031768in}{0.897005in}}%
\pgfpathlineto{\pgfqpoint{1.031873in}{1.306624in}}%
\pgfpathlineto{\pgfqpoint{1.032497in}{0.873656in}}%
\pgfpathlineto{\pgfqpoint{1.032810in}{0.966583in}}%
\pgfpathlineto{\pgfqpoint{1.033539in}{1.321530in}}%
\pgfpathlineto{\pgfqpoint{1.033435in}{0.881325in}}%
\pgfpathlineto{\pgfqpoint{1.033851in}{1.084933in}}%
\pgfpathlineto{\pgfqpoint{1.034789in}{0.895588in}}%
\pgfpathlineto{\pgfqpoint{1.034372in}{1.229387in}}%
\pgfpathlineto{\pgfqpoint{1.034893in}{0.964061in}}%
\pgfpathlineto{\pgfqpoint{1.036142in}{1.294291in}}%
\pgfpathlineto{\pgfqpoint{1.035205in}{0.735260in}}%
\pgfpathlineto{\pgfqpoint{1.036247in}{1.246802in}}%
\pgfpathlineto{\pgfqpoint{1.037080in}{0.757530in}}%
\pgfpathlineto{\pgfqpoint{1.037184in}{1.265930in}}%
\pgfpathlineto{\pgfqpoint{1.037392in}{1.062374in}}%
\pgfpathlineto{\pgfqpoint{1.037705in}{0.637463in}}%
\pgfpathlineto{\pgfqpoint{1.037600in}{1.222473in}}%
\pgfpathlineto{\pgfqpoint{1.038017in}{0.987551in}}%
\pgfpathlineto{\pgfqpoint{1.038121in}{1.307679in}}%
\pgfpathlineto{\pgfqpoint{1.039058in}{0.827285in}}%
\pgfpathlineto{\pgfqpoint{1.039371in}{1.177202in}}%
\pgfpathlineto{\pgfqpoint{1.039579in}{0.626191in}}%
\pgfpathlineto{\pgfqpoint{1.040204in}{1.073059in}}%
\pgfpathlineto{\pgfqpoint{1.040933in}{0.816745in}}%
\pgfpathlineto{\pgfqpoint{1.040620in}{1.291535in}}%
\pgfpathlineto{\pgfqpoint{1.041141in}{1.017659in}}%
\pgfpathlineto{\pgfqpoint{1.041662in}{1.370637in}}%
\pgfpathlineto{\pgfqpoint{1.041870in}{0.915975in}}%
\pgfpathlineto{\pgfqpoint{1.042183in}{1.271807in}}%
\pgfpathlineto{\pgfqpoint{1.042912in}{1.298739in}}%
\pgfpathlineto{\pgfqpoint{1.043328in}{0.814597in}}%
\pgfpathlineto{\pgfqpoint{1.044474in}{1.270202in}}%
\pgfpathlineto{\pgfqpoint{1.045099in}{0.842167in}}%
\pgfpathlineto{\pgfqpoint{1.045619in}{1.131549in}}%
\pgfpathlineto{\pgfqpoint{1.046452in}{0.841932in}}%
\pgfpathlineto{\pgfqpoint{1.046036in}{1.241056in}}%
\pgfpathlineto{\pgfqpoint{1.046661in}{1.139051in}}%
\pgfpathlineto{\pgfqpoint{1.046765in}{1.138960in}}%
\pgfpathlineto{\pgfqpoint{1.047390in}{0.964830in}}%
\pgfpathlineto{\pgfqpoint{1.046973in}{1.260789in}}%
\pgfpathlineto{\pgfqpoint{1.047910in}{1.048605in}}%
\pgfpathlineto{\pgfqpoint{1.048014in}{1.404920in}}%
\pgfpathlineto{\pgfqpoint{1.048743in}{1.009590in}}%
\pgfpathlineto{\pgfqpoint{1.048848in}{1.042287in}}%
\pgfpathlineto{\pgfqpoint{1.048952in}{0.588571in}}%
\pgfpathlineto{\pgfqpoint{1.049264in}{1.288065in}}%
\pgfpathlineto{\pgfqpoint{1.049889in}{1.096898in}}%
\pgfpathlineto{\pgfqpoint{1.050306in}{0.853707in}}%
\pgfpathlineto{\pgfqpoint{1.050930in}{1.289779in}}%
\pgfpathlineto{\pgfqpoint{1.052180in}{0.834288in}}%
\pgfpathlineto{\pgfqpoint{1.052284in}{1.242285in}}%
\pgfpathlineto{\pgfqpoint{1.053326in}{1.017050in}}%
\pgfpathlineto{\pgfqpoint{1.054055in}{0.713416in}}%
\pgfpathlineto{\pgfqpoint{1.054367in}{1.230759in}}%
\pgfpathlineto{\pgfqpoint{1.054575in}{0.587649in}}%
\pgfpathlineto{\pgfqpoint{1.055096in}{1.458793in}}%
\pgfpathlineto{\pgfqpoint{1.055513in}{0.747195in}}%
\pgfpathlineto{\pgfqpoint{1.056658in}{1.162724in}}%
\pgfpathlineto{\pgfqpoint{1.056762in}{1.128630in}}%
\pgfpathlineto{\pgfqpoint{1.057595in}{1.263780in}}%
\pgfpathlineto{\pgfqpoint{1.057908in}{0.773077in}}%
\pgfpathlineto{\pgfqpoint{1.058429in}{1.248082in}}%
\pgfpathlineto{\pgfqpoint{1.059053in}{0.929977in}}%
\pgfpathlineto{\pgfqpoint{1.059678in}{1.189794in}}%
\pgfpathlineto{\pgfqpoint{1.059991in}{0.879949in}}%
\pgfpathlineto{\pgfqpoint{1.060199in}{1.145183in}}%
\pgfpathlineto{\pgfqpoint{1.060303in}{0.773480in}}%
\pgfpathlineto{\pgfqpoint{1.060720in}{1.252553in}}%
\pgfpathlineto{\pgfqpoint{1.061240in}{1.033368in}}%
\pgfpathlineto{\pgfqpoint{1.062074in}{1.428510in}}%
\pgfpathlineto{\pgfqpoint{1.061969in}{0.668698in}}%
\pgfpathlineto{\pgfqpoint{1.062490in}{1.398352in}}%
\pgfpathlineto{\pgfqpoint{1.063531in}{0.806765in}}%
\pgfpathlineto{\pgfqpoint{1.063740in}{1.048835in}}%
\pgfpathlineto{\pgfqpoint{1.063844in}{1.075749in}}%
\pgfpathlineto{\pgfqpoint{1.063948in}{0.977492in}}%
\pgfpathlineto{\pgfqpoint{1.064052in}{1.506029in}}%
\pgfpathlineto{\pgfqpoint{1.064677in}{0.857371in}}%
\pgfpathlineto{\pgfqpoint{1.065094in}{1.200114in}}%
\pgfpathlineto{\pgfqpoint{1.065823in}{1.290540in}}%
\pgfpathlineto{\pgfqpoint{1.065302in}{0.928991in}}%
\pgfpathlineto{\pgfqpoint{1.066031in}{1.147567in}}%
\pgfpathlineto{\pgfqpoint{1.066760in}{0.806202in}}%
\pgfpathlineto{\pgfqpoint{1.066968in}{1.152698in}}%
\pgfpathlineto{\pgfqpoint{1.067176in}{0.996621in}}%
\pgfpathlineto{\pgfqpoint{1.067905in}{0.884772in}}%
\pgfpathlineto{\pgfqpoint{1.067697in}{1.190490in}}%
\pgfpathlineto{\pgfqpoint{1.068218in}{0.978061in}}%
\pgfpathlineto{\pgfqpoint{1.069259in}{1.298501in}}%
\pgfpathlineto{\pgfqpoint{1.068634in}{0.867380in}}%
\pgfpathlineto{\pgfqpoint{1.069363in}{1.098997in}}%
\pgfpathlineto{\pgfqpoint{1.069572in}{1.261854in}}%
\pgfpathlineto{\pgfqpoint{1.069780in}{0.980561in}}%
\pgfpathlineto{\pgfqpoint{1.069988in}{1.089204in}}%
\pgfpathlineto{\pgfqpoint{1.070405in}{0.732369in}}%
\pgfpathlineto{\pgfqpoint{1.070509in}{1.279520in}}%
\pgfpathlineto{\pgfqpoint{1.071030in}{1.234803in}}%
\pgfpathlineto{\pgfqpoint{1.071238in}{0.848560in}}%
\pgfpathlineto{\pgfqpoint{1.071759in}{1.310897in}}%
\pgfpathlineto{\pgfqpoint{1.072383in}{1.089446in}}%
\pgfpathlineto{\pgfqpoint{1.072488in}{1.238202in}}%
\pgfpathlineto{\pgfqpoint{1.072904in}{0.850142in}}%
\pgfpathlineto{\pgfqpoint{1.073425in}{1.007360in}}%
\pgfpathlineto{\pgfqpoint{1.073529in}{0.902935in}}%
\pgfpathlineto{\pgfqpoint{1.074154in}{1.218539in}}%
\pgfpathlineto{\pgfqpoint{1.074258in}{1.010927in}}%
\pgfpathlineto{\pgfqpoint{1.074466in}{1.278678in}}%
\pgfpathlineto{\pgfqpoint{1.074779in}{0.747842in}}%
\pgfpathlineto{\pgfqpoint{1.075299in}{1.111387in}}%
\pgfpathlineto{\pgfqpoint{1.075404in}{0.728086in}}%
\pgfpathlineto{\pgfqpoint{1.076028in}{1.294399in}}%
\pgfpathlineto{\pgfqpoint{1.076341in}{1.082455in}}%
\pgfpathlineto{\pgfqpoint{1.077278in}{1.511323in}}%
\pgfpathlineto{\pgfqpoint{1.076549in}{0.745865in}}%
\pgfpathlineto{\pgfqpoint{1.077382in}{1.172836in}}%
\pgfpathlineto{\pgfqpoint{1.077799in}{0.790468in}}%
\pgfpathlineto{\pgfqpoint{1.078215in}{1.208147in}}%
\pgfpathlineto{\pgfqpoint{1.078528in}{1.033386in}}%
\pgfpathlineto{\pgfqpoint{1.079361in}{1.354079in}}%
\pgfpathlineto{\pgfqpoint{1.079257in}{0.745073in}}%
\pgfpathlineto{\pgfqpoint{1.079569in}{1.112641in}}%
\pgfpathlineto{\pgfqpoint{1.079986in}{0.819288in}}%
\pgfpathlineto{\pgfqpoint{1.080090in}{1.388577in}}%
\pgfpathlineto{\pgfqpoint{1.080402in}{0.989515in}}%
\pgfpathlineto{\pgfqpoint{1.081235in}{1.405799in}}%
\pgfpathlineto{\pgfqpoint{1.080715in}{0.872635in}}%
\pgfpathlineto{\pgfqpoint{1.081548in}{1.292649in}}%
\pgfpathlineto{\pgfqpoint{1.082381in}{0.749517in}}%
\pgfpathlineto{\pgfqpoint{1.082693in}{1.026755in}}%
\pgfpathlineto{\pgfqpoint{1.082798in}{1.161238in}}%
\pgfpathlineto{\pgfqpoint{1.083214in}{0.875507in}}%
\pgfpathlineto{\pgfqpoint{1.083735in}{0.991919in}}%
\pgfpathlineto{\pgfqpoint{1.083943in}{1.207917in}}%
\pgfpathlineto{\pgfqpoint{1.084047in}{1.004937in}}%
\pgfpathlineto{\pgfqpoint{1.084151in}{0.857226in}}%
\pgfpathlineto{\pgfqpoint{1.084568in}{1.290255in}}%
\pgfpathlineto{\pgfqpoint{1.084985in}{1.021910in}}%
\pgfpathlineto{\pgfqpoint{1.085818in}{1.271236in}}%
\pgfpathlineto{\pgfqpoint{1.085193in}{0.825753in}}%
\pgfpathlineto{\pgfqpoint{1.086130in}{1.066575in}}%
\pgfpathlineto{\pgfqpoint{1.086963in}{0.786779in}}%
\pgfpathlineto{\pgfqpoint{1.086443in}{1.395565in}}%
\pgfpathlineto{\pgfqpoint{1.087172in}{0.849752in}}%
\pgfpathlineto{\pgfqpoint{1.087276in}{1.304430in}}%
\pgfpathlineto{\pgfqpoint{1.088317in}{1.065643in}}%
\pgfpathlineto{\pgfqpoint{1.089150in}{1.427096in}}%
\pgfpathlineto{\pgfqpoint{1.088838in}{0.896492in}}%
\pgfpathlineto{\pgfqpoint{1.089254in}{1.048312in}}%
\pgfpathlineto{\pgfqpoint{1.089879in}{0.821780in}}%
\pgfpathlineto{\pgfqpoint{1.089671in}{1.299916in}}%
\pgfpathlineto{\pgfqpoint{1.090296in}{1.016337in}}%
\pgfpathlineto{\pgfqpoint{1.091233in}{1.178094in}}%
\pgfpathlineto{\pgfqpoint{1.090608in}{0.953774in}}%
\pgfpathlineto{\pgfqpoint{1.091441in}{1.120783in}}%
\pgfpathlineto{\pgfqpoint{1.091545in}{0.821810in}}%
\pgfpathlineto{\pgfqpoint{1.092379in}{1.206003in}}%
\pgfpathlineto{\pgfqpoint{1.092483in}{0.994332in}}%
\pgfpathlineto{\pgfqpoint{1.093212in}{1.388391in}}%
\pgfpathlineto{\pgfqpoint{1.093316in}{0.929015in}}%
\pgfpathlineto{\pgfqpoint{1.093524in}{0.976453in}}%
\pgfpathlineto{\pgfqpoint{1.093628in}{0.926509in}}%
\pgfpathlineto{\pgfqpoint{1.093837in}{1.074342in}}%
\pgfpathlineto{\pgfqpoint{1.093941in}{1.482725in}}%
\pgfpathlineto{\pgfqpoint{1.094357in}{0.749495in}}%
\pgfpathlineto{\pgfqpoint{1.094878in}{0.980218in}}%
\pgfpathlineto{\pgfqpoint{1.095399in}{0.867135in}}%
\pgfpathlineto{\pgfqpoint{1.095815in}{1.318723in}}%
\pgfpathlineto{\pgfqpoint{1.096857in}{0.767058in}}%
\pgfpathlineto{\pgfqpoint{1.096961in}{0.999989in}}%
\pgfpathlineto{\pgfqpoint{1.097273in}{1.308405in}}%
\pgfpathlineto{\pgfqpoint{1.097794in}{0.782988in}}%
\pgfpathlineto{\pgfqpoint{1.098002in}{1.205351in}}%
\pgfpathlineto{\pgfqpoint{1.098835in}{1.403942in}}%
\pgfpathlineto{\pgfqpoint{1.099252in}{0.903514in}}%
\pgfpathlineto{\pgfqpoint{1.099981in}{1.128229in}}%
\pgfpathlineto{\pgfqpoint{1.099564in}{0.895783in}}%
\pgfpathlineto{\pgfqpoint{1.100293in}{0.978700in}}%
\pgfpathlineto{\pgfqpoint{1.100397in}{0.878096in}}%
\pgfpathlineto{\pgfqpoint{1.100502in}{1.316277in}}%
\pgfpathlineto{\pgfqpoint{1.100814in}{0.917771in}}%
\pgfpathlineto{\pgfqpoint{1.100918in}{1.485292in}}%
\pgfpathlineto{\pgfqpoint{1.101022in}{0.896545in}}%
\pgfpathlineto{\pgfqpoint{1.101855in}{1.016436in}}%
\pgfpathlineto{\pgfqpoint{1.102064in}{1.357904in}}%
\pgfpathlineto{\pgfqpoint{1.102168in}{0.848893in}}%
\pgfpathlineto{\pgfqpoint{1.103001in}{1.196924in}}%
\pgfpathlineto{\pgfqpoint{1.103313in}{0.825252in}}%
\pgfpathlineto{\pgfqpoint{1.104251in}{0.993595in}}%
\pgfpathlineto{\pgfqpoint{1.104459in}{0.815233in}}%
\pgfpathlineto{\pgfqpoint{1.104563in}{1.351202in}}%
\pgfpathlineto{\pgfqpoint{1.104980in}{1.115478in}}%
\pgfpathlineto{\pgfqpoint{1.105500in}{0.857990in}}%
\pgfpathlineto{\pgfqpoint{1.105813in}{1.434223in}}%
\pgfpathlineto{\pgfqpoint{1.106333in}{0.838219in}}%
\pgfpathlineto{\pgfqpoint{1.106958in}{1.204977in}}%
\pgfpathlineto{\pgfqpoint{1.108000in}{0.834449in}}%
\pgfpathlineto{\pgfqpoint{1.107167in}{1.291901in}}%
\pgfpathlineto{\pgfqpoint{1.108104in}{0.919324in}}%
\pgfpathlineto{\pgfqpoint{1.108208in}{1.248425in}}%
\pgfpathlineto{\pgfqpoint{1.109145in}{1.135466in}}%
\pgfpathlineto{\pgfqpoint{1.109770in}{0.910435in}}%
\pgfpathlineto{\pgfqpoint{1.109978in}{1.235351in}}%
\pgfpathlineto{\pgfqpoint{1.110291in}{0.957731in}}%
\pgfpathlineto{\pgfqpoint{1.111124in}{1.213392in}}%
\pgfpathlineto{\pgfqpoint{1.110499in}{0.832410in}}%
\pgfpathlineto{\pgfqpoint{1.111436in}{1.052822in}}%
\pgfpathlineto{\pgfqpoint{1.111853in}{0.797626in}}%
\pgfpathlineto{\pgfqpoint{1.112269in}{1.229416in}}%
\pgfpathlineto{\pgfqpoint{1.112686in}{0.936460in}}%
\pgfpathlineto{\pgfqpoint{1.113519in}{1.300135in}}%
\pgfpathlineto{\pgfqpoint{1.113207in}{0.804234in}}%
\pgfpathlineto{\pgfqpoint{1.113623in}{1.013814in}}%
\pgfpathlineto{\pgfqpoint{1.113727in}{0.730902in}}%
\pgfpathlineto{\pgfqpoint{1.114352in}{1.224856in}}%
\pgfpathlineto{\pgfqpoint{1.114665in}{0.814498in}}%
\pgfpathlineto{\pgfqpoint{1.114873in}{0.788794in}}%
\pgfpathlineto{\pgfqpoint{1.116019in}{1.287517in}}%
\pgfpathlineto{\pgfqpoint{1.116852in}{0.723825in}}%
\pgfpathlineto{\pgfqpoint{1.116748in}{1.431886in}}%
\pgfpathlineto{\pgfqpoint{1.117164in}{0.968051in}}%
\pgfpathlineto{\pgfqpoint{1.117372in}{0.920711in}}%
\pgfpathlineto{\pgfqpoint{1.117477in}{1.301555in}}%
\pgfpathlineto{\pgfqpoint{1.118101in}{0.734321in}}%
\pgfpathlineto{\pgfqpoint{1.118518in}{1.076102in}}%
\pgfpathlineto{\pgfqpoint{1.119039in}{0.776205in}}%
\pgfpathlineto{\pgfqpoint{1.118726in}{1.172119in}}%
\pgfpathlineto{\pgfqpoint{1.119664in}{0.929116in}}%
\pgfpathlineto{\pgfqpoint{1.119976in}{1.223095in}}%
\pgfpathlineto{\pgfqpoint{1.119872in}{0.829254in}}%
\pgfpathlineto{\pgfqpoint{1.120809in}{1.148834in}}%
\pgfpathlineto{\pgfqpoint{1.121017in}{0.763986in}}%
\pgfpathlineto{\pgfqpoint{1.121538in}{1.190156in}}%
\pgfpathlineto{\pgfqpoint{1.121850in}{1.150493in}}%
\pgfpathlineto{\pgfqpoint{1.121955in}{1.195727in}}%
\pgfpathlineto{\pgfqpoint{1.122163in}{0.940914in}}%
\pgfpathlineto{\pgfqpoint{1.122267in}{0.793650in}}%
\pgfpathlineto{\pgfqpoint{1.122788in}{1.155198in}}%
\pgfpathlineto{\pgfqpoint{1.122996in}{1.045793in}}%
\pgfpathlineto{\pgfqpoint{1.123100in}{1.228047in}}%
\pgfpathlineto{\pgfqpoint{1.123621in}{0.929252in}}%
\pgfpathlineto{\pgfqpoint{1.123933in}{1.021585in}}%
\pgfpathlineto{\pgfqpoint{1.124350in}{0.847551in}}%
\pgfpathlineto{\pgfqpoint{1.124142in}{1.313838in}}%
\pgfpathlineto{\pgfqpoint{1.124871in}{1.197509in}}%
\pgfpathlineto{\pgfqpoint{1.124975in}{1.259695in}}%
\pgfpathlineto{\pgfqpoint{1.125287in}{0.901981in}}%
\pgfpathlineto{\pgfqpoint{1.125704in}{1.214271in}}%
\pgfpathlineto{\pgfqpoint{1.125912in}{0.765255in}}%
\pgfpathlineto{\pgfqpoint{1.126537in}{1.276724in}}%
\pgfpathlineto{\pgfqpoint{1.126849in}{0.980713in}}%
\pgfpathlineto{\pgfqpoint{1.127578in}{1.341986in}}%
\pgfpathlineto{\pgfqpoint{1.127891in}{1.059572in}}%
\pgfpathlineto{\pgfqpoint{1.127995in}{0.704018in}}%
\pgfpathlineto{\pgfqpoint{1.128724in}{1.204203in}}%
\pgfpathlineto{\pgfqpoint{1.128828in}{0.998547in}}%
\pgfpathlineto{\pgfqpoint{1.128932in}{1.360024in}}%
\pgfpathlineto{\pgfqpoint{1.129453in}{0.731910in}}%
\pgfpathlineto{\pgfqpoint{1.129869in}{1.078461in}}%
\pgfpathlineto{\pgfqpoint{1.130390in}{0.682362in}}%
\pgfpathlineto{\pgfqpoint{1.130182in}{1.177792in}}%
\pgfpathlineto{\pgfqpoint{1.130911in}{1.147807in}}%
\pgfpathlineto{\pgfqpoint{1.131223in}{0.831099in}}%
\pgfpathlineto{\pgfqpoint{1.131119in}{1.229743in}}%
\pgfpathlineto{\pgfqpoint{1.132265in}{0.964348in}}%
\pgfpathlineto{\pgfqpoint{1.132889in}{0.944149in}}%
\pgfpathlineto{\pgfqpoint{1.133306in}{1.421445in}}%
\pgfpathlineto{\pgfqpoint{1.134347in}{0.808747in}}%
\pgfpathlineto{\pgfqpoint{1.134452in}{1.353153in}}%
\pgfpathlineto{\pgfqpoint{1.135493in}{1.065269in}}%
\pgfpathlineto{\pgfqpoint{1.136534in}{1.213908in}}%
\pgfpathlineto{\pgfqpoint{1.135910in}{0.900627in}}%
\pgfpathlineto{\pgfqpoint{1.136639in}{1.155739in}}%
\pgfpathlineto{\pgfqpoint{1.137472in}{0.766721in}}%
\pgfpathlineto{\pgfqpoint{1.137888in}{1.056473in}}%
\pgfpathlineto{\pgfqpoint{1.137992in}{1.333238in}}%
\pgfpathlineto{\pgfqpoint{1.138721in}{0.651908in}}%
\pgfpathlineto{\pgfqpoint{1.138930in}{1.280378in}}%
\pgfpathlineto{\pgfqpoint{1.139554in}{0.798232in}}%
\pgfpathlineto{\pgfqpoint{1.140075in}{1.026532in}}%
\pgfpathlineto{\pgfqpoint{1.140179in}{1.038958in}}%
\pgfpathlineto{\pgfqpoint{1.140283in}{1.022761in}}%
\pgfpathlineto{\pgfqpoint{1.141221in}{1.313595in}}%
\pgfpathlineto{\pgfqpoint{1.140700in}{0.867522in}}%
\pgfpathlineto{\pgfqpoint{1.141325in}{1.206697in}}%
\pgfpathlineto{\pgfqpoint{1.141637in}{0.679164in}}%
\pgfpathlineto{\pgfqpoint{1.142470in}{1.124170in}}%
\pgfpathlineto{\pgfqpoint{1.142575in}{1.306163in}}%
\pgfpathlineto{\pgfqpoint{1.142991in}{0.935085in}}%
\pgfpathlineto{\pgfqpoint{1.143512in}{1.098000in}}%
\pgfpathlineto{\pgfqpoint{1.143824in}{1.360325in}}%
\pgfpathlineto{\pgfqpoint{1.144449in}{0.638120in}}%
\pgfpathlineto{\pgfqpoint{1.144970in}{1.284469in}}%
\pgfpathlineto{\pgfqpoint{1.145595in}{1.219860in}}%
\pgfpathlineto{\pgfqpoint{1.146636in}{0.837035in}}%
\pgfpathlineto{\pgfqpoint{1.146115in}{1.256608in}}%
\pgfpathlineto{\pgfqpoint{1.146740in}{0.982558in}}%
\pgfpathlineto{\pgfqpoint{1.147157in}{0.800069in}}%
\pgfpathlineto{\pgfqpoint{1.147365in}{1.420933in}}%
\pgfpathlineto{\pgfqpoint{1.147573in}{0.960018in}}%
\pgfpathlineto{\pgfqpoint{1.147677in}{1.468943in}}%
\pgfpathlineto{\pgfqpoint{1.147782in}{0.860471in}}%
\pgfpathlineto{\pgfqpoint{1.148615in}{0.906296in}}%
\pgfpathlineto{\pgfqpoint{1.149031in}{1.238497in}}%
\pgfpathlineto{\pgfqpoint{1.149552in}{0.797689in}}%
\pgfpathlineto{\pgfqpoint{1.149656in}{1.102875in}}%
\pgfpathlineto{\pgfqpoint{1.149760in}{0.900388in}}%
\pgfpathlineto{\pgfqpoint{1.149864in}{1.305447in}}%
\pgfpathlineto{\pgfqpoint{1.150802in}{0.965489in}}%
\pgfpathlineto{\pgfqpoint{1.151427in}{1.333013in}}%
\pgfpathlineto{\pgfqpoint{1.151010in}{0.747857in}}%
\pgfpathlineto{\pgfqpoint{1.151947in}{1.112117in}}%
\pgfpathlineto{\pgfqpoint{1.152676in}{1.307081in}}%
\pgfpathlineto{\pgfqpoint{1.153197in}{0.729869in}}%
\pgfpathlineto{\pgfqpoint{1.154342in}{1.261536in}}%
\pgfpathlineto{\pgfqpoint{1.155071in}{0.794373in}}%
\pgfpathlineto{\pgfqpoint{1.155488in}{1.051542in}}%
\pgfpathlineto{\pgfqpoint{1.156009in}{0.855037in}}%
\pgfpathlineto{\pgfqpoint{1.155905in}{1.258444in}}%
\pgfpathlineto{\pgfqpoint{1.156529in}{0.978001in}}%
\pgfpathlineto{\pgfqpoint{1.156634in}{1.179469in}}%
\pgfpathlineto{\pgfqpoint{1.157050in}{0.847863in}}%
\pgfpathlineto{\pgfqpoint{1.157571in}{1.084287in}}%
\pgfpathlineto{\pgfqpoint{1.157675in}{0.815765in}}%
\pgfpathlineto{\pgfqpoint{1.157779in}{1.331501in}}%
\pgfpathlineto{\pgfqpoint{1.158612in}{1.058457in}}%
\pgfpathlineto{\pgfqpoint{1.159133in}{1.362309in}}%
\pgfpathlineto{\pgfqpoint{1.158821in}{0.873163in}}%
\pgfpathlineto{\pgfqpoint{1.159550in}{1.339403in}}%
\pgfpathlineto{\pgfqpoint{1.159966in}{0.686063in}}%
\pgfpathlineto{\pgfqpoint{1.160487in}{1.509177in}}%
\pgfpathlineto{\pgfqpoint{1.160695in}{1.129086in}}%
\pgfpathlineto{\pgfqpoint{1.161528in}{0.715781in}}%
\pgfpathlineto{\pgfqpoint{1.161216in}{1.321288in}}%
\pgfpathlineto{\pgfqpoint{1.161632in}{1.001154in}}%
\pgfpathlineto{\pgfqpoint{1.161736in}{1.451129in}}%
\pgfpathlineto{\pgfqpoint{1.162361in}{0.696528in}}%
\pgfpathlineto{\pgfqpoint{1.162674in}{1.080039in}}%
\pgfpathlineto{\pgfqpoint{1.163507in}{0.808263in}}%
\pgfpathlineto{\pgfqpoint{1.163299in}{1.229945in}}%
\pgfpathlineto{\pgfqpoint{1.163715in}{1.059219in}}%
\pgfpathlineto{\pgfqpoint{1.163923in}{1.254209in}}%
\pgfpathlineto{\pgfqpoint{1.164028in}{1.019446in}}%
\pgfpathlineto{\pgfqpoint{1.164132in}{1.036603in}}%
\pgfpathlineto{\pgfqpoint{1.164757in}{1.266292in}}%
\pgfpathlineto{\pgfqpoint{1.165381in}{0.742254in}}%
\pgfpathlineto{\pgfqpoint{1.165694in}{1.484061in}}%
\pgfpathlineto{\pgfqpoint{1.166423in}{0.910458in}}%
\pgfpathlineto{\pgfqpoint{1.166735in}{0.727623in}}%
\pgfpathlineto{\pgfqpoint{1.166944in}{1.009969in}}%
\pgfpathlineto{\pgfqpoint{1.167048in}{0.961439in}}%
\pgfpathlineto{\pgfqpoint{1.167152in}{1.236773in}}%
\pgfpathlineto{\pgfqpoint{1.167464in}{0.790720in}}%
\pgfpathlineto{\pgfqpoint{1.168089in}{0.997404in}}%
\pgfpathlineto{\pgfqpoint{1.168402in}{1.107093in}}%
\pgfpathlineto{\pgfqpoint{1.168506in}{0.920436in}}%
\pgfpathlineto{\pgfqpoint{1.169339in}{1.258662in}}%
\pgfpathlineto{\pgfqpoint{1.169235in}{0.701175in}}%
\pgfpathlineto{\pgfqpoint{1.169547in}{0.833815in}}%
\pgfpathlineto{\pgfqpoint{1.169651in}{0.837002in}}%
\pgfpathlineto{\pgfqpoint{1.169755in}{1.299754in}}%
\pgfpathlineto{\pgfqpoint{1.170068in}{0.721385in}}%
\pgfpathlineto{\pgfqpoint{1.170797in}{1.202016in}}%
\pgfpathlineto{\pgfqpoint{1.171213in}{1.214510in}}%
\pgfpathlineto{\pgfqpoint{1.172046in}{0.628356in}}%
\pgfpathlineto{\pgfqpoint{1.172151in}{1.299619in}}%
\pgfpathlineto{\pgfqpoint{1.173192in}{1.149024in}}%
\pgfpathlineto{\pgfqpoint{1.173817in}{0.796821in}}%
\pgfpathlineto{\pgfqpoint{1.173400in}{1.238077in}}%
\pgfpathlineto{\pgfqpoint{1.174233in}{1.227607in}}%
\pgfpathlineto{\pgfqpoint{1.174754in}{0.714337in}}%
\pgfpathlineto{\pgfqpoint{1.174858in}{1.299927in}}%
\pgfpathlineto{\pgfqpoint{1.175275in}{1.236943in}}%
\pgfpathlineto{\pgfqpoint{1.175379in}{1.381882in}}%
\pgfpathlineto{\pgfqpoint{1.175796in}{0.827359in}}%
\pgfpathlineto{\pgfqpoint{1.176212in}{1.225393in}}%
\pgfpathlineto{\pgfqpoint{1.176316in}{0.809394in}}%
\pgfpathlineto{\pgfqpoint{1.177358in}{0.988723in}}%
\pgfpathlineto{\pgfqpoint{1.177670in}{1.363348in}}%
\pgfpathlineto{\pgfqpoint{1.177878in}{0.852167in}}%
\pgfpathlineto{\pgfqpoint{1.178399in}{1.103802in}}%
\pgfpathlineto{\pgfqpoint{1.178920in}{0.737880in}}%
\pgfpathlineto{\pgfqpoint{1.179232in}{1.296970in}}%
\pgfpathlineto{\pgfqpoint{1.179440in}{1.089395in}}%
\pgfpathlineto{\pgfqpoint{1.179857in}{1.346617in}}%
\pgfpathlineto{\pgfqpoint{1.179649in}{0.907542in}}%
\pgfpathlineto{\pgfqpoint{1.180169in}{0.970587in}}%
\pgfpathlineto{\pgfqpoint{1.180690in}{1.133145in}}%
\pgfpathlineto{\pgfqpoint{1.180898in}{0.907562in}}%
\pgfpathlineto{\pgfqpoint{1.181315in}{1.223445in}}%
\pgfpathlineto{\pgfqpoint{1.181419in}{0.875386in}}%
\pgfpathlineto{\pgfqpoint{1.181940in}{0.915822in}}%
\pgfpathlineto{\pgfqpoint{1.182565in}{1.221597in}}%
\pgfpathlineto{\pgfqpoint{1.182461in}{0.821310in}}%
\pgfpathlineto{\pgfqpoint{1.183190in}{1.124515in}}%
\pgfpathlineto{\pgfqpoint{1.183294in}{0.927681in}}%
\pgfpathlineto{\pgfqpoint{1.184023in}{1.420165in}}%
\pgfpathlineto{\pgfqpoint{1.184231in}{1.007505in}}%
\pgfpathlineto{\pgfqpoint{1.184439in}{1.211530in}}%
\pgfpathlineto{\pgfqpoint{1.184648in}{0.765515in}}%
\pgfpathlineto{\pgfqpoint{1.185168in}{0.923358in}}%
\pgfpathlineto{\pgfqpoint{1.185585in}{0.805252in}}%
\pgfpathlineto{\pgfqpoint{1.185897in}{1.194284in}}%
\pgfpathlineto{\pgfqpoint{1.186105in}{1.377552in}}%
\pgfpathlineto{\pgfqpoint{1.186418in}{0.993013in}}%
\pgfpathlineto{\pgfqpoint{1.186626in}{1.279284in}}%
\pgfpathlineto{\pgfqpoint{1.186730in}{0.856646in}}%
\pgfpathlineto{\pgfqpoint{1.187668in}{1.204792in}}%
\pgfpathlineto{\pgfqpoint{1.187772in}{1.442168in}}%
\pgfpathlineto{\pgfqpoint{1.188605in}{0.931444in}}%
\pgfpathlineto{\pgfqpoint{1.188709in}{1.215305in}}%
\pgfpathlineto{\pgfqpoint{1.189126in}{0.832628in}}%
\pgfpathlineto{\pgfqpoint{1.189438in}{1.304904in}}%
\pgfpathlineto{\pgfqpoint{1.189855in}{0.857886in}}%
\pgfpathlineto{\pgfqpoint{1.190063in}{1.281599in}}%
\pgfpathlineto{\pgfqpoint{1.190792in}{0.826865in}}%
\pgfpathlineto{\pgfqpoint{1.191000in}{1.048247in}}%
\pgfpathlineto{\pgfqpoint{1.191104in}{1.163632in}}%
\pgfpathlineto{\pgfqpoint{1.191729in}{0.886263in}}%
\pgfpathlineto{\pgfqpoint{1.191937in}{1.088394in}}%
\pgfpathlineto{\pgfqpoint{1.192354in}{0.887720in}}%
\pgfpathlineto{\pgfqpoint{1.192771in}{1.397497in}}%
\pgfpathlineto{\pgfqpoint{1.193083in}{0.947443in}}%
\pgfpathlineto{\pgfqpoint{1.193187in}{1.207042in}}%
\pgfpathlineto{\pgfqpoint{1.193291in}{0.868358in}}%
\pgfpathlineto{\pgfqpoint{1.194229in}{1.037359in}}%
\pgfpathlineto{\pgfqpoint{1.194645in}{1.278911in}}%
\pgfpathlineto{\pgfqpoint{1.194749in}{0.939902in}}%
\pgfpathlineto{\pgfqpoint{1.194957in}{1.274177in}}%
\pgfpathlineto{\pgfqpoint{1.195166in}{0.840485in}}%
\pgfpathlineto{\pgfqpoint{1.195478in}{1.328588in}}%
\pgfpathlineto{\pgfqpoint{1.196103in}{1.060607in}}%
\pgfpathlineto{\pgfqpoint{1.196207in}{1.194537in}}%
\pgfpathlineto{\pgfqpoint{1.196311in}{0.983054in}}%
\pgfpathlineto{\pgfqpoint{1.197040in}{0.999572in}}%
\pgfpathlineto{\pgfqpoint{1.197144in}{0.837173in}}%
\pgfpathlineto{\pgfqpoint{1.197353in}{1.332052in}}%
\pgfpathlineto{\pgfqpoint{1.198082in}{0.919353in}}%
\pgfpathlineto{\pgfqpoint{1.198811in}{1.286122in}}%
\pgfpathlineto{\pgfqpoint{1.198290in}{0.824395in}}%
\pgfpathlineto{\pgfqpoint{1.199331in}{1.118554in}}%
\pgfpathlineto{\pgfqpoint{1.199748in}{0.860247in}}%
\pgfpathlineto{\pgfqpoint{1.200269in}{1.287879in}}%
\pgfpathlineto{\pgfqpoint{1.200373in}{1.182922in}}%
\pgfpathlineto{\pgfqpoint{1.200477in}{1.195763in}}%
\pgfpathlineto{\pgfqpoint{1.200685in}{0.800228in}}%
\pgfpathlineto{\pgfqpoint{1.200789in}{1.340970in}}%
\pgfpathlineto{\pgfqpoint{1.201623in}{0.896096in}}%
\pgfpathlineto{\pgfqpoint{1.202768in}{1.390754in}}%
\pgfpathlineto{\pgfqpoint{1.201935in}{0.655804in}}%
\pgfpathlineto{\pgfqpoint{1.202872in}{1.328420in}}%
\pgfpathlineto{\pgfqpoint{1.203914in}{0.862582in}}%
\pgfpathlineto{\pgfqpoint{1.204018in}{0.922940in}}%
\pgfpathlineto{\pgfqpoint{1.204122in}{0.813122in}}%
\pgfpathlineto{\pgfqpoint{1.204538in}{1.255379in}}%
\pgfpathlineto{\pgfqpoint{1.204851in}{1.041381in}}%
\pgfpathlineto{\pgfqpoint{1.204955in}{1.330243in}}%
\pgfpathlineto{\pgfqpoint{1.205267in}{0.844750in}}%
\pgfpathlineto{\pgfqpoint{1.205892in}{1.027144in}}%
\pgfpathlineto{\pgfqpoint{1.205996in}{0.688816in}}%
\pgfpathlineto{\pgfqpoint{1.206934in}{1.249865in}}%
\pgfpathlineto{\pgfqpoint{1.207559in}{0.805357in}}%
\pgfpathlineto{\pgfqpoint{1.207871in}{1.001621in}}%
\pgfpathlineto{\pgfqpoint{1.207975in}{1.441562in}}%
\pgfpathlineto{\pgfqpoint{1.208288in}{0.730307in}}%
\pgfpathlineto{\pgfqpoint{1.208912in}{1.167726in}}%
\pgfpathlineto{\pgfqpoint{1.209537in}{1.196441in}}%
\pgfpathlineto{\pgfqpoint{1.210162in}{0.895284in}}%
\pgfpathlineto{\pgfqpoint{1.211203in}{1.203806in}}%
\pgfpathlineto{\pgfqpoint{1.210683in}{0.876591in}}%
\pgfpathlineto{\pgfqpoint{1.211308in}{0.966906in}}%
\pgfpathlineto{\pgfqpoint{1.211412in}{0.963685in}}%
\pgfpathlineto{\pgfqpoint{1.211516in}{0.901410in}}%
\pgfpathlineto{\pgfqpoint{1.211724in}{1.266940in}}%
\pgfpathlineto{\pgfqpoint{1.211932in}{1.018715in}}%
\pgfpathlineto{\pgfqpoint{1.212037in}{1.269578in}}%
\pgfpathlineto{\pgfqpoint{1.212349in}{0.879276in}}%
\pgfpathlineto{\pgfqpoint{1.212974in}{1.021627in}}%
\pgfpathlineto{\pgfqpoint{1.213390in}{0.768189in}}%
\pgfpathlineto{\pgfqpoint{1.213286in}{1.219779in}}%
\pgfpathlineto{\pgfqpoint{1.214015in}{1.026987in}}%
\pgfpathlineto{\pgfqpoint{1.214640in}{1.395569in}}%
\pgfpathlineto{\pgfqpoint{1.214224in}{0.872128in}}%
\pgfpathlineto{\pgfqpoint{1.215057in}{1.235305in}}%
\pgfpathlineto{\pgfqpoint{1.215577in}{0.846702in}}%
\pgfpathlineto{\pgfqpoint{1.215890in}{1.258601in}}%
\pgfpathlineto{\pgfqpoint{1.216306in}{0.989074in}}%
\pgfpathlineto{\pgfqpoint{1.216723in}{1.155635in}}%
\pgfpathlineto{\pgfqpoint{1.216827in}{0.988847in}}%
\pgfpathlineto{\pgfqpoint{1.217035in}{1.047544in}}%
\pgfpathlineto{\pgfqpoint{1.217973in}{0.882424in}}%
\pgfpathlineto{\pgfqpoint{1.217869in}{1.241077in}}%
\pgfpathlineto{\pgfqpoint{1.218077in}{0.948291in}}%
\pgfpathlineto{\pgfqpoint{1.218181in}{1.462428in}}%
\pgfpathlineto{\pgfqpoint{1.218598in}{0.749131in}}%
\pgfpathlineto{\pgfqpoint{1.219222in}{1.186101in}}%
\pgfpathlineto{\pgfqpoint{1.219431in}{0.978893in}}%
\pgfpathlineto{\pgfqpoint{1.219743in}{1.316655in}}%
\pgfpathlineto{\pgfqpoint{1.219951in}{0.716183in}}%
\pgfpathlineto{\pgfqpoint{1.220889in}{1.106695in}}%
\pgfpathlineto{\pgfqpoint{1.221409in}{0.842930in}}%
\pgfpathlineto{\pgfqpoint{1.221305in}{1.356374in}}%
\pgfpathlineto{\pgfqpoint{1.221722in}{1.118707in}}%
\pgfpathlineto{\pgfqpoint{1.221826in}{1.240379in}}%
\pgfpathlineto{\pgfqpoint{1.222347in}{0.768995in}}%
\pgfpathlineto{\pgfqpoint{1.222659in}{0.956562in}}%
\pgfpathlineto{\pgfqpoint{1.222867in}{1.173017in}}%
\pgfpathlineto{\pgfqpoint{1.223076in}{0.865372in}}%
\pgfpathlineto{\pgfqpoint{1.223909in}{1.250514in}}%
\pgfpathlineto{\pgfqpoint{1.223284in}{0.843606in}}%
\pgfpathlineto{\pgfqpoint{1.224221in}{0.989886in}}%
\pgfpathlineto{\pgfqpoint{1.225158in}{1.270914in}}%
\pgfpathlineto{\pgfqpoint{1.224638in}{0.789277in}}%
\pgfpathlineto{\pgfqpoint{1.225367in}{1.020528in}}%
\pgfpathlineto{\pgfqpoint{1.225471in}{0.887189in}}%
\pgfpathlineto{\pgfqpoint{1.226200in}{1.274580in}}%
\pgfpathlineto{\pgfqpoint{1.226304in}{1.252839in}}%
\pgfpathlineto{\pgfqpoint{1.226616in}{0.669679in}}%
\pgfpathlineto{\pgfqpoint{1.227554in}{0.775190in}}%
\pgfpathlineto{\pgfqpoint{1.228387in}{0.723332in}}%
\pgfpathlineto{\pgfqpoint{1.228699in}{1.196647in}}%
\pgfpathlineto{\pgfqpoint{1.229012in}{0.832927in}}%
\pgfpathlineto{\pgfqpoint{1.229324in}{1.200436in}}%
\pgfpathlineto{\pgfqpoint{1.229845in}{1.029454in}}%
\pgfpathlineto{\pgfqpoint{1.230157in}{1.172994in}}%
\pgfpathlineto{\pgfqpoint{1.230053in}{0.876802in}}%
\pgfpathlineto{\pgfqpoint{1.230886in}{1.073169in}}%
\pgfpathlineto{\pgfqpoint{1.230990in}{0.704542in}}%
\pgfpathlineto{\pgfqpoint{1.231303in}{1.206488in}}%
\pgfpathlineto{\pgfqpoint{1.231928in}{0.924049in}}%
\pgfpathlineto{\pgfqpoint{1.232240in}{1.421082in}}%
\pgfpathlineto{\pgfqpoint{1.232448in}{0.847546in}}%
\pgfpathlineto{\pgfqpoint{1.233073in}{1.151617in}}%
\pgfpathlineto{\pgfqpoint{1.233802in}{1.234404in}}%
\pgfpathlineto{\pgfqpoint{1.234115in}{0.852972in}}%
\pgfpathlineto{\pgfqpoint{1.234844in}{1.281982in}}%
\pgfpathlineto{\pgfqpoint{1.235260in}{1.075527in}}%
\pgfpathlineto{\pgfqpoint{1.235677in}{0.657165in}}%
\pgfpathlineto{\pgfqpoint{1.235468in}{1.302703in}}%
\pgfpathlineto{\pgfqpoint{1.236197in}{1.172691in}}%
\pgfpathlineto{\pgfqpoint{1.236301in}{1.218604in}}%
\pgfpathlineto{\pgfqpoint{1.236510in}{1.004340in}}%
\pgfpathlineto{\pgfqpoint{1.236718in}{1.057049in}}%
\pgfpathlineto{\pgfqpoint{1.237135in}{1.280395in}}%
\pgfpathlineto{\pgfqpoint{1.237864in}{0.903898in}}%
\pgfpathlineto{\pgfqpoint{1.238072in}{0.661391in}}%
\pgfpathlineto{\pgfqpoint{1.239113in}{1.224060in}}%
\pgfpathlineto{\pgfqpoint{1.239530in}{0.848150in}}%
\pgfpathlineto{\pgfqpoint{1.239946in}{1.260065in}}%
\pgfpathlineto{\pgfqpoint{1.240259in}{1.025782in}}%
\pgfpathlineto{\pgfqpoint{1.241196in}{0.860188in}}%
\pgfpathlineto{\pgfqpoint{1.241300in}{1.368886in}}%
\pgfpathlineto{\pgfqpoint{1.242029in}{0.790845in}}%
\pgfpathlineto{\pgfqpoint{1.242446in}{0.827504in}}%
\pgfpathlineto{\pgfqpoint{1.242550in}{1.288287in}}%
\pgfpathlineto{\pgfqpoint{1.243487in}{0.962786in}}%
\pgfpathlineto{\pgfqpoint{1.243591in}{0.783914in}}%
\pgfpathlineto{\pgfqpoint{1.244112in}{1.309840in}}%
\pgfpathlineto{\pgfqpoint{1.244424in}{1.188195in}}%
\pgfpathlineto{\pgfqpoint{1.244737in}{0.804269in}}%
\pgfpathlineto{\pgfqpoint{1.245362in}{1.195521in}}%
\pgfpathlineto{\pgfqpoint{1.245466in}{0.874097in}}%
\pgfpathlineto{\pgfqpoint{1.245987in}{1.401547in}}%
\pgfpathlineto{\pgfqpoint{1.246299in}{0.704881in}}%
\pgfpathlineto{\pgfqpoint{1.246611in}{1.048419in}}%
\pgfpathlineto{\pgfqpoint{1.246820in}{0.753185in}}%
\pgfpathlineto{\pgfqpoint{1.247028in}{1.267911in}}%
\pgfpathlineto{\pgfqpoint{1.247757in}{0.942506in}}%
\pgfpathlineto{\pgfqpoint{1.248903in}{1.318290in}}%
\pgfpathlineto{\pgfqpoint{1.248069in}{0.861470in}}%
\pgfpathlineto{\pgfqpoint{1.249007in}{1.135368in}}%
\pgfpathlineto{\pgfqpoint{1.249944in}{1.320974in}}%
\pgfpathlineto{\pgfqpoint{1.250048in}{0.867766in}}%
\pgfpathlineto{\pgfqpoint{1.250152in}{1.279766in}}%
\pgfpathlineto{\pgfqpoint{1.250777in}{0.656549in}}%
\pgfpathlineto{\pgfqpoint{1.251194in}{1.022603in}}%
\pgfpathlineto{\pgfqpoint{1.251923in}{1.191637in}}%
\pgfpathlineto{\pgfqpoint{1.251610in}{0.956409in}}%
\pgfpathlineto{\pgfqpoint{1.252131in}{1.058646in}}%
\pgfpathlineto{\pgfqpoint{1.252756in}{1.159199in}}%
\pgfpathlineto{\pgfqpoint{1.253381in}{0.796372in}}%
\pgfpathlineto{\pgfqpoint{1.254630in}{1.229090in}}%
\pgfpathlineto{\pgfqpoint{1.254839in}{0.728180in}}%
\pgfpathlineto{\pgfqpoint{1.255463in}{1.268913in}}%
\pgfpathlineto{\pgfqpoint{1.255672in}{1.211681in}}%
\pgfpathlineto{\pgfqpoint{1.255880in}{1.264809in}}%
\pgfpathlineto{\pgfqpoint{1.255984in}{1.095989in}}%
\pgfpathlineto{\pgfqpoint{1.257026in}{0.688084in}}%
\pgfpathlineto{\pgfqpoint{1.256505in}{1.310375in}}%
\pgfpathlineto{\pgfqpoint{1.257130in}{0.980868in}}%
\pgfpathlineto{\pgfqpoint{1.257650in}{1.316152in}}%
\pgfpathlineto{\pgfqpoint{1.257546in}{0.900729in}}%
\pgfpathlineto{\pgfqpoint{1.258275in}{1.129021in}}%
\pgfpathlineto{\pgfqpoint{1.258588in}{1.485255in}}%
\pgfpathlineto{\pgfqpoint{1.258692in}{0.903213in}}%
\pgfpathlineto{\pgfqpoint{1.258796in}{0.802956in}}%
\pgfpathlineto{\pgfqpoint{1.259525in}{1.130274in}}%
\pgfpathlineto{\pgfqpoint{1.259629in}{0.979833in}}%
\pgfpathlineto{\pgfqpoint{1.259837in}{1.185495in}}%
\pgfpathlineto{\pgfqpoint{1.260254in}{0.769038in}}%
\pgfpathlineto{\pgfqpoint{1.260358in}{0.703930in}}%
\pgfpathlineto{\pgfqpoint{1.260566in}{1.126426in}}%
\pgfpathlineto{\pgfqpoint{1.261504in}{1.367796in}}%
\pgfpathlineto{\pgfqpoint{1.261295in}{0.742247in}}%
\pgfpathlineto{\pgfqpoint{1.261712in}{1.296569in}}%
\pgfpathlineto{\pgfqpoint{1.262128in}{0.728502in}}%
\pgfpathlineto{\pgfqpoint{1.262857in}{1.030064in}}%
\pgfpathlineto{\pgfqpoint{1.262962in}{1.230682in}}%
\pgfpathlineto{\pgfqpoint{1.263691in}{0.887906in}}%
\pgfpathlineto{\pgfqpoint{1.263899in}{1.155124in}}%
\pgfpathlineto{\pgfqpoint{1.264315in}{0.964990in}}%
\pgfpathlineto{\pgfqpoint{1.264732in}{1.285088in}}%
\pgfpathlineto{\pgfqpoint{1.264836in}{1.012738in}}%
\pgfpathlineto{\pgfqpoint{1.265357in}{1.299843in}}%
\pgfpathlineto{\pgfqpoint{1.265253in}{0.708989in}}%
\pgfpathlineto{\pgfqpoint{1.265878in}{0.999006in}}%
\pgfpathlineto{\pgfqpoint{1.266398in}{0.744759in}}%
\pgfpathlineto{\pgfqpoint{1.266086in}{1.119032in}}%
\pgfpathlineto{\pgfqpoint{1.266919in}{1.053822in}}%
\pgfpathlineto{\pgfqpoint{1.267752in}{0.834540in}}%
\pgfpathlineto{\pgfqpoint{1.267127in}{1.346715in}}%
\pgfpathlineto{\pgfqpoint{1.268065in}{0.990175in}}%
\pgfpathlineto{\pgfqpoint{1.268793in}{1.190467in}}%
\pgfpathlineto{\pgfqpoint{1.268898in}{0.794163in}}%
\pgfpathlineto{\pgfqpoint{1.269106in}{1.016280in}}%
\pgfpathlineto{\pgfqpoint{1.269418in}{1.328951in}}%
\pgfpathlineto{\pgfqpoint{1.269522in}{0.929212in}}%
\pgfpathlineto{\pgfqpoint{1.270043in}{1.425708in}}%
\pgfpathlineto{\pgfqpoint{1.270564in}{0.800729in}}%
\pgfpathlineto{\pgfqpoint{1.271501in}{1.338098in}}%
\pgfpathlineto{\pgfqpoint{1.271814in}{1.205530in}}%
\pgfpathlineto{\pgfqpoint{1.271918in}{1.214949in}}%
\pgfpathlineto{\pgfqpoint{1.272543in}{0.855924in}}%
\pgfpathlineto{\pgfqpoint{1.273063in}{0.946453in}}%
\pgfpathlineto{\pgfqpoint{1.273376in}{1.081631in}}%
\pgfpathlineto{\pgfqpoint{1.273272in}{0.815188in}}%
\pgfpathlineto{\pgfqpoint{1.273896in}{0.934501in}}%
\pgfpathlineto{\pgfqpoint{1.274001in}{0.837371in}}%
\pgfpathlineto{\pgfqpoint{1.274105in}{1.319332in}}%
\pgfpathlineto{\pgfqpoint{1.274417in}{1.202619in}}%
\pgfpathlineto{\pgfqpoint{1.274521in}{1.572437in}}%
\pgfpathlineto{\pgfqpoint{1.275354in}{0.837629in}}%
\pgfpathlineto{\pgfqpoint{1.275459in}{1.193216in}}%
\pgfpathlineto{\pgfqpoint{1.276500in}{0.744326in}}%
\pgfpathlineto{\pgfqpoint{1.277229in}{1.430198in}}%
\pgfpathlineto{\pgfqpoint{1.277645in}{0.961526in}}%
\pgfpathlineto{\pgfqpoint{1.278687in}{1.179464in}}%
\pgfpathlineto{\pgfqpoint{1.277958in}{0.755705in}}%
\pgfpathlineto{\pgfqpoint{1.278791in}{1.090515in}}%
\pgfpathlineto{\pgfqpoint{1.279520in}{0.863093in}}%
\pgfpathlineto{\pgfqpoint{1.279208in}{1.266608in}}%
\pgfpathlineto{\pgfqpoint{1.279624in}{0.991172in}}%
\pgfpathlineto{\pgfqpoint{1.279832in}{1.245083in}}%
\pgfpathlineto{\pgfqpoint{1.280249in}{0.784171in}}%
\pgfpathlineto{\pgfqpoint{1.280770in}{1.226165in}}%
\pgfpathlineto{\pgfqpoint{1.281186in}{0.796541in}}%
\pgfpathlineto{\pgfqpoint{1.281499in}{1.445824in}}%
\pgfpathlineto{\pgfqpoint{1.281811in}{1.088255in}}%
\pgfpathlineto{\pgfqpoint{1.282332in}{0.969219in}}%
\pgfpathlineto{\pgfqpoint{1.282644in}{1.175533in}}%
\pgfpathlineto{\pgfqpoint{1.283686in}{0.767564in}}%
\pgfpathlineto{\pgfqpoint{1.282957in}{1.334962in}}%
\pgfpathlineto{\pgfqpoint{1.283790in}{0.860989in}}%
\pgfpathlineto{\pgfqpoint{1.284519in}{1.361262in}}%
\pgfpathlineto{\pgfqpoint{1.284415in}{0.848223in}}%
\pgfpathlineto{\pgfqpoint{1.284935in}{1.195498in}}%
\pgfpathlineto{\pgfqpoint{1.285039in}{1.193081in}}%
\pgfpathlineto{\pgfqpoint{1.285144in}{1.211839in}}%
\pgfpathlineto{\pgfqpoint{1.285977in}{0.793105in}}%
\pgfpathlineto{\pgfqpoint{1.286185in}{1.019846in}}%
\pgfpathlineto{\pgfqpoint{1.286497in}{1.290304in}}%
\pgfpathlineto{\pgfqpoint{1.287226in}{1.161682in}}%
\pgfpathlineto{\pgfqpoint{1.288060in}{0.815455in}}%
\pgfpathlineto{\pgfqpoint{1.287955in}{1.355038in}}%
\pgfpathlineto{\pgfqpoint{1.288268in}{1.058964in}}%
\pgfpathlineto{\pgfqpoint{1.288372in}{1.221203in}}%
\pgfpathlineto{\pgfqpoint{1.288997in}{0.853842in}}%
\pgfpathlineto{\pgfqpoint{1.289205in}{1.127734in}}%
\pgfpathlineto{\pgfqpoint{1.289830in}{0.718488in}}%
\pgfpathlineto{\pgfqpoint{1.289413in}{1.289928in}}%
\pgfpathlineto{\pgfqpoint{1.290351in}{0.830605in}}%
\pgfpathlineto{\pgfqpoint{1.291288in}{1.208377in}}%
\pgfpathlineto{\pgfqpoint{1.291496in}{1.075170in}}%
\pgfpathlineto{\pgfqpoint{1.291600in}{0.800694in}}%
\pgfpathlineto{\pgfqpoint{1.291913in}{1.340998in}}%
\pgfpathlineto{\pgfqpoint{1.292642in}{0.916249in}}%
\pgfpathlineto{\pgfqpoint{1.293891in}{1.356673in}}%
\pgfpathlineto{\pgfqpoint{1.293058in}{0.748599in}}%
\pgfpathlineto{\pgfqpoint{1.293996in}{1.136223in}}%
\pgfpathlineto{\pgfqpoint{1.294725in}{0.745098in}}%
\pgfpathlineto{\pgfqpoint{1.294308in}{1.236783in}}%
\pgfpathlineto{\pgfqpoint{1.295037in}{1.061962in}}%
\pgfpathlineto{\pgfqpoint{1.295870in}{1.452270in}}%
\pgfpathlineto{\pgfqpoint{1.295766in}{0.802159in}}%
\pgfpathlineto{\pgfqpoint{1.295974in}{1.044809in}}%
\pgfpathlineto{\pgfqpoint{1.296703in}{1.147164in}}%
\pgfpathlineto{\pgfqpoint{1.297120in}{0.706490in}}%
\pgfpathlineto{\pgfqpoint{1.297953in}{1.311639in}}%
\pgfpathlineto{\pgfqpoint{1.298265in}{1.155096in}}%
\pgfpathlineto{\pgfqpoint{1.298890in}{0.770136in}}%
\pgfpathlineto{\pgfqpoint{1.298786in}{1.251279in}}%
\pgfpathlineto{\pgfqpoint{1.299411in}{1.043252in}}%
\pgfpathlineto{\pgfqpoint{1.300036in}{0.958342in}}%
\pgfpathlineto{\pgfqpoint{1.299619in}{1.141158in}}%
\pgfpathlineto{\pgfqpoint{1.300244in}{0.976631in}}%
\pgfpathlineto{\pgfqpoint{1.300452in}{0.581141in}}%
\pgfpathlineto{\pgfqpoint{1.301285in}{1.254834in}}%
\pgfpathlineto{\pgfqpoint{1.302014in}{0.749974in}}%
\pgfpathlineto{\pgfqpoint{1.302431in}{0.994977in}}%
\pgfpathlineto{\pgfqpoint{1.303160in}{1.332888in}}%
\pgfpathlineto{\pgfqpoint{1.302848in}{0.857457in}}%
\pgfpathlineto{\pgfqpoint{1.303472in}{1.059832in}}%
\pgfpathlineto{\pgfqpoint{1.304306in}{0.912484in}}%
\pgfpathlineto{\pgfqpoint{1.303993in}{1.256042in}}%
\pgfpathlineto{\pgfqpoint{1.304514in}{0.947801in}}%
\pgfpathlineto{\pgfqpoint{1.305347in}{1.365747in}}%
\pgfpathlineto{\pgfqpoint{1.305243in}{0.750866in}}%
\pgfpathlineto{\pgfqpoint{1.305451in}{1.012588in}}%
\pgfpathlineto{\pgfqpoint{1.305555in}{0.756779in}}%
\pgfpathlineto{\pgfqpoint{1.305972in}{1.199734in}}%
\pgfpathlineto{\pgfqpoint{1.306388in}{1.038845in}}%
\pgfpathlineto{\pgfqpoint{1.306493in}{1.314914in}}%
\pgfpathlineto{\pgfqpoint{1.307117in}{0.798100in}}%
\pgfpathlineto{\pgfqpoint{1.307430in}{1.272928in}}%
\pgfpathlineto{\pgfqpoint{1.307951in}{0.864586in}}%
\pgfpathlineto{\pgfqpoint{1.307742in}{1.401842in}}%
\pgfpathlineto{\pgfqpoint{1.308575in}{0.890493in}}%
\pgfpathlineto{\pgfqpoint{1.308888in}{0.720126in}}%
\pgfpathlineto{\pgfqpoint{1.308784in}{1.216699in}}%
\pgfpathlineto{\pgfqpoint{1.308992in}{1.024192in}}%
\pgfpathlineto{\pgfqpoint{1.309096in}{1.512024in}}%
\pgfpathlineto{\pgfqpoint{1.309304in}{0.889578in}}%
\pgfpathlineto{\pgfqpoint{1.310033in}{1.090167in}}%
\pgfpathlineto{\pgfqpoint{1.310346in}{1.243085in}}%
\pgfpathlineto{\pgfqpoint{1.310762in}{0.907764in}}%
\pgfpathlineto{\pgfqpoint{1.310971in}{0.989024in}}%
\pgfpathlineto{\pgfqpoint{1.311700in}{0.772594in}}%
\pgfpathlineto{\pgfqpoint{1.311283in}{1.156817in}}%
\pgfpathlineto{\pgfqpoint{1.311804in}{0.838125in}}%
\pgfpathlineto{\pgfqpoint{1.312845in}{1.414318in}}%
\pgfpathlineto{\pgfqpoint{1.312949in}{1.345481in}}%
\pgfpathlineto{\pgfqpoint{1.313991in}{0.801918in}}%
\pgfpathlineto{\pgfqpoint{1.314095in}{1.088850in}}%
\pgfpathlineto{\pgfqpoint{1.314616in}{0.899280in}}%
\pgfpathlineto{\pgfqpoint{1.315240in}{1.343026in}}%
\pgfpathlineto{\pgfqpoint{1.316074in}{0.620318in}}%
\pgfpathlineto{\pgfqpoint{1.316386in}{0.910507in}}%
\pgfpathlineto{\pgfqpoint{1.317011in}{1.218870in}}%
\pgfpathlineto{\pgfqpoint{1.316803in}{0.898678in}}%
\pgfpathlineto{\pgfqpoint{1.317532in}{1.121174in}}%
\pgfpathlineto{\pgfqpoint{1.317636in}{0.607041in}}%
\pgfpathlineto{\pgfqpoint{1.317844in}{1.265111in}}%
\pgfpathlineto{\pgfqpoint{1.318573in}{0.889995in}}%
\pgfpathlineto{\pgfqpoint{1.318677in}{1.280212in}}%
\pgfpathlineto{\pgfqpoint{1.319094in}{0.788467in}}%
\pgfpathlineto{\pgfqpoint{1.319718in}{1.158085in}}%
\pgfpathlineto{\pgfqpoint{1.320239in}{1.486378in}}%
\pgfpathlineto{\pgfqpoint{1.320031in}{0.852149in}}%
\pgfpathlineto{\pgfqpoint{1.320656in}{1.089681in}}%
\pgfpathlineto{\pgfqpoint{1.320760in}{1.098434in}}%
\pgfpathlineto{\pgfqpoint{1.320864in}{1.048798in}}%
\pgfpathlineto{\pgfqpoint{1.320968in}{0.798962in}}%
\pgfpathlineto{\pgfqpoint{1.321176in}{1.231966in}}%
\pgfpathlineto{\pgfqpoint{1.322010in}{0.865076in}}%
\pgfpathlineto{\pgfqpoint{1.322218in}{1.262627in}}%
\pgfpathlineto{\pgfqpoint{1.323051in}{1.020238in}}%
\pgfpathlineto{\pgfqpoint{1.323155in}{0.879135in}}%
\pgfpathlineto{\pgfqpoint{1.323259in}{1.339591in}}%
\pgfpathlineto{\pgfqpoint{1.323884in}{1.166400in}}%
\pgfpathlineto{\pgfqpoint{1.323988in}{1.307934in}}%
\pgfpathlineto{\pgfqpoint{1.324197in}{0.865332in}}%
\pgfpathlineto{\pgfqpoint{1.324613in}{0.996391in}}%
\pgfpathlineto{\pgfqpoint{1.324717in}{0.773791in}}%
\pgfpathlineto{\pgfqpoint{1.324926in}{1.343823in}}%
\pgfpathlineto{\pgfqpoint{1.325655in}{1.008252in}}%
\pgfpathlineto{\pgfqpoint{1.325759in}{0.937274in}}%
\pgfpathlineto{\pgfqpoint{1.326383in}{1.176529in}}%
\pgfpathlineto{\pgfqpoint{1.326488in}{1.242558in}}%
\pgfpathlineto{\pgfqpoint{1.326696in}{0.862114in}}%
\pgfpathlineto{\pgfqpoint{1.326904in}{1.060641in}}%
\pgfpathlineto{\pgfqpoint{1.327008in}{0.784575in}}%
\pgfpathlineto{\pgfqpoint{1.327112in}{1.258105in}}%
\pgfpathlineto{\pgfqpoint{1.327946in}{0.940768in}}%
\pgfpathlineto{\pgfqpoint{1.328362in}{0.778812in}}%
\pgfpathlineto{\pgfqpoint{1.328987in}{1.390295in}}%
\pgfpathlineto{\pgfqpoint{1.330028in}{0.790688in}}%
\pgfpathlineto{\pgfqpoint{1.330133in}{1.133439in}}%
\pgfpathlineto{\pgfqpoint{1.330549in}{0.825521in}}%
\pgfpathlineto{\pgfqpoint{1.331278in}{0.940717in}}%
\pgfpathlineto{\pgfqpoint{1.331799in}{1.464226in}}%
\pgfpathlineto{\pgfqpoint{1.332111in}{0.792673in}}%
\pgfpathlineto{\pgfqpoint{1.332528in}{1.235859in}}%
\pgfpathlineto{\pgfqpoint{1.333673in}{0.856208in}}%
\pgfpathlineto{\pgfqpoint{1.334194in}{1.364229in}}%
\pgfpathlineto{\pgfqpoint{1.334923in}{1.009239in}}%
\pgfpathlineto{\pgfqpoint{1.335444in}{0.724390in}}%
\pgfpathlineto{\pgfqpoint{1.335652in}{1.335931in}}%
\pgfpathlineto{\pgfqpoint{1.335964in}{1.097568in}}%
\pgfpathlineto{\pgfqpoint{1.337006in}{0.940473in}}%
\pgfpathlineto{\pgfqpoint{1.336485in}{1.234637in}}%
\pgfpathlineto{\pgfqpoint{1.337110in}{0.965671in}}%
\pgfpathlineto{\pgfqpoint{1.337631in}{1.366176in}}%
\pgfpathlineto{\pgfqpoint{1.337735in}{0.646816in}}%
\pgfpathlineto{\pgfqpoint{1.338256in}{1.068831in}}%
\pgfpathlineto{\pgfqpoint{1.338464in}{0.724929in}}%
\pgfpathlineto{\pgfqpoint{1.338985in}{1.310416in}}%
\pgfpathlineto{\pgfqpoint{1.339089in}{1.407338in}}%
\pgfpathlineto{\pgfqpoint{1.339401in}{0.881286in}}%
\pgfpathlineto{\pgfqpoint{1.339714in}{1.017847in}}%
\pgfpathlineto{\pgfqpoint{1.339818in}{0.704798in}}%
\pgfpathlineto{\pgfqpoint{1.340651in}{1.203368in}}%
\pgfpathlineto{\pgfqpoint{1.340859in}{0.836060in}}%
\pgfpathlineto{\pgfqpoint{1.341067in}{1.239335in}}%
\pgfpathlineto{\pgfqpoint{1.341796in}{0.742821in}}%
\pgfpathlineto{\pgfqpoint{1.343046in}{1.300853in}}%
\pgfpathlineto{\pgfqpoint{1.344192in}{0.829085in}}%
\pgfpathlineto{\pgfqpoint{1.343254in}{1.335733in}}%
\pgfpathlineto{\pgfqpoint{1.344296in}{0.888429in}}%
\pgfpathlineto{\pgfqpoint{1.345025in}{1.226953in}}%
\pgfpathlineto{\pgfqpoint{1.344712in}{0.801418in}}%
\pgfpathlineto{\pgfqpoint{1.345337in}{1.058546in}}%
\pgfpathlineto{\pgfqpoint{1.345858in}{0.738000in}}%
\pgfpathlineto{\pgfqpoint{1.345545in}{1.272069in}}%
\pgfpathlineto{\pgfqpoint{1.346170in}{0.844326in}}%
\pgfpathlineto{\pgfqpoint{1.346587in}{1.254531in}}%
\pgfpathlineto{\pgfqpoint{1.347212in}{1.079099in}}%
\pgfpathlineto{\pgfqpoint{1.347941in}{0.823166in}}%
\pgfpathlineto{\pgfqpoint{1.347732in}{1.266664in}}%
\pgfpathlineto{\pgfqpoint{1.348253in}{1.091579in}}%
\pgfpathlineto{\pgfqpoint{1.348982in}{0.922343in}}%
\pgfpathlineto{\pgfqpoint{1.348566in}{1.125449in}}%
\pgfpathlineto{\pgfqpoint{1.349295in}{1.046999in}}%
\pgfpathlineto{\pgfqpoint{1.349399in}{1.088749in}}%
\pgfpathlineto{\pgfqpoint{1.349503in}{0.512508in}}%
\pgfpathlineto{\pgfqpoint{1.349815in}{1.275199in}}%
\pgfpathlineto{\pgfqpoint{1.350440in}{0.936351in}}%
\pgfpathlineto{\pgfqpoint{1.350857in}{1.453902in}}%
\pgfpathlineto{\pgfqpoint{1.350648in}{0.936228in}}%
\pgfpathlineto{\pgfqpoint{1.351169in}{1.126423in}}%
\pgfpathlineto{\pgfqpoint{1.351273in}{0.818280in}}%
\pgfpathlineto{\pgfqpoint{1.351898in}{1.326155in}}%
\pgfpathlineto{\pgfqpoint{1.352210in}{1.001063in}}%
\pgfpathlineto{\pgfqpoint{1.352523in}{0.750415in}}%
\pgfpathlineto{\pgfqpoint{1.353252in}{1.354417in}}%
\pgfpathlineto{\pgfqpoint{1.354293in}{0.740513in}}%
\pgfpathlineto{\pgfqpoint{1.354397in}{1.165750in}}%
\pgfpathlineto{\pgfqpoint{1.355231in}{0.902952in}}%
\pgfpathlineto{\pgfqpoint{1.354606in}{1.255698in}}%
\pgfpathlineto{\pgfqpoint{1.355543in}{1.048111in}}%
\pgfpathlineto{\pgfqpoint{1.355647in}{1.164276in}}%
\pgfpathlineto{\pgfqpoint{1.355751in}{0.758268in}}%
\pgfpathlineto{\pgfqpoint{1.356272in}{1.022664in}}%
\pgfpathlineto{\pgfqpoint{1.356376in}{0.525326in}}%
\pgfpathlineto{\pgfqpoint{1.357001in}{1.280665in}}%
\pgfpathlineto{\pgfqpoint{1.357313in}{0.980761in}}%
\pgfpathlineto{\pgfqpoint{1.357938in}{1.273428in}}%
\pgfpathlineto{\pgfqpoint{1.357834in}{0.966498in}}%
\pgfpathlineto{\pgfqpoint{1.358355in}{1.205676in}}%
\pgfpathlineto{\pgfqpoint{1.358459in}{0.847261in}}%
\pgfpathlineto{\pgfqpoint{1.359396in}{1.285599in}}%
\pgfpathlineto{\pgfqpoint{1.359500in}{1.047151in}}%
\pgfpathlineto{\pgfqpoint{1.359917in}{0.858388in}}%
\pgfpathlineto{\pgfqpoint{1.360229in}{1.244960in}}%
\pgfpathlineto{\pgfqpoint{1.360438in}{0.949305in}}%
\pgfpathlineto{\pgfqpoint{1.361271in}{1.294606in}}%
\pgfpathlineto{\pgfqpoint{1.361375in}{0.843966in}}%
\pgfpathlineto{\pgfqpoint{1.361479in}{0.865371in}}%
\pgfpathlineto{\pgfqpoint{1.361583in}{0.816872in}}%
\pgfpathlineto{\pgfqpoint{1.361687in}{1.108896in}}%
\pgfpathlineto{\pgfqpoint{1.361791in}{1.078621in}}%
\pgfpathlineto{\pgfqpoint{1.362520in}{1.452202in}}%
\pgfpathlineto{\pgfqpoint{1.362416in}{0.808660in}}%
\pgfpathlineto{\pgfqpoint{1.362937in}{1.158352in}}%
\pgfpathlineto{\pgfqpoint{1.363458in}{0.935232in}}%
\pgfpathlineto{\pgfqpoint{1.363562in}{1.262566in}}%
\pgfpathlineto{\pgfqpoint{1.364083in}{1.100717in}}%
\pgfpathlineto{\pgfqpoint{1.364707in}{1.394532in}}%
\pgfpathlineto{\pgfqpoint{1.364812in}{0.847499in}}%
\pgfpathlineto{\pgfqpoint{1.364916in}{0.863535in}}%
\pgfpathlineto{\pgfqpoint{1.365645in}{1.435826in}}%
\pgfpathlineto{\pgfqpoint{1.365749in}{0.800980in}}%
\pgfpathlineto{\pgfqpoint{1.366061in}{1.185070in}}%
\pgfpathlineto{\pgfqpoint{1.366790in}{0.904542in}}%
\pgfpathlineto{\pgfqpoint{1.366686in}{1.206910in}}%
\pgfpathlineto{\pgfqpoint{1.367103in}{0.997467in}}%
\pgfpathlineto{\pgfqpoint{1.367415in}{1.437006in}}%
\pgfpathlineto{\pgfqpoint{1.367832in}{0.960504in}}%
\pgfpathlineto{\pgfqpoint{1.368144in}{1.070908in}}%
\pgfpathlineto{\pgfqpoint{1.369081in}{0.784878in}}%
\pgfpathlineto{\pgfqpoint{1.368665in}{1.229457in}}%
\pgfpathlineto{\pgfqpoint{1.369290in}{0.954324in}}%
\pgfpathlineto{\pgfqpoint{1.369394in}{0.937277in}}%
\pgfpathlineto{\pgfqpoint{1.369498in}{1.082351in}}%
\pgfpathlineto{\pgfqpoint{1.369706in}{0.972623in}}%
\pgfpathlineto{\pgfqpoint{1.369914in}{1.307216in}}%
\pgfpathlineto{\pgfqpoint{1.370123in}{0.771798in}}%
\pgfpathlineto{\pgfqpoint{1.370748in}{1.076406in}}%
\pgfpathlineto{\pgfqpoint{1.370956in}{0.922532in}}%
\pgfpathlineto{\pgfqpoint{1.371685in}{1.195521in}}%
\pgfpathlineto{\pgfqpoint{1.371789in}{1.225346in}}%
\pgfpathlineto{\pgfqpoint{1.371997in}{1.021725in}}%
\pgfpathlineto{\pgfqpoint{1.372101in}{1.082238in}}%
\pgfpathlineto{\pgfqpoint{1.372206in}{0.885328in}}%
\pgfpathlineto{\pgfqpoint{1.372414in}{1.163518in}}%
\pgfpathlineto{\pgfqpoint{1.373247in}{1.001049in}}%
\pgfpathlineto{\pgfqpoint{1.373976in}{0.923241in}}%
\pgfpathlineto{\pgfqpoint{1.374393in}{1.348241in}}%
\pgfpathlineto{\pgfqpoint{1.375226in}{0.635623in}}%
\pgfpathlineto{\pgfqpoint{1.375434in}{1.355843in}}%
\pgfpathlineto{\pgfqpoint{1.375642in}{0.846181in}}%
\pgfpathlineto{\pgfqpoint{1.376684in}{1.474947in}}%
\pgfpathlineto{\pgfqpoint{1.376892in}{1.468927in}}%
\pgfpathlineto{\pgfqpoint{1.377517in}{0.860308in}}%
\pgfpathlineto{\pgfqpoint{1.378037in}{1.123965in}}%
\pgfpathlineto{\pgfqpoint{1.378142in}{1.226090in}}%
\pgfpathlineto{\pgfqpoint{1.378246in}{0.811089in}}%
\pgfpathlineto{\pgfqpoint{1.378871in}{0.905456in}}%
\pgfpathlineto{\pgfqpoint{1.378975in}{0.769939in}}%
\pgfpathlineto{\pgfqpoint{1.379183in}{1.249554in}}%
\pgfpathlineto{\pgfqpoint{1.379912in}{0.881779in}}%
\pgfpathlineto{\pgfqpoint{1.380016in}{1.217569in}}%
\pgfpathlineto{\pgfqpoint{1.381058in}{1.004551in}}%
\pgfpathlineto{\pgfqpoint{1.381682in}{1.246642in}}%
\pgfpathlineto{\pgfqpoint{1.381474in}{0.952969in}}%
\pgfpathlineto{\pgfqpoint{1.381787in}{1.226155in}}%
\pgfpathlineto{\pgfqpoint{1.382620in}{0.836146in}}%
\pgfpathlineto{\pgfqpoint{1.382828in}{1.251830in}}%
\pgfpathlineto{\pgfqpoint{1.382932in}{1.028522in}}%
\pgfpathlineto{\pgfqpoint{1.383453in}{1.409385in}}%
\pgfpathlineto{\pgfqpoint{1.383349in}{0.728358in}}%
\pgfpathlineto{\pgfqpoint{1.383869in}{1.096469in}}%
\pgfpathlineto{\pgfqpoint{1.383973in}{0.842120in}}%
\pgfpathlineto{\pgfqpoint{1.384390in}{1.218792in}}%
\pgfpathlineto{\pgfqpoint{1.384911in}{1.197778in}}%
\pgfpathlineto{\pgfqpoint{1.385119in}{0.850071in}}%
\pgfpathlineto{\pgfqpoint{1.385536in}{1.371252in}}%
\pgfpathlineto{\pgfqpoint{1.386056in}{0.939601in}}%
\pgfpathlineto{\pgfqpoint{1.386889in}{1.288759in}}%
\pgfpathlineto{\pgfqpoint{1.387098in}{0.958013in}}%
\pgfpathlineto{\pgfqpoint{1.387618in}{1.160354in}}%
\pgfpathlineto{\pgfqpoint{1.387514in}{0.909095in}}%
\pgfpathlineto{\pgfqpoint{1.387723in}{1.066729in}}%
\pgfpathlineto{\pgfqpoint{1.387827in}{0.802207in}}%
\pgfpathlineto{\pgfqpoint{1.388243in}{1.331624in}}%
\pgfpathlineto{\pgfqpoint{1.388764in}{1.101863in}}%
\pgfpathlineto{\pgfqpoint{1.389493in}{1.378703in}}%
\pgfpathlineto{\pgfqpoint{1.389389in}{0.950278in}}%
\pgfpathlineto{\pgfqpoint{1.389597in}{1.102192in}}%
\pgfpathlineto{\pgfqpoint{1.389701in}{0.931526in}}%
\pgfpathlineto{\pgfqpoint{1.389910in}{1.359307in}}%
\pgfpathlineto{\pgfqpoint{1.390639in}{1.069109in}}%
\pgfpathlineto{\pgfqpoint{1.390743in}{1.274164in}}%
\pgfpathlineto{\pgfqpoint{1.391680in}{0.946535in}}%
\pgfpathlineto{\pgfqpoint{1.391784in}{0.945462in}}%
\pgfpathlineto{\pgfqpoint{1.391888in}{1.365653in}}%
\pgfpathlineto{\pgfqpoint{1.392513in}{0.593257in}}%
\pgfpathlineto{\pgfqpoint{1.392825in}{1.062968in}}%
\pgfpathlineto{\pgfqpoint{1.392930in}{0.815510in}}%
\pgfpathlineto{\pgfqpoint{1.393659in}{1.345737in}}%
\pgfpathlineto{\pgfqpoint{1.393867in}{0.848048in}}%
\pgfpathlineto{\pgfqpoint{1.394075in}{1.385192in}}%
\pgfpathlineto{\pgfqpoint{1.394388in}{0.842416in}}%
\pgfpathlineto{\pgfqpoint{1.394908in}{0.950386in}}%
\pgfpathlineto{\pgfqpoint{1.395012in}{0.864266in}}%
\pgfpathlineto{\pgfqpoint{1.395741in}{1.122671in}}%
\pgfpathlineto{\pgfqpoint{1.395846in}{1.065779in}}%
\pgfpathlineto{\pgfqpoint{1.396470in}{0.833437in}}%
\pgfpathlineto{\pgfqpoint{1.396575in}{1.200395in}}%
\pgfpathlineto{\pgfqpoint{1.396783in}{1.067433in}}%
\pgfpathlineto{\pgfqpoint{1.396887in}{1.225172in}}%
\pgfpathlineto{\pgfqpoint{1.397616in}{0.937435in}}%
\pgfpathlineto{\pgfqpoint{1.397720in}{0.749830in}}%
\pgfpathlineto{\pgfqpoint{1.398241in}{1.250111in}}%
\pgfpathlineto{\pgfqpoint{1.398553in}{0.976495in}}%
\pgfpathlineto{\pgfqpoint{1.399491in}{1.178429in}}%
\pgfpathlineto{\pgfqpoint{1.398970in}{0.762705in}}%
\pgfpathlineto{\pgfqpoint{1.399595in}{1.038976in}}%
\pgfpathlineto{\pgfqpoint{1.400428in}{0.702289in}}%
\pgfpathlineto{\pgfqpoint{1.399907in}{1.441384in}}%
\pgfpathlineto{\pgfqpoint{1.400948in}{0.763271in}}%
\pgfpathlineto{\pgfqpoint{1.401261in}{1.216708in}}%
\pgfpathlineto{\pgfqpoint{1.402094in}{0.905233in}}%
\pgfpathlineto{\pgfqpoint{1.402511in}{1.221624in}}%
\pgfpathlineto{\pgfqpoint{1.402615in}{0.880845in}}%
\pgfpathlineto{\pgfqpoint{1.403135in}{1.183260in}}%
\pgfpathlineto{\pgfqpoint{1.403969in}{0.829438in}}%
\pgfpathlineto{\pgfqpoint{1.403864in}{1.323295in}}%
\pgfpathlineto{\pgfqpoint{1.404177in}{0.920750in}}%
\pgfpathlineto{\pgfqpoint{1.404385in}{0.816677in}}%
\pgfpathlineto{\pgfqpoint{1.405322in}{1.298635in}}%
\pgfpathlineto{\pgfqpoint{1.405947in}{0.830963in}}%
\pgfpathlineto{\pgfqpoint{1.406572in}{0.942274in}}%
\pgfpathlineto{\pgfqpoint{1.407197in}{1.493686in}}%
\pgfpathlineto{\pgfqpoint{1.407093in}{0.794508in}}%
\pgfpathlineto{\pgfqpoint{1.407822in}{1.239776in}}%
\pgfpathlineto{\pgfqpoint{1.408759in}{0.578131in}}%
\pgfpathlineto{\pgfqpoint{1.408342in}{1.289156in}}%
\pgfpathlineto{\pgfqpoint{1.408967in}{1.053902in}}%
\pgfpathlineto{\pgfqpoint{1.409071in}{1.074728in}}%
\pgfpathlineto{\pgfqpoint{1.409176in}{1.062086in}}%
\pgfpathlineto{\pgfqpoint{1.409280in}{0.859399in}}%
\pgfpathlineto{\pgfqpoint{1.409696in}{1.193182in}}%
\pgfpathlineto{\pgfqpoint{1.410217in}{1.069987in}}%
\pgfpathlineto{\pgfqpoint{1.410634in}{1.378206in}}%
\pgfpathlineto{\pgfqpoint{1.410842in}{0.490731in}}%
\pgfpathlineto{\pgfqpoint{1.411258in}{0.995656in}}%
\pgfpathlineto{\pgfqpoint{1.411363in}{0.988578in}}%
\pgfpathlineto{\pgfqpoint{1.411571in}{0.825156in}}%
\pgfpathlineto{\pgfqpoint{1.412508in}{1.272323in}}%
\pgfpathlineto{\pgfqpoint{1.412612in}{0.753490in}}%
\pgfpathlineto{\pgfqpoint{1.413550in}{0.990635in}}%
\pgfpathlineto{\pgfqpoint{1.413654in}{1.375160in}}%
\pgfpathlineto{\pgfqpoint{1.414174in}{0.753902in}}%
\pgfpathlineto{\pgfqpoint{1.414591in}{1.188409in}}%
\pgfpathlineto{\pgfqpoint{1.415528in}{0.855971in}}%
\pgfpathlineto{\pgfqpoint{1.414799in}{1.291473in}}%
\pgfpathlineto{\pgfqpoint{1.415737in}{0.995453in}}%
\pgfpathlineto{\pgfqpoint{1.415841in}{1.550163in}}%
\pgfpathlineto{\pgfqpoint{1.416257in}{0.778037in}}%
\pgfpathlineto{\pgfqpoint{1.416778in}{1.087815in}}%
\pgfpathlineto{\pgfqpoint{1.417507in}{0.728761in}}%
\pgfpathlineto{\pgfqpoint{1.417194in}{1.306524in}}%
\pgfpathlineto{\pgfqpoint{1.417923in}{0.943922in}}%
\pgfpathlineto{\pgfqpoint{1.418757in}{1.415420in}}%
\pgfpathlineto{\pgfqpoint{1.418132in}{0.837313in}}%
\pgfpathlineto{\pgfqpoint{1.419069in}{1.220944in}}%
\pgfpathlineto{\pgfqpoint{1.419486in}{0.621949in}}%
\pgfpathlineto{\pgfqpoint{1.419277in}{1.256023in}}%
\pgfpathlineto{\pgfqpoint{1.420215in}{1.041192in}}%
\pgfpathlineto{\pgfqpoint{1.420319in}{1.378826in}}%
\pgfpathlineto{\pgfqpoint{1.421048in}{1.026675in}}%
\pgfpathlineto{\pgfqpoint{1.421256in}{1.267677in}}%
\pgfpathlineto{\pgfqpoint{1.421881in}{0.845021in}}%
\pgfpathlineto{\pgfqpoint{1.421985in}{1.334903in}}%
\pgfpathlineto{\pgfqpoint{1.422297in}{0.921609in}}%
\pgfpathlineto{\pgfqpoint{1.422402in}{1.450439in}}%
\pgfpathlineto{\pgfqpoint{1.423339in}{0.758956in}}%
\pgfpathlineto{\pgfqpoint{1.423547in}{1.319634in}}%
\pgfpathlineto{\pgfqpoint{1.424797in}{1.205636in}}%
\pgfpathlineto{\pgfqpoint{1.425838in}{0.761933in}}%
\pgfpathlineto{\pgfqpoint{1.425630in}{1.433586in}}%
\pgfpathlineto{\pgfqpoint{1.426046in}{0.998890in}}%
\pgfpathlineto{\pgfqpoint{1.426880in}{1.376294in}}%
\pgfpathlineto{\pgfqpoint{1.426463in}{0.832824in}}%
\pgfpathlineto{\pgfqpoint{1.427088in}{1.045831in}}%
\pgfpathlineto{\pgfqpoint{1.427713in}{0.859783in}}%
\pgfpathlineto{\pgfqpoint{1.427921in}{1.183044in}}%
\pgfpathlineto{\pgfqpoint{1.428025in}{1.105814in}}%
\pgfpathlineto{\pgfqpoint{1.428338in}{1.293023in}}%
\pgfpathlineto{\pgfqpoint{1.428546in}{0.997480in}}%
\pgfpathlineto{\pgfqpoint{1.428858in}{0.695736in}}%
\pgfpathlineto{\pgfqpoint{1.429379in}{1.173074in}}%
\pgfpathlineto{\pgfqpoint{1.429483in}{1.088052in}}%
\pgfpathlineto{\pgfqpoint{1.429691in}{1.099160in}}%
\pgfpathlineto{\pgfqpoint{1.429796in}{1.067773in}}%
\pgfpathlineto{\pgfqpoint{1.430212in}{1.293566in}}%
\pgfpathlineto{\pgfqpoint{1.430108in}{0.840022in}}%
\pgfpathlineto{\pgfqpoint{1.430837in}{1.285345in}}%
\pgfpathlineto{\pgfqpoint{1.431045in}{0.931195in}}%
\pgfpathlineto{\pgfqpoint{1.431983in}{1.029311in}}%
\pgfpathlineto{\pgfqpoint{1.432295in}{1.355532in}}%
\pgfpathlineto{\pgfqpoint{1.432191in}{0.706875in}}%
\pgfpathlineto{\pgfqpoint{1.433024in}{1.092466in}}%
\pgfpathlineto{\pgfqpoint{1.433857in}{1.192762in}}%
\pgfpathlineto{\pgfqpoint{1.434065in}{0.926586in}}%
\pgfpathlineto{\pgfqpoint{1.434378in}{1.176135in}}%
\pgfpathlineto{\pgfqpoint{1.434586in}{0.792847in}}%
\pgfpathlineto{\pgfqpoint{1.435211in}{1.011853in}}%
\pgfpathlineto{\pgfqpoint{1.435940in}{1.259685in}}%
\pgfpathlineto{\pgfqpoint{1.435836in}{0.937323in}}%
\pgfpathlineto{\pgfqpoint{1.436044in}{1.019433in}}%
\pgfpathlineto{\pgfqpoint{1.436877in}{0.796362in}}%
\pgfpathlineto{\pgfqpoint{1.436565in}{1.378666in}}%
\pgfpathlineto{\pgfqpoint{1.437190in}{0.962409in}}%
\pgfpathlineto{\pgfqpoint{1.437814in}{1.287603in}}%
\pgfpathlineto{\pgfqpoint{1.438023in}{0.736661in}}%
\pgfpathlineto{\pgfqpoint{1.439064in}{1.279971in}}%
\pgfpathlineto{\pgfqpoint{1.439168in}{0.957482in}}%
\pgfpathlineto{\pgfqpoint{1.439272in}{0.966863in}}%
\pgfpathlineto{\pgfqpoint{1.439793in}{1.379360in}}%
\pgfpathlineto{\pgfqpoint{1.439689in}{0.826316in}}%
\pgfpathlineto{\pgfqpoint{1.440210in}{1.335062in}}%
\pgfpathlineto{\pgfqpoint{1.440314in}{0.859607in}}%
\pgfpathlineto{\pgfqpoint{1.440418in}{1.345491in}}%
\pgfpathlineto{\pgfqpoint{1.441355in}{0.972625in}}%
\pgfpathlineto{\pgfqpoint{1.441876in}{0.792783in}}%
\pgfpathlineto{\pgfqpoint{1.441772in}{1.231722in}}%
\pgfpathlineto{\pgfqpoint{1.441980in}{1.210451in}}%
\pgfpathlineto{\pgfqpoint{1.442084in}{1.422113in}}%
\pgfpathlineto{\pgfqpoint{1.442501in}{0.852622in}}%
\pgfpathlineto{\pgfqpoint{1.443021in}{1.141336in}}%
\pgfpathlineto{\pgfqpoint{1.443750in}{0.917979in}}%
\pgfpathlineto{\pgfqpoint{1.443855in}{1.243386in}}%
\pgfpathlineto{\pgfqpoint{1.443959in}{0.961023in}}%
\pgfpathlineto{\pgfqpoint{1.444063in}{1.219950in}}%
\pgfpathlineto{\pgfqpoint{1.444584in}{0.797940in}}%
\pgfpathlineto{\pgfqpoint{1.445000in}{1.202552in}}%
\pgfpathlineto{\pgfqpoint{1.445313in}{0.842298in}}%
\pgfpathlineto{\pgfqpoint{1.445937in}{1.303478in}}%
\pgfpathlineto{\pgfqpoint{1.446042in}{1.102939in}}%
\pgfpathlineto{\pgfqpoint{1.446771in}{1.337059in}}%
\pgfpathlineto{\pgfqpoint{1.446354in}{0.854145in}}%
\pgfpathlineto{\pgfqpoint{1.446979in}{1.227292in}}%
\pgfpathlineto{\pgfqpoint{1.447291in}{0.858996in}}%
\pgfpathlineto{\pgfqpoint{1.448020in}{1.369663in}}%
\pgfpathlineto{\pgfqpoint{1.448124in}{1.040380in}}%
\pgfpathlineto{\pgfqpoint{1.448958in}{1.234872in}}%
\pgfpathlineto{\pgfqpoint{1.448749in}{0.708398in}}%
\pgfpathlineto{\pgfqpoint{1.449166in}{1.119645in}}%
\pgfpathlineto{\pgfqpoint{1.449686in}{1.232553in}}%
\pgfpathlineto{\pgfqpoint{1.450207in}{0.722892in}}%
\pgfpathlineto{\pgfqpoint{1.451040in}{1.327919in}}%
\pgfpathlineto{\pgfqpoint{1.451457in}{1.267822in}}%
\pgfpathlineto{\pgfqpoint{1.451561in}{0.813853in}}%
\pgfpathlineto{\pgfqpoint{1.451665in}{1.338465in}}%
\pgfpathlineto{\pgfqpoint{1.452602in}{0.942246in}}%
\pgfpathlineto{\pgfqpoint{1.452707in}{1.433939in}}%
\pgfpathlineto{\pgfqpoint{1.453331in}{0.685712in}}%
\pgfpathlineto{\pgfqpoint{1.453644in}{1.139601in}}%
\pgfpathlineto{\pgfqpoint{1.454165in}{0.834291in}}%
\pgfpathlineto{\pgfqpoint{1.454269in}{1.160864in}}%
\pgfpathlineto{\pgfqpoint{1.454373in}{1.156046in}}%
\pgfpathlineto{\pgfqpoint{1.454477in}{1.286864in}}%
\pgfpathlineto{\pgfqpoint{1.454581in}{0.938151in}}%
\pgfpathlineto{\pgfqpoint{1.455414in}{1.149173in}}%
\pgfpathlineto{\pgfqpoint{1.455623in}{0.859174in}}%
\pgfpathlineto{\pgfqpoint{1.455831in}{1.276720in}}%
\pgfpathlineto{\pgfqpoint{1.456560in}{0.963696in}}%
\pgfpathlineto{\pgfqpoint{1.456664in}{1.291645in}}%
\pgfpathlineto{\pgfqpoint{1.457289in}{0.730657in}}%
\pgfpathlineto{\pgfqpoint{1.457601in}{0.888927in}}%
\pgfpathlineto{\pgfqpoint{1.457705in}{0.894288in}}%
\pgfpathlineto{\pgfqpoint{1.458226in}{0.881731in}}%
\pgfpathlineto{\pgfqpoint{1.458747in}{1.392815in}}%
\pgfpathlineto{\pgfqpoint{1.458851in}{0.848667in}}%
\pgfpathlineto{\pgfqpoint{1.459788in}{1.218788in}}%
\pgfpathlineto{\pgfqpoint{1.459892in}{1.284681in}}%
\pgfpathlineto{\pgfqpoint{1.460205in}{1.003900in}}%
\pgfpathlineto{\pgfqpoint{1.460621in}{1.116365in}}%
\pgfpathlineto{\pgfqpoint{1.461038in}{0.806382in}}%
\pgfpathlineto{\pgfqpoint{1.461559in}{1.256853in}}%
\pgfpathlineto{\pgfqpoint{1.462183in}{1.036612in}}%
\pgfpathlineto{\pgfqpoint{1.462496in}{1.398642in}}%
\pgfpathlineto{\pgfqpoint{1.463121in}{0.842226in}}%
\pgfpathlineto{\pgfqpoint{1.463641in}{1.114256in}}%
\pgfpathlineto{\pgfqpoint{1.463746in}{1.261950in}}%
\pgfpathlineto{\pgfqpoint{1.464162in}{0.882661in}}%
\pgfpathlineto{\pgfqpoint{1.464683in}{1.075681in}}%
\pgfpathlineto{\pgfqpoint{1.465620in}{1.517911in}}%
\pgfpathlineto{\pgfqpoint{1.464891in}{0.936598in}}%
\pgfpathlineto{\pgfqpoint{1.465828in}{1.114844in}}%
\pgfpathlineto{\pgfqpoint{1.466349in}{0.778015in}}%
\pgfpathlineto{\pgfqpoint{1.466766in}{1.019031in}}%
\pgfpathlineto{\pgfqpoint{1.467495in}{1.355410in}}%
\pgfpathlineto{\pgfqpoint{1.466974in}{0.836470in}}%
\pgfpathlineto{\pgfqpoint{1.467807in}{1.014163in}}%
\pgfpathlineto{\pgfqpoint{1.467911in}{0.767125in}}%
\pgfpathlineto{\pgfqpoint{1.468640in}{1.240569in}}%
\pgfpathlineto{\pgfqpoint{1.468848in}{1.097358in}}%
\pgfpathlineto{\pgfqpoint{1.468953in}{1.078157in}}%
\pgfpathlineto{\pgfqpoint{1.469057in}{1.268834in}}%
\pgfpathlineto{\pgfqpoint{1.469265in}{0.863953in}}%
\pgfpathlineto{\pgfqpoint{1.470098in}{1.168694in}}%
\pgfpathlineto{\pgfqpoint{1.470723in}{0.851905in}}%
\pgfpathlineto{\pgfqpoint{1.470306in}{1.284518in}}%
\pgfpathlineto{\pgfqpoint{1.471244in}{0.946279in}}%
\pgfpathlineto{\pgfqpoint{1.471452in}{1.227983in}}%
\pgfpathlineto{\pgfqpoint{1.471660in}{0.877367in}}%
\pgfpathlineto{\pgfqpoint{1.472181in}{0.949882in}}%
\pgfpathlineto{\pgfqpoint{1.472285in}{0.743613in}}%
\pgfpathlineto{\pgfqpoint{1.473118in}{1.238652in}}%
\pgfpathlineto{\pgfqpoint{1.473222in}{1.228594in}}%
\pgfpathlineto{\pgfqpoint{1.473847in}{0.758391in}}%
\pgfpathlineto{\pgfqpoint{1.474368in}{1.075355in}}%
\pgfpathlineto{\pgfqpoint{1.474993in}{1.014759in}}%
\pgfpathlineto{\pgfqpoint{1.475513in}{1.251358in}}%
\pgfpathlineto{\pgfqpoint{1.475930in}{0.853535in}}%
\pgfpathlineto{\pgfqpoint{1.476659in}{0.982699in}}%
\pgfpathlineto{\pgfqpoint{1.477284in}{1.262734in}}%
\pgfpathlineto{\pgfqpoint{1.477180in}{0.855965in}}%
\pgfpathlineto{\pgfqpoint{1.477700in}{1.043858in}}%
\pgfpathlineto{\pgfqpoint{1.478013in}{0.945276in}}%
\pgfpathlineto{\pgfqpoint{1.478429in}{1.152430in}}%
\pgfpathlineto{\pgfqpoint{1.478534in}{1.328656in}}%
\pgfpathlineto{\pgfqpoint{1.478846in}{0.760201in}}%
\pgfpathlineto{\pgfqpoint{1.479263in}{1.073643in}}%
\pgfpathlineto{\pgfqpoint{1.479471in}{0.787642in}}%
\pgfpathlineto{\pgfqpoint{1.480096in}{1.240958in}}%
\pgfpathlineto{\pgfqpoint{1.480304in}{0.929713in}}%
\pgfpathlineto{\pgfqpoint{1.480616in}{1.209539in}}%
\pgfpathlineto{\pgfqpoint{1.480825in}{0.794504in}}%
\pgfpathlineto{\pgfqpoint{1.481450in}{1.137371in}}%
\pgfpathlineto{\pgfqpoint{1.481658in}{0.737481in}}%
\pgfpathlineto{\pgfqpoint{1.482074in}{1.309406in}}%
\pgfpathlineto{\pgfqpoint{1.482595in}{1.015485in}}%
\pgfpathlineto{\pgfqpoint{1.482699in}{1.009780in}}%
\pgfpathlineto{\pgfqpoint{1.483220in}{1.354589in}}%
\pgfpathlineto{\pgfqpoint{1.483116in}{0.889221in}}%
\pgfpathlineto{\pgfqpoint{1.483845in}{1.085623in}}%
\pgfpathlineto{\pgfqpoint{1.484470in}{0.828347in}}%
\pgfpathlineto{\pgfqpoint{1.484365in}{1.161979in}}%
\pgfpathlineto{\pgfqpoint{1.484990in}{1.021417in}}%
\pgfpathlineto{\pgfqpoint{1.485407in}{1.313076in}}%
\pgfpathlineto{\pgfqpoint{1.485615in}{0.802284in}}%
\pgfpathlineto{\pgfqpoint{1.485928in}{1.111851in}}%
\pgfpathlineto{\pgfqpoint{1.486448in}{0.855724in}}%
\pgfpathlineto{\pgfqpoint{1.486344in}{1.148578in}}%
\pgfpathlineto{\pgfqpoint{1.486969in}{1.080129in}}%
\pgfpathlineto{\pgfqpoint{1.487281in}{1.146232in}}%
\pgfpathlineto{\pgfqpoint{1.487490in}{0.908998in}}%
\pgfpathlineto{\pgfqpoint{1.487594in}{1.383266in}}%
\pgfpathlineto{\pgfqpoint{1.488635in}{1.081314in}}%
\pgfpathlineto{\pgfqpoint{1.488844in}{1.136385in}}%
\pgfpathlineto{\pgfqpoint{1.488948in}{1.089070in}}%
\pgfpathlineto{\pgfqpoint{1.489052in}{0.876364in}}%
\pgfpathlineto{\pgfqpoint{1.489468in}{1.208484in}}%
\pgfpathlineto{\pgfqpoint{1.489989in}{1.121786in}}%
\pgfpathlineto{\pgfqpoint{1.490510in}{0.827268in}}%
\pgfpathlineto{\pgfqpoint{1.490406in}{1.152134in}}%
\pgfpathlineto{\pgfqpoint{1.490822in}{1.132547in}}%
\pgfpathlineto{\pgfqpoint{1.490926in}{1.273421in}}%
\pgfpathlineto{\pgfqpoint{1.491551in}{0.920023in}}%
\pgfpathlineto{\pgfqpoint{1.491864in}{1.146042in}}%
\pgfpathlineto{\pgfqpoint{1.492697in}{0.954552in}}%
\pgfpathlineto{\pgfqpoint{1.492593in}{1.298932in}}%
\pgfpathlineto{\pgfqpoint{1.492801in}{0.963979in}}%
\pgfpathlineto{\pgfqpoint{1.493113in}{1.289299in}}%
\pgfpathlineto{\pgfqpoint{1.493426in}{0.787875in}}%
\pgfpathlineto{\pgfqpoint{1.493842in}{1.287445in}}%
\pgfpathlineto{\pgfqpoint{1.494467in}{0.768901in}}%
\pgfpathlineto{\pgfqpoint{1.494988in}{1.098984in}}%
\pgfpathlineto{\pgfqpoint{1.495092in}{1.222776in}}%
\pgfpathlineto{\pgfqpoint{1.495404in}{0.683966in}}%
\pgfpathlineto{\pgfqpoint{1.496133in}{1.151959in}}%
\pgfpathlineto{\pgfqpoint{1.496446in}{0.902085in}}%
\pgfpathlineto{\pgfqpoint{1.496342in}{1.290326in}}%
\pgfpathlineto{\pgfqpoint{1.497383in}{1.018990in}}%
\pgfpathlineto{\pgfqpoint{1.498008in}{1.251978in}}%
\pgfpathlineto{\pgfqpoint{1.497591in}{0.880912in}}%
\pgfpathlineto{\pgfqpoint{1.498529in}{1.113398in}}%
\pgfpathlineto{\pgfqpoint{1.498633in}{0.786060in}}%
\pgfpathlineto{\pgfqpoint{1.498945in}{1.241895in}}%
\pgfpathlineto{\pgfqpoint{1.499674in}{1.004949in}}%
\pgfpathlineto{\pgfqpoint{1.499778in}{0.797973in}}%
\pgfpathlineto{\pgfqpoint{1.500299in}{1.223301in}}%
\pgfpathlineto{\pgfqpoint{1.500716in}{0.938436in}}%
\pgfpathlineto{\pgfqpoint{1.501861in}{1.355890in}}%
\pgfpathlineto{\pgfqpoint{1.501653in}{0.819549in}}%
\pgfpathlineto{\pgfqpoint{1.501965in}{1.280742in}}%
\pgfpathlineto{\pgfqpoint{1.502382in}{0.806902in}}%
\pgfpathlineto{\pgfqpoint{1.502590in}{1.289975in}}%
\pgfpathlineto{\pgfqpoint{1.503007in}{0.994826in}}%
\pgfpathlineto{\pgfqpoint{1.503111in}{1.353102in}}%
\pgfpathlineto{\pgfqpoint{1.503736in}{0.884322in}}%
\pgfpathlineto{\pgfqpoint{1.504048in}{1.077008in}}%
\pgfpathlineto{\pgfqpoint{1.504152in}{0.868710in}}%
\pgfpathlineto{\pgfqpoint{1.504881in}{1.256560in}}%
\pgfpathlineto{\pgfqpoint{1.505090in}{1.003770in}}%
\pgfpathlineto{\pgfqpoint{1.505610in}{1.325578in}}%
\pgfpathlineto{\pgfqpoint{1.506235in}{1.291048in}}%
\pgfpathlineto{\pgfqpoint{1.506652in}{0.732263in}}%
\pgfpathlineto{\pgfqpoint{1.507381in}{0.934570in}}%
\pgfpathlineto{\pgfqpoint{1.507797in}{1.349875in}}%
\pgfpathlineto{\pgfqpoint{1.508110in}{0.830482in}}%
\pgfpathlineto{\pgfqpoint{1.508422in}{1.208417in}}%
\pgfpathlineto{\pgfqpoint{1.509255in}{0.803811in}}%
\pgfpathlineto{\pgfqpoint{1.509568in}{0.978395in}}%
\pgfpathlineto{\pgfqpoint{1.510609in}{1.259454in}}%
\pgfpathlineto{\pgfqpoint{1.510713in}{1.229102in}}%
\pgfpathlineto{\pgfqpoint{1.511442in}{0.870008in}}%
\pgfpathlineto{\pgfqpoint{1.511859in}{0.892822in}}%
\pgfpathlineto{\pgfqpoint{1.512484in}{1.238860in}}%
\pgfpathlineto{\pgfqpoint{1.512171in}{0.885105in}}%
\pgfpathlineto{\pgfqpoint{1.513004in}{0.997119in}}%
\pgfpathlineto{\pgfqpoint{1.513213in}{1.103215in}}%
\pgfpathlineto{\pgfqpoint{1.513421in}{1.044461in}}%
\pgfpathlineto{\pgfqpoint{1.514046in}{0.788609in}}%
\pgfpathlineto{\pgfqpoint{1.513837in}{1.290653in}}%
\pgfpathlineto{\pgfqpoint{1.514462in}{0.829516in}}%
\pgfpathlineto{\pgfqpoint{1.514983in}{1.298569in}}%
\pgfpathlineto{\pgfqpoint{1.514775in}{0.688870in}}%
\pgfpathlineto{\pgfqpoint{1.515608in}{1.208744in}}%
\pgfpathlineto{\pgfqpoint{1.515712in}{0.783522in}}%
\pgfpathlineto{\pgfqpoint{1.516233in}{1.232334in}}%
\pgfpathlineto{\pgfqpoint{1.516649in}{1.231359in}}%
\pgfpathlineto{\pgfqpoint{1.517899in}{0.794519in}}%
\pgfpathlineto{\pgfqpoint{1.518524in}{1.227021in}}%
\pgfpathlineto{\pgfqpoint{1.518940in}{0.931051in}}%
\pgfpathlineto{\pgfqpoint{1.519044in}{0.876083in}}%
\pgfpathlineto{\pgfqpoint{1.519149in}{1.169004in}}%
\pgfpathlineto{\pgfqpoint{1.519669in}{0.942050in}}%
\pgfpathlineto{\pgfqpoint{1.520294in}{1.300479in}}%
\pgfpathlineto{\pgfqpoint{1.520086in}{0.808100in}}%
\pgfpathlineto{\pgfqpoint{1.520711in}{0.984356in}}%
\pgfpathlineto{\pgfqpoint{1.520919in}{0.893234in}}%
\pgfpathlineto{\pgfqpoint{1.521023in}{1.460725in}}%
\pgfpathlineto{\pgfqpoint{1.521960in}{0.771327in}}%
\pgfpathlineto{\pgfqpoint{1.522065in}{1.316639in}}%
\pgfpathlineto{\pgfqpoint{1.522794in}{0.780032in}}%
\pgfpathlineto{\pgfqpoint{1.523210in}{1.009078in}}%
\pgfpathlineto{\pgfqpoint{1.523314in}{1.222111in}}%
\pgfpathlineto{\pgfqpoint{1.523835in}{0.683855in}}%
\pgfpathlineto{\pgfqpoint{1.524251in}{1.185123in}}%
\pgfpathlineto{\pgfqpoint{1.524564in}{0.824051in}}%
\pgfpathlineto{\pgfqpoint{1.524772in}{1.340991in}}%
\pgfpathlineto{\pgfqpoint{1.525293in}{0.898056in}}%
\pgfpathlineto{\pgfqpoint{1.525709in}{0.794466in}}%
\pgfpathlineto{\pgfqpoint{1.526438in}{1.336652in}}%
\pgfpathlineto{\pgfqpoint{1.526959in}{0.828520in}}%
\pgfpathlineto{\pgfqpoint{1.527584in}{0.994530in}}%
\pgfpathlineto{\pgfqpoint{1.528521in}{1.261340in}}%
\pgfpathlineto{\pgfqpoint{1.527792in}{0.817033in}}%
\pgfpathlineto{\pgfqpoint{1.528625in}{1.008687in}}%
\pgfpathlineto{\pgfqpoint{1.528730in}{0.969803in}}%
\pgfpathlineto{\pgfqpoint{1.528834in}{1.182186in}}%
\pgfpathlineto{\pgfqpoint{1.529042in}{1.026660in}}%
\pgfpathlineto{\pgfqpoint{1.529771in}{1.259294in}}%
\pgfpathlineto{\pgfqpoint{1.529563in}{0.916919in}}%
\pgfpathlineto{\pgfqpoint{1.530083in}{1.030414in}}%
\pgfpathlineto{\pgfqpoint{1.530292in}{1.164279in}}%
\pgfpathlineto{\pgfqpoint{1.530812in}{0.906257in}}%
\pgfpathlineto{\pgfqpoint{1.530917in}{1.078876in}}%
\pgfpathlineto{\pgfqpoint{1.531021in}{0.736768in}}%
\pgfpathlineto{\pgfqpoint{1.531645in}{1.263288in}}%
\pgfpathlineto{\pgfqpoint{1.531958in}{1.074630in}}%
\pgfpathlineto{\pgfqpoint{1.532374in}{1.250308in}}%
\pgfpathlineto{\pgfqpoint{1.532687in}{0.833028in}}%
\pgfpathlineto{\pgfqpoint{1.532999in}{1.102786in}}%
\pgfpathlineto{\pgfqpoint{1.533312in}{0.750930in}}%
\pgfpathlineto{\pgfqpoint{1.533416in}{1.233601in}}%
\pgfpathlineto{\pgfqpoint{1.534145in}{0.947968in}}%
\pgfpathlineto{\pgfqpoint{1.534457in}{1.310655in}}%
\pgfpathlineto{\pgfqpoint{1.534353in}{0.937314in}}%
\pgfpathlineto{\pgfqpoint{1.535290in}{1.157238in}}%
\pgfpathlineto{\pgfqpoint{1.535395in}{1.147705in}}%
\pgfpathlineto{\pgfqpoint{1.536019in}{0.821081in}}%
\pgfpathlineto{\pgfqpoint{1.536332in}{1.319309in}}%
\pgfpathlineto{\pgfqpoint{1.536540in}{0.991525in}}%
\pgfpathlineto{\pgfqpoint{1.536644in}{0.996984in}}%
\pgfpathlineto{\pgfqpoint{1.537477in}{1.308827in}}%
\pgfpathlineto{\pgfqpoint{1.537269in}{0.874360in}}%
\pgfpathlineto{\pgfqpoint{1.537686in}{1.069996in}}%
\pgfpathlineto{\pgfqpoint{1.538311in}{1.181759in}}%
\pgfpathlineto{\pgfqpoint{1.538727in}{0.763538in}}%
\pgfpathlineto{\pgfqpoint{1.538831in}{1.351823in}}%
\pgfpathlineto{\pgfqpoint{1.539873in}{1.038896in}}%
\pgfpathlineto{\pgfqpoint{1.539977in}{0.795965in}}%
\pgfpathlineto{\pgfqpoint{1.540393in}{1.264789in}}%
\pgfpathlineto{\pgfqpoint{1.540914in}{1.162863in}}%
\pgfpathlineto{\pgfqpoint{1.541539in}{0.946627in}}%
\pgfpathlineto{\pgfqpoint{1.541851in}{1.207704in}}%
\pgfpathlineto{\pgfqpoint{1.542060in}{1.110159in}}%
\pgfpathlineto{\pgfqpoint{1.542789in}{0.781066in}}%
\pgfpathlineto{\pgfqpoint{1.542684in}{1.264710in}}%
\pgfpathlineto{\pgfqpoint{1.543101in}{1.114728in}}%
\pgfpathlineto{\pgfqpoint{1.543830in}{1.287790in}}%
\pgfpathlineto{\pgfqpoint{1.543518in}{0.741830in}}%
\pgfpathlineto{\pgfqpoint{1.544142in}{1.139911in}}%
\pgfpathlineto{\pgfqpoint{1.544559in}{0.835045in}}%
\pgfpathlineto{\pgfqpoint{1.544663in}{1.244381in}}%
\pgfpathlineto{\pgfqpoint{1.545288in}{1.047555in}}%
\pgfpathlineto{\pgfqpoint{1.545705in}{0.876556in}}%
\pgfpathlineto{\pgfqpoint{1.545809in}{1.236688in}}%
\pgfpathlineto{\pgfqpoint{1.546017in}{1.063847in}}%
\pgfpathlineto{\pgfqpoint{1.546642in}{1.325568in}}%
\pgfpathlineto{\pgfqpoint{1.546538in}{0.903679in}}%
\pgfpathlineto{\pgfqpoint{1.547163in}{1.175937in}}%
\pgfpathlineto{\pgfqpoint{1.547267in}{0.703024in}}%
\pgfpathlineto{\pgfqpoint{1.547891in}{1.325543in}}%
\pgfpathlineto{\pgfqpoint{1.548204in}{0.994738in}}%
\pgfpathlineto{\pgfqpoint{1.548620in}{1.373115in}}%
\pgfpathlineto{\pgfqpoint{1.548933in}{0.879566in}}%
\pgfpathlineto{\pgfqpoint{1.549349in}{1.173016in}}%
\pgfpathlineto{\pgfqpoint{1.549454in}{1.235146in}}%
\pgfpathlineto{\pgfqpoint{1.549558in}{0.830116in}}%
\pgfpathlineto{\pgfqpoint{1.550078in}{1.203750in}}%
\pgfpathlineto{\pgfqpoint{1.550807in}{1.261036in}}%
\pgfpathlineto{\pgfqpoint{1.551328in}{0.752601in}}%
\pgfpathlineto{\pgfqpoint{1.551953in}{1.349129in}}%
\pgfpathlineto{\pgfqpoint{1.552474in}{1.007602in}}%
\pgfpathlineto{\pgfqpoint{1.553099in}{0.909305in}}%
\pgfpathlineto{\pgfqpoint{1.552786in}{1.258396in}}%
\pgfpathlineto{\pgfqpoint{1.553203in}{1.130070in}}%
\pgfpathlineto{\pgfqpoint{1.553307in}{1.370845in}}%
\pgfpathlineto{\pgfqpoint{1.553828in}{0.651004in}}%
\pgfpathlineto{\pgfqpoint{1.554244in}{1.164782in}}%
\pgfpathlineto{\pgfqpoint{1.554557in}{1.269528in}}%
\pgfpathlineto{\pgfqpoint{1.555390in}{0.751768in}}%
\pgfpathlineto{\pgfqpoint{1.555806in}{1.238086in}}%
\pgfpathlineto{\pgfqpoint{1.556535in}{1.006563in}}%
\pgfpathlineto{\pgfqpoint{1.557264in}{0.642690in}}%
\pgfpathlineto{\pgfqpoint{1.557160in}{1.278074in}}%
\pgfpathlineto{\pgfqpoint{1.557577in}{1.127468in}}%
\pgfpathlineto{\pgfqpoint{1.557993in}{0.779072in}}%
\pgfpathlineto{\pgfqpoint{1.558514in}{1.205001in}}%
\pgfpathlineto{\pgfqpoint{1.558826in}{0.979289in}}%
\pgfpathlineto{\pgfqpoint{1.559451in}{1.406860in}}%
\pgfpathlineto{\pgfqpoint{1.559139in}{0.873975in}}%
\pgfpathlineto{\pgfqpoint{1.559868in}{1.039880in}}%
\pgfpathlineto{\pgfqpoint{1.560284in}{0.847584in}}%
\pgfpathlineto{\pgfqpoint{1.560180in}{1.044154in}}%
\pgfpathlineto{\pgfqpoint{1.560388in}{1.042974in}}%
\pgfpathlineto{\pgfqpoint{1.560493in}{1.273998in}}%
\pgfpathlineto{\pgfqpoint{1.560805in}{0.804346in}}%
\pgfpathlineto{\pgfqpoint{1.561430in}{1.127878in}}%
\pgfpathlineto{\pgfqpoint{1.562159in}{1.277161in}}%
\pgfpathlineto{\pgfqpoint{1.562575in}{0.783672in}}%
\pgfpathlineto{\pgfqpoint{1.562680in}{1.245795in}}%
\pgfpathlineto{\pgfqpoint{1.563617in}{0.997985in}}%
\pgfpathlineto{\pgfqpoint{1.564242in}{0.819026in}}%
\pgfpathlineto{\pgfqpoint{1.564346in}{1.146043in}}%
\pgfpathlineto{\pgfqpoint{1.564658in}{1.287591in}}%
\pgfpathlineto{\pgfqpoint{1.565075in}{0.850844in}}%
\pgfpathlineto{\pgfqpoint{1.565283in}{0.940267in}}%
\pgfpathlineto{\pgfqpoint{1.565387in}{0.939650in}}%
\pgfpathlineto{\pgfqpoint{1.566220in}{1.425773in}}%
\pgfpathlineto{\pgfqpoint{1.565595in}{0.777641in}}%
\pgfpathlineto{\pgfqpoint{1.566429in}{1.020278in}}%
\pgfpathlineto{\pgfqpoint{1.566949in}{0.835459in}}%
\pgfpathlineto{\pgfqpoint{1.566845in}{1.230924in}}%
\pgfpathlineto{\pgfqpoint{1.567366in}{1.055132in}}%
\pgfpathlineto{\pgfqpoint{1.567678in}{1.282960in}}%
\pgfpathlineto{\pgfqpoint{1.567887in}{0.856658in}}%
\pgfpathlineto{\pgfqpoint{1.568199in}{1.153735in}}%
\pgfpathlineto{\pgfqpoint{1.568303in}{0.865499in}}%
\pgfpathlineto{\pgfqpoint{1.569240in}{1.089083in}}%
\pgfpathlineto{\pgfqpoint{1.569865in}{0.852229in}}%
\pgfpathlineto{\pgfqpoint{1.570074in}{1.111104in}}%
\pgfpathlineto{\pgfqpoint{1.570178in}{0.866570in}}%
\pgfpathlineto{\pgfqpoint{1.570594in}{1.253546in}}%
\pgfpathlineto{\pgfqpoint{1.571219in}{0.934606in}}%
\pgfpathlineto{\pgfqpoint{1.571427in}{0.839715in}}%
\pgfpathlineto{\pgfqpoint{1.571740in}{1.149769in}}%
\pgfpathlineto{\pgfqpoint{1.572052in}{0.954003in}}%
\pgfpathlineto{\pgfqpoint{1.572156in}{1.223401in}}%
\pgfpathlineto{\pgfqpoint{1.572469in}{0.896204in}}%
\pgfpathlineto{\pgfqpoint{1.573094in}{1.035569in}}%
\pgfpathlineto{\pgfqpoint{1.573302in}{0.845717in}}%
\pgfpathlineto{\pgfqpoint{1.573927in}{1.191198in}}%
\pgfpathlineto{\pgfqpoint{1.574031in}{1.063502in}}%
\pgfpathlineto{\pgfqpoint{1.574864in}{1.287291in}}%
\pgfpathlineto{\pgfqpoint{1.574343in}{0.478138in}}%
\pgfpathlineto{\pgfqpoint{1.575176in}{1.177088in}}%
\pgfpathlineto{\pgfqpoint{1.575801in}{0.826899in}}%
\pgfpathlineto{\pgfqpoint{1.575489in}{1.211524in}}%
\pgfpathlineto{\pgfqpoint{1.576010in}{1.082512in}}%
\pgfpathlineto{\pgfqpoint{1.576114in}{1.300370in}}%
\pgfpathlineto{\pgfqpoint{1.576218in}{0.824343in}}%
\pgfpathlineto{\pgfqpoint{1.577051in}{1.046606in}}%
\pgfpathlineto{\pgfqpoint{1.577676in}{1.392769in}}%
\pgfpathlineto{\pgfqpoint{1.577363in}{0.880426in}}%
\pgfpathlineto{\pgfqpoint{1.577884in}{1.174971in}}%
\pgfpathlineto{\pgfqpoint{1.578301in}{0.749863in}}%
\pgfpathlineto{\pgfqpoint{1.578821in}{1.218751in}}%
\pgfpathlineto{\pgfqpoint{1.578926in}{1.034597in}}%
\pgfpathlineto{\pgfqpoint{1.579446in}{1.216953in}}%
\pgfpathlineto{\pgfqpoint{1.579342in}{0.894193in}}%
\pgfpathlineto{\pgfqpoint{1.579967in}{1.028304in}}%
\pgfpathlineto{\pgfqpoint{1.580175in}{0.806271in}}%
\pgfpathlineto{\pgfqpoint{1.580384in}{1.130906in}}%
\pgfpathlineto{\pgfqpoint{1.580904in}{1.002204in}}%
\pgfpathlineto{\pgfqpoint{1.581008in}{1.372192in}}%
\pgfpathlineto{\pgfqpoint{1.581946in}{0.870400in}}%
\pgfpathlineto{\pgfqpoint{1.582883in}{1.155028in}}%
\pgfpathlineto{\pgfqpoint{1.583195in}{1.141061in}}%
\pgfpathlineto{\pgfqpoint{1.584028in}{0.896565in}}%
\pgfpathlineto{\pgfqpoint{1.583612in}{1.355828in}}%
\pgfpathlineto{\pgfqpoint{1.584237in}{1.049799in}}%
\pgfpathlineto{\pgfqpoint{1.584862in}{0.873855in}}%
\pgfpathlineto{\pgfqpoint{1.585382in}{1.332629in}}%
\pgfpathlineto{\pgfqpoint{1.585486in}{0.854322in}}%
\pgfpathlineto{\pgfqpoint{1.586528in}{1.004898in}}%
\pgfpathlineto{\pgfqpoint{1.587673in}{1.291031in}}%
\pgfpathlineto{\pgfqpoint{1.586736in}{0.889801in}}%
\pgfpathlineto{\pgfqpoint{1.587778in}{1.193188in}}%
\pgfpathlineto{\pgfqpoint{1.587882in}{1.191431in}}%
\pgfpathlineto{\pgfqpoint{1.588402in}{0.869385in}}%
\pgfpathlineto{\pgfqpoint{1.588611in}{1.395262in}}%
\pgfpathlineto{\pgfqpoint{1.588923in}{1.086235in}}%
\pgfpathlineto{\pgfqpoint{1.589548in}{1.247435in}}%
\pgfpathlineto{\pgfqpoint{1.589131in}{0.948749in}}%
\pgfpathlineto{\pgfqpoint{1.589860in}{1.068190in}}%
\pgfpathlineto{\pgfqpoint{1.590173in}{0.801733in}}%
\pgfpathlineto{\pgfqpoint{1.590277in}{1.207923in}}%
\pgfpathlineto{\pgfqpoint{1.591006in}{0.950557in}}%
\pgfpathlineto{\pgfqpoint{1.591110in}{0.929923in}}%
\pgfpathlineto{\pgfqpoint{1.592256in}{1.351141in}}%
\pgfpathlineto{\pgfqpoint{1.591943in}{0.773503in}}%
\pgfpathlineto{\pgfqpoint{1.592360in}{1.323092in}}%
\pgfpathlineto{\pgfqpoint{1.592672in}{0.710079in}}%
\pgfpathlineto{\pgfqpoint{1.593505in}{1.091491in}}%
\pgfpathlineto{\pgfqpoint{1.594547in}{1.287548in}}%
\pgfpathlineto{\pgfqpoint{1.594130in}{0.896935in}}%
\pgfpathlineto{\pgfqpoint{1.594651in}{1.127765in}}%
\pgfpathlineto{\pgfqpoint{1.595172in}{0.920150in}}%
\pgfpathlineto{\pgfqpoint{1.595588in}{1.276705in}}%
\pgfpathlineto{\pgfqpoint{1.595692in}{0.974607in}}%
\pgfpathlineto{\pgfqpoint{1.595796in}{1.300049in}}%
\pgfpathlineto{\pgfqpoint{1.596109in}{0.795708in}}%
\pgfpathlineto{\pgfqpoint{1.596734in}{0.996496in}}%
\pgfpathlineto{\pgfqpoint{1.596838in}{0.999619in}}%
\pgfpathlineto{\pgfqpoint{1.597254in}{1.264708in}}%
\pgfpathlineto{\pgfqpoint{1.597879in}{0.760699in}}%
\pgfpathlineto{\pgfqpoint{1.598712in}{1.229984in}}%
\pgfpathlineto{\pgfqpoint{1.599025in}{0.939376in}}%
\pgfpathlineto{\pgfqpoint{1.599129in}{0.929690in}}%
\pgfpathlineto{\pgfqpoint{1.600274in}{1.317709in}}%
\pgfpathlineto{\pgfqpoint{1.599545in}{0.860936in}}%
\pgfpathlineto{\pgfqpoint{1.600379in}{1.281041in}}%
\pgfpathlineto{\pgfqpoint{1.601212in}{0.803835in}}%
\pgfpathlineto{\pgfqpoint{1.601524in}{1.193043in}}%
\pgfpathlineto{\pgfqpoint{1.601837in}{0.763860in}}%
\pgfpathlineto{\pgfqpoint{1.602357in}{1.345781in}}%
\pgfpathlineto{\pgfqpoint{1.602878in}{0.932530in}}%
\pgfpathlineto{\pgfqpoint{1.603607in}{1.147699in}}%
\pgfpathlineto{\pgfqpoint{1.603503in}{0.793955in}}%
\pgfpathlineto{\pgfqpoint{1.604024in}{1.117839in}}%
\pgfpathlineto{\pgfqpoint{1.604232in}{0.807161in}}%
\pgfpathlineto{\pgfqpoint{1.604857in}{1.315823in}}%
\pgfpathlineto{\pgfqpoint{1.605169in}{1.013589in}}%
\pgfpathlineto{\pgfqpoint{1.605794in}{1.199245in}}%
\pgfpathlineto{\pgfqpoint{1.605690in}{0.890212in}}%
\pgfpathlineto{\pgfqpoint{1.606210in}{0.957709in}}%
\pgfpathlineto{\pgfqpoint{1.606731in}{0.862134in}}%
\pgfpathlineto{\pgfqpoint{1.606627in}{1.325660in}}%
\pgfpathlineto{\pgfqpoint{1.606835in}{1.094359in}}%
\pgfpathlineto{\pgfqpoint{1.606939in}{1.284274in}}%
\pgfpathlineto{\pgfqpoint{1.607356in}{0.851248in}}%
\pgfpathlineto{\pgfqpoint{1.607773in}{0.886891in}}%
\pgfpathlineto{\pgfqpoint{1.607877in}{0.693171in}}%
\pgfpathlineto{\pgfqpoint{1.608189in}{1.289013in}}%
\pgfpathlineto{\pgfqpoint{1.608606in}{0.968528in}}%
\pgfpathlineto{\pgfqpoint{1.608710in}{1.260600in}}%
\pgfpathlineto{\pgfqpoint{1.608918in}{0.745389in}}%
\pgfpathlineto{\pgfqpoint{1.609647in}{1.247063in}}%
\pgfpathlineto{\pgfqpoint{1.609855in}{1.327941in}}%
\pgfpathlineto{\pgfqpoint{1.610689in}{0.889262in}}%
\pgfpathlineto{\pgfqpoint{1.610793in}{1.421779in}}%
\pgfpathlineto{\pgfqpoint{1.611522in}{0.654359in}}%
\pgfpathlineto{\pgfqpoint{1.611834in}{1.101803in}}%
\pgfpathlineto{\pgfqpoint{1.612563in}{0.784956in}}%
\pgfpathlineto{\pgfqpoint{1.612667in}{1.203665in}}%
\pgfpathlineto{\pgfqpoint{1.612876in}{1.027930in}}%
\pgfpathlineto{\pgfqpoint{1.613396in}{1.218347in}}%
\pgfpathlineto{\pgfqpoint{1.613709in}{0.799741in}}%
\pgfpathlineto{\pgfqpoint{1.613917in}{1.059194in}}%
\pgfpathlineto{\pgfqpoint{1.614854in}{1.232174in}}%
\pgfpathlineto{\pgfqpoint{1.614958in}{0.834885in}}%
\pgfpathlineto{\pgfqpoint{1.616000in}{1.314095in}}%
\pgfpathlineto{\pgfqpoint{1.615375in}{0.690436in}}%
\pgfpathlineto{\pgfqpoint{1.616208in}{1.148163in}}%
\pgfpathlineto{\pgfqpoint{1.617041in}{0.845724in}}%
\pgfpathlineto{\pgfqpoint{1.616833in}{1.275766in}}%
\pgfpathlineto{\pgfqpoint{1.617249in}{1.162369in}}%
\pgfpathlineto{\pgfqpoint{1.617458in}{0.961763in}}%
\pgfpathlineto{\pgfqpoint{1.617562in}{1.482128in}}%
\pgfpathlineto{\pgfqpoint{1.617666in}{0.866622in}}%
\pgfpathlineto{\pgfqpoint{1.618603in}{1.137238in}}%
\pgfpathlineto{\pgfqpoint{1.618916in}{1.243832in}}%
\pgfpathlineto{\pgfqpoint{1.619541in}{1.054081in}}%
\pgfpathlineto{\pgfqpoint{1.619749in}{1.276823in}}%
\pgfpathlineto{\pgfqpoint{1.619853in}{0.832776in}}%
\pgfpathlineto{\pgfqpoint{1.620478in}{1.037704in}}%
\pgfpathlineto{\pgfqpoint{1.620894in}{0.651720in}}%
\pgfpathlineto{\pgfqpoint{1.621207in}{1.247460in}}%
\pgfpathlineto{\pgfqpoint{1.621519in}{1.117372in}}%
\pgfpathlineto{\pgfqpoint{1.621623in}{1.241362in}}%
\pgfpathlineto{\pgfqpoint{1.622352in}{0.878304in}}%
\pgfpathlineto{\pgfqpoint{1.622456in}{0.922907in}}%
\pgfpathlineto{\pgfqpoint{1.622665in}{0.835317in}}%
\pgfpathlineto{\pgfqpoint{1.622769in}{1.075672in}}%
\pgfpathlineto{\pgfqpoint{1.623290in}{1.301728in}}%
\pgfpathlineto{\pgfqpoint{1.623081in}{0.833041in}}%
\pgfpathlineto{\pgfqpoint{1.623810in}{0.977829in}}%
\pgfpathlineto{\pgfqpoint{1.624435in}{1.200202in}}%
\pgfpathlineto{\pgfqpoint{1.624852in}{0.855663in}}%
\pgfpathlineto{\pgfqpoint{1.625477in}{1.185597in}}%
\pgfpathlineto{\pgfqpoint{1.625997in}{0.991521in}}%
\pgfpathlineto{\pgfqpoint{1.626622in}{1.343296in}}%
\pgfpathlineto{\pgfqpoint{1.626830in}{0.942471in}}%
\pgfpathlineto{\pgfqpoint{1.627351in}{1.228878in}}%
\pgfpathlineto{\pgfqpoint{1.628601in}{0.849748in}}%
\pgfpathlineto{\pgfqpoint{1.628705in}{0.924978in}}%
\pgfpathlineto{\pgfqpoint{1.629538in}{1.334299in}}%
\pgfpathlineto{\pgfqpoint{1.629122in}{0.708219in}}%
\pgfpathlineto{\pgfqpoint{1.629746in}{1.331381in}}%
\pgfpathlineto{\pgfqpoint{1.630059in}{0.867098in}}%
\pgfpathlineto{\pgfqpoint{1.630892in}{1.173259in}}%
\pgfpathlineto{\pgfqpoint{1.631308in}{0.809108in}}%
\pgfpathlineto{\pgfqpoint{1.631517in}{1.356758in}}%
\pgfpathlineto{\pgfqpoint{1.632142in}{0.861171in}}%
\pgfpathlineto{\pgfqpoint{1.632871in}{1.332539in}}%
\pgfpathlineto{\pgfqpoint{1.633391in}{1.292597in}}%
\pgfpathlineto{\pgfqpoint{1.633808in}{0.760206in}}%
\pgfpathlineto{\pgfqpoint{1.634537in}{1.168891in}}%
\pgfpathlineto{\pgfqpoint{1.635474in}{0.761757in}}%
\pgfpathlineto{\pgfqpoint{1.635266in}{1.431997in}}%
\pgfpathlineto{\pgfqpoint{1.635682in}{1.072159in}}%
\pgfpathlineto{\pgfqpoint{1.635787in}{1.066999in}}%
\pgfpathlineto{\pgfqpoint{1.636307in}{1.149010in}}%
\pgfpathlineto{\pgfqpoint{1.636620in}{0.853055in}}%
\pgfpathlineto{\pgfqpoint{1.636724in}{1.313347in}}%
\pgfpathlineto{\pgfqpoint{1.637765in}{1.032636in}}%
\pgfpathlineto{\pgfqpoint{1.637869in}{1.399153in}}%
\pgfpathlineto{\pgfqpoint{1.638494in}{0.873610in}}%
\pgfpathlineto{\pgfqpoint{1.638807in}{1.139212in}}%
\pgfpathlineto{\pgfqpoint{1.638911in}{0.688326in}}%
\pgfpathlineto{\pgfqpoint{1.639640in}{1.274242in}}%
\pgfpathlineto{\pgfqpoint{1.639848in}{0.779587in}}%
\pgfpathlineto{\pgfqpoint{1.640160in}{1.230159in}}%
\pgfpathlineto{\pgfqpoint{1.640994in}{1.054600in}}%
\pgfpathlineto{\pgfqpoint{1.641618in}{1.241035in}}%
\pgfpathlineto{\pgfqpoint{1.641202in}{0.896935in}}%
\pgfpathlineto{\pgfqpoint{1.642139in}{1.112550in}}%
\pgfpathlineto{\pgfqpoint{1.642972in}{0.870567in}}%
\pgfpathlineto{\pgfqpoint{1.642868in}{1.216662in}}%
\pgfpathlineto{\pgfqpoint{1.643076in}{0.992124in}}%
\pgfpathlineto{\pgfqpoint{1.643910in}{1.291568in}}%
\pgfpathlineto{\pgfqpoint{1.643701in}{0.831438in}}%
\pgfpathlineto{\pgfqpoint{1.644118in}{1.007426in}}%
\pgfpathlineto{\pgfqpoint{1.644430in}{0.968535in}}%
\pgfpathlineto{\pgfqpoint{1.644534in}{1.125652in}}%
\pgfpathlineto{\pgfqpoint{1.644639in}{1.083147in}}%
\pgfpathlineto{\pgfqpoint{1.644847in}{0.899832in}}%
\pgfpathlineto{\pgfqpoint{1.645472in}{1.228404in}}%
\pgfpathlineto{\pgfqpoint{1.646201in}{0.868157in}}%
\pgfpathlineto{\pgfqpoint{1.645888in}{1.440755in}}%
\pgfpathlineto{\pgfqpoint{1.646617in}{1.014828in}}%
\pgfpathlineto{\pgfqpoint{1.647034in}{1.233893in}}%
\pgfpathlineto{\pgfqpoint{1.647554in}{0.914192in}}%
\pgfpathlineto{\pgfqpoint{1.648075in}{1.333294in}}%
\pgfpathlineto{\pgfqpoint{1.648492in}{0.857724in}}%
\pgfpathlineto{\pgfqpoint{1.648700in}{1.137774in}}%
\pgfpathlineto{\pgfqpoint{1.649012in}{1.195478in}}%
\pgfpathlineto{\pgfqpoint{1.649950in}{0.835739in}}%
\pgfpathlineto{\pgfqpoint{1.650366in}{1.280524in}}%
\pgfpathlineto{\pgfqpoint{1.650991in}{0.996651in}}%
\pgfpathlineto{\pgfqpoint{1.651095in}{0.777709in}}%
\pgfpathlineto{\pgfqpoint{1.651928in}{1.196055in}}%
\pgfpathlineto{\pgfqpoint{1.652033in}{0.888465in}}%
\pgfpathlineto{\pgfqpoint{1.652345in}{1.365114in}}%
\pgfpathlineto{\pgfqpoint{1.652657in}{0.776312in}}%
\pgfpathlineto{\pgfqpoint{1.653074in}{0.971478in}}%
\pgfpathlineto{\pgfqpoint{1.653907in}{0.815825in}}%
\pgfpathlineto{\pgfqpoint{1.653491in}{1.280441in}}%
\pgfpathlineto{\pgfqpoint{1.654011in}{0.875965in}}%
\pgfpathlineto{\pgfqpoint{1.654428in}{1.328577in}}%
\pgfpathlineto{\pgfqpoint{1.654844in}{0.641410in}}%
\pgfpathlineto{\pgfqpoint{1.655157in}{1.106203in}}%
\pgfpathlineto{\pgfqpoint{1.655886in}{0.839449in}}%
\pgfpathlineto{\pgfqpoint{1.655990in}{1.202435in}}%
\pgfpathlineto{\pgfqpoint{1.656094in}{0.831745in}}%
\pgfpathlineto{\pgfqpoint{1.657031in}{1.391937in}}%
\pgfpathlineto{\pgfqpoint{1.657135in}{0.842499in}}%
\pgfpathlineto{\pgfqpoint{1.657552in}{1.293447in}}%
\pgfpathlineto{\pgfqpoint{1.657656in}{0.782762in}}%
\pgfpathlineto{\pgfqpoint{1.658281in}{1.026791in}}%
\pgfpathlineto{\pgfqpoint{1.658489in}{1.111246in}}%
\pgfpathlineto{\pgfqpoint{1.658698in}{0.894297in}}%
\pgfpathlineto{\pgfqpoint{1.658802in}{1.465572in}}%
\pgfpathlineto{\pgfqpoint{1.659010in}{0.728499in}}%
\pgfpathlineto{\pgfqpoint{1.659739in}{1.013146in}}%
\pgfpathlineto{\pgfqpoint{1.659843in}{0.999697in}}%
\pgfpathlineto{\pgfqpoint{1.659947in}{1.042247in}}%
\pgfpathlineto{\pgfqpoint{1.660780in}{1.228755in}}%
\pgfpathlineto{\pgfqpoint{1.660885in}{0.956504in}}%
\pgfpathlineto{\pgfqpoint{1.661197in}{1.081033in}}%
\pgfpathlineto{\pgfqpoint{1.661301in}{0.781937in}}%
\pgfpathlineto{\pgfqpoint{1.662030in}{1.257294in}}%
\pgfpathlineto{\pgfqpoint{1.662447in}{1.109044in}}%
\pgfpathlineto{\pgfqpoint{1.662759in}{0.776654in}}%
\pgfpathlineto{\pgfqpoint{1.663176in}{1.252833in}}%
\pgfpathlineto{\pgfqpoint{1.663384in}{1.012237in}}%
\pgfpathlineto{\pgfqpoint{1.663800in}{1.320058in}}%
\pgfpathlineto{\pgfqpoint{1.664217in}{0.829686in}}%
\pgfpathlineto{\pgfqpoint{1.664425in}{0.890393in}}%
\pgfpathlineto{\pgfqpoint{1.665571in}{1.342532in}}%
\pgfpathlineto{\pgfqpoint{1.664634in}{0.879301in}}%
\pgfpathlineto{\pgfqpoint{1.665779in}{1.000882in}}%
\pgfpathlineto{\pgfqpoint{1.666300in}{0.809597in}}%
\pgfpathlineto{\pgfqpoint{1.666508in}{1.267780in}}%
\pgfpathlineto{\pgfqpoint{1.666716in}{0.923961in}}%
\pgfpathlineto{\pgfqpoint{1.667029in}{1.260404in}}%
\pgfpathlineto{\pgfqpoint{1.667341in}{0.802297in}}%
\pgfpathlineto{\pgfqpoint{1.667862in}{1.063831in}}%
\pgfpathlineto{\pgfqpoint{1.668487in}{0.837238in}}%
\pgfpathlineto{\pgfqpoint{1.668695in}{1.216108in}}%
\pgfpathlineto{\pgfqpoint{1.668903in}{0.976370in}}%
\pgfpathlineto{\pgfqpoint{1.669737in}{0.813582in}}%
\pgfpathlineto{\pgfqpoint{1.669945in}{1.232077in}}%
\pgfpathlineto{\pgfqpoint{1.670882in}{1.369723in}}%
\pgfpathlineto{\pgfqpoint{1.671090in}{0.748072in}}%
\pgfpathlineto{\pgfqpoint{1.672340in}{1.364922in}}%
\pgfpathlineto{\pgfqpoint{1.673381in}{0.763631in}}%
\pgfpathlineto{\pgfqpoint{1.673486in}{1.058016in}}%
\pgfpathlineto{\pgfqpoint{1.673590in}{1.064639in}}%
\pgfpathlineto{\pgfqpoint{1.673694in}{1.148006in}}%
\pgfpathlineto{\pgfqpoint{1.674110in}{0.717678in}}%
\pgfpathlineto{\pgfqpoint{1.674527in}{0.978079in}}%
\pgfpathlineto{\pgfqpoint{1.674631in}{0.975126in}}%
\pgfpathlineto{\pgfqpoint{1.674839in}{0.579445in}}%
\pgfpathlineto{\pgfqpoint{1.674944in}{1.275676in}}%
\pgfpathlineto{\pgfqpoint{1.675673in}{0.996569in}}%
\pgfpathlineto{\pgfqpoint{1.675777in}{1.300623in}}%
\pgfpathlineto{\pgfqpoint{1.676610in}{0.723963in}}%
\pgfpathlineto{\pgfqpoint{1.676714in}{0.888942in}}%
\pgfpathlineto{\pgfqpoint{1.677755in}{1.289275in}}%
\pgfpathlineto{\pgfqpoint{1.677964in}{1.191173in}}%
\pgfpathlineto{\pgfqpoint{1.678068in}{0.610874in}}%
\pgfpathlineto{\pgfqpoint{1.678380in}{1.348826in}}%
\pgfpathlineto{\pgfqpoint{1.679109in}{0.823277in}}%
\pgfpathlineto{\pgfqpoint{1.679942in}{1.286254in}}%
\pgfpathlineto{\pgfqpoint{1.679630in}{0.814637in}}%
\pgfpathlineto{\pgfqpoint{1.680255in}{1.042302in}}%
\pgfpathlineto{\pgfqpoint{1.681192in}{0.774590in}}%
\pgfpathlineto{\pgfqpoint{1.680984in}{1.179442in}}%
\pgfpathlineto{\pgfqpoint{1.681296in}{0.983256in}}%
\pgfpathlineto{\pgfqpoint{1.681504in}{1.233748in}}%
\pgfpathlineto{\pgfqpoint{1.682129in}{0.957121in}}%
\pgfpathlineto{\pgfqpoint{1.682338in}{1.052423in}}%
\pgfpathlineto{\pgfqpoint{1.682650in}{0.949433in}}%
\pgfpathlineto{\pgfqpoint{1.682546in}{1.128776in}}%
\pgfpathlineto{\pgfqpoint{1.683379in}{1.083846in}}%
\pgfpathlineto{\pgfqpoint{1.683900in}{0.823072in}}%
\pgfpathlineto{\pgfqpoint{1.683587in}{1.304615in}}%
\pgfpathlineto{\pgfqpoint{1.684629in}{0.966746in}}%
\pgfpathlineto{\pgfqpoint{1.685149in}{1.295802in}}%
\pgfpathlineto{\pgfqpoint{1.685358in}{0.952945in}}%
\pgfpathlineto{\pgfqpoint{1.685670in}{0.966864in}}%
\pgfpathlineto{\pgfqpoint{1.685774in}{0.866245in}}%
\pgfpathlineto{\pgfqpoint{1.686399in}{1.240692in}}%
\pgfpathlineto{\pgfqpoint{1.686503in}{1.244351in}}%
\pgfpathlineto{\pgfqpoint{1.687545in}{0.860989in}}%
\pgfpathlineto{\pgfqpoint{1.688378in}{1.270301in}}%
\pgfpathlineto{\pgfqpoint{1.687961in}{0.857826in}}%
\pgfpathlineto{\pgfqpoint{1.688586in}{0.937469in}}%
\pgfpathlineto{\pgfqpoint{1.688690in}{0.949708in}}%
\pgfpathlineto{\pgfqpoint{1.689315in}{1.262570in}}%
\pgfpathlineto{\pgfqpoint{1.689419in}{0.767859in}}%
\pgfpathlineto{\pgfqpoint{1.689732in}{0.857569in}}%
\pgfpathlineto{\pgfqpoint{1.690773in}{1.247408in}}%
\pgfpathlineto{\pgfqpoint{1.691085in}{1.123683in}}%
\pgfpathlineto{\pgfqpoint{1.691294in}{1.131209in}}%
\pgfpathlineto{\pgfqpoint{1.692231in}{0.831768in}}%
\pgfpathlineto{\pgfqpoint{1.692543in}{0.803749in}}%
\pgfpathlineto{\pgfqpoint{1.693377in}{1.134961in}}%
\pgfpathlineto{\pgfqpoint{1.694210in}{0.854199in}}%
\pgfpathlineto{\pgfqpoint{1.694314in}{1.302698in}}%
\pgfpathlineto{\pgfqpoint{1.694522in}{1.059365in}}%
\pgfpathlineto{\pgfqpoint{1.695355in}{0.882939in}}%
\pgfpathlineto{\pgfqpoint{1.695251in}{1.325971in}}%
\pgfpathlineto{\pgfqpoint{1.695668in}{0.896986in}}%
\pgfpathlineto{\pgfqpoint{1.695980in}{0.688145in}}%
\pgfpathlineto{\pgfqpoint{1.696709in}{1.231892in}}%
\pgfpathlineto{\pgfqpoint{1.697126in}{1.382119in}}%
\pgfpathlineto{\pgfqpoint{1.697855in}{0.737901in}}%
\pgfpathlineto{\pgfqpoint{1.698896in}{1.147461in}}%
\pgfpathlineto{\pgfqpoint{1.698271in}{0.617519in}}%
\pgfpathlineto{\pgfqpoint{1.699000in}{0.903459in}}%
\pgfpathlineto{\pgfqpoint{1.699521in}{1.323028in}}%
\pgfpathlineto{\pgfqpoint{1.700146in}{1.214138in}}%
\pgfpathlineto{\pgfqpoint{1.700354in}{0.710161in}}%
\pgfpathlineto{\pgfqpoint{1.700979in}{1.298724in}}%
\pgfpathlineto{\pgfqpoint{1.701395in}{1.039874in}}%
\pgfpathlineto{\pgfqpoint{1.702124in}{1.227353in}}%
\pgfpathlineto{\pgfqpoint{1.701812in}{0.861422in}}%
\pgfpathlineto{\pgfqpoint{1.702333in}{1.012346in}}%
\pgfpathlineto{\pgfqpoint{1.702437in}{0.881146in}}%
\pgfpathlineto{\pgfqpoint{1.702645in}{1.217292in}}%
\pgfpathlineto{\pgfqpoint{1.703374in}{1.042629in}}%
\pgfpathlineto{\pgfqpoint{1.703895in}{1.192454in}}%
\pgfpathlineto{\pgfqpoint{1.703791in}{0.940107in}}%
\pgfpathlineto{\pgfqpoint{1.704207in}{0.944405in}}%
\pgfpathlineto{\pgfqpoint{1.704624in}{0.815687in}}%
\pgfpathlineto{\pgfqpoint{1.704415in}{1.246254in}}%
\pgfpathlineto{\pgfqpoint{1.704936in}{1.072436in}}%
\pgfpathlineto{\pgfqpoint{1.705040in}{1.288864in}}%
\pgfpathlineto{\pgfqpoint{1.705769in}{0.809262in}}%
\pgfpathlineto{\pgfqpoint{1.705873in}{1.199642in}}%
\pgfpathlineto{\pgfqpoint{1.706707in}{0.887775in}}%
\pgfpathlineto{\pgfqpoint{1.706915in}{1.212045in}}%
\pgfpathlineto{\pgfqpoint{1.707019in}{1.053544in}}%
\pgfpathlineto{\pgfqpoint{1.707540in}{1.373025in}}%
\pgfpathlineto{\pgfqpoint{1.707748in}{0.808566in}}%
\pgfpathlineto{\pgfqpoint{1.708060in}{1.175541in}}%
\pgfpathlineto{\pgfqpoint{1.708894in}{0.791441in}}%
\pgfpathlineto{\pgfqpoint{1.709102in}{1.240830in}}%
\pgfpathlineto{\pgfqpoint{1.709206in}{1.006137in}}%
\pgfpathlineto{\pgfqpoint{1.709310in}{1.010414in}}%
\pgfpathlineto{\pgfqpoint{1.709518in}{1.270708in}}%
\pgfpathlineto{\pgfqpoint{1.709831in}{0.816435in}}%
\pgfpathlineto{\pgfqpoint{1.710352in}{0.972970in}}%
\pgfpathlineto{\pgfqpoint{1.710560in}{0.881315in}}%
\pgfpathlineto{\pgfqpoint{1.710872in}{1.308309in}}%
\pgfpathlineto{\pgfqpoint{1.711081in}{0.991023in}}%
\pgfpathlineto{\pgfqpoint{1.711705in}{1.289837in}}%
\pgfpathlineto{\pgfqpoint{1.711601in}{0.621397in}}%
\pgfpathlineto{\pgfqpoint{1.712122in}{1.078171in}}%
\pgfpathlineto{\pgfqpoint{1.712434in}{0.862214in}}%
\pgfpathlineto{\pgfqpoint{1.712330in}{1.144719in}}%
\pgfpathlineto{\pgfqpoint{1.713163in}{1.088934in}}%
\pgfpathlineto{\pgfqpoint{1.713996in}{1.196053in}}%
\pgfpathlineto{\pgfqpoint{1.713372in}{0.778833in}}%
\pgfpathlineto{\pgfqpoint{1.714101in}{1.058997in}}%
\pgfpathlineto{\pgfqpoint{1.714205in}{0.997957in}}%
\pgfpathlineto{\pgfqpoint{1.714725in}{1.214679in}}%
\pgfpathlineto{\pgfqpoint{1.715038in}{1.062190in}}%
\pgfpathlineto{\pgfqpoint{1.715350in}{1.209956in}}%
\pgfpathlineto{\pgfqpoint{1.715663in}{0.894689in}}%
\pgfpathlineto{\pgfqpoint{1.715871in}{1.197523in}}%
\pgfpathlineto{\pgfqpoint{1.715975in}{0.822678in}}%
\pgfpathlineto{\pgfqpoint{1.716288in}{1.224913in}}%
\pgfpathlineto{\pgfqpoint{1.717017in}{0.883017in}}%
\pgfpathlineto{\pgfqpoint{1.717121in}{1.270147in}}%
\pgfpathlineto{\pgfqpoint{1.717225in}{0.859543in}}%
\pgfpathlineto{\pgfqpoint{1.718058in}{1.043305in}}%
\pgfpathlineto{\pgfqpoint{1.718370in}{0.690331in}}%
\pgfpathlineto{\pgfqpoint{1.718579in}{1.217593in}}%
\pgfpathlineto{\pgfqpoint{1.718891in}{1.150443in}}%
\pgfpathlineto{\pgfqpoint{1.719412in}{1.400696in}}%
\pgfpathlineto{\pgfqpoint{1.719308in}{0.768890in}}%
\pgfpathlineto{\pgfqpoint{1.719933in}{1.038312in}}%
\pgfpathlineto{\pgfqpoint{1.720037in}{1.037578in}}%
\pgfpathlineto{\pgfqpoint{1.720870in}{1.191250in}}%
\pgfpathlineto{\pgfqpoint{1.720349in}{0.880025in}}%
\pgfpathlineto{\pgfqpoint{1.721078in}{1.080409in}}%
\pgfpathlineto{\pgfqpoint{1.721182in}{0.558443in}}%
\pgfpathlineto{\pgfqpoint{1.721807in}{1.201522in}}%
\pgfpathlineto{\pgfqpoint{1.722119in}{0.800664in}}%
\pgfpathlineto{\pgfqpoint{1.722953in}{1.315179in}}%
\pgfpathlineto{\pgfqpoint{1.722848in}{0.773255in}}%
\pgfpathlineto{\pgfqpoint{1.723265in}{1.187345in}}%
\pgfpathlineto{\pgfqpoint{1.723369in}{0.840958in}}%
\pgfpathlineto{\pgfqpoint{1.724202in}{1.300485in}}%
\pgfpathlineto{\pgfqpoint{1.724411in}{1.042208in}}%
\pgfpathlineto{\pgfqpoint{1.724827in}{1.217469in}}%
\pgfpathlineto{\pgfqpoint{1.725348in}{0.918611in}}%
\pgfpathlineto{\pgfqpoint{1.725452in}{1.014593in}}%
\pgfpathlineto{\pgfqpoint{1.725660in}{0.810156in}}%
\pgfpathlineto{\pgfqpoint{1.725764in}{1.180516in}}%
\pgfpathlineto{\pgfqpoint{1.725869in}{0.647769in}}%
\pgfpathlineto{\pgfqpoint{1.725973in}{1.452860in}}%
\pgfpathlineto{\pgfqpoint{1.726806in}{1.040326in}}%
\pgfpathlineto{\pgfqpoint{1.727014in}{0.861270in}}%
\pgfpathlineto{\pgfqpoint{1.727847in}{1.244688in}}%
\pgfpathlineto{\pgfqpoint{1.729097in}{0.653338in}}%
\pgfpathlineto{\pgfqpoint{1.730034in}{1.254809in}}%
\pgfpathlineto{\pgfqpoint{1.730347in}{1.214915in}}%
\pgfpathlineto{\pgfqpoint{1.730555in}{1.328813in}}%
\pgfpathlineto{\pgfqpoint{1.731492in}{0.910193in}}%
\pgfpathlineto{\pgfqpoint{1.731596in}{1.367757in}}%
\pgfpathlineto{\pgfqpoint{1.731700in}{1.004746in}}%
\pgfpathlineto{\pgfqpoint{1.731805in}{1.008907in}}%
\pgfpathlineto{\pgfqpoint{1.731909in}{1.003248in}}%
\pgfpathlineto{\pgfqpoint{1.732325in}{0.853594in}}%
\pgfpathlineto{\pgfqpoint{1.732429in}{0.938841in}}%
\pgfpathlineto{\pgfqpoint{1.732534in}{1.340389in}}%
\pgfpathlineto{\pgfqpoint{1.733054in}{0.794025in}}%
\pgfpathlineto{\pgfqpoint{1.733471in}{1.010030in}}%
\pgfpathlineto{\pgfqpoint{1.733992in}{0.734676in}}%
\pgfpathlineto{\pgfqpoint{1.733679in}{1.086377in}}%
\pgfpathlineto{\pgfqpoint{1.734616in}{0.888077in}}%
\pgfpathlineto{\pgfqpoint{1.735345in}{1.187338in}}%
\pgfpathlineto{\pgfqpoint{1.735033in}{0.861981in}}%
\pgfpathlineto{\pgfqpoint{1.735866in}{1.155605in}}%
\pgfpathlineto{\pgfqpoint{1.736595in}{0.937157in}}%
\pgfpathlineto{\pgfqpoint{1.736074in}{1.289685in}}%
\pgfpathlineto{\pgfqpoint{1.737012in}{1.128717in}}%
\pgfpathlineto{\pgfqpoint{1.737636in}{1.550219in}}%
\pgfpathlineto{\pgfqpoint{1.737220in}{0.844903in}}%
\pgfpathlineto{\pgfqpoint{1.737845in}{1.042667in}}%
\pgfpathlineto{\pgfqpoint{1.737949in}{0.978470in}}%
\pgfpathlineto{\pgfqpoint{1.738574in}{1.251588in}}%
\pgfpathlineto{\pgfqpoint{1.738678in}{1.257495in}}%
\pgfpathlineto{\pgfqpoint{1.738886in}{0.857197in}}%
\pgfpathlineto{\pgfqpoint{1.739094in}{1.282428in}}%
\pgfpathlineto{\pgfqpoint{1.739823in}{1.097092in}}%
\pgfpathlineto{\pgfqpoint{1.740657in}{1.306535in}}%
\pgfpathlineto{\pgfqpoint{1.740761in}{0.864772in}}%
\pgfpathlineto{\pgfqpoint{1.740865in}{1.238290in}}%
\pgfpathlineto{\pgfqpoint{1.741073in}{0.790418in}}%
\pgfpathlineto{\pgfqpoint{1.742010in}{1.124492in}}%
\pgfpathlineto{\pgfqpoint{1.742219in}{0.844205in}}%
\pgfpathlineto{\pgfqpoint{1.742635in}{1.179143in}}%
\pgfpathlineto{\pgfqpoint{1.743052in}{0.851412in}}%
\pgfpathlineto{\pgfqpoint{1.743260in}{1.131295in}}%
\pgfpathlineto{\pgfqpoint{1.743364in}{0.703398in}}%
\pgfpathlineto{\pgfqpoint{1.744197in}{0.948963in}}%
\pgfpathlineto{\pgfqpoint{1.744302in}{0.960331in}}%
\pgfpathlineto{\pgfqpoint{1.744406in}{0.895526in}}%
\pgfpathlineto{\pgfqpoint{1.745030in}{1.278889in}}%
\pgfpathlineto{\pgfqpoint{1.745239in}{0.756553in}}%
\pgfpathlineto{\pgfqpoint{1.745551in}{1.059221in}}%
\pgfpathlineto{\pgfqpoint{1.746072in}{1.177770in}}%
\pgfpathlineto{\pgfqpoint{1.746801in}{0.566441in}}%
\pgfpathlineto{\pgfqpoint{1.747946in}{1.241599in}}%
\pgfpathlineto{\pgfqpoint{1.748051in}{1.108685in}}%
\pgfpathlineto{\pgfqpoint{1.748675in}{0.850720in}}%
\pgfpathlineto{\pgfqpoint{1.748988in}{1.175409in}}%
\pgfpathlineto{\pgfqpoint{1.749092in}{1.195314in}}%
\pgfpathlineto{\pgfqpoint{1.749509in}{0.787959in}}%
\pgfpathlineto{\pgfqpoint{1.749717in}{1.412686in}}%
\pgfpathlineto{\pgfqpoint{1.750238in}{0.998105in}}%
\pgfpathlineto{\pgfqpoint{1.750758in}{1.537001in}}%
\pgfpathlineto{\pgfqpoint{1.750654in}{0.832417in}}%
\pgfpathlineto{\pgfqpoint{1.751383in}{1.243168in}}%
\pgfpathlineto{\pgfqpoint{1.752216in}{0.866379in}}%
\pgfpathlineto{\pgfqpoint{1.751904in}{1.263280in}}%
\pgfpathlineto{\pgfqpoint{1.752529in}{1.103562in}}%
\pgfpathlineto{\pgfqpoint{1.753049in}{0.704245in}}%
\pgfpathlineto{\pgfqpoint{1.753362in}{1.231501in}}%
\pgfpathlineto{\pgfqpoint{1.753570in}{1.080673in}}%
\pgfpathlineto{\pgfqpoint{1.753674in}{1.087440in}}%
\pgfpathlineto{\pgfqpoint{1.754195in}{0.956682in}}%
\pgfpathlineto{\pgfqpoint{1.754924in}{1.327746in}}%
\pgfpathlineto{\pgfqpoint{1.755861in}{0.849374in}}%
\pgfpathlineto{\pgfqpoint{1.755965in}{1.048291in}}%
\pgfpathlineto{\pgfqpoint{1.756069in}{1.279304in}}%
\pgfpathlineto{\pgfqpoint{1.756694in}{0.627013in}}%
\pgfpathlineto{\pgfqpoint{1.757007in}{0.933789in}}%
\pgfpathlineto{\pgfqpoint{1.757944in}{0.823535in}}%
\pgfpathlineto{\pgfqpoint{1.758256in}{1.282850in}}%
\pgfpathlineto{\pgfqpoint{1.759402in}{0.836238in}}%
\pgfpathlineto{\pgfqpoint{1.759506in}{1.267396in}}%
\pgfpathlineto{\pgfqpoint{1.760443in}{0.758426in}}%
\pgfpathlineto{\pgfqpoint{1.760548in}{1.071548in}}%
\pgfpathlineto{\pgfqpoint{1.760964in}{0.880152in}}%
\pgfpathlineto{\pgfqpoint{1.761068in}{1.289866in}}%
\pgfpathlineto{\pgfqpoint{1.761693in}{0.954887in}}%
\pgfpathlineto{\pgfqpoint{1.762630in}{1.232749in}}%
\pgfpathlineto{\pgfqpoint{1.762734in}{0.948533in}}%
\pgfpathlineto{\pgfqpoint{1.763359in}{1.213255in}}%
\pgfpathlineto{\pgfqpoint{1.763776in}{0.912049in}}%
\pgfpathlineto{\pgfqpoint{1.763880in}{1.137071in}}%
\pgfpathlineto{\pgfqpoint{1.764609in}{1.424745in}}%
\pgfpathlineto{\pgfqpoint{1.764921in}{0.781614in}}%
\pgfpathlineto{\pgfqpoint{1.766067in}{1.436363in}}%
\pgfpathlineto{\pgfqpoint{1.767317in}{0.911544in}}%
\pgfpathlineto{\pgfqpoint{1.767837in}{1.422533in}}%
\pgfpathlineto{\pgfqpoint{1.767525in}{0.786160in}}%
\pgfpathlineto{\pgfqpoint{1.768462in}{1.197688in}}%
\pgfpathlineto{\pgfqpoint{1.769400in}{0.618482in}}%
\pgfpathlineto{\pgfqpoint{1.769295in}{1.275855in}}%
\pgfpathlineto{\pgfqpoint{1.769504in}{1.129051in}}%
\pgfpathlineto{\pgfqpoint{1.769816in}{1.044644in}}%
\pgfpathlineto{\pgfqpoint{1.770128in}{1.273159in}}%
\pgfpathlineto{\pgfqpoint{1.770545in}{0.742242in}}%
\pgfpathlineto{\pgfqpoint{1.770441in}{1.300681in}}%
\pgfpathlineto{\pgfqpoint{1.771274in}{0.967346in}}%
\pgfpathlineto{\pgfqpoint{1.771378in}{1.281506in}}%
\pgfpathlineto{\pgfqpoint{1.772107in}{0.806954in}}%
\pgfpathlineto{\pgfqpoint{1.772315in}{1.055297in}}%
\pgfpathlineto{\pgfqpoint{1.772628in}{0.751421in}}%
\pgfpathlineto{\pgfqpoint{1.772524in}{1.363317in}}%
\pgfpathlineto{\pgfqpoint{1.773357in}{1.012063in}}%
\pgfpathlineto{\pgfqpoint{1.773461in}{1.340825in}}%
\pgfpathlineto{\pgfqpoint{1.773982in}{0.642976in}}%
\pgfpathlineto{\pgfqpoint{1.774502in}{1.272090in}}%
\pgfpathlineto{\pgfqpoint{1.775544in}{0.810211in}}%
\pgfpathlineto{\pgfqpoint{1.776377in}{1.348167in}}%
\pgfpathlineto{\pgfqpoint{1.776689in}{1.097201in}}%
\pgfpathlineto{\pgfqpoint{1.777314in}{1.310304in}}%
\pgfpathlineto{\pgfqpoint{1.776898in}{0.808723in}}%
\pgfpathlineto{\pgfqpoint{1.777418in}{1.022777in}}%
\pgfpathlineto{\pgfqpoint{1.777731in}{0.863190in}}%
\pgfpathlineto{\pgfqpoint{1.777835in}{1.264746in}}%
\pgfpathlineto{\pgfqpoint{1.778460in}{0.902895in}}%
\pgfpathlineto{\pgfqpoint{1.778564in}{1.413349in}}%
\pgfpathlineto{\pgfqpoint{1.779293in}{0.874192in}}%
\pgfpathlineto{\pgfqpoint{1.779501in}{1.051477in}}%
\pgfpathlineto{\pgfqpoint{1.779605in}{1.036928in}}%
\pgfpathlineto{\pgfqpoint{1.779814in}{1.037125in}}%
\pgfpathlineto{\pgfqpoint{1.780647in}{1.270711in}}%
\pgfpathlineto{\pgfqpoint{1.780230in}{0.608608in}}%
\pgfpathlineto{\pgfqpoint{1.780855in}{1.015726in}}%
\pgfpathlineto{\pgfqpoint{1.780959in}{0.966151in}}%
\pgfpathlineto{\pgfqpoint{1.781272in}{1.307988in}}%
\pgfpathlineto{\pgfqpoint{1.781584in}{1.064187in}}%
\pgfpathlineto{\pgfqpoint{1.781688in}{1.174298in}}%
\pgfpathlineto{\pgfqpoint{1.781792in}{0.798979in}}%
\pgfpathlineto{\pgfqpoint{1.782521in}{0.895148in}}%
\pgfpathlineto{\pgfqpoint{1.782625in}{0.771432in}}%
\pgfpathlineto{\pgfqpoint{1.783042in}{1.245515in}}%
\pgfpathlineto{\pgfqpoint{1.783563in}{0.930473in}}%
\pgfpathlineto{\pgfqpoint{1.784500in}{1.289709in}}%
\pgfpathlineto{\pgfqpoint{1.783875in}{0.784540in}}%
\pgfpathlineto{\pgfqpoint{1.784604in}{1.153942in}}%
\pgfpathlineto{\pgfqpoint{1.785333in}{0.780099in}}%
\pgfpathlineto{\pgfqpoint{1.785541in}{1.308130in}}%
\pgfpathlineto{\pgfqpoint{1.785646in}{0.835195in}}%
\pgfpathlineto{\pgfqpoint{1.786583in}{1.211802in}}%
\pgfpathlineto{\pgfqpoint{1.786791in}{0.935255in}}%
\pgfpathlineto{\pgfqpoint{1.786895in}{0.870419in}}%
\pgfpathlineto{\pgfqpoint{1.787103in}{1.284412in}}%
\pgfpathlineto{\pgfqpoint{1.787624in}{1.071883in}}%
\pgfpathlineto{\pgfqpoint{1.788457in}{1.301003in}}%
\pgfpathlineto{\pgfqpoint{1.787937in}{0.846447in}}%
\pgfpathlineto{\pgfqpoint{1.788561in}{1.053840in}}%
\pgfpathlineto{\pgfqpoint{1.789186in}{0.818651in}}%
\pgfpathlineto{\pgfqpoint{1.789082in}{1.117786in}}%
\pgfpathlineto{\pgfqpoint{1.789603in}{1.104624in}}%
\pgfpathlineto{\pgfqpoint{1.789707in}{1.329560in}}%
\pgfpathlineto{\pgfqpoint{1.790019in}{0.910907in}}%
\pgfpathlineto{\pgfqpoint{1.790540in}{1.205870in}}%
\pgfpathlineto{\pgfqpoint{1.791477in}{0.712666in}}%
\pgfpathlineto{\pgfqpoint{1.791165in}{1.235128in}}%
\pgfpathlineto{\pgfqpoint{1.791582in}{1.019319in}}%
\pgfpathlineto{\pgfqpoint{1.792311in}{1.294593in}}%
\pgfpathlineto{\pgfqpoint{1.792519in}{0.888359in}}%
\pgfpathlineto{\pgfqpoint{1.792623in}{1.044390in}}%
\pgfpathlineto{\pgfqpoint{1.792727in}{0.727504in}}%
\pgfpathlineto{\pgfqpoint{1.793040in}{1.300249in}}%
\pgfpathlineto{\pgfqpoint{1.793664in}{1.025564in}}%
\pgfpathlineto{\pgfqpoint{1.793977in}{0.962591in}}%
\pgfpathlineto{\pgfqpoint{1.793873in}{1.098110in}}%
\pgfpathlineto{\pgfqpoint{1.794497in}{1.060121in}}%
\pgfpathlineto{\pgfqpoint{1.795226in}{1.352460in}}%
\pgfpathlineto{\pgfqpoint{1.795331in}{0.920137in}}%
\pgfpathlineto{\pgfqpoint{1.795539in}{0.997265in}}%
\pgfpathlineto{\pgfqpoint{1.795851in}{1.095179in}}%
\pgfpathlineto{\pgfqpoint{1.796372in}{0.720940in}}%
\pgfpathlineto{\pgfqpoint{1.796684in}{1.433104in}}%
\pgfpathlineto{\pgfqpoint{1.796997in}{0.952890in}}%
\pgfpathlineto{\pgfqpoint{1.797934in}{1.177683in}}%
\pgfpathlineto{\pgfqpoint{1.797309in}{0.861090in}}%
\pgfpathlineto{\pgfqpoint{1.798038in}{1.138523in}}%
\pgfpathlineto{\pgfqpoint{1.798871in}{1.412483in}}%
\pgfpathlineto{\pgfqpoint{1.799184in}{0.931928in}}%
\pgfpathlineto{\pgfqpoint{1.799392in}{1.373921in}}%
\pgfpathlineto{\pgfqpoint{1.800746in}{1.350495in}}%
\pgfpathlineto{\pgfqpoint{1.801683in}{0.614524in}}%
\pgfpathlineto{\pgfqpoint{1.801892in}{0.982071in}}%
\pgfpathlineto{\pgfqpoint{1.801996in}{1.257548in}}%
\pgfpathlineto{\pgfqpoint{1.802620in}{0.877550in}}%
\pgfpathlineto{\pgfqpoint{1.802933in}{1.060994in}}%
\pgfpathlineto{\pgfqpoint{1.803037in}{1.065608in}}%
\pgfpathlineto{\pgfqpoint{1.803141in}{1.280931in}}%
\pgfpathlineto{\pgfqpoint{1.803349in}{0.656545in}}%
\pgfpathlineto{\pgfqpoint{1.804078in}{1.084053in}}%
\pgfpathlineto{\pgfqpoint{1.804703in}{0.683583in}}%
\pgfpathlineto{\pgfqpoint{1.804391in}{1.382685in}}%
\pgfpathlineto{\pgfqpoint{1.805120in}{1.065532in}}%
\pgfpathlineto{\pgfqpoint{1.805328in}{1.428076in}}%
\pgfpathlineto{\pgfqpoint{1.805641in}{0.841066in}}%
\pgfpathlineto{\pgfqpoint{1.806265in}{1.177239in}}%
\pgfpathlineto{\pgfqpoint{1.806682in}{0.778229in}}%
\pgfpathlineto{\pgfqpoint{1.806578in}{1.226700in}}%
\pgfpathlineto{\pgfqpoint{1.807411in}{1.000118in}}%
\pgfpathlineto{\pgfqpoint{1.807619in}{0.964365in}}%
\pgfpathlineto{\pgfqpoint{1.807932in}{1.178525in}}%
\pgfpathlineto{\pgfqpoint{1.808348in}{0.947339in}}%
\pgfpathlineto{\pgfqpoint{1.808765in}{1.131804in}}%
\pgfpathlineto{\pgfqpoint{1.808869in}{1.134441in}}%
\pgfpathlineto{\pgfqpoint{1.809390in}{1.413763in}}%
\pgfpathlineto{\pgfqpoint{1.809494in}{0.848039in}}%
\pgfpathlineto{\pgfqpoint{1.809702in}{1.117507in}}%
\pgfpathlineto{\pgfqpoint{1.810431in}{0.711102in}}%
\pgfpathlineto{\pgfqpoint{1.810535in}{1.308932in}}%
\pgfpathlineto{\pgfqpoint{1.810848in}{1.039278in}}%
\pgfpathlineto{\pgfqpoint{1.810952in}{1.039778in}}%
\pgfpathlineto{\pgfqpoint{1.811160in}{1.299607in}}%
\pgfpathlineto{\pgfqpoint{1.811681in}{0.986188in}}%
\pgfpathlineto{\pgfqpoint{1.811993in}{1.213149in}}%
\pgfpathlineto{\pgfqpoint{1.812410in}{0.852853in}}%
\pgfpathlineto{\pgfqpoint{1.812514in}{1.288396in}}%
\pgfpathlineto{\pgfqpoint{1.813139in}{1.086362in}}%
\pgfpathlineto{\pgfqpoint{1.813764in}{1.271935in}}%
\pgfpathlineto{\pgfqpoint{1.813555in}{1.004676in}}%
\pgfpathlineto{\pgfqpoint{1.814076in}{1.097349in}}%
\pgfpathlineto{\pgfqpoint{1.814180in}{0.833421in}}%
\pgfpathlineto{\pgfqpoint{1.814701in}{1.300365in}}%
\pgfpathlineto{\pgfqpoint{1.815117in}{0.935625in}}%
\pgfpathlineto{\pgfqpoint{1.815430in}{1.283017in}}%
\pgfpathlineto{\pgfqpoint{1.815846in}{0.757511in}}%
\pgfpathlineto{\pgfqpoint{1.816367in}{1.151351in}}%
\pgfpathlineto{\pgfqpoint{1.816992in}{0.814017in}}%
\pgfpathlineto{\pgfqpoint{1.816575in}{1.251710in}}%
\pgfpathlineto{\pgfqpoint{1.817617in}{1.033564in}}%
\pgfpathlineto{\pgfqpoint{1.817721in}{1.350499in}}%
\pgfpathlineto{\pgfqpoint{1.818242in}{0.987179in}}%
\pgfpathlineto{\pgfqpoint{1.818658in}{1.079091in}}%
\pgfpathlineto{\pgfqpoint{1.819075in}{1.193023in}}%
\pgfpathlineto{\pgfqpoint{1.818971in}{0.933559in}}%
\pgfpathlineto{\pgfqpoint{1.819491in}{1.186155in}}%
\pgfpathlineto{\pgfqpoint{1.820324in}{0.901009in}}%
\pgfpathlineto{\pgfqpoint{1.820116in}{1.341964in}}%
\pgfpathlineto{\pgfqpoint{1.820533in}{0.997924in}}%
\pgfpathlineto{\pgfqpoint{1.821470in}{1.289559in}}%
\pgfpathlineto{\pgfqpoint{1.821158in}{0.884385in}}%
\pgfpathlineto{\pgfqpoint{1.821782in}{1.211798in}}%
\pgfpathlineto{\pgfqpoint{1.822199in}{0.729977in}}%
\pgfpathlineto{\pgfqpoint{1.822407in}{1.227181in}}%
\pgfpathlineto{\pgfqpoint{1.823240in}{0.993211in}}%
\pgfpathlineto{\pgfqpoint{1.824282in}{1.424912in}}%
\pgfpathlineto{\pgfqpoint{1.823865in}{0.952787in}}%
\pgfpathlineto{\pgfqpoint{1.824386in}{1.108633in}}%
\pgfpathlineto{\pgfqpoint{1.824490in}{0.755184in}}%
\pgfpathlineto{\pgfqpoint{1.825219in}{1.110587in}}%
\pgfpathlineto{\pgfqpoint{1.825427in}{1.098663in}}%
\pgfpathlineto{\pgfqpoint{1.826156in}{0.844449in}}%
\pgfpathlineto{\pgfqpoint{1.825844in}{1.336463in}}%
\pgfpathlineto{\pgfqpoint{1.826365in}{1.120702in}}%
\pgfpathlineto{\pgfqpoint{1.827198in}{0.929580in}}%
\pgfpathlineto{\pgfqpoint{1.827510in}{1.300712in}}%
\pgfpathlineto{\pgfqpoint{1.828656in}{0.736789in}}%
\pgfpathlineto{\pgfqpoint{1.828760in}{1.234899in}}%
\pgfpathlineto{\pgfqpoint{1.829801in}{1.119661in}}%
\pgfpathlineto{\pgfqpoint{1.830114in}{0.854701in}}%
\pgfpathlineto{\pgfqpoint{1.830010in}{1.250241in}}%
\pgfpathlineto{\pgfqpoint{1.830947in}{1.031187in}}%
\pgfpathlineto{\pgfqpoint{1.831572in}{0.894590in}}%
\pgfpathlineto{\pgfqpoint{1.831155in}{1.392187in}}%
\pgfpathlineto{\pgfqpoint{1.831780in}{1.149494in}}%
\pgfpathlineto{\pgfqpoint{1.832092in}{1.486700in}}%
\pgfpathlineto{\pgfqpoint{1.832301in}{0.956791in}}%
\pgfpathlineto{\pgfqpoint{1.832613in}{1.190155in}}%
\pgfpathlineto{\pgfqpoint{1.833655in}{0.768190in}}%
\pgfpathlineto{\pgfqpoint{1.832821in}{1.401977in}}%
\pgfpathlineto{\pgfqpoint{1.833759in}{1.041051in}}%
\pgfpathlineto{\pgfqpoint{1.833863in}{1.217482in}}%
\pgfpathlineto{\pgfqpoint{1.834696in}{0.757372in}}%
\pgfpathlineto{\pgfqpoint{1.834800in}{1.079128in}}%
\pgfpathlineto{\pgfqpoint{1.835217in}{1.205003in}}%
\pgfpathlineto{\pgfqpoint{1.835633in}{0.877306in}}%
\pgfpathlineto{\pgfqpoint{1.835737in}{1.356194in}}%
\pgfpathlineto{\pgfqpoint{1.836466in}{0.845980in}}%
\pgfpathlineto{\pgfqpoint{1.836675in}{0.964213in}}%
\pgfpathlineto{\pgfqpoint{1.836883in}{0.862668in}}%
\pgfpathlineto{\pgfqpoint{1.836987in}{1.219595in}}%
\pgfpathlineto{\pgfqpoint{1.837091in}{1.289434in}}%
\pgfpathlineto{\pgfqpoint{1.837612in}{0.939829in}}%
\pgfpathlineto{\pgfqpoint{1.837820in}{1.087516in}}%
\pgfpathlineto{\pgfqpoint{1.838237in}{0.931008in}}%
\pgfpathlineto{\pgfqpoint{1.838028in}{1.244372in}}%
\pgfpathlineto{\pgfqpoint{1.838653in}{1.135188in}}%
\pgfpathlineto{\pgfqpoint{1.838757in}{1.332742in}}%
\pgfpathlineto{\pgfqpoint{1.839174in}{0.861790in}}%
\pgfpathlineto{\pgfqpoint{1.839695in}{1.225806in}}%
\pgfpathlineto{\pgfqpoint{1.840632in}{0.697290in}}%
\pgfpathlineto{\pgfqpoint{1.840944in}{0.924360in}}%
\pgfpathlineto{\pgfqpoint{1.841257in}{1.152665in}}%
\pgfpathlineto{\pgfqpoint{1.841361in}{0.708187in}}%
\pgfpathlineto{\pgfqpoint{1.842090in}{1.006055in}}%
\pgfpathlineto{\pgfqpoint{1.842507in}{1.179004in}}%
\pgfpathlineto{\pgfqpoint{1.843236in}{0.706263in}}%
\pgfpathlineto{\pgfqpoint{1.844069in}{1.280346in}}%
\pgfpathlineto{\pgfqpoint{1.844485in}{1.164304in}}%
\pgfpathlineto{\pgfqpoint{1.844589in}{0.806441in}}%
\pgfpathlineto{\pgfqpoint{1.845527in}{0.930124in}}%
\pgfpathlineto{\pgfqpoint{1.846256in}{1.402877in}}%
\pgfpathlineto{\pgfqpoint{1.845839in}{0.786066in}}%
\pgfpathlineto{\pgfqpoint{1.846672in}{1.042982in}}%
\pgfpathlineto{\pgfqpoint{1.846880in}{0.847570in}}%
\pgfpathlineto{\pgfqpoint{1.847193in}{1.234595in}}%
\pgfpathlineto{\pgfqpoint{1.847297in}{1.377585in}}%
\pgfpathlineto{\pgfqpoint{1.848026in}{0.859217in}}%
\pgfpathlineto{\pgfqpoint{1.848859in}{1.180568in}}%
\pgfpathlineto{\pgfqpoint{1.848651in}{0.720339in}}%
\pgfpathlineto{\pgfqpoint{1.849067in}{0.931348in}}%
\pgfpathlineto{\pgfqpoint{1.849172in}{0.780069in}}%
\pgfpathlineto{\pgfqpoint{1.849276in}{1.287658in}}%
\pgfpathlineto{\pgfqpoint{1.850005in}{1.079587in}}%
\pgfpathlineto{\pgfqpoint{1.850109in}{1.242747in}}%
\pgfpathlineto{\pgfqpoint{1.850421in}{0.827899in}}%
\pgfpathlineto{\pgfqpoint{1.851046in}{1.066660in}}%
\pgfpathlineto{\pgfqpoint{1.851567in}{0.817727in}}%
\pgfpathlineto{\pgfqpoint{1.851463in}{1.280039in}}%
\pgfpathlineto{\pgfqpoint{1.852087in}{1.022931in}}%
\pgfpathlineto{\pgfqpoint{1.852921in}{1.342314in}}%
\pgfpathlineto{\pgfqpoint{1.852712in}{0.842740in}}%
\pgfpathlineto{\pgfqpoint{1.853025in}{0.937859in}}%
\pgfpathlineto{\pgfqpoint{1.853129in}{0.878914in}}%
\pgfpathlineto{\pgfqpoint{1.853545in}{1.261099in}}%
\pgfpathlineto{\pgfqpoint{1.853962in}{0.980362in}}%
\pgfpathlineto{\pgfqpoint{1.854587in}{1.374940in}}%
\pgfpathlineto{\pgfqpoint{1.854274in}{0.790833in}}%
\pgfpathlineto{\pgfqpoint{1.855003in}{1.011779in}}%
\pgfpathlineto{\pgfqpoint{1.855108in}{1.011577in}}%
\pgfpathlineto{\pgfqpoint{1.855316in}{1.278306in}}%
\pgfpathlineto{\pgfqpoint{1.855628in}{0.900467in}}%
\pgfpathlineto{\pgfqpoint{1.856149in}{1.021746in}}%
\pgfpathlineto{\pgfqpoint{1.856461in}{0.793029in}}%
\pgfpathlineto{\pgfqpoint{1.856357in}{1.132589in}}%
\pgfpathlineto{\pgfqpoint{1.856774in}{0.919009in}}%
\pgfpathlineto{\pgfqpoint{1.857295in}{1.252934in}}%
\pgfpathlineto{\pgfqpoint{1.857815in}{0.894168in}}%
\pgfpathlineto{\pgfqpoint{1.857919in}{0.843492in}}%
\pgfpathlineto{\pgfqpoint{1.858232in}{1.189170in}}%
\pgfpathlineto{\pgfqpoint{1.858544in}{0.884152in}}%
\pgfpathlineto{\pgfqpoint{1.859169in}{1.226021in}}%
\pgfpathlineto{\pgfqpoint{1.859482in}{0.830123in}}%
\pgfpathlineto{\pgfqpoint{1.859586in}{1.135254in}}%
\pgfpathlineto{\pgfqpoint{1.859690in}{0.939562in}}%
\pgfpathlineto{\pgfqpoint{1.860419in}{1.363589in}}%
\pgfpathlineto{\pgfqpoint{1.860731in}{1.054640in}}%
\pgfpathlineto{\pgfqpoint{1.861356in}{0.896858in}}%
\pgfpathlineto{\pgfqpoint{1.861460in}{1.249962in}}%
\pgfpathlineto{\pgfqpoint{1.861877in}{0.972330in}}%
\pgfpathlineto{\pgfqpoint{1.862085in}{1.213964in}}%
\pgfpathlineto{\pgfqpoint{1.862293in}{0.828981in}}%
\pgfpathlineto{\pgfqpoint{1.862918in}{0.947768in}}%
\pgfpathlineto{\pgfqpoint{1.863751in}{1.224329in}}%
\pgfpathlineto{\pgfqpoint{1.863335in}{0.942829in}}%
\pgfpathlineto{\pgfqpoint{1.863855in}{1.128787in}}%
\pgfpathlineto{\pgfqpoint{1.863960in}{0.820543in}}%
\pgfpathlineto{\pgfqpoint{1.864168in}{1.143099in}}%
\pgfpathlineto{\pgfqpoint{1.865001in}{1.019941in}}%
\pgfpathlineto{\pgfqpoint{1.865730in}{0.808629in}}%
\pgfpathlineto{\pgfqpoint{1.865522in}{1.132493in}}%
\pgfpathlineto{\pgfqpoint{1.865938in}{0.897010in}}%
\pgfpathlineto{\pgfqpoint{1.866563in}{1.283147in}}%
\pgfpathlineto{\pgfqpoint{1.866251in}{0.810568in}}%
\pgfpathlineto{\pgfqpoint{1.867084in}{0.996705in}}%
\pgfpathlineto{\pgfqpoint{1.867292in}{1.291018in}}%
\pgfpathlineto{\pgfqpoint{1.867396in}{0.951481in}}%
\pgfpathlineto{\pgfqpoint{1.867709in}{1.093897in}}%
\pgfpathlineto{\pgfqpoint{1.867813in}{0.818229in}}%
\pgfpathlineto{\pgfqpoint{1.868333in}{1.490264in}}%
\pgfpathlineto{\pgfqpoint{1.868750in}{1.164848in}}%
\pgfpathlineto{\pgfqpoint{1.868854in}{1.169256in}}%
\pgfpathlineto{\pgfqpoint{1.869896in}{0.752959in}}%
\pgfpathlineto{\pgfqpoint{1.869062in}{1.238059in}}%
\pgfpathlineto{\pgfqpoint{1.870104in}{1.086269in}}%
\pgfpathlineto{\pgfqpoint{1.870729in}{1.300291in}}%
\pgfpathlineto{\pgfqpoint{1.870312in}{0.829686in}}%
\pgfpathlineto{\pgfqpoint{1.871145in}{0.983168in}}%
\pgfpathlineto{\pgfqpoint{1.871666in}{1.117178in}}%
\pgfpathlineto{\pgfqpoint{1.871562in}{0.711123in}}%
\pgfpathlineto{\pgfqpoint{1.871978in}{1.086443in}}%
\pgfpathlineto{\pgfqpoint{1.872083in}{0.843641in}}%
\pgfpathlineto{\pgfqpoint{1.872187in}{1.337497in}}%
\pgfpathlineto{\pgfqpoint{1.873124in}{1.001516in}}%
\pgfpathlineto{\pgfqpoint{1.873749in}{1.290950in}}%
\pgfpathlineto{\pgfqpoint{1.874061in}{0.790543in}}%
\pgfpathlineto{\pgfqpoint{1.874790in}{1.291313in}}%
\pgfpathlineto{\pgfqpoint{1.875207in}{1.144692in}}%
\pgfpathlineto{\pgfqpoint{1.876248in}{0.815921in}}%
\pgfpathlineto{\pgfqpoint{1.876352in}{1.037523in}}%
\pgfpathlineto{\pgfqpoint{1.877185in}{1.206715in}}%
\pgfpathlineto{\pgfqpoint{1.876769in}{0.821366in}}%
\pgfpathlineto{\pgfqpoint{1.877602in}{1.155142in}}%
\pgfpathlineto{\pgfqpoint{1.878019in}{0.692838in}}%
\pgfpathlineto{\pgfqpoint{1.877914in}{1.287218in}}%
\pgfpathlineto{\pgfqpoint{1.878643in}{1.002312in}}%
\pgfpathlineto{\pgfqpoint{1.879060in}{1.263187in}}%
\pgfpathlineto{\pgfqpoint{1.879268in}{0.850555in}}%
\pgfpathlineto{\pgfqpoint{1.879581in}{1.128412in}}%
\pgfpathlineto{\pgfqpoint{1.880206in}{0.858378in}}%
\pgfpathlineto{\pgfqpoint{1.880310in}{1.536466in}}%
\pgfpathlineto{\pgfqpoint{1.880726in}{1.035893in}}%
\pgfpathlineto{\pgfqpoint{1.881351in}{1.338793in}}%
\pgfpathlineto{\pgfqpoint{1.881247in}{0.841612in}}%
\pgfpathlineto{\pgfqpoint{1.881664in}{1.022910in}}%
\pgfpathlineto{\pgfqpoint{1.882080in}{1.200338in}}%
\pgfpathlineto{\pgfqpoint{1.882601in}{0.796027in}}%
\pgfpathlineto{\pgfqpoint{1.883434in}{1.388137in}}%
\pgfpathlineto{\pgfqpoint{1.883746in}{1.275075in}}%
\pgfpathlineto{\pgfqpoint{1.884267in}{1.343858in}}%
\pgfpathlineto{\pgfqpoint{1.884684in}{0.843718in}}%
\pgfpathlineto{\pgfqpoint{1.884788in}{1.376028in}}%
\pgfpathlineto{\pgfqpoint{1.885100in}{0.798536in}}%
\pgfpathlineto{\pgfqpoint{1.885829in}{1.254155in}}%
\pgfpathlineto{\pgfqpoint{1.885933in}{0.968003in}}%
\pgfpathlineto{\pgfqpoint{1.886975in}{1.157155in}}%
\pgfpathlineto{\pgfqpoint{1.887079in}{1.160358in}}%
\pgfpathlineto{\pgfqpoint{1.887600in}{0.645473in}}%
\pgfpathlineto{\pgfqpoint{1.887495in}{1.240504in}}%
\pgfpathlineto{\pgfqpoint{1.888224in}{1.000009in}}%
\pgfpathlineto{\pgfqpoint{1.888537in}{0.755411in}}%
\pgfpathlineto{\pgfqpoint{1.889266in}{1.421788in}}%
\pgfpathlineto{\pgfqpoint{1.889682in}{0.722754in}}%
\pgfpathlineto{\pgfqpoint{1.890411in}{0.796421in}}%
\pgfpathlineto{\pgfqpoint{1.890932in}{1.264105in}}%
\pgfpathlineto{\pgfqpoint{1.891661in}{1.134580in}}%
\pgfpathlineto{\pgfqpoint{1.891974in}{1.264409in}}%
\pgfpathlineto{\pgfqpoint{1.892703in}{0.838596in}}%
\pgfpathlineto{\pgfqpoint{1.893327in}{1.255699in}}%
\pgfpathlineto{\pgfqpoint{1.893848in}{1.112127in}}%
\pgfpathlineto{\pgfqpoint{1.893952in}{1.106789in}}%
\pgfpathlineto{\pgfqpoint{1.894889in}{0.823133in}}%
\pgfpathlineto{\pgfqpoint{1.894473in}{1.370973in}}%
\pgfpathlineto{\pgfqpoint{1.894994in}{0.889678in}}%
\pgfpathlineto{\pgfqpoint{1.895202in}{1.312066in}}%
\pgfpathlineto{\pgfqpoint{1.895618in}{0.788118in}}%
\pgfpathlineto{\pgfqpoint{1.896139in}{1.295010in}}%
\pgfpathlineto{\pgfqpoint{1.896452in}{0.840558in}}%
\pgfpathlineto{\pgfqpoint{1.897285in}{1.070876in}}%
\pgfpathlineto{\pgfqpoint{1.897389in}{1.088193in}}%
\pgfpathlineto{\pgfqpoint{1.897493in}{1.064795in}}%
\pgfpathlineto{\pgfqpoint{1.897805in}{0.728990in}}%
\pgfpathlineto{\pgfqpoint{1.898118in}{1.178052in}}%
\pgfpathlineto{\pgfqpoint{1.898534in}{1.079197in}}%
\pgfpathlineto{\pgfqpoint{1.898743in}{1.021458in}}%
\pgfpathlineto{\pgfqpoint{1.898847in}{1.108348in}}%
\pgfpathlineto{\pgfqpoint{1.899055in}{1.071906in}}%
\pgfpathlineto{\pgfqpoint{1.899472in}{0.858755in}}%
\pgfpathlineto{\pgfqpoint{1.899784in}{1.200147in}}%
\pgfpathlineto{\pgfqpoint{1.900097in}{0.870795in}}%
\pgfpathlineto{\pgfqpoint{1.900513in}{1.271546in}}%
\pgfpathlineto{\pgfqpoint{1.900826in}{0.794305in}}%
\pgfpathlineto{\pgfqpoint{1.901242in}{1.060377in}}%
\pgfpathlineto{\pgfqpoint{1.901346in}{1.078458in}}%
\pgfpathlineto{\pgfqpoint{1.901450in}{0.950380in}}%
\pgfpathlineto{\pgfqpoint{1.901659in}{1.471625in}}%
\pgfpathlineto{\pgfqpoint{1.902388in}{0.887400in}}%
\pgfpathlineto{\pgfqpoint{1.902492in}{1.113683in}}%
\pgfpathlineto{\pgfqpoint{1.902700in}{0.675530in}}%
\pgfpathlineto{\pgfqpoint{1.903012in}{1.251939in}}%
\pgfpathlineto{\pgfqpoint{1.903533in}{0.962369in}}%
\pgfpathlineto{\pgfqpoint{1.904366in}{1.248406in}}%
\pgfpathlineto{\pgfqpoint{1.904262in}{0.757779in}}%
\pgfpathlineto{\pgfqpoint{1.904575in}{1.111745in}}%
\pgfpathlineto{\pgfqpoint{1.905304in}{1.345443in}}%
\pgfpathlineto{\pgfqpoint{1.905616in}{0.829902in}}%
\pgfpathlineto{\pgfqpoint{1.905720in}{1.365174in}}%
\pgfpathlineto{\pgfqpoint{1.906657in}{1.207633in}}%
\pgfpathlineto{\pgfqpoint{1.906970in}{0.840221in}}%
\pgfpathlineto{\pgfqpoint{1.906866in}{1.288943in}}%
\pgfpathlineto{\pgfqpoint{1.907803in}{1.020634in}}%
\pgfpathlineto{\pgfqpoint{1.908011in}{1.222538in}}%
\pgfpathlineto{\pgfqpoint{1.908636in}{0.865986in}}%
\pgfpathlineto{\pgfqpoint{1.908740in}{1.043640in}}%
\pgfpathlineto{\pgfqpoint{1.909157in}{0.774756in}}%
\pgfpathlineto{\pgfqpoint{1.909053in}{1.173791in}}%
\pgfpathlineto{\pgfqpoint{1.909782in}{0.967560in}}%
\pgfpathlineto{\pgfqpoint{1.910511in}{1.280754in}}%
\pgfpathlineto{\pgfqpoint{1.909990in}{0.715686in}}%
\pgfpathlineto{\pgfqpoint{1.910823in}{1.131628in}}%
\pgfpathlineto{\pgfqpoint{1.910927in}{0.817583in}}%
\pgfpathlineto{\pgfqpoint{1.911760in}{1.212545in}}%
\pgfpathlineto{\pgfqpoint{1.911969in}{0.870317in}}%
\pgfpathlineto{\pgfqpoint{1.912385in}{0.798338in}}%
\pgfpathlineto{\pgfqpoint{1.913114in}{1.398411in}}%
\pgfpathlineto{\pgfqpoint{1.913635in}{0.967970in}}%
\pgfpathlineto{\pgfqpoint{1.914156in}{1.235716in}}%
\pgfpathlineto{\pgfqpoint{1.914260in}{1.358157in}}%
\pgfpathlineto{\pgfqpoint{1.914885in}{0.922538in}}%
\pgfpathlineto{\pgfqpoint{1.915093in}{1.309394in}}%
\pgfpathlineto{\pgfqpoint{1.915301in}{0.857216in}}%
\pgfpathlineto{\pgfqpoint{1.916134in}{1.089205in}}%
\pgfpathlineto{\pgfqpoint{1.917072in}{1.372917in}}%
\pgfpathlineto{\pgfqpoint{1.916655in}{0.894810in}}%
\pgfpathlineto{\pgfqpoint{1.917280in}{1.223076in}}%
\pgfpathlineto{\pgfqpoint{1.917800in}{0.908391in}}%
\pgfpathlineto{\pgfqpoint{1.917592in}{1.406399in}}%
\pgfpathlineto{\pgfqpoint{1.918529in}{0.960740in}}%
\pgfpathlineto{\pgfqpoint{1.918738in}{1.326588in}}%
\pgfpathlineto{\pgfqpoint{1.919363in}{0.944353in}}%
\pgfpathlineto{\pgfqpoint{1.919675in}{1.231614in}}%
\pgfpathlineto{\pgfqpoint{1.920404in}{0.887239in}}%
\pgfpathlineto{\pgfqpoint{1.919987in}{1.339451in}}%
\pgfpathlineto{\pgfqpoint{1.920821in}{0.954422in}}%
\pgfpathlineto{\pgfqpoint{1.921654in}{0.741563in}}%
\pgfpathlineto{\pgfqpoint{1.921862in}{1.174380in}}%
\pgfpathlineto{\pgfqpoint{1.922591in}{1.447693in}}%
\pgfpathlineto{\pgfqpoint{1.923008in}{0.811194in}}%
\pgfpathlineto{\pgfqpoint{1.923841in}{1.208829in}}%
\pgfpathlineto{\pgfqpoint{1.923737in}{0.691336in}}%
\pgfpathlineto{\pgfqpoint{1.924153in}{1.192373in}}%
\pgfpathlineto{\pgfqpoint{1.924986in}{0.928964in}}%
\pgfpathlineto{\pgfqpoint{1.924361in}{1.232773in}}%
\pgfpathlineto{\pgfqpoint{1.925299in}{1.059100in}}%
\pgfpathlineto{\pgfqpoint{1.925403in}{1.214153in}}%
\pgfpathlineto{\pgfqpoint{1.925715in}{0.966108in}}%
\pgfpathlineto{\pgfqpoint{1.926340in}{1.170501in}}%
\pgfpathlineto{\pgfqpoint{1.926757in}{0.804741in}}%
\pgfpathlineto{\pgfqpoint{1.927069in}{1.331112in}}%
\pgfpathlineto{\pgfqpoint{1.927486in}{1.000332in}}%
\pgfpathlineto{\pgfqpoint{1.928319in}{1.500717in}}%
\pgfpathlineto{\pgfqpoint{1.928110in}{0.803435in}}%
\pgfpathlineto{\pgfqpoint{1.928527in}{1.222733in}}%
\pgfpathlineto{\pgfqpoint{1.929568in}{0.889401in}}%
\pgfpathlineto{\pgfqpoint{1.929152in}{1.346252in}}%
\pgfpathlineto{\pgfqpoint{1.929673in}{1.012188in}}%
\pgfpathlineto{\pgfqpoint{1.929985in}{1.412637in}}%
\pgfpathlineto{\pgfqpoint{1.930506in}{0.927108in}}%
\pgfpathlineto{\pgfqpoint{1.930714in}{0.978103in}}%
\pgfpathlineto{\pgfqpoint{1.930818in}{0.791922in}}%
\pgfpathlineto{\pgfqpoint{1.931651in}{1.268682in}}%
\pgfpathlineto{\pgfqpoint{1.931755in}{0.964476in}}%
\pgfpathlineto{\pgfqpoint{1.932172in}{1.508926in}}%
\pgfpathlineto{\pgfqpoint{1.932068in}{0.860142in}}%
\pgfpathlineto{\pgfqpoint{1.932797in}{1.073281in}}%
\pgfpathlineto{\pgfqpoint{1.932901in}{0.833276in}}%
\pgfpathlineto{\pgfqpoint{1.933734in}{1.265070in}}%
\pgfpathlineto{\pgfqpoint{1.933838in}{1.186084in}}%
\pgfpathlineto{\pgfqpoint{1.934671in}{0.818494in}}%
\pgfpathlineto{\pgfqpoint{1.934151in}{1.557730in}}%
\pgfpathlineto{\pgfqpoint{1.934880in}{1.110025in}}%
\pgfpathlineto{\pgfqpoint{1.934984in}{1.279614in}}%
\pgfpathlineto{\pgfqpoint{1.935296in}{0.736566in}}%
\pgfpathlineto{\pgfqpoint{1.935817in}{0.940096in}}%
\pgfpathlineto{\pgfqpoint{1.936025in}{0.758252in}}%
\pgfpathlineto{\pgfqpoint{1.936129in}{1.053742in}}%
\pgfpathlineto{\pgfqpoint{1.936233in}{0.937005in}}%
\pgfpathlineto{\pgfqpoint{1.936338in}{1.439715in}}%
\pgfpathlineto{\pgfqpoint{1.936650in}{0.823737in}}%
\pgfpathlineto{\pgfqpoint{1.937379in}{1.326608in}}%
\pgfpathlineto{\pgfqpoint{1.938004in}{1.007067in}}%
\pgfpathlineto{\pgfqpoint{1.938629in}{1.025594in}}%
\pgfpathlineto{\pgfqpoint{1.939045in}{1.251154in}}%
\pgfpathlineto{\pgfqpoint{1.938837in}{0.984630in}}%
\pgfpathlineto{\pgfqpoint{1.939254in}{1.074038in}}%
\pgfpathlineto{\pgfqpoint{1.939358in}{0.841636in}}%
\pgfpathlineto{\pgfqpoint{1.940295in}{0.995618in}}%
\pgfpathlineto{\pgfqpoint{1.940920in}{1.304936in}}%
\pgfpathlineto{\pgfqpoint{1.940712in}{0.956961in}}%
\pgfpathlineto{\pgfqpoint{1.941649in}{1.211108in}}%
\pgfpathlineto{\pgfqpoint{1.942378in}{0.780596in}}%
\pgfpathlineto{\pgfqpoint{1.942690in}{0.813868in}}%
\pgfpathlineto{\pgfqpoint{1.943627in}{1.396793in}}%
\pgfpathlineto{\pgfqpoint{1.943836in}{1.033428in}}%
\pgfpathlineto{\pgfqpoint{1.944565in}{1.276548in}}%
\pgfpathlineto{\pgfqpoint{1.944669in}{0.707813in}}%
\pgfpathlineto{\pgfqpoint{1.945398in}{1.404365in}}%
\pgfpathlineto{\pgfqpoint{1.945814in}{0.966026in}}%
\pgfpathlineto{\pgfqpoint{1.946127in}{0.752705in}}%
\pgfpathlineto{\pgfqpoint{1.946439in}{1.028418in}}%
\pgfpathlineto{\pgfqpoint{1.946648in}{1.107399in}}%
\pgfpathlineto{\pgfqpoint{1.946856in}{0.968781in}}%
\pgfpathlineto{\pgfqpoint{1.946960in}{0.765837in}}%
\pgfpathlineto{\pgfqpoint{1.947481in}{1.192006in}}%
\pgfpathlineto{\pgfqpoint{1.948001in}{0.832784in}}%
\pgfpathlineto{\pgfqpoint{1.948730in}{1.210936in}}%
\pgfpathlineto{\pgfqpoint{1.948939in}{0.723370in}}%
\pgfpathlineto{\pgfqpoint{1.949147in}{1.031232in}}%
\pgfpathlineto{\pgfqpoint{1.949355in}{0.962023in}}%
\pgfpathlineto{\pgfqpoint{1.949459in}{1.086103in}}%
\pgfpathlineto{\pgfqpoint{1.949668in}{1.050524in}}%
\pgfpathlineto{\pgfqpoint{1.950084in}{1.256851in}}%
\pgfpathlineto{\pgfqpoint{1.949980in}{0.982712in}}%
\pgfpathlineto{\pgfqpoint{1.950605in}{1.190511in}}%
\pgfpathlineto{\pgfqpoint{1.950709in}{0.822232in}}%
\pgfpathlineto{\pgfqpoint{1.951021in}{1.300550in}}%
\pgfpathlineto{\pgfqpoint{1.951646in}{0.904546in}}%
\pgfpathlineto{\pgfqpoint{1.952271in}{1.226557in}}%
\pgfpathlineto{\pgfqpoint{1.952375in}{0.899743in}}%
\pgfpathlineto{\pgfqpoint{1.952688in}{0.940813in}}%
\pgfpathlineto{\pgfqpoint{1.953000in}{0.833141in}}%
\pgfpathlineto{\pgfqpoint{1.952896in}{1.069602in}}%
\pgfpathlineto{\pgfqpoint{1.953104in}{0.906144in}}%
\pgfpathlineto{\pgfqpoint{1.953208in}{1.355829in}}%
\pgfpathlineto{\pgfqpoint{1.954042in}{0.754810in}}%
\pgfpathlineto{\pgfqpoint{1.954250in}{1.078236in}}%
\pgfpathlineto{\pgfqpoint{1.954875in}{1.204666in}}%
\pgfpathlineto{\pgfqpoint{1.954458in}{0.920269in}}%
\pgfpathlineto{\pgfqpoint{1.954979in}{1.187654in}}%
\pgfpathlineto{\pgfqpoint{1.956020in}{1.225595in}}%
\pgfpathlineto{\pgfqpoint{1.956124in}{0.640009in}}%
\pgfpathlineto{\pgfqpoint{1.956958in}{1.522141in}}%
\pgfpathlineto{\pgfqpoint{1.957270in}{0.920119in}}%
\pgfpathlineto{\pgfqpoint{1.957582in}{1.174903in}}%
\pgfpathlineto{\pgfqpoint{1.957791in}{0.889026in}}%
\pgfpathlineto{\pgfqpoint{1.958416in}{1.164382in}}%
\pgfpathlineto{\pgfqpoint{1.958936in}{0.891573in}}%
\pgfpathlineto{\pgfqpoint{1.959144in}{1.247660in}}%
\pgfpathlineto{\pgfqpoint{1.959561in}{1.120623in}}%
\pgfpathlineto{\pgfqpoint{1.959769in}{0.756942in}}%
\pgfpathlineto{\pgfqpoint{1.959978in}{1.201737in}}%
\pgfpathlineto{\pgfqpoint{1.960082in}{0.833110in}}%
\pgfpathlineto{\pgfqpoint{1.960707in}{1.364089in}}%
\pgfpathlineto{\pgfqpoint{1.961123in}{0.957095in}}%
\pgfpathlineto{\pgfqpoint{1.961227in}{1.171545in}}%
\pgfpathlineto{\pgfqpoint{1.961748in}{0.932898in}}%
\pgfpathlineto{\pgfqpoint{1.962269in}{1.093130in}}%
\pgfpathlineto{\pgfqpoint{1.962373in}{0.844264in}}%
\pgfpathlineto{\pgfqpoint{1.963310in}{1.252573in}}%
\pgfpathlineto{\pgfqpoint{1.964039in}{0.799120in}}%
\pgfpathlineto{\pgfqpoint{1.963727in}{1.299098in}}%
\pgfpathlineto{\pgfqpoint{1.964560in}{0.981007in}}%
\pgfpathlineto{\pgfqpoint{1.964664in}{1.449262in}}%
\pgfpathlineto{\pgfqpoint{1.965497in}{0.851456in}}%
\pgfpathlineto{\pgfqpoint{1.965705in}{1.315781in}}%
\pgfpathlineto{\pgfqpoint{1.966955in}{0.601474in}}%
\pgfpathlineto{\pgfqpoint{1.967788in}{1.363430in}}%
\pgfpathlineto{\pgfqpoint{1.968101in}{1.079077in}}%
\pgfpathlineto{\pgfqpoint{1.968309in}{1.165957in}}%
\pgfpathlineto{\pgfqpoint{1.969246in}{0.777218in}}%
\pgfpathlineto{\pgfqpoint{1.969663in}{1.142223in}}%
\pgfpathlineto{\pgfqpoint{1.969975in}{0.747670in}}%
\pgfpathlineto{\pgfqpoint{1.970392in}{0.914493in}}%
\pgfpathlineto{\pgfqpoint{1.970704in}{1.299721in}}%
\pgfpathlineto{\pgfqpoint{1.970912in}{0.784007in}}%
\pgfpathlineto{\pgfqpoint{1.971225in}{0.917616in}}%
\pgfpathlineto{\pgfqpoint{1.971329in}{0.618245in}}%
\pgfpathlineto{\pgfqpoint{1.971433in}{1.252361in}}%
\pgfpathlineto{\pgfqpoint{1.972162in}{0.918976in}}%
\pgfpathlineto{\pgfqpoint{1.972266in}{1.276848in}}%
\pgfpathlineto{\pgfqpoint{1.972370in}{0.823570in}}%
\pgfpathlineto{\pgfqpoint{1.973308in}{1.062340in}}%
\pgfpathlineto{\pgfqpoint{1.973412in}{1.320979in}}%
\pgfpathlineto{\pgfqpoint{1.974141in}{0.855166in}}%
\pgfpathlineto{\pgfqpoint{1.974349in}{0.934103in}}%
\pgfpathlineto{\pgfqpoint{1.974453in}{0.899233in}}%
\pgfpathlineto{\pgfqpoint{1.974557in}{0.948392in}}%
\pgfpathlineto{\pgfqpoint{1.974662in}{1.419156in}}%
\pgfpathlineto{\pgfqpoint{1.975078in}{0.783402in}}%
\pgfpathlineto{\pgfqpoint{1.975599in}{1.162634in}}%
\pgfpathlineto{\pgfqpoint{1.976015in}{1.260685in}}%
\pgfpathlineto{\pgfqpoint{1.976744in}{0.726923in}}%
\pgfpathlineto{\pgfqpoint{1.977161in}{1.300591in}}%
\pgfpathlineto{\pgfqpoint{1.977890in}{0.984577in}}%
\pgfpathlineto{\pgfqpoint{1.979035in}{1.355934in}}%
\pgfpathlineto{\pgfqpoint{1.978098in}{0.753946in}}%
\pgfpathlineto{\pgfqpoint{1.979140in}{1.222911in}}%
\pgfpathlineto{\pgfqpoint{1.979764in}{0.732018in}}%
\pgfpathlineto{\pgfqpoint{1.980077in}{1.312107in}}%
\pgfpathlineto{\pgfqpoint{1.980285in}{1.058993in}}%
\pgfpathlineto{\pgfqpoint{1.980702in}{1.417191in}}%
\pgfpathlineto{\pgfqpoint{1.981118in}{1.002214in}}%
\pgfpathlineto{\pgfqpoint{1.981327in}{1.261156in}}%
\pgfpathlineto{\pgfqpoint{1.981535in}{0.805245in}}%
\pgfpathlineto{\pgfqpoint{1.982056in}{1.270256in}}%
\pgfpathlineto{\pgfqpoint{1.982472in}{1.172241in}}%
\pgfpathlineto{\pgfqpoint{1.982576in}{1.343251in}}%
\pgfpathlineto{\pgfqpoint{1.983409in}{0.912929in}}%
\pgfpathlineto{\pgfqpoint{1.983513in}{1.117986in}}%
\pgfpathlineto{\pgfqpoint{1.983722in}{0.835512in}}%
\pgfpathlineto{\pgfqpoint{1.984138in}{1.223381in}}%
\pgfpathlineto{\pgfqpoint{1.984659in}{0.962718in}}%
\pgfpathlineto{\pgfqpoint{1.985180in}{0.728149in}}%
\pgfpathlineto{\pgfqpoint{1.985284in}{1.193178in}}%
\pgfpathlineto{\pgfqpoint{1.985596in}{0.829834in}}%
\pgfpathlineto{\pgfqpoint{1.986013in}{1.393710in}}%
\pgfpathlineto{\pgfqpoint{1.986742in}{1.094107in}}%
\pgfpathlineto{\pgfqpoint{1.987679in}{0.779698in}}%
\pgfpathlineto{\pgfqpoint{1.986950in}{1.276113in}}%
\pgfpathlineto{\pgfqpoint{1.988096in}{0.871712in}}%
\pgfpathlineto{\pgfqpoint{1.988616in}{1.380240in}}%
\pgfpathlineto{\pgfqpoint{1.988408in}{0.705008in}}%
\pgfpathlineto{\pgfqpoint{1.989241in}{1.048735in}}%
\pgfpathlineto{\pgfqpoint{1.989554in}{0.835771in}}%
\pgfpathlineto{\pgfqpoint{1.989866in}{1.333321in}}%
\pgfpathlineto{\pgfqpoint{1.990387in}{0.886344in}}%
\pgfpathlineto{\pgfqpoint{1.991220in}{1.136911in}}%
\pgfpathlineto{\pgfqpoint{1.991012in}{0.763492in}}%
\pgfpathlineto{\pgfqpoint{1.991636in}{0.974755in}}%
\pgfpathlineto{\pgfqpoint{1.991741in}{0.960662in}}%
\pgfpathlineto{\pgfqpoint{1.991845in}{1.079698in}}%
\pgfpathlineto{\pgfqpoint{1.991949in}{1.393071in}}%
\pgfpathlineto{\pgfqpoint{1.992574in}{0.705989in}}%
\pgfpathlineto{\pgfqpoint{1.992886in}{1.010981in}}%
\pgfpathlineto{\pgfqpoint{1.992990in}{0.933624in}}%
\pgfpathlineto{\pgfqpoint{1.993199in}{1.187273in}}%
\pgfpathlineto{\pgfqpoint{1.993719in}{1.163606in}}%
\pgfpathlineto{\pgfqpoint{1.994240in}{1.412764in}}%
\pgfpathlineto{\pgfqpoint{1.993928in}{0.671523in}}%
\pgfpathlineto{\pgfqpoint{1.994552in}{1.048984in}}%
\pgfpathlineto{\pgfqpoint{1.995073in}{1.339599in}}%
\pgfpathlineto{\pgfqpoint{1.995594in}{0.901162in}}%
\pgfpathlineto{\pgfqpoint{1.996531in}{1.355449in}}%
\pgfpathlineto{\pgfqpoint{1.996739in}{1.223521in}}%
\pgfpathlineto{\pgfqpoint{1.997052in}{0.860572in}}%
\pgfpathlineto{\pgfqpoint{1.997677in}{1.294261in}}%
\pgfpathlineto{\pgfqpoint{1.997885in}{1.043797in}}%
\pgfpathlineto{\pgfqpoint{1.997989in}{1.032328in}}%
\pgfpathlineto{\pgfqpoint{1.998093in}{1.124751in}}%
\pgfpathlineto{\pgfqpoint{1.998614in}{1.315873in}}%
\pgfpathlineto{\pgfqpoint{1.998718in}{0.999997in}}%
\pgfpathlineto{\pgfqpoint{1.999135in}{1.221155in}}%
\pgfpathlineto{\pgfqpoint{1.999864in}{0.908906in}}%
\pgfpathlineto{\pgfqpoint{1.999447in}{1.272170in}}%
\pgfpathlineto{\pgfqpoint{2.000176in}{1.061026in}}%
\pgfpathlineto{\pgfqpoint{2.000384in}{1.280544in}}%
\pgfpathlineto{\pgfqpoint{2.000488in}{0.928448in}}%
\pgfpathlineto{\pgfqpoint{2.001217in}{1.153118in}}%
\pgfpathlineto{\pgfqpoint{2.001946in}{0.714364in}}%
\pgfpathlineto{\pgfqpoint{2.001530in}{1.268458in}}%
\pgfpathlineto{\pgfqpoint{2.002363in}{0.884604in}}%
\pgfpathlineto{\pgfqpoint{2.002675in}{1.343849in}}%
\pgfpathlineto{\pgfqpoint{2.002884in}{0.856744in}}%
\pgfpathlineto{\pgfqpoint{2.004029in}{1.226907in}}%
\pgfpathlineto{\pgfqpoint{2.004967in}{0.746588in}}%
\pgfpathlineto{\pgfqpoint{2.005175in}{0.990875in}}%
\pgfpathlineto{\pgfqpoint{2.005383in}{1.248396in}}%
\pgfpathlineto{\pgfqpoint{2.005487in}{0.927956in}}%
\pgfpathlineto{\pgfqpoint{2.006425in}{1.350704in}}%
\pgfpathlineto{\pgfqpoint{2.006112in}{0.626552in}}%
\pgfpathlineto{\pgfqpoint{2.006633in}{1.101698in}}%
\pgfpathlineto{\pgfqpoint{2.007258in}{0.746037in}}%
\pgfpathlineto{\pgfqpoint{2.006945in}{1.369654in}}%
\pgfpathlineto{\pgfqpoint{2.007883in}{1.010777in}}%
\pgfpathlineto{\pgfqpoint{2.008403in}{1.231621in}}%
\pgfpathlineto{\pgfqpoint{2.008195in}{0.901832in}}%
\pgfpathlineto{\pgfqpoint{2.008716in}{0.922164in}}%
\pgfpathlineto{\pgfqpoint{2.008820in}{0.620381in}}%
\pgfpathlineto{\pgfqpoint{2.009236in}{1.536702in}}%
\pgfpathlineto{\pgfqpoint{2.009757in}{1.105264in}}%
\pgfpathlineto{\pgfqpoint{2.009965in}{1.322854in}}%
\pgfpathlineto{\pgfqpoint{2.010278in}{0.844272in}}%
\pgfpathlineto{\pgfqpoint{2.010798in}{1.107058in}}%
\pgfpathlineto{\pgfqpoint{2.010903in}{1.115477in}}%
\pgfpathlineto{\pgfqpoint{2.011007in}{1.066087in}}%
\pgfpathlineto{\pgfqpoint{2.011111in}{1.080416in}}%
\pgfpathlineto{\pgfqpoint{2.011632in}{0.905758in}}%
\pgfpathlineto{\pgfqpoint{2.011319in}{1.200433in}}%
\pgfpathlineto{\pgfqpoint{2.012256in}{1.005589in}}%
\pgfpathlineto{\pgfqpoint{2.012777in}{1.247294in}}%
\pgfpathlineto{\pgfqpoint{2.013298in}{0.958043in}}%
\pgfpathlineto{\pgfqpoint{2.013402in}{1.070374in}}%
\pgfpathlineto{\pgfqpoint{2.013506in}{1.216862in}}%
\pgfpathlineto{\pgfqpoint{2.014131in}{0.847710in}}%
\pgfpathlineto{\pgfqpoint{2.014443in}{1.115360in}}%
\pgfpathlineto{\pgfqpoint{2.015589in}{0.858538in}}%
\pgfpathlineto{\pgfqpoint{2.015277in}{1.418471in}}%
\pgfpathlineto{\pgfqpoint{2.015693in}{0.935280in}}%
\pgfpathlineto{\pgfqpoint{2.016006in}{1.336030in}}%
\pgfpathlineto{\pgfqpoint{2.016110in}{0.871574in}}%
\pgfpathlineto{\pgfqpoint{2.016734in}{0.993032in}}%
\pgfpathlineto{\pgfqpoint{2.017047in}{0.919847in}}%
\pgfpathlineto{\pgfqpoint{2.017255in}{1.205122in}}%
\pgfpathlineto{\pgfqpoint{2.017359in}{0.785692in}}%
\pgfpathlineto{\pgfqpoint{2.017463in}{1.291691in}}%
\pgfpathlineto{\pgfqpoint{2.018401in}{0.977536in}}%
\pgfpathlineto{\pgfqpoint{2.018505in}{0.920359in}}%
\pgfpathlineto{\pgfqpoint{2.018609in}{1.234017in}}%
\pgfpathlineto{\pgfqpoint{2.019130in}{0.967971in}}%
\pgfpathlineto{\pgfqpoint{2.019234in}{1.244177in}}%
\pgfpathlineto{\pgfqpoint{2.019963in}{0.913122in}}%
\pgfpathlineto{\pgfqpoint{2.020171in}{0.930115in}}%
\pgfpathlineto{\pgfqpoint{2.020275in}{0.935120in}}%
\pgfpathlineto{\pgfqpoint{2.020379in}{0.916264in}}%
\pgfpathlineto{\pgfqpoint{2.020900in}{1.369959in}}%
\pgfpathlineto{\pgfqpoint{2.021004in}{0.906622in}}%
\pgfpathlineto{\pgfqpoint{2.021629in}{1.139343in}}%
\pgfpathlineto{\pgfqpoint{2.021733in}{1.152545in}}%
\pgfpathlineto{\pgfqpoint{2.021942in}{0.908832in}}%
\pgfpathlineto{\pgfqpoint{2.022566in}{1.386444in}}%
\pgfpathlineto{\pgfqpoint{2.022775in}{0.990506in}}%
\pgfpathlineto{\pgfqpoint{2.023295in}{1.323230in}}%
\pgfpathlineto{\pgfqpoint{2.023191in}{0.948647in}}%
\pgfpathlineto{\pgfqpoint{2.023816in}{1.255768in}}%
\pgfpathlineto{\pgfqpoint{2.024441in}{0.835173in}}%
\pgfpathlineto{\pgfqpoint{2.024857in}{1.155028in}}%
\pgfpathlineto{\pgfqpoint{2.024962in}{1.189192in}}%
\pgfpathlineto{\pgfqpoint{2.025066in}{1.042659in}}%
\pgfpathlineto{\pgfqpoint{2.025170in}{1.088027in}}%
\pgfpathlineto{\pgfqpoint{2.026211in}{0.727276in}}%
\pgfpathlineto{\pgfqpoint{2.025795in}{1.388665in}}%
\pgfpathlineto{\pgfqpoint{2.026315in}{0.938513in}}%
\pgfpathlineto{\pgfqpoint{2.027044in}{1.289420in}}%
\pgfpathlineto{\pgfqpoint{2.026732in}{0.679681in}}%
\pgfpathlineto{\pgfqpoint{2.027253in}{0.926381in}}%
\pgfpathlineto{\pgfqpoint{2.027357in}{0.660596in}}%
\pgfpathlineto{\pgfqpoint{2.027773in}{1.368950in}}%
\pgfpathlineto{\pgfqpoint{2.028294in}{0.763493in}}%
\pgfpathlineto{\pgfqpoint{2.028607in}{1.253792in}}%
\pgfpathlineto{\pgfqpoint{2.029231in}{0.702891in}}%
\pgfpathlineto{\pgfqpoint{2.029440in}{1.031775in}}%
\pgfpathlineto{\pgfqpoint{2.029960in}{0.770523in}}%
\pgfpathlineto{\pgfqpoint{2.029648in}{1.166361in}}%
\pgfpathlineto{\pgfqpoint{2.030065in}{0.888374in}}%
\pgfpathlineto{\pgfqpoint{2.030169in}{1.405601in}}%
\pgfpathlineto{\pgfqpoint{2.031210in}{1.163181in}}%
\pgfpathlineto{\pgfqpoint{2.031835in}{1.294289in}}%
\pgfpathlineto{\pgfqpoint{2.031939in}{0.901032in}}%
\pgfpathlineto{\pgfqpoint{2.032252in}{1.130537in}}%
\pgfpathlineto{\pgfqpoint{2.033293in}{0.778514in}}%
\pgfpathlineto{\pgfqpoint{2.032460in}{1.360553in}}%
\pgfpathlineto{\pgfqpoint{2.033397in}{0.818251in}}%
\pgfpathlineto{\pgfqpoint{2.033709in}{1.288000in}}%
\pgfpathlineto{\pgfqpoint{2.034438in}{0.767052in}}%
\pgfpathlineto{\pgfqpoint{2.034647in}{1.262747in}}%
\pgfpathlineto{\pgfqpoint{2.035688in}{0.848667in}}%
\pgfpathlineto{\pgfqpoint{2.035896in}{0.859093in}}%
\pgfpathlineto{\pgfqpoint{2.036834in}{1.305261in}}%
\pgfpathlineto{\pgfqpoint{2.037146in}{1.159263in}}%
\pgfpathlineto{\pgfqpoint{2.037667in}{1.264615in}}%
\pgfpathlineto{\pgfqpoint{2.038188in}{0.886858in}}%
\pgfpathlineto{\pgfqpoint{2.038396in}{1.319161in}}%
\pgfpathlineto{\pgfqpoint{2.039021in}{0.870595in}}%
\pgfpathlineto{\pgfqpoint{2.039333in}{1.160762in}}%
\pgfpathlineto{\pgfqpoint{2.040062in}{0.797763in}}%
\pgfpathlineto{\pgfqpoint{2.039646in}{1.200147in}}%
\pgfpathlineto{\pgfqpoint{2.040375in}{1.104070in}}%
\pgfpathlineto{\pgfqpoint{2.040895in}{1.398235in}}%
\pgfpathlineto{\pgfqpoint{2.041103in}{0.900824in}}%
\pgfpathlineto{\pgfqpoint{2.041520in}{1.177689in}}%
\pgfpathlineto{\pgfqpoint{2.041624in}{1.234618in}}%
\pgfpathlineto{\pgfqpoint{2.041937in}{0.951595in}}%
\pgfpathlineto{\pgfqpoint{2.042353in}{1.103213in}}%
\pgfpathlineto{\pgfqpoint{2.043082in}{1.193364in}}%
\pgfpathlineto{\pgfqpoint{2.043290in}{0.746772in}}%
\pgfpathlineto{\pgfqpoint{2.043395in}{1.275643in}}%
\pgfpathlineto{\pgfqpoint{2.044436in}{0.998660in}}%
\pgfpathlineto{\pgfqpoint{2.044540in}{0.758221in}}%
\pgfpathlineto{\pgfqpoint{2.044957in}{1.241925in}}%
\pgfpathlineto{\pgfqpoint{2.045477in}{1.064079in}}%
\pgfpathlineto{\pgfqpoint{2.045998in}{1.408337in}}%
\pgfpathlineto{\pgfqpoint{2.045790in}{0.925325in}}%
\pgfpathlineto{\pgfqpoint{2.046311in}{1.270851in}}%
\pgfpathlineto{\pgfqpoint{2.046415in}{0.841704in}}%
\pgfpathlineto{\pgfqpoint{2.047456in}{1.020269in}}%
\pgfpathlineto{\pgfqpoint{2.048393in}{1.313123in}}%
\pgfpathlineto{\pgfqpoint{2.048185in}{0.889780in}}%
\pgfpathlineto{\pgfqpoint{2.048498in}{1.016382in}}%
\pgfpathlineto{\pgfqpoint{2.048602in}{0.640750in}}%
\pgfpathlineto{\pgfqpoint{2.049018in}{1.250068in}}%
\pgfpathlineto{\pgfqpoint{2.049539in}{0.896959in}}%
\pgfpathlineto{\pgfqpoint{2.049643in}{0.916847in}}%
\pgfpathlineto{\pgfqpoint{2.050268in}{1.269199in}}%
\pgfpathlineto{\pgfqpoint{2.050684in}{0.875601in}}%
\pgfpathlineto{\pgfqpoint{2.050789in}{1.018070in}}%
\pgfpathlineto{\pgfqpoint{2.050997in}{0.786081in}}%
\pgfpathlineto{\pgfqpoint{2.051309in}{1.229803in}}%
\pgfpathlineto{\pgfqpoint{2.051726in}{0.963807in}}%
\pgfpathlineto{\pgfqpoint{2.051830in}{1.327009in}}%
\pgfpathlineto{\pgfqpoint{2.052663in}{0.842927in}}%
\pgfpathlineto{\pgfqpoint{2.052871in}{1.229831in}}%
\pgfpathlineto{\pgfqpoint{2.053288in}{0.889375in}}%
\pgfpathlineto{\pgfqpoint{2.053705in}{1.346913in}}%
\pgfpathlineto{\pgfqpoint{2.054017in}{0.968941in}}%
\pgfpathlineto{\pgfqpoint{2.054121in}{1.340149in}}%
\pgfpathlineto{\pgfqpoint{2.055058in}{0.929582in}}%
\pgfpathlineto{\pgfqpoint{2.055475in}{0.891532in}}%
\pgfpathlineto{\pgfqpoint{2.056100in}{1.267230in}}%
\pgfpathlineto{\pgfqpoint{2.056621in}{0.863710in}}%
\pgfpathlineto{\pgfqpoint{2.057245in}{0.866103in}}%
\pgfpathlineto{\pgfqpoint{2.058287in}{1.221589in}}%
\pgfpathlineto{\pgfqpoint{2.058183in}{0.798380in}}%
\pgfpathlineto{\pgfqpoint{2.058391in}{1.196715in}}%
\pgfpathlineto{\pgfqpoint{2.058495in}{0.867184in}}%
\pgfpathlineto{\pgfqpoint{2.059120in}{1.348317in}}%
\pgfpathlineto{\pgfqpoint{2.059432in}{0.942180in}}%
\pgfpathlineto{\pgfqpoint{2.060161in}{1.367291in}}%
\pgfpathlineto{\pgfqpoint{2.059641in}{0.811548in}}%
\pgfpathlineto{\pgfqpoint{2.060474in}{1.012745in}}%
\pgfpathlineto{\pgfqpoint{2.060578in}{0.887424in}}%
\pgfpathlineto{\pgfqpoint{2.060682in}{1.238614in}}%
\pgfpathlineto{\pgfqpoint{2.061515in}{1.078395in}}%
\pgfpathlineto{\pgfqpoint{2.061619in}{1.220427in}}%
\pgfpathlineto{\pgfqpoint{2.062244in}{0.731607in}}%
\pgfpathlineto{\pgfqpoint{2.062557in}{1.119724in}}%
\pgfpathlineto{\pgfqpoint{2.063598in}{0.784604in}}%
\pgfpathlineto{\pgfqpoint{2.063181in}{1.321812in}}%
\pgfpathlineto{\pgfqpoint{2.063702in}{1.027853in}}%
\pgfpathlineto{\pgfqpoint{2.063806in}{1.195079in}}%
\pgfpathlineto{\pgfqpoint{2.064015in}{0.837589in}}%
\pgfpathlineto{\pgfqpoint{2.064744in}{1.165289in}}%
\pgfpathlineto{\pgfqpoint{2.065577in}{1.278440in}}%
\pgfpathlineto{\pgfqpoint{2.065785in}{0.767543in}}%
\pgfpathlineto{\pgfqpoint{2.066618in}{1.394514in}}%
\pgfpathlineto{\pgfqpoint{2.066410in}{0.729869in}}%
\pgfpathlineto{\pgfqpoint{2.066930in}{1.226418in}}%
\pgfpathlineto{\pgfqpoint{2.067868in}{0.882569in}}%
\pgfpathlineto{\pgfqpoint{2.068180in}{0.915581in}}%
\pgfpathlineto{\pgfqpoint{2.068805in}{1.250240in}}%
\pgfpathlineto{\pgfqpoint{2.069534in}{1.250217in}}%
\pgfpathlineto{\pgfqpoint{2.070575in}{0.832620in}}%
\pgfpathlineto{\pgfqpoint{2.069846in}{1.251480in}}%
\pgfpathlineto{\pgfqpoint{2.070784in}{0.919309in}}%
\pgfpathlineto{\pgfqpoint{2.070888in}{0.835363in}}%
\pgfpathlineto{\pgfqpoint{2.071200in}{1.254048in}}%
\pgfpathlineto{\pgfqpoint{2.071409in}{1.085213in}}%
\pgfpathlineto{\pgfqpoint{2.072346in}{1.296366in}}%
\pgfpathlineto{\pgfqpoint{2.071617in}{0.827356in}}%
\pgfpathlineto{\pgfqpoint{2.072450in}{1.180601in}}%
\pgfpathlineto{\pgfqpoint{2.072762in}{0.807147in}}%
\pgfpathlineto{\pgfqpoint{2.073387in}{1.273320in}}%
\pgfpathlineto{\pgfqpoint{2.073596in}{0.885493in}}%
\pgfpathlineto{\pgfqpoint{2.073908in}{1.118908in}}%
\pgfpathlineto{\pgfqpoint{2.074116in}{0.773432in}}%
\pgfpathlineto{\pgfqpoint{2.074741in}{1.080985in}}%
\pgfpathlineto{\pgfqpoint{2.075366in}{0.565209in}}%
\pgfpathlineto{\pgfqpoint{2.075053in}{1.154075in}}%
\pgfpathlineto{\pgfqpoint{2.075782in}{1.100536in}}%
\pgfpathlineto{\pgfqpoint{2.076511in}{1.233689in}}%
\pgfpathlineto{\pgfqpoint{2.076303in}{0.858027in}}%
\pgfpathlineto{\pgfqpoint{2.076616in}{1.063302in}}%
\pgfpathlineto{\pgfqpoint{2.077449in}{1.331416in}}%
\pgfpathlineto{\pgfqpoint{2.077761in}{0.740897in}}%
\pgfpathlineto{\pgfqpoint{2.078386in}{1.378437in}}%
\pgfpathlineto{\pgfqpoint{2.078907in}{1.271041in}}%
\pgfpathlineto{\pgfqpoint{2.079844in}{0.780389in}}%
\pgfpathlineto{\pgfqpoint{2.079948in}{1.383818in}}%
\pgfpathlineto{\pgfqpoint{2.080052in}{0.989600in}}%
\pgfpathlineto{\pgfqpoint{2.080365in}{1.280075in}}%
\pgfpathlineto{\pgfqpoint{2.080573in}{0.879158in}}%
\pgfpathlineto{\pgfqpoint{2.081094in}{1.146353in}}%
\pgfpathlineto{\pgfqpoint{2.081719in}{0.762841in}}%
\pgfpathlineto{\pgfqpoint{2.081510in}{1.354484in}}%
\pgfpathlineto{\pgfqpoint{2.082239in}{0.918810in}}%
\pgfpathlineto{\pgfqpoint{2.082343in}{1.409323in}}%
\pgfpathlineto{\pgfqpoint{2.082968in}{0.576836in}}%
\pgfpathlineto{\pgfqpoint{2.083385in}{1.120641in}}%
\pgfpathlineto{\pgfqpoint{2.083905in}{0.924960in}}%
\pgfpathlineto{\pgfqpoint{2.083697in}{1.231799in}}%
\pgfpathlineto{\pgfqpoint{2.084218in}{1.068964in}}%
\pgfpathlineto{\pgfqpoint{2.084322in}{1.373959in}}%
\pgfpathlineto{\pgfqpoint{2.084530in}{0.922020in}}%
\pgfpathlineto{\pgfqpoint{2.085259in}{0.994607in}}%
\pgfpathlineto{\pgfqpoint{2.085572in}{0.899817in}}%
\pgfpathlineto{\pgfqpoint{2.085468in}{1.226093in}}%
\pgfpathlineto{\pgfqpoint{2.085884in}{1.191983in}}%
\pgfpathlineto{\pgfqpoint{2.085988in}{1.233406in}}%
\pgfpathlineto{\pgfqpoint{2.086092in}{0.944436in}}%
\pgfpathlineto{\pgfqpoint{2.086509in}{1.230871in}}%
\pgfpathlineto{\pgfqpoint{2.087238in}{0.762470in}}%
\pgfpathlineto{\pgfqpoint{2.087655in}{0.841903in}}%
\pgfpathlineto{\pgfqpoint{2.088279in}{1.296766in}}%
\pgfpathlineto{\pgfqpoint{2.088175in}{0.831179in}}%
\pgfpathlineto{\pgfqpoint{2.088696in}{1.052465in}}%
\pgfpathlineto{\pgfqpoint{2.089321in}{0.640283in}}%
\pgfpathlineto{\pgfqpoint{2.089008in}{1.287609in}}%
\pgfpathlineto{\pgfqpoint{2.089737in}{1.131790in}}%
\pgfpathlineto{\pgfqpoint{2.089946in}{1.199603in}}%
\pgfpathlineto{\pgfqpoint{2.090050in}{1.119626in}}%
\pgfpathlineto{\pgfqpoint{2.090883in}{0.704194in}}%
\pgfpathlineto{\pgfqpoint{2.090362in}{1.392542in}}%
\pgfpathlineto{\pgfqpoint{2.091091in}{1.048606in}}%
\pgfpathlineto{\pgfqpoint{2.091195in}{1.242367in}}%
\pgfpathlineto{\pgfqpoint{2.091404in}{0.812369in}}%
\pgfpathlineto{\pgfqpoint{2.092237in}{1.146987in}}%
\pgfpathlineto{\pgfqpoint{2.092445in}{1.069051in}}%
\pgfpathlineto{\pgfqpoint{2.092549in}{1.156520in}}%
\pgfpathlineto{\pgfqpoint{2.092653in}{0.698956in}}%
\pgfpathlineto{\pgfqpoint{2.092966in}{1.224951in}}%
\pgfpathlineto{\pgfqpoint{2.093695in}{0.934772in}}%
\pgfpathlineto{\pgfqpoint{2.093799in}{0.940556in}}%
\pgfpathlineto{\pgfqpoint{2.094215in}{0.736825in}}%
\pgfpathlineto{\pgfqpoint{2.094736in}{1.345715in}}%
\pgfpathlineto{\pgfqpoint{2.094840in}{0.711119in}}%
\pgfpathlineto{\pgfqpoint{2.095882in}{1.036061in}}%
\pgfpathlineto{\pgfqpoint{2.096194in}{0.977396in}}%
\pgfpathlineto{\pgfqpoint{2.096090in}{1.209256in}}%
\pgfpathlineto{\pgfqpoint{2.096923in}{1.002815in}}%
\pgfpathlineto{\pgfqpoint{2.097756in}{1.201302in}}%
\pgfpathlineto{\pgfqpoint{2.097444in}{0.831351in}}%
\pgfpathlineto{\pgfqpoint{2.097860in}{1.054194in}}%
\pgfpathlineto{\pgfqpoint{2.098381in}{0.750136in}}%
\pgfpathlineto{\pgfqpoint{2.098277in}{1.163271in}}%
\pgfpathlineto{\pgfqpoint{2.099006in}{0.947027in}}%
\pgfpathlineto{\pgfqpoint{2.099631in}{1.313803in}}%
\pgfpathlineto{\pgfqpoint{2.099527in}{0.921572in}}%
\pgfpathlineto{\pgfqpoint{2.099735in}{1.195262in}}%
\pgfpathlineto{\pgfqpoint{2.099839in}{0.780794in}}%
\pgfpathlineto{\pgfqpoint{2.100672in}{1.253415in}}%
\pgfpathlineto{\pgfqpoint{2.100776in}{1.120592in}}%
\pgfpathlineto{\pgfqpoint{2.100880in}{1.120048in}}%
\pgfpathlineto{\pgfqpoint{2.100985in}{1.282927in}}%
\pgfpathlineto{\pgfqpoint{2.101714in}{0.850998in}}%
\pgfpathlineto{\pgfqpoint{2.101922in}{1.093125in}}%
\pgfpathlineto{\pgfqpoint{2.102859in}{0.913470in}}%
\pgfpathlineto{\pgfqpoint{2.102547in}{1.257336in}}%
\pgfpathlineto{\pgfqpoint{2.102963in}{1.152715in}}%
\pgfpathlineto{\pgfqpoint{2.103380in}{0.787439in}}%
\pgfpathlineto{\pgfqpoint{2.103172in}{1.242697in}}%
\pgfpathlineto{\pgfqpoint{2.104213in}{0.927082in}}%
\pgfpathlineto{\pgfqpoint{2.105150in}{1.280015in}}%
\pgfpathlineto{\pgfqpoint{2.105463in}{1.102168in}}%
\pgfpathlineto{\pgfqpoint{2.106400in}{0.984237in}}%
\pgfpathlineto{\pgfqpoint{2.106608in}{1.200508in}}%
\pgfpathlineto{\pgfqpoint{2.107025in}{1.240928in}}%
\pgfpathlineto{\pgfqpoint{2.107129in}{1.032147in}}%
\pgfpathlineto{\pgfqpoint{2.107650in}{0.790386in}}%
\pgfpathlineto{\pgfqpoint{2.107337in}{1.245250in}}%
\pgfpathlineto{\pgfqpoint{2.108066in}{1.140557in}}%
\pgfpathlineto{\pgfqpoint{2.108170in}{1.160190in}}%
\pgfpathlineto{\pgfqpoint{2.109003in}{0.839068in}}%
\pgfpathlineto{\pgfqpoint{2.109108in}{1.220553in}}%
\pgfpathlineto{\pgfqpoint{2.109212in}{1.129077in}}%
\pgfpathlineto{\pgfqpoint{2.109420in}{1.243678in}}%
\pgfpathlineto{\pgfqpoint{2.109524in}{0.932982in}}%
\pgfpathlineto{\pgfqpoint{2.109628in}{1.001538in}}%
\pgfpathlineto{\pgfqpoint{2.109837in}{1.515364in}}%
\pgfpathlineto{\pgfqpoint{2.110774in}{0.727155in}}%
\pgfpathlineto{\pgfqpoint{2.111711in}{1.411912in}}%
\pgfpathlineto{\pgfqpoint{2.112024in}{1.339329in}}%
\pgfpathlineto{\pgfqpoint{2.113065in}{0.866690in}}%
\pgfpathlineto{\pgfqpoint{2.113169in}{1.048108in}}%
\pgfpathlineto{\pgfqpoint{2.113794in}{1.328504in}}%
\pgfpathlineto{\pgfqpoint{2.114106in}{0.943351in}}%
\pgfpathlineto{\pgfqpoint{2.114211in}{1.128053in}}%
\pgfpathlineto{\pgfqpoint{2.114627in}{0.960471in}}%
\pgfpathlineto{\pgfqpoint{2.114835in}{1.146534in}}%
\pgfpathlineto{\pgfqpoint{2.115460in}{1.056810in}}%
\pgfpathlineto{\pgfqpoint{2.115564in}{1.450397in}}%
\pgfpathlineto{\pgfqpoint{2.116502in}{0.913192in}}%
\pgfpathlineto{\pgfqpoint{2.117231in}{1.468919in}}%
\pgfpathlineto{\pgfqpoint{2.117335in}{0.811275in}}%
\pgfpathlineto{\pgfqpoint{2.117751in}{1.179196in}}%
\pgfpathlineto{\pgfqpoint{2.118480in}{0.920582in}}%
\pgfpathlineto{\pgfqpoint{2.117960in}{1.252587in}}%
\pgfpathlineto{\pgfqpoint{2.118897in}{1.031586in}}%
\pgfpathlineto{\pgfqpoint{2.119105in}{1.049375in}}%
\pgfpathlineto{\pgfqpoint{2.119313in}{1.193224in}}%
\pgfpathlineto{\pgfqpoint{2.119418in}{0.944436in}}%
\pgfpathlineto{\pgfqpoint{2.119522in}{0.929878in}}%
\pgfpathlineto{\pgfqpoint{2.120147in}{1.282528in}}%
\pgfpathlineto{\pgfqpoint{2.119834in}{0.868685in}}%
\pgfpathlineto{\pgfqpoint{2.120563in}{1.193133in}}%
\pgfpathlineto{\pgfqpoint{2.121605in}{0.870848in}}%
\pgfpathlineto{\pgfqpoint{2.121292in}{1.246771in}}%
\pgfpathlineto{\pgfqpoint{2.121709in}{1.068001in}}%
\pgfpathlineto{\pgfqpoint{2.122438in}{1.310291in}}%
\pgfpathlineto{\pgfqpoint{2.122229in}{0.809846in}}%
\pgfpathlineto{\pgfqpoint{2.122750in}{1.060547in}}%
\pgfpathlineto{\pgfqpoint{2.123375in}{0.805389in}}%
\pgfpathlineto{\pgfqpoint{2.123583in}{1.082633in}}%
\pgfpathlineto{\pgfqpoint{2.123791in}{0.980560in}}%
\pgfpathlineto{\pgfqpoint{2.124000in}{1.206906in}}%
\pgfpathlineto{\pgfqpoint{2.124729in}{0.872515in}}%
\pgfpathlineto{\pgfqpoint{2.124833in}{0.992509in}}%
\pgfpathlineto{\pgfqpoint{2.124937in}{0.749448in}}%
\pgfpathlineto{\pgfqpoint{2.125041in}{1.334987in}}%
\pgfpathlineto{\pgfqpoint{2.125874in}{1.151765in}}%
\pgfpathlineto{\pgfqpoint{2.125978in}{1.167951in}}%
\pgfpathlineto{\pgfqpoint{2.126603in}{1.291489in}}%
\pgfpathlineto{\pgfqpoint{2.127124in}{0.790375in}}%
\pgfpathlineto{\pgfqpoint{2.128061in}{1.161702in}}%
\pgfpathlineto{\pgfqpoint{2.127749in}{0.699457in}}%
\pgfpathlineto{\pgfqpoint{2.128165in}{0.972646in}}%
\pgfpathlineto{\pgfqpoint{2.128270in}{0.737394in}}%
\pgfpathlineto{\pgfqpoint{2.128582in}{1.229033in}}%
\pgfpathlineto{\pgfqpoint{2.129103in}{1.164928in}}%
\pgfpathlineto{\pgfqpoint{2.129207in}{1.154600in}}%
\pgfpathlineto{\pgfqpoint{2.129728in}{0.831378in}}%
\pgfpathlineto{\pgfqpoint{2.129936in}{1.327606in}}%
\pgfpathlineto{\pgfqpoint{2.130352in}{1.005796in}}%
\pgfpathlineto{\pgfqpoint{2.130873in}{0.917160in}}%
\pgfpathlineto{\pgfqpoint{2.131394in}{1.362400in}}%
\pgfpathlineto{\pgfqpoint{2.132435in}{0.824798in}}%
\pgfpathlineto{\pgfqpoint{2.132539in}{0.985452in}}%
\pgfpathlineto{\pgfqpoint{2.132643in}{0.885791in}}%
\pgfpathlineto{\pgfqpoint{2.133060in}{1.315642in}}%
\pgfpathlineto{\pgfqpoint{2.133372in}{1.099103in}}%
\pgfpathlineto{\pgfqpoint{2.133789in}{0.801129in}}%
\pgfpathlineto{\pgfqpoint{2.134414in}{1.275588in}}%
\pgfpathlineto{\pgfqpoint{2.134830in}{0.864816in}}%
\pgfpathlineto{\pgfqpoint{2.135559in}{1.111572in}}%
\pgfpathlineto{\pgfqpoint{2.135872in}{0.520316in}}%
\pgfpathlineto{\pgfqpoint{2.135976in}{1.361410in}}%
\pgfpathlineto{\pgfqpoint{2.136705in}{1.033378in}}%
\pgfpathlineto{\pgfqpoint{2.137122in}{1.377742in}}%
\pgfpathlineto{\pgfqpoint{2.136913in}{0.887642in}}%
\pgfpathlineto{\pgfqpoint{2.137746in}{0.995526in}}%
\pgfpathlineto{\pgfqpoint{2.138371in}{1.230020in}}%
\pgfpathlineto{\pgfqpoint{2.138892in}{0.733849in}}%
\pgfpathlineto{\pgfqpoint{2.139621in}{1.301734in}}%
\pgfpathlineto{\pgfqpoint{2.140037in}{0.864700in}}%
\pgfpathlineto{\pgfqpoint{2.140662in}{1.450812in}}%
\pgfpathlineto{\pgfqpoint{2.141183in}{1.192116in}}%
\pgfpathlineto{\pgfqpoint{2.142224in}{0.886668in}}%
\pgfpathlineto{\pgfqpoint{2.141391in}{1.299311in}}%
\pgfpathlineto{\pgfqpoint{2.142433in}{1.042836in}}%
\pgfpathlineto{\pgfqpoint{2.143266in}{0.994900in}}%
\pgfpathlineto{\pgfqpoint{2.143578in}{1.385590in}}%
\pgfpathlineto{\pgfqpoint{2.143891in}{1.446378in}}%
\pgfpathlineto{\pgfqpoint{2.144724in}{0.859605in}}%
\pgfpathlineto{\pgfqpoint{2.145765in}{1.210993in}}%
\pgfpathlineto{\pgfqpoint{2.145974in}{0.829343in}}%
\pgfpathlineto{\pgfqpoint{2.146598in}{1.323569in}}%
\pgfpathlineto{\pgfqpoint{2.146807in}{1.262928in}}%
\pgfpathlineto{\pgfqpoint{2.147640in}{0.622969in}}%
\pgfpathlineto{\pgfqpoint{2.148265in}{0.913345in}}%
\pgfpathlineto{\pgfqpoint{2.148889in}{1.317306in}}%
\pgfpathlineto{\pgfqpoint{2.148681in}{0.838872in}}%
\pgfpathlineto{\pgfqpoint{2.149306in}{1.184409in}}%
\pgfpathlineto{\pgfqpoint{2.150243in}{0.697143in}}%
\pgfpathlineto{\pgfqpoint{2.150347in}{1.213942in}}%
\pgfpathlineto{\pgfqpoint{2.151076in}{0.599752in}}%
\pgfpathlineto{\pgfqpoint{2.151389in}{1.025343in}}%
\pgfpathlineto{\pgfqpoint{2.151805in}{0.772812in}}%
\pgfpathlineto{\pgfqpoint{2.152534in}{1.443031in}}%
\pgfpathlineto{\pgfqpoint{2.153784in}{0.750497in}}%
\pgfpathlineto{\pgfqpoint{2.154721in}{1.276119in}}%
\pgfpathlineto{\pgfqpoint{2.155138in}{1.189881in}}%
\pgfpathlineto{\pgfqpoint{2.155346in}{0.812515in}}%
\pgfpathlineto{\pgfqpoint{2.156283in}{0.941216in}}%
\pgfpathlineto{\pgfqpoint{2.156492in}{0.817288in}}%
\pgfpathlineto{\pgfqpoint{2.156596in}{1.064374in}}%
\pgfpathlineto{\pgfqpoint{2.156700in}{0.970652in}}%
\pgfpathlineto{\pgfqpoint{2.157117in}{1.320409in}}%
\pgfpathlineto{\pgfqpoint{2.156908in}{0.841610in}}%
\pgfpathlineto{\pgfqpoint{2.157846in}{1.073802in}}%
\pgfpathlineto{\pgfqpoint{2.158783in}{0.790644in}}%
\pgfpathlineto{\pgfqpoint{2.158158in}{1.224146in}}%
\pgfpathlineto{\pgfqpoint{2.158887in}{0.889503in}}%
\pgfpathlineto{\pgfqpoint{2.158991in}{1.184603in}}%
\pgfpathlineto{\pgfqpoint{2.159199in}{0.826750in}}%
\pgfpathlineto{\pgfqpoint{2.160033in}{1.056777in}}%
\pgfpathlineto{\pgfqpoint{2.160241in}{1.296725in}}%
\pgfpathlineto{\pgfqpoint{2.160866in}{0.857995in}}%
\pgfpathlineto{\pgfqpoint{2.160970in}{1.325512in}}%
\pgfpathlineto{\pgfqpoint{2.162011in}{1.082902in}}%
\pgfpathlineto{\pgfqpoint{2.162324in}{0.931729in}}%
\pgfpathlineto{\pgfqpoint{2.162949in}{1.258878in}}%
\pgfpathlineto{\pgfqpoint{2.163053in}{0.933972in}}%
\pgfpathlineto{\pgfqpoint{2.163678in}{1.228706in}}%
\pgfpathlineto{\pgfqpoint{2.164094in}{1.070719in}}%
\pgfpathlineto{\pgfqpoint{2.164198in}{0.878592in}}%
\pgfpathlineto{\pgfqpoint{2.164511in}{1.320664in}}%
\pgfpathlineto{\pgfqpoint{2.165240in}{0.959782in}}%
\pgfpathlineto{\pgfqpoint{2.165864in}{1.324085in}}%
\pgfpathlineto{\pgfqpoint{2.165552in}{0.762067in}}%
\pgfpathlineto{\pgfqpoint{2.166385in}{1.294763in}}%
\pgfpathlineto{\pgfqpoint{2.166906in}{0.826797in}}%
\pgfpathlineto{\pgfqpoint{2.167635in}{0.965017in}}%
\pgfpathlineto{\pgfqpoint{2.167739in}{0.967645in}}%
\pgfpathlineto{\pgfqpoint{2.168051in}{0.907293in}}%
\pgfpathlineto{\pgfqpoint{2.168156in}{1.213274in}}%
\pgfpathlineto{\pgfqpoint{2.168364in}{0.925081in}}%
\pgfpathlineto{\pgfqpoint{2.168780in}{1.290514in}}%
\pgfpathlineto{\pgfqpoint{2.168885in}{0.806644in}}%
\pgfpathlineto{\pgfqpoint{2.169718in}{1.476250in}}%
\pgfpathlineto{\pgfqpoint{2.169926in}{1.070573in}}%
\pgfpathlineto{\pgfqpoint{2.170343in}{0.776827in}}%
\pgfpathlineto{\pgfqpoint{2.170655in}{1.339663in}}%
\pgfpathlineto{\pgfqpoint{2.170967in}{1.031297in}}%
\pgfpathlineto{\pgfqpoint{2.171072in}{1.041460in}}%
\pgfpathlineto{\pgfqpoint{2.171488in}{0.897567in}}%
\pgfpathlineto{\pgfqpoint{2.172217in}{1.301752in}}%
\pgfpathlineto{\pgfqpoint{2.173363in}{0.719092in}}%
\pgfpathlineto{\pgfqpoint{2.172634in}{1.330821in}}%
\pgfpathlineto{\pgfqpoint{2.173467in}{0.890887in}}%
\pgfpathlineto{\pgfqpoint{2.174716in}{1.340146in}}%
\pgfpathlineto{\pgfqpoint{2.174508in}{0.837466in}}%
\pgfpathlineto{\pgfqpoint{2.174821in}{1.218937in}}%
\pgfpathlineto{\pgfqpoint{2.175341in}{0.715048in}}%
\pgfpathlineto{\pgfqpoint{2.175966in}{1.052836in}}%
\pgfpathlineto{\pgfqpoint{2.176487in}{0.634706in}}%
\pgfpathlineto{\pgfqpoint{2.176695in}{1.189159in}}%
\pgfpathlineto{\pgfqpoint{2.176903in}{1.159383in}}%
\pgfpathlineto{\pgfqpoint{2.177008in}{1.223898in}}%
\pgfpathlineto{\pgfqpoint{2.177112in}{0.568382in}}%
\pgfpathlineto{\pgfqpoint{2.177737in}{1.336846in}}%
\pgfpathlineto{\pgfqpoint{2.178153in}{0.790472in}}%
\pgfpathlineto{\pgfqpoint{2.178466in}{1.318666in}}%
\pgfpathlineto{\pgfqpoint{2.178361in}{0.716479in}}%
\pgfpathlineto{\pgfqpoint{2.179611in}{1.070405in}}%
\pgfpathlineto{\pgfqpoint{2.179715in}{0.837149in}}%
\pgfpathlineto{\pgfqpoint{2.179924in}{1.213193in}}%
\pgfpathlineto{\pgfqpoint{2.180652in}{0.956490in}}%
\pgfpathlineto{\pgfqpoint{2.181694in}{1.368843in}}%
\pgfpathlineto{\pgfqpoint{2.181173in}{0.699870in}}%
\pgfpathlineto{\pgfqpoint{2.181798in}{1.265585in}}%
\pgfpathlineto{\pgfqpoint{2.182423in}{0.903657in}}%
\pgfpathlineto{\pgfqpoint{2.182944in}{0.940589in}}%
\pgfpathlineto{\pgfqpoint{2.183360in}{1.425280in}}%
\pgfpathlineto{\pgfqpoint{2.184089in}{1.415881in}}%
\pgfpathlineto{\pgfqpoint{2.184818in}{0.685527in}}%
\pgfpathlineto{\pgfqpoint{2.185131in}{1.160506in}}%
\pgfpathlineto{\pgfqpoint{2.185235in}{1.349486in}}%
\pgfpathlineto{\pgfqpoint{2.185547in}{0.882800in}}%
\pgfpathlineto{\pgfqpoint{2.186172in}{1.060519in}}%
\pgfpathlineto{\pgfqpoint{2.186484in}{0.697344in}}%
\pgfpathlineto{\pgfqpoint{2.186901in}{1.128233in}}%
\pgfpathlineto{\pgfqpoint{2.187318in}{0.847937in}}%
\pgfpathlineto{\pgfqpoint{2.188255in}{1.186458in}}%
\pgfpathlineto{\pgfqpoint{2.188359in}{1.156094in}}%
\pgfpathlineto{\pgfqpoint{2.188463in}{0.744889in}}%
\pgfpathlineto{\pgfqpoint{2.188671in}{1.278466in}}%
\pgfpathlineto{\pgfqpoint{2.189504in}{0.913347in}}%
\pgfpathlineto{\pgfqpoint{2.190129in}{1.403747in}}%
\pgfpathlineto{\pgfqpoint{2.190858in}{1.169267in}}%
\pgfpathlineto{\pgfqpoint{2.190962in}{0.848850in}}%
\pgfpathlineto{\pgfqpoint{2.191691in}{1.226566in}}%
\pgfpathlineto{\pgfqpoint{2.191900in}{1.100894in}}%
\pgfpathlineto{\pgfqpoint{2.192004in}{1.114917in}}%
\pgfpathlineto{\pgfqpoint{2.192837in}{0.854783in}}%
\pgfpathlineto{\pgfqpoint{2.192941in}{1.287852in}}%
\pgfpathlineto{\pgfqpoint{2.193045in}{1.133881in}}%
\pgfpathlineto{\pgfqpoint{2.194191in}{0.900018in}}%
\pgfpathlineto{\pgfqpoint{2.193358in}{1.187460in}}%
\pgfpathlineto{\pgfqpoint{2.194295in}{0.988565in}}%
\pgfpathlineto{\pgfqpoint{2.194607in}{1.255506in}}%
\pgfpathlineto{\pgfqpoint{2.194712in}{0.701428in}}%
\pgfpathlineto{\pgfqpoint{2.194920in}{0.979337in}}%
\pgfpathlineto{\pgfqpoint{2.195024in}{0.636517in}}%
\pgfpathlineto{\pgfqpoint{2.195441in}{1.285196in}}%
\pgfpathlineto{\pgfqpoint{2.196065in}{0.849370in}}%
\pgfpathlineto{\pgfqpoint{2.196690in}{1.271878in}}%
\pgfpathlineto{\pgfqpoint{2.197211in}{1.075576in}}%
\pgfpathlineto{\pgfqpoint{2.197836in}{1.199083in}}%
\pgfpathlineto{\pgfqpoint{2.197627in}{0.927873in}}%
\pgfpathlineto{\pgfqpoint{2.198148in}{1.162414in}}%
\pgfpathlineto{\pgfqpoint{2.199190in}{0.869381in}}%
\pgfpathlineto{\pgfqpoint{2.198565in}{1.275858in}}%
\pgfpathlineto{\pgfqpoint{2.199398in}{0.936126in}}%
\pgfpathlineto{\pgfqpoint{2.199814in}{1.305635in}}%
\pgfpathlineto{\pgfqpoint{2.199710in}{0.873989in}}%
\pgfpathlineto{\pgfqpoint{2.200439in}{0.994799in}}%
\pgfpathlineto{\pgfqpoint{2.200856in}{0.843188in}}%
\pgfpathlineto{\pgfqpoint{2.200648in}{1.160494in}}%
\pgfpathlineto{\pgfqpoint{2.201168in}{0.884806in}}%
\pgfpathlineto{\pgfqpoint{2.201481in}{0.713984in}}%
\pgfpathlineto{\pgfqpoint{2.202210in}{1.270948in}}%
\pgfpathlineto{\pgfqpoint{2.203355in}{0.790247in}}%
\pgfpathlineto{\pgfqpoint{2.203980in}{1.342172in}}%
\pgfpathlineto{\pgfqpoint{2.204501in}{1.169787in}}%
\pgfpathlineto{\pgfqpoint{2.204605in}{1.191375in}}%
\pgfpathlineto{\pgfqpoint{2.204917in}{0.825255in}}%
\pgfpathlineto{\pgfqpoint{2.205230in}{1.336861in}}%
\pgfpathlineto{\pgfqpoint{2.205750in}{0.881109in}}%
\pgfpathlineto{\pgfqpoint{2.206792in}{1.711955in}}%
\pgfpathlineto{\pgfqpoint{2.207000in}{1.180629in}}%
\pgfpathlineto{\pgfqpoint{2.207625in}{0.836956in}}%
\pgfpathlineto{\pgfqpoint{2.207417in}{1.330500in}}%
\pgfpathlineto{\pgfqpoint{2.208146in}{0.970931in}}%
\pgfpathlineto{\pgfqpoint{2.208250in}{1.358546in}}%
\pgfpathlineto{\pgfqpoint{2.208875in}{0.759103in}}%
\pgfpathlineto{\pgfqpoint{2.209187in}{1.084497in}}%
\pgfpathlineto{\pgfqpoint{2.209604in}{0.876103in}}%
\pgfpathlineto{\pgfqpoint{2.209500in}{1.131077in}}%
\pgfpathlineto{\pgfqpoint{2.209916in}{1.083539in}}%
\pgfpathlineto{\pgfqpoint{2.210124in}{0.874359in}}%
\pgfpathlineto{\pgfqpoint{2.210958in}{1.214546in}}%
\pgfpathlineto{\pgfqpoint{2.211999in}{0.865110in}}%
\pgfpathlineto{\pgfqpoint{2.211270in}{1.344317in}}%
\pgfpathlineto{\pgfqpoint{2.212103in}{0.939616in}}%
\pgfpathlineto{\pgfqpoint{2.212207in}{0.965370in}}%
\pgfpathlineto{\pgfqpoint{2.212311in}{1.383220in}}%
\pgfpathlineto{\pgfqpoint{2.213145in}{0.851450in}}%
\pgfpathlineto{\pgfqpoint{2.213353in}{1.254678in}}%
\pgfpathlineto{\pgfqpoint{2.214394in}{0.820640in}}%
\pgfpathlineto{\pgfqpoint{2.214498in}{1.051620in}}%
\pgfpathlineto{\pgfqpoint{2.215436in}{0.912821in}}%
\pgfpathlineto{\pgfqpoint{2.215019in}{1.257147in}}%
\pgfpathlineto{\pgfqpoint{2.215540in}{1.066217in}}%
\pgfpathlineto{\pgfqpoint{2.215852in}{0.881506in}}%
\pgfpathlineto{\pgfqpoint{2.216165in}{1.155259in}}%
\pgfpathlineto{\pgfqpoint{2.216269in}{1.151846in}}%
\pgfpathlineto{\pgfqpoint{2.217206in}{1.284068in}}%
\pgfpathlineto{\pgfqpoint{2.216477in}{0.869234in}}%
\pgfpathlineto{\pgfqpoint{2.217310in}{1.134777in}}%
\pgfpathlineto{\pgfqpoint{2.217727in}{0.917217in}}%
\pgfpathlineto{\pgfqpoint{2.217831in}{1.327149in}}%
\pgfpathlineto{\pgfqpoint{2.218352in}{1.130136in}}%
\pgfpathlineto{\pgfqpoint{2.218456in}{1.146578in}}%
\pgfpathlineto{\pgfqpoint{2.219185in}{0.823113in}}%
\pgfpathlineto{\pgfqpoint{2.219497in}{1.233140in}}%
\pgfpathlineto{\pgfqpoint{2.219601in}{0.996478in}}%
\pgfpathlineto{\pgfqpoint{2.220330in}{1.457848in}}%
\pgfpathlineto{\pgfqpoint{2.220122in}{0.897320in}}%
\pgfpathlineto{\pgfqpoint{2.220643in}{1.169967in}}%
\pgfpathlineto{\pgfqpoint{2.221059in}{0.760023in}}%
\pgfpathlineto{\pgfqpoint{2.221268in}{1.231489in}}%
\pgfpathlineto{\pgfqpoint{2.221788in}{0.874678in}}%
\pgfpathlineto{\pgfqpoint{2.222725in}{1.228983in}}%
\pgfpathlineto{\pgfqpoint{2.222101in}{0.871528in}}%
\pgfpathlineto{\pgfqpoint{2.222934in}{1.150316in}}%
\pgfpathlineto{\pgfqpoint{2.223142in}{1.183303in}}%
\pgfpathlineto{\pgfqpoint{2.223246in}{1.059943in}}%
\pgfpathlineto{\pgfqpoint{2.223350in}{0.920189in}}%
\pgfpathlineto{\pgfqpoint{2.223663in}{1.321225in}}%
\pgfpathlineto{\pgfqpoint{2.224183in}{1.276717in}}%
\pgfpathlineto{\pgfqpoint{2.224392in}{0.871877in}}%
\pgfpathlineto{\pgfqpoint{2.225225in}{1.056793in}}%
\pgfpathlineto{\pgfqpoint{2.225329in}{1.417688in}}%
\pgfpathlineto{\pgfqpoint{2.225746in}{0.911002in}}%
\pgfpathlineto{\pgfqpoint{2.226266in}{1.147141in}}%
\pgfpathlineto{\pgfqpoint{2.226370in}{0.881026in}}%
\pgfpathlineto{\pgfqpoint{2.226579in}{1.238575in}}%
\pgfpathlineto{\pgfqpoint{2.227308in}{1.082021in}}%
\pgfpathlineto{\pgfqpoint{2.227412in}{1.205806in}}%
\pgfpathlineto{\pgfqpoint{2.228037in}{0.860823in}}%
\pgfpathlineto{\pgfqpoint{2.228141in}{1.153147in}}%
\pgfpathlineto{\pgfqpoint{2.228557in}{0.826905in}}%
\pgfpathlineto{\pgfqpoint{2.228349in}{1.242879in}}%
\pgfpathlineto{\pgfqpoint{2.229286in}{0.916557in}}%
\pgfpathlineto{\pgfqpoint{2.229599in}{1.301634in}}%
\pgfpathlineto{\pgfqpoint{2.229703in}{0.757530in}}%
\pgfpathlineto{\pgfqpoint{2.230328in}{1.140007in}}%
\pgfpathlineto{\pgfqpoint{2.230536in}{0.850290in}}%
\pgfpathlineto{\pgfqpoint{2.230953in}{1.169081in}}%
\pgfpathlineto{\pgfqpoint{2.231473in}{1.023282in}}%
\pgfpathlineto{\pgfqpoint{2.232411in}{0.788761in}}%
\pgfpathlineto{\pgfqpoint{2.231994in}{1.242334in}}%
\pgfpathlineto{\pgfqpoint{2.232515in}{1.075335in}}%
\pgfpathlineto{\pgfqpoint{2.233244in}{1.417925in}}%
\pgfpathlineto{\pgfqpoint{2.233348in}{0.871691in}}%
\pgfpathlineto{\pgfqpoint{2.233452in}{1.131751in}}%
\pgfpathlineto{\pgfqpoint{2.234389in}{0.967460in}}%
\pgfpathlineto{\pgfqpoint{2.234077in}{1.380370in}}%
\pgfpathlineto{\pgfqpoint{2.234598in}{1.028224in}}%
\pgfpathlineto{\pgfqpoint{2.235014in}{1.279394in}}%
\pgfpathlineto{\pgfqpoint{2.235535in}{0.934866in}}%
\pgfpathlineto{\pgfqpoint{2.235639in}{1.193562in}}%
\pgfpathlineto{\pgfqpoint{2.235743in}{0.876214in}}%
\pgfpathlineto{\pgfqpoint{2.236576in}{1.268702in}}%
\pgfpathlineto{\pgfqpoint{2.236785in}{0.883993in}}%
\pgfpathlineto{\pgfqpoint{2.236889in}{0.899600in}}%
\pgfpathlineto{\pgfqpoint{2.236993in}{0.747702in}}%
\pgfpathlineto{\pgfqpoint{2.237514in}{1.126529in}}%
\pgfpathlineto{\pgfqpoint{2.237930in}{0.922221in}}%
\pgfpathlineto{\pgfqpoint{2.238347in}{1.314248in}}%
\pgfpathlineto{\pgfqpoint{2.238659in}{0.890994in}}%
\pgfpathlineto{\pgfqpoint{2.239180in}{1.231638in}}%
\pgfpathlineto{\pgfqpoint{2.240325in}{0.645362in}}%
\pgfpathlineto{\pgfqpoint{2.241575in}{1.307645in}}%
\pgfpathlineto{\pgfqpoint{2.242200in}{0.943150in}}%
\pgfpathlineto{\pgfqpoint{2.242721in}{1.053190in}}%
\pgfpathlineto{\pgfqpoint{2.243345in}{1.322717in}}%
\pgfpathlineto{\pgfqpoint{2.243033in}{0.832685in}}%
\pgfpathlineto{\pgfqpoint{2.243658in}{0.975404in}}%
\pgfpathlineto{\pgfqpoint{2.243762in}{0.972805in}}%
\pgfpathlineto{\pgfqpoint{2.244283in}{1.458501in}}%
\pgfpathlineto{\pgfqpoint{2.244179in}{0.936166in}}%
\pgfpathlineto{\pgfqpoint{2.244803in}{1.298381in}}%
\pgfpathlineto{\pgfqpoint{2.245637in}{0.847263in}}%
\pgfpathlineto{\pgfqpoint{2.245324in}{1.333458in}}%
\pgfpathlineto{\pgfqpoint{2.245949in}{1.128133in}}%
\pgfpathlineto{\pgfqpoint{2.247094in}{0.886129in}}%
\pgfpathlineto{\pgfqpoint{2.246678in}{1.321455in}}%
\pgfpathlineto{\pgfqpoint{2.247199in}{0.954498in}}%
\pgfpathlineto{\pgfqpoint{2.248136in}{1.349193in}}%
\pgfpathlineto{\pgfqpoint{2.247615in}{0.835382in}}%
\pgfpathlineto{\pgfqpoint{2.248240in}{1.038777in}}%
\pgfpathlineto{\pgfqpoint{2.248344in}{1.033821in}}%
\pgfpathlineto{\pgfqpoint{2.248448in}{0.677124in}}%
\pgfpathlineto{\pgfqpoint{2.248761in}{1.283849in}}%
\pgfpathlineto{\pgfqpoint{2.249386in}{1.090487in}}%
\pgfpathlineto{\pgfqpoint{2.250219in}{1.288800in}}%
\pgfpathlineto{\pgfqpoint{2.249906in}{0.676726in}}%
\pgfpathlineto{\pgfqpoint{2.250323in}{1.180074in}}%
\pgfpathlineto{\pgfqpoint{2.250948in}{1.216534in}}%
\pgfpathlineto{\pgfqpoint{2.251573in}{0.857863in}}%
\pgfpathlineto{\pgfqpoint{2.252093in}{1.277481in}}%
\pgfpathlineto{\pgfqpoint{2.252718in}{1.048082in}}%
\pgfpathlineto{\pgfqpoint{2.253239in}{1.304375in}}%
\pgfpathlineto{\pgfqpoint{2.253760in}{0.841685in}}%
\pgfpathlineto{\pgfqpoint{2.253864in}{1.267647in}}%
\pgfpathlineto{\pgfqpoint{2.254176in}{0.819171in}}%
\pgfpathlineto{\pgfqpoint{2.254905in}{1.039398in}}%
\pgfpathlineto{\pgfqpoint{2.255634in}{0.760890in}}%
\pgfpathlineto{\pgfqpoint{2.255217in}{1.454870in}}%
\pgfpathlineto{\pgfqpoint{2.255842in}{1.079777in}}%
\pgfpathlineto{\pgfqpoint{2.256363in}{1.274311in}}%
\pgfpathlineto{\pgfqpoint{2.256571in}{0.871138in}}%
\pgfpathlineto{\pgfqpoint{2.256988in}{1.167522in}}%
\pgfpathlineto{\pgfqpoint{2.257092in}{1.176316in}}%
\pgfpathlineto{\pgfqpoint{2.258029in}{0.649650in}}%
\pgfpathlineto{\pgfqpoint{2.257404in}{1.274303in}}%
\pgfpathlineto{\pgfqpoint{2.258238in}{0.992976in}}%
\pgfpathlineto{\pgfqpoint{2.258654in}{1.258594in}}%
\pgfpathlineto{\pgfqpoint{2.258446in}{0.920915in}}%
\pgfpathlineto{\pgfqpoint{2.259487in}{1.148398in}}%
\pgfpathlineto{\pgfqpoint{2.259591in}{0.677102in}}%
\pgfpathlineto{\pgfqpoint{2.260529in}{1.202392in}}%
\pgfpathlineto{\pgfqpoint{2.260633in}{0.913415in}}%
\pgfpathlineto{\pgfqpoint{2.260945in}{0.698574in}}%
\pgfpathlineto{\pgfqpoint{2.261987in}{1.221856in}}%
\pgfpathlineto{\pgfqpoint{2.262507in}{0.868053in}}%
\pgfpathlineto{\pgfqpoint{2.262299in}{1.294224in}}%
\pgfpathlineto{\pgfqpoint{2.263028in}{1.093974in}}%
\pgfusepath{stroke}%
\end{pgfscope}%
\begin{pgfscope}%
\pgfsetrectcap%
\pgfsetmiterjoin%
\pgfsetlinewidth{0.803000pt}%
\definecolor{currentstroke}{rgb}{0.000000,0.000000,0.000000}%
\pgfsetstrokecolor{currentstroke}%
\pgfsetdash{}{0pt}%
\pgfpathmoveto{\pgfqpoint{0.471688in}{0.416447in}}%
\pgfpathlineto{\pgfqpoint{0.471688in}{1.773646in}}%
\pgfusepath{stroke}%
\end{pgfscope}%
\begin{pgfscope}%
\pgfsetrectcap%
\pgfsetmiterjoin%
\pgfsetlinewidth{0.803000pt}%
\definecolor{currentstroke}{rgb}{0.000000,0.000000,0.000000}%
\pgfsetstrokecolor{currentstroke}%
\pgfsetdash{}{0pt}%
\pgfpathmoveto{\pgfqpoint{2.348330in}{0.416447in}}%
\pgfpathlineto{\pgfqpoint{2.348330in}{1.773646in}}%
\pgfusepath{stroke}%
\end{pgfscope}%
\begin{pgfscope}%
\pgfsetrectcap%
\pgfsetmiterjoin%
\pgfsetlinewidth{0.803000pt}%
\definecolor{currentstroke}{rgb}{0.000000,0.000000,0.000000}%
\pgfsetstrokecolor{currentstroke}%
\pgfsetdash{}{0pt}%
\pgfpathmoveto{\pgfqpoint{0.471688in}{0.416447in}}%
\pgfpathlineto{\pgfqpoint{2.348330in}{0.416447in}}%
\pgfusepath{stroke}%
\end{pgfscope}%
\begin{pgfscope}%
\pgfsetrectcap%
\pgfsetmiterjoin%
\pgfsetlinewidth{0.803000pt}%
\definecolor{currentstroke}{rgb}{0.000000,0.000000,0.000000}%
\pgfsetstrokecolor{currentstroke}%
\pgfsetdash{}{0pt}%
\pgfpathmoveto{\pgfqpoint{0.471688in}{1.773646in}}%
\pgfpathlineto{\pgfqpoint{2.348330in}{1.773646in}}%
\pgfusepath{stroke}%
\end{pgfscope}%
\begin{pgfscope}%
\pgfsetbuttcap%
\pgfsetmiterjoin%
\definecolor{currentfill}{rgb}{1.000000,1.000000,1.000000}%
\pgfsetfillcolor{currentfill}%
\pgfsetfillopacity{0.800000}%
\pgfsetlinewidth{1.003750pt}%
\definecolor{currentstroke}{rgb}{0.800000,0.800000,0.800000}%
\pgfsetstrokecolor{currentstroke}%
\pgfsetstrokeopacity{0.800000}%
\pgfsetdash{}{0pt}%
\pgfpathmoveto{\pgfqpoint{0.549465in}{1.529869in}}%
\pgfpathlineto{\pgfqpoint{1.518576in}{1.529869in}}%
\pgfpathquadraticcurveto{\pgfqpoint{1.540799in}{1.529869in}}{\pgfqpoint{1.540799in}{1.552091in}}%
\pgfpathlineto{\pgfqpoint{1.540799in}{1.695868in}}%
\pgfpathquadraticcurveto{\pgfqpoint{1.540799in}{1.718091in}}{\pgfqpoint{1.518576in}{1.718091in}}%
\pgfpathlineto{\pgfqpoint{0.549465in}{1.718091in}}%
\pgfpathquadraticcurveto{\pgfqpoint{0.527243in}{1.718091in}}{\pgfqpoint{0.527243in}{1.695868in}}%
\pgfpathlineto{\pgfqpoint{0.527243in}{1.552091in}}%
\pgfpathquadraticcurveto{\pgfqpoint{0.527243in}{1.529869in}}{\pgfqpoint{0.549465in}{1.529869in}}%
\pgfpathlineto{\pgfqpoint{0.549465in}{1.529869in}}%
\pgfpathclose%
\pgfusepath{stroke,fill}%
\end{pgfscope}%
\begin{pgfscope}%
\pgfsetrectcap%
\pgfsetroundjoin%
\pgfsetlinewidth{1.505625pt}%
\definecolor{currentstroke}{rgb}{0.000000,0.447059,0.698039}%
\pgfsetstrokecolor{currentstroke}%
\pgfsetdash{}{0pt}%
\pgfpathmoveto{\pgfqpoint{0.571688in}{1.634757in}}%
\pgfpathlineto{\pgfqpoint{0.682799in}{1.634757in}}%
\pgfpathlineto{\pgfqpoint{0.793910in}{1.634757in}}%
\pgfusepath{stroke}%
\end{pgfscope}%
\begin{pgfscope}%
\definecolor{textcolor}{rgb}{0.000000,0.000000,0.000000}%
\pgfsetstrokecolor{textcolor}%
\pgfsetfillcolor{textcolor}%
\pgftext[x=0.882799in,y=1.595868in,left,base]{\color{textcolor}\rmfamily\fontsize{8.000000}{9.600000}\selectfont White noise}%
\end{pgfscope}%
\end{pgfpicture}%
\makeatother%
\endgroup%

        } % scalebox
        \caption{Time domain}
        \label{fig:white_noise_time}
    \end{subfigure}
    \hfill
    \begin{subfigure}{0.32\linewidth}
        \centering
        \scalebox{0.75}{%
            %% Creator: Matplotlib, PGF backend
%%
%% To include the figure in your LaTeX document, write
%%   \input{<filename>.pgf}
%%
%% Make sure the required packages are loaded in your preamble
%%   \usepackage{pgf}
%%
%% Also ensure that all the required font packages are loaded; for instance,
%% the lmodern package is sometimes necessary when using math font.
%%   \usepackage{lmodern}
%%
%% Figures using additional raster images can only be included by \input if
%% they are in the same directory as the main LaTeX file. For loading figures
%% from other directories you can use the `import` package
%%   \usepackage{import}
%%
%% and then include the figures with
%%   \import{<path to file>}{<filename>.pgf}
%%
%% Matplotlib used the following preamble
%%   \usepackage{siunitx}
%%   \usepackage{fontspec}
%%   \makeatletter\@ifpackageloaded{underscore}{}{\usepackage[strings]{underscore}}\makeatother
%%
\begingroup%
\makeatletter%
\begin{pgfpicture}%
\pgfpathrectangle{\pgfpointorigin}{\pgfqpoint{2.440000in}{1.830000in}}%
\pgfusepath{use as bounding box, clip}%
\begin{pgfscope}%
\pgfsetbuttcap%
\pgfsetmiterjoin%
\definecolor{currentfill}{rgb}{1.000000,1.000000,1.000000}%
\pgfsetfillcolor{currentfill}%
\pgfsetlinewidth{0.000000pt}%
\definecolor{currentstroke}{rgb}{1.000000,1.000000,1.000000}%
\pgfsetstrokecolor{currentstroke}%
\pgfsetdash{}{0pt}%
\pgfpathmoveto{\pgfqpoint{0.000000in}{0.000000in}}%
\pgfpathlineto{\pgfqpoint{2.440000in}{0.000000in}}%
\pgfpathlineto{\pgfqpoint{2.440000in}{1.830000in}}%
\pgfpathlineto{\pgfqpoint{0.000000in}{1.830000in}}%
\pgfpathlineto{\pgfqpoint{0.000000in}{0.000000in}}%
\pgfpathclose%
\pgfusepath{fill}%
\end{pgfscope}%
\begin{pgfscope}%
\pgfsetbuttcap%
\pgfsetmiterjoin%
\definecolor{currentfill}{rgb}{1.000000,1.000000,1.000000}%
\pgfsetfillcolor{currentfill}%
\pgfsetlinewidth{0.000000pt}%
\definecolor{currentstroke}{rgb}{0.000000,0.000000,0.000000}%
\pgfsetstrokecolor{currentstroke}%
\pgfsetstrokeopacity{0.000000}%
\pgfsetdash{}{0pt}%
\pgfpathmoveto{\pgfqpoint{0.514278in}{0.417642in}}%
\pgfpathlineto{\pgfqpoint{2.398330in}{0.417642in}}%
\pgfpathlineto{\pgfqpoint{2.398330in}{1.788330in}}%
\pgfpathlineto{\pgfqpoint{0.514278in}{1.788330in}}%
\pgfpathlineto{\pgfqpoint{0.514278in}{0.417642in}}%
\pgfpathclose%
\pgfusepath{fill}%
\end{pgfscope}%
\begin{pgfscope}%
\pgfpathrectangle{\pgfqpoint{0.514278in}{0.417642in}}{\pgfqpoint{1.884052in}{1.370688in}}%
\pgfusepath{clip}%
\pgfsetrectcap%
\pgfsetroundjoin%
\pgfsetlinewidth{0.803000pt}%
\definecolor{currentstroke}{rgb}{0.450000,0.450000,0.450000}%
\pgfsetstrokecolor{currentstroke}%
\pgfsetdash{}{0pt}%
\pgfpathmoveto{\pgfqpoint{0.916624in}{0.417642in}}%
\pgfpathlineto{\pgfqpoint{0.916624in}{1.788330in}}%
\pgfusepath{stroke}%
\end{pgfscope}%
\begin{pgfscope}%
\pgfsetbuttcap%
\pgfsetroundjoin%
\definecolor{currentfill}{rgb}{0.000000,0.000000,0.000000}%
\pgfsetfillcolor{currentfill}%
\pgfsetlinewidth{0.803000pt}%
\definecolor{currentstroke}{rgb}{0.000000,0.000000,0.000000}%
\pgfsetstrokecolor{currentstroke}%
\pgfsetdash{}{0pt}%
\pgfsys@defobject{currentmarker}{\pgfqpoint{0.000000in}{-0.048611in}}{\pgfqpoint{0.000000in}{0.000000in}}{%
\pgfpathmoveto{\pgfqpoint{0.000000in}{0.000000in}}%
\pgfpathlineto{\pgfqpoint{0.000000in}{-0.048611in}}%
\pgfusepath{stroke,fill}%
}%
\begin{pgfscope}%
\pgfsys@transformshift{0.916624in}{0.417642in}%
\pgfsys@useobject{currentmarker}{}%
\end{pgfscope}%
\end{pgfscope}%
\begin{pgfscope}%
\definecolor{textcolor}{rgb}{0.000000,0.000000,0.000000}%
\pgfsetstrokecolor{textcolor}%
\pgfsetfillcolor{textcolor}%
\pgftext[x=0.916624in,y=0.320420in,,top]{\color{textcolor}\rmfamily\fontsize{8.000000}{9.600000}\selectfont \(\displaystyle {10^{-3}}\)}%
\end{pgfscope}%
\begin{pgfscope}%
\pgfpathrectangle{\pgfqpoint{0.514278in}{0.417642in}}{\pgfqpoint{1.884052in}{1.370688in}}%
\pgfusepath{clip}%
\pgfsetrectcap%
\pgfsetroundjoin%
\pgfsetlinewidth{0.803000pt}%
\definecolor{currentstroke}{rgb}{0.450000,0.450000,0.450000}%
\pgfsetstrokecolor{currentstroke}%
\pgfsetdash{}{0pt}%
\pgfpathmoveto{\pgfqpoint{1.433903in}{0.417642in}}%
\pgfpathlineto{\pgfqpoint{1.433903in}{1.788330in}}%
\pgfusepath{stroke}%
\end{pgfscope}%
\begin{pgfscope}%
\pgfsetbuttcap%
\pgfsetroundjoin%
\definecolor{currentfill}{rgb}{0.000000,0.000000,0.000000}%
\pgfsetfillcolor{currentfill}%
\pgfsetlinewidth{0.803000pt}%
\definecolor{currentstroke}{rgb}{0.000000,0.000000,0.000000}%
\pgfsetstrokecolor{currentstroke}%
\pgfsetdash{}{0pt}%
\pgfsys@defobject{currentmarker}{\pgfqpoint{0.000000in}{-0.048611in}}{\pgfqpoint{0.000000in}{0.000000in}}{%
\pgfpathmoveto{\pgfqpoint{0.000000in}{0.000000in}}%
\pgfpathlineto{\pgfqpoint{0.000000in}{-0.048611in}}%
\pgfusepath{stroke,fill}%
}%
\begin{pgfscope}%
\pgfsys@transformshift{1.433903in}{0.417642in}%
\pgfsys@useobject{currentmarker}{}%
\end{pgfscope}%
\end{pgfscope}%
\begin{pgfscope}%
\definecolor{textcolor}{rgb}{0.000000,0.000000,0.000000}%
\pgfsetstrokecolor{textcolor}%
\pgfsetfillcolor{textcolor}%
\pgftext[x=1.433903in,y=0.320420in,,top]{\color{textcolor}\rmfamily\fontsize{8.000000}{9.600000}\selectfont \(\displaystyle {10^{-2}}\)}%
\end{pgfscope}%
\begin{pgfscope}%
\pgfpathrectangle{\pgfqpoint{0.514278in}{0.417642in}}{\pgfqpoint{1.884052in}{1.370688in}}%
\pgfusepath{clip}%
\pgfsetrectcap%
\pgfsetroundjoin%
\pgfsetlinewidth{0.803000pt}%
\definecolor{currentstroke}{rgb}{0.450000,0.450000,0.450000}%
\pgfsetstrokecolor{currentstroke}%
\pgfsetdash{}{0pt}%
\pgfpathmoveto{\pgfqpoint{1.951183in}{0.417642in}}%
\pgfpathlineto{\pgfqpoint{1.951183in}{1.788330in}}%
\pgfusepath{stroke}%
\end{pgfscope}%
\begin{pgfscope}%
\pgfsetbuttcap%
\pgfsetroundjoin%
\definecolor{currentfill}{rgb}{0.000000,0.000000,0.000000}%
\pgfsetfillcolor{currentfill}%
\pgfsetlinewidth{0.803000pt}%
\definecolor{currentstroke}{rgb}{0.000000,0.000000,0.000000}%
\pgfsetstrokecolor{currentstroke}%
\pgfsetdash{}{0pt}%
\pgfsys@defobject{currentmarker}{\pgfqpoint{0.000000in}{-0.048611in}}{\pgfqpoint{0.000000in}{0.000000in}}{%
\pgfpathmoveto{\pgfqpoint{0.000000in}{0.000000in}}%
\pgfpathlineto{\pgfqpoint{0.000000in}{-0.048611in}}%
\pgfusepath{stroke,fill}%
}%
\begin{pgfscope}%
\pgfsys@transformshift{1.951183in}{0.417642in}%
\pgfsys@useobject{currentmarker}{}%
\end{pgfscope}%
\end{pgfscope}%
\begin{pgfscope}%
\definecolor{textcolor}{rgb}{0.000000,0.000000,0.000000}%
\pgfsetstrokecolor{textcolor}%
\pgfsetfillcolor{textcolor}%
\pgftext[x=1.951183in,y=0.320420in,,top]{\color{textcolor}\rmfamily\fontsize{8.000000}{9.600000}\selectfont \(\displaystyle {10^{-1}}\)}%
\end{pgfscope}%
\begin{pgfscope}%
\pgfpathrectangle{\pgfqpoint{0.514278in}{0.417642in}}{\pgfqpoint{1.884052in}{1.370688in}}%
\pgfusepath{clip}%
\pgfsetrectcap%
\pgfsetroundjoin%
\pgfsetlinewidth{0.803000pt}%
\definecolor{currentstroke}{rgb}{0.850000,0.850000,0.850000}%
\pgfsetstrokecolor{currentstroke}%
\pgfsetdash{}{0pt}%
\pgfpathmoveto{\pgfqpoint{0.555061in}{0.417642in}}%
\pgfpathlineto{\pgfqpoint{0.555061in}{1.788330in}}%
\pgfusepath{stroke}%
\end{pgfscope}%
\begin{pgfscope}%
\pgfsetbuttcap%
\pgfsetroundjoin%
\definecolor{currentfill}{rgb}{0.000000,0.000000,0.000000}%
\pgfsetfillcolor{currentfill}%
\pgfsetlinewidth{0.602250pt}%
\definecolor{currentstroke}{rgb}{0.000000,0.000000,0.000000}%
\pgfsetstrokecolor{currentstroke}%
\pgfsetdash{}{0pt}%
\pgfsys@defobject{currentmarker}{\pgfqpoint{0.000000in}{-0.027778in}}{\pgfqpoint{0.000000in}{0.000000in}}{%
\pgfpathmoveto{\pgfqpoint{0.000000in}{0.000000in}}%
\pgfpathlineto{\pgfqpoint{0.000000in}{-0.027778in}}%
\pgfusepath{stroke,fill}%
}%
\begin{pgfscope}%
\pgfsys@transformshift{0.555061in}{0.417642in}%
\pgfsys@useobject{currentmarker}{}%
\end{pgfscope}%
\end{pgfscope}%
\begin{pgfscope}%
\pgfpathrectangle{\pgfqpoint{0.514278in}{0.417642in}}{\pgfqpoint{1.884052in}{1.370688in}}%
\pgfusepath{clip}%
\pgfsetrectcap%
\pgfsetroundjoin%
\pgfsetlinewidth{0.803000pt}%
\definecolor{currentstroke}{rgb}{0.850000,0.850000,0.850000}%
\pgfsetstrokecolor{currentstroke}%
\pgfsetdash{}{0pt}%
\pgfpathmoveto{\pgfqpoint{0.646149in}{0.417642in}}%
\pgfpathlineto{\pgfqpoint{0.646149in}{1.788330in}}%
\pgfusepath{stroke}%
\end{pgfscope}%
\begin{pgfscope}%
\pgfsetbuttcap%
\pgfsetroundjoin%
\definecolor{currentfill}{rgb}{0.000000,0.000000,0.000000}%
\pgfsetfillcolor{currentfill}%
\pgfsetlinewidth{0.602250pt}%
\definecolor{currentstroke}{rgb}{0.000000,0.000000,0.000000}%
\pgfsetstrokecolor{currentstroke}%
\pgfsetdash{}{0pt}%
\pgfsys@defobject{currentmarker}{\pgfqpoint{0.000000in}{-0.027778in}}{\pgfqpoint{0.000000in}{0.000000in}}{%
\pgfpathmoveto{\pgfqpoint{0.000000in}{0.000000in}}%
\pgfpathlineto{\pgfqpoint{0.000000in}{-0.027778in}}%
\pgfusepath{stroke,fill}%
}%
\begin{pgfscope}%
\pgfsys@transformshift{0.646149in}{0.417642in}%
\pgfsys@useobject{currentmarker}{}%
\end{pgfscope}%
\end{pgfscope}%
\begin{pgfscope}%
\pgfpathrectangle{\pgfqpoint{0.514278in}{0.417642in}}{\pgfqpoint{1.884052in}{1.370688in}}%
\pgfusepath{clip}%
\pgfsetrectcap%
\pgfsetroundjoin%
\pgfsetlinewidth{0.803000pt}%
\definecolor{currentstroke}{rgb}{0.850000,0.850000,0.850000}%
\pgfsetstrokecolor{currentstroke}%
\pgfsetdash{}{0pt}%
\pgfpathmoveto{\pgfqpoint{0.710777in}{0.417642in}}%
\pgfpathlineto{\pgfqpoint{0.710777in}{1.788330in}}%
\pgfusepath{stroke}%
\end{pgfscope}%
\begin{pgfscope}%
\pgfsetbuttcap%
\pgfsetroundjoin%
\definecolor{currentfill}{rgb}{0.000000,0.000000,0.000000}%
\pgfsetfillcolor{currentfill}%
\pgfsetlinewidth{0.602250pt}%
\definecolor{currentstroke}{rgb}{0.000000,0.000000,0.000000}%
\pgfsetstrokecolor{currentstroke}%
\pgfsetdash{}{0pt}%
\pgfsys@defobject{currentmarker}{\pgfqpoint{0.000000in}{-0.027778in}}{\pgfqpoint{0.000000in}{0.000000in}}{%
\pgfpathmoveto{\pgfqpoint{0.000000in}{0.000000in}}%
\pgfpathlineto{\pgfqpoint{0.000000in}{-0.027778in}}%
\pgfusepath{stroke,fill}%
}%
\begin{pgfscope}%
\pgfsys@transformshift{0.710777in}{0.417642in}%
\pgfsys@useobject{currentmarker}{}%
\end{pgfscope}%
\end{pgfscope}%
\begin{pgfscope}%
\pgfpathrectangle{\pgfqpoint{0.514278in}{0.417642in}}{\pgfqpoint{1.884052in}{1.370688in}}%
\pgfusepath{clip}%
\pgfsetrectcap%
\pgfsetroundjoin%
\pgfsetlinewidth{0.803000pt}%
\definecolor{currentstroke}{rgb}{0.850000,0.850000,0.850000}%
\pgfsetstrokecolor{currentstroke}%
\pgfsetdash{}{0pt}%
\pgfpathmoveto{\pgfqpoint{0.760907in}{0.417642in}}%
\pgfpathlineto{\pgfqpoint{0.760907in}{1.788330in}}%
\pgfusepath{stroke}%
\end{pgfscope}%
\begin{pgfscope}%
\pgfsetbuttcap%
\pgfsetroundjoin%
\definecolor{currentfill}{rgb}{0.000000,0.000000,0.000000}%
\pgfsetfillcolor{currentfill}%
\pgfsetlinewidth{0.602250pt}%
\definecolor{currentstroke}{rgb}{0.000000,0.000000,0.000000}%
\pgfsetstrokecolor{currentstroke}%
\pgfsetdash{}{0pt}%
\pgfsys@defobject{currentmarker}{\pgfqpoint{0.000000in}{-0.027778in}}{\pgfqpoint{0.000000in}{0.000000in}}{%
\pgfpathmoveto{\pgfqpoint{0.000000in}{0.000000in}}%
\pgfpathlineto{\pgfqpoint{0.000000in}{-0.027778in}}%
\pgfusepath{stroke,fill}%
}%
\begin{pgfscope}%
\pgfsys@transformshift{0.760907in}{0.417642in}%
\pgfsys@useobject{currentmarker}{}%
\end{pgfscope}%
\end{pgfscope}%
\begin{pgfscope}%
\pgfpathrectangle{\pgfqpoint{0.514278in}{0.417642in}}{\pgfqpoint{1.884052in}{1.370688in}}%
\pgfusepath{clip}%
\pgfsetrectcap%
\pgfsetroundjoin%
\pgfsetlinewidth{0.803000pt}%
\definecolor{currentstroke}{rgb}{0.850000,0.850000,0.850000}%
\pgfsetstrokecolor{currentstroke}%
\pgfsetdash{}{0pt}%
\pgfpathmoveto{\pgfqpoint{0.801866in}{0.417642in}}%
\pgfpathlineto{\pgfqpoint{0.801866in}{1.788330in}}%
\pgfusepath{stroke}%
\end{pgfscope}%
\begin{pgfscope}%
\pgfsetbuttcap%
\pgfsetroundjoin%
\definecolor{currentfill}{rgb}{0.000000,0.000000,0.000000}%
\pgfsetfillcolor{currentfill}%
\pgfsetlinewidth{0.602250pt}%
\definecolor{currentstroke}{rgb}{0.000000,0.000000,0.000000}%
\pgfsetstrokecolor{currentstroke}%
\pgfsetdash{}{0pt}%
\pgfsys@defobject{currentmarker}{\pgfqpoint{0.000000in}{-0.027778in}}{\pgfqpoint{0.000000in}{0.000000in}}{%
\pgfpathmoveto{\pgfqpoint{0.000000in}{0.000000in}}%
\pgfpathlineto{\pgfqpoint{0.000000in}{-0.027778in}}%
\pgfusepath{stroke,fill}%
}%
\begin{pgfscope}%
\pgfsys@transformshift{0.801866in}{0.417642in}%
\pgfsys@useobject{currentmarker}{}%
\end{pgfscope}%
\end{pgfscope}%
\begin{pgfscope}%
\pgfpathrectangle{\pgfqpoint{0.514278in}{0.417642in}}{\pgfqpoint{1.884052in}{1.370688in}}%
\pgfusepath{clip}%
\pgfsetrectcap%
\pgfsetroundjoin%
\pgfsetlinewidth{0.803000pt}%
\definecolor{currentstroke}{rgb}{0.850000,0.850000,0.850000}%
\pgfsetstrokecolor{currentstroke}%
\pgfsetdash{}{0pt}%
\pgfpathmoveto{\pgfqpoint{0.836496in}{0.417642in}}%
\pgfpathlineto{\pgfqpoint{0.836496in}{1.788330in}}%
\pgfusepath{stroke}%
\end{pgfscope}%
\begin{pgfscope}%
\pgfsetbuttcap%
\pgfsetroundjoin%
\definecolor{currentfill}{rgb}{0.000000,0.000000,0.000000}%
\pgfsetfillcolor{currentfill}%
\pgfsetlinewidth{0.602250pt}%
\definecolor{currentstroke}{rgb}{0.000000,0.000000,0.000000}%
\pgfsetstrokecolor{currentstroke}%
\pgfsetdash{}{0pt}%
\pgfsys@defobject{currentmarker}{\pgfqpoint{0.000000in}{-0.027778in}}{\pgfqpoint{0.000000in}{0.000000in}}{%
\pgfpathmoveto{\pgfqpoint{0.000000in}{0.000000in}}%
\pgfpathlineto{\pgfqpoint{0.000000in}{-0.027778in}}%
\pgfusepath{stroke,fill}%
}%
\begin{pgfscope}%
\pgfsys@transformshift{0.836496in}{0.417642in}%
\pgfsys@useobject{currentmarker}{}%
\end{pgfscope}%
\end{pgfscope}%
\begin{pgfscope}%
\pgfpathrectangle{\pgfqpoint{0.514278in}{0.417642in}}{\pgfqpoint{1.884052in}{1.370688in}}%
\pgfusepath{clip}%
\pgfsetrectcap%
\pgfsetroundjoin%
\pgfsetlinewidth{0.803000pt}%
\definecolor{currentstroke}{rgb}{0.850000,0.850000,0.850000}%
\pgfsetstrokecolor{currentstroke}%
\pgfsetdash{}{0pt}%
\pgfpathmoveto{\pgfqpoint{0.866494in}{0.417642in}}%
\pgfpathlineto{\pgfqpoint{0.866494in}{1.788330in}}%
\pgfusepath{stroke}%
\end{pgfscope}%
\begin{pgfscope}%
\pgfsetbuttcap%
\pgfsetroundjoin%
\definecolor{currentfill}{rgb}{0.000000,0.000000,0.000000}%
\pgfsetfillcolor{currentfill}%
\pgfsetlinewidth{0.602250pt}%
\definecolor{currentstroke}{rgb}{0.000000,0.000000,0.000000}%
\pgfsetstrokecolor{currentstroke}%
\pgfsetdash{}{0pt}%
\pgfsys@defobject{currentmarker}{\pgfqpoint{0.000000in}{-0.027778in}}{\pgfqpoint{0.000000in}{0.000000in}}{%
\pgfpathmoveto{\pgfqpoint{0.000000in}{0.000000in}}%
\pgfpathlineto{\pgfqpoint{0.000000in}{-0.027778in}}%
\pgfusepath{stroke,fill}%
}%
\begin{pgfscope}%
\pgfsys@transformshift{0.866494in}{0.417642in}%
\pgfsys@useobject{currentmarker}{}%
\end{pgfscope}%
\end{pgfscope}%
\begin{pgfscope}%
\pgfpathrectangle{\pgfqpoint{0.514278in}{0.417642in}}{\pgfqpoint{1.884052in}{1.370688in}}%
\pgfusepath{clip}%
\pgfsetrectcap%
\pgfsetroundjoin%
\pgfsetlinewidth{0.803000pt}%
\definecolor{currentstroke}{rgb}{0.850000,0.850000,0.850000}%
\pgfsetstrokecolor{currentstroke}%
\pgfsetdash{}{0pt}%
\pgfpathmoveto{\pgfqpoint{0.892954in}{0.417642in}}%
\pgfpathlineto{\pgfqpoint{0.892954in}{1.788330in}}%
\pgfusepath{stroke}%
\end{pgfscope}%
\begin{pgfscope}%
\pgfsetbuttcap%
\pgfsetroundjoin%
\definecolor{currentfill}{rgb}{0.000000,0.000000,0.000000}%
\pgfsetfillcolor{currentfill}%
\pgfsetlinewidth{0.602250pt}%
\definecolor{currentstroke}{rgb}{0.000000,0.000000,0.000000}%
\pgfsetstrokecolor{currentstroke}%
\pgfsetdash{}{0pt}%
\pgfsys@defobject{currentmarker}{\pgfqpoint{0.000000in}{-0.027778in}}{\pgfqpoint{0.000000in}{0.000000in}}{%
\pgfpathmoveto{\pgfqpoint{0.000000in}{0.000000in}}%
\pgfpathlineto{\pgfqpoint{0.000000in}{-0.027778in}}%
\pgfusepath{stroke,fill}%
}%
\begin{pgfscope}%
\pgfsys@transformshift{0.892954in}{0.417642in}%
\pgfsys@useobject{currentmarker}{}%
\end{pgfscope}%
\end{pgfscope}%
\begin{pgfscope}%
\pgfpathrectangle{\pgfqpoint{0.514278in}{0.417642in}}{\pgfqpoint{1.884052in}{1.370688in}}%
\pgfusepath{clip}%
\pgfsetrectcap%
\pgfsetroundjoin%
\pgfsetlinewidth{0.803000pt}%
\definecolor{currentstroke}{rgb}{0.850000,0.850000,0.850000}%
\pgfsetstrokecolor{currentstroke}%
\pgfsetdash{}{0pt}%
\pgfpathmoveto{\pgfqpoint{1.072340in}{0.417642in}}%
\pgfpathlineto{\pgfqpoint{1.072340in}{1.788330in}}%
\pgfusepath{stroke}%
\end{pgfscope}%
\begin{pgfscope}%
\pgfsetbuttcap%
\pgfsetroundjoin%
\definecolor{currentfill}{rgb}{0.000000,0.000000,0.000000}%
\pgfsetfillcolor{currentfill}%
\pgfsetlinewidth{0.602250pt}%
\definecolor{currentstroke}{rgb}{0.000000,0.000000,0.000000}%
\pgfsetstrokecolor{currentstroke}%
\pgfsetdash{}{0pt}%
\pgfsys@defobject{currentmarker}{\pgfqpoint{0.000000in}{-0.027778in}}{\pgfqpoint{0.000000in}{0.000000in}}{%
\pgfpathmoveto{\pgfqpoint{0.000000in}{0.000000in}}%
\pgfpathlineto{\pgfqpoint{0.000000in}{-0.027778in}}%
\pgfusepath{stroke,fill}%
}%
\begin{pgfscope}%
\pgfsys@transformshift{1.072340in}{0.417642in}%
\pgfsys@useobject{currentmarker}{}%
\end{pgfscope}%
\end{pgfscope}%
\begin{pgfscope}%
\pgfpathrectangle{\pgfqpoint{0.514278in}{0.417642in}}{\pgfqpoint{1.884052in}{1.370688in}}%
\pgfusepath{clip}%
\pgfsetrectcap%
\pgfsetroundjoin%
\pgfsetlinewidth{0.803000pt}%
\definecolor{currentstroke}{rgb}{0.850000,0.850000,0.850000}%
\pgfsetstrokecolor{currentstroke}%
\pgfsetdash{}{0pt}%
\pgfpathmoveto{\pgfqpoint{1.163429in}{0.417642in}}%
\pgfpathlineto{\pgfqpoint{1.163429in}{1.788330in}}%
\pgfusepath{stroke}%
\end{pgfscope}%
\begin{pgfscope}%
\pgfsetbuttcap%
\pgfsetroundjoin%
\definecolor{currentfill}{rgb}{0.000000,0.000000,0.000000}%
\pgfsetfillcolor{currentfill}%
\pgfsetlinewidth{0.602250pt}%
\definecolor{currentstroke}{rgb}{0.000000,0.000000,0.000000}%
\pgfsetstrokecolor{currentstroke}%
\pgfsetdash{}{0pt}%
\pgfsys@defobject{currentmarker}{\pgfqpoint{0.000000in}{-0.027778in}}{\pgfqpoint{0.000000in}{0.000000in}}{%
\pgfpathmoveto{\pgfqpoint{0.000000in}{0.000000in}}%
\pgfpathlineto{\pgfqpoint{0.000000in}{-0.027778in}}%
\pgfusepath{stroke,fill}%
}%
\begin{pgfscope}%
\pgfsys@transformshift{1.163429in}{0.417642in}%
\pgfsys@useobject{currentmarker}{}%
\end{pgfscope}%
\end{pgfscope}%
\begin{pgfscope}%
\pgfpathrectangle{\pgfqpoint{0.514278in}{0.417642in}}{\pgfqpoint{1.884052in}{1.370688in}}%
\pgfusepath{clip}%
\pgfsetrectcap%
\pgfsetroundjoin%
\pgfsetlinewidth{0.803000pt}%
\definecolor{currentstroke}{rgb}{0.850000,0.850000,0.850000}%
\pgfsetstrokecolor{currentstroke}%
\pgfsetdash{}{0pt}%
\pgfpathmoveto{\pgfqpoint{1.228057in}{0.417642in}}%
\pgfpathlineto{\pgfqpoint{1.228057in}{1.788330in}}%
\pgfusepath{stroke}%
\end{pgfscope}%
\begin{pgfscope}%
\pgfsetbuttcap%
\pgfsetroundjoin%
\definecolor{currentfill}{rgb}{0.000000,0.000000,0.000000}%
\pgfsetfillcolor{currentfill}%
\pgfsetlinewidth{0.602250pt}%
\definecolor{currentstroke}{rgb}{0.000000,0.000000,0.000000}%
\pgfsetstrokecolor{currentstroke}%
\pgfsetdash{}{0pt}%
\pgfsys@defobject{currentmarker}{\pgfqpoint{0.000000in}{-0.027778in}}{\pgfqpoint{0.000000in}{0.000000in}}{%
\pgfpathmoveto{\pgfqpoint{0.000000in}{0.000000in}}%
\pgfpathlineto{\pgfqpoint{0.000000in}{-0.027778in}}%
\pgfusepath{stroke,fill}%
}%
\begin{pgfscope}%
\pgfsys@transformshift{1.228057in}{0.417642in}%
\pgfsys@useobject{currentmarker}{}%
\end{pgfscope}%
\end{pgfscope}%
\begin{pgfscope}%
\pgfpathrectangle{\pgfqpoint{0.514278in}{0.417642in}}{\pgfqpoint{1.884052in}{1.370688in}}%
\pgfusepath{clip}%
\pgfsetrectcap%
\pgfsetroundjoin%
\pgfsetlinewidth{0.803000pt}%
\definecolor{currentstroke}{rgb}{0.850000,0.850000,0.850000}%
\pgfsetstrokecolor{currentstroke}%
\pgfsetdash{}{0pt}%
\pgfpathmoveto{\pgfqpoint{1.278187in}{0.417642in}}%
\pgfpathlineto{\pgfqpoint{1.278187in}{1.788330in}}%
\pgfusepath{stroke}%
\end{pgfscope}%
\begin{pgfscope}%
\pgfsetbuttcap%
\pgfsetroundjoin%
\definecolor{currentfill}{rgb}{0.000000,0.000000,0.000000}%
\pgfsetfillcolor{currentfill}%
\pgfsetlinewidth{0.602250pt}%
\definecolor{currentstroke}{rgb}{0.000000,0.000000,0.000000}%
\pgfsetstrokecolor{currentstroke}%
\pgfsetdash{}{0pt}%
\pgfsys@defobject{currentmarker}{\pgfqpoint{0.000000in}{-0.027778in}}{\pgfqpoint{0.000000in}{0.000000in}}{%
\pgfpathmoveto{\pgfqpoint{0.000000in}{0.000000in}}%
\pgfpathlineto{\pgfqpoint{0.000000in}{-0.027778in}}%
\pgfusepath{stroke,fill}%
}%
\begin{pgfscope}%
\pgfsys@transformshift{1.278187in}{0.417642in}%
\pgfsys@useobject{currentmarker}{}%
\end{pgfscope}%
\end{pgfscope}%
\begin{pgfscope}%
\pgfpathrectangle{\pgfqpoint{0.514278in}{0.417642in}}{\pgfqpoint{1.884052in}{1.370688in}}%
\pgfusepath{clip}%
\pgfsetrectcap%
\pgfsetroundjoin%
\pgfsetlinewidth{0.803000pt}%
\definecolor{currentstroke}{rgb}{0.850000,0.850000,0.850000}%
\pgfsetstrokecolor{currentstroke}%
\pgfsetdash{}{0pt}%
\pgfpathmoveto{\pgfqpoint{1.319146in}{0.417642in}}%
\pgfpathlineto{\pgfqpoint{1.319146in}{1.788330in}}%
\pgfusepath{stroke}%
\end{pgfscope}%
\begin{pgfscope}%
\pgfsetbuttcap%
\pgfsetroundjoin%
\definecolor{currentfill}{rgb}{0.000000,0.000000,0.000000}%
\pgfsetfillcolor{currentfill}%
\pgfsetlinewidth{0.602250pt}%
\definecolor{currentstroke}{rgb}{0.000000,0.000000,0.000000}%
\pgfsetstrokecolor{currentstroke}%
\pgfsetdash{}{0pt}%
\pgfsys@defobject{currentmarker}{\pgfqpoint{0.000000in}{-0.027778in}}{\pgfqpoint{0.000000in}{0.000000in}}{%
\pgfpathmoveto{\pgfqpoint{0.000000in}{0.000000in}}%
\pgfpathlineto{\pgfqpoint{0.000000in}{-0.027778in}}%
\pgfusepath{stroke,fill}%
}%
\begin{pgfscope}%
\pgfsys@transformshift{1.319146in}{0.417642in}%
\pgfsys@useobject{currentmarker}{}%
\end{pgfscope}%
\end{pgfscope}%
\begin{pgfscope}%
\pgfpathrectangle{\pgfqpoint{0.514278in}{0.417642in}}{\pgfqpoint{1.884052in}{1.370688in}}%
\pgfusepath{clip}%
\pgfsetrectcap%
\pgfsetroundjoin%
\pgfsetlinewidth{0.803000pt}%
\definecolor{currentstroke}{rgb}{0.850000,0.850000,0.850000}%
\pgfsetstrokecolor{currentstroke}%
\pgfsetdash{}{0pt}%
\pgfpathmoveto{\pgfqpoint{1.353776in}{0.417642in}}%
\pgfpathlineto{\pgfqpoint{1.353776in}{1.788330in}}%
\pgfusepath{stroke}%
\end{pgfscope}%
\begin{pgfscope}%
\pgfsetbuttcap%
\pgfsetroundjoin%
\definecolor{currentfill}{rgb}{0.000000,0.000000,0.000000}%
\pgfsetfillcolor{currentfill}%
\pgfsetlinewidth{0.602250pt}%
\definecolor{currentstroke}{rgb}{0.000000,0.000000,0.000000}%
\pgfsetstrokecolor{currentstroke}%
\pgfsetdash{}{0pt}%
\pgfsys@defobject{currentmarker}{\pgfqpoint{0.000000in}{-0.027778in}}{\pgfqpoint{0.000000in}{0.000000in}}{%
\pgfpathmoveto{\pgfqpoint{0.000000in}{0.000000in}}%
\pgfpathlineto{\pgfqpoint{0.000000in}{-0.027778in}}%
\pgfusepath{stroke,fill}%
}%
\begin{pgfscope}%
\pgfsys@transformshift{1.353776in}{0.417642in}%
\pgfsys@useobject{currentmarker}{}%
\end{pgfscope}%
\end{pgfscope}%
\begin{pgfscope}%
\pgfpathrectangle{\pgfqpoint{0.514278in}{0.417642in}}{\pgfqpoint{1.884052in}{1.370688in}}%
\pgfusepath{clip}%
\pgfsetrectcap%
\pgfsetroundjoin%
\pgfsetlinewidth{0.803000pt}%
\definecolor{currentstroke}{rgb}{0.850000,0.850000,0.850000}%
\pgfsetstrokecolor{currentstroke}%
\pgfsetdash{}{0pt}%
\pgfpathmoveto{\pgfqpoint{1.383774in}{0.417642in}}%
\pgfpathlineto{\pgfqpoint{1.383774in}{1.788330in}}%
\pgfusepath{stroke}%
\end{pgfscope}%
\begin{pgfscope}%
\pgfsetbuttcap%
\pgfsetroundjoin%
\definecolor{currentfill}{rgb}{0.000000,0.000000,0.000000}%
\pgfsetfillcolor{currentfill}%
\pgfsetlinewidth{0.602250pt}%
\definecolor{currentstroke}{rgb}{0.000000,0.000000,0.000000}%
\pgfsetstrokecolor{currentstroke}%
\pgfsetdash{}{0pt}%
\pgfsys@defobject{currentmarker}{\pgfqpoint{0.000000in}{-0.027778in}}{\pgfqpoint{0.000000in}{0.000000in}}{%
\pgfpathmoveto{\pgfqpoint{0.000000in}{0.000000in}}%
\pgfpathlineto{\pgfqpoint{0.000000in}{-0.027778in}}%
\pgfusepath{stroke,fill}%
}%
\begin{pgfscope}%
\pgfsys@transformshift{1.383774in}{0.417642in}%
\pgfsys@useobject{currentmarker}{}%
\end{pgfscope}%
\end{pgfscope}%
\begin{pgfscope}%
\pgfpathrectangle{\pgfqpoint{0.514278in}{0.417642in}}{\pgfqpoint{1.884052in}{1.370688in}}%
\pgfusepath{clip}%
\pgfsetrectcap%
\pgfsetroundjoin%
\pgfsetlinewidth{0.803000pt}%
\definecolor{currentstroke}{rgb}{0.850000,0.850000,0.850000}%
\pgfsetstrokecolor{currentstroke}%
\pgfsetdash{}{0pt}%
\pgfpathmoveto{\pgfqpoint{1.410234in}{0.417642in}}%
\pgfpathlineto{\pgfqpoint{1.410234in}{1.788330in}}%
\pgfusepath{stroke}%
\end{pgfscope}%
\begin{pgfscope}%
\pgfsetbuttcap%
\pgfsetroundjoin%
\definecolor{currentfill}{rgb}{0.000000,0.000000,0.000000}%
\pgfsetfillcolor{currentfill}%
\pgfsetlinewidth{0.602250pt}%
\definecolor{currentstroke}{rgb}{0.000000,0.000000,0.000000}%
\pgfsetstrokecolor{currentstroke}%
\pgfsetdash{}{0pt}%
\pgfsys@defobject{currentmarker}{\pgfqpoint{0.000000in}{-0.027778in}}{\pgfqpoint{0.000000in}{0.000000in}}{%
\pgfpathmoveto{\pgfqpoint{0.000000in}{0.000000in}}%
\pgfpathlineto{\pgfqpoint{0.000000in}{-0.027778in}}%
\pgfusepath{stroke,fill}%
}%
\begin{pgfscope}%
\pgfsys@transformshift{1.410234in}{0.417642in}%
\pgfsys@useobject{currentmarker}{}%
\end{pgfscope}%
\end{pgfscope}%
\begin{pgfscope}%
\pgfpathrectangle{\pgfqpoint{0.514278in}{0.417642in}}{\pgfqpoint{1.884052in}{1.370688in}}%
\pgfusepath{clip}%
\pgfsetrectcap%
\pgfsetroundjoin%
\pgfsetlinewidth{0.803000pt}%
\definecolor{currentstroke}{rgb}{0.850000,0.850000,0.850000}%
\pgfsetstrokecolor{currentstroke}%
\pgfsetdash{}{0pt}%
\pgfpathmoveto{\pgfqpoint{1.589620in}{0.417642in}}%
\pgfpathlineto{\pgfqpoint{1.589620in}{1.788330in}}%
\pgfusepath{stroke}%
\end{pgfscope}%
\begin{pgfscope}%
\pgfsetbuttcap%
\pgfsetroundjoin%
\definecolor{currentfill}{rgb}{0.000000,0.000000,0.000000}%
\pgfsetfillcolor{currentfill}%
\pgfsetlinewidth{0.602250pt}%
\definecolor{currentstroke}{rgb}{0.000000,0.000000,0.000000}%
\pgfsetstrokecolor{currentstroke}%
\pgfsetdash{}{0pt}%
\pgfsys@defobject{currentmarker}{\pgfqpoint{0.000000in}{-0.027778in}}{\pgfqpoint{0.000000in}{0.000000in}}{%
\pgfpathmoveto{\pgfqpoint{0.000000in}{0.000000in}}%
\pgfpathlineto{\pgfqpoint{0.000000in}{-0.027778in}}%
\pgfusepath{stroke,fill}%
}%
\begin{pgfscope}%
\pgfsys@transformshift{1.589620in}{0.417642in}%
\pgfsys@useobject{currentmarker}{}%
\end{pgfscope}%
\end{pgfscope}%
\begin{pgfscope}%
\pgfpathrectangle{\pgfqpoint{0.514278in}{0.417642in}}{\pgfqpoint{1.884052in}{1.370688in}}%
\pgfusepath{clip}%
\pgfsetrectcap%
\pgfsetroundjoin%
\pgfsetlinewidth{0.803000pt}%
\definecolor{currentstroke}{rgb}{0.850000,0.850000,0.850000}%
\pgfsetstrokecolor{currentstroke}%
\pgfsetdash{}{0pt}%
\pgfpathmoveto{\pgfqpoint{1.680709in}{0.417642in}}%
\pgfpathlineto{\pgfqpoint{1.680709in}{1.788330in}}%
\pgfusepath{stroke}%
\end{pgfscope}%
\begin{pgfscope}%
\pgfsetbuttcap%
\pgfsetroundjoin%
\definecolor{currentfill}{rgb}{0.000000,0.000000,0.000000}%
\pgfsetfillcolor{currentfill}%
\pgfsetlinewidth{0.602250pt}%
\definecolor{currentstroke}{rgb}{0.000000,0.000000,0.000000}%
\pgfsetstrokecolor{currentstroke}%
\pgfsetdash{}{0pt}%
\pgfsys@defobject{currentmarker}{\pgfqpoint{0.000000in}{-0.027778in}}{\pgfqpoint{0.000000in}{0.000000in}}{%
\pgfpathmoveto{\pgfqpoint{0.000000in}{0.000000in}}%
\pgfpathlineto{\pgfqpoint{0.000000in}{-0.027778in}}%
\pgfusepath{stroke,fill}%
}%
\begin{pgfscope}%
\pgfsys@transformshift{1.680709in}{0.417642in}%
\pgfsys@useobject{currentmarker}{}%
\end{pgfscope}%
\end{pgfscope}%
\begin{pgfscope}%
\pgfpathrectangle{\pgfqpoint{0.514278in}{0.417642in}}{\pgfqpoint{1.884052in}{1.370688in}}%
\pgfusepath{clip}%
\pgfsetrectcap%
\pgfsetroundjoin%
\pgfsetlinewidth{0.803000pt}%
\definecolor{currentstroke}{rgb}{0.850000,0.850000,0.850000}%
\pgfsetstrokecolor{currentstroke}%
\pgfsetdash{}{0pt}%
\pgfpathmoveto{\pgfqpoint{1.745337in}{0.417642in}}%
\pgfpathlineto{\pgfqpoint{1.745337in}{1.788330in}}%
\pgfusepath{stroke}%
\end{pgfscope}%
\begin{pgfscope}%
\pgfsetbuttcap%
\pgfsetroundjoin%
\definecolor{currentfill}{rgb}{0.000000,0.000000,0.000000}%
\pgfsetfillcolor{currentfill}%
\pgfsetlinewidth{0.602250pt}%
\definecolor{currentstroke}{rgb}{0.000000,0.000000,0.000000}%
\pgfsetstrokecolor{currentstroke}%
\pgfsetdash{}{0pt}%
\pgfsys@defobject{currentmarker}{\pgfqpoint{0.000000in}{-0.027778in}}{\pgfqpoint{0.000000in}{0.000000in}}{%
\pgfpathmoveto{\pgfqpoint{0.000000in}{0.000000in}}%
\pgfpathlineto{\pgfqpoint{0.000000in}{-0.027778in}}%
\pgfusepath{stroke,fill}%
}%
\begin{pgfscope}%
\pgfsys@transformshift{1.745337in}{0.417642in}%
\pgfsys@useobject{currentmarker}{}%
\end{pgfscope}%
\end{pgfscope}%
\begin{pgfscope}%
\pgfpathrectangle{\pgfqpoint{0.514278in}{0.417642in}}{\pgfqpoint{1.884052in}{1.370688in}}%
\pgfusepath{clip}%
\pgfsetrectcap%
\pgfsetroundjoin%
\pgfsetlinewidth{0.803000pt}%
\definecolor{currentstroke}{rgb}{0.850000,0.850000,0.850000}%
\pgfsetstrokecolor{currentstroke}%
\pgfsetdash{}{0pt}%
\pgfpathmoveto{\pgfqpoint{1.795466in}{0.417642in}}%
\pgfpathlineto{\pgfqpoint{1.795466in}{1.788330in}}%
\pgfusepath{stroke}%
\end{pgfscope}%
\begin{pgfscope}%
\pgfsetbuttcap%
\pgfsetroundjoin%
\definecolor{currentfill}{rgb}{0.000000,0.000000,0.000000}%
\pgfsetfillcolor{currentfill}%
\pgfsetlinewidth{0.602250pt}%
\definecolor{currentstroke}{rgb}{0.000000,0.000000,0.000000}%
\pgfsetstrokecolor{currentstroke}%
\pgfsetdash{}{0pt}%
\pgfsys@defobject{currentmarker}{\pgfqpoint{0.000000in}{-0.027778in}}{\pgfqpoint{0.000000in}{0.000000in}}{%
\pgfpathmoveto{\pgfqpoint{0.000000in}{0.000000in}}%
\pgfpathlineto{\pgfqpoint{0.000000in}{-0.027778in}}%
\pgfusepath{stroke,fill}%
}%
\begin{pgfscope}%
\pgfsys@transformshift{1.795466in}{0.417642in}%
\pgfsys@useobject{currentmarker}{}%
\end{pgfscope}%
\end{pgfscope}%
\begin{pgfscope}%
\pgfpathrectangle{\pgfqpoint{0.514278in}{0.417642in}}{\pgfqpoint{1.884052in}{1.370688in}}%
\pgfusepath{clip}%
\pgfsetrectcap%
\pgfsetroundjoin%
\pgfsetlinewidth{0.803000pt}%
\definecolor{currentstroke}{rgb}{0.850000,0.850000,0.850000}%
\pgfsetstrokecolor{currentstroke}%
\pgfsetdash{}{0pt}%
\pgfpathmoveto{\pgfqpoint{1.836425in}{0.417642in}}%
\pgfpathlineto{\pgfqpoint{1.836425in}{1.788330in}}%
\pgfusepath{stroke}%
\end{pgfscope}%
\begin{pgfscope}%
\pgfsetbuttcap%
\pgfsetroundjoin%
\definecolor{currentfill}{rgb}{0.000000,0.000000,0.000000}%
\pgfsetfillcolor{currentfill}%
\pgfsetlinewidth{0.602250pt}%
\definecolor{currentstroke}{rgb}{0.000000,0.000000,0.000000}%
\pgfsetstrokecolor{currentstroke}%
\pgfsetdash{}{0pt}%
\pgfsys@defobject{currentmarker}{\pgfqpoint{0.000000in}{-0.027778in}}{\pgfqpoint{0.000000in}{0.000000in}}{%
\pgfpathmoveto{\pgfqpoint{0.000000in}{0.000000in}}%
\pgfpathlineto{\pgfqpoint{0.000000in}{-0.027778in}}%
\pgfusepath{stroke,fill}%
}%
\begin{pgfscope}%
\pgfsys@transformshift{1.836425in}{0.417642in}%
\pgfsys@useobject{currentmarker}{}%
\end{pgfscope}%
\end{pgfscope}%
\begin{pgfscope}%
\pgfpathrectangle{\pgfqpoint{0.514278in}{0.417642in}}{\pgfqpoint{1.884052in}{1.370688in}}%
\pgfusepath{clip}%
\pgfsetrectcap%
\pgfsetroundjoin%
\pgfsetlinewidth{0.803000pt}%
\definecolor{currentstroke}{rgb}{0.850000,0.850000,0.850000}%
\pgfsetstrokecolor{currentstroke}%
\pgfsetdash{}{0pt}%
\pgfpathmoveto{\pgfqpoint{1.871056in}{0.417642in}}%
\pgfpathlineto{\pgfqpoint{1.871056in}{1.788330in}}%
\pgfusepath{stroke}%
\end{pgfscope}%
\begin{pgfscope}%
\pgfsetbuttcap%
\pgfsetroundjoin%
\definecolor{currentfill}{rgb}{0.000000,0.000000,0.000000}%
\pgfsetfillcolor{currentfill}%
\pgfsetlinewidth{0.602250pt}%
\definecolor{currentstroke}{rgb}{0.000000,0.000000,0.000000}%
\pgfsetstrokecolor{currentstroke}%
\pgfsetdash{}{0pt}%
\pgfsys@defobject{currentmarker}{\pgfqpoint{0.000000in}{-0.027778in}}{\pgfqpoint{0.000000in}{0.000000in}}{%
\pgfpathmoveto{\pgfqpoint{0.000000in}{0.000000in}}%
\pgfpathlineto{\pgfqpoint{0.000000in}{-0.027778in}}%
\pgfusepath{stroke,fill}%
}%
\begin{pgfscope}%
\pgfsys@transformshift{1.871056in}{0.417642in}%
\pgfsys@useobject{currentmarker}{}%
\end{pgfscope}%
\end{pgfscope}%
\begin{pgfscope}%
\pgfpathrectangle{\pgfqpoint{0.514278in}{0.417642in}}{\pgfqpoint{1.884052in}{1.370688in}}%
\pgfusepath{clip}%
\pgfsetrectcap%
\pgfsetroundjoin%
\pgfsetlinewidth{0.803000pt}%
\definecolor{currentstroke}{rgb}{0.850000,0.850000,0.850000}%
\pgfsetstrokecolor{currentstroke}%
\pgfsetdash{}{0pt}%
\pgfpathmoveto{\pgfqpoint{1.901054in}{0.417642in}}%
\pgfpathlineto{\pgfqpoint{1.901054in}{1.788330in}}%
\pgfusepath{stroke}%
\end{pgfscope}%
\begin{pgfscope}%
\pgfsetbuttcap%
\pgfsetroundjoin%
\definecolor{currentfill}{rgb}{0.000000,0.000000,0.000000}%
\pgfsetfillcolor{currentfill}%
\pgfsetlinewidth{0.602250pt}%
\definecolor{currentstroke}{rgb}{0.000000,0.000000,0.000000}%
\pgfsetstrokecolor{currentstroke}%
\pgfsetdash{}{0pt}%
\pgfsys@defobject{currentmarker}{\pgfqpoint{0.000000in}{-0.027778in}}{\pgfqpoint{0.000000in}{0.000000in}}{%
\pgfpathmoveto{\pgfqpoint{0.000000in}{0.000000in}}%
\pgfpathlineto{\pgfqpoint{0.000000in}{-0.027778in}}%
\pgfusepath{stroke,fill}%
}%
\begin{pgfscope}%
\pgfsys@transformshift{1.901054in}{0.417642in}%
\pgfsys@useobject{currentmarker}{}%
\end{pgfscope}%
\end{pgfscope}%
\begin{pgfscope}%
\pgfpathrectangle{\pgfqpoint{0.514278in}{0.417642in}}{\pgfqpoint{1.884052in}{1.370688in}}%
\pgfusepath{clip}%
\pgfsetrectcap%
\pgfsetroundjoin%
\pgfsetlinewidth{0.803000pt}%
\definecolor{currentstroke}{rgb}{0.850000,0.850000,0.850000}%
\pgfsetstrokecolor{currentstroke}%
\pgfsetdash{}{0pt}%
\pgfpathmoveto{\pgfqpoint{1.927514in}{0.417642in}}%
\pgfpathlineto{\pgfqpoint{1.927514in}{1.788330in}}%
\pgfusepath{stroke}%
\end{pgfscope}%
\begin{pgfscope}%
\pgfsetbuttcap%
\pgfsetroundjoin%
\definecolor{currentfill}{rgb}{0.000000,0.000000,0.000000}%
\pgfsetfillcolor{currentfill}%
\pgfsetlinewidth{0.602250pt}%
\definecolor{currentstroke}{rgb}{0.000000,0.000000,0.000000}%
\pgfsetstrokecolor{currentstroke}%
\pgfsetdash{}{0pt}%
\pgfsys@defobject{currentmarker}{\pgfqpoint{0.000000in}{-0.027778in}}{\pgfqpoint{0.000000in}{0.000000in}}{%
\pgfpathmoveto{\pgfqpoint{0.000000in}{0.000000in}}%
\pgfpathlineto{\pgfqpoint{0.000000in}{-0.027778in}}%
\pgfusepath{stroke,fill}%
}%
\begin{pgfscope}%
\pgfsys@transformshift{1.927514in}{0.417642in}%
\pgfsys@useobject{currentmarker}{}%
\end{pgfscope}%
\end{pgfscope}%
\begin{pgfscope}%
\pgfpathrectangle{\pgfqpoint{0.514278in}{0.417642in}}{\pgfqpoint{1.884052in}{1.370688in}}%
\pgfusepath{clip}%
\pgfsetrectcap%
\pgfsetroundjoin%
\pgfsetlinewidth{0.803000pt}%
\definecolor{currentstroke}{rgb}{0.850000,0.850000,0.850000}%
\pgfsetstrokecolor{currentstroke}%
\pgfsetdash{}{0pt}%
\pgfpathmoveto{\pgfqpoint{2.106900in}{0.417642in}}%
\pgfpathlineto{\pgfqpoint{2.106900in}{1.788330in}}%
\pgfusepath{stroke}%
\end{pgfscope}%
\begin{pgfscope}%
\pgfsetbuttcap%
\pgfsetroundjoin%
\definecolor{currentfill}{rgb}{0.000000,0.000000,0.000000}%
\pgfsetfillcolor{currentfill}%
\pgfsetlinewidth{0.602250pt}%
\definecolor{currentstroke}{rgb}{0.000000,0.000000,0.000000}%
\pgfsetstrokecolor{currentstroke}%
\pgfsetdash{}{0pt}%
\pgfsys@defobject{currentmarker}{\pgfqpoint{0.000000in}{-0.027778in}}{\pgfqpoint{0.000000in}{0.000000in}}{%
\pgfpathmoveto{\pgfqpoint{0.000000in}{0.000000in}}%
\pgfpathlineto{\pgfqpoint{0.000000in}{-0.027778in}}%
\pgfusepath{stroke,fill}%
}%
\begin{pgfscope}%
\pgfsys@transformshift{2.106900in}{0.417642in}%
\pgfsys@useobject{currentmarker}{}%
\end{pgfscope}%
\end{pgfscope}%
\begin{pgfscope}%
\pgfpathrectangle{\pgfqpoint{0.514278in}{0.417642in}}{\pgfqpoint{1.884052in}{1.370688in}}%
\pgfusepath{clip}%
\pgfsetrectcap%
\pgfsetroundjoin%
\pgfsetlinewidth{0.803000pt}%
\definecolor{currentstroke}{rgb}{0.850000,0.850000,0.850000}%
\pgfsetstrokecolor{currentstroke}%
\pgfsetdash{}{0pt}%
\pgfpathmoveto{\pgfqpoint{2.197988in}{0.417642in}}%
\pgfpathlineto{\pgfqpoint{2.197988in}{1.788330in}}%
\pgfusepath{stroke}%
\end{pgfscope}%
\begin{pgfscope}%
\pgfsetbuttcap%
\pgfsetroundjoin%
\definecolor{currentfill}{rgb}{0.000000,0.000000,0.000000}%
\pgfsetfillcolor{currentfill}%
\pgfsetlinewidth{0.602250pt}%
\definecolor{currentstroke}{rgb}{0.000000,0.000000,0.000000}%
\pgfsetstrokecolor{currentstroke}%
\pgfsetdash{}{0pt}%
\pgfsys@defobject{currentmarker}{\pgfqpoint{0.000000in}{-0.027778in}}{\pgfqpoint{0.000000in}{0.000000in}}{%
\pgfpathmoveto{\pgfqpoint{0.000000in}{0.000000in}}%
\pgfpathlineto{\pgfqpoint{0.000000in}{-0.027778in}}%
\pgfusepath{stroke,fill}%
}%
\begin{pgfscope}%
\pgfsys@transformshift{2.197988in}{0.417642in}%
\pgfsys@useobject{currentmarker}{}%
\end{pgfscope}%
\end{pgfscope}%
\begin{pgfscope}%
\pgfpathrectangle{\pgfqpoint{0.514278in}{0.417642in}}{\pgfqpoint{1.884052in}{1.370688in}}%
\pgfusepath{clip}%
\pgfsetrectcap%
\pgfsetroundjoin%
\pgfsetlinewidth{0.803000pt}%
\definecolor{currentstroke}{rgb}{0.850000,0.850000,0.850000}%
\pgfsetstrokecolor{currentstroke}%
\pgfsetdash{}{0pt}%
\pgfpathmoveto{\pgfqpoint{2.262617in}{0.417642in}}%
\pgfpathlineto{\pgfqpoint{2.262617in}{1.788330in}}%
\pgfusepath{stroke}%
\end{pgfscope}%
\begin{pgfscope}%
\pgfsetbuttcap%
\pgfsetroundjoin%
\definecolor{currentfill}{rgb}{0.000000,0.000000,0.000000}%
\pgfsetfillcolor{currentfill}%
\pgfsetlinewidth{0.602250pt}%
\definecolor{currentstroke}{rgb}{0.000000,0.000000,0.000000}%
\pgfsetstrokecolor{currentstroke}%
\pgfsetdash{}{0pt}%
\pgfsys@defobject{currentmarker}{\pgfqpoint{0.000000in}{-0.027778in}}{\pgfqpoint{0.000000in}{0.000000in}}{%
\pgfpathmoveto{\pgfqpoint{0.000000in}{0.000000in}}%
\pgfpathlineto{\pgfqpoint{0.000000in}{-0.027778in}}%
\pgfusepath{stroke,fill}%
}%
\begin{pgfscope}%
\pgfsys@transformshift{2.262617in}{0.417642in}%
\pgfsys@useobject{currentmarker}{}%
\end{pgfscope}%
\end{pgfscope}%
\begin{pgfscope}%
\pgfpathrectangle{\pgfqpoint{0.514278in}{0.417642in}}{\pgfqpoint{1.884052in}{1.370688in}}%
\pgfusepath{clip}%
\pgfsetrectcap%
\pgfsetroundjoin%
\pgfsetlinewidth{0.803000pt}%
\definecolor{currentstroke}{rgb}{0.850000,0.850000,0.850000}%
\pgfsetstrokecolor{currentstroke}%
\pgfsetdash{}{0pt}%
\pgfpathmoveto{\pgfqpoint{2.312746in}{0.417642in}}%
\pgfpathlineto{\pgfqpoint{2.312746in}{1.788330in}}%
\pgfusepath{stroke}%
\end{pgfscope}%
\begin{pgfscope}%
\pgfsetbuttcap%
\pgfsetroundjoin%
\definecolor{currentfill}{rgb}{0.000000,0.000000,0.000000}%
\pgfsetfillcolor{currentfill}%
\pgfsetlinewidth{0.602250pt}%
\definecolor{currentstroke}{rgb}{0.000000,0.000000,0.000000}%
\pgfsetstrokecolor{currentstroke}%
\pgfsetdash{}{0pt}%
\pgfsys@defobject{currentmarker}{\pgfqpoint{0.000000in}{-0.027778in}}{\pgfqpoint{0.000000in}{0.000000in}}{%
\pgfpathmoveto{\pgfqpoint{0.000000in}{0.000000in}}%
\pgfpathlineto{\pgfqpoint{0.000000in}{-0.027778in}}%
\pgfusepath{stroke,fill}%
}%
\begin{pgfscope}%
\pgfsys@transformshift{2.312746in}{0.417642in}%
\pgfsys@useobject{currentmarker}{}%
\end{pgfscope}%
\end{pgfscope}%
\begin{pgfscope}%
\pgfpathrectangle{\pgfqpoint{0.514278in}{0.417642in}}{\pgfqpoint{1.884052in}{1.370688in}}%
\pgfusepath{clip}%
\pgfsetrectcap%
\pgfsetroundjoin%
\pgfsetlinewidth{0.803000pt}%
\definecolor{currentstroke}{rgb}{0.850000,0.850000,0.850000}%
\pgfsetstrokecolor{currentstroke}%
\pgfsetdash{}{0pt}%
\pgfpathmoveto{\pgfqpoint{2.353705in}{0.417642in}}%
\pgfpathlineto{\pgfqpoint{2.353705in}{1.788330in}}%
\pgfusepath{stroke}%
\end{pgfscope}%
\begin{pgfscope}%
\pgfsetbuttcap%
\pgfsetroundjoin%
\definecolor{currentfill}{rgb}{0.000000,0.000000,0.000000}%
\pgfsetfillcolor{currentfill}%
\pgfsetlinewidth{0.602250pt}%
\definecolor{currentstroke}{rgb}{0.000000,0.000000,0.000000}%
\pgfsetstrokecolor{currentstroke}%
\pgfsetdash{}{0pt}%
\pgfsys@defobject{currentmarker}{\pgfqpoint{0.000000in}{-0.027778in}}{\pgfqpoint{0.000000in}{0.000000in}}{%
\pgfpathmoveto{\pgfqpoint{0.000000in}{0.000000in}}%
\pgfpathlineto{\pgfqpoint{0.000000in}{-0.027778in}}%
\pgfusepath{stroke,fill}%
}%
\begin{pgfscope}%
\pgfsys@transformshift{2.353705in}{0.417642in}%
\pgfsys@useobject{currentmarker}{}%
\end{pgfscope}%
\end{pgfscope}%
\begin{pgfscope}%
\pgfpathrectangle{\pgfqpoint{0.514278in}{0.417642in}}{\pgfqpoint{1.884052in}{1.370688in}}%
\pgfusepath{clip}%
\pgfsetrectcap%
\pgfsetroundjoin%
\pgfsetlinewidth{0.803000pt}%
\definecolor{currentstroke}{rgb}{0.850000,0.850000,0.850000}%
\pgfsetstrokecolor{currentstroke}%
\pgfsetdash{}{0pt}%
\pgfpathmoveto{\pgfqpoint{2.388335in}{0.417642in}}%
\pgfpathlineto{\pgfqpoint{2.388335in}{1.788330in}}%
\pgfusepath{stroke}%
\end{pgfscope}%
\begin{pgfscope}%
\pgfsetbuttcap%
\pgfsetroundjoin%
\definecolor{currentfill}{rgb}{0.000000,0.000000,0.000000}%
\pgfsetfillcolor{currentfill}%
\pgfsetlinewidth{0.602250pt}%
\definecolor{currentstroke}{rgb}{0.000000,0.000000,0.000000}%
\pgfsetstrokecolor{currentstroke}%
\pgfsetdash{}{0pt}%
\pgfsys@defobject{currentmarker}{\pgfqpoint{0.000000in}{-0.027778in}}{\pgfqpoint{0.000000in}{0.000000in}}{%
\pgfpathmoveto{\pgfqpoint{0.000000in}{0.000000in}}%
\pgfpathlineto{\pgfqpoint{0.000000in}{-0.027778in}}%
\pgfusepath{stroke,fill}%
}%
\begin{pgfscope}%
\pgfsys@transformshift{2.388335in}{0.417642in}%
\pgfsys@useobject{currentmarker}{}%
\end{pgfscope}%
\end{pgfscope}%
\begin{pgfscope}%
\definecolor{textcolor}{rgb}{0.000000,0.000000,0.000000}%
\pgfsetstrokecolor{textcolor}%
\pgfsetfillcolor{textcolor}%
\pgftext[x=1.456304in,y=0.165003in,,top]{\color{textcolor}\rmfamily\fontsize{10.000000}{12.000000}\selectfont Frequency in \(\displaystyle \unit{\Hz}\)}%
\end{pgfscope}%
\begin{pgfscope}%
\pgfpathrectangle{\pgfqpoint{0.514278in}{0.417642in}}{\pgfqpoint{1.884052in}{1.370688in}}%
\pgfusepath{clip}%
\pgfsetrectcap%
\pgfsetroundjoin%
\pgfsetlinewidth{0.803000pt}%
\definecolor{currentstroke}{rgb}{0.450000,0.450000,0.450000}%
\pgfsetstrokecolor{currentstroke}%
\pgfsetdash{}{0pt}%
\pgfpathmoveto{\pgfqpoint{0.514278in}{0.640555in}}%
\pgfpathlineto{\pgfqpoint{2.398330in}{0.640555in}}%
\pgfusepath{stroke}%
\end{pgfscope}%
\begin{pgfscope}%
\pgfsetbuttcap%
\pgfsetroundjoin%
\definecolor{currentfill}{rgb}{0.000000,0.000000,0.000000}%
\pgfsetfillcolor{currentfill}%
\pgfsetlinewidth{0.803000pt}%
\definecolor{currentstroke}{rgb}{0.000000,0.000000,0.000000}%
\pgfsetstrokecolor{currentstroke}%
\pgfsetdash{}{0pt}%
\pgfsys@defobject{currentmarker}{\pgfqpoint{-0.048611in}{0.000000in}}{\pgfqpoint{-0.000000in}{0.000000in}}{%
\pgfpathmoveto{\pgfqpoint{-0.000000in}{0.000000in}}%
\pgfpathlineto{\pgfqpoint{-0.048611in}{0.000000in}}%
\pgfusepath{stroke,fill}%
}%
\begin{pgfscope}%
\pgfsys@transformshift{0.514278in}{0.640555in}%
\pgfsys@useobject{currentmarker}{}%
\end{pgfscope}%
\end{pgfscope}%
\begin{pgfscope}%
\definecolor{textcolor}{rgb}{0.000000,0.000000,0.000000}%
\pgfsetstrokecolor{textcolor}%
\pgfsetfillcolor{textcolor}%
\pgftext[x=0.241129in, y=0.601402in, left, base]{\color{textcolor}\rmfamily\fontsize{8.000000}{9.600000}\selectfont \(\displaystyle {10^{0}}\)}%
\end{pgfscope}%
\begin{pgfscope}%
\pgfpathrectangle{\pgfqpoint{0.514278in}{0.417642in}}{\pgfqpoint{1.884052in}{1.370688in}}%
\pgfusepath{clip}%
\pgfsetrectcap%
\pgfsetroundjoin%
\pgfsetlinewidth{0.803000pt}%
\definecolor{currentstroke}{rgb}{0.450000,0.450000,0.450000}%
\pgfsetstrokecolor{currentstroke}%
\pgfsetdash{}{0pt}%
\pgfpathmoveto{\pgfqpoint{0.514278in}{0.983227in}}%
\pgfpathlineto{\pgfqpoint{2.398330in}{0.983227in}}%
\pgfusepath{stroke}%
\end{pgfscope}%
\begin{pgfscope}%
\pgfsetbuttcap%
\pgfsetroundjoin%
\definecolor{currentfill}{rgb}{0.000000,0.000000,0.000000}%
\pgfsetfillcolor{currentfill}%
\pgfsetlinewidth{0.803000pt}%
\definecolor{currentstroke}{rgb}{0.000000,0.000000,0.000000}%
\pgfsetstrokecolor{currentstroke}%
\pgfsetdash{}{0pt}%
\pgfsys@defobject{currentmarker}{\pgfqpoint{-0.048611in}{0.000000in}}{\pgfqpoint{-0.000000in}{0.000000in}}{%
\pgfpathmoveto{\pgfqpoint{-0.000000in}{0.000000in}}%
\pgfpathlineto{\pgfqpoint{-0.048611in}{0.000000in}}%
\pgfusepath{stroke,fill}%
}%
\begin{pgfscope}%
\pgfsys@transformshift{0.514278in}{0.983227in}%
\pgfsys@useobject{currentmarker}{}%
\end{pgfscope}%
\end{pgfscope}%
\begin{pgfscope}%
\definecolor{textcolor}{rgb}{0.000000,0.000000,0.000000}%
\pgfsetstrokecolor{textcolor}%
\pgfsetfillcolor{textcolor}%
\pgftext[x=0.241129in, y=0.944074in, left, base]{\color{textcolor}\rmfamily\fontsize{8.000000}{9.600000}\selectfont \(\displaystyle {10^{2}}\)}%
\end{pgfscope}%
\begin{pgfscope}%
\pgfpathrectangle{\pgfqpoint{0.514278in}{0.417642in}}{\pgfqpoint{1.884052in}{1.370688in}}%
\pgfusepath{clip}%
\pgfsetrectcap%
\pgfsetroundjoin%
\pgfsetlinewidth{0.803000pt}%
\definecolor{currentstroke}{rgb}{0.450000,0.450000,0.450000}%
\pgfsetstrokecolor{currentstroke}%
\pgfsetdash{}{0pt}%
\pgfpathmoveto{\pgfqpoint{0.514278in}{1.325899in}}%
\pgfpathlineto{\pgfqpoint{2.398330in}{1.325899in}}%
\pgfusepath{stroke}%
\end{pgfscope}%
\begin{pgfscope}%
\pgfsetbuttcap%
\pgfsetroundjoin%
\definecolor{currentfill}{rgb}{0.000000,0.000000,0.000000}%
\pgfsetfillcolor{currentfill}%
\pgfsetlinewidth{0.803000pt}%
\definecolor{currentstroke}{rgb}{0.000000,0.000000,0.000000}%
\pgfsetstrokecolor{currentstroke}%
\pgfsetdash{}{0pt}%
\pgfsys@defobject{currentmarker}{\pgfqpoint{-0.048611in}{0.000000in}}{\pgfqpoint{-0.000000in}{0.000000in}}{%
\pgfpathmoveto{\pgfqpoint{-0.000000in}{0.000000in}}%
\pgfpathlineto{\pgfqpoint{-0.048611in}{0.000000in}}%
\pgfusepath{stroke,fill}%
}%
\begin{pgfscope}%
\pgfsys@transformshift{0.514278in}{1.325899in}%
\pgfsys@useobject{currentmarker}{}%
\end{pgfscope}%
\end{pgfscope}%
\begin{pgfscope}%
\definecolor{textcolor}{rgb}{0.000000,0.000000,0.000000}%
\pgfsetstrokecolor{textcolor}%
\pgfsetfillcolor{textcolor}%
\pgftext[x=0.241129in, y=1.286746in, left, base]{\color{textcolor}\rmfamily\fontsize{8.000000}{9.600000}\selectfont \(\displaystyle {10^{4}}\)}%
\end{pgfscope}%
\begin{pgfscope}%
\pgfpathrectangle{\pgfqpoint{0.514278in}{0.417642in}}{\pgfqpoint{1.884052in}{1.370688in}}%
\pgfusepath{clip}%
\pgfsetrectcap%
\pgfsetroundjoin%
\pgfsetlinewidth{0.803000pt}%
\definecolor{currentstroke}{rgb}{0.450000,0.450000,0.450000}%
\pgfsetstrokecolor{currentstroke}%
\pgfsetdash{}{0pt}%
\pgfpathmoveto{\pgfqpoint{0.514278in}{1.668571in}}%
\pgfpathlineto{\pgfqpoint{2.398330in}{1.668571in}}%
\pgfusepath{stroke}%
\end{pgfscope}%
\begin{pgfscope}%
\pgfsetbuttcap%
\pgfsetroundjoin%
\definecolor{currentfill}{rgb}{0.000000,0.000000,0.000000}%
\pgfsetfillcolor{currentfill}%
\pgfsetlinewidth{0.803000pt}%
\definecolor{currentstroke}{rgb}{0.000000,0.000000,0.000000}%
\pgfsetstrokecolor{currentstroke}%
\pgfsetdash{}{0pt}%
\pgfsys@defobject{currentmarker}{\pgfqpoint{-0.048611in}{0.000000in}}{\pgfqpoint{-0.000000in}{0.000000in}}{%
\pgfpathmoveto{\pgfqpoint{-0.000000in}{0.000000in}}%
\pgfpathlineto{\pgfqpoint{-0.048611in}{0.000000in}}%
\pgfusepath{stroke,fill}%
}%
\begin{pgfscope}%
\pgfsys@transformshift{0.514278in}{1.668571in}%
\pgfsys@useobject{currentmarker}{}%
\end{pgfscope}%
\end{pgfscope}%
\begin{pgfscope}%
\definecolor{textcolor}{rgb}{0.000000,0.000000,0.000000}%
\pgfsetstrokecolor{textcolor}%
\pgfsetfillcolor{textcolor}%
\pgftext[x=0.241129in, y=1.629418in, left, base]{\color{textcolor}\rmfamily\fontsize{8.000000}{9.600000}\selectfont \(\displaystyle {10^{6}}\)}%
\end{pgfscope}%
\begin{pgfscope}%
\definecolor{textcolor}{rgb}{0.000000,0.000000,0.000000}%
\pgfsetstrokecolor{textcolor}%
\pgfsetfillcolor{textcolor}%
\pgftext[x=0.185574in,y=1.102986in,,bottom,rotate=90.000000]{\color{textcolor}\rmfamily\fontsize{10.000000}{12.000000}\selectfont  \(\displaystyle S_y(f)\) in \(\displaystyle \unit{1 \per \Hz}\)}%
\end{pgfscope}%
\begin{pgfscope}%
\pgfpathrectangle{\pgfqpoint{0.514278in}{0.417642in}}{\pgfqpoint{1.884052in}{1.370688in}}%
\pgfusepath{clip}%
\pgfsetbuttcap%
\pgfsetroundjoin%
\pgfsetlinewidth{1.505625pt}%
\definecolor{currentstroke}{rgb}{0.000000,0.447059,0.698039}%
\pgfsetstrokecolor{currentstroke}%
\pgfsetdash{{5.550000pt}{2.400000pt}}{0.000000pt}%
\pgfpathmoveto{\pgfqpoint{0.599917in}{0.692133in}}%
\pgfpathlineto{\pgfqpoint{2.312691in}{0.692133in}}%
\pgfpathlineto{\pgfqpoint{2.312691in}{0.692133in}}%
\pgfusepath{stroke}%
\end{pgfscope}%
\begin{pgfscope}%
\pgfpathrectangle{\pgfqpoint{0.514278in}{0.417642in}}{\pgfqpoint{1.884052in}{1.370688in}}%
\pgfusepath{clip}%
\pgfsetbuttcap%
\pgfsetroundjoin%
\definecolor{currentfill}{rgb}{0.000000,0.447059,0.698039}%
\pgfsetfillcolor{currentfill}%
\pgfsetlinewidth{1.003750pt}%
\definecolor{currentstroke}{rgb}{0.000000,0.447059,0.698039}%
\pgfsetstrokecolor{currentstroke}%
\pgfsetdash{}{0pt}%
\pgfsys@defobject{currentmarker}{\pgfqpoint{-0.006944in}{-0.006944in}}{\pgfqpoint{0.006944in}{0.006944in}}{%
\pgfpathmoveto{\pgfqpoint{0.000000in}{-0.006944in}}%
\pgfpathcurveto{\pgfqpoint{0.001842in}{-0.006944in}}{\pgfqpoint{0.003608in}{-0.006213in}}{\pgfqpoint{0.004910in}{-0.004910in}}%
\pgfpathcurveto{\pgfqpoint{0.006213in}{-0.003608in}}{\pgfqpoint{0.006944in}{-0.001842in}}{\pgfqpoint{0.006944in}{0.000000in}}%
\pgfpathcurveto{\pgfqpoint{0.006944in}{0.001842in}}{\pgfqpoint{0.006213in}{0.003608in}}{\pgfqpoint{0.004910in}{0.004910in}}%
\pgfpathcurveto{\pgfqpoint{0.003608in}{0.006213in}}{\pgfqpoint{0.001842in}{0.006944in}}{\pgfqpoint{0.000000in}{0.006944in}}%
\pgfpathcurveto{\pgfqpoint{-0.001842in}{0.006944in}}{\pgfqpoint{-0.003608in}{0.006213in}}{\pgfqpoint{-0.004910in}{0.004910in}}%
\pgfpathcurveto{\pgfqpoint{-0.006213in}{0.003608in}}{\pgfqpoint{-0.006944in}{0.001842in}}{\pgfqpoint{-0.006944in}{0.000000in}}%
\pgfpathcurveto{\pgfqpoint{-0.006944in}{-0.001842in}}{\pgfqpoint{-0.006213in}{-0.003608in}}{\pgfqpoint{-0.004910in}{-0.004910in}}%
\pgfpathcurveto{\pgfqpoint{-0.003608in}{-0.006213in}}{\pgfqpoint{-0.001842in}{-0.006944in}}{\pgfqpoint{0.000000in}{-0.006944in}}%
\pgfpathlineto{\pgfqpoint{0.000000in}{-0.006944in}}%
\pgfpathclose%
\pgfusepath{stroke,fill}%
}%
\begin{pgfscope}%
\pgfsys@transformshift{-226.701573in}{0.624720in}%
\pgfsys@useobject{currentmarker}{}%
\end{pgfscope}%
\begin{pgfscope}%
\pgfsys@transformshift{0.599917in}{0.663678in}%
\pgfsys@useobject{currentmarker}{}%
\end{pgfscope}%
\begin{pgfscope}%
\pgfsys@transformshift{0.755634in}{0.618518in}%
\pgfsys@useobject{currentmarker}{}%
\end{pgfscope}%
\begin{pgfscope}%
\pgfsys@transformshift{0.846722in}{0.651871in}%
\pgfsys@useobject{currentmarker}{}%
\end{pgfscope}%
\begin{pgfscope}%
\pgfsys@transformshift{0.911351in}{0.646106in}%
\pgfsys@useobject{currentmarker}{}%
\end{pgfscope}%
\begin{pgfscope}%
\pgfsys@transformshift{0.961480in}{0.650617in}%
\pgfsys@useobject{currentmarker}{}%
\end{pgfscope}%
\begin{pgfscope}%
\pgfsys@transformshift{1.002439in}{0.619497in}%
\pgfsys@useobject{currentmarker}{}%
\end{pgfscope}%
\begin{pgfscope}%
\pgfsys@transformshift{1.037069in}{0.703448in}%
\pgfsys@useobject{currentmarker}{}%
\end{pgfscope}%
\begin{pgfscope}%
\pgfsys@transformshift{1.067067in}{0.721385in}%
\pgfsys@useobject{currentmarker}{}%
\end{pgfscope}%
\begin{pgfscope}%
\pgfsys@transformshift{1.093527in}{0.711558in}%
\pgfsys@useobject{currentmarker}{}%
\end{pgfscope}%
\begin{pgfscope}%
\pgfsys@transformshift{1.117197in}{0.701313in}%
\pgfsys@useobject{currentmarker}{}%
\end{pgfscope}%
\begin{pgfscope}%
\pgfsys@transformshift{1.138608in}{0.674215in}%
\pgfsys@useobject{currentmarker}{}%
\end{pgfscope}%
\begin{pgfscope}%
\pgfsys@transformshift{1.158156in}{0.714359in}%
\pgfsys@useobject{currentmarker}{}%
\end{pgfscope}%
\begin{pgfscope}%
\pgfsys@transformshift{1.176137in}{0.690859in}%
\pgfsys@useobject{currentmarker}{}%
\end{pgfscope}%
\begin{pgfscope}%
\pgfsys@transformshift{1.192786in}{0.630191in}%
\pgfsys@useobject{currentmarker}{}%
\end{pgfscope}%
\begin{pgfscope}%
\pgfsys@transformshift{1.208285in}{0.701226in}%
\pgfsys@useobject{currentmarker}{}%
\end{pgfscope}%
\begin{pgfscope}%
\pgfsys@transformshift{1.222784in}{0.729309in}%
\pgfsys@useobject{currentmarker}{}%
\end{pgfscope}%
\begin{pgfscope}%
\pgfsys@transformshift{1.236403in}{0.708058in}%
\pgfsys@useobject{currentmarker}{}%
\end{pgfscope}%
\begin{pgfscope}%
\pgfsys@transformshift{1.249244in}{0.605219in}%
\pgfsys@useobject{currentmarker}{}%
\end{pgfscope}%
\begin{pgfscope}%
\pgfsys@transformshift{1.261390in}{0.661534in}%
\pgfsys@useobject{currentmarker}{}%
\end{pgfscope}%
\begin{pgfscope}%
\pgfsys@transformshift{1.272914in}{0.731254in}%
\pgfsys@useobject{currentmarker}{}%
\end{pgfscope}%
\begin{pgfscope}%
\pgfsys@transformshift{1.283874in}{0.748974in}%
\pgfsys@useobject{currentmarker}{}%
\end{pgfscope}%
\begin{pgfscope}%
\pgfsys@transformshift{1.294325in}{0.714574in}%
\pgfsys@useobject{currentmarker}{}%
\end{pgfscope}%
\begin{pgfscope}%
\pgfsys@transformshift{1.304311in}{0.694932in}%
\pgfsys@useobject{currentmarker}{}%
\end{pgfscope}%
\begin{pgfscope}%
\pgfsys@transformshift{1.313872in}{0.651474in}%
\pgfsys@useobject{currentmarker}{}%
\end{pgfscope}%
\begin{pgfscope}%
\pgfsys@transformshift{1.323043in}{0.677275in}%
\pgfsys@useobject{currentmarker}{}%
\end{pgfscope}%
\begin{pgfscope}%
\pgfsys@transformshift{1.331854in}{0.659492in}%
\pgfsys@useobject{currentmarker}{}%
\end{pgfscope}%
\begin{pgfscope}%
\pgfsys@transformshift{1.340333in}{0.657028in}%
\pgfsys@useobject{currentmarker}{}%
\end{pgfscope}%
\begin{pgfscope}%
\pgfsys@transformshift{1.348503in}{0.706173in}%
\pgfsys@useobject{currentmarker}{}%
\end{pgfscope}%
\begin{pgfscope}%
\pgfsys@transformshift{1.356386in}{0.713395in}%
\pgfsys@useobject{currentmarker}{}%
\end{pgfscope}%
\begin{pgfscope}%
\pgfsys@transformshift{1.364002in}{0.673664in}%
\pgfsys@useobject{currentmarker}{}%
\end{pgfscope}%
\begin{pgfscope}%
\pgfsys@transformshift{1.371368in}{0.699535in}%
\pgfsys@useobject{currentmarker}{}%
\end{pgfscope}%
\begin{pgfscope}%
\pgfsys@transformshift{1.378501in}{0.685568in}%
\pgfsys@useobject{currentmarker}{}%
\end{pgfscope}%
\begin{pgfscope}%
\pgfsys@transformshift{1.385414in}{0.673357in}%
\pgfsys@useobject{currentmarker}{}%
\end{pgfscope}%
\begin{pgfscope}%
\pgfsys@transformshift{1.392120in}{0.711047in}%
\pgfsys@useobject{currentmarker}{}%
\end{pgfscope}%
\begin{pgfscope}%
\pgfsys@transformshift{1.398632in}{0.674407in}%
\pgfsys@useobject{currentmarker}{}%
\end{pgfscope}%
\begin{pgfscope}%
\pgfsys@transformshift{1.404961in}{0.709521in}%
\pgfsys@useobject{currentmarker}{}%
\end{pgfscope}%
\begin{pgfscope}%
\pgfsys@transformshift{1.411116in}{0.699352in}%
\pgfsys@useobject{currentmarker}{}%
\end{pgfscope}%
\begin{pgfscope}%
\pgfsys@transformshift{1.417107in}{0.670861in}%
\pgfsys@useobject{currentmarker}{}%
\end{pgfscope}%
\begin{pgfscope}%
\pgfsys@transformshift{1.422943in}{0.630255in}%
\pgfsys@useobject{currentmarker}{}%
\end{pgfscope}%
\begin{pgfscope}%
\pgfsys@transformshift{1.428630in}{0.680685in}%
\pgfsys@useobject{currentmarker}{}%
\end{pgfscope}%
\begin{pgfscope}%
\pgfsys@transformshift{1.434178in}{0.708422in}%
\pgfsys@useobject{currentmarker}{}%
\end{pgfscope}%
\begin{pgfscope}%
\pgfsys@transformshift{1.439591in}{0.724003in}%
\pgfsys@useobject{currentmarker}{}%
\end{pgfscope}%
\begin{pgfscope}%
\pgfsys@transformshift{1.444877in}{0.740658in}%
\pgfsys@useobject{currentmarker}{}%
\end{pgfscope}%
\begin{pgfscope}%
\pgfsys@transformshift{1.450042in}{0.691679in}%
\pgfsys@useobject{currentmarker}{}%
\end{pgfscope}%
\begin{pgfscope}%
\pgfsys@transformshift{1.455090in}{0.615527in}%
\pgfsys@useobject{currentmarker}{}%
\end{pgfscope}%
\begin{pgfscope}%
\pgfsys@transformshift{1.460028in}{0.645323in}%
\pgfsys@useobject{currentmarker}{}%
\end{pgfscope}%
\begin{pgfscope}%
\pgfsys@transformshift{1.464859in}{0.669445in}%
\pgfsys@useobject{currentmarker}{}%
\end{pgfscope}%
\begin{pgfscope}%
\pgfsys@transformshift{1.469589in}{0.698805in}%
\pgfsys@useobject{currentmarker}{}%
\end{pgfscope}%
\begin{pgfscope}%
\pgfsys@transformshift{1.474221in}{0.686711in}%
\pgfsys@useobject{currentmarker}{}%
\end{pgfscope}%
\begin{pgfscope}%
\pgfsys@transformshift{1.478760in}{0.640861in}%
\pgfsys@useobject{currentmarker}{}%
\end{pgfscope}%
\begin{pgfscope}%
\pgfsys@transformshift{1.483209in}{0.584360in}%
\pgfsys@useobject{currentmarker}{}%
\end{pgfscope}%
\begin{pgfscope}%
\pgfsys@transformshift{1.487571in}{0.626187in}%
\pgfsys@useobject{currentmarker}{}%
\end{pgfscope}%
\begin{pgfscope}%
\pgfsys@transformshift{1.491850in}{0.693923in}%
\pgfsys@useobject{currentmarker}{}%
\end{pgfscope}%
\begin{pgfscope}%
\pgfsys@transformshift{1.496049in}{0.697953in}%
\pgfsys@useobject{currentmarker}{}%
\end{pgfscope}%
\begin{pgfscope}%
\pgfsys@transformshift{1.500172in}{0.709626in}%
\pgfsys@useobject{currentmarker}{}%
\end{pgfscope}%
\begin{pgfscope}%
\pgfsys@transformshift{1.504219in}{0.716719in}%
\pgfsys@useobject{currentmarker}{}%
\end{pgfscope}%
\begin{pgfscope}%
\pgfsys@transformshift{1.508196in}{0.728854in}%
\pgfsys@useobject{currentmarker}{}%
\end{pgfscope}%
\begin{pgfscope}%
\pgfsys@transformshift{1.512103in}{0.722589in}%
\pgfsys@useobject{currentmarker}{}%
\end{pgfscope}%
\begin{pgfscope}%
\pgfsys@transformshift{1.515943in}{0.701960in}%
\pgfsys@useobject{currentmarker}{}%
\end{pgfscope}%
\begin{pgfscope}%
\pgfsys@transformshift{1.519719in}{0.683477in}%
\pgfsys@useobject{currentmarker}{}%
\end{pgfscope}%
\begin{pgfscope}%
\pgfsys@transformshift{1.523432in}{0.677670in}%
\pgfsys@useobject{currentmarker}{}%
\end{pgfscope}%
\begin{pgfscope}%
\pgfsys@transformshift{1.527085in}{0.623546in}%
\pgfsys@useobject{currentmarker}{}%
\end{pgfscope}%
\begin{pgfscope}%
\pgfsys@transformshift{1.530680in}{0.633649in}%
\pgfsys@useobject{currentmarker}{}%
\end{pgfscope}%
\begin{pgfscope}%
\pgfsys@transformshift{1.534217in}{0.669744in}%
\pgfsys@useobject{currentmarker}{}%
\end{pgfscope}%
\begin{pgfscope}%
\pgfsys@transformshift{1.537700in}{0.642444in}%
\pgfsys@useobject{currentmarker}{}%
\end{pgfscope}%
\begin{pgfscope}%
\pgfsys@transformshift{1.541130in}{0.643139in}%
\pgfsys@useobject{currentmarker}{}%
\end{pgfscope}%
\begin{pgfscope}%
\pgfsys@transformshift{1.544509in}{0.675283in}%
\pgfsys@useobject{currentmarker}{}%
\end{pgfscope}%
\begin{pgfscope}%
\pgfsys@transformshift{1.547837in}{0.709297in}%
\pgfsys@useobject{currentmarker}{}%
\end{pgfscope}%
\begin{pgfscope}%
\pgfsys@transformshift{1.551117in}{0.700611in}%
\pgfsys@useobject{currentmarker}{}%
\end{pgfscope}%
\begin{pgfscope}%
\pgfsys@transformshift{1.554349in}{0.634289in}%
\pgfsys@useobject{currentmarker}{}%
\end{pgfscope}%
\begin{pgfscope}%
\pgfsys@transformshift{1.557536in}{0.658252in}%
\pgfsys@useobject{currentmarker}{}%
\end{pgfscope}%
\begin{pgfscope}%
\pgfsys@transformshift{1.560678in}{0.709045in}%
\pgfsys@useobject{currentmarker}{}%
\end{pgfscope}%
\begin{pgfscope}%
\pgfsys@transformshift{1.563776in}{0.712386in}%
\pgfsys@useobject{currentmarker}{}%
\end{pgfscope}%
\begin{pgfscope}%
\pgfsys@transformshift{1.566833in}{0.675398in}%
\pgfsys@useobject{currentmarker}{}%
\end{pgfscope}%
\begin{pgfscope}%
\pgfsys@transformshift{1.569848in}{0.681361in}%
\pgfsys@useobject{currentmarker}{}%
\end{pgfscope}%
\begin{pgfscope}%
\pgfsys@transformshift{1.572824in}{0.702359in}%
\pgfsys@useobject{currentmarker}{}%
\end{pgfscope}%
\begin{pgfscope}%
\pgfsys@transformshift{1.575761in}{0.681252in}%
\pgfsys@useobject{currentmarker}{}%
\end{pgfscope}%
\begin{pgfscope}%
\pgfsys@transformshift{1.578659in}{0.633352in}%
\pgfsys@useobject{currentmarker}{}%
\end{pgfscope}%
\begin{pgfscope}%
\pgfsys@transformshift{1.581521in}{0.642407in}%
\pgfsys@useobject{currentmarker}{}%
\end{pgfscope}%
\begin{pgfscope}%
\pgfsys@transformshift{1.584347in}{0.695630in}%
\pgfsys@useobject{currentmarker}{}%
\end{pgfscope}%
\begin{pgfscope}%
\pgfsys@transformshift{1.587138in}{0.685366in}%
\pgfsys@useobject{currentmarker}{}%
\end{pgfscope}%
\begin{pgfscope}%
\pgfsys@transformshift{1.589894in}{0.673601in}%
\pgfsys@useobject{currentmarker}{}%
\end{pgfscope}%
\begin{pgfscope}%
\pgfsys@transformshift{1.592617in}{0.655245in}%
\pgfsys@useobject{currentmarker}{}%
\end{pgfscope}%
\begin{pgfscope}%
\pgfsys@transformshift{1.595308in}{0.637186in}%
\pgfsys@useobject{currentmarker}{}%
\end{pgfscope}%
\begin{pgfscope}%
\pgfsys@transformshift{1.597966in}{0.636657in}%
\pgfsys@useobject{currentmarker}{}%
\end{pgfscope}%
\begin{pgfscope}%
\pgfsys@transformshift{1.600594in}{0.674325in}%
\pgfsys@useobject{currentmarker}{}%
\end{pgfscope}%
\begin{pgfscope}%
\pgfsys@transformshift{1.603191in}{0.639240in}%
\pgfsys@useobject{currentmarker}{}%
\end{pgfscope}%
\begin{pgfscope}%
\pgfsys@transformshift{1.605759in}{0.661485in}%
\pgfsys@useobject{currentmarker}{}%
\end{pgfscope}%
\begin{pgfscope}%
\pgfsys@transformshift{1.608297in}{0.615765in}%
\pgfsys@useobject{currentmarker}{}%
\end{pgfscope}%
\begin{pgfscope}%
\pgfsys@transformshift{1.610807in}{0.622078in}%
\pgfsys@useobject{currentmarker}{}%
\end{pgfscope}%
\begin{pgfscope}%
\pgfsys@transformshift{1.613290in}{0.691726in}%
\pgfsys@useobject{currentmarker}{}%
\end{pgfscope}%
\begin{pgfscope}%
\pgfsys@transformshift{1.615745in}{0.683584in}%
\pgfsys@useobject{currentmarker}{}%
\end{pgfscope}%
\begin{pgfscope}%
\pgfsys@transformshift{1.618173in}{0.691561in}%
\pgfsys@useobject{currentmarker}{}%
\end{pgfscope}%
\begin{pgfscope}%
\pgfsys@transformshift{1.620576in}{0.680920in}%
\pgfsys@useobject{currentmarker}{}%
\end{pgfscope}%
\begin{pgfscope}%
\pgfsys@transformshift{1.622953in}{0.661756in}%
\pgfsys@useobject{currentmarker}{}%
\end{pgfscope}%
\begin{pgfscope}%
\pgfsys@transformshift{1.625306in}{0.679807in}%
\pgfsys@useobject{currentmarker}{}%
\end{pgfscope}%
\begin{pgfscope}%
\pgfsys@transformshift{1.627634in}{0.665479in}%
\pgfsys@useobject{currentmarker}{}%
\end{pgfscope}%
\begin{pgfscope}%
\pgfsys@transformshift{1.629938in}{0.650788in}%
\pgfsys@useobject{currentmarker}{}%
\end{pgfscope}%
\begin{pgfscope}%
\pgfsys@transformshift{1.632219in}{0.671671in}%
\pgfsys@useobject{currentmarker}{}%
\end{pgfscope}%
\begin{pgfscope}%
\pgfsys@transformshift{1.634477in}{0.654724in}%
\pgfsys@useobject{currentmarker}{}%
\end{pgfscope}%
\begin{pgfscope}%
\pgfsys@transformshift{1.636712in}{0.628973in}%
\pgfsys@useobject{currentmarker}{}%
\end{pgfscope}%
\begin{pgfscope}%
\pgfsys@transformshift{1.638925in}{0.662926in}%
\pgfsys@useobject{currentmarker}{}%
\end{pgfscope}%
\begin{pgfscope}%
\pgfsys@transformshift{1.641117in}{0.671421in}%
\pgfsys@useobject{currentmarker}{}%
\end{pgfscope}%
\begin{pgfscope}%
\pgfsys@transformshift{1.643288in}{0.710784in}%
\pgfsys@useobject{currentmarker}{}%
\end{pgfscope}%
\begin{pgfscope}%
\pgfsys@transformshift{1.645437in}{0.675700in}%
\pgfsys@useobject{currentmarker}{}%
\end{pgfscope}%
\begin{pgfscope}%
\pgfsys@transformshift{1.647567in}{0.679357in}%
\pgfsys@useobject{currentmarker}{}%
\end{pgfscope}%
\begin{pgfscope}%
\pgfsys@transformshift{1.649676in}{0.691826in}%
\pgfsys@useobject{currentmarker}{}%
\end{pgfscope}%
\begin{pgfscope}%
\pgfsys@transformshift{1.651766in}{0.697662in}%
\pgfsys@useobject{currentmarker}{}%
\end{pgfscope}%
\begin{pgfscope}%
\pgfsys@transformshift{1.653837in}{0.714151in}%
\pgfsys@useobject{currentmarker}{}%
\end{pgfscope}%
\begin{pgfscope}%
\pgfsys@transformshift{1.655888in}{0.702506in}%
\pgfsys@useobject{currentmarker}{}%
\end{pgfscope}%
\begin{pgfscope}%
\pgfsys@transformshift{1.657921in}{0.668027in}%
\pgfsys@useobject{currentmarker}{}%
\end{pgfscope}%
\begin{pgfscope}%
\pgfsys@transformshift{1.659936in}{0.695746in}%
\pgfsys@useobject{currentmarker}{}%
\end{pgfscope}%
\begin{pgfscope}%
\pgfsys@transformshift{1.661933in}{0.677445in}%
\pgfsys@useobject{currentmarker}{}%
\end{pgfscope}%
\begin{pgfscope}%
\pgfsys@transformshift{1.663912in}{0.676534in}%
\pgfsys@useobject{currentmarker}{}%
\end{pgfscope}%
\begin{pgfscope}%
\pgfsys@transformshift{1.665874in}{0.675255in}%
\pgfsys@useobject{currentmarker}{}%
\end{pgfscope}%
\begin{pgfscope}%
\pgfsys@transformshift{1.667819in}{0.635969in}%
\pgfsys@useobject{currentmarker}{}%
\end{pgfscope}%
\begin{pgfscope}%
\pgfsys@transformshift{1.669748in}{0.659485in}%
\pgfsys@useobject{currentmarker}{}%
\end{pgfscope}%
\begin{pgfscope}%
\pgfsys@transformshift{1.671660in}{0.705846in}%
\pgfsys@useobject{currentmarker}{}%
\end{pgfscope}%
\begin{pgfscope}%
\pgfsys@transformshift{1.673556in}{0.693948in}%
\pgfsys@useobject{currentmarker}{}%
\end{pgfscope}%
\begin{pgfscope}%
\pgfsys@transformshift{1.675435in}{0.665499in}%
\pgfsys@useobject{currentmarker}{}%
\end{pgfscope}%
\begin{pgfscope}%
\pgfsys@transformshift{1.677300in}{0.667706in}%
\pgfsys@useobject{currentmarker}{}%
\end{pgfscope}%
\begin{pgfscope}%
\pgfsys@transformshift{1.679149in}{0.686616in}%
\pgfsys@useobject{currentmarker}{}%
\end{pgfscope}%
\begin{pgfscope}%
\pgfsys@transformshift{1.680983in}{0.654105in}%
\pgfsys@useobject{currentmarker}{}%
\end{pgfscope}%
\begin{pgfscope}%
\pgfsys@transformshift{1.682802in}{0.685246in}%
\pgfsys@useobject{currentmarker}{}%
\end{pgfscope}%
\begin{pgfscope}%
\pgfsys@transformshift{1.684606in}{0.677587in}%
\pgfsys@useobject{currentmarker}{}%
\end{pgfscope}%
\begin{pgfscope}%
\pgfsys@transformshift{1.686396in}{0.683820in}%
\pgfsys@useobject{currentmarker}{}%
\end{pgfscope}%
\begin{pgfscope}%
\pgfsys@transformshift{1.688172in}{0.667612in}%
\pgfsys@useobject{currentmarker}{}%
\end{pgfscope}%
\begin{pgfscope}%
\pgfsys@transformshift{1.689934in}{0.681745in}%
\pgfsys@useobject{currentmarker}{}%
\end{pgfscope}%
\begin{pgfscope}%
\pgfsys@transformshift{1.691682in}{0.680677in}%
\pgfsys@useobject{currentmarker}{}%
\end{pgfscope}%
\begin{pgfscope}%
\pgfsys@transformshift{1.693417in}{0.654107in}%
\pgfsys@useobject{currentmarker}{}%
\end{pgfscope}%
\begin{pgfscope}%
\pgfsys@transformshift{1.695139in}{0.694216in}%
\pgfsys@useobject{currentmarker}{}%
\end{pgfscope}%
\begin{pgfscope}%
\pgfsys@transformshift{1.696847in}{0.708866in}%
\pgfsys@useobject{currentmarker}{}%
\end{pgfscope}%
\begin{pgfscope}%
\pgfsys@transformshift{1.698543in}{0.681582in}%
\pgfsys@useobject{currentmarker}{}%
\end{pgfscope}%
\begin{pgfscope}%
\pgfsys@transformshift{1.700225in}{0.684852in}%
\pgfsys@useobject{currentmarker}{}%
\end{pgfscope}%
\begin{pgfscope}%
\pgfsys@transformshift{1.701896in}{0.716700in}%
\pgfsys@useobject{currentmarker}{}%
\end{pgfscope}%
\begin{pgfscope}%
\pgfsys@transformshift{1.703554in}{0.706415in}%
\pgfsys@useobject{currentmarker}{}%
\end{pgfscope}%
\begin{pgfscope}%
\pgfsys@transformshift{1.705199in}{0.744503in}%
\pgfsys@useobject{currentmarker}{}%
\end{pgfscope}%
\begin{pgfscope}%
\pgfsys@transformshift{1.706833in}{0.729507in}%
\pgfsys@useobject{currentmarker}{}%
\end{pgfscope}%
\begin{pgfscope}%
\pgfsys@transformshift{1.708455in}{0.698267in}%
\pgfsys@useobject{currentmarker}{}%
\end{pgfscope}%
\begin{pgfscope}%
\pgfsys@transformshift{1.710066in}{0.705614in}%
\pgfsys@useobject{currentmarker}{}%
\end{pgfscope}%
\begin{pgfscope}%
\pgfsys@transformshift{1.711665in}{0.673054in}%
\pgfsys@useobject{currentmarker}{}%
\end{pgfscope}%
\begin{pgfscope}%
\pgfsys@transformshift{1.713252in}{0.743824in}%
\pgfsys@useobject{currentmarker}{}%
\end{pgfscope}%
\begin{pgfscope}%
\pgfsys@transformshift{1.714829in}{0.697156in}%
\pgfsys@useobject{currentmarker}{}%
\end{pgfscope}%
\begin{pgfscope}%
\pgfsys@transformshift{1.716394in}{0.703619in}%
\pgfsys@useobject{currentmarker}{}%
\end{pgfscope}%
\begin{pgfscope}%
\pgfsys@transformshift{1.717949in}{0.635238in}%
\pgfsys@useobject{currentmarker}{}%
\end{pgfscope}%
\begin{pgfscope}%
\pgfsys@transformshift{1.719493in}{0.594479in}%
\pgfsys@useobject{currentmarker}{}%
\end{pgfscope}%
\begin{pgfscope}%
\pgfsys@transformshift{1.721026in}{0.654857in}%
\pgfsys@useobject{currentmarker}{}%
\end{pgfscope}%
\begin{pgfscope}%
\pgfsys@transformshift{1.722550in}{0.689086in}%
\pgfsys@useobject{currentmarker}{}%
\end{pgfscope}%
\begin{pgfscope}%
\pgfsys@transformshift{1.724062in}{0.702642in}%
\pgfsys@useobject{currentmarker}{}%
\end{pgfscope}%
\begin{pgfscope}%
\pgfsys@transformshift{1.725565in}{0.726628in}%
\pgfsys@useobject{currentmarker}{}%
\end{pgfscope}%
\begin{pgfscope}%
\pgfsys@transformshift{1.727058in}{0.690670in}%
\pgfsys@useobject{currentmarker}{}%
\end{pgfscope}%
\begin{pgfscope}%
\pgfsys@transformshift{1.728541in}{0.690783in}%
\pgfsys@useobject{currentmarker}{}%
\end{pgfscope}%
\begin{pgfscope}%
\pgfsys@transformshift{1.730014in}{0.704044in}%
\pgfsys@useobject{currentmarker}{}%
\end{pgfscope}%
\begin{pgfscope}%
\pgfsys@transformshift{1.731477in}{0.655639in}%
\pgfsys@useobject{currentmarker}{}%
\end{pgfscope}%
\begin{pgfscope}%
\pgfsys@transformshift{1.732931in}{0.679834in}%
\pgfsys@useobject{currentmarker}{}%
\end{pgfscope}%
\begin{pgfscope}%
\pgfsys@transformshift{1.734376in}{0.693864in}%
\pgfsys@useobject{currentmarker}{}%
\end{pgfscope}%
\begin{pgfscope}%
\pgfsys@transformshift{1.735812in}{0.725502in}%
\pgfsys@useobject{currentmarker}{}%
\end{pgfscope}%
\begin{pgfscope}%
\pgfsys@transformshift{1.737238in}{0.741002in}%
\pgfsys@useobject{currentmarker}{}%
\end{pgfscope}%
\begin{pgfscope}%
\pgfsys@transformshift{1.738655in}{0.720827in}%
\pgfsys@useobject{currentmarker}{}%
\end{pgfscope}%
\begin{pgfscope}%
\pgfsys@transformshift{1.740064in}{0.657793in}%
\pgfsys@useobject{currentmarker}{}%
\end{pgfscope}%
\begin{pgfscope}%
\pgfsys@transformshift{1.741463in}{0.587835in}%
\pgfsys@useobject{currentmarker}{}%
\end{pgfscope}%
\begin{pgfscope}%
\pgfsys@transformshift{1.742854in}{0.650198in}%
\pgfsys@useobject{currentmarker}{}%
\end{pgfscope}%
\begin{pgfscope}%
\pgfsys@transformshift{1.744237in}{0.682555in}%
\pgfsys@useobject{currentmarker}{}%
\end{pgfscope}%
\begin{pgfscope}%
\pgfsys@transformshift{1.745611in}{0.716846in}%
\pgfsys@useobject{currentmarker}{}%
\end{pgfscope}%
\begin{pgfscope}%
\pgfsys@transformshift{1.746977in}{0.736044in}%
\pgfsys@useobject{currentmarker}{}%
\end{pgfscope}%
\begin{pgfscope}%
\pgfsys@transformshift{1.748334in}{0.699158in}%
\pgfsys@useobject{currentmarker}{}%
\end{pgfscope}%
\begin{pgfscope}%
\pgfsys@transformshift{1.749683in}{0.648137in}%
\pgfsys@useobject{currentmarker}{}%
\end{pgfscope}%
\begin{pgfscope}%
\pgfsys@transformshift{1.751025in}{0.649598in}%
\pgfsys@useobject{currentmarker}{}%
\end{pgfscope}%
\begin{pgfscope}%
\pgfsys@transformshift{1.752358in}{0.613845in}%
\pgfsys@useobject{currentmarker}{}%
\end{pgfscope}%
\begin{pgfscope}%
\pgfsys@transformshift{1.753683in}{0.626339in}%
\pgfsys@useobject{currentmarker}{}%
\end{pgfscope}%
\begin{pgfscope}%
\pgfsys@transformshift{1.755001in}{0.637002in}%
\pgfsys@useobject{currentmarker}{}%
\end{pgfscope}%
\begin{pgfscope}%
\pgfsys@transformshift{1.756311in}{0.664409in}%
\pgfsys@useobject{currentmarker}{}%
\end{pgfscope}%
\begin{pgfscope}%
\pgfsys@transformshift{1.757613in}{0.645225in}%
\pgfsys@useobject{currentmarker}{}%
\end{pgfscope}%
\begin{pgfscope}%
\pgfsys@transformshift{1.758908in}{0.594307in}%
\pgfsys@useobject{currentmarker}{}%
\end{pgfscope}%
\begin{pgfscope}%
\pgfsys@transformshift{1.760195in}{0.577254in}%
\pgfsys@useobject{currentmarker}{}%
\end{pgfscope}%
\begin{pgfscope}%
\pgfsys@transformshift{1.761475in}{0.650974in}%
\pgfsys@useobject{currentmarker}{}%
\end{pgfscope}%
\begin{pgfscope}%
\pgfsys@transformshift{1.762748in}{0.705925in}%
\pgfsys@useobject{currentmarker}{}%
\end{pgfscope}%
\begin{pgfscope}%
\pgfsys@transformshift{1.764014in}{0.684145in}%
\pgfsys@useobject{currentmarker}{}%
\end{pgfscope}%
\begin{pgfscope}%
\pgfsys@transformshift{1.765272in}{0.677040in}%
\pgfsys@useobject{currentmarker}{}%
\end{pgfscope}%
\begin{pgfscope}%
\pgfsys@transformshift{1.766524in}{0.718769in}%
\pgfsys@useobject{currentmarker}{}%
\end{pgfscope}%
\begin{pgfscope}%
\pgfsys@transformshift{1.767769in}{0.725821in}%
\pgfsys@useobject{currentmarker}{}%
\end{pgfscope}%
\begin{pgfscope}%
\pgfsys@transformshift{1.769006in}{0.705053in}%
\pgfsys@useobject{currentmarker}{}%
\end{pgfscope}%
\begin{pgfscope}%
\pgfsys@transformshift{1.770237in}{0.680563in}%
\pgfsys@useobject{currentmarker}{}%
\end{pgfscope}%
\begin{pgfscope}%
\pgfsys@transformshift{1.771462in}{0.687243in}%
\pgfsys@useobject{currentmarker}{}%
\end{pgfscope}%
\begin{pgfscope}%
\pgfsys@transformshift{1.772679in}{0.718924in}%
\pgfsys@useobject{currentmarker}{}%
\end{pgfscope}%
\begin{pgfscope}%
\pgfsys@transformshift{1.773890in}{0.690900in}%
\pgfsys@useobject{currentmarker}{}%
\end{pgfscope}%
\begin{pgfscope}%
\pgfsys@transformshift{1.775095in}{0.685715in}%
\pgfsys@useobject{currentmarker}{}%
\end{pgfscope}%
\begin{pgfscope}%
\pgfsys@transformshift{1.776293in}{0.728935in}%
\pgfsys@useobject{currentmarker}{}%
\end{pgfscope}%
\begin{pgfscope}%
\pgfsys@transformshift{1.777485in}{0.711548in}%
\pgfsys@useobject{currentmarker}{}%
\end{pgfscope}%
\begin{pgfscope}%
\pgfsys@transformshift{1.778670in}{0.676584in}%
\pgfsys@useobject{currentmarker}{}%
\end{pgfscope}%
\begin{pgfscope}%
\pgfsys@transformshift{1.779849in}{0.733642in}%
\pgfsys@useobject{currentmarker}{}%
\end{pgfscope}%
\begin{pgfscope}%
\pgfsys@transformshift{1.781023in}{0.723508in}%
\pgfsys@useobject{currentmarker}{}%
\end{pgfscope}%
\begin{pgfscope}%
\pgfsys@transformshift{1.782190in}{0.678302in}%
\pgfsys@useobject{currentmarker}{}%
\end{pgfscope}%
\begin{pgfscope}%
\pgfsys@transformshift{1.783351in}{0.692014in}%
\pgfsys@useobject{currentmarker}{}%
\end{pgfscope}%
\begin{pgfscope}%
\pgfsys@transformshift{1.784506in}{0.719710in}%
\pgfsys@useobject{currentmarker}{}%
\end{pgfscope}%
\begin{pgfscope}%
\pgfsys@transformshift{1.785655in}{0.640502in}%
\pgfsys@useobject{currentmarker}{}%
\end{pgfscope}%
\begin{pgfscope}%
\pgfsys@transformshift{1.786798in}{0.630401in}%
\pgfsys@useobject{currentmarker}{}%
\end{pgfscope}%
\begin{pgfscope}%
\pgfsys@transformshift{1.787936in}{0.702047in}%
\pgfsys@useobject{currentmarker}{}%
\end{pgfscope}%
\begin{pgfscope}%
\pgfsys@transformshift{1.789067in}{0.730444in}%
\pgfsys@useobject{currentmarker}{}%
\end{pgfscope}%
\begin{pgfscope}%
\pgfsys@transformshift{1.790193in}{0.735588in}%
\pgfsys@useobject{currentmarker}{}%
\end{pgfscope}%
\begin{pgfscope}%
\pgfsys@transformshift{1.791314in}{0.761941in}%
\pgfsys@useobject{currentmarker}{}%
\end{pgfscope}%
\begin{pgfscope}%
\pgfsys@transformshift{1.792429in}{0.762888in}%
\pgfsys@useobject{currentmarker}{}%
\end{pgfscope}%
\begin{pgfscope}%
\pgfsys@transformshift{1.793538in}{0.712199in}%
\pgfsys@useobject{currentmarker}{}%
\end{pgfscope}%
\begin{pgfscope}%
\pgfsys@transformshift{1.794642in}{0.672417in}%
\pgfsys@useobject{currentmarker}{}%
\end{pgfscope}%
\begin{pgfscope}%
\pgfsys@transformshift{1.795741in}{0.631643in}%
\pgfsys@useobject{currentmarker}{}%
\end{pgfscope}%
\begin{pgfscope}%
\pgfsys@transformshift{1.796834in}{0.657503in}%
\pgfsys@useobject{currentmarker}{}%
\end{pgfscope}%
\begin{pgfscope}%
\pgfsys@transformshift{1.797922in}{0.711392in}%
\pgfsys@useobject{currentmarker}{}%
\end{pgfscope}%
\begin{pgfscope}%
\pgfsys@transformshift{1.799004in}{0.690356in}%
\pgfsys@useobject{currentmarker}{}%
\end{pgfscope}%
\begin{pgfscope}%
\pgfsys@transformshift{1.800082in}{0.684105in}%
\pgfsys@useobject{currentmarker}{}%
\end{pgfscope}%
\begin{pgfscope}%
\pgfsys@transformshift{1.801154in}{0.704880in}%
\pgfsys@useobject{currentmarker}{}%
\end{pgfscope}%
\begin{pgfscope}%
\pgfsys@transformshift{1.802221in}{0.653204in}%
\pgfsys@useobject{currentmarker}{}%
\end{pgfscope}%
\begin{pgfscope}%
\pgfsys@transformshift{1.803284in}{0.669124in}%
\pgfsys@useobject{currentmarker}{}%
\end{pgfscope}%
\begin{pgfscope}%
\pgfsys@transformshift{1.804341in}{0.661449in}%
\pgfsys@useobject{currentmarker}{}%
\end{pgfscope}%
\begin{pgfscope}%
\pgfsys@transformshift{1.805393in}{0.666156in}%
\pgfsys@useobject{currentmarker}{}%
\end{pgfscope}%
\begin{pgfscope}%
\pgfsys@transformshift{1.806440in}{0.680729in}%
\pgfsys@useobject{currentmarker}{}%
\end{pgfscope}%
\begin{pgfscope}%
\pgfsys@transformshift{1.807483in}{0.682014in}%
\pgfsys@useobject{currentmarker}{}%
\end{pgfscope}%
\begin{pgfscope}%
\pgfsys@transformshift{1.808520in}{0.694038in}%
\pgfsys@useobject{currentmarker}{}%
\end{pgfscope}%
\begin{pgfscope}%
\pgfsys@transformshift{1.809553in}{0.720240in}%
\pgfsys@useobject{currentmarker}{}%
\end{pgfscope}%
\begin{pgfscope}%
\pgfsys@transformshift{1.810581in}{0.689912in}%
\pgfsys@useobject{currentmarker}{}%
\end{pgfscope}%
\begin{pgfscope}%
\pgfsys@transformshift{1.811605in}{0.688881in}%
\pgfsys@useobject{currentmarker}{}%
\end{pgfscope}%
\begin{pgfscope}%
\pgfsys@transformshift{1.812624in}{0.660018in}%
\pgfsys@useobject{currentmarker}{}%
\end{pgfscope}%
\begin{pgfscope}%
\pgfsys@transformshift{1.813638in}{0.651573in}%
\pgfsys@useobject{currentmarker}{}%
\end{pgfscope}%
\begin{pgfscope}%
\pgfsys@transformshift{1.814648in}{0.677834in}%
\pgfsys@useobject{currentmarker}{}%
\end{pgfscope}%
\begin{pgfscope}%
\pgfsys@transformshift{1.815653in}{0.701839in}%
\pgfsys@useobject{currentmarker}{}%
\end{pgfscope}%
\begin{pgfscope}%
\pgfsys@transformshift{1.816653in}{0.697798in}%
\pgfsys@useobject{currentmarker}{}%
\end{pgfscope}%
\begin{pgfscope}%
\pgfsys@transformshift{1.817650in}{0.682010in}%
\pgfsys@useobject{currentmarker}{}%
\end{pgfscope}%
\begin{pgfscope}%
\pgfsys@transformshift{1.818642in}{0.645968in}%
\pgfsys@useobject{currentmarker}{}%
\end{pgfscope}%
\begin{pgfscope}%
\pgfsys@transformshift{1.819629in}{0.643957in}%
\pgfsys@useobject{currentmarker}{}%
\end{pgfscope}%
\begin{pgfscope}%
\pgfsys@transformshift{1.820612in}{0.685035in}%
\pgfsys@useobject{currentmarker}{}%
\end{pgfscope}%
\begin{pgfscope}%
\pgfsys@transformshift{1.821591in}{0.633501in}%
\pgfsys@useobject{currentmarker}{}%
\end{pgfscope}%
\begin{pgfscope}%
\pgfsys@transformshift{1.822566in}{0.598493in}%
\pgfsys@useobject{currentmarker}{}%
\end{pgfscope}%
\begin{pgfscope}%
\pgfsys@transformshift{1.823536in}{0.660533in}%
\pgfsys@useobject{currentmarker}{}%
\end{pgfscope}%
\begin{pgfscope}%
\pgfsys@transformshift{1.824502in}{0.682253in}%
\pgfsys@useobject{currentmarker}{}%
\end{pgfscope}%
\begin{pgfscope}%
\pgfsys@transformshift{1.825464in}{0.665096in}%
\pgfsys@useobject{currentmarker}{}%
\end{pgfscope}%
\begin{pgfscope}%
\pgfsys@transformshift{1.826423in}{0.720235in}%
\pgfsys@useobject{currentmarker}{}%
\end{pgfscope}%
\begin{pgfscope}%
\pgfsys@transformshift{1.827376in}{0.733376in}%
\pgfsys@useobject{currentmarker}{}%
\end{pgfscope}%
\begin{pgfscope}%
\pgfsys@transformshift{1.828326in}{0.682711in}%
\pgfsys@useobject{currentmarker}{}%
\end{pgfscope}%
\begin{pgfscope}%
\pgfsys@transformshift{1.829272in}{0.689959in}%
\pgfsys@useobject{currentmarker}{}%
\end{pgfscope}%
\begin{pgfscope}%
\pgfsys@transformshift{1.830214in}{0.679436in}%
\pgfsys@useobject{currentmarker}{}%
\end{pgfscope}%
\begin{pgfscope}%
\pgfsys@transformshift{1.831152in}{0.698813in}%
\pgfsys@useobject{currentmarker}{}%
\end{pgfscope}%
\begin{pgfscope}%
\pgfsys@transformshift{1.832086in}{0.709568in}%
\pgfsys@useobject{currentmarker}{}%
\end{pgfscope}%
\begin{pgfscope}%
\pgfsys@transformshift{1.833017in}{0.696279in}%
\pgfsys@useobject{currentmarker}{}%
\end{pgfscope}%
\begin{pgfscope}%
\pgfsys@transformshift{1.833943in}{0.636491in}%
\pgfsys@useobject{currentmarker}{}%
\end{pgfscope}%
\begin{pgfscope}%
\pgfsys@transformshift{1.834866in}{0.649742in}%
\pgfsys@useobject{currentmarker}{}%
\end{pgfscope}%
\begin{pgfscope}%
\pgfsys@transformshift{1.835784in}{0.661139in}%
\pgfsys@useobject{currentmarker}{}%
\end{pgfscope}%
\begin{pgfscope}%
\pgfsys@transformshift{1.836699in}{0.649508in}%
\pgfsys@useobject{currentmarker}{}%
\end{pgfscope}%
\begin{pgfscope}%
\pgfsys@transformshift{1.837611in}{0.665376in}%
\pgfsys@useobject{currentmarker}{}%
\end{pgfscope}%
\begin{pgfscope}%
\pgfsys@transformshift{1.838518in}{0.713712in}%
\pgfsys@useobject{currentmarker}{}%
\end{pgfscope}%
\begin{pgfscope}%
\pgfsys@transformshift{1.839423in}{0.722228in}%
\pgfsys@useobject{currentmarker}{}%
\end{pgfscope}%
\begin{pgfscope}%
\pgfsys@transformshift{1.840323in}{0.656915in}%
\pgfsys@useobject{currentmarker}{}%
\end{pgfscope}%
\begin{pgfscope}%
\pgfsys@transformshift{1.841220in}{0.668250in}%
\pgfsys@useobject{currentmarker}{}%
\end{pgfscope}%
\begin{pgfscope}%
\pgfsys@transformshift{1.842113in}{0.701839in}%
\pgfsys@useobject{currentmarker}{}%
\end{pgfscope}%
\begin{pgfscope}%
\pgfsys@transformshift{1.843003in}{0.715192in}%
\pgfsys@useobject{currentmarker}{}%
\end{pgfscope}%
\begin{pgfscope}%
\pgfsys@transformshift{1.843889in}{0.706440in}%
\pgfsys@useobject{currentmarker}{}%
\end{pgfscope}%
\begin{pgfscope}%
\pgfsys@transformshift{1.844772in}{0.724235in}%
\pgfsys@useobject{currentmarker}{}%
\end{pgfscope}%
\begin{pgfscope}%
\pgfsys@transformshift{1.845651in}{0.732808in}%
\pgfsys@useobject{currentmarker}{}%
\end{pgfscope}%
\begin{pgfscope}%
\pgfsys@transformshift{1.846527in}{0.704651in}%
\pgfsys@useobject{currentmarker}{}%
\end{pgfscope}%
\begin{pgfscope}%
\pgfsys@transformshift{1.847399in}{0.698406in}%
\pgfsys@useobject{currentmarker}{}%
\end{pgfscope}%
\begin{pgfscope}%
\pgfsys@transformshift{1.848268in}{0.716320in}%
\pgfsys@useobject{currentmarker}{}%
\end{pgfscope}%
\begin{pgfscope}%
\pgfsys@transformshift{1.849134in}{0.673464in}%
\pgfsys@useobject{currentmarker}{}%
\end{pgfscope}%
\begin{pgfscope}%
\pgfsys@transformshift{1.849996in}{0.706274in}%
\pgfsys@useobject{currentmarker}{}%
\end{pgfscope}%
\begin{pgfscope}%
\pgfsys@transformshift{1.850855in}{0.734680in}%
\pgfsys@useobject{currentmarker}{}%
\end{pgfscope}%
\begin{pgfscope}%
\pgfsys@transformshift{1.851711in}{0.691395in}%
\pgfsys@useobject{currentmarker}{}%
\end{pgfscope}%
\begin{pgfscope}%
\pgfsys@transformshift{1.852564in}{0.691859in}%
\pgfsys@useobject{currentmarker}{}%
\end{pgfscope}%
\begin{pgfscope}%
\pgfsys@transformshift{1.853413in}{0.689912in}%
\pgfsys@useobject{currentmarker}{}%
\end{pgfscope}%
\begin{pgfscope}%
\pgfsys@transformshift{1.854259in}{0.665906in}%
\pgfsys@useobject{currentmarker}{}%
\end{pgfscope}%
\begin{pgfscope}%
\pgfsys@transformshift{1.855102in}{0.629813in}%
\pgfsys@useobject{currentmarker}{}%
\end{pgfscope}%
\begin{pgfscope}%
\pgfsys@transformshift{1.855942in}{0.607697in}%
\pgfsys@useobject{currentmarker}{}%
\end{pgfscope}%
\begin{pgfscope}%
\pgfsys@transformshift{1.856779in}{0.642579in}%
\pgfsys@useobject{currentmarker}{}%
\end{pgfscope}%
\begin{pgfscope}%
\pgfsys@transformshift{1.857612in}{0.669067in}%
\pgfsys@useobject{currentmarker}{}%
\end{pgfscope}%
\begin{pgfscope}%
\pgfsys@transformshift{1.858443in}{0.661314in}%
\pgfsys@useobject{currentmarker}{}%
\end{pgfscope}%
\begin{pgfscope}%
\pgfsys@transformshift{1.859270in}{0.679590in}%
\pgfsys@useobject{currentmarker}{}%
\end{pgfscope}%
\begin{pgfscope}%
\pgfsys@transformshift{1.860095in}{0.714941in}%
\pgfsys@useobject{currentmarker}{}%
\end{pgfscope}%
\begin{pgfscope}%
\pgfsys@transformshift{1.860916in}{0.650290in}%
\pgfsys@useobject{currentmarker}{}%
\end{pgfscope}%
\begin{pgfscope}%
\pgfsys@transformshift{1.861735in}{0.644860in}%
\pgfsys@useobject{currentmarker}{}%
\end{pgfscope}%
\begin{pgfscope}%
\pgfsys@transformshift{1.862550in}{0.656477in}%
\pgfsys@useobject{currentmarker}{}%
\end{pgfscope}%
\begin{pgfscope}%
\pgfsys@transformshift{1.863362in}{0.697925in}%
\pgfsys@useobject{currentmarker}{}%
\end{pgfscope}%
\begin{pgfscope}%
\pgfsys@transformshift{1.864172in}{0.687373in}%
\pgfsys@useobject{currentmarker}{}%
\end{pgfscope}%
\begin{pgfscope}%
\pgfsys@transformshift{1.864979in}{0.688657in}%
\pgfsys@useobject{currentmarker}{}%
\end{pgfscope}%
\begin{pgfscope}%
\pgfsys@transformshift{1.865782in}{0.683943in}%
\pgfsys@useobject{currentmarker}{}%
\end{pgfscope}%
\begin{pgfscope}%
\pgfsys@transformshift{1.866583in}{0.658410in}%
\pgfsys@useobject{currentmarker}{}%
\end{pgfscope}%
\begin{pgfscope}%
\pgfsys@transformshift{1.867381in}{0.695066in}%
\pgfsys@useobject{currentmarker}{}%
\end{pgfscope}%
\begin{pgfscope}%
\pgfsys@transformshift{1.868177in}{0.689595in}%
\pgfsys@useobject{currentmarker}{}%
\end{pgfscope}%
\begin{pgfscope}%
\pgfsys@transformshift{1.868969in}{0.725096in}%
\pgfsys@useobject{currentmarker}{}%
\end{pgfscope}%
\begin{pgfscope}%
\pgfsys@transformshift{1.869759in}{0.699007in}%
\pgfsys@useobject{currentmarker}{}%
\end{pgfscope}%
\begin{pgfscope}%
\pgfsys@transformshift{1.870546in}{0.609878in}%
\pgfsys@useobject{currentmarker}{}%
\end{pgfscope}%
\begin{pgfscope}%
\pgfsys@transformshift{1.871330in}{0.672790in}%
\pgfsys@useobject{currentmarker}{}%
\end{pgfscope}%
\begin{pgfscope}%
\pgfsys@transformshift{1.872111in}{0.696135in}%
\pgfsys@useobject{currentmarker}{}%
\end{pgfscope}%
\begin{pgfscope}%
\pgfsys@transformshift{1.872890in}{0.656174in}%
\pgfsys@useobject{currentmarker}{}%
\end{pgfscope}%
\begin{pgfscope}%
\pgfsys@transformshift{1.873666in}{0.638168in}%
\pgfsys@useobject{currentmarker}{}%
\end{pgfscope}%
\begin{pgfscope}%
\pgfsys@transformshift{1.874439in}{0.650093in}%
\pgfsys@useobject{currentmarker}{}%
\end{pgfscope}%
\begin{pgfscope}%
\pgfsys@transformshift{1.875210in}{0.704595in}%
\pgfsys@useobject{currentmarker}{}%
\end{pgfscope}%
\begin{pgfscope}%
\pgfsys@transformshift{1.875978in}{0.723683in}%
\pgfsys@useobject{currentmarker}{}%
\end{pgfscope}%
\begin{pgfscope}%
\pgfsys@transformshift{1.876743in}{0.718174in}%
\pgfsys@useobject{currentmarker}{}%
\end{pgfscope}%
\begin{pgfscope}%
\pgfsys@transformshift{1.877506in}{0.655121in}%
\pgfsys@useobject{currentmarker}{}%
\end{pgfscope}%
\begin{pgfscope}%
\pgfsys@transformshift{1.878266in}{0.652059in}%
\pgfsys@useobject{currentmarker}{}%
\end{pgfscope}%
\begin{pgfscope}%
\pgfsys@transformshift{1.879024in}{0.662019in}%
\pgfsys@useobject{currentmarker}{}%
\end{pgfscope}%
\begin{pgfscope}%
\pgfsys@transformshift{1.879779in}{0.696260in}%
\pgfsys@useobject{currentmarker}{}%
\end{pgfscope}%
\begin{pgfscope}%
\pgfsys@transformshift{1.880532in}{0.696172in}%
\pgfsys@useobject{currentmarker}{}%
\end{pgfscope}%
\begin{pgfscope}%
\pgfsys@transformshift{1.881282in}{0.663663in}%
\pgfsys@useobject{currentmarker}{}%
\end{pgfscope}%
\begin{pgfscope}%
\pgfsys@transformshift{1.882029in}{0.700161in}%
\pgfsys@useobject{currentmarker}{}%
\end{pgfscope}%
\begin{pgfscope}%
\pgfsys@transformshift{1.882774in}{0.696647in}%
\pgfsys@useobject{currentmarker}{}%
\end{pgfscope}%
\begin{pgfscope}%
\pgfsys@transformshift{1.883517in}{0.665000in}%
\pgfsys@useobject{currentmarker}{}%
\end{pgfscope}%
\begin{pgfscope}%
\pgfsys@transformshift{1.884257in}{0.716636in}%
\pgfsys@useobject{currentmarker}{}%
\end{pgfscope}%
\begin{pgfscope}%
\pgfsys@transformshift{1.884995in}{0.715462in}%
\pgfsys@useobject{currentmarker}{}%
\end{pgfscope}%
\begin{pgfscope}%
\pgfsys@transformshift{1.885730in}{0.672654in}%
\pgfsys@useobject{currentmarker}{}%
\end{pgfscope}%
\begin{pgfscope}%
\pgfsys@transformshift{1.886463in}{0.656649in}%
\pgfsys@useobject{currentmarker}{}%
\end{pgfscope}%
\begin{pgfscope}%
\pgfsys@transformshift{1.887194in}{0.636036in}%
\pgfsys@useobject{currentmarker}{}%
\end{pgfscope}%
\begin{pgfscope}%
\pgfsys@transformshift{1.887922in}{0.723125in}%
\pgfsys@useobject{currentmarker}{}%
\end{pgfscope}%
\begin{pgfscope}%
\pgfsys@transformshift{1.888648in}{0.735564in}%
\pgfsys@useobject{currentmarker}{}%
\end{pgfscope}%
\begin{pgfscope}%
\pgfsys@transformshift{1.889372in}{0.690903in}%
\pgfsys@useobject{currentmarker}{}%
\end{pgfscope}%
\begin{pgfscope}%
\pgfsys@transformshift{1.890093in}{0.673500in}%
\pgfsys@useobject{currentmarker}{}%
\end{pgfscope}%
\begin{pgfscope}%
\pgfsys@transformshift{1.890812in}{0.672641in}%
\pgfsys@useobject{currentmarker}{}%
\end{pgfscope}%
\begin{pgfscope}%
\pgfsys@transformshift{1.891528in}{0.728284in}%
\pgfsys@useobject{currentmarker}{}%
\end{pgfscope}%
\begin{pgfscope}%
\pgfsys@transformshift{1.892243in}{0.706159in}%
\pgfsys@useobject{currentmarker}{}%
\end{pgfscope}%
\begin{pgfscope}%
\pgfsys@transformshift{1.892955in}{0.669587in}%
\pgfsys@useobject{currentmarker}{}%
\end{pgfscope}%
\begin{pgfscope}%
\pgfsys@transformshift{1.893664in}{0.723837in}%
\pgfsys@useobject{currentmarker}{}%
\end{pgfscope}%
\begin{pgfscope}%
\pgfsys@transformshift{1.894372in}{0.719862in}%
\pgfsys@useobject{currentmarker}{}%
\end{pgfscope}%
\begin{pgfscope}%
\pgfsys@transformshift{1.895077in}{0.707533in}%
\pgfsys@useobject{currentmarker}{}%
\end{pgfscope}%
\begin{pgfscope}%
\pgfsys@transformshift{1.895780in}{0.631641in}%
\pgfsys@useobject{currentmarker}{}%
\end{pgfscope}%
\begin{pgfscope}%
\pgfsys@transformshift{1.896481in}{0.671742in}%
\pgfsys@useobject{currentmarker}{}%
\end{pgfscope}%
\begin{pgfscope}%
\pgfsys@transformshift{1.897180in}{0.681387in}%
\pgfsys@useobject{currentmarker}{}%
\end{pgfscope}%
\begin{pgfscope}%
\pgfsys@transformshift{1.897877in}{0.688453in}%
\pgfsys@useobject{currentmarker}{}%
\end{pgfscope}%
\begin{pgfscope}%
\pgfsys@transformshift{1.898571in}{0.682889in}%
\pgfsys@useobject{currentmarker}{}%
\end{pgfscope}%
\begin{pgfscope}%
\pgfsys@transformshift{1.899264in}{0.682615in}%
\pgfsys@useobject{currentmarker}{}%
\end{pgfscope}%
\begin{pgfscope}%
\pgfsys@transformshift{1.899954in}{0.732756in}%
\pgfsys@useobject{currentmarker}{}%
\end{pgfscope}%
\begin{pgfscope}%
\pgfsys@transformshift{1.900642in}{0.745336in}%
\pgfsys@useobject{currentmarker}{}%
\end{pgfscope}%
\begin{pgfscope}%
\pgfsys@transformshift{1.901328in}{0.695922in}%
\pgfsys@useobject{currentmarker}{}%
\end{pgfscope}%
\begin{pgfscope}%
\pgfsys@transformshift{1.902012in}{0.672189in}%
\pgfsys@useobject{currentmarker}{}%
\end{pgfscope}%
\begin{pgfscope}%
\pgfsys@transformshift{1.902693in}{0.697951in}%
\pgfsys@useobject{currentmarker}{}%
\end{pgfscope}%
\begin{pgfscope}%
\pgfsys@transformshift{1.903373in}{0.777981in}%
\pgfsys@useobject{currentmarker}{}%
\end{pgfscope}%
\begin{pgfscope}%
\pgfsys@transformshift{1.904051in}{0.767909in}%
\pgfsys@useobject{currentmarker}{}%
\end{pgfscope}%
\begin{pgfscope}%
\pgfsys@transformshift{1.904726in}{0.730316in}%
\pgfsys@useobject{currentmarker}{}%
\end{pgfscope}%
\begin{pgfscope}%
\pgfsys@transformshift{1.905400in}{0.732346in}%
\pgfsys@useobject{currentmarker}{}%
\end{pgfscope}%
\begin{pgfscope}%
\pgfsys@transformshift{1.906072in}{0.683454in}%
\pgfsys@useobject{currentmarker}{}%
\end{pgfscope}%
\begin{pgfscope}%
\pgfsys@transformshift{1.906741in}{0.666853in}%
\pgfsys@useobject{currentmarker}{}%
\end{pgfscope}%
\begin{pgfscope}%
\pgfsys@transformshift{1.907409in}{0.669589in}%
\pgfsys@useobject{currentmarker}{}%
\end{pgfscope}%
\begin{pgfscope}%
\pgfsys@transformshift{1.908075in}{0.647607in}%
\pgfsys@useobject{currentmarker}{}%
\end{pgfscope}%
\begin{pgfscope}%
\pgfsys@transformshift{1.908738in}{0.641904in}%
\pgfsys@useobject{currentmarker}{}%
\end{pgfscope}%
\begin{pgfscope}%
\pgfsys@transformshift{1.909400in}{0.654571in}%
\pgfsys@useobject{currentmarker}{}%
\end{pgfscope}%
\begin{pgfscope}%
\pgfsys@transformshift{1.910060in}{0.647251in}%
\pgfsys@useobject{currentmarker}{}%
\end{pgfscope}%
\begin{pgfscope}%
\pgfsys@transformshift{1.910717in}{0.650332in}%
\pgfsys@useobject{currentmarker}{}%
\end{pgfscope}%
\begin{pgfscope}%
\pgfsys@transformshift{1.911373in}{0.704919in}%
\pgfsys@useobject{currentmarker}{}%
\end{pgfscope}%
\begin{pgfscope}%
\pgfsys@transformshift{1.912027in}{0.699830in}%
\pgfsys@useobject{currentmarker}{}%
\end{pgfscope}%
\begin{pgfscope}%
\pgfsys@transformshift{1.912680in}{0.735111in}%
\pgfsys@useobject{currentmarker}{}%
\end{pgfscope}%
\begin{pgfscope}%
\pgfsys@transformshift{1.913330in}{0.745838in}%
\pgfsys@useobject{currentmarker}{}%
\end{pgfscope}%
\begin{pgfscope}%
\pgfsys@transformshift{1.913978in}{0.654917in}%
\pgfsys@useobject{currentmarker}{}%
\end{pgfscope}%
\begin{pgfscope}%
\pgfsys@transformshift{1.914625in}{0.705131in}%
\pgfsys@useobject{currentmarker}{}%
\end{pgfscope}%
\begin{pgfscope}%
\pgfsys@transformshift{1.915269in}{0.693982in}%
\pgfsys@useobject{currentmarker}{}%
\end{pgfscope}%
\begin{pgfscope}%
\pgfsys@transformshift{1.915912in}{0.676394in}%
\pgfsys@useobject{currentmarker}{}%
\end{pgfscope}%
\begin{pgfscope}%
\pgfsys@transformshift{1.916553in}{0.655772in}%
\pgfsys@useobject{currentmarker}{}%
\end{pgfscope}%
\begin{pgfscope}%
\pgfsys@transformshift{1.917192in}{0.584977in}%
\pgfsys@useobject{currentmarker}{}%
\end{pgfscope}%
\begin{pgfscope}%
\pgfsys@transformshift{1.917829in}{0.678108in}%
\pgfsys@useobject{currentmarker}{}%
\end{pgfscope}%
\begin{pgfscope}%
\pgfsys@transformshift{1.918465in}{0.708965in}%
\pgfsys@useobject{currentmarker}{}%
\end{pgfscope}%
\begin{pgfscope}%
\pgfsys@transformshift{1.919099in}{0.681839in}%
\pgfsys@useobject{currentmarker}{}%
\end{pgfscope}%
\begin{pgfscope}%
\pgfsys@transformshift{1.919731in}{0.680839in}%
\pgfsys@useobject{currentmarker}{}%
\end{pgfscope}%
\begin{pgfscope}%
\pgfsys@transformshift{1.920361in}{0.681939in}%
\pgfsys@useobject{currentmarker}{}%
\end{pgfscope}%
\begin{pgfscope}%
\pgfsys@transformshift{1.920989in}{0.672246in}%
\pgfsys@useobject{currentmarker}{}%
\end{pgfscope}%
\begin{pgfscope}%
\pgfsys@transformshift{1.921616in}{0.625546in}%
\pgfsys@useobject{currentmarker}{}%
\end{pgfscope}%
\begin{pgfscope}%
\pgfsys@transformshift{1.922241in}{0.679980in}%
\pgfsys@useobject{currentmarker}{}%
\end{pgfscope}%
\begin{pgfscope}%
\pgfsys@transformshift{1.922864in}{0.628091in}%
\pgfsys@useobject{currentmarker}{}%
\end{pgfscope}%
\begin{pgfscope}%
\pgfsys@transformshift{1.923485in}{0.691263in}%
\pgfsys@useobject{currentmarker}{}%
\end{pgfscope}%
\begin{pgfscope}%
\pgfsys@transformshift{1.924105in}{0.722589in}%
\pgfsys@useobject{currentmarker}{}%
\end{pgfscope}%
\begin{pgfscope}%
\pgfsys@transformshift{1.924723in}{0.693513in}%
\pgfsys@useobject{currentmarker}{}%
\end{pgfscope}%
\begin{pgfscope}%
\pgfsys@transformshift{1.925339in}{0.679451in}%
\pgfsys@useobject{currentmarker}{}%
\end{pgfscope}%
\begin{pgfscope}%
\pgfsys@transformshift{1.925954in}{0.641727in}%
\pgfsys@useobject{currentmarker}{}%
\end{pgfscope}%
\begin{pgfscope}%
\pgfsys@transformshift{1.926567in}{0.677110in}%
\pgfsys@useobject{currentmarker}{}%
\end{pgfscope}%
\begin{pgfscope}%
\pgfsys@transformshift{1.927178in}{0.648725in}%
\pgfsys@useobject{currentmarker}{}%
\end{pgfscope}%
\begin{pgfscope}%
\pgfsys@transformshift{1.927788in}{0.631012in}%
\pgfsys@useobject{currentmarker}{}%
\end{pgfscope}%
\begin{pgfscope}%
\pgfsys@transformshift{1.928396in}{0.613763in}%
\pgfsys@useobject{currentmarker}{}%
\end{pgfscope}%
\begin{pgfscope}%
\pgfsys@transformshift{1.929002in}{0.653333in}%
\pgfsys@useobject{currentmarker}{}%
\end{pgfscope}%
\begin{pgfscope}%
\pgfsys@transformshift{1.929607in}{0.703955in}%
\pgfsys@useobject{currentmarker}{}%
\end{pgfscope}%
\begin{pgfscope}%
\pgfsys@transformshift{1.930210in}{0.682711in}%
\pgfsys@useobject{currentmarker}{}%
\end{pgfscope}%
\begin{pgfscope}%
\pgfsys@transformshift{1.930811in}{0.650598in}%
\pgfsys@useobject{currentmarker}{}%
\end{pgfscope}%
\begin{pgfscope}%
\pgfsys@transformshift{1.931411in}{0.671542in}%
\pgfsys@useobject{currentmarker}{}%
\end{pgfscope}%
\begin{pgfscope}%
\pgfsys@transformshift{1.932010in}{0.668796in}%
\pgfsys@useobject{currentmarker}{}%
\end{pgfscope}%
\begin{pgfscope}%
\pgfsys@transformshift{1.932606in}{0.700689in}%
\pgfsys@useobject{currentmarker}{}%
\end{pgfscope}%
\begin{pgfscope}%
\pgfsys@transformshift{1.933201in}{0.679642in}%
\pgfsys@useobject{currentmarker}{}%
\end{pgfscope}%
\begin{pgfscope}%
\pgfsys@transformshift{1.933795in}{0.667919in}%
\pgfsys@useobject{currentmarker}{}%
\end{pgfscope}%
\begin{pgfscope}%
\pgfsys@transformshift{1.934387in}{0.684634in}%
\pgfsys@useobject{currentmarker}{}%
\end{pgfscope}%
\begin{pgfscope}%
\pgfsys@transformshift{1.934977in}{0.661890in}%
\pgfsys@useobject{currentmarker}{}%
\end{pgfscope}%
\begin{pgfscope}%
\pgfsys@transformshift{1.935566in}{0.590834in}%
\pgfsys@useobject{currentmarker}{}%
\end{pgfscope}%
\begin{pgfscope}%
\pgfsys@transformshift{1.936154in}{0.678117in}%
\pgfsys@useobject{currentmarker}{}%
\end{pgfscope}%
\begin{pgfscope}%
\pgfsys@transformshift{1.936739in}{0.679335in}%
\pgfsys@useobject{currentmarker}{}%
\end{pgfscope}%
\begin{pgfscope}%
\pgfsys@transformshift{1.937324in}{0.674719in}%
\pgfsys@useobject{currentmarker}{}%
\end{pgfscope}%
\begin{pgfscope}%
\pgfsys@transformshift{1.937906in}{0.713566in}%
\pgfsys@useobject{currentmarker}{}%
\end{pgfscope}%
\begin{pgfscope}%
\pgfsys@transformshift{1.938488in}{0.720304in}%
\pgfsys@useobject{currentmarker}{}%
\end{pgfscope}%
\begin{pgfscope}%
\pgfsys@transformshift{1.939067in}{0.728576in}%
\pgfsys@useobject{currentmarker}{}%
\end{pgfscope}%
\begin{pgfscope}%
\pgfsys@transformshift{1.939646in}{0.649249in}%
\pgfsys@useobject{currentmarker}{}%
\end{pgfscope}%
\begin{pgfscope}%
\pgfsys@transformshift{1.940222in}{0.657277in}%
\pgfsys@useobject{currentmarker}{}%
\end{pgfscope}%
\begin{pgfscope}%
\pgfsys@transformshift{1.940798in}{0.644641in}%
\pgfsys@useobject{currentmarker}{}%
\end{pgfscope}%
\begin{pgfscope}%
\pgfsys@transformshift{1.941371in}{0.671918in}%
\pgfsys@useobject{currentmarker}{}%
\end{pgfscope}%
\begin{pgfscope}%
\pgfsys@transformshift{1.941944in}{0.688440in}%
\pgfsys@useobject{currentmarker}{}%
\end{pgfscope}%
\begin{pgfscope}%
\pgfsys@transformshift{1.942515in}{0.693452in}%
\pgfsys@useobject{currentmarker}{}%
\end{pgfscope}%
\begin{pgfscope}%
\pgfsys@transformshift{1.943084in}{0.704868in}%
\pgfsys@useobject{currentmarker}{}%
\end{pgfscope}%
\begin{pgfscope}%
\pgfsys@transformshift{1.943652in}{0.697632in}%
\pgfsys@useobject{currentmarker}{}%
\end{pgfscope}%
\begin{pgfscope}%
\pgfsys@transformshift{1.944219in}{0.694514in}%
\pgfsys@useobject{currentmarker}{}%
\end{pgfscope}%
\begin{pgfscope}%
\pgfsys@transformshift{1.944784in}{0.690763in}%
\pgfsys@useobject{currentmarker}{}%
\end{pgfscope}%
\begin{pgfscope}%
\pgfsys@transformshift{1.945348in}{0.701112in}%
\pgfsys@useobject{currentmarker}{}%
\end{pgfscope}%
\begin{pgfscope}%
\pgfsys@transformshift{1.945910in}{0.705555in}%
\pgfsys@useobject{currentmarker}{}%
\end{pgfscope}%
\begin{pgfscope}%
\pgfsys@transformshift{1.946471in}{0.681147in}%
\pgfsys@useobject{currentmarker}{}%
\end{pgfscope}%
\begin{pgfscope}%
\pgfsys@transformshift{1.947031in}{0.665070in}%
\pgfsys@useobject{currentmarker}{}%
\end{pgfscope}%
\begin{pgfscope}%
\pgfsys@transformshift{1.947589in}{0.662996in}%
\pgfsys@useobject{currentmarker}{}%
\end{pgfscope}%
\begin{pgfscope}%
\pgfsys@transformshift{1.948145in}{0.726122in}%
\pgfsys@useobject{currentmarker}{}%
\end{pgfscope}%
\begin{pgfscope}%
\pgfsys@transformshift{1.948701in}{0.714915in}%
\pgfsys@useobject{currentmarker}{}%
\end{pgfscope}%
\begin{pgfscope}%
\pgfsys@transformshift{1.949255in}{0.699401in}%
\pgfsys@useobject{currentmarker}{}%
\end{pgfscope}%
\begin{pgfscope}%
\pgfsys@transformshift{1.949807in}{0.685883in}%
\pgfsys@useobject{currentmarker}{}%
\end{pgfscope}%
\begin{pgfscope}%
\pgfsys@transformshift{1.950359in}{0.707987in}%
\pgfsys@useobject{currentmarker}{}%
\end{pgfscope}%
\begin{pgfscope}%
\pgfsys@transformshift{1.950909in}{0.684266in}%
\pgfsys@useobject{currentmarker}{}%
\end{pgfscope}%
\begin{pgfscope}%
\pgfsys@transformshift{1.951457in}{0.684038in}%
\pgfsys@useobject{currentmarker}{}%
\end{pgfscope}%
\begin{pgfscope}%
\pgfsys@transformshift{1.952005in}{0.652041in}%
\pgfsys@useobject{currentmarker}{}%
\end{pgfscope}%
\begin{pgfscope}%
\pgfsys@transformshift{1.952550in}{0.682749in}%
\pgfsys@useobject{currentmarker}{}%
\end{pgfscope}%
\begin{pgfscope}%
\pgfsys@transformshift{1.953095in}{0.726677in}%
\pgfsys@useobject{currentmarker}{}%
\end{pgfscope}%
\begin{pgfscope}%
\pgfsys@transformshift{1.953638in}{0.724472in}%
\pgfsys@useobject{currentmarker}{}%
\end{pgfscope}%
\begin{pgfscope}%
\pgfsys@transformshift{1.954180in}{0.706414in}%
\pgfsys@useobject{currentmarker}{}%
\end{pgfscope}%
\begin{pgfscope}%
\pgfsys@transformshift{1.954721in}{0.743811in}%
\pgfsys@useobject{currentmarker}{}%
\end{pgfscope}%
\begin{pgfscope}%
\pgfsys@transformshift{1.955260in}{0.729630in}%
\pgfsys@useobject{currentmarker}{}%
\end{pgfscope}%
\begin{pgfscope}%
\pgfsys@transformshift{1.955799in}{0.682835in}%
\pgfsys@useobject{currentmarker}{}%
\end{pgfscope}%
\begin{pgfscope}%
\pgfsys@transformshift{1.956335in}{0.702734in}%
\pgfsys@useobject{currentmarker}{}%
\end{pgfscope}%
\begin{pgfscope}%
\pgfsys@transformshift{1.956871in}{0.722498in}%
\pgfsys@useobject{currentmarker}{}%
\end{pgfscope}%
\begin{pgfscope}%
\pgfsys@transformshift{1.957405in}{0.713929in}%
\pgfsys@useobject{currentmarker}{}%
\end{pgfscope}%
\begin{pgfscope}%
\pgfsys@transformshift{1.957938in}{0.654413in}%
\pgfsys@useobject{currentmarker}{}%
\end{pgfscope}%
\begin{pgfscope}%
\pgfsys@transformshift{1.958470in}{0.687912in}%
\pgfsys@useobject{currentmarker}{}%
\end{pgfscope}%
\begin{pgfscope}%
\pgfsys@transformshift{1.959000in}{0.731379in}%
\pgfsys@useobject{currentmarker}{}%
\end{pgfscope}%
\begin{pgfscope}%
\pgfsys@transformshift{1.959529in}{0.694951in}%
\pgfsys@useobject{currentmarker}{}%
\end{pgfscope}%
\begin{pgfscope}%
\pgfsys@transformshift{1.960057in}{0.668903in}%
\pgfsys@useobject{currentmarker}{}%
\end{pgfscope}%
\begin{pgfscope}%
\pgfsys@transformshift{1.960584in}{0.626513in}%
\pgfsys@useobject{currentmarker}{}%
\end{pgfscope}%
\begin{pgfscope}%
\pgfsys@transformshift{1.961110in}{0.658328in}%
\pgfsys@useobject{currentmarker}{}%
\end{pgfscope}%
\begin{pgfscope}%
\pgfsys@transformshift{1.961634in}{0.679921in}%
\pgfsys@useobject{currentmarker}{}%
\end{pgfscope}%
\begin{pgfscope}%
\pgfsys@transformshift{1.962157in}{0.669094in}%
\pgfsys@useobject{currentmarker}{}%
\end{pgfscope}%
\begin{pgfscope}%
\pgfsys@transformshift{1.962679in}{0.689480in}%
\pgfsys@useobject{currentmarker}{}%
\end{pgfscope}%
\begin{pgfscope}%
\pgfsys@transformshift{1.963199in}{0.712407in}%
\pgfsys@useobject{currentmarker}{}%
\end{pgfscope}%
\begin{pgfscope}%
\pgfsys@transformshift{1.963719in}{0.685599in}%
\pgfsys@useobject{currentmarker}{}%
\end{pgfscope}%
\begin{pgfscope}%
\pgfsys@transformshift{1.964237in}{0.674118in}%
\pgfsys@useobject{currentmarker}{}%
\end{pgfscope}%
\begin{pgfscope}%
\pgfsys@transformshift{1.964754in}{0.711782in}%
\pgfsys@useobject{currentmarker}{}%
\end{pgfscope}%
\begin{pgfscope}%
\pgfsys@transformshift{1.965270in}{0.715649in}%
\pgfsys@useobject{currentmarker}{}%
\end{pgfscope}%
\begin{pgfscope}%
\pgfsys@transformshift{1.965785in}{0.666679in}%
\pgfsys@useobject{currentmarker}{}%
\end{pgfscope}%
\begin{pgfscope}%
\pgfsys@transformshift{1.966298in}{0.669613in}%
\pgfsys@useobject{currentmarker}{}%
\end{pgfscope}%
\begin{pgfscope}%
\pgfsys@transformshift{1.966810in}{0.719980in}%
\pgfsys@useobject{currentmarker}{}%
\end{pgfscope}%
\begin{pgfscope}%
\pgfsys@transformshift{1.967322in}{0.731727in}%
\pgfsys@useobject{currentmarker}{}%
\end{pgfscope}%
\begin{pgfscope}%
\pgfsys@transformshift{1.967832in}{0.707206in}%
\pgfsys@useobject{currentmarker}{}%
\end{pgfscope}%
\begin{pgfscope}%
\pgfsys@transformshift{1.968340in}{0.658470in}%
\pgfsys@useobject{currentmarker}{}%
\end{pgfscope}%
\begin{pgfscope}%
\pgfsys@transformshift{1.968848in}{0.653897in}%
\pgfsys@useobject{currentmarker}{}%
\end{pgfscope}%
\begin{pgfscope}%
\pgfsys@transformshift{1.969355in}{0.689508in}%
\pgfsys@useobject{currentmarker}{}%
\end{pgfscope}%
\begin{pgfscope}%
\pgfsys@transformshift{1.969860in}{0.662645in}%
\pgfsys@useobject{currentmarker}{}%
\end{pgfscope}%
\begin{pgfscope}%
\pgfsys@transformshift{1.970364in}{0.695372in}%
\pgfsys@useobject{currentmarker}{}%
\end{pgfscope}%
\begin{pgfscope}%
\pgfsys@transformshift{1.970868in}{0.677700in}%
\pgfsys@useobject{currentmarker}{}%
\end{pgfscope}%
\begin{pgfscope}%
\pgfsys@transformshift{1.971370in}{0.656267in}%
\pgfsys@useobject{currentmarker}{}%
\end{pgfscope}%
\begin{pgfscope}%
\pgfsys@transformshift{1.971870in}{0.647691in}%
\pgfsys@useobject{currentmarker}{}%
\end{pgfscope}%
\begin{pgfscope}%
\pgfsys@transformshift{1.972370in}{0.642958in}%
\pgfsys@useobject{currentmarker}{}%
\end{pgfscope}%
\begin{pgfscope}%
\pgfsys@transformshift{1.972869in}{0.712616in}%
\pgfsys@useobject{currentmarker}{}%
\end{pgfscope}%
\begin{pgfscope}%
\pgfsys@transformshift{1.973366in}{0.683745in}%
\pgfsys@useobject{currentmarker}{}%
\end{pgfscope}%
\begin{pgfscope}%
\pgfsys@transformshift{1.973863in}{0.658137in}%
\pgfsys@useobject{currentmarker}{}%
\end{pgfscope}%
\begin{pgfscope}%
\pgfsys@transformshift{1.974358in}{0.683492in}%
\pgfsys@useobject{currentmarker}{}%
\end{pgfscope}%
\begin{pgfscope}%
\pgfsys@transformshift{1.974853in}{0.693674in}%
\pgfsys@useobject{currentmarker}{}%
\end{pgfscope}%
\begin{pgfscope}%
\pgfsys@transformshift{1.975346in}{0.714881in}%
\pgfsys@useobject{currentmarker}{}%
\end{pgfscope}%
\begin{pgfscope}%
\pgfsys@transformshift{1.975838in}{0.707588in}%
\pgfsys@useobject{currentmarker}{}%
\end{pgfscope}%
\begin{pgfscope}%
\pgfsys@transformshift{1.976329in}{0.717928in}%
\pgfsys@useobject{currentmarker}{}%
\end{pgfscope}%
\begin{pgfscope}%
\pgfsys@transformshift{1.976819in}{0.703013in}%
\pgfsys@useobject{currentmarker}{}%
\end{pgfscope}%
\begin{pgfscope}%
\pgfsys@transformshift{1.977308in}{0.692257in}%
\pgfsys@useobject{currentmarker}{}%
\end{pgfscope}%
\begin{pgfscope}%
\pgfsys@transformshift{1.977796in}{0.690604in}%
\pgfsys@useobject{currentmarker}{}%
\end{pgfscope}%
\begin{pgfscope}%
\pgfsys@transformshift{1.978282in}{0.671391in}%
\pgfsys@useobject{currentmarker}{}%
\end{pgfscope}%
\begin{pgfscope}%
\pgfsys@transformshift{1.978768in}{0.661497in}%
\pgfsys@useobject{currentmarker}{}%
\end{pgfscope}%
\begin{pgfscope}%
\pgfsys@transformshift{1.979253in}{0.689898in}%
\pgfsys@useobject{currentmarker}{}%
\end{pgfscope}%
\begin{pgfscope}%
\pgfsys@transformshift{1.979736in}{0.667160in}%
\pgfsys@useobject{currentmarker}{}%
\end{pgfscope}%
\begin{pgfscope}%
\pgfsys@transformshift{1.980219in}{0.676039in}%
\pgfsys@useobject{currentmarker}{}%
\end{pgfscope}%
\begin{pgfscope}%
\pgfsys@transformshift{1.980701in}{0.683001in}%
\pgfsys@useobject{currentmarker}{}%
\end{pgfscope}%
\begin{pgfscope}%
\pgfsys@transformshift{1.981181in}{0.723273in}%
\pgfsys@useobject{currentmarker}{}%
\end{pgfscope}%
\begin{pgfscope}%
\pgfsys@transformshift{1.981661in}{0.715346in}%
\pgfsys@useobject{currentmarker}{}%
\end{pgfscope}%
\begin{pgfscope}%
\pgfsys@transformshift{1.982139in}{0.715285in}%
\pgfsys@useobject{currentmarker}{}%
\end{pgfscope}%
\begin{pgfscope}%
\pgfsys@transformshift{1.982617in}{0.712700in}%
\pgfsys@useobject{currentmarker}{}%
\end{pgfscope}%
\begin{pgfscope}%
\pgfsys@transformshift{1.983093in}{0.691870in}%
\pgfsys@useobject{currentmarker}{}%
\end{pgfscope}%
\begin{pgfscope}%
\pgfsys@transformshift{1.983569in}{0.691911in}%
\pgfsys@useobject{currentmarker}{}%
\end{pgfscope}%
\begin{pgfscope}%
\pgfsys@transformshift{1.984043in}{0.666193in}%
\pgfsys@useobject{currentmarker}{}%
\end{pgfscope}%
\begin{pgfscope}%
\pgfsys@transformshift{1.984517in}{0.692592in}%
\pgfsys@useobject{currentmarker}{}%
\end{pgfscope}%
\begin{pgfscope}%
\pgfsys@transformshift{1.984989in}{0.707882in}%
\pgfsys@useobject{currentmarker}{}%
\end{pgfscope}%
\begin{pgfscope}%
\pgfsys@transformshift{1.985460in}{0.688824in}%
\pgfsys@useobject{currentmarker}{}%
\end{pgfscope}%
\begin{pgfscope}%
\pgfsys@transformshift{1.985931in}{0.659792in}%
\pgfsys@useobject{currentmarker}{}%
\end{pgfscope}%
\begin{pgfscope}%
\pgfsys@transformshift{1.986400in}{0.672267in}%
\pgfsys@useobject{currentmarker}{}%
\end{pgfscope}%
\begin{pgfscope}%
\pgfsys@transformshift{1.986869in}{0.639878in}%
\pgfsys@useobject{currentmarker}{}%
\end{pgfscope}%
\begin{pgfscope}%
\pgfsys@transformshift{1.987336in}{0.668738in}%
\pgfsys@useobject{currentmarker}{}%
\end{pgfscope}%
\begin{pgfscope}%
\pgfsys@transformshift{1.987803in}{0.700702in}%
\pgfsys@useobject{currentmarker}{}%
\end{pgfscope}%
\begin{pgfscope}%
\pgfsys@transformshift{1.988269in}{0.672664in}%
\pgfsys@useobject{currentmarker}{}%
\end{pgfscope}%
\begin{pgfscope}%
\pgfsys@transformshift{1.988733in}{0.633296in}%
\pgfsys@useobject{currentmarker}{}%
\end{pgfscope}%
\begin{pgfscope}%
\pgfsys@transformshift{1.989197in}{0.657388in}%
\pgfsys@useobject{currentmarker}{}%
\end{pgfscope}%
\begin{pgfscope}%
\pgfsys@transformshift{1.989660in}{0.720565in}%
\pgfsys@useobject{currentmarker}{}%
\end{pgfscope}%
\begin{pgfscope}%
\pgfsys@transformshift{1.990121in}{0.733014in}%
\pgfsys@useobject{currentmarker}{}%
\end{pgfscope}%
\begin{pgfscope}%
\pgfsys@transformshift{1.990582in}{0.716427in}%
\pgfsys@useobject{currentmarker}{}%
\end{pgfscope}%
\begin{pgfscope}%
\pgfsys@transformshift{1.991042in}{0.661652in}%
\pgfsys@useobject{currentmarker}{}%
\end{pgfscope}%
\begin{pgfscope}%
\pgfsys@transformshift{1.991501in}{0.646272in}%
\pgfsys@useobject{currentmarker}{}%
\end{pgfscope}%
\begin{pgfscope}%
\pgfsys@transformshift{1.991959in}{0.679121in}%
\pgfsys@useobject{currentmarker}{}%
\end{pgfscope}%
\begin{pgfscope}%
\pgfsys@transformshift{1.992416in}{0.696830in}%
\pgfsys@useobject{currentmarker}{}%
\end{pgfscope}%
\begin{pgfscope}%
\pgfsys@transformshift{1.992872in}{0.689479in}%
\pgfsys@useobject{currentmarker}{}%
\end{pgfscope}%
\begin{pgfscope}%
\pgfsys@transformshift{1.993328in}{0.713746in}%
\pgfsys@useobject{currentmarker}{}%
\end{pgfscope}%
\begin{pgfscope}%
\pgfsys@transformshift{1.993782in}{0.697627in}%
\pgfsys@useobject{currentmarker}{}%
\end{pgfscope}%
\begin{pgfscope}%
\pgfsys@transformshift{1.994235in}{0.678248in}%
\pgfsys@useobject{currentmarker}{}%
\end{pgfscope}%
\begin{pgfscope}%
\pgfsys@transformshift{1.994688in}{0.674137in}%
\pgfsys@useobject{currentmarker}{}%
\end{pgfscope}%
\begin{pgfscope}%
\pgfsys@transformshift{1.995139in}{0.737682in}%
\pgfsys@useobject{currentmarker}{}%
\end{pgfscope}%
\begin{pgfscope}%
\pgfsys@transformshift{1.995590in}{0.735312in}%
\pgfsys@useobject{currentmarker}{}%
\end{pgfscope}%
\begin{pgfscope}%
\pgfsys@transformshift{1.996040in}{0.684619in}%
\pgfsys@useobject{currentmarker}{}%
\end{pgfscope}%
\begin{pgfscope}%
\pgfsys@transformshift{1.996488in}{0.672899in}%
\pgfsys@useobject{currentmarker}{}%
\end{pgfscope}%
\begin{pgfscope}%
\pgfsys@transformshift{1.996936in}{0.691313in}%
\pgfsys@useobject{currentmarker}{}%
\end{pgfscope}%
\begin{pgfscope}%
\pgfsys@transformshift{1.997384in}{0.708423in}%
\pgfsys@useobject{currentmarker}{}%
\end{pgfscope}%
\begin{pgfscope}%
\pgfsys@transformshift{1.997830in}{0.681862in}%
\pgfsys@useobject{currentmarker}{}%
\end{pgfscope}%
\begin{pgfscope}%
\pgfsys@transformshift{1.998275in}{0.689469in}%
\pgfsys@useobject{currentmarker}{}%
\end{pgfscope}%
\begin{pgfscope}%
\pgfsys@transformshift{1.998719in}{0.720516in}%
\pgfsys@useobject{currentmarker}{}%
\end{pgfscope}%
\begin{pgfscope}%
\pgfsys@transformshift{1.999163in}{0.701198in}%
\pgfsys@useobject{currentmarker}{}%
\end{pgfscope}%
\begin{pgfscope}%
\pgfsys@transformshift{1.999606in}{0.643096in}%
\pgfsys@useobject{currentmarker}{}%
\end{pgfscope}%
\begin{pgfscope}%
\pgfsys@transformshift{2.000047in}{0.693557in}%
\pgfsys@useobject{currentmarker}{}%
\end{pgfscope}%
\begin{pgfscope}%
\pgfsys@transformshift{2.000488in}{0.694030in}%
\pgfsys@useobject{currentmarker}{}%
\end{pgfscope}%
\begin{pgfscope}%
\pgfsys@transformshift{2.000928in}{0.713641in}%
\pgfsys@useobject{currentmarker}{}%
\end{pgfscope}%
\begin{pgfscope}%
\pgfsys@transformshift{2.001368in}{0.682006in}%
\pgfsys@useobject{currentmarker}{}%
\end{pgfscope}%
\begin{pgfscope}%
\pgfsys@transformshift{2.001806in}{0.683628in}%
\pgfsys@useobject{currentmarker}{}%
\end{pgfscope}%
\begin{pgfscope}%
\pgfsys@transformshift{2.002243in}{0.727113in}%
\pgfsys@useobject{currentmarker}{}%
\end{pgfscope}%
\begin{pgfscope}%
\pgfsys@transformshift{2.002680in}{0.615398in}%
\pgfsys@useobject{currentmarker}{}%
\end{pgfscope}%
\begin{pgfscope}%
\pgfsys@transformshift{2.003116in}{0.627671in}%
\pgfsys@useobject{currentmarker}{}%
\end{pgfscope}%
\begin{pgfscope}%
\pgfsys@transformshift{2.003551in}{0.665065in}%
\pgfsys@useobject{currentmarker}{}%
\end{pgfscope}%
\begin{pgfscope}%
\pgfsys@transformshift{2.003985in}{0.727121in}%
\pgfsys@useobject{currentmarker}{}%
\end{pgfscope}%
\begin{pgfscope}%
\pgfsys@transformshift{2.004418in}{0.706399in}%
\pgfsys@useobject{currentmarker}{}%
\end{pgfscope}%
\begin{pgfscope}%
\pgfsys@transformshift{2.004851in}{0.613323in}%
\pgfsys@useobject{currentmarker}{}%
\end{pgfscope}%
\begin{pgfscope}%
\pgfsys@transformshift{2.005282in}{0.645379in}%
\pgfsys@useobject{currentmarker}{}%
\end{pgfscope}%
\begin{pgfscope}%
\pgfsys@transformshift{2.005713in}{0.631321in}%
\pgfsys@useobject{currentmarker}{}%
\end{pgfscope}%
\begin{pgfscope}%
\pgfsys@transformshift{2.006143in}{0.722630in}%
\pgfsys@useobject{currentmarker}{}%
\end{pgfscope}%
\begin{pgfscope}%
\pgfsys@transformshift{2.006572in}{0.678407in}%
\pgfsys@useobject{currentmarker}{}%
\end{pgfscope}%
\begin{pgfscope}%
\pgfsys@transformshift{2.007000in}{0.659561in}%
\pgfsys@useobject{currentmarker}{}%
\end{pgfscope}%
\begin{pgfscope}%
\pgfsys@transformshift{2.007428in}{0.698347in}%
\pgfsys@useobject{currentmarker}{}%
\end{pgfscope}%
\begin{pgfscope}%
\pgfsys@transformshift{2.007855in}{0.725342in}%
\pgfsys@useobject{currentmarker}{}%
\end{pgfscope}%
\begin{pgfscope}%
\pgfsys@transformshift{2.008280in}{0.719942in}%
\pgfsys@useobject{currentmarker}{}%
\end{pgfscope}%
\begin{pgfscope}%
\pgfsys@transformshift{2.008706in}{0.678150in}%
\pgfsys@useobject{currentmarker}{}%
\end{pgfscope}%
\begin{pgfscope}%
\pgfsys@transformshift{2.009130in}{0.643497in}%
\pgfsys@useobject{currentmarker}{}%
\end{pgfscope}%
\begin{pgfscope}%
\pgfsys@transformshift{2.009553in}{0.677671in}%
\pgfsys@useobject{currentmarker}{}%
\end{pgfscope}%
\begin{pgfscope}%
\pgfsys@transformshift{2.009976in}{0.720252in}%
\pgfsys@useobject{currentmarker}{}%
\end{pgfscope}%
\begin{pgfscope}%
\pgfsys@transformshift{2.010398in}{0.699281in}%
\pgfsys@useobject{currentmarker}{}%
\end{pgfscope}%
\begin{pgfscope}%
\pgfsys@transformshift{2.010819in}{0.664797in}%
\pgfsys@useobject{currentmarker}{}%
\end{pgfscope}%
\begin{pgfscope}%
\pgfsys@transformshift{2.011239in}{0.683037in}%
\pgfsys@useobject{currentmarker}{}%
\end{pgfscope}%
\begin{pgfscope}%
\pgfsys@transformshift{2.011659in}{0.709691in}%
\pgfsys@useobject{currentmarker}{}%
\end{pgfscope}%
\begin{pgfscope}%
\pgfsys@transformshift{2.012078in}{0.720091in}%
\pgfsys@useobject{currentmarker}{}%
\end{pgfscope}%
\begin{pgfscope}%
\pgfsys@transformshift{2.012495in}{0.676865in}%
\pgfsys@useobject{currentmarker}{}%
\end{pgfscope}%
\begin{pgfscope}%
\pgfsys@transformshift{2.012913in}{0.679947in}%
\pgfsys@useobject{currentmarker}{}%
\end{pgfscope}%
\begin{pgfscope}%
\pgfsys@transformshift{2.013329in}{0.669626in}%
\pgfsys@useobject{currentmarker}{}%
\end{pgfscope}%
\begin{pgfscope}%
\pgfsys@transformshift{2.013745in}{0.695657in}%
\pgfsys@useobject{currentmarker}{}%
\end{pgfscope}%
\begin{pgfscope}%
\pgfsys@transformshift{2.014160in}{0.688982in}%
\pgfsys@useobject{currentmarker}{}%
\end{pgfscope}%
\begin{pgfscope}%
\pgfsys@transformshift{2.014574in}{0.711201in}%
\pgfsys@useobject{currentmarker}{}%
\end{pgfscope}%
\begin{pgfscope}%
\pgfsys@transformshift{2.014987in}{0.716887in}%
\pgfsys@useobject{currentmarker}{}%
\end{pgfscope}%
\begin{pgfscope}%
\pgfsys@transformshift{2.015400in}{0.681504in}%
\pgfsys@useobject{currentmarker}{}%
\end{pgfscope}%
\begin{pgfscope}%
\pgfsys@transformshift{2.015811in}{0.657144in}%
\pgfsys@useobject{currentmarker}{}%
\end{pgfscope}%
\begin{pgfscope}%
\pgfsys@transformshift{2.016223in}{0.696478in}%
\pgfsys@useobject{currentmarker}{}%
\end{pgfscope}%
\begin{pgfscope}%
\pgfsys@transformshift{2.016633in}{0.700675in}%
\pgfsys@useobject{currentmarker}{}%
\end{pgfscope}%
\begin{pgfscope}%
\pgfsys@transformshift{2.017042in}{0.676169in}%
\pgfsys@useobject{currentmarker}{}%
\end{pgfscope}%
\begin{pgfscope}%
\pgfsys@transformshift{2.017451in}{0.660714in}%
\pgfsys@useobject{currentmarker}{}%
\end{pgfscope}%
\begin{pgfscope}%
\pgfsys@transformshift{2.017859in}{0.672194in}%
\pgfsys@useobject{currentmarker}{}%
\end{pgfscope}%
\begin{pgfscope}%
\pgfsys@transformshift{2.018267in}{0.651177in}%
\pgfsys@useobject{currentmarker}{}%
\end{pgfscope}%
\begin{pgfscope}%
\pgfsys@transformshift{2.018673in}{0.694315in}%
\pgfsys@useobject{currentmarker}{}%
\end{pgfscope}%
\begin{pgfscope}%
\pgfsys@transformshift{2.019079in}{0.735697in}%
\pgfsys@useobject{currentmarker}{}%
\end{pgfscope}%
\begin{pgfscope}%
\pgfsys@transformshift{2.019484in}{0.711121in}%
\pgfsys@useobject{currentmarker}{}%
\end{pgfscope}%
\begin{pgfscope}%
\pgfsys@transformshift{2.019889in}{0.701407in}%
\pgfsys@useobject{currentmarker}{}%
\end{pgfscope}%
\begin{pgfscope}%
\pgfsys@transformshift{2.020292in}{0.713142in}%
\pgfsys@useobject{currentmarker}{}%
\end{pgfscope}%
\begin{pgfscope}%
\pgfsys@transformshift{2.020695in}{0.709072in}%
\pgfsys@useobject{currentmarker}{}%
\end{pgfscope}%
\begin{pgfscope}%
\pgfsys@transformshift{2.021098in}{0.695291in}%
\pgfsys@useobject{currentmarker}{}%
\end{pgfscope}%
\begin{pgfscope}%
\pgfsys@transformshift{2.021499in}{0.708589in}%
\pgfsys@useobject{currentmarker}{}%
\end{pgfscope}%
\begin{pgfscope}%
\pgfsys@transformshift{2.021900in}{0.694725in}%
\pgfsys@useobject{currentmarker}{}%
\end{pgfscope}%
\begin{pgfscope}%
\pgfsys@transformshift{2.022300in}{0.678544in}%
\pgfsys@useobject{currentmarker}{}%
\end{pgfscope}%
\begin{pgfscope}%
\pgfsys@transformshift{2.022699in}{0.669755in}%
\pgfsys@useobject{currentmarker}{}%
\end{pgfscope}%
\begin{pgfscope}%
\pgfsys@transformshift{2.023098in}{0.709804in}%
\pgfsys@useobject{currentmarker}{}%
\end{pgfscope}%
\begin{pgfscope}%
\pgfsys@transformshift{2.023496in}{0.725305in}%
\pgfsys@useobject{currentmarker}{}%
\end{pgfscope}%
\begin{pgfscope}%
\pgfsys@transformshift{2.023893in}{0.707500in}%
\pgfsys@useobject{currentmarker}{}%
\end{pgfscope}%
\begin{pgfscope}%
\pgfsys@transformshift{2.024290in}{0.727241in}%
\pgfsys@useobject{currentmarker}{}%
\end{pgfscope}%
\begin{pgfscope}%
\pgfsys@transformshift{2.024686in}{0.679337in}%
\pgfsys@useobject{currentmarker}{}%
\end{pgfscope}%
\begin{pgfscope}%
\pgfsys@transformshift{2.025081in}{0.627676in}%
\pgfsys@useobject{currentmarker}{}%
\end{pgfscope}%
\begin{pgfscope}%
\pgfsys@transformshift{2.025475in}{0.654181in}%
\pgfsys@useobject{currentmarker}{}%
\end{pgfscope}%
\begin{pgfscope}%
\pgfsys@transformshift{2.025869in}{0.716069in}%
\pgfsys@useobject{currentmarker}{}%
\end{pgfscope}%
\begin{pgfscope}%
\pgfsys@transformshift{2.026262in}{0.740961in}%
\pgfsys@useobject{currentmarker}{}%
\end{pgfscope}%
\begin{pgfscope}%
\pgfsys@transformshift{2.026655in}{0.712647in}%
\pgfsys@useobject{currentmarker}{}%
\end{pgfscope}%
\begin{pgfscope}%
\pgfsys@transformshift{2.027046in}{0.667001in}%
\pgfsys@useobject{currentmarker}{}%
\end{pgfscope}%
\begin{pgfscope}%
\pgfsys@transformshift{2.027437in}{0.659820in}%
\pgfsys@useobject{currentmarker}{}%
\end{pgfscope}%
\begin{pgfscope}%
\pgfsys@transformshift{2.027828in}{0.712602in}%
\pgfsys@useobject{currentmarker}{}%
\end{pgfscope}%
\begin{pgfscope}%
\pgfsys@transformshift{2.028217in}{0.691164in}%
\pgfsys@useobject{currentmarker}{}%
\end{pgfscope}%
\begin{pgfscope}%
\pgfsys@transformshift{2.028606in}{0.693104in}%
\pgfsys@useobject{currentmarker}{}%
\end{pgfscope}%
\begin{pgfscope}%
\pgfsys@transformshift{2.028995in}{0.662758in}%
\pgfsys@useobject{currentmarker}{}%
\end{pgfscope}%
\begin{pgfscope}%
\pgfsys@transformshift{2.029382in}{0.654668in}%
\pgfsys@useobject{currentmarker}{}%
\end{pgfscope}%
\begin{pgfscope}%
\pgfsys@transformshift{2.029769in}{0.723919in}%
\pgfsys@useobject{currentmarker}{}%
\end{pgfscope}%
\begin{pgfscope}%
\pgfsys@transformshift{2.030156in}{0.710734in}%
\pgfsys@useobject{currentmarker}{}%
\end{pgfscope}%
\begin{pgfscope}%
\pgfsys@transformshift{2.030541in}{0.651053in}%
\pgfsys@useobject{currentmarker}{}%
\end{pgfscope}%
\begin{pgfscope}%
\pgfsys@transformshift{2.030926in}{0.642579in}%
\pgfsys@useobject{currentmarker}{}%
\end{pgfscope}%
\begin{pgfscope}%
\pgfsys@transformshift{2.031311in}{0.682166in}%
\pgfsys@useobject{currentmarker}{}%
\end{pgfscope}%
\begin{pgfscope}%
\pgfsys@transformshift{2.031694in}{0.713128in}%
\pgfsys@useobject{currentmarker}{}%
\end{pgfscope}%
\begin{pgfscope}%
\pgfsys@transformshift{2.032078in}{0.707779in}%
\pgfsys@useobject{currentmarker}{}%
\end{pgfscope}%
\begin{pgfscope}%
\pgfsys@transformshift{2.032460in}{0.688467in}%
\pgfsys@useobject{currentmarker}{}%
\end{pgfscope}%
\begin{pgfscope}%
\pgfsys@transformshift{2.032842in}{0.662656in}%
\pgfsys@useobject{currentmarker}{}%
\end{pgfscope}%
\begin{pgfscope}%
\pgfsys@transformshift{2.033223in}{0.709500in}%
\pgfsys@useobject{currentmarker}{}%
\end{pgfscope}%
\begin{pgfscope}%
\pgfsys@transformshift{2.033603in}{0.664410in}%
\pgfsys@useobject{currentmarker}{}%
\end{pgfscope}%
\begin{pgfscope}%
\pgfsys@transformshift{2.033983in}{0.693663in}%
\pgfsys@useobject{currentmarker}{}%
\end{pgfscope}%
\begin{pgfscope}%
\pgfsys@transformshift{2.034362in}{0.708342in}%
\pgfsys@useobject{currentmarker}{}%
\end{pgfscope}%
\begin{pgfscope}%
\pgfsys@transformshift{2.034741in}{0.643594in}%
\pgfsys@useobject{currentmarker}{}%
\end{pgfscope}%
\begin{pgfscope}%
\pgfsys@transformshift{2.035119in}{0.696481in}%
\pgfsys@useobject{currentmarker}{}%
\end{pgfscope}%
\begin{pgfscope}%
\pgfsys@transformshift{2.035496in}{0.706928in}%
\pgfsys@useobject{currentmarker}{}%
\end{pgfscope}%
\begin{pgfscope}%
\pgfsys@transformshift{2.035872in}{0.694965in}%
\pgfsys@useobject{currentmarker}{}%
\end{pgfscope}%
\begin{pgfscope}%
\pgfsys@transformshift{2.036248in}{0.737347in}%
\pgfsys@useobject{currentmarker}{}%
\end{pgfscope}%
\begin{pgfscope}%
\pgfsys@transformshift{2.036624in}{0.733288in}%
\pgfsys@useobject{currentmarker}{}%
\end{pgfscope}%
\begin{pgfscope}%
\pgfsys@transformshift{2.036998in}{0.692424in}%
\pgfsys@useobject{currentmarker}{}%
\end{pgfscope}%
\begin{pgfscope}%
\pgfsys@transformshift{2.037373in}{0.685369in}%
\pgfsys@useobject{currentmarker}{}%
\end{pgfscope}%
\begin{pgfscope}%
\pgfsys@transformshift{2.037746in}{0.698289in}%
\pgfsys@useobject{currentmarker}{}%
\end{pgfscope}%
\begin{pgfscope}%
\pgfsys@transformshift{2.038119in}{0.696410in}%
\pgfsys@useobject{currentmarker}{}%
\end{pgfscope}%
\begin{pgfscope}%
\pgfsys@transformshift{2.038491in}{0.699623in}%
\pgfsys@useobject{currentmarker}{}%
\end{pgfscope}%
\begin{pgfscope}%
\pgfsys@transformshift{2.038863in}{0.646670in}%
\pgfsys@useobject{currentmarker}{}%
\end{pgfscope}%
\begin{pgfscope}%
\pgfsys@transformshift{2.039234in}{0.681845in}%
\pgfsys@useobject{currentmarker}{}%
\end{pgfscope}%
\begin{pgfscope}%
\pgfsys@transformshift{2.039604in}{0.679177in}%
\pgfsys@useobject{currentmarker}{}%
\end{pgfscope}%
\begin{pgfscope}%
\pgfsys@transformshift{2.039974in}{0.667761in}%
\pgfsys@useobject{currentmarker}{}%
\end{pgfscope}%
\begin{pgfscope}%
\pgfsys@transformshift{2.040343in}{0.711891in}%
\pgfsys@useobject{currentmarker}{}%
\end{pgfscope}%
\begin{pgfscope}%
\pgfsys@transformshift{2.040712in}{0.679644in}%
\pgfsys@useobject{currentmarker}{}%
\end{pgfscope}%
\begin{pgfscope}%
\pgfsys@transformshift{2.041080in}{0.683760in}%
\pgfsys@useobject{currentmarker}{}%
\end{pgfscope}%
\begin{pgfscope}%
\pgfsys@transformshift{2.041447in}{0.704164in}%
\pgfsys@useobject{currentmarker}{}%
\end{pgfscope}%
\begin{pgfscope}%
\pgfsys@transformshift{2.041814in}{0.729598in}%
\pgfsys@useobject{currentmarker}{}%
\end{pgfscope}%
\begin{pgfscope}%
\pgfsys@transformshift{2.042180in}{0.767870in}%
\pgfsys@useobject{currentmarker}{}%
\end{pgfscope}%
\begin{pgfscope}%
\pgfsys@transformshift{2.042546in}{0.673908in}%
\pgfsys@useobject{currentmarker}{}%
\end{pgfscope}%
\begin{pgfscope}%
\pgfsys@transformshift{2.042911in}{0.682671in}%
\pgfsys@useobject{currentmarker}{}%
\end{pgfscope}%
\begin{pgfscope}%
\pgfsys@transformshift{2.043275in}{0.741426in}%
\pgfsys@useobject{currentmarker}{}%
\end{pgfscope}%
\begin{pgfscope}%
\pgfsys@transformshift{2.043639in}{0.707865in}%
\pgfsys@useobject{currentmarker}{}%
\end{pgfscope}%
\begin{pgfscope}%
\pgfsys@transformshift{2.044002in}{0.670264in}%
\pgfsys@useobject{currentmarker}{}%
\end{pgfscope}%
\begin{pgfscope}%
\pgfsys@transformshift{2.044365in}{0.718361in}%
\pgfsys@useobject{currentmarker}{}%
\end{pgfscope}%
\begin{pgfscope}%
\pgfsys@transformshift{2.044727in}{0.698193in}%
\pgfsys@useobject{currentmarker}{}%
\end{pgfscope}%
\begin{pgfscope}%
\pgfsys@transformshift{2.045088in}{0.663369in}%
\pgfsys@useobject{currentmarker}{}%
\end{pgfscope}%
\begin{pgfscope}%
\pgfsys@transformshift{2.045449in}{0.668999in}%
\pgfsys@useobject{currentmarker}{}%
\end{pgfscope}%
\begin{pgfscope}%
\pgfsys@transformshift{2.045809in}{0.689613in}%
\pgfsys@useobject{currentmarker}{}%
\end{pgfscope}%
\begin{pgfscope}%
\pgfsys@transformshift{2.046169in}{0.698936in}%
\pgfsys@useobject{currentmarker}{}%
\end{pgfscope}%
\begin{pgfscope}%
\pgfsys@transformshift{2.046528in}{0.677767in}%
\pgfsys@useobject{currentmarker}{}%
\end{pgfscope}%
\begin{pgfscope}%
\pgfsys@transformshift{2.046887in}{0.649689in}%
\pgfsys@useobject{currentmarker}{}%
\end{pgfscope}%
\begin{pgfscope}%
\pgfsys@transformshift{2.047245in}{0.695263in}%
\pgfsys@useobject{currentmarker}{}%
\end{pgfscope}%
\begin{pgfscope}%
\pgfsys@transformshift{2.047602in}{0.702306in}%
\pgfsys@useobject{currentmarker}{}%
\end{pgfscope}%
\begin{pgfscope}%
\pgfsys@transformshift{2.047959in}{0.702447in}%
\pgfsys@useobject{currentmarker}{}%
\end{pgfscope}%
\begin{pgfscope}%
\pgfsys@transformshift{2.048316in}{0.703295in}%
\pgfsys@useobject{currentmarker}{}%
\end{pgfscope}%
\begin{pgfscope}%
\pgfsys@transformshift{2.048671in}{0.700192in}%
\pgfsys@useobject{currentmarker}{}%
\end{pgfscope}%
\begin{pgfscope}%
\pgfsys@transformshift{2.049027in}{0.684587in}%
\pgfsys@useobject{currentmarker}{}%
\end{pgfscope}%
\begin{pgfscope}%
\pgfsys@transformshift{2.049381in}{0.692237in}%
\pgfsys@useobject{currentmarker}{}%
\end{pgfscope}%
\begin{pgfscope}%
\pgfsys@transformshift{2.049735in}{0.680570in}%
\pgfsys@useobject{currentmarker}{}%
\end{pgfscope}%
\begin{pgfscope}%
\pgfsys@transformshift{2.050089in}{0.700380in}%
\pgfsys@useobject{currentmarker}{}%
\end{pgfscope}%
\begin{pgfscope}%
\pgfsys@transformshift{2.050442in}{0.716194in}%
\pgfsys@useobject{currentmarker}{}%
\end{pgfscope}%
\begin{pgfscope}%
\pgfsys@transformshift{2.050794in}{0.661851in}%
\pgfsys@useobject{currentmarker}{}%
\end{pgfscope}%
\begin{pgfscope}%
\pgfsys@transformshift{2.051146in}{0.634471in}%
\pgfsys@useobject{currentmarker}{}%
\end{pgfscope}%
\begin{pgfscope}%
\pgfsys@transformshift{2.051497in}{0.721721in}%
\pgfsys@useobject{currentmarker}{}%
\end{pgfscope}%
\begin{pgfscope}%
\pgfsys@transformshift{2.051848in}{0.735382in}%
\pgfsys@useobject{currentmarker}{}%
\end{pgfscope}%
\begin{pgfscope}%
\pgfsys@transformshift{2.052198in}{0.677580in}%
\pgfsys@useobject{currentmarker}{}%
\end{pgfscope}%
\begin{pgfscope}%
\pgfsys@transformshift{2.052548in}{0.737037in}%
\pgfsys@useobject{currentmarker}{}%
\end{pgfscope}%
\begin{pgfscope}%
\pgfsys@transformshift{2.052897in}{0.738914in}%
\pgfsys@useobject{currentmarker}{}%
\end{pgfscope}%
\begin{pgfscope}%
\pgfsys@transformshift{2.053245in}{0.702406in}%
\pgfsys@useobject{currentmarker}{}%
\end{pgfscope}%
\begin{pgfscope}%
\pgfsys@transformshift{2.053593in}{0.639855in}%
\pgfsys@useobject{currentmarker}{}%
\end{pgfscope}%
\begin{pgfscope}%
\pgfsys@transformshift{2.053941in}{0.634808in}%
\pgfsys@useobject{currentmarker}{}%
\end{pgfscope}%
\begin{pgfscope}%
\pgfsys@transformshift{2.054288in}{0.659948in}%
\pgfsys@useobject{currentmarker}{}%
\end{pgfscope}%
\begin{pgfscope}%
\pgfsys@transformshift{2.054634in}{0.673911in}%
\pgfsys@useobject{currentmarker}{}%
\end{pgfscope}%
\begin{pgfscope}%
\pgfsys@transformshift{2.054980in}{0.696914in}%
\pgfsys@useobject{currentmarker}{}%
\end{pgfscope}%
\begin{pgfscope}%
\pgfsys@transformshift{2.055326in}{0.707234in}%
\pgfsys@useobject{currentmarker}{}%
\end{pgfscope}%
\begin{pgfscope}%
\pgfsys@transformshift{2.055670in}{0.717028in}%
\pgfsys@useobject{currentmarker}{}%
\end{pgfscope}%
\begin{pgfscope}%
\pgfsys@transformshift{2.056015in}{0.701907in}%
\pgfsys@useobject{currentmarker}{}%
\end{pgfscope}%
\begin{pgfscope}%
\pgfsys@transformshift{2.056358in}{0.726260in}%
\pgfsys@useobject{currentmarker}{}%
\end{pgfscope}%
\begin{pgfscope}%
\pgfsys@transformshift{2.056702in}{0.701550in}%
\pgfsys@useobject{currentmarker}{}%
\end{pgfscope}%
\begin{pgfscope}%
\pgfsys@transformshift{2.057044in}{0.640781in}%
\pgfsys@useobject{currentmarker}{}%
\end{pgfscope}%
\begin{pgfscope}%
\pgfsys@transformshift{2.057387in}{0.650793in}%
\pgfsys@useobject{currentmarker}{}%
\end{pgfscope}%
\begin{pgfscope}%
\pgfsys@transformshift{2.057728in}{0.695484in}%
\pgfsys@useobject{currentmarker}{}%
\end{pgfscope}%
\begin{pgfscope}%
\pgfsys@transformshift{2.058069in}{0.695348in}%
\pgfsys@useobject{currentmarker}{}%
\end{pgfscope}%
\begin{pgfscope}%
\pgfsys@transformshift{2.058410in}{0.634681in}%
\pgfsys@useobject{currentmarker}{}%
\end{pgfscope}%
\begin{pgfscope}%
\pgfsys@transformshift{2.058750in}{0.679720in}%
\pgfsys@useobject{currentmarker}{}%
\end{pgfscope}%
\begin{pgfscope}%
\pgfsys@transformshift{2.059090in}{0.722419in}%
\pgfsys@useobject{currentmarker}{}%
\end{pgfscope}%
\begin{pgfscope}%
\pgfsys@transformshift{2.059429in}{0.737158in}%
\pgfsys@useobject{currentmarker}{}%
\end{pgfscope}%
\begin{pgfscope}%
\pgfsys@transformshift{2.059767in}{0.678155in}%
\pgfsys@useobject{currentmarker}{}%
\end{pgfscope}%
\begin{pgfscope}%
\pgfsys@transformshift{2.060106in}{0.675984in}%
\pgfsys@useobject{currentmarker}{}%
\end{pgfscope}%
\begin{pgfscope}%
\pgfsys@transformshift{2.060443in}{0.703933in}%
\pgfsys@useobject{currentmarker}{}%
\end{pgfscope}%
\begin{pgfscope}%
\pgfsys@transformshift{2.060780in}{0.688599in}%
\pgfsys@useobject{currentmarker}{}%
\end{pgfscope}%
\begin{pgfscope}%
\pgfsys@transformshift{2.061117in}{0.705560in}%
\pgfsys@useobject{currentmarker}{}%
\end{pgfscope}%
\begin{pgfscope}%
\pgfsys@transformshift{2.061453in}{0.736225in}%
\pgfsys@useobject{currentmarker}{}%
\end{pgfscope}%
\begin{pgfscope}%
\pgfsys@transformshift{2.061788in}{0.674044in}%
\pgfsys@useobject{currentmarker}{}%
\end{pgfscope}%
\begin{pgfscope}%
\pgfsys@transformshift{2.062123in}{0.674447in}%
\pgfsys@useobject{currentmarker}{}%
\end{pgfscope}%
\begin{pgfscope}%
\pgfsys@transformshift{2.062458in}{0.704118in}%
\pgfsys@useobject{currentmarker}{}%
\end{pgfscope}%
\begin{pgfscope}%
\pgfsys@transformshift{2.062792in}{0.638117in}%
\pgfsys@useobject{currentmarker}{}%
\end{pgfscope}%
\begin{pgfscope}%
\pgfsys@transformshift{2.063126in}{0.717221in}%
\pgfsys@useobject{currentmarker}{}%
\end{pgfscope}%
\begin{pgfscope}%
\pgfsys@transformshift{2.063459in}{0.697802in}%
\pgfsys@useobject{currentmarker}{}%
\end{pgfscope}%
\begin{pgfscope}%
\pgfsys@transformshift{2.063791in}{0.673404in}%
\pgfsys@useobject{currentmarker}{}%
\end{pgfscope}%
\begin{pgfscope}%
\pgfsys@transformshift{2.064123in}{0.632849in}%
\pgfsys@useobject{currentmarker}{}%
\end{pgfscope}%
\begin{pgfscope}%
\pgfsys@transformshift{2.064455in}{0.689476in}%
\pgfsys@useobject{currentmarker}{}%
\end{pgfscope}%
\begin{pgfscope}%
\pgfsys@transformshift{2.064786in}{0.705455in}%
\pgfsys@useobject{currentmarker}{}%
\end{pgfscope}%
\begin{pgfscope}%
\pgfsys@transformshift{2.065117in}{0.705583in}%
\pgfsys@useobject{currentmarker}{}%
\end{pgfscope}%
\begin{pgfscope}%
\pgfsys@transformshift{2.065447in}{0.694901in}%
\pgfsys@useobject{currentmarker}{}%
\end{pgfscope}%
\begin{pgfscope}%
\pgfsys@transformshift{2.065776in}{0.682553in}%
\pgfsys@useobject{currentmarker}{}%
\end{pgfscope}%
\begin{pgfscope}%
\pgfsys@transformshift{2.066106in}{0.664829in}%
\pgfsys@useobject{currentmarker}{}%
\end{pgfscope}%
\begin{pgfscope}%
\pgfsys@transformshift{2.066434in}{0.703591in}%
\pgfsys@useobject{currentmarker}{}%
\end{pgfscope}%
\begin{pgfscope}%
\pgfsys@transformshift{2.066762in}{0.704026in}%
\pgfsys@useobject{currentmarker}{}%
\end{pgfscope}%
\begin{pgfscope}%
\pgfsys@transformshift{2.067090in}{0.685056in}%
\pgfsys@useobject{currentmarker}{}%
\end{pgfscope}%
\begin{pgfscope}%
\pgfsys@transformshift{2.067417in}{0.656870in}%
\pgfsys@useobject{currentmarker}{}%
\end{pgfscope}%
\begin{pgfscope}%
\pgfsys@transformshift{2.067744in}{0.648502in}%
\pgfsys@useobject{currentmarker}{}%
\end{pgfscope}%
\begin{pgfscope}%
\pgfsys@transformshift{2.068070in}{0.658419in}%
\pgfsys@useobject{currentmarker}{}%
\end{pgfscope}%
\begin{pgfscope}%
\pgfsys@transformshift{2.068396in}{0.668409in}%
\pgfsys@useobject{currentmarker}{}%
\end{pgfscope}%
\begin{pgfscope}%
\pgfsys@transformshift{2.068722in}{0.676179in}%
\pgfsys@useobject{currentmarker}{}%
\end{pgfscope}%
\begin{pgfscope}%
\pgfsys@transformshift{2.069046in}{0.698195in}%
\pgfsys@useobject{currentmarker}{}%
\end{pgfscope}%
\begin{pgfscope}%
\pgfsys@transformshift{2.069371in}{0.678860in}%
\pgfsys@useobject{currentmarker}{}%
\end{pgfscope}%
\begin{pgfscope}%
\pgfsys@transformshift{2.069695in}{0.665803in}%
\pgfsys@useobject{currentmarker}{}%
\end{pgfscope}%
\begin{pgfscope}%
\pgfsys@transformshift{2.070018in}{0.648618in}%
\pgfsys@useobject{currentmarker}{}%
\end{pgfscope}%
\begin{pgfscope}%
\pgfsys@transformshift{2.070341in}{0.681331in}%
\pgfsys@useobject{currentmarker}{}%
\end{pgfscope}%
\begin{pgfscope}%
\pgfsys@transformshift{2.070664in}{0.679585in}%
\pgfsys@useobject{currentmarker}{}%
\end{pgfscope}%
\begin{pgfscope}%
\pgfsys@transformshift{2.070986in}{0.673417in}%
\pgfsys@useobject{currentmarker}{}%
\end{pgfscope}%
\begin{pgfscope}%
\pgfsys@transformshift{2.071308in}{0.703366in}%
\pgfsys@useobject{currentmarker}{}%
\end{pgfscope}%
\begin{pgfscope}%
\pgfsys@transformshift{2.071629in}{0.722559in}%
\pgfsys@useobject{currentmarker}{}%
\end{pgfscope}%
\begin{pgfscope}%
\pgfsys@transformshift{2.071949in}{0.712303in}%
\pgfsys@useobject{currentmarker}{}%
\end{pgfscope}%
\begin{pgfscope}%
\pgfsys@transformshift{2.072270in}{0.669601in}%
\pgfsys@useobject{currentmarker}{}%
\end{pgfscope}%
\begin{pgfscope}%
\pgfsys@transformshift{2.072589in}{0.708022in}%
\pgfsys@useobject{currentmarker}{}%
\end{pgfscope}%
\begin{pgfscope}%
\pgfsys@transformshift{2.072909in}{0.709569in}%
\pgfsys@useobject{currentmarker}{}%
\end{pgfscope}%
\begin{pgfscope}%
\pgfsys@transformshift{2.073228in}{0.676417in}%
\pgfsys@useobject{currentmarker}{}%
\end{pgfscope}%
\begin{pgfscope}%
\pgfsys@transformshift{2.073546in}{0.688587in}%
\pgfsys@useobject{currentmarker}{}%
\end{pgfscope}%
\begin{pgfscope}%
\pgfsys@transformshift{2.073864in}{0.648466in}%
\pgfsys@useobject{currentmarker}{}%
\end{pgfscope}%
\begin{pgfscope}%
\pgfsys@transformshift{2.074182in}{0.690561in}%
\pgfsys@useobject{currentmarker}{}%
\end{pgfscope}%
\begin{pgfscope}%
\pgfsys@transformshift{2.074499in}{0.680756in}%
\pgfsys@useobject{currentmarker}{}%
\end{pgfscope}%
\begin{pgfscope}%
\pgfsys@transformshift{2.074815in}{0.663431in}%
\pgfsys@useobject{currentmarker}{}%
\end{pgfscope}%
\begin{pgfscope}%
\pgfsys@transformshift{2.075131in}{0.636224in}%
\pgfsys@useobject{currentmarker}{}%
\end{pgfscope}%
\begin{pgfscope}%
\pgfsys@transformshift{2.075447in}{0.695127in}%
\pgfsys@useobject{currentmarker}{}%
\end{pgfscope}%
\begin{pgfscope}%
\pgfsys@transformshift{2.075763in}{0.715822in}%
\pgfsys@useobject{currentmarker}{}%
\end{pgfscope}%
\begin{pgfscope}%
\pgfsys@transformshift{2.076077in}{0.702885in}%
\pgfsys@useobject{currentmarker}{}%
\end{pgfscope}%
\begin{pgfscope}%
\pgfsys@transformshift{2.076392in}{0.655556in}%
\pgfsys@useobject{currentmarker}{}%
\end{pgfscope}%
\begin{pgfscope}%
\pgfsys@transformshift{2.076706in}{0.644361in}%
\pgfsys@useobject{currentmarker}{}%
\end{pgfscope}%
\begin{pgfscope}%
\pgfsys@transformshift{2.077019in}{0.739902in}%
\pgfsys@useobject{currentmarker}{}%
\end{pgfscope}%
\begin{pgfscope}%
\pgfsys@transformshift{2.077332in}{0.735866in}%
\pgfsys@useobject{currentmarker}{}%
\end{pgfscope}%
\begin{pgfscope}%
\pgfsys@transformshift{2.077645in}{0.702597in}%
\pgfsys@useobject{currentmarker}{}%
\end{pgfscope}%
\begin{pgfscope}%
\pgfsys@transformshift{2.077957in}{0.686718in}%
\pgfsys@useobject{currentmarker}{}%
\end{pgfscope}%
\begin{pgfscope}%
\pgfsys@transformshift{2.078269in}{0.692620in}%
\pgfsys@useobject{currentmarker}{}%
\end{pgfscope}%
\begin{pgfscope}%
\pgfsys@transformshift{2.078581in}{0.711564in}%
\pgfsys@useobject{currentmarker}{}%
\end{pgfscope}%
\begin{pgfscope}%
\pgfsys@transformshift{2.078891in}{0.666198in}%
\pgfsys@useobject{currentmarker}{}%
\end{pgfscope}%
\begin{pgfscope}%
\pgfsys@transformshift{2.079202in}{0.727136in}%
\pgfsys@useobject{currentmarker}{}%
\end{pgfscope}%
\begin{pgfscope}%
\pgfsys@transformshift{2.079512in}{0.720165in}%
\pgfsys@useobject{currentmarker}{}%
\end{pgfscope}%
\begin{pgfscope}%
\pgfsys@transformshift{2.079822in}{0.734293in}%
\pgfsys@useobject{currentmarker}{}%
\end{pgfscope}%
\begin{pgfscope}%
\pgfsys@transformshift{2.080131in}{0.728889in}%
\pgfsys@useobject{currentmarker}{}%
\end{pgfscope}%
\begin{pgfscope}%
\pgfsys@transformshift{2.080440in}{0.673558in}%
\pgfsys@useobject{currentmarker}{}%
\end{pgfscope}%
\begin{pgfscope}%
\pgfsys@transformshift{2.080748in}{0.662423in}%
\pgfsys@useobject{currentmarker}{}%
\end{pgfscope}%
\begin{pgfscope}%
\pgfsys@transformshift{2.081056in}{0.679043in}%
\pgfsys@useobject{currentmarker}{}%
\end{pgfscope}%
\begin{pgfscope}%
\pgfsys@transformshift{2.081364in}{0.678654in}%
\pgfsys@useobject{currentmarker}{}%
\end{pgfscope}%
\begin{pgfscope}%
\pgfsys@transformshift{2.081671in}{0.661289in}%
\pgfsys@useobject{currentmarker}{}%
\end{pgfscope}%
\begin{pgfscope}%
\pgfsys@transformshift{2.081977in}{0.704366in}%
\pgfsys@useobject{currentmarker}{}%
\end{pgfscope}%
\begin{pgfscope}%
\pgfsys@transformshift{2.082284in}{0.714085in}%
\pgfsys@useobject{currentmarker}{}%
\end{pgfscope}%
\begin{pgfscope}%
\pgfsys@transformshift{2.082589in}{0.712764in}%
\pgfsys@useobject{currentmarker}{}%
\end{pgfscope}%
\begin{pgfscope}%
\pgfsys@transformshift{2.082895in}{0.709400in}%
\pgfsys@useobject{currentmarker}{}%
\end{pgfscope}%
\begin{pgfscope}%
\pgfsys@transformshift{2.083200in}{0.653037in}%
\pgfsys@useobject{currentmarker}{}%
\end{pgfscope}%
\begin{pgfscope}%
\pgfsys@transformshift{2.083505in}{0.683106in}%
\pgfsys@useobject{currentmarker}{}%
\end{pgfscope}%
\begin{pgfscope}%
\pgfsys@transformshift{2.083809in}{0.696453in}%
\pgfsys@useobject{currentmarker}{}%
\end{pgfscope}%
\begin{pgfscope}%
\pgfsys@transformshift{2.084113in}{0.651662in}%
\pgfsys@useobject{currentmarker}{}%
\end{pgfscope}%
\begin{pgfscope}%
\pgfsys@transformshift{2.084416in}{0.594616in}%
\pgfsys@useobject{currentmarker}{}%
\end{pgfscope}%
\begin{pgfscope}%
\pgfsys@transformshift{2.084719in}{0.703342in}%
\pgfsys@useobject{currentmarker}{}%
\end{pgfscope}%
\begin{pgfscope}%
\pgfsys@transformshift{2.085021in}{0.721562in}%
\pgfsys@useobject{currentmarker}{}%
\end{pgfscope}%
\begin{pgfscope}%
\pgfsys@transformshift{2.085324in}{0.699956in}%
\pgfsys@useobject{currentmarker}{}%
\end{pgfscope}%
\begin{pgfscope}%
\pgfsys@transformshift{2.085625in}{0.686056in}%
\pgfsys@useobject{currentmarker}{}%
\end{pgfscope}%
\begin{pgfscope}%
\pgfsys@transformshift{2.085927in}{0.689240in}%
\pgfsys@useobject{currentmarker}{}%
\end{pgfscope}%
\begin{pgfscope}%
\pgfsys@transformshift{2.086228in}{0.642229in}%
\pgfsys@useobject{currentmarker}{}%
\end{pgfscope}%
\begin{pgfscope}%
\pgfsys@transformshift{2.086528in}{0.674210in}%
\pgfsys@useobject{currentmarker}{}%
\end{pgfscope}%
\begin{pgfscope}%
\pgfsys@transformshift{2.086828in}{0.727240in}%
\pgfsys@useobject{currentmarker}{}%
\end{pgfscope}%
\begin{pgfscope}%
\pgfsys@transformshift{2.087128in}{0.689039in}%
\pgfsys@useobject{currentmarker}{}%
\end{pgfscope}%
\begin{pgfscope}%
\pgfsys@transformshift{2.087427in}{0.697024in}%
\pgfsys@useobject{currentmarker}{}%
\end{pgfscope}%
\begin{pgfscope}%
\pgfsys@transformshift{2.087726in}{0.724743in}%
\pgfsys@useobject{currentmarker}{}%
\end{pgfscope}%
\begin{pgfscope}%
\pgfsys@transformshift{2.088025in}{0.710351in}%
\pgfsys@useobject{currentmarker}{}%
\end{pgfscope}%
\begin{pgfscope}%
\pgfsys@transformshift{2.088323in}{0.707674in}%
\pgfsys@useobject{currentmarker}{}%
\end{pgfscope}%
\begin{pgfscope}%
\pgfsys@transformshift{2.088621in}{0.700164in}%
\pgfsys@useobject{currentmarker}{}%
\end{pgfscope}%
\begin{pgfscope}%
\pgfsys@transformshift{2.088918in}{0.649266in}%
\pgfsys@useobject{currentmarker}{}%
\end{pgfscope}%
\begin{pgfscope}%
\pgfsys@transformshift{2.089215in}{0.684446in}%
\pgfsys@useobject{currentmarker}{}%
\end{pgfscope}%
\begin{pgfscope}%
\pgfsys@transformshift{2.089512in}{0.685066in}%
\pgfsys@useobject{currentmarker}{}%
\end{pgfscope}%
\begin{pgfscope}%
\pgfsys@transformshift{2.089808in}{0.669152in}%
\pgfsys@useobject{currentmarker}{}%
\end{pgfscope}%
\begin{pgfscope}%
\pgfsys@transformshift{2.090104in}{0.718749in}%
\pgfsys@useobject{currentmarker}{}%
\end{pgfscope}%
\begin{pgfscope}%
\pgfsys@transformshift{2.090399in}{0.726863in}%
\pgfsys@useobject{currentmarker}{}%
\end{pgfscope}%
\begin{pgfscope}%
\pgfsys@transformshift{2.090694in}{0.638029in}%
\pgfsys@useobject{currentmarker}{}%
\end{pgfscope}%
\begin{pgfscope}%
\pgfsys@transformshift{2.090989in}{0.684571in}%
\pgfsys@useobject{currentmarker}{}%
\end{pgfscope}%
\begin{pgfscope}%
\pgfsys@transformshift{2.091283in}{0.643534in}%
\pgfsys@useobject{currentmarker}{}%
\end{pgfscope}%
\begin{pgfscope}%
\pgfsys@transformshift{2.091577in}{0.700220in}%
\pgfsys@useobject{currentmarker}{}%
\end{pgfscope}%
\begin{pgfscope}%
\pgfsys@transformshift{2.091870in}{0.697467in}%
\pgfsys@useobject{currentmarker}{}%
\end{pgfscope}%
\begin{pgfscope}%
\pgfsys@transformshift{2.092163in}{0.679173in}%
\pgfsys@useobject{currentmarker}{}%
\end{pgfscope}%
\begin{pgfscope}%
\pgfsys@transformshift{2.092456in}{0.702046in}%
\pgfsys@useobject{currentmarker}{}%
\end{pgfscope}%
\begin{pgfscope}%
\pgfsys@transformshift{2.092748in}{0.698780in}%
\pgfsys@useobject{currentmarker}{}%
\end{pgfscope}%
\begin{pgfscope}%
\pgfsys@transformshift{2.093040in}{0.645798in}%
\pgfsys@useobject{currentmarker}{}%
\end{pgfscope}%
\begin{pgfscope}%
\pgfsys@transformshift{2.093332in}{0.606716in}%
\pgfsys@useobject{currentmarker}{}%
\end{pgfscope}%
\begin{pgfscope}%
\pgfsys@transformshift{2.093623in}{0.682926in}%
\pgfsys@useobject{currentmarker}{}%
\end{pgfscope}%
\begin{pgfscope}%
\pgfsys@transformshift{2.093914in}{0.706566in}%
\pgfsys@useobject{currentmarker}{}%
\end{pgfscope}%
\begin{pgfscope}%
\pgfsys@transformshift{2.094204in}{0.711886in}%
\pgfsys@useobject{currentmarker}{}%
\end{pgfscope}%
\begin{pgfscope}%
\pgfsys@transformshift{2.094494in}{0.713413in}%
\pgfsys@useobject{currentmarker}{}%
\end{pgfscope}%
\begin{pgfscope}%
\pgfsys@transformshift{2.094784in}{0.687541in}%
\pgfsys@useobject{currentmarker}{}%
\end{pgfscope}%
\begin{pgfscope}%
\pgfsys@transformshift{2.095073in}{0.611318in}%
\pgfsys@useobject{currentmarker}{}%
\end{pgfscope}%
\begin{pgfscope}%
\pgfsys@transformshift{2.095362in}{0.664599in}%
\pgfsys@useobject{currentmarker}{}%
\end{pgfscope}%
\begin{pgfscope}%
\pgfsys@transformshift{2.095651in}{0.692065in}%
\pgfsys@useobject{currentmarker}{}%
\end{pgfscope}%
\begin{pgfscope}%
\pgfsys@transformshift{2.095939in}{0.716379in}%
\pgfsys@useobject{currentmarker}{}%
\end{pgfscope}%
\begin{pgfscope}%
\pgfsys@transformshift{2.096227in}{0.715210in}%
\pgfsys@useobject{currentmarker}{}%
\end{pgfscope}%
\begin{pgfscope}%
\pgfsys@transformshift{2.096514in}{0.698681in}%
\pgfsys@useobject{currentmarker}{}%
\end{pgfscope}%
\begin{pgfscope}%
\pgfsys@transformshift{2.096801in}{0.710577in}%
\pgfsys@useobject{currentmarker}{}%
\end{pgfscope}%
\begin{pgfscope}%
\pgfsys@transformshift{2.097088in}{0.723632in}%
\pgfsys@useobject{currentmarker}{}%
\end{pgfscope}%
\begin{pgfscope}%
\pgfsys@transformshift{2.097375in}{0.747204in}%
\pgfsys@useobject{currentmarker}{}%
\end{pgfscope}%
\begin{pgfscope}%
\pgfsys@transformshift{2.097661in}{0.739059in}%
\pgfsys@useobject{currentmarker}{}%
\end{pgfscope}%
\begin{pgfscope}%
\pgfsys@transformshift{2.097946in}{0.738377in}%
\pgfsys@useobject{currentmarker}{}%
\end{pgfscope}%
\begin{pgfscope}%
\pgfsys@transformshift{2.098231in}{0.703604in}%
\pgfsys@useobject{currentmarker}{}%
\end{pgfscope}%
\begin{pgfscope}%
\pgfsys@transformshift{2.098516in}{0.681517in}%
\pgfsys@useobject{currentmarker}{}%
\end{pgfscope}%
\begin{pgfscope}%
\pgfsys@transformshift{2.098801in}{0.700475in}%
\pgfsys@useobject{currentmarker}{}%
\end{pgfscope}%
\begin{pgfscope}%
\pgfsys@transformshift{2.099085in}{0.730318in}%
\pgfsys@useobject{currentmarker}{}%
\end{pgfscope}%
\begin{pgfscope}%
\pgfsys@transformshift{2.099369in}{0.743861in}%
\pgfsys@useobject{currentmarker}{}%
\end{pgfscope}%
\begin{pgfscope}%
\pgfsys@transformshift{2.099652in}{0.697812in}%
\pgfsys@useobject{currentmarker}{}%
\end{pgfscope}%
\begin{pgfscope}%
\pgfsys@transformshift{2.099936in}{0.682909in}%
\pgfsys@useobject{currentmarker}{}%
\end{pgfscope}%
\begin{pgfscope}%
\pgfsys@transformshift{2.100218in}{0.703960in}%
\pgfsys@useobject{currentmarker}{}%
\end{pgfscope}%
\begin{pgfscope}%
\pgfsys@transformshift{2.100501in}{0.711933in}%
\pgfsys@useobject{currentmarker}{}%
\end{pgfscope}%
\begin{pgfscope}%
\pgfsys@transformshift{2.100783in}{0.694501in}%
\pgfsys@useobject{currentmarker}{}%
\end{pgfscope}%
\begin{pgfscope}%
\pgfsys@transformshift{2.101064in}{0.724423in}%
\pgfsys@useobject{currentmarker}{}%
\end{pgfscope}%
\begin{pgfscope}%
\pgfsys@transformshift{2.101346in}{0.702244in}%
\pgfsys@useobject{currentmarker}{}%
\end{pgfscope}%
\begin{pgfscope}%
\pgfsys@transformshift{2.101627in}{0.676372in}%
\pgfsys@useobject{currentmarker}{}%
\end{pgfscope}%
\begin{pgfscope}%
\pgfsys@transformshift{2.101907in}{0.707221in}%
\pgfsys@useobject{currentmarker}{}%
\end{pgfscope}%
\begin{pgfscope}%
\pgfsys@transformshift{2.102188in}{0.699492in}%
\pgfsys@useobject{currentmarker}{}%
\end{pgfscope}%
\begin{pgfscope}%
\pgfsys@transformshift{2.102468in}{0.665012in}%
\pgfsys@useobject{currentmarker}{}%
\end{pgfscope}%
\begin{pgfscope}%
\pgfsys@transformshift{2.102747in}{0.657291in}%
\pgfsys@useobject{currentmarker}{}%
\end{pgfscope}%
\begin{pgfscope}%
\pgfsys@transformshift{2.103026in}{0.746681in}%
\pgfsys@useobject{currentmarker}{}%
\end{pgfscope}%
\begin{pgfscope}%
\pgfsys@transformshift{2.103305in}{0.735866in}%
\pgfsys@useobject{currentmarker}{}%
\end{pgfscope}%
\begin{pgfscope}%
\pgfsys@transformshift{2.103584in}{0.710850in}%
\pgfsys@useobject{currentmarker}{}%
\end{pgfscope}%
\begin{pgfscope}%
\pgfsys@transformshift{2.103862in}{0.712241in}%
\pgfsys@useobject{currentmarker}{}%
\end{pgfscope}%
\begin{pgfscope}%
\pgfsys@transformshift{2.104140in}{0.705714in}%
\pgfsys@useobject{currentmarker}{}%
\end{pgfscope}%
\begin{pgfscope}%
\pgfsys@transformshift{2.104417in}{0.615861in}%
\pgfsys@useobject{currentmarker}{}%
\end{pgfscope}%
\begin{pgfscope}%
\pgfsys@transformshift{2.104695in}{0.679406in}%
\pgfsys@useobject{currentmarker}{}%
\end{pgfscope}%
\begin{pgfscope}%
\pgfsys@transformshift{2.104972in}{0.704359in}%
\pgfsys@useobject{currentmarker}{}%
\end{pgfscope}%
\begin{pgfscope}%
\pgfsys@transformshift{2.105248in}{0.658729in}%
\pgfsys@useobject{currentmarker}{}%
\end{pgfscope}%
\begin{pgfscope}%
\pgfsys@transformshift{2.105524in}{0.649500in}%
\pgfsys@useobject{currentmarker}{}%
\end{pgfscope}%
\begin{pgfscope}%
\pgfsys@transformshift{2.105800in}{0.680622in}%
\pgfsys@useobject{currentmarker}{}%
\end{pgfscope}%
\begin{pgfscope}%
\pgfsys@transformshift{2.106075in}{0.713255in}%
\pgfsys@useobject{currentmarker}{}%
\end{pgfscope}%
\begin{pgfscope}%
\pgfsys@transformshift{2.106351in}{0.692285in}%
\pgfsys@useobject{currentmarker}{}%
\end{pgfscope}%
\begin{pgfscope}%
\pgfsys@transformshift{2.106625in}{0.703750in}%
\pgfsys@useobject{currentmarker}{}%
\end{pgfscope}%
\begin{pgfscope}%
\pgfsys@transformshift{2.106900in}{0.701087in}%
\pgfsys@useobject{currentmarker}{}%
\end{pgfscope}%
\begin{pgfscope}%
\pgfsys@transformshift{2.107174in}{0.689841in}%
\pgfsys@useobject{currentmarker}{}%
\end{pgfscope}%
\begin{pgfscope}%
\pgfsys@transformshift{2.107448in}{0.698794in}%
\pgfsys@useobject{currentmarker}{}%
\end{pgfscope}%
\begin{pgfscope}%
\pgfsys@transformshift{2.107721in}{0.675974in}%
\pgfsys@useobject{currentmarker}{}%
\end{pgfscope}%
\begin{pgfscope}%
\pgfsys@transformshift{2.107994in}{0.676578in}%
\pgfsys@useobject{currentmarker}{}%
\end{pgfscope}%
\begin{pgfscope}%
\pgfsys@transformshift{2.108267in}{0.687730in}%
\pgfsys@useobject{currentmarker}{}%
\end{pgfscope}%
\begin{pgfscope}%
\pgfsys@transformshift{2.108540in}{0.701167in}%
\pgfsys@useobject{currentmarker}{}%
\end{pgfscope}%
\begin{pgfscope}%
\pgfsys@transformshift{2.108812in}{0.665261in}%
\pgfsys@useobject{currentmarker}{}%
\end{pgfscope}%
\begin{pgfscope}%
\pgfsys@transformshift{2.109084in}{0.686478in}%
\pgfsys@useobject{currentmarker}{}%
\end{pgfscope}%
\begin{pgfscope}%
\pgfsys@transformshift{2.109355in}{0.740210in}%
\pgfsys@useobject{currentmarker}{}%
\end{pgfscope}%
\begin{pgfscope}%
\pgfsys@transformshift{2.109626in}{0.735376in}%
\pgfsys@useobject{currentmarker}{}%
\end{pgfscope}%
\begin{pgfscope}%
\pgfsys@transformshift{2.109897in}{0.687587in}%
\pgfsys@useobject{currentmarker}{}%
\end{pgfscope}%
\begin{pgfscope}%
\pgfsys@transformshift{2.110168in}{0.672287in}%
\pgfsys@useobject{currentmarker}{}%
\end{pgfscope}%
\begin{pgfscope}%
\pgfsys@transformshift{2.110438in}{0.694955in}%
\pgfsys@useobject{currentmarker}{}%
\end{pgfscope}%
\begin{pgfscope}%
\pgfsys@transformshift{2.110708in}{0.663092in}%
\pgfsys@useobject{currentmarker}{}%
\end{pgfscope}%
\begin{pgfscope}%
\pgfsys@transformshift{2.110977in}{0.612800in}%
\pgfsys@useobject{currentmarker}{}%
\end{pgfscope}%
\begin{pgfscope}%
\pgfsys@transformshift{2.111246in}{0.669606in}%
\pgfsys@useobject{currentmarker}{}%
\end{pgfscope}%
\begin{pgfscope}%
\pgfsys@transformshift{2.111515in}{0.720428in}%
\pgfsys@useobject{currentmarker}{}%
\end{pgfscope}%
\begin{pgfscope}%
\pgfsys@transformshift{2.111784in}{0.710795in}%
\pgfsys@useobject{currentmarker}{}%
\end{pgfscope}%
\begin{pgfscope}%
\pgfsys@transformshift{2.112052in}{0.712077in}%
\pgfsys@useobject{currentmarker}{}%
\end{pgfscope}%
\begin{pgfscope}%
\pgfsys@transformshift{2.112320in}{0.709327in}%
\pgfsys@useobject{currentmarker}{}%
\end{pgfscope}%
\begin{pgfscope}%
\pgfsys@transformshift{2.112588in}{0.690607in}%
\pgfsys@useobject{currentmarker}{}%
\end{pgfscope}%
\begin{pgfscope}%
\pgfsys@transformshift{2.112855in}{0.716323in}%
\pgfsys@useobject{currentmarker}{}%
\end{pgfscope}%
\begin{pgfscope}%
\pgfsys@transformshift{2.113122in}{0.683422in}%
\pgfsys@useobject{currentmarker}{}%
\end{pgfscope}%
\begin{pgfscope}%
\pgfsys@transformshift{2.113388in}{0.702446in}%
\pgfsys@useobject{currentmarker}{}%
\end{pgfscope}%
\begin{pgfscope}%
\pgfsys@transformshift{2.113655in}{0.691182in}%
\pgfsys@useobject{currentmarker}{}%
\end{pgfscope}%
\begin{pgfscope}%
\pgfsys@transformshift{2.113921in}{0.716278in}%
\pgfsys@useobject{currentmarker}{}%
\end{pgfscope}%
\begin{pgfscope}%
\pgfsys@transformshift{2.114187in}{0.755392in}%
\pgfsys@useobject{currentmarker}{}%
\end{pgfscope}%
\begin{pgfscope}%
\pgfsys@transformshift{2.114452in}{0.730724in}%
\pgfsys@useobject{currentmarker}{}%
\end{pgfscope}%
\begin{pgfscope}%
\pgfsys@transformshift{2.114717in}{0.669069in}%
\pgfsys@useobject{currentmarker}{}%
\end{pgfscope}%
\begin{pgfscope}%
\pgfsys@transformshift{2.114982in}{0.642417in}%
\pgfsys@useobject{currentmarker}{}%
\end{pgfscope}%
\begin{pgfscope}%
\pgfsys@transformshift{2.115246in}{0.670023in}%
\pgfsys@useobject{currentmarker}{}%
\end{pgfscope}%
\begin{pgfscope}%
\pgfsys@transformshift{2.115510in}{0.718623in}%
\pgfsys@useobject{currentmarker}{}%
\end{pgfscope}%
\begin{pgfscope}%
\pgfsys@transformshift{2.115774in}{0.693194in}%
\pgfsys@useobject{currentmarker}{}%
\end{pgfscope}%
\begin{pgfscope}%
\pgfsys@transformshift{2.116038in}{0.623022in}%
\pgfsys@useobject{currentmarker}{}%
\end{pgfscope}%
\begin{pgfscope}%
\pgfsys@transformshift{2.116301in}{0.638122in}%
\pgfsys@useobject{currentmarker}{}%
\end{pgfscope}%
\begin{pgfscope}%
\pgfsys@transformshift{2.116564in}{0.642397in}%
\pgfsys@useobject{currentmarker}{}%
\end{pgfscope}%
\begin{pgfscope}%
\pgfsys@transformshift{2.116826in}{0.683617in}%
\pgfsys@useobject{currentmarker}{}%
\end{pgfscope}%
\begin{pgfscope}%
\pgfsys@transformshift{2.117089in}{0.693917in}%
\pgfsys@useobject{currentmarker}{}%
\end{pgfscope}%
\begin{pgfscope}%
\pgfsys@transformshift{2.117351in}{0.705735in}%
\pgfsys@useobject{currentmarker}{}%
\end{pgfscope}%
\begin{pgfscope}%
\pgfsys@transformshift{2.117612in}{0.724175in}%
\pgfsys@useobject{currentmarker}{}%
\end{pgfscope}%
\begin{pgfscope}%
\pgfsys@transformshift{2.117874in}{0.693062in}%
\pgfsys@useobject{currentmarker}{}%
\end{pgfscope}%
\begin{pgfscope}%
\pgfsys@transformshift{2.118135in}{0.683443in}%
\pgfsys@useobject{currentmarker}{}%
\end{pgfscope}%
\begin{pgfscope}%
\pgfsys@transformshift{2.118396in}{0.674757in}%
\pgfsys@useobject{currentmarker}{}%
\end{pgfscope}%
\begin{pgfscope}%
\pgfsys@transformshift{2.118656in}{0.681696in}%
\pgfsys@useobject{currentmarker}{}%
\end{pgfscope}%
\begin{pgfscope}%
\pgfsys@transformshift{2.118916in}{0.668855in}%
\pgfsys@useobject{currentmarker}{}%
\end{pgfscope}%
\begin{pgfscope}%
\pgfsys@transformshift{2.119176in}{0.655898in}%
\pgfsys@useobject{currentmarker}{}%
\end{pgfscope}%
\begin{pgfscope}%
\pgfsys@transformshift{2.119436in}{0.666543in}%
\pgfsys@useobject{currentmarker}{}%
\end{pgfscope}%
\begin{pgfscope}%
\pgfsys@transformshift{2.119695in}{0.659745in}%
\pgfsys@useobject{currentmarker}{}%
\end{pgfscope}%
\begin{pgfscope}%
\pgfsys@transformshift{2.119954in}{0.650855in}%
\pgfsys@useobject{currentmarker}{}%
\end{pgfscope}%
\begin{pgfscope}%
\pgfsys@transformshift{2.120213in}{0.688619in}%
\pgfsys@useobject{currentmarker}{}%
\end{pgfscope}%
\begin{pgfscope}%
\pgfsys@transformshift{2.120471in}{0.704388in}%
\pgfsys@useobject{currentmarker}{}%
\end{pgfscope}%
\begin{pgfscope}%
\pgfsys@transformshift{2.120729in}{0.704412in}%
\pgfsys@useobject{currentmarker}{}%
\end{pgfscope}%
\begin{pgfscope}%
\pgfsys@transformshift{2.120987in}{0.683473in}%
\pgfsys@useobject{currentmarker}{}%
\end{pgfscope}%
\begin{pgfscope}%
\pgfsys@transformshift{2.121244in}{0.705541in}%
\pgfsys@useobject{currentmarker}{}%
\end{pgfscope}%
\begin{pgfscope}%
\pgfsys@transformshift{2.121501in}{0.742238in}%
\pgfsys@useobject{currentmarker}{}%
\end{pgfscope}%
\begin{pgfscope}%
\pgfsys@transformshift{2.121758in}{0.738039in}%
\pgfsys@useobject{currentmarker}{}%
\end{pgfscope}%
\begin{pgfscope}%
\pgfsys@transformshift{2.122015in}{0.694400in}%
\pgfsys@useobject{currentmarker}{}%
\end{pgfscope}%
\begin{pgfscope}%
\pgfsys@transformshift{2.122271in}{0.689326in}%
\pgfsys@useobject{currentmarker}{}%
\end{pgfscope}%
\begin{pgfscope}%
\pgfsys@transformshift{2.122527in}{0.734128in}%
\pgfsys@useobject{currentmarker}{}%
\end{pgfscope}%
\begin{pgfscope}%
\pgfsys@transformshift{2.122783in}{0.732815in}%
\pgfsys@useobject{currentmarker}{}%
\end{pgfscope}%
\begin{pgfscope}%
\pgfsys@transformshift{2.123038in}{0.722618in}%
\pgfsys@useobject{currentmarker}{}%
\end{pgfscope}%
\begin{pgfscope}%
\pgfsys@transformshift{2.123293in}{0.661240in}%
\pgfsys@useobject{currentmarker}{}%
\end{pgfscope}%
\begin{pgfscope}%
\pgfsys@transformshift{2.123548in}{0.691315in}%
\pgfsys@useobject{currentmarker}{}%
\end{pgfscope}%
\begin{pgfscope}%
\pgfsys@transformshift{2.123803in}{0.709684in}%
\pgfsys@useobject{currentmarker}{}%
\end{pgfscope}%
\begin{pgfscope}%
\pgfsys@transformshift{2.124057in}{0.700585in}%
\pgfsys@useobject{currentmarker}{}%
\end{pgfscope}%
\begin{pgfscope}%
\pgfsys@transformshift{2.124311in}{0.674797in}%
\pgfsys@useobject{currentmarker}{}%
\end{pgfscope}%
\begin{pgfscope}%
\pgfsys@transformshift{2.124565in}{0.683385in}%
\pgfsys@useobject{currentmarker}{}%
\end{pgfscope}%
\begin{pgfscope}%
\pgfsys@transformshift{2.124818in}{0.714241in}%
\pgfsys@useobject{currentmarker}{}%
\end{pgfscope}%
\begin{pgfscope}%
\pgfsys@transformshift{2.125071in}{0.720918in}%
\pgfsys@useobject{currentmarker}{}%
\end{pgfscope}%
\begin{pgfscope}%
\pgfsys@transformshift{2.125324in}{0.710821in}%
\pgfsys@useobject{currentmarker}{}%
\end{pgfscope}%
\begin{pgfscope}%
\pgfsys@transformshift{2.125577in}{0.672604in}%
\pgfsys@useobject{currentmarker}{}%
\end{pgfscope}%
\begin{pgfscope}%
\pgfsys@transformshift{2.125829in}{0.715140in}%
\pgfsys@useobject{currentmarker}{}%
\end{pgfscope}%
\begin{pgfscope}%
\pgfsys@transformshift{2.126081in}{0.688109in}%
\pgfsys@useobject{currentmarker}{}%
\end{pgfscope}%
\begin{pgfscope}%
\pgfsys@transformshift{2.126333in}{0.699245in}%
\pgfsys@useobject{currentmarker}{}%
\end{pgfscope}%
\begin{pgfscope}%
\pgfsys@transformshift{2.126584in}{0.727374in}%
\pgfsys@useobject{currentmarker}{}%
\end{pgfscope}%
\begin{pgfscope}%
\pgfsys@transformshift{2.126835in}{0.702729in}%
\pgfsys@useobject{currentmarker}{}%
\end{pgfscope}%
\begin{pgfscope}%
\pgfsys@transformshift{2.127086in}{0.680082in}%
\pgfsys@useobject{currentmarker}{}%
\end{pgfscope}%
\begin{pgfscope}%
\pgfsys@transformshift{2.127337in}{0.693391in}%
\pgfsys@useobject{currentmarker}{}%
\end{pgfscope}%
\begin{pgfscope}%
\pgfsys@transformshift{2.127587in}{0.696358in}%
\pgfsys@useobject{currentmarker}{}%
\end{pgfscope}%
\begin{pgfscope}%
\pgfsys@transformshift{2.127837in}{0.646178in}%
\pgfsys@useobject{currentmarker}{}%
\end{pgfscope}%
\begin{pgfscope}%
\pgfsys@transformshift{2.128087in}{0.537770in}%
\pgfsys@useobject{currentmarker}{}%
\end{pgfscope}%
\begin{pgfscope}%
\pgfsys@transformshift{2.128336in}{0.641500in}%
\pgfsys@useobject{currentmarker}{}%
\end{pgfscope}%
\begin{pgfscope}%
\pgfsys@transformshift{2.128586in}{0.655168in}%
\pgfsys@useobject{currentmarker}{}%
\end{pgfscope}%
\begin{pgfscope}%
\pgfsys@transformshift{2.128835in}{0.667188in}%
\pgfsys@useobject{currentmarker}{}%
\end{pgfscope}%
\begin{pgfscope}%
\pgfsys@transformshift{2.129083in}{0.688585in}%
\pgfsys@useobject{currentmarker}{}%
\end{pgfscope}%
\begin{pgfscope}%
\pgfsys@transformshift{2.129332in}{0.712099in}%
\pgfsys@useobject{currentmarker}{}%
\end{pgfscope}%
\begin{pgfscope}%
\pgfsys@transformshift{2.129580in}{0.710192in}%
\pgfsys@useobject{currentmarker}{}%
\end{pgfscope}%
\begin{pgfscope}%
\pgfsys@transformshift{2.129827in}{0.695364in}%
\pgfsys@useobject{currentmarker}{}%
\end{pgfscope}%
\begin{pgfscope}%
\pgfsys@transformshift{2.130075in}{0.712881in}%
\pgfsys@useobject{currentmarker}{}%
\end{pgfscope}%
\begin{pgfscope}%
\pgfsys@transformshift{2.130322in}{0.669897in}%
\pgfsys@useobject{currentmarker}{}%
\end{pgfscope}%
\begin{pgfscope}%
\pgfsys@transformshift{2.130569in}{0.648365in}%
\pgfsys@useobject{currentmarker}{}%
\end{pgfscope}%
\begin{pgfscope}%
\pgfsys@transformshift{2.130816in}{0.627377in}%
\pgfsys@useobject{currentmarker}{}%
\end{pgfscope}%
\begin{pgfscope}%
\pgfsys@transformshift{2.131062in}{0.668525in}%
\pgfsys@useobject{currentmarker}{}%
\end{pgfscope}%
\begin{pgfscope}%
\pgfsys@transformshift{2.131309in}{0.678622in}%
\pgfsys@useobject{currentmarker}{}%
\end{pgfscope}%
\begin{pgfscope}%
\pgfsys@transformshift{2.131555in}{0.684508in}%
\pgfsys@useobject{currentmarker}{}%
\end{pgfscope}%
\begin{pgfscope}%
\pgfsys@transformshift{2.131800in}{0.701809in}%
\pgfsys@useobject{currentmarker}{}%
\end{pgfscope}%
\begin{pgfscope}%
\pgfsys@transformshift{2.132046in}{0.674729in}%
\pgfsys@useobject{currentmarker}{}%
\end{pgfscope}%
\begin{pgfscope}%
\pgfsys@transformshift{2.132291in}{0.696212in}%
\pgfsys@useobject{currentmarker}{}%
\end{pgfscope}%
\begin{pgfscope}%
\pgfsys@transformshift{2.132536in}{0.721686in}%
\pgfsys@useobject{currentmarker}{}%
\end{pgfscope}%
\begin{pgfscope}%
\pgfsys@transformshift{2.132780in}{0.698003in}%
\pgfsys@useobject{currentmarker}{}%
\end{pgfscope}%
\begin{pgfscope}%
\pgfsys@transformshift{2.133025in}{0.650063in}%
\pgfsys@useobject{currentmarker}{}%
\end{pgfscope}%
\begin{pgfscope}%
\pgfsys@transformshift{2.133269in}{0.686652in}%
\pgfsys@useobject{currentmarker}{}%
\end{pgfscope}%
\begin{pgfscope}%
\pgfsys@transformshift{2.133512in}{0.656210in}%
\pgfsys@useobject{currentmarker}{}%
\end{pgfscope}%
\begin{pgfscope}%
\pgfsys@transformshift{2.133756in}{0.671656in}%
\pgfsys@useobject{currentmarker}{}%
\end{pgfscope}%
\begin{pgfscope}%
\pgfsys@transformshift{2.133999in}{0.696658in}%
\pgfsys@useobject{currentmarker}{}%
\end{pgfscope}%
\begin{pgfscope}%
\pgfsys@transformshift{2.134242in}{0.745930in}%
\pgfsys@useobject{currentmarker}{}%
\end{pgfscope}%
\begin{pgfscope}%
\pgfsys@transformshift{2.134485in}{0.724559in}%
\pgfsys@useobject{currentmarker}{}%
\end{pgfscope}%
\begin{pgfscope}%
\pgfsys@transformshift{2.134727in}{0.700064in}%
\pgfsys@useobject{currentmarker}{}%
\end{pgfscope}%
\begin{pgfscope}%
\pgfsys@transformshift{2.134970in}{0.730558in}%
\pgfsys@useobject{currentmarker}{}%
\end{pgfscope}%
\begin{pgfscope}%
\pgfsys@transformshift{2.135212in}{0.713248in}%
\pgfsys@useobject{currentmarker}{}%
\end{pgfscope}%
\begin{pgfscope}%
\pgfsys@transformshift{2.135453in}{0.715279in}%
\pgfsys@useobject{currentmarker}{}%
\end{pgfscope}%
\begin{pgfscope}%
\pgfsys@transformshift{2.135695in}{0.705786in}%
\pgfsys@useobject{currentmarker}{}%
\end{pgfscope}%
\begin{pgfscope}%
\pgfsys@transformshift{2.135936in}{0.704108in}%
\pgfsys@useobject{currentmarker}{}%
\end{pgfscope}%
\begin{pgfscope}%
\pgfsys@transformshift{2.136177in}{0.701909in}%
\pgfsys@useobject{currentmarker}{}%
\end{pgfscope}%
\begin{pgfscope}%
\pgfsys@transformshift{2.136417in}{0.743148in}%
\pgfsys@useobject{currentmarker}{}%
\end{pgfscope}%
\begin{pgfscope}%
\pgfsys@transformshift{2.136658in}{0.715475in}%
\pgfsys@useobject{currentmarker}{}%
\end{pgfscope}%
\begin{pgfscope}%
\pgfsys@transformshift{2.136898in}{0.693576in}%
\pgfsys@useobject{currentmarker}{}%
\end{pgfscope}%
\begin{pgfscope}%
\pgfsys@transformshift{2.137138in}{0.700562in}%
\pgfsys@useobject{currentmarker}{}%
\end{pgfscope}%
\begin{pgfscope}%
\pgfsys@transformshift{2.137377in}{0.670131in}%
\pgfsys@useobject{currentmarker}{}%
\end{pgfscope}%
\begin{pgfscope}%
\pgfsys@transformshift{2.137617in}{0.653931in}%
\pgfsys@useobject{currentmarker}{}%
\end{pgfscope}%
\begin{pgfscope}%
\pgfsys@transformshift{2.137856in}{0.637718in}%
\pgfsys@useobject{currentmarker}{}%
\end{pgfscope}%
\begin{pgfscope}%
\pgfsys@transformshift{2.138095in}{0.683129in}%
\pgfsys@useobject{currentmarker}{}%
\end{pgfscope}%
\begin{pgfscope}%
\pgfsys@transformshift{2.138333in}{0.675043in}%
\pgfsys@useobject{currentmarker}{}%
\end{pgfscope}%
\begin{pgfscope}%
\pgfsys@transformshift{2.138572in}{0.687880in}%
\pgfsys@useobject{currentmarker}{}%
\end{pgfscope}%
\begin{pgfscope}%
\pgfsys@transformshift{2.138810in}{0.707329in}%
\pgfsys@useobject{currentmarker}{}%
\end{pgfscope}%
\begin{pgfscope}%
\pgfsys@transformshift{2.139048in}{0.702201in}%
\pgfsys@useobject{currentmarker}{}%
\end{pgfscope}%
\begin{pgfscope}%
\pgfsys@transformshift{2.139285in}{0.706183in}%
\pgfsys@useobject{currentmarker}{}%
\end{pgfscope}%
\begin{pgfscope}%
\pgfsys@transformshift{2.139523in}{0.694124in}%
\pgfsys@useobject{currentmarker}{}%
\end{pgfscope}%
\begin{pgfscope}%
\pgfsys@transformshift{2.139760in}{0.729781in}%
\pgfsys@useobject{currentmarker}{}%
\end{pgfscope}%
\begin{pgfscope}%
\pgfsys@transformshift{2.139997in}{0.708043in}%
\pgfsys@useobject{currentmarker}{}%
\end{pgfscope}%
\begin{pgfscope}%
\pgfsys@transformshift{2.140233in}{0.677962in}%
\pgfsys@useobject{currentmarker}{}%
\end{pgfscope}%
\begin{pgfscope}%
\pgfsys@transformshift{2.140470in}{0.643594in}%
\pgfsys@useobject{currentmarker}{}%
\end{pgfscope}%
\begin{pgfscope}%
\pgfsys@transformshift{2.140706in}{0.639047in}%
\pgfsys@useobject{currentmarker}{}%
\end{pgfscope}%
\begin{pgfscope}%
\pgfsys@transformshift{2.140942in}{0.657941in}%
\pgfsys@useobject{currentmarker}{}%
\end{pgfscope}%
\begin{pgfscope}%
\pgfsys@transformshift{2.141177in}{0.648285in}%
\pgfsys@useobject{currentmarker}{}%
\end{pgfscope}%
\begin{pgfscope}%
\pgfsys@transformshift{2.141412in}{0.716770in}%
\pgfsys@useobject{currentmarker}{}%
\end{pgfscope}%
\begin{pgfscope}%
\pgfsys@transformshift{2.141648in}{0.721059in}%
\pgfsys@useobject{currentmarker}{}%
\end{pgfscope}%
\begin{pgfscope}%
\pgfsys@transformshift{2.141882in}{0.686773in}%
\pgfsys@useobject{currentmarker}{}%
\end{pgfscope}%
\begin{pgfscope}%
\pgfsys@transformshift{2.142117in}{0.685631in}%
\pgfsys@useobject{currentmarker}{}%
\end{pgfscope}%
\begin{pgfscope}%
\pgfsys@transformshift{2.142351in}{0.679741in}%
\pgfsys@useobject{currentmarker}{}%
\end{pgfscope}%
\begin{pgfscope}%
\pgfsys@transformshift{2.142586in}{0.640645in}%
\pgfsys@useobject{currentmarker}{}%
\end{pgfscope}%
\begin{pgfscope}%
\pgfsys@transformshift{2.142820in}{0.650478in}%
\pgfsys@useobject{currentmarker}{}%
\end{pgfscope}%
\begin{pgfscope}%
\pgfsys@transformshift{2.143053in}{0.692080in}%
\pgfsys@useobject{currentmarker}{}%
\end{pgfscope}%
\begin{pgfscope}%
\pgfsys@transformshift{2.143287in}{0.700815in}%
\pgfsys@useobject{currentmarker}{}%
\end{pgfscope}%
\begin{pgfscope}%
\pgfsys@transformshift{2.143520in}{0.714060in}%
\pgfsys@useobject{currentmarker}{}%
\end{pgfscope}%
\begin{pgfscope}%
\pgfsys@transformshift{2.143753in}{0.705327in}%
\pgfsys@useobject{currentmarker}{}%
\end{pgfscope}%
\begin{pgfscope}%
\pgfsys@transformshift{2.143985in}{0.690826in}%
\pgfsys@useobject{currentmarker}{}%
\end{pgfscope}%
\begin{pgfscope}%
\pgfsys@transformshift{2.144218in}{0.745522in}%
\pgfsys@useobject{currentmarker}{}%
\end{pgfscope}%
\begin{pgfscope}%
\pgfsys@transformshift{2.144450in}{0.741101in}%
\pgfsys@useobject{currentmarker}{}%
\end{pgfscope}%
\begin{pgfscope}%
\pgfsys@transformshift{2.144682in}{0.690825in}%
\pgfsys@useobject{currentmarker}{}%
\end{pgfscope}%
\begin{pgfscope}%
\pgfsys@transformshift{2.144914in}{0.687563in}%
\pgfsys@useobject{currentmarker}{}%
\end{pgfscope}%
\begin{pgfscope}%
\pgfsys@transformshift{2.145145in}{0.634590in}%
\pgfsys@useobject{currentmarker}{}%
\end{pgfscope}%
\begin{pgfscope}%
\pgfsys@transformshift{2.145376in}{0.643679in}%
\pgfsys@useobject{currentmarker}{}%
\end{pgfscope}%
\begin{pgfscope}%
\pgfsys@transformshift{2.145607in}{0.682764in}%
\pgfsys@useobject{currentmarker}{}%
\end{pgfscope}%
\begin{pgfscope}%
\pgfsys@transformshift{2.145838in}{0.702387in}%
\pgfsys@useobject{currentmarker}{}%
\end{pgfscope}%
\begin{pgfscope}%
\pgfsys@transformshift{2.146069in}{0.729294in}%
\pgfsys@useobject{currentmarker}{}%
\end{pgfscope}%
\begin{pgfscope}%
\pgfsys@transformshift{2.146299in}{0.707105in}%
\pgfsys@useobject{currentmarker}{}%
\end{pgfscope}%
\begin{pgfscope}%
\pgfsys@transformshift{2.146529in}{0.699953in}%
\pgfsys@useobject{currentmarker}{}%
\end{pgfscope}%
\begin{pgfscope}%
\pgfsys@transformshift{2.146759in}{0.673328in}%
\pgfsys@useobject{currentmarker}{}%
\end{pgfscope}%
\begin{pgfscope}%
\pgfsys@transformshift{2.146988in}{0.622389in}%
\pgfsys@useobject{currentmarker}{}%
\end{pgfscope}%
\begin{pgfscope}%
\pgfsys@transformshift{2.147218in}{0.614056in}%
\pgfsys@useobject{currentmarker}{}%
\end{pgfscope}%
\begin{pgfscope}%
\pgfsys@transformshift{2.147447in}{0.656328in}%
\pgfsys@useobject{currentmarker}{}%
\end{pgfscope}%
\begin{pgfscope}%
\pgfsys@transformshift{2.147676in}{0.663367in}%
\pgfsys@useobject{currentmarker}{}%
\end{pgfscope}%
\begin{pgfscope}%
\pgfsys@transformshift{2.147904in}{0.635906in}%
\pgfsys@useobject{currentmarker}{}%
\end{pgfscope}%
\begin{pgfscope}%
\pgfsys@transformshift{2.148133in}{0.667915in}%
\pgfsys@useobject{currentmarker}{}%
\end{pgfscope}%
\begin{pgfscope}%
\pgfsys@transformshift{2.148361in}{0.684408in}%
\pgfsys@useobject{currentmarker}{}%
\end{pgfscope}%
\begin{pgfscope}%
\pgfsys@transformshift{2.148589in}{0.685674in}%
\pgfsys@useobject{currentmarker}{}%
\end{pgfscope}%
\begin{pgfscope}%
\pgfsys@transformshift{2.148817in}{0.702998in}%
\pgfsys@useobject{currentmarker}{}%
\end{pgfscope}%
\begin{pgfscope}%
\pgfsys@transformshift{2.149044in}{0.652902in}%
\pgfsys@useobject{currentmarker}{}%
\end{pgfscope}%
\begin{pgfscope}%
\pgfsys@transformshift{2.149271in}{0.637779in}%
\pgfsys@useobject{currentmarker}{}%
\end{pgfscope}%
\begin{pgfscope}%
\pgfsys@transformshift{2.149499in}{0.683954in}%
\pgfsys@useobject{currentmarker}{}%
\end{pgfscope}%
\begin{pgfscope}%
\pgfsys@transformshift{2.149725in}{0.701856in}%
\pgfsys@useobject{currentmarker}{}%
\end{pgfscope}%
\begin{pgfscope}%
\pgfsys@transformshift{2.149952in}{0.685950in}%
\pgfsys@useobject{currentmarker}{}%
\end{pgfscope}%
\begin{pgfscope}%
\pgfsys@transformshift{2.150178in}{0.681614in}%
\pgfsys@useobject{currentmarker}{}%
\end{pgfscope}%
\begin{pgfscope}%
\pgfsys@transformshift{2.150404in}{0.698868in}%
\pgfsys@useobject{currentmarker}{}%
\end{pgfscope}%
\begin{pgfscope}%
\pgfsys@transformshift{2.150630in}{0.725756in}%
\pgfsys@useobject{currentmarker}{}%
\end{pgfscope}%
\begin{pgfscope}%
\pgfsys@transformshift{2.150856in}{0.738601in}%
\pgfsys@useobject{currentmarker}{}%
\end{pgfscope}%
\begin{pgfscope}%
\pgfsys@transformshift{2.151081in}{0.705758in}%
\pgfsys@useobject{currentmarker}{}%
\end{pgfscope}%
\begin{pgfscope}%
\pgfsys@transformshift{2.151307in}{0.711785in}%
\pgfsys@useobject{currentmarker}{}%
\end{pgfscope}%
\begin{pgfscope}%
\pgfsys@transformshift{2.151532in}{0.723298in}%
\pgfsys@useobject{currentmarker}{}%
\end{pgfscope}%
\begin{pgfscope}%
\pgfsys@transformshift{2.151756in}{0.753283in}%
\pgfsys@useobject{currentmarker}{}%
\end{pgfscope}%
\begin{pgfscope}%
\pgfsys@transformshift{2.151981in}{0.742932in}%
\pgfsys@useobject{currentmarker}{}%
\end{pgfscope}%
\begin{pgfscope}%
\pgfsys@transformshift{2.152205in}{0.703405in}%
\pgfsys@useobject{currentmarker}{}%
\end{pgfscope}%
\begin{pgfscope}%
\pgfsys@transformshift{2.152429in}{0.694349in}%
\pgfsys@useobject{currentmarker}{}%
\end{pgfscope}%
\begin{pgfscope}%
\pgfsys@transformshift{2.152653in}{0.701611in}%
\pgfsys@useobject{currentmarker}{}%
\end{pgfscope}%
\begin{pgfscope}%
\pgfsys@transformshift{2.152877in}{0.706722in}%
\pgfsys@useobject{currentmarker}{}%
\end{pgfscope}%
\begin{pgfscope}%
\pgfsys@transformshift{2.153100in}{0.712846in}%
\pgfsys@useobject{currentmarker}{}%
\end{pgfscope}%
\begin{pgfscope}%
\pgfsys@transformshift{2.153323in}{0.713600in}%
\pgfsys@useobject{currentmarker}{}%
\end{pgfscope}%
\begin{pgfscope}%
\pgfsys@transformshift{2.153546in}{0.692770in}%
\pgfsys@useobject{currentmarker}{}%
\end{pgfscope}%
\begin{pgfscope}%
\pgfsys@transformshift{2.153769in}{0.683829in}%
\pgfsys@useobject{currentmarker}{}%
\end{pgfscope}%
\begin{pgfscope}%
\pgfsys@transformshift{2.153992in}{0.666580in}%
\pgfsys@useobject{currentmarker}{}%
\end{pgfscope}%
\begin{pgfscope}%
\pgfsys@transformshift{2.154214in}{0.625795in}%
\pgfsys@useobject{currentmarker}{}%
\end{pgfscope}%
\begin{pgfscope}%
\pgfsys@transformshift{2.154436in}{0.633319in}%
\pgfsys@useobject{currentmarker}{}%
\end{pgfscope}%
\begin{pgfscope}%
\pgfsys@transformshift{2.154658in}{0.649150in}%
\pgfsys@useobject{currentmarker}{}%
\end{pgfscope}%
\begin{pgfscope}%
\pgfsys@transformshift{2.154880in}{0.641417in}%
\pgfsys@useobject{currentmarker}{}%
\end{pgfscope}%
\begin{pgfscope}%
\pgfsys@transformshift{2.155101in}{0.616325in}%
\pgfsys@useobject{currentmarker}{}%
\end{pgfscope}%
\begin{pgfscope}%
\pgfsys@transformshift{2.155322in}{0.651545in}%
\pgfsys@useobject{currentmarker}{}%
\end{pgfscope}%
\begin{pgfscope}%
\pgfsys@transformshift{2.155543in}{0.668711in}%
\pgfsys@useobject{currentmarker}{}%
\end{pgfscope}%
\begin{pgfscope}%
\pgfsys@transformshift{2.155764in}{0.697307in}%
\pgfsys@useobject{currentmarker}{}%
\end{pgfscope}%
\begin{pgfscope}%
\pgfsys@transformshift{2.155985in}{0.708155in}%
\pgfsys@useobject{currentmarker}{}%
\end{pgfscope}%
\begin{pgfscope}%
\pgfsys@transformshift{2.156205in}{0.725422in}%
\pgfsys@useobject{currentmarker}{}%
\end{pgfscope}%
\begin{pgfscope}%
\pgfsys@transformshift{2.156425in}{0.697613in}%
\pgfsys@useobject{currentmarker}{}%
\end{pgfscope}%
\begin{pgfscope}%
\pgfsys@transformshift{2.156645in}{0.676023in}%
\pgfsys@useobject{currentmarker}{}%
\end{pgfscope}%
\begin{pgfscope}%
\pgfsys@transformshift{2.156865in}{0.714686in}%
\pgfsys@useobject{currentmarker}{}%
\end{pgfscope}%
\begin{pgfscope}%
\pgfsys@transformshift{2.157084in}{0.705988in}%
\pgfsys@useobject{currentmarker}{}%
\end{pgfscope}%
\begin{pgfscope}%
\pgfsys@transformshift{2.157304in}{0.648566in}%
\pgfsys@useobject{currentmarker}{}%
\end{pgfscope}%
\begin{pgfscope}%
\pgfsys@transformshift{2.157523in}{0.690857in}%
\pgfsys@useobject{currentmarker}{}%
\end{pgfscope}%
\begin{pgfscope}%
\pgfsys@transformshift{2.157741in}{0.715701in}%
\pgfsys@useobject{currentmarker}{}%
\end{pgfscope}%
\begin{pgfscope}%
\pgfsys@transformshift{2.157960in}{0.727885in}%
\pgfsys@useobject{currentmarker}{}%
\end{pgfscope}%
\begin{pgfscope}%
\pgfsys@transformshift{2.158179in}{0.713630in}%
\pgfsys@useobject{currentmarker}{}%
\end{pgfscope}%
\begin{pgfscope}%
\pgfsys@transformshift{2.158397in}{0.678393in}%
\pgfsys@useobject{currentmarker}{}%
\end{pgfscope}%
\begin{pgfscope}%
\pgfsys@transformshift{2.158615in}{0.677546in}%
\pgfsys@useobject{currentmarker}{}%
\end{pgfscope}%
\begin{pgfscope}%
\pgfsys@transformshift{2.158833in}{0.666964in}%
\pgfsys@useobject{currentmarker}{}%
\end{pgfscope}%
\begin{pgfscope}%
\pgfsys@transformshift{2.159050in}{0.683486in}%
\pgfsys@useobject{currentmarker}{}%
\end{pgfscope}%
\begin{pgfscope}%
\pgfsys@transformshift{2.159268in}{0.717500in}%
\pgfsys@useobject{currentmarker}{}%
\end{pgfscope}%
\begin{pgfscope}%
\pgfsys@transformshift{2.159485in}{0.697298in}%
\pgfsys@useobject{currentmarker}{}%
\end{pgfscope}%
\begin{pgfscope}%
\pgfsys@transformshift{2.159702in}{0.698609in}%
\pgfsys@useobject{currentmarker}{}%
\end{pgfscope}%
\begin{pgfscope}%
\pgfsys@transformshift{2.159918in}{0.709981in}%
\pgfsys@useobject{currentmarker}{}%
\end{pgfscope}%
\begin{pgfscope}%
\pgfsys@transformshift{2.160135in}{0.711468in}%
\pgfsys@useobject{currentmarker}{}%
\end{pgfscope}%
\begin{pgfscope}%
\pgfsys@transformshift{2.160351in}{0.695103in}%
\pgfsys@useobject{currentmarker}{}%
\end{pgfscope}%
\begin{pgfscope}%
\pgfsys@transformshift{2.160567in}{0.627772in}%
\pgfsys@useobject{currentmarker}{}%
\end{pgfscope}%
\begin{pgfscope}%
\pgfsys@transformshift{2.160783in}{0.673360in}%
\pgfsys@useobject{currentmarker}{}%
\end{pgfscope}%
\begin{pgfscope}%
\pgfsys@transformshift{2.160999in}{0.635575in}%
\pgfsys@useobject{currentmarker}{}%
\end{pgfscope}%
\begin{pgfscope}%
\pgfsys@transformshift{2.161214in}{0.664083in}%
\pgfsys@useobject{currentmarker}{}%
\end{pgfscope}%
\begin{pgfscope}%
\pgfsys@transformshift{2.161430in}{0.742918in}%
\pgfsys@useobject{currentmarker}{}%
\end{pgfscope}%
\begin{pgfscope}%
\pgfsys@transformshift{2.161645in}{0.731537in}%
\pgfsys@useobject{currentmarker}{}%
\end{pgfscope}%
\begin{pgfscope}%
\pgfsys@transformshift{2.161860in}{0.682687in}%
\pgfsys@useobject{currentmarker}{}%
\end{pgfscope}%
\begin{pgfscope}%
\pgfsys@transformshift{2.162074in}{0.711775in}%
\pgfsys@useobject{currentmarker}{}%
\end{pgfscope}%
\begin{pgfscope}%
\pgfsys@transformshift{2.162289in}{0.680837in}%
\pgfsys@useobject{currentmarker}{}%
\end{pgfscope}%
\begin{pgfscope}%
\pgfsys@transformshift{2.162503in}{0.642610in}%
\pgfsys@useobject{currentmarker}{}%
\end{pgfscope}%
\begin{pgfscope}%
\pgfsys@transformshift{2.162717in}{0.701346in}%
\pgfsys@useobject{currentmarker}{}%
\end{pgfscope}%
\begin{pgfscope}%
\pgfsys@transformshift{2.162931in}{0.728721in}%
\pgfsys@useobject{currentmarker}{}%
\end{pgfscope}%
\begin{pgfscope}%
\pgfsys@transformshift{2.163145in}{0.686114in}%
\pgfsys@useobject{currentmarker}{}%
\end{pgfscope}%
\begin{pgfscope}%
\pgfsys@transformshift{2.163358in}{0.645799in}%
\pgfsys@useobject{currentmarker}{}%
\end{pgfscope}%
\begin{pgfscope}%
\pgfsys@transformshift{2.163571in}{0.681333in}%
\pgfsys@useobject{currentmarker}{}%
\end{pgfscope}%
\begin{pgfscope}%
\pgfsys@transformshift{2.163784in}{0.718259in}%
\pgfsys@useobject{currentmarker}{}%
\end{pgfscope}%
\begin{pgfscope}%
\pgfsys@transformshift{2.163997in}{0.718614in}%
\pgfsys@useobject{currentmarker}{}%
\end{pgfscope}%
\begin{pgfscope}%
\pgfsys@transformshift{2.164210in}{0.684764in}%
\pgfsys@useobject{currentmarker}{}%
\end{pgfscope}%
\begin{pgfscope}%
\pgfsys@transformshift{2.164422in}{0.624946in}%
\pgfsys@useobject{currentmarker}{}%
\end{pgfscope}%
\begin{pgfscope}%
\pgfsys@transformshift{2.164635in}{0.597614in}%
\pgfsys@useobject{currentmarker}{}%
\end{pgfscope}%
\begin{pgfscope}%
\pgfsys@transformshift{2.164847in}{0.608665in}%
\pgfsys@useobject{currentmarker}{}%
\end{pgfscope}%
\begin{pgfscope}%
\pgfsys@transformshift{2.165058in}{0.641295in}%
\pgfsys@useobject{currentmarker}{}%
\end{pgfscope}%
\begin{pgfscope}%
\pgfsys@transformshift{2.165270in}{0.670957in}%
\pgfsys@useobject{currentmarker}{}%
\end{pgfscope}%
\begin{pgfscope}%
\pgfsys@transformshift{2.165481in}{0.672730in}%
\pgfsys@useobject{currentmarker}{}%
\end{pgfscope}%
\begin{pgfscope}%
\pgfsys@transformshift{2.165693in}{0.647972in}%
\pgfsys@useobject{currentmarker}{}%
\end{pgfscope}%
\begin{pgfscope}%
\pgfsys@transformshift{2.165904in}{0.691811in}%
\pgfsys@useobject{currentmarker}{}%
\end{pgfscope}%
\begin{pgfscope}%
\pgfsys@transformshift{2.166115in}{0.708925in}%
\pgfsys@useobject{currentmarker}{}%
\end{pgfscope}%
\begin{pgfscope}%
\pgfsys@transformshift{2.166325in}{0.696077in}%
\pgfsys@useobject{currentmarker}{}%
\end{pgfscope}%
\begin{pgfscope}%
\pgfsys@transformshift{2.166536in}{0.701931in}%
\pgfsys@useobject{currentmarker}{}%
\end{pgfscope}%
\begin{pgfscope}%
\pgfsys@transformshift{2.166746in}{0.701145in}%
\pgfsys@useobject{currentmarker}{}%
\end{pgfscope}%
\begin{pgfscope}%
\pgfsys@transformshift{2.166956in}{0.679993in}%
\pgfsys@useobject{currentmarker}{}%
\end{pgfscope}%
\begin{pgfscope}%
\pgfsys@transformshift{2.167166in}{0.688442in}%
\pgfsys@useobject{currentmarker}{}%
\end{pgfscope}%
\begin{pgfscope}%
\pgfsys@transformshift{2.167375in}{0.696333in}%
\pgfsys@useobject{currentmarker}{}%
\end{pgfscope}%
\begin{pgfscope}%
\pgfsys@transformshift{2.167585in}{0.644643in}%
\pgfsys@useobject{currentmarker}{}%
\end{pgfscope}%
\begin{pgfscope}%
\pgfsys@transformshift{2.167794in}{0.679035in}%
\pgfsys@useobject{currentmarker}{}%
\end{pgfscope}%
\begin{pgfscope}%
\pgfsys@transformshift{2.168003in}{0.696387in}%
\pgfsys@useobject{currentmarker}{}%
\end{pgfscope}%
\begin{pgfscope}%
\pgfsys@transformshift{2.168212in}{0.749094in}%
\pgfsys@useobject{currentmarker}{}%
\end{pgfscope}%
\begin{pgfscope}%
\pgfsys@transformshift{2.168421in}{0.738825in}%
\pgfsys@useobject{currentmarker}{}%
\end{pgfscope}%
\begin{pgfscope}%
\pgfsys@transformshift{2.168629in}{0.695960in}%
\pgfsys@useobject{currentmarker}{}%
\end{pgfscope}%
\begin{pgfscope}%
\pgfsys@transformshift{2.168838in}{0.687099in}%
\pgfsys@useobject{currentmarker}{}%
\end{pgfscope}%
\begin{pgfscope}%
\pgfsys@transformshift{2.169046in}{0.644372in}%
\pgfsys@useobject{currentmarker}{}%
\end{pgfscope}%
\begin{pgfscope}%
\pgfsys@transformshift{2.169254in}{0.675323in}%
\pgfsys@useobject{currentmarker}{}%
\end{pgfscope}%
\begin{pgfscope}%
\pgfsys@transformshift{2.169461in}{0.672887in}%
\pgfsys@useobject{currentmarker}{}%
\end{pgfscope}%
\begin{pgfscope}%
\pgfsys@transformshift{2.169669in}{0.658194in}%
\pgfsys@useobject{currentmarker}{}%
\end{pgfscope}%
\begin{pgfscope}%
\pgfsys@transformshift{2.169876in}{0.665375in}%
\pgfsys@useobject{currentmarker}{}%
\end{pgfscope}%
\begin{pgfscope}%
\pgfsys@transformshift{2.170083in}{0.713265in}%
\pgfsys@useobject{currentmarker}{}%
\end{pgfscope}%
\begin{pgfscope}%
\pgfsys@transformshift{2.170290in}{0.703282in}%
\pgfsys@useobject{currentmarker}{}%
\end{pgfscope}%
\begin{pgfscope}%
\pgfsys@transformshift{2.170497in}{0.685662in}%
\pgfsys@useobject{currentmarker}{}%
\end{pgfscope}%
\begin{pgfscope}%
\pgfsys@transformshift{2.170704in}{0.731755in}%
\pgfsys@useobject{currentmarker}{}%
\end{pgfscope}%
\begin{pgfscope}%
\pgfsys@transformshift{2.170910in}{0.733342in}%
\pgfsys@useobject{currentmarker}{}%
\end{pgfscope}%
\begin{pgfscope}%
\pgfsys@transformshift{2.171116in}{0.646644in}%
\pgfsys@useobject{currentmarker}{}%
\end{pgfscope}%
\begin{pgfscope}%
\pgfsys@transformshift{2.171322in}{0.625717in}%
\pgfsys@useobject{currentmarker}{}%
\end{pgfscope}%
\begin{pgfscope}%
\pgfsys@transformshift{2.171528in}{0.697848in}%
\pgfsys@useobject{currentmarker}{}%
\end{pgfscope}%
\begin{pgfscope}%
\pgfsys@transformshift{2.171734in}{0.713975in}%
\pgfsys@useobject{currentmarker}{}%
\end{pgfscope}%
\begin{pgfscope}%
\pgfsys@transformshift{2.171939in}{0.677776in}%
\pgfsys@useobject{currentmarker}{}%
\end{pgfscope}%
\begin{pgfscope}%
\pgfsys@transformshift{2.172144in}{0.654346in}%
\pgfsys@useobject{currentmarker}{}%
\end{pgfscope}%
\begin{pgfscope}%
\pgfsys@transformshift{2.172350in}{0.700642in}%
\pgfsys@useobject{currentmarker}{}%
\end{pgfscope}%
\begin{pgfscope}%
\pgfsys@transformshift{2.172554in}{0.685776in}%
\pgfsys@useobject{currentmarker}{}%
\end{pgfscope}%
\begin{pgfscope}%
\pgfsys@transformshift{2.172759in}{0.649506in}%
\pgfsys@useobject{currentmarker}{}%
\end{pgfscope}%
\begin{pgfscope}%
\pgfsys@transformshift{2.172964in}{0.655355in}%
\pgfsys@useobject{currentmarker}{}%
\end{pgfscope}%
\begin{pgfscope}%
\pgfsys@transformshift{2.173168in}{0.711693in}%
\pgfsys@useobject{currentmarker}{}%
\end{pgfscope}%
\begin{pgfscope}%
\pgfsys@transformshift{2.173372in}{0.742795in}%
\pgfsys@useobject{currentmarker}{}%
\end{pgfscope}%
\begin{pgfscope}%
\pgfsys@transformshift{2.173576in}{0.684455in}%
\pgfsys@useobject{currentmarker}{}%
\end{pgfscope}%
\begin{pgfscope}%
\pgfsys@transformshift{2.173780in}{0.711509in}%
\pgfsys@useobject{currentmarker}{}%
\end{pgfscope}%
\begin{pgfscope}%
\pgfsys@transformshift{2.173983in}{0.721011in}%
\pgfsys@useobject{currentmarker}{}%
\end{pgfscope}%
\begin{pgfscope}%
\pgfsys@transformshift{2.174187in}{0.689218in}%
\pgfsys@useobject{currentmarker}{}%
\end{pgfscope}%
\begin{pgfscope}%
\pgfsys@transformshift{2.174390in}{0.629086in}%
\pgfsys@useobject{currentmarker}{}%
\end{pgfscope}%
\begin{pgfscope}%
\pgfsys@transformshift{2.174593in}{0.636479in}%
\pgfsys@useobject{currentmarker}{}%
\end{pgfscope}%
\begin{pgfscope}%
\pgfsys@transformshift{2.174796in}{0.670238in}%
\pgfsys@useobject{currentmarker}{}%
\end{pgfscope}%
\begin{pgfscope}%
\pgfsys@transformshift{2.174999in}{0.680191in}%
\pgfsys@useobject{currentmarker}{}%
\end{pgfscope}%
\begin{pgfscope}%
\pgfsys@transformshift{2.175201in}{0.677697in}%
\pgfsys@useobject{currentmarker}{}%
\end{pgfscope}%
\begin{pgfscope}%
\pgfsys@transformshift{2.175403in}{0.694819in}%
\pgfsys@useobject{currentmarker}{}%
\end{pgfscope}%
\begin{pgfscope}%
\pgfsys@transformshift{2.175605in}{0.713414in}%
\pgfsys@useobject{currentmarker}{}%
\end{pgfscope}%
\begin{pgfscope}%
\pgfsys@transformshift{2.175807in}{0.722881in}%
\pgfsys@useobject{currentmarker}{}%
\end{pgfscope}%
\begin{pgfscope}%
\pgfsys@transformshift{2.176009in}{0.719058in}%
\pgfsys@useobject{currentmarker}{}%
\end{pgfscope}%
\begin{pgfscope}%
\pgfsys@transformshift{2.176211in}{0.696879in}%
\pgfsys@useobject{currentmarker}{}%
\end{pgfscope}%
\begin{pgfscope}%
\pgfsys@transformshift{2.176412in}{0.708081in}%
\pgfsys@useobject{currentmarker}{}%
\end{pgfscope}%
\begin{pgfscope}%
\pgfsys@transformshift{2.176613in}{0.674268in}%
\pgfsys@useobject{currentmarker}{}%
\end{pgfscope}%
\begin{pgfscope}%
\pgfsys@transformshift{2.176814in}{0.652944in}%
\pgfsys@useobject{currentmarker}{}%
\end{pgfscope}%
\begin{pgfscope}%
\pgfsys@transformshift{2.177015in}{0.676330in}%
\pgfsys@useobject{currentmarker}{}%
\end{pgfscope}%
\begin{pgfscope}%
\pgfsys@transformshift{2.177216in}{0.685890in}%
\pgfsys@useobject{currentmarker}{}%
\end{pgfscope}%
\begin{pgfscope}%
\pgfsys@transformshift{2.177416in}{0.658205in}%
\pgfsys@useobject{currentmarker}{}%
\end{pgfscope}%
\begin{pgfscope}%
\pgfsys@transformshift{2.177617in}{0.661454in}%
\pgfsys@useobject{currentmarker}{}%
\end{pgfscope}%
\begin{pgfscope}%
\pgfsys@transformshift{2.177817in}{0.742054in}%
\pgfsys@useobject{currentmarker}{}%
\end{pgfscope}%
\begin{pgfscope}%
\pgfsys@transformshift{2.178017in}{0.713410in}%
\pgfsys@useobject{currentmarker}{}%
\end{pgfscope}%
\begin{pgfscope}%
\pgfsys@transformshift{2.178217in}{0.702302in}%
\pgfsys@useobject{currentmarker}{}%
\end{pgfscope}%
\begin{pgfscope}%
\pgfsys@transformshift{2.178416in}{0.666123in}%
\pgfsys@useobject{currentmarker}{}%
\end{pgfscope}%
\begin{pgfscope}%
\pgfsys@transformshift{2.178616in}{0.665098in}%
\pgfsys@useobject{currentmarker}{}%
\end{pgfscope}%
\begin{pgfscope}%
\pgfsys@transformshift{2.178815in}{0.699389in}%
\pgfsys@useobject{currentmarker}{}%
\end{pgfscope}%
\begin{pgfscope}%
\pgfsys@transformshift{2.179014in}{0.666185in}%
\pgfsys@useobject{currentmarker}{}%
\end{pgfscope}%
\begin{pgfscope}%
\pgfsys@transformshift{2.179213in}{0.645196in}%
\pgfsys@useobject{currentmarker}{}%
\end{pgfscope}%
\begin{pgfscope}%
\pgfsys@transformshift{2.179411in}{0.667116in}%
\pgfsys@useobject{currentmarker}{}%
\end{pgfscope}%
\begin{pgfscope}%
\pgfsys@transformshift{2.179610in}{0.719953in}%
\pgfsys@useobject{currentmarker}{}%
\end{pgfscope}%
\begin{pgfscope}%
\pgfsys@transformshift{2.179808in}{0.694452in}%
\pgfsys@useobject{currentmarker}{}%
\end{pgfscope}%
\begin{pgfscope}%
\pgfsys@transformshift{2.180007in}{0.697109in}%
\pgfsys@useobject{currentmarker}{}%
\end{pgfscope}%
\begin{pgfscope}%
\pgfsys@transformshift{2.180205in}{0.709678in}%
\pgfsys@useobject{currentmarker}{}%
\end{pgfscope}%
\begin{pgfscope}%
\pgfsys@transformshift{2.180402in}{0.677380in}%
\pgfsys@useobject{currentmarker}{}%
\end{pgfscope}%
\begin{pgfscope}%
\pgfsys@transformshift{2.180600in}{0.699566in}%
\pgfsys@useobject{currentmarker}{}%
\end{pgfscope}%
\begin{pgfscope}%
\pgfsys@transformshift{2.180798in}{0.715782in}%
\pgfsys@useobject{currentmarker}{}%
\end{pgfscope}%
\begin{pgfscope}%
\pgfsys@transformshift{2.180995in}{0.713088in}%
\pgfsys@useobject{currentmarker}{}%
\end{pgfscope}%
\begin{pgfscope}%
\pgfsys@transformshift{2.181192in}{0.692449in}%
\pgfsys@useobject{currentmarker}{}%
\end{pgfscope}%
\begin{pgfscope}%
\pgfsys@transformshift{2.181389in}{0.674588in}%
\pgfsys@useobject{currentmarker}{}%
\end{pgfscope}%
\begin{pgfscope}%
\pgfsys@transformshift{2.181586in}{0.677246in}%
\pgfsys@useobject{currentmarker}{}%
\end{pgfscope}%
\begin{pgfscope}%
\pgfsys@transformshift{2.181782in}{0.710036in}%
\pgfsys@useobject{currentmarker}{}%
\end{pgfscope}%
\begin{pgfscope}%
\pgfsys@transformshift{2.181979in}{0.654459in}%
\pgfsys@useobject{currentmarker}{}%
\end{pgfscope}%
\begin{pgfscope}%
\pgfsys@transformshift{2.182175in}{0.657518in}%
\pgfsys@useobject{currentmarker}{}%
\end{pgfscope}%
\begin{pgfscope}%
\pgfsys@transformshift{2.182371in}{0.725195in}%
\pgfsys@useobject{currentmarker}{}%
\end{pgfscope}%
\begin{pgfscope}%
\pgfsys@transformshift{2.182567in}{0.737707in}%
\pgfsys@useobject{currentmarker}{}%
\end{pgfscope}%
\begin{pgfscope}%
\pgfsys@transformshift{2.182763in}{0.700662in}%
\pgfsys@useobject{currentmarker}{}%
\end{pgfscope}%
\begin{pgfscope}%
\pgfsys@transformshift{2.182959in}{0.694466in}%
\pgfsys@useobject{currentmarker}{}%
\end{pgfscope}%
\begin{pgfscope}%
\pgfsys@transformshift{2.183154in}{0.698805in}%
\pgfsys@useobject{currentmarker}{}%
\end{pgfscope}%
\begin{pgfscope}%
\pgfsys@transformshift{2.183349in}{0.672956in}%
\pgfsys@useobject{currentmarker}{}%
\end{pgfscope}%
\begin{pgfscope}%
\pgfsys@transformshift{2.183544in}{0.651904in}%
\pgfsys@useobject{currentmarker}{}%
\end{pgfscope}%
\begin{pgfscope}%
\pgfsys@transformshift{2.183739in}{0.644893in}%
\pgfsys@useobject{currentmarker}{}%
\end{pgfscope}%
\begin{pgfscope}%
\pgfsys@transformshift{2.183934in}{0.664196in}%
\pgfsys@useobject{currentmarker}{}%
\end{pgfscope}%
\begin{pgfscope}%
\pgfsys@transformshift{2.184129in}{0.698353in}%
\pgfsys@useobject{currentmarker}{}%
\end{pgfscope}%
\begin{pgfscope}%
\pgfsys@transformshift{2.184323in}{0.699403in}%
\pgfsys@useobject{currentmarker}{}%
\end{pgfscope}%
\begin{pgfscope}%
\pgfsys@transformshift{2.184517in}{0.711595in}%
\pgfsys@useobject{currentmarker}{}%
\end{pgfscope}%
\begin{pgfscope}%
\pgfsys@transformshift{2.184711in}{0.714044in}%
\pgfsys@useobject{currentmarker}{}%
\end{pgfscope}%
\begin{pgfscope}%
\pgfsys@transformshift{2.184905in}{0.723616in}%
\pgfsys@useobject{currentmarker}{}%
\end{pgfscope}%
\begin{pgfscope}%
\pgfsys@transformshift{2.185099in}{0.697467in}%
\pgfsys@useobject{currentmarker}{}%
\end{pgfscope}%
\begin{pgfscope}%
\pgfsys@transformshift{2.185293in}{0.669222in}%
\pgfsys@useobject{currentmarker}{}%
\end{pgfscope}%
\begin{pgfscope}%
\pgfsys@transformshift{2.185486in}{0.665594in}%
\pgfsys@useobject{currentmarker}{}%
\end{pgfscope}%
\begin{pgfscope}%
\pgfsys@transformshift{2.185679in}{0.634014in}%
\pgfsys@useobject{currentmarker}{}%
\end{pgfscope}%
\begin{pgfscope}%
\pgfsys@transformshift{2.185872in}{0.636374in}%
\pgfsys@useobject{currentmarker}{}%
\end{pgfscope}%
\begin{pgfscope}%
\pgfsys@transformshift{2.186065in}{0.671994in}%
\pgfsys@useobject{currentmarker}{}%
\end{pgfscope}%
\begin{pgfscope}%
\pgfsys@transformshift{2.186258in}{0.701422in}%
\pgfsys@useobject{currentmarker}{}%
\end{pgfscope}%
\begin{pgfscope}%
\pgfsys@transformshift{2.186451in}{0.703415in}%
\pgfsys@useobject{currentmarker}{}%
\end{pgfscope}%
\begin{pgfscope}%
\pgfsys@transformshift{2.186643in}{0.716826in}%
\pgfsys@useobject{currentmarker}{}%
\end{pgfscope}%
\begin{pgfscope}%
\pgfsys@transformshift{2.186835in}{0.740229in}%
\pgfsys@useobject{currentmarker}{}%
\end{pgfscope}%
\begin{pgfscope}%
\pgfsys@transformshift{2.187028in}{0.738678in}%
\pgfsys@useobject{currentmarker}{}%
\end{pgfscope}%
\begin{pgfscope}%
\pgfsys@transformshift{2.187219in}{0.756909in}%
\pgfsys@useobject{currentmarker}{}%
\end{pgfscope}%
\begin{pgfscope}%
\pgfsys@transformshift{2.187411in}{0.736522in}%
\pgfsys@useobject{currentmarker}{}%
\end{pgfscope}%
\begin{pgfscope}%
\pgfsys@transformshift{2.187603in}{0.693028in}%
\pgfsys@useobject{currentmarker}{}%
\end{pgfscope}%
\begin{pgfscope}%
\pgfsys@transformshift{2.187794in}{0.721176in}%
\pgfsys@useobject{currentmarker}{}%
\end{pgfscope}%
\begin{pgfscope}%
\pgfsys@transformshift{2.187986in}{0.713530in}%
\pgfsys@useobject{currentmarker}{}%
\end{pgfscope}%
\begin{pgfscope}%
\pgfsys@transformshift{2.188177in}{0.684664in}%
\pgfsys@useobject{currentmarker}{}%
\end{pgfscope}%
\begin{pgfscope}%
\pgfsys@transformshift{2.188368in}{0.651424in}%
\pgfsys@useobject{currentmarker}{}%
\end{pgfscope}%
\begin{pgfscope}%
\pgfsys@transformshift{2.188558in}{0.666047in}%
\pgfsys@useobject{currentmarker}{}%
\end{pgfscope}%
\begin{pgfscope}%
\pgfsys@transformshift{2.188749in}{0.708207in}%
\pgfsys@useobject{currentmarker}{}%
\end{pgfscope}%
\begin{pgfscope}%
\pgfsys@transformshift{2.188939in}{0.690829in}%
\pgfsys@useobject{currentmarker}{}%
\end{pgfscope}%
\begin{pgfscope}%
\pgfsys@transformshift{2.189130in}{0.675269in}%
\pgfsys@useobject{currentmarker}{}%
\end{pgfscope}%
\begin{pgfscope}%
\pgfsys@transformshift{2.189320in}{0.674532in}%
\pgfsys@useobject{currentmarker}{}%
\end{pgfscope}%
\begin{pgfscope}%
\pgfsys@transformshift{2.189510in}{0.696183in}%
\pgfsys@useobject{currentmarker}{}%
\end{pgfscope}%
\begin{pgfscope}%
\pgfsys@transformshift{2.189700in}{0.702756in}%
\pgfsys@useobject{currentmarker}{}%
\end{pgfscope}%
\begin{pgfscope}%
\pgfsys@transformshift{2.189889in}{0.713369in}%
\pgfsys@useobject{currentmarker}{}%
\end{pgfscope}%
\begin{pgfscope}%
\pgfsys@transformshift{2.190079in}{0.690744in}%
\pgfsys@useobject{currentmarker}{}%
\end{pgfscope}%
\begin{pgfscope}%
\pgfsys@transformshift{2.190268in}{0.690628in}%
\pgfsys@useobject{currentmarker}{}%
\end{pgfscope}%
\begin{pgfscope}%
\pgfsys@transformshift{2.190457in}{0.649342in}%
\pgfsys@useobject{currentmarker}{}%
\end{pgfscope}%
\begin{pgfscope}%
\pgfsys@transformshift{2.190646in}{0.578642in}%
\pgfsys@useobject{currentmarker}{}%
\end{pgfscope}%
\begin{pgfscope}%
\pgfsys@transformshift{2.190835in}{0.649178in}%
\pgfsys@useobject{currentmarker}{}%
\end{pgfscope}%
\begin{pgfscope}%
\pgfsys@transformshift{2.191024in}{0.694780in}%
\pgfsys@useobject{currentmarker}{}%
\end{pgfscope}%
\begin{pgfscope}%
\pgfsys@transformshift{2.191213in}{0.689650in}%
\pgfsys@useobject{currentmarker}{}%
\end{pgfscope}%
\begin{pgfscope}%
\pgfsys@transformshift{2.191401in}{0.670119in}%
\pgfsys@useobject{currentmarker}{}%
\end{pgfscope}%
\begin{pgfscope}%
\pgfsys@transformshift{2.191589in}{0.642852in}%
\pgfsys@useobject{currentmarker}{}%
\end{pgfscope}%
\begin{pgfscope}%
\pgfsys@transformshift{2.191777in}{0.665341in}%
\pgfsys@useobject{currentmarker}{}%
\end{pgfscope}%
\begin{pgfscope}%
\pgfsys@transformshift{2.191965in}{0.702653in}%
\pgfsys@useobject{currentmarker}{}%
\end{pgfscope}%
\begin{pgfscope}%
\pgfsys@transformshift{2.192153in}{0.668017in}%
\pgfsys@useobject{currentmarker}{}%
\end{pgfscope}%
\begin{pgfscope}%
\pgfsys@transformshift{2.192340in}{0.711816in}%
\pgfsys@useobject{currentmarker}{}%
\end{pgfscope}%
\begin{pgfscope}%
\pgfsys@transformshift{2.192528in}{0.741393in}%
\pgfsys@useobject{currentmarker}{}%
\end{pgfscope}%
\begin{pgfscope}%
\pgfsys@transformshift{2.192715in}{0.700353in}%
\pgfsys@useobject{currentmarker}{}%
\end{pgfscope}%
\begin{pgfscope}%
\pgfsys@transformshift{2.192902in}{0.688983in}%
\pgfsys@useobject{currentmarker}{}%
\end{pgfscope}%
\begin{pgfscope}%
\pgfsys@transformshift{2.193089in}{0.710088in}%
\pgfsys@useobject{currentmarker}{}%
\end{pgfscope}%
\begin{pgfscope}%
\pgfsys@transformshift{2.193276in}{0.710933in}%
\pgfsys@useobject{currentmarker}{}%
\end{pgfscope}%
\begin{pgfscope}%
\pgfsys@transformshift{2.193463in}{0.683686in}%
\pgfsys@useobject{currentmarker}{}%
\end{pgfscope}%
\begin{pgfscope}%
\pgfsys@transformshift{2.193649in}{0.692132in}%
\pgfsys@useobject{currentmarker}{}%
\end{pgfscope}%
\begin{pgfscope}%
\pgfsys@transformshift{2.193836in}{0.707899in}%
\pgfsys@useobject{currentmarker}{}%
\end{pgfscope}%
\begin{pgfscope}%
\pgfsys@transformshift{2.194022in}{0.689390in}%
\pgfsys@useobject{currentmarker}{}%
\end{pgfscope}%
\begin{pgfscope}%
\pgfsys@transformshift{2.194208in}{0.686547in}%
\pgfsys@useobject{currentmarker}{}%
\end{pgfscope}%
\begin{pgfscope}%
\pgfsys@transformshift{2.194394in}{0.688339in}%
\pgfsys@useobject{currentmarker}{}%
\end{pgfscope}%
\begin{pgfscope}%
\pgfsys@transformshift{2.194580in}{0.706259in}%
\pgfsys@useobject{currentmarker}{}%
\end{pgfscope}%
\begin{pgfscope}%
\pgfsys@transformshift{2.194765in}{0.749049in}%
\pgfsys@useobject{currentmarker}{}%
\end{pgfscope}%
\begin{pgfscope}%
\pgfsys@transformshift{2.194951in}{0.730673in}%
\pgfsys@useobject{currentmarker}{}%
\end{pgfscope}%
\begin{pgfscope}%
\pgfsys@transformshift{2.195136in}{0.700597in}%
\pgfsys@useobject{currentmarker}{}%
\end{pgfscope}%
\begin{pgfscope}%
\pgfsys@transformshift{2.195321in}{0.708377in}%
\pgfsys@useobject{currentmarker}{}%
\end{pgfscope}%
\begin{pgfscope}%
\pgfsys@transformshift{2.195506in}{0.707488in}%
\pgfsys@useobject{currentmarker}{}%
\end{pgfscope}%
\begin{pgfscope}%
\pgfsys@transformshift{2.195691in}{0.620732in}%
\pgfsys@useobject{currentmarker}{}%
\end{pgfscope}%
\begin{pgfscope}%
\pgfsys@transformshift{2.195875in}{0.676553in}%
\pgfsys@useobject{currentmarker}{}%
\end{pgfscope}%
\begin{pgfscope}%
\pgfsys@transformshift{2.196060in}{0.682485in}%
\pgfsys@useobject{currentmarker}{}%
\end{pgfscope}%
\begin{pgfscope}%
\pgfsys@transformshift{2.196244in}{0.711136in}%
\pgfsys@useobject{currentmarker}{}%
\end{pgfscope}%
\begin{pgfscope}%
\pgfsys@transformshift{2.196429in}{0.693215in}%
\pgfsys@useobject{currentmarker}{}%
\end{pgfscope}%
\begin{pgfscope}%
\pgfsys@transformshift{2.196613in}{0.688525in}%
\pgfsys@useobject{currentmarker}{}%
\end{pgfscope}%
\begin{pgfscope}%
\pgfsys@transformshift{2.196797in}{0.700968in}%
\pgfsys@useobject{currentmarker}{}%
\end{pgfscope}%
\begin{pgfscope}%
\pgfsys@transformshift{2.196980in}{0.682053in}%
\pgfsys@useobject{currentmarker}{}%
\end{pgfscope}%
\begin{pgfscope}%
\pgfsys@transformshift{2.197164in}{0.749821in}%
\pgfsys@useobject{currentmarker}{}%
\end{pgfscope}%
\begin{pgfscope}%
\pgfsys@transformshift{2.197347in}{0.735825in}%
\pgfsys@useobject{currentmarker}{}%
\end{pgfscope}%
\begin{pgfscope}%
\pgfsys@transformshift{2.197531in}{0.719072in}%
\pgfsys@useobject{currentmarker}{}%
\end{pgfscope}%
\begin{pgfscope}%
\pgfsys@transformshift{2.197714in}{0.712546in}%
\pgfsys@useobject{currentmarker}{}%
\end{pgfscope}%
\begin{pgfscope}%
\pgfsys@transformshift{2.197897in}{0.691153in}%
\pgfsys@useobject{currentmarker}{}%
\end{pgfscope}%
\begin{pgfscope}%
\pgfsys@transformshift{2.198080in}{0.659651in}%
\pgfsys@useobject{currentmarker}{}%
\end{pgfscope}%
\begin{pgfscope}%
\pgfsys@transformshift{2.198262in}{0.696666in}%
\pgfsys@useobject{currentmarker}{}%
\end{pgfscope}%
\begin{pgfscope}%
\pgfsys@transformshift{2.198445in}{0.705294in}%
\pgfsys@useobject{currentmarker}{}%
\end{pgfscope}%
\begin{pgfscope}%
\pgfsys@transformshift{2.198627in}{0.696474in}%
\pgfsys@useobject{currentmarker}{}%
\end{pgfscope}%
\begin{pgfscope}%
\pgfsys@transformshift{2.198810in}{0.680355in}%
\pgfsys@useobject{currentmarker}{}%
\end{pgfscope}%
\begin{pgfscope}%
\pgfsys@transformshift{2.198992in}{0.681122in}%
\pgfsys@useobject{currentmarker}{}%
\end{pgfscope}%
\begin{pgfscope}%
\pgfsys@transformshift{2.199174in}{0.691101in}%
\pgfsys@useobject{currentmarker}{}%
\end{pgfscope}%
\begin{pgfscope}%
\pgfsys@transformshift{2.199356in}{0.690425in}%
\pgfsys@useobject{currentmarker}{}%
\end{pgfscope}%
\begin{pgfscope}%
\pgfsys@transformshift{2.199537in}{0.699674in}%
\pgfsys@useobject{currentmarker}{}%
\end{pgfscope}%
\begin{pgfscope}%
\pgfsys@transformshift{2.199719in}{0.692043in}%
\pgfsys@useobject{currentmarker}{}%
\end{pgfscope}%
\begin{pgfscope}%
\pgfsys@transformshift{2.199900in}{0.675038in}%
\pgfsys@useobject{currentmarker}{}%
\end{pgfscope}%
\begin{pgfscope}%
\pgfsys@transformshift{2.200081in}{0.677072in}%
\pgfsys@useobject{currentmarker}{}%
\end{pgfscope}%
\begin{pgfscope}%
\pgfsys@transformshift{2.200263in}{0.700637in}%
\pgfsys@useobject{currentmarker}{}%
\end{pgfscope}%
\begin{pgfscope}%
\pgfsys@transformshift{2.200444in}{0.700685in}%
\pgfsys@useobject{currentmarker}{}%
\end{pgfscope}%
\begin{pgfscope}%
\pgfsys@transformshift{2.200624in}{0.689386in}%
\pgfsys@useobject{currentmarker}{}%
\end{pgfscope}%
\begin{pgfscope}%
\pgfsys@transformshift{2.200805in}{0.651971in}%
\pgfsys@useobject{currentmarker}{}%
\end{pgfscope}%
\begin{pgfscope}%
\pgfsys@transformshift{2.200986in}{0.675351in}%
\pgfsys@useobject{currentmarker}{}%
\end{pgfscope}%
\begin{pgfscope}%
\pgfsys@transformshift{2.201166in}{0.702693in}%
\pgfsys@useobject{currentmarker}{}%
\end{pgfscope}%
\begin{pgfscope}%
\pgfsys@transformshift{2.201346in}{0.638826in}%
\pgfsys@useobject{currentmarker}{}%
\end{pgfscope}%
\begin{pgfscope}%
\pgfsys@transformshift{2.201526in}{0.614525in}%
\pgfsys@useobject{currentmarker}{}%
\end{pgfscope}%
\begin{pgfscope}%
\pgfsys@transformshift{2.201706in}{0.627732in}%
\pgfsys@useobject{currentmarker}{}%
\end{pgfscope}%
\begin{pgfscope}%
\pgfsys@transformshift{2.201886in}{0.642268in}%
\pgfsys@useobject{currentmarker}{}%
\end{pgfscope}%
\begin{pgfscope}%
\pgfsys@transformshift{2.202066in}{0.672782in}%
\pgfsys@useobject{currentmarker}{}%
\end{pgfscope}%
\begin{pgfscope}%
\pgfsys@transformshift{2.202245in}{0.637925in}%
\pgfsys@useobject{currentmarker}{}%
\end{pgfscope}%
\begin{pgfscope}%
\pgfsys@transformshift{2.202424in}{0.649807in}%
\pgfsys@useobject{currentmarker}{}%
\end{pgfscope}%
\begin{pgfscope}%
\pgfsys@transformshift{2.202604in}{0.665030in}%
\pgfsys@useobject{currentmarker}{}%
\end{pgfscope}%
\begin{pgfscope}%
\pgfsys@transformshift{2.202783in}{0.682822in}%
\pgfsys@useobject{currentmarker}{}%
\end{pgfscope}%
\begin{pgfscope}%
\pgfsys@transformshift{2.202962in}{0.695607in}%
\pgfsys@useobject{currentmarker}{}%
\end{pgfscope}%
\begin{pgfscope}%
\pgfsys@transformshift{2.203140in}{0.678478in}%
\pgfsys@useobject{currentmarker}{}%
\end{pgfscope}%
\begin{pgfscope}%
\pgfsys@transformshift{2.203319in}{0.650414in}%
\pgfsys@useobject{currentmarker}{}%
\end{pgfscope}%
\begin{pgfscope}%
\pgfsys@transformshift{2.203498in}{0.676657in}%
\pgfsys@useobject{currentmarker}{}%
\end{pgfscope}%
\begin{pgfscope}%
\pgfsys@transformshift{2.203676in}{0.735882in}%
\pgfsys@useobject{currentmarker}{}%
\end{pgfscope}%
\begin{pgfscope}%
\pgfsys@transformshift{2.203854in}{0.730899in}%
\pgfsys@useobject{currentmarker}{}%
\end{pgfscope}%
\begin{pgfscope}%
\pgfsys@transformshift{2.204032in}{0.711051in}%
\pgfsys@useobject{currentmarker}{}%
\end{pgfscope}%
\begin{pgfscope}%
\pgfsys@transformshift{2.204210in}{0.647065in}%
\pgfsys@useobject{currentmarker}{}%
\end{pgfscope}%
\begin{pgfscope}%
\pgfsys@transformshift{2.204388in}{0.657818in}%
\pgfsys@useobject{currentmarker}{}%
\end{pgfscope}%
\begin{pgfscope}%
\pgfsys@transformshift{2.204566in}{0.700146in}%
\pgfsys@useobject{currentmarker}{}%
\end{pgfscope}%
\begin{pgfscope}%
\pgfsys@transformshift{2.204743in}{0.714005in}%
\pgfsys@useobject{currentmarker}{}%
\end{pgfscope}%
\begin{pgfscope}%
\pgfsys@transformshift{2.204921in}{0.622037in}%
\pgfsys@useobject{currentmarker}{}%
\end{pgfscope}%
\begin{pgfscope}%
\pgfsys@transformshift{2.205098in}{0.724855in}%
\pgfsys@useobject{currentmarker}{}%
\end{pgfscope}%
\begin{pgfscope}%
\pgfsys@transformshift{2.205275in}{0.710749in}%
\pgfsys@useobject{currentmarker}{}%
\end{pgfscope}%
\begin{pgfscope}%
\pgfsys@transformshift{2.205452in}{0.691575in}%
\pgfsys@useobject{currentmarker}{}%
\end{pgfscope}%
\begin{pgfscope}%
\pgfsys@transformshift{2.205629in}{0.703546in}%
\pgfsys@useobject{currentmarker}{}%
\end{pgfscope}%
\begin{pgfscope}%
\pgfsys@transformshift{2.205805in}{0.680496in}%
\pgfsys@useobject{currentmarker}{}%
\end{pgfscope}%
\begin{pgfscope}%
\pgfsys@transformshift{2.205982in}{0.658757in}%
\pgfsys@useobject{currentmarker}{}%
\end{pgfscope}%
\begin{pgfscope}%
\pgfsys@transformshift{2.206158in}{0.716500in}%
\pgfsys@useobject{currentmarker}{}%
\end{pgfscope}%
\begin{pgfscope}%
\pgfsys@transformshift{2.206335in}{0.715837in}%
\pgfsys@useobject{currentmarker}{}%
\end{pgfscope}%
\begin{pgfscope}%
\pgfsys@transformshift{2.206511in}{0.676328in}%
\pgfsys@useobject{currentmarker}{}%
\end{pgfscope}%
\begin{pgfscope}%
\pgfsys@transformshift{2.206687in}{0.652973in}%
\pgfsys@useobject{currentmarker}{}%
\end{pgfscope}%
\begin{pgfscope}%
\pgfsys@transformshift{2.206863in}{0.684971in}%
\pgfsys@useobject{currentmarker}{}%
\end{pgfscope}%
\begin{pgfscope}%
\pgfsys@transformshift{2.207038in}{0.684302in}%
\pgfsys@useobject{currentmarker}{}%
\end{pgfscope}%
\begin{pgfscope}%
\pgfsys@transformshift{2.207214in}{0.666102in}%
\pgfsys@useobject{currentmarker}{}%
\end{pgfscope}%
\begin{pgfscope}%
\pgfsys@transformshift{2.207389in}{0.718138in}%
\pgfsys@useobject{currentmarker}{}%
\end{pgfscope}%
\begin{pgfscope}%
\pgfsys@transformshift{2.207565in}{0.721955in}%
\pgfsys@useobject{currentmarker}{}%
\end{pgfscope}%
\begin{pgfscope}%
\pgfsys@transformshift{2.207740in}{0.648573in}%
\pgfsys@useobject{currentmarker}{}%
\end{pgfscope}%
\begin{pgfscope}%
\pgfsys@transformshift{2.207915in}{0.639416in}%
\pgfsys@useobject{currentmarker}{}%
\end{pgfscope}%
\begin{pgfscope}%
\pgfsys@transformshift{2.208090in}{0.640161in}%
\pgfsys@useobject{currentmarker}{}%
\end{pgfscope}%
\begin{pgfscope}%
\pgfsys@transformshift{2.208264in}{0.630922in}%
\pgfsys@useobject{currentmarker}{}%
\end{pgfscope}%
\begin{pgfscope}%
\pgfsys@transformshift{2.208439in}{0.668975in}%
\pgfsys@useobject{currentmarker}{}%
\end{pgfscope}%
\begin{pgfscope}%
\pgfsys@transformshift{2.208614in}{0.609224in}%
\pgfsys@useobject{currentmarker}{}%
\end{pgfscope}%
\begin{pgfscope}%
\pgfsys@transformshift{2.208788in}{0.649926in}%
\pgfsys@useobject{currentmarker}{}%
\end{pgfscope}%
\begin{pgfscope}%
\pgfsys@transformshift{2.208962in}{0.660497in}%
\pgfsys@useobject{currentmarker}{}%
\end{pgfscope}%
\begin{pgfscope}%
\pgfsys@transformshift{2.209136in}{0.685356in}%
\pgfsys@useobject{currentmarker}{}%
\end{pgfscope}%
\begin{pgfscope}%
\pgfsys@transformshift{2.209310in}{0.701833in}%
\pgfsys@useobject{currentmarker}{}%
\end{pgfscope}%
\begin{pgfscope}%
\pgfsys@transformshift{2.209484in}{0.696108in}%
\pgfsys@useobject{currentmarker}{}%
\end{pgfscope}%
\begin{pgfscope}%
\pgfsys@transformshift{2.209658in}{0.694036in}%
\pgfsys@useobject{currentmarker}{}%
\end{pgfscope}%
\begin{pgfscope}%
\pgfsys@transformshift{2.209831in}{0.665283in}%
\pgfsys@useobject{currentmarker}{}%
\end{pgfscope}%
\begin{pgfscope}%
\pgfsys@transformshift{2.210005in}{0.686610in}%
\pgfsys@useobject{currentmarker}{}%
\end{pgfscope}%
\begin{pgfscope}%
\pgfsys@transformshift{2.210178in}{0.659683in}%
\pgfsys@useobject{currentmarker}{}%
\end{pgfscope}%
\begin{pgfscope}%
\pgfsys@transformshift{2.210351in}{0.633098in}%
\pgfsys@useobject{currentmarker}{}%
\end{pgfscope}%
\begin{pgfscope}%
\pgfsys@transformshift{2.210524in}{0.653570in}%
\pgfsys@useobject{currentmarker}{}%
\end{pgfscope}%
\begin{pgfscope}%
\pgfsys@transformshift{2.210697in}{0.657469in}%
\pgfsys@useobject{currentmarker}{}%
\end{pgfscope}%
\begin{pgfscope}%
\pgfsys@transformshift{2.210870in}{0.651103in}%
\pgfsys@useobject{currentmarker}{}%
\end{pgfscope}%
\begin{pgfscope}%
\pgfsys@transformshift{2.211042in}{0.661211in}%
\pgfsys@useobject{currentmarker}{}%
\end{pgfscope}%
\begin{pgfscope}%
\pgfsys@transformshift{2.211215in}{0.699584in}%
\pgfsys@useobject{currentmarker}{}%
\end{pgfscope}%
\begin{pgfscope}%
\pgfsys@transformshift{2.211387in}{0.694151in}%
\pgfsys@useobject{currentmarker}{}%
\end{pgfscope}%
\begin{pgfscope}%
\pgfsys@transformshift{2.211559in}{0.650357in}%
\pgfsys@useobject{currentmarker}{}%
\end{pgfscope}%
\begin{pgfscope}%
\pgfsys@transformshift{2.211731in}{0.647085in}%
\pgfsys@useobject{currentmarker}{}%
\end{pgfscope}%
\begin{pgfscope}%
\pgfsys@transformshift{2.211903in}{0.638594in}%
\pgfsys@useobject{currentmarker}{}%
\end{pgfscope}%
\begin{pgfscope}%
\pgfsys@transformshift{2.212075in}{0.600482in}%
\pgfsys@useobject{currentmarker}{}%
\end{pgfscope}%
\begin{pgfscope}%
\pgfsys@transformshift{2.212247in}{0.672459in}%
\pgfsys@useobject{currentmarker}{}%
\end{pgfscope}%
\begin{pgfscope}%
\pgfsys@transformshift{2.212418in}{0.688343in}%
\pgfsys@useobject{currentmarker}{}%
\end{pgfscope}%
\begin{pgfscope}%
\pgfsys@transformshift{2.212590in}{0.674747in}%
\pgfsys@useobject{currentmarker}{}%
\end{pgfscope}%
\begin{pgfscope}%
\pgfsys@transformshift{2.212761in}{0.660007in}%
\pgfsys@useobject{currentmarker}{}%
\end{pgfscope}%
\begin{pgfscope}%
\pgfsys@transformshift{2.212932in}{0.705731in}%
\pgfsys@useobject{currentmarker}{}%
\end{pgfscope}%
\begin{pgfscope}%
\pgfsys@transformshift{2.213103in}{0.704263in}%
\pgfsys@useobject{currentmarker}{}%
\end{pgfscope}%
\begin{pgfscope}%
\pgfsys@transformshift{2.213274in}{0.681712in}%
\pgfsys@useobject{currentmarker}{}%
\end{pgfscope}%
\begin{pgfscope}%
\pgfsys@transformshift{2.213445in}{0.712115in}%
\pgfsys@useobject{currentmarker}{}%
\end{pgfscope}%
\begin{pgfscope}%
\pgfsys@transformshift{2.213616in}{0.682670in}%
\pgfsys@useobject{currentmarker}{}%
\end{pgfscope}%
\begin{pgfscope}%
\pgfsys@transformshift{2.213786in}{0.687291in}%
\pgfsys@useobject{currentmarker}{}%
\end{pgfscope}%
\begin{pgfscope}%
\pgfsys@transformshift{2.213957in}{0.685866in}%
\pgfsys@useobject{currentmarker}{}%
\end{pgfscope}%
\begin{pgfscope}%
\pgfsys@transformshift{2.214127in}{0.692897in}%
\pgfsys@useobject{currentmarker}{}%
\end{pgfscope}%
\begin{pgfscope}%
\pgfsys@transformshift{2.214297in}{0.648852in}%
\pgfsys@useobject{currentmarker}{}%
\end{pgfscope}%
\begin{pgfscope}%
\pgfsys@transformshift{2.214467in}{0.658796in}%
\pgfsys@useobject{currentmarker}{}%
\end{pgfscope}%
\begin{pgfscope}%
\pgfsys@transformshift{2.214637in}{0.724582in}%
\pgfsys@useobject{currentmarker}{}%
\end{pgfscope}%
\begin{pgfscope}%
\pgfsys@transformshift{2.214807in}{0.700298in}%
\pgfsys@useobject{currentmarker}{}%
\end{pgfscope}%
\begin{pgfscope}%
\pgfsys@transformshift{2.214976in}{0.669909in}%
\pgfsys@useobject{currentmarker}{}%
\end{pgfscope}%
\begin{pgfscope}%
\pgfsys@transformshift{2.215146in}{0.667744in}%
\pgfsys@useobject{currentmarker}{}%
\end{pgfscope}%
\begin{pgfscope}%
\pgfsys@transformshift{2.215315in}{0.686014in}%
\pgfsys@useobject{currentmarker}{}%
\end{pgfscope}%
\begin{pgfscope}%
\pgfsys@transformshift{2.215484in}{0.674212in}%
\pgfsys@useobject{currentmarker}{}%
\end{pgfscope}%
\begin{pgfscope}%
\pgfsys@transformshift{2.215653in}{0.723279in}%
\pgfsys@useobject{currentmarker}{}%
\end{pgfscope}%
\begin{pgfscope}%
\pgfsys@transformshift{2.215822in}{0.727257in}%
\pgfsys@useobject{currentmarker}{}%
\end{pgfscope}%
\begin{pgfscope}%
\pgfsys@transformshift{2.215991in}{0.684906in}%
\pgfsys@useobject{currentmarker}{}%
\end{pgfscope}%
\begin{pgfscope}%
\pgfsys@transformshift{2.216160in}{0.657604in}%
\pgfsys@useobject{currentmarker}{}%
\end{pgfscope}%
\begin{pgfscope}%
\pgfsys@transformshift{2.216328in}{0.691976in}%
\pgfsys@useobject{currentmarker}{}%
\end{pgfscope}%
\begin{pgfscope}%
\pgfsys@transformshift{2.216497in}{0.679774in}%
\pgfsys@useobject{currentmarker}{}%
\end{pgfscope}%
\begin{pgfscope}%
\pgfsys@transformshift{2.216665in}{0.720841in}%
\pgfsys@useobject{currentmarker}{}%
\end{pgfscope}%
\begin{pgfscope}%
\pgfsys@transformshift{2.216833in}{0.711150in}%
\pgfsys@useobject{currentmarker}{}%
\end{pgfscope}%
\begin{pgfscope}%
\pgfsys@transformshift{2.217002in}{0.684779in}%
\pgfsys@useobject{currentmarker}{}%
\end{pgfscope}%
\begin{pgfscope}%
\pgfsys@transformshift{2.217170in}{0.668456in}%
\pgfsys@useobject{currentmarker}{}%
\end{pgfscope}%
\begin{pgfscope}%
\pgfsys@transformshift{2.217337in}{0.689799in}%
\pgfsys@useobject{currentmarker}{}%
\end{pgfscope}%
\begin{pgfscope}%
\pgfsys@transformshift{2.217505in}{0.705746in}%
\pgfsys@useobject{currentmarker}{}%
\end{pgfscope}%
\begin{pgfscope}%
\pgfsys@transformshift{2.217673in}{0.699868in}%
\pgfsys@useobject{currentmarker}{}%
\end{pgfscope}%
\begin{pgfscope}%
\pgfsys@transformshift{2.217840in}{0.740808in}%
\pgfsys@useobject{currentmarker}{}%
\end{pgfscope}%
\begin{pgfscope}%
\pgfsys@transformshift{2.218007in}{0.696147in}%
\pgfsys@useobject{currentmarker}{}%
\end{pgfscope}%
\begin{pgfscope}%
\pgfsys@transformshift{2.218175in}{0.661122in}%
\pgfsys@useobject{currentmarker}{}%
\end{pgfscope}%
\begin{pgfscope}%
\pgfsys@transformshift{2.218342in}{0.646083in}%
\pgfsys@useobject{currentmarker}{}%
\end{pgfscope}%
\begin{pgfscope}%
\pgfsys@transformshift{2.218509in}{0.585375in}%
\pgfsys@useobject{currentmarker}{}%
\end{pgfscope}%
\begin{pgfscope}%
\pgfsys@transformshift{2.218676in}{0.671025in}%
\pgfsys@useobject{currentmarker}{}%
\end{pgfscope}%
\begin{pgfscope}%
\pgfsys@transformshift{2.218842in}{0.657438in}%
\pgfsys@useobject{currentmarker}{}%
\end{pgfscope}%
\begin{pgfscope}%
\pgfsys@transformshift{2.219009in}{0.696004in}%
\pgfsys@useobject{currentmarker}{}%
\end{pgfscope}%
\begin{pgfscope}%
\pgfsys@transformshift{2.219175in}{0.678122in}%
\pgfsys@useobject{currentmarker}{}%
\end{pgfscope}%
\begin{pgfscope}%
\pgfsys@transformshift{2.219342in}{0.717930in}%
\pgfsys@useobject{currentmarker}{}%
\end{pgfscope}%
\begin{pgfscope}%
\pgfsys@transformshift{2.219508in}{0.751045in}%
\pgfsys@useobject{currentmarker}{}%
\end{pgfscope}%
\begin{pgfscope}%
\pgfsys@transformshift{2.219674in}{0.661198in}%
\pgfsys@useobject{currentmarker}{}%
\end{pgfscope}%
\begin{pgfscope}%
\pgfsys@transformshift{2.219840in}{0.678201in}%
\pgfsys@useobject{currentmarker}{}%
\end{pgfscope}%
\begin{pgfscope}%
\pgfsys@transformshift{2.220006in}{0.716313in}%
\pgfsys@useobject{currentmarker}{}%
\end{pgfscope}%
\begin{pgfscope}%
\pgfsys@transformshift{2.220172in}{0.729731in}%
\pgfsys@useobject{currentmarker}{}%
\end{pgfscope}%
\begin{pgfscope}%
\pgfsys@transformshift{2.220337in}{0.698281in}%
\pgfsys@useobject{currentmarker}{}%
\end{pgfscope}%
\begin{pgfscope}%
\pgfsys@transformshift{2.220503in}{0.651955in}%
\pgfsys@useobject{currentmarker}{}%
\end{pgfscope}%
\begin{pgfscope}%
\pgfsys@transformshift{2.220668in}{0.658593in}%
\pgfsys@useobject{currentmarker}{}%
\end{pgfscope}%
\begin{pgfscope}%
\pgfsys@transformshift{2.220833in}{0.662169in}%
\pgfsys@useobject{currentmarker}{}%
\end{pgfscope}%
\begin{pgfscope}%
\pgfsys@transformshift{2.220998in}{0.661269in}%
\pgfsys@useobject{currentmarker}{}%
\end{pgfscope}%
\begin{pgfscope}%
\pgfsys@transformshift{2.221163in}{0.691774in}%
\pgfsys@useobject{currentmarker}{}%
\end{pgfscope}%
\begin{pgfscope}%
\pgfsys@transformshift{2.221328in}{0.684020in}%
\pgfsys@useobject{currentmarker}{}%
\end{pgfscope}%
\begin{pgfscope}%
\pgfsys@transformshift{2.221493in}{0.675992in}%
\pgfsys@useobject{currentmarker}{}%
\end{pgfscope}%
\begin{pgfscope}%
\pgfsys@transformshift{2.221658in}{0.737301in}%
\pgfsys@useobject{currentmarker}{}%
\end{pgfscope}%
\begin{pgfscope}%
\pgfsys@transformshift{2.221822in}{0.695076in}%
\pgfsys@useobject{currentmarker}{}%
\end{pgfscope}%
\begin{pgfscope}%
\pgfsys@transformshift{2.221987in}{0.644947in}%
\pgfsys@useobject{currentmarker}{}%
\end{pgfscope}%
\begin{pgfscope}%
\pgfsys@transformshift{2.222151in}{0.661242in}%
\pgfsys@useobject{currentmarker}{}%
\end{pgfscope}%
\begin{pgfscope}%
\pgfsys@transformshift{2.222315in}{0.657020in}%
\pgfsys@useobject{currentmarker}{}%
\end{pgfscope}%
\begin{pgfscope}%
\pgfsys@transformshift{2.222479in}{0.659632in}%
\pgfsys@useobject{currentmarker}{}%
\end{pgfscope}%
\begin{pgfscope}%
\pgfsys@transformshift{2.222643in}{0.716656in}%
\pgfsys@useobject{currentmarker}{}%
\end{pgfscope}%
\begin{pgfscope}%
\pgfsys@transformshift{2.222807in}{0.726925in}%
\pgfsys@useobject{currentmarker}{}%
\end{pgfscope}%
\begin{pgfscope}%
\pgfsys@transformshift{2.222971in}{0.715557in}%
\pgfsys@useobject{currentmarker}{}%
\end{pgfscope}%
\begin{pgfscope}%
\pgfsys@transformshift{2.223134in}{0.701332in}%
\pgfsys@useobject{currentmarker}{}%
\end{pgfscope}%
\begin{pgfscope}%
\pgfsys@transformshift{2.223298in}{0.724275in}%
\pgfsys@useobject{currentmarker}{}%
\end{pgfscope}%
\begin{pgfscope}%
\pgfsys@transformshift{2.223461in}{0.714163in}%
\pgfsys@useobject{currentmarker}{}%
\end{pgfscope}%
\begin{pgfscope}%
\pgfsys@transformshift{2.223624in}{0.657777in}%
\pgfsys@useobject{currentmarker}{}%
\end{pgfscope}%
\begin{pgfscope}%
\pgfsys@transformshift{2.223787in}{0.686653in}%
\pgfsys@useobject{currentmarker}{}%
\end{pgfscope}%
\begin{pgfscope}%
\pgfsys@transformshift{2.223950in}{0.678309in}%
\pgfsys@useobject{currentmarker}{}%
\end{pgfscope}%
\begin{pgfscope}%
\pgfsys@transformshift{2.224113in}{0.702459in}%
\pgfsys@useobject{currentmarker}{}%
\end{pgfscope}%
\begin{pgfscope}%
\pgfsys@transformshift{2.224276in}{0.738780in}%
\pgfsys@useobject{currentmarker}{}%
\end{pgfscope}%
\begin{pgfscope}%
\pgfsys@transformshift{2.224438in}{0.757536in}%
\pgfsys@useobject{currentmarker}{}%
\end{pgfscope}%
\begin{pgfscope}%
\pgfsys@transformshift{2.224601in}{0.705056in}%
\pgfsys@useobject{currentmarker}{}%
\end{pgfscope}%
\begin{pgfscope}%
\pgfsys@transformshift{2.224763in}{0.682339in}%
\pgfsys@useobject{currentmarker}{}%
\end{pgfscope}%
\begin{pgfscope}%
\pgfsys@transformshift{2.224925in}{0.684487in}%
\pgfsys@useobject{currentmarker}{}%
\end{pgfscope}%
\begin{pgfscope}%
\pgfsys@transformshift{2.225088in}{0.703625in}%
\pgfsys@useobject{currentmarker}{}%
\end{pgfscope}%
\begin{pgfscope}%
\pgfsys@transformshift{2.225250in}{0.702026in}%
\pgfsys@useobject{currentmarker}{}%
\end{pgfscope}%
\begin{pgfscope}%
\pgfsys@transformshift{2.225412in}{0.647286in}%
\pgfsys@useobject{currentmarker}{}%
\end{pgfscope}%
\begin{pgfscope}%
\pgfsys@transformshift{2.225573in}{0.721000in}%
\pgfsys@useobject{currentmarker}{}%
\end{pgfscope}%
\begin{pgfscope}%
\pgfsys@transformshift{2.225735in}{0.721766in}%
\pgfsys@useobject{currentmarker}{}%
\end{pgfscope}%
\begin{pgfscope}%
\pgfsys@transformshift{2.225897in}{0.707266in}%
\pgfsys@useobject{currentmarker}{}%
\end{pgfscope}%
\begin{pgfscope}%
\pgfsys@transformshift{2.226058in}{0.679520in}%
\pgfsys@useobject{currentmarker}{}%
\end{pgfscope}%
\begin{pgfscope}%
\pgfsys@transformshift{2.226219in}{0.659990in}%
\pgfsys@useobject{currentmarker}{}%
\end{pgfscope}%
\begin{pgfscope}%
\pgfsys@transformshift{2.226381in}{0.686797in}%
\pgfsys@useobject{currentmarker}{}%
\end{pgfscope}%
\begin{pgfscope}%
\pgfsys@transformshift{2.226542in}{0.702805in}%
\pgfsys@useobject{currentmarker}{}%
\end{pgfscope}%
\begin{pgfscope}%
\pgfsys@transformshift{2.226703in}{0.622765in}%
\pgfsys@useobject{currentmarker}{}%
\end{pgfscope}%
\begin{pgfscope}%
\pgfsys@transformshift{2.226863in}{0.640439in}%
\pgfsys@useobject{currentmarker}{}%
\end{pgfscope}%
\begin{pgfscope}%
\pgfsys@transformshift{2.227024in}{0.656246in}%
\pgfsys@useobject{currentmarker}{}%
\end{pgfscope}%
\begin{pgfscope}%
\pgfsys@transformshift{2.227185in}{0.673155in}%
\pgfsys@useobject{currentmarker}{}%
\end{pgfscope}%
\begin{pgfscope}%
\pgfsys@transformshift{2.227345in}{0.667212in}%
\pgfsys@useobject{currentmarker}{}%
\end{pgfscope}%
\begin{pgfscope}%
\pgfsys@transformshift{2.227506in}{0.680015in}%
\pgfsys@useobject{currentmarker}{}%
\end{pgfscope}%
\begin{pgfscope}%
\pgfsys@transformshift{2.227666in}{0.686046in}%
\pgfsys@useobject{currentmarker}{}%
\end{pgfscope}%
\begin{pgfscope}%
\pgfsys@transformshift{2.227826in}{0.666267in}%
\pgfsys@useobject{currentmarker}{}%
\end{pgfscope}%
\begin{pgfscope}%
\pgfsys@transformshift{2.227986in}{0.661303in}%
\pgfsys@useobject{currentmarker}{}%
\end{pgfscope}%
\begin{pgfscope}%
\pgfsys@transformshift{2.228146in}{0.652090in}%
\pgfsys@useobject{currentmarker}{}%
\end{pgfscope}%
\begin{pgfscope}%
\pgfsys@transformshift{2.228306in}{0.697238in}%
\pgfsys@useobject{currentmarker}{}%
\end{pgfscope}%
\begin{pgfscope}%
\pgfsys@transformshift{2.228466in}{0.666765in}%
\pgfsys@useobject{currentmarker}{}%
\end{pgfscope}%
\begin{pgfscope}%
\pgfsys@transformshift{2.228625in}{0.695329in}%
\pgfsys@useobject{currentmarker}{}%
\end{pgfscope}%
\begin{pgfscope}%
\pgfsys@transformshift{2.228785in}{0.686566in}%
\pgfsys@useobject{currentmarker}{}%
\end{pgfscope}%
\begin{pgfscope}%
\pgfsys@transformshift{2.228944in}{0.725253in}%
\pgfsys@useobject{currentmarker}{}%
\end{pgfscope}%
\begin{pgfscope}%
\pgfsys@transformshift{2.229104in}{0.740777in}%
\pgfsys@useobject{currentmarker}{}%
\end{pgfscope}%
\begin{pgfscope}%
\pgfsys@transformshift{2.229263in}{0.751138in}%
\pgfsys@useobject{currentmarker}{}%
\end{pgfscope}%
\begin{pgfscope}%
\pgfsys@transformshift{2.229422in}{0.758356in}%
\pgfsys@useobject{currentmarker}{}%
\end{pgfscope}%
\begin{pgfscope}%
\pgfsys@transformshift{2.229581in}{0.748521in}%
\pgfsys@useobject{currentmarker}{}%
\end{pgfscope}%
\begin{pgfscope}%
\pgfsys@transformshift{2.229740in}{0.755899in}%
\pgfsys@useobject{currentmarker}{}%
\end{pgfscope}%
\begin{pgfscope}%
\pgfsys@transformshift{2.229898in}{0.725562in}%
\pgfsys@useobject{currentmarker}{}%
\end{pgfscope}%
\begin{pgfscope}%
\pgfsys@transformshift{2.230057in}{0.679957in}%
\pgfsys@useobject{currentmarker}{}%
\end{pgfscope}%
\begin{pgfscope}%
\pgfsys@transformshift{2.230215in}{0.698274in}%
\pgfsys@useobject{currentmarker}{}%
\end{pgfscope}%
\begin{pgfscope}%
\pgfsys@transformshift{2.230374in}{0.731303in}%
\pgfsys@useobject{currentmarker}{}%
\end{pgfscope}%
\begin{pgfscope}%
\pgfsys@transformshift{2.230532in}{0.663577in}%
\pgfsys@useobject{currentmarker}{}%
\end{pgfscope}%
\begin{pgfscope}%
\pgfsys@transformshift{2.230690in}{0.711493in}%
\pgfsys@useobject{currentmarker}{}%
\end{pgfscope}%
\begin{pgfscope}%
\pgfsys@transformshift{2.230848in}{0.698840in}%
\pgfsys@useobject{currentmarker}{}%
\end{pgfscope}%
\begin{pgfscope}%
\pgfsys@transformshift{2.231006in}{0.653367in}%
\pgfsys@useobject{currentmarker}{}%
\end{pgfscope}%
\begin{pgfscope}%
\pgfsys@transformshift{2.231164in}{0.707880in}%
\pgfsys@useobject{currentmarker}{}%
\end{pgfscope}%
\begin{pgfscope}%
\pgfsys@transformshift{2.231322in}{0.717248in}%
\pgfsys@useobject{currentmarker}{}%
\end{pgfscope}%
\begin{pgfscope}%
\pgfsys@transformshift{2.231479in}{0.676444in}%
\pgfsys@useobject{currentmarker}{}%
\end{pgfscope}%
\begin{pgfscope}%
\pgfsys@transformshift{2.231637in}{0.665145in}%
\pgfsys@useobject{currentmarker}{}%
\end{pgfscope}%
\begin{pgfscope}%
\pgfsys@transformshift{2.231794in}{0.693127in}%
\pgfsys@useobject{currentmarker}{}%
\end{pgfscope}%
\begin{pgfscope}%
\pgfsys@transformshift{2.231951in}{0.693485in}%
\pgfsys@useobject{currentmarker}{}%
\end{pgfscope}%
\begin{pgfscope}%
\pgfsys@transformshift{2.232109in}{0.691696in}%
\pgfsys@useobject{currentmarker}{}%
\end{pgfscope}%
\begin{pgfscope}%
\pgfsys@transformshift{2.232266in}{0.687416in}%
\pgfsys@useobject{currentmarker}{}%
\end{pgfscope}%
\begin{pgfscope}%
\pgfsys@transformshift{2.232423in}{0.665613in}%
\pgfsys@useobject{currentmarker}{}%
\end{pgfscope}%
\begin{pgfscope}%
\pgfsys@transformshift{2.232579in}{0.683715in}%
\pgfsys@useobject{currentmarker}{}%
\end{pgfscope}%
\begin{pgfscope}%
\pgfsys@transformshift{2.232736in}{0.686296in}%
\pgfsys@useobject{currentmarker}{}%
\end{pgfscope}%
\begin{pgfscope}%
\pgfsys@transformshift{2.232893in}{0.715749in}%
\pgfsys@useobject{currentmarker}{}%
\end{pgfscope}%
\begin{pgfscope}%
\pgfsys@transformshift{2.233049in}{0.709108in}%
\pgfsys@useobject{currentmarker}{}%
\end{pgfscope}%
\begin{pgfscope}%
\pgfsys@transformshift{2.233206in}{0.673375in}%
\pgfsys@useobject{currentmarker}{}%
\end{pgfscope}%
\begin{pgfscope}%
\pgfsys@transformshift{2.233362in}{0.704116in}%
\pgfsys@useobject{currentmarker}{}%
\end{pgfscope}%
\begin{pgfscope}%
\pgfsys@transformshift{2.233518in}{0.679770in}%
\pgfsys@useobject{currentmarker}{}%
\end{pgfscope}%
\begin{pgfscope}%
\pgfsys@transformshift{2.233674in}{0.672353in}%
\pgfsys@useobject{currentmarker}{}%
\end{pgfscope}%
\begin{pgfscope}%
\pgfsys@transformshift{2.233830in}{0.662293in}%
\pgfsys@useobject{currentmarker}{}%
\end{pgfscope}%
\begin{pgfscope}%
\pgfsys@transformshift{2.233986in}{0.671686in}%
\pgfsys@useobject{currentmarker}{}%
\end{pgfscope}%
\begin{pgfscope}%
\pgfsys@transformshift{2.234142in}{0.679862in}%
\pgfsys@useobject{currentmarker}{}%
\end{pgfscope}%
\begin{pgfscope}%
\pgfsys@transformshift{2.234297in}{0.722726in}%
\pgfsys@useobject{currentmarker}{}%
\end{pgfscope}%
\begin{pgfscope}%
\pgfsys@transformshift{2.234453in}{0.726783in}%
\pgfsys@useobject{currentmarker}{}%
\end{pgfscope}%
\begin{pgfscope}%
\pgfsys@transformshift{2.234608in}{0.733836in}%
\pgfsys@useobject{currentmarker}{}%
\end{pgfscope}%
\begin{pgfscope}%
\pgfsys@transformshift{2.234763in}{0.711932in}%
\pgfsys@useobject{currentmarker}{}%
\end{pgfscope}%
\begin{pgfscope}%
\pgfsys@transformshift{2.234919in}{0.684367in}%
\pgfsys@useobject{currentmarker}{}%
\end{pgfscope}%
\begin{pgfscope}%
\pgfsys@transformshift{2.235074in}{0.634948in}%
\pgfsys@useobject{currentmarker}{}%
\end{pgfscope}%
\begin{pgfscope}%
\pgfsys@transformshift{2.235229in}{0.737733in}%
\pgfsys@useobject{currentmarker}{}%
\end{pgfscope}%
\begin{pgfscope}%
\pgfsys@transformshift{2.235384in}{0.740202in}%
\pgfsys@useobject{currentmarker}{}%
\end{pgfscope}%
\begin{pgfscope}%
\pgfsys@transformshift{2.235538in}{0.641980in}%
\pgfsys@useobject{currentmarker}{}%
\end{pgfscope}%
\begin{pgfscope}%
\pgfsys@transformshift{2.235693in}{0.648868in}%
\pgfsys@useobject{currentmarker}{}%
\end{pgfscope}%
\begin{pgfscope}%
\pgfsys@transformshift{2.235848in}{0.663565in}%
\pgfsys@useobject{currentmarker}{}%
\end{pgfscope}%
\begin{pgfscope}%
\pgfsys@transformshift{2.236002in}{0.650692in}%
\pgfsys@useobject{currentmarker}{}%
\end{pgfscope}%
\begin{pgfscope}%
\pgfsys@transformshift{2.236156in}{0.711447in}%
\pgfsys@useobject{currentmarker}{}%
\end{pgfscope}%
\begin{pgfscope}%
\pgfsys@transformshift{2.236311in}{0.695723in}%
\pgfsys@useobject{currentmarker}{}%
\end{pgfscope}%
\begin{pgfscope}%
\pgfsys@transformshift{2.236465in}{0.691482in}%
\pgfsys@useobject{currentmarker}{}%
\end{pgfscope}%
\begin{pgfscope}%
\pgfsys@transformshift{2.236619in}{0.644823in}%
\pgfsys@useobject{currentmarker}{}%
\end{pgfscope}%
\begin{pgfscope}%
\pgfsys@transformshift{2.236773in}{0.674541in}%
\pgfsys@useobject{currentmarker}{}%
\end{pgfscope}%
\begin{pgfscope}%
\pgfsys@transformshift{2.236927in}{0.656272in}%
\pgfsys@useobject{currentmarker}{}%
\end{pgfscope}%
\begin{pgfscope}%
\pgfsys@transformshift{2.237080in}{0.666804in}%
\pgfsys@useobject{currentmarker}{}%
\end{pgfscope}%
\begin{pgfscope}%
\pgfsys@transformshift{2.237234in}{0.657752in}%
\pgfsys@useobject{currentmarker}{}%
\end{pgfscope}%
\begin{pgfscope}%
\pgfsys@transformshift{2.237387in}{0.676573in}%
\pgfsys@useobject{currentmarker}{}%
\end{pgfscope}%
\begin{pgfscope}%
\pgfsys@transformshift{2.237541in}{0.711751in}%
\pgfsys@useobject{currentmarker}{}%
\end{pgfscope}%
\begin{pgfscope}%
\pgfsys@transformshift{2.237694in}{0.681665in}%
\pgfsys@useobject{currentmarker}{}%
\end{pgfscope}%
\begin{pgfscope}%
\pgfsys@transformshift{2.237847in}{0.672634in}%
\pgfsys@useobject{currentmarker}{}%
\end{pgfscope}%
\begin{pgfscope}%
\pgfsys@transformshift{2.238000in}{0.703249in}%
\pgfsys@useobject{currentmarker}{}%
\end{pgfscope}%
\begin{pgfscope}%
\pgfsys@transformshift{2.238153in}{0.661057in}%
\pgfsys@useobject{currentmarker}{}%
\end{pgfscope}%
\begin{pgfscope}%
\pgfsys@transformshift{2.238306in}{0.659780in}%
\pgfsys@useobject{currentmarker}{}%
\end{pgfscope}%
\begin{pgfscope}%
\pgfsys@transformshift{2.238459in}{0.707272in}%
\pgfsys@useobject{currentmarker}{}%
\end{pgfscope}%
\begin{pgfscope}%
\pgfsys@transformshift{2.238612in}{0.683002in}%
\pgfsys@useobject{currentmarker}{}%
\end{pgfscope}%
\begin{pgfscope}%
\pgfsys@transformshift{2.238764in}{0.705149in}%
\pgfsys@useobject{currentmarker}{}%
\end{pgfscope}%
\begin{pgfscope}%
\pgfsys@transformshift{2.238917in}{0.722769in}%
\pgfsys@useobject{currentmarker}{}%
\end{pgfscope}%
\begin{pgfscope}%
\pgfsys@transformshift{2.239069in}{0.741962in}%
\pgfsys@useobject{currentmarker}{}%
\end{pgfscope}%
\begin{pgfscope}%
\pgfsys@transformshift{2.239221in}{0.696466in}%
\pgfsys@useobject{currentmarker}{}%
\end{pgfscope}%
\begin{pgfscope}%
\pgfsys@transformshift{2.239373in}{0.704415in}%
\pgfsys@useobject{currentmarker}{}%
\end{pgfscope}%
\begin{pgfscope}%
\pgfsys@transformshift{2.239525in}{0.719826in}%
\pgfsys@useobject{currentmarker}{}%
\end{pgfscope}%
\begin{pgfscope}%
\pgfsys@transformshift{2.239677in}{0.748941in}%
\pgfsys@useobject{currentmarker}{}%
\end{pgfscope}%
\begin{pgfscope}%
\pgfsys@transformshift{2.239829in}{0.731963in}%
\pgfsys@useobject{currentmarker}{}%
\end{pgfscope}%
\begin{pgfscope}%
\pgfsys@transformshift{2.239981in}{0.660837in}%
\pgfsys@useobject{currentmarker}{}%
\end{pgfscope}%
\begin{pgfscope}%
\pgfsys@transformshift{2.240133in}{0.645675in}%
\pgfsys@useobject{currentmarker}{}%
\end{pgfscope}%
\begin{pgfscope}%
\pgfsys@transformshift{2.240284in}{0.650799in}%
\pgfsys@useobject{currentmarker}{}%
\end{pgfscope}%
\begin{pgfscope}%
\pgfsys@transformshift{2.240436in}{0.605374in}%
\pgfsys@useobject{currentmarker}{}%
\end{pgfscope}%
\begin{pgfscope}%
\pgfsys@transformshift{2.240587in}{0.660979in}%
\pgfsys@useobject{currentmarker}{}%
\end{pgfscope}%
\begin{pgfscope}%
\pgfsys@transformshift{2.240738in}{0.648619in}%
\pgfsys@useobject{currentmarker}{}%
\end{pgfscope}%
\begin{pgfscope}%
\pgfsys@transformshift{2.240889in}{0.705784in}%
\pgfsys@useobject{currentmarker}{}%
\end{pgfscope}%
\begin{pgfscope}%
\pgfsys@transformshift{2.241040in}{0.696269in}%
\pgfsys@useobject{currentmarker}{}%
\end{pgfscope}%
\begin{pgfscope}%
\pgfsys@transformshift{2.241191in}{0.724454in}%
\pgfsys@useobject{currentmarker}{}%
\end{pgfscope}%
\begin{pgfscope}%
\pgfsys@transformshift{2.241342in}{0.749613in}%
\pgfsys@useobject{currentmarker}{}%
\end{pgfscope}%
\begin{pgfscope}%
\pgfsys@transformshift{2.241493in}{0.731932in}%
\pgfsys@useobject{currentmarker}{}%
\end{pgfscope}%
\begin{pgfscope}%
\pgfsys@transformshift{2.241643in}{0.709987in}%
\pgfsys@useobject{currentmarker}{}%
\end{pgfscope}%
\begin{pgfscope}%
\pgfsys@transformshift{2.241794in}{0.713527in}%
\pgfsys@useobject{currentmarker}{}%
\end{pgfscope}%
\begin{pgfscope}%
\pgfsys@transformshift{2.241944in}{0.692584in}%
\pgfsys@useobject{currentmarker}{}%
\end{pgfscope}%
\begin{pgfscope}%
\pgfsys@transformshift{2.242095in}{0.652717in}%
\pgfsys@useobject{currentmarker}{}%
\end{pgfscope}%
\begin{pgfscope}%
\pgfsys@transformshift{2.242245in}{0.645505in}%
\pgfsys@useobject{currentmarker}{}%
\end{pgfscope}%
\begin{pgfscope}%
\pgfsys@transformshift{2.242395in}{0.691403in}%
\pgfsys@useobject{currentmarker}{}%
\end{pgfscope}%
\begin{pgfscope}%
\pgfsys@transformshift{2.242545in}{0.675150in}%
\pgfsys@useobject{currentmarker}{}%
\end{pgfscope}%
\begin{pgfscope}%
\pgfsys@transformshift{2.242695in}{0.639311in}%
\pgfsys@useobject{currentmarker}{}%
\end{pgfscope}%
\begin{pgfscope}%
\pgfsys@transformshift{2.242845in}{0.661447in}%
\pgfsys@useobject{currentmarker}{}%
\end{pgfscope}%
\begin{pgfscope}%
\pgfsys@transformshift{2.242994in}{0.647048in}%
\pgfsys@useobject{currentmarker}{}%
\end{pgfscope}%
\begin{pgfscope}%
\pgfsys@transformshift{2.243144in}{0.655932in}%
\pgfsys@useobject{currentmarker}{}%
\end{pgfscope}%
\begin{pgfscope}%
\pgfsys@transformshift{2.243294in}{0.679296in}%
\pgfsys@useobject{currentmarker}{}%
\end{pgfscope}%
\begin{pgfscope}%
\pgfsys@transformshift{2.243443in}{0.723797in}%
\pgfsys@useobject{currentmarker}{}%
\end{pgfscope}%
\begin{pgfscope}%
\pgfsys@transformshift{2.243592in}{0.701801in}%
\pgfsys@useobject{currentmarker}{}%
\end{pgfscope}%
\begin{pgfscope}%
\pgfsys@transformshift{2.243742in}{0.702370in}%
\pgfsys@useobject{currentmarker}{}%
\end{pgfscope}%
\begin{pgfscope}%
\pgfsys@transformshift{2.243891in}{0.712814in}%
\pgfsys@useobject{currentmarker}{}%
\end{pgfscope}%
\begin{pgfscope}%
\pgfsys@transformshift{2.244040in}{0.700526in}%
\pgfsys@useobject{currentmarker}{}%
\end{pgfscope}%
\begin{pgfscope}%
\pgfsys@transformshift{2.244189in}{0.668277in}%
\pgfsys@useobject{currentmarker}{}%
\end{pgfscope}%
\begin{pgfscope}%
\pgfsys@transformshift{2.244337in}{0.654914in}%
\pgfsys@useobject{currentmarker}{}%
\end{pgfscope}%
\begin{pgfscope}%
\pgfsys@transformshift{2.244486in}{0.696492in}%
\pgfsys@useobject{currentmarker}{}%
\end{pgfscope}%
\begin{pgfscope}%
\pgfsys@transformshift{2.244635in}{0.676456in}%
\pgfsys@useobject{currentmarker}{}%
\end{pgfscope}%
\begin{pgfscope}%
\pgfsys@transformshift{2.244783in}{0.667008in}%
\pgfsys@useobject{currentmarker}{}%
\end{pgfscope}%
\begin{pgfscope}%
\pgfsys@transformshift{2.244932in}{0.648812in}%
\pgfsys@useobject{currentmarker}{}%
\end{pgfscope}%
\begin{pgfscope}%
\pgfsys@transformshift{2.245080in}{0.625223in}%
\pgfsys@useobject{currentmarker}{}%
\end{pgfscope}%
\begin{pgfscope}%
\pgfsys@transformshift{2.245228in}{0.668583in}%
\pgfsys@useobject{currentmarker}{}%
\end{pgfscope}%
\begin{pgfscope}%
\pgfsys@transformshift{2.245377in}{0.679072in}%
\pgfsys@useobject{currentmarker}{}%
\end{pgfscope}%
\begin{pgfscope}%
\pgfsys@transformshift{2.245525in}{0.720256in}%
\pgfsys@useobject{currentmarker}{}%
\end{pgfscope}%
\begin{pgfscope}%
\pgfsys@transformshift{2.245672in}{0.743194in}%
\pgfsys@useobject{currentmarker}{}%
\end{pgfscope}%
\begin{pgfscope}%
\pgfsys@transformshift{2.245820in}{0.696647in}%
\pgfsys@useobject{currentmarker}{}%
\end{pgfscope}%
\begin{pgfscope}%
\pgfsys@transformshift{2.245968in}{0.626885in}%
\pgfsys@useobject{currentmarker}{}%
\end{pgfscope}%
\begin{pgfscope}%
\pgfsys@transformshift{2.246116in}{0.641761in}%
\pgfsys@useobject{currentmarker}{}%
\end{pgfscope}%
\begin{pgfscope}%
\pgfsys@transformshift{2.246263in}{0.686672in}%
\pgfsys@useobject{currentmarker}{}%
\end{pgfscope}%
\begin{pgfscope}%
\pgfsys@transformshift{2.246411in}{0.677292in}%
\pgfsys@useobject{currentmarker}{}%
\end{pgfscope}%
\begin{pgfscope}%
\pgfsys@transformshift{2.246558in}{0.707727in}%
\pgfsys@useobject{currentmarker}{}%
\end{pgfscope}%
\begin{pgfscope}%
\pgfsys@transformshift{2.246705in}{0.679275in}%
\pgfsys@useobject{currentmarker}{}%
\end{pgfscope}%
\begin{pgfscope}%
\pgfsys@transformshift{2.246853in}{0.688945in}%
\pgfsys@useobject{currentmarker}{}%
\end{pgfscope}%
\begin{pgfscope}%
\pgfsys@transformshift{2.247000in}{0.733506in}%
\pgfsys@useobject{currentmarker}{}%
\end{pgfscope}%
\begin{pgfscope}%
\pgfsys@transformshift{2.247147in}{0.708092in}%
\pgfsys@useobject{currentmarker}{}%
\end{pgfscope}%
\begin{pgfscope}%
\pgfsys@transformshift{2.247293in}{0.661198in}%
\pgfsys@useobject{currentmarker}{}%
\end{pgfscope}%
\begin{pgfscope}%
\pgfsys@transformshift{2.247440in}{0.655308in}%
\pgfsys@useobject{currentmarker}{}%
\end{pgfscope}%
\begin{pgfscope}%
\pgfsys@transformshift{2.247587in}{0.666091in}%
\pgfsys@useobject{currentmarker}{}%
\end{pgfscope}%
\begin{pgfscope}%
\pgfsys@transformshift{2.247734in}{0.689839in}%
\pgfsys@useobject{currentmarker}{}%
\end{pgfscope}%
\begin{pgfscope}%
\pgfsys@transformshift{2.247880in}{0.697435in}%
\pgfsys@useobject{currentmarker}{}%
\end{pgfscope}%
\begin{pgfscope}%
\pgfsys@transformshift{2.248026in}{0.709277in}%
\pgfsys@useobject{currentmarker}{}%
\end{pgfscope}%
\begin{pgfscope}%
\pgfsys@transformshift{2.248173in}{0.647779in}%
\pgfsys@useobject{currentmarker}{}%
\end{pgfscope}%
\begin{pgfscope}%
\pgfsys@transformshift{2.248319in}{0.666248in}%
\pgfsys@useobject{currentmarker}{}%
\end{pgfscope}%
\begin{pgfscope}%
\pgfsys@transformshift{2.248465in}{0.664228in}%
\pgfsys@useobject{currentmarker}{}%
\end{pgfscope}%
\begin{pgfscope}%
\pgfsys@transformshift{2.248611in}{0.643787in}%
\pgfsys@useobject{currentmarker}{}%
\end{pgfscope}%
\begin{pgfscope}%
\pgfsys@transformshift{2.248757in}{0.661420in}%
\pgfsys@useobject{currentmarker}{}%
\end{pgfscope}%
\begin{pgfscope}%
\pgfsys@transformshift{2.248903in}{0.681459in}%
\pgfsys@useobject{currentmarker}{}%
\end{pgfscope}%
\begin{pgfscope}%
\pgfsys@transformshift{2.249049in}{0.701286in}%
\pgfsys@useobject{currentmarker}{}%
\end{pgfscope}%
\begin{pgfscope}%
\pgfsys@transformshift{2.249194in}{0.699312in}%
\pgfsys@useobject{currentmarker}{}%
\end{pgfscope}%
\begin{pgfscope}%
\pgfsys@transformshift{2.249340in}{0.710291in}%
\pgfsys@useobject{currentmarker}{}%
\end{pgfscope}%
\begin{pgfscope}%
\pgfsys@transformshift{2.249485in}{0.696681in}%
\pgfsys@useobject{currentmarker}{}%
\end{pgfscope}%
\begin{pgfscope}%
\pgfsys@transformshift{2.249631in}{0.662323in}%
\pgfsys@useobject{currentmarker}{}%
\end{pgfscope}%
\begin{pgfscope}%
\pgfsys@transformshift{2.249776in}{0.691154in}%
\pgfsys@useobject{currentmarker}{}%
\end{pgfscope}%
\begin{pgfscope}%
\pgfsys@transformshift{2.249921in}{0.701195in}%
\pgfsys@useobject{currentmarker}{}%
\end{pgfscope}%
\begin{pgfscope}%
\pgfsys@transformshift{2.250066in}{0.607596in}%
\pgfsys@useobject{currentmarker}{}%
\end{pgfscope}%
\begin{pgfscope}%
\pgfsys@transformshift{2.250211in}{0.626246in}%
\pgfsys@useobject{currentmarker}{}%
\end{pgfscope}%
\begin{pgfscope}%
\pgfsys@transformshift{2.250356in}{0.648008in}%
\pgfsys@useobject{currentmarker}{}%
\end{pgfscope}%
\begin{pgfscope}%
\pgfsys@transformshift{2.250501in}{0.720145in}%
\pgfsys@useobject{currentmarker}{}%
\end{pgfscope}%
\begin{pgfscope}%
\pgfsys@transformshift{2.250645in}{0.720709in}%
\pgfsys@useobject{currentmarker}{}%
\end{pgfscope}%
\begin{pgfscope}%
\pgfsys@transformshift{2.250790in}{0.693837in}%
\pgfsys@useobject{currentmarker}{}%
\end{pgfscope}%
\begin{pgfscope}%
\pgfsys@transformshift{2.250935in}{0.664196in}%
\pgfsys@useobject{currentmarker}{}%
\end{pgfscope}%
\begin{pgfscope}%
\pgfsys@transformshift{2.251079in}{0.699913in}%
\pgfsys@useobject{currentmarker}{}%
\end{pgfscope}%
\begin{pgfscope}%
\pgfsys@transformshift{2.251223in}{0.711852in}%
\pgfsys@useobject{currentmarker}{}%
\end{pgfscope}%
\begin{pgfscope}%
\pgfsys@transformshift{2.251368in}{0.690207in}%
\pgfsys@useobject{currentmarker}{}%
\end{pgfscope}%
\begin{pgfscope}%
\pgfsys@transformshift{2.251512in}{0.689711in}%
\pgfsys@useobject{currentmarker}{}%
\end{pgfscope}%
\begin{pgfscope}%
\pgfsys@transformshift{2.251656in}{0.704059in}%
\pgfsys@useobject{currentmarker}{}%
\end{pgfscope}%
\begin{pgfscope}%
\pgfsys@transformshift{2.251800in}{0.718119in}%
\pgfsys@useobject{currentmarker}{}%
\end{pgfscope}%
\begin{pgfscope}%
\pgfsys@transformshift{2.251944in}{0.706381in}%
\pgfsys@useobject{currentmarker}{}%
\end{pgfscope}%
\begin{pgfscope}%
\pgfsys@transformshift{2.252087in}{0.722297in}%
\pgfsys@useobject{currentmarker}{}%
\end{pgfscope}%
\begin{pgfscope}%
\pgfsys@transformshift{2.252231in}{0.721144in}%
\pgfsys@useobject{currentmarker}{}%
\end{pgfscope}%
\begin{pgfscope}%
\pgfsys@transformshift{2.252375in}{0.668920in}%
\pgfsys@useobject{currentmarker}{}%
\end{pgfscope}%
\begin{pgfscope}%
\pgfsys@transformshift{2.252518in}{0.665444in}%
\pgfsys@useobject{currentmarker}{}%
\end{pgfscope}%
\begin{pgfscope}%
\pgfsys@transformshift{2.252662in}{0.649127in}%
\pgfsys@useobject{currentmarker}{}%
\end{pgfscope}%
\begin{pgfscope}%
\pgfsys@transformshift{2.252805in}{0.687528in}%
\pgfsys@useobject{currentmarker}{}%
\end{pgfscope}%
\begin{pgfscope}%
\pgfsys@transformshift{2.252948in}{0.675091in}%
\pgfsys@useobject{currentmarker}{}%
\end{pgfscope}%
\begin{pgfscope}%
\pgfsys@transformshift{2.253091in}{0.687771in}%
\pgfsys@useobject{currentmarker}{}%
\end{pgfscope}%
\begin{pgfscope}%
\pgfsys@transformshift{2.253234in}{0.706607in}%
\pgfsys@useobject{currentmarker}{}%
\end{pgfscope}%
\begin{pgfscope}%
\pgfsys@transformshift{2.253377in}{0.671624in}%
\pgfsys@useobject{currentmarker}{}%
\end{pgfscope}%
\begin{pgfscope}%
\pgfsys@transformshift{2.253520in}{0.704291in}%
\pgfsys@useobject{currentmarker}{}%
\end{pgfscope}%
\begin{pgfscope}%
\pgfsys@transformshift{2.253663in}{0.735978in}%
\pgfsys@useobject{currentmarker}{}%
\end{pgfscope}%
\begin{pgfscope}%
\pgfsys@transformshift{2.253806in}{0.705543in}%
\pgfsys@useobject{currentmarker}{}%
\end{pgfscope}%
\begin{pgfscope}%
\pgfsys@transformshift{2.253948in}{0.673976in}%
\pgfsys@useobject{currentmarker}{}%
\end{pgfscope}%
\begin{pgfscope}%
\pgfsys@transformshift{2.254091in}{0.698799in}%
\pgfsys@useobject{currentmarker}{}%
\end{pgfscope}%
\begin{pgfscope}%
\pgfsys@transformshift{2.254233in}{0.691079in}%
\pgfsys@useobject{currentmarker}{}%
\end{pgfscope}%
\begin{pgfscope}%
\pgfsys@transformshift{2.254375in}{0.666402in}%
\pgfsys@useobject{currentmarker}{}%
\end{pgfscope}%
\begin{pgfscope}%
\pgfsys@transformshift{2.254518in}{0.651965in}%
\pgfsys@useobject{currentmarker}{}%
\end{pgfscope}%
\begin{pgfscope}%
\pgfsys@transformshift{2.254660in}{0.694072in}%
\pgfsys@useobject{currentmarker}{}%
\end{pgfscope}%
\begin{pgfscope}%
\pgfsys@transformshift{2.254802in}{0.726310in}%
\pgfsys@useobject{currentmarker}{}%
\end{pgfscope}%
\begin{pgfscope}%
\pgfsys@transformshift{2.254944in}{0.690760in}%
\pgfsys@useobject{currentmarker}{}%
\end{pgfscope}%
\begin{pgfscope}%
\pgfsys@transformshift{2.255086in}{0.684186in}%
\pgfsys@useobject{currentmarker}{}%
\end{pgfscope}%
\begin{pgfscope}%
\pgfsys@transformshift{2.255227in}{0.698056in}%
\pgfsys@useobject{currentmarker}{}%
\end{pgfscope}%
\begin{pgfscope}%
\pgfsys@transformshift{2.255369in}{0.701274in}%
\pgfsys@useobject{currentmarker}{}%
\end{pgfscope}%
\begin{pgfscope}%
\pgfsys@transformshift{2.255511in}{0.675253in}%
\pgfsys@useobject{currentmarker}{}%
\end{pgfscope}%
\begin{pgfscope}%
\pgfsys@transformshift{2.255652in}{0.712363in}%
\pgfsys@useobject{currentmarker}{}%
\end{pgfscope}%
\begin{pgfscope}%
\pgfsys@transformshift{2.255794in}{0.693217in}%
\pgfsys@useobject{currentmarker}{}%
\end{pgfscope}%
\begin{pgfscope}%
\pgfsys@transformshift{2.255935in}{0.711714in}%
\pgfsys@useobject{currentmarker}{}%
\end{pgfscope}%
\begin{pgfscope}%
\pgfsys@transformshift{2.256076in}{0.666845in}%
\pgfsys@useobject{currentmarker}{}%
\end{pgfscope}%
\begin{pgfscope}%
\pgfsys@transformshift{2.256217in}{0.674149in}%
\pgfsys@useobject{currentmarker}{}%
\end{pgfscope}%
\begin{pgfscope}%
\pgfsys@transformshift{2.256358in}{0.667928in}%
\pgfsys@useobject{currentmarker}{}%
\end{pgfscope}%
\begin{pgfscope}%
\pgfsys@transformshift{2.256499in}{0.643141in}%
\pgfsys@useobject{currentmarker}{}%
\end{pgfscope}%
\begin{pgfscope}%
\pgfsys@transformshift{2.256640in}{0.702011in}%
\pgfsys@useobject{currentmarker}{}%
\end{pgfscope}%
\begin{pgfscope}%
\pgfsys@transformshift{2.256781in}{0.666684in}%
\pgfsys@useobject{currentmarker}{}%
\end{pgfscope}%
\begin{pgfscope}%
\pgfsys@transformshift{2.256922in}{0.666329in}%
\pgfsys@useobject{currentmarker}{}%
\end{pgfscope}%
\begin{pgfscope}%
\pgfsys@transformshift{2.257062in}{0.708523in}%
\pgfsys@useobject{currentmarker}{}%
\end{pgfscope}%
\begin{pgfscope}%
\pgfsys@transformshift{2.257203in}{0.709929in}%
\pgfsys@useobject{currentmarker}{}%
\end{pgfscope}%
\begin{pgfscope}%
\pgfsys@transformshift{2.257343in}{0.684925in}%
\pgfsys@useobject{currentmarker}{}%
\end{pgfscope}%
\begin{pgfscope}%
\pgfsys@transformshift{2.257484in}{0.648857in}%
\pgfsys@useobject{currentmarker}{}%
\end{pgfscope}%
\begin{pgfscope}%
\pgfsys@transformshift{2.257624in}{0.633283in}%
\pgfsys@useobject{currentmarker}{}%
\end{pgfscope}%
\begin{pgfscope}%
\pgfsys@transformshift{2.257764in}{0.658897in}%
\pgfsys@useobject{currentmarker}{}%
\end{pgfscope}%
\begin{pgfscope}%
\pgfsys@transformshift{2.257904in}{0.655753in}%
\pgfsys@useobject{currentmarker}{}%
\end{pgfscope}%
\begin{pgfscope}%
\pgfsys@transformshift{2.258044in}{0.647393in}%
\pgfsys@useobject{currentmarker}{}%
\end{pgfscope}%
\begin{pgfscope}%
\pgfsys@transformshift{2.258184in}{0.687425in}%
\pgfsys@useobject{currentmarker}{}%
\end{pgfscope}%
\begin{pgfscope}%
\pgfsys@transformshift{2.258324in}{0.640621in}%
\pgfsys@useobject{currentmarker}{}%
\end{pgfscope}%
\begin{pgfscope}%
\pgfsys@transformshift{2.258464in}{0.603059in}%
\pgfsys@useobject{currentmarker}{}%
\end{pgfscope}%
\begin{pgfscope}%
\pgfsys@transformshift{2.258604in}{0.703940in}%
\pgfsys@useobject{currentmarker}{}%
\end{pgfscope}%
\begin{pgfscope}%
\pgfsys@transformshift{2.258743in}{0.720699in}%
\pgfsys@useobject{currentmarker}{}%
\end{pgfscope}%
\begin{pgfscope}%
\pgfsys@transformshift{2.258883in}{0.704533in}%
\pgfsys@useobject{currentmarker}{}%
\end{pgfscope}%
\begin{pgfscope}%
\pgfsys@transformshift{2.259022in}{0.740334in}%
\pgfsys@useobject{currentmarker}{}%
\end{pgfscope}%
\begin{pgfscope}%
\pgfsys@transformshift{2.259161in}{0.714317in}%
\pgfsys@useobject{currentmarker}{}%
\end{pgfscope}%
\begin{pgfscope}%
\pgfsys@transformshift{2.259301in}{0.688347in}%
\pgfsys@useobject{currentmarker}{}%
\end{pgfscope}%
\begin{pgfscope}%
\pgfsys@transformshift{2.259440in}{0.640970in}%
\pgfsys@useobject{currentmarker}{}%
\end{pgfscope}%
\begin{pgfscope}%
\pgfsys@transformshift{2.259579in}{0.643069in}%
\pgfsys@useobject{currentmarker}{}%
\end{pgfscope}%
\begin{pgfscope}%
\pgfsys@transformshift{2.259718in}{0.675350in}%
\pgfsys@useobject{currentmarker}{}%
\end{pgfscope}%
\begin{pgfscope}%
\pgfsys@transformshift{2.259857in}{0.681239in}%
\pgfsys@useobject{currentmarker}{}%
\end{pgfscope}%
\begin{pgfscope}%
\pgfsys@transformshift{2.259995in}{0.671943in}%
\pgfsys@useobject{currentmarker}{}%
\end{pgfscope}%
\begin{pgfscope}%
\pgfsys@transformshift{2.260134in}{0.701763in}%
\pgfsys@useobject{currentmarker}{}%
\end{pgfscope}%
\begin{pgfscope}%
\pgfsys@transformshift{2.260273in}{0.745728in}%
\pgfsys@useobject{currentmarker}{}%
\end{pgfscope}%
\begin{pgfscope}%
\pgfsys@transformshift{2.260411in}{0.700523in}%
\pgfsys@useobject{currentmarker}{}%
\end{pgfscope}%
\begin{pgfscope}%
\pgfsys@transformshift{2.260550in}{0.729340in}%
\pgfsys@useobject{currentmarker}{}%
\end{pgfscope}%
\begin{pgfscope}%
\pgfsys@transformshift{2.260688in}{0.731554in}%
\pgfsys@useobject{currentmarker}{}%
\end{pgfscope}%
\begin{pgfscope}%
\pgfsys@transformshift{2.260827in}{0.695235in}%
\pgfsys@useobject{currentmarker}{}%
\end{pgfscope}%
\begin{pgfscope}%
\pgfsys@transformshift{2.260965in}{0.652564in}%
\pgfsys@useobject{currentmarker}{}%
\end{pgfscope}%
\begin{pgfscope}%
\pgfsys@transformshift{2.261103in}{0.705764in}%
\pgfsys@useobject{currentmarker}{}%
\end{pgfscope}%
\begin{pgfscope}%
\pgfsys@transformshift{2.261241in}{0.738737in}%
\pgfsys@useobject{currentmarker}{}%
\end{pgfscope}%
\begin{pgfscope}%
\pgfsys@transformshift{2.261379in}{0.713694in}%
\pgfsys@useobject{currentmarker}{}%
\end{pgfscope}%
\begin{pgfscope}%
\pgfsys@transformshift{2.261517in}{0.692730in}%
\pgfsys@useobject{currentmarker}{}%
\end{pgfscope}%
\begin{pgfscope}%
\pgfsys@transformshift{2.261654in}{0.717327in}%
\pgfsys@useobject{currentmarker}{}%
\end{pgfscope}%
\begin{pgfscope}%
\pgfsys@transformshift{2.261792in}{0.701264in}%
\pgfsys@useobject{currentmarker}{}%
\end{pgfscope}%
\begin{pgfscope}%
\pgfsys@transformshift{2.261930in}{0.648040in}%
\pgfsys@useobject{currentmarker}{}%
\end{pgfscope}%
\begin{pgfscope}%
\pgfsys@transformshift{2.262067in}{0.695017in}%
\pgfsys@useobject{currentmarker}{}%
\end{pgfscope}%
\begin{pgfscope}%
\pgfsys@transformshift{2.262205in}{0.686584in}%
\pgfsys@useobject{currentmarker}{}%
\end{pgfscope}%
\begin{pgfscope}%
\pgfsys@transformshift{2.262342in}{0.673170in}%
\pgfsys@useobject{currentmarker}{}%
\end{pgfscope}%
\begin{pgfscope}%
\pgfsys@transformshift{2.262479in}{0.668230in}%
\pgfsys@useobject{currentmarker}{}%
\end{pgfscope}%
\begin{pgfscope}%
\pgfsys@transformshift{2.262617in}{0.695651in}%
\pgfsys@useobject{currentmarker}{}%
\end{pgfscope}%
\begin{pgfscope}%
\pgfsys@transformshift{2.262754in}{0.677191in}%
\pgfsys@useobject{currentmarker}{}%
\end{pgfscope}%
\begin{pgfscope}%
\pgfsys@transformshift{2.262891in}{0.650184in}%
\pgfsys@useobject{currentmarker}{}%
\end{pgfscope}%
\begin{pgfscope}%
\pgfsys@transformshift{2.263028in}{0.717962in}%
\pgfsys@useobject{currentmarker}{}%
\end{pgfscope}%
\begin{pgfscope}%
\pgfsys@transformshift{2.263165in}{0.706230in}%
\pgfsys@useobject{currentmarker}{}%
\end{pgfscope}%
\begin{pgfscope}%
\pgfsys@transformshift{2.263301in}{0.710286in}%
\pgfsys@useobject{currentmarker}{}%
\end{pgfscope}%
\begin{pgfscope}%
\pgfsys@transformshift{2.263438in}{0.694894in}%
\pgfsys@useobject{currentmarker}{}%
\end{pgfscope}%
\begin{pgfscope}%
\pgfsys@transformshift{2.263575in}{0.670663in}%
\pgfsys@useobject{currentmarker}{}%
\end{pgfscope}%
\begin{pgfscope}%
\pgfsys@transformshift{2.263711in}{0.671463in}%
\pgfsys@useobject{currentmarker}{}%
\end{pgfscope}%
\begin{pgfscope}%
\pgfsys@transformshift{2.263848in}{0.638951in}%
\pgfsys@useobject{currentmarker}{}%
\end{pgfscope}%
\begin{pgfscope}%
\pgfsys@transformshift{2.263984in}{0.608143in}%
\pgfsys@useobject{currentmarker}{}%
\end{pgfscope}%
\begin{pgfscope}%
\pgfsys@transformshift{2.264120in}{0.687217in}%
\pgfsys@useobject{currentmarker}{}%
\end{pgfscope}%
\begin{pgfscope}%
\pgfsys@transformshift{2.264256in}{0.659945in}%
\pgfsys@useobject{currentmarker}{}%
\end{pgfscope}%
\begin{pgfscope}%
\pgfsys@transformshift{2.264392in}{0.667699in}%
\pgfsys@useobject{currentmarker}{}%
\end{pgfscope}%
\begin{pgfscope}%
\pgfsys@transformshift{2.264529in}{0.736234in}%
\pgfsys@useobject{currentmarker}{}%
\end{pgfscope}%
\begin{pgfscope}%
\pgfsys@transformshift{2.264664in}{0.747768in}%
\pgfsys@useobject{currentmarker}{}%
\end{pgfscope}%
\begin{pgfscope}%
\pgfsys@transformshift{2.264800in}{0.737056in}%
\pgfsys@useobject{currentmarker}{}%
\end{pgfscope}%
\begin{pgfscope}%
\pgfsys@transformshift{2.264936in}{0.692515in}%
\pgfsys@useobject{currentmarker}{}%
\end{pgfscope}%
\begin{pgfscope}%
\pgfsys@transformshift{2.265072in}{0.729252in}%
\pgfsys@useobject{currentmarker}{}%
\end{pgfscope}%
\begin{pgfscope}%
\pgfsys@transformshift{2.265207in}{0.724098in}%
\pgfsys@useobject{currentmarker}{}%
\end{pgfscope}%
\begin{pgfscope}%
\pgfsys@transformshift{2.265343in}{0.627153in}%
\pgfsys@useobject{currentmarker}{}%
\end{pgfscope}%
\begin{pgfscope}%
\pgfsys@transformshift{2.265478in}{0.671592in}%
\pgfsys@useobject{currentmarker}{}%
\end{pgfscope}%
\begin{pgfscope}%
\pgfsys@transformshift{2.265614in}{0.720012in}%
\pgfsys@useobject{currentmarker}{}%
\end{pgfscope}%
\begin{pgfscope}%
\pgfsys@transformshift{2.265749in}{0.684455in}%
\pgfsys@useobject{currentmarker}{}%
\end{pgfscope}%
\begin{pgfscope}%
\pgfsys@transformshift{2.265884in}{0.662993in}%
\pgfsys@useobject{currentmarker}{}%
\end{pgfscope}%
\begin{pgfscope}%
\pgfsys@transformshift{2.266019in}{0.673080in}%
\pgfsys@useobject{currentmarker}{}%
\end{pgfscope}%
\begin{pgfscope}%
\pgfsys@transformshift{2.266154in}{0.691437in}%
\pgfsys@useobject{currentmarker}{}%
\end{pgfscope}%
\begin{pgfscope}%
\pgfsys@transformshift{2.266289in}{0.685604in}%
\pgfsys@useobject{currentmarker}{}%
\end{pgfscope}%
\begin{pgfscope}%
\pgfsys@transformshift{2.266424in}{0.691666in}%
\pgfsys@useobject{currentmarker}{}%
\end{pgfscope}%
\begin{pgfscope}%
\pgfsys@transformshift{2.266559in}{0.705084in}%
\pgfsys@useobject{currentmarker}{}%
\end{pgfscope}%
\begin{pgfscope}%
\pgfsys@transformshift{2.266694in}{0.732878in}%
\pgfsys@useobject{currentmarker}{}%
\end{pgfscope}%
\begin{pgfscope}%
\pgfsys@transformshift{2.266828in}{0.717866in}%
\pgfsys@useobject{currentmarker}{}%
\end{pgfscope}%
\begin{pgfscope}%
\pgfsys@transformshift{2.266963in}{0.719829in}%
\pgfsys@useobject{currentmarker}{}%
\end{pgfscope}%
\begin{pgfscope}%
\pgfsys@transformshift{2.267098in}{0.712061in}%
\pgfsys@useobject{currentmarker}{}%
\end{pgfscope}%
\begin{pgfscope}%
\pgfsys@transformshift{2.267232in}{0.719081in}%
\pgfsys@useobject{currentmarker}{}%
\end{pgfscope}%
\begin{pgfscope}%
\pgfsys@transformshift{2.267366in}{0.705006in}%
\pgfsys@useobject{currentmarker}{}%
\end{pgfscope}%
\begin{pgfscope}%
\pgfsys@transformshift{2.267501in}{0.660491in}%
\pgfsys@useobject{currentmarker}{}%
\end{pgfscope}%
\begin{pgfscope}%
\pgfsys@transformshift{2.267635in}{0.636766in}%
\pgfsys@useobject{currentmarker}{}%
\end{pgfscope}%
\begin{pgfscope}%
\pgfsys@transformshift{2.267769in}{0.690835in}%
\pgfsys@useobject{currentmarker}{}%
\end{pgfscope}%
\begin{pgfscope}%
\pgfsys@transformshift{2.267903in}{0.720019in}%
\pgfsys@useobject{currentmarker}{}%
\end{pgfscope}%
\begin{pgfscope}%
\pgfsys@transformshift{2.268037in}{0.623526in}%
\pgfsys@useobject{currentmarker}{}%
\end{pgfscope}%
\begin{pgfscope}%
\pgfsys@transformshift{2.268171in}{0.638314in}%
\pgfsys@useobject{currentmarker}{}%
\end{pgfscope}%
\begin{pgfscope}%
\pgfsys@transformshift{2.268304in}{0.713365in}%
\pgfsys@useobject{currentmarker}{}%
\end{pgfscope}%
\begin{pgfscope}%
\pgfsys@transformshift{2.268438in}{0.684976in}%
\pgfsys@useobject{currentmarker}{}%
\end{pgfscope}%
\begin{pgfscope}%
\pgfsys@transformshift{2.268572in}{0.710740in}%
\pgfsys@useobject{currentmarker}{}%
\end{pgfscope}%
\begin{pgfscope}%
\pgfsys@transformshift{2.268705in}{0.720564in}%
\pgfsys@useobject{currentmarker}{}%
\end{pgfscope}%
\begin{pgfscope}%
\pgfsys@transformshift{2.268839in}{0.699921in}%
\pgfsys@useobject{currentmarker}{}%
\end{pgfscope}%
\begin{pgfscope}%
\pgfsys@transformshift{2.268972in}{0.672164in}%
\pgfsys@useobject{currentmarker}{}%
\end{pgfscope}%
\begin{pgfscope}%
\pgfsys@transformshift{2.269105in}{0.646146in}%
\pgfsys@useobject{currentmarker}{}%
\end{pgfscope}%
\begin{pgfscope}%
\pgfsys@transformshift{2.269238in}{0.733471in}%
\pgfsys@useobject{currentmarker}{}%
\end{pgfscope}%
\begin{pgfscope}%
\pgfsys@transformshift{2.269371in}{0.752755in}%
\pgfsys@useobject{currentmarker}{}%
\end{pgfscope}%
\begin{pgfscope}%
\pgfsys@transformshift{2.269505in}{0.698591in}%
\pgfsys@useobject{currentmarker}{}%
\end{pgfscope}%
\begin{pgfscope}%
\pgfsys@transformshift{2.269638in}{0.681363in}%
\pgfsys@useobject{currentmarker}{}%
\end{pgfscope}%
\begin{pgfscope}%
\pgfsys@transformshift{2.269770in}{0.657057in}%
\pgfsys@useobject{currentmarker}{}%
\end{pgfscope}%
\begin{pgfscope}%
\pgfsys@transformshift{2.269903in}{0.691373in}%
\pgfsys@useobject{currentmarker}{}%
\end{pgfscope}%
\begin{pgfscope}%
\pgfsys@transformshift{2.270036in}{0.719245in}%
\pgfsys@useobject{currentmarker}{}%
\end{pgfscope}%
\begin{pgfscope}%
\pgfsys@transformshift{2.270169in}{0.634740in}%
\pgfsys@useobject{currentmarker}{}%
\end{pgfscope}%
\begin{pgfscope}%
\pgfsys@transformshift{2.270301in}{0.693355in}%
\pgfsys@useobject{currentmarker}{}%
\end{pgfscope}%
\begin{pgfscope}%
\pgfsys@transformshift{2.270434in}{0.700867in}%
\pgfsys@useobject{currentmarker}{}%
\end{pgfscope}%
\begin{pgfscope}%
\pgfsys@transformshift{2.270566in}{0.641875in}%
\pgfsys@useobject{currentmarker}{}%
\end{pgfscope}%
\begin{pgfscope}%
\pgfsys@transformshift{2.270698in}{0.678905in}%
\pgfsys@useobject{currentmarker}{}%
\end{pgfscope}%
\begin{pgfscope}%
\pgfsys@transformshift{2.270831in}{0.696778in}%
\pgfsys@useobject{currentmarker}{}%
\end{pgfscope}%
\begin{pgfscope}%
\pgfsys@transformshift{2.270963in}{0.726094in}%
\pgfsys@useobject{currentmarker}{}%
\end{pgfscope}%
\begin{pgfscope}%
\pgfsys@transformshift{2.271095in}{0.726314in}%
\pgfsys@useobject{currentmarker}{}%
\end{pgfscope}%
\begin{pgfscope}%
\pgfsys@transformshift{2.271227in}{0.655297in}%
\pgfsys@useobject{currentmarker}{}%
\end{pgfscope}%
\begin{pgfscope}%
\pgfsys@transformshift{2.271359in}{0.680052in}%
\pgfsys@useobject{currentmarker}{}%
\end{pgfscope}%
\begin{pgfscope}%
\pgfsys@transformshift{2.271491in}{0.723121in}%
\pgfsys@useobject{currentmarker}{}%
\end{pgfscope}%
\begin{pgfscope}%
\pgfsys@transformshift{2.271623in}{0.702500in}%
\pgfsys@useobject{currentmarker}{}%
\end{pgfscope}%
\begin{pgfscope}%
\pgfsys@transformshift{2.271754in}{0.703693in}%
\pgfsys@useobject{currentmarker}{}%
\end{pgfscope}%
\begin{pgfscope}%
\pgfsys@transformshift{2.271886in}{0.710979in}%
\pgfsys@useobject{currentmarker}{}%
\end{pgfscope}%
\begin{pgfscope}%
\pgfsys@transformshift{2.272018in}{0.705691in}%
\pgfsys@useobject{currentmarker}{}%
\end{pgfscope}%
\begin{pgfscope}%
\pgfsys@transformshift{2.272149in}{0.715745in}%
\pgfsys@useobject{currentmarker}{}%
\end{pgfscope}%
\begin{pgfscope}%
\pgfsys@transformshift{2.272281in}{0.705681in}%
\pgfsys@useobject{currentmarker}{}%
\end{pgfscope}%
\begin{pgfscope}%
\pgfsys@transformshift{2.272412in}{0.648929in}%
\pgfsys@useobject{currentmarker}{}%
\end{pgfscope}%
\begin{pgfscope}%
\pgfsys@transformshift{2.272543in}{0.637630in}%
\pgfsys@useobject{currentmarker}{}%
\end{pgfscope}%
\begin{pgfscope}%
\pgfsys@transformshift{2.272674in}{0.656656in}%
\pgfsys@useobject{currentmarker}{}%
\end{pgfscope}%
\begin{pgfscope}%
\pgfsys@transformshift{2.272805in}{0.680087in}%
\pgfsys@useobject{currentmarker}{}%
\end{pgfscope}%
\begin{pgfscope}%
\pgfsys@transformshift{2.272936in}{0.651217in}%
\pgfsys@useobject{currentmarker}{}%
\end{pgfscope}%
\begin{pgfscope}%
\pgfsys@transformshift{2.273067in}{0.651466in}%
\pgfsys@useobject{currentmarker}{}%
\end{pgfscope}%
\begin{pgfscope}%
\pgfsys@transformshift{2.273198in}{0.677017in}%
\pgfsys@useobject{currentmarker}{}%
\end{pgfscope}%
\begin{pgfscope}%
\pgfsys@transformshift{2.273329in}{0.711601in}%
\pgfsys@useobject{currentmarker}{}%
\end{pgfscope}%
\begin{pgfscope}%
\pgfsys@transformshift{2.273460in}{0.694757in}%
\pgfsys@useobject{currentmarker}{}%
\end{pgfscope}%
\begin{pgfscope}%
\pgfsys@transformshift{2.273590in}{0.657173in}%
\pgfsys@useobject{currentmarker}{}%
\end{pgfscope}%
\begin{pgfscope}%
\pgfsys@transformshift{2.273721in}{0.699199in}%
\pgfsys@useobject{currentmarker}{}%
\end{pgfscope}%
\begin{pgfscope}%
\pgfsys@transformshift{2.273852in}{0.675985in}%
\pgfsys@useobject{currentmarker}{}%
\end{pgfscope}%
\begin{pgfscope}%
\pgfsys@transformshift{2.273982in}{0.701432in}%
\pgfsys@useobject{currentmarker}{}%
\end{pgfscope}%
\begin{pgfscope}%
\pgfsys@transformshift{2.274112in}{0.699791in}%
\pgfsys@useobject{currentmarker}{}%
\end{pgfscope}%
\begin{pgfscope}%
\pgfsys@transformshift{2.274243in}{0.690907in}%
\pgfsys@useobject{currentmarker}{}%
\end{pgfscope}%
\begin{pgfscope}%
\pgfsys@transformshift{2.274373in}{0.655310in}%
\pgfsys@useobject{currentmarker}{}%
\end{pgfscope}%
\begin{pgfscope}%
\pgfsys@transformshift{2.274503in}{0.711355in}%
\pgfsys@useobject{currentmarker}{}%
\end{pgfscope}%
\begin{pgfscope}%
\pgfsys@transformshift{2.274633in}{0.726805in}%
\pgfsys@useobject{currentmarker}{}%
\end{pgfscope}%
\begin{pgfscope}%
\pgfsys@transformshift{2.274763in}{0.669294in}%
\pgfsys@useobject{currentmarker}{}%
\end{pgfscope}%
\begin{pgfscope}%
\pgfsys@transformshift{2.274893in}{0.674684in}%
\pgfsys@useobject{currentmarker}{}%
\end{pgfscope}%
\begin{pgfscope}%
\pgfsys@transformshift{2.275023in}{0.676552in}%
\pgfsys@useobject{currentmarker}{}%
\end{pgfscope}%
\begin{pgfscope}%
\pgfsys@transformshift{2.275152in}{0.645829in}%
\pgfsys@useobject{currentmarker}{}%
\end{pgfscope}%
\begin{pgfscope}%
\pgfsys@transformshift{2.275282in}{0.656811in}%
\pgfsys@useobject{currentmarker}{}%
\end{pgfscope}%
\begin{pgfscope}%
\pgfsys@transformshift{2.275412in}{0.688773in}%
\pgfsys@useobject{currentmarker}{}%
\end{pgfscope}%
\begin{pgfscope}%
\pgfsys@transformshift{2.275541in}{0.756668in}%
\pgfsys@useobject{currentmarker}{}%
\end{pgfscope}%
\begin{pgfscope}%
\pgfsys@transformshift{2.275671in}{0.722126in}%
\pgfsys@useobject{currentmarker}{}%
\end{pgfscope}%
\begin{pgfscope}%
\pgfsys@transformshift{2.275800in}{0.634528in}%
\pgfsys@useobject{currentmarker}{}%
\end{pgfscope}%
\begin{pgfscope}%
\pgfsys@transformshift{2.275929in}{0.638446in}%
\pgfsys@useobject{currentmarker}{}%
\end{pgfscope}%
\begin{pgfscope}%
\pgfsys@transformshift{2.276058in}{0.671414in}%
\pgfsys@useobject{currentmarker}{}%
\end{pgfscope}%
\begin{pgfscope}%
\pgfsys@transformshift{2.276188in}{0.692651in}%
\pgfsys@useobject{currentmarker}{}%
\end{pgfscope}%
\begin{pgfscope}%
\pgfsys@transformshift{2.276317in}{0.680170in}%
\pgfsys@useobject{currentmarker}{}%
\end{pgfscope}%
\begin{pgfscope}%
\pgfsys@transformshift{2.276446in}{0.610918in}%
\pgfsys@useobject{currentmarker}{}%
\end{pgfscope}%
\begin{pgfscope}%
\pgfsys@transformshift{2.276575in}{0.716587in}%
\pgfsys@useobject{currentmarker}{}%
\end{pgfscope}%
\begin{pgfscope}%
\pgfsys@transformshift{2.276703in}{0.774854in}%
\pgfsys@useobject{currentmarker}{}%
\end{pgfscope}%
\begin{pgfscope}%
\pgfsys@transformshift{2.276832in}{0.778969in}%
\pgfsys@useobject{currentmarker}{}%
\end{pgfscope}%
\begin{pgfscope}%
\pgfsys@transformshift{2.276961in}{0.703245in}%
\pgfsys@useobject{currentmarker}{}%
\end{pgfscope}%
\begin{pgfscope}%
\pgfsys@transformshift{2.277090in}{0.629540in}%
\pgfsys@useobject{currentmarker}{}%
\end{pgfscope}%
\begin{pgfscope}%
\pgfsys@transformshift{2.277218in}{0.663104in}%
\pgfsys@useobject{currentmarker}{}%
\end{pgfscope}%
\begin{pgfscope}%
\pgfsys@transformshift{2.277347in}{0.673977in}%
\pgfsys@useobject{currentmarker}{}%
\end{pgfscope}%
\begin{pgfscope}%
\pgfsys@transformshift{2.277475in}{0.672853in}%
\pgfsys@useobject{currentmarker}{}%
\end{pgfscope}%
\begin{pgfscope}%
\pgfsys@transformshift{2.277603in}{0.687483in}%
\pgfsys@useobject{currentmarker}{}%
\end{pgfscope}%
\begin{pgfscope}%
\pgfsys@transformshift{2.277732in}{0.702890in}%
\pgfsys@useobject{currentmarker}{}%
\end{pgfscope}%
\begin{pgfscope}%
\pgfsys@transformshift{2.277860in}{0.721473in}%
\pgfsys@useobject{currentmarker}{}%
\end{pgfscope}%
\begin{pgfscope}%
\pgfsys@transformshift{2.277988in}{0.658342in}%
\pgfsys@useobject{currentmarker}{}%
\end{pgfscope}%
\begin{pgfscope}%
\pgfsys@transformshift{2.278116in}{0.668705in}%
\pgfsys@useobject{currentmarker}{}%
\end{pgfscope}%
\begin{pgfscope}%
\pgfsys@transformshift{2.278244in}{0.693601in}%
\pgfsys@useobject{currentmarker}{}%
\end{pgfscope}%
\begin{pgfscope}%
\pgfsys@transformshift{2.278372in}{0.709506in}%
\pgfsys@useobject{currentmarker}{}%
\end{pgfscope}%
\begin{pgfscope}%
\pgfsys@transformshift{2.278500in}{0.690800in}%
\pgfsys@useobject{currentmarker}{}%
\end{pgfscope}%
\begin{pgfscope}%
\pgfsys@transformshift{2.278627in}{0.671381in}%
\pgfsys@useobject{currentmarker}{}%
\end{pgfscope}%
\begin{pgfscope}%
\pgfsys@transformshift{2.278755in}{0.693400in}%
\pgfsys@useobject{currentmarker}{}%
\end{pgfscope}%
\begin{pgfscope}%
\pgfsys@transformshift{2.278883in}{0.708275in}%
\pgfsys@useobject{currentmarker}{}%
\end{pgfscope}%
\begin{pgfscope}%
\pgfsys@transformshift{2.279010in}{0.711693in}%
\pgfsys@useobject{currentmarker}{}%
\end{pgfscope}%
\begin{pgfscope}%
\pgfsys@transformshift{2.279138in}{0.664223in}%
\pgfsys@useobject{currentmarker}{}%
\end{pgfscope}%
\begin{pgfscope}%
\pgfsys@transformshift{2.279265in}{0.648991in}%
\pgfsys@useobject{currentmarker}{}%
\end{pgfscope}%
\begin{pgfscope}%
\pgfsys@transformshift{2.279392in}{0.683228in}%
\pgfsys@useobject{currentmarker}{}%
\end{pgfscope}%
\begin{pgfscope}%
\pgfsys@transformshift{2.279520in}{0.721054in}%
\pgfsys@useobject{currentmarker}{}%
\end{pgfscope}%
\begin{pgfscope}%
\pgfsys@transformshift{2.279647in}{0.678127in}%
\pgfsys@useobject{currentmarker}{}%
\end{pgfscope}%
\begin{pgfscope}%
\pgfsys@transformshift{2.279774in}{0.635688in}%
\pgfsys@useobject{currentmarker}{}%
\end{pgfscope}%
\begin{pgfscope}%
\pgfsys@transformshift{2.279901in}{0.630343in}%
\pgfsys@useobject{currentmarker}{}%
\end{pgfscope}%
\begin{pgfscope}%
\pgfsys@transformshift{2.280028in}{0.677635in}%
\pgfsys@useobject{currentmarker}{}%
\end{pgfscope}%
\begin{pgfscope}%
\pgfsys@transformshift{2.280155in}{0.665910in}%
\pgfsys@useobject{currentmarker}{}%
\end{pgfscope}%
\begin{pgfscope}%
\pgfsys@transformshift{2.280282in}{0.693542in}%
\pgfsys@useobject{currentmarker}{}%
\end{pgfscope}%
\begin{pgfscope}%
\pgfsys@transformshift{2.280408in}{0.698578in}%
\pgfsys@useobject{currentmarker}{}%
\end{pgfscope}%
\begin{pgfscope}%
\pgfsys@transformshift{2.280535in}{0.711778in}%
\pgfsys@useobject{currentmarker}{}%
\end{pgfscope}%
\begin{pgfscope}%
\pgfsys@transformshift{2.280662in}{0.702844in}%
\pgfsys@useobject{currentmarker}{}%
\end{pgfscope}%
\begin{pgfscope}%
\pgfsys@transformshift{2.280788in}{0.703045in}%
\pgfsys@useobject{currentmarker}{}%
\end{pgfscope}%
\begin{pgfscope}%
\pgfsys@transformshift{2.280915in}{0.731331in}%
\pgfsys@useobject{currentmarker}{}%
\end{pgfscope}%
\begin{pgfscope}%
\pgfsys@transformshift{2.281041in}{0.707720in}%
\pgfsys@useobject{currentmarker}{}%
\end{pgfscope}%
\begin{pgfscope}%
\pgfsys@transformshift{2.281167in}{0.696575in}%
\pgfsys@useobject{currentmarker}{}%
\end{pgfscope}%
\begin{pgfscope}%
\pgfsys@transformshift{2.281294in}{0.703734in}%
\pgfsys@useobject{currentmarker}{}%
\end{pgfscope}%
\begin{pgfscope}%
\pgfsys@transformshift{2.281420in}{0.672502in}%
\pgfsys@useobject{currentmarker}{}%
\end{pgfscope}%
\begin{pgfscope}%
\pgfsys@transformshift{2.281546in}{0.646724in}%
\pgfsys@useobject{currentmarker}{}%
\end{pgfscope}%
\begin{pgfscope}%
\pgfsys@transformshift{2.281672in}{0.719133in}%
\pgfsys@useobject{currentmarker}{}%
\end{pgfscope}%
\begin{pgfscope}%
\pgfsys@transformshift{2.281798in}{0.703440in}%
\pgfsys@useobject{currentmarker}{}%
\end{pgfscope}%
\begin{pgfscope}%
\pgfsys@transformshift{2.281924in}{0.667709in}%
\pgfsys@useobject{currentmarker}{}%
\end{pgfscope}%
\begin{pgfscope}%
\pgfsys@transformshift{2.282050in}{0.660650in}%
\pgfsys@useobject{currentmarker}{}%
\end{pgfscope}%
\begin{pgfscope}%
\pgfsys@transformshift{2.282175in}{0.722714in}%
\pgfsys@useobject{currentmarker}{}%
\end{pgfscope}%
\begin{pgfscope}%
\pgfsys@transformshift{2.282301in}{0.721469in}%
\pgfsys@useobject{currentmarker}{}%
\end{pgfscope}%
\begin{pgfscope}%
\pgfsys@transformshift{2.282427in}{0.723666in}%
\pgfsys@useobject{currentmarker}{}%
\end{pgfscope}%
\begin{pgfscope}%
\pgfsys@transformshift{2.282552in}{0.695727in}%
\pgfsys@useobject{currentmarker}{}%
\end{pgfscope}%
\begin{pgfscope}%
\pgfsys@transformshift{2.282678in}{0.692894in}%
\pgfsys@useobject{currentmarker}{}%
\end{pgfscope}%
\begin{pgfscope}%
\pgfsys@transformshift{2.282803in}{0.662008in}%
\pgfsys@useobject{currentmarker}{}%
\end{pgfscope}%
\begin{pgfscope}%
\pgfsys@transformshift{2.282928in}{0.700781in}%
\pgfsys@useobject{currentmarker}{}%
\end{pgfscope}%
\begin{pgfscope}%
\pgfsys@transformshift{2.283054in}{0.704049in}%
\pgfsys@useobject{currentmarker}{}%
\end{pgfscope}%
\begin{pgfscope}%
\pgfsys@transformshift{2.283179in}{0.678944in}%
\pgfsys@useobject{currentmarker}{}%
\end{pgfscope}%
\begin{pgfscope}%
\pgfsys@transformshift{2.283304in}{0.691229in}%
\pgfsys@useobject{currentmarker}{}%
\end{pgfscope}%
\begin{pgfscope}%
\pgfsys@transformshift{2.283429in}{0.731591in}%
\pgfsys@useobject{currentmarker}{}%
\end{pgfscope}%
\begin{pgfscope}%
\pgfsys@transformshift{2.283554in}{0.701590in}%
\pgfsys@useobject{currentmarker}{}%
\end{pgfscope}%
\begin{pgfscope}%
\pgfsys@transformshift{2.283679in}{0.686443in}%
\pgfsys@useobject{currentmarker}{}%
\end{pgfscope}%
\begin{pgfscope}%
\pgfsys@transformshift{2.283804in}{0.727133in}%
\pgfsys@useobject{currentmarker}{}%
\end{pgfscope}%
\begin{pgfscope}%
\pgfsys@transformshift{2.283928in}{0.693761in}%
\pgfsys@useobject{currentmarker}{}%
\end{pgfscope}%
\begin{pgfscope}%
\pgfsys@transformshift{2.284053in}{0.702641in}%
\pgfsys@useobject{currentmarker}{}%
\end{pgfscope}%
\begin{pgfscope}%
\pgfsys@transformshift{2.284178in}{0.703194in}%
\pgfsys@useobject{currentmarker}{}%
\end{pgfscope}%
\begin{pgfscope}%
\pgfsys@transformshift{2.284302in}{0.714416in}%
\pgfsys@useobject{currentmarker}{}%
\end{pgfscope}%
\begin{pgfscope}%
\pgfsys@transformshift{2.284427in}{0.666375in}%
\pgfsys@useobject{currentmarker}{}%
\end{pgfscope}%
\begin{pgfscope}%
\pgfsys@transformshift{2.284551in}{0.686282in}%
\pgfsys@useobject{currentmarker}{}%
\end{pgfscope}%
\begin{pgfscope}%
\pgfsys@transformshift{2.284676in}{0.654418in}%
\pgfsys@useobject{currentmarker}{}%
\end{pgfscope}%
\begin{pgfscope}%
\pgfsys@transformshift{2.284800in}{0.621316in}%
\pgfsys@useobject{currentmarker}{}%
\end{pgfscope}%
\begin{pgfscope}%
\pgfsys@transformshift{2.284924in}{0.633103in}%
\pgfsys@useobject{currentmarker}{}%
\end{pgfscope}%
\begin{pgfscope}%
\pgfsys@transformshift{2.285048in}{0.639488in}%
\pgfsys@useobject{currentmarker}{}%
\end{pgfscope}%
\begin{pgfscope}%
\pgfsys@transformshift{2.285172in}{0.700319in}%
\pgfsys@useobject{currentmarker}{}%
\end{pgfscope}%
\begin{pgfscope}%
\pgfsys@transformshift{2.285296in}{0.726304in}%
\pgfsys@useobject{currentmarker}{}%
\end{pgfscope}%
\begin{pgfscope}%
\pgfsys@transformshift{2.285420in}{0.710266in}%
\pgfsys@useobject{currentmarker}{}%
\end{pgfscope}%
\begin{pgfscope}%
\pgfsys@transformshift{2.285544in}{0.689100in}%
\pgfsys@useobject{currentmarker}{}%
\end{pgfscope}%
\begin{pgfscope}%
\pgfsys@transformshift{2.285668in}{0.714135in}%
\pgfsys@useobject{currentmarker}{}%
\end{pgfscope}%
\begin{pgfscope}%
\pgfsys@transformshift{2.285792in}{0.709999in}%
\pgfsys@useobject{currentmarker}{}%
\end{pgfscope}%
\begin{pgfscope}%
\pgfsys@transformshift{2.285915in}{0.711591in}%
\pgfsys@useobject{currentmarker}{}%
\end{pgfscope}%
\begin{pgfscope}%
\pgfsys@transformshift{2.286039in}{0.689829in}%
\pgfsys@useobject{currentmarker}{}%
\end{pgfscope}%
\begin{pgfscope}%
\pgfsys@transformshift{2.286163in}{0.719483in}%
\pgfsys@useobject{currentmarker}{}%
\end{pgfscope}%
\begin{pgfscope}%
\pgfsys@transformshift{2.286286in}{0.736449in}%
\pgfsys@useobject{currentmarker}{}%
\end{pgfscope}%
\begin{pgfscope}%
\pgfsys@transformshift{2.286409in}{0.665133in}%
\pgfsys@useobject{currentmarker}{}%
\end{pgfscope}%
\begin{pgfscope}%
\pgfsys@transformshift{2.286533in}{0.687265in}%
\pgfsys@useobject{currentmarker}{}%
\end{pgfscope}%
\begin{pgfscope}%
\pgfsys@transformshift{2.286656in}{0.704019in}%
\pgfsys@useobject{currentmarker}{}%
\end{pgfscope}%
\begin{pgfscope}%
\pgfsys@transformshift{2.286779in}{0.710308in}%
\pgfsys@useobject{currentmarker}{}%
\end{pgfscope}%
\begin{pgfscope}%
\pgfsys@transformshift{2.286902in}{0.728616in}%
\pgfsys@useobject{currentmarker}{}%
\end{pgfscope}%
\begin{pgfscope}%
\pgfsys@transformshift{2.287025in}{0.698451in}%
\pgfsys@useobject{currentmarker}{}%
\end{pgfscope}%
\begin{pgfscope}%
\pgfsys@transformshift{2.287148in}{0.669708in}%
\pgfsys@useobject{currentmarker}{}%
\end{pgfscope}%
\begin{pgfscope}%
\pgfsys@transformshift{2.287271in}{0.647351in}%
\pgfsys@useobject{currentmarker}{}%
\end{pgfscope}%
\begin{pgfscope}%
\pgfsys@transformshift{2.287394in}{0.678809in}%
\pgfsys@useobject{currentmarker}{}%
\end{pgfscope}%
\begin{pgfscope}%
\pgfsys@transformshift{2.287517in}{0.727124in}%
\pgfsys@useobject{currentmarker}{}%
\end{pgfscope}%
\begin{pgfscope}%
\pgfsys@transformshift{2.287640in}{0.694741in}%
\pgfsys@useobject{currentmarker}{}%
\end{pgfscope}%
\begin{pgfscope}%
\pgfsys@transformshift{2.287762in}{0.692225in}%
\pgfsys@useobject{currentmarker}{}%
\end{pgfscope}%
\begin{pgfscope}%
\pgfsys@transformshift{2.287885in}{0.711866in}%
\pgfsys@useobject{currentmarker}{}%
\end{pgfscope}%
\begin{pgfscope}%
\pgfsys@transformshift{2.288007in}{0.698509in}%
\pgfsys@useobject{currentmarker}{}%
\end{pgfscope}%
\begin{pgfscope}%
\pgfsys@transformshift{2.288130in}{0.691019in}%
\pgfsys@useobject{currentmarker}{}%
\end{pgfscope}%
\begin{pgfscope}%
\pgfsys@transformshift{2.288252in}{0.693699in}%
\pgfsys@useobject{currentmarker}{}%
\end{pgfscope}%
\begin{pgfscope}%
\pgfsys@transformshift{2.288375in}{0.661100in}%
\pgfsys@useobject{currentmarker}{}%
\end{pgfscope}%
\begin{pgfscope}%
\pgfsys@transformshift{2.288497in}{0.710186in}%
\pgfsys@useobject{currentmarker}{}%
\end{pgfscope}%
\begin{pgfscope}%
\pgfsys@transformshift{2.288619in}{0.727255in}%
\pgfsys@useobject{currentmarker}{}%
\end{pgfscope}%
\begin{pgfscope}%
\pgfsys@transformshift{2.288741in}{0.695623in}%
\pgfsys@useobject{currentmarker}{}%
\end{pgfscope}%
\begin{pgfscope}%
\pgfsys@transformshift{2.288863in}{0.700188in}%
\pgfsys@useobject{currentmarker}{}%
\end{pgfscope}%
\begin{pgfscope}%
\pgfsys@transformshift{2.288985in}{0.686409in}%
\pgfsys@useobject{currentmarker}{}%
\end{pgfscope}%
\begin{pgfscope}%
\pgfsys@transformshift{2.289107in}{0.668974in}%
\pgfsys@useobject{currentmarker}{}%
\end{pgfscope}%
\begin{pgfscope}%
\pgfsys@transformshift{2.289229in}{0.691811in}%
\pgfsys@useobject{currentmarker}{}%
\end{pgfscope}%
\begin{pgfscope}%
\pgfsys@transformshift{2.289351in}{0.716433in}%
\pgfsys@useobject{currentmarker}{}%
\end{pgfscope}%
\begin{pgfscope}%
\pgfsys@transformshift{2.289473in}{0.706222in}%
\pgfsys@useobject{currentmarker}{}%
\end{pgfscope}%
\begin{pgfscope}%
\pgfsys@transformshift{2.289594in}{0.653794in}%
\pgfsys@useobject{currentmarker}{}%
\end{pgfscope}%
\begin{pgfscope}%
\pgfsys@transformshift{2.289716in}{0.681182in}%
\pgfsys@useobject{currentmarker}{}%
\end{pgfscope}%
\begin{pgfscope}%
\pgfsys@transformshift{2.289837in}{0.755559in}%
\pgfsys@useobject{currentmarker}{}%
\end{pgfscope}%
\begin{pgfscope}%
\pgfsys@transformshift{2.289959in}{0.771019in}%
\pgfsys@useobject{currentmarker}{}%
\end{pgfscope}%
\begin{pgfscope}%
\pgfsys@transformshift{2.290080in}{0.701150in}%
\pgfsys@useobject{currentmarker}{}%
\end{pgfscope}%
\begin{pgfscope}%
\pgfsys@transformshift{2.290202in}{0.714864in}%
\pgfsys@useobject{currentmarker}{}%
\end{pgfscope}%
\begin{pgfscope}%
\pgfsys@transformshift{2.290323in}{0.723854in}%
\pgfsys@useobject{currentmarker}{}%
\end{pgfscope}%
\begin{pgfscope}%
\pgfsys@transformshift{2.290444in}{0.663172in}%
\pgfsys@useobject{currentmarker}{}%
\end{pgfscope}%
\begin{pgfscope}%
\pgfsys@transformshift{2.290565in}{0.723456in}%
\pgfsys@useobject{currentmarker}{}%
\end{pgfscope}%
\begin{pgfscope}%
\pgfsys@transformshift{2.290686in}{0.713712in}%
\pgfsys@useobject{currentmarker}{}%
\end{pgfscope}%
\begin{pgfscope}%
\pgfsys@transformshift{2.290807in}{0.699458in}%
\pgfsys@useobject{currentmarker}{}%
\end{pgfscope}%
\begin{pgfscope}%
\pgfsys@transformshift{2.290928in}{0.701659in}%
\pgfsys@useobject{currentmarker}{}%
\end{pgfscope}%
\begin{pgfscope}%
\pgfsys@transformshift{2.291049in}{0.711828in}%
\pgfsys@useobject{currentmarker}{}%
\end{pgfscope}%
\begin{pgfscope}%
\pgfsys@transformshift{2.291170in}{0.694029in}%
\pgfsys@useobject{currentmarker}{}%
\end{pgfscope}%
\begin{pgfscope}%
\pgfsys@transformshift{2.291291in}{0.731750in}%
\pgfsys@useobject{currentmarker}{}%
\end{pgfscope}%
\begin{pgfscope}%
\pgfsys@transformshift{2.291411in}{0.688668in}%
\pgfsys@useobject{currentmarker}{}%
\end{pgfscope}%
\begin{pgfscope}%
\pgfsys@transformshift{2.291532in}{0.696094in}%
\pgfsys@useobject{currentmarker}{}%
\end{pgfscope}%
\begin{pgfscope}%
\pgfsys@transformshift{2.291653in}{0.718694in}%
\pgfsys@useobject{currentmarker}{}%
\end{pgfscope}%
\begin{pgfscope}%
\pgfsys@transformshift{2.291773in}{0.690980in}%
\pgfsys@useobject{currentmarker}{}%
\end{pgfscope}%
\begin{pgfscope}%
\pgfsys@transformshift{2.291893in}{0.677523in}%
\pgfsys@useobject{currentmarker}{}%
\end{pgfscope}%
\begin{pgfscope}%
\pgfsys@transformshift{2.292014in}{0.674714in}%
\pgfsys@useobject{currentmarker}{}%
\end{pgfscope}%
\begin{pgfscope}%
\pgfsys@transformshift{2.292134in}{0.695218in}%
\pgfsys@useobject{currentmarker}{}%
\end{pgfscope}%
\begin{pgfscope}%
\pgfsys@transformshift{2.292254in}{0.680646in}%
\pgfsys@useobject{currentmarker}{}%
\end{pgfscope}%
\begin{pgfscope}%
\pgfsys@transformshift{2.292374in}{0.670211in}%
\pgfsys@useobject{currentmarker}{}%
\end{pgfscope}%
\begin{pgfscope}%
\pgfsys@transformshift{2.292495in}{0.664592in}%
\pgfsys@useobject{currentmarker}{}%
\end{pgfscope}%
\begin{pgfscope}%
\pgfsys@transformshift{2.292615in}{0.703667in}%
\pgfsys@useobject{currentmarker}{}%
\end{pgfscope}%
\begin{pgfscope}%
\pgfsys@transformshift{2.292735in}{0.702049in}%
\pgfsys@useobject{currentmarker}{}%
\end{pgfscope}%
\begin{pgfscope}%
\pgfsys@transformshift{2.292855in}{0.687211in}%
\pgfsys@useobject{currentmarker}{}%
\end{pgfscope}%
\begin{pgfscope}%
\pgfsys@transformshift{2.292974in}{0.655472in}%
\pgfsys@useobject{currentmarker}{}%
\end{pgfscope}%
\begin{pgfscope}%
\pgfsys@transformshift{2.293094in}{0.647486in}%
\pgfsys@useobject{currentmarker}{}%
\end{pgfscope}%
\begin{pgfscope}%
\pgfsys@transformshift{2.293214in}{0.702027in}%
\pgfsys@useobject{currentmarker}{}%
\end{pgfscope}%
\begin{pgfscope}%
\pgfsys@transformshift{2.293334in}{0.702857in}%
\pgfsys@useobject{currentmarker}{}%
\end{pgfscope}%
\begin{pgfscope}%
\pgfsys@transformshift{2.293453in}{0.668111in}%
\pgfsys@useobject{currentmarker}{}%
\end{pgfscope}%
\begin{pgfscope}%
\pgfsys@transformshift{2.293573in}{0.714585in}%
\pgfsys@useobject{currentmarker}{}%
\end{pgfscope}%
\begin{pgfscope}%
\pgfsys@transformshift{2.293692in}{0.698123in}%
\pgfsys@useobject{currentmarker}{}%
\end{pgfscope}%
\begin{pgfscope}%
\pgfsys@transformshift{2.293812in}{0.668135in}%
\pgfsys@useobject{currentmarker}{}%
\end{pgfscope}%
\begin{pgfscope}%
\pgfsys@transformshift{2.293931in}{0.693478in}%
\pgfsys@useobject{currentmarker}{}%
\end{pgfscope}%
\begin{pgfscope}%
\pgfsys@transformshift{2.294050in}{0.702663in}%
\pgfsys@useobject{currentmarker}{}%
\end{pgfscope}%
\begin{pgfscope}%
\pgfsys@transformshift{2.294169in}{0.712650in}%
\pgfsys@useobject{currentmarker}{}%
\end{pgfscope}%
\begin{pgfscope}%
\pgfsys@transformshift{2.294288in}{0.730968in}%
\pgfsys@useobject{currentmarker}{}%
\end{pgfscope}%
\begin{pgfscope}%
\pgfsys@transformshift{2.294408in}{0.727104in}%
\pgfsys@useobject{currentmarker}{}%
\end{pgfscope}%
\begin{pgfscope}%
\pgfsys@transformshift{2.294527in}{0.692845in}%
\pgfsys@useobject{currentmarker}{}%
\end{pgfscope}%
\begin{pgfscope}%
\pgfsys@transformshift{2.294646in}{0.666759in}%
\pgfsys@useobject{currentmarker}{}%
\end{pgfscope}%
\begin{pgfscope}%
\pgfsys@transformshift{2.294764in}{0.682048in}%
\pgfsys@useobject{currentmarker}{}%
\end{pgfscope}%
\begin{pgfscope}%
\pgfsys@transformshift{2.294883in}{0.668633in}%
\pgfsys@useobject{currentmarker}{}%
\end{pgfscope}%
\begin{pgfscope}%
\pgfsys@transformshift{2.295002in}{0.659203in}%
\pgfsys@useobject{currentmarker}{}%
\end{pgfscope}%
\begin{pgfscope}%
\pgfsys@transformshift{2.295121in}{0.647868in}%
\pgfsys@useobject{currentmarker}{}%
\end{pgfscope}%
\begin{pgfscope}%
\pgfsys@transformshift{2.295239in}{0.701177in}%
\pgfsys@useobject{currentmarker}{}%
\end{pgfscope}%
\begin{pgfscope}%
\pgfsys@transformshift{2.295358in}{0.730335in}%
\pgfsys@useobject{currentmarker}{}%
\end{pgfscope}%
\begin{pgfscope}%
\pgfsys@transformshift{2.295476in}{0.720673in}%
\pgfsys@useobject{currentmarker}{}%
\end{pgfscope}%
\begin{pgfscope}%
\pgfsys@transformshift{2.295595in}{0.676803in}%
\pgfsys@useobject{currentmarker}{}%
\end{pgfscope}%
\begin{pgfscope}%
\pgfsys@transformshift{2.295713in}{0.682513in}%
\pgfsys@useobject{currentmarker}{}%
\end{pgfscope}%
\begin{pgfscope}%
\pgfsys@transformshift{2.295832in}{0.681956in}%
\pgfsys@useobject{currentmarker}{}%
\end{pgfscope}%
\begin{pgfscope}%
\pgfsys@transformshift{2.295950in}{0.700567in}%
\pgfsys@useobject{currentmarker}{}%
\end{pgfscope}%
\begin{pgfscope}%
\pgfsys@transformshift{2.296068in}{0.676486in}%
\pgfsys@useobject{currentmarker}{}%
\end{pgfscope}%
\begin{pgfscope}%
\pgfsys@transformshift{2.296186in}{0.673388in}%
\pgfsys@useobject{currentmarker}{}%
\end{pgfscope}%
\begin{pgfscope}%
\pgfsys@transformshift{2.296304in}{0.662260in}%
\pgfsys@useobject{currentmarker}{}%
\end{pgfscope}%
\begin{pgfscope}%
\pgfsys@transformshift{2.296422in}{0.624237in}%
\pgfsys@useobject{currentmarker}{}%
\end{pgfscope}%
\begin{pgfscope}%
\pgfsys@transformshift{2.296540in}{0.651792in}%
\pgfsys@useobject{currentmarker}{}%
\end{pgfscope}%
\begin{pgfscope}%
\pgfsys@transformshift{2.296658in}{0.712339in}%
\pgfsys@useobject{currentmarker}{}%
\end{pgfscope}%
\begin{pgfscope}%
\pgfsys@transformshift{2.296776in}{0.719236in}%
\pgfsys@useobject{currentmarker}{}%
\end{pgfscope}%
\begin{pgfscope}%
\pgfsys@transformshift{2.296894in}{0.694647in}%
\pgfsys@useobject{currentmarker}{}%
\end{pgfscope}%
\begin{pgfscope}%
\pgfsys@transformshift{2.297012in}{0.686537in}%
\pgfsys@useobject{currentmarker}{}%
\end{pgfscope}%
\begin{pgfscope}%
\pgfsys@transformshift{2.297129in}{0.630181in}%
\pgfsys@useobject{currentmarker}{}%
\end{pgfscope}%
\begin{pgfscope}%
\pgfsys@transformshift{2.297247in}{0.659137in}%
\pgfsys@useobject{currentmarker}{}%
\end{pgfscope}%
\begin{pgfscope}%
\pgfsys@transformshift{2.297364in}{0.711367in}%
\pgfsys@useobject{currentmarker}{}%
\end{pgfscope}%
\begin{pgfscope}%
\pgfsys@transformshift{2.297482in}{0.704890in}%
\pgfsys@useobject{currentmarker}{}%
\end{pgfscope}%
\begin{pgfscope}%
\pgfsys@transformshift{2.297599in}{0.669345in}%
\pgfsys@useobject{currentmarker}{}%
\end{pgfscope}%
\begin{pgfscope}%
\pgfsys@transformshift{2.297717in}{0.660290in}%
\pgfsys@useobject{currentmarker}{}%
\end{pgfscope}%
\begin{pgfscope}%
\pgfsys@transformshift{2.297834in}{0.669023in}%
\pgfsys@useobject{currentmarker}{}%
\end{pgfscope}%
\begin{pgfscope}%
\pgfsys@transformshift{2.297951in}{0.684629in}%
\pgfsys@useobject{currentmarker}{}%
\end{pgfscope}%
\begin{pgfscope}%
\pgfsys@transformshift{2.298068in}{0.674896in}%
\pgfsys@useobject{currentmarker}{}%
\end{pgfscope}%
\begin{pgfscope}%
\pgfsys@transformshift{2.298185in}{0.681418in}%
\pgfsys@useobject{currentmarker}{}%
\end{pgfscope}%
\begin{pgfscope}%
\pgfsys@transformshift{2.298302in}{0.700216in}%
\pgfsys@useobject{currentmarker}{}%
\end{pgfscope}%
\begin{pgfscope}%
\pgfsys@transformshift{2.298419in}{0.644460in}%
\pgfsys@useobject{currentmarker}{}%
\end{pgfscope}%
\begin{pgfscope}%
\pgfsys@transformshift{2.298536in}{0.657686in}%
\pgfsys@useobject{currentmarker}{}%
\end{pgfscope}%
\begin{pgfscope}%
\pgfsys@transformshift{2.298653in}{0.651396in}%
\pgfsys@useobject{currentmarker}{}%
\end{pgfscope}%
\begin{pgfscope}%
\pgfsys@transformshift{2.298770in}{0.647251in}%
\pgfsys@useobject{currentmarker}{}%
\end{pgfscope}%
\begin{pgfscope}%
\pgfsys@transformshift{2.298887in}{0.698020in}%
\pgfsys@useobject{currentmarker}{}%
\end{pgfscope}%
\begin{pgfscope}%
\pgfsys@transformshift{2.299003in}{0.681690in}%
\pgfsys@useobject{currentmarker}{}%
\end{pgfscope}%
\begin{pgfscope}%
\pgfsys@transformshift{2.299120in}{0.679153in}%
\pgfsys@useobject{currentmarker}{}%
\end{pgfscope}%
\begin{pgfscope}%
\pgfsys@transformshift{2.299236in}{0.711757in}%
\pgfsys@useobject{currentmarker}{}%
\end{pgfscope}%
\begin{pgfscope}%
\pgfsys@transformshift{2.299353in}{0.713783in}%
\pgfsys@useobject{currentmarker}{}%
\end{pgfscope}%
\begin{pgfscope}%
\pgfsys@transformshift{2.299469in}{0.713424in}%
\pgfsys@useobject{currentmarker}{}%
\end{pgfscope}%
\begin{pgfscope}%
\pgfsys@transformshift{2.299586in}{0.722065in}%
\pgfsys@useobject{currentmarker}{}%
\end{pgfscope}%
\begin{pgfscope}%
\pgfsys@transformshift{2.299702in}{0.700110in}%
\pgfsys@useobject{currentmarker}{}%
\end{pgfscope}%
\begin{pgfscope}%
\pgfsys@transformshift{2.299818in}{0.695800in}%
\pgfsys@useobject{currentmarker}{}%
\end{pgfscope}%
\begin{pgfscope}%
\pgfsys@transformshift{2.299934in}{0.660623in}%
\pgfsys@useobject{currentmarker}{}%
\end{pgfscope}%
\begin{pgfscope}%
\pgfsys@transformshift{2.300051in}{0.708406in}%
\pgfsys@useobject{currentmarker}{}%
\end{pgfscope}%
\begin{pgfscope}%
\pgfsys@transformshift{2.300167in}{0.704480in}%
\pgfsys@useobject{currentmarker}{}%
\end{pgfscope}%
\begin{pgfscope}%
\pgfsys@transformshift{2.300283in}{0.663695in}%
\pgfsys@useobject{currentmarker}{}%
\end{pgfscope}%
\begin{pgfscope}%
\pgfsys@transformshift{2.300399in}{0.690357in}%
\pgfsys@useobject{currentmarker}{}%
\end{pgfscope}%
\begin{pgfscope}%
\pgfsys@transformshift{2.300515in}{0.689635in}%
\pgfsys@useobject{currentmarker}{}%
\end{pgfscope}%
\begin{pgfscope}%
\pgfsys@transformshift{2.300630in}{0.610372in}%
\pgfsys@useobject{currentmarker}{}%
\end{pgfscope}%
\begin{pgfscope}%
\pgfsys@transformshift{2.300746in}{0.669117in}%
\pgfsys@useobject{currentmarker}{}%
\end{pgfscope}%
\begin{pgfscope}%
\pgfsys@transformshift{2.300862in}{0.672333in}%
\pgfsys@useobject{currentmarker}{}%
\end{pgfscope}%
\begin{pgfscope}%
\pgfsys@transformshift{2.300977in}{0.652687in}%
\pgfsys@useobject{currentmarker}{}%
\end{pgfscope}%
\begin{pgfscope}%
\pgfsys@transformshift{2.301093in}{0.688972in}%
\pgfsys@useobject{currentmarker}{}%
\end{pgfscope}%
\begin{pgfscope}%
\pgfsys@transformshift{2.301209in}{0.690778in}%
\pgfsys@useobject{currentmarker}{}%
\end{pgfscope}%
\begin{pgfscope}%
\pgfsys@transformshift{2.301324in}{0.703628in}%
\pgfsys@useobject{currentmarker}{}%
\end{pgfscope}%
\begin{pgfscope}%
\pgfsys@transformshift{2.301439in}{0.650830in}%
\pgfsys@useobject{currentmarker}{}%
\end{pgfscope}%
\begin{pgfscope}%
\pgfsys@transformshift{2.301555in}{0.711084in}%
\pgfsys@useobject{currentmarker}{}%
\end{pgfscope}%
\begin{pgfscope}%
\pgfsys@transformshift{2.301670in}{0.725936in}%
\pgfsys@useobject{currentmarker}{}%
\end{pgfscope}%
\begin{pgfscope}%
\pgfsys@transformshift{2.301785in}{0.677206in}%
\pgfsys@useobject{currentmarker}{}%
\end{pgfscope}%
\begin{pgfscope}%
\pgfsys@transformshift{2.301901in}{0.656372in}%
\pgfsys@useobject{currentmarker}{}%
\end{pgfscope}%
\begin{pgfscope}%
\pgfsys@transformshift{2.302016in}{0.611925in}%
\pgfsys@useobject{currentmarker}{}%
\end{pgfscope}%
\begin{pgfscope}%
\pgfsys@transformshift{2.302131in}{0.647378in}%
\pgfsys@useobject{currentmarker}{}%
\end{pgfscope}%
\begin{pgfscope}%
\pgfsys@transformshift{2.302246in}{0.702354in}%
\pgfsys@useobject{currentmarker}{}%
\end{pgfscope}%
\begin{pgfscope}%
\pgfsys@transformshift{2.302361in}{0.691732in}%
\pgfsys@useobject{currentmarker}{}%
\end{pgfscope}%
\begin{pgfscope}%
\pgfsys@transformshift{2.302476in}{0.744008in}%
\pgfsys@useobject{currentmarker}{}%
\end{pgfscope}%
\begin{pgfscope}%
\pgfsys@transformshift{2.302590in}{0.724019in}%
\pgfsys@useobject{currentmarker}{}%
\end{pgfscope}%
\begin{pgfscope}%
\pgfsys@transformshift{2.302705in}{0.676087in}%
\pgfsys@useobject{currentmarker}{}%
\end{pgfscope}%
\begin{pgfscope}%
\pgfsys@transformshift{2.302820in}{0.701958in}%
\pgfsys@useobject{currentmarker}{}%
\end{pgfscope}%
\begin{pgfscope}%
\pgfsys@transformshift{2.302934in}{0.691789in}%
\pgfsys@useobject{currentmarker}{}%
\end{pgfscope}%
\begin{pgfscope}%
\pgfsys@transformshift{2.303049in}{0.678858in}%
\pgfsys@useobject{currentmarker}{}%
\end{pgfscope}%
\begin{pgfscope}%
\pgfsys@transformshift{2.303164in}{0.695758in}%
\pgfsys@useobject{currentmarker}{}%
\end{pgfscope}%
\begin{pgfscope}%
\pgfsys@transformshift{2.303278in}{0.674349in}%
\pgfsys@useobject{currentmarker}{}%
\end{pgfscope}%
\begin{pgfscope}%
\pgfsys@transformshift{2.303392in}{0.694564in}%
\pgfsys@useobject{currentmarker}{}%
\end{pgfscope}%
\begin{pgfscope}%
\pgfsys@transformshift{2.303507in}{0.668463in}%
\pgfsys@useobject{currentmarker}{}%
\end{pgfscope}%
\begin{pgfscope}%
\pgfsys@transformshift{2.303621in}{0.693891in}%
\pgfsys@useobject{currentmarker}{}%
\end{pgfscope}%
\begin{pgfscope}%
\pgfsys@transformshift{2.303735in}{0.684243in}%
\pgfsys@useobject{currentmarker}{}%
\end{pgfscope}%
\begin{pgfscope}%
\pgfsys@transformshift{2.303850in}{0.681860in}%
\pgfsys@useobject{currentmarker}{}%
\end{pgfscope}%
\begin{pgfscope}%
\pgfsys@transformshift{2.303964in}{0.682685in}%
\pgfsys@useobject{currentmarker}{}%
\end{pgfscope}%
\begin{pgfscope}%
\pgfsys@transformshift{2.304078in}{0.702691in}%
\pgfsys@useobject{currentmarker}{}%
\end{pgfscope}%
\begin{pgfscope}%
\pgfsys@transformshift{2.304192in}{0.683335in}%
\pgfsys@useobject{currentmarker}{}%
\end{pgfscope}%
\begin{pgfscope}%
\pgfsys@transformshift{2.304306in}{0.685018in}%
\pgfsys@useobject{currentmarker}{}%
\end{pgfscope}%
\begin{pgfscope}%
\pgfsys@transformshift{2.304420in}{0.657241in}%
\pgfsys@useobject{currentmarker}{}%
\end{pgfscope}%
\begin{pgfscope}%
\pgfsys@transformshift{2.304533in}{0.620030in}%
\pgfsys@useobject{currentmarker}{}%
\end{pgfscope}%
\begin{pgfscope}%
\pgfsys@transformshift{2.304647in}{0.647755in}%
\pgfsys@useobject{currentmarker}{}%
\end{pgfscope}%
\begin{pgfscope}%
\pgfsys@transformshift{2.304761in}{0.663109in}%
\pgfsys@useobject{currentmarker}{}%
\end{pgfscope}%
\begin{pgfscope}%
\pgfsys@transformshift{2.304875in}{0.673240in}%
\pgfsys@useobject{currentmarker}{}%
\end{pgfscope}%
\begin{pgfscope}%
\pgfsys@transformshift{2.304988in}{0.685274in}%
\pgfsys@useobject{currentmarker}{}%
\end{pgfscope}%
\begin{pgfscope}%
\pgfsys@transformshift{2.305102in}{0.706652in}%
\pgfsys@useobject{currentmarker}{}%
\end{pgfscope}%
\begin{pgfscope}%
\pgfsys@transformshift{2.305215in}{0.703773in}%
\pgfsys@useobject{currentmarker}{}%
\end{pgfscope}%
\begin{pgfscope}%
\pgfsys@transformshift{2.305329in}{0.655876in}%
\pgfsys@useobject{currentmarker}{}%
\end{pgfscope}%
\begin{pgfscope}%
\pgfsys@transformshift{2.305442in}{0.656922in}%
\pgfsys@useobject{currentmarker}{}%
\end{pgfscope}%
\begin{pgfscope}%
\pgfsys@transformshift{2.305555in}{0.686950in}%
\pgfsys@useobject{currentmarker}{}%
\end{pgfscope}%
\begin{pgfscope}%
\pgfsys@transformshift{2.305669in}{0.722768in}%
\pgfsys@useobject{currentmarker}{}%
\end{pgfscope}%
\begin{pgfscope}%
\pgfsys@transformshift{2.305782in}{0.682267in}%
\pgfsys@useobject{currentmarker}{}%
\end{pgfscope}%
\begin{pgfscope}%
\pgfsys@transformshift{2.305895in}{0.720518in}%
\pgfsys@useobject{currentmarker}{}%
\end{pgfscope}%
\begin{pgfscope}%
\pgfsys@transformshift{2.306008in}{0.670210in}%
\pgfsys@useobject{currentmarker}{}%
\end{pgfscope}%
\begin{pgfscope}%
\pgfsys@transformshift{2.306121in}{0.671884in}%
\pgfsys@useobject{currentmarker}{}%
\end{pgfscope}%
\begin{pgfscope}%
\pgfsys@transformshift{2.306234in}{0.672703in}%
\pgfsys@useobject{currentmarker}{}%
\end{pgfscope}%
\begin{pgfscope}%
\pgfsys@transformshift{2.306347in}{0.635077in}%
\pgfsys@useobject{currentmarker}{}%
\end{pgfscope}%
\begin{pgfscope}%
\pgfsys@transformshift{2.306460in}{0.607463in}%
\pgfsys@useobject{currentmarker}{}%
\end{pgfscope}%
\begin{pgfscope}%
\pgfsys@transformshift{2.306573in}{0.691357in}%
\pgfsys@useobject{currentmarker}{}%
\end{pgfscope}%
\begin{pgfscope}%
\pgfsys@transformshift{2.306685in}{0.694229in}%
\pgfsys@useobject{currentmarker}{}%
\end{pgfscope}%
\begin{pgfscope}%
\pgfsys@transformshift{2.306798in}{0.666563in}%
\pgfsys@useobject{currentmarker}{}%
\end{pgfscope}%
\begin{pgfscope}%
\pgfsys@transformshift{2.306911in}{0.664029in}%
\pgfsys@useobject{currentmarker}{}%
\end{pgfscope}%
\begin{pgfscope}%
\pgfsys@transformshift{2.307023in}{0.689675in}%
\pgfsys@useobject{currentmarker}{}%
\end{pgfscope}%
\begin{pgfscope}%
\pgfsys@transformshift{2.307136in}{0.721194in}%
\pgfsys@useobject{currentmarker}{}%
\end{pgfscope}%
\begin{pgfscope}%
\pgfsys@transformshift{2.307248in}{0.741196in}%
\pgfsys@useobject{currentmarker}{}%
\end{pgfscope}%
\begin{pgfscope}%
\pgfsys@transformshift{2.307361in}{0.711213in}%
\pgfsys@useobject{currentmarker}{}%
\end{pgfscope}%
\begin{pgfscope}%
\pgfsys@transformshift{2.307473in}{0.708915in}%
\pgfsys@useobject{currentmarker}{}%
\end{pgfscope}%
\begin{pgfscope}%
\pgfsys@transformshift{2.307585in}{0.704658in}%
\pgfsys@useobject{currentmarker}{}%
\end{pgfscope}%
\begin{pgfscope}%
\pgfsys@transformshift{2.307698in}{0.683894in}%
\pgfsys@useobject{currentmarker}{}%
\end{pgfscope}%
\begin{pgfscope}%
\pgfsys@transformshift{2.307810in}{0.700205in}%
\pgfsys@useobject{currentmarker}{}%
\end{pgfscope}%
\begin{pgfscope}%
\pgfsys@transformshift{2.307922in}{0.632312in}%
\pgfsys@useobject{currentmarker}{}%
\end{pgfscope}%
\begin{pgfscope}%
\pgfsys@transformshift{2.308034in}{0.654943in}%
\pgfsys@useobject{currentmarker}{}%
\end{pgfscope}%
\begin{pgfscope}%
\pgfsys@transformshift{2.308146in}{0.655636in}%
\pgfsys@useobject{currentmarker}{}%
\end{pgfscope}%
\begin{pgfscope}%
\pgfsys@transformshift{2.308258in}{0.676511in}%
\pgfsys@useobject{currentmarker}{}%
\end{pgfscope}%
\begin{pgfscope}%
\pgfsys@transformshift{2.308370in}{0.688069in}%
\pgfsys@useobject{currentmarker}{}%
\end{pgfscope}%
\begin{pgfscope}%
\pgfsys@transformshift{2.308482in}{0.669588in}%
\pgfsys@useobject{currentmarker}{}%
\end{pgfscope}%
\begin{pgfscope}%
\pgfsys@transformshift{2.308594in}{0.687506in}%
\pgfsys@useobject{currentmarker}{}%
\end{pgfscope}%
\begin{pgfscope}%
\pgfsys@transformshift{2.308705in}{0.695060in}%
\pgfsys@useobject{currentmarker}{}%
\end{pgfscope}%
\begin{pgfscope}%
\pgfsys@transformshift{2.308817in}{0.637399in}%
\pgfsys@useobject{currentmarker}{}%
\end{pgfscope}%
\begin{pgfscope}%
\pgfsys@transformshift{2.308929in}{0.732979in}%
\pgfsys@useobject{currentmarker}{}%
\end{pgfscope}%
\begin{pgfscope}%
\pgfsys@transformshift{2.309040in}{0.713554in}%
\pgfsys@useobject{currentmarker}{}%
\end{pgfscope}%
\begin{pgfscope}%
\pgfsys@transformshift{2.309152in}{0.710393in}%
\pgfsys@useobject{currentmarker}{}%
\end{pgfscope}%
\begin{pgfscope}%
\pgfsys@transformshift{2.309263in}{0.721601in}%
\pgfsys@useobject{currentmarker}{}%
\end{pgfscope}%
\begin{pgfscope}%
\pgfsys@transformshift{2.309375in}{0.670824in}%
\pgfsys@useobject{currentmarker}{}%
\end{pgfscope}%
\begin{pgfscope}%
\pgfsys@transformshift{2.309486in}{0.698575in}%
\pgfsys@useobject{currentmarker}{}%
\end{pgfscope}%
\begin{pgfscope}%
\pgfsys@transformshift{2.309597in}{0.712754in}%
\pgfsys@useobject{currentmarker}{}%
\end{pgfscope}%
\begin{pgfscope}%
\pgfsys@transformshift{2.309708in}{0.659761in}%
\pgfsys@useobject{currentmarker}{}%
\end{pgfscope}%
\begin{pgfscope}%
\pgfsys@transformshift{2.309820in}{0.648335in}%
\pgfsys@useobject{currentmarker}{}%
\end{pgfscope}%
\begin{pgfscope}%
\pgfsys@transformshift{2.309931in}{0.699341in}%
\pgfsys@useobject{currentmarker}{}%
\end{pgfscope}%
\begin{pgfscope}%
\pgfsys@transformshift{2.310042in}{0.654900in}%
\pgfsys@useobject{currentmarker}{}%
\end{pgfscope}%
\begin{pgfscope}%
\pgfsys@transformshift{2.310153in}{0.626986in}%
\pgfsys@useobject{currentmarker}{}%
\end{pgfscope}%
\begin{pgfscope}%
\pgfsys@transformshift{2.310264in}{0.692107in}%
\pgfsys@useobject{currentmarker}{}%
\end{pgfscope}%
\begin{pgfscope}%
\pgfsys@transformshift{2.310375in}{0.692365in}%
\pgfsys@useobject{currentmarker}{}%
\end{pgfscope}%
\begin{pgfscope}%
\pgfsys@transformshift{2.310486in}{0.702024in}%
\pgfsys@useobject{currentmarker}{}%
\end{pgfscope}%
\begin{pgfscope}%
\pgfsys@transformshift{2.310596in}{0.624117in}%
\pgfsys@useobject{currentmarker}{}%
\end{pgfscope}%
\begin{pgfscope}%
\pgfsys@transformshift{2.310707in}{0.653013in}%
\pgfsys@useobject{currentmarker}{}%
\end{pgfscope}%
\begin{pgfscope}%
\pgfsys@transformshift{2.310818in}{0.651099in}%
\pgfsys@useobject{currentmarker}{}%
\end{pgfscope}%
\begin{pgfscope}%
\pgfsys@transformshift{2.310928in}{0.664690in}%
\pgfsys@useobject{currentmarker}{}%
\end{pgfscope}%
\begin{pgfscope}%
\pgfsys@transformshift{2.311039in}{0.663477in}%
\pgfsys@useobject{currentmarker}{}%
\end{pgfscope}%
\begin{pgfscope}%
\pgfsys@transformshift{2.311150in}{0.674024in}%
\pgfsys@useobject{currentmarker}{}%
\end{pgfscope}%
\begin{pgfscope}%
\pgfsys@transformshift{2.311260in}{0.623809in}%
\pgfsys@useobject{currentmarker}{}%
\end{pgfscope}%
\begin{pgfscope}%
\pgfsys@transformshift{2.311370in}{0.676013in}%
\pgfsys@useobject{currentmarker}{}%
\end{pgfscope}%
\begin{pgfscope}%
\pgfsys@transformshift{2.311481in}{0.662475in}%
\pgfsys@useobject{currentmarker}{}%
\end{pgfscope}%
\begin{pgfscope}%
\pgfsys@transformshift{2.311591in}{0.669260in}%
\pgfsys@useobject{currentmarker}{}%
\end{pgfscope}%
\begin{pgfscope}%
\pgfsys@transformshift{2.311701in}{0.668396in}%
\pgfsys@useobject{currentmarker}{}%
\end{pgfscope}%
\begin{pgfscope}%
\pgfsys@transformshift{2.311812in}{0.705033in}%
\pgfsys@useobject{currentmarker}{}%
\end{pgfscope}%
\begin{pgfscope}%
\pgfsys@transformshift{2.311922in}{0.697084in}%
\pgfsys@useobject{currentmarker}{}%
\end{pgfscope}%
\begin{pgfscope}%
\pgfsys@transformshift{2.312032in}{0.698491in}%
\pgfsys@useobject{currentmarker}{}%
\end{pgfscope}%
\begin{pgfscope}%
\pgfsys@transformshift{2.312142in}{0.736602in}%
\pgfsys@useobject{currentmarker}{}%
\end{pgfscope}%
\begin{pgfscope}%
\pgfsys@transformshift{2.312252in}{0.725863in}%
\pgfsys@useobject{currentmarker}{}%
\end{pgfscope}%
\begin{pgfscope}%
\pgfsys@transformshift{2.312362in}{0.740974in}%
\pgfsys@useobject{currentmarker}{}%
\end{pgfscope}%
\begin{pgfscope}%
\pgfsys@transformshift{2.312472in}{0.724632in}%
\pgfsys@useobject{currentmarker}{}%
\end{pgfscope}%
\begin{pgfscope}%
\pgfsys@transformshift{2.312582in}{0.688752in}%
\pgfsys@useobject{currentmarker}{}%
\end{pgfscope}%
\begin{pgfscope}%
\pgfsys@transformshift{2.312691in}{0.657374in}%
\pgfsys@useobject{currentmarker}{}%
\end{pgfscope}%
\end{pgfscope}%
\begin{pgfscope}%
\pgfsetrectcap%
\pgfsetmiterjoin%
\pgfsetlinewidth{0.803000pt}%
\definecolor{currentstroke}{rgb}{0.000000,0.000000,0.000000}%
\pgfsetstrokecolor{currentstroke}%
\pgfsetdash{}{0pt}%
\pgfpathmoveto{\pgfqpoint{0.514278in}{0.417642in}}%
\pgfpathlineto{\pgfqpoint{0.514278in}{1.788330in}}%
\pgfusepath{stroke}%
\end{pgfscope}%
\begin{pgfscope}%
\pgfsetrectcap%
\pgfsetmiterjoin%
\pgfsetlinewidth{0.803000pt}%
\definecolor{currentstroke}{rgb}{0.000000,0.000000,0.000000}%
\pgfsetstrokecolor{currentstroke}%
\pgfsetdash{}{0pt}%
\pgfpathmoveto{\pgfqpoint{2.398330in}{0.417642in}}%
\pgfpathlineto{\pgfqpoint{2.398330in}{1.788330in}}%
\pgfusepath{stroke}%
\end{pgfscope}%
\begin{pgfscope}%
\pgfsetrectcap%
\pgfsetmiterjoin%
\pgfsetlinewidth{0.803000pt}%
\definecolor{currentstroke}{rgb}{0.000000,0.000000,0.000000}%
\pgfsetstrokecolor{currentstroke}%
\pgfsetdash{}{0pt}%
\pgfpathmoveto{\pgfqpoint{0.514278in}{0.417642in}}%
\pgfpathlineto{\pgfqpoint{2.398330in}{0.417642in}}%
\pgfusepath{stroke}%
\end{pgfscope}%
\begin{pgfscope}%
\pgfsetrectcap%
\pgfsetmiterjoin%
\pgfsetlinewidth{0.803000pt}%
\definecolor{currentstroke}{rgb}{0.000000,0.000000,0.000000}%
\pgfsetstrokecolor{currentstroke}%
\pgfsetdash{}{0pt}%
\pgfpathmoveto{\pgfqpoint{0.514278in}{1.788330in}}%
\pgfpathlineto{\pgfqpoint{2.398330in}{1.788330in}}%
\pgfusepath{stroke}%
\end{pgfscope}%
\begin{pgfscope}%
\pgfsetbuttcap%
\pgfsetmiterjoin%
\definecolor{currentfill}{rgb}{1.000000,1.000000,1.000000}%
\pgfsetfillcolor{currentfill}%
\pgfsetfillopacity{0.800000}%
\pgfsetlinewidth{1.003750pt}%
\definecolor{currentstroke}{rgb}{0.800000,0.800000,0.800000}%
\pgfsetstrokecolor{currentstroke}%
\pgfsetstrokeopacity{0.800000}%
\pgfsetdash{}{0pt}%
\pgfpathmoveto{\pgfqpoint{1.712264in}{1.522420in}}%
\pgfpathlineto{\pgfqpoint{2.320552in}{1.522420in}}%
\pgfpathquadraticcurveto{\pgfqpoint{2.342774in}{1.522420in}}{\pgfqpoint{2.342774in}{1.544642in}}%
\pgfpathlineto{\pgfqpoint{2.342774in}{1.710552in}}%
\pgfpathquadraticcurveto{\pgfqpoint{2.342774in}{1.732774in}}{\pgfqpoint{2.320552in}{1.732774in}}%
\pgfpathlineto{\pgfqpoint{1.712264in}{1.732774in}}%
\pgfpathquadraticcurveto{\pgfqpoint{1.690042in}{1.732774in}}{\pgfqpoint{1.690042in}{1.710552in}}%
\pgfpathlineto{\pgfqpoint{1.690042in}{1.544642in}}%
\pgfpathquadraticcurveto{\pgfqpoint{1.690042in}{1.522420in}}{\pgfqpoint{1.712264in}{1.522420in}}%
\pgfpathlineto{\pgfqpoint{1.712264in}{1.522420in}}%
\pgfpathclose%
\pgfusepath{stroke,fill}%
\end{pgfscope}%
\begin{pgfscope}%
\pgfsetbuttcap%
\pgfsetroundjoin%
\pgfsetlinewidth{1.505625pt}%
\definecolor{currentstroke}{rgb}{0.000000,0.447059,0.698039}%
\pgfsetstrokecolor{currentstroke}%
\pgfsetdash{{5.550000pt}{2.400000pt}}{0.000000pt}%
\pgfpathmoveto{\pgfqpoint{1.734486in}{1.627358in}}%
\pgfpathlineto{\pgfqpoint{1.845597in}{1.627358in}}%
\pgfpathlineto{\pgfqpoint{1.956708in}{1.627358in}}%
\pgfusepath{stroke}%
\end{pgfscope}%
\begin{pgfscope}%
\definecolor{textcolor}{rgb}{0.000000,0.000000,0.000000}%
\pgfsetstrokecolor{textcolor}%
\pgfsetfillcolor{textcolor}%
\pgftext[x=2.045597in,y=1.588469in,left,base]{\color{textcolor}\rmfamily\fontsize{8.000000}{9.600000}\selectfont \(\displaystyle h_{0}f^{0}\)}%
\end{pgfscope}%
\end{pgfpicture}%
\makeatother%
\endgroup%

        } % scalebox
        \caption{Power spectral density}
        \label{fig:white_noise_psd}
    \end{subfigure}
    \hfill
    \begin{subfigure}{0.32\linewidth}
        \centering
        \scalebox{0.75}{%
            %% Creator: Matplotlib, PGF backend
%%
%% To include the figure in your LaTeX document, write
%%   \input{<filename>.pgf}
%%
%% Make sure the required packages are loaded in your preamble
%%   \usepackage{pgf}
%%
%% Also ensure that all the required font packages are loaded; for instance,
%% the lmodern package is sometimes necessary when using math font.
%%   \usepackage{lmodern}
%%
%% Figures using additional raster images can only be included by \input if
%% they are in the same directory as the main LaTeX file. For loading figures
%% from other directories you can use the `import` package
%%   \usepackage{import}
%%
%% and then include the figures with
%%   \import{<path to file>}{<filename>.pgf}
%%
%% Matplotlib used the following preamble
%%   \usepackage{siunitx}
%%   \usepackage{fontspec}
%%   \makeatletter\@ifpackageloaded{underscore}{}{\usepackage[strings]{underscore}}\makeatother
%%
\begingroup%
\makeatletter%
\begin{pgfpicture}%
\pgfpathrectangle{\pgfpointorigin}{\pgfqpoint{2.440000in}{1.830000in}}%
\pgfusepath{use as bounding box, clip}%
\begin{pgfscope}%
\pgfsetbuttcap%
\pgfsetmiterjoin%
\definecolor{currentfill}{rgb}{1.000000,1.000000,1.000000}%
\pgfsetfillcolor{currentfill}%
\pgfsetlinewidth{0.000000pt}%
\definecolor{currentstroke}{rgb}{1.000000,1.000000,1.000000}%
\pgfsetstrokecolor{currentstroke}%
\pgfsetdash{}{0pt}%
\pgfpathmoveto{\pgfqpoint{0.000000in}{0.000000in}}%
\pgfpathlineto{\pgfqpoint{2.440000in}{0.000000in}}%
\pgfpathlineto{\pgfqpoint{2.440000in}{1.830000in}}%
\pgfpathlineto{\pgfqpoint{0.000000in}{1.830000in}}%
\pgfpathlineto{\pgfqpoint{0.000000in}{0.000000in}}%
\pgfpathclose%
\pgfusepath{fill}%
\end{pgfscope}%
\begin{pgfscope}%
\pgfsetbuttcap%
\pgfsetmiterjoin%
\definecolor{currentfill}{rgb}{1.000000,1.000000,1.000000}%
\pgfsetfillcolor{currentfill}%
\pgfsetlinewidth{0.000000pt}%
\definecolor{currentstroke}{rgb}{0.000000,0.000000,0.000000}%
\pgfsetstrokecolor{currentstroke}%
\pgfsetstrokeopacity{0.000000}%
\pgfsetdash{}{0pt}%
\pgfpathmoveto{\pgfqpoint{0.589510in}{0.417642in}}%
\pgfpathlineto{\pgfqpoint{2.398330in}{0.417642in}}%
\pgfpathlineto{\pgfqpoint{2.398330in}{1.788330in}}%
\pgfpathlineto{\pgfqpoint{0.589510in}{1.788330in}}%
\pgfpathlineto{\pgfqpoint{0.589510in}{0.417642in}}%
\pgfpathclose%
\pgfusepath{fill}%
\end{pgfscope}%
\begin{pgfscope}%
\pgfpathrectangle{\pgfqpoint{0.589510in}{0.417642in}}{\pgfqpoint{1.808820in}{1.370688in}}%
\pgfusepath{clip}%
\pgfsetrectcap%
\pgfsetroundjoin%
\pgfsetlinewidth{0.803000pt}%
\definecolor{currentstroke}{rgb}{0.450000,0.450000,0.450000}%
\pgfsetstrokecolor{currentstroke}%
\pgfsetdash{}{0pt}%
\pgfpathmoveto{\pgfqpoint{0.671729in}{0.417642in}}%
\pgfpathlineto{\pgfqpoint{0.671729in}{1.788330in}}%
\pgfusepath{stroke}%
\end{pgfscope}%
\begin{pgfscope}%
\pgfsetbuttcap%
\pgfsetroundjoin%
\definecolor{currentfill}{rgb}{0.000000,0.000000,0.000000}%
\pgfsetfillcolor{currentfill}%
\pgfsetlinewidth{0.803000pt}%
\definecolor{currentstroke}{rgb}{0.000000,0.000000,0.000000}%
\pgfsetstrokecolor{currentstroke}%
\pgfsetdash{}{0pt}%
\pgfsys@defobject{currentmarker}{\pgfqpoint{0.000000in}{-0.048611in}}{\pgfqpoint{0.000000in}{0.000000in}}{%
\pgfpathmoveto{\pgfqpoint{0.000000in}{0.000000in}}%
\pgfpathlineto{\pgfqpoint{0.000000in}{-0.048611in}}%
\pgfusepath{stroke,fill}%
}%
\begin{pgfscope}%
\pgfsys@transformshift{0.671729in}{0.417642in}%
\pgfsys@useobject{currentmarker}{}%
\end{pgfscope}%
\end{pgfscope}%
\begin{pgfscope}%
\definecolor{textcolor}{rgb}{0.000000,0.000000,0.000000}%
\pgfsetstrokecolor{textcolor}%
\pgfsetfillcolor{textcolor}%
\pgftext[x=0.671729in,y=0.320420in,,top]{\color{textcolor}\rmfamily\fontsize{8.000000}{9.600000}\selectfont \(\displaystyle {10^{0}}\)}%
\end{pgfscope}%
\begin{pgfscope}%
\pgfpathrectangle{\pgfqpoint{0.589510in}{0.417642in}}{\pgfqpoint{1.808820in}{1.370688in}}%
\pgfusepath{clip}%
\pgfsetrectcap%
\pgfsetroundjoin%
\pgfsetlinewidth{0.803000pt}%
\definecolor{currentstroke}{rgb}{0.450000,0.450000,0.450000}%
\pgfsetstrokecolor{currentstroke}%
\pgfsetdash{}{0pt}%
\pgfpathmoveto{\pgfqpoint{1.128240in}{0.417642in}}%
\pgfpathlineto{\pgfqpoint{1.128240in}{1.788330in}}%
\pgfusepath{stroke}%
\end{pgfscope}%
\begin{pgfscope}%
\pgfsetbuttcap%
\pgfsetroundjoin%
\definecolor{currentfill}{rgb}{0.000000,0.000000,0.000000}%
\pgfsetfillcolor{currentfill}%
\pgfsetlinewidth{0.803000pt}%
\definecolor{currentstroke}{rgb}{0.000000,0.000000,0.000000}%
\pgfsetstrokecolor{currentstroke}%
\pgfsetdash{}{0pt}%
\pgfsys@defobject{currentmarker}{\pgfqpoint{0.000000in}{-0.048611in}}{\pgfqpoint{0.000000in}{0.000000in}}{%
\pgfpathmoveto{\pgfqpoint{0.000000in}{0.000000in}}%
\pgfpathlineto{\pgfqpoint{0.000000in}{-0.048611in}}%
\pgfusepath{stroke,fill}%
}%
\begin{pgfscope}%
\pgfsys@transformshift{1.128240in}{0.417642in}%
\pgfsys@useobject{currentmarker}{}%
\end{pgfscope}%
\end{pgfscope}%
\begin{pgfscope}%
\definecolor{textcolor}{rgb}{0.000000,0.000000,0.000000}%
\pgfsetstrokecolor{textcolor}%
\pgfsetfillcolor{textcolor}%
\pgftext[x=1.128240in,y=0.320420in,,top]{\color{textcolor}\rmfamily\fontsize{8.000000}{9.600000}\selectfont \(\displaystyle {10^{1}}\)}%
\end{pgfscope}%
\begin{pgfscope}%
\pgfpathrectangle{\pgfqpoint{0.589510in}{0.417642in}}{\pgfqpoint{1.808820in}{1.370688in}}%
\pgfusepath{clip}%
\pgfsetrectcap%
\pgfsetroundjoin%
\pgfsetlinewidth{0.803000pt}%
\definecolor{currentstroke}{rgb}{0.450000,0.450000,0.450000}%
\pgfsetstrokecolor{currentstroke}%
\pgfsetdash{}{0pt}%
\pgfpathmoveto{\pgfqpoint{1.584752in}{0.417642in}}%
\pgfpathlineto{\pgfqpoint{1.584752in}{1.788330in}}%
\pgfusepath{stroke}%
\end{pgfscope}%
\begin{pgfscope}%
\pgfsetbuttcap%
\pgfsetroundjoin%
\definecolor{currentfill}{rgb}{0.000000,0.000000,0.000000}%
\pgfsetfillcolor{currentfill}%
\pgfsetlinewidth{0.803000pt}%
\definecolor{currentstroke}{rgb}{0.000000,0.000000,0.000000}%
\pgfsetstrokecolor{currentstroke}%
\pgfsetdash{}{0pt}%
\pgfsys@defobject{currentmarker}{\pgfqpoint{0.000000in}{-0.048611in}}{\pgfqpoint{0.000000in}{0.000000in}}{%
\pgfpathmoveto{\pgfqpoint{0.000000in}{0.000000in}}%
\pgfpathlineto{\pgfqpoint{0.000000in}{-0.048611in}}%
\pgfusepath{stroke,fill}%
}%
\begin{pgfscope}%
\pgfsys@transformshift{1.584752in}{0.417642in}%
\pgfsys@useobject{currentmarker}{}%
\end{pgfscope}%
\end{pgfscope}%
\begin{pgfscope}%
\definecolor{textcolor}{rgb}{0.000000,0.000000,0.000000}%
\pgfsetstrokecolor{textcolor}%
\pgfsetfillcolor{textcolor}%
\pgftext[x=1.584752in,y=0.320420in,,top]{\color{textcolor}\rmfamily\fontsize{8.000000}{9.600000}\selectfont \(\displaystyle {10^{2}}\)}%
\end{pgfscope}%
\begin{pgfscope}%
\pgfpathrectangle{\pgfqpoint{0.589510in}{0.417642in}}{\pgfqpoint{1.808820in}{1.370688in}}%
\pgfusepath{clip}%
\pgfsetrectcap%
\pgfsetroundjoin%
\pgfsetlinewidth{0.803000pt}%
\definecolor{currentstroke}{rgb}{0.450000,0.450000,0.450000}%
\pgfsetstrokecolor{currentstroke}%
\pgfsetdash{}{0pt}%
\pgfpathmoveto{\pgfqpoint{2.041264in}{0.417642in}}%
\pgfpathlineto{\pgfqpoint{2.041264in}{1.788330in}}%
\pgfusepath{stroke}%
\end{pgfscope}%
\begin{pgfscope}%
\pgfsetbuttcap%
\pgfsetroundjoin%
\definecolor{currentfill}{rgb}{0.000000,0.000000,0.000000}%
\pgfsetfillcolor{currentfill}%
\pgfsetlinewidth{0.803000pt}%
\definecolor{currentstroke}{rgb}{0.000000,0.000000,0.000000}%
\pgfsetstrokecolor{currentstroke}%
\pgfsetdash{}{0pt}%
\pgfsys@defobject{currentmarker}{\pgfqpoint{0.000000in}{-0.048611in}}{\pgfqpoint{0.000000in}{0.000000in}}{%
\pgfpathmoveto{\pgfqpoint{0.000000in}{0.000000in}}%
\pgfpathlineto{\pgfqpoint{0.000000in}{-0.048611in}}%
\pgfusepath{stroke,fill}%
}%
\begin{pgfscope}%
\pgfsys@transformshift{2.041264in}{0.417642in}%
\pgfsys@useobject{currentmarker}{}%
\end{pgfscope}%
\end{pgfscope}%
\begin{pgfscope}%
\definecolor{textcolor}{rgb}{0.000000,0.000000,0.000000}%
\pgfsetstrokecolor{textcolor}%
\pgfsetfillcolor{textcolor}%
\pgftext[x=2.041264in,y=0.320420in,,top]{\color{textcolor}\rmfamily\fontsize{8.000000}{9.600000}\selectfont \(\displaystyle {10^{3}}\)}%
\end{pgfscope}%
\begin{pgfscope}%
\pgfpathrectangle{\pgfqpoint{0.589510in}{0.417642in}}{\pgfqpoint{1.808820in}{1.370688in}}%
\pgfusepath{clip}%
\pgfsetrectcap%
\pgfsetroundjoin%
\pgfsetlinewidth{0.803000pt}%
\definecolor{currentstroke}{rgb}{0.850000,0.850000,0.850000}%
\pgfsetstrokecolor{currentstroke}%
\pgfsetdash{}{0pt}%
\pgfpathmoveto{\pgfqpoint{0.601014in}{0.417642in}}%
\pgfpathlineto{\pgfqpoint{0.601014in}{1.788330in}}%
\pgfusepath{stroke}%
\end{pgfscope}%
\begin{pgfscope}%
\pgfsetbuttcap%
\pgfsetroundjoin%
\definecolor{currentfill}{rgb}{0.000000,0.000000,0.000000}%
\pgfsetfillcolor{currentfill}%
\pgfsetlinewidth{0.602250pt}%
\definecolor{currentstroke}{rgb}{0.000000,0.000000,0.000000}%
\pgfsetstrokecolor{currentstroke}%
\pgfsetdash{}{0pt}%
\pgfsys@defobject{currentmarker}{\pgfqpoint{0.000000in}{-0.027778in}}{\pgfqpoint{0.000000in}{0.000000in}}{%
\pgfpathmoveto{\pgfqpoint{0.000000in}{0.000000in}}%
\pgfpathlineto{\pgfqpoint{0.000000in}{-0.027778in}}%
\pgfusepath{stroke,fill}%
}%
\begin{pgfscope}%
\pgfsys@transformshift{0.601014in}{0.417642in}%
\pgfsys@useobject{currentmarker}{}%
\end{pgfscope}%
\end{pgfscope}%
\begin{pgfscope}%
\pgfpathrectangle{\pgfqpoint{0.589510in}{0.417642in}}{\pgfqpoint{1.808820in}{1.370688in}}%
\pgfusepath{clip}%
\pgfsetrectcap%
\pgfsetroundjoin%
\pgfsetlinewidth{0.803000pt}%
\definecolor{currentstroke}{rgb}{0.850000,0.850000,0.850000}%
\pgfsetstrokecolor{currentstroke}%
\pgfsetdash{}{0pt}%
\pgfpathmoveto{\pgfqpoint{0.627488in}{0.417642in}}%
\pgfpathlineto{\pgfqpoint{0.627488in}{1.788330in}}%
\pgfusepath{stroke}%
\end{pgfscope}%
\begin{pgfscope}%
\pgfsetbuttcap%
\pgfsetroundjoin%
\definecolor{currentfill}{rgb}{0.000000,0.000000,0.000000}%
\pgfsetfillcolor{currentfill}%
\pgfsetlinewidth{0.602250pt}%
\definecolor{currentstroke}{rgb}{0.000000,0.000000,0.000000}%
\pgfsetstrokecolor{currentstroke}%
\pgfsetdash{}{0pt}%
\pgfsys@defobject{currentmarker}{\pgfqpoint{0.000000in}{-0.027778in}}{\pgfqpoint{0.000000in}{0.000000in}}{%
\pgfpathmoveto{\pgfqpoint{0.000000in}{0.000000in}}%
\pgfpathlineto{\pgfqpoint{0.000000in}{-0.027778in}}%
\pgfusepath{stroke,fill}%
}%
\begin{pgfscope}%
\pgfsys@transformshift{0.627488in}{0.417642in}%
\pgfsys@useobject{currentmarker}{}%
\end{pgfscope}%
\end{pgfscope}%
\begin{pgfscope}%
\pgfpathrectangle{\pgfqpoint{0.589510in}{0.417642in}}{\pgfqpoint{1.808820in}{1.370688in}}%
\pgfusepath{clip}%
\pgfsetrectcap%
\pgfsetroundjoin%
\pgfsetlinewidth{0.803000pt}%
\definecolor{currentstroke}{rgb}{0.850000,0.850000,0.850000}%
\pgfsetstrokecolor{currentstroke}%
\pgfsetdash{}{0pt}%
\pgfpathmoveto{\pgfqpoint{0.650840in}{0.417642in}}%
\pgfpathlineto{\pgfqpoint{0.650840in}{1.788330in}}%
\pgfusepath{stroke}%
\end{pgfscope}%
\begin{pgfscope}%
\pgfsetbuttcap%
\pgfsetroundjoin%
\definecolor{currentfill}{rgb}{0.000000,0.000000,0.000000}%
\pgfsetfillcolor{currentfill}%
\pgfsetlinewidth{0.602250pt}%
\definecolor{currentstroke}{rgb}{0.000000,0.000000,0.000000}%
\pgfsetstrokecolor{currentstroke}%
\pgfsetdash{}{0pt}%
\pgfsys@defobject{currentmarker}{\pgfqpoint{0.000000in}{-0.027778in}}{\pgfqpoint{0.000000in}{0.000000in}}{%
\pgfpathmoveto{\pgfqpoint{0.000000in}{0.000000in}}%
\pgfpathlineto{\pgfqpoint{0.000000in}{-0.027778in}}%
\pgfusepath{stroke,fill}%
}%
\begin{pgfscope}%
\pgfsys@transformshift{0.650840in}{0.417642in}%
\pgfsys@useobject{currentmarker}{}%
\end{pgfscope}%
\end{pgfscope}%
\begin{pgfscope}%
\pgfpathrectangle{\pgfqpoint{0.589510in}{0.417642in}}{\pgfqpoint{1.808820in}{1.370688in}}%
\pgfusepath{clip}%
\pgfsetrectcap%
\pgfsetroundjoin%
\pgfsetlinewidth{0.803000pt}%
\definecolor{currentstroke}{rgb}{0.850000,0.850000,0.850000}%
\pgfsetstrokecolor{currentstroke}%
\pgfsetdash{}{0pt}%
\pgfpathmoveto{\pgfqpoint{0.809153in}{0.417642in}}%
\pgfpathlineto{\pgfqpoint{0.809153in}{1.788330in}}%
\pgfusepath{stroke}%
\end{pgfscope}%
\begin{pgfscope}%
\pgfsetbuttcap%
\pgfsetroundjoin%
\definecolor{currentfill}{rgb}{0.000000,0.000000,0.000000}%
\pgfsetfillcolor{currentfill}%
\pgfsetlinewidth{0.602250pt}%
\definecolor{currentstroke}{rgb}{0.000000,0.000000,0.000000}%
\pgfsetstrokecolor{currentstroke}%
\pgfsetdash{}{0pt}%
\pgfsys@defobject{currentmarker}{\pgfqpoint{0.000000in}{-0.027778in}}{\pgfqpoint{0.000000in}{0.000000in}}{%
\pgfpathmoveto{\pgfqpoint{0.000000in}{0.000000in}}%
\pgfpathlineto{\pgfqpoint{0.000000in}{-0.027778in}}%
\pgfusepath{stroke,fill}%
}%
\begin{pgfscope}%
\pgfsys@transformshift{0.809153in}{0.417642in}%
\pgfsys@useobject{currentmarker}{}%
\end{pgfscope}%
\end{pgfscope}%
\begin{pgfscope}%
\pgfpathrectangle{\pgfqpoint{0.589510in}{0.417642in}}{\pgfqpoint{1.808820in}{1.370688in}}%
\pgfusepath{clip}%
\pgfsetrectcap%
\pgfsetroundjoin%
\pgfsetlinewidth{0.803000pt}%
\definecolor{currentstroke}{rgb}{0.850000,0.850000,0.850000}%
\pgfsetstrokecolor{currentstroke}%
\pgfsetdash{}{0pt}%
\pgfpathmoveto{\pgfqpoint{0.889540in}{0.417642in}}%
\pgfpathlineto{\pgfqpoint{0.889540in}{1.788330in}}%
\pgfusepath{stroke}%
\end{pgfscope}%
\begin{pgfscope}%
\pgfsetbuttcap%
\pgfsetroundjoin%
\definecolor{currentfill}{rgb}{0.000000,0.000000,0.000000}%
\pgfsetfillcolor{currentfill}%
\pgfsetlinewidth{0.602250pt}%
\definecolor{currentstroke}{rgb}{0.000000,0.000000,0.000000}%
\pgfsetstrokecolor{currentstroke}%
\pgfsetdash{}{0pt}%
\pgfsys@defobject{currentmarker}{\pgfqpoint{0.000000in}{-0.027778in}}{\pgfqpoint{0.000000in}{0.000000in}}{%
\pgfpathmoveto{\pgfqpoint{0.000000in}{0.000000in}}%
\pgfpathlineto{\pgfqpoint{0.000000in}{-0.027778in}}%
\pgfusepath{stroke,fill}%
}%
\begin{pgfscope}%
\pgfsys@transformshift{0.889540in}{0.417642in}%
\pgfsys@useobject{currentmarker}{}%
\end{pgfscope}%
\end{pgfscope}%
\begin{pgfscope}%
\pgfpathrectangle{\pgfqpoint{0.589510in}{0.417642in}}{\pgfqpoint{1.808820in}{1.370688in}}%
\pgfusepath{clip}%
\pgfsetrectcap%
\pgfsetroundjoin%
\pgfsetlinewidth{0.803000pt}%
\definecolor{currentstroke}{rgb}{0.850000,0.850000,0.850000}%
\pgfsetstrokecolor{currentstroke}%
\pgfsetdash{}{0pt}%
\pgfpathmoveto{\pgfqpoint{0.946576in}{0.417642in}}%
\pgfpathlineto{\pgfqpoint{0.946576in}{1.788330in}}%
\pgfusepath{stroke}%
\end{pgfscope}%
\begin{pgfscope}%
\pgfsetbuttcap%
\pgfsetroundjoin%
\definecolor{currentfill}{rgb}{0.000000,0.000000,0.000000}%
\pgfsetfillcolor{currentfill}%
\pgfsetlinewidth{0.602250pt}%
\definecolor{currentstroke}{rgb}{0.000000,0.000000,0.000000}%
\pgfsetstrokecolor{currentstroke}%
\pgfsetdash{}{0pt}%
\pgfsys@defobject{currentmarker}{\pgfqpoint{0.000000in}{-0.027778in}}{\pgfqpoint{0.000000in}{0.000000in}}{%
\pgfpathmoveto{\pgfqpoint{0.000000in}{0.000000in}}%
\pgfpathlineto{\pgfqpoint{0.000000in}{-0.027778in}}%
\pgfusepath{stroke,fill}%
}%
\begin{pgfscope}%
\pgfsys@transformshift{0.946576in}{0.417642in}%
\pgfsys@useobject{currentmarker}{}%
\end{pgfscope}%
\end{pgfscope}%
\begin{pgfscope}%
\pgfpathrectangle{\pgfqpoint{0.589510in}{0.417642in}}{\pgfqpoint{1.808820in}{1.370688in}}%
\pgfusepath{clip}%
\pgfsetrectcap%
\pgfsetroundjoin%
\pgfsetlinewidth{0.803000pt}%
\definecolor{currentstroke}{rgb}{0.850000,0.850000,0.850000}%
\pgfsetstrokecolor{currentstroke}%
\pgfsetdash{}{0pt}%
\pgfpathmoveto{\pgfqpoint{0.990817in}{0.417642in}}%
\pgfpathlineto{\pgfqpoint{0.990817in}{1.788330in}}%
\pgfusepath{stroke}%
\end{pgfscope}%
\begin{pgfscope}%
\pgfsetbuttcap%
\pgfsetroundjoin%
\definecolor{currentfill}{rgb}{0.000000,0.000000,0.000000}%
\pgfsetfillcolor{currentfill}%
\pgfsetlinewidth{0.602250pt}%
\definecolor{currentstroke}{rgb}{0.000000,0.000000,0.000000}%
\pgfsetstrokecolor{currentstroke}%
\pgfsetdash{}{0pt}%
\pgfsys@defobject{currentmarker}{\pgfqpoint{0.000000in}{-0.027778in}}{\pgfqpoint{0.000000in}{0.000000in}}{%
\pgfpathmoveto{\pgfqpoint{0.000000in}{0.000000in}}%
\pgfpathlineto{\pgfqpoint{0.000000in}{-0.027778in}}%
\pgfusepath{stroke,fill}%
}%
\begin{pgfscope}%
\pgfsys@transformshift{0.990817in}{0.417642in}%
\pgfsys@useobject{currentmarker}{}%
\end{pgfscope}%
\end{pgfscope}%
\begin{pgfscope}%
\pgfpathrectangle{\pgfqpoint{0.589510in}{0.417642in}}{\pgfqpoint{1.808820in}{1.370688in}}%
\pgfusepath{clip}%
\pgfsetrectcap%
\pgfsetroundjoin%
\pgfsetlinewidth{0.803000pt}%
\definecolor{currentstroke}{rgb}{0.850000,0.850000,0.850000}%
\pgfsetstrokecolor{currentstroke}%
\pgfsetdash{}{0pt}%
\pgfpathmoveto{\pgfqpoint{1.026964in}{0.417642in}}%
\pgfpathlineto{\pgfqpoint{1.026964in}{1.788330in}}%
\pgfusepath{stroke}%
\end{pgfscope}%
\begin{pgfscope}%
\pgfsetbuttcap%
\pgfsetroundjoin%
\definecolor{currentfill}{rgb}{0.000000,0.000000,0.000000}%
\pgfsetfillcolor{currentfill}%
\pgfsetlinewidth{0.602250pt}%
\definecolor{currentstroke}{rgb}{0.000000,0.000000,0.000000}%
\pgfsetstrokecolor{currentstroke}%
\pgfsetdash{}{0pt}%
\pgfsys@defobject{currentmarker}{\pgfqpoint{0.000000in}{-0.027778in}}{\pgfqpoint{0.000000in}{0.000000in}}{%
\pgfpathmoveto{\pgfqpoint{0.000000in}{0.000000in}}%
\pgfpathlineto{\pgfqpoint{0.000000in}{-0.027778in}}%
\pgfusepath{stroke,fill}%
}%
\begin{pgfscope}%
\pgfsys@transformshift{1.026964in}{0.417642in}%
\pgfsys@useobject{currentmarker}{}%
\end{pgfscope}%
\end{pgfscope}%
\begin{pgfscope}%
\pgfpathrectangle{\pgfqpoint{0.589510in}{0.417642in}}{\pgfqpoint{1.808820in}{1.370688in}}%
\pgfusepath{clip}%
\pgfsetrectcap%
\pgfsetroundjoin%
\pgfsetlinewidth{0.803000pt}%
\definecolor{currentstroke}{rgb}{0.850000,0.850000,0.850000}%
\pgfsetstrokecolor{currentstroke}%
\pgfsetdash{}{0pt}%
\pgfpathmoveto{\pgfqpoint{1.057526in}{0.417642in}}%
\pgfpathlineto{\pgfqpoint{1.057526in}{1.788330in}}%
\pgfusepath{stroke}%
\end{pgfscope}%
\begin{pgfscope}%
\pgfsetbuttcap%
\pgfsetroundjoin%
\definecolor{currentfill}{rgb}{0.000000,0.000000,0.000000}%
\pgfsetfillcolor{currentfill}%
\pgfsetlinewidth{0.602250pt}%
\definecolor{currentstroke}{rgb}{0.000000,0.000000,0.000000}%
\pgfsetstrokecolor{currentstroke}%
\pgfsetdash{}{0pt}%
\pgfsys@defobject{currentmarker}{\pgfqpoint{0.000000in}{-0.027778in}}{\pgfqpoint{0.000000in}{0.000000in}}{%
\pgfpathmoveto{\pgfqpoint{0.000000in}{0.000000in}}%
\pgfpathlineto{\pgfqpoint{0.000000in}{-0.027778in}}%
\pgfusepath{stroke,fill}%
}%
\begin{pgfscope}%
\pgfsys@transformshift{1.057526in}{0.417642in}%
\pgfsys@useobject{currentmarker}{}%
\end{pgfscope}%
\end{pgfscope}%
\begin{pgfscope}%
\pgfpathrectangle{\pgfqpoint{0.589510in}{0.417642in}}{\pgfqpoint{1.808820in}{1.370688in}}%
\pgfusepath{clip}%
\pgfsetrectcap%
\pgfsetroundjoin%
\pgfsetlinewidth{0.803000pt}%
\definecolor{currentstroke}{rgb}{0.850000,0.850000,0.850000}%
\pgfsetstrokecolor{currentstroke}%
\pgfsetdash{}{0pt}%
\pgfpathmoveto{\pgfqpoint{1.084000in}{0.417642in}}%
\pgfpathlineto{\pgfqpoint{1.084000in}{1.788330in}}%
\pgfusepath{stroke}%
\end{pgfscope}%
\begin{pgfscope}%
\pgfsetbuttcap%
\pgfsetroundjoin%
\definecolor{currentfill}{rgb}{0.000000,0.000000,0.000000}%
\pgfsetfillcolor{currentfill}%
\pgfsetlinewidth{0.602250pt}%
\definecolor{currentstroke}{rgb}{0.000000,0.000000,0.000000}%
\pgfsetstrokecolor{currentstroke}%
\pgfsetdash{}{0pt}%
\pgfsys@defobject{currentmarker}{\pgfqpoint{0.000000in}{-0.027778in}}{\pgfqpoint{0.000000in}{0.000000in}}{%
\pgfpathmoveto{\pgfqpoint{0.000000in}{0.000000in}}%
\pgfpathlineto{\pgfqpoint{0.000000in}{-0.027778in}}%
\pgfusepath{stroke,fill}%
}%
\begin{pgfscope}%
\pgfsys@transformshift{1.084000in}{0.417642in}%
\pgfsys@useobject{currentmarker}{}%
\end{pgfscope}%
\end{pgfscope}%
\begin{pgfscope}%
\pgfpathrectangle{\pgfqpoint{0.589510in}{0.417642in}}{\pgfqpoint{1.808820in}{1.370688in}}%
\pgfusepath{clip}%
\pgfsetrectcap%
\pgfsetroundjoin%
\pgfsetlinewidth{0.803000pt}%
\definecolor{currentstroke}{rgb}{0.850000,0.850000,0.850000}%
\pgfsetstrokecolor{currentstroke}%
\pgfsetdash{}{0pt}%
\pgfpathmoveto{\pgfqpoint{1.107352in}{0.417642in}}%
\pgfpathlineto{\pgfqpoint{1.107352in}{1.788330in}}%
\pgfusepath{stroke}%
\end{pgfscope}%
\begin{pgfscope}%
\pgfsetbuttcap%
\pgfsetroundjoin%
\definecolor{currentfill}{rgb}{0.000000,0.000000,0.000000}%
\pgfsetfillcolor{currentfill}%
\pgfsetlinewidth{0.602250pt}%
\definecolor{currentstroke}{rgb}{0.000000,0.000000,0.000000}%
\pgfsetstrokecolor{currentstroke}%
\pgfsetdash{}{0pt}%
\pgfsys@defobject{currentmarker}{\pgfqpoint{0.000000in}{-0.027778in}}{\pgfqpoint{0.000000in}{0.000000in}}{%
\pgfpathmoveto{\pgfqpoint{0.000000in}{0.000000in}}%
\pgfpathlineto{\pgfqpoint{0.000000in}{-0.027778in}}%
\pgfusepath{stroke,fill}%
}%
\begin{pgfscope}%
\pgfsys@transformshift{1.107352in}{0.417642in}%
\pgfsys@useobject{currentmarker}{}%
\end{pgfscope}%
\end{pgfscope}%
\begin{pgfscope}%
\pgfpathrectangle{\pgfqpoint{0.589510in}{0.417642in}}{\pgfqpoint{1.808820in}{1.370688in}}%
\pgfusepath{clip}%
\pgfsetrectcap%
\pgfsetroundjoin%
\pgfsetlinewidth{0.803000pt}%
\definecolor{currentstroke}{rgb}{0.850000,0.850000,0.850000}%
\pgfsetstrokecolor{currentstroke}%
\pgfsetdash{}{0pt}%
\pgfpathmoveto{\pgfqpoint{1.265664in}{0.417642in}}%
\pgfpathlineto{\pgfqpoint{1.265664in}{1.788330in}}%
\pgfusepath{stroke}%
\end{pgfscope}%
\begin{pgfscope}%
\pgfsetbuttcap%
\pgfsetroundjoin%
\definecolor{currentfill}{rgb}{0.000000,0.000000,0.000000}%
\pgfsetfillcolor{currentfill}%
\pgfsetlinewidth{0.602250pt}%
\definecolor{currentstroke}{rgb}{0.000000,0.000000,0.000000}%
\pgfsetstrokecolor{currentstroke}%
\pgfsetdash{}{0pt}%
\pgfsys@defobject{currentmarker}{\pgfqpoint{0.000000in}{-0.027778in}}{\pgfqpoint{0.000000in}{0.000000in}}{%
\pgfpathmoveto{\pgfqpoint{0.000000in}{0.000000in}}%
\pgfpathlineto{\pgfqpoint{0.000000in}{-0.027778in}}%
\pgfusepath{stroke,fill}%
}%
\begin{pgfscope}%
\pgfsys@transformshift{1.265664in}{0.417642in}%
\pgfsys@useobject{currentmarker}{}%
\end{pgfscope}%
\end{pgfscope}%
\begin{pgfscope}%
\pgfpathrectangle{\pgfqpoint{0.589510in}{0.417642in}}{\pgfqpoint{1.808820in}{1.370688in}}%
\pgfusepath{clip}%
\pgfsetrectcap%
\pgfsetroundjoin%
\pgfsetlinewidth{0.803000pt}%
\definecolor{currentstroke}{rgb}{0.850000,0.850000,0.850000}%
\pgfsetstrokecolor{currentstroke}%
\pgfsetdash{}{0pt}%
\pgfpathmoveto{\pgfqpoint{1.346052in}{0.417642in}}%
\pgfpathlineto{\pgfqpoint{1.346052in}{1.788330in}}%
\pgfusepath{stroke}%
\end{pgfscope}%
\begin{pgfscope}%
\pgfsetbuttcap%
\pgfsetroundjoin%
\definecolor{currentfill}{rgb}{0.000000,0.000000,0.000000}%
\pgfsetfillcolor{currentfill}%
\pgfsetlinewidth{0.602250pt}%
\definecolor{currentstroke}{rgb}{0.000000,0.000000,0.000000}%
\pgfsetstrokecolor{currentstroke}%
\pgfsetdash{}{0pt}%
\pgfsys@defobject{currentmarker}{\pgfqpoint{0.000000in}{-0.027778in}}{\pgfqpoint{0.000000in}{0.000000in}}{%
\pgfpathmoveto{\pgfqpoint{0.000000in}{0.000000in}}%
\pgfpathlineto{\pgfqpoint{0.000000in}{-0.027778in}}%
\pgfusepath{stroke,fill}%
}%
\begin{pgfscope}%
\pgfsys@transformshift{1.346052in}{0.417642in}%
\pgfsys@useobject{currentmarker}{}%
\end{pgfscope}%
\end{pgfscope}%
\begin{pgfscope}%
\pgfpathrectangle{\pgfqpoint{0.589510in}{0.417642in}}{\pgfqpoint{1.808820in}{1.370688in}}%
\pgfusepath{clip}%
\pgfsetrectcap%
\pgfsetroundjoin%
\pgfsetlinewidth{0.803000pt}%
\definecolor{currentstroke}{rgb}{0.850000,0.850000,0.850000}%
\pgfsetstrokecolor{currentstroke}%
\pgfsetdash{}{0pt}%
\pgfpathmoveto{\pgfqpoint{1.403088in}{0.417642in}}%
\pgfpathlineto{\pgfqpoint{1.403088in}{1.788330in}}%
\pgfusepath{stroke}%
\end{pgfscope}%
\begin{pgfscope}%
\pgfsetbuttcap%
\pgfsetroundjoin%
\definecolor{currentfill}{rgb}{0.000000,0.000000,0.000000}%
\pgfsetfillcolor{currentfill}%
\pgfsetlinewidth{0.602250pt}%
\definecolor{currentstroke}{rgb}{0.000000,0.000000,0.000000}%
\pgfsetstrokecolor{currentstroke}%
\pgfsetdash{}{0pt}%
\pgfsys@defobject{currentmarker}{\pgfqpoint{0.000000in}{-0.027778in}}{\pgfqpoint{0.000000in}{0.000000in}}{%
\pgfpathmoveto{\pgfqpoint{0.000000in}{0.000000in}}%
\pgfpathlineto{\pgfqpoint{0.000000in}{-0.027778in}}%
\pgfusepath{stroke,fill}%
}%
\begin{pgfscope}%
\pgfsys@transformshift{1.403088in}{0.417642in}%
\pgfsys@useobject{currentmarker}{}%
\end{pgfscope}%
\end{pgfscope}%
\begin{pgfscope}%
\pgfpathrectangle{\pgfqpoint{0.589510in}{0.417642in}}{\pgfqpoint{1.808820in}{1.370688in}}%
\pgfusepath{clip}%
\pgfsetrectcap%
\pgfsetroundjoin%
\pgfsetlinewidth{0.803000pt}%
\definecolor{currentstroke}{rgb}{0.850000,0.850000,0.850000}%
\pgfsetstrokecolor{currentstroke}%
\pgfsetdash{}{0pt}%
\pgfpathmoveto{\pgfqpoint{1.447328in}{0.417642in}}%
\pgfpathlineto{\pgfqpoint{1.447328in}{1.788330in}}%
\pgfusepath{stroke}%
\end{pgfscope}%
\begin{pgfscope}%
\pgfsetbuttcap%
\pgfsetroundjoin%
\definecolor{currentfill}{rgb}{0.000000,0.000000,0.000000}%
\pgfsetfillcolor{currentfill}%
\pgfsetlinewidth{0.602250pt}%
\definecolor{currentstroke}{rgb}{0.000000,0.000000,0.000000}%
\pgfsetstrokecolor{currentstroke}%
\pgfsetdash{}{0pt}%
\pgfsys@defobject{currentmarker}{\pgfqpoint{0.000000in}{-0.027778in}}{\pgfqpoint{0.000000in}{0.000000in}}{%
\pgfpathmoveto{\pgfqpoint{0.000000in}{0.000000in}}%
\pgfpathlineto{\pgfqpoint{0.000000in}{-0.027778in}}%
\pgfusepath{stroke,fill}%
}%
\begin{pgfscope}%
\pgfsys@transformshift{1.447328in}{0.417642in}%
\pgfsys@useobject{currentmarker}{}%
\end{pgfscope}%
\end{pgfscope}%
\begin{pgfscope}%
\pgfpathrectangle{\pgfqpoint{0.589510in}{0.417642in}}{\pgfqpoint{1.808820in}{1.370688in}}%
\pgfusepath{clip}%
\pgfsetrectcap%
\pgfsetroundjoin%
\pgfsetlinewidth{0.803000pt}%
\definecolor{currentstroke}{rgb}{0.850000,0.850000,0.850000}%
\pgfsetstrokecolor{currentstroke}%
\pgfsetdash{}{0pt}%
\pgfpathmoveto{\pgfqpoint{1.483475in}{0.417642in}}%
\pgfpathlineto{\pgfqpoint{1.483475in}{1.788330in}}%
\pgfusepath{stroke}%
\end{pgfscope}%
\begin{pgfscope}%
\pgfsetbuttcap%
\pgfsetroundjoin%
\definecolor{currentfill}{rgb}{0.000000,0.000000,0.000000}%
\pgfsetfillcolor{currentfill}%
\pgfsetlinewidth{0.602250pt}%
\definecolor{currentstroke}{rgb}{0.000000,0.000000,0.000000}%
\pgfsetstrokecolor{currentstroke}%
\pgfsetdash{}{0pt}%
\pgfsys@defobject{currentmarker}{\pgfqpoint{0.000000in}{-0.027778in}}{\pgfqpoint{0.000000in}{0.000000in}}{%
\pgfpathmoveto{\pgfqpoint{0.000000in}{0.000000in}}%
\pgfpathlineto{\pgfqpoint{0.000000in}{-0.027778in}}%
\pgfusepath{stroke,fill}%
}%
\begin{pgfscope}%
\pgfsys@transformshift{1.483475in}{0.417642in}%
\pgfsys@useobject{currentmarker}{}%
\end{pgfscope}%
\end{pgfscope}%
\begin{pgfscope}%
\pgfpathrectangle{\pgfqpoint{0.589510in}{0.417642in}}{\pgfqpoint{1.808820in}{1.370688in}}%
\pgfusepath{clip}%
\pgfsetrectcap%
\pgfsetroundjoin%
\pgfsetlinewidth{0.803000pt}%
\definecolor{currentstroke}{rgb}{0.850000,0.850000,0.850000}%
\pgfsetstrokecolor{currentstroke}%
\pgfsetdash{}{0pt}%
\pgfpathmoveto{\pgfqpoint{1.514037in}{0.417642in}}%
\pgfpathlineto{\pgfqpoint{1.514037in}{1.788330in}}%
\pgfusepath{stroke}%
\end{pgfscope}%
\begin{pgfscope}%
\pgfsetbuttcap%
\pgfsetroundjoin%
\definecolor{currentfill}{rgb}{0.000000,0.000000,0.000000}%
\pgfsetfillcolor{currentfill}%
\pgfsetlinewidth{0.602250pt}%
\definecolor{currentstroke}{rgb}{0.000000,0.000000,0.000000}%
\pgfsetstrokecolor{currentstroke}%
\pgfsetdash{}{0pt}%
\pgfsys@defobject{currentmarker}{\pgfqpoint{0.000000in}{-0.027778in}}{\pgfqpoint{0.000000in}{0.000000in}}{%
\pgfpathmoveto{\pgfqpoint{0.000000in}{0.000000in}}%
\pgfpathlineto{\pgfqpoint{0.000000in}{-0.027778in}}%
\pgfusepath{stroke,fill}%
}%
\begin{pgfscope}%
\pgfsys@transformshift{1.514037in}{0.417642in}%
\pgfsys@useobject{currentmarker}{}%
\end{pgfscope}%
\end{pgfscope}%
\begin{pgfscope}%
\pgfpathrectangle{\pgfqpoint{0.589510in}{0.417642in}}{\pgfqpoint{1.808820in}{1.370688in}}%
\pgfusepath{clip}%
\pgfsetrectcap%
\pgfsetroundjoin%
\pgfsetlinewidth{0.803000pt}%
\definecolor{currentstroke}{rgb}{0.850000,0.850000,0.850000}%
\pgfsetstrokecolor{currentstroke}%
\pgfsetdash{}{0pt}%
\pgfpathmoveto{\pgfqpoint{1.540511in}{0.417642in}}%
\pgfpathlineto{\pgfqpoint{1.540511in}{1.788330in}}%
\pgfusepath{stroke}%
\end{pgfscope}%
\begin{pgfscope}%
\pgfsetbuttcap%
\pgfsetroundjoin%
\definecolor{currentfill}{rgb}{0.000000,0.000000,0.000000}%
\pgfsetfillcolor{currentfill}%
\pgfsetlinewidth{0.602250pt}%
\definecolor{currentstroke}{rgb}{0.000000,0.000000,0.000000}%
\pgfsetstrokecolor{currentstroke}%
\pgfsetdash{}{0pt}%
\pgfsys@defobject{currentmarker}{\pgfqpoint{0.000000in}{-0.027778in}}{\pgfqpoint{0.000000in}{0.000000in}}{%
\pgfpathmoveto{\pgfqpoint{0.000000in}{0.000000in}}%
\pgfpathlineto{\pgfqpoint{0.000000in}{-0.027778in}}%
\pgfusepath{stroke,fill}%
}%
\begin{pgfscope}%
\pgfsys@transformshift{1.540511in}{0.417642in}%
\pgfsys@useobject{currentmarker}{}%
\end{pgfscope}%
\end{pgfscope}%
\begin{pgfscope}%
\pgfpathrectangle{\pgfqpoint{0.589510in}{0.417642in}}{\pgfqpoint{1.808820in}{1.370688in}}%
\pgfusepath{clip}%
\pgfsetrectcap%
\pgfsetroundjoin%
\pgfsetlinewidth{0.803000pt}%
\definecolor{currentstroke}{rgb}{0.850000,0.850000,0.850000}%
\pgfsetstrokecolor{currentstroke}%
\pgfsetdash{}{0pt}%
\pgfpathmoveto{\pgfqpoint{1.563863in}{0.417642in}}%
\pgfpathlineto{\pgfqpoint{1.563863in}{1.788330in}}%
\pgfusepath{stroke}%
\end{pgfscope}%
\begin{pgfscope}%
\pgfsetbuttcap%
\pgfsetroundjoin%
\definecolor{currentfill}{rgb}{0.000000,0.000000,0.000000}%
\pgfsetfillcolor{currentfill}%
\pgfsetlinewidth{0.602250pt}%
\definecolor{currentstroke}{rgb}{0.000000,0.000000,0.000000}%
\pgfsetstrokecolor{currentstroke}%
\pgfsetdash{}{0pt}%
\pgfsys@defobject{currentmarker}{\pgfqpoint{0.000000in}{-0.027778in}}{\pgfqpoint{0.000000in}{0.000000in}}{%
\pgfpathmoveto{\pgfqpoint{0.000000in}{0.000000in}}%
\pgfpathlineto{\pgfqpoint{0.000000in}{-0.027778in}}%
\pgfusepath{stroke,fill}%
}%
\begin{pgfscope}%
\pgfsys@transformshift{1.563863in}{0.417642in}%
\pgfsys@useobject{currentmarker}{}%
\end{pgfscope}%
\end{pgfscope}%
\begin{pgfscope}%
\pgfpathrectangle{\pgfqpoint{0.589510in}{0.417642in}}{\pgfqpoint{1.808820in}{1.370688in}}%
\pgfusepath{clip}%
\pgfsetrectcap%
\pgfsetroundjoin%
\pgfsetlinewidth{0.803000pt}%
\definecolor{currentstroke}{rgb}{0.850000,0.850000,0.850000}%
\pgfsetstrokecolor{currentstroke}%
\pgfsetdash{}{0pt}%
\pgfpathmoveto{\pgfqpoint{1.722176in}{0.417642in}}%
\pgfpathlineto{\pgfqpoint{1.722176in}{1.788330in}}%
\pgfusepath{stroke}%
\end{pgfscope}%
\begin{pgfscope}%
\pgfsetbuttcap%
\pgfsetroundjoin%
\definecolor{currentfill}{rgb}{0.000000,0.000000,0.000000}%
\pgfsetfillcolor{currentfill}%
\pgfsetlinewidth{0.602250pt}%
\definecolor{currentstroke}{rgb}{0.000000,0.000000,0.000000}%
\pgfsetstrokecolor{currentstroke}%
\pgfsetdash{}{0pt}%
\pgfsys@defobject{currentmarker}{\pgfqpoint{0.000000in}{-0.027778in}}{\pgfqpoint{0.000000in}{0.000000in}}{%
\pgfpathmoveto{\pgfqpoint{0.000000in}{0.000000in}}%
\pgfpathlineto{\pgfqpoint{0.000000in}{-0.027778in}}%
\pgfusepath{stroke,fill}%
}%
\begin{pgfscope}%
\pgfsys@transformshift{1.722176in}{0.417642in}%
\pgfsys@useobject{currentmarker}{}%
\end{pgfscope}%
\end{pgfscope}%
\begin{pgfscope}%
\pgfpathrectangle{\pgfqpoint{0.589510in}{0.417642in}}{\pgfqpoint{1.808820in}{1.370688in}}%
\pgfusepath{clip}%
\pgfsetrectcap%
\pgfsetroundjoin%
\pgfsetlinewidth{0.803000pt}%
\definecolor{currentstroke}{rgb}{0.850000,0.850000,0.850000}%
\pgfsetstrokecolor{currentstroke}%
\pgfsetdash{}{0pt}%
\pgfpathmoveto{\pgfqpoint{1.802563in}{0.417642in}}%
\pgfpathlineto{\pgfqpoint{1.802563in}{1.788330in}}%
\pgfusepath{stroke}%
\end{pgfscope}%
\begin{pgfscope}%
\pgfsetbuttcap%
\pgfsetroundjoin%
\definecolor{currentfill}{rgb}{0.000000,0.000000,0.000000}%
\pgfsetfillcolor{currentfill}%
\pgfsetlinewidth{0.602250pt}%
\definecolor{currentstroke}{rgb}{0.000000,0.000000,0.000000}%
\pgfsetstrokecolor{currentstroke}%
\pgfsetdash{}{0pt}%
\pgfsys@defobject{currentmarker}{\pgfqpoint{0.000000in}{-0.027778in}}{\pgfqpoint{0.000000in}{0.000000in}}{%
\pgfpathmoveto{\pgfqpoint{0.000000in}{0.000000in}}%
\pgfpathlineto{\pgfqpoint{0.000000in}{-0.027778in}}%
\pgfusepath{stroke,fill}%
}%
\begin{pgfscope}%
\pgfsys@transformshift{1.802563in}{0.417642in}%
\pgfsys@useobject{currentmarker}{}%
\end{pgfscope}%
\end{pgfscope}%
\begin{pgfscope}%
\pgfpathrectangle{\pgfqpoint{0.589510in}{0.417642in}}{\pgfqpoint{1.808820in}{1.370688in}}%
\pgfusepath{clip}%
\pgfsetrectcap%
\pgfsetroundjoin%
\pgfsetlinewidth{0.803000pt}%
\definecolor{currentstroke}{rgb}{0.850000,0.850000,0.850000}%
\pgfsetstrokecolor{currentstroke}%
\pgfsetdash{}{0pt}%
\pgfpathmoveto{\pgfqpoint{1.859599in}{0.417642in}}%
\pgfpathlineto{\pgfqpoint{1.859599in}{1.788330in}}%
\pgfusepath{stroke}%
\end{pgfscope}%
\begin{pgfscope}%
\pgfsetbuttcap%
\pgfsetroundjoin%
\definecolor{currentfill}{rgb}{0.000000,0.000000,0.000000}%
\pgfsetfillcolor{currentfill}%
\pgfsetlinewidth{0.602250pt}%
\definecolor{currentstroke}{rgb}{0.000000,0.000000,0.000000}%
\pgfsetstrokecolor{currentstroke}%
\pgfsetdash{}{0pt}%
\pgfsys@defobject{currentmarker}{\pgfqpoint{0.000000in}{-0.027778in}}{\pgfqpoint{0.000000in}{0.000000in}}{%
\pgfpathmoveto{\pgfqpoint{0.000000in}{0.000000in}}%
\pgfpathlineto{\pgfqpoint{0.000000in}{-0.027778in}}%
\pgfusepath{stroke,fill}%
}%
\begin{pgfscope}%
\pgfsys@transformshift{1.859599in}{0.417642in}%
\pgfsys@useobject{currentmarker}{}%
\end{pgfscope}%
\end{pgfscope}%
\begin{pgfscope}%
\pgfpathrectangle{\pgfqpoint{0.589510in}{0.417642in}}{\pgfqpoint{1.808820in}{1.370688in}}%
\pgfusepath{clip}%
\pgfsetrectcap%
\pgfsetroundjoin%
\pgfsetlinewidth{0.803000pt}%
\definecolor{currentstroke}{rgb}{0.850000,0.850000,0.850000}%
\pgfsetstrokecolor{currentstroke}%
\pgfsetdash{}{0pt}%
\pgfpathmoveto{\pgfqpoint{1.903840in}{0.417642in}}%
\pgfpathlineto{\pgfqpoint{1.903840in}{1.788330in}}%
\pgfusepath{stroke}%
\end{pgfscope}%
\begin{pgfscope}%
\pgfsetbuttcap%
\pgfsetroundjoin%
\definecolor{currentfill}{rgb}{0.000000,0.000000,0.000000}%
\pgfsetfillcolor{currentfill}%
\pgfsetlinewidth{0.602250pt}%
\definecolor{currentstroke}{rgb}{0.000000,0.000000,0.000000}%
\pgfsetstrokecolor{currentstroke}%
\pgfsetdash{}{0pt}%
\pgfsys@defobject{currentmarker}{\pgfqpoint{0.000000in}{-0.027778in}}{\pgfqpoint{0.000000in}{0.000000in}}{%
\pgfpathmoveto{\pgfqpoint{0.000000in}{0.000000in}}%
\pgfpathlineto{\pgfqpoint{0.000000in}{-0.027778in}}%
\pgfusepath{stroke,fill}%
}%
\begin{pgfscope}%
\pgfsys@transformshift{1.903840in}{0.417642in}%
\pgfsys@useobject{currentmarker}{}%
\end{pgfscope}%
\end{pgfscope}%
\begin{pgfscope}%
\pgfpathrectangle{\pgfqpoint{0.589510in}{0.417642in}}{\pgfqpoint{1.808820in}{1.370688in}}%
\pgfusepath{clip}%
\pgfsetrectcap%
\pgfsetroundjoin%
\pgfsetlinewidth{0.803000pt}%
\definecolor{currentstroke}{rgb}{0.850000,0.850000,0.850000}%
\pgfsetstrokecolor{currentstroke}%
\pgfsetdash{}{0pt}%
\pgfpathmoveto{\pgfqpoint{1.939987in}{0.417642in}}%
\pgfpathlineto{\pgfqpoint{1.939987in}{1.788330in}}%
\pgfusepath{stroke}%
\end{pgfscope}%
\begin{pgfscope}%
\pgfsetbuttcap%
\pgfsetroundjoin%
\definecolor{currentfill}{rgb}{0.000000,0.000000,0.000000}%
\pgfsetfillcolor{currentfill}%
\pgfsetlinewidth{0.602250pt}%
\definecolor{currentstroke}{rgb}{0.000000,0.000000,0.000000}%
\pgfsetstrokecolor{currentstroke}%
\pgfsetdash{}{0pt}%
\pgfsys@defobject{currentmarker}{\pgfqpoint{0.000000in}{-0.027778in}}{\pgfqpoint{0.000000in}{0.000000in}}{%
\pgfpathmoveto{\pgfqpoint{0.000000in}{0.000000in}}%
\pgfpathlineto{\pgfqpoint{0.000000in}{-0.027778in}}%
\pgfusepath{stroke,fill}%
}%
\begin{pgfscope}%
\pgfsys@transformshift{1.939987in}{0.417642in}%
\pgfsys@useobject{currentmarker}{}%
\end{pgfscope}%
\end{pgfscope}%
\begin{pgfscope}%
\pgfpathrectangle{\pgfqpoint{0.589510in}{0.417642in}}{\pgfqpoint{1.808820in}{1.370688in}}%
\pgfusepath{clip}%
\pgfsetrectcap%
\pgfsetroundjoin%
\pgfsetlinewidth{0.803000pt}%
\definecolor{currentstroke}{rgb}{0.850000,0.850000,0.850000}%
\pgfsetstrokecolor{currentstroke}%
\pgfsetdash{}{0pt}%
\pgfpathmoveto{\pgfqpoint{1.970549in}{0.417642in}}%
\pgfpathlineto{\pgfqpoint{1.970549in}{1.788330in}}%
\pgfusepath{stroke}%
\end{pgfscope}%
\begin{pgfscope}%
\pgfsetbuttcap%
\pgfsetroundjoin%
\definecolor{currentfill}{rgb}{0.000000,0.000000,0.000000}%
\pgfsetfillcolor{currentfill}%
\pgfsetlinewidth{0.602250pt}%
\definecolor{currentstroke}{rgb}{0.000000,0.000000,0.000000}%
\pgfsetstrokecolor{currentstroke}%
\pgfsetdash{}{0pt}%
\pgfsys@defobject{currentmarker}{\pgfqpoint{0.000000in}{-0.027778in}}{\pgfqpoint{0.000000in}{0.000000in}}{%
\pgfpathmoveto{\pgfqpoint{0.000000in}{0.000000in}}%
\pgfpathlineto{\pgfqpoint{0.000000in}{-0.027778in}}%
\pgfusepath{stroke,fill}%
}%
\begin{pgfscope}%
\pgfsys@transformshift{1.970549in}{0.417642in}%
\pgfsys@useobject{currentmarker}{}%
\end{pgfscope}%
\end{pgfscope}%
\begin{pgfscope}%
\pgfpathrectangle{\pgfqpoint{0.589510in}{0.417642in}}{\pgfqpoint{1.808820in}{1.370688in}}%
\pgfusepath{clip}%
\pgfsetrectcap%
\pgfsetroundjoin%
\pgfsetlinewidth{0.803000pt}%
\definecolor{currentstroke}{rgb}{0.850000,0.850000,0.850000}%
\pgfsetstrokecolor{currentstroke}%
\pgfsetdash{}{0pt}%
\pgfpathmoveto{\pgfqpoint{1.997023in}{0.417642in}}%
\pgfpathlineto{\pgfqpoint{1.997023in}{1.788330in}}%
\pgfusepath{stroke}%
\end{pgfscope}%
\begin{pgfscope}%
\pgfsetbuttcap%
\pgfsetroundjoin%
\definecolor{currentfill}{rgb}{0.000000,0.000000,0.000000}%
\pgfsetfillcolor{currentfill}%
\pgfsetlinewidth{0.602250pt}%
\definecolor{currentstroke}{rgb}{0.000000,0.000000,0.000000}%
\pgfsetstrokecolor{currentstroke}%
\pgfsetdash{}{0pt}%
\pgfsys@defobject{currentmarker}{\pgfqpoint{0.000000in}{-0.027778in}}{\pgfqpoint{0.000000in}{0.000000in}}{%
\pgfpathmoveto{\pgfqpoint{0.000000in}{0.000000in}}%
\pgfpathlineto{\pgfqpoint{0.000000in}{-0.027778in}}%
\pgfusepath{stroke,fill}%
}%
\begin{pgfscope}%
\pgfsys@transformshift{1.997023in}{0.417642in}%
\pgfsys@useobject{currentmarker}{}%
\end{pgfscope}%
\end{pgfscope}%
\begin{pgfscope}%
\pgfpathrectangle{\pgfqpoint{0.589510in}{0.417642in}}{\pgfqpoint{1.808820in}{1.370688in}}%
\pgfusepath{clip}%
\pgfsetrectcap%
\pgfsetroundjoin%
\pgfsetlinewidth{0.803000pt}%
\definecolor{currentstroke}{rgb}{0.850000,0.850000,0.850000}%
\pgfsetstrokecolor{currentstroke}%
\pgfsetdash{}{0pt}%
\pgfpathmoveto{\pgfqpoint{2.020375in}{0.417642in}}%
\pgfpathlineto{\pgfqpoint{2.020375in}{1.788330in}}%
\pgfusepath{stroke}%
\end{pgfscope}%
\begin{pgfscope}%
\pgfsetbuttcap%
\pgfsetroundjoin%
\definecolor{currentfill}{rgb}{0.000000,0.000000,0.000000}%
\pgfsetfillcolor{currentfill}%
\pgfsetlinewidth{0.602250pt}%
\definecolor{currentstroke}{rgb}{0.000000,0.000000,0.000000}%
\pgfsetstrokecolor{currentstroke}%
\pgfsetdash{}{0pt}%
\pgfsys@defobject{currentmarker}{\pgfqpoint{0.000000in}{-0.027778in}}{\pgfqpoint{0.000000in}{0.000000in}}{%
\pgfpathmoveto{\pgfqpoint{0.000000in}{0.000000in}}%
\pgfpathlineto{\pgfqpoint{0.000000in}{-0.027778in}}%
\pgfusepath{stroke,fill}%
}%
\begin{pgfscope}%
\pgfsys@transformshift{2.020375in}{0.417642in}%
\pgfsys@useobject{currentmarker}{}%
\end{pgfscope}%
\end{pgfscope}%
\begin{pgfscope}%
\pgfpathrectangle{\pgfqpoint{0.589510in}{0.417642in}}{\pgfqpoint{1.808820in}{1.370688in}}%
\pgfusepath{clip}%
\pgfsetrectcap%
\pgfsetroundjoin%
\pgfsetlinewidth{0.803000pt}%
\definecolor{currentstroke}{rgb}{0.850000,0.850000,0.850000}%
\pgfsetstrokecolor{currentstroke}%
\pgfsetdash{}{0pt}%
\pgfpathmoveto{\pgfqpoint{2.178687in}{0.417642in}}%
\pgfpathlineto{\pgfqpoint{2.178687in}{1.788330in}}%
\pgfusepath{stroke}%
\end{pgfscope}%
\begin{pgfscope}%
\pgfsetbuttcap%
\pgfsetroundjoin%
\definecolor{currentfill}{rgb}{0.000000,0.000000,0.000000}%
\pgfsetfillcolor{currentfill}%
\pgfsetlinewidth{0.602250pt}%
\definecolor{currentstroke}{rgb}{0.000000,0.000000,0.000000}%
\pgfsetstrokecolor{currentstroke}%
\pgfsetdash{}{0pt}%
\pgfsys@defobject{currentmarker}{\pgfqpoint{0.000000in}{-0.027778in}}{\pgfqpoint{0.000000in}{0.000000in}}{%
\pgfpathmoveto{\pgfqpoint{0.000000in}{0.000000in}}%
\pgfpathlineto{\pgfqpoint{0.000000in}{-0.027778in}}%
\pgfusepath{stroke,fill}%
}%
\begin{pgfscope}%
\pgfsys@transformshift{2.178687in}{0.417642in}%
\pgfsys@useobject{currentmarker}{}%
\end{pgfscope}%
\end{pgfscope}%
\begin{pgfscope}%
\pgfpathrectangle{\pgfqpoint{0.589510in}{0.417642in}}{\pgfqpoint{1.808820in}{1.370688in}}%
\pgfusepath{clip}%
\pgfsetrectcap%
\pgfsetroundjoin%
\pgfsetlinewidth{0.803000pt}%
\definecolor{currentstroke}{rgb}{0.850000,0.850000,0.850000}%
\pgfsetstrokecolor{currentstroke}%
\pgfsetdash{}{0pt}%
\pgfpathmoveto{\pgfqpoint{2.259075in}{0.417642in}}%
\pgfpathlineto{\pgfqpoint{2.259075in}{1.788330in}}%
\pgfusepath{stroke}%
\end{pgfscope}%
\begin{pgfscope}%
\pgfsetbuttcap%
\pgfsetroundjoin%
\definecolor{currentfill}{rgb}{0.000000,0.000000,0.000000}%
\pgfsetfillcolor{currentfill}%
\pgfsetlinewidth{0.602250pt}%
\definecolor{currentstroke}{rgb}{0.000000,0.000000,0.000000}%
\pgfsetstrokecolor{currentstroke}%
\pgfsetdash{}{0pt}%
\pgfsys@defobject{currentmarker}{\pgfqpoint{0.000000in}{-0.027778in}}{\pgfqpoint{0.000000in}{0.000000in}}{%
\pgfpathmoveto{\pgfqpoint{0.000000in}{0.000000in}}%
\pgfpathlineto{\pgfqpoint{0.000000in}{-0.027778in}}%
\pgfusepath{stroke,fill}%
}%
\begin{pgfscope}%
\pgfsys@transformshift{2.259075in}{0.417642in}%
\pgfsys@useobject{currentmarker}{}%
\end{pgfscope}%
\end{pgfscope}%
\begin{pgfscope}%
\pgfpathrectangle{\pgfqpoint{0.589510in}{0.417642in}}{\pgfqpoint{1.808820in}{1.370688in}}%
\pgfusepath{clip}%
\pgfsetrectcap%
\pgfsetroundjoin%
\pgfsetlinewidth{0.803000pt}%
\definecolor{currentstroke}{rgb}{0.850000,0.850000,0.850000}%
\pgfsetstrokecolor{currentstroke}%
\pgfsetdash{}{0pt}%
\pgfpathmoveto{\pgfqpoint{2.316111in}{0.417642in}}%
\pgfpathlineto{\pgfqpoint{2.316111in}{1.788330in}}%
\pgfusepath{stroke}%
\end{pgfscope}%
\begin{pgfscope}%
\pgfsetbuttcap%
\pgfsetroundjoin%
\definecolor{currentfill}{rgb}{0.000000,0.000000,0.000000}%
\pgfsetfillcolor{currentfill}%
\pgfsetlinewidth{0.602250pt}%
\definecolor{currentstroke}{rgb}{0.000000,0.000000,0.000000}%
\pgfsetstrokecolor{currentstroke}%
\pgfsetdash{}{0pt}%
\pgfsys@defobject{currentmarker}{\pgfqpoint{0.000000in}{-0.027778in}}{\pgfqpoint{0.000000in}{0.000000in}}{%
\pgfpathmoveto{\pgfqpoint{0.000000in}{0.000000in}}%
\pgfpathlineto{\pgfqpoint{0.000000in}{-0.027778in}}%
\pgfusepath{stroke,fill}%
}%
\begin{pgfscope}%
\pgfsys@transformshift{2.316111in}{0.417642in}%
\pgfsys@useobject{currentmarker}{}%
\end{pgfscope}%
\end{pgfscope}%
\begin{pgfscope}%
\pgfpathrectangle{\pgfqpoint{0.589510in}{0.417642in}}{\pgfqpoint{1.808820in}{1.370688in}}%
\pgfusepath{clip}%
\pgfsetrectcap%
\pgfsetroundjoin%
\pgfsetlinewidth{0.803000pt}%
\definecolor{currentstroke}{rgb}{0.850000,0.850000,0.850000}%
\pgfsetstrokecolor{currentstroke}%
\pgfsetdash{}{0pt}%
\pgfpathmoveto{\pgfqpoint{2.360351in}{0.417642in}}%
\pgfpathlineto{\pgfqpoint{2.360351in}{1.788330in}}%
\pgfusepath{stroke}%
\end{pgfscope}%
\begin{pgfscope}%
\pgfsetbuttcap%
\pgfsetroundjoin%
\definecolor{currentfill}{rgb}{0.000000,0.000000,0.000000}%
\pgfsetfillcolor{currentfill}%
\pgfsetlinewidth{0.602250pt}%
\definecolor{currentstroke}{rgb}{0.000000,0.000000,0.000000}%
\pgfsetstrokecolor{currentstroke}%
\pgfsetdash{}{0pt}%
\pgfsys@defobject{currentmarker}{\pgfqpoint{0.000000in}{-0.027778in}}{\pgfqpoint{0.000000in}{0.000000in}}{%
\pgfpathmoveto{\pgfqpoint{0.000000in}{0.000000in}}%
\pgfpathlineto{\pgfqpoint{0.000000in}{-0.027778in}}%
\pgfusepath{stroke,fill}%
}%
\begin{pgfscope}%
\pgfsys@transformshift{2.360351in}{0.417642in}%
\pgfsys@useobject{currentmarker}{}%
\end{pgfscope}%
\end{pgfscope}%
\begin{pgfscope}%
\pgfpathrectangle{\pgfqpoint{0.589510in}{0.417642in}}{\pgfqpoint{1.808820in}{1.370688in}}%
\pgfusepath{clip}%
\pgfsetrectcap%
\pgfsetroundjoin%
\pgfsetlinewidth{0.803000pt}%
\definecolor{currentstroke}{rgb}{0.850000,0.850000,0.850000}%
\pgfsetstrokecolor{currentstroke}%
\pgfsetdash{}{0pt}%
\pgfpathmoveto{\pgfqpoint{2.396499in}{0.417642in}}%
\pgfpathlineto{\pgfqpoint{2.396499in}{1.788330in}}%
\pgfusepath{stroke}%
\end{pgfscope}%
\begin{pgfscope}%
\pgfsetbuttcap%
\pgfsetroundjoin%
\definecolor{currentfill}{rgb}{0.000000,0.000000,0.000000}%
\pgfsetfillcolor{currentfill}%
\pgfsetlinewidth{0.602250pt}%
\definecolor{currentstroke}{rgb}{0.000000,0.000000,0.000000}%
\pgfsetstrokecolor{currentstroke}%
\pgfsetdash{}{0pt}%
\pgfsys@defobject{currentmarker}{\pgfqpoint{0.000000in}{-0.027778in}}{\pgfqpoint{0.000000in}{0.000000in}}{%
\pgfpathmoveto{\pgfqpoint{0.000000in}{0.000000in}}%
\pgfpathlineto{\pgfqpoint{0.000000in}{-0.027778in}}%
\pgfusepath{stroke,fill}%
}%
\begin{pgfscope}%
\pgfsys@transformshift{2.396499in}{0.417642in}%
\pgfsys@useobject{currentmarker}{}%
\end{pgfscope}%
\end{pgfscope}%
\begin{pgfscope}%
\definecolor{textcolor}{rgb}{0.000000,0.000000,0.000000}%
\pgfsetstrokecolor{textcolor}%
\pgfsetfillcolor{textcolor}%
\pgftext[x=1.493920in,y=0.165003in,,top]{\color{textcolor}\rmfamily\fontsize{10.000000}{12.000000}\selectfont \(\displaystyle \tau\) in \unit{\second}}%
\end{pgfscope}%
\begin{pgfscope}%
\pgfpathrectangle{\pgfqpoint{0.589510in}{0.417642in}}{\pgfqpoint{1.808820in}{1.370688in}}%
\pgfusepath{clip}%
\pgfsetrectcap%
\pgfsetroundjoin%
\pgfsetlinewidth{0.803000pt}%
\definecolor{currentstroke}{rgb}{0.450000,0.450000,0.450000}%
\pgfsetstrokecolor{currentstroke}%
\pgfsetdash{}{0pt}%
\pgfpathmoveto{\pgfqpoint{0.589510in}{0.417642in}}%
\pgfpathlineto{\pgfqpoint{2.398330in}{0.417642in}}%
\pgfusepath{stroke}%
\end{pgfscope}%
\begin{pgfscope}%
\pgfsetbuttcap%
\pgfsetroundjoin%
\definecolor{currentfill}{rgb}{0.000000,0.000000,0.000000}%
\pgfsetfillcolor{currentfill}%
\pgfsetlinewidth{0.803000pt}%
\definecolor{currentstroke}{rgb}{0.000000,0.000000,0.000000}%
\pgfsetstrokecolor{currentstroke}%
\pgfsetdash{}{0pt}%
\pgfsys@defobject{currentmarker}{\pgfqpoint{-0.048611in}{0.000000in}}{\pgfqpoint{-0.000000in}{0.000000in}}{%
\pgfpathmoveto{\pgfqpoint{-0.000000in}{0.000000in}}%
\pgfpathlineto{\pgfqpoint{-0.048611in}{0.000000in}}%
\pgfusepath{stroke,fill}%
}%
\begin{pgfscope}%
\pgfsys@transformshift{0.589510in}{0.417642in}%
\pgfsys@useobject{currentmarker}{}%
\end{pgfscope}%
\end{pgfscope}%
\begin{pgfscope}%
\definecolor{textcolor}{rgb}{0.000000,0.000000,0.000000}%
\pgfsetstrokecolor{textcolor}%
\pgfsetfillcolor{textcolor}%
\pgftext[x=0.236114in, y=0.378489in, left, base]{\color{textcolor}\rmfamily\fontsize{8.000000}{9.600000}\selectfont \(\displaystyle {10^{-2}}\)}%
\end{pgfscope}%
\begin{pgfscope}%
\pgfpathrectangle{\pgfqpoint{0.589510in}{0.417642in}}{\pgfqpoint{1.808820in}{1.370688in}}%
\pgfusepath{clip}%
\pgfsetrectcap%
\pgfsetroundjoin%
\pgfsetlinewidth{0.803000pt}%
\definecolor{currentstroke}{rgb}{0.450000,0.450000,0.450000}%
\pgfsetstrokecolor{currentstroke}%
\pgfsetdash{}{0pt}%
\pgfpathmoveto{\pgfqpoint{0.589510in}{0.826865in}}%
\pgfpathlineto{\pgfqpoint{2.398330in}{0.826865in}}%
\pgfusepath{stroke}%
\end{pgfscope}%
\begin{pgfscope}%
\pgfsetbuttcap%
\pgfsetroundjoin%
\definecolor{currentfill}{rgb}{0.000000,0.000000,0.000000}%
\pgfsetfillcolor{currentfill}%
\pgfsetlinewidth{0.803000pt}%
\definecolor{currentstroke}{rgb}{0.000000,0.000000,0.000000}%
\pgfsetstrokecolor{currentstroke}%
\pgfsetdash{}{0pt}%
\pgfsys@defobject{currentmarker}{\pgfqpoint{-0.048611in}{0.000000in}}{\pgfqpoint{-0.000000in}{0.000000in}}{%
\pgfpathmoveto{\pgfqpoint{-0.000000in}{0.000000in}}%
\pgfpathlineto{\pgfqpoint{-0.048611in}{0.000000in}}%
\pgfusepath{stroke,fill}%
}%
\begin{pgfscope}%
\pgfsys@transformshift{0.589510in}{0.826865in}%
\pgfsys@useobject{currentmarker}{}%
\end{pgfscope}%
\end{pgfscope}%
\begin{pgfscope}%
\definecolor{textcolor}{rgb}{0.000000,0.000000,0.000000}%
\pgfsetstrokecolor{textcolor}%
\pgfsetfillcolor{textcolor}%
\pgftext[x=0.316361in, y=0.787713in, left, base]{\color{textcolor}\rmfamily\fontsize{8.000000}{9.600000}\selectfont \(\displaystyle {10^{0}}\)}%
\end{pgfscope}%
\begin{pgfscope}%
\pgfpathrectangle{\pgfqpoint{0.589510in}{0.417642in}}{\pgfqpoint{1.808820in}{1.370688in}}%
\pgfusepath{clip}%
\pgfsetrectcap%
\pgfsetroundjoin%
\pgfsetlinewidth{0.803000pt}%
\definecolor{currentstroke}{rgb}{0.450000,0.450000,0.450000}%
\pgfsetstrokecolor{currentstroke}%
\pgfsetdash{}{0pt}%
\pgfpathmoveto{\pgfqpoint{0.589510in}{1.236089in}}%
\pgfpathlineto{\pgfqpoint{2.398330in}{1.236089in}}%
\pgfusepath{stroke}%
\end{pgfscope}%
\begin{pgfscope}%
\pgfsetbuttcap%
\pgfsetroundjoin%
\definecolor{currentfill}{rgb}{0.000000,0.000000,0.000000}%
\pgfsetfillcolor{currentfill}%
\pgfsetlinewidth{0.803000pt}%
\definecolor{currentstroke}{rgb}{0.000000,0.000000,0.000000}%
\pgfsetstrokecolor{currentstroke}%
\pgfsetdash{}{0pt}%
\pgfsys@defobject{currentmarker}{\pgfqpoint{-0.048611in}{0.000000in}}{\pgfqpoint{-0.000000in}{0.000000in}}{%
\pgfpathmoveto{\pgfqpoint{-0.000000in}{0.000000in}}%
\pgfpathlineto{\pgfqpoint{-0.048611in}{0.000000in}}%
\pgfusepath{stroke,fill}%
}%
\begin{pgfscope}%
\pgfsys@transformshift{0.589510in}{1.236089in}%
\pgfsys@useobject{currentmarker}{}%
\end{pgfscope}%
\end{pgfscope}%
\begin{pgfscope}%
\definecolor{textcolor}{rgb}{0.000000,0.000000,0.000000}%
\pgfsetstrokecolor{textcolor}%
\pgfsetfillcolor{textcolor}%
\pgftext[x=0.316361in, y=1.196936in, left, base]{\color{textcolor}\rmfamily\fontsize{8.000000}{9.600000}\selectfont \(\displaystyle {10^{2}}\)}%
\end{pgfscope}%
\begin{pgfscope}%
\pgfpathrectangle{\pgfqpoint{0.589510in}{0.417642in}}{\pgfqpoint{1.808820in}{1.370688in}}%
\pgfusepath{clip}%
\pgfsetrectcap%
\pgfsetroundjoin%
\pgfsetlinewidth{0.803000pt}%
\definecolor{currentstroke}{rgb}{0.450000,0.450000,0.450000}%
\pgfsetstrokecolor{currentstroke}%
\pgfsetdash{}{0pt}%
\pgfpathmoveto{\pgfqpoint{0.589510in}{1.645313in}}%
\pgfpathlineto{\pgfqpoint{2.398330in}{1.645313in}}%
\pgfusepath{stroke}%
\end{pgfscope}%
\begin{pgfscope}%
\pgfsetbuttcap%
\pgfsetroundjoin%
\definecolor{currentfill}{rgb}{0.000000,0.000000,0.000000}%
\pgfsetfillcolor{currentfill}%
\pgfsetlinewidth{0.803000pt}%
\definecolor{currentstroke}{rgb}{0.000000,0.000000,0.000000}%
\pgfsetstrokecolor{currentstroke}%
\pgfsetdash{}{0pt}%
\pgfsys@defobject{currentmarker}{\pgfqpoint{-0.048611in}{0.000000in}}{\pgfqpoint{-0.000000in}{0.000000in}}{%
\pgfpathmoveto{\pgfqpoint{-0.000000in}{0.000000in}}%
\pgfpathlineto{\pgfqpoint{-0.048611in}{0.000000in}}%
\pgfusepath{stroke,fill}%
}%
\begin{pgfscope}%
\pgfsys@transformshift{0.589510in}{1.645313in}%
\pgfsys@useobject{currentmarker}{}%
\end{pgfscope}%
\end{pgfscope}%
\begin{pgfscope}%
\definecolor{textcolor}{rgb}{0.000000,0.000000,0.000000}%
\pgfsetstrokecolor{textcolor}%
\pgfsetfillcolor{textcolor}%
\pgftext[x=0.316361in, y=1.606160in, left, base]{\color{textcolor}\rmfamily\fontsize{8.000000}{9.600000}\selectfont \(\displaystyle {10^{4}}\)}%
\end{pgfscope}%
\begin{pgfscope}%
\definecolor{textcolor}{rgb}{0.000000,0.000000,0.000000}%
\pgfsetstrokecolor{textcolor}%
\pgfsetfillcolor{textcolor}%
\pgftext[x=0.180559in,y=1.102986in,,bottom,rotate=90.000000]{\color{textcolor}\rmfamily\fontsize{10.000000}{12.000000}\selectfont ADEV \(\displaystyle \sigma_A(\tau)\)}%
\end{pgfscope}%
\begin{pgfscope}%
\pgfpathrectangle{\pgfqpoint{0.589510in}{0.417642in}}{\pgfqpoint{1.808820in}{1.370688in}}%
\pgfusepath{clip}%
\pgfsetbuttcap%
\pgfsetroundjoin%
\pgfsetlinewidth{1.505625pt}%
\definecolor{currentstroke}{rgb}{0.003922,0.450980,0.698039}%
\pgfsetstrokecolor{currentstroke}%
\pgfsetdash{{5.550000pt}{2.400000pt}}{0.000000pt}%
\pgfpathmoveto{\pgfqpoint{0.671729in}{0.826865in}}%
\pgfpathlineto{\pgfqpoint{0.809153in}{0.796068in}}%
\pgfpathlineto{\pgfqpoint{0.946576in}{0.765271in}}%
\pgfpathlineto{\pgfqpoint{1.128240in}{0.724560in}}%
\pgfpathlineto{\pgfqpoint{1.265664in}{0.693762in}}%
\pgfpathlineto{\pgfqpoint{1.403088in}{0.662965in}}%
\pgfpathlineto{\pgfqpoint{1.584752in}{0.622254in}}%
\pgfpathlineto{\pgfqpoint{1.722176in}{0.591457in}}%
\pgfpathlineto{\pgfqpoint{1.859599in}{0.560659in}}%
\pgfpathlineto{\pgfqpoint{2.041264in}{0.519948in}}%
\pgfpathlineto{\pgfqpoint{2.178687in}{0.489151in}}%
\pgfpathlineto{\pgfqpoint{2.316111in}{0.458354in}}%
\pgfusepath{stroke}%
\end{pgfscope}%
\begin{pgfscope}%
\pgfpathrectangle{\pgfqpoint{0.589510in}{0.417642in}}{\pgfqpoint{1.808820in}{1.370688in}}%
\pgfusepath{clip}%
\pgfsetbuttcap%
\pgfsetroundjoin%
\definecolor{currentfill}{rgb}{0.003922,0.450980,0.698039}%
\pgfsetfillcolor{currentfill}%
\pgfsetlinewidth{1.003750pt}%
\definecolor{currentstroke}{rgb}{0.003922,0.450980,0.698039}%
\pgfsetstrokecolor{currentstroke}%
\pgfsetdash{}{0pt}%
\pgfsys@defobject{currentmarker}{\pgfqpoint{-0.020833in}{-0.020833in}}{\pgfqpoint{0.020833in}{0.020833in}}{%
\pgfpathmoveto{\pgfqpoint{0.000000in}{-0.020833in}}%
\pgfpathcurveto{\pgfqpoint{0.005525in}{-0.020833in}}{\pgfqpoint{0.010825in}{-0.018638in}}{\pgfqpoint{0.014731in}{-0.014731in}}%
\pgfpathcurveto{\pgfqpoint{0.018638in}{-0.010825in}}{\pgfqpoint{0.020833in}{-0.005525in}}{\pgfqpoint{0.020833in}{0.000000in}}%
\pgfpathcurveto{\pgfqpoint{0.020833in}{0.005525in}}{\pgfqpoint{0.018638in}{0.010825in}}{\pgfqpoint{0.014731in}{0.014731in}}%
\pgfpathcurveto{\pgfqpoint{0.010825in}{0.018638in}}{\pgfqpoint{0.005525in}{0.020833in}}{\pgfqpoint{0.000000in}{0.020833in}}%
\pgfpathcurveto{\pgfqpoint{-0.005525in}{0.020833in}}{\pgfqpoint{-0.010825in}{0.018638in}}{\pgfqpoint{-0.014731in}{0.014731in}}%
\pgfpathcurveto{\pgfqpoint{-0.018638in}{0.010825in}}{\pgfqpoint{-0.020833in}{0.005525in}}{\pgfqpoint{-0.020833in}{0.000000in}}%
\pgfpathcurveto{\pgfqpoint{-0.020833in}{-0.005525in}}{\pgfqpoint{-0.018638in}{-0.010825in}}{\pgfqpoint{-0.014731in}{-0.014731in}}%
\pgfpathcurveto{\pgfqpoint{-0.010825in}{-0.018638in}}{\pgfqpoint{-0.005525in}{-0.020833in}}{\pgfqpoint{0.000000in}{-0.020833in}}%
\pgfpathlineto{\pgfqpoint{0.000000in}{-0.020833in}}%
\pgfpathclose%
\pgfusepath{stroke,fill}%
}%
\begin{pgfscope}%
\pgfsys@transformshift{0.671729in}{0.827492in}%
\pgfsys@useobject{currentmarker}{}%
\end{pgfscope}%
\begin{pgfscope}%
\pgfsys@transformshift{0.809153in}{0.796721in}%
\pgfsys@useobject{currentmarker}{}%
\end{pgfscope}%
\begin{pgfscope}%
\pgfsys@transformshift{0.946576in}{0.765394in}%
\pgfsys@useobject{currentmarker}{}%
\end{pgfscope}%
\begin{pgfscope}%
\pgfsys@transformshift{1.128240in}{0.722837in}%
\pgfsys@useobject{currentmarker}{}%
\end{pgfscope}%
\begin{pgfscope}%
\pgfsys@transformshift{1.265664in}{0.689068in}%
\pgfsys@useobject{currentmarker}{}%
\end{pgfscope}%
\begin{pgfscope}%
\pgfsys@transformshift{1.403088in}{0.662183in}%
\pgfsys@useobject{currentmarker}{}%
\end{pgfscope}%
\begin{pgfscope}%
\pgfsys@transformshift{1.584752in}{0.624431in}%
\pgfsys@useobject{currentmarker}{}%
\end{pgfscope}%
\begin{pgfscope}%
\pgfsys@transformshift{1.722176in}{0.589132in}%
\pgfsys@useobject{currentmarker}{}%
\end{pgfscope}%
\begin{pgfscope}%
\pgfsys@transformshift{1.859599in}{0.544283in}%
\pgfsys@useobject{currentmarker}{}%
\end{pgfscope}%
\begin{pgfscope}%
\pgfsys@transformshift{2.041264in}{0.497009in}%
\pgfsys@useobject{currentmarker}{}%
\end{pgfscope}%
\begin{pgfscope}%
\pgfsys@transformshift{2.178687in}{0.507181in}%
\pgfsys@useobject{currentmarker}{}%
\end{pgfscope}%
\begin{pgfscope}%
\pgfsys@transformshift{2.316111in}{0.455299in}%
\pgfsys@useobject{currentmarker}{}%
\end{pgfscope}%
\end{pgfscope}%
\begin{pgfscope}%
\pgfsetrectcap%
\pgfsetmiterjoin%
\pgfsetlinewidth{0.803000pt}%
\definecolor{currentstroke}{rgb}{0.000000,0.000000,0.000000}%
\pgfsetstrokecolor{currentstroke}%
\pgfsetdash{}{0pt}%
\pgfpathmoveto{\pgfqpoint{0.589510in}{0.417642in}}%
\pgfpathlineto{\pgfqpoint{0.589510in}{1.788330in}}%
\pgfusepath{stroke}%
\end{pgfscope}%
\begin{pgfscope}%
\pgfsetrectcap%
\pgfsetmiterjoin%
\pgfsetlinewidth{0.803000pt}%
\definecolor{currentstroke}{rgb}{0.000000,0.000000,0.000000}%
\pgfsetstrokecolor{currentstroke}%
\pgfsetdash{}{0pt}%
\pgfpathmoveto{\pgfqpoint{2.398330in}{0.417642in}}%
\pgfpathlineto{\pgfqpoint{2.398330in}{1.788330in}}%
\pgfusepath{stroke}%
\end{pgfscope}%
\begin{pgfscope}%
\pgfsetrectcap%
\pgfsetmiterjoin%
\pgfsetlinewidth{0.803000pt}%
\definecolor{currentstroke}{rgb}{0.000000,0.000000,0.000000}%
\pgfsetstrokecolor{currentstroke}%
\pgfsetdash{}{0pt}%
\pgfpathmoveto{\pgfqpoint{0.589510in}{0.417642in}}%
\pgfpathlineto{\pgfqpoint{2.398330in}{0.417642in}}%
\pgfusepath{stroke}%
\end{pgfscope}%
\begin{pgfscope}%
\pgfsetrectcap%
\pgfsetmiterjoin%
\pgfsetlinewidth{0.803000pt}%
\definecolor{currentstroke}{rgb}{0.000000,0.000000,0.000000}%
\pgfsetstrokecolor{currentstroke}%
\pgfsetdash{}{0pt}%
\pgfpathmoveto{\pgfqpoint{0.589510in}{1.788330in}}%
\pgfpathlineto{\pgfqpoint{2.398330in}{1.788330in}}%
\pgfusepath{stroke}%
\end{pgfscope}%
\begin{pgfscope}%
\pgfsetbuttcap%
\pgfsetmiterjoin%
\definecolor{currentfill}{rgb}{1.000000,1.000000,1.000000}%
\pgfsetfillcolor{currentfill}%
\pgfsetfillopacity{0.800000}%
\pgfsetlinewidth{1.003750pt}%
\definecolor{currentstroke}{rgb}{0.800000,0.800000,0.800000}%
\pgfsetstrokecolor{currentstroke}%
\pgfsetstrokeopacity{0.800000}%
\pgfsetdash{}{0pt}%
\pgfpathmoveto{\pgfqpoint{1.289694in}{1.471662in}}%
\pgfpathlineto{\pgfqpoint{2.320552in}{1.471662in}}%
\pgfpathquadraticcurveto{\pgfqpoint{2.342774in}{1.471662in}}{\pgfqpoint{2.342774in}{1.493884in}}%
\pgfpathlineto{\pgfqpoint{2.342774in}{1.710552in}}%
\pgfpathquadraticcurveto{\pgfqpoint{2.342774in}{1.732774in}}{\pgfqpoint{2.320552in}{1.732774in}}%
\pgfpathlineto{\pgfqpoint{1.289694in}{1.732774in}}%
\pgfpathquadraticcurveto{\pgfqpoint{1.267472in}{1.732774in}}{\pgfqpoint{1.267472in}{1.710552in}}%
\pgfpathlineto{\pgfqpoint{1.267472in}{1.493884in}}%
\pgfpathquadraticcurveto{\pgfqpoint{1.267472in}{1.471662in}}{\pgfqpoint{1.289694in}{1.471662in}}%
\pgfpathlineto{\pgfqpoint{1.289694in}{1.471662in}}%
\pgfpathclose%
\pgfusepath{stroke,fill}%
\end{pgfscope}%
\begin{pgfscope}%
\pgfsetbuttcap%
\pgfsetroundjoin%
\pgfsetlinewidth{1.505625pt}%
\definecolor{currentstroke}{rgb}{0.003922,0.450980,0.698039}%
\pgfsetstrokecolor{currentstroke}%
\pgfsetdash{{5.550000pt}{2.400000pt}}{0.000000pt}%
\pgfpathmoveto{\pgfqpoint{1.311916in}{1.595930in}}%
\pgfpathlineto{\pgfqpoint{1.423028in}{1.595930in}}%
\pgfpathlineto{\pgfqpoint{1.534139in}{1.595930in}}%
\pgfusepath{stroke}%
\end{pgfscope}%
\begin{pgfscope}%
\definecolor{textcolor}{rgb}{0.000000,0.000000,0.000000}%
\pgfsetstrokecolor{textcolor}%
\pgfsetfillcolor{textcolor}%
\pgftext[x=1.623028in,y=1.557041in,left,base]{\color{textcolor}\rmfamily\fontsize{8.000000}{9.600000}\selectfont \(\displaystyle \propto\sqrt{h_{0}}\tau^{-0.5}\)}%
\end{pgfscope}%
\end{pgfpicture}%
\makeatother%
\endgroup%

        } % scalebox
        \caption{Allan deviation}
        \label{fig:white_noise_adev}
    \end{subfigure}
    \caption{Different representations of white noise.}
    \label{fig:white_noise_simulated}
\end{figure}

From this simulation, several features can be observed. First of all, the power spectral density is flat and constant with $h_0 = 2$, which is in accordance with table \ref{tab:adev_alpha} and the normalization mentioned earlier. Figure \ref{fig:white_noise_adev} shows the typical $\tau^{-\frac 1 2}$ dependence of white noise in the Allan deviation plot. This immediately explains, why filtering white noise scales with $\frac{1}{\sqrt{n}}$ with $n$ being the number of samples averaged.

\clearpage
\subsubsection{Burst Noise}
\label{sec:theory_burst_noise}
Burst noise, popcorn noise, or sometimes referred to as random telegraph signal is a random bi-stable change in a signal and is caused by a generation recombination processes. This, for example, happens in semiconductors if there is a site, that can trap an electrons for a prologned period of time and then randomly release it. Imporities causing lattice defects are discussed in this context \cite{kay2012operational,burst_noise_psd,popcorn_noise_orgin,technote_ti_popcorn_noise}. Such latttice defects can also be introduced by ion implantation during doping. Fortunately, this type of noise has become less prevalent in modern manufacturing processes, because the quality of the semiconductors has improved. But if a trap site is located very close to an important structure, for example a high precision Zener diode, its effect might be so strong, that it can be clearly seen.

The discussion is split into two parts. First the power spectral density is calculated and then the Allan variance is caclulated using that result.

The spectral density of burst noise caused by a single trap site was derived in \cite{burst_noise_wiener_khinchin} by \citeauthor{burst_noise_wiener_khinchin}. The author used the autocorrelation function of the burst noise signal and applied the Wiener-Khinchin (Wiener-Хи́нчин) theorem, which connects the autocorrelation function with the power spectral density. A more detailed derivation can be found in \cite{fundamentals_of_noise_processes}, in this paper the preconditions, like stationarity of the process, are also discussed. The burst noise signal consists of two energie levels, called $0$ and $1$, split by $\Delta y$. Multiple burst noise signals can be superimposed in a real device. This would then result in mutiple levels, but they can be treated separately. The measurement interval over an even number of transitions, so that one ends in the same state as the measurement has started, is the time $T$. The mean lifetime of the levels is called $\bar \tau_0$ and $\bar \tau_1$:
\begin{equation}
    \bar \tau_{0} \approx \frac 1 N \sum_{i}^N \tau_{0,i} \qquad \bar \tau_{1} \approx \frac 1 N \sum_{i}^N \tau_{1,i}
\end{equation}

Figure \ref{fig:burst_noise} shows a burst noise signal along with the definitions above.

\begin{figure}[hb]
    \centering
    \scalebox{1}{%
        \import{figures/}{burst_noise.tex}
    } % scalebox
    \caption{A random burst noise signal.}
    \label{fig:burst_noise}
\end{figure}

Using these definitions, one can then derive \cite{burst_noise_wiener_khinchin}:
\begin{align}
    R_{xx} (T) &= \Delta y^2 \cdot \frac{\bar \tau_1 \bar \tau_0 e^{-\left(\frac{1}{\bar \tau_1}+\frac{1}{\bar \tau_0}\right)T}}{\left(\bar \tau_1 + \bar \tau_0\right)^2} \quad \text{and} \label{eqn:burst_noise_correlation}\\
    S(\omega) &= 4 R_{xx}(0) \frac{\frac{1}{\bar \tau_1} + \frac{1}{\bar \tau_0}}{\left(\frac{1}{\bar \tau_1} + \frac{1}{\bar \tau_0}\right)^2 + \omega^2} \qquad \omega > 0 . \label{eqn:burst_noise_psd}
\end{align}
Note, that the power spectral density is the one-sided version, hence an additional factor of $2$ is included. The d.c. term was ommitted here and can usually be neglected, because it is not relevant for calculating the power spectral density as it only contributes a single peak at $\omega=0$. Using the following definitions of the average time constant and the duty cycle

\begin{align}
    \frac{1}{\bar \tau} &= \frac{1}{\bar \tau_1} + \frac{1}{\bar \tau_0} \quad \mathrm{and} \label{eqn:definition_bar_tau}\\
    D_i &= \frac{\bar \tau_i}{\bar \tau_1 + \bar \tau_0} \quad i \in \{0 ; 1\}
\end{align}

equations \ref{eqn:burst_noise_correlation} and \ref{eqn:burst_noise_psd} can be rewritten to give a more intuitive form:

\begin{align}
    R_{xx} (T) &= \Delta y^2 D_1 D_0 \, e^{-\left(\frac{1}{\bar \tau_1}+\frac{1}{\bar \tau_0}\right)T}\\
    S(\omega) &= 4 R_{xx}(0) \frac{\bar \tau}{1 + \omega^2 \bar \tau^2} \label{eqn:burst_noise_lorentzian}
\end{align}

The special case $\bar \tau_0 = \bar \tau_1$ with $D_i=\frac 1 2$ is the previously mentioned case of random telegraph noise.

$R_{xx} (0)$ can be identified as the mean squared value of $y$:
\begin{equation}
    y_{RMS} = \sqrt{R_{xx}(0)} \,.
\end{equation}

Equation \ref{eqn:burst_noise_lorentzian} is a Lorentzian function and from this it can be easily seen, that a single trap site has a power spectral density, which is proportional to $\frac{1}{f^2}$ at high frequencies and is flat at low frequencies.

With the spectral density in hand, it is now possible to calculate the Allan variance as it was done by \citeauthor{allen_dev_flicker} in \cite{allen_dev_flicker} for the classic example of random telegraph noise where $\bar \tau_1 = \bar \tau_0$. Do note, that table I given by \citeauthor{allen_dev_flicker} shows the total number of events instead of the instantationous number of events typically given. Hence, their notation must be multiplied by $\frac{1}{\tau^2}$ (or $\frac{1}{T^2}$ in their notation). For the generic case with $\bar \tau_1$, $\bar \tau_0$ and the definition of $\bar \tau$ given in equation \ref{eqn:definition_bar_tau} one finds for the Allan variance of burst noise:
\begin{equation}
    \sigma^2_A(\tau) = R_{xx}(0) \frac{\bar \tau^2}{\tau^2} \left(4 e^{-\frac{\tau}{\bar \tau}} - e^{-\frac{2 \tau}{\bar \tau}} + 2 \frac{\tau}{\bar \tau} - 3 \right) \label{eqn:burst_noise_avar}
\end{equation}

Having arrived at equations \ref{eqn:burst_noise_lorentzian} and \ref{eqn:burst_noise_avar} of the power spectral density and Allan variance, it it now possible to model it. For this purpose, parts of the Python library \textit{qtt} \cite{qtt} was used. The algorithm written by \citeauthor{qtt} implements continous-time Markov chains to simulate the burst noise signal. The result can be see in figure \ref{fig:burst_noise_simulated}. For these simulations one time constant, namely the lifetime of the lower state $\bar \tau_0$ was held constant, while the lifetime of the upper state was varied to show the effect of different $\bar \tau$. By looking at the time domain in figure \ref{fig:burst_noise_time} it can be seen, that the maximum average number of state changes can be observed, when $\bar \tau_1 = \bar \tau_0$. If $\bar \tau_1 > \bar \tau_0$ the system will favour the upper, while if $\bar \tau_1 < \bar \tau_0$ it will favour the lower state instead. This explaines why the noise is strongest for random telegraph noise when $\bar \tau_1 = \bar \tau_0$, which can also be seen in power spectral density in figure \ref{fig:burst_noise_psd}. Looking at the Allan deviation in figure \ref{fig:burst_noise_adev} confirms this, but also shows another interesting implication as it shows an obvious maximum. If the application allows a choice over the sampling interval $\tau$, the effect of the burst noise can mitigated by staying well clear of the maximum.

The small deviation from the analytical solution in figure \ref{fig:burst_noise_adev}  at large $\tau$ is a typical so called end-of-data error. As it was discussed above, the Allan deviation can only be estimated given a limited number of samples using equation \ref{eqn:adev_estimator} and going to longer $\tau$ means there are fewer samples to average over.

\begin{figure}[ht]
    \centering
    \begin{subfigure}{0.8\linewidth}
        \centering
        \scalebox{1}{%
            %% Creator: Matplotlib, PGF backend
%%
%% To include the figure in your LaTeX document, write
%%   \input{<filename>.pgf}
%%
%% Make sure the required packages are loaded in your preamble
%%   \usepackage{pgf}
%%
%% Also ensure that all the required font packages are loaded; for instance,
%% the lmodern package is sometimes necessary when using math font.
%%   \usepackage{lmodern}
%%
%% Figures using additional raster images can only be included by \input if
%% they are in the same directory as the main LaTeX file. For loading figures
%% from other directories you can use the `import` package
%%   \usepackage{import}
%%
%% and then include the figures with
%%   \import{<path to file>}{<filename>.pgf}
%%
%% Matplotlib used the following preamble
%%   \usepackage{siunitx}
%%   \usepackage{fontspec}
%%
\begingroup%
\makeatletter%
\begin{pgfpicture}%
\pgfpathrectangle{\pgfpointorigin}{\pgfqpoint{4.060000in}{2.510000in}}%
\pgfusepath{use as bounding box, clip}%
\begin{pgfscope}%
\pgfsetbuttcap%
\pgfsetmiterjoin%
\definecolor{currentfill}{rgb}{1.000000,1.000000,1.000000}%
\pgfsetfillcolor{currentfill}%
\pgfsetlinewidth{0.000000pt}%
\definecolor{currentstroke}{rgb}{1.000000,1.000000,1.000000}%
\pgfsetstrokecolor{currentstroke}%
\pgfsetdash{}{0pt}%
\pgfpathmoveto{\pgfqpoint{0.000000in}{0.000000in}}%
\pgfpathlineto{\pgfqpoint{4.060000in}{0.000000in}}%
\pgfpathlineto{\pgfqpoint{4.060000in}{2.510000in}}%
\pgfpathlineto{\pgfqpoint{0.000000in}{2.510000in}}%
\pgfpathlineto{\pgfqpoint{0.000000in}{0.000000in}}%
\pgfpathclose%
\pgfusepath{fill}%
\end{pgfscope}%
\begin{pgfscope}%
\pgfsetbuttcap%
\pgfsetmiterjoin%
\definecolor{currentfill}{rgb}{1.000000,1.000000,1.000000}%
\pgfsetfillcolor{currentfill}%
\pgfsetlinewidth{0.000000pt}%
\definecolor{currentstroke}{rgb}{0.000000,0.000000,0.000000}%
\pgfsetstrokecolor{currentstroke}%
\pgfsetstrokeopacity{0.000000}%
\pgfsetdash{}{0pt}%
\pgfpathmoveto{\pgfqpoint{0.471687in}{0.416447in}}%
\pgfpathlineto{\pgfqpoint{4.018330in}{0.416447in}}%
\pgfpathlineto{\pgfqpoint{4.018330in}{2.468330in}}%
\pgfpathlineto{\pgfqpoint{0.471687in}{2.468330in}}%
\pgfpathlineto{\pgfqpoint{0.471687in}{0.416447in}}%
\pgfpathclose%
\pgfusepath{fill}%
\end{pgfscope}%
\begin{pgfscope}%
\pgfpathrectangle{\pgfqpoint{0.471687in}{0.416447in}}{\pgfqpoint{3.546642in}{2.051883in}}%
\pgfusepath{clip}%
\pgfsetrectcap%
\pgfsetroundjoin%
\pgfsetlinewidth{0.803000pt}%
\definecolor{currentstroke}{rgb}{0.450000,0.450000,0.450000}%
\pgfsetstrokecolor{currentstroke}%
\pgfsetdash{}{0pt}%
\pgfpathmoveto{\pgfqpoint{0.632899in}{0.416447in}}%
\pgfpathlineto{\pgfqpoint{0.632899in}{2.468330in}}%
\pgfusepath{stroke}%
\end{pgfscope}%
\begin{pgfscope}%
\pgfsetbuttcap%
\pgfsetroundjoin%
\definecolor{currentfill}{rgb}{0.000000,0.000000,0.000000}%
\pgfsetfillcolor{currentfill}%
\pgfsetlinewidth{0.803000pt}%
\definecolor{currentstroke}{rgb}{0.000000,0.000000,0.000000}%
\pgfsetstrokecolor{currentstroke}%
\pgfsetdash{}{0pt}%
\pgfsys@defobject{currentmarker}{\pgfqpoint{0.000000in}{-0.048611in}}{\pgfqpoint{0.000000in}{0.000000in}}{%
\pgfpathmoveto{\pgfqpoint{0.000000in}{0.000000in}}%
\pgfpathlineto{\pgfqpoint{0.000000in}{-0.048611in}}%
\pgfusepath{stroke,fill}%
}%
\begin{pgfscope}%
\pgfsys@transformshift{0.632899in}{0.416447in}%
\pgfsys@useobject{currentmarker}{}%
\end{pgfscope}%
\end{pgfscope}%
\begin{pgfscope}%
\definecolor{textcolor}{rgb}{0.000000,0.000000,0.000000}%
\pgfsetstrokecolor{textcolor}%
\pgfsetfillcolor{textcolor}%
\pgftext[x=0.632899in,y=0.319225in,,top]{\color{textcolor}\rmfamily\fontsize{8.000000}{9.600000}\selectfont \(\displaystyle {0}\)}%
\end{pgfscope}%
\begin{pgfscope}%
\pgfpathrectangle{\pgfqpoint{0.471687in}{0.416447in}}{\pgfqpoint{3.546642in}{2.051883in}}%
\pgfusepath{clip}%
\pgfsetrectcap%
\pgfsetroundjoin%
\pgfsetlinewidth{0.803000pt}%
\definecolor{currentstroke}{rgb}{0.450000,0.450000,0.450000}%
\pgfsetstrokecolor{currentstroke}%
\pgfsetdash{}{0pt}%
\pgfpathmoveto{\pgfqpoint{1.278065in}{0.416447in}}%
\pgfpathlineto{\pgfqpoint{1.278065in}{2.468330in}}%
\pgfusepath{stroke}%
\end{pgfscope}%
\begin{pgfscope}%
\pgfsetbuttcap%
\pgfsetroundjoin%
\definecolor{currentfill}{rgb}{0.000000,0.000000,0.000000}%
\pgfsetfillcolor{currentfill}%
\pgfsetlinewidth{0.803000pt}%
\definecolor{currentstroke}{rgb}{0.000000,0.000000,0.000000}%
\pgfsetstrokecolor{currentstroke}%
\pgfsetdash{}{0pt}%
\pgfsys@defobject{currentmarker}{\pgfqpoint{0.000000in}{-0.048611in}}{\pgfqpoint{0.000000in}{0.000000in}}{%
\pgfpathmoveto{\pgfqpoint{0.000000in}{0.000000in}}%
\pgfpathlineto{\pgfqpoint{0.000000in}{-0.048611in}}%
\pgfusepath{stroke,fill}%
}%
\begin{pgfscope}%
\pgfsys@transformshift{1.278065in}{0.416447in}%
\pgfsys@useobject{currentmarker}{}%
\end{pgfscope}%
\end{pgfscope}%
\begin{pgfscope}%
\definecolor{textcolor}{rgb}{0.000000,0.000000,0.000000}%
\pgfsetstrokecolor{textcolor}%
\pgfsetfillcolor{textcolor}%
\pgftext[x=1.278065in,y=0.319225in,,top]{\color{textcolor}\rmfamily\fontsize{8.000000}{9.600000}\selectfont \(\displaystyle {2}\)}%
\end{pgfscope}%
\begin{pgfscope}%
\pgfpathrectangle{\pgfqpoint{0.471687in}{0.416447in}}{\pgfqpoint{3.546642in}{2.051883in}}%
\pgfusepath{clip}%
\pgfsetrectcap%
\pgfsetroundjoin%
\pgfsetlinewidth{0.803000pt}%
\definecolor{currentstroke}{rgb}{0.450000,0.450000,0.450000}%
\pgfsetstrokecolor{currentstroke}%
\pgfsetdash{}{0pt}%
\pgfpathmoveto{\pgfqpoint{1.923232in}{0.416447in}}%
\pgfpathlineto{\pgfqpoint{1.923232in}{2.468330in}}%
\pgfusepath{stroke}%
\end{pgfscope}%
\begin{pgfscope}%
\pgfsetbuttcap%
\pgfsetroundjoin%
\definecolor{currentfill}{rgb}{0.000000,0.000000,0.000000}%
\pgfsetfillcolor{currentfill}%
\pgfsetlinewidth{0.803000pt}%
\definecolor{currentstroke}{rgb}{0.000000,0.000000,0.000000}%
\pgfsetstrokecolor{currentstroke}%
\pgfsetdash{}{0pt}%
\pgfsys@defobject{currentmarker}{\pgfqpoint{0.000000in}{-0.048611in}}{\pgfqpoint{0.000000in}{0.000000in}}{%
\pgfpathmoveto{\pgfqpoint{0.000000in}{0.000000in}}%
\pgfpathlineto{\pgfqpoint{0.000000in}{-0.048611in}}%
\pgfusepath{stroke,fill}%
}%
\begin{pgfscope}%
\pgfsys@transformshift{1.923232in}{0.416447in}%
\pgfsys@useobject{currentmarker}{}%
\end{pgfscope}%
\end{pgfscope}%
\begin{pgfscope}%
\definecolor{textcolor}{rgb}{0.000000,0.000000,0.000000}%
\pgfsetstrokecolor{textcolor}%
\pgfsetfillcolor{textcolor}%
\pgftext[x=1.923232in,y=0.319225in,,top]{\color{textcolor}\rmfamily\fontsize{8.000000}{9.600000}\selectfont \(\displaystyle {4}\)}%
\end{pgfscope}%
\begin{pgfscope}%
\pgfpathrectangle{\pgfqpoint{0.471687in}{0.416447in}}{\pgfqpoint{3.546642in}{2.051883in}}%
\pgfusepath{clip}%
\pgfsetrectcap%
\pgfsetroundjoin%
\pgfsetlinewidth{0.803000pt}%
\definecolor{currentstroke}{rgb}{0.450000,0.450000,0.450000}%
\pgfsetstrokecolor{currentstroke}%
\pgfsetdash{}{0pt}%
\pgfpathmoveto{\pgfqpoint{2.568399in}{0.416447in}}%
\pgfpathlineto{\pgfqpoint{2.568399in}{2.468330in}}%
\pgfusepath{stroke}%
\end{pgfscope}%
\begin{pgfscope}%
\pgfsetbuttcap%
\pgfsetroundjoin%
\definecolor{currentfill}{rgb}{0.000000,0.000000,0.000000}%
\pgfsetfillcolor{currentfill}%
\pgfsetlinewidth{0.803000pt}%
\definecolor{currentstroke}{rgb}{0.000000,0.000000,0.000000}%
\pgfsetstrokecolor{currentstroke}%
\pgfsetdash{}{0pt}%
\pgfsys@defobject{currentmarker}{\pgfqpoint{0.000000in}{-0.048611in}}{\pgfqpoint{0.000000in}{0.000000in}}{%
\pgfpathmoveto{\pgfqpoint{0.000000in}{0.000000in}}%
\pgfpathlineto{\pgfqpoint{0.000000in}{-0.048611in}}%
\pgfusepath{stroke,fill}%
}%
\begin{pgfscope}%
\pgfsys@transformshift{2.568399in}{0.416447in}%
\pgfsys@useobject{currentmarker}{}%
\end{pgfscope}%
\end{pgfscope}%
\begin{pgfscope}%
\definecolor{textcolor}{rgb}{0.000000,0.000000,0.000000}%
\pgfsetstrokecolor{textcolor}%
\pgfsetfillcolor{textcolor}%
\pgftext[x=2.568399in,y=0.319225in,,top]{\color{textcolor}\rmfamily\fontsize{8.000000}{9.600000}\selectfont \(\displaystyle {6}\)}%
\end{pgfscope}%
\begin{pgfscope}%
\pgfpathrectangle{\pgfqpoint{0.471687in}{0.416447in}}{\pgfqpoint{3.546642in}{2.051883in}}%
\pgfusepath{clip}%
\pgfsetrectcap%
\pgfsetroundjoin%
\pgfsetlinewidth{0.803000pt}%
\definecolor{currentstroke}{rgb}{0.450000,0.450000,0.450000}%
\pgfsetstrokecolor{currentstroke}%
\pgfsetdash{}{0pt}%
\pgfpathmoveto{\pgfqpoint{3.213565in}{0.416447in}}%
\pgfpathlineto{\pgfqpoint{3.213565in}{2.468330in}}%
\pgfusepath{stroke}%
\end{pgfscope}%
\begin{pgfscope}%
\pgfsetbuttcap%
\pgfsetroundjoin%
\definecolor{currentfill}{rgb}{0.000000,0.000000,0.000000}%
\pgfsetfillcolor{currentfill}%
\pgfsetlinewidth{0.803000pt}%
\definecolor{currentstroke}{rgb}{0.000000,0.000000,0.000000}%
\pgfsetstrokecolor{currentstroke}%
\pgfsetdash{}{0pt}%
\pgfsys@defobject{currentmarker}{\pgfqpoint{0.000000in}{-0.048611in}}{\pgfqpoint{0.000000in}{0.000000in}}{%
\pgfpathmoveto{\pgfqpoint{0.000000in}{0.000000in}}%
\pgfpathlineto{\pgfqpoint{0.000000in}{-0.048611in}}%
\pgfusepath{stroke,fill}%
}%
\begin{pgfscope}%
\pgfsys@transformshift{3.213565in}{0.416447in}%
\pgfsys@useobject{currentmarker}{}%
\end{pgfscope}%
\end{pgfscope}%
\begin{pgfscope}%
\definecolor{textcolor}{rgb}{0.000000,0.000000,0.000000}%
\pgfsetstrokecolor{textcolor}%
\pgfsetfillcolor{textcolor}%
\pgftext[x=3.213565in,y=0.319225in,,top]{\color{textcolor}\rmfamily\fontsize{8.000000}{9.600000}\selectfont \(\displaystyle {8}\)}%
\end{pgfscope}%
\begin{pgfscope}%
\pgfpathrectangle{\pgfqpoint{0.471687in}{0.416447in}}{\pgfqpoint{3.546642in}{2.051883in}}%
\pgfusepath{clip}%
\pgfsetrectcap%
\pgfsetroundjoin%
\pgfsetlinewidth{0.803000pt}%
\definecolor{currentstroke}{rgb}{0.450000,0.450000,0.450000}%
\pgfsetstrokecolor{currentstroke}%
\pgfsetdash{}{0pt}%
\pgfpathmoveto{\pgfqpoint{3.858732in}{0.416447in}}%
\pgfpathlineto{\pgfqpoint{3.858732in}{2.468330in}}%
\pgfusepath{stroke}%
\end{pgfscope}%
\begin{pgfscope}%
\pgfsetbuttcap%
\pgfsetroundjoin%
\definecolor{currentfill}{rgb}{0.000000,0.000000,0.000000}%
\pgfsetfillcolor{currentfill}%
\pgfsetlinewidth{0.803000pt}%
\definecolor{currentstroke}{rgb}{0.000000,0.000000,0.000000}%
\pgfsetstrokecolor{currentstroke}%
\pgfsetdash{}{0pt}%
\pgfsys@defobject{currentmarker}{\pgfqpoint{0.000000in}{-0.048611in}}{\pgfqpoint{0.000000in}{0.000000in}}{%
\pgfpathmoveto{\pgfqpoint{0.000000in}{0.000000in}}%
\pgfpathlineto{\pgfqpoint{0.000000in}{-0.048611in}}%
\pgfusepath{stroke,fill}%
}%
\begin{pgfscope}%
\pgfsys@transformshift{3.858732in}{0.416447in}%
\pgfsys@useobject{currentmarker}{}%
\end{pgfscope}%
\end{pgfscope}%
\begin{pgfscope}%
\definecolor{textcolor}{rgb}{0.000000,0.000000,0.000000}%
\pgfsetstrokecolor{textcolor}%
\pgfsetfillcolor{textcolor}%
\pgftext[x=3.858732in,y=0.319225in,,top]{\color{textcolor}\rmfamily\fontsize{8.000000}{9.600000}\selectfont \(\displaystyle {10}\)}%
\end{pgfscope}%
\begin{pgfscope}%
\definecolor{textcolor}{rgb}{0.000000,0.000000,0.000000}%
\pgfsetstrokecolor{textcolor}%
\pgfsetfillcolor{textcolor}%
\pgftext[x=2.245009in,y=0.165003in,,top]{\color{textcolor}\rmfamily\fontsize{10.000000}{12.000000}\selectfont Time in \unit{\second}}%
\end{pgfscope}%
\begin{pgfscope}%
\pgfpathrectangle{\pgfqpoint{0.471687in}{0.416447in}}{\pgfqpoint{3.546642in}{2.051883in}}%
\pgfusepath{clip}%
\pgfsetrectcap%
\pgfsetroundjoin%
\pgfsetlinewidth{0.803000pt}%
\definecolor{currentstroke}{rgb}{0.450000,0.450000,0.450000}%
\pgfsetstrokecolor{currentstroke}%
\pgfsetdash{}{0pt}%
\pgfpathmoveto{\pgfqpoint{0.471687in}{0.509715in}}%
\pgfpathlineto{\pgfqpoint{4.018330in}{0.509715in}}%
\pgfusepath{stroke}%
\end{pgfscope}%
\begin{pgfscope}%
\pgfsetbuttcap%
\pgfsetroundjoin%
\definecolor{currentfill}{rgb}{0.000000,0.000000,0.000000}%
\pgfsetfillcolor{currentfill}%
\pgfsetlinewidth{0.803000pt}%
\definecolor{currentstroke}{rgb}{0.000000,0.000000,0.000000}%
\pgfsetstrokecolor{currentstroke}%
\pgfsetdash{}{0pt}%
\pgfsys@defobject{currentmarker}{\pgfqpoint{-0.048611in}{0.000000in}}{\pgfqpoint{-0.000000in}{0.000000in}}{%
\pgfpathmoveto{\pgfqpoint{-0.000000in}{0.000000in}}%
\pgfpathlineto{\pgfqpoint{-0.048611in}{0.000000in}}%
\pgfusepath{stroke,fill}%
}%
\begin{pgfscope}%
\pgfsys@transformshift{0.471687in}{0.509715in}%
\pgfsys@useobject{currentmarker}{}%
\end{pgfscope}%
\end{pgfscope}%
\begin{pgfscope}%
\definecolor{textcolor}{rgb}{0.000000,0.000000,0.000000}%
\pgfsetstrokecolor{textcolor}%
\pgfsetfillcolor{textcolor}%
\pgftext[x=0.223614in, y=0.471159in, left, base]{\color{textcolor}\rmfamily\fontsize{8.000000}{9.600000}\selectfont \(\displaystyle {0.0}\)}%
\end{pgfscope}%
\begin{pgfscope}%
\pgfpathrectangle{\pgfqpoint{0.471687in}{0.416447in}}{\pgfqpoint{3.546642in}{2.051883in}}%
\pgfusepath{clip}%
\pgfsetrectcap%
\pgfsetroundjoin%
\pgfsetlinewidth{0.803000pt}%
\definecolor{currentstroke}{rgb}{0.450000,0.450000,0.450000}%
\pgfsetstrokecolor{currentstroke}%
\pgfsetdash{}{0pt}%
\pgfpathmoveto{\pgfqpoint{0.471687in}{0.820606in}}%
\pgfpathlineto{\pgfqpoint{4.018330in}{0.820606in}}%
\pgfusepath{stroke}%
\end{pgfscope}%
\begin{pgfscope}%
\pgfsetbuttcap%
\pgfsetroundjoin%
\definecolor{currentfill}{rgb}{0.000000,0.000000,0.000000}%
\pgfsetfillcolor{currentfill}%
\pgfsetlinewidth{0.803000pt}%
\definecolor{currentstroke}{rgb}{0.000000,0.000000,0.000000}%
\pgfsetstrokecolor{currentstroke}%
\pgfsetdash{}{0pt}%
\pgfsys@defobject{currentmarker}{\pgfqpoint{-0.048611in}{0.000000in}}{\pgfqpoint{-0.000000in}{0.000000in}}{%
\pgfpathmoveto{\pgfqpoint{-0.000000in}{0.000000in}}%
\pgfpathlineto{\pgfqpoint{-0.048611in}{0.000000in}}%
\pgfusepath{stroke,fill}%
}%
\begin{pgfscope}%
\pgfsys@transformshift{0.471687in}{0.820606in}%
\pgfsys@useobject{currentmarker}{}%
\end{pgfscope}%
\end{pgfscope}%
\begin{pgfscope}%
\definecolor{textcolor}{rgb}{0.000000,0.000000,0.000000}%
\pgfsetstrokecolor{textcolor}%
\pgfsetfillcolor{textcolor}%
\pgftext[x=0.223614in, y=0.782051in, left, base]{\color{textcolor}\rmfamily\fontsize{8.000000}{9.600000}\selectfont \(\displaystyle {0.5}\)}%
\end{pgfscope}%
\begin{pgfscope}%
\pgfpathrectangle{\pgfqpoint{0.471687in}{0.416447in}}{\pgfqpoint{3.546642in}{2.051883in}}%
\pgfusepath{clip}%
\pgfsetrectcap%
\pgfsetroundjoin%
\pgfsetlinewidth{0.803000pt}%
\definecolor{currentstroke}{rgb}{0.450000,0.450000,0.450000}%
\pgfsetstrokecolor{currentstroke}%
\pgfsetdash{}{0pt}%
\pgfpathmoveto{\pgfqpoint{0.471687in}{1.131497in}}%
\pgfpathlineto{\pgfqpoint{4.018330in}{1.131497in}}%
\pgfusepath{stroke}%
\end{pgfscope}%
\begin{pgfscope}%
\pgfsetbuttcap%
\pgfsetroundjoin%
\definecolor{currentfill}{rgb}{0.000000,0.000000,0.000000}%
\pgfsetfillcolor{currentfill}%
\pgfsetlinewidth{0.803000pt}%
\definecolor{currentstroke}{rgb}{0.000000,0.000000,0.000000}%
\pgfsetstrokecolor{currentstroke}%
\pgfsetdash{}{0pt}%
\pgfsys@defobject{currentmarker}{\pgfqpoint{-0.048611in}{0.000000in}}{\pgfqpoint{-0.000000in}{0.000000in}}{%
\pgfpathmoveto{\pgfqpoint{-0.000000in}{0.000000in}}%
\pgfpathlineto{\pgfqpoint{-0.048611in}{0.000000in}}%
\pgfusepath{stroke,fill}%
}%
\begin{pgfscope}%
\pgfsys@transformshift{0.471687in}{1.131497in}%
\pgfsys@useobject{currentmarker}{}%
\end{pgfscope}%
\end{pgfscope}%
\begin{pgfscope}%
\definecolor{textcolor}{rgb}{0.000000,0.000000,0.000000}%
\pgfsetstrokecolor{textcolor}%
\pgfsetfillcolor{textcolor}%
\pgftext[x=0.223614in, y=1.092942in, left, base]{\color{textcolor}\rmfamily\fontsize{8.000000}{9.600000}\selectfont \(\displaystyle {1.0}\)}%
\end{pgfscope}%
\begin{pgfscope}%
\pgfpathrectangle{\pgfqpoint{0.471687in}{0.416447in}}{\pgfqpoint{3.546642in}{2.051883in}}%
\pgfusepath{clip}%
\pgfsetrectcap%
\pgfsetroundjoin%
\pgfsetlinewidth{0.803000pt}%
\definecolor{currentstroke}{rgb}{0.450000,0.450000,0.450000}%
\pgfsetstrokecolor{currentstroke}%
\pgfsetdash{}{0pt}%
\pgfpathmoveto{\pgfqpoint{0.471687in}{1.442389in}}%
\pgfpathlineto{\pgfqpoint{4.018330in}{1.442389in}}%
\pgfusepath{stroke}%
\end{pgfscope}%
\begin{pgfscope}%
\pgfsetbuttcap%
\pgfsetroundjoin%
\definecolor{currentfill}{rgb}{0.000000,0.000000,0.000000}%
\pgfsetfillcolor{currentfill}%
\pgfsetlinewidth{0.803000pt}%
\definecolor{currentstroke}{rgb}{0.000000,0.000000,0.000000}%
\pgfsetstrokecolor{currentstroke}%
\pgfsetdash{}{0pt}%
\pgfsys@defobject{currentmarker}{\pgfqpoint{-0.048611in}{0.000000in}}{\pgfqpoint{-0.000000in}{0.000000in}}{%
\pgfpathmoveto{\pgfqpoint{-0.000000in}{0.000000in}}%
\pgfpathlineto{\pgfqpoint{-0.048611in}{0.000000in}}%
\pgfusepath{stroke,fill}%
}%
\begin{pgfscope}%
\pgfsys@transformshift{0.471687in}{1.442389in}%
\pgfsys@useobject{currentmarker}{}%
\end{pgfscope}%
\end{pgfscope}%
\begin{pgfscope}%
\definecolor{textcolor}{rgb}{0.000000,0.000000,0.000000}%
\pgfsetstrokecolor{textcolor}%
\pgfsetfillcolor{textcolor}%
\pgftext[x=0.223614in, y=1.403833in, left, base]{\color{textcolor}\rmfamily\fontsize{8.000000}{9.600000}\selectfont \(\displaystyle {1.5}\)}%
\end{pgfscope}%
\begin{pgfscope}%
\pgfpathrectangle{\pgfqpoint{0.471687in}{0.416447in}}{\pgfqpoint{3.546642in}{2.051883in}}%
\pgfusepath{clip}%
\pgfsetrectcap%
\pgfsetroundjoin%
\pgfsetlinewidth{0.803000pt}%
\definecolor{currentstroke}{rgb}{0.450000,0.450000,0.450000}%
\pgfsetstrokecolor{currentstroke}%
\pgfsetdash{}{0pt}%
\pgfpathmoveto{\pgfqpoint{0.471687in}{1.753280in}}%
\pgfpathlineto{\pgfqpoint{4.018330in}{1.753280in}}%
\pgfusepath{stroke}%
\end{pgfscope}%
\begin{pgfscope}%
\pgfsetbuttcap%
\pgfsetroundjoin%
\definecolor{currentfill}{rgb}{0.000000,0.000000,0.000000}%
\pgfsetfillcolor{currentfill}%
\pgfsetlinewidth{0.803000pt}%
\definecolor{currentstroke}{rgb}{0.000000,0.000000,0.000000}%
\pgfsetstrokecolor{currentstroke}%
\pgfsetdash{}{0pt}%
\pgfsys@defobject{currentmarker}{\pgfqpoint{-0.048611in}{0.000000in}}{\pgfqpoint{-0.000000in}{0.000000in}}{%
\pgfpathmoveto{\pgfqpoint{-0.000000in}{0.000000in}}%
\pgfpathlineto{\pgfqpoint{-0.048611in}{0.000000in}}%
\pgfusepath{stroke,fill}%
}%
\begin{pgfscope}%
\pgfsys@transformshift{0.471687in}{1.753280in}%
\pgfsys@useobject{currentmarker}{}%
\end{pgfscope}%
\end{pgfscope}%
\begin{pgfscope}%
\definecolor{textcolor}{rgb}{0.000000,0.000000,0.000000}%
\pgfsetstrokecolor{textcolor}%
\pgfsetfillcolor{textcolor}%
\pgftext[x=0.223614in, y=1.714724in, left, base]{\color{textcolor}\rmfamily\fontsize{8.000000}{9.600000}\selectfont \(\displaystyle {2.0}\)}%
\end{pgfscope}%
\begin{pgfscope}%
\pgfpathrectangle{\pgfqpoint{0.471687in}{0.416447in}}{\pgfqpoint{3.546642in}{2.051883in}}%
\pgfusepath{clip}%
\pgfsetrectcap%
\pgfsetroundjoin%
\pgfsetlinewidth{0.803000pt}%
\definecolor{currentstroke}{rgb}{0.450000,0.450000,0.450000}%
\pgfsetstrokecolor{currentstroke}%
\pgfsetdash{}{0pt}%
\pgfpathmoveto{\pgfqpoint{0.471687in}{2.064171in}}%
\pgfpathlineto{\pgfqpoint{4.018330in}{2.064171in}}%
\pgfusepath{stroke}%
\end{pgfscope}%
\begin{pgfscope}%
\pgfsetbuttcap%
\pgfsetroundjoin%
\definecolor{currentfill}{rgb}{0.000000,0.000000,0.000000}%
\pgfsetfillcolor{currentfill}%
\pgfsetlinewidth{0.803000pt}%
\definecolor{currentstroke}{rgb}{0.000000,0.000000,0.000000}%
\pgfsetstrokecolor{currentstroke}%
\pgfsetdash{}{0pt}%
\pgfsys@defobject{currentmarker}{\pgfqpoint{-0.048611in}{0.000000in}}{\pgfqpoint{-0.000000in}{0.000000in}}{%
\pgfpathmoveto{\pgfqpoint{-0.000000in}{0.000000in}}%
\pgfpathlineto{\pgfqpoint{-0.048611in}{0.000000in}}%
\pgfusepath{stroke,fill}%
}%
\begin{pgfscope}%
\pgfsys@transformshift{0.471687in}{2.064171in}%
\pgfsys@useobject{currentmarker}{}%
\end{pgfscope}%
\end{pgfscope}%
\begin{pgfscope}%
\definecolor{textcolor}{rgb}{0.000000,0.000000,0.000000}%
\pgfsetstrokecolor{textcolor}%
\pgfsetfillcolor{textcolor}%
\pgftext[x=0.223614in, y=2.025616in, left, base]{\color{textcolor}\rmfamily\fontsize{8.000000}{9.600000}\selectfont \(\displaystyle {2.5}\)}%
\end{pgfscope}%
\begin{pgfscope}%
\pgfpathrectangle{\pgfqpoint{0.471687in}{0.416447in}}{\pgfqpoint{3.546642in}{2.051883in}}%
\pgfusepath{clip}%
\pgfsetrectcap%
\pgfsetroundjoin%
\pgfsetlinewidth{0.803000pt}%
\definecolor{currentstroke}{rgb}{0.450000,0.450000,0.450000}%
\pgfsetstrokecolor{currentstroke}%
\pgfsetdash{}{0pt}%
\pgfpathmoveto{\pgfqpoint{0.471687in}{2.375063in}}%
\pgfpathlineto{\pgfqpoint{4.018330in}{2.375063in}}%
\pgfusepath{stroke}%
\end{pgfscope}%
\begin{pgfscope}%
\pgfsetbuttcap%
\pgfsetroundjoin%
\definecolor{currentfill}{rgb}{0.000000,0.000000,0.000000}%
\pgfsetfillcolor{currentfill}%
\pgfsetlinewidth{0.803000pt}%
\definecolor{currentstroke}{rgb}{0.000000,0.000000,0.000000}%
\pgfsetstrokecolor{currentstroke}%
\pgfsetdash{}{0pt}%
\pgfsys@defobject{currentmarker}{\pgfqpoint{-0.048611in}{0.000000in}}{\pgfqpoint{-0.000000in}{0.000000in}}{%
\pgfpathmoveto{\pgfqpoint{-0.000000in}{0.000000in}}%
\pgfpathlineto{\pgfqpoint{-0.048611in}{0.000000in}}%
\pgfusepath{stroke,fill}%
}%
\begin{pgfscope}%
\pgfsys@transformshift{0.471687in}{2.375063in}%
\pgfsys@useobject{currentmarker}{}%
\end{pgfscope}%
\end{pgfscope}%
\begin{pgfscope}%
\definecolor{textcolor}{rgb}{0.000000,0.000000,0.000000}%
\pgfsetstrokecolor{textcolor}%
\pgfsetfillcolor{textcolor}%
\pgftext[x=0.223614in, y=2.336507in, left, base]{\color{textcolor}\rmfamily\fontsize{8.000000}{9.600000}\selectfont \(\displaystyle {3.0}\)}%
\end{pgfscope}%
\begin{pgfscope}%
\definecolor{textcolor}{rgb}{0.000000,0.000000,0.000000}%
\pgfsetstrokecolor{textcolor}%
\pgfsetfillcolor{textcolor}%
\pgftext[x=0.168059in,y=1.442389in,,bottom,rotate=90.000000]{\color{textcolor}\rmfamily\fontsize{10.000000}{12.000000}\selectfont Amplitude in arb. unit}%
\end{pgfscope}%
\begin{pgfscope}%
\pgfpathrectangle{\pgfqpoint{0.471687in}{0.416447in}}{\pgfqpoint{3.546642in}{2.051883in}}%
\pgfusepath{clip}%
\pgfsetrectcap%
\pgfsetroundjoin%
\pgfsetlinewidth{1.505625pt}%
\definecolor{currentstroke}{rgb}{0.000000,0.447059,0.698039}%
\pgfsetstrokecolor{currentstroke}%
\pgfsetdash{}{0pt}%
\pgfpathmoveto{\pgfqpoint{0.632899in}{0.509715in}}%
\pgfpathlineto{\pgfqpoint{0.736125in}{0.509715in}}%
\pgfpathlineto{\pgfqpoint{0.736125in}{1.131497in}}%
\pgfpathlineto{\pgfqpoint{0.747416in}{1.131497in}}%
\pgfpathlineto{\pgfqpoint{0.747416in}{0.509715in}}%
\pgfpathlineto{\pgfqpoint{0.973224in}{0.509715in}}%
\pgfpathlineto{\pgfqpoint{0.973224in}{1.131497in}}%
\pgfpathlineto{\pgfqpoint{0.986127in}{1.131497in}}%
\pgfpathlineto{\pgfqpoint{0.986127in}{0.509715in}}%
\pgfpathlineto{\pgfqpoint{1.071612in}{0.509715in}}%
\pgfpathlineto{\pgfqpoint{1.071612in}{1.131497in}}%
\pgfpathlineto{\pgfqpoint{1.105483in}{1.131497in}}%
\pgfpathlineto{\pgfqpoint{1.105483in}{0.509715in}}%
\pgfpathlineto{\pgfqpoint{1.902264in}{0.509715in}}%
\pgfpathlineto{\pgfqpoint{1.902264in}{1.131497in}}%
\pgfpathlineto{\pgfqpoint{1.986136in}{1.131497in}}%
\pgfpathlineto{\pgfqpoint{1.986136in}{0.509715in}}%
\pgfpathlineto{\pgfqpoint{2.245815in}{0.509715in}}%
\pgfpathlineto{\pgfqpoint{2.245815in}{1.131497in}}%
\pgfpathlineto{\pgfqpoint{2.260331in}{1.131497in}}%
\pgfpathlineto{\pgfqpoint{2.260331in}{0.509715in}}%
\pgfpathlineto{\pgfqpoint{2.316784in}{0.509715in}}%
\pgfpathlineto{\pgfqpoint{2.316784in}{1.131497in}}%
\pgfpathlineto{\pgfqpoint{2.337751in}{1.131497in}}%
\pgfpathlineto{\pgfqpoint{2.337751in}{0.509715in}}%
\pgfpathlineto{\pgfqpoint{2.665174in}{0.509715in}}%
\pgfpathlineto{\pgfqpoint{2.665174in}{1.131497in}}%
\pgfpathlineto{\pgfqpoint{2.668399in}{1.131497in}}%
\pgfpathlineto{\pgfqpoint{2.668399in}{0.509715in}}%
\pgfpathlineto{\pgfqpoint{2.978079in}{0.509715in}}%
\pgfpathlineto{\pgfqpoint{2.978079in}{1.131497in}}%
\pgfpathlineto{\pgfqpoint{2.992596in}{1.131497in}}%
\pgfpathlineto{\pgfqpoint{2.992596in}{0.509715in}}%
\pgfpathlineto{\pgfqpoint{3.534536in}{0.509715in}}%
\pgfpathlineto{\pgfqpoint{3.534536in}{1.131497in}}%
\pgfpathlineto{\pgfqpoint{3.578084in}{1.131497in}}%
\pgfpathlineto{\pgfqpoint{3.578084in}{0.509715in}}%
\pgfpathlineto{\pgfqpoint{3.857119in}{0.509715in}}%
\pgfpathlineto{\pgfqpoint{3.857119in}{0.509715in}}%
\pgfusepath{stroke}%
\end{pgfscope}%
\begin{pgfscope}%
\pgfpathrectangle{\pgfqpoint{0.471687in}{0.416447in}}{\pgfqpoint{3.546642in}{2.051883in}}%
\pgfusepath{clip}%
\pgfsetrectcap%
\pgfsetroundjoin%
\pgfsetlinewidth{1.505625pt}%
\definecolor{currentstroke}{rgb}{0.000000,0.619608,0.450980}%
\pgfsetstrokecolor{currentstroke}%
\pgfsetdash{}{0pt}%
\pgfpathmoveto{\pgfqpoint{0.632899in}{1.131497in}}%
\pgfpathlineto{\pgfqpoint{0.736125in}{1.131497in}}%
\pgfpathlineto{\pgfqpoint{0.736125in}{1.753280in}}%
\pgfpathlineto{\pgfqpoint{0.916772in}{1.753280in}}%
\pgfpathlineto{\pgfqpoint{0.916772in}{1.131497in}}%
\pgfpathlineto{\pgfqpoint{1.142580in}{1.131497in}}%
\pgfpathlineto{\pgfqpoint{1.142580in}{1.753280in}}%
\pgfpathlineto{\pgfqpoint{1.616778in}{1.753280in}}%
\pgfpathlineto{\pgfqpoint{1.616778in}{1.131497in}}%
\pgfpathlineto{\pgfqpoint{1.702262in}{1.131497in}}%
\pgfpathlineto{\pgfqpoint{1.702262in}{1.753280in}}%
\pgfpathlineto{\pgfqpoint{2.094201in}{1.753280in}}%
\pgfpathlineto{\pgfqpoint{2.094201in}{1.131497in}}%
\pgfpathlineto{\pgfqpoint{2.890982in}{1.131497in}}%
\pgfpathlineto{\pgfqpoint{2.890982in}{1.753280in}}%
\pgfpathlineto{\pgfqpoint{3.857119in}{1.753280in}}%
\pgfpathlineto{\pgfqpoint{3.857119in}{1.753280in}}%
\pgfusepath{stroke}%
\end{pgfscope}%
\begin{pgfscope}%
\pgfpathrectangle{\pgfqpoint{0.471687in}{0.416447in}}{\pgfqpoint{3.546642in}{2.051883in}}%
\pgfusepath{clip}%
\pgfsetrectcap%
\pgfsetroundjoin%
\pgfsetlinewidth{1.505625pt}%
\definecolor{currentstroke}{rgb}{0.835294,0.368627,0.000000}%
\pgfsetstrokecolor{currentstroke}%
\pgfsetdash{}{0pt}%
\pgfpathmoveto{\pgfqpoint{0.632899in}{2.375063in}}%
\pgfpathlineto{\pgfqpoint{0.811932in}{2.375063in}}%
\pgfpathlineto{\pgfqpoint{0.811932in}{1.753280in}}%
\pgfpathlineto{\pgfqpoint{0.916772in}{1.753280in}}%
\pgfpathlineto{\pgfqpoint{0.916772in}{2.375063in}}%
\pgfpathlineto{\pgfqpoint{3.857119in}{2.375063in}}%
\pgfpathlineto{\pgfqpoint{3.857119in}{2.375063in}}%
\pgfusepath{stroke}%
\end{pgfscope}%
\begin{pgfscope}%
\pgfsetrectcap%
\pgfsetmiterjoin%
\pgfsetlinewidth{0.803000pt}%
\definecolor{currentstroke}{rgb}{0.000000,0.000000,0.000000}%
\pgfsetstrokecolor{currentstroke}%
\pgfsetdash{}{0pt}%
\pgfpathmoveto{\pgfqpoint{0.471687in}{0.416447in}}%
\pgfpathlineto{\pgfqpoint{0.471687in}{2.468330in}}%
\pgfusepath{stroke}%
\end{pgfscope}%
\begin{pgfscope}%
\pgfsetrectcap%
\pgfsetmiterjoin%
\pgfsetlinewidth{0.803000pt}%
\definecolor{currentstroke}{rgb}{0.000000,0.000000,0.000000}%
\pgfsetstrokecolor{currentstroke}%
\pgfsetdash{}{0pt}%
\pgfpathmoveto{\pgfqpoint{4.018330in}{0.416447in}}%
\pgfpathlineto{\pgfqpoint{4.018330in}{2.468330in}}%
\pgfusepath{stroke}%
\end{pgfscope}%
\begin{pgfscope}%
\pgfsetrectcap%
\pgfsetmiterjoin%
\pgfsetlinewidth{0.803000pt}%
\definecolor{currentstroke}{rgb}{0.000000,0.000000,0.000000}%
\pgfsetstrokecolor{currentstroke}%
\pgfsetdash{}{0pt}%
\pgfpathmoveto{\pgfqpoint{0.471687in}{0.416447in}}%
\pgfpathlineto{\pgfqpoint{4.018330in}{0.416447in}}%
\pgfusepath{stroke}%
\end{pgfscope}%
\begin{pgfscope}%
\pgfsetrectcap%
\pgfsetmiterjoin%
\pgfsetlinewidth{0.803000pt}%
\definecolor{currentstroke}{rgb}{0.000000,0.000000,0.000000}%
\pgfsetstrokecolor{currentstroke}%
\pgfsetdash{}{0pt}%
\pgfpathmoveto{\pgfqpoint{0.471687in}{2.468330in}}%
\pgfpathlineto{\pgfqpoint{4.018330in}{2.468330in}}%
\pgfusepath{stroke}%
\end{pgfscope}%
\begin{pgfscope}%
\pgfsetbuttcap%
\pgfsetmiterjoin%
\definecolor{currentfill}{rgb}{1.000000,1.000000,1.000000}%
\pgfsetfillcolor{currentfill}%
\pgfsetfillopacity{0.800000}%
\pgfsetlinewidth{1.003750pt}%
\definecolor{currentstroke}{rgb}{0.800000,0.800000,0.800000}%
\pgfsetstrokecolor{currentstroke}%
\pgfsetstrokeopacity{0.800000}%
\pgfsetdash{}{0pt}%
\pgfpathmoveto{\pgfqpoint{3.100242in}{1.914775in}}%
\pgfpathlineto{\pgfqpoint{3.940552in}{1.914775in}}%
\pgfpathquadraticcurveto{\pgfqpoint{3.962774in}{1.914775in}}{\pgfqpoint{3.962774in}{1.936997in}}%
\pgfpathlineto{\pgfqpoint{3.962774in}{2.390552in}}%
\pgfpathquadraticcurveto{\pgfqpoint{3.962774in}{2.412774in}}{\pgfqpoint{3.940552in}{2.412774in}}%
\pgfpathlineto{\pgfqpoint{3.100242in}{2.412774in}}%
\pgfpathquadraticcurveto{\pgfqpoint{3.078020in}{2.412774in}}{\pgfqpoint{3.078020in}{2.390552in}}%
\pgfpathlineto{\pgfqpoint{3.078020in}{1.936997in}}%
\pgfpathquadraticcurveto{\pgfqpoint{3.078020in}{1.914775in}}{\pgfqpoint{3.100242in}{1.914775in}}%
\pgfpathlineto{\pgfqpoint{3.100242in}{1.914775in}}%
\pgfpathclose%
\pgfusepath{stroke,fill}%
\end{pgfscope}%
\begin{pgfscope}%
\pgfsetrectcap%
\pgfsetroundjoin%
\pgfsetlinewidth{1.505625pt}%
\definecolor{currentstroke}{rgb}{0.000000,0.447059,0.698039}%
\pgfsetstrokecolor{currentstroke}%
\pgfsetdash{}{0pt}%
\pgfpathmoveto{\pgfqpoint{3.122464in}{2.329441in}}%
\pgfpathlineto{\pgfqpoint{3.122464in}{2.329441in}}%
\pgfpathlineto{\pgfqpoint{3.233575in}{2.329441in}}%
\pgfpathlineto{\pgfqpoint{3.233575in}{2.329441in}}%
\pgfpathlineto{\pgfqpoint{3.344686in}{2.329441in}}%
\pgfusepath{stroke}%
\end{pgfscope}%
\begin{pgfscope}%
\definecolor{textcolor}{rgb}{0.000000,0.000000,0.000000}%
\pgfsetstrokecolor{textcolor}%
\pgfsetfillcolor{textcolor}%
\pgftext[x=3.433575in,y=2.290552in,left,base]{\color{textcolor}\rmfamily\fontsize{8.000000}{9.600000}\selectfont \(\displaystyle \bar\tau_1=\qty{0.1}{\s}\)}%
\end{pgfscope}%
\begin{pgfscope}%
\pgfsetrectcap%
\pgfsetroundjoin%
\pgfsetlinewidth{1.505625pt}%
\definecolor{currentstroke}{rgb}{0.000000,0.619608,0.450980}%
\pgfsetstrokecolor{currentstroke}%
\pgfsetdash{}{0pt}%
\pgfpathmoveto{\pgfqpoint{3.122464in}{2.174552in}}%
\pgfpathlineto{\pgfqpoint{3.122464in}{2.174552in}}%
\pgfpathlineto{\pgfqpoint{3.233575in}{2.174552in}}%
\pgfpathlineto{\pgfqpoint{3.233575in}{2.174552in}}%
\pgfpathlineto{\pgfqpoint{3.344686in}{2.174552in}}%
\pgfusepath{stroke}%
\end{pgfscope}%
\begin{pgfscope}%
\definecolor{textcolor}{rgb}{0.000000,0.000000,0.000000}%
\pgfsetstrokecolor{textcolor}%
\pgfsetfillcolor{textcolor}%
\pgftext[x=3.433575in,y=2.135663in,left,base]{\color{textcolor}\rmfamily\fontsize{8.000000}{9.600000}\selectfont \(\displaystyle \bar\tau_1=\qty{1}{\s}\)}%
\end{pgfscope}%
\begin{pgfscope}%
\pgfsetrectcap%
\pgfsetroundjoin%
\pgfsetlinewidth{1.505625pt}%
\definecolor{currentstroke}{rgb}{0.835294,0.368627,0.000000}%
\pgfsetstrokecolor{currentstroke}%
\pgfsetdash{}{0pt}%
\pgfpathmoveto{\pgfqpoint{3.122464in}{2.019664in}}%
\pgfpathlineto{\pgfqpoint{3.122464in}{2.019664in}}%
\pgfpathlineto{\pgfqpoint{3.233575in}{2.019664in}}%
\pgfpathlineto{\pgfqpoint{3.233575in}{2.019664in}}%
\pgfpathlineto{\pgfqpoint{3.344686in}{2.019664in}}%
\pgfusepath{stroke}%
\end{pgfscope}%
\begin{pgfscope}%
\definecolor{textcolor}{rgb}{0.000000,0.000000,0.000000}%
\pgfsetstrokecolor{textcolor}%
\pgfsetfillcolor{textcolor}%
\pgftext[x=3.433575in,y=1.980775in,left,base]{\color{textcolor}\rmfamily\fontsize{8.000000}{9.600000}\selectfont \(\displaystyle \bar\tau_1=\qty{10}{\s}\)}%
\end{pgfscope}%
\end{pgfpicture}%
\makeatother%
\endgroup%

        } % scalebox
        \caption{Time domain}
        \label{fig:burst_noise_time}
    \end{subfigure}
    \begin{subfigure}{0.8\linewidth}
        \centering
        \scalebox{1}{%
            %% Creator: Matplotlib, PGF backend
%%
%% To include the figure in your LaTeX document, write
%%   \input{<filename>.pgf}
%%
%% Make sure the required packages are loaded in your preamble
%%   \usepackage{pgf}
%%
%% Also ensure that all the required font packages are loaded; for instance,
%% the lmodern package is sometimes necessary when using math font.
%%   \usepackage{lmodern}
%%
%% Figures using additional raster images can only be included by \input if
%% they are in the same directory as the main LaTeX file. For loading figures
%% from other directories you can use the `import` package
%%   \usepackage{import}
%%
%% and then include the figures with
%%   \import{<path to file>}{<filename>.pgf}
%%
%% Matplotlib used the following preamble
%%   \usepackage{siunitx}
%%   \usepackage{fontspec}
%%
\begingroup%
\makeatletter%
\begin{pgfpicture}%
\pgfpathrectangle{\pgfpointorigin}{\pgfqpoint{5.490000in}{3.390000in}}%
\pgfusepath{use as bounding box, clip}%
\begin{pgfscope}%
\pgfsetbuttcap%
\pgfsetmiterjoin%
\definecolor{currentfill}{rgb}{1.000000,1.000000,1.000000}%
\pgfsetfillcolor{currentfill}%
\pgfsetlinewidth{0.000000pt}%
\definecolor{currentstroke}{rgb}{1.000000,1.000000,1.000000}%
\pgfsetstrokecolor{currentstroke}%
\pgfsetdash{}{0pt}%
\pgfpathmoveto{\pgfqpoint{0.000000in}{0.000000in}}%
\pgfpathlineto{\pgfqpoint{5.490000in}{0.000000in}}%
\pgfpathlineto{\pgfqpoint{5.490000in}{3.390000in}}%
\pgfpathlineto{\pgfqpoint{0.000000in}{3.390000in}}%
\pgfpathlineto{\pgfqpoint{0.000000in}{0.000000in}}%
\pgfpathclose%
\pgfusepath{fill}%
\end{pgfscope}%
\begin{pgfscope}%
\pgfsetbuttcap%
\pgfsetmiterjoin%
\definecolor{currentfill}{rgb}{1.000000,1.000000,1.000000}%
\pgfsetfillcolor{currentfill}%
\pgfsetlinewidth{0.000000pt}%
\definecolor{currentstroke}{rgb}{0.000000,0.000000,0.000000}%
\pgfsetstrokecolor{currentstroke}%
\pgfsetstrokeopacity{0.000000}%
\pgfsetdash{}{0pt}%
\pgfpathmoveto{\pgfqpoint{0.605343in}{0.417642in}}%
\pgfpathlineto{\pgfqpoint{5.448330in}{0.417642in}}%
\pgfpathlineto{\pgfqpoint{5.448330in}{3.348330in}}%
\pgfpathlineto{\pgfqpoint{0.605343in}{3.348330in}}%
\pgfpathlineto{\pgfqpoint{0.605343in}{0.417642in}}%
\pgfpathclose%
\pgfusepath{fill}%
\end{pgfscope}%
\begin{pgfscope}%
\pgfpathrectangle{\pgfqpoint{0.605343in}{0.417642in}}{\pgfqpoint{4.842987in}{2.930688in}}%
\pgfusepath{clip}%
\pgfsetrectcap%
\pgfsetroundjoin%
\pgfsetlinewidth{0.803000pt}%
\definecolor{currentstroke}{rgb}{0.450000,0.450000,0.450000}%
\pgfsetstrokecolor{currentstroke}%
\pgfsetdash{}{0pt}%
\pgfpathmoveto{\pgfqpoint{0.825479in}{0.417642in}}%
\pgfpathlineto{\pgfqpoint{0.825479in}{3.348330in}}%
\pgfusepath{stroke}%
\end{pgfscope}%
\begin{pgfscope}%
\pgfsetbuttcap%
\pgfsetroundjoin%
\definecolor{currentfill}{rgb}{0.000000,0.000000,0.000000}%
\pgfsetfillcolor{currentfill}%
\pgfsetlinewidth{0.803000pt}%
\definecolor{currentstroke}{rgb}{0.000000,0.000000,0.000000}%
\pgfsetstrokecolor{currentstroke}%
\pgfsetdash{}{0pt}%
\pgfsys@defobject{currentmarker}{\pgfqpoint{0.000000in}{-0.048611in}}{\pgfqpoint{0.000000in}{0.000000in}}{%
\pgfpathmoveto{\pgfqpoint{0.000000in}{0.000000in}}%
\pgfpathlineto{\pgfqpoint{0.000000in}{-0.048611in}}%
\pgfusepath{stroke,fill}%
}%
\begin{pgfscope}%
\pgfsys@transformshift{0.825479in}{0.417642in}%
\pgfsys@useobject{currentmarker}{}%
\end{pgfscope}%
\end{pgfscope}%
\begin{pgfscope}%
\definecolor{textcolor}{rgb}{0.000000,0.000000,0.000000}%
\pgfsetstrokecolor{textcolor}%
\pgfsetfillcolor{textcolor}%
\pgftext[x=0.825479in,y=0.320420in,,top]{\color{textcolor}\rmfamily\fontsize{8.000000}{9.600000}\selectfont \(\displaystyle {10^{-3}}\)}%
\end{pgfscope}%
\begin{pgfscope}%
\pgfpathrectangle{\pgfqpoint{0.605343in}{0.417642in}}{\pgfqpoint{4.842987in}{2.930688in}}%
\pgfusepath{clip}%
\pgfsetrectcap%
\pgfsetroundjoin%
\pgfsetlinewidth{0.803000pt}%
\definecolor{currentstroke}{rgb}{0.450000,0.450000,0.450000}%
\pgfsetstrokecolor{currentstroke}%
\pgfsetdash{}{0pt}%
\pgfpathmoveto{\pgfqpoint{1.559265in}{0.417642in}}%
\pgfpathlineto{\pgfqpoint{1.559265in}{3.348330in}}%
\pgfusepath{stroke}%
\end{pgfscope}%
\begin{pgfscope}%
\pgfsetbuttcap%
\pgfsetroundjoin%
\definecolor{currentfill}{rgb}{0.000000,0.000000,0.000000}%
\pgfsetfillcolor{currentfill}%
\pgfsetlinewidth{0.803000pt}%
\definecolor{currentstroke}{rgb}{0.000000,0.000000,0.000000}%
\pgfsetstrokecolor{currentstroke}%
\pgfsetdash{}{0pt}%
\pgfsys@defobject{currentmarker}{\pgfqpoint{0.000000in}{-0.048611in}}{\pgfqpoint{0.000000in}{0.000000in}}{%
\pgfpathmoveto{\pgfqpoint{0.000000in}{0.000000in}}%
\pgfpathlineto{\pgfqpoint{0.000000in}{-0.048611in}}%
\pgfusepath{stroke,fill}%
}%
\begin{pgfscope}%
\pgfsys@transformshift{1.559265in}{0.417642in}%
\pgfsys@useobject{currentmarker}{}%
\end{pgfscope}%
\end{pgfscope}%
\begin{pgfscope}%
\definecolor{textcolor}{rgb}{0.000000,0.000000,0.000000}%
\pgfsetstrokecolor{textcolor}%
\pgfsetfillcolor{textcolor}%
\pgftext[x=1.559265in,y=0.320420in,,top]{\color{textcolor}\rmfamily\fontsize{8.000000}{9.600000}\selectfont \(\displaystyle {10^{-2}}\)}%
\end{pgfscope}%
\begin{pgfscope}%
\pgfpathrectangle{\pgfqpoint{0.605343in}{0.417642in}}{\pgfqpoint{4.842987in}{2.930688in}}%
\pgfusepath{clip}%
\pgfsetrectcap%
\pgfsetroundjoin%
\pgfsetlinewidth{0.803000pt}%
\definecolor{currentstroke}{rgb}{0.450000,0.450000,0.450000}%
\pgfsetstrokecolor{currentstroke}%
\pgfsetdash{}{0pt}%
\pgfpathmoveto{\pgfqpoint{2.293051in}{0.417642in}}%
\pgfpathlineto{\pgfqpoint{2.293051in}{3.348330in}}%
\pgfusepath{stroke}%
\end{pgfscope}%
\begin{pgfscope}%
\pgfsetbuttcap%
\pgfsetroundjoin%
\definecolor{currentfill}{rgb}{0.000000,0.000000,0.000000}%
\pgfsetfillcolor{currentfill}%
\pgfsetlinewidth{0.803000pt}%
\definecolor{currentstroke}{rgb}{0.000000,0.000000,0.000000}%
\pgfsetstrokecolor{currentstroke}%
\pgfsetdash{}{0pt}%
\pgfsys@defobject{currentmarker}{\pgfqpoint{0.000000in}{-0.048611in}}{\pgfqpoint{0.000000in}{0.000000in}}{%
\pgfpathmoveto{\pgfqpoint{0.000000in}{0.000000in}}%
\pgfpathlineto{\pgfqpoint{0.000000in}{-0.048611in}}%
\pgfusepath{stroke,fill}%
}%
\begin{pgfscope}%
\pgfsys@transformshift{2.293051in}{0.417642in}%
\pgfsys@useobject{currentmarker}{}%
\end{pgfscope}%
\end{pgfscope}%
\begin{pgfscope}%
\definecolor{textcolor}{rgb}{0.000000,0.000000,0.000000}%
\pgfsetstrokecolor{textcolor}%
\pgfsetfillcolor{textcolor}%
\pgftext[x=2.293051in,y=0.320420in,,top]{\color{textcolor}\rmfamily\fontsize{8.000000}{9.600000}\selectfont \(\displaystyle {10^{-1}}\)}%
\end{pgfscope}%
\begin{pgfscope}%
\pgfpathrectangle{\pgfqpoint{0.605343in}{0.417642in}}{\pgfqpoint{4.842987in}{2.930688in}}%
\pgfusepath{clip}%
\pgfsetrectcap%
\pgfsetroundjoin%
\pgfsetlinewidth{0.803000pt}%
\definecolor{currentstroke}{rgb}{0.450000,0.450000,0.450000}%
\pgfsetstrokecolor{currentstroke}%
\pgfsetdash{}{0pt}%
\pgfpathmoveto{\pgfqpoint{3.026837in}{0.417642in}}%
\pgfpathlineto{\pgfqpoint{3.026837in}{3.348330in}}%
\pgfusepath{stroke}%
\end{pgfscope}%
\begin{pgfscope}%
\pgfsetbuttcap%
\pgfsetroundjoin%
\definecolor{currentfill}{rgb}{0.000000,0.000000,0.000000}%
\pgfsetfillcolor{currentfill}%
\pgfsetlinewidth{0.803000pt}%
\definecolor{currentstroke}{rgb}{0.000000,0.000000,0.000000}%
\pgfsetstrokecolor{currentstroke}%
\pgfsetdash{}{0pt}%
\pgfsys@defobject{currentmarker}{\pgfqpoint{0.000000in}{-0.048611in}}{\pgfqpoint{0.000000in}{0.000000in}}{%
\pgfpathmoveto{\pgfqpoint{0.000000in}{0.000000in}}%
\pgfpathlineto{\pgfqpoint{0.000000in}{-0.048611in}}%
\pgfusepath{stroke,fill}%
}%
\begin{pgfscope}%
\pgfsys@transformshift{3.026837in}{0.417642in}%
\pgfsys@useobject{currentmarker}{}%
\end{pgfscope}%
\end{pgfscope}%
\begin{pgfscope}%
\definecolor{textcolor}{rgb}{0.000000,0.000000,0.000000}%
\pgfsetstrokecolor{textcolor}%
\pgfsetfillcolor{textcolor}%
\pgftext[x=3.026837in,y=0.320420in,,top]{\color{textcolor}\rmfamily\fontsize{8.000000}{9.600000}\selectfont \(\displaystyle {10^{0}}\)}%
\end{pgfscope}%
\begin{pgfscope}%
\pgfpathrectangle{\pgfqpoint{0.605343in}{0.417642in}}{\pgfqpoint{4.842987in}{2.930688in}}%
\pgfusepath{clip}%
\pgfsetrectcap%
\pgfsetroundjoin%
\pgfsetlinewidth{0.803000pt}%
\definecolor{currentstroke}{rgb}{0.450000,0.450000,0.450000}%
\pgfsetstrokecolor{currentstroke}%
\pgfsetdash{}{0pt}%
\pgfpathmoveto{\pgfqpoint{3.760623in}{0.417642in}}%
\pgfpathlineto{\pgfqpoint{3.760623in}{3.348330in}}%
\pgfusepath{stroke}%
\end{pgfscope}%
\begin{pgfscope}%
\pgfsetbuttcap%
\pgfsetroundjoin%
\definecolor{currentfill}{rgb}{0.000000,0.000000,0.000000}%
\pgfsetfillcolor{currentfill}%
\pgfsetlinewidth{0.803000pt}%
\definecolor{currentstroke}{rgb}{0.000000,0.000000,0.000000}%
\pgfsetstrokecolor{currentstroke}%
\pgfsetdash{}{0pt}%
\pgfsys@defobject{currentmarker}{\pgfqpoint{0.000000in}{-0.048611in}}{\pgfqpoint{0.000000in}{0.000000in}}{%
\pgfpathmoveto{\pgfqpoint{0.000000in}{0.000000in}}%
\pgfpathlineto{\pgfqpoint{0.000000in}{-0.048611in}}%
\pgfusepath{stroke,fill}%
}%
\begin{pgfscope}%
\pgfsys@transformshift{3.760623in}{0.417642in}%
\pgfsys@useobject{currentmarker}{}%
\end{pgfscope}%
\end{pgfscope}%
\begin{pgfscope}%
\definecolor{textcolor}{rgb}{0.000000,0.000000,0.000000}%
\pgfsetstrokecolor{textcolor}%
\pgfsetfillcolor{textcolor}%
\pgftext[x=3.760623in,y=0.320420in,,top]{\color{textcolor}\rmfamily\fontsize{8.000000}{9.600000}\selectfont \(\displaystyle {10^{1}}\)}%
\end{pgfscope}%
\begin{pgfscope}%
\pgfpathrectangle{\pgfqpoint{0.605343in}{0.417642in}}{\pgfqpoint{4.842987in}{2.930688in}}%
\pgfusepath{clip}%
\pgfsetrectcap%
\pgfsetroundjoin%
\pgfsetlinewidth{0.803000pt}%
\definecolor{currentstroke}{rgb}{0.450000,0.450000,0.450000}%
\pgfsetstrokecolor{currentstroke}%
\pgfsetdash{}{0pt}%
\pgfpathmoveto{\pgfqpoint{4.494408in}{0.417642in}}%
\pgfpathlineto{\pgfqpoint{4.494408in}{3.348330in}}%
\pgfusepath{stroke}%
\end{pgfscope}%
\begin{pgfscope}%
\pgfsetbuttcap%
\pgfsetroundjoin%
\definecolor{currentfill}{rgb}{0.000000,0.000000,0.000000}%
\pgfsetfillcolor{currentfill}%
\pgfsetlinewidth{0.803000pt}%
\definecolor{currentstroke}{rgb}{0.000000,0.000000,0.000000}%
\pgfsetstrokecolor{currentstroke}%
\pgfsetdash{}{0pt}%
\pgfsys@defobject{currentmarker}{\pgfqpoint{0.000000in}{-0.048611in}}{\pgfqpoint{0.000000in}{0.000000in}}{%
\pgfpathmoveto{\pgfqpoint{0.000000in}{0.000000in}}%
\pgfpathlineto{\pgfqpoint{0.000000in}{-0.048611in}}%
\pgfusepath{stroke,fill}%
}%
\begin{pgfscope}%
\pgfsys@transformshift{4.494408in}{0.417642in}%
\pgfsys@useobject{currentmarker}{}%
\end{pgfscope}%
\end{pgfscope}%
\begin{pgfscope}%
\definecolor{textcolor}{rgb}{0.000000,0.000000,0.000000}%
\pgfsetstrokecolor{textcolor}%
\pgfsetfillcolor{textcolor}%
\pgftext[x=4.494408in,y=0.320420in,,top]{\color{textcolor}\rmfamily\fontsize{8.000000}{9.600000}\selectfont \(\displaystyle {10^{2}}\)}%
\end{pgfscope}%
\begin{pgfscope}%
\pgfpathrectangle{\pgfqpoint{0.605343in}{0.417642in}}{\pgfqpoint{4.842987in}{2.930688in}}%
\pgfusepath{clip}%
\pgfsetrectcap%
\pgfsetroundjoin%
\pgfsetlinewidth{0.803000pt}%
\definecolor{currentstroke}{rgb}{0.450000,0.450000,0.450000}%
\pgfsetstrokecolor{currentstroke}%
\pgfsetdash{}{0pt}%
\pgfpathmoveto{\pgfqpoint{5.228194in}{0.417642in}}%
\pgfpathlineto{\pgfqpoint{5.228194in}{3.348330in}}%
\pgfusepath{stroke}%
\end{pgfscope}%
\begin{pgfscope}%
\pgfsetbuttcap%
\pgfsetroundjoin%
\definecolor{currentfill}{rgb}{0.000000,0.000000,0.000000}%
\pgfsetfillcolor{currentfill}%
\pgfsetlinewidth{0.803000pt}%
\definecolor{currentstroke}{rgb}{0.000000,0.000000,0.000000}%
\pgfsetstrokecolor{currentstroke}%
\pgfsetdash{}{0pt}%
\pgfsys@defobject{currentmarker}{\pgfqpoint{0.000000in}{-0.048611in}}{\pgfqpoint{0.000000in}{0.000000in}}{%
\pgfpathmoveto{\pgfqpoint{0.000000in}{0.000000in}}%
\pgfpathlineto{\pgfqpoint{0.000000in}{-0.048611in}}%
\pgfusepath{stroke,fill}%
}%
\begin{pgfscope}%
\pgfsys@transformshift{5.228194in}{0.417642in}%
\pgfsys@useobject{currentmarker}{}%
\end{pgfscope}%
\end{pgfscope}%
\begin{pgfscope}%
\definecolor{textcolor}{rgb}{0.000000,0.000000,0.000000}%
\pgfsetstrokecolor{textcolor}%
\pgfsetfillcolor{textcolor}%
\pgftext[x=5.228194in,y=0.320420in,,top]{\color{textcolor}\rmfamily\fontsize{8.000000}{9.600000}\selectfont \(\displaystyle {10^{3}}\)}%
\end{pgfscope}%
\begin{pgfscope}%
\pgfpathrectangle{\pgfqpoint{0.605343in}{0.417642in}}{\pgfqpoint{4.842987in}{2.930688in}}%
\pgfusepath{clip}%
\pgfsetrectcap%
\pgfsetroundjoin%
\pgfsetlinewidth{0.803000pt}%
\definecolor{currentstroke}{rgb}{0.850000,0.850000,0.850000}%
\pgfsetstrokecolor{currentstroke}%
\pgfsetdash{}{0pt}%
\pgfpathmoveto{\pgfqpoint{0.662690in}{0.417642in}}%
\pgfpathlineto{\pgfqpoint{0.662690in}{3.348330in}}%
\pgfusepath{stroke}%
\end{pgfscope}%
\begin{pgfscope}%
\pgfsetbuttcap%
\pgfsetroundjoin%
\definecolor{currentfill}{rgb}{0.000000,0.000000,0.000000}%
\pgfsetfillcolor{currentfill}%
\pgfsetlinewidth{0.602250pt}%
\definecolor{currentstroke}{rgb}{0.000000,0.000000,0.000000}%
\pgfsetstrokecolor{currentstroke}%
\pgfsetdash{}{0pt}%
\pgfsys@defobject{currentmarker}{\pgfqpoint{0.000000in}{-0.027778in}}{\pgfqpoint{0.000000in}{0.000000in}}{%
\pgfpathmoveto{\pgfqpoint{0.000000in}{0.000000in}}%
\pgfpathlineto{\pgfqpoint{0.000000in}{-0.027778in}}%
\pgfusepath{stroke,fill}%
}%
\begin{pgfscope}%
\pgfsys@transformshift{0.662690in}{0.417642in}%
\pgfsys@useobject{currentmarker}{}%
\end{pgfscope}%
\end{pgfscope}%
\begin{pgfscope}%
\pgfpathrectangle{\pgfqpoint{0.605343in}{0.417642in}}{\pgfqpoint{4.842987in}{2.930688in}}%
\pgfusepath{clip}%
\pgfsetrectcap%
\pgfsetroundjoin%
\pgfsetlinewidth{0.803000pt}%
\definecolor{currentstroke}{rgb}{0.850000,0.850000,0.850000}%
\pgfsetstrokecolor{currentstroke}%
\pgfsetdash{}{0pt}%
\pgfpathmoveto{\pgfqpoint{0.711814in}{0.417642in}}%
\pgfpathlineto{\pgfqpoint{0.711814in}{3.348330in}}%
\pgfusepath{stroke}%
\end{pgfscope}%
\begin{pgfscope}%
\pgfsetbuttcap%
\pgfsetroundjoin%
\definecolor{currentfill}{rgb}{0.000000,0.000000,0.000000}%
\pgfsetfillcolor{currentfill}%
\pgfsetlinewidth{0.602250pt}%
\definecolor{currentstroke}{rgb}{0.000000,0.000000,0.000000}%
\pgfsetstrokecolor{currentstroke}%
\pgfsetdash{}{0pt}%
\pgfsys@defobject{currentmarker}{\pgfqpoint{0.000000in}{-0.027778in}}{\pgfqpoint{0.000000in}{0.000000in}}{%
\pgfpathmoveto{\pgfqpoint{0.000000in}{0.000000in}}%
\pgfpathlineto{\pgfqpoint{0.000000in}{-0.027778in}}%
\pgfusepath{stroke,fill}%
}%
\begin{pgfscope}%
\pgfsys@transformshift{0.711814in}{0.417642in}%
\pgfsys@useobject{currentmarker}{}%
\end{pgfscope}%
\end{pgfscope}%
\begin{pgfscope}%
\pgfpathrectangle{\pgfqpoint{0.605343in}{0.417642in}}{\pgfqpoint{4.842987in}{2.930688in}}%
\pgfusepath{clip}%
\pgfsetrectcap%
\pgfsetroundjoin%
\pgfsetlinewidth{0.803000pt}%
\definecolor{currentstroke}{rgb}{0.850000,0.850000,0.850000}%
\pgfsetstrokecolor{currentstroke}%
\pgfsetdash{}{0pt}%
\pgfpathmoveto{\pgfqpoint{0.754368in}{0.417642in}}%
\pgfpathlineto{\pgfqpoint{0.754368in}{3.348330in}}%
\pgfusepath{stroke}%
\end{pgfscope}%
\begin{pgfscope}%
\pgfsetbuttcap%
\pgfsetroundjoin%
\definecolor{currentfill}{rgb}{0.000000,0.000000,0.000000}%
\pgfsetfillcolor{currentfill}%
\pgfsetlinewidth{0.602250pt}%
\definecolor{currentstroke}{rgb}{0.000000,0.000000,0.000000}%
\pgfsetstrokecolor{currentstroke}%
\pgfsetdash{}{0pt}%
\pgfsys@defobject{currentmarker}{\pgfqpoint{0.000000in}{-0.027778in}}{\pgfqpoint{0.000000in}{0.000000in}}{%
\pgfpathmoveto{\pgfqpoint{0.000000in}{0.000000in}}%
\pgfpathlineto{\pgfqpoint{0.000000in}{-0.027778in}}%
\pgfusepath{stroke,fill}%
}%
\begin{pgfscope}%
\pgfsys@transformshift{0.754368in}{0.417642in}%
\pgfsys@useobject{currentmarker}{}%
\end{pgfscope}%
\end{pgfscope}%
\begin{pgfscope}%
\pgfpathrectangle{\pgfqpoint{0.605343in}{0.417642in}}{\pgfqpoint{4.842987in}{2.930688in}}%
\pgfusepath{clip}%
\pgfsetrectcap%
\pgfsetroundjoin%
\pgfsetlinewidth{0.803000pt}%
\definecolor{currentstroke}{rgb}{0.850000,0.850000,0.850000}%
\pgfsetstrokecolor{currentstroke}%
\pgfsetdash{}{0pt}%
\pgfpathmoveto{\pgfqpoint{0.791903in}{0.417642in}}%
\pgfpathlineto{\pgfqpoint{0.791903in}{3.348330in}}%
\pgfusepath{stroke}%
\end{pgfscope}%
\begin{pgfscope}%
\pgfsetbuttcap%
\pgfsetroundjoin%
\definecolor{currentfill}{rgb}{0.000000,0.000000,0.000000}%
\pgfsetfillcolor{currentfill}%
\pgfsetlinewidth{0.602250pt}%
\definecolor{currentstroke}{rgb}{0.000000,0.000000,0.000000}%
\pgfsetstrokecolor{currentstroke}%
\pgfsetdash{}{0pt}%
\pgfsys@defobject{currentmarker}{\pgfqpoint{0.000000in}{-0.027778in}}{\pgfqpoint{0.000000in}{0.000000in}}{%
\pgfpathmoveto{\pgfqpoint{0.000000in}{0.000000in}}%
\pgfpathlineto{\pgfqpoint{0.000000in}{-0.027778in}}%
\pgfusepath{stroke,fill}%
}%
\begin{pgfscope}%
\pgfsys@transformshift{0.791903in}{0.417642in}%
\pgfsys@useobject{currentmarker}{}%
\end{pgfscope}%
\end{pgfscope}%
\begin{pgfscope}%
\pgfpathrectangle{\pgfqpoint{0.605343in}{0.417642in}}{\pgfqpoint{4.842987in}{2.930688in}}%
\pgfusepath{clip}%
\pgfsetrectcap%
\pgfsetroundjoin%
\pgfsetlinewidth{0.803000pt}%
\definecolor{currentstroke}{rgb}{0.850000,0.850000,0.850000}%
\pgfsetstrokecolor{currentstroke}%
\pgfsetdash{}{0pt}%
\pgfpathmoveto{\pgfqpoint{1.046371in}{0.417642in}}%
\pgfpathlineto{\pgfqpoint{1.046371in}{3.348330in}}%
\pgfusepath{stroke}%
\end{pgfscope}%
\begin{pgfscope}%
\pgfsetbuttcap%
\pgfsetroundjoin%
\definecolor{currentfill}{rgb}{0.000000,0.000000,0.000000}%
\pgfsetfillcolor{currentfill}%
\pgfsetlinewidth{0.602250pt}%
\definecolor{currentstroke}{rgb}{0.000000,0.000000,0.000000}%
\pgfsetstrokecolor{currentstroke}%
\pgfsetdash{}{0pt}%
\pgfsys@defobject{currentmarker}{\pgfqpoint{0.000000in}{-0.027778in}}{\pgfqpoint{0.000000in}{0.000000in}}{%
\pgfpathmoveto{\pgfqpoint{0.000000in}{0.000000in}}%
\pgfpathlineto{\pgfqpoint{0.000000in}{-0.027778in}}%
\pgfusepath{stroke,fill}%
}%
\begin{pgfscope}%
\pgfsys@transformshift{1.046371in}{0.417642in}%
\pgfsys@useobject{currentmarker}{}%
\end{pgfscope}%
\end{pgfscope}%
\begin{pgfscope}%
\pgfpathrectangle{\pgfqpoint{0.605343in}{0.417642in}}{\pgfqpoint{4.842987in}{2.930688in}}%
\pgfusepath{clip}%
\pgfsetrectcap%
\pgfsetroundjoin%
\pgfsetlinewidth{0.803000pt}%
\definecolor{currentstroke}{rgb}{0.850000,0.850000,0.850000}%
\pgfsetstrokecolor{currentstroke}%
\pgfsetdash{}{0pt}%
\pgfpathmoveto{\pgfqpoint{1.175584in}{0.417642in}}%
\pgfpathlineto{\pgfqpoint{1.175584in}{3.348330in}}%
\pgfusepath{stroke}%
\end{pgfscope}%
\begin{pgfscope}%
\pgfsetbuttcap%
\pgfsetroundjoin%
\definecolor{currentfill}{rgb}{0.000000,0.000000,0.000000}%
\pgfsetfillcolor{currentfill}%
\pgfsetlinewidth{0.602250pt}%
\definecolor{currentstroke}{rgb}{0.000000,0.000000,0.000000}%
\pgfsetstrokecolor{currentstroke}%
\pgfsetdash{}{0pt}%
\pgfsys@defobject{currentmarker}{\pgfqpoint{0.000000in}{-0.027778in}}{\pgfqpoint{0.000000in}{0.000000in}}{%
\pgfpathmoveto{\pgfqpoint{0.000000in}{0.000000in}}%
\pgfpathlineto{\pgfqpoint{0.000000in}{-0.027778in}}%
\pgfusepath{stroke,fill}%
}%
\begin{pgfscope}%
\pgfsys@transformshift{1.175584in}{0.417642in}%
\pgfsys@useobject{currentmarker}{}%
\end{pgfscope}%
\end{pgfscope}%
\begin{pgfscope}%
\pgfpathrectangle{\pgfqpoint{0.605343in}{0.417642in}}{\pgfqpoint{4.842987in}{2.930688in}}%
\pgfusepath{clip}%
\pgfsetrectcap%
\pgfsetroundjoin%
\pgfsetlinewidth{0.803000pt}%
\definecolor{currentstroke}{rgb}{0.850000,0.850000,0.850000}%
\pgfsetstrokecolor{currentstroke}%
\pgfsetdash{}{0pt}%
\pgfpathmoveto{\pgfqpoint{1.267262in}{0.417642in}}%
\pgfpathlineto{\pgfqpoint{1.267262in}{3.348330in}}%
\pgfusepath{stroke}%
\end{pgfscope}%
\begin{pgfscope}%
\pgfsetbuttcap%
\pgfsetroundjoin%
\definecolor{currentfill}{rgb}{0.000000,0.000000,0.000000}%
\pgfsetfillcolor{currentfill}%
\pgfsetlinewidth{0.602250pt}%
\definecolor{currentstroke}{rgb}{0.000000,0.000000,0.000000}%
\pgfsetstrokecolor{currentstroke}%
\pgfsetdash{}{0pt}%
\pgfsys@defobject{currentmarker}{\pgfqpoint{0.000000in}{-0.027778in}}{\pgfqpoint{0.000000in}{0.000000in}}{%
\pgfpathmoveto{\pgfqpoint{0.000000in}{0.000000in}}%
\pgfpathlineto{\pgfqpoint{0.000000in}{-0.027778in}}%
\pgfusepath{stroke,fill}%
}%
\begin{pgfscope}%
\pgfsys@transformshift{1.267262in}{0.417642in}%
\pgfsys@useobject{currentmarker}{}%
\end{pgfscope}%
\end{pgfscope}%
\begin{pgfscope}%
\pgfpathrectangle{\pgfqpoint{0.605343in}{0.417642in}}{\pgfqpoint{4.842987in}{2.930688in}}%
\pgfusepath{clip}%
\pgfsetrectcap%
\pgfsetroundjoin%
\pgfsetlinewidth{0.803000pt}%
\definecolor{currentstroke}{rgb}{0.850000,0.850000,0.850000}%
\pgfsetstrokecolor{currentstroke}%
\pgfsetdash{}{0pt}%
\pgfpathmoveto{\pgfqpoint{1.338374in}{0.417642in}}%
\pgfpathlineto{\pgfqpoint{1.338374in}{3.348330in}}%
\pgfusepath{stroke}%
\end{pgfscope}%
\begin{pgfscope}%
\pgfsetbuttcap%
\pgfsetroundjoin%
\definecolor{currentfill}{rgb}{0.000000,0.000000,0.000000}%
\pgfsetfillcolor{currentfill}%
\pgfsetlinewidth{0.602250pt}%
\definecolor{currentstroke}{rgb}{0.000000,0.000000,0.000000}%
\pgfsetstrokecolor{currentstroke}%
\pgfsetdash{}{0pt}%
\pgfsys@defobject{currentmarker}{\pgfqpoint{0.000000in}{-0.027778in}}{\pgfqpoint{0.000000in}{0.000000in}}{%
\pgfpathmoveto{\pgfqpoint{0.000000in}{0.000000in}}%
\pgfpathlineto{\pgfqpoint{0.000000in}{-0.027778in}}%
\pgfusepath{stroke,fill}%
}%
\begin{pgfscope}%
\pgfsys@transformshift{1.338374in}{0.417642in}%
\pgfsys@useobject{currentmarker}{}%
\end{pgfscope}%
\end{pgfscope}%
\begin{pgfscope}%
\pgfpathrectangle{\pgfqpoint{0.605343in}{0.417642in}}{\pgfqpoint{4.842987in}{2.930688in}}%
\pgfusepath{clip}%
\pgfsetrectcap%
\pgfsetroundjoin%
\pgfsetlinewidth{0.803000pt}%
\definecolor{currentstroke}{rgb}{0.850000,0.850000,0.850000}%
\pgfsetstrokecolor{currentstroke}%
\pgfsetdash{}{0pt}%
\pgfpathmoveto{\pgfqpoint{1.396476in}{0.417642in}}%
\pgfpathlineto{\pgfqpoint{1.396476in}{3.348330in}}%
\pgfusepath{stroke}%
\end{pgfscope}%
\begin{pgfscope}%
\pgfsetbuttcap%
\pgfsetroundjoin%
\definecolor{currentfill}{rgb}{0.000000,0.000000,0.000000}%
\pgfsetfillcolor{currentfill}%
\pgfsetlinewidth{0.602250pt}%
\definecolor{currentstroke}{rgb}{0.000000,0.000000,0.000000}%
\pgfsetstrokecolor{currentstroke}%
\pgfsetdash{}{0pt}%
\pgfsys@defobject{currentmarker}{\pgfqpoint{0.000000in}{-0.027778in}}{\pgfqpoint{0.000000in}{0.000000in}}{%
\pgfpathmoveto{\pgfqpoint{0.000000in}{0.000000in}}%
\pgfpathlineto{\pgfqpoint{0.000000in}{-0.027778in}}%
\pgfusepath{stroke,fill}%
}%
\begin{pgfscope}%
\pgfsys@transformshift{1.396476in}{0.417642in}%
\pgfsys@useobject{currentmarker}{}%
\end{pgfscope}%
\end{pgfscope}%
\begin{pgfscope}%
\pgfpathrectangle{\pgfqpoint{0.605343in}{0.417642in}}{\pgfqpoint{4.842987in}{2.930688in}}%
\pgfusepath{clip}%
\pgfsetrectcap%
\pgfsetroundjoin%
\pgfsetlinewidth{0.803000pt}%
\definecolor{currentstroke}{rgb}{0.850000,0.850000,0.850000}%
\pgfsetstrokecolor{currentstroke}%
\pgfsetdash{}{0pt}%
\pgfpathmoveto{\pgfqpoint{1.445600in}{0.417642in}}%
\pgfpathlineto{\pgfqpoint{1.445600in}{3.348330in}}%
\pgfusepath{stroke}%
\end{pgfscope}%
\begin{pgfscope}%
\pgfsetbuttcap%
\pgfsetroundjoin%
\definecolor{currentfill}{rgb}{0.000000,0.000000,0.000000}%
\pgfsetfillcolor{currentfill}%
\pgfsetlinewidth{0.602250pt}%
\definecolor{currentstroke}{rgb}{0.000000,0.000000,0.000000}%
\pgfsetstrokecolor{currentstroke}%
\pgfsetdash{}{0pt}%
\pgfsys@defobject{currentmarker}{\pgfqpoint{0.000000in}{-0.027778in}}{\pgfqpoint{0.000000in}{0.000000in}}{%
\pgfpathmoveto{\pgfqpoint{0.000000in}{0.000000in}}%
\pgfpathlineto{\pgfqpoint{0.000000in}{-0.027778in}}%
\pgfusepath{stroke,fill}%
}%
\begin{pgfscope}%
\pgfsys@transformshift{1.445600in}{0.417642in}%
\pgfsys@useobject{currentmarker}{}%
\end{pgfscope}%
\end{pgfscope}%
\begin{pgfscope}%
\pgfpathrectangle{\pgfqpoint{0.605343in}{0.417642in}}{\pgfqpoint{4.842987in}{2.930688in}}%
\pgfusepath{clip}%
\pgfsetrectcap%
\pgfsetroundjoin%
\pgfsetlinewidth{0.803000pt}%
\definecolor{currentstroke}{rgb}{0.850000,0.850000,0.850000}%
\pgfsetstrokecolor{currentstroke}%
\pgfsetdash{}{0pt}%
\pgfpathmoveto{\pgfqpoint{1.488154in}{0.417642in}}%
\pgfpathlineto{\pgfqpoint{1.488154in}{3.348330in}}%
\pgfusepath{stroke}%
\end{pgfscope}%
\begin{pgfscope}%
\pgfsetbuttcap%
\pgfsetroundjoin%
\definecolor{currentfill}{rgb}{0.000000,0.000000,0.000000}%
\pgfsetfillcolor{currentfill}%
\pgfsetlinewidth{0.602250pt}%
\definecolor{currentstroke}{rgb}{0.000000,0.000000,0.000000}%
\pgfsetstrokecolor{currentstroke}%
\pgfsetdash{}{0pt}%
\pgfsys@defobject{currentmarker}{\pgfqpoint{0.000000in}{-0.027778in}}{\pgfqpoint{0.000000in}{0.000000in}}{%
\pgfpathmoveto{\pgfqpoint{0.000000in}{0.000000in}}%
\pgfpathlineto{\pgfqpoint{0.000000in}{-0.027778in}}%
\pgfusepath{stroke,fill}%
}%
\begin{pgfscope}%
\pgfsys@transformshift{1.488154in}{0.417642in}%
\pgfsys@useobject{currentmarker}{}%
\end{pgfscope}%
\end{pgfscope}%
\begin{pgfscope}%
\pgfpathrectangle{\pgfqpoint{0.605343in}{0.417642in}}{\pgfqpoint{4.842987in}{2.930688in}}%
\pgfusepath{clip}%
\pgfsetrectcap%
\pgfsetroundjoin%
\pgfsetlinewidth{0.803000pt}%
\definecolor{currentstroke}{rgb}{0.850000,0.850000,0.850000}%
\pgfsetstrokecolor{currentstroke}%
\pgfsetdash{}{0pt}%
\pgfpathmoveto{\pgfqpoint{1.525689in}{0.417642in}}%
\pgfpathlineto{\pgfqpoint{1.525689in}{3.348330in}}%
\pgfusepath{stroke}%
\end{pgfscope}%
\begin{pgfscope}%
\pgfsetbuttcap%
\pgfsetroundjoin%
\definecolor{currentfill}{rgb}{0.000000,0.000000,0.000000}%
\pgfsetfillcolor{currentfill}%
\pgfsetlinewidth{0.602250pt}%
\definecolor{currentstroke}{rgb}{0.000000,0.000000,0.000000}%
\pgfsetstrokecolor{currentstroke}%
\pgfsetdash{}{0pt}%
\pgfsys@defobject{currentmarker}{\pgfqpoint{0.000000in}{-0.027778in}}{\pgfqpoint{0.000000in}{0.000000in}}{%
\pgfpathmoveto{\pgfqpoint{0.000000in}{0.000000in}}%
\pgfpathlineto{\pgfqpoint{0.000000in}{-0.027778in}}%
\pgfusepath{stroke,fill}%
}%
\begin{pgfscope}%
\pgfsys@transformshift{1.525689in}{0.417642in}%
\pgfsys@useobject{currentmarker}{}%
\end{pgfscope}%
\end{pgfscope}%
\begin{pgfscope}%
\pgfpathrectangle{\pgfqpoint{0.605343in}{0.417642in}}{\pgfqpoint{4.842987in}{2.930688in}}%
\pgfusepath{clip}%
\pgfsetrectcap%
\pgfsetroundjoin%
\pgfsetlinewidth{0.803000pt}%
\definecolor{currentstroke}{rgb}{0.850000,0.850000,0.850000}%
\pgfsetstrokecolor{currentstroke}%
\pgfsetdash{}{0pt}%
\pgfpathmoveto{\pgfqpoint{1.780157in}{0.417642in}}%
\pgfpathlineto{\pgfqpoint{1.780157in}{3.348330in}}%
\pgfusepath{stroke}%
\end{pgfscope}%
\begin{pgfscope}%
\pgfsetbuttcap%
\pgfsetroundjoin%
\definecolor{currentfill}{rgb}{0.000000,0.000000,0.000000}%
\pgfsetfillcolor{currentfill}%
\pgfsetlinewidth{0.602250pt}%
\definecolor{currentstroke}{rgb}{0.000000,0.000000,0.000000}%
\pgfsetstrokecolor{currentstroke}%
\pgfsetdash{}{0pt}%
\pgfsys@defobject{currentmarker}{\pgfqpoint{0.000000in}{-0.027778in}}{\pgfqpoint{0.000000in}{0.000000in}}{%
\pgfpathmoveto{\pgfqpoint{0.000000in}{0.000000in}}%
\pgfpathlineto{\pgfqpoint{0.000000in}{-0.027778in}}%
\pgfusepath{stroke,fill}%
}%
\begin{pgfscope}%
\pgfsys@transformshift{1.780157in}{0.417642in}%
\pgfsys@useobject{currentmarker}{}%
\end{pgfscope}%
\end{pgfscope}%
\begin{pgfscope}%
\pgfpathrectangle{\pgfqpoint{0.605343in}{0.417642in}}{\pgfqpoint{4.842987in}{2.930688in}}%
\pgfusepath{clip}%
\pgfsetrectcap%
\pgfsetroundjoin%
\pgfsetlinewidth{0.803000pt}%
\definecolor{currentstroke}{rgb}{0.850000,0.850000,0.850000}%
\pgfsetstrokecolor{currentstroke}%
\pgfsetdash{}{0pt}%
\pgfpathmoveto{\pgfqpoint{1.909370in}{0.417642in}}%
\pgfpathlineto{\pgfqpoint{1.909370in}{3.348330in}}%
\pgfusepath{stroke}%
\end{pgfscope}%
\begin{pgfscope}%
\pgfsetbuttcap%
\pgfsetroundjoin%
\definecolor{currentfill}{rgb}{0.000000,0.000000,0.000000}%
\pgfsetfillcolor{currentfill}%
\pgfsetlinewidth{0.602250pt}%
\definecolor{currentstroke}{rgb}{0.000000,0.000000,0.000000}%
\pgfsetstrokecolor{currentstroke}%
\pgfsetdash{}{0pt}%
\pgfsys@defobject{currentmarker}{\pgfqpoint{0.000000in}{-0.027778in}}{\pgfqpoint{0.000000in}{0.000000in}}{%
\pgfpathmoveto{\pgfqpoint{0.000000in}{0.000000in}}%
\pgfpathlineto{\pgfqpoint{0.000000in}{-0.027778in}}%
\pgfusepath{stroke,fill}%
}%
\begin{pgfscope}%
\pgfsys@transformshift{1.909370in}{0.417642in}%
\pgfsys@useobject{currentmarker}{}%
\end{pgfscope}%
\end{pgfscope}%
\begin{pgfscope}%
\pgfpathrectangle{\pgfqpoint{0.605343in}{0.417642in}}{\pgfqpoint{4.842987in}{2.930688in}}%
\pgfusepath{clip}%
\pgfsetrectcap%
\pgfsetroundjoin%
\pgfsetlinewidth{0.803000pt}%
\definecolor{currentstroke}{rgb}{0.850000,0.850000,0.850000}%
\pgfsetstrokecolor{currentstroke}%
\pgfsetdash{}{0pt}%
\pgfpathmoveto{\pgfqpoint{2.001048in}{0.417642in}}%
\pgfpathlineto{\pgfqpoint{2.001048in}{3.348330in}}%
\pgfusepath{stroke}%
\end{pgfscope}%
\begin{pgfscope}%
\pgfsetbuttcap%
\pgfsetroundjoin%
\definecolor{currentfill}{rgb}{0.000000,0.000000,0.000000}%
\pgfsetfillcolor{currentfill}%
\pgfsetlinewidth{0.602250pt}%
\definecolor{currentstroke}{rgb}{0.000000,0.000000,0.000000}%
\pgfsetstrokecolor{currentstroke}%
\pgfsetdash{}{0pt}%
\pgfsys@defobject{currentmarker}{\pgfqpoint{0.000000in}{-0.027778in}}{\pgfqpoint{0.000000in}{0.000000in}}{%
\pgfpathmoveto{\pgfqpoint{0.000000in}{0.000000in}}%
\pgfpathlineto{\pgfqpoint{0.000000in}{-0.027778in}}%
\pgfusepath{stroke,fill}%
}%
\begin{pgfscope}%
\pgfsys@transformshift{2.001048in}{0.417642in}%
\pgfsys@useobject{currentmarker}{}%
\end{pgfscope}%
\end{pgfscope}%
\begin{pgfscope}%
\pgfpathrectangle{\pgfqpoint{0.605343in}{0.417642in}}{\pgfqpoint{4.842987in}{2.930688in}}%
\pgfusepath{clip}%
\pgfsetrectcap%
\pgfsetroundjoin%
\pgfsetlinewidth{0.803000pt}%
\definecolor{currentstroke}{rgb}{0.850000,0.850000,0.850000}%
\pgfsetstrokecolor{currentstroke}%
\pgfsetdash{}{0pt}%
\pgfpathmoveto{\pgfqpoint{2.072159in}{0.417642in}}%
\pgfpathlineto{\pgfqpoint{2.072159in}{3.348330in}}%
\pgfusepath{stroke}%
\end{pgfscope}%
\begin{pgfscope}%
\pgfsetbuttcap%
\pgfsetroundjoin%
\definecolor{currentfill}{rgb}{0.000000,0.000000,0.000000}%
\pgfsetfillcolor{currentfill}%
\pgfsetlinewidth{0.602250pt}%
\definecolor{currentstroke}{rgb}{0.000000,0.000000,0.000000}%
\pgfsetstrokecolor{currentstroke}%
\pgfsetdash{}{0pt}%
\pgfsys@defobject{currentmarker}{\pgfqpoint{0.000000in}{-0.027778in}}{\pgfqpoint{0.000000in}{0.000000in}}{%
\pgfpathmoveto{\pgfqpoint{0.000000in}{0.000000in}}%
\pgfpathlineto{\pgfqpoint{0.000000in}{-0.027778in}}%
\pgfusepath{stroke,fill}%
}%
\begin{pgfscope}%
\pgfsys@transformshift{2.072159in}{0.417642in}%
\pgfsys@useobject{currentmarker}{}%
\end{pgfscope}%
\end{pgfscope}%
\begin{pgfscope}%
\pgfpathrectangle{\pgfqpoint{0.605343in}{0.417642in}}{\pgfqpoint{4.842987in}{2.930688in}}%
\pgfusepath{clip}%
\pgfsetrectcap%
\pgfsetroundjoin%
\pgfsetlinewidth{0.803000pt}%
\definecolor{currentstroke}{rgb}{0.850000,0.850000,0.850000}%
\pgfsetstrokecolor{currentstroke}%
\pgfsetdash{}{0pt}%
\pgfpathmoveto{\pgfqpoint{2.130261in}{0.417642in}}%
\pgfpathlineto{\pgfqpoint{2.130261in}{3.348330in}}%
\pgfusepath{stroke}%
\end{pgfscope}%
\begin{pgfscope}%
\pgfsetbuttcap%
\pgfsetroundjoin%
\definecolor{currentfill}{rgb}{0.000000,0.000000,0.000000}%
\pgfsetfillcolor{currentfill}%
\pgfsetlinewidth{0.602250pt}%
\definecolor{currentstroke}{rgb}{0.000000,0.000000,0.000000}%
\pgfsetstrokecolor{currentstroke}%
\pgfsetdash{}{0pt}%
\pgfsys@defobject{currentmarker}{\pgfqpoint{0.000000in}{-0.027778in}}{\pgfqpoint{0.000000in}{0.000000in}}{%
\pgfpathmoveto{\pgfqpoint{0.000000in}{0.000000in}}%
\pgfpathlineto{\pgfqpoint{0.000000in}{-0.027778in}}%
\pgfusepath{stroke,fill}%
}%
\begin{pgfscope}%
\pgfsys@transformshift{2.130261in}{0.417642in}%
\pgfsys@useobject{currentmarker}{}%
\end{pgfscope}%
\end{pgfscope}%
\begin{pgfscope}%
\pgfpathrectangle{\pgfqpoint{0.605343in}{0.417642in}}{\pgfqpoint{4.842987in}{2.930688in}}%
\pgfusepath{clip}%
\pgfsetrectcap%
\pgfsetroundjoin%
\pgfsetlinewidth{0.803000pt}%
\definecolor{currentstroke}{rgb}{0.850000,0.850000,0.850000}%
\pgfsetstrokecolor{currentstroke}%
\pgfsetdash{}{0pt}%
\pgfpathmoveto{\pgfqpoint{2.179386in}{0.417642in}}%
\pgfpathlineto{\pgfqpoint{2.179386in}{3.348330in}}%
\pgfusepath{stroke}%
\end{pgfscope}%
\begin{pgfscope}%
\pgfsetbuttcap%
\pgfsetroundjoin%
\definecolor{currentfill}{rgb}{0.000000,0.000000,0.000000}%
\pgfsetfillcolor{currentfill}%
\pgfsetlinewidth{0.602250pt}%
\definecolor{currentstroke}{rgb}{0.000000,0.000000,0.000000}%
\pgfsetstrokecolor{currentstroke}%
\pgfsetdash{}{0pt}%
\pgfsys@defobject{currentmarker}{\pgfqpoint{0.000000in}{-0.027778in}}{\pgfqpoint{0.000000in}{0.000000in}}{%
\pgfpathmoveto{\pgfqpoint{0.000000in}{0.000000in}}%
\pgfpathlineto{\pgfqpoint{0.000000in}{-0.027778in}}%
\pgfusepath{stroke,fill}%
}%
\begin{pgfscope}%
\pgfsys@transformshift{2.179386in}{0.417642in}%
\pgfsys@useobject{currentmarker}{}%
\end{pgfscope}%
\end{pgfscope}%
\begin{pgfscope}%
\pgfpathrectangle{\pgfqpoint{0.605343in}{0.417642in}}{\pgfqpoint{4.842987in}{2.930688in}}%
\pgfusepath{clip}%
\pgfsetrectcap%
\pgfsetroundjoin%
\pgfsetlinewidth{0.803000pt}%
\definecolor{currentstroke}{rgb}{0.850000,0.850000,0.850000}%
\pgfsetstrokecolor{currentstroke}%
\pgfsetdash{}{0pt}%
\pgfpathmoveto{\pgfqpoint{2.221940in}{0.417642in}}%
\pgfpathlineto{\pgfqpoint{2.221940in}{3.348330in}}%
\pgfusepath{stroke}%
\end{pgfscope}%
\begin{pgfscope}%
\pgfsetbuttcap%
\pgfsetroundjoin%
\definecolor{currentfill}{rgb}{0.000000,0.000000,0.000000}%
\pgfsetfillcolor{currentfill}%
\pgfsetlinewidth{0.602250pt}%
\definecolor{currentstroke}{rgb}{0.000000,0.000000,0.000000}%
\pgfsetstrokecolor{currentstroke}%
\pgfsetdash{}{0pt}%
\pgfsys@defobject{currentmarker}{\pgfqpoint{0.000000in}{-0.027778in}}{\pgfqpoint{0.000000in}{0.000000in}}{%
\pgfpathmoveto{\pgfqpoint{0.000000in}{0.000000in}}%
\pgfpathlineto{\pgfqpoint{0.000000in}{-0.027778in}}%
\pgfusepath{stroke,fill}%
}%
\begin{pgfscope}%
\pgfsys@transformshift{2.221940in}{0.417642in}%
\pgfsys@useobject{currentmarker}{}%
\end{pgfscope}%
\end{pgfscope}%
\begin{pgfscope}%
\pgfpathrectangle{\pgfqpoint{0.605343in}{0.417642in}}{\pgfqpoint{4.842987in}{2.930688in}}%
\pgfusepath{clip}%
\pgfsetrectcap%
\pgfsetroundjoin%
\pgfsetlinewidth{0.803000pt}%
\definecolor{currentstroke}{rgb}{0.850000,0.850000,0.850000}%
\pgfsetstrokecolor{currentstroke}%
\pgfsetdash{}{0pt}%
\pgfpathmoveto{\pgfqpoint{2.259475in}{0.417642in}}%
\pgfpathlineto{\pgfqpoint{2.259475in}{3.348330in}}%
\pgfusepath{stroke}%
\end{pgfscope}%
\begin{pgfscope}%
\pgfsetbuttcap%
\pgfsetroundjoin%
\definecolor{currentfill}{rgb}{0.000000,0.000000,0.000000}%
\pgfsetfillcolor{currentfill}%
\pgfsetlinewidth{0.602250pt}%
\definecolor{currentstroke}{rgb}{0.000000,0.000000,0.000000}%
\pgfsetstrokecolor{currentstroke}%
\pgfsetdash{}{0pt}%
\pgfsys@defobject{currentmarker}{\pgfqpoint{0.000000in}{-0.027778in}}{\pgfqpoint{0.000000in}{0.000000in}}{%
\pgfpathmoveto{\pgfqpoint{0.000000in}{0.000000in}}%
\pgfpathlineto{\pgfqpoint{0.000000in}{-0.027778in}}%
\pgfusepath{stroke,fill}%
}%
\begin{pgfscope}%
\pgfsys@transformshift{2.259475in}{0.417642in}%
\pgfsys@useobject{currentmarker}{}%
\end{pgfscope}%
\end{pgfscope}%
\begin{pgfscope}%
\pgfpathrectangle{\pgfqpoint{0.605343in}{0.417642in}}{\pgfqpoint{4.842987in}{2.930688in}}%
\pgfusepath{clip}%
\pgfsetrectcap%
\pgfsetroundjoin%
\pgfsetlinewidth{0.803000pt}%
\definecolor{currentstroke}{rgb}{0.850000,0.850000,0.850000}%
\pgfsetstrokecolor{currentstroke}%
\pgfsetdash{}{0pt}%
\pgfpathmoveto{\pgfqpoint{2.513942in}{0.417642in}}%
\pgfpathlineto{\pgfqpoint{2.513942in}{3.348330in}}%
\pgfusepath{stroke}%
\end{pgfscope}%
\begin{pgfscope}%
\pgfsetbuttcap%
\pgfsetroundjoin%
\definecolor{currentfill}{rgb}{0.000000,0.000000,0.000000}%
\pgfsetfillcolor{currentfill}%
\pgfsetlinewidth{0.602250pt}%
\definecolor{currentstroke}{rgb}{0.000000,0.000000,0.000000}%
\pgfsetstrokecolor{currentstroke}%
\pgfsetdash{}{0pt}%
\pgfsys@defobject{currentmarker}{\pgfqpoint{0.000000in}{-0.027778in}}{\pgfqpoint{0.000000in}{0.000000in}}{%
\pgfpathmoveto{\pgfqpoint{0.000000in}{0.000000in}}%
\pgfpathlineto{\pgfqpoint{0.000000in}{-0.027778in}}%
\pgfusepath{stroke,fill}%
}%
\begin{pgfscope}%
\pgfsys@transformshift{2.513942in}{0.417642in}%
\pgfsys@useobject{currentmarker}{}%
\end{pgfscope}%
\end{pgfscope}%
\begin{pgfscope}%
\pgfpathrectangle{\pgfqpoint{0.605343in}{0.417642in}}{\pgfqpoint{4.842987in}{2.930688in}}%
\pgfusepath{clip}%
\pgfsetrectcap%
\pgfsetroundjoin%
\pgfsetlinewidth{0.803000pt}%
\definecolor{currentstroke}{rgb}{0.850000,0.850000,0.850000}%
\pgfsetstrokecolor{currentstroke}%
\pgfsetdash{}{0pt}%
\pgfpathmoveto{\pgfqpoint{2.643156in}{0.417642in}}%
\pgfpathlineto{\pgfqpoint{2.643156in}{3.348330in}}%
\pgfusepath{stroke}%
\end{pgfscope}%
\begin{pgfscope}%
\pgfsetbuttcap%
\pgfsetroundjoin%
\definecolor{currentfill}{rgb}{0.000000,0.000000,0.000000}%
\pgfsetfillcolor{currentfill}%
\pgfsetlinewidth{0.602250pt}%
\definecolor{currentstroke}{rgb}{0.000000,0.000000,0.000000}%
\pgfsetstrokecolor{currentstroke}%
\pgfsetdash{}{0pt}%
\pgfsys@defobject{currentmarker}{\pgfqpoint{0.000000in}{-0.027778in}}{\pgfqpoint{0.000000in}{0.000000in}}{%
\pgfpathmoveto{\pgfqpoint{0.000000in}{0.000000in}}%
\pgfpathlineto{\pgfqpoint{0.000000in}{-0.027778in}}%
\pgfusepath{stroke,fill}%
}%
\begin{pgfscope}%
\pgfsys@transformshift{2.643156in}{0.417642in}%
\pgfsys@useobject{currentmarker}{}%
\end{pgfscope}%
\end{pgfscope}%
\begin{pgfscope}%
\pgfpathrectangle{\pgfqpoint{0.605343in}{0.417642in}}{\pgfqpoint{4.842987in}{2.930688in}}%
\pgfusepath{clip}%
\pgfsetrectcap%
\pgfsetroundjoin%
\pgfsetlinewidth{0.803000pt}%
\definecolor{currentstroke}{rgb}{0.850000,0.850000,0.850000}%
\pgfsetstrokecolor{currentstroke}%
\pgfsetdash{}{0pt}%
\pgfpathmoveto{\pgfqpoint{2.734834in}{0.417642in}}%
\pgfpathlineto{\pgfqpoint{2.734834in}{3.348330in}}%
\pgfusepath{stroke}%
\end{pgfscope}%
\begin{pgfscope}%
\pgfsetbuttcap%
\pgfsetroundjoin%
\definecolor{currentfill}{rgb}{0.000000,0.000000,0.000000}%
\pgfsetfillcolor{currentfill}%
\pgfsetlinewidth{0.602250pt}%
\definecolor{currentstroke}{rgb}{0.000000,0.000000,0.000000}%
\pgfsetstrokecolor{currentstroke}%
\pgfsetdash{}{0pt}%
\pgfsys@defobject{currentmarker}{\pgfqpoint{0.000000in}{-0.027778in}}{\pgfqpoint{0.000000in}{0.000000in}}{%
\pgfpathmoveto{\pgfqpoint{0.000000in}{0.000000in}}%
\pgfpathlineto{\pgfqpoint{0.000000in}{-0.027778in}}%
\pgfusepath{stroke,fill}%
}%
\begin{pgfscope}%
\pgfsys@transformshift{2.734834in}{0.417642in}%
\pgfsys@useobject{currentmarker}{}%
\end{pgfscope}%
\end{pgfscope}%
\begin{pgfscope}%
\pgfpathrectangle{\pgfqpoint{0.605343in}{0.417642in}}{\pgfqpoint{4.842987in}{2.930688in}}%
\pgfusepath{clip}%
\pgfsetrectcap%
\pgfsetroundjoin%
\pgfsetlinewidth{0.803000pt}%
\definecolor{currentstroke}{rgb}{0.850000,0.850000,0.850000}%
\pgfsetstrokecolor{currentstroke}%
\pgfsetdash{}{0pt}%
\pgfpathmoveto{\pgfqpoint{2.805945in}{0.417642in}}%
\pgfpathlineto{\pgfqpoint{2.805945in}{3.348330in}}%
\pgfusepath{stroke}%
\end{pgfscope}%
\begin{pgfscope}%
\pgfsetbuttcap%
\pgfsetroundjoin%
\definecolor{currentfill}{rgb}{0.000000,0.000000,0.000000}%
\pgfsetfillcolor{currentfill}%
\pgfsetlinewidth{0.602250pt}%
\definecolor{currentstroke}{rgb}{0.000000,0.000000,0.000000}%
\pgfsetstrokecolor{currentstroke}%
\pgfsetdash{}{0pt}%
\pgfsys@defobject{currentmarker}{\pgfqpoint{0.000000in}{-0.027778in}}{\pgfqpoint{0.000000in}{0.000000in}}{%
\pgfpathmoveto{\pgfqpoint{0.000000in}{0.000000in}}%
\pgfpathlineto{\pgfqpoint{0.000000in}{-0.027778in}}%
\pgfusepath{stroke,fill}%
}%
\begin{pgfscope}%
\pgfsys@transformshift{2.805945in}{0.417642in}%
\pgfsys@useobject{currentmarker}{}%
\end{pgfscope}%
\end{pgfscope}%
\begin{pgfscope}%
\pgfpathrectangle{\pgfqpoint{0.605343in}{0.417642in}}{\pgfqpoint{4.842987in}{2.930688in}}%
\pgfusepath{clip}%
\pgfsetrectcap%
\pgfsetroundjoin%
\pgfsetlinewidth{0.803000pt}%
\definecolor{currentstroke}{rgb}{0.850000,0.850000,0.850000}%
\pgfsetstrokecolor{currentstroke}%
\pgfsetdash{}{0pt}%
\pgfpathmoveto{\pgfqpoint{2.864047in}{0.417642in}}%
\pgfpathlineto{\pgfqpoint{2.864047in}{3.348330in}}%
\pgfusepath{stroke}%
\end{pgfscope}%
\begin{pgfscope}%
\pgfsetbuttcap%
\pgfsetroundjoin%
\definecolor{currentfill}{rgb}{0.000000,0.000000,0.000000}%
\pgfsetfillcolor{currentfill}%
\pgfsetlinewidth{0.602250pt}%
\definecolor{currentstroke}{rgb}{0.000000,0.000000,0.000000}%
\pgfsetstrokecolor{currentstroke}%
\pgfsetdash{}{0pt}%
\pgfsys@defobject{currentmarker}{\pgfqpoint{0.000000in}{-0.027778in}}{\pgfqpoint{0.000000in}{0.000000in}}{%
\pgfpathmoveto{\pgfqpoint{0.000000in}{0.000000in}}%
\pgfpathlineto{\pgfqpoint{0.000000in}{-0.027778in}}%
\pgfusepath{stroke,fill}%
}%
\begin{pgfscope}%
\pgfsys@transformshift{2.864047in}{0.417642in}%
\pgfsys@useobject{currentmarker}{}%
\end{pgfscope}%
\end{pgfscope}%
\begin{pgfscope}%
\pgfpathrectangle{\pgfqpoint{0.605343in}{0.417642in}}{\pgfqpoint{4.842987in}{2.930688in}}%
\pgfusepath{clip}%
\pgfsetrectcap%
\pgfsetroundjoin%
\pgfsetlinewidth{0.803000pt}%
\definecolor{currentstroke}{rgb}{0.850000,0.850000,0.850000}%
\pgfsetstrokecolor{currentstroke}%
\pgfsetdash{}{0pt}%
\pgfpathmoveto{\pgfqpoint{2.913172in}{0.417642in}}%
\pgfpathlineto{\pgfqpoint{2.913172in}{3.348330in}}%
\pgfusepath{stroke}%
\end{pgfscope}%
\begin{pgfscope}%
\pgfsetbuttcap%
\pgfsetroundjoin%
\definecolor{currentfill}{rgb}{0.000000,0.000000,0.000000}%
\pgfsetfillcolor{currentfill}%
\pgfsetlinewidth{0.602250pt}%
\definecolor{currentstroke}{rgb}{0.000000,0.000000,0.000000}%
\pgfsetstrokecolor{currentstroke}%
\pgfsetdash{}{0pt}%
\pgfsys@defobject{currentmarker}{\pgfqpoint{0.000000in}{-0.027778in}}{\pgfqpoint{0.000000in}{0.000000in}}{%
\pgfpathmoveto{\pgfqpoint{0.000000in}{0.000000in}}%
\pgfpathlineto{\pgfqpoint{0.000000in}{-0.027778in}}%
\pgfusepath{stroke,fill}%
}%
\begin{pgfscope}%
\pgfsys@transformshift{2.913172in}{0.417642in}%
\pgfsys@useobject{currentmarker}{}%
\end{pgfscope}%
\end{pgfscope}%
\begin{pgfscope}%
\pgfpathrectangle{\pgfqpoint{0.605343in}{0.417642in}}{\pgfqpoint{4.842987in}{2.930688in}}%
\pgfusepath{clip}%
\pgfsetrectcap%
\pgfsetroundjoin%
\pgfsetlinewidth{0.803000pt}%
\definecolor{currentstroke}{rgb}{0.850000,0.850000,0.850000}%
\pgfsetstrokecolor{currentstroke}%
\pgfsetdash{}{0pt}%
\pgfpathmoveto{\pgfqpoint{2.955726in}{0.417642in}}%
\pgfpathlineto{\pgfqpoint{2.955726in}{3.348330in}}%
\pgfusepath{stroke}%
\end{pgfscope}%
\begin{pgfscope}%
\pgfsetbuttcap%
\pgfsetroundjoin%
\definecolor{currentfill}{rgb}{0.000000,0.000000,0.000000}%
\pgfsetfillcolor{currentfill}%
\pgfsetlinewidth{0.602250pt}%
\definecolor{currentstroke}{rgb}{0.000000,0.000000,0.000000}%
\pgfsetstrokecolor{currentstroke}%
\pgfsetdash{}{0pt}%
\pgfsys@defobject{currentmarker}{\pgfqpoint{0.000000in}{-0.027778in}}{\pgfqpoint{0.000000in}{0.000000in}}{%
\pgfpathmoveto{\pgfqpoint{0.000000in}{0.000000in}}%
\pgfpathlineto{\pgfqpoint{0.000000in}{-0.027778in}}%
\pgfusepath{stroke,fill}%
}%
\begin{pgfscope}%
\pgfsys@transformshift{2.955726in}{0.417642in}%
\pgfsys@useobject{currentmarker}{}%
\end{pgfscope}%
\end{pgfscope}%
\begin{pgfscope}%
\pgfpathrectangle{\pgfqpoint{0.605343in}{0.417642in}}{\pgfqpoint{4.842987in}{2.930688in}}%
\pgfusepath{clip}%
\pgfsetrectcap%
\pgfsetroundjoin%
\pgfsetlinewidth{0.803000pt}%
\definecolor{currentstroke}{rgb}{0.850000,0.850000,0.850000}%
\pgfsetstrokecolor{currentstroke}%
\pgfsetdash{}{0pt}%
\pgfpathmoveto{\pgfqpoint{2.993261in}{0.417642in}}%
\pgfpathlineto{\pgfqpoint{2.993261in}{3.348330in}}%
\pgfusepath{stroke}%
\end{pgfscope}%
\begin{pgfscope}%
\pgfsetbuttcap%
\pgfsetroundjoin%
\definecolor{currentfill}{rgb}{0.000000,0.000000,0.000000}%
\pgfsetfillcolor{currentfill}%
\pgfsetlinewidth{0.602250pt}%
\definecolor{currentstroke}{rgb}{0.000000,0.000000,0.000000}%
\pgfsetstrokecolor{currentstroke}%
\pgfsetdash{}{0pt}%
\pgfsys@defobject{currentmarker}{\pgfqpoint{0.000000in}{-0.027778in}}{\pgfqpoint{0.000000in}{0.000000in}}{%
\pgfpathmoveto{\pgfqpoint{0.000000in}{0.000000in}}%
\pgfpathlineto{\pgfqpoint{0.000000in}{-0.027778in}}%
\pgfusepath{stroke,fill}%
}%
\begin{pgfscope}%
\pgfsys@transformshift{2.993261in}{0.417642in}%
\pgfsys@useobject{currentmarker}{}%
\end{pgfscope}%
\end{pgfscope}%
\begin{pgfscope}%
\pgfpathrectangle{\pgfqpoint{0.605343in}{0.417642in}}{\pgfqpoint{4.842987in}{2.930688in}}%
\pgfusepath{clip}%
\pgfsetrectcap%
\pgfsetroundjoin%
\pgfsetlinewidth{0.803000pt}%
\definecolor{currentstroke}{rgb}{0.850000,0.850000,0.850000}%
\pgfsetstrokecolor{currentstroke}%
\pgfsetdash{}{0pt}%
\pgfpathmoveto{\pgfqpoint{3.247728in}{0.417642in}}%
\pgfpathlineto{\pgfqpoint{3.247728in}{3.348330in}}%
\pgfusepath{stroke}%
\end{pgfscope}%
\begin{pgfscope}%
\pgfsetbuttcap%
\pgfsetroundjoin%
\definecolor{currentfill}{rgb}{0.000000,0.000000,0.000000}%
\pgfsetfillcolor{currentfill}%
\pgfsetlinewidth{0.602250pt}%
\definecolor{currentstroke}{rgb}{0.000000,0.000000,0.000000}%
\pgfsetstrokecolor{currentstroke}%
\pgfsetdash{}{0pt}%
\pgfsys@defobject{currentmarker}{\pgfqpoint{0.000000in}{-0.027778in}}{\pgfqpoint{0.000000in}{0.000000in}}{%
\pgfpathmoveto{\pgfqpoint{0.000000in}{0.000000in}}%
\pgfpathlineto{\pgfqpoint{0.000000in}{-0.027778in}}%
\pgfusepath{stroke,fill}%
}%
\begin{pgfscope}%
\pgfsys@transformshift{3.247728in}{0.417642in}%
\pgfsys@useobject{currentmarker}{}%
\end{pgfscope}%
\end{pgfscope}%
\begin{pgfscope}%
\pgfpathrectangle{\pgfqpoint{0.605343in}{0.417642in}}{\pgfqpoint{4.842987in}{2.930688in}}%
\pgfusepath{clip}%
\pgfsetrectcap%
\pgfsetroundjoin%
\pgfsetlinewidth{0.803000pt}%
\definecolor{currentstroke}{rgb}{0.850000,0.850000,0.850000}%
\pgfsetstrokecolor{currentstroke}%
\pgfsetdash{}{0pt}%
\pgfpathmoveto{\pgfqpoint{3.376942in}{0.417642in}}%
\pgfpathlineto{\pgfqpoint{3.376942in}{3.348330in}}%
\pgfusepath{stroke}%
\end{pgfscope}%
\begin{pgfscope}%
\pgfsetbuttcap%
\pgfsetroundjoin%
\definecolor{currentfill}{rgb}{0.000000,0.000000,0.000000}%
\pgfsetfillcolor{currentfill}%
\pgfsetlinewidth{0.602250pt}%
\definecolor{currentstroke}{rgb}{0.000000,0.000000,0.000000}%
\pgfsetstrokecolor{currentstroke}%
\pgfsetdash{}{0pt}%
\pgfsys@defobject{currentmarker}{\pgfqpoint{0.000000in}{-0.027778in}}{\pgfqpoint{0.000000in}{0.000000in}}{%
\pgfpathmoveto{\pgfqpoint{0.000000in}{0.000000in}}%
\pgfpathlineto{\pgfqpoint{0.000000in}{-0.027778in}}%
\pgfusepath{stroke,fill}%
}%
\begin{pgfscope}%
\pgfsys@transformshift{3.376942in}{0.417642in}%
\pgfsys@useobject{currentmarker}{}%
\end{pgfscope}%
\end{pgfscope}%
\begin{pgfscope}%
\pgfpathrectangle{\pgfqpoint{0.605343in}{0.417642in}}{\pgfqpoint{4.842987in}{2.930688in}}%
\pgfusepath{clip}%
\pgfsetrectcap%
\pgfsetroundjoin%
\pgfsetlinewidth{0.803000pt}%
\definecolor{currentstroke}{rgb}{0.850000,0.850000,0.850000}%
\pgfsetstrokecolor{currentstroke}%
\pgfsetdash{}{0pt}%
\pgfpathmoveto{\pgfqpoint{3.468620in}{0.417642in}}%
\pgfpathlineto{\pgfqpoint{3.468620in}{3.348330in}}%
\pgfusepath{stroke}%
\end{pgfscope}%
\begin{pgfscope}%
\pgfsetbuttcap%
\pgfsetroundjoin%
\definecolor{currentfill}{rgb}{0.000000,0.000000,0.000000}%
\pgfsetfillcolor{currentfill}%
\pgfsetlinewidth{0.602250pt}%
\definecolor{currentstroke}{rgb}{0.000000,0.000000,0.000000}%
\pgfsetstrokecolor{currentstroke}%
\pgfsetdash{}{0pt}%
\pgfsys@defobject{currentmarker}{\pgfqpoint{0.000000in}{-0.027778in}}{\pgfqpoint{0.000000in}{0.000000in}}{%
\pgfpathmoveto{\pgfqpoint{0.000000in}{0.000000in}}%
\pgfpathlineto{\pgfqpoint{0.000000in}{-0.027778in}}%
\pgfusepath{stroke,fill}%
}%
\begin{pgfscope}%
\pgfsys@transformshift{3.468620in}{0.417642in}%
\pgfsys@useobject{currentmarker}{}%
\end{pgfscope}%
\end{pgfscope}%
\begin{pgfscope}%
\pgfpathrectangle{\pgfqpoint{0.605343in}{0.417642in}}{\pgfqpoint{4.842987in}{2.930688in}}%
\pgfusepath{clip}%
\pgfsetrectcap%
\pgfsetroundjoin%
\pgfsetlinewidth{0.803000pt}%
\definecolor{currentstroke}{rgb}{0.850000,0.850000,0.850000}%
\pgfsetstrokecolor{currentstroke}%
\pgfsetdash{}{0pt}%
\pgfpathmoveto{\pgfqpoint{3.539731in}{0.417642in}}%
\pgfpathlineto{\pgfqpoint{3.539731in}{3.348330in}}%
\pgfusepath{stroke}%
\end{pgfscope}%
\begin{pgfscope}%
\pgfsetbuttcap%
\pgfsetroundjoin%
\definecolor{currentfill}{rgb}{0.000000,0.000000,0.000000}%
\pgfsetfillcolor{currentfill}%
\pgfsetlinewidth{0.602250pt}%
\definecolor{currentstroke}{rgb}{0.000000,0.000000,0.000000}%
\pgfsetstrokecolor{currentstroke}%
\pgfsetdash{}{0pt}%
\pgfsys@defobject{currentmarker}{\pgfqpoint{0.000000in}{-0.027778in}}{\pgfqpoint{0.000000in}{0.000000in}}{%
\pgfpathmoveto{\pgfqpoint{0.000000in}{0.000000in}}%
\pgfpathlineto{\pgfqpoint{0.000000in}{-0.027778in}}%
\pgfusepath{stroke,fill}%
}%
\begin{pgfscope}%
\pgfsys@transformshift{3.539731in}{0.417642in}%
\pgfsys@useobject{currentmarker}{}%
\end{pgfscope}%
\end{pgfscope}%
\begin{pgfscope}%
\pgfpathrectangle{\pgfqpoint{0.605343in}{0.417642in}}{\pgfqpoint{4.842987in}{2.930688in}}%
\pgfusepath{clip}%
\pgfsetrectcap%
\pgfsetroundjoin%
\pgfsetlinewidth{0.803000pt}%
\definecolor{currentstroke}{rgb}{0.850000,0.850000,0.850000}%
\pgfsetstrokecolor{currentstroke}%
\pgfsetdash{}{0pt}%
\pgfpathmoveto{\pgfqpoint{3.597833in}{0.417642in}}%
\pgfpathlineto{\pgfqpoint{3.597833in}{3.348330in}}%
\pgfusepath{stroke}%
\end{pgfscope}%
\begin{pgfscope}%
\pgfsetbuttcap%
\pgfsetroundjoin%
\definecolor{currentfill}{rgb}{0.000000,0.000000,0.000000}%
\pgfsetfillcolor{currentfill}%
\pgfsetlinewidth{0.602250pt}%
\definecolor{currentstroke}{rgb}{0.000000,0.000000,0.000000}%
\pgfsetstrokecolor{currentstroke}%
\pgfsetdash{}{0pt}%
\pgfsys@defobject{currentmarker}{\pgfqpoint{0.000000in}{-0.027778in}}{\pgfqpoint{0.000000in}{0.000000in}}{%
\pgfpathmoveto{\pgfqpoint{0.000000in}{0.000000in}}%
\pgfpathlineto{\pgfqpoint{0.000000in}{-0.027778in}}%
\pgfusepath{stroke,fill}%
}%
\begin{pgfscope}%
\pgfsys@transformshift{3.597833in}{0.417642in}%
\pgfsys@useobject{currentmarker}{}%
\end{pgfscope}%
\end{pgfscope}%
\begin{pgfscope}%
\pgfpathrectangle{\pgfqpoint{0.605343in}{0.417642in}}{\pgfqpoint{4.842987in}{2.930688in}}%
\pgfusepath{clip}%
\pgfsetrectcap%
\pgfsetroundjoin%
\pgfsetlinewidth{0.803000pt}%
\definecolor{currentstroke}{rgb}{0.850000,0.850000,0.850000}%
\pgfsetstrokecolor{currentstroke}%
\pgfsetdash{}{0pt}%
\pgfpathmoveto{\pgfqpoint{3.646958in}{0.417642in}}%
\pgfpathlineto{\pgfqpoint{3.646958in}{3.348330in}}%
\pgfusepath{stroke}%
\end{pgfscope}%
\begin{pgfscope}%
\pgfsetbuttcap%
\pgfsetroundjoin%
\definecolor{currentfill}{rgb}{0.000000,0.000000,0.000000}%
\pgfsetfillcolor{currentfill}%
\pgfsetlinewidth{0.602250pt}%
\definecolor{currentstroke}{rgb}{0.000000,0.000000,0.000000}%
\pgfsetstrokecolor{currentstroke}%
\pgfsetdash{}{0pt}%
\pgfsys@defobject{currentmarker}{\pgfqpoint{0.000000in}{-0.027778in}}{\pgfqpoint{0.000000in}{0.000000in}}{%
\pgfpathmoveto{\pgfqpoint{0.000000in}{0.000000in}}%
\pgfpathlineto{\pgfqpoint{0.000000in}{-0.027778in}}%
\pgfusepath{stroke,fill}%
}%
\begin{pgfscope}%
\pgfsys@transformshift{3.646958in}{0.417642in}%
\pgfsys@useobject{currentmarker}{}%
\end{pgfscope}%
\end{pgfscope}%
\begin{pgfscope}%
\pgfpathrectangle{\pgfqpoint{0.605343in}{0.417642in}}{\pgfqpoint{4.842987in}{2.930688in}}%
\pgfusepath{clip}%
\pgfsetrectcap%
\pgfsetroundjoin%
\pgfsetlinewidth{0.803000pt}%
\definecolor{currentstroke}{rgb}{0.850000,0.850000,0.850000}%
\pgfsetstrokecolor{currentstroke}%
\pgfsetdash{}{0pt}%
\pgfpathmoveto{\pgfqpoint{3.689511in}{0.417642in}}%
\pgfpathlineto{\pgfqpoint{3.689511in}{3.348330in}}%
\pgfusepath{stroke}%
\end{pgfscope}%
\begin{pgfscope}%
\pgfsetbuttcap%
\pgfsetroundjoin%
\definecolor{currentfill}{rgb}{0.000000,0.000000,0.000000}%
\pgfsetfillcolor{currentfill}%
\pgfsetlinewidth{0.602250pt}%
\definecolor{currentstroke}{rgb}{0.000000,0.000000,0.000000}%
\pgfsetstrokecolor{currentstroke}%
\pgfsetdash{}{0pt}%
\pgfsys@defobject{currentmarker}{\pgfqpoint{0.000000in}{-0.027778in}}{\pgfqpoint{0.000000in}{0.000000in}}{%
\pgfpathmoveto{\pgfqpoint{0.000000in}{0.000000in}}%
\pgfpathlineto{\pgfqpoint{0.000000in}{-0.027778in}}%
\pgfusepath{stroke,fill}%
}%
\begin{pgfscope}%
\pgfsys@transformshift{3.689511in}{0.417642in}%
\pgfsys@useobject{currentmarker}{}%
\end{pgfscope}%
\end{pgfscope}%
\begin{pgfscope}%
\pgfpathrectangle{\pgfqpoint{0.605343in}{0.417642in}}{\pgfqpoint{4.842987in}{2.930688in}}%
\pgfusepath{clip}%
\pgfsetrectcap%
\pgfsetroundjoin%
\pgfsetlinewidth{0.803000pt}%
\definecolor{currentstroke}{rgb}{0.850000,0.850000,0.850000}%
\pgfsetstrokecolor{currentstroke}%
\pgfsetdash{}{0pt}%
\pgfpathmoveto{\pgfqpoint{3.727046in}{0.417642in}}%
\pgfpathlineto{\pgfqpoint{3.727046in}{3.348330in}}%
\pgfusepath{stroke}%
\end{pgfscope}%
\begin{pgfscope}%
\pgfsetbuttcap%
\pgfsetroundjoin%
\definecolor{currentfill}{rgb}{0.000000,0.000000,0.000000}%
\pgfsetfillcolor{currentfill}%
\pgfsetlinewidth{0.602250pt}%
\definecolor{currentstroke}{rgb}{0.000000,0.000000,0.000000}%
\pgfsetstrokecolor{currentstroke}%
\pgfsetdash{}{0pt}%
\pgfsys@defobject{currentmarker}{\pgfqpoint{0.000000in}{-0.027778in}}{\pgfqpoint{0.000000in}{0.000000in}}{%
\pgfpathmoveto{\pgfqpoint{0.000000in}{0.000000in}}%
\pgfpathlineto{\pgfqpoint{0.000000in}{-0.027778in}}%
\pgfusepath{stroke,fill}%
}%
\begin{pgfscope}%
\pgfsys@transformshift{3.727046in}{0.417642in}%
\pgfsys@useobject{currentmarker}{}%
\end{pgfscope}%
\end{pgfscope}%
\begin{pgfscope}%
\pgfpathrectangle{\pgfqpoint{0.605343in}{0.417642in}}{\pgfqpoint{4.842987in}{2.930688in}}%
\pgfusepath{clip}%
\pgfsetrectcap%
\pgfsetroundjoin%
\pgfsetlinewidth{0.803000pt}%
\definecolor{currentstroke}{rgb}{0.850000,0.850000,0.850000}%
\pgfsetstrokecolor{currentstroke}%
\pgfsetdash{}{0pt}%
\pgfpathmoveto{\pgfqpoint{3.981514in}{0.417642in}}%
\pgfpathlineto{\pgfqpoint{3.981514in}{3.348330in}}%
\pgfusepath{stroke}%
\end{pgfscope}%
\begin{pgfscope}%
\pgfsetbuttcap%
\pgfsetroundjoin%
\definecolor{currentfill}{rgb}{0.000000,0.000000,0.000000}%
\pgfsetfillcolor{currentfill}%
\pgfsetlinewidth{0.602250pt}%
\definecolor{currentstroke}{rgb}{0.000000,0.000000,0.000000}%
\pgfsetstrokecolor{currentstroke}%
\pgfsetdash{}{0pt}%
\pgfsys@defobject{currentmarker}{\pgfqpoint{0.000000in}{-0.027778in}}{\pgfqpoint{0.000000in}{0.000000in}}{%
\pgfpathmoveto{\pgfqpoint{0.000000in}{0.000000in}}%
\pgfpathlineto{\pgfqpoint{0.000000in}{-0.027778in}}%
\pgfusepath{stroke,fill}%
}%
\begin{pgfscope}%
\pgfsys@transformshift{3.981514in}{0.417642in}%
\pgfsys@useobject{currentmarker}{}%
\end{pgfscope}%
\end{pgfscope}%
\begin{pgfscope}%
\pgfpathrectangle{\pgfqpoint{0.605343in}{0.417642in}}{\pgfqpoint{4.842987in}{2.930688in}}%
\pgfusepath{clip}%
\pgfsetrectcap%
\pgfsetroundjoin%
\pgfsetlinewidth{0.803000pt}%
\definecolor{currentstroke}{rgb}{0.850000,0.850000,0.850000}%
\pgfsetstrokecolor{currentstroke}%
\pgfsetdash{}{0pt}%
\pgfpathmoveto{\pgfqpoint{4.110727in}{0.417642in}}%
\pgfpathlineto{\pgfqpoint{4.110727in}{3.348330in}}%
\pgfusepath{stroke}%
\end{pgfscope}%
\begin{pgfscope}%
\pgfsetbuttcap%
\pgfsetroundjoin%
\definecolor{currentfill}{rgb}{0.000000,0.000000,0.000000}%
\pgfsetfillcolor{currentfill}%
\pgfsetlinewidth{0.602250pt}%
\definecolor{currentstroke}{rgb}{0.000000,0.000000,0.000000}%
\pgfsetstrokecolor{currentstroke}%
\pgfsetdash{}{0pt}%
\pgfsys@defobject{currentmarker}{\pgfqpoint{0.000000in}{-0.027778in}}{\pgfqpoint{0.000000in}{0.000000in}}{%
\pgfpathmoveto{\pgfqpoint{0.000000in}{0.000000in}}%
\pgfpathlineto{\pgfqpoint{0.000000in}{-0.027778in}}%
\pgfusepath{stroke,fill}%
}%
\begin{pgfscope}%
\pgfsys@transformshift{4.110727in}{0.417642in}%
\pgfsys@useobject{currentmarker}{}%
\end{pgfscope}%
\end{pgfscope}%
\begin{pgfscope}%
\pgfpathrectangle{\pgfqpoint{0.605343in}{0.417642in}}{\pgfqpoint{4.842987in}{2.930688in}}%
\pgfusepath{clip}%
\pgfsetrectcap%
\pgfsetroundjoin%
\pgfsetlinewidth{0.803000pt}%
\definecolor{currentstroke}{rgb}{0.850000,0.850000,0.850000}%
\pgfsetstrokecolor{currentstroke}%
\pgfsetdash{}{0pt}%
\pgfpathmoveto{\pgfqpoint{4.202406in}{0.417642in}}%
\pgfpathlineto{\pgfqpoint{4.202406in}{3.348330in}}%
\pgfusepath{stroke}%
\end{pgfscope}%
\begin{pgfscope}%
\pgfsetbuttcap%
\pgfsetroundjoin%
\definecolor{currentfill}{rgb}{0.000000,0.000000,0.000000}%
\pgfsetfillcolor{currentfill}%
\pgfsetlinewidth{0.602250pt}%
\definecolor{currentstroke}{rgb}{0.000000,0.000000,0.000000}%
\pgfsetstrokecolor{currentstroke}%
\pgfsetdash{}{0pt}%
\pgfsys@defobject{currentmarker}{\pgfqpoint{0.000000in}{-0.027778in}}{\pgfqpoint{0.000000in}{0.000000in}}{%
\pgfpathmoveto{\pgfqpoint{0.000000in}{0.000000in}}%
\pgfpathlineto{\pgfqpoint{0.000000in}{-0.027778in}}%
\pgfusepath{stroke,fill}%
}%
\begin{pgfscope}%
\pgfsys@transformshift{4.202406in}{0.417642in}%
\pgfsys@useobject{currentmarker}{}%
\end{pgfscope}%
\end{pgfscope}%
\begin{pgfscope}%
\pgfpathrectangle{\pgfqpoint{0.605343in}{0.417642in}}{\pgfqpoint{4.842987in}{2.930688in}}%
\pgfusepath{clip}%
\pgfsetrectcap%
\pgfsetroundjoin%
\pgfsetlinewidth{0.803000pt}%
\definecolor{currentstroke}{rgb}{0.850000,0.850000,0.850000}%
\pgfsetstrokecolor{currentstroke}%
\pgfsetdash{}{0pt}%
\pgfpathmoveto{\pgfqpoint{4.273517in}{0.417642in}}%
\pgfpathlineto{\pgfqpoint{4.273517in}{3.348330in}}%
\pgfusepath{stroke}%
\end{pgfscope}%
\begin{pgfscope}%
\pgfsetbuttcap%
\pgfsetroundjoin%
\definecolor{currentfill}{rgb}{0.000000,0.000000,0.000000}%
\pgfsetfillcolor{currentfill}%
\pgfsetlinewidth{0.602250pt}%
\definecolor{currentstroke}{rgb}{0.000000,0.000000,0.000000}%
\pgfsetstrokecolor{currentstroke}%
\pgfsetdash{}{0pt}%
\pgfsys@defobject{currentmarker}{\pgfqpoint{0.000000in}{-0.027778in}}{\pgfqpoint{0.000000in}{0.000000in}}{%
\pgfpathmoveto{\pgfqpoint{0.000000in}{0.000000in}}%
\pgfpathlineto{\pgfqpoint{0.000000in}{-0.027778in}}%
\pgfusepath{stroke,fill}%
}%
\begin{pgfscope}%
\pgfsys@transformshift{4.273517in}{0.417642in}%
\pgfsys@useobject{currentmarker}{}%
\end{pgfscope}%
\end{pgfscope}%
\begin{pgfscope}%
\pgfpathrectangle{\pgfqpoint{0.605343in}{0.417642in}}{\pgfqpoint{4.842987in}{2.930688in}}%
\pgfusepath{clip}%
\pgfsetrectcap%
\pgfsetroundjoin%
\pgfsetlinewidth{0.803000pt}%
\definecolor{currentstroke}{rgb}{0.850000,0.850000,0.850000}%
\pgfsetstrokecolor{currentstroke}%
\pgfsetdash{}{0pt}%
\pgfpathmoveto{\pgfqpoint{4.331619in}{0.417642in}}%
\pgfpathlineto{\pgfqpoint{4.331619in}{3.348330in}}%
\pgfusepath{stroke}%
\end{pgfscope}%
\begin{pgfscope}%
\pgfsetbuttcap%
\pgfsetroundjoin%
\definecolor{currentfill}{rgb}{0.000000,0.000000,0.000000}%
\pgfsetfillcolor{currentfill}%
\pgfsetlinewidth{0.602250pt}%
\definecolor{currentstroke}{rgb}{0.000000,0.000000,0.000000}%
\pgfsetstrokecolor{currentstroke}%
\pgfsetdash{}{0pt}%
\pgfsys@defobject{currentmarker}{\pgfqpoint{0.000000in}{-0.027778in}}{\pgfqpoint{0.000000in}{0.000000in}}{%
\pgfpathmoveto{\pgfqpoint{0.000000in}{0.000000in}}%
\pgfpathlineto{\pgfqpoint{0.000000in}{-0.027778in}}%
\pgfusepath{stroke,fill}%
}%
\begin{pgfscope}%
\pgfsys@transformshift{4.331619in}{0.417642in}%
\pgfsys@useobject{currentmarker}{}%
\end{pgfscope}%
\end{pgfscope}%
\begin{pgfscope}%
\pgfpathrectangle{\pgfqpoint{0.605343in}{0.417642in}}{\pgfqpoint{4.842987in}{2.930688in}}%
\pgfusepath{clip}%
\pgfsetrectcap%
\pgfsetroundjoin%
\pgfsetlinewidth{0.803000pt}%
\definecolor{currentstroke}{rgb}{0.850000,0.850000,0.850000}%
\pgfsetstrokecolor{currentstroke}%
\pgfsetdash{}{0pt}%
\pgfpathmoveto{\pgfqpoint{4.380744in}{0.417642in}}%
\pgfpathlineto{\pgfqpoint{4.380744in}{3.348330in}}%
\pgfusepath{stroke}%
\end{pgfscope}%
\begin{pgfscope}%
\pgfsetbuttcap%
\pgfsetroundjoin%
\definecolor{currentfill}{rgb}{0.000000,0.000000,0.000000}%
\pgfsetfillcolor{currentfill}%
\pgfsetlinewidth{0.602250pt}%
\definecolor{currentstroke}{rgb}{0.000000,0.000000,0.000000}%
\pgfsetstrokecolor{currentstroke}%
\pgfsetdash{}{0pt}%
\pgfsys@defobject{currentmarker}{\pgfqpoint{0.000000in}{-0.027778in}}{\pgfqpoint{0.000000in}{0.000000in}}{%
\pgfpathmoveto{\pgfqpoint{0.000000in}{0.000000in}}%
\pgfpathlineto{\pgfqpoint{0.000000in}{-0.027778in}}%
\pgfusepath{stroke,fill}%
}%
\begin{pgfscope}%
\pgfsys@transformshift{4.380744in}{0.417642in}%
\pgfsys@useobject{currentmarker}{}%
\end{pgfscope}%
\end{pgfscope}%
\begin{pgfscope}%
\pgfpathrectangle{\pgfqpoint{0.605343in}{0.417642in}}{\pgfqpoint{4.842987in}{2.930688in}}%
\pgfusepath{clip}%
\pgfsetrectcap%
\pgfsetroundjoin%
\pgfsetlinewidth{0.803000pt}%
\definecolor{currentstroke}{rgb}{0.850000,0.850000,0.850000}%
\pgfsetstrokecolor{currentstroke}%
\pgfsetdash{}{0pt}%
\pgfpathmoveto{\pgfqpoint{4.423297in}{0.417642in}}%
\pgfpathlineto{\pgfqpoint{4.423297in}{3.348330in}}%
\pgfusepath{stroke}%
\end{pgfscope}%
\begin{pgfscope}%
\pgfsetbuttcap%
\pgfsetroundjoin%
\definecolor{currentfill}{rgb}{0.000000,0.000000,0.000000}%
\pgfsetfillcolor{currentfill}%
\pgfsetlinewidth{0.602250pt}%
\definecolor{currentstroke}{rgb}{0.000000,0.000000,0.000000}%
\pgfsetstrokecolor{currentstroke}%
\pgfsetdash{}{0pt}%
\pgfsys@defobject{currentmarker}{\pgfqpoint{0.000000in}{-0.027778in}}{\pgfqpoint{0.000000in}{0.000000in}}{%
\pgfpathmoveto{\pgfqpoint{0.000000in}{0.000000in}}%
\pgfpathlineto{\pgfqpoint{0.000000in}{-0.027778in}}%
\pgfusepath{stroke,fill}%
}%
\begin{pgfscope}%
\pgfsys@transformshift{4.423297in}{0.417642in}%
\pgfsys@useobject{currentmarker}{}%
\end{pgfscope}%
\end{pgfscope}%
\begin{pgfscope}%
\pgfpathrectangle{\pgfqpoint{0.605343in}{0.417642in}}{\pgfqpoint{4.842987in}{2.930688in}}%
\pgfusepath{clip}%
\pgfsetrectcap%
\pgfsetroundjoin%
\pgfsetlinewidth{0.803000pt}%
\definecolor{currentstroke}{rgb}{0.850000,0.850000,0.850000}%
\pgfsetstrokecolor{currentstroke}%
\pgfsetdash{}{0pt}%
\pgfpathmoveto{\pgfqpoint{4.460832in}{0.417642in}}%
\pgfpathlineto{\pgfqpoint{4.460832in}{3.348330in}}%
\pgfusepath{stroke}%
\end{pgfscope}%
\begin{pgfscope}%
\pgfsetbuttcap%
\pgfsetroundjoin%
\definecolor{currentfill}{rgb}{0.000000,0.000000,0.000000}%
\pgfsetfillcolor{currentfill}%
\pgfsetlinewidth{0.602250pt}%
\definecolor{currentstroke}{rgb}{0.000000,0.000000,0.000000}%
\pgfsetstrokecolor{currentstroke}%
\pgfsetdash{}{0pt}%
\pgfsys@defobject{currentmarker}{\pgfqpoint{0.000000in}{-0.027778in}}{\pgfqpoint{0.000000in}{0.000000in}}{%
\pgfpathmoveto{\pgfqpoint{0.000000in}{0.000000in}}%
\pgfpathlineto{\pgfqpoint{0.000000in}{-0.027778in}}%
\pgfusepath{stroke,fill}%
}%
\begin{pgfscope}%
\pgfsys@transformshift{4.460832in}{0.417642in}%
\pgfsys@useobject{currentmarker}{}%
\end{pgfscope}%
\end{pgfscope}%
\begin{pgfscope}%
\pgfpathrectangle{\pgfqpoint{0.605343in}{0.417642in}}{\pgfqpoint{4.842987in}{2.930688in}}%
\pgfusepath{clip}%
\pgfsetrectcap%
\pgfsetroundjoin%
\pgfsetlinewidth{0.803000pt}%
\definecolor{currentstroke}{rgb}{0.850000,0.850000,0.850000}%
\pgfsetstrokecolor{currentstroke}%
\pgfsetdash{}{0pt}%
\pgfpathmoveto{\pgfqpoint{4.715300in}{0.417642in}}%
\pgfpathlineto{\pgfqpoint{4.715300in}{3.348330in}}%
\pgfusepath{stroke}%
\end{pgfscope}%
\begin{pgfscope}%
\pgfsetbuttcap%
\pgfsetroundjoin%
\definecolor{currentfill}{rgb}{0.000000,0.000000,0.000000}%
\pgfsetfillcolor{currentfill}%
\pgfsetlinewidth{0.602250pt}%
\definecolor{currentstroke}{rgb}{0.000000,0.000000,0.000000}%
\pgfsetstrokecolor{currentstroke}%
\pgfsetdash{}{0pt}%
\pgfsys@defobject{currentmarker}{\pgfqpoint{0.000000in}{-0.027778in}}{\pgfqpoint{0.000000in}{0.000000in}}{%
\pgfpathmoveto{\pgfqpoint{0.000000in}{0.000000in}}%
\pgfpathlineto{\pgfqpoint{0.000000in}{-0.027778in}}%
\pgfusepath{stroke,fill}%
}%
\begin{pgfscope}%
\pgfsys@transformshift{4.715300in}{0.417642in}%
\pgfsys@useobject{currentmarker}{}%
\end{pgfscope}%
\end{pgfscope}%
\begin{pgfscope}%
\pgfpathrectangle{\pgfqpoint{0.605343in}{0.417642in}}{\pgfqpoint{4.842987in}{2.930688in}}%
\pgfusepath{clip}%
\pgfsetrectcap%
\pgfsetroundjoin%
\pgfsetlinewidth{0.803000pt}%
\definecolor{currentstroke}{rgb}{0.850000,0.850000,0.850000}%
\pgfsetstrokecolor{currentstroke}%
\pgfsetdash{}{0pt}%
\pgfpathmoveto{\pgfqpoint{4.844513in}{0.417642in}}%
\pgfpathlineto{\pgfqpoint{4.844513in}{3.348330in}}%
\pgfusepath{stroke}%
\end{pgfscope}%
\begin{pgfscope}%
\pgfsetbuttcap%
\pgfsetroundjoin%
\definecolor{currentfill}{rgb}{0.000000,0.000000,0.000000}%
\pgfsetfillcolor{currentfill}%
\pgfsetlinewidth{0.602250pt}%
\definecolor{currentstroke}{rgb}{0.000000,0.000000,0.000000}%
\pgfsetstrokecolor{currentstroke}%
\pgfsetdash{}{0pt}%
\pgfsys@defobject{currentmarker}{\pgfqpoint{0.000000in}{-0.027778in}}{\pgfqpoint{0.000000in}{0.000000in}}{%
\pgfpathmoveto{\pgfqpoint{0.000000in}{0.000000in}}%
\pgfpathlineto{\pgfqpoint{0.000000in}{-0.027778in}}%
\pgfusepath{stroke,fill}%
}%
\begin{pgfscope}%
\pgfsys@transformshift{4.844513in}{0.417642in}%
\pgfsys@useobject{currentmarker}{}%
\end{pgfscope}%
\end{pgfscope}%
\begin{pgfscope}%
\pgfpathrectangle{\pgfqpoint{0.605343in}{0.417642in}}{\pgfqpoint{4.842987in}{2.930688in}}%
\pgfusepath{clip}%
\pgfsetrectcap%
\pgfsetroundjoin%
\pgfsetlinewidth{0.803000pt}%
\definecolor{currentstroke}{rgb}{0.850000,0.850000,0.850000}%
\pgfsetstrokecolor{currentstroke}%
\pgfsetdash{}{0pt}%
\pgfpathmoveto{\pgfqpoint{4.936192in}{0.417642in}}%
\pgfpathlineto{\pgfqpoint{4.936192in}{3.348330in}}%
\pgfusepath{stroke}%
\end{pgfscope}%
\begin{pgfscope}%
\pgfsetbuttcap%
\pgfsetroundjoin%
\definecolor{currentfill}{rgb}{0.000000,0.000000,0.000000}%
\pgfsetfillcolor{currentfill}%
\pgfsetlinewidth{0.602250pt}%
\definecolor{currentstroke}{rgb}{0.000000,0.000000,0.000000}%
\pgfsetstrokecolor{currentstroke}%
\pgfsetdash{}{0pt}%
\pgfsys@defobject{currentmarker}{\pgfqpoint{0.000000in}{-0.027778in}}{\pgfqpoint{0.000000in}{0.000000in}}{%
\pgfpathmoveto{\pgfqpoint{0.000000in}{0.000000in}}%
\pgfpathlineto{\pgfqpoint{0.000000in}{-0.027778in}}%
\pgfusepath{stroke,fill}%
}%
\begin{pgfscope}%
\pgfsys@transformshift{4.936192in}{0.417642in}%
\pgfsys@useobject{currentmarker}{}%
\end{pgfscope}%
\end{pgfscope}%
\begin{pgfscope}%
\pgfpathrectangle{\pgfqpoint{0.605343in}{0.417642in}}{\pgfqpoint{4.842987in}{2.930688in}}%
\pgfusepath{clip}%
\pgfsetrectcap%
\pgfsetroundjoin%
\pgfsetlinewidth{0.803000pt}%
\definecolor{currentstroke}{rgb}{0.850000,0.850000,0.850000}%
\pgfsetstrokecolor{currentstroke}%
\pgfsetdash{}{0pt}%
\pgfpathmoveto{\pgfqpoint{5.007303in}{0.417642in}}%
\pgfpathlineto{\pgfqpoint{5.007303in}{3.348330in}}%
\pgfusepath{stroke}%
\end{pgfscope}%
\begin{pgfscope}%
\pgfsetbuttcap%
\pgfsetroundjoin%
\definecolor{currentfill}{rgb}{0.000000,0.000000,0.000000}%
\pgfsetfillcolor{currentfill}%
\pgfsetlinewidth{0.602250pt}%
\definecolor{currentstroke}{rgb}{0.000000,0.000000,0.000000}%
\pgfsetstrokecolor{currentstroke}%
\pgfsetdash{}{0pt}%
\pgfsys@defobject{currentmarker}{\pgfqpoint{0.000000in}{-0.027778in}}{\pgfqpoint{0.000000in}{0.000000in}}{%
\pgfpathmoveto{\pgfqpoint{0.000000in}{0.000000in}}%
\pgfpathlineto{\pgfqpoint{0.000000in}{-0.027778in}}%
\pgfusepath{stroke,fill}%
}%
\begin{pgfscope}%
\pgfsys@transformshift{5.007303in}{0.417642in}%
\pgfsys@useobject{currentmarker}{}%
\end{pgfscope}%
\end{pgfscope}%
\begin{pgfscope}%
\pgfpathrectangle{\pgfqpoint{0.605343in}{0.417642in}}{\pgfqpoint{4.842987in}{2.930688in}}%
\pgfusepath{clip}%
\pgfsetrectcap%
\pgfsetroundjoin%
\pgfsetlinewidth{0.803000pt}%
\definecolor{currentstroke}{rgb}{0.850000,0.850000,0.850000}%
\pgfsetstrokecolor{currentstroke}%
\pgfsetdash{}{0pt}%
\pgfpathmoveto{\pgfqpoint{5.065405in}{0.417642in}}%
\pgfpathlineto{\pgfqpoint{5.065405in}{3.348330in}}%
\pgfusepath{stroke}%
\end{pgfscope}%
\begin{pgfscope}%
\pgfsetbuttcap%
\pgfsetroundjoin%
\definecolor{currentfill}{rgb}{0.000000,0.000000,0.000000}%
\pgfsetfillcolor{currentfill}%
\pgfsetlinewidth{0.602250pt}%
\definecolor{currentstroke}{rgb}{0.000000,0.000000,0.000000}%
\pgfsetstrokecolor{currentstroke}%
\pgfsetdash{}{0pt}%
\pgfsys@defobject{currentmarker}{\pgfqpoint{0.000000in}{-0.027778in}}{\pgfqpoint{0.000000in}{0.000000in}}{%
\pgfpathmoveto{\pgfqpoint{0.000000in}{0.000000in}}%
\pgfpathlineto{\pgfqpoint{0.000000in}{-0.027778in}}%
\pgfusepath{stroke,fill}%
}%
\begin{pgfscope}%
\pgfsys@transformshift{5.065405in}{0.417642in}%
\pgfsys@useobject{currentmarker}{}%
\end{pgfscope}%
\end{pgfscope}%
\begin{pgfscope}%
\pgfpathrectangle{\pgfqpoint{0.605343in}{0.417642in}}{\pgfqpoint{4.842987in}{2.930688in}}%
\pgfusepath{clip}%
\pgfsetrectcap%
\pgfsetroundjoin%
\pgfsetlinewidth{0.803000pt}%
\definecolor{currentstroke}{rgb}{0.850000,0.850000,0.850000}%
\pgfsetstrokecolor{currentstroke}%
\pgfsetdash{}{0pt}%
\pgfpathmoveto{\pgfqpoint{5.114529in}{0.417642in}}%
\pgfpathlineto{\pgfqpoint{5.114529in}{3.348330in}}%
\pgfusepath{stroke}%
\end{pgfscope}%
\begin{pgfscope}%
\pgfsetbuttcap%
\pgfsetroundjoin%
\definecolor{currentfill}{rgb}{0.000000,0.000000,0.000000}%
\pgfsetfillcolor{currentfill}%
\pgfsetlinewidth{0.602250pt}%
\definecolor{currentstroke}{rgb}{0.000000,0.000000,0.000000}%
\pgfsetstrokecolor{currentstroke}%
\pgfsetdash{}{0pt}%
\pgfsys@defobject{currentmarker}{\pgfqpoint{0.000000in}{-0.027778in}}{\pgfqpoint{0.000000in}{0.000000in}}{%
\pgfpathmoveto{\pgfqpoint{0.000000in}{0.000000in}}%
\pgfpathlineto{\pgfqpoint{0.000000in}{-0.027778in}}%
\pgfusepath{stroke,fill}%
}%
\begin{pgfscope}%
\pgfsys@transformshift{5.114529in}{0.417642in}%
\pgfsys@useobject{currentmarker}{}%
\end{pgfscope}%
\end{pgfscope}%
\begin{pgfscope}%
\pgfpathrectangle{\pgfqpoint{0.605343in}{0.417642in}}{\pgfqpoint{4.842987in}{2.930688in}}%
\pgfusepath{clip}%
\pgfsetrectcap%
\pgfsetroundjoin%
\pgfsetlinewidth{0.803000pt}%
\definecolor{currentstroke}{rgb}{0.850000,0.850000,0.850000}%
\pgfsetstrokecolor{currentstroke}%
\pgfsetdash{}{0pt}%
\pgfpathmoveto{\pgfqpoint{5.157083in}{0.417642in}}%
\pgfpathlineto{\pgfqpoint{5.157083in}{3.348330in}}%
\pgfusepath{stroke}%
\end{pgfscope}%
\begin{pgfscope}%
\pgfsetbuttcap%
\pgfsetroundjoin%
\definecolor{currentfill}{rgb}{0.000000,0.000000,0.000000}%
\pgfsetfillcolor{currentfill}%
\pgfsetlinewidth{0.602250pt}%
\definecolor{currentstroke}{rgb}{0.000000,0.000000,0.000000}%
\pgfsetstrokecolor{currentstroke}%
\pgfsetdash{}{0pt}%
\pgfsys@defobject{currentmarker}{\pgfqpoint{0.000000in}{-0.027778in}}{\pgfqpoint{0.000000in}{0.000000in}}{%
\pgfpathmoveto{\pgfqpoint{0.000000in}{0.000000in}}%
\pgfpathlineto{\pgfqpoint{0.000000in}{-0.027778in}}%
\pgfusepath{stroke,fill}%
}%
\begin{pgfscope}%
\pgfsys@transformshift{5.157083in}{0.417642in}%
\pgfsys@useobject{currentmarker}{}%
\end{pgfscope}%
\end{pgfscope}%
\begin{pgfscope}%
\pgfpathrectangle{\pgfqpoint{0.605343in}{0.417642in}}{\pgfqpoint{4.842987in}{2.930688in}}%
\pgfusepath{clip}%
\pgfsetrectcap%
\pgfsetroundjoin%
\pgfsetlinewidth{0.803000pt}%
\definecolor{currentstroke}{rgb}{0.850000,0.850000,0.850000}%
\pgfsetstrokecolor{currentstroke}%
\pgfsetdash{}{0pt}%
\pgfpathmoveto{\pgfqpoint{5.194618in}{0.417642in}}%
\pgfpathlineto{\pgfqpoint{5.194618in}{3.348330in}}%
\pgfusepath{stroke}%
\end{pgfscope}%
\begin{pgfscope}%
\pgfsetbuttcap%
\pgfsetroundjoin%
\definecolor{currentfill}{rgb}{0.000000,0.000000,0.000000}%
\pgfsetfillcolor{currentfill}%
\pgfsetlinewidth{0.602250pt}%
\definecolor{currentstroke}{rgb}{0.000000,0.000000,0.000000}%
\pgfsetstrokecolor{currentstroke}%
\pgfsetdash{}{0pt}%
\pgfsys@defobject{currentmarker}{\pgfqpoint{0.000000in}{-0.027778in}}{\pgfqpoint{0.000000in}{0.000000in}}{%
\pgfpathmoveto{\pgfqpoint{0.000000in}{0.000000in}}%
\pgfpathlineto{\pgfqpoint{0.000000in}{-0.027778in}}%
\pgfusepath{stroke,fill}%
}%
\begin{pgfscope}%
\pgfsys@transformshift{5.194618in}{0.417642in}%
\pgfsys@useobject{currentmarker}{}%
\end{pgfscope}%
\end{pgfscope}%
\begin{pgfscope}%
\definecolor{textcolor}{rgb}{0.000000,0.000000,0.000000}%
\pgfsetstrokecolor{textcolor}%
\pgfsetfillcolor{textcolor}%
\pgftext[x=3.026837in,y=0.165003in,,top]{\color{textcolor}\rmfamily\fontsize{10.000000}{12.000000}\selectfont Frequency in \unit{\Hz}}%
\end{pgfscope}%
\begin{pgfscope}%
\pgfpathrectangle{\pgfqpoint{0.605343in}{0.417642in}}{\pgfqpoint{4.842987in}{2.930688in}}%
\pgfusepath{clip}%
\pgfsetrectcap%
\pgfsetroundjoin%
\pgfsetlinewidth{0.803000pt}%
\definecolor{currentstroke}{rgb}{0.450000,0.450000,0.450000}%
\pgfsetstrokecolor{currentstroke}%
\pgfsetdash{}{0pt}%
\pgfpathmoveto{\pgfqpoint{0.605343in}{0.856755in}}%
\pgfpathlineto{\pgfqpoint{5.448330in}{0.856755in}}%
\pgfusepath{stroke}%
\end{pgfscope}%
\begin{pgfscope}%
\pgfsetbuttcap%
\pgfsetroundjoin%
\definecolor{currentfill}{rgb}{0.000000,0.000000,0.000000}%
\pgfsetfillcolor{currentfill}%
\pgfsetlinewidth{0.803000pt}%
\definecolor{currentstroke}{rgb}{0.000000,0.000000,0.000000}%
\pgfsetstrokecolor{currentstroke}%
\pgfsetdash{}{0pt}%
\pgfsys@defobject{currentmarker}{\pgfqpoint{-0.048611in}{0.000000in}}{\pgfqpoint{-0.000000in}{0.000000in}}{%
\pgfpathmoveto{\pgfqpoint{-0.000000in}{0.000000in}}%
\pgfpathlineto{\pgfqpoint{-0.048611in}{0.000000in}}%
\pgfusepath{stroke,fill}%
}%
\begin{pgfscope}%
\pgfsys@transformshift{0.605343in}{0.856755in}%
\pgfsys@useobject{currentmarker}{}%
\end{pgfscope}%
\end{pgfscope}%
\begin{pgfscope}%
\definecolor{textcolor}{rgb}{0.000000,0.000000,0.000000}%
\pgfsetstrokecolor{textcolor}%
\pgfsetfillcolor{textcolor}%
\pgftext[x=0.251948in, y=0.817602in, left, base]{\color{textcolor}\rmfamily\fontsize{8.000000}{9.600000}\selectfont \(\displaystyle {10^{-8}}\)}%
\end{pgfscope}%
\begin{pgfscope}%
\pgfpathrectangle{\pgfqpoint{0.605343in}{0.417642in}}{\pgfqpoint{4.842987in}{2.930688in}}%
\pgfusepath{clip}%
\pgfsetrectcap%
\pgfsetroundjoin%
\pgfsetlinewidth{0.803000pt}%
\definecolor{currentstroke}{rgb}{0.450000,0.450000,0.450000}%
\pgfsetstrokecolor{currentstroke}%
\pgfsetdash{}{0pt}%
\pgfpathmoveto{\pgfqpoint{0.605343in}{1.469399in}}%
\pgfpathlineto{\pgfqpoint{5.448330in}{1.469399in}}%
\pgfusepath{stroke}%
\end{pgfscope}%
\begin{pgfscope}%
\pgfsetbuttcap%
\pgfsetroundjoin%
\definecolor{currentfill}{rgb}{0.000000,0.000000,0.000000}%
\pgfsetfillcolor{currentfill}%
\pgfsetlinewidth{0.803000pt}%
\definecolor{currentstroke}{rgb}{0.000000,0.000000,0.000000}%
\pgfsetstrokecolor{currentstroke}%
\pgfsetdash{}{0pt}%
\pgfsys@defobject{currentmarker}{\pgfqpoint{-0.048611in}{0.000000in}}{\pgfqpoint{-0.000000in}{0.000000in}}{%
\pgfpathmoveto{\pgfqpoint{-0.000000in}{0.000000in}}%
\pgfpathlineto{\pgfqpoint{-0.048611in}{0.000000in}}%
\pgfusepath{stroke,fill}%
}%
\begin{pgfscope}%
\pgfsys@transformshift{0.605343in}{1.469399in}%
\pgfsys@useobject{currentmarker}{}%
\end{pgfscope}%
\end{pgfscope}%
\begin{pgfscope}%
\definecolor{textcolor}{rgb}{0.000000,0.000000,0.000000}%
\pgfsetstrokecolor{textcolor}%
\pgfsetfillcolor{textcolor}%
\pgftext[x=0.251948in, y=1.430246in, left, base]{\color{textcolor}\rmfamily\fontsize{8.000000}{9.600000}\selectfont \(\displaystyle {10^{-6}}\)}%
\end{pgfscope}%
\begin{pgfscope}%
\pgfpathrectangle{\pgfqpoint{0.605343in}{0.417642in}}{\pgfqpoint{4.842987in}{2.930688in}}%
\pgfusepath{clip}%
\pgfsetrectcap%
\pgfsetroundjoin%
\pgfsetlinewidth{0.803000pt}%
\definecolor{currentstroke}{rgb}{0.450000,0.450000,0.450000}%
\pgfsetstrokecolor{currentstroke}%
\pgfsetdash{}{0pt}%
\pgfpathmoveto{\pgfqpoint{0.605343in}{2.082042in}}%
\pgfpathlineto{\pgfqpoint{5.448330in}{2.082042in}}%
\pgfusepath{stroke}%
\end{pgfscope}%
\begin{pgfscope}%
\pgfsetbuttcap%
\pgfsetroundjoin%
\definecolor{currentfill}{rgb}{0.000000,0.000000,0.000000}%
\pgfsetfillcolor{currentfill}%
\pgfsetlinewidth{0.803000pt}%
\definecolor{currentstroke}{rgb}{0.000000,0.000000,0.000000}%
\pgfsetstrokecolor{currentstroke}%
\pgfsetdash{}{0pt}%
\pgfsys@defobject{currentmarker}{\pgfqpoint{-0.048611in}{0.000000in}}{\pgfqpoint{-0.000000in}{0.000000in}}{%
\pgfpathmoveto{\pgfqpoint{-0.000000in}{0.000000in}}%
\pgfpathlineto{\pgfqpoint{-0.048611in}{0.000000in}}%
\pgfusepath{stroke,fill}%
}%
\begin{pgfscope}%
\pgfsys@transformshift{0.605343in}{2.082042in}%
\pgfsys@useobject{currentmarker}{}%
\end{pgfscope}%
\end{pgfscope}%
\begin{pgfscope}%
\definecolor{textcolor}{rgb}{0.000000,0.000000,0.000000}%
\pgfsetstrokecolor{textcolor}%
\pgfsetfillcolor{textcolor}%
\pgftext[x=0.251948in, y=2.042890in, left, base]{\color{textcolor}\rmfamily\fontsize{8.000000}{9.600000}\selectfont \(\displaystyle {10^{-4}}\)}%
\end{pgfscope}%
\begin{pgfscope}%
\pgfpathrectangle{\pgfqpoint{0.605343in}{0.417642in}}{\pgfqpoint{4.842987in}{2.930688in}}%
\pgfusepath{clip}%
\pgfsetrectcap%
\pgfsetroundjoin%
\pgfsetlinewidth{0.803000pt}%
\definecolor{currentstroke}{rgb}{0.450000,0.450000,0.450000}%
\pgfsetstrokecolor{currentstroke}%
\pgfsetdash{}{0pt}%
\pgfpathmoveto{\pgfqpoint{0.605343in}{2.694686in}}%
\pgfpathlineto{\pgfqpoint{5.448330in}{2.694686in}}%
\pgfusepath{stroke}%
\end{pgfscope}%
\begin{pgfscope}%
\pgfsetbuttcap%
\pgfsetroundjoin%
\definecolor{currentfill}{rgb}{0.000000,0.000000,0.000000}%
\pgfsetfillcolor{currentfill}%
\pgfsetlinewidth{0.803000pt}%
\definecolor{currentstroke}{rgb}{0.000000,0.000000,0.000000}%
\pgfsetstrokecolor{currentstroke}%
\pgfsetdash{}{0pt}%
\pgfsys@defobject{currentmarker}{\pgfqpoint{-0.048611in}{0.000000in}}{\pgfqpoint{-0.000000in}{0.000000in}}{%
\pgfpathmoveto{\pgfqpoint{-0.000000in}{0.000000in}}%
\pgfpathlineto{\pgfqpoint{-0.048611in}{0.000000in}}%
\pgfusepath{stroke,fill}%
}%
\begin{pgfscope}%
\pgfsys@transformshift{0.605343in}{2.694686in}%
\pgfsys@useobject{currentmarker}{}%
\end{pgfscope}%
\end{pgfscope}%
\begin{pgfscope}%
\definecolor{textcolor}{rgb}{0.000000,0.000000,0.000000}%
\pgfsetstrokecolor{textcolor}%
\pgfsetfillcolor{textcolor}%
\pgftext[x=0.251948in, y=2.655534in, left, base]{\color{textcolor}\rmfamily\fontsize{8.000000}{9.600000}\selectfont \(\displaystyle {10^{-2}}\)}%
\end{pgfscope}%
\begin{pgfscope}%
\pgfpathrectangle{\pgfqpoint{0.605343in}{0.417642in}}{\pgfqpoint{4.842987in}{2.930688in}}%
\pgfusepath{clip}%
\pgfsetrectcap%
\pgfsetroundjoin%
\pgfsetlinewidth{0.803000pt}%
\definecolor{currentstroke}{rgb}{0.450000,0.450000,0.450000}%
\pgfsetstrokecolor{currentstroke}%
\pgfsetdash{}{0pt}%
\pgfpathmoveto{\pgfqpoint{0.605343in}{3.307330in}}%
\pgfpathlineto{\pgfqpoint{5.448330in}{3.307330in}}%
\pgfusepath{stroke}%
\end{pgfscope}%
\begin{pgfscope}%
\pgfsetbuttcap%
\pgfsetroundjoin%
\definecolor{currentfill}{rgb}{0.000000,0.000000,0.000000}%
\pgfsetfillcolor{currentfill}%
\pgfsetlinewidth{0.803000pt}%
\definecolor{currentstroke}{rgb}{0.000000,0.000000,0.000000}%
\pgfsetstrokecolor{currentstroke}%
\pgfsetdash{}{0pt}%
\pgfsys@defobject{currentmarker}{\pgfqpoint{-0.048611in}{0.000000in}}{\pgfqpoint{-0.000000in}{0.000000in}}{%
\pgfpathmoveto{\pgfqpoint{-0.000000in}{0.000000in}}%
\pgfpathlineto{\pgfqpoint{-0.048611in}{0.000000in}}%
\pgfusepath{stroke,fill}%
}%
\begin{pgfscope}%
\pgfsys@transformshift{0.605343in}{3.307330in}%
\pgfsys@useobject{currentmarker}{}%
\end{pgfscope}%
\end{pgfscope}%
\begin{pgfscope}%
\definecolor{textcolor}{rgb}{0.000000,0.000000,0.000000}%
\pgfsetstrokecolor{textcolor}%
\pgfsetfillcolor{textcolor}%
\pgftext[x=0.332194in, y=3.268178in, left, base]{\color{textcolor}\rmfamily\fontsize{8.000000}{9.600000}\selectfont \(\displaystyle {10^{0}}\)}%
\end{pgfscope}%
\begin{pgfscope}%
\definecolor{textcolor}{rgb}{0.000000,0.000000,0.000000}%
\pgfsetstrokecolor{textcolor}%
\pgfsetfillcolor{textcolor}%
\pgftext[x=0.196393in,y=1.882986in,,bottom,rotate=90.000000]{\color{textcolor}\rmfamily\fontsize{10.000000}{12.000000}\selectfont Power spectral density in \(\displaystyle \unit{\V^2 \per \Hz}\)}%
\end{pgfscope}%
\begin{pgfscope}%
\pgfpathrectangle{\pgfqpoint{0.605343in}{0.417642in}}{\pgfqpoint{4.842987in}{2.930688in}}%
\pgfusepath{clip}%
\pgfsetrectcap%
\pgfsetroundjoin%
\pgfsetlinewidth{1.003750pt}%
\definecolor{currentstroke}{rgb}{0.000000,0.447059,0.698039}%
\pgfsetstrokecolor{currentstroke}%
\pgfsetstrokeopacity{0.700000}%
\pgfsetdash{}{0pt}%
\pgfpathmoveto{\pgfqpoint{0.825479in}{3.215117in}}%
\pgfpathlineto{\pgfqpoint{0.915331in}{3.215116in}}%
\pgfpathlineto{\pgfqpoint{1.005182in}{3.215114in}}%
\pgfpathlineto{\pgfqpoint{1.095033in}{3.215111in}}%
\pgfpathlineto{\pgfqpoint{1.184885in}{3.215106in}}%
\pgfpathlineto{\pgfqpoint{1.274736in}{3.215096in}}%
\pgfpathlineto{\pgfqpoint{1.364587in}{3.215080in}}%
\pgfpathlineto{\pgfqpoint{1.454439in}{3.215050in}}%
\pgfpathlineto{\pgfqpoint{1.544290in}{3.214999in}}%
\pgfpathlineto{\pgfqpoint{1.634141in}{3.214908in}}%
\pgfpathlineto{\pgfqpoint{1.723992in}{3.214750in}}%
\pgfpathlineto{\pgfqpoint{1.813844in}{3.214471in}}%
\pgfpathlineto{\pgfqpoint{1.903695in}{3.213983in}}%
\pgfpathlineto{\pgfqpoint{1.993546in}{3.213129in}}%
\pgfpathlineto{\pgfqpoint{2.083398in}{3.211642in}}%
\pgfpathlineto{\pgfqpoint{2.173249in}{3.209067in}}%
\pgfpathlineto{\pgfqpoint{2.263100in}{3.204660in}}%
\pgfpathlineto{\pgfqpoint{2.352952in}{3.197252in}}%
\pgfpathlineto{\pgfqpoint{2.442803in}{3.185154in}}%
\pgfpathlineto{\pgfqpoint{2.532654in}{3.166241in}}%
\pgfpathlineto{\pgfqpoint{2.622506in}{3.138387in}}%
\pgfpathlineto{\pgfqpoint{2.712357in}{3.100247in}}%
\pgfpathlineto{\pgfqpoint{2.802208in}{3.051916in}}%
\pgfpathlineto{\pgfqpoint{2.892060in}{2.994874in}}%
\pgfpathlineto{\pgfqpoint{2.981911in}{2.931284in}}%
\pgfpathlineto{\pgfqpoint{3.071762in}{2.863236in}}%
\pgfpathlineto{\pgfqpoint{3.161614in}{2.792353in}}%
\pgfpathlineto{\pgfqpoint{3.251465in}{2.719747in}}%
\pgfpathlineto{\pgfqpoint{3.341316in}{2.646121in}}%
\pgfpathlineto{\pgfqpoint{3.431168in}{2.571902in}}%
\pgfpathlineto{\pgfqpoint{3.521019in}{2.497340in}}%
\pgfpathlineto{\pgfqpoint{3.610870in}{2.422583in}}%
\pgfpathlineto{\pgfqpoint{3.700722in}{2.347714in}}%
\pgfpathlineto{\pgfqpoint{3.790573in}{2.272781in}}%
\pgfpathlineto{\pgfqpoint{3.880424in}{2.197811in}}%
\pgfpathlineto{\pgfqpoint{3.970276in}{2.122821in}}%
\pgfpathlineto{\pgfqpoint{4.060127in}{2.047819in}}%
\pgfpathlineto{\pgfqpoint{4.149978in}{1.972810in}}%
\pgfpathlineto{\pgfqpoint{4.239830in}{1.897798in}}%
\pgfpathlineto{\pgfqpoint{4.329681in}{1.822783in}}%
\pgfpathlineto{\pgfqpoint{4.419532in}{1.747767in}}%
\pgfpathlineto{\pgfqpoint{4.509384in}{1.672750in}}%
\pgfpathlineto{\pgfqpoint{4.599235in}{1.597733in}}%
\pgfpathlineto{\pgfqpoint{4.689086in}{1.522716in}}%
\pgfpathlineto{\pgfqpoint{4.778938in}{1.447698in}}%
\pgfpathlineto{\pgfqpoint{4.868789in}{1.372681in}}%
\pgfpathlineto{\pgfqpoint{4.958640in}{1.297663in}}%
\pgfpathlineto{\pgfqpoint{5.048492in}{1.222646in}}%
\pgfpathlineto{\pgfqpoint{5.138343in}{1.147628in}}%
\pgfpathlineto{\pgfqpoint{5.228194in}{1.072611in}}%
\pgfusepath{stroke}%
\end{pgfscope}%
\begin{pgfscope}%
\pgfpathrectangle{\pgfqpoint{0.605343in}{0.417642in}}{\pgfqpoint{4.842987in}{2.930688in}}%
\pgfusepath{clip}%
\pgfsetrectcap%
\pgfsetroundjoin%
\pgfsetlinewidth{1.003750pt}%
\definecolor{currentstroke}{rgb}{0.000000,0.619608,0.450980}%
\pgfsetstrokecolor{currentstroke}%
\pgfsetstrokeopacity{0.700000}%
\pgfsetdash{}{0pt}%
\pgfpathmoveto{\pgfqpoint{0.825479in}{3.147390in}}%
\pgfpathlineto{\pgfqpoint{0.915331in}{3.147386in}}%
\pgfpathlineto{\pgfqpoint{1.005182in}{3.147381in}}%
\pgfpathlineto{\pgfqpoint{1.095033in}{3.147371in}}%
\pgfpathlineto{\pgfqpoint{1.184885in}{3.147353in}}%
\pgfpathlineto{\pgfqpoint{1.274736in}{3.147321in}}%
\pgfpathlineto{\pgfqpoint{1.364587in}{3.147266in}}%
\pgfpathlineto{\pgfqpoint{1.454439in}{3.147169in}}%
\pgfpathlineto{\pgfqpoint{1.544290in}{3.147000in}}%
\pgfpathlineto{\pgfqpoint{1.634141in}{3.146701in}}%
\pgfpathlineto{\pgfqpoint{1.723992in}{3.146179in}}%
\pgfpathlineto{\pgfqpoint{1.813844in}{3.145266in}}%
\pgfpathlineto{\pgfqpoint{1.903695in}{3.143677in}}%
\pgfpathlineto{\pgfqpoint{1.993546in}{3.140928in}}%
\pgfpathlineto{\pgfqpoint{2.083398in}{3.136232in}}%
\pgfpathlineto{\pgfqpoint{2.173249in}{3.128359in}}%
\pgfpathlineto{\pgfqpoint{2.263100in}{3.115559in}}%
\pgfpathlineto{\pgfqpoint{2.352952in}{3.095675in}}%
\pgfpathlineto{\pgfqpoint{2.442803in}{3.066629in}}%
\pgfpathlineto{\pgfqpoint{2.532654in}{3.027227in}}%
\pgfpathlineto{\pgfqpoint{2.622506in}{2.977747in}}%
\pgfpathlineto{\pgfqpoint{2.712357in}{2.919797in}}%
\pgfpathlineto{\pgfqpoint{2.802208in}{2.855568in}}%
\pgfpathlineto{\pgfqpoint{2.892060in}{2.787103in}}%
\pgfpathlineto{\pgfqpoint{2.981911in}{2.715963in}}%
\pgfpathlineto{\pgfqpoint{3.071762in}{2.643204in}}%
\pgfpathlineto{\pgfqpoint{3.161614in}{2.569488in}}%
\pgfpathlineto{\pgfqpoint{3.251465in}{2.495217in}}%
\pgfpathlineto{\pgfqpoint{3.341316in}{2.420626in}}%
\pgfpathlineto{\pgfqpoint{3.431168in}{2.345852in}}%
\pgfpathlineto{\pgfqpoint{3.521019in}{2.270973in}}%
\pgfpathlineto{\pgfqpoint{3.610870in}{2.196034in}}%
\pgfpathlineto{\pgfqpoint{3.700722in}{2.121061in}}%
\pgfpathlineto{\pgfqpoint{3.790573in}{2.046069in}}%
\pgfpathlineto{\pgfqpoint{3.880424in}{1.971066in}}%
\pgfpathlineto{\pgfqpoint{3.970276in}{1.896057in}}%
\pgfpathlineto{\pgfqpoint{4.060127in}{1.821044in}}%
\pgfpathlineto{\pgfqpoint{4.149978in}{1.746029in}}%
\pgfpathlineto{\pgfqpoint{4.239830in}{1.671013in}}%
\pgfpathlineto{\pgfqpoint{4.329681in}{1.595996in}}%
\pgfpathlineto{\pgfqpoint{4.419532in}{1.520979in}}%
\pgfpathlineto{\pgfqpoint{4.509384in}{1.445962in}}%
\pgfpathlineto{\pgfqpoint{4.599235in}{1.370944in}}%
\pgfpathlineto{\pgfqpoint{4.689086in}{1.295927in}}%
\pgfpathlineto{\pgfqpoint{4.778938in}{1.220909in}}%
\pgfpathlineto{\pgfqpoint{4.868789in}{1.145892in}}%
\pgfpathlineto{\pgfqpoint{4.958640in}{1.070874in}}%
\pgfpathlineto{\pgfqpoint{5.048492in}{0.995856in}}%
\pgfpathlineto{\pgfqpoint{5.138343in}{0.920839in}}%
\pgfpathlineto{\pgfqpoint{5.228194in}{0.845821in}}%
\pgfusepath{stroke}%
\end{pgfscope}%
\begin{pgfscope}%
\pgfpathrectangle{\pgfqpoint{0.605343in}{0.417642in}}{\pgfqpoint{4.842987in}{2.930688in}}%
\pgfusepath{clip}%
\pgfsetrectcap%
\pgfsetroundjoin%
\pgfsetlinewidth{1.003750pt}%
\definecolor{currentstroke}{rgb}{0.835294,0.368627,0.000000}%
\pgfsetstrokecolor{currentstroke}%
\pgfsetstrokeopacity{0.700000}%
\pgfsetdash{}{0pt}%
\pgfpathmoveto{\pgfqpoint{0.825479in}{2.875134in}}%
\pgfpathlineto{\pgfqpoint{0.915331in}{2.875130in}}%
\pgfpathlineto{\pgfqpoint{1.005182in}{2.875123in}}%
\pgfpathlineto{\pgfqpoint{1.095033in}{2.875111in}}%
\pgfpathlineto{\pgfqpoint{1.184885in}{2.875090in}}%
\pgfpathlineto{\pgfqpoint{1.274736in}{2.875053in}}%
\pgfpathlineto{\pgfqpoint{1.364587in}{2.874988in}}%
\pgfpathlineto{\pgfqpoint{1.454439in}{2.874873in}}%
\pgfpathlineto{\pgfqpoint{1.544290in}{2.874672in}}%
\pgfpathlineto{\pgfqpoint{1.634141in}{2.874318in}}%
\pgfpathlineto{\pgfqpoint{1.723992in}{2.873700in}}%
\pgfpathlineto{\pgfqpoint{1.813844in}{2.872619in}}%
\pgfpathlineto{\pgfqpoint{1.903695in}{2.870741in}}%
\pgfpathlineto{\pgfqpoint{1.993546in}{2.867504in}}%
\pgfpathlineto{\pgfqpoint{2.083398in}{2.861999in}}%
\pgfpathlineto{\pgfqpoint{2.173249in}{2.852842in}}%
\pgfpathlineto{\pgfqpoint{2.263100in}{2.838136in}}%
\pgfpathlineto{\pgfqpoint{2.352952in}{2.815673in}}%
\pgfpathlineto{\pgfqpoint{2.442803in}{2.783557in}}%
\pgfpathlineto{\pgfqpoint{2.532654in}{2.741002in}}%
\pgfpathlineto{\pgfqpoint{2.622506in}{2.688745in}}%
\pgfpathlineto{\pgfqpoint{2.712357in}{2.628659in}}%
\pgfpathlineto{\pgfqpoint{2.802208in}{2.562951in}}%
\pgfpathlineto{\pgfqpoint{2.892060in}{2.493536in}}%
\pgfpathlineto{\pgfqpoint{2.981911in}{2.421815in}}%
\pgfpathlineto{\pgfqpoint{3.071762in}{2.348710in}}%
\pgfpathlineto{\pgfqpoint{3.161614in}{2.274793in}}%
\pgfpathlineto{\pgfqpoint{3.251465in}{2.200406in}}%
\pgfpathlineto{\pgfqpoint{3.341316in}{2.125748in}}%
\pgfpathlineto{\pgfqpoint{3.431168in}{2.050936in}}%
\pgfpathlineto{\pgfqpoint{3.521019in}{1.976035in}}%
\pgfpathlineto{\pgfqpoint{3.610870in}{1.901084in}}%
\pgfpathlineto{\pgfqpoint{3.700722in}{1.826105in}}%
\pgfpathlineto{\pgfqpoint{3.790573in}{1.751109in}}%
\pgfpathlineto{\pgfqpoint{3.880424in}{1.676103in}}%
\pgfpathlineto{\pgfqpoint{3.970276in}{1.601093in}}%
\pgfpathlineto{\pgfqpoint{4.060127in}{1.526079in}}%
\pgfpathlineto{\pgfqpoint{4.149978in}{1.451064in}}%
\pgfpathlineto{\pgfqpoint{4.239830in}{1.376047in}}%
\pgfpathlineto{\pgfqpoint{4.329681in}{1.301030in}}%
\pgfpathlineto{\pgfqpoint{4.419532in}{1.226013in}}%
\pgfpathlineto{\pgfqpoint{4.509384in}{1.150996in}}%
\pgfpathlineto{\pgfqpoint{4.599235in}{1.075978in}}%
\pgfpathlineto{\pgfqpoint{4.689086in}{1.000961in}}%
\pgfpathlineto{\pgfqpoint{4.778938in}{0.925943in}}%
\pgfpathlineto{\pgfqpoint{4.868789in}{0.850926in}}%
\pgfpathlineto{\pgfqpoint{4.958640in}{0.775908in}}%
\pgfpathlineto{\pgfqpoint{5.048492in}{0.700890in}}%
\pgfpathlineto{\pgfqpoint{5.138343in}{0.625873in}}%
\pgfpathlineto{\pgfqpoint{5.228194in}{0.550855in}}%
\pgfusepath{stroke}%
\end{pgfscope}%
\begin{pgfscope}%
\pgfsetrectcap%
\pgfsetmiterjoin%
\pgfsetlinewidth{0.803000pt}%
\definecolor{currentstroke}{rgb}{0.000000,0.000000,0.000000}%
\pgfsetstrokecolor{currentstroke}%
\pgfsetdash{}{0pt}%
\pgfpathmoveto{\pgfqpoint{0.605343in}{0.417642in}}%
\pgfpathlineto{\pgfqpoint{0.605343in}{3.348330in}}%
\pgfusepath{stroke}%
\end{pgfscope}%
\begin{pgfscope}%
\pgfsetrectcap%
\pgfsetmiterjoin%
\pgfsetlinewidth{0.803000pt}%
\definecolor{currentstroke}{rgb}{0.000000,0.000000,0.000000}%
\pgfsetstrokecolor{currentstroke}%
\pgfsetdash{}{0pt}%
\pgfpathmoveto{\pgfqpoint{5.448330in}{0.417642in}}%
\pgfpathlineto{\pgfqpoint{5.448330in}{3.348330in}}%
\pgfusepath{stroke}%
\end{pgfscope}%
\begin{pgfscope}%
\pgfsetrectcap%
\pgfsetmiterjoin%
\pgfsetlinewidth{0.803000pt}%
\definecolor{currentstroke}{rgb}{0.000000,0.000000,0.000000}%
\pgfsetstrokecolor{currentstroke}%
\pgfsetdash{}{0pt}%
\pgfpathmoveto{\pgfqpoint{0.605343in}{0.417642in}}%
\pgfpathlineto{\pgfqpoint{5.448330in}{0.417642in}}%
\pgfusepath{stroke}%
\end{pgfscope}%
\begin{pgfscope}%
\pgfsetrectcap%
\pgfsetmiterjoin%
\pgfsetlinewidth{0.803000pt}%
\definecolor{currentstroke}{rgb}{0.000000,0.000000,0.000000}%
\pgfsetstrokecolor{currentstroke}%
\pgfsetdash{}{0pt}%
\pgfpathmoveto{\pgfqpoint{0.605343in}{3.348330in}}%
\pgfpathlineto{\pgfqpoint{5.448330in}{3.348330in}}%
\pgfusepath{stroke}%
\end{pgfscope}%
\begin{pgfscope}%
\pgfsetbuttcap%
\pgfsetmiterjoin%
\definecolor{currentfill}{rgb}{1.000000,1.000000,1.000000}%
\pgfsetfillcolor{currentfill}%
\pgfsetfillopacity{0.800000}%
\pgfsetlinewidth{1.003750pt}%
\definecolor{currentstroke}{rgb}{0.800000,0.800000,0.800000}%
\pgfsetstrokecolor{currentstroke}%
\pgfsetstrokeopacity{0.800000}%
\pgfsetdash{}{0pt}%
\pgfpathmoveto{\pgfqpoint{4.570239in}{2.794775in}}%
\pgfpathlineto{\pgfqpoint{5.370552in}{2.794775in}}%
\pgfpathquadraticcurveto{\pgfqpoint{5.392774in}{2.794775in}}{\pgfqpoint{5.392774in}{2.816997in}}%
\pgfpathlineto{\pgfqpoint{5.392774in}{3.270552in}}%
\pgfpathquadraticcurveto{\pgfqpoint{5.392774in}{3.292774in}}{\pgfqpoint{5.370552in}{3.292774in}}%
\pgfpathlineto{\pgfqpoint{4.570239in}{3.292774in}}%
\pgfpathquadraticcurveto{\pgfqpoint{4.548017in}{3.292774in}}{\pgfqpoint{4.548017in}{3.270552in}}%
\pgfpathlineto{\pgfqpoint{4.548017in}{2.816997in}}%
\pgfpathquadraticcurveto{\pgfqpoint{4.548017in}{2.794775in}}{\pgfqpoint{4.570239in}{2.794775in}}%
\pgfpathlineto{\pgfqpoint{4.570239in}{2.794775in}}%
\pgfpathclose%
\pgfusepath{stroke,fill}%
\end{pgfscope}%
\begin{pgfscope}%
\pgfsetrectcap%
\pgfsetroundjoin%
\pgfsetlinewidth{1.003750pt}%
\definecolor{currentstroke}{rgb}{0.000000,0.447059,0.698039}%
\pgfsetstrokecolor{currentstroke}%
\pgfsetstrokeopacity{0.700000}%
\pgfsetdash{}{0pt}%
\pgfpathmoveto{\pgfqpoint{4.592461in}{3.209441in}}%
\pgfpathlineto{\pgfqpoint{4.703572in}{3.209441in}}%
\pgfpathlineto{\pgfqpoint{4.814683in}{3.209441in}}%
\pgfusepath{stroke}%
\end{pgfscope}%
\begin{pgfscope}%
\definecolor{textcolor}{rgb}{0.000000,0.000000,0.000000}%
\pgfsetstrokecolor{textcolor}%
\pgfsetfillcolor{textcolor}%
\pgftext[x=4.903572in,y=3.170552in,left,base]{\color{textcolor}\rmfamily\fontsize{8.000000}{9.600000}\selectfont \(\displaystyle \tau_1=1\)}%
\end{pgfscope}%
\begin{pgfscope}%
\pgfsetrectcap%
\pgfsetroundjoin%
\pgfsetlinewidth{1.003750pt}%
\definecolor{currentstroke}{rgb}{0.000000,0.619608,0.450980}%
\pgfsetstrokecolor{currentstroke}%
\pgfsetstrokeopacity{0.700000}%
\pgfsetdash{}{0pt}%
\pgfpathmoveto{\pgfqpoint{4.592461in}{3.054552in}}%
\pgfpathlineto{\pgfqpoint{4.703572in}{3.054552in}}%
\pgfpathlineto{\pgfqpoint{4.814683in}{3.054552in}}%
\pgfusepath{stroke}%
\end{pgfscope}%
\begin{pgfscope}%
\definecolor{textcolor}{rgb}{0.000000,0.000000,0.000000}%
\pgfsetstrokecolor{textcolor}%
\pgfsetfillcolor{textcolor}%
\pgftext[x=4.903572in,y=3.015663in,left,base]{\color{textcolor}\rmfamily\fontsize{8.000000}{9.600000}\selectfont \(\displaystyle \tau_1=10\)}%
\end{pgfscope}%
\begin{pgfscope}%
\pgfsetrectcap%
\pgfsetroundjoin%
\pgfsetlinewidth{1.003750pt}%
\definecolor{currentstroke}{rgb}{0.835294,0.368627,0.000000}%
\pgfsetstrokecolor{currentstroke}%
\pgfsetstrokeopacity{0.700000}%
\pgfsetdash{}{0pt}%
\pgfpathmoveto{\pgfqpoint{4.592461in}{2.899664in}}%
\pgfpathlineto{\pgfqpoint{4.703572in}{2.899664in}}%
\pgfpathlineto{\pgfqpoint{4.814683in}{2.899664in}}%
\pgfusepath{stroke}%
\end{pgfscope}%
\begin{pgfscope}%
\definecolor{textcolor}{rgb}{0.000000,0.000000,0.000000}%
\pgfsetstrokecolor{textcolor}%
\pgfsetfillcolor{textcolor}%
\pgftext[x=4.903572in,y=2.860775in,left,base]{\color{textcolor}\rmfamily\fontsize{8.000000}{9.600000}\selectfont \(\displaystyle \tau_1=100\)}%
\end{pgfscope}%
\end{pgfpicture}%
\makeatother%
\endgroup%

        } % scalebox
        \caption{Power spectral density}
        \label{fig:burst_noise_psd}
    \end{subfigure}
    \begin{subfigure}{0.8\linewidth}
        \centering
        \scalebox{1}{%
            %% Creator: Matplotlib, PGF backend
%%
%% To include the figure in your LaTeX document, write
%%   \input{<filename>.pgf}
%%
%% Make sure the required packages are loaded in your preamble
%%   \usepackage{pgf}
%%
%% Also ensure that all the required font packages are loaded; for instance,
%% the lmodern package is sometimes necessary when using math font.
%%   \usepackage{lmodern}
%%
%% Figures using additional raster images can only be included by \input if
%% they are in the same directory as the main LaTeX file. For loading figures
%% from other directories you can use the `import` package
%%   \usepackage{import}
%%
%% and then include the figures with
%%   \import{<path to file>}{<filename>.pgf}
%%
%% Matplotlib used the following preamble
%%   \def\mathdefault#1{#1}
%%   \everymath=\expandafter{\the\everymath\displaystyle}
%%   \usepackage{siunitx}
%%   \sisetup{per-mode = symbol}%
%%   \ifdefined\pdftexversion\else  % non-pdftex case.
%%     \usepackage{fontspec}
%%   \fi
%%   \makeatletter\@ifpackageloaded{underscore}{}{\usepackage[strings]{underscore}}\makeatother
%%
\begingroup%
\makeatletter%
\begin{pgfpicture}%
\pgfpathrectangle{\pgfpointorigin}{\pgfqpoint{4.068242in}{2.514312in}}%
\pgfusepath{use as bounding box, clip}%
\begin{pgfscope}%
\pgfsetbuttcap%
\pgfsetmiterjoin%
\definecolor{currentfill}{rgb}{1.000000,1.000000,1.000000}%
\pgfsetfillcolor{currentfill}%
\pgfsetlinewidth{0.000000pt}%
\definecolor{currentstroke}{rgb}{1.000000,1.000000,1.000000}%
\pgfsetstrokecolor{currentstroke}%
\pgfsetdash{}{0pt}%
\pgfpathmoveto{\pgfqpoint{0.000000in}{0.000000in}}%
\pgfpathlineto{\pgfqpoint{4.068242in}{0.000000in}}%
\pgfpathlineto{\pgfqpoint{4.068242in}{2.514312in}}%
\pgfpathlineto{\pgfqpoint{0.000000in}{2.514312in}}%
\pgfpathlineto{\pgfqpoint{0.000000in}{0.000000in}}%
\pgfpathclose%
\pgfusepath{fill}%
\end{pgfscope}%
\begin{pgfscope}%
\pgfsetbuttcap%
\pgfsetmiterjoin%
\definecolor{currentfill}{rgb}{1.000000,1.000000,1.000000}%
\pgfsetfillcolor{currentfill}%
\pgfsetlinewidth{0.000000pt}%
\definecolor{currentstroke}{rgb}{0.000000,0.000000,0.000000}%
\pgfsetstrokecolor{currentstroke}%
\pgfsetstrokeopacity{0.000000}%
\pgfsetdash{}{0pt}%
\pgfpathmoveto{\pgfqpoint{0.589510in}{0.417642in}}%
\pgfpathlineto{\pgfqpoint{4.026572in}{0.417642in}}%
\pgfpathlineto{\pgfqpoint{4.026572in}{2.472642in}}%
\pgfpathlineto{\pgfqpoint{0.589510in}{2.472642in}}%
\pgfpathlineto{\pgfqpoint{0.589510in}{0.417642in}}%
\pgfpathclose%
\pgfusepath{fill}%
\end{pgfscope}%
\begin{pgfscope}%
\pgfpathrectangle{\pgfqpoint{0.589510in}{0.417642in}}{\pgfqpoint{3.437062in}{2.055000in}}%
\pgfusepath{clip}%
\pgfsetrectcap%
\pgfsetroundjoin%
\pgfsetlinewidth{0.803000pt}%
\definecolor{currentstroke}{rgb}{0.450000,0.450000,0.450000}%
\pgfsetstrokecolor{currentstroke}%
\pgfsetdash{}{0pt}%
\pgfpathmoveto{\pgfqpoint{0.745740in}{0.417642in}}%
\pgfpathlineto{\pgfqpoint{0.745740in}{2.472642in}}%
\pgfusepath{stroke}%
\end{pgfscope}%
\begin{pgfscope}%
\pgfsetbuttcap%
\pgfsetroundjoin%
\definecolor{currentfill}{rgb}{0.000000,0.000000,0.000000}%
\pgfsetfillcolor{currentfill}%
\pgfsetlinewidth{0.803000pt}%
\definecolor{currentstroke}{rgb}{0.000000,0.000000,0.000000}%
\pgfsetstrokecolor{currentstroke}%
\pgfsetdash{}{0pt}%
\pgfsys@defobject{currentmarker}{\pgfqpoint{0.000000in}{-0.048611in}}{\pgfqpoint{0.000000in}{0.000000in}}{%
\pgfpathmoveto{\pgfqpoint{0.000000in}{0.000000in}}%
\pgfpathlineto{\pgfqpoint{0.000000in}{-0.048611in}}%
\pgfusepath{stroke,fill}%
}%
\begin{pgfscope}%
\pgfsys@transformshift{0.745740in}{0.417642in}%
\pgfsys@useobject{currentmarker}{}%
\end{pgfscope}%
\end{pgfscope}%
\begin{pgfscope}%
\definecolor{textcolor}{rgb}{0.000000,0.000000,0.000000}%
\pgfsetstrokecolor{textcolor}%
\pgfsetfillcolor{textcolor}%
\pgftext[x=0.745740in,y=0.320420in,,top]{\color{textcolor}{\rmfamily\fontsize{8.000000}{9.600000}\selectfont\catcode`\^=\active\def^{\ifmmode\sp\else\^{}\fi}\catcode`\%=\active\def%{\%}$\mathdefault{10^{-2}}$}}%
\end{pgfscope}%
\begin{pgfscope}%
\pgfpathrectangle{\pgfqpoint{0.589510in}{0.417642in}}{\pgfqpoint{3.437062in}{2.055000in}}%
\pgfusepath{clip}%
\pgfsetrectcap%
\pgfsetroundjoin%
\pgfsetlinewidth{0.803000pt}%
\definecolor{currentstroke}{rgb}{0.450000,0.450000,0.450000}%
\pgfsetstrokecolor{currentstroke}%
\pgfsetdash{}{0pt}%
\pgfpathmoveto{\pgfqpoint{1.526890in}{0.417642in}}%
\pgfpathlineto{\pgfqpoint{1.526890in}{2.472642in}}%
\pgfusepath{stroke}%
\end{pgfscope}%
\begin{pgfscope}%
\pgfsetbuttcap%
\pgfsetroundjoin%
\definecolor{currentfill}{rgb}{0.000000,0.000000,0.000000}%
\pgfsetfillcolor{currentfill}%
\pgfsetlinewidth{0.803000pt}%
\definecolor{currentstroke}{rgb}{0.000000,0.000000,0.000000}%
\pgfsetstrokecolor{currentstroke}%
\pgfsetdash{}{0pt}%
\pgfsys@defobject{currentmarker}{\pgfqpoint{0.000000in}{-0.048611in}}{\pgfqpoint{0.000000in}{0.000000in}}{%
\pgfpathmoveto{\pgfqpoint{0.000000in}{0.000000in}}%
\pgfpathlineto{\pgfqpoint{0.000000in}{-0.048611in}}%
\pgfusepath{stroke,fill}%
}%
\begin{pgfscope}%
\pgfsys@transformshift{1.526890in}{0.417642in}%
\pgfsys@useobject{currentmarker}{}%
\end{pgfscope}%
\end{pgfscope}%
\begin{pgfscope}%
\definecolor{textcolor}{rgb}{0.000000,0.000000,0.000000}%
\pgfsetstrokecolor{textcolor}%
\pgfsetfillcolor{textcolor}%
\pgftext[x=1.526890in,y=0.320420in,,top]{\color{textcolor}{\rmfamily\fontsize{8.000000}{9.600000}\selectfont\catcode`\^=\active\def^{\ifmmode\sp\else\^{}\fi}\catcode`\%=\active\def%{\%}$\mathdefault{10^{-1}}$}}%
\end{pgfscope}%
\begin{pgfscope}%
\pgfpathrectangle{\pgfqpoint{0.589510in}{0.417642in}}{\pgfqpoint{3.437062in}{2.055000in}}%
\pgfusepath{clip}%
\pgfsetrectcap%
\pgfsetroundjoin%
\pgfsetlinewidth{0.803000pt}%
\definecolor{currentstroke}{rgb}{0.450000,0.450000,0.450000}%
\pgfsetstrokecolor{currentstroke}%
\pgfsetdash{}{0pt}%
\pgfpathmoveto{\pgfqpoint{2.308041in}{0.417642in}}%
\pgfpathlineto{\pgfqpoint{2.308041in}{2.472642in}}%
\pgfusepath{stroke}%
\end{pgfscope}%
\begin{pgfscope}%
\pgfsetbuttcap%
\pgfsetroundjoin%
\definecolor{currentfill}{rgb}{0.000000,0.000000,0.000000}%
\pgfsetfillcolor{currentfill}%
\pgfsetlinewidth{0.803000pt}%
\definecolor{currentstroke}{rgb}{0.000000,0.000000,0.000000}%
\pgfsetstrokecolor{currentstroke}%
\pgfsetdash{}{0pt}%
\pgfsys@defobject{currentmarker}{\pgfqpoint{0.000000in}{-0.048611in}}{\pgfqpoint{0.000000in}{0.000000in}}{%
\pgfpathmoveto{\pgfqpoint{0.000000in}{0.000000in}}%
\pgfpathlineto{\pgfqpoint{0.000000in}{-0.048611in}}%
\pgfusepath{stroke,fill}%
}%
\begin{pgfscope}%
\pgfsys@transformshift{2.308041in}{0.417642in}%
\pgfsys@useobject{currentmarker}{}%
\end{pgfscope}%
\end{pgfscope}%
\begin{pgfscope}%
\definecolor{textcolor}{rgb}{0.000000,0.000000,0.000000}%
\pgfsetstrokecolor{textcolor}%
\pgfsetfillcolor{textcolor}%
\pgftext[x=2.308041in,y=0.320420in,,top]{\color{textcolor}{\rmfamily\fontsize{8.000000}{9.600000}\selectfont\catcode`\^=\active\def^{\ifmmode\sp\else\^{}\fi}\catcode`\%=\active\def%{\%}$\mathdefault{10^{0}}$}}%
\end{pgfscope}%
\begin{pgfscope}%
\pgfpathrectangle{\pgfqpoint{0.589510in}{0.417642in}}{\pgfqpoint{3.437062in}{2.055000in}}%
\pgfusepath{clip}%
\pgfsetrectcap%
\pgfsetroundjoin%
\pgfsetlinewidth{0.803000pt}%
\definecolor{currentstroke}{rgb}{0.450000,0.450000,0.450000}%
\pgfsetstrokecolor{currentstroke}%
\pgfsetdash{}{0pt}%
\pgfpathmoveto{\pgfqpoint{3.089191in}{0.417642in}}%
\pgfpathlineto{\pgfqpoint{3.089191in}{2.472642in}}%
\pgfusepath{stroke}%
\end{pgfscope}%
\begin{pgfscope}%
\pgfsetbuttcap%
\pgfsetroundjoin%
\definecolor{currentfill}{rgb}{0.000000,0.000000,0.000000}%
\pgfsetfillcolor{currentfill}%
\pgfsetlinewidth{0.803000pt}%
\definecolor{currentstroke}{rgb}{0.000000,0.000000,0.000000}%
\pgfsetstrokecolor{currentstroke}%
\pgfsetdash{}{0pt}%
\pgfsys@defobject{currentmarker}{\pgfqpoint{0.000000in}{-0.048611in}}{\pgfqpoint{0.000000in}{0.000000in}}{%
\pgfpathmoveto{\pgfqpoint{0.000000in}{0.000000in}}%
\pgfpathlineto{\pgfqpoint{0.000000in}{-0.048611in}}%
\pgfusepath{stroke,fill}%
}%
\begin{pgfscope}%
\pgfsys@transformshift{3.089191in}{0.417642in}%
\pgfsys@useobject{currentmarker}{}%
\end{pgfscope}%
\end{pgfscope}%
\begin{pgfscope}%
\definecolor{textcolor}{rgb}{0.000000,0.000000,0.000000}%
\pgfsetstrokecolor{textcolor}%
\pgfsetfillcolor{textcolor}%
\pgftext[x=3.089191in,y=0.320420in,,top]{\color{textcolor}{\rmfamily\fontsize{8.000000}{9.600000}\selectfont\catcode`\^=\active\def^{\ifmmode\sp\else\^{}\fi}\catcode`\%=\active\def%{\%}$\mathdefault{10^{1}}$}}%
\end{pgfscope}%
\begin{pgfscope}%
\pgfpathrectangle{\pgfqpoint{0.589510in}{0.417642in}}{\pgfqpoint{3.437062in}{2.055000in}}%
\pgfusepath{clip}%
\pgfsetrectcap%
\pgfsetroundjoin%
\pgfsetlinewidth{0.803000pt}%
\definecolor{currentstroke}{rgb}{0.450000,0.450000,0.450000}%
\pgfsetstrokecolor{currentstroke}%
\pgfsetdash{}{0pt}%
\pgfpathmoveto{\pgfqpoint{3.870342in}{0.417642in}}%
\pgfpathlineto{\pgfqpoint{3.870342in}{2.472642in}}%
\pgfusepath{stroke}%
\end{pgfscope}%
\begin{pgfscope}%
\pgfsetbuttcap%
\pgfsetroundjoin%
\definecolor{currentfill}{rgb}{0.000000,0.000000,0.000000}%
\pgfsetfillcolor{currentfill}%
\pgfsetlinewidth{0.803000pt}%
\definecolor{currentstroke}{rgb}{0.000000,0.000000,0.000000}%
\pgfsetstrokecolor{currentstroke}%
\pgfsetdash{}{0pt}%
\pgfsys@defobject{currentmarker}{\pgfqpoint{0.000000in}{-0.048611in}}{\pgfqpoint{0.000000in}{0.000000in}}{%
\pgfpathmoveto{\pgfqpoint{0.000000in}{0.000000in}}%
\pgfpathlineto{\pgfqpoint{0.000000in}{-0.048611in}}%
\pgfusepath{stroke,fill}%
}%
\begin{pgfscope}%
\pgfsys@transformshift{3.870342in}{0.417642in}%
\pgfsys@useobject{currentmarker}{}%
\end{pgfscope}%
\end{pgfscope}%
\begin{pgfscope}%
\definecolor{textcolor}{rgb}{0.000000,0.000000,0.000000}%
\pgfsetstrokecolor{textcolor}%
\pgfsetfillcolor{textcolor}%
\pgftext[x=3.870342in,y=0.320420in,,top]{\color{textcolor}{\rmfamily\fontsize{8.000000}{9.600000}\selectfont\catcode`\^=\active\def^{\ifmmode\sp\else\^{}\fi}\catcode`\%=\active\def%{\%}$\mathdefault{10^{2}}$}}%
\end{pgfscope}%
\begin{pgfscope}%
\pgfpathrectangle{\pgfqpoint{0.589510in}{0.417642in}}{\pgfqpoint{3.437062in}{2.055000in}}%
\pgfusepath{clip}%
\pgfsetrectcap%
\pgfsetroundjoin%
\pgfsetlinewidth{0.803000pt}%
\definecolor{currentstroke}{rgb}{0.850000,0.850000,0.850000}%
\pgfsetstrokecolor{currentstroke}%
\pgfsetdash{}{0pt}%
\pgfpathmoveto{\pgfqpoint{0.624738in}{0.417642in}}%
\pgfpathlineto{\pgfqpoint{0.624738in}{2.472642in}}%
\pgfusepath{stroke}%
\end{pgfscope}%
\begin{pgfscope}%
\pgfsetbuttcap%
\pgfsetroundjoin%
\definecolor{currentfill}{rgb}{0.000000,0.000000,0.000000}%
\pgfsetfillcolor{currentfill}%
\pgfsetlinewidth{0.602250pt}%
\definecolor{currentstroke}{rgb}{0.000000,0.000000,0.000000}%
\pgfsetstrokecolor{currentstroke}%
\pgfsetdash{}{0pt}%
\pgfsys@defobject{currentmarker}{\pgfqpoint{0.000000in}{-0.027778in}}{\pgfqpoint{0.000000in}{0.000000in}}{%
\pgfpathmoveto{\pgfqpoint{0.000000in}{0.000000in}}%
\pgfpathlineto{\pgfqpoint{0.000000in}{-0.027778in}}%
\pgfusepath{stroke,fill}%
}%
\begin{pgfscope}%
\pgfsys@transformshift{0.624738in}{0.417642in}%
\pgfsys@useobject{currentmarker}{}%
\end{pgfscope}%
\end{pgfscope}%
\begin{pgfscope}%
\pgfpathrectangle{\pgfqpoint{0.589510in}{0.417642in}}{\pgfqpoint{3.437062in}{2.055000in}}%
\pgfusepath{clip}%
\pgfsetrectcap%
\pgfsetroundjoin%
\pgfsetlinewidth{0.803000pt}%
\definecolor{currentstroke}{rgb}{0.850000,0.850000,0.850000}%
\pgfsetstrokecolor{currentstroke}%
\pgfsetdash{}{0pt}%
\pgfpathmoveto{\pgfqpoint{0.670039in}{0.417642in}}%
\pgfpathlineto{\pgfqpoint{0.670039in}{2.472642in}}%
\pgfusepath{stroke}%
\end{pgfscope}%
\begin{pgfscope}%
\pgfsetbuttcap%
\pgfsetroundjoin%
\definecolor{currentfill}{rgb}{0.000000,0.000000,0.000000}%
\pgfsetfillcolor{currentfill}%
\pgfsetlinewidth{0.602250pt}%
\definecolor{currentstroke}{rgb}{0.000000,0.000000,0.000000}%
\pgfsetstrokecolor{currentstroke}%
\pgfsetdash{}{0pt}%
\pgfsys@defobject{currentmarker}{\pgfqpoint{0.000000in}{-0.027778in}}{\pgfqpoint{0.000000in}{0.000000in}}{%
\pgfpathmoveto{\pgfqpoint{0.000000in}{0.000000in}}%
\pgfpathlineto{\pgfqpoint{0.000000in}{-0.027778in}}%
\pgfusepath{stroke,fill}%
}%
\begin{pgfscope}%
\pgfsys@transformshift{0.670039in}{0.417642in}%
\pgfsys@useobject{currentmarker}{}%
\end{pgfscope}%
\end{pgfscope}%
\begin{pgfscope}%
\pgfpathrectangle{\pgfqpoint{0.589510in}{0.417642in}}{\pgfqpoint{3.437062in}{2.055000in}}%
\pgfusepath{clip}%
\pgfsetrectcap%
\pgfsetroundjoin%
\pgfsetlinewidth{0.803000pt}%
\definecolor{currentstroke}{rgb}{0.850000,0.850000,0.850000}%
\pgfsetstrokecolor{currentstroke}%
\pgfsetdash{}{0pt}%
\pgfpathmoveto{\pgfqpoint{0.709996in}{0.417642in}}%
\pgfpathlineto{\pgfqpoint{0.709996in}{2.472642in}}%
\pgfusepath{stroke}%
\end{pgfscope}%
\begin{pgfscope}%
\pgfsetbuttcap%
\pgfsetroundjoin%
\definecolor{currentfill}{rgb}{0.000000,0.000000,0.000000}%
\pgfsetfillcolor{currentfill}%
\pgfsetlinewidth{0.602250pt}%
\definecolor{currentstroke}{rgb}{0.000000,0.000000,0.000000}%
\pgfsetstrokecolor{currentstroke}%
\pgfsetdash{}{0pt}%
\pgfsys@defobject{currentmarker}{\pgfqpoint{0.000000in}{-0.027778in}}{\pgfqpoint{0.000000in}{0.000000in}}{%
\pgfpathmoveto{\pgfqpoint{0.000000in}{0.000000in}}%
\pgfpathlineto{\pgfqpoint{0.000000in}{-0.027778in}}%
\pgfusepath{stroke,fill}%
}%
\begin{pgfscope}%
\pgfsys@transformshift{0.709996in}{0.417642in}%
\pgfsys@useobject{currentmarker}{}%
\end{pgfscope}%
\end{pgfscope}%
\begin{pgfscope}%
\pgfpathrectangle{\pgfqpoint{0.589510in}{0.417642in}}{\pgfqpoint{3.437062in}{2.055000in}}%
\pgfusepath{clip}%
\pgfsetrectcap%
\pgfsetroundjoin%
\pgfsetlinewidth{0.803000pt}%
\definecolor{currentstroke}{rgb}{0.850000,0.850000,0.850000}%
\pgfsetstrokecolor{currentstroke}%
\pgfsetdash{}{0pt}%
\pgfpathmoveto{\pgfqpoint{0.980890in}{0.417642in}}%
\pgfpathlineto{\pgfqpoint{0.980890in}{2.472642in}}%
\pgfusepath{stroke}%
\end{pgfscope}%
\begin{pgfscope}%
\pgfsetbuttcap%
\pgfsetroundjoin%
\definecolor{currentfill}{rgb}{0.000000,0.000000,0.000000}%
\pgfsetfillcolor{currentfill}%
\pgfsetlinewidth{0.602250pt}%
\definecolor{currentstroke}{rgb}{0.000000,0.000000,0.000000}%
\pgfsetstrokecolor{currentstroke}%
\pgfsetdash{}{0pt}%
\pgfsys@defobject{currentmarker}{\pgfqpoint{0.000000in}{-0.027778in}}{\pgfqpoint{0.000000in}{0.000000in}}{%
\pgfpathmoveto{\pgfqpoint{0.000000in}{0.000000in}}%
\pgfpathlineto{\pgfqpoint{0.000000in}{-0.027778in}}%
\pgfusepath{stroke,fill}%
}%
\begin{pgfscope}%
\pgfsys@transformshift{0.980890in}{0.417642in}%
\pgfsys@useobject{currentmarker}{}%
\end{pgfscope}%
\end{pgfscope}%
\begin{pgfscope}%
\pgfpathrectangle{\pgfqpoint{0.589510in}{0.417642in}}{\pgfqpoint{3.437062in}{2.055000in}}%
\pgfusepath{clip}%
\pgfsetrectcap%
\pgfsetroundjoin%
\pgfsetlinewidth{0.803000pt}%
\definecolor{currentstroke}{rgb}{0.850000,0.850000,0.850000}%
\pgfsetstrokecolor{currentstroke}%
\pgfsetdash{}{0pt}%
\pgfpathmoveto{\pgfqpoint{1.118443in}{0.417642in}}%
\pgfpathlineto{\pgfqpoint{1.118443in}{2.472642in}}%
\pgfusepath{stroke}%
\end{pgfscope}%
\begin{pgfscope}%
\pgfsetbuttcap%
\pgfsetroundjoin%
\definecolor{currentfill}{rgb}{0.000000,0.000000,0.000000}%
\pgfsetfillcolor{currentfill}%
\pgfsetlinewidth{0.602250pt}%
\definecolor{currentstroke}{rgb}{0.000000,0.000000,0.000000}%
\pgfsetstrokecolor{currentstroke}%
\pgfsetdash{}{0pt}%
\pgfsys@defobject{currentmarker}{\pgfqpoint{0.000000in}{-0.027778in}}{\pgfqpoint{0.000000in}{0.000000in}}{%
\pgfpathmoveto{\pgfqpoint{0.000000in}{0.000000in}}%
\pgfpathlineto{\pgfqpoint{0.000000in}{-0.027778in}}%
\pgfusepath{stroke,fill}%
}%
\begin{pgfscope}%
\pgfsys@transformshift{1.118443in}{0.417642in}%
\pgfsys@useobject{currentmarker}{}%
\end{pgfscope}%
\end{pgfscope}%
\begin{pgfscope}%
\pgfpathrectangle{\pgfqpoint{0.589510in}{0.417642in}}{\pgfqpoint{3.437062in}{2.055000in}}%
\pgfusepath{clip}%
\pgfsetrectcap%
\pgfsetroundjoin%
\pgfsetlinewidth{0.803000pt}%
\definecolor{currentstroke}{rgb}{0.850000,0.850000,0.850000}%
\pgfsetstrokecolor{currentstroke}%
\pgfsetdash{}{0pt}%
\pgfpathmoveto{\pgfqpoint{1.216039in}{0.417642in}}%
\pgfpathlineto{\pgfqpoint{1.216039in}{2.472642in}}%
\pgfusepath{stroke}%
\end{pgfscope}%
\begin{pgfscope}%
\pgfsetbuttcap%
\pgfsetroundjoin%
\definecolor{currentfill}{rgb}{0.000000,0.000000,0.000000}%
\pgfsetfillcolor{currentfill}%
\pgfsetlinewidth{0.602250pt}%
\definecolor{currentstroke}{rgb}{0.000000,0.000000,0.000000}%
\pgfsetstrokecolor{currentstroke}%
\pgfsetdash{}{0pt}%
\pgfsys@defobject{currentmarker}{\pgfqpoint{0.000000in}{-0.027778in}}{\pgfqpoint{0.000000in}{0.000000in}}{%
\pgfpathmoveto{\pgfqpoint{0.000000in}{0.000000in}}%
\pgfpathlineto{\pgfqpoint{0.000000in}{-0.027778in}}%
\pgfusepath{stroke,fill}%
}%
\begin{pgfscope}%
\pgfsys@transformshift{1.216039in}{0.417642in}%
\pgfsys@useobject{currentmarker}{}%
\end{pgfscope}%
\end{pgfscope}%
\begin{pgfscope}%
\pgfpathrectangle{\pgfqpoint{0.589510in}{0.417642in}}{\pgfqpoint{3.437062in}{2.055000in}}%
\pgfusepath{clip}%
\pgfsetrectcap%
\pgfsetroundjoin%
\pgfsetlinewidth{0.803000pt}%
\definecolor{currentstroke}{rgb}{0.850000,0.850000,0.850000}%
\pgfsetstrokecolor{currentstroke}%
\pgfsetdash{}{0pt}%
\pgfpathmoveto{\pgfqpoint{1.291741in}{0.417642in}}%
\pgfpathlineto{\pgfqpoint{1.291741in}{2.472642in}}%
\pgfusepath{stroke}%
\end{pgfscope}%
\begin{pgfscope}%
\pgfsetbuttcap%
\pgfsetroundjoin%
\definecolor{currentfill}{rgb}{0.000000,0.000000,0.000000}%
\pgfsetfillcolor{currentfill}%
\pgfsetlinewidth{0.602250pt}%
\definecolor{currentstroke}{rgb}{0.000000,0.000000,0.000000}%
\pgfsetstrokecolor{currentstroke}%
\pgfsetdash{}{0pt}%
\pgfsys@defobject{currentmarker}{\pgfqpoint{0.000000in}{-0.027778in}}{\pgfqpoint{0.000000in}{0.000000in}}{%
\pgfpathmoveto{\pgfqpoint{0.000000in}{0.000000in}}%
\pgfpathlineto{\pgfqpoint{0.000000in}{-0.027778in}}%
\pgfusepath{stroke,fill}%
}%
\begin{pgfscope}%
\pgfsys@transformshift{1.291741in}{0.417642in}%
\pgfsys@useobject{currentmarker}{}%
\end{pgfscope}%
\end{pgfscope}%
\begin{pgfscope}%
\pgfpathrectangle{\pgfqpoint{0.589510in}{0.417642in}}{\pgfqpoint{3.437062in}{2.055000in}}%
\pgfusepath{clip}%
\pgfsetrectcap%
\pgfsetroundjoin%
\pgfsetlinewidth{0.803000pt}%
\definecolor{currentstroke}{rgb}{0.850000,0.850000,0.850000}%
\pgfsetstrokecolor{currentstroke}%
\pgfsetdash{}{0pt}%
\pgfpathmoveto{\pgfqpoint{1.353593in}{0.417642in}}%
\pgfpathlineto{\pgfqpoint{1.353593in}{2.472642in}}%
\pgfusepath{stroke}%
\end{pgfscope}%
\begin{pgfscope}%
\pgfsetbuttcap%
\pgfsetroundjoin%
\definecolor{currentfill}{rgb}{0.000000,0.000000,0.000000}%
\pgfsetfillcolor{currentfill}%
\pgfsetlinewidth{0.602250pt}%
\definecolor{currentstroke}{rgb}{0.000000,0.000000,0.000000}%
\pgfsetstrokecolor{currentstroke}%
\pgfsetdash{}{0pt}%
\pgfsys@defobject{currentmarker}{\pgfqpoint{0.000000in}{-0.027778in}}{\pgfqpoint{0.000000in}{0.000000in}}{%
\pgfpathmoveto{\pgfqpoint{0.000000in}{0.000000in}}%
\pgfpathlineto{\pgfqpoint{0.000000in}{-0.027778in}}%
\pgfusepath{stroke,fill}%
}%
\begin{pgfscope}%
\pgfsys@transformshift{1.353593in}{0.417642in}%
\pgfsys@useobject{currentmarker}{}%
\end{pgfscope}%
\end{pgfscope}%
\begin{pgfscope}%
\pgfpathrectangle{\pgfqpoint{0.589510in}{0.417642in}}{\pgfqpoint{3.437062in}{2.055000in}}%
\pgfusepath{clip}%
\pgfsetrectcap%
\pgfsetroundjoin%
\pgfsetlinewidth{0.803000pt}%
\definecolor{currentstroke}{rgb}{0.850000,0.850000,0.850000}%
\pgfsetstrokecolor{currentstroke}%
\pgfsetdash{}{0pt}%
\pgfpathmoveto{\pgfqpoint{1.405889in}{0.417642in}}%
\pgfpathlineto{\pgfqpoint{1.405889in}{2.472642in}}%
\pgfusepath{stroke}%
\end{pgfscope}%
\begin{pgfscope}%
\pgfsetbuttcap%
\pgfsetroundjoin%
\definecolor{currentfill}{rgb}{0.000000,0.000000,0.000000}%
\pgfsetfillcolor{currentfill}%
\pgfsetlinewidth{0.602250pt}%
\definecolor{currentstroke}{rgb}{0.000000,0.000000,0.000000}%
\pgfsetstrokecolor{currentstroke}%
\pgfsetdash{}{0pt}%
\pgfsys@defobject{currentmarker}{\pgfqpoint{0.000000in}{-0.027778in}}{\pgfqpoint{0.000000in}{0.000000in}}{%
\pgfpathmoveto{\pgfqpoint{0.000000in}{0.000000in}}%
\pgfpathlineto{\pgfqpoint{0.000000in}{-0.027778in}}%
\pgfusepath{stroke,fill}%
}%
\begin{pgfscope}%
\pgfsys@transformshift{1.405889in}{0.417642in}%
\pgfsys@useobject{currentmarker}{}%
\end{pgfscope}%
\end{pgfscope}%
\begin{pgfscope}%
\pgfpathrectangle{\pgfqpoint{0.589510in}{0.417642in}}{\pgfqpoint{3.437062in}{2.055000in}}%
\pgfusepath{clip}%
\pgfsetrectcap%
\pgfsetroundjoin%
\pgfsetlinewidth{0.803000pt}%
\definecolor{currentstroke}{rgb}{0.850000,0.850000,0.850000}%
\pgfsetstrokecolor{currentstroke}%
\pgfsetdash{}{0pt}%
\pgfpathmoveto{\pgfqpoint{1.451189in}{0.417642in}}%
\pgfpathlineto{\pgfqpoint{1.451189in}{2.472642in}}%
\pgfusepath{stroke}%
\end{pgfscope}%
\begin{pgfscope}%
\pgfsetbuttcap%
\pgfsetroundjoin%
\definecolor{currentfill}{rgb}{0.000000,0.000000,0.000000}%
\pgfsetfillcolor{currentfill}%
\pgfsetlinewidth{0.602250pt}%
\definecolor{currentstroke}{rgb}{0.000000,0.000000,0.000000}%
\pgfsetstrokecolor{currentstroke}%
\pgfsetdash{}{0pt}%
\pgfsys@defobject{currentmarker}{\pgfqpoint{0.000000in}{-0.027778in}}{\pgfqpoint{0.000000in}{0.000000in}}{%
\pgfpathmoveto{\pgfqpoint{0.000000in}{0.000000in}}%
\pgfpathlineto{\pgfqpoint{0.000000in}{-0.027778in}}%
\pgfusepath{stroke,fill}%
}%
\begin{pgfscope}%
\pgfsys@transformshift{1.451189in}{0.417642in}%
\pgfsys@useobject{currentmarker}{}%
\end{pgfscope}%
\end{pgfscope}%
\begin{pgfscope}%
\pgfpathrectangle{\pgfqpoint{0.589510in}{0.417642in}}{\pgfqpoint{3.437062in}{2.055000in}}%
\pgfusepath{clip}%
\pgfsetrectcap%
\pgfsetroundjoin%
\pgfsetlinewidth{0.803000pt}%
\definecolor{currentstroke}{rgb}{0.850000,0.850000,0.850000}%
\pgfsetstrokecolor{currentstroke}%
\pgfsetdash{}{0pt}%
\pgfpathmoveto{\pgfqpoint{1.491147in}{0.417642in}}%
\pgfpathlineto{\pgfqpoint{1.491147in}{2.472642in}}%
\pgfusepath{stroke}%
\end{pgfscope}%
\begin{pgfscope}%
\pgfsetbuttcap%
\pgfsetroundjoin%
\definecolor{currentfill}{rgb}{0.000000,0.000000,0.000000}%
\pgfsetfillcolor{currentfill}%
\pgfsetlinewidth{0.602250pt}%
\definecolor{currentstroke}{rgb}{0.000000,0.000000,0.000000}%
\pgfsetstrokecolor{currentstroke}%
\pgfsetdash{}{0pt}%
\pgfsys@defobject{currentmarker}{\pgfqpoint{0.000000in}{-0.027778in}}{\pgfqpoint{0.000000in}{0.000000in}}{%
\pgfpathmoveto{\pgfqpoint{0.000000in}{0.000000in}}%
\pgfpathlineto{\pgfqpoint{0.000000in}{-0.027778in}}%
\pgfusepath{stroke,fill}%
}%
\begin{pgfscope}%
\pgfsys@transformshift{1.491147in}{0.417642in}%
\pgfsys@useobject{currentmarker}{}%
\end{pgfscope}%
\end{pgfscope}%
\begin{pgfscope}%
\pgfpathrectangle{\pgfqpoint{0.589510in}{0.417642in}}{\pgfqpoint{3.437062in}{2.055000in}}%
\pgfusepath{clip}%
\pgfsetrectcap%
\pgfsetroundjoin%
\pgfsetlinewidth{0.803000pt}%
\definecolor{currentstroke}{rgb}{0.850000,0.850000,0.850000}%
\pgfsetstrokecolor{currentstroke}%
\pgfsetdash{}{0pt}%
\pgfpathmoveto{\pgfqpoint{1.762040in}{0.417642in}}%
\pgfpathlineto{\pgfqpoint{1.762040in}{2.472642in}}%
\pgfusepath{stroke}%
\end{pgfscope}%
\begin{pgfscope}%
\pgfsetbuttcap%
\pgfsetroundjoin%
\definecolor{currentfill}{rgb}{0.000000,0.000000,0.000000}%
\pgfsetfillcolor{currentfill}%
\pgfsetlinewidth{0.602250pt}%
\definecolor{currentstroke}{rgb}{0.000000,0.000000,0.000000}%
\pgfsetstrokecolor{currentstroke}%
\pgfsetdash{}{0pt}%
\pgfsys@defobject{currentmarker}{\pgfqpoint{0.000000in}{-0.027778in}}{\pgfqpoint{0.000000in}{0.000000in}}{%
\pgfpathmoveto{\pgfqpoint{0.000000in}{0.000000in}}%
\pgfpathlineto{\pgfqpoint{0.000000in}{-0.027778in}}%
\pgfusepath{stroke,fill}%
}%
\begin{pgfscope}%
\pgfsys@transformshift{1.762040in}{0.417642in}%
\pgfsys@useobject{currentmarker}{}%
\end{pgfscope}%
\end{pgfscope}%
\begin{pgfscope}%
\pgfpathrectangle{\pgfqpoint{0.589510in}{0.417642in}}{\pgfqpoint{3.437062in}{2.055000in}}%
\pgfusepath{clip}%
\pgfsetrectcap%
\pgfsetroundjoin%
\pgfsetlinewidth{0.803000pt}%
\definecolor{currentstroke}{rgb}{0.850000,0.850000,0.850000}%
\pgfsetstrokecolor{currentstroke}%
\pgfsetdash{}{0pt}%
\pgfpathmoveto{\pgfqpoint{1.899594in}{0.417642in}}%
\pgfpathlineto{\pgfqpoint{1.899594in}{2.472642in}}%
\pgfusepath{stroke}%
\end{pgfscope}%
\begin{pgfscope}%
\pgfsetbuttcap%
\pgfsetroundjoin%
\definecolor{currentfill}{rgb}{0.000000,0.000000,0.000000}%
\pgfsetfillcolor{currentfill}%
\pgfsetlinewidth{0.602250pt}%
\definecolor{currentstroke}{rgb}{0.000000,0.000000,0.000000}%
\pgfsetstrokecolor{currentstroke}%
\pgfsetdash{}{0pt}%
\pgfsys@defobject{currentmarker}{\pgfqpoint{0.000000in}{-0.027778in}}{\pgfqpoint{0.000000in}{0.000000in}}{%
\pgfpathmoveto{\pgfqpoint{0.000000in}{0.000000in}}%
\pgfpathlineto{\pgfqpoint{0.000000in}{-0.027778in}}%
\pgfusepath{stroke,fill}%
}%
\begin{pgfscope}%
\pgfsys@transformshift{1.899594in}{0.417642in}%
\pgfsys@useobject{currentmarker}{}%
\end{pgfscope}%
\end{pgfscope}%
\begin{pgfscope}%
\pgfpathrectangle{\pgfqpoint{0.589510in}{0.417642in}}{\pgfqpoint{3.437062in}{2.055000in}}%
\pgfusepath{clip}%
\pgfsetrectcap%
\pgfsetroundjoin%
\pgfsetlinewidth{0.803000pt}%
\definecolor{currentstroke}{rgb}{0.850000,0.850000,0.850000}%
\pgfsetstrokecolor{currentstroke}%
\pgfsetdash{}{0pt}%
\pgfpathmoveto{\pgfqpoint{1.997190in}{0.417642in}}%
\pgfpathlineto{\pgfqpoint{1.997190in}{2.472642in}}%
\pgfusepath{stroke}%
\end{pgfscope}%
\begin{pgfscope}%
\pgfsetbuttcap%
\pgfsetroundjoin%
\definecolor{currentfill}{rgb}{0.000000,0.000000,0.000000}%
\pgfsetfillcolor{currentfill}%
\pgfsetlinewidth{0.602250pt}%
\definecolor{currentstroke}{rgb}{0.000000,0.000000,0.000000}%
\pgfsetstrokecolor{currentstroke}%
\pgfsetdash{}{0pt}%
\pgfsys@defobject{currentmarker}{\pgfqpoint{0.000000in}{-0.027778in}}{\pgfqpoint{0.000000in}{0.000000in}}{%
\pgfpathmoveto{\pgfqpoint{0.000000in}{0.000000in}}%
\pgfpathlineto{\pgfqpoint{0.000000in}{-0.027778in}}%
\pgfusepath{stroke,fill}%
}%
\begin{pgfscope}%
\pgfsys@transformshift{1.997190in}{0.417642in}%
\pgfsys@useobject{currentmarker}{}%
\end{pgfscope}%
\end{pgfscope}%
\begin{pgfscope}%
\pgfpathrectangle{\pgfqpoint{0.589510in}{0.417642in}}{\pgfqpoint{3.437062in}{2.055000in}}%
\pgfusepath{clip}%
\pgfsetrectcap%
\pgfsetroundjoin%
\pgfsetlinewidth{0.803000pt}%
\definecolor{currentstroke}{rgb}{0.850000,0.850000,0.850000}%
\pgfsetstrokecolor{currentstroke}%
\pgfsetdash{}{0pt}%
\pgfpathmoveto{\pgfqpoint{2.072891in}{0.417642in}}%
\pgfpathlineto{\pgfqpoint{2.072891in}{2.472642in}}%
\pgfusepath{stroke}%
\end{pgfscope}%
\begin{pgfscope}%
\pgfsetbuttcap%
\pgfsetroundjoin%
\definecolor{currentfill}{rgb}{0.000000,0.000000,0.000000}%
\pgfsetfillcolor{currentfill}%
\pgfsetlinewidth{0.602250pt}%
\definecolor{currentstroke}{rgb}{0.000000,0.000000,0.000000}%
\pgfsetstrokecolor{currentstroke}%
\pgfsetdash{}{0pt}%
\pgfsys@defobject{currentmarker}{\pgfqpoint{0.000000in}{-0.027778in}}{\pgfqpoint{0.000000in}{0.000000in}}{%
\pgfpathmoveto{\pgfqpoint{0.000000in}{0.000000in}}%
\pgfpathlineto{\pgfqpoint{0.000000in}{-0.027778in}}%
\pgfusepath{stroke,fill}%
}%
\begin{pgfscope}%
\pgfsys@transformshift{2.072891in}{0.417642in}%
\pgfsys@useobject{currentmarker}{}%
\end{pgfscope}%
\end{pgfscope}%
\begin{pgfscope}%
\pgfpathrectangle{\pgfqpoint{0.589510in}{0.417642in}}{\pgfqpoint{3.437062in}{2.055000in}}%
\pgfusepath{clip}%
\pgfsetrectcap%
\pgfsetroundjoin%
\pgfsetlinewidth{0.803000pt}%
\definecolor{currentstroke}{rgb}{0.850000,0.850000,0.850000}%
\pgfsetstrokecolor{currentstroke}%
\pgfsetdash{}{0pt}%
\pgfpathmoveto{\pgfqpoint{2.134743in}{0.417642in}}%
\pgfpathlineto{\pgfqpoint{2.134743in}{2.472642in}}%
\pgfusepath{stroke}%
\end{pgfscope}%
\begin{pgfscope}%
\pgfsetbuttcap%
\pgfsetroundjoin%
\definecolor{currentfill}{rgb}{0.000000,0.000000,0.000000}%
\pgfsetfillcolor{currentfill}%
\pgfsetlinewidth{0.602250pt}%
\definecolor{currentstroke}{rgb}{0.000000,0.000000,0.000000}%
\pgfsetstrokecolor{currentstroke}%
\pgfsetdash{}{0pt}%
\pgfsys@defobject{currentmarker}{\pgfqpoint{0.000000in}{-0.027778in}}{\pgfqpoint{0.000000in}{0.000000in}}{%
\pgfpathmoveto{\pgfqpoint{0.000000in}{0.000000in}}%
\pgfpathlineto{\pgfqpoint{0.000000in}{-0.027778in}}%
\pgfusepath{stroke,fill}%
}%
\begin{pgfscope}%
\pgfsys@transformshift{2.134743in}{0.417642in}%
\pgfsys@useobject{currentmarker}{}%
\end{pgfscope}%
\end{pgfscope}%
\begin{pgfscope}%
\pgfpathrectangle{\pgfqpoint{0.589510in}{0.417642in}}{\pgfqpoint{3.437062in}{2.055000in}}%
\pgfusepath{clip}%
\pgfsetrectcap%
\pgfsetroundjoin%
\pgfsetlinewidth{0.803000pt}%
\definecolor{currentstroke}{rgb}{0.850000,0.850000,0.850000}%
\pgfsetstrokecolor{currentstroke}%
\pgfsetdash{}{0pt}%
\pgfpathmoveto{\pgfqpoint{2.187039in}{0.417642in}}%
\pgfpathlineto{\pgfqpoint{2.187039in}{2.472642in}}%
\pgfusepath{stroke}%
\end{pgfscope}%
\begin{pgfscope}%
\pgfsetbuttcap%
\pgfsetroundjoin%
\definecolor{currentfill}{rgb}{0.000000,0.000000,0.000000}%
\pgfsetfillcolor{currentfill}%
\pgfsetlinewidth{0.602250pt}%
\definecolor{currentstroke}{rgb}{0.000000,0.000000,0.000000}%
\pgfsetstrokecolor{currentstroke}%
\pgfsetdash{}{0pt}%
\pgfsys@defobject{currentmarker}{\pgfqpoint{0.000000in}{-0.027778in}}{\pgfqpoint{0.000000in}{0.000000in}}{%
\pgfpathmoveto{\pgfqpoint{0.000000in}{0.000000in}}%
\pgfpathlineto{\pgfqpoint{0.000000in}{-0.027778in}}%
\pgfusepath{stroke,fill}%
}%
\begin{pgfscope}%
\pgfsys@transformshift{2.187039in}{0.417642in}%
\pgfsys@useobject{currentmarker}{}%
\end{pgfscope}%
\end{pgfscope}%
\begin{pgfscope}%
\pgfpathrectangle{\pgfqpoint{0.589510in}{0.417642in}}{\pgfqpoint{3.437062in}{2.055000in}}%
\pgfusepath{clip}%
\pgfsetrectcap%
\pgfsetroundjoin%
\pgfsetlinewidth{0.803000pt}%
\definecolor{currentstroke}{rgb}{0.850000,0.850000,0.850000}%
\pgfsetstrokecolor{currentstroke}%
\pgfsetdash{}{0pt}%
\pgfpathmoveto{\pgfqpoint{2.232339in}{0.417642in}}%
\pgfpathlineto{\pgfqpoint{2.232339in}{2.472642in}}%
\pgfusepath{stroke}%
\end{pgfscope}%
\begin{pgfscope}%
\pgfsetbuttcap%
\pgfsetroundjoin%
\definecolor{currentfill}{rgb}{0.000000,0.000000,0.000000}%
\pgfsetfillcolor{currentfill}%
\pgfsetlinewidth{0.602250pt}%
\definecolor{currentstroke}{rgb}{0.000000,0.000000,0.000000}%
\pgfsetstrokecolor{currentstroke}%
\pgfsetdash{}{0pt}%
\pgfsys@defobject{currentmarker}{\pgfqpoint{0.000000in}{-0.027778in}}{\pgfqpoint{0.000000in}{0.000000in}}{%
\pgfpathmoveto{\pgfqpoint{0.000000in}{0.000000in}}%
\pgfpathlineto{\pgfqpoint{0.000000in}{-0.027778in}}%
\pgfusepath{stroke,fill}%
}%
\begin{pgfscope}%
\pgfsys@transformshift{2.232339in}{0.417642in}%
\pgfsys@useobject{currentmarker}{}%
\end{pgfscope}%
\end{pgfscope}%
\begin{pgfscope}%
\pgfpathrectangle{\pgfqpoint{0.589510in}{0.417642in}}{\pgfqpoint{3.437062in}{2.055000in}}%
\pgfusepath{clip}%
\pgfsetrectcap%
\pgfsetroundjoin%
\pgfsetlinewidth{0.803000pt}%
\definecolor{currentstroke}{rgb}{0.850000,0.850000,0.850000}%
\pgfsetstrokecolor{currentstroke}%
\pgfsetdash{}{0pt}%
\pgfpathmoveto{\pgfqpoint{2.272297in}{0.417642in}}%
\pgfpathlineto{\pgfqpoint{2.272297in}{2.472642in}}%
\pgfusepath{stroke}%
\end{pgfscope}%
\begin{pgfscope}%
\pgfsetbuttcap%
\pgfsetroundjoin%
\definecolor{currentfill}{rgb}{0.000000,0.000000,0.000000}%
\pgfsetfillcolor{currentfill}%
\pgfsetlinewidth{0.602250pt}%
\definecolor{currentstroke}{rgb}{0.000000,0.000000,0.000000}%
\pgfsetstrokecolor{currentstroke}%
\pgfsetdash{}{0pt}%
\pgfsys@defobject{currentmarker}{\pgfqpoint{0.000000in}{-0.027778in}}{\pgfqpoint{0.000000in}{0.000000in}}{%
\pgfpathmoveto{\pgfqpoint{0.000000in}{0.000000in}}%
\pgfpathlineto{\pgfqpoint{0.000000in}{-0.027778in}}%
\pgfusepath{stroke,fill}%
}%
\begin{pgfscope}%
\pgfsys@transformshift{2.272297in}{0.417642in}%
\pgfsys@useobject{currentmarker}{}%
\end{pgfscope}%
\end{pgfscope}%
\begin{pgfscope}%
\pgfpathrectangle{\pgfqpoint{0.589510in}{0.417642in}}{\pgfqpoint{3.437062in}{2.055000in}}%
\pgfusepath{clip}%
\pgfsetrectcap%
\pgfsetroundjoin%
\pgfsetlinewidth{0.803000pt}%
\definecolor{currentstroke}{rgb}{0.850000,0.850000,0.850000}%
\pgfsetstrokecolor{currentstroke}%
\pgfsetdash{}{0pt}%
\pgfpathmoveto{\pgfqpoint{2.543190in}{0.417642in}}%
\pgfpathlineto{\pgfqpoint{2.543190in}{2.472642in}}%
\pgfusepath{stroke}%
\end{pgfscope}%
\begin{pgfscope}%
\pgfsetbuttcap%
\pgfsetroundjoin%
\definecolor{currentfill}{rgb}{0.000000,0.000000,0.000000}%
\pgfsetfillcolor{currentfill}%
\pgfsetlinewidth{0.602250pt}%
\definecolor{currentstroke}{rgb}{0.000000,0.000000,0.000000}%
\pgfsetstrokecolor{currentstroke}%
\pgfsetdash{}{0pt}%
\pgfsys@defobject{currentmarker}{\pgfqpoint{0.000000in}{-0.027778in}}{\pgfqpoint{0.000000in}{0.000000in}}{%
\pgfpathmoveto{\pgfqpoint{0.000000in}{0.000000in}}%
\pgfpathlineto{\pgfqpoint{0.000000in}{-0.027778in}}%
\pgfusepath{stroke,fill}%
}%
\begin{pgfscope}%
\pgfsys@transformshift{2.543190in}{0.417642in}%
\pgfsys@useobject{currentmarker}{}%
\end{pgfscope}%
\end{pgfscope}%
\begin{pgfscope}%
\pgfpathrectangle{\pgfqpoint{0.589510in}{0.417642in}}{\pgfqpoint{3.437062in}{2.055000in}}%
\pgfusepath{clip}%
\pgfsetrectcap%
\pgfsetroundjoin%
\pgfsetlinewidth{0.803000pt}%
\definecolor{currentstroke}{rgb}{0.850000,0.850000,0.850000}%
\pgfsetstrokecolor{currentstroke}%
\pgfsetdash{}{0pt}%
\pgfpathmoveto{\pgfqpoint{2.680744in}{0.417642in}}%
\pgfpathlineto{\pgfqpoint{2.680744in}{2.472642in}}%
\pgfusepath{stroke}%
\end{pgfscope}%
\begin{pgfscope}%
\pgfsetbuttcap%
\pgfsetroundjoin%
\definecolor{currentfill}{rgb}{0.000000,0.000000,0.000000}%
\pgfsetfillcolor{currentfill}%
\pgfsetlinewidth{0.602250pt}%
\definecolor{currentstroke}{rgb}{0.000000,0.000000,0.000000}%
\pgfsetstrokecolor{currentstroke}%
\pgfsetdash{}{0pt}%
\pgfsys@defobject{currentmarker}{\pgfqpoint{0.000000in}{-0.027778in}}{\pgfqpoint{0.000000in}{0.000000in}}{%
\pgfpathmoveto{\pgfqpoint{0.000000in}{0.000000in}}%
\pgfpathlineto{\pgfqpoint{0.000000in}{-0.027778in}}%
\pgfusepath{stroke,fill}%
}%
\begin{pgfscope}%
\pgfsys@transformshift{2.680744in}{0.417642in}%
\pgfsys@useobject{currentmarker}{}%
\end{pgfscope}%
\end{pgfscope}%
\begin{pgfscope}%
\pgfpathrectangle{\pgfqpoint{0.589510in}{0.417642in}}{\pgfqpoint{3.437062in}{2.055000in}}%
\pgfusepath{clip}%
\pgfsetrectcap%
\pgfsetroundjoin%
\pgfsetlinewidth{0.803000pt}%
\definecolor{currentstroke}{rgb}{0.850000,0.850000,0.850000}%
\pgfsetstrokecolor{currentstroke}%
\pgfsetdash{}{0pt}%
\pgfpathmoveto{\pgfqpoint{2.778340in}{0.417642in}}%
\pgfpathlineto{\pgfqpoint{2.778340in}{2.472642in}}%
\pgfusepath{stroke}%
\end{pgfscope}%
\begin{pgfscope}%
\pgfsetbuttcap%
\pgfsetroundjoin%
\definecolor{currentfill}{rgb}{0.000000,0.000000,0.000000}%
\pgfsetfillcolor{currentfill}%
\pgfsetlinewidth{0.602250pt}%
\definecolor{currentstroke}{rgb}{0.000000,0.000000,0.000000}%
\pgfsetstrokecolor{currentstroke}%
\pgfsetdash{}{0pt}%
\pgfsys@defobject{currentmarker}{\pgfqpoint{0.000000in}{-0.027778in}}{\pgfqpoint{0.000000in}{0.000000in}}{%
\pgfpathmoveto{\pgfqpoint{0.000000in}{0.000000in}}%
\pgfpathlineto{\pgfqpoint{0.000000in}{-0.027778in}}%
\pgfusepath{stroke,fill}%
}%
\begin{pgfscope}%
\pgfsys@transformshift{2.778340in}{0.417642in}%
\pgfsys@useobject{currentmarker}{}%
\end{pgfscope}%
\end{pgfscope}%
\begin{pgfscope}%
\pgfpathrectangle{\pgfqpoint{0.589510in}{0.417642in}}{\pgfqpoint{3.437062in}{2.055000in}}%
\pgfusepath{clip}%
\pgfsetrectcap%
\pgfsetroundjoin%
\pgfsetlinewidth{0.803000pt}%
\definecolor{currentstroke}{rgb}{0.850000,0.850000,0.850000}%
\pgfsetstrokecolor{currentstroke}%
\pgfsetdash{}{0pt}%
\pgfpathmoveto{\pgfqpoint{2.854041in}{0.417642in}}%
\pgfpathlineto{\pgfqpoint{2.854041in}{2.472642in}}%
\pgfusepath{stroke}%
\end{pgfscope}%
\begin{pgfscope}%
\pgfsetbuttcap%
\pgfsetroundjoin%
\definecolor{currentfill}{rgb}{0.000000,0.000000,0.000000}%
\pgfsetfillcolor{currentfill}%
\pgfsetlinewidth{0.602250pt}%
\definecolor{currentstroke}{rgb}{0.000000,0.000000,0.000000}%
\pgfsetstrokecolor{currentstroke}%
\pgfsetdash{}{0pt}%
\pgfsys@defobject{currentmarker}{\pgfqpoint{0.000000in}{-0.027778in}}{\pgfqpoint{0.000000in}{0.000000in}}{%
\pgfpathmoveto{\pgfqpoint{0.000000in}{0.000000in}}%
\pgfpathlineto{\pgfqpoint{0.000000in}{-0.027778in}}%
\pgfusepath{stroke,fill}%
}%
\begin{pgfscope}%
\pgfsys@transformshift{2.854041in}{0.417642in}%
\pgfsys@useobject{currentmarker}{}%
\end{pgfscope}%
\end{pgfscope}%
\begin{pgfscope}%
\pgfpathrectangle{\pgfqpoint{0.589510in}{0.417642in}}{\pgfqpoint{3.437062in}{2.055000in}}%
\pgfusepath{clip}%
\pgfsetrectcap%
\pgfsetroundjoin%
\pgfsetlinewidth{0.803000pt}%
\definecolor{currentstroke}{rgb}{0.850000,0.850000,0.850000}%
\pgfsetstrokecolor{currentstroke}%
\pgfsetdash{}{0pt}%
\pgfpathmoveto{\pgfqpoint{2.915894in}{0.417642in}}%
\pgfpathlineto{\pgfqpoint{2.915894in}{2.472642in}}%
\pgfusepath{stroke}%
\end{pgfscope}%
\begin{pgfscope}%
\pgfsetbuttcap%
\pgfsetroundjoin%
\definecolor{currentfill}{rgb}{0.000000,0.000000,0.000000}%
\pgfsetfillcolor{currentfill}%
\pgfsetlinewidth{0.602250pt}%
\definecolor{currentstroke}{rgb}{0.000000,0.000000,0.000000}%
\pgfsetstrokecolor{currentstroke}%
\pgfsetdash{}{0pt}%
\pgfsys@defobject{currentmarker}{\pgfqpoint{0.000000in}{-0.027778in}}{\pgfqpoint{0.000000in}{0.000000in}}{%
\pgfpathmoveto{\pgfqpoint{0.000000in}{0.000000in}}%
\pgfpathlineto{\pgfqpoint{0.000000in}{-0.027778in}}%
\pgfusepath{stroke,fill}%
}%
\begin{pgfscope}%
\pgfsys@transformshift{2.915894in}{0.417642in}%
\pgfsys@useobject{currentmarker}{}%
\end{pgfscope}%
\end{pgfscope}%
\begin{pgfscope}%
\pgfpathrectangle{\pgfqpoint{0.589510in}{0.417642in}}{\pgfqpoint{3.437062in}{2.055000in}}%
\pgfusepath{clip}%
\pgfsetrectcap%
\pgfsetroundjoin%
\pgfsetlinewidth{0.803000pt}%
\definecolor{currentstroke}{rgb}{0.850000,0.850000,0.850000}%
\pgfsetstrokecolor{currentstroke}%
\pgfsetdash{}{0pt}%
\pgfpathmoveto{\pgfqpoint{2.968189in}{0.417642in}}%
\pgfpathlineto{\pgfqpoint{2.968189in}{2.472642in}}%
\pgfusepath{stroke}%
\end{pgfscope}%
\begin{pgfscope}%
\pgfsetbuttcap%
\pgfsetroundjoin%
\definecolor{currentfill}{rgb}{0.000000,0.000000,0.000000}%
\pgfsetfillcolor{currentfill}%
\pgfsetlinewidth{0.602250pt}%
\definecolor{currentstroke}{rgb}{0.000000,0.000000,0.000000}%
\pgfsetstrokecolor{currentstroke}%
\pgfsetdash{}{0pt}%
\pgfsys@defobject{currentmarker}{\pgfqpoint{0.000000in}{-0.027778in}}{\pgfqpoint{0.000000in}{0.000000in}}{%
\pgfpathmoveto{\pgfqpoint{0.000000in}{0.000000in}}%
\pgfpathlineto{\pgfqpoint{0.000000in}{-0.027778in}}%
\pgfusepath{stroke,fill}%
}%
\begin{pgfscope}%
\pgfsys@transformshift{2.968189in}{0.417642in}%
\pgfsys@useobject{currentmarker}{}%
\end{pgfscope}%
\end{pgfscope}%
\begin{pgfscope}%
\pgfpathrectangle{\pgfqpoint{0.589510in}{0.417642in}}{\pgfqpoint{3.437062in}{2.055000in}}%
\pgfusepath{clip}%
\pgfsetrectcap%
\pgfsetroundjoin%
\pgfsetlinewidth{0.803000pt}%
\definecolor{currentstroke}{rgb}{0.850000,0.850000,0.850000}%
\pgfsetstrokecolor{currentstroke}%
\pgfsetdash{}{0pt}%
\pgfpathmoveto{\pgfqpoint{3.013490in}{0.417642in}}%
\pgfpathlineto{\pgfqpoint{3.013490in}{2.472642in}}%
\pgfusepath{stroke}%
\end{pgfscope}%
\begin{pgfscope}%
\pgfsetbuttcap%
\pgfsetroundjoin%
\definecolor{currentfill}{rgb}{0.000000,0.000000,0.000000}%
\pgfsetfillcolor{currentfill}%
\pgfsetlinewidth{0.602250pt}%
\definecolor{currentstroke}{rgb}{0.000000,0.000000,0.000000}%
\pgfsetstrokecolor{currentstroke}%
\pgfsetdash{}{0pt}%
\pgfsys@defobject{currentmarker}{\pgfqpoint{0.000000in}{-0.027778in}}{\pgfqpoint{0.000000in}{0.000000in}}{%
\pgfpathmoveto{\pgfqpoint{0.000000in}{0.000000in}}%
\pgfpathlineto{\pgfqpoint{0.000000in}{-0.027778in}}%
\pgfusepath{stroke,fill}%
}%
\begin{pgfscope}%
\pgfsys@transformshift{3.013490in}{0.417642in}%
\pgfsys@useobject{currentmarker}{}%
\end{pgfscope}%
\end{pgfscope}%
\begin{pgfscope}%
\pgfpathrectangle{\pgfqpoint{0.589510in}{0.417642in}}{\pgfqpoint{3.437062in}{2.055000in}}%
\pgfusepath{clip}%
\pgfsetrectcap%
\pgfsetroundjoin%
\pgfsetlinewidth{0.803000pt}%
\definecolor{currentstroke}{rgb}{0.850000,0.850000,0.850000}%
\pgfsetstrokecolor{currentstroke}%
\pgfsetdash{}{0pt}%
\pgfpathmoveto{\pgfqpoint{3.053448in}{0.417642in}}%
\pgfpathlineto{\pgfqpoint{3.053448in}{2.472642in}}%
\pgfusepath{stroke}%
\end{pgfscope}%
\begin{pgfscope}%
\pgfsetbuttcap%
\pgfsetroundjoin%
\definecolor{currentfill}{rgb}{0.000000,0.000000,0.000000}%
\pgfsetfillcolor{currentfill}%
\pgfsetlinewidth{0.602250pt}%
\definecolor{currentstroke}{rgb}{0.000000,0.000000,0.000000}%
\pgfsetstrokecolor{currentstroke}%
\pgfsetdash{}{0pt}%
\pgfsys@defobject{currentmarker}{\pgfqpoint{0.000000in}{-0.027778in}}{\pgfqpoint{0.000000in}{0.000000in}}{%
\pgfpathmoveto{\pgfqpoint{0.000000in}{0.000000in}}%
\pgfpathlineto{\pgfqpoint{0.000000in}{-0.027778in}}%
\pgfusepath{stroke,fill}%
}%
\begin{pgfscope}%
\pgfsys@transformshift{3.053448in}{0.417642in}%
\pgfsys@useobject{currentmarker}{}%
\end{pgfscope}%
\end{pgfscope}%
\begin{pgfscope}%
\pgfpathrectangle{\pgfqpoint{0.589510in}{0.417642in}}{\pgfqpoint{3.437062in}{2.055000in}}%
\pgfusepath{clip}%
\pgfsetrectcap%
\pgfsetroundjoin%
\pgfsetlinewidth{0.803000pt}%
\definecolor{currentstroke}{rgb}{0.850000,0.850000,0.850000}%
\pgfsetstrokecolor{currentstroke}%
\pgfsetdash{}{0pt}%
\pgfpathmoveto{\pgfqpoint{3.324341in}{0.417642in}}%
\pgfpathlineto{\pgfqpoint{3.324341in}{2.472642in}}%
\pgfusepath{stroke}%
\end{pgfscope}%
\begin{pgfscope}%
\pgfsetbuttcap%
\pgfsetroundjoin%
\definecolor{currentfill}{rgb}{0.000000,0.000000,0.000000}%
\pgfsetfillcolor{currentfill}%
\pgfsetlinewidth{0.602250pt}%
\definecolor{currentstroke}{rgb}{0.000000,0.000000,0.000000}%
\pgfsetstrokecolor{currentstroke}%
\pgfsetdash{}{0pt}%
\pgfsys@defobject{currentmarker}{\pgfqpoint{0.000000in}{-0.027778in}}{\pgfqpoint{0.000000in}{0.000000in}}{%
\pgfpathmoveto{\pgfqpoint{0.000000in}{0.000000in}}%
\pgfpathlineto{\pgfqpoint{0.000000in}{-0.027778in}}%
\pgfusepath{stroke,fill}%
}%
\begin{pgfscope}%
\pgfsys@transformshift{3.324341in}{0.417642in}%
\pgfsys@useobject{currentmarker}{}%
\end{pgfscope}%
\end{pgfscope}%
\begin{pgfscope}%
\pgfpathrectangle{\pgfqpoint{0.589510in}{0.417642in}}{\pgfqpoint{3.437062in}{2.055000in}}%
\pgfusepath{clip}%
\pgfsetrectcap%
\pgfsetroundjoin%
\pgfsetlinewidth{0.803000pt}%
\definecolor{currentstroke}{rgb}{0.850000,0.850000,0.850000}%
\pgfsetstrokecolor{currentstroke}%
\pgfsetdash{}{0pt}%
\pgfpathmoveto{\pgfqpoint{3.461895in}{0.417642in}}%
\pgfpathlineto{\pgfqpoint{3.461895in}{2.472642in}}%
\pgfusepath{stroke}%
\end{pgfscope}%
\begin{pgfscope}%
\pgfsetbuttcap%
\pgfsetroundjoin%
\definecolor{currentfill}{rgb}{0.000000,0.000000,0.000000}%
\pgfsetfillcolor{currentfill}%
\pgfsetlinewidth{0.602250pt}%
\definecolor{currentstroke}{rgb}{0.000000,0.000000,0.000000}%
\pgfsetstrokecolor{currentstroke}%
\pgfsetdash{}{0pt}%
\pgfsys@defobject{currentmarker}{\pgfqpoint{0.000000in}{-0.027778in}}{\pgfqpoint{0.000000in}{0.000000in}}{%
\pgfpathmoveto{\pgfqpoint{0.000000in}{0.000000in}}%
\pgfpathlineto{\pgfqpoint{0.000000in}{-0.027778in}}%
\pgfusepath{stroke,fill}%
}%
\begin{pgfscope}%
\pgfsys@transformshift{3.461895in}{0.417642in}%
\pgfsys@useobject{currentmarker}{}%
\end{pgfscope}%
\end{pgfscope}%
\begin{pgfscope}%
\pgfpathrectangle{\pgfqpoint{0.589510in}{0.417642in}}{\pgfqpoint{3.437062in}{2.055000in}}%
\pgfusepath{clip}%
\pgfsetrectcap%
\pgfsetroundjoin%
\pgfsetlinewidth{0.803000pt}%
\definecolor{currentstroke}{rgb}{0.850000,0.850000,0.850000}%
\pgfsetstrokecolor{currentstroke}%
\pgfsetdash{}{0pt}%
\pgfpathmoveto{\pgfqpoint{3.559491in}{0.417642in}}%
\pgfpathlineto{\pgfqpoint{3.559491in}{2.472642in}}%
\pgfusepath{stroke}%
\end{pgfscope}%
\begin{pgfscope}%
\pgfsetbuttcap%
\pgfsetroundjoin%
\definecolor{currentfill}{rgb}{0.000000,0.000000,0.000000}%
\pgfsetfillcolor{currentfill}%
\pgfsetlinewidth{0.602250pt}%
\definecolor{currentstroke}{rgb}{0.000000,0.000000,0.000000}%
\pgfsetstrokecolor{currentstroke}%
\pgfsetdash{}{0pt}%
\pgfsys@defobject{currentmarker}{\pgfqpoint{0.000000in}{-0.027778in}}{\pgfqpoint{0.000000in}{0.000000in}}{%
\pgfpathmoveto{\pgfqpoint{0.000000in}{0.000000in}}%
\pgfpathlineto{\pgfqpoint{0.000000in}{-0.027778in}}%
\pgfusepath{stroke,fill}%
}%
\begin{pgfscope}%
\pgfsys@transformshift{3.559491in}{0.417642in}%
\pgfsys@useobject{currentmarker}{}%
\end{pgfscope}%
\end{pgfscope}%
\begin{pgfscope}%
\pgfpathrectangle{\pgfqpoint{0.589510in}{0.417642in}}{\pgfqpoint{3.437062in}{2.055000in}}%
\pgfusepath{clip}%
\pgfsetrectcap%
\pgfsetroundjoin%
\pgfsetlinewidth{0.803000pt}%
\definecolor{currentstroke}{rgb}{0.850000,0.850000,0.850000}%
\pgfsetstrokecolor{currentstroke}%
\pgfsetdash{}{0pt}%
\pgfpathmoveto{\pgfqpoint{3.635192in}{0.417642in}}%
\pgfpathlineto{\pgfqpoint{3.635192in}{2.472642in}}%
\pgfusepath{stroke}%
\end{pgfscope}%
\begin{pgfscope}%
\pgfsetbuttcap%
\pgfsetroundjoin%
\definecolor{currentfill}{rgb}{0.000000,0.000000,0.000000}%
\pgfsetfillcolor{currentfill}%
\pgfsetlinewidth{0.602250pt}%
\definecolor{currentstroke}{rgb}{0.000000,0.000000,0.000000}%
\pgfsetstrokecolor{currentstroke}%
\pgfsetdash{}{0pt}%
\pgfsys@defobject{currentmarker}{\pgfqpoint{0.000000in}{-0.027778in}}{\pgfqpoint{0.000000in}{0.000000in}}{%
\pgfpathmoveto{\pgfqpoint{0.000000in}{0.000000in}}%
\pgfpathlineto{\pgfqpoint{0.000000in}{-0.027778in}}%
\pgfusepath{stroke,fill}%
}%
\begin{pgfscope}%
\pgfsys@transformshift{3.635192in}{0.417642in}%
\pgfsys@useobject{currentmarker}{}%
\end{pgfscope}%
\end{pgfscope}%
\begin{pgfscope}%
\pgfpathrectangle{\pgfqpoint{0.589510in}{0.417642in}}{\pgfqpoint{3.437062in}{2.055000in}}%
\pgfusepath{clip}%
\pgfsetrectcap%
\pgfsetroundjoin%
\pgfsetlinewidth{0.803000pt}%
\definecolor{currentstroke}{rgb}{0.850000,0.850000,0.850000}%
\pgfsetstrokecolor{currentstroke}%
\pgfsetdash{}{0pt}%
\pgfpathmoveto{\pgfqpoint{3.697044in}{0.417642in}}%
\pgfpathlineto{\pgfqpoint{3.697044in}{2.472642in}}%
\pgfusepath{stroke}%
\end{pgfscope}%
\begin{pgfscope}%
\pgfsetbuttcap%
\pgfsetroundjoin%
\definecolor{currentfill}{rgb}{0.000000,0.000000,0.000000}%
\pgfsetfillcolor{currentfill}%
\pgfsetlinewidth{0.602250pt}%
\definecolor{currentstroke}{rgb}{0.000000,0.000000,0.000000}%
\pgfsetstrokecolor{currentstroke}%
\pgfsetdash{}{0pt}%
\pgfsys@defobject{currentmarker}{\pgfqpoint{0.000000in}{-0.027778in}}{\pgfqpoint{0.000000in}{0.000000in}}{%
\pgfpathmoveto{\pgfqpoint{0.000000in}{0.000000in}}%
\pgfpathlineto{\pgfqpoint{0.000000in}{-0.027778in}}%
\pgfusepath{stroke,fill}%
}%
\begin{pgfscope}%
\pgfsys@transformshift{3.697044in}{0.417642in}%
\pgfsys@useobject{currentmarker}{}%
\end{pgfscope}%
\end{pgfscope}%
\begin{pgfscope}%
\pgfpathrectangle{\pgfqpoint{0.589510in}{0.417642in}}{\pgfqpoint{3.437062in}{2.055000in}}%
\pgfusepath{clip}%
\pgfsetrectcap%
\pgfsetroundjoin%
\pgfsetlinewidth{0.803000pt}%
\definecolor{currentstroke}{rgb}{0.850000,0.850000,0.850000}%
\pgfsetstrokecolor{currentstroke}%
\pgfsetdash{}{0pt}%
\pgfpathmoveto{\pgfqpoint{3.749340in}{0.417642in}}%
\pgfpathlineto{\pgfqpoint{3.749340in}{2.472642in}}%
\pgfusepath{stroke}%
\end{pgfscope}%
\begin{pgfscope}%
\pgfsetbuttcap%
\pgfsetroundjoin%
\definecolor{currentfill}{rgb}{0.000000,0.000000,0.000000}%
\pgfsetfillcolor{currentfill}%
\pgfsetlinewidth{0.602250pt}%
\definecolor{currentstroke}{rgb}{0.000000,0.000000,0.000000}%
\pgfsetstrokecolor{currentstroke}%
\pgfsetdash{}{0pt}%
\pgfsys@defobject{currentmarker}{\pgfqpoint{0.000000in}{-0.027778in}}{\pgfqpoint{0.000000in}{0.000000in}}{%
\pgfpathmoveto{\pgfqpoint{0.000000in}{0.000000in}}%
\pgfpathlineto{\pgfqpoint{0.000000in}{-0.027778in}}%
\pgfusepath{stroke,fill}%
}%
\begin{pgfscope}%
\pgfsys@transformshift{3.749340in}{0.417642in}%
\pgfsys@useobject{currentmarker}{}%
\end{pgfscope}%
\end{pgfscope}%
\begin{pgfscope}%
\pgfpathrectangle{\pgfqpoint{0.589510in}{0.417642in}}{\pgfqpoint{3.437062in}{2.055000in}}%
\pgfusepath{clip}%
\pgfsetrectcap%
\pgfsetroundjoin%
\pgfsetlinewidth{0.803000pt}%
\definecolor{currentstroke}{rgb}{0.850000,0.850000,0.850000}%
\pgfsetstrokecolor{currentstroke}%
\pgfsetdash{}{0pt}%
\pgfpathmoveto{\pgfqpoint{3.794640in}{0.417642in}}%
\pgfpathlineto{\pgfqpoint{3.794640in}{2.472642in}}%
\pgfusepath{stroke}%
\end{pgfscope}%
\begin{pgfscope}%
\pgfsetbuttcap%
\pgfsetroundjoin%
\definecolor{currentfill}{rgb}{0.000000,0.000000,0.000000}%
\pgfsetfillcolor{currentfill}%
\pgfsetlinewidth{0.602250pt}%
\definecolor{currentstroke}{rgb}{0.000000,0.000000,0.000000}%
\pgfsetstrokecolor{currentstroke}%
\pgfsetdash{}{0pt}%
\pgfsys@defobject{currentmarker}{\pgfqpoint{0.000000in}{-0.027778in}}{\pgfqpoint{0.000000in}{0.000000in}}{%
\pgfpathmoveto{\pgfqpoint{0.000000in}{0.000000in}}%
\pgfpathlineto{\pgfqpoint{0.000000in}{-0.027778in}}%
\pgfusepath{stroke,fill}%
}%
\begin{pgfscope}%
\pgfsys@transformshift{3.794640in}{0.417642in}%
\pgfsys@useobject{currentmarker}{}%
\end{pgfscope}%
\end{pgfscope}%
\begin{pgfscope}%
\pgfpathrectangle{\pgfqpoint{0.589510in}{0.417642in}}{\pgfqpoint{3.437062in}{2.055000in}}%
\pgfusepath{clip}%
\pgfsetrectcap%
\pgfsetroundjoin%
\pgfsetlinewidth{0.803000pt}%
\definecolor{currentstroke}{rgb}{0.850000,0.850000,0.850000}%
\pgfsetstrokecolor{currentstroke}%
\pgfsetdash{}{0pt}%
\pgfpathmoveto{\pgfqpoint{3.834598in}{0.417642in}}%
\pgfpathlineto{\pgfqpoint{3.834598in}{2.472642in}}%
\pgfusepath{stroke}%
\end{pgfscope}%
\begin{pgfscope}%
\pgfsetbuttcap%
\pgfsetroundjoin%
\definecolor{currentfill}{rgb}{0.000000,0.000000,0.000000}%
\pgfsetfillcolor{currentfill}%
\pgfsetlinewidth{0.602250pt}%
\definecolor{currentstroke}{rgb}{0.000000,0.000000,0.000000}%
\pgfsetstrokecolor{currentstroke}%
\pgfsetdash{}{0pt}%
\pgfsys@defobject{currentmarker}{\pgfqpoint{0.000000in}{-0.027778in}}{\pgfqpoint{0.000000in}{0.000000in}}{%
\pgfpathmoveto{\pgfqpoint{0.000000in}{0.000000in}}%
\pgfpathlineto{\pgfqpoint{0.000000in}{-0.027778in}}%
\pgfusepath{stroke,fill}%
}%
\begin{pgfscope}%
\pgfsys@transformshift{3.834598in}{0.417642in}%
\pgfsys@useobject{currentmarker}{}%
\end{pgfscope}%
\end{pgfscope}%
\begin{pgfscope}%
\definecolor{textcolor}{rgb}{0.000000,0.000000,0.000000}%
\pgfsetstrokecolor{textcolor}%
\pgfsetfillcolor{textcolor}%
\pgftext[x=2.308041in,y=0.165003in,,top]{\color{textcolor}{\rmfamily\fontsize{10.000000}{12.000000}\selectfont\catcode`\^=\active\def^{\ifmmode\sp\else\^{}\fi}\catcode`\%=\active\def%{\%}$\tau$ in \unit{\second}}}%
\end{pgfscope}%
\begin{pgfscope}%
\pgfpathrectangle{\pgfqpoint{0.589510in}{0.417642in}}{\pgfqpoint{3.437062in}{2.055000in}}%
\pgfusepath{clip}%
\pgfsetrectcap%
\pgfsetroundjoin%
\pgfsetlinewidth{0.803000pt}%
\definecolor{currentstroke}{rgb}{0.450000,0.450000,0.450000}%
\pgfsetstrokecolor{currentstroke}%
\pgfsetdash{}{0pt}%
\pgfpathmoveto{\pgfqpoint{0.589510in}{1.728662in}}%
\pgfpathlineto{\pgfqpoint{4.026572in}{1.728662in}}%
\pgfusepath{stroke}%
\end{pgfscope}%
\begin{pgfscope}%
\pgfsetbuttcap%
\pgfsetroundjoin%
\definecolor{currentfill}{rgb}{0.000000,0.000000,0.000000}%
\pgfsetfillcolor{currentfill}%
\pgfsetlinewidth{0.803000pt}%
\definecolor{currentstroke}{rgb}{0.000000,0.000000,0.000000}%
\pgfsetstrokecolor{currentstroke}%
\pgfsetdash{}{0pt}%
\pgfsys@defobject{currentmarker}{\pgfqpoint{-0.048611in}{0.000000in}}{\pgfqpoint{-0.000000in}{0.000000in}}{%
\pgfpathmoveto{\pgfqpoint{-0.000000in}{0.000000in}}%
\pgfpathlineto{\pgfqpoint{-0.048611in}{0.000000in}}%
\pgfusepath{stroke,fill}%
}%
\begin{pgfscope}%
\pgfsys@transformshift{0.589510in}{1.728662in}%
\pgfsys@useobject{currentmarker}{}%
\end{pgfscope}%
\end{pgfscope}%
\begin{pgfscope}%
\definecolor{textcolor}{rgb}{0.000000,0.000000,0.000000}%
\pgfsetstrokecolor{textcolor}%
\pgfsetfillcolor{textcolor}%
\pgftext[x=0.236114in, y=1.689509in, left, base]{\color{textcolor}{\rmfamily\fontsize{8.000000}{9.600000}\selectfont\catcode`\^=\active\def^{\ifmmode\sp\else\^{}\fi}\catcode`\%=\active\def%{\%}$\mathdefault{10^{-1}}$}}%
\end{pgfscope}%
\begin{pgfscope}%
\pgfpathrectangle{\pgfqpoint{0.589510in}{0.417642in}}{\pgfqpoint{3.437062in}{2.055000in}}%
\pgfusepath{clip}%
\pgfsetrectcap%
\pgfsetroundjoin%
\pgfsetlinewidth{0.803000pt}%
\definecolor{currentstroke}{rgb}{0.850000,0.850000,0.850000}%
\pgfsetstrokecolor{currentstroke}%
\pgfsetdash{}{0pt}%
\pgfpathmoveto{\pgfqpoint{0.589510in}{0.795335in}}%
\pgfpathlineto{\pgfqpoint{4.026572in}{0.795335in}}%
\pgfusepath{stroke}%
\end{pgfscope}%
\begin{pgfscope}%
\pgfsetbuttcap%
\pgfsetroundjoin%
\definecolor{currentfill}{rgb}{0.000000,0.000000,0.000000}%
\pgfsetfillcolor{currentfill}%
\pgfsetlinewidth{0.602250pt}%
\definecolor{currentstroke}{rgb}{0.000000,0.000000,0.000000}%
\pgfsetstrokecolor{currentstroke}%
\pgfsetdash{}{0pt}%
\pgfsys@defobject{currentmarker}{\pgfqpoint{-0.027778in}{0.000000in}}{\pgfqpoint{-0.000000in}{0.000000in}}{%
\pgfpathmoveto{\pgfqpoint{-0.000000in}{0.000000in}}%
\pgfpathlineto{\pgfqpoint{-0.027778in}{0.000000in}}%
\pgfusepath{stroke,fill}%
}%
\begin{pgfscope}%
\pgfsys@transformshift{0.589510in}{0.795335in}%
\pgfsys@useobject{currentmarker}{}%
\end{pgfscope}%
\end{pgfscope}%
\begin{pgfscope}%
\pgfpathrectangle{\pgfqpoint{0.589510in}{0.417642in}}{\pgfqpoint{3.437062in}{2.055000in}}%
\pgfusepath{clip}%
\pgfsetrectcap%
\pgfsetroundjoin%
\pgfsetlinewidth{0.803000pt}%
\definecolor{currentstroke}{rgb}{0.850000,0.850000,0.850000}%
\pgfsetstrokecolor{currentstroke}%
\pgfsetdash{}{0pt}%
\pgfpathmoveto{\pgfqpoint{0.589510in}{1.030467in}}%
\pgfpathlineto{\pgfqpoint{4.026572in}{1.030467in}}%
\pgfusepath{stroke}%
\end{pgfscope}%
\begin{pgfscope}%
\pgfsetbuttcap%
\pgfsetroundjoin%
\definecolor{currentfill}{rgb}{0.000000,0.000000,0.000000}%
\pgfsetfillcolor{currentfill}%
\pgfsetlinewidth{0.602250pt}%
\definecolor{currentstroke}{rgb}{0.000000,0.000000,0.000000}%
\pgfsetstrokecolor{currentstroke}%
\pgfsetdash{}{0pt}%
\pgfsys@defobject{currentmarker}{\pgfqpoint{-0.027778in}{0.000000in}}{\pgfqpoint{-0.000000in}{0.000000in}}{%
\pgfpathmoveto{\pgfqpoint{-0.000000in}{0.000000in}}%
\pgfpathlineto{\pgfqpoint{-0.027778in}{0.000000in}}%
\pgfusepath{stroke,fill}%
}%
\begin{pgfscope}%
\pgfsys@transformshift{0.589510in}{1.030467in}%
\pgfsys@useobject{currentmarker}{}%
\end{pgfscope}%
\end{pgfscope}%
\begin{pgfscope}%
\pgfpathrectangle{\pgfqpoint{0.589510in}{0.417642in}}{\pgfqpoint{3.437062in}{2.055000in}}%
\pgfusepath{clip}%
\pgfsetrectcap%
\pgfsetroundjoin%
\pgfsetlinewidth{0.803000pt}%
\definecolor{currentstroke}{rgb}{0.850000,0.850000,0.850000}%
\pgfsetstrokecolor{currentstroke}%
\pgfsetdash{}{0pt}%
\pgfpathmoveto{\pgfqpoint{0.589510in}{1.197297in}}%
\pgfpathlineto{\pgfqpoint{4.026572in}{1.197297in}}%
\pgfusepath{stroke}%
\end{pgfscope}%
\begin{pgfscope}%
\pgfsetbuttcap%
\pgfsetroundjoin%
\definecolor{currentfill}{rgb}{0.000000,0.000000,0.000000}%
\pgfsetfillcolor{currentfill}%
\pgfsetlinewidth{0.602250pt}%
\definecolor{currentstroke}{rgb}{0.000000,0.000000,0.000000}%
\pgfsetstrokecolor{currentstroke}%
\pgfsetdash{}{0pt}%
\pgfsys@defobject{currentmarker}{\pgfqpoint{-0.027778in}{0.000000in}}{\pgfqpoint{-0.000000in}{0.000000in}}{%
\pgfpathmoveto{\pgfqpoint{-0.000000in}{0.000000in}}%
\pgfpathlineto{\pgfqpoint{-0.027778in}{0.000000in}}%
\pgfusepath{stroke,fill}%
}%
\begin{pgfscope}%
\pgfsys@transformshift{0.589510in}{1.197297in}%
\pgfsys@useobject{currentmarker}{}%
\end{pgfscope}%
\end{pgfscope}%
\begin{pgfscope}%
\pgfpathrectangle{\pgfqpoint{0.589510in}{0.417642in}}{\pgfqpoint{3.437062in}{2.055000in}}%
\pgfusepath{clip}%
\pgfsetrectcap%
\pgfsetroundjoin%
\pgfsetlinewidth{0.803000pt}%
\definecolor{currentstroke}{rgb}{0.850000,0.850000,0.850000}%
\pgfsetstrokecolor{currentstroke}%
\pgfsetdash{}{0pt}%
\pgfpathmoveto{\pgfqpoint{0.589510in}{1.326699in}}%
\pgfpathlineto{\pgfqpoint{4.026572in}{1.326699in}}%
\pgfusepath{stroke}%
\end{pgfscope}%
\begin{pgfscope}%
\pgfsetbuttcap%
\pgfsetroundjoin%
\definecolor{currentfill}{rgb}{0.000000,0.000000,0.000000}%
\pgfsetfillcolor{currentfill}%
\pgfsetlinewidth{0.602250pt}%
\definecolor{currentstroke}{rgb}{0.000000,0.000000,0.000000}%
\pgfsetstrokecolor{currentstroke}%
\pgfsetdash{}{0pt}%
\pgfsys@defobject{currentmarker}{\pgfqpoint{-0.027778in}{0.000000in}}{\pgfqpoint{-0.000000in}{0.000000in}}{%
\pgfpathmoveto{\pgfqpoint{-0.000000in}{0.000000in}}%
\pgfpathlineto{\pgfqpoint{-0.027778in}{0.000000in}}%
\pgfusepath{stroke,fill}%
}%
\begin{pgfscope}%
\pgfsys@transformshift{0.589510in}{1.326699in}%
\pgfsys@useobject{currentmarker}{}%
\end{pgfscope}%
\end{pgfscope}%
\begin{pgfscope}%
\pgfpathrectangle{\pgfqpoint{0.589510in}{0.417642in}}{\pgfqpoint{3.437062in}{2.055000in}}%
\pgfusepath{clip}%
\pgfsetrectcap%
\pgfsetroundjoin%
\pgfsetlinewidth{0.803000pt}%
\definecolor{currentstroke}{rgb}{0.850000,0.850000,0.850000}%
\pgfsetstrokecolor{currentstroke}%
\pgfsetdash{}{0pt}%
\pgfpathmoveto{\pgfqpoint{0.589510in}{1.432429in}}%
\pgfpathlineto{\pgfqpoint{4.026572in}{1.432429in}}%
\pgfusepath{stroke}%
\end{pgfscope}%
\begin{pgfscope}%
\pgfsetbuttcap%
\pgfsetroundjoin%
\definecolor{currentfill}{rgb}{0.000000,0.000000,0.000000}%
\pgfsetfillcolor{currentfill}%
\pgfsetlinewidth{0.602250pt}%
\definecolor{currentstroke}{rgb}{0.000000,0.000000,0.000000}%
\pgfsetstrokecolor{currentstroke}%
\pgfsetdash{}{0pt}%
\pgfsys@defobject{currentmarker}{\pgfqpoint{-0.027778in}{0.000000in}}{\pgfqpoint{-0.000000in}{0.000000in}}{%
\pgfpathmoveto{\pgfqpoint{-0.000000in}{0.000000in}}%
\pgfpathlineto{\pgfqpoint{-0.027778in}{0.000000in}}%
\pgfusepath{stroke,fill}%
}%
\begin{pgfscope}%
\pgfsys@transformshift{0.589510in}{1.432429in}%
\pgfsys@useobject{currentmarker}{}%
\end{pgfscope}%
\end{pgfscope}%
\begin{pgfscope}%
\pgfpathrectangle{\pgfqpoint{0.589510in}{0.417642in}}{\pgfqpoint{3.437062in}{2.055000in}}%
\pgfusepath{clip}%
\pgfsetrectcap%
\pgfsetroundjoin%
\pgfsetlinewidth{0.803000pt}%
\definecolor{currentstroke}{rgb}{0.850000,0.850000,0.850000}%
\pgfsetstrokecolor{currentstroke}%
\pgfsetdash{}{0pt}%
\pgfpathmoveto{\pgfqpoint{0.589510in}{1.521823in}}%
\pgfpathlineto{\pgfqpoint{4.026572in}{1.521823in}}%
\pgfusepath{stroke}%
\end{pgfscope}%
\begin{pgfscope}%
\pgfsetbuttcap%
\pgfsetroundjoin%
\definecolor{currentfill}{rgb}{0.000000,0.000000,0.000000}%
\pgfsetfillcolor{currentfill}%
\pgfsetlinewidth{0.602250pt}%
\definecolor{currentstroke}{rgb}{0.000000,0.000000,0.000000}%
\pgfsetstrokecolor{currentstroke}%
\pgfsetdash{}{0pt}%
\pgfsys@defobject{currentmarker}{\pgfqpoint{-0.027778in}{0.000000in}}{\pgfqpoint{-0.000000in}{0.000000in}}{%
\pgfpathmoveto{\pgfqpoint{-0.000000in}{0.000000in}}%
\pgfpathlineto{\pgfqpoint{-0.027778in}{0.000000in}}%
\pgfusepath{stroke,fill}%
}%
\begin{pgfscope}%
\pgfsys@transformshift{0.589510in}{1.521823in}%
\pgfsys@useobject{currentmarker}{}%
\end{pgfscope}%
\end{pgfscope}%
\begin{pgfscope}%
\pgfpathrectangle{\pgfqpoint{0.589510in}{0.417642in}}{\pgfqpoint{3.437062in}{2.055000in}}%
\pgfusepath{clip}%
\pgfsetrectcap%
\pgfsetroundjoin%
\pgfsetlinewidth{0.803000pt}%
\definecolor{currentstroke}{rgb}{0.850000,0.850000,0.850000}%
\pgfsetstrokecolor{currentstroke}%
\pgfsetdash{}{0pt}%
\pgfpathmoveto{\pgfqpoint{0.589510in}{1.599259in}}%
\pgfpathlineto{\pgfqpoint{4.026572in}{1.599259in}}%
\pgfusepath{stroke}%
\end{pgfscope}%
\begin{pgfscope}%
\pgfsetbuttcap%
\pgfsetroundjoin%
\definecolor{currentfill}{rgb}{0.000000,0.000000,0.000000}%
\pgfsetfillcolor{currentfill}%
\pgfsetlinewidth{0.602250pt}%
\definecolor{currentstroke}{rgb}{0.000000,0.000000,0.000000}%
\pgfsetstrokecolor{currentstroke}%
\pgfsetdash{}{0pt}%
\pgfsys@defobject{currentmarker}{\pgfqpoint{-0.027778in}{0.000000in}}{\pgfqpoint{-0.000000in}{0.000000in}}{%
\pgfpathmoveto{\pgfqpoint{-0.000000in}{0.000000in}}%
\pgfpathlineto{\pgfqpoint{-0.027778in}{0.000000in}}%
\pgfusepath{stroke,fill}%
}%
\begin{pgfscope}%
\pgfsys@transformshift{0.589510in}{1.599259in}%
\pgfsys@useobject{currentmarker}{}%
\end{pgfscope}%
\end{pgfscope}%
\begin{pgfscope}%
\pgfpathrectangle{\pgfqpoint{0.589510in}{0.417642in}}{\pgfqpoint{3.437062in}{2.055000in}}%
\pgfusepath{clip}%
\pgfsetrectcap%
\pgfsetroundjoin%
\pgfsetlinewidth{0.803000pt}%
\definecolor{currentstroke}{rgb}{0.850000,0.850000,0.850000}%
\pgfsetstrokecolor{currentstroke}%
\pgfsetdash{}{0pt}%
\pgfpathmoveto{\pgfqpoint{0.589510in}{1.667562in}}%
\pgfpathlineto{\pgfqpoint{4.026572in}{1.667562in}}%
\pgfusepath{stroke}%
\end{pgfscope}%
\begin{pgfscope}%
\pgfsetbuttcap%
\pgfsetroundjoin%
\definecolor{currentfill}{rgb}{0.000000,0.000000,0.000000}%
\pgfsetfillcolor{currentfill}%
\pgfsetlinewidth{0.602250pt}%
\definecolor{currentstroke}{rgb}{0.000000,0.000000,0.000000}%
\pgfsetstrokecolor{currentstroke}%
\pgfsetdash{}{0pt}%
\pgfsys@defobject{currentmarker}{\pgfqpoint{-0.027778in}{0.000000in}}{\pgfqpoint{-0.000000in}{0.000000in}}{%
\pgfpathmoveto{\pgfqpoint{-0.000000in}{0.000000in}}%
\pgfpathlineto{\pgfqpoint{-0.027778in}{0.000000in}}%
\pgfusepath{stroke,fill}%
}%
\begin{pgfscope}%
\pgfsys@transformshift{0.589510in}{1.667562in}%
\pgfsys@useobject{currentmarker}{}%
\end{pgfscope}%
\end{pgfscope}%
\begin{pgfscope}%
\pgfpathrectangle{\pgfqpoint{0.589510in}{0.417642in}}{\pgfqpoint{3.437062in}{2.055000in}}%
\pgfusepath{clip}%
\pgfsetrectcap%
\pgfsetroundjoin%
\pgfsetlinewidth{0.803000pt}%
\definecolor{currentstroke}{rgb}{0.850000,0.850000,0.850000}%
\pgfsetstrokecolor{currentstroke}%
\pgfsetdash{}{0pt}%
\pgfpathmoveto{\pgfqpoint{0.589510in}{2.130624in}}%
\pgfpathlineto{\pgfqpoint{4.026572in}{2.130624in}}%
\pgfusepath{stroke}%
\end{pgfscope}%
\begin{pgfscope}%
\pgfsetbuttcap%
\pgfsetroundjoin%
\definecolor{currentfill}{rgb}{0.000000,0.000000,0.000000}%
\pgfsetfillcolor{currentfill}%
\pgfsetlinewidth{0.602250pt}%
\definecolor{currentstroke}{rgb}{0.000000,0.000000,0.000000}%
\pgfsetstrokecolor{currentstroke}%
\pgfsetdash{}{0pt}%
\pgfsys@defobject{currentmarker}{\pgfqpoint{-0.027778in}{0.000000in}}{\pgfqpoint{-0.000000in}{0.000000in}}{%
\pgfpathmoveto{\pgfqpoint{-0.000000in}{0.000000in}}%
\pgfpathlineto{\pgfqpoint{-0.027778in}{0.000000in}}%
\pgfusepath{stroke,fill}%
}%
\begin{pgfscope}%
\pgfsys@transformshift{0.589510in}{2.130624in}%
\pgfsys@useobject{currentmarker}{}%
\end{pgfscope}%
\end{pgfscope}%
\begin{pgfscope}%
\pgfpathrectangle{\pgfqpoint{0.589510in}{0.417642in}}{\pgfqpoint{3.437062in}{2.055000in}}%
\pgfusepath{clip}%
\pgfsetrectcap%
\pgfsetroundjoin%
\pgfsetlinewidth{0.803000pt}%
\definecolor{currentstroke}{rgb}{0.850000,0.850000,0.850000}%
\pgfsetstrokecolor{currentstroke}%
\pgfsetdash{}{0pt}%
\pgfpathmoveto{\pgfqpoint{0.589510in}{2.365756in}}%
\pgfpathlineto{\pgfqpoint{4.026572in}{2.365756in}}%
\pgfusepath{stroke}%
\end{pgfscope}%
\begin{pgfscope}%
\pgfsetbuttcap%
\pgfsetroundjoin%
\definecolor{currentfill}{rgb}{0.000000,0.000000,0.000000}%
\pgfsetfillcolor{currentfill}%
\pgfsetlinewidth{0.602250pt}%
\definecolor{currentstroke}{rgb}{0.000000,0.000000,0.000000}%
\pgfsetstrokecolor{currentstroke}%
\pgfsetdash{}{0pt}%
\pgfsys@defobject{currentmarker}{\pgfqpoint{-0.027778in}{0.000000in}}{\pgfqpoint{-0.000000in}{0.000000in}}{%
\pgfpathmoveto{\pgfqpoint{-0.000000in}{0.000000in}}%
\pgfpathlineto{\pgfqpoint{-0.027778in}{0.000000in}}%
\pgfusepath{stroke,fill}%
}%
\begin{pgfscope}%
\pgfsys@transformshift{0.589510in}{2.365756in}%
\pgfsys@useobject{currentmarker}{}%
\end{pgfscope}%
\end{pgfscope}%
\begin{pgfscope}%
\definecolor{textcolor}{rgb}{0.000000,0.000000,0.000000}%
\pgfsetstrokecolor{textcolor}%
\pgfsetfillcolor{textcolor}%
\pgftext[x=0.180559in,y=1.445142in,,bottom,rotate=90.000000]{\color{textcolor}{\rmfamily\fontsize{10.000000}{12.000000}\selectfont\catcode`\^=\active\def^{\ifmmode\sp\else\^{}\fi}\catcode`\%=\active\def%{\%}ADEV $\sigma_A(\tau)$}}%
\end{pgfscope}%
\begin{pgfscope}%
\pgfpathrectangle{\pgfqpoint{0.589510in}{0.417642in}}{\pgfqpoint{3.437062in}{2.055000in}}%
\pgfusepath{clip}%
\pgfsetbuttcap%
\pgfsetroundjoin%
\pgfsetlinewidth{1.505625pt}%
\definecolor{currentstroke}{rgb}{0.003922,0.450980,0.698039}%
\pgfsetstrokecolor{currentstroke}%
\pgfsetdash{{5.550000pt}{2.400000pt}}{0.000000pt}%
\pgfpathmoveto{\pgfqpoint{0.745740in}{1.559778in}}%
\pgfpathlineto{\pgfqpoint{0.883294in}{1.665683in}}%
\pgfpathlineto{\pgfqpoint{1.056591in}{1.790827in}}%
\pgfpathlineto{\pgfqpoint{1.255997in}{1.916697in}}%
\pgfpathlineto{\pgfqpoint{1.405889in}{1.991613in}}%
\pgfpathlineto{\pgfqpoint{1.574304in}{2.046417in}}%
\pgfpathlineto{\pgfqpoint{1.735592in}{2.060844in}}%
\pgfpathlineto{\pgfqpoint{1.899594in}{2.031622in}}%
\pgfpathlineto{\pgfqpoint{2.062558in}{1.961556in}}%
\pgfpathlineto{\pgfqpoint{2.225918in}{1.861613in}}%
\pgfpathlineto{\pgfqpoint{2.390460in}{1.743494in}}%
\pgfpathlineto{\pgfqpoint{2.554861in}{1.616021in}}%
\pgfpathlineto{\pgfqpoint{2.719191in}{1.483315in}}%
\pgfpathlineto{\pgfqpoint{2.883588in}{1.347478in}}%
\pgfpathlineto{\pgfqpoint{3.048129in}{1.209689in}}%
\pgfpathlineto{\pgfqpoint{3.212542in}{1.070902in}}%
\pgfpathlineto{\pgfqpoint{3.376951in}{0.931446in}}%
\pgfpathlineto{\pgfqpoint{3.541419in}{0.791530in}}%
\pgfpathlineto{\pgfqpoint{3.705890in}{0.651360in}}%
\pgfpathlineto{\pgfqpoint{3.870342in}{0.511051in}}%
\pgfusepath{stroke}%
\end{pgfscope}%
\begin{pgfscope}%
\pgfpathrectangle{\pgfqpoint{0.589510in}{0.417642in}}{\pgfqpoint{3.437062in}{2.055000in}}%
\pgfusepath{clip}%
\pgfsetbuttcap%
\pgfsetroundjoin%
\definecolor{currentfill}{rgb}{0.003922,0.450980,0.698039}%
\pgfsetfillcolor{currentfill}%
\pgfsetlinewidth{1.003750pt}%
\definecolor{currentstroke}{rgb}{0.003922,0.450980,0.698039}%
\pgfsetstrokecolor{currentstroke}%
\pgfsetdash{}{0pt}%
\pgfsys@defobject{currentmarker}{\pgfqpoint{-0.020833in}{-0.020833in}}{\pgfqpoint{0.020833in}{0.020833in}}{%
\pgfpathmoveto{\pgfqpoint{0.000000in}{-0.020833in}}%
\pgfpathcurveto{\pgfqpoint{0.005525in}{-0.020833in}}{\pgfqpoint{0.010825in}{-0.018638in}}{\pgfqpoint{0.014731in}{-0.014731in}}%
\pgfpathcurveto{\pgfqpoint{0.018638in}{-0.010825in}}{\pgfqpoint{0.020833in}{-0.005525in}}{\pgfqpoint{0.020833in}{0.000000in}}%
\pgfpathcurveto{\pgfqpoint{0.020833in}{0.005525in}}{\pgfqpoint{0.018638in}{0.010825in}}{\pgfqpoint{0.014731in}{0.014731in}}%
\pgfpathcurveto{\pgfqpoint{0.010825in}{0.018638in}}{\pgfqpoint{0.005525in}{0.020833in}}{\pgfqpoint{0.000000in}{0.020833in}}%
\pgfpathcurveto{\pgfqpoint{-0.005525in}{0.020833in}}{\pgfqpoint{-0.010825in}{0.018638in}}{\pgfqpoint{-0.014731in}{0.014731in}}%
\pgfpathcurveto{\pgfqpoint{-0.018638in}{0.010825in}}{\pgfqpoint{-0.020833in}{0.005525in}}{\pgfqpoint{-0.020833in}{0.000000in}}%
\pgfpathcurveto{\pgfqpoint{-0.020833in}{-0.005525in}}{\pgfqpoint{-0.018638in}{-0.010825in}}{\pgfqpoint{-0.014731in}{-0.014731in}}%
\pgfpathcurveto{\pgfqpoint{-0.010825in}{-0.018638in}}{\pgfqpoint{-0.005525in}{-0.020833in}}{\pgfqpoint{0.000000in}{-0.020833in}}%
\pgfpathlineto{\pgfqpoint{0.000000in}{-0.020833in}}%
\pgfpathclose%
\pgfusepath{stroke,fill}%
}%
\begin{pgfscope}%
\pgfsys@transformshift{0.745740in}{1.595855in}%
\pgfsys@useobject{currentmarker}{}%
\end{pgfscope}%
\begin{pgfscope}%
\pgfsys@transformshift{0.883294in}{1.682541in}%
\pgfsys@useobject{currentmarker}{}%
\end{pgfscope}%
\begin{pgfscope}%
\pgfsys@transformshift{1.056591in}{1.796965in}%
\pgfsys@useobject{currentmarker}{}%
\end{pgfscope}%
\begin{pgfscope}%
\pgfsys@transformshift{1.255997in}{1.918175in}%
\pgfsys@useobject{currentmarker}{}%
\end{pgfscope}%
\begin{pgfscope}%
\pgfsys@transformshift{1.405889in}{1.991475in}%
\pgfsys@useobject{currentmarker}{}%
\end{pgfscope}%
\begin{pgfscope}%
\pgfsys@transformshift{1.574304in}{2.045269in}%
\pgfsys@useobject{currentmarker}{}%
\end{pgfscope}%
\begin{pgfscope}%
\pgfsys@transformshift{1.735592in}{2.059350in}%
\pgfsys@useobject{currentmarker}{}%
\end{pgfscope}%
\begin{pgfscope}%
\pgfsys@transformshift{1.899594in}{2.030599in}%
\pgfsys@useobject{currentmarker}{}%
\end{pgfscope}%
\begin{pgfscope}%
\pgfsys@transformshift{2.062558in}{1.961571in}%
\pgfsys@useobject{currentmarker}{}%
\end{pgfscope}%
\begin{pgfscope}%
\pgfsys@transformshift{2.225918in}{1.860671in}%
\pgfsys@useobject{currentmarker}{}%
\end{pgfscope}%
\begin{pgfscope}%
\pgfsys@transformshift{2.390460in}{1.741341in}%
\pgfsys@useobject{currentmarker}{}%
\end{pgfscope}%
\begin{pgfscope}%
\pgfsys@transformshift{2.554861in}{1.613268in}%
\pgfsys@useobject{currentmarker}{}%
\end{pgfscope}%
\begin{pgfscope}%
\pgfsys@transformshift{2.719191in}{1.479603in}%
\pgfsys@useobject{currentmarker}{}%
\end{pgfscope}%
\begin{pgfscope}%
\pgfsys@transformshift{2.883588in}{1.344093in}%
\pgfsys@useobject{currentmarker}{}%
\end{pgfscope}%
\begin{pgfscope}%
\pgfsys@transformshift{3.048129in}{1.211443in}%
\pgfsys@useobject{currentmarker}{}%
\end{pgfscope}%
\begin{pgfscope}%
\pgfsys@transformshift{3.212542in}{1.071670in}%
\pgfsys@useobject{currentmarker}{}%
\end{pgfscope}%
\begin{pgfscope}%
\pgfsys@transformshift{3.376951in}{0.932113in}%
\pgfsys@useobject{currentmarker}{}%
\end{pgfscope}%
\begin{pgfscope}%
\pgfsys@transformshift{3.541419in}{0.793388in}%
\pgfsys@useobject{currentmarker}{}%
\end{pgfscope}%
\begin{pgfscope}%
\pgfsys@transformshift{3.705890in}{0.660482in}%
\pgfsys@useobject{currentmarker}{}%
\end{pgfscope}%
\begin{pgfscope}%
\pgfsys@transformshift{3.870342in}{0.518722in}%
\pgfsys@useobject{currentmarker}{}%
\end{pgfscope}%
\end{pgfscope}%
\begin{pgfscope}%
\pgfpathrectangle{\pgfqpoint{0.589510in}{0.417642in}}{\pgfqpoint{3.437062in}{2.055000in}}%
\pgfusepath{clip}%
\pgfsetbuttcap%
\pgfsetroundjoin%
\pgfsetlinewidth{1.505625pt}%
\definecolor{currentstroke}{rgb}{0.007843,0.619608,0.450980}%
\pgfsetstrokecolor{currentstroke}%
\pgfsetdash{{5.550000pt}{2.400000pt}}{0.000000pt}%
\pgfpathmoveto{\pgfqpoint{0.745740in}{1.405773in}}%
\pgfpathlineto{\pgfqpoint{0.883294in}{1.521174in}}%
\pgfpathlineto{\pgfqpoint{1.056591in}{1.664973in}}%
\pgfpathlineto{\pgfqpoint{1.255997in}{1.826817in}}%
\pgfpathlineto{\pgfqpoint{1.405889in}{1.944282in}}%
\pgfpathlineto{\pgfqpoint{1.574304in}{2.069309in}}%
\pgfpathlineto{\pgfqpoint{1.735592in}{2.178354in}}%
\pgfpathlineto{\pgfqpoint{1.899594in}{2.272797in}}%
\pgfpathlineto{\pgfqpoint{2.062558in}{2.342510in}}%
\pgfpathlineto{\pgfqpoint{2.225918in}{2.379020in}}%
\pgfpathlineto{\pgfqpoint{2.390460in}{2.373738in}}%
\pgfpathlineto{\pgfqpoint{2.554861in}{2.324040in}}%
\pgfpathlineto{\pgfqpoint{2.719191in}{2.237475in}}%
\pgfpathlineto{\pgfqpoint{2.883588in}{2.127186in}}%
\pgfpathlineto{\pgfqpoint{3.048129in}{2.003793in}}%
\pgfpathlineto{\pgfqpoint{3.212542in}{1.873390in}}%
\pgfpathlineto{\pgfqpoint{3.376951in}{1.738933in}}%
\pgfpathlineto{\pgfqpoint{3.541419in}{1.602036in}}%
\pgfpathlineto{\pgfqpoint{3.705890in}{1.463703in}}%
\pgfpathlineto{\pgfqpoint{3.870342in}{1.324517in}}%
\pgfusepath{stroke}%
\end{pgfscope}%
\begin{pgfscope}%
\pgfpathrectangle{\pgfqpoint{0.589510in}{0.417642in}}{\pgfqpoint{3.437062in}{2.055000in}}%
\pgfusepath{clip}%
\pgfsetbuttcap%
\pgfsetroundjoin%
\definecolor{currentfill}{rgb}{0.007843,0.619608,0.450980}%
\pgfsetfillcolor{currentfill}%
\pgfsetlinewidth{1.003750pt}%
\definecolor{currentstroke}{rgb}{0.007843,0.619608,0.450980}%
\pgfsetstrokecolor{currentstroke}%
\pgfsetdash{}{0pt}%
\pgfsys@defobject{currentmarker}{\pgfqpoint{-0.020833in}{-0.020833in}}{\pgfqpoint{0.020833in}{0.020833in}}{%
\pgfpathmoveto{\pgfqpoint{0.000000in}{-0.020833in}}%
\pgfpathcurveto{\pgfqpoint{0.005525in}{-0.020833in}}{\pgfqpoint{0.010825in}{-0.018638in}}{\pgfqpoint{0.014731in}{-0.014731in}}%
\pgfpathcurveto{\pgfqpoint{0.018638in}{-0.010825in}}{\pgfqpoint{0.020833in}{-0.005525in}}{\pgfqpoint{0.020833in}{0.000000in}}%
\pgfpathcurveto{\pgfqpoint{0.020833in}{0.005525in}}{\pgfqpoint{0.018638in}{0.010825in}}{\pgfqpoint{0.014731in}{0.014731in}}%
\pgfpathcurveto{\pgfqpoint{0.010825in}{0.018638in}}{\pgfqpoint{0.005525in}{0.020833in}}{\pgfqpoint{0.000000in}{0.020833in}}%
\pgfpathcurveto{\pgfqpoint{-0.005525in}{0.020833in}}{\pgfqpoint{-0.010825in}{0.018638in}}{\pgfqpoint{-0.014731in}{0.014731in}}%
\pgfpathcurveto{\pgfqpoint{-0.018638in}{0.010825in}}{\pgfqpoint{-0.020833in}{0.005525in}}{\pgfqpoint{-0.020833in}{0.000000in}}%
\pgfpathcurveto{\pgfqpoint{-0.020833in}{-0.005525in}}{\pgfqpoint{-0.018638in}{-0.010825in}}{\pgfqpoint{-0.014731in}{-0.014731in}}%
\pgfpathcurveto{\pgfqpoint{-0.010825in}{-0.018638in}}{\pgfqpoint{-0.005525in}{-0.020833in}}{\pgfqpoint{0.000000in}{-0.020833in}}%
\pgfpathlineto{\pgfqpoint{0.000000in}{-0.020833in}}%
\pgfpathclose%
\pgfusepath{stroke,fill}%
}%
\begin{pgfscope}%
\pgfsys@transformshift{0.745740in}{1.439137in}%
\pgfsys@useobject{currentmarker}{}%
\end{pgfscope}%
\begin{pgfscope}%
\pgfsys@transformshift{0.883294in}{1.535960in}%
\pgfsys@useobject{currentmarker}{}%
\end{pgfscope}%
\begin{pgfscope}%
\pgfsys@transformshift{1.056591in}{1.669780in}%
\pgfsys@useobject{currentmarker}{}%
\end{pgfscope}%
\begin{pgfscope}%
\pgfsys@transformshift{1.255997in}{1.827681in}%
\pgfsys@useobject{currentmarker}{}%
\end{pgfscope}%
\begin{pgfscope}%
\pgfsys@transformshift{1.405889in}{1.944202in}%
\pgfsys@useobject{currentmarker}{}%
\end{pgfscope}%
\begin{pgfscope}%
\pgfsys@transformshift{1.574304in}{2.068925in}%
\pgfsys@useobject{currentmarker}{}%
\end{pgfscope}%
\begin{pgfscope}%
\pgfsys@transformshift{1.735592in}{2.177903in}%
\pgfsys@useobject{currentmarker}{}%
\end{pgfscope}%
\begin{pgfscope}%
\pgfsys@transformshift{1.899594in}{2.272358in}%
\pgfsys@useobject{currentmarker}{}%
\end{pgfscope}%
\begin{pgfscope}%
\pgfsys@transformshift{2.062558in}{2.342387in}%
\pgfsys@useobject{currentmarker}{}%
\end{pgfscope}%
\begin{pgfscope}%
\pgfsys@transformshift{2.225918in}{2.379233in}%
\pgfsys@useobject{currentmarker}{}%
\end{pgfscope}%
\begin{pgfscope}%
\pgfsys@transformshift{2.390460in}{2.374266in}%
\pgfsys@useobject{currentmarker}{}%
\end{pgfscope}%
\begin{pgfscope}%
\pgfsys@transformshift{2.554861in}{2.325972in}%
\pgfsys@useobject{currentmarker}{}%
\end{pgfscope}%
\begin{pgfscope}%
\pgfsys@transformshift{2.719191in}{2.239006in}%
\pgfsys@useobject{currentmarker}{}%
\end{pgfscope}%
\begin{pgfscope}%
\pgfsys@transformshift{2.883588in}{2.125576in}%
\pgfsys@useobject{currentmarker}{}%
\end{pgfscope}%
\begin{pgfscope}%
\pgfsys@transformshift{3.048129in}{2.002036in}%
\pgfsys@useobject{currentmarker}{}%
\end{pgfscope}%
\begin{pgfscope}%
\pgfsys@transformshift{3.212542in}{1.870563in}%
\pgfsys@useobject{currentmarker}{}%
\end{pgfscope}%
\begin{pgfscope}%
\pgfsys@transformshift{3.376951in}{1.732331in}%
\pgfsys@useobject{currentmarker}{}%
\end{pgfscope}%
\begin{pgfscope}%
\pgfsys@transformshift{3.541419in}{1.599236in}%
\pgfsys@useobject{currentmarker}{}%
\end{pgfscope}%
\begin{pgfscope}%
\pgfsys@transformshift{3.705890in}{1.469932in}%
\pgfsys@useobject{currentmarker}{}%
\end{pgfscope}%
\begin{pgfscope}%
\pgfsys@transformshift{3.870342in}{1.322460in}%
\pgfsys@useobject{currentmarker}{}%
\end{pgfscope}%
\end{pgfscope}%
\begin{pgfscope}%
\pgfpathrectangle{\pgfqpoint{0.589510in}{0.417642in}}{\pgfqpoint{3.437062in}{2.055000in}}%
\pgfusepath{clip}%
\pgfsetbuttcap%
\pgfsetroundjoin%
\pgfsetlinewidth{1.505625pt}%
\definecolor{currentstroke}{rgb}{0.835294,0.368627,0.000000}%
\pgfsetstrokecolor{currentstroke}%
\pgfsetdash{{5.550000pt}{2.400000pt}}{0.000000pt}%
\pgfpathmoveto{\pgfqpoint{0.745740in}{0.913425in}}%
\pgfpathlineto{\pgfqpoint{0.883294in}{1.029799in}}%
\pgfpathlineto{\pgfqpoint{1.056591in}{1.175532in}}%
\pgfpathlineto{\pgfqpoint{1.255997in}{1.341213in}}%
\pgfpathlineto{\pgfqpoint{1.405889in}{1.463413in}}%
\pgfpathlineto{\pgfqpoint{1.574304in}{1.596792in}}%
\pgfpathlineto{\pgfqpoint{1.735592in}{1.718399in}}%
\pgfpathlineto{\pgfqpoint{1.899594in}{1.832381in}}%
\pgfpathlineto{\pgfqpoint{2.062558in}{1.930803in}}%
\pgfpathlineto{\pgfqpoint{2.225918in}{2.007370in}}%
\pgfpathlineto{\pgfqpoint{2.390460in}{2.052996in}}%
\pgfpathlineto{\pgfqpoint{2.554861in}{2.058063in}}%
\pgfpathlineto{\pgfqpoint{2.719191in}{2.018461in}}%
\pgfpathlineto{\pgfqpoint{2.883588in}{1.939376in}}%
\pgfpathlineto{\pgfqpoint{3.048129in}{1.833376in}}%
\pgfpathlineto{\pgfqpoint{3.212542in}{1.712430in}}%
\pgfpathlineto{\pgfqpoint{3.376951in}{1.583345in}}%
\pgfpathlineto{\pgfqpoint{3.541419in}{1.449602in}}%
\pgfpathlineto{\pgfqpoint{3.705890in}{1.313155in}}%
\pgfpathlineto{\pgfqpoint{3.870342in}{1.175110in}}%
\pgfusepath{stroke}%
\end{pgfscope}%
\begin{pgfscope}%
\pgfpathrectangle{\pgfqpoint{0.589510in}{0.417642in}}{\pgfqpoint{3.437062in}{2.055000in}}%
\pgfusepath{clip}%
\pgfsetbuttcap%
\pgfsetroundjoin%
\definecolor{currentfill}{rgb}{0.835294,0.368627,0.000000}%
\pgfsetfillcolor{currentfill}%
\pgfsetlinewidth{1.003750pt}%
\definecolor{currentstroke}{rgb}{0.835294,0.368627,0.000000}%
\pgfsetstrokecolor{currentstroke}%
\pgfsetdash{}{0pt}%
\pgfsys@defobject{currentmarker}{\pgfqpoint{-0.020833in}{-0.020833in}}{\pgfqpoint{0.020833in}{0.020833in}}{%
\pgfpathmoveto{\pgfqpoint{0.000000in}{-0.020833in}}%
\pgfpathcurveto{\pgfqpoint{0.005525in}{-0.020833in}}{\pgfqpoint{0.010825in}{-0.018638in}}{\pgfqpoint{0.014731in}{-0.014731in}}%
\pgfpathcurveto{\pgfqpoint{0.018638in}{-0.010825in}}{\pgfqpoint{0.020833in}{-0.005525in}}{\pgfqpoint{0.020833in}{0.000000in}}%
\pgfpathcurveto{\pgfqpoint{0.020833in}{0.005525in}}{\pgfqpoint{0.018638in}{0.010825in}}{\pgfqpoint{0.014731in}{0.014731in}}%
\pgfpathcurveto{\pgfqpoint{0.010825in}{0.018638in}}{\pgfqpoint{0.005525in}{0.020833in}}{\pgfqpoint{0.000000in}{0.020833in}}%
\pgfpathcurveto{\pgfqpoint{-0.005525in}{0.020833in}}{\pgfqpoint{-0.010825in}{0.018638in}}{\pgfqpoint{-0.014731in}{0.014731in}}%
\pgfpathcurveto{\pgfqpoint{-0.018638in}{0.010825in}}{\pgfqpoint{-0.020833in}{0.005525in}}{\pgfqpoint{-0.020833in}{0.000000in}}%
\pgfpathcurveto{\pgfqpoint{-0.020833in}{-0.005525in}}{\pgfqpoint{-0.018638in}{-0.010825in}}{\pgfqpoint{-0.014731in}{-0.014731in}}%
\pgfpathcurveto{\pgfqpoint{-0.010825in}{-0.018638in}}{\pgfqpoint{-0.005525in}{-0.020833in}}{\pgfqpoint{0.000000in}{-0.020833in}}%
\pgfpathlineto{\pgfqpoint{0.000000in}{-0.020833in}}%
\pgfpathclose%
\pgfusepath{stroke,fill}%
}%
\begin{pgfscope}%
\pgfsys@transformshift{0.745740in}{0.947418in}%
\pgfsys@useobject{currentmarker}{}%
\end{pgfscope}%
\begin{pgfscope}%
\pgfsys@transformshift{0.883294in}{1.045270in}%
\pgfsys@useobject{currentmarker}{}%
\end{pgfscope}%
\begin{pgfscope}%
\pgfsys@transformshift{1.056591in}{1.181028in}%
\pgfsys@useobject{currentmarker}{}%
\end{pgfscope}%
\begin{pgfscope}%
\pgfsys@transformshift{1.255997in}{1.342847in}%
\pgfsys@useobject{currentmarker}{}%
\end{pgfscope}%
\begin{pgfscope}%
\pgfsys@transformshift{1.405889in}{1.464266in}%
\pgfsys@useobject{currentmarker}{}%
\end{pgfscope}%
\begin{pgfscope}%
\pgfsys@transformshift{1.574304in}{1.597467in}%
\pgfsys@useobject{currentmarker}{}%
\end{pgfscope}%
\begin{pgfscope}%
\pgfsys@transformshift{1.735592in}{1.718836in}%
\pgfsys@useobject{currentmarker}{}%
\end{pgfscope}%
\begin{pgfscope}%
\pgfsys@transformshift{1.899594in}{1.832070in}%
\pgfsys@useobject{currentmarker}{}%
\end{pgfscope}%
\begin{pgfscope}%
\pgfsys@transformshift{2.062558in}{1.929928in}%
\pgfsys@useobject{currentmarker}{}%
\end{pgfscope}%
\begin{pgfscope}%
\pgfsys@transformshift{2.225918in}{2.006037in}%
\pgfsys@useobject{currentmarker}{}%
\end{pgfscope}%
\begin{pgfscope}%
\pgfsys@transformshift{2.390460in}{2.051126in}%
\pgfsys@useobject{currentmarker}{}%
\end{pgfscope}%
\begin{pgfscope}%
\pgfsys@transformshift{2.554861in}{2.054991in}%
\pgfsys@useobject{currentmarker}{}%
\end{pgfscope}%
\begin{pgfscope}%
\pgfsys@transformshift{2.719191in}{2.013598in}%
\pgfsys@useobject{currentmarker}{}%
\end{pgfscope}%
\begin{pgfscope}%
\pgfsys@transformshift{2.883588in}{1.934049in}%
\pgfsys@useobject{currentmarker}{}%
\end{pgfscope}%
\begin{pgfscope}%
\pgfsys@transformshift{3.048129in}{1.829443in}%
\pgfsys@useobject{currentmarker}{}%
\end{pgfscope}%
\begin{pgfscope}%
\pgfsys@transformshift{3.212542in}{1.710452in}%
\pgfsys@useobject{currentmarker}{}%
\end{pgfscope}%
\begin{pgfscope}%
\pgfsys@transformshift{3.376951in}{1.583831in}%
\pgfsys@useobject{currentmarker}{}%
\end{pgfscope}%
\begin{pgfscope}%
\pgfsys@transformshift{3.541419in}{1.446900in}%
\pgfsys@useobject{currentmarker}{}%
\end{pgfscope}%
\begin{pgfscope}%
\pgfsys@transformshift{3.705890in}{1.301542in}%
\pgfsys@useobject{currentmarker}{}%
\end{pgfscope}%
\begin{pgfscope}%
\pgfsys@transformshift{3.870342in}{1.156551in}%
\pgfsys@useobject{currentmarker}{}%
\end{pgfscope}%
\end{pgfscope}%
\begin{pgfscope}%
\pgfsetrectcap%
\pgfsetmiterjoin%
\pgfsetlinewidth{0.803000pt}%
\definecolor{currentstroke}{rgb}{0.000000,0.000000,0.000000}%
\pgfsetstrokecolor{currentstroke}%
\pgfsetdash{}{0pt}%
\pgfpathmoveto{\pgfqpoint{0.589510in}{0.417642in}}%
\pgfpathlineto{\pgfqpoint{0.589510in}{2.472642in}}%
\pgfusepath{stroke}%
\end{pgfscope}%
\begin{pgfscope}%
\pgfsetrectcap%
\pgfsetmiterjoin%
\pgfsetlinewidth{0.803000pt}%
\definecolor{currentstroke}{rgb}{0.000000,0.000000,0.000000}%
\pgfsetstrokecolor{currentstroke}%
\pgfsetdash{}{0pt}%
\pgfpathmoveto{\pgfqpoint{4.026572in}{0.417642in}}%
\pgfpathlineto{\pgfqpoint{4.026572in}{2.472642in}}%
\pgfusepath{stroke}%
\end{pgfscope}%
\begin{pgfscope}%
\pgfsetrectcap%
\pgfsetmiterjoin%
\pgfsetlinewidth{0.803000pt}%
\definecolor{currentstroke}{rgb}{0.000000,0.000000,0.000000}%
\pgfsetstrokecolor{currentstroke}%
\pgfsetdash{}{0pt}%
\pgfpathmoveto{\pgfqpoint{0.589510in}{0.417642in}}%
\pgfpathlineto{\pgfqpoint{4.026572in}{0.417642in}}%
\pgfusepath{stroke}%
\end{pgfscope}%
\begin{pgfscope}%
\pgfsetrectcap%
\pgfsetmiterjoin%
\pgfsetlinewidth{0.803000pt}%
\definecolor{currentstroke}{rgb}{0.000000,0.000000,0.000000}%
\pgfsetstrokecolor{currentstroke}%
\pgfsetdash{}{0pt}%
\pgfpathmoveto{\pgfqpoint{0.589510in}{2.472642in}}%
\pgfpathlineto{\pgfqpoint{4.026572in}{2.472642in}}%
\pgfusepath{stroke}%
\end{pgfscope}%
\begin{pgfscope}%
\pgfsetbuttcap%
\pgfsetmiterjoin%
\definecolor{currentfill}{rgb}{1.000000,1.000000,1.000000}%
\pgfsetfillcolor{currentfill}%
\pgfsetfillopacity{0.800000}%
\pgfsetlinewidth{1.003750pt}%
\definecolor{currentstroke}{rgb}{0.800000,0.800000,0.800000}%
\pgfsetstrokecolor{currentstroke}%
\pgfsetstrokeopacity{0.800000}%
\pgfsetdash{}{0pt}%
\pgfpathmoveto{\pgfqpoint{3.108484in}{1.919086in}}%
\pgfpathlineto{\pgfqpoint{3.948794in}{1.919086in}}%
\pgfpathquadraticcurveto{\pgfqpoint{3.971016in}{1.919086in}}{\pgfqpoint{3.971016in}{1.941309in}}%
\pgfpathlineto{\pgfqpoint{3.971016in}{2.394864in}}%
\pgfpathquadraticcurveto{\pgfqpoint{3.971016in}{2.417086in}}{\pgfqpoint{3.948794in}{2.417086in}}%
\pgfpathlineto{\pgfqpoint{3.108484in}{2.417086in}}%
\pgfpathquadraticcurveto{\pgfqpoint{3.086261in}{2.417086in}}{\pgfqpoint{3.086261in}{2.394864in}}%
\pgfpathlineto{\pgfqpoint{3.086261in}{1.941309in}}%
\pgfpathquadraticcurveto{\pgfqpoint{3.086261in}{1.919086in}}{\pgfqpoint{3.108484in}{1.919086in}}%
\pgfpathlineto{\pgfqpoint{3.108484in}{1.919086in}}%
\pgfpathclose%
\pgfusepath{stroke,fill}%
\end{pgfscope}%
\begin{pgfscope}%
\pgfsetbuttcap%
\pgfsetroundjoin%
\definecolor{currentfill}{rgb}{0.003922,0.450980,0.698039}%
\pgfsetfillcolor{currentfill}%
\pgfsetlinewidth{1.003750pt}%
\definecolor{currentstroke}{rgb}{0.003922,0.450980,0.698039}%
\pgfsetstrokecolor{currentstroke}%
\pgfsetdash{}{0pt}%
\pgfsys@defobject{currentmarker}{\pgfqpoint{-0.020833in}{-0.020833in}}{\pgfqpoint{0.020833in}{0.020833in}}{%
\pgfpathmoveto{\pgfqpoint{0.000000in}{-0.020833in}}%
\pgfpathcurveto{\pgfqpoint{0.005525in}{-0.020833in}}{\pgfqpoint{0.010825in}{-0.018638in}}{\pgfqpoint{0.014731in}{-0.014731in}}%
\pgfpathcurveto{\pgfqpoint{0.018638in}{-0.010825in}}{\pgfqpoint{0.020833in}{-0.005525in}}{\pgfqpoint{0.020833in}{0.000000in}}%
\pgfpathcurveto{\pgfqpoint{0.020833in}{0.005525in}}{\pgfqpoint{0.018638in}{0.010825in}}{\pgfqpoint{0.014731in}{0.014731in}}%
\pgfpathcurveto{\pgfqpoint{0.010825in}{0.018638in}}{\pgfqpoint{0.005525in}{0.020833in}}{\pgfqpoint{0.000000in}{0.020833in}}%
\pgfpathcurveto{\pgfqpoint{-0.005525in}{0.020833in}}{\pgfqpoint{-0.010825in}{0.018638in}}{\pgfqpoint{-0.014731in}{0.014731in}}%
\pgfpathcurveto{\pgfqpoint{-0.018638in}{0.010825in}}{\pgfqpoint{-0.020833in}{0.005525in}}{\pgfqpoint{-0.020833in}{0.000000in}}%
\pgfpathcurveto{\pgfqpoint{-0.020833in}{-0.005525in}}{\pgfqpoint{-0.018638in}{-0.010825in}}{\pgfqpoint{-0.014731in}{-0.014731in}}%
\pgfpathcurveto{\pgfqpoint{-0.010825in}{-0.018638in}}{\pgfqpoint{-0.005525in}{-0.020833in}}{\pgfqpoint{0.000000in}{-0.020833in}}%
\pgfpathlineto{\pgfqpoint{0.000000in}{-0.020833in}}%
\pgfpathclose%
\pgfusepath{stroke,fill}%
}%
\begin{pgfscope}%
\pgfsys@transformshift{3.241817in}{2.333753in}%
\pgfsys@useobject{currentmarker}{}%
\end{pgfscope}%
\end{pgfscope}%
\begin{pgfscope}%
\definecolor{textcolor}{rgb}{0.000000,0.000000,0.000000}%
\pgfsetstrokecolor{textcolor}%
\pgfsetfillcolor{textcolor}%
\pgftext[x=3.441817in,y=2.294864in,left,base]{\color{textcolor}{\rmfamily\fontsize{8.000000}{9.600000}\selectfont\catcode`\^=\active\def^{\ifmmode\sp\else\^{}\fi}\catcode`\%=\active\def%{\%}$\bar\tau_1=\qty{0.1}{\s}$}}%
\end{pgfscope}%
\begin{pgfscope}%
\pgfsetbuttcap%
\pgfsetroundjoin%
\definecolor{currentfill}{rgb}{0.007843,0.619608,0.450980}%
\pgfsetfillcolor{currentfill}%
\pgfsetlinewidth{1.003750pt}%
\definecolor{currentstroke}{rgb}{0.007843,0.619608,0.450980}%
\pgfsetstrokecolor{currentstroke}%
\pgfsetdash{}{0pt}%
\pgfsys@defobject{currentmarker}{\pgfqpoint{-0.020833in}{-0.020833in}}{\pgfqpoint{0.020833in}{0.020833in}}{%
\pgfpathmoveto{\pgfqpoint{0.000000in}{-0.020833in}}%
\pgfpathcurveto{\pgfqpoint{0.005525in}{-0.020833in}}{\pgfqpoint{0.010825in}{-0.018638in}}{\pgfqpoint{0.014731in}{-0.014731in}}%
\pgfpathcurveto{\pgfqpoint{0.018638in}{-0.010825in}}{\pgfqpoint{0.020833in}{-0.005525in}}{\pgfqpoint{0.020833in}{0.000000in}}%
\pgfpathcurveto{\pgfqpoint{0.020833in}{0.005525in}}{\pgfqpoint{0.018638in}{0.010825in}}{\pgfqpoint{0.014731in}{0.014731in}}%
\pgfpathcurveto{\pgfqpoint{0.010825in}{0.018638in}}{\pgfqpoint{0.005525in}{0.020833in}}{\pgfqpoint{0.000000in}{0.020833in}}%
\pgfpathcurveto{\pgfqpoint{-0.005525in}{0.020833in}}{\pgfqpoint{-0.010825in}{0.018638in}}{\pgfqpoint{-0.014731in}{0.014731in}}%
\pgfpathcurveto{\pgfqpoint{-0.018638in}{0.010825in}}{\pgfqpoint{-0.020833in}{0.005525in}}{\pgfqpoint{-0.020833in}{0.000000in}}%
\pgfpathcurveto{\pgfqpoint{-0.020833in}{-0.005525in}}{\pgfqpoint{-0.018638in}{-0.010825in}}{\pgfqpoint{-0.014731in}{-0.014731in}}%
\pgfpathcurveto{\pgfqpoint{-0.010825in}{-0.018638in}}{\pgfqpoint{-0.005525in}{-0.020833in}}{\pgfqpoint{0.000000in}{-0.020833in}}%
\pgfpathlineto{\pgfqpoint{0.000000in}{-0.020833in}}%
\pgfpathclose%
\pgfusepath{stroke,fill}%
}%
\begin{pgfscope}%
\pgfsys@transformshift{3.241817in}{2.178864in}%
\pgfsys@useobject{currentmarker}{}%
\end{pgfscope}%
\end{pgfscope}%
\begin{pgfscope}%
\definecolor{textcolor}{rgb}{0.000000,0.000000,0.000000}%
\pgfsetstrokecolor{textcolor}%
\pgfsetfillcolor{textcolor}%
\pgftext[x=3.441817in,y=2.139975in,left,base]{\color{textcolor}{\rmfamily\fontsize{8.000000}{9.600000}\selectfont\catcode`\^=\active\def^{\ifmmode\sp\else\^{}\fi}\catcode`\%=\active\def%{\%}$\bar\tau_1=\qty{1}{\s}$}}%
\end{pgfscope}%
\begin{pgfscope}%
\pgfsetbuttcap%
\pgfsetroundjoin%
\definecolor{currentfill}{rgb}{0.835294,0.368627,0.000000}%
\pgfsetfillcolor{currentfill}%
\pgfsetlinewidth{1.003750pt}%
\definecolor{currentstroke}{rgb}{0.835294,0.368627,0.000000}%
\pgfsetstrokecolor{currentstroke}%
\pgfsetdash{}{0pt}%
\pgfsys@defobject{currentmarker}{\pgfqpoint{-0.020833in}{-0.020833in}}{\pgfqpoint{0.020833in}{0.020833in}}{%
\pgfpathmoveto{\pgfqpoint{0.000000in}{-0.020833in}}%
\pgfpathcurveto{\pgfqpoint{0.005525in}{-0.020833in}}{\pgfqpoint{0.010825in}{-0.018638in}}{\pgfqpoint{0.014731in}{-0.014731in}}%
\pgfpathcurveto{\pgfqpoint{0.018638in}{-0.010825in}}{\pgfqpoint{0.020833in}{-0.005525in}}{\pgfqpoint{0.020833in}{0.000000in}}%
\pgfpathcurveto{\pgfqpoint{0.020833in}{0.005525in}}{\pgfqpoint{0.018638in}{0.010825in}}{\pgfqpoint{0.014731in}{0.014731in}}%
\pgfpathcurveto{\pgfqpoint{0.010825in}{0.018638in}}{\pgfqpoint{0.005525in}{0.020833in}}{\pgfqpoint{0.000000in}{0.020833in}}%
\pgfpathcurveto{\pgfqpoint{-0.005525in}{0.020833in}}{\pgfqpoint{-0.010825in}{0.018638in}}{\pgfqpoint{-0.014731in}{0.014731in}}%
\pgfpathcurveto{\pgfqpoint{-0.018638in}{0.010825in}}{\pgfqpoint{-0.020833in}{0.005525in}}{\pgfqpoint{-0.020833in}{0.000000in}}%
\pgfpathcurveto{\pgfqpoint{-0.020833in}{-0.005525in}}{\pgfqpoint{-0.018638in}{-0.010825in}}{\pgfqpoint{-0.014731in}{-0.014731in}}%
\pgfpathcurveto{\pgfqpoint{-0.010825in}{-0.018638in}}{\pgfqpoint{-0.005525in}{-0.020833in}}{\pgfqpoint{0.000000in}{-0.020833in}}%
\pgfpathlineto{\pgfqpoint{0.000000in}{-0.020833in}}%
\pgfpathclose%
\pgfusepath{stroke,fill}%
}%
\begin{pgfscope}%
\pgfsys@transformshift{3.241817in}{2.023975in}%
\pgfsys@useobject{currentmarker}{}%
\end{pgfscope}%
\end{pgfscope}%
\begin{pgfscope}%
\definecolor{textcolor}{rgb}{0.000000,0.000000,0.000000}%
\pgfsetstrokecolor{textcolor}%
\pgfsetfillcolor{textcolor}%
\pgftext[x=3.441817in,y=1.985086in,left,base]{\color{textcolor}{\rmfamily\fontsize{8.000000}{9.600000}\selectfont\catcode`\^=\active\def^{\ifmmode\sp\else\^{}\fi}\catcode`\%=\active\def%{\%}$\bar\tau_1=\qty{10}{\s}$}}%
\end{pgfscope}%
\end{pgfpicture}%
\makeatother%
\endgroup%

        } % scalebox
        \caption{Allan deviation}
        \label{fig:burst_noise_adev}
    \end{subfigure}
    \caption{Different representations of burst noise for different $\bar \tau_1$ and fixed $\bar \tau_0 = \qty{1}{\s}$.}
    \label{fig:burst_noise_simulated}
\end{figure}

The burst noise equations can used to gain further insight into other types of noise. The first one is Shot noise, which is commonly found in photodetectors and lasers. Here, electrons or photons are created at discrete intervals resulting in an instantationous signal. This means, that the lifetime of the upper level is very short in comparison to the lower level ($\tau_1 \ll \tau_0$) equation \ref{eqn:burst_noise_psd} becomes:
\begin{align}
    S_{Shot}(\omega) = S_{\tau_1 \ll \tau_0}(\omega) &= 4 \Delta y^2 \frac{\tau_1}{\tau_0} \frac{\frac{1}{\bar \tau_1}}{\left(\frac{1}{\bar \tau_1}\right)^2 + \omega^2}\nonumber\\
    &= 4 \Delta y^2 \frac{1}{\tau_0} \frac{1}{\frac{1}{\tau_1^2}+\omega^2}\\
    \overset{\omega \ll 1/\tau_0}&{\approx} 4 \Delta y^2 \frac{\tau_1^2}{\tau_0} = \text{const.}
\end{align}

For the typical case, a very large number of such events happen. When not counting single events, but rather a stream, the relation $\omega \ll 1/\tau_0$ is valid and hence the results is a white spectrum as $S_{Shot}(\omega)$ is constant with respect to $\omega$ --- just as observed in photodetectors and lasers.

The other interesting case is a case, where many trap sites with different time constants are contributing to the noise. This can change the shape of the spectrum from $f^{-2}$ to $f^{-1}$ and is discussed in the next section.

\clearpage
\subsubsection{Flicker Noise}
\label{sec:flicker_noise}
Flicker noise is also called $\frac 1 f$-noise and it can be observed in many naturally occuring phenomenen. Its origin is not clear, although there have been many explanations. An overview can be found in \cite{flicker_noise_overview, flicker_noise_overview2, origins_1_f_noise}. This work concentrates on flicker noise in electronic devices. In thick-film resistors, for example, it was shown to extend over at least 6 decades without any visible flattening \cite{1_f_noise_thick_film}. In transistors, flicker noise is caused by the existance of generation-recombination noise or burst noise discussed in the previous section \cite{origins_1_f_noise}. If there are many uncorrelated trap sites, that contribute to the total noise, the envelope of the noise spectral density changes from $\frac{1}{f^2}$ to $\frac{1}{f^1}$ as shown in figure \ref{fig:flicker_noise_evelope}

\begin{figure}[hb]
    \centering
    %% Creator: Matplotlib, PGF backend
%%
%% To include the figure in your LaTeX document, write
%%   \input{<filename>.pgf}
%%
%% Make sure the required packages are loaded in your preamble
%%   \usepackage{pgf}
%%
%% Also ensure that all the required font packages are loaded; for instance,
%% the lmodern package is sometimes necessary when using math font.
%%   \usepackage{lmodern}
%%
%% Figures using additional raster images can only be included by \input if
%% they are in the same directory as the main LaTeX file. For loading figures
%% from other directories you can use the `import` package
%%   \usepackage{import}
%%
%% and then include the figures with
%%   \import{<path to file>}{<filename>.pgf}
%%
%% Matplotlib used the following preamble
%%   \usepackage{siunitx}
%%   \usepackage{fontspec}
%%   \makeatletter\@ifpackageloaded{underscore}{}{\usepackage[strings]{underscore}}\makeatother
%%
\begingroup%
\makeatletter%
\begin{pgfpicture}%
\pgfpathrectangle{\pgfpointorigin}{\pgfqpoint{4.060000in}{2.510000in}}%
\pgfusepath{use as bounding box, clip}%
\begin{pgfscope}%
\pgfsetbuttcap%
\pgfsetmiterjoin%
\definecolor{currentfill}{rgb}{1.000000,1.000000,1.000000}%
\pgfsetfillcolor{currentfill}%
\pgfsetlinewidth{0.000000pt}%
\definecolor{currentstroke}{rgb}{1.000000,1.000000,1.000000}%
\pgfsetstrokecolor{currentstroke}%
\pgfsetdash{}{0pt}%
\pgfpathmoveto{\pgfqpoint{0.000000in}{0.000000in}}%
\pgfpathlineto{\pgfqpoint{4.060000in}{0.000000in}}%
\pgfpathlineto{\pgfqpoint{4.060000in}{2.510000in}}%
\pgfpathlineto{\pgfqpoint{0.000000in}{2.510000in}}%
\pgfpathlineto{\pgfqpoint{0.000000in}{0.000000in}}%
\pgfpathclose%
\pgfusepath{fill}%
\end{pgfscope}%
\begin{pgfscope}%
\pgfsetbuttcap%
\pgfsetmiterjoin%
\definecolor{currentfill}{rgb}{1.000000,1.000000,1.000000}%
\pgfsetfillcolor{currentfill}%
\pgfsetlinewidth{0.000000pt}%
\definecolor{currentstroke}{rgb}{0.000000,0.000000,0.000000}%
\pgfsetstrokecolor{currentstroke}%
\pgfsetstrokeopacity{0.000000}%
\pgfsetdash{}{0pt}%
\pgfpathmoveto{\pgfqpoint{0.594525in}{0.417642in}}%
\pgfpathlineto{\pgfqpoint{4.018330in}{0.417642in}}%
\pgfpathlineto{\pgfqpoint{4.018330in}{2.429177in}}%
\pgfpathlineto{\pgfqpoint{0.594525in}{2.429177in}}%
\pgfpathlineto{\pgfqpoint{0.594525in}{0.417642in}}%
\pgfpathclose%
\pgfusepath{fill}%
\end{pgfscope}%
\begin{pgfscope}%
\pgfpathrectangle{\pgfqpoint{0.594525in}{0.417642in}}{\pgfqpoint{3.423805in}{2.011535in}}%
\pgfusepath{clip}%
\pgfsetrectcap%
\pgfsetroundjoin%
\pgfsetlinewidth{0.803000pt}%
\definecolor{currentstroke}{rgb}{0.450000,0.450000,0.450000}%
\pgfsetstrokecolor{currentstroke}%
\pgfsetdash{}{0pt}%
\pgfpathmoveto{\pgfqpoint{0.750152in}{0.417642in}}%
\pgfpathlineto{\pgfqpoint{0.750152in}{2.429177in}}%
\pgfusepath{stroke}%
\end{pgfscope}%
\begin{pgfscope}%
\pgfsetbuttcap%
\pgfsetroundjoin%
\definecolor{currentfill}{rgb}{0.000000,0.000000,0.000000}%
\pgfsetfillcolor{currentfill}%
\pgfsetlinewidth{0.803000pt}%
\definecolor{currentstroke}{rgb}{0.000000,0.000000,0.000000}%
\pgfsetstrokecolor{currentstroke}%
\pgfsetdash{}{0pt}%
\pgfsys@defobject{currentmarker}{\pgfqpoint{0.000000in}{-0.048611in}}{\pgfqpoint{0.000000in}{0.000000in}}{%
\pgfpathmoveto{\pgfqpoint{0.000000in}{0.000000in}}%
\pgfpathlineto{\pgfqpoint{0.000000in}{-0.048611in}}%
\pgfusepath{stroke,fill}%
}%
\begin{pgfscope}%
\pgfsys@transformshift{0.750152in}{0.417642in}%
\pgfsys@useobject{currentmarker}{}%
\end{pgfscope}%
\end{pgfscope}%
\begin{pgfscope}%
\definecolor{textcolor}{rgb}{0.000000,0.000000,0.000000}%
\pgfsetstrokecolor{textcolor}%
\pgfsetfillcolor{textcolor}%
\pgftext[x=0.750152in,y=0.320420in,,top]{\color{textcolor}\rmfamily\fontsize{8.000000}{9.600000}\selectfont \(\displaystyle {10^{-2}}\)}%
\end{pgfscope}%
\begin{pgfscope}%
\pgfpathrectangle{\pgfqpoint{0.594525in}{0.417642in}}{\pgfqpoint{3.423805in}{2.011535in}}%
\pgfusepath{clip}%
\pgfsetrectcap%
\pgfsetroundjoin%
\pgfsetlinewidth{0.803000pt}%
\definecolor{currentstroke}{rgb}{0.450000,0.450000,0.450000}%
\pgfsetstrokecolor{currentstroke}%
\pgfsetdash{}{0pt}%
\pgfpathmoveto{\pgfqpoint{1.528290in}{0.417642in}}%
\pgfpathlineto{\pgfqpoint{1.528290in}{2.429177in}}%
\pgfusepath{stroke}%
\end{pgfscope}%
\begin{pgfscope}%
\pgfsetbuttcap%
\pgfsetroundjoin%
\definecolor{currentfill}{rgb}{0.000000,0.000000,0.000000}%
\pgfsetfillcolor{currentfill}%
\pgfsetlinewidth{0.803000pt}%
\definecolor{currentstroke}{rgb}{0.000000,0.000000,0.000000}%
\pgfsetstrokecolor{currentstroke}%
\pgfsetdash{}{0pt}%
\pgfsys@defobject{currentmarker}{\pgfqpoint{0.000000in}{-0.048611in}}{\pgfqpoint{0.000000in}{0.000000in}}{%
\pgfpathmoveto{\pgfqpoint{0.000000in}{0.000000in}}%
\pgfpathlineto{\pgfqpoint{0.000000in}{-0.048611in}}%
\pgfusepath{stroke,fill}%
}%
\begin{pgfscope}%
\pgfsys@transformshift{1.528290in}{0.417642in}%
\pgfsys@useobject{currentmarker}{}%
\end{pgfscope}%
\end{pgfscope}%
\begin{pgfscope}%
\definecolor{textcolor}{rgb}{0.000000,0.000000,0.000000}%
\pgfsetstrokecolor{textcolor}%
\pgfsetfillcolor{textcolor}%
\pgftext[x=1.528290in,y=0.320420in,,top]{\color{textcolor}\rmfamily\fontsize{8.000000}{9.600000}\selectfont \(\displaystyle {10^{-1}}\)}%
\end{pgfscope}%
\begin{pgfscope}%
\pgfpathrectangle{\pgfqpoint{0.594525in}{0.417642in}}{\pgfqpoint{3.423805in}{2.011535in}}%
\pgfusepath{clip}%
\pgfsetrectcap%
\pgfsetroundjoin%
\pgfsetlinewidth{0.803000pt}%
\definecolor{currentstroke}{rgb}{0.450000,0.450000,0.450000}%
\pgfsetstrokecolor{currentstroke}%
\pgfsetdash{}{0pt}%
\pgfpathmoveto{\pgfqpoint{2.306427in}{0.417642in}}%
\pgfpathlineto{\pgfqpoint{2.306427in}{2.429177in}}%
\pgfusepath{stroke}%
\end{pgfscope}%
\begin{pgfscope}%
\pgfsetbuttcap%
\pgfsetroundjoin%
\definecolor{currentfill}{rgb}{0.000000,0.000000,0.000000}%
\pgfsetfillcolor{currentfill}%
\pgfsetlinewidth{0.803000pt}%
\definecolor{currentstroke}{rgb}{0.000000,0.000000,0.000000}%
\pgfsetstrokecolor{currentstroke}%
\pgfsetdash{}{0pt}%
\pgfsys@defobject{currentmarker}{\pgfqpoint{0.000000in}{-0.048611in}}{\pgfqpoint{0.000000in}{0.000000in}}{%
\pgfpathmoveto{\pgfqpoint{0.000000in}{0.000000in}}%
\pgfpathlineto{\pgfqpoint{0.000000in}{-0.048611in}}%
\pgfusepath{stroke,fill}%
}%
\begin{pgfscope}%
\pgfsys@transformshift{2.306427in}{0.417642in}%
\pgfsys@useobject{currentmarker}{}%
\end{pgfscope}%
\end{pgfscope}%
\begin{pgfscope}%
\definecolor{textcolor}{rgb}{0.000000,0.000000,0.000000}%
\pgfsetstrokecolor{textcolor}%
\pgfsetfillcolor{textcolor}%
\pgftext[x=2.306427in,y=0.320420in,,top]{\color{textcolor}\rmfamily\fontsize{8.000000}{9.600000}\selectfont \(\displaystyle {10^{0}}\)}%
\end{pgfscope}%
\begin{pgfscope}%
\pgfpathrectangle{\pgfqpoint{0.594525in}{0.417642in}}{\pgfqpoint{3.423805in}{2.011535in}}%
\pgfusepath{clip}%
\pgfsetrectcap%
\pgfsetroundjoin%
\pgfsetlinewidth{0.803000pt}%
\definecolor{currentstroke}{rgb}{0.450000,0.450000,0.450000}%
\pgfsetstrokecolor{currentstroke}%
\pgfsetdash{}{0pt}%
\pgfpathmoveto{\pgfqpoint{3.084565in}{0.417642in}}%
\pgfpathlineto{\pgfqpoint{3.084565in}{2.429177in}}%
\pgfusepath{stroke}%
\end{pgfscope}%
\begin{pgfscope}%
\pgfsetbuttcap%
\pgfsetroundjoin%
\definecolor{currentfill}{rgb}{0.000000,0.000000,0.000000}%
\pgfsetfillcolor{currentfill}%
\pgfsetlinewidth{0.803000pt}%
\definecolor{currentstroke}{rgb}{0.000000,0.000000,0.000000}%
\pgfsetstrokecolor{currentstroke}%
\pgfsetdash{}{0pt}%
\pgfsys@defobject{currentmarker}{\pgfqpoint{0.000000in}{-0.048611in}}{\pgfqpoint{0.000000in}{0.000000in}}{%
\pgfpathmoveto{\pgfqpoint{0.000000in}{0.000000in}}%
\pgfpathlineto{\pgfqpoint{0.000000in}{-0.048611in}}%
\pgfusepath{stroke,fill}%
}%
\begin{pgfscope}%
\pgfsys@transformshift{3.084565in}{0.417642in}%
\pgfsys@useobject{currentmarker}{}%
\end{pgfscope}%
\end{pgfscope}%
\begin{pgfscope}%
\definecolor{textcolor}{rgb}{0.000000,0.000000,0.000000}%
\pgfsetstrokecolor{textcolor}%
\pgfsetfillcolor{textcolor}%
\pgftext[x=3.084565in,y=0.320420in,,top]{\color{textcolor}\rmfamily\fontsize{8.000000}{9.600000}\selectfont \(\displaystyle {10^{1}}\)}%
\end{pgfscope}%
\begin{pgfscope}%
\pgfpathrectangle{\pgfqpoint{0.594525in}{0.417642in}}{\pgfqpoint{3.423805in}{2.011535in}}%
\pgfusepath{clip}%
\pgfsetrectcap%
\pgfsetroundjoin%
\pgfsetlinewidth{0.803000pt}%
\definecolor{currentstroke}{rgb}{0.450000,0.450000,0.450000}%
\pgfsetstrokecolor{currentstroke}%
\pgfsetdash{}{0pt}%
\pgfpathmoveto{\pgfqpoint{3.862702in}{0.417642in}}%
\pgfpathlineto{\pgfqpoint{3.862702in}{2.429177in}}%
\pgfusepath{stroke}%
\end{pgfscope}%
\begin{pgfscope}%
\pgfsetbuttcap%
\pgfsetroundjoin%
\definecolor{currentfill}{rgb}{0.000000,0.000000,0.000000}%
\pgfsetfillcolor{currentfill}%
\pgfsetlinewidth{0.803000pt}%
\definecolor{currentstroke}{rgb}{0.000000,0.000000,0.000000}%
\pgfsetstrokecolor{currentstroke}%
\pgfsetdash{}{0pt}%
\pgfsys@defobject{currentmarker}{\pgfqpoint{0.000000in}{-0.048611in}}{\pgfqpoint{0.000000in}{0.000000in}}{%
\pgfpathmoveto{\pgfqpoint{0.000000in}{0.000000in}}%
\pgfpathlineto{\pgfqpoint{0.000000in}{-0.048611in}}%
\pgfusepath{stroke,fill}%
}%
\begin{pgfscope}%
\pgfsys@transformshift{3.862702in}{0.417642in}%
\pgfsys@useobject{currentmarker}{}%
\end{pgfscope}%
\end{pgfscope}%
\begin{pgfscope}%
\definecolor{textcolor}{rgb}{0.000000,0.000000,0.000000}%
\pgfsetstrokecolor{textcolor}%
\pgfsetfillcolor{textcolor}%
\pgftext[x=3.862702in,y=0.320420in,,top]{\color{textcolor}\rmfamily\fontsize{8.000000}{9.600000}\selectfont \(\displaystyle {10^{2}}\)}%
\end{pgfscope}%
\begin{pgfscope}%
\pgfpathrectangle{\pgfqpoint{0.594525in}{0.417642in}}{\pgfqpoint{3.423805in}{2.011535in}}%
\pgfusepath{clip}%
\pgfsetrectcap%
\pgfsetroundjoin%
\pgfsetlinewidth{0.803000pt}%
\definecolor{currentstroke}{rgb}{0.850000,0.850000,0.850000}%
\pgfsetstrokecolor{currentstroke}%
\pgfsetdash{}{0pt}%
\pgfpathmoveto{\pgfqpoint{0.629617in}{0.417642in}}%
\pgfpathlineto{\pgfqpoint{0.629617in}{2.429177in}}%
\pgfusepath{stroke}%
\end{pgfscope}%
\begin{pgfscope}%
\pgfsetbuttcap%
\pgfsetroundjoin%
\definecolor{currentfill}{rgb}{0.000000,0.000000,0.000000}%
\pgfsetfillcolor{currentfill}%
\pgfsetlinewidth{0.602250pt}%
\definecolor{currentstroke}{rgb}{0.000000,0.000000,0.000000}%
\pgfsetstrokecolor{currentstroke}%
\pgfsetdash{}{0pt}%
\pgfsys@defobject{currentmarker}{\pgfqpoint{0.000000in}{-0.027778in}}{\pgfqpoint{0.000000in}{0.000000in}}{%
\pgfpathmoveto{\pgfqpoint{0.000000in}{0.000000in}}%
\pgfpathlineto{\pgfqpoint{0.000000in}{-0.027778in}}%
\pgfusepath{stroke,fill}%
}%
\begin{pgfscope}%
\pgfsys@transformshift{0.629617in}{0.417642in}%
\pgfsys@useobject{currentmarker}{}%
\end{pgfscope}%
\end{pgfscope}%
\begin{pgfscope}%
\pgfpathrectangle{\pgfqpoint{0.594525in}{0.417642in}}{\pgfqpoint{3.423805in}{2.011535in}}%
\pgfusepath{clip}%
\pgfsetrectcap%
\pgfsetroundjoin%
\pgfsetlinewidth{0.803000pt}%
\definecolor{currentstroke}{rgb}{0.850000,0.850000,0.850000}%
\pgfsetstrokecolor{currentstroke}%
\pgfsetdash{}{0pt}%
\pgfpathmoveto{\pgfqpoint{0.674743in}{0.417642in}}%
\pgfpathlineto{\pgfqpoint{0.674743in}{2.429177in}}%
\pgfusepath{stroke}%
\end{pgfscope}%
\begin{pgfscope}%
\pgfsetbuttcap%
\pgfsetroundjoin%
\definecolor{currentfill}{rgb}{0.000000,0.000000,0.000000}%
\pgfsetfillcolor{currentfill}%
\pgfsetlinewidth{0.602250pt}%
\definecolor{currentstroke}{rgb}{0.000000,0.000000,0.000000}%
\pgfsetstrokecolor{currentstroke}%
\pgfsetdash{}{0pt}%
\pgfsys@defobject{currentmarker}{\pgfqpoint{0.000000in}{-0.027778in}}{\pgfqpoint{0.000000in}{0.000000in}}{%
\pgfpathmoveto{\pgfqpoint{0.000000in}{0.000000in}}%
\pgfpathlineto{\pgfqpoint{0.000000in}{-0.027778in}}%
\pgfusepath{stroke,fill}%
}%
\begin{pgfscope}%
\pgfsys@transformshift{0.674743in}{0.417642in}%
\pgfsys@useobject{currentmarker}{}%
\end{pgfscope}%
\end{pgfscope}%
\begin{pgfscope}%
\pgfpathrectangle{\pgfqpoint{0.594525in}{0.417642in}}{\pgfqpoint{3.423805in}{2.011535in}}%
\pgfusepath{clip}%
\pgfsetrectcap%
\pgfsetroundjoin%
\pgfsetlinewidth{0.803000pt}%
\definecolor{currentstroke}{rgb}{0.850000,0.850000,0.850000}%
\pgfsetstrokecolor{currentstroke}%
\pgfsetdash{}{0pt}%
\pgfpathmoveto{\pgfqpoint{0.714547in}{0.417642in}}%
\pgfpathlineto{\pgfqpoint{0.714547in}{2.429177in}}%
\pgfusepath{stroke}%
\end{pgfscope}%
\begin{pgfscope}%
\pgfsetbuttcap%
\pgfsetroundjoin%
\definecolor{currentfill}{rgb}{0.000000,0.000000,0.000000}%
\pgfsetfillcolor{currentfill}%
\pgfsetlinewidth{0.602250pt}%
\definecolor{currentstroke}{rgb}{0.000000,0.000000,0.000000}%
\pgfsetstrokecolor{currentstroke}%
\pgfsetdash{}{0pt}%
\pgfsys@defobject{currentmarker}{\pgfqpoint{0.000000in}{-0.027778in}}{\pgfqpoint{0.000000in}{0.000000in}}{%
\pgfpathmoveto{\pgfqpoint{0.000000in}{0.000000in}}%
\pgfpathlineto{\pgfqpoint{0.000000in}{-0.027778in}}%
\pgfusepath{stroke,fill}%
}%
\begin{pgfscope}%
\pgfsys@transformshift{0.714547in}{0.417642in}%
\pgfsys@useobject{currentmarker}{}%
\end{pgfscope}%
\end{pgfscope}%
\begin{pgfscope}%
\pgfpathrectangle{\pgfqpoint{0.594525in}{0.417642in}}{\pgfqpoint{3.423805in}{2.011535in}}%
\pgfusepath{clip}%
\pgfsetrectcap%
\pgfsetroundjoin%
\pgfsetlinewidth{0.803000pt}%
\definecolor{currentstroke}{rgb}{0.850000,0.850000,0.850000}%
\pgfsetstrokecolor{currentstroke}%
\pgfsetdash{}{0pt}%
\pgfpathmoveto{\pgfqpoint{0.984395in}{0.417642in}}%
\pgfpathlineto{\pgfqpoint{0.984395in}{2.429177in}}%
\pgfusepath{stroke}%
\end{pgfscope}%
\begin{pgfscope}%
\pgfsetbuttcap%
\pgfsetroundjoin%
\definecolor{currentfill}{rgb}{0.000000,0.000000,0.000000}%
\pgfsetfillcolor{currentfill}%
\pgfsetlinewidth{0.602250pt}%
\definecolor{currentstroke}{rgb}{0.000000,0.000000,0.000000}%
\pgfsetstrokecolor{currentstroke}%
\pgfsetdash{}{0pt}%
\pgfsys@defobject{currentmarker}{\pgfqpoint{0.000000in}{-0.027778in}}{\pgfqpoint{0.000000in}{0.000000in}}{%
\pgfpathmoveto{\pgfqpoint{0.000000in}{0.000000in}}%
\pgfpathlineto{\pgfqpoint{0.000000in}{-0.027778in}}%
\pgfusepath{stroke,fill}%
}%
\begin{pgfscope}%
\pgfsys@transformshift{0.984395in}{0.417642in}%
\pgfsys@useobject{currentmarker}{}%
\end{pgfscope}%
\end{pgfscope}%
\begin{pgfscope}%
\pgfpathrectangle{\pgfqpoint{0.594525in}{0.417642in}}{\pgfqpoint{3.423805in}{2.011535in}}%
\pgfusepath{clip}%
\pgfsetrectcap%
\pgfsetroundjoin%
\pgfsetlinewidth{0.803000pt}%
\definecolor{currentstroke}{rgb}{0.850000,0.850000,0.850000}%
\pgfsetstrokecolor{currentstroke}%
\pgfsetdash{}{0pt}%
\pgfpathmoveto{\pgfqpoint{1.121418in}{0.417642in}}%
\pgfpathlineto{\pgfqpoint{1.121418in}{2.429177in}}%
\pgfusepath{stroke}%
\end{pgfscope}%
\begin{pgfscope}%
\pgfsetbuttcap%
\pgfsetroundjoin%
\definecolor{currentfill}{rgb}{0.000000,0.000000,0.000000}%
\pgfsetfillcolor{currentfill}%
\pgfsetlinewidth{0.602250pt}%
\definecolor{currentstroke}{rgb}{0.000000,0.000000,0.000000}%
\pgfsetstrokecolor{currentstroke}%
\pgfsetdash{}{0pt}%
\pgfsys@defobject{currentmarker}{\pgfqpoint{0.000000in}{-0.027778in}}{\pgfqpoint{0.000000in}{0.000000in}}{%
\pgfpathmoveto{\pgfqpoint{0.000000in}{0.000000in}}%
\pgfpathlineto{\pgfqpoint{0.000000in}{-0.027778in}}%
\pgfusepath{stroke,fill}%
}%
\begin{pgfscope}%
\pgfsys@transformshift{1.121418in}{0.417642in}%
\pgfsys@useobject{currentmarker}{}%
\end{pgfscope}%
\end{pgfscope}%
\begin{pgfscope}%
\pgfpathrectangle{\pgfqpoint{0.594525in}{0.417642in}}{\pgfqpoint{3.423805in}{2.011535in}}%
\pgfusepath{clip}%
\pgfsetrectcap%
\pgfsetroundjoin%
\pgfsetlinewidth{0.803000pt}%
\definecolor{currentstroke}{rgb}{0.850000,0.850000,0.850000}%
\pgfsetstrokecolor{currentstroke}%
\pgfsetdash{}{0pt}%
\pgfpathmoveto{\pgfqpoint{1.218638in}{0.417642in}}%
\pgfpathlineto{\pgfqpoint{1.218638in}{2.429177in}}%
\pgfusepath{stroke}%
\end{pgfscope}%
\begin{pgfscope}%
\pgfsetbuttcap%
\pgfsetroundjoin%
\definecolor{currentfill}{rgb}{0.000000,0.000000,0.000000}%
\pgfsetfillcolor{currentfill}%
\pgfsetlinewidth{0.602250pt}%
\definecolor{currentstroke}{rgb}{0.000000,0.000000,0.000000}%
\pgfsetstrokecolor{currentstroke}%
\pgfsetdash{}{0pt}%
\pgfsys@defobject{currentmarker}{\pgfqpoint{0.000000in}{-0.027778in}}{\pgfqpoint{0.000000in}{0.000000in}}{%
\pgfpathmoveto{\pgfqpoint{0.000000in}{0.000000in}}%
\pgfpathlineto{\pgfqpoint{0.000000in}{-0.027778in}}%
\pgfusepath{stroke,fill}%
}%
\begin{pgfscope}%
\pgfsys@transformshift{1.218638in}{0.417642in}%
\pgfsys@useobject{currentmarker}{}%
\end{pgfscope}%
\end{pgfscope}%
\begin{pgfscope}%
\pgfpathrectangle{\pgfqpoint{0.594525in}{0.417642in}}{\pgfqpoint{3.423805in}{2.011535in}}%
\pgfusepath{clip}%
\pgfsetrectcap%
\pgfsetroundjoin%
\pgfsetlinewidth{0.803000pt}%
\definecolor{currentstroke}{rgb}{0.850000,0.850000,0.850000}%
\pgfsetstrokecolor{currentstroke}%
\pgfsetdash{}{0pt}%
\pgfpathmoveto{\pgfqpoint{1.294047in}{0.417642in}}%
\pgfpathlineto{\pgfqpoint{1.294047in}{2.429177in}}%
\pgfusepath{stroke}%
\end{pgfscope}%
\begin{pgfscope}%
\pgfsetbuttcap%
\pgfsetroundjoin%
\definecolor{currentfill}{rgb}{0.000000,0.000000,0.000000}%
\pgfsetfillcolor{currentfill}%
\pgfsetlinewidth{0.602250pt}%
\definecolor{currentstroke}{rgb}{0.000000,0.000000,0.000000}%
\pgfsetstrokecolor{currentstroke}%
\pgfsetdash{}{0pt}%
\pgfsys@defobject{currentmarker}{\pgfqpoint{0.000000in}{-0.027778in}}{\pgfqpoint{0.000000in}{0.000000in}}{%
\pgfpathmoveto{\pgfqpoint{0.000000in}{0.000000in}}%
\pgfpathlineto{\pgfqpoint{0.000000in}{-0.027778in}}%
\pgfusepath{stroke,fill}%
}%
\begin{pgfscope}%
\pgfsys@transformshift{1.294047in}{0.417642in}%
\pgfsys@useobject{currentmarker}{}%
\end{pgfscope}%
\end{pgfscope}%
\begin{pgfscope}%
\pgfpathrectangle{\pgfqpoint{0.594525in}{0.417642in}}{\pgfqpoint{3.423805in}{2.011535in}}%
\pgfusepath{clip}%
\pgfsetrectcap%
\pgfsetroundjoin%
\pgfsetlinewidth{0.803000pt}%
\definecolor{currentstroke}{rgb}{0.850000,0.850000,0.850000}%
\pgfsetstrokecolor{currentstroke}%
\pgfsetdash{}{0pt}%
\pgfpathmoveto{\pgfqpoint{1.355661in}{0.417642in}}%
\pgfpathlineto{\pgfqpoint{1.355661in}{2.429177in}}%
\pgfusepath{stroke}%
\end{pgfscope}%
\begin{pgfscope}%
\pgfsetbuttcap%
\pgfsetroundjoin%
\definecolor{currentfill}{rgb}{0.000000,0.000000,0.000000}%
\pgfsetfillcolor{currentfill}%
\pgfsetlinewidth{0.602250pt}%
\definecolor{currentstroke}{rgb}{0.000000,0.000000,0.000000}%
\pgfsetstrokecolor{currentstroke}%
\pgfsetdash{}{0pt}%
\pgfsys@defobject{currentmarker}{\pgfqpoint{0.000000in}{-0.027778in}}{\pgfqpoint{0.000000in}{0.000000in}}{%
\pgfpathmoveto{\pgfqpoint{0.000000in}{0.000000in}}%
\pgfpathlineto{\pgfqpoint{0.000000in}{-0.027778in}}%
\pgfusepath{stroke,fill}%
}%
\begin{pgfscope}%
\pgfsys@transformshift{1.355661in}{0.417642in}%
\pgfsys@useobject{currentmarker}{}%
\end{pgfscope}%
\end{pgfscope}%
\begin{pgfscope}%
\pgfpathrectangle{\pgfqpoint{0.594525in}{0.417642in}}{\pgfqpoint{3.423805in}{2.011535in}}%
\pgfusepath{clip}%
\pgfsetrectcap%
\pgfsetroundjoin%
\pgfsetlinewidth{0.803000pt}%
\definecolor{currentstroke}{rgb}{0.850000,0.850000,0.850000}%
\pgfsetstrokecolor{currentstroke}%
\pgfsetdash{}{0pt}%
\pgfpathmoveto{\pgfqpoint{1.407755in}{0.417642in}}%
\pgfpathlineto{\pgfqpoint{1.407755in}{2.429177in}}%
\pgfusepath{stroke}%
\end{pgfscope}%
\begin{pgfscope}%
\pgfsetbuttcap%
\pgfsetroundjoin%
\definecolor{currentfill}{rgb}{0.000000,0.000000,0.000000}%
\pgfsetfillcolor{currentfill}%
\pgfsetlinewidth{0.602250pt}%
\definecolor{currentstroke}{rgb}{0.000000,0.000000,0.000000}%
\pgfsetstrokecolor{currentstroke}%
\pgfsetdash{}{0pt}%
\pgfsys@defobject{currentmarker}{\pgfqpoint{0.000000in}{-0.027778in}}{\pgfqpoint{0.000000in}{0.000000in}}{%
\pgfpathmoveto{\pgfqpoint{0.000000in}{0.000000in}}%
\pgfpathlineto{\pgfqpoint{0.000000in}{-0.027778in}}%
\pgfusepath{stroke,fill}%
}%
\begin{pgfscope}%
\pgfsys@transformshift{1.407755in}{0.417642in}%
\pgfsys@useobject{currentmarker}{}%
\end{pgfscope}%
\end{pgfscope}%
\begin{pgfscope}%
\pgfpathrectangle{\pgfqpoint{0.594525in}{0.417642in}}{\pgfqpoint{3.423805in}{2.011535in}}%
\pgfusepath{clip}%
\pgfsetrectcap%
\pgfsetroundjoin%
\pgfsetlinewidth{0.803000pt}%
\definecolor{currentstroke}{rgb}{0.850000,0.850000,0.850000}%
\pgfsetstrokecolor{currentstroke}%
\pgfsetdash{}{0pt}%
\pgfpathmoveto{\pgfqpoint{1.452880in}{0.417642in}}%
\pgfpathlineto{\pgfqpoint{1.452880in}{2.429177in}}%
\pgfusepath{stroke}%
\end{pgfscope}%
\begin{pgfscope}%
\pgfsetbuttcap%
\pgfsetroundjoin%
\definecolor{currentfill}{rgb}{0.000000,0.000000,0.000000}%
\pgfsetfillcolor{currentfill}%
\pgfsetlinewidth{0.602250pt}%
\definecolor{currentstroke}{rgb}{0.000000,0.000000,0.000000}%
\pgfsetstrokecolor{currentstroke}%
\pgfsetdash{}{0pt}%
\pgfsys@defobject{currentmarker}{\pgfqpoint{0.000000in}{-0.027778in}}{\pgfqpoint{0.000000in}{0.000000in}}{%
\pgfpathmoveto{\pgfqpoint{0.000000in}{0.000000in}}%
\pgfpathlineto{\pgfqpoint{0.000000in}{-0.027778in}}%
\pgfusepath{stroke,fill}%
}%
\begin{pgfscope}%
\pgfsys@transformshift{1.452880in}{0.417642in}%
\pgfsys@useobject{currentmarker}{}%
\end{pgfscope}%
\end{pgfscope}%
\begin{pgfscope}%
\pgfpathrectangle{\pgfqpoint{0.594525in}{0.417642in}}{\pgfqpoint{3.423805in}{2.011535in}}%
\pgfusepath{clip}%
\pgfsetrectcap%
\pgfsetroundjoin%
\pgfsetlinewidth{0.803000pt}%
\definecolor{currentstroke}{rgb}{0.850000,0.850000,0.850000}%
\pgfsetstrokecolor{currentstroke}%
\pgfsetdash{}{0pt}%
\pgfpathmoveto{\pgfqpoint{1.492684in}{0.417642in}}%
\pgfpathlineto{\pgfqpoint{1.492684in}{2.429177in}}%
\pgfusepath{stroke}%
\end{pgfscope}%
\begin{pgfscope}%
\pgfsetbuttcap%
\pgfsetroundjoin%
\definecolor{currentfill}{rgb}{0.000000,0.000000,0.000000}%
\pgfsetfillcolor{currentfill}%
\pgfsetlinewidth{0.602250pt}%
\definecolor{currentstroke}{rgb}{0.000000,0.000000,0.000000}%
\pgfsetstrokecolor{currentstroke}%
\pgfsetdash{}{0pt}%
\pgfsys@defobject{currentmarker}{\pgfqpoint{0.000000in}{-0.027778in}}{\pgfqpoint{0.000000in}{0.000000in}}{%
\pgfpathmoveto{\pgfqpoint{0.000000in}{0.000000in}}%
\pgfpathlineto{\pgfqpoint{0.000000in}{-0.027778in}}%
\pgfusepath{stroke,fill}%
}%
\begin{pgfscope}%
\pgfsys@transformshift{1.492684in}{0.417642in}%
\pgfsys@useobject{currentmarker}{}%
\end{pgfscope}%
\end{pgfscope}%
\begin{pgfscope}%
\pgfpathrectangle{\pgfqpoint{0.594525in}{0.417642in}}{\pgfqpoint{3.423805in}{2.011535in}}%
\pgfusepath{clip}%
\pgfsetrectcap%
\pgfsetroundjoin%
\pgfsetlinewidth{0.803000pt}%
\definecolor{currentstroke}{rgb}{0.850000,0.850000,0.850000}%
\pgfsetstrokecolor{currentstroke}%
\pgfsetdash{}{0pt}%
\pgfpathmoveto{\pgfqpoint{1.762533in}{0.417642in}}%
\pgfpathlineto{\pgfqpoint{1.762533in}{2.429177in}}%
\pgfusepath{stroke}%
\end{pgfscope}%
\begin{pgfscope}%
\pgfsetbuttcap%
\pgfsetroundjoin%
\definecolor{currentfill}{rgb}{0.000000,0.000000,0.000000}%
\pgfsetfillcolor{currentfill}%
\pgfsetlinewidth{0.602250pt}%
\definecolor{currentstroke}{rgb}{0.000000,0.000000,0.000000}%
\pgfsetstrokecolor{currentstroke}%
\pgfsetdash{}{0pt}%
\pgfsys@defobject{currentmarker}{\pgfqpoint{0.000000in}{-0.027778in}}{\pgfqpoint{0.000000in}{0.000000in}}{%
\pgfpathmoveto{\pgfqpoint{0.000000in}{0.000000in}}%
\pgfpathlineto{\pgfqpoint{0.000000in}{-0.027778in}}%
\pgfusepath{stroke,fill}%
}%
\begin{pgfscope}%
\pgfsys@transformshift{1.762533in}{0.417642in}%
\pgfsys@useobject{currentmarker}{}%
\end{pgfscope}%
\end{pgfscope}%
\begin{pgfscope}%
\pgfpathrectangle{\pgfqpoint{0.594525in}{0.417642in}}{\pgfqpoint{3.423805in}{2.011535in}}%
\pgfusepath{clip}%
\pgfsetrectcap%
\pgfsetroundjoin%
\pgfsetlinewidth{0.803000pt}%
\definecolor{currentstroke}{rgb}{0.850000,0.850000,0.850000}%
\pgfsetstrokecolor{currentstroke}%
\pgfsetdash{}{0pt}%
\pgfpathmoveto{\pgfqpoint{1.899556in}{0.417642in}}%
\pgfpathlineto{\pgfqpoint{1.899556in}{2.429177in}}%
\pgfusepath{stroke}%
\end{pgfscope}%
\begin{pgfscope}%
\pgfsetbuttcap%
\pgfsetroundjoin%
\definecolor{currentfill}{rgb}{0.000000,0.000000,0.000000}%
\pgfsetfillcolor{currentfill}%
\pgfsetlinewidth{0.602250pt}%
\definecolor{currentstroke}{rgb}{0.000000,0.000000,0.000000}%
\pgfsetstrokecolor{currentstroke}%
\pgfsetdash{}{0pt}%
\pgfsys@defobject{currentmarker}{\pgfqpoint{0.000000in}{-0.027778in}}{\pgfqpoint{0.000000in}{0.000000in}}{%
\pgfpathmoveto{\pgfqpoint{0.000000in}{0.000000in}}%
\pgfpathlineto{\pgfqpoint{0.000000in}{-0.027778in}}%
\pgfusepath{stroke,fill}%
}%
\begin{pgfscope}%
\pgfsys@transformshift{1.899556in}{0.417642in}%
\pgfsys@useobject{currentmarker}{}%
\end{pgfscope}%
\end{pgfscope}%
\begin{pgfscope}%
\pgfpathrectangle{\pgfqpoint{0.594525in}{0.417642in}}{\pgfqpoint{3.423805in}{2.011535in}}%
\pgfusepath{clip}%
\pgfsetrectcap%
\pgfsetroundjoin%
\pgfsetlinewidth{0.803000pt}%
\definecolor{currentstroke}{rgb}{0.850000,0.850000,0.850000}%
\pgfsetstrokecolor{currentstroke}%
\pgfsetdash{}{0pt}%
\pgfpathmoveto{\pgfqpoint{1.996775in}{0.417642in}}%
\pgfpathlineto{\pgfqpoint{1.996775in}{2.429177in}}%
\pgfusepath{stroke}%
\end{pgfscope}%
\begin{pgfscope}%
\pgfsetbuttcap%
\pgfsetroundjoin%
\definecolor{currentfill}{rgb}{0.000000,0.000000,0.000000}%
\pgfsetfillcolor{currentfill}%
\pgfsetlinewidth{0.602250pt}%
\definecolor{currentstroke}{rgb}{0.000000,0.000000,0.000000}%
\pgfsetstrokecolor{currentstroke}%
\pgfsetdash{}{0pt}%
\pgfsys@defobject{currentmarker}{\pgfqpoint{0.000000in}{-0.027778in}}{\pgfqpoint{0.000000in}{0.000000in}}{%
\pgfpathmoveto{\pgfqpoint{0.000000in}{0.000000in}}%
\pgfpathlineto{\pgfqpoint{0.000000in}{-0.027778in}}%
\pgfusepath{stroke,fill}%
}%
\begin{pgfscope}%
\pgfsys@transformshift{1.996775in}{0.417642in}%
\pgfsys@useobject{currentmarker}{}%
\end{pgfscope}%
\end{pgfscope}%
\begin{pgfscope}%
\pgfpathrectangle{\pgfqpoint{0.594525in}{0.417642in}}{\pgfqpoint{3.423805in}{2.011535in}}%
\pgfusepath{clip}%
\pgfsetrectcap%
\pgfsetroundjoin%
\pgfsetlinewidth{0.803000pt}%
\definecolor{currentstroke}{rgb}{0.850000,0.850000,0.850000}%
\pgfsetstrokecolor{currentstroke}%
\pgfsetdash{}{0pt}%
\pgfpathmoveto{\pgfqpoint{2.072185in}{0.417642in}}%
\pgfpathlineto{\pgfqpoint{2.072185in}{2.429177in}}%
\pgfusepath{stroke}%
\end{pgfscope}%
\begin{pgfscope}%
\pgfsetbuttcap%
\pgfsetroundjoin%
\definecolor{currentfill}{rgb}{0.000000,0.000000,0.000000}%
\pgfsetfillcolor{currentfill}%
\pgfsetlinewidth{0.602250pt}%
\definecolor{currentstroke}{rgb}{0.000000,0.000000,0.000000}%
\pgfsetstrokecolor{currentstroke}%
\pgfsetdash{}{0pt}%
\pgfsys@defobject{currentmarker}{\pgfqpoint{0.000000in}{-0.027778in}}{\pgfqpoint{0.000000in}{0.000000in}}{%
\pgfpathmoveto{\pgfqpoint{0.000000in}{0.000000in}}%
\pgfpathlineto{\pgfqpoint{0.000000in}{-0.027778in}}%
\pgfusepath{stroke,fill}%
}%
\begin{pgfscope}%
\pgfsys@transformshift{2.072185in}{0.417642in}%
\pgfsys@useobject{currentmarker}{}%
\end{pgfscope}%
\end{pgfscope}%
\begin{pgfscope}%
\pgfpathrectangle{\pgfqpoint{0.594525in}{0.417642in}}{\pgfqpoint{3.423805in}{2.011535in}}%
\pgfusepath{clip}%
\pgfsetrectcap%
\pgfsetroundjoin%
\pgfsetlinewidth{0.803000pt}%
\definecolor{currentstroke}{rgb}{0.850000,0.850000,0.850000}%
\pgfsetstrokecolor{currentstroke}%
\pgfsetdash{}{0pt}%
\pgfpathmoveto{\pgfqpoint{2.133799in}{0.417642in}}%
\pgfpathlineto{\pgfqpoint{2.133799in}{2.429177in}}%
\pgfusepath{stroke}%
\end{pgfscope}%
\begin{pgfscope}%
\pgfsetbuttcap%
\pgfsetroundjoin%
\definecolor{currentfill}{rgb}{0.000000,0.000000,0.000000}%
\pgfsetfillcolor{currentfill}%
\pgfsetlinewidth{0.602250pt}%
\definecolor{currentstroke}{rgb}{0.000000,0.000000,0.000000}%
\pgfsetstrokecolor{currentstroke}%
\pgfsetdash{}{0pt}%
\pgfsys@defobject{currentmarker}{\pgfqpoint{0.000000in}{-0.027778in}}{\pgfqpoint{0.000000in}{0.000000in}}{%
\pgfpathmoveto{\pgfqpoint{0.000000in}{0.000000in}}%
\pgfpathlineto{\pgfqpoint{0.000000in}{-0.027778in}}%
\pgfusepath{stroke,fill}%
}%
\begin{pgfscope}%
\pgfsys@transformshift{2.133799in}{0.417642in}%
\pgfsys@useobject{currentmarker}{}%
\end{pgfscope}%
\end{pgfscope}%
\begin{pgfscope}%
\pgfpathrectangle{\pgfqpoint{0.594525in}{0.417642in}}{\pgfqpoint{3.423805in}{2.011535in}}%
\pgfusepath{clip}%
\pgfsetrectcap%
\pgfsetroundjoin%
\pgfsetlinewidth{0.803000pt}%
\definecolor{currentstroke}{rgb}{0.850000,0.850000,0.850000}%
\pgfsetstrokecolor{currentstroke}%
\pgfsetdash{}{0pt}%
\pgfpathmoveto{\pgfqpoint{2.185892in}{0.417642in}}%
\pgfpathlineto{\pgfqpoint{2.185892in}{2.429177in}}%
\pgfusepath{stroke}%
\end{pgfscope}%
\begin{pgfscope}%
\pgfsetbuttcap%
\pgfsetroundjoin%
\definecolor{currentfill}{rgb}{0.000000,0.000000,0.000000}%
\pgfsetfillcolor{currentfill}%
\pgfsetlinewidth{0.602250pt}%
\definecolor{currentstroke}{rgb}{0.000000,0.000000,0.000000}%
\pgfsetstrokecolor{currentstroke}%
\pgfsetdash{}{0pt}%
\pgfsys@defobject{currentmarker}{\pgfqpoint{0.000000in}{-0.027778in}}{\pgfqpoint{0.000000in}{0.000000in}}{%
\pgfpathmoveto{\pgfqpoint{0.000000in}{0.000000in}}%
\pgfpathlineto{\pgfqpoint{0.000000in}{-0.027778in}}%
\pgfusepath{stroke,fill}%
}%
\begin{pgfscope}%
\pgfsys@transformshift{2.185892in}{0.417642in}%
\pgfsys@useobject{currentmarker}{}%
\end{pgfscope}%
\end{pgfscope}%
\begin{pgfscope}%
\pgfpathrectangle{\pgfqpoint{0.594525in}{0.417642in}}{\pgfqpoint{3.423805in}{2.011535in}}%
\pgfusepath{clip}%
\pgfsetrectcap%
\pgfsetroundjoin%
\pgfsetlinewidth{0.803000pt}%
\definecolor{currentstroke}{rgb}{0.850000,0.850000,0.850000}%
\pgfsetstrokecolor{currentstroke}%
\pgfsetdash{}{0pt}%
\pgfpathmoveto{\pgfqpoint{2.231018in}{0.417642in}}%
\pgfpathlineto{\pgfqpoint{2.231018in}{2.429177in}}%
\pgfusepath{stroke}%
\end{pgfscope}%
\begin{pgfscope}%
\pgfsetbuttcap%
\pgfsetroundjoin%
\definecolor{currentfill}{rgb}{0.000000,0.000000,0.000000}%
\pgfsetfillcolor{currentfill}%
\pgfsetlinewidth{0.602250pt}%
\definecolor{currentstroke}{rgb}{0.000000,0.000000,0.000000}%
\pgfsetstrokecolor{currentstroke}%
\pgfsetdash{}{0pt}%
\pgfsys@defobject{currentmarker}{\pgfqpoint{0.000000in}{-0.027778in}}{\pgfqpoint{0.000000in}{0.000000in}}{%
\pgfpathmoveto{\pgfqpoint{0.000000in}{0.000000in}}%
\pgfpathlineto{\pgfqpoint{0.000000in}{-0.027778in}}%
\pgfusepath{stroke,fill}%
}%
\begin{pgfscope}%
\pgfsys@transformshift{2.231018in}{0.417642in}%
\pgfsys@useobject{currentmarker}{}%
\end{pgfscope}%
\end{pgfscope}%
\begin{pgfscope}%
\pgfpathrectangle{\pgfqpoint{0.594525in}{0.417642in}}{\pgfqpoint{3.423805in}{2.011535in}}%
\pgfusepath{clip}%
\pgfsetrectcap%
\pgfsetroundjoin%
\pgfsetlinewidth{0.803000pt}%
\definecolor{currentstroke}{rgb}{0.850000,0.850000,0.850000}%
\pgfsetstrokecolor{currentstroke}%
\pgfsetdash{}{0pt}%
\pgfpathmoveto{\pgfqpoint{2.270822in}{0.417642in}}%
\pgfpathlineto{\pgfqpoint{2.270822in}{2.429177in}}%
\pgfusepath{stroke}%
\end{pgfscope}%
\begin{pgfscope}%
\pgfsetbuttcap%
\pgfsetroundjoin%
\definecolor{currentfill}{rgb}{0.000000,0.000000,0.000000}%
\pgfsetfillcolor{currentfill}%
\pgfsetlinewidth{0.602250pt}%
\definecolor{currentstroke}{rgb}{0.000000,0.000000,0.000000}%
\pgfsetstrokecolor{currentstroke}%
\pgfsetdash{}{0pt}%
\pgfsys@defobject{currentmarker}{\pgfqpoint{0.000000in}{-0.027778in}}{\pgfqpoint{0.000000in}{0.000000in}}{%
\pgfpathmoveto{\pgfqpoint{0.000000in}{0.000000in}}%
\pgfpathlineto{\pgfqpoint{0.000000in}{-0.027778in}}%
\pgfusepath{stroke,fill}%
}%
\begin{pgfscope}%
\pgfsys@transformshift{2.270822in}{0.417642in}%
\pgfsys@useobject{currentmarker}{}%
\end{pgfscope}%
\end{pgfscope}%
\begin{pgfscope}%
\pgfpathrectangle{\pgfqpoint{0.594525in}{0.417642in}}{\pgfqpoint{3.423805in}{2.011535in}}%
\pgfusepath{clip}%
\pgfsetrectcap%
\pgfsetroundjoin%
\pgfsetlinewidth{0.803000pt}%
\definecolor{currentstroke}{rgb}{0.850000,0.850000,0.850000}%
\pgfsetstrokecolor{currentstroke}%
\pgfsetdash{}{0pt}%
\pgfpathmoveto{\pgfqpoint{2.540670in}{0.417642in}}%
\pgfpathlineto{\pgfqpoint{2.540670in}{2.429177in}}%
\pgfusepath{stroke}%
\end{pgfscope}%
\begin{pgfscope}%
\pgfsetbuttcap%
\pgfsetroundjoin%
\definecolor{currentfill}{rgb}{0.000000,0.000000,0.000000}%
\pgfsetfillcolor{currentfill}%
\pgfsetlinewidth{0.602250pt}%
\definecolor{currentstroke}{rgb}{0.000000,0.000000,0.000000}%
\pgfsetstrokecolor{currentstroke}%
\pgfsetdash{}{0pt}%
\pgfsys@defobject{currentmarker}{\pgfqpoint{0.000000in}{-0.027778in}}{\pgfqpoint{0.000000in}{0.000000in}}{%
\pgfpathmoveto{\pgfqpoint{0.000000in}{0.000000in}}%
\pgfpathlineto{\pgfqpoint{0.000000in}{-0.027778in}}%
\pgfusepath{stroke,fill}%
}%
\begin{pgfscope}%
\pgfsys@transformshift{2.540670in}{0.417642in}%
\pgfsys@useobject{currentmarker}{}%
\end{pgfscope}%
\end{pgfscope}%
\begin{pgfscope}%
\pgfpathrectangle{\pgfqpoint{0.594525in}{0.417642in}}{\pgfqpoint{3.423805in}{2.011535in}}%
\pgfusepath{clip}%
\pgfsetrectcap%
\pgfsetroundjoin%
\pgfsetlinewidth{0.803000pt}%
\definecolor{currentstroke}{rgb}{0.850000,0.850000,0.850000}%
\pgfsetstrokecolor{currentstroke}%
\pgfsetdash{}{0pt}%
\pgfpathmoveto{\pgfqpoint{2.677693in}{0.417642in}}%
\pgfpathlineto{\pgfqpoint{2.677693in}{2.429177in}}%
\pgfusepath{stroke}%
\end{pgfscope}%
\begin{pgfscope}%
\pgfsetbuttcap%
\pgfsetroundjoin%
\definecolor{currentfill}{rgb}{0.000000,0.000000,0.000000}%
\pgfsetfillcolor{currentfill}%
\pgfsetlinewidth{0.602250pt}%
\definecolor{currentstroke}{rgb}{0.000000,0.000000,0.000000}%
\pgfsetstrokecolor{currentstroke}%
\pgfsetdash{}{0pt}%
\pgfsys@defobject{currentmarker}{\pgfqpoint{0.000000in}{-0.027778in}}{\pgfqpoint{0.000000in}{0.000000in}}{%
\pgfpathmoveto{\pgfqpoint{0.000000in}{0.000000in}}%
\pgfpathlineto{\pgfqpoint{0.000000in}{-0.027778in}}%
\pgfusepath{stroke,fill}%
}%
\begin{pgfscope}%
\pgfsys@transformshift{2.677693in}{0.417642in}%
\pgfsys@useobject{currentmarker}{}%
\end{pgfscope}%
\end{pgfscope}%
\begin{pgfscope}%
\pgfpathrectangle{\pgfqpoint{0.594525in}{0.417642in}}{\pgfqpoint{3.423805in}{2.011535in}}%
\pgfusepath{clip}%
\pgfsetrectcap%
\pgfsetroundjoin%
\pgfsetlinewidth{0.803000pt}%
\definecolor{currentstroke}{rgb}{0.850000,0.850000,0.850000}%
\pgfsetstrokecolor{currentstroke}%
\pgfsetdash{}{0pt}%
\pgfpathmoveto{\pgfqpoint{2.774913in}{0.417642in}}%
\pgfpathlineto{\pgfqpoint{2.774913in}{2.429177in}}%
\pgfusepath{stroke}%
\end{pgfscope}%
\begin{pgfscope}%
\pgfsetbuttcap%
\pgfsetroundjoin%
\definecolor{currentfill}{rgb}{0.000000,0.000000,0.000000}%
\pgfsetfillcolor{currentfill}%
\pgfsetlinewidth{0.602250pt}%
\definecolor{currentstroke}{rgb}{0.000000,0.000000,0.000000}%
\pgfsetstrokecolor{currentstroke}%
\pgfsetdash{}{0pt}%
\pgfsys@defobject{currentmarker}{\pgfqpoint{0.000000in}{-0.027778in}}{\pgfqpoint{0.000000in}{0.000000in}}{%
\pgfpathmoveto{\pgfqpoint{0.000000in}{0.000000in}}%
\pgfpathlineto{\pgfqpoint{0.000000in}{-0.027778in}}%
\pgfusepath{stroke,fill}%
}%
\begin{pgfscope}%
\pgfsys@transformshift{2.774913in}{0.417642in}%
\pgfsys@useobject{currentmarker}{}%
\end{pgfscope}%
\end{pgfscope}%
\begin{pgfscope}%
\pgfpathrectangle{\pgfqpoint{0.594525in}{0.417642in}}{\pgfqpoint{3.423805in}{2.011535in}}%
\pgfusepath{clip}%
\pgfsetrectcap%
\pgfsetroundjoin%
\pgfsetlinewidth{0.803000pt}%
\definecolor{currentstroke}{rgb}{0.850000,0.850000,0.850000}%
\pgfsetstrokecolor{currentstroke}%
\pgfsetdash{}{0pt}%
\pgfpathmoveto{\pgfqpoint{2.850322in}{0.417642in}}%
\pgfpathlineto{\pgfqpoint{2.850322in}{2.429177in}}%
\pgfusepath{stroke}%
\end{pgfscope}%
\begin{pgfscope}%
\pgfsetbuttcap%
\pgfsetroundjoin%
\definecolor{currentfill}{rgb}{0.000000,0.000000,0.000000}%
\pgfsetfillcolor{currentfill}%
\pgfsetlinewidth{0.602250pt}%
\definecolor{currentstroke}{rgb}{0.000000,0.000000,0.000000}%
\pgfsetstrokecolor{currentstroke}%
\pgfsetdash{}{0pt}%
\pgfsys@defobject{currentmarker}{\pgfqpoint{0.000000in}{-0.027778in}}{\pgfqpoint{0.000000in}{0.000000in}}{%
\pgfpathmoveto{\pgfqpoint{0.000000in}{0.000000in}}%
\pgfpathlineto{\pgfqpoint{0.000000in}{-0.027778in}}%
\pgfusepath{stroke,fill}%
}%
\begin{pgfscope}%
\pgfsys@transformshift{2.850322in}{0.417642in}%
\pgfsys@useobject{currentmarker}{}%
\end{pgfscope}%
\end{pgfscope}%
\begin{pgfscope}%
\pgfpathrectangle{\pgfqpoint{0.594525in}{0.417642in}}{\pgfqpoint{3.423805in}{2.011535in}}%
\pgfusepath{clip}%
\pgfsetrectcap%
\pgfsetroundjoin%
\pgfsetlinewidth{0.803000pt}%
\definecolor{currentstroke}{rgb}{0.850000,0.850000,0.850000}%
\pgfsetstrokecolor{currentstroke}%
\pgfsetdash{}{0pt}%
\pgfpathmoveto{\pgfqpoint{2.911936in}{0.417642in}}%
\pgfpathlineto{\pgfqpoint{2.911936in}{2.429177in}}%
\pgfusepath{stroke}%
\end{pgfscope}%
\begin{pgfscope}%
\pgfsetbuttcap%
\pgfsetroundjoin%
\definecolor{currentfill}{rgb}{0.000000,0.000000,0.000000}%
\pgfsetfillcolor{currentfill}%
\pgfsetlinewidth{0.602250pt}%
\definecolor{currentstroke}{rgb}{0.000000,0.000000,0.000000}%
\pgfsetstrokecolor{currentstroke}%
\pgfsetdash{}{0pt}%
\pgfsys@defobject{currentmarker}{\pgfqpoint{0.000000in}{-0.027778in}}{\pgfqpoint{0.000000in}{0.000000in}}{%
\pgfpathmoveto{\pgfqpoint{0.000000in}{0.000000in}}%
\pgfpathlineto{\pgfqpoint{0.000000in}{-0.027778in}}%
\pgfusepath{stroke,fill}%
}%
\begin{pgfscope}%
\pgfsys@transformshift{2.911936in}{0.417642in}%
\pgfsys@useobject{currentmarker}{}%
\end{pgfscope}%
\end{pgfscope}%
\begin{pgfscope}%
\pgfpathrectangle{\pgfqpoint{0.594525in}{0.417642in}}{\pgfqpoint{3.423805in}{2.011535in}}%
\pgfusepath{clip}%
\pgfsetrectcap%
\pgfsetroundjoin%
\pgfsetlinewidth{0.803000pt}%
\definecolor{currentstroke}{rgb}{0.850000,0.850000,0.850000}%
\pgfsetstrokecolor{currentstroke}%
\pgfsetdash{}{0pt}%
\pgfpathmoveto{\pgfqpoint{2.964030in}{0.417642in}}%
\pgfpathlineto{\pgfqpoint{2.964030in}{2.429177in}}%
\pgfusepath{stroke}%
\end{pgfscope}%
\begin{pgfscope}%
\pgfsetbuttcap%
\pgfsetroundjoin%
\definecolor{currentfill}{rgb}{0.000000,0.000000,0.000000}%
\pgfsetfillcolor{currentfill}%
\pgfsetlinewidth{0.602250pt}%
\definecolor{currentstroke}{rgb}{0.000000,0.000000,0.000000}%
\pgfsetstrokecolor{currentstroke}%
\pgfsetdash{}{0pt}%
\pgfsys@defobject{currentmarker}{\pgfqpoint{0.000000in}{-0.027778in}}{\pgfqpoint{0.000000in}{0.000000in}}{%
\pgfpathmoveto{\pgfqpoint{0.000000in}{0.000000in}}%
\pgfpathlineto{\pgfqpoint{0.000000in}{-0.027778in}}%
\pgfusepath{stroke,fill}%
}%
\begin{pgfscope}%
\pgfsys@transformshift{2.964030in}{0.417642in}%
\pgfsys@useobject{currentmarker}{}%
\end{pgfscope}%
\end{pgfscope}%
\begin{pgfscope}%
\pgfpathrectangle{\pgfqpoint{0.594525in}{0.417642in}}{\pgfqpoint{3.423805in}{2.011535in}}%
\pgfusepath{clip}%
\pgfsetrectcap%
\pgfsetroundjoin%
\pgfsetlinewidth{0.803000pt}%
\definecolor{currentstroke}{rgb}{0.850000,0.850000,0.850000}%
\pgfsetstrokecolor{currentstroke}%
\pgfsetdash{}{0pt}%
\pgfpathmoveto{\pgfqpoint{3.009156in}{0.417642in}}%
\pgfpathlineto{\pgfqpoint{3.009156in}{2.429177in}}%
\pgfusepath{stroke}%
\end{pgfscope}%
\begin{pgfscope}%
\pgfsetbuttcap%
\pgfsetroundjoin%
\definecolor{currentfill}{rgb}{0.000000,0.000000,0.000000}%
\pgfsetfillcolor{currentfill}%
\pgfsetlinewidth{0.602250pt}%
\definecolor{currentstroke}{rgb}{0.000000,0.000000,0.000000}%
\pgfsetstrokecolor{currentstroke}%
\pgfsetdash{}{0pt}%
\pgfsys@defobject{currentmarker}{\pgfqpoint{0.000000in}{-0.027778in}}{\pgfqpoint{0.000000in}{0.000000in}}{%
\pgfpathmoveto{\pgfqpoint{0.000000in}{0.000000in}}%
\pgfpathlineto{\pgfqpoint{0.000000in}{-0.027778in}}%
\pgfusepath{stroke,fill}%
}%
\begin{pgfscope}%
\pgfsys@transformshift{3.009156in}{0.417642in}%
\pgfsys@useobject{currentmarker}{}%
\end{pgfscope}%
\end{pgfscope}%
\begin{pgfscope}%
\pgfpathrectangle{\pgfqpoint{0.594525in}{0.417642in}}{\pgfqpoint{3.423805in}{2.011535in}}%
\pgfusepath{clip}%
\pgfsetrectcap%
\pgfsetroundjoin%
\pgfsetlinewidth{0.803000pt}%
\definecolor{currentstroke}{rgb}{0.850000,0.850000,0.850000}%
\pgfsetstrokecolor{currentstroke}%
\pgfsetdash{}{0pt}%
\pgfpathmoveto{\pgfqpoint{3.048959in}{0.417642in}}%
\pgfpathlineto{\pgfqpoint{3.048959in}{2.429177in}}%
\pgfusepath{stroke}%
\end{pgfscope}%
\begin{pgfscope}%
\pgfsetbuttcap%
\pgfsetroundjoin%
\definecolor{currentfill}{rgb}{0.000000,0.000000,0.000000}%
\pgfsetfillcolor{currentfill}%
\pgfsetlinewidth{0.602250pt}%
\definecolor{currentstroke}{rgb}{0.000000,0.000000,0.000000}%
\pgfsetstrokecolor{currentstroke}%
\pgfsetdash{}{0pt}%
\pgfsys@defobject{currentmarker}{\pgfqpoint{0.000000in}{-0.027778in}}{\pgfqpoint{0.000000in}{0.000000in}}{%
\pgfpathmoveto{\pgfqpoint{0.000000in}{0.000000in}}%
\pgfpathlineto{\pgfqpoint{0.000000in}{-0.027778in}}%
\pgfusepath{stroke,fill}%
}%
\begin{pgfscope}%
\pgfsys@transformshift{3.048959in}{0.417642in}%
\pgfsys@useobject{currentmarker}{}%
\end{pgfscope}%
\end{pgfscope}%
\begin{pgfscope}%
\pgfpathrectangle{\pgfqpoint{0.594525in}{0.417642in}}{\pgfqpoint{3.423805in}{2.011535in}}%
\pgfusepath{clip}%
\pgfsetrectcap%
\pgfsetroundjoin%
\pgfsetlinewidth{0.803000pt}%
\definecolor{currentstroke}{rgb}{0.850000,0.850000,0.850000}%
\pgfsetstrokecolor{currentstroke}%
\pgfsetdash{}{0pt}%
\pgfpathmoveto{\pgfqpoint{3.318808in}{0.417642in}}%
\pgfpathlineto{\pgfqpoint{3.318808in}{2.429177in}}%
\pgfusepath{stroke}%
\end{pgfscope}%
\begin{pgfscope}%
\pgfsetbuttcap%
\pgfsetroundjoin%
\definecolor{currentfill}{rgb}{0.000000,0.000000,0.000000}%
\pgfsetfillcolor{currentfill}%
\pgfsetlinewidth{0.602250pt}%
\definecolor{currentstroke}{rgb}{0.000000,0.000000,0.000000}%
\pgfsetstrokecolor{currentstroke}%
\pgfsetdash{}{0pt}%
\pgfsys@defobject{currentmarker}{\pgfqpoint{0.000000in}{-0.027778in}}{\pgfqpoint{0.000000in}{0.000000in}}{%
\pgfpathmoveto{\pgfqpoint{0.000000in}{0.000000in}}%
\pgfpathlineto{\pgfqpoint{0.000000in}{-0.027778in}}%
\pgfusepath{stroke,fill}%
}%
\begin{pgfscope}%
\pgfsys@transformshift{3.318808in}{0.417642in}%
\pgfsys@useobject{currentmarker}{}%
\end{pgfscope}%
\end{pgfscope}%
\begin{pgfscope}%
\pgfpathrectangle{\pgfqpoint{0.594525in}{0.417642in}}{\pgfqpoint{3.423805in}{2.011535in}}%
\pgfusepath{clip}%
\pgfsetrectcap%
\pgfsetroundjoin%
\pgfsetlinewidth{0.803000pt}%
\definecolor{currentstroke}{rgb}{0.850000,0.850000,0.850000}%
\pgfsetstrokecolor{currentstroke}%
\pgfsetdash{}{0pt}%
\pgfpathmoveto{\pgfqpoint{3.455831in}{0.417642in}}%
\pgfpathlineto{\pgfqpoint{3.455831in}{2.429177in}}%
\pgfusepath{stroke}%
\end{pgfscope}%
\begin{pgfscope}%
\pgfsetbuttcap%
\pgfsetroundjoin%
\definecolor{currentfill}{rgb}{0.000000,0.000000,0.000000}%
\pgfsetfillcolor{currentfill}%
\pgfsetlinewidth{0.602250pt}%
\definecolor{currentstroke}{rgb}{0.000000,0.000000,0.000000}%
\pgfsetstrokecolor{currentstroke}%
\pgfsetdash{}{0pt}%
\pgfsys@defobject{currentmarker}{\pgfqpoint{0.000000in}{-0.027778in}}{\pgfqpoint{0.000000in}{0.000000in}}{%
\pgfpathmoveto{\pgfqpoint{0.000000in}{0.000000in}}%
\pgfpathlineto{\pgfqpoint{0.000000in}{-0.027778in}}%
\pgfusepath{stroke,fill}%
}%
\begin{pgfscope}%
\pgfsys@transformshift{3.455831in}{0.417642in}%
\pgfsys@useobject{currentmarker}{}%
\end{pgfscope}%
\end{pgfscope}%
\begin{pgfscope}%
\pgfpathrectangle{\pgfqpoint{0.594525in}{0.417642in}}{\pgfqpoint{3.423805in}{2.011535in}}%
\pgfusepath{clip}%
\pgfsetrectcap%
\pgfsetroundjoin%
\pgfsetlinewidth{0.803000pt}%
\definecolor{currentstroke}{rgb}{0.850000,0.850000,0.850000}%
\pgfsetstrokecolor{currentstroke}%
\pgfsetdash{}{0pt}%
\pgfpathmoveto{\pgfqpoint{3.553050in}{0.417642in}}%
\pgfpathlineto{\pgfqpoint{3.553050in}{2.429177in}}%
\pgfusepath{stroke}%
\end{pgfscope}%
\begin{pgfscope}%
\pgfsetbuttcap%
\pgfsetroundjoin%
\definecolor{currentfill}{rgb}{0.000000,0.000000,0.000000}%
\pgfsetfillcolor{currentfill}%
\pgfsetlinewidth{0.602250pt}%
\definecolor{currentstroke}{rgb}{0.000000,0.000000,0.000000}%
\pgfsetstrokecolor{currentstroke}%
\pgfsetdash{}{0pt}%
\pgfsys@defobject{currentmarker}{\pgfqpoint{0.000000in}{-0.027778in}}{\pgfqpoint{0.000000in}{0.000000in}}{%
\pgfpathmoveto{\pgfqpoint{0.000000in}{0.000000in}}%
\pgfpathlineto{\pgfqpoint{0.000000in}{-0.027778in}}%
\pgfusepath{stroke,fill}%
}%
\begin{pgfscope}%
\pgfsys@transformshift{3.553050in}{0.417642in}%
\pgfsys@useobject{currentmarker}{}%
\end{pgfscope}%
\end{pgfscope}%
\begin{pgfscope}%
\pgfpathrectangle{\pgfqpoint{0.594525in}{0.417642in}}{\pgfqpoint{3.423805in}{2.011535in}}%
\pgfusepath{clip}%
\pgfsetrectcap%
\pgfsetroundjoin%
\pgfsetlinewidth{0.803000pt}%
\definecolor{currentstroke}{rgb}{0.850000,0.850000,0.850000}%
\pgfsetstrokecolor{currentstroke}%
\pgfsetdash{}{0pt}%
\pgfpathmoveto{\pgfqpoint{3.628460in}{0.417642in}}%
\pgfpathlineto{\pgfqpoint{3.628460in}{2.429177in}}%
\pgfusepath{stroke}%
\end{pgfscope}%
\begin{pgfscope}%
\pgfsetbuttcap%
\pgfsetroundjoin%
\definecolor{currentfill}{rgb}{0.000000,0.000000,0.000000}%
\pgfsetfillcolor{currentfill}%
\pgfsetlinewidth{0.602250pt}%
\definecolor{currentstroke}{rgb}{0.000000,0.000000,0.000000}%
\pgfsetstrokecolor{currentstroke}%
\pgfsetdash{}{0pt}%
\pgfsys@defobject{currentmarker}{\pgfqpoint{0.000000in}{-0.027778in}}{\pgfqpoint{0.000000in}{0.000000in}}{%
\pgfpathmoveto{\pgfqpoint{0.000000in}{0.000000in}}%
\pgfpathlineto{\pgfqpoint{0.000000in}{-0.027778in}}%
\pgfusepath{stroke,fill}%
}%
\begin{pgfscope}%
\pgfsys@transformshift{3.628460in}{0.417642in}%
\pgfsys@useobject{currentmarker}{}%
\end{pgfscope}%
\end{pgfscope}%
\begin{pgfscope}%
\pgfpathrectangle{\pgfqpoint{0.594525in}{0.417642in}}{\pgfqpoint{3.423805in}{2.011535in}}%
\pgfusepath{clip}%
\pgfsetrectcap%
\pgfsetroundjoin%
\pgfsetlinewidth{0.803000pt}%
\definecolor{currentstroke}{rgb}{0.850000,0.850000,0.850000}%
\pgfsetstrokecolor{currentstroke}%
\pgfsetdash{}{0pt}%
\pgfpathmoveto{\pgfqpoint{3.690074in}{0.417642in}}%
\pgfpathlineto{\pgfqpoint{3.690074in}{2.429177in}}%
\pgfusepath{stroke}%
\end{pgfscope}%
\begin{pgfscope}%
\pgfsetbuttcap%
\pgfsetroundjoin%
\definecolor{currentfill}{rgb}{0.000000,0.000000,0.000000}%
\pgfsetfillcolor{currentfill}%
\pgfsetlinewidth{0.602250pt}%
\definecolor{currentstroke}{rgb}{0.000000,0.000000,0.000000}%
\pgfsetstrokecolor{currentstroke}%
\pgfsetdash{}{0pt}%
\pgfsys@defobject{currentmarker}{\pgfqpoint{0.000000in}{-0.027778in}}{\pgfqpoint{0.000000in}{0.000000in}}{%
\pgfpathmoveto{\pgfqpoint{0.000000in}{0.000000in}}%
\pgfpathlineto{\pgfqpoint{0.000000in}{-0.027778in}}%
\pgfusepath{stroke,fill}%
}%
\begin{pgfscope}%
\pgfsys@transformshift{3.690074in}{0.417642in}%
\pgfsys@useobject{currentmarker}{}%
\end{pgfscope}%
\end{pgfscope}%
\begin{pgfscope}%
\pgfpathrectangle{\pgfqpoint{0.594525in}{0.417642in}}{\pgfqpoint{3.423805in}{2.011535in}}%
\pgfusepath{clip}%
\pgfsetrectcap%
\pgfsetroundjoin%
\pgfsetlinewidth{0.803000pt}%
\definecolor{currentstroke}{rgb}{0.850000,0.850000,0.850000}%
\pgfsetstrokecolor{currentstroke}%
\pgfsetdash{}{0pt}%
\pgfpathmoveto{\pgfqpoint{3.742167in}{0.417642in}}%
\pgfpathlineto{\pgfqpoint{3.742167in}{2.429177in}}%
\pgfusepath{stroke}%
\end{pgfscope}%
\begin{pgfscope}%
\pgfsetbuttcap%
\pgfsetroundjoin%
\definecolor{currentfill}{rgb}{0.000000,0.000000,0.000000}%
\pgfsetfillcolor{currentfill}%
\pgfsetlinewidth{0.602250pt}%
\definecolor{currentstroke}{rgb}{0.000000,0.000000,0.000000}%
\pgfsetstrokecolor{currentstroke}%
\pgfsetdash{}{0pt}%
\pgfsys@defobject{currentmarker}{\pgfqpoint{0.000000in}{-0.027778in}}{\pgfqpoint{0.000000in}{0.000000in}}{%
\pgfpathmoveto{\pgfqpoint{0.000000in}{0.000000in}}%
\pgfpathlineto{\pgfqpoint{0.000000in}{-0.027778in}}%
\pgfusepath{stroke,fill}%
}%
\begin{pgfscope}%
\pgfsys@transformshift{3.742167in}{0.417642in}%
\pgfsys@useobject{currentmarker}{}%
\end{pgfscope}%
\end{pgfscope}%
\begin{pgfscope}%
\pgfpathrectangle{\pgfqpoint{0.594525in}{0.417642in}}{\pgfqpoint{3.423805in}{2.011535in}}%
\pgfusepath{clip}%
\pgfsetrectcap%
\pgfsetroundjoin%
\pgfsetlinewidth{0.803000pt}%
\definecolor{currentstroke}{rgb}{0.850000,0.850000,0.850000}%
\pgfsetstrokecolor{currentstroke}%
\pgfsetdash{}{0pt}%
\pgfpathmoveto{\pgfqpoint{3.787293in}{0.417642in}}%
\pgfpathlineto{\pgfqpoint{3.787293in}{2.429177in}}%
\pgfusepath{stroke}%
\end{pgfscope}%
\begin{pgfscope}%
\pgfsetbuttcap%
\pgfsetroundjoin%
\definecolor{currentfill}{rgb}{0.000000,0.000000,0.000000}%
\pgfsetfillcolor{currentfill}%
\pgfsetlinewidth{0.602250pt}%
\definecolor{currentstroke}{rgb}{0.000000,0.000000,0.000000}%
\pgfsetstrokecolor{currentstroke}%
\pgfsetdash{}{0pt}%
\pgfsys@defobject{currentmarker}{\pgfqpoint{0.000000in}{-0.027778in}}{\pgfqpoint{0.000000in}{0.000000in}}{%
\pgfpathmoveto{\pgfqpoint{0.000000in}{0.000000in}}%
\pgfpathlineto{\pgfqpoint{0.000000in}{-0.027778in}}%
\pgfusepath{stroke,fill}%
}%
\begin{pgfscope}%
\pgfsys@transformshift{3.787293in}{0.417642in}%
\pgfsys@useobject{currentmarker}{}%
\end{pgfscope}%
\end{pgfscope}%
\begin{pgfscope}%
\pgfpathrectangle{\pgfqpoint{0.594525in}{0.417642in}}{\pgfqpoint{3.423805in}{2.011535in}}%
\pgfusepath{clip}%
\pgfsetrectcap%
\pgfsetroundjoin%
\pgfsetlinewidth{0.803000pt}%
\definecolor{currentstroke}{rgb}{0.850000,0.850000,0.850000}%
\pgfsetstrokecolor{currentstroke}%
\pgfsetdash{}{0pt}%
\pgfpathmoveto{\pgfqpoint{3.827097in}{0.417642in}}%
\pgfpathlineto{\pgfqpoint{3.827097in}{2.429177in}}%
\pgfusepath{stroke}%
\end{pgfscope}%
\begin{pgfscope}%
\pgfsetbuttcap%
\pgfsetroundjoin%
\definecolor{currentfill}{rgb}{0.000000,0.000000,0.000000}%
\pgfsetfillcolor{currentfill}%
\pgfsetlinewidth{0.602250pt}%
\definecolor{currentstroke}{rgb}{0.000000,0.000000,0.000000}%
\pgfsetstrokecolor{currentstroke}%
\pgfsetdash{}{0pt}%
\pgfsys@defobject{currentmarker}{\pgfqpoint{0.000000in}{-0.027778in}}{\pgfqpoint{0.000000in}{0.000000in}}{%
\pgfpathmoveto{\pgfqpoint{0.000000in}{0.000000in}}%
\pgfpathlineto{\pgfqpoint{0.000000in}{-0.027778in}}%
\pgfusepath{stroke,fill}%
}%
\begin{pgfscope}%
\pgfsys@transformshift{3.827097in}{0.417642in}%
\pgfsys@useobject{currentmarker}{}%
\end{pgfscope}%
\end{pgfscope}%
\begin{pgfscope}%
\definecolor{textcolor}{rgb}{0.000000,0.000000,0.000000}%
\pgfsetstrokecolor{textcolor}%
\pgfsetfillcolor{textcolor}%
\pgftext[x=2.306427in,y=0.165003in,,top]{\color{textcolor}\rmfamily\fontsize{10.000000}{12.000000}\selectfont Frequency in \(\displaystyle \unit{\Hz}\)}%
\end{pgfscope}%
\begin{pgfscope}%
\pgfpathrectangle{\pgfqpoint{0.594525in}{0.417642in}}{\pgfqpoint{3.423805in}{2.011535in}}%
\pgfusepath{clip}%
\pgfsetrectcap%
\pgfsetroundjoin%
\pgfsetlinewidth{0.803000pt}%
\definecolor{currentstroke}{rgb}{0.450000,0.450000,0.450000}%
\pgfsetstrokecolor{currentstroke}%
\pgfsetdash{}{0pt}%
\pgfpathmoveto{\pgfqpoint{0.594525in}{0.417642in}}%
\pgfpathlineto{\pgfqpoint{4.018330in}{0.417642in}}%
\pgfusepath{stroke}%
\end{pgfscope}%
\begin{pgfscope}%
\pgfsetbuttcap%
\pgfsetroundjoin%
\definecolor{currentfill}{rgb}{0.000000,0.000000,0.000000}%
\pgfsetfillcolor{currentfill}%
\pgfsetlinewidth{0.803000pt}%
\definecolor{currentstroke}{rgb}{0.000000,0.000000,0.000000}%
\pgfsetstrokecolor{currentstroke}%
\pgfsetdash{}{0pt}%
\pgfsys@defobject{currentmarker}{\pgfqpoint{-0.048611in}{0.000000in}}{\pgfqpoint{-0.000000in}{0.000000in}}{%
\pgfpathmoveto{\pgfqpoint{-0.000000in}{0.000000in}}%
\pgfpathlineto{\pgfqpoint{-0.048611in}{0.000000in}}%
\pgfusepath{stroke,fill}%
}%
\begin{pgfscope}%
\pgfsys@transformshift{0.594525in}{0.417642in}%
\pgfsys@useobject{currentmarker}{}%
\end{pgfscope}%
\end{pgfscope}%
\begin{pgfscope}%
\definecolor{textcolor}{rgb}{0.000000,0.000000,0.000000}%
\pgfsetstrokecolor{textcolor}%
\pgfsetfillcolor{textcolor}%
\pgftext[x=0.241129in, y=0.378489in, left, base]{\color{textcolor}\rmfamily\fontsize{8.000000}{9.600000}\selectfont \(\displaystyle {10^{-4}}\)}%
\end{pgfscope}%
\begin{pgfscope}%
\pgfpathrectangle{\pgfqpoint{0.594525in}{0.417642in}}{\pgfqpoint{3.423805in}{2.011535in}}%
\pgfusepath{clip}%
\pgfsetrectcap%
\pgfsetroundjoin%
\pgfsetlinewidth{0.803000pt}%
\definecolor{currentstroke}{rgb}{0.450000,0.450000,0.450000}%
\pgfsetstrokecolor{currentstroke}%
\pgfsetdash{}{0pt}%
\pgfpathmoveto{\pgfqpoint{0.594525in}{0.819949in}}%
\pgfpathlineto{\pgfqpoint{4.018330in}{0.819949in}}%
\pgfusepath{stroke}%
\end{pgfscope}%
\begin{pgfscope}%
\pgfsetbuttcap%
\pgfsetroundjoin%
\definecolor{currentfill}{rgb}{0.000000,0.000000,0.000000}%
\pgfsetfillcolor{currentfill}%
\pgfsetlinewidth{0.803000pt}%
\definecolor{currentstroke}{rgb}{0.000000,0.000000,0.000000}%
\pgfsetstrokecolor{currentstroke}%
\pgfsetdash{}{0pt}%
\pgfsys@defobject{currentmarker}{\pgfqpoint{-0.048611in}{0.000000in}}{\pgfqpoint{-0.000000in}{0.000000in}}{%
\pgfpathmoveto{\pgfqpoint{-0.000000in}{0.000000in}}%
\pgfpathlineto{\pgfqpoint{-0.048611in}{0.000000in}}%
\pgfusepath{stroke,fill}%
}%
\begin{pgfscope}%
\pgfsys@transformshift{0.594525in}{0.819949in}%
\pgfsys@useobject{currentmarker}{}%
\end{pgfscope}%
\end{pgfscope}%
\begin{pgfscope}%
\definecolor{textcolor}{rgb}{0.000000,0.000000,0.000000}%
\pgfsetstrokecolor{textcolor}%
\pgfsetfillcolor{textcolor}%
\pgftext[x=0.241129in, y=0.780796in, left, base]{\color{textcolor}\rmfamily\fontsize{8.000000}{9.600000}\selectfont \(\displaystyle {10^{-3}}\)}%
\end{pgfscope}%
\begin{pgfscope}%
\pgfpathrectangle{\pgfqpoint{0.594525in}{0.417642in}}{\pgfqpoint{3.423805in}{2.011535in}}%
\pgfusepath{clip}%
\pgfsetrectcap%
\pgfsetroundjoin%
\pgfsetlinewidth{0.803000pt}%
\definecolor{currentstroke}{rgb}{0.450000,0.450000,0.450000}%
\pgfsetstrokecolor{currentstroke}%
\pgfsetdash{}{0pt}%
\pgfpathmoveto{\pgfqpoint{0.594525in}{1.222256in}}%
\pgfpathlineto{\pgfqpoint{4.018330in}{1.222256in}}%
\pgfusepath{stroke}%
\end{pgfscope}%
\begin{pgfscope}%
\pgfsetbuttcap%
\pgfsetroundjoin%
\definecolor{currentfill}{rgb}{0.000000,0.000000,0.000000}%
\pgfsetfillcolor{currentfill}%
\pgfsetlinewidth{0.803000pt}%
\definecolor{currentstroke}{rgb}{0.000000,0.000000,0.000000}%
\pgfsetstrokecolor{currentstroke}%
\pgfsetdash{}{0pt}%
\pgfsys@defobject{currentmarker}{\pgfqpoint{-0.048611in}{0.000000in}}{\pgfqpoint{-0.000000in}{0.000000in}}{%
\pgfpathmoveto{\pgfqpoint{-0.000000in}{0.000000in}}%
\pgfpathlineto{\pgfqpoint{-0.048611in}{0.000000in}}%
\pgfusepath{stroke,fill}%
}%
\begin{pgfscope}%
\pgfsys@transformshift{0.594525in}{1.222256in}%
\pgfsys@useobject{currentmarker}{}%
\end{pgfscope}%
\end{pgfscope}%
\begin{pgfscope}%
\definecolor{textcolor}{rgb}{0.000000,0.000000,0.000000}%
\pgfsetstrokecolor{textcolor}%
\pgfsetfillcolor{textcolor}%
\pgftext[x=0.241129in, y=1.183103in, left, base]{\color{textcolor}\rmfamily\fontsize{8.000000}{9.600000}\selectfont \(\displaystyle {10^{-2}}\)}%
\end{pgfscope}%
\begin{pgfscope}%
\pgfpathrectangle{\pgfqpoint{0.594525in}{0.417642in}}{\pgfqpoint{3.423805in}{2.011535in}}%
\pgfusepath{clip}%
\pgfsetrectcap%
\pgfsetroundjoin%
\pgfsetlinewidth{0.803000pt}%
\definecolor{currentstroke}{rgb}{0.450000,0.450000,0.450000}%
\pgfsetstrokecolor{currentstroke}%
\pgfsetdash{}{0pt}%
\pgfpathmoveto{\pgfqpoint{0.594525in}{1.624563in}}%
\pgfpathlineto{\pgfqpoint{4.018330in}{1.624563in}}%
\pgfusepath{stroke}%
\end{pgfscope}%
\begin{pgfscope}%
\pgfsetbuttcap%
\pgfsetroundjoin%
\definecolor{currentfill}{rgb}{0.000000,0.000000,0.000000}%
\pgfsetfillcolor{currentfill}%
\pgfsetlinewidth{0.803000pt}%
\definecolor{currentstroke}{rgb}{0.000000,0.000000,0.000000}%
\pgfsetstrokecolor{currentstroke}%
\pgfsetdash{}{0pt}%
\pgfsys@defobject{currentmarker}{\pgfqpoint{-0.048611in}{0.000000in}}{\pgfqpoint{-0.000000in}{0.000000in}}{%
\pgfpathmoveto{\pgfqpoint{-0.000000in}{0.000000in}}%
\pgfpathlineto{\pgfqpoint{-0.048611in}{0.000000in}}%
\pgfusepath{stroke,fill}%
}%
\begin{pgfscope}%
\pgfsys@transformshift{0.594525in}{1.624563in}%
\pgfsys@useobject{currentmarker}{}%
\end{pgfscope}%
\end{pgfscope}%
\begin{pgfscope}%
\definecolor{textcolor}{rgb}{0.000000,0.000000,0.000000}%
\pgfsetstrokecolor{textcolor}%
\pgfsetfillcolor{textcolor}%
\pgftext[x=0.241129in, y=1.585410in, left, base]{\color{textcolor}\rmfamily\fontsize{8.000000}{9.600000}\selectfont \(\displaystyle {10^{-1}}\)}%
\end{pgfscope}%
\begin{pgfscope}%
\pgfpathrectangle{\pgfqpoint{0.594525in}{0.417642in}}{\pgfqpoint{3.423805in}{2.011535in}}%
\pgfusepath{clip}%
\pgfsetrectcap%
\pgfsetroundjoin%
\pgfsetlinewidth{0.803000pt}%
\definecolor{currentstroke}{rgb}{0.450000,0.450000,0.450000}%
\pgfsetstrokecolor{currentstroke}%
\pgfsetdash{}{0pt}%
\pgfpathmoveto{\pgfqpoint{0.594525in}{2.026870in}}%
\pgfpathlineto{\pgfqpoint{4.018330in}{2.026870in}}%
\pgfusepath{stroke}%
\end{pgfscope}%
\begin{pgfscope}%
\pgfsetbuttcap%
\pgfsetroundjoin%
\definecolor{currentfill}{rgb}{0.000000,0.000000,0.000000}%
\pgfsetfillcolor{currentfill}%
\pgfsetlinewidth{0.803000pt}%
\definecolor{currentstroke}{rgb}{0.000000,0.000000,0.000000}%
\pgfsetstrokecolor{currentstroke}%
\pgfsetdash{}{0pt}%
\pgfsys@defobject{currentmarker}{\pgfqpoint{-0.048611in}{0.000000in}}{\pgfqpoint{-0.000000in}{0.000000in}}{%
\pgfpathmoveto{\pgfqpoint{-0.000000in}{0.000000in}}%
\pgfpathlineto{\pgfqpoint{-0.048611in}{0.000000in}}%
\pgfusepath{stroke,fill}%
}%
\begin{pgfscope}%
\pgfsys@transformshift{0.594525in}{2.026870in}%
\pgfsys@useobject{currentmarker}{}%
\end{pgfscope}%
\end{pgfscope}%
\begin{pgfscope}%
\definecolor{textcolor}{rgb}{0.000000,0.000000,0.000000}%
\pgfsetstrokecolor{textcolor}%
\pgfsetfillcolor{textcolor}%
\pgftext[x=0.321376in, y=1.987717in, left, base]{\color{textcolor}\rmfamily\fontsize{8.000000}{9.600000}\selectfont \(\displaystyle {10^{0}}\)}%
\end{pgfscope}%
\begin{pgfscope}%
\pgfpathrectangle{\pgfqpoint{0.594525in}{0.417642in}}{\pgfqpoint{3.423805in}{2.011535in}}%
\pgfusepath{clip}%
\pgfsetrectcap%
\pgfsetroundjoin%
\pgfsetlinewidth{0.803000pt}%
\definecolor{currentstroke}{rgb}{0.450000,0.450000,0.450000}%
\pgfsetstrokecolor{currentstroke}%
\pgfsetdash{}{0pt}%
\pgfpathmoveto{\pgfqpoint{0.594525in}{2.429177in}}%
\pgfpathlineto{\pgfqpoint{4.018330in}{2.429177in}}%
\pgfusepath{stroke}%
\end{pgfscope}%
\begin{pgfscope}%
\pgfsetbuttcap%
\pgfsetroundjoin%
\definecolor{currentfill}{rgb}{0.000000,0.000000,0.000000}%
\pgfsetfillcolor{currentfill}%
\pgfsetlinewidth{0.803000pt}%
\definecolor{currentstroke}{rgb}{0.000000,0.000000,0.000000}%
\pgfsetstrokecolor{currentstroke}%
\pgfsetdash{}{0pt}%
\pgfsys@defobject{currentmarker}{\pgfqpoint{-0.048611in}{0.000000in}}{\pgfqpoint{-0.000000in}{0.000000in}}{%
\pgfpathmoveto{\pgfqpoint{-0.000000in}{0.000000in}}%
\pgfpathlineto{\pgfqpoint{-0.048611in}{0.000000in}}%
\pgfusepath{stroke,fill}%
}%
\begin{pgfscope}%
\pgfsys@transformshift{0.594525in}{2.429177in}%
\pgfsys@useobject{currentmarker}{}%
\end{pgfscope}%
\end{pgfscope}%
\begin{pgfscope}%
\definecolor{textcolor}{rgb}{0.000000,0.000000,0.000000}%
\pgfsetstrokecolor{textcolor}%
\pgfsetfillcolor{textcolor}%
\pgftext[x=0.321376in, y=2.390024in, left, base]{\color{textcolor}\rmfamily\fontsize{8.000000}{9.600000}\selectfont \(\displaystyle {10^{1}}\)}%
\end{pgfscope}%
\begin{pgfscope}%
\pgfpathrectangle{\pgfqpoint{0.594525in}{0.417642in}}{\pgfqpoint{3.423805in}{2.011535in}}%
\pgfusepath{clip}%
\pgfsetrectcap%
\pgfsetroundjoin%
\pgfsetlinewidth{0.803000pt}%
\definecolor{currentstroke}{rgb}{0.850000,0.850000,0.850000}%
\pgfsetstrokecolor{currentstroke}%
\pgfsetdash{}{0pt}%
\pgfpathmoveto{\pgfqpoint{0.594525in}{0.538748in}}%
\pgfpathlineto{\pgfqpoint{4.018330in}{0.538748in}}%
\pgfusepath{stroke}%
\end{pgfscope}%
\begin{pgfscope}%
\pgfsetbuttcap%
\pgfsetroundjoin%
\definecolor{currentfill}{rgb}{0.000000,0.000000,0.000000}%
\pgfsetfillcolor{currentfill}%
\pgfsetlinewidth{0.602250pt}%
\definecolor{currentstroke}{rgb}{0.000000,0.000000,0.000000}%
\pgfsetstrokecolor{currentstroke}%
\pgfsetdash{}{0pt}%
\pgfsys@defobject{currentmarker}{\pgfqpoint{-0.027778in}{0.000000in}}{\pgfqpoint{-0.000000in}{0.000000in}}{%
\pgfpathmoveto{\pgfqpoint{-0.000000in}{0.000000in}}%
\pgfpathlineto{\pgfqpoint{-0.027778in}{0.000000in}}%
\pgfusepath{stroke,fill}%
}%
\begin{pgfscope}%
\pgfsys@transformshift{0.594525in}{0.538748in}%
\pgfsys@useobject{currentmarker}{}%
\end{pgfscope}%
\end{pgfscope}%
\begin{pgfscope}%
\pgfpathrectangle{\pgfqpoint{0.594525in}{0.417642in}}{\pgfqpoint{3.423805in}{2.011535in}}%
\pgfusepath{clip}%
\pgfsetrectcap%
\pgfsetroundjoin%
\pgfsetlinewidth{0.803000pt}%
\definecolor{currentstroke}{rgb}{0.850000,0.850000,0.850000}%
\pgfsetstrokecolor{currentstroke}%
\pgfsetdash{}{0pt}%
\pgfpathmoveto{\pgfqpoint{0.594525in}{0.609591in}}%
\pgfpathlineto{\pgfqpoint{4.018330in}{0.609591in}}%
\pgfusepath{stroke}%
\end{pgfscope}%
\begin{pgfscope}%
\pgfsetbuttcap%
\pgfsetroundjoin%
\definecolor{currentfill}{rgb}{0.000000,0.000000,0.000000}%
\pgfsetfillcolor{currentfill}%
\pgfsetlinewidth{0.602250pt}%
\definecolor{currentstroke}{rgb}{0.000000,0.000000,0.000000}%
\pgfsetstrokecolor{currentstroke}%
\pgfsetdash{}{0pt}%
\pgfsys@defobject{currentmarker}{\pgfqpoint{-0.027778in}{0.000000in}}{\pgfqpoint{-0.000000in}{0.000000in}}{%
\pgfpathmoveto{\pgfqpoint{-0.000000in}{0.000000in}}%
\pgfpathlineto{\pgfqpoint{-0.027778in}{0.000000in}}%
\pgfusepath{stroke,fill}%
}%
\begin{pgfscope}%
\pgfsys@transformshift{0.594525in}{0.609591in}%
\pgfsys@useobject{currentmarker}{}%
\end{pgfscope}%
\end{pgfscope}%
\begin{pgfscope}%
\pgfpathrectangle{\pgfqpoint{0.594525in}{0.417642in}}{\pgfqpoint{3.423805in}{2.011535in}}%
\pgfusepath{clip}%
\pgfsetrectcap%
\pgfsetroundjoin%
\pgfsetlinewidth{0.803000pt}%
\definecolor{currentstroke}{rgb}{0.850000,0.850000,0.850000}%
\pgfsetstrokecolor{currentstroke}%
\pgfsetdash{}{0pt}%
\pgfpathmoveto{\pgfqpoint{0.594525in}{0.659855in}}%
\pgfpathlineto{\pgfqpoint{4.018330in}{0.659855in}}%
\pgfusepath{stroke}%
\end{pgfscope}%
\begin{pgfscope}%
\pgfsetbuttcap%
\pgfsetroundjoin%
\definecolor{currentfill}{rgb}{0.000000,0.000000,0.000000}%
\pgfsetfillcolor{currentfill}%
\pgfsetlinewidth{0.602250pt}%
\definecolor{currentstroke}{rgb}{0.000000,0.000000,0.000000}%
\pgfsetstrokecolor{currentstroke}%
\pgfsetdash{}{0pt}%
\pgfsys@defobject{currentmarker}{\pgfqpoint{-0.027778in}{0.000000in}}{\pgfqpoint{-0.000000in}{0.000000in}}{%
\pgfpathmoveto{\pgfqpoint{-0.000000in}{0.000000in}}%
\pgfpathlineto{\pgfqpoint{-0.027778in}{0.000000in}}%
\pgfusepath{stroke,fill}%
}%
\begin{pgfscope}%
\pgfsys@transformshift{0.594525in}{0.659855in}%
\pgfsys@useobject{currentmarker}{}%
\end{pgfscope}%
\end{pgfscope}%
\begin{pgfscope}%
\pgfpathrectangle{\pgfqpoint{0.594525in}{0.417642in}}{\pgfqpoint{3.423805in}{2.011535in}}%
\pgfusepath{clip}%
\pgfsetrectcap%
\pgfsetroundjoin%
\pgfsetlinewidth{0.803000pt}%
\definecolor{currentstroke}{rgb}{0.850000,0.850000,0.850000}%
\pgfsetstrokecolor{currentstroke}%
\pgfsetdash{}{0pt}%
\pgfpathmoveto{\pgfqpoint{0.594525in}{0.698843in}}%
\pgfpathlineto{\pgfqpoint{4.018330in}{0.698843in}}%
\pgfusepath{stroke}%
\end{pgfscope}%
\begin{pgfscope}%
\pgfsetbuttcap%
\pgfsetroundjoin%
\definecolor{currentfill}{rgb}{0.000000,0.000000,0.000000}%
\pgfsetfillcolor{currentfill}%
\pgfsetlinewidth{0.602250pt}%
\definecolor{currentstroke}{rgb}{0.000000,0.000000,0.000000}%
\pgfsetstrokecolor{currentstroke}%
\pgfsetdash{}{0pt}%
\pgfsys@defobject{currentmarker}{\pgfqpoint{-0.027778in}{0.000000in}}{\pgfqpoint{-0.000000in}{0.000000in}}{%
\pgfpathmoveto{\pgfqpoint{-0.000000in}{0.000000in}}%
\pgfpathlineto{\pgfqpoint{-0.027778in}{0.000000in}}%
\pgfusepath{stroke,fill}%
}%
\begin{pgfscope}%
\pgfsys@transformshift{0.594525in}{0.698843in}%
\pgfsys@useobject{currentmarker}{}%
\end{pgfscope}%
\end{pgfscope}%
\begin{pgfscope}%
\pgfpathrectangle{\pgfqpoint{0.594525in}{0.417642in}}{\pgfqpoint{3.423805in}{2.011535in}}%
\pgfusepath{clip}%
\pgfsetrectcap%
\pgfsetroundjoin%
\pgfsetlinewidth{0.803000pt}%
\definecolor{currentstroke}{rgb}{0.850000,0.850000,0.850000}%
\pgfsetstrokecolor{currentstroke}%
\pgfsetdash{}{0pt}%
\pgfpathmoveto{\pgfqpoint{0.594525in}{0.730698in}}%
\pgfpathlineto{\pgfqpoint{4.018330in}{0.730698in}}%
\pgfusepath{stroke}%
\end{pgfscope}%
\begin{pgfscope}%
\pgfsetbuttcap%
\pgfsetroundjoin%
\definecolor{currentfill}{rgb}{0.000000,0.000000,0.000000}%
\pgfsetfillcolor{currentfill}%
\pgfsetlinewidth{0.602250pt}%
\definecolor{currentstroke}{rgb}{0.000000,0.000000,0.000000}%
\pgfsetstrokecolor{currentstroke}%
\pgfsetdash{}{0pt}%
\pgfsys@defobject{currentmarker}{\pgfqpoint{-0.027778in}{0.000000in}}{\pgfqpoint{-0.000000in}{0.000000in}}{%
\pgfpathmoveto{\pgfqpoint{-0.000000in}{0.000000in}}%
\pgfpathlineto{\pgfqpoint{-0.027778in}{0.000000in}}%
\pgfusepath{stroke,fill}%
}%
\begin{pgfscope}%
\pgfsys@transformshift{0.594525in}{0.730698in}%
\pgfsys@useobject{currentmarker}{}%
\end{pgfscope}%
\end{pgfscope}%
\begin{pgfscope}%
\pgfpathrectangle{\pgfqpoint{0.594525in}{0.417642in}}{\pgfqpoint{3.423805in}{2.011535in}}%
\pgfusepath{clip}%
\pgfsetrectcap%
\pgfsetroundjoin%
\pgfsetlinewidth{0.803000pt}%
\definecolor{currentstroke}{rgb}{0.850000,0.850000,0.850000}%
\pgfsetstrokecolor{currentstroke}%
\pgfsetdash{}{0pt}%
\pgfpathmoveto{\pgfqpoint{0.594525in}{0.757631in}}%
\pgfpathlineto{\pgfqpoint{4.018330in}{0.757631in}}%
\pgfusepath{stroke}%
\end{pgfscope}%
\begin{pgfscope}%
\pgfsetbuttcap%
\pgfsetroundjoin%
\definecolor{currentfill}{rgb}{0.000000,0.000000,0.000000}%
\pgfsetfillcolor{currentfill}%
\pgfsetlinewidth{0.602250pt}%
\definecolor{currentstroke}{rgb}{0.000000,0.000000,0.000000}%
\pgfsetstrokecolor{currentstroke}%
\pgfsetdash{}{0pt}%
\pgfsys@defobject{currentmarker}{\pgfqpoint{-0.027778in}{0.000000in}}{\pgfqpoint{-0.000000in}{0.000000in}}{%
\pgfpathmoveto{\pgfqpoint{-0.000000in}{0.000000in}}%
\pgfpathlineto{\pgfqpoint{-0.027778in}{0.000000in}}%
\pgfusepath{stroke,fill}%
}%
\begin{pgfscope}%
\pgfsys@transformshift{0.594525in}{0.757631in}%
\pgfsys@useobject{currentmarker}{}%
\end{pgfscope}%
\end{pgfscope}%
\begin{pgfscope}%
\pgfpathrectangle{\pgfqpoint{0.594525in}{0.417642in}}{\pgfqpoint{3.423805in}{2.011535in}}%
\pgfusepath{clip}%
\pgfsetrectcap%
\pgfsetroundjoin%
\pgfsetlinewidth{0.803000pt}%
\definecolor{currentstroke}{rgb}{0.850000,0.850000,0.850000}%
\pgfsetstrokecolor{currentstroke}%
\pgfsetdash{}{0pt}%
\pgfpathmoveto{\pgfqpoint{0.594525in}{0.780961in}}%
\pgfpathlineto{\pgfqpoint{4.018330in}{0.780961in}}%
\pgfusepath{stroke}%
\end{pgfscope}%
\begin{pgfscope}%
\pgfsetbuttcap%
\pgfsetroundjoin%
\definecolor{currentfill}{rgb}{0.000000,0.000000,0.000000}%
\pgfsetfillcolor{currentfill}%
\pgfsetlinewidth{0.602250pt}%
\definecolor{currentstroke}{rgb}{0.000000,0.000000,0.000000}%
\pgfsetstrokecolor{currentstroke}%
\pgfsetdash{}{0pt}%
\pgfsys@defobject{currentmarker}{\pgfqpoint{-0.027778in}{0.000000in}}{\pgfqpoint{-0.000000in}{0.000000in}}{%
\pgfpathmoveto{\pgfqpoint{-0.000000in}{0.000000in}}%
\pgfpathlineto{\pgfqpoint{-0.027778in}{0.000000in}}%
\pgfusepath{stroke,fill}%
}%
\begin{pgfscope}%
\pgfsys@transformshift{0.594525in}{0.780961in}%
\pgfsys@useobject{currentmarker}{}%
\end{pgfscope}%
\end{pgfscope}%
\begin{pgfscope}%
\pgfpathrectangle{\pgfqpoint{0.594525in}{0.417642in}}{\pgfqpoint{3.423805in}{2.011535in}}%
\pgfusepath{clip}%
\pgfsetrectcap%
\pgfsetroundjoin%
\pgfsetlinewidth{0.803000pt}%
\definecolor{currentstroke}{rgb}{0.850000,0.850000,0.850000}%
\pgfsetstrokecolor{currentstroke}%
\pgfsetdash{}{0pt}%
\pgfpathmoveto{\pgfqpoint{0.594525in}{0.801540in}}%
\pgfpathlineto{\pgfqpoint{4.018330in}{0.801540in}}%
\pgfusepath{stroke}%
\end{pgfscope}%
\begin{pgfscope}%
\pgfsetbuttcap%
\pgfsetroundjoin%
\definecolor{currentfill}{rgb}{0.000000,0.000000,0.000000}%
\pgfsetfillcolor{currentfill}%
\pgfsetlinewidth{0.602250pt}%
\definecolor{currentstroke}{rgb}{0.000000,0.000000,0.000000}%
\pgfsetstrokecolor{currentstroke}%
\pgfsetdash{}{0pt}%
\pgfsys@defobject{currentmarker}{\pgfqpoint{-0.027778in}{0.000000in}}{\pgfqpoint{-0.000000in}{0.000000in}}{%
\pgfpathmoveto{\pgfqpoint{-0.000000in}{0.000000in}}%
\pgfpathlineto{\pgfqpoint{-0.027778in}{0.000000in}}%
\pgfusepath{stroke,fill}%
}%
\begin{pgfscope}%
\pgfsys@transformshift{0.594525in}{0.801540in}%
\pgfsys@useobject{currentmarker}{}%
\end{pgfscope}%
\end{pgfscope}%
\begin{pgfscope}%
\pgfpathrectangle{\pgfqpoint{0.594525in}{0.417642in}}{\pgfqpoint{3.423805in}{2.011535in}}%
\pgfusepath{clip}%
\pgfsetrectcap%
\pgfsetroundjoin%
\pgfsetlinewidth{0.803000pt}%
\definecolor{currentstroke}{rgb}{0.850000,0.850000,0.850000}%
\pgfsetstrokecolor{currentstroke}%
\pgfsetdash{}{0pt}%
\pgfpathmoveto{\pgfqpoint{0.594525in}{0.941055in}}%
\pgfpathlineto{\pgfqpoint{4.018330in}{0.941055in}}%
\pgfusepath{stroke}%
\end{pgfscope}%
\begin{pgfscope}%
\pgfsetbuttcap%
\pgfsetroundjoin%
\definecolor{currentfill}{rgb}{0.000000,0.000000,0.000000}%
\pgfsetfillcolor{currentfill}%
\pgfsetlinewidth{0.602250pt}%
\definecolor{currentstroke}{rgb}{0.000000,0.000000,0.000000}%
\pgfsetstrokecolor{currentstroke}%
\pgfsetdash{}{0pt}%
\pgfsys@defobject{currentmarker}{\pgfqpoint{-0.027778in}{0.000000in}}{\pgfqpoint{-0.000000in}{0.000000in}}{%
\pgfpathmoveto{\pgfqpoint{-0.000000in}{0.000000in}}%
\pgfpathlineto{\pgfqpoint{-0.027778in}{0.000000in}}%
\pgfusepath{stroke,fill}%
}%
\begin{pgfscope}%
\pgfsys@transformshift{0.594525in}{0.941055in}%
\pgfsys@useobject{currentmarker}{}%
\end{pgfscope}%
\end{pgfscope}%
\begin{pgfscope}%
\pgfpathrectangle{\pgfqpoint{0.594525in}{0.417642in}}{\pgfqpoint{3.423805in}{2.011535in}}%
\pgfusepath{clip}%
\pgfsetrectcap%
\pgfsetroundjoin%
\pgfsetlinewidth{0.803000pt}%
\definecolor{currentstroke}{rgb}{0.850000,0.850000,0.850000}%
\pgfsetstrokecolor{currentstroke}%
\pgfsetdash{}{0pt}%
\pgfpathmoveto{\pgfqpoint{0.594525in}{1.011898in}}%
\pgfpathlineto{\pgfqpoint{4.018330in}{1.011898in}}%
\pgfusepath{stroke}%
\end{pgfscope}%
\begin{pgfscope}%
\pgfsetbuttcap%
\pgfsetroundjoin%
\definecolor{currentfill}{rgb}{0.000000,0.000000,0.000000}%
\pgfsetfillcolor{currentfill}%
\pgfsetlinewidth{0.602250pt}%
\definecolor{currentstroke}{rgb}{0.000000,0.000000,0.000000}%
\pgfsetstrokecolor{currentstroke}%
\pgfsetdash{}{0pt}%
\pgfsys@defobject{currentmarker}{\pgfqpoint{-0.027778in}{0.000000in}}{\pgfqpoint{-0.000000in}{0.000000in}}{%
\pgfpathmoveto{\pgfqpoint{-0.000000in}{0.000000in}}%
\pgfpathlineto{\pgfqpoint{-0.027778in}{0.000000in}}%
\pgfusepath{stroke,fill}%
}%
\begin{pgfscope}%
\pgfsys@transformshift{0.594525in}{1.011898in}%
\pgfsys@useobject{currentmarker}{}%
\end{pgfscope}%
\end{pgfscope}%
\begin{pgfscope}%
\pgfpathrectangle{\pgfqpoint{0.594525in}{0.417642in}}{\pgfqpoint{3.423805in}{2.011535in}}%
\pgfusepath{clip}%
\pgfsetrectcap%
\pgfsetroundjoin%
\pgfsetlinewidth{0.803000pt}%
\definecolor{currentstroke}{rgb}{0.850000,0.850000,0.850000}%
\pgfsetstrokecolor{currentstroke}%
\pgfsetdash{}{0pt}%
\pgfpathmoveto{\pgfqpoint{0.594525in}{1.062162in}}%
\pgfpathlineto{\pgfqpoint{4.018330in}{1.062162in}}%
\pgfusepath{stroke}%
\end{pgfscope}%
\begin{pgfscope}%
\pgfsetbuttcap%
\pgfsetroundjoin%
\definecolor{currentfill}{rgb}{0.000000,0.000000,0.000000}%
\pgfsetfillcolor{currentfill}%
\pgfsetlinewidth{0.602250pt}%
\definecolor{currentstroke}{rgb}{0.000000,0.000000,0.000000}%
\pgfsetstrokecolor{currentstroke}%
\pgfsetdash{}{0pt}%
\pgfsys@defobject{currentmarker}{\pgfqpoint{-0.027778in}{0.000000in}}{\pgfqpoint{-0.000000in}{0.000000in}}{%
\pgfpathmoveto{\pgfqpoint{-0.000000in}{0.000000in}}%
\pgfpathlineto{\pgfqpoint{-0.027778in}{0.000000in}}%
\pgfusepath{stroke,fill}%
}%
\begin{pgfscope}%
\pgfsys@transformshift{0.594525in}{1.062162in}%
\pgfsys@useobject{currentmarker}{}%
\end{pgfscope}%
\end{pgfscope}%
\begin{pgfscope}%
\pgfpathrectangle{\pgfqpoint{0.594525in}{0.417642in}}{\pgfqpoint{3.423805in}{2.011535in}}%
\pgfusepath{clip}%
\pgfsetrectcap%
\pgfsetroundjoin%
\pgfsetlinewidth{0.803000pt}%
\definecolor{currentstroke}{rgb}{0.850000,0.850000,0.850000}%
\pgfsetstrokecolor{currentstroke}%
\pgfsetdash{}{0pt}%
\pgfpathmoveto{\pgfqpoint{0.594525in}{1.101150in}}%
\pgfpathlineto{\pgfqpoint{4.018330in}{1.101150in}}%
\pgfusepath{stroke}%
\end{pgfscope}%
\begin{pgfscope}%
\pgfsetbuttcap%
\pgfsetroundjoin%
\definecolor{currentfill}{rgb}{0.000000,0.000000,0.000000}%
\pgfsetfillcolor{currentfill}%
\pgfsetlinewidth{0.602250pt}%
\definecolor{currentstroke}{rgb}{0.000000,0.000000,0.000000}%
\pgfsetstrokecolor{currentstroke}%
\pgfsetdash{}{0pt}%
\pgfsys@defobject{currentmarker}{\pgfqpoint{-0.027778in}{0.000000in}}{\pgfqpoint{-0.000000in}{0.000000in}}{%
\pgfpathmoveto{\pgfqpoint{-0.000000in}{0.000000in}}%
\pgfpathlineto{\pgfqpoint{-0.027778in}{0.000000in}}%
\pgfusepath{stroke,fill}%
}%
\begin{pgfscope}%
\pgfsys@transformshift{0.594525in}{1.101150in}%
\pgfsys@useobject{currentmarker}{}%
\end{pgfscope}%
\end{pgfscope}%
\begin{pgfscope}%
\pgfpathrectangle{\pgfqpoint{0.594525in}{0.417642in}}{\pgfqpoint{3.423805in}{2.011535in}}%
\pgfusepath{clip}%
\pgfsetrectcap%
\pgfsetroundjoin%
\pgfsetlinewidth{0.803000pt}%
\definecolor{currentstroke}{rgb}{0.850000,0.850000,0.850000}%
\pgfsetstrokecolor{currentstroke}%
\pgfsetdash{}{0pt}%
\pgfpathmoveto{\pgfqpoint{0.594525in}{1.133005in}}%
\pgfpathlineto{\pgfqpoint{4.018330in}{1.133005in}}%
\pgfusepath{stroke}%
\end{pgfscope}%
\begin{pgfscope}%
\pgfsetbuttcap%
\pgfsetroundjoin%
\definecolor{currentfill}{rgb}{0.000000,0.000000,0.000000}%
\pgfsetfillcolor{currentfill}%
\pgfsetlinewidth{0.602250pt}%
\definecolor{currentstroke}{rgb}{0.000000,0.000000,0.000000}%
\pgfsetstrokecolor{currentstroke}%
\pgfsetdash{}{0pt}%
\pgfsys@defobject{currentmarker}{\pgfqpoint{-0.027778in}{0.000000in}}{\pgfqpoint{-0.000000in}{0.000000in}}{%
\pgfpathmoveto{\pgfqpoint{-0.000000in}{0.000000in}}%
\pgfpathlineto{\pgfqpoint{-0.027778in}{0.000000in}}%
\pgfusepath{stroke,fill}%
}%
\begin{pgfscope}%
\pgfsys@transformshift{0.594525in}{1.133005in}%
\pgfsys@useobject{currentmarker}{}%
\end{pgfscope}%
\end{pgfscope}%
\begin{pgfscope}%
\pgfpathrectangle{\pgfqpoint{0.594525in}{0.417642in}}{\pgfqpoint{3.423805in}{2.011535in}}%
\pgfusepath{clip}%
\pgfsetrectcap%
\pgfsetroundjoin%
\pgfsetlinewidth{0.803000pt}%
\definecolor{currentstroke}{rgb}{0.850000,0.850000,0.850000}%
\pgfsetstrokecolor{currentstroke}%
\pgfsetdash{}{0pt}%
\pgfpathmoveto{\pgfqpoint{0.594525in}{1.159938in}}%
\pgfpathlineto{\pgfqpoint{4.018330in}{1.159938in}}%
\pgfusepath{stroke}%
\end{pgfscope}%
\begin{pgfscope}%
\pgfsetbuttcap%
\pgfsetroundjoin%
\definecolor{currentfill}{rgb}{0.000000,0.000000,0.000000}%
\pgfsetfillcolor{currentfill}%
\pgfsetlinewidth{0.602250pt}%
\definecolor{currentstroke}{rgb}{0.000000,0.000000,0.000000}%
\pgfsetstrokecolor{currentstroke}%
\pgfsetdash{}{0pt}%
\pgfsys@defobject{currentmarker}{\pgfqpoint{-0.027778in}{0.000000in}}{\pgfqpoint{-0.000000in}{0.000000in}}{%
\pgfpathmoveto{\pgfqpoint{-0.000000in}{0.000000in}}%
\pgfpathlineto{\pgfqpoint{-0.027778in}{0.000000in}}%
\pgfusepath{stroke,fill}%
}%
\begin{pgfscope}%
\pgfsys@transformshift{0.594525in}{1.159938in}%
\pgfsys@useobject{currentmarker}{}%
\end{pgfscope}%
\end{pgfscope}%
\begin{pgfscope}%
\pgfpathrectangle{\pgfqpoint{0.594525in}{0.417642in}}{\pgfqpoint{3.423805in}{2.011535in}}%
\pgfusepath{clip}%
\pgfsetrectcap%
\pgfsetroundjoin%
\pgfsetlinewidth{0.803000pt}%
\definecolor{currentstroke}{rgb}{0.850000,0.850000,0.850000}%
\pgfsetstrokecolor{currentstroke}%
\pgfsetdash{}{0pt}%
\pgfpathmoveto{\pgfqpoint{0.594525in}{1.183268in}}%
\pgfpathlineto{\pgfqpoint{4.018330in}{1.183268in}}%
\pgfusepath{stroke}%
\end{pgfscope}%
\begin{pgfscope}%
\pgfsetbuttcap%
\pgfsetroundjoin%
\definecolor{currentfill}{rgb}{0.000000,0.000000,0.000000}%
\pgfsetfillcolor{currentfill}%
\pgfsetlinewidth{0.602250pt}%
\definecolor{currentstroke}{rgb}{0.000000,0.000000,0.000000}%
\pgfsetstrokecolor{currentstroke}%
\pgfsetdash{}{0pt}%
\pgfsys@defobject{currentmarker}{\pgfqpoint{-0.027778in}{0.000000in}}{\pgfqpoint{-0.000000in}{0.000000in}}{%
\pgfpathmoveto{\pgfqpoint{-0.000000in}{0.000000in}}%
\pgfpathlineto{\pgfqpoint{-0.027778in}{0.000000in}}%
\pgfusepath{stroke,fill}%
}%
\begin{pgfscope}%
\pgfsys@transformshift{0.594525in}{1.183268in}%
\pgfsys@useobject{currentmarker}{}%
\end{pgfscope}%
\end{pgfscope}%
\begin{pgfscope}%
\pgfpathrectangle{\pgfqpoint{0.594525in}{0.417642in}}{\pgfqpoint{3.423805in}{2.011535in}}%
\pgfusepath{clip}%
\pgfsetrectcap%
\pgfsetroundjoin%
\pgfsetlinewidth{0.803000pt}%
\definecolor{currentstroke}{rgb}{0.850000,0.850000,0.850000}%
\pgfsetstrokecolor{currentstroke}%
\pgfsetdash{}{0pt}%
\pgfpathmoveto{\pgfqpoint{0.594525in}{1.203847in}}%
\pgfpathlineto{\pgfqpoint{4.018330in}{1.203847in}}%
\pgfusepath{stroke}%
\end{pgfscope}%
\begin{pgfscope}%
\pgfsetbuttcap%
\pgfsetroundjoin%
\definecolor{currentfill}{rgb}{0.000000,0.000000,0.000000}%
\pgfsetfillcolor{currentfill}%
\pgfsetlinewidth{0.602250pt}%
\definecolor{currentstroke}{rgb}{0.000000,0.000000,0.000000}%
\pgfsetstrokecolor{currentstroke}%
\pgfsetdash{}{0pt}%
\pgfsys@defobject{currentmarker}{\pgfqpoint{-0.027778in}{0.000000in}}{\pgfqpoint{-0.000000in}{0.000000in}}{%
\pgfpathmoveto{\pgfqpoint{-0.000000in}{0.000000in}}%
\pgfpathlineto{\pgfqpoint{-0.027778in}{0.000000in}}%
\pgfusepath{stroke,fill}%
}%
\begin{pgfscope}%
\pgfsys@transformshift{0.594525in}{1.203847in}%
\pgfsys@useobject{currentmarker}{}%
\end{pgfscope}%
\end{pgfscope}%
\begin{pgfscope}%
\pgfpathrectangle{\pgfqpoint{0.594525in}{0.417642in}}{\pgfqpoint{3.423805in}{2.011535in}}%
\pgfusepath{clip}%
\pgfsetrectcap%
\pgfsetroundjoin%
\pgfsetlinewidth{0.803000pt}%
\definecolor{currentstroke}{rgb}{0.850000,0.850000,0.850000}%
\pgfsetstrokecolor{currentstroke}%
\pgfsetdash{}{0pt}%
\pgfpathmoveto{\pgfqpoint{0.594525in}{1.343363in}}%
\pgfpathlineto{\pgfqpoint{4.018330in}{1.343363in}}%
\pgfusepath{stroke}%
\end{pgfscope}%
\begin{pgfscope}%
\pgfsetbuttcap%
\pgfsetroundjoin%
\definecolor{currentfill}{rgb}{0.000000,0.000000,0.000000}%
\pgfsetfillcolor{currentfill}%
\pgfsetlinewidth{0.602250pt}%
\definecolor{currentstroke}{rgb}{0.000000,0.000000,0.000000}%
\pgfsetstrokecolor{currentstroke}%
\pgfsetdash{}{0pt}%
\pgfsys@defobject{currentmarker}{\pgfqpoint{-0.027778in}{0.000000in}}{\pgfqpoint{-0.000000in}{0.000000in}}{%
\pgfpathmoveto{\pgfqpoint{-0.000000in}{0.000000in}}%
\pgfpathlineto{\pgfqpoint{-0.027778in}{0.000000in}}%
\pgfusepath{stroke,fill}%
}%
\begin{pgfscope}%
\pgfsys@transformshift{0.594525in}{1.343363in}%
\pgfsys@useobject{currentmarker}{}%
\end{pgfscope}%
\end{pgfscope}%
\begin{pgfscope}%
\pgfpathrectangle{\pgfqpoint{0.594525in}{0.417642in}}{\pgfqpoint{3.423805in}{2.011535in}}%
\pgfusepath{clip}%
\pgfsetrectcap%
\pgfsetroundjoin%
\pgfsetlinewidth{0.803000pt}%
\definecolor{currentstroke}{rgb}{0.850000,0.850000,0.850000}%
\pgfsetstrokecolor{currentstroke}%
\pgfsetdash{}{0pt}%
\pgfpathmoveto{\pgfqpoint{0.594525in}{1.414205in}}%
\pgfpathlineto{\pgfqpoint{4.018330in}{1.414205in}}%
\pgfusepath{stroke}%
\end{pgfscope}%
\begin{pgfscope}%
\pgfsetbuttcap%
\pgfsetroundjoin%
\definecolor{currentfill}{rgb}{0.000000,0.000000,0.000000}%
\pgfsetfillcolor{currentfill}%
\pgfsetlinewidth{0.602250pt}%
\definecolor{currentstroke}{rgb}{0.000000,0.000000,0.000000}%
\pgfsetstrokecolor{currentstroke}%
\pgfsetdash{}{0pt}%
\pgfsys@defobject{currentmarker}{\pgfqpoint{-0.027778in}{0.000000in}}{\pgfqpoint{-0.000000in}{0.000000in}}{%
\pgfpathmoveto{\pgfqpoint{-0.000000in}{0.000000in}}%
\pgfpathlineto{\pgfqpoint{-0.027778in}{0.000000in}}%
\pgfusepath{stroke,fill}%
}%
\begin{pgfscope}%
\pgfsys@transformshift{0.594525in}{1.414205in}%
\pgfsys@useobject{currentmarker}{}%
\end{pgfscope}%
\end{pgfscope}%
\begin{pgfscope}%
\pgfpathrectangle{\pgfqpoint{0.594525in}{0.417642in}}{\pgfqpoint{3.423805in}{2.011535in}}%
\pgfusepath{clip}%
\pgfsetrectcap%
\pgfsetroundjoin%
\pgfsetlinewidth{0.803000pt}%
\definecolor{currentstroke}{rgb}{0.850000,0.850000,0.850000}%
\pgfsetstrokecolor{currentstroke}%
\pgfsetdash{}{0pt}%
\pgfpathmoveto{\pgfqpoint{0.594525in}{1.464469in}}%
\pgfpathlineto{\pgfqpoint{4.018330in}{1.464469in}}%
\pgfusepath{stroke}%
\end{pgfscope}%
\begin{pgfscope}%
\pgfsetbuttcap%
\pgfsetroundjoin%
\definecolor{currentfill}{rgb}{0.000000,0.000000,0.000000}%
\pgfsetfillcolor{currentfill}%
\pgfsetlinewidth{0.602250pt}%
\definecolor{currentstroke}{rgb}{0.000000,0.000000,0.000000}%
\pgfsetstrokecolor{currentstroke}%
\pgfsetdash{}{0pt}%
\pgfsys@defobject{currentmarker}{\pgfqpoint{-0.027778in}{0.000000in}}{\pgfqpoint{-0.000000in}{0.000000in}}{%
\pgfpathmoveto{\pgfqpoint{-0.000000in}{0.000000in}}%
\pgfpathlineto{\pgfqpoint{-0.027778in}{0.000000in}}%
\pgfusepath{stroke,fill}%
}%
\begin{pgfscope}%
\pgfsys@transformshift{0.594525in}{1.464469in}%
\pgfsys@useobject{currentmarker}{}%
\end{pgfscope}%
\end{pgfscope}%
\begin{pgfscope}%
\pgfpathrectangle{\pgfqpoint{0.594525in}{0.417642in}}{\pgfqpoint{3.423805in}{2.011535in}}%
\pgfusepath{clip}%
\pgfsetrectcap%
\pgfsetroundjoin%
\pgfsetlinewidth{0.803000pt}%
\definecolor{currentstroke}{rgb}{0.850000,0.850000,0.850000}%
\pgfsetstrokecolor{currentstroke}%
\pgfsetdash{}{0pt}%
\pgfpathmoveto{\pgfqpoint{0.594525in}{1.503457in}}%
\pgfpathlineto{\pgfqpoint{4.018330in}{1.503457in}}%
\pgfusepath{stroke}%
\end{pgfscope}%
\begin{pgfscope}%
\pgfsetbuttcap%
\pgfsetroundjoin%
\definecolor{currentfill}{rgb}{0.000000,0.000000,0.000000}%
\pgfsetfillcolor{currentfill}%
\pgfsetlinewidth{0.602250pt}%
\definecolor{currentstroke}{rgb}{0.000000,0.000000,0.000000}%
\pgfsetstrokecolor{currentstroke}%
\pgfsetdash{}{0pt}%
\pgfsys@defobject{currentmarker}{\pgfqpoint{-0.027778in}{0.000000in}}{\pgfqpoint{-0.000000in}{0.000000in}}{%
\pgfpathmoveto{\pgfqpoint{-0.000000in}{0.000000in}}%
\pgfpathlineto{\pgfqpoint{-0.027778in}{0.000000in}}%
\pgfusepath{stroke,fill}%
}%
\begin{pgfscope}%
\pgfsys@transformshift{0.594525in}{1.503457in}%
\pgfsys@useobject{currentmarker}{}%
\end{pgfscope}%
\end{pgfscope}%
\begin{pgfscope}%
\pgfpathrectangle{\pgfqpoint{0.594525in}{0.417642in}}{\pgfqpoint{3.423805in}{2.011535in}}%
\pgfusepath{clip}%
\pgfsetrectcap%
\pgfsetroundjoin%
\pgfsetlinewidth{0.803000pt}%
\definecolor{currentstroke}{rgb}{0.850000,0.850000,0.850000}%
\pgfsetstrokecolor{currentstroke}%
\pgfsetdash{}{0pt}%
\pgfpathmoveto{\pgfqpoint{0.594525in}{1.535312in}}%
\pgfpathlineto{\pgfqpoint{4.018330in}{1.535312in}}%
\pgfusepath{stroke}%
\end{pgfscope}%
\begin{pgfscope}%
\pgfsetbuttcap%
\pgfsetroundjoin%
\definecolor{currentfill}{rgb}{0.000000,0.000000,0.000000}%
\pgfsetfillcolor{currentfill}%
\pgfsetlinewidth{0.602250pt}%
\definecolor{currentstroke}{rgb}{0.000000,0.000000,0.000000}%
\pgfsetstrokecolor{currentstroke}%
\pgfsetdash{}{0pt}%
\pgfsys@defobject{currentmarker}{\pgfqpoint{-0.027778in}{0.000000in}}{\pgfqpoint{-0.000000in}{0.000000in}}{%
\pgfpathmoveto{\pgfqpoint{-0.000000in}{0.000000in}}%
\pgfpathlineto{\pgfqpoint{-0.027778in}{0.000000in}}%
\pgfusepath{stroke,fill}%
}%
\begin{pgfscope}%
\pgfsys@transformshift{0.594525in}{1.535312in}%
\pgfsys@useobject{currentmarker}{}%
\end{pgfscope}%
\end{pgfscope}%
\begin{pgfscope}%
\pgfpathrectangle{\pgfqpoint{0.594525in}{0.417642in}}{\pgfqpoint{3.423805in}{2.011535in}}%
\pgfusepath{clip}%
\pgfsetrectcap%
\pgfsetroundjoin%
\pgfsetlinewidth{0.803000pt}%
\definecolor{currentstroke}{rgb}{0.850000,0.850000,0.850000}%
\pgfsetstrokecolor{currentstroke}%
\pgfsetdash{}{0pt}%
\pgfpathmoveto{\pgfqpoint{0.594525in}{1.562245in}}%
\pgfpathlineto{\pgfqpoint{4.018330in}{1.562245in}}%
\pgfusepath{stroke}%
\end{pgfscope}%
\begin{pgfscope}%
\pgfsetbuttcap%
\pgfsetroundjoin%
\definecolor{currentfill}{rgb}{0.000000,0.000000,0.000000}%
\pgfsetfillcolor{currentfill}%
\pgfsetlinewidth{0.602250pt}%
\definecolor{currentstroke}{rgb}{0.000000,0.000000,0.000000}%
\pgfsetstrokecolor{currentstroke}%
\pgfsetdash{}{0pt}%
\pgfsys@defobject{currentmarker}{\pgfqpoint{-0.027778in}{0.000000in}}{\pgfqpoint{-0.000000in}{0.000000in}}{%
\pgfpathmoveto{\pgfqpoint{-0.000000in}{0.000000in}}%
\pgfpathlineto{\pgfqpoint{-0.027778in}{0.000000in}}%
\pgfusepath{stroke,fill}%
}%
\begin{pgfscope}%
\pgfsys@transformshift{0.594525in}{1.562245in}%
\pgfsys@useobject{currentmarker}{}%
\end{pgfscope}%
\end{pgfscope}%
\begin{pgfscope}%
\pgfpathrectangle{\pgfqpoint{0.594525in}{0.417642in}}{\pgfqpoint{3.423805in}{2.011535in}}%
\pgfusepath{clip}%
\pgfsetrectcap%
\pgfsetroundjoin%
\pgfsetlinewidth{0.803000pt}%
\definecolor{currentstroke}{rgb}{0.850000,0.850000,0.850000}%
\pgfsetstrokecolor{currentstroke}%
\pgfsetdash{}{0pt}%
\pgfpathmoveto{\pgfqpoint{0.594525in}{1.585576in}}%
\pgfpathlineto{\pgfqpoint{4.018330in}{1.585576in}}%
\pgfusepath{stroke}%
\end{pgfscope}%
\begin{pgfscope}%
\pgfsetbuttcap%
\pgfsetroundjoin%
\definecolor{currentfill}{rgb}{0.000000,0.000000,0.000000}%
\pgfsetfillcolor{currentfill}%
\pgfsetlinewidth{0.602250pt}%
\definecolor{currentstroke}{rgb}{0.000000,0.000000,0.000000}%
\pgfsetstrokecolor{currentstroke}%
\pgfsetdash{}{0pt}%
\pgfsys@defobject{currentmarker}{\pgfqpoint{-0.027778in}{0.000000in}}{\pgfqpoint{-0.000000in}{0.000000in}}{%
\pgfpathmoveto{\pgfqpoint{-0.000000in}{0.000000in}}%
\pgfpathlineto{\pgfqpoint{-0.027778in}{0.000000in}}%
\pgfusepath{stroke,fill}%
}%
\begin{pgfscope}%
\pgfsys@transformshift{0.594525in}{1.585576in}%
\pgfsys@useobject{currentmarker}{}%
\end{pgfscope}%
\end{pgfscope}%
\begin{pgfscope}%
\pgfpathrectangle{\pgfqpoint{0.594525in}{0.417642in}}{\pgfqpoint{3.423805in}{2.011535in}}%
\pgfusepath{clip}%
\pgfsetrectcap%
\pgfsetroundjoin%
\pgfsetlinewidth{0.803000pt}%
\definecolor{currentstroke}{rgb}{0.850000,0.850000,0.850000}%
\pgfsetstrokecolor{currentstroke}%
\pgfsetdash{}{0pt}%
\pgfpathmoveto{\pgfqpoint{0.594525in}{1.606155in}}%
\pgfpathlineto{\pgfqpoint{4.018330in}{1.606155in}}%
\pgfusepath{stroke}%
\end{pgfscope}%
\begin{pgfscope}%
\pgfsetbuttcap%
\pgfsetroundjoin%
\definecolor{currentfill}{rgb}{0.000000,0.000000,0.000000}%
\pgfsetfillcolor{currentfill}%
\pgfsetlinewidth{0.602250pt}%
\definecolor{currentstroke}{rgb}{0.000000,0.000000,0.000000}%
\pgfsetstrokecolor{currentstroke}%
\pgfsetdash{}{0pt}%
\pgfsys@defobject{currentmarker}{\pgfqpoint{-0.027778in}{0.000000in}}{\pgfqpoint{-0.000000in}{0.000000in}}{%
\pgfpathmoveto{\pgfqpoint{-0.000000in}{0.000000in}}%
\pgfpathlineto{\pgfqpoint{-0.027778in}{0.000000in}}%
\pgfusepath{stroke,fill}%
}%
\begin{pgfscope}%
\pgfsys@transformshift{0.594525in}{1.606155in}%
\pgfsys@useobject{currentmarker}{}%
\end{pgfscope}%
\end{pgfscope}%
\begin{pgfscope}%
\pgfpathrectangle{\pgfqpoint{0.594525in}{0.417642in}}{\pgfqpoint{3.423805in}{2.011535in}}%
\pgfusepath{clip}%
\pgfsetrectcap%
\pgfsetroundjoin%
\pgfsetlinewidth{0.803000pt}%
\definecolor{currentstroke}{rgb}{0.850000,0.850000,0.850000}%
\pgfsetstrokecolor{currentstroke}%
\pgfsetdash{}{0pt}%
\pgfpathmoveto{\pgfqpoint{0.594525in}{1.745670in}}%
\pgfpathlineto{\pgfqpoint{4.018330in}{1.745670in}}%
\pgfusepath{stroke}%
\end{pgfscope}%
\begin{pgfscope}%
\pgfsetbuttcap%
\pgfsetroundjoin%
\definecolor{currentfill}{rgb}{0.000000,0.000000,0.000000}%
\pgfsetfillcolor{currentfill}%
\pgfsetlinewidth{0.602250pt}%
\definecolor{currentstroke}{rgb}{0.000000,0.000000,0.000000}%
\pgfsetstrokecolor{currentstroke}%
\pgfsetdash{}{0pt}%
\pgfsys@defobject{currentmarker}{\pgfqpoint{-0.027778in}{0.000000in}}{\pgfqpoint{-0.000000in}{0.000000in}}{%
\pgfpathmoveto{\pgfqpoint{-0.000000in}{0.000000in}}%
\pgfpathlineto{\pgfqpoint{-0.027778in}{0.000000in}}%
\pgfusepath{stroke,fill}%
}%
\begin{pgfscope}%
\pgfsys@transformshift{0.594525in}{1.745670in}%
\pgfsys@useobject{currentmarker}{}%
\end{pgfscope}%
\end{pgfscope}%
\begin{pgfscope}%
\pgfpathrectangle{\pgfqpoint{0.594525in}{0.417642in}}{\pgfqpoint{3.423805in}{2.011535in}}%
\pgfusepath{clip}%
\pgfsetrectcap%
\pgfsetroundjoin%
\pgfsetlinewidth{0.803000pt}%
\definecolor{currentstroke}{rgb}{0.850000,0.850000,0.850000}%
\pgfsetstrokecolor{currentstroke}%
\pgfsetdash{}{0pt}%
\pgfpathmoveto{\pgfqpoint{0.594525in}{1.816512in}}%
\pgfpathlineto{\pgfqpoint{4.018330in}{1.816512in}}%
\pgfusepath{stroke}%
\end{pgfscope}%
\begin{pgfscope}%
\pgfsetbuttcap%
\pgfsetroundjoin%
\definecolor{currentfill}{rgb}{0.000000,0.000000,0.000000}%
\pgfsetfillcolor{currentfill}%
\pgfsetlinewidth{0.602250pt}%
\definecolor{currentstroke}{rgb}{0.000000,0.000000,0.000000}%
\pgfsetstrokecolor{currentstroke}%
\pgfsetdash{}{0pt}%
\pgfsys@defobject{currentmarker}{\pgfqpoint{-0.027778in}{0.000000in}}{\pgfqpoint{-0.000000in}{0.000000in}}{%
\pgfpathmoveto{\pgfqpoint{-0.000000in}{0.000000in}}%
\pgfpathlineto{\pgfqpoint{-0.027778in}{0.000000in}}%
\pgfusepath{stroke,fill}%
}%
\begin{pgfscope}%
\pgfsys@transformshift{0.594525in}{1.816512in}%
\pgfsys@useobject{currentmarker}{}%
\end{pgfscope}%
\end{pgfscope}%
\begin{pgfscope}%
\pgfpathrectangle{\pgfqpoint{0.594525in}{0.417642in}}{\pgfqpoint{3.423805in}{2.011535in}}%
\pgfusepath{clip}%
\pgfsetrectcap%
\pgfsetroundjoin%
\pgfsetlinewidth{0.803000pt}%
\definecolor{currentstroke}{rgb}{0.850000,0.850000,0.850000}%
\pgfsetstrokecolor{currentstroke}%
\pgfsetdash{}{0pt}%
\pgfpathmoveto{\pgfqpoint{0.594525in}{1.866776in}}%
\pgfpathlineto{\pgfqpoint{4.018330in}{1.866776in}}%
\pgfusepath{stroke}%
\end{pgfscope}%
\begin{pgfscope}%
\pgfsetbuttcap%
\pgfsetroundjoin%
\definecolor{currentfill}{rgb}{0.000000,0.000000,0.000000}%
\pgfsetfillcolor{currentfill}%
\pgfsetlinewidth{0.602250pt}%
\definecolor{currentstroke}{rgb}{0.000000,0.000000,0.000000}%
\pgfsetstrokecolor{currentstroke}%
\pgfsetdash{}{0pt}%
\pgfsys@defobject{currentmarker}{\pgfqpoint{-0.027778in}{0.000000in}}{\pgfqpoint{-0.000000in}{0.000000in}}{%
\pgfpathmoveto{\pgfqpoint{-0.000000in}{0.000000in}}%
\pgfpathlineto{\pgfqpoint{-0.027778in}{0.000000in}}%
\pgfusepath{stroke,fill}%
}%
\begin{pgfscope}%
\pgfsys@transformshift{0.594525in}{1.866776in}%
\pgfsys@useobject{currentmarker}{}%
\end{pgfscope}%
\end{pgfscope}%
\begin{pgfscope}%
\pgfpathrectangle{\pgfqpoint{0.594525in}{0.417642in}}{\pgfqpoint{3.423805in}{2.011535in}}%
\pgfusepath{clip}%
\pgfsetrectcap%
\pgfsetroundjoin%
\pgfsetlinewidth{0.803000pt}%
\definecolor{currentstroke}{rgb}{0.850000,0.850000,0.850000}%
\pgfsetstrokecolor{currentstroke}%
\pgfsetdash{}{0pt}%
\pgfpathmoveto{\pgfqpoint{0.594525in}{1.905764in}}%
\pgfpathlineto{\pgfqpoint{4.018330in}{1.905764in}}%
\pgfusepath{stroke}%
\end{pgfscope}%
\begin{pgfscope}%
\pgfsetbuttcap%
\pgfsetroundjoin%
\definecolor{currentfill}{rgb}{0.000000,0.000000,0.000000}%
\pgfsetfillcolor{currentfill}%
\pgfsetlinewidth{0.602250pt}%
\definecolor{currentstroke}{rgb}{0.000000,0.000000,0.000000}%
\pgfsetstrokecolor{currentstroke}%
\pgfsetdash{}{0pt}%
\pgfsys@defobject{currentmarker}{\pgfqpoint{-0.027778in}{0.000000in}}{\pgfqpoint{-0.000000in}{0.000000in}}{%
\pgfpathmoveto{\pgfqpoint{-0.000000in}{0.000000in}}%
\pgfpathlineto{\pgfqpoint{-0.027778in}{0.000000in}}%
\pgfusepath{stroke,fill}%
}%
\begin{pgfscope}%
\pgfsys@transformshift{0.594525in}{1.905764in}%
\pgfsys@useobject{currentmarker}{}%
\end{pgfscope}%
\end{pgfscope}%
\begin{pgfscope}%
\pgfpathrectangle{\pgfqpoint{0.594525in}{0.417642in}}{\pgfqpoint{3.423805in}{2.011535in}}%
\pgfusepath{clip}%
\pgfsetrectcap%
\pgfsetroundjoin%
\pgfsetlinewidth{0.803000pt}%
\definecolor{currentstroke}{rgb}{0.850000,0.850000,0.850000}%
\pgfsetstrokecolor{currentstroke}%
\pgfsetdash{}{0pt}%
\pgfpathmoveto{\pgfqpoint{0.594525in}{1.937619in}}%
\pgfpathlineto{\pgfqpoint{4.018330in}{1.937619in}}%
\pgfusepath{stroke}%
\end{pgfscope}%
\begin{pgfscope}%
\pgfsetbuttcap%
\pgfsetroundjoin%
\definecolor{currentfill}{rgb}{0.000000,0.000000,0.000000}%
\pgfsetfillcolor{currentfill}%
\pgfsetlinewidth{0.602250pt}%
\definecolor{currentstroke}{rgb}{0.000000,0.000000,0.000000}%
\pgfsetstrokecolor{currentstroke}%
\pgfsetdash{}{0pt}%
\pgfsys@defobject{currentmarker}{\pgfqpoint{-0.027778in}{0.000000in}}{\pgfqpoint{-0.000000in}{0.000000in}}{%
\pgfpathmoveto{\pgfqpoint{-0.000000in}{0.000000in}}%
\pgfpathlineto{\pgfqpoint{-0.027778in}{0.000000in}}%
\pgfusepath{stroke,fill}%
}%
\begin{pgfscope}%
\pgfsys@transformshift{0.594525in}{1.937619in}%
\pgfsys@useobject{currentmarker}{}%
\end{pgfscope}%
\end{pgfscope}%
\begin{pgfscope}%
\pgfpathrectangle{\pgfqpoint{0.594525in}{0.417642in}}{\pgfqpoint{3.423805in}{2.011535in}}%
\pgfusepath{clip}%
\pgfsetrectcap%
\pgfsetroundjoin%
\pgfsetlinewidth{0.803000pt}%
\definecolor{currentstroke}{rgb}{0.850000,0.850000,0.850000}%
\pgfsetstrokecolor{currentstroke}%
\pgfsetdash{}{0pt}%
\pgfpathmoveto{\pgfqpoint{0.594525in}{1.964552in}}%
\pgfpathlineto{\pgfqpoint{4.018330in}{1.964552in}}%
\pgfusepath{stroke}%
\end{pgfscope}%
\begin{pgfscope}%
\pgfsetbuttcap%
\pgfsetroundjoin%
\definecolor{currentfill}{rgb}{0.000000,0.000000,0.000000}%
\pgfsetfillcolor{currentfill}%
\pgfsetlinewidth{0.602250pt}%
\definecolor{currentstroke}{rgb}{0.000000,0.000000,0.000000}%
\pgfsetstrokecolor{currentstroke}%
\pgfsetdash{}{0pt}%
\pgfsys@defobject{currentmarker}{\pgfqpoint{-0.027778in}{0.000000in}}{\pgfqpoint{-0.000000in}{0.000000in}}{%
\pgfpathmoveto{\pgfqpoint{-0.000000in}{0.000000in}}%
\pgfpathlineto{\pgfqpoint{-0.027778in}{0.000000in}}%
\pgfusepath{stroke,fill}%
}%
\begin{pgfscope}%
\pgfsys@transformshift{0.594525in}{1.964552in}%
\pgfsys@useobject{currentmarker}{}%
\end{pgfscope}%
\end{pgfscope}%
\begin{pgfscope}%
\pgfpathrectangle{\pgfqpoint{0.594525in}{0.417642in}}{\pgfqpoint{3.423805in}{2.011535in}}%
\pgfusepath{clip}%
\pgfsetrectcap%
\pgfsetroundjoin%
\pgfsetlinewidth{0.803000pt}%
\definecolor{currentstroke}{rgb}{0.850000,0.850000,0.850000}%
\pgfsetstrokecolor{currentstroke}%
\pgfsetdash{}{0pt}%
\pgfpathmoveto{\pgfqpoint{0.594525in}{1.987883in}}%
\pgfpathlineto{\pgfqpoint{4.018330in}{1.987883in}}%
\pgfusepath{stroke}%
\end{pgfscope}%
\begin{pgfscope}%
\pgfsetbuttcap%
\pgfsetroundjoin%
\definecolor{currentfill}{rgb}{0.000000,0.000000,0.000000}%
\pgfsetfillcolor{currentfill}%
\pgfsetlinewidth{0.602250pt}%
\definecolor{currentstroke}{rgb}{0.000000,0.000000,0.000000}%
\pgfsetstrokecolor{currentstroke}%
\pgfsetdash{}{0pt}%
\pgfsys@defobject{currentmarker}{\pgfqpoint{-0.027778in}{0.000000in}}{\pgfqpoint{-0.000000in}{0.000000in}}{%
\pgfpathmoveto{\pgfqpoint{-0.000000in}{0.000000in}}%
\pgfpathlineto{\pgfqpoint{-0.027778in}{0.000000in}}%
\pgfusepath{stroke,fill}%
}%
\begin{pgfscope}%
\pgfsys@transformshift{0.594525in}{1.987883in}%
\pgfsys@useobject{currentmarker}{}%
\end{pgfscope}%
\end{pgfscope}%
\begin{pgfscope}%
\pgfpathrectangle{\pgfqpoint{0.594525in}{0.417642in}}{\pgfqpoint{3.423805in}{2.011535in}}%
\pgfusepath{clip}%
\pgfsetrectcap%
\pgfsetroundjoin%
\pgfsetlinewidth{0.803000pt}%
\definecolor{currentstroke}{rgb}{0.850000,0.850000,0.850000}%
\pgfsetstrokecolor{currentstroke}%
\pgfsetdash{}{0pt}%
\pgfpathmoveto{\pgfqpoint{0.594525in}{2.008462in}}%
\pgfpathlineto{\pgfqpoint{4.018330in}{2.008462in}}%
\pgfusepath{stroke}%
\end{pgfscope}%
\begin{pgfscope}%
\pgfsetbuttcap%
\pgfsetroundjoin%
\definecolor{currentfill}{rgb}{0.000000,0.000000,0.000000}%
\pgfsetfillcolor{currentfill}%
\pgfsetlinewidth{0.602250pt}%
\definecolor{currentstroke}{rgb}{0.000000,0.000000,0.000000}%
\pgfsetstrokecolor{currentstroke}%
\pgfsetdash{}{0pt}%
\pgfsys@defobject{currentmarker}{\pgfqpoint{-0.027778in}{0.000000in}}{\pgfqpoint{-0.000000in}{0.000000in}}{%
\pgfpathmoveto{\pgfqpoint{-0.000000in}{0.000000in}}%
\pgfpathlineto{\pgfqpoint{-0.027778in}{0.000000in}}%
\pgfusepath{stroke,fill}%
}%
\begin{pgfscope}%
\pgfsys@transformshift{0.594525in}{2.008462in}%
\pgfsys@useobject{currentmarker}{}%
\end{pgfscope}%
\end{pgfscope}%
\begin{pgfscope}%
\pgfpathrectangle{\pgfqpoint{0.594525in}{0.417642in}}{\pgfqpoint{3.423805in}{2.011535in}}%
\pgfusepath{clip}%
\pgfsetrectcap%
\pgfsetroundjoin%
\pgfsetlinewidth{0.803000pt}%
\definecolor{currentstroke}{rgb}{0.850000,0.850000,0.850000}%
\pgfsetstrokecolor{currentstroke}%
\pgfsetdash{}{0pt}%
\pgfpathmoveto{\pgfqpoint{0.594525in}{2.147977in}}%
\pgfpathlineto{\pgfqpoint{4.018330in}{2.147977in}}%
\pgfusepath{stroke}%
\end{pgfscope}%
\begin{pgfscope}%
\pgfsetbuttcap%
\pgfsetroundjoin%
\definecolor{currentfill}{rgb}{0.000000,0.000000,0.000000}%
\pgfsetfillcolor{currentfill}%
\pgfsetlinewidth{0.602250pt}%
\definecolor{currentstroke}{rgb}{0.000000,0.000000,0.000000}%
\pgfsetstrokecolor{currentstroke}%
\pgfsetdash{}{0pt}%
\pgfsys@defobject{currentmarker}{\pgfqpoint{-0.027778in}{0.000000in}}{\pgfqpoint{-0.000000in}{0.000000in}}{%
\pgfpathmoveto{\pgfqpoint{-0.000000in}{0.000000in}}%
\pgfpathlineto{\pgfqpoint{-0.027778in}{0.000000in}}%
\pgfusepath{stroke,fill}%
}%
\begin{pgfscope}%
\pgfsys@transformshift{0.594525in}{2.147977in}%
\pgfsys@useobject{currentmarker}{}%
\end{pgfscope}%
\end{pgfscope}%
\begin{pgfscope}%
\pgfpathrectangle{\pgfqpoint{0.594525in}{0.417642in}}{\pgfqpoint{3.423805in}{2.011535in}}%
\pgfusepath{clip}%
\pgfsetrectcap%
\pgfsetroundjoin%
\pgfsetlinewidth{0.803000pt}%
\definecolor{currentstroke}{rgb}{0.850000,0.850000,0.850000}%
\pgfsetstrokecolor{currentstroke}%
\pgfsetdash{}{0pt}%
\pgfpathmoveto{\pgfqpoint{0.594525in}{2.218819in}}%
\pgfpathlineto{\pgfqpoint{4.018330in}{2.218819in}}%
\pgfusepath{stroke}%
\end{pgfscope}%
\begin{pgfscope}%
\pgfsetbuttcap%
\pgfsetroundjoin%
\definecolor{currentfill}{rgb}{0.000000,0.000000,0.000000}%
\pgfsetfillcolor{currentfill}%
\pgfsetlinewidth{0.602250pt}%
\definecolor{currentstroke}{rgb}{0.000000,0.000000,0.000000}%
\pgfsetstrokecolor{currentstroke}%
\pgfsetdash{}{0pt}%
\pgfsys@defobject{currentmarker}{\pgfqpoint{-0.027778in}{0.000000in}}{\pgfqpoint{-0.000000in}{0.000000in}}{%
\pgfpathmoveto{\pgfqpoint{-0.000000in}{0.000000in}}%
\pgfpathlineto{\pgfqpoint{-0.027778in}{0.000000in}}%
\pgfusepath{stroke,fill}%
}%
\begin{pgfscope}%
\pgfsys@transformshift{0.594525in}{2.218819in}%
\pgfsys@useobject{currentmarker}{}%
\end{pgfscope}%
\end{pgfscope}%
\begin{pgfscope}%
\pgfpathrectangle{\pgfqpoint{0.594525in}{0.417642in}}{\pgfqpoint{3.423805in}{2.011535in}}%
\pgfusepath{clip}%
\pgfsetrectcap%
\pgfsetroundjoin%
\pgfsetlinewidth{0.803000pt}%
\definecolor{currentstroke}{rgb}{0.850000,0.850000,0.850000}%
\pgfsetstrokecolor{currentstroke}%
\pgfsetdash{}{0pt}%
\pgfpathmoveto{\pgfqpoint{0.594525in}{2.269083in}}%
\pgfpathlineto{\pgfqpoint{4.018330in}{2.269083in}}%
\pgfusepath{stroke}%
\end{pgfscope}%
\begin{pgfscope}%
\pgfsetbuttcap%
\pgfsetroundjoin%
\definecolor{currentfill}{rgb}{0.000000,0.000000,0.000000}%
\pgfsetfillcolor{currentfill}%
\pgfsetlinewidth{0.602250pt}%
\definecolor{currentstroke}{rgb}{0.000000,0.000000,0.000000}%
\pgfsetstrokecolor{currentstroke}%
\pgfsetdash{}{0pt}%
\pgfsys@defobject{currentmarker}{\pgfqpoint{-0.027778in}{0.000000in}}{\pgfqpoint{-0.000000in}{0.000000in}}{%
\pgfpathmoveto{\pgfqpoint{-0.000000in}{0.000000in}}%
\pgfpathlineto{\pgfqpoint{-0.027778in}{0.000000in}}%
\pgfusepath{stroke,fill}%
}%
\begin{pgfscope}%
\pgfsys@transformshift{0.594525in}{2.269083in}%
\pgfsys@useobject{currentmarker}{}%
\end{pgfscope}%
\end{pgfscope}%
\begin{pgfscope}%
\pgfpathrectangle{\pgfqpoint{0.594525in}{0.417642in}}{\pgfqpoint{3.423805in}{2.011535in}}%
\pgfusepath{clip}%
\pgfsetrectcap%
\pgfsetroundjoin%
\pgfsetlinewidth{0.803000pt}%
\definecolor{currentstroke}{rgb}{0.850000,0.850000,0.850000}%
\pgfsetstrokecolor{currentstroke}%
\pgfsetdash{}{0pt}%
\pgfpathmoveto{\pgfqpoint{0.594525in}{2.308071in}}%
\pgfpathlineto{\pgfqpoint{4.018330in}{2.308071in}}%
\pgfusepath{stroke}%
\end{pgfscope}%
\begin{pgfscope}%
\pgfsetbuttcap%
\pgfsetroundjoin%
\definecolor{currentfill}{rgb}{0.000000,0.000000,0.000000}%
\pgfsetfillcolor{currentfill}%
\pgfsetlinewidth{0.602250pt}%
\definecolor{currentstroke}{rgb}{0.000000,0.000000,0.000000}%
\pgfsetstrokecolor{currentstroke}%
\pgfsetdash{}{0pt}%
\pgfsys@defobject{currentmarker}{\pgfqpoint{-0.027778in}{0.000000in}}{\pgfqpoint{-0.000000in}{0.000000in}}{%
\pgfpathmoveto{\pgfqpoint{-0.000000in}{0.000000in}}%
\pgfpathlineto{\pgfqpoint{-0.027778in}{0.000000in}}%
\pgfusepath{stroke,fill}%
}%
\begin{pgfscope}%
\pgfsys@transformshift{0.594525in}{2.308071in}%
\pgfsys@useobject{currentmarker}{}%
\end{pgfscope}%
\end{pgfscope}%
\begin{pgfscope}%
\pgfpathrectangle{\pgfqpoint{0.594525in}{0.417642in}}{\pgfqpoint{3.423805in}{2.011535in}}%
\pgfusepath{clip}%
\pgfsetrectcap%
\pgfsetroundjoin%
\pgfsetlinewidth{0.803000pt}%
\definecolor{currentstroke}{rgb}{0.850000,0.850000,0.850000}%
\pgfsetstrokecolor{currentstroke}%
\pgfsetdash{}{0pt}%
\pgfpathmoveto{\pgfqpoint{0.594525in}{2.339926in}}%
\pgfpathlineto{\pgfqpoint{4.018330in}{2.339926in}}%
\pgfusepath{stroke}%
\end{pgfscope}%
\begin{pgfscope}%
\pgfsetbuttcap%
\pgfsetroundjoin%
\definecolor{currentfill}{rgb}{0.000000,0.000000,0.000000}%
\pgfsetfillcolor{currentfill}%
\pgfsetlinewidth{0.602250pt}%
\definecolor{currentstroke}{rgb}{0.000000,0.000000,0.000000}%
\pgfsetstrokecolor{currentstroke}%
\pgfsetdash{}{0pt}%
\pgfsys@defobject{currentmarker}{\pgfqpoint{-0.027778in}{0.000000in}}{\pgfqpoint{-0.000000in}{0.000000in}}{%
\pgfpathmoveto{\pgfqpoint{-0.000000in}{0.000000in}}%
\pgfpathlineto{\pgfqpoint{-0.027778in}{0.000000in}}%
\pgfusepath{stroke,fill}%
}%
\begin{pgfscope}%
\pgfsys@transformshift{0.594525in}{2.339926in}%
\pgfsys@useobject{currentmarker}{}%
\end{pgfscope}%
\end{pgfscope}%
\begin{pgfscope}%
\pgfpathrectangle{\pgfqpoint{0.594525in}{0.417642in}}{\pgfqpoint{3.423805in}{2.011535in}}%
\pgfusepath{clip}%
\pgfsetrectcap%
\pgfsetroundjoin%
\pgfsetlinewidth{0.803000pt}%
\definecolor{currentstroke}{rgb}{0.850000,0.850000,0.850000}%
\pgfsetstrokecolor{currentstroke}%
\pgfsetdash{}{0pt}%
\pgfpathmoveto{\pgfqpoint{0.594525in}{2.366859in}}%
\pgfpathlineto{\pgfqpoint{4.018330in}{2.366859in}}%
\pgfusepath{stroke}%
\end{pgfscope}%
\begin{pgfscope}%
\pgfsetbuttcap%
\pgfsetroundjoin%
\definecolor{currentfill}{rgb}{0.000000,0.000000,0.000000}%
\pgfsetfillcolor{currentfill}%
\pgfsetlinewidth{0.602250pt}%
\definecolor{currentstroke}{rgb}{0.000000,0.000000,0.000000}%
\pgfsetstrokecolor{currentstroke}%
\pgfsetdash{}{0pt}%
\pgfsys@defobject{currentmarker}{\pgfqpoint{-0.027778in}{0.000000in}}{\pgfqpoint{-0.000000in}{0.000000in}}{%
\pgfpathmoveto{\pgfqpoint{-0.000000in}{0.000000in}}%
\pgfpathlineto{\pgfqpoint{-0.027778in}{0.000000in}}%
\pgfusepath{stroke,fill}%
}%
\begin{pgfscope}%
\pgfsys@transformshift{0.594525in}{2.366859in}%
\pgfsys@useobject{currentmarker}{}%
\end{pgfscope}%
\end{pgfscope}%
\begin{pgfscope}%
\pgfpathrectangle{\pgfqpoint{0.594525in}{0.417642in}}{\pgfqpoint{3.423805in}{2.011535in}}%
\pgfusepath{clip}%
\pgfsetrectcap%
\pgfsetroundjoin%
\pgfsetlinewidth{0.803000pt}%
\definecolor{currentstroke}{rgb}{0.850000,0.850000,0.850000}%
\pgfsetstrokecolor{currentstroke}%
\pgfsetdash{}{0pt}%
\pgfpathmoveto{\pgfqpoint{0.594525in}{2.390190in}}%
\pgfpathlineto{\pgfqpoint{4.018330in}{2.390190in}}%
\pgfusepath{stroke}%
\end{pgfscope}%
\begin{pgfscope}%
\pgfsetbuttcap%
\pgfsetroundjoin%
\definecolor{currentfill}{rgb}{0.000000,0.000000,0.000000}%
\pgfsetfillcolor{currentfill}%
\pgfsetlinewidth{0.602250pt}%
\definecolor{currentstroke}{rgb}{0.000000,0.000000,0.000000}%
\pgfsetstrokecolor{currentstroke}%
\pgfsetdash{}{0pt}%
\pgfsys@defobject{currentmarker}{\pgfqpoint{-0.027778in}{0.000000in}}{\pgfqpoint{-0.000000in}{0.000000in}}{%
\pgfpathmoveto{\pgfqpoint{-0.000000in}{0.000000in}}%
\pgfpathlineto{\pgfqpoint{-0.027778in}{0.000000in}}%
\pgfusepath{stroke,fill}%
}%
\begin{pgfscope}%
\pgfsys@transformshift{0.594525in}{2.390190in}%
\pgfsys@useobject{currentmarker}{}%
\end{pgfscope}%
\end{pgfscope}%
\begin{pgfscope}%
\pgfpathrectangle{\pgfqpoint{0.594525in}{0.417642in}}{\pgfqpoint{3.423805in}{2.011535in}}%
\pgfusepath{clip}%
\pgfsetrectcap%
\pgfsetroundjoin%
\pgfsetlinewidth{0.803000pt}%
\definecolor{currentstroke}{rgb}{0.850000,0.850000,0.850000}%
\pgfsetstrokecolor{currentstroke}%
\pgfsetdash{}{0pt}%
\pgfpathmoveto{\pgfqpoint{0.594525in}{2.410769in}}%
\pgfpathlineto{\pgfqpoint{4.018330in}{2.410769in}}%
\pgfusepath{stroke}%
\end{pgfscope}%
\begin{pgfscope}%
\pgfsetbuttcap%
\pgfsetroundjoin%
\definecolor{currentfill}{rgb}{0.000000,0.000000,0.000000}%
\pgfsetfillcolor{currentfill}%
\pgfsetlinewidth{0.602250pt}%
\definecolor{currentstroke}{rgb}{0.000000,0.000000,0.000000}%
\pgfsetstrokecolor{currentstroke}%
\pgfsetdash{}{0pt}%
\pgfsys@defobject{currentmarker}{\pgfqpoint{-0.027778in}{0.000000in}}{\pgfqpoint{-0.000000in}{0.000000in}}{%
\pgfpathmoveto{\pgfqpoint{-0.000000in}{0.000000in}}%
\pgfpathlineto{\pgfqpoint{-0.027778in}{0.000000in}}%
\pgfusepath{stroke,fill}%
}%
\begin{pgfscope}%
\pgfsys@transformshift{0.594525in}{2.410769in}%
\pgfsys@useobject{currentmarker}{}%
\end{pgfscope}%
\end{pgfscope}%
\begin{pgfscope}%
\definecolor{textcolor}{rgb}{0.000000,0.000000,0.000000}%
\pgfsetstrokecolor{textcolor}%
\pgfsetfillcolor{textcolor}%
\pgftext[x=0.185574in,y=1.423410in,,bottom,rotate=90.000000]{\color{textcolor}\rmfamily\fontsize{10.000000}{12.000000}\selectfont  \(\displaystyle S_y(f)\) in \(\displaystyle \unit{1 \per \Hz}\)}%
\end{pgfscope}%
\begin{pgfscope}%
\pgfpathrectangle{\pgfqpoint{0.594525in}{0.417642in}}{\pgfqpoint{3.423805in}{2.011535in}}%
\pgfusepath{clip}%
\pgfsetbuttcap%
\pgfsetroundjoin%
\pgfsetlinewidth{1.505625pt}%
\definecolor{currentstroke}{rgb}{0.000000,0.447059,0.698039}%
\pgfsetstrokecolor{currentstroke}%
\pgfsetdash{{5.550000pt}{2.400000pt}}{0.000000pt}%
\pgfpathmoveto{\pgfqpoint{0.750152in}{1.101150in}}%
\pgfpathlineto{\pgfqpoint{0.913971in}{1.101150in}}%
\pgfpathlineto{\pgfqpoint{1.077789in}{1.101149in}}%
\pgfpathlineto{\pgfqpoint{1.241608in}{1.101149in}}%
\pgfpathlineto{\pgfqpoint{1.405426in}{1.101149in}}%
\pgfpathlineto{\pgfqpoint{1.569244in}{1.101147in}}%
\pgfpathlineto{\pgfqpoint{1.733063in}{1.101144in}}%
\pgfpathlineto{\pgfqpoint{1.896881in}{1.101134in}}%
\pgfpathlineto{\pgfqpoint{2.060700in}{1.101109in}}%
\pgfpathlineto{\pgfqpoint{2.224518in}{1.101043in}}%
\pgfpathlineto{\pgfqpoint{2.388337in}{1.100870in}}%
\pgfpathlineto{\pgfqpoint{2.552155in}{1.100413in}}%
\pgfpathlineto{\pgfqpoint{2.715973in}{1.099214in}}%
\pgfpathlineto{\pgfqpoint{2.879792in}{1.096091in}}%
\pgfpathlineto{\pgfqpoint{3.043610in}{1.088116in}}%
\pgfpathlineto{\pgfqpoint{3.207429in}{1.068682in}}%
\pgfpathlineto{\pgfqpoint{3.371247in}{1.025885in}}%
\pgfpathlineto{\pgfqpoint{3.535066in}{0.946760in}}%
\pgfpathlineto{\pgfqpoint{3.698884in}{0.829160in}}%
\pgfpathlineto{\pgfqpoint{3.862702in}{0.684273in}}%
\pgfusepath{stroke}%
\end{pgfscope}%
\begin{pgfscope}%
\pgfpathrectangle{\pgfqpoint{0.594525in}{0.417642in}}{\pgfqpoint{3.423805in}{2.011535in}}%
\pgfusepath{clip}%
\pgfsetbuttcap%
\pgfsetroundjoin%
\pgfsetlinewidth{1.505625pt}%
\definecolor{currentstroke}{rgb}{0.000000,0.619608,0.450980}%
\pgfsetstrokecolor{currentstroke}%
\pgfsetdash{{5.550000pt}{2.400000pt}}{0.000000pt}%
\pgfpathmoveto{\pgfqpoint{0.750152in}{1.503455in}}%
\pgfpathlineto{\pgfqpoint{0.913971in}{1.503452in}}%
\pgfpathlineto{\pgfqpoint{1.077789in}{1.503445in}}%
\pgfpathlineto{\pgfqpoint{1.241608in}{1.503425in}}%
\pgfpathlineto{\pgfqpoint{1.405426in}{1.503373in}}%
\pgfpathlineto{\pgfqpoint{1.569244in}{1.503237in}}%
\pgfpathlineto{\pgfqpoint{1.733063in}{1.502878in}}%
\pgfpathlineto{\pgfqpoint{1.896881in}{1.501936in}}%
\pgfpathlineto{\pgfqpoint{2.060700in}{1.499475in}}%
\pgfpathlineto{\pgfqpoint{2.224518in}{1.493147in}}%
\pgfpathlineto{\pgfqpoint{2.388337in}{1.477486in}}%
\pgfpathlineto{\pgfqpoint{2.552155in}{1.441876in}}%
\pgfpathlineto{\pgfqpoint{2.715973in}{1.372655in}}%
\pgfpathlineto{\pgfqpoint{2.879792in}{1.263993in}}%
\pgfpathlineto{\pgfqpoint{3.043610in}{1.124575in}}%
\pgfpathlineto{\pgfqpoint{3.207429in}{0.968046in}}%
\pgfpathlineto{\pgfqpoint{3.371247in}{0.803791in}}%
\pgfpathlineto{\pgfqpoint{3.535066in}{0.636387in}}%
\pgfpathlineto{\pgfqpoint{3.698884in}{0.467755in}}%
\pgfpathlineto{\pgfqpoint{3.757118in}{0.407642in}}%
\pgfusepath{stroke}%
\end{pgfscope}%
\begin{pgfscope}%
\pgfpathrectangle{\pgfqpoint{0.594525in}{0.417642in}}{\pgfqpoint{3.423805in}{2.011535in}}%
\pgfusepath{clip}%
\pgfsetbuttcap%
\pgfsetroundjoin%
\pgfsetlinewidth{1.505625pt}%
\definecolor{currentstroke}{rgb}{0.835294,0.368627,0.000000}%
\pgfsetstrokecolor{currentstroke}%
\pgfsetdash{{5.550000pt}{2.400000pt}}{0.000000pt}%
\pgfpathmoveto{\pgfqpoint{0.750152in}{1.905591in}}%
\pgfpathlineto{\pgfqpoint{0.913971in}{1.905310in}}%
\pgfpathlineto{\pgfqpoint{1.077789in}{1.904569in}}%
\pgfpathlineto{\pgfqpoint{1.241608in}{1.902631in}}%
\pgfpathlineto{\pgfqpoint{1.405426in}{1.897622in}}%
\pgfpathlineto{\pgfqpoint{1.569244in}{1.885066in}}%
\pgfpathlineto{\pgfqpoint{1.733063in}{1.855727in}}%
\pgfpathlineto{\pgfqpoint{1.896881in}{1.795997in}}%
\pgfpathlineto{\pgfqpoint{2.060700in}{1.696883in}}%
\pgfpathlineto{\pgfqpoint{2.224518in}{1.563834in}}%
\pgfpathlineto{\pgfqpoint{2.388337in}{1.410478in}}%
\pgfpathlineto{\pgfqpoint{2.552155in}{1.247574in}}%
\pgfpathlineto{\pgfqpoint{2.715973in}{1.080708in}}%
\pgfpathlineto{\pgfqpoint{2.879792in}{0.912283in}}%
\pgfpathlineto{\pgfqpoint{3.043610in}{0.743258in}}%
\pgfpathlineto{\pgfqpoint{3.207429in}{0.574006in}}%
\pgfpathlineto{\pgfqpoint{3.368369in}{0.407642in}}%
\pgfusepath{stroke}%
\end{pgfscope}%
\begin{pgfscope}%
\pgfpathrectangle{\pgfqpoint{0.594525in}{0.417642in}}{\pgfqpoint{3.423805in}{2.011535in}}%
\pgfusepath{clip}%
\pgfsetbuttcap%
\pgfsetroundjoin%
\pgfsetlinewidth{1.505625pt}%
\definecolor{currentstroke}{rgb}{0.800000,0.474510,0.654902}%
\pgfsetstrokecolor{currentstroke}%
\pgfsetdash{{5.550000pt}{2.400000pt}}{0.000000pt}%
\pgfpathmoveto{\pgfqpoint{0.750152in}{2.291625in}}%
\pgfpathlineto{\pgfqpoint{0.913971in}{2.267659in}}%
\pgfpathlineto{\pgfqpoint{1.077789in}{2.216791in}}%
\pgfpathlineto{\pgfqpoint{1.241608in}{2.127610in}}%
\pgfpathlineto{\pgfqpoint{1.405426in}{2.001847in}}%
\pgfpathlineto{\pgfqpoint{1.569244in}{1.852341in}}%
\pgfpathlineto{\pgfqpoint{1.733063in}{1.691125in}}%
\pgfpathlineto{\pgfqpoint{1.896881in}{1.524937in}}%
\pgfpathlineto{\pgfqpoint{2.060700in}{1.356775in}}%
\pgfpathlineto{\pgfqpoint{2.224518in}{1.187852in}}%
\pgfpathlineto{\pgfqpoint{2.388337in}{1.018638in}}%
\pgfpathlineto{\pgfqpoint{2.552155in}{0.849313in}}%
\pgfpathlineto{\pgfqpoint{2.715973in}{0.679946in}}%
\pgfpathlineto{\pgfqpoint{2.879792in}{0.510563in}}%
\pgfpathlineto{\pgfqpoint{2.979329in}{0.407642in}}%
\pgfusepath{stroke}%
\end{pgfscope}%
\begin{pgfscope}%
\pgfpathrectangle{\pgfqpoint{0.594525in}{0.417642in}}{\pgfqpoint{3.423805in}{2.011535in}}%
\pgfusepath{clip}%
\pgfsetrectcap%
\pgfsetroundjoin%
\pgfsetlinewidth{1.505625pt}%
\definecolor{currentstroke}{rgb}{0.000000,0.000000,0.000000}%
\pgfsetstrokecolor{currentstroke}%
\pgfsetdash{}{0pt}%
\pgfpathmoveto{\pgfqpoint{0.750152in}{2.311714in}}%
\pgfpathlineto{\pgfqpoint{0.913971in}{2.290485in}}%
\pgfpathlineto{\pgfqpoint{1.077789in}{2.246597in}}%
\pgfpathlineto{\pgfqpoint{1.241608in}{2.174364in}}%
\pgfpathlineto{\pgfqpoint{1.405426in}{2.085506in}}%
\pgfpathlineto{\pgfqpoint{1.569244in}{2.002007in}}%
\pgfpathlineto{\pgfqpoint{1.733063in}{1.930696in}}%
\pgfpathlineto{\pgfqpoint{1.896881in}{1.856833in}}%
\pgfpathlineto{\pgfqpoint{2.060700in}{1.767596in}}%
\pgfpathlineto{\pgfqpoint{2.224518in}{1.671721in}}%
\pgfpathlineto{\pgfqpoint{2.388337in}{1.586817in}}%
\pgfpathlineto{\pgfqpoint{2.552155in}{1.513206in}}%
\pgfpathlineto{\pgfqpoint{2.715973in}{1.433444in}}%
\pgfpathlineto{\pgfqpoint{2.879792in}{1.338240in}}%
\pgfpathlineto{\pgfqpoint{3.043610in}{1.239973in}}%
\pgfpathlineto{\pgfqpoint{3.207429in}{1.153723in}}%
\pgfpathlineto{\pgfqpoint{3.371247in}{1.073322in}}%
\pgfpathlineto{\pgfqpoint{3.535066in}{0.976858in}}%
\pgfpathlineto{\pgfqpoint{3.698884in}{0.852102in}}%
\pgfpathlineto{\pgfqpoint{3.862702in}{0.704407in}}%
\pgfusepath{stroke}%
\end{pgfscope}%
\begin{pgfscope}%
\pgfsetrectcap%
\pgfsetmiterjoin%
\pgfsetlinewidth{0.803000pt}%
\definecolor{currentstroke}{rgb}{0.000000,0.000000,0.000000}%
\pgfsetstrokecolor{currentstroke}%
\pgfsetdash{}{0pt}%
\pgfpathmoveto{\pgfqpoint{0.594525in}{0.417642in}}%
\pgfpathlineto{\pgfqpoint{0.594525in}{2.429177in}}%
\pgfusepath{stroke}%
\end{pgfscope}%
\begin{pgfscope}%
\pgfsetrectcap%
\pgfsetmiterjoin%
\pgfsetlinewidth{0.803000pt}%
\definecolor{currentstroke}{rgb}{0.000000,0.000000,0.000000}%
\pgfsetstrokecolor{currentstroke}%
\pgfsetdash{}{0pt}%
\pgfpathmoveto{\pgfqpoint{4.018330in}{0.417642in}}%
\pgfpathlineto{\pgfqpoint{4.018330in}{2.429177in}}%
\pgfusepath{stroke}%
\end{pgfscope}%
\begin{pgfscope}%
\pgfsetrectcap%
\pgfsetmiterjoin%
\pgfsetlinewidth{0.803000pt}%
\definecolor{currentstroke}{rgb}{0.000000,0.000000,0.000000}%
\pgfsetstrokecolor{currentstroke}%
\pgfsetdash{}{0pt}%
\pgfpathmoveto{\pgfqpoint{0.594525in}{0.417642in}}%
\pgfpathlineto{\pgfqpoint{4.018330in}{0.417642in}}%
\pgfusepath{stroke}%
\end{pgfscope}%
\begin{pgfscope}%
\pgfsetrectcap%
\pgfsetmiterjoin%
\pgfsetlinewidth{0.803000pt}%
\definecolor{currentstroke}{rgb}{0.000000,0.000000,0.000000}%
\pgfsetstrokecolor{currentstroke}%
\pgfsetdash{}{0pt}%
\pgfpathmoveto{\pgfqpoint{0.594525in}{2.429177in}}%
\pgfpathlineto{\pgfqpoint{4.018330in}{2.429177in}}%
\pgfusepath{stroke}%
\end{pgfscope}%
\begin{pgfscope}%
\pgfsetbuttcap%
\pgfsetmiterjoin%
\definecolor{currentfill}{rgb}{1.000000,1.000000,1.000000}%
\pgfsetfillcolor{currentfill}%
\pgfsetfillopacity{0.800000}%
\pgfsetlinewidth{1.003750pt}%
\definecolor{currentstroke}{rgb}{0.800000,0.800000,0.800000}%
\pgfsetstrokecolor{currentstroke}%
\pgfsetstrokeopacity{0.800000}%
\pgfsetdash{}{0pt}%
\pgfpathmoveto{\pgfqpoint{0.672303in}{0.473197in}}%
\pgfpathlineto{\pgfqpoint{1.839313in}{0.473197in}}%
\pgfpathquadraticcurveto{\pgfqpoint{1.861536in}{0.473197in}}{\pgfqpoint{1.861536in}{0.495420in}}%
\pgfpathlineto{\pgfqpoint{1.861536in}{1.258752in}}%
\pgfpathquadraticcurveto{\pgfqpoint{1.861536in}{1.280975in}}{\pgfqpoint{1.839313in}{1.280975in}}%
\pgfpathlineto{\pgfqpoint{0.672303in}{1.280975in}}%
\pgfpathquadraticcurveto{\pgfqpoint{0.650080in}{1.280975in}}{\pgfqpoint{0.650080in}{1.258752in}}%
\pgfpathlineto{\pgfqpoint{0.650080in}{0.495420in}}%
\pgfpathquadraticcurveto{\pgfqpoint{0.650080in}{0.473197in}}{\pgfqpoint{0.672303in}{0.473197in}}%
\pgfpathlineto{\pgfqpoint{0.672303in}{0.473197in}}%
\pgfpathclose%
\pgfusepath{stroke,fill}%
\end{pgfscope}%
\begin{pgfscope}%
\pgfsetbuttcap%
\pgfsetroundjoin%
\pgfsetlinewidth{1.505625pt}%
\definecolor{currentstroke}{rgb}{0.000000,0.447059,0.698039}%
\pgfsetstrokecolor{currentstroke}%
\pgfsetdash{{5.550000pt}{2.400000pt}}{0.000000pt}%
\pgfpathmoveto{\pgfqpoint{0.694525in}{1.197641in}}%
\pgfpathlineto{\pgfqpoint{0.805636in}{1.197641in}}%
\pgfpathlineto{\pgfqpoint{0.916747in}{1.197641in}}%
\pgfusepath{stroke}%
\end{pgfscope}%
\begin{pgfscope}%
\definecolor{textcolor}{rgb}{0.000000,0.000000,0.000000}%
\pgfsetstrokecolor{textcolor}%
\pgfsetfillcolor{textcolor}%
\pgftext[x=1.005636in,y=1.158752in,left,base]{\color{textcolor}\rmfamily\fontsize{8.000000}{9.600000}\selectfont \(\displaystyle \bar\tau_1=\tau_0=\qty{0.01}{\s}\)}%
\end{pgfscope}%
\begin{pgfscope}%
\pgfsetbuttcap%
\pgfsetroundjoin%
\pgfsetlinewidth{1.505625pt}%
\definecolor{currentstroke}{rgb}{0.000000,0.619608,0.450980}%
\pgfsetstrokecolor{currentstroke}%
\pgfsetdash{{5.550000pt}{2.400000pt}}{0.000000pt}%
\pgfpathmoveto{\pgfqpoint{0.694525in}{1.042752in}}%
\pgfpathlineto{\pgfqpoint{0.805636in}{1.042752in}}%
\pgfpathlineto{\pgfqpoint{0.916747in}{1.042752in}}%
\pgfusepath{stroke}%
\end{pgfscope}%
\begin{pgfscope}%
\definecolor{textcolor}{rgb}{0.000000,0.000000,0.000000}%
\pgfsetstrokecolor{textcolor}%
\pgfsetfillcolor{textcolor}%
\pgftext[x=1.005636in,y=1.003864in,left,base]{\color{textcolor}\rmfamily\fontsize{8.000000}{9.600000}\selectfont \(\displaystyle \bar\tau_1=\tau_0=\qty{0.1}{\s}\)}%
\end{pgfscope}%
\begin{pgfscope}%
\pgfsetbuttcap%
\pgfsetroundjoin%
\pgfsetlinewidth{1.505625pt}%
\definecolor{currentstroke}{rgb}{0.835294,0.368627,0.000000}%
\pgfsetstrokecolor{currentstroke}%
\pgfsetdash{{5.550000pt}{2.400000pt}}{0.000000pt}%
\pgfpathmoveto{\pgfqpoint{0.694525in}{0.887864in}}%
\pgfpathlineto{\pgfqpoint{0.805636in}{0.887864in}}%
\pgfpathlineto{\pgfqpoint{0.916747in}{0.887864in}}%
\pgfusepath{stroke}%
\end{pgfscope}%
\begin{pgfscope}%
\definecolor{textcolor}{rgb}{0.000000,0.000000,0.000000}%
\pgfsetstrokecolor{textcolor}%
\pgfsetfillcolor{textcolor}%
\pgftext[x=1.005636in,y=0.848975in,left,base]{\color{textcolor}\rmfamily\fontsize{8.000000}{9.600000}\selectfont \(\displaystyle \bar\tau_1=\tau_0=\qty{1}{\s}\)}%
\end{pgfscope}%
\begin{pgfscope}%
\pgfsetbuttcap%
\pgfsetroundjoin%
\pgfsetlinewidth{1.505625pt}%
\definecolor{currentstroke}{rgb}{0.800000,0.474510,0.654902}%
\pgfsetstrokecolor{currentstroke}%
\pgfsetdash{{5.550000pt}{2.400000pt}}{0.000000pt}%
\pgfpathmoveto{\pgfqpoint{0.694525in}{0.732975in}}%
\pgfpathlineto{\pgfqpoint{0.805636in}{0.732975in}}%
\pgfpathlineto{\pgfqpoint{0.916747in}{0.732975in}}%
\pgfusepath{stroke}%
\end{pgfscope}%
\begin{pgfscope}%
\definecolor{textcolor}{rgb}{0.000000,0.000000,0.000000}%
\pgfsetstrokecolor{textcolor}%
\pgfsetfillcolor{textcolor}%
\pgftext[x=1.005636in,y=0.694086in,left,base]{\color{textcolor}\rmfamily\fontsize{8.000000}{9.600000}\selectfont \(\displaystyle \bar\tau_1=\tau_0=\qty{10}{\s}\)}%
\end{pgfscope}%
\begin{pgfscope}%
\pgfsetrectcap%
\pgfsetroundjoin%
\pgfsetlinewidth{1.505625pt}%
\definecolor{currentstroke}{rgb}{0.000000,0.000000,0.000000}%
\pgfsetstrokecolor{currentstroke}%
\pgfsetdash{}{0pt}%
\pgfpathmoveto{\pgfqpoint{0.694525in}{0.578086in}}%
\pgfpathlineto{\pgfqpoint{0.805636in}{0.578086in}}%
\pgfpathlineto{\pgfqpoint{0.916747in}{0.578086in}}%
\pgfusepath{stroke}%
\end{pgfscope}%
\begin{pgfscope}%
\definecolor{textcolor}{rgb}{0.000000,0.000000,0.000000}%
\pgfsetstrokecolor{textcolor}%
\pgfsetfillcolor{textcolor}%
\pgftext[x=1.005636in,y=0.539197in,left,base]{\color{textcolor}\rmfamily\fontsize{8.000000}{9.600000}\selectfont Envelope}%
\end{pgfscope}%
\end{pgfpicture}%
\makeatother%
\endgroup%

    \caption{Multiple overlapping Lorentzian noise sources forming a $\frac 1 f$-like shape.}
    \label{fig:flicker_noise_evelope}
\end{figure}

Given that no trap site can store an electron indefinetely, the number of trap sites $N$ with a certain time constant $\frac 1 2 \bar \tau = \bar \tau_0 = \bar \tau_1$ must decline for longer time scales. Assuming $N$ is inversely proportional to the time constant $\bar \tau$
\begin{equation}
    N(\tau) \propto \frac{1}{\bar \tau}\,, \label{eqn:flicker_noise_weight_function}
\end{equation}

which can be motivated if the trapping process is thermally activated \cite{1_f_noise_motivation} and using equation \ref{eqn:burst_noise_lorentzian} from the previous section, multiplying the weight function \ref{eqn:flicker_noise_weight_function} and integrating over all possible storage times gives:

\begin{align}
    S(\omega) &= \lim_{t \to \infty} \int_0^t N(\bar \tau) \, 4 R_{xx}(0) \frac{\bar \tau}{1 + \omega^2 \bar \tau^2} \, d\bar\tau \nonumber\\
    \overset{\bar \tau_0 = \bar \tau_1}&{=} 4 R_{xx}(0)\, C_N \lim_{t \to \infty} \int_0^t \frac{1}{1 + \omega^2 \bar\tau^2} \, d\bar\tau \nonumber\\
    &= \frac{4 R_{xx}(0)\, C_N}{\omega} \lim_{t \to \infty}  \arctan{\bar\tau \omega} \Big|_{\bar\tau=0}^t \nonumber\\
    &= \frac{4 R_{xx}(0)\, C_N}{\omega} \cdot \frac{\pi}{2} \nonumber\\
    &= \frac{2 \pi R_{xx}(0)\, C_N}{\omega}\\
    S(f) &= h_{-1} f^{-1}
\end{align}

$C_N$ is the proportionality constant of \ref{eqn:flicker_noise_weight_function} and $h_{-1}$ is the power coefficient introduced in \ref{eqn:power_law}. This shows, that for a large number of distributed trap sites, a noise spectrum of $f^{-1}$ is found.

Using equation \ref{eqn:psd_to_adev}, the Allan variance can be calculated from the power spectral density:
\begin{align}
    \sigma_A^2(\tau) &= 2 h_{-1} \int_0^\infty \frac{1}{f} \frac{\sin^4\left( \pi f \tau \right)}{(\pi f \tau)^2}\,df \nonumber\\
    &=2 \ln 2 \, h_{-1}
\end{align}

Again, using the \textit{AllanTools} library \cite{allantools}, flicker noise was simulated to give an impresion of its properties.

\begin{figure}[ht]
    \centering
    \begin{subfigure}{0.32\linewidth}
        \centering
        \scalebox{0.75}{%
            %% Creator: Matplotlib, PGF backend
%%
%% To include the figure in your LaTeX document, write
%%   \input{<filename>.pgf}
%%
%% Make sure the required packages are loaded in your preamble
%%   \usepackage{pgf}
%%
%% Also ensure that all the required font packages are loaded; for instance,
%% the lmodern package is sometimes necessary when using math font.
%%   \usepackage{lmodern}
%%
%% Figures using additional raster images can only be included by \input if
%% they are in the same directory as the main LaTeX file. For loading figures
%% from other directories you can use the `import` package
%%   \usepackage{import}
%%
%% and then include the figures with
%%   \import{<path to file>}{<filename>.pgf}
%%
%% Matplotlib used the following preamble
%%   \usepackage{siunitx}
%%   \usepackage{fontspec}
%%   \makeatletter\@ifpackageloaded{underscore}{}{\usepackage[strings]{underscore}}\makeatother
%%
\begingroup%
\makeatletter%
\begin{pgfpicture}%
\pgfpathrectangle{\pgfpointorigin}{\pgfqpoint{2.440000in}{1.830000in}}%
\pgfusepath{use as bounding box, clip}%
\begin{pgfscope}%
\pgfsetbuttcap%
\pgfsetmiterjoin%
\definecolor{currentfill}{rgb}{1.000000,1.000000,1.000000}%
\pgfsetfillcolor{currentfill}%
\pgfsetlinewidth{0.000000pt}%
\definecolor{currentstroke}{rgb}{1.000000,1.000000,1.000000}%
\pgfsetstrokecolor{currentstroke}%
\pgfsetdash{}{0pt}%
\pgfpathmoveto{\pgfqpoint{0.000000in}{0.000000in}}%
\pgfpathlineto{\pgfqpoint{2.440000in}{0.000000in}}%
\pgfpathlineto{\pgfqpoint{2.440000in}{1.830000in}}%
\pgfpathlineto{\pgfqpoint{0.000000in}{1.830000in}}%
\pgfpathlineto{\pgfqpoint{0.000000in}{0.000000in}}%
\pgfpathclose%
\pgfusepath{fill}%
\end{pgfscope}%
\begin{pgfscope}%
\pgfsetbuttcap%
\pgfsetmiterjoin%
\definecolor{currentfill}{rgb}{1.000000,1.000000,1.000000}%
\pgfsetfillcolor{currentfill}%
\pgfsetlinewidth{0.000000pt}%
\definecolor{currentstroke}{rgb}{0.000000,0.000000,0.000000}%
\pgfsetstrokecolor{currentstroke}%
\pgfsetstrokeopacity{0.000000}%
\pgfsetdash{}{0pt}%
\pgfpathmoveto{\pgfqpoint{0.530716in}{0.416447in}}%
\pgfpathlineto{\pgfqpoint{2.398330in}{0.416447in}}%
\pgfpathlineto{\pgfqpoint{2.398330in}{1.788330in}}%
\pgfpathlineto{\pgfqpoint{0.530716in}{1.788330in}}%
\pgfpathlineto{\pgfqpoint{0.530716in}{0.416447in}}%
\pgfpathclose%
\pgfusepath{fill}%
\end{pgfscope}%
\begin{pgfscope}%
\pgfpathrectangle{\pgfqpoint{0.530716in}{0.416447in}}{\pgfqpoint{1.867614in}{1.371882in}}%
\pgfusepath{clip}%
\pgfsetrectcap%
\pgfsetroundjoin%
\pgfsetlinewidth{0.803000pt}%
\definecolor{currentstroke}{rgb}{0.450000,0.450000,0.450000}%
\pgfsetstrokecolor{currentstroke}%
\pgfsetdash{}{0pt}%
\pgfpathmoveto{\pgfqpoint{0.615608in}{0.416447in}}%
\pgfpathlineto{\pgfqpoint{0.615608in}{1.788330in}}%
\pgfusepath{stroke}%
\end{pgfscope}%
\begin{pgfscope}%
\pgfsetbuttcap%
\pgfsetroundjoin%
\definecolor{currentfill}{rgb}{0.000000,0.000000,0.000000}%
\pgfsetfillcolor{currentfill}%
\pgfsetlinewidth{0.803000pt}%
\definecolor{currentstroke}{rgb}{0.000000,0.000000,0.000000}%
\pgfsetstrokecolor{currentstroke}%
\pgfsetdash{}{0pt}%
\pgfsys@defobject{currentmarker}{\pgfqpoint{0.000000in}{-0.048611in}}{\pgfqpoint{0.000000in}{0.000000in}}{%
\pgfpathmoveto{\pgfqpoint{0.000000in}{0.000000in}}%
\pgfpathlineto{\pgfqpoint{0.000000in}{-0.048611in}}%
\pgfusepath{stroke,fill}%
}%
\begin{pgfscope}%
\pgfsys@transformshift{0.615608in}{0.416447in}%
\pgfsys@useobject{currentmarker}{}%
\end{pgfscope}%
\end{pgfscope}%
\begin{pgfscope}%
\definecolor{textcolor}{rgb}{0.000000,0.000000,0.000000}%
\pgfsetstrokecolor{textcolor}%
\pgfsetfillcolor{textcolor}%
\pgftext[x=0.615608in,y=0.319225in,,top]{\color{textcolor}\rmfamily\fontsize{8.000000}{9.600000}\selectfont \(\displaystyle {0}\)}%
\end{pgfscope}%
\begin{pgfscope}%
\pgfpathrectangle{\pgfqpoint{0.530716in}{0.416447in}}{\pgfqpoint{1.867614in}{1.371882in}}%
\pgfusepath{clip}%
\pgfsetrectcap%
\pgfsetroundjoin%
\pgfsetlinewidth{0.803000pt}%
\definecolor{currentstroke}{rgb}{0.450000,0.450000,0.450000}%
\pgfsetstrokecolor{currentstroke}%
\pgfsetdash{}{0pt}%
\pgfpathmoveto{\pgfqpoint{1.133808in}{0.416447in}}%
\pgfpathlineto{\pgfqpoint{1.133808in}{1.788330in}}%
\pgfusepath{stroke}%
\end{pgfscope}%
\begin{pgfscope}%
\pgfsetbuttcap%
\pgfsetroundjoin%
\definecolor{currentfill}{rgb}{0.000000,0.000000,0.000000}%
\pgfsetfillcolor{currentfill}%
\pgfsetlinewidth{0.803000pt}%
\definecolor{currentstroke}{rgb}{0.000000,0.000000,0.000000}%
\pgfsetstrokecolor{currentstroke}%
\pgfsetdash{}{0pt}%
\pgfsys@defobject{currentmarker}{\pgfqpoint{0.000000in}{-0.048611in}}{\pgfqpoint{0.000000in}{0.000000in}}{%
\pgfpathmoveto{\pgfqpoint{0.000000in}{0.000000in}}%
\pgfpathlineto{\pgfqpoint{0.000000in}{-0.048611in}}%
\pgfusepath{stroke,fill}%
}%
\begin{pgfscope}%
\pgfsys@transformshift{1.133808in}{0.416447in}%
\pgfsys@useobject{currentmarker}{}%
\end{pgfscope}%
\end{pgfscope}%
\begin{pgfscope}%
\definecolor{textcolor}{rgb}{0.000000,0.000000,0.000000}%
\pgfsetstrokecolor{textcolor}%
\pgfsetfillcolor{textcolor}%
\pgftext[x=1.133808in,y=0.319225in,,top]{\color{textcolor}\rmfamily\fontsize{8.000000}{9.600000}\selectfont \(\displaystyle {5000}\)}%
\end{pgfscope}%
\begin{pgfscope}%
\pgfpathrectangle{\pgfqpoint{0.530716in}{0.416447in}}{\pgfqpoint{1.867614in}{1.371882in}}%
\pgfusepath{clip}%
\pgfsetrectcap%
\pgfsetroundjoin%
\pgfsetlinewidth{0.803000pt}%
\definecolor{currentstroke}{rgb}{0.450000,0.450000,0.450000}%
\pgfsetstrokecolor{currentstroke}%
\pgfsetdash{}{0pt}%
\pgfpathmoveto{\pgfqpoint{1.652008in}{0.416447in}}%
\pgfpathlineto{\pgfqpoint{1.652008in}{1.788330in}}%
\pgfusepath{stroke}%
\end{pgfscope}%
\begin{pgfscope}%
\pgfsetbuttcap%
\pgfsetroundjoin%
\definecolor{currentfill}{rgb}{0.000000,0.000000,0.000000}%
\pgfsetfillcolor{currentfill}%
\pgfsetlinewidth{0.803000pt}%
\definecolor{currentstroke}{rgb}{0.000000,0.000000,0.000000}%
\pgfsetstrokecolor{currentstroke}%
\pgfsetdash{}{0pt}%
\pgfsys@defobject{currentmarker}{\pgfqpoint{0.000000in}{-0.048611in}}{\pgfqpoint{0.000000in}{0.000000in}}{%
\pgfpathmoveto{\pgfqpoint{0.000000in}{0.000000in}}%
\pgfpathlineto{\pgfqpoint{0.000000in}{-0.048611in}}%
\pgfusepath{stroke,fill}%
}%
\begin{pgfscope}%
\pgfsys@transformshift{1.652008in}{0.416447in}%
\pgfsys@useobject{currentmarker}{}%
\end{pgfscope}%
\end{pgfscope}%
\begin{pgfscope}%
\definecolor{textcolor}{rgb}{0.000000,0.000000,0.000000}%
\pgfsetstrokecolor{textcolor}%
\pgfsetfillcolor{textcolor}%
\pgftext[x=1.652008in,y=0.319225in,,top]{\color{textcolor}\rmfamily\fontsize{8.000000}{9.600000}\selectfont \(\displaystyle {10000}\)}%
\end{pgfscope}%
\begin{pgfscope}%
\pgfpathrectangle{\pgfqpoint{0.530716in}{0.416447in}}{\pgfqpoint{1.867614in}{1.371882in}}%
\pgfusepath{clip}%
\pgfsetrectcap%
\pgfsetroundjoin%
\pgfsetlinewidth{0.803000pt}%
\definecolor{currentstroke}{rgb}{0.450000,0.450000,0.450000}%
\pgfsetstrokecolor{currentstroke}%
\pgfsetdash{}{0pt}%
\pgfpathmoveto{\pgfqpoint{2.170208in}{0.416447in}}%
\pgfpathlineto{\pgfqpoint{2.170208in}{1.788330in}}%
\pgfusepath{stroke}%
\end{pgfscope}%
\begin{pgfscope}%
\pgfsetbuttcap%
\pgfsetroundjoin%
\definecolor{currentfill}{rgb}{0.000000,0.000000,0.000000}%
\pgfsetfillcolor{currentfill}%
\pgfsetlinewidth{0.803000pt}%
\definecolor{currentstroke}{rgb}{0.000000,0.000000,0.000000}%
\pgfsetstrokecolor{currentstroke}%
\pgfsetdash{}{0pt}%
\pgfsys@defobject{currentmarker}{\pgfqpoint{0.000000in}{-0.048611in}}{\pgfqpoint{0.000000in}{0.000000in}}{%
\pgfpathmoveto{\pgfqpoint{0.000000in}{0.000000in}}%
\pgfpathlineto{\pgfqpoint{0.000000in}{-0.048611in}}%
\pgfusepath{stroke,fill}%
}%
\begin{pgfscope}%
\pgfsys@transformshift{2.170208in}{0.416447in}%
\pgfsys@useobject{currentmarker}{}%
\end{pgfscope}%
\end{pgfscope}%
\begin{pgfscope}%
\definecolor{textcolor}{rgb}{0.000000,0.000000,0.000000}%
\pgfsetstrokecolor{textcolor}%
\pgfsetfillcolor{textcolor}%
\pgftext[x=2.170208in,y=0.319225in,,top]{\color{textcolor}\rmfamily\fontsize{8.000000}{9.600000}\selectfont \(\displaystyle {15000}\)}%
\end{pgfscope}%
\begin{pgfscope}%
\definecolor{textcolor}{rgb}{0.000000,0.000000,0.000000}%
\pgfsetstrokecolor{textcolor}%
\pgfsetfillcolor{textcolor}%
\pgftext[x=1.464523in,y=0.165003in,,top]{\color{textcolor}\rmfamily\fontsize{10.000000}{12.000000}\selectfont Time in \(\displaystyle \unit{\second}\)}%
\end{pgfscope}%
\begin{pgfscope}%
\pgfpathrectangle{\pgfqpoint{0.530716in}{0.416447in}}{\pgfqpoint{1.867614in}{1.371882in}}%
\pgfusepath{clip}%
\pgfsetrectcap%
\pgfsetroundjoin%
\pgfsetlinewidth{0.803000pt}%
\definecolor{currentstroke}{rgb}{0.450000,0.450000,0.450000}%
\pgfsetstrokecolor{currentstroke}%
\pgfsetdash{}{0pt}%
\pgfpathmoveto{\pgfqpoint{0.530716in}{0.416447in}}%
\pgfpathlineto{\pgfqpoint{2.398330in}{0.416447in}}%
\pgfusepath{stroke}%
\end{pgfscope}%
\begin{pgfscope}%
\pgfsetbuttcap%
\pgfsetroundjoin%
\definecolor{currentfill}{rgb}{0.000000,0.000000,0.000000}%
\pgfsetfillcolor{currentfill}%
\pgfsetlinewidth{0.803000pt}%
\definecolor{currentstroke}{rgb}{0.000000,0.000000,0.000000}%
\pgfsetstrokecolor{currentstroke}%
\pgfsetdash{}{0pt}%
\pgfsys@defobject{currentmarker}{\pgfqpoint{-0.048611in}{0.000000in}}{\pgfqpoint{-0.000000in}{0.000000in}}{%
\pgfpathmoveto{\pgfqpoint{-0.000000in}{0.000000in}}%
\pgfpathlineto{\pgfqpoint{-0.048611in}{0.000000in}}%
\pgfusepath{stroke,fill}%
}%
\begin{pgfscope}%
\pgfsys@transformshift{0.530716in}{0.416447in}%
\pgfsys@useobject{currentmarker}{}%
\end{pgfscope}%
\end{pgfscope}%
\begin{pgfscope}%
\definecolor{textcolor}{rgb}{0.000000,0.000000,0.000000}%
\pgfsetstrokecolor{textcolor}%
\pgfsetfillcolor{textcolor}%
\pgftext[x=0.223614in, y=0.377892in, left, base]{\color{textcolor}\rmfamily\fontsize{8.000000}{9.600000}\selectfont \(\displaystyle {\ensuremath{-}15}\)}%
\end{pgfscope}%
\begin{pgfscope}%
\pgfpathrectangle{\pgfqpoint{0.530716in}{0.416447in}}{\pgfqpoint{1.867614in}{1.371882in}}%
\pgfusepath{clip}%
\pgfsetrectcap%
\pgfsetroundjoin%
\pgfsetlinewidth{0.803000pt}%
\definecolor{currentstroke}{rgb}{0.450000,0.450000,0.450000}%
\pgfsetstrokecolor{currentstroke}%
\pgfsetdash{}{0pt}%
\pgfpathmoveto{\pgfqpoint{0.530716in}{0.645095in}}%
\pgfpathlineto{\pgfqpoint{2.398330in}{0.645095in}}%
\pgfusepath{stroke}%
\end{pgfscope}%
\begin{pgfscope}%
\pgfsetbuttcap%
\pgfsetroundjoin%
\definecolor{currentfill}{rgb}{0.000000,0.000000,0.000000}%
\pgfsetfillcolor{currentfill}%
\pgfsetlinewidth{0.803000pt}%
\definecolor{currentstroke}{rgb}{0.000000,0.000000,0.000000}%
\pgfsetstrokecolor{currentstroke}%
\pgfsetdash{}{0pt}%
\pgfsys@defobject{currentmarker}{\pgfqpoint{-0.048611in}{0.000000in}}{\pgfqpoint{-0.000000in}{0.000000in}}{%
\pgfpathmoveto{\pgfqpoint{-0.000000in}{0.000000in}}%
\pgfpathlineto{\pgfqpoint{-0.048611in}{0.000000in}}%
\pgfusepath{stroke,fill}%
}%
\begin{pgfscope}%
\pgfsys@transformshift{0.530716in}{0.645095in}%
\pgfsys@useobject{currentmarker}{}%
\end{pgfscope}%
\end{pgfscope}%
\begin{pgfscope}%
\definecolor{textcolor}{rgb}{0.000000,0.000000,0.000000}%
\pgfsetstrokecolor{textcolor}%
\pgfsetfillcolor{textcolor}%
\pgftext[x=0.223614in, y=0.606539in, left, base]{\color{textcolor}\rmfamily\fontsize{8.000000}{9.600000}\selectfont \(\displaystyle {\ensuremath{-}10}\)}%
\end{pgfscope}%
\begin{pgfscope}%
\pgfpathrectangle{\pgfqpoint{0.530716in}{0.416447in}}{\pgfqpoint{1.867614in}{1.371882in}}%
\pgfusepath{clip}%
\pgfsetrectcap%
\pgfsetroundjoin%
\pgfsetlinewidth{0.803000pt}%
\definecolor{currentstroke}{rgb}{0.450000,0.450000,0.450000}%
\pgfsetstrokecolor{currentstroke}%
\pgfsetdash{}{0pt}%
\pgfpathmoveto{\pgfqpoint{0.530716in}{0.873742in}}%
\pgfpathlineto{\pgfqpoint{2.398330in}{0.873742in}}%
\pgfusepath{stroke}%
\end{pgfscope}%
\begin{pgfscope}%
\pgfsetbuttcap%
\pgfsetroundjoin%
\definecolor{currentfill}{rgb}{0.000000,0.000000,0.000000}%
\pgfsetfillcolor{currentfill}%
\pgfsetlinewidth{0.803000pt}%
\definecolor{currentstroke}{rgb}{0.000000,0.000000,0.000000}%
\pgfsetstrokecolor{currentstroke}%
\pgfsetdash{}{0pt}%
\pgfsys@defobject{currentmarker}{\pgfqpoint{-0.048611in}{0.000000in}}{\pgfqpoint{-0.000000in}{0.000000in}}{%
\pgfpathmoveto{\pgfqpoint{-0.000000in}{0.000000in}}%
\pgfpathlineto{\pgfqpoint{-0.048611in}{0.000000in}}%
\pgfusepath{stroke,fill}%
}%
\begin{pgfscope}%
\pgfsys@transformshift{0.530716in}{0.873742in}%
\pgfsys@useobject{currentmarker}{}%
\end{pgfscope}%
\end{pgfscope}%
\begin{pgfscope}%
\definecolor{textcolor}{rgb}{0.000000,0.000000,0.000000}%
\pgfsetstrokecolor{textcolor}%
\pgfsetfillcolor{textcolor}%
\pgftext[x=0.282643in, y=0.835186in, left, base]{\color{textcolor}\rmfamily\fontsize{8.000000}{9.600000}\selectfont \(\displaystyle {\ensuremath{-}5}\)}%
\end{pgfscope}%
\begin{pgfscope}%
\pgfpathrectangle{\pgfqpoint{0.530716in}{0.416447in}}{\pgfqpoint{1.867614in}{1.371882in}}%
\pgfusepath{clip}%
\pgfsetrectcap%
\pgfsetroundjoin%
\pgfsetlinewidth{0.803000pt}%
\definecolor{currentstroke}{rgb}{0.450000,0.450000,0.450000}%
\pgfsetstrokecolor{currentstroke}%
\pgfsetdash{}{0pt}%
\pgfpathmoveto{\pgfqpoint{0.530716in}{1.102389in}}%
\pgfpathlineto{\pgfqpoint{2.398330in}{1.102389in}}%
\pgfusepath{stroke}%
\end{pgfscope}%
\begin{pgfscope}%
\pgfsetbuttcap%
\pgfsetroundjoin%
\definecolor{currentfill}{rgb}{0.000000,0.000000,0.000000}%
\pgfsetfillcolor{currentfill}%
\pgfsetlinewidth{0.803000pt}%
\definecolor{currentstroke}{rgb}{0.000000,0.000000,0.000000}%
\pgfsetstrokecolor{currentstroke}%
\pgfsetdash{}{0pt}%
\pgfsys@defobject{currentmarker}{\pgfqpoint{-0.048611in}{0.000000in}}{\pgfqpoint{-0.000000in}{0.000000in}}{%
\pgfpathmoveto{\pgfqpoint{-0.000000in}{0.000000in}}%
\pgfpathlineto{\pgfqpoint{-0.048611in}{0.000000in}}%
\pgfusepath{stroke,fill}%
}%
\begin{pgfscope}%
\pgfsys@transformshift{0.530716in}{1.102389in}%
\pgfsys@useobject{currentmarker}{}%
\end{pgfscope}%
\end{pgfscope}%
\begin{pgfscope}%
\definecolor{textcolor}{rgb}{0.000000,0.000000,0.000000}%
\pgfsetstrokecolor{textcolor}%
\pgfsetfillcolor{textcolor}%
\pgftext[x=0.374465in, y=1.063833in, left, base]{\color{textcolor}\rmfamily\fontsize{8.000000}{9.600000}\selectfont \(\displaystyle {0}\)}%
\end{pgfscope}%
\begin{pgfscope}%
\pgfpathrectangle{\pgfqpoint{0.530716in}{0.416447in}}{\pgfqpoint{1.867614in}{1.371882in}}%
\pgfusepath{clip}%
\pgfsetrectcap%
\pgfsetroundjoin%
\pgfsetlinewidth{0.803000pt}%
\definecolor{currentstroke}{rgb}{0.450000,0.450000,0.450000}%
\pgfsetstrokecolor{currentstroke}%
\pgfsetdash{}{0pt}%
\pgfpathmoveto{\pgfqpoint{0.530716in}{1.331036in}}%
\pgfpathlineto{\pgfqpoint{2.398330in}{1.331036in}}%
\pgfusepath{stroke}%
\end{pgfscope}%
\begin{pgfscope}%
\pgfsetbuttcap%
\pgfsetroundjoin%
\definecolor{currentfill}{rgb}{0.000000,0.000000,0.000000}%
\pgfsetfillcolor{currentfill}%
\pgfsetlinewidth{0.803000pt}%
\definecolor{currentstroke}{rgb}{0.000000,0.000000,0.000000}%
\pgfsetstrokecolor{currentstroke}%
\pgfsetdash{}{0pt}%
\pgfsys@defobject{currentmarker}{\pgfqpoint{-0.048611in}{0.000000in}}{\pgfqpoint{-0.000000in}{0.000000in}}{%
\pgfpathmoveto{\pgfqpoint{-0.000000in}{0.000000in}}%
\pgfpathlineto{\pgfqpoint{-0.048611in}{0.000000in}}%
\pgfusepath{stroke,fill}%
}%
\begin{pgfscope}%
\pgfsys@transformshift{0.530716in}{1.331036in}%
\pgfsys@useobject{currentmarker}{}%
\end{pgfscope}%
\end{pgfscope}%
\begin{pgfscope}%
\definecolor{textcolor}{rgb}{0.000000,0.000000,0.000000}%
\pgfsetstrokecolor{textcolor}%
\pgfsetfillcolor{textcolor}%
\pgftext[x=0.374465in, y=1.292480in, left, base]{\color{textcolor}\rmfamily\fontsize{8.000000}{9.600000}\selectfont \(\displaystyle {5}\)}%
\end{pgfscope}%
\begin{pgfscope}%
\pgfpathrectangle{\pgfqpoint{0.530716in}{0.416447in}}{\pgfqpoint{1.867614in}{1.371882in}}%
\pgfusepath{clip}%
\pgfsetrectcap%
\pgfsetroundjoin%
\pgfsetlinewidth{0.803000pt}%
\definecolor{currentstroke}{rgb}{0.450000,0.450000,0.450000}%
\pgfsetstrokecolor{currentstroke}%
\pgfsetdash{}{0pt}%
\pgfpathmoveto{\pgfqpoint{0.530716in}{1.559683in}}%
\pgfpathlineto{\pgfqpoint{2.398330in}{1.559683in}}%
\pgfusepath{stroke}%
\end{pgfscope}%
\begin{pgfscope}%
\pgfsetbuttcap%
\pgfsetroundjoin%
\definecolor{currentfill}{rgb}{0.000000,0.000000,0.000000}%
\pgfsetfillcolor{currentfill}%
\pgfsetlinewidth{0.803000pt}%
\definecolor{currentstroke}{rgb}{0.000000,0.000000,0.000000}%
\pgfsetstrokecolor{currentstroke}%
\pgfsetdash{}{0pt}%
\pgfsys@defobject{currentmarker}{\pgfqpoint{-0.048611in}{0.000000in}}{\pgfqpoint{-0.000000in}{0.000000in}}{%
\pgfpathmoveto{\pgfqpoint{-0.000000in}{0.000000in}}%
\pgfpathlineto{\pgfqpoint{-0.048611in}{0.000000in}}%
\pgfusepath{stroke,fill}%
}%
\begin{pgfscope}%
\pgfsys@transformshift{0.530716in}{1.559683in}%
\pgfsys@useobject{currentmarker}{}%
\end{pgfscope}%
\end{pgfscope}%
\begin{pgfscope}%
\definecolor{textcolor}{rgb}{0.000000,0.000000,0.000000}%
\pgfsetstrokecolor{textcolor}%
\pgfsetfillcolor{textcolor}%
\pgftext[x=0.315437in, y=1.521127in, left, base]{\color{textcolor}\rmfamily\fontsize{8.000000}{9.600000}\selectfont \(\displaystyle {10}\)}%
\end{pgfscope}%
\begin{pgfscope}%
\pgfpathrectangle{\pgfqpoint{0.530716in}{0.416447in}}{\pgfqpoint{1.867614in}{1.371882in}}%
\pgfusepath{clip}%
\pgfsetrectcap%
\pgfsetroundjoin%
\pgfsetlinewidth{0.803000pt}%
\definecolor{currentstroke}{rgb}{0.450000,0.450000,0.450000}%
\pgfsetstrokecolor{currentstroke}%
\pgfsetdash{}{0pt}%
\pgfpathmoveto{\pgfqpoint{0.530716in}{1.788330in}}%
\pgfpathlineto{\pgfqpoint{2.398330in}{1.788330in}}%
\pgfusepath{stroke}%
\end{pgfscope}%
\begin{pgfscope}%
\pgfsetbuttcap%
\pgfsetroundjoin%
\definecolor{currentfill}{rgb}{0.000000,0.000000,0.000000}%
\pgfsetfillcolor{currentfill}%
\pgfsetlinewidth{0.803000pt}%
\definecolor{currentstroke}{rgb}{0.000000,0.000000,0.000000}%
\pgfsetstrokecolor{currentstroke}%
\pgfsetdash{}{0pt}%
\pgfsys@defobject{currentmarker}{\pgfqpoint{-0.048611in}{0.000000in}}{\pgfqpoint{-0.000000in}{0.000000in}}{%
\pgfpathmoveto{\pgfqpoint{-0.000000in}{0.000000in}}%
\pgfpathlineto{\pgfqpoint{-0.048611in}{0.000000in}}%
\pgfusepath{stroke,fill}%
}%
\begin{pgfscope}%
\pgfsys@transformshift{0.530716in}{1.788330in}%
\pgfsys@useobject{currentmarker}{}%
\end{pgfscope}%
\end{pgfscope}%
\begin{pgfscope}%
\definecolor{textcolor}{rgb}{0.000000,0.000000,0.000000}%
\pgfsetstrokecolor{textcolor}%
\pgfsetfillcolor{textcolor}%
\pgftext[x=0.315437in, y=1.749774in, left, base]{\color{textcolor}\rmfamily\fontsize{8.000000}{9.600000}\selectfont \(\displaystyle {15}\)}%
\end{pgfscope}%
\begin{pgfscope}%
\definecolor{textcolor}{rgb}{0.000000,0.000000,0.000000}%
\pgfsetstrokecolor{textcolor}%
\pgfsetfillcolor{textcolor}%
\pgftext[x=0.168059in,y=1.102389in,,bottom,rotate=90.000000]{\color{textcolor}\rmfamily\fontsize{10.000000}{12.000000}\selectfont Ampl. in arb. unit}%
\end{pgfscope}%
\begin{pgfscope}%
\pgfpathrectangle{\pgfqpoint{0.530716in}{0.416447in}}{\pgfqpoint{1.867614in}{1.371882in}}%
\pgfusepath{clip}%
\pgfsetrectcap%
\pgfsetroundjoin%
\pgfsetlinewidth{1.505625pt}%
\definecolor{currentstroke}{rgb}{0.000000,0.619608,0.450980}%
\pgfsetstrokecolor{currentstroke}%
\pgfsetdash{}{0pt}%
\pgfpathmoveto{\pgfqpoint{0.615608in}{1.109968in}}%
\pgfpathlineto{\pgfqpoint{0.616126in}{1.247326in}}%
\pgfpathlineto{\pgfqpoint{0.616748in}{1.156083in}}%
\pgfpathlineto{\pgfqpoint{0.617473in}{0.927633in}}%
\pgfpathlineto{\pgfqpoint{0.617991in}{0.975135in}}%
\pgfpathlineto{\pgfqpoint{0.618717in}{1.122777in}}%
\pgfpathlineto{\pgfqpoint{0.618199in}{0.946292in}}%
\pgfpathlineto{\pgfqpoint{0.619028in}{1.079568in}}%
\pgfpathlineto{\pgfqpoint{0.619442in}{0.878404in}}%
\pgfpathlineto{\pgfqpoint{0.620271in}{0.923986in}}%
\pgfpathlineto{\pgfqpoint{0.621204in}{1.096724in}}%
\pgfpathlineto{\pgfqpoint{0.620582in}{0.896272in}}%
\pgfpathlineto{\pgfqpoint{0.621411in}{0.999151in}}%
\pgfpathlineto{\pgfqpoint{0.622448in}{1.116002in}}%
\pgfpathlineto{\pgfqpoint{0.622033in}{0.913785in}}%
\pgfpathlineto{\pgfqpoint{0.622552in}{1.103577in}}%
\pgfpathlineto{\pgfqpoint{0.623692in}{0.934054in}}%
\pgfpathlineto{\pgfqpoint{0.623070in}{1.206526in}}%
\pgfpathlineto{\pgfqpoint{0.623795in}{0.986921in}}%
\pgfpathlineto{\pgfqpoint{0.624002in}{1.135508in}}%
\pgfpathlineto{\pgfqpoint{0.625039in}{1.051314in}}%
\pgfpathlineto{\pgfqpoint{0.625246in}{1.029245in}}%
\pgfpathlineto{\pgfqpoint{0.626179in}{0.934340in}}%
\pgfpathlineto{\pgfqpoint{0.625557in}{1.043243in}}%
\pgfpathlineto{\pgfqpoint{0.626283in}{0.960896in}}%
\pgfpathlineto{\pgfqpoint{0.627215in}{1.184323in}}%
\pgfpathlineto{\pgfqpoint{0.626904in}{0.926771in}}%
\pgfpathlineto{\pgfqpoint{0.627526in}{1.080835in}}%
\pgfpathlineto{\pgfqpoint{0.627630in}{0.995270in}}%
\pgfpathlineto{\pgfqpoint{0.628148in}{1.170213in}}%
\pgfpathlineto{\pgfqpoint{0.628355in}{1.102870in}}%
\pgfpathlineto{\pgfqpoint{0.628459in}{1.238820in}}%
\pgfpathlineto{\pgfqpoint{0.628977in}{0.964016in}}%
\pgfpathlineto{\pgfqpoint{0.629288in}{1.040594in}}%
\pgfpathlineto{\pgfqpoint{0.630221in}{0.974169in}}%
\pgfpathlineto{\pgfqpoint{0.630117in}{1.126885in}}%
\pgfpathlineto{\pgfqpoint{0.630325in}{1.033228in}}%
\pgfpathlineto{\pgfqpoint{0.630532in}{1.102945in}}%
\pgfpathlineto{\pgfqpoint{0.630739in}{0.936584in}}%
\pgfpathlineto{\pgfqpoint{0.631465in}{1.064450in}}%
\pgfpathlineto{\pgfqpoint{0.631568in}{1.005820in}}%
\pgfpathlineto{\pgfqpoint{0.631672in}{1.157160in}}%
\pgfpathlineto{\pgfqpoint{0.632397in}{1.053697in}}%
\pgfpathlineto{\pgfqpoint{0.632812in}{1.261184in}}%
\pgfpathlineto{\pgfqpoint{0.633226in}{1.028221in}}%
\pgfpathlineto{\pgfqpoint{0.633537in}{1.102756in}}%
\pgfpathlineto{\pgfqpoint{0.634056in}{1.318893in}}%
\pgfpathlineto{\pgfqpoint{0.634366in}{1.089169in}}%
\pgfpathlineto{\pgfqpoint{0.634781in}{1.185422in}}%
\pgfpathlineto{\pgfqpoint{0.636128in}{1.004114in}}%
\pgfpathlineto{\pgfqpoint{0.636232in}{1.069302in}}%
\pgfpathlineto{\pgfqpoint{0.637165in}{1.384778in}}%
\pgfpathlineto{\pgfqpoint{0.636750in}{1.024661in}}%
\pgfpathlineto{\pgfqpoint{0.637787in}{1.273526in}}%
\pgfpathlineto{\pgfqpoint{0.638616in}{1.075928in}}%
\pgfpathlineto{\pgfqpoint{0.638305in}{1.327134in}}%
\pgfpathlineto{\pgfqpoint{0.639030in}{1.085869in}}%
\pgfpathlineto{\pgfqpoint{0.639134in}{1.070207in}}%
\pgfpathlineto{\pgfqpoint{0.639238in}{1.148529in}}%
\pgfpathlineto{\pgfqpoint{0.639652in}{1.078496in}}%
\pgfpathlineto{\pgfqpoint{0.639756in}{1.248867in}}%
\pgfpathlineto{\pgfqpoint{0.639963in}{1.047850in}}%
\pgfpathlineto{\pgfqpoint{0.640689in}{1.186677in}}%
\pgfpathlineto{\pgfqpoint{0.641103in}{1.043034in}}%
\pgfpathlineto{\pgfqpoint{0.641621in}{1.279634in}}%
\pgfpathlineto{\pgfqpoint{0.641725in}{1.272612in}}%
\pgfpathlineto{\pgfqpoint{0.642658in}{0.915205in}}%
\pgfpathlineto{\pgfqpoint{0.642969in}{0.947263in}}%
\pgfpathlineto{\pgfqpoint{0.643487in}{1.143344in}}%
\pgfpathlineto{\pgfqpoint{0.644109in}{1.091762in}}%
\pgfpathlineto{\pgfqpoint{0.644212in}{1.044015in}}%
\pgfpathlineto{\pgfqpoint{0.644730in}{1.204014in}}%
\pgfpathlineto{\pgfqpoint{0.644834in}{1.060292in}}%
\pgfpathlineto{\pgfqpoint{0.644938in}{1.234721in}}%
\pgfpathlineto{\pgfqpoint{0.645041in}{1.030227in}}%
\pgfpathlineto{\pgfqpoint{0.645974in}{1.118830in}}%
\pgfpathlineto{\pgfqpoint{0.646078in}{1.056137in}}%
\pgfpathlineto{\pgfqpoint{0.646492in}{1.182899in}}%
\pgfpathlineto{\pgfqpoint{0.647011in}{1.138926in}}%
\pgfpathlineto{\pgfqpoint{0.647114in}{1.140008in}}%
\pgfpathlineto{\pgfqpoint{0.648047in}{1.290493in}}%
\pgfpathlineto{\pgfqpoint{0.647632in}{1.125555in}}%
\pgfpathlineto{\pgfqpoint{0.648254in}{1.258669in}}%
\pgfpathlineto{\pgfqpoint{0.649187in}{1.102484in}}%
\pgfpathlineto{\pgfqpoint{0.648980in}{1.310698in}}%
\pgfpathlineto{\pgfqpoint{0.649291in}{1.222125in}}%
\pgfpathlineto{\pgfqpoint{0.649602in}{1.263061in}}%
\pgfpathlineto{\pgfqpoint{0.649809in}{1.154456in}}%
\pgfpathlineto{\pgfqpoint{0.649912in}{1.182549in}}%
\pgfpathlineto{\pgfqpoint{0.650742in}{1.085100in}}%
\pgfpathlineto{\pgfqpoint{0.650120in}{1.224078in}}%
\pgfpathlineto{\pgfqpoint{0.651156in}{1.094216in}}%
\pgfpathlineto{\pgfqpoint{0.651985in}{1.186370in}}%
\pgfpathlineto{\pgfqpoint{0.651467in}{0.983281in}}%
\pgfpathlineto{\pgfqpoint{0.652193in}{1.131481in}}%
\pgfpathlineto{\pgfqpoint{0.652711in}{1.051937in}}%
\pgfpathlineto{\pgfqpoint{0.652918in}{1.211500in}}%
\pgfpathlineto{\pgfqpoint{0.653125in}{1.169683in}}%
\pgfpathlineto{\pgfqpoint{0.654265in}{1.355290in}}%
\pgfpathlineto{\pgfqpoint{0.653747in}{1.094167in}}%
\pgfpathlineto{\pgfqpoint{0.654473in}{1.271570in}}%
\pgfpathlineto{\pgfqpoint{0.655094in}{1.030819in}}%
\pgfpathlineto{\pgfqpoint{0.654680in}{1.375345in}}%
\pgfpathlineto{\pgfqpoint{0.655613in}{1.258499in}}%
\pgfpathlineto{\pgfqpoint{0.655716in}{1.260332in}}%
\pgfpathlineto{\pgfqpoint{0.656545in}{1.068641in}}%
\pgfpathlineto{\pgfqpoint{0.656338in}{1.270436in}}%
\pgfpathlineto{\pgfqpoint{0.656753in}{1.172868in}}%
\pgfpathlineto{\pgfqpoint{0.656856in}{1.254799in}}%
\pgfpathlineto{\pgfqpoint{0.657582in}{1.079117in}}%
\pgfpathlineto{\pgfqpoint{0.657789in}{1.124243in}}%
\pgfpathlineto{\pgfqpoint{0.659136in}{1.370961in}}%
\pgfpathlineto{\pgfqpoint{0.658204in}{1.014680in}}%
\pgfpathlineto{\pgfqpoint{0.659344in}{1.318152in}}%
\pgfpathlineto{\pgfqpoint{0.659447in}{1.315279in}}%
\pgfpathlineto{\pgfqpoint{0.659551in}{1.369055in}}%
\pgfpathlineto{\pgfqpoint{0.660069in}{1.130721in}}%
\pgfpathlineto{\pgfqpoint{0.660173in}{1.035011in}}%
\pgfpathlineto{\pgfqpoint{0.660691in}{1.317579in}}%
\pgfpathlineto{\pgfqpoint{0.661106in}{1.171398in}}%
\pgfpathlineto{\pgfqpoint{0.661209in}{1.162462in}}%
\pgfpathlineto{\pgfqpoint{0.662038in}{1.016417in}}%
\pgfpathlineto{\pgfqpoint{0.662349in}{1.049586in}}%
\pgfpathlineto{\pgfqpoint{0.663178in}{1.149404in}}%
\pgfpathlineto{\pgfqpoint{0.663075in}{0.960070in}}%
\pgfpathlineto{\pgfqpoint{0.663489in}{1.085031in}}%
\pgfpathlineto{\pgfqpoint{0.663697in}{1.067189in}}%
\pgfpathlineto{\pgfqpoint{0.663800in}{1.120083in}}%
\pgfpathlineto{\pgfqpoint{0.663904in}{1.155310in}}%
\pgfpathlineto{\pgfqpoint{0.664318in}{0.921190in}}%
\pgfpathlineto{\pgfqpoint{0.664422in}{0.901534in}}%
\pgfpathlineto{\pgfqpoint{0.664526in}{1.070855in}}%
\pgfpathlineto{\pgfqpoint{0.665044in}{1.325961in}}%
\pgfpathlineto{\pgfqpoint{0.665459in}{1.040316in}}%
\pgfpathlineto{\pgfqpoint{0.665562in}{1.109233in}}%
\pgfpathlineto{\pgfqpoint{0.665977in}{0.969365in}}%
\pgfpathlineto{\pgfqpoint{0.666391in}{1.216163in}}%
\pgfpathlineto{\pgfqpoint{0.666599in}{1.091381in}}%
\pgfpathlineto{\pgfqpoint{0.667428in}{1.234182in}}%
\pgfpathlineto{\pgfqpoint{0.667220in}{0.999829in}}%
\pgfpathlineto{\pgfqpoint{0.667635in}{1.143840in}}%
\pgfpathlineto{\pgfqpoint{0.668775in}{1.000218in}}%
\pgfpathlineto{\pgfqpoint{0.669293in}{1.113969in}}%
\pgfpathlineto{\pgfqpoint{0.669397in}{0.976961in}}%
\pgfpathlineto{\pgfqpoint{0.669811in}{1.017805in}}%
\pgfpathlineto{\pgfqpoint{0.670019in}{0.877657in}}%
\pgfpathlineto{\pgfqpoint{0.670433in}{1.023914in}}%
\pgfpathlineto{\pgfqpoint{0.670951in}{0.947396in}}%
\pgfpathlineto{\pgfqpoint{0.671055in}{0.903614in}}%
\pgfpathlineto{\pgfqpoint{0.671470in}{1.032991in}}%
\pgfpathlineto{\pgfqpoint{0.671677in}{1.008405in}}%
\pgfpathlineto{\pgfqpoint{0.671781in}{1.100311in}}%
\pgfpathlineto{\pgfqpoint{0.671884in}{0.882713in}}%
\pgfpathlineto{\pgfqpoint{0.672817in}{1.054638in}}%
\pgfpathlineto{\pgfqpoint{0.672921in}{1.062854in}}%
\pgfpathlineto{\pgfqpoint{0.673128in}{0.995090in}}%
\pgfpathlineto{\pgfqpoint{0.673232in}{0.983791in}}%
\pgfpathlineto{\pgfqpoint{0.673335in}{1.031025in}}%
\pgfpathlineto{\pgfqpoint{0.673542in}{1.030278in}}%
\pgfpathlineto{\pgfqpoint{0.673646in}{1.168069in}}%
\pgfpathlineto{\pgfqpoint{0.674372in}{0.868565in}}%
\pgfpathlineto{\pgfqpoint{0.674682in}{1.091968in}}%
\pgfpathlineto{\pgfqpoint{0.675097in}{0.912814in}}%
\pgfpathlineto{\pgfqpoint{0.674993in}{1.093222in}}%
\pgfpathlineto{\pgfqpoint{0.675823in}{1.072513in}}%
\pgfpathlineto{\pgfqpoint{0.675926in}{1.209647in}}%
\pgfpathlineto{\pgfqpoint{0.676341in}{1.026099in}}%
\pgfpathlineto{\pgfqpoint{0.676859in}{1.077727in}}%
\pgfpathlineto{\pgfqpoint{0.677066in}{1.147373in}}%
\pgfpathlineto{\pgfqpoint{0.677273in}{1.061705in}}%
\pgfpathlineto{\pgfqpoint{0.677377in}{1.002634in}}%
\pgfpathlineto{\pgfqpoint{0.677999in}{1.171161in}}%
\pgfpathlineto{\pgfqpoint{0.678103in}{1.148351in}}%
\pgfpathlineto{\pgfqpoint{0.678206in}{1.249771in}}%
\pgfpathlineto{\pgfqpoint{0.678517in}{0.955343in}}%
\pgfpathlineto{\pgfqpoint{0.679139in}{1.231168in}}%
\pgfpathlineto{\pgfqpoint{0.679761in}{1.099783in}}%
\pgfpathlineto{\pgfqpoint{0.680072in}{1.287756in}}%
\pgfpathlineto{\pgfqpoint{0.680175in}{1.331526in}}%
\pgfpathlineto{\pgfqpoint{0.680383in}{1.141880in}}%
\pgfpathlineto{\pgfqpoint{0.680694in}{1.235282in}}%
\pgfpathlineto{\pgfqpoint{0.681730in}{0.979615in}}%
\pgfpathlineto{\pgfqpoint{0.681834in}{1.012818in}}%
\pgfpathlineto{\pgfqpoint{0.682455in}{0.862686in}}%
\pgfpathlineto{\pgfqpoint{0.682974in}{1.177631in}}%
\pgfpathlineto{\pgfqpoint{0.683285in}{1.237057in}}%
\pgfpathlineto{\pgfqpoint{0.684321in}{1.018330in}}%
\pgfpathlineto{\pgfqpoint{0.684632in}{1.161639in}}%
\pgfpathlineto{\pgfqpoint{0.684736in}{0.932602in}}%
\pgfpathlineto{\pgfqpoint{0.685254in}{1.008606in}}%
\pgfpathlineto{\pgfqpoint{0.685565in}{0.905783in}}%
\pgfpathlineto{\pgfqpoint{0.685876in}{1.157345in}}%
\pgfpathlineto{\pgfqpoint{0.686187in}{1.030815in}}%
\pgfpathlineto{\pgfqpoint{0.687119in}{1.149627in}}%
\pgfpathlineto{\pgfqpoint{0.686394in}{0.990265in}}%
\pgfpathlineto{\pgfqpoint{0.687430in}{1.130410in}}%
\pgfpathlineto{\pgfqpoint{0.688467in}{0.921278in}}%
\pgfpathlineto{\pgfqpoint{0.688156in}{1.136896in}}%
\pgfpathlineto{\pgfqpoint{0.688674in}{0.940937in}}%
\pgfpathlineto{\pgfqpoint{0.689607in}{1.133108in}}%
\pgfpathlineto{\pgfqpoint{0.688881in}{0.923344in}}%
\pgfpathlineto{\pgfqpoint{0.689814in}{1.101194in}}%
\pgfpathlineto{\pgfqpoint{0.690332in}{0.930051in}}%
\pgfpathlineto{\pgfqpoint{0.691058in}{0.983864in}}%
\pgfpathlineto{\pgfqpoint{0.692094in}{1.198619in}}%
\pgfpathlineto{\pgfqpoint{0.691472in}{0.909891in}}%
\pgfpathlineto{\pgfqpoint{0.692198in}{1.018679in}}%
\pgfpathlineto{\pgfqpoint{0.692612in}{1.064261in}}%
\pgfpathlineto{\pgfqpoint{0.693441in}{0.856163in}}%
\pgfpathlineto{\pgfqpoint{0.694478in}{1.202750in}}%
\pgfpathlineto{\pgfqpoint{0.694685in}{1.170551in}}%
\pgfpathlineto{\pgfqpoint{0.695203in}{1.223114in}}%
\pgfpathlineto{\pgfqpoint{0.695825in}{1.064803in}}%
\pgfpathlineto{\pgfqpoint{0.696447in}{1.017195in}}%
\pgfpathlineto{\pgfqpoint{0.696240in}{1.126195in}}%
\pgfpathlineto{\pgfqpoint{0.696551in}{1.085595in}}%
\pgfpathlineto{\pgfqpoint{0.696965in}{1.205680in}}%
\pgfpathlineto{\pgfqpoint{0.697276in}{1.036144in}}%
\pgfpathlineto{\pgfqpoint{0.697587in}{1.135212in}}%
\pgfpathlineto{\pgfqpoint{0.697794in}{1.001011in}}%
\pgfpathlineto{\pgfqpoint{0.698416in}{1.172742in}}%
\pgfpathlineto{\pgfqpoint{0.698727in}{1.081647in}}%
\pgfpathlineto{\pgfqpoint{0.699245in}{1.159623in}}%
\pgfpathlineto{\pgfqpoint{0.699556in}{1.049042in}}%
\pgfpathlineto{\pgfqpoint{0.700696in}{1.332203in}}%
\pgfpathlineto{\pgfqpoint{0.699971in}{1.046985in}}%
\pgfpathlineto{\pgfqpoint{0.700800in}{1.213566in}}%
\pgfpathlineto{\pgfqpoint{0.701836in}{1.025260in}}%
\pgfpathlineto{\pgfqpoint{0.701111in}{1.324386in}}%
\pgfpathlineto{\pgfqpoint{0.702043in}{1.093651in}}%
\pgfpathlineto{\pgfqpoint{0.702769in}{1.068868in}}%
\pgfpathlineto{\pgfqpoint{0.703287in}{1.313606in}}%
\pgfpathlineto{\pgfqpoint{0.703805in}{1.069494in}}%
\pgfpathlineto{\pgfqpoint{0.704427in}{1.103830in}}%
\pgfpathlineto{\pgfqpoint{0.705567in}{1.321166in}}%
\pgfpathlineto{\pgfqpoint{0.704738in}{1.043167in}}%
\pgfpathlineto{\pgfqpoint{0.705671in}{1.230464in}}%
\pgfpathlineto{\pgfqpoint{0.705982in}{1.076347in}}%
\pgfpathlineto{\pgfqpoint{0.706189in}{1.243016in}}%
\pgfpathlineto{\pgfqpoint{0.706604in}{1.238745in}}%
\pgfpathlineto{\pgfqpoint{0.706707in}{1.369455in}}%
\pgfpathlineto{\pgfqpoint{0.706915in}{1.190005in}}%
\pgfpathlineto{\pgfqpoint{0.707744in}{1.344204in}}%
\pgfpathlineto{\pgfqpoint{0.707847in}{1.332380in}}%
\pgfpathlineto{\pgfqpoint{0.708055in}{1.396677in}}%
\pgfpathlineto{\pgfqpoint{0.708158in}{1.431033in}}%
\pgfpathlineto{\pgfqpoint{0.708676in}{1.291253in}}%
\pgfpathlineto{\pgfqpoint{0.708780in}{1.324712in}}%
\pgfpathlineto{\pgfqpoint{0.709402in}{1.205474in}}%
\pgfpathlineto{\pgfqpoint{0.709091in}{1.367225in}}%
\pgfpathlineto{\pgfqpoint{0.709609in}{1.318540in}}%
\pgfpathlineto{\pgfqpoint{0.709713in}{1.438912in}}%
\pgfpathlineto{\pgfqpoint{0.710646in}{1.300159in}}%
\pgfpathlineto{\pgfqpoint{0.710853in}{1.380412in}}%
\pgfpathlineto{\pgfqpoint{0.711164in}{1.242706in}}%
\pgfpathlineto{\pgfqpoint{0.711267in}{1.345182in}}%
\pgfpathlineto{\pgfqpoint{0.712200in}{1.105091in}}%
\pgfpathlineto{\pgfqpoint{0.711993in}{1.389306in}}%
\pgfpathlineto{\pgfqpoint{0.712511in}{1.132968in}}%
\pgfpathlineto{\pgfqpoint{0.713340in}{1.341996in}}%
\pgfpathlineto{\pgfqpoint{0.712926in}{1.110635in}}%
\pgfpathlineto{\pgfqpoint{0.713547in}{1.118039in}}%
\pgfpathlineto{\pgfqpoint{0.714066in}{1.201082in}}%
\pgfpathlineto{\pgfqpoint{0.714169in}{1.106315in}}%
\pgfpathlineto{\pgfqpoint{0.715102in}{1.362534in}}%
\pgfpathlineto{\pgfqpoint{0.715413in}{1.298016in}}%
\pgfpathlineto{\pgfqpoint{0.715724in}{1.408682in}}%
\pgfpathlineto{\pgfqpoint{0.716138in}{1.218169in}}%
\pgfpathlineto{\pgfqpoint{0.716346in}{1.227198in}}%
\pgfpathlineto{\pgfqpoint{0.716553in}{1.098795in}}%
\pgfpathlineto{\pgfqpoint{0.716657in}{1.136028in}}%
\pgfpathlineto{\pgfqpoint{0.716864in}{0.988168in}}%
\pgfpathlineto{\pgfqpoint{0.717693in}{1.220958in}}%
\pgfpathlineto{\pgfqpoint{0.717797in}{1.242328in}}%
\pgfpathlineto{\pgfqpoint{0.717900in}{1.191113in}}%
\pgfpathlineto{\pgfqpoint{0.718108in}{1.202484in}}%
\pgfpathlineto{\pgfqpoint{0.718522in}{1.045396in}}%
\pgfpathlineto{\pgfqpoint{0.719144in}{1.274342in}}%
\pgfpathlineto{\pgfqpoint{0.719248in}{1.285523in}}%
\pgfpathlineto{\pgfqpoint{0.719351in}{1.243777in}}%
\pgfpathlineto{\pgfqpoint{0.719455in}{1.182102in}}%
\pgfpathlineto{\pgfqpoint{0.720180in}{1.336145in}}%
\pgfpathlineto{\pgfqpoint{0.720284in}{1.298841in}}%
\pgfpathlineto{\pgfqpoint{0.720388in}{1.418685in}}%
\pgfpathlineto{\pgfqpoint{0.721320in}{1.238205in}}%
\pgfpathlineto{\pgfqpoint{0.721424in}{1.247921in}}%
\pgfpathlineto{\pgfqpoint{0.721631in}{1.439234in}}%
\pgfpathlineto{\pgfqpoint{0.722357in}{1.160599in}}%
\pgfpathlineto{\pgfqpoint{0.722461in}{1.245068in}}%
\pgfpathlineto{\pgfqpoint{0.723290in}{1.373145in}}%
\pgfpathlineto{\pgfqpoint{0.722979in}{1.143850in}}%
\pgfpathlineto{\pgfqpoint{0.723601in}{1.303081in}}%
\pgfpathlineto{\pgfqpoint{0.724015in}{1.205716in}}%
\pgfpathlineto{\pgfqpoint{0.724222in}{1.398695in}}%
\pgfpathlineto{\pgfqpoint{0.724533in}{1.341115in}}%
\pgfpathlineto{\pgfqpoint{0.724844in}{1.240083in}}%
\pgfpathlineto{\pgfqpoint{0.725052in}{1.344355in}}%
\pgfpathlineto{\pgfqpoint{0.725362in}{1.341679in}}%
\pgfpathlineto{\pgfqpoint{0.725984in}{1.081774in}}%
\pgfpathlineto{\pgfqpoint{0.726502in}{1.234568in}}%
\pgfpathlineto{\pgfqpoint{0.727021in}{1.090505in}}%
\pgfpathlineto{\pgfqpoint{0.727228in}{1.250662in}}%
\pgfpathlineto{\pgfqpoint{0.728057in}{1.336278in}}%
\pgfpathlineto{\pgfqpoint{0.727643in}{1.176681in}}%
\pgfpathlineto{\pgfqpoint{0.728264in}{1.265814in}}%
\pgfpathlineto{\pgfqpoint{0.729197in}{1.127633in}}%
\pgfpathlineto{\pgfqpoint{0.728886in}{1.311202in}}%
\pgfpathlineto{\pgfqpoint{0.729404in}{1.260618in}}%
\pgfpathlineto{\pgfqpoint{0.729612in}{1.077335in}}%
\pgfpathlineto{\pgfqpoint{0.730441in}{1.328026in}}%
\pgfpathlineto{\pgfqpoint{0.730544in}{1.256550in}}%
\pgfpathlineto{\pgfqpoint{0.731166in}{1.449202in}}%
\pgfpathlineto{\pgfqpoint{0.731374in}{1.343516in}}%
\pgfpathlineto{\pgfqpoint{0.731684in}{1.449794in}}%
\pgfpathlineto{\pgfqpoint{0.732306in}{1.278835in}}%
\pgfpathlineto{\pgfqpoint{0.733239in}{1.449483in}}%
\pgfpathlineto{\pgfqpoint{0.733965in}{1.371395in}}%
\pgfpathlineto{\pgfqpoint{0.735001in}{1.097645in}}%
\pgfpathlineto{\pgfqpoint{0.735312in}{1.222663in}}%
\pgfpathlineto{\pgfqpoint{0.736348in}{1.425951in}}%
\pgfpathlineto{\pgfqpoint{0.735726in}{1.133940in}}%
\pgfpathlineto{\pgfqpoint{0.736452in}{1.396287in}}%
\pgfpathlineto{\pgfqpoint{0.736556in}{1.395618in}}%
\pgfpathlineto{\pgfqpoint{0.736763in}{1.423626in}}%
\pgfpathlineto{\pgfqpoint{0.737177in}{1.296976in}}%
\pgfpathlineto{\pgfqpoint{0.737281in}{1.296220in}}%
\pgfpathlineto{\pgfqpoint{0.737696in}{1.154854in}}%
\pgfpathlineto{\pgfqpoint{0.738007in}{1.343568in}}%
\pgfpathlineto{\pgfqpoint{0.738421in}{1.189255in}}%
\pgfpathlineto{\pgfqpoint{0.738628in}{1.258773in}}%
\pgfpathlineto{\pgfqpoint{0.738939in}{1.131070in}}%
\pgfpathlineto{\pgfqpoint{0.739147in}{1.195327in}}%
\pgfpathlineto{\pgfqpoint{0.739250in}{1.087135in}}%
\pgfpathlineto{\pgfqpoint{0.739768in}{1.308428in}}%
\pgfpathlineto{\pgfqpoint{0.740183in}{1.272624in}}%
\pgfpathlineto{\pgfqpoint{0.740701in}{1.103929in}}%
\pgfpathlineto{\pgfqpoint{0.740805in}{1.279946in}}%
\pgfpathlineto{\pgfqpoint{0.741323in}{1.178076in}}%
\pgfpathlineto{\pgfqpoint{0.741841in}{1.331514in}}%
\pgfpathlineto{\pgfqpoint{0.741738in}{1.156497in}}%
\pgfpathlineto{\pgfqpoint{0.742463in}{1.227608in}}%
\pgfpathlineto{\pgfqpoint{0.742670in}{1.220213in}}%
\pgfpathlineto{\pgfqpoint{0.742774in}{1.290268in}}%
\pgfpathlineto{\pgfqpoint{0.743085in}{1.187699in}}%
\pgfpathlineto{\pgfqpoint{0.743914in}{1.442133in}}%
\pgfpathlineto{\pgfqpoint{0.745261in}{1.095221in}}%
\pgfpathlineto{\pgfqpoint{0.745676in}{1.303923in}}%
\pgfpathlineto{\pgfqpoint{0.746505in}{1.232119in}}%
\pgfpathlineto{\pgfqpoint{0.746920in}{1.114817in}}%
\pgfpathlineto{\pgfqpoint{0.747230in}{1.283516in}}%
\pgfpathlineto{\pgfqpoint{0.747541in}{1.282703in}}%
\pgfpathlineto{\pgfqpoint{0.747645in}{1.254953in}}%
\pgfpathlineto{\pgfqpoint{0.747749in}{1.401384in}}%
\pgfpathlineto{\pgfqpoint{0.748371in}{1.314264in}}%
\pgfpathlineto{\pgfqpoint{0.748474in}{1.352980in}}%
\pgfpathlineto{\pgfqpoint{0.748785in}{1.097009in}}%
\pgfpathlineto{\pgfqpoint{0.748889in}{1.204242in}}%
\pgfpathlineto{\pgfqpoint{0.749096in}{1.024667in}}%
\pgfpathlineto{\pgfqpoint{0.749303in}{1.321989in}}%
\pgfpathlineto{\pgfqpoint{0.749925in}{1.248186in}}%
\pgfpathlineto{\pgfqpoint{0.750340in}{1.300076in}}%
\pgfpathlineto{\pgfqpoint{0.750547in}{1.201464in}}%
\pgfpathlineto{\pgfqpoint{0.750651in}{1.261258in}}%
\pgfpathlineto{\pgfqpoint{0.750754in}{1.184976in}}%
\pgfpathlineto{\pgfqpoint{0.751480in}{1.377112in}}%
\pgfpathlineto{\pgfqpoint{0.751583in}{1.357403in}}%
\pgfpathlineto{\pgfqpoint{0.753138in}{1.082597in}}%
\pgfpathlineto{\pgfqpoint{0.753656in}{1.341465in}}%
\pgfpathlineto{\pgfqpoint{0.754382in}{1.230683in}}%
\pgfpathlineto{\pgfqpoint{0.754485in}{1.225815in}}%
\pgfpathlineto{\pgfqpoint{0.754589in}{1.254673in}}%
\pgfpathlineto{\pgfqpoint{0.754796in}{1.341118in}}%
\pgfpathlineto{\pgfqpoint{0.755003in}{1.054117in}}%
\pgfpathlineto{\pgfqpoint{0.755418in}{1.082418in}}%
\pgfpathlineto{\pgfqpoint{0.755936in}{0.966725in}}%
\pgfpathlineto{\pgfqpoint{0.755729in}{1.181405in}}%
\pgfpathlineto{\pgfqpoint{0.756454in}{1.087211in}}%
\pgfpathlineto{\pgfqpoint{0.757387in}{1.045208in}}%
\pgfpathlineto{\pgfqpoint{0.757594in}{1.210298in}}%
\pgfpathlineto{\pgfqpoint{0.757802in}{0.994439in}}%
\pgfpathlineto{\pgfqpoint{0.758424in}{1.245027in}}%
\pgfpathlineto{\pgfqpoint{0.758631in}{1.163722in}}%
\pgfpathlineto{\pgfqpoint{0.759253in}{1.365120in}}%
\pgfpathlineto{\pgfqpoint{0.759771in}{1.354662in}}%
\pgfpathlineto{\pgfqpoint{0.760600in}{1.124259in}}%
\pgfpathlineto{\pgfqpoint{0.760911in}{1.282703in}}%
\pgfpathlineto{\pgfqpoint{0.761222in}{1.133328in}}%
\pgfpathlineto{\pgfqpoint{0.761947in}{1.418904in}}%
\pgfpathlineto{\pgfqpoint{0.763191in}{1.119258in}}%
\pgfpathlineto{\pgfqpoint{0.763295in}{1.334814in}}%
\pgfpathlineto{\pgfqpoint{0.764331in}{1.243833in}}%
\pgfpathlineto{\pgfqpoint{0.764849in}{1.409041in}}%
\pgfpathlineto{\pgfqpoint{0.765367in}{1.235881in}}%
\pgfpathlineto{\pgfqpoint{0.765471in}{1.296183in}}%
\pgfpathlineto{\pgfqpoint{0.766093in}{1.527159in}}%
\pgfpathlineto{\pgfqpoint{0.766300in}{1.290792in}}%
\pgfpathlineto{\pgfqpoint{0.766715in}{1.438197in}}%
\pgfpathlineto{\pgfqpoint{0.767233in}{1.248492in}}%
\pgfpathlineto{\pgfqpoint{0.768062in}{1.311692in}}%
\pgfpathlineto{\pgfqpoint{0.768166in}{1.314648in}}%
\pgfpathlineto{\pgfqpoint{0.768788in}{1.247716in}}%
\pgfpathlineto{\pgfqpoint{0.769099in}{1.398559in}}%
\pgfpathlineto{\pgfqpoint{0.769202in}{1.312527in}}%
\pgfpathlineto{\pgfqpoint{0.769409in}{1.378801in}}%
\pgfpathlineto{\pgfqpoint{0.769513in}{1.243577in}}%
\pgfpathlineto{\pgfqpoint{0.769720in}{1.120023in}}%
\pgfpathlineto{\pgfqpoint{0.770031in}{1.409615in}}%
\pgfpathlineto{\pgfqpoint{0.770342in}{1.398026in}}%
\pgfpathlineto{\pgfqpoint{0.770549in}{1.535165in}}%
\pgfpathlineto{\pgfqpoint{0.771171in}{1.354033in}}%
\pgfpathlineto{\pgfqpoint{0.771379in}{1.358290in}}%
\pgfpathlineto{\pgfqpoint{0.772311in}{1.200569in}}%
\pgfpathlineto{\pgfqpoint{0.771897in}{1.482621in}}%
\pgfpathlineto{\pgfqpoint{0.772519in}{1.273165in}}%
\pgfpathlineto{\pgfqpoint{0.773762in}{1.526496in}}%
\pgfpathlineto{\pgfqpoint{0.775006in}{1.041784in}}%
\pgfpathlineto{\pgfqpoint{0.775110in}{1.152414in}}%
\pgfpathlineto{\pgfqpoint{0.775317in}{1.377804in}}%
\pgfpathlineto{\pgfqpoint{0.776250in}{1.297696in}}%
\pgfpathlineto{\pgfqpoint{0.776975in}{1.363585in}}%
\pgfpathlineto{\pgfqpoint{0.777390in}{1.095914in}}%
\pgfpathlineto{\pgfqpoint{0.777908in}{1.292308in}}%
\pgfpathlineto{\pgfqpoint{0.778426in}{1.223707in}}%
\pgfpathlineto{\pgfqpoint{0.778841in}{1.103244in}}%
\pgfpathlineto{\pgfqpoint{0.779359in}{1.269975in}}%
\pgfpathlineto{\pgfqpoint{0.779463in}{1.165576in}}%
\pgfpathlineto{\pgfqpoint{0.779773in}{1.269323in}}%
\pgfpathlineto{\pgfqpoint{0.780292in}{1.136907in}}%
\pgfpathlineto{\pgfqpoint{0.780395in}{0.975741in}}%
\pgfpathlineto{\pgfqpoint{0.781017in}{1.270501in}}%
\pgfpathlineto{\pgfqpoint{0.781328in}{1.156036in}}%
\pgfpathlineto{\pgfqpoint{0.781432in}{1.262207in}}%
\pgfpathlineto{\pgfqpoint{0.781846in}{1.089053in}}%
\pgfpathlineto{\pgfqpoint{0.782364in}{1.177707in}}%
\pgfpathlineto{\pgfqpoint{0.782468in}{1.079604in}}%
\pgfpathlineto{\pgfqpoint{0.782883in}{1.393544in}}%
\pgfpathlineto{\pgfqpoint{0.783297in}{1.293719in}}%
\pgfpathlineto{\pgfqpoint{0.783504in}{1.150406in}}%
\pgfpathlineto{\pgfqpoint{0.784126in}{1.352389in}}%
\pgfpathlineto{\pgfqpoint{0.784230in}{1.307490in}}%
\pgfpathlineto{\pgfqpoint{0.784334in}{1.348532in}}%
\pgfpathlineto{\pgfqpoint{0.784645in}{1.226549in}}%
\pgfpathlineto{\pgfqpoint{0.784955in}{1.238532in}}%
\pgfpathlineto{\pgfqpoint{0.785059in}{1.184567in}}%
\pgfpathlineto{\pgfqpoint{0.785266in}{1.383250in}}%
\pgfpathlineto{\pgfqpoint{0.785888in}{1.224645in}}%
\pgfpathlineto{\pgfqpoint{0.786406in}{1.504051in}}%
\pgfpathlineto{\pgfqpoint{0.787236in}{1.419919in}}%
\pgfpathlineto{\pgfqpoint{0.787754in}{1.220462in}}%
\pgfpathlineto{\pgfqpoint{0.787546in}{1.447459in}}%
\pgfpathlineto{\pgfqpoint{0.788376in}{1.324154in}}%
\pgfpathlineto{\pgfqpoint{0.788479in}{1.338902in}}%
\pgfpathlineto{\pgfqpoint{0.788790in}{1.258324in}}%
\pgfpathlineto{\pgfqpoint{0.788894in}{1.209483in}}%
\pgfpathlineto{\pgfqpoint{0.789412in}{1.437326in}}%
\pgfpathlineto{\pgfqpoint{0.789619in}{1.376621in}}%
\pgfpathlineto{\pgfqpoint{0.790034in}{1.238588in}}%
\pgfpathlineto{\pgfqpoint{0.790759in}{1.287793in}}%
\pgfpathlineto{\pgfqpoint{0.791174in}{1.417919in}}%
\pgfpathlineto{\pgfqpoint{0.790967in}{1.235377in}}%
\pgfpathlineto{\pgfqpoint{0.792003in}{1.414696in}}%
\pgfpathlineto{\pgfqpoint{0.792936in}{1.213456in}}%
\pgfpathlineto{\pgfqpoint{0.792418in}{1.491742in}}%
\pgfpathlineto{\pgfqpoint{0.793454in}{1.329175in}}%
\pgfpathlineto{\pgfqpoint{0.794179in}{1.291671in}}%
\pgfpathlineto{\pgfqpoint{0.794698in}{1.481443in}}%
\pgfpathlineto{\pgfqpoint{0.795527in}{1.199202in}}%
\pgfpathlineto{\pgfqpoint{0.795941in}{1.323603in}}%
\pgfpathlineto{\pgfqpoint{0.796045in}{1.328099in}}%
\pgfpathlineto{\pgfqpoint{0.796770in}{1.427844in}}%
\pgfpathlineto{\pgfqpoint{0.796874in}{1.199182in}}%
\pgfpathlineto{\pgfqpoint{0.796978in}{1.324726in}}%
\pgfpathlineto{\pgfqpoint{0.797185in}{1.167167in}}%
\pgfpathlineto{\pgfqpoint{0.797703in}{1.449935in}}%
\pgfpathlineto{\pgfqpoint{0.797910in}{1.523406in}}%
\pgfpathlineto{\pgfqpoint{0.798221in}{1.313897in}}%
\pgfpathlineto{\pgfqpoint{0.798325in}{1.337234in}}%
\pgfpathlineto{\pgfqpoint{0.799465in}{1.265942in}}%
\pgfpathlineto{\pgfqpoint{0.799050in}{1.495366in}}%
\pgfpathlineto{\pgfqpoint{0.799569in}{1.270126in}}%
\pgfpathlineto{\pgfqpoint{0.799672in}{1.298604in}}%
\pgfpathlineto{\pgfqpoint{0.800191in}{1.151392in}}%
\pgfpathlineto{\pgfqpoint{0.800398in}{1.263376in}}%
\pgfpathlineto{\pgfqpoint{0.801020in}{1.153462in}}%
\pgfpathlineto{\pgfqpoint{0.800916in}{1.323327in}}%
\pgfpathlineto{\pgfqpoint{0.801538in}{1.211759in}}%
\pgfpathlineto{\pgfqpoint{0.801641in}{1.213909in}}%
\pgfpathlineto{\pgfqpoint{0.802574in}{1.060827in}}%
\pgfpathlineto{\pgfqpoint{0.802056in}{1.385532in}}%
\pgfpathlineto{\pgfqpoint{0.802678in}{1.155497in}}%
\pgfpathlineto{\pgfqpoint{0.803403in}{1.318404in}}%
\pgfpathlineto{\pgfqpoint{0.802885in}{1.091980in}}%
\pgfpathlineto{\pgfqpoint{0.803611in}{1.194635in}}%
\pgfpathlineto{\pgfqpoint{0.803714in}{1.073754in}}%
\pgfpathlineto{\pgfqpoint{0.804751in}{1.109272in}}%
\pgfpathlineto{\pgfqpoint{0.805373in}{1.291865in}}%
\pgfpathlineto{\pgfqpoint{0.805891in}{1.204092in}}%
\pgfpathlineto{\pgfqpoint{0.805994in}{1.206039in}}%
\pgfpathlineto{\pgfqpoint{0.806098in}{1.195597in}}%
\pgfpathlineto{\pgfqpoint{0.806202in}{1.195921in}}%
\pgfpathlineto{\pgfqpoint{0.806305in}{1.073943in}}%
\pgfpathlineto{\pgfqpoint{0.807134in}{1.233030in}}%
\pgfpathlineto{\pgfqpoint{0.807238in}{1.169874in}}%
\pgfpathlineto{\pgfqpoint{0.807860in}{1.322866in}}%
\pgfpathlineto{\pgfqpoint{0.808378in}{1.306507in}}%
\pgfpathlineto{\pgfqpoint{0.808793in}{1.160129in}}%
\pgfpathlineto{\pgfqpoint{0.809000in}{1.331539in}}%
\pgfpathlineto{\pgfqpoint{0.810140in}{1.141125in}}%
\pgfpathlineto{\pgfqpoint{0.809829in}{1.365732in}}%
\pgfpathlineto{\pgfqpoint{0.810244in}{1.141727in}}%
\pgfpathlineto{\pgfqpoint{0.810451in}{1.338943in}}%
\pgfpathlineto{\pgfqpoint{0.811384in}{1.245539in}}%
\pgfpathlineto{\pgfqpoint{0.812005in}{1.181255in}}%
\pgfpathlineto{\pgfqpoint{0.812627in}{1.325767in}}%
\pgfpathlineto{\pgfqpoint{0.813042in}{1.173764in}}%
\pgfpathlineto{\pgfqpoint{0.813249in}{1.108029in}}%
\pgfpathlineto{\pgfqpoint{0.813456in}{1.302804in}}%
\pgfpathlineto{\pgfqpoint{0.813560in}{1.404391in}}%
\pgfpathlineto{\pgfqpoint{0.814182in}{1.164577in}}%
\pgfpathlineto{\pgfqpoint{0.814389in}{1.166594in}}%
\pgfpathlineto{\pgfqpoint{0.815115in}{1.025580in}}%
\pgfpathlineto{\pgfqpoint{0.814907in}{1.309962in}}%
\pgfpathlineto{\pgfqpoint{0.815322in}{1.204344in}}%
\pgfpathlineto{\pgfqpoint{0.815944in}{1.374002in}}%
\pgfpathlineto{\pgfqpoint{0.815529in}{1.162979in}}%
\pgfpathlineto{\pgfqpoint{0.816462in}{1.277308in}}%
\pgfpathlineto{\pgfqpoint{0.816566in}{1.282925in}}%
\pgfpathlineto{\pgfqpoint{0.816669in}{1.237941in}}%
\pgfpathlineto{\pgfqpoint{0.816773in}{1.262578in}}%
\pgfpathlineto{\pgfqpoint{0.817602in}{1.148076in}}%
\pgfpathlineto{\pgfqpoint{0.817395in}{1.324326in}}%
\pgfpathlineto{\pgfqpoint{0.818017in}{1.161849in}}%
\pgfpathlineto{\pgfqpoint{0.818328in}{1.381415in}}%
\pgfpathlineto{\pgfqpoint{0.818224in}{1.098965in}}%
\pgfpathlineto{\pgfqpoint{0.819157in}{1.322156in}}%
\pgfpathlineto{\pgfqpoint{0.819778in}{1.118201in}}%
\pgfpathlineto{\pgfqpoint{0.819675in}{1.352333in}}%
\pgfpathlineto{\pgfqpoint{0.820193in}{1.228953in}}%
\pgfpathlineto{\pgfqpoint{0.820400in}{1.348904in}}%
\pgfpathlineto{\pgfqpoint{0.820815in}{1.149184in}}%
\pgfpathlineto{\pgfqpoint{0.821333in}{1.327251in}}%
\pgfpathlineto{\pgfqpoint{0.821437in}{1.329859in}}%
\pgfpathlineto{\pgfqpoint{0.821540in}{1.257378in}}%
\pgfpathlineto{\pgfqpoint{0.822266in}{1.425246in}}%
\pgfpathlineto{\pgfqpoint{0.822577in}{1.303225in}}%
\pgfpathlineto{\pgfqpoint{0.823199in}{1.108260in}}%
\pgfpathlineto{\pgfqpoint{0.823613in}{1.238779in}}%
\pgfpathlineto{\pgfqpoint{0.823717in}{1.291529in}}%
\pgfpathlineto{\pgfqpoint{0.824131in}{1.103933in}}%
\pgfpathlineto{\pgfqpoint{0.824442in}{1.171672in}}%
\pgfpathlineto{\pgfqpoint{0.825271in}{0.955039in}}%
\pgfpathlineto{\pgfqpoint{0.825064in}{1.258697in}}%
\pgfpathlineto{\pgfqpoint{0.825479in}{1.228689in}}%
\pgfpathlineto{\pgfqpoint{0.826411in}{1.440773in}}%
\pgfpathlineto{\pgfqpoint{0.825997in}{1.153957in}}%
\pgfpathlineto{\pgfqpoint{0.826722in}{1.280429in}}%
\pgfpathlineto{\pgfqpoint{0.827344in}{1.138283in}}%
\pgfpathlineto{\pgfqpoint{0.827862in}{1.212175in}}%
\pgfpathlineto{\pgfqpoint{0.827966in}{1.266467in}}%
\pgfpathlineto{\pgfqpoint{0.828484in}{1.084091in}}%
\pgfpathlineto{\pgfqpoint{0.829002in}{1.244901in}}%
\pgfpathlineto{\pgfqpoint{0.829624in}{1.024350in}}%
\pgfpathlineto{\pgfqpoint{0.829935in}{1.254898in}}%
\pgfpathlineto{\pgfqpoint{0.830142in}{1.328258in}}%
\pgfpathlineto{\pgfqpoint{0.830557in}{1.206231in}}%
\pgfpathlineto{\pgfqpoint{0.830868in}{1.266801in}}%
\pgfpathlineto{\pgfqpoint{0.831386in}{1.041728in}}%
\pgfpathlineto{\pgfqpoint{0.831179in}{1.277884in}}%
\pgfpathlineto{\pgfqpoint{0.832112in}{1.201596in}}%
\pgfpathlineto{\pgfqpoint{0.832319in}{1.235625in}}%
\pgfpathlineto{\pgfqpoint{0.832941in}{1.070480in}}%
\pgfpathlineto{\pgfqpoint{0.832733in}{1.245722in}}%
\pgfpathlineto{\pgfqpoint{0.833355in}{1.095658in}}%
\pgfpathlineto{\pgfqpoint{0.833666in}{1.252742in}}%
\pgfpathlineto{\pgfqpoint{0.834392in}{1.177872in}}%
\pgfpathlineto{\pgfqpoint{0.835014in}{0.952925in}}%
\pgfpathlineto{\pgfqpoint{0.835221in}{1.240551in}}%
\pgfpathlineto{\pgfqpoint{0.835428in}{1.190412in}}%
\pgfpathlineto{\pgfqpoint{0.835843in}{1.108322in}}%
\pgfpathlineto{\pgfqpoint{0.836154in}{1.259698in}}%
\pgfpathlineto{\pgfqpoint{0.836361in}{1.232773in}}%
\pgfpathlineto{\pgfqpoint{0.836983in}{1.317258in}}%
\pgfpathlineto{\pgfqpoint{0.837086in}{1.177823in}}%
\pgfpathlineto{\pgfqpoint{0.837812in}{1.293867in}}%
\pgfpathlineto{\pgfqpoint{0.838226in}{1.049257in}}%
\pgfpathlineto{\pgfqpoint{0.839781in}{1.328888in}}%
\pgfpathlineto{\pgfqpoint{0.840092in}{1.093890in}}%
\pgfpathlineto{\pgfqpoint{0.840817in}{1.131177in}}%
\pgfpathlineto{\pgfqpoint{0.841232in}{1.320548in}}%
\pgfpathlineto{\pgfqpoint{0.841854in}{1.095170in}}%
\pgfpathlineto{\pgfqpoint{0.841957in}{1.265092in}}%
\pgfpathlineto{\pgfqpoint{0.842476in}{1.076763in}}%
\pgfpathlineto{\pgfqpoint{0.842683in}{1.299864in}}%
\pgfpathlineto{\pgfqpoint{0.843097in}{1.190393in}}%
\pgfpathlineto{\pgfqpoint{0.843305in}{1.137869in}}%
\pgfpathlineto{\pgfqpoint{0.843408in}{1.193616in}}%
\pgfpathlineto{\pgfqpoint{0.843512in}{1.306836in}}%
\pgfpathlineto{\pgfqpoint{0.843927in}{1.119952in}}%
\pgfpathlineto{\pgfqpoint{0.844445in}{1.216823in}}%
\pgfpathlineto{\pgfqpoint{0.844548in}{1.128384in}}%
\pgfpathlineto{\pgfqpoint{0.845378in}{1.330923in}}%
\pgfpathlineto{\pgfqpoint{0.845481in}{1.243765in}}%
\pgfpathlineto{\pgfqpoint{0.846414in}{1.112433in}}%
\pgfpathlineto{\pgfqpoint{0.846207in}{1.336755in}}%
\pgfpathlineto{\pgfqpoint{0.846621in}{1.172993in}}%
\pgfpathlineto{\pgfqpoint{0.846829in}{1.089139in}}%
\pgfpathlineto{\pgfqpoint{0.847865in}{1.342354in}}%
\pgfpathlineto{\pgfqpoint{0.849109in}{1.107504in}}%
\pgfpathlineto{\pgfqpoint{0.849523in}{1.011156in}}%
\pgfpathlineto{\pgfqpoint{0.850249in}{1.278596in}}%
\pgfpathlineto{\pgfqpoint{0.851285in}{1.047801in}}%
\pgfpathlineto{\pgfqpoint{0.851492in}{1.057898in}}%
\pgfpathlineto{\pgfqpoint{0.851803in}{1.036380in}}%
\pgfpathlineto{\pgfqpoint{0.852840in}{1.297332in}}%
\pgfpathlineto{\pgfqpoint{0.852943in}{1.296336in}}%
\pgfpathlineto{\pgfqpoint{0.853254in}{1.134244in}}%
\pgfpathlineto{\pgfqpoint{0.854187in}{1.192456in}}%
\pgfpathlineto{\pgfqpoint{0.854394in}{1.470915in}}%
\pgfpathlineto{\pgfqpoint{0.855431in}{1.372372in}}%
\pgfpathlineto{\pgfqpoint{0.856260in}{1.422505in}}%
\pgfpathlineto{\pgfqpoint{0.855845in}{1.217055in}}%
\pgfpathlineto{\pgfqpoint{0.856467in}{1.380983in}}%
\pgfpathlineto{\pgfqpoint{0.857193in}{1.194674in}}%
\pgfpathlineto{\pgfqpoint{0.857607in}{1.327591in}}%
\pgfpathlineto{\pgfqpoint{0.858436in}{1.133366in}}%
\pgfpathlineto{\pgfqpoint{0.858229in}{1.332637in}}%
\pgfpathlineto{\pgfqpoint{0.858954in}{1.221875in}}%
\pgfpathlineto{\pgfqpoint{0.859265in}{1.333899in}}%
\pgfpathlineto{\pgfqpoint{0.859887in}{1.151980in}}%
\pgfpathlineto{\pgfqpoint{0.859991in}{1.238276in}}%
\pgfpathlineto{\pgfqpoint{0.860302in}{1.062870in}}%
\pgfpathlineto{\pgfqpoint{0.860613in}{1.279131in}}%
\pgfpathlineto{\pgfqpoint{0.861131in}{1.179491in}}%
\pgfpathlineto{\pgfqpoint{0.861649in}{1.337885in}}%
\pgfpathlineto{\pgfqpoint{0.862271in}{1.108571in}}%
\pgfpathlineto{\pgfqpoint{0.863100in}{1.249270in}}%
\pgfpathlineto{\pgfqpoint{0.863307in}{1.077886in}}%
\pgfpathlineto{\pgfqpoint{0.863929in}{0.980896in}}%
\pgfpathlineto{\pgfqpoint{0.863618in}{1.078665in}}%
\pgfpathlineto{\pgfqpoint{0.864447in}{1.044174in}}%
\pgfpathlineto{\pgfqpoint{0.864551in}{1.051697in}}%
\pgfpathlineto{\pgfqpoint{0.865380in}{1.270453in}}%
\pgfpathlineto{\pgfqpoint{0.865691in}{1.200808in}}%
\pgfpathlineto{\pgfqpoint{0.866209in}{1.114449in}}%
\pgfpathlineto{\pgfqpoint{0.866106in}{1.225647in}}%
\pgfpathlineto{\pgfqpoint{0.866313in}{1.225217in}}%
\pgfpathlineto{\pgfqpoint{0.866416in}{1.277058in}}%
\pgfpathlineto{\pgfqpoint{0.866727in}{1.098230in}}%
\pgfpathlineto{\pgfqpoint{0.867349in}{1.243844in}}%
\pgfpathlineto{\pgfqpoint{0.868282in}{1.073541in}}%
\pgfpathlineto{\pgfqpoint{0.868386in}{1.250374in}}%
\pgfpathlineto{\pgfqpoint{0.868489in}{1.299364in}}%
\pgfpathlineto{\pgfqpoint{0.869215in}{1.142016in}}%
\pgfpathlineto{\pgfqpoint{0.869526in}{1.071462in}}%
\pgfpathlineto{\pgfqpoint{0.869837in}{1.155340in}}%
\pgfpathlineto{\pgfqpoint{0.870769in}{1.142154in}}%
\pgfpathlineto{\pgfqpoint{0.870977in}{1.272206in}}%
\pgfpathlineto{\pgfqpoint{0.871806in}{1.166436in}}%
\pgfpathlineto{\pgfqpoint{0.871599in}{1.407235in}}%
\pgfpathlineto{\pgfqpoint{0.872117in}{1.230420in}}%
\pgfpathlineto{\pgfqpoint{0.872739in}{1.416057in}}%
\pgfpathlineto{\pgfqpoint{0.872324in}{1.186161in}}%
\pgfpathlineto{\pgfqpoint{0.873153in}{1.256460in}}%
\pgfpathlineto{\pgfqpoint{0.873360in}{1.257162in}}%
\pgfpathlineto{\pgfqpoint{0.873464in}{1.121537in}}%
\pgfpathlineto{\pgfqpoint{0.873879in}{1.272667in}}%
\pgfpathlineto{\pgfqpoint{0.874397in}{1.250694in}}%
\pgfpathlineto{\pgfqpoint{0.874500in}{1.249892in}}%
\pgfpathlineto{\pgfqpoint{0.874708in}{1.313847in}}%
\pgfpathlineto{\pgfqpoint{0.875019in}{1.148428in}}%
\pgfpathlineto{\pgfqpoint{0.875122in}{1.130828in}}%
\pgfpathlineto{\pgfqpoint{0.875226in}{1.371541in}}%
\pgfpathlineto{\pgfqpoint{0.876262in}{1.247114in}}%
\pgfpathlineto{\pgfqpoint{0.876366in}{1.087606in}}%
\pgfpathlineto{\pgfqpoint{0.876781in}{1.374907in}}%
\pgfpathlineto{\pgfqpoint{0.877299in}{1.306267in}}%
\pgfpathlineto{\pgfqpoint{0.877506in}{1.433267in}}%
\pgfpathlineto{\pgfqpoint{0.877817in}{1.234817in}}%
\pgfpathlineto{\pgfqpoint{0.877921in}{1.296044in}}%
\pgfpathlineto{\pgfqpoint{0.878853in}{1.129552in}}%
\pgfpathlineto{\pgfqpoint{0.878439in}{1.326121in}}%
\pgfpathlineto{\pgfqpoint{0.879061in}{1.163542in}}%
\pgfpathlineto{\pgfqpoint{0.879682in}{1.386181in}}%
\pgfpathlineto{\pgfqpoint{0.879890in}{1.104197in}}%
\pgfpathlineto{\pgfqpoint{0.880097in}{1.279445in}}%
\pgfpathlineto{\pgfqpoint{0.880408in}{1.060990in}}%
\pgfpathlineto{\pgfqpoint{0.881444in}{1.084034in}}%
\pgfpathlineto{\pgfqpoint{0.882792in}{1.377654in}}%
\pgfpathlineto{\pgfqpoint{0.883724in}{1.187211in}}%
\pgfpathlineto{\pgfqpoint{0.883310in}{1.390941in}}%
\pgfpathlineto{\pgfqpoint{0.884139in}{1.240680in}}%
\pgfpathlineto{\pgfqpoint{0.884346in}{1.301176in}}%
\pgfpathlineto{\pgfqpoint{0.884968in}{1.089505in}}%
\pgfpathlineto{\pgfqpoint{0.885486in}{1.121500in}}%
\pgfpathlineto{\pgfqpoint{0.885694in}{0.978563in}}%
\pgfpathlineto{\pgfqpoint{0.886004in}{1.219183in}}%
\pgfpathlineto{\pgfqpoint{0.886315in}{1.147077in}}%
\pgfpathlineto{\pgfqpoint{0.887352in}{1.349280in}}%
\pgfpathlineto{\pgfqpoint{0.887766in}{1.339789in}}%
\pgfpathlineto{\pgfqpoint{0.887974in}{1.238289in}}%
\pgfpathlineto{\pgfqpoint{0.888595in}{1.400103in}}%
\pgfpathlineto{\pgfqpoint{0.888699in}{1.356971in}}%
\pgfpathlineto{\pgfqpoint{0.889114in}{1.431771in}}%
\pgfpathlineto{\pgfqpoint{0.889425in}{1.249388in}}%
\pgfpathlineto{\pgfqpoint{0.889839in}{1.120320in}}%
\pgfpathlineto{\pgfqpoint{0.890254in}{1.304091in}}%
\pgfpathlineto{\pgfqpoint{0.890357in}{1.294446in}}%
\pgfpathlineto{\pgfqpoint{0.890461in}{1.338741in}}%
\pgfpathlineto{\pgfqpoint{0.891186in}{1.206041in}}%
\pgfpathlineto{\pgfqpoint{0.891705in}{1.138251in}}%
\pgfpathlineto{\pgfqpoint{0.892119in}{1.255459in}}%
\pgfpathlineto{\pgfqpoint{0.892223in}{1.186796in}}%
\pgfpathlineto{\pgfqpoint{0.892430in}{1.248890in}}%
\pgfpathlineto{\pgfqpoint{0.892741in}{1.048205in}}%
\pgfpathlineto{\pgfqpoint{0.893259in}{1.184931in}}%
\pgfpathlineto{\pgfqpoint{0.893985in}{1.110140in}}%
\pgfpathlineto{\pgfqpoint{0.894192in}{1.236239in}}%
\pgfpathlineto{\pgfqpoint{0.894399in}{1.156601in}}%
\pgfpathlineto{\pgfqpoint{0.895436in}{1.342359in}}%
\pgfpathlineto{\pgfqpoint{0.895539in}{1.432406in}}%
\pgfpathlineto{\pgfqpoint{0.896368in}{1.300833in}}%
\pgfpathlineto{\pgfqpoint{0.896576in}{1.393687in}}%
\pgfpathlineto{\pgfqpoint{0.896887in}{1.165521in}}%
\pgfpathlineto{\pgfqpoint{0.897405in}{1.410658in}}%
\pgfpathlineto{\pgfqpoint{0.897612in}{1.233042in}}%
\pgfpathlineto{\pgfqpoint{0.898856in}{1.496813in}}%
\pgfpathlineto{\pgfqpoint{0.899892in}{1.240674in}}%
\pgfpathlineto{\pgfqpoint{0.900100in}{1.333411in}}%
\pgfpathlineto{\pgfqpoint{0.900203in}{1.251913in}}%
\pgfpathlineto{\pgfqpoint{0.900618in}{1.477453in}}%
\pgfpathlineto{\pgfqpoint{0.901240in}{1.286252in}}%
\pgfpathlineto{\pgfqpoint{0.901343in}{1.231634in}}%
\pgfpathlineto{\pgfqpoint{0.902069in}{1.368034in}}%
\pgfpathlineto{\pgfqpoint{0.902172in}{1.335511in}}%
\pgfpathlineto{\pgfqpoint{0.902276in}{1.359493in}}%
\pgfpathlineto{\pgfqpoint{0.902691in}{1.249659in}}%
\pgfpathlineto{\pgfqpoint{0.902794in}{1.264444in}}%
\pgfpathlineto{\pgfqpoint{0.903623in}{1.315794in}}%
\pgfpathlineto{\pgfqpoint{0.903934in}{1.145205in}}%
\pgfpathlineto{\pgfqpoint{0.904660in}{1.209812in}}%
\pgfpathlineto{\pgfqpoint{0.904349in}{1.085739in}}%
\pgfpathlineto{\pgfqpoint{0.904867in}{1.151246in}}%
\pgfpathlineto{\pgfqpoint{0.905385in}{1.051201in}}%
\pgfpathlineto{\pgfqpoint{0.905282in}{1.180535in}}%
\pgfpathlineto{\pgfqpoint{0.905696in}{1.159665in}}%
\pgfpathlineto{\pgfqpoint{0.905800in}{1.345307in}}%
\pgfpathlineto{\pgfqpoint{0.906214in}{1.080711in}}%
\pgfpathlineto{\pgfqpoint{0.906732in}{1.154986in}}%
\pgfpathlineto{\pgfqpoint{0.906836in}{1.121539in}}%
\pgfpathlineto{\pgfqpoint{0.907147in}{1.323286in}}%
\pgfpathlineto{\pgfqpoint{0.907354in}{1.305011in}}%
\pgfpathlineto{\pgfqpoint{0.907458in}{1.299199in}}%
\pgfpathlineto{\pgfqpoint{0.907562in}{1.301486in}}%
\pgfpathlineto{\pgfqpoint{0.908080in}{1.134839in}}%
\pgfpathlineto{\pgfqpoint{0.908391in}{1.341281in}}%
\pgfpathlineto{\pgfqpoint{0.908598in}{1.334747in}}%
\pgfpathlineto{\pgfqpoint{0.908909in}{1.141678in}}%
\pgfpathlineto{\pgfqpoint{0.909531in}{1.396737in}}%
\pgfpathlineto{\pgfqpoint{0.909634in}{1.478674in}}%
\pgfpathlineto{\pgfqpoint{0.910360in}{1.209225in}}%
\pgfpathlineto{\pgfqpoint{0.910878in}{1.154277in}}%
\pgfpathlineto{\pgfqpoint{0.911293in}{1.285996in}}%
\pgfpathlineto{\pgfqpoint{0.911085in}{1.147068in}}%
\pgfpathlineto{\pgfqpoint{0.912018in}{1.254711in}}%
\pgfpathlineto{\pgfqpoint{0.913158in}{1.092273in}}%
\pgfpathlineto{\pgfqpoint{0.912847in}{1.290417in}}%
\pgfpathlineto{\pgfqpoint{0.913262in}{1.120464in}}%
\pgfpathlineto{\pgfqpoint{0.913365in}{1.154750in}}%
\pgfpathlineto{\pgfqpoint{0.913676in}{1.079988in}}%
\pgfpathlineto{\pgfqpoint{0.913884in}{1.130969in}}%
\pgfpathlineto{\pgfqpoint{0.914402in}{0.887371in}}%
\pgfpathlineto{\pgfqpoint{0.914920in}{1.098045in}}%
\pgfpathlineto{\pgfqpoint{0.915024in}{1.102740in}}%
\pgfpathlineto{\pgfqpoint{0.915127in}{0.954582in}}%
\pgfpathlineto{\pgfqpoint{0.915542in}{1.383787in}}%
\pgfpathlineto{\pgfqpoint{0.916060in}{1.184805in}}%
\pgfpathlineto{\pgfqpoint{0.916786in}{1.058188in}}%
\pgfpathlineto{\pgfqpoint{0.916993in}{1.185904in}}%
\pgfpathlineto{\pgfqpoint{0.917304in}{1.097496in}}%
\pgfpathlineto{\pgfqpoint{0.918029in}{1.306605in}}%
\pgfpathlineto{\pgfqpoint{0.918547in}{1.221755in}}%
\pgfpathlineto{\pgfqpoint{0.919687in}{1.387617in}}%
\pgfpathlineto{\pgfqpoint{0.919377in}{1.205962in}}%
\pgfpathlineto{\pgfqpoint{0.919895in}{1.266975in}}%
\pgfpathlineto{\pgfqpoint{0.920931in}{1.006840in}}%
\pgfpathlineto{\pgfqpoint{0.920206in}{1.273600in}}%
\pgfpathlineto{\pgfqpoint{0.921138in}{1.020191in}}%
\pgfpathlineto{\pgfqpoint{0.922071in}{1.332589in}}%
\pgfpathlineto{\pgfqpoint{0.922278in}{1.177209in}}%
\pgfpathlineto{\pgfqpoint{0.922693in}{1.236485in}}%
\pgfpathlineto{\pgfqpoint{0.923522in}{1.021447in}}%
\pgfpathlineto{\pgfqpoint{0.924559in}{1.304790in}}%
\pgfpathlineto{\pgfqpoint{0.924662in}{1.285793in}}%
\pgfpathlineto{\pgfqpoint{0.925180in}{1.107395in}}%
\pgfpathlineto{\pgfqpoint{0.925388in}{1.347692in}}%
\pgfpathlineto{\pgfqpoint{0.925802in}{1.186532in}}%
\pgfpathlineto{\pgfqpoint{0.926735in}{1.257883in}}%
\pgfpathlineto{\pgfqpoint{0.926424in}{1.070285in}}%
\pgfpathlineto{\pgfqpoint{0.926839in}{1.243996in}}%
\pgfpathlineto{\pgfqpoint{0.927564in}{1.259399in}}%
\pgfpathlineto{\pgfqpoint{0.928186in}{0.918341in}}%
\pgfpathlineto{\pgfqpoint{0.928808in}{1.188177in}}%
\pgfpathlineto{\pgfqpoint{0.929430in}{1.112763in}}%
\pgfpathlineto{\pgfqpoint{0.929533in}{0.946480in}}%
\pgfpathlineto{\pgfqpoint{0.930466in}{1.047928in}}%
\pgfpathlineto{\pgfqpoint{0.931088in}{0.984202in}}%
\pgfpathlineto{\pgfqpoint{0.931502in}{1.196495in}}%
\pgfpathlineto{\pgfqpoint{0.932435in}{1.015075in}}%
\pgfpathlineto{\pgfqpoint{0.932642in}{1.080727in}}%
\pgfpathlineto{\pgfqpoint{0.933057in}{0.959856in}}%
\pgfpathlineto{\pgfqpoint{0.933264in}{1.171181in}}%
\pgfpathlineto{\pgfqpoint{0.933472in}{1.126028in}}%
\pgfpathlineto{\pgfqpoint{0.934715in}{1.284188in}}%
\pgfpathlineto{\pgfqpoint{0.933679in}{1.049387in}}%
\pgfpathlineto{\pgfqpoint{0.934819in}{1.247370in}}%
\pgfpathlineto{\pgfqpoint{0.935752in}{1.006590in}}%
\pgfpathlineto{\pgfqpoint{0.935959in}{1.111843in}}%
\pgfpathlineto{\pgfqpoint{0.937099in}{1.317894in}}%
\pgfpathlineto{\pgfqpoint{0.936166in}{1.077224in}}%
\pgfpathlineto{\pgfqpoint{0.937306in}{1.243373in}}%
\pgfpathlineto{\pgfqpoint{0.938135in}{1.036082in}}%
\pgfpathlineto{\pgfqpoint{0.938550in}{1.073282in}}%
\pgfpathlineto{\pgfqpoint{0.938757in}{1.262709in}}%
\pgfpathlineto{\pgfqpoint{0.939483in}{1.044853in}}%
\pgfpathlineto{\pgfqpoint{0.939690in}{1.169636in}}%
\pgfpathlineto{\pgfqpoint{0.940105in}{1.027465in}}%
\pgfpathlineto{\pgfqpoint{0.940519in}{1.232130in}}%
\pgfpathlineto{\pgfqpoint{0.940830in}{1.151472in}}%
\pgfpathlineto{\pgfqpoint{0.940934in}{1.216133in}}%
\pgfpathlineto{\pgfqpoint{0.941037in}{1.041402in}}%
\pgfpathlineto{\pgfqpoint{0.941763in}{1.096098in}}%
\pgfpathlineto{\pgfqpoint{0.942592in}{0.893348in}}%
\pgfpathlineto{\pgfqpoint{0.941970in}{1.142698in}}%
\pgfpathlineto{\pgfqpoint{0.943006in}{1.023786in}}%
\pgfpathlineto{\pgfqpoint{0.943110in}{1.076697in}}%
\pgfpathlineto{\pgfqpoint{0.943525in}{0.887218in}}%
\pgfpathlineto{\pgfqpoint{0.943939in}{0.978405in}}%
\pgfpathlineto{\pgfqpoint{0.944665in}{0.875535in}}%
\pgfpathlineto{\pgfqpoint{0.944354in}{1.030358in}}%
\pgfpathlineto{\pgfqpoint{0.944872in}{1.013260in}}%
\pgfpathlineto{\pgfqpoint{0.946323in}{1.280077in}}%
\pgfpathlineto{\pgfqpoint{0.945494in}{0.964683in}}%
\pgfpathlineto{\pgfqpoint{0.946427in}{1.149572in}}%
\pgfpathlineto{\pgfqpoint{0.946841in}{0.941993in}}%
\pgfpathlineto{\pgfqpoint{0.947463in}{1.166731in}}%
\pgfpathlineto{\pgfqpoint{0.947567in}{1.055770in}}%
\pgfpathlineto{\pgfqpoint{0.947774in}{1.004276in}}%
\pgfpathlineto{\pgfqpoint{0.947981in}{1.139151in}}%
\pgfpathlineto{\pgfqpoint{0.948085in}{1.138049in}}%
\pgfpathlineto{\pgfqpoint{0.948707in}{1.322428in}}%
\pgfpathlineto{\pgfqpoint{0.948292in}{1.068125in}}%
\pgfpathlineto{\pgfqpoint{0.949121in}{1.168784in}}%
\pgfpathlineto{\pgfqpoint{0.949225in}{1.077340in}}%
\pgfpathlineto{\pgfqpoint{0.949432in}{1.236350in}}%
\pgfpathlineto{\pgfqpoint{0.950158in}{1.187332in}}%
\pgfpathlineto{\pgfqpoint{0.950469in}{1.013912in}}%
\pgfpathlineto{\pgfqpoint{0.950572in}{1.198587in}}%
\pgfpathlineto{\pgfqpoint{0.951298in}{1.149853in}}%
\pgfpathlineto{\pgfqpoint{0.951401in}{1.340405in}}%
\pgfpathlineto{\pgfqpoint{0.952230in}{1.048532in}}%
\pgfpathlineto{\pgfqpoint{0.952334in}{1.150797in}}%
\pgfpathlineto{\pgfqpoint{0.953267in}{1.286898in}}%
\pgfpathlineto{\pgfqpoint{0.953060in}{1.113670in}}%
\pgfpathlineto{\pgfqpoint{0.953474in}{1.208568in}}%
\pgfpathlineto{\pgfqpoint{0.953578in}{1.138011in}}%
\pgfpathlineto{\pgfqpoint{0.953889in}{1.355646in}}%
\pgfpathlineto{\pgfqpoint{0.954511in}{1.153986in}}%
\pgfpathlineto{\pgfqpoint{0.955029in}{1.320674in}}%
\pgfpathlineto{\pgfqpoint{0.955547in}{1.221100in}}%
\pgfpathlineto{\pgfqpoint{0.956480in}{1.251882in}}%
\pgfpathlineto{\pgfqpoint{0.956583in}{1.037284in}}%
\pgfpathlineto{\pgfqpoint{0.957102in}{1.225965in}}%
\pgfpathlineto{\pgfqpoint{0.957723in}{1.148093in}}%
\pgfpathlineto{\pgfqpoint{0.957827in}{1.076822in}}%
\pgfpathlineto{\pgfqpoint{0.958345in}{1.326586in}}%
\pgfpathlineto{\pgfqpoint{0.958656in}{1.217004in}}%
\pgfpathlineto{\pgfqpoint{0.959485in}{1.377535in}}%
\pgfpathlineto{\pgfqpoint{0.959278in}{1.141956in}}%
\pgfpathlineto{\pgfqpoint{0.959693in}{1.210996in}}%
\pgfpathlineto{\pgfqpoint{0.959796in}{1.121160in}}%
\pgfpathlineto{\pgfqpoint{0.960625in}{1.269945in}}%
\pgfpathlineto{\pgfqpoint{0.960729in}{1.128787in}}%
\pgfpathlineto{\pgfqpoint{0.961040in}{1.341162in}}%
\pgfpathlineto{\pgfqpoint{0.961247in}{1.124612in}}%
\pgfpathlineto{\pgfqpoint{0.961869in}{1.176580in}}%
\pgfpathlineto{\pgfqpoint{0.962387in}{1.353759in}}%
\pgfpathlineto{\pgfqpoint{0.963009in}{1.259328in}}%
\pgfpathlineto{\pgfqpoint{0.964045in}{0.954990in}}%
\pgfpathlineto{\pgfqpoint{0.964356in}{1.087723in}}%
\pgfpathlineto{\pgfqpoint{0.965082in}{1.283766in}}%
\pgfpathlineto{\pgfqpoint{0.964875in}{1.054416in}}%
\pgfpathlineto{\pgfqpoint{0.965393in}{1.092690in}}%
\pgfpathlineto{\pgfqpoint{0.965496in}{1.047306in}}%
\pgfpathlineto{\pgfqpoint{0.965911in}{1.263874in}}%
\pgfpathlineto{\pgfqpoint{0.966015in}{1.224710in}}%
\pgfpathlineto{\pgfqpoint{0.966222in}{1.304083in}}%
\pgfpathlineto{\pgfqpoint{0.966844in}{1.103992in}}%
\pgfpathlineto{\pgfqpoint{0.966947in}{1.097213in}}%
\pgfpathlineto{\pgfqpoint{0.967155in}{1.137856in}}%
\pgfpathlineto{\pgfqpoint{0.967673in}{1.385204in}}%
\pgfpathlineto{\pgfqpoint{0.967362in}{1.123465in}}%
\pgfpathlineto{\pgfqpoint{0.968295in}{1.277742in}}%
\pgfpathlineto{\pgfqpoint{0.968398in}{1.082488in}}%
\pgfpathlineto{\pgfqpoint{0.969124in}{1.348393in}}%
\pgfpathlineto{\pgfqpoint{0.969435in}{1.202590in}}%
\pgfpathlineto{\pgfqpoint{0.969538in}{1.267494in}}%
\pgfpathlineto{\pgfqpoint{0.969953in}{1.136402in}}%
\pgfpathlineto{\pgfqpoint{0.970264in}{1.219304in}}%
\pgfpathlineto{\pgfqpoint{0.970367in}{1.082099in}}%
\pgfpathlineto{\pgfqpoint{0.971093in}{1.364555in}}%
\pgfpathlineto{\pgfqpoint{0.971300in}{1.297011in}}%
\pgfpathlineto{\pgfqpoint{0.971404in}{1.440822in}}%
\pgfpathlineto{\pgfqpoint{0.971818in}{1.201993in}}%
\pgfpathlineto{\pgfqpoint{0.972440in}{1.350144in}}%
\pgfpathlineto{\pgfqpoint{0.973477in}{0.961442in}}%
\pgfpathlineto{\pgfqpoint{0.973580in}{1.213648in}}%
\pgfpathlineto{\pgfqpoint{0.974202in}{1.351366in}}%
\pgfpathlineto{\pgfqpoint{0.973995in}{1.110914in}}%
\pgfpathlineto{\pgfqpoint{0.974513in}{1.298348in}}%
\pgfpathlineto{\pgfqpoint{0.975653in}{1.063995in}}%
\pgfpathlineto{\pgfqpoint{0.976068in}{1.253993in}}%
\pgfpathlineto{\pgfqpoint{0.976275in}{1.008153in}}%
\pgfpathlineto{\pgfqpoint{0.976793in}{1.195130in}}%
\pgfpathlineto{\pgfqpoint{0.976897in}{1.201733in}}%
\pgfpathlineto{\pgfqpoint{0.977104in}{1.026542in}}%
\pgfpathlineto{\pgfqpoint{0.977726in}{1.266578in}}%
\pgfpathlineto{\pgfqpoint{0.977933in}{1.210124in}}%
\pgfpathlineto{\pgfqpoint{0.978762in}{1.070148in}}%
\pgfpathlineto{\pgfqpoint{0.978244in}{1.251657in}}%
\pgfpathlineto{\pgfqpoint{0.979177in}{1.130997in}}%
\pgfpathlineto{\pgfqpoint{0.979488in}{1.273211in}}%
\pgfpathlineto{\pgfqpoint{0.980213in}{1.063760in}}%
\pgfpathlineto{\pgfqpoint{0.980317in}{1.163273in}}%
\pgfpathlineto{\pgfqpoint{0.980421in}{0.987761in}}%
\pgfpathlineto{\pgfqpoint{0.981042in}{1.233876in}}%
\pgfpathlineto{\pgfqpoint{0.981353in}{1.176816in}}%
\pgfpathlineto{\pgfqpoint{0.981871in}{1.319899in}}%
\pgfpathlineto{\pgfqpoint{0.981664in}{1.094966in}}%
\pgfpathlineto{\pgfqpoint{0.982597in}{1.310361in}}%
\pgfpathlineto{\pgfqpoint{0.983530in}{1.114040in}}%
\pgfpathlineto{\pgfqpoint{0.983115in}{1.369839in}}%
\pgfpathlineto{\pgfqpoint{0.983633in}{1.193079in}}%
\pgfpathlineto{\pgfqpoint{0.983737in}{1.366788in}}%
\pgfpathlineto{\pgfqpoint{0.984359in}{1.129265in}}%
\pgfpathlineto{\pgfqpoint{0.984566in}{1.137910in}}%
\pgfpathlineto{\pgfqpoint{0.984670in}{1.015439in}}%
\pgfpathlineto{\pgfqpoint{0.985395in}{1.291372in}}%
\pgfpathlineto{\pgfqpoint{0.985499in}{1.232250in}}%
\pgfpathlineto{\pgfqpoint{0.986432in}{1.084710in}}%
\pgfpathlineto{\pgfqpoint{0.986743in}{1.387764in}}%
\pgfpathlineto{\pgfqpoint{0.987261in}{1.112010in}}%
\pgfpathlineto{\pgfqpoint{0.987779in}{1.399369in}}%
\pgfpathlineto{\pgfqpoint{0.987883in}{1.209085in}}%
\pgfpathlineto{\pgfqpoint{0.988090in}{1.356860in}}%
\pgfpathlineto{\pgfqpoint{0.988712in}{1.185057in}}%
\pgfpathlineto{\pgfqpoint{0.989023in}{1.328099in}}%
\pgfpathlineto{\pgfqpoint{0.989437in}{1.175661in}}%
\pgfpathlineto{\pgfqpoint{0.990266in}{1.201366in}}%
\pgfpathlineto{\pgfqpoint{0.990370in}{1.205471in}}%
\pgfpathlineto{\pgfqpoint{0.990577in}{1.323337in}}%
\pgfpathlineto{\pgfqpoint{0.990888in}{1.103817in}}%
\pgfpathlineto{\pgfqpoint{0.991406in}{1.240476in}}%
\pgfpathlineto{\pgfqpoint{0.991510in}{1.115827in}}%
\pgfpathlineto{\pgfqpoint{0.992546in}{1.205185in}}%
\pgfpathlineto{\pgfqpoint{0.992650in}{1.237998in}}%
\pgfpathlineto{\pgfqpoint{0.993168in}{1.116601in}}%
\pgfpathlineto{\pgfqpoint{0.993272in}{1.072945in}}%
\pgfpathlineto{\pgfqpoint{0.993997in}{1.234262in}}%
\pgfpathlineto{\pgfqpoint{0.994101in}{1.156872in}}%
\pgfpathlineto{\pgfqpoint{0.994723in}{1.299748in}}%
\pgfpathlineto{\pgfqpoint{0.994516in}{1.054487in}}%
\pgfpathlineto{\pgfqpoint{0.995241in}{1.283949in}}%
\pgfpathlineto{\pgfqpoint{0.995345in}{1.153056in}}%
\pgfpathlineto{\pgfqpoint{0.995967in}{1.316315in}}%
\pgfpathlineto{\pgfqpoint{0.996381in}{1.245433in}}%
\pgfpathlineto{\pgfqpoint{0.996588in}{1.227122in}}%
\pgfpathlineto{\pgfqpoint{0.997521in}{1.397281in}}%
\pgfpathlineto{\pgfqpoint{0.996899in}{1.176815in}}%
\pgfpathlineto{\pgfqpoint{0.997728in}{1.387960in}}%
\pgfpathlineto{\pgfqpoint{0.998558in}{1.309242in}}%
\pgfpathlineto{\pgfqpoint{0.998350in}{1.432232in}}%
\pgfpathlineto{\pgfqpoint{0.998868in}{1.334758in}}%
\pgfpathlineto{\pgfqpoint{0.999076in}{1.458046in}}%
\pgfpathlineto{\pgfqpoint{0.999594in}{1.275678in}}%
\pgfpathlineto{\pgfqpoint{0.999905in}{1.341792in}}%
\pgfpathlineto{\pgfqpoint{1.000319in}{1.110240in}}%
\pgfpathlineto{\pgfqpoint{1.000734in}{1.511188in}}%
\pgfpathlineto{\pgfqpoint{1.000941in}{1.359568in}}%
\pgfpathlineto{\pgfqpoint{1.001149in}{1.488308in}}%
\pgfpathlineto{\pgfqpoint{1.001356in}{1.208982in}}%
\pgfpathlineto{\pgfqpoint{1.001770in}{1.378511in}}%
\pgfpathlineto{\pgfqpoint{1.002289in}{1.090667in}}%
\pgfpathlineto{\pgfqpoint{1.003014in}{1.114532in}}%
\pgfpathlineto{\pgfqpoint{1.003118in}{1.117897in}}%
\pgfpathlineto{\pgfqpoint{1.003740in}{1.031388in}}%
\pgfpathlineto{\pgfqpoint{1.004361in}{1.248550in}}%
\pgfpathlineto{\pgfqpoint{1.004672in}{1.029891in}}%
\pgfpathlineto{\pgfqpoint{1.004880in}{1.270069in}}%
\pgfpathlineto{\pgfqpoint{1.005398in}{1.151296in}}%
\pgfpathlineto{\pgfqpoint{1.006227in}{1.284848in}}%
\pgfpathlineto{\pgfqpoint{1.005812in}{1.145317in}}%
\pgfpathlineto{\pgfqpoint{1.006434in}{1.169918in}}%
\pgfpathlineto{\pgfqpoint{1.007056in}{0.937839in}}%
\pgfpathlineto{\pgfqpoint{1.007989in}{1.020808in}}%
\pgfpathlineto{\pgfqpoint{1.008300in}{1.169527in}}%
\pgfpathlineto{\pgfqpoint{1.009129in}{1.061511in}}%
\pgfpathlineto{\pgfqpoint{1.010062in}{0.912295in}}%
\pgfpathlineto{\pgfqpoint{1.010165in}{1.031876in}}%
\pgfpathlineto{\pgfqpoint{1.010269in}{1.087994in}}%
\pgfpathlineto{\pgfqpoint{1.010891in}{0.887642in}}%
\pgfpathlineto{\pgfqpoint{1.011098in}{0.969040in}}%
\pgfpathlineto{\pgfqpoint{1.011305in}{0.909687in}}%
\pgfpathlineto{\pgfqpoint{1.011616in}{0.991874in}}%
\pgfpathlineto{\pgfqpoint{1.012031in}{0.968634in}}%
\pgfpathlineto{\pgfqpoint{1.012653in}{1.127119in}}%
\pgfpathlineto{\pgfqpoint{1.013793in}{0.965358in}}%
\pgfpathlineto{\pgfqpoint{1.014104in}{1.177954in}}%
\pgfpathlineto{\pgfqpoint{1.014933in}{1.042284in}}%
\pgfpathlineto{\pgfqpoint{1.015036in}{1.050438in}}%
\pgfpathlineto{\pgfqpoint{1.015140in}{0.965809in}}%
\pgfpathlineto{\pgfqpoint{1.015244in}{1.059627in}}%
\pgfpathlineto{\pgfqpoint{1.016176in}{1.024294in}}%
\pgfpathlineto{\pgfqpoint{1.016487in}{1.151337in}}%
\pgfpathlineto{\pgfqpoint{1.016902in}{0.929764in}}%
\pgfpathlineto{\pgfqpoint{1.017316in}{1.092565in}}%
\pgfpathlineto{\pgfqpoint{1.018249in}{0.872521in}}%
\pgfpathlineto{\pgfqpoint{1.018456in}{0.978010in}}%
\pgfpathlineto{\pgfqpoint{1.019389in}{1.121962in}}%
\pgfpathlineto{\pgfqpoint{1.018975in}{0.899807in}}%
\pgfpathlineto{\pgfqpoint{1.019596in}{1.119071in}}%
\pgfpathlineto{\pgfqpoint{1.020322in}{1.203611in}}%
\pgfpathlineto{\pgfqpoint{1.020736in}{0.988479in}}%
\pgfpathlineto{\pgfqpoint{1.021773in}{1.216846in}}%
\pgfpathlineto{\pgfqpoint{1.021151in}{0.964484in}}%
\pgfpathlineto{\pgfqpoint{1.021980in}{1.117752in}}%
\pgfpathlineto{\pgfqpoint{1.022084in}{1.008478in}}%
\pgfpathlineto{\pgfqpoint{1.022913in}{1.209241in}}%
\pgfpathlineto{\pgfqpoint{1.023224in}{1.183678in}}%
\pgfpathlineto{\pgfqpoint{1.023431in}{1.256976in}}%
\pgfpathlineto{\pgfqpoint{1.024157in}{0.972229in}}%
\pgfpathlineto{\pgfqpoint{1.024571in}{1.150962in}}%
\pgfpathlineto{\pgfqpoint{1.024778in}{1.029738in}}%
\pgfpathlineto{\pgfqpoint{1.025089in}{1.224853in}}%
\pgfpathlineto{\pgfqpoint{1.025608in}{1.094191in}}%
\pgfpathlineto{\pgfqpoint{1.026644in}{1.041392in}}%
\pgfpathlineto{\pgfqpoint{1.026851in}{1.270833in}}%
\pgfpathlineto{\pgfqpoint{1.027059in}{1.107736in}}%
\pgfpathlineto{\pgfqpoint{1.027680in}{1.282827in}}%
\pgfpathlineto{\pgfqpoint{1.027888in}{1.156279in}}%
\pgfpathlineto{\pgfqpoint{1.028199in}{1.485452in}}%
\pgfpathlineto{\pgfqpoint{1.028509in}{1.129165in}}%
\pgfpathlineto{\pgfqpoint{1.028924in}{1.209211in}}%
\pgfpathlineto{\pgfqpoint{1.029650in}{0.952712in}}%
\pgfpathlineto{\pgfqpoint{1.030064in}{1.071809in}}%
\pgfpathlineto{\pgfqpoint{1.030375in}{1.142863in}}%
\pgfpathlineto{\pgfqpoint{1.030479in}{1.035545in}}%
\pgfpathlineto{\pgfqpoint{1.030582in}{1.121731in}}%
\pgfpathlineto{\pgfqpoint{1.030686in}{0.926874in}}%
\pgfpathlineto{\pgfqpoint{1.031722in}{0.993258in}}%
\pgfpathlineto{\pgfqpoint{1.032137in}{1.193736in}}%
\pgfpathlineto{\pgfqpoint{1.032344in}{0.934276in}}%
\pgfpathlineto{\pgfqpoint{1.032759in}{0.976249in}}%
\pgfpathlineto{\pgfqpoint{1.033173in}{1.161218in}}%
\pgfpathlineto{\pgfqpoint{1.033381in}{0.984338in}}%
\pgfpathlineto{\pgfqpoint{1.033484in}{0.835528in}}%
\pgfpathlineto{\pgfqpoint{1.034313in}{1.079699in}}%
\pgfpathlineto{\pgfqpoint{1.034417in}{0.946541in}}%
\pgfpathlineto{\pgfqpoint{1.034935in}{1.263960in}}%
\pgfpathlineto{\pgfqpoint{1.035661in}{1.052137in}}%
\pgfpathlineto{\pgfqpoint{1.036075in}{1.215526in}}%
\pgfpathlineto{\pgfqpoint{1.036697in}{1.072666in}}%
\pgfpathlineto{\pgfqpoint{1.037008in}{1.123659in}}%
\pgfpathlineto{\pgfqpoint{1.037112in}{1.033918in}}%
\pgfpathlineto{\pgfqpoint{1.037319in}{1.159516in}}%
\pgfpathlineto{\pgfqpoint{1.038044in}{0.924810in}}%
\pgfpathlineto{\pgfqpoint{1.038148in}{0.973716in}}%
\pgfpathlineto{\pgfqpoint{1.039806in}{1.200143in}}%
\pgfpathlineto{\pgfqpoint{1.038666in}{0.961454in}}%
\pgfpathlineto{\pgfqpoint{1.040014in}{1.146831in}}%
\pgfpathlineto{\pgfqpoint{1.040532in}{1.000406in}}%
\pgfpathlineto{\pgfqpoint{1.040739in}{1.161005in}}%
\pgfpathlineto{\pgfqpoint{1.041050in}{1.100582in}}%
\pgfpathlineto{\pgfqpoint{1.041672in}{1.241134in}}%
\pgfpathlineto{\pgfqpoint{1.041775in}{1.081701in}}%
\pgfpathlineto{\pgfqpoint{1.042086in}{1.138425in}}%
\pgfpathlineto{\pgfqpoint{1.042397in}{1.004876in}}%
\pgfpathlineto{\pgfqpoint{1.043123in}{1.009269in}}%
\pgfpathlineto{\pgfqpoint{1.043848in}{1.245543in}}%
\pgfpathlineto{\pgfqpoint{1.044263in}{1.168160in}}%
\pgfpathlineto{\pgfqpoint{1.044677in}{1.298012in}}%
\pgfpathlineto{\pgfqpoint{1.044988in}{1.116932in}}%
\pgfpathlineto{\pgfqpoint{1.045092in}{0.976157in}}%
\pgfpathlineto{\pgfqpoint{1.045714in}{1.193364in}}%
\pgfpathlineto{\pgfqpoint{1.046025in}{1.189741in}}%
\pgfpathlineto{\pgfqpoint{1.046957in}{1.006669in}}%
\pgfpathlineto{\pgfqpoint{1.047372in}{1.083128in}}%
\pgfpathlineto{\pgfqpoint{1.047994in}{0.990785in}}%
\pgfpathlineto{\pgfqpoint{1.048512in}{1.210644in}}%
\pgfpathlineto{\pgfqpoint{1.049134in}{1.033687in}}%
\pgfpathlineto{\pgfqpoint{1.049341in}{1.242533in}}%
\pgfpathlineto{\pgfqpoint{1.049652in}{1.100885in}}%
\pgfpathlineto{\pgfqpoint{1.049963in}{1.214771in}}%
\pgfpathlineto{\pgfqpoint{1.050067in}{1.000063in}}%
\pgfpathlineto{\pgfqpoint{1.050585in}{0.809075in}}%
\pgfpathlineto{\pgfqpoint{1.050378in}{1.088117in}}%
\pgfpathlineto{\pgfqpoint{1.051207in}{0.953797in}}%
\pgfpathlineto{\pgfqpoint{1.051518in}{0.796038in}}%
\pgfpathlineto{\pgfqpoint{1.051414in}{0.995000in}}%
\pgfpathlineto{\pgfqpoint{1.052243in}{0.893953in}}%
\pgfpathlineto{\pgfqpoint{1.053487in}{1.100352in}}%
\pgfpathlineto{\pgfqpoint{1.052865in}{0.775949in}}%
\pgfpathlineto{\pgfqpoint{1.053590in}{1.068153in}}%
\pgfpathlineto{\pgfqpoint{1.053798in}{1.043183in}}%
\pgfpathlineto{\pgfqpoint{1.054419in}{0.895922in}}%
\pgfpathlineto{\pgfqpoint{1.054730in}{1.101463in}}%
\pgfpathlineto{\pgfqpoint{1.054938in}{0.915796in}}%
\pgfpathlineto{\pgfqpoint{1.055663in}{1.146391in}}%
\pgfpathlineto{\pgfqpoint{1.056181in}{1.108450in}}%
\pgfpathlineto{\pgfqpoint{1.056285in}{1.017867in}}%
\pgfpathlineto{\pgfqpoint{1.056700in}{1.157510in}}%
\pgfpathlineto{\pgfqpoint{1.057218in}{1.134640in}}%
\pgfpathlineto{\pgfqpoint{1.058461in}{0.835198in}}%
\pgfpathlineto{\pgfqpoint{1.058772in}{0.876412in}}%
\pgfpathlineto{\pgfqpoint{1.059912in}{1.131906in}}%
\pgfpathlineto{\pgfqpoint{1.060223in}{1.067222in}}%
\pgfpathlineto{\pgfqpoint{1.060534in}{1.106035in}}%
\pgfpathlineto{\pgfqpoint{1.061363in}{0.862919in}}%
\pgfpathlineto{\pgfqpoint{1.061467in}{0.964838in}}%
\pgfpathlineto{\pgfqpoint{1.061674in}{0.812589in}}%
\pgfpathlineto{\pgfqpoint{1.062503in}{0.950356in}}%
\pgfpathlineto{\pgfqpoint{1.062814in}{0.899963in}}%
\pgfpathlineto{\pgfqpoint{1.062918in}{1.013579in}}%
\pgfpathlineto{\pgfqpoint{1.063333in}{0.969255in}}%
\pgfpathlineto{\pgfqpoint{1.064058in}{1.130671in}}%
\pgfpathlineto{\pgfqpoint{1.063643in}{0.923298in}}%
\pgfpathlineto{\pgfqpoint{1.064473in}{1.060651in}}%
\pgfpathlineto{\pgfqpoint{1.064680in}{1.075044in}}%
\pgfpathlineto{\pgfqpoint{1.064783in}{1.010560in}}%
\pgfpathlineto{\pgfqpoint{1.065094in}{0.862147in}}%
\pgfpathlineto{\pgfqpoint{1.065613in}{1.137044in}}%
\pgfpathlineto{\pgfqpoint{1.065820in}{1.050171in}}%
\pgfpathlineto{\pgfqpoint{1.065924in}{1.043514in}}%
\pgfpathlineto{\pgfqpoint{1.066649in}{0.821784in}}%
\pgfpathlineto{\pgfqpoint{1.067064in}{0.948918in}}%
\pgfpathlineto{\pgfqpoint{1.067893in}{1.153094in}}%
\pgfpathlineto{\pgfqpoint{1.068307in}{1.032474in}}%
\pgfpathlineto{\pgfqpoint{1.069033in}{0.858922in}}%
\pgfpathlineto{\pgfqpoint{1.069136in}{1.075921in}}%
\pgfpathlineto{\pgfqpoint{1.069344in}{0.973165in}}%
\pgfpathlineto{\pgfqpoint{1.069551in}{1.017246in}}%
\pgfpathlineto{\pgfqpoint{1.070069in}{0.928404in}}%
\pgfpathlineto{\pgfqpoint{1.070173in}{0.860909in}}%
\pgfpathlineto{\pgfqpoint{1.070898in}{1.058830in}}%
\pgfpathlineto{\pgfqpoint{1.071209in}{0.891101in}}%
\pgfpathlineto{\pgfqpoint{1.072453in}{1.106810in}}%
\pgfpathlineto{\pgfqpoint{1.073489in}{0.788994in}}%
\pgfpathlineto{\pgfqpoint{1.073697in}{0.878096in}}%
\pgfpathlineto{\pgfqpoint{1.073904in}{1.041696in}}%
\pgfpathlineto{\pgfqpoint{1.074629in}{0.818428in}}%
\pgfpathlineto{\pgfqpoint{1.074733in}{0.734994in}}%
\pgfpathlineto{\pgfqpoint{1.075251in}{0.987179in}}%
\pgfpathlineto{\pgfqpoint{1.075458in}{0.914538in}}%
\pgfpathlineto{\pgfqpoint{1.076288in}{0.831234in}}%
\pgfpathlineto{\pgfqpoint{1.076598in}{1.025764in}}%
\pgfpathlineto{\pgfqpoint{1.077531in}{0.824584in}}%
\pgfpathlineto{\pgfqpoint{1.077117in}{1.071626in}}%
\pgfpathlineto{\pgfqpoint{1.077635in}{0.937088in}}%
\pgfpathlineto{\pgfqpoint{1.078982in}{1.175182in}}%
\pgfpathlineto{\pgfqpoint{1.080330in}{0.946631in}}%
\pgfpathlineto{\pgfqpoint{1.080537in}{0.988400in}}%
\pgfpathlineto{\pgfqpoint{1.081366in}{1.151137in}}%
\pgfpathlineto{\pgfqpoint{1.081159in}{0.974195in}}%
\pgfpathlineto{\pgfqpoint{1.081573in}{1.122308in}}%
\pgfpathlineto{\pgfqpoint{1.082091in}{0.816975in}}%
\pgfpathlineto{\pgfqpoint{1.082713in}{1.041781in}}%
\pgfpathlineto{\pgfqpoint{1.083439in}{0.903775in}}%
\pgfpathlineto{\pgfqpoint{1.083957in}{0.916521in}}%
\pgfpathlineto{\pgfqpoint{1.084475in}{1.171407in}}%
\pgfpathlineto{\pgfqpoint{1.085201in}{1.110719in}}%
\pgfpathlineto{\pgfqpoint{1.086237in}{0.970147in}}%
\pgfpathlineto{\pgfqpoint{1.086133in}{1.156270in}}%
\pgfpathlineto{\pgfqpoint{1.086341in}{0.977679in}}%
\pgfpathlineto{\pgfqpoint{1.087273in}{1.168213in}}%
\pgfpathlineto{\pgfqpoint{1.087584in}{1.052916in}}%
\pgfpathlineto{\pgfqpoint{1.088103in}{0.986162in}}%
\pgfpathlineto{\pgfqpoint{1.088206in}{1.138414in}}%
\pgfpathlineto{\pgfqpoint{1.088621in}{1.043088in}}%
\pgfpathlineto{\pgfqpoint{1.088932in}{1.136385in}}%
\pgfpathlineto{\pgfqpoint{1.088828in}{1.001353in}}%
\pgfpathlineto{\pgfqpoint{1.089657in}{1.016611in}}%
\pgfpathlineto{\pgfqpoint{1.089761in}{0.950364in}}%
\pgfpathlineto{\pgfqpoint{1.089864in}{1.121594in}}%
\pgfpathlineto{\pgfqpoint{1.090590in}{1.058837in}}%
\pgfpathlineto{\pgfqpoint{1.090901in}{1.187935in}}%
\pgfpathlineto{\pgfqpoint{1.091523in}{0.935610in}}%
\pgfpathlineto{\pgfqpoint{1.091626in}{1.005686in}}%
\pgfpathlineto{\pgfqpoint{1.092559in}{1.262556in}}%
\pgfpathlineto{\pgfqpoint{1.093077in}{1.172413in}}%
\pgfpathlineto{\pgfqpoint{1.094010in}{0.921751in}}%
\pgfpathlineto{\pgfqpoint{1.094321in}{1.005337in}}%
\pgfpathlineto{\pgfqpoint{1.094839in}{1.179577in}}%
\pgfpathlineto{\pgfqpoint{1.095357in}{1.010917in}}%
\pgfpathlineto{\pgfqpoint{1.095668in}{1.126171in}}%
\pgfpathlineto{\pgfqpoint{1.095772in}{1.051498in}}%
\pgfpathlineto{\pgfqpoint{1.096290in}{0.734354in}}%
\pgfpathlineto{\pgfqpoint{1.096808in}{1.051623in}}%
\pgfpathlineto{\pgfqpoint{1.097223in}{0.906096in}}%
\pgfpathlineto{\pgfqpoint{1.097948in}{1.198810in}}%
\pgfpathlineto{\pgfqpoint{1.098363in}{1.001028in}}%
\pgfpathlineto{\pgfqpoint{1.099088in}{1.006387in}}%
\pgfpathlineto{\pgfqpoint{1.099192in}{1.143103in}}%
\pgfpathlineto{\pgfqpoint{1.099607in}{0.963841in}}%
\pgfpathlineto{\pgfqpoint{1.100228in}{1.119094in}}%
\pgfpathlineto{\pgfqpoint{1.100747in}{1.158973in}}%
\pgfpathlineto{\pgfqpoint{1.100643in}{1.040813in}}%
\pgfpathlineto{\pgfqpoint{1.100850in}{1.062810in}}%
\pgfpathlineto{\pgfqpoint{1.101368in}{0.914148in}}%
\pgfpathlineto{\pgfqpoint{1.101058in}{1.084874in}}%
\pgfpathlineto{\pgfqpoint{1.101990in}{1.043020in}}%
\pgfpathlineto{\pgfqpoint{1.102094in}{1.030417in}}%
\pgfpathlineto{\pgfqpoint{1.102198in}{1.071777in}}%
\pgfpathlineto{\pgfqpoint{1.102716in}{0.930809in}}%
\pgfpathlineto{\pgfqpoint{1.103234in}{1.153745in}}%
\pgfpathlineto{\pgfqpoint{1.103649in}{1.051629in}}%
\pgfpathlineto{\pgfqpoint{1.104167in}{1.097787in}}%
\pgfpathlineto{\pgfqpoint{1.104270in}{1.258664in}}%
\pgfpathlineto{\pgfqpoint{1.105099in}{1.158908in}}%
\pgfpathlineto{\pgfqpoint{1.105203in}{0.938127in}}%
\pgfpathlineto{\pgfqpoint{1.105514in}{1.174583in}}%
\pgfpathlineto{\pgfqpoint{1.106136in}{1.095979in}}%
\pgfpathlineto{\pgfqpoint{1.107172in}{1.224512in}}%
\pgfpathlineto{\pgfqpoint{1.106550in}{0.993855in}}%
\pgfpathlineto{\pgfqpoint{1.107276in}{1.144204in}}%
\pgfpathlineto{\pgfqpoint{1.107380in}{1.146643in}}%
\pgfpathlineto{\pgfqpoint{1.108416in}{0.980590in}}%
\pgfpathlineto{\pgfqpoint{1.107794in}{1.155808in}}%
\pgfpathlineto{\pgfqpoint{1.108520in}{1.120951in}}%
\pgfpathlineto{\pgfqpoint{1.109452in}{1.000409in}}%
\pgfpathlineto{\pgfqpoint{1.109141in}{1.126700in}}%
\pgfpathlineto{\pgfqpoint{1.109556in}{1.023918in}}%
\pgfpathlineto{\pgfqpoint{1.109763in}{1.140788in}}%
\pgfpathlineto{\pgfqpoint{1.110281in}{0.929789in}}%
\pgfpathlineto{\pgfqpoint{1.110592in}{1.138953in}}%
\pgfpathlineto{\pgfqpoint{1.110800in}{0.835606in}}%
\pgfpathlineto{\pgfqpoint{1.111836in}{0.955872in}}%
\pgfpathlineto{\pgfqpoint{1.112251in}{1.004959in}}%
\pgfpathlineto{\pgfqpoint{1.112354in}{0.993272in}}%
\pgfpathlineto{\pgfqpoint{1.112769in}{0.790004in}}%
\pgfpathlineto{\pgfqpoint{1.112562in}{0.996285in}}%
\pgfpathlineto{\pgfqpoint{1.113598in}{0.829730in}}%
\pgfpathlineto{\pgfqpoint{1.114634in}{1.040012in}}%
\pgfpathlineto{\pgfqpoint{1.114842in}{1.006769in}}%
\pgfpathlineto{\pgfqpoint{1.115256in}{0.914949in}}%
\pgfpathlineto{\pgfqpoint{1.115774in}{1.034416in}}%
\pgfpathlineto{\pgfqpoint{1.115878in}{1.071853in}}%
\pgfpathlineto{\pgfqpoint{1.116189in}{0.941760in}}%
\pgfpathlineto{\pgfqpoint{1.116396in}{1.048010in}}%
\pgfpathlineto{\pgfqpoint{1.116500in}{0.894007in}}%
\pgfpathlineto{\pgfqpoint{1.117433in}{0.984242in}}%
\pgfpathlineto{\pgfqpoint{1.118158in}{0.791072in}}%
\pgfpathlineto{\pgfqpoint{1.118676in}{1.205373in}}%
\pgfpathlineto{\pgfqpoint{1.119713in}{0.879132in}}%
\pgfpathlineto{\pgfqpoint{1.120127in}{0.954328in}}%
\pgfpathlineto{\pgfqpoint{1.120231in}{1.187513in}}%
\pgfpathlineto{\pgfqpoint{1.120853in}{0.908134in}}%
\pgfpathlineto{\pgfqpoint{1.121267in}{1.087350in}}%
\pgfpathlineto{\pgfqpoint{1.121371in}{1.145115in}}%
\pgfpathlineto{\pgfqpoint{1.121889in}{0.998542in}}%
\pgfpathlineto{\pgfqpoint{1.122200in}{1.125415in}}%
\pgfpathlineto{\pgfqpoint{1.123029in}{0.866156in}}%
\pgfpathlineto{\pgfqpoint{1.123547in}{0.880715in}}%
\pgfpathlineto{\pgfqpoint{1.125724in}{1.214124in}}%
\pgfpathlineto{\pgfqpoint{1.125827in}{1.181686in}}%
\pgfpathlineto{\pgfqpoint{1.126553in}{0.939925in}}%
\pgfpathlineto{\pgfqpoint{1.126968in}{1.116201in}}%
\pgfpathlineto{\pgfqpoint{1.127278in}{1.193878in}}%
\pgfpathlineto{\pgfqpoint{1.127797in}{1.043320in}}%
\pgfpathlineto{\pgfqpoint{1.128833in}{1.208121in}}%
\pgfpathlineto{\pgfqpoint{1.128211in}{1.018691in}}%
\pgfpathlineto{\pgfqpoint{1.128937in}{1.184044in}}%
\pgfpathlineto{\pgfqpoint{1.130180in}{1.002881in}}%
\pgfpathlineto{\pgfqpoint{1.130595in}{1.230156in}}%
\pgfpathlineto{\pgfqpoint{1.131113in}{0.902379in}}%
\pgfpathlineto{\pgfqpoint{1.131320in}{1.048624in}}%
\pgfpathlineto{\pgfqpoint{1.131631in}{0.900211in}}%
\pgfpathlineto{\pgfqpoint{1.132564in}{1.243229in}}%
\pgfpathlineto{\pgfqpoint{1.132668in}{1.012678in}}%
\pgfpathlineto{\pgfqpoint{1.133393in}{1.312966in}}%
\pgfpathlineto{\pgfqpoint{1.133704in}{1.116956in}}%
\pgfpathlineto{\pgfqpoint{1.133911in}{0.984308in}}%
\pgfpathlineto{\pgfqpoint{1.134430in}{1.120492in}}%
\pgfpathlineto{\pgfqpoint{1.134637in}{1.059847in}}%
\pgfpathlineto{\pgfqpoint{1.134741in}{1.178478in}}%
\pgfpathlineto{\pgfqpoint{1.135362in}{0.946714in}}%
\pgfpathlineto{\pgfqpoint{1.135777in}{1.111572in}}%
\pgfpathlineto{\pgfqpoint{1.136088in}{0.966963in}}%
\pgfpathlineto{\pgfqpoint{1.136191in}{1.181918in}}%
\pgfpathlineto{\pgfqpoint{1.136813in}{0.994582in}}%
\pgfpathlineto{\pgfqpoint{1.137435in}{1.355347in}}%
\pgfpathlineto{\pgfqpoint{1.137953in}{1.111443in}}%
\pgfpathlineto{\pgfqpoint{1.138368in}{1.230835in}}%
\pgfpathlineto{\pgfqpoint{1.138472in}{1.045508in}}%
\pgfpathlineto{\pgfqpoint{1.139197in}{1.153117in}}%
\pgfpathlineto{\pgfqpoint{1.140233in}{1.022364in}}%
\pgfpathlineto{\pgfqpoint{1.140026in}{1.164074in}}%
\pgfpathlineto{\pgfqpoint{1.140337in}{1.119644in}}%
\pgfpathlineto{\pgfqpoint{1.141270in}{1.015281in}}%
\pgfpathlineto{\pgfqpoint{1.140648in}{1.210727in}}%
\pgfpathlineto{\pgfqpoint{1.141477in}{1.091067in}}%
\pgfpathlineto{\pgfqpoint{1.141581in}{1.094328in}}%
\pgfpathlineto{\pgfqpoint{1.142514in}{1.292936in}}%
\pgfpathlineto{\pgfqpoint{1.142306in}{1.058909in}}%
\pgfpathlineto{\pgfqpoint{1.142824in}{1.195956in}}%
\pgfpathlineto{\pgfqpoint{1.143239in}{0.979895in}}%
\pgfpathlineto{\pgfqpoint{1.143964in}{1.070155in}}%
\pgfpathlineto{\pgfqpoint{1.144483in}{1.155899in}}%
\pgfpathlineto{\pgfqpoint{1.144897in}{1.032138in}}%
\pgfpathlineto{\pgfqpoint{1.145001in}{1.076371in}}%
\pgfpathlineto{\pgfqpoint{1.145415in}{1.049523in}}%
\pgfpathlineto{\pgfqpoint{1.145208in}{1.246114in}}%
\pgfpathlineto{\pgfqpoint{1.145623in}{1.128345in}}%
\pgfpathlineto{\pgfqpoint{1.145726in}{1.229634in}}%
\pgfpathlineto{\pgfqpoint{1.146141in}{1.020813in}}%
\pgfpathlineto{\pgfqpoint{1.146555in}{1.114958in}}%
\pgfpathlineto{\pgfqpoint{1.147592in}{1.029648in}}%
\pgfpathlineto{\pgfqpoint{1.147488in}{1.161756in}}%
\pgfpathlineto{\pgfqpoint{1.147696in}{1.079438in}}%
\pgfpathlineto{\pgfqpoint{1.148628in}{1.165660in}}%
\pgfpathlineto{\pgfqpoint{1.148214in}{1.037894in}}%
\pgfpathlineto{\pgfqpoint{1.148836in}{1.121298in}}%
\pgfpathlineto{\pgfqpoint{1.148939in}{1.105606in}}%
\pgfpathlineto{\pgfqpoint{1.149043in}{1.162026in}}%
\pgfpathlineto{\pgfqpoint{1.149146in}{1.158776in}}%
\pgfpathlineto{\pgfqpoint{1.149976in}{1.325981in}}%
\pgfpathlineto{\pgfqpoint{1.149665in}{1.085052in}}%
\pgfpathlineto{\pgfqpoint{1.150183in}{1.115026in}}%
\pgfpathlineto{\pgfqpoint{1.150805in}{1.197936in}}%
\pgfpathlineto{\pgfqpoint{1.150390in}{1.010973in}}%
\pgfpathlineto{\pgfqpoint{1.151116in}{1.073575in}}%
\pgfpathlineto{\pgfqpoint{1.152048in}{0.942181in}}%
\pgfpathlineto{\pgfqpoint{1.151841in}{1.184269in}}%
\pgfpathlineto{\pgfqpoint{1.152256in}{1.028396in}}%
\pgfpathlineto{\pgfqpoint{1.152774in}{1.131850in}}%
\pgfpathlineto{\pgfqpoint{1.152878in}{0.962541in}}%
\pgfpathlineto{\pgfqpoint{1.153085in}{1.038602in}}%
\pgfpathlineto{\pgfqpoint{1.153810in}{0.927711in}}%
\pgfpathlineto{\pgfqpoint{1.153292in}{1.133585in}}%
\pgfpathlineto{\pgfqpoint{1.154225in}{0.977050in}}%
\pgfpathlineto{\pgfqpoint{1.155054in}{1.231055in}}%
\pgfpathlineto{\pgfqpoint{1.155469in}{1.106961in}}%
\pgfpathlineto{\pgfqpoint{1.156401in}{0.993273in}}%
\pgfpathlineto{\pgfqpoint{1.156505in}{1.161871in}}%
\pgfpathlineto{\pgfqpoint{1.156816in}{1.028617in}}%
\pgfpathlineto{\pgfqpoint{1.156919in}{1.262504in}}%
\pgfpathlineto{\pgfqpoint{1.157438in}{1.194222in}}%
\pgfpathlineto{\pgfqpoint{1.158060in}{1.288532in}}%
\pgfpathlineto{\pgfqpoint{1.158163in}{1.116790in}}%
\pgfpathlineto{\pgfqpoint{1.158474in}{1.237774in}}%
\pgfpathlineto{\pgfqpoint{1.158992in}{1.257177in}}%
\pgfpathlineto{\pgfqpoint{1.160132in}{0.969874in}}%
\pgfpathlineto{\pgfqpoint{1.161065in}{1.181267in}}%
\pgfpathlineto{\pgfqpoint{1.160443in}{0.905646in}}%
\pgfpathlineto{\pgfqpoint{1.161376in}{1.098632in}}%
\pgfpathlineto{\pgfqpoint{1.161480in}{1.016306in}}%
\pgfpathlineto{\pgfqpoint{1.161791in}{1.319626in}}%
\pgfpathlineto{\pgfqpoint{1.162516in}{1.020371in}}%
\pgfpathlineto{\pgfqpoint{1.163760in}{1.282190in}}%
\pgfpathlineto{\pgfqpoint{1.163863in}{1.249560in}}%
\pgfpathlineto{\pgfqpoint{1.164071in}{1.084950in}}%
\pgfpathlineto{\pgfqpoint{1.164900in}{1.156544in}}%
\pgfpathlineto{\pgfqpoint{1.165314in}{1.257104in}}%
\pgfpathlineto{\pgfqpoint{1.165729in}{1.071473in}}%
\pgfpathlineto{\pgfqpoint{1.165936in}{1.197667in}}%
\pgfpathlineto{\pgfqpoint{1.166454in}{1.009003in}}%
\pgfpathlineto{\pgfqpoint{1.167076in}{1.144896in}}%
\pgfpathlineto{\pgfqpoint{1.167802in}{1.011153in}}%
\pgfpathlineto{\pgfqpoint{1.167698in}{1.147853in}}%
\pgfpathlineto{\pgfqpoint{1.168113in}{1.081904in}}%
\pgfpathlineto{\pgfqpoint{1.168216in}{1.163700in}}%
\pgfpathlineto{\pgfqpoint{1.168942in}{1.009903in}}%
\pgfpathlineto{\pgfqpoint{1.169045in}{1.075022in}}%
\pgfpathlineto{\pgfqpoint{1.169771in}{0.854693in}}%
\pgfpathlineto{\pgfqpoint{1.169460in}{1.129566in}}%
\pgfpathlineto{\pgfqpoint{1.170185in}{1.039056in}}%
\pgfpathlineto{\pgfqpoint{1.170289in}{1.115331in}}%
\pgfpathlineto{\pgfqpoint{1.170807in}{0.876762in}}%
\pgfpathlineto{\pgfqpoint{1.171222in}{0.999256in}}%
\pgfpathlineto{\pgfqpoint{1.171533in}{0.879540in}}%
\pgfpathlineto{\pgfqpoint{1.171947in}{1.093861in}}%
\pgfpathlineto{\pgfqpoint{1.172362in}{0.938312in}}%
\pgfpathlineto{\pgfqpoint{1.172673in}{1.191325in}}%
\pgfpathlineto{\pgfqpoint{1.173295in}{0.921427in}}%
\pgfpathlineto{\pgfqpoint{1.173709in}{1.142900in}}%
\pgfpathlineto{\pgfqpoint{1.174020in}{0.837911in}}%
\pgfpathlineto{\pgfqpoint{1.174953in}{0.888455in}}%
\pgfpathlineto{\pgfqpoint{1.175782in}{0.857811in}}%
\pgfpathlineto{\pgfqpoint{1.176093in}{1.045231in}}%
\pgfpathlineto{\pgfqpoint{1.177337in}{0.832992in}}%
\pgfpathlineto{\pgfqpoint{1.177855in}{1.031891in}}%
\pgfpathlineto{\pgfqpoint{1.178477in}{0.904682in}}%
\pgfpathlineto{\pgfqpoint{1.178580in}{0.892596in}}%
\pgfpathlineto{\pgfqpoint{1.178684in}{0.960824in}}%
\pgfpathlineto{\pgfqpoint{1.178891in}{0.923112in}}%
\pgfpathlineto{\pgfqpoint{1.178995in}{1.014502in}}%
\pgfpathlineto{\pgfqpoint{1.179513in}{0.883264in}}%
\pgfpathlineto{\pgfqpoint{1.179824in}{0.937915in}}%
\pgfpathlineto{\pgfqpoint{1.179928in}{0.853824in}}%
\pgfpathlineto{\pgfqpoint{1.180549in}{1.059253in}}%
\pgfpathlineto{\pgfqpoint{1.180757in}{1.024149in}}%
\pgfpathlineto{\pgfqpoint{1.180860in}{1.092440in}}%
\pgfpathlineto{\pgfqpoint{1.181171in}{0.921207in}}%
\pgfpathlineto{\pgfqpoint{1.181689in}{0.976424in}}%
\pgfpathlineto{\pgfqpoint{1.181793in}{0.856958in}}%
\pgfpathlineto{\pgfqpoint{1.182415in}{1.077676in}}%
\pgfpathlineto{\pgfqpoint{1.182726in}{0.973983in}}%
\pgfpathlineto{\pgfqpoint{1.182829in}{0.978560in}}%
\pgfpathlineto{\pgfqpoint{1.183451in}{1.180260in}}%
\pgfpathlineto{\pgfqpoint{1.183866in}{0.914352in}}%
\pgfpathlineto{\pgfqpoint{1.183970in}{1.040931in}}%
\pgfpathlineto{\pgfqpoint{1.184073in}{1.040155in}}%
\pgfpathlineto{\pgfqpoint{1.184799in}{1.168423in}}%
\pgfpathlineto{\pgfqpoint{1.184488in}{0.983274in}}%
\pgfpathlineto{\pgfqpoint{1.185110in}{1.030055in}}%
\pgfpathlineto{\pgfqpoint{1.186250in}{0.821334in}}%
\pgfpathlineto{\pgfqpoint{1.185628in}{1.083106in}}%
\pgfpathlineto{\pgfqpoint{1.186457in}{0.925422in}}%
\pgfpathlineto{\pgfqpoint{1.186975in}{1.100972in}}%
\pgfpathlineto{\pgfqpoint{1.187390in}{0.952733in}}%
\pgfpathlineto{\pgfqpoint{1.187493in}{0.886477in}}%
\pgfpathlineto{\pgfqpoint{1.188219in}{1.000600in}}%
\pgfpathlineto{\pgfqpoint{1.188426in}{0.962275in}}%
\pgfpathlineto{\pgfqpoint{1.188530in}{0.962154in}}%
\pgfpathlineto{\pgfqpoint{1.189462in}{0.944036in}}%
\pgfpathlineto{\pgfqpoint{1.189877in}{1.210259in}}%
\pgfpathlineto{\pgfqpoint{1.190188in}{0.977269in}}%
\pgfpathlineto{\pgfqpoint{1.191224in}{0.992429in}}%
\pgfpathlineto{\pgfqpoint{1.192468in}{1.170397in}}%
\pgfpathlineto{\pgfqpoint{1.192572in}{1.152145in}}%
\pgfpathlineto{\pgfqpoint{1.193297in}{0.887481in}}%
\pgfpathlineto{\pgfqpoint{1.193712in}{0.953180in}}%
\pgfpathlineto{\pgfqpoint{1.194230in}{1.192474in}}%
\pgfpathlineto{\pgfqpoint{1.194541in}{0.881831in}}%
\pgfpathlineto{\pgfqpoint{1.194852in}{1.057946in}}%
\pgfpathlineto{\pgfqpoint{1.195059in}{1.003919in}}%
\pgfpathlineto{\pgfqpoint{1.195163in}{1.057577in}}%
\pgfpathlineto{\pgfqpoint{1.195266in}{1.128758in}}%
\pgfpathlineto{\pgfqpoint{1.195370in}{0.970164in}}%
\pgfpathlineto{\pgfqpoint{1.196303in}{1.107101in}}%
\pgfpathlineto{\pgfqpoint{1.196406in}{1.100858in}}%
\pgfpathlineto{\pgfqpoint{1.197443in}{0.936826in}}%
\pgfpathlineto{\pgfqpoint{1.197132in}{1.211929in}}%
\pgfpathlineto{\pgfqpoint{1.197546in}{1.009792in}}%
\pgfpathlineto{\pgfqpoint{1.197650in}{1.011663in}}%
\pgfpathlineto{\pgfqpoint{1.198686in}{1.129378in}}%
\pgfpathlineto{\pgfqpoint{1.198065in}{0.918445in}}%
\pgfpathlineto{\pgfqpoint{1.198894in}{1.079916in}}%
\pgfpathlineto{\pgfqpoint{1.198997in}{1.018191in}}%
\pgfpathlineto{\pgfqpoint{1.199619in}{1.222330in}}%
\pgfpathlineto{\pgfqpoint{1.199930in}{1.093730in}}%
\pgfpathlineto{\pgfqpoint{1.200966in}{1.180627in}}%
\pgfpathlineto{\pgfqpoint{1.200241in}{0.888153in}}%
\pgfpathlineto{\pgfqpoint{1.201070in}{1.121298in}}%
\pgfpathlineto{\pgfqpoint{1.201277in}{1.098871in}}%
\pgfpathlineto{\pgfqpoint{1.201899in}{1.192690in}}%
\pgfpathlineto{\pgfqpoint{1.201588in}{1.036359in}}%
\pgfpathlineto{\pgfqpoint{1.202314in}{1.068814in}}%
\pgfpathlineto{\pgfqpoint{1.202625in}{0.924571in}}%
\pgfpathlineto{\pgfqpoint{1.203143in}{1.227590in}}%
\pgfpathlineto{\pgfqpoint{1.203350in}{1.073692in}}%
\pgfpathlineto{\pgfqpoint{1.203454in}{1.279386in}}%
\pgfpathlineto{\pgfqpoint{1.204387in}{1.011645in}}%
\pgfpathlineto{\pgfqpoint{1.204801in}{1.090543in}}%
\pgfpathlineto{\pgfqpoint{1.204698in}{0.945111in}}%
\pgfpathlineto{\pgfqpoint{1.205216in}{1.019558in}}%
\pgfpathlineto{\pgfqpoint{1.205319in}{0.906034in}}%
\pgfpathlineto{\pgfqpoint{1.205630in}{1.091074in}}%
\pgfpathlineto{\pgfqpoint{1.206252in}{1.043847in}}%
\pgfpathlineto{\pgfqpoint{1.206770in}{0.870930in}}%
\pgfpathlineto{\pgfqpoint{1.207081in}{0.965062in}}%
\pgfpathlineto{\pgfqpoint{1.207185in}{1.123455in}}%
\pgfpathlineto{\pgfqpoint{1.208118in}{1.092621in}}%
\pgfpathlineto{\pgfqpoint{1.208947in}{0.823841in}}%
\pgfpathlineto{\pgfqpoint{1.208429in}{1.120679in}}%
\pgfpathlineto{\pgfqpoint{1.209258in}{0.952223in}}%
\pgfpathlineto{\pgfqpoint{1.209465in}{0.969838in}}%
\pgfpathlineto{\pgfqpoint{1.210087in}{1.106012in}}%
\pgfpathlineto{\pgfqpoint{1.209776in}{0.965495in}}%
\pgfpathlineto{\pgfqpoint{1.210605in}{1.002229in}}%
\pgfpathlineto{\pgfqpoint{1.210812in}{0.839856in}}%
\pgfpathlineto{\pgfqpoint{1.211641in}{1.013445in}}%
\pgfpathlineto{\pgfqpoint{1.211745in}{0.888542in}}%
\pgfpathlineto{\pgfqpoint{1.212574in}{1.005007in}}%
\pgfpathlineto{\pgfqpoint{1.211952in}{0.867299in}}%
\pgfpathlineto{\pgfqpoint{1.212678in}{0.920816in}}%
\pgfpathlineto{\pgfqpoint{1.212781in}{0.855202in}}%
\pgfpathlineto{\pgfqpoint{1.213507in}{1.043149in}}%
\pgfpathlineto{\pgfqpoint{1.213611in}{0.982367in}}%
\pgfpathlineto{\pgfqpoint{1.213714in}{1.028474in}}%
\pgfpathlineto{\pgfqpoint{1.214025in}{0.895205in}}%
\pgfpathlineto{\pgfqpoint{1.214543in}{0.906872in}}%
\pgfpathlineto{\pgfqpoint{1.214647in}{0.898907in}}%
\pgfpathlineto{\pgfqpoint{1.214751in}{0.963860in}}%
\pgfpathlineto{\pgfqpoint{1.215269in}{1.151692in}}%
\pgfpathlineto{\pgfqpoint{1.215683in}{0.783578in}}%
\pgfpathlineto{\pgfqpoint{1.215787in}{0.904434in}}%
\pgfpathlineto{\pgfqpoint{1.216202in}{1.155412in}}%
\pgfpathlineto{\pgfqpoint{1.217134in}{1.094160in}}%
\pgfpathlineto{\pgfqpoint{1.218067in}{0.903660in}}%
\pgfpathlineto{\pgfqpoint{1.217756in}{1.189647in}}%
\pgfpathlineto{\pgfqpoint{1.218274in}{0.971953in}}%
\pgfpathlineto{\pgfqpoint{1.219000in}{1.049905in}}%
\pgfpathlineto{\pgfqpoint{1.218896in}{0.927771in}}%
\pgfpathlineto{\pgfqpoint{1.219103in}{0.978826in}}%
\pgfpathlineto{\pgfqpoint{1.219207in}{0.880062in}}%
\pgfpathlineto{\pgfqpoint{1.219622in}{1.071936in}}%
\pgfpathlineto{\pgfqpoint{1.220140in}{1.019417in}}%
\pgfpathlineto{\pgfqpoint{1.221073in}{0.747979in}}%
\pgfpathlineto{\pgfqpoint{1.220451in}{1.032390in}}%
\pgfpathlineto{\pgfqpoint{1.221280in}{0.859421in}}%
\pgfpathlineto{\pgfqpoint{1.221384in}{1.090963in}}%
\pgfpathlineto{\pgfqpoint{1.222316in}{0.758987in}}%
\pgfpathlineto{\pgfqpoint{1.222420in}{0.685871in}}%
\pgfpathlineto{\pgfqpoint{1.222938in}{0.962500in}}%
\pgfpathlineto{\pgfqpoint{1.223353in}{0.803636in}}%
\pgfpathlineto{\pgfqpoint{1.223871in}{0.778225in}}%
\pgfpathlineto{\pgfqpoint{1.223975in}{0.829138in}}%
\pgfpathlineto{\pgfqpoint{1.224285in}{0.920095in}}%
\pgfpathlineto{\pgfqpoint{1.224493in}{0.741683in}}%
\pgfpathlineto{\pgfqpoint{1.224804in}{0.843494in}}%
\pgfpathlineto{\pgfqpoint{1.225736in}{0.660787in}}%
\pgfpathlineto{\pgfqpoint{1.225426in}{0.898778in}}%
\pgfpathlineto{\pgfqpoint{1.225944in}{0.752924in}}%
\pgfpathlineto{\pgfqpoint{1.226876in}{0.907129in}}%
\pgfpathlineto{\pgfqpoint{1.226255in}{0.672284in}}%
\pgfpathlineto{\pgfqpoint{1.226980in}{0.867949in}}%
\pgfpathlineto{\pgfqpoint{1.227706in}{0.615829in}}%
\pgfpathlineto{\pgfqpoint{1.228120in}{0.809781in}}%
\pgfpathlineto{\pgfqpoint{1.228742in}{0.739182in}}%
\pgfpathlineto{\pgfqpoint{1.229260in}{0.910620in}}%
\pgfpathlineto{\pgfqpoint{1.229882in}{0.937010in}}%
\pgfpathlineto{\pgfqpoint{1.230400in}{0.698090in}}%
\pgfpathlineto{\pgfqpoint{1.231022in}{1.058702in}}%
\pgfpathlineto{\pgfqpoint{1.231644in}{0.958779in}}%
\pgfpathlineto{\pgfqpoint{1.232058in}{0.778140in}}%
\pgfpathlineto{\pgfqpoint{1.231851in}{0.984637in}}%
\pgfpathlineto{\pgfqpoint{1.232784in}{0.925876in}}%
\pgfpathlineto{\pgfqpoint{1.233924in}{1.091033in}}%
\pgfpathlineto{\pgfqpoint{1.233509in}{0.901803in}}%
\pgfpathlineto{\pgfqpoint{1.234028in}{1.060522in}}%
\pgfpathlineto{\pgfqpoint{1.234546in}{0.885512in}}%
\pgfpathlineto{\pgfqpoint{1.234857in}{1.068029in}}%
\pgfpathlineto{\pgfqpoint{1.235064in}{1.042826in}}%
\pgfpathlineto{\pgfqpoint{1.235686in}{1.199218in}}%
\pgfpathlineto{\pgfqpoint{1.235271in}{0.962997in}}%
\pgfpathlineto{\pgfqpoint{1.236100in}{1.076351in}}%
\pgfpathlineto{\pgfqpoint{1.236930in}{1.139774in}}%
\pgfpathlineto{\pgfqpoint{1.237551in}{0.902536in}}%
\pgfpathlineto{\pgfqpoint{1.237759in}{0.926396in}}%
\pgfpathlineto{\pgfqpoint{1.238070in}{0.862694in}}%
\pgfpathlineto{\pgfqpoint{1.239002in}{1.134875in}}%
\pgfpathlineto{\pgfqpoint{1.239624in}{1.236599in}}%
\pgfpathlineto{\pgfqpoint{1.240453in}{0.864169in}}%
\pgfpathlineto{\pgfqpoint{1.241697in}{1.198014in}}%
\pgfpathlineto{\pgfqpoint{1.240868in}{0.846702in}}%
\pgfpathlineto{\pgfqpoint{1.241904in}{1.076733in}}%
\pgfpathlineto{\pgfqpoint{1.242422in}{0.966632in}}%
\pgfpathlineto{\pgfqpoint{1.242215in}{1.142882in}}%
\pgfpathlineto{\pgfqpoint{1.242941in}{1.103598in}}%
\pgfpathlineto{\pgfqpoint{1.243873in}{1.314206in}}%
\pgfpathlineto{\pgfqpoint{1.244081in}{1.182522in}}%
\pgfpathlineto{\pgfqpoint{1.245013in}{1.249973in}}%
\pgfpathlineto{\pgfqpoint{1.245428in}{0.976304in}}%
\pgfpathlineto{\pgfqpoint{1.245843in}{1.222906in}}%
\pgfpathlineto{\pgfqpoint{1.246568in}{1.077864in}}%
\pgfpathlineto{\pgfqpoint{1.246672in}{1.133304in}}%
\pgfpathlineto{\pgfqpoint{1.247294in}{1.023828in}}%
\pgfpathlineto{\pgfqpoint{1.247501in}{1.083871in}}%
\pgfpathlineto{\pgfqpoint{1.247915in}{0.976052in}}%
\pgfpathlineto{\pgfqpoint{1.248330in}{1.220515in}}%
\pgfpathlineto{\pgfqpoint{1.248641in}{1.051653in}}%
\pgfpathlineto{\pgfqpoint{1.248745in}{1.141268in}}%
\pgfpathlineto{\pgfqpoint{1.249263in}{0.974253in}}%
\pgfpathlineto{\pgfqpoint{1.249677in}{1.027099in}}%
\pgfpathlineto{\pgfqpoint{1.249885in}{1.047761in}}%
\pgfpathlineto{\pgfqpoint{1.250714in}{0.988405in}}%
\pgfpathlineto{\pgfqpoint{1.250506in}{1.195862in}}%
\pgfpathlineto{\pgfqpoint{1.250921in}{1.018476in}}%
\pgfpathlineto{\pgfqpoint{1.251025in}{1.167348in}}%
\pgfpathlineto{\pgfqpoint{1.251336in}{0.983036in}}%
\pgfpathlineto{\pgfqpoint{1.252061in}{1.061173in}}%
\pgfpathlineto{\pgfqpoint{1.252890in}{1.170226in}}%
\pgfpathlineto{\pgfqpoint{1.252683in}{0.966233in}}%
\pgfpathlineto{\pgfqpoint{1.252994in}{1.048131in}}%
\pgfpathlineto{\pgfqpoint{1.253408in}{1.082218in}}%
\pgfpathlineto{\pgfqpoint{1.254237in}{0.891587in}}%
\pgfpathlineto{\pgfqpoint{1.254548in}{1.081967in}}%
\pgfpathlineto{\pgfqpoint{1.255377in}{0.995682in}}%
\pgfpathlineto{\pgfqpoint{1.255688in}{0.911244in}}%
\pgfpathlineto{\pgfqpoint{1.255792in}{1.067793in}}%
\pgfpathlineto{\pgfqpoint{1.255999in}{1.118025in}}%
\pgfpathlineto{\pgfqpoint{1.256103in}{1.054978in}}%
\pgfpathlineto{\pgfqpoint{1.256207in}{0.934316in}}%
\pgfpathlineto{\pgfqpoint{1.257036in}{1.184606in}}%
\pgfpathlineto{\pgfqpoint{1.257139in}{1.058826in}}%
\pgfpathlineto{\pgfqpoint{1.257347in}{1.120795in}}%
\pgfpathlineto{\pgfqpoint{1.258176in}{0.885519in}}%
\pgfpathlineto{\pgfqpoint{1.258383in}{1.159039in}}%
\pgfpathlineto{\pgfqpoint{1.259316in}{1.028856in}}%
\pgfpathlineto{\pgfqpoint{1.259627in}{0.865413in}}%
\pgfpathlineto{\pgfqpoint{1.260145in}{1.079385in}}%
\pgfpathlineto{\pgfqpoint{1.260352in}{1.018251in}}%
\pgfpathlineto{\pgfqpoint{1.260456in}{1.147210in}}%
\pgfpathlineto{\pgfqpoint{1.260767in}{0.961376in}}%
\pgfpathlineto{\pgfqpoint{1.261389in}{1.014516in}}%
\pgfpathlineto{\pgfqpoint{1.261492in}{0.851773in}}%
\pgfpathlineto{\pgfqpoint{1.262425in}{1.047916in}}%
\pgfpathlineto{\pgfqpoint{1.262943in}{1.087668in}}%
\pgfpathlineto{\pgfqpoint{1.263047in}{0.942602in}}%
\pgfpathlineto{\pgfqpoint{1.263254in}{0.997533in}}%
\pgfpathlineto{\pgfqpoint{1.263358in}{0.988176in}}%
\pgfpathlineto{\pgfqpoint{1.263461in}{1.185173in}}%
\pgfpathlineto{\pgfqpoint{1.263772in}{0.911652in}}%
\pgfpathlineto{\pgfqpoint{1.264394in}{1.085180in}}%
\pgfpathlineto{\pgfqpoint{1.264705in}{1.005247in}}%
\pgfpathlineto{\pgfqpoint{1.265120in}{1.131145in}}%
\pgfpathlineto{\pgfqpoint{1.265327in}{1.068655in}}%
\pgfpathlineto{\pgfqpoint{1.265431in}{1.121102in}}%
\pgfpathlineto{\pgfqpoint{1.266156in}{0.974418in}}%
\pgfpathlineto{\pgfqpoint{1.266363in}{1.033043in}}%
\pgfpathlineto{\pgfqpoint{1.266674in}{1.141506in}}%
\pgfpathlineto{\pgfqpoint{1.266882in}{1.022833in}}%
\pgfpathlineto{\pgfqpoint{1.266985in}{0.970742in}}%
\pgfpathlineto{\pgfqpoint{1.267503in}{1.158514in}}%
\pgfpathlineto{\pgfqpoint{1.267814in}{1.033196in}}%
\pgfpathlineto{\pgfqpoint{1.267918in}{1.110347in}}%
\pgfpathlineto{\pgfqpoint{1.268540in}{0.909133in}}%
\pgfpathlineto{\pgfqpoint{1.268747in}{1.093677in}}%
\pgfpathlineto{\pgfqpoint{1.268851in}{0.921924in}}%
\pgfpathlineto{\pgfqpoint{1.269576in}{1.108680in}}%
\pgfpathlineto{\pgfqpoint{1.269783in}{1.058647in}}%
\pgfpathlineto{\pgfqpoint{1.270094in}{1.253741in}}%
\pgfpathlineto{\pgfqpoint{1.270923in}{1.123266in}}%
\pgfpathlineto{\pgfqpoint{1.271027in}{1.027868in}}%
\pgfpathlineto{\pgfqpoint{1.271545in}{1.216498in}}%
\pgfpathlineto{\pgfqpoint{1.272064in}{1.084544in}}%
\pgfpathlineto{\pgfqpoint{1.272167in}{1.131147in}}%
\pgfpathlineto{\pgfqpoint{1.272582in}{1.024144in}}%
\pgfpathlineto{\pgfqpoint{1.273100in}{1.065062in}}%
\pgfpathlineto{\pgfqpoint{1.274033in}{0.925526in}}%
\pgfpathlineto{\pgfqpoint{1.273618in}{1.221265in}}%
\pgfpathlineto{\pgfqpoint{1.274136in}{0.979384in}}%
\pgfpathlineto{\pgfqpoint{1.275173in}{1.207578in}}%
\pgfpathlineto{\pgfqpoint{1.274551in}{0.861698in}}%
\pgfpathlineto{\pgfqpoint{1.275276in}{1.116464in}}%
\pgfpathlineto{\pgfqpoint{1.275380in}{0.945234in}}%
\pgfpathlineto{\pgfqpoint{1.276209in}{1.143834in}}%
\pgfpathlineto{\pgfqpoint{1.276313in}{1.126220in}}%
\pgfpathlineto{\pgfqpoint{1.276935in}{1.065362in}}%
\pgfpathlineto{\pgfqpoint{1.276727in}{1.251534in}}%
\pgfpathlineto{\pgfqpoint{1.277142in}{1.110903in}}%
\pgfpathlineto{\pgfqpoint{1.277246in}{1.183985in}}%
\pgfpathlineto{\pgfqpoint{1.277971in}{0.949266in}}%
\pgfpathlineto{\pgfqpoint{1.279318in}{1.123871in}}%
\pgfpathlineto{\pgfqpoint{1.278696in}{0.936648in}}%
\pgfpathlineto{\pgfqpoint{1.279837in}{1.084630in}}%
\pgfpathlineto{\pgfqpoint{1.280458in}{0.926885in}}%
\pgfpathlineto{\pgfqpoint{1.280977in}{0.945887in}}%
\pgfpathlineto{\pgfqpoint{1.281702in}{1.162318in}}%
\pgfpathlineto{\pgfqpoint{1.282013in}{0.868293in}}%
\pgfpathlineto{\pgfqpoint{1.282220in}{0.840500in}}%
\pgfpathlineto{\pgfqpoint{1.282324in}{0.919909in}}%
\pgfpathlineto{\pgfqpoint{1.282635in}{0.888677in}}%
\pgfpathlineto{\pgfqpoint{1.283360in}{0.993241in}}%
\pgfpathlineto{\pgfqpoint{1.283153in}{0.828638in}}%
\pgfpathlineto{\pgfqpoint{1.283671in}{0.888691in}}%
\pgfpathlineto{\pgfqpoint{1.284604in}{0.758277in}}%
\pgfpathlineto{\pgfqpoint{1.284086in}{0.959839in}}%
\pgfpathlineto{\pgfqpoint{1.284811in}{0.830801in}}%
\pgfpathlineto{\pgfqpoint{1.285537in}{0.904566in}}%
\pgfpathlineto{\pgfqpoint{1.285122in}{0.786255in}}%
\pgfpathlineto{\pgfqpoint{1.285951in}{0.844558in}}%
\pgfpathlineto{\pgfqpoint{1.286470in}{0.686805in}}%
\pgfpathlineto{\pgfqpoint{1.287091in}{0.914145in}}%
\pgfpathlineto{\pgfqpoint{1.287195in}{0.755529in}}%
\pgfpathlineto{\pgfqpoint{1.287610in}{1.030120in}}%
\pgfpathlineto{\pgfqpoint{1.288128in}{0.950372in}}%
\pgfpathlineto{\pgfqpoint{1.288231in}{0.875333in}}%
\pgfpathlineto{\pgfqpoint{1.288853in}{1.050048in}}%
\pgfpathlineto{\pgfqpoint{1.289061in}{0.960133in}}%
\pgfpathlineto{\pgfqpoint{1.290201in}{1.118737in}}%
\pgfpathlineto{\pgfqpoint{1.289475in}{0.879002in}}%
\pgfpathlineto{\pgfqpoint{1.290304in}{1.115812in}}%
\pgfpathlineto{\pgfqpoint{1.291030in}{0.853713in}}%
\pgfpathlineto{\pgfqpoint{1.291444in}{0.939727in}}%
\pgfpathlineto{\pgfqpoint{1.292584in}{1.147791in}}%
\pgfpathlineto{\pgfqpoint{1.292688in}{1.041728in}}%
\pgfpathlineto{\pgfqpoint{1.292792in}{1.102886in}}%
\pgfpathlineto{\pgfqpoint{1.293413in}{0.825616in}}%
\pgfpathlineto{\pgfqpoint{1.293517in}{0.798135in}}%
\pgfpathlineto{\pgfqpoint{1.293724in}{0.908385in}}%
\pgfpathlineto{\pgfqpoint{1.294139in}{0.894341in}}%
\pgfpathlineto{\pgfqpoint{1.294864in}{0.847183in}}%
\pgfpathlineto{\pgfqpoint{1.295486in}{1.089499in}}%
\pgfpathlineto{\pgfqpoint{1.296004in}{1.162481in}}%
\pgfpathlineto{\pgfqpoint{1.296523in}{0.993710in}}%
\pgfpathlineto{\pgfqpoint{1.296626in}{1.177596in}}%
\pgfpathlineto{\pgfqpoint{1.297352in}{0.925084in}}%
\pgfpathlineto{\pgfqpoint{1.297559in}{1.007228in}}%
\pgfpathlineto{\pgfqpoint{1.297766in}{0.895166in}}%
\pgfpathlineto{\pgfqpoint{1.297870in}{1.062389in}}%
\pgfpathlineto{\pgfqpoint{1.297974in}{1.030498in}}%
\pgfpathlineto{\pgfqpoint{1.298077in}{1.120101in}}%
\pgfpathlineto{\pgfqpoint{1.298906in}{0.884412in}}%
\pgfpathlineto{\pgfqpoint{1.299010in}{1.041539in}}%
\pgfpathlineto{\pgfqpoint{1.299321in}{0.853950in}}%
\pgfpathlineto{\pgfqpoint{1.299425in}{1.042160in}}%
\pgfpathlineto{\pgfqpoint{1.300254in}{0.945360in}}%
\pgfpathlineto{\pgfqpoint{1.300357in}{0.898783in}}%
\pgfpathlineto{\pgfqpoint{1.300875in}{1.111977in}}%
\pgfpathlineto{\pgfqpoint{1.301083in}{1.054871in}}%
\pgfpathlineto{\pgfqpoint{1.301290in}{1.234580in}}%
\pgfpathlineto{\pgfqpoint{1.301601in}{0.887105in}}%
\pgfpathlineto{\pgfqpoint{1.302016in}{0.964557in}}%
\pgfpathlineto{\pgfqpoint{1.302119in}{0.847018in}}%
\pgfpathlineto{\pgfqpoint{1.302326in}{1.057917in}}%
\pgfpathlineto{\pgfqpoint{1.303156in}{0.904424in}}%
\pgfpathlineto{\pgfqpoint{1.304192in}{1.019840in}}%
\pgfpathlineto{\pgfqpoint{1.303674in}{0.848865in}}%
\pgfpathlineto{\pgfqpoint{1.304296in}{1.010534in}}%
\pgfpathlineto{\pgfqpoint{1.304399in}{0.920286in}}%
\pgfpathlineto{\pgfqpoint{1.305228in}{1.079956in}}%
\pgfpathlineto{\pgfqpoint{1.305332in}{0.935652in}}%
\pgfpathlineto{\pgfqpoint{1.305747in}{1.074076in}}%
\pgfpathlineto{\pgfqpoint{1.306057in}{0.851782in}}%
\pgfpathlineto{\pgfqpoint{1.306265in}{0.866978in}}%
\pgfpathlineto{\pgfqpoint{1.306368in}{0.827290in}}%
\pgfpathlineto{\pgfqpoint{1.306679in}{0.974494in}}%
\pgfpathlineto{\pgfqpoint{1.307094in}{0.966826in}}%
\pgfpathlineto{\pgfqpoint{1.307301in}{1.049208in}}%
\pgfpathlineto{\pgfqpoint{1.307716in}{0.915177in}}%
\pgfpathlineto{\pgfqpoint{1.308130in}{0.980059in}}%
\pgfpathlineto{\pgfqpoint{1.308234in}{0.979043in}}%
\pgfpathlineto{\pgfqpoint{1.308338in}{0.987656in}}%
\pgfpathlineto{\pgfqpoint{1.308648in}{0.815221in}}%
\pgfpathlineto{\pgfqpoint{1.309374in}{0.873291in}}%
\pgfpathlineto{\pgfqpoint{1.309581in}{1.030859in}}%
\pgfpathlineto{\pgfqpoint{1.310203in}{0.786354in}}%
\pgfpathlineto{\pgfqpoint{1.310307in}{0.836882in}}%
\pgfpathlineto{\pgfqpoint{1.310410in}{0.787611in}}%
\pgfpathlineto{\pgfqpoint{1.310929in}{0.984576in}}%
\pgfpathlineto{\pgfqpoint{1.312069in}{1.134629in}}%
\pgfpathlineto{\pgfqpoint{1.311343in}{0.883203in}}%
\pgfpathlineto{\pgfqpoint{1.312172in}{1.105847in}}%
\pgfpathlineto{\pgfqpoint{1.312276in}{0.904640in}}%
\pgfpathlineto{\pgfqpoint{1.313001in}{1.138923in}}%
\pgfpathlineto{\pgfqpoint{1.313312in}{0.978050in}}%
\pgfpathlineto{\pgfqpoint{1.313830in}{1.259317in}}%
\pgfpathlineto{\pgfqpoint{1.314245in}{0.903129in}}%
\pgfpathlineto{\pgfqpoint{1.314660in}{0.834751in}}%
\pgfpathlineto{\pgfqpoint{1.315074in}{0.980379in}}%
\pgfpathlineto{\pgfqpoint{1.315281in}{0.912278in}}%
\pgfpathlineto{\pgfqpoint{1.315592in}{0.719358in}}%
\pgfpathlineto{\pgfqpoint{1.316421in}{1.094948in}}%
\pgfpathlineto{\pgfqpoint{1.316940in}{1.148471in}}%
\pgfpathlineto{\pgfqpoint{1.317562in}{0.848027in}}%
\pgfpathlineto{\pgfqpoint{1.318183in}{1.083542in}}%
\pgfpathlineto{\pgfqpoint{1.318805in}{1.078327in}}%
\pgfpathlineto{\pgfqpoint{1.319012in}{0.920544in}}%
\pgfpathlineto{\pgfqpoint{1.319842in}{1.141640in}}%
\pgfpathlineto{\pgfqpoint{1.319945in}{1.186559in}}%
\pgfpathlineto{\pgfqpoint{1.320360in}{1.000380in}}%
\pgfpathlineto{\pgfqpoint{1.320567in}{1.067579in}}%
\pgfpathlineto{\pgfqpoint{1.321603in}{0.773509in}}%
\pgfpathlineto{\pgfqpoint{1.322018in}{0.719108in}}%
\pgfpathlineto{\pgfqpoint{1.321811in}{0.852126in}}%
\pgfpathlineto{\pgfqpoint{1.322225in}{0.789756in}}%
\pgfpathlineto{\pgfqpoint{1.322433in}{1.039372in}}%
\pgfpathlineto{\pgfqpoint{1.323365in}{0.931753in}}%
\pgfpathlineto{\pgfqpoint{1.324091in}{0.796117in}}%
\pgfpathlineto{\pgfqpoint{1.323987in}{0.936734in}}%
\pgfpathlineto{\pgfqpoint{1.324505in}{0.810783in}}%
\pgfpathlineto{\pgfqpoint{1.325231in}{1.151196in}}%
\pgfpathlineto{\pgfqpoint{1.325645in}{0.935045in}}%
\pgfpathlineto{\pgfqpoint{1.325749in}{0.823359in}}%
\pgfpathlineto{\pgfqpoint{1.326578in}{1.063092in}}%
\pgfpathlineto{\pgfqpoint{1.326682in}{1.137946in}}%
\pgfpathlineto{\pgfqpoint{1.327407in}{0.921121in}}%
\pgfpathlineto{\pgfqpoint{1.327615in}{1.037359in}}%
\pgfpathlineto{\pgfqpoint{1.328444in}{0.805088in}}%
\pgfpathlineto{\pgfqpoint{1.328858in}{0.873724in}}%
\pgfpathlineto{\pgfqpoint{1.329687in}{1.186228in}}%
\pgfpathlineto{\pgfqpoint{1.329169in}{0.782636in}}%
\pgfpathlineto{\pgfqpoint{1.330102in}{0.927798in}}%
\pgfpathlineto{\pgfqpoint{1.330827in}{0.951156in}}%
\pgfpathlineto{\pgfqpoint{1.331657in}{0.708225in}}%
\pgfpathlineto{\pgfqpoint{1.332486in}{1.078881in}}%
\pgfpathlineto{\pgfqpoint{1.332900in}{0.982974in}}%
\pgfpathlineto{\pgfqpoint{1.333729in}{0.777750in}}%
\pgfpathlineto{\pgfqpoint{1.334040in}{0.880223in}}%
\pgfpathlineto{\pgfqpoint{1.334662in}{0.836159in}}%
\pgfpathlineto{\pgfqpoint{1.335077in}{1.031681in}}%
\pgfpathlineto{\pgfqpoint{1.336320in}{0.737202in}}%
\pgfpathlineto{\pgfqpoint{1.336631in}{1.098895in}}%
\pgfpathlineto{\pgfqpoint{1.337771in}{1.023954in}}%
\pgfpathlineto{\pgfqpoint{1.337875in}{0.922041in}}%
\pgfpathlineto{\pgfqpoint{1.338704in}{1.185581in}}%
\pgfpathlineto{\pgfqpoint{1.338911in}{0.934633in}}%
\pgfpathlineto{\pgfqpoint{1.339015in}{1.073009in}}%
\pgfpathlineto{\pgfqpoint{1.339533in}{0.852651in}}%
\pgfpathlineto{\pgfqpoint{1.340051in}{1.072867in}}%
\pgfpathlineto{\pgfqpoint{1.340259in}{1.129753in}}%
\pgfpathlineto{\pgfqpoint{1.340673in}{1.013982in}}%
\pgfpathlineto{\pgfqpoint{1.340984in}{1.088913in}}%
\pgfpathlineto{\pgfqpoint{1.341088in}{0.934545in}}%
\pgfpathlineto{\pgfqpoint{1.341917in}{1.250694in}}%
\pgfpathlineto{\pgfqpoint{1.342021in}{1.196434in}}%
\pgfpathlineto{\pgfqpoint{1.342746in}{1.070798in}}%
\pgfpathlineto{\pgfqpoint{1.342331in}{1.208436in}}%
\pgfpathlineto{\pgfqpoint{1.342953in}{1.141459in}}%
\pgfpathlineto{\pgfqpoint{1.343057in}{1.267743in}}%
\pgfpathlineto{\pgfqpoint{1.343886in}{1.005907in}}%
\pgfpathlineto{\pgfqpoint{1.343990in}{1.071128in}}%
\pgfpathlineto{\pgfqpoint{1.344922in}{0.858573in}}%
\pgfpathlineto{\pgfqpoint{1.344612in}{1.161336in}}%
\pgfpathlineto{\pgfqpoint{1.345130in}{0.997389in}}%
\pgfpathlineto{\pgfqpoint{1.346063in}{0.867212in}}%
\pgfpathlineto{\pgfqpoint{1.345337in}{1.012390in}}%
\pgfpathlineto{\pgfqpoint{1.346166in}{0.987389in}}%
\pgfpathlineto{\pgfqpoint{1.346373in}{1.059756in}}%
\pgfpathlineto{\pgfqpoint{1.346581in}{1.036108in}}%
\pgfpathlineto{\pgfqpoint{1.346684in}{0.906384in}}%
\pgfpathlineto{\pgfqpoint{1.347203in}{1.161150in}}%
\pgfpathlineto{\pgfqpoint{1.347721in}{0.997981in}}%
\pgfpathlineto{\pgfqpoint{1.347928in}{1.002060in}}%
\pgfpathlineto{\pgfqpoint{1.348964in}{1.233690in}}%
\pgfpathlineto{\pgfqpoint{1.348446in}{0.910905in}}%
\pgfpathlineto{\pgfqpoint{1.349068in}{1.187239in}}%
\pgfpathlineto{\pgfqpoint{1.350001in}{0.891680in}}%
\pgfpathlineto{\pgfqpoint{1.349379in}{1.204900in}}%
\pgfpathlineto{\pgfqpoint{1.350312in}{0.992322in}}%
\pgfpathlineto{\pgfqpoint{1.350934in}{1.191669in}}%
\pgfpathlineto{\pgfqpoint{1.351245in}{0.951398in}}%
\pgfpathlineto{\pgfqpoint{1.351348in}{0.986765in}}%
\pgfpathlineto{\pgfqpoint{1.351763in}{1.053619in}}%
\pgfpathlineto{\pgfqpoint{1.351866in}{1.009360in}}%
\pgfpathlineto{\pgfqpoint{1.352177in}{0.805306in}}%
\pgfpathlineto{\pgfqpoint{1.352592in}{1.027016in}}%
\pgfpathlineto{\pgfqpoint{1.352903in}{0.968143in}}%
\pgfpathlineto{\pgfqpoint{1.353006in}{1.099518in}}%
\pgfpathlineto{\pgfqpoint{1.353939in}{0.887228in}}%
\pgfpathlineto{\pgfqpoint{1.354354in}{1.084582in}}%
\pgfpathlineto{\pgfqpoint{1.355079in}{0.986804in}}%
\pgfpathlineto{\pgfqpoint{1.355390in}{1.068557in}}%
\pgfpathlineto{\pgfqpoint{1.356219in}{0.800982in}}%
\pgfpathlineto{\pgfqpoint{1.356634in}{1.017261in}}%
\pgfpathlineto{\pgfqpoint{1.357359in}{0.874753in}}%
\pgfpathlineto{\pgfqpoint{1.359225in}{1.110553in}}%
\pgfpathlineto{\pgfqpoint{1.357981in}{0.859650in}}%
\pgfpathlineto{\pgfqpoint{1.359432in}{0.982787in}}%
\pgfpathlineto{\pgfqpoint{1.359847in}{1.029405in}}%
\pgfpathlineto{\pgfqpoint{1.360261in}{0.862759in}}%
\pgfpathlineto{\pgfqpoint{1.360572in}{0.859092in}}%
\pgfpathlineto{\pgfqpoint{1.361505in}{1.137949in}}%
\pgfpathlineto{\pgfqpoint{1.362127in}{0.883974in}}%
\pgfpathlineto{\pgfqpoint{1.362645in}{1.061153in}}%
\pgfpathlineto{\pgfqpoint{1.362749in}{1.190148in}}%
\pgfpathlineto{\pgfqpoint{1.362956in}{0.975455in}}%
\pgfpathlineto{\pgfqpoint{1.363578in}{0.983399in}}%
\pgfpathlineto{\pgfqpoint{1.363889in}{0.850586in}}%
\pgfpathlineto{\pgfqpoint{1.364096in}{1.154176in}}%
\pgfpathlineto{\pgfqpoint{1.364614in}{1.006665in}}%
\pgfpathlineto{\pgfqpoint{1.365340in}{1.118697in}}%
\pgfpathlineto{\pgfqpoint{1.364821in}{0.953822in}}%
\pgfpathlineto{\pgfqpoint{1.365650in}{0.993152in}}%
\pgfpathlineto{\pgfqpoint{1.366791in}{0.821841in}}%
\pgfpathlineto{\pgfqpoint{1.365858in}{1.036597in}}%
\pgfpathlineto{\pgfqpoint{1.366998in}{0.874452in}}%
\pgfpathlineto{\pgfqpoint{1.367931in}{1.244766in}}%
\pgfpathlineto{\pgfqpoint{1.368345in}{1.143736in}}%
\pgfpathlineto{\pgfqpoint{1.368967in}{0.954466in}}%
\pgfpathlineto{\pgfqpoint{1.369382in}{1.157260in}}%
\pgfpathlineto{\pgfqpoint{1.369589in}{0.996518in}}%
\pgfpathlineto{\pgfqpoint{1.370211in}{1.220143in}}%
\pgfpathlineto{\pgfqpoint{1.370936in}{1.166417in}}%
\pgfpathlineto{\pgfqpoint{1.371869in}{0.916350in}}%
\pgfpathlineto{\pgfqpoint{1.372076in}{0.965194in}}%
\pgfpathlineto{\pgfqpoint{1.372594in}{0.800861in}}%
\pgfpathlineto{\pgfqpoint{1.372491in}{1.023040in}}%
\pgfpathlineto{\pgfqpoint{1.372698in}{0.974987in}}%
\pgfpathlineto{\pgfqpoint{1.372802in}{1.065432in}}%
\pgfpathlineto{\pgfqpoint{1.373527in}{0.815734in}}%
\pgfpathlineto{\pgfqpoint{1.373631in}{0.981859in}}%
\pgfpathlineto{\pgfqpoint{1.374045in}{0.856753in}}%
\pgfpathlineto{\pgfqpoint{1.374667in}{1.023269in}}%
\pgfpathlineto{\pgfqpoint{1.375393in}{1.241488in}}%
\pgfpathlineto{\pgfqpoint{1.374978in}{0.939042in}}%
\pgfpathlineto{\pgfqpoint{1.375807in}{1.130807in}}%
\pgfpathlineto{\pgfqpoint{1.376118in}{1.138214in}}%
\pgfpathlineto{\pgfqpoint{1.377051in}{0.901320in}}%
\pgfpathlineto{\pgfqpoint{1.378191in}{1.151180in}}%
\pgfpathlineto{\pgfqpoint{1.379124in}{1.036568in}}%
\pgfpathlineto{\pgfqpoint{1.378916in}{1.236974in}}%
\pgfpathlineto{\pgfqpoint{1.379331in}{1.072813in}}%
\pgfpathlineto{\pgfqpoint{1.379435in}{1.074023in}}%
\pgfpathlineto{\pgfqpoint{1.379642in}{0.925071in}}%
\pgfpathlineto{\pgfqpoint{1.379849in}{1.161875in}}%
\pgfpathlineto{\pgfqpoint{1.380264in}{1.146129in}}%
\pgfpathlineto{\pgfqpoint{1.380678in}{1.063434in}}%
\pgfpathlineto{\pgfqpoint{1.381404in}{1.212626in}}%
\pgfpathlineto{\pgfqpoint{1.381922in}{0.958148in}}%
\pgfpathlineto{\pgfqpoint{1.382855in}{1.043528in}}%
\pgfpathlineto{\pgfqpoint{1.383269in}{0.997137in}}%
\pgfpathlineto{\pgfqpoint{1.383891in}{1.278969in}}%
\pgfpathlineto{\pgfqpoint{1.384928in}{0.968235in}}%
\pgfpathlineto{\pgfqpoint{1.385031in}{1.063310in}}%
\pgfpathlineto{\pgfqpoint{1.386378in}{0.875743in}}%
\pgfpathlineto{\pgfqpoint{1.386689in}{1.233765in}}%
\pgfpathlineto{\pgfqpoint{1.387726in}{1.112579in}}%
\pgfpathlineto{\pgfqpoint{1.387933in}{1.124314in}}%
\pgfpathlineto{\pgfqpoint{1.388969in}{0.875403in}}%
\pgfpathlineto{\pgfqpoint{1.390213in}{1.138826in}}%
\pgfpathlineto{\pgfqpoint{1.390317in}{0.923979in}}%
\pgfpathlineto{\pgfqpoint{1.391353in}{1.077601in}}%
\pgfpathlineto{\pgfqpoint{1.391560in}{1.070548in}}%
\pgfpathlineto{\pgfqpoint{1.391664in}{1.111375in}}%
\pgfpathlineto{\pgfqpoint{1.392182in}{0.947254in}}%
\pgfpathlineto{\pgfqpoint{1.392286in}{0.952584in}}%
\pgfpathlineto{\pgfqpoint{1.392390in}{0.945386in}}%
\pgfpathlineto{\pgfqpoint{1.392493in}{1.119621in}}%
\pgfpathlineto{\pgfqpoint{1.393322in}{0.865170in}}%
\pgfpathlineto{\pgfqpoint{1.393426in}{0.959593in}}%
\pgfpathlineto{\pgfqpoint{1.393633in}{0.899793in}}%
\pgfpathlineto{\pgfqpoint{1.393944in}{1.175285in}}%
\pgfpathlineto{\pgfqpoint{1.394255in}{0.961796in}}%
\pgfpathlineto{\pgfqpoint{1.394462in}{1.146071in}}%
\pgfpathlineto{\pgfqpoint{1.394670in}{0.896236in}}%
\pgfpathlineto{\pgfqpoint{1.395292in}{1.017542in}}%
\pgfpathlineto{\pgfqpoint{1.396639in}{0.795299in}}%
\pgfpathlineto{\pgfqpoint{1.395499in}{1.061722in}}%
\pgfpathlineto{\pgfqpoint{1.396742in}{0.829056in}}%
\pgfpathlineto{\pgfqpoint{1.398090in}{1.197798in}}%
\pgfpathlineto{\pgfqpoint{1.398712in}{1.087080in}}%
\pgfpathlineto{\pgfqpoint{1.399748in}{0.892437in}}%
\pgfpathlineto{\pgfqpoint{1.400059in}{0.903298in}}%
\pgfpathlineto{\pgfqpoint{1.400370in}{1.047661in}}%
\pgfpathlineto{\pgfqpoint{1.400681in}{0.783726in}}%
\pgfpathlineto{\pgfqpoint{1.400888in}{0.859699in}}%
\pgfpathlineto{\pgfqpoint{1.400992in}{0.795777in}}%
\pgfpathlineto{\pgfqpoint{1.401510in}{1.041478in}}%
\pgfpathlineto{\pgfqpoint{1.401717in}{0.899971in}}%
\pgfpathlineto{\pgfqpoint{1.402546in}{1.061313in}}%
\pgfpathlineto{\pgfqpoint{1.402754in}{0.875078in}}%
\pgfpathlineto{\pgfqpoint{1.402857in}{0.858352in}}%
\pgfpathlineto{\pgfqpoint{1.402961in}{0.977260in}}%
\pgfpathlineto{\pgfqpoint{1.403272in}{0.929252in}}%
\pgfpathlineto{\pgfqpoint{1.403375in}{0.982243in}}%
\pgfpathlineto{\pgfqpoint{1.403894in}{0.870782in}}%
\pgfpathlineto{\pgfqpoint{1.404205in}{0.931026in}}%
\pgfpathlineto{\pgfqpoint{1.404308in}{0.676778in}}%
\pgfpathlineto{\pgfqpoint{1.404930in}{0.982924in}}%
\pgfpathlineto{\pgfqpoint{1.405345in}{0.850278in}}%
\pgfpathlineto{\pgfqpoint{1.405552in}{0.810008in}}%
\pgfpathlineto{\pgfqpoint{1.405656in}{1.036411in}}%
\pgfpathlineto{\pgfqpoint{1.405759in}{1.015890in}}%
\pgfpathlineto{\pgfqpoint{1.406692in}{1.134333in}}%
\pgfpathlineto{\pgfqpoint{1.406070in}{0.890858in}}%
\pgfpathlineto{\pgfqpoint{1.406796in}{1.032303in}}%
\pgfpathlineto{\pgfqpoint{1.407314in}{0.809275in}}%
\pgfpathlineto{\pgfqpoint{1.407106in}{1.070847in}}%
\pgfpathlineto{\pgfqpoint{1.407936in}{0.810327in}}%
\pgfpathlineto{\pgfqpoint{1.408972in}{1.073943in}}%
\pgfpathlineto{\pgfqpoint{1.409076in}{0.873330in}}%
\pgfpathlineto{\pgfqpoint{1.409905in}{1.060014in}}%
\pgfpathlineto{\pgfqpoint{1.410216in}{0.968359in}}%
\pgfpathlineto{\pgfqpoint{1.410319in}{0.970406in}}%
\pgfpathlineto{\pgfqpoint{1.410423in}{1.025584in}}%
\pgfpathlineto{\pgfqpoint{1.410527in}{0.862632in}}%
\pgfpathlineto{\pgfqpoint{1.411045in}{0.894167in}}%
\pgfpathlineto{\pgfqpoint{1.411148in}{0.666173in}}%
\pgfpathlineto{\pgfqpoint{1.411978in}{0.999948in}}%
\pgfpathlineto{\pgfqpoint{1.412081in}{0.926617in}}%
\pgfpathlineto{\pgfqpoint{1.413118in}{1.098860in}}%
\pgfpathlineto{\pgfqpoint{1.412599in}{0.908319in}}%
\pgfpathlineto{\pgfqpoint{1.413532in}{1.085344in}}%
\pgfpathlineto{\pgfqpoint{1.414672in}{0.886070in}}%
\pgfpathlineto{\pgfqpoint{1.414776in}{0.915979in}}%
\pgfpathlineto{\pgfqpoint{1.416020in}{1.121012in}}%
\pgfpathlineto{\pgfqpoint{1.416330in}{0.844060in}}%
\pgfpathlineto{\pgfqpoint{1.417160in}{0.878806in}}%
\pgfpathlineto{\pgfqpoint{1.417263in}{1.121643in}}%
\pgfpathlineto{\pgfqpoint{1.418300in}{1.112078in}}%
\pgfpathlineto{\pgfqpoint{1.418611in}{1.176026in}}%
\pgfpathlineto{\pgfqpoint{1.419336in}{1.009538in}}%
\pgfpathlineto{\pgfqpoint{1.419647in}{0.965723in}}%
\pgfpathlineto{\pgfqpoint{1.420372in}{1.273305in}}%
\pgfpathlineto{\pgfqpoint{1.420476in}{1.056965in}}%
\pgfpathlineto{\pgfqpoint{1.421512in}{1.105089in}}%
\pgfpathlineto{\pgfqpoint{1.421616in}{1.106869in}}%
\pgfpathlineto{\pgfqpoint{1.421720in}{1.047787in}}%
\pgfpathlineto{\pgfqpoint{1.422134in}{1.286260in}}%
\pgfpathlineto{\pgfqpoint{1.422652in}{1.117294in}}%
\pgfpathlineto{\pgfqpoint{1.422756in}{1.195902in}}%
\pgfpathlineto{\pgfqpoint{1.423482in}{1.031118in}}%
\pgfpathlineto{\pgfqpoint{1.423689in}{1.073354in}}%
\pgfpathlineto{\pgfqpoint{1.424103in}{0.936362in}}%
\pgfpathlineto{\pgfqpoint{1.424622in}{1.132647in}}%
\pgfpathlineto{\pgfqpoint{1.424725in}{1.188546in}}%
\pgfpathlineto{\pgfqpoint{1.424829in}{0.987464in}}%
\pgfpathlineto{\pgfqpoint{1.425451in}{1.059278in}}%
\pgfpathlineto{\pgfqpoint{1.425658in}{0.971254in}}%
\pgfpathlineto{\pgfqpoint{1.426384in}{1.119789in}}%
\pgfpathlineto{\pgfqpoint{1.426487in}{1.164390in}}%
\pgfpathlineto{\pgfqpoint{1.426902in}{1.006498in}}%
\pgfpathlineto{\pgfqpoint{1.427316in}{1.070593in}}%
\pgfpathlineto{\pgfqpoint{1.427627in}{1.094353in}}%
\pgfpathlineto{\pgfqpoint{1.427524in}{1.021062in}}%
\pgfpathlineto{\pgfqpoint{1.428042in}{1.041361in}}%
\pgfpathlineto{\pgfqpoint{1.428145in}{1.016020in}}%
\pgfpathlineto{\pgfqpoint{1.428456in}{1.153882in}}%
\pgfpathlineto{\pgfqpoint{1.428560in}{1.202031in}}%
\pgfpathlineto{\pgfqpoint{1.428975in}{1.040728in}}%
\pgfpathlineto{\pgfqpoint{1.429389in}{1.144712in}}%
\pgfpathlineto{\pgfqpoint{1.429907in}{0.801748in}}%
\pgfpathlineto{\pgfqpoint{1.430633in}{0.906895in}}%
\pgfpathlineto{\pgfqpoint{1.431566in}{1.324607in}}%
\pgfpathlineto{\pgfqpoint{1.431876in}{1.156922in}}%
\pgfpathlineto{\pgfqpoint{1.432187in}{1.032188in}}%
\pgfpathlineto{\pgfqpoint{1.432809in}{1.199611in}}%
\pgfpathlineto{\pgfqpoint{1.433016in}{1.120637in}}%
\pgfpathlineto{\pgfqpoint{1.434364in}{0.882973in}}%
\pgfpathlineto{\pgfqpoint{1.433327in}{1.171722in}}%
\pgfpathlineto{\pgfqpoint{1.434571in}{0.894150in}}%
\pgfpathlineto{\pgfqpoint{1.435400in}{1.093541in}}%
\pgfpathlineto{\pgfqpoint{1.435711in}{1.045789in}}%
\pgfpathlineto{\pgfqpoint{1.435815in}{1.051204in}}%
\pgfpathlineto{\pgfqpoint{1.436022in}{1.008707in}}%
\pgfpathlineto{\pgfqpoint{1.436126in}{0.978630in}}%
\pgfpathlineto{\pgfqpoint{1.436437in}{1.157181in}}%
\pgfpathlineto{\pgfqpoint{1.436748in}{1.073533in}}%
\pgfpathlineto{\pgfqpoint{1.437680in}{1.170985in}}%
\pgfpathlineto{\pgfqpoint{1.437266in}{0.968979in}}%
\pgfpathlineto{\pgfqpoint{1.437888in}{1.157670in}}%
\pgfpathlineto{\pgfqpoint{1.437991in}{0.969522in}}%
\pgfpathlineto{\pgfqpoint{1.438198in}{1.232593in}}%
\pgfpathlineto{\pgfqpoint{1.438924in}{1.007548in}}%
\pgfpathlineto{\pgfqpoint{1.439753in}{0.989145in}}%
\pgfpathlineto{\pgfqpoint{1.440168in}{1.284770in}}%
\pgfpathlineto{\pgfqpoint{1.440997in}{1.095811in}}%
\pgfpathlineto{\pgfqpoint{1.441308in}{1.097080in}}%
\pgfpathlineto{\pgfqpoint{1.441515in}{1.258970in}}%
\pgfpathlineto{\pgfqpoint{1.442137in}{1.078292in}}%
\pgfpathlineto{\pgfqpoint{1.442344in}{1.142652in}}%
\pgfpathlineto{\pgfqpoint{1.442551in}{1.007951in}}%
\pgfpathlineto{\pgfqpoint{1.442862in}{1.186260in}}%
\pgfpathlineto{\pgfqpoint{1.443380in}{1.108020in}}%
\pgfpathlineto{\pgfqpoint{1.444521in}{1.366990in}}%
\pgfpathlineto{\pgfqpoint{1.444728in}{1.211922in}}%
\pgfpathlineto{\pgfqpoint{1.444831in}{1.200639in}}%
\pgfpathlineto{\pgfqpoint{1.444935in}{1.248306in}}%
\pgfpathlineto{\pgfqpoint{1.445350in}{1.341865in}}%
\pgfpathlineto{\pgfqpoint{1.445971in}{1.230180in}}%
\pgfpathlineto{\pgfqpoint{1.447112in}{0.875924in}}%
\pgfpathlineto{\pgfqpoint{1.446490in}{1.328883in}}%
\pgfpathlineto{\pgfqpoint{1.447733in}{0.990480in}}%
\pgfpathlineto{\pgfqpoint{1.448252in}{1.255169in}}%
\pgfpathlineto{\pgfqpoint{1.448873in}{1.124925in}}%
\pgfpathlineto{\pgfqpoint{1.449910in}{0.968618in}}%
\pgfpathlineto{\pgfqpoint{1.449288in}{1.201420in}}%
\pgfpathlineto{\pgfqpoint{1.450117in}{0.990378in}}%
\pgfpathlineto{\pgfqpoint{1.450221in}{0.983302in}}%
\pgfpathlineto{\pgfqpoint{1.450324in}{1.046255in}}%
\pgfpathlineto{\pgfqpoint{1.450428in}{1.048071in}}%
\pgfpathlineto{\pgfqpoint{1.451050in}{0.943115in}}%
\pgfpathlineto{\pgfqpoint{1.451153in}{1.065955in}}%
\pgfpathlineto{\pgfqpoint{1.451361in}{1.022847in}}%
\pgfpathlineto{\pgfqpoint{1.451983in}{1.152220in}}%
\pgfpathlineto{\pgfqpoint{1.452294in}{0.919251in}}%
\pgfpathlineto{\pgfqpoint{1.452397in}{1.007111in}}%
\pgfpathlineto{\pgfqpoint{1.452604in}{1.106540in}}%
\pgfpathlineto{\pgfqpoint{1.452708in}{1.098655in}}%
\pgfpathlineto{\pgfqpoint{1.452812in}{1.172827in}}%
\pgfpathlineto{\pgfqpoint{1.453537in}{0.940645in}}%
\pgfpathlineto{\pgfqpoint{1.453744in}{1.048887in}}%
\pgfpathlineto{\pgfqpoint{1.453952in}{0.957491in}}%
\pgfpathlineto{\pgfqpoint{1.454055in}{1.059014in}}%
\pgfpathlineto{\pgfqpoint{1.454366in}{1.028692in}}%
\pgfpathlineto{\pgfqpoint{1.454677in}{1.227572in}}%
\pgfpathlineto{\pgfqpoint{1.455403in}{0.908930in}}%
\pgfpathlineto{\pgfqpoint{1.456335in}{0.824460in}}%
\pgfpathlineto{\pgfqpoint{1.455817in}{1.071952in}}%
\pgfpathlineto{\pgfqpoint{1.456543in}{0.879113in}}%
\pgfpathlineto{\pgfqpoint{1.456646in}{0.838316in}}%
\pgfpathlineto{\pgfqpoint{1.457061in}{1.013507in}}%
\pgfpathlineto{\pgfqpoint{1.457476in}{0.943267in}}%
\pgfpathlineto{\pgfqpoint{1.458408in}{1.142705in}}%
\pgfpathlineto{\pgfqpoint{1.458616in}{0.998832in}}%
\pgfpathlineto{\pgfqpoint{1.459341in}{0.847891in}}%
\pgfpathlineto{\pgfqpoint{1.458823in}{1.021224in}}%
\pgfpathlineto{\pgfqpoint{1.459652in}{0.964433in}}%
\pgfpathlineto{\pgfqpoint{1.459859in}{1.105004in}}%
\pgfpathlineto{\pgfqpoint{1.460481in}{0.886136in}}%
\pgfpathlineto{\pgfqpoint{1.460792in}{0.991572in}}%
\pgfpathlineto{\pgfqpoint{1.461414in}{1.053321in}}%
\pgfpathlineto{\pgfqpoint{1.461621in}{0.893449in}}%
\pgfpathlineto{\pgfqpoint{1.461828in}{1.188682in}}%
\pgfpathlineto{\pgfqpoint{1.462761in}{1.108451in}}%
\pgfpathlineto{\pgfqpoint{1.462968in}{1.225770in}}%
\pgfpathlineto{\pgfqpoint{1.463279in}{0.907123in}}%
\pgfpathlineto{\pgfqpoint{1.463694in}{1.056588in}}%
\pgfpathlineto{\pgfqpoint{1.463798in}{0.963708in}}%
\pgfpathlineto{\pgfqpoint{1.464316in}{1.147193in}}%
\pgfpathlineto{\pgfqpoint{1.464730in}{1.090390in}}%
\pgfpathlineto{\pgfqpoint{1.465145in}{1.337853in}}%
\pgfpathlineto{\pgfqpoint{1.465352in}{0.942822in}}%
\pgfpathlineto{\pgfqpoint{1.465663in}{1.040006in}}%
\pgfpathlineto{\pgfqpoint{1.465767in}{1.042103in}}%
\pgfpathlineto{\pgfqpoint{1.465870in}{1.031905in}}%
\pgfpathlineto{\pgfqpoint{1.466181in}{0.951553in}}%
\pgfpathlineto{\pgfqpoint{1.467010in}{1.128666in}}%
\pgfpathlineto{\pgfqpoint{1.467425in}{0.855758in}}%
\pgfpathlineto{\pgfqpoint{1.468047in}{1.001053in}}%
\pgfpathlineto{\pgfqpoint{1.468150in}{1.167261in}}%
\pgfpathlineto{\pgfqpoint{1.468669in}{0.871485in}}%
\pgfpathlineto{\pgfqpoint{1.469083in}{1.101206in}}%
\pgfpathlineto{\pgfqpoint{1.470016in}{0.958322in}}%
\pgfpathlineto{\pgfqpoint{1.469290in}{1.133406in}}%
\pgfpathlineto{\pgfqpoint{1.470223in}{0.989940in}}%
\pgfpathlineto{\pgfqpoint{1.470327in}{1.241447in}}%
\pgfpathlineto{\pgfqpoint{1.470845in}{0.924384in}}%
\pgfpathlineto{\pgfqpoint{1.471363in}{1.060448in}}%
\pgfpathlineto{\pgfqpoint{1.471674in}{1.168193in}}%
\pgfpathlineto{\pgfqpoint{1.471571in}{1.035831in}}%
\pgfpathlineto{\pgfqpoint{1.471881in}{1.084545in}}%
\pgfpathlineto{\pgfqpoint{1.473022in}{0.926186in}}%
\pgfpathlineto{\pgfqpoint{1.473851in}{1.169165in}}%
\pgfpathlineto{\pgfqpoint{1.473954in}{0.924578in}}%
\pgfpathlineto{\pgfqpoint{1.474265in}{1.051770in}}%
\pgfpathlineto{\pgfqpoint{1.474576in}{0.866389in}}%
\pgfpathlineto{\pgfqpoint{1.474991in}{1.217794in}}%
\pgfpathlineto{\pgfqpoint{1.475094in}{1.192712in}}%
\pgfpathlineto{\pgfqpoint{1.475302in}{1.290314in}}%
\pgfpathlineto{\pgfqpoint{1.475509in}{1.173435in}}%
\pgfpathlineto{\pgfqpoint{1.475820in}{1.183592in}}%
\pgfpathlineto{\pgfqpoint{1.476442in}{1.251914in}}%
\pgfpathlineto{\pgfqpoint{1.476753in}{1.074764in}}%
\pgfpathlineto{\pgfqpoint{1.476856in}{1.289702in}}%
\pgfpathlineto{\pgfqpoint{1.477789in}{0.995287in}}%
\pgfpathlineto{\pgfqpoint{1.477893in}{0.975116in}}%
\pgfpathlineto{\pgfqpoint{1.477996in}{1.161913in}}%
\pgfpathlineto{\pgfqpoint{1.478307in}{1.269836in}}%
\pgfpathlineto{\pgfqpoint{1.478825in}{1.133995in}}%
\pgfpathlineto{\pgfqpoint{1.479240in}{1.258682in}}%
\pgfpathlineto{\pgfqpoint{1.480276in}{1.013406in}}%
\pgfpathlineto{\pgfqpoint{1.480069in}{1.308257in}}%
\pgfpathlineto{\pgfqpoint{1.480691in}{1.155003in}}%
\pgfpathlineto{\pgfqpoint{1.481313in}{1.293561in}}%
\pgfpathlineto{\pgfqpoint{1.480898in}{1.070439in}}%
\pgfpathlineto{\pgfqpoint{1.481727in}{1.118787in}}%
\pgfpathlineto{\pgfqpoint{1.482142in}{1.061800in}}%
\pgfpathlineto{\pgfqpoint{1.482038in}{1.165619in}}%
\pgfpathlineto{\pgfqpoint{1.482453in}{1.158006in}}%
\pgfpathlineto{\pgfqpoint{1.482764in}{1.274297in}}%
\pgfpathlineto{\pgfqpoint{1.483075in}{1.111506in}}%
\pgfpathlineto{\pgfqpoint{1.483178in}{1.155529in}}%
\pgfpathlineto{\pgfqpoint{1.483696in}{0.978065in}}%
\pgfpathlineto{\pgfqpoint{1.484215in}{1.104123in}}%
\pgfpathlineto{\pgfqpoint{1.485251in}{1.202983in}}%
\pgfpathlineto{\pgfqpoint{1.484940in}{1.030196in}}%
\pgfpathlineto{\pgfqpoint{1.485355in}{1.126899in}}%
\pgfpathlineto{\pgfqpoint{1.485458in}{1.068707in}}%
\pgfpathlineto{\pgfqpoint{1.486184in}{1.237459in}}%
\pgfpathlineto{\pgfqpoint{1.486287in}{1.234132in}}%
\pgfpathlineto{\pgfqpoint{1.486598in}{1.055515in}}%
\pgfpathlineto{\pgfqpoint{1.486495in}{1.271886in}}%
\pgfpathlineto{\pgfqpoint{1.487635in}{1.124927in}}%
\pgfpathlineto{\pgfqpoint{1.488464in}{1.050122in}}%
\pgfpathlineto{\pgfqpoint{1.488257in}{1.169394in}}%
\pgfpathlineto{\pgfqpoint{1.488671in}{1.108590in}}%
\pgfpathlineto{\pgfqpoint{1.488775in}{1.165489in}}%
\pgfpathlineto{\pgfqpoint{1.488982in}{0.976124in}}%
\pgfpathlineto{\pgfqpoint{1.489604in}{1.042814in}}%
\pgfpathlineto{\pgfqpoint{1.490329in}{1.159491in}}%
\pgfpathlineto{\pgfqpoint{1.490537in}{1.008796in}}%
\pgfpathlineto{\pgfqpoint{1.490640in}{1.064662in}}%
\pgfpathlineto{\pgfqpoint{1.490951in}{1.226236in}}%
\pgfpathlineto{\pgfqpoint{1.491677in}{0.959215in}}%
\pgfpathlineto{\pgfqpoint{1.492195in}{1.157149in}}%
\pgfpathlineto{\pgfqpoint{1.492402in}{0.924182in}}%
\pgfpathlineto{\pgfqpoint{1.492817in}{1.024853in}}%
\pgfpathlineto{\pgfqpoint{1.493024in}{1.159069in}}%
\pgfpathlineto{\pgfqpoint{1.493231in}{1.015871in}}%
\pgfpathlineto{\pgfqpoint{1.493957in}{1.060117in}}%
\pgfpathlineto{\pgfqpoint{1.494060in}{0.952748in}}%
\pgfpathlineto{\pgfqpoint{1.494786in}{1.187874in}}%
\pgfpathlineto{\pgfqpoint{1.494890in}{1.176892in}}%
\pgfpathlineto{\pgfqpoint{1.495822in}{0.998852in}}%
\pgfpathlineto{\pgfqpoint{1.496030in}{1.111431in}}%
\pgfpathlineto{\pgfqpoint{1.496444in}{1.274306in}}%
\pgfpathlineto{\pgfqpoint{1.496237in}{1.007102in}}%
\pgfpathlineto{\pgfqpoint{1.497066in}{1.177148in}}%
\pgfpathlineto{\pgfqpoint{1.498102in}{1.064054in}}%
\pgfpathlineto{\pgfqpoint{1.498413in}{1.189424in}}%
\pgfpathlineto{\pgfqpoint{1.499035in}{0.957372in}}%
\pgfpathlineto{\pgfqpoint{1.499139in}{0.990190in}}%
\pgfpathlineto{\pgfqpoint{1.500279in}{1.257293in}}%
\pgfpathlineto{\pgfqpoint{1.499657in}{0.936003in}}%
\pgfpathlineto{\pgfqpoint{1.500590in}{1.162032in}}%
\pgfpathlineto{\pgfqpoint{1.501315in}{1.234913in}}%
\pgfpathlineto{\pgfqpoint{1.501730in}{1.040220in}}%
\pgfpathlineto{\pgfqpoint{1.502352in}{1.265395in}}%
\pgfpathlineto{\pgfqpoint{1.502974in}{1.222935in}}%
\pgfpathlineto{\pgfqpoint{1.503077in}{1.038629in}}%
\pgfpathlineto{\pgfqpoint{1.504114in}{1.166177in}}%
\pgfpathlineto{\pgfqpoint{1.504217in}{1.187072in}}%
\pgfpathlineto{\pgfqpoint{1.504424in}{1.093523in}}%
\pgfpathlineto{\pgfqpoint{1.504528in}{0.967568in}}%
\pgfpathlineto{\pgfqpoint{1.505357in}{1.219788in}}%
\pgfpathlineto{\pgfqpoint{1.505565in}{1.017421in}}%
\pgfpathlineto{\pgfqpoint{1.505979in}{1.183535in}}%
\pgfpathlineto{\pgfqpoint{1.505875in}{1.001664in}}%
\pgfpathlineto{\pgfqpoint{1.506808in}{1.075085in}}%
\pgfpathlineto{\pgfqpoint{1.506912in}{1.035252in}}%
\pgfpathlineto{\pgfqpoint{1.507326in}{1.245596in}}%
\pgfpathlineto{\pgfqpoint{1.507430in}{1.096469in}}%
\pgfpathlineto{\pgfqpoint{1.507534in}{1.249035in}}%
\pgfpathlineto{\pgfqpoint{1.507637in}{1.018377in}}%
\pgfpathlineto{\pgfqpoint{1.508363in}{1.034603in}}%
\pgfpathlineto{\pgfqpoint{1.508466in}{0.965902in}}%
\pgfpathlineto{\pgfqpoint{1.508777in}{1.204600in}}%
\pgfpathlineto{\pgfqpoint{1.509192in}{1.194793in}}%
\pgfpathlineto{\pgfqpoint{1.509296in}{1.223034in}}%
\pgfpathlineto{\pgfqpoint{1.509814in}{1.140157in}}%
\pgfpathlineto{\pgfqpoint{1.510436in}{1.056109in}}%
\pgfpathlineto{\pgfqpoint{1.510228in}{1.239662in}}%
\pgfpathlineto{\pgfqpoint{1.510850in}{1.085825in}}%
\pgfpathlineto{\pgfqpoint{1.510954in}{1.216466in}}%
\pgfpathlineto{\pgfqpoint{1.511576in}{0.937981in}}%
\pgfpathlineto{\pgfqpoint{1.511783in}{1.059919in}}%
\pgfpathlineto{\pgfqpoint{1.511990in}{0.958107in}}%
\pgfpathlineto{\pgfqpoint{1.512716in}{1.161137in}}%
\pgfpathlineto{\pgfqpoint{1.513130in}{1.084599in}}%
\pgfpathlineto{\pgfqpoint{1.513027in}{1.263569in}}%
\pgfpathlineto{\pgfqpoint{1.513856in}{1.138246in}}%
\pgfpathlineto{\pgfqpoint{1.514167in}{1.293530in}}%
\pgfpathlineto{\pgfqpoint{1.514892in}{1.249971in}}%
\pgfpathlineto{\pgfqpoint{1.515307in}{1.022220in}}%
\pgfpathlineto{\pgfqpoint{1.515929in}{1.312923in}}%
\pgfpathlineto{\pgfqpoint{1.516758in}{1.387017in}}%
\pgfpathlineto{\pgfqpoint{1.517069in}{1.130063in}}%
\pgfpathlineto{\pgfqpoint{1.518001in}{1.334611in}}%
\pgfpathlineto{\pgfqpoint{1.517379in}{1.096083in}}%
\pgfpathlineto{\pgfqpoint{1.518209in}{1.218213in}}%
\pgfpathlineto{\pgfqpoint{1.518416in}{1.118555in}}%
\pgfpathlineto{\pgfqpoint{1.518830in}{1.257656in}}%
\pgfpathlineto{\pgfqpoint{1.519349in}{1.197968in}}%
\pgfpathlineto{\pgfqpoint{1.519867in}{1.398942in}}%
\pgfpathlineto{\pgfqpoint{1.519763in}{1.171012in}}%
\pgfpathlineto{\pgfqpoint{1.520489in}{1.249663in}}%
\pgfpathlineto{\pgfqpoint{1.520696in}{1.058236in}}%
\pgfpathlineto{\pgfqpoint{1.521525in}{1.190454in}}%
\pgfpathlineto{\pgfqpoint{1.521732in}{1.346005in}}%
\pgfpathlineto{\pgfqpoint{1.522147in}{1.060838in}}%
\pgfpathlineto{\pgfqpoint{1.522561in}{1.229243in}}%
\pgfpathlineto{\pgfqpoint{1.523598in}{1.112190in}}%
\pgfpathlineto{\pgfqpoint{1.523391in}{1.303698in}}%
\pgfpathlineto{\pgfqpoint{1.523702in}{1.194317in}}%
\pgfpathlineto{\pgfqpoint{1.523909in}{1.148312in}}%
\pgfpathlineto{\pgfqpoint{1.524323in}{1.250617in}}%
\pgfpathlineto{\pgfqpoint{1.524427in}{1.194026in}}%
\pgfpathlineto{\pgfqpoint{1.524842in}{1.325467in}}%
\pgfpathlineto{\pgfqpoint{1.524945in}{1.173012in}}%
\pgfpathlineto{\pgfqpoint{1.525567in}{1.274885in}}%
\pgfpathlineto{\pgfqpoint{1.525671in}{1.324111in}}%
\pgfpathlineto{\pgfqpoint{1.525878in}{1.134833in}}%
\pgfpathlineto{\pgfqpoint{1.526396in}{1.174562in}}%
\pgfpathlineto{\pgfqpoint{1.527018in}{0.977831in}}%
\pgfpathlineto{\pgfqpoint{1.527433in}{1.236035in}}%
\pgfpathlineto{\pgfqpoint{1.527536in}{1.131833in}}%
\pgfpathlineto{\pgfqpoint{1.528054in}{1.029683in}}%
\pgfpathlineto{\pgfqpoint{1.527951in}{1.186511in}}%
\pgfpathlineto{\pgfqpoint{1.528573in}{1.154730in}}%
\pgfpathlineto{\pgfqpoint{1.529713in}{1.279666in}}%
\pgfpathlineto{\pgfqpoint{1.529194in}{1.153728in}}%
\pgfpathlineto{\pgfqpoint{1.529816in}{1.245111in}}%
\pgfpathlineto{\pgfqpoint{1.530853in}{1.073370in}}%
\pgfpathlineto{\pgfqpoint{1.530024in}{1.247985in}}%
\pgfpathlineto{\pgfqpoint{1.530956in}{1.164470in}}%
\pgfpathlineto{\pgfqpoint{1.531578in}{1.194133in}}%
\pgfpathlineto{\pgfqpoint{1.531371in}{1.041813in}}%
\pgfpathlineto{\pgfqpoint{1.531889in}{1.159297in}}%
\pgfpathlineto{\pgfqpoint{1.532511in}{1.098917in}}%
\pgfpathlineto{\pgfqpoint{1.532096in}{1.174500in}}%
\pgfpathlineto{\pgfqpoint{1.532615in}{1.163636in}}%
\pgfpathlineto{\pgfqpoint{1.532718in}{1.274596in}}%
\pgfpathlineto{\pgfqpoint{1.533029in}{0.967163in}}%
\pgfpathlineto{\pgfqpoint{1.533547in}{0.977819in}}%
\pgfpathlineto{\pgfqpoint{1.533651in}{0.911945in}}%
\pgfpathlineto{\pgfqpoint{1.534273in}{1.113637in}}%
\pgfpathlineto{\pgfqpoint{1.534480in}{1.025072in}}%
\pgfpathlineto{\pgfqpoint{1.534791in}{1.090261in}}%
\pgfpathlineto{\pgfqpoint{1.534998in}{0.949128in}}%
\pgfpathlineto{\pgfqpoint{1.535413in}{1.017546in}}%
\pgfpathlineto{\pgfqpoint{1.535827in}{0.866519in}}%
\pgfpathlineto{\pgfqpoint{1.536242in}{1.122557in}}%
\pgfpathlineto{\pgfqpoint{1.536346in}{1.099715in}}%
\pgfpathlineto{\pgfqpoint{1.536449in}{1.098931in}}%
\pgfpathlineto{\pgfqpoint{1.537278in}{0.931529in}}%
\pgfpathlineto{\pgfqpoint{1.537382in}{1.124420in}}%
\pgfpathlineto{\pgfqpoint{1.537486in}{1.107679in}}%
\pgfpathlineto{\pgfqpoint{1.537589in}{1.102733in}}%
\pgfpathlineto{\pgfqpoint{1.538626in}{0.950082in}}%
\pgfpathlineto{\pgfqpoint{1.538729in}{1.041430in}}%
\pgfpathlineto{\pgfqpoint{1.539558in}{1.168136in}}%
\pgfpathlineto{\pgfqpoint{1.539040in}{0.972866in}}%
\pgfpathlineto{\pgfqpoint{1.539662in}{1.071967in}}%
\pgfpathlineto{\pgfqpoint{1.539973in}{1.171522in}}%
\pgfpathlineto{\pgfqpoint{1.540802in}{0.948287in}}%
\pgfpathlineto{\pgfqpoint{1.542149in}{1.244532in}}%
\pgfpathlineto{\pgfqpoint{1.542253in}{1.157044in}}%
\pgfpathlineto{\pgfqpoint{1.542357in}{1.162882in}}%
\pgfpathlineto{\pgfqpoint{1.543289in}{1.007819in}}%
\pgfpathlineto{\pgfqpoint{1.543497in}{1.019221in}}%
\pgfpathlineto{\pgfqpoint{1.543704in}{1.002988in}}%
\pgfpathlineto{\pgfqpoint{1.544533in}{1.117058in}}%
\pgfpathlineto{\pgfqpoint{1.544637in}{0.988842in}}%
\pgfpathlineto{\pgfqpoint{1.545570in}{1.205960in}}%
\pgfpathlineto{\pgfqpoint{1.545673in}{1.101253in}}%
\pgfpathlineto{\pgfqpoint{1.546606in}{1.150182in}}%
\pgfpathlineto{\pgfqpoint{1.547331in}{1.269254in}}%
\pgfpathlineto{\pgfqpoint{1.547539in}{1.073538in}}%
\pgfpathlineto{\pgfqpoint{1.547642in}{1.125254in}}%
\pgfpathlineto{\pgfqpoint{1.548368in}{1.292327in}}%
\pgfpathlineto{\pgfqpoint{1.548575in}{1.107117in}}%
\pgfpathlineto{\pgfqpoint{1.548679in}{1.148039in}}%
\pgfpathlineto{\pgfqpoint{1.549922in}{0.951799in}}%
\pgfpathlineto{\pgfqpoint{1.549197in}{1.248899in}}%
\pgfpathlineto{\pgfqpoint{1.550026in}{0.972616in}}%
\pgfpathlineto{\pgfqpoint{1.551062in}{1.189109in}}%
\pgfpathlineto{\pgfqpoint{1.551581in}{1.147715in}}%
\pgfpathlineto{\pgfqpoint{1.552203in}{1.212119in}}%
\pgfpathlineto{\pgfqpoint{1.552721in}{1.046461in}}%
\pgfpathlineto{\pgfqpoint{1.553446in}{1.167918in}}%
\pgfpathlineto{\pgfqpoint{1.552928in}{0.999440in}}%
\pgfpathlineto{\pgfqpoint{1.553757in}{1.081749in}}%
\pgfpathlineto{\pgfqpoint{1.554275in}{0.947340in}}%
\pgfpathlineto{\pgfqpoint{1.554690in}{1.092693in}}%
\pgfpathlineto{\pgfqpoint{1.554794in}{1.027043in}}%
\pgfpathlineto{\pgfqpoint{1.556037in}{1.294812in}}%
\pgfpathlineto{\pgfqpoint{1.555104in}{1.001350in}}%
\pgfpathlineto{\pgfqpoint{1.556141in}{1.197257in}}%
\pgfpathlineto{\pgfqpoint{1.556452in}{0.983598in}}%
\pgfpathlineto{\pgfqpoint{1.556970in}{1.218706in}}%
\pgfpathlineto{\pgfqpoint{1.557074in}{1.139830in}}%
\pgfpathlineto{\pgfqpoint{1.557281in}{1.343321in}}%
\pgfpathlineto{\pgfqpoint{1.557799in}{1.103274in}}%
\pgfpathlineto{\pgfqpoint{1.558110in}{1.231658in}}%
\pgfpathlineto{\pgfqpoint{1.558317in}{1.071659in}}%
\pgfpathlineto{\pgfqpoint{1.558939in}{1.267784in}}%
\pgfpathlineto{\pgfqpoint{1.559250in}{1.200600in}}%
\pgfpathlineto{\pgfqpoint{1.559354in}{1.206468in}}%
\pgfpathlineto{\pgfqpoint{1.560597in}{1.380791in}}%
\pgfpathlineto{\pgfqpoint{1.560805in}{1.125792in}}%
\pgfpathlineto{\pgfqpoint{1.561634in}{1.294510in}}%
\pgfpathlineto{\pgfqpoint{1.561841in}{1.422732in}}%
\pgfpathlineto{\pgfqpoint{1.562152in}{1.202253in}}%
\pgfpathlineto{\pgfqpoint{1.562463in}{1.292970in}}%
\pgfpathlineto{\pgfqpoint{1.563292in}{1.089668in}}%
\pgfpathlineto{\pgfqpoint{1.563707in}{1.169381in}}%
\pgfpathlineto{\pgfqpoint{1.565261in}{1.348556in}}%
\pgfpathlineto{\pgfqpoint{1.566194in}{1.138200in}}%
\pgfpathlineto{\pgfqpoint{1.566401in}{1.192998in}}%
\pgfpathlineto{\pgfqpoint{1.566505in}{1.286266in}}%
\pgfpathlineto{\pgfqpoint{1.566712in}{1.142750in}}%
\pgfpathlineto{\pgfqpoint{1.567438in}{1.165763in}}%
\pgfpathlineto{\pgfqpoint{1.567852in}{1.254038in}}%
\pgfpathlineto{\pgfqpoint{1.568474in}{1.012940in}}%
\pgfpathlineto{\pgfqpoint{1.568681in}{1.243220in}}%
\pgfpathlineto{\pgfqpoint{1.569510in}{0.967829in}}%
\pgfpathlineto{\pgfqpoint{1.569614in}{1.111831in}}%
\pgfpathlineto{\pgfqpoint{1.569718in}{0.970504in}}%
\pgfpathlineto{\pgfqpoint{1.570547in}{1.168781in}}%
\pgfpathlineto{\pgfqpoint{1.570650in}{1.215460in}}%
\pgfpathlineto{\pgfqpoint{1.571376in}{1.047956in}}%
\pgfpathlineto{\pgfqpoint{1.571894in}{0.932473in}}%
\pgfpathlineto{\pgfqpoint{1.572205in}{1.083117in}}%
\pgfpathlineto{\pgfqpoint{1.572412in}{1.063832in}}%
\pgfpathlineto{\pgfqpoint{1.572516in}{1.151207in}}%
\pgfpathlineto{\pgfqpoint{1.573034in}{0.933797in}}%
\pgfpathlineto{\pgfqpoint{1.573345in}{1.039216in}}%
\pgfpathlineto{\pgfqpoint{1.574174in}{0.939542in}}%
\pgfpathlineto{\pgfqpoint{1.574278in}{1.097614in}}%
\pgfpathlineto{\pgfqpoint{1.574485in}{0.945033in}}%
\pgfpathlineto{\pgfqpoint{1.575003in}{1.151813in}}%
\pgfpathlineto{\pgfqpoint{1.574900in}{0.910267in}}%
\pgfpathlineto{\pgfqpoint{1.575729in}{1.055831in}}%
\pgfpathlineto{\pgfqpoint{1.575936in}{0.902470in}}%
\pgfpathlineto{\pgfqpoint{1.576351in}{1.175613in}}%
\pgfpathlineto{\pgfqpoint{1.576454in}{1.052211in}}%
\pgfpathlineto{\pgfqpoint{1.576558in}{1.142425in}}%
\pgfpathlineto{\pgfqpoint{1.576765in}{0.947743in}}%
\pgfpathlineto{\pgfqpoint{1.577594in}{1.084742in}}%
\pgfpathlineto{\pgfqpoint{1.577698in}{1.080656in}}%
\pgfpathlineto{\pgfqpoint{1.577802in}{0.897712in}}%
\pgfpathlineto{\pgfqpoint{1.578734in}{1.166760in}}%
\pgfpathlineto{\pgfqpoint{1.579667in}{0.949849in}}%
\pgfpathlineto{\pgfqpoint{1.579874in}{1.116963in}}%
\pgfpathlineto{\pgfqpoint{1.580393in}{1.265746in}}%
\pgfpathlineto{\pgfqpoint{1.580807in}{1.136673in}}%
\pgfpathlineto{\pgfqpoint{1.581844in}{0.878674in}}%
\pgfpathlineto{\pgfqpoint{1.582051in}{0.927518in}}%
\pgfpathlineto{\pgfqpoint{1.583087in}{1.172352in}}%
\pgfpathlineto{\pgfqpoint{1.583295in}{1.104045in}}%
\pgfpathlineto{\pgfqpoint{1.583398in}{1.099304in}}%
\pgfpathlineto{\pgfqpoint{1.583502in}{1.030523in}}%
\pgfpathlineto{\pgfqpoint{1.584331in}{1.175862in}}%
\pgfpathlineto{\pgfqpoint{1.584435in}{1.122610in}}%
\pgfpathlineto{\pgfqpoint{1.584642in}{1.164162in}}%
\pgfpathlineto{\pgfqpoint{1.584745in}{1.072543in}}%
\pgfpathlineto{\pgfqpoint{1.584849in}{1.110841in}}%
\pgfpathlineto{\pgfqpoint{1.584953in}{0.959994in}}%
\pgfpathlineto{\pgfqpoint{1.585678in}{1.251033in}}%
\pgfpathlineto{\pgfqpoint{1.585886in}{1.177443in}}%
\pgfpathlineto{\pgfqpoint{1.585989in}{1.238556in}}%
\pgfpathlineto{\pgfqpoint{1.586611in}{1.083500in}}%
\pgfpathlineto{\pgfqpoint{1.586818in}{1.086979in}}%
\pgfpathlineto{\pgfqpoint{1.586922in}{1.131541in}}%
\pgfpathlineto{\pgfqpoint{1.587233in}{0.925465in}}%
\pgfpathlineto{\pgfqpoint{1.587647in}{1.073824in}}%
\pgfpathlineto{\pgfqpoint{1.587958in}{0.940756in}}%
\pgfpathlineto{\pgfqpoint{1.588580in}{1.152518in}}%
\pgfpathlineto{\pgfqpoint{1.588684in}{1.106207in}}%
\pgfpathlineto{\pgfqpoint{1.588787in}{1.203324in}}%
\pgfpathlineto{\pgfqpoint{1.589513in}{1.087034in}}%
\pgfpathlineto{\pgfqpoint{1.589720in}{1.176790in}}%
\pgfpathlineto{\pgfqpoint{1.589927in}{1.015075in}}%
\pgfpathlineto{\pgfqpoint{1.590238in}{1.225285in}}%
\pgfpathlineto{\pgfqpoint{1.590860in}{1.061538in}}%
\pgfpathlineto{\pgfqpoint{1.591378in}{1.156742in}}%
\pgfpathlineto{\pgfqpoint{1.591275in}{0.995230in}}%
\pgfpathlineto{\pgfqpoint{1.592000in}{1.138504in}}%
\pgfpathlineto{\pgfqpoint{1.592208in}{1.183481in}}%
\pgfpathlineto{\pgfqpoint{1.592518in}{1.059243in}}%
\pgfpathlineto{\pgfqpoint{1.592622in}{0.948729in}}%
\pgfpathlineto{\pgfqpoint{1.593244in}{1.177739in}}%
\pgfpathlineto{\pgfqpoint{1.593555in}{1.077437in}}%
\pgfpathlineto{\pgfqpoint{1.593659in}{1.140678in}}%
\pgfpathlineto{\pgfqpoint{1.594488in}{0.975526in}}%
\pgfpathlineto{\pgfqpoint{1.594695in}{0.983726in}}%
\pgfpathlineto{\pgfqpoint{1.595731in}{1.149849in}}%
\pgfpathlineto{\pgfqpoint{1.595939in}{1.121356in}}%
\pgfpathlineto{\pgfqpoint{1.596664in}{0.994894in}}%
\pgfpathlineto{\pgfqpoint{1.596560in}{1.165047in}}%
\pgfpathlineto{\pgfqpoint{1.596975in}{1.073760in}}%
\pgfpathlineto{\pgfqpoint{1.597079in}{1.174872in}}%
\pgfpathlineto{\pgfqpoint{1.597390in}{0.891572in}}%
\pgfpathlineto{\pgfqpoint{1.598011in}{1.147884in}}%
\pgfpathlineto{\pgfqpoint{1.598426in}{0.959024in}}%
\pgfpathlineto{\pgfqpoint{1.599255in}{1.057786in}}%
\pgfpathlineto{\pgfqpoint{1.599566in}{0.970231in}}%
\pgfpathlineto{\pgfqpoint{1.600602in}{1.239222in}}%
\pgfpathlineto{\pgfqpoint{1.601121in}{1.015923in}}%
\pgfpathlineto{\pgfqpoint{1.601742in}{1.191146in}}%
\pgfpathlineto{\pgfqpoint{1.602468in}{1.379203in}}%
\pgfpathlineto{\pgfqpoint{1.602053in}{1.119403in}}%
\pgfpathlineto{\pgfqpoint{1.602675in}{1.164346in}}%
\pgfpathlineto{\pgfqpoint{1.603297in}{1.268546in}}%
\pgfpathlineto{\pgfqpoint{1.603815in}{1.090083in}}%
\pgfpathlineto{\pgfqpoint{1.604023in}{1.071485in}}%
\pgfpathlineto{\pgfqpoint{1.604955in}{1.221300in}}%
\pgfpathlineto{\pgfqpoint{1.605163in}{1.012773in}}%
\pgfpathlineto{\pgfqpoint{1.605784in}{1.250071in}}%
\pgfpathlineto{\pgfqpoint{1.606199in}{1.100922in}}%
\pgfpathlineto{\pgfqpoint{1.606406in}{1.052152in}}%
\pgfpathlineto{\pgfqpoint{1.606510in}{1.123454in}}%
\pgfpathlineto{\pgfqpoint{1.607132in}{1.290074in}}%
\pgfpathlineto{\pgfqpoint{1.606924in}{1.091575in}}%
\pgfpathlineto{\pgfqpoint{1.607546in}{1.104132in}}%
\pgfpathlineto{\pgfqpoint{1.608583in}{0.932187in}}%
\pgfpathlineto{\pgfqpoint{1.608375in}{1.130154in}}%
\pgfpathlineto{\pgfqpoint{1.608686in}{1.014889in}}%
\pgfpathlineto{\pgfqpoint{1.609723in}{1.183131in}}%
\pgfpathlineto{\pgfqpoint{1.609205in}{0.926291in}}%
\pgfpathlineto{\pgfqpoint{1.609826in}{1.141869in}}%
\pgfpathlineto{\pgfqpoint{1.610448in}{0.930915in}}%
\pgfpathlineto{\pgfqpoint{1.610863in}{1.122788in}}%
\pgfpathlineto{\pgfqpoint{1.610966in}{1.201853in}}%
\pgfpathlineto{\pgfqpoint{1.611070in}{0.951340in}}%
\pgfpathlineto{\pgfqpoint{1.611692in}{1.048518in}}%
\pgfpathlineto{\pgfqpoint{1.611796in}{0.933253in}}%
\pgfpathlineto{\pgfqpoint{1.612106in}{1.100861in}}%
\pgfpathlineto{\pgfqpoint{1.612728in}{1.055762in}}%
\pgfpathlineto{\pgfqpoint{1.613557in}{1.285936in}}%
\pgfpathlineto{\pgfqpoint{1.613039in}{0.991230in}}%
\pgfpathlineto{\pgfqpoint{1.613868in}{1.097289in}}%
\pgfpathlineto{\pgfqpoint{1.614283in}{1.175676in}}%
\pgfpathlineto{\pgfqpoint{1.614076in}{1.016353in}}%
\pgfpathlineto{\pgfqpoint{1.614490in}{1.082153in}}%
\pgfpathlineto{\pgfqpoint{1.614594in}{0.973971in}}%
\pgfpathlineto{\pgfqpoint{1.615527in}{1.108614in}}%
\pgfpathlineto{\pgfqpoint{1.616148in}{1.271379in}}%
\pgfpathlineto{\pgfqpoint{1.616356in}{1.086780in}}%
\pgfpathlineto{\pgfqpoint{1.616667in}{1.172939in}}%
\pgfpathlineto{\pgfqpoint{1.617185in}{1.222422in}}%
\pgfpathlineto{\pgfqpoint{1.617081in}{1.118439in}}%
\pgfpathlineto{\pgfqpoint{1.617392in}{1.139357in}}%
\pgfpathlineto{\pgfqpoint{1.618014in}{0.999227in}}%
\pgfpathlineto{\pgfqpoint{1.618221in}{1.200577in}}%
\pgfpathlineto{\pgfqpoint{1.618325in}{1.081547in}}%
\pgfpathlineto{\pgfqpoint{1.618428in}{1.215750in}}%
\pgfpathlineto{\pgfqpoint{1.619361in}{0.994587in}}%
\pgfpathlineto{\pgfqpoint{1.619465in}{0.950689in}}%
\pgfpathlineto{\pgfqpoint{1.619776in}{1.092759in}}%
\pgfpathlineto{\pgfqpoint{1.619879in}{1.060187in}}%
\pgfpathlineto{\pgfqpoint{1.619983in}{1.244651in}}%
\pgfpathlineto{\pgfqpoint{1.620709in}{1.055945in}}%
\pgfpathlineto{\pgfqpoint{1.621019in}{1.152565in}}%
\pgfpathlineto{\pgfqpoint{1.622056in}{1.011326in}}%
\pgfpathlineto{\pgfqpoint{1.621538in}{1.231595in}}%
\pgfpathlineto{\pgfqpoint{1.622367in}{1.077660in}}%
\pgfpathlineto{\pgfqpoint{1.622470in}{1.099454in}}%
\pgfpathlineto{\pgfqpoint{1.622574in}{1.019648in}}%
\pgfpathlineto{\pgfqpoint{1.623196in}{1.081615in}}%
\pgfpathlineto{\pgfqpoint{1.623921in}{0.953643in}}%
\pgfpathlineto{\pgfqpoint{1.624129in}{1.143695in}}%
\pgfpathlineto{\pgfqpoint{1.624232in}{1.012904in}}%
\pgfpathlineto{\pgfqpoint{1.624751in}{1.140197in}}%
\pgfpathlineto{\pgfqpoint{1.625165in}{0.944645in}}%
\pgfpathlineto{\pgfqpoint{1.625372in}{1.069873in}}%
\pgfpathlineto{\pgfqpoint{1.625476in}{1.103037in}}%
\pgfpathlineto{\pgfqpoint{1.626201in}{0.985188in}}%
\pgfpathlineto{\pgfqpoint{1.626305in}{1.046536in}}%
\pgfpathlineto{\pgfqpoint{1.627134in}{0.947343in}}%
\pgfpathlineto{\pgfqpoint{1.626616in}{1.128292in}}%
\pgfpathlineto{\pgfqpoint{1.627342in}{1.036602in}}%
\pgfpathlineto{\pgfqpoint{1.627549in}{1.019550in}}%
\pgfpathlineto{\pgfqpoint{1.627860in}{1.113462in}}%
\pgfpathlineto{\pgfqpoint{1.628067in}{0.749224in}}%
\pgfpathlineto{\pgfqpoint{1.628896in}{1.077696in}}%
\pgfpathlineto{\pgfqpoint{1.629103in}{1.022261in}}%
\pgfpathlineto{\pgfqpoint{1.629207in}{1.106102in}}%
\pgfpathlineto{\pgfqpoint{1.629518in}{0.934311in}}%
\pgfpathlineto{\pgfqpoint{1.629829in}{1.135654in}}%
\pgfpathlineto{\pgfqpoint{1.630347in}{1.060904in}}%
\pgfpathlineto{\pgfqpoint{1.631073in}{0.970038in}}%
\pgfpathlineto{\pgfqpoint{1.630658in}{1.089082in}}%
\pgfpathlineto{\pgfqpoint{1.631280in}{1.086746in}}%
\pgfpathlineto{\pgfqpoint{1.631383in}{1.225099in}}%
\pgfpathlineto{\pgfqpoint{1.632005in}{0.957283in}}%
\pgfpathlineto{\pgfqpoint{1.632213in}{0.988019in}}%
\pgfpathlineto{\pgfqpoint{1.632316in}{0.967016in}}%
\pgfpathlineto{\pgfqpoint{1.632524in}{1.102363in}}%
\pgfpathlineto{\pgfqpoint{1.633042in}{0.986544in}}%
\pgfpathlineto{\pgfqpoint{1.633353in}{1.102398in}}%
\pgfpathlineto{\pgfqpoint{1.633871in}{0.873946in}}%
\pgfpathlineto{\pgfqpoint{1.634078in}{0.986044in}}%
\pgfpathlineto{\pgfqpoint{1.634389in}{0.890179in}}%
\pgfpathlineto{\pgfqpoint{1.634804in}{1.098044in}}%
\pgfpathlineto{\pgfqpoint{1.635011in}{1.156111in}}%
\pgfpathlineto{\pgfqpoint{1.635218in}{0.966983in}}%
\pgfpathlineto{\pgfqpoint{1.635633in}{0.930686in}}%
\pgfpathlineto{\pgfqpoint{1.635425in}{1.071936in}}%
\pgfpathlineto{\pgfqpoint{1.636151in}{0.951384in}}%
\pgfpathlineto{\pgfqpoint{1.637291in}{1.144869in}}%
\pgfpathlineto{\pgfqpoint{1.637395in}{1.116288in}}%
\pgfpathlineto{\pgfqpoint{1.637498in}{1.111789in}}%
\pgfpathlineto{\pgfqpoint{1.637602in}{1.119614in}}%
\pgfpathlineto{\pgfqpoint{1.638535in}{0.972199in}}%
\pgfpathlineto{\pgfqpoint{1.638638in}{0.991447in}}%
\pgfpathlineto{\pgfqpoint{1.639053in}{1.194810in}}%
\pgfpathlineto{\pgfqpoint{1.639260in}{0.985390in}}%
\pgfpathlineto{\pgfqpoint{1.639778in}{1.060094in}}%
\pgfpathlineto{\pgfqpoint{1.640607in}{1.120527in}}%
\pgfpathlineto{\pgfqpoint{1.640400in}{0.964026in}}%
\pgfpathlineto{\pgfqpoint{1.640815in}{1.047206in}}%
\pgfpathlineto{\pgfqpoint{1.640918in}{1.043902in}}%
\pgfpathlineto{\pgfqpoint{1.641126in}{0.993470in}}%
\pgfpathlineto{\pgfqpoint{1.641540in}{1.163687in}}%
\pgfpathlineto{\pgfqpoint{1.641851in}{1.065495in}}%
\pgfpathlineto{\pgfqpoint{1.642058in}{1.017177in}}%
\pgfpathlineto{\pgfqpoint{1.643198in}{1.275941in}}%
\pgfpathlineto{\pgfqpoint{1.643820in}{1.019183in}}%
\pgfpathlineto{\pgfqpoint{1.644442in}{1.126768in}}%
\pgfpathlineto{\pgfqpoint{1.644753in}{1.036907in}}%
\pgfpathlineto{\pgfqpoint{1.645375in}{1.158617in}}%
\pgfpathlineto{\pgfqpoint{1.646308in}{0.932238in}}%
\pgfpathlineto{\pgfqpoint{1.645997in}{1.280922in}}%
\pgfpathlineto{\pgfqpoint{1.646515in}{1.007701in}}%
\pgfpathlineto{\pgfqpoint{1.646722in}{1.145383in}}%
\pgfpathlineto{\pgfqpoint{1.647655in}{1.118671in}}%
\pgfpathlineto{\pgfqpoint{1.647862in}{1.006021in}}%
\pgfpathlineto{\pgfqpoint{1.648173in}{1.174270in}}%
\pgfpathlineto{\pgfqpoint{1.648795in}{1.010020in}}%
\pgfpathlineto{\pgfqpoint{1.649417in}{1.245045in}}%
\pgfpathlineto{\pgfqpoint{1.649728in}{0.974863in}}%
\pgfpathlineto{\pgfqpoint{1.649831in}{1.010447in}}%
\pgfpathlineto{\pgfqpoint{1.649935in}{1.007927in}}%
\pgfpathlineto{\pgfqpoint{1.650142in}{1.082884in}}%
\pgfpathlineto{\pgfqpoint{1.650557in}{0.906788in}}%
\pgfpathlineto{\pgfqpoint{1.651075in}{1.076624in}}%
\pgfpathlineto{\pgfqpoint{1.651179in}{0.951252in}}%
\pgfpathlineto{\pgfqpoint{1.651386in}{1.141638in}}%
\pgfpathlineto{\pgfqpoint{1.652215in}{1.014769in}}%
\pgfpathlineto{\pgfqpoint{1.653044in}{1.137696in}}%
\pgfpathlineto{\pgfqpoint{1.652733in}{0.978496in}}%
\pgfpathlineto{\pgfqpoint{1.653355in}{1.090461in}}%
\pgfpathlineto{\pgfqpoint{1.653562in}{0.992001in}}%
\pgfpathlineto{\pgfqpoint{1.653873in}{1.150405in}}%
\pgfpathlineto{\pgfqpoint{1.654495in}{1.269093in}}%
\pgfpathlineto{\pgfqpoint{1.654702in}{1.172392in}}%
\pgfpathlineto{\pgfqpoint{1.655428in}{1.001424in}}%
\pgfpathlineto{\pgfqpoint{1.655117in}{1.201002in}}%
\pgfpathlineto{\pgfqpoint{1.655843in}{1.103663in}}%
\pgfpathlineto{\pgfqpoint{1.655946in}{1.226108in}}%
\pgfpathlineto{\pgfqpoint{1.656464in}{1.060666in}}%
\pgfpathlineto{\pgfqpoint{1.656983in}{1.158827in}}%
\pgfpathlineto{\pgfqpoint{1.658019in}{0.931814in}}%
\pgfpathlineto{\pgfqpoint{1.658123in}{0.996986in}}%
\pgfpathlineto{\pgfqpoint{1.658434in}{1.183461in}}%
\pgfpathlineto{\pgfqpoint{1.659159in}{1.030399in}}%
\pgfpathlineto{\pgfqpoint{1.659263in}{0.970892in}}%
\pgfpathlineto{\pgfqpoint{1.659574in}{1.137727in}}%
\pgfpathlineto{\pgfqpoint{1.660092in}{0.977063in}}%
\pgfpathlineto{\pgfqpoint{1.661128in}{1.170128in}}%
\pgfpathlineto{\pgfqpoint{1.660921in}{0.975538in}}%
\pgfpathlineto{\pgfqpoint{1.661232in}{1.087753in}}%
\pgfpathlineto{\pgfqpoint{1.661439in}{0.879817in}}%
\pgfpathlineto{\pgfqpoint{1.662268in}{1.130885in}}%
\pgfpathlineto{\pgfqpoint{1.662683in}{0.940889in}}%
\pgfpathlineto{\pgfqpoint{1.663305in}{1.014981in}}%
\pgfpathlineto{\pgfqpoint{1.664341in}{1.188545in}}%
\pgfpathlineto{\pgfqpoint{1.664237in}{0.974084in}}%
\pgfpathlineto{\pgfqpoint{1.664445in}{1.080090in}}%
\pgfpathlineto{\pgfqpoint{1.664548in}{1.128912in}}%
\pgfpathlineto{\pgfqpoint{1.665066in}{0.873248in}}%
\pgfpathlineto{\pgfqpoint{1.665170in}{0.977032in}}%
\pgfpathlineto{\pgfqpoint{1.666103in}{0.781182in}}%
\pgfpathlineto{\pgfqpoint{1.665377in}{0.987742in}}%
\pgfpathlineto{\pgfqpoint{1.666414in}{0.879225in}}%
\pgfpathlineto{\pgfqpoint{1.666932in}{1.006443in}}%
\pgfpathlineto{\pgfqpoint{1.667243in}{0.817508in}}%
\pgfpathlineto{\pgfqpoint{1.667450in}{0.894333in}}%
\pgfpathlineto{\pgfqpoint{1.667761in}{0.791392in}}%
\pgfpathlineto{\pgfqpoint{1.667968in}{0.950537in}}%
\pgfpathlineto{\pgfqpoint{1.668279in}{0.915213in}}%
\pgfpathlineto{\pgfqpoint{1.668383in}{0.994927in}}%
\pgfpathlineto{\pgfqpoint{1.668901in}{0.755325in}}%
\pgfpathlineto{\pgfqpoint{1.669316in}{0.877111in}}%
\pgfpathlineto{\pgfqpoint{1.669419in}{0.875862in}}%
\pgfpathlineto{\pgfqpoint{1.670456in}{1.119484in}}%
\pgfpathlineto{\pgfqpoint{1.670559in}{0.968907in}}%
\pgfpathlineto{\pgfqpoint{1.671078in}{1.219150in}}%
\pgfpathlineto{\pgfqpoint{1.671803in}{1.112462in}}%
\pgfpathlineto{\pgfqpoint{1.672010in}{1.041436in}}%
\pgfpathlineto{\pgfqpoint{1.672425in}{1.187723in}}%
\pgfpathlineto{\pgfqpoint{1.672736in}{1.134547in}}%
\pgfpathlineto{\pgfqpoint{1.673254in}{1.247539in}}%
\pgfpathlineto{\pgfqpoint{1.673358in}{1.076915in}}%
\pgfpathlineto{\pgfqpoint{1.673461in}{1.107717in}}%
\pgfpathlineto{\pgfqpoint{1.674394in}{0.867458in}}%
\pgfpathlineto{\pgfqpoint{1.673876in}{1.136723in}}%
\pgfpathlineto{\pgfqpoint{1.674601in}{1.062546in}}%
\pgfpathlineto{\pgfqpoint{1.675223in}{1.139544in}}%
\pgfpathlineto{\pgfqpoint{1.674912in}{0.930417in}}%
\pgfpathlineto{\pgfqpoint{1.675534in}{1.031174in}}%
\pgfpathlineto{\pgfqpoint{1.676156in}{0.868709in}}%
\pgfpathlineto{\pgfqpoint{1.675741in}{1.088876in}}%
\pgfpathlineto{\pgfqpoint{1.676674in}{1.014207in}}%
\pgfpathlineto{\pgfqpoint{1.677192in}{1.143281in}}%
\pgfpathlineto{\pgfqpoint{1.677400in}{0.961409in}}%
\pgfpathlineto{\pgfqpoint{1.677711in}{1.062995in}}%
\pgfpathlineto{\pgfqpoint{1.678332in}{0.957163in}}%
\pgfpathlineto{\pgfqpoint{1.678021in}{1.106705in}}%
\pgfpathlineto{\pgfqpoint{1.678851in}{1.023852in}}%
\pgfpathlineto{\pgfqpoint{1.679265in}{1.111900in}}%
\pgfpathlineto{\pgfqpoint{1.679887in}{1.047814in}}%
\pgfpathlineto{\pgfqpoint{1.679991in}{1.029183in}}%
\pgfpathlineto{\pgfqpoint{1.680198in}{1.179167in}}%
\pgfpathlineto{\pgfqpoint{1.680509in}{1.114866in}}%
\pgfpathlineto{\pgfqpoint{1.681131in}{1.199387in}}%
\pgfpathlineto{\pgfqpoint{1.681649in}{1.136276in}}%
\pgfpathlineto{\pgfqpoint{1.682582in}{0.862473in}}%
\pgfpathlineto{\pgfqpoint{1.682789in}{1.103942in}}%
\pgfpathlineto{\pgfqpoint{1.682893in}{1.097094in}}%
\pgfpathlineto{\pgfqpoint{1.683203in}{1.210957in}}%
\pgfpathlineto{\pgfqpoint{1.683514in}{1.012773in}}%
\pgfpathlineto{\pgfqpoint{1.683929in}{1.119581in}}%
\pgfpathlineto{\pgfqpoint{1.684758in}{1.016108in}}%
\pgfpathlineto{\pgfqpoint{1.684654in}{1.179377in}}%
\pgfpathlineto{\pgfqpoint{1.684862in}{1.137948in}}%
\pgfpathlineto{\pgfqpoint{1.684965in}{1.247971in}}%
\pgfpathlineto{\pgfqpoint{1.685587in}{1.066764in}}%
\pgfpathlineto{\pgfqpoint{1.686002in}{1.217168in}}%
\pgfpathlineto{\pgfqpoint{1.687245in}{0.955277in}}%
\pgfpathlineto{\pgfqpoint{1.686313in}{1.242876in}}%
\pgfpathlineto{\pgfqpoint{1.687349in}{0.985558in}}%
\pgfpathlineto{\pgfqpoint{1.687453in}{0.961307in}}%
\pgfpathlineto{\pgfqpoint{1.687764in}{1.091209in}}%
\pgfpathlineto{\pgfqpoint{1.688696in}{1.250527in}}%
\pgfpathlineto{\pgfqpoint{1.688489in}{1.003602in}}%
\pgfpathlineto{\pgfqpoint{1.688800in}{1.196115in}}%
\pgfpathlineto{\pgfqpoint{1.690044in}{0.952831in}}%
\pgfpathlineto{\pgfqpoint{1.691287in}{1.250826in}}%
\pgfpathlineto{\pgfqpoint{1.692324in}{0.965759in}}%
\pgfpathlineto{\pgfqpoint{1.692946in}{1.061510in}}%
\pgfpathlineto{\pgfqpoint{1.693049in}{1.183571in}}%
\pgfpathlineto{\pgfqpoint{1.693257in}{0.987423in}}%
\pgfpathlineto{\pgfqpoint{1.693982in}{1.130857in}}%
\pgfpathlineto{\pgfqpoint{1.694708in}{1.046052in}}%
\pgfpathlineto{\pgfqpoint{1.694189in}{1.166590in}}%
\pgfpathlineto{\pgfqpoint{1.695018in}{1.161780in}}%
\pgfpathlineto{\pgfqpoint{1.695848in}{1.080293in}}%
\pgfpathlineto{\pgfqpoint{1.695433in}{1.205833in}}%
\pgfpathlineto{\pgfqpoint{1.696159in}{1.090809in}}%
\pgfpathlineto{\pgfqpoint{1.697091in}{1.028864in}}%
\pgfpathlineto{\pgfqpoint{1.697299in}{1.186376in}}%
\pgfpathlineto{\pgfqpoint{1.697817in}{1.042430in}}%
\pgfpathlineto{\pgfqpoint{1.698439in}{1.057082in}}%
\pgfpathlineto{\pgfqpoint{1.699268in}{1.244500in}}%
\pgfpathlineto{\pgfqpoint{1.698646in}{1.036040in}}%
\pgfpathlineto{\pgfqpoint{1.699579in}{1.092081in}}%
\pgfpathlineto{\pgfqpoint{1.700200in}{0.978329in}}%
\pgfpathlineto{\pgfqpoint{1.700408in}{1.155386in}}%
\pgfpathlineto{\pgfqpoint{1.700511in}{1.176591in}}%
\pgfpathlineto{\pgfqpoint{1.700926in}{1.041997in}}%
\pgfpathlineto{\pgfqpoint{1.701030in}{1.099584in}}%
\pgfpathlineto{\pgfqpoint{1.701237in}{0.985333in}}%
\pgfpathlineto{\pgfqpoint{1.701651in}{1.218988in}}%
\pgfpathlineto{\pgfqpoint{1.702066in}{1.159180in}}%
\pgfpathlineto{\pgfqpoint{1.703310in}{0.918965in}}%
\pgfpathlineto{\pgfqpoint{1.702377in}{1.185208in}}%
\pgfpathlineto{\pgfqpoint{1.703621in}{1.065905in}}%
\pgfpathlineto{\pgfqpoint{1.703828in}{1.201887in}}%
\pgfpathlineto{\pgfqpoint{1.704450in}{0.997017in}}%
\pgfpathlineto{\pgfqpoint{1.704657in}{1.010889in}}%
\pgfpathlineto{\pgfqpoint{1.705072in}{0.949332in}}%
\pgfpathlineto{\pgfqpoint{1.705279in}{1.091796in}}%
\pgfpathlineto{\pgfqpoint{1.705382in}{0.988371in}}%
\pgfpathlineto{\pgfqpoint{1.705693in}{1.164072in}}%
\pgfpathlineto{\pgfqpoint{1.706004in}{0.950109in}}%
\pgfpathlineto{\pgfqpoint{1.706419in}{1.003596in}}%
\pgfpathlineto{\pgfqpoint{1.706833in}{1.157157in}}%
\pgfpathlineto{\pgfqpoint{1.707352in}{0.942645in}}%
\pgfpathlineto{\pgfqpoint{1.707766in}{1.085080in}}%
\pgfpathlineto{\pgfqpoint{1.708181in}{0.823341in}}%
\pgfpathlineto{\pgfqpoint{1.708492in}{0.998469in}}%
\pgfpathlineto{\pgfqpoint{1.708803in}{1.058872in}}%
\pgfpathlineto{\pgfqpoint{1.709217in}{0.835371in}}%
\pgfpathlineto{\pgfqpoint{1.709321in}{0.974642in}}%
\pgfpathlineto{\pgfqpoint{1.709424in}{0.859451in}}%
\pgfpathlineto{\pgfqpoint{1.709839in}{1.044396in}}%
\pgfpathlineto{\pgfqpoint{1.710254in}{0.996096in}}%
\pgfpathlineto{\pgfqpoint{1.710875in}{1.159703in}}%
\pgfpathlineto{\pgfqpoint{1.711186in}{0.929692in}}%
\pgfpathlineto{\pgfqpoint{1.711290in}{1.006925in}}%
\pgfpathlineto{\pgfqpoint{1.712119in}{1.157342in}}%
\pgfpathlineto{\pgfqpoint{1.712534in}{0.861109in}}%
\pgfpathlineto{\pgfqpoint{1.713777in}{1.105349in}}%
\pgfpathlineto{\pgfqpoint{1.714606in}{0.888626in}}%
\pgfpathlineto{\pgfqpoint{1.714088in}{1.153025in}}%
\pgfpathlineto{\pgfqpoint{1.715125in}{0.994208in}}%
\pgfpathlineto{\pgfqpoint{1.715954in}{1.143042in}}%
\pgfpathlineto{\pgfqpoint{1.716057in}{0.967706in}}%
\pgfpathlineto{\pgfqpoint{1.716265in}{1.046665in}}%
\pgfpathlineto{\pgfqpoint{1.717197in}{1.259583in}}%
\pgfpathlineto{\pgfqpoint{1.717405in}{1.153057in}}%
\pgfpathlineto{\pgfqpoint{1.717923in}{0.966545in}}%
\pgfpathlineto{\pgfqpoint{1.718648in}{1.003262in}}%
\pgfpathlineto{\pgfqpoint{1.718856in}{1.180504in}}%
\pgfpathlineto{\pgfqpoint{1.719167in}{0.982856in}}%
\pgfpathlineto{\pgfqpoint{1.719478in}{1.024134in}}%
\pgfpathlineto{\pgfqpoint{1.719581in}{0.924435in}}%
\pgfpathlineto{\pgfqpoint{1.719892in}{1.139143in}}%
\pgfpathlineto{\pgfqpoint{1.720410in}{1.097236in}}%
\pgfpathlineto{\pgfqpoint{1.721032in}{1.157737in}}%
\pgfpathlineto{\pgfqpoint{1.720618in}{0.989360in}}%
\pgfpathlineto{\pgfqpoint{1.721447in}{1.103874in}}%
\pgfpathlineto{\pgfqpoint{1.721758in}{0.936064in}}%
\pgfpathlineto{\pgfqpoint{1.722587in}{1.029149in}}%
\pgfpathlineto{\pgfqpoint{1.723623in}{1.097557in}}%
\pgfpathlineto{\pgfqpoint{1.723001in}{0.877932in}}%
\pgfpathlineto{\pgfqpoint{1.723727in}{1.052589in}}%
\pgfpathlineto{\pgfqpoint{1.724141in}{1.223968in}}%
\pgfpathlineto{\pgfqpoint{1.724349in}{0.992245in}}%
\pgfpathlineto{\pgfqpoint{1.724867in}{1.135368in}}%
\pgfpathlineto{\pgfqpoint{1.726007in}{0.910379in}}%
\pgfpathlineto{\pgfqpoint{1.725592in}{1.220471in}}%
\pgfpathlineto{\pgfqpoint{1.726110in}{0.975502in}}%
\pgfpathlineto{\pgfqpoint{1.726318in}{1.130898in}}%
\pgfpathlineto{\pgfqpoint{1.726629in}{0.925684in}}%
\pgfpathlineto{\pgfqpoint{1.727147in}{0.987931in}}%
\pgfpathlineto{\pgfqpoint{1.728080in}{0.658662in}}%
\pgfpathlineto{\pgfqpoint{1.728494in}{0.761234in}}%
\pgfpathlineto{\pgfqpoint{1.729012in}{0.989575in}}%
\pgfpathlineto{\pgfqpoint{1.729738in}{0.966475in}}%
\pgfpathlineto{\pgfqpoint{1.729945in}{0.778001in}}%
\pgfpathlineto{\pgfqpoint{1.730878in}{0.901383in}}%
\pgfpathlineto{\pgfqpoint{1.732122in}{1.278780in}}%
\pgfpathlineto{\pgfqpoint{1.731292in}{0.819069in}}%
\pgfpathlineto{\pgfqpoint{1.732225in}{1.199737in}}%
\pgfpathlineto{\pgfqpoint{1.732847in}{0.921180in}}%
\pgfpathlineto{\pgfqpoint{1.733262in}{1.133885in}}%
\pgfpathlineto{\pgfqpoint{1.733365in}{1.162660in}}%
\pgfpathlineto{\pgfqpoint{1.733573in}{0.973266in}}%
\pgfpathlineto{\pgfqpoint{1.733780in}{0.999966in}}%
\pgfpathlineto{\pgfqpoint{1.734402in}{0.910777in}}%
\pgfpathlineto{\pgfqpoint{1.734194in}{1.090170in}}%
\pgfpathlineto{\pgfqpoint{1.734609in}{0.976598in}}%
\pgfpathlineto{\pgfqpoint{1.735231in}{1.098373in}}%
\pgfpathlineto{\pgfqpoint{1.735645in}{0.973822in}}%
\pgfpathlineto{\pgfqpoint{1.735749in}{1.033112in}}%
\pgfpathlineto{\pgfqpoint{1.735853in}{0.973350in}}%
\pgfpathlineto{\pgfqpoint{1.736682in}{1.012688in}}%
\pgfpathlineto{\pgfqpoint{1.736785in}{1.130256in}}%
\pgfpathlineto{\pgfqpoint{1.737096in}{0.915292in}}%
\pgfpathlineto{\pgfqpoint{1.737718in}{0.942118in}}%
\pgfpathlineto{\pgfqpoint{1.737822in}{0.923536in}}%
\pgfpathlineto{\pgfqpoint{1.737925in}{1.050651in}}%
\pgfpathlineto{\pgfqpoint{1.738236in}{0.974255in}}%
\pgfpathlineto{\pgfqpoint{1.738340in}{1.099441in}}%
\pgfpathlineto{\pgfqpoint{1.739169in}{0.845170in}}%
\pgfpathlineto{\pgfqpoint{1.739273in}{0.933842in}}%
\pgfpathlineto{\pgfqpoint{1.739687in}{1.068537in}}%
\pgfpathlineto{\pgfqpoint{1.740516in}{0.886854in}}%
\pgfpathlineto{\pgfqpoint{1.741553in}{1.021560in}}%
\pgfpathlineto{\pgfqpoint{1.741138in}{0.847662in}}%
\pgfpathlineto{\pgfqpoint{1.741656in}{0.990001in}}%
\pgfpathlineto{\pgfqpoint{1.741864in}{0.896469in}}%
\pgfpathlineto{\pgfqpoint{1.742486in}{1.128425in}}%
\pgfpathlineto{\pgfqpoint{1.742693in}{1.030814in}}%
\pgfpathlineto{\pgfqpoint{1.742797in}{1.082789in}}%
\pgfpathlineto{\pgfqpoint{1.743211in}{0.894699in}}%
\pgfpathlineto{\pgfqpoint{1.743626in}{1.057670in}}%
\pgfpathlineto{\pgfqpoint{1.743833in}{0.920365in}}%
\pgfpathlineto{\pgfqpoint{1.744869in}{0.922335in}}%
\pgfpathlineto{\pgfqpoint{1.745698in}{0.805821in}}%
\pgfpathlineto{\pgfqpoint{1.746217in}{0.866466in}}%
\pgfpathlineto{\pgfqpoint{1.746424in}{0.806183in}}%
\pgfpathlineto{\pgfqpoint{1.747253in}{0.972384in}}%
\pgfpathlineto{\pgfqpoint{1.748082in}{0.800381in}}%
\pgfpathlineto{\pgfqpoint{1.747460in}{0.992562in}}%
\pgfpathlineto{\pgfqpoint{1.748289in}{0.829717in}}%
\pgfpathlineto{\pgfqpoint{1.748393in}{0.983135in}}%
\pgfpathlineto{\pgfqpoint{1.749119in}{0.723281in}}%
\pgfpathlineto{\pgfqpoint{1.749326in}{0.785692in}}%
\pgfpathlineto{\pgfqpoint{1.749429in}{0.760258in}}%
\pgfpathlineto{\pgfqpoint{1.749740in}{0.912867in}}%
\pgfpathlineto{\pgfqpoint{1.750259in}{1.052088in}}%
\pgfpathlineto{\pgfqpoint{1.750155in}{0.877302in}}%
\pgfpathlineto{\pgfqpoint{1.750880in}{0.949071in}}%
\pgfpathlineto{\pgfqpoint{1.751399in}{0.728220in}}%
\pgfpathlineto{\pgfqpoint{1.751917in}{0.853806in}}%
\pgfpathlineto{\pgfqpoint{1.752642in}{1.036843in}}%
\pgfpathlineto{\pgfqpoint{1.752124in}{0.823673in}}%
\pgfpathlineto{\pgfqpoint{1.753264in}{1.016276in}}%
\pgfpathlineto{\pgfqpoint{1.753368in}{1.006601in}}%
\pgfpathlineto{\pgfqpoint{1.753471in}{0.823267in}}%
\pgfpathlineto{\pgfqpoint{1.754093in}{1.083848in}}%
\pgfpathlineto{\pgfqpoint{1.754404in}{1.068336in}}%
\pgfpathlineto{\pgfqpoint{1.754922in}{0.896167in}}%
\pgfpathlineto{\pgfqpoint{1.755337in}{1.070432in}}%
\pgfpathlineto{\pgfqpoint{1.755648in}{0.996231in}}%
\pgfpathlineto{\pgfqpoint{1.755959in}{1.116141in}}%
\pgfpathlineto{\pgfqpoint{1.756270in}{0.969819in}}%
\pgfpathlineto{\pgfqpoint{1.756684in}{0.994842in}}%
\pgfpathlineto{\pgfqpoint{1.756892in}{0.940133in}}%
\pgfpathlineto{\pgfqpoint{1.756995in}{1.032752in}}%
\pgfpathlineto{\pgfqpoint{1.757306in}{0.945331in}}%
\pgfpathlineto{\pgfqpoint{1.757410in}{0.879425in}}%
\pgfpathlineto{\pgfqpoint{1.758135in}{1.119977in}}%
\pgfpathlineto{\pgfqpoint{1.759793in}{0.865443in}}%
\pgfpathlineto{\pgfqpoint{1.760726in}{1.167255in}}%
\pgfpathlineto{\pgfqpoint{1.761244in}{1.003444in}}%
\pgfpathlineto{\pgfqpoint{1.761970in}{0.823509in}}%
\pgfpathlineto{\pgfqpoint{1.762177in}{1.039169in}}%
\pgfpathlineto{\pgfqpoint{1.762384in}{0.959283in}}%
\pgfpathlineto{\pgfqpoint{1.762592in}{1.090048in}}%
\pgfpathlineto{\pgfqpoint{1.763006in}{0.872267in}}%
\pgfpathlineto{\pgfqpoint{1.763317in}{0.941232in}}%
\pgfpathlineto{\pgfqpoint{1.763628in}{0.822553in}}%
\pgfpathlineto{\pgfqpoint{1.763939in}{1.097462in}}%
\pgfpathlineto{\pgfqpoint{1.764665in}{0.822182in}}%
\pgfpathlineto{\pgfqpoint{1.765286in}{0.972378in}}%
\pgfpathlineto{\pgfqpoint{1.765390in}{1.032479in}}%
\pgfpathlineto{\pgfqpoint{1.766116in}{0.903465in}}%
\pgfpathlineto{\pgfqpoint{1.766323in}{1.010551in}}%
\pgfpathlineto{\pgfqpoint{1.766945in}{0.845430in}}%
\pgfpathlineto{\pgfqpoint{1.767463in}{0.967710in}}%
\pgfpathlineto{\pgfqpoint{1.768499in}{1.128289in}}%
\pgfpathlineto{\pgfqpoint{1.767670in}{0.965351in}}%
\pgfpathlineto{\pgfqpoint{1.768603in}{1.012940in}}%
\pgfpathlineto{\pgfqpoint{1.769743in}{0.875323in}}%
\pgfpathlineto{\pgfqpoint{1.769328in}{1.080529in}}%
\pgfpathlineto{\pgfqpoint{1.769847in}{0.890389in}}%
\pgfpathlineto{\pgfqpoint{1.770572in}{1.075873in}}%
\pgfpathlineto{\pgfqpoint{1.770054in}{0.851723in}}%
\pgfpathlineto{\pgfqpoint{1.771194in}{0.966224in}}%
\pgfpathlineto{\pgfqpoint{1.771401in}{0.769861in}}%
\pgfpathlineto{\pgfqpoint{1.772023in}{1.068440in}}%
\pgfpathlineto{\pgfqpoint{1.772541in}{1.172737in}}%
\pgfpathlineto{\pgfqpoint{1.772334in}{0.853146in}}%
\pgfpathlineto{\pgfqpoint{1.772956in}{1.033120in}}%
\pgfpathlineto{\pgfqpoint{1.773370in}{0.968854in}}%
\pgfpathlineto{\pgfqpoint{1.773163in}{1.081489in}}%
\pgfpathlineto{\pgfqpoint{1.773785in}{1.053840in}}%
\pgfpathlineto{\pgfqpoint{1.773889in}{1.102690in}}%
\pgfpathlineto{\pgfqpoint{1.774199in}{0.811130in}}%
\pgfpathlineto{\pgfqpoint{1.774614in}{0.932715in}}%
\pgfpathlineto{\pgfqpoint{1.774718in}{0.883106in}}%
\pgfpathlineto{\pgfqpoint{1.775029in}{1.050144in}}%
\pgfpathlineto{\pgfqpoint{1.775650in}{0.911639in}}%
\pgfpathlineto{\pgfqpoint{1.776272in}{1.047303in}}%
\pgfpathlineto{\pgfqpoint{1.775858in}{0.839287in}}%
\pgfpathlineto{\pgfqpoint{1.776894in}{1.021427in}}%
\pgfpathlineto{\pgfqpoint{1.776998in}{0.962807in}}%
\pgfpathlineto{\pgfqpoint{1.777309in}{1.120569in}}%
\pgfpathlineto{\pgfqpoint{1.777723in}{1.109016in}}%
\pgfpathlineto{\pgfqpoint{1.777930in}{1.182024in}}%
\pgfpathlineto{\pgfqpoint{1.778345in}{1.023986in}}%
\pgfpathlineto{\pgfqpoint{1.778552in}{1.038661in}}%
\pgfpathlineto{\pgfqpoint{1.778863in}{0.839935in}}%
\pgfpathlineto{\pgfqpoint{1.779071in}{1.154802in}}%
\pgfpathlineto{\pgfqpoint{1.779589in}{1.052176in}}%
\pgfpathlineto{\pgfqpoint{1.780003in}{0.952449in}}%
\pgfpathlineto{\pgfqpoint{1.780418in}{1.007865in}}%
\pgfpathlineto{\pgfqpoint{1.781454in}{1.165531in}}%
\pgfpathlineto{\pgfqpoint{1.781558in}{1.109062in}}%
\pgfpathlineto{\pgfqpoint{1.781972in}{1.111411in}}%
\pgfpathlineto{\pgfqpoint{1.782802in}{0.876161in}}%
\pgfpathlineto{\pgfqpoint{1.784149in}{1.254635in}}%
\pgfpathlineto{\pgfqpoint{1.784253in}{1.176400in}}%
\pgfpathlineto{\pgfqpoint{1.784356in}{1.180479in}}%
\pgfpathlineto{\pgfqpoint{1.785393in}{0.956252in}}%
\pgfpathlineto{\pgfqpoint{1.784563in}{1.251074in}}%
\pgfpathlineto{\pgfqpoint{1.785496in}{1.132361in}}%
\pgfpathlineto{\pgfqpoint{1.785600in}{1.130145in}}%
\pgfpathlineto{\pgfqpoint{1.785911in}{1.271938in}}%
\pgfpathlineto{\pgfqpoint{1.786222in}{1.008001in}}%
\pgfpathlineto{\pgfqpoint{1.786740in}{1.044345in}}%
\pgfpathlineto{\pgfqpoint{1.786844in}{0.995733in}}%
\pgfpathlineto{\pgfqpoint{1.787984in}{0.795557in}}%
\pgfpathlineto{\pgfqpoint{1.789227in}{1.083171in}}%
\pgfpathlineto{\pgfqpoint{1.789435in}{1.080322in}}%
\pgfpathlineto{\pgfqpoint{1.790264in}{0.937810in}}%
\pgfpathlineto{\pgfqpoint{1.790056in}{1.111084in}}%
\pgfpathlineto{\pgfqpoint{1.790471in}{0.944700in}}%
\pgfpathlineto{\pgfqpoint{1.790575in}{1.204993in}}%
\pgfpathlineto{\pgfqpoint{1.791611in}{1.156033in}}%
\pgfpathlineto{\pgfqpoint{1.791818in}{0.998198in}}%
\pgfpathlineto{\pgfqpoint{1.792336in}{1.240539in}}%
\pgfpathlineto{\pgfqpoint{1.792751in}{1.140711in}}%
\pgfpathlineto{\pgfqpoint{1.792855in}{1.217723in}}%
\pgfpathlineto{\pgfqpoint{1.793684in}{1.059464in}}%
\pgfpathlineto{\pgfqpoint{1.793787in}{1.174216in}}%
\pgfpathlineto{\pgfqpoint{1.794409in}{0.902102in}}%
\pgfpathlineto{\pgfqpoint{1.794927in}{1.086261in}}%
\pgfpathlineto{\pgfqpoint{1.796275in}{0.879289in}}%
\pgfpathlineto{\pgfqpoint{1.795757in}{1.091974in}}%
\pgfpathlineto{\pgfqpoint{1.796689in}{0.906920in}}%
\pgfpathlineto{\pgfqpoint{1.797933in}{1.086797in}}%
\pgfpathlineto{\pgfqpoint{1.797311in}{0.856302in}}%
\pgfpathlineto{\pgfqpoint{1.798037in}{0.990604in}}%
\pgfpathlineto{\pgfqpoint{1.798140in}{0.861002in}}%
\pgfpathlineto{\pgfqpoint{1.798969in}{1.006705in}}%
\pgfpathlineto{\pgfqpoint{1.799073in}{0.997841in}}%
\pgfpathlineto{\pgfqpoint{1.799177in}{1.031112in}}%
\pgfpathlineto{\pgfqpoint{1.799488in}{0.903852in}}%
\pgfpathlineto{\pgfqpoint{1.799591in}{0.921001in}}%
\pgfpathlineto{\pgfqpoint{1.799695in}{0.716690in}}%
\pgfpathlineto{\pgfqpoint{1.800213in}{0.930375in}}%
\pgfpathlineto{\pgfqpoint{1.800731in}{0.799739in}}%
\pgfpathlineto{\pgfqpoint{1.801975in}{0.987531in}}%
\pgfpathlineto{\pgfqpoint{1.802390in}{0.736832in}}%
\pgfpathlineto{\pgfqpoint{1.802804in}{1.056072in}}%
\pgfpathlineto{\pgfqpoint{1.803115in}{0.929082in}}%
\pgfpathlineto{\pgfqpoint{1.803633in}{1.125001in}}%
\pgfpathlineto{\pgfqpoint{1.803530in}{0.844572in}}%
\pgfpathlineto{\pgfqpoint{1.804359in}{1.009925in}}%
\pgfpathlineto{\pgfqpoint{1.805188in}{0.886728in}}%
\pgfpathlineto{\pgfqpoint{1.804877in}{1.082699in}}%
\pgfpathlineto{\pgfqpoint{1.805499in}{0.939991in}}%
\pgfpathlineto{\pgfqpoint{1.805913in}{0.772182in}}%
\pgfpathlineto{\pgfqpoint{1.806535in}{0.913768in}}%
\pgfpathlineto{\pgfqpoint{1.807261in}{0.902244in}}%
\pgfpathlineto{\pgfqpoint{1.807779in}{1.144817in}}%
\pgfpathlineto{\pgfqpoint{1.808712in}{0.964162in}}%
\pgfpathlineto{\pgfqpoint{1.808401in}{1.166350in}}%
\pgfpathlineto{\pgfqpoint{1.809022in}{1.049415in}}%
\pgfpathlineto{\pgfqpoint{1.809437in}{1.157164in}}%
\pgfpathlineto{\pgfqpoint{1.809748in}{0.901109in}}%
\pgfpathlineto{\pgfqpoint{1.809852in}{0.892699in}}%
\pgfpathlineto{\pgfqpoint{1.809955in}{0.960000in}}%
\pgfpathlineto{\pgfqpoint{1.810266in}{1.036855in}}%
\pgfpathlineto{\pgfqpoint{1.810473in}{0.893540in}}%
\pgfpathlineto{\pgfqpoint{1.810681in}{0.925287in}}%
\pgfpathlineto{\pgfqpoint{1.810784in}{0.842635in}}%
\pgfpathlineto{\pgfqpoint{1.811406in}{1.146948in}}%
\pgfpathlineto{\pgfqpoint{1.811510in}{1.044541in}}%
\pgfpathlineto{\pgfqpoint{1.812028in}{1.179893in}}%
\pgfpathlineto{\pgfqpoint{1.812235in}{1.012067in}}%
\pgfpathlineto{\pgfqpoint{1.812754in}{1.168842in}}%
\pgfpathlineto{\pgfqpoint{1.812857in}{1.200204in}}%
\pgfpathlineto{\pgfqpoint{1.813272in}{1.007039in}}%
\pgfpathlineto{\pgfqpoint{1.813375in}{1.071413in}}%
\pgfpathlineto{\pgfqpoint{1.813790in}{0.934221in}}%
\pgfpathlineto{\pgfqpoint{1.813997in}{1.129947in}}%
\pgfpathlineto{\pgfqpoint{1.814515in}{1.035243in}}%
\pgfpathlineto{\pgfqpoint{1.815448in}{1.136969in}}%
\pgfpathlineto{\pgfqpoint{1.815241in}{1.015347in}}%
\pgfpathlineto{\pgfqpoint{1.815552in}{1.043470in}}%
\pgfpathlineto{\pgfqpoint{1.815863in}{1.038882in}}%
\pgfpathlineto{\pgfqpoint{1.816070in}{1.105180in}}%
\pgfpathlineto{\pgfqpoint{1.817003in}{1.180488in}}%
\pgfpathlineto{\pgfqpoint{1.816795in}{0.959963in}}%
\pgfpathlineto{\pgfqpoint{1.817210in}{1.163812in}}%
\pgfpathlineto{\pgfqpoint{1.817417in}{1.228206in}}%
\pgfpathlineto{\pgfqpoint{1.818350in}{0.935274in}}%
\pgfpathlineto{\pgfqpoint{1.818868in}{1.167030in}}%
\pgfpathlineto{\pgfqpoint{1.818661in}{0.908451in}}%
\pgfpathlineto{\pgfqpoint{1.819697in}{1.116165in}}%
\pgfpathlineto{\pgfqpoint{1.820527in}{0.980228in}}%
\pgfpathlineto{\pgfqpoint{1.820216in}{1.193538in}}%
\pgfpathlineto{\pgfqpoint{1.820630in}{1.182686in}}%
\pgfpathlineto{\pgfqpoint{1.821045in}{1.308982in}}%
\pgfpathlineto{\pgfqpoint{1.820941in}{1.156003in}}%
\pgfpathlineto{\pgfqpoint{1.821563in}{1.160559in}}%
\pgfpathlineto{\pgfqpoint{1.821667in}{1.144167in}}%
\pgfpathlineto{\pgfqpoint{1.821874in}{1.187496in}}%
\pgfpathlineto{\pgfqpoint{1.821977in}{1.163312in}}%
\pgfpathlineto{\pgfqpoint{1.822185in}{0.996007in}}%
\pgfpathlineto{\pgfqpoint{1.823221in}{1.301549in}}%
\pgfpathlineto{\pgfqpoint{1.823325in}{1.087672in}}%
\pgfpathlineto{\pgfqpoint{1.824361in}{1.154085in}}%
\pgfpathlineto{\pgfqpoint{1.824568in}{1.178398in}}%
\pgfpathlineto{\pgfqpoint{1.825812in}{0.946904in}}%
\pgfpathlineto{\pgfqpoint{1.826019in}{0.907992in}}%
\pgfpathlineto{\pgfqpoint{1.826123in}{0.939602in}}%
\pgfpathlineto{\pgfqpoint{1.826745in}{0.927400in}}%
\pgfpathlineto{\pgfqpoint{1.827263in}{1.211088in}}%
\pgfpathlineto{\pgfqpoint{1.827989in}{0.933742in}}%
\pgfpathlineto{\pgfqpoint{1.828403in}{1.096893in}}%
\pgfpathlineto{\pgfqpoint{1.828507in}{1.093074in}}%
\pgfpathlineto{\pgfqpoint{1.828610in}{1.121232in}}%
\pgfpathlineto{\pgfqpoint{1.828714in}{1.037141in}}%
\pgfpathlineto{\pgfqpoint{1.829129in}{1.233528in}}%
\pgfpathlineto{\pgfqpoint{1.829647in}{1.052033in}}%
\pgfpathlineto{\pgfqpoint{1.830061in}{1.181488in}}%
\pgfpathlineto{\pgfqpoint{1.830476in}{0.977066in}}%
\pgfpathlineto{\pgfqpoint{1.830891in}{1.176492in}}%
\pgfpathlineto{\pgfqpoint{1.831201in}{1.042627in}}%
\pgfpathlineto{\pgfqpoint{1.831305in}{1.239950in}}%
\pgfpathlineto{\pgfqpoint{1.832031in}{1.082705in}}%
\pgfpathlineto{\pgfqpoint{1.832860in}{1.171032in}}%
\pgfpathlineto{\pgfqpoint{1.832963in}{0.938347in}}%
\pgfpathlineto{\pgfqpoint{1.833896in}{1.102192in}}%
\pgfpathlineto{\pgfqpoint{1.834000in}{1.211313in}}%
\pgfpathlineto{\pgfqpoint{1.834518in}{1.037022in}}%
\pgfpathlineto{\pgfqpoint{1.834932in}{1.108267in}}%
\pgfpathlineto{\pgfqpoint{1.835347in}{0.912271in}}%
\pgfpathlineto{\pgfqpoint{1.835762in}{1.167275in}}%
\pgfpathlineto{\pgfqpoint{1.836073in}{1.029962in}}%
\pgfpathlineto{\pgfqpoint{1.836591in}{0.884022in}}%
\pgfpathlineto{\pgfqpoint{1.837109in}{1.032593in}}%
\pgfpathlineto{\pgfqpoint{1.837834in}{1.189325in}}%
\pgfpathlineto{\pgfqpoint{1.837420in}{0.966687in}}%
\pgfpathlineto{\pgfqpoint{1.838249in}{1.135014in}}%
\pgfpathlineto{\pgfqpoint{1.839078in}{0.912679in}}%
\pgfpathlineto{\pgfqpoint{1.839285in}{1.067072in}}%
\pgfpathlineto{\pgfqpoint{1.839596in}{0.941449in}}%
\pgfpathlineto{\pgfqpoint{1.840425in}{1.145465in}}%
\pgfpathlineto{\pgfqpoint{1.840633in}{0.988487in}}%
\pgfpathlineto{\pgfqpoint{1.841151in}{1.208522in}}%
\pgfpathlineto{\pgfqpoint{1.841669in}{1.030539in}}%
\pgfpathlineto{\pgfqpoint{1.842395in}{1.159570in}}%
\pgfpathlineto{\pgfqpoint{1.841980in}{0.920009in}}%
\pgfpathlineto{\pgfqpoint{1.842809in}{1.109934in}}%
\pgfpathlineto{\pgfqpoint{1.843431in}{0.888009in}}%
\pgfpathlineto{\pgfqpoint{1.843224in}{1.134000in}}%
\pgfpathlineto{\pgfqpoint{1.843846in}{1.111800in}}%
\pgfpathlineto{\pgfqpoint{1.844053in}{1.039314in}}%
\pgfpathlineto{\pgfqpoint{1.844156in}{0.891430in}}%
\pgfpathlineto{\pgfqpoint{1.844986in}{1.145679in}}%
\pgfpathlineto{\pgfqpoint{1.845193in}{0.991357in}}%
\pgfpathlineto{\pgfqpoint{1.845711in}{1.128606in}}%
\pgfpathlineto{\pgfqpoint{1.845400in}{0.876779in}}%
\pgfpathlineto{\pgfqpoint{1.846126in}{1.004535in}}%
\pgfpathlineto{\pgfqpoint{1.846229in}{0.916349in}}%
\pgfpathlineto{\pgfqpoint{1.847162in}{0.994952in}}%
\pgfpathlineto{\pgfqpoint{1.847887in}{1.118052in}}%
\pgfpathlineto{\pgfqpoint{1.847680in}{0.953947in}}%
\pgfpathlineto{\pgfqpoint{1.848198in}{0.978617in}}%
\pgfpathlineto{\pgfqpoint{1.848302in}{0.978629in}}%
\pgfpathlineto{\pgfqpoint{1.848613in}{0.884727in}}%
\pgfpathlineto{\pgfqpoint{1.848717in}{1.016509in}}%
\pgfpathlineto{\pgfqpoint{1.848820in}{1.040751in}}%
\pgfpathlineto{\pgfqpoint{1.848924in}{0.933384in}}%
\pgfpathlineto{\pgfqpoint{1.849028in}{0.816026in}}%
\pgfpathlineto{\pgfqpoint{1.849338in}{1.085935in}}%
\pgfpathlineto{\pgfqpoint{1.849960in}{0.873765in}}%
\pgfpathlineto{\pgfqpoint{1.850893in}{1.023012in}}%
\pgfpathlineto{\pgfqpoint{1.850271in}{0.841053in}}%
\pgfpathlineto{\pgfqpoint{1.851204in}{0.964127in}}%
\pgfpathlineto{\pgfqpoint{1.851308in}{0.947758in}}%
\pgfpathlineto{\pgfqpoint{1.851411in}{0.973624in}}%
\pgfpathlineto{\pgfqpoint{1.851515in}{1.145138in}}%
\pgfpathlineto{\pgfqpoint{1.851929in}{0.959226in}}%
\pgfpathlineto{\pgfqpoint{1.852551in}{1.099543in}}%
\pgfpathlineto{\pgfqpoint{1.852759in}{0.994545in}}%
\pgfpathlineto{\pgfqpoint{1.853380in}{1.262089in}}%
\pgfpathlineto{\pgfqpoint{1.853484in}{1.182110in}}%
\pgfpathlineto{\pgfqpoint{1.853588in}{1.194234in}}%
\pgfpathlineto{\pgfqpoint{1.854313in}{0.792141in}}%
\pgfpathlineto{\pgfqpoint{1.854831in}{0.942339in}}%
\pgfpathlineto{\pgfqpoint{1.855764in}{1.068280in}}%
\pgfpathlineto{\pgfqpoint{1.855453in}{0.891064in}}%
\pgfpathlineto{\pgfqpoint{1.855868in}{1.047190in}}%
\pgfpathlineto{\pgfqpoint{1.856801in}{0.787247in}}%
\pgfpathlineto{\pgfqpoint{1.856904in}{0.872859in}}%
\pgfpathlineto{\pgfqpoint{1.857319in}{0.738214in}}%
\pgfpathlineto{\pgfqpoint{1.858044in}{1.156396in}}%
\pgfpathlineto{\pgfqpoint{1.859288in}{0.853187in}}%
\pgfpathlineto{\pgfqpoint{1.860532in}{0.995526in}}%
\pgfpathlineto{\pgfqpoint{1.860635in}{0.973713in}}%
\pgfpathlineto{\pgfqpoint{1.860946in}{0.926153in}}%
\pgfpathlineto{\pgfqpoint{1.861464in}{1.017967in}}%
\pgfpathlineto{\pgfqpoint{1.861983in}{1.171995in}}%
\pgfpathlineto{\pgfqpoint{1.861879in}{0.976876in}}%
\pgfpathlineto{\pgfqpoint{1.862501in}{1.001474in}}%
\pgfpathlineto{\pgfqpoint{1.862604in}{1.004709in}}%
\pgfpathlineto{\pgfqpoint{1.863019in}{0.850955in}}%
\pgfpathlineto{\pgfqpoint{1.863744in}{1.132846in}}%
\pgfpathlineto{\pgfqpoint{1.864574in}{1.198969in}}%
\pgfpathlineto{\pgfqpoint{1.864988in}{0.999166in}}%
\pgfpathlineto{\pgfqpoint{1.865092in}{1.150080in}}%
\pgfpathlineto{\pgfqpoint{1.865506in}{0.976049in}}%
\pgfpathlineto{\pgfqpoint{1.866128in}{1.042029in}}%
\pgfpathlineto{\pgfqpoint{1.866854in}{0.993893in}}%
\pgfpathlineto{\pgfqpoint{1.867475in}{1.223440in}}%
\pgfpathlineto{\pgfqpoint{1.868615in}{1.028880in}}%
\pgfpathlineto{\pgfqpoint{1.868305in}{1.236206in}}%
\pgfpathlineto{\pgfqpoint{1.868823in}{1.055854in}}%
\pgfpathlineto{\pgfqpoint{1.869134in}{1.192303in}}%
\pgfpathlineto{\pgfqpoint{1.869652in}{0.999442in}}%
\pgfpathlineto{\pgfqpoint{1.869963in}{1.123691in}}%
\pgfpathlineto{\pgfqpoint{1.870066in}{1.021520in}}%
\pgfpathlineto{\pgfqpoint{1.870274in}{1.195862in}}%
\pgfpathlineto{\pgfqpoint{1.870999in}{1.099261in}}%
\pgfpathlineto{\pgfqpoint{1.871517in}{1.044055in}}%
\pgfpathlineto{\pgfqpoint{1.872036in}{1.158440in}}%
\pgfpathlineto{\pgfqpoint{1.872139in}{1.085521in}}%
\pgfpathlineto{\pgfqpoint{1.872657in}{1.257116in}}%
\pgfpathlineto{\pgfqpoint{1.873072in}{1.113854in}}%
\pgfpathlineto{\pgfqpoint{1.873797in}{1.010442in}}%
\pgfpathlineto{\pgfqpoint{1.874419in}{1.218932in}}%
\pgfpathlineto{\pgfqpoint{1.874730in}{0.955607in}}%
\pgfpathlineto{\pgfqpoint{1.875767in}{1.048147in}}%
\pgfpathlineto{\pgfqpoint{1.876803in}{1.281929in}}%
\pgfpathlineto{\pgfqpoint{1.876907in}{1.218568in}}%
\pgfpathlineto{\pgfqpoint{1.877632in}{0.867433in}}%
\pgfpathlineto{\pgfqpoint{1.878150in}{0.914165in}}%
\pgfpathlineto{\pgfqpoint{1.878358in}{1.132938in}}%
\pgfpathlineto{\pgfqpoint{1.879394in}{1.129849in}}%
\pgfpathlineto{\pgfqpoint{1.879705in}{1.017191in}}%
\pgfpathlineto{\pgfqpoint{1.880327in}{1.182011in}}%
\pgfpathlineto{\pgfqpoint{1.880430in}{1.180005in}}%
\pgfpathlineto{\pgfqpoint{1.880741in}{1.247881in}}%
\pgfpathlineto{\pgfqpoint{1.880845in}{1.114047in}}%
\pgfpathlineto{\pgfqpoint{1.881156in}{0.940158in}}%
\pgfpathlineto{\pgfqpoint{1.881363in}{1.162737in}}%
\pgfpathlineto{\pgfqpoint{1.881881in}{1.091041in}}%
\pgfpathlineto{\pgfqpoint{1.882503in}{1.191904in}}%
\pgfpathlineto{\pgfqpoint{1.882607in}{1.052305in}}%
\pgfpathlineto{\pgfqpoint{1.883125in}{1.140425in}}%
\pgfpathlineto{\pgfqpoint{1.884058in}{1.039697in}}%
\pgfpathlineto{\pgfqpoint{1.883643in}{1.222580in}}%
\pgfpathlineto{\pgfqpoint{1.884161in}{1.081647in}}%
\pgfpathlineto{\pgfqpoint{1.884576in}{1.382260in}}%
\pgfpathlineto{\pgfqpoint{1.885405in}{1.287879in}}%
\pgfpathlineto{\pgfqpoint{1.886545in}{1.024351in}}%
\pgfpathlineto{\pgfqpoint{1.886856in}{1.122877in}}%
\pgfpathlineto{\pgfqpoint{1.887167in}{0.919655in}}%
\pgfpathlineto{\pgfqpoint{1.887685in}{1.095821in}}%
\pgfpathlineto{\pgfqpoint{1.888100in}{0.995073in}}%
\pgfpathlineto{\pgfqpoint{1.888203in}{1.176183in}}%
\pgfpathlineto{\pgfqpoint{1.888618in}{1.118051in}}%
\pgfpathlineto{\pgfqpoint{1.888722in}{1.154591in}}%
\pgfpathlineto{\pgfqpoint{1.889240in}{1.023491in}}%
\pgfpathlineto{\pgfqpoint{1.889343in}{0.957736in}}%
\pgfpathlineto{\pgfqpoint{1.889758in}{1.227893in}}%
\pgfpathlineto{\pgfqpoint{1.890173in}{1.128303in}}%
\pgfpathlineto{\pgfqpoint{1.890276in}{1.127265in}}%
\pgfpathlineto{\pgfqpoint{1.890794in}{1.067503in}}%
\pgfpathlineto{\pgfqpoint{1.890484in}{1.198470in}}%
\pgfpathlineto{\pgfqpoint{1.890898in}{1.138848in}}%
\pgfpathlineto{\pgfqpoint{1.891313in}{1.326928in}}%
\pgfpathlineto{\pgfqpoint{1.891624in}{1.097697in}}%
\pgfpathlineto{\pgfqpoint{1.892142in}{1.292258in}}%
\pgfpathlineto{\pgfqpoint{1.893075in}{0.971131in}}%
\pgfpathlineto{\pgfqpoint{1.893489in}{1.069669in}}%
\pgfpathlineto{\pgfqpoint{1.893696in}{1.147262in}}%
\pgfpathlineto{\pgfqpoint{1.893904in}{0.947623in}}%
\pgfpathlineto{\pgfqpoint{1.894525in}{1.052768in}}%
\pgfpathlineto{\pgfqpoint{1.895666in}{0.853193in}}%
\pgfpathlineto{\pgfqpoint{1.894940in}{1.106831in}}%
\pgfpathlineto{\pgfqpoint{1.895769in}{0.931838in}}%
\pgfpathlineto{\pgfqpoint{1.896598in}{1.147009in}}%
\pgfpathlineto{\pgfqpoint{1.896909in}{1.133228in}}%
\pgfpathlineto{\pgfqpoint{1.897013in}{0.989777in}}%
\pgfpathlineto{\pgfqpoint{1.897427in}{1.138185in}}%
\pgfpathlineto{\pgfqpoint{1.898049in}{1.052094in}}%
\pgfpathlineto{\pgfqpoint{1.898671in}{1.227147in}}%
\pgfpathlineto{\pgfqpoint{1.898257in}{0.942647in}}%
\pgfpathlineto{\pgfqpoint{1.899086in}{1.066676in}}%
\pgfpathlineto{\pgfqpoint{1.899293in}{0.965553in}}%
\pgfpathlineto{\pgfqpoint{1.899604in}{1.105824in}}%
\pgfpathlineto{\pgfqpoint{1.899811in}{1.233577in}}%
\pgfpathlineto{\pgfqpoint{1.900433in}{1.036374in}}%
\pgfpathlineto{\pgfqpoint{1.900744in}{1.145336in}}%
\pgfpathlineto{\pgfqpoint{1.901573in}{0.909247in}}%
\pgfpathlineto{\pgfqpoint{1.900951in}{1.150840in}}%
\pgfpathlineto{\pgfqpoint{1.901988in}{1.089282in}}%
\pgfpathlineto{\pgfqpoint{1.902091in}{1.174690in}}%
\pgfpathlineto{\pgfqpoint{1.902817in}{0.965005in}}%
\pgfpathlineto{\pgfqpoint{1.902920in}{0.938979in}}%
\pgfpathlineto{\pgfqpoint{1.903024in}{1.055552in}}%
\pgfpathlineto{\pgfqpoint{1.903646in}{0.997114in}}%
\pgfpathlineto{\pgfqpoint{1.903853in}{1.172114in}}%
\pgfpathlineto{\pgfqpoint{1.904786in}{1.093975in}}%
\pgfpathlineto{\pgfqpoint{1.905200in}{0.931224in}}%
\pgfpathlineto{\pgfqpoint{1.905304in}{1.111555in}}%
\pgfpathlineto{\pgfqpoint{1.905822in}{0.999775in}}%
\pgfpathlineto{\pgfqpoint{1.905926in}{1.108020in}}%
\pgfpathlineto{\pgfqpoint{1.906651in}{0.919162in}}%
\pgfpathlineto{\pgfqpoint{1.906755in}{0.978778in}}%
\pgfpathlineto{\pgfqpoint{1.906859in}{0.890803in}}%
\pgfpathlineto{\pgfqpoint{1.907688in}{1.129715in}}%
\pgfpathlineto{\pgfqpoint{1.907791in}{1.129796in}}%
\pgfpathlineto{\pgfqpoint{1.907895in}{1.192801in}}%
\pgfpathlineto{\pgfqpoint{1.907999in}{1.057669in}}%
\pgfpathlineto{\pgfqpoint{1.908724in}{1.126349in}}%
\pgfpathlineto{\pgfqpoint{1.909139in}{0.980817in}}%
\pgfpathlineto{\pgfqpoint{1.909657in}{1.136366in}}%
\pgfpathlineto{\pgfqpoint{1.909761in}{1.089931in}}%
\pgfpathlineto{\pgfqpoint{1.909968in}{1.155539in}}%
\pgfpathlineto{\pgfqpoint{1.910279in}{0.969033in}}%
\pgfpathlineto{\pgfqpoint{1.910797in}{1.075484in}}%
\pgfpathlineto{\pgfqpoint{1.911833in}{0.918996in}}%
\pgfpathlineto{\pgfqpoint{1.912041in}{0.956739in}}%
\pgfpathlineto{\pgfqpoint{1.912766in}{1.184668in}}%
\pgfpathlineto{\pgfqpoint{1.913181in}{1.056616in}}%
\pgfpathlineto{\pgfqpoint{1.913699in}{0.965914in}}%
\pgfpathlineto{\pgfqpoint{1.913906in}{1.110836in}}%
\pgfpathlineto{\pgfqpoint{1.914217in}{1.028632in}}%
\pgfpathlineto{\pgfqpoint{1.914424in}{1.132699in}}%
\pgfpathlineto{\pgfqpoint{1.914632in}{0.954577in}}%
\pgfpathlineto{\pgfqpoint{1.915046in}{0.966082in}}%
\pgfpathlineto{\pgfqpoint{1.915668in}{0.909882in}}%
\pgfpathlineto{\pgfqpoint{1.915979in}{0.949186in}}%
\pgfpathlineto{\pgfqpoint{1.916186in}{1.048570in}}%
\pgfpathlineto{\pgfqpoint{1.916704in}{0.898447in}}%
\pgfpathlineto{\pgfqpoint{1.917015in}{0.898649in}}%
\pgfpathlineto{\pgfqpoint{1.917845in}{0.941702in}}%
\pgfpathlineto{\pgfqpoint{1.918259in}{0.809653in}}%
\pgfpathlineto{\pgfqpoint{1.918881in}{0.980719in}}%
\pgfpathlineto{\pgfqpoint{1.918570in}{0.798296in}}%
\pgfpathlineto{\pgfqpoint{1.919399in}{0.870613in}}%
\pgfpathlineto{\pgfqpoint{1.921368in}{1.148826in}}%
\pgfpathlineto{\pgfqpoint{1.921576in}{1.047800in}}%
\pgfpathlineto{\pgfqpoint{1.922197in}{0.949920in}}%
\pgfpathlineto{\pgfqpoint{1.922094in}{1.126137in}}%
\pgfpathlineto{\pgfqpoint{1.922405in}{1.044968in}}%
\pgfpathlineto{\pgfqpoint{1.923027in}{1.201091in}}%
\pgfpathlineto{\pgfqpoint{1.922612in}{0.989262in}}%
\pgfpathlineto{\pgfqpoint{1.923545in}{1.097404in}}%
\pgfpathlineto{\pgfqpoint{1.923856in}{0.907495in}}%
\pgfpathlineto{\pgfqpoint{1.924477in}{1.115580in}}%
\pgfpathlineto{\pgfqpoint{1.924788in}{0.975575in}}%
\pgfpathlineto{\pgfqpoint{1.926136in}{1.210561in}}%
\pgfpathlineto{\pgfqpoint{1.926239in}{1.120676in}}%
\pgfpathlineto{\pgfqpoint{1.926343in}{0.993273in}}%
\pgfpathlineto{\pgfqpoint{1.927172in}{1.268920in}}%
\pgfpathlineto{\pgfqpoint{1.929038in}{0.951639in}}%
\pgfpathlineto{\pgfqpoint{1.929141in}{1.034731in}}%
\pgfpathlineto{\pgfqpoint{1.929867in}{1.162855in}}%
\pgfpathlineto{\pgfqpoint{1.930074in}{1.007154in}}%
\pgfpathlineto{\pgfqpoint{1.930178in}{1.154797in}}%
\pgfpathlineto{\pgfqpoint{1.930281in}{0.941109in}}%
\pgfpathlineto{\pgfqpoint{1.931214in}{1.111944in}}%
\pgfpathlineto{\pgfqpoint{1.931318in}{1.178796in}}%
\pgfpathlineto{\pgfqpoint{1.932043in}{0.976299in}}%
\pgfpathlineto{\pgfqpoint{1.932250in}{1.104653in}}%
\pgfpathlineto{\pgfqpoint{1.932458in}{0.954800in}}%
\pgfpathlineto{\pgfqpoint{1.932561in}{1.229711in}}%
\pgfpathlineto{\pgfqpoint{1.933287in}{1.139752in}}%
\pgfpathlineto{\pgfqpoint{1.934012in}{0.987088in}}%
\pgfpathlineto{\pgfqpoint{1.933598in}{1.197604in}}%
\pgfpathlineto{\pgfqpoint{1.934323in}{1.136305in}}%
\pgfpathlineto{\pgfqpoint{1.934427in}{1.156505in}}%
\pgfpathlineto{\pgfqpoint{1.934531in}{1.098862in}}%
\pgfpathlineto{\pgfqpoint{1.934738in}{1.124997in}}%
\pgfpathlineto{\pgfqpoint{1.935049in}{0.973128in}}%
\pgfpathlineto{\pgfqpoint{1.935671in}{1.190000in}}%
\pgfpathlineto{\pgfqpoint{1.935878in}{1.031261in}}%
\pgfpathlineto{\pgfqpoint{1.936500in}{1.262056in}}%
\pgfpathlineto{\pgfqpoint{1.936914in}{1.022437in}}%
\pgfpathlineto{\pgfqpoint{1.937018in}{1.222005in}}%
\pgfpathlineto{\pgfqpoint{1.937329in}{0.992693in}}%
\pgfpathlineto{\pgfqpoint{1.938158in}{1.108333in}}%
\pgfpathlineto{\pgfqpoint{1.939298in}{1.228532in}}%
\pgfpathlineto{\pgfqpoint{1.939505in}{1.191912in}}%
\pgfpathlineto{\pgfqpoint{1.939713in}{1.214386in}}%
\pgfpathlineto{\pgfqpoint{1.940853in}{0.970996in}}%
\pgfpathlineto{\pgfqpoint{1.941474in}{1.324569in}}%
\pgfpathlineto{\pgfqpoint{1.942096in}{1.238601in}}%
\pgfpathlineto{\pgfqpoint{1.942614in}{0.968951in}}%
\pgfpathlineto{\pgfqpoint{1.943236in}{1.091990in}}%
\pgfpathlineto{\pgfqpoint{1.944169in}{1.227941in}}%
\pgfpathlineto{\pgfqpoint{1.943962in}{1.046393in}}%
\pgfpathlineto{\pgfqpoint{1.944480in}{1.177426in}}%
\pgfpathlineto{\pgfqpoint{1.945102in}{1.033764in}}%
\pgfpathlineto{\pgfqpoint{1.945516in}{1.179686in}}%
\pgfpathlineto{\pgfqpoint{1.945724in}{1.068727in}}%
\pgfpathlineto{\pgfqpoint{1.946346in}{0.982643in}}%
\pgfpathlineto{\pgfqpoint{1.946138in}{1.103724in}}%
\pgfpathlineto{\pgfqpoint{1.946553in}{1.040753in}}%
\pgfpathlineto{\pgfqpoint{1.946864in}{1.242514in}}%
\pgfpathlineto{\pgfqpoint{1.947175in}{0.981753in}}%
\pgfpathlineto{\pgfqpoint{1.947589in}{1.071857in}}%
\pgfpathlineto{\pgfqpoint{1.947796in}{0.924871in}}%
\pgfpathlineto{\pgfqpoint{1.948315in}{1.125904in}}%
\pgfpathlineto{\pgfqpoint{1.948729in}{1.013888in}}%
\pgfpathlineto{\pgfqpoint{1.949144in}{1.101300in}}%
\pgfpathlineto{\pgfqpoint{1.949247in}{1.005436in}}%
\pgfpathlineto{\pgfqpoint{1.949662in}{1.074339in}}%
\pgfpathlineto{\pgfqpoint{1.949973in}{0.838865in}}%
\pgfpathlineto{\pgfqpoint{1.950284in}{1.083387in}}%
\pgfpathlineto{\pgfqpoint{1.950802in}{0.995930in}}%
\pgfpathlineto{\pgfqpoint{1.951217in}{1.026145in}}%
\pgfpathlineto{\pgfqpoint{1.951320in}{0.979879in}}%
\pgfpathlineto{\pgfqpoint{1.951424in}{1.070380in}}%
\pgfpathlineto{\pgfqpoint{1.952253in}{0.920926in}}%
\pgfpathlineto{\pgfqpoint{1.952357in}{1.021285in}}%
\pgfpathlineto{\pgfqpoint{1.952978in}{0.869568in}}%
\pgfpathlineto{\pgfqpoint{1.952668in}{1.091672in}}%
\pgfpathlineto{\pgfqpoint{1.953600in}{0.937186in}}%
\pgfpathlineto{\pgfqpoint{1.953808in}{1.228369in}}%
\pgfpathlineto{\pgfqpoint{1.954533in}{0.921064in}}%
\pgfpathlineto{\pgfqpoint{1.954637in}{0.999266in}}%
\pgfpathlineto{\pgfqpoint{1.954844in}{0.810094in}}%
\pgfpathlineto{\pgfqpoint{1.955155in}{1.051866in}}%
\pgfpathlineto{\pgfqpoint{1.955880in}{0.949517in}}%
\pgfpathlineto{\pgfqpoint{1.956399in}{0.872611in}}%
\pgfpathlineto{\pgfqpoint{1.956917in}{1.063808in}}%
\pgfpathlineto{\pgfqpoint{1.957020in}{0.944621in}}%
\pgfpathlineto{\pgfqpoint{1.957850in}{1.158127in}}%
\pgfpathlineto{\pgfqpoint{1.957953in}{1.007793in}}%
\pgfpathlineto{\pgfqpoint{1.958264in}{0.986405in}}%
\pgfpathlineto{\pgfqpoint{1.958990in}{1.159505in}}%
\pgfpathlineto{\pgfqpoint{1.959404in}{0.989255in}}%
\pgfpathlineto{\pgfqpoint{1.960026in}{1.066763in}}%
\pgfpathlineto{\pgfqpoint{1.960337in}{1.163956in}}%
\pgfpathlineto{\pgfqpoint{1.960959in}{0.961917in}}%
\pgfpathlineto{\pgfqpoint{1.961166in}{1.094158in}}%
\pgfpathlineto{\pgfqpoint{1.962099in}{0.858369in}}%
\pgfpathlineto{\pgfqpoint{1.962306in}{0.957284in}}%
\pgfpathlineto{\pgfqpoint{1.962617in}{1.072285in}}%
\pgfpathlineto{\pgfqpoint{1.962721in}{0.942619in}}%
\pgfpathlineto{\pgfqpoint{1.962928in}{1.021503in}}%
\pgfpathlineto{\pgfqpoint{1.964068in}{0.859667in}}%
\pgfpathlineto{\pgfqpoint{1.964483in}{0.801621in}}%
\pgfpathlineto{\pgfqpoint{1.965208in}{1.121539in}}%
\pgfpathlineto{\pgfqpoint{1.966037in}{1.009499in}}%
\pgfpathlineto{\pgfqpoint{1.966244in}{1.113814in}}%
\pgfpathlineto{\pgfqpoint{1.966555in}{1.262382in}}%
\pgfpathlineto{\pgfqpoint{1.967074in}{1.087138in}}%
\pgfpathlineto{\pgfqpoint{1.967281in}{1.089858in}}%
\pgfpathlineto{\pgfqpoint{1.967384in}{1.013386in}}%
\pgfpathlineto{\pgfqpoint{1.968110in}{1.226609in}}%
\pgfpathlineto{\pgfqpoint{1.968214in}{1.183366in}}%
\pgfpathlineto{\pgfqpoint{1.968317in}{1.306770in}}%
\pgfpathlineto{\pgfqpoint{1.968732in}{1.059859in}}%
\pgfpathlineto{\pgfqpoint{1.969354in}{1.268726in}}%
\pgfpathlineto{\pgfqpoint{1.969665in}{1.404735in}}%
\pgfpathlineto{\pgfqpoint{1.970597in}{1.170267in}}%
\pgfpathlineto{\pgfqpoint{1.971115in}{1.335278in}}%
\pgfpathlineto{\pgfqpoint{1.971634in}{1.159022in}}%
\pgfpathlineto{\pgfqpoint{1.971737in}{1.256293in}}%
\pgfpathlineto{\pgfqpoint{1.972463in}{1.130940in}}%
\pgfpathlineto{\pgfqpoint{1.972048in}{1.293057in}}%
\pgfpathlineto{\pgfqpoint{1.972670in}{1.252561in}}%
\pgfpathlineto{\pgfqpoint{1.972774in}{1.327824in}}%
\pgfpathlineto{\pgfqpoint{1.973499in}{1.070920in}}%
\pgfpathlineto{\pgfqpoint{1.973603in}{1.145846in}}%
\pgfpathlineto{\pgfqpoint{1.973810in}{0.947232in}}%
\pgfpathlineto{\pgfqpoint{1.974432in}{1.175917in}}%
\pgfpathlineto{\pgfqpoint{1.974536in}{1.132106in}}%
\pgfpathlineto{\pgfqpoint{1.974639in}{1.314331in}}%
\pgfpathlineto{\pgfqpoint{1.975468in}{1.064010in}}%
\pgfpathlineto{\pgfqpoint{1.975572in}{1.099869in}}%
\pgfpathlineto{\pgfqpoint{1.976401in}{1.175770in}}%
\pgfpathlineto{\pgfqpoint{1.975779in}{0.957454in}}%
\pgfpathlineto{\pgfqpoint{1.976816in}{1.157944in}}%
\pgfpathlineto{\pgfqpoint{1.977023in}{1.061883in}}%
\pgfpathlineto{\pgfqpoint{1.977438in}{1.189380in}}%
\pgfpathlineto{\pgfqpoint{1.977852in}{1.141178in}}%
\pgfpathlineto{\pgfqpoint{1.978059in}{1.200877in}}%
\pgfpathlineto{\pgfqpoint{1.978267in}{1.103519in}}%
\pgfpathlineto{\pgfqpoint{1.978474in}{1.165729in}}%
\pgfpathlineto{\pgfqpoint{1.978992in}{1.044534in}}%
\pgfpathlineto{\pgfqpoint{1.979096in}{1.215019in}}%
\pgfpathlineto{\pgfqpoint{1.979614in}{1.130264in}}%
\pgfpathlineto{\pgfqpoint{1.980339in}{1.259036in}}%
\pgfpathlineto{\pgfqpoint{1.980132in}{0.998067in}}%
\pgfpathlineto{\pgfqpoint{1.980961in}{1.234832in}}%
\pgfpathlineto{\pgfqpoint{1.981583in}{1.109638in}}%
\pgfpathlineto{\pgfqpoint{1.981169in}{1.302483in}}%
\pgfpathlineto{\pgfqpoint{1.981894in}{1.113762in}}%
\pgfpathlineto{\pgfqpoint{1.981998in}{1.296023in}}%
\pgfpathlineto{\pgfqpoint{1.982930in}{1.003079in}}%
\pgfpathlineto{\pgfqpoint{1.983034in}{0.953432in}}%
\pgfpathlineto{\pgfqpoint{1.983449in}{1.177418in}}%
\pgfpathlineto{\pgfqpoint{1.983552in}{1.161492in}}%
\pgfpathlineto{\pgfqpoint{1.984174in}{1.335625in}}%
\pgfpathlineto{\pgfqpoint{1.984070in}{1.086321in}}%
\pgfpathlineto{\pgfqpoint{1.984692in}{1.253052in}}%
\pgfpathlineto{\pgfqpoint{1.985521in}{1.109959in}}%
\pgfpathlineto{\pgfqpoint{1.985314in}{1.271495in}}%
\pgfpathlineto{\pgfqpoint{1.985625in}{1.223441in}}%
\pgfpathlineto{\pgfqpoint{1.986143in}{1.491040in}}%
\pgfpathlineto{\pgfqpoint{1.986454in}{1.213755in}}%
\pgfpathlineto{\pgfqpoint{1.986558in}{1.272918in}}%
\pgfpathlineto{\pgfqpoint{1.986661in}{1.140160in}}%
\pgfpathlineto{\pgfqpoint{1.987076in}{1.353040in}}%
\pgfpathlineto{\pgfqpoint{1.987698in}{1.140332in}}%
\pgfpathlineto{\pgfqpoint{1.987802in}{1.136922in}}%
\pgfpathlineto{\pgfqpoint{1.988009in}{1.001863in}}%
\pgfpathlineto{\pgfqpoint{1.988320in}{1.274238in}}%
\pgfpathlineto{\pgfqpoint{1.988942in}{1.087397in}}%
\pgfpathlineto{\pgfqpoint{1.989149in}{1.100704in}}%
\pgfpathlineto{\pgfqpoint{1.989252in}{1.100477in}}%
\pgfpathlineto{\pgfqpoint{1.990289in}{1.302173in}}%
\pgfpathlineto{\pgfqpoint{1.990393in}{1.234306in}}%
\pgfpathlineto{\pgfqpoint{1.991014in}{1.284701in}}%
\pgfpathlineto{\pgfqpoint{1.990911in}{1.186266in}}%
\pgfpathlineto{\pgfqpoint{1.991222in}{1.238543in}}%
\pgfpathlineto{\pgfqpoint{1.992154in}{1.088371in}}%
\pgfpathlineto{\pgfqpoint{1.992362in}{1.122475in}}%
\pgfpathlineto{\pgfqpoint{1.993605in}{1.369354in}}%
\pgfpathlineto{\pgfqpoint{1.994642in}{1.028903in}}%
\pgfpathlineto{\pgfqpoint{1.994849in}{1.070979in}}%
\pgfpathlineto{\pgfqpoint{1.994953in}{1.074390in}}%
\pgfpathlineto{\pgfqpoint{1.995575in}{1.325883in}}%
\pgfpathlineto{\pgfqpoint{1.995989in}{1.127581in}}%
\pgfpathlineto{\pgfqpoint{1.996093in}{1.098403in}}%
\pgfpathlineto{\pgfqpoint{1.996196in}{1.227252in}}%
\pgfpathlineto{\pgfqpoint{1.996404in}{1.221674in}}%
\pgfpathlineto{\pgfqpoint{1.996611in}{1.081465in}}%
\pgfpathlineto{\pgfqpoint{1.997544in}{1.353493in}}%
\pgfpathlineto{\pgfqpoint{1.998891in}{0.991706in}}%
\pgfpathlineto{\pgfqpoint{1.999098in}{0.998796in}}%
\pgfpathlineto{\pgfqpoint{1.999513in}{1.201701in}}%
\pgfpathlineto{\pgfqpoint{2.000238in}{1.024031in}}%
\pgfpathlineto{\pgfqpoint{2.000446in}{0.994697in}}%
\pgfpathlineto{\pgfqpoint{2.000549in}{1.041723in}}%
\pgfpathlineto{\pgfqpoint{2.000653in}{1.129528in}}%
\pgfpathlineto{\pgfqpoint{2.000860in}{0.932170in}}%
\pgfpathlineto{\pgfqpoint{2.001689in}{1.085528in}}%
\pgfpathlineto{\pgfqpoint{2.001897in}{1.026746in}}%
\pgfpathlineto{\pgfqpoint{2.002311in}{1.221012in}}%
\pgfpathlineto{\pgfqpoint{2.002415in}{1.204657in}}%
\pgfpathlineto{\pgfqpoint{2.003037in}{1.246074in}}%
\pgfpathlineto{\pgfqpoint{2.002622in}{1.081100in}}%
\pgfpathlineto{\pgfqpoint{2.003451in}{1.201918in}}%
\pgfpathlineto{\pgfqpoint{2.004695in}{1.013599in}}%
\pgfpathlineto{\pgfqpoint{2.005006in}{0.957150in}}%
\pgfpathlineto{\pgfqpoint{2.005835in}{1.197540in}}%
\pgfpathlineto{\pgfqpoint{2.006042in}{1.006023in}}%
\pgfpathlineto{\pgfqpoint{2.006975in}{1.059263in}}%
\pgfpathlineto{\pgfqpoint{2.007908in}{1.138660in}}%
\pgfpathlineto{\pgfqpoint{2.008011in}{0.913223in}}%
\pgfpathlineto{\pgfqpoint{2.008840in}{1.334173in}}%
\pgfpathlineto{\pgfqpoint{2.009255in}{1.190996in}}%
\pgfpathlineto{\pgfqpoint{2.010084in}{1.060772in}}%
\pgfpathlineto{\pgfqpoint{2.009462in}{1.202391in}}%
\pgfpathlineto{\pgfqpoint{2.010499in}{1.098628in}}%
\pgfpathlineto{\pgfqpoint{2.010810in}{1.067851in}}%
\pgfpathlineto{\pgfqpoint{2.011535in}{1.234978in}}%
\pgfpathlineto{\pgfqpoint{2.011742in}{1.010234in}}%
\pgfpathlineto{\pgfqpoint{2.012571in}{1.241693in}}%
\pgfpathlineto{\pgfqpoint{2.012675in}{1.127199in}}%
\pgfpathlineto{\pgfqpoint{2.012779in}{1.231069in}}%
\pgfpathlineto{\pgfqpoint{2.013608in}{1.091506in}}%
\pgfpathlineto{\pgfqpoint{2.013712in}{1.172868in}}%
\pgfpathlineto{\pgfqpoint{2.014230in}{1.055903in}}%
\pgfpathlineto{\pgfqpoint{2.014333in}{1.189701in}}%
\pgfpathlineto{\pgfqpoint{2.014748in}{1.172124in}}%
\pgfpathlineto{\pgfqpoint{2.015266in}{1.299860in}}%
\pgfpathlineto{\pgfqpoint{2.015473in}{1.149007in}}%
\pgfpathlineto{\pgfqpoint{2.015784in}{1.181125in}}%
\pgfpathlineto{\pgfqpoint{2.016199in}{1.056866in}}%
\pgfpathlineto{\pgfqpoint{2.016510in}{1.307185in}}%
\pgfpathlineto{\pgfqpoint{2.016821in}{1.100858in}}%
\pgfpathlineto{\pgfqpoint{2.017339in}{1.069565in}}%
\pgfpathlineto{\pgfqpoint{2.018064in}{1.362500in}}%
\pgfpathlineto{\pgfqpoint{2.018790in}{0.998956in}}%
\pgfpathlineto{\pgfqpoint{2.019515in}{1.244259in}}%
\pgfpathlineto{\pgfqpoint{2.019619in}{1.356300in}}%
\pgfpathlineto{\pgfqpoint{2.020448in}{1.070772in}}%
\pgfpathlineto{\pgfqpoint{2.020759in}{1.198602in}}%
\pgfpathlineto{\pgfqpoint{2.021070in}{1.056404in}}%
\pgfpathlineto{\pgfqpoint{2.021485in}{1.138002in}}%
\pgfpathlineto{\pgfqpoint{2.021795in}{0.947463in}}%
\pgfpathlineto{\pgfqpoint{2.022521in}{1.163986in}}%
\pgfpathlineto{\pgfqpoint{2.022832in}{0.967100in}}%
\pgfpathlineto{\pgfqpoint{2.023039in}{0.979768in}}%
\pgfpathlineto{\pgfqpoint{2.023143in}{0.817101in}}%
\pgfpathlineto{\pgfqpoint{2.023454in}{1.083536in}}%
\pgfpathlineto{\pgfqpoint{2.023972in}{0.993120in}}%
\pgfpathlineto{\pgfqpoint{2.024179in}{0.982726in}}%
\pgfpathlineto{\pgfqpoint{2.025216in}{1.228125in}}%
\pgfpathlineto{\pgfqpoint{2.025734in}{1.302457in}}%
\pgfpathlineto{\pgfqpoint{2.025837in}{1.214138in}}%
\pgfpathlineto{\pgfqpoint{2.026874in}{1.001163in}}%
\pgfpathlineto{\pgfqpoint{2.026459in}{1.258733in}}%
\pgfpathlineto{\pgfqpoint{2.026977in}{1.013888in}}%
\pgfpathlineto{\pgfqpoint{2.027807in}{1.248542in}}%
\pgfpathlineto{\pgfqpoint{2.028117in}{1.075638in}}%
\pgfpathlineto{\pgfqpoint{2.028843in}{0.960777in}}%
\pgfpathlineto{\pgfqpoint{2.028325in}{1.137511in}}%
\pgfpathlineto{\pgfqpoint{2.029154in}{1.077954in}}%
\pgfpathlineto{\pgfqpoint{2.029983in}{0.900643in}}%
\pgfpathlineto{\pgfqpoint{2.029568in}{1.089808in}}%
\pgfpathlineto{\pgfqpoint{2.030190in}{1.037093in}}%
\pgfpathlineto{\pgfqpoint{2.031434in}{1.320550in}}%
\pgfpathlineto{\pgfqpoint{2.031538in}{1.099850in}}%
\pgfpathlineto{\pgfqpoint{2.032470in}{1.451406in}}%
\pgfpathlineto{\pgfqpoint{2.033299in}{1.173432in}}%
\pgfpathlineto{\pgfqpoint{2.033714in}{1.291014in}}%
\pgfpathlineto{\pgfqpoint{2.034336in}{1.440095in}}%
\pgfpathlineto{\pgfqpoint{2.034750in}{1.258650in}}%
\pgfpathlineto{\pgfqpoint{2.035061in}{1.368137in}}%
\pgfpathlineto{\pgfqpoint{2.035476in}{1.156632in}}%
\pgfpathlineto{\pgfqpoint{2.035683in}{1.135684in}}%
\pgfpathlineto{\pgfqpoint{2.035787in}{1.209552in}}%
\pgfpathlineto{\pgfqpoint{2.035890in}{1.282958in}}%
\pgfpathlineto{\pgfqpoint{2.036512in}{1.103920in}}%
\pgfpathlineto{\pgfqpoint{2.036720in}{1.144628in}}%
\pgfpathlineto{\pgfqpoint{2.036823in}{1.142363in}}%
\pgfpathlineto{\pgfqpoint{2.036927in}{1.005856in}}%
\pgfpathlineto{\pgfqpoint{2.037756in}{1.294140in}}%
\pgfpathlineto{\pgfqpoint{2.037860in}{1.172673in}}%
\pgfpathlineto{\pgfqpoint{2.039207in}{1.348834in}}%
\pgfpathlineto{\pgfqpoint{2.040140in}{0.954283in}}%
\pgfpathlineto{\pgfqpoint{2.040554in}{1.180615in}}%
\pgfpathlineto{\pgfqpoint{2.041072in}{1.105083in}}%
\pgfpathlineto{\pgfqpoint{2.041591in}{1.295855in}}%
\pgfpathlineto{\pgfqpoint{2.041280in}{1.097838in}}%
\pgfpathlineto{\pgfqpoint{2.042109in}{1.111628in}}%
\pgfpathlineto{\pgfqpoint{2.042731in}{0.974795in}}%
\pgfpathlineto{\pgfqpoint{2.043456in}{1.019338in}}%
\pgfpathlineto{\pgfqpoint{2.043663in}{1.229423in}}%
\pgfpathlineto{\pgfqpoint{2.044285in}{0.967821in}}%
\pgfpathlineto{\pgfqpoint{2.044596in}{1.066966in}}%
\pgfpathlineto{\pgfqpoint{2.044700in}{1.022095in}}%
\pgfpathlineto{\pgfqpoint{2.045425in}{1.144466in}}%
\pgfpathlineto{\pgfqpoint{2.045529in}{1.224596in}}%
\pgfpathlineto{\pgfqpoint{2.045840in}{0.965139in}}%
\pgfpathlineto{\pgfqpoint{2.046358in}{1.069956in}}%
\pgfpathlineto{\pgfqpoint{2.046462in}{1.031695in}}%
\pgfpathlineto{\pgfqpoint{2.046773in}{1.292971in}}%
\pgfpathlineto{\pgfqpoint{2.047291in}{1.086571in}}%
\pgfpathlineto{\pgfqpoint{2.048431in}{1.358634in}}%
\pgfpathlineto{\pgfqpoint{2.047498in}{1.076992in}}%
\pgfpathlineto{\pgfqpoint{2.048638in}{1.274707in}}%
\pgfpathlineto{\pgfqpoint{2.049260in}{1.163404in}}%
\pgfpathlineto{\pgfqpoint{2.049467in}{1.313285in}}%
\pgfpathlineto{\pgfqpoint{2.049675in}{1.239069in}}%
\pgfpathlineto{\pgfqpoint{2.050296in}{1.380604in}}%
\pgfpathlineto{\pgfqpoint{2.050815in}{1.376154in}}%
\pgfpathlineto{\pgfqpoint{2.051644in}{1.200072in}}%
\pgfpathlineto{\pgfqpoint{2.051955in}{1.316279in}}%
\pgfpathlineto{\pgfqpoint{2.052473in}{1.385223in}}%
\pgfpathlineto{\pgfqpoint{2.052162in}{1.253331in}}%
\pgfpathlineto{\pgfqpoint{2.052991in}{1.274115in}}%
\pgfpathlineto{\pgfqpoint{2.053198in}{1.394513in}}%
\pgfpathlineto{\pgfqpoint{2.053509in}{1.292878in}}%
\pgfpathlineto{\pgfqpoint{2.054131in}{1.128481in}}%
\pgfpathlineto{\pgfqpoint{2.054442in}{1.373267in}}%
\pgfpathlineto{\pgfqpoint{2.054546in}{1.217136in}}%
\pgfpathlineto{\pgfqpoint{2.054960in}{1.190851in}}%
\pgfpathlineto{\pgfqpoint{2.055686in}{1.354885in}}%
\pgfpathlineto{\pgfqpoint{2.056618in}{1.000420in}}%
\pgfpathlineto{\pgfqpoint{2.056929in}{1.065441in}}%
\pgfpathlineto{\pgfqpoint{2.057344in}{1.299613in}}%
\pgfpathlineto{\pgfqpoint{2.057759in}{0.954378in}}%
\pgfpathlineto{\pgfqpoint{2.058173in}{1.232430in}}%
\pgfpathlineto{\pgfqpoint{2.058380in}{1.190120in}}%
\pgfpathlineto{\pgfqpoint{2.058484in}{1.299186in}}%
\pgfpathlineto{\pgfqpoint{2.058588in}{1.385502in}}%
\pgfpathlineto{\pgfqpoint{2.058899in}{1.024742in}}%
\pgfpathlineto{\pgfqpoint{2.059209in}{1.183904in}}%
\pgfpathlineto{\pgfqpoint{2.060453in}{0.919825in}}%
\pgfpathlineto{\pgfqpoint{2.061593in}{1.394143in}}%
\pgfpathlineto{\pgfqpoint{2.062111in}{1.339951in}}%
\pgfpathlineto{\pgfqpoint{2.063251in}{1.160345in}}%
\pgfpathlineto{\pgfqpoint{2.063355in}{1.177966in}}%
\pgfpathlineto{\pgfqpoint{2.064391in}{1.313534in}}%
\pgfpathlineto{\pgfqpoint{2.064495in}{1.225473in}}%
\pgfpathlineto{\pgfqpoint{2.065117in}{1.297255in}}%
\pgfpathlineto{\pgfqpoint{2.065221in}{1.193241in}}%
\pgfpathlineto{\pgfqpoint{2.065739in}{1.055334in}}%
\pgfpathlineto{\pgfqpoint{2.066361in}{1.166955in}}%
\pgfpathlineto{\pgfqpoint{2.066568in}{1.141925in}}%
\pgfpathlineto{\pgfqpoint{2.066672in}{1.305496in}}%
\pgfpathlineto{\pgfqpoint{2.066775in}{1.287606in}}%
\pgfpathlineto{\pgfqpoint{2.066879in}{1.413328in}}%
\pgfpathlineto{\pgfqpoint{2.067293in}{1.141942in}}%
\pgfpathlineto{\pgfqpoint{2.067708in}{1.195909in}}%
\pgfpathlineto{\pgfqpoint{2.068226in}{1.249519in}}%
\pgfpathlineto{\pgfqpoint{2.068744in}{1.100171in}}%
\pgfpathlineto{\pgfqpoint{2.069781in}{1.330264in}}%
\pgfpathlineto{\pgfqpoint{2.068952in}{1.060688in}}%
\pgfpathlineto{\pgfqpoint{2.069884in}{1.161464in}}%
\pgfpathlineto{\pgfqpoint{2.069988in}{1.167264in}}%
\pgfpathlineto{\pgfqpoint{2.070092in}{1.127016in}}%
\pgfpathlineto{\pgfqpoint{2.070817in}{1.268472in}}%
\pgfpathlineto{\pgfqpoint{2.071128in}{1.146051in}}%
\pgfpathlineto{\pgfqpoint{2.071335in}{1.164316in}}%
\pgfpathlineto{\pgfqpoint{2.071439in}{1.117408in}}%
\pgfpathlineto{\pgfqpoint{2.071957in}{1.039460in}}%
\pgfpathlineto{\pgfqpoint{2.072164in}{1.163001in}}%
\pgfpathlineto{\pgfqpoint{2.072475in}{1.302182in}}%
\pgfpathlineto{\pgfqpoint{2.072994in}{1.098155in}}%
\pgfpathlineto{\pgfqpoint{2.073305in}{1.209862in}}%
\pgfpathlineto{\pgfqpoint{2.073926in}{1.043146in}}%
\pgfpathlineto{\pgfqpoint{2.074134in}{1.261384in}}%
\pgfpathlineto{\pgfqpoint{2.074445in}{1.177950in}}%
\pgfpathlineto{\pgfqpoint{2.074859in}{1.262093in}}%
\pgfpathlineto{\pgfqpoint{2.075688in}{1.082791in}}%
\pgfpathlineto{\pgfqpoint{2.076517in}{1.168032in}}%
\pgfpathlineto{\pgfqpoint{2.075999in}{1.013205in}}%
\pgfpathlineto{\pgfqpoint{2.076725in}{1.145329in}}%
\pgfpathlineto{\pgfqpoint{2.076828in}{0.987641in}}%
\pgfpathlineto{\pgfqpoint{2.077450in}{1.288052in}}%
\pgfpathlineto{\pgfqpoint{2.077865in}{1.081703in}}%
\pgfpathlineto{\pgfqpoint{2.078072in}{1.094269in}}%
\pgfpathlineto{\pgfqpoint{2.078176in}{1.076924in}}%
\pgfpathlineto{\pgfqpoint{2.078901in}{0.898619in}}%
\pgfpathlineto{\pgfqpoint{2.078590in}{1.200138in}}%
\pgfpathlineto{\pgfqpoint{2.079212in}{1.024364in}}%
\pgfpathlineto{\pgfqpoint{2.079316in}{1.196338in}}%
\pgfpathlineto{\pgfqpoint{2.079834in}{0.949805in}}%
\pgfpathlineto{\pgfqpoint{2.080352in}{1.103856in}}%
\pgfpathlineto{\pgfqpoint{2.080559in}{1.152578in}}%
\pgfpathlineto{\pgfqpoint{2.080767in}{0.933184in}}%
\pgfpathlineto{\pgfqpoint{2.081285in}{1.079369in}}%
\pgfpathlineto{\pgfqpoint{2.081492in}{0.904058in}}%
\pgfpathlineto{\pgfqpoint{2.082218in}{1.154837in}}%
\pgfpathlineto{\pgfqpoint{2.082321in}{1.083594in}}%
\pgfpathlineto{\pgfqpoint{2.083358in}{1.279964in}}%
\pgfpathlineto{\pgfqpoint{2.083669in}{1.200618in}}%
\pgfpathlineto{\pgfqpoint{2.084394in}{1.381905in}}%
\pgfpathlineto{\pgfqpoint{2.084083in}{1.148775in}}%
\pgfpathlineto{\pgfqpoint{2.084498in}{1.188381in}}%
\pgfpathlineto{\pgfqpoint{2.084912in}{1.049741in}}%
\pgfpathlineto{\pgfqpoint{2.084705in}{1.286161in}}%
\pgfpathlineto{\pgfqpoint{2.085534in}{1.170912in}}%
\pgfpathlineto{\pgfqpoint{2.085949in}{1.040801in}}%
\pgfpathlineto{\pgfqpoint{2.086467in}{1.218086in}}%
\pgfpathlineto{\pgfqpoint{2.087400in}{1.011746in}}%
\pgfpathlineto{\pgfqpoint{2.087607in}{1.178591in}}%
\pgfpathlineto{\pgfqpoint{2.088021in}{1.030731in}}%
\pgfpathlineto{\pgfqpoint{2.088436in}{1.236763in}}%
\pgfpathlineto{\pgfqpoint{2.088747in}{1.155119in}}%
\pgfpathlineto{\pgfqpoint{2.088954in}{1.132065in}}%
\pgfpathlineto{\pgfqpoint{2.089058in}{1.194668in}}%
\pgfpathlineto{\pgfqpoint{2.089887in}{1.382610in}}%
\pgfpathlineto{\pgfqpoint{2.089680in}{1.177003in}}%
\pgfpathlineto{\pgfqpoint{2.090301in}{1.296536in}}%
\pgfpathlineto{\pgfqpoint{2.091545in}{1.099663in}}%
\pgfpathlineto{\pgfqpoint{2.091131in}{1.299983in}}%
\pgfpathlineto{\pgfqpoint{2.091649in}{1.146214in}}%
\pgfpathlineto{\pgfqpoint{2.092374in}{1.421136in}}%
\pgfpathlineto{\pgfqpoint{2.093100in}{1.401254in}}%
\pgfpathlineto{\pgfqpoint{2.093203in}{1.413437in}}%
\pgfpathlineto{\pgfqpoint{2.093307in}{1.366258in}}%
\pgfpathlineto{\pgfqpoint{2.094343in}{1.274001in}}%
\pgfpathlineto{\pgfqpoint{2.094136in}{1.372179in}}%
\pgfpathlineto{\pgfqpoint{2.094447in}{1.277786in}}%
\pgfpathlineto{\pgfqpoint{2.094551in}{1.347821in}}%
\pgfpathlineto{\pgfqpoint{2.094758in}{1.111162in}}%
\pgfpathlineto{\pgfqpoint{2.095380in}{1.288376in}}%
\pgfpathlineto{\pgfqpoint{2.096002in}{1.096988in}}%
\pgfpathlineto{\pgfqpoint{2.095587in}{1.310859in}}%
\pgfpathlineto{\pgfqpoint{2.096520in}{1.162656in}}%
\pgfpathlineto{\pgfqpoint{2.096727in}{1.128214in}}%
\pgfpathlineto{\pgfqpoint{2.096934in}{1.171706in}}%
\pgfpathlineto{\pgfqpoint{2.097453in}{1.391085in}}%
\pgfpathlineto{\pgfqpoint{2.097245in}{1.123984in}}%
\pgfpathlineto{\pgfqpoint{2.097971in}{1.199801in}}%
\pgfpathlineto{\pgfqpoint{2.098385in}{1.104678in}}%
\pgfpathlineto{\pgfqpoint{2.099111in}{1.167760in}}%
\pgfpathlineto{\pgfqpoint{2.099525in}{1.321668in}}%
\pgfpathlineto{\pgfqpoint{2.099940in}{1.239712in}}%
\pgfpathlineto{\pgfqpoint{2.100355in}{0.980587in}}%
\pgfpathlineto{\pgfqpoint{2.101080in}{1.013735in}}%
\pgfpathlineto{\pgfqpoint{2.101806in}{1.261767in}}%
\pgfpathlineto{\pgfqpoint{2.102220in}{1.140826in}}%
\pgfpathlineto{\pgfqpoint{2.102427in}{1.008500in}}%
\pgfpathlineto{\pgfqpoint{2.102946in}{1.224204in}}%
\pgfpathlineto{\pgfqpoint{2.103256in}{1.304229in}}%
\pgfpathlineto{\pgfqpoint{2.103153in}{1.165322in}}%
\pgfpathlineto{\pgfqpoint{2.103567in}{1.180527in}}%
\pgfpathlineto{\pgfqpoint{2.104189in}{1.029291in}}%
\pgfpathlineto{\pgfqpoint{2.103878in}{1.200157in}}%
\pgfpathlineto{\pgfqpoint{2.104500in}{1.102062in}}%
\pgfpathlineto{\pgfqpoint{2.105537in}{1.340286in}}%
\pgfpathlineto{\pgfqpoint{2.104915in}{1.058980in}}%
\pgfpathlineto{\pgfqpoint{2.105744in}{1.316547in}}%
\pgfpathlineto{\pgfqpoint{2.106262in}{1.388030in}}%
\pgfpathlineto{\pgfqpoint{2.106884in}{1.144341in}}%
\pgfpathlineto{\pgfqpoint{2.107506in}{1.302340in}}%
\pgfpathlineto{\pgfqpoint{2.107920in}{1.220458in}}%
\pgfpathlineto{\pgfqpoint{2.108749in}{1.126754in}}%
\pgfpathlineto{\pgfqpoint{2.108231in}{1.263338in}}%
\pgfpathlineto{\pgfqpoint{2.109060in}{1.179188in}}%
\pgfpathlineto{\pgfqpoint{2.109579in}{1.009140in}}%
\pgfpathlineto{\pgfqpoint{2.109786in}{1.208921in}}%
\pgfpathlineto{\pgfqpoint{2.110304in}{1.093522in}}%
\pgfpathlineto{\pgfqpoint{2.110408in}{1.084309in}}%
\pgfpathlineto{\pgfqpoint{2.111030in}{1.043189in}}%
\pgfpathlineto{\pgfqpoint{2.111548in}{1.302381in}}%
\pgfpathlineto{\pgfqpoint{2.111962in}{1.091393in}}%
\pgfpathlineto{\pgfqpoint{2.112377in}{1.312278in}}%
\pgfpathlineto{\pgfqpoint{2.112791in}{1.126743in}}%
\pgfpathlineto{\pgfqpoint{2.112999in}{1.248157in}}%
\pgfpathlineto{\pgfqpoint{2.113621in}{1.035676in}}%
\pgfpathlineto{\pgfqpoint{2.113931in}{1.216868in}}%
\pgfpathlineto{\pgfqpoint{2.114139in}{1.090208in}}%
\pgfpathlineto{\pgfqpoint{2.114761in}{1.299799in}}%
\pgfpathlineto{\pgfqpoint{2.115071in}{1.141111in}}%
\pgfpathlineto{\pgfqpoint{2.115175in}{1.223705in}}%
\pgfpathlineto{\pgfqpoint{2.115590in}{1.075890in}}%
\pgfpathlineto{\pgfqpoint{2.116108in}{1.165385in}}%
\pgfpathlineto{\pgfqpoint{2.117144in}{0.980450in}}%
\pgfpathlineto{\pgfqpoint{2.116937in}{1.204379in}}%
\pgfpathlineto{\pgfqpoint{2.117248in}{1.084037in}}%
\pgfpathlineto{\pgfqpoint{2.117455in}{1.041793in}}%
\pgfpathlineto{\pgfqpoint{2.117766in}{0.982767in}}%
\pgfpathlineto{\pgfqpoint{2.118699in}{1.251735in}}%
\pgfpathlineto{\pgfqpoint{2.119528in}{1.050424in}}%
\pgfpathlineto{\pgfqpoint{2.120046in}{1.117212in}}%
\pgfpathlineto{\pgfqpoint{2.121186in}{1.362472in}}%
\pgfpathlineto{\pgfqpoint{2.121290in}{1.258979in}}%
\pgfpathlineto{\pgfqpoint{2.122326in}{1.053507in}}%
\pgfpathlineto{\pgfqpoint{2.121704in}{1.327128in}}%
\pgfpathlineto{\pgfqpoint{2.122430in}{1.134905in}}%
\pgfpathlineto{\pgfqpoint{2.122948in}{1.125626in}}%
\pgfpathlineto{\pgfqpoint{2.123674in}{1.283747in}}%
\pgfpathlineto{\pgfqpoint{2.123777in}{1.288212in}}%
\pgfpathlineto{\pgfqpoint{2.124088in}{1.048975in}}%
\pgfpathlineto{\pgfqpoint{2.125125in}{1.085925in}}%
\pgfpathlineto{\pgfqpoint{2.125228in}{1.147809in}}%
\pgfpathlineto{\pgfqpoint{2.125643in}{0.983900in}}%
\pgfpathlineto{\pgfqpoint{2.126161in}{1.047608in}}%
\pgfpathlineto{\pgfqpoint{2.126368in}{1.127004in}}%
\pgfpathlineto{\pgfqpoint{2.126472in}{1.027911in}}%
\pgfpathlineto{\pgfqpoint{2.126576in}{1.061358in}}%
\pgfpathlineto{\pgfqpoint{2.126679in}{0.844853in}}%
\pgfpathlineto{\pgfqpoint{2.127197in}{1.103215in}}%
\pgfpathlineto{\pgfqpoint{2.127716in}{0.957639in}}%
\pgfpathlineto{\pgfqpoint{2.128752in}{1.187785in}}%
\pgfpathlineto{\pgfqpoint{2.128130in}{0.869266in}}%
\pgfpathlineto{\pgfqpoint{2.128959in}{1.146985in}}%
\pgfpathlineto{\pgfqpoint{2.129063in}{0.972653in}}%
\pgfpathlineto{\pgfqpoint{2.129685in}{1.267609in}}%
\pgfpathlineto{\pgfqpoint{2.129996in}{1.188661in}}%
\pgfpathlineto{\pgfqpoint{2.130099in}{1.099626in}}%
\pgfpathlineto{\pgfqpoint{2.130825in}{1.426485in}}%
\pgfpathlineto{\pgfqpoint{2.131136in}{1.108190in}}%
\pgfpathlineto{\pgfqpoint{2.131239in}{1.320640in}}%
\pgfpathlineto{\pgfqpoint{2.132276in}{1.184329in}}%
\pgfpathlineto{\pgfqpoint{2.132794in}{1.287740in}}%
\pgfpathlineto{\pgfqpoint{2.133001in}{1.080460in}}%
\pgfpathlineto{\pgfqpoint{2.133416in}{1.283747in}}%
\pgfpathlineto{\pgfqpoint{2.134245in}{0.881363in}}%
\pgfpathlineto{\pgfqpoint{2.133623in}{1.354117in}}%
\pgfpathlineto{\pgfqpoint{2.134659in}{1.133211in}}%
\pgfpathlineto{\pgfqpoint{2.134867in}{1.179067in}}%
\pgfpathlineto{\pgfqpoint{2.135592in}{1.340320in}}%
\pgfpathlineto{\pgfqpoint{2.135178in}{1.113041in}}%
\pgfpathlineto{\pgfqpoint{2.135903in}{1.211648in}}%
\pgfpathlineto{\pgfqpoint{2.136940in}{1.114284in}}%
\pgfpathlineto{\pgfqpoint{2.136421in}{1.280397in}}%
\pgfpathlineto{\pgfqpoint{2.137043in}{1.182364in}}%
\pgfpathlineto{\pgfqpoint{2.137250in}{1.284893in}}%
\pgfpathlineto{\pgfqpoint{2.137665in}{1.147779in}}%
\pgfpathlineto{\pgfqpoint{2.138080in}{1.205940in}}%
\pgfpathlineto{\pgfqpoint{2.138494in}{1.074166in}}%
\pgfpathlineto{\pgfqpoint{2.138390in}{1.235709in}}%
\pgfpathlineto{\pgfqpoint{2.139220in}{1.148246in}}%
\pgfpathlineto{\pgfqpoint{2.140360in}{1.317684in}}%
\pgfpathlineto{\pgfqpoint{2.139427in}{1.070891in}}%
\pgfpathlineto{\pgfqpoint{2.140463in}{1.215853in}}%
\pgfpathlineto{\pgfqpoint{2.140567in}{1.025342in}}%
\pgfpathlineto{\pgfqpoint{2.141189in}{1.254406in}}%
\pgfpathlineto{\pgfqpoint{2.141500in}{1.121773in}}%
\pgfpathlineto{\pgfqpoint{2.141603in}{1.301874in}}%
\pgfpathlineto{\pgfqpoint{2.142122in}{1.071202in}}%
\pgfpathlineto{\pgfqpoint{2.142536in}{1.208248in}}%
\pgfpathlineto{\pgfqpoint{2.143054in}{1.045700in}}%
\pgfpathlineto{\pgfqpoint{2.143572in}{1.227010in}}%
\pgfpathlineto{\pgfqpoint{2.143883in}{1.048724in}}%
\pgfpathlineto{\pgfqpoint{2.143780in}{1.244970in}}%
\pgfpathlineto{\pgfqpoint{2.144713in}{1.195733in}}%
\pgfpathlineto{\pgfqpoint{2.144816in}{1.210461in}}%
\pgfpathlineto{\pgfqpoint{2.145023in}{1.097343in}}%
\pgfpathlineto{\pgfqpoint{2.145127in}{1.142645in}}%
\pgfpathlineto{\pgfqpoint{2.145438in}{0.994524in}}%
\pgfpathlineto{\pgfqpoint{2.145645in}{1.195558in}}%
\pgfpathlineto{\pgfqpoint{2.145853in}{1.167754in}}%
\pgfpathlineto{\pgfqpoint{2.145956in}{1.293141in}}%
\pgfpathlineto{\pgfqpoint{2.146060in}{1.060327in}}%
\pgfpathlineto{\pgfqpoint{2.146889in}{1.232469in}}%
\pgfpathlineto{\pgfqpoint{2.147200in}{1.110585in}}%
\pgfpathlineto{\pgfqpoint{2.147822in}{1.241113in}}%
\pgfpathlineto{\pgfqpoint{2.148029in}{1.171301in}}%
\pgfpathlineto{\pgfqpoint{2.148651in}{1.063132in}}%
\pgfpathlineto{\pgfqpoint{2.148340in}{1.183793in}}%
\pgfpathlineto{\pgfqpoint{2.148858in}{1.162444in}}%
\pgfpathlineto{\pgfqpoint{2.148962in}{1.225361in}}%
\pgfpathlineto{\pgfqpoint{2.149584in}{1.008183in}}%
\pgfpathlineto{\pgfqpoint{2.149791in}{1.131125in}}%
\pgfpathlineto{\pgfqpoint{2.150724in}{1.009398in}}%
\pgfpathlineto{\pgfqpoint{2.151449in}{0.994866in}}%
\pgfpathlineto{\pgfqpoint{2.151864in}{1.231478in}}%
\pgfpathlineto{\pgfqpoint{2.152900in}{1.110417in}}%
\pgfpathlineto{\pgfqpoint{2.152278in}{1.302053in}}%
\pgfpathlineto{\pgfqpoint{2.153211in}{1.134263in}}%
\pgfpathlineto{\pgfqpoint{2.154351in}{1.298940in}}%
\pgfpathlineto{\pgfqpoint{2.154455in}{1.251192in}}%
\pgfpathlineto{\pgfqpoint{2.154662in}{1.045321in}}%
\pgfpathlineto{\pgfqpoint{2.155595in}{1.094382in}}%
\pgfpathlineto{\pgfqpoint{2.155802in}{1.091413in}}%
\pgfpathlineto{\pgfqpoint{2.155906in}{1.094496in}}%
\pgfpathlineto{\pgfqpoint{2.156942in}{1.305651in}}%
\pgfpathlineto{\pgfqpoint{2.157253in}{1.302164in}}%
\pgfpathlineto{\pgfqpoint{2.157564in}{1.209379in}}%
\pgfpathlineto{\pgfqpoint{2.157875in}{1.317246in}}%
\pgfpathlineto{\pgfqpoint{2.158082in}{1.314153in}}%
\pgfpathlineto{\pgfqpoint{2.158186in}{1.369213in}}%
\pgfpathlineto{\pgfqpoint{2.158704in}{1.206909in}}%
\pgfpathlineto{\pgfqpoint{2.159015in}{1.091215in}}%
\pgfpathlineto{\pgfqpoint{2.159326in}{1.221503in}}%
\pgfpathlineto{\pgfqpoint{2.159844in}{1.154277in}}%
\pgfpathlineto{\pgfqpoint{2.160155in}{1.038290in}}%
\pgfpathlineto{\pgfqpoint{2.160569in}{1.280069in}}%
\pgfpathlineto{\pgfqpoint{2.160880in}{1.041684in}}%
\pgfpathlineto{\pgfqpoint{2.161502in}{1.347643in}}%
\pgfpathlineto{\pgfqpoint{2.162020in}{1.124887in}}%
\pgfpathlineto{\pgfqpoint{2.162850in}{1.324265in}}%
\pgfpathlineto{\pgfqpoint{2.162642in}{1.092313in}}%
\pgfpathlineto{\pgfqpoint{2.163264in}{1.323470in}}%
\pgfpathlineto{\pgfqpoint{2.164197in}{1.081560in}}%
\pgfpathlineto{\pgfqpoint{2.164404in}{1.261921in}}%
\pgfpathlineto{\pgfqpoint{2.164611in}{1.179915in}}%
\pgfpathlineto{\pgfqpoint{2.164922in}{1.328483in}}%
\pgfpathlineto{\pgfqpoint{2.165648in}{1.196566in}}%
\pgfpathlineto{\pgfqpoint{2.165751in}{1.165923in}}%
\pgfpathlineto{\pgfqpoint{2.165959in}{1.247026in}}%
\pgfpathlineto{\pgfqpoint{2.166581in}{1.209108in}}%
\pgfpathlineto{\pgfqpoint{2.166684in}{1.393910in}}%
\pgfpathlineto{\pgfqpoint{2.167721in}{1.270208in}}%
\pgfpathlineto{\pgfqpoint{2.168239in}{1.098811in}}%
\pgfpathlineto{\pgfqpoint{2.168342in}{1.348789in}}%
\pgfpathlineto{\pgfqpoint{2.168757in}{1.314024in}}%
\pgfpathlineto{\pgfqpoint{2.169172in}{1.364114in}}%
\pgfpathlineto{\pgfqpoint{2.168964in}{1.264097in}}%
\pgfpathlineto{\pgfqpoint{2.169482in}{1.281556in}}%
\pgfpathlineto{\pgfqpoint{2.169793in}{1.381637in}}%
\pgfpathlineto{\pgfqpoint{2.170623in}{1.196603in}}%
\pgfpathlineto{\pgfqpoint{2.171659in}{1.382893in}}%
\pgfpathlineto{\pgfqpoint{2.170933in}{1.163420in}}%
\pgfpathlineto{\pgfqpoint{2.171763in}{1.286547in}}%
\pgfpathlineto{\pgfqpoint{2.171970in}{1.212079in}}%
\pgfpathlineto{\pgfqpoint{2.172177in}{1.287897in}}%
\pgfpathlineto{\pgfqpoint{2.172281in}{1.276562in}}%
\pgfpathlineto{\pgfqpoint{2.172384in}{1.357081in}}%
\pgfpathlineto{\pgfqpoint{2.172695in}{1.187344in}}%
\pgfpathlineto{\pgfqpoint{2.173214in}{1.201883in}}%
\pgfpathlineto{\pgfqpoint{2.173524in}{1.331275in}}%
\pgfpathlineto{\pgfqpoint{2.174457in}{1.030788in}}%
\pgfpathlineto{\pgfqpoint{2.175390in}{1.247222in}}%
\pgfpathlineto{\pgfqpoint{2.175908in}{1.099663in}}%
\pgfpathlineto{\pgfqpoint{2.176012in}{0.984257in}}%
\pgfpathlineto{\pgfqpoint{2.176323in}{1.230210in}}%
\pgfpathlineto{\pgfqpoint{2.176841in}{1.224256in}}%
\pgfpathlineto{\pgfqpoint{2.177048in}{1.270661in}}%
\pgfpathlineto{\pgfqpoint{2.177255in}{1.178146in}}%
\pgfpathlineto{\pgfqpoint{2.178188in}{1.078682in}}%
\pgfpathlineto{\pgfqpoint{2.177670in}{1.320460in}}%
\pgfpathlineto{\pgfqpoint{2.178396in}{1.124250in}}%
\pgfpathlineto{\pgfqpoint{2.178706in}{1.163030in}}%
\pgfpathlineto{\pgfqpoint{2.179328in}{0.930035in}}%
\pgfpathlineto{\pgfqpoint{2.179846in}{1.110066in}}%
\pgfpathlineto{\pgfqpoint{2.180779in}{1.051303in}}%
\pgfpathlineto{\pgfqpoint{2.180987in}{1.259432in}}%
\pgfpathlineto{\pgfqpoint{2.181505in}{1.287431in}}%
\pgfpathlineto{\pgfqpoint{2.182334in}{1.069756in}}%
\pgfpathlineto{\pgfqpoint{2.182437in}{1.245490in}}%
\pgfpathlineto{\pgfqpoint{2.183370in}{1.166020in}}%
\pgfpathlineto{\pgfqpoint{2.183681in}{1.033779in}}%
\pgfpathlineto{\pgfqpoint{2.184096in}{1.234665in}}%
\pgfpathlineto{\pgfqpoint{2.184407in}{1.192388in}}%
\pgfpathlineto{\pgfqpoint{2.184510in}{1.264096in}}%
\pgfpathlineto{\pgfqpoint{2.185236in}{1.009996in}}%
\pgfpathlineto{\pgfqpoint{2.185443in}{1.192932in}}%
\pgfpathlineto{\pgfqpoint{2.185858in}{1.054163in}}%
\pgfpathlineto{\pgfqpoint{2.186687in}{1.061244in}}%
\pgfpathlineto{\pgfqpoint{2.186790in}{1.056169in}}%
\pgfpathlineto{\pgfqpoint{2.186894in}{0.832213in}}%
\pgfpathlineto{\pgfqpoint{2.187827in}{1.147448in}}%
\pgfpathlineto{\pgfqpoint{2.188863in}{1.014403in}}%
\pgfpathlineto{\pgfqpoint{2.188656in}{1.221400in}}%
\pgfpathlineto{\pgfqpoint{2.188967in}{1.051797in}}%
\pgfpathlineto{\pgfqpoint{2.189900in}{0.939228in}}%
\pgfpathlineto{\pgfqpoint{2.189485in}{1.159707in}}%
\pgfpathlineto{\pgfqpoint{2.190003in}{1.029869in}}%
\pgfpathlineto{\pgfqpoint{2.190936in}{1.246729in}}%
\pgfpathlineto{\pgfqpoint{2.191351in}{1.216503in}}%
\pgfpathlineto{\pgfqpoint{2.191558in}{1.226738in}}%
\pgfpathlineto{\pgfqpoint{2.191661in}{1.386981in}}%
\pgfpathlineto{\pgfqpoint{2.192594in}{1.223496in}}%
\pgfpathlineto{\pgfqpoint{2.193009in}{1.346303in}}%
\pgfpathlineto{\pgfqpoint{2.193216in}{1.174024in}}%
\pgfpathlineto{\pgfqpoint{2.193631in}{1.234523in}}%
\pgfpathlineto{\pgfqpoint{2.194252in}{1.166016in}}%
\pgfpathlineto{\pgfqpoint{2.194356in}{1.240614in}}%
\pgfpathlineto{\pgfqpoint{2.194460in}{1.228361in}}%
\pgfpathlineto{\pgfqpoint{2.194874in}{1.423845in}}%
\pgfpathlineto{\pgfqpoint{2.195600in}{1.306880in}}%
\pgfpathlineto{\pgfqpoint{2.195911in}{1.189764in}}%
\pgfpathlineto{\pgfqpoint{2.196636in}{1.290793in}}%
\pgfpathlineto{\pgfqpoint{2.196947in}{1.130967in}}%
\pgfpathlineto{\pgfqpoint{2.197880in}{1.391565in}}%
\pgfpathlineto{\pgfqpoint{2.198813in}{1.012397in}}%
\pgfpathlineto{\pgfqpoint{2.199227in}{1.112282in}}%
\pgfpathlineto{\pgfqpoint{2.199953in}{1.322424in}}%
\pgfpathlineto{\pgfqpoint{2.199642in}{1.057741in}}%
\pgfpathlineto{\pgfqpoint{2.200367in}{1.215198in}}%
\pgfpathlineto{\pgfqpoint{2.201196in}{0.938014in}}%
\pgfpathlineto{\pgfqpoint{2.201715in}{1.073554in}}%
\pgfpathlineto{\pgfqpoint{2.202440in}{1.173193in}}%
\pgfpathlineto{\pgfqpoint{2.202025in}{0.937609in}}%
\pgfpathlineto{\pgfqpoint{2.202647in}{1.045499in}}%
\pgfpathlineto{\pgfqpoint{2.202751in}{0.943778in}}%
\pgfpathlineto{\pgfqpoint{2.203476in}{1.303298in}}%
\pgfpathlineto{\pgfqpoint{2.204927in}{0.912199in}}%
\pgfpathlineto{\pgfqpoint{2.205031in}{1.013928in}}%
\pgfpathlineto{\pgfqpoint{2.205135in}{1.002056in}}%
\pgfpathlineto{\pgfqpoint{2.205238in}{1.011628in}}%
\pgfpathlineto{\pgfqpoint{2.205653in}{1.193733in}}%
\pgfpathlineto{\pgfqpoint{2.206275in}{0.975448in}}%
\pgfpathlineto{\pgfqpoint{2.206793in}{1.082485in}}%
\pgfpathlineto{\pgfqpoint{2.207311in}{0.952408in}}%
\pgfpathlineto{\pgfqpoint{2.207415in}{0.886928in}}%
\pgfpathlineto{\pgfqpoint{2.208037in}{1.102173in}}%
\pgfpathlineto{\pgfqpoint{2.208451in}{0.888431in}}%
\pgfpathlineto{\pgfqpoint{2.209280in}{1.134538in}}%
\pgfpathlineto{\pgfqpoint{2.209591in}{1.113650in}}%
\pgfpathlineto{\pgfqpoint{2.209798in}{0.945709in}}%
\pgfpathlineto{\pgfqpoint{2.210628in}{1.103235in}}%
\pgfpathlineto{\pgfqpoint{2.211146in}{1.178538in}}%
\pgfpathlineto{\pgfqpoint{2.211457in}{1.025362in}}%
\pgfpathlineto{\pgfqpoint{2.211664in}{1.119865in}}%
\pgfpathlineto{\pgfqpoint{2.211768in}{1.005255in}}%
\pgfpathlineto{\pgfqpoint{2.212700in}{1.185246in}}%
\pgfpathlineto{\pgfqpoint{2.213529in}{1.071509in}}%
\pgfpathlineto{\pgfqpoint{2.213115in}{1.186726in}}%
\pgfpathlineto{\pgfqpoint{2.213737in}{1.096190in}}%
\pgfpathlineto{\pgfqpoint{2.213840in}{1.199678in}}%
\pgfpathlineto{\pgfqpoint{2.214151in}{1.067710in}}%
\pgfpathlineto{\pgfqpoint{2.214773in}{1.157161in}}%
\pgfpathlineto{\pgfqpoint{2.215188in}{1.027970in}}%
\pgfpathlineto{\pgfqpoint{2.215602in}{1.245052in}}%
\pgfpathlineto{\pgfqpoint{2.215810in}{1.146456in}}%
\pgfpathlineto{\pgfqpoint{2.216742in}{1.299356in}}%
\pgfpathlineto{\pgfqpoint{2.216431in}{1.023020in}}%
\pgfpathlineto{\pgfqpoint{2.216950in}{1.202849in}}%
\pgfpathlineto{\pgfqpoint{2.217261in}{1.347852in}}%
\pgfpathlineto{\pgfqpoint{2.217779in}{1.111161in}}%
\pgfpathlineto{\pgfqpoint{2.218090in}{1.249277in}}%
\pgfpathlineto{\pgfqpoint{2.218919in}{1.134427in}}%
\pgfpathlineto{\pgfqpoint{2.218401in}{1.322333in}}%
\pgfpathlineto{\pgfqpoint{2.219022in}{1.240183in}}%
\pgfpathlineto{\pgfqpoint{2.219644in}{1.335324in}}%
\pgfpathlineto{\pgfqpoint{2.219748in}{1.154867in}}%
\pgfpathlineto{\pgfqpoint{2.219955in}{1.168419in}}%
\pgfpathlineto{\pgfqpoint{2.220473in}{1.041837in}}%
\pgfpathlineto{\pgfqpoint{2.220577in}{1.296866in}}%
\pgfpathlineto{\pgfqpoint{2.220992in}{1.174618in}}%
\pgfpathlineto{\pgfqpoint{2.221302in}{1.027686in}}%
\pgfpathlineto{\pgfqpoint{2.221406in}{1.135849in}}%
\pgfpathlineto{\pgfqpoint{2.222235in}{1.362039in}}%
\pgfpathlineto{\pgfqpoint{2.222546in}{1.248256in}}%
\pgfpathlineto{\pgfqpoint{2.223272in}{1.188674in}}%
\pgfpathlineto{\pgfqpoint{2.223064in}{1.391287in}}%
\pgfpathlineto{\pgfqpoint{2.223375in}{1.249184in}}%
\pgfpathlineto{\pgfqpoint{2.223479in}{1.369953in}}%
\pgfpathlineto{\pgfqpoint{2.224204in}{1.080966in}}%
\pgfpathlineto{\pgfqpoint{2.224308in}{1.085035in}}%
\pgfpathlineto{\pgfqpoint{2.225655in}{1.336370in}}%
\pgfpathlineto{\pgfqpoint{2.226174in}{1.036963in}}%
\pgfpathlineto{\pgfqpoint{2.226899in}{1.067780in}}%
\pgfpathlineto{\pgfqpoint{2.227003in}{1.088675in}}%
\pgfpathlineto{\pgfqpoint{2.227106in}{0.987020in}}%
\pgfpathlineto{\pgfqpoint{2.227210in}{1.077821in}}%
\pgfpathlineto{\pgfqpoint{2.227832in}{1.159750in}}%
\pgfpathlineto{\pgfqpoint{2.228454in}{0.823538in}}%
\pgfpathlineto{\pgfqpoint{2.228868in}{1.108863in}}%
\pgfpathlineto{\pgfqpoint{2.229179in}{0.818206in}}%
\pgfpathlineto{\pgfqpoint{2.229594in}{1.070951in}}%
\pgfpathlineto{\pgfqpoint{2.229697in}{1.098118in}}%
\pgfpathlineto{\pgfqpoint{2.230112in}{1.041697in}}%
\pgfpathlineto{\pgfqpoint{2.230423in}{1.084375in}}%
\pgfpathlineto{\pgfqpoint{2.231148in}{0.928186in}}%
\pgfpathlineto{\pgfqpoint{2.231459in}{0.942878in}}%
\pgfpathlineto{\pgfqpoint{2.231977in}{0.939613in}}%
\pgfpathlineto{\pgfqpoint{2.232599in}{1.195934in}}%
\pgfpathlineto{\pgfqpoint{2.233325in}{1.024154in}}%
\pgfpathlineto{\pgfqpoint{2.233739in}{1.064270in}}%
\pgfpathlineto{\pgfqpoint{2.234879in}{1.298802in}}%
\pgfpathlineto{\pgfqpoint{2.235087in}{1.221741in}}%
\pgfpathlineto{\pgfqpoint{2.235605in}{0.976830in}}%
\pgfpathlineto{\pgfqpoint{2.236019in}{1.259796in}}%
\pgfpathlineto{\pgfqpoint{2.236227in}{1.097983in}}%
\pgfpathlineto{\pgfqpoint{2.237263in}{0.928293in}}%
\pgfpathlineto{\pgfqpoint{2.236848in}{1.165945in}}%
\pgfpathlineto{\pgfqpoint{2.237367in}{1.034721in}}%
\pgfpathlineto{\pgfqpoint{2.237781in}{1.122917in}}%
\pgfpathlineto{\pgfqpoint{2.238092in}{1.006549in}}%
\pgfpathlineto{\pgfqpoint{2.238299in}{1.050451in}}%
\pgfpathlineto{\pgfqpoint{2.238403in}{1.009469in}}%
\pgfpathlineto{\pgfqpoint{2.239025in}{1.167062in}}%
\pgfpathlineto{\pgfqpoint{2.239129in}{1.178816in}}%
\pgfpathlineto{\pgfqpoint{2.239439in}{1.220990in}}%
\pgfpathlineto{\pgfqpoint{2.240269in}{0.964603in}}%
\pgfpathlineto{\pgfqpoint{2.240372in}{0.958453in}}%
\pgfpathlineto{\pgfqpoint{2.240476in}{0.998527in}}%
\pgfpathlineto{\pgfqpoint{2.240580in}{0.961663in}}%
\pgfpathlineto{\pgfqpoint{2.240890in}{1.222200in}}%
\pgfpathlineto{\pgfqpoint{2.241823in}{1.062646in}}%
\pgfpathlineto{\pgfqpoint{2.242445in}{1.140494in}}%
\pgfpathlineto{\pgfqpoint{2.242341in}{1.028067in}}%
\pgfpathlineto{\pgfqpoint{2.242860in}{1.036421in}}%
\pgfpathlineto{\pgfqpoint{2.243171in}{1.143979in}}%
\pgfpathlineto{\pgfqpoint{2.243585in}{0.919654in}}%
\pgfpathlineto{\pgfqpoint{2.243896in}{1.084628in}}%
\pgfpathlineto{\pgfqpoint{2.244518in}{1.010857in}}%
\pgfpathlineto{\pgfqpoint{2.244829in}{1.143747in}}%
\pgfpathlineto{\pgfqpoint{2.245036in}{1.040969in}}%
\pgfpathlineto{\pgfqpoint{2.245347in}{1.139569in}}%
\pgfpathlineto{\pgfqpoint{2.245451in}{0.935789in}}%
\pgfpathlineto{\pgfqpoint{2.245658in}{0.967668in}}%
\pgfpathlineto{\pgfqpoint{2.245762in}{0.805251in}}%
\pgfpathlineto{\pgfqpoint{2.246280in}{1.138160in}}%
\pgfpathlineto{\pgfqpoint{2.246798in}{0.886627in}}%
\pgfpathlineto{\pgfqpoint{2.247627in}{1.175342in}}%
\pgfpathlineto{\pgfqpoint{2.248145in}{1.154665in}}%
\pgfpathlineto{\pgfqpoint{2.248353in}{1.060412in}}%
\pgfpathlineto{\pgfqpoint{2.248871in}{1.180502in}}%
\pgfpathlineto{\pgfqpoint{2.248974in}{1.126100in}}%
\pgfpathlineto{\pgfqpoint{2.249285in}{1.246399in}}%
\pgfpathlineto{\pgfqpoint{2.249907in}{1.038981in}}%
\pgfpathlineto{\pgfqpoint{2.250011in}{1.068047in}}%
\pgfpathlineto{\pgfqpoint{2.250114in}{1.030678in}}%
\pgfpathlineto{\pgfqpoint{2.250529in}{1.206528in}}%
\pgfpathlineto{\pgfqpoint{2.250944in}{1.098915in}}%
\pgfpathlineto{\pgfqpoint{2.251047in}{1.104087in}}%
\pgfpathlineto{\pgfqpoint{2.251151in}{1.080565in}}%
\pgfpathlineto{\pgfqpoint{2.252187in}{0.869461in}}%
\pgfpathlineto{\pgfqpoint{2.251358in}{1.121130in}}%
\pgfpathlineto{\pgfqpoint{2.252394in}{0.988166in}}%
\pgfpathlineto{\pgfqpoint{2.252705in}{1.024300in}}%
\pgfpathlineto{\pgfqpoint{2.252809in}{0.973526in}}%
\pgfpathlineto{\pgfqpoint{2.252913in}{1.105353in}}%
\pgfpathlineto{\pgfqpoint{2.253431in}{0.959859in}}%
\pgfpathlineto{\pgfqpoint{2.253949in}{1.045044in}}%
\pgfpathlineto{\pgfqpoint{2.254053in}{0.919056in}}%
\pgfpathlineto{\pgfqpoint{2.254675in}{1.196130in}}%
\pgfpathlineto{\pgfqpoint{2.254985in}{1.135106in}}%
\pgfpathlineto{\pgfqpoint{2.255711in}{0.996672in}}%
\pgfpathlineto{\pgfqpoint{2.255504in}{1.197893in}}%
\pgfpathlineto{\pgfqpoint{2.256126in}{1.094109in}}%
\pgfpathlineto{\pgfqpoint{2.256851in}{1.245127in}}%
\pgfpathlineto{\pgfqpoint{2.256436in}{1.035765in}}%
\pgfpathlineto{\pgfqpoint{2.257162in}{1.145128in}}%
\pgfpathlineto{\pgfqpoint{2.257369in}{1.059564in}}%
\pgfpathlineto{\pgfqpoint{2.257473in}{1.394853in}}%
\pgfpathlineto{\pgfqpoint{2.257887in}{1.313420in}}%
\pgfpathlineto{\pgfqpoint{2.258198in}{1.435624in}}%
\pgfpathlineto{\pgfqpoint{2.258302in}{1.240431in}}%
\pgfpathlineto{\pgfqpoint{2.258613in}{1.112217in}}%
\pgfpathlineto{\pgfqpoint{2.259027in}{1.358275in}}%
\pgfpathlineto{\pgfqpoint{2.259131in}{1.275985in}}%
\pgfpathlineto{\pgfqpoint{2.259338in}{1.407144in}}%
\pgfpathlineto{\pgfqpoint{2.259546in}{1.147420in}}%
\pgfpathlineto{\pgfqpoint{2.260167in}{1.233162in}}%
\pgfpathlineto{\pgfqpoint{2.261100in}{1.119522in}}%
\pgfpathlineto{\pgfqpoint{2.260686in}{1.243985in}}%
\pgfpathlineto{\pgfqpoint{2.261308in}{1.127208in}}%
\pgfpathlineto{\pgfqpoint{2.261515in}{1.083097in}}%
\pgfpathlineto{\pgfqpoint{2.261618in}{1.189162in}}%
\pgfpathlineto{\pgfqpoint{2.261826in}{1.171219in}}%
\pgfpathlineto{\pgfqpoint{2.262240in}{1.347489in}}%
\pgfpathlineto{\pgfqpoint{2.262862in}{1.146610in}}%
\pgfpathlineto{\pgfqpoint{2.262966in}{1.326384in}}%
\pgfpathlineto{\pgfqpoint{2.263277in}{1.107767in}}%
\pgfpathlineto{\pgfqpoint{2.264106in}{1.191385in}}%
\pgfpathlineto{\pgfqpoint{2.264209in}{1.272047in}}%
\pgfpathlineto{\pgfqpoint{2.265039in}{1.056102in}}%
\pgfpathlineto{\pgfqpoint{2.265868in}{1.268702in}}%
\pgfpathlineto{\pgfqpoint{2.266282in}{1.142490in}}%
\pgfpathlineto{\pgfqpoint{2.267422in}{1.038926in}}%
\pgfpathlineto{\pgfqpoint{2.267008in}{1.228793in}}%
\pgfpathlineto{\pgfqpoint{2.267526in}{1.056930in}}%
\pgfpathlineto{\pgfqpoint{2.268459in}{1.284069in}}%
\pgfpathlineto{\pgfqpoint{2.268666in}{1.216037in}}%
\pgfpathlineto{\pgfqpoint{2.269391in}{1.289229in}}%
\pgfpathlineto{\pgfqpoint{2.269806in}{1.079505in}}%
\pgfpathlineto{\pgfqpoint{2.270946in}{1.407698in}}%
\pgfpathlineto{\pgfqpoint{2.271257in}{1.321539in}}%
\pgfpathlineto{\pgfqpoint{2.272708in}{1.071174in}}%
\pgfpathlineto{\pgfqpoint{2.272812in}{1.161759in}}%
\pgfpathlineto{\pgfqpoint{2.273744in}{1.278446in}}%
\pgfpathlineto{\pgfqpoint{2.273122in}{1.135494in}}%
\pgfpathlineto{\pgfqpoint{2.273848in}{1.241647in}}%
\pgfpathlineto{\pgfqpoint{2.274055in}{1.156201in}}%
\pgfpathlineto{\pgfqpoint{2.274781in}{1.323349in}}%
\pgfpathlineto{\pgfqpoint{2.274884in}{1.319649in}}%
\pgfpathlineto{\pgfqpoint{2.275506in}{1.138275in}}%
\pgfpathlineto{\pgfqpoint{2.275921in}{1.397071in}}%
\pgfpathlineto{\pgfqpoint{2.276335in}{1.175403in}}%
\pgfpathlineto{\pgfqpoint{2.277268in}{1.232900in}}%
\pgfpathlineto{\pgfqpoint{2.277372in}{1.318841in}}%
\pgfpathlineto{\pgfqpoint{2.278097in}{1.217098in}}%
\pgfpathlineto{\pgfqpoint{2.278408in}{1.276773in}}%
\pgfpathlineto{\pgfqpoint{2.279134in}{1.098403in}}%
\pgfpathlineto{\pgfqpoint{2.279548in}{1.157613in}}%
\pgfpathlineto{\pgfqpoint{2.280170in}{1.320615in}}%
\pgfpathlineto{\pgfqpoint{2.280274in}{1.115457in}}%
\pgfpathlineto{\pgfqpoint{2.280688in}{1.203913in}}%
\pgfpathlineto{\pgfqpoint{2.281206in}{1.021927in}}%
\pgfpathlineto{\pgfqpoint{2.281621in}{1.212380in}}%
\pgfpathlineto{\pgfqpoint{2.281828in}{1.160091in}}%
\pgfpathlineto{\pgfqpoint{2.282968in}{1.003828in}}%
\pgfpathlineto{\pgfqpoint{2.282554in}{1.233083in}}%
\pgfpathlineto{\pgfqpoint{2.283072in}{1.087723in}}%
\pgfpathlineto{\pgfqpoint{2.283797in}{1.329050in}}%
\pgfpathlineto{\pgfqpoint{2.284316in}{1.292677in}}%
\pgfpathlineto{\pgfqpoint{2.284419in}{1.215285in}}%
\pgfpathlineto{\pgfqpoint{2.284627in}{1.400079in}}%
\pgfpathlineto{\pgfqpoint{2.285248in}{1.297106in}}%
\pgfpathlineto{\pgfqpoint{2.285663in}{1.425879in}}%
\pgfpathlineto{\pgfqpoint{2.286077in}{1.230145in}}%
\pgfpathlineto{\pgfqpoint{2.286181in}{1.320334in}}%
\pgfpathlineto{\pgfqpoint{2.286492in}{1.193166in}}%
\pgfpathlineto{\pgfqpoint{2.287114in}{1.454233in}}%
\pgfpathlineto{\pgfqpoint{2.287321in}{1.280060in}}%
\pgfpathlineto{\pgfqpoint{2.288461in}{1.115663in}}%
\pgfpathlineto{\pgfqpoint{2.288565in}{1.134390in}}%
\pgfpathlineto{\pgfqpoint{2.288979in}{1.355192in}}%
\pgfpathlineto{\pgfqpoint{2.289705in}{1.339985in}}%
\pgfpathlineto{\pgfqpoint{2.290845in}{0.993272in}}%
\pgfpathlineto{\pgfqpoint{2.291052in}{1.068922in}}%
\pgfpathlineto{\pgfqpoint{2.291156in}{1.224655in}}%
\pgfpathlineto{\pgfqpoint{2.291985in}{1.009991in}}%
\pgfpathlineto{\pgfqpoint{2.292192in}{1.190577in}}%
\pgfpathlineto{\pgfqpoint{2.292918in}{1.146052in}}%
\pgfpathlineto{\pgfqpoint{2.292607in}{1.239119in}}%
\pgfpathlineto{\pgfqpoint{2.293229in}{1.165405in}}%
\pgfpathlineto{\pgfqpoint{2.293850in}{1.235084in}}%
\pgfpathlineto{\pgfqpoint{2.293643in}{1.050183in}}%
\pgfpathlineto{\pgfqpoint{2.294161in}{1.105360in}}%
\pgfpathlineto{\pgfqpoint{2.294265in}{1.089709in}}%
\pgfpathlineto{\pgfqpoint{2.294369in}{1.128136in}}%
\pgfpathlineto{\pgfqpoint{2.294783in}{1.343513in}}%
\pgfpathlineto{\pgfqpoint{2.295612in}{1.281489in}}%
\pgfpathlineto{\pgfqpoint{2.295820in}{1.423888in}}%
\pgfpathlineto{\pgfqpoint{2.296338in}{1.156247in}}%
\pgfpathlineto{\pgfqpoint{2.296545in}{1.219585in}}%
\pgfpathlineto{\pgfqpoint{2.296649in}{1.206357in}}%
\pgfpathlineto{\pgfqpoint{2.296960in}{1.282123in}}%
\pgfpathlineto{\pgfqpoint{2.297167in}{1.406722in}}%
\pgfpathlineto{\pgfqpoint{2.297685in}{1.193207in}}%
\pgfpathlineto{\pgfqpoint{2.297892in}{1.251028in}}%
\pgfpathlineto{\pgfqpoint{2.298929in}{1.094660in}}%
\pgfpathlineto{\pgfqpoint{2.298618in}{1.388456in}}%
\pgfpathlineto{\pgfqpoint{2.299032in}{1.146040in}}%
\pgfpathlineto{\pgfqpoint{2.299240in}{1.326535in}}%
\pgfpathlineto{\pgfqpoint{2.299758in}{1.106626in}}%
\pgfpathlineto{\pgfqpoint{2.300173in}{1.177575in}}%
\pgfpathlineto{\pgfqpoint{2.300380in}{0.980558in}}%
\pgfpathlineto{\pgfqpoint{2.300794in}{1.270915in}}%
\pgfpathlineto{\pgfqpoint{2.301105in}{1.242278in}}%
\pgfpathlineto{\pgfqpoint{2.301416in}{1.325169in}}%
\pgfpathlineto{\pgfqpoint{2.301934in}{1.139824in}}%
\pgfpathlineto{\pgfqpoint{2.302038in}{1.087419in}}%
\pgfpathlineto{\pgfqpoint{2.302660in}{1.303002in}}%
\pgfpathlineto{\pgfqpoint{2.302867in}{1.156526in}}%
\pgfpathlineto{\pgfqpoint{2.303696in}{1.287435in}}%
\pgfpathlineto{\pgfqpoint{2.303904in}{1.124595in}}%
\pgfpathlineto{\pgfqpoint{2.304318in}{1.249523in}}%
\pgfpathlineto{\pgfqpoint{2.304629in}{1.058646in}}%
\pgfpathlineto{\pgfqpoint{2.304940in}{1.167631in}}%
\pgfpathlineto{\pgfqpoint{2.306080in}{0.941953in}}%
\pgfpathlineto{\pgfqpoint{2.305665in}{1.258227in}}%
\pgfpathlineto{\pgfqpoint{2.306184in}{1.034831in}}%
\pgfpathlineto{\pgfqpoint{2.306909in}{1.222746in}}%
\pgfpathlineto{\pgfqpoint{2.307635in}{1.172246in}}%
\pgfpathlineto{\pgfqpoint{2.308464in}{1.006124in}}%
\pgfpathlineto{\pgfqpoint{2.307946in}{1.273631in}}%
\pgfpathlineto{\pgfqpoint{2.308982in}{1.083011in}}%
\pgfpathlineto{\pgfqpoint{2.309915in}{1.211148in}}%
\pgfpathlineto{\pgfqpoint{2.310018in}{1.010117in}}%
\pgfpathlineto{\pgfqpoint{2.310329in}{0.955770in}}%
\pgfpathlineto{\pgfqpoint{2.310226in}{1.089899in}}%
\pgfpathlineto{\pgfqpoint{2.310847in}{1.047704in}}%
\pgfpathlineto{\pgfqpoint{2.310951in}{1.123597in}}%
\pgfpathlineto{\pgfqpoint{2.311366in}{0.869685in}}%
\pgfpathlineto{\pgfqpoint{2.311780in}{1.025671in}}%
\pgfpathlineto{\pgfqpoint{2.311884in}{0.964972in}}%
\pgfpathlineto{\pgfqpoint{2.312402in}{1.162812in}}%
\pgfpathlineto{\pgfqpoint{2.312609in}{1.072525in}}%
\pgfpathlineto{\pgfqpoint{2.312713in}{1.195085in}}%
\pgfpathlineto{\pgfqpoint{2.313128in}{0.997168in}}%
\pgfpathlineto{\pgfqpoint{2.313438in}{1.058027in}}%
\pgfusepath{stroke}%
\end{pgfscope}%
\begin{pgfscope}%
\pgfsetrectcap%
\pgfsetmiterjoin%
\pgfsetlinewidth{0.803000pt}%
\definecolor{currentstroke}{rgb}{0.000000,0.000000,0.000000}%
\pgfsetstrokecolor{currentstroke}%
\pgfsetdash{}{0pt}%
\pgfpathmoveto{\pgfqpoint{0.530716in}{0.416447in}}%
\pgfpathlineto{\pgfqpoint{0.530716in}{1.788330in}}%
\pgfusepath{stroke}%
\end{pgfscope}%
\begin{pgfscope}%
\pgfsetrectcap%
\pgfsetmiterjoin%
\pgfsetlinewidth{0.803000pt}%
\definecolor{currentstroke}{rgb}{0.000000,0.000000,0.000000}%
\pgfsetstrokecolor{currentstroke}%
\pgfsetdash{}{0pt}%
\pgfpathmoveto{\pgfqpoint{2.398330in}{0.416447in}}%
\pgfpathlineto{\pgfqpoint{2.398330in}{1.788330in}}%
\pgfusepath{stroke}%
\end{pgfscope}%
\begin{pgfscope}%
\pgfsetrectcap%
\pgfsetmiterjoin%
\pgfsetlinewidth{0.803000pt}%
\definecolor{currentstroke}{rgb}{0.000000,0.000000,0.000000}%
\pgfsetstrokecolor{currentstroke}%
\pgfsetdash{}{0pt}%
\pgfpathmoveto{\pgfqpoint{0.530716in}{0.416447in}}%
\pgfpathlineto{\pgfqpoint{2.398330in}{0.416447in}}%
\pgfusepath{stroke}%
\end{pgfscope}%
\begin{pgfscope}%
\pgfsetrectcap%
\pgfsetmiterjoin%
\pgfsetlinewidth{0.803000pt}%
\definecolor{currentstroke}{rgb}{0.000000,0.000000,0.000000}%
\pgfsetstrokecolor{currentstroke}%
\pgfsetdash{}{0pt}%
\pgfpathmoveto{\pgfqpoint{0.530716in}{1.788330in}}%
\pgfpathlineto{\pgfqpoint{2.398330in}{1.788330in}}%
\pgfusepath{stroke}%
\end{pgfscope}%
\begin{pgfscope}%
\pgfsetbuttcap%
\pgfsetmiterjoin%
\definecolor{currentfill}{rgb}{1.000000,1.000000,1.000000}%
\pgfsetfillcolor{currentfill}%
\pgfsetfillopacity{0.800000}%
\pgfsetlinewidth{1.003750pt}%
\definecolor{currentstroke}{rgb}{0.800000,0.800000,0.800000}%
\pgfsetstrokecolor{currentstroke}%
\pgfsetstrokeopacity{0.800000}%
\pgfsetdash{}{0pt}%
\pgfpathmoveto{\pgfqpoint{0.608494in}{1.544552in}}%
\pgfpathlineto{\pgfqpoint{1.608827in}{1.544552in}}%
\pgfpathquadraticcurveto{\pgfqpoint{1.631049in}{1.544552in}}{\pgfqpoint{1.631049in}{1.566775in}}%
\pgfpathlineto{\pgfqpoint{1.631049in}{1.710552in}}%
\pgfpathquadraticcurveto{\pgfqpoint{1.631049in}{1.732774in}}{\pgfqpoint{1.608827in}{1.732774in}}%
\pgfpathlineto{\pgfqpoint{0.608494in}{1.732774in}}%
\pgfpathquadraticcurveto{\pgfqpoint{0.586272in}{1.732774in}}{\pgfqpoint{0.586272in}{1.710552in}}%
\pgfpathlineto{\pgfqpoint{0.586272in}{1.566775in}}%
\pgfpathquadraticcurveto{\pgfqpoint{0.586272in}{1.544552in}}{\pgfqpoint{0.608494in}{1.544552in}}%
\pgfpathlineto{\pgfqpoint{0.608494in}{1.544552in}}%
\pgfpathclose%
\pgfusepath{stroke,fill}%
\end{pgfscope}%
\begin{pgfscope}%
\pgfsetrectcap%
\pgfsetroundjoin%
\pgfsetlinewidth{1.505625pt}%
\definecolor{currentstroke}{rgb}{0.000000,0.619608,0.450980}%
\pgfsetstrokecolor{currentstroke}%
\pgfsetdash{}{0pt}%
\pgfpathmoveto{\pgfqpoint{0.630716in}{1.649441in}}%
\pgfpathlineto{\pgfqpoint{0.741827in}{1.649441in}}%
\pgfpathlineto{\pgfqpoint{0.852938in}{1.649441in}}%
\pgfusepath{stroke}%
\end{pgfscope}%
\begin{pgfscope}%
\definecolor{textcolor}{rgb}{0.000000,0.000000,0.000000}%
\pgfsetstrokecolor{textcolor}%
\pgfsetfillcolor{textcolor}%
\pgftext[x=0.941827in,y=1.610552in,left,base]{\color{textcolor}\rmfamily\fontsize{8.000000}{9.600000}\selectfont Flicker noise}%
\end{pgfscope}%
\end{pgfpicture}%
\makeatother%
\endgroup%

        } % scalebox
        \caption{Time domain}
        \label{fig:flicker_noise_time}
    \end{subfigure}
    \begin{subfigure}{0.32\linewidth}
        \centering
        \scalebox{0.75}{%
            %% Creator: Matplotlib, PGF backend
%%
%% To include the figure in your LaTeX document, write
%%   \input{<filename>.pgf}
%%
%% Make sure the required packages are loaded in your preamble
%%   \usepackage{pgf}
%%
%% Also ensure that all the required font packages are loaded; for instance,
%% the lmodern package is sometimes necessary when using math font.
%%   \usepackage{lmodern}
%%
%% Figures using additional raster images can only be included by \input if
%% they are in the same directory as the main LaTeX file. For loading figures
%% from other directories you can use the `import` package
%%   \usepackage{import}
%%
%% and then include the figures with
%%   \import{<path to file>}{<filename>.pgf}
%%
%% Matplotlib used the following preamble
%%   \usepackage{siunitx}
%%   \usepackage{fontspec}
%%   \makeatletter\@ifpackageloaded{underscore}{}{\usepackage[strings]{underscore}}\makeatother
%%
\begingroup%
\makeatletter%
\begin{pgfpicture}%
\pgfpathrectangle{\pgfpointorigin}{\pgfqpoint{2.440945in}{1.830709in}}%
\pgfusepath{use as bounding box, clip}%
\begin{pgfscope}%
\pgfsetbuttcap%
\pgfsetmiterjoin%
\definecolor{currentfill}{rgb}{1.000000,1.000000,1.000000}%
\pgfsetfillcolor{currentfill}%
\pgfsetlinewidth{0.000000pt}%
\definecolor{currentstroke}{rgb}{1.000000,1.000000,1.000000}%
\pgfsetstrokecolor{currentstroke}%
\pgfsetdash{}{0pt}%
\pgfpathmoveto{\pgfqpoint{0.000000in}{0.000000in}}%
\pgfpathlineto{\pgfqpoint{2.440945in}{0.000000in}}%
\pgfpathlineto{\pgfqpoint{2.440945in}{1.830709in}}%
\pgfpathlineto{\pgfqpoint{0.000000in}{1.830709in}}%
\pgfpathlineto{\pgfqpoint{0.000000in}{0.000000in}}%
\pgfpathclose%
\pgfusepath{fill}%
\end{pgfscope}%
\begin{pgfscope}%
\pgfsetbuttcap%
\pgfsetmiterjoin%
\definecolor{currentfill}{rgb}{1.000000,1.000000,1.000000}%
\pgfsetfillcolor{currentfill}%
\pgfsetlinewidth{0.000000pt}%
\definecolor{currentstroke}{rgb}{0.000000,0.000000,0.000000}%
\pgfsetstrokecolor{currentstroke}%
\pgfsetstrokeopacity{0.000000}%
\pgfsetdash{}{0pt}%
\pgfpathmoveto{\pgfqpoint{0.514278in}{0.417642in}}%
\pgfpathlineto{\pgfqpoint{2.399275in}{0.417642in}}%
\pgfpathlineto{\pgfqpoint{2.399275in}{1.789039in}}%
\pgfpathlineto{\pgfqpoint{0.514278in}{1.789039in}}%
\pgfpathlineto{\pgfqpoint{0.514278in}{0.417642in}}%
\pgfpathclose%
\pgfusepath{fill}%
\end{pgfscope}%
\begin{pgfscope}%
\pgfpathrectangle{\pgfqpoint{0.514278in}{0.417642in}}{\pgfqpoint{1.884996in}{1.371397in}}%
\pgfusepath{clip}%
\pgfsetrectcap%
\pgfsetroundjoin%
\pgfsetlinewidth{0.803000pt}%
\definecolor{currentstroke}{rgb}{0.450000,0.450000,0.450000}%
\pgfsetstrokecolor{currentstroke}%
\pgfsetdash{}{0pt}%
\pgfpathmoveto{\pgfqpoint{0.916826in}{0.417642in}}%
\pgfpathlineto{\pgfqpoint{0.916826in}{1.789039in}}%
\pgfusepath{stroke}%
\end{pgfscope}%
\begin{pgfscope}%
\pgfsetbuttcap%
\pgfsetroundjoin%
\definecolor{currentfill}{rgb}{0.000000,0.000000,0.000000}%
\pgfsetfillcolor{currentfill}%
\pgfsetlinewidth{0.803000pt}%
\definecolor{currentstroke}{rgb}{0.000000,0.000000,0.000000}%
\pgfsetstrokecolor{currentstroke}%
\pgfsetdash{}{0pt}%
\pgfsys@defobject{currentmarker}{\pgfqpoint{0.000000in}{-0.048611in}}{\pgfqpoint{0.000000in}{0.000000in}}{%
\pgfpathmoveto{\pgfqpoint{0.000000in}{0.000000in}}%
\pgfpathlineto{\pgfqpoint{0.000000in}{-0.048611in}}%
\pgfusepath{stroke,fill}%
}%
\begin{pgfscope}%
\pgfsys@transformshift{0.916826in}{0.417642in}%
\pgfsys@useobject{currentmarker}{}%
\end{pgfscope}%
\end{pgfscope}%
\begin{pgfscope}%
\definecolor{textcolor}{rgb}{0.000000,0.000000,0.000000}%
\pgfsetstrokecolor{textcolor}%
\pgfsetfillcolor{textcolor}%
\pgftext[x=0.916826in,y=0.320420in,,top]{\color{textcolor}\rmfamily\fontsize{8.000000}{9.600000}\selectfont \(\displaystyle {10^{-3}}\)}%
\end{pgfscope}%
\begin{pgfscope}%
\pgfpathrectangle{\pgfqpoint{0.514278in}{0.417642in}}{\pgfqpoint{1.884996in}{1.371397in}}%
\pgfusepath{clip}%
\pgfsetrectcap%
\pgfsetroundjoin%
\pgfsetlinewidth{0.803000pt}%
\definecolor{currentstroke}{rgb}{0.450000,0.450000,0.450000}%
\pgfsetstrokecolor{currentstroke}%
\pgfsetdash{}{0pt}%
\pgfpathmoveto{\pgfqpoint{1.434365in}{0.417642in}}%
\pgfpathlineto{\pgfqpoint{1.434365in}{1.789039in}}%
\pgfusepath{stroke}%
\end{pgfscope}%
\begin{pgfscope}%
\pgfsetbuttcap%
\pgfsetroundjoin%
\definecolor{currentfill}{rgb}{0.000000,0.000000,0.000000}%
\pgfsetfillcolor{currentfill}%
\pgfsetlinewidth{0.803000pt}%
\definecolor{currentstroke}{rgb}{0.000000,0.000000,0.000000}%
\pgfsetstrokecolor{currentstroke}%
\pgfsetdash{}{0pt}%
\pgfsys@defobject{currentmarker}{\pgfqpoint{0.000000in}{-0.048611in}}{\pgfqpoint{0.000000in}{0.000000in}}{%
\pgfpathmoveto{\pgfqpoint{0.000000in}{0.000000in}}%
\pgfpathlineto{\pgfqpoint{0.000000in}{-0.048611in}}%
\pgfusepath{stroke,fill}%
}%
\begin{pgfscope}%
\pgfsys@transformshift{1.434365in}{0.417642in}%
\pgfsys@useobject{currentmarker}{}%
\end{pgfscope}%
\end{pgfscope}%
\begin{pgfscope}%
\definecolor{textcolor}{rgb}{0.000000,0.000000,0.000000}%
\pgfsetstrokecolor{textcolor}%
\pgfsetfillcolor{textcolor}%
\pgftext[x=1.434365in,y=0.320420in,,top]{\color{textcolor}\rmfamily\fontsize{8.000000}{9.600000}\selectfont \(\displaystyle {10^{-2}}\)}%
\end{pgfscope}%
\begin{pgfscope}%
\pgfpathrectangle{\pgfqpoint{0.514278in}{0.417642in}}{\pgfqpoint{1.884996in}{1.371397in}}%
\pgfusepath{clip}%
\pgfsetrectcap%
\pgfsetroundjoin%
\pgfsetlinewidth{0.803000pt}%
\definecolor{currentstroke}{rgb}{0.450000,0.450000,0.450000}%
\pgfsetstrokecolor{currentstroke}%
\pgfsetdash{}{0pt}%
\pgfpathmoveto{\pgfqpoint{1.951904in}{0.417642in}}%
\pgfpathlineto{\pgfqpoint{1.951904in}{1.789039in}}%
\pgfusepath{stroke}%
\end{pgfscope}%
\begin{pgfscope}%
\pgfsetbuttcap%
\pgfsetroundjoin%
\definecolor{currentfill}{rgb}{0.000000,0.000000,0.000000}%
\pgfsetfillcolor{currentfill}%
\pgfsetlinewidth{0.803000pt}%
\definecolor{currentstroke}{rgb}{0.000000,0.000000,0.000000}%
\pgfsetstrokecolor{currentstroke}%
\pgfsetdash{}{0pt}%
\pgfsys@defobject{currentmarker}{\pgfqpoint{0.000000in}{-0.048611in}}{\pgfqpoint{0.000000in}{0.000000in}}{%
\pgfpathmoveto{\pgfqpoint{0.000000in}{0.000000in}}%
\pgfpathlineto{\pgfqpoint{0.000000in}{-0.048611in}}%
\pgfusepath{stroke,fill}%
}%
\begin{pgfscope}%
\pgfsys@transformshift{1.951904in}{0.417642in}%
\pgfsys@useobject{currentmarker}{}%
\end{pgfscope}%
\end{pgfscope}%
\begin{pgfscope}%
\definecolor{textcolor}{rgb}{0.000000,0.000000,0.000000}%
\pgfsetstrokecolor{textcolor}%
\pgfsetfillcolor{textcolor}%
\pgftext[x=1.951904in,y=0.320420in,,top]{\color{textcolor}\rmfamily\fontsize{8.000000}{9.600000}\selectfont \(\displaystyle {10^{-1}}\)}%
\end{pgfscope}%
\begin{pgfscope}%
\pgfpathrectangle{\pgfqpoint{0.514278in}{0.417642in}}{\pgfqpoint{1.884996in}{1.371397in}}%
\pgfusepath{clip}%
\pgfsetrectcap%
\pgfsetroundjoin%
\pgfsetlinewidth{0.803000pt}%
\definecolor{currentstroke}{rgb}{0.850000,0.850000,0.850000}%
\pgfsetstrokecolor{currentstroke}%
\pgfsetdash{}{0pt}%
\pgfpathmoveto{\pgfqpoint{0.555081in}{0.417642in}}%
\pgfpathlineto{\pgfqpoint{0.555081in}{1.789039in}}%
\pgfusepath{stroke}%
\end{pgfscope}%
\begin{pgfscope}%
\pgfsetbuttcap%
\pgfsetroundjoin%
\definecolor{currentfill}{rgb}{0.000000,0.000000,0.000000}%
\pgfsetfillcolor{currentfill}%
\pgfsetlinewidth{0.602250pt}%
\definecolor{currentstroke}{rgb}{0.000000,0.000000,0.000000}%
\pgfsetstrokecolor{currentstroke}%
\pgfsetdash{}{0pt}%
\pgfsys@defobject{currentmarker}{\pgfqpoint{0.000000in}{-0.027778in}}{\pgfqpoint{0.000000in}{0.000000in}}{%
\pgfpathmoveto{\pgfqpoint{0.000000in}{0.000000in}}%
\pgfpathlineto{\pgfqpoint{0.000000in}{-0.027778in}}%
\pgfusepath{stroke,fill}%
}%
\begin{pgfscope}%
\pgfsys@transformshift{0.555081in}{0.417642in}%
\pgfsys@useobject{currentmarker}{}%
\end{pgfscope}%
\end{pgfscope}%
\begin{pgfscope}%
\pgfpathrectangle{\pgfqpoint{0.514278in}{0.417642in}}{\pgfqpoint{1.884996in}{1.371397in}}%
\pgfusepath{clip}%
\pgfsetrectcap%
\pgfsetroundjoin%
\pgfsetlinewidth{0.803000pt}%
\definecolor{currentstroke}{rgb}{0.850000,0.850000,0.850000}%
\pgfsetstrokecolor{currentstroke}%
\pgfsetdash{}{0pt}%
\pgfpathmoveto{\pgfqpoint{0.646215in}{0.417642in}}%
\pgfpathlineto{\pgfqpoint{0.646215in}{1.789039in}}%
\pgfusepath{stroke}%
\end{pgfscope}%
\begin{pgfscope}%
\pgfsetbuttcap%
\pgfsetroundjoin%
\definecolor{currentfill}{rgb}{0.000000,0.000000,0.000000}%
\pgfsetfillcolor{currentfill}%
\pgfsetlinewidth{0.602250pt}%
\definecolor{currentstroke}{rgb}{0.000000,0.000000,0.000000}%
\pgfsetstrokecolor{currentstroke}%
\pgfsetdash{}{0pt}%
\pgfsys@defobject{currentmarker}{\pgfqpoint{0.000000in}{-0.027778in}}{\pgfqpoint{0.000000in}{0.000000in}}{%
\pgfpathmoveto{\pgfqpoint{0.000000in}{0.000000in}}%
\pgfpathlineto{\pgfqpoint{0.000000in}{-0.027778in}}%
\pgfusepath{stroke,fill}%
}%
\begin{pgfscope}%
\pgfsys@transformshift{0.646215in}{0.417642in}%
\pgfsys@useobject{currentmarker}{}%
\end{pgfscope}%
\end{pgfscope}%
\begin{pgfscope}%
\pgfpathrectangle{\pgfqpoint{0.514278in}{0.417642in}}{\pgfqpoint{1.884996in}{1.371397in}}%
\pgfusepath{clip}%
\pgfsetrectcap%
\pgfsetroundjoin%
\pgfsetlinewidth{0.803000pt}%
\definecolor{currentstroke}{rgb}{0.850000,0.850000,0.850000}%
\pgfsetstrokecolor{currentstroke}%
\pgfsetdash{}{0pt}%
\pgfpathmoveto{\pgfqpoint{0.710876in}{0.417642in}}%
\pgfpathlineto{\pgfqpoint{0.710876in}{1.789039in}}%
\pgfusepath{stroke}%
\end{pgfscope}%
\begin{pgfscope}%
\pgfsetbuttcap%
\pgfsetroundjoin%
\definecolor{currentfill}{rgb}{0.000000,0.000000,0.000000}%
\pgfsetfillcolor{currentfill}%
\pgfsetlinewidth{0.602250pt}%
\definecolor{currentstroke}{rgb}{0.000000,0.000000,0.000000}%
\pgfsetstrokecolor{currentstroke}%
\pgfsetdash{}{0pt}%
\pgfsys@defobject{currentmarker}{\pgfqpoint{0.000000in}{-0.027778in}}{\pgfqpoint{0.000000in}{0.000000in}}{%
\pgfpathmoveto{\pgfqpoint{0.000000in}{0.000000in}}%
\pgfpathlineto{\pgfqpoint{0.000000in}{-0.027778in}}%
\pgfusepath{stroke,fill}%
}%
\begin{pgfscope}%
\pgfsys@transformshift{0.710876in}{0.417642in}%
\pgfsys@useobject{currentmarker}{}%
\end{pgfscope}%
\end{pgfscope}%
\begin{pgfscope}%
\pgfpathrectangle{\pgfqpoint{0.514278in}{0.417642in}}{\pgfqpoint{1.884996in}{1.371397in}}%
\pgfusepath{clip}%
\pgfsetrectcap%
\pgfsetroundjoin%
\pgfsetlinewidth{0.803000pt}%
\definecolor{currentstroke}{rgb}{0.850000,0.850000,0.850000}%
\pgfsetstrokecolor{currentstroke}%
\pgfsetdash{}{0pt}%
\pgfpathmoveto{\pgfqpoint{0.761031in}{0.417642in}}%
\pgfpathlineto{\pgfqpoint{0.761031in}{1.789039in}}%
\pgfusepath{stroke}%
\end{pgfscope}%
\begin{pgfscope}%
\pgfsetbuttcap%
\pgfsetroundjoin%
\definecolor{currentfill}{rgb}{0.000000,0.000000,0.000000}%
\pgfsetfillcolor{currentfill}%
\pgfsetlinewidth{0.602250pt}%
\definecolor{currentstroke}{rgb}{0.000000,0.000000,0.000000}%
\pgfsetstrokecolor{currentstroke}%
\pgfsetdash{}{0pt}%
\pgfsys@defobject{currentmarker}{\pgfqpoint{0.000000in}{-0.027778in}}{\pgfqpoint{0.000000in}{0.000000in}}{%
\pgfpathmoveto{\pgfqpoint{0.000000in}{0.000000in}}%
\pgfpathlineto{\pgfqpoint{0.000000in}{-0.027778in}}%
\pgfusepath{stroke,fill}%
}%
\begin{pgfscope}%
\pgfsys@transformshift{0.761031in}{0.417642in}%
\pgfsys@useobject{currentmarker}{}%
\end{pgfscope}%
\end{pgfscope}%
\begin{pgfscope}%
\pgfpathrectangle{\pgfqpoint{0.514278in}{0.417642in}}{\pgfqpoint{1.884996in}{1.371397in}}%
\pgfusepath{clip}%
\pgfsetrectcap%
\pgfsetroundjoin%
\pgfsetlinewidth{0.803000pt}%
\definecolor{currentstroke}{rgb}{0.850000,0.850000,0.850000}%
\pgfsetstrokecolor{currentstroke}%
\pgfsetdash{}{0pt}%
\pgfpathmoveto{\pgfqpoint{0.802010in}{0.417642in}}%
\pgfpathlineto{\pgfqpoint{0.802010in}{1.789039in}}%
\pgfusepath{stroke}%
\end{pgfscope}%
\begin{pgfscope}%
\pgfsetbuttcap%
\pgfsetroundjoin%
\definecolor{currentfill}{rgb}{0.000000,0.000000,0.000000}%
\pgfsetfillcolor{currentfill}%
\pgfsetlinewidth{0.602250pt}%
\definecolor{currentstroke}{rgb}{0.000000,0.000000,0.000000}%
\pgfsetstrokecolor{currentstroke}%
\pgfsetdash{}{0pt}%
\pgfsys@defobject{currentmarker}{\pgfqpoint{0.000000in}{-0.027778in}}{\pgfqpoint{0.000000in}{0.000000in}}{%
\pgfpathmoveto{\pgfqpoint{0.000000in}{0.000000in}}%
\pgfpathlineto{\pgfqpoint{0.000000in}{-0.027778in}}%
\pgfusepath{stroke,fill}%
}%
\begin{pgfscope}%
\pgfsys@transformshift{0.802010in}{0.417642in}%
\pgfsys@useobject{currentmarker}{}%
\end{pgfscope}%
\end{pgfscope}%
\begin{pgfscope}%
\pgfpathrectangle{\pgfqpoint{0.514278in}{0.417642in}}{\pgfqpoint{1.884996in}{1.371397in}}%
\pgfusepath{clip}%
\pgfsetrectcap%
\pgfsetroundjoin%
\pgfsetlinewidth{0.803000pt}%
\definecolor{currentstroke}{rgb}{0.850000,0.850000,0.850000}%
\pgfsetstrokecolor{currentstroke}%
\pgfsetdash{}{0pt}%
\pgfpathmoveto{\pgfqpoint{0.836658in}{0.417642in}}%
\pgfpathlineto{\pgfqpoint{0.836658in}{1.789039in}}%
\pgfusepath{stroke}%
\end{pgfscope}%
\begin{pgfscope}%
\pgfsetbuttcap%
\pgfsetroundjoin%
\definecolor{currentfill}{rgb}{0.000000,0.000000,0.000000}%
\pgfsetfillcolor{currentfill}%
\pgfsetlinewidth{0.602250pt}%
\definecolor{currentstroke}{rgb}{0.000000,0.000000,0.000000}%
\pgfsetstrokecolor{currentstroke}%
\pgfsetdash{}{0pt}%
\pgfsys@defobject{currentmarker}{\pgfqpoint{0.000000in}{-0.027778in}}{\pgfqpoint{0.000000in}{0.000000in}}{%
\pgfpathmoveto{\pgfqpoint{0.000000in}{0.000000in}}%
\pgfpathlineto{\pgfqpoint{0.000000in}{-0.027778in}}%
\pgfusepath{stroke,fill}%
}%
\begin{pgfscope}%
\pgfsys@transformshift{0.836658in}{0.417642in}%
\pgfsys@useobject{currentmarker}{}%
\end{pgfscope}%
\end{pgfscope}%
\begin{pgfscope}%
\pgfpathrectangle{\pgfqpoint{0.514278in}{0.417642in}}{\pgfqpoint{1.884996in}{1.371397in}}%
\pgfusepath{clip}%
\pgfsetrectcap%
\pgfsetroundjoin%
\pgfsetlinewidth{0.803000pt}%
\definecolor{currentstroke}{rgb}{0.850000,0.850000,0.850000}%
\pgfsetstrokecolor{currentstroke}%
\pgfsetdash{}{0pt}%
\pgfpathmoveto{\pgfqpoint{0.866671in}{0.417642in}}%
\pgfpathlineto{\pgfqpoint{0.866671in}{1.789039in}}%
\pgfusepath{stroke}%
\end{pgfscope}%
\begin{pgfscope}%
\pgfsetbuttcap%
\pgfsetroundjoin%
\definecolor{currentfill}{rgb}{0.000000,0.000000,0.000000}%
\pgfsetfillcolor{currentfill}%
\pgfsetlinewidth{0.602250pt}%
\definecolor{currentstroke}{rgb}{0.000000,0.000000,0.000000}%
\pgfsetstrokecolor{currentstroke}%
\pgfsetdash{}{0pt}%
\pgfsys@defobject{currentmarker}{\pgfqpoint{0.000000in}{-0.027778in}}{\pgfqpoint{0.000000in}{0.000000in}}{%
\pgfpathmoveto{\pgfqpoint{0.000000in}{0.000000in}}%
\pgfpathlineto{\pgfqpoint{0.000000in}{-0.027778in}}%
\pgfusepath{stroke,fill}%
}%
\begin{pgfscope}%
\pgfsys@transformshift{0.866671in}{0.417642in}%
\pgfsys@useobject{currentmarker}{}%
\end{pgfscope}%
\end{pgfscope}%
\begin{pgfscope}%
\pgfpathrectangle{\pgfqpoint{0.514278in}{0.417642in}}{\pgfqpoint{1.884996in}{1.371397in}}%
\pgfusepath{clip}%
\pgfsetrectcap%
\pgfsetroundjoin%
\pgfsetlinewidth{0.803000pt}%
\definecolor{currentstroke}{rgb}{0.850000,0.850000,0.850000}%
\pgfsetstrokecolor{currentstroke}%
\pgfsetdash{}{0pt}%
\pgfpathmoveto{\pgfqpoint{0.893144in}{0.417642in}}%
\pgfpathlineto{\pgfqpoint{0.893144in}{1.789039in}}%
\pgfusepath{stroke}%
\end{pgfscope}%
\begin{pgfscope}%
\pgfsetbuttcap%
\pgfsetroundjoin%
\definecolor{currentfill}{rgb}{0.000000,0.000000,0.000000}%
\pgfsetfillcolor{currentfill}%
\pgfsetlinewidth{0.602250pt}%
\definecolor{currentstroke}{rgb}{0.000000,0.000000,0.000000}%
\pgfsetstrokecolor{currentstroke}%
\pgfsetdash{}{0pt}%
\pgfsys@defobject{currentmarker}{\pgfqpoint{0.000000in}{-0.027778in}}{\pgfqpoint{0.000000in}{0.000000in}}{%
\pgfpathmoveto{\pgfqpoint{0.000000in}{0.000000in}}%
\pgfpathlineto{\pgfqpoint{0.000000in}{-0.027778in}}%
\pgfusepath{stroke,fill}%
}%
\begin{pgfscope}%
\pgfsys@transformshift{0.893144in}{0.417642in}%
\pgfsys@useobject{currentmarker}{}%
\end{pgfscope}%
\end{pgfscope}%
\begin{pgfscope}%
\pgfpathrectangle{\pgfqpoint{0.514278in}{0.417642in}}{\pgfqpoint{1.884996in}{1.371397in}}%
\pgfusepath{clip}%
\pgfsetrectcap%
\pgfsetroundjoin%
\pgfsetlinewidth{0.803000pt}%
\definecolor{currentstroke}{rgb}{0.850000,0.850000,0.850000}%
\pgfsetstrokecolor{currentstroke}%
\pgfsetdash{}{0pt}%
\pgfpathmoveto{\pgfqpoint{1.072620in}{0.417642in}}%
\pgfpathlineto{\pgfqpoint{1.072620in}{1.789039in}}%
\pgfusepath{stroke}%
\end{pgfscope}%
\begin{pgfscope}%
\pgfsetbuttcap%
\pgfsetroundjoin%
\definecolor{currentfill}{rgb}{0.000000,0.000000,0.000000}%
\pgfsetfillcolor{currentfill}%
\pgfsetlinewidth{0.602250pt}%
\definecolor{currentstroke}{rgb}{0.000000,0.000000,0.000000}%
\pgfsetstrokecolor{currentstroke}%
\pgfsetdash{}{0pt}%
\pgfsys@defobject{currentmarker}{\pgfqpoint{0.000000in}{-0.027778in}}{\pgfqpoint{0.000000in}{0.000000in}}{%
\pgfpathmoveto{\pgfqpoint{0.000000in}{0.000000in}}%
\pgfpathlineto{\pgfqpoint{0.000000in}{-0.027778in}}%
\pgfusepath{stroke,fill}%
}%
\begin{pgfscope}%
\pgfsys@transformshift{1.072620in}{0.417642in}%
\pgfsys@useobject{currentmarker}{}%
\end{pgfscope}%
\end{pgfscope}%
\begin{pgfscope}%
\pgfpathrectangle{\pgfqpoint{0.514278in}{0.417642in}}{\pgfqpoint{1.884996in}{1.371397in}}%
\pgfusepath{clip}%
\pgfsetrectcap%
\pgfsetroundjoin%
\pgfsetlinewidth{0.803000pt}%
\definecolor{currentstroke}{rgb}{0.850000,0.850000,0.850000}%
\pgfsetstrokecolor{currentstroke}%
\pgfsetdash{}{0pt}%
\pgfpathmoveto{\pgfqpoint{1.163754in}{0.417642in}}%
\pgfpathlineto{\pgfqpoint{1.163754in}{1.789039in}}%
\pgfusepath{stroke}%
\end{pgfscope}%
\begin{pgfscope}%
\pgfsetbuttcap%
\pgfsetroundjoin%
\definecolor{currentfill}{rgb}{0.000000,0.000000,0.000000}%
\pgfsetfillcolor{currentfill}%
\pgfsetlinewidth{0.602250pt}%
\definecolor{currentstroke}{rgb}{0.000000,0.000000,0.000000}%
\pgfsetstrokecolor{currentstroke}%
\pgfsetdash{}{0pt}%
\pgfsys@defobject{currentmarker}{\pgfqpoint{0.000000in}{-0.027778in}}{\pgfqpoint{0.000000in}{0.000000in}}{%
\pgfpathmoveto{\pgfqpoint{0.000000in}{0.000000in}}%
\pgfpathlineto{\pgfqpoint{0.000000in}{-0.027778in}}%
\pgfusepath{stroke,fill}%
}%
\begin{pgfscope}%
\pgfsys@transformshift{1.163754in}{0.417642in}%
\pgfsys@useobject{currentmarker}{}%
\end{pgfscope}%
\end{pgfscope}%
\begin{pgfscope}%
\pgfpathrectangle{\pgfqpoint{0.514278in}{0.417642in}}{\pgfqpoint{1.884996in}{1.371397in}}%
\pgfusepath{clip}%
\pgfsetrectcap%
\pgfsetroundjoin%
\pgfsetlinewidth{0.803000pt}%
\definecolor{currentstroke}{rgb}{0.850000,0.850000,0.850000}%
\pgfsetstrokecolor{currentstroke}%
\pgfsetdash{}{0pt}%
\pgfpathmoveto{\pgfqpoint{1.228415in}{0.417642in}}%
\pgfpathlineto{\pgfqpoint{1.228415in}{1.789039in}}%
\pgfusepath{stroke}%
\end{pgfscope}%
\begin{pgfscope}%
\pgfsetbuttcap%
\pgfsetroundjoin%
\definecolor{currentfill}{rgb}{0.000000,0.000000,0.000000}%
\pgfsetfillcolor{currentfill}%
\pgfsetlinewidth{0.602250pt}%
\definecolor{currentstroke}{rgb}{0.000000,0.000000,0.000000}%
\pgfsetstrokecolor{currentstroke}%
\pgfsetdash{}{0pt}%
\pgfsys@defobject{currentmarker}{\pgfqpoint{0.000000in}{-0.027778in}}{\pgfqpoint{0.000000in}{0.000000in}}{%
\pgfpathmoveto{\pgfqpoint{0.000000in}{0.000000in}}%
\pgfpathlineto{\pgfqpoint{0.000000in}{-0.027778in}}%
\pgfusepath{stroke,fill}%
}%
\begin{pgfscope}%
\pgfsys@transformshift{1.228415in}{0.417642in}%
\pgfsys@useobject{currentmarker}{}%
\end{pgfscope}%
\end{pgfscope}%
\begin{pgfscope}%
\pgfpathrectangle{\pgfqpoint{0.514278in}{0.417642in}}{\pgfqpoint{1.884996in}{1.371397in}}%
\pgfusepath{clip}%
\pgfsetrectcap%
\pgfsetroundjoin%
\pgfsetlinewidth{0.803000pt}%
\definecolor{currentstroke}{rgb}{0.850000,0.850000,0.850000}%
\pgfsetstrokecolor{currentstroke}%
\pgfsetdash{}{0pt}%
\pgfpathmoveto{\pgfqpoint{1.278570in}{0.417642in}}%
\pgfpathlineto{\pgfqpoint{1.278570in}{1.789039in}}%
\pgfusepath{stroke}%
\end{pgfscope}%
\begin{pgfscope}%
\pgfsetbuttcap%
\pgfsetroundjoin%
\definecolor{currentfill}{rgb}{0.000000,0.000000,0.000000}%
\pgfsetfillcolor{currentfill}%
\pgfsetlinewidth{0.602250pt}%
\definecolor{currentstroke}{rgb}{0.000000,0.000000,0.000000}%
\pgfsetstrokecolor{currentstroke}%
\pgfsetdash{}{0pt}%
\pgfsys@defobject{currentmarker}{\pgfqpoint{0.000000in}{-0.027778in}}{\pgfqpoint{0.000000in}{0.000000in}}{%
\pgfpathmoveto{\pgfqpoint{0.000000in}{0.000000in}}%
\pgfpathlineto{\pgfqpoint{0.000000in}{-0.027778in}}%
\pgfusepath{stroke,fill}%
}%
\begin{pgfscope}%
\pgfsys@transformshift{1.278570in}{0.417642in}%
\pgfsys@useobject{currentmarker}{}%
\end{pgfscope}%
\end{pgfscope}%
\begin{pgfscope}%
\pgfpathrectangle{\pgfqpoint{0.514278in}{0.417642in}}{\pgfqpoint{1.884996in}{1.371397in}}%
\pgfusepath{clip}%
\pgfsetrectcap%
\pgfsetroundjoin%
\pgfsetlinewidth{0.803000pt}%
\definecolor{currentstroke}{rgb}{0.850000,0.850000,0.850000}%
\pgfsetstrokecolor{currentstroke}%
\pgfsetdash{}{0pt}%
\pgfpathmoveto{\pgfqpoint{1.319549in}{0.417642in}}%
\pgfpathlineto{\pgfqpoint{1.319549in}{1.789039in}}%
\pgfusepath{stroke}%
\end{pgfscope}%
\begin{pgfscope}%
\pgfsetbuttcap%
\pgfsetroundjoin%
\definecolor{currentfill}{rgb}{0.000000,0.000000,0.000000}%
\pgfsetfillcolor{currentfill}%
\pgfsetlinewidth{0.602250pt}%
\definecolor{currentstroke}{rgb}{0.000000,0.000000,0.000000}%
\pgfsetstrokecolor{currentstroke}%
\pgfsetdash{}{0pt}%
\pgfsys@defobject{currentmarker}{\pgfqpoint{0.000000in}{-0.027778in}}{\pgfqpoint{0.000000in}{0.000000in}}{%
\pgfpathmoveto{\pgfqpoint{0.000000in}{0.000000in}}%
\pgfpathlineto{\pgfqpoint{0.000000in}{-0.027778in}}%
\pgfusepath{stroke,fill}%
}%
\begin{pgfscope}%
\pgfsys@transformshift{1.319549in}{0.417642in}%
\pgfsys@useobject{currentmarker}{}%
\end{pgfscope}%
\end{pgfscope}%
\begin{pgfscope}%
\pgfpathrectangle{\pgfqpoint{0.514278in}{0.417642in}}{\pgfqpoint{1.884996in}{1.371397in}}%
\pgfusepath{clip}%
\pgfsetrectcap%
\pgfsetroundjoin%
\pgfsetlinewidth{0.803000pt}%
\definecolor{currentstroke}{rgb}{0.850000,0.850000,0.850000}%
\pgfsetstrokecolor{currentstroke}%
\pgfsetdash{}{0pt}%
\pgfpathmoveto{\pgfqpoint{1.354197in}{0.417642in}}%
\pgfpathlineto{\pgfqpoint{1.354197in}{1.789039in}}%
\pgfusepath{stroke}%
\end{pgfscope}%
\begin{pgfscope}%
\pgfsetbuttcap%
\pgfsetroundjoin%
\definecolor{currentfill}{rgb}{0.000000,0.000000,0.000000}%
\pgfsetfillcolor{currentfill}%
\pgfsetlinewidth{0.602250pt}%
\definecolor{currentstroke}{rgb}{0.000000,0.000000,0.000000}%
\pgfsetstrokecolor{currentstroke}%
\pgfsetdash{}{0pt}%
\pgfsys@defobject{currentmarker}{\pgfqpoint{0.000000in}{-0.027778in}}{\pgfqpoint{0.000000in}{0.000000in}}{%
\pgfpathmoveto{\pgfqpoint{0.000000in}{0.000000in}}%
\pgfpathlineto{\pgfqpoint{0.000000in}{-0.027778in}}%
\pgfusepath{stroke,fill}%
}%
\begin{pgfscope}%
\pgfsys@transformshift{1.354197in}{0.417642in}%
\pgfsys@useobject{currentmarker}{}%
\end{pgfscope}%
\end{pgfscope}%
\begin{pgfscope}%
\pgfpathrectangle{\pgfqpoint{0.514278in}{0.417642in}}{\pgfqpoint{1.884996in}{1.371397in}}%
\pgfusepath{clip}%
\pgfsetrectcap%
\pgfsetroundjoin%
\pgfsetlinewidth{0.803000pt}%
\definecolor{currentstroke}{rgb}{0.850000,0.850000,0.850000}%
\pgfsetstrokecolor{currentstroke}%
\pgfsetdash{}{0pt}%
\pgfpathmoveto{\pgfqpoint{1.384210in}{0.417642in}}%
\pgfpathlineto{\pgfqpoint{1.384210in}{1.789039in}}%
\pgfusepath{stroke}%
\end{pgfscope}%
\begin{pgfscope}%
\pgfsetbuttcap%
\pgfsetroundjoin%
\definecolor{currentfill}{rgb}{0.000000,0.000000,0.000000}%
\pgfsetfillcolor{currentfill}%
\pgfsetlinewidth{0.602250pt}%
\definecolor{currentstroke}{rgb}{0.000000,0.000000,0.000000}%
\pgfsetstrokecolor{currentstroke}%
\pgfsetdash{}{0pt}%
\pgfsys@defobject{currentmarker}{\pgfqpoint{0.000000in}{-0.027778in}}{\pgfqpoint{0.000000in}{0.000000in}}{%
\pgfpathmoveto{\pgfqpoint{0.000000in}{0.000000in}}%
\pgfpathlineto{\pgfqpoint{0.000000in}{-0.027778in}}%
\pgfusepath{stroke,fill}%
}%
\begin{pgfscope}%
\pgfsys@transformshift{1.384210in}{0.417642in}%
\pgfsys@useobject{currentmarker}{}%
\end{pgfscope}%
\end{pgfscope}%
\begin{pgfscope}%
\pgfpathrectangle{\pgfqpoint{0.514278in}{0.417642in}}{\pgfqpoint{1.884996in}{1.371397in}}%
\pgfusepath{clip}%
\pgfsetrectcap%
\pgfsetroundjoin%
\pgfsetlinewidth{0.803000pt}%
\definecolor{currentstroke}{rgb}{0.850000,0.850000,0.850000}%
\pgfsetstrokecolor{currentstroke}%
\pgfsetdash{}{0pt}%
\pgfpathmoveto{\pgfqpoint{1.410683in}{0.417642in}}%
\pgfpathlineto{\pgfqpoint{1.410683in}{1.789039in}}%
\pgfusepath{stroke}%
\end{pgfscope}%
\begin{pgfscope}%
\pgfsetbuttcap%
\pgfsetroundjoin%
\definecolor{currentfill}{rgb}{0.000000,0.000000,0.000000}%
\pgfsetfillcolor{currentfill}%
\pgfsetlinewidth{0.602250pt}%
\definecolor{currentstroke}{rgb}{0.000000,0.000000,0.000000}%
\pgfsetstrokecolor{currentstroke}%
\pgfsetdash{}{0pt}%
\pgfsys@defobject{currentmarker}{\pgfqpoint{0.000000in}{-0.027778in}}{\pgfqpoint{0.000000in}{0.000000in}}{%
\pgfpathmoveto{\pgfqpoint{0.000000in}{0.000000in}}%
\pgfpathlineto{\pgfqpoint{0.000000in}{-0.027778in}}%
\pgfusepath{stroke,fill}%
}%
\begin{pgfscope}%
\pgfsys@transformshift{1.410683in}{0.417642in}%
\pgfsys@useobject{currentmarker}{}%
\end{pgfscope}%
\end{pgfscope}%
\begin{pgfscope}%
\pgfpathrectangle{\pgfqpoint{0.514278in}{0.417642in}}{\pgfqpoint{1.884996in}{1.371397in}}%
\pgfusepath{clip}%
\pgfsetrectcap%
\pgfsetroundjoin%
\pgfsetlinewidth{0.803000pt}%
\definecolor{currentstroke}{rgb}{0.850000,0.850000,0.850000}%
\pgfsetstrokecolor{currentstroke}%
\pgfsetdash{}{0pt}%
\pgfpathmoveto{\pgfqpoint{1.590159in}{0.417642in}}%
\pgfpathlineto{\pgfqpoint{1.590159in}{1.789039in}}%
\pgfusepath{stroke}%
\end{pgfscope}%
\begin{pgfscope}%
\pgfsetbuttcap%
\pgfsetroundjoin%
\definecolor{currentfill}{rgb}{0.000000,0.000000,0.000000}%
\pgfsetfillcolor{currentfill}%
\pgfsetlinewidth{0.602250pt}%
\definecolor{currentstroke}{rgb}{0.000000,0.000000,0.000000}%
\pgfsetstrokecolor{currentstroke}%
\pgfsetdash{}{0pt}%
\pgfsys@defobject{currentmarker}{\pgfqpoint{0.000000in}{-0.027778in}}{\pgfqpoint{0.000000in}{0.000000in}}{%
\pgfpathmoveto{\pgfqpoint{0.000000in}{0.000000in}}%
\pgfpathlineto{\pgfqpoint{0.000000in}{-0.027778in}}%
\pgfusepath{stroke,fill}%
}%
\begin{pgfscope}%
\pgfsys@transformshift{1.590159in}{0.417642in}%
\pgfsys@useobject{currentmarker}{}%
\end{pgfscope}%
\end{pgfscope}%
\begin{pgfscope}%
\pgfpathrectangle{\pgfqpoint{0.514278in}{0.417642in}}{\pgfqpoint{1.884996in}{1.371397in}}%
\pgfusepath{clip}%
\pgfsetrectcap%
\pgfsetroundjoin%
\pgfsetlinewidth{0.803000pt}%
\definecolor{currentstroke}{rgb}{0.850000,0.850000,0.850000}%
\pgfsetstrokecolor{currentstroke}%
\pgfsetdash{}{0pt}%
\pgfpathmoveto{\pgfqpoint{1.681294in}{0.417642in}}%
\pgfpathlineto{\pgfqpoint{1.681294in}{1.789039in}}%
\pgfusepath{stroke}%
\end{pgfscope}%
\begin{pgfscope}%
\pgfsetbuttcap%
\pgfsetroundjoin%
\definecolor{currentfill}{rgb}{0.000000,0.000000,0.000000}%
\pgfsetfillcolor{currentfill}%
\pgfsetlinewidth{0.602250pt}%
\definecolor{currentstroke}{rgb}{0.000000,0.000000,0.000000}%
\pgfsetstrokecolor{currentstroke}%
\pgfsetdash{}{0pt}%
\pgfsys@defobject{currentmarker}{\pgfqpoint{0.000000in}{-0.027778in}}{\pgfqpoint{0.000000in}{0.000000in}}{%
\pgfpathmoveto{\pgfqpoint{0.000000in}{0.000000in}}%
\pgfpathlineto{\pgfqpoint{0.000000in}{-0.027778in}}%
\pgfusepath{stroke,fill}%
}%
\begin{pgfscope}%
\pgfsys@transformshift{1.681294in}{0.417642in}%
\pgfsys@useobject{currentmarker}{}%
\end{pgfscope}%
\end{pgfscope}%
\begin{pgfscope}%
\pgfpathrectangle{\pgfqpoint{0.514278in}{0.417642in}}{\pgfqpoint{1.884996in}{1.371397in}}%
\pgfusepath{clip}%
\pgfsetrectcap%
\pgfsetroundjoin%
\pgfsetlinewidth{0.803000pt}%
\definecolor{currentstroke}{rgb}{0.850000,0.850000,0.850000}%
\pgfsetstrokecolor{currentstroke}%
\pgfsetdash{}{0pt}%
\pgfpathmoveto{\pgfqpoint{1.745954in}{0.417642in}}%
\pgfpathlineto{\pgfqpoint{1.745954in}{1.789039in}}%
\pgfusepath{stroke}%
\end{pgfscope}%
\begin{pgfscope}%
\pgfsetbuttcap%
\pgfsetroundjoin%
\definecolor{currentfill}{rgb}{0.000000,0.000000,0.000000}%
\pgfsetfillcolor{currentfill}%
\pgfsetlinewidth{0.602250pt}%
\definecolor{currentstroke}{rgb}{0.000000,0.000000,0.000000}%
\pgfsetstrokecolor{currentstroke}%
\pgfsetdash{}{0pt}%
\pgfsys@defobject{currentmarker}{\pgfqpoint{0.000000in}{-0.027778in}}{\pgfqpoint{0.000000in}{0.000000in}}{%
\pgfpathmoveto{\pgfqpoint{0.000000in}{0.000000in}}%
\pgfpathlineto{\pgfqpoint{0.000000in}{-0.027778in}}%
\pgfusepath{stroke,fill}%
}%
\begin{pgfscope}%
\pgfsys@transformshift{1.745954in}{0.417642in}%
\pgfsys@useobject{currentmarker}{}%
\end{pgfscope}%
\end{pgfscope}%
\begin{pgfscope}%
\pgfpathrectangle{\pgfqpoint{0.514278in}{0.417642in}}{\pgfqpoint{1.884996in}{1.371397in}}%
\pgfusepath{clip}%
\pgfsetrectcap%
\pgfsetroundjoin%
\pgfsetlinewidth{0.803000pt}%
\definecolor{currentstroke}{rgb}{0.850000,0.850000,0.850000}%
\pgfsetstrokecolor{currentstroke}%
\pgfsetdash{}{0pt}%
\pgfpathmoveto{\pgfqpoint{1.796109in}{0.417642in}}%
\pgfpathlineto{\pgfqpoint{1.796109in}{1.789039in}}%
\pgfusepath{stroke}%
\end{pgfscope}%
\begin{pgfscope}%
\pgfsetbuttcap%
\pgfsetroundjoin%
\definecolor{currentfill}{rgb}{0.000000,0.000000,0.000000}%
\pgfsetfillcolor{currentfill}%
\pgfsetlinewidth{0.602250pt}%
\definecolor{currentstroke}{rgb}{0.000000,0.000000,0.000000}%
\pgfsetstrokecolor{currentstroke}%
\pgfsetdash{}{0pt}%
\pgfsys@defobject{currentmarker}{\pgfqpoint{0.000000in}{-0.027778in}}{\pgfqpoint{0.000000in}{0.000000in}}{%
\pgfpathmoveto{\pgfqpoint{0.000000in}{0.000000in}}%
\pgfpathlineto{\pgfqpoint{0.000000in}{-0.027778in}}%
\pgfusepath{stroke,fill}%
}%
\begin{pgfscope}%
\pgfsys@transformshift{1.796109in}{0.417642in}%
\pgfsys@useobject{currentmarker}{}%
\end{pgfscope}%
\end{pgfscope}%
\begin{pgfscope}%
\pgfpathrectangle{\pgfqpoint{0.514278in}{0.417642in}}{\pgfqpoint{1.884996in}{1.371397in}}%
\pgfusepath{clip}%
\pgfsetrectcap%
\pgfsetroundjoin%
\pgfsetlinewidth{0.803000pt}%
\definecolor{currentstroke}{rgb}{0.850000,0.850000,0.850000}%
\pgfsetstrokecolor{currentstroke}%
\pgfsetdash{}{0pt}%
\pgfpathmoveto{\pgfqpoint{1.837088in}{0.417642in}}%
\pgfpathlineto{\pgfqpoint{1.837088in}{1.789039in}}%
\pgfusepath{stroke}%
\end{pgfscope}%
\begin{pgfscope}%
\pgfsetbuttcap%
\pgfsetroundjoin%
\definecolor{currentfill}{rgb}{0.000000,0.000000,0.000000}%
\pgfsetfillcolor{currentfill}%
\pgfsetlinewidth{0.602250pt}%
\definecolor{currentstroke}{rgb}{0.000000,0.000000,0.000000}%
\pgfsetstrokecolor{currentstroke}%
\pgfsetdash{}{0pt}%
\pgfsys@defobject{currentmarker}{\pgfqpoint{0.000000in}{-0.027778in}}{\pgfqpoint{0.000000in}{0.000000in}}{%
\pgfpathmoveto{\pgfqpoint{0.000000in}{0.000000in}}%
\pgfpathlineto{\pgfqpoint{0.000000in}{-0.027778in}}%
\pgfusepath{stroke,fill}%
}%
\begin{pgfscope}%
\pgfsys@transformshift{1.837088in}{0.417642in}%
\pgfsys@useobject{currentmarker}{}%
\end{pgfscope}%
\end{pgfscope}%
\begin{pgfscope}%
\pgfpathrectangle{\pgfqpoint{0.514278in}{0.417642in}}{\pgfqpoint{1.884996in}{1.371397in}}%
\pgfusepath{clip}%
\pgfsetrectcap%
\pgfsetroundjoin%
\pgfsetlinewidth{0.803000pt}%
\definecolor{currentstroke}{rgb}{0.850000,0.850000,0.850000}%
\pgfsetstrokecolor{currentstroke}%
\pgfsetdash{}{0pt}%
\pgfpathmoveto{\pgfqpoint{1.871736in}{0.417642in}}%
\pgfpathlineto{\pgfqpoint{1.871736in}{1.789039in}}%
\pgfusepath{stroke}%
\end{pgfscope}%
\begin{pgfscope}%
\pgfsetbuttcap%
\pgfsetroundjoin%
\definecolor{currentfill}{rgb}{0.000000,0.000000,0.000000}%
\pgfsetfillcolor{currentfill}%
\pgfsetlinewidth{0.602250pt}%
\definecolor{currentstroke}{rgb}{0.000000,0.000000,0.000000}%
\pgfsetstrokecolor{currentstroke}%
\pgfsetdash{}{0pt}%
\pgfsys@defobject{currentmarker}{\pgfqpoint{0.000000in}{-0.027778in}}{\pgfqpoint{0.000000in}{0.000000in}}{%
\pgfpathmoveto{\pgfqpoint{0.000000in}{0.000000in}}%
\pgfpathlineto{\pgfqpoint{0.000000in}{-0.027778in}}%
\pgfusepath{stroke,fill}%
}%
\begin{pgfscope}%
\pgfsys@transformshift{1.871736in}{0.417642in}%
\pgfsys@useobject{currentmarker}{}%
\end{pgfscope}%
\end{pgfscope}%
\begin{pgfscope}%
\pgfpathrectangle{\pgfqpoint{0.514278in}{0.417642in}}{\pgfqpoint{1.884996in}{1.371397in}}%
\pgfusepath{clip}%
\pgfsetrectcap%
\pgfsetroundjoin%
\pgfsetlinewidth{0.803000pt}%
\definecolor{currentstroke}{rgb}{0.850000,0.850000,0.850000}%
\pgfsetstrokecolor{currentstroke}%
\pgfsetdash{}{0pt}%
\pgfpathmoveto{\pgfqpoint{1.901749in}{0.417642in}}%
\pgfpathlineto{\pgfqpoint{1.901749in}{1.789039in}}%
\pgfusepath{stroke}%
\end{pgfscope}%
\begin{pgfscope}%
\pgfsetbuttcap%
\pgfsetroundjoin%
\definecolor{currentfill}{rgb}{0.000000,0.000000,0.000000}%
\pgfsetfillcolor{currentfill}%
\pgfsetlinewidth{0.602250pt}%
\definecolor{currentstroke}{rgb}{0.000000,0.000000,0.000000}%
\pgfsetstrokecolor{currentstroke}%
\pgfsetdash{}{0pt}%
\pgfsys@defobject{currentmarker}{\pgfqpoint{0.000000in}{-0.027778in}}{\pgfqpoint{0.000000in}{0.000000in}}{%
\pgfpathmoveto{\pgfqpoint{0.000000in}{0.000000in}}%
\pgfpathlineto{\pgfqpoint{0.000000in}{-0.027778in}}%
\pgfusepath{stroke,fill}%
}%
\begin{pgfscope}%
\pgfsys@transformshift{1.901749in}{0.417642in}%
\pgfsys@useobject{currentmarker}{}%
\end{pgfscope}%
\end{pgfscope}%
\begin{pgfscope}%
\pgfpathrectangle{\pgfqpoint{0.514278in}{0.417642in}}{\pgfqpoint{1.884996in}{1.371397in}}%
\pgfusepath{clip}%
\pgfsetrectcap%
\pgfsetroundjoin%
\pgfsetlinewidth{0.803000pt}%
\definecolor{currentstroke}{rgb}{0.850000,0.850000,0.850000}%
\pgfsetstrokecolor{currentstroke}%
\pgfsetdash{}{0pt}%
\pgfpathmoveto{\pgfqpoint{1.928223in}{0.417642in}}%
\pgfpathlineto{\pgfqpoint{1.928223in}{1.789039in}}%
\pgfusepath{stroke}%
\end{pgfscope}%
\begin{pgfscope}%
\pgfsetbuttcap%
\pgfsetroundjoin%
\definecolor{currentfill}{rgb}{0.000000,0.000000,0.000000}%
\pgfsetfillcolor{currentfill}%
\pgfsetlinewidth{0.602250pt}%
\definecolor{currentstroke}{rgb}{0.000000,0.000000,0.000000}%
\pgfsetstrokecolor{currentstroke}%
\pgfsetdash{}{0pt}%
\pgfsys@defobject{currentmarker}{\pgfqpoint{0.000000in}{-0.027778in}}{\pgfqpoint{0.000000in}{0.000000in}}{%
\pgfpathmoveto{\pgfqpoint{0.000000in}{0.000000in}}%
\pgfpathlineto{\pgfqpoint{0.000000in}{-0.027778in}}%
\pgfusepath{stroke,fill}%
}%
\begin{pgfscope}%
\pgfsys@transformshift{1.928223in}{0.417642in}%
\pgfsys@useobject{currentmarker}{}%
\end{pgfscope}%
\end{pgfscope}%
\begin{pgfscope}%
\pgfpathrectangle{\pgfqpoint{0.514278in}{0.417642in}}{\pgfqpoint{1.884996in}{1.371397in}}%
\pgfusepath{clip}%
\pgfsetrectcap%
\pgfsetroundjoin%
\pgfsetlinewidth{0.803000pt}%
\definecolor{currentstroke}{rgb}{0.850000,0.850000,0.850000}%
\pgfsetstrokecolor{currentstroke}%
\pgfsetdash{}{0pt}%
\pgfpathmoveto{\pgfqpoint{2.107699in}{0.417642in}}%
\pgfpathlineto{\pgfqpoint{2.107699in}{1.789039in}}%
\pgfusepath{stroke}%
\end{pgfscope}%
\begin{pgfscope}%
\pgfsetbuttcap%
\pgfsetroundjoin%
\definecolor{currentfill}{rgb}{0.000000,0.000000,0.000000}%
\pgfsetfillcolor{currentfill}%
\pgfsetlinewidth{0.602250pt}%
\definecolor{currentstroke}{rgb}{0.000000,0.000000,0.000000}%
\pgfsetstrokecolor{currentstroke}%
\pgfsetdash{}{0pt}%
\pgfsys@defobject{currentmarker}{\pgfqpoint{0.000000in}{-0.027778in}}{\pgfqpoint{0.000000in}{0.000000in}}{%
\pgfpathmoveto{\pgfqpoint{0.000000in}{0.000000in}}%
\pgfpathlineto{\pgfqpoint{0.000000in}{-0.027778in}}%
\pgfusepath{stroke,fill}%
}%
\begin{pgfscope}%
\pgfsys@transformshift{2.107699in}{0.417642in}%
\pgfsys@useobject{currentmarker}{}%
\end{pgfscope}%
\end{pgfscope}%
\begin{pgfscope}%
\pgfpathrectangle{\pgfqpoint{0.514278in}{0.417642in}}{\pgfqpoint{1.884996in}{1.371397in}}%
\pgfusepath{clip}%
\pgfsetrectcap%
\pgfsetroundjoin%
\pgfsetlinewidth{0.803000pt}%
\definecolor{currentstroke}{rgb}{0.850000,0.850000,0.850000}%
\pgfsetstrokecolor{currentstroke}%
\pgfsetdash{}{0pt}%
\pgfpathmoveto{\pgfqpoint{2.198833in}{0.417642in}}%
\pgfpathlineto{\pgfqpoint{2.198833in}{1.789039in}}%
\pgfusepath{stroke}%
\end{pgfscope}%
\begin{pgfscope}%
\pgfsetbuttcap%
\pgfsetroundjoin%
\definecolor{currentfill}{rgb}{0.000000,0.000000,0.000000}%
\pgfsetfillcolor{currentfill}%
\pgfsetlinewidth{0.602250pt}%
\definecolor{currentstroke}{rgb}{0.000000,0.000000,0.000000}%
\pgfsetstrokecolor{currentstroke}%
\pgfsetdash{}{0pt}%
\pgfsys@defobject{currentmarker}{\pgfqpoint{0.000000in}{-0.027778in}}{\pgfqpoint{0.000000in}{0.000000in}}{%
\pgfpathmoveto{\pgfqpoint{0.000000in}{0.000000in}}%
\pgfpathlineto{\pgfqpoint{0.000000in}{-0.027778in}}%
\pgfusepath{stroke,fill}%
}%
\begin{pgfscope}%
\pgfsys@transformshift{2.198833in}{0.417642in}%
\pgfsys@useobject{currentmarker}{}%
\end{pgfscope}%
\end{pgfscope}%
\begin{pgfscope}%
\pgfpathrectangle{\pgfqpoint{0.514278in}{0.417642in}}{\pgfqpoint{1.884996in}{1.371397in}}%
\pgfusepath{clip}%
\pgfsetrectcap%
\pgfsetroundjoin%
\pgfsetlinewidth{0.803000pt}%
\definecolor{currentstroke}{rgb}{0.850000,0.850000,0.850000}%
\pgfsetstrokecolor{currentstroke}%
\pgfsetdash{}{0pt}%
\pgfpathmoveto{\pgfqpoint{2.263493in}{0.417642in}}%
\pgfpathlineto{\pgfqpoint{2.263493in}{1.789039in}}%
\pgfusepath{stroke}%
\end{pgfscope}%
\begin{pgfscope}%
\pgfsetbuttcap%
\pgfsetroundjoin%
\definecolor{currentfill}{rgb}{0.000000,0.000000,0.000000}%
\pgfsetfillcolor{currentfill}%
\pgfsetlinewidth{0.602250pt}%
\definecolor{currentstroke}{rgb}{0.000000,0.000000,0.000000}%
\pgfsetstrokecolor{currentstroke}%
\pgfsetdash{}{0pt}%
\pgfsys@defobject{currentmarker}{\pgfqpoint{0.000000in}{-0.027778in}}{\pgfqpoint{0.000000in}{0.000000in}}{%
\pgfpathmoveto{\pgfqpoint{0.000000in}{0.000000in}}%
\pgfpathlineto{\pgfqpoint{0.000000in}{-0.027778in}}%
\pgfusepath{stroke,fill}%
}%
\begin{pgfscope}%
\pgfsys@transformshift{2.263493in}{0.417642in}%
\pgfsys@useobject{currentmarker}{}%
\end{pgfscope}%
\end{pgfscope}%
\begin{pgfscope}%
\pgfpathrectangle{\pgfqpoint{0.514278in}{0.417642in}}{\pgfqpoint{1.884996in}{1.371397in}}%
\pgfusepath{clip}%
\pgfsetrectcap%
\pgfsetroundjoin%
\pgfsetlinewidth{0.803000pt}%
\definecolor{currentstroke}{rgb}{0.850000,0.850000,0.850000}%
\pgfsetstrokecolor{currentstroke}%
\pgfsetdash{}{0pt}%
\pgfpathmoveto{\pgfqpoint{2.313648in}{0.417642in}}%
\pgfpathlineto{\pgfqpoint{2.313648in}{1.789039in}}%
\pgfusepath{stroke}%
\end{pgfscope}%
\begin{pgfscope}%
\pgfsetbuttcap%
\pgfsetroundjoin%
\definecolor{currentfill}{rgb}{0.000000,0.000000,0.000000}%
\pgfsetfillcolor{currentfill}%
\pgfsetlinewidth{0.602250pt}%
\definecolor{currentstroke}{rgb}{0.000000,0.000000,0.000000}%
\pgfsetstrokecolor{currentstroke}%
\pgfsetdash{}{0pt}%
\pgfsys@defobject{currentmarker}{\pgfqpoint{0.000000in}{-0.027778in}}{\pgfqpoint{0.000000in}{0.000000in}}{%
\pgfpathmoveto{\pgfqpoint{0.000000in}{0.000000in}}%
\pgfpathlineto{\pgfqpoint{0.000000in}{-0.027778in}}%
\pgfusepath{stroke,fill}%
}%
\begin{pgfscope}%
\pgfsys@transformshift{2.313648in}{0.417642in}%
\pgfsys@useobject{currentmarker}{}%
\end{pgfscope}%
\end{pgfscope}%
\begin{pgfscope}%
\pgfpathrectangle{\pgfqpoint{0.514278in}{0.417642in}}{\pgfqpoint{1.884996in}{1.371397in}}%
\pgfusepath{clip}%
\pgfsetrectcap%
\pgfsetroundjoin%
\pgfsetlinewidth{0.803000pt}%
\definecolor{currentstroke}{rgb}{0.850000,0.850000,0.850000}%
\pgfsetstrokecolor{currentstroke}%
\pgfsetdash{}{0pt}%
\pgfpathmoveto{\pgfqpoint{2.354628in}{0.417642in}}%
\pgfpathlineto{\pgfqpoint{2.354628in}{1.789039in}}%
\pgfusepath{stroke}%
\end{pgfscope}%
\begin{pgfscope}%
\pgfsetbuttcap%
\pgfsetroundjoin%
\definecolor{currentfill}{rgb}{0.000000,0.000000,0.000000}%
\pgfsetfillcolor{currentfill}%
\pgfsetlinewidth{0.602250pt}%
\definecolor{currentstroke}{rgb}{0.000000,0.000000,0.000000}%
\pgfsetstrokecolor{currentstroke}%
\pgfsetdash{}{0pt}%
\pgfsys@defobject{currentmarker}{\pgfqpoint{0.000000in}{-0.027778in}}{\pgfqpoint{0.000000in}{0.000000in}}{%
\pgfpathmoveto{\pgfqpoint{0.000000in}{0.000000in}}%
\pgfpathlineto{\pgfqpoint{0.000000in}{-0.027778in}}%
\pgfusepath{stroke,fill}%
}%
\begin{pgfscope}%
\pgfsys@transformshift{2.354628in}{0.417642in}%
\pgfsys@useobject{currentmarker}{}%
\end{pgfscope}%
\end{pgfscope}%
\begin{pgfscope}%
\pgfpathrectangle{\pgfqpoint{0.514278in}{0.417642in}}{\pgfqpoint{1.884996in}{1.371397in}}%
\pgfusepath{clip}%
\pgfsetrectcap%
\pgfsetroundjoin%
\pgfsetlinewidth{0.803000pt}%
\definecolor{currentstroke}{rgb}{0.850000,0.850000,0.850000}%
\pgfsetstrokecolor{currentstroke}%
\pgfsetdash{}{0pt}%
\pgfpathmoveto{\pgfqpoint{2.389275in}{0.417642in}}%
\pgfpathlineto{\pgfqpoint{2.389275in}{1.789039in}}%
\pgfusepath{stroke}%
\end{pgfscope}%
\begin{pgfscope}%
\pgfsetbuttcap%
\pgfsetroundjoin%
\definecolor{currentfill}{rgb}{0.000000,0.000000,0.000000}%
\pgfsetfillcolor{currentfill}%
\pgfsetlinewidth{0.602250pt}%
\definecolor{currentstroke}{rgb}{0.000000,0.000000,0.000000}%
\pgfsetstrokecolor{currentstroke}%
\pgfsetdash{}{0pt}%
\pgfsys@defobject{currentmarker}{\pgfqpoint{0.000000in}{-0.027778in}}{\pgfqpoint{0.000000in}{0.000000in}}{%
\pgfpathmoveto{\pgfqpoint{0.000000in}{0.000000in}}%
\pgfpathlineto{\pgfqpoint{0.000000in}{-0.027778in}}%
\pgfusepath{stroke,fill}%
}%
\begin{pgfscope}%
\pgfsys@transformshift{2.389275in}{0.417642in}%
\pgfsys@useobject{currentmarker}{}%
\end{pgfscope}%
\end{pgfscope}%
\begin{pgfscope}%
\definecolor{textcolor}{rgb}{0.000000,0.000000,0.000000}%
\pgfsetstrokecolor{textcolor}%
\pgfsetfillcolor{textcolor}%
\pgftext[x=1.456777in,y=0.165003in,,top]{\color{textcolor}\rmfamily\fontsize{10.000000}{12.000000}\selectfont Frequency in \(\displaystyle \unit{\Hz}\)}%
\end{pgfscope}%
\begin{pgfscope}%
\pgfpathrectangle{\pgfqpoint{0.514278in}{0.417642in}}{\pgfqpoint{1.884996in}{1.371397in}}%
\pgfusepath{clip}%
\pgfsetrectcap%
\pgfsetroundjoin%
\pgfsetlinewidth{0.803000pt}%
\definecolor{currentstroke}{rgb}{0.450000,0.450000,0.450000}%
\pgfsetstrokecolor{currentstroke}%
\pgfsetdash{}{0pt}%
\pgfpathmoveto{\pgfqpoint{0.514278in}{0.640670in}}%
\pgfpathlineto{\pgfqpoint{2.399275in}{0.640670in}}%
\pgfusepath{stroke}%
\end{pgfscope}%
\begin{pgfscope}%
\pgfsetbuttcap%
\pgfsetroundjoin%
\definecolor{currentfill}{rgb}{0.000000,0.000000,0.000000}%
\pgfsetfillcolor{currentfill}%
\pgfsetlinewidth{0.803000pt}%
\definecolor{currentstroke}{rgb}{0.000000,0.000000,0.000000}%
\pgfsetstrokecolor{currentstroke}%
\pgfsetdash{}{0pt}%
\pgfsys@defobject{currentmarker}{\pgfqpoint{-0.048611in}{0.000000in}}{\pgfqpoint{-0.000000in}{0.000000in}}{%
\pgfpathmoveto{\pgfqpoint{-0.000000in}{0.000000in}}%
\pgfpathlineto{\pgfqpoint{-0.048611in}{0.000000in}}%
\pgfusepath{stroke,fill}%
}%
\begin{pgfscope}%
\pgfsys@transformshift{0.514278in}{0.640670in}%
\pgfsys@useobject{currentmarker}{}%
\end{pgfscope}%
\end{pgfscope}%
\begin{pgfscope}%
\definecolor{textcolor}{rgb}{0.000000,0.000000,0.000000}%
\pgfsetstrokecolor{textcolor}%
\pgfsetfillcolor{textcolor}%
\pgftext[x=0.241129in, y=0.601518in, left, base]{\color{textcolor}\rmfamily\fontsize{8.000000}{9.600000}\selectfont \(\displaystyle {10^{0}}\)}%
\end{pgfscope}%
\begin{pgfscope}%
\pgfpathrectangle{\pgfqpoint{0.514278in}{0.417642in}}{\pgfqpoint{1.884996in}{1.371397in}}%
\pgfusepath{clip}%
\pgfsetrectcap%
\pgfsetroundjoin%
\pgfsetlinewidth{0.803000pt}%
\definecolor{currentstroke}{rgb}{0.450000,0.450000,0.450000}%
\pgfsetstrokecolor{currentstroke}%
\pgfsetdash{}{0pt}%
\pgfpathmoveto{\pgfqpoint{0.514278in}{0.983520in}}%
\pgfpathlineto{\pgfqpoint{2.399275in}{0.983520in}}%
\pgfusepath{stroke}%
\end{pgfscope}%
\begin{pgfscope}%
\pgfsetbuttcap%
\pgfsetroundjoin%
\definecolor{currentfill}{rgb}{0.000000,0.000000,0.000000}%
\pgfsetfillcolor{currentfill}%
\pgfsetlinewidth{0.803000pt}%
\definecolor{currentstroke}{rgb}{0.000000,0.000000,0.000000}%
\pgfsetstrokecolor{currentstroke}%
\pgfsetdash{}{0pt}%
\pgfsys@defobject{currentmarker}{\pgfqpoint{-0.048611in}{0.000000in}}{\pgfqpoint{-0.000000in}{0.000000in}}{%
\pgfpathmoveto{\pgfqpoint{-0.000000in}{0.000000in}}%
\pgfpathlineto{\pgfqpoint{-0.048611in}{0.000000in}}%
\pgfusepath{stroke,fill}%
}%
\begin{pgfscope}%
\pgfsys@transformshift{0.514278in}{0.983520in}%
\pgfsys@useobject{currentmarker}{}%
\end{pgfscope}%
\end{pgfscope}%
\begin{pgfscope}%
\definecolor{textcolor}{rgb}{0.000000,0.000000,0.000000}%
\pgfsetstrokecolor{textcolor}%
\pgfsetfillcolor{textcolor}%
\pgftext[x=0.241129in, y=0.944367in, left, base]{\color{textcolor}\rmfamily\fontsize{8.000000}{9.600000}\selectfont \(\displaystyle {10^{2}}\)}%
\end{pgfscope}%
\begin{pgfscope}%
\pgfpathrectangle{\pgfqpoint{0.514278in}{0.417642in}}{\pgfqpoint{1.884996in}{1.371397in}}%
\pgfusepath{clip}%
\pgfsetrectcap%
\pgfsetroundjoin%
\pgfsetlinewidth{0.803000pt}%
\definecolor{currentstroke}{rgb}{0.450000,0.450000,0.450000}%
\pgfsetstrokecolor{currentstroke}%
\pgfsetdash{}{0pt}%
\pgfpathmoveto{\pgfqpoint{0.514278in}{1.326369in}}%
\pgfpathlineto{\pgfqpoint{2.399275in}{1.326369in}}%
\pgfusepath{stroke}%
\end{pgfscope}%
\begin{pgfscope}%
\pgfsetbuttcap%
\pgfsetroundjoin%
\definecolor{currentfill}{rgb}{0.000000,0.000000,0.000000}%
\pgfsetfillcolor{currentfill}%
\pgfsetlinewidth{0.803000pt}%
\definecolor{currentstroke}{rgb}{0.000000,0.000000,0.000000}%
\pgfsetstrokecolor{currentstroke}%
\pgfsetdash{}{0pt}%
\pgfsys@defobject{currentmarker}{\pgfqpoint{-0.048611in}{0.000000in}}{\pgfqpoint{-0.000000in}{0.000000in}}{%
\pgfpathmoveto{\pgfqpoint{-0.000000in}{0.000000in}}%
\pgfpathlineto{\pgfqpoint{-0.048611in}{0.000000in}}%
\pgfusepath{stroke,fill}%
}%
\begin{pgfscope}%
\pgfsys@transformshift{0.514278in}{1.326369in}%
\pgfsys@useobject{currentmarker}{}%
\end{pgfscope}%
\end{pgfscope}%
\begin{pgfscope}%
\definecolor{textcolor}{rgb}{0.000000,0.000000,0.000000}%
\pgfsetstrokecolor{textcolor}%
\pgfsetfillcolor{textcolor}%
\pgftext[x=0.241129in, y=1.287216in, left, base]{\color{textcolor}\rmfamily\fontsize{8.000000}{9.600000}\selectfont \(\displaystyle {10^{4}}\)}%
\end{pgfscope}%
\begin{pgfscope}%
\pgfpathrectangle{\pgfqpoint{0.514278in}{0.417642in}}{\pgfqpoint{1.884996in}{1.371397in}}%
\pgfusepath{clip}%
\pgfsetrectcap%
\pgfsetroundjoin%
\pgfsetlinewidth{0.803000pt}%
\definecolor{currentstroke}{rgb}{0.450000,0.450000,0.450000}%
\pgfsetstrokecolor{currentstroke}%
\pgfsetdash{}{0pt}%
\pgfpathmoveto{\pgfqpoint{0.514278in}{1.669218in}}%
\pgfpathlineto{\pgfqpoint{2.399275in}{1.669218in}}%
\pgfusepath{stroke}%
\end{pgfscope}%
\begin{pgfscope}%
\pgfsetbuttcap%
\pgfsetroundjoin%
\definecolor{currentfill}{rgb}{0.000000,0.000000,0.000000}%
\pgfsetfillcolor{currentfill}%
\pgfsetlinewidth{0.803000pt}%
\definecolor{currentstroke}{rgb}{0.000000,0.000000,0.000000}%
\pgfsetstrokecolor{currentstroke}%
\pgfsetdash{}{0pt}%
\pgfsys@defobject{currentmarker}{\pgfqpoint{-0.048611in}{0.000000in}}{\pgfqpoint{-0.000000in}{0.000000in}}{%
\pgfpathmoveto{\pgfqpoint{-0.000000in}{0.000000in}}%
\pgfpathlineto{\pgfqpoint{-0.048611in}{0.000000in}}%
\pgfusepath{stroke,fill}%
}%
\begin{pgfscope}%
\pgfsys@transformshift{0.514278in}{1.669218in}%
\pgfsys@useobject{currentmarker}{}%
\end{pgfscope}%
\end{pgfscope}%
\begin{pgfscope}%
\definecolor{textcolor}{rgb}{0.000000,0.000000,0.000000}%
\pgfsetstrokecolor{textcolor}%
\pgfsetfillcolor{textcolor}%
\pgftext[x=0.241129in, y=1.630065in, left, base]{\color{textcolor}\rmfamily\fontsize{8.000000}{9.600000}\selectfont \(\displaystyle {10^{6}}\)}%
\end{pgfscope}%
\begin{pgfscope}%
\definecolor{textcolor}{rgb}{0.000000,0.000000,0.000000}%
\pgfsetstrokecolor{textcolor}%
\pgfsetfillcolor{textcolor}%
\pgftext[x=0.185574in,y=1.103340in,,bottom,rotate=90.000000]{\color{textcolor}\rmfamily\fontsize{10.000000}{12.000000}\selectfont  \(\displaystyle S_y(f)\) in \(\displaystyle \unit{1 \per \Hz}\)}%
\end{pgfscope}%
\begin{pgfscope}%
\pgfpathrectangle{\pgfqpoint{0.514278in}{0.417642in}}{\pgfqpoint{1.884996in}{1.371397in}}%
\pgfusepath{clip}%
\pgfsetbuttcap%
\pgfsetroundjoin%
\pgfsetlinewidth{1.505625pt}%
\definecolor{currentstroke}{rgb}{0.007843,0.619608,0.450980}%
\pgfsetstrokecolor{currentstroke}%
\pgfsetdash{{5.550000pt}{2.400000pt}}{0.000000pt}%
\pgfpathmoveto{\pgfqpoint{0.599960in}{1.235582in}}%
\pgfpathlineto{\pgfqpoint{2.313593in}{0.667975in}}%
\pgfpathlineto{\pgfqpoint{2.313593in}{0.667975in}}%
\pgfusepath{stroke}%
\end{pgfscope}%
\begin{pgfscope}%
\pgfpathrectangle{\pgfqpoint{0.514278in}{0.417642in}}{\pgfqpoint{1.884996in}{1.371397in}}%
\pgfusepath{clip}%
\pgfsetbuttcap%
\pgfsetroundjoin%
\definecolor{currentfill}{rgb}{0.007843,0.619608,0.450980}%
\pgfsetfillcolor{currentfill}%
\pgfsetlinewidth{1.003750pt}%
\definecolor{currentstroke}{rgb}{0.007843,0.619608,0.450980}%
\pgfsetstrokecolor{currentstroke}%
\pgfsetdash{}{0pt}%
\pgfsys@defobject{currentmarker}{\pgfqpoint{-0.006944in}{-0.006944in}}{\pgfqpoint{0.006944in}{0.006944in}}{%
\pgfpathmoveto{\pgfqpoint{0.000000in}{-0.006944in}}%
\pgfpathcurveto{\pgfqpoint{0.001842in}{-0.006944in}}{\pgfqpoint{0.003608in}{-0.006213in}}{\pgfqpoint{0.004910in}{-0.004910in}}%
\pgfpathcurveto{\pgfqpoint{0.006213in}{-0.003608in}}{\pgfqpoint{0.006944in}{-0.001842in}}{\pgfqpoint{0.006944in}{0.000000in}}%
\pgfpathcurveto{\pgfqpoint{0.006944in}{0.001842in}}{\pgfqpoint{0.006213in}{0.003608in}}{\pgfqpoint{0.004910in}{0.004910in}}%
\pgfpathcurveto{\pgfqpoint{0.003608in}{0.006213in}}{\pgfqpoint{0.001842in}{0.006944in}}{\pgfqpoint{0.000000in}{0.006944in}}%
\pgfpathcurveto{\pgfqpoint{-0.001842in}{0.006944in}}{\pgfqpoint{-0.003608in}{0.006213in}}{\pgfqpoint{-0.004910in}{0.004910in}}%
\pgfpathcurveto{\pgfqpoint{-0.006213in}{0.003608in}}{\pgfqpoint{-0.006944in}{0.001842in}}{\pgfqpoint{-0.006944in}{0.000000in}}%
\pgfpathcurveto{\pgfqpoint{-0.006944in}{-0.001842in}}{\pgfqpoint{-0.006213in}{-0.003608in}}{\pgfqpoint{-0.004910in}{-0.004910in}}%
\pgfpathcurveto{\pgfqpoint{-0.003608in}{-0.006213in}}{\pgfqpoint{-0.001842in}{-0.006944in}}{\pgfqpoint{0.000000in}{-0.006944in}}%
\pgfpathlineto{\pgfqpoint{0.000000in}{-0.006944in}}%
\pgfpathclose%
\pgfusepath{stroke,fill}%
}%
\begin{pgfscope}%
\pgfsys@transformshift{-226.701573in}{1.169189in}%
\pgfsys@useobject{currentmarker}{}%
\end{pgfscope}%
\begin{pgfscope}%
\pgfsys@transformshift{0.599960in}{1.223928in}%
\pgfsys@useobject{currentmarker}{}%
\end{pgfscope}%
\begin{pgfscope}%
\pgfsys@transformshift{0.755755in}{1.133873in}%
\pgfsys@useobject{currentmarker}{}%
\end{pgfscope}%
\begin{pgfscope}%
\pgfsys@transformshift{0.846889in}{1.112772in}%
\pgfsys@useobject{currentmarker}{}%
\end{pgfscope}%
\begin{pgfscope}%
\pgfsys@transformshift{0.911550in}{1.087150in}%
\pgfsys@useobject{currentmarker}{}%
\end{pgfscope}%
\begin{pgfscope}%
\pgfsys@transformshift{0.961704in}{1.077979in}%
\pgfsys@useobject{currentmarker}{}%
\end{pgfscope}%
\begin{pgfscope}%
\pgfsys@transformshift{1.002684in}{1.028039in}%
\pgfsys@useobject{currentmarker}{}%
\end{pgfscope}%
\begin{pgfscope}%
\pgfsys@transformshift{1.037331in}{1.099417in}%
\pgfsys@useobject{currentmarker}{}%
\end{pgfscope}%
\begin{pgfscope}%
\pgfsys@transformshift{1.067345in}{1.110978in}%
\pgfsys@useobject{currentmarker}{}%
\end{pgfscope}%
\begin{pgfscope}%
\pgfsys@transformshift{1.093818in}{1.091707in}%
\pgfsys@useobject{currentmarker}{}%
\end{pgfscope}%
\begin{pgfscope}%
\pgfsys@transformshift{1.117499in}{1.074621in}%
\pgfsys@useobject{currentmarker}{}%
\end{pgfscope}%
\begin{pgfscope}%
\pgfsys@transformshift{1.138922in}{1.038922in}%
\pgfsys@useobject{currentmarker}{}%
\end{pgfscope}%
\begin{pgfscope}%
\pgfsys@transformshift{1.158479in}{1.072460in}%
\pgfsys@useobject{currentmarker}{}%
\end{pgfscope}%
\begin{pgfscope}%
\pgfsys@transformshift{1.176469in}{1.045045in}%
\pgfsys@useobject{currentmarker}{}%
\end{pgfscope}%
\begin{pgfscope}%
\pgfsys@transformshift{1.193126in}{0.976377in}%
\pgfsys@useobject{currentmarker}{}%
\end{pgfscope}%
\begin{pgfscope}%
\pgfsys@transformshift{1.208633in}{1.042403in}%
\pgfsys@useobject{currentmarker}{}%
\end{pgfscope}%
\begin{pgfscope}%
\pgfsys@transformshift{1.223139in}{1.066147in}%
\pgfsys@useobject{currentmarker}{}%
\end{pgfscope}%
\begin{pgfscope}%
\pgfsys@transformshift{1.236766in}{1.041717in}%
\pgfsys@useobject{currentmarker}{}%
\end{pgfscope}%
\begin{pgfscope}%
\pgfsys@transformshift{1.249613in}{0.934765in}%
\pgfsys@useobject{currentmarker}{}%
\end{pgfscope}%
\begin{pgfscope}%
\pgfsys@transformshift{1.261765in}{0.984899in}%
\pgfsys@useobject{currentmarker}{}%
\end{pgfscope}%
\begin{pgfscope}%
\pgfsys@transformshift{1.273294in}{1.050658in}%
\pgfsys@useobject{currentmarker}{}%
\end{pgfscope}%
\begin{pgfscope}%
\pgfsys@transformshift{1.284260in}{1.066325in}%
\pgfsys@useobject{currentmarker}{}%
\end{pgfscope}%
\begin{pgfscope}%
\pgfsys@transformshift{1.294716in}{1.028193in}%
\pgfsys@useobject{currentmarker}{}%
\end{pgfscope}%
\begin{pgfscope}%
\pgfsys@transformshift{1.304708in}{1.005515in}%
\pgfsys@useobject{currentmarker}{}%
\end{pgfscope}%
\begin{pgfscope}%
\pgfsys@transformshift{1.314274in}{0.958468in}%
\pgfsys@useobject{currentmarker}{}%
\end{pgfscope}%
\begin{pgfscope}%
\pgfsys@transformshift{1.323449in}{0.981024in}%
\pgfsys@useobject{currentmarker}{}%
\end{pgfscope}%
\begin{pgfscope}%
\pgfsys@transformshift{1.332264in}{0.960523in}%
\pgfsys@useobject{currentmarker}{}%
\end{pgfscope}%
\begin{pgfscope}%
\pgfsys@transformshift{1.340747in}{0.954804in}%
\pgfsys@useobject{currentmarker}{}%
\end{pgfscope}%
\begin{pgfscope}%
\pgfsys@transformshift{1.348921in}{1.001306in}%
\pgfsys@useobject{currentmarker}{}%
\end{pgfscope}%
\begin{pgfscope}%
\pgfsys@transformshift{1.356808in}{1.006503in}%
\pgfsys@useobject{currentmarker}{}%
\end{pgfscope}%
\begin{pgfscope}%
\pgfsys@transformshift{1.364428in}{0.963668in}%
\pgfsys@useobject{currentmarker}{}%
\end{pgfscope}%
\begin{pgfscope}%
\pgfsys@transformshift{1.371798in}{0.987385in}%
\pgfsys@useobject{currentmarker}{}%
\end{pgfscope}%
\begin{pgfscope}%
\pgfsys@transformshift{1.378934in}{0.971209in}%
\pgfsys@useobject{currentmarker}{}%
\end{pgfscope}%
\begin{pgfscope}%
\pgfsys@transformshift{1.385851in}{0.955870in}%
\pgfsys@useobject{currentmarker}{}%
\end{pgfscope}%
\begin{pgfscope}%
\pgfsys@transformshift{1.392560in}{0.992313in}%
\pgfsys@useobject{currentmarker}{}%
\end{pgfscope}%
\begin{pgfscope}%
\pgfsys@transformshift{1.399076in}{0.953299in}%
\pgfsys@useobject{currentmarker}{}%
\end{pgfscope}%
\begin{pgfscope}%
\pgfsys@transformshift{1.405408in}{0.985941in}%
\pgfsys@useobject{currentmarker}{}%
\end{pgfscope}%
\begin{pgfscope}%
\pgfsys@transformshift{1.411566in}{0.974329in}%
\pgfsys@useobject{currentmarker}{}%
\end{pgfscope}%
\begin{pgfscope}%
\pgfsys@transformshift{1.417560in}{0.943776in}%
\pgfsys@useobject{currentmarker}{}%
\end{pgfscope}%
\begin{pgfscope}%
\pgfsys@transformshift{1.423398in}{0.901112in}%
\pgfsys@useobject{currentmarker}{}%
\end{pgfscope}%
\begin{pgfscope}%
\pgfsys@transformshift{1.429089in}{0.948777in}%
\pgfsys@useobject{currentmarker}{}%
\end{pgfscope}%
\begin{pgfscope}%
\pgfsys@transformshift{1.434639in}{0.975645in}%
\pgfsys@useobject{currentmarker}{}%
\end{pgfscope}%
\begin{pgfscope}%
\pgfsys@transformshift{1.440055in}{0.988812in}%
\pgfsys@useobject{currentmarker}{}%
\end{pgfscope}%
\begin{pgfscope}%
\pgfsys@transformshift{1.445344in}{1.004300in}%
\pgfsys@useobject{currentmarker}{}%
\end{pgfscope}%
\begin{pgfscope}%
\pgfsys@transformshift{1.450511in}{0.954049in}%
\pgfsys@useobject{currentmarker}{}%
\end{pgfscope}%
\begin{pgfscope}%
\pgfsys@transformshift{1.455562in}{0.875688in}%
\pgfsys@useobject{currentmarker}{}%
\end{pgfscope}%
\begin{pgfscope}%
\pgfsys@transformshift{1.460502in}{0.903319in}%
\pgfsys@useobject{currentmarker}{}%
\end{pgfscope}%
\begin{pgfscope}%
\pgfsys@transformshift{1.465336in}{0.926270in}%
\pgfsys@useobject{currentmarker}{}%
\end{pgfscope}%
\begin{pgfscope}%
\pgfsys@transformshift{1.470068in}{0.953962in}%
\pgfsys@useobject{currentmarker}{}%
\end{pgfscope}%
\begin{pgfscope}%
\pgfsys@transformshift{1.474703in}{0.940623in}%
\pgfsys@useobject{currentmarker}{}%
\end{pgfscope}%
\begin{pgfscope}%
\pgfsys@transformshift{1.479244in}{0.893481in}%
\pgfsys@useobject{currentmarker}{}%
\end{pgfscope}%
\begin{pgfscope}%
\pgfsys@transformshift{1.483695in}{0.835354in}%
\pgfsys@useobject{currentmarker}{}%
\end{pgfscope}%
\begin{pgfscope}%
\pgfsys@transformshift{1.488059in}{0.874771in}%
\pgfsys@useobject{currentmarker}{}%
\end{pgfscope}%
\begin{pgfscope}%
\pgfsys@transformshift{1.492340in}{0.941691in}%
\pgfsys@useobject{currentmarker}{}%
\end{pgfscope}%
\begin{pgfscope}%
\pgfsys@transformshift{1.496542in}{0.944407in}%
\pgfsys@useobject{currentmarker}{}%
\end{pgfscope}%
\begin{pgfscope}%
\pgfsys@transformshift{1.500666in}{0.954851in}%
\pgfsys@useobject{currentmarker}{}%
\end{pgfscope}%
\begin{pgfscope}%
\pgfsys@transformshift{1.504716in}{0.960310in}%
\pgfsys@useobject{currentmarker}{}%
\end{pgfscope}%
\begin{pgfscope}%
\pgfsys@transformshift{1.508694in}{0.971391in}%
\pgfsys@useobject{currentmarker}{}%
\end{pgfscope}%
\begin{pgfscope}%
\pgfsys@transformshift{1.512603in}{0.963758in}%
\pgfsys@useobject{currentmarker}{}%
\end{pgfscope}%
\begin{pgfscope}%
\pgfsys@transformshift{1.516445in}{0.942146in}%
\pgfsys@useobject{currentmarker}{}%
\end{pgfscope}%
\begin{pgfscope}%
\pgfsys@transformshift{1.520223in}{0.922040in}%
\pgfsys@useobject{currentmarker}{}%
\end{pgfscope}%
\begin{pgfscope}%
\pgfsys@transformshift{1.523938in}{0.915363in}%
\pgfsys@useobject{currentmarker}{}%
\end{pgfscope}%
\begin{pgfscope}%
\pgfsys@transformshift{1.527593in}{0.859983in}%
\pgfsys@useobject{currentmarker}{}%
\end{pgfscope}%
\begin{pgfscope}%
\pgfsys@transformshift{1.531189in}{0.868357in}%
\pgfsys@useobject{currentmarker}{}%
\end{pgfscope}%
\begin{pgfscope}%
\pgfsys@transformshift{1.534729in}{0.903468in}%
\pgfsys@useobject{currentmarker}{}%
\end{pgfscope}%
\begin{pgfscope}%
\pgfsys@transformshift{1.538214in}{0.875485in}%
\pgfsys@useobject{currentmarker}{}%
\end{pgfscope}%
\begin{pgfscope}%
\pgfsys@transformshift{1.541645in}{0.874498in}%
\pgfsys@useobject{currentmarker}{}%
\end{pgfscope}%
\begin{pgfscope}%
\pgfsys@transformshift{1.545025in}{0.905637in}%
\pgfsys@useobject{currentmarker}{}%
\end{pgfscope}%
\begin{pgfscope}%
\pgfsys@transformshift{1.548355in}{0.938479in}%
\pgfsys@useobject{currentmarker}{}%
\end{pgfscope}%
\begin{pgfscope}%
\pgfsys@transformshift{1.551637in}{0.929154in}%
\pgfsys@useobject{currentmarker}{}%
\end{pgfscope}%
\begin{pgfscope}%
\pgfsys@transformshift{1.554871in}{0.861375in}%
\pgfsys@useobject{currentmarker}{}%
\end{pgfscope}%
\begin{pgfscope}%
\pgfsys@transformshift{1.558059in}{0.884229in}%
\pgfsys@useobject{currentmarker}{}%
\end{pgfscope}%
\begin{pgfscope}%
\pgfsys@transformshift{1.561202in}{0.934031in}%
\pgfsys@useobject{currentmarker}{}%
\end{pgfscope}%
\begin{pgfscope}%
\pgfsys@transformshift{1.564303in}{0.936534in}%
\pgfsys@useobject{currentmarker}{}%
\end{pgfscope}%
\begin{pgfscope}%
\pgfsys@transformshift{1.567361in}{0.898715in}%
\pgfsys@useobject{currentmarker}{}%
\end{pgfscope}%
\begin{pgfscope}%
\pgfsys@transformshift{1.570378in}{0.903260in}%
\pgfsys@useobject{currentmarker}{}%
\end{pgfscope}%
\begin{pgfscope}%
\pgfsys@transformshift{1.573355in}{0.923390in}%
\pgfsys@useobject{currentmarker}{}%
\end{pgfscope}%
\begin{pgfscope}%
\pgfsys@transformshift{1.576293in}{0.901611in}%
\pgfsys@useobject{currentmarker}{}%
\end{pgfscope}%
\begin{pgfscope}%
\pgfsys@transformshift{1.579193in}{0.852478in}%
\pgfsys@useobject{currentmarker}{}%
\end{pgfscope}%
\begin{pgfscope}%
\pgfsys@transformshift{1.582056in}{0.860354in}%
\pgfsys@useobject{currentmarker}{}%
\end{pgfscope}%
\begin{pgfscope}%
\pgfsys@transformshift{1.584884in}{0.912870in}%
\pgfsys@useobject{currentmarker}{}%
\end{pgfscope}%
\begin{pgfscope}%
\pgfsys@transformshift{1.587676in}{0.901766in}%
\pgfsys@useobject{currentmarker}{}%
\end{pgfscope}%
\begin{pgfscope}%
\pgfsys@transformshift{1.590434in}{0.889081in}%
\pgfsys@useobject{currentmarker}{}%
\end{pgfscope}%
\begin{pgfscope}%
\pgfsys@transformshift{1.593158in}{0.869770in}%
\pgfsys@useobject{currentmarker}{}%
\end{pgfscope}%
\begin{pgfscope}%
\pgfsys@transformshift{1.595850in}{0.850916in}%
\pgfsys@useobject{currentmarker}{}%
\end{pgfscope}%
\begin{pgfscope}%
\pgfsys@transformshift{1.598510in}{0.849274in}%
\pgfsys@useobject{currentmarker}{}%
\end{pgfscope}%
\begin{pgfscope}%
\pgfsys@transformshift{1.601139in}{0.886159in}%
\pgfsys@useobject{currentmarker}{}%
\end{pgfscope}%
\begin{pgfscope}%
\pgfsys@transformshift{1.603737in}{0.850326in}%
\pgfsys@useobject{currentmarker}{}%
\end{pgfscope}%
\begin{pgfscope}%
\pgfsys@transformshift{1.606306in}{0.871626in}%
\pgfsys@useobject{currentmarker}{}%
\end{pgfscope}%
\begin{pgfscope}%
\pgfsys@transformshift{1.608846in}{0.825372in}%
\pgfsys@useobject{currentmarker}{}%
\end{pgfscope}%
\begin{pgfscope}%
\pgfsys@transformshift{1.611357in}{0.830224in}%
\pgfsys@useobject{currentmarker}{}%
\end{pgfscope}%
\begin{pgfscope}%
\pgfsys@transformshift{1.613841in}{0.899334in}%
\pgfsys@useobject{currentmarker}{}%
\end{pgfscope}%
\begin{pgfscope}%
\pgfsys@transformshift{1.616297in}{0.890515in}%
\pgfsys@useobject{currentmarker}{}%
\end{pgfscope}%
\begin{pgfscope}%
\pgfsys@transformshift{1.618727in}{0.897581in}%
\pgfsys@useobject{currentmarker}{}%
\end{pgfscope}%
\begin{pgfscope}%
\pgfsys@transformshift{1.621131in}{0.886250in}%
\pgfsys@useobject{currentmarker}{}%
\end{pgfscope}%
\begin{pgfscope}%
\pgfsys@transformshift{1.623510in}{0.866302in}%
\pgfsys@useobject{currentmarker}{}%
\end{pgfscope}%
\begin{pgfscope}%
\pgfsys@transformshift{1.625863in}{0.883376in}%
\pgfsys@useobject{currentmarker}{}%
\end{pgfscope}%
\begin{pgfscope}%
\pgfsys@transformshift{1.628192in}{0.868562in}%
\pgfsys@useobject{currentmarker}{}%
\end{pgfscope}%
\begin{pgfscope}%
\pgfsys@transformshift{1.630498in}{0.852916in}%
\pgfsys@useobject{currentmarker}{}%
\end{pgfscope}%
\begin{pgfscope}%
\pgfsys@transformshift{1.632780in}{0.872972in}%
\pgfsys@useobject{currentmarker}{}%
\end{pgfscope}%
\begin{pgfscope}%
\pgfsys@transformshift{1.635038in}{0.855595in}%
\pgfsys@useobject{currentmarker}{}%
\end{pgfscope}%
\begin{pgfscope}%
\pgfsys@transformshift{1.637275in}{0.828613in}%
\pgfsys@useobject{currentmarker}{}%
\end{pgfscope}%
\begin{pgfscope}%
\pgfsys@transformshift{1.639489in}{0.862219in}%
\pgfsys@useobject{currentmarker}{}%
\end{pgfscope}%
\begin{pgfscope}%
\pgfsys@transformshift{1.641682in}{0.869641in}%
\pgfsys@useobject{currentmarker}{}%
\end{pgfscope}%
\begin{pgfscope}%
\pgfsys@transformshift{1.643854in}{0.908632in}%
\pgfsys@useobject{currentmarker}{}%
\end{pgfscope}%
\begin{pgfscope}%
\pgfsys@transformshift{1.646005in}{0.872792in}%
\pgfsys@useobject{currentmarker}{}%
\end{pgfscope}%
\begin{pgfscope}%
\pgfsys@transformshift{1.648135in}{0.875636in}%
\pgfsys@useobject{currentmarker}{}%
\end{pgfscope}%
\begin{pgfscope}%
\pgfsys@transformshift{1.650246in}{0.887454in}%
\pgfsys@useobject{currentmarker}{}%
\end{pgfscope}%
\begin{pgfscope}%
\pgfsys@transformshift{1.652337in}{0.892554in}%
\pgfsys@useobject{currentmarker}{}%
\end{pgfscope}%
\begin{pgfscope}%
\pgfsys@transformshift{1.654408in}{0.908423in}%
\pgfsys@useobject{currentmarker}{}%
\end{pgfscope}%
\begin{pgfscope}%
\pgfsys@transformshift{1.656461in}{0.896224in}%
\pgfsys@useobject{currentmarker}{}%
\end{pgfscope}%
\begin{pgfscope}%
\pgfsys@transformshift{1.658495in}{0.860875in}%
\pgfsys@useobject{currentmarker}{}%
\end{pgfscope}%
\begin{pgfscope}%
\pgfsys@transformshift{1.660511in}{0.887937in}%
\pgfsys@useobject{currentmarker}{}%
\end{pgfscope}%
\begin{pgfscope}%
\pgfsys@transformshift{1.662509in}{0.869125in}%
\pgfsys@useobject{currentmarker}{}%
\end{pgfscope}%
\begin{pgfscope}%
\pgfsys@transformshift{1.664489in}{0.867503in}%
\pgfsys@useobject{currentmarker}{}%
\end{pgfscope}%
\begin{pgfscope}%
\pgfsys@transformshift{1.666452in}{0.865528in}%
\pgfsys@useobject{currentmarker}{}%
\end{pgfscope}%
\begin{pgfscope}%
\pgfsys@transformshift{1.668398in}{0.825752in}%
\pgfsys@useobject{currentmarker}{}%
\end{pgfscope}%
\begin{pgfscope}%
\pgfsys@transformshift{1.670327in}{0.848347in}%
\pgfsys@useobject{currentmarker}{}%
\end{pgfscope}%
\begin{pgfscope}%
\pgfsys@transformshift{1.672240in}{0.894180in}%
\pgfsys@useobject{currentmarker}{}%
\end{pgfscope}%
\begin{pgfscope}%
\pgfsys@transformshift{1.674137in}{0.881754in}%
\pgfsys@useobject{currentmarker}{}%
\end{pgfscope}%
\begin{pgfscope}%
\pgfsys@transformshift{1.676018in}{0.852703in}%
\pgfsys@useobject{currentmarker}{}%
\end{pgfscope}%
\begin{pgfscope}%
\pgfsys@transformshift{1.677883in}{0.854158in}%
\pgfsys@useobject{currentmarker}{}%
\end{pgfscope}%
\begin{pgfscope}%
\pgfsys@transformshift{1.679733in}{0.872484in}%
\pgfsys@useobject{currentmarker}{}%
\end{pgfscope}%
\begin{pgfscope}%
\pgfsys@transformshift{1.681568in}{0.839506in}%
\pgfsys@useobject{currentmarker}{}%
\end{pgfscope}%
\begin{pgfscope}%
\pgfsys@transformshift{1.683388in}{0.869854in}%
\pgfsys@useobject{currentmarker}{}%
\end{pgfscope}%
\begin{pgfscope}%
\pgfsys@transformshift{1.685193in}{0.861733in}%
\pgfsys@useobject{currentmarker}{}%
\end{pgfscope}%
\begin{pgfscope}%
\pgfsys@transformshift{1.686984in}{0.867266in}%
\pgfsys@useobject{currentmarker}{}%
\end{pgfscope}%
\begin{pgfscope}%
\pgfsys@transformshift{1.688761in}{0.850604in}%
\pgfsys@useobject{currentmarker}{}%
\end{pgfscope}%
\begin{pgfscope}%
\pgfsys@transformshift{1.690524in}{0.863991in}%
\pgfsys@useobject{currentmarker}{}%
\end{pgfscope}%
\begin{pgfscope}%
\pgfsys@transformshift{1.692273in}{0.862511in}%
\pgfsys@useobject{currentmarker}{}%
\end{pgfscope}%
\begin{pgfscope}%
\pgfsys@transformshift{1.694009in}{0.835319in}%
\pgfsys@useobject{currentmarker}{}%
\end{pgfscope}%
\begin{pgfscope}%
\pgfsys@transformshift{1.695731in}{0.874681in}%
\pgfsys@useobject{currentmarker}{}%
\end{pgfscope}%
\begin{pgfscope}%
\pgfsys@transformshift{1.697440in}{0.888982in}%
\pgfsys@useobject{currentmarker}{}%
\end{pgfscope}%
\begin{pgfscope}%
\pgfsys@transformshift{1.699137in}{0.861051in}%
\pgfsys@useobject{currentmarker}{}%
\end{pgfscope}%
\begin{pgfscope}%
\pgfsys@transformshift{1.700820in}{0.863790in}%
\pgfsys@useobject{currentmarker}{}%
\end{pgfscope}%
\begin{pgfscope}%
\pgfsys@transformshift{1.702491in}{0.895101in}%
\pgfsys@useobject{currentmarker}{}%
\end{pgfscope}%
\begin{pgfscope}%
\pgfsys@transformshift{1.704150in}{0.884287in}%
\pgfsys@useobject{currentmarker}{}%
\end{pgfscope}%
\begin{pgfscope}%
\pgfsys@transformshift{1.705797in}{0.921697in}%
\pgfsys@useobject{currentmarker}{}%
\end{pgfscope}%
\begin{pgfscope}%
\pgfsys@transformshift{1.707431in}{0.906466in}%
\pgfsys@useobject{currentmarker}{}%
\end{pgfscope}%
\begin{pgfscope}%
\pgfsys@transformshift{1.709054in}{0.874408in}%
\pgfsys@useobject{currentmarker}{}%
\end{pgfscope}%
\begin{pgfscope}%
\pgfsys@transformshift{1.710665in}{0.881441in}%
\pgfsys@useobject{currentmarker}{}%
\end{pgfscope}%
\begin{pgfscope}%
\pgfsys@transformshift{1.712265in}{0.847954in}%
\pgfsys@useobject{currentmarker}{}%
\end{pgfscope}%
\begin{pgfscope}%
\pgfsys@transformshift{1.713854in}{0.918553in}%
\pgfsys@useobject{currentmarker}{}%
\end{pgfscope}%
\begin{pgfscope}%
\pgfsys@transformshift{1.715431in}{0.871295in}%
\pgfsys@useobject{currentmarker}{}%
\end{pgfscope}%
\begin{pgfscope}%
\pgfsys@transformshift{1.716997in}{0.877249in}%
\pgfsys@useobject{currentmarker}{}%
\end{pgfscope}%
\begin{pgfscope}%
\pgfsys@transformshift{1.718553in}{0.808568in}%
\pgfsys@useobject{currentmarker}{}%
\end{pgfscope}%
\begin{pgfscope}%
\pgfsys@transformshift{1.720098in}{0.766890in}%
\pgfsys@useobject{currentmarker}{}%
\end{pgfscope}%
\begin{pgfscope}%
\pgfsys@transformshift{1.721632in}{0.826777in}%
\pgfsys@useobject{currentmarker}{}%
\end{pgfscope}%
\begin{pgfscope}%
\pgfsys@transformshift{1.723156in}{0.860634in}%
\pgfsys@useobject{currentmarker}{}%
\end{pgfscope}%
\begin{pgfscope}%
\pgfsys@transformshift{1.724669in}{0.873668in}%
\pgfsys@useobject{currentmarker}{}%
\end{pgfscope}%
\begin{pgfscope}%
\pgfsys@transformshift{1.726173in}{0.897229in}%
\pgfsys@useobject{currentmarker}{}%
\end{pgfscope}%
\begin{pgfscope}%
\pgfsys@transformshift{1.727666in}{0.860825in}%
\pgfsys@useobject{currentmarker}{}%
\end{pgfscope}%
\begin{pgfscope}%
\pgfsys@transformshift{1.729150in}{0.860282in}%
\pgfsys@useobject{currentmarker}{}%
\end{pgfscope}%
\begin{pgfscope}%
\pgfsys@transformshift{1.730624in}{0.873236in}%
\pgfsys@useobject{currentmarker}{}%
\end{pgfscope}%
\begin{pgfscope}%
\pgfsys@transformshift{1.732088in}{0.824307in}%
\pgfsys@useobject{currentmarker}{}%
\end{pgfscope}%
\begin{pgfscope}%
\pgfsys@transformshift{1.733543in}{0.847872in}%
\pgfsys@useobject{currentmarker}{}%
\end{pgfscope}%
\begin{pgfscope}%
\pgfsys@transformshift{1.734988in}{0.861568in}%
\pgfsys@useobject{currentmarker}{}%
\end{pgfscope}%
\begin{pgfscope}%
\pgfsys@transformshift{1.736424in}{0.892606in}%
\pgfsys@useobject{currentmarker}{}%
\end{pgfscope}%
\begin{pgfscope}%
\pgfsys@transformshift{1.737851in}{0.907781in}%
\pgfsys@useobject{currentmarker}{}%
\end{pgfscope}%
\begin{pgfscope}%
\pgfsys@transformshift{1.739269in}{0.887192in}%
\pgfsys@useobject{currentmarker}{}%
\end{pgfscope}%
\begin{pgfscope}%
\pgfsys@transformshift{1.740679in}{0.823646in}%
\pgfsys@useobject{currentmarker}{}%
\end{pgfscope}%
\begin{pgfscope}%
\pgfsys@transformshift{1.742079in}{0.753206in}%
\pgfsys@useobject{currentmarker}{}%
\end{pgfscope}%
\begin{pgfscope}%
\pgfsys@transformshift{1.743471in}{0.814960in}%
\pgfsys@useobject{currentmarker}{}%
\end{pgfscope}%
\begin{pgfscope}%
\pgfsys@transformshift{1.744854in}{0.846902in}%
\pgfsys@useobject{currentmarker}{}%
\end{pgfscope}%
\begin{pgfscope}%
\pgfsys@transformshift{1.746229in}{0.880779in}%
\pgfsys@useobject{currentmarker}{}%
\end{pgfscope}%
\begin{pgfscope}%
\pgfsys@transformshift{1.747595in}{0.899568in}%
\pgfsys@useobject{currentmarker}{}%
\end{pgfscope}%
\begin{pgfscope}%
\pgfsys@transformshift{1.748953in}{0.862380in}%
\pgfsys@useobject{currentmarker}{}%
\end{pgfscope}%
\begin{pgfscope}%
\pgfsys@transformshift{1.750303in}{0.810762in}%
\pgfsys@useobject{currentmarker}{}%
\end{pgfscope}%
\begin{pgfscope}%
\pgfsys@transformshift{1.751645in}{0.811822in}%
\pgfsys@useobject{currentmarker}{}%
\end{pgfscope}%
\begin{pgfscope}%
\pgfsys@transformshift{1.752979in}{0.775534in}%
\pgfsys@useobject{currentmarker}{}%
\end{pgfscope}%
\begin{pgfscope}%
\pgfsys@transformshift{1.754305in}{0.787638in}%
\pgfsys@useobject{currentmarker}{}%
\end{pgfscope}%
\begin{pgfscope}%
\pgfsys@transformshift{1.755623in}{0.797787in}%
\pgfsys@useobject{currentmarker}{}%
\end{pgfscope}%
\begin{pgfscope}%
\pgfsys@transformshift{1.756934in}{0.824830in}%
\pgfsys@useobject{currentmarker}{}%
\end{pgfscope}%
\begin{pgfscope}%
\pgfsys@transformshift{1.758237in}{0.805286in}%
\pgfsys@useobject{currentmarker}{}%
\end{pgfscope}%
\begin{pgfscope}%
\pgfsys@transformshift{1.759532in}{0.753892in}%
\pgfsys@useobject{currentmarker}{}%
\end{pgfscope}%
\begin{pgfscope}%
\pgfsys@transformshift{1.760820in}{0.736293in}%
\pgfsys@useobject{currentmarker}{}%
\end{pgfscope}%
\begin{pgfscope}%
\pgfsys@transformshift{1.762101in}{0.809565in}%
\pgfsys@useobject{currentmarker}{}%
\end{pgfscope}%
\begin{pgfscope}%
\pgfsys@transformshift{1.763374in}{0.864193in}%
\pgfsys@useobject{currentmarker}{}%
\end{pgfscope}%
\begin{pgfscope}%
\pgfsys@transformshift{1.764641in}{0.842150in}%
\pgfsys@useobject{currentmarker}{}%
\end{pgfscope}%
\begin{pgfscope}%
\pgfsys@transformshift{1.765900in}{0.834399in}%
\pgfsys@useobject{currentmarker}{}%
\end{pgfscope}%
\begin{pgfscope}%
\pgfsys@transformshift{1.767152in}{0.875827in}%
\pgfsys@useobject{currentmarker}{}%
\end{pgfscope}%
\begin{pgfscope}%
\pgfsys@transformshift{1.768397in}{0.882557in}%
\pgfsys@useobject{currentmarker}{}%
\end{pgfscope}%
\begin{pgfscope}%
\pgfsys@transformshift{1.769636in}{0.861331in}%
\pgfsys@useobject{currentmarker}{}%
\end{pgfscope}%
\begin{pgfscope}%
\pgfsys@transformshift{1.770867in}{0.836480in}%
\pgfsys@useobject{currentmarker}{}%
\end{pgfscope}%
\begin{pgfscope}%
\pgfsys@transformshift{1.772092in}{0.842557in}%
\pgfsys@useobject{currentmarker}{}%
\end{pgfscope}%
\begin{pgfscope}%
\pgfsys@transformshift{1.773310in}{0.874043in}%
\pgfsys@useobject{currentmarker}{}%
\end{pgfscope}%
\begin{pgfscope}%
\pgfsys@transformshift{1.774522in}{0.845546in}%
\pgfsys@useobject{currentmarker}{}%
\end{pgfscope}%
\begin{pgfscope}%
\pgfsys@transformshift{1.775727in}{0.839936in}%
\pgfsys@useobject{currentmarker}{}%
\end{pgfscope}%
\begin{pgfscope}%
\pgfsys@transformshift{1.776926in}{0.882789in}%
\pgfsys@useobject{currentmarker}{}%
\end{pgfscope}%
\begin{pgfscope}%
\pgfsys@transformshift{1.778118in}{0.865075in}%
\pgfsys@useobject{currentmarker}{}%
\end{pgfscope}%
\begin{pgfscope}%
\pgfsys@transformshift{1.779304in}{0.829644in}%
\pgfsys@useobject{currentmarker}{}%
\end{pgfscope}%
\begin{pgfscope}%
\pgfsys@transformshift{1.780484in}{0.886290in}%
\pgfsys@useobject{currentmarker}{}%
\end{pgfscope}%
\begin{pgfscope}%
\pgfsys@transformshift{1.781658in}{0.875900in}%
\pgfsys@useobject{currentmarker}{}%
\end{pgfscope}%
\begin{pgfscope}%
\pgfsys@transformshift{1.782826in}{0.830273in}%
\pgfsys@useobject{currentmarker}{}%
\end{pgfscope}%
\begin{pgfscope}%
\pgfsys@transformshift{1.783987in}{0.843497in}%
\pgfsys@useobject{currentmarker}{}%
\end{pgfscope}%
\begin{pgfscope}%
\pgfsys@transformshift{1.785143in}{0.870880in}%
\pgfsys@useobject{currentmarker}{}%
\end{pgfscope}%
\begin{pgfscope}%
\pgfsys@transformshift{1.786292in}{0.791482in}%
\pgfsys@useobject{currentmarker}{}%
\end{pgfscope}%
\begin{pgfscope}%
\pgfsys@transformshift{1.787436in}{0.780694in}%
\pgfsys@useobject{currentmarker}{}%
\end{pgfscope}%
\begin{pgfscope}%
\pgfsys@transformshift{1.788574in}{0.852013in}%
\pgfsys@useobject{currentmarker}{}%
\end{pgfscope}%
\begin{pgfscope}%
\pgfsys@transformshift{1.789707in}{0.880114in}%
\pgfsys@useobject{currentmarker}{}%
\end{pgfscope}%
\begin{pgfscope}%
\pgfsys@transformshift{1.790833in}{0.884900in}%
\pgfsys@useobject{currentmarker}{}%
\end{pgfscope}%
\begin{pgfscope}%
\pgfsys@transformshift{1.791954in}{0.910836in}%
\pgfsys@useobject{currentmarker}{}%
\end{pgfscope}%
\begin{pgfscope}%
\pgfsys@transformshift{1.793070in}{0.911537in}%
\pgfsys@useobject{currentmarker}{}%
\end{pgfscope}%
\begin{pgfscope}%
\pgfsys@transformshift{1.794180in}{0.860493in}%
\pgfsys@useobject{currentmarker}{}%
\end{pgfscope}%
\begin{pgfscope}%
\pgfsys@transformshift{1.795284in}{0.820264in}%
\pgfsys@useobject{currentmarker}{}%
\end{pgfscope}%
\begin{pgfscope}%
\pgfsys@transformshift{1.796383in}{0.779122in}%
\pgfsys@useobject{currentmarker}{}%
\end{pgfscope}%
\begin{pgfscope}%
\pgfsys@transformshift{1.797477in}{0.804464in}%
\pgfsys@useobject{currentmarker}{}%
\end{pgfscope}%
\begin{pgfscope}%
\pgfsys@transformshift{1.798565in}{0.858141in}%
\pgfsys@useobject{currentmarker}{}%
\end{pgfscope}%
\begin{pgfscope}%
\pgfsys@transformshift{1.799649in}{0.836824in}%
\pgfsys@useobject{currentmarker}{}%
\end{pgfscope}%
\begin{pgfscope}%
\pgfsys@transformshift{1.800727in}{0.830044in}%
\pgfsys@useobject{currentmarker}{}%
\end{pgfscope}%
\begin{pgfscope}%
\pgfsys@transformshift{1.801800in}{0.850609in}%
\pgfsys@useobject{currentmarker}{}%
\end{pgfscope}%
\begin{pgfscope}%
\pgfsys@transformshift{1.802867in}{0.798580in}%
\pgfsys@useobject{currentmarker}{}%
\end{pgfscope}%
\begin{pgfscope}%
\pgfsys@transformshift{1.803930in}{0.814061in}%
\pgfsys@useobject{currentmarker}{}%
\end{pgfscope}%
\begin{pgfscope}%
\pgfsys@transformshift{1.804988in}{0.806145in}%
\pgfsys@useobject{currentmarker}{}%
\end{pgfscope}%
\begin{pgfscope}%
\pgfsys@transformshift{1.806041in}{0.810365in}%
\pgfsys@useobject{currentmarker}{}%
\end{pgfscope}%
\begin{pgfscope}%
\pgfsys@transformshift{1.807088in}{0.824684in}%
\pgfsys@useobject{currentmarker}{}%
\end{pgfscope}%
\begin{pgfscope}%
\pgfsys@transformshift{1.808131in}{0.825589in}%
\pgfsys@useobject{currentmarker}{}%
\end{pgfscope}%
\begin{pgfscope}%
\pgfsys@transformshift{1.809170in}{0.837258in}%
\pgfsys@useobject{currentmarker}{}%
\end{pgfscope}%
\begin{pgfscope}%
\pgfsys@transformshift{1.810203in}{0.863216in}%
\pgfsys@useobject{currentmarker}{}%
\end{pgfscope}%
\begin{pgfscope}%
\pgfsys@transformshift{1.811232in}{0.832546in}%
\pgfsys@useobject{currentmarker}{}%
\end{pgfscope}%
\begin{pgfscope}%
\pgfsys@transformshift{1.812256in}{0.831154in}%
\pgfsys@useobject{currentmarker}{}%
\end{pgfscope}%
\begin{pgfscope}%
\pgfsys@transformshift{1.813275in}{0.801970in}%
\pgfsys@useobject{currentmarker}{}%
\end{pgfscope}%
\begin{pgfscope}%
\pgfsys@transformshift{1.814290in}{0.793106in}%
\pgfsys@useobject{currentmarker}{}%
\end{pgfscope}%
\begin{pgfscope}%
\pgfsys@transformshift{1.815300in}{0.819054in}%
\pgfsys@useobject{currentmarker}{}%
\end{pgfscope}%
\begin{pgfscope}%
\pgfsys@transformshift{1.816306in}{0.842745in}%
\pgfsys@useobject{currentmarker}{}%
\end{pgfscope}%
\begin{pgfscope}%
\pgfsys@transformshift{1.817307in}{0.838437in}%
\pgfsys@useobject{currentmarker}{}%
\end{pgfscope}%
\begin{pgfscope}%
\pgfsys@transformshift{1.818303in}{0.822283in}%
\pgfsys@useobject{currentmarker}{}%
\end{pgfscope}%
\begin{pgfscope}%
\pgfsys@transformshift{1.819296in}{0.785952in}%
\pgfsys@useobject{currentmarker}{}%
\end{pgfscope}%
\begin{pgfscope}%
\pgfsys@transformshift{1.820284in}{0.783481in}%
\pgfsys@useobject{currentmarker}{}%
\end{pgfscope}%
\begin{pgfscope}%
\pgfsys@transformshift{1.821267in}{0.824328in}%
\pgfsys@useobject{currentmarker}{}%
\end{pgfscope}%
\begin{pgfscope}%
\pgfsys@transformshift{1.822247in}{0.772564in}%
\pgfsys@useobject{currentmarker}{}%
\end{pgfscope}%
\begin{pgfscope}%
\pgfsys@transformshift{1.823222in}{0.737011in}%
\pgfsys@useobject{currentmarker}{}%
\end{pgfscope}%
\begin{pgfscope}%
\pgfsys@transformshift{1.824193in}{0.798803in}%
\pgfsys@useobject{currentmarker}{}%
\end{pgfscope}%
\begin{pgfscope}%
\pgfsys@transformshift{1.825160in}{0.820266in}%
\pgfsys@useobject{currentmarker}{}%
\end{pgfscope}%
\begin{pgfscope}%
\pgfsys@transformshift{1.826122in}{0.802825in}%
\pgfsys@useobject{currentmarker}{}%
\end{pgfscope}%
\begin{pgfscope}%
\pgfsys@transformshift{1.827081in}{0.857551in}%
\pgfsys@useobject{currentmarker}{}%
\end{pgfscope}%
\begin{pgfscope}%
\pgfsys@transformshift{1.828035in}{0.870513in}%
\pgfsys@useobject{currentmarker}{}%
\end{pgfscope}%
\begin{pgfscope}%
\pgfsys@transformshift{1.828985in}{0.819518in}%
\pgfsys@useobject{currentmarker}{}%
\end{pgfscope}%
\begin{pgfscope}%
\pgfsys@transformshift{1.829932in}{0.826446in}%
\pgfsys@useobject{currentmarker}{}%
\end{pgfscope}%
\begin{pgfscope}%
\pgfsys@transformshift{1.830874in}{0.815587in}%
\pgfsys@useobject{currentmarker}{}%
\end{pgfscope}%
\begin{pgfscope}%
\pgfsys@transformshift{1.831813in}{0.834643in}%
\pgfsys@useobject{currentmarker}{}%
\end{pgfscope}%
\begin{pgfscope}%
\pgfsys@transformshift{1.832747in}{0.845107in}%
\pgfsys@useobject{currentmarker}{}%
\end{pgfscope}%
\begin{pgfscope}%
\pgfsys@transformshift{1.833678in}{0.831540in}%
\pgfsys@useobject{currentmarker}{}%
\end{pgfscope}%
\begin{pgfscope}%
\pgfsys@transformshift{1.834605in}{0.771497in}%
\pgfsys@useobject{currentmarker}{}%
\end{pgfscope}%
\begin{pgfscope}%
\pgfsys@transformshift{1.835528in}{0.784273in}%
\pgfsys@useobject{currentmarker}{}%
\end{pgfscope}%
\begin{pgfscope}%
\pgfsys@transformshift{1.836447in}{0.795510in}%
\pgfsys@useobject{currentmarker}{}%
\end{pgfscope}%
\begin{pgfscope}%
\pgfsys@transformshift{1.837363in}{0.783491in}%
\pgfsys@useobject{currentmarker}{}%
\end{pgfscope}%
\begin{pgfscope}%
\pgfsys@transformshift{1.838275in}{0.799057in}%
\pgfsys@useobject{currentmarker}{}%
\end{pgfscope}%
\begin{pgfscope}%
\pgfsys@transformshift{1.839183in}{0.847104in}%
\pgfsys@useobject{currentmarker}{}%
\end{pgfscope}%
\begin{pgfscope}%
\pgfsys@transformshift{1.840087in}{0.855430in}%
\pgfsys@useobject{currentmarker}{}%
\end{pgfscope}%
\begin{pgfscope}%
\pgfsys@transformshift{1.840988in}{0.789830in}%
\pgfsys@useobject{currentmarker}{}%
\end{pgfscope}%
\begin{pgfscope}%
\pgfsys@transformshift{1.841885in}{0.800732in}%
\pgfsys@useobject{currentmarker}{}%
\end{pgfscope}%
\begin{pgfscope}%
\pgfsys@transformshift{1.842779in}{0.834040in}%
\pgfsys@useobject{currentmarker}{}%
\end{pgfscope}%
\begin{pgfscope}%
\pgfsys@transformshift{1.843669in}{0.847189in}%
\pgfsys@useobject{currentmarker}{}%
\end{pgfscope}%
\begin{pgfscope}%
\pgfsys@transformshift{1.844556in}{0.838129in}%
\pgfsys@useobject{currentmarker}{}%
\end{pgfscope}%
\begin{pgfscope}%
\pgfsys@transformshift{1.845439in}{0.855621in}%
\pgfsys@useobject{currentmarker}{}%
\end{pgfscope}%
\begin{pgfscope}%
\pgfsys@transformshift{1.846319in}{0.863931in}%
\pgfsys@useobject{currentmarker}{}%
\end{pgfscope}%
\begin{pgfscope}%
\pgfsys@transformshift{1.847195in}{0.835531in}%
\pgfsys@useobject{currentmarker}{}%
\end{pgfscope}%
\begin{pgfscope}%
\pgfsys@transformshift{1.848068in}{0.828915in}%
\pgfsys@useobject{currentmarker}{}%
\end{pgfscope}%
\begin{pgfscope}%
\pgfsys@transformshift{1.848937in}{0.846583in}%
\pgfsys@useobject{currentmarker}{}%
\end{pgfscope}%
\begin{pgfscope}%
\pgfsys@transformshift{1.849803in}{0.803440in}%
\pgfsys@useobject{currentmarker}{}%
\end{pgfscope}%
\begin{pgfscope}%
\pgfsys@transformshift{1.850666in}{0.835901in}%
\pgfsys@useobject{currentmarker}{}%
\end{pgfscope}%
\begin{pgfscope}%
\pgfsys@transformshift{1.851526in}{0.864104in}%
\pgfsys@useobject{currentmarker}{}%
\end{pgfscope}%
\begin{pgfscope}%
\pgfsys@transformshift{1.852382in}{0.820573in}%
\pgfsys@useobject{currentmarker}{}%
\end{pgfscope}%
\begin{pgfscope}%
\pgfsys@transformshift{1.853235in}{0.820652in}%
\pgfsys@useobject{currentmarker}{}%
\end{pgfscope}%
\begin{pgfscope}%
\pgfsys@transformshift{1.854085in}{0.818505in}%
\pgfsys@useobject{currentmarker}{}%
\end{pgfscope}%
\begin{pgfscope}%
\pgfsys@transformshift{1.854931in}{0.794195in}%
\pgfsys@useobject{currentmarker}{}%
\end{pgfscope}%
\begin{pgfscope}%
\pgfsys@transformshift{1.855775in}{0.757885in}%
\pgfsys@useobject{currentmarker}{}%
\end{pgfscope}%
\begin{pgfscope}%
\pgfsys@transformshift{1.856615in}{0.735314in}%
\pgfsys@useobject{currentmarker}{}%
\end{pgfscope}%
\begin{pgfscope}%
\pgfsys@transformshift{1.857452in}{0.769994in}%
\pgfsys@useobject{currentmarker}{}%
\end{pgfscope}%
\begin{pgfscope}%
\pgfsys@transformshift{1.858286in}{0.796215in}%
\pgfsys@useobject{currentmarker}{}%
\end{pgfscope}%
\begin{pgfscope}%
\pgfsys@transformshift{1.859117in}{0.788280in}%
\pgfsys@useobject{currentmarker}{}%
\end{pgfscope}%
\begin{pgfscope}%
\pgfsys@transformshift{1.859945in}{0.806094in}%
\pgfsys@useobject{currentmarker}{}%
\end{pgfscope}%
\begin{pgfscope}%
\pgfsys@transformshift{1.860770in}{0.841395in}%
\pgfsys@useobject{currentmarker}{}%
\end{pgfscope}%
\begin{pgfscope}%
\pgfsys@transformshift{1.861592in}{0.776402in}%
\pgfsys@useobject{currentmarker}{}%
\end{pgfscope}%
\begin{pgfscope}%
\pgfsys@transformshift{1.862410in}{0.770740in}%
\pgfsys@useobject{currentmarker}{}%
\end{pgfscope}%
\begin{pgfscope}%
\pgfsys@transformshift{1.863226in}{0.781994in}%
\pgfsys@useobject{currentmarker}{}%
\end{pgfscope}%
\begin{pgfscope}%
\pgfsys@transformshift{1.864039in}{0.823222in}%
\pgfsys@useobject{currentmarker}{}%
\end{pgfscope}%
\begin{pgfscope}%
\pgfsys@transformshift{1.864849in}{0.812461in}%
\pgfsys@useobject{currentmarker}{}%
\end{pgfscope}%
\begin{pgfscope}%
\pgfsys@transformshift{1.865656in}{0.813410in}%
\pgfsys@useobject{currentmarker}{}%
\end{pgfscope}%
\begin{pgfscope}%
\pgfsys@transformshift{1.866460in}{0.808492in}%
\pgfsys@useobject{currentmarker}{}%
\end{pgfscope}%
\begin{pgfscope}%
\pgfsys@transformshift{1.867262in}{0.782609in}%
\pgfsys@useobject{currentmarker}{}%
\end{pgfscope}%
\begin{pgfscope}%
\pgfsys@transformshift{1.868060in}{0.819064in}%
\pgfsys@useobject{currentmarker}{}%
\end{pgfscope}%
\begin{pgfscope}%
\pgfsys@transformshift{1.868856in}{0.813299in}%
\pgfsys@useobject{currentmarker}{}%
\end{pgfscope}%
\begin{pgfscope}%
\pgfsys@transformshift{1.869648in}{0.848573in}%
\pgfsys@useobject{currentmarker}{}%
\end{pgfscope}%
\begin{pgfscope}%
\pgfsys@transformshift{1.870438in}{0.822306in}%
\pgfsys@useobject{currentmarker}{}%
\end{pgfscope}%
\begin{pgfscope}%
\pgfsys@transformshift{1.871226in}{0.732776in}%
\pgfsys@useobject{currentmarker}{}%
\end{pgfscope}%
\begin{pgfscope}%
\pgfsys@transformshift{1.872010in}{0.795454in}%
\pgfsys@useobject{currentmarker}{}%
\end{pgfscope}%
\begin{pgfscope}%
\pgfsys@transformshift{1.872792in}{0.818603in}%
\pgfsys@useobject{currentmarker}{}%
\end{pgfscope}%
\begin{pgfscope}%
\pgfsys@transformshift{1.873571in}{0.778372in}%
\pgfsys@useobject{currentmarker}{}%
\end{pgfscope}%
\begin{pgfscope}%
\pgfsys@transformshift{1.874348in}{0.760139in}%
\pgfsys@useobject{currentmarker}{}%
\end{pgfscope}%
\begin{pgfscope}%
\pgfsys@transformshift{1.875121in}{0.771675in}%
\pgfsys@useobject{currentmarker}{}%
\end{pgfscope}%
\begin{pgfscope}%
\pgfsys@transformshift{1.875892in}{0.826048in}%
\pgfsys@useobject{currentmarker}{}%
\end{pgfscope}%
\begin{pgfscope}%
\pgfsys@transformshift{1.876661in}{0.844875in}%
\pgfsys@useobject{currentmarker}{}%
\end{pgfscope}%
\begin{pgfscope}%
\pgfsys@transformshift{1.877427in}{0.839180in}%
\pgfsys@useobject{currentmarker}{}%
\end{pgfscope}%
\begin{pgfscope}%
\pgfsys@transformshift{1.878190in}{0.775806in}%
\pgfsys@useobject{currentmarker}{}%
\end{pgfscope}%
\begin{pgfscope}%
\pgfsys@transformshift{1.878950in}{0.772517in}%
\pgfsys@useobject{currentmarker}{}%
\end{pgfscope}%
\begin{pgfscope}%
\pgfsys@transformshift{1.879708in}{0.782184in}%
\pgfsys@useobject{currentmarker}{}%
\end{pgfscope}%
\begin{pgfscope}%
\pgfsys@transformshift{1.880464in}{0.816209in}%
\pgfsys@useobject{currentmarker}{}%
\end{pgfscope}%
\begin{pgfscope}%
\pgfsys@transformshift{1.881217in}{0.815910in}%
\pgfsys@useobject{currentmarker}{}%
\end{pgfscope}%
\begin{pgfscope}%
\pgfsys@transformshift{1.881967in}{0.783113in}%
\pgfsys@useobject{currentmarker}{}%
\end{pgfscope}%
\begin{pgfscope}%
\pgfsys@transformshift{1.882715in}{0.819377in}%
\pgfsys@useobject{currentmarker}{}%
\end{pgfscope}%
\begin{pgfscope}%
\pgfsys@transformshift{1.883461in}{0.815651in}%
\pgfsys@useobject{currentmarker}{}%
\end{pgfscope}%
\begin{pgfscope}%
\pgfsys@transformshift{1.884204in}{0.783751in}%
\pgfsys@useobject{currentmarker}{}%
\end{pgfscope}%
\begin{pgfscope}%
\pgfsys@transformshift{1.884944in}{0.835091in}%
\pgfsys@useobject{currentmarker}{}%
\end{pgfscope}%
\begin{pgfscope}%
\pgfsys@transformshift{1.885683in}{0.833779in}%
\pgfsys@useobject{currentmarker}{}%
\end{pgfscope}%
\begin{pgfscope}%
\pgfsys@transformshift{1.886418in}{0.790706in}%
\pgfsys@useobject{currentmarker}{}%
\end{pgfscope}%
\begin{pgfscope}%
\pgfsys@transformshift{1.887152in}{0.774405in}%
\pgfsys@useobject{currentmarker}{}%
\end{pgfscope}%
\begin{pgfscope}%
\pgfsys@transformshift{1.887883in}{0.753493in}%
\pgfsys@useobject{currentmarker}{}%
\end{pgfscope}%
\begin{pgfscope}%
\pgfsys@transformshift{1.888611in}{0.840397in}%
\pgfsys@useobject{currentmarker}{}%
\end{pgfscope}%
\begin{pgfscope}%
\pgfsys@transformshift{1.889337in}{0.852694in}%
\pgfsys@useobject{currentmarker}{}%
\end{pgfscope}%
\begin{pgfscope}%
\pgfsys@transformshift{1.890061in}{0.807765in}%
\pgfsys@useobject{currentmarker}{}%
\end{pgfscope}%
\begin{pgfscope}%
\pgfsys@transformshift{1.890783in}{0.790088in}%
\pgfsys@useobject{currentmarker}{}%
\end{pgfscope}%
\begin{pgfscope}%
\pgfsys@transformshift{1.891502in}{0.788950in}%
\pgfsys@useobject{currentmarker}{}%
\end{pgfscope}%
\begin{pgfscope}%
\pgfsys@transformshift{1.892219in}{0.844409in}%
\pgfsys@useobject{currentmarker}{}%
\end{pgfscope}%
\begin{pgfscope}%
\pgfsys@transformshift{1.892934in}{0.822148in}%
\pgfsys@useobject{currentmarker}{}%
\end{pgfscope}%
\begin{pgfscope}%
\pgfsys@transformshift{1.893646in}{0.785170in}%
\pgfsys@useobject{currentmarker}{}%
\end{pgfscope}%
\begin{pgfscope}%
\pgfsys@transformshift{1.894356in}{0.839278in}%
\pgfsys@useobject{currentmarker}{}%
\end{pgfscope}%
\begin{pgfscope}%
\pgfsys@transformshift{1.895064in}{0.835073in}%
\pgfsys@useobject{currentmarker}{}%
\end{pgfscope}%
\begin{pgfscope}%
\pgfsys@transformshift{1.895770in}{0.822607in}%
\pgfsys@useobject{currentmarker}{}%
\end{pgfscope}%
\begin{pgfscope}%
\pgfsys@transformshift{1.896473in}{0.746406in}%
\pgfsys@useobject{currentmarker}{}%
\end{pgfscope}%
\begin{pgfscope}%
\pgfsys@transformshift{1.897175in}{0.786192in}%
\pgfsys@useobject{currentmarker}{}%
\end{pgfscope}%
\begin{pgfscope}%
\pgfsys@transformshift{1.897874in}{0.795706in}%
\pgfsys@useobject{currentmarker}{}%
\end{pgfscope}%
\begin{pgfscope}%
\pgfsys@transformshift{1.898571in}{0.802487in}%
\pgfsys@useobject{currentmarker}{}%
\end{pgfscope}%
\begin{pgfscope}%
\pgfsys@transformshift{1.899266in}{0.796826in}%
\pgfsys@useobject{currentmarker}{}%
\end{pgfscope}%
\begin{pgfscope}%
\pgfsys@transformshift{1.899958in}{0.796151in}%
\pgfsys@useobject{currentmarker}{}%
\end{pgfscope}%
\begin{pgfscope}%
\pgfsys@transformshift{1.900649in}{0.846166in}%
\pgfsys@useobject{currentmarker}{}%
\end{pgfscope}%
\begin{pgfscope}%
\pgfsys@transformshift{1.901337in}{0.858544in}%
\pgfsys@useobject{currentmarker}{}%
\end{pgfscope}%
\begin{pgfscope}%
\pgfsys@transformshift{1.902023in}{0.808916in}%
\pgfsys@useobject{currentmarker}{}%
\end{pgfscope}%
\begin{pgfscope}%
\pgfsys@transformshift{1.902708in}{0.784929in}%
\pgfsys@useobject{currentmarker}{}%
\end{pgfscope}%
\begin{pgfscope}%
\pgfsys@transformshift{1.903390in}{0.810391in}%
\pgfsys@useobject{currentmarker}{}%
\end{pgfscope}%
\begin{pgfscope}%
\pgfsys@transformshift{1.904070in}{0.890274in}%
\pgfsys@useobject{currentmarker}{}%
\end{pgfscope}%
\begin{pgfscope}%
\pgfsys@transformshift{1.904748in}{0.880065in}%
\pgfsys@useobject{currentmarker}{}%
\end{pgfscope}%
\begin{pgfscope}%
\pgfsys@transformshift{1.905424in}{0.842189in}%
\pgfsys@useobject{currentmarker}{}%
\end{pgfscope}%
\begin{pgfscope}%
\pgfsys@transformshift{1.906098in}{0.844019in}%
\pgfsys@useobject{currentmarker}{}%
\end{pgfscope}%
\begin{pgfscope}%
\pgfsys@transformshift{1.906770in}{0.794931in}%
\pgfsys@useobject{currentmarker}{}%
\end{pgfscope}%
\begin{pgfscope}%
\pgfsys@transformshift{1.907440in}{0.778038in}%
\pgfsys@useobject{currentmarker}{}%
\end{pgfscope}%
\begin{pgfscope}%
\pgfsys@transformshift{1.908108in}{0.780564in}%
\pgfsys@useobject{currentmarker}{}%
\end{pgfscope}%
\begin{pgfscope}%
\pgfsys@transformshift{1.908774in}{0.758377in}%
\pgfsys@useobject{currentmarker}{}%
\end{pgfscope}%
\begin{pgfscope}%
\pgfsys@transformshift{1.909438in}{0.752435in}%
\pgfsys@useobject{currentmarker}{}%
\end{pgfscope}%
\begin{pgfscope}%
\pgfsys@transformshift{1.910100in}{0.764869in}%
\pgfsys@useobject{currentmarker}{}%
\end{pgfscope}%
\begin{pgfscope}%
\pgfsys@transformshift{1.910760in}{0.757368in}%
\pgfsys@useobject{currentmarker}{}%
\end{pgfscope}%
\begin{pgfscope}%
\pgfsys@transformshift{1.911418in}{0.760172in}%
\pgfsys@useobject{currentmarker}{}%
\end{pgfscope}%
\begin{pgfscope}%
\pgfsys@transformshift{1.912074in}{0.814604in}%
\pgfsys@useobject{currentmarker}{}%
\end{pgfscope}%
\begin{pgfscope}%
\pgfsys@transformshift{1.912728in}{0.809340in}%
\pgfsys@useobject{currentmarker}{}%
\end{pgfscope}%
\begin{pgfscope}%
\pgfsys@transformshift{1.913381in}{0.844349in}%
\pgfsys@useobject{currentmarker}{}%
\end{pgfscope}%
\begin{pgfscope}%
\pgfsys@transformshift{1.914031in}{0.854978in}%
\pgfsys@useobject{currentmarker}{}%
\end{pgfscope}%
\begin{pgfscope}%
\pgfsys@transformshift{1.914680in}{0.763773in}%
\pgfsys@useobject{currentmarker}{}%
\end{pgfscope}%
\begin{pgfscope}%
\pgfsys@transformshift{1.915327in}{0.813770in}%
\pgfsys@useobject{currentmarker}{}%
\end{pgfscope}%
\begin{pgfscope}%
\pgfsys@transformshift{1.915972in}{0.802471in}%
\pgfsys@useobject{currentmarker}{}%
\end{pgfscope}%
\begin{pgfscope}%
\pgfsys@transformshift{1.916615in}{0.784611in}%
\pgfsys@useobject{currentmarker}{}%
\end{pgfscope}%
\begin{pgfscope}%
\pgfsys@transformshift{1.917256in}{0.763797in}%
\pgfsys@useobject{currentmarker}{}%
\end{pgfscope}%
\begin{pgfscope}%
\pgfsys@transformshift{1.917896in}{0.692818in}%
\pgfsys@useobject{currentmarker}{}%
\end{pgfscope}%
\begin{pgfscope}%
\pgfsys@transformshift{1.918533in}{0.785631in}%
\pgfsys@useobject{currentmarker}{}%
\end{pgfscope}%
\begin{pgfscope}%
\pgfsys@transformshift{1.919169in}{0.816414in}%
\pgfsys@useobject{currentmarker}{}%
\end{pgfscope}%
\begin{pgfscope}%
\pgfsys@transformshift{1.919803in}{0.789043in}%
\pgfsys@useobject{currentmarker}{}%
\end{pgfscope}%
\begin{pgfscope}%
\pgfsys@transformshift{1.920435in}{0.787819in}%
\pgfsys@useobject{currentmarker}{}%
\end{pgfscope}%
\begin{pgfscope}%
\pgfsys@transformshift{1.921066in}{0.788735in}%
\pgfsys@useobject{currentmarker}{}%
\end{pgfscope}%
\begin{pgfscope}%
\pgfsys@transformshift{1.921695in}{0.778832in}%
\pgfsys@useobject{currentmarker}{}%
\end{pgfscope}%
\begin{pgfscope}%
\pgfsys@transformshift{1.922322in}{0.731897in}%
\pgfsys@useobject{currentmarker}{}%
\end{pgfscope}%
\begin{pgfscope}%
\pgfsys@transformshift{1.922947in}{0.786133in}%
\pgfsys@useobject{currentmarker}{}%
\end{pgfscope}%
\begin{pgfscope}%
\pgfsys@transformshift{1.923570in}{0.734117in}%
\pgfsys@useobject{currentmarker}{}%
\end{pgfscope}%
\begin{pgfscope}%
\pgfsys@transformshift{1.924192in}{0.796973in}%
\pgfsys@useobject{currentmarker}{}%
\end{pgfscope}%
\begin{pgfscope}%
\pgfsys@transformshift{1.924812in}{0.828206in}%
\pgfsys@useobject{currentmarker}{}%
\end{pgfscope}%
\begin{pgfscope}%
\pgfsys@transformshift{1.925430in}{0.798907in}%
\pgfsys@useobject{currentmarker}{}%
\end{pgfscope}%
\begin{pgfscope}%
\pgfsys@transformshift{1.926047in}{0.784655in}%
\pgfsys@useobject{currentmarker}{}%
\end{pgfscope}%
\begin{pgfscope}%
\pgfsys@transformshift{1.926662in}{0.746672in}%
\pgfsys@useobject{currentmarker}{}%
\end{pgfscope}%
\begin{pgfscope}%
\pgfsys@transformshift{1.927275in}{0.781892in}%
\pgfsys@useobject{currentmarker}{}%
\end{pgfscope}%
\begin{pgfscope}%
\pgfsys@transformshift{1.927887in}{0.753307in}%
\pgfsys@useobject{currentmarker}{}%
\end{pgfscope}%
\begin{pgfscope}%
\pgfsys@transformshift{1.928497in}{0.735415in}%
\pgfsys@useobject{currentmarker}{}%
\end{pgfscope}%
\begin{pgfscope}%
\pgfsys@transformshift{1.929105in}{0.717928in}%
\pgfsys@useobject{currentmarker}{}%
\end{pgfscope}%
\begin{pgfscope}%
\pgfsys@transformshift{1.929712in}{0.757233in}%
\pgfsys@useobject{currentmarker}{}%
\end{pgfscope}%
\begin{pgfscope}%
\pgfsys@transformshift{1.930317in}{0.807773in}%
\pgfsys@useobject{currentmarker}{}%
\end{pgfscope}%
\begin{pgfscope}%
\pgfsys@transformshift{1.930920in}{0.786353in}%
\pgfsys@useobject{currentmarker}{}%
\end{pgfscope}%
\begin{pgfscope}%
\pgfsys@transformshift{1.931522in}{0.754010in}%
\pgfsys@useobject{currentmarker}{}%
\end{pgfscope}%
\begin{pgfscope}%
\pgfsys@transformshift{1.932122in}{0.774742in}%
\pgfsys@useobject{currentmarker}{}%
\end{pgfscope}%
\begin{pgfscope}%
\pgfsys@transformshift{1.932721in}{0.771781in}%
\pgfsys@useobject{currentmarker}{}%
\end{pgfscope}%
\begin{pgfscope}%
\pgfsys@transformshift{1.933318in}{0.803534in}%
\pgfsys@useobject{currentmarker}{}%
\end{pgfscope}%
\begin{pgfscope}%
\pgfsys@transformshift{1.933913in}{0.782299in}%
\pgfsys@useobject{currentmarker}{}%
\end{pgfscope}%
\begin{pgfscope}%
\pgfsys@transformshift{1.934507in}{0.770374in}%
\pgfsys@useobject{currentmarker}{}%
\end{pgfscope}%
\begin{pgfscope}%
\pgfsys@transformshift{1.935099in}{0.786873in}%
\pgfsys@useobject{currentmarker}{}%
\end{pgfscope}%
\begin{pgfscope}%
\pgfsys@transformshift{1.935690in}{0.764014in}%
\pgfsys@useobject{currentmarker}{}%
\end{pgfscope}%
\begin{pgfscope}%
\pgfsys@transformshift{1.936279in}{0.692623in}%
\pgfsys@useobject{currentmarker}{}%
\end{pgfscope}%
\begin{pgfscope}%
\pgfsys@transformshift{1.936867in}{0.779797in}%
\pgfsys@useobject{currentmarker}{}%
\end{pgfscope}%
\begin{pgfscope}%
\pgfsys@transformshift{1.937453in}{0.780849in}%
\pgfsys@useobject{currentmarker}{}%
\end{pgfscope}%
\begin{pgfscope}%
\pgfsys@transformshift{1.938037in}{0.776045in}%
\pgfsys@useobject{currentmarker}{}%
\end{pgfscope}%
\begin{pgfscope}%
\pgfsys@transformshift{1.938620in}{0.814678in}%
\pgfsys@useobject{currentmarker}{}%
\end{pgfscope}%
\begin{pgfscope}%
\pgfsys@transformshift{1.939202in}{0.821257in}%
\pgfsys@useobject{currentmarker}{}%
\end{pgfscope}%
\begin{pgfscope}%
\pgfsys@transformshift{1.939782in}{0.829375in}%
\pgfsys@useobject{currentmarker}{}%
\end{pgfscope}%
\begin{pgfscope}%
\pgfsys@transformshift{1.940360in}{0.749857in}%
\pgfsys@useobject{currentmarker}{}%
\end{pgfscope}%
\begin{pgfscope}%
\pgfsys@transformshift{1.940938in}{0.757648in}%
\pgfsys@useobject{currentmarker}{}%
\end{pgfscope}%
\begin{pgfscope}%
\pgfsys@transformshift{1.941513in}{0.744829in}%
\pgfsys@useobject{currentmarker}{}%
\end{pgfscope}%
\begin{pgfscope}%
\pgfsys@transformshift{1.942087in}{0.771904in}%
\pgfsys@useobject{currentmarker}{}%
\end{pgfscope}%
\begin{pgfscope}%
\pgfsys@transformshift{1.942660in}{0.788285in}%
\pgfsys@useobject{currentmarker}{}%
\end{pgfscope}%
\begin{pgfscope}%
\pgfsys@transformshift{1.943231in}{0.793095in}%
\pgfsys@useobject{currentmarker}{}%
\end{pgfscope}%
\begin{pgfscope}%
\pgfsys@transformshift{1.943801in}{0.804352in}%
\pgfsys@useobject{currentmarker}{}%
\end{pgfscope}%
\begin{pgfscope}%
\pgfsys@transformshift{1.944369in}{0.796928in}%
\pgfsys@useobject{currentmarker}{}%
\end{pgfscope}%
\begin{pgfscope}%
\pgfsys@transformshift{1.944936in}{0.793638in}%
\pgfsys@useobject{currentmarker}{}%
\end{pgfscope}%
\begin{pgfscope}%
\pgfsys@transformshift{1.945501in}{0.789693in}%
\pgfsys@useobject{currentmarker}{}%
\end{pgfscope}%
\begin{pgfscope}%
\pgfsys@transformshift{1.946065in}{0.799834in}%
\pgfsys@useobject{currentmarker}{}%
\end{pgfscope}%
\begin{pgfscope}%
\pgfsys@transformshift{1.946628in}{0.804129in}%
\pgfsys@useobject{currentmarker}{}%
\end{pgfscope}%
\begin{pgfscope}%
\pgfsys@transformshift{1.947189in}{0.779517in}%
\pgfsys@useobject{currentmarker}{}%
\end{pgfscope}%
\begin{pgfscope}%
\pgfsys@transformshift{1.947749in}{0.763290in}%
\pgfsys@useobject{currentmarker}{}%
\end{pgfscope}%
\begin{pgfscope}%
\pgfsys@transformshift{1.948308in}{0.760964in}%
\pgfsys@useobject{currentmarker}{}%
\end{pgfscope}%
\begin{pgfscope}%
\pgfsys@transformshift{1.948865in}{0.823954in}%
\pgfsys@useobject{currentmarker}{}%
\end{pgfscope}%
\begin{pgfscope}%
\pgfsys@transformshift{1.949420in}{0.812643in}%
\pgfsys@useobject{currentmarker}{}%
\end{pgfscope}%
\begin{pgfscope}%
\pgfsys@transformshift{1.949975in}{0.796909in}%
\pgfsys@useobject{currentmarker}{}%
\end{pgfscope}%
\begin{pgfscope}%
\pgfsys@transformshift{1.950527in}{0.783166in}%
\pgfsys@useobject{currentmarker}{}%
\end{pgfscope}%
\begin{pgfscope}%
\pgfsys@transformshift{1.951079in}{0.805153in}%
\pgfsys@useobject{currentmarker}{}%
\end{pgfscope}%
\begin{pgfscope}%
\pgfsys@transformshift{1.951629in}{0.781213in}%
\pgfsys@useobject{currentmarker}{}%
\end{pgfscope}%
\begin{pgfscope}%
\pgfsys@transformshift{1.952178in}{0.780836in}%
\pgfsys@useobject{currentmarker}{}%
\end{pgfscope}%
\begin{pgfscope}%
\pgfsys@transformshift{1.952726in}{0.748647in}%
\pgfsys@useobject{currentmarker}{}%
\end{pgfscope}%
\begin{pgfscope}%
\pgfsys@transformshift{1.953272in}{0.779132in}%
\pgfsys@useobject{currentmarker}{}%
\end{pgfscope}%
\begin{pgfscope}%
\pgfsys@transformshift{1.953817in}{0.822938in}%
\pgfsys@useobject{currentmarker}{}%
\end{pgfscope}%
\begin{pgfscope}%
\pgfsys@transformshift{1.954360in}{0.820584in}%
\pgfsys@useobject{currentmarker}{}%
\end{pgfscope}%
\begin{pgfscope}%
\pgfsys@transformshift{1.954903in}{0.802313in}%
\pgfsys@useobject{currentmarker}{}%
\end{pgfscope}%
\begin{pgfscope}%
\pgfsys@transformshift{1.955444in}{0.839565in}%
\pgfsys@useobject{currentmarker}{}%
\end{pgfscope}%
\begin{pgfscope}%
\pgfsys@transformshift{1.955983in}{0.825247in}%
\pgfsys@useobject{currentmarker}{}%
\end{pgfscope}%
\begin{pgfscope}%
\pgfsys@transformshift{1.956522in}{0.778231in}%
\pgfsys@useobject{currentmarker}{}%
\end{pgfscope}%
\begin{pgfscope}%
\pgfsys@transformshift{1.957059in}{0.797940in}%
\pgfsys@useobject{currentmarker}{}%
\end{pgfscope}%
\begin{pgfscope}%
\pgfsys@transformshift{1.957594in}{0.817551in}%
\pgfsys@useobject{currentmarker}{}%
\end{pgfscope}%
\begin{pgfscope}%
\pgfsys@transformshift{1.958129in}{0.808854in}%
\pgfsys@useobject{currentmarker}{}%
\end{pgfscope}%
\begin{pgfscope}%
\pgfsys@transformshift{1.958662in}{0.749164in}%
\pgfsys@useobject{currentmarker}{}%
\end{pgfscope}%
\begin{pgfscope}%
\pgfsys@transformshift{1.959194in}{0.782375in}%
\pgfsys@useobject{currentmarker}{}%
\end{pgfscope}%
\begin{pgfscope}%
\pgfsys@transformshift{1.959725in}{0.825793in}%
\pgfsys@useobject{currentmarker}{}%
\end{pgfscope}%
\begin{pgfscope}%
\pgfsys@transformshift{1.960254in}{0.789173in}%
\pgfsys@useobject{currentmarker}{}%
\end{pgfscope}%
\begin{pgfscope}%
\pgfsys@transformshift{1.960783in}{0.762954in}%
\pgfsys@useobject{currentmarker}{}%
\end{pgfscope}%
\begin{pgfscope}%
\pgfsys@transformshift{1.961310in}{0.720344in}%
\pgfsys@useobject{currentmarker}{}%
\end{pgfscope}%
\begin{pgfscope}%
\pgfsys@transformshift{1.961835in}{0.751996in}%
\pgfsys@useobject{currentmarker}{}%
\end{pgfscope}%
\begin{pgfscope}%
\pgfsys@transformshift{1.962360in}{0.773453in}%
\pgfsys@useobject{currentmarker}{}%
\end{pgfscope}%
\begin{pgfscope}%
\pgfsys@transformshift{1.962883in}{0.762442in}%
\pgfsys@useobject{currentmarker}{}%
\end{pgfscope}%
\begin{pgfscope}%
\pgfsys@transformshift{1.963405in}{0.782663in}%
\pgfsys@useobject{currentmarker}{}%
\end{pgfscope}%
\begin{pgfscope}%
\pgfsys@transformshift{1.963926in}{0.805440in}%
\pgfsys@useobject{currentmarker}{}%
\end{pgfscope}%
\begin{pgfscope}%
\pgfsys@transformshift{1.964446in}{0.778510in}%
\pgfsys@useobject{currentmarker}{}%
\end{pgfscope}%
\begin{pgfscope}%
\pgfsys@transformshift{1.964964in}{0.766775in}%
\pgfsys@useobject{currentmarker}{}%
\end{pgfscope}%
\begin{pgfscope}%
\pgfsys@transformshift{1.965482in}{0.804315in}%
\pgfsys@useobject{currentmarker}{}%
\end{pgfscope}%
\begin{pgfscope}%
\pgfsys@transformshift{1.965998in}{0.808056in}%
\pgfsys@useobject{currentmarker}{}%
\end{pgfscope}%
\begin{pgfscope}%
\pgfsys@transformshift{1.966513in}{0.758900in}%
\pgfsys@useobject{currentmarker}{}%
\end{pgfscope}%
\begin{pgfscope}%
\pgfsys@transformshift{1.967026in}{0.761614in}%
\pgfsys@useobject{currentmarker}{}%
\end{pgfscope}%
\begin{pgfscope}%
\pgfsys@transformshift{1.967539in}{0.811837in}%
\pgfsys@useobject{currentmarker}{}%
\end{pgfscope}%
\begin{pgfscope}%
\pgfsys@transformshift{1.968050in}{0.823480in}%
\pgfsys@useobject{currentmarker}{}%
\end{pgfscope}%
\begin{pgfscope}%
\pgfsys@transformshift{1.968561in}{0.798780in}%
\pgfsys@useobject{currentmarker}{}%
\end{pgfscope}%
\begin{pgfscope}%
\pgfsys@transformshift{1.969070in}{0.749858in}%
\pgfsys@useobject{currentmarker}{}%
\end{pgfscope}%
\begin{pgfscope}%
\pgfsys@transformshift{1.969578in}{0.745085in}%
\pgfsys@useobject{currentmarker}{}%
\end{pgfscope}%
\begin{pgfscope}%
\pgfsys@transformshift{1.970085in}{0.780587in}%
\pgfsys@useobject{currentmarker}{}%
\end{pgfscope}%
\begin{pgfscope}%
\pgfsys@transformshift{1.970590in}{0.753506in}%
\pgfsys@useobject{currentmarker}{}%
\end{pgfscope}%
\begin{pgfscope}%
\pgfsys@transformshift{1.971095in}{0.786110in}%
\pgfsys@useobject{currentmarker}{}%
\end{pgfscope}%
\begin{pgfscope}%
\pgfsys@transformshift{1.971598in}{0.768295in}%
\pgfsys@useobject{currentmarker}{}%
\end{pgfscope}%
\begin{pgfscope}%
\pgfsys@transformshift{1.972100in}{0.746676in}%
\pgfsys@useobject{currentmarker}{}%
\end{pgfscope}%
\begin{pgfscope}%
\pgfsys@transformshift{1.972601in}{0.737944in}%
\pgfsys@useobject{currentmarker}{}%
\end{pgfscope}%
\begin{pgfscope}%
\pgfsys@transformshift{1.973102in}{0.733020in}%
\pgfsys@useobject{currentmarker}{}%
\end{pgfscope}%
\begin{pgfscope}%
\pgfsys@transformshift{1.973600in}{0.802546in}%
\pgfsys@useobject{currentmarker}{}%
\end{pgfscope}%
\begin{pgfscope}%
\pgfsys@transformshift{1.974098in}{0.773586in}%
\pgfsys@useobject{currentmarker}{}%
\end{pgfscope}%
\begin{pgfscope}%
\pgfsys@transformshift{1.974595in}{0.747690in}%
\pgfsys@useobject{currentmarker}{}%
\end{pgfscope}%
\begin{pgfscope}%
\pgfsys@transformshift{1.975091in}{0.772994in}%
\pgfsys@useobject{currentmarker}{}%
\end{pgfscope}%
\begin{pgfscope}%
\pgfsys@transformshift{1.975585in}{0.782932in}%
\pgfsys@useobject{currentmarker}{}%
\end{pgfscope}%
\begin{pgfscope}%
\pgfsys@transformshift{1.976079in}{0.804070in}%
\pgfsys@useobject{currentmarker}{}%
\end{pgfscope}%
\begin{pgfscope}%
\pgfsys@transformshift{1.976571in}{0.796584in}%
\pgfsys@useobject{currentmarker}{}%
\end{pgfscope}%
\begin{pgfscope}%
\pgfsys@transformshift{1.977062in}{0.806800in}%
\pgfsys@useobject{currentmarker}{}%
\end{pgfscope}%
\begin{pgfscope}%
\pgfsys@transformshift{1.977552in}{0.791734in}%
\pgfsys@useobject{currentmarker}{}%
\end{pgfscope}%
\begin{pgfscope}%
\pgfsys@transformshift{1.978042in}{0.780798in}%
\pgfsys@useobject{currentmarker}{}%
\end{pgfscope}%
\begin{pgfscope}%
\pgfsys@transformshift{1.978530in}{0.779006in}%
\pgfsys@useobject{currentmarker}{}%
\end{pgfscope}%
\begin{pgfscope}%
\pgfsys@transformshift{1.979017in}{0.759611in}%
\pgfsys@useobject{currentmarker}{}%
\end{pgfscope}%
\begin{pgfscope}%
\pgfsys@transformshift{1.979503in}{0.749516in}%
\pgfsys@useobject{currentmarker}{}%
\end{pgfscope}%
\begin{pgfscope}%
\pgfsys@transformshift{1.979988in}{0.777817in}%
\pgfsys@useobject{currentmarker}{}%
\end{pgfscope}%
\begin{pgfscope}%
\pgfsys@transformshift{1.980472in}{0.754936in}%
\pgfsys@useobject{currentmarker}{}%
\end{pgfscope}%
\begin{pgfscope}%
\pgfsys@transformshift{1.980954in}{0.763636in}%
\pgfsys@useobject{currentmarker}{}%
\end{pgfscope}%
\begin{pgfscope}%
\pgfsys@transformshift{1.981436in}{0.770418in}%
\pgfsys@useobject{currentmarker}{}%
\end{pgfscope}%
\begin{pgfscope}%
\pgfsys@transformshift{1.981917in}{0.810585in}%
\pgfsys@useobject{currentmarker}{}%
\end{pgfscope}%
\begin{pgfscope}%
\pgfsys@transformshift{1.982397in}{0.802534in}%
\pgfsys@useobject{currentmarker}{}%
\end{pgfscope}%
\begin{pgfscope}%
\pgfsys@transformshift{1.982875in}{0.802273in}%
\pgfsys@useobject{currentmarker}{}%
\end{pgfscope}%
\begin{pgfscope}%
\pgfsys@transformshift{1.983353in}{0.799605in}%
\pgfsys@useobject{currentmarker}{}%
\end{pgfscope}%
\begin{pgfscope}%
\pgfsys@transformshift{1.983830in}{0.778553in}%
\pgfsys@useobject{currentmarker}{}%
\end{pgfscope}%
\begin{pgfscope}%
\pgfsys@transformshift{1.984306in}{0.778497in}%
\pgfsys@useobject{currentmarker}{}%
\end{pgfscope}%
\begin{pgfscope}%
\pgfsys@transformshift{1.984780in}{0.752571in}%
\pgfsys@useobject{currentmarker}{}%
\end{pgfscope}%
\begin{pgfscope}%
\pgfsys@transformshift{1.985254in}{0.778830in}%
\pgfsys@useobject{currentmarker}{}%
\end{pgfscope}%
\begin{pgfscope}%
\pgfsys@transformshift{1.985727in}{0.793990in}%
\pgfsys@useobject{currentmarker}{}%
\end{pgfscope}%
\begin{pgfscope}%
\pgfsys@transformshift{1.986198in}{0.774780in}%
\pgfsys@useobject{currentmarker}{}%
\end{pgfscope}%
\begin{pgfscope}%
\pgfsys@transformshift{1.986669in}{0.745582in}%
\pgfsys@useobject{currentmarker}{}%
\end{pgfscope}%
\begin{pgfscope}%
\pgfsys@transformshift{1.987139in}{0.757915in}%
\pgfsys@useobject{currentmarker}{}%
\end{pgfscope}%
\begin{pgfscope}%
\pgfsys@transformshift{1.987607in}{0.725364in}%
\pgfsys@useobject{currentmarker}{}%
\end{pgfscope}%
\begin{pgfscope}%
\pgfsys@transformshift{1.988075in}{0.754060in}%
\pgfsys@useobject{currentmarker}{}%
\end{pgfscope}%
\begin{pgfscope}%
\pgfsys@transformshift{1.988542in}{0.785918in}%
\pgfsys@useobject{currentmarker}{}%
\end{pgfscope}%
\begin{pgfscope}%
\pgfsys@transformshift{1.989008in}{0.757755in}%
\pgfsys@useobject{currentmarker}{}%
\end{pgfscope}%
\begin{pgfscope}%
\pgfsys@transformshift{1.989473in}{0.718218in}%
\pgfsys@useobject{currentmarker}{}%
\end{pgfscope}%
\begin{pgfscope}%
\pgfsys@transformshift{1.989937in}{0.742102in}%
\pgfsys@useobject{currentmarker}{}%
\end{pgfscope}%
\begin{pgfscope}%
\pgfsys@transformshift{1.990400in}{0.805181in}%
\pgfsys@useobject{currentmarker}{}%
\end{pgfscope}%
\begin{pgfscope}%
\pgfsys@transformshift{1.990862in}{0.817515in}%
\pgfsys@useobject{currentmarker}{}%
\end{pgfscope}%
\begin{pgfscope}%
\pgfsys@transformshift{1.991323in}{0.800802in}%
\pgfsys@useobject{currentmarker}{}%
\end{pgfscope}%
\begin{pgfscope}%
\pgfsys@transformshift{1.991783in}{0.745832in}%
\pgfsys@useobject{currentmarker}{}%
\end{pgfscope}%
\begin{pgfscope}%
\pgfsys@transformshift{1.992242in}{0.730297in}%
\pgfsys@useobject{currentmarker}{}%
\end{pgfscope}%
\begin{pgfscope}%
\pgfsys@transformshift{1.992700in}{0.762977in}%
\pgfsys@useobject{currentmarker}{}%
\end{pgfscope}%
\begin{pgfscope}%
\pgfsys@transformshift{1.993158in}{0.780594in}%
\pgfsys@useobject{currentmarker}{}%
\end{pgfscope}%
\begin{pgfscope}%
\pgfsys@transformshift{1.993614in}{0.773081in}%
\pgfsys@useobject{currentmarker}{}%
\end{pgfscope}%
\begin{pgfscope}%
\pgfsys@transformshift{1.994069in}{0.797248in}%
\pgfsys@useobject{currentmarker}{}%
\end{pgfscope}%
\begin{pgfscope}%
\pgfsys@transformshift{1.994524in}{0.780958in}%
\pgfsys@useobject{currentmarker}{}%
\end{pgfscope}%
\begin{pgfscope}%
\pgfsys@transformshift{1.994977in}{0.761462in}%
\pgfsys@useobject{currentmarker}{}%
\end{pgfscope}%
\begin{pgfscope}%
\pgfsys@transformshift{1.995430in}{0.757111in}%
\pgfsys@useobject{currentmarker}{}%
\end{pgfscope}%
\begin{pgfscope}%
\pgfsys@transformshift{1.995882in}{0.820583in}%
\pgfsys@useobject{currentmarker}{}%
\end{pgfscope}%
\begin{pgfscope}%
\pgfsys@transformshift{1.996333in}{0.818108in}%
\pgfsys@useobject{currentmarker}{}%
\end{pgfscope}%
\begin{pgfscope}%
\pgfsys@transformshift{1.996783in}{0.767232in}%
\pgfsys@useobject{currentmarker}{}%
\end{pgfscope}%
\begin{pgfscope}%
\pgfsys@transformshift{1.997232in}{0.755348in}%
\pgfsys@useobject{currentmarker}{}%
\end{pgfscope}%
\begin{pgfscope}%
\pgfsys@transformshift{1.997680in}{0.773615in}%
\pgfsys@useobject{currentmarker}{}%
\end{pgfscope}%
\begin{pgfscope}%
\pgfsys@transformshift{1.998127in}{0.790645in}%
\pgfsys@useobject{currentmarker}{}%
\end{pgfscope}%
\begin{pgfscope}%
\pgfsys@transformshift{1.998574in}{0.763911in}%
\pgfsys@useobject{currentmarker}{}%
\end{pgfscope}%
\begin{pgfscope}%
\pgfsys@transformshift{1.999019in}{0.771369in}%
\pgfsys@useobject{currentmarker}{}%
\end{pgfscope}%
\begin{pgfscope}%
\pgfsys@transformshift{1.999464in}{0.802267in}%
\pgfsys@useobject{currentmarker}{}%
\end{pgfscope}%
\begin{pgfscope}%
\pgfsys@transformshift{1.999908in}{0.782888in}%
\pgfsys@useobject{currentmarker}{}%
\end{pgfscope}%
\begin{pgfscope}%
\pgfsys@transformshift{2.000351in}{0.724551in}%
\pgfsys@useobject{currentmarker}{}%
\end{pgfscope}%
\begin{pgfscope}%
\pgfsys@transformshift{2.000793in}{0.774901in}%
\pgfsys@useobject{currentmarker}{}%
\end{pgfscope}%
\begin{pgfscope}%
\pgfsys@transformshift{2.001234in}{0.775237in}%
\pgfsys@useobject{currentmarker}{}%
\end{pgfscope}%
\begin{pgfscope}%
\pgfsys@transformshift{2.001674in}{0.794716in}%
\pgfsys@useobject{currentmarker}{}%
\end{pgfscope}%
\begin{pgfscope}%
\pgfsys@transformshift{2.002113in}{0.762981in}%
\pgfsys@useobject{currentmarker}{}%
\end{pgfscope}%
\begin{pgfscope}%
\pgfsys@transformshift{2.002552in}{0.764346in}%
\pgfsys@useobject{currentmarker}{}%
\end{pgfscope}%
\begin{pgfscope}%
\pgfsys@transformshift{2.002990in}{0.807820in}%
\pgfsys@useobject{currentmarker}{}%
\end{pgfscope}%
\begin{pgfscope}%
\pgfsys@transformshift{2.003427in}{0.695940in}%
\pgfsys@useobject{currentmarker}{}%
\end{pgfscope}%
\begin{pgfscope}%
\pgfsys@transformshift{2.003863in}{0.708013in}%
\pgfsys@useobject{currentmarker}{}%
\end{pgfscope}%
\begin{pgfscope}%
\pgfsys@transformshift{2.004298in}{0.745284in}%
\pgfsys@useobject{currentmarker}{}%
\end{pgfscope}%
\begin{pgfscope}%
\pgfsys@transformshift{2.004732in}{0.807232in}%
\pgfsys@useobject{currentmarker}{}%
\end{pgfscope}%
\begin{pgfscope}%
\pgfsys@transformshift{2.005166in}{0.786427in}%
\pgfsys@useobject{currentmarker}{}%
\end{pgfscope}%
\begin{pgfscope}%
\pgfsys@transformshift{2.005598in}{0.693146in}%
\pgfsys@useobject{currentmarker}{}%
\end{pgfscope}%
\begin{pgfscope}%
\pgfsys@transformshift{2.006030in}{0.725059in}%
\pgfsys@useobject{currentmarker}{}%
\end{pgfscope}%
\begin{pgfscope}%
\pgfsys@transformshift{2.006461in}{0.710819in}%
\pgfsys@useobject{currentmarker}{}%
\end{pgfscope}%
\begin{pgfscope}%
\pgfsys@transformshift{2.006891in}{0.802059in}%
\pgfsys@useobject{currentmarker}{}%
\end{pgfscope}%
\begin{pgfscope}%
\pgfsys@transformshift{2.007321in}{0.757783in}%
\pgfsys@useobject{currentmarker}{}%
\end{pgfscope}%
\begin{pgfscope}%
\pgfsys@transformshift{2.007749in}{0.738642in}%
\pgfsys@useobject{currentmarker}{}%
\end{pgfscope}%
\begin{pgfscope}%
\pgfsys@transformshift{2.008177in}{0.777374in}%
\pgfsys@useobject{currentmarker}{}%
\end{pgfscope}%
\begin{pgfscope}%
\pgfsys@transformshift{2.008604in}{0.804246in}%
\pgfsys@useobject{currentmarker}{}%
\end{pgfscope}%
\begin{pgfscope}%
\pgfsys@transformshift{2.009030in}{0.798750in}%
\pgfsys@useobject{currentmarker}{}%
\end{pgfscope}%
\begin{pgfscope}%
\pgfsys@transformshift{2.009455in}{0.756821in}%
\pgfsys@useobject{currentmarker}{}%
\end{pgfscope}%
\begin{pgfscope}%
\pgfsys@transformshift{2.009880in}{0.721975in}%
\pgfsys@useobject{currentmarker}{}%
\end{pgfscope}%
\begin{pgfscope}%
\pgfsys@transformshift{2.010303in}{0.756000in}%
\pgfsys@useobject{currentmarker}{}%
\end{pgfscope}%
\begin{pgfscope}%
\pgfsys@transformshift{2.010726in}{0.798512in}%
\pgfsys@useobject{currentmarker}{}%
\end{pgfscope}%
\begin{pgfscope}%
\pgfsys@transformshift{2.011148in}{0.777417in}%
\pgfsys@useobject{currentmarker}{}%
\end{pgfscope}%
\begin{pgfscope}%
\pgfsys@transformshift{2.011570in}{0.742773in}%
\pgfsys@useobject{currentmarker}{}%
\end{pgfscope}%
\begin{pgfscope}%
\pgfsys@transformshift{2.011990in}{0.760863in}%
\pgfsys@useobject{currentmarker}{}%
\end{pgfscope}%
\begin{pgfscope}%
\pgfsys@transformshift{2.012410in}{0.787380in}%
\pgfsys@useobject{currentmarker}{}%
\end{pgfscope}%
\begin{pgfscope}%
\pgfsys@transformshift{2.012829in}{0.797718in}%
\pgfsys@useobject{currentmarker}{}%
\end{pgfscope}%
\begin{pgfscope}%
\pgfsys@transformshift{2.013247in}{0.754295in}%
\pgfsys@useobject{currentmarker}{}%
\end{pgfscope}%
\begin{pgfscope}%
\pgfsys@transformshift{2.013664in}{0.757256in}%
\pgfsys@useobject{currentmarker}{}%
\end{pgfscope}%
\begin{pgfscope}%
\pgfsys@transformshift{2.014081in}{0.746771in}%
\pgfsys@useobject{currentmarker}{}%
\end{pgfscope}%
\begin{pgfscope}%
\pgfsys@transformshift{2.014497in}{0.772720in}%
\pgfsys@useobject{currentmarker}{}%
\end{pgfscope}%
\begin{pgfscope}%
\pgfsys@transformshift{2.014912in}{0.765896in}%
\pgfsys@useobject{currentmarker}{}%
\end{pgfscope}%
\begin{pgfscope}%
\pgfsys@transformshift{2.015326in}{0.788009in}%
\pgfsys@useobject{currentmarker}{}%
\end{pgfscope}%
\begin{pgfscope}%
\pgfsys@transformshift{2.015740in}{0.793587in}%
\pgfsys@useobject{currentmarker}{}%
\end{pgfscope}%
\begin{pgfscope}%
\pgfsys@transformshift{2.016152in}{0.758081in}%
\pgfsys@useobject{currentmarker}{}%
\end{pgfscope}%
\begin{pgfscope}%
\pgfsys@transformshift{2.016565in}{0.733514in}%
\pgfsys@useobject{currentmarker}{}%
\end{pgfscope}%
\begin{pgfscope}%
\pgfsys@transformshift{2.016976in}{0.772767in}%
\pgfsys@useobject{currentmarker}{}%
\end{pgfscope}%
\begin{pgfscope}%
\pgfsys@transformshift{2.017386in}{0.776843in}%
\pgfsys@useobject{currentmarker}{}%
\end{pgfscope}%
\begin{pgfscope}%
\pgfsys@transformshift{2.017796in}{0.752223in}%
\pgfsys@useobject{currentmarker}{}%
\end{pgfscope}%
\begin{pgfscope}%
\pgfsys@transformshift{2.018205in}{0.736593in}%
\pgfsys@useobject{currentmarker}{}%
\end{pgfscope}%
\begin{pgfscope}%
\pgfsys@transformshift{2.018613in}{0.747975in}%
\pgfsys@useobject{currentmarker}{}%
\end{pgfscope}%
\begin{pgfscope}%
\pgfsys@transformshift{2.019021in}{0.726829in}%
\pgfsys@useobject{currentmarker}{}%
\end{pgfscope}%
\begin{pgfscope}%
\pgfsys@transformshift{2.019428in}{0.769827in}%
\pgfsys@useobject{currentmarker}{}%
\end{pgfscope}%
\begin{pgfscope}%
\pgfsys@transformshift{2.019834in}{0.811112in}%
\pgfsys@useobject{currentmarker}{}%
\end{pgfscope}%
\begin{pgfscope}%
\pgfsys@transformshift{2.020239in}{0.786421in}%
\pgfsys@useobject{currentmarker}{}%
\end{pgfscope}%
\begin{pgfscope}%
\pgfsys@transformshift{2.020644in}{0.776535in}%
\pgfsys@useobject{currentmarker}{}%
\end{pgfscope}%
\begin{pgfscope}%
\pgfsys@transformshift{2.021048in}{0.788183in}%
\pgfsys@useobject{currentmarker}{}%
\end{pgfscope}%
\begin{pgfscope}%
\pgfsys@transformshift{2.021451in}{0.783972in}%
\pgfsys@useobject{currentmarker}{}%
\end{pgfscope}%
\begin{pgfscope}%
\pgfsys@transformshift{2.021853in}{0.770058in}%
\pgfsys@useobject{currentmarker}{}%
\end{pgfscope}%
\begin{pgfscope}%
\pgfsys@transformshift{2.022255in}{0.783234in}%
\pgfsys@useobject{currentmarker}{}%
\end{pgfscope}%
\begin{pgfscope}%
\pgfsys@transformshift{2.022656in}{0.769267in}%
\pgfsys@useobject{currentmarker}{}%
\end{pgfscope}%
\begin{pgfscope}%
\pgfsys@transformshift{2.023056in}{0.752964in}%
\pgfsys@useobject{currentmarker}{}%
\end{pgfscope}%
\begin{pgfscope}%
\pgfsys@transformshift{2.023456in}{0.744003in}%
\pgfsys@useobject{currentmarker}{}%
\end{pgfscope}%
\begin{pgfscope}%
\pgfsys@transformshift{2.023855in}{0.783936in}%
\pgfsys@useobject{currentmarker}{}%
\end{pgfscope}%
\begin{pgfscope}%
\pgfsys@transformshift{2.024253in}{0.799363in}%
\pgfsys@useobject{currentmarker}{}%
\end{pgfscope}%
\begin{pgfscope}%
\pgfsys@transformshift{2.024650in}{0.781398in}%
\pgfsys@useobject{currentmarker}{}%
\end{pgfscope}%
\begin{pgfscope}%
\pgfsys@transformshift{2.025047in}{0.801048in}%
\pgfsys@useobject{currentmarker}{}%
\end{pgfscope}%
\begin{pgfscope}%
\pgfsys@transformshift{2.025443in}{0.753031in}%
\pgfsys@useobject{currentmarker}{}%
\end{pgfscope}%
\begin{pgfscope}%
\pgfsys@transformshift{2.025839in}{0.701178in}%
\pgfsys@useobject{currentmarker}{}%
\end{pgfscope}%
\begin{pgfscope}%
\pgfsys@transformshift{2.026233in}{0.727554in}%
\pgfsys@useobject{currentmarker}{}%
\end{pgfscope}%
\begin{pgfscope}%
\pgfsys@transformshift{2.026627in}{0.789339in}%
\pgfsys@useobject{currentmarker}{}%
\end{pgfscope}%
\begin{pgfscope}%
\pgfsys@transformshift{2.027021in}{0.814156in}%
\pgfsys@useobject{currentmarker}{}%
\end{pgfscope}%
\begin{pgfscope}%
\pgfsys@transformshift{2.027413in}{0.785721in}%
\pgfsys@useobject{currentmarker}{}%
\end{pgfscope}%
\begin{pgfscope}%
\pgfsys@transformshift{2.027805in}{0.739937in}%
\pgfsys@useobject{currentmarker}{}%
\end{pgfscope}%
\begin{pgfscope}%
\pgfsys@transformshift{2.028196in}{0.732564in}%
\pgfsys@useobject{currentmarker}{}%
\end{pgfscope}%
\begin{pgfscope}%
\pgfsys@transformshift{2.028587in}{0.785297in}%
\pgfsys@useobject{currentmarker}{}%
\end{pgfscope}%
\begin{pgfscope}%
\pgfsys@transformshift{2.028977in}{0.763744in}%
\pgfsys@useobject{currentmarker}{}%
\end{pgfscope}%
\begin{pgfscope}%
\pgfsys@transformshift{2.029366in}{0.765545in}%
\pgfsys@useobject{currentmarker}{}%
\end{pgfscope}%
\begin{pgfscope}%
\pgfsys@transformshift{2.029754in}{0.735086in}%
\pgfsys@useobject{currentmarker}{}%
\end{pgfscope}%
\begin{pgfscope}%
\pgfsys@transformshift{2.030142in}{0.726827in}%
\pgfsys@useobject{currentmarker}{}%
\end{pgfscope}%
\begin{pgfscope}%
\pgfsys@transformshift{2.030530in}{0.796002in}%
\pgfsys@useobject{currentmarker}{}%
\end{pgfscope}%
\begin{pgfscope}%
\pgfsys@transformshift{2.030916in}{0.782729in}%
\pgfsys@useobject{currentmarker}{}%
\end{pgfscope}%
\begin{pgfscope}%
\pgfsys@transformshift{2.031302in}{0.722909in}%
\pgfsys@useobject{currentmarker}{}%
\end{pgfscope}%
\begin{pgfscope}%
\pgfsys@transformshift{2.031687in}{0.714257in}%
\pgfsys@useobject{currentmarker}{}%
\end{pgfscope}%
\begin{pgfscope}%
\pgfsys@transformshift{2.032072in}{0.753730in}%
\pgfsys@useobject{currentmarker}{}%
\end{pgfscope}%
\begin{pgfscope}%
\pgfsys@transformshift{2.032456in}{0.784633in}%
\pgfsys@useobject{currentmarker}{}%
\end{pgfscope}%
\begin{pgfscope}%
\pgfsys@transformshift{2.032839in}{0.779155in}%
\pgfsys@useobject{currentmarker}{}%
\end{pgfscope}%
\begin{pgfscope}%
\pgfsys@transformshift{2.033221in}{0.759748in}%
\pgfsys@useobject{currentmarker}{}%
\end{pgfscope}%
\begin{pgfscope}%
\pgfsys@transformshift{2.033603in}{0.733734in}%
\pgfsys@useobject{currentmarker}{}%
\end{pgfscope}%
\begin{pgfscope}%
\pgfsys@transformshift{2.033985in}{0.780536in}%
\pgfsys@useobject{currentmarker}{}%
\end{pgfscope}%
\begin{pgfscope}%
\pgfsys@transformshift{2.034365in}{0.735291in}%
\pgfsys@useobject{currentmarker}{}%
\end{pgfscope}%
\begin{pgfscope}%
\pgfsys@transformshift{2.034745in}{0.764428in}%
\pgfsys@useobject{currentmarker}{}%
\end{pgfscope}%
\begin{pgfscope}%
\pgfsys@transformshift{2.035125in}{0.779026in}%
\pgfsys@useobject{currentmarker}{}%
\end{pgfscope}%
\begin{pgfscope}%
\pgfsys@transformshift{2.035503in}{0.714103in}%
\pgfsys@useobject{currentmarker}{}%
\end{pgfscope}%
\begin{pgfscope}%
\pgfsys@transformshift{2.035881in}{0.766888in}%
\pgfsys@useobject{currentmarker}{}%
\end{pgfscope}%
\begin{pgfscope}%
\pgfsys@transformshift{2.036259in}{0.777279in}%
\pgfsys@useobject{currentmarker}{}%
\end{pgfscope}%
\begin{pgfscope}%
\pgfsys@transformshift{2.036636in}{0.765160in}%
\pgfsys@useobject{currentmarker}{}%
\end{pgfscope}%
\begin{pgfscope}%
\pgfsys@transformshift{2.037012in}{0.807441in}%
\pgfsys@useobject{currentmarker}{}%
\end{pgfscope}%
\begin{pgfscope}%
\pgfsys@transformshift{2.037387in}{0.803288in}%
\pgfsys@useobject{currentmarker}{}%
\end{pgfscope}%
\begin{pgfscope}%
\pgfsys@transformshift{2.037762in}{0.762306in}%
\pgfsys@useobject{currentmarker}{}%
\end{pgfscope}%
\begin{pgfscope}%
\pgfsys@transformshift{2.038137in}{0.755086in}%
\pgfsys@useobject{currentmarker}{}%
\end{pgfscope}%
\begin{pgfscope}%
\pgfsys@transformshift{2.038510in}{0.767921in}%
\pgfsys@useobject{currentmarker}{}%
\end{pgfscope}%
\begin{pgfscope}%
\pgfsys@transformshift{2.038883in}{0.765910in}%
\pgfsys@useobject{currentmarker}{}%
\end{pgfscope}%
\begin{pgfscope}%
\pgfsys@transformshift{2.039256in}{0.769035in}%
\pgfsys@useobject{currentmarker}{}%
\end{pgfscope}%
\begin{pgfscope}%
\pgfsys@transformshift{2.039627in}{0.715940in}%
\pgfsys@useobject{currentmarker}{}%
\end{pgfscope}%
\begin{pgfscope}%
\pgfsys@transformshift{2.039999in}{0.750994in}%
\pgfsys@useobject{currentmarker}{}%
\end{pgfscope}%
\begin{pgfscope}%
\pgfsys@transformshift{2.040369in}{0.748245in}%
\pgfsys@useobject{currentmarker}{}%
\end{pgfscope}%
\begin{pgfscope}%
\pgfsys@transformshift{2.040739in}{0.736650in}%
\pgfsys@useobject{currentmarker}{}%
\end{pgfscope}%
\begin{pgfscope}%
\pgfsys@transformshift{2.041109in}{0.780715in}%
\pgfsys@useobject{currentmarker}{}%
\end{pgfscope}%
\begin{pgfscope}%
\pgfsys@transformshift{2.041477in}{0.748386in}%
\pgfsys@useobject{currentmarker}{}%
\end{pgfscope}%
\begin{pgfscope}%
\pgfsys@transformshift{2.041846in}{0.752324in}%
\pgfsys@useobject{currentmarker}{}%
\end{pgfscope}%
\begin{pgfscope}%
\pgfsys@transformshift{2.042213in}{0.772688in}%
\pgfsys@useobject{currentmarker}{}%
\end{pgfscope}%
\begin{pgfscope}%
\pgfsys@transformshift{2.042580in}{0.797953in}%
\pgfsys@useobject{currentmarker}{}%
\end{pgfscope}%
\begin{pgfscope}%
\pgfsys@transformshift{2.042946in}{0.836178in}%
\pgfsys@useobject{currentmarker}{}%
\end{pgfscope}%
\begin{pgfscope}%
\pgfsys@transformshift{2.043312in}{0.742134in}%
\pgfsys@useobject{currentmarker}{}%
\end{pgfscope}%
\begin{pgfscope}%
\pgfsys@transformshift{2.043677in}{0.750667in}%
\pgfsys@useobject{currentmarker}{}%
\end{pgfscope}%
\begin{pgfscope}%
\pgfsys@transformshift{2.044042in}{0.809377in}%
\pgfsys@useobject{currentmarker}{}%
\end{pgfscope}%
\begin{pgfscope}%
\pgfsys@transformshift{2.044406in}{0.775719in}%
\pgfsys@useobject{currentmarker}{}%
\end{pgfscope}%
\begin{pgfscope}%
\pgfsys@transformshift{2.044769in}{0.737940in}%
\pgfsys@useobject{currentmarker}{}%
\end{pgfscope}%
\begin{pgfscope}%
\pgfsys@transformshift{2.045132in}{0.785984in}%
\pgfsys@useobject{currentmarker}{}%
\end{pgfscope}%
\begin{pgfscope}%
\pgfsys@transformshift{2.045494in}{0.765688in}%
\pgfsys@useobject{currentmarker}{}%
\end{pgfscope}%
\begin{pgfscope}%
\pgfsys@transformshift{2.045856in}{0.730737in}%
\pgfsys@useobject{currentmarker}{}%
\end{pgfscope}%
\begin{pgfscope}%
\pgfsys@transformshift{2.046217in}{0.736246in}%
\pgfsys@useobject{currentmarker}{}%
\end{pgfscope}%
\begin{pgfscope}%
\pgfsys@transformshift{2.046578in}{0.756753in}%
\pgfsys@useobject{currentmarker}{}%
\end{pgfscope}%
\begin{pgfscope}%
\pgfsys@transformshift{2.046938in}{0.765977in}%
\pgfsys@useobject{currentmarker}{}%
\end{pgfscope}%
\begin{pgfscope}%
\pgfsys@transformshift{2.047297in}{0.744708in}%
\pgfsys@useobject{currentmarker}{}%
\end{pgfscope}%
\begin{pgfscope}%
\pgfsys@transformshift{2.047656in}{0.716493in}%
\pgfsys@useobject{currentmarker}{}%
\end{pgfscope}%
\begin{pgfscope}%
\pgfsys@transformshift{2.048014in}{0.761971in}%
\pgfsys@useobject{currentmarker}{}%
\end{pgfscope}%
\begin{pgfscope}%
\pgfsys@transformshift{2.048371in}{0.768914in}%
\pgfsys@useobject{currentmarker}{}%
\end{pgfscope}%
\begin{pgfscope}%
\pgfsys@transformshift{2.048729in}{0.768934in}%
\pgfsys@useobject{currentmarker}{}%
\end{pgfscope}%
\begin{pgfscope}%
\pgfsys@transformshift{2.049085in}{0.769699in}%
\pgfsys@useobject{currentmarker}{}%
\end{pgfscope}%
\begin{pgfscope}%
\pgfsys@transformshift{2.049441in}{0.766481in}%
\pgfsys@useobject{currentmarker}{}%
\end{pgfscope}%
\begin{pgfscope}%
\pgfsys@transformshift{2.049796in}{0.750743in}%
\pgfsys@useobject{currentmarker}{}%
\end{pgfscope}%
\begin{pgfscope}%
\pgfsys@transformshift{2.050151in}{0.758301in}%
\pgfsys@useobject{currentmarker}{}%
\end{pgfscope}%
\begin{pgfscope}%
\pgfsys@transformshift{2.050505in}{0.746524in}%
\pgfsys@useobject{currentmarker}{}%
\end{pgfscope}%
\begin{pgfscope}%
\pgfsys@transformshift{2.050859in}{0.766215in}%
\pgfsys@useobject{currentmarker}{}%
\end{pgfscope}%
\begin{pgfscope}%
\pgfsys@transformshift{2.051212in}{0.781953in}%
\pgfsys@useobject{currentmarker}{}%
\end{pgfscope}%
\begin{pgfscope}%
\pgfsys@transformshift{2.051565in}{0.727521in}%
\pgfsys@useobject{currentmarker}{}%
\end{pgfscope}%
\begin{pgfscope}%
\pgfsys@transformshift{2.051917in}{0.699938in}%
\pgfsys@useobject{currentmarker}{}%
\end{pgfscope}%
\begin{pgfscope}%
\pgfsys@transformshift{2.052268in}{0.787128in}%
\pgfsys@useobject{currentmarker}{}%
\end{pgfscope}%
\begin{pgfscope}%
\pgfsys@transformshift{2.052619in}{0.800726in}%
\pgfsys@useobject{currentmarker}{}%
\end{pgfscope}%
\begin{pgfscope}%
\pgfsys@transformshift{2.052969in}{0.742788in}%
\pgfsys@useobject{currentmarker}{}%
\end{pgfscope}%
\begin{pgfscope}%
\pgfsys@transformshift{2.053319in}{0.802123in}%
\pgfsys@useobject{currentmarker}{}%
\end{pgfscope}%
\begin{pgfscope}%
\pgfsys@transformshift{2.053669in}{0.803955in}%
\pgfsys@useobject{currentmarker}{}%
\end{pgfscope}%
\begin{pgfscope}%
\pgfsys@transformshift{2.054017in}{0.767312in}%
\pgfsys@useobject{currentmarker}{}%
\end{pgfscope}%
\begin{pgfscope}%
\pgfsys@transformshift{2.054366in}{0.704634in}%
\pgfsys@useobject{currentmarker}{}%
\end{pgfscope}%
\begin{pgfscope}%
\pgfsys@transformshift{2.054713in}{0.699430in}%
\pgfsys@useobject{currentmarker}{}%
\end{pgfscope}%
\begin{pgfscope}%
\pgfsys@transformshift{2.055060in}{0.724507in}%
\pgfsys@useobject{currentmarker}{}%
\end{pgfscope}%
\begin{pgfscope}%
\pgfsys@transformshift{2.055407in}{0.738358in}%
\pgfsys@useobject{currentmarker}{}%
\end{pgfscope}%
\begin{pgfscope}%
\pgfsys@transformshift{2.055753in}{0.761263in}%
\pgfsys@useobject{currentmarker}{}%
\end{pgfscope}%
\begin{pgfscope}%
\pgfsys@transformshift{2.056098in}{0.771489in}%
\pgfsys@useobject{currentmarker}{}%
\end{pgfscope}%
\begin{pgfscope}%
\pgfsys@transformshift{2.056443in}{0.781188in}%
\pgfsys@useobject{currentmarker}{}%
\end{pgfscope}%
\begin{pgfscope}%
\pgfsys@transformshift{2.056788in}{0.765943in}%
\pgfsys@useobject{currentmarker}{}%
\end{pgfscope}%
\begin{pgfscope}%
\pgfsys@transformshift{2.057132in}{0.790219in}%
\pgfsys@useobject{currentmarker}{}%
\end{pgfscope}%
\begin{pgfscope}%
\pgfsys@transformshift{2.057475in}{0.765415in}%
\pgfsys@useobject{currentmarker}{}%
\end{pgfscope}%
\begin{pgfscope}%
\pgfsys@transformshift{2.057818in}{0.704501in}%
\pgfsys@useobject{currentmarker}{}%
\end{pgfscope}%
\begin{pgfscope}%
\pgfsys@transformshift{2.058161in}{0.714358in}%
\pgfsys@useobject{currentmarker}{}%
\end{pgfscope}%
\begin{pgfscope}%
\pgfsys@transformshift{2.058502in}{0.759006in}%
\pgfsys@useobject{currentmarker}{}%
\end{pgfscope}%
\begin{pgfscope}%
\pgfsys@transformshift{2.058844in}{0.758793in}%
\pgfsys@useobject{currentmarker}{}%
\end{pgfscope}%
\begin{pgfscope}%
\pgfsys@transformshift{2.059185in}{0.698004in}%
\pgfsys@useobject{currentmarker}{}%
\end{pgfscope}%
\begin{pgfscope}%
\pgfsys@transformshift{2.059525in}{0.742901in}%
\pgfsys@useobject{currentmarker}{}%
\end{pgfscope}%
\begin{pgfscope}%
\pgfsys@transformshift{2.059865in}{0.785526in}%
\pgfsys@useobject{currentmarker}{}%
\end{pgfscope}%
\begin{pgfscope}%
\pgfsys@transformshift{2.060204in}{0.800209in}%
\pgfsys@useobject{currentmarker}{}%
\end{pgfscope}%
\begin{pgfscope}%
\pgfsys@transformshift{2.060543in}{0.741072in}%
\pgfsys@useobject{currentmarker}{}%
\end{pgfscope}%
\begin{pgfscope}%
\pgfsys@transformshift{2.060881in}{0.738772in}%
\pgfsys@useobject{currentmarker}{}%
\end{pgfscope}%
\begin{pgfscope}%
\pgfsys@transformshift{2.061219in}{0.766643in}%
\pgfsys@useobject{currentmarker}{}%
\end{pgfscope}%
\begin{pgfscope}%
\pgfsys@transformshift{2.061556in}{0.751215in}%
\pgfsys@useobject{currentmarker}{}%
\end{pgfscope}%
\begin{pgfscope}%
\pgfsys@transformshift{2.061893in}{0.768031in}%
\pgfsys@useobject{currentmarker}{}%
\end{pgfscope}%
\begin{pgfscope}%
\pgfsys@transformshift{2.062229in}{0.798650in}%
\pgfsys@useobject{currentmarker}{}%
\end{pgfscope}%
\begin{pgfscope}%
\pgfsys@transformshift{2.062565in}{0.736361in}%
\pgfsys@useobject{currentmarker}{}%
\end{pgfscope}%
\begin{pgfscope}%
\pgfsys@transformshift{2.062900in}{0.736605in}%
\pgfsys@useobject{currentmarker}{}%
\end{pgfscope}%
\begin{pgfscope}%
\pgfsys@transformshift{2.063234in}{0.766239in}%
\pgfsys@useobject{currentmarker}{}%
\end{pgfscope}%
\begin{pgfscope}%
\pgfsys@transformshift{2.063569in}{0.700077in}%
\pgfsys@useobject{currentmarker}{}%
\end{pgfscope}%
\begin{pgfscope}%
\pgfsys@transformshift{2.063902in}{0.779110in}%
\pgfsys@useobject{currentmarker}{}%
\end{pgfscope}%
\begin{pgfscope}%
\pgfsys@transformshift{2.064236in}{0.759621in}%
\pgfsys@useobject{currentmarker}{}%
\end{pgfscope}%
\begin{pgfscope}%
\pgfsys@transformshift{2.064568in}{0.735098in}%
\pgfsys@useobject{currentmarker}{}%
\end{pgfscope}%
\begin{pgfscope}%
\pgfsys@transformshift{2.064901in}{0.694404in}%
\pgfsys@useobject{currentmarker}{}%
\end{pgfscope}%
\begin{pgfscope}%
\pgfsys@transformshift{2.065232in}{0.750958in}%
\pgfsys@useobject{currentmarker}{}%
\end{pgfscope}%
\begin{pgfscope}%
\pgfsys@transformshift{2.065564in}{0.766854in}%
\pgfsys@useobject{currentmarker}{}%
\end{pgfscope}%
\begin{pgfscope}%
\pgfsys@transformshift{2.065894in}{0.766884in}%
\pgfsys@useobject{currentmarker}{}%
\end{pgfscope}%
\begin{pgfscope}%
\pgfsys@transformshift{2.066225in}{0.756106in}%
\pgfsys@useobject{currentmarker}{}%
\end{pgfscope}%
\begin{pgfscope}%
\pgfsys@transformshift{2.066555in}{0.743667in}%
\pgfsys@useobject{currentmarker}{}%
\end{pgfscope}%
\begin{pgfscope}%
\pgfsys@transformshift{2.066884in}{0.725797in}%
\pgfsys@useobject{currentmarker}{}%
\end{pgfscope}%
\begin{pgfscope}%
\pgfsys@transformshift{2.067213in}{0.764477in}%
\pgfsys@useobject{currentmarker}{}%
\end{pgfscope}%
\begin{pgfscope}%
\pgfsys@transformshift{2.067541in}{0.764831in}%
\pgfsys@useobject{currentmarker}{}%
\end{pgfscope}%
\begin{pgfscope}%
\pgfsys@transformshift{2.067869in}{0.745769in}%
\pgfsys@useobject{currentmarker}{}%
\end{pgfscope}%
\begin{pgfscope}%
\pgfsys@transformshift{2.068196in}{0.717465in}%
\pgfsys@useobject{currentmarker}{}%
\end{pgfscope}%
\begin{pgfscope}%
\pgfsys@transformshift{2.068523in}{0.708973in}%
\pgfsys@useobject{currentmarker}{}%
\end{pgfscope}%
\begin{pgfscope}%
\pgfsys@transformshift{2.068850in}{0.718810in}%
\pgfsys@useobject{currentmarker}{}%
\end{pgfscope}%
\begin{pgfscope}%
\pgfsys@transformshift{2.069176in}{0.728700in}%
\pgfsys@useobject{currentmarker}{}%
\end{pgfscope}%
\begin{pgfscope}%
\pgfsys@transformshift{2.069501in}{0.736382in}%
\pgfsys@useobject{currentmarker}{}%
\end{pgfscope}%
\begin{pgfscope}%
\pgfsys@transformshift{2.069826in}{0.758322in}%
\pgfsys@useobject{currentmarker}{}%
\end{pgfscope}%
\begin{pgfscope}%
\pgfsys@transformshift{2.070151in}{0.738880in}%
\pgfsys@useobject{currentmarker}{}%
\end{pgfscope}%
\begin{pgfscope}%
\pgfsys@transformshift{2.070475in}{0.725713in}%
\pgfsys@useobject{currentmarker}{}%
\end{pgfscope}%
\begin{pgfscope}%
\pgfsys@transformshift{2.070799in}{0.708422in}%
\pgfsys@useobject{currentmarker}{}%
\end{pgfscope}%
\begin{pgfscope}%
\pgfsys@transformshift{2.071122in}{0.741043in}%
\pgfsys@useobject{currentmarker}{}%
\end{pgfscope}%
\begin{pgfscope}%
\pgfsys@transformshift{2.071444in}{0.739227in}%
\pgfsys@useobject{currentmarker}{}%
\end{pgfscope}%
\begin{pgfscope}%
\pgfsys@transformshift{2.071767in}{0.732933in}%
\pgfsys@useobject{currentmarker}{}%
\end{pgfscope}%
\begin{pgfscope}%
\pgfsys@transformshift{2.072088in}{0.762782in}%
\pgfsys@useobject{currentmarker}{}%
\end{pgfscope}%
\begin{pgfscope}%
\pgfsys@transformshift{2.072410in}{0.781908in}%
\pgfsys@useobject{currentmarker}{}%
\end{pgfscope}%
\begin{pgfscope}%
\pgfsys@transformshift{2.072731in}{0.771571in}%
\pgfsys@useobject{currentmarker}{}%
\end{pgfscope}%
\begin{pgfscope}%
\pgfsys@transformshift{2.073051in}{0.728758in}%
\pgfsys@useobject{currentmarker}{}%
\end{pgfscope}%
\begin{pgfscope}%
\pgfsys@transformshift{2.073371in}{0.767087in}%
\pgfsys@useobject{currentmarker}{}%
\end{pgfscope}%
\begin{pgfscope}%
\pgfsys@transformshift{2.073691in}{0.768547in}%
\pgfsys@useobject{currentmarker}{}%
\end{pgfscope}%
\begin{pgfscope}%
\pgfsys@transformshift{2.074010in}{0.735282in}%
\pgfsys@useobject{currentmarker}{}%
\end{pgfscope}%
\begin{pgfscope}%
\pgfsys@transformshift{2.074328in}{0.747363in}%
\pgfsys@useobject{currentmarker}{}%
\end{pgfscope}%
\begin{pgfscope}%
\pgfsys@transformshift{2.074646in}{0.707105in}%
\pgfsys@useobject{currentmarker}{}%
\end{pgfscope}%
\begin{pgfscope}%
\pgfsys@transformshift{2.074964in}{0.749147in}%
\pgfsys@useobject{currentmarker}{}%
\end{pgfscope}%
\begin{pgfscope}%
\pgfsys@transformshift{2.075281in}{0.739238in}%
\pgfsys@useobject{currentmarker}{}%
\end{pgfscope}%
\begin{pgfscope}%
\pgfsys@transformshift{2.075598in}{0.721836in}%
\pgfsys@useobject{currentmarker}{}%
\end{pgfscope}%
\begin{pgfscope}%
\pgfsys@transformshift{2.075914in}{0.694480in}%
\pgfsys@useobject{currentmarker}{}%
\end{pgfscope}%
\begin{pgfscope}%
\pgfsys@transformshift{2.076230in}{0.753313in}%
\pgfsys@useobject{currentmarker}{}%
\end{pgfscope}%
\begin{pgfscope}%
\pgfsys@transformshift{2.076546in}{0.773945in}%
\pgfsys@useobject{currentmarker}{}%
\end{pgfscope}%
\begin{pgfscope}%
\pgfsys@transformshift{2.076861in}{0.760914in}%
\pgfsys@useobject{currentmarker}{}%
\end{pgfscope}%
\begin{pgfscope}%
\pgfsys@transformshift{2.077175in}{0.713454in}%
\pgfsys@useobject{currentmarker}{}%
\end{pgfscope}%
\begin{pgfscope}%
\pgfsys@transformshift{2.077489in}{0.702125in}%
\pgfsys@useobject{currentmarker}{}%
\end{pgfscope}%
\begin{pgfscope}%
\pgfsys@transformshift{2.077803in}{0.797648in}%
\pgfsys@useobject{currentmarker}{}%
\end{pgfscope}%
\begin{pgfscope}%
\pgfsys@transformshift{2.078116in}{0.793540in}%
\pgfsys@useobject{currentmarker}{}%
\end{pgfscope}%
\begin{pgfscope}%
\pgfsys@transformshift{2.078429in}{0.760162in}%
\pgfsys@useobject{currentmarker}{}%
\end{pgfscope}%
\begin{pgfscope}%
\pgfsys@transformshift{2.078742in}{0.744165in}%
\pgfsys@useobject{currentmarker}{}%
\end{pgfscope}%
\begin{pgfscope}%
\pgfsys@transformshift{2.079054in}{0.749967in}%
\pgfsys@useobject{currentmarker}{}%
\end{pgfscope}%
\begin{pgfscope}%
\pgfsys@transformshift{2.079365in}{0.768849in}%
\pgfsys@useobject{currentmarker}{}%
\end{pgfscope}%
\begin{pgfscope}%
\pgfsys@transformshift{2.079676in}{0.723357in}%
\pgfsys@useobject{currentmarker}{}%
\end{pgfscope}%
\begin{pgfscope}%
\pgfsys@transformshift{2.079987in}{0.784243in}%
\pgfsys@useobject{currentmarker}{}%
\end{pgfscope}%
\begin{pgfscope}%
\pgfsys@transformshift{2.080297in}{0.777168in}%
\pgfsys@useobject{currentmarker}{}%
\end{pgfscope}%
\begin{pgfscope}%
\pgfsys@transformshift{2.080607in}{0.791192in}%
\pgfsys@useobject{currentmarker}{}%
\end{pgfscope}%
\begin{pgfscope}%
\pgfsys@transformshift{2.080916in}{0.785736in}%
\pgfsys@useobject{currentmarker}{}%
\end{pgfscope}%
\begin{pgfscope}%
\pgfsys@transformshift{2.081225in}{0.730298in}%
\pgfsys@useobject{currentmarker}{}%
\end{pgfscope}%
\begin{pgfscope}%
\pgfsys@transformshift{2.081534in}{0.719031in}%
\pgfsys@useobject{currentmarker}{}%
\end{pgfscope}%
\begin{pgfscope}%
\pgfsys@transformshift{2.081842in}{0.735570in}%
\pgfsys@useobject{currentmarker}{}%
\end{pgfscope}%
\begin{pgfscope}%
\pgfsys@transformshift{2.082150in}{0.735097in}%
\pgfsys@useobject{currentmarker}{}%
\end{pgfscope}%
\begin{pgfscope}%
\pgfsys@transformshift{2.082457in}{0.717640in}%
\pgfsys@useobject{currentmarker}{}%
\end{pgfscope}%
\begin{pgfscope}%
\pgfsys@transformshift{2.082764in}{0.760614in}%
\pgfsys@useobject{currentmarker}{}%
\end{pgfscope}%
\begin{pgfscope}%
\pgfsys@transformshift{2.083070in}{0.770280in}%
\pgfsys@useobject{currentmarker}{}%
\end{pgfscope}%
\begin{pgfscope}%
\pgfsys@transformshift{2.083376in}{0.768843in}%
\pgfsys@useobject{currentmarker}{}%
\end{pgfscope}%
\begin{pgfscope}%
\pgfsys@transformshift{2.083682in}{0.765422in}%
\pgfsys@useobject{currentmarker}{}%
\end{pgfscope}%
\begin{pgfscope}%
\pgfsys@transformshift{2.083987in}{0.708917in}%
\pgfsys@useobject{currentmarker}{}%
\end{pgfscope}%
\begin{pgfscope}%
\pgfsys@transformshift{2.084292in}{0.738905in}%
\pgfsys@useobject{currentmarker}{}%
\end{pgfscope}%
\begin{pgfscope}%
\pgfsys@transformshift{2.084596in}{0.752188in}%
\pgfsys@useobject{currentmarker}{}%
\end{pgfscope}%
\begin{pgfscope}%
\pgfsys@transformshift{2.084900in}{0.707301in}%
\pgfsys@useobject{currentmarker}{}%
\end{pgfscope}%
\begin{pgfscope}%
\pgfsys@transformshift{2.085203in}{0.650101in}%
\pgfsys@useobject{currentmarker}{}%
\end{pgfscope}%
\begin{pgfscope}%
\pgfsys@transformshift{2.085507in}{0.758772in}%
\pgfsys@useobject{currentmarker}{}%
\end{pgfscope}%
\begin{pgfscope}%
\pgfsys@transformshift{2.085809in}{0.776947in}%
\pgfsys@useobject{currentmarker}{}%
\end{pgfscope}%
\begin{pgfscope}%
\pgfsys@transformshift{2.086112in}{0.755248in}%
\pgfsys@useobject{currentmarker}{}%
\end{pgfscope}%
\begin{pgfscope}%
\pgfsys@transformshift{2.086413in}{0.741261in}%
\pgfsys@useobject{currentmarker}{}%
\end{pgfscope}%
\begin{pgfscope}%
\pgfsys@transformshift{2.086715in}{0.744348in}%
\pgfsys@useobject{currentmarker}{}%
\end{pgfscope}%
\begin{pgfscope}%
\pgfsys@transformshift{2.087016in}{0.697243in}%
\pgfsys@useobject{currentmarker}{}%
\end{pgfscope}%
\begin{pgfscope}%
\pgfsys@transformshift{2.087317in}{0.729103in}%
\pgfsys@useobject{currentmarker}{}%
\end{pgfscope}%
\begin{pgfscope}%
\pgfsys@transformshift{2.087617in}{0.782107in}%
\pgfsys@useobject{currentmarker}{}%
\end{pgfscope}%
\begin{pgfscope}%
\pgfsys@transformshift{2.087917in}{0.743802in}%
\pgfsys@useobject{currentmarker}{}%
\end{pgfscope}%
\begin{pgfscope}%
\pgfsys@transformshift{2.088216in}{0.751662in}%
\pgfsys@useobject{currentmarker}{}%
\end{pgfscope}%
\begin{pgfscope}%
\pgfsys@transformshift{2.088516in}{0.779330in}%
\pgfsys@useobject{currentmarker}{}%
\end{pgfscope}%
\begin{pgfscope}%
\pgfsys@transformshift{2.088814in}{0.764847in}%
\pgfsys@useobject{currentmarker}{}%
\end{pgfscope}%
\begin{pgfscope}%
\pgfsys@transformshift{2.089112in}{0.762067in}%
\pgfsys@useobject{currentmarker}{}%
\end{pgfscope}%
\begin{pgfscope}%
\pgfsys@transformshift{2.089410in}{0.754489in}%
\pgfsys@useobject{currentmarker}{}%
\end{pgfscope}%
\begin{pgfscope}%
\pgfsys@transformshift{2.089708in}{0.703449in}%
\pgfsys@useobject{currentmarker}{}%
\end{pgfscope}%
\begin{pgfscope}%
\pgfsys@transformshift{2.090005in}{0.738568in}%
\pgfsys@useobject{currentmarker}{}%
\end{pgfscope}%
\begin{pgfscope}%
\pgfsys@transformshift{2.090302in}{0.739125in}%
\pgfsys@useobject{currentmarker}{}%
\end{pgfscope}%
\begin{pgfscope}%
\pgfsys@transformshift{2.090598in}{0.723088in}%
\pgfsys@useobject{currentmarker}{}%
\end{pgfscope}%
\begin{pgfscope}%
\pgfsys@transformshift{2.090894in}{0.772610in}%
\pgfsys@useobject{currentmarker}{}%
\end{pgfscope}%
\begin{pgfscope}%
\pgfsys@transformshift{2.091190in}{0.780699in}%
\pgfsys@useobject{currentmarker}{}%
\end{pgfscope}%
\begin{pgfscope}%
\pgfsys@transformshift{2.091485in}{0.691717in}%
\pgfsys@useobject{currentmarker}{}%
\end{pgfscope}%
\begin{pgfscope}%
\pgfsys@transformshift{2.091779in}{0.738175in}%
\pgfsys@useobject{currentmarker}{}%
\end{pgfscope}%
\begin{pgfscope}%
\pgfsys@transformshift{2.092074in}{0.697063in}%
\pgfsys@useobject{currentmarker}{}%
\end{pgfscope}%
\begin{pgfscope}%
\pgfsys@transformshift{2.092368in}{0.753642in}%
\pgfsys@useobject{currentmarker}{}%
\end{pgfscope}%
\begin{pgfscope}%
\pgfsys@transformshift{2.092661in}{0.750846in}%
\pgfsys@useobject{currentmarker}{}%
\end{pgfscope}%
\begin{pgfscope}%
\pgfsys@transformshift{2.092955in}{0.732437in}%
\pgfsys@useobject{currentmarker}{}%
\end{pgfscope}%
\begin{pgfscope}%
\pgfsys@transformshift{2.093248in}{0.755229in}%
\pgfsys@useobject{currentmarker}{}%
\end{pgfscope}%
\begin{pgfscope}%
\pgfsys@transformshift{2.093540in}{0.751890in}%
\pgfsys@useobject{currentmarker}{}%
\end{pgfscope}%
\begin{pgfscope}%
\pgfsys@transformshift{2.093832in}{0.698817in}%
\pgfsys@useobject{currentmarker}{}%
\end{pgfscope}%
\begin{pgfscope}%
\pgfsys@transformshift{2.094124in}{0.659611in}%
\pgfsys@useobject{currentmarker}{}%
\end{pgfscope}%
\begin{pgfscope}%
\pgfsys@transformshift{2.094415in}{0.735742in}%
\pgfsys@useobject{currentmarker}{}%
\end{pgfscope}%
\begin{pgfscope}%
\pgfsys@transformshift{2.094706in}{0.759320in}%
\pgfsys@useobject{currentmarker}{}%
\end{pgfscope}%
\begin{pgfscope}%
\pgfsys@transformshift{2.094997in}{0.764573in}%
\pgfsys@useobject{currentmarker}{}%
\end{pgfscope}%
\begin{pgfscope}%
\pgfsys@transformshift{2.095287in}{0.766011in}%
\pgfsys@useobject{currentmarker}{}%
\end{pgfscope}%
\begin{pgfscope}%
\pgfsys@transformshift{2.095577in}{0.740076in}%
\pgfsys@useobject{currentmarker}{}%
\end{pgfscope}%
\begin{pgfscope}%
\pgfsys@transformshift{2.095866in}{0.663696in}%
\pgfsys@useobject{currentmarker}{}%
\end{pgfscope}%
\begin{pgfscope}%
\pgfsys@transformshift{2.096155in}{0.716913in}%
\pgfsys@useobject{currentmarker}{}%
\end{pgfscope}%
\begin{pgfscope}%
\pgfsys@transformshift{2.096444in}{0.744308in}%
\pgfsys@useobject{currentmarker}{}%
\end{pgfscope}%
\begin{pgfscope}%
\pgfsys@transformshift{2.096732in}{0.768567in}%
\pgfsys@useobject{currentmarker}{}%
\end{pgfscope}%
\begin{pgfscope}%
\pgfsys@transformshift{2.097020in}{0.767308in}%
\pgfsys@useobject{currentmarker}{}%
\end{pgfscope}%
\begin{pgfscope}%
\pgfsys@transformshift{2.097308in}{0.750682in}%
\pgfsys@useobject{currentmarker}{}%
\end{pgfscope}%
\begin{pgfscope}%
\pgfsys@transformshift{2.097595in}{0.762500in}%
\pgfsys@useobject{currentmarker}{}%
\end{pgfscope}%
\begin{pgfscope}%
\pgfsys@transformshift{2.097882in}{0.775490in}%
\pgfsys@useobject{currentmarker}{}%
\end{pgfscope}%
\begin{pgfscope}%
\pgfsys@transformshift{2.098169in}{0.798985in}%
\pgfsys@useobject{currentmarker}{}%
\end{pgfscope}%
\begin{pgfscope}%
\pgfsys@transformshift{2.098455in}{0.790745in}%
\pgfsys@useobject{currentmarker}{}%
\end{pgfscope}%
\begin{pgfscope}%
\pgfsys@transformshift{2.098740in}{0.789982in}%
\pgfsys@useobject{currentmarker}{}%
\end{pgfscope}%
\begin{pgfscope}%
\pgfsys@transformshift{2.099026in}{0.755140in}%
\pgfsys@useobject{currentmarker}{}%
\end{pgfscope}%
\begin{pgfscope}%
\pgfsys@transformshift{2.099311in}{0.732940in}%
\pgfsys@useobject{currentmarker}{}%
\end{pgfscope}%
\begin{pgfscope}%
\pgfsys@transformshift{2.099596in}{0.751802in}%
\pgfsys@useobject{currentmarker}{}%
\end{pgfscope}%
\begin{pgfscope}%
\pgfsys@transformshift{2.099880in}{0.781597in}%
\pgfsys@useobject{currentmarker}{}%
\end{pgfscope}%
\begin{pgfscope}%
\pgfsys@transformshift{2.100164in}{0.795070in}%
\pgfsys@useobject{currentmarker}{}%
\end{pgfscope}%
\begin{pgfscope}%
\pgfsys@transformshift{2.100448in}{0.748938in}%
\pgfsys@useobject{currentmarker}{}%
\end{pgfscope}%
\begin{pgfscope}%
\pgfsys@transformshift{2.100731in}{0.733920in}%
\pgfsys@useobject{currentmarker}{}%
\end{pgfscope}%
\begin{pgfscope}%
\pgfsys@transformshift{2.101014in}{0.754896in}%
\pgfsys@useobject{currentmarker}{}%
\end{pgfscope}%
\begin{pgfscope}%
\pgfsys@transformshift{2.101296in}{0.762791in}%
\pgfsys@useobject{currentmarker}{}%
\end{pgfscope}%
\begin{pgfscope}%
\pgfsys@transformshift{2.101578in}{0.745277in}%
\pgfsys@useobject{currentmarker}{}%
\end{pgfscope}%
\begin{pgfscope}%
\pgfsys@transformshift{2.101860in}{0.775102in}%
\pgfsys@useobject{currentmarker}{}%
\end{pgfscope}%
\begin{pgfscope}%
\pgfsys@transformshift{2.102142in}{0.752883in}%
\pgfsys@useobject{currentmarker}{}%
\end{pgfscope}%
\begin{pgfscope}%
\pgfsys@transformshift{2.102423in}{0.726866in}%
\pgfsys@useobject{currentmarker}{}%
\end{pgfscope}%
\begin{pgfscope}%
\pgfsys@transformshift{2.102704in}{0.757677in}%
\pgfsys@useobject{currentmarker}{}%
\end{pgfscope}%
\begin{pgfscope}%
\pgfsys@transformshift{2.102984in}{0.749858in}%
\pgfsys@useobject{currentmarker}{}%
\end{pgfscope}%
\begin{pgfscope}%
\pgfsys@transformshift{2.103264in}{0.715309in}%
\pgfsys@useobject{currentmarker}{}%
\end{pgfscope}%
\begin{pgfscope}%
\pgfsys@transformshift{2.103544in}{0.707442in}%
\pgfsys@useobject{currentmarker}{}%
\end{pgfscope}%
\begin{pgfscope}%
\pgfsys@transformshift{2.103823in}{0.796823in}%
\pgfsys@useobject{currentmarker}{}%
\end{pgfscope}%
\begin{pgfscope}%
\pgfsys@transformshift{2.104102in}{0.785943in}%
\pgfsys@useobject{currentmarker}{}%
\end{pgfscope}%
\begin{pgfscope}%
\pgfsys@transformshift{2.104381in}{0.760831in}%
\pgfsys@useobject{currentmarker}{}%
\end{pgfscope}%
\begin{pgfscope}%
\pgfsys@transformshift{2.104659in}{0.762117in}%
\pgfsys@useobject{currentmarker}{}%
\end{pgfscope}%
\begin{pgfscope}%
\pgfsys@transformshift{2.104937in}{0.755535in}%
\pgfsys@useobject{currentmarker}{}%
\end{pgfscope}%
\begin{pgfscope}%
\pgfsys@transformshift{2.105215in}{0.665531in}%
\pgfsys@useobject{currentmarker}{}%
\end{pgfscope}%
\begin{pgfscope}%
\pgfsys@transformshift{2.105492in}{0.729017in}%
\pgfsys@useobject{currentmarker}{}%
\end{pgfscope}%
\begin{pgfscope}%
\pgfsys@transformshift{2.105769in}{0.753924in}%
\pgfsys@useobject{currentmarker}{}%
\end{pgfscope}%
\begin{pgfscope}%
\pgfsys@transformshift{2.106046in}{0.708229in}%
\pgfsys@useobject{currentmarker}{}%
\end{pgfscope}%
\begin{pgfscope}%
\pgfsys@transformshift{2.106322in}{0.698889in}%
\pgfsys@useobject{currentmarker}{}%
\end{pgfscope}%
\begin{pgfscope}%
\pgfsys@transformshift{2.106598in}{0.729909in}%
\pgfsys@useobject{currentmarker}{}%
\end{pgfscope}%
\begin{pgfscope}%
\pgfsys@transformshift{2.106874in}{0.762514in}%
\pgfsys@useobject{currentmarker}{}%
\end{pgfscope}%
\begin{pgfscope}%
\pgfsys@transformshift{2.107149in}{0.741463in}%
\pgfsys@useobject{currentmarker}{}%
\end{pgfscope}%
\begin{pgfscope}%
\pgfsys@transformshift{2.107424in}{0.752821in}%
\pgfsys@useobject{currentmarker}{}%
\end{pgfscope}%
\begin{pgfscope}%
\pgfsys@transformshift{2.107699in}{0.750121in}%
\pgfsys@useobject{currentmarker}{}%
\end{pgfscope}%
\begin{pgfscope}%
\pgfsys@transformshift{2.107973in}{0.738762in}%
\pgfsys@useobject{currentmarker}{}%
\end{pgfscope}%
\begin{pgfscope}%
\pgfsys@transformshift{2.108247in}{0.747659in}%
\pgfsys@useobject{currentmarker}{}%
\end{pgfscope}%
\begin{pgfscope}%
\pgfsys@transformshift{2.108520in}{0.724718in}%
\pgfsys@useobject{currentmarker}{}%
\end{pgfscope}%
\begin{pgfscope}%
\pgfsys@transformshift{2.108794in}{0.725274in}%
\pgfsys@useobject{currentmarker}{}%
\end{pgfscope}%
\begin{pgfscope}%
\pgfsys@transformshift{2.109067in}{0.736326in}%
\pgfsys@useobject{currentmarker}{}%
\end{pgfscope}%
\begin{pgfscope}%
\pgfsys@transformshift{2.109339in}{0.749728in}%
\pgfsys@useobject{currentmarker}{}%
\end{pgfscope}%
\begin{pgfscope}%
\pgfsys@transformshift{2.109612in}{0.713716in}%
\pgfsys@useobject{currentmarker}{}%
\end{pgfscope}%
\begin{pgfscope}%
\pgfsys@transformshift{2.109883in}{0.734817in}%
\pgfsys@useobject{currentmarker}{}%
\end{pgfscope}%
\begin{pgfscope}%
\pgfsys@transformshift{2.110155in}{0.788539in}%
\pgfsys@useobject{currentmarker}{}%
\end{pgfscope}%
\begin{pgfscope}%
\pgfsys@transformshift{2.110426in}{0.783622in}%
\pgfsys@useobject{currentmarker}{}%
\end{pgfscope}%
\begin{pgfscope}%
\pgfsys@transformshift{2.110697in}{0.735760in}%
\pgfsys@useobject{currentmarker}{}%
\end{pgfscope}%
\begin{pgfscope}%
\pgfsys@transformshift{2.110968in}{0.720328in}%
\pgfsys@useobject{currentmarker}{}%
\end{pgfscope}%
\begin{pgfscope}%
\pgfsys@transformshift{2.111238in}{0.742968in}%
\pgfsys@useobject{currentmarker}{}%
\end{pgfscope}%
\begin{pgfscope}%
\pgfsys@transformshift{2.111508in}{0.711006in}%
\pgfsys@useobject{currentmarker}{}%
\end{pgfscope}%
\begin{pgfscope}%
\pgfsys@transformshift{2.111778in}{0.660595in}%
\pgfsys@useobject{currentmarker}{}%
\end{pgfscope}%
\begin{pgfscope}%
\pgfsys@transformshift{2.112047in}{0.717334in}%
\pgfsys@useobject{currentmarker}{}%
\end{pgfscope}%
\begin{pgfscope}%
\pgfsys@transformshift{2.112316in}{0.768128in}%
\pgfsys@useobject{currentmarker}{}%
\end{pgfscope}%
\begin{pgfscope}%
\pgfsys@transformshift{2.112585in}{0.758423in}%
\pgfsys@useobject{currentmarker}{}%
\end{pgfscope}%
\begin{pgfscope}%
\pgfsys@transformshift{2.112853in}{0.759604in}%
\pgfsys@useobject{currentmarker}{}%
\end{pgfscope}%
\begin{pgfscope}%
\pgfsys@transformshift{2.113121in}{0.756806in}%
\pgfsys@useobject{currentmarker}{}%
\end{pgfscope}%
\begin{pgfscope}%
\pgfsys@transformshift{2.113389in}{0.737992in}%
\pgfsys@useobject{currentmarker}{}%
\end{pgfscope}%
\begin{pgfscope}%
\pgfsys@transformshift{2.113657in}{0.763639in}%
\pgfsys@useobject{currentmarker}{}%
\end{pgfscope}%
\begin{pgfscope}%
\pgfsys@transformshift{2.113924in}{0.730653in}%
\pgfsys@useobject{currentmarker}{}%
\end{pgfscope}%
\begin{pgfscope}%
\pgfsys@transformshift{2.114190in}{0.749609in}%
\pgfsys@useobject{currentmarker}{}%
\end{pgfscope}%
\begin{pgfscope}%
\pgfsys@transformshift{2.114457in}{0.738264in}%
\pgfsys@useobject{currentmarker}{}%
\end{pgfscope}%
\begin{pgfscope}%
\pgfsys@transformshift{2.114723in}{0.763275in}%
\pgfsys@useobject{currentmarker}{}%
\end{pgfscope}%
\begin{pgfscope}%
\pgfsys@transformshift{2.114989in}{0.802353in}%
\pgfsys@useobject{currentmarker}{}%
\end{pgfscope}%
\begin{pgfscope}%
\pgfsys@transformshift{2.115254in}{0.777616in}%
\pgfsys@useobject{currentmarker}{}%
\end{pgfscope}%
\begin{pgfscope}%
\pgfsys@transformshift{2.115520in}{0.715850in}%
\pgfsys@useobject{currentmarker}{}%
\end{pgfscope}%
\begin{pgfscope}%
\pgfsys@transformshift{2.115785in}{0.689104in}%
\pgfsys@useobject{currentmarker}{}%
\end{pgfscope}%
\begin{pgfscope}%
\pgfsys@transformshift{2.116049in}{0.716633in}%
\pgfsys@useobject{currentmarker}{}%
\end{pgfscope}%
\begin{pgfscope}%
\pgfsys@transformshift{2.116313in}{0.765192in}%
\pgfsys@useobject{currentmarker}{}%
\end{pgfscope}%
\begin{pgfscope}%
\pgfsys@transformshift{2.116577in}{0.739715in}%
\pgfsys@useobject{currentmarker}{}%
\end{pgfscope}%
\begin{pgfscope}%
\pgfsys@transformshift{2.116841in}{0.669404in}%
\pgfsys@useobject{currentmarker}{}%
\end{pgfscope}%
\begin{pgfscope}%
\pgfsys@transformshift{2.117104in}{0.684418in}%
\pgfsys@useobject{currentmarker}{}%
\end{pgfscope}%
\begin{pgfscope}%
\pgfsys@transformshift{2.117367in}{0.688615in}%
\pgfsys@useobject{currentmarker}{}%
\end{pgfscope}%
\begin{pgfscope}%
\pgfsys@transformshift{2.117630in}{0.729777in}%
\pgfsys@useobject{currentmarker}{}%
\end{pgfscope}%
\begin{pgfscope}%
\pgfsys@transformshift{2.117893in}{0.740024in}%
\pgfsys@useobject{currentmarker}{}%
\end{pgfscope}%
\begin{pgfscope}%
\pgfsys@transformshift{2.118155in}{0.751779in}%
\pgfsys@useobject{currentmarker}{}%
\end{pgfscope}%
\begin{pgfscope}%
\pgfsys@transformshift{2.118417in}{0.770149in}%
\pgfsys@useobject{currentmarker}{}%
\end{pgfscope}%
\begin{pgfscope}%
\pgfsys@transformshift{2.118678in}{0.738981in}%
\pgfsys@useobject{currentmarker}{}%
\end{pgfscope}%
\begin{pgfscope}%
\pgfsys@transformshift{2.118939in}{0.729257in}%
\pgfsys@useobject{currentmarker}{}%
\end{pgfscope}%
\begin{pgfscope}%
\pgfsys@transformshift{2.119200in}{0.720489in}%
\pgfsys@useobject{currentmarker}{}%
\end{pgfscope}%
\begin{pgfscope}%
\pgfsys@transformshift{2.119461in}{0.727377in}%
\pgfsys@useobject{currentmarker}{}%
\end{pgfscope}%
\begin{pgfscope}%
\pgfsys@transformshift{2.119721in}{0.714443in}%
\pgfsys@useobject{currentmarker}{}%
\end{pgfscope}%
\begin{pgfscope}%
\pgfsys@transformshift{2.119981in}{0.701395in}%
\pgfsys@useobject{currentmarker}{}%
\end{pgfscope}%
\begin{pgfscope}%
\pgfsys@transformshift{2.120241in}{0.711982in}%
\pgfsys@useobject{currentmarker}{}%
\end{pgfscope}%
\begin{pgfscope}%
\pgfsys@transformshift{2.120500in}{0.705109in}%
\pgfsys@useobject{currentmarker}{}%
\end{pgfscope}%
\begin{pgfscope}%
\pgfsys@transformshift{2.120759in}{0.696114in}%
\pgfsys@useobject{currentmarker}{}%
\end{pgfscope}%
\begin{pgfscope}%
\pgfsys@transformshift{2.121018in}{0.733837in}%
\pgfsys@useobject{currentmarker}{}%
\end{pgfscope}%
\begin{pgfscope}%
\pgfsys@transformshift{2.121276in}{0.749546in}%
\pgfsys@useobject{currentmarker}{}%
\end{pgfscope}%
\begin{pgfscope}%
\pgfsys@transformshift{2.121535in}{0.749515in}%
\pgfsys@useobject{currentmarker}{}%
\end{pgfscope}%
\begin{pgfscope}%
\pgfsys@transformshift{2.121793in}{0.728487in}%
\pgfsys@useobject{currentmarker}{}%
\end{pgfscope}%
\begin{pgfscope}%
\pgfsys@transformshift{2.122050in}{0.750473in}%
\pgfsys@useobject{currentmarker}{}%
\end{pgfscope}%
\begin{pgfscope}%
\pgfsys@transformshift{2.122308in}{0.787121in}%
\pgfsys@useobject{currentmarker}{}%
\end{pgfscope}%
\begin{pgfscope}%
\pgfsys@transformshift{2.122565in}{0.782872in}%
\pgfsys@useobject{currentmarker}{}%
\end{pgfscope}%
\begin{pgfscope}%
\pgfsys@transformshift{2.122821in}{0.739143in}%
\pgfsys@useobject{currentmarker}{}%
\end{pgfscope}%
\begin{pgfscope}%
\pgfsys@transformshift{2.123078in}{0.733957in}%
\pgfsys@useobject{currentmarker}{}%
\end{pgfscope}%
\begin{pgfscope}%
\pgfsys@transformshift{2.123334in}{0.778723in}%
\pgfsys@useobject{currentmarker}{}%
\end{pgfscope}%
\begin{pgfscope}%
\pgfsys@transformshift{2.123590in}{0.777327in}%
\pgfsys@useobject{currentmarker}{}%
\end{pgfscope}%
\begin{pgfscope}%
\pgfsys@transformshift{2.123845in}{0.767095in}%
\pgfsys@useobject{currentmarker}{}%
\end{pgfscope}%
\begin{pgfscope}%
\pgfsys@transformshift{2.124101in}{0.705598in}%
\pgfsys@useobject{currentmarker}{}%
\end{pgfscope}%
\begin{pgfscope}%
\pgfsys@transformshift{2.124355in}{0.735614in}%
\pgfsys@useobject{currentmarker}{}%
\end{pgfscope}%
\begin{pgfscope}%
\pgfsys@transformshift{2.124610in}{0.753909in}%
\pgfsys@useobject{currentmarker}{}%
\end{pgfscope}%
\begin{pgfscope}%
\pgfsys@transformshift{2.124865in}{0.744751in}%
\pgfsys@useobject{currentmarker}{}%
\end{pgfscope}%
\begin{pgfscope}%
\pgfsys@transformshift{2.125119in}{0.718861in}%
\pgfsys@useobject{currentmarker}{}%
\end{pgfscope}%
\begin{pgfscope}%
\pgfsys@transformshift{2.125373in}{0.727380in}%
\pgfsys@useobject{currentmarker}{}%
\end{pgfscope}%
\begin{pgfscope}%
\pgfsys@transformshift{2.125626in}{0.758190in}%
\pgfsys@useobject{currentmarker}{}%
\end{pgfscope}%
\begin{pgfscope}%
\pgfsys@transformshift{2.125879in}{0.764796in}%
\pgfsys@useobject{currentmarker}{}%
\end{pgfscope}%
\begin{pgfscope}%
\pgfsys@transformshift{2.126132in}{0.754640in}%
\pgfsys@useobject{currentmarker}{}%
\end{pgfscope}%
\begin{pgfscope}%
\pgfsys@transformshift{2.126385in}{0.716307in}%
\pgfsys@useobject{currentmarker}{}%
\end{pgfscope}%
\begin{pgfscope}%
\pgfsys@transformshift{2.126637in}{0.758806in}%
\pgfsys@useobject{currentmarker}{}%
\end{pgfscope}%
\begin{pgfscope}%
\pgfsys@transformshift{2.126889in}{0.731700in}%
\pgfsys@useobject{currentmarker}{}%
\end{pgfscope}%
\begin{pgfscope}%
\pgfsys@transformshift{2.127141in}{0.742758in}%
\pgfsys@useobject{currentmarker}{}%
\end{pgfscope}%
\begin{pgfscope}%
\pgfsys@transformshift{2.127393in}{0.770833in}%
\pgfsys@useobject{currentmarker}{}%
\end{pgfscope}%
\begin{pgfscope}%
\pgfsys@transformshift{2.127644in}{0.746133in}%
\pgfsys@useobject{currentmarker}{}%
\end{pgfscope}%
\begin{pgfscope}%
\pgfsys@transformshift{2.127895in}{0.723388in}%
\pgfsys@useobject{currentmarker}{}%
\end{pgfscope}%
\begin{pgfscope}%
\pgfsys@transformshift{2.128146in}{0.736629in}%
\pgfsys@useobject{currentmarker}{}%
\end{pgfscope}%
\begin{pgfscope}%
\pgfsys@transformshift{2.128396in}{0.739530in}%
\pgfsys@useobject{currentmarker}{}%
\end{pgfscope}%
\begin{pgfscope}%
\pgfsys@transformshift{2.128646in}{0.689276in}%
\pgfsys@useobject{currentmarker}{}%
\end{pgfscope}%
\begin{pgfscope}%
\pgfsys@transformshift{2.128896in}{0.580656in}%
\pgfsys@useobject{currentmarker}{}%
\end{pgfscope}%
\begin{pgfscope}%
\pgfsys@transformshift{2.129146in}{0.684426in}%
\pgfsys@useobject{currentmarker}{}%
\end{pgfscope}%
\begin{pgfscope}%
\pgfsys@transformshift{2.129395in}{0.698049in}%
\pgfsys@useobject{currentmarker}{}%
\end{pgfscope}%
\begin{pgfscope}%
\pgfsys@transformshift{2.129644in}{0.709993in}%
\pgfsys@useobject{currentmarker}{}%
\end{pgfscope}%
\begin{pgfscope}%
\pgfsys@transformshift{2.129893in}{0.731329in}%
\pgfsys@useobject{currentmarker}{}%
\end{pgfscope}%
\begin{pgfscope}%
\pgfsys@transformshift{2.130142in}{0.754799in}%
\pgfsys@useobject{currentmarker}{}%
\end{pgfscope}%
\begin{pgfscope}%
\pgfsys@transformshift{2.130390in}{0.752829in}%
\pgfsys@useobject{currentmarker}{}%
\end{pgfscope}%
\begin{pgfscope}%
\pgfsys@transformshift{2.130638in}{0.737898in}%
\pgfsys@useobject{currentmarker}{}%
\end{pgfscope}%
\begin{pgfscope}%
\pgfsys@transformshift{2.130885in}{0.755376in}%
\pgfsys@useobject{currentmarker}{}%
\end{pgfscope}%
\begin{pgfscope}%
\pgfsys@transformshift{2.131133in}{0.712311in}%
\pgfsys@useobject{currentmarker}{}%
\end{pgfscope}%
\begin{pgfscope}%
\pgfsys@transformshift{2.131380in}{0.690687in}%
\pgfsys@useobject{currentmarker}{}%
\end{pgfscope}%
\begin{pgfscope}%
\pgfsys@transformshift{2.131627in}{0.669638in}%
\pgfsys@useobject{currentmarker}{}%
\end{pgfscope}%
\begin{pgfscope}%
\pgfsys@transformshift{2.131873in}{0.710706in}%
\pgfsys@useobject{currentmarker}{}%
\end{pgfscope}%
\begin{pgfscope}%
\pgfsys@transformshift{2.132120in}{0.720754in}%
\pgfsys@useobject{currentmarker}{}%
\end{pgfscope}%
\begin{pgfscope}%
\pgfsys@transformshift{2.132366in}{0.726575in}%
\pgfsys@useobject{currentmarker}{}%
\end{pgfscope}%
\begin{pgfscope}%
\pgfsys@transformshift{2.132612in}{0.743830in}%
\pgfsys@useobject{currentmarker}{}%
\end{pgfscope}%
\begin{pgfscope}%
\pgfsys@transformshift{2.132857in}{0.716683in}%
\pgfsys@useobject{currentmarker}{}%
\end{pgfscope}%
\begin{pgfscope}%
\pgfsys@transformshift{2.133102in}{0.738072in}%
\pgfsys@useobject{currentmarker}{}%
\end{pgfscope}%
\begin{pgfscope}%
\pgfsys@transformshift{2.133347in}{0.763504in}%
\pgfsys@useobject{currentmarker}{}%
\end{pgfscope}%
\begin{pgfscope}%
\pgfsys@transformshift{2.133592in}{0.739772in}%
\pgfsys@useobject{currentmarker}{}%
\end{pgfscope}%
\begin{pgfscope}%
\pgfsys@transformshift{2.133836in}{0.691709in}%
\pgfsys@useobject{currentmarker}{}%
\end{pgfscope}%
\begin{pgfscope}%
\pgfsys@transformshift{2.134081in}{0.728267in}%
\pgfsys@useobject{currentmarker}{}%
\end{pgfscope}%
\begin{pgfscope}%
\pgfsys@transformshift{2.134324in}{0.697727in}%
\pgfsys@useobject{currentmarker}{}%
\end{pgfscope}%
\begin{pgfscope}%
\pgfsys@transformshift{2.134568in}{0.713117in}%
\pgfsys@useobject{currentmarker}{}%
\end{pgfscope}%
\begin{pgfscope}%
\pgfsys@transformshift{2.134812in}{0.738045in}%
\pgfsys@useobject{currentmarker}{}%
\end{pgfscope}%
\begin{pgfscope}%
\pgfsys@transformshift{2.135055in}{0.787278in}%
\pgfsys@useobject{currentmarker}{}%
\end{pgfscope}%
\begin{pgfscope}%
\pgfsys@transformshift{2.135298in}{0.765866in}%
\pgfsys@useobject{currentmarker}{}%
\end{pgfscope}%
\begin{pgfscope}%
\pgfsys@transformshift{2.135540in}{0.741258in}%
\pgfsys@useobject{currentmarker}{}%
\end{pgfscope}%
\begin{pgfscope}%
\pgfsys@transformshift{2.135782in}{0.771714in}%
\pgfsys@useobject{currentmarker}{}%
\end{pgfscope}%
\begin{pgfscope}%
\pgfsys@transformshift{2.136025in}{0.754316in}%
\pgfsys@useobject{currentmarker}{}%
\end{pgfscope}%
\begin{pgfscope}%
\pgfsys@transformshift{2.136266in}{0.756294in}%
\pgfsys@useobject{currentmarker}{}%
\end{pgfscope}%
\begin{pgfscope}%
\pgfsys@transformshift{2.136508in}{0.746728in}%
\pgfsys@useobject{currentmarker}{}%
\end{pgfscope}%
\begin{pgfscope}%
\pgfsys@transformshift{2.136749in}{0.744986in}%
\pgfsys@useobject{currentmarker}{}%
\end{pgfscope}%
\begin{pgfscope}%
\pgfsys@transformshift{2.136990in}{0.742700in}%
\pgfsys@useobject{currentmarker}{}%
\end{pgfscope}%
\begin{pgfscope}%
\pgfsys@transformshift{2.137231in}{0.783917in}%
\pgfsys@useobject{currentmarker}{}%
\end{pgfscope}%
\begin{pgfscope}%
\pgfsys@transformshift{2.137471in}{0.756183in}%
\pgfsys@useobject{currentmarker}{}%
\end{pgfscope}%
\begin{pgfscope}%
\pgfsys@transformshift{2.137712in}{0.734182in}%
\pgfsys@useobject{currentmarker}{}%
\end{pgfscope}%
\begin{pgfscope}%
\pgfsys@transformshift{2.137952in}{0.741119in}%
\pgfsys@useobject{currentmarker}{}%
\end{pgfscope}%
\begin{pgfscope}%
\pgfsys@transformshift{2.138192in}{0.710595in}%
\pgfsys@useobject{currentmarker}{}%
\end{pgfscope}%
\begin{pgfscope}%
\pgfsys@transformshift{2.138431in}{0.694334in}%
\pgfsys@useobject{currentmarker}{}%
\end{pgfscope}%
\begin{pgfscope}%
\pgfsys@transformshift{2.138670in}{0.678033in}%
\pgfsys@useobject{currentmarker}{}%
\end{pgfscope}%
\begin{pgfscope}%
\pgfsys@transformshift{2.138909in}{0.723408in}%
\pgfsys@useobject{currentmarker}{}%
\end{pgfscope}%
\begin{pgfscope}%
\pgfsys@transformshift{2.139148in}{0.715270in}%
\pgfsys@useobject{currentmarker}{}%
\end{pgfscope}%
\begin{pgfscope}%
\pgfsys@transformshift{2.139386in}{0.728016in}%
\pgfsys@useobject{currentmarker}{}%
\end{pgfscope}%
\begin{pgfscope}%
\pgfsys@transformshift{2.139625in}{0.747428in}%
\pgfsys@useobject{currentmarker}{}%
\end{pgfscope}%
\begin{pgfscope}%
\pgfsys@transformshift{2.139863in}{0.742223in}%
\pgfsys@useobject{currentmarker}{}%
\end{pgfscope}%
\begin{pgfscope}%
\pgfsys@transformshift{2.140100in}{0.746163in}%
\pgfsys@useobject{currentmarker}{}%
\end{pgfscope}%
\begin{pgfscope}%
\pgfsys@transformshift{2.140338in}{0.734004in}%
\pgfsys@useobject{currentmarker}{}%
\end{pgfscope}%
\begin{pgfscope}%
\pgfsys@transformshift{2.140575in}{0.769615in}%
\pgfsys@useobject{currentmarker}{}%
\end{pgfscope}%
\begin{pgfscope}%
\pgfsys@transformshift{2.140812in}{0.747846in}%
\pgfsys@useobject{currentmarker}{}%
\end{pgfscope}%
\begin{pgfscope}%
\pgfsys@transformshift{2.141049in}{0.717670in}%
\pgfsys@useobject{currentmarker}{}%
\end{pgfscope}%
\begin{pgfscope}%
\pgfsys@transformshift{2.141285in}{0.683237in}%
\pgfsys@useobject{currentmarker}{}%
\end{pgfscope}%
\begin{pgfscope}%
\pgfsys@transformshift{2.141521in}{0.678605in}%
\pgfsys@useobject{currentmarker}{}%
\end{pgfscope}%
\begin{pgfscope}%
\pgfsys@transformshift{2.141757in}{0.697433in}%
\pgfsys@useobject{currentmarker}{}%
\end{pgfscope}%
\begin{pgfscope}%
\pgfsys@transformshift{2.141993in}{0.687721in}%
\pgfsys@useobject{currentmarker}{}%
\end{pgfscope}%
\begin{pgfscope}%
\pgfsys@transformshift{2.142229in}{0.756163in}%
\pgfsys@useobject{currentmarker}{}%
\end{pgfscope}%
\begin{pgfscope}%
\pgfsys@transformshift{2.142464in}{0.760415in}%
\pgfsys@useobject{currentmarker}{}%
\end{pgfscope}%
\begin{pgfscope}%
\pgfsys@transformshift{2.142699in}{0.726030in}%
\pgfsys@useobject{currentmarker}{}%
\end{pgfscope}%
\begin{pgfscope}%
\pgfsys@transformshift{2.142934in}{0.724836in}%
\pgfsys@useobject{currentmarker}{}%
\end{pgfscope}%
\begin{pgfscope}%
\pgfsys@transformshift{2.143168in}{0.718877in}%
\pgfsys@useobject{currentmarker}{}%
\end{pgfscope}%
\begin{pgfscope}%
\pgfsys@transformshift{2.143402in}{0.679703in}%
\pgfsys@useobject{currentmarker}{}%
\end{pgfscope}%
\begin{pgfscope}%
\pgfsys@transformshift{2.143636in}{0.689452in}%
\pgfsys@useobject{currentmarker}{}%
\end{pgfscope}%
\begin{pgfscope}%
\pgfsys@transformshift{2.143870in}{0.731027in}%
\pgfsys@useobject{currentmarker}{}%
\end{pgfscope}%
\begin{pgfscope}%
\pgfsys@transformshift{2.144104in}{0.739697in}%
\pgfsys@useobject{currentmarker}{}%
\end{pgfscope}%
\begin{pgfscope}%
\pgfsys@transformshift{2.144337in}{0.752893in}%
\pgfsys@useobject{currentmarker}{}%
\end{pgfscope}%
\begin{pgfscope}%
\pgfsys@transformshift{2.144570in}{0.744104in}%
\pgfsys@useobject{currentmarker}{}%
\end{pgfscope}%
\begin{pgfscope}%
\pgfsys@transformshift{2.144803in}{0.729498in}%
\pgfsys@useobject{currentmarker}{}%
\end{pgfscope}%
\begin{pgfscope}%
\pgfsys@transformshift{2.145035in}{0.784171in}%
\pgfsys@useobject{currentmarker}{}%
\end{pgfscope}%
\begin{pgfscope}%
\pgfsys@transformshift{2.145268in}{0.779705in}%
\pgfsys@useobject{currentmarker}{}%
\end{pgfscope}%
\begin{pgfscope}%
\pgfsys@transformshift{2.145500in}{0.729325in}%
\pgfsys@useobject{currentmarker}{}%
\end{pgfscope}%
\begin{pgfscope}%
\pgfsys@transformshift{2.145731in}{0.726012in}%
\pgfsys@useobject{currentmarker}{}%
\end{pgfscope}%
\begin{pgfscope}%
\pgfsys@transformshift{2.145963in}{0.672974in}%
\pgfsys@useobject{currentmarker}{}%
\end{pgfscope}%
\begin{pgfscope}%
\pgfsys@transformshift{2.146194in}{0.681961in}%
\pgfsys@useobject{currentmarker}{}%
\end{pgfscope}%
\begin{pgfscope}%
\pgfsys@transformshift{2.146426in}{0.721021in}%
\pgfsys@useobject{currentmarker}{}%
\end{pgfscope}%
\begin{pgfscope}%
\pgfsys@transformshift{2.146656in}{0.740577in}%
\pgfsys@useobject{currentmarker}{}%
\end{pgfscope}%
\begin{pgfscope}%
\pgfsys@transformshift{2.146887in}{0.767440in}%
\pgfsys@useobject{currentmarker}{}%
\end{pgfscope}%
\begin{pgfscope}%
\pgfsys@transformshift{2.147117in}{0.745189in}%
\pgfsys@useobject{currentmarker}{}%
\end{pgfscope}%
\begin{pgfscope}%
\pgfsys@transformshift{2.147348in}{0.737968in}%
\pgfsys@useobject{currentmarker}{}%
\end{pgfscope}%
\begin{pgfscope}%
\pgfsys@transformshift{2.147578in}{0.711279in}%
\pgfsys@useobject{currentmarker}{}%
\end{pgfscope}%
\begin{pgfscope}%
\pgfsys@transformshift{2.147807in}{0.660258in}%
\pgfsys@useobject{currentmarker}{}%
\end{pgfscope}%
\begin{pgfscope}%
\pgfsys@transformshift{2.148037in}{0.651826in}%
\pgfsys@useobject{currentmarker}{}%
\end{pgfscope}%
\begin{pgfscope}%
\pgfsys@transformshift{2.148266in}{0.694073in}%
\pgfsys@useobject{currentmarker}{}%
\end{pgfscope}%
\begin{pgfscope}%
\pgfsys@transformshift{2.148495in}{0.701057in}%
\pgfsys@useobject{currentmarker}{}%
\end{pgfscope}%
\begin{pgfscope}%
\pgfsys@transformshift{2.148724in}{0.673513in}%
\pgfsys@useobject{currentmarker}{}%
\end{pgfscope}%
\begin{pgfscope}%
\pgfsys@transformshift{2.148952in}{0.705473in}%
\pgfsys@useobject{currentmarker}{}%
\end{pgfscope}%
\begin{pgfscope}%
\pgfsys@transformshift{2.149181in}{0.721926in}%
\pgfsys@useobject{currentmarker}{}%
\end{pgfscope}%
\begin{pgfscope}%
\pgfsys@transformshift{2.149409in}{0.723126in}%
\pgfsys@useobject{currentmarker}{}%
\end{pgfscope}%
\begin{pgfscope}%
\pgfsys@transformshift{2.149637in}{0.740408in}%
\pgfsys@useobject{currentmarker}{}%
\end{pgfscope}%
\begin{pgfscope}%
\pgfsys@transformshift{2.149864in}{0.690260in}%
\pgfsys@useobject{currentmarker}{}%
\end{pgfscope}%
\begin{pgfscope}%
\pgfsys@transformshift{2.150092in}{0.675034in}%
\pgfsys@useobject{currentmarker}{}%
\end{pgfscope}%
\begin{pgfscope}%
\pgfsys@transformshift{2.150319in}{0.721163in}%
\pgfsys@useobject{currentmarker}{}%
\end{pgfscope}%
\begin{pgfscope}%
\pgfsys@transformshift{2.150546in}{0.739028in}%
\pgfsys@useobject{currentmarker}{}%
\end{pgfscope}%
\begin{pgfscope}%
\pgfsys@transformshift{2.150772in}{0.723065in}%
\pgfsys@useobject{currentmarker}{}%
\end{pgfscope}%
\begin{pgfscope}%
\pgfsys@transformshift{2.150999in}{0.718657in}%
\pgfsys@useobject{currentmarker}{}%
\end{pgfscope}%
\begin{pgfscope}%
\pgfsys@transformshift{2.151225in}{0.735854in}%
\pgfsys@useobject{currentmarker}{}%
\end{pgfscope}%
\begin{pgfscope}%
\pgfsys@transformshift{2.151451in}{0.762691in}%
\pgfsys@useobject{currentmarker}{}%
\end{pgfscope}%
\begin{pgfscope}%
\pgfsys@transformshift{2.151677in}{0.775493in}%
\pgfsys@useobject{currentmarker}{}%
\end{pgfscope}%
\begin{pgfscope}%
\pgfsys@transformshift{2.151902in}{0.742574in}%
\pgfsys@useobject{currentmarker}{}%
\end{pgfscope}%
\begin{pgfscope}%
\pgfsys@transformshift{2.152128in}{0.748545in}%
\pgfsys@useobject{currentmarker}{}%
\end{pgfscope}%
\begin{pgfscope}%
\pgfsys@transformshift{2.152353in}{0.759999in}%
\pgfsys@useobject{currentmarker}{}%
\end{pgfscope}%
\begin{pgfscope}%
\pgfsys@transformshift{2.152578in}{0.789942in}%
\pgfsys@useobject{currentmarker}{}%
\end{pgfscope}%
\begin{pgfscope}%
\pgfsys@transformshift{2.152802in}{0.779526in}%
\pgfsys@useobject{currentmarker}{}%
\end{pgfscope}%
\begin{pgfscope}%
\pgfsys@transformshift{2.153027in}{0.739926in}%
\pgfsys@useobject{currentmarker}{}%
\end{pgfscope}%
\begin{pgfscope}%
\pgfsys@transformshift{2.153251in}{0.730775in}%
\pgfsys@useobject{currentmarker}{}%
\end{pgfscope}%
\begin{pgfscope}%
\pgfsys@transformshift{2.153475in}{0.738012in}%
\pgfsys@useobject{currentmarker}{}%
\end{pgfscope}%
\begin{pgfscope}%
\pgfsys@transformshift{2.153699in}{0.743065in}%
\pgfsys@useobject{currentmarker}{}%
\end{pgfscope}%
\begin{pgfscope}%
\pgfsys@transformshift{2.153922in}{0.749129in}%
\pgfsys@useobject{currentmarker}{}%
\end{pgfscope}%
\begin{pgfscope}%
\pgfsys@transformshift{2.154145in}{0.749824in}%
\pgfsys@useobject{currentmarker}{}%
\end{pgfscope}%
\begin{pgfscope}%
\pgfsys@transformshift{2.154369in}{0.728947in}%
\pgfsys@useobject{currentmarker}{}%
\end{pgfscope}%
\begin{pgfscope}%
\pgfsys@transformshift{2.154591in}{0.719932in}%
\pgfsys@useobject{currentmarker}{}%
\end{pgfscope}%
\begin{pgfscope}%
\pgfsys@transformshift{2.154814in}{0.702610in}%
\pgfsys@useobject{currentmarker}{}%
\end{pgfscope}%
\begin{pgfscope}%
\pgfsys@transformshift{2.155037in}{0.661762in}%
\pgfsys@useobject{currentmarker}{}%
\end{pgfscope}%
\begin{pgfscope}%
\pgfsys@transformshift{2.155259in}{0.669199in}%
\pgfsys@useobject{currentmarker}{}%
\end{pgfscope}%
\begin{pgfscope}%
\pgfsys@transformshift{2.155481in}{0.684992in}%
\pgfsys@useobject{currentmarker}{}%
\end{pgfscope}%
\begin{pgfscope}%
\pgfsys@transformshift{2.155703in}{0.677193in}%
\pgfsys@useobject{currentmarker}{}%
\end{pgfscope}%
\begin{pgfscope}%
\pgfsys@transformshift{2.155924in}{0.652032in}%
\pgfsys@useobject{currentmarker}{}%
\end{pgfscope}%
\begin{pgfscope}%
\pgfsys@transformshift{2.156145in}{0.687188in}%
\pgfsys@useobject{currentmarker}{}%
\end{pgfscope}%
\begin{pgfscope}%
\pgfsys@transformshift{2.156367in}{0.704337in}%
\pgfsys@useobject{currentmarker}{}%
\end{pgfscope}%
\begin{pgfscope}%
\pgfsys@transformshift{2.156587in}{0.732875in}%
\pgfsys@useobject{currentmarker}{}%
\end{pgfscope}%
\begin{pgfscope}%
\pgfsys@transformshift{2.156808in}{0.743677in}%
\pgfsys@useobject{currentmarker}{}%
\end{pgfscope}%
\begin{pgfscope}%
\pgfsys@transformshift{2.157029in}{0.760899in}%
\pgfsys@useobject{currentmarker}{}%
\end{pgfscope}%
\begin{pgfscope}%
\pgfsys@transformshift{2.157249in}{0.733034in}%
\pgfsys@useobject{currentmarker}{}%
\end{pgfscope}%
\begin{pgfscope}%
\pgfsys@transformshift{2.157469in}{0.711355in}%
\pgfsys@useobject{currentmarker}{}%
\end{pgfscope}%
\begin{pgfscope}%
\pgfsys@transformshift{2.157689in}{0.749981in}%
\pgfsys@useobject{currentmarker}{}%
\end{pgfscope}%
\begin{pgfscope}%
\pgfsys@transformshift{2.157908in}{0.741231in}%
\pgfsys@useobject{currentmarker}{}%
\end{pgfscope}%
\begin{pgfscope}%
\pgfsys@transformshift{2.158128in}{0.683718in}%
\pgfsys@useobject{currentmarker}{}%
\end{pgfscope}%
\begin{pgfscope}%
\pgfsys@transformshift{2.158347in}{0.725973in}%
\pgfsys@useobject{currentmarker}{}%
\end{pgfscope}%
\begin{pgfscope}%
\pgfsys@transformshift{2.158566in}{0.750768in}%
\pgfsys@useobject{currentmarker}{}%
\end{pgfscope}%
\begin{pgfscope}%
\pgfsys@transformshift{2.158785in}{0.762906in}%
\pgfsys@useobject{currentmarker}{}%
\end{pgfscope}%
\begin{pgfscope}%
\pgfsys@transformshift{2.159003in}{0.748588in}%
\pgfsys@useobject{currentmarker}{}%
\end{pgfscope}%
\begin{pgfscope}%
\pgfsys@transformshift{2.159221in}{0.713253in}%
\pgfsys@useobject{currentmarker}{}%
\end{pgfscope}%
\begin{pgfscope}%
\pgfsys@transformshift{2.159440in}{0.712371in}%
\pgfsys@useobject{currentmarker}{}%
\end{pgfscope}%
\begin{pgfscope}%
\pgfsys@transformshift{2.159657in}{0.701712in}%
\pgfsys@useobject{currentmarker}{}%
\end{pgfscope}%
\begin{pgfscope}%
\pgfsys@transformshift{2.159875in}{0.718185in}%
\pgfsys@useobject{currentmarker}{}%
\end{pgfscope}%
\begin{pgfscope}%
\pgfsys@transformshift{2.160093in}{0.752171in}%
\pgfsys@useobject{currentmarker}{}%
\end{pgfscope}%
\begin{pgfscope}%
\pgfsys@transformshift{2.160310in}{0.731908in}%
\pgfsys@useobject{currentmarker}{}%
\end{pgfscope}%
\begin{pgfscope}%
\pgfsys@transformshift{2.160527in}{0.733171in}%
\pgfsys@useobject{currentmarker}{}%
\end{pgfscope}%
\begin{pgfscope}%
\pgfsys@transformshift{2.160744in}{0.744483in}%
\pgfsys@useobject{currentmarker}{}%
\end{pgfscope}%
\begin{pgfscope}%
\pgfsys@transformshift{2.160960in}{0.745905in}%
\pgfsys@useobject{currentmarker}{}%
\end{pgfscope}%
\begin{pgfscope}%
\pgfsys@transformshift{2.161177in}{0.729505in}%
\pgfsys@useobject{currentmarker}{}%
\end{pgfscope}%
\begin{pgfscope}%
\pgfsys@transformshift{2.161393in}{0.662076in}%
\pgfsys@useobject{currentmarker}{}%
\end{pgfscope}%
\begin{pgfscope}%
\pgfsys@transformshift{2.161609in}{0.707627in}%
\pgfsys@useobject{currentmarker}{}%
\end{pgfscope}%
\begin{pgfscope}%
\pgfsys@transformshift{2.161825in}{0.669764in}%
\pgfsys@useobject{currentmarker}{}%
\end{pgfscope}%
\begin{pgfscope}%
\pgfsys@transformshift{2.162040in}{0.698200in}%
\pgfsys@useobject{currentmarker}{}%
\end{pgfscope}%
\begin{pgfscope}%
\pgfsys@transformshift{2.162256in}{0.777038in}%
\pgfsys@useobject{currentmarker}{}%
\end{pgfscope}%
\begin{pgfscope}%
\pgfsys@transformshift{2.162471in}{0.765614in}%
\pgfsys@useobject{currentmarker}{}%
\end{pgfscope}%
\begin{pgfscope}%
\pgfsys@transformshift{2.162686in}{0.716676in}%
\pgfsys@useobject{currentmarker}{}%
\end{pgfscope}%
\begin{pgfscope}%
\pgfsys@transformshift{2.162901in}{0.745723in}%
\pgfsys@useobject{currentmarker}{}%
\end{pgfscope}%
\begin{pgfscope}%
\pgfsys@transformshift{2.163115in}{0.714728in}%
\pgfsys@useobject{currentmarker}{}%
\end{pgfscope}%
\begin{pgfscope}%
\pgfsys@transformshift{2.163330in}{0.676405in}%
\pgfsys@useobject{currentmarker}{}%
\end{pgfscope}%
\begin{pgfscope}%
\pgfsys@transformshift{2.163544in}{0.735114in}%
\pgfsys@useobject{currentmarker}{}%
\end{pgfscope}%
\begin{pgfscope}%
\pgfsys@transformshift{2.163758in}{0.762454in}%
\pgfsys@useobject{currentmarker}{}%
\end{pgfscope}%
\begin{pgfscope}%
\pgfsys@transformshift{2.163972in}{0.719797in}%
\pgfsys@useobject{currentmarker}{}%
\end{pgfscope}%
\begin{pgfscope}%
\pgfsys@transformshift{2.164185in}{0.679386in}%
\pgfsys@useobject{currentmarker}{}%
\end{pgfscope}%
\begin{pgfscope}%
\pgfsys@transformshift{2.164399in}{0.714881in}%
\pgfsys@useobject{currentmarker}{}%
\end{pgfscope}%
\begin{pgfscope}%
\pgfsys@transformshift{2.164612in}{0.751762in}%
\pgfsys@useobject{currentmarker}{}%
\end{pgfscope}%
\begin{pgfscope}%
\pgfsys@transformshift{2.164825in}{0.752088in}%
\pgfsys@useobject{currentmarker}{}%
\end{pgfscope}%
\begin{pgfscope}%
\pgfsys@transformshift{2.165037in}{0.718166in}%
\pgfsys@useobject{currentmarker}{}%
\end{pgfscope}%
\begin{pgfscope}%
\pgfsys@transformshift{2.165250in}{0.658277in}%
\pgfsys@useobject{currentmarker}{}%
\end{pgfscope}%
\begin{pgfscope}%
\pgfsys@transformshift{2.165462in}{0.630853in}%
\pgfsys@useobject{currentmarker}{}%
\end{pgfscope}%
\begin{pgfscope}%
\pgfsys@transformshift{2.165674in}{0.641844in}%
\pgfsys@useobject{currentmarker}{}%
\end{pgfscope}%
\begin{pgfscope}%
\pgfsys@transformshift{2.165886in}{0.674428in}%
\pgfsys@useobject{currentmarker}{}%
\end{pgfscope}%
\begin{pgfscope}%
\pgfsys@transformshift{2.166098in}{0.704081in}%
\pgfsys@useobject{currentmarker}{}%
\end{pgfscope}%
\begin{pgfscope}%
\pgfsys@transformshift{2.166310in}{0.705802in}%
\pgfsys@useobject{currentmarker}{}%
\end{pgfscope}%
\begin{pgfscope}%
\pgfsys@transformshift{2.166521in}{0.680989in}%
\pgfsys@useobject{currentmarker}{}%
\end{pgfscope}%
\begin{pgfscope}%
\pgfsys@transformshift{2.166732in}{0.724757in}%
\pgfsys@useobject{currentmarker}{}%
\end{pgfscope}%
\begin{pgfscope}%
\pgfsys@transformshift{2.166943in}{0.741851in}%
\pgfsys@useobject{currentmarker}{}%
\end{pgfscope}%
\begin{pgfscope}%
\pgfsys@transformshift{2.167154in}{0.728933in}%
\pgfsys@useobject{currentmarker}{}%
\end{pgfscope}%
\begin{pgfscope}%
\pgfsys@transformshift{2.167364in}{0.734742in}%
\pgfsys@useobject{currentmarker}{}%
\end{pgfscope}%
\begin{pgfscope}%
\pgfsys@transformshift{2.167575in}{0.733895in}%
\pgfsys@useobject{currentmarker}{}%
\end{pgfscope}%
\begin{pgfscope}%
\pgfsys@transformshift{2.167785in}{0.712681in}%
\pgfsys@useobject{currentmarker}{}%
\end{pgfscope}%
\begin{pgfscope}%
\pgfsys@transformshift{2.167995in}{0.721090in}%
\pgfsys@useobject{currentmarker}{}%
\end{pgfscope}%
\begin{pgfscope}%
\pgfsys@transformshift{2.168205in}{0.728936in}%
\pgfsys@useobject{currentmarker}{}%
\end{pgfscope}%
\begin{pgfscope}%
\pgfsys@transformshift{2.168414in}{0.677173in}%
\pgfsys@useobject{currentmarker}{}%
\end{pgfscope}%
\begin{pgfscope}%
\pgfsys@transformshift{2.168624in}{0.711506in}%
\pgfsys@useobject{currentmarker}{}%
\end{pgfscope}%
\begin{pgfscope}%
\pgfsys@transformshift{2.168833in}{0.728816in}%
\pgfsys@useobject{currentmarker}{}%
\end{pgfscope}%
\begin{pgfscope}%
\pgfsys@transformshift{2.169042in}{0.781488in}%
\pgfsys@useobject{currentmarker}{}%
\end{pgfscope}%
\begin{pgfscope}%
\pgfsys@transformshift{2.169251in}{0.771199in}%
\pgfsys@useobject{currentmarker}{}%
\end{pgfscope}%
\begin{pgfscope}%
\pgfsys@transformshift{2.169459in}{0.728248in}%
\pgfsys@useobject{currentmarker}{}%
\end{pgfscope}%
\begin{pgfscope}%
\pgfsys@transformshift{2.169668in}{0.719327in}%
\pgfsys@useobject{currentmarker}{}%
\end{pgfscope}%
\begin{pgfscope}%
\pgfsys@transformshift{2.169876in}{0.676532in}%
\pgfsys@useobject{currentmarker}{}%
\end{pgfscope}%
\begin{pgfscope}%
\pgfsys@transformshift{2.170084in}{0.707427in}%
\pgfsys@useobject{currentmarker}{}%
\end{pgfscope}%
\begin{pgfscope}%
\pgfsys@transformshift{2.170292in}{0.704931in}%
\pgfsys@useobject{currentmarker}{}%
\end{pgfscope}%
\begin{pgfscope}%
\pgfsys@transformshift{2.170499in}{0.690180in}%
\pgfsys@useobject{currentmarker}{}%
\end{pgfscope}%
\begin{pgfscope}%
\pgfsys@transformshift{2.170707in}{0.697309in}%
\pgfsys@useobject{currentmarker}{}%
\end{pgfscope}%
\begin{pgfscope}%
\pgfsys@transformshift{2.170914in}{0.745171in}%
\pgfsys@useobject{currentmarker}{}%
\end{pgfscope}%
\begin{pgfscope}%
\pgfsys@transformshift{2.171121in}{0.735156in}%
\pgfsys@useobject{currentmarker}{}%
\end{pgfscope}%
\begin{pgfscope}%
\pgfsys@transformshift{2.171328in}{0.717459in}%
\pgfsys@useobject{currentmarker}{}%
\end{pgfscope}%
\begin{pgfscope}%
\pgfsys@transformshift{2.171535in}{0.763519in}%
\pgfsys@useobject{currentmarker}{}%
\end{pgfscope}%
\begin{pgfscope}%
\pgfsys@transformshift{2.171741in}{0.765076in}%
\pgfsys@useobject{currentmarker}{}%
\end{pgfscope}%
\begin{pgfscope}%
\pgfsys@transformshift{2.171947in}{0.678295in}%
\pgfsys@useobject{currentmarker}{}%
\end{pgfscope}%
\begin{pgfscope}%
\pgfsys@transformshift{2.172153in}{0.657281in}%
\pgfsys@useobject{currentmarker}{}%
\end{pgfscope}%
\begin{pgfscope}%
\pgfsys@transformshift{2.172359in}{0.729391in}%
\pgfsys@useobject{currentmarker}{}%
\end{pgfscope}%
\begin{pgfscope}%
\pgfsys@transformshift{2.172565in}{0.745475in}%
\pgfsys@useobject{currentmarker}{}%
\end{pgfscope}%
\begin{pgfscope}%
\pgfsys@transformshift{2.172771in}{0.709218in}%
\pgfsys@useobject{currentmarker}{}%
\end{pgfscope}%
\begin{pgfscope}%
\pgfsys@transformshift{2.172976in}{0.685701in}%
\pgfsys@useobject{currentmarker}{}%
\end{pgfscope}%
\begin{pgfscope}%
\pgfsys@transformshift{2.173181in}{0.731980in}%
\pgfsys@useobject{currentmarker}{}%
\end{pgfscope}%
\begin{pgfscope}%
\pgfsys@transformshift{2.173386in}{0.717069in}%
\pgfsys@useobject{currentmarker}{}%
\end{pgfscope}%
\begin{pgfscope}%
\pgfsys@transformshift{2.173591in}{0.680726in}%
\pgfsys@useobject{currentmarker}{}%
\end{pgfscope}%
\begin{pgfscope}%
\pgfsys@transformshift{2.173796in}{0.686509in}%
\pgfsys@useobject{currentmarker}{}%
\end{pgfscope}%
\begin{pgfscope}%
\pgfsys@transformshift{2.174000in}{0.742819in}%
\pgfsys@useobject{currentmarker}{}%
\end{pgfscope}%
\begin{pgfscope}%
\pgfsys@transformshift{2.174204in}{0.773910in}%
\pgfsys@useobject{currentmarker}{}%
\end{pgfscope}%
\begin{pgfscope}%
\pgfsys@transformshift{2.174408in}{0.715499in}%
\pgfsys@useobject{currentmarker}{}%
\end{pgfscope}%
\begin{pgfscope}%
\pgfsys@transformshift{2.174612in}{0.742503in}%
\pgfsys@useobject{currentmarker}{}%
\end{pgfscope}%
\begin{pgfscope}%
\pgfsys@transformshift{2.174816in}{0.751963in}%
\pgfsys@useobject{currentmarker}{}%
\end{pgfscope}%
\begin{pgfscope}%
\pgfsys@transformshift{2.175019in}{0.720115in}%
\pgfsys@useobject{currentmarker}{}%
\end{pgfscope}%
\begin{pgfscope}%
\pgfsys@transformshift{2.175223in}{0.659906in}%
\pgfsys@useobject{currentmarker}{}%
\end{pgfscope}%
\begin{pgfscope}%
\pgfsys@transformshift{2.175426in}{0.667215in}%
\pgfsys@useobject{currentmarker}{}%
\end{pgfscope}%
\begin{pgfscope}%
\pgfsys@transformshift{2.175629in}{0.700960in}%
\pgfsys@useobject{currentmarker}{}%
\end{pgfscope}%
\begin{pgfscope}%
\pgfsys@transformshift{2.175831in}{0.710872in}%
\pgfsys@useobject{currentmarker}{}%
\end{pgfscope}%
\begin{pgfscope}%
\pgfsys@transformshift{2.176034in}{0.708315in}%
\pgfsys@useobject{currentmarker}{}%
\end{pgfscope}%
\begin{pgfscope}%
\pgfsys@transformshift{2.176236in}{0.725400in}%
\pgfsys@useobject{currentmarker}{}%
\end{pgfscope}%
\begin{pgfscope}%
\pgfsys@transformshift{2.176439in}{0.743954in}%
\pgfsys@useobject{currentmarker}{}%
\end{pgfscope}%
\begin{pgfscope}%
\pgfsys@transformshift{2.176641in}{0.753375in}%
\pgfsys@useobject{currentmarker}{}%
\end{pgfscope}%
\begin{pgfscope}%
\pgfsys@transformshift{2.176843in}{0.749506in}%
\pgfsys@useobject{currentmarker}{}%
\end{pgfscope}%
\begin{pgfscope}%
\pgfsys@transformshift{2.177044in}{0.727254in}%
\pgfsys@useobject{currentmarker}{}%
\end{pgfscope}%
\begin{pgfscope}%
\pgfsys@transformshift{2.177246in}{0.738427in}%
\pgfsys@useobject{currentmarker}{}%
\end{pgfscope}%
\begin{pgfscope}%
\pgfsys@transformshift{2.177447in}{0.704556in}%
\pgfsys@useobject{currentmarker}{}%
\end{pgfscope}%
\begin{pgfscope}%
\pgfsys@transformshift{2.177648in}{0.683166in}%
\pgfsys@useobject{currentmarker}{}%
\end{pgfscope}%
\begin{pgfscope}%
\pgfsys@transformshift{2.177849in}{0.706498in}%
\pgfsys@useobject{currentmarker}{}%
\end{pgfscope}%
\begin{pgfscope}%
\pgfsys@transformshift{2.178050in}{0.716019in}%
\pgfsys@useobject{currentmarker}{}%
\end{pgfscope}%
\begin{pgfscope}%
\pgfsys@transformshift{2.178250in}{0.688282in}%
\pgfsys@useobject{currentmarker}{}%
\end{pgfscope}%
\begin{pgfscope}%
\pgfsys@transformshift{2.178451in}{0.691456in}%
\pgfsys@useobject{currentmarker}{}%
\end{pgfscope}%
\begin{pgfscope}%
\pgfsys@transformshift{2.178651in}{0.772068in}%
\pgfsys@useobject{currentmarker}{}%
\end{pgfscope}%
\begin{pgfscope}%
\pgfsys@transformshift{2.178851in}{0.743363in}%
\pgfsys@useobject{currentmarker}{}%
\end{pgfscope}%
\begin{pgfscope}%
\pgfsys@transformshift{2.179051in}{0.732207in}%
\pgfsys@useobject{currentmarker}{}%
\end{pgfscope}%
\begin{pgfscope}%
\pgfsys@transformshift{2.179251in}{0.695948in}%
\pgfsys@useobject{currentmarker}{}%
\end{pgfscope}%
\begin{pgfscope}%
\pgfsys@transformshift{2.179450in}{0.694856in}%
\pgfsys@useobject{currentmarker}{}%
\end{pgfscope}%
\begin{pgfscope}%
\pgfsys@transformshift{2.179650in}{0.729139in}%
\pgfsys@useobject{currentmarker}{}%
\end{pgfscope}%
\begin{pgfscope}%
\pgfsys@transformshift{2.179849in}{0.695873in}%
\pgfsys@useobject{currentmarker}{}%
\end{pgfscope}%
\begin{pgfscope}%
\pgfsys@transformshift{2.180048in}{0.674827in}%
\pgfsys@useobject{currentmarker}{}%
\end{pgfscope}%
\begin{pgfscope}%
\pgfsys@transformshift{2.180247in}{0.696679in}%
\pgfsys@useobject{currentmarker}{}%
\end{pgfscope}%
\begin{pgfscope}%
\pgfsys@transformshift{2.180445in}{0.749520in}%
\pgfsys@useobject{currentmarker}{}%
\end{pgfscope}%
\begin{pgfscope}%
\pgfsys@transformshift{2.180644in}{0.723974in}%
\pgfsys@useobject{currentmarker}{}%
\end{pgfscope}%
\begin{pgfscope}%
\pgfsys@transformshift{2.180842in}{0.726545in}%
\pgfsys@useobject{currentmarker}{}%
\end{pgfscope}%
\begin{pgfscope}%
\pgfsys@transformshift{2.181040in}{0.739102in}%
\pgfsys@useobject{currentmarker}{}%
\end{pgfscope}%
\begin{pgfscope}%
\pgfsys@transformshift{2.181238in}{0.706739in}%
\pgfsys@useobject{currentmarker}{}%
\end{pgfscope}%
\begin{pgfscope}%
\pgfsys@transformshift{2.181436in}{0.728875in}%
\pgfsys@useobject{currentmarker}{}%
\end{pgfscope}%
\begin{pgfscope}%
\pgfsys@transformshift{2.181633in}{0.745068in}%
\pgfsys@useobject{currentmarker}{}%
\end{pgfscope}%
\begin{pgfscope}%
\pgfsys@transformshift{2.181831in}{0.742319in}%
\pgfsys@useobject{currentmarker}{}%
\end{pgfscope}%
\begin{pgfscope}%
\pgfsys@transformshift{2.182028in}{0.721603in}%
\pgfsys@useobject{currentmarker}{}%
\end{pgfscope}%
\begin{pgfscope}%
\pgfsys@transformshift{2.182225in}{0.703696in}%
\pgfsys@useobject{currentmarker}{}%
\end{pgfscope}%
\begin{pgfscope}%
\pgfsys@transformshift{2.182422in}{0.706321in}%
\pgfsys@useobject{currentmarker}{}%
\end{pgfscope}%
\begin{pgfscope}%
\pgfsys@transformshift{2.182619in}{0.739068in}%
\pgfsys@useobject{currentmarker}{}%
\end{pgfscope}%
\begin{pgfscope}%
\pgfsys@transformshift{2.182815in}{0.683436in}%
\pgfsys@useobject{currentmarker}{}%
\end{pgfscope}%
\begin{pgfscope}%
\pgfsys@transformshift{2.183012in}{0.686393in}%
\pgfsys@useobject{currentmarker}{}%
\end{pgfscope}%
\begin{pgfscope}%
\pgfsys@transformshift{2.183208in}{0.754084in}%
\pgfsys@useobject{currentmarker}{}%
\end{pgfscope}%
\begin{pgfscope}%
\pgfsys@transformshift{2.183404in}{0.766568in}%
\pgfsys@useobject{currentmarker}{}%
\end{pgfscope}%
\begin{pgfscope}%
\pgfsys@transformshift{2.183600in}{0.729471in}%
\pgfsys@useobject{currentmarker}{}%
\end{pgfscope}%
\begin{pgfscope}%
\pgfsys@transformshift{2.183796in}{0.723207in}%
\pgfsys@useobject{currentmarker}{}%
\end{pgfscope}%
\begin{pgfscope}%
\pgfsys@transformshift{2.183991in}{0.727502in}%
\pgfsys@useobject{currentmarker}{}%
\end{pgfscope}%
\begin{pgfscope}%
\pgfsys@transformshift{2.184187in}{0.701599in}%
\pgfsys@useobject{currentmarker}{}%
\end{pgfscope}%
\begin{pgfscope}%
\pgfsys@transformshift{2.184382in}{0.680490in}%
\pgfsys@useobject{currentmarker}{}%
\end{pgfscope}%
\begin{pgfscope}%
\pgfsys@transformshift{2.184577in}{0.673418in}%
\pgfsys@useobject{currentmarker}{}%
\end{pgfscope}%
\begin{pgfscope}%
\pgfsys@transformshift{2.184772in}{0.692675in}%
\pgfsys@useobject{currentmarker}{}%
\end{pgfscope}%
\begin{pgfscope}%
\pgfsys@transformshift{2.184966in}{0.726810in}%
\pgfsys@useobject{currentmarker}{}%
\end{pgfscope}%
\begin{pgfscope}%
\pgfsys@transformshift{2.185161in}{0.727832in}%
\pgfsys@useobject{currentmarker}{}%
\end{pgfscope}%
\begin{pgfscope}%
\pgfsys@transformshift{2.185355in}{0.739960in}%
\pgfsys@useobject{currentmarker}{}%
\end{pgfscope}%
\begin{pgfscope}%
\pgfsys@transformshift{2.185549in}{0.742367in}%
\pgfsys@useobject{currentmarker}{}%
\end{pgfscope}%
\begin{pgfscope}%
\pgfsys@transformshift{2.185743in}{0.751899in}%
\pgfsys@useobject{currentmarker}{}%
\end{pgfscope}%
\begin{pgfscope}%
\pgfsys@transformshift{2.185937in}{0.725703in}%
\pgfsys@useobject{currentmarker}{}%
\end{pgfscope}%
\begin{pgfscope}%
\pgfsys@transformshift{2.186131in}{0.697385in}%
\pgfsys@useobject{currentmarker}{}%
\end{pgfscope}%
\begin{pgfscope}%
\pgfsys@transformshift{2.186324in}{0.693726in}%
\pgfsys@useobject{currentmarker}{}%
\end{pgfscope}%
\begin{pgfscope}%
\pgfsys@transformshift{2.186518in}{0.662089in}%
\pgfsys@useobject{currentmarker}{}%
\end{pgfscope}%
\begin{pgfscope}%
\pgfsys@transformshift{2.186711in}{0.664394in}%
\pgfsys@useobject{currentmarker}{}%
\end{pgfscope}%
\begin{pgfscope}%
\pgfsys@transformshift{2.186904in}{0.699982in}%
\pgfsys@useobject{currentmarker}{}%
\end{pgfscope}%
\begin{pgfscope}%
\pgfsys@transformshift{2.187097in}{0.729376in}%
\pgfsys@useobject{currentmarker}{}%
\end{pgfscope}%
\begin{pgfscope}%
\pgfsys@transformshift{2.187289in}{0.731325in}%
\pgfsys@useobject{currentmarker}{}%
\end{pgfscope}%
\begin{pgfscope}%
\pgfsys@transformshift{2.187482in}{0.744696in}%
\pgfsys@useobject{currentmarker}{}%
\end{pgfscope}%
\begin{pgfscope}%
\pgfsys@transformshift{2.187674in}{0.768065in}%
\pgfsys@useobject{currentmarker}{}%
\end{pgfscope}%
\begin{pgfscope}%
\pgfsys@transformshift{2.187866in}{0.766467in}%
\pgfsys@useobject{currentmarker}{}%
\end{pgfscope}%
\begin{pgfscope}%
\pgfsys@transformshift{2.188059in}{0.784667in}%
\pgfsys@useobject{currentmarker}{}%
\end{pgfscope}%
\begin{pgfscope}%
\pgfsys@transformshift{2.188250in}{0.764235in}%
\pgfsys@useobject{currentmarker}{}%
\end{pgfscope}%
\begin{pgfscope}%
\pgfsys@transformshift{2.188442in}{0.720639in}%
\pgfsys@useobject{currentmarker}{}%
\end{pgfscope}%
\begin{pgfscope}%
\pgfsys@transformshift{2.188634in}{0.748775in}%
\pgfsys@useobject{currentmarker}{}%
\end{pgfscope}%
\begin{pgfscope}%
\pgfsys@transformshift{2.188825in}{0.741102in}%
\pgfsys@useobject{currentmarker}{}%
\end{pgfscope}%
\begin{pgfscope}%
\pgfsys@transformshift{2.189016in}{0.712175in}%
\pgfsys@useobject{currentmarker}{}%
\end{pgfscope}%
\begin{pgfscope}%
\pgfsys@transformshift{2.189207in}{0.678856in}%
\pgfsys@useobject{currentmarker}{}%
\end{pgfscope}%
\begin{pgfscope}%
\pgfsys@transformshift{2.189398in}{0.693428in}%
\pgfsys@useobject{currentmarker}{}%
\end{pgfscope}%
\begin{pgfscope}%
\pgfsys@transformshift{2.189589in}{0.735576in}%
\pgfsys@useobject{currentmarker}{}%
\end{pgfscope}%
\begin{pgfscope}%
\pgfsys@transformshift{2.189779in}{0.718155in}%
\pgfsys@useobject{currentmarker}{}%
\end{pgfscope}%
\begin{pgfscope}%
\pgfsys@transformshift{2.189970in}{0.702542in}%
\pgfsys@useobject{currentmarker}{}%
\end{pgfscope}%
\begin{pgfscope}%
\pgfsys@transformshift{2.190160in}{0.701739in}%
\pgfsys@useobject{currentmarker}{}%
\end{pgfscope}%
\begin{pgfscope}%
\pgfsys@transformshift{2.190350in}{0.723365in}%
\pgfsys@useobject{currentmarker}{}%
\end{pgfscope}%
\begin{pgfscope}%
\pgfsys@transformshift{2.190540in}{0.729904in}%
\pgfsys@useobject{currentmarker}{}%
\end{pgfscope}%
\begin{pgfscope}%
\pgfsys@transformshift{2.190730in}{0.740476in}%
\pgfsys@useobject{currentmarker}{}%
\end{pgfscope}%
\begin{pgfscope}%
\pgfsys@transformshift{2.190919in}{0.717782in}%
\pgfsys@useobject{currentmarker}{}%
\end{pgfscope}%
\begin{pgfscope}%
\pgfsys@transformshift{2.191109in}{0.717645in}%
\pgfsys@useobject{currentmarker}{}%
\end{pgfscope}%
\begin{pgfscope}%
\pgfsys@transformshift{2.191298in}{0.676298in}%
\pgfsys@useobject{currentmarker}{}%
\end{pgfscope}%
\begin{pgfscope}%
\pgfsys@transformshift{2.191487in}{0.605517in}%
\pgfsys@useobject{currentmarker}{}%
\end{pgfscope}%
\begin{pgfscope}%
\pgfsys@transformshift{2.191676in}{0.676001in}%
\pgfsys@useobject{currentmarker}{}%
\end{pgfscope}%
\begin{pgfscope}%
\pgfsys@transformshift{2.191865in}{0.721601in}%
\pgfsys@useobject{currentmarker}{}%
\end{pgfscope}%
\begin{pgfscope}%
\pgfsys@transformshift{2.192054in}{0.716428in}%
\pgfsys@useobject{currentmarker}{}%
\end{pgfscope}%
\begin{pgfscope}%
\pgfsys@transformshift{2.192242in}{0.696846in}%
\pgfsys@useobject{currentmarker}{}%
\end{pgfscope}%
\begin{pgfscope}%
\pgfsys@transformshift{2.192430in}{0.669526in}%
\pgfsys@useobject{currentmarker}{}%
\end{pgfscope}%
\begin{pgfscope}%
\pgfsys@transformshift{2.192619in}{0.691971in}%
\pgfsys@useobject{currentmarker}{}%
\end{pgfscope}%
\begin{pgfscope}%
\pgfsys@transformshift{2.192807in}{0.729259in}%
\pgfsys@useobject{currentmarker}{}%
\end{pgfscope}%
\begin{pgfscope}%
\pgfsys@transformshift{2.192994in}{0.694577in}%
\pgfsys@useobject{currentmarker}{}%
\end{pgfscope}%
\begin{pgfscope}%
\pgfsys@transformshift{2.193182in}{0.738337in}%
\pgfsys@useobject{currentmarker}{}%
\end{pgfscope}%
\begin{pgfscope}%
\pgfsys@transformshift{2.193370in}{0.767891in}%
\pgfsys@useobject{currentmarker}{}%
\end{pgfscope}%
\begin{pgfscope}%
\pgfsys@transformshift{2.193557in}{0.726793in}%
\pgfsys@useobject{currentmarker}{}%
\end{pgfscope}%
\begin{pgfscope}%
\pgfsys@transformshift{2.193744in}{0.715363in}%
\pgfsys@useobject{currentmarker}{}%
\end{pgfscope}%
\begin{pgfscope}%
\pgfsys@transformshift{2.193931in}{0.736440in}%
\pgfsys@useobject{currentmarker}{}%
\end{pgfscope}%
\begin{pgfscope}%
\pgfsys@transformshift{2.194118in}{0.737253in}%
\pgfsys@useobject{currentmarker}{}%
\end{pgfscope}%
\begin{pgfscope}%
\pgfsys@transformshift{2.194305in}{0.709950in}%
\pgfsys@useobject{currentmarker}{}%
\end{pgfscope}%
\begin{pgfscope}%
\pgfsys@transformshift{2.194492in}{0.718345in}%
\pgfsys@useobject{currentmarker}{}%
\end{pgfscope}%
\begin{pgfscope}%
\pgfsys@transformshift{2.194678in}{0.734083in}%
\pgfsys@useobject{currentmarker}{}%
\end{pgfscope}%
\begin{pgfscope}%
\pgfsys@transformshift{2.194864in}{0.715521in}%
\pgfsys@useobject{currentmarker}{}%
\end{pgfscope}%
\begin{pgfscope}%
\pgfsys@transformshift{2.195050in}{0.712621in}%
\pgfsys@useobject{currentmarker}{}%
\end{pgfscope}%
\begin{pgfscope}%
\pgfsys@transformshift{2.195236in}{0.714380in}%
\pgfsys@useobject{currentmarker}{}%
\end{pgfscope}%
\begin{pgfscope}%
\pgfsys@transformshift{2.195422in}{0.732256in}%
\pgfsys@useobject{currentmarker}{}%
\end{pgfscope}%
\begin{pgfscope}%
\pgfsys@transformshift{2.195608in}{0.775041in}%
\pgfsys@useobject{currentmarker}{}%
\end{pgfscope}%
\begin{pgfscope}%
\pgfsys@transformshift{2.195794in}{0.756623in}%
\pgfsys@useobject{currentmarker}{}%
\end{pgfscope}%
\begin{pgfscope}%
\pgfsys@transformshift{2.195979in}{0.726466in}%
\pgfsys@useobject{currentmarker}{}%
\end{pgfscope}%
\begin{pgfscope}%
\pgfsys@transformshift{2.196164in}{0.734216in}%
\pgfsys@useobject{currentmarker}{}%
\end{pgfscope}%
\begin{pgfscope}%
\pgfsys@transformshift{2.196349in}{0.733288in}%
\pgfsys@useobject{currentmarker}{}%
\end{pgfscope}%
\begin{pgfscope}%
\pgfsys@transformshift{2.196534in}{0.646449in}%
\pgfsys@useobject{currentmarker}{}%
\end{pgfscope}%
\begin{pgfscope}%
\pgfsys@transformshift{2.196719in}{0.702246in}%
\pgfsys@useobject{currentmarker}{}%
\end{pgfscope}%
\begin{pgfscope}%
\pgfsys@transformshift{2.196903in}{0.708147in}%
\pgfsys@useobject{currentmarker}{}%
\end{pgfscope}%
\begin{pgfscope}%
\pgfsys@transformshift{2.197088in}{0.736765in}%
\pgfsys@useobject{currentmarker}{}%
\end{pgfscope}%
\begin{pgfscope}%
\pgfsys@transformshift{2.197272in}{0.718781in}%
\pgfsys@useobject{currentmarker}{}%
\end{pgfscope}%
\begin{pgfscope}%
\pgfsys@transformshift{2.197456in}{0.714060in}%
\pgfsys@useobject{currentmarker}{}%
\end{pgfscope}%
\begin{pgfscope}%
\pgfsys@transformshift{2.197640in}{0.726455in}%
\pgfsys@useobject{currentmarker}{}%
\end{pgfscope}%
\begin{pgfscope}%
\pgfsys@transformshift{2.197824in}{0.707486in}%
\pgfsys@useobject{currentmarker}{}%
\end{pgfscope}%
\begin{pgfscope}%
\pgfsys@transformshift{2.198008in}{0.775248in}%
\pgfsys@useobject{currentmarker}{}%
\end{pgfscope}%
\begin{pgfscope}%
\pgfsys@transformshift{2.198192in}{0.761214in}%
\pgfsys@useobject{currentmarker}{}%
\end{pgfscope}%
\begin{pgfscope}%
\pgfsys@transformshift{2.198375in}{0.744398in}%
\pgfsys@useobject{currentmarker}{}%
\end{pgfscope}%
\begin{pgfscope}%
\pgfsys@transformshift{2.198558in}{0.737847in}%
\pgfsys@useobject{currentmarker}{}%
\end{pgfscope}%
\begin{pgfscope}%
\pgfsys@transformshift{2.198741in}{0.716402in}%
\pgfsys@useobject{currentmarker}{}%
\end{pgfscope}%
\begin{pgfscope}%
\pgfsys@transformshift{2.198924in}{0.684829in}%
\pgfsys@useobject{currentmarker}{}%
\end{pgfscope}%
\begin{pgfscope}%
\pgfsys@transformshift{2.199107in}{0.721821in}%
\pgfsys@useobject{currentmarker}{}%
\end{pgfscope}%
\begin{pgfscope}%
\pgfsys@transformshift{2.199290in}{0.730431in}%
\pgfsys@useobject{currentmarker}{}%
\end{pgfscope}%
\begin{pgfscope}%
\pgfsys@transformshift{2.199472in}{0.721561in}%
\pgfsys@useobject{currentmarker}{}%
\end{pgfscope}%
\begin{pgfscope}%
\pgfsys@transformshift{2.199655in}{0.705400in}%
\pgfsys@useobject{currentmarker}{}%
\end{pgfscope}%
\begin{pgfscope}%
\pgfsys@transformshift{2.199837in}{0.706114in}%
\pgfsys@useobject{currentmarker}{}%
\end{pgfscope}%
\begin{pgfscope}%
\pgfsys@transformshift{2.200019in}{0.716057in}%
\pgfsys@useobject{currentmarker}{}%
\end{pgfscope}%
\begin{pgfscope}%
\pgfsys@transformshift{2.200201in}{0.715337in}%
\pgfsys@useobject{currentmarker}{}%
\end{pgfscope}%
\begin{pgfscope}%
\pgfsys@transformshift{2.200383in}{0.724548in}%
\pgfsys@useobject{currentmarker}{}%
\end{pgfscope}%
\begin{pgfscope}%
\pgfsys@transformshift{2.200564in}{0.716869in}%
\pgfsys@useobject{currentmarker}{}%
\end{pgfscope}%
\begin{pgfscope}%
\pgfsys@transformshift{2.200746in}{0.699830in}%
\pgfsys@useobject{currentmarker}{}%
\end{pgfscope}%
\begin{pgfscope}%
\pgfsys@transformshift{2.200927in}{0.701824in}%
\pgfsys@useobject{currentmarker}{}%
\end{pgfscope}%
\begin{pgfscope}%
\pgfsys@transformshift{2.201108in}{0.725357in}%
\pgfsys@useobject{currentmarker}{}%
\end{pgfscope}%
\begin{pgfscope}%
\pgfsys@transformshift{2.201289in}{0.725354in}%
\pgfsys@useobject{currentmarker}{}%
\end{pgfscope}%
\begin{pgfscope}%
\pgfsys@transformshift{2.201470in}{0.714010in}%
\pgfsys@useobject{currentmarker}{}%
\end{pgfscope}%
\begin{pgfscope}%
\pgfsys@transformshift{2.201651in}{0.676558in}%
\pgfsys@useobject{currentmarker}{}%
\end{pgfscope}%
\begin{pgfscope}%
\pgfsys@transformshift{2.201831in}{0.699883in}%
\pgfsys@useobject{currentmarker}{}%
\end{pgfscope}%
\begin{pgfscope}%
\pgfsys@transformshift{2.202012in}{0.727213in}%
\pgfsys@useobject{currentmarker}{}%
\end{pgfscope}%
\begin{pgfscope}%
\pgfsys@transformshift{2.202192in}{0.663284in}%
\pgfsys@useobject{currentmarker}{}%
\end{pgfscope}%
\begin{pgfscope}%
\pgfsys@transformshift{2.202372in}{0.638932in}%
\pgfsys@useobject{currentmarker}{}%
\end{pgfscope}%
\begin{pgfscope}%
\pgfsys@transformshift{2.202552in}{0.652077in}%
\pgfsys@useobject{currentmarker}{}%
\end{pgfscope}%
\begin{pgfscope}%
\pgfsys@transformshift{2.202732in}{0.666581in}%
\pgfsys@useobject{currentmarker}{}%
\end{pgfscope}%
\begin{pgfscope}%
\pgfsys@transformshift{2.202912in}{0.697078in}%
\pgfsys@useobject{currentmarker}{}%
\end{pgfscope}%
\begin{pgfscope}%
\pgfsys@transformshift{2.203092in}{0.662157in}%
\pgfsys@useobject{currentmarker}{}%
\end{pgfscope}%
\begin{pgfscope}%
\pgfsys@transformshift{2.203271in}{0.674000in}%
\pgfsys@useobject{currentmarker}{}%
\end{pgfscope}%
\begin{pgfscope}%
\pgfsys@transformshift{2.203450in}{0.689189in}%
\pgfsys@useobject{currentmarker}{}%
\end{pgfscope}%
\begin{pgfscope}%
\pgfsys@transformshift{2.203630in}{0.706961in}%
\pgfsys@useobject{currentmarker}{}%
\end{pgfscope}%
\begin{pgfscope}%
\pgfsys@transformshift{2.203809in}{0.719724in}%
\pgfsys@useobject{currentmarker}{}%
\end{pgfscope}%
\begin{pgfscope}%
\pgfsys@transformshift{2.203988in}{0.702542in}%
\pgfsys@useobject{currentmarker}{}%
\end{pgfscope}%
\begin{pgfscope}%
\pgfsys@transformshift{2.204166in}{0.674417in}%
\pgfsys@useobject{currentmarker}{}%
\end{pgfscope}%
\begin{pgfscope}%
\pgfsys@transformshift{2.204345in}{0.700634in}%
\pgfsys@useobject{currentmarker}{}%
\end{pgfscope}%
\begin{pgfscope}%
\pgfsys@transformshift{2.204523in}{0.759846in}%
\pgfsys@useobject{currentmarker}{}%
\end{pgfscope}%
\begin{pgfscope}%
\pgfsys@transformshift{2.204702in}{0.754821in}%
\pgfsys@useobject{currentmarker}{}%
\end{pgfscope}%
\begin{pgfscope}%
\pgfsys@transformshift{2.204880in}{0.734934in}%
\pgfsys@useobject{currentmarker}{}%
\end{pgfscope}%
\begin{pgfscope}%
\pgfsys@transformshift{2.205058in}{0.670861in}%
\pgfsys@useobject{currentmarker}{}%
\end{pgfscope}%
\begin{pgfscope}%
\pgfsys@transformshift{2.205236in}{0.681583in}%
\pgfsys@useobject{currentmarker}{}%
\end{pgfscope}%
\begin{pgfscope}%
\pgfsys@transformshift{2.205413in}{0.723891in}%
\pgfsys@useobject{currentmarker}{}%
\end{pgfscope}%
\begin{pgfscope}%
\pgfsys@transformshift{2.205591in}{0.737738in}%
\pgfsys@useobject{currentmarker}{}%
\end{pgfscope}%
\begin{pgfscope}%
\pgfsys@transformshift{2.205769in}{0.645643in}%
\pgfsys@useobject{currentmarker}{}%
\end{pgfscope}%
\begin{pgfscope}%
\pgfsys@transformshift{2.205946in}{0.748504in}%
\pgfsys@useobject{currentmarker}{}%
\end{pgfscope}%
\begin{pgfscope}%
\pgfsys@transformshift{2.206123in}{0.734348in}%
\pgfsys@useobject{currentmarker}{}%
\end{pgfscope}%
\begin{pgfscope}%
\pgfsys@transformshift{2.206300in}{0.715132in}%
\pgfsys@useobject{currentmarker}{}%
\end{pgfscope}%
\begin{pgfscope}%
\pgfsys@transformshift{2.206477in}{0.727069in}%
\pgfsys@useobject{currentmarker}{}%
\end{pgfscope}%
\begin{pgfscope}%
\pgfsys@transformshift{2.206654in}{0.703978in}%
\pgfsys@useobject{currentmarker}{}%
\end{pgfscope}%
\begin{pgfscope}%
\pgfsys@transformshift{2.206830in}{0.682167in}%
\pgfsys@useobject{currentmarker}{}%
\end{pgfscope}%
\begin{pgfscope}%
\pgfsys@transformshift{2.207007in}{0.739906in}%
\pgfsys@useobject{currentmarker}{}%
\end{pgfscope}%
\begin{pgfscope}%
\pgfsys@transformshift{2.207183in}{0.739218in}%
\pgfsys@useobject{currentmarker}{}%
\end{pgfscope}%
\begin{pgfscope}%
\pgfsys@transformshift{2.207359in}{0.699649in}%
\pgfsys@useobject{currentmarker}{}%
\end{pgfscope}%
\begin{pgfscope}%
\pgfsys@transformshift{2.207536in}{0.676229in}%
\pgfsys@useobject{currentmarker}{}%
\end{pgfscope}%
\begin{pgfscope}%
\pgfsys@transformshift{2.207712in}{0.708207in}%
\pgfsys@useobject{currentmarker}{}%
\end{pgfscope}%
\begin{pgfscope}%
\pgfsys@transformshift{2.207887in}{0.707514in}%
\pgfsys@useobject{currentmarker}{}%
\end{pgfscope}%
\begin{pgfscope}%
\pgfsys@transformshift{2.208063in}{0.689248in}%
\pgfsys@useobject{currentmarker}{}%
\end{pgfscope}%
\begin{pgfscope}%
\pgfsys@transformshift{2.208239in}{0.741295in}%
\pgfsys@useobject{currentmarker}{}%
\end{pgfscope}%
\begin{pgfscope}%
\pgfsys@transformshift{2.208414in}{0.745068in}%
\pgfsys@useobject{currentmarker}{}%
\end{pgfscope}%
\begin{pgfscope}%
\pgfsys@transformshift{2.208589in}{0.671633in}%
\pgfsys@useobject{currentmarker}{}%
\end{pgfscope}%
\begin{pgfscope}%
\pgfsys@transformshift{2.208764in}{0.662407in}%
\pgfsys@useobject{currentmarker}{}%
\end{pgfscope}%
\begin{pgfscope}%
\pgfsys@transformshift{2.208939in}{0.663120in}%
\pgfsys@useobject{currentmarker}{}%
\end{pgfscope}%
\begin{pgfscope}%
\pgfsys@transformshift{2.209114in}{0.653842in}%
\pgfsys@useobject{currentmarker}{}%
\end{pgfscope}%
\begin{pgfscope}%
\pgfsys@transformshift{2.209289in}{0.691873in}%
\pgfsys@useobject{currentmarker}{}%
\end{pgfscope}%
\begin{pgfscope}%
\pgfsys@transformshift{2.209463in}{0.632074in}%
\pgfsys@useobject{currentmarker}{}%
\end{pgfscope}%
\begin{pgfscope}%
\pgfsys@transformshift{2.209638in}{0.672730in}%
\pgfsys@useobject{currentmarker}{}%
\end{pgfscope}%
\begin{pgfscope}%
\pgfsys@transformshift{2.209812in}{0.683276in}%
\pgfsys@useobject{currentmarker}{}%
\end{pgfscope}%
\begin{pgfscope}%
\pgfsys@transformshift{2.209986in}{0.708098in}%
\pgfsys@useobject{currentmarker}{}%
\end{pgfscope}%
\begin{pgfscope}%
\pgfsys@transformshift{2.210160in}{0.724566in}%
\pgfsys@useobject{currentmarker}{}%
\end{pgfscope}%
\begin{pgfscope}%
\pgfsys@transformshift{2.210334in}{0.718782in}%
\pgfsys@useobject{currentmarker}{}%
\end{pgfscope}%
\begin{pgfscope}%
\pgfsys@transformshift{2.210508in}{0.716675in}%
\pgfsys@useobject{currentmarker}{}%
\end{pgfscope}%
\begin{pgfscope}%
\pgfsys@transformshift{2.210682in}{0.687867in}%
\pgfsys@useobject{currentmarker}{}%
\end{pgfscope}%
\begin{pgfscope}%
\pgfsys@transformshift{2.210855in}{0.709187in}%
\pgfsys@useobject{currentmarker}{}%
\end{pgfscope}%
\begin{pgfscope}%
\pgfsys@transformshift{2.211028in}{0.682204in}%
\pgfsys@useobject{currentmarker}{}%
\end{pgfscope}%
\begin{pgfscope}%
\pgfsys@transformshift{2.211202in}{0.655565in}%
\pgfsys@useobject{currentmarker}{}%
\end{pgfscope}%
\begin{pgfscope}%
\pgfsys@transformshift{2.211375in}{0.676005in}%
\pgfsys@useobject{currentmarker}{}%
\end{pgfscope}%
\begin{pgfscope}%
\pgfsys@transformshift{2.211548in}{0.679882in}%
\pgfsys@useobject{currentmarker}{}%
\end{pgfscope}%
\begin{pgfscope}%
\pgfsys@transformshift{2.211721in}{0.673464in}%
\pgfsys@useobject{currentmarker}{}%
\end{pgfscope}%
\begin{pgfscope}%
\pgfsys@transformshift{2.211893in}{0.683523in}%
\pgfsys@useobject{currentmarker}{}%
\end{pgfscope}%
\begin{pgfscope}%
\pgfsys@transformshift{2.212066in}{0.721890in}%
\pgfsys@useobject{currentmarker}{}%
\end{pgfscope}%
\begin{pgfscope}%
\pgfsys@transformshift{2.212238in}{0.716432in}%
\pgfsys@useobject{currentmarker}{}%
\end{pgfscope}%
\begin{pgfscope}%
\pgfsys@transformshift{2.212411in}{0.672571in}%
\pgfsys@useobject{currentmarker}{}%
\end{pgfscope}%
\begin{pgfscope}%
\pgfsys@transformshift{2.212583in}{0.669261in}%
\pgfsys@useobject{currentmarker}{}%
\end{pgfscope}%
\begin{pgfscope}%
\pgfsys@transformshift{2.212755in}{0.660733in}%
\pgfsys@useobject{currentmarker}{}%
\end{pgfscope}%
\begin{pgfscope}%
\pgfsys@transformshift{2.212927in}{0.622567in}%
\pgfsys@useobject{currentmarker}{}%
\end{pgfscope}%
\begin{pgfscope}%
\pgfsys@transformshift{2.213098in}{0.694528in}%
\pgfsys@useobject{currentmarker}{}%
\end{pgfscope}%
\begin{pgfscope}%
\pgfsys@transformshift{2.213270in}{0.710398in}%
\pgfsys@useobject{currentmarker}{}%
\end{pgfscope}%
\begin{pgfscope}%
\pgfsys@transformshift{2.213442in}{0.696765in}%
\pgfsys@useobject{currentmarker}{}%
\end{pgfscope}%
\begin{pgfscope}%
\pgfsys@transformshift{2.213613in}{0.681968in}%
\pgfsys@useobject{currentmarker}{}%
\end{pgfscope}%
\begin{pgfscope}%
\pgfsys@transformshift{2.213784in}{0.727688in}%
\pgfsys@useobject{currentmarker}{}%
\end{pgfscope}%
\begin{pgfscope}%
\pgfsys@transformshift{2.213955in}{0.726179in}%
\pgfsys@useobject{currentmarker}{}%
\end{pgfscope}%
\begin{pgfscope}%
\pgfsys@transformshift{2.214126in}{0.703603in}%
\pgfsys@useobject{currentmarker}{}%
\end{pgfscope}%
\begin{pgfscope}%
\pgfsys@transformshift{2.214297in}{0.733967in}%
\pgfsys@useobject{currentmarker}{}%
\end{pgfscope}%
\begin{pgfscope}%
\pgfsys@transformshift{2.214468in}{0.704482in}%
\pgfsys@useobject{currentmarker}{}%
\end{pgfscope}%
\begin{pgfscope}%
\pgfsys@transformshift{2.214639in}{0.709051in}%
\pgfsys@useobject{currentmarker}{}%
\end{pgfscope}%
\begin{pgfscope}%
\pgfsys@transformshift{2.214809in}{0.707598in}%
\pgfsys@useobject{currentmarker}{}%
\end{pgfscope}%
\begin{pgfscope}%
\pgfsys@transformshift{2.214979in}{0.714595in}%
\pgfsys@useobject{currentmarker}{}%
\end{pgfscope}%
\begin{pgfscope}%
\pgfsys@transformshift{2.215150in}{0.670509in}%
\pgfsys@useobject{currentmarker}{}%
\end{pgfscope}%
\begin{pgfscope}%
\pgfsys@transformshift{2.215320in}{0.680392in}%
\pgfsys@useobject{currentmarker}{}%
\end{pgfscope}%
\begin{pgfscope}%
\pgfsys@transformshift{2.215490in}{0.746188in}%
\pgfsys@useobject{currentmarker}{}%
\end{pgfscope}%
\begin{pgfscope}%
\pgfsys@transformshift{2.215659in}{0.721865in}%
\pgfsys@useobject{currentmarker}{}%
\end{pgfscope}%
\begin{pgfscope}%
\pgfsys@transformshift{2.215829in}{0.691407in}%
\pgfsys@useobject{currentmarker}{}%
\end{pgfscope}%
\begin{pgfscope}%
\pgfsys@transformshift{2.215999in}{0.689233in}%
\pgfsys@useobject{currentmarker}{}%
\end{pgfscope}%
\begin{pgfscope}%
\pgfsys@transformshift{2.216168in}{0.707457in}%
\pgfsys@useobject{currentmarker}{}%
\end{pgfscope}%
\begin{pgfscope}%
\pgfsys@transformshift{2.216337in}{0.695610in}%
\pgfsys@useobject{currentmarker}{}%
\end{pgfscope}%
\begin{pgfscope}%
\pgfsys@transformshift{2.216507in}{0.744680in}%
\pgfsys@useobject{currentmarker}{}%
\end{pgfscope}%
\begin{pgfscope}%
\pgfsys@transformshift{2.216676in}{0.748624in}%
\pgfsys@useobject{currentmarker}{}%
\end{pgfscope}%
\begin{pgfscope}%
\pgfsys@transformshift{2.216845in}{0.706219in}%
\pgfsys@useobject{currentmarker}{}%
\end{pgfscope}%
\begin{pgfscope}%
\pgfsys@transformshift{2.217013in}{0.678859in}%
\pgfsys@useobject{currentmarker}{}%
\end{pgfscope}%
\begin{pgfscope}%
\pgfsys@transformshift{2.217182in}{0.713202in}%
\pgfsys@useobject{currentmarker}{}%
\end{pgfscope}%
\begin{pgfscope}%
\pgfsys@transformshift{2.217351in}{0.700952in}%
\pgfsys@useobject{currentmarker}{}%
\end{pgfscope}%
\begin{pgfscope}%
\pgfsys@transformshift{2.217519in}{0.742016in}%
\pgfsys@useobject{currentmarker}{}%
\end{pgfscope}%
\begin{pgfscope}%
\pgfsys@transformshift{2.217687in}{0.732304in}%
\pgfsys@useobject{currentmarker}{}%
\end{pgfscope}%
\begin{pgfscope}%
\pgfsys@transformshift{2.217856in}{0.705881in}%
\pgfsys@useobject{currentmarker}{}%
\end{pgfscope}%
\begin{pgfscope}%
\pgfsys@transformshift{2.218024in}{0.689518in}%
\pgfsys@useobject{currentmarker}{}%
\end{pgfscope}%
\begin{pgfscope}%
\pgfsys@transformshift{2.218192in}{0.710824in}%
\pgfsys@useobject{currentmarker}{}%
\end{pgfscope}%
\begin{pgfscope}%
\pgfsys@transformshift{2.218359in}{0.726755in}%
\pgfsys@useobject{currentmarker}{}%
\end{pgfscope}%
\begin{pgfscope}%
\pgfsys@transformshift{2.218527in}{0.720829in}%
\pgfsys@useobject{currentmarker}{}%
\end{pgfscope}%
\begin{pgfscope}%
\pgfsys@transformshift{2.218695in}{0.761760in}%
\pgfsys@useobject{currentmarker}{}%
\end{pgfscope}%
\begin{pgfscope}%
\pgfsys@transformshift{2.218862in}{0.717056in}%
\pgfsys@useobject{currentmarker}{}%
\end{pgfscope}%
\begin{pgfscope}%
\pgfsys@transformshift{2.219029in}{0.681954in}%
\pgfsys@useobject{currentmarker}{}%
\end{pgfscope}%
\begin{pgfscope}%
\pgfsys@transformshift{2.219196in}{0.666884in}%
\pgfsys@useobject{currentmarker}{}%
\end{pgfscope}%
\begin{pgfscope}%
\pgfsys@transformshift{2.219364in}{0.606129in}%
\pgfsys@useobject{currentmarker}{}%
\end{pgfscope}%
\begin{pgfscope}%
\pgfsys@transformshift{2.219530in}{0.691758in}%
\pgfsys@useobject{currentmarker}{}%
\end{pgfscope}%
\begin{pgfscope}%
\pgfsys@transformshift{2.219697in}{0.678163in}%
\pgfsys@useobject{currentmarker}{}%
\end{pgfscope}%
\begin{pgfscope}%
\pgfsys@transformshift{2.219864in}{0.716692in}%
\pgfsys@useobject{currentmarker}{}%
\end{pgfscope}%
\begin{pgfscope}%
\pgfsys@transformshift{2.220030in}{0.698782in}%
\pgfsys@useobject{currentmarker}{}%
\end{pgfscope}%
\begin{pgfscope}%
\pgfsys@transformshift{2.220197in}{0.738551in}%
\pgfsys@useobject{currentmarker}{}%
\end{pgfscope}%
\begin{pgfscope}%
\pgfsys@transformshift{2.220363in}{0.771667in}%
\pgfsys@useobject{currentmarker}{}%
\end{pgfscope}%
\begin{pgfscope}%
\pgfsys@transformshift{2.220529in}{0.681734in}%
\pgfsys@useobject{currentmarker}{}%
\end{pgfscope}%
\begin{pgfscope}%
\pgfsys@transformshift{2.220695in}{0.698710in}%
\pgfsys@useobject{currentmarker}{}%
\end{pgfscope}%
\begin{pgfscope}%
\pgfsys@transformshift{2.220861in}{0.736800in}%
\pgfsys@useobject{currentmarker}{}%
\end{pgfscope}%
\begin{pgfscope}%
\pgfsys@transformshift{2.221027in}{0.750201in}%
\pgfsys@useobject{currentmarker}{}%
\end{pgfscope}%
\begin{pgfscope}%
\pgfsys@transformshift{2.221193in}{0.718708in}%
\pgfsys@useobject{currentmarker}{}%
\end{pgfscope}%
\begin{pgfscope}%
\pgfsys@transformshift{2.221358in}{0.672329in}%
\pgfsys@useobject{currentmarker}{}%
\end{pgfscope}%
\begin{pgfscope}%
\pgfsys@transformshift{2.221524in}{0.678924in}%
\pgfsys@useobject{currentmarker}{}%
\end{pgfscope}%
\begin{pgfscope}%
\pgfsys@transformshift{2.221689in}{0.682488in}%
\pgfsys@useobject{currentmarker}{}%
\end{pgfscope}%
\begin{pgfscope}%
\pgfsys@transformshift{2.221854in}{0.681525in}%
\pgfsys@useobject{currentmarker}{}%
\end{pgfscope}%
\begin{pgfscope}%
\pgfsys@transformshift{2.222020in}{0.712029in}%
\pgfsys@useobject{currentmarker}{}%
\end{pgfscope}%
\begin{pgfscope}%
\pgfsys@transformshift{2.222185in}{0.704233in}%
\pgfsys@useobject{currentmarker}{}%
\end{pgfscope}%
\begin{pgfscope}%
\pgfsys@transformshift{2.222349in}{0.696173in}%
\pgfsys@useobject{currentmarker}{}%
\end{pgfscope}%
\begin{pgfscope}%
\pgfsys@transformshift{2.222514in}{0.757469in}%
\pgfsys@useobject{currentmarker}{}%
\end{pgfscope}%
\begin{pgfscope}%
\pgfsys@transformshift{2.222679in}{0.715206in}%
\pgfsys@useobject{currentmarker}{}%
\end{pgfscope}%
\begin{pgfscope}%
\pgfsys@transformshift{2.222843in}{0.665008in}%
\pgfsys@useobject{currentmarker}{}%
\end{pgfscope}%
\begin{pgfscope}%
\pgfsys@transformshift{2.223008in}{0.681274in}%
\pgfsys@useobject{currentmarker}{}%
\end{pgfscope}%
\begin{pgfscope}%
\pgfsys@transformshift{2.223172in}{0.677021in}%
\pgfsys@useobject{currentmarker}{}%
\end{pgfscope}%
\begin{pgfscope}%
\pgfsys@transformshift{2.223336in}{0.679586in}%
\pgfsys@useobject{currentmarker}{}%
\end{pgfscope}%
\begin{pgfscope}%
\pgfsys@transformshift{2.223500in}{0.736616in}%
\pgfsys@useobject{currentmarker}{}%
\end{pgfscope}%
\begin{pgfscope}%
\pgfsys@transformshift{2.223664in}{0.746874in}%
\pgfsys@useobject{currentmarker}{}%
\end{pgfscope}%
\begin{pgfscope}%
\pgfsys@transformshift{2.223828in}{0.735464in}%
\pgfsys@useobject{currentmarker}{}%
\end{pgfscope}%
\begin{pgfscope}%
\pgfsys@transformshift{2.223991in}{0.721201in}%
\pgfsys@useobject{currentmarker}{}%
\end{pgfscope}%
\begin{pgfscope}%
\pgfsys@transformshift{2.224155in}{0.744112in}%
\pgfsys@useobject{currentmarker}{}%
\end{pgfscope}%
\begin{pgfscope}%
\pgfsys@transformshift{2.224318in}{0.733977in}%
\pgfsys@useobject{currentmarker}{}%
\end{pgfscope}%
\begin{pgfscope}%
\pgfsys@transformshift{2.224481in}{0.677522in}%
\pgfsys@useobject{currentmarker}{}%
\end{pgfscope}%
\begin{pgfscope}%
\pgfsys@transformshift{2.224645in}{0.706371in}%
\pgfsys@useobject{currentmarker}{}%
\end{pgfscope}%
\begin{pgfscope}%
\pgfsys@transformshift{2.224808in}{0.698013in}%
\pgfsys@useobject{currentmarker}{}%
\end{pgfscope}%
\begin{pgfscope}%
\pgfsys@transformshift{2.224971in}{0.722121in}%
\pgfsys@useobject{currentmarker}{}%
\end{pgfscope}%
\begin{pgfscope}%
\pgfsys@transformshift{2.225133in}{0.758441in}%
\pgfsys@useobject{currentmarker}{}%
\end{pgfscope}%
\begin{pgfscope}%
\pgfsys@transformshift{2.225296in}{0.777174in}%
\pgfsys@useobject{currentmarker}{}%
\end{pgfscope}%
\begin{pgfscope}%
\pgfsys@transformshift{2.225459in}{0.724638in}%
\pgfsys@useobject{currentmarker}{}%
\end{pgfscope}%
\begin{pgfscope}%
\pgfsys@transformshift{2.225621in}{0.701874in}%
\pgfsys@useobject{currentmarker}{}%
\end{pgfscope}%
\begin{pgfscope}%
\pgfsys@transformshift{2.225783in}{0.703992in}%
\pgfsys@useobject{currentmarker}{}%
\end{pgfscope}%
\begin{pgfscope}%
\pgfsys@transformshift{2.225946in}{0.723095in}%
\pgfsys@useobject{currentmarker}{}%
\end{pgfscope}%
\begin{pgfscope}%
\pgfsys@transformshift{2.226108in}{0.721481in}%
\pgfsys@useobject{currentmarker}{}%
\end{pgfscope}%
\begin{pgfscope}%
\pgfsys@transformshift{2.226270in}{0.666646in}%
\pgfsys@useobject{currentmarker}{}%
\end{pgfscope}%
\begin{pgfscope}%
\pgfsys@transformshift{2.226432in}{0.740396in}%
\pgfsys@useobject{currentmarker}{}%
\end{pgfscope}%
\begin{pgfscope}%
\pgfsys@transformshift{2.226593in}{0.741126in}%
\pgfsys@useobject{currentmarker}{}%
\end{pgfscope}%
\begin{pgfscope}%
\pgfsys@transformshift{2.226755in}{0.726582in}%
\pgfsys@useobject{currentmarker}{}%
\end{pgfscope}%
\begin{pgfscope}%
\pgfsys@transformshift{2.226917in}{0.698789in}%
\pgfsys@useobject{currentmarker}{}%
\end{pgfscope}%
\begin{pgfscope}%
\pgfsys@transformshift{2.227078in}{0.679218in}%
\pgfsys@useobject{currentmarker}{}%
\end{pgfscope}%
\begin{pgfscope}%
\pgfsys@transformshift{2.227239in}{0.705997in}%
\pgfsys@useobject{currentmarker}{}%
\end{pgfscope}%
\begin{pgfscope}%
\pgfsys@transformshift{2.227400in}{0.722009in}%
\pgfsys@useobject{currentmarker}{}%
\end{pgfscope}%
\begin{pgfscope}%
\pgfsys@transformshift{2.227562in}{0.641916in}%
\pgfsys@useobject{currentmarker}{}%
\end{pgfscope}%
\begin{pgfscope}%
\pgfsys@transformshift{2.227722in}{0.659539in}%
\pgfsys@useobject{currentmarker}{}%
\end{pgfscope}%
\begin{pgfscope}%
\pgfsys@transformshift{2.227883in}{0.675330in}%
\pgfsys@useobject{currentmarker}{}%
\end{pgfscope}%
\begin{pgfscope}%
\pgfsys@transformshift{2.228044in}{0.692210in}%
\pgfsys@useobject{currentmarker}{}%
\end{pgfscope}%
\begin{pgfscope}%
\pgfsys@transformshift{2.228205in}{0.686245in}%
\pgfsys@useobject{currentmarker}{}%
\end{pgfscope}%
\begin{pgfscope}%
\pgfsys@transformshift{2.228365in}{0.699014in}%
\pgfsys@useobject{currentmarker}{}%
\end{pgfscope}%
\begin{pgfscope}%
\pgfsys@transformshift{2.228525in}{0.705021in}%
\pgfsys@useobject{currentmarker}{}%
\end{pgfscope}%
\begin{pgfscope}%
\pgfsys@transformshift{2.228686in}{0.685201in}%
\pgfsys@useobject{currentmarker}{}%
\end{pgfscope}%
\begin{pgfscope}%
\pgfsys@transformshift{2.228846in}{0.680196in}%
\pgfsys@useobject{currentmarker}{}%
\end{pgfscope}%
\begin{pgfscope}%
\pgfsys@transformshift{2.229006in}{0.670964in}%
\pgfsys@useobject{currentmarker}{}%
\end{pgfscope}%
\begin{pgfscope}%
\pgfsys@transformshift{2.229166in}{0.716096in}%
\pgfsys@useobject{currentmarker}{}%
\end{pgfscope}%
\begin{pgfscope}%
\pgfsys@transformshift{2.229326in}{0.685561in}%
\pgfsys@useobject{currentmarker}{}%
\end{pgfscope}%
\begin{pgfscope}%
\pgfsys@transformshift{2.229485in}{0.714111in}%
\pgfsys@useobject{currentmarker}{}%
\end{pgfscope}%
\begin{pgfscope}%
\pgfsys@transformshift{2.229645in}{0.705319in}%
\pgfsys@useobject{currentmarker}{}%
\end{pgfscope}%
\begin{pgfscope}%
\pgfsys@transformshift{2.229804in}{0.743994in}%
\pgfsys@useobject{currentmarker}{}%
\end{pgfscope}%
\begin{pgfscope}%
\pgfsys@transformshift{2.229964in}{0.759499in}%
\pgfsys@useobject{currentmarker}{}%
\end{pgfscope}%
\begin{pgfscope}%
\pgfsys@transformshift{2.230123in}{0.769836in}%
\pgfsys@useobject{currentmarker}{}%
\end{pgfscope}%
\begin{pgfscope}%
\pgfsys@transformshift{2.230282in}{0.777032in}%
\pgfsys@useobject{currentmarker}{}%
\end{pgfscope}%
\begin{pgfscope}%
\pgfsys@transformshift{2.230441in}{0.767157in}%
\pgfsys@useobject{currentmarker}{}%
\end{pgfscope}%
\begin{pgfscope}%
\pgfsys@transformshift{2.230600in}{0.774499in}%
\pgfsys@useobject{currentmarker}{}%
\end{pgfscope}%
\begin{pgfscope}%
\pgfsys@transformshift{2.230759in}{0.744127in}%
\pgfsys@useobject{currentmarker}{}%
\end{pgfscope}%
\begin{pgfscope}%
\pgfsys@transformshift{2.230917in}{0.698464in}%
\pgfsys@useobject{currentmarker}{}%
\end{pgfscope}%
\begin{pgfscope}%
\pgfsys@transformshift{2.231076in}{0.716751in}%
\pgfsys@useobject{currentmarker}{}%
\end{pgfscope}%
\begin{pgfscope}%
\pgfsys@transformshift{2.231234in}{0.749789in}%
\pgfsys@useobject{currentmarker}{}%
\end{pgfscope}%
\begin{pgfscope}%
\pgfsys@transformshift{2.231393in}{0.681999in}%
\pgfsys@useobject{currentmarker}{}%
\end{pgfscope}%
\begin{pgfscope}%
\pgfsys@transformshift{2.231551in}{0.729898in}%
\pgfsys@useobject{currentmarker}{}%
\end{pgfscope}%
\begin{pgfscope}%
\pgfsys@transformshift{2.231709in}{0.717214in}%
\pgfsys@useobject{currentmarker}{}%
\end{pgfscope}%
\begin{pgfscope}%
\pgfsys@transformshift{2.231867in}{0.671668in}%
\pgfsys@useobject{currentmarker}{}%
\end{pgfscope}%
\begin{pgfscope}%
\pgfsys@transformshift{2.232025in}{0.726181in}%
\pgfsys@useobject{currentmarker}{}%
\end{pgfscope}%
\begin{pgfscope}%
\pgfsys@transformshift{2.232183in}{0.735537in}%
\pgfsys@useobject{currentmarker}{}%
\end{pgfscope}%
\begin{pgfscope}%
\pgfsys@transformshift{2.232341in}{0.694695in}%
\pgfsys@useobject{currentmarker}{}%
\end{pgfscope}%
\begin{pgfscope}%
\pgfsys@transformshift{2.232498in}{0.683343in}%
\pgfsys@useobject{currentmarker}{}%
\end{pgfscope}%
\begin{pgfscope}%
\pgfsys@transformshift{2.232656in}{0.711313in}%
\pgfsys@useobject{currentmarker}{}%
\end{pgfscope}%
\begin{pgfscope}%
\pgfsys@transformshift{2.232813in}{0.711649in}%
\pgfsys@useobject{currentmarker}{}%
\end{pgfscope}%
\begin{pgfscope}%
\pgfsys@transformshift{2.232970in}{0.709831in}%
\pgfsys@useobject{currentmarker}{}%
\end{pgfscope}%
\begin{pgfscope}%
\pgfsys@transformshift{2.233127in}{0.705522in}%
\pgfsys@useobject{currentmarker}{}%
\end{pgfscope}%
\begin{pgfscope}%
\pgfsys@transformshift{2.233284in}{0.683687in}%
\pgfsys@useobject{currentmarker}{}%
\end{pgfscope}%
\begin{pgfscope}%
\pgfsys@transformshift{2.233441in}{0.701756in}%
\pgfsys@useobject{currentmarker}{}%
\end{pgfscope}%
\begin{pgfscope}%
\pgfsys@transformshift{2.233598in}{0.704296in}%
\pgfsys@useobject{currentmarker}{}%
\end{pgfscope}%
\begin{pgfscope}%
\pgfsys@transformshift{2.233755in}{0.733735in}%
\pgfsys@useobject{currentmarker}{}%
\end{pgfscope}%
\begin{pgfscope}%
\pgfsys@transformshift{2.233911in}{0.727067in}%
\pgfsys@useobject{currentmarker}{}%
\end{pgfscope}%
\begin{pgfscope}%
\pgfsys@transformshift{2.234068in}{0.691282in}%
\pgfsys@useobject{currentmarker}{}%
\end{pgfscope}%
\begin{pgfscope}%
\pgfsys@transformshift{2.234224in}{0.722016in}%
\pgfsys@useobject{currentmarker}{}%
\end{pgfscope}%
\begin{pgfscope}%
\pgfsys@transformshift{2.234380in}{0.697631in}%
\pgfsys@useobject{currentmarker}{}%
\end{pgfscope}%
\begin{pgfscope}%
\pgfsys@transformshift{2.234536in}{0.690177in}%
\pgfsys@useobject{currentmarker}{}%
\end{pgfscope}%
\begin{pgfscope}%
\pgfsys@transformshift{2.234692in}{0.680094in}%
\pgfsys@useobject{currentmarker}{}%
\end{pgfscope}%
\begin{pgfscope}%
\pgfsys@transformshift{2.234848in}{0.689452in}%
\pgfsys@useobject{currentmarker}{}%
\end{pgfscope}%
\begin{pgfscope}%
\pgfsys@transformshift{2.235004in}{0.697590in}%
\pgfsys@useobject{currentmarker}{}%
\end{pgfscope}%
\begin{pgfscope}%
\pgfsys@transformshift{2.235160in}{0.740463in}%
\pgfsys@useobject{currentmarker}{}%
\end{pgfscope}%
\begin{pgfscope}%
\pgfsys@transformshift{2.235316in}{0.744502in}%
\pgfsys@useobject{currentmarker}{}%
\end{pgfscope}%
\begin{pgfscope}%
\pgfsys@transformshift{2.235471in}{0.751524in}%
\pgfsys@useobject{currentmarker}{}%
\end{pgfscope}%
\begin{pgfscope}%
\pgfsys@transformshift{2.235626in}{0.729572in}%
\pgfsys@useobject{currentmarker}{}%
\end{pgfscope}%
\begin{pgfscope}%
\pgfsys@transformshift{2.235782in}{0.701987in}%
\pgfsys@useobject{currentmarker}{}%
\end{pgfscope}%
\begin{pgfscope}%
\pgfsys@transformshift{2.235937in}{0.652469in}%
\pgfsys@useobject{currentmarker}{}%
\end{pgfscope}%
\begin{pgfscope}%
\pgfsys@transformshift{2.236092in}{0.755301in}%
\pgfsys@useobject{currentmarker}{}%
\end{pgfscope}%
\begin{pgfscope}%
\pgfsys@transformshift{2.236247in}{0.757764in}%
\pgfsys@useobject{currentmarker}{}%
\end{pgfscope}%
\begin{pgfscope}%
\pgfsys@transformshift{2.236402in}{0.659488in}%
\pgfsys@useobject{currentmarker}{}%
\end{pgfscope}%
\begin{pgfscope}%
\pgfsys@transformshift{2.236556in}{0.666320in}%
\pgfsys@useobject{currentmarker}{}%
\end{pgfscope}%
\begin{pgfscope}%
\pgfsys@transformshift{2.236711in}{0.680998in}%
\pgfsys@useobject{currentmarker}{}%
\end{pgfscope}%
\begin{pgfscope}%
\pgfsys@transformshift{2.236866in}{0.668075in}%
\pgfsys@useobject{currentmarker}{}%
\end{pgfscope}%
\begin{pgfscope}%
\pgfsys@transformshift{2.237020in}{0.728849in}%
\pgfsys@useobject{currentmarker}{}%
\end{pgfscope}%
\begin{pgfscope}%
\pgfsys@transformshift{2.237174in}{0.713085in}%
\pgfsys@useobject{currentmarker}{}%
\end{pgfscope}%
\begin{pgfscope}%
\pgfsys@transformshift{2.237329in}{0.708821in}%
\pgfsys@useobject{currentmarker}{}%
\end{pgfscope}%
\begin{pgfscope}%
\pgfsys@transformshift{2.237483in}{0.662090in}%
\pgfsys@useobject{currentmarker}{}%
\end{pgfscope}%
\begin{pgfscope}%
\pgfsys@transformshift{2.237637in}{0.691805in}%
\pgfsys@useobject{currentmarker}{}%
\end{pgfscope}%
\begin{pgfscope}%
\pgfsys@transformshift{2.237791in}{0.673486in}%
\pgfsys@useobject{currentmarker}{}%
\end{pgfscope}%
\begin{pgfscope}%
\pgfsys@transformshift{2.237944in}{0.683999in}%
\pgfsys@useobject{currentmarker}{}%
\end{pgfscope}%
\begin{pgfscope}%
\pgfsys@transformshift{2.238098in}{0.674939in}%
\pgfsys@useobject{currentmarker}{}%
\end{pgfscope}%
\begin{pgfscope}%
\pgfsys@transformshift{2.238252in}{0.693731in}%
\pgfsys@useobject{currentmarker}{}%
\end{pgfscope}%
\begin{pgfscope}%
\pgfsys@transformshift{2.238405in}{0.728905in}%
\pgfsys@useobject{currentmarker}{}%
\end{pgfscope}%
\begin{pgfscope}%
\pgfsys@transformshift{2.238558in}{0.698778in}%
\pgfsys@useobject{currentmarker}{}%
\end{pgfscope}%
\begin{pgfscope}%
\pgfsys@transformshift{2.238712in}{0.689717in}%
\pgfsys@useobject{currentmarker}{}%
\end{pgfscope}%
\begin{pgfscope}%
\pgfsys@transformshift{2.238865in}{0.720320in}%
\pgfsys@useobject{currentmarker}{}%
\end{pgfscope}%
\begin{pgfscope}%
\pgfsys@transformshift{2.239018in}{0.678074in}%
\pgfsys@useobject{currentmarker}{}%
\end{pgfscope}%
\begin{pgfscope}%
\pgfsys@transformshift{2.239171in}{0.676762in}%
\pgfsys@useobject{currentmarker}{}%
\end{pgfscope}%
\begin{pgfscope}%
\pgfsys@transformshift{2.239324in}{0.724258in}%
\pgfsys@useobject{currentmarker}{}%
\end{pgfscope}%
\begin{pgfscope}%
\pgfsys@transformshift{2.239476in}{0.699964in}%
\pgfsys@useobject{currentmarker}{}%
\end{pgfscope}%
\begin{pgfscope}%
\pgfsys@transformshift{2.239629in}{0.722085in}%
\pgfsys@useobject{currentmarker}{}%
\end{pgfscope}%
\begin{pgfscope}%
\pgfsys@transformshift{2.239782in}{0.739685in}%
\pgfsys@useobject{currentmarker}{}%
\end{pgfscope}%
\begin{pgfscope}%
\pgfsys@transformshift{2.239934in}{0.758864in}%
\pgfsys@useobject{currentmarker}{}%
\end{pgfscope}%
\begin{pgfscope}%
\pgfsys@transformshift{2.240086in}{0.713306in}%
\pgfsys@useobject{currentmarker}{}%
\end{pgfscope}%
\begin{pgfscope}%
\pgfsys@transformshift{2.240239in}{0.721244in}%
\pgfsys@useobject{currentmarker}{}%
\end{pgfscope}%
\begin{pgfscope}%
\pgfsys@transformshift{2.240391in}{0.736624in}%
\pgfsys@useobject{currentmarker}{}%
\end{pgfscope}%
\begin{pgfscope}%
\pgfsys@transformshift{2.240543in}{0.765730in}%
\pgfsys@useobject{currentmarker}{}%
\end{pgfscope}%
\begin{pgfscope}%
\pgfsys@transformshift{2.240695in}{0.748730in}%
\pgfsys@useobject{currentmarker}{}%
\end{pgfscope}%
\begin{pgfscope}%
\pgfsys@transformshift{2.240847in}{0.677540in}%
\pgfsys@useobject{currentmarker}{}%
\end{pgfscope}%
\begin{pgfscope}%
\pgfsys@transformshift{2.240998in}{0.662334in}%
\pgfsys@useobject{currentmarker}{}%
\end{pgfscope}%
\begin{pgfscope}%
\pgfsys@transformshift{2.241150in}{0.667437in}%
\pgfsys@useobject{currentmarker}{}%
\end{pgfscope}%
\begin{pgfscope}%
\pgfsys@transformshift{2.241301in}{0.621920in}%
\pgfsys@useobject{currentmarker}{}%
\end{pgfscope}%
\begin{pgfscope}%
\pgfsys@transformshift{2.241453in}{0.677563in}%
\pgfsys@useobject{currentmarker}{}%
\end{pgfscope}%
\begin{pgfscope}%
\pgfsys@transformshift{2.241604in}{0.665148in}%
\pgfsys@useobject{currentmarker}{}%
\end{pgfscope}%
\begin{pgfscope}%
\pgfsys@transformshift{2.241755in}{0.722334in}%
\pgfsys@useobject{currentmarker}{}%
\end{pgfscope}%
\begin{pgfscope}%
\pgfsys@transformshift{2.241906in}{0.712801in}%
\pgfsys@useobject{currentmarker}{}%
\end{pgfscope}%
\begin{pgfscope}%
\pgfsys@transformshift{2.242057in}{0.740972in}%
\pgfsys@useobject{currentmarker}{}%
\end{pgfscope}%
\begin{pgfscope}%
\pgfsys@transformshift{2.242208in}{0.766118in}%
\pgfsys@useobject{currentmarker}{}%
\end{pgfscope}%
\begin{pgfscope}%
\pgfsys@transformshift{2.242359in}{0.748389in}%
\pgfsys@useobject{currentmarker}{}%
\end{pgfscope}%
\begin{pgfscope}%
\pgfsys@transformshift{2.242510in}{0.726405in}%
\pgfsys@useobject{currentmarker}{}%
\end{pgfscope}%
\begin{pgfscope}%
\pgfsys@transformshift{2.242660in}{0.729936in}%
\pgfsys@useobject{currentmarker}{}%
\end{pgfscope}%
\begin{pgfscope}%
\pgfsys@transformshift{2.242811in}{0.708956in}%
\pgfsys@useobject{currentmarker}{}%
\end{pgfscope}%
\begin{pgfscope}%
\pgfsys@transformshift{2.242961in}{0.669054in}%
\pgfsys@useobject{currentmarker}{}%
\end{pgfscope}%
\begin{pgfscope}%
\pgfsys@transformshift{2.243112in}{0.661807in}%
\pgfsys@useobject{currentmarker}{}%
\end{pgfscope}%
\begin{pgfscope}%
\pgfsys@transformshift{2.243262in}{0.707696in}%
\pgfsys@useobject{currentmarker}{}%
\end{pgfscope}%
\begin{pgfscope}%
\pgfsys@transformshift{2.243412in}{0.691400in}%
\pgfsys@useobject{currentmarker}{}%
\end{pgfscope}%
\begin{pgfscope}%
\pgfsys@transformshift{2.243562in}{0.655520in}%
\pgfsys@useobject{currentmarker}{}%
\end{pgfscope}%
\begin{pgfscope}%
\pgfsys@transformshift{2.243712in}{0.677648in}%
\pgfsys@useobject{currentmarker}{}%
\end{pgfscope}%
\begin{pgfscope}%
\pgfsys@transformshift{2.243862in}{0.663219in}%
\pgfsys@useobject{currentmarker}{}%
\end{pgfscope}%
\begin{pgfscope}%
\pgfsys@transformshift{2.244011in}{0.672069in}%
\pgfsys@useobject{currentmarker}{}%
\end{pgfscope}%
\begin{pgfscope}%
\pgfsys@transformshift{2.244161in}{0.695424in}%
\pgfsys@useobject{currentmarker}{}%
\end{pgfscope}%
\begin{pgfscope}%
\pgfsys@transformshift{2.244310in}{0.739924in}%
\pgfsys@useobject{currentmarker}{}%
\end{pgfscope}%
\begin{pgfscope}%
\pgfsys@transformshift{2.244460in}{0.717892in}%
\pgfsys@useobject{currentmarker}{}%
\end{pgfscope}%
\begin{pgfscope}%
\pgfsys@transformshift{2.244609in}{0.718431in}%
\pgfsys@useobject{currentmarker}{}%
\end{pgfscope}%
\begin{pgfscope}%
\pgfsys@transformshift{2.244758in}{0.728860in}%
\pgfsys@useobject{currentmarker}{}%
\end{pgfscope}%
\begin{pgfscope}%
\pgfsys@transformshift{2.244907in}{0.716542in}%
\pgfsys@useobject{currentmarker}{}%
\end{pgfscope}%
\begin{pgfscope}%
\pgfsys@transformshift{2.245056in}{0.684243in}%
\pgfsys@useobject{currentmarker}{}%
\end{pgfscope}%
\begin{pgfscope}%
\pgfsys@transformshift{2.245205in}{0.670843in}%
\pgfsys@useobject{currentmarker}{}%
\end{pgfscope}%
\begin{pgfscope}%
\pgfsys@transformshift{2.245354in}{0.712432in}%
\pgfsys@useobject{currentmarker}{}%
\end{pgfscope}%
\begin{pgfscope}%
\pgfsys@transformshift{2.245503in}{0.692356in}%
\pgfsys@useobject{currentmarker}{}%
\end{pgfscope}%
\begin{pgfscope}%
\pgfsys@transformshift{2.245651in}{0.682883in}%
\pgfsys@useobject{currentmarker}{}%
\end{pgfscope}%
\begin{pgfscope}%
\pgfsys@transformshift{2.245800in}{0.664664in}%
\pgfsys@useobject{currentmarker}{}%
\end{pgfscope}%
\begin{pgfscope}%
\pgfsys@transformshift{2.245948in}{0.641016in}%
\pgfsys@useobject{currentmarker}{}%
\end{pgfscope}%
\begin{pgfscope}%
\pgfsys@transformshift{2.246097in}{0.684380in}%
\pgfsys@useobject{currentmarker}{}%
\end{pgfscope}%
\begin{pgfscope}%
\pgfsys@transformshift{2.246245in}{0.694844in}%
\pgfsys@useobject{currentmarker}{}%
\end{pgfscope}%
\begin{pgfscope}%
\pgfsys@transformshift{2.246393in}{0.736035in}%
\pgfsys@useobject{currentmarker}{}%
\end{pgfscope}%
\begin{pgfscope}%
\pgfsys@transformshift{2.246541in}{0.758960in}%
\pgfsys@useobject{currentmarker}{}%
\end{pgfscope}%
\begin{pgfscope}%
\pgfsys@transformshift{2.246689in}{0.712354in}%
\pgfsys@useobject{currentmarker}{}%
\end{pgfscope}%
\begin{pgfscope}%
\pgfsys@transformshift{2.246837in}{0.642552in}%
\pgfsys@useobject{currentmarker}{}%
\end{pgfscope}%
\begin{pgfscope}%
\pgfsys@transformshift{2.246984in}{0.657392in}%
\pgfsys@useobject{currentmarker}{}%
\end{pgfscope}%
\begin{pgfscope}%
\pgfsys@transformshift{2.247132in}{0.702311in}%
\pgfsys@useobject{currentmarker}{}%
\end{pgfscope}%
\begin{pgfscope}%
\pgfsys@transformshift{2.247280in}{0.692910in}%
\pgfsys@useobject{currentmarker}{}%
\end{pgfscope}%
\begin{pgfscope}%
\pgfsys@transformshift{2.247427in}{0.723326in}%
\pgfsys@useobject{currentmarker}{}%
\end{pgfscope}%
\begin{pgfscope}%
\pgfsys@transformshift{2.247574in}{0.694843in}%
\pgfsys@useobject{currentmarker}{}%
\end{pgfscope}%
\begin{pgfscope}%
\pgfsys@transformshift{2.247722in}{0.704490in}%
\pgfsys@useobject{currentmarker}{}%
\end{pgfscope}%
\begin{pgfscope}%
\pgfsys@transformshift{2.247869in}{0.749039in}%
\pgfsys@useobject{currentmarker}{}%
\end{pgfscope}%
\begin{pgfscope}%
\pgfsys@transformshift{2.248016in}{0.723607in}%
\pgfsys@useobject{currentmarker}{}%
\end{pgfscope}%
\begin{pgfscope}%
\pgfsys@transformshift{2.248163in}{0.676666in}%
\pgfsys@useobject{currentmarker}{}%
\end{pgfscope}%
\begin{pgfscope}%
\pgfsys@transformshift{2.248310in}{0.670732in}%
\pgfsys@useobject{currentmarker}{}%
\end{pgfscope}%
\begin{pgfscope}%
\pgfsys@transformshift{2.248456in}{0.681496in}%
\pgfsys@useobject{currentmarker}{}%
\end{pgfscope}%
\begin{pgfscope}%
\pgfsys@transformshift{2.248603in}{0.705240in}%
\pgfsys@useobject{currentmarker}{}%
\end{pgfscope}%
\begin{pgfscope}%
\pgfsys@transformshift{2.248750in}{0.712810in}%
\pgfsys@useobject{currentmarker}{}%
\end{pgfscope}%
\begin{pgfscope}%
\pgfsys@transformshift{2.248896in}{0.724644in}%
\pgfsys@useobject{currentmarker}{}%
\end{pgfscope}%
\begin{pgfscope}%
\pgfsys@transformshift{2.249042in}{0.663082in}%
\pgfsys@useobject{currentmarker}{}%
\end{pgfscope}%
\begin{pgfscope}%
\pgfsys@transformshift{2.249189in}{0.681529in}%
\pgfsys@useobject{currentmarker}{}%
\end{pgfscope}%
\begin{pgfscope}%
\pgfsys@transformshift{2.249335in}{0.679495in}%
\pgfsys@useobject{currentmarker}{}%
\end{pgfscope}%
\begin{pgfscope}%
\pgfsys@transformshift{2.249481in}{0.659021in}%
\pgfsys@useobject{currentmarker}{}%
\end{pgfscope}%
\begin{pgfscope}%
\pgfsys@transformshift{2.249627in}{0.676636in}%
\pgfsys@useobject{currentmarker}{}%
\end{pgfscope}%
\begin{pgfscope}%
\pgfsys@transformshift{2.249773in}{0.696665in}%
\pgfsys@useobject{currentmarker}{}%
\end{pgfscope}%
\begin{pgfscope}%
\pgfsys@transformshift{2.249919in}{0.716474in}%
\pgfsys@useobject{currentmarker}{}%
\end{pgfscope}%
\begin{pgfscope}%
\pgfsys@transformshift{2.250064in}{0.714473in}%
\pgfsys@useobject{currentmarker}{}%
\end{pgfscope}%
\begin{pgfscope}%
\pgfsys@transformshift{2.250210in}{0.725436in}%
\pgfsys@useobject{currentmarker}{}%
\end{pgfscope}%
\begin{pgfscope}%
\pgfsys@transformshift{2.250356in}{0.711804in}%
\pgfsys@useobject{currentmarker}{}%
\end{pgfscope}%
\begin{pgfscope}%
\pgfsys@transformshift{2.250501in}{0.677398in}%
\pgfsys@useobject{currentmarker}{}%
\end{pgfscope}%
\begin{pgfscope}%
\pgfsys@transformshift{2.250646in}{0.706218in}%
\pgfsys@useobject{currentmarker}{}%
\end{pgfscope}%
\begin{pgfscope}%
\pgfsys@transformshift{2.250792in}{0.716252in}%
\pgfsys@useobject{currentmarker}{}%
\end{pgfscope}%
\begin{pgfscope}%
\pgfsys@transformshift{2.250937in}{0.622602in}%
\pgfsys@useobject{currentmarker}{}%
\end{pgfscope}%
\begin{pgfscope}%
\pgfsys@transformshift{2.251082in}{0.641215in}%
\pgfsys@useobject{currentmarker}{}%
\end{pgfscope}%
\begin{pgfscope}%
\pgfsys@transformshift{2.251227in}{0.662960in}%
\pgfsys@useobject{currentmarker}{}%
\end{pgfscope}%
\begin{pgfscope}%
\pgfsys@transformshift{2.251372in}{0.735113in}%
\pgfsys@useobject{currentmarker}{}%
\end{pgfscope}%
\begin{pgfscope}%
\pgfsys@transformshift{2.251516in}{0.735665in}%
\pgfsys@useobject{currentmarker}{}%
\end{pgfscope}%
\begin{pgfscope}%
\pgfsys@transformshift{2.251661in}{0.708749in}%
\pgfsys@useobject{currentmarker}{}%
\end{pgfscope}%
\begin{pgfscope}%
\pgfsys@transformshift{2.251806in}{0.679063in}%
\pgfsys@useobject{currentmarker}{}%
\end{pgfscope}%
\begin{pgfscope}%
\pgfsys@transformshift{2.251950in}{0.714772in}%
\pgfsys@useobject{currentmarker}{}%
\end{pgfscope}%
\begin{pgfscope}%
\pgfsys@transformshift{2.252095in}{0.726715in}%
\pgfsys@useobject{currentmarker}{}%
\end{pgfscope}%
\begin{pgfscope}%
\pgfsys@transformshift{2.252239in}{0.705032in}%
\pgfsys@useobject{currentmarker}{}%
\end{pgfscope}%
\begin{pgfscope}%
\pgfsys@transformshift{2.252383in}{0.704510in}%
\pgfsys@useobject{currentmarker}{}%
\end{pgfscope}%
\begin{pgfscope}%
\pgfsys@transformshift{2.252527in}{0.718843in}%
\pgfsys@useobject{currentmarker}{}%
\end{pgfscope}%
\begin{pgfscope}%
\pgfsys@transformshift{2.252671in}{0.732880in}%
\pgfsys@useobject{currentmarker}{}%
\end{pgfscope}%
\begin{pgfscope}%
\pgfsys@transformshift{2.252815in}{0.721123in}%
\pgfsys@useobject{currentmarker}{}%
\end{pgfscope}%
\begin{pgfscope}%
\pgfsys@transformshift{2.252959in}{0.737019in}%
\pgfsys@useobject{currentmarker}{}%
\end{pgfscope}%
\begin{pgfscope}%
\pgfsys@transformshift{2.253103in}{0.735847in}%
\pgfsys@useobject{currentmarker}{}%
\end{pgfscope}%
\begin{pgfscope}%
\pgfsys@transformshift{2.253246in}{0.683591in}%
\pgfsys@useobject{currentmarker}{}%
\end{pgfscope}%
\begin{pgfscope}%
\pgfsys@transformshift{2.253390in}{0.680078in}%
\pgfsys@useobject{currentmarker}{}%
\end{pgfscope}%
\begin{pgfscope}%
\pgfsys@transformshift{2.253533in}{0.663719in}%
\pgfsys@useobject{currentmarker}{}%
\end{pgfscope}%
\begin{pgfscope}%
\pgfsys@transformshift{2.253677in}{0.702132in}%
\pgfsys@useobject{currentmarker}{}%
\end{pgfscope}%
\begin{pgfscope}%
\pgfsys@transformshift{2.253820in}{0.689677in}%
\pgfsys@useobject{currentmarker}{}%
\end{pgfscope}%
\begin{pgfscope}%
\pgfsys@transformshift{2.253963in}{0.702329in}%
\pgfsys@useobject{currentmarker}{}%
\end{pgfscope}%
\begin{pgfscope}%
\pgfsys@transformshift{2.254106in}{0.721144in}%
\pgfsys@useobject{currentmarker}{}%
\end{pgfscope}%
\begin{pgfscope}%
\pgfsys@transformshift{2.254250in}{0.686102in}%
\pgfsys@useobject{currentmarker}{}%
\end{pgfscope}%
\begin{pgfscope}%
\pgfsys@transformshift{2.254392in}{0.718778in}%
\pgfsys@useobject{currentmarker}{}%
\end{pgfscope}%
\begin{pgfscope}%
\pgfsys@transformshift{2.254535in}{0.750481in}%
\pgfsys@useobject{currentmarker}{}%
\end{pgfscope}%
\begin{pgfscope}%
\pgfsys@transformshift{2.254678in}{0.720000in}%
\pgfsys@useobject{currentmarker}{}%
\end{pgfscope}%
\begin{pgfscope}%
\pgfsys@transformshift{2.254821in}{0.688399in}%
\pgfsys@useobject{currentmarker}{}%
\end{pgfscope}%
\begin{pgfscope}%
\pgfsys@transformshift{2.254963in}{0.713206in}%
\pgfsys@useobject{currentmarker}{}%
\end{pgfscope}%
\begin{pgfscope}%
\pgfsys@transformshift{2.255106in}{0.705462in}%
\pgfsys@useobject{currentmarker}{}%
\end{pgfscope}%
\begin{pgfscope}%
\pgfsys@transformshift{2.255248in}{0.680761in}%
\pgfsys@useobject{currentmarker}{}%
\end{pgfscope}%
\begin{pgfscope}%
\pgfsys@transformshift{2.255390in}{0.666280in}%
\pgfsys@useobject{currentmarker}{}%
\end{pgfscope}%
\begin{pgfscope}%
\pgfsys@transformshift{2.255533in}{0.708373in}%
\pgfsys@useobject{currentmarker}{}%
\end{pgfscope}%
\begin{pgfscope}%
\pgfsys@transformshift{2.255675in}{0.740621in}%
\pgfsys@useobject{currentmarker}{}%
\end{pgfscope}%
\begin{pgfscope}%
\pgfsys@transformshift{2.255817in}{0.705022in}%
\pgfsys@useobject{currentmarker}{}%
\end{pgfscope}%
\begin{pgfscope}%
\pgfsys@transformshift{2.255959in}{0.698443in}%
\pgfsys@useobject{currentmarker}{}%
\end{pgfscope}%
\begin{pgfscope}%
\pgfsys@transformshift{2.256101in}{0.712294in}%
\pgfsys@useobject{currentmarker}{}%
\end{pgfscope}%
\begin{pgfscope}%
\pgfsys@transformshift{2.256242in}{0.715503in}%
\pgfsys@useobject{currentmarker}{}%
\end{pgfscope}%
\begin{pgfscope}%
\pgfsys@transformshift{2.256384in}{0.689414in}%
\pgfsys@useobject{currentmarker}{}%
\end{pgfscope}%
\begin{pgfscope}%
\pgfsys@transformshift{2.256526in}{0.726545in}%
\pgfsys@useobject{currentmarker}{}%
\end{pgfscope}%
\begin{pgfscope}%
\pgfsys@transformshift{2.256667in}{0.707368in}%
\pgfsys@useobject{currentmarker}{}%
\end{pgfscope}%
\begin{pgfscope}%
\pgfsys@transformshift{2.256809in}{0.725852in}%
\pgfsys@useobject{currentmarker}{}%
\end{pgfscope}%
\begin{pgfscope}%
\pgfsys@transformshift{2.256950in}{0.680940in}%
\pgfsys@useobject{currentmarker}{}%
\end{pgfscope}%
\begin{pgfscope}%
\pgfsys@transformshift{2.257091in}{0.688226in}%
\pgfsys@useobject{currentmarker}{}%
\end{pgfscope}%
\begin{pgfscope}%
\pgfsys@transformshift{2.257232in}{0.681984in}%
\pgfsys@useobject{currentmarker}{}%
\end{pgfscope}%
\begin{pgfscope}%
\pgfsys@transformshift{2.257373in}{0.657153in}%
\pgfsys@useobject{currentmarker}{}%
\end{pgfscope}%
\begin{pgfscope}%
\pgfsys@transformshift{2.257514in}{0.716045in}%
\pgfsys@useobject{currentmarker}{}%
\end{pgfscope}%
\begin{pgfscope}%
\pgfsys@transformshift{2.257655in}{0.680681in}%
\pgfsys@useobject{currentmarker}{}%
\end{pgfscope}%
\begin{pgfscope}%
\pgfsys@transformshift{2.257796in}{0.680277in}%
\pgfsys@useobject{currentmarker}{}%
\end{pgfscope}%
\begin{pgfscope}%
\pgfsys@transformshift{2.257937in}{0.722488in}%
\pgfsys@useobject{currentmarker}{}%
\end{pgfscope}%
\begin{pgfscope}%
\pgfsys@transformshift{2.258077in}{0.723869in}%
\pgfsys@useobject{currentmarker}{}%
\end{pgfscope}%
\begin{pgfscope}%
\pgfsys@transformshift{2.258218in}{0.698837in}%
\pgfsys@useobject{currentmarker}{}%
\end{pgfscope}%
\begin{pgfscope}%
\pgfsys@transformshift{2.258358in}{0.662739in}%
\pgfsys@useobject{currentmarker}{}%
\end{pgfscope}%
\begin{pgfscope}%
\pgfsys@transformshift{2.258498in}{0.647127in}%
\pgfsys@useobject{currentmarker}{}%
\end{pgfscope}%
\begin{pgfscope}%
\pgfsys@transformshift{2.258639in}{0.672734in}%
\pgfsys@useobject{currentmarker}{}%
\end{pgfscope}%
\begin{pgfscope}%
\pgfsys@transformshift{2.258779in}{0.669575in}%
\pgfsys@useobject{currentmarker}{}%
\end{pgfscope}%
\begin{pgfscope}%
\pgfsys@transformshift{2.258919in}{0.661187in}%
\pgfsys@useobject{currentmarker}{}%
\end{pgfscope}%
\begin{pgfscope}%
\pgfsys@transformshift{2.259059in}{0.701223in}%
\pgfsys@useobject{currentmarker}{}%
\end{pgfscope}%
\begin{pgfscope}%
\pgfsys@transformshift{2.259199in}{0.654386in}%
\pgfsys@useobject{currentmarker}{}%
\end{pgfscope}%
\begin{pgfscope}%
\pgfsys@transformshift{2.259339in}{0.616757in}%
\pgfsys@useobject{currentmarker}{}%
\end{pgfscope}%
\begin{pgfscope}%
\pgfsys@transformshift{2.259478in}{0.717679in}%
\pgfsys@useobject{currentmarker}{}%
\end{pgfscope}%
\begin{pgfscope}%
\pgfsys@transformshift{2.259618in}{0.734439in}%
\pgfsys@useobject{currentmarker}{}%
\end{pgfscope}%
\begin{pgfscope}%
\pgfsys@transformshift{2.259758in}{0.718237in}%
\pgfsys@useobject{currentmarker}{}%
\end{pgfscope}%
\begin{pgfscope}%
\pgfsys@transformshift{2.259897in}{0.754042in}%
\pgfsys@useobject{currentmarker}{}%
\end{pgfscope}%
\begin{pgfscope}%
\pgfsys@transformshift{2.260037in}{0.727988in}%
\pgfsys@useobject{currentmarker}{}%
\end{pgfscope}%
\begin{pgfscope}%
\pgfsys@transformshift{2.260176in}{0.702001in}%
\pgfsys@useobject{currentmarker}{}%
\end{pgfscope}%
\begin{pgfscope}%
\pgfsys@transformshift{2.260315in}{0.654584in}%
\pgfsys@useobject{currentmarker}{}%
\end{pgfscope}%
\begin{pgfscope}%
\pgfsys@transformshift{2.260454in}{0.656653in}%
\pgfsys@useobject{currentmarker}{}%
\end{pgfscope}%
\begin{pgfscope}%
\pgfsys@transformshift{2.260593in}{0.688915in}%
\pgfsys@useobject{currentmarker}{}%
\end{pgfscope}%
\begin{pgfscope}%
\pgfsys@transformshift{2.260732in}{0.694782in}%
\pgfsys@useobject{currentmarker}{}%
\end{pgfscope}%
\begin{pgfscope}%
\pgfsys@transformshift{2.260871in}{0.685468in}%
\pgfsys@useobject{currentmarker}{}%
\end{pgfscope}%
\begin{pgfscope}%
\pgfsys@transformshift{2.261010in}{0.715285in}%
\pgfsys@useobject{currentmarker}{}%
\end{pgfscope}%
\begin{pgfscope}%
\pgfsys@transformshift{2.261149in}{0.759264in}%
\pgfsys@useobject{currentmarker}{}%
\end{pgfscope}%
\begin{pgfscope}%
\pgfsys@transformshift{2.261287in}{0.714031in}%
\pgfsys@useobject{currentmarker}{}%
\end{pgfscope}%
\begin{pgfscope}%
\pgfsys@transformshift{2.261426in}{0.742826in}%
\pgfsys@useobject{currentmarker}{}%
\end{pgfscope}%
\begin{pgfscope}%
\pgfsys@transformshift{2.261564in}{0.745015in}%
\pgfsys@useobject{currentmarker}{}%
\end{pgfscope}%
\begin{pgfscope}%
\pgfsys@transformshift{2.261703in}{0.708678in}%
\pgfsys@useobject{currentmarker}{}%
\end{pgfscope}%
\begin{pgfscope}%
\pgfsys@transformshift{2.261841in}{0.665939in}%
\pgfsys@useobject{currentmarker}{}%
\end{pgfscope}%
\begin{pgfscope}%
\pgfsys@transformshift{2.261979in}{0.719158in}%
\pgfsys@useobject{currentmarker}{}%
\end{pgfscope}%
\begin{pgfscope}%
\pgfsys@transformshift{2.262117in}{0.752129in}%
\pgfsys@useobject{currentmarker}{}%
\end{pgfscope}%
\begin{pgfscope}%
\pgfsys@transformshift{2.262255in}{0.727061in}%
\pgfsys@useobject{currentmarker}{}%
\end{pgfscope}%
\begin{pgfscope}%
\pgfsys@transformshift{2.262393in}{0.706079in}%
\pgfsys@useobject{currentmarker}{}%
\end{pgfscope}%
\begin{pgfscope}%
\pgfsys@transformshift{2.262531in}{0.730655in}%
\pgfsys@useobject{currentmarker}{}%
\end{pgfscope}%
\begin{pgfscope}%
\pgfsys@transformshift{2.262669in}{0.714559in}%
\pgfsys@useobject{currentmarker}{}%
\end{pgfscope}%
\begin{pgfscope}%
\pgfsys@transformshift{2.262806in}{0.661295in}%
\pgfsys@useobject{currentmarker}{}%
\end{pgfscope}%
\begin{pgfscope}%
\pgfsys@transformshift{2.262944in}{0.708267in}%
\pgfsys@useobject{currentmarker}{}%
\end{pgfscope}%
\begin{pgfscope}%
\pgfsys@transformshift{2.263081in}{0.699828in}%
\pgfsys@useobject{currentmarker}{}%
\end{pgfscope}%
\begin{pgfscope}%
\pgfsys@transformshift{2.263219in}{0.686382in}%
\pgfsys@useobject{currentmarker}{}%
\end{pgfscope}%
\begin{pgfscope}%
\pgfsys@transformshift{2.263356in}{0.681413in}%
\pgfsys@useobject{currentmarker}{}%
\end{pgfscope}%
\begin{pgfscope}%
\pgfsys@transformshift{2.263493in}{0.708833in}%
\pgfsys@useobject{currentmarker}{}%
\end{pgfscope}%
\begin{pgfscope}%
\pgfsys@transformshift{2.263631in}{0.690348in}%
\pgfsys@useobject{currentmarker}{}%
\end{pgfscope}%
\begin{pgfscope}%
\pgfsys@transformshift{2.263768in}{0.663298in}%
\pgfsys@useobject{currentmarker}{}%
\end{pgfscope}%
\begin{pgfscope}%
\pgfsys@transformshift{2.263905in}{0.731098in}%
\pgfsys@useobject{currentmarker}{}%
\end{pgfscope}%
\begin{pgfscope}%
\pgfsys@transformshift{2.264042in}{0.719342in}%
\pgfsys@useobject{currentmarker}{}%
\end{pgfscope}%
\begin{pgfscope}%
\pgfsys@transformshift{2.264179in}{0.723377in}%
\pgfsys@useobject{currentmarker}{}%
\end{pgfscope}%
\begin{pgfscope}%
\pgfsys@transformshift{2.264315in}{0.707965in}%
\pgfsys@useobject{currentmarker}{}%
\end{pgfscope}%
\begin{pgfscope}%
\pgfsys@transformshift{2.264452in}{0.683720in}%
\pgfsys@useobject{currentmarker}{}%
\end{pgfscope}%
\begin{pgfscope}%
\pgfsys@transformshift{2.264589in}{0.684484in}%
\pgfsys@useobject{currentmarker}{}%
\end{pgfscope}%
\begin{pgfscope}%
\pgfsys@transformshift{2.264725in}{0.651951in}%
\pgfsys@useobject{currentmarker}{}%
\end{pgfscope}%
\begin{pgfscope}%
\pgfsys@transformshift{2.264861in}{0.621098in}%
\pgfsys@useobject{currentmarker}{}%
\end{pgfscope}%
\begin{pgfscope}%
\pgfsys@transformshift{2.264998in}{0.700192in}%
\pgfsys@useobject{currentmarker}{}%
\end{pgfscope}%
\begin{pgfscope}%
\pgfsys@transformshift{2.265134in}{0.672906in}%
\pgfsys@useobject{currentmarker}{}%
\end{pgfscope}%
\begin{pgfscope}%
\pgfsys@transformshift{2.265270in}{0.680614in}%
\pgfsys@useobject{currentmarker}{}%
\end{pgfscope}%
\begin{pgfscope}%
\pgfsys@transformshift{2.265406in}{0.749180in}%
\pgfsys@useobject{currentmarker}{}%
\end{pgfscope}%
\begin{pgfscope}%
\pgfsys@transformshift{2.265542in}{0.760705in}%
\pgfsys@useobject{currentmarker}{}%
\end{pgfscope}%
\begin{pgfscope}%
\pgfsys@transformshift{2.265678in}{0.749964in}%
\pgfsys@useobject{currentmarker}{}%
\end{pgfscope}%
\begin{pgfscope}%
\pgfsys@transformshift{2.265814in}{0.705377in}%
\pgfsys@useobject{currentmarker}{}%
\end{pgfscope}%
\begin{pgfscope}%
\pgfsys@transformshift{2.265950in}{0.742119in}%
\pgfsys@useobject{currentmarker}{}%
\end{pgfscope}%
\begin{pgfscope}%
\pgfsys@transformshift{2.266086in}{0.736949in}%
\pgfsys@useobject{currentmarker}{}%
\end{pgfscope}%
\begin{pgfscope}%
\pgfsys@transformshift{2.266221in}{0.639973in}%
\pgfsys@useobject{currentmarker}{}%
\end{pgfscope}%
\begin{pgfscope}%
\pgfsys@transformshift{2.266357in}{0.684384in}%
\pgfsys@useobject{currentmarker}{}%
\end{pgfscope}%
\begin{pgfscope}%
\pgfsys@transformshift{2.266492in}{0.732813in}%
\pgfsys@useobject{currentmarker}{}%
\end{pgfscope}%
\begin{pgfscope}%
\pgfsys@transformshift{2.266628in}{0.697220in}%
\pgfsys@useobject{currentmarker}{}%
\end{pgfscope}%
\begin{pgfscope}%
\pgfsys@transformshift{2.266763in}{0.675731in}%
\pgfsys@useobject{currentmarker}{}%
\end{pgfscope}%
\begin{pgfscope}%
\pgfsys@transformshift{2.266898in}{0.685796in}%
\pgfsys@useobject{currentmarker}{}%
\end{pgfscope}%
\begin{pgfscope}%
\pgfsys@transformshift{2.267033in}{0.704138in}%
\pgfsys@useobject{currentmarker}{}%
\end{pgfscope}%
\begin{pgfscope}%
\pgfsys@transformshift{2.267168in}{0.698292in}%
\pgfsys@useobject{currentmarker}{}%
\end{pgfscope}%
\begin{pgfscope}%
\pgfsys@transformshift{2.267303in}{0.704329in}%
\pgfsys@useobject{currentmarker}{}%
\end{pgfscope}%
\begin{pgfscope}%
\pgfsys@transformshift{2.267438in}{0.717765in}%
\pgfsys@useobject{currentmarker}{}%
\end{pgfscope}%
\begin{pgfscope}%
\pgfsys@transformshift{2.267573in}{0.745543in}%
\pgfsys@useobject{currentmarker}{}%
\end{pgfscope}%
\begin{pgfscope}%
\pgfsys@transformshift{2.267707in}{0.730506in}%
\pgfsys@useobject{currentmarker}{}%
\end{pgfscope}%
\begin{pgfscope}%
\pgfsys@transformshift{2.267842in}{0.732449in}%
\pgfsys@useobject{currentmarker}{}%
\end{pgfscope}%
\begin{pgfscope}%
\pgfsys@transformshift{2.267977in}{0.724661in}%
\pgfsys@useobject{currentmarker}{}%
\end{pgfscope}%
\begin{pgfscope}%
\pgfsys@transformshift{2.268111in}{0.731673in}%
\pgfsys@useobject{currentmarker}{}%
\end{pgfscope}%
\begin{pgfscope}%
\pgfsys@transformshift{2.268246in}{0.717570in}%
\pgfsys@useobject{currentmarker}{}%
\end{pgfscope}%
\begin{pgfscope}%
\pgfsys@transformshift{2.268380in}{0.673013in}%
\pgfsys@useobject{currentmarker}{}%
\end{pgfscope}%
\begin{pgfscope}%
\pgfsys@transformshift{2.268514in}{0.649266in}%
\pgfsys@useobject{currentmarker}{}%
\end{pgfscope}%
\begin{pgfscope}%
\pgfsys@transformshift{2.268648in}{0.703331in}%
\pgfsys@useobject{currentmarker}{}%
\end{pgfscope}%
\begin{pgfscope}%
\pgfsys@transformshift{2.268782in}{0.732532in}%
\pgfsys@useobject{currentmarker}{}%
\end{pgfscope}%
\begin{pgfscope}%
\pgfsys@transformshift{2.268916in}{0.635977in}%
\pgfsys@useobject{currentmarker}{}%
\end{pgfscope}%
\begin{pgfscope}%
\pgfsys@transformshift{2.269050in}{0.650731in}%
\pgfsys@useobject{currentmarker}{}%
\end{pgfscope}%
\begin{pgfscope}%
\pgfsys@transformshift{2.269184in}{0.725812in}%
\pgfsys@useobject{currentmarker}{}%
\end{pgfscope}%
\begin{pgfscope}%
\pgfsys@transformshift{2.269318in}{0.697405in}%
\pgfsys@useobject{currentmarker}{}%
\end{pgfscope}%
\begin{pgfscope}%
\pgfsys@transformshift{2.269451in}{0.723162in}%
\pgfsys@useobject{currentmarker}{}%
\end{pgfscope}%
\begin{pgfscope}%
\pgfsys@transformshift{2.269585in}{0.732974in}%
\pgfsys@useobject{currentmarker}{}%
\end{pgfscope}%
\begin{pgfscope}%
\pgfsys@transformshift{2.269719in}{0.712306in}%
\pgfsys@useobject{currentmarker}{}%
\end{pgfscope}%
\begin{pgfscope}%
\pgfsys@transformshift{2.269852in}{0.684522in}%
\pgfsys@useobject{currentmarker}{}%
\end{pgfscope}%
\begin{pgfscope}%
\pgfsys@transformshift{2.269985in}{0.658479in}%
\pgfsys@useobject{currentmarker}{}%
\end{pgfscope}%
\begin{pgfscope}%
\pgfsys@transformshift{2.270119in}{0.745817in}%
\pgfsys@useobject{currentmarker}{}%
\end{pgfscope}%
\begin{pgfscope}%
\pgfsys@transformshift{2.270252in}{0.765105in}%
\pgfsys@useobject{currentmarker}{}%
\end{pgfscope}%
\begin{pgfscope}%
\pgfsys@transformshift{2.270385in}{0.710887in}%
\pgfsys@useobject{currentmarker}{}%
\end{pgfscope}%
\begin{pgfscope}%
\pgfsys@transformshift{2.270518in}{0.693639in}%
\pgfsys@useobject{currentmarker}{}%
\end{pgfscope}%
\begin{pgfscope}%
\pgfsys@transformshift{2.270651in}{0.669314in}%
\pgfsys@useobject{currentmarker}{}%
\end{pgfscope}%
\begin{pgfscope}%
\pgfsys@transformshift{2.270784in}{0.703615in}%
\pgfsys@useobject{currentmarker}{}%
\end{pgfscope}%
\begin{pgfscope}%
\pgfsys@transformshift{2.270917in}{0.731494in}%
\pgfsys@useobject{currentmarker}{}%
\end{pgfscope}%
\begin{pgfscope}%
\pgfsys@transformshift{2.271049in}{0.646919in}%
\pgfsys@useobject{currentmarker}{}%
\end{pgfscope}%
\begin{pgfscope}%
\pgfsys@transformshift{2.271182in}{0.705548in}%
\pgfsys@useobject{currentmarker}{}%
\end{pgfscope}%
\begin{pgfscope}%
\pgfsys@transformshift{2.271314in}{0.713062in}%
\pgfsys@useobject{currentmarker}{}%
\end{pgfscope}%
\begin{pgfscope}%
\pgfsys@transformshift{2.271447in}{0.654034in}%
\pgfsys@useobject{currentmarker}{}%
\end{pgfscope}%
\begin{pgfscope}%
\pgfsys@transformshift{2.271579in}{0.691041in}%
\pgfsys@useobject{currentmarker}{}%
\end{pgfscope}%
\begin{pgfscope}%
\pgfsys@transformshift{2.271712in}{0.708914in}%
\pgfsys@useobject{currentmarker}{}%
\end{pgfscope}%
\begin{pgfscope}%
\pgfsys@transformshift{2.271844in}{0.738235in}%
\pgfsys@useobject{currentmarker}{}%
\end{pgfscope}%
\begin{pgfscope}%
\pgfsys@transformshift{2.271976in}{0.738445in}%
\pgfsys@useobject{currentmarker}{}%
\end{pgfscope}%
\begin{pgfscope}%
\pgfsys@transformshift{2.272108in}{0.667373in}%
\pgfsys@useobject{currentmarker}{}%
\end{pgfscope}%
\begin{pgfscope}%
\pgfsys@transformshift{2.272240in}{0.692112in}%
\pgfsys@useobject{currentmarker}{}%
\end{pgfscope}%
\begin{pgfscope}%
\pgfsys@transformshift{2.272372in}{0.735193in}%
\pgfsys@useobject{currentmarker}{}%
\end{pgfscope}%
\begin{pgfscope}%
\pgfsys@transformshift{2.272504in}{0.714548in}%
\pgfsys@useobject{currentmarker}{}%
\end{pgfscope}%
\begin{pgfscope}%
\pgfsys@transformshift{2.272636in}{0.715735in}%
\pgfsys@useobject{currentmarker}{}%
\end{pgfscope}%
\begin{pgfscope}%
\pgfsys@transformshift{2.272768in}{0.723007in}%
\pgfsys@useobject{currentmarker}{}%
\end{pgfscope}%
\begin{pgfscope}%
\pgfsys@transformshift{2.272899in}{0.717698in}%
\pgfsys@useobject{currentmarker}{}%
\end{pgfscope}%
\begin{pgfscope}%
\pgfsys@transformshift{2.273031in}{0.727744in}%
\pgfsys@useobject{currentmarker}{}%
\end{pgfscope}%
\begin{pgfscope}%
\pgfsys@transformshift{2.273162in}{0.717657in}%
\pgfsys@useobject{currentmarker}{}%
\end{pgfscope}%
\begin{pgfscope}%
\pgfsys@transformshift{2.273294in}{0.660881in}%
\pgfsys@useobject{currentmarker}{}%
\end{pgfscope}%
\begin{pgfscope}%
\pgfsys@transformshift{2.273425in}{0.649536in}%
\pgfsys@useobject{currentmarker}{}%
\end{pgfscope}%
\begin{pgfscope}%
\pgfsys@transformshift{2.273556in}{0.668551in}%
\pgfsys@useobject{currentmarker}{}%
\end{pgfscope}%
\begin{pgfscope}%
\pgfsys@transformshift{2.273687in}{0.692001in}%
\pgfsys@useobject{currentmarker}{}%
\end{pgfscope}%
\begin{pgfscope}%
\pgfsys@transformshift{2.273818in}{0.663089in}%
\pgfsys@useobject{currentmarker}{}%
\end{pgfscope}%
\begin{pgfscope}%
\pgfsys@transformshift{2.273950in}{0.663322in}%
\pgfsys@useobject{currentmarker}{}%
\end{pgfscope}%
\begin{pgfscope}%
\pgfsys@transformshift{2.274080in}{0.688876in}%
\pgfsys@useobject{currentmarker}{}%
\end{pgfscope}%
\begin{pgfscope}%
\pgfsys@transformshift{2.274211in}{0.723455in}%
\pgfsys@useobject{currentmarker}{}%
\end{pgfscope}%
\begin{pgfscope}%
\pgfsys@transformshift{2.274342in}{0.706596in}%
\pgfsys@useobject{currentmarker}{}%
\end{pgfscope}%
\begin{pgfscope}%
\pgfsys@transformshift{2.274473in}{0.668972in}%
\pgfsys@useobject{currentmarker}{}%
\end{pgfscope}%
\begin{pgfscope}%
\pgfsys@transformshift{2.274603in}{0.711011in}%
\pgfsys@useobject{currentmarker}{}%
\end{pgfscope}%
\begin{pgfscope}%
\pgfsys@transformshift{2.274734in}{0.687775in}%
\pgfsys@useobject{currentmarker}{}%
\end{pgfscope}%
\begin{pgfscope}%
\pgfsys@transformshift{2.274865in}{0.713213in}%
\pgfsys@useobject{currentmarker}{}%
\end{pgfscope}%
\begin{pgfscope}%
\pgfsys@transformshift{2.274995in}{0.711552in}%
\pgfsys@useobject{currentmarker}{}%
\end{pgfscope}%
\begin{pgfscope}%
\pgfsys@transformshift{2.275125in}{0.702659in}%
\pgfsys@useobject{currentmarker}{}%
\end{pgfscope}%
\begin{pgfscope}%
\pgfsys@transformshift{2.275256in}{0.667022in}%
\pgfsys@useobject{currentmarker}{}%
\end{pgfscope}%
\begin{pgfscope}%
\pgfsys@transformshift{2.275386in}{0.723076in}%
\pgfsys@useobject{currentmarker}{}%
\end{pgfscope}%
\begin{pgfscope}%
\pgfsys@transformshift{2.275516in}{0.738531in}%
\pgfsys@useobject{currentmarker}{}%
\end{pgfscope}%
\begin{pgfscope}%
\pgfsys@transformshift{2.275646in}{0.680963in}%
\pgfsys@useobject{currentmarker}{}%
\end{pgfscope}%
\begin{pgfscope}%
\pgfsys@transformshift{2.275776in}{0.686357in}%
\pgfsys@useobject{currentmarker}{}%
\end{pgfscope}%
\begin{pgfscope}%
\pgfsys@transformshift{2.275906in}{0.688201in}%
\pgfsys@useobject{currentmarker}{}%
\end{pgfscope}%
\begin{pgfscope}%
\pgfsys@transformshift{2.276036in}{0.657443in}%
\pgfsys@useobject{currentmarker}{}%
\end{pgfscope}%
\begin{pgfscope}%
\pgfsys@transformshift{2.276165in}{0.668430in}%
\pgfsys@useobject{currentmarker}{}%
\end{pgfscope}%
\begin{pgfscope}%
\pgfsys@transformshift{2.276295in}{0.700388in}%
\pgfsys@useobject{currentmarker}{}%
\end{pgfscope}%
\begin{pgfscope}%
\pgfsys@transformshift{2.276424in}{0.768301in}%
\pgfsys@useobject{currentmarker}{}%
\end{pgfscope}%
\begin{pgfscope}%
\pgfsys@transformshift{2.276554in}{0.733738in}%
\pgfsys@useobject{currentmarker}{}%
\end{pgfscope}%
\begin{pgfscope}%
\pgfsys@transformshift{2.276683in}{0.646094in}%
\pgfsys@useobject{currentmarker}{}%
\end{pgfscope}%
\begin{pgfscope}%
\pgfsys@transformshift{2.276813in}{0.649977in}%
\pgfsys@useobject{currentmarker}{}%
\end{pgfscope}%
\begin{pgfscope}%
\pgfsys@transformshift{2.276942in}{0.682967in}%
\pgfsys@useobject{currentmarker}{}%
\end{pgfscope}%
\begin{pgfscope}%
\pgfsys@transformshift{2.277071in}{0.704185in}%
\pgfsys@useobject{currentmarker}{}%
\end{pgfscope}%
\begin{pgfscope}%
\pgfsys@transformshift{2.277200in}{0.691691in}%
\pgfsys@useobject{currentmarker}{}%
\end{pgfscope}%
\begin{pgfscope}%
\pgfsys@transformshift{2.277329in}{0.622419in}%
\pgfsys@useobject{currentmarker}{}%
\end{pgfscope}%
\begin{pgfscope}%
\pgfsys@transformshift{2.277458in}{0.728093in}%
\pgfsys@useobject{currentmarker}{}%
\end{pgfscope}%
\begin{pgfscope}%
\pgfsys@transformshift{2.277587in}{0.786377in}%
\pgfsys@useobject{currentmarker}{}%
\end{pgfscope}%
\begin{pgfscope}%
\pgfsys@transformshift{2.277716in}{0.790479in}%
\pgfsys@useobject{currentmarker}{}%
\end{pgfscope}%
\begin{pgfscope}%
\pgfsys@transformshift{2.277845in}{0.714713in}%
\pgfsys@useobject{currentmarker}{}%
\end{pgfscope}%
\begin{pgfscope}%
\pgfsys@transformshift{2.277974in}{0.640945in}%
\pgfsys@useobject{currentmarker}{}%
\end{pgfscope}%
\begin{pgfscope}%
\pgfsys@transformshift{2.278102in}{0.674513in}%
\pgfsys@useobject{currentmarker}{}%
\end{pgfscope}%
\begin{pgfscope}%
\pgfsys@transformshift{2.278231in}{0.685379in}%
\pgfsys@useobject{currentmarker}{}%
\end{pgfscope}%
\begin{pgfscope}%
\pgfsys@transformshift{2.278359in}{0.684238in}%
\pgfsys@useobject{currentmarker}{}%
\end{pgfscope}%
\begin{pgfscope}%
\pgfsys@transformshift{2.278488in}{0.698864in}%
\pgfsys@useobject{currentmarker}{}%
\end{pgfscope}%
\begin{pgfscope}%
\pgfsys@transformshift{2.278616in}{0.714252in}%
\pgfsys@useobject{currentmarker}{}%
\end{pgfscope}%
\begin{pgfscope}%
\pgfsys@transformshift{2.278744in}{0.732854in}%
\pgfsys@useobject{currentmarker}{}%
\end{pgfscope}%
\begin{pgfscope}%
\pgfsys@transformshift{2.278872in}{0.669672in}%
\pgfsys@useobject{currentmarker}{}%
\end{pgfscope}%
\begin{pgfscope}%
\pgfsys@transformshift{2.279001in}{0.680015in}%
\pgfsys@useobject{currentmarker}{}%
\end{pgfscope}%
\begin{pgfscope}%
\pgfsys@transformshift{2.279129in}{0.704930in}%
\pgfsys@useobject{currentmarker}{}%
\end{pgfscope}%
\begin{pgfscope}%
\pgfsys@transformshift{2.279257in}{0.720831in}%
\pgfsys@useobject{currentmarker}{}%
\end{pgfscope}%
\begin{pgfscope}%
\pgfsys@transformshift{2.279384in}{0.702083in}%
\pgfsys@useobject{currentmarker}{}%
\end{pgfscope}%
\begin{pgfscope}%
\pgfsys@transformshift{2.279512in}{0.682626in}%
\pgfsys@useobject{currentmarker}{}%
\end{pgfscope}%
\begin{pgfscope}%
\pgfsys@transformshift{2.279640in}{0.704669in}%
\pgfsys@useobject{currentmarker}{}%
\end{pgfscope}%
\begin{pgfscope}%
\pgfsys@transformshift{2.279768in}{0.719532in}%
\pgfsys@useobject{currentmarker}{}%
\end{pgfscope}%
\begin{pgfscope}%
\pgfsys@transformshift{2.279895in}{0.722956in}%
\pgfsys@useobject{currentmarker}{}%
\end{pgfscope}%
\begin{pgfscope}%
\pgfsys@transformshift{2.280023in}{0.675440in}%
\pgfsys@useobject{currentmarker}{}%
\end{pgfscope}%
\begin{pgfscope}%
\pgfsys@transformshift{2.280150in}{0.660190in}%
\pgfsys@useobject{currentmarker}{}%
\end{pgfscope}%
\begin{pgfscope}%
\pgfsys@transformshift{2.280278in}{0.694435in}%
\pgfsys@useobject{currentmarker}{}%
\end{pgfscope}%
\begin{pgfscope}%
\pgfsys@transformshift{2.280405in}{0.732264in}%
\pgfsys@useobject{currentmarker}{}%
\end{pgfscope}%
\begin{pgfscope}%
\pgfsys@transformshift{2.280532in}{0.689298in}%
\pgfsys@useobject{currentmarker}{}%
\end{pgfscope}%
\begin{pgfscope}%
\pgfsys@transformshift{2.280659in}{0.646832in}%
\pgfsys@useobject{currentmarker}{}%
\end{pgfscope}%
\begin{pgfscope}%
\pgfsys@transformshift{2.280786in}{0.641459in}%
\pgfsys@useobject{currentmarker}{}%
\end{pgfscope}%
\begin{pgfscope}%
\pgfsys@transformshift{2.280914in}{0.688768in}%
\pgfsys@useobject{currentmarker}{}%
\end{pgfscope}%
\begin{pgfscope}%
\pgfsys@transformshift{2.281040in}{0.677020in}%
\pgfsys@useobject{currentmarker}{}%
\end{pgfscope}%
\begin{pgfscope}%
\pgfsys@transformshift{2.281167in}{0.704662in}%
\pgfsys@useobject{currentmarker}{}%
\end{pgfscope}%
\begin{pgfscope}%
\pgfsys@transformshift{2.281294in}{0.709688in}%
\pgfsys@useobject{currentmarker}{}%
\end{pgfscope}%
\begin{pgfscope}%
\pgfsys@transformshift{2.281421in}{0.722882in}%
\pgfsys@useobject{currentmarker}{}%
\end{pgfscope}%
\begin{pgfscope}%
\pgfsys@transformshift{2.281548in}{0.713927in}%
\pgfsys@useobject{currentmarker}{}%
\end{pgfscope}%
\begin{pgfscope}%
\pgfsys@transformshift{2.281674in}{0.714127in}%
\pgfsys@useobject{currentmarker}{}%
\end{pgfscope}%
\begin{pgfscope}%
\pgfsys@transformshift{2.281801in}{0.742413in}%
\pgfsys@useobject{currentmarker}{}%
\end{pgfscope}%
\begin{pgfscope}%
\pgfsys@transformshift{2.281927in}{0.718765in}%
\pgfsys@useobject{currentmarker}{}%
\end{pgfscope}%
\begin{pgfscope}%
\pgfsys@transformshift{2.282053in}{0.707617in}%
\pgfsys@useobject{currentmarker}{}%
\end{pgfscope}%
\begin{pgfscope}%
\pgfsys@transformshift{2.282180in}{0.714773in}%
\pgfsys@useobject{currentmarker}{}%
\end{pgfscope}%
\begin{pgfscope}%
\pgfsys@transformshift{2.282306in}{0.683500in}%
\pgfsys@useobject{currentmarker}{}%
\end{pgfscope}%
\begin{pgfscope}%
\pgfsys@transformshift{2.282432in}{0.657683in}%
\pgfsys@useobject{currentmarker}{}%
\end{pgfscope}%
\begin{pgfscope}%
\pgfsys@transformshift{2.282558in}{0.730130in}%
\pgfsys@useobject{currentmarker}{}%
\end{pgfscope}%
\begin{pgfscope}%
\pgfsys@transformshift{2.282684in}{0.714415in}%
\pgfsys@useobject{currentmarker}{}%
\end{pgfscope}%
\begin{pgfscope}%
\pgfsys@transformshift{2.282810in}{0.678671in}%
\pgfsys@useobject{currentmarker}{}%
\end{pgfscope}%
\begin{pgfscope}%
\pgfsys@transformshift{2.282936in}{0.671587in}%
\pgfsys@useobject{currentmarker}{}%
\end{pgfscope}%
\begin{pgfscope}%
\pgfsys@transformshift{2.283062in}{0.733670in}%
\pgfsys@useobject{currentmarker}{}%
\end{pgfscope}%
\begin{pgfscope}%
\pgfsys@transformshift{2.283188in}{0.732399in}%
\pgfsys@useobject{currentmarker}{}%
\end{pgfscope}%
\begin{pgfscope}%
\pgfsys@transformshift{2.283313in}{0.734602in}%
\pgfsys@useobject{currentmarker}{}%
\end{pgfscope}%
\begin{pgfscope}%
\pgfsys@transformshift{2.283439in}{0.706649in}%
\pgfsys@useobject{currentmarker}{}%
\end{pgfscope}%
\begin{pgfscope}%
\pgfsys@transformshift{2.283565in}{0.703793in}%
\pgfsys@useobject{currentmarker}{}%
\end{pgfscope}%
\begin{pgfscope}%
\pgfsys@transformshift{2.283690in}{0.672864in}%
\pgfsys@useobject{currentmarker}{}%
\end{pgfscope}%
\begin{pgfscope}%
\pgfsys@transformshift{2.283815in}{0.711660in}%
\pgfsys@useobject{currentmarker}{}%
\end{pgfscope}%
\begin{pgfscope}%
\pgfsys@transformshift{2.283941in}{0.714909in}%
\pgfsys@useobject{currentmarker}{}%
\end{pgfscope}%
\begin{pgfscope}%
\pgfsys@transformshift{2.284066in}{0.689798in}%
\pgfsys@useobject{currentmarker}{}%
\end{pgfscope}%
\begin{pgfscope}%
\pgfsys@transformshift{2.284191in}{0.702071in}%
\pgfsys@useobject{currentmarker}{}%
\end{pgfscope}%
\begin{pgfscope}%
\pgfsys@transformshift{2.284316in}{0.742441in}%
\pgfsys@useobject{currentmarker}{}%
\end{pgfscope}%
\begin{pgfscope}%
\pgfsys@transformshift{2.284441in}{0.712419in}%
\pgfsys@useobject{currentmarker}{}%
\end{pgfscope}%
\begin{pgfscope}%
\pgfsys@transformshift{2.284566in}{0.697232in}%
\pgfsys@useobject{currentmarker}{}%
\end{pgfscope}%
\begin{pgfscope}%
\pgfsys@transformshift{2.284691in}{0.737951in}%
\pgfsys@useobject{currentmarker}{}%
\end{pgfscope}%
\begin{pgfscope}%
\pgfsys@transformshift{2.284816in}{0.704572in}%
\pgfsys@useobject{currentmarker}{}%
\end{pgfscope}%
\begin{pgfscope}%
\pgfsys@transformshift{2.284941in}{0.713422in}%
\pgfsys@useobject{currentmarker}{}%
\end{pgfscope}%
\begin{pgfscope}%
\pgfsys@transformshift{2.285065in}{0.713941in}%
\pgfsys@useobject{currentmarker}{}%
\end{pgfscope}%
\begin{pgfscope}%
\pgfsys@transformshift{2.285190in}{0.725179in}%
\pgfsys@useobject{currentmarker}{}%
\end{pgfscope}%
\begin{pgfscope}%
\pgfsys@transformshift{2.285315in}{0.677097in}%
\pgfsys@useobject{currentmarker}{}%
\end{pgfscope}%
\begin{pgfscope}%
\pgfsys@transformshift{2.285439in}{0.697009in}%
\pgfsys@useobject{currentmarker}{}%
\end{pgfscope}%
\begin{pgfscope}%
\pgfsys@transformshift{2.285564in}{0.665123in}%
\pgfsys@useobject{currentmarker}{}%
\end{pgfscope}%
\begin{pgfscope}%
\pgfsys@transformshift{2.285688in}{0.631973in}%
\pgfsys@useobject{currentmarker}{}%
\end{pgfscope}%
\begin{pgfscope}%
\pgfsys@transformshift{2.285812in}{0.643770in}%
\pgfsys@useobject{currentmarker}{}%
\end{pgfscope}%
\begin{pgfscope}%
\pgfsys@transformshift{2.285936in}{0.650149in}%
\pgfsys@useobject{currentmarker}{}%
\end{pgfscope}%
\begin{pgfscope}%
\pgfsys@transformshift{2.286061in}{0.711002in}%
\pgfsys@useobject{currentmarker}{}%
\end{pgfscope}%
\begin{pgfscope}%
\pgfsys@transformshift{2.286185in}{0.736984in}%
\pgfsys@useobject{currentmarker}{}%
\end{pgfscope}%
\begin{pgfscope}%
\pgfsys@transformshift{2.286309in}{0.720922in}%
\pgfsys@useobject{currentmarker}{}%
\end{pgfscope}%
\begin{pgfscope}%
\pgfsys@transformshift{2.286433in}{0.699732in}%
\pgfsys@useobject{currentmarker}{}%
\end{pgfscope}%
\begin{pgfscope}%
\pgfsys@transformshift{2.286556in}{0.724797in}%
\pgfsys@useobject{currentmarker}{}%
\end{pgfscope}%
\begin{pgfscope}%
\pgfsys@transformshift{2.286680in}{0.720633in}%
\pgfsys@useobject{currentmarker}{}%
\end{pgfscope}%
\begin{pgfscope}%
\pgfsys@transformshift{2.286804in}{0.722209in}%
\pgfsys@useobject{currentmarker}{}%
\end{pgfscope}%
\begin{pgfscope}%
\pgfsys@transformshift{2.286928in}{0.700440in}%
\pgfsys@useobject{currentmarker}{}%
\end{pgfscope}%
\begin{pgfscope}%
\pgfsys@transformshift{2.287051in}{0.730090in}%
\pgfsys@useobject{currentmarker}{}%
\end{pgfscope}%
\begin{pgfscope}%
\pgfsys@transformshift{2.287175in}{0.747054in}%
\pgfsys@useobject{currentmarker}{}%
\end{pgfscope}%
\begin{pgfscope}%
\pgfsys@transformshift{2.287298in}{0.675692in}%
\pgfsys@useobject{currentmarker}{}%
\end{pgfscope}%
\begin{pgfscope}%
\pgfsys@transformshift{2.287422in}{0.697817in}%
\pgfsys@useobject{currentmarker}{}%
\end{pgfscope}%
\begin{pgfscope}%
\pgfsys@transformshift{2.287545in}{0.714574in}%
\pgfsys@useobject{currentmarker}{}%
\end{pgfscope}%
\begin{pgfscope}%
\pgfsys@transformshift{2.287668in}{0.720861in}%
\pgfsys@useobject{currentmarker}{}%
\end{pgfscope}%
\begin{pgfscope}%
\pgfsys@transformshift{2.287791in}{0.739168in}%
\pgfsys@useobject{currentmarker}{}%
\end{pgfscope}%
\begin{pgfscope}%
\pgfsys@transformshift{2.287915in}{0.708985in}%
\pgfsys@useobject{currentmarker}{}%
\end{pgfscope}%
\begin{pgfscope}%
\pgfsys@transformshift{2.288038in}{0.680209in}%
\pgfsys@useobject{currentmarker}{}%
\end{pgfscope}%
\begin{pgfscope}%
\pgfsys@transformshift{2.288161in}{0.657840in}%
\pgfsys@useobject{currentmarker}{}%
\end{pgfscope}%
\begin{pgfscope}%
\pgfsys@transformshift{2.288283in}{0.689287in}%
\pgfsys@useobject{currentmarker}{}%
\end{pgfscope}%
\begin{pgfscope}%
\pgfsys@transformshift{2.288406in}{0.737630in}%
\pgfsys@useobject{currentmarker}{}%
\end{pgfscope}%
\begin{pgfscope}%
\pgfsys@transformshift{2.288529in}{0.705219in}%
\pgfsys@useobject{currentmarker}{}%
\end{pgfscope}%
\begin{pgfscope}%
\pgfsys@transformshift{2.288652in}{0.702693in}%
\pgfsys@useobject{currentmarker}{}%
\end{pgfscope}%
\begin{pgfscope}%
\pgfsys@transformshift{2.288775in}{0.722332in}%
\pgfsys@useobject{currentmarker}{}%
\end{pgfscope}%
\begin{pgfscope}%
\pgfsys@transformshift{2.288897in}{0.708950in}%
\pgfsys@useobject{currentmarker}{}%
\end{pgfscope}%
\begin{pgfscope}%
\pgfsys@transformshift{2.289020in}{0.701449in}%
\pgfsys@useobject{currentmarker}{}%
\end{pgfscope}%
\begin{pgfscope}%
\pgfsys@transformshift{2.289142in}{0.704127in}%
\pgfsys@useobject{currentmarker}{}%
\end{pgfscope}%
\begin{pgfscope}%
\pgfsys@transformshift{2.289264in}{0.671512in}%
\pgfsys@useobject{currentmarker}{}%
\end{pgfscope}%
\begin{pgfscope}%
\pgfsys@transformshift{2.289387in}{0.720612in}%
\pgfsys@useobject{currentmarker}{}%
\end{pgfscope}%
\begin{pgfscope}%
\pgfsys@transformshift{2.289509in}{0.737685in}%
\pgfsys@useobject{currentmarker}{}%
\end{pgfscope}%
\begin{pgfscope}%
\pgfsys@transformshift{2.289631in}{0.706024in}%
\pgfsys@useobject{currentmarker}{}%
\end{pgfscope}%
\begin{pgfscope}%
\pgfsys@transformshift{2.289753in}{0.710570in}%
\pgfsys@useobject{currentmarker}{}%
\end{pgfscope}%
\begin{pgfscope}%
\pgfsys@transformshift{2.289875in}{0.696791in}%
\pgfsys@useobject{currentmarker}{}%
\end{pgfscope}%
\begin{pgfscope}%
\pgfsys@transformshift{2.289997in}{0.679317in}%
\pgfsys@useobject{currentmarker}{}%
\end{pgfscope}%
\begin{pgfscope}%
\pgfsys@transformshift{2.290119in}{0.702166in}%
\pgfsys@useobject{currentmarker}{}%
\end{pgfscope}%
\begin{pgfscope}%
\pgfsys@transformshift{2.290241in}{0.726785in}%
\pgfsys@useobject{currentmarker}{}%
\end{pgfscope}%
\begin{pgfscope}%
\pgfsys@transformshift{2.290363in}{0.716559in}%
\pgfsys@useobject{currentmarker}{}%
\end{pgfscope}%
\begin{pgfscope}%
\pgfsys@transformshift{2.290485in}{0.664115in}%
\pgfsys@useobject{currentmarker}{}%
\end{pgfscope}%
\begin{pgfscope}%
\pgfsys@transformshift{2.290606in}{0.691506in}%
\pgfsys@useobject{currentmarker}{}%
\end{pgfscope}%
\begin{pgfscope}%
\pgfsys@transformshift{2.290728in}{0.765908in}%
\pgfsys@useobject{currentmarker}{}%
\end{pgfscope}%
\begin{pgfscope}%
\pgfsys@transformshift{2.290849in}{0.781369in}%
\pgfsys@useobject{currentmarker}{}%
\end{pgfscope}%
\begin{pgfscope}%
\pgfsys@transformshift{2.290971in}{0.711453in}%
\pgfsys@useobject{currentmarker}{}%
\end{pgfscope}%
\begin{pgfscope}%
\pgfsys@transformshift{2.291092in}{0.725165in}%
\pgfsys@useobject{currentmarker}{}%
\end{pgfscope}%
\begin{pgfscope}%
\pgfsys@transformshift{2.291214in}{0.734145in}%
\pgfsys@useobject{currentmarker}{}%
\end{pgfscope}%
\begin{pgfscope}%
\pgfsys@transformshift{2.291335in}{0.673424in}%
\pgfsys@useobject{currentmarker}{}%
\end{pgfscope}%
\begin{pgfscope}%
\pgfsys@transformshift{2.291456in}{0.733732in}%
\pgfsys@useobject{currentmarker}{}%
\end{pgfscope}%
\begin{pgfscope}%
\pgfsys@transformshift{2.291577in}{0.723982in}%
\pgfsys@useobject{currentmarker}{}%
\end{pgfscope}%
\begin{pgfscope}%
\pgfsys@transformshift{2.291698in}{0.709716in}%
\pgfsys@useobject{currentmarker}{}%
\end{pgfscope}%
\begin{pgfscope}%
\pgfsys@transformshift{2.291819in}{0.711911in}%
\pgfsys@useobject{currentmarker}{}%
\end{pgfscope}%
\begin{pgfscope}%
\pgfsys@transformshift{2.291940in}{0.722068in}%
\pgfsys@useobject{currentmarker}{}%
\end{pgfscope}%
\begin{pgfscope}%
\pgfsys@transformshift{2.292061in}{0.704247in}%
\pgfsys@useobject{currentmarker}{}%
\end{pgfscope}%
\begin{pgfscope}%
\pgfsys@transformshift{2.292182in}{0.741973in}%
\pgfsys@useobject{currentmarker}{}%
\end{pgfscope}%
\begin{pgfscope}%
\pgfsys@transformshift{2.292303in}{0.698854in}%
\pgfsys@useobject{currentmarker}{}%
\end{pgfscope}%
\begin{pgfscope}%
\pgfsys@transformshift{2.292423in}{0.706286in}%
\pgfsys@useobject{currentmarker}{}%
\end{pgfscope}%
\begin{pgfscope}%
\pgfsys@transformshift{2.292544in}{0.728893in}%
\pgfsys@useobject{currentmarker}{}%
\end{pgfscope}%
\begin{pgfscope}%
\pgfsys@transformshift{2.292665in}{0.701155in}%
\pgfsys@useobject{currentmarker}{}%
\end{pgfscope}%
\begin{pgfscope}%
\pgfsys@transformshift{2.292785in}{0.687686in}%
\pgfsys@useobject{currentmarker}{}%
\end{pgfscope}%
\begin{pgfscope}%
\pgfsys@transformshift{2.292905in}{0.684856in}%
\pgfsys@useobject{currentmarker}{}%
\end{pgfscope}%
\begin{pgfscope}%
\pgfsys@transformshift{2.293026in}{0.705366in}%
\pgfsys@useobject{currentmarker}{}%
\end{pgfscope}%
\begin{pgfscope}%
\pgfsys@transformshift{2.293146in}{0.690791in}%
\pgfsys@useobject{currentmarker}{}%
\end{pgfscope}%
\begin{pgfscope}%
\pgfsys@transformshift{2.293266in}{0.680338in}%
\pgfsys@useobject{currentmarker}{}%
\end{pgfscope}%
\begin{pgfscope}%
\pgfsys@transformshift{2.293386in}{0.674711in}%
\pgfsys@useobject{currentmarker}{}%
\end{pgfscope}%
\begin{pgfscope}%
\pgfsys@transformshift{2.293507in}{0.713793in}%
\pgfsys@useobject{currentmarker}{}%
\end{pgfscope}%
\begin{pgfscope}%
\pgfsys@transformshift{2.293627in}{0.712175in}%
\pgfsys@useobject{currentmarker}{}%
\end{pgfscope}%
\begin{pgfscope}%
\pgfsys@transformshift{2.293747in}{0.697309in}%
\pgfsys@useobject{currentmarker}{}%
\end{pgfscope}%
\begin{pgfscope}%
\pgfsys@transformshift{2.293866in}{0.665552in}%
\pgfsys@useobject{currentmarker}{}%
\end{pgfscope}%
\begin{pgfscope}%
\pgfsys@transformshift{2.293986in}{0.657562in}%
\pgfsys@useobject{currentmarker}{}%
\end{pgfscope}%
\begin{pgfscope}%
\pgfsys@transformshift{2.294106in}{0.712121in}%
\pgfsys@useobject{currentmarker}{}%
\end{pgfscope}%
\begin{pgfscope}%
\pgfsys@transformshift{2.294226in}{0.712943in}%
\pgfsys@useobject{currentmarker}{}%
\end{pgfscope}%
\begin{pgfscope}%
\pgfsys@transformshift{2.294345in}{0.678153in}%
\pgfsys@useobject{currentmarker}{}%
\end{pgfscope}%
\begin{pgfscope}%
\pgfsys@transformshift{2.294465in}{0.724651in}%
\pgfsys@useobject{currentmarker}{}%
\end{pgfscope}%
\begin{pgfscope}%
\pgfsys@transformshift{2.294585in}{0.708174in}%
\pgfsys@useobject{currentmarker}{}%
\end{pgfscope}%
\begin{pgfscope}%
\pgfsys@transformshift{2.294704in}{0.678162in}%
\pgfsys@useobject{currentmarker}{}%
\end{pgfscope}%
\begin{pgfscope}%
\pgfsys@transformshift{2.294823in}{0.703519in}%
\pgfsys@useobject{currentmarker}{}%
\end{pgfscope}%
\begin{pgfscope}%
\pgfsys@transformshift{2.294943in}{0.712702in}%
\pgfsys@useobject{currentmarker}{}%
\end{pgfscope}%
\begin{pgfscope}%
\pgfsys@transformshift{2.295062in}{0.722684in}%
\pgfsys@useobject{currentmarker}{}%
\end{pgfscope}%
\begin{pgfscope}%
\pgfsys@transformshift{2.295181in}{0.741006in}%
\pgfsys@useobject{currentmarker}{}%
\end{pgfscope}%
\begin{pgfscope}%
\pgfsys@transformshift{2.295300in}{0.737123in}%
\pgfsys@useobject{currentmarker}{}%
\end{pgfscope}%
\begin{pgfscope}%
\pgfsys@transformshift{2.295419in}{0.702855in}%
\pgfsys@useobject{currentmarker}{}%
\end{pgfscope}%
\begin{pgfscope}%
\pgfsys@transformshift{2.295538in}{0.676737in}%
\pgfsys@useobject{currentmarker}{}%
\end{pgfscope}%
\begin{pgfscope}%
\pgfsys@transformshift{2.295657in}{0.692031in}%
\pgfsys@useobject{currentmarker}{}%
\end{pgfscope}%
\begin{pgfscope}%
\pgfsys@transformshift{2.295776in}{0.678590in}%
\pgfsys@useobject{currentmarker}{}%
\end{pgfscope}%
\begin{pgfscope}%
\pgfsys@transformshift{2.295895in}{0.669163in}%
\pgfsys@useobject{currentmarker}{}%
\end{pgfscope}%
\begin{pgfscope}%
\pgfsys@transformshift{2.296014in}{0.657821in}%
\pgfsys@useobject{currentmarker}{}%
\end{pgfscope}%
\begin{pgfscope}%
\pgfsys@transformshift{2.296133in}{0.711149in}%
\pgfsys@useobject{currentmarker}{}%
\end{pgfscope}%
\begin{pgfscope}%
\pgfsys@transformshift{2.296251in}{0.740318in}%
\pgfsys@useobject{currentmarker}{}%
\end{pgfscope}%
\begin{pgfscope}%
\pgfsys@transformshift{2.296370in}{0.730645in}%
\pgfsys@useobject{currentmarker}{}%
\end{pgfscope}%
\begin{pgfscope}%
\pgfsys@transformshift{2.296488in}{0.686735in}%
\pgfsys@useobject{currentmarker}{}%
\end{pgfscope}%
\begin{pgfscope}%
\pgfsys@transformshift{2.296607in}{0.692440in}%
\pgfsys@useobject{currentmarker}{}%
\end{pgfscope}%
\begin{pgfscope}%
\pgfsys@transformshift{2.296725in}{0.691881in}%
\pgfsys@useobject{currentmarker}{}%
\end{pgfscope}%
\begin{pgfscope}%
\pgfsys@transformshift{2.296844in}{0.710498in}%
\pgfsys@useobject{currentmarker}{}%
\end{pgfscope}%
\begin{pgfscope}%
\pgfsys@transformshift{2.296962in}{0.686410in}%
\pgfsys@useobject{currentmarker}{}%
\end{pgfscope}%
\begin{pgfscope}%
\pgfsys@transformshift{2.297080in}{0.683287in}%
\pgfsys@useobject{currentmarker}{}%
\end{pgfscope}%
\begin{pgfscope}%
\pgfsys@transformshift{2.297198in}{0.672157in}%
\pgfsys@useobject{currentmarker}{}%
\end{pgfscope}%
\begin{pgfscope}%
\pgfsys@transformshift{2.297316in}{0.634107in}%
\pgfsys@useobject{currentmarker}{}%
\end{pgfscope}%
\begin{pgfscope}%
\pgfsys@transformshift{2.297434in}{0.661659in}%
\pgfsys@useobject{currentmarker}{}%
\end{pgfscope}%
\begin{pgfscope}%
\pgfsys@transformshift{2.297552in}{0.722232in}%
\pgfsys@useobject{currentmarker}{}%
\end{pgfscope}%
\begin{pgfscope}%
\pgfsys@transformshift{2.297670in}{0.729114in}%
\pgfsys@useobject{currentmarker}{}%
\end{pgfscope}%
\begin{pgfscope}%
\pgfsys@transformshift{2.297788in}{0.704515in}%
\pgfsys@useobject{currentmarker}{}%
\end{pgfscope}%
\begin{pgfscope}%
\pgfsys@transformshift{2.297906in}{0.696408in}%
\pgfsys@useobject{currentmarker}{}%
\end{pgfscope}%
\begin{pgfscope}%
\pgfsys@transformshift{2.298023in}{0.640024in}%
\pgfsys@useobject{currentmarker}{}%
\end{pgfscope}%
\begin{pgfscope}%
\pgfsys@transformshift{2.298141in}{0.668973in}%
\pgfsys@useobject{currentmarker}{}%
\end{pgfscope}%
\begin{pgfscope}%
\pgfsys@transformshift{2.298259in}{0.721226in}%
\pgfsys@useobject{currentmarker}{}%
\end{pgfscope}%
\begin{pgfscope}%
\pgfsys@transformshift{2.298376in}{0.714735in}%
\pgfsys@useobject{currentmarker}{}%
\end{pgfscope}%
\begin{pgfscope}%
\pgfsys@transformshift{2.298494in}{0.679163in}%
\pgfsys@useobject{currentmarker}{}%
\end{pgfscope}%
\begin{pgfscope}%
\pgfsys@transformshift{2.298611in}{0.670111in}%
\pgfsys@useobject{currentmarker}{}%
\end{pgfscope}%
\begin{pgfscope}%
\pgfsys@transformshift{2.298728in}{0.678831in}%
\pgfsys@useobject{currentmarker}{}%
\end{pgfscope}%
\begin{pgfscope}%
\pgfsys@transformshift{2.298846in}{0.694435in}%
\pgfsys@useobject{currentmarker}{}%
\end{pgfscope}%
\begin{pgfscope}%
\pgfsys@transformshift{2.298963in}{0.684698in}%
\pgfsys@useobject{currentmarker}{}%
\end{pgfscope}%
\begin{pgfscope}%
\pgfsys@transformshift{2.299080in}{0.691228in}%
\pgfsys@useobject{currentmarker}{}%
\end{pgfscope}%
\begin{pgfscope}%
\pgfsys@transformshift{2.299197in}{0.710030in}%
\pgfsys@useobject{currentmarker}{}%
\end{pgfscope}%
\begin{pgfscope}%
\pgfsys@transformshift{2.299314in}{0.654207in}%
\pgfsys@useobject{currentmarker}{}%
\end{pgfscope}%
\begin{pgfscope}%
\pgfsys@transformshift{2.299431in}{0.667456in}%
\pgfsys@useobject{currentmarker}{}%
\end{pgfscope}%
\begin{pgfscope}%
\pgfsys@transformshift{2.299548in}{0.661154in}%
\pgfsys@useobject{currentmarker}{}%
\end{pgfscope}%
\begin{pgfscope}%
\pgfsys@transformshift{2.299665in}{0.657005in}%
\pgfsys@useobject{currentmarker}{}%
\end{pgfscope}%
\begin{pgfscope}%
\pgfsys@transformshift{2.299782in}{0.707789in}%
\pgfsys@useobject{currentmarker}{}%
\end{pgfscope}%
\begin{pgfscope}%
\pgfsys@transformshift{2.299898in}{0.691450in}%
\pgfsys@useobject{currentmarker}{}%
\end{pgfscope}%
\begin{pgfscope}%
\pgfsys@transformshift{2.300015in}{0.688906in}%
\pgfsys@useobject{currentmarker}{}%
\end{pgfscope}%
\begin{pgfscope}%
\pgfsys@transformshift{2.300132in}{0.721522in}%
\pgfsys@useobject{currentmarker}{}%
\end{pgfscope}%
\begin{pgfscope}%
\pgfsys@transformshift{2.300248in}{0.723547in}%
\pgfsys@useobject{currentmarker}{}%
\end{pgfscope}%
\begin{pgfscope}%
\pgfsys@transformshift{2.300365in}{0.723183in}%
\pgfsys@useobject{currentmarker}{}%
\end{pgfscope}%
\begin{pgfscope}%
\pgfsys@transformshift{2.300481in}{0.731823in}%
\pgfsys@useobject{currentmarker}{}%
\end{pgfscope}%
\begin{pgfscope}%
\pgfsys@transformshift{2.300598in}{0.709856in}%
\pgfsys@useobject{currentmarker}{}%
\end{pgfscope}%
\begin{pgfscope}%
\pgfsys@transformshift{2.300714in}{0.705536in}%
\pgfsys@useobject{currentmarker}{}%
\end{pgfscope}%
\begin{pgfscope}%
\pgfsys@transformshift{2.300830in}{0.670328in}%
\pgfsys@useobject{currentmarker}{}%
\end{pgfscope}%
\begin{pgfscope}%
\pgfsys@transformshift{2.300946in}{0.718140in}%
\pgfsys@useobject{currentmarker}{}%
\end{pgfscope}%
\begin{pgfscope}%
\pgfsys@transformshift{2.301062in}{0.714202in}%
\pgfsys@useobject{currentmarker}{}%
\end{pgfscope}%
\begin{pgfscope}%
\pgfsys@transformshift{2.301178in}{0.673391in}%
\pgfsys@useobject{currentmarker}{}%
\end{pgfscope}%
\begin{pgfscope}%
\pgfsys@transformshift{2.301294in}{0.700064in}%
\pgfsys@useobject{currentmarker}{}%
\end{pgfscope}%
\begin{pgfscope}%
\pgfsys@transformshift{2.301410in}{0.699343in}%
\pgfsys@useobject{currentmarker}{}%
\end{pgfscope}%
\begin{pgfscope}%
\pgfsys@transformshift{2.301526in}{0.620035in}%
\pgfsys@useobject{currentmarker}{}%
\end{pgfscope}%
\begin{pgfscope}%
\pgfsys@transformshift{2.301642in}{0.678801in}%
\pgfsys@useobject{currentmarker}{}%
\end{pgfscope}%
\begin{pgfscope}%
\pgfsys@transformshift{2.301758in}{0.682002in}%
\pgfsys@useobject{currentmarker}{}%
\end{pgfscope}%
\begin{pgfscope}%
\pgfsys@transformshift{2.301874in}{0.662337in}%
\pgfsys@useobject{currentmarker}{}%
\end{pgfscope}%
\begin{pgfscope}%
\pgfsys@transformshift{2.301989in}{0.698650in}%
\pgfsys@useobject{currentmarker}{}%
\end{pgfscope}%
\begin{pgfscope}%
\pgfsys@transformshift{2.302105in}{0.700449in}%
\pgfsys@useobject{currentmarker}{}%
\end{pgfscope}%
\begin{pgfscope}%
\pgfsys@transformshift{2.302220in}{0.713300in}%
\pgfsys@useobject{currentmarker}{}%
\end{pgfscope}%
\begin{pgfscope}%
\pgfsys@transformshift{2.302336in}{0.660434in}%
\pgfsys@useobject{currentmarker}{}%
\end{pgfscope}%
\begin{pgfscope}%
\pgfsys@transformshift{2.302451in}{0.720749in}%
\pgfsys@useobject{currentmarker}{}%
\end{pgfscope}%
\begin{pgfscope}%
\pgfsys@transformshift{2.302567in}{0.735608in}%
\pgfsys@useobject{currentmarker}{}%
\end{pgfscope}%
\begin{pgfscope}%
\pgfsys@transformshift{2.302682in}{0.686856in}%
\pgfsys@useobject{currentmarker}{}%
\end{pgfscope}%
\begin{pgfscope}%
\pgfsys@transformshift{2.302797in}{0.666006in}%
\pgfsys@useobject{currentmarker}{}%
\end{pgfscope}%
\begin{pgfscope}%
\pgfsys@transformshift{2.302912in}{0.621537in}%
\pgfsys@useobject{currentmarker}{}%
\end{pgfscope}%
\begin{pgfscope}%
\pgfsys@transformshift{2.303027in}{0.657001in}%
\pgfsys@useobject{currentmarker}{}%
\end{pgfscope}%
\begin{pgfscope}%
\pgfsys@transformshift{2.303142in}{0.711995in}%
\pgfsys@useobject{currentmarker}{}%
\end{pgfscope}%
\begin{pgfscope}%
\pgfsys@transformshift{2.303257in}{0.701378in}%
\pgfsys@useobject{currentmarker}{}%
\end{pgfscope}%
\begin{pgfscope}%
\pgfsys@transformshift{2.303372in}{0.753669in}%
\pgfsys@useobject{currentmarker}{}%
\end{pgfscope}%
\begin{pgfscope}%
\pgfsys@transformshift{2.303487in}{0.733664in}%
\pgfsys@useobject{currentmarker}{}%
\end{pgfscope}%
\begin{pgfscope}%
\pgfsys@transformshift{2.303602in}{0.685683in}%
\pgfsys@useobject{currentmarker}{}%
\end{pgfscope}%
\begin{pgfscope}%
\pgfsys@transformshift{2.303717in}{0.711573in}%
\pgfsys@useobject{currentmarker}{}%
\end{pgfscope}%
\begin{pgfscope}%
\pgfsys@transformshift{2.303832in}{0.701399in}%
\pgfsys@useobject{currentmarker}{}%
\end{pgfscope}%
\begin{pgfscope}%
\pgfsys@transformshift{2.303946in}{0.688463in}%
\pgfsys@useobject{currentmarker}{}%
\end{pgfscope}%
\begin{pgfscope}%
\pgfsys@transformshift{2.304061in}{0.705369in}%
\pgfsys@useobject{currentmarker}{}%
\end{pgfscope}%
\begin{pgfscope}%
\pgfsys@transformshift{2.304175in}{0.683930in}%
\pgfsys@useobject{currentmarker}{}%
\end{pgfscope}%
\begin{pgfscope}%
\pgfsys@transformshift{2.304290in}{0.704156in}%
\pgfsys@useobject{currentmarker}{}%
\end{pgfscope}%
\begin{pgfscope}%
\pgfsys@transformshift{2.304404in}{0.678038in}%
\pgfsys@useobject{currentmarker}{}%
\end{pgfscope}%
\begin{pgfscope}%
\pgfsys@transformshift{2.304519in}{0.703483in}%
\pgfsys@useobject{currentmarker}{}%
\end{pgfscope}%
\begin{pgfscope}%
\pgfsys@transformshift{2.304633in}{0.693840in}%
\pgfsys@useobject{currentmarker}{}%
\end{pgfscope}%
\begin{pgfscope}%
\pgfsys@transformshift{2.304747in}{0.691440in}%
\pgfsys@useobject{currentmarker}{}%
\end{pgfscope}%
\begin{pgfscope}%
\pgfsys@transformshift{2.304861in}{0.692269in}%
\pgfsys@useobject{currentmarker}{}%
\end{pgfscope}%
\begin{pgfscope}%
\pgfsys@transformshift{2.304975in}{0.712271in}%
\pgfsys@useobject{currentmarker}{}%
\end{pgfscope}%
\begin{pgfscope}%
\pgfsys@transformshift{2.305089in}{0.692906in}%
\pgfsys@useobject{currentmarker}{}%
\end{pgfscope}%
\begin{pgfscope}%
\pgfsys@transformshift{2.305204in}{0.694588in}%
\pgfsys@useobject{currentmarker}{}%
\end{pgfscope}%
\begin{pgfscope}%
\pgfsys@transformshift{2.305317in}{0.666775in}%
\pgfsys@useobject{currentmarker}{}%
\end{pgfscope}%
\begin{pgfscope}%
\pgfsys@transformshift{2.305431in}{0.629551in}%
\pgfsys@useobject{currentmarker}{}%
\end{pgfscope}%
\begin{pgfscope}%
\pgfsys@transformshift{2.305545in}{0.657286in}%
\pgfsys@useobject{currentmarker}{}%
\end{pgfscope}%
\begin{pgfscope}%
\pgfsys@transformshift{2.305659in}{0.672650in}%
\pgfsys@useobject{currentmarker}{}%
\end{pgfscope}%
\begin{pgfscope}%
\pgfsys@transformshift{2.305773in}{0.682784in}%
\pgfsys@useobject{currentmarker}{}%
\end{pgfscope}%
\begin{pgfscope}%
\pgfsys@transformshift{2.305886in}{0.694818in}%
\pgfsys@useobject{currentmarker}{}%
\end{pgfscope}%
\begin{pgfscope}%
\pgfsys@transformshift{2.306000in}{0.716199in}%
\pgfsys@useobject{currentmarker}{}%
\end{pgfscope}%
\begin{pgfscope}%
\pgfsys@transformshift{2.306113in}{0.713327in}%
\pgfsys@useobject{currentmarker}{}%
\end{pgfscope}%
\begin{pgfscope}%
\pgfsys@transformshift{2.306227in}{0.665396in}%
\pgfsys@useobject{currentmarker}{}%
\end{pgfscope}%
\begin{pgfscope}%
\pgfsys@transformshift{2.306340in}{0.666437in}%
\pgfsys@useobject{currentmarker}{}%
\end{pgfscope}%
\begin{pgfscope}%
\pgfsys@transformshift{2.306454in}{0.696480in}%
\pgfsys@useobject{currentmarker}{}%
\end{pgfscope}%
\begin{pgfscope}%
\pgfsys@transformshift{2.306567in}{0.732312in}%
\pgfsys@useobject{currentmarker}{}%
\end{pgfscope}%
\begin{pgfscope}%
\pgfsys@transformshift{2.306680in}{0.691786in}%
\pgfsys@useobject{currentmarker}{}%
\end{pgfscope}%
\begin{pgfscope}%
\pgfsys@transformshift{2.306794in}{0.730060in}%
\pgfsys@useobject{currentmarker}{}%
\end{pgfscope}%
\begin{pgfscope}%
\pgfsys@transformshift{2.306907in}{0.679721in}%
\pgfsys@useobject{currentmarker}{}%
\end{pgfscope}%
\begin{pgfscope}%
\pgfsys@transformshift{2.307020in}{0.681391in}%
\pgfsys@useobject{currentmarker}{}%
\end{pgfscope}%
\begin{pgfscope}%
\pgfsys@transformshift{2.307133in}{0.682218in}%
\pgfsys@useobject{currentmarker}{}%
\end{pgfscope}%
\begin{pgfscope}%
\pgfsys@transformshift{2.307246in}{0.644569in}%
\pgfsys@useobject{currentmarker}{}%
\end{pgfscope}%
\begin{pgfscope}%
\pgfsys@transformshift{2.307359in}{0.616939in}%
\pgfsys@useobject{currentmarker}{}%
\end{pgfscope}%
\begin{pgfscope}%
\pgfsys@transformshift{2.307472in}{0.700867in}%
\pgfsys@useobject{currentmarker}{}%
\end{pgfscope}%
\begin{pgfscope}%
\pgfsys@transformshift{2.307584in}{0.703737in}%
\pgfsys@useobject{currentmarker}{}%
\end{pgfscope}%
\begin{pgfscope}%
\pgfsys@transformshift{2.307697in}{0.676048in}%
\pgfsys@useobject{currentmarker}{}%
\end{pgfscope}%
\begin{pgfscope}%
\pgfsys@transformshift{2.307810in}{0.673511in}%
\pgfsys@useobject{currentmarker}{}%
\end{pgfscope}%
\begin{pgfscope}%
\pgfsys@transformshift{2.307922in}{0.699170in}%
\pgfsys@useobject{currentmarker}{}%
\end{pgfscope}%
\begin{pgfscope}%
\pgfsys@transformshift{2.308035in}{0.730706in}%
\pgfsys@useobject{currentmarker}{}%
\end{pgfscope}%
\begin{pgfscope}%
\pgfsys@transformshift{2.308148in}{0.750720in}%
\pgfsys@useobject{currentmarker}{}%
\end{pgfscope}%
\begin{pgfscope}%
\pgfsys@transformshift{2.308260in}{0.720723in}%
\pgfsys@useobject{currentmarker}{}%
\end{pgfscope}%
\begin{pgfscope}%
\pgfsys@transformshift{2.308372in}{0.718408in}%
\pgfsys@useobject{currentmarker}{}%
\end{pgfscope}%
\begin{pgfscope}%
\pgfsys@transformshift{2.308485in}{0.714159in}%
\pgfsys@useobject{currentmarker}{}%
\end{pgfscope}%
\begin{pgfscope}%
\pgfsys@transformshift{2.308597in}{0.693380in}%
\pgfsys@useobject{currentmarker}{}%
\end{pgfscope}%
\begin{pgfscope}%
\pgfsys@transformshift{2.308709in}{0.709693in}%
\pgfsys@useobject{currentmarker}{}%
\end{pgfscope}%
\begin{pgfscope}%
\pgfsys@transformshift{2.308822in}{0.641766in}%
\pgfsys@useobject{currentmarker}{}%
\end{pgfscope}%
\begin{pgfscope}%
\pgfsys@transformshift{2.308934in}{0.664414in}%
\pgfsys@useobject{currentmarker}{}%
\end{pgfscope}%
\begin{pgfscope}%
\pgfsys@transformshift{2.309046in}{0.665097in}%
\pgfsys@useobject{currentmarker}{}%
\end{pgfscope}%
\begin{pgfscope}%
\pgfsys@transformshift{2.309158in}{0.685991in}%
\pgfsys@useobject{currentmarker}{}%
\end{pgfscope}%
\begin{pgfscope}%
\pgfsys@transformshift{2.309270in}{0.697556in}%
\pgfsys@useobject{currentmarker}{}%
\end{pgfscope}%
\begin{pgfscope}%
\pgfsys@transformshift{2.309382in}{0.679058in}%
\pgfsys@useobject{currentmarker}{}%
\end{pgfscope}%
\begin{pgfscope}%
\pgfsys@transformshift{2.309493in}{0.696982in}%
\pgfsys@useobject{currentmarker}{}%
\end{pgfscope}%
\begin{pgfscope}%
\pgfsys@transformshift{2.309605in}{0.704530in}%
\pgfsys@useobject{currentmarker}{}%
\end{pgfscope}%
\begin{pgfscope}%
\pgfsys@transformshift{2.309717in}{0.646856in}%
\pgfsys@useobject{currentmarker}{}%
\end{pgfscope}%
\begin{pgfscope}%
\pgfsys@transformshift{2.309829in}{0.742471in}%
\pgfsys@useobject{currentmarker}{}%
\end{pgfscope}%
\begin{pgfscope}%
\pgfsys@transformshift{2.309940in}{0.723047in}%
\pgfsys@useobject{currentmarker}{}%
\end{pgfscope}%
\begin{pgfscope}%
\pgfsys@transformshift{2.310052in}{0.719874in}%
\pgfsys@useobject{currentmarker}{}%
\end{pgfscope}%
\begin{pgfscope}%
\pgfsys@transformshift{2.310163in}{0.731081in}%
\pgfsys@useobject{currentmarker}{}%
\end{pgfscope}%
\begin{pgfscope}%
\pgfsys@transformshift{2.310275in}{0.680279in}%
\pgfsys@useobject{currentmarker}{}%
\end{pgfscope}%
\begin{pgfscope}%
\pgfsys@transformshift{2.310386in}{0.708039in}%
\pgfsys@useobject{currentmarker}{}%
\end{pgfscope}%
\begin{pgfscope}%
\pgfsys@transformshift{2.310498in}{0.722225in}%
\pgfsys@useobject{currentmarker}{}%
\end{pgfscope}%
\begin{pgfscope}%
\pgfsys@transformshift{2.310609in}{0.669204in}%
\pgfsys@useobject{currentmarker}{}%
\end{pgfscope}%
\begin{pgfscope}%
\pgfsys@transformshift{2.310720in}{0.657773in}%
\pgfsys@useobject{currentmarker}{}%
\end{pgfscope}%
\begin{pgfscope}%
\pgfsys@transformshift{2.310831in}{0.708807in}%
\pgfsys@useobject{currentmarker}{}%
\end{pgfscope}%
\begin{pgfscope}%
\pgfsys@transformshift{2.310942in}{0.664331in}%
\pgfsys@useobject{currentmarker}{}%
\end{pgfscope}%
\begin{pgfscope}%
\pgfsys@transformshift{2.311054in}{0.636421in}%
\pgfsys@useobject{currentmarker}{}%
\end{pgfscope}%
\begin{pgfscope}%
\pgfsys@transformshift{2.311165in}{0.701558in}%
\pgfsys@useobject{currentmarker}{}%
\end{pgfscope}%
\begin{pgfscope}%
\pgfsys@transformshift{2.311276in}{0.701825in}%
\pgfsys@useobject{currentmarker}{}%
\end{pgfscope}%
\begin{pgfscope}%
\pgfsys@transformshift{2.311386in}{0.711489in}%
\pgfsys@useobject{currentmarker}{}%
\end{pgfscope}%
\begin{pgfscope}%
\pgfsys@transformshift{2.311497in}{0.633541in}%
\pgfsys@useobject{currentmarker}{}%
\end{pgfscope}%
\begin{pgfscope}%
\pgfsys@transformshift{2.311608in}{0.662446in}%
\pgfsys@useobject{currentmarker}{}%
\end{pgfscope}%
\begin{pgfscope}%
\pgfsys@transformshift{2.311719in}{0.660526in}%
\pgfsys@useobject{currentmarker}{}%
\end{pgfscope}%
\begin{pgfscope}%
\pgfsys@transformshift{2.311830in}{0.674123in}%
\pgfsys@useobject{currentmarker}{}%
\end{pgfscope}%
\begin{pgfscope}%
\pgfsys@transformshift{2.311940in}{0.672911in}%
\pgfsys@useobject{currentmarker}{}%
\end{pgfscope}%
\begin{pgfscope}%
\pgfsys@transformshift{2.312051in}{0.683456in}%
\pgfsys@useobject{currentmarker}{}%
\end{pgfscope}%
\begin{pgfscope}%
\pgfsys@transformshift{2.312161in}{0.633243in}%
\pgfsys@useobject{currentmarker}{}%
\end{pgfscope}%
\begin{pgfscope}%
\pgfsys@transformshift{2.312272in}{0.685455in}%
\pgfsys@useobject{currentmarker}{}%
\end{pgfscope}%
\begin{pgfscope}%
\pgfsys@transformshift{2.312382in}{0.671903in}%
\pgfsys@useobject{currentmarker}{}%
\end{pgfscope}%
\begin{pgfscope}%
\pgfsys@transformshift{2.312493in}{0.678689in}%
\pgfsys@useobject{currentmarker}{}%
\end{pgfscope}%
\begin{pgfscope}%
\pgfsys@transformshift{2.312603in}{0.677845in}%
\pgfsys@useobject{currentmarker}{}%
\end{pgfscope}%
\begin{pgfscope}%
\pgfsys@transformshift{2.312713in}{0.714490in}%
\pgfsys@useobject{currentmarker}{}%
\end{pgfscope}%
\begin{pgfscope}%
\pgfsys@transformshift{2.312823in}{0.706529in}%
\pgfsys@useobject{currentmarker}{}%
\end{pgfscope}%
\begin{pgfscope}%
\pgfsys@transformshift{2.312934in}{0.707921in}%
\pgfsys@useobject{currentmarker}{}%
\end{pgfscope}%
\begin{pgfscope}%
\pgfsys@transformshift{2.313044in}{0.746064in}%
\pgfsys@useobject{currentmarker}{}%
\end{pgfscope}%
\begin{pgfscope}%
\pgfsys@transformshift{2.313154in}{0.735332in}%
\pgfsys@useobject{currentmarker}{}%
\end{pgfscope}%
\begin{pgfscope}%
\pgfsys@transformshift{2.313264in}{0.750442in}%
\pgfsys@useobject{currentmarker}{}%
\end{pgfscope}%
\begin{pgfscope}%
\pgfsys@transformshift{2.313374in}{0.734084in}%
\pgfsys@useobject{currentmarker}{}%
\end{pgfscope}%
\begin{pgfscope}%
\pgfsys@transformshift{2.313483in}{0.698187in}%
\pgfsys@useobject{currentmarker}{}%
\end{pgfscope}%
\begin{pgfscope}%
\pgfsys@transformshift{2.313593in}{0.666802in}%
\pgfsys@useobject{currentmarker}{}%
\end{pgfscope}%
\end{pgfscope}%
\begin{pgfscope}%
\pgfsetrectcap%
\pgfsetmiterjoin%
\pgfsetlinewidth{0.803000pt}%
\definecolor{currentstroke}{rgb}{0.000000,0.000000,0.000000}%
\pgfsetstrokecolor{currentstroke}%
\pgfsetdash{}{0pt}%
\pgfpathmoveto{\pgfqpoint{0.514278in}{0.417642in}}%
\pgfpathlineto{\pgfqpoint{0.514278in}{1.789039in}}%
\pgfusepath{stroke}%
\end{pgfscope}%
\begin{pgfscope}%
\pgfsetrectcap%
\pgfsetmiterjoin%
\pgfsetlinewidth{0.803000pt}%
\definecolor{currentstroke}{rgb}{0.000000,0.000000,0.000000}%
\pgfsetstrokecolor{currentstroke}%
\pgfsetdash{}{0pt}%
\pgfpathmoveto{\pgfqpoint{2.399275in}{0.417642in}}%
\pgfpathlineto{\pgfqpoint{2.399275in}{1.789039in}}%
\pgfusepath{stroke}%
\end{pgfscope}%
\begin{pgfscope}%
\pgfsetrectcap%
\pgfsetmiterjoin%
\pgfsetlinewidth{0.803000pt}%
\definecolor{currentstroke}{rgb}{0.000000,0.000000,0.000000}%
\pgfsetstrokecolor{currentstroke}%
\pgfsetdash{}{0pt}%
\pgfpathmoveto{\pgfqpoint{0.514278in}{0.417642in}}%
\pgfpathlineto{\pgfqpoint{2.399275in}{0.417642in}}%
\pgfusepath{stroke}%
\end{pgfscope}%
\begin{pgfscope}%
\pgfsetrectcap%
\pgfsetmiterjoin%
\pgfsetlinewidth{0.803000pt}%
\definecolor{currentstroke}{rgb}{0.000000,0.000000,0.000000}%
\pgfsetstrokecolor{currentstroke}%
\pgfsetdash{}{0pt}%
\pgfpathmoveto{\pgfqpoint{0.514278in}{1.789039in}}%
\pgfpathlineto{\pgfqpoint{2.399275in}{1.789039in}}%
\pgfusepath{stroke}%
\end{pgfscope}%
\begin{pgfscope}%
\pgfsetbuttcap%
\pgfsetmiterjoin%
\definecolor{currentfill}{rgb}{1.000000,1.000000,1.000000}%
\pgfsetfillcolor{currentfill}%
\pgfsetfillopacity{0.800000}%
\pgfsetlinewidth{1.003750pt}%
\definecolor{currentstroke}{rgb}{0.800000,0.800000,0.800000}%
\pgfsetstrokecolor{currentstroke}%
\pgfsetstrokeopacity{0.800000}%
\pgfsetdash{}{0pt}%
\pgfpathmoveto{\pgfqpoint{1.552717in}{1.517728in}}%
\pgfpathlineto{\pgfqpoint{2.321497in}{1.517728in}}%
\pgfpathquadraticcurveto{\pgfqpoint{2.343719in}{1.517728in}}{\pgfqpoint{2.343719in}{1.539950in}}%
\pgfpathlineto{\pgfqpoint{2.343719in}{1.711261in}}%
\pgfpathquadraticcurveto{\pgfqpoint{2.343719in}{1.733483in}}{\pgfqpoint{2.321497in}{1.733483in}}%
\pgfpathlineto{\pgfqpoint{1.552717in}{1.733483in}}%
\pgfpathquadraticcurveto{\pgfqpoint{1.530494in}{1.733483in}}{\pgfqpoint{1.530494in}{1.711261in}}%
\pgfpathlineto{\pgfqpoint{1.530494in}{1.539950in}}%
\pgfpathquadraticcurveto{\pgfqpoint{1.530494in}{1.517728in}}{\pgfqpoint{1.552717in}{1.517728in}}%
\pgfpathlineto{\pgfqpoint{1.552717in}{1.517728in}}%
\pgfpathclose%
\pgfusepath{stroke,fill}%
\end{pgfscope}%
\begin{pgfscope}%
\pgfsetbuttcap%
\pgfsetroundjoin%
\pgfsetlinewidth{1.505625pt}%
\definecolor{currentstroke}{rgb}{0.007843,0.619608,0.450980}%
\pgfsetstrokecolor{currentstroke}%
\pgfsetdash{{5.550000pt}{2.400000pt}}{0.000000pt}%
\pgfpathmoveto{\pgfqpoint{1.574939in}{1.628067in}}%
\pgfpathlineto{\pgfqpoint{1.686050in}{1.628067in}}%
\pgfpathlineto{\pgfqpoint{1.797161in}{1.628067in}}%
\pgfusepath{stroke}%
\end{pgfscope}%
\begin{pgfscope}%
\definecolor{textcolor}{rgb}{0.000000,0.000000,0.000000}%
\pgfsetstrokecolor{textcolor}%
\pgfsetfillcolor{textcolor}%
\pgftext[x=1.886050in,y=1.589178in,left,base]{\color{textcolor}\rmfamily\fontsize{8.000000}{9.600000}\selectfont \(\displaystyle h_{-1}f^{-1}\)}%
\end{pgfscope}%
\end{pgfpicture}%
\makeatother%
\endgroup%

        } % scalebox
        \caption{Power spectral density}
        \label{fig:flicker_noise_psd}
    \end{subfigure}
    \begin{subfigure}{0.32\linewidth}
        \centering
        \scalebox{0.75}{%
            %% Creator: Matplotlib, PGF backend
%%
%% To include the figure in your LaTeX document, write
%%   \input{<filename>.pgf}
%%
%% Make sure the required packages are loaded in your preamble
%%   \usepackage{pgf}
%%
%% Also ensure that all the required font packages are loaded; for instance,
%% the lmodern package is sometimes necessary when using math font.
%%   \usepackage{lmodern}
%%
%% Figures using additional raster images can only be included by \input if
%% they are in the same directory as the main LaTeX file. For loading figures
%% from other directories you can use the `import` package
%%   \usepackage{import}
%%
%% and then include the figures with
%%   \import{<path to file>}{<filename>.pgf}
%%
%% Matplotlib used the following preamble
%%   \usepackage{siunitx}
%%   \usepackage{fontspec}
%%   \makeatletter\@ifpackageloaded{underscore}{}{\usepackage[strings]{underscore}}\makeatother
%%
\begingroup%
\makeatletter%
\begin{pgfpicture}%
\pgfpathrectangle{\pgfpointorigin}{\pgfqpoint{2.440945in}{1.830709in}}%
\pgfusepath{use as bounding box, clip}%
\begin{pgfscope}%
\pgfsetbuttcap%
\pgfsetmiterjoin%
\definecolor{currentfill}{rgb}{1.000000,1.000000,1.000000}%
\pgfsetfillcolor{currentfill}%
\pgfsetlinewidth{0.000000pt}%
\definecolor{currentstroke}{rgb}{1.000000,1.000000,1.000000}%
\pgfsetstrokecolor{currentstroke}%
\pgfsetdash{}{0pt}%
\pgfpathmoveto{\pgfqpoint{0.000000in}{0.000000in}}%
\pgfpathlineto{\pgfqpoint{2.440945in}{0.000000in}}%
\pgfpathlineto{\pgfqpoint{2.440945in}{1.830709in}}%
\pgfpathlineto{\pgfqpoint{0.000000in}{1.830709in}}%
\pgfpathlineto{\pgfqpoint{0.000000in}{0.000000in}}%
\pgfpathclose%
\pgfusepath{fill}%
\end{pgfscope}%
\begin{pgfscope}%
\pgfsetbuttcap%
\pgfsetmiterjoin%
\definecolor{currentfill}{rgb}{1.000000,1.000000,1.000000}%
\pgfsetfillcolor{currentfill}%
\pgfsetlinewidth{0.000000pt}%
\definecolor{currentstroke}{rgb}{0.000000,0.000000,0.000000}%
\pgfsetstrokecolor{currentstroke}%
\pgfsetstrokeopacity{0.000000}%
\pgfsetdash{}{0pt}%
\pgfpathmoveto{\pgfqpoint{0.589510in}{0.417642in}}%
\pgfpathlineto{\pgfqpoint{2.399275in}{0.417642in}}%
\pgfpathlineto{\pgfqpoint{2.399275in}{1.789039in}}%
\pgfpathlineto{\pgfqpoint{0.589510in}{1.789039in}}%
\pgfpathlineto{\pgfqpoint{0.589510in}{0.417642in}}%
\pgfpathclose%
\pgfusepath{fill}%
\end{pgfscope}%
\begin{pgfscope}%
\pgfpathrectangle{\pgfqpoint{0.589510in}{0.417642in}}{\pgfqpoint{1.809765in}{1.371397in}}%
\pgfusepath{clip}%
\pgfsetrectcap%
\pgfsetroundjoin%
\pgfsetlinewidth{0.803000pt}%
\definecolor{currentstroke}{rgb}{0.450000,0.450000,0.450000}%
\pgfsetstrokecolor{currentstroke}%
\pgfsetdash{}{0pt}%
\pgfpathmoveto{\pgfqpoint{0.671772in}{0.417642in}}%
\pgfpathlineto{\pgfqpoint{0.671772in}{1.789039in}}%
\pgfusepath{stroke}%
\end{pgfscope}%
\begin{pgfscope}%
\pgfsetbuttcap%
\pgfsetroundjoin%
\definecolor{currentfill}{rgb}{0.000000,0.000000,0.000000}%
\pgfsetfillcolor{currentfill}%
\pgfsetlinewidth{0.803000pt}%
\definecolor{currentstroke}{rgb}{0.000000,0.000000,0.000000}%
\pgfsetstrokecolor{currentstroke}%
\pgfsetdash{}{0pt}%
\pgfsys@defobject{currentmarker}{\pgfqpoint{0.000000in}{-0.048611in}}{\pgfqpoint{0.000000in}{0.000000in}}{%
\pgfpathmoveto{\pgfqpoint{0.000000in}{0.000000in}}%
\pgfpathlineto{\pgfqpoint{0.000000in}{-0.048611in}}%
\pgfusepath{stroke,fill}%
}%
\begin{pgfscope}%
\pgfsys@transformshift{0.671772in}{0.417642in}%
\pgfsys@useobject{currentmarker}{}%
\end{pgfscope}%
\end{pgfscope}%
\begin{pgfscope}%
\definecolor{textcolor}{rgb}{0.000000,0.000000,0.000000}%
\pgfsetstrokecolor{textcolor}%
\pgfsetfillcolor{textcolor}%
\pgftext[x=0.671772in,y=0.320420in,,top]{\color{textcolor}\rmfamily\fontsize{8.000000}{9.600000}\selectfont \(\displaystyle {10^{0}}\)}%
\end{pgfscope}%
\begin{pgfscope}%
\pgfpathrectangle{\pgfqpoint{0.589510in}{0.417642in}}{\pgfqpoint{1.809765in}{1.371397in}}%
\pgfusepath{clip}%
\pgfsetrectcap%
\pgfsetroundjoin%
\pgfsetlinewidth{0.803000pt}%
\definecolor{currentstroke}{rgb}{0.450000,0.450000,0.450000}%
\pgfsetstrokecolor{currentstroke}%
\pgfsetdash{}{0pt}%
\pgfpathmoveto{\pgfqpoint{1.128522in}{0.417642in}}%
\pgfpathlineto{\pgfqpoint{1.128522in}{1.789039in}}%
\pgfusepath{stroke}%
\end{pgfscope}%
\begin{pgfscope}%
\pgfsetbuttcap%
\pgfsetroundjoin%
\definecolor{currentfill}{rgb}{0.000000,0.000000,0.000000}%
\pgfsetfillcolor{currentfill}%
\pgfsetlinewidth{0.803000pt}%
\definecolor{currentstroke}{rgb}{0.000000,0.000000,0.000000}%
\pgfsetstrokecolor{currentstroke}%
\pgfsetdash{}{0pt}%
\pgfsys@defobject{currentmarker}{\pgfqpoint{0.000000in}{-0.048611in}}{\pgfqpoint{0.000000in}{0.000000in}}{%
\pgfpathmoveto{\pgfqpoint{0.000000in}{0.000000in}}%
\pgfpathlineto{\pgfqpoint{0.000000in}{-0.048611in}}%
\pgfusepath{stroke,fill}%
}%
\begin{pgfscope}%
\pgfsys@transformshift{1.128522in}{0.417642in}%
\pgfsys@useobject{currentmarker}{}%
\end{pgfscope}%
\end{pgfscope}%
\begin{pgfscope}%
\definecolor{textcolor}{rgb}{0.000000,0.000000,0.000000}%
\pgfsetstrokecolor{textcolor}%
\pgfsetfillcolor{textcolor}%
\pgftext[x=1.128522in,y=0.320420in,,top]{\color{textcolor}\rmfamily\fontsize{8.000000}{9.600000}\selectfont \(\displaystyle {10^{1}}\)}%
\end{pgfscope}%
\begin{pgfscope}%
\pgfpathrectangle{\pgfqpoint{0.589510in}{0.417642in}}{\pgfqpoint{1.809765in}{1.371397in}}%
\pgfusepath{clip}%
\pgfsetrectcap%
\pgfsetroundjoin%
\pgfsetlinewidth{0.803000pt}%
\definecolor{currentstroke}{rgb}{0.450000,0.450000,0.450000}%
\pgfsetstrokecolor{currentstroke}%
\pgfsetdash{}{0pt}%
\pgfpathmoveto{\pgfqpoint{1.585272in}{0.417642in}}%
\pgfpathlineto{\pgfqpoint{1.585272in}{1.789039in}}%
\pgfusepath{stroke}%
\end{pgfscope}%
\begin{pgfscope}%
\pgfsetbuttcap%
\pgfsetroundjoin%
\definecolor{currentfill}{rgb}{0.000000,0.000000,0.000000}%
\pgfsetfillcolor{currentfill}%
\pgfsetlinewidth{0.803000pt}%
\definecolor{currentstroke}{rgb}{0.000000,0.000000,0.000000}%
\pgfsetstrokecolor{currentstroke}%
\pgfsetdash{}{0pt}%
\pgfsys@defobject{currentmarker}{\pgfqpoint{0.000000in}{-0.048611in}}{\pgfqpoint{0.000000in}{0.000000in}}{%
\pgfpathmoveto{\pgfqpoint{0.000000in}{0.000000in}}%
\pgfpathlineto{\pgfqpoint{0.000000in}{-0.048611in}}%
\pgfusepath{stroke,fill}%
}%
\begin{pgfscope}%
\pgfsys@transformshift{1.585272in}{0.417642in}%
\pgfsys@useobject{currentmarker}{}%
\end{pgfscope}%
\end{pgfscope}%
\begin{pgfscope}%
\definecolor{textcolor}{rgb}{0.000000,0.000000,0.000000}%
\pgfsetstrokecolor{textcolor}%
\pgfsetfillcolor{textcolor}%
\pgftext[x=1.585272in,y=0.320420in,,top]{\color{textcolor}\rmfamily\fontsize{8.000000}{9.600000}\selectfont \(\displaystyle {10^{2}}\)}%
\end{pgfscope}%
\begin{pgfscope}%
\pgfpathrectangle{\pgfqpoint{0.589510in}{0.417642in}}{\pgfqpoint{1.809765in}{1.371397in}}%
\pgfusepath{clip}%
\pgfsetrectcap%
\pgfsetroundjoin%
\pgfsetlinewidth{0.803000pt}%
\definecolor{currentstroke}{rgb}{0.450000,0.450000,0.450000}%
\pgfsetstrokecolor{currentstroke}%
\pgfsetdash{}{0pt}%
\pgfpathmoveto{\pgfqpoint{2.042022in}{0.417642in}}%
\pgfpathlineto{\pgfqpoint{2.042022in}{1.789039in}}%
\pgfusepath{stroke}%
\end{pgfscope}%
\begin{pgfscope}%
\pgfsetbuttcap%
\pgfsetroundjoin%
\definecolor{currentfill}{rgb}{0.000000,0.000000,0.000000}%
\pgfsetfillcolor{currentfill}%
\pgfsetlinewidth{0.803000pt}%
\definecolor{currentstroke}{rgb}{0.000000,0.000000,0.000000}%
\pgfsetstrokecolor{currentstroke}%
\pgfsetdash{}{0pt}%
\pgfsys@defobject{currentmarker}{\pgfqpoint{0.000000in}{-0.048611in}}{\pgfqpoint{0.000000in}{0.000000in}}{%
\pgfpathmoveto{\pgfqpoint{0.000000in}{0.000000in}}%
\pgfpathlineto{\pgfqpoint{0.000000in}{-0.048611in}}%
\pgfusepath{stroke,fill}%
}%
\begin{pgfscope}%
\pgfsys@transformshift{2.042022in}{0.417642in}%
\pgfsys@useobject{currentmarker}{}%
\end{pgfscope}%
\end{pgfscope}%
\begin{pgfscope}%
\definecolor{textcolor}{rgb}{0.000000,0.000000,0.000000}%
\pgfsetstrokecolor{textcolor}%
\pgfsetfillcolor{textcolor}%
\pgftext[x=2.042022in,y=0.320420in,,top]{\color{textcolor}\rmfamily\fontsize{8.000000}{9.600000}\selectfont \(\displaystyle {10^{3}}\)}%
\end{pgfscope}%
\begin{pgfscope}%
\pgfpathrectangle{\pgfqpoint{0.589510in}{0.417642in}}{\pgfqpoint{1.809765in}{1.371397in}}%
\pgfusepath{clip}%
\pgfsetrectcap%
\pgfsetroundjoin%
\pgfsetlinewidth{0.803000pt}%
\definecolor{currentstroke}{rgb}{0.850000,0.850000,0.850000}%
\pgfsetstrokecolor{currentstroke}%
\pgfsetdash{}{0pt}%
\pgfpathmoveto{\pgfqpoint{0.601020in}{0.417642in}}%
\pgfpathlineto{\pgfqpoint{0.601020in}{1.789039in}}%
\pgfusepath{stroke}%
\end{pgfscope}%
\begin{pgfscope}%
\pgfsetbuttcap%
\pgfsetroundjoin%
\definecolor{currentfill}{rgb}{0.000000,0.000000,0.000000}%
\pgfsetfillcolor{currentfill}%
\pgfsetlinewidth{0.602250pt}%
\definecolor{currentstroke}{rgb}{0.000000,0.000000,0.000000}%
\pgfsetstrokecolor{currentstroke}%
\pgfsetdash{}{0pt}%
\pgfsys@defobject{currentmarker}{\pgfqpoint{0.000000in}{-0.027778in}}{\pgfqpoint{0.000000in}{0.000000in}}{%
\pgfpathmoveto{\pgfqpoint{0.000000in}{0.000000in}}%
\pgfpathlineto{\pgfqpoint{0.000000in}{-0.027778in}}%
\pgfusepath{stroke,fill}%
}%
\begin{pgfscope}%
\pgfsys@transformshift{0.601020in}{0.417642in}%
\pgfsys@useobject{currentmarker}{}%
\end{pgfscope}%
\end{pgfscope}%
\begin{pgfscope}%
\pgfpathrectangle{\pgfqpoint{0.589510in}{0.417642in}}{\pgfqpoint{1.809765in}{1.371397in}}%
\pgfusepath{clip}%
\pgfsetrectcap%
\pgfsetroundjoin%
\pgfsetlinewidth{0.803000pt}%
\definecolor{currentstroke}{rgb}{0.850000,0.850000,0.850000}%
\pgfsetstrokecolor{currentstroke}%
\pgfsetdash{}{0pt}%
\pgfpathmoveto{\pgfqpoint{0.627508in}{0.417642in}}%
\pgfpathlineto{\pgfqpoint{0.627508in}{1.789039in}}%
\pgfusepath{stroke}%
\end{pgfscope}%
\begin{pgfscope}%
\pgfsetbuttcap%
\pgfsetroundjoin%
\definecolor{currentfill}{rgb}{0.000000,0.000000,0.000000}%
\pgfsetfillcolor{currentfill}%
\pgfsetlinewidth{0.602250pt}%
\definecolor{currentstroke}{rgb}{0.000000,0.000000,0.000000}%
\pgfsetstrokecolor{currentstroke}%
\pgfsetdash{}{0pt}%
\pgfsys@defobject{currentmarker}{\pgfqpoint{0.000000in}{-0.027778in}}{\pgfqpoint{0.000000in}{0.000000in}}{%
\pgfpathmoveto{\pgfqpoint{0.000000in}{0.000000in}}%
\pgfpathlineto{\pgfqpoint{0.000000in}{-0.027778in}}%
\pgfusepath{stroke,fill}%
}%
\begin{pgfscope}%
\pgfsys@transformshift{0.627508in}{0.417642in}%
\pgfsys@useobject{currentmarker}{}%
\end{pgfscope}%
\end{pgfscope}%
\begin{pgfscope}%
\pgfpathrectangle{\pgfqpoint{0.589510in}{0.417642in}}{\pgfqpoint{1.809765in}{1.371397in}}%
\pgfusepath{clip}%
\pgfsetrectcap%
\pgfsetroundjoin%
\pgfsetlinewidth{0.803000pt}%
\definecolor{currentstroke}{rgb}{0.850000,0.850000,0.850000}%
\pgfsetstrokecolor{currentstroke}%
\pgfsetdash{}{0pt}%
\pgfpathmoveto{\pgfqpoint{0.650872in}{0.417642in}}%
\pgfpathlineto{\pgfqpoint{0.650872in}{1.789039in}}%
\pgfusepath{stroke}%
\end{pgfscope}%
\begin{pgfscope}%
\pgfsetbuttcap%
\pgfsetroundjoin%
\definecolor{currentfill}{rgb}{0.000000,0.000000,0.000000}%
\pgfsetfillcolor{currentfill}%
\pgfsetlinewidth{0.602250pt}%
\definecolor{currentstroke}{rgb}{0.000000,0.000000,0.000000}%
\pgfsetstrokecolor{currentstroke}%
\pgfsetdash{}{0pt}%
\pgfsys@defobject{currentmarker}{\pgfqpoint{0.000000in}{-0.027778in}}{\pgfqpoint{0.000000in}{0.000000in}}{%
\pgfpathmoveto{\pgfqpoint{0.000000in}{0.000000in}}%
\pgfpathlineto{\pgfqpoint{0.000000in}{-0.027778in}}%
\pgfusepath{stroke,fill}%
}%
\begin{pgfscope}%
\pgfsys@transformshift{0.650872in}{0.417642in}%
\pgfsys@useobject{currentmarker}{}%
\end{pgfscope}%
\end{pgfscope}%
\begin{pgfscope}%
\pgfpathrectangle{\pgfqpoint{0.589510in}{0.417642in}}{\pgfqpoint{1.809765in}{1.371397in}}%
\pgfusepath{clip}%
\pgfsetrectcap%
\pgfsetroundjoin%
\pgfsetlinewidth{0.803000pt}%
\definecolor{currentstroke}{rgb}{0.850000,0.850000,0.850000}%
\pgfsetstrokecolor{currentstroke}%
\pgfsetdash{}{0pt}%
\pgfpathmoveto{\pgfqpoint{0.809267in}{0.417642in}}%
\pgfpathlineto{\pgfqpoint{0.809267in}{1.789039in}}%
\pgfusepath{stroke}%
\end{pgfscope}%
\begin{pgfscope}%
\pgfsetbuttcap%
\pgfsetroundjoin%
\definecolor{currentfill}{rgb}{0.000000,0.000000,0.000000}%
\pgfsetfillcolor{currentfill}%
\pgfsetlinewidth{0.602250pt}%
\definecolor{currentstroke}{rgb}{0.000000,0.000000,0.000000}%
\pgfsetstrokecolor{currentstroke}%
\pgfsetdash{}{0pt}%
\pgfsys@defobject{currentmarker}{\pgfqpoint{0.000000in}{-0.027778in}}{\pgfqpoint{0.000000in}{0.000000in}}{%
\pgfpathmoveto{\pgfqpoint{0.000000in}{0.000000in}}%
\pgfpathlineto{\pgfqpoint{0.000000in}{-0.027778in}}%
\pgfusepath{stroke,fill}%
}%
\begin{pgfscope}%
\pgfsys@transformshift{0.809267in}{0.417642in}%
\pgfsys@useobject{currentmarker}{}%
\end{pgfscope}%
\end{pgfscope}%
\begin{pgfscope}%
\pgfpathrectangle{\pgfqpoint{0.589510in}{0.417642in}}{\pgfqpoint{1.809765in}{1.371397in}}%
\pgfusepath{clip}%
\pgfsetrectcap%
\pgfsetroundjoin%
\pgfsetlinewidth{0.803000pt}%
\definecolor{currentstroke}{rgb}{0.850000,0.850000,0.850000}%
\pgfsetstrokecolor{currentstroke}%
\pgfsetdash{}{0pt}%
\pgfpathmoveto{\pgfqpoint{0.889697in}{0.417642in}}%
\pgfpathlineto{\pgfqpoint{0.889697in}{1.789039in}}%
\pgfusepath{stroke}%
\end{pgfscope}%
\begin{pgfscope}%
\pgfsetbuttcap%
\pgfsetroundjoin%
\definecolor{currentfill}{rgb}{0.000000,0.000000,0.000000}%
\pgfsetfillcolor{currentfill}%
\pgfsetlinewidth{0.602250pt}%
\definecolor{currentstroke}{rgb}{0.000000,0.000000,0.000000}%
\pgfsetstrokecolor{currentstroke}%
\pgfsetdash{}{0pt}%
\pgfsys@defobject{currentmarker}{\pgfqpoint{0.000000in}{-0.027778in}}{\pgfqpoint{0.000000in}{0.000000in}}{%
\pgfpathmoveto{\pgfqpoint{0.000000in}{0.000000in}}%
\pgfpathlineto{\pgfqpoint{0.000000in}{-0.027778in}}%
\pgfusepath{stroke,fill}%
}%
\begin{pgfscope}%
\pgfsys@transformshift{0.889697in}{0.417642in}%
\pgfsys@useobject{currentmarker}{}%
\end{pgfscope}%
\end{pgfscope}%
\begin{pgfscope}%
\pgfpathrectangle{\pgfqpoint{0.589510in}{0.417642in}}{\pgfqpoint{1.809765in}{1.371397in}}%
\pgfusepath{clip}%
\pgfsetrectcap%
\pgfsetroundjoin%
\pgfsetlinewidth{0.803000pt}%
\definecolor{currentstroke}{rgb}{0.850000,0.850000,0.850000}%
\pgfsetstrokecolor{currentstroke}%
\pgfsetdash{}{0pt}%
\pgfpathmoveto{\pgfqpoint{0.946763in}{0.417642in}}%
\pgfpathlineto{\pgfqpoint{0.946763in}{1.789039in}}%
\pgfusepath{stroke}%
\end{pgfscope}%
\begin{pgfscope}%
\pgfsetbuttcap%
\pgfsetroundjoin%
\definecolor{currentfill}{rgb}{0.000000,0.000000,0.000000}%
\pgfsetfillcolor{currentfill}%
\pgfsetlinewidth{0.602250pt}%
\definecolor{currentstroke}{rgb}{0.000000,0.000000,0.000000}%
\pgfsetstrokecolor{currentstroke}%
\pgfsetdash{}{0pt}%
\pgfsys@defobject{currentmarker}{\pgfqpoint{0.000000in}{-0.027778in}}{\pgfqpoint{0.000000in}{0.000000in}}{%
\pgfpathmoveto{\pgfqpoint{0.000000in}{0.000000in}}%
\pgfpathlineto{\pgfqpoint{0.000000in}{-0.027778in}}%
\pgfusepath{stroke,fill}%
}%
\begin{pgfscope}%
\pgfsys@transformshift{0.946763in}{0.417642in}%
\pgfsys@useobject{currentmarker}{}%
\end{pgfscope}%
\end{pgfscope}%
\begin{pgfscope}%
\pgfpathrectangle{\pgfqpoint{0.589510in}{0.417642in}}{\pgfqpoint{1.809765in}{1.371397in}}%
\pgfusepath{clip}%
\pgfsetrectcap%
\pgfsetroundjoin%
\pgfsetlinewidth{0.803000pt}%
\definecolor{currentstroke}{rgb}{0.850000,0.850000,0.850000}%
\pgfsetstrokecolor{currentstroke}%
\pgfsetdash{}{0pt}%
\pgfpathmoveto{\pgfqpoint{0.991026in}{0.417642in}}%
\pgfpathlineto{\pgfqpoint{0.991026in}{1.789039in}}%
\pgfusepath{stroke}%
\end{pgfscope}%
\begin{pgfscope}%
\pgfsetbuttcap%
\pgfsetroundjoin%
\definecolor{currentfill}{rgb}{0.000000,0.000000,0.000000}%
\pgfsetfillcolor{currentfill}%
\pgfsetlinewidth{0.602250pt}%
\definecolor{currentstroke}{rgb}{0.000000,0.000000,0.000000}%
\pgfsetstrokecolor{currentstroke}%
\pgfsetdash{}{0pt}%
\pgfsys@defobject{currentmarker}{\pgfqpoint{0.000000in}{-0.027778in}}{\pgfqpoint{0.000000in}{0.000000in}}{%
\pgfpathmoveto{\pgfqpoint{0.000000in}{0.000000in}}%
\pgfpathlineto{\pgfqpoint{0.000000in}{-0.027778in}}%
\pgfusepath{stroke,fill}%
}%
\begin{pgfscope}%
\pgfsys@transformshift{0.991026in}{0.417642in}%
\pgfsys@useobject{currentmarker}{}%
\end{pgfscope}%
\end{pgfscope}%
\begin{pgfscope}%
\pgfpathrectangle{\pgfqpoint{0.589510in}{0.417642in}}{\pgfqpoint{1.809765in}{1.371397in}}%
\pgfusepath{clip}%
\pgfsetrectcap%
\pgfsetroundjoin%
\pgfsetlinewidth{0.803000pt}%
\definecolor{currentstroke}{rgb}{0.850000,0.850000,0.850000}%
\pgfsetstrokecolor{currentstroke}%
\pgfsetdash{}{0pt}%
\pgfpathmoveto{\pgfqpoint{1.027192in}{0.417642in}}%
\pgfpathlineto{\pgfqpoint{1.027192in}{1.789039in}}%
\pgfusepath{stroke}%
\end{pgfscope}%
\begin{pgfscope}%
\pgfsetbuttcap%
\pgfsetroundjoin%
\definecolor{currentfill}{rgb}{0.000000,0.000000,0.000000}%
\pgfsetfillcolor{currentfill}%
\pgfsetlinewidth{0.602250pt}%
\definecolor{currentstroke}{rgb}{0.000000,0.000000,0.000000}%
\pgfsetstrokecolor{currentstroke}%
\pgfsetdash{}{0pt}%
\pgfsys@defobject{currentmarker}{\pgfqpoint{0.000000in}{-0.027778in}}{\pgfqpoint{0.000000in}{0.000000in}}{%
\pgfpathmoveto{\pgfqpoint{0.000000in}{0.000000in}}%
\pgfpathlineto{\pgfqpoint{0.000000in}{-0.027778in}}%
\pgfusepath{stroke,fill}%
}%
\begin{pgfscope}%
\pgfsys@transformshift{1.027192in}{0.417642in}%
\pgfsys@useobject{currentmarker}{}%
\end{pgfscope}%
\end{pgfscope}%
\begin{pgfscope}%
\pgfpathrectangle{\pgfqpoint{0.589510in}{0.417642in}}{\pgfqpoint{1.809765in}{1.371397in}}%
\pgfusepath{clip}%
\pgfsetrectcap%
\pgfsetroundjoin%
\pgfsetlinewidth{0.803000pt}%
\definecolor{currentstroke}{rgb}{0.850000,0.850000,0.850000}%
\pgfsetstrokecolor{currentstroke}%
\pgfsetdash{}{0pt}%
\pgfpathmoveto{\pgfqpoint{1.057770in}{0.417642in}}%
\pgfpathlineto{\pgfqpoint{1.057770in}{1.789039in}}%
\pgfusepath{stroke}%
\end{pgfscope}%
\begin{pgfscope}%
\pgfsetbuttcap%
\pgfsetroundjoin%
\definecolor{currentfill}{rgb}{0.000000,0.000000,0.000000}%
\pgfsetfillcolor{currentfill}%
\pgfsetlinewidth{0.602250pt}%
\definecolor{currentstroke}{rgb}{0.000000,0.000000,0.000000}%
\pgfsetstrokecolor{currentstroke}%
\pgfsetdash{}{0pt}%
\pgfsys@defobject{currentmarker}{\pgfqpoint{0.000000in}{-0.027778in}}{\pgfqpoint{0.000000in}{0.000000in}}{%
\pgfpathmoveto{\pgfqpoint{0.000000in}{0.000000in}}%
\pgfpathlineto{\pgfqpoint{0.000000in}{-0.027778in}}%
\pgfusepath{stroke,fill}%
}%
\begin{pgfscope}%
\pgfsys@transformshift{1.057770in}{0.417642in}%
\pgfsys@useobject{currentmarker}{}%
\end{pgfscope}%
\end{pgfscope}%
\begin{pgfscope}%
\pgfpathrectangle{\pgfqpoint{0.589510in}{0.417642in}}{\pgfqpoint{1.809765in}{1.371397in}}%
\pgfusepath{clip}%
\pgfsetrectcap%
\pgfsetroundjoin%
\pgfsetlinewidth{0.803000pt}%
\definecolor{currentstroke}{rgb}{0.850000,0.850000,0.850000}%
\pgfsetstrokecolor{currentstroke}%
\pgfsetdash{}{0pt}%
\pgfpathmoveto{\pgfqpoint{1.084258in}{0.417642in}}%
\pgfpathlineto{\pgfqpoint{1.084258in}{1.789039in}}%
\pgfusepath{stroke}%
\end{pgfscope}%
\begin{pgfscope}%
\pgfsetbuttcap%
\pgfsetroundjoin%
\definecolor{currentfill}{rgb}{0.000000,0.000000,0.000000}%
\pgfsetfillcolor{currentfill}%
\pgfsetlinewidth{0.602250pt}%
\definecolor{currentstroke}{rgb}{0.000000,0.000000,0.000000}%
\pgfsetstrokecolor{currentstroke}%
\pgfsetdash{}{0pt}%
\pgfsys@defobject{currentmarker}{\pgfqpoint{0.000000in}{-0.027778in}}{\pgfqpoint{0.000000in}{0.000000in}}{%
\pgfpathmoveto{\pgfqpoint{0.000000in}{0.000000in}}%
\pgfpathlineto{\pgfqpoint{0.000000in}{-0.027778in}}%
\pgfusepath{stroke,fill}%
}%
\begin{pgfscope}%
\pgfsys@transformshift{1.084258in}{0.417642in}%
\pgfsys@useobject{currentmarker}{}%
\end{pgfscope}%
\end{pgfscope}%
\begin{pgfscope}%
\pgfpathrectangle{\pgfqpoint{0.589510in}{0.417642in}}{\pgfqpoint{1.809765in}{1.371397in}}%
\pgfusepath{clip}%
\pgfsetrectcap%
\pgfsetroundjoin%
\pgfsetlinewidth{0.803000pt}%
\definecolor{currentstroke}{rgb}{0.850000,0.850000,0.850000}%
\pgfsetstrokecolor{currentstroke}%
\pgfsetdash{}{0pt}%
\pgfpathmoveto{\pgfqpoint{1.107622in}{0.417642in}}%
\pgfpathlineto{\pgfqpoint{1.107622in}{1.789039in}}%
\pgfusepath{stroke}%
\end{pgfscope}%
\begin{pgfscope}%
\pgfsetbuttcap%
\pgfsetroundjoin%
\definecolor{currentfill}{rgb}{0.000000,0.000000,0.000000}%
\pgfsetfillcolor{currentfill}%
\pgfsetlinewidth{0.602250pt}%
\definecolor{currentstroke}{rgb}{0.000000,0.000000,0.000000}%
\pgfsetstrokecolor{currentstroke}%
\pgfsetdash{}{0pt}%
\pgfsys@defobject{currentmarker}{\pgfqpoint{0.000000in}{-0.027778in}}{\pgfqpoint{0.000000in}{0.000000in}}{%
\pgfpathmoveto{\pgfqpoint{0.000000in}{0.000000in}}%
\pgfpathlineto{\pgfqpoint{0.000000in}{-0.027778in}}%
\pgfusepath{stroke,fill}%
}%
\begin{pgfscope}%
\pgfsys@transformshift{1.107622in}{0.417642in}%
\pgfsys@useobject{currentmarker}{}%
\end{pgfscope}%
\end{pgfscope}%
\begin{pgfscope}%
\pgfpathrectangle{\pgfqpoint{0.589510in}{0.417642in}}{\pgfqpoint{1.809765in}{1.371397in}}%
\pgfusepath{clip}%
\pgfsetrectcap%
\pgfsetroundjoin%
\pgfsetlinewidth{0.803000pt}%
\definecolor{currentstroke}{rgb}{0.850000,0.850000,0.850000}%
\pgfsetstrokecolor{currentstroke}%
\pgfsetdash{}{0pt}%
\pgfpathmoveto{\pgfqpoint{1.266017in}{0.417642in}}%
\pgfpathlineto{\pgfqpoint{1.266017in}{1.789039in}}%
\pgfusepath{stroke}%
\end{pgfscope}%
\begin{pgfscope}%
\pgfsetbuttcap%
\pgfsetroundjoin%
\definecolor{currentfill}{rgb}{0.000000,0.000000,0.000000}%
\pgfsetfillcolor{currentfill}%
\pgfsetlinewidth{0.602250pt}%
\definecolor{currentstroke}{rgb}{0.000000,0.000000,0.000000}%
\pgfsetstrokecolor{currentstroke}%
\pgfsetdash{}{0pt}%
\pgfsys@defobject{currentmarker}{\pgfqpoint{0.000000in}{-0.027778in}}{\pgfqpoint{0.000000in}{0.000000in}}{%
\pgfpathmoveto{\pgfqpoint{0.000000in}{0.000000in}}%
\pgfpathlineto{\pgfqpoint{0.000000in}{-0.027778in}}%
\pgfusepath{stroke,fill}%
}%
\begin{pgfscope}%
\pgfsys@transformshift{1.266017in}{0.417642in}%
\pgfsys@useobject{currentmarker}{}%
\end{pgfscope}%
\end{pgfscope}%
\begin{pgfscope}%
\pgfpathrectangle{\pgfqpoint{0.589510in}{0.417642in}}{\pgfqpoint{1.809765in}{1.371397in}}%
\pgfusepath{clip}%
\pgfsetrectcap%
\pgfsetroundjoin%
\pgfsetlinewidth{0.803000pt}%
\definecolor{currentstroke}{rgb}{0.850000,0.850000,0.850000}%
\pgfsetstrokecolor{currentstroke}%
\pgfsetdash{}{0pt}%
\pgfpathmoveto{\pgfqpoint{1.346447in}{0.417642in}}%
\pgfpathlineto{\pgfqpoint{1.346447in}{1.789039in}}%
\pgfusepath{stroke}%
\end{pgfscope}%
\begin{pgfscope}%
\pgfsetbuttcap%
\pgfsetroundjoin%
\definecolor{currentfill}{rgb}{0.000000,0.000000,0.000000}%
\pgfsetfillcolor{currentfill}%
\pgfsetlinewidth{0.602250pt}%
\definecolor{currentstroke}{rgb}{0.000000,0.000000,0.000000}%
\pgfsetstrokecolor{currentstroke}%
\pgfsetdash{}{0pt}%
\pgfsys@defobject{currentmarker}{\pgfqpoint{0.000000in}{-0.027778in}}{\pgfqpoint{0.000000in}{0.000000in}}{%
\pgfpathmoveto{\pgfqpoint{0.000000in}{0.000000in}}%
\pgfpathlineto{\pgfqpoint{0.000000in}{-0.027778in}}%
\pgfusepath{stroke,fill}%
}%
\begin{pgfscope}%
\pgfsys@transformshift{1.346447in}{0.417642in}%
\pgfsys@useobject{currentmarker}{}%
\end{pgfscope}%
\end{pgfscope}%
\begin{pgfscope}%
\pgfpathrectangle{\pgfqpoint{0.589510in}{0.417642in}}{\pgfqpoint{1.809765in}{1.371397in}}%
\pgfusepath{clip}%
\pgfsetrectcap%
\pgfsetroundjoin%
\pgfsetlinewidth{0.803000pt}%
\definecolor{currentstroke}{rgb}{0.850000,0.850000,0.850000}%
\pgfsetstrokecolor{currentstroke}%
\pgfsetdash{}{0pt}%
\pgfpathmoveto{\pgfqpoint{1.403513in}{0.417642in}}%
\pgfpathlineto{\pgfqpoint{1.403513in}{1.789039in}}%
\pgfusepath{stroke}%
\end{pgfscope}%
\begin{pgfscope}%
\pgfsetbuttcap%
\pgfsetroundjoin%
\definecolor{currentfill}{rgb}{0.000000,0.000000,0.000000}%
\pgfsetfillcolor{currentfill}%
\pgfsetlinewidth{0.602250pt}%
\definecolor{currentstroke}{rgb}{0.000000,0.000000,0.000000}%
\pgfsetstrokecolor{currentstroke}%
\pgfsetdash{}{0pt}%
\pgfsys@defobject{currentmarker}{\pgfqpoint{0.000000in}{-0.027778in}}{\pgfqpoint{0.000000in}{0.000000in}}{%
\pgfpathmoveto{\pgfqpoint{0.000000in}{0.000000in}}%
\pgfpathlineto{\pgfqpoint{0.000000in}{-0.027778in}}%
\pgfusepath{stroke,fill}%
}%
\begin{pgfscope}%
\pgfsys@transformshift{1.403513in}{0.417642in}%
\pgfsys@useobject{currentmarker}{}%
\end{pgfscope}%
\end{pgfscope}%
\begin{pgfscope}%
\pgfpathrectangle{\pgfqpoint{0.589510in}{0.417642in}}{\pgfqpoint{1.809765in}{1.371397in}}%
\pgfusepath{clip}%
\pgfsetrectcap%
\pgfsetroundjoin%
\pgfsetlinewidth{0.803000pt}%
\definecolor{currentstroke}{rgb}{0.850000,0.850000,0.850000}%
\pgfsetstrokecolor{currentstroke}%
\pgfsetdash{}{0pt}%
\pgfpathmoveto{\pgfqpoint{1.447776in}{0.417642in}}%
\pgfpathlineto{\pgfqpoint{1.447776in}{1.789039in}}%
\pgfusepath{stroke}%
\end{pgfscope}%
\begin{pgfscope}%
\pgfsetbuttcap%
\pgfsetroundjoin%
\definecolor{currentfill}{rgb}{0.000000,0.000000,0.000000}%
\pgfsetfillcolor{currentfill}%
\pgfsetlinewidth{0.602250pt}%
\definecolor{currentstroke}{rgb}{0.000000,0.000000,0.000000}%
\pgfsetstrokecolor{currentstroke}%
\pgfsetdash{}{0pt}%
\pgfsys@defobject{currentmarker}{\pgfqpoint{0.000000in}{-0.027778in}}{\pgfqpoint{0.000000in}{0.000000in}}{%
\pgfpathmoveto{\pgfqpoint{0.000000in}{0.000000in}}%
\pgfpathlineto{\pgfqpoint{0.000000in}{-0.027778in}}%
\pgfusepath{stroke,fill}%
}%
\begin{pgfscope}%
\pgfsys@transformshift{1.447776in}{0.417642in}%
\pgfsys@useobject{currentmarker}{}%
\end{pgfscope}%
\end{pgfscope}%
\begin{pgfscope}%
\pgfpathrectangle{\pgfqpoint{0.589510in}{0.417642in}}{\pgfqpoint{1.809765in}{1.371397in}}%
\pgfusepath{clip}%
\pgfsetrectcap%
\pgfsetroundjoin%
\pgfsetlinewidth{0.803000pt}%
\definecolor{currentstroke}{rgb}{0.850000,0.850000,0.850000}%
\pgfsetstrokecolor{currentstroke}%
\pgfsetdash{}{0pt}%
\pgfpathmoveto{\pgfqpoint{1.483942in}{0.417642in}}%
\pgfpathlineto{\pgfqpoint{1.483942in}{1.789039in}}%
\pgfusepath{stroke}%
\end{pgfscope}%
\begin{pgfscope}%
\pgfsetbuttcap%
\pgfsetroundjoin%
\definecolor{currentfill}{rgb}{0.000000,0.000000,0.000000}%
\pgfsetfillcolor{currentfill}%
\pgfsetlinewidth{0.602250pt}%
\definecolor{currentstroke}{rgb}{0.000000,0.000000,0.000000}%
\pgfsetstrokecolor{currentstroke}%
\pgfsetdash{}{0pt}%
\pgfsys@defobject{currentmarker}{\pgfqpoint{0.000000in}{-0.027778in}}{\pgfqpoint{0.000000in}{0.000000in}}{%
\pgfpathmoveto{\pgfqpoint{0.000000in}{0.000000in}}%
\pgfpathlineto{\pgfqpoint{0.000000in}{-0.027778in}}%
\pgfusepath{stroke,fill}%
}%
\begin{pgfscope}%
\pgfsys@transformshift{1.483942in}{0.417642in}%
\pgfsys@useobject{currentmarker}{}%
\end{pgfscope}%
\end{pgfscope}%
\begin{pgfscope}%
\pgfpathrectangle{\pgfqpoint{0.589510in}{0.417642in}}{\pgfqpoint{1.809765in}{1.371397in}}%
\pgfusepath{clip}%
\pgfsetrectcap%
\pgfsetroundjoin%
\pgfsetlinewidth{0.803000pt}%
\definecolor{currentstroke}{rgb}{0.850000,0.850000,0.850000}%
\pgfsetstrokecolor{currentstroke}%
\pgfsetdash{}{0pt}%
\pgfpathmoveto{\pgfqpoint{1.514520in}{0.417642in}}%
\pgfpathlineto{\pgfqpoint{1.514520in}{1.789039in}}%
\pgfusepath{stroke}%
\end{pgfscope}%
\begin{pgfscope}%
\pgfsetbuttcap%
\pgfsetroundjoin%
\definecolor{currentfill}{rgb}{0.000000,0.000000,0.000000}%
\pgfsetfillcolor{currentfill}%
\pgfsetlinewidth{0.602250pt}%
\definecolor{currentstroke}{rgb}{0.000000,0.000000,0.000000}%
\pgfsetstrokecolor{currentstroke}%
\pgfsetdash{}{0pt}%
\pgfsys@defobject{currentmarker}{\pgfqpoint{0.000000in}{-0.027778in}}{\pgfqpoint{0.000000in}{0.000000in}}{%
\pgfpathmoveto{\pgfqpoint{0.000000in}{0.000000in}}%
\pgfpathlineto{\pgfqpoint{0.000000in}{-0.027778in}}%
\pgfusepath{stroke,fill}%
}%
\begin{pgfscope}%
\pgfsys@transformshift{1.514520in}{0.417642in}%
\pgfsys@useobject{currentmarker}{}%
\end{pgfscope}%
\end{pgfscope}%
\begin{pgfscope}%
\pgfpathrectangle{\pgfqpoint{0.589510in}{0.417642in}}{\pgfqpoint{1.809765in}{1.371397in}}%
\pgfusepath{clip}%
\pgfsetrectcap%
\pgfsetroundjoin%
\pgfsetlinewidth{0.803000pt}%
\definecolor{currentstroke}{rgb}{0.850000,0.850000,0.850000}%
\pgfsetstrokecolor{currentstroke}%
\pgfsetdash{}{0pt}%
\pgfpathmoveto{\pgfqpoint{1.541008in}{0.417642in}}%
\pgfpathlineto{\pgfqpoint{1.541008in}{1.789039in}}%
\pgfusepath{stroke}%
\end{pgfscope}%
\begin{pgfscope}%
\pgfsetbuttcap%
\pgfsetroundjoin%
\definecolor{currentfill}{rgb}{0.000000,0.000000,0.000000}%
\pgfsetfillcolor{currentfill}%
\pgfsetlinewidth{0.602250pt}%
\definecolor{currentstroke}{rgb}{0.000000,0.000000,0.000000}%
\pgfsetstrokecolor{currentstroke}%
\pgfsetdash{}{0pt}%
\pgfsys@defobject{currentmarker}{\pgfqpoint{0.000000in}{-0.027778in}}{\pgfqpoint{0.000000in}{0.000000in}}{%
\pgfpathmoveto{\pgfqpoint{0.000000in}{0.000000in}}%
\pgfpathlineto{\pgfqpoint{0.000000in}{-0.027778in}}%
\pgfusepath{stroke,fill}%
}%
\begin{pgfscope}%
\pgfsys@transformshift{1.541008in}{0.417642in}%
\pgfsys@useobject{currentmarker}{}%
\end{pgfscope}%
\end{pgfscope}%
\begin{pgfscope}%
\pgfpathrectangle{\pgfqpoint{0.589510in}{0.417642in}}{\pgfqpoint{1.809765in}{1.371397in}}%
\pgfusepath{clip}%
\pgfsetrectcap%
\pgfsetroundjoin%
\pgfsetlinewidth{0.803000pt}%
\definecolor{currentstroke}{rgb}{0.850000,0.850000,0.850000}%
\pgfsetstrokecolor{currentstroke}%
\pgfsetdash{}{0pt}%
\pgfpathmoveto{\pgfqpoint{1.564372in}{0.417642in}}%
\pgfpathlineto{\pgfqpoint{1.564372in}{1.789039in}}%
\pgfusepath{stroke}%
\end{pgfscope}%
\begin{pgfscope}%
\pgfsetbuttcap%
\pgfsetroundjoin%
\definecolor{currentfill}{rgb}{0.000000,0.000000,0.000000}%
\pgfsetfillcolor{currentfill}%
\pgfsetlinewidth{0.602250pt}%
\definecolor{currentstroke}{rgb}{0.000000,0.000000,0.000000}%
\pgfsetstrokecolor{currentstroke}%
\pgfsetdash{}{0pt}%
\pgfsys@defobject{currentmarker}{\pgfqpoint{0.000000in}{-0.027778in}}{\pgfqpoint{0.000000in}{0.000000in}}{%
\pgfpathmoveto{\pgfqpoint{0.000000in}{0.000000in}}%
\pgfpathlineto{\pgfqpoint{0.000000in}{-0.027778in}}%
\pgfusepath{stroke,fill}%
}%
\begin{pgfscope}%
\pgfsys@transformshift{1.564372in}{0.417642in}%
\pgfsys@useobject{currentmarker}{}%
\end{pgfscope}%
\end{pgfscope}%
\begin{pgfscope}%
\pgfpathrectangle{\pgfqpoint{0.589510in}{0.417642in}}{\pgfqpoint{1.809765in}{1.371397in}}%
\pgfusepath{clip}%
\pgfsetrectcap%
\pgfsetroundjoin%
\pgfsetlinewidth{0.803000pt}%
\definecolor{currentstroke}{rgb}{0.850000,0.850000,0.850000}%
\pgfsetstrokecolor{currentstroke}%
\pgfsetdash{}{0pt}%
\pgfpathmoveto{\pgfqpoint{1.722767in}{0.417642in}}%
\pgfpathlineto{\pgfqpoint{1.722767in}{1.789039in}}%
\pgfusepath{stroke}%
\end{pgfscope}%
\begin{pgfscope}%
\pgfsetbuttcap%
\pgfsetroundjoin%
\definecolor{currentfill}{rgb}{0.000000,0.000000,0.000000}%
\pgfsetfillcolor{currentfill}%
\pgfsetlinewidth{0.602250pt}%
\definecolor{currentstroke}{rgb}{0.000000,0.000000,0.000000}%
\pgfsetstrokecolor{currentstroke}%
\pgfsetdash{}{0pt}%
\pgfsys@defobject{currentmarker}{\pgfqpoint{0.000000in}{-0.027778in}}{\pgfqpoint{0.000000in}{0.000000in}}{%
\pgfpathmoveto{\pgfqpoint{0.000000in}{0.000000in}}%
\pgfpathlineto{\pgfqpoint{0.000000in}{-0.027778in}}%
\pgfusepath{stroke,fill}%
}%
\begin{pgfscope}%
\pgfsys@transformshift{1.722767in}{0.417642in}%
\pgfsys@useobject{currentmarker}{}%
\end{pgfscope}%
\end{pgfscope}%
\begin{pgfscope}%
\pgfpathrectangle{\pgfqpoint{0.589510in}{0.417642in}}{\pgfqpoint{1.809765in}{1.371397in}}%
\pgfusepath{clip}%
\pgfsetrectcap%
\pgfsetroundjoin%
\pgfsetlinewidth{0.803000pt}%
\definecolor{currentstroke}{rgb}{0.850000,0.850000,0.850000}%
\pgfsetstrokecolor{currentstroke}%
\pgfsetdash{}{0pt}%
\pgfpathmoveto{\pgfqpoint{1.803197in}{0.417642in}}%
\pgfpathlineto{\pgfqpoint{1.803197in}{1.789039in}}%
\pgfusepath{stroke}%
\end{pgfscope}%
\begin{pgfscope}%
\pgfsetbuttcap%
\pgfsetroundjoin%
\definecolor{currentfill}{rgb}{0.000000,0.000000,0.000000}%
\pgfsetfillcolor{currentfill}%
\pgfsetlinewidth{0.602250pt}%
\definecolor{currentstroke}{rgb}{0.000000,0.000000,0.000000}%
\pgfsetstrokecolor{currentstroke}%
\pgfsetdash{}{0pt}%
\pgfsys@defobject{currentmarker}{\pgfqpoint{0.000000in}{-0.027778in}}{\pgfqpoint{0.000000in}{0.000000in}}{%
\pgfpathmoveto{\pgfqpoint{0.000000in}{0.000000in}}%
\pgfpathlineto{\pgfqpoint{0.000000in}{-0.027778in}}%
\pgfusepath{stroke,fill}%
}%
\begin{pgfscope}%
\pgfsys@transformshift{1.803197in}{0.417642in}%
\pgfsys@useobject{currentmarker}{}%
\end{pgfscope}%
\end{pgfscope}%
\begin{pgfscope}%
\pgfpathrectangle{\pgfqpoint{0.589510in}{0.417642in}}{\pgfqpoint{1.809765in}{1.371397in}}%
\pgfusepath{clip}%
\pgfsetrectcap%
\pgfsetroundjoin%
\pgfsetlinewidth{0.803000pt}%
\definecolor{currentstroke}{rgb}{0.850000,0.850000,0.850000}%
\pgfsetstrokecolor{currentstroke}%
\pgfsetdash{}{0pt}%
\pgfpathmoveto{\pgfqpoint{1.860263in}{0.417642in}}%
\pgfpathlineto{\pgfqpoint{1.860263in}{1.789039in}}%
\pgfusepath{stroke}%
\end{pgfscope}%
\begin{pgfscope}%
\pgfsetbuttcap%
\pgfsetroundjoin%
\definecolor{currentfill}{rgb}{0.000000,0.000000,0.000000}%
\pgfsetfillcolor{currentfill}%
\pgfsetlinewidth{0.602250pt}%
\definecolor{currentstroke}{rgb}{0.000000,0.000000,0.000000}%
\pgfsetstrokecolor{currentstroke}%
\pgfsetdash{}{0pt}%
\pgfsys@defobject{currentmarker}{\pgfqpoint{0.000000in}{-0.027778in}}{\pgfqpoint{0.000000in}{0.000000in}}{%
\pgfpathmoveto{\pgfqpoint{0.000000in}{0.000000in}}%
\pgfpathlineto{\pgfqpoint{0.000000in}{-0.027778in}}%
\pgfusepath{stroke,fill}%
}%
\begin{pgfscope}%
\pgfsys@transformshift{1.860263in}{0.417642in}%
\pgfsys@useobject{currentmarker}{}%
\end{pgfscope}%
\end{pgfscope}%
\begin{pgfscope}%
\pgfpathrectangle{\pgfqpoint{0.589510in}{0.417642in}}{\pgfqpoint{1.809765in}{1.371397in}}%
\pgfusepath{clip}%
\pgfsetrectcap%
\pgfsetroundjoin%
\pgfsetlinewidth{0.803000pt}%
\definecolor{currentstroke}{rgb}{0.850000,0.850000,0.850000}%
\pgfsetstrokecolor{currentstroke}%
\pgfsetdash{}{0pt}%
\pgfpathmoveto{\pgfqpoint{1.904526in}{0.417642in}}%
\pgfpathlineto{\pgfqpoint{1.904526in}{1.789039in}}%
\pgfusepath{stroke}%
\end{pgfscope}%
\begin{pgfscope}%
\pgfsetbuttcap%
\pgfsetroundjoin%
\definecolor{currentfill}{rgb}{0.000000,0.000000,0.000000}%
\pgfsetfillcolor{currentfill}%
\pgfsetlinewidth{0.602250pt}%
\definecolor{currentstroke}{rgb}{0.000000,0.000000,0.000000}%
\pgfsetstrokecolor{currentstroke}%
\pgfsetdash{}{0pt}%
\pgfsys@defobject{currentmarker}{\pgfqpoint{0.000000in}{-0.027778in}}{\pgfqpoint{0.000000in}{0.000000in}}{%
\pgfpathmoveto{\pgfqpoint{0.000000in}{0.000000in}}%
\pgfpathlineto{\pgfqpoint{0.000000in}{-0.027778in}}%
\pgfusepath{stroke,fill}%
}%
\begin{pgfscope}%
\pgfsys@transformshift{1.904526in}{0.417642in}%
\pgfsys@useobject{currentmarker}{}%
\end{pgfscope}%
\end{pgfscope}%
\begin{pgfscope}%
\pgfpathrectangle{\pgfqpoint{0.589510in}{0.417642in}}{\pgfqpoint{1.809765in}{1.371397in}}%
\pgfusepath{clip}%
\pgfsetrectcap%
\pgfsetroundjoin%
\pgfsetlinewidth{0.803000pt}%
\definecolor{currentstroke}{rgb}{0.850000,0.850000,0.850000}%
\pgfsetstrokecolor{currentstroke}%
\pgfsetdash{}{0pt}%
\pgfpathmoveto{\pgfqpoint{1.940693in}{0.417642in}}%
\pgfpathlineto{\pgfqpoint{1.940693in}{1.789039in}}%
\pgfusepath{stroke}%
\end{pgfscope}%
\begin{pgfscope}%
\pgfsetbuttcap%
\pgfsetroundjoin%
\definecolor{currentfill}{rgb}{0.000000,0.000000,0.000000}%
\pgfsetfillcolor{currentfill}%
\pgfsetlinewidth{0.602250pt}%
\definecolor{currentstroke}{rgb}{0.000000,0.000000,0.000000}%
\pgfsetstrokecolor{currentstroke}%
\pgfsetdash{}{0pt}%
\pgfsys@defobject{currentmarker}{\pgfqpoint{0.000000in}{-0.027778in}}{\pgfqpoint{0.000000in}{0.000000in}}{%
\pgfpathmoveto{\pgfqpoint{0.000000in}{0.000000in}}%
\pgfpathlineto{\pgfqpoint{0.000000in}{-0.027778in}}%
\pgfusepath{stroke,fill}%
}%
\begin{pgfscope}%
\pgfsys@transformshift{1.940693in}{0.417642in}%
\pgfsys@useobject{currentmarker}{}%
\end{pgfscope}%
\end{pgfscope}%
\begin{pgfscope}%
\pgfpathrectangle{\pgfqpoint{0.589510in}{0.417642in}}{\pgfqpoint{1.809765in}{1.371397in}}%
\pgfusepath{clip}%
\pgfsetrectcap%
\pgfsetroundjoin%
\pgfsetlinewidth{0.803000pt}%
\definecolor{currentstroke}{rgb}{0.850000,0.850000,0.850000}%
\pgfsetstrokecolor{currentstroke}%
\pgfsetdash{}{0pt}%
\pgfpathmoveto{\pgfqpoint{1.971270in}{0.417642in}}%
\pgfpathlineto{\pgfqpoint{1.971270in}{1.789039in}}%
\pgfusepath{stroke}%
\end{pgfscope}%
\begin{pgfscope}%
\pgfsetbuttcap%
\pgfsetroundjoin%
\definecolor{currentfill}{rgb}{0.000000,0.000000,0.000000}%
\pgfsetfillcolor{currentfill}%
\pgfsetlinewidth{0.602250pt}%
\definecolor{currentstroke}{rgb}{0.000000,0.000000,0.000000}%
\pgfsetstrokecolor{currentstroke}%
\pgfsetdash{}{0pt}%
\pgfsys@defobject{currentmarker}{\pgfqpoint{0.000000in}{-0.027778in}}{\pgfqpoint{0.000000in}{0.000000in}}{%
\pgfpathmoveto{\pgfqpoint{0.000000in}{0.000000in}}%
\pgfpathlineto{\pgfqpoint{0.000000in}{-0.027778in}}%
\pgfusepath{stroke,fill}%
}%
\begin{pgfscope}%
\pgfsys@transformshift{1.971270in}{0.417642in}%
\pgfsys@useobject{currentmarker}{}%
\end{pgfscope}%
\end{pgfscope}%
\begin{pgfscope}%
\pgfpathrectangle{\pgfqpoint{0.589510in}{0.417642in}}{\pgfqpoint{1.809765in}{1.371397in}}%
\pgfusepath{clip}%
\pgfsetrectcap%
\pgfsetroundjoin%
\pgfsetlinewidth{0.803000pt}%
\definecolor{currentstroke}{rgb}{0.850000,0.850000,0.850000}%
\pgfsetstrokecolor{currentstroke}%
\pgfsetdash{}{0pt}%
\pgfpathmoveto{\pgfqpoint{1.997758in}{0.417642in}}%
\pgfpathlineto{\pgfqpoint{1.997758in}{1.789039in}}%
\pgfusepath{stroke}%
\end{pgfscope}%
\begin{pgfscope}%
\pgfsetbuttcap%
\pgfsetroundjoin%
\definecolor{currentfill}{rgb}{0.000000,0.000000,0.000000}%
\pgfsetfillcolor{currentfill}%
\pgfsetlinewidth{0.602250pt}%
\definecolor{currentstroke}{rgb}{0.000000,0.000000,0.000000}%
\pgfsetstrokecolor{currentstroke}%
\pgfsetdash{}{0pt}%
\pgfsys@defobject{currentmarker}{\pgfqpoint{0.000000in}{-0.027778in}}{\pgfqpoint{0.000000in}{0.000000in}}{%
\pgfpathmoveto{\pgfqpoint{0.000000in}{0.000000in}}%
\pgfpathlineto{\pgfqpoint{0.000000in}{-0.027778in}}%
\pgfusepath{stroke,fill}%
}%
\begin{pgfscope}%
\pgfsys@transformshift{1.997758in}{0.417642in}%
\pgfsys@useobject{currentmarker}{}%
\end{pgfscope}%
\end{pgfscope}%
\begin{pgfscope}%
\pgfpathrectangle{\pgfqpoint{0.589510in}{0.417642in}}{\pgfqpoint{1.809765in}{1.371397in}}%
\pgfusepath{clip}%
\pgfsetrectcap%
\pgfsetroundjoin%
\pgfsetlinewidth{0.803000pt}%
\definecolor{currentstroke}{rgb}{0.850000,0.850000,0.850000}%
\pgfsetstrokecolor{currentstroke}%
\pgfsetdash{}{0pt}%
\pgfpathmoveto{\pgfqpoint{2.021122in}{0.417642in}}%
\pgfpathlineto{\pgfqpoint{2.021122in}{1.789039in}}%
\pgfusepath{stroke}%
\end{pgfscope}%
\begin{pgfscope}%
\pgfsetbuttcap%
\pgfsetroundjoin%
\definecolor{currentfill}{rgb}{0.000000,0.000000,0.000000}%
\pgfsetfillcolor{currentfill}%
\pgfsetlinewidth{0.602250pt}%
\definecolor{currentstroke}{rgb}{0.000000,0.000000,0.000000}%
\pgfsetstrokecolor{currentstroke}%
\pgfsetdash{}{0pt}%
\pgfsys@defobject{currentmarker}{\pgfqpoint{0.000000in}{-0.027778in}}{\pgfqpoint{0.000000in}{0.000000in}}{%
\pgfpathmoveto{\pgfqpoint{0.000000in}{0.000000in}}%
\pgfpathlineto{\pgfqpoint{0.000000in}{-0.027778in}}%
\pgfusepath{stroke,fill}%
}%
\begin{pgfscope}%
\pgfsys@transformshift{2.021122in}{0.417642in}%
\pgfsys@useobject{currentmarker}{}%
\end{pgfscope}%
\end{pgfscope}%
\begin{pgfscope}%
\pgfpathrectangle{\pgfqpoint{0.589510in}{0.417642in}}{\pgfqpoint{1.809765in}{1.371397in}}%
\pgfusepath{clip}%
\pgfsetrectcap%
\pgfsetroundjoin%
\pgfsetlinewidth{0.803000pt}%
\definecolor{currentstroke}{rgb}{0.850000,0.850000,0.850000}%
\pgfsetstrokecolor{currentstroke}%
\pgfsetdash{}{0pt}%
\pgfpathmoveto{\pgfqpoint{2.179517in}{0.417642in}}%
\pgfpathlineto{\pgfqpoint{2.179517in}{1.789039in}}%
\pgfusepath{stroke}%
\end{pgfscope}%
\begin{pgfscope}%
\pgfsetbuttcap%
\pgfsetroundjoin%
\definecolor{currentfill}{rgb}{0.000000,0.000000,0.000000}%
\pgfsetfillcolor{currentfill}%
\pgfsetlinewidth{0.602250pt}%
\definecolor{currentstroke}{rgb}{0.000000,0.000000,0.000000}%
\pgfsetstrokecolor{currentstroke}%
\pgfsetdash{}{0pt}%
\pgfsys@defobject{currentmarker}{\pgfqpoint{0.000000in}{-0.027778in}}{\pgfqpoint{0.000000in}{0.000000in}}{%
\pgfpathmoveto{\pgfqpoint{0.000000in}{0.000000in}}%
\pgfpathlineto{\pgfqpoint{0.000000in}{-0.027778in}}%
\pgfusepath{stroke,fill}%
}%
\begin{pgfscope}%
\pgfsys@transformshift{2.179517in}{0.417642in}%
\pgfsys@useobject{currentmarker}{}%
\end{pgfscope}%
\end{pgfscope}%
\begin{pgfscope}%
\pgfpathrectangle{\pgfqpoint{0.589510in}{0.417642in}}{\pgfqpoint{1.809765in}{1.371397in}}%
\pgfusepath{clip}%
\pgfsetrectcap%
\pgfsetroundjoin%
\pgfsetlinewidth{0.803000pt}%
\definecolor{currentstroke}{rgb}{0.850000,0.850000,0.850000}%
\pgfsetstrokecolor{currentstroke}%
\pgfsetdash{}{0pt}%
\pgfpathmoveto{\pgfqpoint{2.259947in}{0.417642in}}%
\pgfpathlineto{\pgfqpoint{2.259947in}{1.789039in}}%
\pgfusepath{stroke}%
\end{pgfscope}%
\begin{pgfscope}%
\pgfsetbuttcap%
\pgfsetroundjoin%
\definecolor{currentfill}{rgb}{0.000000,0.000000,0.000000}%
\pgfsetfillcolor{currentfill}%
\pgfsetlinewidth{0.602250pt}%
\definecolor{currentstroke}{rgb}{0.000000,0.000000,0.000000}%
\pgfsetstrokecolor{currentstroke}%
\pgfsetdash{}{0pt}%
\pgfsys@defobject{currentmarker}{\pgfqpoint{0.000000in}{-0.027778in}}{\pgfqpoint{0.000000in}{0.000000in}}{%
\pgfpathmoveto{\pgfqpoint{0.000000in}{0.000000in}}%
\pgfpathlineto{\pgfqpoint{0.000000in}{-0.027778in}}%
\pgfusepath{stroke,fill}%
}%
\begin{pgfscope}%
\pgfsys@transformshift{2.259947in}{0.417642in}%
\pgfsys@useobject{currentmarker}{}%
\end{pgfscope}%
\end{pgfscope}%
\begin{pgfscope}%
\pgfpathrectangle{\pgfqpoint{0.589510in}{0.417642in}}{\pgfqpoint{1.809765in}{1.371397in}}%
\pgfusepath{clip}%
\pgfsetrectcap%
\pgfsetroundjoin%
\pgfsetlinewidth{0.803000pt}%
\definecolor{currentstroke}{rgb}{0.850000,0.850000,0.850000}%
\pgfsetstrokecolor{currentstroke}%
\pgfsetdash{}{0pt}%
\pgfpathmoveto{\pgfqpoint{2.317013in}{0.417642in}}%
\pgfpathlineto{\pgfqpoint{2.317013in}{1.789039in}}%
\pgfusepath{stroke}%
\end{pgfscope}%
\begin{pgfscope}%
\pgfsetbuttcap%
\pgfsetroundjoin%
\definecolor{currentfill}{rgb}{0.000000,0.000000,0.000000}%
\pgfsetfillcolor{currentfill}%
\pgfsetlinewidth{0.602250pt}%
\definecolor{currentstroke}{rgb}{0.000000,0.000000,0.000000}%
\pgfsetstrokecolor{currentstroke}%
\pgfsetdash{}{0pt}%
\pgfsys@defobject{currentmarker}{\pgfqpoint{0.000000in}{-0.027778in}}{\pgfqpoint{0.000000in}{0.000000in}}{%
\pgfpathmoveto{\pgfqpoint{0.000000in}{0.000000in}}%
\pgfpathlineto{\pgfqpoint{0.000000in}{-0.027778in}}%
\pgfusepath{stroke,fill}%
}%
\begin{pgfscope}%
\pgfsys@transformshift{2.317013in}{0.417642in}%
\pgfsys@useobject{currentmarker}{}%
\end{pgfscope}%
\end{pgfscope}%
\begin{pgfscope}%
\pgfpathrectangle{\pgfqpoint{0.589510in}{0.417642in}}{\pgfqpoint{1.809765in}{1.371397in}}%
\pgfusepath{clip}%
\pgfsetrectcap%
\pgfsetroundjoin%
\pgfsetlinewidth{0.803000pt}%
\definecolor{currentstroke}{rgb}{0.850000,0.850000,0.850000}%
\pgfsetstrokecolor{currentstroke}%
\pgfsetdash{}{0pt}%
\pgfpathmoveto{\pgfqpoint{2.361277in}{0.417642in}}%
\pgfpathlineto{\pgfqpoint{2.361277in}{1.789039in}}%
\pgfusepath{stroke}%
\end{pgfscope}%
\begin{pgfscope}%
\pgfsetbuttcap%
\pgfsetroundjoin%
\definecolor{currentfill}{rgb}{0.000000,0.000000,0.000000}%
\pgfsetfillcolor{currentfill}%
\pgfsetlinewidth{0.602250pt}%
\definecolor{currentstroke}{rgb}{0.000000,0.000000,0.000000}%
\pgfsetstrokecolor{currentstroke}%
\pgfsetdash{}{0pt}%
\pgfsys@defobject{currentmarker}{\pgfqpoint{0.000000in}{-0.027778in}}{\pgfqpoint{0.000000in}{0.000000in}}{%
\pgfpathmoveto{\pgfqpoint{0.000000in}{0.000000in}}%
\pgfpathlineto{\pgfqpoint{0.000000in}{-0.027778in}}%
\pgfusepath{stroke,fill}%
}%
\begin{pgfscope}%
\pgfsys@transformshift{2.361277in}{0.417642in}%
\pgfsys@useobject{currentmarker}{}%
\end{pgfscope}%
\end{pgfscope}%
\begin{pgfscope}%
\pgfpathrectangle{\pgfqpoint{0.589510in}{0.417642in}}{\pgfqpoint{1.809765in}{1.371397in}}%
\pgfusepath{clip}%
\pgfsetrectcap%
\pgfsetroundjoin%
\pgfsetlinewidth{0.803000pt}%
\definecolor{currentstroke}{rgb}{0.850000,0.850000,0.850000}%
\pgfsetstrokecolor{currentstroke}%
\pgfsetdash{}{0pt}%
\pgfpathmoveto{\pgfqpoint{2.397443in}{0.417642in}}%
\pgfpathlineto{\pgfqpoint{2.397443in}{1.789039in}}%
\pgfusepath{stroke}%
\end{pgfscope}%
\begin{pgfscope}%
\pgfsetbuttcap%
\pgfsetroundjoin%
\definecolor{currentfill}{rgb}{0.000000,0.000000,0.000000}%
\pgfsetfillcolor{currentfill}%
\pgfsetlinewidth{0.602250pt}%
\definecolor{currentstroke}{rgb}{0.000000,0.000000,0.000000}%
\pgfsetstrokecolor{currentstroke}%
\pgfsetdash{}{0pt}%
\pgfsys@defobject{currentmarker}{\pgfqpoint{0.000000in}{-0.027778in}}{\pgfqpoint{0.000000in}{0.000000in}}{%
\pgfpathmoveto{\pgfqpoint{0.000000in}{0.000000in}}%
\pgfpathlineto{\pgfqpoint{0.000000in}{-0.027778in}}%
\pgfusepath{stroke,fill}%
}%
\begin{pgfscope}%
\pgfsys@transformshift{2.397443in}{0.417642in}%
\pgfsys@useobject{currentmarker}{}%
\end{pgfscope}%
\end{pgfscope}%
\begin{pgfscope}%
\definecolor{textcolor}{rgb}{0.000000,0.000000,0.000000}%
\pgfsetstrokecolor{textcolor}%
\pgfsetfillcolor{textcolor}%
\pgftext[x=1.494392in,y=0.165003in,,top]{\color{textcolor}\rmfamily\fontsize{10.000000}{12.000000}\selectfont \(\displaystyle \tau\) in \unit{\second}}%
\end{pgfscope}%
\begin{pgfscope}%
\pgfpathrectangle{\pgfqpoint{0.589510in}{0.417642in}}{\pgfqpoint{1.809765in}{1.371397in}}%
\pgfusepath{clip}%
\pgfsetrectcap%
\pgfsetroundjoin%
\pgfsetlinewidth{0.803000pt}%
\definecolor{currentstroke}{rgb}{0.450000,0.450000,0.450000}%
\pgfsetstrokecolor{currentstroke}%
\pgfsetdash{}{0pt}%
\pgfpathmoveto{\pgfqpoint{0.589510in}{0.417642in}}%
\pgfpathlineto{\pgfqpoint{2.399275in}{0.417642in}}%
\pgfusepath{stroke}%
\end{pgfscope}%
\begin{pgfscope}%
\pgfsetbuttcap%
\pgfsetroundjoin%
\definecolor{currentfill}{rgb}{0.000000,0.000000,0.000000}%
\pgfsetfillcolor{currentfill}%
\pgfsetlinewidth{0.803000pt}%
\definecolor{currentstroke}{rgb}{0.000000,0.000000,0.000000}%
\pgfsetstrokecolor{currentstroke}%
\pgfsetdash{}{0pt}%
\pgfsys@defobject{currentmarker}{\pgfqpoint{-0.048611in}{0.000000in}}{\pgfqpoint{-0.000000in}{0.000000in}}{%
\pgfpathmoveto{\pgfqpoint{-0.000000in}{0.000000in}}%
\pgfpathlineto{\pgfqpoint{-0.048611in}{0.000000in}}%
\pgfusepath{stroke,fill}%
}%
\begin{pgfscope}%
\pgfsys@transformshift{0.589510in}{0.417642in}%
\pgfsys@useobject{currentmarker}{}%
\end{pgfscope}%
\end{pgfscope}%
\begin{pgfscope}%
\definecolor{textcolor}{rgb}{0.000000,0.000000,0.000000}%
\pgfsetstrokecolor{textcolor}%
\pgfsetfillcolor{textcolor}%
\pgftext[x=0.236114in, y=0.378489in, left, base]{\color{textcolor}\rmfamily\fontsize{8.000000}{9.600000}\selectfont \(\displaystyle {10^{-2}}\)}%
\end{pgfscope}%
\begin{pgfscope}%
\pgfpathrectangle{\pgfqpoint{0.589510in}{0.417642in}}{\pgfqpoint{1.809765in}{1.371397in}}%
\pgfusepath{clip}%
\pgfsetrectcap%
\pgfsetroundjoin%
\pgfsetlinewidth{0.803000pt}%
\definecolor{currentstroke}{rgb}{0.450000,0.450000,0.450000}%
\pgfsetstrokecolor{currentstroke}%
\pgfsetdash{}{0pt}%
\pgfpathmoveto{\pgfqpoint{0.589510in}{0.827077in}}%
\pgfpathlineto{\pgfqpoint{2.399275in}{0.827077in}}%
\pgfusepath{stroke}%
\end{pgfscope}%
\begin{pgfscope}%
\pgfsetbuttcap%
\pgfsetroundjoin%
\definecolor{currentfill}{rgb}{0.000000,0.000000,0.000000}%
\pgfsetfillcolor{currentfill}%
\pgfsetlinewidth{0.803000pt}%
\definecolor{currentstroke}{rgb}{0.000000,0.000000,0.000000}%
\pgfsetstrokecolor{currentstroke}%
\pgfsetdash{}{0pt}%
\pgfsys@defobject{currentmarker}{\pgfqpoint{-0.048611in}{0.000000in}}{\pgfqpoint{-0.000000in}{0.000000in}}{%
\pgfpathmoveto{\pgfqpoint{-0.000000in}{0.000000in}}%
\pgfpathlineto{\pgfqpoint{-0.048611in}{0.000000in}}%
\pgfusepath{stroke,fill}%
}%
\begin{pgfscope}%
\pgfsys@transformshift{0.589510in}{0.827077in}%
\pgfsys@useobject{currentmarker}{}%
\end{pgfscope}%
\end{pgfscope}%
\begin{pgfscope}%
\definecolor{textcolor}{rgb}{0.000000,0.000000,0.000000}%
\pgfsetstrokecolor{textcolor}%
\pgfsetfillcolor{textcolor}%
\pgftext[x=0.316361in, y=0.787924in, left, base]{\color{textcolor}\rmfamily\fontsize{8.000000}{9.600000}\selectfont \(\displaystyle {10^{0}}\)}%
\end{pgfscope}%
\begin{pgfscope}%
\pgfpathrectangle{\pgfqpoint{0.589510in}{0.417642in}}{\pgfqpoint{1.809765in}{1.371397in}}%
\pgfusepath{clip}%
\pgfsetrectcap%
\pgfsetroundjoin%
\pgfsetlinewidth{0.803000pt}%
\definecolor{currentstroke}{rgb}{0.450000,0.450000,0.450000}%
\pgfsetstrokecolor{currentstroke}%
\pgfsetdash{}{0pt}%
\pgfpathmoveto{\pgfqpoint{0.589510in}{1.236512in}}%
\pgfpathlineto{\pgfqpoint{2.399275in}{1.236512in}}%
\pgfusepath{stroke}%
\end{pgfscope}%
\begin{pgfscope}%
\pgfsetbuttcap%
\pgfsetroundjoin%
\definecolor{currentfill}{rgb}{0.000000,0.000000,0.000000}%
\pgfsetfillcolor{currentfill}%
\pgfsetlinewidth{0.803000pt}%
\definecolor{currentstroke}{rgb}{0.000000,0.000000,0.000000}%
\pgfsetstrokecolor{currentstroke}%
\pgfsetdash{}{0pt}%
\pgfsys@defobject{currentmarker}{\pgfqpoint{-0.048611in}{0.000000in}}{\pgfqpoint{-0.000000in}{0.000000in}}{%
\pgfpathmoveto{\pgfqpoint{-0.000000in}{0.000000in}}%
\pgfpathlineto{\pgfqpoint{-0.048611in}{0.000000in}}%
\pgfusepath{stroke,fill}%
}%
\begin{pgfscope}%
\pgfsys@transformshift{0.589510in}{1.236512in}%
\pgfsys@useobject{currentmarker}{}%
\end{pgfscope}%
\end{pgfscope}%
\begin{pgfscope}%
\definecolor{textcolor}{rgb}{0.000000,0.000000,0.000000}%
\pgfsetstrokecolor{textcolor}%
\pgfsetfillcolor{textcolor}%
\pgftext[x=0.316361in, y=1.197359in, left, base]{\color{textcolor}\rmfamily\fontsize{8.000000}{9.600000}\selectfont \(\displaystyle {10^{2}}\)}%
\end{pgfscope}%
\begin{pgfscope}%
\pgfpathrectangle{\pgfqpoint{0.589510in}{0.417642in}}{\pgfqpoint{1.809765in}{1.371397in}}%
\pgfusepath{clip}%
\pgfsetrectcap%
\pgfsetroundjoin%
\pgfsetlinewidth{0.803000pt}%
\definecolor{currentstroke}{rgb}{0.450000,0.450000,0.450000}%
\pgfsetstrokecolor{currentstroke}%
\pgfsetdash{}{0pt}%
\pgfpathmoveto{\pgfqpoint{0.589510in}{1.645947in}}%
\pgfpathlineto{\pgfqpoint{2.399275in}{1.645947in}}%
\pgfusepath{stroke}%
\end{pgfscope}%
\begin{pgfscope}%
\pgfsetbuttcap%
\pgfsetroundjoin%
\definecolor{currentfill}{rgb}{0.000000,0.000000,0.000000}%
\pgfsetfillcolor{currentfill}%
\pgfsetlinewidth{0.803000pt}%
\definecolor{currentstroke}{rgb}{0.000000,0.000000,0.000000}%
\pgfsetstrokecolor{currentstroke}%
\pgfsetdash{}{0pt}%
\pgfsys@defobject{currentmarker}{\pgfqpoint{-0.048611in}{0.000000in}}{\pgfqpoint{-0.000000in}{0.000000in}}{%
\pgfpathmoveto{\pgfqpoint{-0.000000in}{0.000000in}}%
\pgfpathlineto{\pgfqpoint{-0.048611in}{0.000000in}}%
\pgfusepath{stroke,fill}%
}%
\begin{pgfscope}%
\pgfsys@transformshift{0.589510in}{1.645947in}%
\pgfsys@useobject{currentmarker}{}%
\end{pgfscope}%
\end{pgfscope}%
\begin{pgfscope}%
\definecolor{textcolor}{rgb}{0.000000,0.000000,0.000000}%
\pgfsetstrokecolor{textcolor}%
\pgfsetfillcolor{textcolor}%
\pgftext[x=0.316361in, y=1.606795in, left, base]{\color{textcolor}\rmfamily\fontsize{8.000000}{9.600000}\selectfont \(\displaystyle {10^{4}}\)}%
\end{pgfscope}%
\begin{pgfscope}%
\definecolor{textcolor}{rgb}{0.000000,0.000000,0.000000}%
\pgfsetstrokecolor{textcolor}%
\pgfsetfillcolor{textcolor}%
\pgftext[x=0.180559in,y=1.103340in,,bottom,rotate=90.000000]{\color{textcolor}\rmfamily\fontsize{10.000000}{12.000000}\selectfont ADEV \(\displaystyle \sigma_A(\tau)\)}%
\end{pgfscope}%
\begin{pgfscope}%
\pgfpathrectangle{\pgfqpoint{0.589510in}{0.417642in}}{\pgfqpoint{1.809765in}{1.371397in}}%
\pgfusepath{clip}%
\pgfsetbuttcap%
\pgfsetroundjoin%
\pgfsetlinewidth{1.505625pt}%
\definecolor{currentstroke}{rgb}{0.007843,0.619608,0.450980}%
\pgfsetstrokecolor{currentstroke}%
\pgfsetdash{{5.550000pt}{2.400000pt}}{0.000000pt}%
\pgfpathmoveto{\pgfqpoint{0.671772in}{0.827077in}}%
\pgfpathlineto{\pgfqpoint{0.809267in}{0.827077in}}%
\pgfpathlineto{\pgfqpoint{0.946763in}{0.827077in}}%
\pgfpathlineto{\pgfqpoint{1.128522in}{0.827077in}}%
\pgfpathlineto{\pgfqpoint{1.266017in}{0.827077in}}%
\pgfpathlineto{\pgfqpoint{1.403513in}{0.827077in}}%
\pgfpathlineto{\pgfqpoint{1.585272in}{0.827077in}}%
\pgfpathlineto{\pgfqpoint{1.722767in}{0.827077in}}%
\pgfpathlineto{\pgfqpoint{1.860263in}{0.827077in}}%
\pgfpathlineto{\pgfqpoint{2.042022in}{0.827077in}}%
\pgfpathlineto{\pgfqpoint{2.179517in}{0.827077in}}%
\pgfpathlineto{\pgfqpoint{2.317013in}{0.827077in}}%
\pgfusepath{stroke}%
\end{pgfscope}%
\begin{pgfscope}%
\pgfpathrectangle{\pgfqpoint{0.589510in}{0.417642in}}{\pgfqpoint{1.809765in}{1.371397in}}%
\pgfusepath{clip}%
\pgfsetbuttcap%
\pgfsetroundjoin%
\definecolor{currentfill}{rgb}{0.007843,0.619608,0.450980}%
\pgfsetfillcolor{currentfill}%
\pgfsetlinewidth{1.003750pt}%
\definecolor{currentstroke}{rgb}{0.007843,0.619608,0.450980}%
\pgfsetstrokecolor{currentstroke}%
\pgfsetdash{}{0pt}%
\pgfsys@defobject{currentmarker}{\pgfqpoint{-0.020833in}{-0.020833in}}{\pgfqpoint{0.020833in}{0.020833in}}{%
\pgfpathmoveto{\pgfqpoint{0.000000in}{-0.020833in}}%
\pgfpathcurveto{\pgfqpoint{0.005525in}{-0.020833in}}{\pgfqpoint{0.010825in}{-0.018638in}}{\pgfqpoint{0.014731in}{-0.014731in}}%
\pgfpathcurveto{\pgfqpoint{0.018638in}{-0.010825in}}{\pgfqpoint{0.020833in}{-0.005525in}}{\pgfqpoint{0.020833in}{0.000000in}}%
\pgfpathcurveto{\pgfqpoint{0.020833in}{0.005525in}}{\pgfqpoint{0.018638in}{0.010825in}}{\pgfqpoint{0.014731in}{0.014731in}}%
\pgfpathcurveto{\pgfqpoint{0.010825in}{0.018638in}}{\pgfqpoint{0.005525in}{0.020833in}}{\pgfqpoint{0.000000in}{0.020833in}}%
\pgfpathcurveto{\pgfqpoint{-0.005525in}{0.020833in}}{\pgfqpoint{-0.010825in}{0.018638in}}{\pgfqpoint{-0.014731in}{0.014731in}}%
\pgfpathcurveto{\pgfqpoint{-0.018638in}{0.010825in}}{\pgfqpoint{-0.020833in}{0.005525in}}{\pgfqpoint{-0.020833in}{0.000000in}}%
\pgfpathcurveto{\pgfqpoint{-0.020833in}{-0.005525in}}{\pgfqpoint{-0.018638in}{-0.010825in}}{\pgfqpoint{-0.014731in}{-0.014731in}}%
\pgfpathcurveto{\pgfqpoint{-0.010825in}{-0.018638in}}{\pgfqpoint{-0.005525in}{-0.020833in}}{\pgfqpoint{0.000000in}{-0.020833in}}%
\pgfpathlineto{\pgfqpoint{0.000000in}{-0.020833in}}%
\pgfpathclose%
\pgfusepath{stroke,fill}%
}%
\begin{pgfscope}%
\pgfsys@transformshift{0.671772in}{0.843965in}%
\pgfsys@useobject{currentmarker}{}%
\end{pgfscope}%
\begin{pgfscope}%
\pgfsys@transformshift{0.809267in}{0.833886in}%
\pgfsys@useobject{currentmarker}{}%
\end{pgfscope}%
\begin{pgfscope}%
\pgfsys@transformshift{0.946763in}{0.828796in}%
\pgfsys@useobject{currentmarker}{}%
\end{pgfscope}%
\begin{pgfscope}%
\pgfsys@transformshift{1.128522in}{0.824754in}%
\pgfsys@useobject{currentmarker}{}%
\end{pgfscope}%
\begin{pgfscope}%
\pgfsys@transformshift{1.266017in}{0.823118in}%
\pgfsys@useobject{currentmarker}{}%
\end{pgfscope}%
\begin{pgfscope}%
\pgfsys@transformshift{1.403513in}{0.827157in}%
\pgfsys@useobject{currentmarker}{}%
\end{pgfscope}%
\begin{pgfscope}%
\pgfsys@transformshift{1.585272in}{0.827683in}%
\pgfsys@useobject{currentmarker}{}%
\end{pgfscope}%
\begin{pgfscope}%
\pgfsys@transformshift{1.722767in}{0.820937in}%
\pgfsys@useobject{currentmarker}{}%
\end{pgfscope}%
\begin{pgfscope}%
\pgfsys@transformshift{1.860263in}{0.810134in}%
\pgfsys@useobject{currentmarker}{}%
\end{pgfscope}%
\begin{pgfscope}%
\pgfsys@transformshift{2.042022in}{0.819270in}%
\pgfsys@useobject{currentmarker}{}%
\end{pgfscope}%
\begin{pgfscope}%
\pgfsys@transformshift{2.179517in}{0.852081in}%
\pgfsys@useobject{currentmarker}{}%
\end{pgfscope}%
\begin{pgfscope}%
\pgfsys@transformshift{2.317013in}{0.835836in}%
\pgfsys@useobject{currentmarker}{}%
\end{pgfscope}%
\end{pgfscope}%
\begin{pgfscope}%
\pgfsetrectcap%
\pgfsetmiterjoin%
\pgfsetlinewidth{0.803000pt}%
\definecolor{currentstroke}{rgb}{0.000000,0.000000,0.000000}%
\pgfsetstrokecolor{currentstroke}%
\pgfsetdash{}{0pt}%
\pgfpathmoveto{\pgfqpoint{0.589510in}{0.417642in}}%
\pgfpathlineto{\pgfqpoint{0.589510in}{1.789039in}}%
\pgfusepath{stroke}%
\end{pgfscope}%
\begin{pgfscope}%
\pgfsetrectcap%
\pgfsetmiterjoin%
\pgfsetlinewidth{0.803000pt}%
\definecolor{currentstroke}{rgb}{0.000000,0.000000,0.000000}%
\pgfsetstrokecolor{currentstroke}%
\pgfsetdash{}{0pt}%
\pgfpathmoveto{\pgfqpoint{2.399275in}{0.417642in}}%
\pgfpathlineto{\pgfqpoint{2.399275in}{1.789039in}}%
\pgfusepath{stroke}%
\end{pgfscope}%
\begin{pgfscope}%
\pgfsetrectcap%
\pgfsetmiterjoin%
\pgfsetlinewidth{0.803000pt}%
\definecolor{currentstroke}{rgb}{0.000000,0.000000,0.000000}%
\pgfsetstrokecolor{currentstroke}%
\pgfsetdash{}{0pt}%
\pgfpathmoveto{\pgfqpoint{0.589510in}{0.417642in}}%
\pgfpathlineto{\pgfqpoint{2.399275in}{0.417642in}}%
\pgfusepath{stroke}%
\end{pgfscope}%
\begin{pgfscope}%
\pgfsetrectcap%
\pgfsetmiterjoin%
\pgfsetlinewidth{0.803000pt}%
\definecolor{currentstroke}{rgb}{0.000000,0.000000,0.000000}%
\pgfsetstrokecolor{currentstroke}%
\pgfsetdash{}{0pt}%
\pgfpathmoveto{\pgfqpoint{0.589510in}{1.789039in}}%
\pgfpathlineto{\pgfqpoint{2.399275in}{1.789039in}}%
\pgfusepath{stroke}%
\end{pgfscope}%
\begin{pgfscope}%
\pgfsetbuttcap%
\pgfsetmiterjoin%
\definecolor{currentfill}{rgb}{1.000000,1.000000,1.000000}%
\pgfsetfillcolor{currentfill}%
\pgfsetfillopacity{0.800000}%
\pgfsetlinewidth{1.003750pt}%
\definecolor{currentstroke}{rgb}{0.800000,0.800000,0.800000}%
\pgfsetstrokecolor{currentstroke}%
\pgfsetstrokeopacity{0.800000}%
\pgfsetdash{}{0pt}%
\pgfpathmoveto{\pgfqpoint{1.212708in}{1.472371in}}%
\pgfpathlineto{\pgfqpoint{2.321497in}{1.472371in}}%
\pgfpathquadraticcurveto{\pgfqpoint{2.343719in}{1.472371in}}{\pgfqpoint{2.343719in}{1.494593in}}%
\pgfpathlineto{\pgfqpoint{2.343719in}{1.711261in}}%
\pgfpathquadraticcurveto{\pgfqpoint{2.343719in}{1.733483in}}{\pgfqpoint{2.321497in}{1.733483in}}%
\pgfpathlineto{\pgfqpoint{1.212708in}{1.733483in}}%
\pgfpathquadraticcurveto{\pgfqpoint{1.190486in}{1.733483in}}{\pgfqpoint{1.190486in}{1.711261in}}%
\pgfpathlineto{\pgfqpoint{1.190486in}{1.494593in}}%
\pgfpathquadraticcurveto{\pgfqpoint{1.190486in}{1.472371in}}{\pgfqpoint{1.212708in}{1.472371in}}%
\pgfpathlineto{\pgfqpoint{1.212708in}{1.472371in}}%
\pgfpathclose%
\pgfusepath{stroke,fill}%
\end{pgfscope}%
\begin{pgfscope}%
\pgfsetbuttcap%
\pgfsetroundjoin%
\pgfsetlinewidth{1.505625pt}%
\definecolor{currentstroke}{rgb}{0.007843,0.619608,0.450980}%
\pgfsetstrokecolor{currentstroke}%
\pgfsetdash{{5.550000pt}{2.400000pt}}{0.000000pt}%
\pgfpathmoveto{\pgfqpoint{1.234930in}{1.602426in}}%
\pgfpathlineto{\pgfqpoint{1.346041in}{1.602426in}}%
\pgfpathlineto{\pgfqpoint{1.457152in}{1.602426in}}%
\pgfusepath{stroke}%
\end{pgfscope}%
\begin{pgfscope}%
\definecolor{textcolor}{rgb}{0.000000,0.000000,0.000000}%
\pgfsetstrokecolor{textcolor}%
\pgfsetfillcolor{textcolor}%
\pgftext[x=1.546041in,y=1.563537in,left,base]{\color{textcolor}\rmfamily\fontsize{8.000000}{9.600000}\selectfont \(\displaystyle \propto\sqrt{h_{-1}}\tau^{+0.0}\)}%
\end{pgfscope}%
\end{pgfpicture}%
\makeatother%
\endgroup%

        } % scalebox
        \caption{Allan deviation}
        \label{fig:flicker_noise_adev}
    \end{subfigure}
    \caption{Different representations of flicker noise.}
    \label{fig:flicker_noise_simulated}
\end{figure}

While it is not immediately evident from the power spectral density, the Allan deviation plot explains very well, why additional filtering does not affect flicker noise. No matter how long the integration time, the variance will the same.

The small wiggles at longer $\tau$ are typical end-of-data errors caused by spectral leakage, because there are insufficient samples to average over \cite{adev_long_tau}. As it was discussed above, the Allan deviation can only be estimated using equation \ref{eqn:adev_estimator} given a limited number of samples. Therefore, at $\frac{\tau}{2}$ there are only $2$ samples left, so there is no averaging possible to improve the estimate of the Allan deviation, which causes the oscillations at low frequencies or large $\tau$.

As a last remark, a commonly used definition in combination with flicker noise is the corner frequency $f_c$. The corner frequency appears in situations, where there is both flicker and white noise present. It is the crossover point in frequency, where the flicker noise is equal compared to the white noise.
\begin{equation}
    f_c = \frac{h_{-1}}{h_0} \label{eqn:corner_frequency}
\end{equation}
It can be graphically extracted from the power spectral density plot by drawing a line trough the flicker noise and the white noise and finding the intersection. This can be seen in figure \ref{fig:adev_example_psd} on page \pageref{fig:adev_example_psd}. The corner frequency can be found where the horizontal dashed blue and green line meet.

\clearpage
\subsubsection{Random Walk}
Random walk noise can be attributed to environmental factors such as temperature \cite{random_walk_fm} and diffusion processes, the latter contributing to the ageing effect seen in semiconductors.
It is a process, where in each time step the change is randomly determined to be either a positve or negative step with equal probability and a fixed step size. Its mean is
\begin{equation}
    \langle y_n \rangle = \langle e_1 + e_2 + \dots e_n \rangle = \underbrace{\langle e_1 \rangle}_{=\,0} + \langle e_2 \rangle + \dots + \langle e_n \rangle = 0 \, ,
\end{equation}
but its variance
\begin{equation}
    \sigma_y^2 = \langle y_n^2 \rangle - \underbrace{\langle y_n \rangle}_{=\,0} = \sigma_{e_1}^2 + \sigma_{e_2}^2 + \dots \sigma_{e_n}^2 = n \sigma_e^2
\end{equation}
goes with $n$ (or $t$). It therefore not a stationary process as can also be seen in figure \ref{fig:random_walk_adev}.

The power spectral density can be calculated \cite{psd_to_adev,noise_generation} to
\begin{equation}
    S(f) = h_{-2} \frac{1}{f^2}
\end{equation}
and the Allan deviation can again be calculated from the spectral density
\begin{align}
    \sigma_A^2(\tau) &= 2 h_{-2} \int_0^\infty \frac{1}{f^2} \frac{\sin^4\left( \pi f \tau \right)}{(\pi f \tau)^2}\,df \nonumber\\
    &=\frac{2}{3} \pi^2 h_{-2}\, \tau
\end{align}

The \textit{AllanTools} library \cite{allantools} can then be used to simulate the random walk.

\begin{figure}[ht]
    \centering
    \begin{subfigure}{0.32\linewidth}
        \centering
        \scalebox{0.75}{%
            %% Creator: Matplotlib, PGF backend
%%
%% To include the figure in your LaTeX document, write
%%   \input{<filename>.pgf}
%%
%% Make sure the required packages are loaded in your preamble
%%   \usepackage{pgf}
%%
%% Also ensure that all the required font packages are loaded; for instance,
%% the lmodern package is sometimes necessary when using math font.
%%   \usepackage{lmodern}
%%
%% Figures using additional raster images can only be included by \input if
%% they are in the same directory as the main LaTeX file. For loading figures
%% from other directories you can use the `import` package
%%   \usepackage{import}
%%
%% and then include the figures with
%%   \import{<path to file>}{<filename>.pgf}
%%
%% Matplotlib used the following preamble
%%   \usepackage{siunitx}
%%   \usepackage{fontspec}
%%   \makeatletter\@ifpackageloaded{underscore}{}{\usepackage[strings]{underscore}}\makeatother
%%
\begingroup%
\makeatletter%
\begin{pgfpicture}%
\pgfpathrectangle{\pgfpointorigin}{\pgfqpoint{2.440000in}{1.830000in}}%
\pgfusepath{use as bounding box, clip}%
\begin{pgfscope}%
\pgfsetbuttcap%
\pgfsetmiterjoin%
\definecolor{currentfill}{rgb}{1.000000,1.000000,1.000000}%
\pgfsetfillcolor{currentfill}%
\pgfsetlinewidth{0.000000pt}%
\definecolor{currentstroke}{rgb}{1.000000,1.000000,1.000000}%
\pgfsetstrokecolor{currentstroke}%
\pgfsetdash{}{0pt}%
\pgfpathmoveto{\pgfqpoint{0.000000in}{0.000000in}}%
\pgfpathlineto{\pgfqpoint{2.440000in}{0.000000in}}%
\pgfpathlineto{\pgfqpoint{2.440000in}{1.830000in}}%
\pgfpathlineto{\pgfqpoint{0.000000in}{1.830000in}}%
\pgfpathlineto{\pgfqpoint{0.000000in}{0.000000in}}%
\pgfpathclose%
\pgfusepath{fill}%
\end{pgfscope}%
\begin{pgfscope}%
\pgfsetbuttcap%
\pgfsetmiterjoin%
\definecolor{currentfill}{rgb}{1.000000,1.000000,1.000000}%
\pgfsetfillcolor{currentfill}%
\pgfsetlinewidth{0.000000pt}%
\definecolor{currentstroke}{rgb}{0.000000,0.000000,0.000000}%
\pgfsetstrokecolor{currentstroke}%
\pgfsetstrokeopacity{0.000000}%
\pgfsetdash{}{0pt}%
\pgfpathmoveto{\pgfqpoint{0.589745in}{0.416447in}}%
\pgfpathlineto{\pgfqpoint{2.398330in}{0.416447in}}%
\pgfpathlineto{\pgfqpoint{2.398330in}{1.788330in}}%
\pgfpathlineto{\pgfqpoint{0.589745in}{1.788330in}}%
\pgfpathlineto{\pgfqpoint{0.589745in}{0.416447in}}%
\pgfpathclose%
\pgfusepath{fill}%
\end{pgfscope}%
\begin{pgfscope}%
\pgfpathrectangle{\pgfqpoint{0.589745in}{0.416447in}}{\pgfqpoint{1.808585in}{1.371883in}}%
\pgfusepath{clip}%
\pgfsetrectcap%
\pgfsetroundjoin%
\pgfsetlinewidth{0.803000pt}%
\definecolor{currentstroke}{rgb}{0.450000,0.450000,0.450000}%
\pgfsetstrokecolor{currentstroke}%
\pgfsetdash{}{0pt}%
\pgfpathmoveto{\pgfqpoint{0.671953in}{0.416447in}}%
\pgfpathlineto{\pgfqpoint{0.671953in}{1.788330in}}%
\pgfusepath{stroke}%
\end{pgfscope}%
\begin{pgfscope}%
\pgfsetbuttcap%
\pgfsetroundjoin%
\definecolor{currentfill}{rgb}{0.000000,0.000000,0.000000}%
\pgfsetfillcolor{currentfill}%
\pgfsetlinewidth{0.803000pt}%
\definecolor{currentstroke}{rgb}{0.000000,0.000000,0.000000}%
\pgfsetstrokecolor{currentstroke}%
\pgfsetdash{}{0pt}%
\pgfsys@defobject{currentmarker}{\pgfqpoint{0.000000in}{-0.048611in}}{\pgfqpoint{0.000000in}{0.000000in}}{%
\pgfpathmoveto{\pgfqpoint{0.000000in}{0.000000in}}%
\pgfpathlineto{\pgfqpoint{0.000000in}{-0.048611in}}%
\pgfusepath{stroke,fill}%
}%
\begin{pgfscope}%
\pgfsys@transformshift{0.671953in}{0.416447in}%
\pgfsys@useobject{currentmarker}{}%
\end{pgfscope}%
\end{pgfscope}%
\begin{pgfscope}%
\definecolor{textcolor}{rgb}{0.000000,0.000000,0.000000}%
\pgfsetstrokecolor{textcolor}%
\pgfsetfillcolor{textcolor}%
\pgftext[x=0.671953in,y=0.319225in,,top]{\color{textcolor}\rmfamily\fontsize{8.000000}{9.600000}\selectfont \(\displaystyle {0}\)}%
\end{pgfscope}%
\begin{pgfscope}%
\pgfpathrectangle{\pgfqpoint{0.589745in}{0.416447in}}{\pgfqpoint{1.808585in}{1.371883in}}%
\pgfusepath{clip}%
\pgfsetrectcap%
\pgfsetroundjoin%
\pgfsetlinewidth{0.803000pt}%
\definecolor{currentstroke}{rgb}{0.450000,0.450000,0.450000}%
\pgfsetstrokecolor{currentstroke}%
\pgfsetdash{}{0pt}%
\pgfpathmoveto{\pgfqpoint{1.173775in}{0.416447in}}%
\pgfpathlineto{\pgfqpoint{1.173775in}{1.788330in}}%
\pgfusepath{stroke}%
\end{pgfscope}%
\begin{pgfscope}%
\pgfsetbuttcap%
\pgfsetroundjoin%
\definecolor{currentfill}{rgb}{0.000000,0.000000,0.000000}%
\pgfsetfillcolor{currentfill}%
\pgfsetlinewidth{0.803000pt}%
\definecolor{currentstroke}{rgb}{0.000000,0.000000,0.000000}%
\pgfsetstrokecolor{currentstroke}%
\pgfsetdash{}{0pt}%
\pgfsys@defobject{currentmarker}{\pgfqpoint{0.000000in}{-0.048611in}}{\pgfqpoint{0.000000in}{0.000000in}}{%
\pgfpathmoveto{\pgfqpoint{0.000000in}{0.000000in}}%
\pgfpathlineto{\pgfqpoint{0.000000in}{-0.048611in}}%
\pgfusepath{stroke,fill}%
}%
\begin{pgfscope}%
\pgfsys@transformshift{1.173775in}{0.416447in}%
\pgfsys@useobject{currentmarker}{}%
\end{pgfscope}%
\end{pgfscope}%
\begin{pgfscope}%
\definecolor{textcolor}{rgb}{0.000000,0.000000,0.000000}%
\pgfsetstrokecolor{textcolor}%
\pgfsetfillcolor{textcolor}%
\pgftext[x=1.173775in,y=0.319225in,,top]{\color{textcolor}\rmfamily\fontsize{8.000000}{9.600000}\selectfont \(\displaystyle {5000}\)}%
\end{pgfscope}%
\begin{pgfscope}%
\pgfpathrectangle{\pgfqpoint{0.589745in}{0.416447in}}{\pgfqpoint{1.808585in}{1.371883in}}%
\pgfusepath{clip}%
\pgfsetrectcap%
\pgfsetroundjoin%
\pgfsetlinewidth{0.803000pt}%
\definecolor{currentstroke}{rgb}{0.450000,0.450000,0.450000}%
\pgfsetstrokecolor{currentstroke}%
\pgfsetdash{}{0pt}%
\pgfpathmoveto{\pgfqpoint{1.675596in}{0.416447in}}%
\pgfpathlineto{\pgfqpoint{1.675596in}{1.788330in}}%
\pgfusepath{stroke}%
\end{pgfscope}%
\begin{pgfscope}%
\pgfsetbuttcap%
\pgfsetroundjoin%
\definecolor{currentfill}{rgb}{0.000000,0.000000,0.000000}%
\pgfsetfillcolor{currentfill}%
\pgfsetlinewidth{0.803000pt}%
\definecolor{currentstroke}{rgb}{0.000000,0.000000,0.000000}%
\pgfsetstrokecolor{currentstroke}%
\pgfsetdash{}{0pt}%
\pgfsys@defobject{currentmarker}{\pgfqpoint{0.000000in}{-0.048611in}}{\pgfqpoint{0.000000in}{0.000000in}}{%
\pgfpathmoveto{\pgfqpoint{0.000000in}{0.000000in}}%
\pgfpathlineto{\pgfqpoint{0.000000in}{-0.048611in}}%
\pgfusepath{stroke,fill}%
}%
\begin{pgfscope}%
\pgfsys@transformshift{1.675596in}{0.416447in}%
\pgfsys@useobject{currentmarker}{}%
\end{pgfscope}%
\end{pgfscope}%
\begin{pgfscope}%
\definecolor{textcolor}{rgb}{0.000000,0.000000,0.000000}%
\pgfsetstrokecolor{textcolor}%
\pgfsetfillcolor{textcolor}%
\pgftext[x=1.675596in,y=0.319225in,,top]{\color{textcolor}\rmfamily\fontsize{8.000000}{9.600000}\selectfont \(\displaystyle {10000}\)}%
\end{pgfscope}%
\begin{pgfscope}%
\pgfpathrectangle{\pgfqpoint{0.589745in}{0.416447in}}{\pgfqpoint{1.808585in}{1.371883in}}%
\pgfusepath{clip}%
\pgfsetrectcap%
\pgfsetroundjoin%
\pgfsetlinewidth{0.803000pt}%
\definecolor{currentstroke}{rgb}{0.450000,0.450000,0.450000}%
\pgfsetstrokecolor{currentstroke}%
\pgfsetdash{}{0pt}%
\pgfpathmoveto{\pgfqpoint{2.177418in}{0.416447in}}%
\pgfpathlineto{\pgfqpoint{2.177418in}{1.788330in}}%
\pgfusepath{stroke}%
\end{pgfscope}%
\begin{pgfscope}%
\pgfsetbuttcap%
\pgfsetroundjoin%
\definecolor{currentfill}{rgb}{0.000000,0.000000,0.000000}%
\pgfsetfillcolor{currentfill}%
\pgfsetlinewidth{0.803000pt}%
\definecolor{currentstroke}{rgb}{0.000000,0.000000,0.000000}%
\pgfsetstrokecolor{currentstroke}%
\pgfsetdash{}{0pt}%
\pgfsys@defobject{currentmarker}{\pgfqpoint{0.000000in}{-0.048611in}}{\pgfqpoint{0.000000in}{0.000000in}}{%
\pgfpathmoveto{\pgfqpoint{0.000000in}{0.000000in}}%
\pgfpathlineto{\pgfqpoint{0.000000in}{-0.048611in}}%
\pgfusepath{stroke,fill}%
}%
\begin{pgfscope}%
\pgfsys@transformshift{2.177418in}{0.416447in}%
\pgfsys@useobject{currentmarker}{}%
\end{pgfscope}%
\end{pgfscope}%
\begin{pgfscope}%
\definecolor{textcolor}{rgb}{0.000000,0.000000,0.000000}%
\pgfsetstrokecolor{textcolor}%
\pgfsetfillcolor{textcolor}%
\pgftext[x=2.177418in,y=0.319225in,,top]{\color{textcolor}\rmfamily\fontsize{8.000000}{9.600000}\selectfont \(\displaystyle {15000}\)}%
\end{pgfscope}%
\begin{pgfscope}%
\definecolor{textcolor}{rgb}{0.000000,0.000000,0.000000}%
\pgfsetstrokecolor{textcolor}%
\pgfsetfillcolor{textcolor}%
\pgftext[x=1.494037in,y=0.165003in,,top]{\color{textcolor}\rmfamily\fontsize{10.000000}{12.000000}\selectfont Time in \(\displaystyle \unit{\second}\)}%
\end{pgfscope}%
\begin{pgfscope}%
\pgfpathrectangle{\pgfqpoint{0.589745in}{0.416447in}}{\pgfqpoint{1.808585in}{1.371883in}}%
\pgfusepath{clip}%
\pgfsetrectcap%
\pgfsetroundjoin%
\pgfsetlinewidth{0.803000pt}%
\definecolor{currentstroke}{rgb}{0.450000,0.450000,0.450000}%
\pgfsetstrokecolor{currentstroke}%
\pgfsetdash{}{0pt}%
\pgfpathmoveto{\pgfqpoint{0.589745in}{0.416447in}}%
\pgfpathlineto{\pgfqpoint{2.398330in}{0.416447in}}%
\pgfusepath{stroke}%
\end{pgfscope}%
\begin{pgfscope}%
\pgfsetbuttcap%
\pgfsetroundjoin%
\definecolor{currentfill}{rgb}{0.000000,0.000000,0.000000}%
\pgfsetfillcolor{currentfill}%
\pgfsetlinewidth{0.803000pt}%
\definecolor{currentstroke}{rgb}{0.000000,0.000000,0.000000}%
\pgfsetstrokecolor{currentstroke}%
\pgfsetdash{}{0pt}%
\pgfsys@defobject{currentmarker}{\pgfqpoint{-0.048611in}{0.000000in}}{\pgfqpoint{-0.000000in}{0.000000in}}{%
\pgfpathmoveto{\pgfqpoint{-0.000000in}{0.000000in}}%
\pgfpathlineto{\pgfqpoint{-0.048611in}{0.000000in}}%
\pgfusepath{stroke,fill}%
}%
\begin{pgfscope}%
\pgfsys@transformshift{0.589745in}{0.416447in}%
\pgfsys@useobject{currentmarker}{}%
\end{pgfscope}%
\end{pgfscope}%
\begin{pgfscope}%
\definecolor{textcolor}{rgb}{0.000000,0.000000,0.000000}%
\pgfsetstrokecolor{textcolor}%
\pgfsetfillcolor{textcolor}%
\pgftext[x=0.223614in, y=0.377892in, left, base]{\color{textcolor}\rmfamily\fontsize{8.000000}{9.600000}\selectfont \(\displaystyle {\ensuremath{-}200}\)}%
\end{pgfscope}%
\begin{pgfscope}%
\pgfpathrectangle{\pgfqpoint{0.589745in}{0.416447in}}{\pgfqpoint{1.808585in}{1.371883in}}%
\pgfusepath{clip}%
\pgfsetrectcap%
\pgfsetroundjoin%
\pgfsetlinewidth{0.803000pt}%
\definecolor{currentstroke}{rgb}{0.450000,0.450000,0.450000}%
\pgfsetstrokecolor{currentstroke}%
\pgfsetdash{}{0pt}%
\pgfpathmoveto{\pgfqpoint{0.589745in}{0.721310in}}%
\pgfpathlineto{\pgfqpoint{2.398330in}{0.721310in}}%
\pgfusepath{stroke}%
\end{pgfscope}%
\begin{pgfscope}%
\pgfsetbuttcap%
\pgfsetroundjoin%
\definecolor{currentfill}{rgb}{0.000000,0.000000,0.000000}%
\pgfsetfillcolor{currentfill}%
\pgfsetlinewidth{0.803000pt}%
\definecolor{currentstroke}{rgb}{0.000000,0.000000,0.000000}%
\pgfsetstrokecolor{currentstroke}%
\pgfsetdash{}{0pt}%
\pgfsys@defobject{currentmarker}{\pgfqpoint{-0.048611in}{0.000000in}}{\pgfqpoint{-0.000000in}{0.000000in}}{%
\pgfpathmoveto{\pgfqpoint{-0.000000in}{0.000000in}}%
\pgfpathlineto{\pgfqpoint{-0.048611in}{0.000000in}}%
\pgfusepath{stroke,fill}%
}%
\begin{pgfscope}%
\pgfsys@transformshift{0.589745in}{0.721310in}%
\pgfsys@useobject{currentmarker}{}%
\end{pgfscope}%
\end{pgfscope}%
\begin{pgfscope}%
\definecolor{textcolor}{rgb}{0.000000,0.000000,0.000000}%
\pgfsetstrokecolor{textcolor}%
\pgfsetfillcolor{textcolor}%
\pgftext[x=0.223614in, y=0.682755in, left, base]{\color{textcolor}\rmfamily\fontsize{8.000000}{9.600000}\selectfont \(\displaystyle {\ensuremath{-}100}\)}%
\end{pgfscope}%
\begin{pgfscope}%
\pgfpathrectangle{\pgfqpoint{0.589745in}{0.416447in}}{\pgfqpoint{1.808585in}{1.371883in}}%
\pgfusepath{clip}%
\pgfsetrectcap%
\pgfsetroundjoin%
\pgfsetlinewidth{0.803000pt}%
\definecolor{currentstroke}{rgb}{0.450000,0.450000,0.450000}%
\pgfsetstrokecolor{currentstroke}%
\pgfsetdash{}{0pt}%
\pgfpathmoveto{\pgfqpoint{0.589745in}{1.026173in}}%
\pgfpathlineto{\pgfqpoint{2.398330in}{1.026173in}}%
\pgfusepath{stroke}%
\end{pgfscope}%
\begin{pgfscope}%
\pgfsetbuttcap%
\pgfsetroundjoin%
\definecolor{currentfill}{rgb}{0.000000,0.000000,0.000000}%
\pgfsetfillcolor{currentfill}%
\pgfsetlinewidth{0.803000pt}%
\definecolor{currentstroke}{rgb}{0.000000,0.000000,0.000000}%
\pgfsetstrokecolor{currentstroke}%
\pgfsetdash{}{0pt}%
\pgfsys@defobject{currentmarker}{\pgfqpoint{-0.048611in}{0.000000in}}{\pgfqpoint{-0.000000in}{0.000000in}}{%
\pgfpathmoveto{\pgfqpoint{-0.000000in}{0.000000in}}%
\pgfpathlineto{\pgfqpoint{-0.048611in}{0.000000in}}%
\pgfusepath{stroke,fill}%
}%
\begin{pgfscope}%
\pgfsys@transformshift{0.589745in}{1.026173in}%
\pgfsys@useobject{currentmarker}{}%
\end{pgfscope}%
\end{pgfscope}%
\begin{pgfscope}%
\definecolor{textcolor}{rgb}{0.000000,0.000000,0.000000}%
\pgfsetstrokecolor{textcolor}%
\pgfsetfillcolor{textcolor}%
\pgftext[x=0.433494in, y=0.987618in, left, base]{\color{textcolor}\rmfamily\fontsize{8.000000}{9.600000}\selectfont \(\displaystyle {0}\)}%
\end{pgfscope}%
\begin{pgfscope}%
\pgfpathrectangle{\pgfqpoint{0.589745in}{0.416447in}}{\pgfqpoint{1.808585in}{1.371883in}}%
\pgfusepath{clip}%
\pgfsetrectcap%
\pgfsetroundjoin%
\pgfsetlinewidth{0.803000pt}%
\definecolor{currentstroke}{rgb}{0.450000,0.450000,0.450000}%
\pgfsetstrokecolor{currentstroke}%
\pgfsetdash{}{0pt}%
\pgfpathmoveto{\pgfqpoint{0.589745in}{1.331036in}}%
\pgfpathlineto{\pgfqpoint{2.398330in}{1.331036in}}%
\pgfusepath{stroke}%
\end{pgfscope}%
\begin{pgfscope}%
\pgfsetbuttcap%
\pgfsetroundjoin%
\definecolor{currentfill}{rgb}{0.000000,0.000000,0.000000}%
\pgfsetfillcolor{currentfill}%
\pgfsetlinewidth{0.803000pt}%
\definecolor{currentstroke}{rgb}{0.000000,0.000000,0.000000}%
\pgfsetstrokecolor{currentstroke}%
\pgfsetdash{}{0pt}%
\pgfsys@defobject{currentmarker}{\pgfqpoint{-0.048611in}{0.000000in}}{\pgfqpoint{-0.000000in}{0.000000in}}{%
\pgfpathmoveto{\pgfqpoint{-0.000000in}{0.000000in}}%
\pgfpathlineto{\pgfqpoint{-0.048611in}{0.000000in}}%
\pgfusepath{stroke,fill}%
}%
\begin{pgfscope}%
\pgfsys@transformshift{0.589745in}{1.331036in}%
\pgfsys@useobject{currentmarker}{}%
\end{pgfscope}%
\end{pgfscope}%
\begin{pgfscope}%
\definecolor{textcolor}{rgb}{0.000000,0.000000,0.000000}%
\pgfsetstrokecolor{textcolor}%
\pgfsetfillcolor{textcolor}%
\pgftext[x=0.315437in, y=1.292480in, left, base]{\color{textcolor}\rmfamily\fontsize{8.000000}{9.600000}\selectfont \(\displaystyle {100}\)}%
\end{pgfscope}%
\begin{pgfscope}%
\pgfpathrectangle{\pgfqpoint{0.589745in}{0.416447in}}{\pgfqpoint{1.808585in}{1.371883in}}%
\pgfusepath{clip}%
\pgfsetrectcap%
\pgfsetroundjoin%
\pgfsetlinewidth{0.803000pt}%
\definecolor{currentstroke}{rgb}{0.450000,0.450000,0.450000}%
\pgfsetstrokecolor{currentstroke}%
\pgfsetdash{}{0pt}%
\pgfpathmoveto{\pgfqpoint{0.589745in}{1.635899in}}%
\pgfpathlineto{\pgfqpoint{2.398330in}{1.635899in}}%
\pgfusepath{stroke}%
\end{pgfscope}%
\begin{pgfscope}%
\pgfsetbuttcap%
\pgfsetroundjoin%
\definecolor{currentfill}{rgb}{0.000000,0.000000,0.000000}%
\pgfsetfillcolor{currentfill}%
\pgfsetlinewidth{0.803000pt}%
\definecolor{currentstroke}{rgb}{0.000000,0.000000,0.000000}%
\pgfsetstrokecolor{currentstroke}%
\pgfsetdash{}{0pt}%
\pgfsys@defobject{currentmarker}{\pgfqpoint{-0.048611in}{0.000000in}}{\pgfqpoint{-0.000000in}{0.000000in}}{%
\pgfpathmoveto{\pgfqpoint{-0.000000in}{0.000000in}}%
\pgfpathlineto{\pgfqpoint{-0.048611in}{0.000000in}}%
\pgfusepath{stroke,fill}%
}%
\begin{pgfscope}%
\pgfsys@transformshift{0.589745in}{1.635899in}%
\pgfsys@useobject{currentmarker}{}%
\end{pgfscope}%
\end{pgfscope}%
\begin{pgfscope}%
\definecolor{textcolor}{rgb}{0.000000,0.000000,0.000000}%
\pgfsetstrokecolor{textcolor}%
\pgfsetfillcolor{textcolor}%
\pgftext[x=0.315437in, y=1.597343in, left, base]{\color{textcolor}\rmfamily\fontsize{8.000000}{9.600000}\selectfont \(\displaystyle {200}\)}%
\end{pgfscope}%
\begin{pgfscope}%
\definecolor{textcolor}{rgb}{0.000000,0.000000,0.000000}%
\pgfsetstrokecolor{textcolor}%
\pgfsetfillcolor{textcolor}%
\pgftext[x=0.168059in,y=1.102389in,,bottom,rotate=90.000000]{\color{textcolor}\rmfamily\fontsize{10.000000}{12.000000}\selectfont Ampl. in arb. unit}%
\end{pgfscope}%
\begin{pgfscope}%
\pgfpathrectangle{\pgfqpoint{0.589745in}{0.416447in}}{\pgfqpoint{1.808585in}{1.371883in}}%
\pgfusepath{clip}%
\pgfsetrectcap%
\pgfsetroundjoin%
\pgfsetlinewidth{1.505625pt}%
\definecolor{currentstroke}{rgb}{0.835294,0.368627,0.000000}%
\pgfsetstrokecolor{currentstroke}%
\pgfsetdash{}{0pt}%
\pgfpathmoveto{\pgfqpoint{0.671953in}{1.028066in}}%
\pgfpathlineto{\pgfqpoint{0.672756in}{1.049832in}}%
\pgfpathlineto{\pgfqpoint{0.673158in}{1.036101in}}%
\pgfpathlineto{\pgfqpoint{0.674663in}{0.997909in}}%
\pgfpathlineto{\pgfqpoint{0.677072in}{0.962747in}}%
\pgfpathlineto{\pgfqpoint{0.674964in}{1.002972in}}%
\pgfpathlineto{\pgfqpoint{0.677272in}{0.971421in}}%
\pgfpathlineto{\pgfqpoint{0.677774in}{0.977175in}}%
\pgfpathlineto{\pgfqpoint{0.678176in}{0.961506in}}%
\pgfpathlineto{\pgfqpoint{0.679179in}{0.994482in}}%
\pgfpathlineto{\pgfqpoint{0.679681in}{0.984354in}}%
\pgfpathlineto{\pgfqpoint{0.679882in}{0.972699in}}%
\pgfpathlineto{\pgfqpoint{0.680785in}{0.979216in}}%
\pgfpathlineto{\pgfqpoint{0.680986in}{0.984843in}}%
\pgfpathlineto{\pgfqpoint{0.681287in}{0.977336in}}%
\pgfpathlineto{\pgfqpoint{0.682291in}{0.954746in}}%
\pgfpathlineto{\pgfqpoint{0.682491in}{0.966839in}}%
\pgfpathlineto{\pgfqpoint{0.682692in}{0.969121in}}%
\pgfpathlineto{\pgfqpoint{0.682792in}{0.968728in}}%
\pgfpathlineto{\pgfqpoint{0.682993in}{0.958456in}}%
\pgfpathlineto{\pgfqpoint{0.683796in}{0.976008in}}%
\pgfpathlineto{\pgfqpoint{0.684398in}{0.990053in}}%
\pgfpathlineto{\pgfqpoint{0.683997in}{0.975383in}}%
\pgfpathlineto{\pgfqpoint{0.684800in}{0.979700in}}%
\pgfpathlineto{\pgfqpoint{0.685302in}{0.963911in}}%
\pgfpathlineto{\pgfqpoint{0.685904in}{0.965256in}}%
\pgfpathlineto{\pgfqpoint{0.686004in}{0.972158in}}%
\pgfpathlineto{\pgfqpoint{0.686606in}{0.956641in}}%
\pgfpathlineto{\pgfqpoint{0.686907in}{0.962288in}}%
\pgfpathlineto{\pgfqpoint{0.688614in}{0.991476in}}%
\pgfpathlineto{\pgfqpoint{0.687409in}{0.959529in}}%
\pgfpathlineto{\pgfqpoint{0.688915in}{0.981503in}}%
\pgfpathlineto{\pgfqpoint{0.689115in}{0.976788in}}%
\pgfpathlineto{\pgfqpoint{0.689617in}{0.992162in}}%
\pgfpathlineto{\pgfqpoint{0.689718in}{0.990765in}}%
\pgfpathlineto{\pgfqpoint{0.689918in}{1.008432in}}%
\pgfpathlineto{\pgfqpoint{0.690822in}{0.993033in}}%
\pgfpathlineto{\pgfqpoint{0.691825in}{0.983116in}}%
\pgfpathlineto{\pgfqpoint{0.691123in}{0.996328in}}%
\pgfpathlineto{\pgfqpoint{0.692026in}{0.987966in}}%
\pgfpathlineto{\pgfqpoint{0.692427in}{0.987023in}}%
\pgfpathlineto{\pgfqpoint{0.693933in}{1.040110in}}%
\pgfpathlineto{\pgfqpoint{0.695238in}{1.017163in}}%
\pgfpathlineto{\pgfqpoint{0.695438in}{1.031831in}}%
\pgfpathlineto{\pgfqpoint{0.696342in}{1.019238in}}%
\pgfpathlineto{\pgfqpoint{0.696643in}{1.011514in}}%
\pgfpathlineto{\pgfqpoint{0.696843in}{1.022975in}}%
\pgfpathlineto{\pgfqpoint{0.697044in}{1.021163in}}%
\pgfpathlineto{\pgfqpoint{0.697245in}{1.037821in}}%
\pgfpathlineto{\pgfqpoint{0.698048in}{1.031419in}}%
\pgfpathlineto{\pgfqpoint{0.698750in}{0.999717in}}%
\pgfpathlineto{\pgfqpoint{0.699252in}{1.006632in}}%
\pgfpathlineto{\pgfqpoint{0.699854in}{1.000105in}}%
\pgfpathlineto{\pgfqpoint{0.700156in}{1.012575in}}%
\pgfpathlineto{\pgfqpoint{0.700256in}{1.006039in}}%
\pgfpathlineto{\pgfqpoint{0.700356in}{1.017302in}}%
\pgfpathlineto{\pgfqpoint{0.701159in}{1.000681in}}%
\pgfpathlineto{\pgfqpoint{0.701360in}{1.007053in}}%
\pgfpathlineto{\pgfqpoint{0.701460in}{1.003394in}}%
\pgfpathlineto{\pgfqpoint{0.701862in}{1.017384in}}%
\pgfpathlineto{\pgfqpoint{0.702163in}{1.013994in}}%
\pgfpathlineto{\pgfqpoint{0.704270in}{1.056851in}}%
\pgfpathlineto{\pgfqpoint{0.704371in}{1.051539in}}%
\pgfpathlineto{\pgfqpoint{0.704471in}{1.045128in}}%
\pgfpathlineto{\pgfqpoint{0.704873in}{1.062037in}}%
\pgfpathlineto{\pgfqpoint{0.705274in}{1.054059in}}%
\pgfpathlineto{\pgfqpoint{0.705374in}{1.059208in}}%
\pgfpathlineto{\pgfqpoint{0.705977in}{1.047238in}}%
\pgfpathlineto{\pgfqpoint{0.706177in}{1.049818in}}%
\pgfpathlineto{\pgfqpoint{0.706880in}{1.025880in}}%
\pgfpathlineto{\pgfqpoint{0.707884in}{1.029316in}}%
\pgfpathlineto{\pgfqpoint{0.709790in}{1.080850in}}%
\pgfpathlineto{\pgfqpoint{0.710092in}{1.068983in}}%
\pgfpathlineto{\pgfqpoint{0.710393in}{1.050983in}}%
\pgfpathlineto{\pgfqpoint{0.711196in}{1.056344in}}%
\pgfpathlineto{\pgfqpoint{0.711898in}{1.073799in}}%
\pgfpathlineto{\pgfqpoint{0.712300in}{1.062491in}}%
\pgfpathlineto{\pgfqpoint{0.713203in}{1.050084in}}%
\pgfpathlineto{\pgfqpoint{0.713002in}{1.064123in}}%
\pgfpathlineto{\pgfqpoint{0.713504in}{1.054633in}}%
\pgfpathlineto{\pgfqpoint{0.714809in}{1.108735in}}%
\pgfpathlineto{\pgfqpoint{0.715009in}{1.093200in}}%
\pgfpathlineto{\pgfqpoint{0.715612in}{1.101723in}}%
\pgfpathlineto{\pgfqpoint{0.716214in}{1.076086in}}%
\pgfpathlineto{\pgfqpoint{0.716314in}{1.075615in}}%
\pgfpathlineto{\pgfqpoint{0.717920in}{1.040832in}}%
\pgfpathlineto{\pgfqpoint{0.718723in}{1.054311in}}%
\pgfpathlineto{\pgfqpoint{0.719024in}{1.047016in}}%
\pgfpathlineto{\pgfqpoint{0.719225in}{1.026861in}}%
\pgfpathlineto{\pgfqpoint{0.719927in}{1.068307in}}%
\pgfpathlineto{\pgfqpoint{0.720128in}{1.062586in}}%
\pgfpathlineto{\pgfqpoint{0.720730in}{1.034548in}}%
\pgfpathlineto{\pgfqpoint{0.721332in}{1.051746in}}%
\pgfpathlineto{\pgfqpoint{0.721533in}{1.062634in}}%
\pgfpathlineto{\pgfqpoint{0.721935in}{1.044227in}}%
\pgfpathlineto{\pgfqpoint{0.722436in}{1.051352in}}%
\pgfpathlineto{\pgfqpoint{0.725748in}{0.981338in}}%
\pgfpathlineto{\pgfqpoint{0.725949in}{0.983605in}}%
\pgfpathlineto{\pgfqpoint{0.726351in}{0.992463in}}%
\pgfpathlineto{\pgfqpoint{0.726551in}{0.975203in}}%
\pgfpathlineto{\pgfqpoint{0.726852in}{0.979138in}}%
\pgfpathlineto{\pgfqpoint{0.727154in}{0.975584in}}%
\pgfpathlineto{\pgfqpoint{0.727455in}{0.984877in}}%
\pgfpathlineto{\pgfqpoint{0.727956in}{0.977703in}}%
\pgfpathlineto{\pgfqpoint{0.728458in}{0.999413in}}%
\pgfpathlineto{\pgfqpoint{0.728759in}{0.981173in}}%
\pgfpathlineto{\pgfqpoint{0.729462in}{0.983337in}}%
\pgfpathlineto{\pgfqpoint{0.729863in}{0.969168in}}%
\pgfpathlineto{\pgfqpoint{0.730566in}{0.992405in}}%
\pgfpathlineto{\pgfqpoint{0.731168in}{0.983211in}}%
\pgfpathlineto{\pgfqpoint{0.732573in}{1.006312in}}%
\pgfpathlineto{\pgfqpoint{0.731871in}{0.981349in}}%
\pgfpathlineto{\pgfqpoint{0.732674in}{1.002226in}}%
\pgfpathlineto{\pgfqpoint{0.732874in}{0.986262in}}%
\pgfpathlineto{\pgfqpoint{0.733677in}{1.007033in}}%
\pgfpathlineto{\pgfqpoint{0.734480in}{1.038358in}}%
\pgfpathlineto{\pgfqpoint{0.735383in}{1.031892in}}%
\pgfpathlineto{\pgfqpoint{0.736889in}{0.988141in}}%
\pgfpathlineto{\pgfqpoint{0.736989in}{0.991817in}}%
\pgfpathlineto{\pgfqpoint{0.737893in}{1.015159in}}%
\pgfpathlineto{\pgfqpoint{0.738194in}{1.008480in}}%
\pgfpathlineto{\pgfqpoint{0.738294in}{1.008307in}}%
\pgfpathlineto{\pgfqpoint{0.739699in}{0.975054in}}%
\pgfpathlineto{\pgfqpoint{0.738796in}{1.008831in}}%
\pgfpathlineto{\pgfqpoint{0.739900in}{0.989247in}}%
\pgfpathlineto{\pgfqpoint{0.740101in}{0.999772in}}%
\pgfpathlineto{\pgfqpoint{0.740502in}{0.987171in}}%
\pgfpathlineto{\pgfqpoint{0.741004in}{0.993788in}}%
\pgfpathlineto{\pgfqpoint{0.741104in}{0.991882in}}%
\pgfpathlineto{\pgfqpoint{0.741706in}{0.998767in}}%
\pgfpathlineto{\pgfqpoint{0.741907in}{0.997040in}}%
\pgfpathlineto{\pgfqpoint{0.742208in}{0.998842in}}%
\pgfpathlineto{\pgfqpoint{0.742409in}{0.987686in}}%
\pgfpathlineto{\pgfqpoint{0.742911in}{0.974178in}}%
\pgfpathlineto{\pgfqpoint{0.743513in}{0.982771in}}%
\pgfpathlineto{\pgfqpoint{0.743814in}{0.989993in}}%
\pgfpathlineto{\pgfqpoint{0.744316in}{0.973249in}}%
\pgfpathlineto{\pgfqpoint{0.744517in}{0.978076in}}%
\pgfpathlineto{\pgfqpoint{0.745420in}{0.961110in}}%
\pgfpathlineto{\pgfqpoint{0.744818in}{0.979037in}}%
\pgfpathlineto{\pgfqpoint{0.745821in}{0.975581in}}%
\pgfpathlineto{\pgfqpoint{0.746022in}{0.989732in}}%
\pgfpathlineto{\pgfqpoint{0.746624in}{0.968994in}}%
\pgfpathlineto{\pgfqpoint{0.746725in}{0.969981in}}%
\pgfpathlineto{\pgfqpoint{0.747327in}{0.938170in}}%
\pgfpathlineto{\pgfqpoint{0.748230in}{0.952971in}}%
\pgfpathlineto{\pgfqpoint{0.749535in}{0.996289in}}%
\pgfpathlineto{\pgfqpoint{0.750237in}{0.979622in}}%
\pgfpathlineto{\pgfqpoint{0.750639in}{0.993517in}}%
\pgfpathlineto{\pgfqpoint{0.750840in}{1.000345in}}%
\pgfpathlineto{\pgfqpoint{0.751442in}{0.989261in}}%
\pgfpathlineto{\pgfqpoint{0.751542in}{0.983553in}}%
\pgfpathlineto{\pgfqpoint{0.752144in}{0.997765in}}%
\pgfpathlineto{\pgfqpoint{0.752445in}{0.993107in}}%
\pgfpathlineto{\pgfqpoint{0.753248in}{0.986423in}}%
\pgfpathlineto{\pgfqpoint{0.753449in}{0.995101in}}%
\pgfpathlineto{\pgfqpoint{0.754754in}{1.033470in}}%
\pgfpathlineto{\pgfqpoint{0.753650in}{0.992706in}}%
\pgfpathlineto{\pgfqpoint{0.755055in}{1.027003in}}%
\pgfpathlineto{\pgfqpoint{0.755557in}{1.005513in}}%
\pgfpathlineto{\pgfqpoint{0.756360in}{1.011107in}}%
\pgfpathlineto{\pgfqpoint{0.756861in}{1.031097in}}%
\pgfpathlineto{\pgfqpoint{0.758166in}{1.025694in}}%
\pgfpathlineto{\pgfqpoint{0.758266in}{1.017690in}}%
\pgfpathlineto{\pgfqpoint{0.758969in}{1.036814in}}%
\pgfpathlineto{\pgfqpoint{0.759571in}{1.030961in}}%
\pgfpathlineto{\pgfqpoint{0.760173in}{1.058417in}}%
\pgfpathlineto{\pgfqpoint{0.760374in}{1.053030in}}%
\pgfpathlineto{\pgfqpoint{0.760675in}{1.071895in}}%
\pgfpathlineto{\pgfqpoint{0.764188in}{1.151844in}}%
\pgfpathlineto{\pgfqpoint{0.764790in}{1.138745in}}%
\pgfpathlineto{\pgfqpoint{0.765192in}{1.145724in}}%
\pgfpathlineto{\pgfqpoint{0.765292in}{1.155658in}}%
\pgfpathlineto{\pgfqpoint{0.765794in}{1.130681in}}%
\pgfpathlineto{\pgfqpoint{0.766095in}{1.132730in}}%
\pgfpathlineto{\pgfqpoint{0.766998in}{1.125637in}}%
\pgfpathlineto{\pgfqpoint{0.766597in}{1.143237in}}%
\pgfpathlineto{\pgfqpoint{0.767199in}{1.125648in}}%
\pgfpathlineto{\pgfqpoint{0.768905in}{1.171845in}}%
\pgfpathlineto{\pgfqpoint{0.767400in}{1.120898in}}%
\pgfpathlineto{\pgfqpoint{0.769608in}{1.159795in}}%
\pgfpathlineto{\pgfqpoint{0.770712in}{1.125184in}}%
\pgfpathlineto{\pgfqpoint{0.770912in}{1.137755in}}%
\pgfpathlineto{\pgfqpoint{0.771213in}{1.137831in}}%
\pgfpathlineto{\pgfqpoint{0.771615in}{1.116854in}}%
\pgfpathlineto{\pgfqpoint{0.771715in}{1.115370in}}%
\pgfpathlineto{\pgfqpoint{0.771916in}{1.128246in}}%
\pgfpathlineto{\pgfqpoint{0.772016in}{1.125230in}}%
\pgfpathlineto{\pgfqpoint{0.775027in}{1.200484in}}%
\pgfpathlineto{\pgfqpoint{0.775228in}{1.193724in}}%
\pgfpathlineto{\pgfqpoint{0.775931in}{1.181994in}}%
\pgfpathlineto{\pgfqpoint{0.776232in}{1.197600in}}%
\pgfpathlineto{\pgfqpoint{0.776332in}{1.195177in}}%
\pgfpathlineto{\pgfqpoint{0.776834in}{1.207868in}}%
\pgfpathlineto{\pgfqpoint{0.776934in}{1.199544in}}%
\pgfpathlineto{\pgfqpoint{0.778239in}{1.221994in}}%
\pgfpathlineto{\pgfqpoint{0.778841in}{1.189170in}}%
\pgfpathlineto{\pgfqpoint{0.779544in}{1.190561in}}%
\pgfpathlineto{\pgfqpoint{0.779945in}{1.182735in}}%
\pgfpathlineto{\pgfqpoint{0.780949in}{1.211720in}}%
\pgfpathlineto{\pgfqpoint{0.781350in}{1.208741in}}%
\pgfpathlineto{\pgfqpoint{0.781651in}{1.216726in}}%
\pgfpathlineto{\pgfqpoint{0.781752in}{1.216300in}}%
\pgfpathlineto{\pgfqpoint{0.781852in}{1.216715in}}%
\pgfpathlineto{\pgfqpoint{0.782354in}{1.202808in}}%
\pgfpathlineto{\pgfqpoint{0.782655in}{1.218950in}}%
\pgfpathlineto{\pgfqpoint{0.782956in}{1.211132in}}%
\pgfpathlineto{\pgfqpoint{0.783056in}{1.207839in}}%
\pgfpathlineto{\pgfqpoint{0.783558in}{1.225298in}}%
\pgfpathlineto{\pgfqpoint{0.783759in}{1.228020in}}%
\pgfpathlineto{\pgfqpoint{0.784662in}{1.265084in}}%
\pgfpathlineto{\pgfqpoint{0.785064in}{1.256308in}}%
\pgfpathlineto{\pgfqpoint{0.786368in}{1.289034in}}%
\pgfpathlineto{\pgfqpoint{0.787171in}{1.278628in}}%
\pgfpathlineto{\pgfqpoint{0.787573in}{1.251662in}}%
\pgfpathlineto{\pgfqpoint{0.788075in}{1.279109in}}%
\pgfpathlineto{\pgfqpoint{0.788275in}{1.264196in}}%
\pgfpathlineto{\pgfqpoint{0.789480in}{1.302526in}}%
\pgfpathlineto{\pgfqpoint{0.788576in}{1.260267in}}%
\pgfpathlineto{\pgfqpoint{0.789580in}{1.302195in}}%
\pgfpathlineto{\pgfqpoint{0.791688in}{1.255855in}}%
\pgfpathlineto{\pgfqpoint{0.791788in}{1.258181in}}%
\pgfpathlineto{\pgfqpoint{0.791889in}{1.258077in}}%
\pgfpathlineto{\pgfqpoint{0.792792in}{1.271761in}}%
\pgfpathlineto{\pgfqpoint{0.792993in}{1.264554in}}%
\pgfpathlineto{\pgfqpoint{0.793494in}{1.252366in}}%
\pgfpathlineto{\pgfqpoint{0.793193in}{1.266386in}}%
\pgfpathlineto{\pgfqpoint{0.794097in}{1.255628in}}%
\pgfpathlineto{\pgfqpoint{0.795201in}{1.282405in}}%
\pgfpathlineto{\pgfqpoint{0.795301in}{1.279462in}}%
\pgfpathlineto{\pgfqpoint{0.795502in}{1.271762in}}%
\pgfpathlineto{\pgfqpoint{0.796104in}{1.298736in}}%
\pgfpathlineto{\pgfqpoint{0.796204in}{1.307334in}}%
\pgfpathlineto{\pgfqpoint{0.797007in}{1.282050in}}%
\pgfpathlineto{\pgfqpoint{0.797107in}{1.283285in}}%
\pgfpathlineto{\pgfqpoint{0.797208in}{1.275069in}}%
\pgfpathlineto{\pgfqpoint{0.797409in}{1.281218in}}%
\pgfpathlineto{\pgfqpoint{0.797509in}{1.270690in}}%
\pgfpathlineto{\pgfqpoint{0.798011in}{1.283805in}}%
\pgfpathlineto{\pgfqpoint{0.798412in}{1.283137in}}%
\pgfpathlineto{\pgfqpoint{0.798513in}{1.285982in}}%
\pgfpathlineto{\pgfqpoint{0.799014in}{1.269190in}}%
\pgfpathlineto{\pgfqpoint{0.799115in}{1.265306in}}%
\pgfpathlineto{\pgfqpoint{0.799416in}{1.280812in}}%
\pgfpathlineto{\pgfqpoint{0.799817in}{1.280527in}}%
\pgfpathlineto{\pgfqpoint{0.800620in}{1.302924in}}%
\pgfpathlineto{\pgfqpoint{0.800921in}{1.282048in}}%
\pgfpathlineto{\pgfqpoint{0.801222in}{1.267536in}}%
\pgfpathlineto{\pgfqpoint{0.802427in}{1.295199in}}%
\pgfpathlineto{\pgfqpoint{0.802828in}{1.288373in}}%
\pgfpathlineto{\pgfqpoint{0.803230in}{1.296714in}}%
\pgfpathlineto{\pgfqpoint{0.803330in}{1.296582in}}%
\pgfpathlineto{\pgfqpoint{0.803732in}{1.311241in}}%
\pgfpathlineto{\pgfqpoint{0.804233in}{1.292873in}}%
\pgfpathlineto{\pgfqpoint{0.804334in}{1.293172in}}%
\pgfpathlineto{\pgfqpoint{0.804735in}{1.298506in}}%
\pgfpathlineto{\pgfqpoint{0.805036in}{1.282354in}}%
\pgfpathlineto{\pgfqpoint{0.805137in}{1.275966in}}%
\pgfpathlineto{\pgfqpoint{0.805739in}{1.301182in}}%
\pgfpathlineto{\pgfqpoint{0.805839in}{1.299402in}}%
\pgfpathlineto{\pgfqpoint{0.806140in}{1.292516in}}%
\pgfpathlineto{\pgfqpoint{0.806642in}{1.302874in}}%
\pgfpathlineto{\pgfqpoint{0.806742in}{1.307888in}}%
\pgfpathlineto{\pgfqpoint{0.807244in}{1.283091in}}%
\pgfpathlineto{\pgfqpoint{0.808649in}{1.248354in}}%
\pgfpathlineto{\pgfqpoint{0.808750in}{1.254672in}}%
\pgfpathlineto{\pgfqpoint{0.809352in}{1.240837in}}%
\pgfpathlineto{\pgfqpoint{0.809553in}{1.242631in}}%
\pgfpathlineto{\pgfqpoint{0.809653in}{1.234111in}}%
\pgfpathlineto{\pgfqpoint{0.810356in}{1.252756in}}%
\pgfpathlineto{\pgfqpoint{0.810456in}{1.250179in}}%
\pgfpathlineto{\pgfqpoint{0.811660in}{1.285384in}}%
\pgfpathlineto{\pgfqpoint{0.811861in}{1.284430in}}%
\pgfpathlineto{\pgfqpoint{0.812363in}{1.269528in}}%
\pgfpathlineto{\pgfqpoint{0.812764in}{1.287679in}}%
\pgfpathlineto{\pgfqpoint{0.812965in}{1.276175in}}%
\pgfpathlineto{\pgfqpoint{0.814470in}{1.327342in}}%
\pgfpathlineto{\pgfqpoint{0.814571in}{1.325415in}}%
\pgfpathlineto{\pgfqpoint{0.814872in}{1.305084in}}%
\pgfpathlineto{\pgfqpoint{0.815876in}{1.315491in}}%
\pgfpathlineto{\pgfqpoint{0.815976in}{1.314178in}}%
\pgfpathlineto{\pgfqpoint{0.816177in}{1.325258in}}%
\pgfpathlineto{\pgfqpoint{0.818284in}{1.374242in}}%
\pgfpathlineto{\pgfqpoint{0.818385in}{1.373511in}}%
\pgfpathlineto{\pgfqpoint{0.818786in}{1.358386in}}%
\pgfpathlineto{\pgfqpoint{0.819388in}{1.374748in}}%
\pgfpathlineto{\pgfqpoint{0.819489in}{1.371528in}}%
\pgfpathlineto{\pgfqpoint{0.820292in}{1.364757in}}%
\pgfpathlineto{\pgfqpoint{0.820091in}{1.371753in}}%
\pgfpathlineto{\pgfqpoint{0.820492in}{1.366603in}}%
\pgfpathlineto{\pgfqpoint{0.820894in}{1.379722in}}%
\pgfpathlineto{\pgfqpoint{0.821195in}{1.354879in}}%
\pgfpathlineto{\pgfqpoint{0.821596in}{1.369161in}}%
\pgfpathlineto{\pgfqpoint{0.821797in}{1.378278in}}%
\pgfpathlineto{\pgfqpoint{0.823303in}{1.425233in}}%
\pgfpathlineto{\pgfqpoint{0.823905in}{1.403461in}}%
\pgfpathlineto{\pgfqpoint{0.824407in}{1.425614in}}%
\pgfpathlineto{\pgfqpoint{0.825109in}{1.444309in}}%
\pgfpathlineto{\pgfqpoint{0.824808in}{1.420460in}}%
\pgfpathlineto{\pgfqpoint{0.825611in}{1.428928in}}%
\pgfpathlineto{\pgfqpoint{0.825812in}{1.425688in}}%
\pgfpathlineto{\pgfqpoint{0.826414in}{1.395677in}}%
\pgfpathlineto{\pgfqpoint{0.827116in}{1.407011in}}%
\pgfpathlineto{\pgfqpoint{0.827317in}{1.417534in}}%
\pgfpathlineto{\pgfqpoint{0.827819in}{1.405732in}}%
\pgfpathlineto{\pgfqpoint{0.828321in}{1.414705in}}%
\pgfpathlineto{\pgfqpoint{0.828421in}{1.414592in}}%
\pgfpathlineto{\pgfqpoint{0.830328in}{1.380650in}}%
\pgfpathlineto{\pgfqpoint{0.831031in}{1.389592in}}%
\pgfpathlineto{\pgfqpoint{0.831332in}{1.383738in}}%
\pgfpathlineto{\pgfqpoint{0.831934in}{1.351921in}}%
\pgfpathlineto{\pgfqpoint{0.832938in}{1.353400in}}%
\pgfpathlineto{\pgfqpoint{0.833038in}{1.354048in}}%
\pgfpathlineto{\pgfqpoint{0.834042in}{1.374943in}}%
\pgfpathlineto{\pgfqpoint{0.833540in}{1.353205in}}%
\pgfpathlineto{\pgfqpoint{0.834343in}{1.374786in}}%
\pgfpathlineto{\pgfqpoint{0.834443in}{1.375384in}}%
\pgfpathlineto{\pgfqpoint{0.834543in}{1.367789in}}%
\pgfpathlineto{\pgfqpoint{0.835246in}{1.389470in}}%
\pgfpathlineto{\pgfqpoint{0.836250in}{1.398289in}}%
\pgfpathlineto{\pgfqpoint{0.836049in}{1.385162in}}%
\pgfpathlineto{\pgfqpoint{0.836350in}{1.392652in}}%
\pgfpathlineto{\pgfqpoint{0.836450in}{1.392780in}}%
\pgfpathlineto{\pgfqpoint{0.837655in}{1.428825in}}%
\pgfpathlineto{\pgfqpoint{0.837956in}{1.427188in}}%
\pgfpathlineto{\pgfqpoint{0.838357in}{1.435739in}}%
\pgfpathlineto{\pgfqpoint{0.838458in}{1.442694in}}%
\pgfpathlineto{\pgfqpoint{0.839160in}{1.426442in}}%
\pgfpathlineto{\pgfqpoint{0.839361in}{1.433768in}}%
\pgfpathlineto{\pgfqpoint{0.839762in}{1.423630in}}%
\pgfpathlineto{\pgfqpoint{0.840063in}{1.436885in}}%
\pgfpathlineto{\pgfqpoint{0.840264in}{1.450190in}}%
\pgfpathlineto{\pgfqpoint{0.841268in}{1.447006in}}%
\pgfpathlineto{\pgfqpoint{0.841770in}{1.443722in}}%
\pgfpathlineto{\pgfqpoint{0.841669in}{1.450830in}}%
\pgfpathlineto{\pgfqpoint{0.841870in}{1.444390in}}%
\pgfpathlineto{\pgfqpoint{0.843175in}{1.477548in}}%
\pgfpathlineto{\pgfqpoint{0.843275in}{1.476848in}}%
\pgfpathlineto{\pgfqpoint{0.843677in}{1.465440in}}%
\pgfpathlineto{\pgfqpoint{0.843476in}{1.477548in}}%
\pgfpathlineto{\pgfqpoint{0.844379in}{1.470815in}}%
\pgfpathlineto{\pgfqpoint{0.845383in}{1.493182in}}%
\pgfpathlineto{\pgfqpoint{0.845885in}{1.492678in}}%
\pgfpathlineto{\pgfqpoint{0.846186in}{1.472941in}}%
\pgfpathlineto{\pgfqpoint{0.846989in}{1.488416in}}%
\pgfpathlineto{\pgfqpoint{0.847390in}{1.502498in}}%
\pgfpathlineto{\pgfqpoint{0.847791in}{1.478683in}}%
\pgfpathlineto{\pgfqpoint{0.847992in}{1.488636in}}%
\pgfpathlineto{\pgfqpoint{0.848093in}{1.487910in}}%
\pgfpathlineto{\pgfqpoint{0.848193in}{1.492942in}}%
\pgfpathlineto{\pgfqpoint{0.849598in}{1.530442in}}%
\pgfpathlineto{\pgfqpoint{0.849698in}{1.527155in}}%
\pgfpathlineto{\pgfqpoint{0.849799in}{1.528182in}}%
\pgfpathlineto{\pgfqpoint{0.849899in}{1.524262in}}%
\pgfpathlineto{\pgfqpoint{0.851505in}{1.482142in}}%
\pgfpathlineto{\pgfqpoint{0.851605in}{1.482749in}}%
\pgfpathlineto{\pgfqpoint{0.852207in}{1.476716in}}%
\pgfpathlineto{\pgfqpoint{0.851906in}{1.486574in}}%
\pgfpathlineto{\pgfqpoint{0.852408in}{1.482510in}}%
\pgfpathlineto{\pgfqpoint{0.852509in}{1.492549in}}%
\pgfpathlineto{\pgfqpoint{0.853010in}{1.462609in}}%
\pgfpathlineto{\pgfqpoint{0.853211in}{1.466706in}}%
\pgfpathlineto{\pgfqpoint{0.853813in}{1.474556in}}%
\pgfpathlineto{\pgfqpoint{0.854516in}{1.446251in}}%
\pgfpathlineto{\pgfqpoint{0.854616in}{1.446657in}}%
\pgfpathlineto{\pgfqpoint{0.854717in}{1.445470in}}%
\pgfpathlineto{\pgfqpoint{0.855118in}{1.439504in}}%
\pgfpathlineto{\pgfqpoint{0.855620in}{1.451504in}}%
\pgfpathlineto{\pgfqpoint{0.855821in}{1.458053in}}%
\pgfpathlineto{\pgfqpoint{0.856523in}{1.449309in}}%
\pgfpathlineto{\pgfqpoint{0.857025in}{1.432383in}}%
\pgfpathlineto{\pgfqpoint{0.857627in}{1.447829in}}%
\pgfpathlineto{\pgfqpoint{0.857728in}{1.441526in}}%
\pgfpathlineto{\pgfqpoint{0.858430in}{1.457054in}}%
\pgfpathlineto{\pgfqpoint{0.858530in}{1.456344in}}%
\pgfpathlineto{\pgfqpoint{0.859233in}{1.465093in}}%
\pgfpathlineto{\pgfqpoint{0.859032in}{1.453692in}}%
\pgfpathlineto{\pgfqpoint{0.859434in}{1.457163in}}%
\pgfpathlineto{\pgfqpoint{0.859735in}{1.448563in}}%
\pgfpathlineto{\pgfqpoint{0.860036in}{1.466969in}}%
\pgfpathlineto{\pgfqpoint{0.860538in}{1.454846in}}%
\pgfpathlineto{\pgfqpoint{0.860638in}{1.463377in}}%
\pgfpathlineto{\pgfqpoint{0.861441in}{1.450456in}}%
\pgfpathlineto{\pgfqpoint{0.861541in}{1.453341in}}%
\pgfpathlineto{\pgfqpoint{0.861642in}{1.453375in}}%
\pgfpathlineto{\pgfqpoint{0.862144in}{1.446705in}}%
\pgfpathlineto{\pgfqpoint{0.862344in}{1.458108in}}%
\pgfpathlineto{\pgfqpoint{0.862445in}{1.456548in}}%
\pgfpathlineto{\pgfqpoint{0.863047in}{1.468409in}}%
\pgfpathlineto{\pgfqpoint{0.863248in}{1.454165in}}%
\pgfpathlineto{\pgfqpoint{0.863348in}{1.447860in}}%
\pgfpathlineto{\pgfqpoint{0.863850in}{1.475559in}}%
\pgfpathlineto{\pgfqpoint{0.864251in}{1.455746in}}%
\pgfpathlineto{\pgfqpoint{0.864452in}{1.452365in}}%
\pgfpathlineto{\pgfqpoint{0.865255in}{1.432688in}}%
\pgfpathlineto{\pgfqpoint{0.864954in}{1.452413in}}%
\pgfpathlineto{\pgfqpoint{0.865556in}{1.442349in}}%
\pgfpathlineto{\pgfqpoint{0.867362in}{1.466323in}}%
\pgfpathlineto{\pgfqpoint{0.865857in}{1.441956in}}%
\pgfpathlineto{\pgfqpoint{0.867463in}{1.466052in}}%
\pgfpathlineto{\pgfqpoint{0.868165in}{1.447954in}}%
\pgfpathlineto{\pgfqpoint{0.868466in}{1.471277in}}%
\pgfpathlineto{\pgfqpoint{0.868567in}{1.470987in}}%
\pgfpathlineto{\pgfqpoint{0.868667in}{1.472495in}}%
\pgfpathlineto{\pgfqpoint{0.869571in}{1.488312in}}%
\pgfpathlineto{\pgfqpoint{0.869872in}{1.477277in}}%
\pgfpathlineto{\pgfqpoint{0.869972in}{1.477178in}}%
\pgfpathlineto{\pgfqpoint{0.870775in}{1.468409in}}%
\pgfpathlineto{\pgfqpoint{0.870273in}{1.486066in}}%
\pgfpathlineto{\pgfqpoint{0.870976in}{1.477996in}}%
\pgfpathlineto{\pgfqpoint{0.872080in}{1.511887in}}%
\pgfpathlineto{\pgfqpoint{0.872682in}{1.497967in}}%
\pgfpathlineto{\pgfqpoint{0.874488in}{1.454058in}}%
\pgfpathlineto{\pgfqpoint{0.874689in}{1.455839in}}%
\pgfpathlineto{\pgfqpoint{0.874789in}{1.463023in}}%
\pgfpathlineto{\pgfqpoint{0.874990in}{1.440168in}}%
\pgfpathlineto{\pgfqpoint{0.875793in}{1.457336in}}%
\pgfpathlineto{\pgfqpoint{0.876195in}{1.482248in}}%
\pgfpathlineto{\pgfqpoint{0.876897in}{1.462291in}}%
\pgfpathlineto{\pgfqpoint{0.878202in}{1.443660in}}%
\pgfpathlineto{\pgfqpoint{0.877600in}{1.467265in}}%
\pgfpathlineto{\pgfqpoint{0.878302in}{1.448277in}}%
\pgfpathlineto{\pgfqpoint{0.879005in}{1.457371in}}%
\pgfpathlineto{\pgfqpoint{0.879105in}{1.454294in}}%
\pgfpathlineto{\pgfqpoint{0.879306in}{1.440707in}}%
\pgfpathlineto{\pgfqpoint{0.879908in}{1.466246in}}%
\pgfpathlineto{\pgfqpoint{0.880109in}{1.461315in}}%
\pgfpathlineto{\pgfqpoint{0.880711in}{1.470988in}}%
\pgfpathlineto{\pgfqpoint{0.880811in}{1.463363in}}%
\pgfpathlineto{\pgfqpoint{0.881413in}{1.436029in}}%
\pgfpathlineto{\pgfqpoint{0.882016in}{1.449770in}}%
\pgfpathlineto{\pgfqpoint{0.882216in}{1.452910in}}%
\pgfpathlineto{\pgfqpoint{0.882317in}{1.447250in}}%
\pgfpathlineto{\pgfqpoint{0.883019in}{1.435659in}}%
\pgfpathlineto{\pgfqpoint{0.883421in}{1.441213in}}%
\pgfpathlineto{\pgfqpoint{0.883521in}{1.444398in}}%
\pgfpathlineto{\pgfqpoint{0.883923in}{1.435589in}}%
\pgfpathlineto{\pgfqpoint{0.884224in}{1.435584in}}%
\pgfpathlineto{\pgfqpoint{0.884525in}{1.415795in}}%
\pgfpathlineto{\pgfqpoint{0.885328in}{1.433844in}}%
\pgfpathlineto{\pgfqpoint{0.886331in}{1.454330in}}%
\pgfpathlineto{\pgfqpoint{0.886532in}{1.440038in}}%
\pgfpathlineto{\pgfqpoint{0.887134in}{1.450068in}}%
\pgfpathlineto{\pgfqpoint{0.887235in}{1.441615in}}%
\pgfpathlineto{\pgfqpoint{0.888339in}{1.423624in}}%
\pgfpathlineto{\pgfqpoint{0.888439in}{1.430391in}}%
\pgfpathlineto{\pgfqpoint{0.889041in}{1.445766in}}%
\pgfpathlineto{\pgfqpoint{0.888640in}{1.427467in}}%
\pgfpathlineto{\pgfqpoint{0.889643in}{1.435924in}}%
\pgfpathlineto{\pgfqpoint{0.890045in}{1.430225in}}%
\pgfpathlineto{\pgfqpoint{0.890948in}{1.447872in}}%
\pgfpathlineto{\pgfqpoint{0.891149in}{1.444290in}}%
\pgfpathlineto{\pgfqpoint{0.891651in}{1.427765in}}%
\pgfpathlineto{\pgfqpoint{0.892353in}{1.433553in}}%
\pgfpathlineto{\pgfqpoint{0.892454in}{1.430166in}}%
\pgfpathlineto{\pgfqpoint{0.892855in}{1.445062in}}%
\pgfpathlineto{\pgfqpoint{0.893156in}{1.435677in}}%
\pgfpathlineto{\pgfqpoint{0.893357in}{1.445392in}}%
\pgfpathlineto{\pgfqpoint{0.894260in}{1.439212in}}%
\pgfpathlineto{\pgfqpoint{0.894360in}{1.438863in}}%
\pgfpathlineto{\pgfqpoint{0.895264in}{1.457351in}}%
\pgfpathlineto{\pgfqpoint{0.895565in}{1.449645in}}%
\pgfpathlineto{\pgfqpoint{0.896468in}{1.434867in}}%
\pgfpathlineto{\pgfqpoint{0.895766in}{1.450205in}}%
\pgfpathlineto{\pgfqpoint{0.896769in}{1.442136in}}%
\pgfpathlineto{\pgfqpoint{0.897171in}{1.453403in}}%
\pgfpathlineto{\pgfqpoint{0.897472in}{1.440634in}}%
\pgfpathlineto{\pgfqpoint{0.897873in}{1.445144in}}%
\pgfpathlineto{\pgfqpoint{0.897974in}{1.445080in}}%
\pgfpathlineto{\pgfqpoint{0.898475in}{1.429097in}}%
\pgfpathlineto{\pgfqpoint{0.899078in}{1.439050in}}%
\pgfpathlineto{\pgfqpoint{0.899178in}{1.445347in}}%
\pgfpathlineto{\pgfqpoint{0.899881in}{1.430246in}}%
\pgfpathlineto{\pgfqpoint{0.899981in}{1.430426in}}%
\pgfpathlineto{\pgfqpoint{0.901085in}{1.408415in}}%
\pgfpathlineto{\pgfqpoint{0.901286in}{1.414992in}}%
\pgfpathlineto{\pgfqpoint{0.901486in}{1.410509in}}%
\pgfpathlineto{\pgfqpoint{0.901587in}{1.418019in}}%
\pgfpathlineto{\pgfqpoint{0.902490in}{1.439972in}}%
\pgfpathlineto{\pgfqpoint{0.902791in}{1.432319in}}%
\pgfpathlineto{\pgfqpoint{0.902891in}{1.430292in}}%
\pgfpathlineto{\pgfqpoint{0.903092in}{1.442151in}}%
\pgfpathlineto{\pgfqpoint{0.904297in}{1.490120in}}%
\pgfpathlineto{\pgfqpoint{0.904397in}{1.486861in}}%
\pgfpathlineto{\pgfqpoint{0.904598in}{1.478103in}}%
\pgfpathlineto{\pgfqpoint{0.904999in}{1.495796in}}%
\pgfpathlineto{\pgfqpoint{0.905099in}{1.495767in}}%
\pgfpathlineto{\pgfqpoint{0.905601in}{1.499646in}}%
\pgfpathlineto{\pgfqpoint{0.905902in}{1.489700in}}%
\pgfpathlineto{\pgfqpoint{0.906003in}{1.487815in}}%
\pgfpathlineto{\pgfqpoint{0.906505in}{1.498498in}}%
\pgfpathlineto{\pgfqpoint{0.906605in}{1.495011in}}%
\pgfpathlineto{\pgfqpoint{0.906906in}{1.503522in}}%
\pgfpathlineto{\pgfqpoint{0.907408in}{1.488083in}}%
\pgfpathlineto{\pgfqpoint{0.907508in}{1.491745in}}%
\pgfpathlineto{\pgfqpoint{0.907609in}{1.489675in}}%
\pgfpathlineto{\pgfqpoint{0.907910in}{1.502139in}}%
\pgfpathlineto{\pgfqpoint{0.908110in}{1.501326in}}%
\pgfpathlineto{\pgfqpoint{0.908211in}{1.502706in}}%
\pgfpathlineto{\pgfqpoint{0.908512in}{1.492134in}}%
\pgfpathlineto{\pgfqpoint{0.908612in}{1.494291in}}%
\pgfpathlineto{\pgfqpoint{0.909516in}{1.466586in}}%
\pgfpathlineto{\pgfqpoint{0.910017in}{1.472887in}}%
\pgfpathlineto{\pgfqpoint{0.910218in}{1.486436in}}%
\pgfpathlineto{\pgfqpoint{0.910820in}{1.466583in}}%
\pgfpathlineto{\pgfqpoint{0.911021in}{1.468141in}}%
\pgfpathlineto{\pgfqpoint{0.911121in}{1.468314in}}%
\pgfpathlineto{\pgfqpoint{0.913028in}{1.418386in}}%
\pgfpathlineto{\pgfqpoint{0.911623in}{1.468379in}}%
\pgfpathlineto{\pgfqpoint{0.913129in}{1.427222in}}%
\pgfpathlineto{\pgfqpoint{0.914433in}{1.443230in}}%
\pgfpathlineto{\pgfqpoint{0.914634in}{1.438289in}}%
\pgfpathlineto{\pgfqpoint{0.915738in}{1.453031in}}%
\pgfpathlineto{\pgfqpoint{0.916641in}{1.438719in}}%
\pgfpathlineto{\pgfqpoint{0.916742in}{1.447957in}}%
\pgfpathlineto{\pgfqpoint{0.916842in}{1.455252in}}%
\pgfpathlineto{\pgfqpoint{0.917645in}{1.443336in}}%
\pgfpathlineto{\pgfqpoint{0.918046in}{1.429972in}}%
\pgfpathlineto{\pgfqpoint{0.918649in}{1.444759in}}%
\pgfpathlineto{\pgfqpoint{0.918849in}{1.449082in}}%
\pgfpathlineto{\pgfqpoint{0.920455in}{1.471615in}}%
\pgfpathlineto{\pgfqpoint{0.919050in}{1.445093in}}%
\pgfpathlineto{\pgfqpoint{0.920656in}{1.468205in}}%
\pgfpathlineto{\pgfqpoint{0.920756in}{1.465006in}}%
\pgfpathlineto{\pgfqpoint{0.920957in}{1.484606in}}%
\pgfpathlineto{\pgfqpoint{0.921359in}{1.480151in}}%
\pgfpathlineto{\pgfqpoint{0.921459in}{1.479911in}}%
\pgfpathlineto{\pgfqpoint{0.921559in}{1.480045in}}%
\pgfpathlineto{\pgfqpoint{0.921961in}{1.463589in}}%
\pgfpathlineto{\pgfqpoint{0.922663in}{1.471890in}}%
\pgfpathlineto{\pgfqpoint{0.923466in}{1.487090in}}%
\pgfpathlineto{\pgfqpoint{0.923265in}{1.464945in}}%
\pgfpathlineto{\pgfqpoint{0.923968in}{1.481305in}}%
\pgfpathlineto{\pgfqpoint{0.924671in}{1.464053in}}%
\pgfpathlineto{\pgfqpoint{0.924972in}{1.482682in}}%
\pgfpathlineto{\pgfqpoint{0.925072in}{1.481027in}}%
\pgfpathlineto{\pgfqpoint{0.925574in}{1.504725in}}%
\pgfpathlineto{\pgfqpoint{0.926477in}{1.500685in}}%
\pgfpathlineto{\pgfqpoint{0.927079in}{1.481231in}}%
\pgfpathlineto{\pgfqpoint{0.927581in}{1.491467in}}%
\pgfpathlineto{\pgfqpoint{0.927681in}{1.502495in}}%
\pgfpathlineto{\pgfqpoint{0.928384in}{1.480451in}}%
\pgfpathlineto{\pgfqpoint{0.928585in}{1.482200in}}%
\pgfpathlineto{\pgfqpoint{0.928685in}{1.487080in}}%
\pgfpathlineto{\pgfqpoint{0.929287in}{1.471758in}}%
\pgfpathlineto{\pgfqpoint{0.930090in}{1.463938in}}%
\pgfpathlineto{\pgfqpoint{0.930191in}{1.477776in}}%
\pgfpathlineto{\pgfqpoint{0.930291in}{1.471905in}}%
\pgfpathlineto{\pgfqpoint{0.931194in}{1.511658in}}%
\pgfpathlineto{\pgfqpoint{0.931796in}{1.502290in}}%
\pgfpathlineto{\pgfqpoint{0.933503in}{1.458839in}}%
\pgfpathlineto{\pgfqpoint{0.932198in}{1.504436in}}%
\pgfpathlineto{\pgfqpoint{0.934205in}{1.469804in}}%
\pgfpathlineto{\pgfqpoint{0.936815in}{1.533121in}}%
\pgfpathlineto{\pgfqpoint{0.934406in}{1.469456in}}%
\pgfpathlineto{\pgfqpoint{0.936915in}{1.531970in}}%
\pgfpathlineto{\pgfqpoint{0.937517in}{1.504775in}}%
\pgfpathlineto{\pgfqpoint{0.938220in}{1.517516in}}%
\pgfpathlineto{\pgfqpoint{0.938722in}{1.524792in}}%
\pgfpathlineto{\pgfqpoint{0.939123in}{1.512004in}}%
\pgfpathlineto{\pgfqpoint{0.939223in}{1.514450in}}%
\pgfpathlineto{\pgfqpoint{0.940528in}{1.487240in}}%
\pgfpathlineto{\pgfqpoint{0.940628in}{1.487744in}}%
\pgfpathlineto{\pgfqpoint{0.940729in}{1.486353in}}%
\pgfpathlineto{\pgfqpoint{0.941231in}{1.492238in}}%
\pgfpathlineto{\pgfqpoint{0.941431in}{1.496385in}}%
\pgfpathlineto{\pgfqpoint{0.941532in}{1.488277in}}%
\pgfpathlineto{\pgfqpoint{0.942034in}{1.488642in}}%
\pgfpathlineto{\pgfqpoint{0.942134in}{1.487932in}}%
\pgfpathlineto{\pgfqpoint{0.942234in}{1.489695in}}%
\pgfpathlineto{\pgfqpoint{0.944041in}{1.550708in}}%
\pgfpathlineto{\pgfqpoint{0.944342in}{1.534172in}}%
\pgfpathlineto{\pgfqpoint{0.945145in}{1.549554in}}%
\pgfpathlineto{\pgfqpoint{0.946650in}{1.595534in}}%
\pgfpathlineto{\pgfqpoint{0.947152in}{1.607858in}}%
\pgfpathlineto{\pgfqpoint{0.947554in}{1.588131in}}%
\pgfpathlineto{\pgfqpoint{0.947955in}{1.606667in}}%
\pgfpathlineto{\pgfqpoint{0.948858in}{1.598303in}}%
\pgfpathlineto{\pgfqpoint{0.949561in}{1.603341in}}%
\pgfpathlineto{\pgfqpoint{0.950364in}{1.581339in}}%
\pgfpathlineto{\pgfqpoint{0.952371in}{1.541935in}}%
\pgfpathlineto{\pgfqpoint{0.950866in}{1.585535in}}%
\pgfpathlineto{\pgfqpoint{0.952471in}{1.544962in}}%
\pgfpathlineto{\pgfqpoint{0.953074in}{1.552260in}}%
\pgfpathlineto{\pgfqpoint{0.953575in}{1.532251in}}%
\pgfpathlineto{\pgfqpoint{0.954679in}{1.558243in}}%
\pgfpathlineto{\pgfqpoint{0.953977in}{1.530210in}}%
\pgfpathlineto{\pgfqpoint{0.954981in}{1.551055in}}%
\pgfpathlineto{\pgfqpoint{0.955081in}{1.550796in}}%
\pgfpathlineto{\pgfqpoint{0.955282in}{1.543100in}}%
\pgfpathlineto{\pgfqpoint{0.955683in}{1.563379in}}%
\pgfpathlineto{\pgfqpoint{0.956085in}{1.555817in}}%
\pgfpathlineto{\pgfqpoint{0.956386in}{1.551502in}}%
\pgfpathlineto{\pgfqpoint{0.956486in}{1.561557in}}%
\pgfpathlineto{\pgfqpoint{0.956887in}{1.582580in}}%
\pgfpathlineto{\pgfqpoint{0.957590in}{1.564024in}}%
\pgfpathlineto{\pgfqpoint{0.958594in}{1.553516in}}%
\pgfpathlineto{\pgfqpoint{0.958794in}{1.560770in}}%
\pgfpathlineto{\pgfqpoint{0.962006in}{1.484182in}}%
\pgfpathlineto{\pgfqpoint{0.962307in}{1.492060in}}%
\pgfpathlineto{\pgfqpoint{0.962408in}{1.512792in}}%
\pgfpathlineto{\pgfqpoint{0.963411in}{1.503096in}}%
\pgfpathlineto{\pgfqpoint{0.963712in}{1.492555in}}%
\pgfpathlineto{\pgfqpoint{0.964314in}{1.502021in}}%
\pgfpathlineto{\pgfqpoint{0.965720in}{1.531188in}}%
\pgfpathlineto{\pgfqpoint{0.966422in}{1.551326in}}%
\pgfpathlineto{\pgfqpoint{0.966924in}{1.541458in}}%
\pgfpathlineto{\pgfqpoint{0.967827in}{1.512081in}}%
\pgfpathlineto{\pgfqpoint{0.967325in}{1.543082in}}%
\pgfpathlineto{\pgfqpoint{0.968329in}{1.528039in}}%
\pgfpathlineto{\pgfqpoint{0.968730in}{1.546935in}}%
\pgfpathlineto{\pgfqpoint{0.969232in}{1.525154in}}%
\pgfpathlineto{\pgfqpoint{0.969433in}{1.530245in}}%
\pgfpathlineto{\pgfqpoint{0.969533in}{1.532107in}}%
\pgfpathlineto{\pgfqpoint{0.969734in}{1.519482in}}%
\pgfpathlineto{\pgfqpoint{0.969834in}{1.520908in}}%
\pgfpathlineto{\pgfqpoint{0.970136in}{1.503223in}}%
\pgfpathlineto{\pgfqpoint{0.970537in}{1.525760in}}%
\pgfpathlineto{\pgfqpoint{0.970838in}{1.525216in}}%
\pgfpathlineto{\pgfqpoint{0.971240in}{1.538721in}}%
\pgfpathlineto{\pgfqpoint{0.971541in}{1.524265in}}%
\pgfpathlineto{\pgfqpoint{0.971641in}{1.524758in}}%
\pgfpathlineto{\pgfqpoint{0.971741in}{1.518790in}}%
\pgfpathlineto{\pgfqpoint{0.972143in}{1.543092in}}%
\pgfpathlineto{\pgfqpoint{0.972444in}{1.537354in}}%
\pgfpathlineto{\pgfqpoint{0.972544in}{1.538715in}}%
\pgfpathlineto{\pgfqpoint{0.972645in}{1.532158in}}%
\pgfpathlineto{\pgfqpoint{0.972845in}{1.533103in}}%
\pgfpathlineto{\pgfqpoint{0.973146in}{1.516302in}}%
\pgfpathlineto{\pgfqpoint{0.973949in}{1.519287in}}%
\pgfpathlineto{\pgfqpoint{0.974050in}{1.526288in}}%
\pgfpathlineto{\pgfqpoint{0.974652in}{1.491538in}}%
\pgfpathlineto{\pgfqpoint{0.976258in}{1.469417in}}%
\pgfpathlineto{\pgfqpoint{0.975254in}{1.499881in}}%
\pgfpathlineto{\pgfqpoint{0.976559in}{1.470279in}}%
\pgfpathlineto{\pgfqpoint{0.976659in}{1.471180in}}%
\pgfpathlineto{\pgfqpoint{0.976860in}{1.465005in}}%
\pgfpathlineto{\pgfqpoint{0.977362in}{1.470916in}}%
\pgfpathlineto{\pgfqpoint{0.978265in}{1.456354in}}%
\pgfpathlineto{\pgfqpoint{0.977864in}{1.473005in}}%
\pgfpathlineto{\pgfqpoint{0.978566in}{1.458058in}}%
\pgfpathlineto{\pgfqpoint{0.978667in}{1.459137in}}%
\pgfpathlineto{\pgfqpoint{0.978867in}{1.450700in}}%
\pgfpathlineto{\pgfqpoint{0.978968in}{1.452376in}}%
\pgfpathlineto{\pgfqpoint{0.979369in}{1.439333in}}%
\pgfpathlineto{\pgfqpoint{0.979871in}{1.455243in}}%
\pgfpathlineto{\pgfqpoint{0.979971in}{1.448540in}}%
\pgfpathlineto{\pgfqpoint{0.981075in}{1.486962in}}%
\pgfpathlineto{\pgfqpoint{0.981276in}{1.470676in}}%
\pgfpathlineto{\pgfqpoint{0.981979in}{1.458325in}}%
\pgfpathlineto{\pgfqpoint{0.981677in}{1.477469in}}%
\pgfpathlineto{\pgfqpoint{0.982480in}{1.466248in}}%
\pgfpathlineto{\pgfqpoint{0.983283in}{1.492356in}}%
\pgfpathlineto{\pgfqpoint{0.983785in}{1.488985in}}%
\pgfpathlineto{\pgfqpoint{0.984688in}{1.461872in}}%
\pgfpathlineto{\pgfqpoint{0.984989in}{1.473598in}}%
\pgfpathlineto{\pgfqpoint{0.985491in}{1.474238in}}%
\pgfpathlineto{\pgfqpoint{0.986194in}{1.459517in}}%
\pgfpathlineto{\pgfqpoint{0.986997in}{1.475159in}}%
\pgfpathlineto{\pgfqpoint{0.987298in}{1.460220in}}%
\pgfpathlineto{\pgfqpoint{0.987499in}{1.464984in}}%
\pgfpathlineto{\pgfqpoint{0.987900in}{1.452168in}}%
\pgfpathlineto{\pgfqpoint{0.988101in}{1.454665in}}%
\pgfpathlineto{\pgfqpoint{0.989506in}{1.403718in}}%
\pgfpathlineto{\pgfqpoint{0.990710in}{1.375490in}}%
\pgfpathlineto{\pgfqpoint{0.991413in}{1.379592in}}%
\pgfpathlineto{\pgfqpoint{0.992216in}{1.413040in}}%
\pgfpathlineto{\pgfqpoint{0.992617in}{1.395509in}}%
\pgfpathlineto{\pgfqpoint{0.992818in}{1.383599in}}%
\pgfpathlineto{\pgfqpoint{0.993320in}{1.397059in}}%
\pgfpathlineto{\pgfqpoint{0.993721in}{1.388760in}}%
\pgfpathlineto{\pgfqpoint{0.994524in}{1.426065in}}%
\pgfpathlineto{\pgfqpoint{0.995427in}{1.425336in}}%
\pgfpathlineto{\pgfqpoint{0.996230in}{1.407598in}}%
\pgfpathlineto{\pgfqpoint{0.995829in}{1.426362in}}%
\pgfpathlineto{\pgfqpoint{0.996732in}{1.410623in}}%
\pgfpathlineto{\pgfqpoint{0.997033in}{1.416617in}}%
\pgfpathlineto{\pgfqpoint{0.997635in}{1.434037in}}%
\pgfpathlineto{\pgfqpoint{0.998137in}{1.426324in}}%
\pgfpathlineto{\pgfqpoint{0.999542in}{1.457610in}}%
\pgfpathlineto{\pgfqpoint{1.000044in}{1.454473in}}%
\pgfpathlineto{\pgfqpoint{1.000144in}{1.449863in}}%
\pgfpathlineto{\pgfqpoint{1.000847in}{1.467953in}}%
\pgfpathlineto{\pgfqpoint{1.000947in}{1.470879in}}%
\pgfpathlineto{\pgfqpoint{1.001249in}{1.455122in}}%
\pgfpathlineto{\pgfqpoint{1.001349in}{1.459750in}}%
\pgfpathlineto{\pgfqpoint{1.002553in}{1.442577in}}%
\pgfpathlineto{\pgfqpoint{1.003356in}{1.438914in}}%
\pgfpathlineto{\pgfqpoint{1.003858in}{1.464350in}}%
\pgfpathlineto{\pgfqpoint{1.003958in}{1.464348in}}%
\pgfpathlineto{\pgfqpoint{1.004059in}{1.458107in}}%
\pgfpathlineto{\pgfqpoint{1.004259in}{1.471496in}}%
\pgfpathlineto{\pgfqpoint{1.004862in}{1.466880in}}%
\pgfpathlineto{\pgfqpoint{1.005062in}{1.480166in}}%
\pgfpathlineto{\pgfqpoint{1.005966in}{1.472430in}}%
\pgfpathlineto{\pgfqpoint{1.007070in}{1.488693in}}%
\pgfpathlineto{\pgfqpoint{1.006166in}{1.469093in}}%
\pgfpathlineto{\pgfqpoint{1.007371in}{1.488487in}}%
\pgfpathlineto{\pgfqpoint{1.007471in}{1.483393in}}%
\pgfpathlineto{\pgfqpoint{1.007873in}{1.502474in}}%
\pgfpathlineto{\pgfqpoint{1.008174in}{1.497208in}}%
\pgfpathlineto{\pgfqpoint{1.008374in}{1.499877in}}%
\pgfpathlineto{\pgfqpoint{1.008575in}{1.495867in}}%
\pgfpathlineto{\pgfqpoint{1.008675in}{1.496282in}}%
\pgfpathlineto{\pgfqpoint{1.009478in}{1.446914in}}%
\pgfpathlineto{\pgfqpoint{1.010181in}{1.451476in}}%
\pgfpathlineto{\pgfqpoint{1.010984in}{1.446643in}}%
\pgfpathlineto{\pgfqpoint{1.011486in}{1.476747in}}%
\pgfpathlineto{\pgfqpoint{1.012389in}{1.461799in}}%
\pgfpathlineto{\pgfqpoint{1.012690in}{1.467566in}}%
\pgfpathlineto{\pgfqpoint{1.013995in}{1.494234in}}%
\pgfpathlineto{\pgfqpoint{1.014095in}{1.490670in}}%
\pgfpathlineto{\pgfqpoint{1.015199in}{1.483269in}}%
\pgfpathlineto{\pgfqpoint{1.014296in}{1.499058in}}%
\pgfpathlineto{\pgfqpoint{1.015300in}{1.485518in}}%
\pgfpathlineto{\pgfqpoint{1.016705in}{1.532308in}}%
\pgfpathlineto{\pgfqpoint{1.015500in}{1.478348in}}%
\pgfpathlineto{\pgfqpoint{1.017809in}{1.519777in}}%
\pgfpathlineto{\pgfqpoint{1.018511in}{1.487294in}}%
\pgfpathlineto{\pgfqpoint{1.019113in}{1.504763in}}%
\pgfpathlineto{\pgfqpoint{1.019515in}{1.521939in}}%
\pgfpathlineto{\pgfqpoint{1.020217in}{1.508938in}}%
\pgfpathlineto{\pgfqpoint{1.020318in}{1.509214in}}%
\pgfpathlineto{\pgfqpoint{1.022024in}{1.473129in}}%
\pgfpathlineto{\pgfqpoint{1.022425in}{1.470069in}}%
\pgfpathlineto{\pgfqpoint{1.022526in}{1.472769in}}%
\pgfpathlineto{\pgfqpoint{1.023128in}{1.490424in}}%
\pgfpathlineto{\pgfqpoint{1.023630in}{1.477724in}}%
\pgfpathlineto{\pgfqpoint{1.024934in}{1.490162in}}%
\pgfpathlineto{\pgfqpoint{1.026038in}{1.456830in}}%
\pgfpathlineto{\pgfqpoint{1.026139in}{1.464737in}}%
\pgfpathlineto{\pgfqpoint{1.027544in}{1.500765in}}%
\pgfpathlineto{\pgfqpoint{1.027845in}{1.513790in}}%
\pgfpathlineto{\pgfqpoint{1.028247in}{1.496646in}}%
\pgfpathlineto{\pgfqpoint{1.028849in}{1.505885in}}%
\pgfpathlineto{\pgfqpoint{1.029049in}{1.498427in}}%
\pgfpathlineto{\pgfqpoint{1.029150in}{1.499656in}}%
\pgfpathlineto{\pgfqpoint{1.029351in}{1.484096in}}%
\pgfpathlineto{\pgfqpoint{1.029953in}{1.503805in}}%
\pgfpathlineto{\pgfqpoint{1.030455in}{1.516654in}}%
\pgfpathlineto{\pgfqpoint{1.030956in}{1.499394in}}%
\pgfpathlineto{\pgfqpoint{1.031057in}{1.493764in}}%
\pgfpathlineto{\pgfqpoint{1.031358in}{1.522154in}}%
\pgfpathlineto{\pgfqpoint{1.031860in}{1.500721in}}%
\pgfpathlineto{\pgfqpoint{1.033465in}{1.534492in}}%
\pgfpathlineto{\pgfqpoint{1.033566in}{1.538622in}}%
\pgfpathlineto{\pgfqpoint{1.034369in}{1.526544in}}%
\pgfpathlineto{\pgfqpoint{1.035071in}{1.540499in}}%
\pgfpathlineto{\pgfqpoint{1.035372in}{1.524660in}}%
\pgfpathlineto{\pgfqpoint{1.035473in}{1.524908in}}%
\pgfpathlineto{\pgfqpoint{1.035573in}{1.534929in}}%
\pgfpathlineto{\pgfqpoint{1.036376in}{1.516619in}}%
\pgfpathlineto{\pgfqpoint{1.036476in}{1.520548in}}%
\pgfpathlineto{\pgfqpoint{1.037079in}{1.526110in}}%
\pgfpathlineto{\pgfqpoint{1.037380in}{1.517295in}}%
\pgfpathlineto{\pgfqpoint{1.037480in}{1.517110in}}%
\pgfpathlineto{\pgfqpoint{1.037681in}{1.506945in}}%
\pgfpathlineto{\pgfqpoint{1.038383in}{1.518703in}}%
\pgfpathlineto{\pgfqpoint{1.038584in}{1.515132in}}%
\pgfpathlineto{\pgfqpoint{1.038885in}{1.506076in}}%
\pgfpathlineto{\pgfqpoint{1.040290in}{1.534197in}}%
\pgfpathlineto{\pgfqpoint{1.040391in}{1.528161in}}%
\pgfpathlineto{\pgfqpoint{1.040993in}{1.538421in}}%
\pgfpathlineto{\pgfqpoint{1.041394in}{1.530933in}}%
\pgfpathlineto{\pgfqpoint{1.043301in}{1.594562in}}%
\pgfpathlineto{\pgfqpoint{1.043402in}{1.589723in}}%
\pgfpathlineto{\pgfqpoint{1.044506in}{1.568697in}}%
\pgfpathlineto{\pgfqpoint{1.044104in}{1.593005in}}%
\pgfpathlineto{\pgfqpoint{1.044706in}{1.573209in}}%
\pgfpathlineto{\pgfqpoint{1.045308in}{1.615252in}}%
\pgfpathlineto{\pgfqpoint{1.045911in}{1.605234in}}%
\pgfpathlineto{\pgfqpoint{1.047215in}{1.564370in}}%
\pgfpathlineto{\pgfqpoint{1.047617in}{1.564454in}}%
\pgfpathlineto{\pgfqpoint{1.048219in}{1.549609in}}%
\pgfpathlineto{\pgfqpoint{1.048620in}{1.551941in}}%
\pgfpathlineto{\pgfqpoint{1.048721in}{1.541544in}}%
\pgfpathlineto{\pgfqpoint{1.049624in}{1.557108in}}%
\pgfpathlineto{\pgfqpoint{1.049724in}{1.557201in}}%
\pgfpathlineto{\pgfqpoint{1.050226in}{1.564814in}}%
\pgfpathlineto{\pgfqpoint{1.051732in}{1.504945in}}%
\pgfpathlineto{\pgfqpoint{1.052234in}{1.512874in}}%
\pgfpathlineto{\pgfqpoint{1.051932in}{1.501941in}}%
\pgfpathlineto{\pgfqpoint{1.052936in}{1.509358in}}%
\pgfpathlineto{\pgfqpoint{1.055245in}{1.436588in}}%
\pgfpathlineto{\pgfqpoint{1.055345in}{1.438010in}}%
\pgfpathlineto{\pgfqpoint{1.055546in}{1.438736in}}%
\pgfpathlineto{\pgfqpoint{1.055646in}{1.435198in}}%
\pgfpathlineto{\pgfqpoint{1.056449in}{1.448141in}}%
\pgfpathlineto{\pgfqpoint{1.055847in}{1.435174in}}%
\pgfpathlineto{\pgfqpoint{1.056650in}{1.435505in}}%
\pgfpathlineto{\pgfqpoint{1.057553in}{1.426129in}}%
\pgfpathlineto{\pgfqpoint{1.056951in}{1.441946in}}%
\pgfpathlineto{\pgfqpoint{1.057653in}{1.434624in}}%
\pgfpathlineto{\pgfqpoint{1.057854in}{1.444758in}}%
\pgfpathlineto{\pgfqpoint{1.058255in}{1.431184in}}%
\pgfpathlineto{\pgfqpoint{1.058657in}{1.432858in}}%
\pgfpathlineto{\pgfqpoint{1.059259in}{1.420570in}}%
\pgfpathlineto{\pgfqpoint{1.060162in}{1.425834in}}%
\pgfpathlineto{\pgfqpoint{1.060664in}{1.401829in}}%
\pgfpathlineto{\pgfqpoint{1.060965in}{1.414422in}}%
\pgfpathlineto{\pgfqpoint{1.061567in}{1.393375in}}%
\pgfpathlineto{\pgfqpoint{1.061668in}{1.393037in}}%
\pgfpathlineto{\pgfqpoint{1.061768in}{1.395868in}}%
\pgfpathlineto{\pgfqpoint{1.062671in}{1.379018in}}%
\pgfpathlineto{\pgfqpoint{1.062872in}{1.391665in}}%
\pgfpathlineto{\pgfqpoint{1.063876in}{1.413727in}}%
\pgfpathlineto{\pgfqpoint{1.064177in}{1.405501in}}%
\pgfpathlineto{\pgfqpoint{1.064879in}{1.391189in}}%
\pgfpathlineto{\pgfqpoint{1.065181in}{1.398785in}}%
\pgfpathlineto{\pgfqpoint{1.065783in}{1.396264in}}%
\pgfpathlineto{\pgfqpoint{1.066385in}{1.410074in}}%
\pgfpathlineto{\pgfqpoint{1.066887in}{1.427743in}}%
\pgfpathlineto{\pgfqpoint{1.067489in}{1.420127in}}%
\pgfpathlineto{\pgfqpoint{1.067690in}{1.405329in}}%
\pgfpathlineto{\pgfqpoint{1.068593in}{1.419881in}}%
\pgfpathlineto{\pgfqpoint{1.068994in}{1.411831in}}%
\pgfpathlineto{\pgfqpoint{1.069496in}{1.425092in}}%
\pgfpathlineto{\pgfqpoint{1.070199in}{1.436979in}}%
\pgfpathlineto{\pgfqpoint{1.069998in}{1.419522in}}%
\pgfpathlineto{\pgfqpoint{1.070801in}{1.434244in}}%
\pgfpathlineto{\pgfqpoint{1.071504in}{1.474518in}}%
\pgfpathlineto{\pgfqpoint{1.072407in}{1.453558in}}%
\pgfpathlineto{\pgfqpoint{1.073009in}{1.423676in}}%
\pgfpathlineto{\pgfqpoint{1.073712in}{1.424907in}}%
\pgfpathlineto{\pgfqpoint{1.073812in}{1.428036in}}%
\pgfpathlineto{\pgfqpoint{1.074113in}{1.407851in}}%
\pgfpathlineto{\pgfqpoint{1.074314in}{1.411315in}}%
\pgfpathlineto{\pgfqpoint{1.075518in}{1.388366in}}%
\pgfpathlineto{\pgfqpoint{1.075719in}{1.388920in}}%
\pgfpathlineto{\pgfqpoint{1.076020in}{1.381743in}}%
\pgfpathlineto{\pgfqpoint{1.076221in}{1.394420in}}%
\pgfpathlineto{\pgfqpoint{1.076321in}{1.398972in}}%
\pgfpathlineto{\pgfqpoint{1.076622in}{1.370424in}}%
\pgfpathlineto{\pgfqpoint{1.077525in}{1.365600in}}%
\pgfpathlineto{\pgfqpoint{1.077425in}{1.374132in}}%
\pgfpathlineto{\pgfqpoint{1.077626in}{1.367182in}}%
\pgfpathlineto{\pgfqpoint{1.079432in}{1.405579in}}%
\pgfpathlineto{\pgfqpoint{1.081239in}{1.367992in}}%
\pgfpathlineto{\pgfqpoint{1.081640in}{1.366421in}}%
\pgfpathlineto{\pgfqpoint{1.082744in}{1.389883in}}%
\pgfpathlineto{\pgfqpoint{1.083447in}{1.372112in}}%
\pgfpathlineto{\pgfqpoint{1.083949in}{1.383325in}}%
\pgfpathlineto{\pgfqpoint{1.084752in}{1.401356in}}%
\pgfpathlineto{\pgfqpoint{1.085053in}{1.388127in}}%
\pgfpathlineto{\pgfqpoint{1.085956in}{1.373644in}}%
\pgfpathlineto{\pgfqpoint{1.086157in}{1.379260in}}%
\pgfpathlineto{\pgfqpoint{1.087662in}{1.418263in}}%
\pgfpathlineto{\pgfqpoint{1.088164in}{1.395183in}}%
\pgfpathlineto{\pgfqpoint{1.088867in}{1.402698in}}%
\pgfpathlineto{\pgfqpoint{1.088967in}{1.405478in}}%
\pgfpathlineto{\pgfqpoint{1.089168in}{1.399526in}}%
\pgfpathlineto{\pgfqpoint{1.089569in}{1.401153in}}%
\pgfpathlineto{\pgfqpoint{1.090773in}{1.374361in}}%
\pgfpathlineto{\pgfqpoint{1.090874in}{1.376956in}}%
\pgfpathlineto{\pgfqpoint{1.090974in}{1.376907in}}%
\pgfpathlineto{\pgfqpoint{1.091777in}{1.374989in}}%
\pgfpathlineto{\pgfqpoint{1.092580in}{1.393239in}}%
\pgfpathlineto{\pgfqpoint{1.094286in}{1.309412in}}%
\pgfpathlineto{\pgfqpoint{1.095390in}{1.289618in}}%
\pgfpathlineto{\pgfqpoint{1.094989in}{1.312605in}}%
\pgfpathlineto{\pgfqpoint{1.095892in}{1.295766in}}%
\pgfpathlineto{\pgfqpoint{1.096294in}{1.309356in}}%
\pgfpathlineto{\pgfqpoint{1.096896in}{1.293176in}}%
\pgfpathlineto{\pgfqpoint{1.096996in}{1.300754in}}%
\pgfpathlineto{\pgfqpoint{1.097899in}{1.293725in}}%
\pgfpathlineto{\pgfqpoint{1.097197in}{1.306594in}}%
\pgfpathlineto{\pgfqpoint{1.098000in}{1.294592in}}%
\pgfpathlineto{\pgfqpoint{1.099104in}{1.320431in}}%
\pgfpathlineto{\pgfqpoint{1.099204in}{1.319946in}}%
\pgfpathlineto{\pgfqpoint{1.100007in}{1.325308in}}%
\pgfpathlineto{\pgfqpoint{1.101211in}{1.277586in}}%
\pgfpathlineto{\pgfqpoint{1.102416in}{1.303213in}}%
\pgfpathlineto{\pgfqpoint{1.102918in}{1.290311in}}%
\pgfpathlineto{\pgfqpoint{1.104523in}{1.237866in}}%
\pgfpathlineto{\pgfqpoint{1.104824in}{1.240341in}}%
\pgfpathlineto{\pgfqpoint{1.105627in}{1.250466in}}%
\pgfpathlineto{\pgfqpoint{1.105025in}{1.239699in}}%
\pgfpathlineto{\pgfqpoint{1.105929in}{1.246227in}}%
\pgfpathlineto{\pgfqpoint{1.106029in}{1.243082in}}%
\pgfpathlineto{\pgfqpoint{1.106430in}{1.258287in}}%
\pgfpathlineto{\pgfqpoint{1.106631in}{1.256180in}}%
\pgfpathlineto{\pgfqpoint{1.106832in}{1.259731in}}%
\pgfpathlineto{\pgfqpoint{1.106932in}{1.256322in}}%
\pgfpathlineto{\pgfqpoint{1.107334in}{1.237637in}}%
\pgfpathlineto{\pgfqpoint{1.107735in}{1.259321in}}%
\pgfpathlineto{\pgfqpoint{1.108137in}{1.247408in}}%
\pgfpathlineto{\pgfqpoint{1.108739in}{1.218831in}}%
\pgfpathlineto{\pgfqpoint{1.109542in}{1.228217in}}%
\pgfpathlineto{\pgfqpoint{1.109943in}{1.243918in}}%
\pgfpathlineto{\pgfqpoint{1.110545in}{1.226859in}}%
\pgfpathlineto{\pgfqpoint{1.111047in}{1.217945in}}%
\pgfpathlineto{\pgfqpoint{1.110947in}{1.230469in}}%
\pgfpathlineto{\pgfqpoint{1.111649in}{1.224426in}}%
\pgfpathlineto{\pgfqpoint{1.111750in}{1.225471in}}%
\pgfpathlineto{\pgfqpoint{1.111950in}{1.220128in}}%
\pgfpathlineto{\pgfqpoint{1.113155in}{1.206166in}}%
\pgfpathlineto{\pgfqpoint{1.112854in}{1.223182in}}%
\pgfpathlineto{\pgfqpoint{1.113255in}{1.211869in}}%
\pgfpathlineto{\pgfqpoint{1.114359in}{1.230931in}}%
\pgfpathlineto{\pgfqpoint{1.114560in}{1.227080in}}%
\pgfpathlineto{\pgfqpoint{1.116768in}{1.164913in}}%
\pgfpathlineto{\pgfqpoint{1.116868in}{1.161993in}}%
\pgfpathlineto{\pgfqpoint{1.117069in}{1.171864in}}%
\pgfpathlineto{\pgfqpoint{1.117671in}{1.168194in}}%
\pgfpathlineto{\pgfqpoint{1.117771in}{1.171786in}}%
\pgfpathlineto{\pgfqpoint{1.118073in}{1.152902in}}%
\pgfpathlineto{\pgfqpoint{1.118474in}{1.161257in}}%
\pgfpathlineto{\pgfqpoint{1.119277in}{1.153143in}}%
\pgfpathlineto{\pgfqpoint{1.118976in}{1.174485in}}%
\pgfpathlineto{\pgfqpoint{1.119478in}{1.162915in}}%
\pgfpathlineto{\pgfqpoint{1.120983in}{1.208788in}}%
\pgfpathlineto{\pgfqpoint{1.122188in}{1.192780in}}%
\pgfpathlineto{\pgfqpoint{1.121184in}{1.210348in}}%
\pgfpathlineto{\pgfqpoint{1.122388in}{1.197440in}}%
\pgfpathlineto{\pgfqpoint{1.122790in}{1.188938in}}%
\pgfpathlineto{\pgfqpoint{1.123191in}{1.203753in}}%
\pgfpathlineto{\pgfqpoint{1.123693in}{1.165425in}}%
\pgfpathlineto{\pgfqpoint{1.124496in}{1.170365in}}%
\pgfpathlineto{\pgfqpoint{1.124596in}{1.175951in}}%
\pgfpathlineto{\pgfqpoint{1.125399in}{1.158851in}}%
\pgfpathlineto{\pgfqpoint{1.125500in}{1.156276in}}%
\pgfpathlineto{\pgfqpoint{1.125801in}{1.175527in}}%
\pgfpathlineto{\pgfqpoint{1.126704in}{1.189839in}}%
\pgfpathlineto{\pgfqpoint{1.127005in}{1.185754in}}%
\pgfpathlineto{\pgfqpoint{1.127607in}{1.192966in}}%
\pgfpathlineto{\pgfqpoint{1.127507in}{1.182694in}}%
\pgfpathlineto{\pgfqpoint{1.127708in}{1.183350in}}%
\pgfpathlineto{\pgfqpoint{1.127808in}{1.180194in}}%
\pgfpathlineto{\pgfqpoint{1.128510in}{1.192705in}}%
\pgfpathlineto{\pgfqpoint{1.128711in}{1.201746in}}%
\pgfpathlineto{\pgfqpoint{1.129313in}{1.188391in}}%
\pgfpathlineto{\pgfqpoint{1.129414in}{1.191852in}}%
\pgfpathlineto{\pgfqpoint{1.129514in}{1.186201in}}%
\pgfpathlineto{\pgfqpoint{1.130317in}{1.198550in}}%
\pgfpathlineto{\pgfqpoint{1.130417in}{1.198624in}}%
\pgfpathlineto{\pgfqpoint{1.131120in}{1.183739in}}%
\pgfpathlineto{\pgfqpoint{1.131722in}{1.190629in}}%
\pgfpathlineto{\pgfqpoint{1.132224in}{1.210321in}}%
\pgfpathlineto{\pgfqpoint{1.132826in}{1.191403in}}%
\pgfpathlineto{\pgfqpoint{1.133027in}{1.192195in}}%
\pgfpathlineto{\pgfqpoint{1.133127in}{1.190729in}}%
\pgfpathlineto{\pgfqpoint{1.133830in}{1.226063in}}%
\pgfpathlineto{\pgfqpoint{1.134532in}{1.219270in}}%
\pgfpathlineto{\pgfqpoint{1.135536in}{1.200199in}}%
\pgfpathlineto{\pgfqpoint{1.135737in}{1.204629in}}%
\pgfpathlineto{\pgfqpoint{1.136439in}{1.217887in}}%
\pgfpathlineto{\pgfqpoint{1.136941in}{1.209759in}}%
\pgfpathlineto{\pgfqpoint{1.137443in}{1.165081in}}%
\pgfpathlineto{\pgfqpoint{1.138145in}{1.180748in}}%
\pgfpathlineto{\pgfqpoint{1.138246in}{1.180565in}}%
\pgfpathlineto{\pgfqpoint{1.138547in}{1.169504in}}%
\pgfpathlineto{\pgfqpoint{1.139049in}{1.191439in}}%
\pgfpathlineto{\pgfqpoint{1.139249in}{1.184805in}}%
\pgfpathlineto{\pgfqpoint{1.140253in}{1.191089in}}%
\pgfpathlineto{\pgfqpoint{1.139450in}{1.181868in}}%
\pgfpathlineto{\pgfqpoint{1.140554in}{1.190200in}}%
\pgfpathlineto{\pgfqpoint{1.141056in}{1.177812in}}%
\pgfpathlineto{\pgfqpoint{1.141457in}{1.193185in}}%
\pgfpathlineto{\pgfqpoint{1.141658in}{1.188920in}}%
\pgfpathlineto{\pgfqpoint{1.141759in}{1.196687in}}%
\pgfpathlineto{\pgfqpoint{1.142461in}{1.171638in}}%
\pgfpathlineto{\pgfqpoint{1.142561in}{1.171201in}}%
\pgfpathlineto{\pgfqpoint{1.142662in}{1.172485in}}%
\pgfpathlineto{\pgfqpoint{1.143364in}{1.188316in}}%
\pgfpathlineto{\pgfqpoint{1.143967in}{1.182584in}}%
\pgfpathlineto{\pgfqpoint{1.144067in}{1.181904in}}%
\pgfpathlineto{\pgfqpoint{1.145773in}{1.227274in}}%
\pgfpathlineto{\pgfqpoint{1.147379in}{1.201372in}}%
\pgfpathlineto{\pgfqpoint{1.147780in}{1.210442in}}%
\pgfpathlineto{\pgfqpoint{1.147981in}{1.221104in}}%
\pgfpathlineto{\pgfqpoint{1.148383in}{1.210011in}}%
\pgfpathlineto{\pgfqpoint{1.148884in}{1.211609in}}%
\pgfpathlineto{\pgfqpoint{1.149888in}{1.211901in}}%
\pgfpathlineto{\pgfqpoint{1.150290in}{1.200386in}}%
\pgfpathlineto{\pgfqpoint{1.150591in}{1.210109in}}%
\pgfpathlineto{\pgfqpoint{1.150992in}{1.196221in}}%
\pgfpathlineto{\pgfqpoint{1.151394in}{1.200791in}}%
\pgfpathlineto{\pgfqpoint{1.151895in}{1.172936in}}%
\pgfpathlineto{\pgfqpoint{1.152598in}{1.172956in}}%
\pgfpathlineto{\pgfqpoint{1.154204in}{1.127069in}}%
\pgfpathlineto{\pgfqpoint{1.155609in}{1.132632in}}%
\pgfpathlineto{\pgfqpoint{1.156010in}{1.126790in}}%
\pgfpathlineto{\pgfqpoint{1.156311in}{1.132724in}}%
\pgfpathlineto{\pgfqpoint{1.156412in}{1.137600in}}%
\pgfpathlineto{\pgfqpoint{1.157215in}{1.124677in}}%
\pgfpathlineto{\pgfqpoint{1.158118in}{1.132328in}}%
\pgfpathlineto{\pgfqpoint{1.158419in}{1.115064in}}%
\pgfpathlineto{\pgfqpoint{1.159423in}{1.149698in}}%
\pgfpathlineto{\pgfqpoint{1.158620in}{1.103735in}}%
\pgfpathlineto{\pgfqpoint{1.160025in}{1.130916in}}%
\pgfpathlineto{\pgfqpoint{1.160527in}{1.120855in}}%
\pgfpathlineto{\pgfqpoint{1.160727in}{1.144284in}}%
\pgfpathlineto{\pgfqpoint{1.161129in}{1.130375in}}%
\pgfpathlineto{\pgfqpoint{1.161430in}{1.121983in}}%
\pgfpathlineto{\pgfqpoint{1.162032in}{1.140945in}}%
\pgfpathlineto{\pgfqpoint{1.162534in}{1.147424in}}%
\pgfpathlineto{\pgfqpoint{1.164642in}{1.098294in}}%
\pgfpathlineto{\pgfqpoint{1.166047in}{1.145810in}}%
\pgfpathlineto{\pgfqpoint{1.166348in}{1.142055in}}%
\pgfpathlineto{\pgfqpoint{1.166749in}{1.128703in}}%
\pgfpathlineto{\pgfqpoint{1.167251in}{1.143191in}}%
\pgfpathlineto{\pgfqpoint{1.168154in}{1.160812in}}%
\pgfpathlineto{\pgfqpoint{1.168355in}{1.149858in}}%
\pgfpathlineto{\pgfqpoint{1.168455in}{1.147822in}}%
\pgfpathlineto{\pgfqpoint{1.168857in}{1.162173in}}%
\pgfpathlineto{\pgfqpoint{1.169058in}{1.169018in}}%
\pgfpathlineto{\pgfqpoint{1.169258in}{1.160649in}}%
\pgfpathlineto{\pgfqpoint{1.169961in}{1.162491in}}%
\pgfpathlineto{\pgfqpoint{1.171165in}{1.141127in}}%
\pgfpathlineto{\pgfqpoint{1.170663in}{1.172922in}}%
\pgfpathlineto{\pgfqpoint{1.171366in}{1.149017in}}%
\pgfpathlineto{\pgfqpoint{1.171466in}{1.151075in}}%
\pgfpathlineto{\pgfqpoint{1.171667in}{1.133024in}}%
\pgfpathlineto{\pgfqpoint{1.171868in}{1.139223in}}%
\pgfpathlineto{\pgfqpoint{1.171968in}{1.138445in}}%
\pgfpathlineto{\pgfqpoint{1.172069in}{1.141428in}}%
\pgfpathlineto{\pgfqpoint{1.173474in}{1.178553in}}%
\pgfpathlineto{\pgfqpoint{1.174176in}{1.155359in}}%
\pgfpathlineto{\pgfqpoint{1.174678in}{1.167287in}}%
\pgfpathlineto{\pgfqpoint{1.174879in}{1.171207in}}%
\pgfpathlineto{\pgfqpoint{1.175280in}{1.156540in}}%
\pgfpathlineto{\pgfqpoint{1.175682in}{1.169940in}}%
\pgfpathlineto{\pgfqpoint{1.176384in}{1.154362in}}%
\pgfpathlineto{\pgfqpoint{1.176886in}{1.165197in}}%
\pgfpathlineto{\pgfqpoint{1.177689in}{1.205608in}}%
\pgfpathlineto{\pgfqpoint{1.178994in}{1.201452in}}%
\pgfpathlineto{\pgfqpoint{1.179997in}{1.185049in}}%
\pgfpathlineto{\pgfqpoint{1.180298in}{1.188060in}}%
\pgfpathlineto{\pgfqpoint{1.180600in}{1.197967in}}%
\pgfpathlineto{\pgfqpoint{1.181001in}{1.185587in}}%
\pgfpathlineto{\pgfqpoint{1.181402in}{1.192875in}}%
\pgfpathlineto{\pgfqpoint{1.182506in}{1.213065in}}%
\pgfpathlineto{\pgfqpoint{1.182005in}{1.192717in}}%
\pgfpathlineto{\pgfqpoint{1.182707in}{1.199682in}}%
\pgfpathlineto{\pgfqpoint{1.182908in}{1.188605in}}%
\pgfpathlineto{\pgfqpoint{1.183309in}{1.199864in}}%
\pgfpathlineto{\pgfqpoint{1.183811in}{1.198333in}}%
\pgfpathlineto{\pgfqpoint{1.184213in}{1.203235in}}%
\pgfpathlineto{\pgfqpoint{1.184313in}{1.197796in}}%
\pgfpathlineto{\pgfqpoint{1.184413in}{1.198039in}}%
\pgfpathlineto{\pgfqpoint{1.184714in}{1.190260in}}%
\pgfpathlineto{\pgfqpoint{1.185317in}{1.206913in}}%
\pgfpathlineto{\pgfqpoint{1.185517in}{1.195445in}}%
\pgfpathlineto{\pgfqpoint{1.185618in}{1.199758in}}%
\pgfpathlineto{\pgfqpoint{1.186421in}{1.186945in}}%
\pgfpathlineto{\pgfqpoint{1.187023in}{1.199478in}}%
\pgfpathlineto{\pgfqpoint{1.187625in}{1.190634in}}%
\pgfpathlineto{\pgfqpoint{1.187725in}{1.187602in}}%
\pgfpathlineto{\pgfqpoint{1.188227in}{1.199189in}}%
\pgfpathlineto{\pgfqpoint{1.188428in}{1.197417in}}%
\pgfpathlineto{\pgfqpoint{1.189432in}{1.222023in}}%
\pgfpathlineto{\pgfqpoint{1.189733in}{1.216525in}}%
\pgfpathlineto{\pgfqpoint{1.191037in}{1.188517in}}%
\pgfpathlineto{\pgfqpoint{1.191238in}{1.200931in}}%
\pgfpathlineto{\pgfqpoint{1.191840in}{1.182495in}}%
\pgfpathlineto{\pgfqpoint{1.193547in}{1.165436in}}%
\pgfpathlineto{\pgfqpoint{1.192141in}{1.188870in}}%
\pgfpathlineto{\pgfqpoint{1.194048in}{1.170742in}}%
\pgfpathlineto{\pgfqpoint{1.194349in}{1.193811in}}%
\pgfpathlineto{\pgfqpoint{1.195253in}{1.186671in}}%
\pgfpathlineto{\pgfqpoint{1.195353in}{1.188208in}}%
\pgfpathlineto{\pgfqpoint{1.195654in}{1.176486in}}%
\pgfpathlineto{\pgfqpoint{1.195955in}{1.184794in}}%
\pgfpathlineto{\pgfqpoint{1.196056in}{1.179889in}}%
\pgfpathlineto{\pgfqpoint{1.196658in}{1.197882in}}%
\pgfpathlineto{\pgfqpoint{1.197661in}{1.220056in}}%
\pgfpathlineto{\pgfqpoint{1.197963in}{1.219034in}}%
\pgfpathlineto{\pgfqpoint{1.198163in}{1.227623in}}%
\pgfpathlineto{\pgfqpoint{1.198866in}{1.214776in}}%
\pgfpathlineto{\pgfqpoint{1.199568in}{1.179506in}}%
\pgfpathlineto{\pgfqpoint{1.200572in}{1.194115in}}%
\pgfpathlineto{\pgfqpoint{1.200873in}{1.218420in}}%
\pgfpathlineto{\pgfqpoint{1.201676in}{1.199636in}}%
\pgfpathlineto{\pgfqpoint{1.201776in}{1.196250in}}%
\pgfpathlineto{\pgfqpoint{1.202178in}{1.216867in}}%
\pgfpathlineto{\pgfqpoint{1.204286in}{1.244144in}}%
\pgfpathlineto{\pgfqpoint{1.202379in}{1.212637in}}%
\pgfpathlineto{\pgfqpoint{1.204386in}{1.243462in}}%
\pgfpathlineto{\pgfqpoint{1.205891in}{1.214310in}}%
\pgfpathlineto{\pgfqpoint{1.205992in}{1.220035in}}%
\pgfpathlineto{\pgfqpoint{1.206393in}{1.219411in}}%
\pgfpathlineto{\pgfqpoint{1.206594in}{1.223072in}}%
\pgfpathlineto{\pgfqpoint{1.207999in}{1.193612in}}%
\pgfpathlineto{\pgfqpoint{1.208601in}{1.174123in}}%
\pgfpathlineto{\pgfqpoint{1.208300in}{1.197752in}}%
\pgfpathlineto{\pgfqpoint{1.209203in}{1.183699in}}%
\pgfpathlineto{\pgfqpoint{1.209304in}{1.188424in}}%
\pgfpathlineto{\pgfqpoint{1.209806in}{1.163791in}}%
\pgfpathlineto{\pgfqpoint{1.210006in}{1.164720in}}%
\pgfpathlineto{\pgfqpoint{1.210107in}{1.159822in}}%
\pgfpathlineto{\pgfqpoint{1.211110in}{1.143664in}}%
\pgfpathlineto{\pgfqpoint{1.210709in}{1.162314in}}%
\pgfpathlineto{\pgfqpoint{1.211311in}{1.149849in}}%
\pgfpathlineto{\pgfqpoint{1.211411in}{1.163430in}}%
\pgfpathlineto{\pgfqpoint{1.212014in}{1.140657in}}%
\pgfpathlineto{\pgfqpoint{1.212415in}{1.161256in}}%
\pgfpathlineto{\pgfqpoint{1.213921in}{1.118441in}}%
\pgfpathlineto{\pgfqpoint{1.214021in}{1.121263in}}%
\pgfpathlineto{\pgfqpoint{1.214121in}{1.121744in}}%
\pgfpathlineto{\pgfqpoint{1.214222in}{1.117247in}}%
\pgfpathlineto{\pgfqpoint{1.214322in}{1.117281in}}%
\pgfpathlineto{\pgfqpoint{1.214723in}{1.120090in}}%
\pgfpathlineto{\pgfqpoint{1.215727in}{1.093103in}}%
\pgfpathlineto{\pgfqpoint{1.215827in}{1.093282in}}%
\pgfpathlineto{\pgfqpoint{1.216430in}{1.099020in}}%
\pgfpathlineto{\pgfqpoint{1.217132in}{1.077335in}}%
\pgfpathlineto{\pgfqpoint{1.217634in}{1.085487in}}%
\pgfpathlineto{\pgfqpoint{1.218035in}{1.074156in}}%
\pgfpathlineto{\pgfqpoint{1.218437in}{1.066991in}}%
\pgfpathlineto{\pgfqpoint{1.218638in}{1.078017in}}%
\pgfpathlineto{\pgfqpoint{1.218939in}{1.068840in}}%
\pgfpathlineto{\pgfqpoint{1.219340in}{1.084533in}}%
\pgfpathlineto{\pgfqpoint{1.220143in}{1.078298in}}%
\pgfpathlineto{\pgfqpoint{1.220344in}{1.067798in}}%
\pgfpathlineto{\pgfqpoint{1.220946in}{1.079616in}}%
\pgfpathlineto{\pgfqpoint{1.221247in}{1.074107in}}%
\pgfpathlineto{\pgfqpoint{1.222150in}{1.098293in}}%
\pgfpathlineto{\pgfqpoint{1.222552in}{1.092608in}}%
\pgfpathlineto{\pgfqpoint{1.223656in}{1.078874in}}%
\pgfpathlineto{\pgfqpoint{1.223254in}{1.101317in}}%
\pgfpathlineto{\pgfqpoint{1.223756in}{1.087320in}}%
\pgfpathlineto{\pgfqpoint{1.224057in}{1.094722in}}%
\pgfpathlineto{\pgfqpoint{1.224459in}{1.083765in}}%
\pgfpathlineto{\pgfqpoint{1.224760in}{1.068448in}}%
\pgfpathlineto{\pgfqpoint{1.225262in}{1.084057in}}%
\pgfpathlineto{\pgfqpoint{1.225563in}{1.082009in}}%
\pgfpathlineto{\pgfqpoint{1.226366in}{1.053078in}}%
\pgfpathlineto{\pgfqpoint{1.226968in}{1.060392in}}%
\pgfpathlineto{\pgfqpoint{1.228574in}{1.097519in}}%
\pgfpathlineto{\pgfqpoint{1.229377in}{1.093408in}}%
\pgfpathlineto{\pgfqpoint{1.228774in}{1.101934in}}%
\pgfpathlineto{\pgfqpoint{1.229577in}{1.099346in}}%
\pgfpathlineto{\pgfqpoint{1.229678in}{1.099192in}}%
\pgfpathlineto{\pgfqpoint{1.230681in}{1.112920in}}%
\pgfpathlineto{\pgfqpoint{1.230079in}{1.094327in}}%
\pgfpathlineto{\pgfqpoint{1.230882in}{1.107634in}}%
\pgfpathlineto{\pgfqpoint{1.231685in}{1.075889in}}%
\pgfpathlineto{\pgfqpoint{1.232187in}{1.102152in}}%
\pgfpathlineto{\pgfqpoint{1.232287in}{1.108317in}}%
\pgfpathlineto{\pgfqpoint{1.232588in}{1.086628in}}%
\pgfpathlineto{\pgfqpoint{1.233090in}{1.092782in}}%
\pgfpathlineto{\pgfqpoint{1.233391in}{1.091540in}}%
\pgfpathlineto{\pgfqpoint{1.234395in}{1.101093in}}%
\pgfpathlineto{\pgfqpoint{1.234596in}{1.092383in}}%
\pgfpathlineto{\pgfqpoint{1.235198in}{1.116114in}}%
\pgfpathlineto{\pgfqpoint{1.235499in}{1.099043in}}%
\pgfpathlineto{\pgfqpoint{1.236101in}{1.082519in}}%
\pgfpathlineto{\pgfqpoint{1.235700in}{1.099383in}}%
\pgfpathlineto{\pgfqpoint{1.236502in}{1.095312in}}%
\pgfpathlineto{\pgfqpoint{1.237506in}{1.111638in}}%
\pgfpathlineto{\pgfqpoint{1.236904in}{1.094087in}}%
\pgfpathlineto{\pgfqpoint{1.237807in}{1.105745in}}%
\pgfpathlineto{\pgfqpoint{1.238811in}{1.111646in}}%
\pgfpathlineto{\pgfqpoint{1.238510in}{1.090764in}}%
\pgfpathlineto{\pgfqpoint{1.239012in}{1.111155in}}%
\pgfpathlineto{\pgfqpoint{1.239413in}{1.107541in}}%
\pgfpathlineto{\pgfqpoint{1.239212in}{1.116421in}}%
\pgfpathlineto{\pgfqpoint{1.239614in}{1.112641in}}%
\pgfpathlineto{\pgfqpoint{1.239714in}{1.119920in}}%
\pgfpathlineto{\pgfqpoint{1.240417in}{1.093287in}}%
\pgfpathlineto{\pgfqpoint{1.241721in}{1.126693in}}%
\pgfpathlineto{\pgfqpoint{1.240718in}{1.091599in}}%
\pgfpathlineto{\pgfqpoint{1.242023in}{1.117233in}}%
\pgfpathlineto{\pgfqpoint{1.243227in}{1.085076in}}%
\pgfpathlineto{\pgfqpoint{1.243628in}{1.090831in}}%
\pgfpathlineto{\pgfqpoint{1.243829in}{1.093512in}}%
\pgfpathlineto{\pgfqpoint{1.244331in}{1.087539in}}%
\pgfpathlineto{\pgfqpoint{1.244431in}{1.076525in}}%
\pgfpathlineto{\pgfqpoint{1.245335in}{1.092884in}}%
\pgfpathlineto{\pgfqpoint{1.246940in}{1.062089in}}%
\pgfpathlineto{\pgfqpoint{1.245736in}{1.094568in}}%
\pgfpathlineto{\pgfqpoint{1.247543in}{1.065255in}}%
\pgfpathlineto{\pgfqpoint{1.247643in}{1.072711in}}%
\pgfpathlineto{\pgfqpoint{1.248345in}{1.049490in}}%
\pgfpathlineto{\pgfqpoint{1.248446in}{1.051426in}}%
\pgfpathlineto{\pgfqpoint{1.250453in}{1.024309in}}%
\pgfpathlineto{\pgfqpoint{1.250654in}{1.027195in}}%
\pgfpathlineto{\pgfqpoint{1.251156in}{1.032494in}}%
\pgfpathlineto{\pgfqpoint{1.250855in}{1.019706in}}%
\pgfpathlineto{\pgfqpoint{1.251356in}{1.029705in}}%
\pgfpathlineto{\pgfqpoint{1.252059in}{1.017050in}}%
\pgfpathlineto{\pgfqpoint{1.252260in}{1.030544in}}%
\pgfpathlineto{\pgfqpoint{1.252360in}{1.029815in}}%
\pgfpathlineto{\pgfqpoint{1.252661in}{1.042628in}}%
\pgfpathlineto{\pgfqpoint{1.252962in}{1.025318in}}%
\pgfpathlineto{\pgfqpoint{1.253063in}{1.012082in}}%
\pgfpathlineto{\pgfqpoint{1.253765in}{1.030015in}}%
\pgfpathlineto{\pgfqpoint{1.253966in}{1.028145in}}%
\pgfpathlineto{\pgfqpoint{1.254167in}{1.024720in}}%
\pgfpathlineto{\pgfqpoint{1.254267in}{1.034324in}}%
\pgfpathlineto{\pgfqpoint{1.255170in}{1.054839in}}%
\pgfpathlineto{\pgfqpoint{1.254668in}{1.027189in}}%
\pgfpathlineto{\pgfqpoint{1.255471in}{1.041148in}}%
\pgfpathlineto{\pgfqpoint{1.256575in}{1.019236in}}%
\pgfpathlineto{\pgfqpoint{1.256876in}{1.027458in}}%
\pgfpathlineto{\pgfqpoint{1.258181in}{0.998493in}}%
\pgfpathlineto{\pgfqpoint{1.258583in}{1.003663in}}%
\pgfpathlineto{\pgfqpoint{1.259586in}{0.958897in}}%
\pgfpathlineto{\pgfqpoint{1.261995in}{0.908773in}}%
\pgfpathlineto{\pgfqpoint{1.260088in}{0.964288in}}%
\pgfpathlineto{\pgfqpoint{1.262095in}{0.916126in}}%
\pgfpathlineto{\pgfqpoint{1.262196in}{0.919676in}}%
\pgfpathlineto{\pgfqpoint{1.262396in}{0.903944in}}%
\pgfpathlineto{\pgfqpoint{1.262698in}{0.904476in}}%
\pgfpathlineto{\pgfqpoint{1.263601in}{0.877219in}}%
\pgfpathlineto{\pgfqpoint{1.263902in}{0.888893in}}%
\pgfpathlineto{\pgfqpoint{1.264002in}{0.888522in}}%
\pgfpathlineto{\pgfqpoint{1.264705in}{0.856991in}}%
\pgfpathlineto{\pgfqpoint{1.265207in}{0.869910in}}%
\pgfpathlineto{\pgfqpoint{1.265307in}{0.872054in}}%
\pgfpathlineto{\pgfqpoint{1.265709in}{0.860416in}}%
\pgfpathlineto{\pgfqpoint{1.265809in}{0.863827in}}%
\pgfpathlineto{\pgfqpoint{1.266511in}{0.855523in}}%
\pgfpathlineto{\pgfqpoint{1.266311in}{0.869322in}}%
\pgfpathlineto{\pgfqpoint{1.266813in}{0.866190in}}%
\pgfpathlineto{\pgfqpoint{1.266913in}{0.868319in}}%
\pgfpathlineto{\pgfqpoint{1.267114in}{0.859011in}}%
\pgfpathlineto{\pgfqpoint{1.267214in}{0.859555in}}%
\pgfpathlineto{\pgfqpoint{1.267314in}{0.847336in}}%
\pgfpathlineto{\pgfqpoint{1.267917in}{0.874993in}}%
\pgfpathlineto{\pgfqpoint{1.268218in}{0.872097in}}%
\pgfpathlineto{\pgfqpoint{1.269221in}{0.854306in}}%
\pgfpathlineto{\pgfqpoint{1.268719in}{0.873929in}}%
\pgfpathlineto{\pgfqpoint{1.269422in}{0.861384in}}%
\pgfpathlineto{\pgfqpoint{1.271229in}{0.902967in}}%
\pgfpathlineto{\pgfqpoint{1.271530in}{0.890582in}}%
\pgfpathlineto{\pgfqpoint{1.272132in}{0.909736in}}%
\pgfpathlineto{\pgfqpoint{1.272433in}{0.930724in}}%
\pgfpathlineto{\pgfqpoint{1.273437in}{0.930427in}}%
\pgfpathlineto{\pgfqpoint{1.273637in}{0.937419in}}%
\pgfpathlineto{\pgfqpoint{1.273938in}{0.927282in}}%
\pgfpathlineto{\pgfqpoint{1.274741in}{0.903421in}}%
\pgfpathlineto{\pgfqpoint{1.275243in}{0.908123in}}%
\pgfpathlineto{\pgfqpoint{1.276648in}{0.948451in}}%
\pgfpathlineto{\pgfqpoint{1.276749in}{0.940496in}}%
\pgfpathlineto{\pgfqpoint{1.277752in}{0.902533in}}%
\pgfpathlineto{\pgfqpoint{1.278053in}{0.916217in}}%
\pgfpathlineto{\pgfqpoint{1.280061in}{0.975389in}}%
\pgfpathlineto{\pgfqpoint{1.281466in}{1.000188in}}%
\pgfpathlineto{\pgfqpoint{1.281566in}{0.996118in}}%
\pgfpathlineto{\pgfqpoint{1.281867in}{0.977582in}}%
\pgfpathlineto{\pgfqpoint{1.282369in}{1.001665in}}%
\pgfpathlineto{\pgfqpoint{1.282570in}{0.999404in}}%
\pgfpathlineto{\pgfqpoint{1.284477in}{0.975136in}}%
\pgfpathlineto{\pgfqpoint{1.284577in}{0.980748in}}%
\pgfpathlineto{\pgfqpoint{1.284677in}{0.993093in}}%
\pgfpathlineto{\pgfqpoint{1.285581in}{0.977737in}}%
\pgfpathlineto{\pgfqpoint{1.285681in}{0.982080in}}%
\pgfpathlineto{\pgfqpoint{1.286283in}{0.974911in}}%
\pgfpathlineto{\pgfqpoint{1.286685in}{0.986181in}}%
\pgfpathlineto{\pgfqpoint{1.286785in}{0.994092in}}%
\pgfpathlineto{\pgfqpoint{1.287588in}{0.980560in}}%
\pgfpathlineto{\pgfqpoint{1.287688in}{0.986315in}}%
\pgfpathlineto{\pgfqpoint{1.287789in}{0.983299in}}%
\pgfpathlineto{\pgfqpoint{1.288491in}{0.992418in}}%
\pgfpathlineto{\pgfqpoint{1.288592in}{0.991256in}}%
\pgfpathlineto{\pgfqpoint{1.288692in}{0.991592in}}%
\pgfpathlineto{\pgfqpoint{1.288792in}{0.989629in}}%
\pgfpathlineto{\pgfqpoint{1.289997in}{0.972568in}}%
\pgfpathlineto{\pgfqpoint{1.289093in}{0.993178in}}%
\pgfpathlineto{\pgfqpoint{1.290097in}{0.973325in}}%
\pgfpathlineto{\pgfqpoint{1.290197in}{0.977927in}}%
\pgfpathlineto{\pgfqpoint{1.290398in}{0.963228in}}%
\pgfpathlineto{\pgfqpoint{1.291101in}{0.970315in}}%
\pgfpathlineto{\pgfqpoint{1.291201in}{0.971015in}}%
\pgfpathlineto{\pgfqpoint{1.291301in}{0.969000in}}%
\pgfpathlineto{\pgfqpoint{1.291803in}{0.955200in}}%
\pgfpathlineto{\pgfqpoint{1.292104in}{0.973324in}}%
\pgfpathlineto{\pgfqpoint{1.292305in}{0.962580in}}%
\pgfpathlineto{\pgfqpoint{1.293108in}{0.989328in}}%
\pgfpathlineto{\pgfqpoint{1.293409in}{0.985142in}}%
\pgfpathlineto{\pgfqpoint{1.294212in}{0.947779in}}%
\pgfpathlineto{\pgfqpoint{1.294513in}{0.963606in}}%
\pgfpathlineto{\pgfqpoint{1.295216in}{0.972575in}}%
\pgfpathlineto{\pgfqpoint{1.295517in}{0.958451in}}%
\pgfpathlineto{\pgfqpoint{1.295918in}{0.942127in}}%
\pgfpathlineto{\pgfqpoint{1.296420in}{0.964991in}}%
\pgfpathlineto{\pgfqpoint{1.296520in}{0.967741in}}%
\pgfpathlineto{\pgfqpoint{1.296721in}{0.958480in}}%
\pgfpathlineto{\pgfqpoint{1.297323in}{0.960249in}}%
\pgfpathlineto{\pgfqpoint{1.298227in}{0.939812in}}%
\pgfpathlineto{\pgfqpoint{1.298427in}{0.951754in}}%
\pgfpathlineto{\pgfqpoint{1.298929in}{0.948471in}}%
\pgfpathlineto{\pgfqpoint{1.298829in}{0.957418in}}%
\pgfpathlineto{\pgfqpoint{1.299230in}{0.948889in}}%
\pgfpathlineto{\pgfqpoint{1.299431in}{0.964412in}}%
\pgfpathlineto{\pgfqpoint{1.300033in}{0.943914in}}%
\pgfpathlineto{\pgfqpoint{1.300334in}{0.954643in}}%
\pgfpathlineto{\pgfqpoint{1.301237in}{0.969060in}}%
\pgfpathlineto{\pgfqpoint{1.300535in}{0.953101in}}%
\pgfpathlineto{\pgfqpoint{1.301840in}{0.965015in}}%
\pgfpathlineto{\pgfqpoint{1.301940in}{0.958630in}}%
\pgfpathlineto{\pgfqpoint{1.302442in}{0.970748in}}%
\pgfpathlineto{\pgfqpoint{1.302843in}{0.962554in}}%
\pgfpathlineto{\pgfqpoint{1.303446in}{0.979984in}}%
\pgfpathlineto{\pgfqpoint{1.303947in}{0.964554in}}%
\pgfpathlineto{\pgfqpoint{1.304248in}{0.952682in}}%
\pgfpathlineto{\pgfqpoint{1.305152in}{0.957642in}}%
\pgfpathlineto{\pgfqpoint{1.306456in}{0.995602in}}%
\pgfpathlineto{\pgfqpoint{1.305352in}{0.957497in}}%
\pgfpathlineto{\pgfqpoint{1.306657in}{0.985547in}}%
\pgfpathlineto{\pgfqpoint{1.306758in}{0.982818in}}%
\pgfpathlineto{\pgfqpoint{1.307159in}{1.000984in}}%
\pgfpathlineto{\pgfqpoint{1.307560in}{0.989343in}}%
\pgfpathlineto{\pgfqpoint{1.307761in}{0.993435in}}%
\pgfpathlineto{\pgfqpoint{1.308062in}{0.991583in}}%
\pgfpathlineto{\pgfqpoint{1.309066in}{0.979486in}}%
\pgfpathlineto{\pgfqpoint{1.309166in}{0.994165in}}%
\pgfpathlineto{\pgfqpoint{1.309969in}{0.965911in}}%
\pgfpathlineto{\pgfqpoint{1.310070in}{0.957995in}}%
\pgfpathlineto{\pgfqpoint{1.310672in}{0.986358in}}%
\pgfpathlineto{\pgfqpoint{1.311073in}{0.966316in}}%
\pgfpathlineto{\pgfqpoint{1.311274in}{0.963957in}}%
\pgfpathlineto{\pgfqpoint{1.311374in}{0.969865in}}%
\pgfpathlineto{\pgfqpoint{1.312177in}{0.999752in}}%
\pgfpathlineto{\pgfqpoint{1.312779in}{0.997655in}}%
\pgfpathlineto{\pgfqpoint{1.313080in}{0.998581in}}%
\pgfpathlineto{\pgfqpoint{1.314285in}{0.964256in}}%
\pgfpathlineto{\pgfqpoint{1.314385in}{0.962325in}}%
\pgfpathlineto{\pgfqpoint{1.314686in}{0.974801in}}%
\pgfpathlineto{\pgfqpoint{1.314787in}{0.974367in}}%
\pgfpathlineto{\pgfqpoint{1.315188in}{0.976955in}}%
\pgfpathlineto{\pgfqpoint{1.315389in}{0.966911in}}%
\pgfpathlineto{\pgfqpoint{1.315590in}{0.970587in}}%
\pgfpathlineto{\pgfqpoint{1.316292in}{0.959048in}}%
\pgfpathlineto{\pgfqpoint{1.315991in}{0.970601in}}%
\pgfpathlineto{\pgfqpoint{1.316794in}{0.962137in}}%
\pgfpathlineto{\pgfqpoint{1.316995in}{0.977205in}}%
\pgfpathlineto{\pgfqpoint{1.317497in}{0.942701in}}%
\pgfpathlineto{\pgfqpoint{1.317597in}{0.943415in}}%
\pgfpathlineto{\pgfqpoint{1.319705in}{0.889459in}}%
\pgfpathlineto{\pgfqpoint{1.321110in}{0.869164in}}%
\pgfpathlineto{\pgfqpoint{1.321210in}{0.869362in}}%
\pgfpathlineto{\pgfqpoint{1.321310in}{0.867700in}}%
\pgfpathlineto{\pgfqpoint{1.321411in}{0.868383in}}%
\pgfpathlineto{\pgfqpoint{1.322515in}{0.836859in}}%
\pgfpathlineto{\pgfqpoint{1.322615in}{0.840572in}}%
\pgfpathlineto{\pgfqpoint{1.323920in}{0.866678in}}%
\pgfpathlineto{\pgfqpoint{1.324121in}{0.864437in}}%
\pgfpathlineto{\pgfqpoint{1.325325in}{0.887079in}}%
\pgfpathlineto{\pgfqpoint{1.324522in}{0.859609in}}%
\pgfpathlineto{\pgfqpoint{1.325425in}{0.883057in}}%
\pgfpathlineto{\pgfqpoint{1.326429in}{0.871528in}}%
\pgfpathlineto{\pgfqpoint{1.325927in}{0.888192in}}%
\pgfpathlineto{\pgfqpoint{1.326529in}{0.875790in}}%
\pgfpathlineto{\pgfqpoint{1.327734in}{0.898013in}}%
\pgfpathlineto{\pgfqpoint{1.327934in}{0.896667in}}%
\pgfpathlineto{\pgfqpoint{1.329841in}{0.851387in}}%
\pgfpathlineto{\pgfqpoint{1.330042in}{0.860878in}}%
\pgfpathlineto{\pgfqpoint{1.331748in}{0.906986in}}%
\pgfpathlineto{\pgfqpoint{1.332551in}{0.880521in}}%
\pgfpathlineto{\pgfqpoint{1.333354in}{0.885086in}}%
\pgfpathlineto{\pgfqpoint{1.333454in}{0.892847in}}%
\pgfpathlineto{\pgfqpoint{1.334057in}{0.871662in}}%
\pgfpathlineto{\pgfqpoint{1.334257in}{0.874046in}}%
\pgfpathlineto{\pgfqpoint{1.334358in}{0.874355in}}%
\pgfpathlineto{\pgfqpoint{1.334558in}{0.881401in}}%
\pgfpathlineto{\pgfqpoint{1.335060in}{0.864660in}}%
\pgfpathlineto{\pgfqpoint{1.335361in}{0.874416in}}%
\pgfpathlineto{\pgfqpoint{1.335462in}{0.867941in}}%
\pgfpathlineto{\pgfqpoint{1.335964in}{0.901827in}}%
\pgfpathlineto{\pgfqpoint{1.336265in}{0.880482in}}%
\pgfpathlineto{\pgfqpoint{1.336465in}{0.883428in}}%
\pgfpathlineto{\pgfqpoint{1.336666in}{0.879765in}}%
\pgfpathlineto{\pgfqpoint{1.337971in}{0.854832in}}%
\pgfpathlineto{\pgfqpoint{1.338673in}{0.842992in}}%
\pgfpathlineto{\pgfqpoint{1.338372in}{0.856199in}}%
\pgfpathlineto{\pgfqpoint{1.338974in}{0.850463in}}%
\pgfpathlineto{\pgfqpoint{1.340279in}{0.868244in}}%
\pgfpathlineto{\pgfqpoint{1.340982in}{0.838748in}}%
\pgfpathlineto{\pgfqpoint{1.340480in}{0.869480in}}%
\pgfpathlineto{\pgfqpoint{1.341785in}{0.853902in}}%
\pgfpathlineto{\pgfqpoint{1.341885in}{0.854063in}}%
\pgfpathlineto{\pgfqpoint{1.343491in}{0.824054in}}%
\pgfpathlineto{\pgfqpoint{1.343993in}{0.834783in}}%
\pgfpathlineto{\pgfqpoint{1.343792in}{0.823566in}}%
\pgfpathlineto{\pgfqpoint{1.344394in}{0.829981in}}%
\pgfpathlineto{\pgfqpoint{1.344896in}{0.801228in}}%
\pgfpathlineto{\pgfqpoint{1.345498in}{0.828368in}}%
\pgfpathlineto{\pgfqpoint{1.346502in}{0.851296in}}%
\pgfpathlineto{\pgfqpoint{1.345699in}{0.819505in}}%
\pgfpathlineto{\pgfqpoint{1.347004in}{0.836374in}}%
\pgfpathlineto{\pgfqpoint{1.347104in}{0.830857in}}%
\pgfpathlineto{\pgfqpoint{1.347305in}{0.849156in}}%
\pgfpathlineto{\pgfqpoint{1.347807in}{0.847612in}}%
\pgfpathlineto{\pgfqpoint{1.348108in}{0.869150in}}%
\pgfpathlineto{\pgfqpoint{1.348609in}{0.836454in}}%
\pgfpathlineto{\pgfqpoint{1.349814in}{0.798736in}}%
\pgfpathlineto{\pgfqpoint{1.350115in}{0.814062in}}%
\pgfpathlineto{\pgfqpoint{1.350516in}{0.819763in}}%
\pgfpathlineto{\pgfqpoint{1.351119in}{0.841016in}}%
\pgfpathlineto{\pgfqpoint{1.350717in}{0.817351in}}%
\pgfpathlineto{\pgfqpoint{1.351620in}{0.819824in}}%
\pgfpathlineto{\pgfqpoint{1.351721in}{0.814032in}}%
\pgfpathlineto{\pgfqpoint{1.352423in}{0.834567in}}%
\pgfpathlineto{\pgfqpoint{1.352724in}{0.837084in}}%
\pgfpathlineto{\pgfqpoint{1.352624in}{0.833470in}}%
\pgfpathlineto{\pgfqpoint{1.352825in}{0.834860in}}%
\pgfpathlineto{\pgfqpoint{1.353126in}{0.828235in}}%
\pgfpathlineto{\pgfqpoint{1.354230in}{0.858264in}}%
\pgfpathlineto{\pgfqpoint{1.356237in}{0.781208in}}%
\pgfpathlineto{\pgfqpoint{1.356538in}{0.796827in}}%
\pgfpathlineto{\pgfqpoint{1.356739in}{0.803725in}}%
\pgfpathlineto{\pgfqpoint{1.357241in}{0.789395in}}%
\pgfpathlineto{\pgfqpoint{1.357341in}{0.791730in}}%
\pgfpathlineto{\pgfqpoint{1.358445in}{0.762878in}}%
\pgfpathlineto{\pgfqpoint{1.358646in}{0.768265in}}%
\pgfpathlineto{\pgfqpoint{1.359148in}{0.799823in}}%
\pgfpathlineto{\pgfqpoint{1.359850in}{0.784792in}}%
\pgfpathlineto{\pgfqpoint{1.360151in}{0.780782in}}%
\pgfpathlineto{\pgfqpoint{1.360452in}{0.790915in}}%
\pgfpathlineto{\pgfqpoint{1.360954in}{0.813642in}}%
\pgfpathlineto{\pgfqpoint{1.361556in}{0.798176in}}%
\pgfpathlineto{\pgfqpoint{1.362962in}{0.757527in}}%
\pgfpathlineto{\pgfqpoint{1.363363in}{0.768544in}}%
\pgfpathlineto{\pgfqpoint{1.363463in}{0.787179in}}%
\pgfpathlineto{\pgfqpoint{1.364266in}{0.758651in}}%
\pgfpathlineto{\pgfqpoint{1.364367in}{0.763650in}}%
\pgfpathlineto{\pgfqpoint{1.364567in}{0.766172in}}%
\pgfpathlineto{\pgfqpoint{1.364768in}{0.761771in}}%
\pgfpathlineto{\pgfqpoint{1.364868in}{0.763289in}}%
\pgfpathlineto{\pgfqpoint{1.365370in}{0.734870in}}%
\pgfpathlineto{\pgfqpoint{1.366073in}{0.751086in}}%
\pgfpathlineto{\pgfqpoint{1.366374in}{0.761524in}}%
\pgfpathlineto{\pgfqpoint{1.366876in}{0.738979in}}%
\pgfpathlineto{\pgfqpoint{1.366976in}{0.739091in}}%
\pgfpathlineto{\pgfqpoint{1.367378in}{0.722957in}}%
\pgfpathlineto{\pgfqpoint{1.367980in}{0.743536in}}%
\pgfpathlineto{\pgfqpoint{1.368381in}{0.729703in}}%
\pgfpathlineto{\pgfqpoint{1.368682in}{0.746733in}}%
\pgfpathlineto{\pgfqpoint{1.369184in}{0.736070in}}%
\pgfpathlineto{\pgfqpoint{1.369285in}{0.739780in}}%
\pgfpathlineto{\pgfqpoint{1.369786in}{0.722393in}}%
\pgfpathlineto{\pgfqpoint{1.369887in}{0.713130in}}%
\pgfpathlineto{\pgfqpoint{1.370589in}{0.747517in}}%
\pgfpathlineto{\pgfqpoint{1.371292in}{0.750170in}}%
\pgfpathlineto{\pgfqpoint{1.371493in}{0.744269in}}%
\pgfpathlineto{\pgfqpoint{1.372195in}{0.782991in}}%
\pgfpathlineto{\pgfqpoint{1.372697in}{0.767645in}}%
\pgfpathlineto{\pgfqpoint{1.372998in}{0.750687in}}%
\pgfpathlineto{\pgfqpoint{1.373600in}{0.774486in}}%
\pgfpathlineto{\pgfqpoint{1.374403in}{0.782751in}}%
\pgfpathlineto{\pgfqpoint{1.374503in}{0.773365in}}%
\pgfpathlineto{\pgfqpoint{1.374704in}{0.775905in}}%
\pgfpathlineto{\pgfqpoint{1.376410in}{0.838031in}}%
\pgfpathlineto{\pgfqpoint{1.377113in}{0.830482in}}%
\pgfpathlineto{\pgfqpoint{1.378719in}{0.795295in}}%
\pgfpathlineto{\pgfqpoint{1.378819in}{0.799509in}}%
\pgfpathlineto{\pgfqpoint{1.379321in}{0.783349in}}%
\pgfpathlineto{\pgfqpoint{1.379823in}{0.798580in}}%
\pgfpathlineto{\pgfqpoint{1.380124in}{0.787894in}}%
\pgfpathlineto{\pgfqpoint{1.380525in}{0.813100in}}%
\pgfpathlineto{\pgfqpoint{1.380726in}{0.809963in}}%
\pgfpathlineto{\pgfqpoint{1.381127in}{0.803135in}}%
\pgfpathlineto{\pgfqpoint{1.381228in}{0.811613in}}%
\pgfpathlineto{\pgfqpoint{1.381629in}{0.794802in}}%
\pgfpathlineto{\pgfqpoint{1.382131in}{0.826626in}}%
\pgfpathlineto{\pgfqpoint{1.382533in}{0.837127in}}%
\pgfpathlineto{\pgfqpoint{1.382834in}{0.827938in}}%
\pgfpathlineto{\pgfqpoint{1.383135in}{0.807594in}}%
\pgfpathlineto{\pgfqpoint{1.383938in}{0.811863in}}%
\pgfpathlineto{\pgfqpoint{1.384038in}{0.826177in}}%
\pgfpathlineto{\pgfqpoint{1.384741in}{0.810912in}}%
\pgfpathlineto{\pgfqpoint{1.384941in}{0.812209in}}%
\pgfpathlineto{\pgfqpoint{1.385443in}{0.785399in}}%
\pgfpathlineto{\pgfqpoint{1.386146in}{0.798835in}}%
\pgfpathlineto{\pgfqpoint{1.386949in}{0.785397in}}%
\pgfpathlineto{\pgfqpoint{1.387250in}{0.789841in}}%
\pgfpathlineto{\pgfqpoint{1.387952in}{0.799706in}}%
\pgfpathlineto{\pgfqpoint{1.388354in}{0.797390in}}%
\pgfpathlineto{\pgfqpoint{1.389157in}{0.767035in}}%
\pgfpathlineto{\pgfqpoint{1.389558in}{0.777893in}}%
\pgfpathlineto{\pgfqpoint{1.390963in}{0.751634in}}%
\pgfpathlineto{\pgfqpoint{1.392067in}{0.780856in}}%
\pgfpathlineto{\pgfqpoint{1.392268in}{0.773572in}}%
\pgfpathlineto{\pgfqpoint{1.393372in}{0.760414in}}%
\pgfpathlineto{\pgfqpoint{1.393673in}{0.760993in}}%
\pgfpathlineto{\pgfqpoint{1.394275in}{0.782947in}}%
\pgfpathlineto{\pgfqpoint{1.394877in}{0.766849in}}%
\pgfpathlineto{\pgfqpoint{1.395178in}{0.777027in}}%
\pgfpathlineto{\pgfqpoint{1.395379in}{0.776417in}}%
\pgfpathlineto{\pgfqpoint{1.395480in}{0.788917in}}%
\pgfpathlineto{\pgfqpoint{1.396383in}{0.771028in}}%
\pgfpathlineto{\pgfqpoint{1.396483in}{0.776872in}}%
\pgfpathlineto{\pgfqpoint{1.396684in}{0.763784in}}%
\pgfpathlineto{\pgfqpoint{1.396784in}{0.780246in}}%
\pgfpathlineto{\pgfqpoint{1.397487in}{0.772551in}}%
\pgfpathlineto{\pgfqpoint{1.398189in}{0.790178in}}%
\pgfpathlineto{\pgfqpoint{1.398591in}{0.781279in}}%
\pgfpathlineto{\pgfqpoint{1.399595in}{0.753193in}}%
\pgfpathlineto{\pgfqpoint{1.399896in}{0.767675in}}%
\pgfpathlineto{\pgfqpoint{1.401301in}{0.811643in}}%
\pgfpathlineto{\pgfqpoint{1.401401in}{0.807237in}}%
\pgfpathlineto{\pgfqpoint{1.401501in}{0.798167in}}%
\pgfpathlineto{\pgfqpoint{1.401903in}{0.810585in}}%
\pgfpathlineto{\pgfqpoint{1.402505in}{0.800482in}}%
\pgfpathlineto{\pgfqpoint{1.403308in}{0.834672in}}%
\pgfpathlineto{\pgfqpoint{1.403709in}{0.812297in}}%
\pgfpathlineto{\pgfqpoint{1.405014in}{0.777264in}}%
\pgfpathlineto{\pgfqpoint{1.405416in}{0.779564in}}%
\pgfpathlineto{\pgfqpoint{1.405917in}{0.758196in}}%
\pgfpathlineto{\pgfqpoint{1.406921in}{0.764438in}}%
\pgfpathlineto{\pgfqpoint{1.408728in}{0.809524in}}%
\pgfpathlineto{\pgfqpoint{1.408928in}{0.802338in}}%
\pgfpathlineto{\pgfqpoint{1.409330in}{0.781068in}}%
\pgfpathlineto{\pgfqpoint{1.410032in}{0.799452in}}%
\pgfpathlineto{\pgfqpoint{1.411136in}{0.824614in}}%
\pgfpathlineto{\pgfqpoint{1.410334in}{0.793555in}}%
\pgfpathlineto{\pgfqpoint{1.411739in}{0.812010in}}%
\pgfpathlineto{\pgfqpoint{1.411839in}{0.801928in}}%
\pgfpathlineto{\pgfqpoint{1.412642in}{0.832622in}}%
\pgfpathlineto{\pgfqpoint{1.413144in}{0.825443in}}%
\pgfpathlineto{\pgfqpoint{1.413445in}{0.833821in}}%
\pgfpathlineto{\pgfqpoint{1.413646in}{0.841006in}}%
\pgfpathlineto{\pgfqpoint{1.414047in}{0.822818in}}%
\pgfpathlineto{\pgfqpoint{1.414549in}{0.836799in}}%
\pgfpathlineto{\pgfqpoint{1.415954in}{0.860106in}}%
\pgfpathlineto{\pgfqpoint{1.414950in}{0.834200in}}%
\pgfpathlineto{\pgfqpoint{1.416155in}{0.854925in}}%
\pgfpathlineto{\pgfqpoint{1.418162in}{0.809485in}}%
\pgfpathlineto{\pgfqpoint{1.418262in}{0.809179in}}%
\pgfpathlineto{\pgfqpoint{1.418363in}{0.801185in}}%
\pgfpathlineto{\pgfqpoint{1.418864in}{0.833224in}}%
\pgfpathlineto{\pgfqpoint{1.419065in}{0.831756in}}%
\pgfpathlineto{\pgfqpoint{1.419366in}{0.839832in}}%
\pgfpathlineto{\pgfqpoint{1.419968in}{0.832784in}}%
\pgfpathlineto{\pgfqpoint{1.420872in}{0.800386in}}%
\pgfpathlineto{\pgfqpoint{1.421474in}{0.804402in}}%
\pgfpathlineto{\pgfqpoint{1.422076in}{0.816627in}}%
\pgfpathlineto{\pgfqpoint{1.421775in}{0.801264in}}%
\pgfpathlineto{\pgfqpoint{1.422678in}{0.814436in}}%
\pgfpathlineto{\pgfqpoint{1.423481in}{0.819257in}}%
\pgfpathlineto{\pgfqpoint{1.424184in}{0.799641in}}%
\pgfpathlineto{\pgfqpoint{1.424284in}{0.810859in}}%
\pgfpathlineto{\pgfqpoint{1.425087in}{0.794163in}}%
\pgfpathlineto{\pgfqpoint{1.425187in}{0.795248in}}%
\pgfpathlineto{\pgfqpoint{1.425489in}{0.784292in}}%
\pgfpathlineto{\pgfqpoint{1.425790in}{0.807351in}}%
\pgfpathlineto{\pgfqpoint{1.426091in}{0.801154in}}%
\pgfpathlineto{\pgfqpoint{1.426191in}{0.810569in}}%
\pgfpathlineto{\pgfqpoint{1.427094in}{0.795821in}}%
\pgfpathlineto{\pgfqpoint{1.427195in}{0.801186in}}%
\pgfpathlineto{\pgfqpoint{1.428499in}{0.760970in}}%
\pgfpathlineto{\pgfqpoint{1.429102in}{0.783373in}}%
\pgfpathlineto{\pgfqpoint{1.429202in}{0.783435in}}%
\pgfpathlineto{\pgfqpoint{1.429302in}{0.778257in}}%
\pgfpathlineto{\pgfqpoint{1.429704in}{0.802964in}}%
\pgfpathlineto{\pgfqpoint{1.430005in}{0.793329in}}%
\pgfpathlineto{\pgfqpoint{1.430206in}{0.801662in}}%
\pgfpathlineto{\pgfqpoint{1.430607in}{0.787776in}}%
\pgfpathlineto{\pgfqpoint{1.431009in}{0.793477in}}%
\pgfpathlineto{\pgfqpoint{1.431811in}{0.770203in}}%
\pgfpathlineto{\pgfqpoint{1.431912in}{0.778037in}}%
\pgfpathlineto{\pgfqpoint{1.432514in}{0.744726in}}%
\pgfpathlineto{\pgfqpoint{1.432715in}{0.754529in}}%
\pgfpathlineto{\pgfqpoint{1.432815in}{0.754380in}}%
\pgfpathlineto{\pgfqpoint{1.433016in}{0.765082in}}%
\pgfpathlineto{\pgfqpoint{1.434019in}{0.764411in}}%
\pgfpathlineto{\pgfqpoint{1.435926in}{0.711205in}}%
\pgfpathlineto{\pgfqpoint{1.436428in}{0.715271in}}%
\pgfpathlineto{\pgfqpoint{1.436529in}{0.717351in}}%
\pgfpathlineto{\pgfqpoint{1.436830in}{0.706209in}}%
\pgfpathlineto{\pgfqpoint{1.436930in}{0.702345in}}%
\pgfpathlineto{\pgfqpoint{1.437332in}{0.729592in}}%
\pgfpathlineto{\pgfqpoint{1.437432in}{0.721110in}}%
\pgfpathlineto{\pgfqpoint{1.438034in}{0.742951in}}%
\pgfpathlineto{\pgfqpoint{1.438536in}{0.730965in}}%
\pgfpathlineto{\pgfqpoint{1.439540in}{0.710341in}}%
\pgfpathlineto{\pgfqpoint{1.439740in}{0.714139in}}%
\pgfpathlineto{\pgfqpoint{1.439841in}{0.713804in}}%
\pgfpathlineto{\pgfqpoint{1.439941in}{0.716353in}}%
\pgfpathlineto{\pgfqpoint{1.441145in}{0.732690in}}%
\pgfpathlineto{\pgfqpoint{1.442852in}{0.678611in}}%
\pgfpathlineto{\pgfqpoint{1.442952in}{0.686755in}}%
\pgfpathlineto{\pgfqpoint{1.444658in}{0.721286in}}%
\pgfpathlineto{\pgfqpoint{1.445862in}{0.704594in}}%
\pgfpathlineto{\pgfqpoint{1.445963in}{0.706353in}}%
\pgfpathlineto{\pgfqpoint{1.447067in}{0.727353in}}%
\pgfpathlineto{\pgfqpoint{1.447167in}{0.719795in}}%
\pgfpathlineto{\pgfqpoint{1.447368in}{0.704474in}}%
\pgfpathlineto{\pgfqpoint{1.448171in}{0.706422in}}%
\pgfpathlineto{\pgfqpoint{1.449576in}{0.748678in}}%
\pgfpathlineto{\pgfqpoint{1.449777in}{0.750467in}}%
\pgfpathlineto{\pgfqpoint{1.450379in}{0.759055in}}%
\pgfpathlineto{\pgfqpoint{1.450580in}{0.744769in}}%
\pgfpathlineto{\pgfqpoint{1.450780in}{0.755475in}}%
\pgfpathlineto{\pgfqpoint{1.450881in}{0.750058in}}%
\pgfpathlineto{\pgfqpoint{1.451282in}{0.774578in}}%
\pgfpathlineto{\pgfqpoint{1.451483in}{0.769138in}}%
\pgfpathlineto{\pgfqpoint{1.452286in}{0.792718in}}%
\pgfpathlineto{\pgfqpoint{1.452788in}{0.784493in}}%
\pgfpathlineto{\pgfqpoint{1.452888in}{0.785063in}}%
\pgfpathlineto{\pgfqpoint{1.453691in}{0.803609in}}%
\pgfpathlineto{\pgfqpoint{1.453992in}{0.792158in}}%
\pgfpathlineto{\pgfqpoint{1.454092in}{0.792548in}}%
\pgfpathlineto{\pgfqpoint{1.454193in}{0.798850in}}%
\pgfpathlineto{\pgfqpoint{1.454795in}{0.773171in}}%
\pgfpathlineto{\pgfqpoint{1.455196in}{0.766201in}}%
\pgfpathlineto{\pgfqpoint{1.455497in}{0.786215in}}%
\pgfpathlineto{\pgfqpoint{1.455698in}{0.779109in}}%
\pgfpathlineto{\pgfqpoint{1.456501in}{0.768146in}}%
\pgfpathlineto{\pgfqpoint{1.457003in}{0.775942in}}%
\pgfpathlineto{\pgfqpoint{1.457204in}{0.787179in}}%
\pgfpathlineto{\pgfqpoint{1.458308in}{0.786022in}}%
\pgfpathlineto{\pgfqpoint{1.458809in}{0.779547in}}%
\pgfpathlineto{\pgfqpoint{1.459010in}{0.780935in}}%
\pgfpathlineto{\pgfqpoint{1.460014in}{0.801293in}}%
\pgfpathlineto{\pgfqpoint{1.460215in}{0.791494in}}%
\pgfpathlineto{\pgfqpoint{1.461218in}{0.761232in}}%
\pgfpathlineto{\pgfqpoint{1.461620in}{0.767780in}}%
\pgfpathlineto{\pgfqpoint{1.462122in}{0.811168in}}%
\pgfpathlineto{\pgfqpoint{1.463226in}{0.808443in}}%
\pgfpathlineto{\pgfqpoint{1.463326in}{0.814625in}}%
\pgfpathlineto{\pgfqpoint{1.464028in}{0.802379in}}%
\pgfpathlineto{\pgfqpoint{1.465032in}{0.768064in}}%
\pgfpathlineto{\pgfqpoint{1.465333in}{0.774015in}}%
\pgfpathlineto{\pgfqpoint{1.465434in}{0.772545in}}%
\pgfpathlineto{\pgfqpoint{1.465735in}{0.782531in}}%
\pgfpathlineto{\pgfqpoint{1.466538in}{0.777981in}}%
\pgfpathlineto{\pgfqpoint{1.467240in}{0.795219in}}%
\pgfpathlineto{\pgfqpoint{1.467642in}{0.785001in}}%
\pgfpathlineto{\pgfqpoint{1.468043in}{0.801580in}}%
\pgfpathlineto{\pgfqpoint{1.468143in}{0.800578in}}%
\pgfpathlineto{\pgfqpoint{1.468545in}{0.814018in}}%
\pgfpathlineto{\pgfqpoint{1.468344in}{0.793349in}}%
\pgfpathlineto{\pgfqpoint{1.469147in}{0.800520in}}%
\pgfpathlineto{\pgfqpoint{1.469247in}{0.795549in}}%
\pgfpathlineto{\pgfqpoint{1.469649in}{0.810922in}}%
\pgfpathlineto{\pgfqpoint{1.470050in}{0.799845in}}%
\pgfpathlineto{\pgfqpoint{1.470853in}{0.833041in}}%
\pgfpathlineto{\pgfqpoint{1.471556in}{0.828717in}}%
\pgfpathlineto{\pgfqpoint{1.471857in}{0.846538in}}%
\pgfpathlineto{\pgfqpoint{1.472660in}{0.832494in}}%
\pgfpathlineto{\pgfqpoint{1.472961in}{0.824542in}}%
\pgfpathlineto{\pgfqpoint{1.473162in}{0.835776in}}%
\pgfpathlineto{\pgfqpoint{1.473663in}{0.835423in}}%
\pgfpathlineto{\pgfqpoint{1.475671in}{0.902108in}}%
\pgfpathlineto{\pgfqpoint{1.475771in}{0.900586in}}%
\pgfpathlineto{\pgfqpoint{1.476674in}{0.915661in}}%
\pgfpathlineto{\pgfqpoint{1.476474in}{0.899298in}}%
\pgfpathlineto{\pgfqpoint{1.476775in}{0.903066in}}%
\pgfpathlineto{\pgfqpoint{1.477778in}{0.852150in}}%
\pgfpathlineto{\pgfqpoint{1.478782in}{0.875218in}}%
\pgfpathlineto{\pgfqpoint{1.479284in}{0.880832in}}%
\pgfpathlineto{\pgfqpoint{1.479986in}{0.853233in}}%
\pgfpathlineto{\pgfqpoint{1.480990in}{0.838313in}}%
\pgfpathlineto{\pgfqpoint{1.481391in}{0.845542in}}%
\pgfpathlineto{\pgfqpoint{1.481893in}{0.857263in}}%
\pgfpathlineto{\pgfqpoint{1.482194in}{0.838857in}}%
\pgfpathlineto{\pgfqpoint{1.482295in}{0.839671in}}%
\pgfpathlineto{\pgfqpoint{1.482395in}{0.839234in}}%
\pgfpathlineto{\pgfqpoint{1.483098in}{0.857815in}}%
\pgfpathlineto{\pgfqpoint{1.483499in}{0.839464in}}%
\pgfpathlineto{\pgfqpoint{1.484703in}{0.864380in}}%
\pgfpathlineto{\pgfqpoint{1.483800in}{0.833491in}}%
\pgfpathlineto{\pgfqpoint{1.484904in}{0.848143in}}%
\pgfpathlineto{\pgfqpoint{1.486410in}{0.789130in}}%
\pgfpathlineto{\pgfqpoint{1.487915in}{0.796403in}}%
\pgfpathlineto{\pgfqpoint{1.488116in}{0.807565in}}%
\pgfpathlineto{\pgfqpoint{1.488417in}{0.794040in}}%
\pgfpathlineto{\pgfqpoint{1.488517in}{0.797230in}}%
\pgfpathlineto{\pgfqpoint{1.489120in}{0.773222in}}%
\pgfpathlineto{\pgfqpoint{1.489621in}{0.792313in}}%
\pgfpathlineto{\pgfqpoint{1.490826in}{0.774141in}}%
\pgfpathlineto{\pgfqpoint{1.491026in}{0.782951in}}%
\pgfpathlineto{\pgfqpoint{1.491127in}{0.782835in}}%
\pgfpathlineto{\pgfqpoint{1.491227in}{0.773499in}}%
\pgfpathlineto{\pgfqpoint{1.491528in}{0.802337in}}%
\pgfpathlineto{\pgfqpoint{1.491930in}{0.801564in}}%
\pgfpathlineto{\pgfqpoint{1.492632in}{0.822761in}}%
\pgfpathlineto{\pgfqpoint{1.492231in}{0.796324in}}%
\pgfpathlineto{\pgfqpoint{1.493234in}{0.806893in}}%
\pgfpathlineto{\pgfqpoint{1.493335in}{0.799890in}}%
\pgfpathlineto{\pgfqpoint{1.493837in}{0.817963in}}%
\pgfpathlineto{\pgfqpoint{1.494138in}{0.815905in}}%
\pgfpathlineto{\pgfqpoint{1.494338in}{0.821745in}}%
\pgfpathlineto{\pgfqpoint{1.494740in}{0.848184in}}%
\pgfpathlineto{\pgfqpoint{1.495442in}{0.826141in}}%
\pgfpathlineto{\pgfqpoint{1.496848in}{0.790080in}}%
\pgfpathlineto{\pgfqpoint{1.497851in}{0.814636in}}%
\pgfpathlineto{\pgfqpoint{1.497952in}{0.804877in}}%
\pgfpathlineto{\pgfqpoint{1.498152in}{0.791023in}}%
\pgfpathlineto{\pgfqpoint{1.498654in}{0.808909in}}%
\pgfpathlineto{\pgfqpoint{1.499056in}{0.803965in}}%
\pgfpathlineto{\pgfqpoint{1.499758in}{0.819584in}}%
\pgfpathlineto{\pgfqpoint{1.499557in}{0.798178in}}%
\pgfpathlineto{\pgfqpoint{1.500059in}{0.801917in}}%
\pgfpathlineto{\pgfqpoint{1.500160in}{0.796504in}}%
\pgfpathlineto{\pgfqpoint{1.500561in}{0.811780in}}%
\pgfpathlineto{\pgfqpoint{1.500862in}{0.808755in}}%
\pgfpathlineto{\pgfqpoint{1.500962in}{0.817828in}}%
\pgfpathlineto{\pgfqpoint{1.501665in}{0.798456in}}%
\pgfpathlineto{\pgfqpoint{1.501765in}{0.802642in}}%
\pgfpathlineto{\pgfqpoint{1.502267in}{0.789634in}}%
\pgfpathlineto{\pgfqpoint{1.502769in}{0.810178in}}%
\pgfpathlineto{\pgfqpoint{1.503873in}{0.785685in}}%
\pgfpathlineto{\pgfqpoint{1.503070in}{0.820024in}}%
\pgfpathlineto{\pgfqpoint{1.504074in}{0.801608in}}%
\pgfpathlineto{\pgfqpoint{1.505579in}{0.853384in}}%
\pgfpathlineto{\pgfqpoint{1.505880in}{0.843832in}}%
\pgfpathlineto{\pgfqpoint{1.506382in}{0.863488in}}%
\pgfpathlineto{\pgfqpoint{1.506583in}{0.866936in}}%
\pgfpathlineto{\pgfqpoint{1.506884in}{0.849360in}}%
\pgfpathlineto{\pgfqpoint{1.506984in}{0.842369in}}%
\pgfpathlineto{\pgfqpoint{1.507386in}{0.869710in}}%
\pgfpathlineto{\pgfqpoint{1.507687in}{0.867878in}}%
\pgfpathlineto{\pgfqpoint{1.507888in}{0.866082in}}%
\pgfpathlineto{\pgfqpoint{1.507988in}{0.872119in}}%
\pgfpathlineto{\pgfqpoint{1.508289in}{0.882120in}}%
\pgfpathlineto{\pgfqpoint{1.508389in}{0.884170in}}%
\pgfpathlineto{\pgfqpoint{1.508791in}{0.871784in}}%
\pgfpathlineto{\pgfqpoint{1.508891in}{0.872246in}}%
\pgfpathlineto{\pgfqpoint{1.508992in}{0.872161in}}%
\pgfpathlineto{\pgfqpoint{1.509092in}{0.885803in}}%
\pgfpathlineto{\pgfqpoint{1.509493in}{0.868039in}}%
\pgfpathlineto{\pgfqpoint{1.509995in}{0.868673in}}%
\pgfpathlineto{\pgfqpoint{1.511802in}{0.897399in}}%
\pgfpathlineto{\pgfqpoint{1.512103in}{0.891843in}}%
\pgfpathlineto{\pgfqpoint{1.512605in}{0.864789in}}%
\pgfpathlineto{\pgfqpoint{1.513207in}{0.879489in}}%
\pgfpathlineto{\pgfqpoint{1.513307in}{0.880141in}}%
\pgfpathlineto{\pgfqpoint{1.513408in}{0.872441in}}%
\pgfpathlineto{\pgfqpoint{1.514110in}{0.883058in}}%
\pgfpathlineto{\pgfqpoint{1.514311in}{0.876810in}}%
\pgfpathlineto{\pgfqpoint{1.515315in}{0.907625in}}%
\pgfpathlineto{\pgfqpoint{1.515716in}{0.893654in}}%
\pgfpathlineto{\pgfqpoint{1.515917in}{0.892664in}}%
\pgfpathlineto{\pgfqpoint{1.516017in}{0.897978in}}%
\pgfpathlineto{\pgfqpoint{1.516820in}{0.886826in}}%
\pgfpathlineto{\pgfqpoint{1.517021in}{0.893327in}}%
\pgfpathlineto{\pgfqpoint{1.518125in}{0.868214in}}%
\pgfpathlineto{\pgfqpoint{1.518727in}{0.873134in}}%
\pgfpathlineto{\pgfqpoint{1.519630in}{0.882523in}}%
\pgfpathlineto{\pgfqpoint{1.519430in}{0.867343in}}%
\pgfpathlineto{\pgfqpoint{1.519831in}{0.880125in}}%
\pgfpathlineto{\pgfqpoint{1.521035in}{0.857219in}}%
\pgfpathlineto{\pgfqpoint{1.521136in}{0.859640in}}%
\pgfpathlineto{\pgfqpoint{1.521236in}{0.853806in}}%
\pgfpathlineto{\pgfqpoint{1.521638in}{0.867602in}}%
\pgfpathlineto{\pgfqpoint{1.522240in}{0.859334in}}%
\pgfpathlineto{\pgfqpoint{1.523544in}{0.873908in}}%
\pgfpathlineto{\pgfqpoint{1.522641in}{0.851423in}}%
\pgfpathlineto{\pgfqpoint{1.523745in}{0.871725in}}%
\pgfpathlineto{\pgfqpoint{1.524347in}{0.860721in}}%
\pgfpathlineto{\pgfqpoint{1.523946in}{0.874593in}}%
\pgfpathlineto{\pgfqpoint{1.524849in}{0.870068in}}%
\pgfpathlineto{\pgfqpoint{1.525150in}{0.890920in}}%
\pgfpathlineto{\pgfqpoint{1.526054in}{0.884846in}}%
\pgfpathlineto{\pgfqpoint{1.526856in}{0.892094in}}%
\pgfpathlineto{\pgfqpoint{1.526555in}{0.882679in}}%
\pgfpathlineto{\pgfqpoint{1.527158in}{0.889313in}}%
\pgfpathlineto{\pgfqpoint{1.528061in}{0.859277in}}%
\pgfpathlineto{\pgfqpoint{1.528362in}{0.875803in}}%
\pgfpathlineto{\pgfqpoint{1.529667in}{0.901654in}}%
\pgfpathlineto{\pgfqpoint{1.529867in}{0.897711in}}%
\pgfpathlineto{\pgfqpoint{1.530068in}{0.886773in}}%
\pgfpathlineto{\pgfqpoint{1.530670in}{0.905015in}}%
\pgfpathlineto{\pgfqpoint{1.530771in}{0.904519in}}%
\pgfpathlineto{\pgfqpoint{1.531273in}{0.918289in}}%
\pgfpathlineto{\pgfqpoint{1.531875in}{0.908496in}}%
\pgfpathlineto{\pgfqpoint{1.533179in}{0.891167in}}%
\pgfpathlineto{\pgfqpoint{1.532577in}{0.916909in}}%
\pgfpathlineto{\pgfqpoint{1.533380in}{0.893088in}}%
\pgfpathlineto{\pgfqpoint{1.533581in}{0.905871in}}%
\pgfpathlineto{\pgfqpoint{1.534083in}{0.888805in}}%
\pgfpathlineto{\pgfqpoint{1.534685in}{0.901422in}}%
\pgfpathlineto{\pgfqpoint{1.535086in}{0.891551in}}%
\pgfpathlineto{\pgfqpoint{1.535488in}{0.916665in}}%
\pgfpathlineto{\pgfqpoint{1.535588in}{0.909160in}}%
\pgfpathlineto{\pgfqpoint{1.535689in}{0.918413in}}%
\pgfpathlineto{\pgfqpoint{1.536491in}{0.896799in}}%
\pgfpathlineto{\pgfqpoint{1.536592in}{0.888893in}}%
\pgfpathlineto{\pgfqpoint{1.537294in}{0.909034in}}%
\pgfpathlineto{\pgfqpoint{1.538298in}{0.924150in}}%
\pgfpathlineto{\pgfqpoint{1.538398in}{0.918276in}}%
\pgfpathlineto{\pgfqpoint{1.538499in}{0.912110in}}%
\pgfpathlineto{\pgfqpoint{1.539302in}{0.923764in}}%
\pgfpathlineto{\pgfqpoint{1.539402in}{0.925096in}}%
\pgfpathlineto{\pgfqpoint{1.539502in}{0.914692in}}%
\pgfpathlineto{\pgfqpoint{1.540004in}{0.891509in}}%
\pgfpathlineto{\pgfqpoint{1.540707in}{0.901460in}}%
\pgfpathlineto{\pgfqpoint{1.540908in}{0.902488in}}%
\pgfpathlineto{\pgfqpoint{1.542313in}{0.940682in}}%
\pgfpathlineto{\pgfqpoint{1.543216in}{0.926546in}}%
\pgfpathlineto{\pgfqpoint{1.542814in}{0.949323in}}%
\pgfpathlineto{\pgfqpoint{1.543617in}{0.931345in}}%
\pgfpathlineto{\pgfqpoint{1.544621in}{0.973082in}}%
\pgfpathlineto{\pgfqpoint{1.545123in}{0.964211in}}%
\pgfpathlineto{\pgfqpoint{1.545223in}{0.956591in}}%
\pgfpathlineto{\pgfqpoint{1.545825in}{0.983388in}}%
\pgfpathlineto{\pgfqpoint{1.546026in}{0.978181in}}%
\pgfpathlineto{\pgfqpoint{1.546327in}{0.967890in}}%
\pgfpathlineto{\pgfqpoint{1.546829in}{0.983670in}}%
\pgfpathlineto{\pgfqpoint{1.547030in}{0.977808in}}%
\pgfpathlineto{\pgfqpoint{1.547331in}{0.981978in}}%
\pgfpathlineto{\pgfqpoint{1.547532in}{0.979255in}}%
\pgfpathlineto{\pgfqpoint{1.547833in}{1.002899in}}%
\pgfpathlineto{\pgfqpoint{1.548535in}{0.978561in}}%
\pgfpathlineto{\pgfqpoint{1.548636in}{0.979655in}}%
\pgfpathlineto{\pgfqpoint{1.549037in}{0.974401in}}%
\pgfpathlineto{\pgfqpoint{1.548836in}{0.986196in}}%
\pgfpathlineto{\pgfqpoint{1.549238in}{0.983120in}}%
\pgfpathlineto{\pgfqpoint{1.549438in}{0.998624in}}%
\pgfpathlineto{\pgfqpoint{1.549940in}{0.981038in}}%
\pgfpathlineto{\pgfqpoint{1.550342in}{0.989831in}}%
\pgfpathlineto{\pgfqpoint{1.550442in}{0.986366in}}%
\pgfpathlineto{\pgfqpoint{1.550944in}{1.000576in}}%
\pgfpathlineto{\pgfqpoint{1.551245in}{0.992595in}}%
\pgfpathlineto{\pgfqpoint{1.552148in}{1.009958in}}%
\pgfpathlineto{\pgfqpoint{1.552249in}{1.005797in}}%
\pgfpathlineto{\pgfqpoint{1.552851in}{1.019893in}}%
\pgfpathlineto{\pgfqpoint{1.553052in}{1.015108in}}%
\pgfpathlineto{\pgfqpoint{1.553252in}{1.026406in}}%
\pgfpathlineto{\pgfqpoint{1.553453in}{1.013899in}}%
\pgfpathlineto{\pgfqpoint{1.553955in}{1.016669in}}%
\pgfpathlineto{\pgfqpoint{1.554657in}{0.983722in}}%
\pgfpathlineto{\pgfqpoint{1.555260in}{0.993589in}}%
\pgfpathlineto{\pgfqpoint{1.555460in}{0.999056in}}%
\pgfpathlineto{\pgfqpoint{1.555761in}{0.983686in}}%
\pgfpathlineto{\pgfqpoint{1.556464in}{0.998540in}}%
\pgfpathlineto{\pgfqpoint{1.556665in}{0.997470in}}%
\pgfpathlineto{\pgfqpoint{1.557066in}{1.003881in}}%
\pgfpathlineto{\pgfqpoint{1.557468in}{1.013610in}}%
\pgfpathlineto{\pgfqpoint{1.558070in}{0.996590in}}%
\pgfpathlineto{\pgfqpoint{1.558772in}{0.989423in}}%
\pgfpathlineto{\pgfqpoint{1.558973in}{1.000048in}}%
\pgfpathlineto{\pgfqpoint{1.559073in}{0.996317in}}%
\pgfpathlineto{\pgfqpoint{1.560077in}{1.006204in}}%
\pgfpathlineto{\pgfqpoint{1.559876in}{0.992801in}}%
\pgfpathlineto{\pgfqpoint{1.560177in}{1.002221in}}%
\pgfpathlineto{\pgfqpoint{1.561382in}{0.953590in}}%
\pgfpathlineto{\pgfqpoint{1.561683in}{0.964263in}}%
\pgfpathlineto{\pgfqpoint{1.563188in}{0.926890in}}%
\pgfpathlineto{\pgfqpoint{1.563389in}{0.930028in}}%
\pgfpathlineto{\pgfqpoint{1.563690in}{0.943259in}}%
\pgfpathlineto{\pgfqpoint{1.564393in}{0.931414in}}%
\pgfpathlineto{\pgfqpoint{1.564493in}{0.925483in}}%
\pgfpathlineto{\pgfqpoint{1.564794in}{0.940693in}}%
\pgfpathlineto{\pgfqpoint{1.565396in}{0.934605in}}%
\pgfpathlineto{\pgfqpoint{1.566300in}{0.920523in}}%
\pgfpathlineto{\pgfqpoint{1.566601in}{0.923495in}}%
\pgfpathlineto{\pgfqpoint{1.567203in}{0.936390in}}%
\pgfpathlineto{\pgfqpoint{1.566902in}{0.921066in}}%
\pgfpathlineto{\pgfqpoint{1.567604in}{0.927480in}}%
\pgfpathlineto{\pgfqpoint{1.568708in}{0.905866in}}%
\pgfpathlineto{\pgfqpoint{1.568809in}{0.917699in}}%
\pgfpathlineto{\pgfqpoint{1.569411in}{0.937671in}}%
\pgfpathlineto{\pgfqpoint{1.570214in}{0.931176in}}%
\pgfpathlineto{\pgfqpoint{1.570816in}{0.919366in}}%
\pgfpathlineto{\pgfqpoint{1.571418in}{0.924658in}}%
\pgfpathlineto{\pgfqpoint{1.572522in}{0.943971in}}%
\pgfpathlineto{\pgfqpoint{1.571719in}{0.918104in}}%
\pgfpathlineto{\pgfqpoint{1.573024in}{0.942623in}}%
\pgfpathlineto{\pgfqpoint{1.573124in}{0.941001in}}%
\pgfpathlineto{\pgfqpoint{1.573426in}{0.952072in}}%
\pgfpathlineto{\pgfqpoint{1.573526in}{0.950882in}}%
\pgfpathlineto{\pgfqpoint{1.574831in}{0.976317in}}%
\pgfpathlineto{\pgfqpoint{1.575734in}{0.971023in}}%
\pgfpathlineto{\pgfqpoint{1.575332in}{0.989273in}}%
\pgfpathlineto{\pgfqpoint{1.575834in}{0.976198in}}%
\pgfpathlineto{\pgfqpoint{1.576135in}{0.985676in}}%
\pgfpathlineto{\pgfqpoint{1.576236in}{0.978820in}}%
\pgfpathlineto{\pgfqpoint{1.576938in}{0.946120in}}%
\pgfpathlineto{\pgfqpoint{1.577440in}{0.958307in}}%
\pgfpathlineto{\pgfqpoint{1.577540in}{0.958557in}}%
\pgfpathlineto{\pgfqpoint{1.577942in}{0.967138in}}%
\pgfpathlineto{\pgfqpoint{1.578544in}{0.958420in}}%
\pgfpathlineto{\pgfqpoint{1.578644in}{0.960508in}}%
\pgfpathlineto{\pgfqpoint{1.578745in}{0.960682in}}%
\pgfpathlineto{\pgfqpoint{1.579347in}{0.976677in}}%
\pgfpathlineto{\pgfqpoint{1.579648in}{0.957606in}}%
\pgfpathlineto{\pgfqpoint{1.579849in}{0.964035in}}%
\pgfpathlineto{\pgfqpoint{1.580150in}{0.969979in}}%
\pgfpathlineto{\pgfqpoint{1.580853in}{0.947443in}}%
\pgfpathlineto{\pgfqpoint{1.581756in}{0.939047in}}%
\pgfpathlineto{\pgfqpoint{1.581354in}{0.947472in}}%
\pgfpathlineto{\pgfqpoint{1.581856in}{0.942075in}}%
\pgfpathlineto{\pgfqpoint{1.582659in}{0.974332in}}%
\pgfpathlineto{\pgfqpoint{1.583061in}{0.951944in}}%
\pgfpathlineto{\pgfqpoint{1.583161in}{0.951377in}}%
\pgfpathlineto{\pgfqpoint{1.584666in}{0.993109in}}%
\pgfpathlineto{\pgfqpoint{1.585168in}{0.976073in}}%
\pgfpathlineto{\pgfqpoint{1.585770in}{0.989583in}}%
\pgfpathlineto{\pgfqpoint{1.588480in}{1.057776in}}%
\pgfpathlineto{\pgfqpoint{1.589383in}{1.052802in}}%
\pgfpathlineto{\pgfqpoint{1.589986in}{1.032901in}}%
\pgfpathlineto{\pgfqpoint{1.590789in}{1.045751in}}%
\pgfpathlineto{\pgfqpoint{1.591591in}{1.073195in}}%
\pgfpathlineto{\pgfqpoint{1.592394in}{1.061975in}}%
\pgfpathlineto{\pgfqpoint{1.593398in}{1.052027in}}%
\pgfpathlineto{\pgfqpoint{1.592796in}{1.066378in}}%
\pgfpathlineto{\pgfqpoint{1.593599in}{1.052886in}}%
\pgfpathlineto{\pgfqpoint{1.593699in}{1.052537in}}%
\pgfpathlineto{\pgfqpoint{1.594703in}{1.030793in}}%
\pgfpathlineto{\pgfqpoint{1.594101in}{1.053612in}}%
\pgfpathlineto{\pgfqpoint{1.594803in}{1.038911in}}%
\pgfpathlineto{\pgfqpoint{1.594904in}{1.046378in}}%
\pgfpathlineto{\pgfqpoint{1.595606in}{1.027095in}}%
\pgfpathlineto{\pgfqpoint{1.596309in}{1.007364in}}%
\pgfpathlineto{\pgfqpoint{1.596710in}{1.019672in}}%
\pgfpathlineto{\pgfqpoint{1.596911in}{1.027464in}}%
\pgfpathlineto{\pgfqpoint{1.597613in}{1.010034in}}%
\pgfpathlineto{\pgfqpoint{1.597714in}{1.010989in}}%
\pgfpathlineto{\pgfqpoint{1.597914in}{1.005082in}}%
\pgfpathlineto{\pgfqpoint{1.599119in}{0.981148in}}%
\pgfpathlineto{\pgfqpoint{1.599219in}{0.985277in}}%
\pgfpathlineto{\pgfqpoint{1.599420in}{0.984860in}}%
\pgfpathlineto{\pgfqpoint{1.600925in}{0.952171in}}%
\pgfpathlineto{\pgfqpoint{1.601026in}{0.966789in}}%
\pgfpathlineto{\pgfqpoint{1.601929in}{0.949515in}}%
\pgfpathlineto{\pgfqpoint{1.602029in}{0.958952in}}%
\pgfpathlineto{\pgfqpoint{1.602531in}{0.971478in}}%
\pgfpathlineto{\pgfqpoint{1.603133in}{0.953975in}}%
\pgfpathlineto{\pgfqpoint{1.603635in}{0.960873in}}%
\pgfpathlineto{\pgfqpoint{1.603736in}{0.947558in}}%
\pgfpathlineto{\pgfqpoint{1.604137in}{0.956704in}}%
\pgfpathlineto{\pgfqpoint{1.605141in}{0.941177in}}%
\pgfpathlineto{\pgfqpoint{1.604639in}{0.959118in}}%
\pgfpathlineto{\pgfqpoint{1.605341in}{0.947873in}}%
\pgfpathlineto{\pgfqpoint{1.606445in}{0.975345in}}%
\pgfpathlineto{\pgfqpoint{1.605542in}{0.941357in}}%
\pgfpathlineto{\pgfqpoint{1.606847in}{0.964588in}}%
\pgfpathlineto{\pgfqpoint{1.607851in}{0.928611in}}%
\pgfpathlineto{\pgfqpoint{1.608252in}{0.946127in}}%
\pgfpathlineto{\pgfqpoint{1.608453in}{0.941458in}}%
\pgfpathlineto{\pgfqpoint{1.608653in}{0.948752in}}%
\pgfpathlineto{\pgfqpoint{1.608754in}{0.947762in}}%
\pgfpathlineto{\pgfqpoint{1.610059in}{0.967707in}}%
\pgfpathlineto{\pgfqpoint{1.610661in}{0.954361in}}%
\pgfpathlineto{\pgfqpoint{1.610360in}{0.970204in}}%
\pgfpathlineto{\pgfqpoint{1.611163in}{0.963889in}}%
\pgfpathlineto{\pgfqpoint{1.612166in}{0.989842in}}%
\pgfpathlineto{\pgfqpoint{1.612467in}{0.978512in}}%
\pgfpathlineto{\pgfqpoint{1.612568in}{0.980260in}}%
\pgfpathlineto{\pgfqpoint{1.612768in}{0.970570in}}%
\pgfpathlineto{\pgfqpoint{1.613872in}{0.941148in}}%
\pgfpathlineto{\pgfqpoint{1.613973in}{0.950370in}}%
\pgfpathlineto{\pgfqpoint{1.615277in}{0.975909in}}%
\pgfpathlineto{\pgfqpoint{1.615378in}{0.973832in}}%
\pgfpathlineto{\pgfqpoint{1.615579in}{0.964356in}}%
\pgfpathlineto{\pgfqpoint{1.615779in}{0.977182in}}%
\pgfpathlineto{\pgfqpoint{1.616482in}{0.971682in}}%
\pgfpathlineto{\pgfqpoint{1.616783in}{0.958365in}}%
\pgfpathlineto{\pgfqpoint{1.617485in}{0.973365in}}%
\pgfpathlineto{\pgfqpoint{1.617787in}{0.978681in}}%
\pgfpathlineto{\pgfqpoint{1.617987in}{0.970284in}}%
\pgfpathlineto{\pgfqpoint{1.618088in}{0.959834in}}%
\pgfpathlineto{\pgfqpoint{1.618288in}{0.975092in}}%
\pgfpathlineto{\pgfqpoint{1.619091in}{0.970903in}}%
\pgfpathlineto{\pgfqpoint{1.619192in}{0.970353in}}%
\pgfpathlineto{\pgfqpoint{1.620095in}{0.944259in}}%
\pgfpathlineto{\pgfqpoint{1.620496in}{0.946142in}}%
\pgfpathlineto{\pgfqpoint{1.621902in}{0.962920in}}%
\pgfpathlineto{\pgfqpoint{1.622002in}{0.953101in}}%
\pgfpathlineto{\pgfqpoint{1.622704in}{0.935763in}}%
\pgfpathlineto{\pgfqpoint{1.622403in}{0.959487in}}%
\pgfpathlineto{\pgfqpoint{1.623106in}{0.946402in}}%
\pgfpathlineto{\pgfqpoint{1.623307in}{0.954997in}}%
\pgfpathlineto{\pgfqpoint{1.623708in}{0.938833in}}%
\pgfpathlineto{\pgfqpoint{1.624210in}{0.950011in}}%
\pgfpathlineto{\pgfqpoint{1.624310in}{0.951116in}}%
\pgfpathlineto{\pgfqpoint{1.624712in}{0.945776in}}%
\pgfpathlineto{\pgfqpoint{1.624812in}{0.939389in}}%
\pgfpathlineto{\pgfqpoint{1.625414in}{0.964503in}}%
\pgfpathlineto{\pgfqpoint{1.626217in}{0.980456in}}%
\pgfpathlineto{\pgfqpoint{1.625615in}{0.960019in}}%
\pgfpathlineto{\pgfqpoint{1.626318in}{0.967830in}}%
\pgfpathlineto{\pgfqpoint{1.626819in}{0.960282in}}%
\pgfpathlineto{\pgfqpoint{1.627020in}{0.976326in}}%
\pgfpathlineto{\pgfqpoint{1.627221in}{0.973129in}}%
\pgfpathlineto{\pgfqpoint{1.627622in}{1.005538in}}%
\pgfpathlineto{\pgfqpoint{1.628726in}{0.996423in}}%
\pgfpathlineto{\pgfqpoint{1.629429in}{0.985421in}}%
\pgfpathlineto{\pgfqpoint{1.629830in}{0.993326in}}%
\pgfpathlineto{\pgfqpoint{1.630031in}{1.000810in}}%
\pgfpathlineto{\pgfqpoint{1.630232in}{0.986653in}}%
\pgfpathlineto{\pgfqpoint{1.630834in}{0.999012in}}%
\pgfpathlineto{\pgfqpoint{1.631436in}{0.984388in}}%
\pgfpathlineto{\pgfqpoint{1.631938in}{0.992567in}}%
\pgfpathlineto{\pgfqpoint{1.632139in}{1.006682in}}%
\pgfpathlineto{\pgfqpoint{1.632640in}{0.982643in}}%
\pgfpathlineto{\pgfqpoint{1.632741in}{0.984782in}}%
\pgfpathlineto{\pgfqpoint{1.634146in}{0.954025in}}%
\pgfpathlineto{\pgfqpoint{1.634246in}{0.958242in}}%
\pgfpathlineto{\pgfqpoint{1.634748in}{0.974383in}}%
\pgfpathlineto{\pgfqpoint{1.635250in}{0.967960in}}%
\pgfpathlineto{\pgfqpoint{1.636153in}{0.954597in}}%
\pgfpathlineto{\pgfqpoint{1.635852in}{0.974633in}}%
\pgfpathlineto{\pgfqpoint{1.636354in}{0.957908in}}%
\pgfpathlineto{\pgfqpoint{1.636454in}{0.961265in}}%
\pgfpathlineto{\pgfqpoint{1.636755in}{0.946359in}}%
\pgfpathlineto{\pgfqpoint{1.637157in}{0.953105in}}%
\pgfpathlineto{\pgfqpoint{1.637859in}{0.944508in}}%
\pgfpathlineto{\pgfqpoint{1.638060in}{0.958165in}}%
\pgfpathlineto{\pgfqpoint{1.638161in}{0.953608in}}%
\pgfpathlineto{\pgfqpoint{1.638361in}{0.974964in}}%
\pgfpathlineto{\pgfqpoint{1.639265in}{0.964437in}}%
\pgfpathlineto{\pgfqpoint{1.639365in}{0.955455in}}%
\pgfpathlineto{\pgfqpoint{1.640268in}{0.961768in}}%
\pgfpathlineto{\pgfqpoint{1.641874in}{0.994696in}}%
\pgfpathlineto{\pgfqpoint{1.642577in}{0.986226in}}%
\pgfpathlineto{\pgfqpoint{1.642677in}{0.977771in}}%
\pgfpathlineto{\pgfqpoint{1.643179in}{0.993386in}}%
\pgfpathlineto{\pgfqpoint{1.643681in}{0.985059in}}%
\pgfpathlineto{\pgfqpoint{1.644182in}{0.968095in}}%
\pgfpathlineto{\pgfqpoint{1.643881in}{0.986179in}}%
\pgfpathlineto{\pgfqpoint{1.644885in}{0.977341in}}%
\pgfpathlineto{\pgfqpoint{1.645086in}{0.981242in}}%
\pgfpathlineto{\pgfqpoint{1.646089in}{0.995988in}}%
\pgfpathlineto{\pgfqpoint{1.645387in}{0.978435in}}%
\pgfpathlineto{\pgfqpoint{1.646190in}{0.988887in}}%
\pgfpathlineto{\pgfqpoint{1.647394in}{0.968709in}}%
\pgfpathlineto{\pgfqpoint{1.647595in}{0.970509in}}%
\pgfpathlineto{\pgfqpoint{1.647695in}{0.970636in}}%
\pgfpathlineto{\pgfqpoint{1.648398in}{0.950700in}}%
\pgfpathlineto{\pgfqpoint{1.648900in}{0.963347in}}%
\pgfpathlineto{\pgfqpoint{1.649301in}{0.969953in}}%
\pgfpathlineto{\pgfqpoint{1.649502in}{0.959501in}}%
\pgfpathlineto{\pgfqpoint{1.650606in}{0.947512in}}%
\pgfpathlineto{\pgfqpoint{1.650706in}{0.949290in}}%
\pgfpathlineto{\pgfqpoint{1.651509in}{0.938265in}}%
\pgfpathlineto{\pgfqpoint{1.651007in}{0.956485in}}%
\pgfpathlineto{\pgfqpoint{1.651910in}{0.940221in}}%
\pgfpathlineto{\pgfqpoint{1.652212in}{0.949197in}}%
\pgfpathlineto{\pgfqpoint{1.652412in}{0.919119in}}%
\pgfpathlineto{\pgfqpoint{1.652513in}{0.920617in}}%
\pgfpathlineto{\pgfqpoint{1.652814in}{0.910377in}}%
\pgfpathlineto{\pgfqpoint{1.653316in}{0.928112in}}%
\pgfpathlineto{\pgfqpoint{1.653416in}{0.925695in}}%
\pgfpathlineto{\pgfqpoint{1.653516in}{0.931353in}}%
\pgfpathlineto{\pgfqpoint{1.654319in}{0.918540in}}%
\pgfpathlineto{\pgfqpoint{1.654420in}{0.922884in}}%
\pgfpathlineto{\pgfqpoint{1.655122in}{0.925723in}}%
\pgfpathlineto{\pgfqpoint{1.654821in}{0.917438in}}%
\pgfpathlineto{\pgfqpoint{1.655222in}{0.925115in}}%
\pgfpathlineto{\pgfqpoint{1.655323in}{0.918867in}}%
\pgfpathlineto{\pgfqpoint{1.655624in}{0.938630in}}%
\pgfpathlineto{\pgfqpoint{1.656326in}{0.924649in}}%
\pgfpathlineto{\pgfqpoint{1.656527in}{0.917968in}}%
\pgfpathlineto{\pgfqpoint{1.657330in}{0.927015in}}%
\pgfpathlineto{\pgfqpoint{1.657531in}{0.930659in}}%
\pgfpathlineto{\pgfqpoint{1.657832in}{0.923046in}}%
\pgfpathlineto{\pgfqpoint{1.658735in}{0.888215in}}%
\pgfpathlineto{\pgfqpoint{1.659036in}{0.910630in}}%
\pgfpathlineto{\pgfqpoint{1.659137in}{0.916800in}}%
\pgfpathlineto{\pgfqpoint{1.659839in}{0.894501in}}%
\pgfpathlineto{\pgfqpoint{1.661144in}{0.884825in}}%
\pgfpathlineto{\pgfqpoint{1.660140in}{0.895983in}}%
\pgfpathlineto{\pgfqpoint{1.661244in}{0.887759in}}%
\pgfpathlineto{\pgfqpoint{1.661646in}{0.907101in}}%
\pgfpathlineto{\pgfqpoint{1.662449in}{0.904870in}}%
\pgfpathlineto{\pgfqpoint{1.662649in}{0.896698in}}%
\pgfpathlineto{\pgfqpoint{1.663051in}{0.915627in}}%
\pgfpathlineto{\pgfqpoint{1.663352in}{0.911627in}}%
\pgfpathlineto{\pgfqpoint{1.663452in}{0.911869in}}%
\pgfpathlineto{\pgfqpoint{1.663553in}{0.910588in}}%
\pgfpathlineto{\pgfqpoint{1.664356in}{0.896645in}}%
\pgfpathlineto{\pgfqpoint{1.664657in}{0.906174in}}%
\pgfpathlineto{\pgfqpoint{1.665159in}{0.896433in}}%
\pgfpathlineto{\pgfqpoint{1.665359in}{0.909948in}}%
\pgfpathlineto{\pgfqpoint{1.667065in}{0.956508in}}%
\pgfpathlineto{\pgfqpoint{1.665961in}{0.909198in}}%
\pgfpathlineto{\pgfqpoint{1.667266in}{0.952675in}}%
\pgfpathlineto{\pgfqpoint{1.667367in}{0.953179in}}%
\pgfpathlineto{\pgfqpoint{1.668772in}{0.935232in}}%
\pgfpathlineto{\pgfqpoint{1.669073in}{0.937849in}}%
\pgfpathlineto{\pgfqpoint{1.669775in}{0.952484in}}%
\pgfpathlineto{\pgfqpoint{1.669474in}{0.922624in}}%
\pgfpathlineto{\pgfqpoint{1.669976in}{0.942055in}}%
\pgfpathlineto{\pgfqpoint{1.670277in}{0.925395in}}%
\pgfpathlineto{\pgfqpoint{1.671180in}{0.929902in}}%
\pgfpathlineto{\pgfqpoint{1.671983in}{0.939227in}}%
\pgfpathlineto{\pgfqpoint{1.671582in}{0.925792in}}%
\pgfpathlineto{\pgfqpoint{1.672385in}{0.933092in}}%
\pgfpathlineto{\pgfqpoint{1.672485in}{0.928273in}}%
\pgfpathlineto{\pgfqpoint{1.672887in}{0.947630in}}%
\pgfpathlineto{\pgfqpoint{1.672987in}{0.944769in}}%
\pgfpathlineto{\pgfqpoint{1.673087in}{0.953610in}}%
\pgfpathlineto{\pgfqpoint{1.673890in}{0.926964in}}%
\pgfpathlineto{\pgfqpoint{1.674392in}{0.904301in}}%
\pgfpathlineto{\pgfqpoint{1.675797in}{0.905618in}}%
\pgfpathlineto{\pgfqpoint{1.676600in}{0.918236in}}%
\pgfpathlineto{\pgfqpoint{1.676299in}{0.904549in}}%
\pgfpathlineto{\pgfqpoint{1.677002in}{0.913393in}}%
\pgfpathlineto{\pgfqpoint{1.677303in}{0.906369in}}%
\pgfpathlineto{\pgfqpoint{1.677704in}{0.934378in}}%
\pgfpathlineto{\pgfqpoint{1.677905in}{0.938762in}}%
\pgfpathlineto{\pgfqpoint{1.678708in}{0.949079in}}%
\pgfpathlineto{\pgfqpoint{1.678908in}{0.935782in}}%
\pgfpathlineto{\pgfqpoint{1.679210in}{0.929908in}}%
\pgfpathlineto{\pgfqpoint{1.679511in}{0.944311in}}%
\pgfpathlineto{\pgfqpoint{1.679711in}{0.941798in}}%
\pgfpathlineto{\pgfqpoint{1.679912in}{0.938081in}}%
\pgfpathlineto{\pgfqpoint{1.680414in}{0.946910in}}%
\pgfpathlineto{\pgfqpoint{1.681418in}{0.912910in}}%
\pgfpathlineto{\pgfqpoint{1.681719in}{0.919584in}}%
\pgfpathlineto{\pgfqpoint{1.681819in}{0.928992in}}%
\pgfpathlineto{\pgfqpoint{1.682622in}{0.913394in}}%
\pgfpathlineto{\pgfqpoint{1.682823in}{0.921035in}}%
\pgfpathlineto{\pgfqpoint{1.683425in}{0.911484in}}%
\pgfpathlineto{\pgfqpoint{1.684027in}{0.928151in}}%
\pgfpathlineto{\pgfqpoint{1.684930in}{0.900669in}}%
\pgfpathlineto{\pgfqpoint{1.685432in}{0.909086in}}%
\pgfpathlineto{\pgfqpoint{1.685633in}{0.917568in}}%
\pgfpathlineto{\pgfqpoint{1.686034in}{0.895493in}}%
\pgfpathlineto{\pgfqpoint{1.686436in}{0.904428in}}%
\pgfpathlineto{\pgfqpoint{1.686536in}{0.900750in}}%
\pgfpathlineto{\pgfqpoint{1.687239in}{0.911325in}}%
\pgfpathlineto{\pgfqpoint{1.687439in}{0.901419in}}%
\pgfpathlineto{\pgfqpoint{1.687741in}{0.915773in}}%
\pgfpathlineto{\pgfqpoint{1.688443in}{0.888064in}}%
\pgfpathlineto{\pgfqpoint{1.688543in}{0.889776in}}%
\pgfpathlineto{\pgfqpoint{1.688845in}{0.878342in}}%
\pgfpathlineto{\pgfqpoint{1.690852in}{0.822977in}}%
\pgfpathlineto{\pgfqpoint{1.690952in}{0.823384in}}%
\pgfpathlineto{\pgfqpoint{1.691053in}{0.828959in}}%
\pgfpathlineto{\pgfqpoint{1.691253in}{0.819877in}}%
\pgfpathlineto{\pgfqpoint{1.691855in}{0.821111in}}%
\pgfpathlineto{\pgfqpoint{1.691956in}{0.808033in}}%
\pgfpathlineto{\pgfqpoint{1.692859in}{0.830576in}}%
\pgfpathlineto{\pgfqpoint{1.693160in}{0.825614in}}%
\pgfpathlineto{\pgfqpoint{1.693461in}{0.842389in}}%
\pgfpathlineto{\pgfqpoint{1.693662in}{0.836077in}}%
\pgfpathlineto{\pgfqpoint{1.695770in}{0.876818in}}%
\pgfpathlineto{\pgfqpoint{1.695970in}{0.877106in}}%
\pgfpathlineto{\pgfqpoint{1.696171in}{0.891430in}}%
\pgfpathlineto{\pgfqpoint{1.696974in}{0.873209in}}%
\pgfpathlineto{\pgfqpoint{1.697275in}{0.858377in}}%
\pgfpathlineto{\pgfqpoint{1.697576in}{0.874349in}}%
\pgfpathlineto{\pgfqpoint{1.697978in}{0.871418in}}%
\pgfpathlineto{\pgfqpoint{1.698078in}{0.876301in}}%
\pgfpathlineto{\pgfqpoint{1.698781in}{0.860122in}}%
\pgfpathlineto{\pgfqpoint{1.699383in}{0.844403in}}%
\pgfpathlineto{\pgfqpoint{1.699885in}{0.854523in}}%
\pgfpathlineto{\pgfqpoint{1.700788in}{0.865984in}}%
\pgfpathlineto{\pgfqpoint{1.700186in}{0.853306in}}%
\pgfpathlineto{\pgfqpoint{1.700989in}{0.862185in}}%
\pgfpathlineto{\pgfqpoint{1.701490in}{0.854810in}}%
\pgfpathlineto{\pgfqpoint{1.701992in}{0.866077in}}%
\pgfpathlineto{\pgfqpoint{1.704100in}{0.897755in}}%
\pgfpathlineto{\pgfqpoint{1.702695in}{0.863167in}}%
\pgfpathlineto{\pgfqpoint{1.704401in}{0.894211in}}%
\pgfpathlineto{\pgfqpoint{1.705204in}{0.858729in}}%
\pgfpathlineto{\pgfqpoint{1.705706in}{0.880046in}}%
\pgfpathlineto{\pgfqpoint{1.705806in}{0.890013in}}%
\pgfpathlineto{\pgfqpoint{1.706107in}{0.879451in}}%
\pgfpathlineto{\pgfqpoint{1.706709in}{0.882285in}}%
\pgfpathlineto{\pgfqpoint{1.706810in}{0.879292in}}%
\pgfpathlineto{\pgfqpoint{1.707211in}{0.891412in}}%
\pgfpathlineto{\pgfqpoint{1.707412in}{0.888583in}}%
\pgfpathlineto{\pgfqpoint{1.708516in}{0.913714in}}%
\pgfpathlineto{\pgfqpoint{1.708616in}{0.909623in}}%
\pgfpathlineto{\pgfqpoint{1.710122in}{0.890427in}}%
\pgfpathlineto{\pgfqpoint{1.709319in}{0.920962in}}%
\pgfpathlineto{\pgfqpoint{1.710222in}{0.895115in}}%
\pgfpathlineto{\pgfqpoint{1.711226in}{0.912820in}}%
\pgfpathlineto{\pgfqpoint{1.710925in}{0.892930in}}%
\pgfpathlineto{\pgfqpoint{1.711326in}{0.902306in}}%
\pgfpathlineto{\pgfqpoint{1.712430in}{0.881050in}}%
\pgfpathlineto{\pgfqpoint{1.713133in}{0.887656in}}%
\pgfpathlineto{\pgfqpoint{1.713735in}{0.912646in}}%
\pgfpathlineto{\pgfqpoint{1.714538in}{0.909778in}}%
\pgfpathlineto{\pgfqpoint{1.714638in}{0.896624in}}%
\pgfpathlineto{\pgfqpoint{1.715341in}{0.911584in}}%
\pgfpathlineto{\pgfqpoint{1.715642in}{0.899942in}}%
\pgfpathlineto{\pgfqpoint{1.715742in}{0.899528in}}%
\pgfpathlineto{\pgfqpoint{1.716746in}{0.913977in}}%
\pgfpathlineto{\pgfqpoint{1.716947in}{0.905537in}}%
\pgfpathlineto{\pgfqpoint{1.717047in}{0.905556in}}%
\pgfpathlineto{\pgfqpoint{1.717147in}{0.904528in}}%
\pgfpathlineto{\pgfqpoint{1.718452in}{0.925253in}}%
\pgfpathlineto{\pgfqpoint{1.719355in}{0.915680in}}%
\pgfpathlineto{\pgfqpoint{1.719155in}{0.925533in}}%
\pgfpathlineto{\pgfqpoint{1.719656in}{0.919054in}}%
\pgfpathlineto{\pgfqpoint{1.720760in}{0.910031in}}%
\pgfpathlineto{\pgfqpoint{1.720259in}{0.920352in}}%
\pgfpathlineto{\pgfqpoint{1.720861in}{0.914866in}}%
\pgfpathlineto{\pgfqpoint{1.721563in}{0.930997in}}%
\pgfpathlineto{\pgfqpoint{1.721262in}{0.910509in}}%
\pgfpathlineto{\pgfqpoint{1.721864in}{0.915457in}}%
\pgfpathlineto{\pgfqpoint{1.722266in}{0.901508in}}%
\pgfpathlineto{\pgfqpoint{1.722768in}{0.922357in}}%
\pgfpathlineto{\pgfqpoint{1.722968in}{0.914699in}}%
\pgfpathlineto{\pgfqpoint{1.724373in}{0.931270in}}%
\pgfpathlineto{\pgfqpoint{1.723269in}{0.905170in}}%
\pgfpathlineto{\pgfqpoint{1.724474in}{0.926065in}}%
\pgfpathlineto{\pgfqpoint{1.725277in}{0.894790in}}%
\pgfpathlineto{\pgfqpoint{1.725779in}{0.917735in}}%
\pgfpathlineto{\pgfqpoint{1.726180in}{0.926388in}}%
\pgfpathlineto{\pgfqpoint{1.726582in}{0.911005in}}%
\pgfpathlineto{\pgfqpoint{1.727987in}{0.895925in}}%
\pgfpathlineto{\pgfqpoint{1.728689in}{0.902973in}}%
\pgfpathlineto{\pgfqpoint{1.728388in}{0.888705in}}%
\pgfpathlineto{\pgfqpoint{1.728990in}{0.902203in}}%
\pgfpathlineto{\pgfqpoint{1.730094in}{0.864818in}}%
\pgfpathlineto{\pgfqpoint{1.730395in}{0.873828in}}%
\pgfpathlineto{\pgfqpoint{1.731499in}{0.846718in}}%
\pgfpathlineto{\pgfqpoint{1.730596in}{0.875544in}}%
\pgfpathlineto{\pgfqpoint{1.732001in}{0.854228in}}%
\pgfpathlineto{\pgfqpoint{1.732603in}{0.875107in}}%
\pgfpathlineto{\pgfqpoint{1.733005in}{0.858388in}}%
\pgfpathlineto{\pgfqpoint{1.734209in}{0.835538in}}%
\pgfpathlineto{\pgfqpoint{1.733808in}{0.860959in}}%
\pgfpathlineto{\pgfqpoint{1.734310in}{0.839901in}}%
\pgfpathlineto{\pgfqpoint{1.735715in}{0.863716in}}%
\pgfpathlineto{\pgfqpoint{1.734912in}{0.837888in}}%
\pgfpathlineto{\pgfqpoint{1.735915in}{0.855686in}}%
\pgfpathlineto{\pgfqpoint{1.736216in}{0.840601in}}%
\pgfpathlineto{\pgfqpoint{1.736919in}{0.857064in}}%
\pgfpathlineto{\pgfqpoint{1.737019in}{0.853472in}}%
\pgfpathlineto{\pgfqpoint{1.737722in}{0.851138in}}%
\pgfpathlineto{\pgfqpoint{1.738525in}{0.875888in}}%
\pgfpathlineto{\pgfqpoint{1.738826in}{0.892423in}}%
\pgfpathlineto{\pgfqpoint{1.739428in}{0.867874in}}%
\pgfpathlineto{\pgfqpoint{1.739830in}{0.863570in}}%
\pgfpathlineto{\pgfqpoint{1.739930in}{0.869216in}}%
\pgfpathlineto{\pgfqpoint{1.740231in}{0.865152in}}%
\pgfpathlineto{\pgfqpoint{1.740432in}{0.877290in}}%
\pgfpathlineto{\pgfqpoint{1.741134in}{0.852059in}}%
\pgfpathlineto{\pgfqpoint{1.741335in}{0.867311in}}%
\pgfpathlineto{\pgfqpoint{1.741636in}{0.859958in}}%
\pgfpathlineto{\pgfqpoint{1.741937in}{0.873683in}}%
\pgfpathlineto{\pgfqpoint{1.742238in}{0.868164in}}%
\pgfpathlineto{\pgfqpoint{1.742740in}{0.878786in}}%
\pgfpathlineto{\pgfqpoint{1.743142in}{0.861201in}}%
\pgfpathlineto{\pgfqpoint{1.744446in}{0.843155in}}%
\pgfpathlineto{\pgfqpoint{1.743643in}{0.866720in}}%
\pgfpathlineto{\pgfqpoint{1.744747in}{0.844186in}}%
\pgfpathlineto{\pgfqpoint{1.745550in}{0.873267in}}%
\pgfpathlineto{\pgfqpoint{1.746253in}{0.866416in}}%
\pgfpathlineto{\pgfqpoint{1.746353in}{0.866587in}}%
\pgfpathlineto{\pgfqpoint{1.747257in}{0.848326in}}%
\pgfpathlineto{\pgfqpoint{1.746855in}{0.875351in}}%
\pgfpathlineto{\pgfqpoint{1.747558in}{0.860414in}}%
\pgfpathlineto{\pgfqpoint{1.747658in}{0.861478in}}%
\pgfpathlineto{\pgfqpoint{1.747758in}{0.860458in}}%
\pgfpathlineto{\pgfqpoint{1.749264in}{0.787746in}}%
\pgfpathlineto{\pgfqpoint{1.749465in}{0.790428in}}%
\pgfpathlineto{\pgfqpoint{1.751673in}{0.754322in}}%
\pgfpathlineto{\pgfqpoint{1.751974in}{0.759855in}}%
\pgfpathlineto{\pgfqpoint{1.753278in}{0.826945in}}%
\pgfpathlineto{\pgfqpoint{1.753680in}{0.819474in}}%
\pgfpathlineto{\pgfqpoint{1.753981in}{0.800366in}}%
\pgfpathlineto{\pgfqpoint{1.754382in}{0.823399in}}%
\pgfpathlineto{\pgfqpoint{1.754884in}{0.807379in}}%
\pgfpathlineto{\pgfqpoint{1.755085in}{0.809999in}}%
\pgfpathlineto{\pgfqpoint{1.755185in}{0.814503in}}%
\pgfpathlineto{\pgfqpoint{1.755486in}{0.800432in}}%
\pgfpathlineto{\pgfqpoint{1.756189in}{0.813505in}}%
\pgfpathlineto{\pgfqpoint{1.757293in}{0.800035in}}%
\pgfpathlineto{\pgfqpoint{1.757594in}{0.801586in}}%
\pgfpathlineto{\pgfqpoint{1.757694in}{0.810591in}}%
\pgfpathlineto{\pgfqpoint{1.758397in}{0.789353in}}%
\pgfpathlineto{\pgfqpoint{1.758497in}{0.790836in}}%
\pgfpathlineto{\pgfqpoint{1.760003in}{0.766650in}}%
\pgfpathlineto{\pgfqpoint{1.759200in}{0.798259in}}%
\pgfpathlineto{\pgfqpoint{1.760304in}{0.766667in}}%
\pgfpathlineto{\pgfqpoint{1.760505in}{0.780211in}}%
\pgfpathlineto{\pgfqpoint{1.761308in}{0.763002in}}%
\pgfpathlineto{\pgfqpoint{1.761408in}{0.764692in}}%
\pgfpathlineto{\pgfqpoint{1.761508in}{0.757746in}}%
\pgfpathlineto{\pgfqpoint{1.761809in}{0.759390in}}%
\pgfpathlineto{\pgfqpoint{1.762211in}{0.750267in}}%
\pgfpathlineto{\pgfqpoint{1.762813in}{0.761024in}}%
\pgfpathlineto{\pgfqpoint{1.763214in}{0.776862in}}%
\pgfpathlineto{\pgfqpoint{1.763716in}{0.763540in}}%
\pgfpathlineto{\pgfqpoint{1.764017in}{0.752655in}}%
\pgfpathlineto{\pgfqpoint{1.764720in}{0.765696in}}%
\pgfpathlineto{\pgfqpoint{1.765121in}{0.771246in}}%
\pgfpathlineto{\pgfqpoint{1.765322in}{0.770890in}}%
\pgfpathlineto{\pgfqpoint{1.767028in}{0.716139in}}%
\pgfpathlineto{\pgfqpoint{1.767129in}{0.719044in}}%
\pgfpathlineto{\pgfqpoint{1.767731in}{0.704840in}}%
\pgfpathlineto{\pgfqpoint{1.767831in}{0.713422in}}%
\pgfpathlineto{\pgfqpoint{1.768835in}{0.692743in}}%
\pgfpathlineto{\pgfqpoint{1.768032in}{0.715168in}}%
\pgfpathlineto{\pgfqpoint{1.769136in}{0.696901in}}%
\pgfpathlineto{\pgfqpoint{1.769236in}{0.698210in}}%
\pgfpathlineto{\pgfqpoint{1.769337in}{0.692558in}}%
\pgfpathlineto{\pgfqpoint{1.769537in}{0.694082in}}%
\pgfpathlineto{\pgfqpoint{1.769939in}{0.672975in}}%
\pgfpathlineto{\pgfqpoint{1.770641in}{0.682804in}}%
\pgfpathlineto{\pgfqpoint{1.771143in}{0.702480in}}%
\pgfpathlineto{\pgfqpoint{1.771745in}{0.690504in}}%
\pgfpathlineto{\pgfqpoint{1.772047in}{0.664170in}}%
\pgfpathlineto{\pgfqpoint{1.772950in}{0.671036in}}%
\pgfpathlineto{\pgfqpoint{1.773753in}{0.688143in}}%
\pgfpathlineto{\pgfqpoint{1.774054in}{0.675212in}}%
\pgfpathlineto{\pgfqpoint{1.776262in}{0.714297in}}%
\pgfpathlineto{\pgfqpoint{1.776362in}{0.713178in}}%
\pgfpathlineto{\pgfqpoint{1.777667in}{0.695190in}}%
\pgfpathlineto{\pgfqpoint{1.777767in}{0.702096in}}%
\pgfpathlineto{\pgfqpoint{1.778470in}{0.715591in}}%
\pgfpathlineto{\pgfqpoint{1.777968in}{0.696450in}}%
\pgfpathlineto{\pgfqpoint{1.778972in}{0.711174in}}%
\pgfpathlineto{\pgfqpoint{1.780076in}{0.686548in}}%
\pgfpathlineto{\pgfqpoint{1.780276in}{0.692202in}}%
\pgfpathlineto{\pgfqpoint{1.780879in}{0.719427in}}%
\pgfpathlineto{\pgfqpoint{1.781581in}{0.705070in}}%
\pgfpathlineto{\pgfqpoint{1.782083in}{0.687383in}}%
\pgfpathlineto{\pgfqpoint{1.781782in}{0.706355in}}%
\pgfpathlineto{\pgfqpoint{1.782585in}{0.698506in}}%
\pgfpathlineto{\pgfqpoint{1.782786in}{0.708449in}}%
\pgfpathlineto{\pgfqpoint{1.783488in}{0.685366in}}%
\pgfpathlineto{\pgfqpoint{1.783689in}{0.673563in}}%
\pgfpathlineto{\pgfqpoint{1.784090in}{0.698496in}}%
\pgfpathlineto{\pgfqpoint{1.784191in}{0.696226in}}%
\pgfpathlineto{\pgfqpoint{1.784492in}{0.704205in}}%
\pgfpathlineto{\pgfqpoint{1.784692in}{0.688418in}}%
\pgfpathlineto{\pgfqpoint{1.785295in}{0.698609in}}%
\pgfpathlineto{\pgfqpoint{1.785395in}{0.701864in}}%
\pgfpathlineto{\pgfqpoint{1.786098in}{0.689485in}}%
\pgfpathlineto{\pgfqpoint{1.786298in}{0.694318in}}%
\pgfpathlineto{\pgfqpoint{1.786700in}{0.683288in}}%
\pgfpathlineto{\pgfqpoint{1.786900in}{0.671718in}}%
\pgfpathlineto{\pgfqpoint{1.787703in}{0.684641in}}%
\pgfpathlineto{\pgfqpoint{1.788406in}{0.707718in}}%
\pgfpathlineto{\pgfqpoint{1.788908in}{0.689735in}}%
\pgfpathlineto{\pgfqpoint{1.789309in}{0.700826in}}%
\pgfpathlineto{\pgfqpoint{1.789610in}{0.682426in}}%
\pgfpathlineto{\pgfqpoint{1.789911in}{0.672466in}}%
\pgfpathlineto{\pgfqpoint{1.790313in}{0.686599in}}%
\pgfpathlineto{\pgfqpoint{1.790413in}{0.691343in}}%
\pgfpathlineto{\pgfqpoint{1.791116in}{0.678539in}}%
\pgfpathlineto{\pgfqpoint{1.791317in}{0.659973in}}%
\pgfpathlineto{\pgfqpoint{1.791919in}{0.682820in}}%
\pgfpathlineto{\pgfqpoint{1.792119in}{0.667428in}}%
\pgfpathlineto{\pgfqpoint{1.793023in}{0.696564in}}%
\pgfpathlineto{\pgfqpoint{1.793424in}{0.694773in}}%
\pgfpathlineto{\pgfqpoint{1.793625in}{0.701085in}}%
\pgfpathlineto{\pgfqpoint{1.793826in}{0.700119in}}%
\pgfpathlineto{\pgfqpoint{1.794428in}{0.673931in}}%
\pgfpathlineto{\pgfqpoint{1.795030in}{0.675042in}}%
\pgfpathlineto{\pgfqpoint{1.795532in}{0.662866in}}%
\pgfpathlineto{\pgfqpoint{1.795833in}{0.674582in}}%
\pgfpathlineto{\pgfqpoint{1.796234in}{0.667555in}}%
\pgfpathlineto{\pgfqpoint{1.797740in}{0.713751in}}%
\pgfpathlineto{\pgfqpoint{1.798442in}{0.686184in}}%
\pgfpathlineto{\pgfqpoint{1.799145in}{0.706627in}}%
\pgfpathlineto{\pgfqpoint{1.799546in}{0.699392in}}%
\pgfpathlineto{\pgfqpoint{1.800249in}{0.705346in}}%
\pgfpathlineto{\pgfqpoint{1.801453in}{0.724519in}}%
\pgfpathlineto{\pgfqpoint{1.802256in}{0.688669in}}%
\pgfpathlineto{\pgfqpoint{1.802658in}{0.703309in}}%
\pgfpathlineto{\pgfqpoint{1.803962in}{0.749420in}}%
\pgfpathlineto{\pgfqpoint{1.804364in}{0.740779in}}%
\pgfpathlineto{\pgfqpoint{1.804765in}{0.722804in}}%
\pgfpathlineto{\pgfqpoint{1.805167in}{0.748085in}}%
\pgfpathlineto{\pgfqpoint{1.805267in}{0.757770in}}%
\pgfpathlineto{\pgfqpoint{1.805869in}{0.731250in}}%
\pgfpathlineto{\pgfqpoint{1.806070in}{0.732594in}}%
\pgfpathlineto{\pgfqpoint{1.807475in}{0.670204in}}%
\pgfpathlineto{\pgfqpoint{1.808278in}{0.687434in}}%
\pgfpathlineto{\pgfqpoint{1.809282in}{0.704472in}}%
\pgfpathlineto{\pgfqpoint{1.809583in}{0.693014in}}%
\pgfpathlineto{\pgfqpoint{1.809683in}{0.689190in}}%
\pgfpathlineto{\pgfqpoint{1.809884in}{0.709749in}}%
\pgfpathlineto{\pgfqpoint{1.810285in}{0.704237in}}%
\pgfpathlineto{\pgfqpoint{1.810586in}{0.700974in}}%
\pgfpathlineto{\pgfqpoint{1.810687in}{0.708073in}}%
\pgfpathlineto{\pgfqpoint{1.811992in}{0.747872in}}%
\pgfpathlineto{\pgfqpoint{1.812192in}{0.740211in}}%
\pgfpathlineto{\pgfqpoint{1.812493in}{0.742956in}}%
\pgfpathlineto{\pgfqpoint{1.812393in}{0.739529in}}%
\pgfpathlineto{\pgfqpoint{1.812594in}{0.740563in}}%
\pgfpathlineto{\pgfqpoint{1.812694in}{0.749637in}}%
\pgfpathlineto{\pgfqpoint{1.813397in}{0.729978in}}%
\pgfpathlineto{\pgfqpoint{1.814501in}{0.715007in}}%
\pgfpathlineto{\pgfqpoint{1.813999in}{0.732125in}}%
\pgfpathlineto{\pgfqpoint{1.814601in}{0.719501in}}%
\pgfpathlineto{\pgfqpoint{1.814802in}{0.726695in}}%
\pgfpathlineto{\pgfqpoint{1.815304in}{0.710261in}}%
\pgfpathlineto{\pgfqpoint{1.816307in}{0.683575in}}%
\pgfpathlineto{\pgfqpoint{1.816809in}{0.690897in}}%
\pgfpathlineto{\pgfqpoint{1.816909in}{0.698977in}}%
\pgfpathlineto{\pgfqpoint{1.817712in}{0.682229in}}%
\pgfpathlineto{\pgfqpoint{1.818114in}{0.689271in}}%
\pgfpathlineto{\pgfqpoint{1.819419in}{0.639825in}}%
\pgfpathlineto{\pgfqpoint{1.820422in}{0.629449in}}%
\pgfpathlineto{\pgfqpoint{1.819920in}{0.641252in}}%
\pgfpathlineto{\pgfqpoint{1.820623in}{0.630529in}}%
\pgfpathlineto{\pgfqpoint{1.820824in}{0.639964in}}%
\pgfpathlineto{\pgfqpoint{1.821225in}{0.613969in}}%
\pgfpathlineto{\pgfqpoint{1.821325in}{0.608517in}}%
\pgfpathlineto{\pgfqpoint{1.821727in}{0.639128in}}%
\pgfpathlineto{\pgfqpoint{1.821928in}{0.633195in}}%
\pgfpathlineto{\pgfqpoint{1.823032in}{0.649246in}}%
\pgfpathlineto{\pgfqpoint{1.822329in}{0.627590in}}%
\pgfpathlineto{\pgfqpoint{1.823232in}{0.642448in}}%
\pgfpathlineto{\pgfqpoint{1.823433in}{0.639143in}}%
\pgfpathlineto{\pgfqpoint{1.823533in}{0.646661in}}%
\pgfpathlineto{\pgfqpoint{1.823734in}{0.653413in}}%
\pgfpathlineto{\pgfqpoint{1.824437in}{0.634400in}}%
\pgfpathlineto{\pgfqpoint{1.824537in}{0.634276in}}%
\pgfpathlineto{\pgfqpoint{1.824838in}{0.612568in}}%
\pgfpathlineto{\pgfqpoint{1.825641in}{0.627502in}}%
\pgfpathlineto{\pgfqpoint{1.825942in}{0.620432in}}%
\pgfpathlineto{\pgfqpoint{1.826344in}{0.635349in}}%
\pgfpathlineto{\pgfqpoint{1.827147in}{0.667477in}}%
\pgfpathlineto{\pgfqpoint{1.827749in}{0.666977in}}%
\pgfpathlineto{\pgfqpoint{1.828050in}{0.676886in}}%
\pgfpathlineto{\pgfqpoint{1.828351in}{0.650236in}}%
\pgfpathlineto{\pgfqpoint{1.829355in}{0.633465in}}%
\pgfpathlineto{\pgfqpoint{1.828853in}{0.654442in}}%
\pgfpathlineto{\pgfqpoint{1.829555in}{0.645961in}}%
\pgfpathlineto{\pgfqpoint{1.831061in}{0.684918in}}%
\pgfpathlineto{\pgfqpoint{1.831161in}{0.684441in}}%
\pgfpathlineto{\pgfqpoint{1.831663in}{0.703831in}}%
\pgfpathlineto{\pgfqpoint{1.832165in}{0.682798in}}%
\pgfpathlineto{\pgfqpoint{1.832265in}{0.676541in}}%
\pgfpathlineto{\pgfqpoint{1.833068in}{0.694994in}}%
\pgfpathlineto{\pgfqpoint{1.833670in}{0.687332in}}%
\pgfpathlineto{\pgfqpoint{1.833871in}{0.696227in}}%
\pgfpathlineto{\pgfqpoint{1.834272in}{0.691874in}}%
\pgfpathlineto{\pgfqpoint{1.834574in}{0.702978in}}%
\pgfpathlineto{\pgfqpoint{1.834975in}{0.688066in}}%
\pgfpathlineto{\pgfqpoint{1.835075in}{0.691047in}}%
\pgfpathlineto{\pgfqpoint{1.835176in}{0.684912in}}%
\pgfpathlineto{\pgfqpoint{1.835778in}{0.713402in}}%
\pgfpathlineto{\pgfqpoint{1.836982in}{0.674772in}}%
\pgfpathlineto{\pgfqpoint{1.837685in}{0.686077in}}%
\pgfpathlineto{\pgfqpoint{1.839592in}{0.740562in}}%
\pgfpathlineto{\pgfqpoint{1.839792in}{0.738971in}}%
\pgfpathlineto{\pgfqpoint{1.839893in}{0.737864in}}%
\pgfpathlineto{\pgfqpoint{1.840094in}{0.743016in}}%
\pgfpathlineto{\pgfqpoint{1.840194in}{0.742850in}}%
\pgfpathlineto{\pgfqpoint{1.841398in}{0.770216in}}%
\pgfpathlineto{\pgfqpoint{1.840395in}{0.735155in}}%
\pgfpathlineto{\pgfqpoint{1.841499in}{0.759000in}}%
\pgfpathlineto{\pgfqpoint{1.841599in}{0.755357in}}%
\pgfpathlineto{\pgfqpoint{1.841900in}{0.771624in}}%
\pgfpathlineto{\pgfqpoint{1.842402in}{0.766333in}}%
\pgfpathlineto{\pgfqpoint{1.842703in}{0.767339in}}%
\pgfpathlineto{\pgfqpoint{1.842603in}{0.762889in}}%
\pgfpathlineto{\pgfqpoint{1.842904in}{0.763979in}}%
\pgfpathlineto{\pgfqpoint{1.844209in}{0.716145in}}%
\pgfpathlineto{\pgfqpoint{1.844409in}{0.726864in}}%
\pgfpathlineto{\pgfqpoint{1.845313in}{0.746301in}}%
\pgfpathlineto{\pgfqpoint{1.844911in}{0.725276in}}%
\pgfpathlineto{\pgfqpoint{1.845614in}{0.735815in}}%
\pgfpathlineto{\pgfqpoint{1.846316in}{0.718291in}}%
\pgfpathlineto{\pgfqpoint{1.846818in}{0.732425in}}%
\pgfpathlineto{\pgfqpoint{1.847521in}{0.759319in}}%
\pgfpathlineto{\pgfqpoint{1.848123in}{0.752183in}}%
\pgfpathlineto{\pgfqpoint{1.848424in}{0.738122in}}%
\pgfpathlineto{\pgfqpoint{1.849126in}{0.746705in}}%
\pgfpathlineto{\pgfqpoint{1.849829in}{0.767381in}}%
\pgfpathlineto{\pgfqpoint{1.850331in}{0.760331in}}%
\pgfpathlineto{\pgfqpoint{1.850431in}{0.759718in}}%
\pgfpathlineto{\pgfqpoint{1.850531in}{0.764329in}}%
\pgfpathlineto{\pgfqpoint{1.850732in}{0.769756in}}%
\pgfpathlineto{\pgfqpoint{1.851033in}{0.746576in}}%
\pgfpathlineto{\pgfqpoint{1.851234in}{0.757775in}}%
\pgfpathlineto{\pgfqpoint{1.851535in}{0.748337in}}%
\pgfpathlineto{\pgfqpoint{1.852037in}{0.767954in}}%
\pgfpathlineto{\pgfqpoint{1.852238in}{0.772633in}}%
\pgfpathlineto{\pgfqpoint{1.853141in}{0.745047in}}%
\pgfpathlineto{\pgfqpoint{1.853442in}{0.754411in}}%
\pgfpathlineto{\pgfqpoint{1.853542in}{0.761291in}}%
\pgfpathlineto{\pgfqpoint{1.854245in}{0.739919in}}%
\pgfpathlineto{\pgfqpoint{1.854747in}{0.719653in}}%
\pgfpathlineto{\pgfqpoint{1.855249in}{0.737337in}}%
\pgfpathlineto{\pgfqpoint{1.855550in}{0.752743in}}%
\pgfpathlineto{\pgfqpoint{1.855851in}{0.735961in}}%
\pgfpathlineto{\pgfqpoint{1.856252in}{0.737396in}}%
\pgfpathlineto{\pgfqpoint{1.857356in}{0.721261in}}%
\pgfpathlineto{\pgfqpoint{1.857457in}{0.729540in}}%
\pgfpathlineto{\pgfqpoint{1.858761in}{0.755032in}}%
\pgfpathlineto{\pgfqpoint{1.859062in}{0.751350in}}%
\pgfpathlineto{\pgfqpoint{1.859765in}{0.731365in}}%
\pgfpathlineto{\pgfqpoint{1.860367in}{0.745695in}}%
\pgfpathlineto{\pgfqpoint{1.861772in}{0.723657in}}%
\pgfpathlineto{\pgfqpoint{1.860769in}{0.748173in}}%
\pgfpathlineto{\pgfqpoint{1.862073in}{0.727164in}}%
\pgfpathlineto{\pgfqpoint{1.862475in}{0.735963in}}%
\pgfpathlineto{\pgfqpoint{1.862274in}{0.724028in}}%
\pgfpathlineto{\pgfqpoint{1.862776in}{0.728737in}}%
\pgfpathlineto{\pgfqpoint{1.863880in}{0.714015in}}%
\pgfpathlineto{\pgfqpoint{1.863378in}{0.731139in}}%
\pgfpathlineto{\pgfqpoint{1.863980in}{0.715595in}}%
\pgfpathlineto{\pgfqpoint{1.865185in}{0.704776in}}%
\pgfpathlineto{\pgfqpoint{1.865285in}{0.715501in}}%
\pgfpathlineto{\pgfqpoint{1.865988in}{0.696554in}}%
\pgfpathlineto{\pgfqpoint{1.866188in}{0.705371in}}%
\pgfpathlineto{\pgfqpoint{1.867593in}{0.667479in}}%
\pgfpathlineto{\pgfqpoint{1.867794in}{0.672936in}}%
\pgfpathlineto{\pgfqpoint{1.868898in}{0.692906in}}%
\pgfpathlineto{\pgfqpoint{1.869099in}{0.685284in}}%
\pgfpathlineto{\pgfqpoint{1.869199in}{0.682687in}}%
\pgfpathlineto{\pgfqpoint{1.869500in}{0.696864in}}%
\pgfpathlineto{\pgfqpoint{1.869701in}{0.696259in}}%
\pgfpathlineto{\pgfqpoint{1.870805in}{0.732686in}}%
\pgfpathlineto{\pgfqpoint{1.870002in}{0.694034in}}%
\pgfpathlineto{\pgfqpoint{1.871207in}{0.721055in}}%
\pgfpathlineto{\pgfqpoint{1.872612in}{0.677657in}}%
\pgfpathlineto{\pgfqpoint{1.872812in}{0.678313in}}%
\pgfpathlineto{\pgfqpoint{1.873013in}{0.688500in}}%
\pgfpathlineto{\pgfqpoint{1.873615in}{0.655041in}}%
\pgfpathlineto{\pgfqpoint{1.874519in}{0.631665in}}%
\pgfpathlineto{\pgfqpoint{1.874117in}{0.655100in}}%
\pgfpathlineto{\pgfqpoint{1.874820in}{0.639272in}}%
\pgfpathlineto{\pgfqpoint{1.875121in}{0.666453in}}%
\pgfpathlineto{\pgfqpoint{1.876024in}{0.648542in}}%
\pgfpathlineto{\pgfqpoint{1.877228in}{0.634625in}}%
\pgfpathlineto{\pgfqpoint{1.876225in}{0.649748in}}%
\pgfpathlineto{\pgfqpoint{1.877329in}{0.638783in}}%
\pgfpathlineto{\pgfqpoint{1.877429in}{0.635369in}}%
\pgfpathlineto{\pgfqpoint{1.878132in}{0.641106in}}%
\pgfpathlineto{\pgfqpoint{1.878332in}{0.640748in}}%
\pgfpathlineto{\pgfqpoint{1.878935in}{0.664418in}}%
\pgfpathlineto{\pgfqpoint{1.879838in}{0.658768in}}%
\pgfpathlineto{\pgfqpoint{1.879938in}{0.646433in}}%
\pgfpathlineto{\pgfqpoint{1.880841in}{0.668502in}}%
\pgfpathlineto{\pgfqpoint{1.881444in}{0.690426in}}%
\pgfpathlineto{\pgfqpoint{1.882347in}{0.679880in}}%
\pgfpathlineto{\pgfqpoint{1.882447in}{0.676690in}}%
\pgfpathlineto{\pgfqpoint{1.883049in}{0.684930in}}%
\pgfpathlineto{\pgfqpoint{1.884354in}{0.713355in}}%
\pgfpathlineto{\pgfqpoint{1.884455in}{0.709093in}}%
\pgfpathlineto{\pgfqpoint{1.884555in}{0.708787in}}%
\pgfpathlineto{\pgfqpoint{1.884655in}{0.704538in}}%
\pgfpathlineto{\pgfqpoint{1.885057in}{0.726214in}}%
\pgfpathlineto{\pgfqpoint{1.885559in}{0.710048in}}%
\pgfpathlineto{\pgfqpoint{1.886161in}{0.723273in}}%
\pgfpathlineto{\pgfqpoint{1.886462in}{0.708996in}}%
\pgfpathlineto{\pgfqpoint{1.886663in}{0.717318in}}%
\pgfpathlineto{\pgfqpoint{1.886863in}{0.710412in}}%
\pgfpathlineto{\pgfqpoint{1.887566in}{0.726798in}}%
\pgfpathlineto{\pgfqpoint{1.888168in}{0.723956in}}%
\pgfpathlineto{\pgfqpoint{1.887967in}{0.732301in}}%
\pgfpathlineto{\pgfqpoint{1.888570in}{0.728416in}}%
\pgfpathlineto{\pgfqpoint{1.889372in}{0.748817in}}%
\pgfpathlineto{\pgfqpoint{1.889975in}{0.745694in}}%
\pgfpathlineto{\pgfqpoint{1.890376in}{0.734149in}}%
\pgfpathlineto{\pgfqpoint{1.890878in}{0.752059in}}%
\pgfpathlineto{\pgfqpoint{1.890978in}{0.756361in}}%
\pgfpathlineto{\pgfqpoint{1.891380in}{0.735912in}}%
\pgfpathlineto{\pgfqpoint{1.891681in}{0.739894in}}%
\pgfpathlineto{\pgfqpoint{1.892283in}{0.734401in}}%
\pgfpathlineto{\pgfqpoint{1.892584in}{0.739665in}}%
\pgfpathlineto{\pgfqpoint{1.893387in}{0.762604in}}%
\pgfpathlineto{\pgfqpoint{1.893588in}{0.748988in}}%
\pgfpathlineto{\pgfqpoint{1.894591in}{0.709633in}}%
\pgfpathlineto{\pgfqpoint{1.894892in}{0.724655in}}%
\pgfpathlineto{\pgfqpoint{1.895194in}{0.710071in}}%
\pgfpathlineto{\pgfqpoint{1.895695in}{0.727467in}}%
\pgfpathlineto{\pgfqpoint{1.897101in}{0.755620in}}%
\pgfpathlineto{\pgfqpoint{1.896097in}{0.723104in}}%
\pgfpathlineto{\pgfqpoint{1.897301in}{0.742559in}}%
\pgfpathlineto{\pgfqpoint{1.897402in}{0.742526in}}%
\pgfpathlineto{\pgfqpoint{1.897502in}{0.731114in}}%
\pgfpathlineto{\pgfqpoint{1.897703in}{0.744058in}}%
\pgfpathlineto{\pgfqpoint{1.898405in}{0.742412in}}%
\pgfpathlineto{\pgfqpoint{1.898807in}{0.753593in}}%
\pgfpathlineto{\pgfqpoint{1.899509in}{0.747181in}}%
\pgfpathlineto{\pgfqpoint{1.899810in}{0.743059in}}%
\pgfpathlineto{\pgfqpoint{1.900914in}{0.782632in}}%
\pgfpathlineto{\pgfqpoint{1.901215in}{0.780117in}}%
\pgfpathlineto{\pgfqpoint{1.901617in}{0.795183in}}%
\pgfpathlineto{\pgfqpoint{1.901416in}{0.778157in}}%
\pgfpathlineto{\pgfqpoint{1.902319in}{0.779832in}}%
\pgfpathlineto{\pgfqpoint{1.902420in}{0.779360in}}%
\pgfpathlineto{\pgfqpoint{1.902520in}{0.785232in}}%
\pgfpathlineto{\pgfqpoint{1.903223in}{0.766511in}}%
\pgfpathlineto{\pgfqpoint{1.903624in}{0.749612in}}%
\pgfpathlineto{\pgfqpoint{1.904327in}{0.762280in}}%
\pgfpathlineto{\pgfqpoint{1.905431in}{0.749287in}}%
\pgfpathlineto{\pgfqpoint{1.904929in}{0.771085in}}%
\pgfpathlineto{\pgfqpoint{1.905531in}{0.755235in}}%
\pgfpathlineto{\pgfqpoint{1.906635in}{0.779445in}}%
\pgfpathlineto{\pgfqpoint{1.906735in}{0.775017in}}%
\pgfpathlineto{\pgfqpoint{1.906836in}{0.771421in}}%
\pgfpathlineto{\pgfqpoint{1.907237in}{0.793453in}}%
\pgfpathlineto{\pgfqpoint{1.907739in}{0.791024in}}%
\pgfpathlineto{\pgfqpoint{1.908442in}{0.819363in}}%
\pgfpathlineto{\pgfqpoint{1.909947in}{0.773047in}}%
\pgfpathlineto{\pgfqpoint{1.910248in}{0.781143in}}%
\pgfpathlineto{\pgfqpoint{1.910750in}{0.772738in}}%
\pgfpathlineto{\pgfqpoint{1.911051in}{0.778086in}}%
\pgfpathlineto{\pgfqpoint{1.911653in}{0.748747in}}%
\pgfpathlineto{\pgfqpoint{1.912456in}{0.765346in}}%
\pgfpathlineto{\pgfqpoint{1.913560in}{0.773107in}}%
\pgfpathlineto{\pgfqpoint{1.912958in}{0.759213in}}%
\pgfpathlineto{\pgfqpoint{1.913661in}{0.768075in}}%
\pgfpathlineto{\pgfqpoint{1.914062in}{0.755216in}}%
\pgfpathlineto{\pgfqpoint{1.914464in}{0.773829in}}%
\pgfpathlineto{\pgfqpoint{1.914765in}{0.766230in}}%
\pgfpathlineto{\pgfqpoint{1.914965in}{0.764092in}}%
\pgfpathlineto{\pgfqpoint{1.915266in}{0.753262in}}%
\pgfpathlineto{\pgfqpoint{1.915568in}{0.778430in}}%
\pgfpathlineto{\pgfqpoint{1.915869in}{0.769820in}}%
\pgfpathlineto{\pgfqpoint{1.915969in}{0.779243in}}%
\pgfpathlineto{\pgfqpoint{1.916772in}{0.766903in}}%
\pgfpathlineto{\pgfqpoint{1.916973in}{0.771996in}}%
\pgfpathlineto{\pgfqpoint{1.917274in}{0.753326in}}%
\pgfpathlineto{\pgfqpoint{1.918077in}{0.767618in}}%
\pgfpathlineto{\pgfqpoint{1.918177in}{0.774398in}}%
\pgfpathlineto{\pgfqpoint{1.918880in}{0.753536in}}%
\pgfpathlineto{\pgfqpoint{1.919281in}{0.747498in}}%
\pgfpathlineto{\pgfqpoint{1.919482in}{0.762744in}}%
\pgfpathlineto{\pgfqpoint{1.919984in}{0.752193in}}%
\pgfpathlineto{\pgfqpoint{1.921088in}{0.738423in}}%
\pgfpathlineto{\pgfqpoint{1.920385in}{0.757395in}}%
\pgfpathlineto{\pgfqpoint{1.921489in}{0.745937in}}%
\pgfpathlineto{\pgfqpoint{1.921589in}{0.748837in}}%
\pgfpathlineto{\pgfqpoint{1.922192in}{0.734666in}}%
\pgfpathlineto{\pgfqpoint{1.922292in}{0.734411in}}%
\pgfpathlineto{\pgfqpoint{1.922392in}{0.726295in}}%
\pgfpathlineto{\pgfqpoint{1.923195in}{0.740469in}}%
\pgfpathlineto{\pgfqpoint{1.923898in}{0.758419in}}%
\pgfpathlineto{\pgfqpoint{1.924400in}{0.747291in}}%
\pgfpathlineto{\pgfqpoint{1.924901in}{0.740978in}}%
\pgfpathlineto{\pgfqpoint{1.925102in}{0.749672in}}%
\pgfpathlineto{\pgfqpoint{1.925203in}{0.749617in}}%
\pgfpathlineto{\pgfqpoint{1.925504in}{0.758911in}}%
\pgfpathlineto{\pgfqpoint{1.925905in}{0.743280in}}%
\pgfpathlineto{\pgfqpoint{1.926206in}{0.751097in}}%
\pgfpathlineto{\pgfqpoint{1.927611in}{0.725699in}}%
\pgfpathlineto{\pgfqpoint{1.928213in}{0.747365in}}%
\pgfpathlineto{\pgfqpoint{1.928916in}{0.735580in}}%
\pgfpathlineto{\pgfqpoint{1.929016in}{0.729924in}}%
\pgfpathlineto{\pgfqpoint{1.929719in}{0.745274in}}%
\pgfpathlineto{\pgfqpoint{1.929920in}{0.733464in}}%
\pgfpathlineto{\pgfqpoint{1.930120in}{0.738890in}}%
\pgfpathlineto{\pgfqpoint{1.930622in}{0.719031in}}%
\pgfpathlineto{\pgfqpoint{1.933332in}{0.667968in}}%
\pgfpathlineto{\pgfqpoint{1.934637in}{0.649413in}}%
\pgfpathlineto{\pgfqpoint{1.935841in}{0.678035in}}%
\pgfpathlineto{\pgfqpoint{1.935942in}{0.673828in}}%
\pgfpathlineto{\pgfqpoint{1.936042in}{0.673934in}}%
\pgfpathlineto{\pgfqpoint{1.937547in}{0.699553in}}%
\pgfpathlineto{\pgfqpoint{1.937648in}{0.691482in}}%
\pgfpathlineto{\pgfqpoint{1.938350in}{0.717888in}}%
\pgfpathlineto{\pgfqpoint{1.938451in}{0.715335in}}%
\pgfpathlineto{\pgfqpoint{1.938551in}{0.715185in}}%
\pgfpathlineto{\pgfqpoint{1.939354in}{0.692275in}}%
\pgfpathlineto{\pgfqpoint{1.939956in}{0.703502in}}%
\pgfpathlineto{\pgfqpoint{1.941462in}{0.729125in}}%
\pgfpathlineto{\pgfqpoint{1.941562in}{0.721911in}}%
\pgfpathlineto{\pgfqpoint{1.942164in}{0.751812in}}%
\pgfpathlineto{\pgfqpoint{1.942365in}{0.757566in}}%
\pgfpathlineto{\pgfqpoint{1.942867in}{0.746704in}}%
\pgfpathlineto{\pgfqpoint{1.943870in}{0.725621in}}%
\pgfpathlineto{\pgfqpoint{1.944171in}{0.733529in}}%
\pgfpathlineto{\pgfqpoint{1.944673in}{0.747060in}}%
\pgfpathlineto{\pgfqpoint{1.945175in}{0.728660in}}%
\pgfpathlineto{\pgfqpoint{1.945275in}{0.734850in}}%
\pgfpathlineto{\pgfqpoint{1.945878in}{0.727593in}}%
\pgfpathlineto{\pgfqpoint{1.945978in}{0.733447in}}%
\pgfpathlineto{\pgfqpoint{1.946078in}{0.740962in}}%
\pgfpathlineto{\pgfqpoint{1.946781in}{0.722484in}}%
\pgfpathlineto{\pgfqpoint{1.946982in}{0.730139in}}%
\pgfpathlineto{\pgfqpoint{1.947182in}{0.719816in}}%
\pgfpathlineto{\pgfqpoint{1.947584in}{0.744473in}}%
\pgfpathlineto{\pgfqpoint{1.947684in}{0.743816in}}%
\pgfpathlineto{\pgfqpoint{1.947885in}{0.749236in}}%
\pgfpathlineto{\pgfqpoint{1.947985in}{0.749718in}}%
\pgfpathlineto{\pgfqpoint{1.948086in}{0.749118in}}%
\pgfpathlineto{\pgfqpoint{1.948788in}{0.733083in}}%
\pgfpathlineto{\pgfqpoint{1.948286in}{0.751710in}}%
\pgfpathlineto{\pgfqpoint{1.949190in}{0.744658in}}%
\pgfpathlineto{\pgfqpoint{1.949390in}{0.747074in}}%
\pgfpathlineto{\pgfqpoint{1.949691in}{0.733110in}}%
\pgfpathlineto{\pgfqpoint{1.950394in}{0.750354in}}%
\pgfpathlineto{\pgfqpoint{1.950494in}{0.742174in}}%
\pgfpathlineto{\pgfqpoint{1.951097in}{0.766889in}}%
\pgfpathlineto{\pgfqpoint{1.951799in}{0.760123in}}%
\pgfpathlineto{\pgfqpoint{1.952201in}{0.746698in}}%
\pgfpathlineto{\pgfqpoint{1.952602in}{0.761692in}}%
\pgfpathlineto{\pgfqpoint{1.952903in}{0.759968in}}%
\pgfpathlineto{\pgfqpoint{1.954208in}{0.783269in}}%
\pgfpathlineto{\pgfqpoint{1.955312in}{0.753648in}}%
\pgfpathlineto{\pgfqpoint{1.955412in}{0.758798in}}%
\pgfpathlineto{\pgfqpoint{1.956014in}{0.790824in}}%
\pgfpathlineto{\pgfqpoint{1.956918in}{0.776254in}}%
\pgfpathlineto{\pgfqpoint{1.957319in}{0.762129in}}%
\pgfpathlineto{\pgfqpoint{1.957921in}{0.776917in}}%
\pgfpathlineto{\pgfqpoint{1.958022in}{0.773673in}}%
\pgfpathlineto{\pgfqpoint{1.958323in}{0.771115in}}%
\pgfpathlineto{\pgfqpoint{1.958423in}{0.777787in}}%
\pgfpathlineto{\pgfqpoint{1.959126in}{0.786468in}}%
\pgfpathlineto{\pgfqpoint{1.959427in}{0.775332in}}%
\pgfpathlineto{\pgfqpoint{1.959527in}{0.780291in}}%
\pgfpathlineto{\pgfqpoint{1.960631in}{0.764900in}}%
\pgfpathlineto{\pgfqpoint{1.959929in}{0.785763in}}%
\pgfpathlineto{\pgfqpoint{1.960832in}{0.765426in}}%
\pgfpathlineto{\pgfqpoint{1.961133in}{0.786590in}}%
\pgfpathlineto{\pgfqpoint{1.961936in}{0.769006in}}%
\pgfpathlineto{\pgfqpoint{1.962438in}{0.757073in}}%
\pgfpathlineto{\pgfqpoint{1.963040in}{0.758745in}}%
\pgfpathlineto{\pgfqpoint{1.963341in}{0.763899in}}%
\pgfpathlineto{\pgfqpoint{1.963441in}{0.757930in}}%
\pgfpathlineto{\pgfqpoint{1.963843in}{0.761178in}}%
\pgfpathlineto{\pgfqpoint{1.964746in}{0.733881in}}%
\pgfpathlineto{\pgfqpoint{1.965248in}{0.736330in}}%
\pgfpathlineto{\pgfqpoint{1.965549in}{0.738778in}}%
\pgfpathlineto{\pgfqpoint{1.965449in}{0.733593in}}%
\pgfpathlineto{\pgfqpoint{1.965649in}{0.736504in}}%
\pgfpathlineto{\pgfqpoint{1.966352in}{0.721015in}}%
\pgfpathlineto{\pgfqpoint{1.966753in}{0.729008in}}%
\pgfpathlineto{\pgfqpoint{1.966854in}{0.731694in}}%
\pgfpathlineto{\pgfqpoint{1.967155in}{0.713413in}}%
\pgfpathlineto{\pgfqpoint{1.967255in}{0.707010in}}%
\pgfpathlineto{\pgfqpoint{1.967857in}{0.732004in}}%
\pgfpathlineto{\pgfqpoint{1.967958in}{0.730533in}}%
\pgfpathlineto{\pgfqpoint{1.968158in}{0.732912in}}%
\pgfpathlineto{\pgfqpoint{1.968460in}{0.721418in}}%
\pgfpathlineto{\pgfqpoint{1.969664in}{0.685749in}}%
\pgfpathlineto{\pgfqpoint{1.970065in}{0.689673in}}%
\pgfpathlineto{\pgfqpoint{1.970366in}{0.687920in}}%
\pgfpathlineto{\pgfqpoint{1.971571in}{0.712104in}}%
\pgfpathlineto{\pgfqpoint{1.971671in}{0.704041in}}%
\pgfpathlineto{\pgfqpoint{1.972374in}{0.715238in}}%
\pgfpathlineto{\pgfqpoint{1.972574in}{0.714283in}}%
\pgfpathlineto{\pgfqpoint{1.972876in}{0.723290in}}%
\pgfpathlineto{\pgfqpoint{1.973277in}{0.710320in}}%
\pgfpathlineto{\pgfqpoint{1.973678in}{0.716394in}}%
\pgfpathlineto{\pgfqpoint{1.973779in}{0.715188in}}%
\pgfpathlineto{\pgfqpoint{1.973980in}{0.722465in}}%
\pgfpathlineto{\pgfqpoint{1.974180in}{0.730790in}}%
\pgfpathlineto{\pgfqpoint{1.974782in}{0.710934in}}%
\pgfpathlineto{\pgfqpoint{1.974983in}{0.719410in}}%
\pgfpathlineto{\pgfqpoint{1.975887in}{0.687544in}}%
\pgfpathlineto{\pgfqpoint{1.976288in}{0.688004in}}%
\pgfpathlineto{\pgfqpoint{1.976388in}{0.696151in}}%
\pgfpathlineto{\pgfqpoint{1.977091in}{0.673669in}}%
\pgfpathlineto{\pgfqpoint{1.978195in}{0.653528in}}%
\pgfpathlineto{\pgfqpoint{1.977593in}{0.674830in}}%
\pgfpathlineto{\pgfqpoint{1.978596in}{0.655222in}}%
\pgfpathlineto{\pgfqpoint{1.980804in}{0.722710in}}%
\pgfpathlineto{\pgfqpoint{1.981005in}{0.710480in}}%
\pgfpathlineto{\pgfqpoint{1.981306in}{0.727149in}}%
\pgfpathlineto{\pgfqpoint{1.981607in}{0.726081in}}%
\pgfpathlineto{\pgfqpoint{1.983313in}{0.783409in}}%
\pgfpathlineto{\pgfqpoint{1.983514in}{0.775269in}}%
\pgfpathlineto{\pgfqpoint{1.984016in}{0.785323in}}%
\pgfpathlineto{\pgfqpoint{1.984217in}{0.784082in}}%
\pgfpathlineto{\pgfqpoint{1.984618in}{0.807522in}}%
\pgfpathlineto{\pgfqpoint{1.985521in}{0.805930in}}%
\pgfpathlineto{\pgfqpoint{1.985923in}{0.800738in}}%
\pgfpathlineto{\pgfqpoint{1.986124in}{0.810626in}}%
\pgfpathlineto{\pgfqpoint{1.986224in}{0.818855in}}%
\pgfpathlineto{\pgfqpoint{1.986927in}{0.802765in}}%
\pgfpathlineto{\pgfqpoint{1.987027in}{0.804657in}}%
\pgfpathlineto{\pgfqpoint{1.987529in}{0.779859in}}%
\pgfpathlineto{\pgfqpoint{1.988031in}{0.806176in}}%
\pgfpathlineto{\pgfqpoint{1.988131in}{0.802984in}}%
\pgfpathlineto{\pgfqpoint{1.988231in}{0.812195in}}%
\pgfpathlineto{\pgfqpoint{1.988934in}{0.796795in}}%
\pgfpathlineto{\pgfqpoint{1.989034in}{0.798642in}}%
\pgfpathlineto{\pgfqpoint{1.989436in}{0.781513in}}%
\pgfpathlineto{\pgfqpoint{1.990138in}{0.796827in}}%
\pgfpathlineto{\pgfqpoint{1.990339in}{0.790141in}}%
\pgfpathlineto{\pgfqpoint{1.990740in}{0.798528in}}%
\pgfpathlineto{\pgfqpoint{1.991142in}{0.797115in}}%
\pgfpathlineto{\pgfqpoint{1.991945in}{0.805871in}}%
\pgfpathlineto{\pgfqpoint{1.992045in}{0.796903in}}%
\pgfpathlineto{\pgfqpoint{1.992146in}{0.801044in}}%
\pgfpathlineto{\pgfqpoint{1.992346in}{0.804321in}}%
\pgfpathlineto{\pgfqpoint{1.993450in}{0.786445in}}%
\pgfpathlineto{\pgfqpoint{1.994354in}{0.821905in}}%
\pgfpathlineto{\pgfqpoint{1.994655in}{0.821325in}}%
\pgfpathlineto{\pgfqpoint{1.995056in}{0.812462in}}%
\pgfpathlineto{\pgfqpoint{1.995458in}{0.831673in}}%
\pgfpathlineto{\pgfqpoint{1.995658in}{0.827031in}}%
\pgfpathlineto{\pgfqpoint{1.996160in}{0.797684in}}%
\pgfpathlineto{\pgfqpoint{1.996863in}{0.820685in}}%
\pgfpathlineto{\pgfqpoint{1.997063in}{0.825193in}}%
\pgfpathlineto{\pgfqpoint{1.997164in}{0.818217in}}%
\pgfpathlineto{\pgfqpoint{1.998368in}{0.848467in}}%
\pgfpathlineto{\pgfqpoint{1.998569in}{0.840505in}}%
\pgfpathlineto{\pgfqpoint{1.998669in}{0.846085in}}%
\pgfpathlineto{\pgfqpoint{2.000175in}{0.889075in}}%
\pgfpathlineto{\pgfqpoint{2.001178in}{0.853328in}}%
\pgfpathlineto{\pgfqpoint{2.001379in}{0.866300in}}%
\pgfpathlineto{\pgfqpoint{2.001479in}{0.873131in}}%
\pgfpathlineto{\pgfqpoint{2.002182in}{0.853148in}}%
\pgfpathlineto{\pgfqpoint{2.002483in}{0.867473in}}%
\pgfpathlineto{\pgfqpoint{2.004089in}{0.891124in}}%
\pgfpathlineto{\pgfqpoint{2.005093in}{0.867957in}}%
\pgfpathlineto{\pgfqpoint{2.005394in}{0.875975in}}%
\pgfpathlineto{\pgfqpoint{2.005494in}{0.872479in}}%
\pgfpathlineto{\pgfqpoint{2.005895in}{0.891334in}}%
\pgfpathlineto{\pgfqpoint{2.006397in}{0.917084in}}%
\pgfpathlineto{\pgfqpoint{2.007100in}{0.896285in}}%
\pgfpathlineto{\pgfqpoint{2.007702in}{0.879869in}}%
\pgfpathlineto{\pgfqpoint{2.008204in}{0.887357in}}%
\pgfpathlineto{\pgfqpoint{2.009207in}{0.906566in}}%
\pgfpathlineto{\pgfqpoint{2.009408in}{0.900865in}}%
\pgfpathlineto{\pgfqpoint{2.009609in}{0.903455in}}%
\pgfpathlineto{\pgfqpoint{2.009709in}{0.899081in}}%
\pgfpathlineto{\pgfqpoint{2.009810in}{0.895094in}}%
\pgfpathlineto{\pgfqpoint{2.010211in}{0.922247in}}%
\pgfpathlineto{\pgfqpoint{2.010613in}{0.911362in}}%
\pgfpathlineto{\pgfqpoint{2.011717in}{0.873731in}}%
\pgfpathlineto{\pgfqpoint{2.011917in}{0.880993in}}%
\pgfpathlineto{\pgfqpoint{2.012118in}{0.889946in}}%
\pgfpathlineto{\pgfqpoint{2.012821in}{0.872313in}}%
\pgfpathlineto{\pgfqpoint{2.014426in}{0.847885in}}%
\pgfpathlineto{\pgfqpoint{2.015932in}{0.884101in}}%
\pgfpathlineto{\pgfqpoint{2.017437in}{0.847741in}}%
\pgfpathlineto{\pgfqpoint{2.018240in}{0.872011in}}%
\pgfpathlineto{\pgfqpoint{2.018642in}{0.858276in}}%
\pgfpathlineto{\pgfqpoint{2.019746in}{0.843651in}}%
\pgfpathlineto{\pgfqpoint{2.019244in}{0.860285in}}%
\pgfpathlineto{\pgfqpoint{2.019846in}{0.849146in}}%
\pgfpathlineto{\pgfqpoint{2.020047in}{0.853125in}}%
\pgfpathlineto{\pgfqpoint{2.020147in}{0.846437in}}%
\pgfpathlineto{\pgfqpoint{2.020248in}{0.852601in}}%
\pgfpathlineto{\pgfqpoint{2.020348in}{0.837709in}}%
\pgfpathlineto{\pgfqpoint{2.021151in}{0.870101in}}%
\pgfpathlineto{\pgfqpoint{2.021352in}{0.880902in}}%
\pgfpathlineto{\pgfqpoint{2.022154in}{0.867635in}}%
\pgfpathlineto{\pgfqpoint{2.022355in}{0.863897in}}%
\pgfpathlineto{\pgfqpoint{2.022957in}{0.870415in}}%
\pgfpathlineto{\pgfqpoint{2.023058in}{0.864569in}}%
\pgfpathlineto{\pgfqpoint{2.023760in}{0.885983in}}%
\pgfpathlineto{\pgfqpoint{2.023961in}{0.867093in}}%
\pgfpathlineto{\pgfqpoint{2.024463in}{0.859452in}}%
\pgfpathlineto{\pgfqpoint{2.025366in}{0.883386in}}%
\pgfpathlineto{\pgfqpoint{2.025567in}{0.884982in}}%
\pgfpathlineto{\pgfqpoint{2.025667in}{0.881829in}}%
\pgfpathlineto{\pgfqpoint{2.026370in}{0.875205in}}%
\pgfpathlineto{\pgfqpoint{2.026269in}{0.882752in}}%
\pgfpathlineto{\pgfqpoint{2.026671in}{0.881566in}}%
\pgfpathlineto{\pgfqpoint{2.027674in}{0.909602in}}%
\pgfpathlineto{\pgfqpoint{2.028176in}{0.899251in}}%
\pgfpathlineto{\pgfqpoint{2.028477in}{0.891014in}}%
\pgfpathlineto{\pgfqpoint{2.028578in}{0.905219in}}%
\pgfpathlineto{\pgfqpoint{2.029180in}{0.896247in}}%
\pgfpathlineto{\pgfqpoint{2.030284in}{0.938864in}}%
\pgfpathlineto{\pgfqpoint{2.029381in}{0.892691in}}%
\pgfpathlineto{\pgfqpoint{2.030786in}{0.914761in}}%
\pgfpathlineto{\pgfqpoint{2.032091in}{0.948578in}}%
\pgfpathlineto{\pgfqpoint{2.030987in}{0.913480in}}%
\pgfpathlineto{\pgfqpoint{2.032392in}{0.929067in}}%
\pgfpathlineto{\pgfqpoint{2.034299in}{0.885871in}}%
\pgfpathlineto{\pgfqpoint{2.032893in}{0.932579in}}%
\pgfpathlineto{\pgfqpoint{2.034399in}{0.894700in}}%
\pgfpathlineto{\pgfqpoint{2.034499in}{0.897326in}}%
\pgfpathlineto{\pgfqpoint{2.034700in}{0.880933in}}%
\pgfpathlineto{\pgfqpoint{2.034800in}{0.881224in}}%
\pgfpathlineto{\pgfqpoint{2.035001in}{0.860640in}}%
\pgfpathlineto{\pgfqpoint{2.036005in}{0.865588in}}%
\pgfpathlineto{\pgfqpoint{2.037510in}{0.914453in}}%
\pgfpathlineto{\pgfqpoint{2.037711in}{0.904887in}}%
\pgfpathlineto{\pgfqpoint{2.037811in}{0.905251in}}%
\pgfpathlineto{\pgfqpoint{2.038715in}{0.886798in}}%
\pgfpathlineto{\pgfqpoint{2.038413in}{0.907313in}}%
\pgfpathlineto{\pgfqpoint{2.039016in}{0.899318in}}%
\pgfpathlineto{\pgfqpoint{2.039618in}{0.913844in}}%
\pgfpathlineto{\pgfqpoint{2.039819in}{0.900447in}}%
\pgfpathlineto{\pgfqpoint{2.041123in}{0.867469in}}%
\pgfpathlineto{\pgfqpoint{2.041625in}{0.854113in}}%
\pgfpathlineto{\pgfqpoint{2.042528in}{0.892202in}}%
\pgfpathlineto{\pgfqpoint{2.042930in}{0.904846in}}%
\pgfpathlineto{\pgfqpoint{2.043331in}{0.901207in}}%
\pgfpathlineto{\pgfqpoint{2.044636in}{0.959089in}}%
\pgfpathlineto{\pgfqpoint{2.044736in}{0.959190in}}%
\pgfpathlineto{\pgfqpoint{2.045038in}{0.948503in}}%
\pgfpathlineto{\pgfqpoint{2.045539in}{0.975066in}}%
\pgfpathlineto{\pgfqpoint{2.045640in}{0.971533in}}%
\pgfpathlineto{\pgfqpoint{2.046543in}{0.994799in}}%
\pgfpathlineto{\pgfqpoint{2.046944in}{0.980533in}}%
\pgfpathlineto{\pgfqpoint{2.048350in}{0.950583in}}%
\pgfpathlineto{\pgfqpoint{2.048450in}{0.955581in}}%
\pgfpathlineto{\pgfqpoint{2.048751in}{0.951878in}}%
\pgfpathlineto{\pgfqpoint{2.050056in}{0.982411in}}%
\pgfpathlineto{\pgfqpoint{2.051461in}{0.950178in}}%
\pgfpathlineto{\pgfqpoint{2.050558in}{0.994802in}}%
\pgfpathlineto{\pgfqpoint{2.051561in}{0.953875in}}%
\pgfpathlineto{\pgfqpoint{2.052464in}{0.969942in}}%
\pgfpathlineto{\pgfqpoint{2.052766in}{0.965348in}}%
\pgfpathlineto{\pgfqpoint{2.052966in}{0.977185in}}%
\pgfpathlineto{\pgfqpoint{2.053669in}{0.949805in}}%
\pgfpathlineto{\pgfqpoint{2.053870in}{0.951886in}}%
\pgfpathlineto{\pgfqpoint{2.054673in}{0.931084in}}%
\pgfpathlineto{\pgfqpoint{2.055074in}{0.938580in}}%
\pgfpathlineto{\pgfqpoint{2.055174in}{0.938307in}}%
\pgfpathlineto{\pgfqpoint{2.055877in}{0.923587in}}%
\pgfpathlineto{\pgfqpoint{2.055375in}{0.940477in}}%
\pgfpathlineto{\pgfqpoint{2.056479in}{0.927868in}}%
\pgfpathlineto{\pgfqpoint{2.056680in}{0.939635in}}%
\pgfpathlineto{\pgfqpoint{2.056981in}{0.917298in}}%
\pgfpathlineto{\pgfqpoint{2.057483in}{0.927233in}}%
\pgfpathlineto{\pgfqpoint{2.057583in}{0.922365in}}%
\pgfpathlineto{\pgfqpoint{2.058085in}{0.947861in}}%
\pgfpathlineto{\pgfqpoint{2.058185in}{0.946450in}}%
\pgfpathlineto{\pgfqpoint{2.058587in}{0.935709in}}%
\pgfpathlineto{\pgfqpoint{2.058988in}{0.949305in}}%
\pgfpathlineto{\pgfqpoint{2.059089in}{0.949106in}}%
\pgfpathlineto{\pgfqpoint{2.061798in}{1.018820in}}%
\pgfpathlineto{\pgfqpoint{2.062601in}{1.010339in}}%
\pgfpathlineto{\pgfqpoint{2.062099in}{1.020762in}}%
\pgfpathlineto{\pgfqpoint{2.062902in}{1.017941in}}%
\pgfpathlineto{\pgfqpoint{2.064107in}{1.049947in}}%
\pgfpathlineto{\pgfqpoint{2.064307in}{1.049535in}}%
\pgfpathlineto{\pgfqpoint{2.064910in}{1.028936in}}%
\pgfpathlineto{\pgfqpoint{2.065010in}{1.024286in}}%
\pgfpathlineto{\pgfqpoint{2.065311in}{1.046036in}}%
\pgfpathlineto{\pgfqpoint{2.065813in}{1.034435in}}%
\pgfpathlineto{\pgfqpoint{2.066515in}{1.053944in}}%
\pgfpathlineto{\pgfqpoint{2.067017in}{1.040165in}}%
\pgfpathlineto{\pgfqpoint{2.067118in}{1.039918in}}%
\pgfpathlineto{\pgfqpoint{2.068623in}{0.993818in}}%
\pgfpathlineto{\pgfqpoint{2.068824in}{1.008572in}}%
\pgfpathlineto{\pgfqpoint{2.069326in}{1.035164in}}%
\pgfpathlineto{\pgfqpoint{2.069928in}{1.018124in}}%
\pgfpathlineto{\pgfqpoint{2.071333in}{0.975728in}}%
\pgfpathlineto{\pgfqpoint{2.071433in}{0.977570in}}%
\pgfpathlineto{\pgfqpoint{2.072838in}{1.037134in}}%
\pgfpathlineto{\pgfqpoint{2.072939in}{1.033610in}}%
\pgfpathlineto{\pgfqpoint{2.073641in}{1.047317in}}%
\pgfpathlineto{\pgfqpoint{2.073942in}{1.035447in}}%
\pgfpathlineto{\pgfqpoint{2.075348in}{1.050268in}}%
\pgfpathlineto{\pgfqpoint{2.076351in}{1.022422in}}%
\pgfpathlineto{\pgfqpoint{2.075649in}{1.053244in}}%
\pgfpathlineto{\pgfqpoint{2.077054in}{1.025128in}}%
\pgfpathlineto{\pgfqpoint{2.078057in}{1.057343in}}%
\pgfpathlineto{\pgfqpoint{2.078258in}{1.045215in}}%
\pgfpathlineto{\pgfqpoint{2.079362in}{1.032000in}}%
\pgfpathlineto{\pgfqpoint{2.078660in}{1.050331in}}%
\pgfpathlineto{\pgfqpoint{2.079663in}{1.034650in}}%
\pgfpathlineto{\pgfqpoint{2.080165in}{1.060689in}}%
\pgfpathlineto{\pgfqpoint{2.081169in}{1.054973in}}%
\pgfpathlineto{\pgfqpoint{2.082373in}{1.023699in}}%
\pgfpathlineto{\pgfqpoint{2.082674in}{1.032804in}}%
\pgfpathlineto{\pgfqpoint{2.082975in}{1.044575in}}%
\pgfpathlineto{\pgfqpoint{2.083377in}{1.031956in}}%
\pgfpathlineto{\pgfqpoint{2.083678in}{1.033571in}}%
\pgfpathlineto{\pgfqpoint{2.084180in}{1.016790in}}%
\pgfpathlineto{\pgfqpoint{2.084782in}{1.025211in}}%
\pgfpathlineto{\pgfqpoint{2.085183in}{1.036444in}}%
\pgfpathlineto{\pgfqpoint{2.085785in}{1.024370in}}%
\pgfpathlineto{\pgfqpoint{2.087391in}{1.000889in}}%
\pgfpathlineto{\pgfqpoint{2.087793in}{1.031005in}}%
\pgfpathlineto{\pgfqpoint{2.088395in}{0.995313in}}%
\pgfpathlineto{\pgfqpoint{2.088696in}{1.008680in}}%
\pgfpathlineto{\pgfqpoint{2.088997in}{0.985772in}}%
\pgfpathlineto{\pgfqpoint{2.089097in}{0.984221in}}%
\pgfpathlineto{\pgfqpoint{2.089198in}{0.988980in}}%
\pgfpathlineto{\pgfqpoint{2.089298in}{0.985749in}}%
\pgfpathlineto{\pgfqpoint{2.089399in}{0.997067in}}%
\pgfpathlineto{\pgfqpoint{2.089900in}{0.980356in}}%
\pgfpathlineto{\pgfqpoint{2.090402in}{0.990484in}}%
\pgfpathlineto{\pgfqpoint{2.090603in}{0.995796in}}%
\pgfpathlineto{\pgfqpoint{2.090804in}{0.978597in}}%
\pgfpathlineto{\pgfqpoint{2.091004in}{0.979624in}}%
\pgfpathlineto{\pgfqpoint{2.091607in}{0.957302in}}%
\pgfpathlineto{\pgfqpoint{2.091205in}{0.980665in}}%
\pgfpathlineto{\pgfqpoint{2.092108in}{0.972485in}}%
\pgfpathlineto{\pgfqpoint{2.093714in}{1.008697in}}%
\pgfpathlineto{\pgfqpoint{2.093815in}{1.006317in}}%
\pgfpathlineto{\pgfqpoint{2.094116in}{1.018199in}}%
\pgfpathlineto{\pgfqpoint{2.094316in}{1.034652in}}%
\pgfpathlineto{\pgfqpoint{2.094919in}{1.008583in}}%
\pgfpathlineto{\pgfqpoint{2.095119in}{1.018005in}}%
\pgfpathlineto{\pgfqpoint{2.097829in}{0.989581in}}%
\pgfpathlineto{\pgfqpoint{2.095621in}{1.020268in}}%
\pgfpathlineto{\pgfqpoint{2.098030in}{0.991938in}}%
\pgfpathlineto{\pgfqpoint{2.099636in}{1.047250in}}%
\pgfpathlineto{\pgfqpoint{2.099736in}{1.046015in}}%
\pgfpathlineto{\pgfqpoint{2.100238in}{1.042533in}}%
\pgfpathlineto{\pgfqpoint{2.100539in}{1.050892in}}%
\pgfpathlineto{\pgfqpoint{2.100940in}{1.056901in}}%
\pgfpathlineto{\pgfqpoint{2.101242in}{1.044273in}}%
\pgfpathlineto{\pgfqpoint{2.101342in}{1.043310in}}%
\pgfpathlineto{\pgfqpoint{2.101543in}{1.048740in}}%
\pgfpathlineto{\pgfqpoint{2.102948in}{1.099351in}}%
\pgfpathlineto{\pgfqpoint{2.103249in}{1.097655in}}%
\pgfpathlineto{\pgfqpoint{2.103851in}{1.111687in}}%
\pgfpathlineto{\pgfqpoint{2.104252in}{1.104015in}}%
\pgfpathlineto{\pgfqpoint{2.104353in}{1.092963in}}%
\pgfpathlineto{\pgfqpoint{2.104654in}{1.106830in}}%
\pgfpathlineto{\pgfqpoint{2.105356in}{1.104393in}}%
\pgfpathlineto{\pgfqpoint{2.106762in}{1.081741in}}%
\pgfpathlineto{\pgfqpoint{2.106862in}{1.090644in}}%
\pgfpathlineto{\pgfqpoint{2.107263in}{1.107930in}}%
\pgfpathlineto{\pgfqpoint{2.107866in}{1.082682in}}%
\pgfpathlineto{\pgfqpoint{2.108066in}{1.082893in}}%
\pgfpathlineto{\pgfqpoint{2.108367in}{1.076436in}}%
\pgfpathlineto{\pgfqpoint{2.108869in}{1.090395in}}%
\pgfpathlineto{\pgfqpoint{2.109271in}{1.098625in}}%
\pgfpathlineto{\pgfqpoint{2.109572in}{1.081329in}}%
\pgfpathlineto{\pgfqpoint{2.109672in}{1.082042in}}%
\pgfpathlineto{\pgfqpoint{2.110877in}{1.051909in}}%
\pgfpathlineto{\pgfqpoint{2.111378in}{1.076343in}}%
\pgfpathlineto{\pgfqpoint{2.112282in}{1.071192in}}%
\pgfpathlineto{\pgfqpoint{2.112783in}{1.082811in}}%
\pgfpathlineto{\pgfqpoint{2.112482in}{1.071179in}}%
\pgfpathlineto{\pgfqpoint{2.113285in}{1.073968in}}%
\pgfpathlineto{\pgfqpoint{2.114189in}{1.054790in}}%
\pgfpathlineto{\pgfqpoint{2.114389in}{1.073860in}}%
\pgfpathlineto{\pgfqpoint{2.115594in}{1.113772in}}%
\pgfpathlineto{\pgfqpoint{2.116196in}{1.103944in}}%
\pgfpathlineto{\pgfqpoint{2.117199in}{1.097563in}}%
\pgfpathlineto{\pgfqpoint{2.116698in}{1.107915in}}%
\pgfpathlineto{\pgfqpoint{2.117300in}{1.102161in}}%
\pgfpathlineto{\pgfqpoint{2.117701in}{1.108235in}}%
\pgfpathlineto{\pgfqpoint{2.117902in}{1.096964in}}%
\pgfpathlineto{\pgfqpoint{2.118203in}{1.101628in}}%
\pgfpathlineto{\pgfqpoint{2.119909in}{1.073236in}}%
\pgfpathlineto{\pgfqpoint{2.120210in}{1.065997in}}%
\pgfpathlineto{\pgfqpoint{2.121415in}{1.099281in}}%
\pgfpathlineto{\pgfqpoint{2.122619in}{1.082596in}}%
\pgfpathlineto{\pgfqpoint{2.122820in}{1.090298in}}%
\pgfpathlineto{\pgfqpoint{2.123824in}{1.100140in}}%
\pgfpathlineto{\pgfqpoint{2.123121in}{1.083281in}}%
\pgfpathlineto{\pgfqpoint{2.123924in}{1.090447in}}%
\pgfpathlineto{\pgfqpoint{2.124125in}{1.094568in}}%
\pgfpathlineto{\pgfqpoint{2.125229in}{1.070580in}}%
\pgfpathlineto{\pgfqpoint{2.126333in}{1.054948in}}%
\pgfpathlineto{\pgfqpoint{2.125831in}{1.072326in}}%
\pgfpathlineto{\pgfqpoint{2.126433in}{1.058391in}}%
\pgfpathlineto{\pgfqpoint{2.127537in}{1.083732in}}%
\pgfpathlineto{\pgfqpoint{2.126734in}{1.047716in}}%
\pgfpathlineto{\pgfqpoint{2.127938in}{1.079901in}}%
\pgfpathlineto{\pgfqpoint{2.128541in}{1.059550in}}%
\pgfpathlineto{\pgfqpoint{2.129042in}{1.076244in}}%
\pgfpathlineto{\pgfqpoint{2.130448in}{1.116116in}}%
\pgfpathlineto{\pgfqpoint{2.130849in}{1.103382in}}%
\pgfpathlineto{\pgfqpoint{2.131050in}{1.088841in}}%
\pgfpathlineto{\pgfqpoint{2.131552in}{1.105319in}}%
\pgfpathlineto{\pgfqpoint{2.131953in}{1.102838in}}%
\pgfpathlineto{\pgfqpoint{2.132455in}{1.109962in}}%
\pgfpathlineto{\pgfqpoint{2.133258in}{1.089960in}}%
\pgfpathlineto{\pgfqpoint{2.133358in}{1.097840in}}%
\pgfpathlineto{\pgfqpoint{2.134061in}{1.074125in}}%
\pgfpathlineto{\pgfqpoint{2.135165in}{1.056333in}}%
\pgfpathlineto{\pgfqpoint{2.136269in}{1.023966in}}%
\pgfpathlineto{\pgfqpoint{2.136369in}{1.030420in}}%
\pgfpathlineto{\pgfqpoint{2.136469in}{1.030749in}}%
\pgfpathlineto{\pgfqpoint{2.136971in}{1.008204in}}%
\pgfpathlineto{\pgfqpoint{2.137473in}{1.031997in}}%
\pgfpathlineto{\pgfqpoint{2.137573in}{1.020733in}}%
\pgfpathlineto{\pgfqpoint{2.139079in}{1.065547in}}%
\pgfpathlineto{\pgfqpoint{2.140083in}{1.091685in}}%
\pgfpathlineto{\pgfqpoint{2.140283in}{1.081695in}}%
\pgfpathlineto{\pgfqpoint{2.140785in}{1.083978in}}%
\pgfpathlineto{\pgfqpoint{2.141086in}{1.075848in}}%
\pgfpathlineto{\pgfqpoint{2.142190in}{1.104620in}}%
\pgfpathlineto{\pgfqpoint{2.141488in}{1.074613in}}%
\pgfpathlineto{\pgfqpoint{2.142291in}{1.098522in}}%
\pgfpathlineto{\pgfqpoint{2.142592in}{1.061258in}}%
\pgfpathlineto{\pgfqpoint{2.143495in}{1.075540in}}%
\pgfpathlineto{\pgfqpoint{2.143896in}{1.096178in}}%
\pgfpathlineto{\pgfqpoint{2.144800in}{1.091705in}}%
\pgfpathlineto{\pgfqpoint{2.145201in}{1.084460in}}%
\pgfpathlineto{\pgfqpoint{2.145502in}{1.098904in}}%
\pgfpathlineto{\pgfqpoint{2.145803in}{1.095644in}}%
\pgfpathlineto{\pgfqpoint{2.146606in}{1.100594in}}%
\pgfpathlineto{\pgfqpoint{2.146907in}{1.087489in}}%
\pgfpathlineto{\pgfqpoint{2.148011in}{1.105269in}}%
\pgfpathlineto{\pgfqpoint{2.147610in}{1.081977in}}%
\pgfpathlineto{\pgfqpoint{2.148112in}{1.105207in}}%
\pgfpathlineto{\pgfqpoint{2.149015in}{1.089045in}}%
\pgfpathlineto{\pgfqpoint{2.148513in}{1.111160in}}%
\pgfpathlineto{\pgfqpoint{2.149216in}{1.096463in}}%
\pgfpathlineto{\pgfqpoint{2.150119in}{1.112911in}}%
\pgfpathlineto{\pgfqpoint{2.149617in}{1.092735in}}%
\pgfpathlineto{\pgfqpoint{2.150320in}{1.099063in}}%
\pgfpathlineto{\pgfqpoint{2.150420in}{1.098863in}}%
\pgfpathlineto{\pgfqpoint{2.150621in}{1.105881in}}%
\pgfpathlineto{\pgfqpoint{2.151123in}{1.084384in}}%
\pgfpathlineto{\pgfqpoint{2.151524in}{1.099782in}}%
\pgfpathlineto{\pgfqpoint{2.151825in}{1.106495in}}%
\pgfpathlineto{\pgfqpoint{2.152126in}{1.088792in}}%
\pgfpathlineto{\pgfqpoint{2.152327in}{1.094156in}}%
\pgfpathlineto{\pgfqpoint{2.152427in}{1.094319in}}%
\pgfpathlineto{\pgfqpoint{2.153531in}{1.075441in}}%
\pgfpathlineto{\pgfqpoint{2.152829in}{1.097995in}}%
\pgfpathlineto{\pgfqpoint{2.153632in}{1.084929in}}%
\pgfpathlineto{\pgfqpoint{2.153933in}{1.098613in}}%
\pgfpathlineto{\pgfqpoint{2.154234in}{1.079855in}}%
\pgfpathlineto{\pgfqpoint{2.154836in}{1.093976in}}%
\pgfpathlineto{\pgfqpoint{2.155137in}{1.085284in}}%
\pgfpathlineto{\pgfqpoint{2.155840in}{1.099229in}}%
\pgfpathlineto{\pgfqpoint{2.158248in}{1.065333in}}%
\pgfpathlineto{\pgfqpoint{2.158750in}{1.074703in}}%
\pgfpathlineto{\pgfqpoint{2.158550in}{1.060294in}}%
\pgfpathlineto{\pgfqpoint{2.159152in}{1.063825in}}%
\pgfpathlineto{\pgfqpoint{2.159252in}{1.055954in}}%
\pgfpathlineto{\pgfqpoint{2.159854in}{1.080895in}}%
\pgfpathlineto{\pgfqpoint{2.160557in}{1.095506in}}%
\pgfpathlineto{\pgfqpoint{2.160958in}{1.086655in}}%
\pgfpathlineto{\pgfqpoint{2.162163in}{1.111714in}}%
\pgfpathlineto{\pgfqpoint{2.162464in}{1.097031in}}%
\pgfpathlineto{\pgfqpoint{2.163568in}{1.082929in}}%
\pgfpathlineto{\pgfqpoint{2.163769in}{1.087113in}}%
\pgfpathlineto{\pgfqpoint{2.165776in}{1.146002in}}%
\pgfpathlineto{\pgfqpoint{2.165876in}{1.145210in}}%
\pgfpathlineto{\pgfqpoint{2.166077in}{1.146191in}}%
\pgfpathlineto{\pgfqpoint{2.167482in}{1.114779in}}%
\pgfpathlineto{\pgfqpoint{2.167683in}{1.105755in}}%
\pgfpathlineto{\pgfqpoint{2.167984in}{1.118618in}}%
\pgfpathlineto{\pgfqpoint{2.168987in}{1.138986in}}%
\pgfpathlineto{\pgfqpoint{2.168385in}{1.109176in}}%
\pgfpathlineto{\pgfqpoint{2.169289in}{1.132597in}}%
\pgfpathlineto{\pgfqpoint{2.170091in}{1.120495in}}%
\pgfpathlineto{\pgfqpoint{2.170292in}{1.133827in}}%
\pgfpathlineto{\pgfqpoint{2.170393in}{1.126778in}}%
\pgfpathlineto{\pgfqpoint{2.170493in}{1.126326in}}%
\pgfpathlineto{\pgfqpoint{2.170694in}{1.142410in}}%
\pgfpathlineto{\pgfqpoint{2.171497in}{1.131600in}}%
\pgfpathlineto{\pgfqpoint{2.171697in}{1.124641in}}%
\pgfpathlineto{\pgfqpoint{2.172299in}{1.145390in}}%
\pgfpathlineto{\pgfqpoint{2.172400in}{1.145369in}}%
\pgfpathlineto{\pgfqpoint{2.172801in}{1.149193in}}%
\pgfpathlineto{\pgfqpoint{2.173102in}{1.142876in}}%
\pgfpathlineto{\pgfqpoint{2.173504in}{1.148533in}}%
\pgfpathlineto{\pgfqpoint{2.173805in}{1.147795in}}%
\pgfpathlineto{\pgfqpoint{2.173705in}{1.148853in}}%
\pgfpathlineto{\pgfqpoint{2.173905in}{1.147890in}}%
\pgfpathlineto{\pgfqpoint{2.174708in}{1.178981in}}%
\pgfpathlineto{\pgfqpoint{2.175110in}{1.168244in}}%
\pgfpathlineto{\pgfqpoint{2.176615in}{1.190431in}}%
\pgfpathlineto{\pgfqpoint{2.175511in}{1.155522in}}%
\pgfpathlineto{\pgfqpoint{2.176716in}{1.188826in}}%
\pgfpathlineto{\pgfqpoint{2.176816in}{1.184023in}}%
\pgfpathlineto{\pgfqpoint{2.177017in}{1.197485in}}%
\pgfpathlineto{\pgfqpoint{2.177820in}{1.187999in}}%
\pgfpathlineto{\pgfqpoint{2.178823in}{1.209592in}}%
\pgfpathlineto{\pgfqpoint{2.178321in}{1.181112in}}%
\pgfpathlineto{\pgfqpoint{2.179024in}{1.204910in}}%
\pgfpathlineto{\pgfqpoint{2.179225in}{1.198572in}}%
\pgfpathlineto{\pgfqpoint{2.179526in}{1.209820in}}%
\pgfpathlineto{\pgfqpoint{2.180128in}{1.203116in}}%
\pgfpathlineto{\pgfqpoint{2.180228in}{1.206026in}}%
\pgfpathlineto{\pgfqpoint{2.180429in}{1.192174in}}%
\pgfpathlineto{\pgfqpoint{2.181031in}{1.198695in}}%
\pgfpathlineto{\pgfqpoint{2.181934in}{1.168662in}}%
\pgfpathlineto{\pgfqpoint{2.182336in}{1.176660in}}%
\pgfpathlineto{\pgfqpoint{2.182436in}{1.181560in}}%
\pgfpathlineto{\pgfqpoint{2.182938in}{1.165494in}}%
\pgfpathlineto{\pgfqpoint{2.183038in}{1.154537in}}%
\pgfpathlineto{\pgfqpoint{2.183942in}{1.176704in}}%
\pgfpathlineto{\pgfqpoint{2.184745in}{1.186519in}}%
\pgfpathlineto{\pgfqpoint{2.184343in}{1.167896in}}%
\pgfpathlineto{\pgfqpoint{2.185046in}{1.179011in}}%
\pgfpathlineto{\pgfqpoint{2.186451in}{1.124880in}}%
\pgfpathlineto{\pgfqpoint{2.186652in}{1.132206in}}%
\pgfpathlineto{\pgfqpoint{2.188458in}{1.160295in}}%
\pgfpathlineto{\pgfqpoint{2.188759in}{1.150825in}}%
\pgfpathlineto{\pgfqpoint{2.189161in}{1.139608in}}%
\pgfpathlineto{\pgfqpoint{2.189663in}{1.152481in}}%
\pgfpathlineto{\pgfqpoint{2.189863in}{1.150901in}}%
\pgfpathlineto{\pgfqpoint{2.190064in}{1.154693in}}%
\pgfpathlineto{\pgfqpoint{2.190365in}{1.142607in}}%
\pgfpathlineto{\pgfqpoint{2.190566in}{1.133030in}}%
\pgfpathlineto{\pgfqpoint{2.191268in}{1.154213in}}%
\pgfpathlineto{\pgfqpoint{2.191369in}{1.157255in}}%
\pgfpathlineto{\pgfqpoint{2.191770in}{1.141685in}}%
\pgfpathlineto{\pgfqpoint{2.192975in}{1.128648in}}%
\pgfpathlineto{\pgfqpoint{2.193075in}{1.133364in}}%
\pgfpathlineto{\pgfqpoint{2.193476in}{1.122707in}}%
\pgfpathlineto{\pgfqpoint{2.194380in}{1.097905in}}%
\pgfpathlineto{\pgfqpoint{2.194681in}{1.101691in}}%
\pgfpathlineto{\pgfqpoint{2.195283in}{1.124750in}}%
\pgfpathlineto{\pgfqpoint{2.195785in}{1.103640in}}%
\pgfpathlineto{\pgfqpoint{2.195885in}{1.102272in}}%
\pgfpathlineto{\pgfqpoint{2.196086in}{1.112266in}}%
\pgfpathlineto{\pgfqpoint{2.196186in}{1.113656in}}%
\pgfpathlineto{\pgfqpoint{2.196387in}{1.108050in}}%
\pgfpathlineto{\pgfqpoint{2.196487in}{1.096532in}}%
\pgfpathlineto{\pgfqpoint{2.197290in}{1.115353in}}%
\pgfpathlineto{\pgfqpoint{2.199498in}{1.173747in}}%
\pgfpathlineto{\pgfqpoint{2.200703in}{1.161319in}}%
\pgfpathlineto{\pgfqpoint{2.200803in}{1.165842in}}%
\pgfpathlineto{\pgfqpoint{2.201104in}{1.179632in}}%
\pgfpathlineto{\pgfqpoint{2.201204in}{1.176723in}}%
\pgfpathlineto{\pgfqpoint{2.201807in}{1.207482in}}%
\pgfpathlineto{\pgfqpoint{2.202409in}{1.193743in}}%
\pgfpathlineto{\pgfqpoint{2.203111in}{1.205482in}}%
\pgfpathlineto{\pgfqpoint{2.203312in}{1.191342in}}%
\pgfpathlineto{\pgfqpoint{2.203412in}{1.187504in}}%
\pgfpathlineto{\pgfqpoint{2.203914in}{1.205864in}}%
\pgfpathlineto{\pgfqpoint{2.204818in}{1.222758in}}%
\pgfpathlineto{\pgfqpoint{2.204918in}{1.207254in}}%
\pgfpathlineto{\pgfqpoint{2.206022in}{1.180371in}}%
\pgfpathlineto{\pgfqpoint{2.206122in}{1.189832in}}%
\pgfpathlineto{\pgfqpoint{2.206524in}{1.202611in}}%
\pgfpathlineto{\pgfqpoint{2.207026in}{1.182063in}}%
\pgfpathlineto{\pgfqpoint{2.207126in}{1.182572in}}%
\pgfpathlineto{\pgfqpoint{2.208431in}{1.149085in}}%
\pgfpathlineto{\pgfqpoint{2.208932in}{1.136411in}}%
\pgfpathlineto{\pgfqpoint{2.208631in}{1.155742in}}%
\pgfpathlineto{\pgfqpoint{2.209535in}{1.145687in}}%
\pgfpathlineto{\pgfqpoint{2.209735in}{1.162066in}}%
\pgfpathlineto{\pgfqpoint{2.210538in}{1.143487in}}%
\pgfpathlineto{\pgfqpoint{2.210639in}{1.151491in}}%
\pgfpathlineto{\pgfqpoint{2.211341in}{1.117793in}}%
\pgfpathlineto{\pgfqpoint{2.211943in}{1.133789in}}%
\pgfpathlineto{\pgfqpoint{2.212144in}{1.137392in}}%
\pgfpathlineto{\pgfqpoint{2.213148in}{1.105200in}}%
\pgfpathlineto{\pgfqpoint{2.214453in}{1.069473in}}%
\pgfpathlineto{\pgfqpoint{2.214954in}{1.072189in}}%
\pgfpathlineto{\pgfqpoint{2.215255in}{1.087900in}}%
\pgfpathlineto{\pgfqpoint{2.215757in}{1.071718in}}%
\pgfpathlineto{\pgfqpoint{2.216058in}{1.072856in}}%
\pgfpathlineto{\pgfqpoint{2.217062in}{1.089729in}}%
\pgfpathlineto{\pgfqpoint{2.217263in}{1.084225in}}%
\pgfpathlineto{\pgfqpoint{2.217664in}{1.078039in}}%
\pgfpathlineto{\pgfqpoint{2.217965in}{1.091889in}}%
\pgfpathlineto{\pgfqpoint{2.218166in}{1.091477in}}%
\pgfpathlineto{\pgfqpoint{2.218266in}{1.090694in}}%
\pgfpathlineto{\pgfqpoint{2.218367in}{1.096068in}}%
\pgfpathlineto{\pgfqpoint{2.218467in}{1.096027in}}%
\pgfpathlineto{\pgfqpoint{2.218969in}{1.100998in}}%
\pgfpathlineto{\pgfqpoint{2.219370in}{1.091347in}}%
\pgfpathlineto{\pgfqpoint{2.219471in}{1.095162in}}%
\pgfpathlineto{\pgfqpoint{2.219973in}{1.093327in}}%
\pgfpathlineto{\pgfqpoint{2.219671in}{1.100851in}}%
\pgfpathlineto{\pgfqpoint{2.220274in}{1.094646in}}%
\pgfpathlineto{\pgfqpoint{2.221077in}{1.086853in}}%
\pgfpathlineto{\pgfqpoint{2.221478in}{1.106868in}}%
\pgfpathlineto{\pgfqpoint{2.222181in}{1.102304in}}%
\pgfpathlineto{\pgfqpoint{2.221779in}{1.115260in}}%
\pgfpathlineto{\pgfqpoint{2.222281in}{1.107246in}}%
\pgfpathlineto{\pgfqpoint{2.224087in}{1.154690in}}%
\pgfpathlineto{\pgfqpoint{2.224589in}{1.141358in}}%
\pgfpathlineto{\pgfqpoint{2.225191in}{1.153199in}}%
\pgfpathlineto{\pgfqpoint{2.225292in}{1.161697in}}%
\pgfpathlineto{\pgfqpoint{2.225794in}{1.139755in}}%
\pgfpathlineto{\pgfqpoint{2.225994in}{1.140542in}}%
\pgfpathlineto{\pgfqpoint{2.226898in}{1.128646in}}%
\pgfpathlineto{\pgfqpoint{2.226396in}{1.149557in}}%
\pgfpathlineto{\pgfqpoint{2.226998in}{1.132843in}}%
\pgfpathlineto{\pgfqpoint{2.228604in}{1.185692in}}%
\pgfpathlineto{\pgfqpoint{2.228805in}{1.176750in}}%
\pgfpathlineto{\pgfqpoint{2.229206in}{1.189697in}}%
\pgfpathlineto{\pgfqpoint{2.229507in}{1.194418in}}%
\pgfpathlineto{\pgfqpoint{2.229808in}{1.171623in}}%
\pgfpathlineto{\pgfqpoint{2.230310in}{1.167367in}}%
\pgfpathlineto{\pgfqpoint{2.230511in}{1.178007in}}%
\pgfpathlineto{\pgfqpoint{2.231113in}{1.193382in}}%
\pgfpathlineto{\pgfqpoint{2.230912in}{1.177175in}}%
\pgfpathlineto{\pgfqpoint{2.231514in}{1.182862in}}%
\pgfpathlineto{\pgfqpoint{2.233020in}{1.129446in}}%
\pgfpathlineto{\pgfqpoint{2.233221in}{1.139332in}}%
\pgfpathlineto{\pgfqpoint{2.233421in}{1.121066in}}%
\pgfpathlineto{\pgfqpoint{2.234525in}{1.078925in}}%
\pgfpathlineto{\pgfqpoint{2.234726in}{1.090190in}}%
\pgfpathlineto{\pgfqpoint{2.235328in}{1.100702in}}%
\pgfpathlineto{\pgfqpoint{2.235830in}{1.091467in}}%
\pgfpathlineto{\pgfqpoint{2.236733in}{1.068862in}}%
\pgfpathlineto{\pgfqpoint{2.236232in}{1.091550in}}%
\pgfpathlineto{\pgfqpoint{2.236934in}{1.076642in}}%
\pgfpathlineto{\pgfqpoint{2.237837in}{1.089450in}}%
\pgfpathlineto{\pgfqpoint{2.237637in}{1.070534in}}%
\pgfpathlineto{\pgfqpoint{2.238138in}{1.086152in}}%
\pgfpathlineto{\pgfqpoint{2.239544in}{1.108097in}}%
\pgfpathlineto{\pgfqpoint{2.238640in}{1.081640in}}%
\pgfpathlineto{\pgfqpoint{2.239644in}{1.104161in}}%
\pgfpathlineto{\pgfqpoint{2.239945in}{1.102702in}}%
\pgfpathlineto{\pgfqpoint{2.240246in}{1.117188in}}%
\pgfpathlineto{\pgfqpoint{2.240848in}{1.099063in}}%
\pgfpathlineto{\pgfqpoint{2.241049in}{1.102977in}}%
\pgfpathlineto{\pgfqpoint{2.241149in}{1.113595in}}%
\pgfpathlineto{\pgfqpoint{2.241451in}{1.100905in}}%
\pgfpathlineto{\pgfqpoint{2.242153in}{1.103633in}}%
\pgfpathlineto{\pgfqpoint{2.243458in}{1.080896in}}%
\pgfpathlineto{\pgfqpoint{2.243859in}{1.083718in}}%
\pgfpathlineto{\pgfqpoint{2.244461in}{1.099329in}}%
\pgfpathlineto{\pgfqpoint{2.244863in}{1.083041in}}%
\pgfpathlineto{\pgfqpoint{2.245565in}{1.061353in}}%
\pgfpathlineto{\pgfqpoint{2.245867in}{1.080392in}}%
\pgfpathlineto{\pgfqpoint{2.246569in}{1.089048in}}%
\pgfpathlineto{\pgfqpoint{2.246770in}{1.080375in}}%
\pgfpathlineto{\pgfqpoint{2.246971in}{1.081791in}}%
\pgfpathlineto{\pgfqpoint{2.248476in}{1.055066in}}%
\pgfpathlineto{\pgfqpoint{2.249480in}{1.064233in}}%
\pgfpathlineto{\pgfqpoint{2.249680in}{1.071949in}}%
\pgfpathlineto{\pgfqpoint{2.250283in}{1.054948in}}%
\pgfpathlineto{\pgfqpoint{2.250684in}{1.025755in}}%
\pgfpathlineto{\pgfqpoint{2.251989in}{1.029190in}}%
\pgfpathlineto{\pgfqpoint{2.254096in}{1.080288in}}%
\pgfpathlineto{\pgfqpoint{2.254197in}{1.080204in}}%
\pgfpathlineto{\pgfqpoint{2.254799in}{1.061936in}}%
\pgfpathlineto{\pgfqpoint{2.255401in}{1.068664in}}%
\pgfpathlineto{\pgfqpoint{2.255502in}{1.072665in}}%
\pgfpathlineto{\pgfqpoint{2.256204in}{1.058154in}}%
\pgfpathlineto{\pgfqpoint{2.256304in}{1.061064in}}%
\pgfpathlineto{\pgfqpoint{2.257609in}{1.032479in}}%
\pgfpathlineto{\pgfqpoint{2.257710in}{1.034828in}}%
\pgfpathlineto{\pgfqpoint{2.257810in}{1.038219in}}%
\pgfpathlineto{\pgfqpoint{2.258312in}{1.025738in}}%
\pgfpathlineto{\pgfqpoint{2.258512in}{1.030606in}}%
\pgfpathlineto{\pgfqpoint{2.258613in}{1.021117in}}%
\pgfpathlineto{\pgfqpoint{2.259215in}{1.040423in}}%
\pgfpathlineto{\pgfqpoint{2.259516in}{1.039942in}}%
\pgfpathlineto{\pgfqpoint{2.259616in}{1.039218in}}%
\pgfpathlineto{\pgfqpoint{2.259717in}{1.043375in}}%
\pgfpathlineto{\pgfqpoint{2.260720in}{1.050572in}}%
\pgfpathlineto{\pgfqpoint{2.260319in}{1.034642in}}%
\pgfpathlineto{\pgfqpoint{2.260821in}{1.048986in}}%
\pgfpathlineto{\pgfqpoint{2.261022in}{1.040561in}}%
\pgfpathlineto{\pgfqpoint{2.261423in}{1.062264in}}%
\pgfpathlineto{\pgfqpoint{2.261824in}{1.051094in}}%
\pgfpathlineto{\pgfqpoint{2.262627in}{1.115089in}}%
\pgfpathlineto{\pgfqpoint{2.263029in}{1.091134in}}%
\pgfpathlineto{\pgfqpoint{2.263430in}{1.110850in}}%
\pgfpathlineto{\pgfqpoint{2.263230in}{1.089849in}}%
\pgfpathlineto{\pgfqpoint{2.263531in}{1.110816in}}%
\pgfpathlineto{\pgfqpoint{2.263731in}{1.128999in}}%
\pgfpathlineto{\pgfqpoint{2.264534in}{1.116115in}}%
\pgfpathlineto{\pgfqpoint{2.265839in}{1.098648in}}%
\pgfpathlineto{\pgfqpoint{2.266843in}{1.137704in}}%
\pgfpathlineto{\pgfqpoint{2.267345in}{1.131135in}}%
\pgfpathlineto{\pgfqpoint{2.267545in}{1.121254in}}%
\pgfpathlineto{\pgfqpoint{2.267947in}{1.134246in}}%
\pgfpathlineto{\pgfqpoint{2.268348in}{1.128830in}}%
\pgfpathlineto{\pgfqpoint{2.268449in}{1.135101in}}%
\pgfpathlineto{\pgfqpoint{2.268950in}{1.119502in}}%
\pgfpathlineto{\pgfqpoint{2.269151in}{1.123264in}}%
\pgfpathlineto{\pgfqpoint{2.269753in}{1.106048in}}%
\pgfpathlineto{\pgfqpoint{2.270054in}{1.125860in}}%
\pgfpathlineto{\pgfqpoint{2.270255in}{1.118165in}}%
\pgfpathlineto{\pgfqpoint{2.270355in}{1.118598in}}%
\pgfpathlineto{\pgfqpoint{2.270556in}{1.114589in}}%
\pgfpathlineto{\pgfqpoint{2.271660in}{1.101907in}}%
\pgfpathlineto{\pgfqpoint{2.271158in}{1.118987in}}%
\pgfpathlineto{\pgfqpoint{2.271761in}{1.107550in}}%
\pgfpathlineto{\pgfqpoint{2.271861in}{1.105584in}}%
\pgfpathlineto{\pgfqpoint{2.272062in}{1.116748in}}%
\pgfpathlineto{\pgfqpoint{2.272162in}{1.114667in}}%
\pgfpathlineto{\pgfqpoint{2.273467in}{1.143809in}}%
\pgfpathlineto{\pgfqpoint{2.273567in}{1.140278in}}%
\pgfpathlineto{\pgfqpoint{2.274069in}{1.125998in}}%
\pgfpathlineto{\pgfqpoint{2.274571in}{1.138771in}}%
\pgfpathlineto{\pgfqpoint{2.275374in}{1.165126in}}%
\pgfpathlineto{\pgfqpoint{2.275775in}{1.149313in}}%
\pgfpathlineto{\pgfqpoint{2.275875in}{1.155688in}}%
\pgfpathlineto{\pgfqpoint{2.276678in}{1.137754in}}%
\pgfpathlineto{\pgfqpoint{2.278385in}{1.164547in}}%
\pgfpathlineto{\pgfqpoint{2.278485in}{1.162815in}}%
\pgfpathlineto{\pgfqpoint{2.278585in}{1.163398in}}%
\pgfpathlineto{\pgfqpoint{2.279890in}{1.189333in}}%
\pgfpathlineto{\pgfqpoint{2.279388in}{1.162850in}}%
\pgfpathlineto{\pgfqpoint{2.279990in}{1.188047in}}%
\pgfpathlineto{\pgfqpoint{2.280392in}{1.177782in}}%
\pgfpathlineto{\pgfqpoint{2.280693in}{1.197706in}}%
\pgfpathlineto{\pgfqpoint{2.280894in}{1.187931in}}%
\pgfpathlineto{\pgfqpoint{2.281797in}{1.202537in}}%
\pgfpathlineto{\pgfqpoint{2.282098in}{1.196709in}}%
\pgfpathlineto{\pgfqpoint{2.282198in}{1.200208in}}%
\pgfpathlineto{\pgfqpoint{2.282600in}{1.184003in}}%
\pgfpathlineto{\pgfqpoint{2.282801in}{1.189964in}}%
\pgfpathlineto{\pgfqpoint{2.283704in}{1.174864in}}%
\pgfpathlineto{\pgfqpoint{2.283804in}{1.182335in}}%
\pgfpathlineto{\pgfqpoint{2.284306in}{1.192528in}}%
\pgfpathlineto{\pgfqpoint{2.284708in}{1.178402in}}%
\pgfpathlineto{\pgfqpoint{2.284908in}{1.164109in}}%
\pgfpathlineto{\pgfqpoint{2.285912in}{1.166232in}}%
\pgfpathlineto{\pgfqpoint{2.286414in}{1.176235in}}%
\pgfpathlineto{\pgfqpoint{2.286614in}{1.157718in}}%
\pgfpathlineto{\pgfqpoint{2.286815in}{1.162649in}}%
\pgfpathlineto{\pgfqpoint{2.286916in}{1.159104in}}%
\pgfpathlineto{\pgfqpoint{2.287417in}{1.180249in}}%
\pgfpathlineto{\pgfqpoint{2.287518in}{1.173688in}}%
\pgfpathlineto{\pgfqpoint{2.290729in}{1.251772in}}%
\pgfpathlineto{\pgfqpoint{2.290930in}{1.240096in}}%
\pgfpathlineto{\pgfqpoint{2.292034in}{1.208428in}}%
\pgfpathlineto{\pgfqpoint{2.292235in}{1.214090in}}%
\pgfpathlineto{\pgfqpoint{2.293138in}{1.236475in}}%
\pgfpathlineto{\pgfqpoint{2.293540in}{1.233383in}}%
\pgfpathlineto{\pgfqpoint{2.295346in}{1.186650in}}%
\pgfpathlineto{\pgfqpoint{2.295547in}{1.198824in}}%
\pgfpathlineto{\pgfqpoint{2.295748in}{1.197411in}}%
\pgfpathlineto{\pgfqpoint{2.295948in}{1.205222in}}%
\pgfpathlineto{\pgfqpoint{2.296551in}{1.200303in}}%
\pgfpathlineto{\pgfqpoint{2.296751in}{1.206759in}}%
\pgfpathlineto{\pgfqpoint{2.297052in}{1.189673in}}%
\pgfpathlineto{\pgfqpoint{2.297353in}{1.193733in}}%
\pgfpathlineto{\pgfqpoint{2.297554in}{1.187975in}}%
\pgfpathlineto{\pgfqpoint{2.297855in}{1.198721in}}%
\pgfpathlineto{\pgfqpoint{2.297956in}{1.194478in}}%
\pgfpathlineto{\pgfqpoint{2.299260in}{1.241774in}}%
\pgfpathlineto{\pgfqpoint{2.299561in}{1.224562in}}%
\pgfpathlineto{\pgfqpoint{2.300264in}{1.239714in}}%
\pgfpathlineto{\pgfqpoint{2.300665in}{1.249700in}}%
\pgfpathlineto{\pgfqpoint{2.301268in}{1.230449in}}%
\pgfpathlineto{\pgfqpoint{2.301368in}{1.230706in}}%
\pgfpathlineto{\pgfqpoint{2.301769in}{1.253889in}}%
\pgfpathlineto{\pgfqpoint{2.302773in}{1.253085in}}%
\pgfpathlineto{\pgfqpoint{2.303476in}{1.220338in}}%
\pgfpathlineto{\pgfqpoint{2.303977in}{1.236834in}}%
\pgfpathlineto{\pgfqpoint{2.304078in}{1.236395in}}%
\pgfpathlineto{\pgfqpoint{2.304178in}{1.239384in}}%
\pgfpathlineto{\pgfqpoint{2.304479in}{1.252031in}}%
\pgfpathlineto{\pgfqpoint{2.305082in}{1.232538in}}%
\pgfpathlineto{\pgfqpoint{2.305182in}{1.234900in}}%
\pgfpathlineto{\pgfqpoint{2.305483in}{1.229980in}}%
\pgfpathlineto{\pgfqpoint{2.305684in}{1.244671in}}%
\pgfpathlineto{\pgfqpoint{2.305985in}{1.239813in}}%
\pgfpathlineto{\pgfqpoint{2.306085in}{1.242991in}}%
\pgfpathlineto{\pgfqpoint{2.306487in}{1.230433in}}%
\pgfpathlineto{\pgfqpoint{2.306788in}{1.237265in}}%
\pgfpathlineto{\pgfqpoint{2.308193in}{1.218239in}}%
\pgfpathlineto{\pgfqpoint{2.308394in}{1.214172in}}%
\pgfpathlineto{\pgfqpoint{2.308996in}{1.194335in}}%
\pgfpathlineto{\pgfqpoint{2.308594in}{1.220479in}}%
\pgfpathlineto{\pgfqpoint{2.309598in}{1.198738in}}%
\pgfpathlineto{\pgfqpoint{2.310903in}{1.223315in}}%
\pgfpathlineto{\pgfqpoint{2.311806in}{1.198059in}}%
\pgfpathlineto{\pgfqpoint{2.312207in}{1.203893in}}%
\pgfpathlineto{\pgfqpoint{2.312709in}{1.213193in}}%
\pgfpathlineto{\pgfqpoint{2.312910in}{1.192138in}}%
\pgfpathlineto{\pgfqpoint{2.313010in}{1.194340in}}%
\pgfpathlineto{\pgfqpoint{2.314114in}{1.154455in}}%
\pgfpathlineto{\pgfqpoint{2.314315in}{1.160139in}}%
\pgfpathlineto{\pgfqpoint{2.315419in}{1.170735in}}%
\pgfpathlineto{\pgfqpoint{2.314616in}{1.154300in}}%
\pgfpathlineto{\pgfqpoint{2.315519in}{1.169695in}}%
\pgfpathlineto{\pgfqpoint{2.316021in}{1.151670in}}%
\pgfpathlineto{\pgfqpoint{2.316122in}{1.153101in}}%
\pgfusepath{stroke}%
\end{pgfscope}%
\begin{pgfscope}%
\pgfsetrectcap%
\pgfsetmiterjoin%
\pgfsetlinewidth{0.803000pt}%
\definecolor{currentstroke}{rgb}{0.000000,0.000000,0.000000}%
\pgfsetstrokecolor{currentstroke}%
\pgfsetdash{}{0pt}%
\pgfpathmoveto{\pgfqpoint{0.589745in}{0.416447in}}%
\pgfpathlineto{\pgfqpoint{0.589745in}{1.788330in}}%
\pgfusepath{stroke}%
\end{pgfscope}%
\begin{pgfscope}%
\pgfsetrectcap%
\pgfsetmiterjoin%
\pgfsetlinewidth{0.803000pt}%
\definecolor{currentstroke}{rgb}{0.000000,0.000000,0.000000}%
\pgfsetstrokecolor{currentstroke}%
\pgfsetdash{}{0pt}%
\pgfpathmoveto{\pgfqpoint{2.398330in}{0.416447in}}%
\pgfpathlineto{\pgfqpoint{2.398330in}{1.788330in}}%
\pgfusepath{stroke}%
\end{pgfscope}%
\begin{pgfscope}%
\pgfsetrectcap%
\pgfsetmiterjoin%
\pgfsetlinewidth{0.803000pt}%
\definecolor{currentstroke}{rgb}{0.000000,0.000000,0.000000}%
\pgfsetstrokecolor{currentstroke}%
\pgfsetdash{}{0pt}%
\pgfpathmoveto{\pgfqpoint{0.589745in}{0.416447in}}%
\pgfpathlineto{\pgfqpoint{2.398330in}{0.416447in}}%
\pgfusepath{stroke}%
\end{pgfscope}%
\begin{pgfscope}%
\pgfsetrectcap%
\pgfsetmiterjoin%
\pgfsetlinewidth{0.803000pt}%
\definecolor{currentstroke}{rgb}{0.000000,0.000000,0.000000}%
\pgfsetstrokecolor{currentstroke}%
\pgfsetdash{}{0pt}%
\pgfpathmoveto{\pgfqpoint{0.589745in}{1.788330in}}%
\pgfpathlineto{\pgfqpoint{2.398330in}{1.788330in}}%
\pgfusepath{stroke}%
\end{pgfscope}%
\begin{pgfscope}%
\pgfsetbuttcap%
\pgfsetmiterjoin%
\definecolor{currentfill}{rgb}{1.000000,1.000000,1.000000}%
\pgfsetfillcolor{currentfill}%
\pgfsetfillopacity{0.800000}%
\pgfsetlinewidth{1.003750pt}%
\definecolor{currentstroke}{rgb}{0.800000,0.800000,0.800000}%
\pgfsetstrokecolor{currentstroke}%
\pgfsetstrokeopacity{0.800000}%
\pgfsetdash{}{0pt}%
\pgfpathmoveto{\pgfqpoint{0.667523in}{1.544552in}}%
\pgfpathlineto{\pgfqpoint{1.732745in}{1.544552in}}%
\pgfpathquadraticcurveto{\pgfqpoint{1.754967in}{1.544552in}}{\pgfqpoint{1.754967in}{1.566775in}}%
\pgfpathlineto{\pgfqpoint{1.754967in}{1.710552in}}%
\pgfpathquadraticcurveto{\pgfqpoint{1.754967in}{1.732774in}}{\pgfqpoint{1.732745in}{1.732774in}}%
\pgfpathlineto{\pgfqpoint{0.667523in}{1.732774in}}%
\pgfpathquadraticcurveto{\pgfqpoint{0.645300in}{1.732774in}}{\pgfqpoint{0.645300in}{1.710552in}}%
\pgfpathlineto{\pgfqpoint{0.645300in}{1.566775in}}%
\pgfpathquadraticcurveto{\pgfqpoint{0.645300in}{1.544552in}}{\pgfqpoint{0.667523in}{1.544552in}}%
\pgfpathlineto{\pgfqpoint{0.667523in}{1.544552in}}%
\pgfpathclose%
\pgfusepath{stroke,fill}%
\end{pgfscope}%
\begin{pgfscope}%
\pgfsetrectcap%
\pgfsetroundjoin%
\pgfsetlinewidth{1.505625pt}%
\definecolor{currentstroke}{rgb}{0.835294,0.368627,0.000000}%
\pgfsetstrokecolor{currentstroke}%
\pgfsetdash{}{0pt}%
\pgfpathmoveto{\pgfqpoint{0.689745in}{1.649441in}}%
\pgfpathlineto{\pgfqpoint{0.800856in}{1.649441in}}%
\pgfpathlineto{\pgfqpoint{0.911967in}{1.649441in}}%
\pgfusepath{stroke}%
\end{pgfscope}%
\begin{pgfscope}%
\definecolor{textcolor}{rgb}{0.000000,0.000000,0.000000}%
\pgfsetstrokecolor{textcolor}%
\pgfsetfillcolor{textcolor}%
\pgftext[x=1.000856in,y=1.610552in,left,base]{\color{textcolor}\rmfamily\fontsize{8.000000}{9.600000}\selectfont Random walk}%
\end{pgfscope}%
\end{pgfpicture}%
\makeatother%
\endgroup%

        } % scalebox
        \caption{Time domain}
        \label{fig:random_walk_time}
    \end{subfigure}
    \begin{subfigure}{0.32\linewidth}
        \centering
        \scalebox{0.75}{%
            %% Creator: Matplotlib, PGF backend
%%
%% To include the figure in your LaTeX document, write
%%   \input{<filename>.pgf}
%%
%% Make sure the required packages are loaded in your preamble
%%   \usepackage{pgf}
%%
%% Also ensure that all the required font packages are loaded; for instance,
%% the lmodern package is sometimes necessary when using math font.
%%   \usepackage{lmodern}
%%
%% Figures using additional raster images can only be included by \input if
%% they are in the same directory as the main LaTeX file. For loading figures
%% from other directories you can use the `import` package
%%   \usepackage{import}
%%
%% and then include the figures with
%%   \import{<path to file>}{<filename>.pgf}
%%
%% Matplotlib used the following preamble
%%   \usepackage{siunitx}
%%   \usepackage{fontspec}
%%   \makeatletter\@ifpackageloaded{underscore}{}{\usepackage[strings]{underscore}}\makeatother
%%
\begingroup%
\makeatletter%
\begin{pgfpicture}%
\pgfpathrectangle{\pgfpointorigin}{\pgfqpoint{2.440000in}{1.830000in}}%
\pgfusepath{use as bounding box, clip}%
\begin{pgfscope}%
\pgfsetbuttcap%
\pgfsetmiterjoin%
\definecolor{currentfill}{rgb}{1.000000,1.000000,1.000000}%
\pgfsetfillcolor{currentfill}%
\pgfsetlinewidth{0.000000pt}%
\definecolor{currentstroke}{rgb}{1.000000,1.000000,1.000000}%
\pgfsetstrokecolor{currentstroke}%
\pgfsetdash{}{0pt}%
\pgfpathmoveto{\pgfqpoint{0.000000in}{0.000000in}}%
\pgfpathlineto{\pgfqpoint{2.440000in}{0.000000in}}%
\pgfpathlineto{\pgfqpoint{2.440000in}{1.830000in}}%
\pgfpathlineto{\pgfqpoint{0.000000in}{1.830000in}}%
\pgfpathlineto{\pgfqpoint{0.000000in}{0.000000in}}%
\pgfpathclose%
\pgfusepath{fill}%
\end{pgfscope}%
\begin{pgfscope}%
\pgfsetbuttcap%
\pgfsetmiterjoin%
\definecolor{currentfill}{rgb}{1.000000,1.000000,1.000000}%
\pgfsetfillcolor{currentfill}%
\pgfsetlinewidth{0.000000pt}%
\definecolor{currentstroke}{rgb}{0.000000,0.000000,0.000000}%
\pgfsetstrokecolor{currentstroke}%
\pgfsetstrokeopacity{0.000000}%
\pgfsetdash{}{0pt}%
\pgfpathmoveto{\pgfqpoint{0.514278in}{0.417642in}}%
\pgfpathlineto{\pgfqpoint{2.398330in}{0.417642in}}%
\pgfpathlineto{\pgfqpoint{2.398330in}{1.788330in}}%
\pgfpathlineto{\pgfqpoint{0.514278in}{1.788330in}}%
\pgfpathlineto{\pgfqpoint{0.514278in}{0.417642in}}%
\pgfpathclose%
\pgfusepath{fill}%
\end{pgfscope}%
\begin{pgfscope}%
\pgfpathrectangle{\pgfqpoint{0.514278in}{0.417642in}}{\pgfqpoint{1.884052in}{1.370688in}}%
\pgfusepath{clip}%
\pgfsetrectcap%
\pgfsetroundjoin%
\pgfsetlinewidth{0.803000pt}%
\definecolor{currentstroke}{rgb}{0.450000,0.450000,0.450000}%
\pgfsetstrokecolor{currentstroke}%
\pgfsetdash{}{0pt}%
\pgfpathmoveto{\pgfqpoint{0.916624in}{0.417642in}}%
\pgfpathlineto{\pgfqpoint{0.916624in}{1.788330in}}%
\pgfusepath{stroke}%
\end{pgfscope}%
\begin{pgfscope}%
\pgfsetbuttcap%
\pgfsetroundjoin%
\definecolor{currentfill}{rgb}{0.000000,0.000000,0.000000}%
\pgfsetfillcolor{currentfill}%
\pgfsetlinewidth{0.803000pt}%
\definecolor{currentstroke}{rgb}{0.000000,0.000000,0.000000}%
\pgfsetstrokecolor{currentstroke}%
\pgfsetdash{}{0pt}%
\pgfsys@defobject{currentmarker}{\pgfqpoint{0.000000in}{-0.048611in}}{\pgfqpoint{0.000000in}{0.000000in}}{%
\pgfpathmoveto{\pgfqpoint{0.000000in}{0.000000in}}%
\pgfpathlineto{\pgfqpoint{0.000000in}{-0.048611in}}%
\pgfusepath{stroke,fill}%
}%
\begin{pgfscope}%
\pgfsys@transformshift{0.916624in}{0.417642in}%
\pgfsys@useobject{currentmarker}{}%
\end{pgfscope}%
\end{pgfscope}%
\begin{pgfscope}%
\definecolor{textcolor}{rgb}{0.000000,0.000000,0.000000}%
\pgfsetstrokecolor{textcolor}%
\pgfsetfillcolor{textcolor}%
\pgftext[x=0.916624in,y=0.320420in,,top]{\color{textcolor}\rmfamily\fontsize{8.000000}{9.600000}\selectfont \(\displaystyle {10^{-3}}\)}%
\end{pgfscope}%
\begin{pgfscope}%
\pgfpathrectangle{\pgfqpoint{0.514278in}{0.417642in}}{\pgfqpoint{1.884052in}{1.370688in}}%
\pgfusepath{clip}%
\pgfsetrectcap%
\pgfsetroundjoin%
\pgfsetlinewidth{0.803000pt}%
\definecolor{currentstroke}{rgb}{0.450000,0.450000,0.450000}%
\pgfsetstrokecolor{currentstroke}%
\pgfsetdash{}{0pt}%
\pgfpathmoveto{\pgfqpoint{1.433903in}{0.417642in}}%
\pgfpathlineto{\pgfqpoint{1.433903in}{1.788330in}}%
\pgfusepath{stroke}%
\end{pgfscope}%
\begin{pgfscope}%
\pgfsetbuttcap%
\pgfsetroundjoin%
\definecolor{currentfill}{rgb}{0.000000,0.000000,0.000000}%
\pgfsetfillcolor{currentfill}%
\pgfsetlinewidth{0.803000pt}%
\definecolor{currentstroke}{rgb}{0.000000,0.000000,0.000000}%
\pgfsetstrokecolor{currentstroke}%
\pgfsetdash{}{0pt}%
\pgfsys@defobject{currentmarker}{\pgfqpoint{0.000000in}{-0.048611in}}{\pgfqpoint{0.000000in}{0.000000in}}{%
\pgfpathmoveto{\pgfqpoint{0.000000in}{0.000000in}}%
\pgfpathlineto{\pgfqpoint{0.000000in}{-0.048611in}}%
\pgfusepath{stroke,fill}%
}%
\begin{pgfscope}%
\pgfsys@transformshift{1.433903in}{0.417642in}%
\pgfsys@useobject{currentmarker}{}%
\end{pgfscope}%
\end{pgfscope}%
\begin{pgfscope}%
\definecolor{textcolor}{rgb}{0.000000,0.000000,0.000000}%
\pgfsetstrokecolor{textcolor}%
\pgfsetfillcolor{textcolor}%
\pgftext[x=1.433903in,y=0.320420in,,top]{\color{textcolor}\rmfamily\fontsize{8.000000}{9.600000}\selectfont \(\displaystyle {10^{-2}}\)}%
\end{pgfscope}%
\begin{pgfscope}%
\pgfpathrectangle{\pgfqpoint{0.514278in}{0.417642in}}{\pgfqpoint{1.884052in}{1.370688in}}%
\pgfusepath{clip}%
\pgfsetrectcap%
\pgfsetroundjoin%
\pgfsetlinewidth{0.803000pt}%
\definecolor{currentstroke}{rgb}{0.450000,0.450000,0.450000}%
\pgfsetstrokecolor{currentstroke}%
\pgfsetdash{}{0pt}%
\pgfpathmoveto{\pgfqpoint{1.951183in}{0.417642in}}%
\pgfpathlineto{\pgfqpoint{1.951183in}{1.788330in}}%
\pgfusepath{stroke}%
\end{pgfscope}%
\begin{pgfscope}%
\pgfsetbuttcap%
\pgfsetroundjoin%
\definecolor{currentfill}{rgb}{0.000000,0.000000,0.000000}%
\pgfsetfillcolor{currentfill}%
\pgfsetlinewidth{0.803000pt}%
\definecolor{currentstroke}{rgb}{0.000000,0.000000,0.000000}%
\pgfsetstrokecolor{currentstroke}%
\pgfsetdash{}{0pt}%
\pgfsys@defobject{currentmarker}{\pgfqpoint{0.000000in}{-0.048611in}}{\pgfqpoint{0.000000in}{0.000000in}}{%
\pgfpathmoveto{\pgfqpoint{0.000000in}{0.000000in}}%
\pgfpathlineto{\pgfqpoint{0.000000in}{-0.048611in}}%
\pgfusepath{stroke,fill}%
}%
\begin{pgfscope}%
\pgfsys@transformshift{1.951183in}{0.417642in}%
\pgfsys@useobject{currentmarker}{}%
\end{pgfscope}%
\end{pgfscope}%
\begin{pgfscope}%
\definecolor{textcolor}{rgb}{0.000000,0.000000,0.000000}%
\pgfsetstrokecolor{textcolor}%
\pgfsetfillcolor{textcolor}%
\pgftext[x=1.951183in,y=0.320420in,,top]{\color{textcolor}\rmfamily\fontsize{8.000000}{9.600000}\selectfont \(\displaystyle {10^{-1}}\)}%
\end{pgfscope}%
\begin{pgfscope}%
\pgfpathrectangle{\pgfqpoint{0.514278in}{0.417642in}}{\pgfqpoint{1.884052in}{1.370688in}}%
\pgfusepath{clip}%
\pgfsetrectcap%
\pgfsetroundjoin%
\pgfsetlinewidth{0.803000pt}%
\definecolor{currentstroke}{rgb}{0.850000,0.850000,0.850000}%
\pgfsetstrokecolor{currentstroke}%
\pgfsetdash{}{0pt}%
\pgfpathmoveto{\pgfqpoint{0.555061in}{0.417642in}}%
\pgfpathlineto{\pgfqpoint{0.555061in}{1.788330in}}%
\pgfusepath{stroke}%
\end{pgfscope}%
\begin{pgfscope}%
\pgfsetbuttcap%
\pgfsetroundjoin%
\definecolor{currentfill}{rgb}{0.000000,0.000000,0.000000}%
\pgfsetfillcolor{currentfill}%
\pgfsetlinewidth{0.602250pt}%
\definecolor{currentstroke}{rgb}{0.000000,0.000000,0.000000}%
\pgfsetstrokecolor{currentstroke}%
\pgfsetdash{}{0pt}%
\pgfsys@defobject{currentmarker}{\pgfqpoint{0.000000in}{-0.027778in}}{\pgfqpoint{0.000000in}{0.000000in}}{%
\pgfpathmoveto{\pgfqpoint{0.000000in}{0.000000in}}%
\pgfpathlineto{\pgfqpoint{0.000000in}{-0.027778in}}%
\pgfusepath{stroke,fill}%
}%
\begin{pgfscope}%
\pgfsys@transformshift{0.555061in}{0.417642in}%
\pgfsys@useobject{currentmarker}{}%
\end{pgfscope}%
\end{pgfscope}%
\begin{pgfscope}%
\pgfpathrectangle{\pgfqpoint{0.514278in}{0.417642in}}{\pgfqpoint{1.884052in}{1.370688in}}%
\pgfusepath{clip}%
\pgfsetrectcap%
\pgfsetroundjoin%
\pgfsetlinewidth{0.803000pt}%
\definecolor{currentstroke}{rgb}{0.850000,0.850000,0.850000}%
\pgfsetstrokecolor{currentstroke}%
\pgfsetdash{}{0pt}%
\pgfpathmoveto{\pgfqpoint{0.646149in}{0.417642in}}%
\pgfpathlineto{\pgfqpoint{0.646149in}{1.788330in}}%
\pgfusepath{stroke}%
\end{pgfscope}%
\begin{pgfscope}%
\pgfsetbuttcap%
\pgfsetroundjoin%
\definecolor{currentfill}{rgb}{0.000000,0.000000,0.000000}%
\pgfsetfillcolor{currentfill}%
\pgfsetlinewidth{0.602250pt}%
\definecolor{currentstroke}{rgb}{0.000000,0.000000,0.000000}%
\pgfsetstrokecolor{currentstroke}%
\pgfsetdash{}{0pt}%
\pgfsys@defobject{currentmarker}{\pgfqpoint{0.000000in}{-0.027778in}}{\pgfqpoint{0.000000in}{0.000000in}}{%
\pgfpathmoveto{\pgfqpoint{0.000000in}{0.000000in}}%
\pgfpathlineto{\pgfqpoint{0.000000in}{-0.027778in}}%
\pgfusepath{stroke,fill}%
}%
\begin{pgfscope}%
\pgfsys@transformshift{0.646149in}{0.417642in}%
\pgfsys@useobject{currentmarker}{}%
\end{pgfscope}%
\end{pgfscope}%
\begin{pgfscope}%
\pgfpathrectangle{\pgfqpoint{0.514278in}{0.417642in}}{\pgfqpoint{1.884052in}{1.370688in}}%
\pgfusepath{clip}%
\pgfsetrectcap%
\pgfsetroundjoin%
\pgfsetlinewidth{0.803000pt}%
\definecolor{currentstroke}{rgb}{0.850000,0.850000,0.850000}%
\pgfsetstrokecolor{currentstroke}%
\pgfsetdash{}{0pt}%
\pgfpathmoveto{\pgfqpoint{0.710777in}{0.417642in}}%
\pgfpathlineto{\pgfqpoint{0.710777in}{1.788330in}}%
\pgfusepath{stroke}%
\end{pgfscope}%
\begin{pgfscope}%
\pgfsetbuttcap%
\pgfsetroundjoin%
\definecolor{currentfill}{rgb}{0.000000,0.000000,0.000000}%
\pgfsetfillcolor{currentfill}%
\pgfsetlinewidth{0.602250pt}%
\definecolor{currentstroke}{rgb}{0.000000,0.000000,0.000000}%
\pgfsetstrokecolor{currentstroke}%
\pgfsetdash{}{0pt}%
\pgfsys@defobject{currentmarker}{\pgfqpoint{0.000000in}{-0.027778in}}{\pgfqpoint{0.000000in}{0.000000in}}{%
\pgfpathmoveto{\pgfqpoint{0.000000in}{0.000000in}}%
\pgfpathlineto{\pgfqpoint{0.000000in}{-0.027778in}}%
\pgfusepath{stroke,fill}%
}%
\begin{pgfscope}%
\pgfsys@transformshift{0.710777in}{0.417642in}%
\pgfsys@useobject{currentmarker}{}%
\end{pgfscope}%
\end{pgfscope}%
\begin{pgfscope}%
\pgfpathrectangle{\pgfqpoint{0.514278in}{0.417642in}}{\pgfqpoint{1.884052in}{1.370688in}}%
\pgfusepath{clip}%
\pgfsetrectcap%
\pgfsetroundjoin%
\pgfsetlinewidth{0.803000pt}%
\definecolor{currentstroke}{rgb}{0.850000,0.850000,0.850000}%
\pgfsetstrokecolor{currentstroke}%
\pgfsetdash{}{0pt}%
\pgfpathmoveto{\pgfqpoint{0.760907in}{0.417642in}}%
\pgfpathlineto{\pgfqpoint{0.760907in}{1.788330in}}%
\pgfusepath{stroke}%
\end{pgfscope}%
\begin{pgfscope}%
\pgfsetbuttcap%
\pgfsetroundjoin%
\definecolor{currentfill}{rgb}{0.000000,0.000000,0.000000}%
\pgfsetfillcolor{currentfill}%
\pgfsetlinewidth{0.602250pt}%
\definecolor{currentstroke}{rgb}{0.000000,0.000000,0.000000}%
\pgfsetstrokecolor{currentstroke}%
\pgfsetdash{}{0pt}%
\pgfsys@defobject{currentmarker}{\pgfqpoint{0.000000in}{-0.027778in}}{\pgfqpoint{0.000000in}{0.000000in}}{%
\pgfpathmoveto{\pgfqpoint{0.000000in}{0.000000in}}%
\pgfpathlineto{\pgfqpoint{0.000000in}{-0.027778in}}%
\pgfusepath{stroke,fill}%
}%
\begin{pgfscope}%
\pgfsys@transformshift{0.760907in}{0.417642in}%
\pgfsys@useobject{currentmarker}{}%
\end{pgfscope}%
\end{pgfscope}%
\begin{pgfscope}%
\pgfpathrectangle{\pgfqpoint{0.514278in}{0.417642in}}{\pgfqpoint{1.884052in}{1.370688in}}%
\pgfusepath{clip}%
\pgfsetrectcap%
\pgfsetroundjoin%
\pgfsetlinewidth{0.803000pt}%
\definecolor{currentstroke}{rgb}{0.850000,0.850000,0.850000}%
\pgfsetstrokecolor{currentstroke}%
\pgfsetdash{}{0pt}%
\pgfpathmoveto{\pgfqpoint{0.801866in}{0.417642in}}%
\pgfpathlineto{\pgfqpoint{0.801866in}{1.788330in}}%
\pgfusepath{stroke}%
\end{pgfscope}%
\begin{pgfscope}%
\pgfsetbuttcap%
\pgfsetroundjoin%
\definecolor{currentfill}{rgb}{0.000000,0.000000,0.000000}%
\pgfsetfillcolor{currentfill}%
\pgfsetlinewidth{0.602250pt}%
\definecolor{currentstroke}{rgb}{0.000000,0.000000,0.000000}%
\pgfsetstrokecolor{currentstroke}%
\pgfsetdash{}{0pt}%
\pgfsys@defobject{currentmarker}{\pgfqpoint{0.000000in}{-0.027778in}}{\pgfqpoint{0.000000in}{0.000000in}}{%
\pgfpathmoveto{\pgfqpoint{0.000000in}{0.000000in}}%
\pgfpathlineto{\pgfqpoint{0.000000in}{-0.027778in}}%
\pgfusepath{stroke,fill}%
}%
\begin{pgfscope}%
\pgfsys@transformshift{0.801866in}{0.417642in}%
\pgfsys@useobject{currentmarker}{}%
\end{pgfscope}%
\end{pgfscope}%
\begin{pgfscope}%
\pgfpathrectangle{\pgfqpoint{0.514278in}{0.417642in}}{\pgfqpoint{1.884052in}{1.370688in}}%
\pgfusepath{clip}%
\pgfsetrectcap%
\pgfsetroundjoin%
\pgfsetlinewidth{0.803000pt}%
\definecolor{currentstroke}{rgb}{0.850000,0.850000,0.850000}%
\pgfsetstrokecolor{currentstroke}%
\pgfsetdash{}{0pt}%
\pgfpathmoveto{\pgfqpoint{0.836496in}{0.417642in}}%
\pgfpathlineto{\pgfqpoint{0.836496in}{1.788330in}}%
\pgfusepath{stroke}%
\end{pgfscope}%
\begin{pgfscope}%
\pgfsetbuttcap%
\pgfsetroundjoin%
\definecolor{currentfill}{rgb}{0.000000,0.000000,0.000000}%
\pgfsetfillcolor{currentfill}%
\pgfsetlinewidth{0.602250pt}%
\definecolor{currentstroke}{rgb}{0.000000,0.000000,0.000000}%
\pgfsetstrokecolor{currentstroke}%
\pgfsetdash{}{0pt}%
\pgfsys@defobject{currentmarker}{\pgfqpoint{0.000000in}{-0.027778in}}{\pgfqpoint{0.000000in}{0.000000in}}{%
\pgfpathmoveto{\pgfqpoint{0.000000in}{0.000000in}}%
\pgfpathlineto{\pgfqpoint{0.000000in}{-0.027778in}}%
\pgfusepath{stroke,fill}%
}%
\begin{pgfscope}%
\pgfsys@transformshift{0.836496in}{0.417642in}%
\pgfsys@useobject{currentmarker}{}%
\end{pgfscope}%
\end{pgfscope}%
\begin{pgfscope}%
\pgfpathrectangle{\pgfqpoint{0.514278in}{0.417642in}}{\pgfqpoint{1.884052in}{1.370688in}}%
\pgfusepath{clip}%
\pgfsetrectcap%
\pgfsetroundjoin%
\pgfsetlinewidth{0.803000pt}%
\definecolor{currentstroke}{rgb}{0.850000,0.850000,0.850000}%
\pgfsetstrokecolor{currentstroke}%
\pgfsetdash{}{0pt}%
\pgfpathmoveto{\pgfqpoint{0.866494in}{0.417642in}}%
\pgfpathlineto{\pgfqpoint{0.866494in}{1.788330in}}%
\pgfusepath{stroke}%
\end{pgfscope}%
\begin{pgfscope}%
\pgfsetbuttcap%
\pgfsetroundjoin%
\definecolor{currentfill}{rgb}{0.000000,0.000000,0.000000}%
\pgfsetfillcolor{currentfill}%
\pgfsetlinewidth{0.602250pt}%
\definecolor{currentstroke}{rgb}{0.000000,0.000000,0.000000}%
\pgfsetstrokecolor{currentstroke}%
\pgfsetdash{}{0pt}%
\pgfsys@defobject{currentmarker}{\pgfqpoint{0.000000in}{-0.027778in}}{\pgfqpoint{0.000000in}{0.000000in}}{%
\pgfpathmoveto{\pgfqpoint{0.000000in}{0.000000in}}%
\pgfpathlineto{\pgfqpoint{0.000000in}{-0.027778in}}%
\pgfusepath{stroke,fill}%
}%
\begin{pgfscope}%
\pgfsys@transformshift{0.866494in}{0.417642in}%
\pgfsys@useobject{currentmarker}{}%
\end{pgfscope}%
\end{pgfscope}%
\begin{pgfscope}%
\pgfpathrectangle{\pgfqpoint{0.514278in}{0.417642in}}{\pgfqpoint{1.884052in}{1.370688in}}%
\pgfusepath{clip}%
\pgfsetrectcap%
\pgfsetroundjoin%
\pgfsetlinewidth{0.803000pt}%
\definecolor{currentstroke}{rgb}{0.850000,0.850000,0.850000}%
\pgfsetstrokecolor{currentstroke}%
\pgfsetdash{}{0pt}%
\pgfpathmoveto{\pgfqpoint{0.892954in}{0.417642in}}%
\pgfpathlineto{\pgfqpoint{0.892954in}{1.788330in}}%
\pgfusepath{stroke}%
\end{pgfscope}%
\begin{pgfscope}%
\pgfsetbuttcap%
\pgfsetroundjoin%
\definecolor{currentfill}{rgb}{0.000000,0.000000,0.000000}%
\pgfsetfillcolor{currentfill}%
\pgfsetlinewidth{0.602250pt}%
\definecolor{currentstroke}{rgb}{0.000000,0.000000,0.000000}%
\pgfsetstrokecolor{currentstroke}%
\pgfsetdash{}{0pt}%
\pgfsys@defobject{currentmarker}{\pgfqpoint{0.000000in}{-0.027778in}}{\pgfqpoint{0.000000in}{0.000000in}}{%
\pgfpathmoveto{\pgfqpoint{0.000000in}{0.000000in}}%
\pgfpathlineto{\pgfqpoint{0.000000in}{-0.027778in}}%
\pgfusepath{stroke,fill}%
}%
\begin{pgfscope}%
\pgfsys@transformshift{0.892954in}{0.417642in}%
\pgfsys@useobject{currentmarker}{}%
\end{pgfscope}%
\end{pgfscope}%
\begin{pgfscope}%
\pgfpathrectangle{\pgfqpoint{0.514278in}{0.417642in}}{\pgfqpoint{1.884052in}{1.370688in}}%
\pgfusepath{clip}%
\pgfsetrectcap%
\pgfsetroundjoin%
\pgfsetlinewidth{0.803000pt}%
\definecolor{currentstroke}{rgb}{0.850000,0.850000,0.850000}%
\pgfsetstrokecolor{currentstroke}%
\pgfsetdash{}{0pt}%
\pgfpathmoveto{\pgfqpoint{1.072340in}{0.417642in}}%
\pgfpathlineto{\pgfqpoint{1.072340in}{1.788330in}}%
\pgfusepath{stroke}%
\end{pgfscope}%
\begin{pgfscope}%
\pgfsetbuttcap%
\pgfsetroundjoin%
\definecolor{currentfill}{rgb}{0.000000,0.000000,0.000000}%
\pgfsetfillcolor{currentfill}%
\pgfsetlinewidth{0.602250pt}%
\definecolor{currentstroke}{rgb}{0.000000,0.000000,0.000000}%
\pgfsetstrokecolor{currentstroke}%
\pgfsetdash{}{0pt}%
\pgfsys@defobject{currentmarker}{\pgfqpoint{0.000000in}{-0.027778in}}{\pgfqpoint{0.000000in}{0.000000in}}{%
\pgfpathmoveto{\pgfqpoint{0.000000in}{0.000000in}}%
\pgfpathlineto{\pgfqpoint{0.000000in}{-0.027778in}}%
\pgfusepath{stroke,fill}%
}%
\begin{pgfscope}%
\pgfsys@transformshift{1.072340in}{0.417642in}%
\pgfsys@useobject{currentmarker}{}%
\end{pgfscope}%
\end{pgfscope}%
\begin{pgfscope}%
\pgfpathrectangle{\pgfqpoint{0.514278in}{0.417642in}}{\pgfqpoint{1.884052in}{1.370688in}}%
\pgfusepath{clip}%
\pgfsetrectcap%
\pgfsetroundjoin%
\pgfsetlinewidth{0.803000pt}%
\definecolor{currentstroke}{rgb}{0.850000,0.850000,0.850000}%
\pgfsetstrokecolor{currentstroke}%
\pgfsetdash{}{0pt}%
\pgfpathmoveto{\pgfqpoint{1.163429in}{0.417642in}}%
\pgfpathlineto{\pgfqpoint{1.163429in}{1.788330in}}%
\pgfusepath{stroke}%
\end{pgfscope}%
\begin{pgfscope}%
\pgfsetbuttcap%
\pgfsetroundjoin%
\definecolor{currentfill}{rgb}{0.000000,0.000000,0.000000}%
\pgfsetfillcolor{currentfill}%
\pgfsetlinewidth{0.602250pt}%
\definecolor{currentstroke}{rgb}{0.000000,0.000000,0.000000}%
\pgfsetstrokecolor{currentstroke}%
\pgfsetdash{}{0pt}%
\pgfsys@defobject{currentmarker}{\pgfqpoint{0.000000in}{-0.027778in}}{\pgfqpoint{0.000000in}{0.000000in}}{%
\pgfpathmoveto{\pgfqpoint{0.000000in}{0.000000in}}%
\pgfpathlineto{\pgfqpoint{0.000000in}{-0.027778in}}%
\pgfusepath{stroke,fill}%
}%
\begin{pgfscope}%
\pgfsys@transformshift{1.163429in}{0.417642in}%
\pgfsys@useobject{currentmarker}{}%
\end{pgfscope}%
\end{pgfscope}%
\begin{pgfscope}%
\pgfpathrectangle{\pgfqpoint{0.514278in}{0.417642in}}{\pgfqpoint{1.884052in}{1.370688in}}%
\pgfusepath{clip}%
\pgfsetrectcap%
\pgfsetroundjoin%
\pgfsetlinewidth{0.803000pt}%
\definecolor{currentstroke}{rgb}{0.850000,0.850000,0.850000}%
\pgfsetstrokecolor{currentstroke}%
\pgfsetdash{}{0pt}%
\pgfpathmoveto{\pgfqpoint{1.228057in}{0.417642in}}%
\pgfpathlineto{\pgfqpoint{1.228057in}{1.788330in}}%
\pgfusepath{stroke}%
\end{pgfscope}%
\begin{pgfscope}%
\pgfsetbuttcap%
\pgfsetroundjoin%
\definecolor{currentfill}{rgb}{0.000000,0.000000,0.000000}%
\pgfsetfillcolor{currentfill}%
\pgfsetlinewidth{0.602250pt}%
\definecolor{currentstroke}{rgb}{0.000000,0.000000,0.000000}%
\pgfsetstrokecolor{currentstroke}%
\pgfsetdash{}{0pt}%
\pgfsys@defobject{currentmarker}{\pgfqpoint{0.000000in}{-0.027778in}}{\pgfqpoint{0.000000in}{0.000000in}}{%
\pgfpathmoveto{\pgfqpoint{0.000000in}{0.000000in}}%
\pgfpathlineto{\pgfqpoint{0.000000in}{-0.027778in}}%
\pgfusepath{stroke,fill}%
}%
\begin{pgfscope}%
\pgfsys@transformshift{1.228057in}{0.417642in}%
\pgfsys@useobject{currentmarker}{}%
\end{pgfscope}%
\end{pgfscope}%
\begin{pgfscope}%
\pgfpathrectangle{\pgfqpoint{0.514278in}{0.417642in}}{\pgfqpoint{1.884052in}{1.370688in}}%
\pgfusepath{clip}%
\pgfsetrectcap%
\pgfsetroundjoin%
\pgfsetlinewidth{0.803000pt}%
\definecolor{currentstroke}{rgb}{0.850000,0.850000,0.850000}%
\pgfsetstrokecolor{currentstroke}%
\pgfsetdash{}{0pt}%
\pgfpathmoveto{\pgfqpoint{1.278187in}{0.417642in}}%
\pgfpathlineto{\pgfqpoint{1.278187in}{1.788330in}}%
\pgfusepath{stroke}%
\end{pgfscope}%
\begin{pgfscope}%
\pgfsetbuttcap%
\pgfsetroundjoin%
\definecolor{currentfill}{rgb}{0.000000,0.000000,0.000000}%
\pgfsetfillcolor{currentfill}%
\pgfsetlinewidth{0.602250pt}%
\definecolor{currentstroke}{rgb}{0.000000,0.000000,0.000000}%
\pgfsetstrokecolor{currentstroke}%
\pgfsetdash{}{0pt}%
\pgfsys@defobject{currentmarker}{\pgfqpoint{0.000000in}{-0.027778in}}{\pgfqpoint{0.000000in}{0.000000in}}{%
\pgfpathmoveto{\pgfqpoint{0.000000in}{0.000000in}}%
\pgfpathlineto{\pgfqpoint{0.000000in}{-0.027778in}}%
\pgfusepath{stroke,fill}%
}%
\begin{pgfscope}%
\pgfsys@transformshift{1.278187in}{0.417642in}%
\pgfsys@useobject{currentmarker}{}%
\end{pgfscope}%
\end{pgfscope}%
\begin{pgfscope}%
\pgfpathrectangle{\pgfqpoint{0.514278in}{0.417642in}}{\pgfqpoint{1.884052in}{1.370688in}}%
\pgfusepath{clip}%
\pgfsetrectcap%
\pgfsetroundjoin%
\pgfsetlinewidth{0.803000pt}%
\definecolor{currentstroke}{rgb}{0.850000,0.850000,0.850000}%
\pgfsetstrokecolor{currentstroke}%
\pgfsetdash{}{0pt}%
\pgfpathmoveto{\pgfqpoint{1.319146in}{0.417642in}}%
\pgfpathlineto{\pgfqpoint{1.319146in}{1.788330in}}%
\pgfusepath{stroke}%
\end{pgfscope}%
\begin{pgfscope}%
\pgfsetbuttcap%
\pgfsetroundjoin%
\definecolor{currentfill}{rgb}{0.000000,0.000000,0.000000}%
\pgfsetfillcolor{currentfill}%
\pgfsetlinewidth{0.602250pt}%
\definecolor{currentstroke}{rgb}{0.000000,0.000000,0.000000}%
\pgfsetstrokecolor{currentstroke}%
\pgfsetdash{}{0pt}%
\pgfsys@defobject{currentmarker}{\pgfqpoint{0.000000in}{-0.027778in}}{\pgfqpoint{0.000000in}{0.000000in}}{%
\pgfpathmoveto{\pgfqpoint{0.000000in}{0.000000in}}%
\pgfpathlineto{\pgfqpoint{0.000000in}{-0.027778in}}%
\pgfusepath{stroke,fill}%
}%
\begin{pgfscope}%
\pgfsys@transformshift{1.319146in}{0.417642in}%
\pgfsys@useobject{currentmarker}{}%
\end{pgfscope}%
\end{pgfscope}%
\begin{pgfscope}%
\pgfpathrectangle{\pgfqpoint{0.514278in}{0.417642in}}{\pgfqpoint{1.884052in}{1.370688in}}%
\pgfusepath{clip}%
\pgfsetrectcap%
\pgfsetroundjoin%
\pgfsetlinewidth{0.803000pt}%
\definecolor{currentstroke}{rgb}{0.850000,0.850000,0.850000}%
\pgfsetstrokecolor{currentstroke}%
\pgfsetdash{}{0pt}%
\pgfpathmoveto{\pgfqpoint{1.353776in}{0.417642in}}%
\pgfpathlineto{\pgfqpoint{1.353776in}{1.788330in}}%
\pgfusepath{stroke}%
\end{pgfscope}%
\begin{pgfscope}%
\pgfsetbuttcap%
\pgfsetroundjoin%
\definecolor{currentfill}{rgb}{0.000000,0.000000,0.000000}%
\pgfsetfillcolor{currentfill}%
\pgfsetlinewidth{0.602250pt}%
\definecolor{currentstroke}{rgb}{0.000000,0.000000,0.000000}%
\pgfsetstrokecolor{currentstroke}%
\pgfsetdash{}{0pt}%
\pgfsys@defobject{currentmarker}{\pgfqpoint{0.000000in}{-0.027778in}}{\pgfqpoint{0.000000in}{0.000000in}}{%
\pgfpathmoveto{\pgfqpoint{0.000000in}{0.000000in}}%
\pgfpathlineto{\pgfqpoint{0.000000in}{-0.027778in}}%
\pgfusepath{stroke,fill}%
}%
\begin{pgfscope}%
\pgfsys@transformshift{1.353776in}{0.417642in}%
\pgfsys@useobject{currentmarker}{}%
\end{pgfscope}%
\end{pgfscope}%
\begin{pgfscope}%
\pgfpathrectangle{\pgfqpoint{0.514278in}{0.417642in}}{\pgfqpoint{1.884052in}{1.370688in}}%
\pgfusepath{clip}%
\pgfsetrectcap%
\pgfsetroundjoin%
\pgfsetlinewidth{0.803000pt}%
\definecolor{currentstroke}{rgb}{0.850000,0.850000,0.850000}%
\pgfsetstrokecolor{currentstroke}%
\pgfsetdash{}{0pt}%
\pgfpathmoveto{\pgfqpoint{1.383774in}{0.417642in}}%
\pgfpathlineto{\pgfqpoint{1.383774in}{1.788330in}}%
\pgfusepath{stroke}%
\end{pgfscope}%
\begin{pgfscope}%
\pgfsetbuttcap%
\pgfsetroundjoin%
\definecolor{currentfill}{rgb}{0.000000,0.000000,0.000000}%
\pgfsetfillcolor{currentfill}%
\pgfsetlinewidth{0.602250pt}%
\definecolor{currentstroke}{rgb}{0.000000,0.000000,0.000000}%
\pgfsetstrokecolor{currentstroke}%
\pgfsetdash{}{0pt}%
\pgfsys@defobject{currentmarker}{\pgfqpoint{0.000000in}{-0.027778in}}{\pgfqpoint{0.000000in}{0.000000in}}{%
\pgfpathmoveto{\pgfqpoint{0.000000in}{0.000000in}}%
\pgfpathlineto{\pgfqpoint{0.000000in}{-0.027778in}}%
\pgfusepath{stroke,fill}%
}%
\begin{pgfscope}%
\pgfsys@transformshift{1.383774in}{0.417642in}%
\pgfsys@useobject{currentmarker}{}%
\end{pgfscope}%
\end{pgfscope}%
\begin{pgfscope}%
\pgfpathrectangle{\pgfqpoint{0.514278in}{0.417642in}}{\pgfqpoint{1.884052in}{1.370688in}}%
\pgfusepath{clip}%
\pgfsetrectcap%
\pgfsetroundjoin%
\pgfsetlinewidth{0.803000pt}%
\definecolor{currentstroke}{rgb}{0.850000,0.850000,0.850000}%
\pgfsetstrokecolor{currentstroke}%
\pgfsetdash{}{0pt}%
\pgfpathmoveto{\pgfqpoint{1.410234in}{0.417642in}}%
\pgfpathlineto{\pgfqpoint{1.410234in}{1.788330in}}%
\pgfusepath{stroke}%
\end{pgfscope}%
\begin{pgfscope}%
\pgfsetbuttcap%
\pgfsetroundjoin%
\definecolor{currentfill}{rgb}{0.000000,0.000000,0.000000}%
\pgfsetfillcolor{currentfill}%
\pgfsetlinewidth{0.602250pt}%
\definecolor{currentstroke}{rgb}{0.000000,0.000000,0.000000}%
\pgfsetstrokecolor{currentstroke}%
\pgfsetdash{}{0pt}%
\pgfsys@defobject{currentmarker}{\pgfqpoint{0.000000in}{-0.027778in}}{\pgfqpoint{0.000000in}{0.000000in}}{%
\pgfpathmoveto{\pgfqpoint{0.000000in}{0.000000in}}%
\pgfpathlineto{\pgfqpoint{0.000000in}{-0.027778in}}%
\pgfusepath{stroke,fill}%
}%
\begin{pgfscope}%
\pgfsys@transformshift{1.410234in}{0.417642in}%
\pgfsys@useobject{currentmarker}{}%
\end{pgfscope}%
\end{pgfscope}%
\begin{pgfscope}%
\pgfpathrectangle{\pgfqpoint{0.514278in}{0.417642in}}{\pgfqpoint{1.884052in}{1.370688in}}%
\pgfusepath{clip}%
\pgfsetrectcap%
\pgfsetroundjoin%
\pgfsetlinewidth{0.803000pt}%
\definecolor{currentstroke}{rgb}{0.850000,0.850000,0.850000}%
\pgfsetstrokecolor{currentstroke}%
\pgfsetdash{}{0pt}%
\pgfpathmoveto{\pgfqpoint{1.589620in}{0.417642in}}%
\pgfpathlineto{\pgfqpoint{1.589620in}{1.788330in}}%
\pgfusepath{stroke}%
\end{pgfscope}%
\begin{pgfscope}%
\pgfsetbuttcap%
\pgfsetroundjoin%
\definecolor{currentfill}{rgb}{0.000000,0.000000,0.000000}%
\pgfsetfillcolor{currentfill}%
\pgfsetlinewidth{0.602250pt}%
\definecolor{currentstroke}{rgb}{0.000000,0.000000,0.000000}%
\pgfsetstrokecolor{currentstroke}%
\pgfsetdash{}{0pt}%
\pgfsys@defobject{currentmarker}{\pgfqpoint{0.000000in}{-0.027778in}}{\pgfqpoint{0.000000in}{0.000000in}}{%
\pgfpathmoveto{\pgfqpoint{0.000000in}{0.000000in}}%
\pgfpathlineto{\pgfqpoint{0.000000in}{-0.027778in}}%
\pgfusepath{stroke,fill}%
}%
\begin{pgfscope}%
\pgfsys@transformshift{1.589620in}{0.417642in}%
\pgfsys@useobject{currentmarker}{}%
\end{pgfscope}%
\end{pgfscope}%
\begin{pgfscope}%
\pgfpathrectangle{\pgfqpoint{0.514278in}{0.417642in}}{\pgfqpoint{1.884052in}{1.370688in}}%
\pgfusepath{clip}%
\pgfsetrectcap%
\pgfsetroundjoin%
\pgfsetlinewidth{0.803000pt}%
\definecolor{currentstroke}{rgb}{0.850000,0.850000,0.850000}%
\pgfsetstrokecolor{currentstroke}%
\pgfsetdash{}{0pt}%
\pgfpathmoveto{\pgfqpoint{1.680709in}{0.417642in}}%
\pgfpathlineto{\pgfqpoint{1.680709in}{1.788330in}}%
\pgfusepath{stroke}%
\end{pgfscope}%
\begin{pgfscope}%
\pgfsetbuttcap%
\pgfsetroundjoin%
\definecolor{currentfill}{rgb}{0.000000,0.000000,0.000000}%
\pgfsetfillcolor{currentfill}%
\pgfsetlinewidth{0.602250pt}%
\definecolor{currentstroke}{rgb}{0.000000,0.000000,0.000000}%
\pgfsetstrokecolor{currentstroke}%
\pgfsetdash{}{0pt}%
\pgfsys@defobject{currentmarker}{\pgfqpoint{0.000000in}{-0.027778in}}{\pgfqpoint{0.000000in}{0.000000in}}{%
\pgfpathmoveto{\pgfqpoint{0.000000in}{0.000000in}}%
\pgfpathlineto{\pgfqpoint{0.000000in}{-0.027778in}}%
\pgfusepath{stroke,fill}%
}%
\begin{pgfscope}%
\pgfsys@transformshift{1.680709in}{0.417642in}%
\pgfsys@useobject{currentmarker}{}%
\end{pgfscope}%
\end{pgfscope}%
\begin{pgfscope}%
\pgfpathrectangle{\pgfqpoint{0.514278in}{0.417642in}}{\pgfqpoint{1.884052in}{1.370688in}}%
\pgfusepath{clip}%
\pgfsetrectcap%
\pgfsetroundjoin%
\pgfsetlinewidth{0.803000pt}%
\definecolor{currentstroke}{rgb}{0.850000,0.850000,0.850000}%
\pgfsetstrokecolor{currentstroke}%
\pgfsetdash{}{0pt}%
\pgfpathmoveto{\pgfqpoint{1.745337in}{0.417642in}}%
\pgfpathlineto{\pgfqpoint{1.745337in}{1.788330in}}%
\pgfusepath{stroke}%
\end{pgfscope}%
\begin{pgfscope}%
\pgfsetbuttcap%
\pgfsetroundjoin%
\definecolor{currentfill}{rgb}{0.000000,0.000000,0.000000}%
\pgfsetfillcolor{currentfill}%
\pgfsetlinewidth{0.602250pt}%
\definecolor{currentstroke}{rgb}{0.000000,0.000000,0.000000}%
\pgfsetstrokecolor{currentstroke}%
\pgfsetdash{}{0pt}%
\pgfsys@defobject{currentmarker}{\pgfqpoint{0.000000in}{-0.027778in}}{\pgfqpoint{0.000000in}{0.000000in}}{%
\pgfpathmoveto{\pgfqpoint{0.000000in}{0.000000in}}%
\pgfpathlineto{\pgfqpoint{0.000000in}{-0.027778in}}%
\pgfusepath{stroke,fill}%
}%
\begin{pgfscope}%
\pgfsys@transformshift{1.745337in}{0.417642in}%
\pgfsys@useobject{currentmarker}{}%
\end{pgfscope}%
\end{pgfscope}%
\begin{pgfscope}%
\pgfpathrectangle{\pgfqpoint{0.514278in}{0.417642in}}{\pgfqpoint{1.884052in}{1.370688in}}%
\pgfusepath{clip}%
\pgfsetrectcap%
\pgfsetroundjoin%
\pgfsetlinewidth{0.803000pt}%
\definecolor{currentstroke}{rgb}{0.850000,0.850000,0.850000}%
\pgfsetstrokecolor{currentstroke}%
\pgfsetdash{}{0pt}%
\pgfpathmoveto{\pgfqpoint{1.795466in}{0.417642in}}%
\pgfpathlineto{\pgfqpoint{1.795466in}{1.788330in}}%
\pgfusepath{stroke}%
\end{pgfscope}%
\begin{pgfscope}%
\pgfsetbuttcap%
\pgfsetroundjoin%
\definecolor{currentfill}{rgb}{0.000000,0.000000,0.000000}%
\pgfsetfillcolor{currentfill}%
\pgfsetlinewidth{0.602250pt}%
\definecolor{currentstroke}{rgb}{0.000000,0.000000,0.000000}%
\pgfsetstrokecolor{currentstroke}%
\pgfsetdash{}{0pt}%
\pgfsys@defobject{currentmarker}{\pgfqpoint{0.000000in}{-0.027778in}}{\pgfqpoint{0.000000in}{0.000000in}}{%
\pgfpathmoveto{\pgfqpoint{0.000000in}{0.000000in}}%
\pgfpathlineto{\pgfqpoint{0.000000in}{-0.027778in}}%
\pgfusepath{stroke,fill}%
}%
\begin{pgfscope}%
\pgfsys@transformshift{1.795466in}{0.417642in}%
\pgfsys@useobject{currentmarker}{}%
\end{pgfscope}%
\end{pgfscope}%
\begin{pgfscope}%
\pgfpathrectangle{\pgfqpoint{0.514278in}{0.417642in}}{\pgfqpoint{1.884052in}{1.370688in}}%
\pgfusepath{clip}%
\pgfsetrectcap%
\pgfsetroundjoin%
\pgfsetlinewidth{0.803000pt}%
\definecolor{currentstroke}{rgb}{0.850000,0.850000,0.850000}%
\pgfsetstrokecolor{currentstroke}%
\pgfsetdash{}{0pt}%
\pgfpathmoveto{\pgfqpoint{1.836425in}{0.417642in}}%
\pgfpathlineto{\pgfqpoint{1.836425in}{1.788330in}}%
\pgfusepath{stroke}%
\end{pgfscope}%
\begin{pgfscope}%
\pgfsetbuttcap%
\pgfsetroundjoin%
\definecolor{currentfill}{rgb}{0.000000,0.000000,0.000000}%
\pgfsetfillcolor{currentfill}%
\pgfsetlinewidth{0.602250pt}%
\definecolor{currentstroke}{rgb}{0.000000,0.000000,0.000000}%
\pgfsetstrokecolor{currentstroke}%
\pgfsetdash{}{0pt}%
\pgfsys@defobject{currentmarker}{\pgfqpoint{0.000000in}{-0.027778in}}{\pgfqpoint{0.000000in}{0.000000in}}{%
\pgfpathmoveto{\pgfqpoint{0.000000in}{0.000000in}}%
\pgfpathlineto{\pgfqpoint{0.000000in}{-0.027778in}}%
\pgfusepath{stroke,fill}%
}%
\begin{pgfscope}%
\pgfsys@transformshift{1.836425in}{0.417642in}%
\pgfsys@useobject{currentmarker}{}%
\end{pgfscope}%
\end{pgfscope}%
\begin{pgfscope}%
\pgfpathrectangle{\pgfqpoint{0.514278in}{0.417642in}}{\pgfqpoint{1.884052in}{1.370688in}}%
\pgfusepath{clip}%
\pgfsetrectcap%
\pgfsetroundjoin%
\pgfsetlinewidth{0.803000pt}%
\definecolor{currentstroke}{rgb}{0.850000,0.850000,0.850000}%
\pgfsetstrokecolor{currentstroke}%
\pgfsetdash{}{0pt}%
\pgfpathmoveto{\pgfqpoint{1.871056in}{0.417642in}}%
\pgfpathlineto{\pgfqpoint{1.871056in}{1.788330in}}%
\pgfusepath{stroke}%
\end{pgfscope}%
\begin{pgfscope}%
\pgfsetbuttcap%
\pgfsetroundjoin%
\definecolor{currentfill}{rgb}{0.000000,0.000000,0.000000}%
\pgfsetfillcolor{currentfill}%
\pgfsetlinewidth{0.602250pt}%
\definecolor{currentstroke}{rgb}{0.000000,0.000000,0.000000}%
\pgfsetstrokecolor{currentstroke}%
\pgfsetdash{}{0pt}%
\pgfsys@defobject{currentmarker}{\pgfqpoint{0.000000in}{-0.027778in}}{\pgfqpoint{0.000000in}{0.000000in}}{%
\pgfpathmoveto{\pgfqpoint{0.000000in}{0.000000in}}%
\pgfpathlineto{\pgfqpoint{0.000000in}{-0.027778in}}%
\pgfusepath{stroke,fill}%
}%
\begin{pgfscope}%
\pgfsys@transformshift{1.871056in}{0.417642in}%
\pgfsys@useobject{currentmarker}{}%
\end{pgfscope}%
\end{pgfscope}%
\begin{pgfscope}%
\pgfpathrectangle{\pgfqpoint{0.514278in}{0.417642in}}{\pgfqpoint{1.884052in}{1.370688in}}%
\pgfusepath{clip}%
\pgfsetrectcap%
\pgfsetroundjoin%
\pgfsetlinewidth{0.803000pt}%
\definecolor{currentstroke}{rgb}{0.850000,0.850000,0.850000}%
\pgfsetstrokecolor{currentstroke}%
\pgfsetdash{}{0pt}%
\pgfpathmoveto{\pgfqpoint{1.901054in}{0.417642in}}%
\pgfpathlineto{\pgfqpoint{1.901054in}{1.788330in}}%
\pgfusepath{stroke}%
\end{pgfscope}%
\begin{pgfscope}%
\pgfsetbuttcap%
\pgfsetroundjoin%
\definecolor{currentfill}{rgb}{0.000000,0.000000,0.000000}%
\pgfsetfillcolor{currentfill}%
\pgfsetlinewidth{0.602250pt}%
\definecolor{currentstroke}{rgb}{0.000000,0.000000,0.000000}%
\pgfsetstrokecolor{currentstroke}%
\pgfsetdash{}{0pt}%
\pgfsys@defobject{currentmarker}{\pgfqpoint{0.000000in}{-0.027778in}}{\pgfqpoint{0.000000in}{0.000000in}}{%
\pgfpathmoveto{\pgfqpoint{0.000000in}{0.000000in}}%
\pgfpathlineto{\pgfqpoint{0.000000in}{-0.027778in}}%
\pgfusepath{stroke,fill}%
}%
\begin{pgfscope}%
\pgfsys@transformshift{1.901054in}{0.417642in}%
\pgfsys@useobject{currentmarker}{}%
\end{pgfscope}%
\end{pgfscope}%
\begin{pgfscope}%
\pgfpathrectangle{\pgfqpoint{0.514278in}{0.417642in}}{\pgfqpoint{1.884052in}{1.370688in}}%
\pgfusepath{clip}%
\pgfsetrectcap%
\pgfsetroundjoin%
\pgfsetlinewidth{0.803000pt}%
\definecolor{currentstroke}{rgb}{0.850000,0.850000,0.850000}%
\pgfsetstrokecolor{currentstroke}%
\pgfsetdash{}{0pt}%
\pgfpathmoveto{\pgfqpoint{1.927514in}{0.417642in}}%
\pgfpathlineto{\pgfqpoint{1.927514in}{1.788330in}}%
\pgfusepath{stroke}%
\end{pgfscope}%
\begin{pgfscope}%
\pgfsetbuttcap%
\pgfsetroundjoin%
\definecolor{currentfill}{rgb}{0.000000,0.000000,0.000000}%
\pgfsetfillcolor{currentfill}%
\pgfsetlinewidth{0.602250pt}%
\definecolor{currentstroke}{rgb}{0.000000,0.000000,0.000000}%
\pgfsetstrokecolor{currentstroke}%
\pgfsetdash{}{0pt}%
\pgfsys@defobject{currentmarker}{\pgfqpoint{0.000000in}{-0.027778in}}{\pgfqpoint{0.000000in}{0.000000in}}{%
\pgfpathmoveto{\pgfqpoint{0.000000in}{0.000000in}}%
\pgfpathlineto{\pgfqpoint{0.000000in}{-0.027778in}}%
\pgfusepath{stroke,fill}%
}%
\begin{pgfscope}%
\pgfsys@transformshift{1.927514in}{0.417642in}%
\pgfsys@useobject{currentmarker}{}%
\end{pgfscope}%
\end{pgfscope}%
\begin{pgfscope}%
\pgfpathrectangle{\pgfqpoint{0.514278in}{0.417642in}}{\pgfqpoint{1.884052in}{1.370688in}}%
\pgfusepath{clip}%
\pgfsetrectcap%
\pgfsetroundjoin%
\pgfsetlinewidth{0.803000pt}%
\definecolor{currentstroke}{rgb}{0.850000,0.850000,0.850000}%
\pgfsetstrokecolor{currentstroke}%
\pgfsetdash{}{0pt}%
\pgfpathmoveto{\pgfqpoint{2.106900in}{0.417642in}}%
\pgfpathlineto{\pgfqpoint{2.106900in}{1.788330in}}%
\pgfusepath{stroke}%
\end{pgfscope}%
\begin{pgfscope}%
\pgfsetbuttcap%
\pgfsetroundjoin%
\definecolor{currentfill}{rgb}{0.000000,0.000000,0.000000}%
\pgfsetfillcolor{currentfill}%
\pgfsetlinewidth{0.602250pt}%
\definecolor{currentstroke}{rgb}{0.000000,0.000000,0.000000}%
\pgfsetstrokecolor{currentstroke}%
\pgfsetdash{}{0pt}%
\pgfsys@defobject{currentmarker}{\pgfqpoint{0.000000in}{-0.027778in}}{\pgfqpoint{0.000000in}{0.000000in}}{%
\pgfpathmoveto{\pgfqpoint{0.000000in}{0.000000in}}%
\pgfpathlineto{\pgfqpoint{0.000000in}{-0.027778in}}%
\pgfusepath{stroke,fill}%
}%
\begin{pgfscope}%
\pgfsys@transformshift{2.106900in}{0.417642in}%
\pgfsys@useobject{currentmarker}{}%
\end{pgfscope}%
\end{pgfscope}%
\begin{pgfscope}%
\pgfpathrectangle{\pgfqpoint{0.514278in}{0.417642in}}{\pgfqpoint{1.884052in}{1.370688in}}%
\pgfusepath{clip}%
\pgfsetrectcap%
\pgfsetroundjoin%
\pgfsetlinewidth{0.803000pt}%
\definecolor{currentstroke}{rgb}{0.850000,0.850000,0.850000}%
\pgfsetstrokecolor{currentstroke}%
\pgfsetdash{}{0pt}%
\pgfpathmoveto{\pgfqpoint{2.197988in}{0.417642in}}%
\pgfpathlineto{\pgfqpoint{2.197988in}{1.788330in}}%
\pgfusepath{stroke}%
\end{pgfscope}%
\begin{pgfscope}%
\pgfsetbuttcap%
\pgfsetroundjoin%
\definecolor{currentfill}{rgb}{0.000000,0.000000,0.000000}%
\pgfsetfillcolor{currentfill}%
\pgfsetlinewidth{0.602250pt}%
\definecolor{currentstroke}{rgb}{0.000000,0.000000,0.000000}%
\pgfsetstrokecolor{currentstroke}%
\pgfsetdash{}{0pt}%
\pgfsys@defobject{currentmarker}{\pgfqpoint{0.000000in}{-0.027778in}}{\pgfqpoint{0.000000in}{0.000000in}}{%
\pgfpathmoveto{\pgfqpoint{0.000000in}{0.000000in}}%
\pgfpathlineto{\pgfqpoint{0.000000in}{-0.027778in}}%
\pgfusepath{stroke,fill}%
}%
\begin{pgfscope}%
\pgfsys@transformshift{2.197988in}{0.417642in}%
\pgfsys@useobject{currentmarker}{}%
\end{pgfscope}%
\end{pgfscope}%
\begin{pgfscope}%
\pgfpathrectangle{\pgfqpoint{0.514278in}{0.417642in}}{\pgfqpoint{1.884052in}{1.370688in}}%
\pgfusepath{clip}%
\pgfsetrectcap%
\pgfsetroundjoin%
\pgfsetlinewidth{0.803000pt}%
\definecolor{currentstroke}{rgb}{0.850000,0.850000,0.850000}%
\pgfsetstrokecolor{currentstroke}%
\pgfsetdash{}{0pt}%
\pgfpathmoveto{\pgfqpoint{2.262617in}{0.417642in}}%
\pgfpathlineto{\pgfqpoint{2.262617in}{1.788330in}}%
\pgfusepath{stroke}%
\end{pgfscope}%
\begin{pgfscope}%
\pgfsetbuttcap%
\pgfsetroundjoin%
\definecolor{currentfill}{rgb}{0.000000,0.000000,0.000000}%
\pgfsetfillcolor{currentfill}%
\pgfsetlinewidth{0.602250pt}%
\definecolor{currentstroke}{rgb}{0.000000,0.000000,0.000000}%
\pgfsetstrokecolor{currentstroke}%
\pgfsetdash{}{0pt}%
\pgfsys@defobject{currentmarker}{\pgfqpoint{0.000000in}{-0.027778in}}{\pgfqpoint{0.000000in}{0.000000in}}{%
\pgfpathmoveto{\pgfqpoint{0.000000in}{0.000000in}}%
\pgfpathlineto{\pgfqpoint{0.000000in}{-0.027778in}}%
\pgfusepath{stroke,fill}%
}%
\begin{pgfscope}%
\pgfsys@transformshift{2.262617in}{0.417642in}%
\pgfsys@useobject{currentmarker}{}%
\end{pgfscope}%
\end{pgfscope}%
\begin{pgfscope}%
\pgfpathrectangle{\pgfqpoint{0.514278in}{0.417642in}}{\pgfqpoint{1.884052in}{1.370688in}}%
\pgfusepath{clip}%
\pgfsetrectcap%
\pgfsetroundjoin%
\pgfsetlinewidth{0.803000pt}%
\definecolor{currentstroke}{rgb}{0.850000,0.850000,0.850000}%
\pgfsetstrokecolor{currentstroke}%
\pgfsetdash{}{0pt}%
\pgfpathmoveto{\pgfqpoint{2.312746in}{0.417642in}}%
\pgfpathlineto{\pgfqpoint{2.312746in}{1.788330in}}%
\pgfusepath{stroke}%
\end{pgfscope}%
\begin{pgfscope}%
\pgfsetbuttcap%
\pgfsetroundjoin%
\definecolor{currentfill}{rgb}{0.000000,0.000000,0.000000}%
\pgfsetfillcolor{currentfill}%
\pgfsetlinewidth{0.602250pt}%
\definecolor{currentstroke}{rgb}{0.000000,0.000000,0.000000}%
\pgfsetstrokecolor{currentstroke}%
\pgfsetdash{}{0pt}%
\pgfsys@defobject{currentmarker}{\pgfqpoint{0.000000in}{-0.027778in}}{\pgfqpoint{0.000000in}{0.000000in}}{%
\pgfpathmoveto{\pgfqpoint{0.000000in}{0.000000in}}%
\pgfpathlineto{\pgfqpoint{0.000000in}{-0.027778in}}%
\pgfusepath{stroke,fill}%
}%
\begin{pgfscope}%
\pgfsys@transformshift{2.312746in}{0.417642in}%
\pgfsys@useobject{currentmarker}{}%
\end{pgfscope}%
\end{pgfscope}%
\begin{pgfscope}%
\pgfpathrectangle{\pgfqpoint{0.514278in}{0.417642in}}{\pgfqpoint{1.884052in}{1.370688in}}%
\pgfusepath{clip}%
\pgfsetrectcap%
\pgfsetroundjoin%
\pgfsetlinewidth{0.803000pt}%
\definecolor{currentstroke}{rgb}{0.850000,0.850000,0.850000}%
\pgfsetstrokecolor{currentstroke}%
\pgfsetdash{}{0pt}%
\pgfpathmoveto{\pgfqpoint{2.353705in}{0.417642in}}%
\pgfpathlineto{\pgfqpoint{2.353705in}{1.788330in}}%
\pgfusepath{stroke}%
\end{pgfscope}%
\begin{pgfscope}%
\pgfsetbuttcap%
\pgfsetroundjoin%
\definecolor{currentfill}{rgb}{0.000000,0.000000,0.000000}%
\pgfsetfillcolor{currentfill}%
\pgfsetlinewidth{0.602250pt}%
\definecolor{currentstroke}{rgb}{0.000000,0.000000,0.000000}%
\pgfsetstrokecolor{currentstroke}%
\pgfsetdash{}{0pt}%
\pgfsys@defobject{currentmarker}{\pgfqpoint{0.000000in}{-0.027778in}}{\pgfqpoint{0.000000in}{0.000000in}}{%
\pgfpathmoveto{\pgfqpoint{0.000000in}{0.000000in}}%
\pgfpathlineto{\pgfqpoint{0.000000in}{-0.027778in}}%
\pgfusepath{stroke,fill}%
}%
\begin{pgfscope}%
\pgfsys@transformshift{2.353705in}{0.417642in}%
\pgfsys@useobject{currentmarker}{}%
\end{pgfscope}%
\end{pgfscope}%
\begin{pgfscope}%
\pgfpathrectangle{\pgfqpoint{0.514278in}{0.417642in}}{\pgfqpoint{1.884052in}{1.370688in}}%
\pgfusepath{clip}%
\pgfsetrectcap%
\pgfsetroundjoin%
\pgfsetlinewidth{0.803000pt}%
\definecolor{currentstroke}{rgb}{0.850000,0.850000,0.850000}%
\pgfsetstrokecolor{currentstroke}%
\pgfsetdash{}{0pt}%
\pgfpathmoveto{\pgfqpoint{2.388335in}{0.417642in}}%
\pgfpathlineto{\pgfqpoint{2.388335in}{1.788330in}}%
\pgfusepath{stroke}%
\end{pgfscope}%
\begin{pgfscope}%
\pgfsetbuttcap%
\pgfsetroundjoin%
\definecolor{currentfill}{rgb}{0.000000,0.000000,0.000000}%
\pgfsetfillcolor{currentfill}%
\pgfsetlinewidth{0.602250pt}%
\definecolor{currentstroke}{rgb}{0.000000,0.000000,0.000000}%
\pgfsetstrokecolor{currentstroke}%
\pgfsetdash{}{0pt}%
\pgfsys@defobject{currentmarker}{\pgfqpoint{0.000000in}{-0.027778in}}{\pgfqpoint{0.000000in}{0.000000in}}{%
\pgfpathmoveto{\pgfqpoint{0.000000in}{0.000000in}}%
\pgfpathlineto{\pgfqpoint{0.000000in}{-0.027778in}}%
\pgfusepath{stroke,fill}%
}%
\begin{pgfscope}%
\pgfsys@transformshift{2.388335in}{0.417642in}%
\pgfsys@useobject{currentmarker}{}%
\end{pgfscope}%
\end{pgfscope}%
\begin{pgfscope}%
\definecolor{textcolor}{rgb}{0.000000,0.000000,0.000000}%
\pgfsetstrokecolor{textcolor}%
\pgfsetfillcolor{textcolor}%
\pgftext[x=1.456304in,y=0.165003in,,top]{\color{textcolor}\rmfamily\fontsize{10.000000}{12.000000}\selectfont Frequency in \(\displaystyle \unit{\Hz}\)}%
\end{pgfscope}%
\begin{pgfscope}%
\pgfpathrectangle{\pgfqpoint{0.514278in}{0.417642in}}{\pgfqpoint{1.884052in}{1.370688in}}%
\pgfusepath{clip}%
\pgfsetrectcap%
\pgfsetroundjoin%
\pgfsetlinewidth{0.803000pt}%
\definecolor{currentstroke}{rgb}{0.450000,0.450000,0.450000}%
\pgfsetstrokecolor{currentstroke}%
\pgfsetdash{}{0pt}%
\pgfpathmoveto{\pgfqpoint{0.514278in}{0.640555in}}%
\pgfpathlineto{\pgfqpoint{2.398330in}{0.640555in}}%
\pgfusepath{stroke}%
\end{pgfscope}%
\begin{pgfscope}%
\pgfsetbuttcap%
\pgfsetroundjoin%
\definecolor{currentfill}{rgb}{0.000000,0.000000,0.000000}%
\pgfsetfillcolor{currentfill}%
\pgfsetlinewidth{0.803000pt}%
\definecolor{currentstroke}{rgb}{0.000000,0.000000,0.000000}%
\pgfsetstrokecolor{currentstroke}%
\pgfsetdash{}{0pt}%
\pgfsys@defobject{currentmarker}{\pgfqpoint{-0.048611in}{0.000000in}}{\pgfqpoint{-0.000000in}{0.000000in}}{%
\pgfpathmoveto{\pgfqpoint{-0.000000in}{0.000000in}}%
\pgfpathlineto{\pgfqpoint{-0.048611in}{0.000000in}}%
\pgfusepath{stroke,fill}%
}%
\begin{pgfscope}%
\pgfsys@transformshift{0.514278in}{0.640555in}%
\pgfsys@useobject{currentmarker}{}%
\end{pgfscope}%
\end{pgfscope}%
\begin{pgfscope}%
\definecolor{textcolor}{rgb}{0.000000,0.000000,0.000000}%
\pgfsetstrokecolor{textcolor}%
\pgfsetfillcolor{textcolor}%
\pgftext[x=0.241129in, y=0.601402in, left, base]{\color{textcolor}\rmfamily\fontsize{8.000000}{9.600000}\selectfont \(\displaystyle {10^{0}}\)}%
\end{pgfscope}%
\begin{pgfscope}%
\pgfpathrectangle{\pgfqpoint{0.514278in}{0.417642in}}{\pgfqpoint{1.884052in}{1.370688in}}%
\pgfusepath{clip}%
\pgfsetrectcap%
\pgfsetroundjoin%
\pgfsetlinewidth{0.803000pt}%
\definecolor{currentstroke}{rgb}{0.450000,0.450000,0.450000}%
\pgfsetstrokecolor{currentstroke}%
\pgfsetdash{}{0pt}%
\pgfpathmoveto{\pgfqpoint{0.514278in}{0.983227in}}%
\pgfpathlineto{\pgfqpoint{2.398330in}{0.983227in}}%
\pgfusepath{stroke}%
\end{pgfscope}%
\begin{pgfscope}%
\pgfsetbuttcap%
\pgfsetroundjoin%
\definecolor{currentfill}{rgb}{0.000000,0.000000,0.000000}%
\pgfsetfillcolor{currentfill}%
\pgfsetlinewidth{0.803000pt}%
\definecolor{currentstroke}{rgb}{0.000000,0.000000,0.000000}%
\pgfsetstrokecolor{currentstroke}%
\pgfsetdash{}{0pt}%
\pgfsys@defobject{currentmarker}{\pgfqpoint{-0.048611in}{0.000000in}}{\pgfqpoint{-0.000000in}{0.000000in}}{%
\pgfpathmoveto{\pgfqpoint{-0.000000in}{0.000000in}}%
\pgfpathlineto{\pgfqpoint{-0.048611in}{0.000000in}}%
\pgfusepath{stroke,fill}%
}%
\begin{pgfscope}%
\pgfsys@transformshift{0.514278in}{0.983227in}%
\pgfsys@useobject{currentmarker}{}%
\end{pgfscope}%
\end{pgfscope}%
\begin{pgfscope}%
\definecolor{textcolor}{rgb}{0.000000,0.000000,0.000000}%
\pgfsetstrokecolor{textcolor}%
\pgfsetfillcolor{textcolor}%
\pgftext[x=0.241129in, y=0.944074in, left, base]{\color{textcolor}\rmfamily\fontsize{8.000000}{9.600000}\selectfont \(\displaystyle {10^{2}}\)}%
\end{pgfscope}%
\begin{pgfscope}%
\pgfpathrectangle{\pgfqpoint{0.514278in}{0.417642in}}{\pgfqpoint{1.884052in}{1.370688in}}%
\pgfusepath{clip}%
\pgfsetrectcap%
\pgfsetroundjoin%
\pgfsetlinewidth{0.803000pt}%
\definecolor{currentstroke}{rgb}{0.450000,0.450000,0.450000}%
\pgfsetstrokecolor{currentstroke}%
\pgfsetdash{}{0pt}%
\pgfpathmoveto{\pgfqpoint{0.514278in}{1.325899in}}%
\pgfpathlineto{\pgfqpoint{2.398330in}{1.325899in}}%
\pgfusepath{stroke}%
\end{pgfscope}%
\begin{pgfscope}%
\pgfsetbuttcap%
\pgfsetroundjoin%
\definecolor{currentfill}{rgb}{0.000000,0.000000,0.000000}%
\pgfsetfillcolor{currentfill}%
\pgfsetlinewidth{0.803000pt}%
\definecolor{currentstroke}{rgb}{0.000000,0.000000,0.000000}%
\pgfsetstrokecolor{currentstroke}%
\pgfsetdash{}{0pt}%
\pgfsys@defobject{currentmarker}{\pgfqpoint{-0.048611in}{0.000000in}}{\pgfqpoint{-0.000000in}{0.000000in}}{%
\pgfpathmoveto{\pgfqpoint{-0.000000in}{0.000000in}}%
\pgfpathlineto{\pgfqpoint{-0.048611in}{0.000000in}}%
\pgfusepath{stroke,fill}%
}%
\begin{pgfscope}%
\pgfsys@transformshift{0.514278in}{1.325899in}%
\pgfsys@useobject{currentmarker}{}%
\end{pgfscope}%
\end{pgfscope}%
\begin{pgfscope}%
\definecolor{textcolor}{rgb}{0.000000,0.000000,0.000000}%
\pgfsetstrokecolor{textcolor}%
\pgfsetfillcolor{textcolor}%
\pgftext[x=0.241129in, y=1.286746in, left, base]{\color{textcolor}\rmfamily\fontsize{8.000000}{9.600000}\selectfont \(\displaystyle {10^{4}}\)}%
\end{pgfscope}%
\begin{pgfscope}%
\pgfpathrectangle{\pgfqpoint{0.514278in}{0.417642in}}{\pgfqpoint{1.884052in}{1.370688in}}%
\pgfusepath{clip}%
\pgfsetrectcap%
\pgfsetroundjoin%
\pgfsetlinewidth{0.803000pt}%
\definecolor{currentstroke}{rgb}{0.450000,0.450000,0.450000}%
\pgfsetstrokecolor{currentstroke}%
\pgfsetdash{}{0pt}%
\pgfpathmoveto{\pgfqpoint{0.514278in}{1.668571in}}%
\pgfpathlineto{\pgfqpoint{2.398330in}{1.668571in}}%
\pgfusepath{stroke}%
\end{pgfscope}%
\begin{pgfscope}%
\pgfsetbuttcap%
\pgfsetroundjoin%
\definecolor{currentfill}{rgb}{0.000000,0.000000,0.000000}%
\pgfsetfillcolor{currentfill}%
\pgfsetlinewidth{0.803000pt}%
\definecolor{currentstroke}{rgb}{0.000000,0.000000,0.000000}%
\pgfsetstrokecolor{currentstroke}%
\pgfsetdash{}{0pt}%
\pgfsys@defobject{currentmarker}{\pgfqpoint{-0.048611in}{0.000000in}}{\pgfqpoint{-0.000000in}{0.000000in}}{%
\pgfpathmoveto{\pgfqpoint{-0.000000in}{0.000000in}}%
\pgfpathlineto{\pgfqpoint{-0.048611in}{0.000000in}}%
\pgfusepath{stroke,fill}%
}%
\begin{pgfscope}%
\pgfsys@transformshift{0.514278in}{1.668571in}%
\pgfsys@useobject{currentmarker}{}%
\end{pgfscope}%
\end{pgfscope}%
\begin{pgfscope}%
\definecolor{textcolor}{rgb}{0.000000,0.000000,0.000000}%
\pgfsetstrokecolor{textcolor}%
\pgfsetfillcolor{textcolor}%
\pgftext[x=0.241129in, y=1.629418in, left, base]{\color{textcolor}\rmfamily\fontsize{8.000000}{9.600000}\selectfont \(\displaystyle {10^{6}}\)}%
\end{pgfscope}%
\begin{pgfscope}%
\definecolor{textcolor}{rgb}{0.000000,0.000000,0.000000}%
\pgfsetstrokecolor{textcolor}%
\pgfsetfillcolor{textcolor}%
\pgftext[x=0.185574in,y=1.102986in,,bottom,rotate=90.000000]{\color{textcolor}\rmfamily\fontsize{10.000000}{12.000000}\selectfont  \(\displaystyle S_y(f)\) in \(\displaystyle \unit{1 \per \Hz}\)}%
\end{pgfscope}%
\begin{pgfscope}%
\pgfpathrectangle{\pgfqpoint{0.514278in}{0.417642in}}{\pgfqpoint{1.884052in}{1.370688in}}%
\pgfusepath{clip}%
\pgfsetbuttcap%
\pgfsetroundjoin%
\pgfsetlinewidth{1.505625pt}%
\definecolor{currentstroke}{rgb}{0.835294,0.368627,0.000000}%
\pgfsetstrokecolor{currentstroke}%
\pgfsetdash{{5.550000pt}{2.400000pt}}{0.000000pt}%
\pgfpathmoveto{\pgfqpoint{0.599917in}{1.738185in}}%
\pgfpathlineto{\pgfqpoint{2.312691in}{0.603558in}}%
\pgfpathlineto{\pgfqpoint{2.312691in}{0.603558in}}%
\pgfusepath{stroke}%
\end{pgfscope}%
\begin{pgfscope}%
\pgfpathrectangle{\pgfqpoint{0.514278in}{0.417642in}}{\pgfqpoint{1.884052in}{1.370688in}}%
\pgfusepath{clip}%
\pgfsetbuttcap%
\pgfsetroundjoin%
\definecolor{currentfill}{rgb}{0.835294,0.368627,0.000000}%
\pgfsetfillcolor{currentfill}%
\pgfsetlinewidth{1.003750pt}%
\definecolor{currentstroke}{rgb}{0.835294,0.368627,0.000000}%
\pgfsetstrokecolor{currentstroke}%
\pgfsetdash{}{0pt}%
\pgfsys@defobject{currentmarker}{\pgfqpoint{-0.006944in}{-0.006944in}}{\pgfqpoint{0.006944in}{0.006944in}}{%
\pgfpathmoveto{\pgfqpoint{0.000000in}{-0.006944in}}%
\pgfpathcurveto{\pgfqpoint{0.001842in}{-0.006944in}}{\pgfqpoint{0.003608in}{-0.006213in}}{\pgfqpoint{0.004910in}{-0.004910in}}%
\pgfpathcurveto{\pgfqpoint{0.006213in}{-0.003608in}}{\pgfqpoint{0.006944in}{-0.001842in}}{\pgfqpoint{0.006944in}{0.000000in}}%
\pgfpathcurveto{\pgfqpoint{0.006944in}{0.001842in}}{\pgfqpoint{0.006213in}{0.003608in}}{\pgfqpoint{0.004910in}{0.004910in}}%
\pgfpathcurveto{\pgfqpoint{0.003608in}{0.006213in}}{\pgfqpoint{0.001842in}{0.006944in}}{\pgfqpoint{0.000000in}{0.006944in}}%
\pgfpathcurveto{\pgfqpoint{-0.001842in}{0.006944in}}{\pgfqpoint{-0.003608in}{0.006213in}}{\pgfqpoint{-0.004910in}{0.004910in}}%
\pgfpathcurveto{\pgfqpoint{-0.006213in}{0.003608in}}{\pgfqpoint{-0.006944in}{0.001842in}}{\pgfqpoint{-0.006944in}{0.000000in}}%
\pgfpathcurveto{\pgfqpoint{-0.006944in}{-0.001842in}}{\pgfqpoint{-0.006213in}{-0.003608in}}{\pgfqpoint{-0.004910in}{-0.004910in}}%
\pgfpathcurveto{\pgfqpoint{-0.003608in}{-0.006213in}}{\pgfqpoint{-0.001842in}{-0.006944in}}{\pgfqpoint{0.000000in}{-0.006944in}}%
\pgfpathlineto{\pgfqpoint{0.000000in}{-0.006944in}}%
\pgfpathclose%
\pgfusepath{stroke,fill}%
}%
\begin{pgfscope}%
\pgfsys@transformshift{-226.701573in}{1.699596in}%
\pgfsys@useobject{currentmarker}{}%
\end{pgfscope}%
\begin{pgfscope}%
\pgfsys@transformshift{0.599917in}{1.767760in}%
\pgfsys@useobject{currentmarker}{}%
\end{pgfscope}%
\begin{pgfscope}%
\pgfsys@transformshift{0.755634in}{1.633926in}%
\pgfsys@useobject{currentmarker}{}%
\end{pgfscope}%
\begin{pgfscope}%
\pgfsys@transformshift{0.846722in}{1.542294in}%
\pgfsys@useobject{currentmarker}{}%
\end{pgfscope}%
\begin{pgfscope}%
\pgfsys@transformshift{0.911351in}{1.484904in}%
\pgfsys@useobject{currentmarker}{}%
\end{pgfscope}%
\begin{pgfscope}%
\pgfsys@transformshift{0.961480in}{1.467589in}%
\pgfsys@useobject{currentmarker}{}%
\end{pgfscope}%
\begin{pgfscope}%
\pgfsys@transformshift{1.002439in}{1.396364in}%
\pgfsys@useobject{currentmarker}{}%
\end{pgfscope}%
\begin{pgfscope}%
\pgfsys@transformshift{1.037069in}{1.455179in}%
\pgfsys@useobject{currentmarker}{}%
\end{pgfscope}%
\begin{pgfscope}%
\pgfsys@transformshift{1.067067in}{1.459825in}%
\pgfsys@useobject{currentmarker}{}%
\end{pgfscope}%
\begin{pgfscope}%
\pgfsys@transformshift{1.093527in}{1.432011in}%
\pgfsys@useobject{currentmarker}{}%
\end{pgfscope}%
\begin{pgfscope}%
\pgfsys@transformshift{1.117197in}{1.407185in}%
\pgfsys@useobject{currentmarker}{}%
\end{pgfscope}%
\begin{pgfscope}%
\pgfsys@transformshift{1.138608in}{1.363179in}%
\pgfsys@useobject{currentmarker}{}%
\end{pgfscope}%
\begin{pgfscope}%
\pgfsys@transformshift{1.158156in}{1.390307in}%
\pgfsys@useobject{currentmarker}{}%
\end{pgfscope}%
\begin{pgfscope}%
\pgfsys@transformshift{1.176137in}{1.358330in}%
\pgfsys@useobject{currentmarker}{}%
\end{pgfscope}%
\begin{pgfscope}%
\pgfsys@transformshift{1.192786in}{1.282640in}%
\pgfsys@useobject{currentmarker}{}%
\end{pgfscope}%
\begin{pgfscope}%
\pgfsys@transformshift{1.208285in}{1.342931in}%
\pgfsys@useobject{currentmarker}{}%
\end{pgfscope}%
\begin{pgfscope}%
\pgfsys@transformshift{1.222784in}{1.362450in}%
\pgfsys@useobject{currentmarker}{}%
\end{pgfscope}%
\begin{pgfscope}%
\pgfsys@transformshift{1.236403in}{1.334753in}%
\pgfsys@useobject{currentmarker}{}%
\end{pgfscope}%
\begin{pgfscope}%
\pgfsys@transformshift{1.249244in}{1.223887in}%
\pgfsys@useobject{currentmarker}{}%
\end{pgfscope}%
\begin{pgfscope}%
\pgfsys@transformshift{1.261390in}{1.267876in}%
\pgfsys@useobject{currentmarker}{}%
\end{pgfscope}%
\begin{pgfscope}%
\pgfsys@transformshift{1.272914in}{1.329337in}%
\pgfsys@useobject{currentmarker}{}%
\end{pgfscope}%
\begin{pgfscope}%
\pgfsys@transformshift{1.283874in}{1.343124in}%
\pgfsys@useobject{currentmarker}{}%
\end{pgfscope}%
\begin{pgfscope}%
\pgfsys@transformshift{1.294325in}{1.301182in}%
\pgfsys@useobject{currentmarker}{}%
\end{pgfscope}%
\begin{pgfscope}%
\pgfsys@transformshift{1.304311in}{1.275590in}%
\pgfsys@useobject{currentmarker}{}%
\end{pgfscope}%
\begin{pgfscope}%
\pgfsys@transformshift{1.313872in}{1.224897in}%
\pgfsys@useobject{currentmarker}{}%
\end{pgfscope}%
\begin{pgfscope}%
\pgfsys@transformshift{1.323043in}{1.244226in}%
\pgfsys@useobject{currentmarker}{}%
\end{pgfscope}%
\begin{pgfscope}%
\pgfsys@transformshift{1.331854in}{1.221070in}%
\pgfsys@useobject{currentmarker}{}%
\end{pgfscope}%
\begin{pgfscope}%
\pgfsys@transformshift{1.340333in}{1.212021in}%
\pgfsys@useobject{currentmarker}{}%
\end{pgfscope}%
\begin{pgfscope}%
\pgfsys@transformshift{1.348503in}{1.255859in}%
\pgfsys@useobject{currentmarker}{}%
\end{pgfscope}%
\begin{pgfscope}%
\pgfsys@transformshift{1.356386in}{1.259043in}%
\pgfsys@useobject{currentmarker}{}%
\end{pgfscope}%
\begin{pgfscope}%
\pgfsys@transformshift{1.364002in}{1.213137in}%
\pgfsys@useobject{currentmarker}{}%
\end{pgfscope}%
\begin{pgfscope}%
\pgfsys@transformshift{1.371368in}{1.234630in}%
\pgfsys@useobject{currentmarker}{}%
\end{pgfscope}%
\begin{pgfscope}%
\pgfsys@transformshift{1.378501in}{1.216343in}%
\pgfsys@useobject{currentmarker}{}%
\end{pgfscope}%
\begin{pgfscope}%
\pgfsys@transformshift{1.385414in}{1.197913in}%
\pgfsys@useobject{currentmarker}{}%
\end{pgfscope}%
\begin{pgfscope}%
\pgfsys@transformshift{1.392120in}{1.232938in}%
\pgfsys@useobject{currentmarker}{}%
\end{pgfscope}%
\begin{pgfscope}%
\pgfsys@transformshift{1.398632in}{1.191697in}%
\pgfsys@useobject{currentmarker}{}%
\end{pgfscope}%
\begin{pgfscope}%
\pgfsys@transformshift{1.404961in}{1.221783in}%
\pgfsys@useobject{currentmarker}{}%
\end{pgfscope}%
\begin{pgfscope}%
\pgfsys@transformshift{1.411116in}{1.208746in}%
\pgfsys@useobject{currentmarker}{}%
\end{pgfscope}%
\begin{pgfscope}%
\pgfsys@transformshift{1.417107in}{1.176183in}%
\pgfsys@useobject{currentmarker}{}%
\end{pgfscope}%
\begin{pgfscope}%
\pgfsys@transformshift{1.422943in}{1.131458in}%
\pgfsys@useobject{currentmarker}{}%
\end{pgfscope}%
\begin{pgfscope}%
\pgfsys@transformshift{1.428630in}{1.176332in}%
\pgfsys@useobject{currentmarker}{}%
\end{pgfscope}%
\begin{pgfscope}%
\pgfsys@transformshift{1.434178in}{1.202293in}%
\pgfsys@useobject{currentmarker}{}%
\end{pgfscope}%
\begin{pgfscope}%
\pgfsys@transformshift{1.439591in}{1.213066in}%
\pgfsys@useobject{currentmarker}{}%
\end{pgfscope}%
\begin{pgfscope}%
\pgfsys@transformshift{1.444877in}{1.227325in}%
\pgfsys@useobject{currentmarker}{}%
\end{pgfscope}%
\begin{pgfscope}%
\pgfsys@transformshift{1.450042in}{1.175888in}%
\pgfsys@useobject{currentmarker}{}%
\end{pgfscope}%
\begin{pgfscope}%
\pgfsys@transformshift{1.455090in}{1.095359in}%
\pgfsys@useobject{currentmarker}{}%
\end{pgfscope}%
\begin{pgfscope}%
\pgfsys@transformshift{1.460028in}{1.120846in}%
\pgfsys@useobject{currentmarker}{}%
\end{pgfscope}%
\begin{pgfscope}%
\pgfsys@transformshift{1.464859in}{1.142584in}%
\pgfsys@useobject{currentmarker}{}%
\end{pgfscope}%
\begin{pgfscope}%
\pgfsys@transformshift{1.469589in}{1.168559in}%
\pgfsys@useobject{currentmarker}{}%
\end{pgfscope}%
\begin{pgfscope}%
\pgfsys@transformshift{1.474221in}{1.154005in}%
\pgfsys@useobject{currentmarker}{}%
\end{pgfscope}%
\begin{pgfscope}%
\pgfsys@transformshift{1.478760in}{1.105622in}%
\pgfsys@useobject{currentmarker}{}%
\end{pgfscope}%
\begin{pgfscope}%
\pgfsys@transformshift{1.483209in}{1.045925in}%
\pgfsys@useobject{currentmarker}{}%
\end{pgfscope}%
\begin{pgfscope}%
\pgfsys@transformshift{1.487571in}{1.082888in}%
\pgfsys@useobject{currentmarker}{}%
\end{pgfscope}%
\begin{pgfscope}%
\pgfsys@transformshift{1.491850in}{1.148928in}%
\pgfsys@useobject{currentmarker}{}%
\end{pgfscope}%
\begin{pgfscope}%
\pgfsys@transformshift{1.496049in}{1.150320in}%
\pgfsys@useobject{currentmarker}{}%
\end{pgfscope}%
\begin{pgfscope}%
\pgfsys@transformshift{1.500172in}{1.159522in}%
\pgfsys@useobject{currentmarker}{}%
\end{pgfscope}%
\begin{pgfscope}%
\pgfsys@transformshift{1.504219in}{1.163355in}%
\pgfsys@useobject{currentmarker}{}%
\end{pgfscope}%
\begin{pgfscope}%
\pgfsys@transformshift{1.508196in}{1.173362in}%
\pgfsys@useobject{currentmarker}{}%
\end{pgfscope}%
\begin{pgfscope}%
\pgfsys@transformshift{1.512103in}{1.164352in}%
\pgfsys@useobject{currentmarker}{}%
\end{pgfscope}%
\begin{pgfscope}%
\pgfsys@transformshift{1.515943in}{1.141812in}%
\pgfsys@useobject{currentmarker}{}%
\end{pgfscope}%
\begin{pgfscope}%
\pgfsys@transformshift{1.519719in}{1.120098in}%
\pgfsys@useobject{currentmarker}{}%
\end{pgfscope}%
\begin{pgfscope}%
\pgfsys@transformshift{1.523432in}{1.112526in}%
\pgfsys@useobject{currentmarker}{}%
\end{pgfscope}%
\begin{pgfscope}%
\pgfsys@transformshift{1.527085in}{1.055989in}%
\pgfsys@useobject{currentmarker}{}%
\end{pgfscope}%
\begin{pgfscope}%
\pgfsys@transformshift{1.530680in}{1.062603in}%
\pgfsys@useobject{currentmarker}{}%
\end{pgfscope}%
\begin{pgfscope}%
\pgfsys@transformshift{1.534217in}{1.096690in}%
\pgfsys@useobject{currentmarker}{}%
\end{pgfscope}%
\begin{pgfscope}%
\pgfsys@transformshift{1.537700in}{1.068067in}%
\pgfsys@useobject{currentmarker}{}%
\end{pgfscope}%
\begin{pgfscope}%
\pgfsys@transformshift{1.541130in}{1.065379in}%
\pgfsys@useobject{currentmarker}{}%
\end{pgfscope}%
\begin{pgfscope}%
\pgfsys@transformshift{1.544509in}{1.095491in}%
\pgfsys@useobject{currentmarker}{}%
\end{pgfscope}%
\begin{pgfscope}%
\pgfsys@transformshift{1.547837in}{1.127129in}%
\pgfsys@useobject{currentmarker}{}%
\end{pgfscope}%
\begin{pgfscope}%
\pgfsys@transformshift{1.551117in}{1.117168in}%
\pgfsys@useobject{currentmarker}{}%
\end{pgfscope}%
\begin{pgfscope}%
\pgfsys@transformshift{1.554349in}{1.048005in}%
\pgfsys@useobject{currentmarker}{}%
\end{pgfscope}%
\begin{pgfscope}%
\pgfsys@transformshift{1.557536in}{1.069736in}%
\pgfsys@useobject{currentmarker}{}%
\end{pgfscope}%
\begin{pgfscope}%
\pgfsys@transformshift{1.560678in}{1.118483in}%
\pgfsys@useobject{currentmarker}{}%
\end{pgfscope}%
\begin{pgfscope}%
\pgfsys@transformshift{1.563776in}{1.120149in}%
\pgfsys@useobject{currentmarker}{}%
\end{pgfscope}%
\begin{pgfscope}%
\pgfsys@transformshift{1.566833in}{1.081543in}%
\pgfsys@useobject{currentmarker}{}%
\end{pgfscope}%
\begin{pgfscope}%
\pgfsys@transformshift{1.569848in}{1.084658in}%
\pgfsys@useobject{currentmarker}{}%
\end{pgfscope}%
\begin{pgfscope}%
\pgfsys@transformshift{1.572824in}{1.103899in}%
\pgfsys@useobject{currentmarker}{}%
\end{pgfscope}%
\begin{pgfscope}%
\pgfsys@transformshift{1.575761in}{1.081469in}%
\pgfsys@useobject{currentmarker}{}%
\end{pgfscope}%
\begin{pgfscope}%
\pgfsys@transformshift{1.578659in}{1.031169in}%
\pgfsys@useobject{currentmarker}{}%
\end{pgfscope}%
\begin{pgfscope}%
\pgfsys@transformshift{1.581521in}{1.037846in}%
\pgfsys@useobject{currentmarker}{}%
\end{pgfscope}%
\begin{pgfscope}%
\pgfsys@transformshift{1.584347in}{1.089595in}%
\pgfsys@useobject{currentmarker}{}%
\end{pgfscope}%
\begin{pgfscope}%
\pgfsys@transformshift{1.587138in}{1.077670in}%
\pgfsys@useobject{currentmarker}{}%
\end{pgfscope}%
\begin{pgfscope}%
\pgfsys@transformshift{1.589894in}{1.064079in}%
\pgfsys@useobject{currentmarker}{}%
\end{pgfscope}%
\begin{pgfscope}%
\pgfsys@transformshift{1.592617in}{1.043820in}%
\pgfsys@useobject{currentmarker}{}%
\end{pgfscope}%
\begin{pgfscope}%
\pgfsys@transformshift{1.595308in}{1.024211in}%
\pgfsys@useobject{currentmarker}{}%
\end{pgfscope}%
\begin{pgfscope}%
\pgfsys@transformshift{1.597966in}{1.021445in}%
\pgfsys@useobject{currentmarker}{}%
\end{pgfscope}%
\begin{pgfscope}%
\pgfsys@transformshift{1.600594in}{1.057507in}%
\pgfsys@useobject{currentmarker}{}%
\end{pgfscope}%
\begin{pgfscope}%
\pgfsys@transformshift{1.603191in}{1.020981in}%
\pgfsys@useobject{currentmarker}{}%
\end{pgfscope}%
\begin{pgfscope}%
\pgfsys@transformshift{1.605759in}{1.041287in}%
\pgfsys@useobject{currentmarker}{}%
\end{pgfscope}%
\begin{pgfscope}%
\pgfsys@transformshift{1.608297in}{0.994575in}%
\pgfsys@useobject{currentmarker}{}%
\end{pgfscope}%
\begin{pgfscope}%
\pgfsys@transformshift{1.610807in}{0.997938in}%
\pgfsys@useobject{currentmarker}{}%
\end{pgfscope}%
\begin{pgfscope}%
\pgfsys@transformshift{1.613290in}{1.066445in}%
\pgfsys@useobject{currentmarker}{}%
\end{pgfscope}%
\begin{pgfscope}%
\pgfsys@transformshift{1.615745in}{1.056958in}%
\pgfsys@useobject{currentmarker}{}%
\end{pgfscope}%
\begin{pgfscope}%
\pgfsys@transformshift{1.618173in}{1.063108in}%
\pgfsys@useobject{currentmarker}{}%
\end{pgfscope}%
\begin{pgfscope}%
\pgfsys@transformshift{1.620576in}{1.051093in}%
\pgfsys@useobject{currentmarker}{}%
\end{pgfscope}%
\begin{pgfscope}%
\pgfsys@transformshift{1.622953in}{1.030392in}%
\pgfsys@useobject{currentmarker}{}%
\end{pgfscope}%
\begin{pgfscope}%
\pgfsys@transformshift{1.625306in}{1.046458in}%
\pgfsys@useobject{currentmarker}{}%
\end{pgfscope}%
\begin{pgfscope}%
\pgfsys@transformshift{1.627634in}{1.031185in}%
\pgfsys@useobject{currentmarker}{}%
\end{pgfscope}%
\begin{pgfscope}%
\pgfsys@transformshift{1.629938in}{1.014596in}%
\pgfsys@useobject{currentmarker}{}%
\end{pgfscope}%
\begin{pgfscope}%
\pgfsys@transformshift{1.632219in}{1.033804in}%
\pgfsys@useobject{currentmarker}{}%
\end{pgfscope}%
\begin{pgfscope}%
\pgfsys@transformshift{1.634477in}{1.016011in}%
\pgfsys@useobject{currentmarker}{}%
\end{pgfscope}%
\begin{pgfscope}%
\pgfsys@transformshift{1.636712in}{0.987823in}%
\pgfsys@useobject{currentmarker}{}%
\end{pgfscope}%
\begin{pgfscope}%
\pgfsys@transformshift{1.638925in}{1.021058in}%
\pgfsys@useobject{currentmarker}{}%
\end{pgfscope}%
\begin{pgfscope}%
\pgfsys@transformshift{1.641117in}{1.027398in}%
\pgfsys@useobject{currentmarker}{}%
\end{pgfscope}%
\begin{pgfscope}%
\pgfsys@transformshift{1.643288in}{1.065967in}%
\pgfsys@useobject{currentmarker}{}%
\end{pgfscope}%
\begin{pgfscope}%
\pgfsys@transformshift{1.645437in}{1.029419in}%
\pgfsys@useobject{currentmarker}{}%
\end{pgfscope}%
\begin{pgfscope}%
\pgfsys@transformshift{1.647567in}{1.031442in}%
\pgfsys@useobject{currentmarker}{}%
\end{pgfscope}%
\begin{pgfscope}%
\pgfsys@transformshift{1.649676in}{1.042601in}%
\pgfsys@useobject{currentmarker}{}%
\end{pgfscope}%
\begin{pgfscope}%
\pgfsys@transformshift{1.651766in}{1.046952in}%
\pgfsys@useobject{currentmarker}{}%
\end{pgfscope}%
\begin{pgfscope}%
\pgfsys@transformshift{1.653837in}{1.062189in}%
\pgfsys@useobject{currentmarker}{}%
\end{pgfscope}%
\begin{pgfscope}%
\pgfsys@transformshift{1.655888in}{1.049449in}%
\pgfsys@useobject{currentmarker}{}%
\end{pgfscope}%
\begin{pgfscope}%
\pgfsys@transformshift{1.657921in}{1.013267in}%
\pgfsys@useobject{currentmarker}{}%
\end{pgfscope}%
\begin{pgfscope}%
\pgfsys@transformshift{1.659936in}{1.039645in}%
\pgfsys@useobject{currentmarker}{}%
\end{pgfscope}%
\begin{pgfscope}%
\pgfsys@transformshift{1.661933in}{1.020337in}%
\pgfsys@useobject{currentmarker}{}%
\end{pgfscope}%
\begin{pgfscope}%
\pgfsys@transformshift{1.663912in}{1.018009in}%
\pgfsys@useobject{currentmarker}{}%
\end{pgfscope}%
\begin{pgfscope}%
\pgfsys@transformshift{1.665874in}{1.015334in}%
\pgfsys@useobject{currentmarker}{}%
\end{pgfscope}%
\begin{pgfscope}%
\pgfsys@transformshift{1.667819in}{0.975119in}%
\pgfsys@useobject{currentmarker}{}%
\end{pgfscope}%
\begin{pgfscope}%
\pgfsys@transformshift{1.669748in}{0.996764in}%
\pgfsys@useobject{currentmarker}{}%
\end{pgfscope}%
\begin{pgfscope}%
\pgfsys@transformshift{1.671660in}{1.042020in}%
\pgfsys@useobject{currentmarker}{}%
\end{pgfscope}%
\begin{pgfscope}%
\pgfsys@transformshift{1.673556in}{1.029084in}%
\pgfsys@useobject{currentmarker}{}%
\end{pgfscope}%
\begin{pgfscope}%
\pgfsys@transformshift{1.675435in}{0.999451in}%
\pgfsys@useobject{currentmarker}{}%
\end{pgfscope}%
\begin{pgfscope}%
\pgfsys@transformshift{1.677300in}{1.000164in}%
\pgfsys@useobject{currentmarker}{}%
\end{pgfscope}%
\begin{pgfscope}%
\pgfsys@transformshift{1.679149in}{1.017880in}%
\pgfsys@useobject{currentmarker}{}%
\end{pgfscope}%
\begin{pgfscope}%
\pgfsys@transformshift{1.680983in}{0.984477in}%
\pgfsys@useobject{currentmarker}{}%
\end{pgfscope}%
\begin{pgfscope}%
\pgfsys@transformshift{1.682802in}{1.013991in}%
\pgfsys@useobject{currentmarker}{}%
\end{pgfscope}%
\begin{pgfscope}%
\pgfsys@transformshift{1.684606in}{1.005423in}%
\pgfsys@useobject{currentmarker}{}%
\end{pgfscope}%
\begin{pgfscope}%
\pgfsys@transformshift{1.686396in}{1.010247in}%
\pgfsys@useobject{currentmarker}{}%
\end{pgfscope}%
\begin{pgfscope}%
\pgfsys@transformshift{1.688172in}{0.993147in}%
\pgfsys@useobject{currentmarker}{}%
\end{pgfscope}%
\begin{pgfscope}%
\pgfsys@transformshift{1.689934in}{1.005775in}%
\pgfsys@useobject{currentmarker}{}%
\end{pgfscope}%
\begin{pgfscope}%
\pgfsys@transformshift{1.691682in}{1.003888in}%
\pgfsys@useobject{currentmarker}{}%
\end{pgfscope}%
\begin{pgfscope}%
\pgfsys@transformshift{1.693417in}{0.976093in}%
\pgfsys@useobject{currentmarker}{}%
\end{pgfscope}%
\begin{pgfscope}%
\pgfsys@transformshift{1.695139in}{1.014675in}%
\pgfsys@useobject{currentmarker}{}%
\end{pgfscope}%
\begin{pgfscope}%
\pgfsys@transformshift{1.696847in}{1.028608in}%
\pgfsys@useobject{currentmarker}{}%
\end{pgfscope}%
\begin{pgfscope}%
\pgfsys@transformshift{1.698543in}{1.000062in}%
\pgfsys@useobject{currentmarker}{}%
\end{pgfscope}%
\begin{pgfscope}%
\pgfsys@transformshift{1.700225in}{1.002268in}%
\pgfsys@useobject{currentmarker}{}%
\end{pgfscope}%
\begin{pgfscope}%
\pgfsys@transformshift{1.701896in}{1.033007in}%
\pgfsys@useobject{currentmarker}{}%
\end{pgfscope}%
\begin{pgfscope}%
\pgfsys@transformshift{1.703554in}{1.021678in}%
\pgfsys@useobject{currentmarker}{}%
\end{pgfscope}%
\begin{pgfscope}%
\pgfsys@transformshift{1.705199in}{1.058368in}%
\pgfsys@useobject{currentmarker}{}%
\end{pgfscope}%
\begin{pgfscope}%
\pgfsys@transformshift{1.706833in}{1.042920in}%
\pgfsys@useobject{currentmarker}{}%
\end{pgfscope}%
\begin{pgfscope}%
\pgfsys@transformshift{1.708455in}{1.010079in}%
\pgfsys@useobject{currentmarker}{}%
\end{pgfscope}%
\begin{pgfscope}%
\pgfsys@transformshift{1.710066in}{1.016787in}%
\pgfsys@useobject{currentmarker}{}%
\end{pgfscope}%
\begin{pgfscope}%
\pgfsys@transformshift{1.711665in}{0.982408in}%
\pgfsys@useobject{currentmarker}{}%
\end{pgfscope}%
\begin{pgfscope}%
\pgfsys@transformshift{1.713252in}{1.052763in}%
\pgfsys@useobject{currentmarker}{}%
\end{pgfscope}%
\begin{pgfscope}%
\pgfsys@transformshift{1.714829in}{1.004967in}%
\pgfsys@useobject{currentmarker}{}%
\end{pgfscope}%
\begin{pgfscope}%
\pgfsys@transformshift{1.716394in}{1.010402in}%
\pgfsys@useobject{currentmarker}{}%
\end{pgfscope}%
\begin{pgfscope}%
\pgfsys@transformshift{1.717949in}{0.941492in}%
\pgfsys@useobject{currentmarker}{}%
\end{pgfscope}%
\begin{pgfscope}%
\pgfsys@transformshift{1.719493in}{0.898942in}%
\pgfsys@useobject{currentmarker}{}%
\end{pgfscope}%
\begin{pgfscope}%
\pgfsys@transformshift{1.721026in}{0.958275in}%
\pgfsys@useobject{currentmarker}{}%
\end{pgfscope}%
\begin{pgfscope}%
\pgfsys@transformshift{1.722550in}{0.991722in}%
\pgfsys@useobject{currentmarker}{}%
\end{pgfscope}%
\begin{pgfscope}%
\pgfsys@transformshift{1.724062in}{1.004222in}%
\pgfsys@useobject{currentmarker}{}%
\end{pgfscope}%
\begin{pgfscope}%
\pgfsys@transformshift{1.725565in}{1.027335in}%
\pgfsys@useobject{currentmarker}{}%
\end{pgfscope}%
\begin{pgfscope}%
\pgfsys@transformshift{1.727058in}{0.990523in}%
\pgfsys@useobject{currentmarker}{}%
\end{pgfscope}%
\begin{pgfscope}%
\pgfsys@transformshift{1.728541in}{0.989322in}%
\pgfsys@useobject{currentmarker}{}%
\end{pgfscope}%
\begin{pgfscope}%
\pgfsys@transformshift{1.730014in}{1.001958in}%
\pgfsys@useobject{currentmarker}{}%
\end{pgfscope}%
\begin{pgfscope}%
\pgfsys@transformshift{1.731477in}{0.952554in}%
\pgfsys@useobject{currentmarker}{}%
\end{pgfscope}%
\begin{pgfscope}%
\pgfsys@transformshift{1.732931in}{0.975463in}%
\pgfsys@useobject{currentmarker}{}%
\end{pgfscope}%
\begin{pgfscope}%
\pgfsys@transformshift{1.734376in}{0.988815in}%
\pgfsys@useobject{currentmarker}{}%
\end{pgfscope}%
\begin{pgfscope}%
\pgfsys@transformshift{1.735812in}{1.019218in}%
\pgfsys@useobject{currentmarker}{}%
\end{pgfscope}%
\begin{pgfscope}%
\pgfsys@transformshift{1.737238in}{1.034054in}%
\pgfsys@useobject{currentmarker}{}%
\end{pgfscope}%
\begin{pgfscope}%
\pgfsys@transformshift{1.738655in}{1.013071in}%
\pgfsys@useobject{currentmarker}{}%
\end{pgfscope}%
\begin{pgfscope}%
\pgfsys@transformshift{1.740064in}{0.949076in}%
\pgfsys@useobject{currentmarker}{}%
\end{pgfscope}%
\begin{pgfscope}%
\pgfsys@transformshift{1.741463in}{0.878230in}%
\pgfsys@useobject{currentmarker}{}%
\end{pgfscope}%
\begin{pgfscope}%
\pgfsys@transformshift{1.742854in}{0.939312in}%
\pgfsys@useobject{currentmarker}{}%
\end{pgfscope}%
\begin{pgfscope}%
\pgfsys@transformshift{1.744237in}{0.970803in}%
\pgfsys@useobject{currentmarker}{}%
\end{pgfscope}%
\begin{pgfscope}%
\pgfsys@transformshift{1.745611in}{1.004233in}%
\pgfsys@useobject{currentmarker}{}%
\end{pgfscope}%
\begin{pgfscope}%
\pgfsys@transformshift{1.746977in}{1.022592in}%
\pgfsys@useobject{currentmarker}{}%
\end{pgfscope}%
\begin{pgfscope}%
\pgfsys@transformshift{1.748334in}{0.985141in}%
\pgfsys@useobject{currentmarker}{}%
\end{pgfscope}%
\begin{pgfscope}%
\pgfsys@transformshift{1.749683in}{0.932980in}%
\pgfsys@useobject{currentmarker}{}%
\end{pgfscope}%
\begin{pgfscope}%
\pgfsys@transformshift{1.751025in}{0.933641in}%
\pgfsys@useobject{currentmarker}{}%
\end{pgfscope}%
\begin{pgfscope}%
\pgfsys@transformshift{1.752358in}{0.896854in}%
\pgfsys@useobject{currentmarker}{}%
\end{pgfscope}%
\begin{pgfscope}%
\pgfsys@transformshift{1.753683in}{0.908551in}%
\pgfsys@useobject{currentmarker}{}%
\end{pgfscope}%
\begin{pgfscope}%
\pgfsys@transformshift{1.755001in}{0.918177in}%
\pgfsys@useobject{currentmarker}{}%
\end{pgfscope}%
\begin{pgfscope}%
\pgfsys@transformshift{1.756311in}{0.944832in}%
\pgfsys@useobject{currentmarker}{}%
\end{pgfscope}%
\begin{pgfscope}%
\pgfsys@transformshift{1.757613in}{0.924943in}%
\pgfsys@useobject{currentmarker}{}%
\end{pgfscope}%
\begin{pgfscope}%
\pgfsys@transformshift{1.758908in}{0.873133in}%
\pgfsys@useobject{currentmarker}{}%
\end{pgfscope}%
\begin{pgfscope}%
\pgfsys@transformshift{1.760195in}{0.854998in}%
\pgfsys@useobject{currentmarker}{}%
\end{pgfscope}%
\begin{pgfscope}%
\pgfsys@transformshift{1.761475in}{0.927751in}%
\pgfsys@useobject{currentmarker}{}%
\end{pgfscope}%
\begin{pgfscope}%
\pgfsys@transformshift{1.762748in}{0.981997in}%
\pgfsys@useobject{currentmarker}{}%
\end{pgfscope}%
\begin{pgfscope}%
\pgfsys@transformshift{1.764014in}{0.959718in}%
\pgfsys@useobject{currentmarker}{}%
\end{pgfscope}%
\begin{pgfscope}%
\pgfsys@transformshift{1.765272in}{0.951327in}%
\pgfsys@useobject{currentmarker}{}%
\end{pgfscope}%
\begin{pgfscope}%
\pgfsys@transformshift{1.766524in}{0.992409in}%
\pgfsys@useobject{currentmarker}{}%
\end{pgfscope}%
\begin{pgfscope}%
\pgfsys@transformshift{1.767769in}{0.998812in}%
\pgfsys@useobject{currentmarker}{}%
\end{pgfscope}%
\begin{pgfscope}%
\pgfsys@transformshift{1.769006in}{0.977150in}%
\pgfsys@useobject{currentmarker}{}%
\end{pgfscope}%
\begin{pgfscope}%
\pgfsys@transformshift{1.770237in}{0.951963in}%
\pgfsys@useobject{currentmarker}{}%
\end{pgfscope}%
\begin{pgfscope}%
\pgfsys@transformshift{1.771462in}{0.957431in}%
\pgfsys@useobject{currentmarker}{}%
\end{pgfscope}%
\begin{pgfscope}%
\pgfsys@transformshift{1.772679in}{0.988688in}%
\pgfsys@useobject{currentmarker}{}%
\end{pgfscope}%
\begin{pgfscope}%
\pgfsys@transformshift{1.773890in}{0.959751in}%
\pgfsys@useobject{currentmarker}{}%
\end{pgfscope}%
\begin{pgfscope}%
\pgfsys@transformshift{1.775095in}{0.953720in}%
\pgfsys@useobject{currentmarker}{}%
\end{pgfscope}%
\begin{pgfscope}%
\pgfsys@transformshift{1.776293in}{0.996163in}%
\pgfsys@useobject{currentmarker}{}%
\end{pgfscope}%
\begin{pgfscope}%
\pgfsys@transformshift{1.777485in}{0.978138in}%
\pgfsys@useobject{currentmarker}{}%
\end{pgfscope}%
\begin{pgfscope}%
\pgfsys@transformshift{1.778670in}{0.942278in}%
\pgfsys@useobject{currentmarker}{}%
\end{pgfscope}%
\begin{pgfscope}%
\pgfsys@transformshift{1.779849in}{0.998451in}%
\pgfsys@useobject{currentmarker}{}%
\end{pgfscope}%
\begin{pgfscope}%
\pgfsys@transformshift{1.781023in}{0.987817in}%
\pgfsys@useobject{currentmarker}{}%
\end{pgfscope}%
\begin{pgfscope}%
\pgfsys@transformshift{1.782190in}{0.941817in}%
\pgfsys@useobject{currentmarker}{}%
\end{pgfscope}%
\begin{pgfscope}%
\pgfsys@transformshift{1.783351in}{0.954538in}%
\pgfsys@useobject{currentmarker}{}%
\end{pgfscope}%
\begin{pgfscope}%
\pgfsys@transformshift{1.784506in}{0.981581in}%
\pgfsys@useobject{currentmarker}{}%
\end{pgfscope}%
\begin{pgfscope}%
\pgfsys@transformshift{1.785655in}{0.902075in}%
\pgfsys@useobject{currentmarker}{}%
\end{pgfscope}%
\begin{pgfscope}%
\pgfsys@transformshift{1.786798in}{0.890611in}%
\pgfsys@useobject{currentmarker}{}%
\end{pgfscope}%
\begin{pgfscope}%
\pgfsys@transformshift{1.787936in}{0.961528in}%
\pgfsys@useobject{currentmarker}{}%
\end{pgfscope}%
\begin{pgfscope}%
\pgfsys@transformshift{1.789067in}{0.989306in}%
\pgfsys@useobject{currentmarker}{}%
\end{pgfscope}%
\begin{pgfscope}%
\pgfsys@transformshift{1.790193in}{0.993728in}%
\pgfsys@useobject{currentmarker}{}%
\end{pgfscope}%
\begin{pgfscope}%
\pgfsys@transformshift{1.791314in}{1.019221in}%
\pgfsys@useobject{currentmarker}{}%
\end{pgfscope}%
\begin{pgfscope}%
\pgfsys@transformshift{1.792429in}{1.019675in}%
\pgfsys@useobject{currentmarker}{}%
\end{pgfscope}%
\begin{pgfscope}%
\pgfsys@transformshift{1.793538in}{0.968329in}%
\pgfsys@useobject{currentmarker}{}%
\end{pgfscope}%
\begin{pgfscope}%
\pgfsys@transformshift{1.794642in}{0.927693in}%
\pgfsys@useobject{currentmarker}{}%
\end{pgfscope}%
\begin{pgfscope}%
\pgfsys@transformshift{1.795741in}{0.886227in}%
\pgfsys@useobject{currentmarker}{}%
\end{pgfscope}%
\begin{pgfscope}%
\pgfsys@transformshift{1.796834in}{0.911023in}%
\pgfsys@useobject{currentmarker}{}%
\end{pgfscope}%
\begin{pgfscope}%
\pgfsys@transformshift{1.797922in}{0.964434in}%
\pgfsys@useobject{currentmarker}{}%
\end{pgfscope}%
\begin{pgfscope}%
\pgfsys@transformshift{1.799004in}{0.942858in}%
\pgfsys@useobject{currentmarker}{}%
\end{pgfscope}%
\begin{pgfscope}%
\pgfsys@transformshift{1.800082in}{0.935556in}%
\pgfsys@useobject{currentmarker}{}%
\end{pgfscope}%
\begin{pgfscope}%
\pgfsys@transformshift{1.801154in}{0.955890in}%
\pgfsys@useobject{currentmarker}{}%
\end{pgfscope}%
\begin{pgfscope}%
\pgfsys@transformshift{1.802221in}{0.903562in}%
\pgfsys@useobject{currentmarker}{}%
\end{pgfscope}%
\begin{pgfscope}%
\pgfsys@transformshift{1.803284in}{0.918587in}%
\pgfsys@useobject{currentmarker}{}%
\end{pgfscope}%
\begin{pgfscope}%
\pgfsys@transformshift{1.804341in}{0.910437in}%
\pgfsys@useobject{currentmarker}{}%
\end{pgfscope}%
\begin{pgfscope}%
\pgfsys@transformshift{1.805393in}{0.914167in}%
\pgfsys@useobject{currentmarker}{}%
\end{pgfscope}%
\begin{pgfscope}%
\pgfsys@transformshift{1.806440in}{0.928218in}%
\pgfsys@useobject{currentmarker}{}%
\end{pgfscope}%
\begin{pgfscope}%
\pgfsys@transformshift{1.807483in}{0.928742in}%
\pgfsys@useobject{currentmarker}{}%
\end{pgfscope}%
\begin{pgfscope}%
\pgfsys@transformshift{1.808520in}{0.940043in}%
\pgfsys@useobject{currentmarker}{}%
\end{pgfscope}%
\begin{pgfscope}%
\pgfsys@transformshift{1.809553in}{0.965731in}%
\pgfsys@useobject{currentmarker}{}%
\end{pgfscope}%
\begin{pgfscope}%
\pgfsys@transformshift{1.810581in}{0.934750in}%
\pgfsys@useobject{currentmarker}{}%
\end{pgfscope}%
\begin{pgfscope}%
\pgfsys@transformshift{1.811605in}{0.932998in}%
\pgfsys@useobject{currentmarker}{}%
\end{pgfscope}%
\begin{pgfscope}%
\pgfsys@transformshift{1.812624in}{0.903523in}%
\pgfsys@useobject{currentmarker}{}%
\end{pgfscope}%
\begin{pgfscope}%
\pgfsys@transformshift{1.813638in}{0.894253in}%
\pgfsys@useobject{currentmarker}{}%
\end{pgfscope}%
\begin{pgfscope}%
\pgfsys@transformshift{1.814648in}{0.919857in}%
\pgfsys@useobject{currentmarker}{}%
\end{pgfscope}%
\begin{pgfscope}%
\pgfsys@transformshift{1.815653in}{0.943210in}%
\pgfsys@useobject{currentmarker}{}%
\end{pgfscope}%
\begin{pgfscope}%
\pgfsys@transformshift{1.816653in}{0.938640in}%
\pgfsys@useobject{currentmarker}{}%
\end{pgfscope}%
\begin{pgfscope}%
\pgfsys@transformshift{1.817650in}{0.922136in}%
\pgfsys@useobject{currentmarker}{}%
\end{pgfscope}%
\begin{pgfscope}%
\pgfsys@transformshift{1.818642in}{0.885553in}%
\pgfsys@useobject{currentmarker}{}%
\end{pgfscope}%
\begin{pgfscope}%
\pgfsys@transformshift{1.819629in}{0.882626in}%
\pgfsys@useobject{currentmarker}{}%
\end{pgfscope}%
\begin{pgfscope}%
\pgfsys@transformshift{1.820612in}{0.923199in}%
\pgfsys@useobject{currentmarker}{}%
\end{pgfscope}%
\begin{pgfscope}%
\pgfsys@transformshift{1.821591in}{0.871260in}%
\pgfsys@useobject{currentmarker}{}%
\end{pgfscope}%
\begin{pgfscope}%
\pgfsys@transformshift{1.822566in}{0.835198in}%
\pgfsys@useobject{currentmarker}{}%
\end{pgfscope}%
\begin{pgfscope}%
\pgfsys@transformshift{1.823536in}{0.896678in}%
\pgfsys@useobject{currentmarker}{}%
\end{pgfscope}%
\begin{pgfscope}%
\pgfsys@transformshift{1.824502in}{0.917862in}%
\pgfsys@useobject{currentmarker}{}%
\end{pgfscope}%
\begin{pgfscope}%
\pgfsys@transformshift{1.825464in}{0.900156in}%
\pgfsys@useobject{currentmarker}{}%
\end{pgfscope}%
\begin{pgfscope}%
\pgfsys@transformshift{1.826423in}{0.954410in}%
\pgfsys@useobject{currentmarker}{}%
\end{pgfscope}%
\begin{pgfscope}%
\pgfsys@transformshift{1.827376in}{0.967181in}%
\pgfsys@useobject{currentmarker}{}%
\end{pgfscope}%
\begin{pgfscope}%
\pgfsys@transformshift{1.828326in}{0.915909in}%
\pgfsys@useobject{currentmarker}{}%
\end{pgfscope}%
\begin{pgfscope}%
\pgfsys@transformshift{1.829272in}{0.922509in}%
\pgfsys@useobject{currentmarker}{}%
\end{pgfscope}%
\begin{pgfscope}%
\pgfsys@transformshift{1.830214in}{0.911325in}%
\pgfsys@useobject{currentmarker}{}%
\end{pgfscope}%
\begin{pgfscope}%
\pgfsys@transformshift{1.831152in}{0.930041in}%
\pgfsys@useobject{currentmarker}{}%
\end{pgfscope}%
\begin{pgfscope}%
\pgfsys@transformshift{1.832086in}{0.940203in}%
\pgfsys@useobject{currentmarker}{}%
\end{pgfscope}%
\begin{pgfscope}%
\pgfsys@transformshift{1.833017in}{0.926372in}%
\pgfsys@useobject{currentmarker}{}%
\end{pgfscope}%
\begin{pgfscope}%
\pgfsys@transformshift{1.833943in}{0.866135in}%
\pgfsys@useobject{currentmarker}{}%
\end{pgfscope}%
\begin{pgfscope}%
\pgfsys@transformshift{1.834866in}{0.878424in}%
\pgfsys@useobject{currentmarker}{}%
\end{pgfscope}%
\begin{pgfscope}%
\pgfsys@transformshift{1.835784in}{0.889490in}%
\pgfsys@useobject{currentmarker}{}%
\end{pgfscope}%
\begin{pgfscope}%
\pgfsys@transformshift{1.836699in}{0.877095in}%
\pgfsys@useobject{currentmarker}{}%
\end{pgfscope}%
\begin{pgfscope}%
\pgfsys@transformshift{1.837611in}{0.892343in}%
\pgfsys@useobject{currentmarker}{}%
\end{pgfscope}%
\begin{pgfscope}%
\pgfsys@transformshift{1.838518in}{0.940050in}%
\pgfsys@useobject{currentmarker}{}%
\end{pgfscope}%
\begin{pgfscope}%
\pgfsys@transformshift{1.839423in}{0.948179in}%
\pgfsys@useobject{currentmarker}{}%
\end{pgfscope}%
\begin{pgfscope}%
\pgfsys@transformshift{1.840323in}{0.882358in}%
\pgfsys@useobject{currentmarker}{}%
\end{pgfscope}%
\begin{pgfscope}%
\pgfsys@transformshift{1.841220in}{0.892817in}%
\pgfsys@useobject{currentmarker}{}%
\end{pgfscope}%
\begin{pgfscope}%
\pgfsys@transformshift{1.842113in}{0.925810in}%
\pgfsys@useobject{currentmarker}{}%
\end{pgfscope}%
\begin{pgfscope}%
\pgfsys@transformshift{1.843003in}{0.938740in}%
\pgfsys@useobject{currentmarker}{}%
\end{pgfscope}%
\begin{pgfscope}%
\pgfsys@transformshift{1.843889in}{0.929383in}%
\pgfsys@useobject{currentmarker}{}%
\end{pgfscope}%
\begin{pgfscope}%
\pgfsys@transformshift{1.844772in}{0.946552in}%
\pgfsys@useobject{currentmarker}{}%
\end{pgfscope}%
\begin{pgfscope}%
\pgfsys@transformshift{1.845651in}{0.954592in}%
\pgfsys@useobject{currentmarker}{}%
\end{pgfscope}%
\begin{pgfscope}%
\pgfsys@transformshift{1.846527in}{0.925979in}%
\pgfsys@useobject{currentmarker}{}%
\end{pgfscope}%
\begin{pgfscope}%
\pgfsys@transformshift{1.847399in}{0.918996in}%
\pgfsys@useobject{currentmarker}{}%
\end{pgfscope}%
\begin{pgfscope}%
\pgfsys@transformshift{1.848268in}{0.936402in}%
\pgfsys@useobject{currentmarker}{}%
\end{pgfscope}%
\begin{pgfscope}%
\pgfsys@transformshift{1.849134in}{0.893017in}%
\pgfsys@useobject{currentmarker}{}%
\end{pgfscope}%
\begin{pgfscope}%
\pgfsys@transformshift{1.849996in}{0.925095in}%
\pgfsys@useobject{currentmarker}{}%
\end{pgfscope}%
\begin{pgfscope}%
\pgfsys@transformshift{1.850855in}{0.953066in}%
\pgfsys@useobject{currentmarker}{}%
\end{pgfscope}%
\begin{pgfscope}%
\pgfsys@transformshift{1.851711in}{0.909335in}%
\pgfsys@useobject{currentmarker}{}%
\end{pgfscope}%
\begin{pgfscope}%
\pgfsys@transformshift{1.852564in}{0.909027in}%
\pgfsys@useobject{currentmarker}{}%
\end{pgfscope}%
\begin{pgfscope}%
\pgfsys@transformshift{1.853413in}{0.906683in}%
\pgfsys@useobject{currentmarker}{}%
\end{pgfscope}%
\begin{pgfscope}%
\pgfsys@transformshift{1.854259in}{0.882094in}%
\pgfsys@useobject{currentmarker}{}%
\end{pgfscope}%
\begin{pgfscope}%
\pgfsys@transformshift{1.855102in}{0.845605in}%
\pgfsys@useobject{currentmarker}{}%
\end{pgfscope}%
\begin{pgfscope}%
\pgfsys@transformshift{1.855942in}{0.822602in}%
\pgfsys@useobject{currentmarker}{}%
\end{pgfscope}%
\begin{pgfscope}%
\pgfsys@transformshift{1.856779in}{0.857045in}%
\pgfsys@useobject{currentmarker}{}%
\end{pgfscope}%
\begin{pgfscope}%
\pgfsys@transformshift{1.857612in}{0.882969in}%
\pgfsys@useobject{currentmarker}{}%
\end{pgfscope}%
\begin{pgfscope}%
\pgfsys@transformshift{1.858443in}{0.874862in}%
\pgfsys@useobject{currentmarker}{}%
\end{pgfscope}%
\begin{pgfscope}%
\pgfsys@transformshift{1.859270in}{0.892196in}%
\pgfsys@useobject{currentmarker}{}%
\end{pgfscope}%
\begin{pgfscope}%
\pgfsys@transformshift{1.860095in}{0.927409in}%
\pgfsys@useobject{currentmarker}{}%
\end{pgfscope}%
\begin{pgfscope}%
\pgfsys@transformshift{1.860916in}{0.862143in}%
\pgfsys@useobject{currentmarker}{}%
\end{pgfscope}%
\begin{pgfscope}%
\pgfsys@transformshift{1.861735in}{0.856254in}%
\pgfsys@useobject{currentmarker}{}%
\end{pgfscope}%
\begin{pgfscope}%
\pgfsys@transformshift{1.862550in}{0.867133in}%
\pgfsys@useobject{currentmarker}{}%
\end{pgfscope}%
\begin{pgfscope}%
\pgfsys@transformshift{1.863362in}{0.908099in}%
\pgfsys@useobject{currentmarker}{}%
\end{pgfscope}%
\begin{pgfscope}%
\pgfsys@transformshift{1.864172in}{0.897139in}%
\pgfsys@useobject{currentmarker}{}%
\end{pgfscope}%
\begin{pgfscope}%
\pgfsys@transformshift{1.864979in}{0.897752in}%
\pgfsys@useobject{currentmarker}{}%
\end{pgfscope}%
\begin{pgfscope}%
\pgfsys@transformshift{1.865782in}{0.892636in}%
\pgfsys@useobject{currentmarker}{}%
\end{pgfscope}%
\begin{pgfscope}%
\pgfsys@transformshift{1.866583in}{0.866430in}%
\pgfsys@useobject{currentmarker}{}%
\end{pgfscope}%
\begin{pgfscope}%
\pgfsys@transformshift{1.867381in}{0.902646in}%
\pgfsys@useobject{currentmarker}{}%
\end{pgfscope}%
\begin{pgfscope}%
\pgfsys@transformshift{1.868177in}{0.896592in}%
\pgfsys@useobject{currentmarker}{}%
\end{pgfscope}%
\begin{pgfscope}%
\pgfsys@transformshift{1.868969in}{0.931603in}%
\pgfsys@useobject{currentmarker}{}%
\end{pgfscope}%
\begin{pgfscope}%
\pgfsys@transformshift{1.869759in}{0.905184in}%
\pgfsys@useobject{currentmarker}{}%
\end{pgfscope}%
\begin{pgfscope}%
\pgfsys@transformshift{1.870546in}{0.815348in}%
\pgfsys@useobject{currentmarker}{}%
\end{pgfscope}%
\begin{pgfscope}%
\pgfsys@transformshift{1.871330in}{0.877727in}%
\pgfsys@useobject{currentmarker}{}%
\end{pgfscope}%
\begin{pgfscope}%
\pgfsys@transformshift{1.872111in}{0.900657in}%
\pgfsys@useobject{currentmarker}{}%
\end{pgfscope}%
\begin{pgfscope}%
\pgfsys@transformshift{1.872890in}{0.860195in}%
\pgfsys@useobject{currentmarker}{}%
\end{pgfscope}%
\begin{pgfscope}%
\pgfsys@transformshift{1.873666in}{0.841754in}%
\pgfsys@useobject{currentmarker}{}%
\end{pgfscope}%
\begin{pgfscope}%
\pgfsys@transformshift{1.874439in}{0.852890in}%
\pgfsys@useobject{currentmarker}{}%
\end{pgfscope}%
\begin{pgfscope}%
\pgfsys@transformshift{1.875210in}{0.907077in}%
\pgfsys@useobject{currentmarker}{}%
\end{pgfscope}%
\begin{pgfscope}%
\pgfsys@transformshift{1.875978in}{0.925623in}%
\pgfsys@useobject{currentmarker}{}%
\end{pgfscope}%
\begin{pgfscope}%
\pgfsys@transformshift{1.876743in}{0.919749in}%
\pgfsys@useobject{currentmarker}{}%
\end{pgfscope}%
\begin{pgfscope}%
\pgfsys@transformshift{1.877506in}{0.856119in}%
\pgfsys@useobject{currentmarker}{}%
\end{pgfscope}%
\begin{pgfscope}%
\pgfsys@transformshift{1.878266in}{0.852607in}%
\pgfsys@useobject{currentmarker}{}%
\end{pgfscope}%
\begin{pgfscope}%
\pgfsys@transformshift{1.879024in}{0.861972in}%
\pgfsys@useobject{currentmarker}{}%
\end{pgfscope}%
\begin{pgfscope}%
\pgfsys@transformshift{1.879779in}{0.895745in}%
\pgfsys@useobject{currentmarker}{}%
\end{pgfscope}%
\begin{pgfscope}%
\pgfsys@transformshift{1.880532in}{0.895235in}%
\pgfsys@useobject{currentmarker}{}%
\end{pgfscope}%
\begin{pgfscope}%
\pgfsys@transformshift{1.881282in}{0.862184in}%
\pgfsys@useobject{currentmarker}{}%
\end{pgfscope}%
\begin{pgfscope}%
\pgfsys@transformshift{1.882029in}{0.898176in}%
\pgfsys@useobject{currentmarker}{}%
\end{pgfscope}%
\begin{pgfscope}%
\pgfsys@transformshift{1.882774in}{0.894241in}%
\pgfsys@useobject{currentmarker}{}%
\end{pgfscope}%
\begin{pgfscope}%
\pgfsys@transformshift{1.883517in}{0.862124in}%
\pgfsys@useobject{currentmarker}{}%
\end{pgfscope}%
\begin{pgfscope}%
\pgfsys@transformshift{1.884257in}{0.913113in}%
\pgfsys@useobject{currentmarker}{}%
\end{pgfscope}%
\begin{pgfscope}%
\pgfsys@transformshift{1.884995in}{0.911665in}%
\pgfsys@useobject{currentmarker}{}%
\end{pgfscope}%
\begin{pgfscope}%
\pgfsys@transformshift{1.885730in}{0.868371in}%
\pgfsys@useobject{currentmarker}{}%
\end{pgfscope}%
\begin{pgfscope}%
\pgfsys@transformshift{1.886463in}{0.851790in}%
\pgfsys@useobject{currentmarker}{}%
\end{pgfscope}%
\begin{pgfscope}%
\pgfsys@transformshift{1.887194in}{0.830603in}%
\pgfsys@useobject{currentmarker}{}%
\end{pgfscope}%
\begin{pgfscope}%
\pgfsys@transformshift{1.887922in}{0.917230in}%
\pgfsys@useobject{currentmarker}{}%
\end{pgfscope}%
\begin{pgfscope}%
\pgfsys@transformshift{1.888648in}{0.929372in}%
\pgfsys@useobject{currentmarker}{}%
\end{pgfscope}%
\begin{pgfscope}%
\pgfsys@transformshift{1.889372in}{0.884223in}%
\pgfsys@useobject{currentmarker}{}%
\end{pgfscope}%
\begin{pgfscope}%
\pgfsys@transformshift{1.890093in}{0.866289in}%
\pgfsys@useobject{currentmarker}{}%
\end{pgfscope}%
\begin{pgfscope}%
\pgfsys@transformshift{1.890812in}{0.864875in}%
\pgfsys@useobject{currentmarker}{}%
\end{pgfscope}%
\begin{pgfscope}%
\pgfsys@transformshift{1.891528in}{0.920091in}%
\pgfsys@useobject{currentmarker}{}%
\end{pgfscope}%
\begin{pgfscope}%
\pgfsys@transformshift{1.892243in}{0.897719in}%
\pgfsys@useobject{currentmarker}{}%
\end{pgfscope}%
\begin{pgfscope}%
\pgfsys@transformshift{1.892955in}{0.860372in}%
\pgfsys@useobject{currentmarker}{}%
\end{pgfscope}%
\begin{pgfscope}%
\pgfsys@transformshift{1.893664in}{0.914282in}%
\pgfsys@useobject{currentmarker}{}%
\end{pgfscope}%
\begin{pgfscope}%
\pgfsys@transformshift{1.894372in}{0.909851in}%
\pgfsys@useobject{currentmarker}{}%
\end{pgfscope}%
\begin{pgfscope}%
\pgfsys@transformshift{1.895077in}{0.897261in}%
\pgfsys@useobject{currentmarker}{}%
\end{pgfscope}%
\begin{pgfscope}%
\pgfsys@transformshift{1.895780in}{0.820830in}%
\pgfsys@useobject{currentmarker}{}%
\end{pgfscope}%
\begin{pgfscope}%
\pgfsys@transformshift{1.896481in}{0.860261in}%
\pgfsys@useobject{currentmarker}{}%
\end{pgfscope}%
\begin{pgfscope}%
\pgfsys@transformshift{1.897180in}{0.869633in}%
\pgfsys@useobject{currentmarker}{}%
\end{pgfscope}%
\begin{pgfscope}%
\pgfsys@transformshift{1.897877in}{0.876122in}%
\pgfsys@useobject{currentmarker}{}%
\end{pgfscope}%
\begin{pgfscope}%
\pgfsys@transformshift{1.898571in}{0.870369in}%
\pgfsys@useobject{currentmarker}{}%
\end{pgfscope}%
\begin{pgfscope}%
\pgfsys@transformshift{1.899264in}{0.869295in}%
\pgfsys@useobject{currentmarker}{}%
\end{pgfscope}%
\begin{pgfscope}%
\pgfsys@transformshift{1.899954in}{0.919133in}%
\pgfsys@useobject{currentmarker}{}%
\end{pgfscope}%
\begin{pgfscope}%
\pgfsys@transformshift{1.900642in}{0.931295in}%
\pgfsys@useobject{currentmarker}{}%
\end{pgfscope}%
\begin{pgfscope}%
\pgfsys@transformshift{1.901328in}{0.881504in}%
\pgfsys@useobject{currentmarker}{}%
\end{pgfscope}%
\begin{pgfscope}%
\pgfsys@transformshift{1.902012in}{0.857289in}%
\pgfsys@useobject{currentmarker}{}%
\end{pgfscope}%
\begin{pgfscope}%
\pgfsys@transformshift{1.902693in}{0.882424in}%
\pgfsys@useobject{currentmarker}{}%
\end{pgfscope}%
\begin{pgfscope}%
\pgfsys@transformshift{1.903373in}{0.962077in}%
\pgfsys@useobject{currentmarker}{}%
\end{pgfscope}%
\begin{pgfscope}%
\pgfsys@transformshift{1.904051in}{0.951743in}%
\pgfsys@useobject{currentmarker}{}%
\end{pgfscope}%
\begin{pgfscope}%
\pgfsys@transformshift{1.904726in}{0.913622in}%
\pgfsys@useobject{currentmarker}{}%
\end{pgfscope}%
\begin{pgfscope}%
\pgfsys@transformshift{1.905400in}{0.915250in}%
\pgfsys@useobject{currentmarker}{}%
\end{pgfscope}%
\begin{pgfscope}%
\pgfsys@transformshift{1.906072in}{0.866016in}%
\pgfsys@useobject{currentmarker}{}%
\end{pgfscope}%
\begin{pgfscope}%
\pgfsys@transformshift{1.906741in}{0.848849in}%
\pgfsys@useobject{currentmarker}{}%
\end{pgfscope}%
\begin{pgfscope}%
\pgfsys@transformshift{1.907409in}{0.851163in}%
\pgfsys@useobject{currentmarker}{}%
\end{pgfscope}%
\begin{pgfscope}%
\pgfsys@transformshift{1.908075in}{0.828793in}%
\pgfsys@useobject{currentmarker}{}%
\end{pgfscope}%
\begin{pgfscope}%
\pgfsys@transformshift{1.908738in}{0.822618in}%
\pgfsys@useobject{currentmarker}{}%
\end{pgfscope}%
\begin{pgfscope}%
\pgfsys@transformshift{1.909400in}{0.834806in}%
\pgfsys@useobject{currentmarker}{}%
\end{pgfscope}%
\begin{pgfscope}%
\pgfsys@transformshift{1.910060in}{0.827132in}%
\pgfsys@useobject{currentmarker}{}%
\end{pgfscope}%
\begin{pgfscope}%
\pgfsys@transformshift{1.910717in}{0.829656in}%
\pgfsys@useobject{currentmarker}{}%
\end{pgfscope}%
\begin{pgfscope}%
\pgfsys@transformshift{1.911373in}{0.883877in}%
\pgfsys@useobject{currentmarker}{}%
\end{pgfscope}%
\begin{pgfscope}%
\pgfsys@transformshift{1.912027in}{0.878445in}%
\pgfsys@useobject{currentmarker}{}%
\end{pgfscope}%
\begin{pgfscope}%
\pgfsys@transformshift{1.912680in}{0.913145in}%
\pgfsys@useobject{currentmarker}{}%
\end{pgfscope}%
\begin{pgfscope}%
\pgfsys@transformshift{1.913330in}{0.923664in}%
\pgfsys@useobject{currentmarker}{}%
\end{pgfscope}%
\begin{pgfscope}%
\pgfsys@transformshift{1.913978in}{0.832269in}%
\pgfsys@useobject{currentmarker}{}%
\end{pgfscope}%
\begin{pgfscope}%
\pgfsys@transformshift{1.914625in}{0.881998in}%
\pgfsys@useobject{currentmarker}{}%
\end{pgfscope}%
\begin{pgfscope}%
\pgfsys@transformshift{1.915269in}{0.870560in}%
\pgfsys@useobject{currentmarker}{}%
\end{pgfscope}%
\begin{pgfscope}%
\pgfsys@transformshift{1.915912in}{0.852447in}%
\pgfsys@useobject{currentmarker}{}%
\end{pgfscope}%
\begin{pgfscope}%
\pgfsys@transformshift{1.916553in}{0.831464in}%
\pgfsys@useobject{currentmarker}{}%
\end{pgfscope}%
\begin{pgfscope}%
\pgfsys@transformshift{1.917192in}{0.760373in}%
\pgfsys@useobject{currentmarker}{}%
\end{pgfscope}%
\begin{pgfscope}%
\pgfsys@transformshift{1.917829in}{0.852771in}%
\pgfsys@useobject{currentmarker}{}%
\end{pgfscope}%
\begin{pgfscope}%
\pgfsys@transformshift{1.918465in}{0.883449in}%
\pgfsys@useobject{currentmarker}{}%
\end{pgfscope}%
\begin{pgfscope}%
\pgfsys@transformshift{1.919099in}{0.855862in}%
\pgfsys@useobject{currentmarker}{}%
\end{pgfscope}%
\begin{pgfscope}%
\pgfsys@transformshift{1.919731in}{0.854415in}%
\pgfsys@useobject{currentmarker}{}%
\end{pgfscope}%
\begin{pgfscope}%
\pgfsys@transformshift{1.920361in}{0.855146in}%
\pgfsys@useobject{currentmarker}{}%
\end{pgfscope}%
\begin{pgfscope}%
\pgfsys@transformshift{1.920989in}{0.845043in}%
\pgfsys@useobject{currentmarker}{}%
\end{pgfscope}%
\begin{pgfscope}%
\pgfsys@transformshift{1.921616in}{0.797923in}%
\pgfsys@useobject{currentmarker}{}%
\end{pgfscope}%
\begin{pgfscope}%
\pgfsys@transformshift{1.922241in}{0.851903in}%
\pgfsys@useobject{currentmarker}{}%
\end{pgfscope}%
\begin{pgfscope}%
\pgfsys@transformshift{1.922864in}{0.799815in}%
\pgfsys@useobject{currentmarker}{}%
\end{pgfscope}%
\begin{pgfscope}%
\pgfsys@transformshift{1.923485in}{0.862288in}%
\pgfsys@useobject{currentmarker}{}%
\end{pgfscope}%
\begin{pgfscope}%
\pgfsys@transformshift{1.924105in}{0.893396in}%
\pgfsys@useobject{currentmarker}{}%
\end{pgfscope}%
\begin{pgfscope}%
\pgfsys@transformshift{1.924723in}{0.863905in}%
\pgfsys@useobject{currentmarker}{}%
\end{pgfscope}%
\begin{pgfscope}%
\pgfsys@transformshift{1.925339in}{0.849477in}%
\pgfsys@useobject{currentmarker}{}%
\end{pgfscope}%
\begin{pgfscope}%
\pgfsys@transformshift{1.925954in}{0.811275in}%
\pgfsys@useobject{currentmarker}{}%
\end{pgfscope}%
\begin{pgfscope}%
\pgfsys@transformshift{1.926567in}{0.846296in}%
\pgfsys@useobject{currentmarker}{}%
\end{pgfscope}%
\begin{pgfscope}%
\pgfsys@transformshift{1.927178in}{0.817540in}%
\pgfsys@useobject{currentmarker}{}%
\end{pgfscope}%
\begin{pgfscope}%
\pgfsys@transformshift{1.927788in}{0.799489in}%
\pgfsys@useobject{currentmarker}{}%
\end{pgfscope}%
\begin{pgfscope}%
\pgfsys@transformshift{1.928396in}{0.781782in}%
\pgfsys@useobject{currentmarker}{}%
\end{pgfscope}%
\begin{pgfscope}%
\pgfsys@transformshift{1.929002in}{0.820781in}%
\pgfsys@useobject{currentmarker}{}%
\end{pgfscope}%
\begin{pgfscope}%
\pgfsys@transformshift{1.929607in}{0.871187in}%
\pgfsys@useobject{currentmarker}{}%
\end{pgfscope}%
\begin{pgfscope}%
\pgfsys@transformshift{1.930210in}{0.849613in}%
\pgfsys@useobject{currentmarker}{}%
\end{pgfscope}%
\begin{pgfscope}%
\pgfsys@transformshift{1.930811in}{0.817072in}%
\pgfsys@useobject{currentmarker}{}%
\end{pgfscope}%
\begin{pgfscope}%
\pgfsys@transformshift{1.931411in}{0.837572in}%
\pgfsys@useobject{currentmarker}{}%
\end{pgfscope}%
\begin{pgfscope}%
\pgfsys@transformshift{1.932010in}{0.834399in}%
\pgfsys@useobject{currentmarker}{}%
\end{pgfscope}%
\begin{pgfscope}%
\pgfsys@transformshift{1.932606in}{0.865978in}%
\pgfsys@useobject{currentmarker}{}%
\end{pgfscope}%
\begin{pgfscope}%
\pgfsys@transformshift{1.933201in}{0.844577in}%
\pgfsys@useobject{currentmarker}{}%
\end{pgfscope}%
\begin{pgfscope}%
\pgfsys@transformshift{1.933795in}{0.832464in}%
\pgfsys@useobject{currentmarker}{}%
\end{pgfscope}%
\begin{pgfscope}%
\pgfsys@transformshift{1.934387in}{0.848728in}%
\pgfsys@useobject{currentmarker}{}%
\end{pgfscope}%
\begin{pgfscope}%
\pgfsys@transformshift{1.934977in}{0.825778in}%
\pgfsys@useobject{currentmarker}{}%
\end{pgfscope}%
\begin{pgfscope}%
\pgfsys@transformshift{1.935566in}{0.754126in}%
\pgfsys@useobject{currentmarker}{}%
\end{pgfscope}%
\begin{pgfscope}%
\pgfsys@transformshift{1.936154in}{0.841103in}%
\pgfsys@useobject{currentmarker}{}%
\end{pgfscope}%
\begin{pgfscope}%
\pgfsys@transformshift{1.936739in}{0.841987in}%
\pgfsys@useobject{currentmarker}{}%
\end{pgfscope}%
\begin{pgfscope}%
\pgfsys@transformshift{1.937324in}{0.836999in}%
\pgfsys@useobject{currentmarker}{}%
\end{pgfscope}%
\begin{pgfscope}%
\pgfsys@transformshift{1.937906in}{0.875378in}%
\pgfsys@useobject{currentmarker}{}%
\end{pgfscope}%
\begin{pgfscope}%
\pgfsys@transformshift{1.938488in}{0.881792in}%
\pgfsys@useobject{currentmarker}{}%
\end{pgfscope}%
\begin{pgfscope}%
\pgfsys@transformshift{1.939067in}{0.889746in}%
\pgfsys@useobject{currentmarker}{}%
\end{pgfscope}%
\begin{pgfscope}%
\pgfsys@transformshift{1.939646in}{0.810120in}%
\pgfsys@useobject{currentmarker}{}%
\end{pgfscope}%
\begin{pgfscope}%
\pgfsys@transformshift{1.940222in}{0.817668in}%
\pgfsys@useobject{currentmarker}{}%
\end{pgfscope}%
\begin{pgfscope}%
\pgfsys@transformshift{1.940798in}{0.804678in}%
\pgfsys@useobject{currentmarker}{}%
\end{pgfscope}%
\begin{pgfscope}%
\pgfsys@transformshift{1.941371in}{0.831522in}%
\pgfsys@useobject{currentmarker}{}%
\end{pgfscope}%
\begin{pgfscope}%
\pgfsys@transformshift{1.941944in}{0.847745in}%
\pgfsys@useobject{currentmarker}{}%
\end{pgfscope}%
\begin{pgfscope}%
\pgfsys@transformshift{1.942515in}{0.852349in}%
\pgfsys@useobject{currentmarker}{}%
\end{pgfscope}%
\begin{pgfscope}%
\pgfsys@transformshift{1.943084in}{0.863436in}%
\pgfsys@useobject{currentmarker}{}%
\end{pgfscope}%
\begin{pgfscope}%
\pgfsys@transformshift{1.943652in}{0.855830in}%
\pgfsys@useobject{currentmarker}{}%
\end{pgfscope}%
\begin{pgfscope}%
\pgfsys@transformshift{1.944219in}{0.852371in}%
\pgfsys@useobject{currentmarker}{}%
\end{pgfscope}%
\begin{pgfscope}%
\pgfsys@transformshift{1.944784in}{0.848238in}%
\pgfsys@useobject{currentmarker}{}%
\end{pgfscope}%
\begin{pgfscope}%
\pgfsys@transformshift{1.945348in}{0.858159in}%
\pgfsys@useobject{currentmarker}{}%
\end{pgfscope}%
\begin{pgfscope}%
\pgfsys@transformshift{1.945910in}{0.862303in}%
\pgfsys@useobject{currentmarker}{}%
\end{pgfscope}%
\begin{pgfscope}%
\pgfsys@transformshift{1.946471in}{0.837512in}%
\pgfsys@useobject{currentmarker}{}%
\end{pgfscope}%
\begin{pgfscope}%
\pgfsys@transformshift{1.947031in}{0.821151in}%
\pgfsys@useobject{currentmarker}{}%
\end{pgfscope}%
\begin{pgfscope}%
\pgfsys@transformshift{1.947589in}{0.818575in}%
\pgfsys@useobject{currentmarker}{}%
\end{pgfscope}%
\begin{pgfscope}%
\pgfsys@transformshift{1.948145in}{0.881364in}%
\pgfsys@useobject{currentmarker}{}%
\end{pgfscope}%
\begin{pgfscope}%
\pgfsys@transformshift{1.948701in}{0.869961in}%
\pgfsys@useobject{currentmarker}{}%
\end{pgfscope}%
\begin{pgfscope}%
\pgfsys@transformshift{1.949255in}{0.854023in}%
\pgfsys@useobject{currentmarker}{}%
\end{pgfscope}%
\begin{pgfscope}%
\pgfsys@transformshift{1.949807in}{0.840069in}%
\pgfsys@useobject{currentmarker}{}%
\end{pgfscope}%
\begin{pgfscope}%
\pgfsys@transformshift{1.950359in}{0.861918in}%
\pgfsys@useobject{currentmarker}{}%
\end{pgfscope}%
\begin{pgfscope}%
\pgfsys@transformshift{1.950909in}{0.837782in}%
\pgfsys@useobject{currentmarker}{}%
\end{pgfscope}%
\begin{pgfscope}%
\pgfsys@transformshift{1.951457in}{0.837257in}%
\pgfsys@useobject{currentmarker}{}%
\end{pgfscope}%
\begin{pgfscope}%
\pgfsys@transformshift{1.952005in}{0.804909in}%
\pgfsys@useobject{currentmarker}{}%
\end{pgfscope}%
\begin{pgfscope}%
\pgfsys@transformshift{1.952550in}{0.835141in}%
\pgfsys@useobject{currentmarker}{}%
\end{pgfscope}%
\begin{pgfscope}%
\pgfsys@transformshift{1.953095in}{0.878779in}%
\pgfsys@useobject{currentmarker}{}%
\end{pgfscope}%
\begin{pgfscope}%
\pgfsys@transformshift{1.953638in}{0.876278in}%
\pgfsys@useobject{currentmarker}{}%
\end{pgfscope}%
\begin{pgfscope}%
\pgfsys@transformshift{1.954180in}{0.857814in}%
\pgfsys@useobject{currentmarker}{}%
\end{pgfscope}%
\begin{pgfscope}%
\pgfsys@transformshift{1.954721in}{0.894881in}%
\pgfsys@useobject{currentmarker}{}%
\end{pgfscope}%
\begin{pgfscope}%
\pgfsys@transformshift{1.955260in}{0.880440in}%
\pgfsys@useobject{currentmarker}{}%
\end{pgfscope}%
\begin{pgfscope}%
\pgfsys@transformshift{1.955799in}{0.833254in}%
\pgfsys@useobject{currentmarker}{}%
\end{pgfscope}%
\begin{pgfscope}%
\pgfsys@transformshift{1.956335in}{0.852752in}%
\pgfsys@useobject{currentmarker}{}%
\end{pgfscope}%
\begin{pgfscope}%
\pgfsys@transformshift{1.956871in}{0.872189in}%
\pgfsys@useobject{currentmarker}{}%
\end{pgfscope}%
\begin{pgfscope}%
\pgfsys@transformshift{1.957405in}{0.863373in}%
\pgfsys@useobject{currentmarker}{}%
\end{pgfscope}%
\begin{pgfscope}%
\pgfsys@transformshift{1.957938in}{0.803570in}%
\pgfsys@useobject{currentmarker}{}%
\end{pgfscope}%
\begin{pgfscope}%
\pgfsys@transformshift{1.958470in}{0.836460in}%
\pgfsys@useobject{currentmarker}{}%
\end{pgfscope}%
\begin{pgfscope}%
\pgfsys@transformshift{1.959000in}{0.879784in}%
\pgfsys@useobject{currentmarker}{}%
\end{pgfscope}%
\begin{pgfscope}%
\pgfsys@transformshift{1.959529in}{0.843010in}%
\pgfsys@useobject{currentmarker}{}%
\end{pgfscope}%
\begin{pgfscope}%
\pgfsys@transformshift{1.960057in}{0.816646in}%
\pgfsys@useobject{currentmarker}{}%
\end{pgfscope}%
\begin{pgfscope}%
\pgfsys@transformshift{1.960584in}{0.773861in}%
\pgfsys@useobject{currentmarker}{}%
\end{pgfscope}%
\begin{pgfscope}%
\pgfsys@transformshift{1.961110in}{0.805317in}%
\pgfsys@useobject{currentmarker}{}%
\end{pgfscope}%
\begin{pgfscope}%
\pgfsys@transformshift{1.961634in}{0.826616in}%
\pgfsys@useobject{currentmarker}{}%
\end{pgfscope}%
\begin{pgfscope}%
\pgfsys@transformshift{1.962157in}{0.815432in}%
\pgfsys@useobject{currentmarker}{}%
\end{pgfscope}%
\begin{pgfscope}%
\pgfsys@transformshift{1.962679in}{0.835468in}%
\pgfsys@useobject{currentmarker}{}%
\end{pgfscope}%
\begin{pgfscope}%
\pgfsys@transformshift{1.963199in}{0.858071in}%
\pgfsys@useobject{currentmarker}{}%
\end{pgfscope}%
\begin{pgfscope}%
\pgfsys@transformshift{1.963719in}{0.831046in}%
\pgfsys@useobject{currentmarker}{}%
\end{pgfscope}%
\begin{pgfscope}%
\pgfsys@transformshift{1.964237in}{0.819070in}%
\pgfsys@useobject{currentmarker}{}%
\end{pgfscope}%
\begin{pgfscope}%
\pgfsys@transformshift{1.964754in}{0.856447in}%
\pgfsys@useobject{currentmarker}{}%
\end{pgfscope}%
\begin{pgfscope}%
\pgfsys@transformshift{1.965270in}{0.860059in}%
\pgfsys@useobject{currentmarker}{}%
\end{pgfscope}%
\begin{pgfscope}%
\pgfsys@transformshift{1.965785in}{0.810766in}%
\pgfsys@useobject{currentmarker}{}%
\end{pgfscope}%
\begin{pgfscope}%
\pgfsys@transformshift{1.966298in}{0.813258in}%
\pgfsys@useobject{currentmarker}{}%
\end{pgfscope}%
\begin{pgfscope}%
\pgfsys@transformshift{1.966810in}{0.863284in}%
\pgfsys@useobject{currentmarker}{}%
\end{pgfscope}%
\begin{pgfscope}%
\pgfsys@transformshift{1.967322in}{0.874811in}%
\pgfsys@useobject{currentmarker}{}%
\end{pgfscope}%
\begin{pgfscope}%
\pgfsys@transformshift{1.967832in}{0.849959in}%
\pgfsys@useobject{currentmarker}{}%
\end{pgfscope}%
\begin{pgfscope}%
\pgfsys@transformshift{1.968340in}{0.800901in}%
\pgfsys@useobject{currentmarker}{}%
\end{pgfscope}%
\begin{pgfscope}%
\pgfsys@transformshift{1.968848in}{0.795933in}%
\pgfsys@useobject{currentmarker}{}%
\end{pgfscope}%
\begin{pgfscope}%
\pgfsys@transformshift{1.969355in}{0.831290in}%
\pgfsys@useobject{currentmarker}{}%
\end{pgfscope}%
\begin{pgfscope}%
\pgfsys@transformshift{1.969860in}{0.804019in}%
\pgfsys@useobject{currentmarker}{}%
\end{pgfscope}%
\begin{pgfscope}%
\pgfsys@transformshift{1.970364in}{0.836466in}%
\pgfsys@useobject{currentmarker}{}%
\end{pgfscope}%
\begin{pgfscope}%
\pgfsys@transformshift{1.970868in}{0.818526in}%
\pgfsys@useobject{currentmarker}{}%
\end{pgfscope}%
\begin{pgfscope}%
\pgfsys@transformshift{1.971370in}{0.796743in}%
\pgfsys@useobject{currentmarker}{}%
\end{pgfscope}%
\begin{pgfscope}%
\pgfsys@transformshift{1.971870in}{0.787864in}%
\pgfsys@useobject{currentmarker}{}%
\end{pgfscope}%
\begin{pgfscope}%
\pgfsys@transformshift{1.972370in}{0.782755in}%
\pgfsys@useobject{currentmarker}{}%
\end{pgfscope}%
\begin{pgfscope}%
\pgfsys@transformshift{1.972869in}{0.852076in}%
\pgfsys@useobject{currentmarker}{}%
\end{pgfscope}%
\begin{pgfscope}%
\pgfsys@transformshift{1.973366in}{0.823058in}%
\pgfsys@useobject{currentmarker}{}%
\end{pgfscope}%
\begin{pgfscope}%
\pgfsys@transformshift{1.973863in}{0.796900in}%
\pgfsys@useobject{currentmarker}{}%
\end{pgfscope}%
\begin{pgfscope}%
\pgfsys@transformshift{1.974358in}{0.822127in}%
\pgfsys@useobject{currentmarker}{}%
\end{pgfscope}%
\begin{pgfscope}%
\pgfsys@transformshift{1.974853in}{0.831811in}%
\pgfsys@useobject{currentmarker}{}%
\end{pgfscope}%
\begin{pgfscope}%
\pgfsys@transformshift{1.975346in}{0.852858in}%
\pgfsys@useobject{currentmarker}{}%
\end{pgfscope}%
\begin{pgfscope}%
\pgfsys@transformshift{1.975838in}{0.845188in}%
\pgfsys@useobject{currentmarker}{}%
\end{pgfscope}%
\begin{pgfscope}%
\pgfsys@transformshift{1.976329in}{0.855269in}%
\pgfsys@useobject{currentmarker}{}%
\end{pgfscope}%
\begin{pgfscope}%
\pgfsys@transformshift{1.976819in}{0.840067in}%
\pgfsys@useobject{currentmarker}{}%
\end{pgfscope}%
\begin{pgfscope}%
\pgfsys@transformshift{1.977308in}{0.828964in}%
\pgfsys@useobject{currentmarker}{}%
\end{pgfscope}%
\begin{pgfscope}%
\pgfsys@transformshift{1.977796in}{0.827034in}%
\pgfsys@useobject{currentmarker}{}%
\end{pgfscope}%
\begin{pgfscope}%
\pgfsys@transformshift{1.978282in}{0.807477in}%
\pgfsys@useobject{currentmarker}{}%
\end{pgfscope}%
\begin{pgfscope}%
\pgfsys@transformshift{1.978768in}{0.797191in}%
\pgfsys@useobject{currentmarker}{}%
\end{pgfscope}%
\begin{pgfscope}%
\pgfsys@transformshift{1.979253in}{0.825363in}%
\pgfsys@useobject{currentmarker}{}%
\end{pgfscope}%
\begin{pgfscope}%
\pgfsys@transformshift{1.979736in}{0.802362in}%
\pgfsys@useobject{currentmarker}{}%
\end{pgfscope}%
\begin{pgfscope}%
\pgfsys@transformshift{1.980219in}{0.810875in}%
\pgfsys@useobject{currentmarker}{}%
\end{pgfscope}%
\begin{pgfscope}%
\pgfsys@transformshift{1.980701in}{0.817469in}%
\pgfsys@useobject{currentmarker}{}%
\end{pgfscope}%
\begin{pgfscope}%
\pgfsys@transformshift{1.981181in}{0.857488in}%
\pgfsys@useobject{currentmarker}{}%
\end{pgfscope}%
\begin{pgfscope}%
\pgfsys@transformshift{1.981661in}{0.849323in}%
\pgfsys@useobject{currentmarker}{}%
\end{pgfscope}%
\begin{pgfscope}%
\pgfsys@transformshift{1.982139in}{0.848861in}%
\pgfsys@useobject{currentmarker}{}%
\end{pgfscope}%
\begin{pgfscope}%
\pgfsys@transformshift{1.982617in}{0.846113in}%
\pgfsys@useobject{currentmarker}{}%
\end{pgfscope}%
\begin{pgfscope}%
\pgfsys@transformshift{1.983093in}{0.824861in}%
\pgfsys@useobject{currentmarker}{}%
\end{pgfscope}%
\begin{pgfscope}%
\pgfsys@transformshift{1.983569in}{0.824708in}%
\pgfsys@useobject{currentmarker}{}%
\end{pgfscope}%
\begin{pgfscope}%
\pgfsys@transformshift{1.984043in}{0.798602in}%
\pgfsys@useobject{currentmarker}{}%
\end{pgfscope}%
\begin{pgfscope}%
\pgfsys@transformshift{1.984517in}{0.824694in}%
\pgfsys@useobject{currentmarker}{}%
\end{pgfscope}%
\begin{pgfscope}%
\pgfsys@transformshift{1.984989in}{0.839707in}%
\pgfsys@useobject{currentmarker}{}%
\end{pgfscope}%
\begin{pgfscope}%
\pgfsys@transformshift{1.985460in}{0.820366in}%
\pgfsys@useobject{currentmarker}{}%
\end{pgfscope}%
\begin{pgfscope}%
\pgfsys@transformshift{1.985931in}{0.791033in}%
\pgfsys@useobject{currentmarker}{}%
\end{pgfscope}%
\begin{pgfscope}%
\pgfsys@transformshift{1.986400in}{0.803211in}%
\pgfsys@useobject{currentmarker}{}%
\end{pgfscope}%
\begin{pgfscope}%
\pgfsys@transformshift{1.986869in}{0.770530in}%
\pgfsys@useobject{currentmarker}{}%
\end{pgfscope}%
\begin{pgfscope}%
\pgfsys@transformshift{1.987336in}{0.799032in}%
\pgfsys@useobject{currentmarker}{}%
\end{pgfscope}%
\begin{pgfscope}%
\pgfsys@transformshift{1.987803in}{0.830753in}%
\pgfsys@useobject{currentmarker}{}%
\end{pgfscope}%
\begin{pgfscope}%
\pgfsys@transformshift{1.988269in}{0.802493in}%
\pgfsys@useobject{currentmarker}{}%
\end{pgfscope}%
\begin{pgfscope}%
\pgfsys@transformshift{1.988733in}{0.762829in}%
\pgfsys@useobject{currentmarker}{}%
\end{pgfscope}%
\begin{pgfscope}%
\pgfsys@transformshift{1.989197in}{0.786480in}%
\pgfsys@useobject{currentmarker}{}%
\end{pgfscope}%
\begin{pgfscope}%
\pgfsys@transformshift{1.989660in}{0.849395in}%
\pgfsys@useobject{currentmarker}{}%
\end{pgfscope}%
\begin{pgfscope}%
\pgfsys@transformshift{1.990121in}{0.861601in}%
\pgfsys@useobject{currentmarker}{}%
\end{pgfscope}%
\begin{pgfscope}%
\pgfsys@transformshift{1.990582in}{0.844780in}%
\pgfsys@useobject{currentmarker}{}%
\end{pgfscope}%
\begin{pgfscope}%
\pgfsys@transformshift{1.991042in}{0.789673in}%
\pgfsys@useobject{currentmarker}{}%
\end{pgfscope}%
\begin{pgfscope}%
\pgfsys@transformshift{1.991501in}{0.773997in}%
\pgfsys@useobject{currentmarker}{}%
\end{pgfscope}%
\begin{pgfscope}%
\pgfsys@transformshift{1.991959in}{0.806476in}%
\pgfsys@useobject{currentmarker}{}%
\end{pgfscope}%
\begin{pgfscope}%
\pgfsys@transformshift{1.992416in}{0.823982in}%
\pgfsys@useobject{currentmarker}{}%
\end{pgfscope}%
\begin{pgfscope}%
\pgfsys@transformshift{1.992872in}{0.816315in}%
\pgfsys@useobject{currentmarker}{}%
\end{pgfscope}%
\begin{pgfscope}%
\pgfsys@transformshift{1.993328in}{0.840357in}%
\pgfsys@useobject{currentmarker}{}%
\end{pgfscope}%
\begin{pgfscope}%
\pgfsys@transformshift{1.993782in}{0.823913in}%
\pgfsys@useobject{currentmarker}{}%
\end{pgfscope}%
\begin{pgfscope}%
\pgfsys@transformshift{1.994235in}{0.804318in}%
\pgfsys@useobject{currentmarker}{}%
\end{pgfscope}%
\begin{pgfscope}%
\pgfsys@transformshift{1.994688in}{0.799733in}%
\pgfsys@useobject{currentmarker}{}%
\end{pgfscope}%
\begin{pgfscope}%
\pgfsys@transformshift{1.995139in}{0.863067in}%
\pgfsys@useobject{currentmarker}{}%
\end{pgfscope}%
\begin{pgfscope}%
\pgfsys@transformshift{1.995590in}{0.860490in}%
\pgfsys@useobject{currentmarker}{}%
\end{pgfscope}%
\begin{pgfscope}%
\pgfsys@transformshift{1.996040in}{0.809483in}%
\pgfsys@useobject{currentmarker}{}%
\end{pgfscope}%
\begin{pgfscope}%
\pgfsys@transformshift{1.996488in}{0.797447in}%
\pgfsys@useobject{currentmarker}{}%
\end{pgfscope}%
\begin{pgfscope}%
\pgfsys@transformshift{1.996936in}{0.815548in}%
\pgfsys@useobject{currentmarker}{}%
\end{pgfscope}%
\begin{pgfscope}%
\pgfsys@transformshift{1.997384in}{0.832481in}%
\pgfsys@useobject{currentmarker}{}%
\end{pgfscope}%
\begin{pgfscope}%
\pgfsys@transformshift{1.997830in}{0.805600in}%
\pgfsys@useobject{currentmarker}{}%
\end{pgfscope}%
\begin{pgfscope}%
\pgfsys@transformshift{1.998275in}{0.812903in}%
\pgfsys@useobject{currentmarker}{}%
\end{pgfscope}%
\begin{pgfscope}%
\pgfsys@transformshift{1.998719in}{0.843620in}%
\pgfsys@useobject{currentmarker}{}%
\end{pgfscope}%
\begin{pgfscope}%
\pgfsys@transformshift{1.999163in}{0.824199in}%
\pgfsys@useobject{currentmarker}{}%
\end{pgfscope}%
\begin{pgfscope}%
\pgfsys@transformshift{1.999606in}{0.765688in}%
\pgfsys@useobject{currentmarker}{}%
\end{pgfscope}%
\begin{pgfscope}%
\pgfsys@transformshift{2.000047in}{0.815875in}%
\pgfsys@useobject{currentmarker}{}%
\end{pgfscope}%
\begin{pgfscope}%
\pgfsys@transformshift{2.000488in}{0.816073in}%
\pgfsys@useobject{currentmarker}{}%
\end{pgfscope}%
\begin{pgfscope}%
\pgfsys@transformshift{2.000928in}{0.835400in}%
\pgfsys@useobject{currentmarker}{}%
\end{pgfscope}%
\begin{pgfscope}%
\pgfsys@transformshift{2.001368in}{0.803598in}%
\pgfsys@useobject{currentmarker}{}%
\end{pgfscope}%
\begin{pgfscope}%
\pgfsys@transformshift{2.001806in}{0.804704in}%
\pgfsys@useobject{currentmarker}{}%
\end{pgfscope}%
\begin{pgfscope}%
\pgfsys@transformshift{2.002243in}{0.848122in}%
\pgfsys@useobject{currentmarker}{}%
\end{pgfscope}%
\begin{pgfscope}%
\pgfsys@transformshift{2.002680in}{0.736194in}%
\pgfsys@useobject{currentmarker}{}%
\end{pgfscope}%
\begin{pgfscope}%
\pgfsys@transformshift{2.003116in}{0.748053in}%
\pgfsys@useobject{currentmarker}{}%
\end{pgfscope}%
\begin{pgfscope}%
\pgfsys@transformshift{2.003551in}{0.785164in}%
\pgfsys@useobject{currentmarker}{}%
\end{pgfscope}%
\begin{pgfscope}%
\pgfsys@transformshift{2.003985in}{0.846940in}%
\pgfsys@useobject{currentmarker}{}%
\end{pgfscope}%
\begin{pgfscope}%
\pgfsys@transformshift{2.004418in}{0.826073in}%
\pgfsys@useobject{currentmarker}{}%
\end{pgfscope}%
\begin{pgfscope}%
\pgfsys@transformshift{2.004851in}{0.732683in}%
\pgfsys@useobject{currentmarker}{}%
\end{pgfscope}%
\begin{pgfscope}%
\pgfsys@transformshift{2.005282in}{0.764420in}%
\pgfsys@useobject{currentmarker}{}%
\end{pgfscope}%
\begin{pgfscope}%
\pgfsys@transformshift{2.005713in}{0.750013in}%
\pgfsys@useobject{currentmarker}{}%
\end{pgfscope}%
\begin{pgfscope}%
\pgfsys@transformshift{2.006143in}{0.841089in}%
\pgfsys@useobject{currentmarker}{}%
\end{pgfscope}%
\begin{pgfscope}%
\pgfsys@transformshift{2.006572in}{0.796806in}%
\pgfsys@useobject{currentmarker}{}%
\end{pgfscope}%
\begin{pgfscope}%
\pgfsys@transformshift{2.007000in}{0.777390in}%
\pgfsys@useobject{currentmarker}{}%
\end{pgfscope}%
\begin{pgfscope}%
\pgfsys@transformshift{2.007428in}{0.816028in}%
\pgfsys@useobject{currentmarker}{}%
\end{pgfscope}%
\begin{pgfscope}%
\pgfsys@transformshift{2.007855in}{0.842749in}%
\pgfsys@useobject{currentmarker}{}%
\end{pgfscope}%
\begin{pgfscope}%
\pgfsys@transformshift{2.008280in}{0.837164in}%
\pgfsys@useobject{currentmarker}{}%
\end{pgfscope}%
\begin{pgfscope}%
\pgfsys@transformshift{2.008706in}{0.795140in}%
\pgfsys@useobject{currentmarker}{}%
\end{pgfscope}%
\begin{pgfscope}%
\pgfsys@transformshift{2.009130in}{0.760137in}%
\pgfsys@useobject{currentmarker}{}%
\end{pgfscope}%
\begin{pgfscope}%
\pgfsys@transformshift{2.009553in}{0.793978in}%
\pgfsys@useobject{currentmarker}{}%
\end{pgfscope}%
\begin{pgfscope}%
\pgfsys@transformshift{2.009976in}{0.836378in}%
\pgfsys@useobject{currentmarker}{}%
\end{pgfscope}%
\begin{pgfscope}%
\pgfsys@transformshift{2.010398in}{0.815180in}%
\pgfsys@useobject{currentmarker}{}%
\end{pgfscope}%
\begin{pgfscope}%
\pgfsys@transformshift{2.010819in}{0.780410in}%
\pgfsys@useobject{currentmarker}{}%
\end{pgfscope}%
\begin{pgfscope}%
\pgfsys@transformshift{2.011239in}{0.798333in}%
\pgfsys@useobject{currentmarker}{}%
\end{pgfscope}%
\begin{pgfscope}%
\pgfsys@transformshift{2.011659in}{0.824686in}%
\pgfsys@useobject{currentmarker}{}%
\end{pgfscope}%
\begin{pgfscope}%
\pgfsys@transformshift{2.012078in}{0.834952in}%
\pgfsys@useobject{currentmarker}{}%
\end{pgfscope}%
\begin{pgfscope}%
\pgfsys@transformshift{2.012495in}{0.791375in}%
\pgfsys@useobject{currentmarker}{}%
\end{pgfscope}%
\begin{pgfscope}%
\pgfsys@transformshift{2.012913in}{0.794212in}%
\pgfsys@useobject{currentmarker}{}%
\end{pgfscope}%
\begin{pgfscope}%
\pgfsys@transformshift{2.013329in}{0.783574in}%
\pgfsys@useobject{currentmarker}{}%
\end{pgfscope}%
\begin{pgfscope}%
\pgfsys@transformshift{2.013745in}{0.809414in}%
\pgfsys@useobject{currentmarker}{}%
\end{pgfscope}%
\begin{pgfscope}%
\pgfsys@transformshift{2.014160in}{0.802448in}%
\pgfsys@useobject{currentmarker}{}%
\end{pgfscope}%
\begin{pgfscope}%
\pgfsys@transformshift{2.014574in}{0.824433in}%
\pgfsys@useobject{currentmarker}{}%
\end{pgfscope}%
\begin{pgfscope}%
\pgfsys@transformshift{2.014987in}{0.829897in}%
\pgfsys@useobject{currentmarker}{}%
\end{pgfscope}%
\begin{pgfscope}%
\pgfsys@transformshift{2.015400in}{0.794305in}%
\pgfsys@useobject{currentmarker}{}%
\end{pgfscope}%
\begin{pgfscope}%
\pgfsys@transformshift{2.015811in}{0.769556in}%
\pgfsys@useobject{currentmarker}{}%
\end{pgfscope}%
\begin{pgfscope}%
\pgfsys@transformshift{2.016223in}{0.808688in}%
\pgfsys@useobject{currentmarker}{}%
\end{pgfscope}%
\begin{pgfscope}%
\pgfsys@transformshift{2.016633in}{0.812639in}%
\pgfsys@useobject{currentmarker}{}%
\end{pgfscope}%
\begin{pgfscope}%
\pgfsys@transformshift{2.017042in}{0.787931in}%
\pgfsys@useobject{currentmarker}{}%
\end{pgfscope}%
\begin{pgfscope}%
\pgfsys@transformshift{2.017451in}{0.772141in}%
\pgfsys@useobject{currentmarker}{}%
\end{pgfscope}%
\begin{pgfscope}%
\pgfsys@transformshift{2.017859in}{0.783414in}%
\pgfsys@useobject{currentmarker}{}%
\end{pgfscope}%
\begin{pgfscope}%
\pgfsys@transformshift{2.018267in}{0.762160in}%
\pgfsys@useobject{currentmarker}{}%
\end{pgfscope}%
\begin{pgfscope}%
\pgfsys@transformshift{2.018673in}{0.804974in}%
\pgfsys@useobject{currentmarker}{}%
\end{pgfscope}%
\begin{pgfscope}%
\pgfsys@transformshift{2.019079in}{0.846118in}%
\pgfsys@useobject{currentmarker}{}%
\end{pgfscope}%
\begin{pgfscope}%
\pgfsys@transformshift{2.019484in}{0.821338in}%
\pgfsys@useobject{currentmarker}{}%
\end{pgfscope}%
\begin{pgfscope}%
\pgfsys@transformshift{2.019889in}{0.811291in}%
\pgfsys@useobject{currentmarker}{}%
\end{pgfscope}%
\begin{pgfscope}%
\pgfsys@transformshift{2.020292in}{0.822839in}%
\pgfsys@useobject{currentmarker}{}%
\end{pgfscope}%
\begin{pgfscope}%
\pgfsys@transformshift{2.020695in}{0.818493in}%
\pgfsys@useobject{currentmarker}{}%
\end{pgfscope}%
\begin{pgfscope}%
\pgfsys@transformshift{2.021098in}{0.804460in}%
\pgfsys@useobject{currentmarker}{}%
\end{pgfscope}%
\begin{pgfscope}%
\pgfsys@transformshift{2.021499in}{0.817500in}%
\pgfsys@useobject{currentmarker}{}%
\end{pgfscope}%
\begin{pgfscope}%
\pgfsys@transformshift{2.021900in}{0.803444in}%
\pgfsys@useobject{currentmarker}{}%
\end{pgfscope}%
\begin{pgfscope}%
\pgfsys@transformshift{2.022300in}{0.787036in}%
\pgfsys@useobject{currentmarker}{}%
\end{pgfscope}%
\begin{pgfscope}%
\pgfsys@transformshift{2.022699in}{0.777913in}%
\pgfsys@useobject{currentmarker}{}%
\end{pgfscope}%
\begin{pgfscope}%
\pgfsys@transformshift{2.023098in}{0.817688in}%
\pgfsys@useobject{currentmarker}{}%
\end{pgfscope}%
\begin{pgfscope}%
\pgfsys@transformshift{2.023496in}{0.833024in}%
\pgfsys@useobject{currentmarker}{}%
\end{pgfscope}%
\begin{pgfscope}%
\pgfsys@transformshift{2.023893in}{0.814919in}%
\pgfsys@useobject{currentmarker}{}%
\end{pgfscope}%
\begin{pgfscope}%
\pgfsys@transformshift{2.024290in}{0.834457in}%
\pgfsys@useobject{currentmarker}{}%
\end{pgfscope}%
\begin{pgfscope}%
\pgfsys@transformshift{2.024686in}{0.786378in}%
\pgfsys@useobject{currentmarker}{}%
\end{pgfscope}%
\begin{pgfscope}%
\pgfsys@transformshift{2.025081in}{0.734385in}%
\pgfsys@useobject{currentmarker}{}%
\end{pgfscope}%
\begin{pgfscope}%
\pgfsys@transformshift{2.025475in}{0.760606in}%
\pgfsys@useobject{currentmarker}{}%
\end{pgfscope}%
\begin{pgfscope}%
\pgfsys@transformshift{2.025869in}{0.822223in}%
\pgfsys@useobject{currentmarker}{}%
\end{pgfscope}%
\begin{pgfscope}%
\pgfsys@transformshift{2.026262in}{0.846939in}%
\pgfsys@useobject{currentmarker}{}%
\end{pgfscope}%
\begin{pgfscope}%
\pgfsys@transformshift{2.026655in}{0.818413in}%
\pgfsys@useobject{currentmarker}{}%
\end{pgfscope}%
\begin{pgfscope}%
\pgfsys@transformshift{2.027046in}{0.772538in}%
\pgfsys@useobject{currentmarker}{}%
\end{pgfscope}%
\begin{pgfscope}%
\pgfsys@transformshift{2.027437in}{0.764981in}%
\pgfsys@useobject{currentmarker}{}%
\end{pgfscope}%
\begin{pgfscope}%
\pgfsys@transformshift{2.027828in}{0.817610in}%
\pgfsys@useobject{currentmarker}{}%
\end{pgfscope}%
\begin{pgfscope}%
\pgfsys@transformshift{2.028217in}{0.795965in}%
\pgfsys@useobject{currentmarker}{}%
\end{pgfscope}%
\begin{pgfscope}%
\pgfsys@transformshift{2.028606in}{0.797624in}%
\pgfsys@useobject{currentmarker}{}%
\end{pgfscope}%
\begin{pgfscope}%
\pgfsys@transformshift{2.028995in}{0.767084in}%
\pgfsys@useobject{currentmarker}{}%
\end{pgfscope}%
\begin{pgfscope}%
\pgfsys@transformshift{2.029382in}{0.758665in}%
\pgfsys@useobject{currentmarker}{}%
\end{pgfscope}%
\begin{pgfscope}%
\pgfsys@transformshift{2.029769in}{0.827693in}%
\pgfsys@useobject{currentmarker}{}%
\end{pgfscope}%
\begin{pgfscope}%
\pgfsys@transformshift{2.030156in}{0.814346in}%
\pgfsys@useobject{currentmarker}{}%
\end{pgfscope}%
\begin{pgfscope}%
\pgfsys@transformshift{2.030541in}{0.754448in}%
\pgfsys@useobject{currentmarker}{}%
\end{pgfscope}%
\begin{pgfscope}%
\pgfsys@transformshift{2.030926in}{0.745627in}%
\pgfsys@useobject{currentmarker}{}%
\end{pgfscope}%
\begin{pgfscope}%
\pgfsys@transformshift{2.031311in}{0.784945in}%
\pgfsys@useobject{currentmarker}{}%
\end{pgfscope}%
\begin{pgfscope}%
\pgfsys@transformshift{2.031694in}{0.815758in}%
\pgfsys@useobject{currentmarker}{}%
\end{pgfscope}%
\begin{pgfscope}%
\pgfsys@transformshift{2.032078in}{0.810155in}%
\pgfsys@useobject{currentmarker}{}%
\end{pgfscope}%
\begin{pgfscope}%
\pgfsys@transformshift{2.032460in}{0.790675in}%
\pgfsys@useobject{currentmarker}{}%
\end{pgfscope}%
\begin{pgfscope}%
\pgfsys@transformshift{2.032842in}{0.764484in}%
\pgfsys@useobject{currentmarker}{}%
\end{pgfscope}%
\begin{pgfscope}%
\pgfsys@transformshift{2.033223in}{0.811195in}%
\pgfsys@useobject{currentmarker}{}%
\end{pgfscope}%
\begin{pgfscope}%
\pgfsys@transformshift{2.033603in}{0.765842in}%
\pgfsys@useobject{currentmarker}{}%
\end{pgfscope}%
\begin{pgfscope}%
\pgfsys@transformshift{2.033983in}{0.794832in}%
\pgfsys@useobject{currentmarker}{}%
\end{pgfscope}%
\begin{pgfscope}%
\pgfsys@transformshift{2.034362in}{0.809335in}%
\pgfsys@useobject{currentmarker}{}%
\end{pgfscope}%
\begin{pgfscope}%
\pgfsys@transformshift{2.034741in}{0.744305in}%
\pgfsys@useobject{currentmarker}{}%
\end{pgfscope}%
\begin{pgfscope}%
\pgfsys@transformshift{2.035119in}{0.796932in}%
\pgfsys@useobject{currentmarker}{}%
\end{pgfscope}%
\begin{pgfscope}%
\pgfsys@transformshift{2.035496in}{0.807257in}%
\pgfsys@useobject{currentmarker}{}%
\end{pgfscope}%
\begin{pgfscope}%
\pgfsys@transformshift{2.035872in}{0.794995in}%
\pgfsys@useobject{currentmarker}{}%
\end{pgfscope}%
\begin{pgfscope}%
\pgfsys@transformshift{2.036248in}{0.837130in}%
\pgfsys@useobject{currentmarker}{}%
\end{pgfscope}%
\begin{pgfscope}%
\pgfsys@transformshift{2.036624in}{0.832889in}%
\pgfsys@useobject{currentmarker}{}%
\end{pgfscope}%
\begin{pgfscope}%
\pgfsys@transformshift{2.036998in}{0.791830in}%
\pgfsys@useobject{currentmarker}{}%
\end{pgfscope}%
\begin{pgfscope}%
\pgfsys@transformshift{2.037373in}{0.784453in}%
\pgfsys@useobject{currentmarker}{}%
\end{pgfscope}%
\begin{pgfscope}%
\pgfsys@transformshift{2.037746in}{0.797190in}%
\pgfsys@useobject{currentmarker}{}%
\end{pgfscope}%
\begin{pgfscope}%
\pgfsys@transformshift{2.038119in}{0.795050in}%
\pgfsys@useobject{currentmarker}{}%
\end{pgfscope}%
\begin{pgfscope}%
\pgfsys@transformshift{2.038491in}{0.798083in}%
\pgfsys@useobject{currentmarker}{}%
\end{pgfscope}%
\begin{pgfscope}%
\pgfsys@transformshift{2.038863in}{0.744900in}%
\pgfsys@useobject{currentmarker}{}%
\end{pgfscope}%
\begin{pgfscope}%
\pgfsys@transformshift{2.039234in}{0.779797in}%
\pgfsys@useobject{currentmarker}{}%
\end{pgfscope}%
\begin{pgfscope}%
\pgfsys@transformshift{2.039604in}{0.776970in}%
\pgfsys@useobject{currentmarker}{}%
\end{pgfscope}%
\begin{pgfscope}%
\pgfsys@transformshift{2.039974in}{0.765209in}%
\pgfsys@useobject{currentmarker}{}%
\end{pgfscope}%
\begin{pgfscope}%
\pgfsys@transformshift{2.040343in}{0.809163in}%
\pgfsys@useobject{currentmarker}{}%
\end{pgfscope}%
\begin{pgfscope}%
\pgfsys@transformshift{2.040712in}{0.776786in}%
\pgfsys@useobject{currentmarker}{}%
\end{pgfscope}%
\begin{pgfscope}%
\pgfsys@transformshift{2.041080in}{0.780541in}%
\pgfsys@useobject{currentmarker}{}%
\end{pgfscope}%
\begin{pgfscope}%
\pgfsys@transformshift{2.041447in}{0.800843in}%
\pgfsys@useobject{currentmarker}{}%
\end{pgfscope}%
\begin{pgfscope}%
\pgfsys@transformshift{2.041814in}{0.825914in}%
\pgfsys@useobject{currentmarker}{}%
\end{pgfscope}%
\begin{pgfscope}%
\pgfsys@transformshift{2.042180in}{0.864052in}%
\pgfsys@useobject{currentmarker}{}%
\end{pgfscope}%
\begin{pgfscope}%
\pgfsys@transformshift{2.042546in}{0.770022in}%
\pgfsys@useobject{currentmarker}{}%
\end{pgfscope}%
\begin{pgfscope}%
\pgfsys@transformshift{2.042911in}{0.778317in}%
\pgfsys@useobject{currentmarker}{}%
\end{pgfscope}%
\begin{pgfscope}%
\pgfsys@transformshift{2.043275in}{0.836923in}%
\pgfsys@useobject{currentmarker}{}%
\end{pgfscope}%
\begin{pgfscope}%
\pgfsys@transformshift{2.043639in}{0.803202in}%
\pgfsys@useobject{currentmarker}{}%
\end{pgfscope}%
\begin{pgfscope}%
\pgfsys@transformshift{2.044002in}{0.765283in}%
\pgfsys@useobject{currentmarker}{}%
\end{pgfscope}%
\begin{pgfscope}%
\pgfsys@transformshift{2.044365in}{0.813224in}%
\pgfsys@useobject{currentmarker}{}%
\end{pgfscope}%
\begin{pgfscope}%
\pgfsys@transformshift{2.044727in}{0.792821in}%
\pgfsys@useobject{currentmarker}{}%
\end{pgfscope}%
\begin{pgfscope}%
\pgfsys@transformshift{2.045088in}{0.757780in}%
\pgfsys@useobject{currentmarker}{}%
\end{pgfscope}%
\begin{pgfscope}%
\pgfsys@transformshift{2.045449in}{0.763163in}%
\pgfsys@useobject{currentmarker}{}%
\end{pgfscope}%
\begin{pgfscope}%
\pgfsys@transformshift{2.045809in}{0.783541in}%
\pgfsys@useobject{currentmarker}{}%
\end{pgfscope}%
\begin{pgfscope}%
\pgfsys@transformshift{2.046169in}{0.792656in}%
\pgfsys@useobject{currentmarker}{}%
\end{pgfscope}%
\begin{pgfscope}%
\pgfsys@transformshift{2.046528in}{0.771310in}%
\pgfsys@useobject{currentmarker}{}%
\end{pgfscope}%
\begin{pgfscope}%
\pgfsys@transformshift{2.046887in}{0.742986in}%
\pgfsys@useobject{currentmarker}{}%
\end{pgfscope}%
\begin{pgfscope}%
\pgfsys@transformshift{2.047245in}{0.788321in}%
\pgfsys@useobject{currentmarker}{}%
\end{pgfscope}%
\begin{pgfscope}%
\pgfsys@transformshift{2.047602in}{0.795158in}%
\pgfsys@useobject{currentmarker}{}%
\end{pgfscope}%
\begin{pgfscope}%
\pgfsys@transformshift{2.047959in}{0.795058in}%
\pgfsys@useobject{currentmarker}{}%
\end{pgfscope}%
\begin{pgfscope}%
\pgfsys@transformshift{2.048316in}{0.795739in}%
\pgfsys@useobject{currentmarker}{}%
\end{pgfscope}%
\begin{pgfscope}%
\pgfsys@transformshift{2.048671in}{0.792409in}%
\pgfsys@useobject{currentmarker}{}%
\end{pgfscope}%
\begin{pgfscope}%
\pgfsys@transformshift{2.049027in}{0.776553in}%
\pgfsys@useobject{currentmarker}{}%
\end{pgfscope}%
\begin{pgfscope}%
\pgfsys@transformshift{2.049381in}{0.784012in}%
\pgfsys@useobject{currentmarker}{}%
\end{pgfscope}%
\begin{pgfscope}%
\pgfsys@transformshift{2.049735in}{0.772137in}%
\pgfsys@useobject{currentmarker}{}%
\end{pgfscope}%
\begin{pgfscope}%
\pgfsys@transformshift{2.050089in}{0.791689in}%
\pgfsys@useobject{currentmarker}{}%
\end{pgfscope}%
\begin{pgfscope}%
\pgfsys@transformshift{2.050442in}{0.807334in}%
\pgfsys@useobject{currentmarker}{}%
\end{pgfscope}%
\begin{pgfscope}%
\pgfsys@transformshift{2.050794in}{0.752869in}%
\pgfsys@useobject{currentmarker}{}%
\end{pgfscope}%
\begin{pgfscope}%
\pgfsys@transformshift{2.051146in}{0.725112in}%
\pgfsys@useobject{currentmarker}{}%
\end{pgfscope}%
\begin{pgfscope}%
\pgfsys@transformshift{2.051497in}{0.812152in}%
\pgfsys@useobject{currentmarker}{}%
\end{pgfscope}%
\begin{pgfscope}%
\pgfsys@transformshift{2.051848in}{0.825674in}%
\pgfsys@useobject{currentmarker}{}%
\end{pgfscope}%
\begin{pgfscope}%
\pgfsys@transformshift{2.052198in}{0.767659in}%
\pgfsys@useobject{currentmarker}{}%
\end{pgfscope}%
\begin{pgfscope}%
\pgfsys@transformshift{2.052548in}{0.826811in}%
\pgfsys@useobject{currentmarker}{}%
\end{pgfscope}%
\begin{pgfscope}%
\pgfsys@transformshift{2.052897in}{0.828595in}%
\pgfsys@useobject{currentmarker}{}%
\end{pgfscope}%
\begin{pgfscope}%
\pgfsys@transformshift{2.053245in}{0.791855in}%
\pgfsys@useobject{currentmarker}{}%
\end{pgfscope}%
\begin{pgfscope}%
\pgfsys@transformshift{2.053593in}{0.729114in}%
\pgfsys@useobject{currentmarker}{}%
\end{pgfscope}%
\begin{pgfscope}%
\pgfsys@transformshift{2.053941in}{0.723759in}%
\pgfsys@useobject{currentmarker}{}%
\end{pgfscope}%
\begin{pgfscope}%
\pgfsys@transformshift{2.054288in}{0.748747in}%
\pgfsys@useobject{currentmarker}{}%
\end{pgfscope}%
\begin{pgfscope}%
\pgfsys@transformshift{2.054634in}{0.762472in}%
\pgfsys@useobject{currentmarker}{}%
\end{pgfscope}%
\begin{pgfscope}%
\pgfsys@transformshift{2.054980in}{0.785257in}%
\pgfsys@useobject{currentmarker}{}%
\end{pgfscope}%
\begin{pgfscope}%
\pgfsys@transformshift{2.055326in}{0.795376in}%
\pgfsys@useobject{currentmarker}{}%
\end{pgfscope}%
\begin{pgfscope}%
\pgfsys@transformshift{2.055670in}{0.804972in}%
\pgfsys@useobject{currentmarker}{}%
\end{pgfscope}%
\begin{pgfscope}%
\pgfsys@transformshift{2.056015in}{0.789617in}%
\pgfsys@useobject{currentmarker}{}%
\end{pgfscope}%
\begin{pgfscope}%
\pgfsys@transformshift{2.056358in}{0.813792in}%
\pgfsys@useobject{currentmarker}{}%
\end{pgfscope}%
\begin{pgfscope}%
\pgfsys@transformshift{2.056702in}{0.788919in}%
\pgfsys@useobject{currentmarker}{}%
\end{pgfscope}%
\begin{pgfscope}%
\pgfsys@transformshift{2.057044in}{0.727923in}%
\pgfsys@useobject{currentmarker}{}%
\end{pgfscope}%
\begin{pgfscope}%
\pgfsys@transformshift{2.057387in}{0.737615in}%
\pgfsys@useobject{currentmarker}{}%
\end{pgfscope}%
\begin{pgfscope}%
\pgfsys@transformshift{2.057728in}{0.782174in}%
\pgfsys@useobject{currentmarker}{}%
\end{pgfscope}%
\begin{pgfscope}%
\pgfsys@transformshift{2.058069in}{0.781884in}%
\pgfsys@useobject{currentmarker}{}%
\end{pgfscope}%
\begin{pgfscope}%
\pgfsys@transformshift{2.058410in}{0.721037in}%
\pgfsys@useobject{currentmarker}{}%
\end{pgfscope}%
\begin{pgfscope}%
\pgfsys@transformshift{2.058750in}{0.765744in}%
\pgfsys@useobject{currentmarker}{}%
\end{pgfscope}%
\begin{pgfscope}%
\pgfsys@transformshift{2.059090in}{0.808250in}%
\pgfsys@useobject{currentmarker}{}%
\end{pgfscope}%
\begin{pgfscope}%
\pgfsys@transformshift{2.059429in}{0.822864in}%
\pgfsys@useobject{currentmarker}{}%
\end{pgfscope}%
\begin{pgfscope}%
\pgfsys@transformshift{2.059767in}{0.763653in}%
\pgfsys@useobject{currentmarker}{}%
\end{pgfscope}%
\begin{pgfscope}%
\pgfsys@transformshift{2.060106in}{0.761226in}%
\pgfsys@useobject{currentmarker}{}%
\end{pgfscope}%
\begin{pgfscope}%
\pgfsys@transformshift{2.060443in}{0.788990in}%
\pgfsys@useobject{currentmarker}{}%
\end{pgfscope}%
\begin{pgfscope}%
\pgfsys@transformshift{2.060780in}{0.773484in}%
\pgfsys@useobject{currentmarker}{}%
\end{pgfscope}%
\begin{pgfscope}%
\pgfsys@transformshift{2.061117in}{0.790140in}%
\pgfsys@useobject{currentmarker}{}%
\end{pgfscope}%
\begin{pgfscope}%
\pgfsys@transformshift{2.061453in}{0.820679in}%
\pgfsys@useobject{currentmarker}{}%
\end{pgfscope}%
\begin{pgfscope}%
\pgfsys@transformshift{2.061788in}{0.758347in}%
\pgfsys@useobject{currentmarker}{}%
\end{pgfscope}%
\begin{pgfscope}%
\pgfsys@transformshift{2.062123in}{0.758432in}%
\pgfsys@useobject{currentmarker}{}%
\end{pgfscope}%
\begin{pgfscope}%
\pgfsys@transformshift{2.062458in}{0.787997in}%
\pgfsys@useobject{currentmarker}{}%
\end{pgfscope}%
\begin{pgfscope}%
\pgfsys@transformshift{2.062792in}{0.721743in}%
\pgfsys@useobject{currentmarker}{}%
\end{pgfscope}%
\begin{pgfscope}%
\pgfsys@transformshift{2.063126in}{0.800623in}%
\pgfsys@useobject{currentmarker}{}%
\end{pgfscope}%
\begin{pgfscope}%
\pgfsys@transformshift{2.063459in}{0.781085in}%
\pgfsys@useobject{currentmarker}{}%
\end{pgfscope}%
\begin{pgfscope}%
\pgfsys@transformshift{2.063791in}{0.756461in}%
\pgfsys@useobject{currentmarker}{}%
\end{pgfscope}%
\begin{pgfscope}%
\pgfsys@transformshift{2.064123in}{0.715672in}%
\pgfsys@useobject{currentmarker}{}%
\end{pgfscope}%
\begin{pgfscope}%
\pgfsys@transformshift{2.064455in}{0.772093in}%
\pgfsys@useobject{currentmarker}{}%
\end{pgfscope}%
\begin{pgfscope}%
\pgfsys@transformshift{2.064786in}{0.787890in}%
\pgfsys@useobject{currentmarker}{}%
\end{pgfscope}%
\begin{pgfscope}%
\pgfsys@transformshift{2.065117in}{0.787823in}%
\pgfsys@useobject{currentmarker}{}%
\end{pgfscope}%
\begin{pgfscope}%
\pgfsys@transformshift{2.065447in}{0.776960in}%
\pgfsys@useobject{currentmarker}{}%
\end{pgfscope}%
\begin{pgfscope}%
\pgfsys@transformshift{2.065776in}{0.764442in}%
\pgfsys@useobject{currentmarker}{}%
\end{pgfscope}%
\begin{pgfscope}%
\pgfsys@transformshift{2.066106in}{0.746445in}%
\pgfsys@useobject{currentmarker}{}%
\end{pgfscope}%
\begin{pgfscope}%
\pgfsys@transformshift{2.066434in}{0.785003in}%
\pgfsys@useobject{currentmarker}{}%
\end{pgfscope}%
\begin{pgfscope}%
\pgfsys@transformshift{2.066762in}{0.785275in}%
\pgfsys@useobject{currentmarker}{}%
\end{pgfscope}%
\begin{pgfscope}%
\pgfsys@transformshift{2.067090in}{0.766141in}%
\pgfsys@useobject{currentmarker}{}%
\end{pgfscope}%
\begin{pgfscope}%
\pgfsys@transformshift{2.067417in}{0.737749in}%
\pgfsys@useobject{currentmarker}{}%
\end{pgfscope}%
\begin{pgfscope}%
\pgfsys@transformshift{2.067744in}{0.729141in}%
\pgfsys@useobject{currentmarker}{}%
\end{pgfscope}%
\begin{pgfscope}%
\pgfsys@transformshift{2.068070in}{0.738890in}%
\pgfsys@useobject{currentmarker}{}%
\end{pgfscope}%
\begin{pgfscope}%
\pgfsys@transformshift{2.068396in}{0.748667in}%
\pgfsys@useobject{currentmarker}{}%
\end{pgfscope}%
\begin{pgfscope}%
\pgfsys@transformshift{2.068722in}{0.756254in}%
\pgfsys@useobject{currentmarker}{}%
\end{pgfscope}%
\begin{pgfscope}%
\pgfsys@transformshift{2.069046in}{0.778096in}%
\pgfsys@useobject{currentmarker}{}%
\end{pgfscope}%
\begin{pgfscope}%
\pgfsys@transformshift{2.069371in}{0.758567in}%
\pgfsys@useobject{currentmarker}{}%
\end{pgfscope}%
\begin{pgfscope}%
\pgfsys@transformshift{2.069695in}{0.745304in}%
\pgfsys@useobject{currentmarker}{}%
\end{pgfscope}%
\begin{pgfscope}%
\pgfsys@transformshift{2.070018in}{0.727923in}%
\pgfsys@useobject{currentmarker}{}%
\end{pgfscope}%
\begin{pgfscope}%
\pgfsys@transformshift{2.070341in}{0.760419in}%
\pgfsys@useobject{currentmarker}{}%
\end{pgfscope}%
\begin{pgfscope}%
\pgfsys@transformshift{2.070664in}{0.758535in}%
\pgfsys@useobject{currentmarker}{}%
\end{pgfscope}%
\begin{pgfscope}%
\pgfsys@transformshift{2.070986in}{0.752122in}%
\pgfsys@useobject{currentmarker}{}%
\end{pgfscope}%
\begin{pgfscope}%
\pgfsys@transformshift{2.071308in}{0.781840in}%
\pgfsys@useobject{currentmarker}{}%
\end{pgfscope}%
\begin{pgfscope}%
\pgfsys@transformshift{2.071629in}{0.800878in}%
\pgfsys@useobject{currentmarker}{}%
\end{pgfscope}%
\begin{pgfscope}%
\pgfsys@transformshift{2.071949in}{0.790472in}%
\pgfsys@useobject{currentmarker}{}%
\end{pgfscope}%
\begin{pgfscope}%
\pgfsys@transformshift{2.072270in}{0.747592in}%
\pgfsys@useobject{currentmarker}{}%
\end{pgfscope}%
\begin{pgfscope}%
\pgfsys@transformshift{2.072589in}{0.785789in}%
\pgfsys@useobject{currentmarker}{}%
\end{pgfscope}%
\begin{pgfscope}%
\pgfsys@transformshift{2.072909in}{0.787160in}%
\pgfsys@useobject{currentmarker}{}%
\end{pgfscope}%
\begin{pgfscope}%
\pgfsys@transformshift{2.073228in}{0.753818in}%
\pgfsys@useobject{currentmarker}{}%
\end{pgfscope}%
\begin{pgfscope}%
\pgfsys@transformshift{2.073546in}{0.765796in}%
\pgfsys@useobject{currentmarker}{}%
\end{pgfscope}%
\begin{pgfscope}%
\pgfsys@transformshift{2.073864in}{0.725444in}%
\pgfsys@useobject{currentmarker}{}%
\end{pgfscope}%
\begin{pgfscope}%
\pgfsys@transformshift{2.074182in}{0.767389in}%
\pgfsys@useobject{currentmarker}{}%
\end{pgfscope}%
\begin{pgfscope}%
\pgfsys@transformshift{2.074499in}{0.757386in}%
\pgfsys@useobject{currentmarker}{}%
\end{pgfscope}%
\begin{pgfscope}%
\pgfsys@transformshift{2.074815in}{0.739924in}%
\pgfsys@useobject{currentmarker}{}%
\end{pgfscope}%
\begin{pgfscope}%
\pgfsys@transformshift{2.075131in}{0.712449in}%
\pgfsys@useobject{currentmarker}{}%
\end{pgfscope}%
\begin{pgfscope}%
\pgfsys@transformshift{2.075447in}{0.771152in}%
\pgfsys@useobject{currentmarker}{}%
\end{pgfscope}%
\begin{pgfscope}%
\pgfsys@transformshift{2.075763in}{0.791699in}%
\pgfsys@useobject{currentmarker}{}%
\end{pgfscope}%
\begin{pgfscope}%
\pgfsys@transformshift{2.076077in}{0.778587in}%
\pgfsys@useobject{currentmarker}{}%
\end{pgfscope}%
\begin{pgfscope}%
\pgfsys@transformshift{2.076392in}{0.731044in}%
\pgfsys@useobject{currentmarker}{}%
\end{pgfscope}%
\begin{pgfscope}%
\pgfsys@transformshift{2.076706in}{0.719593in}%
\pgfsys@useobject{currentmarker}{}%
\end{pgfscope}%
\begin{pgfscope}%
\pgfsys@transformshift{2.077019in}{0.815000in}%
\pgfsys@useobject{currentmarker}{}%
\end{pgfscope}%
\begin{pgfscope}%
\pgfsys@transformshift{2.077332in}{0.810825in}%
\pgfsys@useobject{currentmarker}{}%
\end{pgfscope}%
\begin{pgfscope}%
\pgfsys@transformshift{2.077645in}{0.777372in}%
\pgfsys@useobject{currentmarker}{}%
\end{pgfscope}%
\begin{pgfscope}%
\pgfsys@transformshift{2.077957in}{0.761272in}%
\pgfsys@useobject{currentmarker}{}%
\end{pgfscope}%
\begin{pgfscope}%
\pgfsys@transformshift{2.078269in}{0.766970in}%
\pgfsys@useobject{currentmarker}{}%
\end{pgfscope}%
\begin{pgfscope}%
\pgfsys@transformshift{2.078581in}{0.785769in}%
\pgfsys@useobject{currentmarker}{}%
\end{pgfscope}%
\begin{pgfscope}%
\pgfsys@transformshift{2.078891in}{0.740198in}%
\pgfsys@useobject{currentmarker}{}%
\end{pgfscope}%
\begin{pgfscope}%
\pgfsys@transformshift{2.079202in}{0.800970in}%
\pgfsys@useobject{currentmarker}{}%
\end{pgfscope}%
\begin{pgfscope}%
\pgfsys@transformshift{2.079512in}{0.793799in}%
\pgfsys@useobject{currentmarker}{}%
\end{pgfscope}%
\begin{pgfscope}%
\pgfsys@transformshift{2.079822in}{0.807703in}%
\pgfsys@useobject{currentmarker}{}%
\end{pgfscope}%
\begin{pgfscope}%
\pgfsys@transformshift{2.080131in}{0.802201in}%
\pgfsys@useobject{currentmarker}{}%
\end{pgfscope}%
\begin{pgfscope}%
\pgfsys@transformshift{2.080440in}{0.746714in}%
\pgfsys@useobject{currentmarker}{}%
\end{pgfscope}%
\begin{pgfscope}%
\pgfsys@transformshift{2.080748in}{0.735326in}%
\pgfsys@useobject{currentmarker}{}%
\end{pgfscope}%
\begin{pgfscope}%
\pgfsys@transformshift{2.081056in}{0.751767in}%
\pgfsys@useobject{currentmarker}{}%
\end{pgfscope}%
\begin{pgfscope}%
\pgfsys@transformshift{2.081364in}{0.751211in}%
\pgfsys@useobject{currentmarker}{}%
\end{pgfscope}%
\begin{pgfscope}%
\pgfsys@transformshift{2.081671in}{0.733680in}%
\pgfsys@useobject{currentmarker}{}%
\end{pgfscope}%
\begin{pgfscope}%
\pgfsys@transformshift{2.081977in}{0.776505in}%
\pgfsys@useobject{currentmarker}{}%
\end{pgfscope}%
\begin{pgfscope}%
\pgfsys@transformshift{2.082284in}{0.786109in}%
\pgfsys@useobject{currentmarker}{}%
\end{pgfscope}%
\begin{pgfscope}%
\pgfsys@transformshift{2.082589in}{0.784557in}%
\pgfsys@useobject{currentmarker}{}%
\end{pgfscope}%
\begin{pgfscope}%
\pgfsys@transformshift{2.082895in}{0.781083in}%
\pgfsys@useobject{currentmarker}{}%
\end{pgfscope}%
\begin{pgfscope}%
\pgfsys@transformshift{2.083200in}{0.724494in}%
\pgfsys@useobject{currentmarker}{}%
\end{pgfscope}%
\begin{pgfscope}%
\pgfsys@transformshift{2.083505in}{0.754370in}%
\pgfsys@useobject{currentmarker}{}%
\end{pgfscope}%
\begin{pgfscope}%
\pgfsys@transformshift{2.083809in}{0.767576in}%
\pgfsys@useobject{currentmarker}{}%
\end{pgfscope}%
\begin{pgfscope}%
\pgfsys@transformshift{2.084113in}{0.722638in}%
\pgfsys@useobject{currentmarker}{}%
\end{pgfscope}%
\begin{pgfscope}%
\pgfsys@transformshift{2.084416in}{0.665345in}%
\pgfsys@useobject{currentmarker}{}%
\end{pgfscope}%
\begin{pgfscope}%
\pgfsys@transformshift{2.084719in}{0.773847in}%
\pgfsys@useobject{currentmarker}{}%
\end{pgfscope}%
\begin{pgfscope}%
\pgfsys@transformshift{2.085021in}{0.791958in}%
\pgfsys@useobject{currentmarker}{}%
\end{pgfscope}%
\begin{pgfscope}%
\pgfsys@transformshift{2.085324in}{0.770189in}%
\pgfsys@useobject{currentmarker}{}%
\end{pgfscope}%
\begin{pgfscope}%
\pgfsys@transformshift{2.085625in}{0.756130in}%
\pgfsys@useobject{currentmarker}{}%
\end{pgfscope}%
\begin{pgfscope}%
\pgfsys@transformshift{2.085927in}{0.759117in}%
\pgfsys@useobject{currentmarker}{}%
\end{pgfscope}%
\begin{pgfscope}%
\pgfsys@transformshift{2.086228in}{0.711966in}%
\pgfsys@useobject{currentmarker}{}%
\end{pgfscope}%
\begin{pgfscope}%
\pgfsys@transformshift{2.086528in}{0.743672in}%
\pgfsys@useobject{currentmarker}{}%
\end{pgfscope}%
\begin{pgfscope}%
\pgfsys@transformshift{2.086828in}{0.796596in}%
\pgfsys@useobject{currentmarker}{}%
\end{pgfscope}%
\begin{pgfscope}%
\pgfsys@transformshift{2.087128in}{0.758227in}%
\pgfsys@useobject{currentmarker}{}%
\end{pgfscope}%
\begin{pgfscope}%
\pgfsys@transformshift{2.087427in}{0.765954in}%
\pgfsys@useobject{currentmarker}{}%
\end{pgfscope}%
\begin{pgfscope}%
\pgfsys@transformshift{2.087726in}{0.793542in}%
\pgfsys@useobject{currentmarker}{}%
\end{pgfscope}%
\begin{pgfscope}%
\pgfsys@transformshift{2.088025in}{0.778983in}%
\pgfsys@useobject{currentmarker}{}%
\end{pgfscope}%
\begin{pgfscope}%
\pgfsys@transformshift{2.088323in}{0.776103in}%
\pgfsys@useobject{currentmarker}{}%
\end{pgfscope}%
\begin{pgfscope}%
\pgfsys@transformshift{2.088621in}{0.768464in}%
\pgfsys@useobject{currentmarker}{}%
\end{pgfscope}%
\begin{pgfscope}%
\pgfsys@transformshift{2.088918in}{0.717334in}%
\pgfsys@useobject{currentmarker}{}%
\end{pgfscope}%
\begin{pgfscope}%
\pgfsys@transformshift{2.089215in}{0.752356in}%
\pgfsys@useobject{currentmarker}{}%
\end{pgfscope}%
\begin{pgfscope}%
\pgfsys@transformshift{2.089512in}{0.752851in}%
\pgfsys@useobject{currentmarker}{}%
\end{pgfscope}%
\begin{pgfscope}%
\pgfsys@transformshift{2.089808in}{0.736706in}%
\pgfsys@useobject{currentmarker}{}%
\end{pgfscope}%
\begin{pgfscope}%
\pgfsys@transformshift{2.090104in}{0.786102in}%
\pgfsys@useobject{currentmarker}{}%
\end{pgfscope}%
\begin{pgfscope}%
\pgfsys@transformshift{2.090399in}{0.794158in}%
\pgfsys@useobject{currentmarker}{}%
\end{pgfscope}%
\begin{pgfscope}%
\pgfsys@transformshift{2.090694in}{0.705120in}%
\pgfsys@useobject{currentmarker}{}%
\end{pgfscope}%
\begin{pgfscope}%
\pgfsys@transformshift{2.090989in}{0.751445in}%
\pgfsys@useobject{currentmarker}{}%
\end{pgfscope}%
\begin{pgfscope}%
\pgfsys@transformshift{2.091283in}{0.710301in}%
\pgfsys@useobject{currentmarker}{}%
\end{pgfscope}%
\begin{pgfscope}%
\pgfsys@transformshift{2.091577in}{0.766715in}%
\pgfsys@useobject{currentmarker}{}%
\end{pgfscope}%
\begin{pgfscope}%
\pgfsys@transformshift{2.091870in}{0.763879in}%
\pgfsys@useobject{currentmarker}{}%
\end{pgfscope}%
\begin{pgfscope}%
\pgfsys@transformshift{2.092163in}{0.745374in}%
\pgfsys@useobject{currentmarker}{}%
\end{pgfscope}%
\begin{pgfscope}%
\pgfsys@transformshift{2.092456in}{0.768062in}%
\pgfsys@useobject{currentmarker}{}%
\end{pgfscope}%
\begin{pgfscope}%
\pgfsys@transformshift{2.092748in}{0.764653in}%
\pgfsys@useobject{currentmarker}{}%
\end{pgfscope}%
\begin{pgfscope}%
\pgfsys@transformshift{2.093040in}{0.711543in}%
\pgfsys@useobject{currentmarker}{}%
\end{pgfscope}%
\begin{pgfscope}%
\pgfsys@transformshift{2.093332in}{0.672255in}%
\pgfsys@useobject{currentmarker}{}%
\end{pgfscope}%
\begin{pgfscope}%
\pgfsys@transformshift{2.093623in}{0.748229in}%
\pgfsys@useobject{currentmarker}{}%
\end{pgfscope}%
\begin{pgfscope}%
\pgfsys@transformshift{2.093914in}{0.771720in}%
\pgfsys@useobject{currentmarker}{}%
\end{pgfscope}%
\begin{pgfscope}%
\pgfsys@transformshift{2.094204in}{0.776900in}%
\pgfsys@useobject{currentmarker}{}%
\end{pgfscope}%
\begin{pgfscope}%
\pgfsys@transformshift{2.094494in}{0.778248in}%
\pgfsys@useobject{currentmarker}{}%
\end{pgfscope}%
\begin{pgfscope}%
\pgfsys@transformshift{2.094784in}{0.752275in}%
\pgfsys@useobject{currentmarker}{}%
\end{pgfscope}%
\begin{pgfscope}%
\pgfsys@transformshift{2.095073in}{0.675818in}%
\pgfsys@useobject{currentmarker}{}%
\end{pgfscope}%
\begin{pgfscope}%
\pgfsys@transformshift{2.095362in}{0.728916in}%
\pgfsys@useobject{currentmarker}{}%
\end{pgfscope}%
\begin{pgfscope}%
\pgfsys@transformshift{2.095651in}{0.756213in}%
\pgfsys@useobject{currentmarker}{}%
\end{pgfscope}%
\begin{pgfscope}%
\pgfsys@transformshift{2.095939in}{0.780391in}%
\pgfsys@useobject{currentmarker}{}%
\end{pgfscope}%
\begin{pgfscope}%
\pgfsys@transformshift{2.096227in}{0.779043in}%
\pgfsys@useobject{currentmarker}{}%
\end{pgfscope}%
\begin{pgfscope}%
\pgfsys@transformshift{2.096514in}{0.762338in}%
\pgfsys@useobject{currentmarker}{}%
\end{pgfscope}%
\begin{pgfscope}%
\pgfsys@transformshift{2.096801in}{0.774066in}%
\pgfsys@useobject{currentmarker}{}%
\end{pgfscope}%
\begin{pgfscope}%
\pgfsys@transformshift{2.097088in}{0.786978in}%
\pgfsys@useobject{currentmarker}{}%
\end{pgfscope}%
\begin{pgfscope}%
\pgfsys@transformshift{2.097375in}{0.810369in}%
\pgfsys@useobject{currentmarker}{}%
\end{pgfscope}%
\begin{pgfscope}%
\pgfsys@transformshift{2.097661in}{0.802044in}%
\pgfsys@useobject{currentmarker}{}%
\end{pgfscope}%
\begin{pgfscope}%
\pgfsys@transformshift{2.097946in}{0.801201in}%
\pgfsys@useobject{currentmarker}{}%
\end{pgfscope}%
\begin{pgfscope}%
\pgfsys@transformshift{2.098231in}{0.766326in}%
\pgfsys@useobject{currentmarker}{}%
\end{pgfscope}%
\begin{pgfscope}%
\pgfsys@transformshift{2.098516in}{0.744037in}%
\pgfsys@useobject{currentmarker}{}%
\end{pgfscope}%
\begin{pgfscope}%
\pgfsys@transformshift{2.098801in}{0.762782in}%
\pgfsys@useobject{currentmarker}{}%
\end{pgfscope}%
\begin{pgfscope}%
\pgfsys@transformshift{2.099085in}{0.792498in}%
\pgfsys@useobject{currentmarker}{}%
\end{pgfscope}%
\begin{pgfscope}%
\pgfsys@transformshift{2.099369in}{0.805887in}%
\pgfsys@useobject{currentmarker}{}%
\end{pgfscope}%
\begin{pgfscope}%
\pgfsys@transformshift{2.099652in}{0.759720in}%
\pgfsys@useobject{currentmarker}{}%
\end{pgfscope}%
\begin{pgfscope}%
\pgfsys@transformshift{2.099936in}{0.744603in}%
\pgfsys@useobject{currentmarker}{}%
\end{pgfscope}%
\begin{pgfscope}%
\pgfsys@transformshift{2.100218in}{0.765481in}%
\pgfsys@useobject{currentmarker}{}%
\end{pgfscope}%
\begin{pgfscope}%
\pgfsys@transformshift{2.100501in}{0.773291in}%
\pgfsys@useobject{currentmarker}{}%
\end{pgfscope}%
\begin{pgfscope}%
\pgfsys@transformshift{2.100783in}{0.755713in}%
\pgfsys@useobject{currentmarker}{}%
\end{pgfscope}%
\begin{pgfscope}%
\pgfsys@transformshift{2.101064in}{0.785410in}%
\pgfsys@useobject{currentmarker}{}%
\end{pgfscope}%
\begin{pgfscope}%
\pgfsys@transformshift{2.101346in}{0.763174in}%
\pgfsys@useobject{currentmarker}{}%
\end{pgfscope}%
\begin{pgfscope}%
\pgfsys@transformshift{2.101627in}{0.737039in}%
\pgfsys@useobject{currentmarker}{}%
\end{pgfscope}%
\begin{pgfscope}%
\pgfsys@transformshift{2.101907in}{0.767780in}%
\pgfsys@useobject{currentmarker}{}%
\end{pgfscope}%
\begin{pgfscope}%
\pgfsys@transformshift{2.102188in}{0.759880in}%
\pgfsys@useobject{currentmarker}{}%
\end{pgfscope}%
\begin{pgfscope}%
\pgfsys@transformshift{2.102468in}{0.725297in}%
\pgfsys@useobject{currentmarker}{}%
\end{pgfscope}%
\begin{pgfscope}%
\pgfsys@transformshift{2.102747in}{0.717293in}%
\pgfsys@useobject{currentmarker}{}%
\end{pgfscope}%
\begin{pgfscope}%
\pgfsys@transformshift{2.103026in}{0.806572in}%
\pgfsys@useobject{currentmarker}{}%
\end{pgfscope}%
\begin{pgfscope}%
\pgfsys@transformshift{2.103305in}{0.795638in}%
\pgfsys@useobject{currentmarker}{}%
\end{pgfscope}%
\begin{pgfscope}%
\pgfsys@transformshift{2.103584in}{0.770456in}%
\pgfsys@useobject{currentmarker}{}%
\end{pgfscope}%
\begin{pgfscope}%
\pgfsys@transformshift{2.103862in}{0.771636in}%
\pgfsys@useobject{currentmarker}{}%
\end{pgfscope}%
\begin{pgfscope}%
\pgfsys@transformshift{2.104140in}{0.765007in}%
\pgfsys@useobject{currentmarker}{}%
\end{pgfscope}%
\begin{pgfscope}%
\pgfsys@transformshift{2.104417in}{0.674943in}%
\pgfsys@useobject{currentmarker}{}%
\end{pgfscope}%
\begin{pgfscope}%
\pgfsys@transformshift{2.104695in}{0.738306in}%
\pgfsys@useobject{currentmarker}{}%
\end{pgfscope}%
\begin{pgfscope}%
\pgfsys@transformshift{2.104972in}{0.763139in}%
\pgfsys@useobject{currentmarker}{}%
\end{pgfscope}%
\begin{pgfscope}%
\pgfsys@transformshift{2.105248in}{0.717427in}%
\pgfsys@useobject{currentmarker}{}%
\end{pgfscope}%
\begin{pgfscope}%
\pgfsys@transformshift{2.105524in}{0.707985in}%
\pgfsys@useobject{currentmarker}{}%
\end{pgfscope}%
\begin{pgfscope}%
\pgfsys@transformshift{2.105800in}{0.738873in}%
\pgfsys@useobject{currentmarker}{}%
\end{pgfscope}%
\begin{pgfscope}%
\pgfsys@transformshift{2.106075in}{0.771415in}%
\pgfsys@useobject{currentmarker}{}%
\end{pgfscope}%
\begin{pgfscope}%
\pgfsys@transformshift{2.106351in}{0.750304in}%
\pgfsys@useobject{currentmarker}{}%
\end{pgfscope}%
\begin{pgfscope}%
\pgfsys@transformshift{2.106625in}{0.761546in}%
\pgfsys@useobject{currentmarker}{}%
\end{pgfscope}%
\begin{pgfscope}%
\pgfsys@transformshift{2.106900in}{0.758811in}%
\pgfsys@useobject{currentmarker}{}%
\end{pgfscope}%
\begin{pgfscope}%
\pgfsys@transformshift{2.107174in}{0.747350in}%
\pgfsys@useobject{currentmarker}{}%
\end{pgfscope}%
\begin{pgfscope}%
\pgfsys@transformshift{2.107448in}{0.756182in}%
\pgfsys@useobject{currentmarker}{}%
\end{pgfscope}%
\begin{pgfscope}%
\pgfsys@transformshift{2.107721in}{0.733143in}%
\pgfsys@useobject{currentmarker}{}%
\end{pgfscope}%
\begin{pgfscope}%
\pgfsys@transformshift{2.107994in}{0.733651in}%
\pgfsys@useobject{currentmarker}{}%
\end{pgfscope}%
\begin{pgfscope}%
\pgfsys@transformshift{2.108267in}{0.744592in}%
\pgfsys@useobject{currentmarker}{}%
\end{pgfscope}%
\begin{pgfscope}%
\pgfsys@transformshift{2.108540in}{0.757944in}%
\pgfsys@useobject{currentmarker}{}%
\end{pgfscope}%
\begin{pgfscope}%
\pgfsys@transformshift{2.108812in}{0.721864in}%
\pgfsys@useobject{currentmarker}{}%
\end{pgfscope}%
\begin{pgfscope}%
\pgfsys@transformshift{2.109084in}{0.742827in}%
\pgfsys@useobject{currentmarker}{}%
\end{pgfscope}%
\begin{pgfscope}%
\pgfsys@transformshift{2.109355in}{0.796482in}%
\pgfsys@useobject{currentmarker}{}%
\end{pgfscope}%
\begin{pgfscope}%
\pgfsys@transformshift{2.109626in}{0.791489in}%
\pgfsys@useobject{currentmarker}{}%
\end{pgfscope}%
\begin{pgfscope}%
\pgfsys@transformshift{2.109897in}{0.743602in}%
\pgfsys@useobject{currentmarker}{}%
\end{pgfscope}%
\begin{pgfscope}%
\pgfsys@transformshift{2.110168in}{0.728055in}%
\pgfsys@useobject{currentmarker}{}%
\end{pgfscope}%
\begin{pgfscope}%
\pgfsys@transformshift{2.110438in}{0.750642in}%
\pgfsys@useobject{currentmarker}{}%
\end{pgfscope}%
\begin{pgfscope}%
\pgfsys@transformshift{2.110708in}{0.718617in}%
\pgfsys@useobject{currentmarker}{}%
\end{pgfscope}%
\begin{pgfscope}%
\pgfsys@transformshift{2.110977in}{0.668139in}%
\pgfsys@useobject{currentmarker}{}%
\end{pgfscope}%
\begin{pgfscope}%
\pgfsys@transformshift{2.111246in}{0.724750in}%
\pgfsys@useobject{currentmarker}{}%
\end{pgfscope}%
\begin{pgfscope}%
\pgfsys@transformshift{2.111515in}{0.775464in}%
\pgfsys@useobject{currentmarker}{}%
\end{pgfscope}%
\begin{pgfscope}%
\pgfsys@transformshift{2.111784in}{0.765697in}%
\pgfsys@useobject{currentmarker}{}%
\end{pgfscope}%
\begin{pgfscope}%
\pgfsys@transformshift{2.112052in}{0.766776in}%
\pgfsys@useobject{currentmarker}{}%
\end{pgfscope}%
\begin{pgfscope}%
\pgfsys@transformshift{2.112320in}{0.763934in}%
\pgfsys@useobject{currentmarker}{}%
\end{pgfscope}%
\begin{pgfscope}%
\pgfsys@transformshift{2.112588in}{0.745045in}%
\pgfsys@useobject{currentmarker}{}%
\end{pgfscope}%
\begin{pgfscope}%
\pgfsys@transformshift{2.112855in}{0.770596in}%
\pgfsys@useobject{currentmarker}{}%
\end{pgfscope}%
\begin{pgfscope}%
\pgfsys@transformshift{2.113122in}{0.737559in}%
\pgfsys@useobject{currentmarker}{}%
\end{pgfscope}%
\begin{pgfscope}%
\pgfsys@transformshift{2.113388in}{0.756427in}%
\pgfsys@useobject{currentmarker}{}%
\end{pgfscope}%
\begin{pgfscope}%
\pgfsys@transformshift{2.113655in}{0.745014in}%
\pgfsys@useobject{currentmarker}{}%
\end{pgfscope}%
\begin{pgfscope}%
\pgfsys@transformshift{2.113921in}{0.769913in}%
\pgfsys@useobject{currentmarker}{}%
\end{pgfscope}%
\begin{pgfscope}%
\pgfsys@transformshift{2.114187in}{0.808915in}%
\pgfsys@useobject{currentmarker}{}%
\end{pgfscope}%
\begin{pgfscope}%
\pgfsys@transformshift{2.114452in}{0.784133in}%
\pgfsys@useobject{currentmarker}{}%
\end{pgfscope}%
\begin{pgfscope}%
\pgfsys@transformshift{2.114717in}{0.722321in}%
\pgfsys@useobject{currentmarker}{}%
\end{pgfscope}%
\begin{pgfscope}%
\pgfsys@transformshift{2.114982in}{0.695509in}%
\pgfsys@useobject{currentmarker}{}%
\end{pgfscope}%
\begin{pgfscope}%
\pgfsys@transformshift{2.115246in}{0.722933in}%
\pgfsys@useobject{currentmarker}{}%
\end{pgfscope}%
\begin{pgfscope}%
\pgfsys@transformshift{2.115510in}{0.771400in}%
\pgfsys@useobject{currentmarker}{}%
\end{pgfscope}%
\begin{pgfscope}%
\pgfsys@transformshift{2.115774in}{0.745901in}%
\pgfsys@useobject{currentmarker}{}%
\end{pgfscope}%
\begin{pgfscope}%
\pgfsys@transformshift{2.116038in}{0.675524in}%
\pgfsys@useobject{currentmarker}{}%
\end{pgfscope}%
\begin{pgfscope}%
\pgfsys@transformshift{2.116301in}{0.690438in}%
\pgfsys@useobject{currentmarker}{}%
\end{pgfscope}%
\begin{pgfscope}%
\pgfsys@transformshift{2.116564in}{0.694552in}%
\pgfsys@useobject{currentmarker}{}%
\end{pgfscope}%
\begin{pgfscope}%
\pgfsys@transformshift{2.116826in}{0.735613in}%
\pgfsys@useobject{currentmarker}{}%
\end{pgfscope}%
\begin{pgfscope}%
\pgfsys@transformshift{2.117089in}{0.745796in}%
\pgfsys@useobject{currentmarker}{}%
\end{pgfscope}%
\begin{pgfscope}%
\pgfsys@transformshift{2.117351in}{0.757476in}%
\pgfsys@useobject{currentmarker}{}%
\end{pgfscope}%
\begin{pgfscope}%
\pgfsys@transformshift{2.117612in}{0.775756in}%
\pgfsys@useobject{currentmarker}{}%
\end{pgfscope}%
\begin{pgfscope}%
\pgfsys@transformshift{2.117874in}{0.744567in}%
\pgfsys@useobject{currentmarker}{}%
\end{pgfscope}%
\begin{pgfscope}%
\pgfsys@transformshift{2.118135in}{0.734748in}%
\pgfsys@useobject{currentmarker}{}%
\end{pgfscope}%
\begin{pgfscope}%
\pgfsys@transformshift{2.118396in}{0.725906in}%
\pgfsys@useobject{currentmarker}{}%
\end{pgfscope}%
\begin{pgfscope}%
\pgfsys@transformshift{2.118656in}{0.732736in}%
\pgfsys@useobject{currentmarker}{}%
\end{pgfscope}%
\begin{pgfscope}%
\pgfsys@transformshift{2.118916in}{0.719722in}%
\pgfsys@useobject{currentmarker}{}%
\end{pgfscope}%
\begin{pgfscope}%
\pgfsys@transformshift{2.119176in}{0.706597in}%
\pgfsys@useobject{currentmarker}{}%
\end{pgfscope}%
\begin{pgfscope}%
\pgfsys@transformshift{2.119436in}{0.717115in}%
\pgfsys@useobject{currentmarker}{}%
\end{pgfscope}%
\begin{pgfscope}%
\pgfsys@transformshift{2.119695in}{0.710174in}%
\pgfsys@useobject{currentmarker}{}%
\end{pgfscope}%
\begin{pgfscope}%
\pgfsys@transformshift{2.119954in}{0.701083in}%
\pgfsys@useobject{currentmarker}{}%
\end{pgfscope}%
\begin{pgfscope}%
\pgfsys@transformshift{2.120213in}{0.738727in}%
\pgfsys@useobject{currentmarker}{}%
\end{pgfscope}%
\begin{pgfscope}%
\pgfsys@transformshift{2.120471in}{0.754360in}%
\pgfsys@useobject{currentmarker}{}%
\end{pgfscope}%
\begin{pgfscope}%
\pgfsys@transformshift{2.120729in}{0.754275in}%
\pgfsys@useobject{currentmarker}{}%
\end{pgfscope}%
\begin{pgfscope}%
\pgfsys@transformshift{2.120987in}{0.733179in}%
\pgfsys@useobject{currentmarker}{}%
\end{pgfscope}%
\begin{pgfscope}%
\pgfsys@transformshift{2.121244in}{0.755059in}%
\pgfsys@useobject{currentmarker}{}%
\end{pgfscope}%
\begin{pgfscope}%
\pgfsys@transformshift{2.121501in}{0.791621in}%
\pgfsys@useobject{currentmarker}{}%
\end{pgfscope}%
\begin{pgfscope}%
\pgfsys@transformshift{2.121758in}{0.787327in}%
\pgfsys@useobject{currentmarker}{}%
\end{pgfscope}%
\begin{pgfscope}%
\pgfsys@transformshift{2.122015in}{0.743552in}%
\pgfsys@useobject{currentmarker}{}%
\end{pgfscope}%
\begin{pgfscope}%
\pgfsys@transformshift{2.122271in}{0.738259in}%
\pgfsys@useobject{currentmarker}{}%
\end{pgfscope}%
\begin{pgfscope}%
\pgfsys@transformshift{2.122527in}{0.782943in}%
\pgfsys@useobject{currentmarker}{}%
\end{pgfscope}%
\begin{pgfscope}%
\pgfsys@transformshift{2.122783in}{0.781467in}%
\pgfsys@useobject{currentmarker}{}%
\end{pgfscope}%
\begin{pgfscope}%
\pgfsys@transformshift{2.123038in}{0.771210in}%
\pgfsys@useobject{currentmarker}{}%
\end{pgfscope}%
\begin{pgfscope}%
\pgfsys@transformshift{2.123293in}{0.709657in}%
\pgfsys@useobject{currentmarker}{}%
\end{pgfscope}%
\begin{pgfscope}%
\pgfsys@transformshift{2.123548in}{0.739582in}%
\pgfsys@useobject{currentmarker}{}%
\end{pgfscope}%
\begin{pgfscope}%
\pgfsys@transformshift{2.123803in}{0.757784in}%
\pgfsys@useobject{currentmarker}{}%
\end{pgfscope}%
\begin{pgfscope}%
\pgfsys@transformshift{2.124057in}{0.748578in}%
\pgfsys@useobject{currentmarker}{}%
\end{pgfscope}%
\begin{pgfscope}%
\pgfsys@transformshift{2.124311in}{0.722612in}%
\pgfsys@useobject{currentmarker}{}%
\end{pgfscope}%
\begin{pgfscope}%
\pgfsys@transformshift{2.124565in}{0.731054in}%
\pgfsys@useobject{currentmarker}{}%
\end{pgfscope}%
\begin{pgfscope}%
\pgfsys@transformshift{2.124818in}{0.761786in}%
\pgfsys@useobject{currentmarker}{}%
\end{pgfscope}%
\begin{pgfscope}%
\pgfsys@transformshift{2.125071in}{0.768314in}%
\pgfsys@useobject{currentmarker}{}%
\end{pgfscope}%
\begin{pgfscope}%
\pgfsys@transformshift{2.125324in}{0.758108in}%
\pgfsys@useobject{currentmarker}{}%
\end{pgfscope}%
\begin{pgfscope}%
\pgfsys@transformshift{2.125577in}{0.719701in}%
\pgfsys@useobject{currentmarker}{}%
\end{pgfscope}%
\begin{pgfscope}%
\pgfsys@transformshift{2.125829in}{0.762119in}%
\pgfsys@useobject{currentmarker}{}%
\end{pgfscope}%
\begin{pgfscope}%
\pgfsys@transformshift{2.126081in}{0.734965in}%
\pgfsys@useobject{currentmarker}{}%
\end{pgfscope}%
\begin{pgfscope}%
\pgfsys@transformshift{2.126333in}{0.745933in}%
\pgfsys@useobject{currentmarker}{}%
\end{pgfscope}%
\begin{pgfscope}%
\pgfsys@transformshift{2.126584in}{0.773927in}%
\pgfsys@useobject{currentmarker}{}%
\end{pgfscope}%
\begin{pgfscope}%
\pgfsys@transformshift{2.126835in}{0.749195in}%
\pgfsys@useobject{currentmarker}{}%
\end{pgfscope}%
\begin{pgfscope}%
\pgfsys@transformshift{2.127086in}{0.726377in}%
\pgfsys@useobject{currentmarker}{}%
\end{pgfscope}%
\begin{pgfscope}%
\pgfsys@transformshift{2.127337in}{0.739536in}%
\pgfsys@useobject{currentmarker}{}%
\end{pgfscope}%
\begin{pgfscope}%
\pgfsys@transformshift{2.127587in}{0.742369in}%
\pgfsys@useobject{currentmarker}{}%
\end{pgfscope}%
\begin{pgfscope}%
\pgfsys@transformshift{2.127837in}{0.692092in}%
\pgfsys@useobject{currentmarker}{}%
\end{pgfscope}%
\begin{pgfscope}%
\pgfsys@transformshift{2.128087in}{0.583372in}%
\pgfsys@useobject{currentmarker}{}%
\end{pgfscope}%
\begin{pgfscope}%
\pgfsys@transformshift{2.128336in}{0.687074in}%
\pgfsys@useobject{currentmarker}{}%
\end{pgfscope}%
\begin{pgfscope}%
\pgfsys@transformshift{2.128586in}{0.700639in}%
\pgfsys@useobject{currentmarker}{}%
\end{pgfscope}%
\begin{pgfscope}%
\pgfsys@transformshift{2.128835in}{0.712495in}%
\pgfsys@useobject{currentmarker}{}%
\end{pgfscope}%
\begin{pgfscope}%
\pgfsys@transformshift{2.129083in}{0.733748in}%
\pgfsys@useobject{currentmarker}{}%
\end{pgfscope}%
\begin{pgfscope}%
\pgfsys@transformshift{2.129332in}{0.757149in}%
\pgfsys@useobject{currentmarker}{}%
\end{pgfscope}%
\begin{pgfscope}%
\pgfsys@transformshift{2.129580in}{0.755118in}%
\pgfsys@useobject{currentmarker}{}%
\end{pgfscope}%
\begin{pgfscope}%
\pgfsys@transformshift{2.129827in}{0.740099in}%
\pgfsys@useobject{currentmarker}{}%
\end{pgfscope}%
\begin{pgfscope}%
\pgfsys@transformshift{2.130075in}{0.757521in}%
\pgfsys@useobject{currentmarker}{}%
\end{pgfscope}%
\begin{pgfscope}%
\pgfsys@transformshift{2.130322in}{0.714419in}%
\pgfsys@useobject{currentmarker}{}%
\end{pgfscope}%
\begin{pgfscope}%
\pgfsys@transformshift{2.130569in}{0.692725in}%
\pgfsys@useobject{currentmarker}{}%
\end{pgfscope}%
\begin{pgfscope}%
\pgfsys@transformshift{2.130816in}{0.671637in}%
\pgfsys@useobject{currentmarker}{}%
\end{pgfscope}%
\begin{pgfscope}%
\pgfsys@transformshift{2.131062in}{0.712582in}%
\pgfsys@useobject{currentmarker}{}%
\end{pgfscope}%
\begin{pgfscope}%
\pgfsys@transformshift{2.131309in}{0.722571in}%
\pgfsys@useobject{currentmarker}{}%
\end{pgfscope}%
\begin{pgfscope}%
\pgfsys@transformshift{2.131555in}{0.728320in}%
\pgfsys@useobject{currentmarker}{}%
\end{pgfscope}%
\begin{pgfscope}%
\pgfsys@transformshift{2.131800in}{0.745514in}%
\pgfsys@useobject{currentmarker}{}%
\end{pgfscope}%
\begin{pgfscope}%
\pgfsys@transformshift{2.132046in}{0.718325in}%
\pgfsys@useobject{currentmarker}{}%
\end{pgfscope}%
\begin{pgfscope}%
\pgfsys@transformshift{2.132291in}{0.739599in}%
\pgfsys@useobject{currentmarker}{}%
\end{pgfscope}%
\begin{pgfscope}%
\pgfsys@transformshift{2.132536in}{0.764963in}%
\pgfsys@useobject{currentmarker}{}%
\end{pgfscope}%
\begin{pgfscope}%
\pgfsys@transformshift{2.132780in}{0.741206in}%
\pgfsys@useobject{currentmarker}{}%
\end{pgfscope}%
\begin{pgfscope}%
\pgfsys@transformshift{2.133025in}{0.693069in}%
\pgfsys@useobject{currentmarker}{}%
\end{pgfscope}%
\begin{pgfscope}%
\pgfsys@transformshift{2.133269in}{0.729560in}%
\pgfsys@useobject{currentmarker}{}%
\end{pgfscope}%
\begin{pgfscope}%
\pgfsys@transformshift{2.133512in}{0.698952in}%
\pgfsys@useobject{currentmarker}{}%
\end{pgfscope}%
\begin{pgfscope}%
\pgfsys@transformshift{2.133756in}{0.714272in}%
\pgfsys@useobject{currentmarker}{}%
\end{pgfscope}%
\begin{pgfscope}%
\pgfsys@transformshift{2.133999in}{0.739100in}%
\pgfsys@useobject{currentmarker}{}%
\end{pgfscope}%
\begin{pgfscope}%
\pgfsys@transformshift{2.134242in}{0.788243in}%
\pgfsys@useobject{currentmarker}{}%
\end{pgfscope}%
\begin{pgfscope}%
\pgfsys@transformshift{2.134485in}{0.766811in}%
\pgfsys@useobject{currentmarker}{}%
\end{pgfscope}%
\begin{pgfscope}%
\pgfsys@transformshift{2.134727in}{0.742117in}%
\pgfsys@useobject{currentmarker}{}%
\end{pgfscope}%
\begin{pgfscope}%
\pgfsys@transformshift{2.134970in}{0.772503in}%
\pgfsys@useobject{currentmarker}{}%
\end{pgfscope}%
\begin{pgfscope}%
\pgfsys@transformshift{2.135212in}{0.755034in}%
\pgfsys@useobject{currentmarker}{}%
\end{pgfscope}%
\begin{pgfscope}%
\pgfsys@transformshift{2.135453in}{0.756957in}%
\pgfsys@useobject{currentmarker}{}%
\end{pgfscope}%
\begin{pgfscope}%
\pgfsys@transformshift{2.135695in}{0.747327in}%
\pgfsys@useobject{currentmarker}{}%
\end{pgfscope}%
\begin{pgfscope}%
\pgfsys@transformshift{2.135936in}{0.745524in}%
\pgfsys@useobject{currentmarker}{}%
\end{pgfscope}%
\begin{pgfscope}%
\pgfsys@transformshift{2.136177in}{0.743153in}%
\pgfsys@useobject{currentmarker}{}%
\end{pgfscope}%
\begin{pgfscope}%
\pgfsys@transformshift{2.136417in}{0.784306in}%
\pgfsys@useobject{currentmarker}{}%
\end{pgfscope}%
\begin{pgfscope}%
\pgfsys@transformshift{2.136658in}{0.756540in}%
\pgfsys@useobject{currentmarker}{}%
\end{pgfscope}%
\begin{pgfscope}%
\pgfsys@transformshift{2.136898in}{0.734460in}%
\pgfsys@useobject{currentmarker}{}%
\end{pgfscope}%
\begin{pgfscope}%
\pgfsys@transformshift{2.137138in}{0.741341in}%
\pgfsys@useobject{currentmarker}{}%
\end{pgfscope}%
\begin{pgfscope}%
\pgfsys@transformshift{2.137377in}{0.710754in}%
\pgfsys@useobject{currentmarker}{}%
\end{pgfscope}%
\begin{pgfscope}%
\pgfsys@transformshift{2.137617in}{0.694449in}%
\pgfsys@useobject{currentmarker}{}%
\end{pgfscope}%
\begin{pgfscope}%
\pgfsys@transformshift{2.137856in}{0.678078in}%
\pgfsys@useobject{currentmarker}{}%
\end{pgfscope}%
\begin{pgfscope}%
\pgfsys@transformshift{2.138095in}{0.723370in}%
\pgfsys@useobject{currentmarker}{}%
\end{pgfscope}%
\begin{pgfscope}%
\pgfsys@transformshift{2.138333in}{0.715188in}%
\pgfsys@useobject{currentmarker}{}%
\end{pgfscope}%
\begin{pgfscope}%
\pgfsys@transformshift{2.138572in}{0.727830in}%
\pgfsys@useobject{currentmarker}{}%
\end{pgfscope}%
\begin{pgfscope}%
\pgfsys@transformshift{2.138810in}{0.747185in}%
\pgfsys@useobject{currentmarker}{}%
\end{pgfscope}%
\begin{pgfscope}%
\pgfsys@transformshift{2.139048in}{0.741907in}%
\pgfsys@useobject{currentmarker}{}%
\end{pgfscope}%
\begin{pgfscope}%
\pgfsys@transformshift{2.139285in}{0.745801in}%
\pgfsys@useobject{currentmarker}{}%
\end{pgfscope}%
\begin{pgfscope}%
\pgfsys@transformshift{2.139523in}{0.733556in}%
\pgfsys@useobject{currentmarker}{}%
\end{pgfscope}%
\begin{pgfscope}%
\pgfsys@transformshift{2.139760in}{0.769085in}%
\pgfsys@useobject{currentmarker}{}%
\end{pgfscope}%
\begin{pgfscope}%
\pgfsys@transformshift{2.139997in}{0.747306in}%
\pgfsys@useobject{currentmarker}{}%
\end{pgfscope}%
\begin{pgfscope}%
\pgfsys@transformshift{2.140233in}{0.717067in}%
\pgfsys@useobject{currentmarker}{}%
\end{pgfscope}%
\begin{pgfscope}%
\pgfsys@transformshift{2.140470in}{0.682604in}%
\pgfsys@useobject{currentmarker}{}%
\end{pgfscope}%
\begin{pgfscope}%
\pgfsys@transformshift{2.140706in}{0.677891in}%
\pgfsys@useobject{currentmarker}{}%
\end{pgfscope}%
\begin{pgfscope}%
\pgfsys@transformshift{2.140942in}{0.696633in}%
\pgfsys@useobject{currentmarker}{}%
\end{pgfscope}%
\begin{pgfscope}%
\pgfsys@transformshift{2.141177in}{0.686876in}%
\pgfsys@useobject{currentmarker}{}%
\end{pgfscope}%
\begin{pgfscope}%
\pgfsys@transformshift{2.141412in}{0.755205in}%
\pgfsys@useobject{currentmarker}{}%
\end{pgfscope}%
\begin{pgfscope}%
\pgfsys@transformshift{2.141648in}{0.759415in}%
\pgfsys@useobject{currentmarker}{}%
\end{pgfscope}%
\begin{pgfscope}%
\pgfsys@transformshift{2.141882in}{0.724966in}%
\pgfsys@useobject{currentmarker}{}%
\end{pgfscope}%
\begin{pgfscope}%
\pgfsys@transformshift{2.142117in}{0.723721in}%
\pgfsys@useobject{currentmarker}{}%
\end{pgfscope}%
\begin{pgfscope}%
\pgfsys@transformshift{2.142351in}{0.717700in}%
\pgfsys@useobject{currentmarker}{}%
\end{pgfscope}%
\begin{pgfscope}%
\pgfsys@transformshift{2.142586in}{0.678488in}%
\pgfsys@useobject{currentmarker}{}%
\end{pgfscope}%
\begin{pgfscope}%
\pgfsys@transformshift{2.142820in}{0.688144in}%
\pgfsys@useobject{currentmarker}{}%
\end{pgfscope}%
\begin{pgfscope}%
\pgfsys@transformshift{2.143053in}{0.729649in}%
\pgfsys@useobject{currentmarker}{}%
\end{pgfscope}%
\begin{pgfscope}%
\pgfsys@transformshift{2.143287in}{0.738245in}%
\pgfsys@useobject{currentmarker}{}%
\end{pgfscope}%
\begin{pgfscope}%
\pgfsys@transformshift{2.143520in}{0.751378in}%
\pgfsys@useobject{currentmarker}{}%
\end{pgfscope}%
\begin{pgfscope}%
\pgfsys@transformshift{2.143753in}{0.742542in}%
\pgfsys@useobject{currentmarker}{}%
\end{pgfscope}%
\begin{pgfscope}%
\pgfsys@transformshift{2.143985in}{0.727847in}%
\pgfsys@useobject{currentmarker}{}%
\end{pgfscope}%
\begin{pgfscope}%
\pgfsys@transformshift{2.144218in}{0.782439in}%
\pgfsys@useobject{currentmarker}{}%
\end{pgfscope}%
\begin{pgfscope}%
\pgfsys@transformshift{2.144450in}{0.777934in}%
\pgfsys@useobject{currentmarker}{}%
\end{pgfscope}%
\begin{pgfscope}%
\pgfsys@transformshift{2.144682in}{0.727502in}%
\pgfsys@useobject{currentmarker}{}%
\end{pgfscope}%
\begin{pgfscope}%
\pgfsys@transformshift{2.144914in}{0.724140in}%
\pgfsys@useobject{currentmarker}{}%
\end{pgfscope}%
\begin{pgfscope}%
\pgfsys@transformshift{2.145145in}{0.671094in}%
\pgfsys@useobject{currentmarker}{}%
\end{pgfscope}%
\begin{pgfscope}%
\pgfsys@transformshift{2.145376in}{0.679968in}%
\pgfsys@useobject{currentmarker}{}%
\end{pgfscope}%
\begin{pgfscope}%
\pgfsys@transformshift{2.145607in}{0.718964in}%
\pgfsys@useobject{currentmarker}{}%
\end{pgfscope}%
\begin{pgfscope}%
\pgfsys@transformshift{2.145838in}{0.738431in}%
\pgfsys@useobject{currentmarker}{}%
\end{pgfscope}%
\begin{pgfscope}%
\pgfsys@transformshift{2.146069in}{0.765222in}%
\pgfsys@useobject{currentmarker}{}%
\end{pgfscope}%
\begin{pgfscope}%
\pgfsys@transformshift{2.146299in}{0.742933in}%
\pgfsys@useobject{currentmarker}{}%
\end{pgfscope}%
\begin{pgfscope}%
\pgfsys@transformshift{2.146529in}{0.735651in}%
\pgfsys@useobject{currentmarker}{}%
\end{pgfscope}%
\begin{pgfscope}%
\pgfsys@transformshift{2.146759in}{0.708925in}%
\pgfsys@useobject{currentmarker}{}%
\end{pgfscope}%
\begin{pgfscope}%
\pgfsys@transformshift{2.146988in}{0.657875in}%
\pgfsys@useobject{currentmarker}{}%
\end{pgfscope}%
\begin{pgfscope}%
\pgfsys@transformshift{2.147218in}{0.649352in}%
\pgfsys@useobject{currentmarker}{}%
\end{pgfscope}%
\begin{pgfscope}%
\pgfsys@transformshift{2.147447in}{0.691531in}%
\pgfsys@useobject{currentmarker}{}%
\end{pgfscope}%
\begin{pgfscope}%
\pgfsys@transformshift{2.147676in}{0.698453in}%
\pgfsys@useobject{currentmarker}{}%
\end{pgfscope}%
\begin{pgfscope}%
\pgfsys@transformshift{2.147904in}{0.670853in}%
\pgfsys@useobject{currentmarker}{}%
\end{pgfscope}%
\begin{pgfscope}%
\pgfsys@transformshift{2.148133in}{0.702732in}%
\pgfsys@useobject{currentmarker}{}%
\end{pgfscope}%
\begin{pgfscope}%
\pgfsys@transformshift{2.148361in}{0.719129in}%
\pgfsys@useobject{currentmarker}{}%
\end{pgfscope}%
\begin{pgfscope}%
\pgfsys@transformshift{2.148589in}{0.720260in}%
\pgfsys@useobject{currentmarker}{}%
\end{pgfscope}%
\begin{pgfscope}%
\pgfsys@transformshift{2.148817in}{0.737484in}%
\pgfsys@useobject{currentmarker}{}%
\end{pgfscope}%
\begin{pgfscope}%
\pgfsys@transformshift{2.149044in}{0.687334in}%
\pgfsys@useobject{currentmarker}{}%
\end{pgfscope}%
\begin{pgfscope}%
\pgfsys@transformshift{2.149271in}{0.672022in}%
\pgfsys@useobject{currentmarker}{}%
\end{pgfscope}%
\begin{pgfscope}%
\pgfsys@transformshift{2.149499in}{0.718057in}%
\pgfsys@useobject{currentmarker}{}%
\end{pgfscope}%
\begin{pgfscope}%
\pgfsys@transformshift{2.149725in}{0.735866in}%
\pgfsys@useobject{currentmarker}{}%
\end{pgfscope}%
\begin{pgfscope}%
\pgfsys@transformshift{2.149952in}{0.719864in}%
\pgfsys@useobject{currentmarker}{}%
\end{pgfscope}%
\begin{pgfscope}%
\pgfsys@transformshift{2.150178in}{0.715389in}%
\pgfsys@useobject{currentmarker}{}%
\end{pgfscope}%
\begin{pgfscope}%
\pgfsys@transformshift{2.150404in}{0.732510in}%
\pgfsys@useobject{currentmarker}{}%
\end{pgfscope}%
\begin{pgfscope}%
\pgfsys@transformshift{2.150630in}{0.759267in}%
\pgfsys@useobject{currentmarker}{}%
\end{pgfscope}%
\begin{pgfscope}%
\pgfsys@transformshift{2.150856in}{0.772014in}%
\pgfsys@useobject{currentmarker}{}%
\end{pgfscope}%
\begin{pgfscope}%
\pgfsys@transformshift{2.151081in}{0.739052in}%
\pgfsys@useobject{currentmarker}{}%
\end{pgfscope}%
\begin{pgfscope}%
\pgfsys@transformshift{2.151307in}{0.744961in}%
\pgfsys@useobject{currentmarker}{}%
\end{pgfscope}%
\begin{pgfscope}%
\pgfsys@transformshift{2.151532in}{0.756346in}%
\pgfsys@useobject{currentmarker}{}%
\end{pgfscope}%
\begin{pgfscope}%
\pgfsys@transformshift{2.151756in}{0.786215in}%
\pgfsys@useobject{currentmarker}{}%
\end{pgfscope}%
\begin{pgfscope}%
\pgfsys@transformshift{2.151981in}{0.775746in}%
\pgfsys@useobject{currentmarker}{}%
\end{pgfscope}%
\begin{pgfscope}%
\pgfsys@transformshift{2.152205in}{0.736112in}%
\pgfsys@useobject{currentmarker}{}%
\end{pgfscope}%
\begin{pgfscope}%
\pgfsys@transformshift{2.152429in}{0.726876in}%
\pgfsys@useobject{currentmarker}{}%
\end{pgfscope}%
\begin{pgfscope}%
\pgfsys@transformshift{2.152653in}{0.734080in}%
\pgfsys@useobject{currentmarker}{}%
\end{pgfscope}%
\begin{pgfscope}%
\pgfsys@transformshift{2.152877in}{0.739071in}%
\pgfsys@useobject{currentmarker}{}%
\end{pgfscope}%
\begin{pgfscope}%
\pgfsys@transformshift{2.153100in}{0.745069in}%
\pgfsys@useobject{currentmarker}{}%
\end{pgfscope}%
\begin{pgfscope}%
\pgfsys@transformshift{2.153323in}{0.745703in}%
\pgfsys@useobject{currentmarker}{}%
\end{pgfscope}%
\begin{pgfscope}%
\pgfsys@transformshift{2.153546in}{0.724801in}%
\pgfsys@useobject{currentmarker}{}%
\end{pgfscope}%
\begin{pgfscope}%
\pgfsys@transformshift{2.153769in}{0.715722in}%
\pgfsys@useobject{currentmarker}{}%
\end{pgfscope}%
\begin{pgfscope}%
\pgfsys@transformshift{2.153992in}{0.698345in}%
\pgfsys@useobject{currentmarker}{}%
\end{pgfscope}%
\begin{pgfscope}%
\pgfsys@transformshift{2.154214in}{0.657475in}%
\pgfsys@useobject{currentmarker}{}%
\end{pgfscope}%
\begin{pgfscope}%
\pgfsys@transformshift{2.154436in}{0.664817in}%
\pgfsys@useobject{currentmarker}{}%
\end{pgfscope}%
\begin{pgfscope}%
\pgfsys@transformshift{2.154658in}{0.680556in}%
\pgfsys@useobject{currentmarker}{}%
\end{pgfscope}%
\begin{pgfscope}%
\pgfsys@transformshift{2.154880in}{0.672700in}%
\pgfsys@useobject{currentmarker}{}%
\end{pgfscope}%
\begin{pgfscope}%
\pgfsys@transformshift{2.155101in}{0.647495in}%
\pgfsys@useobject{currentmarker}{}%
\end{pgfscope}%
\begin{pgfscope}%
\pgfsys@transformshift{2.155322in}{0.682551in}%
\pgfsys@useobject{currentmarker}{}%
\end{pgfscope}%
\begin{pgfscope}%
\pgfsys@transformshift{2.155543in}{0.699665in}%
\pgfsys@useobject{currentmarker}{}%
\end{pgfscope}%
\begin{pgfscope}%
\pgfsys@transformshift{2.155764in}{0.728117in}%
\pgfsys@useobject{currentmarker}{}%
\end{pgfscope}%
\begin{pgfscope}%
\pgfsys@transformshift{2.155985in}{0.738862in}%
\pgfsys@useobject{currentmarker}{}%
\end{pgfscope}%
\begin{pgfscope}%
\pgfsys@transformshift{2.156205in}{0.756020in}%
\pgfsys@useobject{currentmarker}{}%
\end{pgfscope}%
\begin{pgfscope}%
\pgfsys@transformshift{2.156425in}{0.728128in}%
\pgfsys@useobject{currentmarker}{}%
\end{pgfscope}%
\begin{pgfscope}%
\pgfsys@transformshift{2.156645in}{0.706383in}%
\pgfsys@useobject{currentmarker}{}%
\end{pgfscope}%
\begin{pgfscope}%
\pgfsys@transformshift{2.156865in}{0.744930in}%
\pgfsys@useobject{currentmarker}{}%
\end{pgfscope}%
\begin{pgfscope}%
\pgfsys@transformshift{2.157084in}{0.736139in}%
\pgfsys@useobject{currentmarker}{}%
\end{pgfscope}%
\begin{pgfscope}%
\pgfsys@transformshift{2.157304in}{0.678594in}%
\pgfsys@useobject{currentmarker}{}%
\end{pgfscope}%
\begin{pgfscope}%
\pgfsys@transformshift{2.157523in}{0.720768in}%
\pgfsys@useobject{currentmarker}{}%
\end{pgfscope}%
\begin{pgfscope}%
\pgfsys@transformshift{2.157741in}{0.745489in}%
\pgfsys@useobject{currentmarker}{}%
\end{pgfscope}%
\begin{pgfscope}%
\pgfsys@transformshift{2.157960in}{0.757569in}%
\pgfsys@useobject{currentmarker}{}%
\end{pgfscope}%
\begin{pgfscope}%
\pgfsys@transformshift{2.158179in}{0.743203in}%
\pgfsys@useobject{currentmarker}{}%
\end{pgfscope}%
\begin{pgfscope}%
\pgfsys@transformshift{2.158397in}{0.707806in}%
\pgfsys@useobject{currentmarker}{}%
\end{pgfscope}%
\begin{pgfscope}%
\pgfsys@transformshift{2.158615in}{0.706890in}%
\pgfsys@useobject{currentmarker}{}%
\end{pgfscope}%
\begin{pgfscope}%
\pgfsys@transformshift{2.158833in}{0.696165in}%
\pgfsys@useobject{currentmarker}{}%
\end{pgfscope}%
\begin{pgfscope}%
\pgfsys@transformshift{2.159050in}{0.712572in}%
\pgfsys@useobject{currentmarker}{}%
\end{pgfscope}%
\begin{pgfscope}%
\pgfsys@transformshift{2.159268in}{0.746496in}%
\pgfsys@useobject{currentmarker}{}%
\end{pgfscope}%
\begin{pgfscope}%
\pgfsys@transformshift{2.159485in}{0.726192in}%
\pgfsys@useobject{currentmarker}{}%
\end{pgfscope}%
\begin{pgfscope}%
\pgfsys@transformshift{2.159702in}{0.727406in}%
\pgfsys@useobject{currentmarker}{}%
\end{pgfscope}%
\begin{pgfscope}%
\pgfsys@transformshift{2.159918in}{0.738646in}%
\pgfsys@useobject{currentmarker}{}%
\end{pgfscope}%
\begin{pgfscope}%
\pgfsys@transformshift{2.160135in}{0.740001in}%
\pgfsys@useobject{currentmarker}{}%
\end{pgfscope}%
\begin{pgfscope}%
\pgfsys@transformshift{2.160351in}{0.723584in}%
\pgfsys@useobject{currentmarker}{}%
\end{pgfscope}%
\begin{pgfscope}%
\pgfsys@transformshift{2.160567in}{0.656126in}%
\pgfsys@useobject{currentmarker}{}%
\end{pgfscope}%
\begin{pgfscope}%
\pgfsys@transformshift{2.160783in}{0.701592in}%
\pgfsys@useobject{currentmarker}{}%
\end{pgfscope}%
\begin{pgfscope}%
\pgfsys@transformshift{2.160999in}{0.663691in}%
\pgfsys@useobject{currentmarker}{}%
\end{pgfscope}%
\begin{pgfscope}%
\pgfsys@transformshift{2.161214in}{0.692026in}%
\pgfsys@useobject{currentmarker}{}%
\end{pgfscope}%
\begin{pgfscope}%
\pgfsys@transformshift{2.161430in}{0.770786in}%
\pgfsys@useobject{currentmarker}{}%
\end{pgfscope}%
\begin{pgfscope}%
\pgfsys@transformshift{2.161645in}{0.759329in}%
\pgfsys@useobject{currentmarker}{}%
\end{pgfscope}%
\begin{pgfscope}%
\pgfsys@transformshift{2.161860in}{0.710355in}%
\pgfsys@useobject{currentmarker}{}%
\end{pgfscope}%
\begin{pgfscope}%
\pgfsys@transformshift{2.162074in}{0.739331in}%
\pgfsys@useobject{currentmarker}{}%
\end{pgfscope}%
\begin{pgfscope}%
\pgfsys@transformshift{2.162289in}{0.708310in}%
\pgfsys@useobject{currentmarker}{}%
\end{pgfscope}%
\begin{pgfscope}%
\pgfsys@transformshift{2.162503in}{0.669930in}%
\pgfsys@useobject{currentmarker}{}%
\end{pgfscope}%
\begin{pgfscope}%
\pgfsys@transformshift{2.162717in}{0.728552in}%
\pgfsys@useobject{currentmarker}{}%
\end{pgfscope}%
\begin{pgfscope}%
\pgfsys@transformshift{2.162931in}{0.755829in}%
\pgfsys@useobject{currentmarker}{}%
\end{pgfscope}%
\begin{pgfscope}%
\pgfsys@transformshift{2.163145in}{0.713167in}%
\pgfsys@useobject{currentmarker}{}%
\end{pgfscope}%
\begin{pgfscope}%
\pgfsys@transformshift{2.163358in}{0.672702in}%
\pgfsys@useobject{currentmarker}{}%
\end{pgfscope}%
\begin{pgfscope}%
\pgfsys@transformshift{2.163571in}{0.708122in}%
\pgfsys@useobject{currentmarker}{}%
\end{pgfscope}%
\begin{pgfscope}%
\pgfsys@transformshift{2.163784in}{0.744918in}%
\pgfsys@useobject{currentmarker}{}%
\end{pgfscope}%
\begin{pgfscope}%
\pgfsys@transformshift{2.163997in}{0.745216in}%
\pgfsys@useobject{currentmarker}{}%
\end{pgfscope}%
\begin{pgfscope}%
\pgfsys@transformshift{2.164210in}{0.711256in}%
\pgfsys@useobject{currentmarker}{}%
\end{pgfscope}%
\begin{pgfscope}%
\pgfsys@transformshift{2.164422in}{0.651359in}%
\pgfsys@useobject{currentmarker}{}%
\end{pgfscope}%
\begin{pgfscope}%
\pgfsys@transformshift{2.164635in}{0.623871in}%
\pgfsys@useobject{currentmarker}{}%
\end{pgfscope}%
\begin{pgfscope}%
\pgfsys@transformshift{2.164847in}{0.634790in}%
\pgfsys@useobject{currentmarker}{}%
\end{pgfscope}%
\begin{pgfscope}%
\pgfsys@transformshift{2.165058in}{0.667293in}%
\pgfsys@useobject{currentmarker}{}%
\end{pgfscope}%
\begin{pgfscope}%
\pgfsys@transformshift{2.165270in}{0.696907in}%
\pgfsys@useobject{currentmarker}{}%
\end{pgfscope}%
\begin{pgfscope}%
\pgfsys@transformshift{2.165481in}{0.698575in}%
\pgfsys@useobject{currentmarker}{}%
\end{pgfscope}%
\begin{pgfscope}%
\pgfsys@transformshift{2.165693in}{0.673732in}%
\pgfsys@useobject{currentmarker}{}%
\end{pgfscope}%
\begin{pgfscope}%
\pgfsys@transformshift{2.165904in}{0.717383in}%
\pgfsys@useobject{currentmarker}{}%
\end{pgfscope}%
\begin{pgfscope}%
\pgfsys@transformshift{2.166115in}{0.734440in}%
\pgfsys@useobject{currentmarker}{}%
\end{pgfscope}%
\begin{pgfscope}%
\pgfsys@transformshift{2.166325in}{0.721465in}%
\pgfsys@useobject{currentmarker}{}%
\end{pgfscope}%
\begin{pgfscope}%
\pgfsys@transformshift{2.166536in}{0.727223in}%
\pgfsys@useobject{currentmarker}{}%
\end{pgfscope}%
\begin{pgfscope}%
\pgfsys@transformshift{2.166746in}{0.726318in}%
\pgfsys@useobject{currentmarker}{}%
\end{pgfscope}%
\begin{pgfscope}%
\pgfsys@transformshift{2.166956in}{0.705061in}%
\pgfsys@useobject{currentmarker}{}%
\end{pgfscope}%
\begin{pgfscope}%
\pgfsys@transformshift{2.167166in}{0.713422in}%
\pgfsys@useobject{currentmarker}{}%
\end{pgfscope}%
\begin{pgfscope}%
\pgfsys@transformshift{2.167375in}{0.721215in}%
\pgfsys@useobject{currentmarker}{}%
\end{pgfscope}%
\begin{pgfscope}%
\pgfsys@transformshift{2.167585in}{0.669433in}%
\pgfsys@useobject{currentmarker}{}%
\end{pgfscope}%
\begin{pgfscope}%
\pgfsys@transformshift{2.167794in}{0.703672in}%
\pgfsys@useobject{currentmarker}{}%
\end{pgfscope}%
\begin{pgfscope}%
\pgfsys@transformshift{2.168003in}{0.720923in}%
\pgfsys@useobject{currentmarker}{}%
\end{pgfscope}%
\begin{pgfscope}%
\pgfsys@transformshift{2.168212in}{0.773505in}%
\pgfsys@useobject{currentmarker}{}%
\end{pgfscope}%
\begin{pgfscope}%
\pgfsys@transformshift{2.168421in}{0.763206in}%
\pgfsys@useobject{currentmarker}{}%
\end{pgfscope}%
\begin{pgfscope}%
\pgfsys@transformshift{2.168629in}{0.720212in}%
\pgfsys@useobject{currentmarker}{}%
\end{pgfscope}%
\begin{pgfscope}%
\pgfsys@transformshift{2.168838in}{0.711242in}%
\pgfsys@useobject{currentmarker}{}%
\end{pgfscope}%
\begin{pgfscope}%
\pgfsys@transformshift{2.169046in}{0.668423in}%
\pgfsys@useobject{currentmarker}{}%
\end{pgfscope}%
\begin{pgfscope}%
\pgfsys@transformshift{2.169254in}{0.699231in}%
\pgfsys@useobject{currentmarker}{}%
\end{pgfscope}%
\begin{pgfscope}%
\pgfsys@transformshift{2.169461in}{0.696676in}%
\pgfsys@useobject{currentmarker}{}%
\end{pgfscope}%
\begin{pgfscope}%
\pgfsys@transformshift{2.169669in}{0.681883in}%
\pgfsys@useobject{currentmarker}{}%
\end{pgfscope}%
\begin{pgfscope}%
\pgfsys@transformshift{2.169876in}{0.688953in}%
\pgfsys@useobject{currentmarker}{}%
\end{pgfscope}%
\begin{pgfscope}%
\pgfsys@transformshift{2.170083in}{0.736737in}%
\pgfsys@useobject{currentmarker}{}%
\end{pgfscope}%
\begin{pgfscope}%
\pgfsys@transformshift{2.170290in}{0.726701in}%
\pgfsys@useobject{currentmarker}{}%
\end{pgfscope}%
\begin{pgfscope}%
\pgfsys@transformshift{2.170497in}{0.708945in}%
\pgfsys@useobject{currentmarker}{}%
\end{pgfscope}%
\begin{pgfscope}%
\pgfsys@transformshift{2.170704in}{0.754924in}%
\pgfsys@useobject{currentmarker}{}%
\end{pgfscope}%
\begin{pgfscope}%
\pgfsys@transformshift{2.170910in}{0.756449in}%
\pgfsys@useobject{currentmarker}{}%
\end{pgfscope}%
\begin{pgfscope}%
\pgfsys@transformshift{2.171116in}{0.669675in}%
\pgfsys@useobject{currentmarker}{}%
\end{pgfscope}%
\begin{pgfscope}%
\pgfsys@transformshift{2.171322in}{0.648596in}%
\pgfsys@useobject{currentmarker}{}%
\end{pgfscope}%
\begin{pgfscope}%
\pgfsys@transformshift{2.171528in}{0.720611in}%
\pgfsys@useobject{currentmarker}{}%
\end{pgfscope}%
\begin{pgfscope}%
\pgfsys@transformshift{2.171734in}{0.736635in}%
\pgfsys@useobject{currentmarker}{}%
\end{pgfscope}%
\begin{pgfscope}%
\pgfsys@transformshift{2.171939in}{0.700356in}%
\pgfsys@useobject{currentmarker}{}%
\end{pgfscope}%
\begin{pgfscope}%
\pgfsys@transformshift{2.172144in}{0.676777in}%
\pgfsys@useobject{currentmarker}{}%
\end{pgfscope}%
\begin{pgfscope}%
\pgfsys@transformshift{2.172350in}{0.722993in}%
\pgfsys@useobject{currentmarker}{}%
\end{pgfscope}%
\begin{pgfscope}%
\pgfsys@transformshift{2.172554in}{0.708050in}%
\pgfsys@useobject{currentmarker}{}%
\end{pgfscope}%
\begin{pgfscope}%
\pgfsys@transformshift{2.172759in}{0.671673in}%
\pgfsys@useobject{currentmarker}{}%
\end{pgfscope}%
\begin{pgfscope}%
\pgfsys@transformshift{2.172964in}{0.677384in}%
\pgfsys@useobject{currentmarker}{}%
\end{pgfscope}%
\begin{pgfscope}%
\pgfsys@transformshift{2.173168in}{0.733607in}%
\pgfsys@useobject{currentmarker}{}%
\end{pgfscope}%
\begin{pgfscope}%
\pgfsys@transformshift{2.173372in}{0.764656in}%
\pgfsys@useobject{currentmarker}{}%
\end{pgfscope}%
\begin{pgfscope}%
\pgfsys@transformshift{2.173576in}{0.706235in}%
\pgfsys@useobject{currentmarker}{}%
\end{pgfscope}%
\begin{pgfscope}%
\pgfsys@transformshift{2.173780in}{0.733161in}%
\pgfsys@useobject{currentmarker}{}%
\end{pgfscope}%
\begin{pgfscope}%
\pgfsys@transformshift{2.173983in}{0.742567in}%
\pgfsys@useobject{currentmarker}{}%
\end{pgfscope}%
\begin{pgfscope}%
\pgfsys@transformshift{2.174187in}{0.710698in}%
\pgfsys@useobject{currentmarker}{}%
\end{pgfscope}%
\begin{pgfscope}%
\pgfsys@transformshift{2.174390in}{0.650475in}%
\pgfsys@useobject{currentmarker}{}%
\end{pgfscope}%
\begin{pgfscope}%
\pgfsys@transformshift{2.174593in}{0.657691in}%
\pgfsys@useobject{currentmarker}{}%
\end{pgfscope}%
\begin{pgfscope}%
\pgfsys@transformshift{2.174796in}{0.691387in}%
\pgfsys@useobject{currentmarker}{}%
\end{pgfscope}%
\begin{pgfscope}%
\pgfsys@transformshift{2.174999in}{0.701248in}%
\pgfsys@useobject{currentmarker}{}%
\end{pgfscope}%
\begin{pgfscope}%
\pgfsys@transformshift{2.175201in}{0.698631in}%
\pgfsys@useobject{currentmarker}{}%
\end{pgfscope}%
\begin{pgfscope}%
\pgfsys@transformshift{2.175403in}{0.715662in}%
\pgfsys@useobject{currentmarker}{}%
\end{pgfscope}%
\begin{pgfscope}%
\pgfsys@transformshift{2.175605in}{0.734154in}%
\pgfsys@useobject{currentmarker}{}%
\end{pgfscope}%
\begin{pgfscope}%
\pgfsys@transformshift{2.175807in}{0.743521in}%
\pgfsys@useobject{currentmarker}{}%
\end{pgfscope}%
\begin{pgfscope}%
\pgfsys@transformshift{2.176009in}{0.739611in}%
\pgfsys@useobject{currentmarker}{}%
\end{pgfscope}%
\begin{pgfscope}%
\pgfsys@transformshift{2.176211in}{0.717308in}%
\pgfsys@useobject{currentmarker}{}%
\end{pgfscope}%
\begin{pgfscope}%
\pgfsys@transformshift{2.176412in}{0.728441in}%
\pgfsys@useobject{currentmarker}{}%
\end{pgfscope}%
\begin{pgfscope}%
\pgfsys@transformshift{2.176613in}{0.694546in}%
\pgfsys@useobject{currentmarker}{}%
\end{pgfscope}%
\begin{pgfscope}%
\pgfsys@transformshift{2.176814in}{0.673112in}%
\pgfsys@useobject{currentmarker}{}%
\end{pgfscope}%
\begin{pgfscope}%
\pgfsys@transformshift{2.177015in}{0.696367in}%
\pgfsys@useobject{currentmarker}{}%
\end{pgfscope}%
\begin{pgfscope}%
\pgfsys@transformshift{2.177216in}{0.705838in}%
\pgfsys@useobject{currentmarker}{}%
\end{pgfscope}%
\begin{pgfscope}%
\pgfsys@transformshift{2.177416in}{0.678078in}%
\pgfsys@useobject{currentmarker}{}%
\end{pgfscope}%
\begin{pgfscope}%
\pgfsys@transformshift{2.177617in}{0.681173in}%
\pgfsys@useobject{currentmarker}{}%
\end{pgfscope}%
\begin{pgfscope}%
\pgfsys@transformshift{2.177817in}{0.761714in}%
\pgfsys@useobject{currentmarker}{}%
\end{pgfscope}%
\begin{pgfscope}%
\pgfsys@transformshift{2.178017in}{0.732977in}%
\pgfsys@useobject{currentmarker}{}%
\end{pgfscope}%
\begin{pgfscope}%
\pgfsys@transformshift{2.178217in}{0.721785in}%
\pgfsys@useobject{currentmarker}{}%
\end{pgfscope}%
\begin{pgfscope}%
\pgfsys@transformshift{2.178416in}{0.685483in}%
\pgfsys@useobject{currentmarker}{}%
\end{pgfscope}%
\begin{pgfscope}%
\pgfsys@transformshift{2.178616in}{0.684326in}%
\pgfsys@useobject{currentmarker}{}%
\end{pgfscope}%
\begin{pgfscope}%
\pgfsys@transformshift{2.178815in}{0.718565in}%
\pgfsys@useobject{currentmarker}{}%
\end{pgfscope}%
\begin{pgfscope}%
\pgfsys@transformshift{2.179014in}{0.685271in}%
\pgfsys@useobject{currentmarker}{}%
\end{pgfscope}%
\begin{pgfscope}%
\pgfsys@transformshift{2.179213in}{0.664190in}%
\pgfsys@useobject{currentmarker}{}%
\end{pgfscope}%
\begin{pgfscope}%
\pgfsys@transformshift{2.179411in}{0.685953in}%
\pgfsys@useobject{currentmarker}{}%
\end{pgfscope}%
\begin{pgfscope}%
\pgfsys@transformshift{2.179610in}{0.738743in}%
\pgfsys@useobject{currentmarker}{}%
\end{pgfscope}%
\begin{pgfscope}%
\pgfsys@transformshift{2.179808in}{0.713178in}%
\pgfsys@useobject{currentmarker}{}%
\end{pgfscope}%
\begin{pgfscope}%
\pgfsys@transformshift{2.180007in}{0.715660in}%
\pgfsys@useobject{currentmarker}{}%
\end{pgfscope}%
\begin{pgfscope}%
\pgfsys@transformshift{2.180205in}{0.728192in}%
\pgfsys@useobject{currentmarker}{}%
\end{pgfscope}%
\begin{pgfscope}%
\pgfsys@transformshift{2.180402in}{0.695799in}%
\pgfsys@useobject{currentmarker}{}%
\end{pgfscope}%
\begin{pgfscope}%
\pgfsys@transformshift{2.180600in}{0.717860in}%
\pgfsys@useobject{currentmarker}{}%
\end{pgfscope}%
\begin{pgfscope}%
\pgfsys@transformshift{2.180798in}{0.734015in}%
\pgfsys@useobject{currentmarker}{}%
\end{pgfscope}%
\begin{pgfscope}%
\pgfsys@transformshift{2.180995in}{0.731213in}%
\pgfsys@useobject{currentmarker}{}%
\end{pgfscope}%
\begin{pgfscope}%
\pgfsys@transformshift{2.181192in}{0.710442in}%
\pgfsys@useobject{currentmarker}{}%
\end{pgfscope}%
\begin{pgfscope}%
\pgfsys@transformshift{2.181389in}{0.692507in}%
\pgfsys@useobject{currentmarker}{}%
\end{pgfscope}%
\begin{pgfscope}%
\pgfsys@transformshift{2.181586in}{0.695097in}%
\pgfsys@useobject{currentmarker}{}%
\end{pgfscope}%
\begin{pgfscope}%
\pgfsys@transformshift{2.181782in}{0.727766in}%
\pgfsys@useobject{currentmarker}{}%
\end{pgfscope}%
\begin{pgfscope}%
\pgfsys@transformshift{2.181979in}{0.672138in}%
\pgfsys@useobject{currentmarker}{}%
\end{pgfscope}%
\begin{pgfscope}%
\pgfsys@transformshift{2.182175in}{0.674989in}%
\pgfsys@useobject{currentmarker}{}%
\end{pgfscope}%
\begin{pgfscope}%
\pgfsys@transformshift{2.182371in}{0.742624in}%
\pgfsys@useobject{currentmarker}{}%
\end{pgfscope}%
\begin{pgfscope}%
\pgfsys@transformshift{2.182567in}{0.755067in}%
\pgfsys@useobject{currentmarker}{}%
\end{pgfscope}%
\begin{pgfscope}%
\pgfsys@transformshift{2.182763in}{0.717957in}%
\pgfsys@useobject{currentmarker}{}%
\end{pgfscope}%
\begin{pgfscope}%
\pgfsys@transformshift{2.182959in}{0.711630in}%
\pgfsys@useobject{currentmarker}{}%
\end{pgfscope}%
\begin{pgfscope}%
\pgfsys@transformshift{2.183154in}{0.715877in}%
\pgfsys@useobject{currentmarker}{}%
\end{pgfscope}%
\begin{pgfscope}%
\pgfsys@transformshift{2.183349in}{0.689947in}%
\pgfsys@useobject{currentmarker}{}%
\end{pgfscope}%
\begin{pgfscope}%
\pgfsys@transformshift{2.183544in}{0.668804in}%
\pgfsys@useobject{currentmarker}{}%
\end{pgfscope}%
\begin{pgfscope}%
\pgfsys@transformshift{2.183739in}{0.661677in}%
\pgfsys@useobject{currentmarker}{}%
\end{pgfscope}%
\begin{pgfscope}%
\pgfsys@transformshift{2.183934in}{0.680868in}%
\pgfsys@useobject{currentmarker}{}%
\end{pgfscope}%
\begin{pgfscope}%
\pgfsys@transformshift{2.184129in}{0.714945in}%
\pgfsys@useobject{currentmarker}{}%
\end{pgfscope}%
\begin{pgfscope}%
\pgfsys@transformshift{2.184323in}{0.715940in}%
\pgfsys@useobject{currentmarker}{}%
\end{pgfscope}%
\begin{pgfscope}%
\pgfsys@transformshift{2.184517in}{0.727992in}%
\pgfsys@useobject{currentmarker}{}%
\end{pgfscope}%
\begin{pgfscope}%
\pgfsys@transformshift{2.184711in}{0.730353in}%
\pgfsys@useobject{currentmarker}{}%
\end{pgfscope}%
\begin{pgfscope}%
\pgfsys@transformshift{2.184905in}{0.739834in}%
\pgfsys@useobject{currentmarker}{}%
\end{pgfscope}%
\begin{pgfscope}%
\pgfsys@transformshift{2.185099in}{0.713620in}%
\pgfsys@useobject{currentmarker}{}%
\end{pgfscope}%
\begin{pgfscope}%
\pgfsys@transformshift{2.185293in}{0.685257in}%
\pgfsys@useobject{currentmarker}{}%
\end{pgfscope}%
\begin{pgfscope}%
\pgfsys@transformshift{2.185486in}{0.681571in}%
\pgfsys@useobject{currentmarker}{}%
\end{pgfscope}%
\begin{pgfscope}%
\pgfsys@transformshift{2.185679in}{0.649910in}%
\pgfsys@useobject{currentmarker}{}%
\end{pgfscope}%
\begin{pgfscope}%
\pgfsys@transformshift{2.185872in}{0.652157in}%
\pgfsys@useobject{currentmarker}{}%
\end{pgfscope}%
\begin{pgfscope}%
\pgfsys@transformshift{2.186065in}{0.687677in}%
\pgfsys@useobject{currentmarker}{}%
\end{pgfscope}%
\begin{pgfscope}%
\pgfsys@transformshift{2.186258in}{0.717005in}%
\pgfsys@useobject{currentmarker}{}%
\end{pgfscope}%
\begin{pgfscope}%
\pgfsys@transformshift{2.186451in}{0.718910in}%
\pgfsys@useobject{currentmarker}{}%
\end{pgfscope}%
\begin{pgfscope}%
\pgfsys@transformshift{2.186643in}{0.732227in}%
\pgfsys@useobject{currentmarker}{}%
\end{pgfscope}%
\begin{pgfscope}%
\pgfsys@transformshift{2.186835in}{0.755536in}%
\pgfsys@useobject{currentmarker}{}%
\end{pgfscope}%
\begin{pgfscope}%
\pgfsys@transformshift{2.187028in}{0.753894in}%
\pgfsys@useobject{currentmarker}{}%
\end{pgfscope}%
\begin{pgfscope}%
\pgfsys@transformshift{2.187219in}{0.772044in}%
\pgfsys@useobject{currentmarker}{}%
\end{pgfscope}%
\begin{pgfscope}%
\pgfsys@transformshift{2.187411in}{0.751588in}%
\pgfsys@useobject{currentmarker}{}%
\end{pgfscope}%
\begin{pgfscope}%
\pgfsys@transformshift{2.187603in}{0.707936in}%
\pgfsys@useobject{currentmarker}{}%
\end{pgfscope}%
\begin{pgfscope}%
\pgfsys@transformshift{2.187794in}{0.736030in}%
\pgfsys@useobject{currentmarker}{}%
\end{pgfscope}%
\begin{pgfscope}%
\pgfsys@transformshift{2.187986in}{0.728339in}%
\pgfsys@useobject{currentmarker}{}%
\end{pgfscope}%
\begin{pgfscope}%
\pgfsys@transformshift{2.188177in}{0.699380in}%
\pgfsys@useobject{currentmarker}{}%
\end{pgfscope}%
\begin{pgfscope}%
\pgfsys@transformshift{2.188368in}{0.666017in}%
\pgfsys@useobject{currentmarker}{}%
\end{pgfscope}%
\begin{pgfscope}%
\pgfsys@transformshift{2.188558in}{0.680523in}%
\pgfsys@useobject{currentmarker}{}%
\end{pgfscope}%
\begin{pgfscope}%
\pgfsys@transformshift{2.188749in}{0.722615in}%
\pgfsys@useobject{currentmarker}{}%
\end{pgfscope}%
\begin{pgfscope}%
\pgfsys@transformshift{2.188939in}{0.705168in}%
\pgfsys@useobject{currentmarker}{}%
\end{pgfscope}%
\begin{pgfscope}%
\pgfsys@transformshift{2.189130in}{0.689519in}%
\pgfsys@useobject{currentmarker}{}%
\end{pgfscope}%
\begin{pgfscope}%
\pgfsys@transformshift{2.189320in}{0.688651in}%
\pgfsys@useobject{currentmarker}{}%
\end{pgfscope}%
\begin{pgfscope}%
\pgfsys@transformshift{2.189510in}{0.710229in}%
\pgfsys@useobject{currentmarker}{}%
\end{pgfscope}%
\begin{pgfscope}%
\pgfsys@transformshift{2.189700in}{0.716728in}%
\pgfsys@useobject{currentmarker}{}%
\end{pgfscope}%
\begin{pgfscope}%
\pgfsys@transformshift{2.189889in}{0.727247in}%
\pgfsys@useobject{currentmarker}{}%
\end{pgfscope}%
\begin{pgfscope}%
\pgfsys@transformshift{2.190079in}{0.704509in}%
\pgfsys@useobject{currentmarker}{}%
\end{pgfscope}%
\begin{pgfscope}%
\pgfsys@transformshift{2.190268in}{0.704351in}%
\pgfsys@useobject{currentmarker}{}%
\end{pgfscope}%
\begin{pgfscope}%
\pgfsys@transformshift{2.190457in}{0.662984in}%
\pgfsys@useobject{currentmarker}{}%
\end{pgfscope}%
\begin{pgfscope}%
\pgfsys@transformshift{2.190646in}{0.592196in}%
\pgfsys@useobject{currentmarker}{}%
\end{pgfscope}%
\begin{pgfscope}%
\pgfsys@transformshift{2.190835in}{0.662557in}%
\pgfsys@useobject{currentmarker}{}%
\end{pgfscope}%
\begin{pgfscope}%
\pgfsys@transformshift{2.191024in}{0.708106in}%
\pgfsys@useobject{currentmarker}{}%
\end{pgfscope}%
\begin{pgfscope}%
\pgfsys@transformshift{2.191213in}{0.702896in}%
\pgfsys@useobject{currentmarker}{}%
\end{pgfscope}%
\begin{pgfscope}%
\pgfsys@transformshift{2.191401in}{0.683284in}%
\pgfsys@useobject{currentmarker}{}%
\end{pgfscope}%
\begin{pgfscope}%
\pgfsys@transformshift{2.191589in}{0.655938in}%
\pgfsys@useobject{currentmarker}{}%
\end{pgfscope}%
\begin{pgfscope}%
\pgfsys@transformshift{2.191777in}{0.678317in}%
\pgfsys@useobject{currentmarker}{}%
\end{pgfscope}%
\begin{pgfscope}%
\pgfsys@transformshift{2.191965in}{0.715542in}%
\pgfsys@useobject{currentmarker}{}%
\end{pgfscope}%
\begin{pgfscope}%
\pgfsys@transformshift{2.192153in}{0.680849in}%
\pgfsys@useobject{currentmarker}{}%
\end{pgfscope}%
\begin{pgfscope}%
\pgfsys@transformshift{2.192340in}{0.724524in}%
\pgfsys@useobject{currentmarker}{}%
\end{pgfscope}%
\begin{pgfscope}%
\pgfsys@transformshift{2.192528in}{0.754026in}%
\pgfsys@useobject{currentmarker}{}%
\end{pgfscope}%
\begin{pgfscope}%
\pgfsys@transformshift{2.192715in}{0.712912in}%
\pgfsys@useobject{currentmarker}{}%
\end{pgfscope}%
\begin{pgfscope}%
\pgfsys@transformshift{2.192902in}{0.701435in}%
\pgfsys@useobject{currentmarker}{}%
\end{pgfscope}%
\begin{pgfscope}%
\pgfsys@transformshift{2.193089in}{0.722462in}%
\pgfsys@useobject{currentmarker}{}%
\end{pgfscope}%
\begin{pgfscope}%
\pgfsys@transformshift{2.193276in}{0.723242in}%
\pgfsys@useobject{currentmarker}{}%
\end{pgfscope}%
\begin{pgfscope}%
\pgfsys@transformshift{2.193463in}{0.695910in}%
\pgfsys@useobject{currentmarker}{}%
\end{pgfscope}%
\begin{pgfscope}%
\pgfsys@transformshift{2.193649in}{0.704246in}%
\pgfsys@useobject{currentmarker}{}%
\end{pgfscope}%
\begin{pgfscope}%
\pgfsys@transformshift{2.193836in}{0.719939in}%
\pgfsys@useobject{currentmarker}{}%
\end{pgfscope}%
\begin{pgfscope}%
\pgfsys@transformshift{2.194022in}{0.701344in}%
\pgfsys@useobject{currentmarker}{}%
\end{pgfscope}%
\begin{pgfscope}%
\pgfsys@transformshift{2.194208in}{0.698389in}%
\pgfsys@useobject{currentmarker}{}%
\end{pgfscope}%
\begin{pgfscope}%
\pgfsys@transformshift{2.194394in}{0.700112in}%
\pgfsys@useobject{currentmarker}{}%
\end{pgfscope}%
\begin{pgfscope}%
\pgfsys@transformshift{2.194580in}{0.717927in}%
\pgfsys@useobject{currentmarker}{}%
\end{pgfscope}%
\begin{pgfscope}%
\pgfsys@transformshift{2.194765in}{0.760663in}%
\pgfsys@useobject{currentmarker}{}%
\end{pgfscope}%
\begin{pgfscope}%
\pgfsys@transformshift{2.194951in}{0.742220in}%
\pgfsys@useobject{currentmarker}{}%
\end{pgfscope}%
\begin{pgfscope}%
\pgfsys@transformshift{2.195136in}{0.712015in}%
\pgfsys@useobject{currentmarker}{}%
\end{pgfscope}%
\begin{pgfscope}%
\pgfsys@transformshift{2.195321in}{0.719726in}%
\pgfsys@useobject{currentmarker}{}%
\end{pgfscope}%
\begin{pgfscope}%
\pgfsys@transformshift{2.195506in}{0.718760in}%
\pgfsys@useobject{currentmarker}{}%
\end{pgfscope}%
\begin{pgfscope}%
\pgfsys@transformshift{2.195691in}{0.631928in}%
\pgfsys@useobject{currentmarker}{}%
\end{pgfscope}%
\begin{pgfscope}%
\pgfsys@transformshift{2.195875in}{0.687643in}%
\pgfsys@useobject{currentmarker}{}%
\end{pgfscope}%
\begin{pgfscope}%
\pgfsys@transformshift{2.196060in}{0.693508in}%
\pgfsys@useobject{currentmarker}{}%
\end{pgfscope}%
\begin{pgfscope}%
\pgfsys@transformshift{2.196244in}{0.722062in}%
\pgfsys@useobject{currentmarker}{}%
\end{pgfscope}%
\begin{pgfscope}%
\pgfsys@transformshift{2.196429in}{0.704035in}%
\pgfsys@useobject{currentmarker}{}%
\end{pgfscope}%
\begin{pgfscope}%
\pgfsys@transformshift{2.196613in}{0.699288in}%
\pgfsys@useobject{currentmarker}{}%
\end{pgfscope}%
\begin{pgfscope}%
\pgfsys@transformshift{2.196797in}{0.711622in}%
\pgfsys@useobject{currentmarker}{}%
\end{pgfscope}%
\begin{pgfscope}%
\pgfsys@transformshift{2.196980in}{0.692619in}%
\pgfsys@useobject{currentmarker}{}%
\end{pgfscope}%
\begin{pgfscope}%
\pgfsys@transformshift{2.197164in}{0.760304in}%
\pgfsys@useobject{currentmarker}{}%
\end{pgfscope}%
\begin{pgfscope}%
\pgfsys@transformshift{2.197347in}{0.746247in}%
\pgfsys@useobject{currentmarker}{}%
\end{pgfscope}%
\begin{pgfscope}%
\pgfsys@transformshift{2.197531in}{0.729386in}%
\pgfsys@useobject{currentmarker}{}%
\end{pgfscope}%
\begin{pgfscope}%
\pgfsys@transformshift{2.197714in}{0.722816in}%
\pgfsys@useobject{currentmarker}{}%
\end{pgfscope}%
\begin{pgfscope}%
\pgfsys@transformshift{2.197897in}{0.701341in}%
\pgfsys@useobject{currentmarker}{}%
\end{pgfscope}%
\begin{pgfscope}%
\pgfsys@transformshift{2.198080in}{0.669729in}%
\pgfsys@useobject{currentmarker}{}%
\end{pgfscope}%
\begin{pgfscope}%
\pgfsys@transformshift{2.198262in}{0.706661in}%
\pgfsys@useobject{currentmarker}{}%
\end{pgfscope}%
\begin{pgfscope}%
\pgfsys@transformshift{2.198445in}{0.715244in}%
\pgfsys@useobject{currentmarker}{}%
\end{pgfscope}%
\begin{pgfscope}%
\pgfsys@transformshift{2.198627in}{0.706333in}%
\pgfsys@useobject{currentmarker}{}%
\end{pgfscope}%
\begin{pgfscope}%
\pgfsys@transformshift{2.198810in}{0.690146in}%
\pgfsys@useobject{currentmarker}{}%
\end{pgfscope}%
\begin{pgfscope}%
\pgfsys@transformshift{2.198992in}{0.690807in}%
\pgfsys@useobject{currentmarker}{}%
\end{pgfscope}%
\begin{pgfscope}%
\pgfsys@transformshift{2.199174in}{0.700704in}%
\pgfsys@useobject{currentmarker}{}%
\end{pgfscope}%
\begin{pgfscope}%
\pgfsys@transformshift{2.199356in}{0.699940in}%
\pgfsys@useobject{currentmarker}{}%
\end{pgfscope}%
\begin{pgfscope}%
\pgfsys@transformshift{2.199537in}{0.709103in}%
\pgfsys@useobject{currentmarker}{}%
\end{pgfscope}%
\begin{pgfscope}%
\pgfsys@transformshift{2.199719in}{0.701384in}%
\pgfsys@useobject{currentmarker}{}%
\end{pgfscope}%
\begin{pgfscope}%
\pgfsys@transformshift{2.199900in}{0.684330in}%
\pgfsys@useobject{currentmarker}{}%
\end{pgfscope}%
\begin{pgfscope}%
\pgfsys@transformshift{2.200081in}{0.686280in}%
\pgfsys@useobject{currentmarker}{}%
\end{pgfscope}%
\begin{pgfscope}%
\pgfsys@transformshift{2.200263in}{0.709757in}%
\pgfsys@useobject{currentmarker}{}%
\end{pgfscope}%
\begin{pgfscope}%
\pgfsys@transformshift{2.200444in}{0.709704in}%
\pgfsys@useobject{currentmarker}{}%
\end{pgfscope}%
\begin{pgfscope}%
\pgfsys@transformshift{2.200624in}{0.698326in}%
\pgfsys@useobject{currentmarker}{}%
\end{pgfscope}%
\begin{pgfscope}%
\pgfsys@transformshift{2.200805in}{0.660877in}%
\pgfsys@useobject{currentmarker}{}%
\end{pgfscope}%
\begin{pgfscope}%
\pgfsys@transformshift{2.200986in}{0.684121in}%
\pgfsys@useobject{currentmarker}{}%
\end{pgfscope}%
\begin{pgfscope}%
\pgfsys@transformshift{2.201166in}{0.711412in}%
\pgfsys@useobject{currentmarker}{}%
\end{pgfscope}%
\begin{pgfscope}%
\pgfsys@transformshift{2.201346in}{0.647488in}%
\pgfsys@useobject{currentmarker}{}%
\end{pgfscope}%
\begin{pgfscope}%
\pgfsys@transformshift{2.201526in}{0.623108in}%
\pgfsys@useobject{currentmarker}{}%
\end{pgfscope}%
\begin{pgfscope}%
\pgfsys@transformshift{2.201706in}{0.636178in}%
\pgfsys@useobject{currentmarker}{}%
\end{pgfscope}%
\begin{pgfscope}%
\pgfsys@transformshift{2.201886in}{0.650636in}%
\pgfsys@useobject{currentmarker}{}%
\end{pgfscope}%
\begin{pgfscope}%
\pgfsys@transformshift{2.202066in}{0.681083in}%
\pgfsys@useobject{currentmarker}{}%
\end{pgfscope}%
\begin{pgfscope}%
\pgfsys@transformshift{2.202245in}{0.646135in}%
\pgfsys@useobject{currentmarker}{}%
\end{pgfscope}%
\begin{pgfscope}%
\pgfsys@transformshift{2.202424in}{0.657928in}%
\pgfsys@useobject{currentmarker}{}%
\end{pgfscope}%
\begin{pgfscope}%
\pgfsys@transformshift{2.202604in}{0.673066in}%
\pgfsys@useobject{currentmarker}{}%
\end{pgfscope}%
\begin{pgfscope}%
\pgfsys@transformshift{2.202783in}{0.690799in}%
\pgfsys@useobject{currentmarker}{}%
\end{pgfscope}%
\begin{pgfscope}%
\pgfsys@transformshift{2.202962in}{0.703526in}%
\pgfsys@useobject{currentmarker}{}%
\end{pgfscope}%
\begin{pgfscope}%
\pgfsys@transformshift{2.203140in}{0.686310in}%
\pgfsys@useobject{currentmarker}{}%
\end{pgfscope}%
\begin{pgfscope}%
\pgfsys@transformshift{2.203319in}{0.658153in}%
\pgfsys@useobject{currentmarker}{}%
\end{pgfscope}%
\begin{pgfscope}%
\pgfsys@transformshift{2.203498in}{0.684317in}%
\pgfsys@useobject{currentmarker}{}%
\end{pgfscope}%
\begin{pgfscope}%
\pgfsys@transformshift{2.203676in}{0.743454in}%
\pgfsys@useobject{currentmarker}{}%
\end{pgfscope}%
\begin{pgfscope}%
\pgfsys@transformshift{2.203854in}{0.738394in}%
\pgfsys@useobject{currentmarker}{}%
\end{pgfscope}%
\begin{pgfscope}%
\pgfsys@transformshift{2.204032in}{0.718487in}%
\pgfsys@useobject{currentmarker}{}%
\end{pgfscope}%
\begin{pgfscope}%
\pgfsys@transformshift{2.204210in}{0.654393in}%
\pgfsys@useobject{currentmarker}{}%
\end{pgfscope}%
\begin{pgfscope}%
\pgfsys@transformshift{2.204388in}{0.665074in}%
\pgfsys@useobject{currentmarker}{}%
\end{pgfscope}%
\begin{pgfscope}%
\pgfsys@transformshift{2.204566in}{0.707318in}%
\pgfsys@useobject{currentmarker}{}%
\end{pgfscope}%
\begin{pgfscope}%
\pgfsys@transformshift{2.204743in}{0.721140in}%
\pgfsys@useobject{currentmarker}{}%
\end{pgfscope}%
\begin{pgfscope}%
\pgfsys@transformshift{2.204921in}{0.629012in}%
\pgfsys@useobject{currentmarker}{}%
\end{pgfscope}%
\begin{pgfscope}%
\pgfsys@transformshift{2.205098in}{0.731811in}%
\pgfsys@useobject{currentmarker}{}%
\end{pgfscope}%
\begin{pgfscope}%
\pgfsys@transformshift{2.205275in}{0.717617in}%
\pgfsys@useobject{currentmarker}{}%
\end{pgfscope}%
\begin{pgfscope}%
\pgfsys@transformshift{2.205452in}{0.698380in}%
\pgfsys@useobject{currentmarker}{}%
\end{pgfscope}%
\begin{pgfscope}%
\pgfsys@transformshift{2.205629in}{0.710271in}%
\pgfsys@useobject{currentmarker}{}%
\end{pgfscope}%
\begin{pgfscope}%
\pgfsys@transformshift{2.205805in}{0.687162in}%
\pgfsys@useobject{currentmarker}{}%
\end{pgfscope}%
\begin{pgfscope}%
\pgfsys@transformshift{2.205982in}{0.665303in}%
\pgfsys@useobject{currentmarker}{}%
\end{pgfscope}%
\begin{pgfscope}%
\pgfsys@transformshift{2.206158in}{0.722979in}%
\pgfsys@useobject{currentmarker}{}%
\end{pgfscope}%
\begin{pgfscope}%
\pgfsys@transformshift{2.206335in}{0.722266in}%
\pgfsys@useobject{currentmarker}{}%
\end{pgfscope}%
\begin{pgfscope}%
\pgfsys@transformshift{2.206511in}{0.682677in}%
\pgfsys@useobject{currentmarker}{}%
\end{pgfscope}%
\begin{pgfscope}%
\pgfsys@transformshift{2.206687in}{0.659216in}%
\pgfsys@useobject{currentmarker}{}%
\end{pgfscope}%
\begin{pgfscope}%
\pgfsys@transformshift{2.206863in}{0.691140in}%
\pgfsys@useobject{currentmarker}{}%
\end{pgfscope}%
\begin{pgfscope}%
\pgfsys@transformshift{2.207038in}{0.690426in}%
\pgfsys@useobject{currentmarker}{}%
\end{pgfscope}%
\begin{pgfscope}%
\pgfsys@transformshift{2.207214in}{0.672113in}%
\pgfsys@useobject{currentmarker}{}%
\end{pgfscope}%
\begin{pgfscope}%
\pgfsys@transformshift{2.207389in}{0.724117in}%
\pgfsys@useobject{currentmarker}{}%
\end{pgfscope}%
\begin{pgfscope}%
\pgfsys@transformshift{2.207565in}{0.727842in}%
\pgfsys@useobject{currentmarker}{}%
\end{pgfscope}%
\begin{pgfscope}%
\pgfsys@transformshift{2.207740in}{0.654429in}%
\pgfsys@useobject{currentmarker}{}%
\end{pgfscope}%
\begin{pgfscope}%
\pgfsys@transformshift{2.207915in}{0.645144in}%
\pgfsys@useobject{currentmarker}{}%
\end{pgfscope}%
\begin{pgfscope}%
\pgfsys@transformshift{2.208090in}{0.645824in}%
\pgfsys@useobject{currentmarker}{}%
\end{pgfscope}%
\begin{pgfscope}%
\pgfsys@transformshift{2.208264in}{0.636517in}%
\pgfsys@useobject{currentmarker}{}%
\end{pgfscope}%
\begin{pgfscope}%
\pgfsys@transformshift{2.208439in}{0.674486in}%
\pgfsys@useobject{currentmarker}{}%
\end{pgfscope}%
\begin{pgfscope}%
\pgfsys@transformshift{2.208614in}{0.614702in}%
\pgfsys@useobject{currentmarker}{}%
\end{pgfscope}%
\begin{pgfscope}%
\pgfsys@transformshift{2.208788in}{0.655269in}%
\pgfsys@useobject{currentmarker}{}%
\end{pgfscope}%
\begin{pgfscope}%
\pgfsys@transformshift{2.208962in}{0.665779in}%
\pgfsys@useobject{currentmarker}{}%
\end{pgfscope}%
\begin{pgfscope}%
\pgfsys@transformshift{2.209136in}{0.690538in}%
\pgfsys@useobject{currentmarker}{}%
\end{pgfscope}%
\begin{pgfscope}%
\pgfsys@transformshift{2.209310in}{0.706981in}%
\pgfsys@useobject{currentmarker}{}%
\end{pgfscope}%
\begin{pgfscope}%
\pgfsys@transformshift{2.209484in}{0.701144in}%
\pgfsys@useobject{currentmarker}{}%
\end{pgfscope}%
\begin{pgfscope}%
\pgfsys@transformshift{2.209658in}{0.699004in}%
\pgfsys@useobject{currentmarker}{}%
\end{pgfscope}%
\begin{pgfscope}%
\pgfsys@transformshift{2.209831in}{0.670171in}%
\pgfsys@useobject{currentmarker}{}%
\end{pgfscope}%
\begin{pgfscope}%
\pgfsys@transformshift{2.210005in}{0.691461in}%
\pgfsys@useobject{currentmarker}{}%
\end{pgfscope}%
\begin{pgfscope}%
\pgfsys@transformshift{2.210178in}{0.664451in}%
\pgfsys@useobject{currentmarker}{}%
\end{pgfscope}%
\begin{pgfscope}%
\pgfsys@transformshift{2.210351in}{0.637785in}%
\pgfsys@useobject{currentmarker}{}%
\end{pgfscope}%
\begin{pgfscope}%
\pgfsys@transformshift{2.210524in}{0.658172in}%
\pgfsys@useobject{currentmarker}{}%
\end{pgfscope}%
\begin{pgfscope}%
\pgfsys@transformshift{2.210697in}{0.662022in}%
\pgfsys@useobject{currentmarker}{}%
\end{pgfscope}%
\begin{pgfscope}%
\pgfsys@transformshift{2.210870in}{0.655558in}%
\pgfsys@useobject{currentmarker}{}%
\end{pgfscope}%
\begin{pgfscope}%
\pgfsys@transformshift{2.211042in}{0.665558in}%
\pgfsys@useobject{currentmarker}{}%
\end{pgfscope}%
\begin{pgfscope}%
\pgfsys@transformshift{2.211215in}{0.703880in}%
\pgfsys@useobject{currentmarker}{}%
\end{pgfscope}%
\begin{pgfscope}%
\pgfsys@transformshift{2.211387in}{0.698403in}%
\pgfsys@useobject{currentmarker}{}%
\end{pgfscope}%
\begin{pgfscope}%
\pgfsys@transformshift{2.211559in}{0.654520in}%
\pgfsys@useobject{currentmarker}{}%
\end{pgfscope}%
\begin{pgfscope}%
\pgfsys@transformshift{2.211731in}{0.651175in}%
\pgfsys@useobject{currentmarker}{}%
\end{pgfscope}%
\begin{pgfscope}%
\pgfsys@transformshift{2.211903in}{0.642618in}%
\pgfsys@useobject{currentmarker}{}%
\end{pgfscope}%
\begin{pgfscope}%
\pgfsys@transformshift{2.212075in}{0.604438in}%
\pgfsys@useobject{currentmarker}{}%
\end{pgfscope}%
\begin{pgfscope}%
\pgfsys@transformshift{2.212247in}{0.676310in}%
\pgfsys@useobject{currentmarker}{}%
\end{pgfscope}%
\begin{pgfscope}%
\pgfsys@transformshift{2.212418in}{0.692149in}%
\pgfsys@useobject{currentmarker}{}%
\end{pgfscope}%
\begin{pgfscope}%
\pgfsys@transformshift{2.212590in}{0.678494in}%
\pgfsys@useobject{currentmarker}{}%
\end{pgfscope}%
\begin{pgfscope}%
\pgfsys@transformshift{2.212761in}{0.663655in}%
\pgfsys@useobject{currentmarker}{}%
\end{pgfscope}%
\begin{pgfscope}%
\pgfsys@transformshift{2.212932in}{0.709323in}%
\pgfsys@useobject{currentmarker}{}%
\end{pgfscope}%
\begin{pgfscope}%
\pgfsys@transformshift{2.213103in}{0.707774in}%
\pgfsys@useobject{currentmarker}{}%
\end{pgfscope}%
\begin{pgfscope}%
\pgfsys@transformshift{2.213274in}{0.685196in}%
\pgfsys@useobject{currentmarker}{}%
\end{pgfscope}%
\begin{pgfscope}%
\pgfsys@transformshift{2.213445in}{0.715490in}%
\pgfsys@useobject{currentmarker}{}%
\end{pgfscope}%
\begin{pgfscope}%
\pgfsys@transformshift{2.213616in}{0.685996in}%
\pgfsys@useobject{currentmarker}{}%
\end{pgfscope}%
\begin{pgfscope}%
\pgfsys@transformshift{2.213786in}{0.690508in}%
\pgfsys@useobject{currentmarker}{}%
\end{pgfscope}%
\begin{pgfscope}%
\pgfsys@transformshift{2.213957in}{0.689029in}%
\pgfsys@useobject{currentmarker}{}%
\end{pgfscope}%
\begin{pgfscope}%
\pgfsys@transformshift{2.214127in}{0.695985in}%
\pgfsys@useobject{currentmarker}{}%
\end{pgfscope}%
\begin{pgfscope}%
\pgfsys@transformshift{2.214297in}{0.651903in}%
\pgfsys@useobject{currentmarker}{}%
\end{pgfscope}%
\begin{pgfscope}%
\pgfsys@transformshift{2.214467in}{0.661715in}%
\pgfsys@useobject{currentmarker}{}%
\end{pgfscope}%
\begin{pgfscope}%
\pgfsys@transformshift{2.214637in}{0.727454in}%
\pgfsys@useobject{currentmarker}{}%
\end{pgfscope}%
\begin{pgfscope}%
\pgfsys@transformshift{2.214807in}{0.703116in}%
\pgfsys@useobject{currentmarker}{}%
\end{pgfscope}%
\begin{pgfscope}%
\pgfsys@transformshift{2.214976in}{0.672621in}%
\pgfsys@useobject{currentmarker}{}%
\end{pgfscope}%
\begin{pgfscope}%
\pgfsys@transformshift{2.215146in}{0.670440in}%
\pgfsys@useobject{currentmarker}{}%
\end{pgfscope}%
\begin{pgfscope}%
\pgfsys@transformshift{2.215315in}{0.688599in}%
\pgfsys@useobject{currentmarker}{}%
\end{pgfscope}%
\begin{pgfscope}%
\pgfsys@transformshift{2.215484in}{0.676720in}%
\pgfsys@useobject{currentmarker}{}%
\end{pgfscope}%
\begin{pgfscope}%
\pgfsys@transformshift{2.215653in}{0.725742in}%
\pgfsys@useobject{currentmarker}{}%
\end{pgfscope}%
\begin{pgfscope}%
\pgfsys@transformshift{2.215822in}{0.729649in}%
\pgfsys@useobject{currentmarker}{}%
\end{pgfscope}%
\begin{pgfscope}%
\pgfsys@transformshift{2.215991in}{0.687232in}%
\pgfsys@useobject{currentmarker}{}%
\end{pgfscope}%
\begin{pgfscope}%
\pgfsys@transformshift{2.216160in}{0.659842in}%
\pgfsys@useobject{currentmarker}{}%
\end{pgfscope}%
\begin{pgfscope}%
\pgfsys@transformshift{2.216328in}{0.694122in}%
\pgfsys@useobject{currentmarker}{}%
\end{pgfscope}%
\begin{pgfscope}%
\pgfsys@transformshift{2.216497in}{0.681836in}%
\pgfsys@useobject{currentmarker}{}%
\end{pgfscope}%
\begin{pgfscope}%
\pgfsys@transformshift{2.216665in}{0.722855in}%
\pgfsys@useobject{currentmarker}{}%
\end{pgfscope}%
\begin{pgfscope}%
\pgfsys@transformshift{2.216833in}{0.713131in}%
\pgfsys@useobject{currentmarker}{}%
\end{pgfscope}%
\begin{pgfscope}%
\pgfsys@transformshift{2.217002in}{0.686684in}%
\pgfsys@useobject{currentmarker}{}%
\end{pgfscope}%
\begin{pgfscope}%
\pgfsys@transformshift{2.217170in}{0.670298in}%
\pgfsys@useobject{currentmarker}{}%
\end{pgfscope}%
\begin{pgfscope}%
\pgfsys@transformshift{2.217337in}{0.691544in}%
\pgfsys@useobject{currentmarker}{}%
\end{pgfscope}%
\begin{pgfscope}%
\pgfsys@transformshift{2.217505in}{0.707442in}%
\pgfsys@useobject{currentmarker}{}%
\end{pgfscope}%
\begin{pgfscope}%
\pgfsys@transformshift{2.217673in}{0.701476in}%
\pgfsys@useobject{currentmarker}{}%
\end{pgfscope}%
\begin{pgfscope}%
\pgfsys@transformshift{2.217840in}{0.742355in}%
\pgfsys@useobject{currentmarker}{}%
\end{pgfscope}%
\begin{pgfscope}%
\pgfsys@transformshift{2.218007in}{0.697654in}%
\pgfsys@useobject{currentmarker}{}%
\end{pgfscope}%
\begin{pgfscope}%
\pgfsys@transformshift{2.218175in}{0.662512in}%
\pgfsys@useobject{currentmarker}{}%
\end{pgfscope}%
\begin{pgfscope}%
\pgfsys@transformshift{2.218342in}{0.647427in}%
\pgfsys@useobject{currentmarker}{}%
\end{pgfscope}%
\begin{pgfscope}%
\pgfsys@transformshift{2.218509in}{0.586686in}%
\pgfsys@useobject{currentmarker}{}%
\end{pgfscope}%
\begin{pgfscope}%
\pgfsys@transformshift{2.218676in}{0.672205in}%
\pgfsys@useobject{currentmarker}{}%
\end{pgfscope}%
\begin{pgfscope}%
\pgfsys@transformshift{2.218842in}{0.658616in}%
\pgfsys@useobject{currentmarker}{}%
\end{pgfscope}%
\begin{pgfscope}%
\pgfsys@transformshift{2.219009in}{0.697068in}%
\pgfsys@useobject{currentmarker}{}%
\end{pgfscope}%
\begin{pgfscope}%
\pgfsys@transformshift{2.219175in}{0.679150in}%
\pgfsys@useobject{currentmarker}{}%
\end{pgfscope}%
\begin{pgfscope}%
\pgfsys@transformshift{2.219342in}{0.718839in}%
\pgfsys@useobject{currentmarker}{}%
\end{pgfscope}%
\begin{pgfscope}%
\pgfsys@transformshift{2.219508in}{0.751922in}%
\pgfsys@useobject{currentmarker}{}%
\end{pgfscope}%
\begin{pgfscope}%
\pgfsys@transformshift{2.219674in}{0.661995in}%
\pgfsys@useobject{currentmarker}{}%
\end{pgfscope}%
\begin{pgfscope}%
\pgfsys@transformshift{2.219840in}{0.678927in}%
\pgfsys@useobject{currentmarker}{}%
\end{pgfscope}%
\begin{pgfscope}%
\pgfsys@transformshift{2.220006in}{0.716956in}%
\pgfsys@useobject{currentmarker}{}%
\end{pgfscope}%
\begin{pgfscope}%
\pgfsys@transformshift{2.220172in}{0.730325in}%
\pgfsys@useobject{currentmarker}{}%
\end{pgfscope}%
\begin{pgfscope}%
\pgfsys@transformshift{2.220337in}{0.698822in}%
\pgfsys@useobject{currentmarker}{}%
\end{pgfscope}%
\begin{pgfscope}%
\pgfsys@transformshift{2.220503in}{0.652438in}%
\pgfsys@useobject{currentmarker}{}%
\end{pgfscope}%
\begin{pgfscope}%
\pgfsys@transformshift{2.220668in}{0.658983in}%
\pgfsys@useobject{currentmarker}{}%
\end{pgfscope}%
\begin{pgfscope}%
\pgfsys@transformshift{2.220833in}{0.662532in}%
\pgfsys@useobject{currentmarker}{}%
\end{pgfscope}%
\begin{pgfscope}%
\pgfsys@transformshift{2.220998in}{0.661507in}%
\pgfsys@useobject{currentmarker}{}%
\end{pgfscope}%
\begin{pgfscope}%
\pgfsys@transformshift{2.221163in}{0.691979in}%
\pgfsys@useobject{currentmarker}{}%
\end{pgfscope}%
\begin{pgfscope}%
\pgfsys@transformshift{2.221328in}{0.684149in}%
\pgfsys@useobject{currentmarker}{}%
\end{pgfscope}%
\begin{pgfscope}%
\pgfsys@transformshift{2.221493in}{0.676064in}%
\pgfsys@useobject{currentmarker}{}%
\end{pgfscope}%
\begin{pgfscope}%
\pgfsys@transformshift{2.221658in}{0.737285in}%
\pgfsys@useobject{currentmarker}{}%
\end{pgfscope}%
\begin{pgfscope}%
\pgfsys@transformshift{2.221822in}{0.695028in}%
\pgfsys@useobject{currentmarker}{}%
\end{pgfscope}%
\begin{pgfscope}%
\pgfsys@transformshift{2.221987in}{0.644811in}%
\pgfsys@useobject{currentmarker}{}%
\end{pgfscope}%
\begin{pgfscope}%
\pgfsys@transformshift{2.222151in}{0.661033in}%
\pgfsys@useobject{currentmarker}{}%
\end{pgfscope}%
\begin{pgfscope}%
\pgfsys@transformshift{2.222315in}{0.656753in}%
\pgfsys@useobject{currentmarker}{}%
\end{pgfscope}%
\begin{pgfscope}%
\pgfsys@transformshift{2.222479in}{0.659268in}%
\pgfsys@useobject{currentmarker}{}%
\end{pgfscope}%
\begin{pgfscope}%
\pgfsys@transformshift{2.222643in}{0.716246in}%
\pgfsys@useobject{currentmarker}{}%
\end{pgfscope}%
\begin{pgfscope}%
\pgfsys@transformshift{2.222807in}{0.726481in}%
\pgfsys@useobject{currentmarker}{}%
\end{pgfscope}%
\begin{pgfscope}%
\pgfsys@transformshift{2.222971in}{0.715042in}%
\pgfsys@useobject{currentmarker}{}%
\end{pgfscope}%
\begin{pgfscope}%
\pgfsys@transformshift{2.223134in}{0.700755in}%
\pgfsys@useobject{currentmarker}{}%
\end{pgfscope}%
\begin{pgfscope}%
\pgfsys@transformshift{2.223298in}{0.723611in}%
\pgfsys@useobject{currentmarker}{}%
\end{pgfscope}%
\begin{pgfscope}%
\pgfsys@transformshift{2.223461in}{0.713462in}%
\pgfsys@useobject{currentmarker}{}%
\end{pgfscope}%
\begin{pgfscope}%
\pgfsys@transformshift{2.223624in}{0.656997in}%
\pgfsys@useobject{currentmarker}{}%
\end{pgfscope}%
\begin{pgfscope}%
\pgfsys@transformshift{2.223787in}{0.685790in}%
\pgfsys@useobject{currentmarker}{}%
\end{pgfscope}%
\begin{pgfscope}%
\pgfsys@transformshift{2.223950in}{0.677426in}%
\pgfsys@useobject{currentmarker}{}%
\end{pgfscope}%
\begin{pgfscope}%
\pgfsys@transformshift{2.224113in}{0.701467in}%
\pgfsys@useobject{currentmarker}{}%
\end{pgfscope}%
\begin{pgfscope}%
\pgfsys@transformshift{2.224276in}{0.737747in}%
\pgfsys@useobject{currentmarker}{}%
\end{pgfscope}%
\begin{pgfscope}%
\pgfsys@transformshift{2.224438in}{0.756439in}%
\pgfsys@useobject{currentmarker}{}%
\end{pgfscope}%
\begin{pgfscope}%
\pgfsys@transformshift{2.224601in}{0.703901in}%
\pgfsys@useobject{currentmarker}{}%
\end{pgfscope}%
\begin{pgfscope}%
\pgfsys@transformshift{2.224763in}{0.681114in}%
\pgfsys@useobject{currentmarker}{}%
\end{pgfscope}%
\begin{pgfscope}%
\pgfsys@transformshift{2.224925in}{0.683200in}%
\pgfsys@useobject{currentmarker}{}%
\end{pgfscope}%
\begin{pgfscope}%
\pgfsys@transformshift{2.225088in}{0.702248in}%
\pgfsys@useobject{currentmarker}{}%
\end{pgfscope}%
\begin{pgfscope}%
\pgfsys@transformshift{2.225250in}{0.700620in}%
\pgfsys@useobject{currentmarker}{}%
\end{pgfscope}%
\begin{pgfscope}%
\pgfsys@transformshift{2.225412in}{0.645747in}%
\pgfsys@useobject{currentmarker}{}%
\end{pgfscope}%
\begin{pgfscope}%
\pgfsys@transformshift{2.225573in}{0.719457in}%
\pgfsys@useobject{currentmarker}{}%
\end{pgfscope}%
\begin{pgfscope}%
\pgfsys@transformshift{2.225735in}{0.720150in}%
\pgfsys@useobject{currentmarker}{}%
\end{pgfscope}%
\begin{pgfscope}%
\pgfsys@transformshift{2.225897in}{0.705578in}%
\pgfsys@useobject{currentmarker}{}%
\end{pgfscope}%
\begin{pgfscope}%
\pgfsys@transformshift{2.226058in}{0.677766in}%
\pgfsys@useobject{currentmarker}{}%
\end{pgfscope}%
\begin{pgfscope}%
\pgfsys@transformshift{2.226219in}{0.658175in}%
\pgfsys@useobject{currentmarker}{}%
\end{pgfscope}%
\begin{pgfscope}%
\pgfsys@transformshift{2.226381in}{0.684897in}%
\pgfsys@useobject{currentmarker}{}%
\end{pgfscope}%
\begin{pgfscope}%
\pgfsys@transformshift{2.226542in}{0.700898in}%
\pgfsys@useobject{currentmarker}{}%
\end{pgfscope}%
\begin{pgfscope}%
\pgfsys@transformshift{2.226703in}{0.620834in}%
\pgfsys@useobject{currentmarker}{}%
\end{pgfscope}%
\begin{pgfscope}%
\pgfsys@transformshift{2.226863in}{0.638387in}%
\pgfsys@useobject{currentmarker}{}%
\end{pgfscope}%
\begin{pgfscope}%
\pgfsys@transformshift{2.227024in}{0.654146in}%
\pgfsys@useobject{currentmarker}{}%
\end{pgfscope}%
\begin{pgfscope}%
\pgfsys@transformshift{2.227185in}{0.670980in}%
\pgfsys@useobject{currentmarker}{}%
\end{pgfscope}%
\begin{pgfscope}%
\pgfsys@transformshift{2.227345in}{0.664999in}%
\pgfsys@useobject{currentmarker}{}%
\end{pgfscope}%
\begin{pgfscope}%
\pgfsys@transformshift{2.227506in}{0.677720in}%
\pgfsys@useobject{currentmarker}{}%
\end{pgfscope}%
\begin{pgfscope}%
\pgfsys@transformshift{2.227666in}{0.683698in}%
\pgfsys@useobject{currentmarker}{}%
\end{pgfscope}%
\begin{pgfscope}%
\pgfsys@transformshift{2.227826in}{0.663858in}%
\pgfsys@useobject{currentmarker}{}%
\end{pgfscope}%
\begin{pgfscope}%
\pgfsys@transformshift{2.227986in}{0.658817in}%
\pgfsys@useobject{currentmarker}{}%
\end{pgfscope}%
\begin{pgfscope}%
\pgfsys@transformshift{2.228146in}{0.649575in}%
\pgfsys@useobject{currentmarker}{}%
\end{pgfscope}%
\begin{pgfscope}%
\pgfsys@transformshift{2.228306in}{0.694644in}%
\pgfsys@useobject{currentmarker}{}%
\end{pgfscope}%
\begin{pgfscope}%
\pgfsys@transformshift{2.228466in}{0.664077in}%
\pgfsys@useobject{currentmarker}{}%
\end{pgfscope}%
\begin{pgfscope}%
\pgfsys@transformshift{2.228625in}{0.692586in}%
\pgfsys@useobject{currentmarker}{}%
\end{pgfscope}%
\begin{pgfscope}%
\pgfsys@transformshift{2.228785in}{0.683773in}%
\pgfsys@useobject{currentmarker}{}%
\end{pgfscope}%
\begin{pgfscope}%
\pgfsys@transformshift{2.228944in}{0.722397in}%
\pgfsys@useobject{currentmarker}{}%
\end{pgfscope}%
\begin{pgfscope}%
\pgfsys@transformshift{2.229104in}{0.737867in}%
\pgfsys@useobject{currentmarker}{}%
\end{pgfscope}%
\begin{pgfscope}%
\pgfsys@transformshift{2.229263in}{0.748168in}%
\pgfsys@useobject{currentmarker}{}%
\end{pgfscope}%
\begin{pgfscope}%
\pgfsys@transformshift{2.229422in}{0.755335in}%
\pgfsys@useobject{currentmarker}{}%
\end{pgfscope}%
\begin{pgfscope}%
\pgfsys@transformshift{2.229581in}{0.745430in}%
\pgfsys@useobject{currentmarker}{}%
\end{pgfscope}%
\begin{pgfscope}%
\pgfsys@transformshift{2.229740in}{0.752728in}%
\pgfsys@useobject{currentmarker}{}%
\end{pgfscope}%
\begin{pgfscope}%
\pgfsys@transformshift{2.229898in}{0.722353in}%
\pgfsys@useobject{currentmarker}{}%
\end{pgfscope}%
\begin{pgfscope}%
\pgfsys@transformshift{2.230057in}{0.676680in}%
\pgfsys@useobject{currentmarker}{}%
\end{pgfscope}%
\begin{pgfscope}%
\pgfsys@transformshift{2.230215in}{0.694917in}%
\pgfsys@useobject{currentmarker}{}%
\end{pgfscope}%
\begin{pgfscope}%
\pgfsys@transformshift{2.230374in}{0.727930in}%
\pgfsys@useobject{currentmarker}{}%
\end{pgfscope}%
\begin{pgfscope}%
\pgfsys@transformshift{2.230532in}{0.660147in}%
\pgfsys@useobject{currentmarker}{}%
\end{pgfscope}%
\begin{pgfscope}%
\pgfsys@transformshift{2.230690in}{0.707979in}%
\pgfsys@useobject{currentmarker}{}%
\end{pgfscope}%
\begin{pgfscope}%
\pgfsys@transformshift{2.230848in}{0.695276in}%
\pgfsys@useobject{currentmarker}{}%
\end{pgfscope}%
\begin{pgfscope}%
\pgfsys@transformshift{2.231006in}{0.649705in}%
\pgfsys@useobject{currentmarker}{}%
\end{pgfscope}%
\begin{pgfscope}%
\pgfsys@transformshift{2.231164in}{0.704161in}%
\pgfsys@useobject{currentmarker}{}%
\end{pgfscope}%
\begin{pgfscope}%
\pgfsys@transformshift{2.231322in}{0.713497in}%
\pgfsys@useobject{currentmarker}{}%
\end{pgfscope}%
\begin{pgfscope}%
\pgfsys@transformshift{2.231479in}{0.672658in}%
\pgfsys@useobject{currentmarker}{}%
\end{pgfscope}%
\begin{pgfscope}%
\pgfsys@transformshift{2.231637in}{0.661265in}%
\pgfsys@useobject{currentmarker}{}%
\end{pgfscope}%
\begin{pgfscope}%
\pgfsys@transformshift{2.231794in}{0.689194in}%
\pgfsys@useobject{currentmarker}{}%
\end{pgfscope}%
\begin{pgfscope}%
\pgfsys@transformshift{2.231951in}{0.689508in}%
\pgfsys@useobject{currentmarker}{}%
\end{pgfscope}%
\begin{pgfscope}%
\pgfsys@transformshift{2.232109in}{0.687662in}%
\pgfsys@useobject{currentmarker}{}%
\end{pgfscope}%
\begin{pgfscope}%
\pgfsys@transformshift{2.232266in}{0.683328in}%
\pgfsys@useobject{currentmarker}{}%
\end{pgfscope}%
\begin{pgfscope}%
\pgfsys@transformshift{2.232423in}{0.661486in}%
\pgfsys@useobject{currentmarker}{}%
\end{pgfscope}%
\begin{pgfscope}%
\pgfsys@transformshift{2.232579in}{0.679501in}%
\pgfsys@useobject{currentmarker}{}%
\end{pgfscope}%
\begin{pgfscope}%
\pgfsys@transformshift{2.232736in}{0.681998in}%
\pgfsys@useobject{currentmarker}{}%
\end{pgfscope}%
\begin{pgfscope}%
\pgfsys@transformshift{2.232893in}{0.711393in}%
\pgfsys@useobject{currentmarker}{}%
\end{pgfscope}%
\begin{pgfscope}%
\pgfsys@transformshift{2.233049in}{0.704705in}%
\pgfsys@useobject{currentmarker}{}%
\end{pgfscope}%
\begin{pgfscope}%
\pgfsys@transformshift{2.233206in}{0.668906in}%
\pgfsys@useobject{currentmarker}{}%
\end{pgfscope}%
\begin{pgfscope}%
\pgfsys@transformshift{2.233362in}{0.699601in}%
\pgfsys@useobject{currentmarker}{}%
\end{pgfscope}%
\begin{pgfscope}%
\pgfsys@transformshift{2.233518in}{0.675201in}%
\pgfsys@useobject{currentmarker}{}%
\end{pgfscope}%
\begin{pgfscope}%
\pgfsys@transformshift{2.233674in}{0.667717in}%
\pgfsys@useobject{currentmarker}{}%
\end{pgfscope}%
\begin{pgfscope}%
\pgfsys@transformshift{2.233830in}{0.657621in}%
\pgfsys@useobject{currentmarker}{}%
\end{pgfscope}%
\begin{pgfscope}%
\pgfsys@transformshift{2.233986in}{0.666936in}%
\pgfsys@useobject{currentmarker}{}%
\end{pgfscope}%
\begin{pgfscope}%
\pgfsys@transformshift{2.234142in}{0.675027in}%
\pgfsys@useobject{currentmarker}{}%
\end{pgfscope}%
\begin{pgfscope}%
\pgfsys@transformshift{2.234297in}{0.717864in}%
\pgfsys@useobject{currentmarker}{}%
\end{pgfscope}%
\begin{pgfscope}%
\pgfsys@transformshift{2.234453in}{0.721882in}%
\pgfsys@useobject{currentmarker}{}%
\end{pgfscope}%
\begin{pgfscope}%
\pgfsys@transformshift{2.234608in}{0.728866in}%
\pgfsys@useobject{currentmarker}{}%
\end{pgfscope}%
\begin{pgfscope}%
\pgfsys@transformshift{2.234763in}{0.706888in}%
\pgfsys@useobject{currentmarker}{}%
\end{pgfscope}%
\begin{pgfscope}%
\pgfsys@transformshift{2.234919in}{0.679311in}%
\pgfsys@useobject{currentmarker}{}%
\end{pgfscope}%
\begin{pgfscope}%
\pgfsys@transformshift{2.235074in}{0.629745in}%
\pgfsys@useobject{currentmarker}{}%
\end{pgfscope}%
\begin{pgfscope}%
\pgfsys@transformshift{2.235229in}{0.732518in}%
\pgfsys@useobject{currentmarker}{}%
\end{pgfscope}%
\begin{pgfscope}%
\pgfsys@transformshift{2.235384in}{0.734973in}%
\pgfsys@useobject{currentmarker}{}%
\end{pgfscope}%
\begin{pgfscope}%
\pgfsys@transformshift{2.235538in}{0.636745in}%
\pgfsys@useobject{currentmarker}{}%
\end{pgfscope}%
\begin{pgfscope}%
\pgfsys@transformshift{2.235693in}{0.643513in}%
\pgfsys@useobject{currentmarker}{}%
\end{pgfscope}%
\begin{pgfscope}%
\pgfsys@transformshift{2.235848in}{0.658157in}%
\pgfsys@useobject{currentmarker}{}%
\end{pgfscope}%
\begin{pgfscope}%
\pgfsys@transformshift{2.236002in}{0.645197in}%
\pgfsys@useobject{currentmarker}{}%
\end{pgfscope}%
\begin{pgfscope}%
\pgfsys@transformshift{2.236156in}{0.705928in}%
\pgfsys@useobject{currentmarker}{}%
\end{pgfscope}%
\begin{pgfscope}%
\pgfsys@transformshift{2.236311in}{0.690141in}%
\pgfsys@useobject{currentmarker}{}%
\end{pgfscope}%
\begin{pgfscope}%
\pgfsys@transformshift{2.236465in}{0.685857in}%
\pgfsys@useobject{currentmarker}{}%
\end{pgfscope}%
\begin{pgfscope}%
\pgfsys@transformshift{2.236619in}{0.639104in}%
\pgfsys@useobject{currentmarker}{}%
\end{pgfscope}%
\begin{pgfscope}%
\pgfsys@transformshift{2.236773in}{0.668784in}%
\pgfsys@useobject{currentmarker}{}%
\end{pgfscope}%
\begin{pgfscope}%
\pgfsys@transformshift{2.236927in}{0.650435in}%
\pgfsys@useobject{currentmarker}{}%
\end{pgfscope}%
\begin{pgfscope}%
\pgfsys@transformshift{2.237080in}{0.660918in}%
\pgfsys@useobject{currentmarker}{}%
\end{pgfscope}%
\begin{pgfscope}%
\pgfsys@transformshift{2.237234in}{0.651859in}%
\pgfsys@useobject{currentmarker}{}%
\end{pgfscope}%
\begin{pgfscope}%
\pgfsys@transformshift{2.237387in}{0.670603in}%
\pgfsys@useobject{currentmarker}{}%
\end{pgfscope}%
\begin{pgfscope}%
\pgfsys@transformshift{2.237541in}{0.705735in}%
\pgfsys@useobject{currentmarker}{}%
\end{pgfscope}%
\begin{pgfscope}%
\pgfsys@transformshift{2.237694in}{0.675600in}%
\pgfsys@useobject{currentmarker}{}%
\end{pgfscope}%
\begin{pgfscope}%
\pgfsys@transformshift{2.237847in}{0.666518in}%
\pgfsys@useobject{currentmarker}{}%
\end{pgfscope}%
\begin{pgfscope}%
\pgfsys@transformshift{2.238000in}{0.697076in}%
\pgfsys@useobject{currentmarker}{}%
\end{pgfscope}%
\begin{pgfscope}%
\pgfsys@transformshift{2.238153in}{0.654820in}%
\pgfsys@useobject{currentmarker}{}%
\end{pgfscope}%
\begin{pgfscope}%
\pgfsys@transformshift{2.238306in}{0.653475in}%
\pgfsys@useobject{currentmarker}{}%
\end{pgfscope}%
\begin{pgfscope}%
\pgfsys@transformshift{2.238459in}{0.700927in}%
\pgfsys@useobject{currentmarker}{}%
\end{pgfscope}%
\begin{pgfscope}%
\pgfsys@transformshift{2.238612in}{0.676631in}%
\pgfsys@useobject{currentmarker}{}%
\end{pgfscope}%
\begin{pgfscope}%
\pgfsys@transformshift{2.238764in}{0.698705in}%
\pgfsys@useobject{currentmarker}{}%
\end{pgfscope}%
\begin{pgfscope}%
\pgfsys@transformshift{2.238917in}{0.716267in}%
\pgfsys@useobject{currentmarker}{}%
\end{pgfscope}%
\begin{pgfscope}%
\pgfsys@transformshift{2.239069in}{0.735411in}%
\pgfsys@useobject{currentmarker}{}%
\end{pgfscope}%
\begin{pgfscope}%
\pgfsys@transformshift{2.239221in}{0.689838in}%
\pgfsys@useobject{currentmarker}{}%
\end{pgfscope}%
\begin{pgfscope}%
\pgfsys@transformshift{2.239373in}{0.697758in}%
\pgfsys@useobject{currentmarker}{}%
\end{pgfscope}%
\begin{pgfscope}%
\pgfsys@transformshift{2.239525in}{0.713092in}%
\pgfsys@useobject{currentmarker}{}%
\end{pgfscope}%
\begin{pgfscope}%
\pgfsys@transformshift{2.239677in}{0.742158in}%
\pgfsys@useobject{currentmarker}{}%
\end{pgfscope}%
\begin{pgfscope}%
\pgfsys@transformshift{2.239829in}{0.725152in}%
\pgfsys@useobject{currentmarker}{}%
\end{pgfscope}%
\begin{pgfscope}%
\pgfsys@transformshift{2.239981in}{0.653973in}%
\pgfsys@useobject{currentmarker}{}%
\end{pgfscope}%
\begin{pgfscope}%
\pgfsys@transformshift{2.240133in}{0.638739in}%
\pgfsys@useobject{currentmarker}{}%
\end{pgfscope}%
\begin{pgfscope}%
\pgfsys@transformshift{2.240284in}{0.643816in}%
\pgfsys@useobject{currentmarker}{}%
\end{pgfscope}%
\begin{pgfscope}%
\pgfsys@transformshift{2.240436in}{0.598254in}%
\pgfsys@useobject{currentmarker}{}%
\end{pgfscope}%
\begin{pgfscope}%
\pgfsys@transformshift{2.240587in}{0.653877in}%
\pgfsys@useobject{currentmarker}{}%
\end{pgfscope}%
\begin{pgfscope}%
\pgfsys@transformshift{2.240738in}{0.641419in}%
\pgfsys@useobject{currentmarker}{}%
\end{pgfscope}%
\begin{pgfscope}%
\pgfsys@transformshift{2.240889in}{0.698568in}%
\pgfsys@useobject{currentmarker}{}%
\end{pgfscope}%
\begin{pgfscope}%
\pgfsys@transformshift{2.241040in}{0.689027in}%
\pgfsys@useobject{currentmarker}{}%
\end{pgfscope}%
\begin{pgfscope}%
\pgfsys@transformshift{2.241191in}{0.717154in}%
\pgfsys@useobject{currentmarker}{}%
\end{pgfscope}%
\begin{pgfscope}%
\pgfsys@transformshift{2.241342in}{0.742262in}%
\pgfsys@useobject{currentmarker}{}%
\end{pgfscope}%
\begin{pgfscope}%
\pgfsys@transformshift{2.241493in}{0.724503in}%
\pgfsys@useobject{currentmarker}{}%
\end{pgfscope}%
\begin{pgfscope}%
\pgfsys@transformshift{2.241643in}{0.702501in}%
\pgfsys@useobject{currentmarker}{}%
\end{pgfscope}%
\begin{pgfscope}%
\pgfsys@transformshift{2.241794in}{0.706021in}%
\pgfsys@useobject{currentmarker}{}%
\end{pgfscope}%
\begin{pgfscope}%
\pgfsys@transformshift{2.241944in}{0.685025in}%
\pgfsys@useobject{currentmarker}{}%
\end{pgfscope}%
\begin{pgfscope}%
\pgfsys@transformshift{2.242095in}{0.645130in}%
\pgfsys@useobject{currentmarker}{}%
\end{pgfscope}%
\begin{pgfscope}%
\pgfsys@transformshift{2.242245in}{0.637856in}%
\pgfsys@useobject{currentmarker}{}%
\end{pgfscope}%
\begin{pgfscope}%
\pgfsys@transformshift{2.242395in}{0.683687in}%
\pgfsys@useobject{currentmarker}{}%
\end{pgfscope}%
\begin{pgfscope}%
\pgfsys@transformshift{2.242545in}{0.667367in}%
\pgfsys@useobject{currentmarker}{}%
\end{pgfscope}%
\begin{pgfscope}%
\pgfsys@transformshift{2.242695in}{0.631481in}%
\pgfsys@useobject{currentmarker}{}%
\end{pgfscope}%
\begin{pgfscope}%
\pgfsys@transformshift{2.242845in}{0.653579in}%
\pgfsys@useobject{currentmarker}{}%
\end{pgfscope}%
\begin{pgfscope}%
\pgfsys@transformshift{2.242994in}{0.639135in}%
\pgfsys@useobject{currentmarker}{}%
\end{pgfscope}%
\begin{pgfscope}%
\pgfsys@transformshift{2.243144in}{0.647941in}%
\pgfsys@useobject{currentmarker}{}%
\end{pgfscope}%
\begin{pgfscope}%
\pgfsys@transformshift{2.243294in}{0.671264in}%
\pgfsys@useobject{currentmarker}{}%
\end{pgfscope}%
\begin{pgfscope}%
\pgfsys@transformshift{2.243443in}{0.715717in}%
\pgfsys@useobject{currentmarker}{}%
\end{pgfscope}%
\begin{pgfscope}%
\pgfsys@transformshift{2.243592in}{0.693670in}%
\pgfsys@useobject{currentmarker}{}%
\end{pgfscope}%
\begin{pgfscope}%
\pgfsys@transformshift{2.243742in}{0.694179in}%
\pgfsys@useobject{currentmarker}{}%
\end{pgfscope}%
\begin{pgfscope}%
\pgfsys@transformshift{2.243891in}{0.704584in}%
\pgfsys@useobject{currentmarker}{}%
\end{pgfscope}%
\begin{pgfscope}%
\pgfsys@transformshift{2.244040in}{0.692247in}%
\pgfsys@useobject{currentmarker}{}%
\end{pgfscope}%
\begin{pgfscope}%
\pgfsys@transformshift{2.244189in}{0.659932in}%
\pgfsys@useobject{currentmarker}{}%
\end{pgfscope}%
\begin{pgfscope}%
\pgfsys@transformshift{2.244337in}{0.646508in}%
\pgfsys@useobject{currentmarker}{}%
\end{pgfscope}%
\begin{pgfscope}%
\pgfsys@transformshift{2.244486in}{0.688065in}%
\pgfsys@useobject{currentmarker}{}%
\end{pgfscope}%
\begin{pgfscope}%
\pgfsys@transformshift{2.244635in}{0.667971in}%
\pgfsys@useobject{currentmarker}{}%
\end{pgfscope}%
\begin{pgfscope}%
\pgfsys@transformshift{2.244783in}{0.658482in}%
\pgfsys@useobject{currentmarker}{}%
\end{pgfscope}%
\begin{pgfscope}%
\pgfsys@transformshift{2.244932in}{0.640259in}%
\pgfsys@useobject{currentmarker}{}%
\end{pgfscope}%
\begin{pgfscope}%
\pgfsys@transformshift{2.245080in}{0.616577in}%
\pgfsys@useobject{currentmarker}{}%
\end{pgfscope}%
\begin{pgfscope}%
\pgfsys@transformshift{2.245228in}{0.659901in}%
\pgfsys@useobject{currentmarker}{}%
\end{pgfscope}%
\begin{pgfscope}%
\pgfsys@transformshift{2.245377in}{0.670329in}%
\pgfsys@useobject{currentmarker}{}%
\end{pgfscope}%
\begin{pgfscope}%
\pgfsys@transformshift{2.245525in}{0.711482in}%
\pgfsys@useobject{currentmarker}{}%
\end{pgfscope}%
\begin{pgfscope}%
\pgfsys@transformshift{2.245672in}{0.734372in}%
\pgfsys@useobject{currentmarker}{}%
\end{pgfscope}%
\begin{pgfscope}%
\pgfsys@transformshift{2.245820in}{0.687755in}%
\pgfsys@useobject{currentmarker}{}%
\end{pgfscope}%
\begin{pgfscope}%
\pgfsys@transformshift{2.245968in}{0.617986in}%
\pgfsys@useobject{currentmarker}{}%
\end{pgfscope}%
\begin{pgfscope}%
\pgfsys@transformshift{2.246116in}{0.632775in}%
\pgfsys@useobject{currentmarker}{}%
\end{pgfscope}%
\begin{pgfscope}%
\pgfsys@transformshift{2.246263in}{0.677654in}%
\pgfsys@useobject{currentmarker}{}%
\end{pgfscope}%
\begin{pgfscope}%
\pgfsys@transformshift{2.246411in}{0.668243in}%
\pgfsys@useobject{currentmarker}{}%
\end{pgfscope}%
\begin{pgfscope}%
\pgfsys@transformshift{2.246558in}{0.698607in}%
\pgfsys@useobject{currentmarker}{}%
\end{pgfscope}%
\begin{pgfscope}%
\pgfsys@transformshift{2.246705in}{0.670122in}%
\pgfsys@useobject{currentmarker}{}%
\end{pgfscope}%
\begin{pgfscope}%
\pgfsys@transformshift{2.246853in}{0.679737in}%
\pgfsys@useobject{currentmarker}{}%
\end{pgfscope}%
\begin{pgfscope}%
\pgfsys@transformshift{2.247000in}{0.724229in}%
\pgfsys@useobject{currentmarker}{}%
\end{pgfscope}%
\begin{pgfscope}%
\pgfsys@transformshift{2.247147in}{0.698804in}%
\pgfsys@useobject{currentmarker}{}%
\end{pgfscope}%
\begin{pgfscope}%
\pgfsys@transformshift{2.247293in}{0.651864in}%
\pgfsys@useobject{currentmarker}{}%
\end{pgfscope}%
\begin{pgfscope}%
\pgfsys@transformshift{2.247440in}{0.645893in}%
\pgfsys@useobject{currentmarker}{}%
\end{pgfscope}%
\begin{pgfscope}%
\pgfsys@transformshift{2.247587in}{0.656626in}%
\pgfsys@useobject{currentmarker}{}%
\end{pgfscope}%
\begin{pgfscope}%
\pgfsys@transformshift{2.247734in}{0.680341in}%
\pgfsys@useobject{currentmarker}{}%
\end{pgfscope}%
\begin{pgfscope}%
\pgfsys@transformshift{2.247880in}{0.687878in}%
\pgfsys@useobject{currentmarker}{}%
\end{pgfscope}%
\begin{pgfscope}%
\pgfsys@transformshift{2.248026in}{0.699692in}%
\pgfsys@useobject{currentmarker}{}%
\end{pgfscope}%
\begin{pgfscope}%
\pgfsys@transformshift{2.248173in}{0.638129in}%
\pgfsys@useobject{currentmarker}{}%
\end{pgfscope}%
\begin{pgfscope}%
\pgfsys@transformshift{2.248319in}{0.656535in}%
\pgfsys@useobject{currentmarker}{}%
\end{pgfscope}%
\begin{pgfscope}%
\pgfsys@transformshift{2.248465in}{0.654489in}%
\pgfsys@useobject{currentmarker}{}%
\end{pgfscope}%
\begin{pgfscope}%
\pgfsys@transformshift{2.248611in}{0.634004in}%
\pgfsys@useobject{currentmarker}{}%
\end{pgfscope}%
\begin{pgfscope}%
\pgfsys@transformshift{2.248757in}{0.651584in}%
\pgfsys@useobject{currentmarker}{}%
\end{pgfscope}%
\begin{pgfscope}%
\pgfsys@transformshift{2.248903in}{0.671580in}%
\pgfsys@useobject{currentmarker}{}%
\end{pgfscope}%
\begin{pgfscope}%
\pgfsys@transformshift{2.249049in}{0.691352in}%
\pgfsys@useobject{currentmarker}{}%
\end{pgfscope}%
\begin{pgfscope}%
\pgfsys@transformshift{2.249194in}{0.689325in}%
\pgfsys@useobject{currentmarker}{}%
\end{pgfscope}%
\begin{pgfscope}%
\pgfsys@transformshift{2.249340in}{0.700260in}%
\pgfsys@useobject{currentmarker}{}%
\end{pgfscope}%
\begin{pgfscope}%
\pgfsys@transformshift{2.249485in}{0.686621in}%
\pgfsys@useobject{currentmarker}{}%
\end{pgfscope}%
\begin{pgfscope}%
\pgfsys@transformshift{2.249631in}{0.652202in}%
\pgfsys@useobject{currentmarker}{}%
\end{pgfscope}%
\begin{pgfscope}%
\pgfsys@transformshift{2.249776in}{0.680983in}%
\pgfsys@useobject{currentmarker}{}%
\end{pgfscope}%
\begin{pgfscope}%
\pgfsys@transformshift{2.249921in}{0.690999in}%
\pgfsys@useobject{currentmarker}{}%
\end{pgfscope}%
\begin{pgfscope}%
\pgfsys@transformshift{2.250066in}{0.597394in}%
\pgfsys@useobject{currentmarker}{}%
\end{pgfscope}%
\begin{pgfscope}%
\pgfsys@transformshift{2.250211in}{0.615952in}%
\pgfsys@useobject{currentmarker}{}%
\end{pgfscope}%
\begin{pgfscope}%
\pgfsys@transformshift{2.250356in}{0.637657in}%
\pgfsys@useobject{currentmarker}{}%
\end{pgfscope}%
\begin{pgfscope}%
\pgfsys@transformshift{2.250501in}{0.709751in}%
\pgfsys@useobject{currentmarker}{}%
\end{pgfscope}%
\begin{pgfscope}%
\pgfsys@transformshift{2.250645in}{0.710290in}%
\pgfsys@useobject{currentmarker}{}%
\end{pgfscope}%
\begin{pgfscope}%
\pgfsys@transformshift{2.250790in}{0.683359in}%
\pgfsys@useobject{currentmarker}{}%
\end{pgfscope}%
\begin{pgfscope}%
\pgfsys@transformshift{2.250935in}{0.653658in}%
\pgfsys@useobject{currentmarker}{}%
\end{pgfscope}%
\begin{pgfscope}%
\pgfsys@transformshift{2.251079in}{0.689323in}%
\pgfsys@useobject{currentmarker}{}%
\end{pgfscope}%
\begin{pgfscope}%
\pgfsys@transformshift{2.251223in}{0.701257in}%
\pgfsys@useobject{currentmarker}{}%
\end{pgfscope}%
\begin{pgfscope}%
\pgfsys@transformshift{2.251368in}{0.679560in}%
\pgfsys@useobject{currentmarker}{}%
\end{pgfscope}%
\begin{pgfscope}%
\pgfsys@transformshift{2.251512in}{0.679012in}%
\pgfsys@useobject{currentmarker}{}%
\end{pgfscope}%
\begin{pgfscope}%
\pgfsys@transformshift{2.251656in}{0.693315in}%
\pgfsys@useobject{currentmarker}{}%
\end{pgfscope}%
\begin{pgfscope}%
\pgfsys@transformshift{2.251800in}{0.707314in}%
\pgfsys@useobject{currentmarker}{}%
\end{pgfscope}%
\begin{pgfscope}%
\pgfsys@transformshift{2.251944in}{0.695550in}%
\pgfsys@useobject{currentmarker}{}%
\end{pgfscope}%
\begin{pgfscope}%
\pgfsys@transformshift{2.252087in}{0.711409in}%
\pgfsys@useobject{currentmarker}{}%
\end{pgfscope}%
\begin{pgfscope}%
\pgfsys@transformshift{2.252231in}{0.710220in}%
\pgfsys@useobject{currentmarker}{}%
\end{pgfscope}%
\begin{pgfscope}%
\pgfsys@transformshift{2.252375in}{0.657986in}%
\pgfsys@useobject{currentmarker}{}%
\end{pgfscope}%
\begin{pgfscope}%
\pgfsys@transformshift{2.252518in}{0.654439in}%
\pgfsys@useobject{currentmarker}{}%
\end{pgfscope}%
\begin{pgfscope}%
\pgfsys@transformshift{2.252662in}{0.638055in}%
\pgfsys@useobject{currentmarker}{}%
\end{pgfscope}%
\begin{pgfscope}%
\pgfsys@transformshift{2.252805in}{0.676442in}%
\pgfsys@useobject{currentmarker}{}%
\end{pgfscope}%
\begin{pgfscope}%
\pgfsys@transformshift{2.252948in}{0.663980in}%
\pgfsys@useobject{currentmarker}{}%
\end{pgfscope}%
\begin{pgfscope}%
\pgfsys@transformshift{2.253091in}{0.676592in}%
\pgfsys@useobject{currentmarker}{}%
\end{pgfscope}%
\begin{pgfscope}%
\pgfsys@transformshift{2.253234in}{0.695366in}%
\pgfsys@useobject{currentmarker}{}%
\end{pgfscope}%
\begin{pgfscope}%
\pgfsys@transformshift{2.253377in}{0.660301in}%
\pgfsys@useobject{currentmarker}{}%
\end{pgfscope}%
\begin{pgfscope}%
\pgfsys@transformshift{2.253520in}{0.692952in}%
\pgfsys@useobject{currentmarker}{}%
\end{pgfscope}%
\begin{pgfscope}%
\pgfsys@transformshift{2.253663in}{0.724638in}%
\pgfsys@useobject{currentmarker}{}%
\end{pgfscope}%
\begin{pgfscope}%
\pgfsys@transformshift{2.253806in}{0.694144in}%
\pgfsys@useobject{currentmarker}{}%
\end{pgfscope}%
\begin{pgfscope}%
\pgfsys@transformshift{2.253948in}{0.662542in}%
\pgfsys@useobject{currentmarker}{}%
\end{pgfscope}%
\begin{pgfscope}%
\pgfsys@transformshift{2.254091in}{0.687305in}%
\pgfsys@useobject{currentmarker}{}%
\end{pgfscope}%
\begin{pgfscope}%
\pgfsys@transformshift{2.254233in}{0.679545in}%
\pgfsys@useobject{currentmarker}{}%
\end{pgfscope}%
\begin{pgfscope}%
\pgfsys@transformshift{2.254375in}{0.654846in}%
\pgfsys@useobject{currentmarker}{}%
\end{pgfscope}%
\begin{pgfscope}%
\pgfsys@transformshift{2.254518in}{0.640337in}%
\pgfsys@useobject{currentmarker}{}%
\end{pgfscope}%
\begin{pgfscope}%
\pgfsys@transformshift{2.254660in}{0.682372in}%
\pgfsys@useobject{currentmarker}{}%
\end{pgfscope}%
\begin{pgfscope}%
\pgfsys@transformshift{2.254802in}{0.714596in}%
\pgfsys@useobject{currentmarker}{}%
\end{pgfscope}%
\begin{pgfscope}%
\pgfsys@transformshift{2.254944in}{0.678984in}%
\pgfsys@useobject{currentmarker}{}%
\end{pgfscope}%
\begin{pgfscope}%
\pgfsys@transformshift{2.255086in}{0.672408in}%
\pgfsys@useobject{currentmarker}{}%
\end{pgfscope}%
\begin{pgfscope}%
\pgfsys@transformshift{2.255227in}{0.686225in}%
\pgfsys@useobject{currentmarker}{}%
\end{pgfscope}%
\begin{pgfscope}%
\pgfsys@transformshift{2.255369in}{0.689422in}%
\pgfsys@useobject{currentmarker}{}%
\end{pgfscope}%
\begin{pgfscope}%
\pgfsys@transformshift{2.255511in}{0.663292in}%
\pgfsys@useobject{currentmarker}{}%
\end{pgfscope}%
\begin{pgfscope}%
\pgfsys@transformshift{2.255652in}{0.700407in}%
\pgfsys@useobject{currentmarker}{}%
\end{pgfscope}%
\begin{pgfscope}%
\pgfsys@transformshift{2.255794in}{0.681218in}%
\pgfsys@useobject{currentmarker}{}%
\end{pgfscope}%
\begin{pgfscope}%
\pgfsys@transformshift{2.255935in}{0.699670in}%
\pgfsys@useobject{currentmarker}{}%
\end{pgfscope}%
\begin{pgfscope}%
\pgfsys@transformshift{2.256076in}{0.654761in}%
\pgfsys@useobject{currentmarker}{}%
\end{pgfscope}%
\begin{pgfscope}%
\pgfsys@transformshift{2.256217in}{0.662021in}%
\pgfsys@useobject{currentmarker}{}%
\end{pgfscope}%
\begin{pgfscope}%
\pgfsys@transformshift{2.256358in}{0.655766in}%
\pgfsys@useobject{currentmarker}{}%
\end{pgfscope}%
\begin{pgfscope}%
\pgfsys@transformshift{2.256499in}{0.630917in}%
\pgfsys@useobject{currentmarker}{}%
\end{pgfscope}%
\begin{pgfscope}%
\pgfsys@transformshift{2.256640in}{0.689770in}%
\pgfsys@useobject{currentmarker}{}%
\end{pgfscope}%
\begin{pgfscope}%
\pgfsys@transformshift{2.256781in}{0.654405in}%
\pgfsys@useobject{currentmarker}{}%
\end{pgfscope}%
\begin{pgfscope}%
\pgfsys@transformshift{2.256922in}{0.653953in}%
\pgfsys@useobject{currentmarker}{}%
\end{pgfscope}%
\begin{pgfscope}%
\pgfsys@transformshift{2.257062in}{0.696136in}%
\pgfsys@useobject{currentmarker}{}%
\end{pgfscope}%
\begin{pgfscope}%
\pgfsys@transformshift{2.257203in}{0.697490in}%
\pgfsys@useobject{currentmarker}{}%
\end{pgfscope}%
\begin{pgfscope}%
\pgfsys@transformshift{2.257343in}{0.672456in}%
\pgfsys@useobject{currentmarker}{}%
\end{pgfscope}%
\begin{pgfscope}%
\pgfsys@transformshift{2.257484in}{0.636366in}%
\pgfsys@useobject{currentmarker}{}%
\end{pgfscope}%
\begin{pgfscope}%
\pgfsys@transformshift{2.257624in}{0.620733in}%
\pgfsys@useobject{currentmarker}{}%
\end{pgfscope}%
\begin{pgfscope}%
\pgfsys@transformshift{2.257764in}{0.646307in}%
\pgfsys@useobject{currentmarker}{}%
\end{pgfscope}%
\begin{pgfscope}%
\pgfsys@transformshift{2.257904in}{0.643135in}%
\pgfsys@useobject{currentmarker}{}%
\end{pgfscope}%
\begin{pgfscope}%
\pgfsys@transformshift{2.258044in}{0.634727in}%
\pgfsys@useobject{currentmarker}{}%
\end{pgfscope}%
\begin{pgfscope}%
\pgfsys@transformshift{2.258184in}{0.674726in}%
\pgfsys@useobject{currentmarker}{}%
\end{pgfscope}%
\begin{pgfscope}%
\pgfsys@transformshift{2.258324in}{0.627905in}%
\pgfsys@useobject{currentmarker}{}%
\end{pgfscope}%
\begin{pgfscope}%
\pgfsys@transformshift{2.258464in}{0.590248in}%
\pgfsys@useobject{currentmarker}{}%
\end{pgfscope}%
\begin{pgfscope}%
\pgfsys@transformshift{2.258604in}{0.691105in}%
\pgfsys@useobject{currentmarker}{}%
\end{pgfscope}%
\begin{pgfscope}%
\pgfsys@transformshift{2.258743in}{0.707850in}%
\pgfsys@useobject{currentmarker}{}%
\end{pgfscope}%
\begin{pgfscope}%
\pgfsys@transformshift{2.258883in}{0.691628in}%
\pgfsys@useobject{currentmarker}{}%
\end{pgfscope}%
\begin{pgfscope}%
\pgfsys@transformshift{2.259022in}{0.727400in}%
\pgfsys@useobject{currentmarker}{}%
\end{pgfscope}%
\begin{pgfscope}%
\pgfsys@transformshift{2.259161in}{0.701337in}%
\pgfsys@useobject{currentmarker}{}%
\end{pgfscope}%
\begin{pgfscope}%
\pgfsys@transformshift{2.259301in}{0.675359in}%
\pgfsys@useobject{currentmarker}{}%
\end{pgfscope}%
\begin{pgfscope}%
\pgfsys@transformshift{2.259440in}{0.627953in}%
\pgfsys@useobject{currentmarker}{}%
\end{pgfscope}%
\begin{pgfscope}%
\pgfsys@transformshift{2.259579in}{0.629988in}%
\pgfsys@useobject{currentmarker}{}%
\end{pgfscope}%
\begin{pgfscope}%
\pgfsys@transformshift{2.259718in}{0.662198in}%
\pgfsys@useobject{currentmarker}{}%
\end{pgfscope}%
\begin{pgfscope}%
\pgfsys@transformshift{2.259857in}{0.668037in}%
\pgfsys@useobject{currentmarker}{}%
\end{pgfscope}%
\begin{pgfscope}%
\pgfsys@transformshift{2.259995in}{0.658715in}%
\pgfsys@useobject{currentmarker}{}%
\end{pgfscope}%
\begin{pgfscope}%
\pgfsys@transformshift{2.260134in}{0.688499in}%
\pgfsys@useobject{currentmarker}{}%
\end{pgfscope}%
\begin{pgfscope}%
\pgfsys@transformshift{2.260273in}{0.732446in}%
\pgfsys@useobject{currentmarker}{}%
\end{pgfscope}%
\begin{pgfscope}%
\pgfsys@transformshift{2.260411in}{0.687231in}%
\pgfsys@useobject{currentmarker}{}%
\end{pgfscope}%
\begin{pgfscope}%
\pgfsys@transformshift{2.260550in}{0.715975in}%
\pgfsys@useobject{currentmarker}{}%
\end{pgfscope}%
\begin{pgfscope}%
\pgfsys@transformshift{2.260688in}{0.718137in}%
\pgfsys@useobject{currentmarker}{}%
\end{pgfscope}%
\begin{pgfscope}%
\pgfsys@transformshift{2.260827in}{0.681819in}%
\pgfsys@useobject{currentmarker}{}%
\end{pgfscope}%
\begin{pgfscope}%
\pgfsys@transformshift{2.260965in}{0.639055in}%
\pgfsys@useobject{currentmarker}{}%
\end{pgfscope}%
\begin{pgfscope}%
\pgfsys@transformshift{2.261103in}{0.692238in}%
\pgfsys@useobject{currentmarker}{}%
\end{pgfscope}%
\begin{pgfscope}%
\pgfsys@transformshift{2.261241in}{0.725174in}%
\pgfsys@useobject{currentmarker}{}%
\end{pgfscope}%
\begin{pgfscope}%
\pgfsys@transformshift{2.261379in}{0.700106in}%
\pgfsys@useobject{currentmarker}{}%
\end{pgfscope}%
\begin{pgfscope}%
\pgfsys@transformshift{2.261517in}{0.679129in}%
\pgfsys@useobject{currentmarker}{}%
\end{pgfscope}%
\begin{pgfscope}%
\pgfsys@transformshift{2.261654in}{0.703659in}%
\pgfsys@useobject{currentmarker}{}%
\end{pgfscope}%
\begin{pgfscope}%
\pgfsys@transformshift{2.261792in}{0.687547in}%
\pgfsys@useobject{currentmarker}{}%
\end{pgfscope}%
\begin{pgfscope}%
\pgfsys@transformshift{2.261930in}{0.634297in}%
\pgfsys@useobject{currentmarker}{}%
\end{pgfscope}%
\begin{pgfscope}%
\pgfsys@transformshift{2.262067in}{0.681215in}%
\pgfsys@useobject{currentmarker}{}%
\end{pgfscope}%
\begin{pgfscope}%
\pgfsys@transformshift{2.262205in}{0.672779in}%
\pgfsys@useobject{currentmarker}{}%
\end{pgfscope}%
\begin{pgfscope}%
\pgfsys@transformshift{2.262342in}{0.659315in}%
\pgfsys@useobject{currentmarker}{}%
\end{pgfscope}%
\begin{pgfscope}%
\pgfsys@transformshift{2.262479in}{0.654322in}%
\pgfsys@useobject{currentmarker}{}%
\end{pgfscope}%
\begin{pgfscope}%
\pgfsys@transformshift{2.262617in}{0.681712in}%
\pgfsys@useobject{currentmarker}{}%
\end{pgfscope}%
\begin{pgfscope}%
\pgfsys@transformshift{2.262754in}{0.663223in}%
\pgfsys@useobject{currentmarker}{}%
\end{pgfscope}%
\begin{pgfscope}%
\pgfsys@transformshift{2.262891in}{0.636156in}%
\pgfsys@useobject{currentmarker}{}%
\end{pgfscope}%
\begin{pgfscope}%
\pgfsys@transformshift{2.263028in}{0.703908in}%
\pgfsys@useobject{currentmarker}{}%
\end{pgfscope}%
\begin{pgfscope}%
\pgfsys@transformshift{2.263165in}{0.692141in}%
\pgfsys@useobject{currentmarker}{}%
\end{pgfscope}%
\begin{pgfscope}%
\pgfsys@transformshift{2.263301in}{0.696151in}%
\pgfsys@useobject{currentmarker}{}%
\end{pgfscope}%
\begin{pgfscope}%
\pgfsys@transformshift{2.263438in}{0.680734in}%
\pgfsys@useobject{currentmarker}{}%
\end{pgfscope}%
\begin{pgfscope}%
\pgfsys@transformshift{2.263575in}{0.656500in}%
\pgfsys@useobject{currentmarker}{}%
\end{pgfscope}%
\begin{pgfscope}%
\pgfsys@transformshift{2.263711in}{0.657229in}%
\pgfsys@useobject{currentmarker}{}%
\end{pgfscope}%
\begin{pgfscope}%
\pgfsys@transformshift{2.263848in}{0.624708in}%
\pgfsys@useobject{currentmarker}{}%
\end{pgfscope}%
\begin{pgfscope}%
\pgfsys@transformshift{2.263984in}{0.593841in}%
\pgfsys@useobject{currentmarker}{}%
\end{pgfscope}%
\begin{pgfscope}%
\pgfsys@transformshift{2.264120in}{0.672873in}%
\pgfsys@useobject{currentmarker}{}%
\end{pgfscope}%
\begin{pgfscope}%
\pgfsys@transformshift{2.264256in}{0.645601in}%
\pgfsys@useobject{currentmarker}{}%
\end{pgfscope}%
\begin{pgfscope}%
\pgfsys@transformshift{2.264392in}{0.653257in}%
\pgfsys@useobject{currentmarker}{}%
\end{pgfscope}%
\begin{pgfscope}%
\pgfsys@transformshift{2.264529in}{0.721783in}%
\pgfsys@useobject{currentmarker}{}%
\end{pgfscope}%
\begin{pgfscope}%
\pgfsys@transformshift{2.264664in}{0.733285in}%
\pgfsys@useobject{currentmarker}{}%
\end{pgfscope}%
\begin{pgfscope}%
\pgfsys@transformshift{2.264800in}{0.722527in}%
\pgfsys@useobject{currentmarker}{}%
\end{pgfscope}%
\begin{pgfscope}%
\pgfsys@transformshift{2.264936in}{0.677941in}%
\pgfsys@useobject{currentmarker}{}%
\end{pgfscope}%
\begin{pgfscope}%
\pgfsys@transformshift{2.265072in}{0.714649in}%
\pgfsys@useobject{currentmarker}{}%
\end{pgfscope}%
\begin{pgfscope}%
\pgfsys@transformshift{2.265207in}{0.709469in}%
\pgfsys@useobject{currentmarker}{}%
\end{pgfscope}%
\begin{pgfscope}%
\pgfsys@transformshift{2.265343in}{0.612561in}%
\pgfsys@useobject{currentmarker}{}%
\end{pgfscope}%
\begin{pgfscope}%
\pgfsys@transformshift{2.265478in}{0.656899in}%
\pgfsys@useobject{currentmarker}{}%
\end{pgfscope}%
\begin{pgfscope}%
\pgfsys@transformshift{2.265614in}{0.705286in}%
\pgfsys@useobject{currentmarker}{}%
\end{pgfscope}%
\begin{pgfscope}%
\pgfsys@transformshift{2.265749in}{0.669695in}%
\pgfsys@useobject{currentmarker}{}%
\end{pgfscope}%
\begin{pgfscope}%
\pgfsys@transformshift{2.265884in}{0.648200in}%
\pgfsys@useobject{currentmarker}{}%
\end{pgfscope}%
\begin{pgfscope}%
\pgfsys@transformshift{2.266019in}{0.658233in}%
\pgfsys@useobject{currentmarker}{}%
\end{pgfscope}%
\begin{pgfscope}%
\pgfsys@transformshift{2.266154in}{0.676541in}%
\pgfsys@useobject{currentmarker}{}%
\end{pgfscope}%
\begin{pgfscope}%
\pgfsys@transformshift{2.266289in}{0.670689in}%
\pgfsys@useobject{currentmarker}{}%
\end{pgfscope}%
\begin{pgfscope}%
\pgfsys@transformshift{2.266424in}{0.676695in}%
\pgfsys@useobject{currentmarker}{}%
\end{pgfscope}%
\begin{pgfscope}%
\pgfsys@transformshift{2.266559in}{0.690134in}%
\pgfsys@useobject{currentmarker}{}%
\end{pgfscope}%
\begin{pgfscope}%
\pgfsys@transformshift{2.266694in}{0.717867in}%
\pgfsys@useobject{currentmarker}{}%
\end{pgfscope}%
\begin{pgfscope}%
\pgfsys@transformshift{2.266828in}{0.702821in}%
\pgfsys@useobject{currentmarker}{}%
\end{pgfscope}%
\begin{pgfscope}%
\pgfsys@transformshift{2.266963in}{0.704742in}%
\pgfsys@useobject{currentmarker}{}%
\end{pgfscope}%
\begin{pgfscope}%
\pgfsys@transformshift{2.267098in}{0.696943in}%
\pgfsys@useobject{currentmarker}{}%
\end{pgfscope}%
\begin{pgfscope}%
\pgfsys@transformshift{2.267232in}{0.703939in}%
\pgfsys@useobject{currentmarker}{}%
\end{pgfscope}%
\begin{pgfscope}%
\pgfsys@transformshift{2.267366in}{0.689822in}%
\pgfsys@useobject{currentmarker}{}%
\end{pgfscope}%
\begin{pgfscope}%
\pgfsys@transformshift{2.267501in}{0.645271in}%
\pgfsys@useobject{currentmarker}{}%
\end{pgfscope}%
\begin{pgfscope}%
\pgfsys@transformshift{2.267635in}{0.621526in}%
\pgfsys@useobject{currentmarker}{}%
\end{pgfscope}%
\begin{pgfscope}%
\pgfsys@transformshift{2.267769in}{0.675530in}%
\pgfsys@useobject{currentmarker}{}%
\end{pgfscope}%
\begin{pgfscope}%
\pgfsys@transformshift{2.267903in}{0.704719in}%
\pgfsys@useobject{currentmarker}{}%
\end{pgfscope}%
\begin{pgfscope}%
\pgfsys@transformshift{2.268037in}{0.608201in}%
\pgfsys@useobject{currentmarker}{}%
\end{pgfscope}%
\begin{pgfscope}%
\pgfsys@transformshift{2.268171in}{0.622905in}%
\pgfsys@useobject{currentmarker}{}%
\end{pgfscope}%
\begin{pgfscope}%
\pgfsys@transformshift{2.268304in}{0.697939in}%
\pgfsys@useobject{currentmarker}{}%
\end{pgfscope}%
\begin{pgfscope}%
\pgfsys@transformshift{2.268438in}{0.669543in}%
\pgfsys@useobject{currentmarker}{}%
\end{pgfscope}%
\begin{pgfscope}%
\pgfsys@transformshift{2.268572in}{0.695268in}%
\pgfsys@useobject{currentmarker}{}%
\end{pgfscope}%
\begin{pgfscope}%
\pgfsys@transformshift{2.268705in}{0.705057in}%
\pgfsys@useobject{currentmarker}{}%
\end{pgfscope}%
\begin{pgfscope}%
\pgfsys@transformshift{2.268839in}{0.684384in}%
\pgfsys@useobject{currentmarker}{}%
\end{pgfscope}%
\begin{pgfscope}%
\pgfsys@transformshift{2.268972in}{0.656603in}%
\pgfsys@useobject{currentmarker}{}%
\end{pgfscope}%
\begin{pgfscope}%
\pgfsys@transformshift{2.269105in}{0.630561in}%
\pgfsys@useobject{currentmarker}{}%
\end{pgfscope}%
\begin{pgfscope}%
\pgfsys@transformshift{2.269238in}{0.717822in}%
\pgfsys@useobject{currentmarker}{}%
\end{pgfscope}%
\begin{pgfscope}%
\pgfsys@transformshift{2.269371in}{0.737095in}%
\pgfsys@useobject{currentmarker}{}%
\end{pgfscope}%
\begin{pgfscope}%
\pgfsys@transformshift{2.269505in}{0.682879in}%
\pgfsys@useobject{currentmarker}{}%
\end{pgfscope}%
\begin{pgfscope}%
\pgfsys@transformshift{2.269638in}{0.665629in}%
\pgfsys@useobject{currentmarker}{}%
\end{pgfscope}%
\begin{pgfscope}%
\pgfsys@transformshift{2.269770in}{0.641309in}%
\pgfsys@useobject{currentmarker}{}%
\end{pgfscope}%
\begin{pgfscope}%
\pgfsys@transformshift{2.269903in}{0.675560in}%
\pgfsys@useobject{currentmarker}{}%
\end{pgfscope}%
\begin{pgfscope}%
\pgfsys@transformshift{2.270036in}{0.703418in}%
\pgfsys@useobject{currentmarker}{}%
\end{pgfscope}%
\begin{pgfscope}%
\pgfsys@transformshift{2.270169in}{0.618859in}%
\pgfsys@useobject{currentmarker}{}%
\end{pgfscope}%
\begin{pgfscope}%
\pgfsys@transformshift{2.270301in}{0.677442in}%
\pgfsys@useobject{currentmarker}{}%
\end{pgfscope}%
\begin{pgfscope}%
\pgfsys@transformshift{2.270434in}{0.684949in}%
\pgfsys@useobject{currentmarker}{}%
\end{pgfscope}%
\begin{pgfscope}%
\pgfsys@transformshift{2.270566in}{0.625947in}%
\pgfsys@useobject{currentmarker}{}%
\end{pgfscope}%
\begin{pgfscope}%
\pgfsys@transformshift{2.270698in}{0.662892in}%
\pgfsys@useobject{currentmarker}{}%
\end{pgfscope}%
\begin{pgfscope}%
\pgfsys@transformshift{2.270831in}{0.680748in}%
\pgfsys@useobject{currentmarker}{}%
\end{pgfscope}%
\begin{pgfscope}%
\pgfsys@transformshift{2.270963in}{0.710043in}%
\pgfsys@useobject{currentmarker}{}%
\end{pgfscope}%
\begin{pgfscope}%
\pgfsys@transformshift{2.271095in}{0.710242in}%
\pgfsys@useobject{currentmarker}{}%
\end{pgfscope}%
\begin{pgfscope}%
\pgfsys@transformshift{2.271227in}{0.639190in}%
\pgfsys@useobject{currentmarker}{}%
\end{pgfscope}%
\begin{pgfscope}%
\pgfsys@transformshift{2.271359in}{0.663887in}%
\pgfsys@useobject{currentmarker}{}%
\end{pgfscope}%
\begin{pgfscope}%
\pgfsys@transformshift{2.271491in}{0.706935in}%
\pgfsys@useobject{currentmarker}{}%
\end{pgfscope}%
\begin{pgfscope}%
\pgfsys@transformshift{2.271623in}{0.686287in}%
\pgfsys@useobject{currentmarker}{}%
\end{pgfscope}%
\begin{pgfscope}%
\pgfsys@transformshift{2.271754in}{0.687468in}%
\pgfsys@useobject{currentmarker}{}%
\end{pgfscope}%
\begin{pgfscope}%
\pgfsys@transformshift{2.271886in}{0.694717in}%
\pgfsys@useobject{currentmarker}{}%
\end{pgfscope}%
\begin{pgfscope}%
\pgfsys@transformshift{2.272018in}{0.689393in}%
\pgfsys@useobject{currentmarker}{}%
\end{pgfscope}%
\begin{pgfscope}%
\pgfsys@transformshift{2.272149in}{0.699422in}%
\pgfsys@useobject{currentmarker}{}%
\end{pgfscope}%
\begin{pgfscope}%
\pgfsys@transformshift{2.272281in}{0.689321in}%
\pgfsys@useobject{currentmarker}{}%
\end{pgfscope}%
\begin{pgfscope}%
\pgfsys@transformshift{2.272412in}{0.632581in}%
\pgfsys@useobject{currentmarker}{}%
\end{pgfscope}%
\begin{pgfscope}%
\pgfsys@transformshift{2.272543in}{0.621202in}%
\pgfsys@useobject{currentmarker}{}%
\end{pgfscope}%
\begin{pgfscope}%
\pgfsys@transformshift{2.272674in}{0.640186in}%
\pgfsys@useobject{currentmarker}{}%
\end{pgfscope}%
\begin{pgfscope}%
\pgfsys@transformshift{2.272805in}{0.663630in}%
\pgfsys@useobject{currentmarker}{}%
\end{pgfscope}%
\begin{pgfscope}%
\pgfsys@transformshift{2.272936in}{0.634707in}%
\pgfsys@useobject{currentmarker}{}%
\end{pgfscope}%
\begin{pgfscope}%
\pgfsys@transformshift{2.273067in}{0.634922in}%
\pgfsys@useobject{currentmarker}{}%
\end{pgfscope}%
\begin{pgfscope}%
\pgfsys@transformshift{2.273198in}{0.660453in}%
\pgfsys@useobject{currentmarker}{}%
\end{pgfscope}%
\begin{pgfscope}%
\pgfsys@transformshift{2.273329in}{0.694990in}%
\pgfsys@useobject{currentmarker}{}%
\end{pgfscope}%
\begin{pgfscope}%
\pgfsys@transformshift{2.273460in}{0.678135in}%
\pgfsys@useobject{currentmarker}{}%
\end{pgfscope}%
\begin{pgfscope}%
\pgfsys@transformshift{2.273590in}{0.640511in}%
\pgfsys@useobject{currentmarker}{}%
\end{pgfscope}%
\begin{pgfscope}%
\pgfsys@transformshift{2.273721in}{0.682518in}%
\pgfsys@useobject{currentmarker}{}%
\end{pgfscope}%
\begin{pgfscope}%
\pgfsys@transformshift{2.273852in}{0.659285in}%
\pgfsys@useobject{currentmarker}{}%
\end{pgfscope}%
\begin{pgfscope}%
\pgfsys@transformshift{2.273982in}{0.684687in}%
\pgfsys@useobject{currentmarker}{}%
\end{pgfscope}%
\begin{pgfscope}%
\pgfsys@transformshift{2.274112in}{0.683009in}%
\pgfsys@useobject{currentmarker}{}%
\end{pgfscope}%
\begin{pgfscope}%
\pgfsys@transformshift{2.274243in}{0.674116in}%
\pgfsys@useobject{currentmarker}{}%
\end{pgfscope}%
\begin{pgfscope}%
\pgfsys@transformshift{2.274373in}{0.638475in}%
\pgfsys@useobject{currentmarker}{}%
\end{pgfscope}%
\begin{pgfscope}%
\pgfsys@transformshift{2.274503in}{0.694480in}%
\pgfsys@useobject{currentmarker}{}%
\end{pgfscope}%
\begin{pgfscope}%
\pgfsys@transformshift{2.274633in}{0.709923in}%
\pgfsys@useobject{currentmarker}{}%
\end{pgfscope}%
\begin{pgfscope}%
\pgfsys@transformshift{2.274763in}{0.652359in}%
\pgfsys@useobject{currentmarker}{}%
\end{pgfscope}%
\begin{pgfscope}%
\pgfsys@transformshift{2.274893in}{0.657751in}%
\pgfsys@useobject{currentmarker}{}%
\end{pgfscope}%
\begin{pgfscope}%
\pgfsys@transformshift{2.275023in}{0.659568in}%
\pgfsys@useobject{currentmarker}{}%
\end{pgfscope}%
\begin{pgfscope}%
\pgfsys@transformshift{2.275152in}{0.628808in}%
\pgfsys@useobject{currentmarker}{}%
\end{pgfscope}%
\begin{pgfscope}%
\pgfsys@transformshift{2.275282in}{0.639788in}%
\pgfsys@useobject{currentmarker}{}%
\end{pgfscope}%
\begin{pgfscope}%
\pgfsys@transformshift{2.275412in}{0.671710in}%
\pgfsys@useobject{currentmarker}{}%
\end{pgfscope}%
\begin{pgfscope}%
\pgfsys@transformshift{2.275541in}{0.739571in}%
\pgfsys@useobject{currentmarker}{}%
\end{pgfscope}%
\begin{pgfscope}%
\pgfsys@transformshift{2.275671in}{0.705023in}%
\pgfsys@useobject{currentmarker}{}%
\end{pgfscope}%
\begin{pgfscope}%
\pgfsys@transformshift{2.275800in}{0.617422in}%
\pgfsys@useobject{currentmarker}{}%
\end{pgfscope}%
\begin{pgfscope}%
\pgfsys@transformshift{2.275929in}{0.621267in}%
\pgfsys@useobject{currentmarker}{}%
\end{pgfscope}%
\begin{pgfscope}%
\pgfsys@transformshift{2.276058in}{0.654245in}%
\pgfsys@useobject{currentmarker}{}%
\end{pgfscope}%
\begin{pgfscope}%
\pgfsys@transformshift{2.276188in}{0.675421in}%
\pgfsys@useobject{currentmarker}{}%
\end{pgfscope}%
\begin{pgfscope}%
\pgfsys@transformshift{2.276317in}{0.662926in}%
\pgfsys@useobject{currentmarker}{}%
\end{pgfscope}%
\begin{pgfscope}%
\pgfsys@transformshift{2.276446in}{0.593707in}%
\pgfsys@useobject{currentmarker}{}%
\end{pgfscope}%
\begin{pgfscope}%
\pgfsys@transformshift{2.276575in}{0.699276in}%
\pgfsys@useobject{currentmarker}{}%
\end{pgfscope}%
\begin{pgfscope}%
\pgfsys@transformshift{2.276703in}{0.757518in}%
\pgfsys@useobject{currentmarker}{}%
\end{pgfscope}%
\begin{pgfscope}%
\pgfsys@transformshift{2.276832in}{0.761603in}%
\pgfsys@useobject{currentmarker}{}%
\end{pgfscope}%
\begin{pgfscope}%
\pgfsys@transformshift{2.276961in}{0.685873in}%
\pgfsys@useobject{currentmarker}{}%
\end{pgfscope}%
\begin{pgfscope}%
\pgfsys@transformshift{2.277090in}{0.612118in}%
\pgfsys@useobject{currentmarker}{}%
\end{pgfscope}%
\begin{pgfscope}%
\pgfsys@transformshift{2.277218in}{0.645655in}%
\pgfsys@useobject{currentmarker}{}%
\end{pgfscope}%
\begin{pgfscope}%
\pgfsys@transformshift{2.277347in}{0.656503in}%
\pgfsys@useobject{currentmarker}{}%
\end{pgfscope}%
\begin{pgfscope}%
\pgfsys@transformshift{2.277475in}{0.655345in}%
\pgfsys@useobject{currentmarker}{}%
\end{pgfscope}%
\begin{pgfscope}%
\pgfsys@transformshift{2.277603in}{0.669952in}%
\pgfsys@useobject{currentmarker}{}%
\end{pgfscope}%
\begin{pgfscope}%
\pgfsys@transformshift{2.277732in}{0.685307in}%
\pgfsys@useobject{currentmarker}{}%
\end{pgfscope}%
\begin{pgfscope}%
\pgfsys@transformshift{2.277860in}{0.703907in}%
\pgfsys@useobject{currentmarker}{}%
\end{pgfscope}%
\begin{pgfscope}%
\pgfsys@transformshift{2.277988in}{0.640740in}%
\pgfsys@useobject{currentmarker}{}%
\end{pgfscope}%
\begin{pgfscope}%
\pgfsys@transformshift{2.278116in}{0.651054in}%
\pgfsys@useobject{currentmarker}{}%
\end{pgfscope}%
\begin{pgfscope}%
\pgfsys@transformshift{2.278244in}{0.675960in}%
\pgfsys@useobject{currentmarker}{}%
\end{pgfscope}%
\begin{pgfscope}%
\pgfsys@transformshift{2.278372in}{0.691841in}%
\pgfsys@useobject{currentmarker}{}%
\end{pgfscope}%
\begin{pgfscope}%
\pgfsys@transformshift{2.278500in}{0.673071in}%
\pgfsys@useobject{currentmarker}{}%
\end{pgfscope}%
\begin{pgfscope}%
\pgfsys@transformshift{2.278627in}{0.653595in}%
\pgfsys@useobject{currentmarker}{}%
\end{pgfscope}%
\begin{pgfscope}%
\pgfsys@transformshift{2.278755in}{0.675640in}%
\pgfsys@useobject{currentmarker}{}%
\end{pgfscope}%
\begin{pgfscope}%
\pgfsys@transformshift{2.278883in}{0.690476in}%
\pgfsys@useobject{currentmarker}{}%
\end{pgfscope}%
\begin{pgfscope}%
\pgfsys@transformshift{2.279010in}{0.693903in}%
\pgfsys@useobject{currentmarker}{}%
\end{pgfscope}%
\begin{pgfscope}%
\pgfsys@transformshift{2.279138in}{0.646390in}%
\pgfsys@useobject{currentmarker}{}%
\end{pgfscope}%
\begin{pgfscope}%
\pgfsys@transformshift{2.279265in}{0.631137in}%
\pgfsys@useobject{currentmarker}{}%
\end{pgfscope}%
\begin{pgfscope}%
\pgfsys@transformshift{2.279392in}{0.665355in}%
\pgfsys@useobject{currentmarker}{}%
\end{pgfscope}%
\begin{pgfscope}%
\pgfsys@transformshift{2.279520in}{0.703147in}%
\pgfsys@useobject{currentmarker}{}%
\end{pgfscope}%
\begin{pgfscope}%
\pgfsys@transformshift{2.279647in}{0.660187in}%
\pgfsys@useobject{currentmarker}{}%
\end{pgfscope}%
\begin{pgfscope}%
\pgfsys@transformshift{2.279774in}{0.617738in}%
\pgfsys@useobject{currentmarker}{}%
\end{pgfscope}%
\begin{pgfscope}%
\pgfsys@transformshift{2.279901in}{0.612342in}%
\pgfsys@useobject{currentmarker}{}%
\end{pgfscope}%
\begin{pgfscope}%
\pgfsys@transformshift{2.280028in}{0.659620in}%
\pgfsys@useobject{currentmarker}{}%
\end{pgfscope}%
\begin{pgfscope}%
\pgfsys@transformshift{2.280155in}{0.647861in}%
\pgfsys@useobject{currentmarker}{}%
\end{pgfscope}%
\begin{pgfscope}%
\pgfsys@transformshift{2.280282in}{0.675484in}%
\pgfsys@useobject{currentmarker}{}%
\end{pgfscope}%
\begin{pgfscope}%
\pgfsys@transformshift{2.280408in}{0.680495in}%
\pgfsys@useobject{currentmarker}{}%
\end{pgfscope}%
\begin{pgfscope}%
\pgfsys@transformshift{2.280535in}{0.693669in}%
\pgfsys@useobject{currentmarker}{}%
\end{pgfscope}%
\begin{pgfscope}%
\pgfsys@transformshift{2.280662in}{0.684703in}%
\pgfsys@useobject{currentmarker}{}%
\end{pgfscope}%
\begin{pgfscope}%
\pgfsys@transformshift{2.280788in}{0.684902in}%
\pgfsys@useobject{currentmarker}{}%
\end{pgfscope}%
\begin{pgfscope}%
\pgfsys@transformshift{2.280915in}{0.713158in}%
\pgfsys@useobject{currentmarker}{}%
\end{pgfscope}%
\begin{pgfscope}%
\pgfsys@transformshift{2.281041in}{0.689498in}%
\pgfsys@useobject{currentmarker}{}%
\end{pgfscope}%
\begin{pgfscope}%
\pgfsys@transformshift{2.281167in}{0.678358in}%
\pgfsys@useobject{currentmarker}{}%
\end{pgfscope}%
\begin{pgfscope}%
\pgfsys@transformshift{2.281294in}{0.685504in}%
\pgfsys@useobject{currentmarker}{}%
\end{pgfscope}%
\begin{pgfscope}%
\pgfsys@transformshift{2.281420in}{0.654222in}%
\pgfsys@useobject{currentmarker}{}%
\end{pgfscope}%
\begin{pgfscope}%
\pgfsys@transformshift{2.281546in}{0.628392in}%
\pgfsys@useobject{currentmarker}{}%
\end{pgfscope}%
\begin{pgfscope}%
\pgfsys@transformshift{2.281672in}{0.700803in}%
\pgfsys@useobject{currentmarker}{}%
\end{pgfscope}%
\begin{pgfscope}%
\pgfsys@transformshift{2.281798in}{0.685082in}%
\pgfsys@useobject{currentmarker}{}%
\end{pgfscope}%
\begin{pgfscope}%
\pgfsys@transformshift{2.281924in}{0.649362in}%
\pgfsys@useobject{currentmarker}{}%
\end{pgfscope}%
\begin{pgfscope}%
\pgfsys@transformshift{2.282050in}{0.642260in}%
\pgfsys@useobject{currentmarker}{}%
\end{pgfscope}%
\begin{pgfscope}%
\pgfsys@transformshift{2.282175in}{0.704298in}%
\pgfsys@useobject{currentmarker}{}%
\end{pgfscope}%
\begin{pgfscope}%
\pgfsys@transformshift{2.282301in}{0.703003in}%
\pgfsys@useobject{currentmarker}{}%
\end{pgfscope}%
\begin{pgfscope}%
\pgfsys@transformshift{2.282427in}{0.705209in}%
\pgfsys@useobject{currentmarker}{}%
\end{pgfscope}%
\begin{pgfscope}%
\pgfsys@transformshift{2.282552in}{0.677271in}%
\pgfsys@useobject{currentmarker}{}%
\end{pgfscope}%
\begin{pgfscope}%
\pgfsys@transformshift{2.282678in}{0.674395in}%
\pgfsys@useobject{currentmarker}{}%
\end{pgfscope}%
\begin{pgfscope}%
\pgfsys@transformshift{2.282803in}{0.643454in}%
\pgfsys@useobject{currentmarker}{}%
\end{pgfscope}%
\begin{pgfscope}%
\pgfsys@transformshift{2.282928in}{0.682233in}%
\pgfsys@useobject{currentmarker}{}%
\end{pgfscope}%
\begin{pgfscope}%
\pgfsys@transformshift{2.283054in}{0.685461in}%
\pgfsys@useobject{currentmarker}{}%
\end{pgfscope}%
\begin{pgfscope}%
\pgfsys@transformshift{2.283179in}{0.660369in}%
\pgfsys@useobject{currentmarker}{}%
\end{pgfscope}%
\begin{pgfscope}%
\pgfsys@transformshift{2.283304in}{0.672619in}%
\pgfsys@useobject{currentmarker}{}%
\end{pgfscope}%
\begin{pgfscope}%
\pgfsys@transformshift{2.283429in}{0.712954in}%
\pgfsys@useobject{currentmarker}{}%
\end{pgfscope}%
\begin{pgfscope}%
\pgfsys@transformshift{2.283554in}{0.682941in}%
\pgfsys@useobject{currentmarker}{}%
\end{pgfscope}%
\begin{pgfscope}%
\pgfsys@transformshift{2.283679in}{0.667730in}%
\pgfsys@useobject{currentmarker}{}%
\end{pgfscope}%
\begin{pgfscope}%
\pgfsys@transformshift{2.283804in}{0.708436in}%
\pgfsys@useobject{currentmarker}{}%
\end{pgfscope}%
\begin{pgfscope}%
\pgfsys@transformshift{2.283928in}{0.675085in}%
\pgfsys@useobject{currentmarker}{}%
\end{pgfscope}%
\begin{pgfscope}%
\pgfsys@transformshift{2.284053in}{0.683896in}%
\pgfsys@useobject{currentmarker}{}%
\end{pgfscope}%
\begin{pgfscope}%
\pgfsys@transformshift{2.284178in}{0.684381in}%
\pgfsys@useobject{currentmarker}{}%
\end{pgfscope}%
\begin{pgfscope}%
\pgfsys@transformshift{2.284302in}{0.695623in}%
\pgfsys@useobject{currentmarker}{}%
\end{pgfscope}%
\begin{pgfscope}%
\pgfsys@transformshift{2.284427in}{0.647550in}%
\pgfsys@useobject{currentmarker}{}%
\end{pgfscope}%
\begin{pgfscope}%
\pgfsys@transformshift{2.284551in}{0.667445in}%
\pgfsys@useobject{currentmarker}{}%
\end{pgfscope}%
\begin{pgfscope}%
\pgfsys@transformshift{2.284676in}{0.635571in}%
\pgfsys@useobject{currentmarker}{}%
\end{pgfscope}%
\begin{pgfscope}%
\pgfsys@transformshift{2.284800in}{0.602407in}%
\pgfsys@useobject{currentmarker}{}%
\end{pgfscope}%
\begin{pgfscope}%
\pgfsys@transformshift{2.284924in}{0.614202in}%
\pgfsys@useobject{currentmarker}{}%
\end{pgfscope}%
\begin{pgfscope}%
\pgfsys@transformshift{2.285048in}{0.620568in}%
\pgfsys@useobject{currentmarker}{}%
\end{pgfscope}%
\begin{pgfscope}%
\pgfsys@transformshift{2.285172in}{0.681380in}%
\pgfsys@useobject{currentmarker}{}%
\end{pgfscope}%
\begin{pgfscope}%
\pgfsys@transformshift{2.285296in}{0.707333in}%
\pgfsys@useobject{currentmarker}{}%
\end{pgfscope}%
\begin{pgfscope}%
\pgfsys@transformshift{2.285420in}{0.691262in}%
\pgfsys@useobject{currentmarker}{}%
\end{pgfscope}%
\begin{pgfscope}%
\pgfsys@transformshift{2.285544in}{0.670071in}%
\pgfsys@useobject{currentmarker}{}%
\end{pgfscope}%
\begin{pgfscope}%
\pgfsys@transformshift{2.285668in}{0.695140in}%
\pgfsys@useobject{currentmarker}{}%
\end{pgfscope}%
\begin{pgfscope}%
\pgfsys@transformshift{2.285792in}{0.690953in}%
\pgfsys@useobject{currentmarker}{}%
\end{pgfscope}%
\begin{pgfscope}%
\pgfsys@transformshift{2.285915in}{0.692511in}%
\pgfsys@useobject{currentmarker}{}%
\end{pgfscope}%
\begin{pgfscope}%
\pgfsys@transformshift{2.286039in}{0.670758in}%
\pgfsys@useobject{currentmarker}{}%
\end{pgfscope}%
\begin{pgfscope}%
\pgfsys@transformshift{2.286163in}{0.700373in}%
\pgfsys@useobject{currentmarker}{}%
\end{pgfscope}%
\begin{pgfscope}%
\pgfsys@transformshift{2.286286in}{0.717317in}%
\pgfsys@useobject{currentmarker}{}%
\end{pgfscope}%
\begin{pgfscope}%
\pgfsys@transformshift{2.286409in}{0.645983in}%
\pgfsys@useobject{currentmarker}{}%
\end{pgfscope}%
\begin{pgfscope}%
\pgfsys@transformshift{2.286533in}{0.668079in}%
\pgfsys@useobject{currentmarker}{}%
\end{pgfscope}%
\begin{pgfscope}%
\pgfsys@transformshift{2.286656in}{0.684820in}%
\pgfsys@useobject{currentmarker}{}%
\end{pgfscope}%
\begin{pgfscope}%
\pgfsys@transformshift{2.286779in}{0.691099in}%
\pgfsys@useobject{currentmarker}{}%
\end{pgfscope}%
\begin{pgfscope}%
\pgfsys@transformshift{2.286902in}{0.709386in}%
\pgfsys@useobject{currentmarker}{}%
\end{pgfscope}%
\begin{pgfscope}%
\pgfsys@transformshift{2.287025in}{0.679216in}%
\pgfsys@useobject{currentmarker}{}%
\end{pgfscope}%
\begin{pgfscope}%
\pgfsys@transformshift{2.287148in}{0.650438in}%
\pgfsys@useobject{currentmarker}{}%
\end{pgfscope}%
\begin{pgfscope}%
\pgfsys@transformshift{2.287271in}{0.628079in}%
\pgfsys@useobject{currentmarker}{}%
\end{pgfscope}%
\begin{pgfscope}%
\pgfsys@transformshift{2.287394in}{0.659483in}%
\pgfsys@useobject{currentmarker}{}%
\end{pgfscope}%
\begin{pgfscope}%
\pgfsys@transformshift{2.287517in}{0.707804in}%
\pgfsys@useobject{currentmarker}{}%
\end{pgfscope}%
\begin{pgfscope}%
\pgfsys@transformshift{2.287640in}{0.675398in}%
\pgfsys@useobject{currentmarker}{}%
\end{pgfscope}%
\begin{pgfscope}%
\pgfsys@transformshift{2.287762in}{0.672865in}%
\pgfsys@useobject{currentmarker}{}%
\end{pgfscope}%
\begin{pgfscope}%
\pgfsys@transformshift{2.287885in}{0.692482in}%
\pgfsys@useobject{currentmarker}{}%
\end{pgfscope}%
\begin{pgfscope}%
\pgfsys@transformshift{2.288007in}{0.679088in}%
\pgfsys@useobject{currentmarker}{}%
\end{pgfscope}%
\begin{pgfscope}%
\pgfsys@transformshift{2.288130in}{0.671585in}%
\pgfsys@useobject{currentmarker}{}%
\end{pgfscope}%
\begin{pgfscope}%
\pgfsys@transformshift{2.288252in}{0.674258in}%
\pgfsys@useobject{currentmarker}{}%
\end{pgfscope}%
\begin{pgfscope}%
\pgfsys@transformshift{2.288375in}{0.641660in}%
\pgfsys@useobject{currentmarker}{}%
\end{pgfscope}%
\begin{pgfscope}%
\pgfsys@transformshift{2.288497in}{0.690724in}%
\pgfsys@useobject{currentmarker}{}%
\end{pgfscope}%
\begin{pgfscope}%
\pgfsys@transformshift{2.288619in}{0.707782in}%
\pgfsys@useobject{currentmarker}{}%
\end{pgfscope}%
\begin{pgfscope}%
\pgfsys@transformshift{2.288741in}{0.676124in}%
\pgfsys@useobject{currentmarker}{}%
\end{pgfscope}%
\begin{pgfscope}%
\pgfsys@transformshift{2.288863in}{0.680647in}%
\pgfsys@useobject{currentmarker}{}%
\end{pgfscope}%
\begin{pgfscope}%
\pgfsys@transformshift{2.288985in}{0.666883in}%
\pgfsys@useobject{currentmarker}{}%
\end{pgfscope}%
\begin{pgfscope}%
\pgfsys@transformshift{2.289107in}{0.649389in}%
\pgfsys@useobject{currentmarker}{}%
\end{pgfscope}%
\begin{pgfscope}%
\pgfsys@transformshift{2.289229in}{0.672225in}%
\pgfsys@useobject{currentmarker}{}%
\end{pgfscope}%
\begin{pgfscope}%
\pgfsys@transformshift{2.289351in}{0.696815in}%
\pgfsys@useobject{currentmarker}{}%
\end{pgfscope}%
\begin{pgfscope}%
\pgfsys@transformshift{2.289473in}{0.686586in}%
\pgfsys@useobject{currentmarker}{}%
\end{pgfscope}%
\begin{pgfscope}%
\pgfsys@transformshift{2.289594in}{0.634180in}%
\pgfsys@useobject{currentmarker}{}%
\end{pgfscope}%
\begin{pgfscope}%
\pgfsys@transformshift{2.289716in}{0.661545in}%
\pgfsys@useobject{currentmarker}{}%
\end{pgfscope}%
\begin{pgfscope}%
\pgfsys@transformshift{2.289837in}{0.735895in}%
\pgfsys@useobject{currentmarker}{}%
\end{pgfscope}%
\begin{pgfscope}%
\pgfsys@transformshift{2.289959in}{0.751341in}%
\pgfsys@useobject{currentmarker}{}%
\end{pgfscope}%
\begin{pgfscope}%
\pgfsys@transformshift{2.290080in}{0.681450in}%
\pgfsys@useobject{currentmarker}{}%
\end{pgfscope}%
\begin{pgfscope}%
\pgfsys@transformshift{2.290202in}{0.695147in}%
\pgfsys@useobject{currentmarker}{}%
\end{pgfscope}%
\begin{pgfscope}%
\pgfsys@transformshift{2.290323in}{0.704108in}%
\pgfsys@useobject{currentmarker}{}%
\end{pgfscope}%
\begin{pgfscope}%
\pgfsys@transformshift{2.290444in}{0.643409in}%
\pgfsys@useobject{currentmarker}{}%
\end{pgfscope}%
\begin{pgfscope}%
\pgfsys@transformshift{2.290565in}{0.703679in}%
\pgfsys@useobject{currentmarker}{}%
\end{pgfscope}%
\begin{pgfscope}%
\pgfsys@transformshift{2.290686in}{0.693934in}%
\pgfsys@useobject{currentmarker}{}%
\end{pgfscope}%
\begin{pgfscope}%
\pgfsys@transformshift{2.290807in}{0.679671in}%
\pgfsys@useobject{currentmarker}{}%
\end{pgfscope}%
\begin{pgfscope}%
\pgfsys@transformshift{2.290928in}{0.681857in}%
\pgfsys@useobject{currentmarker}{}%
\end{pgfscope}%
\begin{pgfscope}%
\pgfsys@transformshift{2.291049in}{0.691992in}%
\pgfsys@useobject{currentmarker}{}%
\end{pgfscope}%
\begin{pgfscope}%
\pgfsys@transformshift{2.291170in}{0.674166in}%
\pgfsys@useobject{currentmarker}{}%
\end{pgfscope}%
\begin{pgfscope}%
\pgfsys@transformshift{2.291291in}{0.711859in}%
\pgfsys@useobject{currentmarker}{}%
\end{pgfscope}%
\begin{pgfscope}%
\pgfsys@transformshift{2.291411in}{0.668748in}%
\pgfsys@useobject{currentmarker}{}%
\end{pgfscope}%
\begin{pgfscope}%
\pgfsys@transformshift{2.291532in}{0.676178in}%
\pgfsys@useobject{currentmarker}{}%
\end{pgfscope}%
\begin{pgfscope}%
\pgfsys@transformshift{2.291653in}{0.698768in}%
\pgfsys@useobject{currentmarker}{}%
\end{pgfscope}%
\begin{pgfscope}%
\pgfsys@transformshift{2.291773in}{0.671036in}%
\pgfsys@useobject{currentmarker}{}%
\end{pgfscope}%
\begin{pgfscope}%
\pgfsys@transformshift{2.291893in}{0.657567in}%
\pgfsys@useobject{currentmarker}{}%
\end{pgfscope}%
\begin{pgfscope}%
\pgfsys@transformshift{2.292014in}{0.654721in}%
\pgfsys@useobject{currentmarker}{}%
\end{pgfscope}%
\begin{pgfscope}%
\pgfsys@transformshift{2.292134in}{0.675216in}%
\pgfsys@useobject{currentmarker}{}%
\end{pgfscope}%
\begin{pgfscope}%
\pgfsys@transformshift{2.292254in}{0.660652in}%
\pgfsys@useobject{currentmarker}{}%
\end{pgfscope}%
\begin{pgfscope}%
\pgfsys@transformshift{2.292374in}{0.650192in}%
\pgfsys@useobject{currentmarker}{}%
\end{pgfscope}%
\begin{pgfscope}%
\pgfsys@transformshift{2.292495in}{0.644563in}%
\pgfsys@useobject{currentmarker}{}%
\end{pgfscope}%
\begin{pgfscope}%
\pgfsys@transformshift{2.292615in}{0.683611in}%
\pgfsys@useobject{currentmarker}{}%
\end{pgfscope}%
\begin{pgfscope}%
\pgfsys@transformshift{2.292735in}{0.681995in}%
\pgfsys@useobject{currentmarker}{}%
\end{pgfscope}%
\begin{pgfscope}%
\pgfsys@transformshift{2.292855in}{0.667117in}%
\pgfsys@useobject{currentmarker}{}%
\end{pgfscope}%
\begin{pgfscope}%
\pgfsys@transformshift{2.292974in}{0.635375in}%
\pgfsys@useobject{currentmarker}{}%
\end{pgfscope}%
\begin{pgfscope}%
\pgfsys@transformshift{2.293094in}{0.627388in}%
\pgfsys@useobject{currentmarker}{}%
\end{pgfscope}%
\begin{pgfscope}%
\pgfsys@transformshift{2.293214in}{0.681910in}%
\pgfsys@useobject{currentmarker}{}%
\end{pgfscope}%
\begin{pgfscope}%
\pgfsys@transformshift{2.293334in}{0.682722in}%
\pgfsys@useobject{currentmarker}{}%
\end{pgfscope}%
\begin{pgfscope}%
\pgfsys@transformshift{2.293453in}{0.647924in}%
\pgfsys@useobject{currentmarker}{}%
\end{pgfscope}%
\begin{pgfscope}%
\pgfsys@transformshift{2.293573in}{0.694397in}%
\pgfsys@useobject{currentmarker}{}%
\end{pgfscope}%
\begin{pgfscope}%
\pgfsys@transformshift{2.293692in}{0.677924in}%
\pgfsys@useobject{currentmarker}{}%
\end{pgfscope}%
\begin{pgfscope}%
\pgfsys@transformshift{2.293812in}{0.647918in}%
\pgfsys@useobject{currentmarker}{}%
\end{pgfscope}%
\begin{pgfscope}%
\pgfsys@transformshift{2.293931in}{0.673262in}%
\pgfsys@useobject{currentmarker}{}%
\end{pgfscope}%
\begin{pgfscope}%
\pgfsys@transformshift{2.294050in}{0.682434in}%
\pgfsys@useobject{currentmarker}{}%
\end{pgfscope}%
\begin{pgfscope}%
\pgfsys@transformshift{2.294169in}{0.692400in}%
\pgfsys@useobject{currentmarker}{}%
\end{pgfscope}%
\begin{pgfscope}%
\pgfsys@transformshift{2.294288in}{0.710708in}%
\pgfsys@useobject{currentmarker}{}%
\end{pgfscope}%
\begin{pgfscope}%
\pgfsys@transformshift{2.294408in}{0.706811in}%
\pgfsys@useobject{currentmarker}{}%
\end{pgfscope}%
\begin{pgfscope}%
\pgfsys@transformshift{2.294527in}{0.672568in}%
\pgfsys@useobject{currentmarker}{}%
\end{pgfscope}%
\begin{pgfscope}%
\pgfsys@transformshift{2.294646in}{0.646445in}%
\pgfsys@useobject{currentmarker}{}%
\end{pgfscope}%
\begin{pgfscope}%
\pgfsys@transformshift{2.294764in}{0.661729in}%
\pgfsys@useobject{currentmarker}{}%
\end{pgfscope}%
\begin{pgfscope}%
\pgfsys@transformshift{2.294883in}{0.648276in}%
\pgfsys@useobject{currentmarker}{}%
\end{pgfscope}%
\begin{pgfscope}%
\pgfsys@transformshift{2.295002in}{0.638861in}%
\pgfsys@useobject{currentmarker}{}%
\end{pgfscope}%
\begin{pgfscope}%
\pgfsys@transformshift{2.295121in}{0.627525in}%
\pgfsys@useobject{currentmarker}{}%
\end{pgfscope}%
\begin{pgfscope}%
\pgfsys@transformshift{2.295239in}{0.680816in}%
\pgfsys@useobject{currentmarker}{}%
\end{pgfscope}%
\begin{pgfscope}%
\pgfsys@transformshift{2.295358in}{0.709967in}%
\pgfsys@useobject{currentmarker}{}%
\end{pgfscope}%
\begin{pgfscope}%
\pgfsys@transformshift{2.295476in}{0.700293in}%
\pgfsys@useobject{currentmarker}{}%
\end{pgfscope}%
\begin{pgfscope}%
\pgfsys@transformshift{2.295595in}{0.656387in}%
\pgfsys@useobject{currentmarker}{}%
\end{pgfscope}%
\begin{pgfscope}%
\pgfsys@transformshift{2.295713in}{0.662081in}%
\pgfsys@useobject{currentmarker}{}%
\end{pgfscope}%
\begin{pgfscope}%
\pgfsys@transformshift{2.295832in}{0.661520in}%
\pgfsys@useobject{currentmarker}{}%
\end{pgfscope}%
\begin{pgfscope}%
\pgfsys@transformshift{2.295950in}{0.680124in}%
\pgfsys@useobject{currentmarker}{}%
\end{pgfscope}%
\begin{pgfscope}%
\pgfsys@transformshift{2.296068in}{0.656054in}%
\pgfsys@useobject{currentmarker}{}%
\end{pgfscope}%
\begin{pgfscope}%
\pgfsys@transformshift{2.296186in}{0.652909in}%
\pgfsys@useobject{currentmarker}{}%
\end{pgfscope}%
\begin{pgfscope}%
\pgfsys@transformshift{2.296304in}{0.641789in}%
\pgfsys@useobject{currentmarker}{}%
\end{pgfscope}%
\begin{pgfscope}%
\pgfsys@transformshift{2.296422in}{0.603751in}%
\pgfsys@useobject{currentmarker}{}%
\end{pgfscope}%
\begin{pgfscope}%
\pgfsys@transformshift{2.296540in}{0.631273in}%
\pgfsys@useobject{currentmarker}{}%
\end{pgfscope}%
\begin{pgfscope}%
\pgfsys@transformshift{2.296658in}{0.691810in}%
\pgfsys@useobject{currentmarker}{}%
\end{pgfscope}%
\begin{pgfscope}%
\pgfsys@transformshift{2.296776in}{0.698668in}%
\pgfsys@useobject{currentmarker}{}%
\end{pgfscope}%
\begin{pgfscope}%
\pgfsys@transformshift{2.296894in}{0.674084in}%
\pgfsys@useobject{currentmarker}{}%
\end{pgfscope}%
\begin{pgfscope}%
\pgfsys@transformshift{2.297012in}{0.665989in}%
\pgfsys@useobject{currentmarker}{}%
\end{pgfscope}%
\begin{pgfscope}%
\pgfsys@transformshift{2.297129in}{0.609636in}%
\pgfsys@useobject{currentmarker}{}%
\end{pgfscope}%
\begin{pgfscope}%
\pgfsys@transformshift{2.297247in}{0.638547in}%
\pgfsys@useobject{currentmarker}{}%
\end{pgfscope}%
\begin{pgfscope}%
\pgfsys@transformshift{2.297364in}{0.690769in}%
\pgfsys@useobject{currentmarker}{}%
\end{pgfscope}%
\begin{pgfscope}%
\pgfsys@transformshift{2.297482in}{0.684271in}%
\pgfsys@useobject{currentmarker}{}%
\end{pgfscope}%
\begin{pgfscope}%
\pgfsys@transformshift{2.297599in}{0.648709in}%
\pgfsys@useobject{currentmarker}{}%
\end{pgfscope}%
\begin{pgfscope}%
\pgfsys@transformshift{2.297717in}{0.639670in}%
\pgfsys@useobject{currentmarker}{}%
\end{pgfscope}%
\begin{pgfscope}%
\pgfsys@transformshift{2.297834in}{0.648369in}%
\pgfsys@useobject{currentmarker}{}%
\end{pgfscope}%
\begin{pgfscope}%
\pgfsys@transformshift{2.297951in}{0.663953in}%
\pgfsys@useobject{currentmarker}{}%
\end{pgfscope}%
\begin{pgfscope}%
\pgfsys@transformshift{2.298068in}{0.654222in}%
\pgfsys@useobject{currentmarker}{}%
\end{pgfscope}%
\begin{pgfscope}%
\pgfsys@transformshift{2.298185in}{0.660754in}%
\pgfsys@useobject{currentmarker}{}%
\end{pgfscope}%
\begin{pgfscope}%
\pgfsys@transformshift{2.298302in}{0.679540in}%
\pgfsys@useobject{currentmarker}{}%
\end{pgfscope}%
\begin{pgfscope}%
\pgfsys@transformshift{2.298419in}{0.623709in}%
\pgfsys@useobject{currentmarker}{}%
\end{pgfscope}%
\begin{pgfscope}%
\pgfsys@transformshift{2.298536in}{0.636966in}%
\pgfsys@useobject{currentmarker}{}%
\end{pgfscope}%
\begin{pgfscope}%
\pgfsys@transformshift{2.298653in}{0.630659in}%
\pgfsys@useobject{currentmarker}{}%
\end{pgfscope}%
\begin{pgfscope}%
\pgfsys@transformshift{2.298770in}{0.626511in}%
\pgfsys@useobject{currentmarker}{}%
\end{pgfscope}%
\begin{pgfscope}%
\pgfsys@transformshift{2.298887in}{0.677257in}%
\pgfsys@useobject{currentmarker}{}%
\end{pgfscope}%
\begin{pgfscope}%
\pgfsys@transformshift{2.299003in}{0.660927in}%
\pgfsys@useobject{currentmarker}{}%
\end{pgfscope}%
\begin{pgfscope}%
\pgfsys@transformshift{2.299120in}{0.658376in}%
\pgfsys@useobject{currentmarker}{}%
\end{pgfscope}%
\begin{pgfscope}%
\pgfsys@transformshift{2.299236in}{0.690972in}%
\pgfsys@useobject{currentmarker}{}%
\end{pgfscope}%
\begin{pgfscope}%
\pgfsys@transformshift{2.299353in}{0.692994in}%
\pgfsys@useobject{currentmarker}{}%
\end{pgfscope}%
\begin{pgfscope}%
\pgfsys@transformshift{2.299469in}{0.692625in}%
\pgfsys@useobject{currentmarker}{}%
\end{pgfscope}%
\begin{pgfscope}%
\pgfsys@transformshift{2.299586in}{0.701255in}%
\pgfsys@useobject{currentmarker}{}%
\end{pgfscope}%
\begin{pgfscope}%
\pgfsys@transformshift{2.299702in}{0.679299in}%
\pgfsys@useobject{currentmarker}{}%
\end{pgfscope}%
\begin{pgfscope}%
\pgfsys@transformshift{2.299818in}{0.674972in}%
\pgfsys@useobject{currentmarker}{}%
\end{pgfscope}%
\begin{pgfscope}%
\pgfsys@transformshift{2.299934in}{0.639771in}%
\pgfsys@useobject{currentmarker}{}%
\end{pgfscope}%
\begin{pgfscope}%
\pgfsys@transformshift{2.300051in}{0.687561in}%
\pgfsys@useobject{currentmarker}{}%
\end{pgfscope}%
\begin{pgfscope}%
\pgfsys@transformshift{2.300167in}{0.683616in}%
\pgfsys@useobject{currentmarker}{}%
\end{pgfscope}%
\begin{pgfscope}%
\pgfsys@transformshift{2.300283in}{0.642821in}%
\pgfsys@useobject{currentmarker}{}%
\end{pgfscope}%
\begin{pgfscope}%
\pgfsys@transformshift{2.300399in}{0.669478in}%
\pgfsys@useobject{currentmarker}{}%
\end{pgfscope}%
\begin{pgfscope}%
\pgfsys@transformshift{2.300515in}{0.668758in}%
\pgfsys@useobject{currentmarker}{}%
\end{pgfscope}%
\begin{pgfscope}%
\pgfsys@transformshift{2.300630in}{0.589486in}%
\pgfsys@useobject{currentmarker}{}%
\end{pgfscope}%
\begin{pgfscope}%
\pgfsys@transformshift{2.300746in}{0.648213in}%
\pgfsys@useobject{currentmarker}{}%
\end{pgfscope}%
\begin{pgfscope}%
\pgfsys@transformshift{2.300862in}{0.651395in}%
\pgfsys@useobject{currentmarker}{}%
\end{pgfscope}%
\begin{pgfscope}%
\pgfsys@transformshift{2.300977in}{0.631733in}%
\pgfsys@useobject{currentmarker}{}%
\end{pgfscope}%
\begin{pgfscope}%
\pgfsys@transformshift{2.301093in}{0.668036in}%
\pgfsys@useobject{currentmarker}{}%
\end{pgfscope}%
\begin{pgfscope}%
\pgfsys@transformshift{2.301209in}{0.669826in}%
\pgfsys@useobject{currentmarker}{}%
\end{pgfscope}%
\begin{pgfscope}%
\pgfsys@transformshift{2.301324in}{0.682665in}%
\pgfsys@useobject{currentmarker}{}%
\end{pgfscope}%
\begin{pgfscope}%
\pgfsys@transformshift{2.301439in}{0.629785in}%
\pgfsys@useobject{currentmarker}{}%
\end{pgfscope}%
\begin{pgfscope}%
\pgfsys@transformshift{2.301555in}{0.690100in}%
\pgfsys@useobject{currentmarker}{}%
\end{pgfscope}%
\begin{pgfscope}%
\pgfsys@transformshift{2.301670in}{0.704950in}%
\pgfsys@useobject{currentmarker}{}%
\end{pgfscope}%
\begin{pgfscope}%
\pgfsys@transformshift{2.301785in}{0.656226in}%
\pgfsys@useobject{currentmarker}{}%
\end{pgfscope}%
\begin{pgfscope}%
\pgfsys@transformshift{2.301901in}{0.635383in}%
\pgfsys@useobject{currentmarker}{}%
\end{pgfscope}%
\begin{pgfscope}%
\pgfsys@transformshift{2.302016in}{0.590936in}%
\pgfsys@useobject{currentmarker}{}%
\end{pgfscope}%
\begin{pgfscope}%
\pgfsys@transformshift{2.302131in}{0.626375in}%
\pgfsys@useobject{currentmarker}{}%
\end{pgfscope}%
\begin{pgfscope}%
\pgfsys@transformshift{2.302246in}{0.681330in}%
\pgfsys@useobject{currentmarker}{}%
\end{pgfscope}%
\begin{pgfscope}%
\pgfsys@transformshift{2.302361in}{0.670729in}%
\pgfsys@useobject{currentmarker}{}%
\end{pgfscope}%
\begin{pgfscope}%
\pgfsys@transformshift{2.302476in}{0.722981in}%
\pgfsys@useobject{currentmarker}{}%
\end{pgfscope}%
\begin{pgfscope}%
\pgfsys@transformshift{2.302590in}{0.702981in}%
\pgfsys@useobject{currentmarker}{}%
\end{pgfscope}%
\begin{pgfscope}%
\pgfsys@transformshift{2.302705in}{0.655000in}%
\pgfsys@useobject{currentmarker}{}%
\end{pgfscope}%
\begin{pgfscope}%
\pgfsys@transformshift{2.302820in}{0.680883in}%
\pgfsys@useobject{currentmarker}{}%
\end{pgfscope}%
\begin{pgfscope}%
\pgfsys@transformshift{2.302934in}{0.670715in}%
\pgfsys@useobject{currentmarker}{}%
\end{pgfscope}%
\begin{pgfscope}%
\pgfsys@transformshift{2.303049in}{0.657788in}%
\pgfsys@useobject{currentmarker}{}%
\end{pgfscope}%
\begin{pgfscope}%
\pgfsys@transformshift{2.303164in}{0.674681in}%
\pgfsys@useobject{currentmarker}{}%
\end{pgfscope}%
\begin{pgfscope}%
\pgfsys@transformshift{2.303278in}{0.653235in}%
\pgfsys@useobject{currentmarker}{}%
\end{pgfscope}%
\begin{pgfscope}%
\pgfsys@transformshift{2.303392in}{0.673450in}%
\pgfsys@useobject{currentmarker}{}%
\end{pgfscope}%
\begin{pgfscope}%
\pgfsys@transformshift{2.303507in}{0.647342in}%
\pgfsys@useobject{currentmarker}{}%
\end{pgfscope}%
\begin{pgfscope}%
\pgfsys@transformshift{2.303621in}{0.672778in}%
\pgfsys@useobject{currentmarker}{}%
\end{pgfscope}%
\begin{pgfscope}%
\pgfsys@transformshift{2.303735in}{0.663151in}%
\pgfsys@useobject{currentmarker}{}%
\end{pgfscope}%
\begin{pgfscope}%
\pgfsys@transformshift{2.303850in}{0.660736in}%
\pgfsys@useobject{currentmarker}{}%
\end{pgfscope}%
\begin{pgfscope}%
\pgfsys@transformshift{2.303964in}{0.661568in}%
\pgfsys@useobject{currentmarker}{}%
\end{pgfscope}%
\begin{pgfscope}%
\pgfsys@transformshift{2.304078in}{0.681546in}%
\pgfsys@useobject{currentmarker}{}%
\end{pgfscope}%
\begin{pgfscope}%
\pgfsys@transformshift{2.304192in}{0.662191in}%
\pgfsys@useobject{currentmarker}{}%
\end{pgfscope}%
\begin{pgfscope}%
\pgfsys@transformshift{2.304306in}{0.663871in}%
\pgfsys@useobject{currentmarker}{}%
\end{pgfscope}%
\begin{pgfscope}%
\pgfsys@transformshift{2.304420in}{0.636050in}%
\pgfsys@useobject{currentmarker}{}%
\end{pgfscope}%
\begin{pgfscope}%
\pgfsys@transformshift{2.304533in}{0.598852in}%
\pgfsys@useobject{currentmarker}{}%
\end{pgfscope}%
\begin{pgfscope}%
\pgfsys@transformshift{2.304647in}{0.626567in}%
\pgfsys@useobject{currentmarker}{}%
\end{pgfscope}%
\begin{pgfscope}%
\pgfsys@transformshift{2.304761in}{0.641927in}%
\pgfsys@useobject{currentmarker}{}%
\end{pgfscope}%
\begin{pgfscope}%
\pgfsys@transformshift{2.304875in}{0.652052in}%
\pgfsys@useobject{currentmarker}{}%
\end{pgfscope}%
\begin{pgfscope}%
\pgfsys@transformshift{2.304988in}{0.664075in}%
\pgfsys@useobject{currentmarker}{}%
\end{pgfscope}%
\begin{pgfscope}%
\pgfsys@transformshift{2.305102in}{0.685435in}%
\pgfsys@useobject{currentmarker}{}%
\end{pgfscope}%
\begin{pgfscope}%
\pgfsys@transformshift{2.305215in}{0.682574in}%
\pgfsys@useobject{currentmarker}{}%
\end{pgfscope}%
\begin{pgfscope}%
\pgfsys@transformshift{2.305329in}{0.634659in}%
\pgfsys@useobject{currentmarker}{}%
\end{pgfscope}%
\begin{pgfscope}%
\pgfsys@transformshift{2.305442in}{0.635693in}%
\pgfsys@useobject{currentmarker}{}%
\end{pgfscope}%
\begin{pgfscope}%
\pgfsys@transformshift{2.305555in}{0.665719in}%
\pgfsys@useobject{currentmarker}{}%
\end{pgfscope}%
\begin{pgfscope}%
\pgfsys@transformshift{2.305669in}{0.701529in}%
\pgfsys@useobject{currentmarker}{}%
\end{pgfscope}%
\begin{pgfscope}%
\pgfsys@transformshift{2.305782in}{0.661021in}%
\pgfsys@useobject{currentmarker}{}%
\end{pgfscope}%
\begin{pgfscope}%
\pgfsys@transformshift{2.305895in}{0.699278in}%
\pgfsys@useobject{currentmarker}{}%
\end{pgfscope}%
\begin{pgfscope}%
\pgfsys@transformshift{2.306008in}{0.648960in}%
\pgfsys@useobject{currentmarker}{}%
\end{pgfscope}%
\begin{pgfscope}%
\pgfsys@transformshift{2.306121in}{0.650623in}%
\pgfsys@useobject{currentmarker}{}%
\end{pgfscope}%
\begin{pgfscope}%
\pgfsys@transformshift{2.306234in}{0.651458in}%
\pgfsys@useobject{currentmarker}{}%
\end{pgfscope}%
\begin{pgfscope}%
\pgfsys@transformshift{2.306347in}{0.613825in}%
\pgfsys@useobject{currentmarker}{}%
\end{pgfscope}%
\begin{pgfscope}%
\pgfsys@transformshift{2.306460in}{0.586209in}%
\pgfsys@useobject{currentmarker}{}%
\end{pgfscope}%
\begin{pgfscope}%
\pgfsys@transformshift{2.306573in}{0.670083in}%
\pgfsys@useobject{currentmarker}{}%
\end{pgfscope}%
\begin{pgfscope}%
\pgfsys@transformshift{2.306685in}{0.672948in}%
\pgfsys@useobject{currentmarker}{}%
\end{pgfscope}%
\begin{pgfscope}%
\pgfsys@transformshift{2.306798in}{0.645266in}%
\pgfsys@useobject{currentmarker}{}%
\end{pgfscope}%
\begin{pgfscope}%
\pgfsys@transformshift{2.306911in}{0.642729in}%
\pgfsys@useobject{currentmarker}{}%
\end{pgfscope}%
\begin{pgfscope}%
\pgfsys@transformshift{2.307023in}{0.668373in}%
\pgfsys@useobject{currentmarker}{}%
\end{pgfscope}%
\begin{pgfscope}%
\pgfsys@transformshift{2.307136in}{0.699892in}%
\pgfsys@useobject{currentmarker}{}%
\end{pgfscope}%
\begin{pgfscope}%
\pgfsys@transformshift{2.307248in}{0.719898in}%
\pgfsys@useobject{currentmarker}{}%
\end{pgfscope}%
\begin{pgfscope}%
\pgfsys@transformshift{2.307361in}{0.689919in}%
\pgfsys@useobject{currentmarker}{}%
\end{pgfscope}%
\begin{pgfscope}%
\pgfsys@transformshift{2.307473in}{0.687589in}%
\pgfsys@useobject{currentmarker}{}%
\end{pgfscope}%
\begin{pgfscope}%
\pgfsys@transformshift{2.307585in}{0.683352in}%
\pgfsys@useobject{currentmarker}{}%
\end{pgfscope}%
\begin{pgfscope}%
\pgfsys@transformshift{2.307698in}{0.662579in}%
\pgfsys@useobject{currentmarker}{}%
\end{pgfscope}%
\begin{pgfscope}%
\pgfsys@transformshift{2.307810in}{0.678877in}%
\pgfsys@useobject{currentmarker}{}%
\end{pgfscope}%
\begin{pgfscope}%
\pgfsys@transformshift{2.307922in}{0.610988in}%
\pgfsys@useobject{currentmarker}{}%
\end{pgfscope}%
\begin{pgfscope}%
\pgfsys@transformshift{2.308034in}{0.633629in}%
\pgfsys@useobject{currentmarker}{}%
\end{pgfscope}%
\begin{pgfscope}%
\pgfsys@transformshift{2.308146in}{0.634300in}%
\pgfsys@useobject{currentmarker}{}%
\end{pgfscope}%
\begin{pgfscope}%
\pgfsys@transformshift{2.308258in}{0.655192in}%
\pgfsys@useobject{currentmarker}{}%
\end{pgfscope}%
\begin{pgfscope}%
\pgfsys@transformshift{2.308370in}{0.666752in}%
\pgfsys@useobject{currentmarker}{}%
\end{pgfscope}%
\begin{pgfscope}%
\pgfsys@transformshift{2.308482in}{0.648257in}%
\pgfsys@useobject{currentmarker}{}%
\end{pgfscope}%
\begin{pgfscope}%
\pgfsys@transformshift{2.308594in}{0.666169in}%
\pgfsys@useobject{currentmarker}{}%
\end{pgfscope}%
\begin{pgfscope}%
\pgfsys@transformshift{2.308705in}{0.673702in}%
\pgfsys@useobject{currentmarker}{}%
\end{pgfscope}%
\begin{pgfscope}%
\pgfsys@transformshift{2.308817in}{0.616075in}%
\pgfsys@useobject{currentmarker}{}%
\end{pgfscope}%
\begin{pgfscope}%
\pgfsys@transformshift{2.308929in}{0.711626in}%
\pgfsys@useobject{currentmarker}{}%
\end{pgfscope}%
\begin{pgfscope}%
\pgfsys@transformshift{2.309040in}{0.692222in}%
\pgfsys@useobject{currentmarker}{}%
\end{pgfscope}%
\begin{pgfscope}%
\pgfsys@transformshift{2.309152in}{0.689041in}%
\pgfsys@useobject{currentmarker}{}%
\end{pgfscope}%
\begin{pgfscope}%
\pgfsys@transformshift{2.309263in}{0.700235in}%
\pgfsys@useobject{currentmarker}{}%
\end{pgfscope}%
\begin{pgfscope}%
\pgfsys@transformshift{2.309375in}{0.649461in}%
\pgfsys@useobject{currentmarker}{}%
\end{pgfscope}%
\begin{pgfscope}%
\pgfsys@transformshift{2.309486in}{0.677203in}%
\pgfsys@useobject{currentmarker}{}%
\end{pgfscope}%
\begin{pgfscope}%
\pgfsys@transformshift{2.309597in}{0.691380in}%
\pgfsys@useobject{currentmarker}{}%
\end{pgfscope}%
\begin{pgfscope}%
\pgfsys@transformshift{2.309708in}{0.638386in}%
\pgfsys@useobject{currentmarker}{}%
\end{pgfscope}%
\begin{pgfscope}%
\pgfsys@transformshift{2.309820in}{0.626961in}%
\pgfsys@useobject{currentmarker}{}%
\end{pgfscope}%
\begin{pgfscope}%
\pgfsys@transformshift{2.309931in}{0.677970in}%
\pgfsys@useobject{currentmarker}{}%
\end{pgfscope}%
\begin{pgfscope}%
\pgfsys@transformshift{2.310042in}{0.633506in}%
\pgfsys@useobject{currentmarker}{}%
\end{pgfscope}%
\begin{pgfscope}%
\pgfsys@transformshift{2.310153in}{0.605629in}%
\pgfsys@useobject{currentmarker}{}%
\end{pgfscope}%
\begin{pgfscope}%
\pgfsys@transformshift{2.310264in}{0.670713in}%
\pgfsys@useobject{currentmarker}{}%
\end{pgfscope}%
\begin{pgfscope}%
\pgfsys@transformshift{2.310375in}{0.670991in}%
\pgfsys@useobject{currentmarker}{}%
\end{pgfscope}%
\begin{pgfscope}%
\pgfsys@transformshift{2.310486in}{0.680648in}%
\pgfsys@useobject{currentmarker}{}%
\end{pgfscope}%
\begin{pgfscope}%
\pgfsys@transformshift{2.310596in}{0.602741in}%
\pgfsys@useobject{currentmarker}{}%
\end{pgfscope}%
\begin{pgfscope}%
\pgfsys@transformshift{2.310707in}{0.631625in}%
\pgfsys@useobject{currentmarker}{}%
\end{pgfscope}%
\begin{pgfscope}%
\pgfsys@transformshift{2.310818in}{0.629701in}%
\pgfsys@useobject{currentmarker}{}%
\end{pgfscope}%
\begin{pgfscope}%
\pgfsys@transformshift{2.310928in}{0.643290in}%
\pgfsys@useobject{currentmarker}{}%
\end{pgfscope}%
\begin{pgfscope}%
\pgfsys@transformshift{2.311039in}{0.642079in}%
\pgfsys@useobject{currentmarker}{}%
\end{pgfscope}%
\begin{pgfscope}%
\pgfsys@transformshift{2.311150in}{0.652611in}%
\pgfsys@useobject{currentmarker}{}%
\end{pgfscope}%
\begin{pgfscope}%
\pgfsys@transformshift{2.311260in}{0.602453in}%
\pgfsys@useobject{currentmarker}{}%
\end{pgfscope}%
\begin{pgfscope}%
\pgfsys@transformshift{2.311370in}{0.654620in}%
\pgfsys@useobject{currentmarker}{}%
\end{pgfscope}%
\begin{pgfscope}%
\pgfsys@transformshift{2.311481in}{0.641068in}%
\pgfsys@useobject{currentmarker}{}%
\end{pgfscope}%
\begin{pgfscope}%
\pgfsys@transformshift{2.311591in}{0.647847in}%
\pgfsys@useobject{currentmarker}{}%
\end{pgfscope}%
\begin{pgfscope}%
\pgfsys@transformshift{2.311701in}{0.647023in}%
\pgfsys@useobject{currentmarker}{}%
\end{pgfscope}%
\begin{pgfscope}%
\pgfsys@transformshift{2.311812in}{0.683638in}%
\pgfsys@useobject{currentmarker}{}%
\end{pgfscope}%
\begin{pgfscope}%
\pgfsys@transformshift{2.311922in}{0.675673in}%
\pgfsys@useobject{currentmarker}{}%
\end{pgfscope}%
\begin{pgfscope}%
\pgfsys@transformshift{2.312032in}{0.677050in}%
\pgfsys@useobject{currentmarker}{}%
\end{pgfscope}%
\begin{pgfscope}%
\pgfsys@transformshift{2.312142in}{0.715185in}%
\pgfsys@useobject{currentmarker}{}%
\end{pgfscope}%
\begin{pgfscope}%
\pgfsys@transformshift{2.312252in}{0.704471in}%
\pgfsys@useobject{currentmarker}{}%
\end{pgfscope}%
\begin{pgfscope}%
\pgfsys@transformshift{2.312362in}{0.719565in}%
\pgfsys@useobject{currentmarker}{}%
\end{pgfscope}%
\begin{pgfscope}%
\pgfsys@transformshift{2.312472in}{0.703206in}%
\pgfsys@useobject{currentmarker}{}%
\end{pgfscope}%
\begin{pgfscope}%
\pgfsys@transformshift{2.312582in}{0.667330in}%
\pgfsys@useobject{currentmarker}{}%
\end{pgfscope}%
\begin{pgfscope}%
\pgfsys@transformshift{2.312691in}{0.635971in}%
\pgfsys@useobject{currentmarker}{}%
\end{pgfscope}%
\end{pgfscope}%
\begin{pgfscope}%
\pgfsetrectcap%
\pgfsetmiterjoin%
\pgfsetlinewidth{0.803000pt}%
\definecolor{currentstroke}{rgb}{0.000000,0.000000,0.000000}%
\pgfsetstrokecolor{currentstroke}%
\pgfsetdash{}{0pt}%
\pgfpathmoveto{\pgfqpoint{0.514278in}{0.417642in}}%
\pgfpathlineto{\pgfqpoint{0.514278in}{1.788330in}}%
\pgfusepath{stroke}%
\end{pgfscope}%
\begin{pgfscope}%
\pgfsetrectcap%
\pgfsetmiterjoin%
\pgfsetlinewidth{0.803000pt}%
\definecolor{currentstroke}{rgb}{0.000000,0.000000,0.000000}%
\pgfsetstrokecolor{currentstroke}%
\pgfsetdash{}{0pt}%
\pgfpathmoveto{\pgfqpoint{2.398330in}{0.417642in}}%
\pgfpathlineto{\pgfqpoint{2.398330in}{1.788330in}}%
\pgfusepath{stroke}%
\end{pgfscope}%
\begin{pgfscope}%
\pgfsetrectcap%
\pgfsetmiterjoin%
\pgfsetlinewidth{0.803000pt}%
\definecolor{currentstroke}{rgb}{0.000000,0.000000,0.000000}%
\pgfsetstrokecolor{currentstroke}%
\pgfsetdash{}{0pt}%
\pgfpathmoveto{\pgfqpoint{0.514278in}{0.417642in}}%
\pgfpathlineto{\pgfqpoint{2.398330in}{0.417642in}}%
\pgfusepath{stroke}%
\end{pgfscope}%
\begin{pgfscope}%
\pgfsetrectcap%
\pgfsetmiterjoin%
\pgfsetlinewidth{0.803000pt}%
\definecolor{currentstroke}{rgb}{0.000000,0.000000,0.000000}%
\pgfsetstrokecolor{currentstroke}%
\pgfsetdash{}{0pt}%
\pgfpathmoveto{\pgfqpoint{0.514278in}{1.788330in}}%
\pgfpathlineto{\pgfqpoint{2.398330in}{1.788330in}}%
\pgfusepath{stroke}%
\end{pgfscope}%
\begin{pgfscope}%
\pgfsetbuttcap%
\pgfsetmiterjoin%
\definecolor{currentfill}{rgb}{1.000000,1.000000,1.000000}%
\pgfsetfillcolor{currentfill}%
\pgfsetfillopacity{0.800000}%
\pgfsetlinewidth{1.003750pt}%
\definecolor{currentstroke}{rgb}{0.800000,0.800000,0.800000}%
\pgfsetstrokecolor{currentstroke}%
\pgfsetstrokeopacity{0.800000}%
\pgfsetdash{}{0pt}%
\pgfpathmoveto{\pgfqpoint{1.551772in}{1.517019in}}%
\pgfpathlineto{\pgfqpoint{2.320552in}{1.517019in}}%
\pgfpathquadraticcurveto{\pgfqpoint{2.342774in}{1.517019in}}{\pgfqpoint{2.342774in}{1.539241in}}%
\pgfpathlineto{\pgfqpoint{2.342774in}{1.710552in}}%
\pgfpathquadraticcurveto{\pgfqpoint{2.342774in}{1.732774in}}{\pgfqpoint{2.320552in}{1.732774in}}%
\pgfpathlineto{\pgfqpoint{1.551772in}{1.732774in}}%
\pgfpathquadraticcurveto{\pgfqpoint{1.529549in}{1.732774in}}{\pgfqpoint{1.529549in}{1.710552in}}%
\pgfpathlineto{\pgfqpoint{1.529549in}{1.539241in}}%
\pgfpathquadraticcurveto{\pgfqpoint{1.529549in}{1.517019in}}{\pgfqpoint{1.551772in}{1.517019in}}%
\pgfpathlineto{\pgfqpoint{1.551772in}{1.517019in}}%
\pgfpathclose%
\pgfusepath{stroke,fill}%
\end{pgfscope}%
\begin{pgfscope}%
\pgfsetbuttcap%
\pgfsetroundjoin%
\pgfsetlinewidth{1.505625pt}%
\definecolor{currentstroke}{rgb}{0.835294,0.368627,0.000000}%
\pgfsetstrokecolor{currentstroke}%
\pgfsetdash{{5.550000pt}{2.400000pt}}{0.000000pt}%
\pgfpathmoveto{\pgfqpoint{1.573994in}{1.627358in}}%
\pgfpathlineto{\pgfqpoint{1.685105in}{1.627358in}}%
\pgfpathlineto{\pgfqpoint{1.796216in}{1.627358in}}%
\pgfusepath{stroke}%
\end{pgfscope}%
\begin{pgfscope}%
\definecolor{textcolor}{rgb}{0.000000,0.000000,0.000000}%
\pgfsetstrokecolor{textcolor}%
\pgfsetfillcolor{textcolor}%
\pgftext[x=1.885105in,y=1.588469in,left,base]{\color{textcolor}\rmfamily\fontsize{8.000000}{9.600000}\selectfont \(\displaystyle h_{-2}f^{-2}\)}%
\end{pgfscope}%
\end{pgfpicture}%
\makeatother%
\endgroup%

        } % scalebox
        \caption{Power spectral density}
        \label{fig:random_walk_psd}
    \end{subfigure}
    \begin{subfigure}{0.32\linewidth}
        \centering
        \scalebox{0.75}{%
            \input{images/random_walk_noise_adev.pgf}
        } % scalebox
        \caption{Allan deviation}
        \label{fig:random_walk_adev}
    \end{subfigure}
    \caption{Different representations of random walk noise.}
    \label{fig:random_walk_noise_simulated}
\end{figure}


\clearpage
\subsubsection{Drift}
Finally, the last feature of the Allan deviation plot, that needs to be discussed is drift. Drift happens at very long time scales and descriped a linear dependence of measurand on the time. This is also part of the ageing effect. \citeauthor{adev_drift} discussed the effect of drift \cite{adev_drift} on the Allan variance and found the following relationship:
\begin{align}
    \sigma_A^2(\tau) = \frac{D^2}{2} \tau^2
\end{align}
with slope of the drift $D$.

\begin{figure}[ht]
    \centering
    \begin{subfigure}{0.32\linewidth}
        \centering
        \scalebox{0.75}{%
            \input{images/drift_time.pgf}
        } % scalebox
        \caption{Time domain}
        \label{fig:drift_time}
    \end{subfigure}
    \begin{subfigure}{0.32\linewidth}
        \centering
        \scalebox{0.75}{%
            %% Creator: Matplotlib, PGF backend
%%
%% To include the figure in your LaTeX document, write
%%   \input{<filename>.pgf}
%%
%% Make sure the required packages are loaded in your preamble
%%   \usepackage{pgf}
%%
%% Also ensure that all the required font packages are loaded; for instance,
%% the lmodern package is sometimes necessary when using math font.
%%   \usepackage{lmodern}
%%
%% Figures using additional raster images can only be included by \input if
%% they are in the same directory as the main LaTeX file. For loading figures
%% from other directories you can use the `import` package
%%   \usepackage{import}
%%
%% and then include the figures with
%%   \import{<path to file>}{<filename>.pgf}
%%
%% Matplotlib used the following preamble
%%   \def\mathdefault#1{#1}
%%   \everymath=\expandafter{\the\everymath\displaystyle}
%%   \usepackage{siunitx}
%%   \sisetup{per-mode = symbol}%
%%   \ifdefined\pdftexversion\else  % non-pdftex case.
%%     \usepackage{fontspec}
%%   \fi
%%   \makeatletter\@ifpackageloaded{underscore}{}{\usepackage[strings]{underscore}}\makeatother
%%
\begingroup%
\makeatletter%
\begin{pgfpicture}%
\pgfpathrectangle{\pgfpointorigin}{\pgfqpoint{2.440945in}{1.830709in}}%
\pgfusepath{use as bounding box, clip}%
\begin{pgfscope}%
\pgfsetbuttcap%
\pgfsetmiterjoin%
\definecolor{currentfill}{rgb}{1.000000,1.000000,1.000000}%
\pgfsetfillcolor{currentfill}%
\pgfsetlinewidth{0.000000pt}%
\definecolor{currentstroke}{rgb}{1.000000,1.000000,1.000000}%
\pgfsetstrokecolor{currentstroke}%
\pgfsetdash{}{0pt}%
\pgfpathmoveto{\pgfqpoint{0.000000in}{0.000000in}}%
\pgfpathlineto{\pgfqpoint{2.440945in}{0.000000in}}%
\pgfpathlineto{\pgfqpoint{2.440945in}{1.830709in}}%
\pgfpathlineto{\pgfqpoint{0.000000in}{1.830709in}}%
\pgfpathlineto{\pgfqpoint{0.000000in}{0.000000in}}%
\pgfpathclose%
\pgfusepath{fill}%
\end{pgfscope}%
\begin{pgfscope}%
\pgfsetbuttcap%
\pgfsetmiterjoin%
\definecolor{currentfill}{rgb}{1.000000,1.000000,1.000000}%
\pgfsetfillcolor{currentfill}%
\pgfsetlinewidth{0.000000pt}%
\definecolor{currentstroke}{rgb}{0.000000,0.000000,0.000000}%
\pgfsetstrokecolor{currentstroke}%
\pgfsetstrokeopacity{0.000000}%
\pgfsetdash{}{0pt}%
\pgfpathmoveto{\pgfqpoint{0.589510in}{0.417642in}}%
\pgfpathlineto{\pgfqpoint{2.399275in}{0.417642in}}%
\pgfpathlineto{\pgfqpoint{2.399275in}{1.789039in}}%
\pgfpathlineto{\pgfqpoint{0.589510in}{1.789039in}}%
\pgfpathlineto{\pgfqpoint{0.589510in}{0.417642in}}%
\pgfpathclose%
\pgfusepath{fill}%
\end{pgfscope}%
\begin{pgfscope}%
\pgfpathrectangle{\pgfqpoint{0.589510in}{0.417642in}}{\pgfqpoint{1.809765in}{1.371397in}}%
\pgfusepath{clip}%
\pgfsetrectcap%
\pgfsetroundjoin%
\pgfsetlinewidth{0.803000pt}%
\definecolor{currentstroke}{rgb}{0.450000,0.450000,0.450000}%
\pgfsetstrokecolor{currentstroke}%
\pgfsetdash{}{0pt}%
\pgfpathmoveto{\pgfqpoint{0.671772in}{0.417642in}}%
\pgfpathlineto{\pgfqpoint{0.671772in}{1.789039in}}%
\pgfusepath{stroke}%
\end{pgfscope}%
\begin{pgfscope}%
\pgfsetbuttcap%
\pgfsetroundjoin%
\definecolor{currentfill}{rgb}{0.000000,0.000000,0.000000}%
\pgfsetfillcolor{currentfill}%
\pgfsetlinewidth{0.803000pt}%
\definecolor{currentstroke}{rgb}{0.000000,0.000000,0.000000}%
\pgfsetstrokecolor{currentstroke}%
\pgfsetdash{}{0pt}%
\pgfsys@defobject{currentmarker}{\pgfqpoint{0.000000in}{-0.048611in}}{\pgfqpoint{0.000000in}{0.000000in}}{%
\pgfpathmoveto{\pgfqpoint{0.000000in}{0.000000in}}%
\pgfpathlineto{\pgfqpoint{0.000000in}{-0.048611in}}%
\pgfusepath{stroke,fill}%
}%
\begin{pgfscope}%
\pgfsys@transformshift{0.671772in}{0.417642in}%
\pgfsys@useobject{currentmarker}{}%
\end{pgfscope}%
\end{pgfscope}%
\begin{pgfscope}%
\definecolor{textcolor}{rgb}{0.000000,0.000000,0.000000}%
\pgfsetstrokecolor{textcolor}%
\pgfsetfillcolor{textcolor}%
\pgftext[x=0.671772in,y=0.320420in,,top]{\color{textcolor}{\rmfamily\fontsize{8.000000}{9.600000}\selectfont\catcode`\^=\active\def^{\ifmmode\sp\else\^{}\fi}\catcode`\%=\active\def%{\%}$\mathdefault{10^{0}}$}}%
\end{pgfscope}%
\begin{pgfscope}%
\pgfpathrectangle{\pgfqpoint{0.589510in}{0.417642in}}{\pgfqpoint{1.809765in}{1.371397in}}%
\pgfusepath{clip}%
\pgfsetrectcap%
\pgfsetroundjoin%
\pgfsetlinewidth{0.803000pt}%
\definecolor{currentstroke}{rgb}{0.450000,0.450000,0.450000}%
\pgfsetstrokecolor{currentstroke}%
\pgfsetdash{}{0pt}%
\pgfpathmoveto{\pgfqpoint{1.128522in}{0.417642in}}%
\pgfpathlineto{\pgfqpoint{1.128522in}{1.789039in}}%
\pgfusepath{stroke}%
\end{pgfscope}%
\begin{pgfscope}%
\pgfsetbuttcap%
\pgfsetroundjoin%
\definecolor{currentfill}{rgb}{0.000000,0.000000,0.000000}%
\pgfsetfillcolor{currentfill}%
\pgfsetlinewidth{0.803000pt}%
\definecolor{currentstroke}{rgb}{0.000000,0.000000,0.000000}%
\pgfsetstrokecolor{currentstroke}%
\pgfsetdash{}{0pt}%
\pgfsys@defobject{currentmarker}{\pgfqpoint{0.000000in}{-0.048611in}}{\pgfqpoint{0.000000in}{0.000000in}}{%
\pgfpathmoveto{\pgfqpoint{0.000000in}{0.000000in}}%
\pgfpathlineto{\pgfqpoint{0.000000in}{-0.048611in}}%
\pgfusepath{stroke,fill}%
}%
\begin{pgfscope}%
\pgfsys@transformshift{1.128522in}{0.417642in}%
\pgfsys@useobject{currentmarker}{}%
\end{pgfscope}%
\end{pgfscope}%
\begin{pgfscope}%
\definecolor{textcolor}{rgb}{0.000000,0.000000,0.000000}%
\pgfsetstrokecolor{textcolor}%
\pgfsetfillcolor{textcolor}%
\pgftext[x=1.128522in,y=0.320420in,,top]{\color{textcolor}{\rmfamily\fontsize{8.000000}{9.600000}\selectfont\catcode`\^=\active\def^{\ifmmode\sp\else\^{}\fi}\catcode`\%=\active\def%{\%}$\mathdefault{10^{1}}$}}%
\end{pgfscope}%
\begin{pgfscope}%
\pgfpathrectangle{\pgfqpoint{0.589510in}{0.417642in}}{\pgfqpoint{1.809765in}{1.371397in}}%
\pgfusepath{clip}%
\pgfsetrectcap%
\pgfsetroundjoin%
\pgfsetlinewidth{0.803000pt}%
\definecolor{currentstroke}{rgb}{0.450000,0.450000,0.450000}%
\pgfsetstrokecolor{currentstroke}%
\pgfsetdash{}{0pt}%
\pgfpathmoveto{\pgfqpoint{1.585272in}{0.417642in}}%
\pgfpathlineto{\pgfqpoint{1.585272in}{1.789039in}}%
\pgfusepath{stroke}%
\end{pgfscope}%
\begin{pgfscope}%
\pgfsetbuttcap%
\pgfsetroundjoin%
\definecolor{currentfill}{rgb}{0.000000,0.000000,0.000000}%
\pgfsetfillcolor{currentfill}%
\pgfsetlinewidth{0.803000pt}%
\definecolor{currentstroke}{rgb}{0.000000,0.000000,0.000000}%
\pgfsetstrokecolor{currentstroke}%
\pgfsetdash{}{0pt}%
\pgfsys@defobject{currentmarker}{\pgfqpoint{0.000000in}{-0.048611in}}{\pgfqpoint{0.000000in}{0.000000in}}{%
\pgfpathmoveto{\pgfqpoint{0.000000in}{0.000000in}}%
\pgfpathlineto{\pgfqpoint{0.000000in}{-0.048611in}}%
\pgfusepath{stroke,fill}%
}%
\begin{pgfscope}%
\pgfsys@transformshift{1.585272in}{0.417642in}%
\pgfsys@useobject{currentmarker}{}%
\end{pgfscope}%
\end{pgfscope}%
\begin{pgfscope}%
\definecolor{textcolor}{rgb}{0.000000,0.000000,0.000000}%
\pgfsetstrokecolor{textcolor}%
\pgfsetfillcolor{textcolor}%
\pgftext[x=1.585272in,y=0.320420in,,top]{\color{textcolor}{\rmfamily\fontsize{8.000000}{9.600000}\selectfont\catcode`\^=\active\def^{\ifmmode\sp\else\^{}\fi}\catcode`\%=\active\def%{\%}$\mathdefault{10^{2}}$}}%
\end{pgfscope}%
\begin{pgfscope}%
\pgfpathrectangle{\pgfqpoint{0.589510in}{0.417642in}}{\pgfqpoint{1.809765in}{1.371397in}}%
\pgfusepath{clip}%
\pgfsetrectcap%
\pgfsetroundjoin%
\pgfsetlinewidth{0.803000pt}%
\definecolor{currentstroke}{rgb}{0.450000,0.450000,0.450000}%
\pgfsetstrokecolor{currentstroke}%
\pgfsetdash{}{0pt}%
\pgfpathmoveto{\pgfqpoint{2.042022in}{0.417642in}}%
\pgfpathlineto{\pgfqpoint{2.042022in}{1.789039in}}%
\pgfusepath{stroke}%
\end{pgfscope}%
\begin{pgfscope}%
\pgfsetbuttcap%
\pgfsetroundjoin%
\definecolor{currentfill}{rgb}{0.000000,0.000000,0.000000}%
\pgfsetfillcolor{currentfill}%
\pgfsetlinewidth{0.803000pt}%
\definecolor{currentstroke}{rgb}{0.000000,0.000000,0.000000}%
\pgfsetstrokecolor{currentstroke}%
\pgfsetdash{}{0pt}%
\pgfsys@defobject{currentmarker}{\pgfqpoint{0.000000in}{-0.048611in}}{\pgfqpoint{0.000000in}{0.000000in}}{%
\pgfpathmoveto{\pgfqpoint{0.000000in}{0.000000in}}%
\pgfpathlineto{\pgfqpoint{0.000000in}{-0.048611in}}%
\pgfusepath{stroke,fill}%
}%
\begin{pgfscope}%
\pgfsys@transformshift{2.042022in}{0.417642in}%
\pgfsys@useobject{currentmarker}{}%
\end{pgfscope}%
\end{pgfscope}%
\begin{pgfscope}%
\definecolor{textcolor}{rgb}{0.000000,0.000000,0.000000}%
\pgfsetstrokecolor{textcolor}%
\pgfsetfillcolor{textcolor}%
\pgftext[x=2.042022in,y=0.320420in,,top]{\color{textcolor}{\rmfamily\fontsize{8.000000}{9.600000}\selectfont\catcode`\^=\active\def^{\ifmmode\sp\else\^{}\fi}\catcode`\%=\active\def%{\%}$\mathdefault{10^{3}}$}}%
\end{pgfscope}%
\begin{pgfscope}%
\pgfpathrectangle{\pgfqpoint{0.589510in}{0.417642in}}{\pgfqpoint{1.809765in}{1.371397in}}%
\pgfusepath{clip}%
\pgfsetrectcap%
\pgfsetroundjoin%
\pgfsetlinewidth{0.803000pt}%
\definecolor{currentstroke}{rgb}{0.850000,0.850000,0.850000}%
\pgfsetstrokecolor{currentstroke}%
\pgfsetdash{}{0pt}%
\pgfpathmoveto{\pgfqpoint{0.601020in}{0.417642in}}%
\pgfpathlineto{\pgfqpoint{0.601020in}{1.789039in}}%
\pgfusepath{stroke}%
\end{pgfscope}%
\begin{pgfscope}%
\pgfsetbuttcap%
\pgfsetroundjoin%
\definecolor{currentfill}{rgb}{0.000000,0.000000,0.000000}%
\pgfsetfillcolor{currentfill}%
\pgfsetlinewidth{0.602250pt}%
\definecolor{currentstroke}{rgb}{0.000000,0.000000,0.000000}%
\pgfsetstrokecolor{currentstroke}%
\pgfsetdash{}{0pt}%
\pgfsys@defobject{currentmarker}{\pgfqpoint{0.000000in}{-0.027778in}}{\pgfqpoint{0.000000in}{0.000000in}}{%
\pgfpathmoveto{\pgfqpoint{0.000000in}{0.000000in}}%
\pgfpathlineto{\pgfqpoint{0.000000in}{-0.027778in}}%
\pgfusepath{stroke,fill}%
}%
\begin{pgfscope}%
\pgfsys@transformshift{0.601020in}{0.417642in}%
\pgfsys@useobject{currentmarker}{}%
\end{pgfscope}%
\end{pgfscope}%
\begin{pgfscope}%
\pgfpathrectangle{\pgfqpoint{0.589510in}{0.417642in}}{\pgfqpoint{1.809765in}{1.371397in}}%
\pgfusepath{clip}%
\pgfsetrectcap%
\pgfsetroundjoin%
\pgfsetlinewidth{0.803000pt}%
\definecolor{currentstroke}{rgb}{0.850000,0.850000,0.850000}%
\pgfsetstrokecolor{currentstroke}%
\pgfsetdash{}{0pt}%
\pgfpathmoveto{\pgfqpoint{0.627508in}{0.417642in}}%
\pgfpathlineto{\pgfqpoint{0.627508in}{1.789039in}}%
\pgfusepath{stroke}%
\end{pgfscope}%
\begin{pgfscope}%
\pgfsetbuttcap%
\pgfsetroundjoin%
\definecolor{currentfill}{rgb}{0.000000,0.000000,0.000000}%
\pgfsetfillcolor{currentfill}%
\pgfsetlinewidth{0.602250pt}%
\definecolor{currentstroke}{rgb}{0.000000,0.000000,0.000000}%
\pgfsetstrokecolor{currentstroke}%
\pgfsetdash{}{0pt}%
\pgfsys@defobject{currentmarker}{\pgfqpoint{0.000000in}{-0.027778in}}{\pgfqpoint{0.000000in}{0.000000in}}{%
\pgfpathmoveto{\pgfqpoint{0.000000in}{0.000000in}}%
\pgfpathlineto{\pgfqpoint{0.000000in}{-0.027778in}}%
\pgfusepath{stroke,fill}%
}%
\begin{pgfscope}%
\pgfsys@transformshift{0.627508in}{0.417642in}%
\pgfsys@useobject{currentmarker}{}%
\end{pgfscope}%
\end{pgfscope}%
\begin{pgfscope}%
\pgfpathrectangle{\pgfqpoint{0.589510in}{0.417642in}}{\pgfqpoint{1.809765in}{1.371397in}}%
\pgfusepath{clip}%
\pgfsetrectcap%
\pgfsetroundjoin%
\pgfsetlinewidth{0.803000pt}%
\definecolor{currentstroke}{rgb}{0.850000,0.850000,0.850000}%
\pgfsetstrokecolor{currentstroke}%
\pgfsetdash{}{0pt}%
\pgfpathmoveto{\pgfqpoint{0.650872in}{0.417642in}}%
\pgfpathlineto{\pgfqpoint{0.650872in}{1.789039in}}%
\pgfusepath{stroke}%
\end{pgfscope}%
\begin{pgfscope}%
\pgfsetbuttcap%
\pgfsetroundjoin%
\definecolor{currentfill}{rgb}{0.000000,0.000000,0.000000}%
\pgfsetfillcolor{currentfill}%
\pgfsetlinewidth{0.602250pt}%
\definecolor{currentstroke}{rgb}{0.000000,0.000000,0.000000}%
\pgfsetstrokecolor{currentstroke}%
\pgfsetdash{}{0pt}%
\pgfsys@defobject{currentmarker}{\pgfqpoint{0.000000in}{-0.027778in}}{\pgfqpoint{0.000000in}{0.000000in}}{%
\pgfpathmoveto{\pgfqpoint{0.000000in}{0.000000in}}%
\pgfpathlineto{\pgfqpoint{0.000000in}{-0.027778in}}%
\pgfusepath{stroke,fill}%
}%
\begin{pgfscope}%
\pgfsys@transformshift{0.650872in}{0.417642in}%
\pgfsys@useobject{currentmarker}{}%
\end{pgfscope}%
\end{pgfscope}%
\begin{pgfscope}%
\pgfpathrectangle{\pgfqpoint{0.589510in}{0.417642in}}{\pgfqpoint{1.809765in}{1.371397in}}%
\pgfusepath{clip}%
\pgfsetrectcap%
\pgfsetroundjoin%
\pgfsetlinewidth{0.803000pt}%
\definecolor{currentstroke}{rgb}{0.850000,0.850000,0.850000}%
\pgfsetstrokecolor{currentstroke}%
\pgfsetdash{}{0pt}%
\pgfpathmoveto{\pgfqpoint{0.809267in}{0.417642in}}%
\pgfpathlineto{\pgfqpoint{0.809267in}{1.789039in}}%
\pgfusepath{stroke}%
\end{pgfscope}%
\begin{pgfscope}%
\pgfsetbuttcap%
\pgfsetroundjoin%
\definecolor{currentfill}{rgb}{0.000000,0.000000,0.000000}%
\pgfsetfillcolor{currentfill}%
\pgfsetlinewidth{0.602250pt}%
\definecolor{currentstroke}{rgb}{0.000000,0.000000,0.000000}%
\pgfsetstrokecolor{currentstroke}%
\pgfsetdash{}{0pt}%
\pgfsys@defobject{currentmarker}{\pgfqpoint{0.000000in}{-0.027778in}}{\pgfqpoint{0.000000in}{0.000000in}}{%
\pgfpathmoveto{\pgfqpoint{0.000000in}{0.000000in}}%
\pgfpathlineto{\pgfqpoint{0.000000in}{-0.027778in}}%
\pgfusepath{stroke,fill}%
}%
\begin{pgfscope}%
\pgfsys@transformshift{0.809267in}{0.417642in}%
\pgfsys@useobject{currentmarker}{}%
\end{pgfscope}%
\end{pgfscope}%
\begin{pgfscope}%
\pgfpathrectangle{\pgfqpoint{0.589510in}{0.417642in}}{\pgfqpoint{1.809765in}{1.371397in}}%
\pgfusepath{clip}%
\pgfsetrectcap%
\pgfsetroundjoin%
\pgfsetlinewidth{0.803000pt}%
\definecolor{currentstroke}{rgb}{0.850000,0.850000,0.850000}%
\pgfsetstrokecolor{currentstroke}%
\pgfsetdash{}{0pt}%
\pgfpathmoveto{\pgfqpoint{0.889697in}{0.417642in}}%
\pgfpathlineto{\pgfqpoint{0.889697in}{1.789039in}}%
\pgfusepath{stroke}%
\end{pgfscope}%
\begin{pgfscope}%
\pgfsetbuttcap%
\pgfsetroundjoin%
\definecolor{currentfill}{rgb}{0.000000,0.000000,0.000000}%
\pgfsetfillcolor{currentfill}%
\pgfsetlinewidth{0.602250pt}%
\definecolor{currentstroke}{rgb}{0.000000,0.000000,0.000000}%
\pgfsetstrokecolor{currentstroke}%
\pgfsetdash{}{0pt}%
\pgfsys@defobject{currentmarker}{\pgfqpoint{0.000000in}{-0.027778in}}{\pgfqpoint{0.000000in}{0.000000in}}{%
\pgfpathmoveto{\pgfqpoint{0.000000in}{0.000000in}}%
\pgfpathlineto{\pgfqpoint{0.000000in}{-0.027778in}}%
\pgfusepath{stroke,fill}%
}%
\begin{pgfscope}%
\pgfsys@transformshift{0.889697in}{0.417642in}%
\pgfsys@useobject{currentmarker}{}%
\end{pgfscope}%
\end{pgfscope}%
\begin{pgfscope}%
\pgfpathrectangle{\pgfqpoint{0.589510in}{0.417642in}}{\pgfqpoint{1.809765in}{1.371397in}}%
\pgfusepath{clip}%
\pgfsetrectcap%
\pgfsetroundjoin%
\pgfsetlinewidth{0.803000pt}%
\definecolor{currentstroke}{rgb}{0.850000,0.850000,0.850000}%
\pgfsetstrokecolor{currentstroke}%
\pgfsetdash{}{0pt}%
\pgfpathmoveto{\pgfqpoint{0.946763in}{0.417642in}}%
\pgfpathlineto{\pgfqpoint{0.946763in}{1.789039in}}%
\pgfusepath{stroke}%
\end{pgfscope}%
\begin{pgfscope}%
\pgfsetbuttcap%
\pgfsetroundjoin%
\definecolor{currentfill}{rgb}{0.000000,0.000000,0.000000}%
\pgfsetfillcolor{currentfill}%
\pgfsetlinewidth{0.602250pt}%
\definecolor{currentstroke}{rgb}{0.000000,0.000000,0.000000}%
\pgfsetstrokecolor{currentstroke}%
\pgfsetdash{}{0pt}%
\pgfsys@defobject{currentmarker}{\pgfqpoint{0.000000in}{-0.027778in}}{\pgfqpoint{0.000000in}{0.000000in}}{%
\pgfpathmoveto{\pgfqpoint{0.000000in}{0.000000in}}%
\pgfpathlineto{\pgfqpoint{0.000000in}{-0.027778in}}%
\pgfusepath{stroke,fill}%
}%
\begin{pgfscope}%
\pgfsys@transformshift{0.946763in}{0.417642in}%
\pgfsys@useobject{currentmarker}{}%
\end{pgfscope}%
\end{pgfscope}%
\begin{pgfscope}%
\pgfpathrectangle{\pgfqpoint{0.589510in}{0.417642in}}{\pgfqpoint{1.809765in}{1.371397in}}%
\pgfusepath{clip}%
\pgfsetrectcap%
\pgfsetroundjoin%
\pgfsetlinewidth{0.803000pt}%
\definecolor{currentstroke}{rgb}{0.850000,0.850000,0.850000}%
\pgfsetstrokecolor{currentstroke}%
\pgfsetdash{}{0pt}%
\pgfpathmoveto{\pgfqpoint{0.991026in}{0.417642in}}%
\pgfpathlineto{\pgfqpoint{0.991026in}{1.789039in}}%
\pgfusepath{stroke}%
\end{pgfscope}%
\begin{pgfscope}%
\pgfsetbuttcap%
\pgfsetroundjoin%
\definecolor{currentfill}{rgb}{0.000000,0.000000,0.000000}%
\pgfsetfillcolor{currentfill}%
\pgfsetlinewidth{0.602250pt}%
\definecolor{currentstroke}{rgb}{0.000000,0.000000,0.000000}%
\pgfsetstrokecolor{currentstroke}%
\pgfsetdash{}{0pt}%
\pgfsys@defobject{currentmarker}{\pgfqpoint{0.000000in}{-0.027778in}}{\pgfqpoint{0.000000in}{0.000000in}}{%
\pgfpathmoveto{\pgfqpoint{0.000000in}{0.000000in}}%
\pgfpathlineto{\pgfqpoint{0.000000in}{-0.027778in}}%
\pgfusepath{stroke,fill}%
}%
\begin{pgfscope}%
\pgfsys@transformshift{0.991026in}{0.417642in}%
\pgfsys@useobject{currentmarker}{}%
\end{pgfscope}%
\end{pgfscope}%
\begin{pgfscope}%
\pgfpathrectangle{\pgfqpoint{0.589510in}{0.417642in}}{\pgfqpoint{1.809765in}{1.371397in}}%
\pgfusepath{clip}%
\pgfsetrectcap%
\pgfsetroundjoin%
\pgfsetlinewidth{0.803000pt}%
\definecolor{currentstroke}{rgb}{0.850000,0.850000,0.850000}%
\pgfsetstrokecolor{currentstroke}%
\pgfsetdash{}{0pt}%
\pgfpathmoveto{\pgfqpoint{1.027192in}{0.417642in}}%
\pgfpathlineto{\pgfqpoint{1.027192in}{1.789039in}}%
\pgfusepath{stroke}%
\end{pgfscope}%
\begin{pgfscope}%
\pgfsetbuttcap%
\pgfsetroundjoin%
\definecolor{currentfill}{rgb}{0.000000,0.000000,0.000000}%
\pgfsetfillcolor{currentfill}%
\pgfsetlinewidth{0.602250pt}%
\definecolor{currentstroke}{rgb}{0.000000,0.000000,0.000000}%
\pgfsetstrokecolor{currentstroke}%
\pgfsetdash{}{0pt}%
\pgfsys@defobject{currentmarker}{\pgfqpoint{0.000000in}{-0.027778in}}{\pgfqpoint{0.000000in}{0.000000in}}{%
\pgfpathmoveto{\pgfqpoint{0.000000in}{0.000000in}}%
\pgfpathlineto{\pgfqpoint{0.000000in}{-0.027778in}}%
\pgfusepath{stroke,fill}%
}%
\begin{pgfscope}%
\pgfsys@transformshift{1.027192in}{0.417642in}%
\pgfsys@useobject{currentmarker}{}%
\end{pgfscope}%
\end{pgfscope}%
\begin{pgfscope}%
\pgfpathrectangle{\pgfqpoint{0.589510in}{0.417642in}}{\pgfqpoint{1.809765in}{1.371397in}}%
\pgfusepath{clip}%
\pgfsetrectcap%
\pgfsetroundjoin%
\pgfsetlinewidth{0.803000pt}%
\definecolor{currentstroke}{rgb}{0.850000,0.850000,0.850000}%
\pgfsetstrokecolor{currentstroke}%
\pgfsetdash{}{0pt}%
\pgfpathmoveto{\pgfqpoint{1.057770in}{0.417642in}}%
\pgfpathlineto{\pgfqpoint{1.057770in}{1.789039in}}%
\pgfusepath{stroke}%
\end{pgfscope}%
\begin{pgfscope}%
\pgfsetbuttcap%
\pgfsetroundjoin%
\definecolor{currentfill}{rgb}{0.000000,0.000000,0.000000}%
\pgfsetfillcolor{currentfill}%
\pgfsetlinewidth{0.602250pt}%
\definecolor{currentstroke}{rgb}{0.000000,0.000000,0.000000}%
\pgfsetstrokecolor{currentstroke}%
\pgfsetdash{}{0pt}%
\pgfsys@defobject{currentmarker}{\pgfqpoint{0.000000in}{-0.027778in}}{\pgfqpoint{0.000000in}{0.000000in}}{%
\pgfpathmoveto{\pgfqpoint{0.000000in}{0.000000in}}%
\pgfpathlineto{\pgfqpoint{0.000000in}{-0.027778in}}%
\pgfusepath{stroke,fill}%
}%
\begin{pgfscope}%
\pgfsys@transformshift{1.057770in}{0.417642in}%
\pgfsys@useobject{currentmarker}{}%
\end{pgfscope}%
\end{pgfscope}%
\begin{pgfscope}%
\pgfpathrectangle{\pgfqpoint{0.589510in}{0.417642in}}{\pgfqpoint{1.809765in}{1.371397in}}%
\pgfusepath{clip}%
\pgfsetrectcap%
\pgfsetroundjoin%
\pgfsetlinewidth{0.803000pt}%
\definecolor{currentstroke}{rgb}{0.850000,0.850000,0.850000}%
\pgfsetstrokecolor{currentstroke}%
\pgfsetdash{}{0pt}%
\pgfpathmoveto{\pgfqpoint{1.084258in}{0.417642in}}%
\pgfpathlineto{\pgfqpoint{1.084258in}{1.789039in}}%
\pgfusepath{stroke}%
\end{pgfscope}%
\begin{pgfscope}%
\pgfsetbuttcap%
\pgfsetroundjoin%
\definecolor{currentfill}{rgb}{0.000000,0.000000,0.000000}%
\pgfsetfillcolor{currentfill}%
\pgfsetlinewidth{0.602250pt}%
\definecolor{currentstroke}{rgb}{0.000000,0.000000,0.000000}%
\pgfsetstrokecolor{currentstroke}%
\pgfsetdash{}{0pt}%
\pgfsys@defobject{currentmarker}{\pgfqpoint{0.000000in}{-0.027778in}}{\pgfqpoint{0.000000in}{0.000000in}}{%
\pgfpathmoveto{\pgfqpoint{0.000000in}{0.000000in}}%
\pgfpathlineto{\pgfqpoint{0.000000in}{-0.027778in}}%
\pgfusepath{stroke,fill}%
}%
\begin{pgfscope}%
\pgfsys@transformshift{1.084258in}{0.417642in}%
\pgfsys@useobject{currentmarker}{}%
\end{pgfscope}%
\end{pgfscope}%
\begin{pgfscope}%
\pgfpathrectangle{\pgfqpoint{0.589510in}{0.417642in}}{\pgfqpoint{1.809765in}{1.371397in}}%
\pgfusepath{clip}%
\pgfsetrectcap%
\pgfsetroundjoin%
\pgfsetlinewidth{0.803000pt}%
\definecolor{currentstroke}{rgb}{0.850000,0.850000,0.850000}%
\pgfsetstrokecolor{currentstroke}%
\pgfsetdash{}{0pt}%
\pgfpathmoveto{\pgfqpoint{1.107622in}{0.417642in}}%
\pgfpathlineto{\pgfqpoint{1.107622in}{1.789039in}}%
\pgfusepath{stroke}%
\end{pgfscope}%
\begin{pgfscope}%
\pgfsetbuttcap%
\pgfsetroundjoin%
\definecolor{currentfill}{rgb}{0.000000,0.000000,0.000000}%
\pgfsetfillcolor{currentfill}%
\pgfsetlinewidth{0.602250pt}%
\definecolor{currentstroke}{rgb}{0.000000,0.000000,0.000000}%
\pgfsetstrokecolor{currentstroke}%
\pgfsetdash{}{0pt}%
\pgfsys@defobject{currentmarker}{\pgfqpoint{0.000000in}{-0.027778in}}{\pgfqpoint{0.000000in}{0.000000in}}{%
\pgfpathmoveto{\pgfqpoint{0.000000in}{0.000000in}}%
\pgfpathlineto{\pgfqpoint{0.000000in}{-0.027778in}}%
\pgfusepath{stroke,fill}%
}%
\begin{pgfscope}%
\pgfsys@transformshift{1.107622in}{0.417642in}%
\pgfsys@useobject{currentmarker}{}%
\end{pgfscope}%
\end{pgfscope}%
\begin{pgfscope}%
\pgfpathrectangle{\pgfqpoint{0.589510in}{0.417642in}}{\pgfqpoint{1.809765in}{1.371397in}}%
\pgfusepath{clip}%
\pgfsetrectcap%
\pgfsetroundjoin%
\pgfsetlinewidth{0.803000pt}%
\definecolor{currentstroke}{rgb}{0.850000,0.850000,0.850000}%
\pgfsetstrokecolor{currentstroke}%
\pgfsetdash{}{0pt}%
\pgfpathmoveto{\pgfqpoint{1.266017in}{0.417642in}}%
\pgfpathlineto{\pgfqpoint{1.266017in}{1.789039in}}%
\pgfusepath{stroke}%
\end{pgfscope}%
\begin{pgfscope}%
\pgfsetbuttcap%
\pgfsetroundjoin%
\definecolor{currentfill}{rgb}{0.000000,0.000000,0.000000}%
\pgfsetfillcolor{currentfill}%
\pgfsetlinewidth{0.602250pt}%
\definecolor{currentstroke}{rgb}{0.000000,0.000000,0.000000}%
\pgfsetstrokecolor{currentstroke}%
\pgfsetdash{}{0pt}%
\pgfsys@defobject{currentmarker}{\pgfqpoint{0.000000in}{-0.027778in}}{\pgfqpoint{0.000000in}{0.000000in}}{%
\pgfpathmoveto{\pgfqpoint{0.000000in}{0.000000in}}%
\pgfpathlineto{\pgfqpoint{0.000000in}{-0.027778in}}%
\pgfusepath{stroke,fill}%
}%
\begin{pgfscope}%
\pgfsys@transformshift{1.266017in}{0.417642in}%
\pgfsys@useobject{currentmarker}{}%
\end{pgfscope}%
\end{pgfscope}%
\begin{pgfscope}%
\pgfpathrectangle{\pgfqpoint{0.589510in}{0.417642in}}{\pgfqpoint{1.809765in}{1.371397in}}%
\pgfusepath{clip}%
\pgfsetrectcap%
\pgfsetroundjoin%
\pgfsetlinewidth{0.803000pt}%
\definecolor{currentstroke}{rgb}{0.850000,0.850000,0.850000}%
\pgfsetstrokecolor{currentstroke}%
\pgfsetdash{}{0pt}%
\pgfpathmoveto{\pgfqpoint{1.346447in}{0.417642in}}%
\pgfpathlineto{\pgfqpoint{1.346447in}{1.789039in}}%
\pgfusepath{stroke}%
\end{pgfscope}%
\begin{pgfscope}%
\pgfsetbuttcap%
\pgfsetroundjoin%
\definecolor{currentfill}{rgb}{0.000000,0.000000,0.000000}%
\pgfsetfillcolor{currentfill}%
\pgfsetlinewidth{0.602250pt}%
\definecolor{currentstroke}{rgb}{0.000000,0.000000,0.000000}%
\pgfsetstrokecolor{currentstroke}%
\pgfsetdash{}{0pt}%
\pgfsys@defobject{currentmarker}{\pgfqpoint{0.000000in}{-0.027778in}}{\pgfqpoint{0.000000in}{0.000000in}}{%
\pgfpathmoveto{\pgfqpoint{0.000000in}{0.000000in}}%
\pgfpathlineto{\pgfqpoint{0.000000in}{-0.027778in}}%
\pgfusepath{stroke,fill}%
}%
\begin{pgfscope}%
\pgfsys@transformshift{1.346447in}{0.417642in}%
\pgfsys@useobject{currentmarker}{}%
\end{pgfscope}%
\end{pgfscope}%
\begin{pgfscope}%
\pgfpathrectangle{\pgfqpoint{0.589510in}{0.417642in}}{\pgfqpoint{1.809765in}{1.371397in}}%
\pgfusepath{clip}%
\pgfsetrectcap%
\pgfsetroundjoin%
\pgfsetlinewidth{0.803000pt}%
\definecolor{currentstroke}{rgb}{0.850000,0.850000,0.850000}%
\pgfsetstrokecolor{currentstroke}%
\pgfsetdash{}{0pt}%
\pgfpathmoveto{\pgfqpoint{1.403513in}{0.417642in}}%
\pgfpathlineto{\pgfqpoint{1.403513in}{1.789039in}}%
\pgfusepath{stroke}%
\end{pgfscope}%
\begin{pgfscope}%
\pgfsetbuttcap%
\pgfsetroundjoin%
\definecolor{currentfill}{rgb}{0.000000,0.000000,0.000000}%
\pgfsetfillcolor{currentfill}%
\pgfsetlinewidth{0.602250pt}%
\definecolor{currentstroke}{rgb}{0.000000,0.000000,0.000000}%
\pgfsetstrokecolor{currentstroke}%
\pgfsetdash{}{0pt}%
\pgfsys@defobject{currentmarker}{\pgfqpoint{0.000000in}{-0.027778in}}{\pgfqpoint{0.000000in}{0.000000in}}{%
\pgfpathmoveto{\pgfqpoint{0.000000in}{0.000000in}}%
\pgfpathlineto{\pgfqpoint{0.000000in}{-0.027778in}}%
\pgfusepath{stroke,fill}%
}%
\begin{pgfscope}%
\pgfsys@transformshift{1.403513in}{0.417642in}%
\pgfsys@useobject{currentmarker}{}%
\end{pgfscope}%
\end{pgfscope}%
\begin{pgfscope}%
\pgfpathrectangle{\pgfqpoint{0.589510in}{0.417642in}}{\pgfqpoint{1.809765in}{1.371397in}}%
\pgfusepath{clip}%
\pgfsetrectcap%
\pgfsetroundjoin%
\pgfsetlinewidth{0.803000pt}%
\definecolor{currentstroke}{rgb}{0.850000,0.850000,0.850000}%
\pgfsetstrokecolor{currentstroke}%
\pgfsetdash{}{0pt}%
\pgfpathmoveto{\pgfqpoint{1.447776in}{0.417642in}}%
\pgfpathlineto{\pgfqpoint{1.447776in}{1.789039in}}%
\pgfusepath{stroke}%
\end{pgfscope}%
\begin{pgfscope}%
\pgfsetbuttcap%
\pgfsetroundjoin%
\definecolor{currentfill}{rgb}{0.000000,0.000000,0.000000}%
\pgfsetfillcolor{currentfill}%
\pgfsetlinewidth{0.602250pt}%
\definecolor{currentstroke}{rgb}{0.000000,0.000000,0.000000}%
\pgfsetstrokecolor{currentstroke}%
\pgfsetdash{}{0pt}%
\pgfsys@defobject{currentmarker}{\pgfqpoint{0.000000in}{-0.027778in}}{\pgfqpoint{0.000000in}{0.000000in}}{%
\pgfpathmoveto{\pgfqpoint{0.000000in}{0.000000in}}%
\pgfpathlineto{\pgfqpoint{0.000000in}{-0.027778in}}%
\pgfusepath{stroke,fill}%
}%
\begin{pgfscope}%
\pgfsys@transformshift{1.447776in}{0.417642in}%
\pgfsys@useobject{currentmarker}{}%
\end{pgfscope}%
\end{pgfscope}%
\begin{pgfscope}%
\pgfpathrectangle{\pgfqpoint{0.589510in}{0.417642in}}{\pgfqpoint{1.809765in}{1.371397in}}%
\pgfusepath{clip}%
\pgfsetrectcap%
\pgfsetroundjoin%
\pgfsetlinewidth{0.803000pt}%
\definecolor{currentstroke}{rgb}{0.850000,0.850000,0.850000}%
\pgfsetstrokecolor{currentstroke}%
\pgfsetdash{}{0pt}%
\pgfpathmoveto{\pgfqpoint{1.483942in}{0.417642in}}%
\pgfpathlineto{\pgfqpoint{1.483942in}{1.789039in}}%
\pgfusepath{stroke}%
\end{pgfscope}%
\begin{pgfscope}%
\pgfsetbuttcap%
\pgfsetroundjoin%
\definecolor{currentfill}{rgb}{0.000000,0.000000,0.000000}%
\pgfsetfillcolor{currentfill}%
\pgfsetlinewidth{0.602250pt}%
\definecolor{currentstroke}{rgb}{0.000000,0.000000,0.000000}%
\pgfsetstrokecolor{currentstroke}%
\pgfsetdash{}{0pt}%
\pgfsys@defobject{currentmarker}{\pgfqpoint{0.000000in}{-0.027778in}}{\pgfqpoint{0.000000in}{0.000000in}}{%
\pgfpathmoveto{\pgfqpoint{0.000000in}{0.000000in}}%
\pgfpathlineto{\pgfqpoint{0.000000in}{-0.027778in}}%
\pgfusepath{stroke,fill}%
}%
\begin{pgfscope}%
\pgfsys@transformshift{1.483942in}{0.417642in}%
\pgfsys@useobject{currentmarker}{}%
\end{pgfscope}%
\end{pgfscope}%
\begin{pgfscope}%
\pgfpathrectangle{\pgfqpoint{0.589510in}{0.417642in}}{\pgfqpoint{1.809765in}{1.371397in}}%
\pgfusepath{clip}%
\pgfsetrectcap%
\pgfsetroundjoin%
\pgfsetlinewidth{0.803000pt}%
\definecolor{currentstroke}{rgb}{0.850000,0.850000,0.850000}%
\pgfsetstrokecolor{currentstroke}%
\pgfsetdash{}{0pt}%
\pgfpathmoveto{\pgfqpoint{1.514520in}{0.417642in}}%
\pgfpathlineto{\pgfqpoint{1.514520in}{1.789039in}}%
\pgfusepath{stroke}%
\end{pgfscope}%
\begin{pgfscope}%
\pgfsetbuttcap%
\pgfsetroundjoin%
\definecolor{currentfill}{rgb}{0.000000,0.000000,0.000000}%
\pgfsetfillcolor{currentfill}%
\pgfsetlinewidth{0.602250pt}%
\definecolor{currentstroke}{rgb}{0.000000,0.000000,0.000000}%
\pgfsetstrokecolor{currentstroke}%
\pgfsetdash{}{0pt}%
\pgfsys@defobject{currentmarker}{\pgfqpoint{0.000000in}{-0.027778in}}{\pgfqpoint{0.000000in}{0.000000in}}{%
\pgfpathmoveto{\pgfqpoint{0.000000in}{0.000000in}}%
\pgfpathlineto{\pgfqpoint{0.000000in}{-0.027778in}}%
\pgfusepath{stroke,fill}%
}%
\begin{pgfscope}%
\pgfsys@transformshift{1.514520in}{0.417642in}%
\pgfsys@useobject{currentmarker}{}%
\end{pgfscope}%
\end{pgfscope}%
\begin{pgfscope}%
\pgfpathrectangle{\pgfqpoint{0.589510in}{0.417642in}}{\pgfqpoint{1.809765in}{1.371397in}}%
\pgfusepath{clip}%
\pgfsetrectcap%
\pgfsetroundjoin%
\pgfsetlinewidth{0.803000pt}%
\definecolor{currentstroke}{rgb}{0.850000,0.850000,0.850000}%
\pgfsetstrokecolor{currentstroke}%
\pgfsetdash{}{0pt}%
\pgfpathmoveto{\pgfqpoint{1.541008in}{0.417642in}}%
\pgfpathlineto{\pgfqpoint{1.541008in}{1.789039in}}%
\pgfusepath{stroke}%
\end{pgfscope}%
\begin{pgfscope}%
\pgfsetbuttcap%
\pgfsetroundjoin%
\definecolor{currentfill}{rgb}{0.000000,0.000000,0.000000}%
\pgfsetfillcolor{currentfill}%
\pgfsetlinewidth{0.602250pt}%
\definecolor{currentstroke}{rgb}{0.000000,0.000000,0.000000}%
\pgfsetstrokecolor{currentstroke}%
\pgfsetdash{}{0pt}%
\pgfsys@defobject{currentmarker}{\pgfqpoint{0.000000in}{-0.027778in}}{\pgfqpoint{0.000000in}{0.000000in}}{%
\pgfpathmoveto{\pgfqpoint{0.000000in}{0.000000in}}%
\pgfpathlineto{\pgfqpoint{0.000000in}{-0.027778in}}%
\pgfusepath{stroke,fill}%
}%
\begin{pgfscope}%
\pgfsys@transformshift{1.541008in}{0.417642in}%
\pgfsys@useobject{currentmarker}{}%
\end{pgfscope}%
\end{pgfscope}%
\begin{pgfscope}%
\pgfpathrectangle{\pgfqpoint{0.589510in}{0.417642in}}{\pgfqpoint{1.809765in}{1.371397in}}%
\pgfusepath{clip}%
\pgfsetrectcap%
\pgfsetroundjoin%
\pgfsetlinewidth{0.803000pt}%
\definecolor{currentstroke}{rgb}{0.850000,0.850000,0.850000}%
\pgfsetstrokecolor{currentstroke}%
\pgfsetdash{}{0pt}%
\pgfpathmoveto{\pgfqpoint{1.564372in}{0.417642in}}%
\pgfpathlineto{\pgfqpoint{1.564372in}{1.789039in}}%
\pgfusepath{stroke}%
\end{pgfscope}%
\begin{pgfscope}%
\pgfsetbuttcap%
\pgfsetroundjoin%
\definecolor{currentfill}{rgb}{0.000000,0.000000,0.000000}%
\pgfsetfillcolor{currentfill}%
\pgfsetlinewidth{0.602250pt}%
\definecolor{currentstroke}{rgb}{0.000000,0.000000,0.000000}%
\pgfsetstrokecolor{currentstroke}%
\pgfsetdash{}{0pt}%
\pgfsys@defobject{currentmarker}{\pgfqpoint{0.000000in}{-0.027778in}}{\pgfqpoint{0.000000in}{0.000000in}}{%
\pgfpathmoveto{\pgfqpoint{0.000000in}{0.000000in}}%
\pgfpathlineto{\pgfqpoint{0.000000in}{-0.027778in}}%
\pgfusepath{stroke,fill}%
}%
\begin{pgfscope}%
\pgfsys@transformshift{1.564372in}{0.417642in}%
\pgfsys@useobject{currentmarker}{}%
\end{pgfscope}%
\end{pgfscope}%
\begin{pgfscope}%
\pgfpathrectangle{\pgfqpoint{0.589510in}{0.417642in}}{\pgfqpoint{1.809765in}{1.371397in}}%
\pgfusepath{clip}%
\pgfsetrectcap%
\pgfsetroundjoin%
\pgfsetlinewidth{0.803000pt}%
\definecolor{currentstroke}{rgb}{0.850000,0.850000,0.850000}%
\pgfsetstrokecolor{currentstroke}%
\pgfsetdash{}{0pt}%
\pgfpathmoveto{\pgfqpoint{1.722767in}{0.417642in}}%
\pgfpathlineto{\pgfqpoint{1.722767in}{1.789039in}}%
\pgfusepath{stroke}%
\end{pgfscope}%
\begin{pgfscope}%
\pgfsetbuttcap%
\pgfsetroundjoin%
\definecolor{currentfill}{rgb}{0.000000,0.000000,0.000000}%
\pgfsetfillcolor{currentfill}%
\pgfsetlinewidth{0.602250pt}%
\definecolor{currentstroke}{rgb}{0.000000,0.000000,0.000000}%
\pgfsetstrokecolor{currentstroke}%
\pgfsetdash{}{0pt}%
\pgfsys@defobject{currentmarker}{\pgfqpoint{0.000000in}{-0.027778in}}{\pgfqpoint{0.000000in}{0.000000in}}{%
\pgfpathmoveto{\pgfqpoint{0.000000in}{0.000000in}}%
\pgfpathlineto{\pgfqpoint{0.000000in}{-0.027778in}}%
\pgfusepath{stroke,fill}%
}%
\begin{pgfscope}%
\pgfsys@transformshift{1.722767in}{0.417642in}%
\pgfsys@useobject{currentmarker}{}%
\end{pgfscope}%
\end{pgfscope}%
\begin{pgfscope}%
\pgfpathrectangle{\pgfqpoint{0.589510in}{0.417642in}}{\pgfqpoint{1.809765in}{1.371397in}}%
\pgfusepath{clip}%
\pgfsetrectcap%
\pgfsetroundjoin%
\pgfsetlinewidth{0.803000pt}%
\definecolor{currentstroke}{rgb}{0.850000,0.850000,0.850000}%
\pgfsetstrokecolor{currentstroke}%
\pgfsetdash{}{0pt}%
\pgfpathmoveto{\pgfqpoint{1.803197in}{0.417642in}}%
\pgfpathlineto{\pgfqpoint{1.803197in}{1.789039in}}%
\pgfusepath{stroke}%
\end{pgfscope}%
\begin{pgfscope}%
\pgfsetbuttcap%
\pgfsetroundjoin%
\definecolor{currentfill}{rgb}{0.000000,0.000000,0.000000}%
\pgfsetfillcolor{currentfill}%
\pgfsetlinewidth{0.602250pt}%
\definecolor{currentstroke}{rgb}{0.000000,0.000000,0.000000}%
\pgfsetstrokecolor{currentstroke}%
\pgfsetdash{}{0pt}%
\pgfsys@defobject{currentmarker}{\pgfqpoint{0.000000in}{-0.027778in}}{\pgfqpoint{0.000000in}{0.000000in}}{%
\pgfpathmoveto{\pgfqpoint{0.000000in}{0.000000in}}%
\pgfpathlineto{\pgfqpoint{0.000000in}{-0.027778in}}%
\pgfusepath{stroke,fill}%
}%
\begin{pgfscope}%
\pgfsys@transformshift{1.803197in}{0.417642in}%
\pgfsys@useobject{currentmarker}{}%
\end{pgfscope}%
\end{pgfscope}%
\begin{pgfscope}%
\pgfpathrectangle{\pgfqpoint{0.589510in}{0.417642in}}{\pgfqpoint{1.809765in}{1.371397in}}%
\pgfusepath{clip}%
\pgfsetrectcap%
\pgfsetroundjoin%
\pgfsetlinewidth{0.803000pt}%
\definecolor{currentstroke}{rgb}{0.850000,0.850000,0.850000}%
\pgfsetstrokecolor{currentstroke}%
\pgfsetdash{}{0pt}%
\pgfpathmoveto{\pgfqpoint{1.860263in}{0.417642in}}%
\pgfpathlineto{\pgfqpoint{1.860263in}{1.789039in}}%
\pgfusepath{stroke}%
\end{pgfscope}%
\begin{pgfscope}%
\pgfsetbuttcap%
\pgfsetroundjoin%
\definecolor{currentfill}{rgb}{0.000000,0.000000,0.000000}%
\pgfsetfillcolor{currentfill}%
\pgfsetlinewidth{0.602250pt}%
\definecolor{currentstroke}{rgb}{0.000000,0.000000,0.000000}%
\pgfsetstrokecolor{currentstroke}%
\pgfsetdash{}{0pt}%
\pgfsys@defobject{currentmarker}{\pgfqpoint{0.000000in}{-0.027778in}}{\pgfqpoint{0.000000in}{0.000000in}}{%
\pgfpathmoveto{\pgfqpoint{0.000000in}{0.000000in}}%
\pgfpathlineto{\pgfqpoint{0.000000in}{-0.027778in}}%
\pgfusepath{stroke,fill}%
}%
\begin{pgfscope}%
\pgfsys@transformshift{1.860263in}{0.417642in}%
\pgfsys@useobject{currentmarker}{}%
\end{pgfscope}%
\end{pgfscope}%
\begin{pgfscope}%
\pgfpathrectangle{\pgfqpoint{0.589510in}{0.417642in}}{\pgfqpoint{1.809765in}{1.371397in}}%
\pgfusepath{clip}%
\pgfsetrectcap%
\pgfsetroundjoin%
\pgfsetlinewidth{0.803000pt}%
\definecolor{currentstroke}{rgb}{0.850000,0.850000,0.850000}%
\pgfsetstrokecolor{currentstroke}%
\pgfsetdash{}{0pt}%
\pgfpathmoveto{\pgfqpoint{1.904526in}{0.417642in}}%
\pgfpathlineto{\pgfqpoint{1.904526in}{1.789039in}}%
\pgfusepath{stroke}%
\end{pgfscope}%
\begin{pgfscope}%
\pgfsetbuttcap%
\pgfsetroundjoin%
\definecolor{currentfill}{rgb}{0.000000,0.000000,0.000000}%
\pgfsetfillcolor{currentfill}%
\pgfsetlinewidth{0.602250pt}%
\definecolor{currentstroke}{rgb}{0.000000,0.000000,0.000000}%
\pgfsetstrokecolor{currentstroke}%
\pgfsetdash{}{0pt}%
\pgfsys@defobject{currentmarker}{\pgfqpoint{0.000000in}{-0.027778in}}{\pgfqpoint{0.000000in}{0.000000in}}{%
\pgfpathmoveto{\pgfqpoint{0.000000in}{0.000000in}}%
\pgfpathlineto{\pgfqpoint{0.000000in}{-0.027778in}}%
\pgfusepath{stroke,fill}%
}%
\begin{pgfscope}%
\pgfsys@transformshift{1.904526in}{0.417642in}%
\pgfsys@useobject{currentmarker}{}%
\end{pgfscope}%
\end{pgfscope}%
\begin{pgfscope}%
\pgfpathrectangle{\pgfqpoint{0.589510in}{0.417642in}}{\pgfqpoint{1.809765in}{1.371397in}}%
\pgfusepath{clip}%
\pgfsetrectcap%
\pgfsetroundjoin%
\pgfsetlinewidth{0.803000pt}%
\definecolor{currentstroke}{rgb}{0.850000,0.850000,0.850000}%
\pgfsetstrokecolor{currentstroke}%
\pgfsetdash{}{0pt}%
\pgfpathmoveto{\pgfqpoint{1.940693in}{0.417642in}}%
\pgfpathlineto{\pgfqpoint{1.940693in}{1.789039in}}%
\pgfusepath{stroke}%
\end{pgfscope}%
\begin{pgfscope}%
\pgfsetbuttcap%
\pgfsetroundjoin%
\definecolor{currentfill}{rgb}{0.000000,0.000000,0.000000}%
\pgfsetfillcolor{currentfill}%
\pgfsetlinewidth{0.602250pt}%
\definecolor{currentstroke}{rgb}{0.000000,0.000000,0.000000}%
\pgfsetstrokecolor{currentstroke}%
\pgfsetdash{}{0pt}%
\pgfsys@defobject{currentmarker}{\pgfqpoint{0.000000in}{-0.027778in}}{\pgfqpoint{0.000000in}{0.000000in}}{%
\pgfpathmoveto{\pgfqpoint{0.000000in}{0.000000in}}%
\pgfpathlineto{\pgfqpoint{0.000000in}{-0.027778in}}%
\pgfusepath{stroke,fill}%
}%
\begin{pgfscope}%
\pgfsys@transformshift{1.940693in}{0.417642in}%
\pgfsys@useobject{currentmarker}{}%
\end{pgfscope}%
\end{pgfscope}%
\begin{pgfscope}%
\pgfpathrectangle{\pgfqpoint{0.589510in}{0.417642in}}{\pgfqpoint{1.809765in}{1.371397in}}%
\pgfusepath{clip}%
\pgfsetrectcap%
\pgfsetroundjoin%
\pgfsetlinewidth{0.803000pt}%
\definecolor{currentstroke}{rgb}{0.850000,0.850000,0.850000}%
\pgfsetstrokecolor{currentstroke}%
\pgfsetdash{}{0pt}%
\pgfpathmoveto{\pgfqpoint{1.971270in}{0.417642in}}%
\pgfpathlineto{\pgfqpoint{1.971270in}{1.789039in}}%
\pgfusepath{stroke}%
\end{pgfscope}%
\begin{pgfscope}%
\pgfsetbuttcap%
\pgfsetroundjoin%
\definecolor{currentfill}{rgb}{0.000000,0.000000,0.000000}%
\pgfsetfillcolor{currentfill}%
\pgfsetlinewidth{0.602250pt}%
\definecolor{currentstroke}{rgb}{0.000000,0.000000,0.000000}%
\pgfsetstrokecolor{currentstroke}%
\pgfsetdash{}{0pt}%
\pgfsys@defobject{currentmarker}{\pgfqpoint{0.000000in}{-0.027778in}}{\pgfqpoint{0.000000in}{0.000000in}}{%
\pgfpathmoveto{\pgfqpoint{0.000000in}{0.000000in}}%
\pgfpathlineto{\pgfqpoint{0.000000in}{-0.027778in}}%
\pgfusepath{stroke,fill}%
}%
\begin{pgfscope}%
\pgfsys@transformshift{1.971270in}{0.417642in}%
\pgfsys@useobject{currentmarker}{}%
\end{pgfscope}%
\end{pgfscope}%
\begin{pgfscope}%
\pgfpathrectangle{\pgfqpoint{0.589510in}{0.417642in}}{\pgfqpoint{1.809765in}{1.371397in}}%
\pgfusepath{clip}%
\pgfsetrectcap%
\pgfsetroundjoin%
\pgfsetlinewidth{0.803000pt}%
\definecolor{currentstroke}{rgb}{0.850000,0.850000,0.850000}%
\pgfsetstrokecolor{currentstroke}%
\pgfsetdash{}{0pt}%
\pgfpathmoveto{\pgfqpoint{1.997758in}{0.417642in}}%
\pgfpathlineto{\pgfqpoint{1.997758in}{1.789039in}}%
\pgfusepath{stroke}%
\end{pgfscope}%
\begin{pgfscope}%
\pgfsetbuttcap%
\pgfsetroundjoin%
\definecolor{currentfill}{rgb}{0.000000,0.000000,0.000000}%
\pgfsetfillcolor{currentfill}%
\pgfsetlinewidth{0.602250pt}%
\definecolor{currentstroke}{rgb}{0.000000,0.000000,0.000000}%
\pgfsetstrokecolor{currentstroke}%
\pgfsetdash{}{0pt}%
\pgfsys@defobject{currentmarker}{\pgfqpoint{0.000000in}{-0.027778in}}{\pgfqpoint{0.000000in}{0.000000in}}{%
\pgfpathmoveto{\pgfqpoint{0.000000in}{0.000000in}}%
\pgfpathlineto{\pgfqpoint{0.000000in}{-0.027778in}}%
\pgfusepath{stroke,fill}%
}%
\begin{pgfscope}%
\pgfsys@transformshift{1.997758in}{0.417642in}%
\pgfsys@useobject{currentmarker}{}%
\end{pgfscope}%
\end{pgfscope}%
\begin{pgfscope}%
\pgfpathrectangle{\pgfqpoint{0.589510in}{0.417642in}}{\pgfqpoint{1.809765in}{1.371397in}}%
\pgfusepath{clip}%
\pgfsetrectcap%
\pgfsetroundjoin%
\pgfsetlinewidth{0.803000pt}%
\definecolor{currentstroke}{rgb}{0.850000,0.850000,0.850000}%
\pgfsetstrokecolor{currentstroke}%
\pgfsetdash{}{0pt}%
\pgfpathmoveto{\pgfqpoint{2.021122in}{0.417642in}}%
\pgfpathlineto{\pgfqpoint{2.021122in}{1.789039in}}%
\pgfusepath{stroke}%
\end{pgfscope}%
\begin{pgfscope}%
\pgfsetbuttcap%
\pgfsetroundjoin%
\definecolor{currentfill}{rgb}{0.000000,0.000000,0.000000}%
\pgfsetfillcolor{currentfill}%
\pgfsetlinewidth{0.602250pt}%
\definecolor{currentstroke}{rgb}{0.000000,0.000000,0.000000}%
\pgfsetstrokecolor{currentstroke}%
\pgfsetdash{}{0pt}%
\pgfsys@defobject{currentmarker}{\pgfqpoint{0.000000in}{-0.027778in}}{\pgfqpoint{0.000000in}{0.000000in}}{%
\pgfpathmoveto{\pgfqpoint{0.000000in}{0.000000in}}%
\pgfpathlineto{\pgfqpoint{0.000000in}{-0.027778in}}%
\pgfusepath{stroke,fill}%
}%
\begin{pgfscope}%
\pgfsys@transformshift{2.021122in}{0.417642in}%
\pgfsys@useobject{currentmarker}{}%
\end{pgfscope}%
\end{pgfscope}%
\begin{pgfscope}%
\pgfpathrectangle{\pgfqpoint{0.589510in}{0.417642in}}{\pgfqpoint{1.809765in}{1.371397in}}%
\pgfusepath{clip}%
\pgfsetrectcap%
\pgfsetroundjoin%
\pgfsetlinewidth{0.803000pt}%
\definecolor{currentstroke}{rgb}{0.850000,0.850000,0.850000}%
\pgfsetstrokecolor{currentstroke}%
\pgfsetdash{}{0pt}%
\pgfpathmoveto{\pgfqpoint{2.179517in}{0.417642in}}%
\pgfpathlineto{\pgfqpoint{2.179517in}{1.789039in}}%
\pgfusepath{stroke}%
\end{pgfscope}%
\begin{pgfscope}%
\pgfsetbuttcap%
\pgfsetroundjoin%
\definecolor{currentfill}{rgb}{0.000000,0.000000,0.000000}%
\pgfsetfillcolor{currentfill}%
\pgfsetlinewidth{0.602250pt}%
\definecolor{currentstroke}{rgb}{0.000000,0.000000,0.000000}%
\pgfsetstrokecolor{currentstroke}%
\pgfsetdash{}{0pt}%
\pgfsys@defobject{currentmarker}{\pgfqpoint{0.000000in}{-0.027778in}}{\pgfqpoint{0.000000in}{0.000000in}}{%
\pgfpathmoveto{\pgfqpoint{0.000000in}{0.000000in}}%
\pgfpathlineto{\pgfqpoint{0.000000in}{-0.027778in}}%
\pgfusepath{stroke,fill}%
}%
\begin{pgfscope}%
\pgfsys@transformshift{2.179517in}{0.417642in}%
\pgfsys@useobject{currentmarker}{}%
\end{pgfscope}%
\end{pgfscope}%
\begin{pgfscope}%
\pgfpathrectangle{\pgfqpoint{0.589510in}{0.417642in}}{\pgfqpoint{1.809765in}{1.371397in}}%
\pgfusepath{clip}%
\pgfsetrectcap%
\pgfsetroundjoin%
\pgfsetlinewidth{0.803000pt}%
\definecolor{currentstroke}{rgb}{0.850000,0.850000,0.850000}%
\pgfsetstrokecolor{currentstroke}%
\pgfsetdash{}{0pt}%
\pgfpathmoveto{\pgfqpoint{2.259947in}{0.417642in}}%
\pgfpathlineto{\pgfqpoint{2.259947in}{1.789039in}}%
\pgfusepath{stroke}%
\end{pgfscope}%
\begin{pgfscope}%
\pgfsetbuttcap%
\pgfsetroundjoin%
\definecolor{currentfill}{rgb}{0.000000,0.000000,0.000000}%
\pgfsetfillcolor{currentfill}%
\pgfsetlinewidth{0.602250pt}%
\definecolor{currentstroke}{rgb}{0.000000,0.000000,0.000000}%
\pgfsetstrokecolor{currentstroke}%
\pgfsetdash{}{0pt}%
\pgfsys@defobject{currentmarker}{\pgfqpoint{0.000000in}{-0.027778in}}{\pgfqpoint{0.000000in}{0.000000in}}{%
\pgfpathmoveto{\pgfqpoint{0.000000in}{0.000000in}}%
\pgfpathlineto{\pgfqpoint{0.000000in}{-0.027778in}}%
\pgfusepath{stroke,fill}%
}%
\begin{pgfscope}%
\pgfsys@transformshift{2.259947in}{0.417642in}%
\pgfsys@useobject{currentmarker}{}%
\end{pgfscope}%
\end{pgfscope}%
\begin{pgfscope}%
\pgfpathrectangle{\pgfqpoint{0.589510in}{0.417642in}}{\pgfqpoint{1.809765in}{1.371397in}}%
\pgfusepath{clip}%
\pgfsetrectcap%
\pgfsetroundjoin%
\pgfsetlinewidth{0.803000pt}%
\definecolor{currentstroke}{rgb}{0.850000,0.850000,0.850000}%
\pgfsetstrokecolor{currentstroke}%
\pgfsetdash{}{0pt}%
\pgfpathmoveto{\pgfqpoint{2.317013in}{0.417642in}}%
\pgfpathlineto{\pgfqpoint{2.317013in}{1.789039in}}%
\pgfusepath{stroke}%
\end{pgfscope}%
\begin{pgfscope}%
\pgfsetbuttcap%
\pgfsetroundjoin%
\definecolor{currentfill}{rgb}{0.000000,0.000000,0.000000}%
\pgfsetfillcolor{currentfill}%
\pgfsetlinewidth{0.602250pt}%
\definecolor{currentstroke}{rgb}{0.000000,0.000000,0.000000}%
\pgfsetstrokecolor{currentstroke}%
\pgfsetdash{}{0pt}%
\pgfsys@defobject{currentmarker}{\pgfqpoint{0.000000in}{-0.027778in}}{\pgfqpoint{0.000000in}{0.000000in}}{%
\pgfpathmoveto{\pgfqpoint{0.000000in}{0.000000in}}%
\pgfpathlineto{\pgfqpoint{0.000000in}{-0.027778in}}%
\pgfusepath{stroke,fill}%
}%
\begin{pgfscope}%
\pgfsys@transformshift{2.317013in}{0.417642in}%
\pgfsys@useobject{currentmarker}{}%
\end{pgfscope}%
\end{pgfscope}%
\begin{pgfscope}%
\pgfpathrectangle{\pgfqpoint{0.589510in}{0.417642in}}{\pgfqpoint{1.809765in}{1.371397in}}%
\pgfusepath{clip}%
\pgfsetrectcap%
\pgfsetroundjoin%
\pgfsetlinewidth{0.803000pt}%
\definecolor{currentstroke}{rgb}{0.850000,0.850000,0.850000}%
\pgfsetstrokecolor{currentstroke}%
\pgfsetdash{}{0pt}%
\pgfpathmoveto{\pgfqpoint{2.361277in}{0.417642in}}%
\pgfpathlineto{\pgfqpoint{2.361277in}{1.789039in}}%
\pgfusepath{stroke}%
\end{pgfscope}%
\begin{pgfscope}%
\pgfsetbuttcap%
\pgfsetroundjoin%
\definecolor{currentfill}{rgb}{0.000000,0.000000,0.000000}%
\pgfsetfillcolor{currentfill}%
\pgfsetlinewidth{0.602250pt}%
\definecolor{currentstroke}{rgb}{0.000000,0.000000,0.000000}%
\pgfsetstrokecolor{currentstroke}%
\pgfsetdash{}{0pt}%
\pgfsys@defobject{currentmarker}{\pgfqpoint{0.000000in}{-0.027778in}}{\pgfqpoint{0.000000in}{0.000000in}}{%
\pgfpathmoveto{\pgfqpoint{0.000000in}{0.000000in}}%
\pgfpathlineto{\pgfqpoint{0.000000in}{-0.027778in}}%
\pgfusepath{stroke,fill}%
}%
\begin{pgfscope}%
\pgfsys@transformshift{2.361277in}{0.417642in}%
\pgfsys@useobject{currentmarker}{}%
\end{pgfscope}%
\end{pgfscope}%
\begin{pgfscope}%
\pgfpathrectangle{\pgfqpoint{0.589510in}{0.417642in}}{\pgfqpoint{1.809765in}{1.371397in}}%
\pgfusepath{clip}%
\pgfsetrectcap%
\pgfsetroundjoin%
\pgfsetlinewidth{0.803000pt}%
\definecolor{currentstroke}{rgb}{0.850000,0.850000,0.850000}%
\pgfsetstrokecolor{currentstroke}%
\pgfsetdash{}{0pt}%
\pgfpathmoveto{\pgfqpoint{2.397443in}{0.417642in}}%
\pgfpathlineto{\pgfqpoint{2.397443in}{1.789039in}}%
\pgfusepath{stroke}%
\end{pgfscope}%
\begin{pgfscope}%
\pgfsetbuttcap%
\pgfsetroundjoin%
\definecolor{currentfill}{rgb}{0.000000,0.000000,0.000000}%
\pgfsetfillcolor{currentfill}%
\pgfsetlinewidth{0.602250pt}%
\definecolor{currentstroke}{rgb}{0.000000,0.000000,0.000000}%
\pgfsetstrokecolor{currentstroke}%
\pgfsetdash{}{0pt}%
\pgfsys@defobject{currentmarker}{\pgfqpoint{0.000000in}{-0.027778in}}{\pgfqpoint{0.000000in}{0.000000in}}{%
\pgfpathmoveto{\pgfqpoint{0.000000in}{0.000000in}}%
\pgfpathlineto{\pgfqpoint{0.000000in}{-0.027778in}}%
\pgfusepath{stroke,fill}%
}%
\begin{pgfscope}%
\pgfsys@transformshift{2.397443in}{0.417642in}%
\pgfsys@useobject{currentmarker}{}%
\end{pgfscope}%
\end{pgfscope}%
\begin{pgfscope}%
\definecolor{textcolor}{rgb}{0.000000,0.000000,0.000000}%
\pgfsetstrokecolor{textcolor}%
\pgfsetfillcolor{textcolor}%
\pgftext[x=1.494392in,y=0.165003in,,top]{\color{textcolor}{\rmfamily\fontsize{10.000000}{12.000000}\selectfont\catcode`\^=\active\def^{\ifmmode\sp\else\^{}\fi}\catcode`\%=\active\def%{\%}$\tau$ in \unit{\second}}}%
\end{pgfscope}%
\begin{pgfscope}%
\pgfpathrectangle{\pgfqpoint{0.589510in}{0.417642in}}{\pgfqpoint{1.809765in}{1.371397in}}%
\pgfusepath{clip}%
\pgfsetrectcap%
\pgfsetroundjoin%
\pgfsetlinewidth{0.803000pt}%
\definecolor{currentstroke}{rgb}{0.450000,0.450000,0.450000}%
\pgfsetstrokecolor{currentstroke}%
\pgfsetdash{}{0pt}%
\pgfpathmoveto{\pgfqpoint{0.589510in}{0.417642in}}%
\pgfpathlineto{\pgfqpoint{2.399275in}{0.417642in}}%
\pgfusepath{stroke}%
\end{pgfscope}%
\begin{pgfscope}%
\pgfsetbuttcap%
\pgfsetroundjoin%
\definecolor{currentfill}{rgb}{0.000000,0.000000,0.000000}%
\pgfsetfillcolor{currentfill}%
\pgfsetlinewidth{0.803000pt}%
\definecolor{currentstroke}{rgb}{0.000000,0.000000,0.000000}%
\pgfsetstrokecolor{currentstroke}%
\pgfsetdash{}{0pt}%
\pgfsys@defobject{currentmarker}{\pgfqpoint{-0.048611in}{0.000000in}}{\pgfqpoint{-0.000000in}{0.000000in}}{%
\pgfpathmoveto{\pgfqpoint{-0.000000in}{0.000000in}}%
\pgfpathlineto{\pgfqpoint{-0.048611in}{0.000000in}}%
\pgfusepath{stroke,fill}%
}%
\begin{pgfscope}%
\pgfsys@transformshift{0.589510in}{0.417642in}%
\pgfsys@useobject{currentmarker}{}%
\end{pgfscope}%
\end{pgfscope}%
\begin{pgfscope}%
\definecolor{textcolor}{rgb}{0.000000,0.000000,0.000000}%
\pgfsetstrokecolor{textcolor}%
\pgfsetfillcolor{textcolor}%
\pgftext[x=0.236114in, y=0.378489in, left, base]{\color{textcolor}{\rmfamily\fontsize{8.000000}{9.600000}\selectfont\catcode`\^=\active\def^{\ifmmode\sp\else\^{}\fi}\catcode`\%=\active\def%{\%}$\mathdefault{10^{-2}}$}}%
\end{pgfscope}%
\begin{pgfscope}%
\pgfpathrectangle{\pgfqpoint{0.589510in}{0.417642in}}{\pgfqpoint{1.809765in}{1.371397in}}%
\pgfusepath{clip}%
\pgfsetrectcap%
\pgfsetroundjoin%
\pgfsetlinewidth{0.803000pt}%
\definecolor{currentstroke}{rgb}{0.450000,0.450000,0.450000}%
\pgfsetstrokecolor{currentstroke}%
\pgfsetdash{}{0pt}%
\pgfpathmoveto{\pgfqpoint{0.589510in}{0.622360in}}%
\pgfpathlineto{\pgfqpoint{2.399275in}{0.622360in}}%
\pgfusepath{stroke}%
\end{pgfscope}%
\begin{pgfscope}%
\pgfsetbuttcap%
\pgfsetroundjoin%
\definecolor{currentfill}{rgb}{0.000000,0.000000,0.000000}%
\pgfsetfillcolor{currentfill}%
\pgfsetlinewidth{0.803000pt}%
\definecolor{currentstroke}{rgb}{0.000000,0.000000,0.000000}%
\pgfsetstrokecolor{currentstroke}%
\pgfsetdash{}{0pt}%
\pgfsys@defobject{currentmarker}{\pgfqpoint{-0.048611in}{0.000000in}}{\pgfqpoint{-0.000000in}{0.000000in}}{%
\pgfpathmoveto{\pgfqpoint{-0.000000in}{0.000000in}}%
\pgfpathlineto{\pgfqpoint{-0.048611in}{0.000000in}}%
\pgfusepath{stroke,fill}%
}%
\begin{pgfscope}%
\pgfsys@transformshift{0.589510in}{0.622360in}%
\pgfsys@useobject{currentmarker}{}%
\end{pgfscope}%
\end{pgfscope}%
\begin{pgfscope}%
\definecolor{textcolor}{rgb}{0.000000,0.000000,0.000000}%
\pgfsetstrokecolor{textcolor}%
\pgfsetfillcolor{textcolor}%
\pgftext[x=0.236114in, y=0.583207in, left, base]{\color{textcolor}{\rmfamily\fontsize{8.000000}{9.600000}\selectfont\catcode`\^=\active\def^{\ifmmode\sp\else\^{}\fi}\catcode`\%=\active\def%{\%}$\mathdefault{10^{-1}}$}}%
\end{pgfscope}%
\begin{pgfscope}%
\pgfpathrectangle{\pgfqpoint{0.589510in}{0.417642in}}{\pgfqpoint{1.809765in}{1.371397in}}%
\pgfusepath{clip}%
\pgfsetrectcap%
\pgfsetroundjoin%
\pgfsetlinewidth{0.803000pt}%
\definecolor{currentstroke}{rgb}{0.450000,0.450000,0.450000}%
\pgfsetstrokecolor{currentstroke}%
\pgfsetdash{}{0pt}%
\pgfpathmoveto{\pgfqpoint{0.589510in}{0.827077in}}%
\pgfpathlineto{\pgfqpoint{2.399275in}{0.827077in}}%
\pgfusepath{stroke}%
\end{pgfscope}%
\begin{pgfscope}%
\pgfsetbuttcap%
\pgfsetroundjoin%
\definecolor{currentfill}{rgb}{0.000000,0.000000,0.000000}%
\pgfsetfillcolor{currentfill}%
\pgfsetlinewidth{0.803000pt}%
\definecolor{currentstroke}{rgb}{0.000000,0.000000,0.000000}%
\pgfsetstrokecolor{currentstroke}%
\pgfsetdash{}{0pt}%
\pgfsys@defobject{currentmarker}{\pgfqpoint{-0.048611in}{0.000000in}}{\pgfqpoint{-0.000000in}{0.000000in}}{%
\pgfpathmoveto{\pgfqpoint{-0.000000in}{0.000000in}}%
\pgfpathlineto{\pgfqpoint{-0.048611in}{0.000000in}}%
\pgfusepath{stroke,fill}%
}%
\begin{pgfscope}%
\pgfsys@transformshift{0.589510in}{0.827077in}%
\pgfsys@useobject{currentmarker}{}%
\end{pgfscope}%
\end{pgfscope}%
\begin{pgfscope}%
\definecolor{textcolor}{rgb}{0.000000,0.000000,0.000000}%
\pgfsetstrokecolor{textcolor}%
\pgfsetfillcolor{textcolor}%
\pgftext[x=0.316361in, y=0.787924in, left, base]{\color{textcolor}{\rmfamily\fontsize{8.000000}{9.600000}\selectfont\catcode`\^=\active\def^{\ifmmode\sp\else\^{}\fi}\catcode`\%=\active\def%{\%}$\mathdefault{10^{0}}$}}%
\end{pgfscope}%
\begin{pgfscope}%
\pgfpathrectangle{\pgfqpoint{0.589510in}{0.417642in}}{\pgfqpoint{1.809765in}{1.371397in}}%
\pgfusepath{clip}%
\pgfsetrectcap%
\pgfsetroundjoin%
\pgfsetlinewidth{0.803000pt}%
\definecolor{currentstroke}{rgb}{0.450000,0.450000,0.450000}%
\pgfsetstrokecolor{currentstroke}%
\pgfsetdash{}{0pt}%
\pgfpathmoveto{\pgfqpoint{0.589510in}{1.031795in}}%
\pgfpathlineto{\pgfqpoint{2.399275in}{1.031795in}}%
\pgfusepath{stroke}%
\end{pgfscope}%
\begin{pgfscope}%
\pgfsetbuttcap%
\pgfsetroundjoin%
\definecolor{currentfill}{rgb}{0.000000,0.000000,0.000000}%
\pgfsetfillcolor{currentfill}%
\pgfsetlinewidth{0.803000pt}%
\definecolor{currentstroke}{rgb}{0.000000,0.000000,0.000000}%
\pgfsetstrokecolor{currentstroke}%
\pgfsetdash{}{0pt}%
\pgfsys@defobject{currentmarker}{\pgfqpoint{-0.048611in}{0.000000in}}{\pgfqpoint{-0.000000in}{0.000000in}}{%
\pgfpathmoveto{\pgfqpoint{-0.000000in}{0.000000in}}%
\pgfpathlineto{\pgfqpoint{-0.048611in}{0.000000in}}%
\pgfusepath{stroke,fill}%
}%
\begin{pgfscope}%
\pgfsys@transformshift{0.589510in}{1.031795in}%
\pgfsys@useobject{currentmarker}{}%
\end{pgfscope}%
\end{pgfscope}%
\begin{pgfscope}%
\definecolor{textcolor}{rgb}{0.000000,0.000000,0.000000}%
\pgfsetstrokecolor{textcolor}%
\pgfsetfillcolor{textcolor}%
\pgftext[x=0.316361in, y=0.992642in, left, base]{\color{textcolor}{\rmfamily\fontsize{8.000000}{9.600000}\selectfont\catcode`\^=\active\def^{\ifmmode\sp\else\^{}\fi}\catcode`\%=\active\def%{\%}$\mathdefault{10^{1}}$}}%
\end{pgfscope}%
\begin{pgfscope}%
\pgfpathrectangle{\pgfqpoint{0.589510in}{0.417642in}}{\pgfqpoint{1.809765in}{1.371397in}}%
\pgfusepath{clip}%
\pgfsetrectcap%
\pgfsetroundjoin%
\pgfsetlinewidth{0.803000pt}%
\definecolor{currentstroke}{rgb}{0.450000,0.450000,0.450000}%
\pgfsetstrokecolor{currentstroke}%
\pgfsetdash{}{0pt}%
\pgfpathmoveto{\pgfqpoint{0.589510in}{1.236512in}}%
\pgfpathlineto{\pgfqpoint{2.399275in}{1.236512in}}%
\pgfusepath{stroke}%
\end{pgfscope}%
\begin{pgfscope}%
\pgfsetbuttcap%
\pgfsetroundjoin%
\definecolor{currentfill}{rgb}{0.000000,0.000000,0.000000}%
\pgfsetfillcolor{currentfill}%
\pgfsetlinewidth{0.803000pt}%
\definecolor{currentstroke}{rgb}{0.000000,0.000000,0.000000}%
\pgfsetstrokecolor{currentstroke}%
\pgfsetdash{}{0pt}%
\pgfsys@defobject{currentmarker}{\pgfqpoint{-0.048611in}{0.000000in}}{\pgfqpoint{-0.000000in}{0.000000in}}{%
\pgfpathmoveto{\pgfqpoint{-0.000000in}{0.000000in}}%
\pgfpathlineto{\pgfqpoint{-0.048611in}{0.000000in}}%
\pgfusepath{stroke,fill}%
}%
\begin{pgfscope}%
\pgfsys@transformshift{0.589510in}{1.236512in}%
\pgfsys@useobject{currentmarker}{}%
\end{pgfscope}%
\end{pgfscope}%
\begin{pgfscope}%
\definecolor{textcolor}{rgb}{0.000000,0.000000,0.000000}%
\pgfsetstrokecolor{textcolor}%
\pgfsetfillcolor{textcolor}%
\pgftext[x=0.316361in, y=1.197359in, left, base]{\color{textcolor}{\rmfamily\fontsize{8.000000}{9.600000}\selectfont\catcode`\^=\active\def^{\ifmmode\sp\else\^{}\fi}\catcode`\%=\active\def%{\%}$\mathdefault{10^{2}}$}}%
\end{pgfscope}%
\begin{pgfscope}%
\pgfpathrectangle{\pgfqpoint{0.589510in}{0.417642in}}{\pgfqpoint{1.809765in}{1.371397in}}%
\pgfusepath{clip}%
\pgfsetrectcap%
\pgfsetroundjoin%
\pgfsetlinewidth{0.803000pt}%
\definecolor{currentstroke}{rgb}{0.450000,0.450000,0.450000}%
\pgfsetstrokecolor{currentstroke}%
\pgfsetdash{}{0pt}%
\pgfpathmoveto{\pgfqpoint{0.589510in}{1.441230in}}%
\pgfpathlineto{\pgfqpoint{2.399275in}{1.441230in}}%
\pgfusepath{stroke}%
\end{pgfscope}%
\begin{pgfscope}%
\pgfsetbuttcap%
\pgfsetroundjoin%
\definecolor{currentfill}{rgb}{0.000000,0.000000,0.000000}%
\pgfsetfillcolor{currentfill}%
\pgfsetlinewidth{0.803000pt}%
\definecolor{currentstroke}{rgb}{0.000000,0.000000,0.000000}%
\pgfsetstrokecolor{currentstroke}%
\pgfsetdash{}{0pt}%
\pgfsys@defobject{currentmarker}{\pgfqpoint{-0.048611in}{0.000000in}}{\pgfqpoint{-0.000000in}{0.000000in}}{%
\pgfpathmoveto{\pgfqpoint{-0.000000in}{0.000000in}}%
\pgfpathlineto{\pgfqpoint{-0.048611in}{0.000000in}}%
\pgfusepath{stroke,fill}%
}%
\begin{pgfscope}%
\pgfsys@transformshift{0.589510in}{1.441230in}%
\pgfsys@useobject{currentmarker}{}%
\end{pgfscope}%
\end{pgfscope}%
\begin{pgfscope}%
\definecolor{textcolor}{rgb}{0.000000,0.000000,0.000000}%
\pgfsetstrokecolor{textcolor}%
\pgfsetfillcolor{textcolor}%
\pgftext[x=0.316361in, y=1.402077in, left, base]{\color{textcolor}{\rmfamily\fontsize{8.000000}{9.600000}\selectfont\catcode`\^=\active\def^{\ifmmode\sp\else\^{}\fi}\catcode`\%=\active\def%{\%}$\mathdefault{10^{3}}$}}%
\end{pgfscope}%
\begin{pgfscope}%
\pgfpathrectangle{\pgfqpoint{0.589510in}{0.417642in}}{\pgfqpoint{1.809765in}{1.371397in}}%
\pgfusepath{clip}%
\pgfsetrectcap%
\pgfsetroundjoin%
\pgfsetlinewidth{0.803000pt}%
\definecolor{currentstroke}{rgb}{0.450000,0.450000,0.450000}%
\pgfsetstrokecolor{currentstroke}%
\pgfsetdash{}{0pt}%
\pgfpathmoveto{\pgfqpoint{0.589510in}{1.645947in}}%
\pgfpathlineto{\pgfqpoint{2.399275in}{1.645947in}}%
\pgfusepath{stroke}%
\end{pgfscope}%
\begin{pgfscope}%
\pgfsetbuttcap%
\pgfsetroundjoin%
\definecolor{currentfill}{rgb}{0.000000,0.000000,0.000000}%
\pgfsetfillcolor{currentfill}%
\pgfsetlinewidth{0.803000pt}%
\definecolor{currentstroke}{rgb}{0.000000,0.000000,0.000000}%
\pgfsetstrokecolor{currentstroke}%
\pgfsetdash{}{0pt}%
\pgfsys@defobject{currentmarker}{\pgfqpoint{-0.048611in}{0.000000in}}{\pgfqpoint{-0.000000in}{0.000000in}}{%
\pgfpathmoveto{\pgfqpoint{-0.000000in}{0.000000in}}%
\pgfpathlineto{\pgfqpoint{-0.048611in}{0.000000in}}%
\pgfusepath{stroke,fill}%
}%
\begin{pgfscope}%
\pgfsys@transformshift{0.589510in}{1.645947in}%
\pgfsys@useobject{currentmarker}{}%
\end{pgfscope}%
\end{pgfscope}%
\begin{pgfscope}%
\definecolor{textcolor}{rgb}{0.000000,0.000000,0.000000}%
\pgfsetstrokecolor{textcolor}%
\pgfsetfillcolor{textcolor}%
\pgftext[x=0.316361in, y=1.606795in, left, base]{\color{textcolor}{\rmfamily\fontsize{8.000000}{9.600000}\selectfont\catcode`\^=\active\def^{\ifmmode\sp\else\^{}\fi}\catcode`\%=\active\def%{\%}$\mathdefault{10^{4}}$}}%
\end{pgfscope}%
\begin{pgfscope}%
\pgfsetbuttcap%
\pgfsetroundjoin%
\definecolor{currentfill}{rgb}{0.000000,0.000000,0.000000}%
\pgfsetfillcolor{currentfill}%
\pgfsetlinewidth{0.602250pt}%
\definecolor{currentstroke}{rgb}{0.000000,0.000000,0.000000}%
\pgfsetstrokecolor{currentstroke}%
\pgfsetdash{}{0pt}%
\pgfsys@defobject{currentmarker}{\pgfqpoint{-0.027778in}{0.000000in}}{\pgfqpoint{-0.000000in}{0.000000in}}{%
\pgfpathmoveto{\pgfqpoint{-0.000000in}{0.000000in}}%
\pgfpathlineto{\pgfqpoint{-0.027778in}{0.000000in}}%
\pgfusepath{stroke,fill}%
}%
\begin{pgfscope}%
\pgfsys@transformshift{0.589510in}{0.479268in}%
\pgfsys@useobject{currentmarker}{}%
\end{pgfscope}%
\end{pgfscope}%
\begin{pgfscope}%
\pgfsetbuttcap%
\pgfsetroundjoin%
\definecolor{currentfill}{rgb}{0.000000,0.000000,0.000000}%
\pgfsetfillcolor{currentfill}%
\pgfsetlinewidth{0.602250pt}%
\definecolor{currentstroke}{rgb}{0.000000,0.000000,0.000000}%
\pgfsetstrokecolor{currentstroke}%
\pgfsetdash{}{0pt}%
\pgfsys@defobject{currentmarker}{\pgfqpoint{-0.027778in}{0.000000in}}{\pgfqpoint{-0.000000in}{0.000000in}}{%
\pgfpathmoveto{\pgfqpoint{-0.000000in}{0.000000in}}%
\pgfpathlineto{\pgfqpoint{-0.027778in}{0.000000in}}%
\pgfusepath{stroke,fill}%
}%
\begin{pgfscope}%
\pgfsys@transformshift{0.589510in}{0.515317in}%
\pgfsys@useobject{currentmarker}{}%
\end{pgfscope}%
\end{pgfscope}%
\begin{pgfscope}%
\pgfsetbuttcap%
\pgfsetroundjoin%
\definecolor{currentfill}{rgb}{0.000000,0.000000,0.000000}%
\pgfsetfillcolor{currentfill}%
\pgfsetlinewidth{0.602250pt}%
\definecolor{currentstroke}{rgb}{0.000000,0.000000,0.000000}%
\pgfsetstrokecolor{currentstroke}%
\pgfsetdash{}{0pt}%
\pgfsys@defobject{currentmarker}{\pgfqpoint{-0.027778in}{0.000000in}}{\pgfqpoint{-0.000000in}{0.000000in}}{%
\pgfpathmoveto{\pgfqpoint{-0.000000in}{0.000000in}}%
\pgfpathlineto{\pgfqpoint{-0.027778in}{0.000000in}}%
\pgfusepath{stroke,fill}%
}%
\begin{pgfscope}%
\pgfsys@transformshift{0.589510in}{0.540894in}%
\pgfsys@useobject{currentmarker}{}%
\end{pgfscope}%
\end{pgfscope}%
\begin{pgfscope}%
\pgfsetbuttcap%
\pgfsetroundjoin%
\definecolor{currentfill}{rgb}{0.000000,0.000000,0.000000}%
\pgfsetfillcolor{currentfill}%
\pgfsetlinewidth{0.602250pt}%
\definecolor{currentstroke}{rgb}{0.000000,0.000000,0.000000}%
\pgfsetstrokecolor{currentstroke}%
\pgfsetdash{}{0pt}%
\pgfsys@defobject{currentmarker}{\pgfqpoint{-0.027778in}{0.000000in}}{\pgfqpoint{-0.000000in}{0.000000in}}{%
\pgfpathmoveto{\pgfqpoint{-0.000000in}{0.000000in}}%
\pgfpathlineto{\pgfqpoint{-0.027778in}{0.000000in}}%
\pgfusepath{stroke,fill}%
}%
\begin{pgfscope}%
\pgfsys@transformshift{0.589510in}{0.560733in}%
\pgfsys@useobject{currentmarker}{}%
\end{pgfscope}%
\end{pgfscope}%
\begin{pgfscope}%
\pgfsetbuttcap%
\pgfsetroundjoin%
\definecolor{currentfill}{rgb}{0.000000,0.000000,0.000000}%
\pgfsetfillcolor{currentfill}%
\pgfsetlinewidth{0.602250pt}%
\definecolor{currentstroke}{rgb}{0.000000,0.000000,0.000000}%
\pgfsetstrokecolor{currentstroke}%
\pgfsetdash{}{0pt}%
\pgfsys@defobject{currentmarker}{\pgfqpoint{-0.027778in}{0.000000in}}{\pgfqpoint{-0.000000in}{0.000000in}}{%
\pgfpathmoveto{\pgfqpoint{-0.000000in}{0.000000in}}%
\pgfpathlineto{\pgfqpoint{-0.027778in}{0.000000in}}%
\pgfusepath{stroke,fill}%
}%
\begin{pgfscope}%
\pgfsys@transformshift{0.589510in}{0.576943in}%
\pgfsys@useobject{currentmarker}{}%
\end{pgfscope}%
\end{pgfscope}%
\begin{pgfscope}%
\pgfsetbuttcap%
\pgfsetroundjoin%
\definecolor{currentfill}{rgb}{0.000000,0.000000,0.000000}%
\pgfsetfillcolor{currentfill}%
\pgfsetlinewidth{0.602250pt}%
\definecolor{currentstroke}{rgb}{0.000000,0.000000,0.000000}%
\pgfsetstrokecolor{currentstroke}%
\pgfsetdash{}{0pt}%
\pgfsys@defobject{currentmarker}{\pgfqpoint{-0.027778in}{0.000000in}}{\pgfqpoint{-0.000000in}{0.000000in}}{%
\pgfpathmoveto{\pgfqpoint{-0.000000in}{0.000000in}}%
\pgfpathlineto{\pgfqpoint{-0.027778in}{0.000000in}}%
\pgfusepath{stroke,fill}%
}%
\begin{pgfscope}%
\pgfsys@transformshift{0.589510in}{0.590648in}%
\pgfsys@useobject{currentmarker}{}%
\end{pgfscope}%
\end{pgfscope}%
\begin{pgfscope}%
\pgfsetbuttcap%
\pgfsetroundjoin%
\definecolor{currentfill}{rgb}{0.000000,0.000000,0.000000}%
\pgfsetfillcolor{currentfill}%
\pgfsetlinewidth{0.602250pt}%
\definecolor{currentstroke}{rgb}{0.000000,0.000000,0.000000}%
\pgfsetstrokecolor{currentstroke}%
\pgfsetdash{}{0pt}%
\pgfsys@defobject{currentmarker}{\pgfqpoint{-0.027778in}{0.000000in}}{\pgfqpoint{-0.000000in}{0.000000in}}{%
\pgfpathmoveto{\pgfqpoint{-0.000000in}{0.000000in}}%
\pgfpathlineto{\pgfqpoint{-0.027778in}{0.000000in}}%
\pgfusepath{stroke,fill}%
}%
\begin{pgfscope}%
\pgfsys@transformshift{0.589510in}{0.602520in}%
\pgfsys@useobject{currentmarker}{}%
\end{pgfscope}%
\end{pgfscope}%
\begin{pgfscope}%
\pgfsetbuttcap%
\pgfsetroundjoin%
\definecolor{currentfill}{rgb}{0.000000,0.000000,0.000000}%
\pgfsetfillcolor{currentfill}%
\pgfsetlinewidth{0.602250pt}%
\definecolor{currentstroke}{rgb}{0.000000,0.000000,0.000000}%
\pgfsetstrokecolor{currentstroke}%
\pgfsetdash{}{0pt}%
\pgfsys@defobject{currentmarker}{\pgfqpoint{-0.027778in}{0.000000in}}{\pgfqpoint{-0.000000in}{0.000000in}}{%
\pgfpathmoveto{\pgfqpoint{-0.000000in}{0.000000in}}%
\pgfpathlineto{\pgfqpoint{-0.027778in}{0.000000in}}%
\pgfusepath{stroke,fill}%
}%
\begin{pgfscope}%
\pgfsys@transformshift{0.589510in}{0.612992in}%
\pgfsys@useobject{currentmarker}{}%
\end{pgfscope}%
\end{pgfscope}%
\begin{pgfscope}%
\pgfsetbuttcap%
\pgfsetroundjoin%
\definecolor{currentfill}{rgb}{0.000000,0.000000,0.000000}%
\pgfsetfillcolor{currentfill}%
\pgfsetlinewidth{0.602250pt}%
\definecolor{currentstroke}{rgb}{0.000000,0.000000,0.000000}%
\pgfsetstrokecolor{currentstroke}%
\pgfsetdash{}{0pt}%
\pgfsys@defobject{currentmarker}{\pgfqpoint{-0.027778in}{0.000000in}}{\pgfqpoint{-0.000000in}{0.000000in}}{%
\pgfpathmoveto{\pgfqpoint{-0.000000in}{0.000000in}}%
\pgfpathlineto{\pgfqpoint{-0.027778in}{0.000000in}}%
\pgfusepath{stroke,fill}%
}%
\begin{pgfscope}%
\pgfsys@transformshift{0.589510in}{0.683986in}%
\pgfsys@useobject{currentmarker}{}%
\end{pgfscope}%
\end{pgfscope}%
\begin{pgfscope}%
\pgfsetbuttcap%
\pgfsetroundjoin%
\definecolor{currentfill}{rgb}{0.000000,0.000000,0.000000}%
\pgfsetfillcolor{currentfill}%
\pgfsetlinewidth{0.602250pt}%
\definecolor{currentstroke}{rgb}{0.000000,0.000000,0.000000}%
\pgfsetstrokecolor{currentstroke}%
\pgfsetdash{}{0pt}%
\pgfsys@defobject{currentmarker}{\pgfqpoint{-0.027778in}{0.000000in}}{\pgfqpoint{-0.000000in}{0.000000in}}{%
\pgfpathmoveto{\pgfqpoint{-0.000000in}{0.000000in}}%
\pgfpathlineto{\pgfqpoint{-0.027778in}{0.000000in}}%
\pgfusepath{stroke,fill}%
}%
\begin{pgfscope}%
\pgfsys@transformshift{0.589510in}{0.720035in}%
\pgfsys@useobject{currentmarker}{}%
\end{pgfscope}%
\end{pgfscope}%
\begin{pgfscope}%
\pgfsetbuttcap%
\pgfsetroundjoin%
\definecolor{currentfill}{rgb}{0.000000,0.000000,0.000000}%
\pgfsetfillcolor{currentfill}%
\pgfsetlinewidth{0.602250pt}%
\definecolor{currentstroke}{rgb}{0.000000,0.000000,0.000000}%
\pgfsetstrokecolor{currentstroke}%
\pgfsetdash{}{0pt}%
\pgfsys@defobject{currentmarker}{\pgfqpoint{-0.027778in}{0.000000in}}{\pgfqpoint{-0.000000in}{0.000000in}}{%
\pgfpathmoveto{\pgfqpoint{-0.000000in}{0.000000in}}%
\pgfpathlineto{\pgfqpoint{-0.027778in}{0.000000in}}%
\pgfusepath{stroke,fill}%
}%
\begin{pgfscope}%
\pgfsys@transformshift{0.589510in}{0.745612in}%
\pgfsys@useobject{currentmarker}{}%
\end{pgfscope}%
\end{pgfscope}%
\begin{pgfscope}%
\pgfsetbuttcap%
\pgfsetroundjoin%
\definecolor{currentfill}{rgb}{0.000000,0.000000,0.000000}%
\pgfsetfillcolor{currentfill}%
\pgfsetlinewidth{0.602250pt}%
\definecolor{currentstroke}{rgb}{0.000000,0.000000,0.000000}%
\pgfsetstrokecolor{currentstroke}%
\pgfsetdash{}{0pt}%
\pgfsys@defobject{currentmarker}{\pgfqpoint{-0.027778in}{0.000000in}}{\pgfqpoint{-0.000000in}{0.000000in}}{%
\pgfpathmoveto{\pgfqpoint{-0.000000in}{0.000000in}}%
\pgfpathlineto{\pgfqpoint{-0.027778in}{0.000000in}}%
\pgfusepath{stroke,fill}%
}%
\begin{pgfscope}%
\pgfsys@transformshift{0.589510in}{0.765451in}%
\pgfsys@useobject{currentmarker}{}%
\end{pgfscope}%
\end{pgfscope}%
\begin{pgfscope}%
\pgfsetbuttcap%
\pgfsetroundjoin%
\definecolor{currentfill}{rgb}{0.000000,0.000000,0.000000}%
\pgfsetfillcolor{currentfill}%
\pgfsetlinewidth{0.602250pt}%
\definecolor{currentstroke}{rgb}{0.000000,0.000000,0.000000}%
\pgfsetstrokecolor{currentstroke}%
\pgfsetdash{}{0pt}%
\pgfsys@defobject{currentmarker}{\pgfqpoint{-0.027778in}{0.000000in}}{\pgfqpoint{-0.000000in}{0.000000in}}{%
\pgfpathmoveto{\pgfqpoint{-0.000000in}{0.000000in}}%
\pgfpathlineto{\pgfqpoint{-0.027778in}{0.000000in}}%
\pgfusepath{stroke,fill}%
}%
\begin{pgfscope}%
\pgfsys@transformshift{0.589510in}{0.781661in}%
\pgfsys@useobject{currentmarker}{}%
\end{pgfscope}%
\end{pgfscope}%
\begin{pgfscope}%
\pgfsetbuttcap%
\pgfsetroundjoin%
\definecolor{currentfill}{rgb}{0.000000,0.000000,0.000000}%
\pgfsetfillcolor{currentfill}%
\pgfsetlinewidth{0.602250pt}%
\definecolor{currentstroke}{rgb}{0.000000,0.000000,0.000000}%
\pgfsetstrokecolor{currentstroke}%
\pgfsetdash{}{0pt}%
\pgfsys@defobject{currentmarker}{\pgfqpoint{-0.027778in}{0.000000in}}{\pgfqpoint{-0.000000in}{0.000000in}}{%
\pgfpathmoveto{\pgfqpoint{-0.000000in}{0.000000in}}%
\pgfpathlineto{\pgfqpoint{-0.027778in}{0.000000in}}%
\pgfusepath{stroke,fill}%
}%
\begin{pgfscope}%
\pgfsys@transformshift{0.589510in}{0.795366in}%
\pgfsys@useobject{currentmarker}{}%
\end{pgfscope}%
\end{pgfscope}%
\begin{pgfscope}%
\pgfsetbuttcap%
\pgfsetroundjoin%
\definecolor{currentfill}{rgb}{0.000000,0.000000,0.000000}%
\pgfsetfillcolor{currentfill}%
\pgfsetlinewidth{0.602250pt}%
\definecolor{currentstroke}{rgb}{0.000000,0.000000,0.000000}%
\pgfsetstrokecolor{currentstroke}%
\pgfsetdash{}{0pt}%
\pgfsys@defobject{currentmarker}{\pgfqpoint{-0.027778in}{0.000000in}}{\pgfqpoint{-0.000000in}{0.000000in}}{%
\pgfpathmoveto{\pgfqpoint{-0.000000in}{0.000000in}}%
\pgfpathlineto{\pgfqpoint{-0.027778in}{0.000000in}}%
\pgfusepath{stroke,fill}%
}%
\begin{pgfscope}%
\pgfsys@transformshift{0.589510in}{0.807238in}%
\pgfsys@useobject{currentmarker}{}%
\end{pgfscope}%
\end{pgfscope}%
\begin{pgfscope}%
\pgfsetbuttcap%
\pgfsetroundjoin%
\definecolor{currentfill}{rgb}{0.000000,0.000000,0.000000}%
\pgfsetfillcolor{currentfill}%
\pgfsetlinewidth{0.602250pt}%
\definecolor{currentstroke}{rgb}{0.000000,0.000000,0.000000}%
\pgfsetstrokecolor{currentstroke}%
\pgfsetdash{}{0pt}%
\pgfsys@defobject{currentmarker}{\pgfqpoint{-0.027778in}{0.000000in}}{\pgfqpoint{-0.000000in}{0.000000in}}{%
\pgfpathmoveto{\pgfqpoint{-0.000000in}{0.000000in}}%
\pgfpathlineto{\pgfqpoint{-0.027778in}{0.000000in}}%
\pgfusepath{stroke,fill}%
}%
\begin{pgfscope}%
\pgfsys@transformshift{0.589510in}{0.817710in}%
\pgfsys@useobject{currentmarker}{}%
\end{pgfscope}%
\end{pgfscope}%
\begin{pgfscope}%
\pgfsetbuttcap%
\pgfsetroundjoin%
\definecolor{currentfill}{rgb}{0.000000,0.000000,0.000000}%
\pgfsetfillcolor{currentfill}%
\pgfsetlinewidth{0.602250pt}%
\definecolor{currentstroke}{rgb}{0.000000,0.000000,0.000000}%
\pgfsetstrokecolor{currentstroke}%
\pgfsetdash{}{0pt}%
\pgfsys@defobject{currentmarker}{\pgfqpoint{-0.027778in}{0.000000in}}{\pgfqpoint{-0.000000in}{0.000000in}}{%
\pgfpathmoveto{\pgfqpoint{-0.000000in}{0.000000in}}%
\pgfpathlineto{\pgfqpoint{-0.027778in}{0.000000in}}%
\pgfusepath{stroke,fill}%
}%
\begin{pgfscope}%
\pgfsys@transformshift{0.589510in}{0.888703in}%
\pgfsys@useobject{currentmarker}{}%
\end{pgfscope}%
\end{pgfscope}%
\begin{pgfscope}%
\pgfsetbuttcap%
\pgfsetroundjoin%
\definecolor{currentfill}{rgb}{0.000000,0.000000,0.000000}%
\pgfsetfillcolor{currentfill}%
\pgfsetlinewidth{0.602250pt}%
\definecolor{currentstroke}{rgb}{0.000000,0.000000,0.000000}%
\pgfsetstrokecolor{currentstroke}%
\pgfsetdash{}{0pt}%
\pgfsys@defobject{currentmarker}{\pgfqpoint{-0.027778in}{0.000000in}}{\pgfqpoint{-0.000000in}{0.000000in}}{%
\pgfpathmoveto{\pgfqpoint{-0.000000in}{0.000000in}}%
\pgfpathlineto{\pgfqpoint{-0.027778in}{0.000000in}}%
\pgfusepath{stroke,fill}%
}%
\begin{pgfscope}%
\pgfsys@transformshift{0.589510in}{0.924752in}%
\pgfsys@useobject{currentmarker}{}%
\end{pgfscope}%
\end{pgfscope}%
\begin{pgfscope}%
\pgfsetbuttcap%
\pgfsetroundjoin%
\definecolor{currentfill}{rgb}{0.000000,0.000000,0.000000}%
\pgfsetfillcolor{currentfill}%
\pgfsetlinewidth{0.602250pt}%
\definecolor{currentstroke}{rgb}{0.000000,0.000000,0.000000}%
\pgfsetstrokecolor{currentstroke}%
\pgfsetdash{}{0pt}%
\pgfsys@defobject{currentmarker}{\pgfqpoint{-0.027778in}{0.000000in}}{\pgfqpoint{-0.000000in}{0.000000in}}{%
\pgfpathmoveto{\pgfqpoint{-0.000000in}{0.000000in}}%
\pgfpathlineto{\pgfqpoint{-0.027778in}{0.000000in}}%
\pgfusepath{stroke,fill}%
}%
\begin{pgfscope}%
\pgfsys@transformshift{0.589510in}{0.950329in}%
\pgfsys@useobject{currentmarker}{}%
\end{pgfscope}%
\end{pgfscope}%
\begin{pgfscope}%
\pgfsetbuttcap%
\pgfsetroundjoin%
\definecolor{currentfill}{rgb}{0.000000,0.000000,0.000000}%
\pgfsetfillcolor{currentfill}%
\pgfsetlinewidth{0.602250pt}%
\definecolor{currentstroke}{rgb}{0.000000,0.000000,0.000000}%
\pgfsetstrokecolor{currentstroke}%
\pgfsetdash{}{0pt}%
\pgfsys@defobject{currentmarker}{\pgfqpoint{-0.027778in}{0.000000in}}{\pgfqpoint{-0.000000in}{0.000000in}}{%
\pgfpathmoveto{\pgfqpoint{-0.000000in}{0.000000in}}%
\pgfpathlineto{\pgfqpoint{-0.027778in}{0.000000in}}%
\pgfusepath{stroke,fill}%
}%
\begin{pgfscope}%
\pgfsys@transformshift{0.589510in}{0.970168in}%
\pgfsys@useobject{currentmarker}{}%
\end{pgfscope}%
\end{pgfscope}%
\begin{pgfscope}%
\pgfsetbuttcap%
\pgfsetroundjoin%
\definecolor{currentfill}{rgb}{0.000000,0.000000,0.000000}%
\pgfsetfillcolor{currentfill}%
\pgfsetlinewidth{0.602250pt}%
\definecolor{currentstroke}{rgb}{0.000000,0.000000,0.000000}%
\pgfsetstrokecolor{currentstroke}%
\pgfsetdash{}{0pt}%
\pgfsys@defobject{currentmarker}{\pgfqpoint{-0.027778in}{0.000000in}}{\pgfqpoint{-0.000000in}{0.000000in}}{%
\pgfpathmoveto{\pgfqpoint{-0.000000in}{0.000000in}}%
\pgfpathlineto{\pgfqpoint{-0.027778in}{0.000000in}}%
\pgfusepath{stroke,fill}%
}%
\begin{pgfscope}%
\pgfsys@transformshift{0.589510in}{0.986378in}%
\pgfsys@useobject{currentmarker}{}%
\end{pgfscope}%
\end{pgfscope}%
\begin{pgfscope}%
\pgfsetbuttcap%
\pgfsetroundjoin%
\definecolor{currentfill}{rgb}{0.000000,0.000000,0.000000}%
\pgfsetfillcolor{currentfill}%
\pgfsetlinewidth{0.602250pt}%
\definecolor{currentstroke}{rgb}{0.000000,0.000000,0.000000}%
\pgfsetstrokecolor{currentstroke}%
\pgfsetdash{}{0pt}%
\pgfsys@defobject{currentmarker}{\pgfqpoint{-0.027778in}{0.000000in}}{\pgfqpoint{-0.000000in}{0.000000in}}{%
\pgfpathmoveto{\pgfqpoint{-0.000000in}{0.000000in}}%
\pgfpathlineto{\pgfqpoint{-0.027778in}{0.000000in}}%
\pgfusepath{stroke,fill}%
}%
\begin{pgfscope}%
\pgfsys@transformshift{0.589510in}{1.000083in}%
\pgfsys@useobject{currentmarker}{}%
\end{pgfscope}%
\end{pgfscope}%
\begin{pgfscope}%
\pgfsetbuttcap%
\pgfsetroundjoin%
\definecolor{currentfill}{rgb}{0.000000,0.000000,0.000000}%
\pgfsetfillcolor{currentfill}%
\pgfsetlinewidth{0.602250pt}%
\definecolor{currentstroke}{rgb}{0.000000,0.000000,0.000000}%
\pgfsetstrokecolor{currentstroke}%
\pgfsetdash{}{0pt}%
\pgfsys@defobject{currentmarker}{\pgfqpoint{-0.027778in}{0.000000in}}{\pgfqpoint{-0.000000in}{0.000000in}}{%
\pgfpathmoveto{\pgfqpoint{-0.000000in}{0.000000in}}%
\pgfpathlineto{\pgfqpoint{-0.027778in}{0.000000in}}%
\pgfusepath{stroke,fill}%
}%
\begin{pgfscope}%
\pgfsys@transformshift{0.589510in}{1.011955in}%
\pgfsys@useobject{currentmarker}{}%
\end{pgfscope}%
\end{pgfscope}%
\begin{pgfscope}%
\pgfsetbuttcap%
\pgfsetroundjoin%
\definecolor{currentfill}{rgb}{0.000000,0.000000,0.000000}%
\pgfsetfillcolor{currentfill}%
\pgfsetlinewidth{0.602250pt}%
\definecolor{currentstroke}{rgb}{0.000000,0.000000,0.000000}%
\pgfsetstrokecolor{currentstroke}%
\pgfsetdash{}{0pt}%
\pgfsys@defobject{currentmarker}{\pgfqpoint{-0.027778in}{0.000000in}}{\pgfqpoint{-0.000000in}{0.000000in}}{%
\pgfpathmoveto{\pgfqpoint{-0.000000in}{0.000000in}}%
\pgfpathlineto{\pgfqpoint{-0.027778in}{0.000000in}}%
\pgfusepath{stroke,fill}%
}%
\begin{pgfscope}%
\pgfsys@transformshift{0.589510in}{1.022427in}%
\pgfsys@useobject{currentmarker}{}%
\end{pgfscope}%
\end{pgfscope}%
\begin{pgfscope}%
\pgfsetbuttcap%
\pgfsetroundjoin%
\definecolor{currentfill}{rgb}{0.000000,0.000000,0.000000}%
\pgfsetfillcolor{currentfill}%
\pgfsetlinewidth{0.602250pt}%
\definecolor{currentstroke}{rgb}{0.000000,0.000000,0.000000}%
\pgfsetstrokecolor{currentstroke}%
\pgfsetdash{}{0pt}%
\pgfsys@defobject{currentmarker}{\pgfqpoint{-0.027778in}{0.000000in}}{\pgfqpoint{-0.000000in}{0.000000in}}{%
\pgfpathmoveto{\pgfqpoint{-0.000000in}{0.000000in}}%
\pgfpathlineto{\pgfqpoint{-0.027778in}{0.000000in}}%
\pgfusepath{stroke,fill}%
}%
\begin{pgfscope}%
\pgfsys@transformshift{0.589510in}{1.093421in}%
\pgfsys@useobject{currentmarker}{}%
\end{pgfscope}%
\end{pgfscope}%
\begin{pgfscope}%
\pgfsetbuttcap%
\pgfsetroundjoin%
\definecolor{currentfill}{rgb}{0.000000,0.000000,0.000000}%
\pgfsetfillcolor{currentfill}%
\pgfsetlinewidth{0.602250pt}%
\definecolor{currentstroke}{rgb}{0.000000,0.000000,0.000000}%
\pgfsetstrokecolor{currentstroke}%
\pgfsetdash{}{0pt}%
\pgfsys@defobject{currentmarker}{\pgfqpoint{-0.027778in}{0.000000in}}{\pgfqpoint{-0.000000in}{0.000000in}}{%
\pgfpathmoveto{\pgfqpoint{-0.000000in}{0.000000in}}%
\pgfpathlineto{\pgfqpoint{-0.027778in}{0.000000in}}%
\pgfusepath{stroke,fill}%
}%
\begin{pgfscope}%
\pgfsys@transformshift{0.589510in}{1.129470in}%
\pgfsys@useobject{currentmarker}{}%
\end{pgfscope}%
\end{pgfscope}%
\begin{pgfscope}%
\pgfsetbuttcap%
\pgfsetroundjoin%
\definecolor{currentfill}{rgb}{0.000000,0.000000,0.000000}%
\pgfsetfillcolor{currentfill}%
\pgfsetlinewidth{0.602250pt}%
\definecolor{currentstroke}{rgb}{0.000000,0.000000,0.000000}%
\pgfsetstrokecolor{currentstroke}%
\pgfsetdash{}{0pt}%
\pgfsys@defobject{currentmarker}{\pgfqpoint{-0.027778in}{0.000000in}}{\pgfqpoint{-0.000000in}{0.000000in}}{%
\pgfpathmoveto{\pgfqpoint{-0.000000in}{0.000000in}}%
\pgfpathlineto{\pgfqpoint{-0.027778in}{0.000000in}}%
\pgfusepath{stroke,fill}%
}%
\begin{pgfscope}%
\pgfsys@transformshift{0.589510in}{1.155047in}%
\pgfsys@useobject{currentmarker}{}%
\end{pgfscope}%
\end{pgfscope}%
\begin{pgfscope}%
\pgfsetbuttcap%
\pgfsetroundjoin%
\definecolor{currentfill}{rgb}{0.000000,0.000000,0.000000}%
\pgfsetfillcolor{currentfill}%
\pgfsetlinewidth{0.602250pt}%
\definecolor{currentstroke}{rgb}{0.000000,0.000000,0.000000}%
\pgfsetstrokecolor{currentstroke}%
\pgfsetdash{}{0pt}%
\pgfsys@defobject{currentmarker}{\pgfqpoint{-0.027778in}{0.000000in}}{\pgfqpoint{-0.000000in}{0.000000in}}{%
\pgfpathmoveto{\pgfqpoint{-0.000000in}{0.000000in}}%
\pgfpathlineto{\pgfqpoint{-0.027778in}{0.000000in}}%
\pgfusepath{stroke,fill}%
}%
\begin{pgfscope}%
\pgfsys@transformshift{0.589510in}{1.174886in}%
\pgfsys@useobject{currentmarker}{}%
\end{pgfscope}%
\end{pgfscope}%
\begin{pgfscope}%
\pgfsetbuttcap%
\pgfsetroundjoin%
\definecolor{currentfill}{rgb}{0.000000,0.000000,0.000000}%
\pgfsetfillcolor{currentfill}%
\pgfsetlinewidth{0.602250pt}%
\definecolor{currentstroke}{rgb}{0.000000,0.000000,0.000000}%
\pgfsetstrokecolor{currentstroke}%
\pgfsetdash{}{0pt}%
\pgfsys@defobject{currentmarker}{\pgfqpoint{-0.027778in}{0.000000in}}{\pgfqpoint{-0.000000in}{0.000000in}}{%
\pgfpathmoveto{\pgfqpoint{-0.000000in}{0.000000in}}%
\pgfpathlineto{\pgfqpoint{-0.027778in}{0.000000in}}%
\pgfusepath{stroke,fill}%
}%
\begin{pgfscope}%
\pgfsys@transformshift{0.589510in}{1.191096in}%
\pgfsys@useobject{currentmarker}{}%
\end{pgfscope}%
\end{pgfscope}%
\begin{pgfscope}%
\pgfsetbuttcap%
\pgfsetroundjoin%
\definecolor{currentfill}{rgb}{0.000000,0.000000,0.000000}%
\pgfsetfillcolor{currentfill}%
\pgfsetlinewidth{0.602250pt}%
\definecolor{currentstroke}{rgb}{0.000000,0.000000,0.000000}%
\pgfsetstrokecolor{currentstroke}%
\pgfsetdash{}{0pt}%
\pgfsys@defobject{currentmarker}{\pgfqpoint{-0.027778in}{0.000000in}}{\pgfqpoint{-0.000000in}{0.000000in}}{%
\pgfpathmoveto{\pgfqpoint{-0.000000in}{0.000000in}}%
\pgfpathlineto{\pgfqpoint{-0.027778in}{0.000000in}}%
\pgfusepath{stroke,fill}%
}%
\begin{pgfscope}%
\pgfsys@transformshift{0.589510in}{1.204801in}%
\pgfsys@useobject{currentmarker}{}%
\end{pgfscope}%
\end{pgfscope}%
\begin{pgfscope}%
\pgfsetbuttcap%
\pgfsetroundjoin%
\definecolor{currentfill}{rgb}{0.000000,0.000000,0.000000}%
\pgfsetfillcolor{currentfill}%
\pgfsetlinewidth{0.602250pt}%
\definecolor{currentstroke}{rgb}{0.000000,0.000000,0.000000}%
\pgfsetstrokecolor{currentstroke}%
\pgfsetdash{}{0pt}%
\pgfsys@defobject{currentmarker}{\pgfqpoint{-0.027778in}{0.000000in}}{\pgfqpoint{-0.000000in}{0.000000in}}{%
\pgfpathmoveto{\pgfqpoint{-0.000000in}{0.000000in}}%
\pgfpathlineto{\pgfqpoint{-0.027778in}{0.000000in}}%
\pgfusepath{stroke,fill}%
}%
\begin{pgfscope}%
\pgfsys@transformshift{0.589510in}{1.216673in}%
\pgfsys@useobject{currentmarker}{}%
\end{pgfscope}%
\end{pgfscope}%
\begin{pgfscope}%
\pgfsetbuttcap%
\pgfsetroundjoin%
\definecolor{currentfill}{rgb}{0.000000,0.000000,0.000000}%
\pgfsetfillcolor{currentfill}%
\pgfsetlinewidth{0.602250pt}%
\definecolor{currentstroke}{rgb}{0.000000,0.000000,0.000000}%
\pgfsetstrokecolor{currentstroke}%
\pgfsetdash{}{0pt}%
\pgfsys@defobject{currentmarker}{\pgfqpoint{-0.027778in}{0.000000in}}{\pgfqpoint{-0.000000in}{0.000000in}}{%
\pgfpathmoveto{\pgfqpoint{-0.000000in}{0.000000in}}%
\pgfpathlineto{\pgfqpoint{-0.027778in}{0.000000in}}%
\pgfusepath{stroke,fill}%
}%
\begin{pgfscope}%
\pgfsys@transformshift{0.589510in}{1.227145in}%
\pgfsys@useobject{currentmarker}{}%
\end{pgfscope}%
\end{pgfscope}%
\begin{pgfscope}%
\pgfsetbuttcap%
\pgfsetroundjoin%
\definecolor{currentfill}{rgb}{0.000000,0.000000,0.000000}%
\pgfsetfillcolor{currentfill}%
\pgfsetlinewidth{0.602250pt}%
\definecolor{currentstroke}{rgb}{0.000000,0.000000,0.000000}%
\pgfsetstrokecolor{currentstroke}%
\pgfsetdash{}{0pt}%
\pgfsys@defobject{currentmarker}{\pgfqpoint{-0.027778in}{0.000000in}}{\pgfqpoint{-0.000000in}{0.000000in}}{%
\pgfpathmoveto{\pgfqpoint{-0.000000in}{0.000000in}}%
\pgfpathlineto{\pgfqpoint{-0.027778in}{0.000000in}}%
\pgfusepath{stroke,fill}%
}%
\begin{pgfscope}%
\pgfsys@transformshift{0.589510in}{1.298138in}%
\pgfsys@useobject{currentmarker}{}%
\end{pgfscope}%
\end{pgfscope}%
\begin{pgfscope}%
\pgfsetbuttcap%
\pgfsetroundjoin%
\definecolor{currentfill}{rgb}{0.000000,0.000000,0.000000}%
\pgfsetfillcolor{currentfill}%
\pgfsetlinewidth{0.602250pt}%
\definecolor{currentstroke}{rgb}{0.000000,0.000000,0.000000}%
\pgfsetstrokecolor{currentstroke}%
\pgfsetdash{}{0pt}%
\pgfsys@defobject{currentmarker}{\pgfqpoint{-0.027778in}{0.000000in}}{\pgfqpoint{-0.000000in}{0.000000in}}{%
\pgfpathmoveto{\pgfqpoint{-0.000000in}{0.000000in}}%
\pgfpathlineto{\pgfqpoint{-0.027778in}{0.000000in}}%
\pgfusepath{stroke,fill}%
}%
\begin{pgfscope}%
\pgfsys@transformshift{0.589510in}{1.334187in}%
\pgfsys@useobject{currentmarker}{}%
\end{pgfscope}%
\end{pgfscope}%
\begin{pgfscope}%
\pgfsetbuttcap%
\pgfsetroundjoin%
\definecolor{currentfill}{rgb}{0.000000,0.000000,0.000000}%
\pgfsetfillcolor{currentfill}%
\pgfsetlinewidth{0.602250pt}%
\definecolor{currentstroke}{rgb}{0.000000,0.000000,0.000000}%
\pgfsetstrokecolor{currentstroke}%
\pgfsetdash{}{0pt}%
\pgfsys@defobject{currentmarker}{\pgfqpoint{-0.027778in}{0.000000in}}{\pgfqpoint{-0.000000in}{0.000000in}}{%
\pgfpathmoveto{\pgfqpoint{-0.000000in}{0.000000in}}%
\pgfpathlineto{\pgfqpoint{-0.027778in}{0.000000in}}%
\pgfusepath{stroke,fill}%
}%
\begin{pgfscope}%
\pgfsys@transformshift{0.589510in}{1.359764in}%
\pgfsys@useobject{currentmarker}{}%
\end{pgfscope}%
\end{pgfscope}%
\begin{pgfscope}%
\pgfsetbuttcap%
\pgfsetroundjoin%
\definecolor{currentfill}{rgb}{0.000000,0.000000,0.000000}%
\pgfsetfillcolor{currentfill}%
\pgfsetlinewidth{0.602250pt}%
\definecolor{currentstroke}{rgb}{0.000000,0.000000,0.000000}%
\pgfsetstrokecolor{currentstroke}%
\pgfsetdash{}{0pt}%
\pgfsys@defobject{currentmarker}{\pgfqpoint{-0.027778in}{0.000000in}}{\pgfqpoint{-0.000000in}{0.000000in}}{%
\pgfpathmoveto{\pgfqpoint{-0.000000in}{0.000000in}}%
\pgfpathlineto{\pgfqpoint{-0.027778in}{0.000000in}}%
\pgfusepath{stroke,fill}%
}%
\begin{pgfscope}%
\pgfsys@transformshift{0.589510in}{1.379604in}%
\pgfsys@useobject{currentmarker}{}%
\end{pgfscope}%
\end{pgfscope}%
\begin{pgfscope}%
\pgfsetbuttcap%
\pgfsetroundjoin%
\definecolor{currentfill}{rgb}{0.000000,0.000000,0.000000}%
\pgfsetfillcolor{currentfill}%
\pgfsetlinewidth{0.602250pt}%
\definecolor{currentstroke}{rgb}{0.000000,0.000000,0.000000}%
\pgfsetstrokecolor{currentstroke}%
\pgfsetdash{}{0pt}%
\pgfsys@defobject{currentmarker}{\pgfqpoint{-0.027778in}{0.000000in}}{\pgfqpoint{-0.000000in}{0.000000in}}{%
\pgfpathmoveto{\pgfqpoint{-0.000000in}{0.000000in}}%
\pgfpathlineto{\pgfqpoint{-0.027778in}{0.000000in}}%
\pgfusepath{stroke,fill}%
}%
\begin{pgfscope}%
\pgfsys@transformshift{0.589510in}{1.395813in}%
\pgfsys@useobject{currentmarker}{}%
\end{pgfscope}%
\end{pgfscope}%
\begin{pgfscope}%
\pgfsetbuttcap%
\pgfsetroundjoin%
\definecolor{currentfill}{rgb}{0.000000,0.000000,0.000000}%
\pgfsetfillcolor{currentfill}%
\pgfsetlinewidth{0.602250pt}%
\definecolor{currentstroke}{rgb}{0.000000,0.000000,0.000000}%
\pgfsetstrokecolor{currentstroke}%
\pgfsetdash{}{0pt}%
\pgfsys@defobject{currentmarker}{\pgfqpoint{-0.027778in}{0.000000in}}{\pgfqpoint{-0.000000in}{0.000000in}}{%
\pgfpathmoveto{\pgfqpoint{-0.000000in}{0.000000in}}%
\pgfpathlineto{\pgfqpoint{-0.027778in}{0.000000in}}%
\pgfusepath{stroke,fill}%
}%
\begin{pgfscope}%
\pgfsys@transformshift{0.589510in}{1.409519in}%
\pgfsys@useobject{currentmarker}{}%
\end{pgfscope}%
\end{pgfscope}%
\begin{pgfscope}%
\pgfsetbuttcap%
\pgfsetroundjoin%
\definecolor{currentfill}{rgb}{0.000000,0.000000,0.000000}%
\pgfsetfillcolor{currentfill}%
\pgfsetlinewidth{0.602250pt}%
\definecolor{currentstroke}{rgb}{0.000000,0.000000,0.000000}%
\pgfsetstrokecolor{currentstroke}%
\pgfsetdash{}{0pt}%
\pgfsys@defobject{currentmarker}{\pgfqpoint{-0.027778in}{0.000000in}}{\pgfqpoint{-0.000000in}{0.000000in}}{%
\pgfpathmoveto{\pgfqpoint{-0.000000in}{0.000000in}}%
\pgfpathlineto{\pgfqpoint{-0.027778in}{0.000000in}}%
\pgfusepath{stroke,fill}%
}%
\begin{pgfscope}%
\pgfsys@transformshift{0.589510in}{1.421391in}%
\pgfsys@useobject{currentmarker}{}%
\end{pgfscope}%
\end{pgfscope}%
\begin{pgfscope}%
\pgfsetbuttcap%
\pgfsetroundjoin%
\definecolor{currentfill}{rgb}{0.000000,0.000000,0.000000}%
\pgfsetfillcolor{currentfill}%
\pgfsetlinewidth{0.602250pt}%
\definecolor{currentstroke}{rgb}{0.000000,0.000000,0.000000}%
\pgfsetstrokecolor{currentstroke}%
\pgfsetdash{}{0pt}%
\pgfsys@defobject{currentmarker}{\pgfqpoint{-0.027778in}{0.000000in}}{\pgfqpoint{-0.000000in}{0.000000in}}{%
\pgfpathmoveto{\pgfqpoint{-0.000000in}{0.000000in}}%
\pgfpathlineto{\pgfqpoint{-0.027778in}{0.000000in}}%
\pgfusepath{stroke,fill}%
}%
\begin{pgfscope}%
\pgfsys@transformshift{0.589510in}{1.431862in}%
\pgfsys@useobject{currentmarker}{}%
\end{pgfscope}%
\end{pgfscope}%
\begin{pgfscope}%
\pgfsetbuttcap%
\pgfsetroundjoin%
\definecolor{currentfill}{rgb}{0.000000,0.000000,0.000000}%
\pgfsetfillcolor{currentfill}%
\pgfsetlinewidth{0.602250pt}%
\definecolor{currentstroke}{rgb}{0.000000,0.000000,0.000000}%
\pgfsetstrokecolor{currentstroke}%
\pgfsetdash{}{0pt}%
\pgfsys@defobject{currentmarker}{\pgfqpoint{-0.027778in}{0.000000in}}{\pgfqpoint{-0.000000in}{0.000000in}}{%
\pgfpathmoveto{\pgfqpoint{-0.000000in}{0.000000in}}%
\pgfpathlineto{\pgfqpoint{-0.027778in}{0.000000in}}%
\pgfusepath{stroke,fill}%
}%
\begin{pgfscope}%
\pgfsys@transformshift{0.589510in}{1.502856in}%
\pgfsys@useobject{currentmarker}{}%
\end{pgfscope}%
\end{pgfscope}%
\begin{pgfscope}%
\pgfsetbuttcap%
\pgfsetroundjoin%
\definecolor{currentfill}{rgb}{0.000000,0.000000,0.000000}%
\pgfsetfillcolor{currentfill}%
\pgfsetlinewidth{0.602250pt}%
\definecolor{currentstroke}{rgb}{0.000000,0.000000,0.000000}%
\pgfsetstrokecolor{currentstroke}%
\pgfsetdash{}{0pt}%
\pgfsys@defobject{currentmarker}{\pgfqpoint{-0.027778in}{0.000000in}}{\pgfqpoint{-0.000000in}{0.000000in}}{%
\pgfpathmoveto{\pgfqpoint{-0.000000in}{0.000000in}}%
\pgfpathlineto{\pgfqpoint{-0.027778in}{0.000000in}}%
\pgfusepath{stroke,fill}%
}%
\begin{pgfscope}%
\pgfsys@transformshift{0.589510in}{1.538905in}%
\pgfsys@useobject{currentmarker}{}%
\end{pgfscope}%
\end{pgfscope}%
\begin{pgfscope}%
\pgfsetbuttcap%
\pgfsetroundjoin%
\definecolor{currentfill}{rgb}{0.000000,0.000000,0.000000}%
\pgfsetfillcolor{currentfill}%
\pgfsetlinewidth{0.602250pt}%
\definecolor{currentstroke}{rgb}{0.000000,0.000000,0.000000}%
\pgfsetstrokecolor{currentstroke}%
\pgfsetdash{}{0pt}%
\pgfsys@defobject{currentmarker}{\pgfqpoint{-0.027778in}{0.000000in}}{\pgfqpoint{-0.000000in}{0.000000in}}{%
\pgfpathmoveto{\pgfqpoint{-0.000000in}{0.000000in}}%
\pgfpathlineto{\pgfqpoint{-0.027778in}{0.000000in}}%
\pgfusepath{stroke,fill}%
}%
\begin{pgfscope}%
\pgfsys@transformshift{0.589510in}{1.564482in}%
\pgfsys@useobject{currentmarker}{}%
\end{pgfscope}%
\end{pgfscope}%
\begin{pgfscope}%
\pgfsetbuttcap%
\pgfsetroundjoin%
\definecolor{currentfill}{rgb}{0.000000,0.000000,0.000000}%
\pgfsetfillcolor{currentfill}%
\pgfsetlinewidth{0.602250pt}%
\definecolor{currentstroke}{rgb}{0.000000,0.000000,0.000000}%
\pgfsetstrokecolor{currentstroke}%
\pgfsetdash{}{0pt}%
\pgfsys@defobject{currentmarker}{\pgfqpoint{-0.027778in}{0.000000in}}{\pgfqpoint{-0.000000in}{0.000000in}}{%
\pgfpathmoveto{\pgfqpoint{-0.000000in}{0.000000in}}%
\pgfpathlineto{\pgfqpoint{-0.027778in}{0.000000in}}%
\pgfusepath{stroke,fill}%
}%
\begin{pgfscope}%
\pgfsys@transformshift{0.589510in}{1.584321in}%
\pgfsys@useobject{currentmarker}{}%
\end{pgfscope}%
\end{pgfscope}%
\begin{pgfscope}%
\pgfsetbuttcap%
\pgfsetroundjoin%
\definecolor{currentfill}{rgb}{0.000000,0.000000,0.000000}%
\pgfsetfillcolor{currentfill}%
\pgfsetlinewidth{0.602250pt}%
\definecolor{currentstroke}{rgb}{0.000000,0.000000,0.000000}%
\pgfsetstrokecolor{currentstroke}%
\pgfsetdash{}{0pt}%
\pgfsys@defobject{currentmarker}{\pgfqpoint{-0.027778in}{0.000000in}}{\pgfqpoint{-0.000000in}{0.000000in}}{%
\pgfpathmoveto{\pgfqpoint{-0.000000in}{0.000000in}}%
\pgfpathlineto{\pgfqpoint{-0.027778in}{0.000000in}}%
\pgfusepath{stroke,fill}%
}%
\begin{pgfscope}%
\pgfsys@transformshift{0.589510in}{1.600531in}%
\pgfsys@useobject{currentmarker}{}%
\end{pgfscope}%
\end{pgfscope}%
\begin{pgfscope}%
\pgfsetbuttcap%
\pgfsetroundjoin%
\definecolor{currentfill}{rgb}{0.000000,0.000000,0.000000}%
\pgfsetfillcolor{currentfill}%
\pgfsetlinewidth{0.602250pt}%
\definecolor{currentstroke}{rgb}{0.000000,0.000000,0.000000}%
\pgfsetstrokecolor{currentstroke}%
\pgfsetdash{}{0pt}%
\pgfsys@defobject{currentmarker}{\pgfqpoint{-0.027778in}{0.000000in}}{\pgfqpoint{-0.000000in}{0.000000in}}{%
\pgfpathmoveto{\pgfqpoint{-0.000000in}{0.000000in}}%
\pgfpathlineto{\pgfqpoint{-0.027778in}{0.000000in}}%
\pgfusepath{stroke,fill}%
}%
\begin{pgfscope}%
\pgfsys@transformshift{0.589510in}{1.614236in}%
\pgfsys@useobject{currentmarker}{}%
\end{pgfscope}%
\end{pgfscope}%
\begin{pgfscope}%
\pgfsetbuttcap%
\pgfsetroundjoin%
\definecolor{currentfill}{rgb}{0.000000,0.000000,0.000000}%
\pgfsetfillcolor{currentfill}%
\pgfsetlinewidth{0.602250pt}%
\definecolor{currentstroke}{rgb}{0.000000,0.000000,0.000000}%
\pgfsetstrokecolor{currentstroke}%
\pgfsetdash{}{0pt}%
\pgfsys@defobject{currentmarker}{\pgfqpoint{-0.027778in}{0.000000in}}{\pgfqpoint{-0.000000in}{0.000000in}}{%
\pgfpathmoveto{\pgfqpoint{-0.000000in}{0.000000in}}%
\pgfpathlineto{\pgfqpoint{-0.027778in}{0.000000in}}%
\pgfusepath{stroke,fill}%
}%
\begin{pgfscope}%
\pgfsys@transformshift{0.589510in}{1.626108in}%
\pgfsys@useobject{currentmarker}{}%
\end{pgfscope}%
\end{pgfscope}%
\begin{pgfscope}%
\pgfsetbuttcap%
\pgfsetroundjoin%
\definecolor{currentfill}{rgb}{0.000000,0.000000,0.000000}%
\pgfsetfillcolor{currentfill}%
\pgfsetlinewidth{0.602250pt}%
\definecolor{currentstroke}{rgb}{0.000000,0.000000,0.000000}%
\pgfsetstrokecolor{currentstroke}%
\pgfsetdash{}{0pt}%
\pgfsys@defobject{currentmarker}{\pgfqpoint{-0.027778in}{0.000000in}}{\pgfqpoint{-0.000000in}{0.000000in}}{%
\pgfpathmoveto{\pgfqpoint{-0.000000in}{0.000000in}}%
\pgfpathlineto{\pgfqpoint{-0.027778in}{0.000000in}}%
\pgfusepath{stroke,fill}%
}%
\begin{pgfscope}%
\pgfsys@transformshift{0.589510in}{1.636580in}%
\pgfsys@useobject{currentmarker}{}%
\end{pgfscope}%
\end{pgfscope}%
\begin{pgfscope}%
\pgfsetbuttcap%
\pgfsetroundjoin%
\definecolor{currentfill}{rgb}{0.000000,0.000000,0.000000}%
\pgfsetfillcolor{currentfill}%
\pgfsetlinewidth{0.602250pt}%
\definecolor{currentstroke}{rgb}{0.000000,0.000000,0.000000}%
\pgfsetstrokecolor{currentstroke}%
\pgfsetdash{}{0pt}%
\pgfsys@defobject{currentmarker}{\pgfqpoint{-0.027778in}{0.000000in}}{\pgfqpoint{-0.000000in}{0.000000in}}{%
\pgfpathmoveto{\pgfqpoint{-0.000000in}{0.000000in}}%
\pgfpathlineto{\pgfqpoint{-0.027778in}{0.000000in}}%
\pgfusepath{stroke,fill}%
}%
\begin{pgfscope}%
\pgfsys@transformshift{0.589510in}{1.707573in}%
\pgfsys@useobject{currentmarker}{}%
\end{pgfscope}%
\end{pgfscope}%
\begin{pgfscope}%
\pgfsetbuttcap%
\pgfsetroundjoin%
\definecolor{currentfill}{rgb}{0.000000,0.000000,0.000000}%
\pgfsetfillcolor{currentfill}%
\pgfsetlinewidth{0.602250pt}%
\definecolor{currentstroke}{rgb}{0.000000,0.000000,0.000000}%
\pgfsetstrokecolor{currentstroke}%
\pgfsetdash{}{0pt}%
\pgfsys@defobject{currentmarker}{\pgfqpoint{-0.027778in}{0.000000in}}{\pgfqpoint{-0.000000in}{0.000000in}}{%
\pgfpathmoveto{\pgfqpoint{-0.000000in}{0.000000in}}%
\pgfpathlineto{\pgfqpoint{-0.027778in}{0.000000in}}%
\pgfusepath{stroke,fill}%
}%
\begin{pgfscope}%
\pgfsys@transformshift{0.589510in}{1.743622in}%
\pgfsys@useobject{currentmarker}{}%
\end{pgfscope}%
\end{pgfscope}%
\begin{pgfscope}%
\pgfsetbuttcap%
\pgfsetroundjoin%
\definecolor{currentfill}{rgb}{0.000000,0.000000,0.000000}%
\pgfsetfillcolor{currentfill}%
\pgfsetlinewidth{0.602250pt}%
\definecolor{currentstroke}{rgb}{0.000000,0.000000,0.000000}%
\pgfsetstrokecolor{currentstroke}%
\pgfsetdash{}{0pt}%
\pgfsys@defobject{currentmarker}{\pgfqpoint{-0.027778in}{0.000000in}}{\pgfqpoint{-0.000000in}{0.000000in}}{%
\pgfpathmoveto{\pgfqpoint{-0.000000in}{0.000000in}}%
\pgfpathlineto{\pgfqpoint{-0.027778in}{0.000000in}}%
\pgfusepath{stroke,fill}%
}%
\begin{pgfscope}%
\pgfsys@transformshift{0.589510in}{1.769200in}%
\pgfsys@useobject{currentmarker}{}%
\end{pgfscope}%
\end{pgfscope}%
\begin{pgfscope}%
\pgfsetbuttcap%
\pgfsetroundjoin%
\definecolor{currentfill}{rgb}{0.000000,0.000000,0.000000}%
\pgfsetfillcolor{currentfill}%
\pgfsetlinewidth{0.602250pt}%
\definecolor{currentstroke}{rgb}{0.000000,0.000000,0.000000}%
\pgfsetstrokecolor{currentstroke}%
\pgfsetdash{}{0pt}%
\pgfsys@defobject{currentmarker}{\pgfqpoint{-0.027778in}{0.000000in}}{\pgfqpoint{-0.000000in}{0.000000in}}{%
\pgfpathmoveto{\pgfqpoint{-0.000000in}{0.000000in}}%
\pgfpathlineto{\pgfqpoint{-0.027778in}{0.000000in}}%
\pgfusepath{stroke,fill}%
}%
\begin{pgfscope}%
\pgfsys@transformshift{0.589510in}{1.789039in}%
\pgfsys@useobject{currentmarker}{}%
\end{pgfscope}%
\end{pgfscope}%
\begin{pgfscope}%
\definecolor{textcolor}{rgb}{0.000000,0.000000,0.000000}%
\pgfsetstrokecolor{textcolor}%
\pgfsetfillcolor{textcolor}%
\pgftext[x=0.180559in,y=1.103340in,,bottom,rotate=90.000000]{\color{textcolor}{\rmfamily\fontsize{10.000000}{12.000000}\selectfont\catcode`\^=\active\def^{\ifmmode\sp\else\^{}\fi}\catcode`\%=\active\def%{\%}ADEV $\sigma_A(\tau)$}}%
\end{pgfscope}%
\begin{pgfscope}%
\pgfpathrectangle{\pgfqpoint{0.589510in}{0.417642in}}{\pgfqpoint{1.809765in}{1.371397in}}%
\pgfusepath{clip}%
\pgfsetbuttcap%
\pgfsetroundjoin%
\pgfsetlinewidth{1.505625pt}%
\definecolor{currentstroke}{rgb}{0.800000,0.470588,0.737255}%
\pgfsetstrokecolor{currentstroke}%
\pgfsetdash{{5.550000pt}{2.400000pt}}{0.000000pt}%
\pgfpathmoveto{\pgfqpoint{0.671772in}{0.827077in}}%
\pgfpathlineto{\pgfqpoint{0.809267in}{0.888703in}}%
\pgfpathlineto{\pgfqpoint{0.946763in}{0.950329in}}%
\pgfpathlineto{\pgfqpoint{1.128522in}{1.031795in}}%
\pgfpathlineto{\pgfqpoint{1.266017in}{1.093421in}}%
\pgfpathlineto{\pgfqpoint{1.403513in}{1.155047in}}%
\pgfpathlineto{\pgfqpoint{1.585272in}{1.236512in}}%
\pgfpathlineto{\pgfqpoint{1.722767in}{1.298138in}}%
\pgfpathlineto{\pgfqpoint{1.860263in}{1.359764in}}%
\pgfpathlineto{\pgfqpoint{2.042022in}{1.441230in}}%
\pgfpathlineto{\pgfqpoint{2.179517in}{1.502856in}}%
\pgfpathlineto{\pgfqpoint{2.317013in}{1.564482in}}%
\pgfusepath{stroke}%
\end{pgfscope}%
\begin{pgfscope}%
\pgfpathrectangle{\pgfqpoint{0.589510in}{0.417642in}}{\pgfqpoint{1.809765in}{1.371397in}}%
\pgfusepath{clip}%
\pgfsetbuttcap%
\pgfsetroundjoin%
\definecolor{currentfill}{rgb}{0.800000,0.470588,0.737255}%
\pgfsetfillcolor{currentfill}%
\pgfsetlinewidth{1.003750pt}%
\definecolor{currentstroke}{rgb}{0.800000,0.470588,0.737255}%
\pgfsetstrokecolor{currentstroke}%
\pgfsetdash{}{0pt}%
\pgfsys@defobject{currentmarker}{\pgfqpoint{-0.020833in}{-0.020833in}}{\pgfqpoint{0.020833in}{0.020833in}}{%
\pgfpathmoveto{\pgfqpoint{0.000000in}{-0.020833in}}%
\pgfpathcurveto{\pgfqpoint{0.005525in}{-0.020833in}}{\pgfqpoint{0.010825in}{-0.018638in}}{\pgfqpoint{0.014731in}{-0.014731in}}%
\pgfpathcurveto{\pgfqpoint{0.018638in}{-0.010825in}}{\pgfqpoint{0.020833in}{-0.005525in}}{\pgfqpoint{0.020833in}{0.000000in}}%
\pgfpathcurveto{\pgfqpoint{0.020833in}{0.005525in}}{\pgfqpoint{0.018638in}{0.010825in}}{\pgfqpoint{0.014731in}{0.014731in}}%
\pgfpathcurveto{\pgfqpoint{0.010825in}{0.018638in}}{\pgfqpoint{0.005525in}{0.020833in}}{\pgfqpoint{0.000000in}{0.020833in}}%
\pgfpathcurveto{\pgfqpoint{-0.005525in}{0.020833in}}{\pgfqpoint{-0.010825in}{0.018638in}}{\pgfqpoint{-0.014731in}{0.014731in}}%
\pgfpathcurveto{\pgfqpoint{-0.018638in}{0.010825in}}{\pgfqpoint{-0.020833in}{0.005525in}}{\pgfqpoint{-0.020833in}{0.000000in}}%
\pgfpathcurveto{\pgfqpoint{-0.020833in}{-0.005525in}}{\pgfqpoint{-0.018638in}{-0.010825in}}{\pgfqpoint{-0.014731in}{-0.014731in}}%
\pgfpathcurveto{\pgfqpoint{-0.010825in}{-0.018638in}}{\pgfqpoint{-0.005525in}{-0.020833in}}{\pgfqpoint{0.000000in}{-0.020833in}}%
\pgfpathlineto{\pgfqpoint{0.000000in}{-0.020833in}}%
\pgfpathclose%
\pgfusepath{stroke,fill}%
}%
\begin{pgfscope}%
\pgfsys@transformshift{0.671772in}{0.827077in}%
\pgfsys@useobject{currentmarker}{}%
\end{pgfscope}%
\begin{pgfscope}%
\pgfsys@transformshift{0.809267in}{0.888703in}%
\pgfsys@useobject{currentmarker}{}%
\end{pgfscope}%
\begin{pgfscope}%
\pgfsys@transformshift{0.946763in}{0.950329in}%
\pgfsys@useobject{currentmarker}{}%
\end{pgfscope}%
\begin{pgfscope}%
\pgfsys@transformshift{1.128522in}{1.031795in}%
\pgfsys@useobject{currentmarker}{}%
\end{pgfscope}%
\begin{pgfscope}%
\pgfsys@transformshift{1.266017in}{1.093421in}%
\pgfsys@useobject{currentmarker}{}%
\end{pgfscope}%
\begin{pgfscope}%
\pgfsys@transformshift{1.403513in}{1.155047in}%
\pgfsys@useobject{currentmarker}{}%
\end{pgfscope}%
\begin{pgfscope}%
\pgfsys@transformshift{1.585272in}{1.236512in}%
\pgfsys@useobject{currentmarker}{}%
\end{pgfscope}%
\begin{pgfscope}%
\pgfsys@transformshift{1.722767in}{1.298138in}%
\pgfsys@useobject{currentmarker}{}%
\end{pgfscope}%
\begin{pgfscope}%
\pgfsys@transformshift{1.860263in}{1.359764in}%
\pgfsys@useobject{currentmarker}{}%
\end{pgfscope}%
\begin{pgfscope}%
\pgfsys@transformshift{2.042022in}{1.441230in}%
\pgfsys@useobject{currentmarker}{}%
\end{pgfscope}%
\begin{pgfscope}%
\pgfsys@transformshift{2.179517in}{1.502856in}%
\pgfsys@useobject{currentmarker}{}%
\end{pgfscope}%
\begin{pgfscope}%
\pgfsys@transformshift{2.317013in}{1.564482in}%
\pgfsys@useobject{currentmarker}{}%
\end{pgfscope}%
\end{pgfscope}%
\begin{pgfscope}%
\pgfsetrectcap%
\pgfsetmiterjoin%
\pgfsetlinewidth{0.803000pt}%
\definecolor{currentstroke}{rgb}{0.000000,0.000000,0.000000}%
\pgfsetstrokecolor{currentstroke}%
\pgfsetdash{}{0pt}%
\pgfpathmoveto{\pgfqpoint{0.589510in}{0.417642in}}%
\pgfpathlineto{\pgfqpoint{0.589510in}{1.789039in}}%
\pgfusepath{stroke}%
\end{pgfscope}%
\begin{pgfscope}%
\pgfsetrectcap%
\pgfsetmiterjoin%
\pgfsetlinewidth{0.803000pt}%
\definecolor{currentstroke}{rgb}{0.000000,0.000000,0.000000}%
\pgfsetstrokecolor{currentstroke}%
\pgfsetdash{}{0pt}%
\pgfpathmoveto{\pgfqpoint{2.399275in}{0.417642in}}%
\pgfpathlineto{\pgfqpoint{2.399275in}{1.789039in}}%
\pgfusepath{stroke}%
\end{pgfscope}%
\begin{pgfscope}%
\pgfsetrectcap%
\pgfsetmiterjoin%
\pgfsetlinewidth{0.803000pt}%
\definecolor{currentstroke}{rgb}{0.000000,0.000000,0.000000}%
\pgfsetstrokecolor{currentstroke}%
\pgfsetdash{}{0pt}%
\pgfpathmoveto{\pgfqpoint{0.589510in}{0.417642in}}%
\pgfpathlineto{\pgfqpoint{2.399275in}{0.417642in}}%
\pgfusepath{stroke}%
\end{pgfscope}%
\begin{pgfscope}%
\pgfsetrectcap%
\pgfsetmiterjoin%
\pgfsetlinewidth{0.803000pt}%
\definecolor{currentstroke}{rgb}{0.000000,0.000000,0.000000}%
\pgfsetstrokecolor{currentstroke}%
\pgfsetdash{}{0pt}%
\pgfpathmoveto{\pgfqpoint{0.589510in}{1.789039in}}%
\pgfpathlineto{\pgfqpoint{2.399275in}{1.789039in}}%
\pgfusepath{stroke}%
\end{pgfscope}%
\begin{pgfscope}%
\pgfsetbuttcap%
\pgfsetmiterjoin%
\definecolor{currentfill}{rgb}{1.000000,1.000000,1.000000}%
\pgfsetfillcolor{currentfill}%
\pgfsetfillopacity{0.800000}%
\pgfsetlinewidth{1.003750pt}%
\definecolor{currentstroke}{rgb}{0.800000,0.800000,0.800000}%
\pgfsetstrokecolor{currentstroke}%
\pgfsetstrokeopacity{0.800000}%
\pgfsetdash{}{0pt}%
\pgfpathmoveto{\pgfqpoint{0.667288in}{1.544733in}}%
\pgfpathlineto{\pgfqpoint{1.448613in}{1.544733in}}%
\pgfpathquadraticcurveto{\pgfqpoint{1.470835in}{1.544733in}}{\pgfqpoint{1.470835in}{1.566956in}}%
\pgfpathlineto{\pgfqpoint{1.470835in}{1.711261in}}%
\pgfpathquadraticcurveto{\pgfqpoint{1.470835in}{1.733483in}}{\pgfqpoint{1.448613in}{1.733483in}}%
\pgfpathlineto{\pgfqpoint{0.667288in}{1.733483in}}%
\pgfpathquadraticcurveto{\pgfqpoint{0.645065in}{1.733483in}}{\pgfqpoint{0.645065in}{1.711261in}}%
\pgfpathlineto{\pgfqpoint{0.645065in}{1.566956in}}%
\pgfpathquadraticcurveto{\pgfqpoint{0.645065in}{1.544733in}}{\pgfqpoint{0.667288in}{1.544733in}}%
\pgfpathlineto{\pgfqpoint{0.667288in}{1.544733in}}%
\pgfpathclose%
\pgfusepath{stroke,fill}%
\end{pgfscope}%
\begin{pgfscope}%
\pgfsetbuttcap%
\pgfsetroundjoin%
\pgfsetlinewidth{1.505625pt}%
\definecolor{currentstroke}{rgb}{0.800000,0.470588,0.737255}%
\pgfsetstrokecolor{currentstroke}%
\pgfsetdash{{5.550000pt}{2.400000pt}}{0.000000pt}%
\pgfpathmoveto{\pgfqpoint{0.689510in}{1.649622in}}%
\pgfpathlineto{\pgfqpoint{0.800621in}{1.649622in}}%
\pgfpathlineto{\pgfqpoint{0.911732in}{1.649622in}}%
\pgfusepath{stroke}%
\end{pgfscope}%
\begin{pgfscope}%
\definecolor{textcolor}{rgb}{0.000000,0.000000,0.000000}%
\pgfsetstrokecolor{textcolor}%
\pgfsetfillcolor{textcolor}%
\pgftext[x=1.000621in,y=1.610733in,left,base]{\color{textcolor}{\rmfamily\fontsize{8.000000}{9.600000}\selectfont\catcode`\^=\active\def^{\ifmmode\sp\else\^{}\fi}\catcode`\%=\active\def%{\%}$\displaystyle \propto D\tau^{+1}$}}%
\end{pgfscope}%
\end{pgfpicture}%
\makeatother%
\endgroup%

        } % scalebox
        \caption{Allan deviation}
        \label{fig:drift_adev}
    \end{subfigure}
    \caption{Different representations of linear drift.}
    \label{fig:drift_noise_simulated}
\end{figure}

\clearpage
\subsubsection{Dead Time}
\label{sec:dead_time}
The coefficients given here were derived using the assumption, that all sampling in a measurement are continous and the dead time $\theta = 0$. Unfortunately, measurement sometime have a dead time, that is non negligible. This problem was extensively discussed by \citeauthor{psd_to_adev} \cite{psd_to_adev}. \citeauthor{adev_frequency_counter} even developed special models to account for the algorithms of modern frequency counters \cite{adev_frequency_counter}. While some frequency counters support gapless measurements, the situation is entirely different for digitizers and digital multimeters. Several settings commonly used affect the dead time, which can be considerable. It is therefore important to discuss typical measurement settings for voltmeters to estimate the errors that arise from those settings. The focus of this discussion lies on the dead time introduced by digital multimeters, but the application is not limited to this field.

The most commonly used settings, that affect the dead time of a voltmeter are auto-zeroing and line synchronization. Auto-zeroing is done by adding additional measurements to the normal input integration cycle. To correct for the zero offset drift a zero measurement is added where the adc is switched to the low terminal. Additionally, some devices add a reading of the reference voltage to correct for gain errors. The implementation details and type of measurements are manufacturer dependent and must be determinded for every multimeter used.

The other setting, that can be enabled in voltmeters, is the line synchronization to increase the noise rejection of the instrument. This setting synchronizes the start of a measurement to the zero crossing of the power line. Depending on the instrument, this might cause a delay of one power line cycle (PLC) after each measurment if the instrument is not capable of processing the previous measurement while at the same time recording another one.

A simple measurement with dead time is shown in figure \ref{fig:allan_variance_definitions} on page \pageref{fig:allan_variance_definitions}. The model assumes, that the dead time is constant and is always added after the actual integration time $\tau$. This is rarely true for real measurement data as many devices and even ADCs use internal averaging and auto-zeroing to produce a measurement. The actual dead time is therefore spread over the whole measurement and not limited to the end of the measurement. An example is the Keysight \device{3458A} DMM, which automatically switches to averaging when selecting integration times greater than \qty{10}{\plc}. The reason is simple, for longer integration times, more and more flicker noise starts contributing to the measurement. The measurement is therefore split into single measurements of \qty{10}{\plc} and using auto-zeroing the flicker noise is suppressed. This is discussed in more detail as an example in section \ref{sec:autozero}. The mathematical problem of a distributed dead time was already noted by \citeauthor{adev_noise_types} \cite{adev_noise_types} and it is distinctively different from the calculations made by \citeauthor{psd_to_adev} for a single dead time at the end of the measurement. The exact mathematical treatment is complex and is beyond the scope of this work, especially considering, that auto-zeroing does a lot more than just adding dead time at the end of the measurement. Fortunately using a few assumptions the problem can be greatly simplified.

An interesting observation can be made for white noise. Since it is uncorrelated, it makes no difference whether it is sampled in full, or only partially, therefore the Allan deviation for a white noise process with or without dead time is the same:
\begin{equation}
    \sigma^2(N,T, \tau) = \sigma^2(N=2,T=\tau, \tau) = \sigma_A^2(\tau) \frac 1 2 h_0 \tau^{-1}
\end{equation}

Consequently, if the dead time is added at a frequency high enough, so that the input amplifier output is dominated by white noise, the dead time will have no influence on the Allan variance.

Finally, \citeauthor{psd_to_adev} \cite{psd_to_adev} notes that for measurement durations or averaging times $T \gg T_0$, the Allan variance with respect to $T$ shows an asymptotic behaviour of $\sigma_A^2(T) \to \sigma_A^2(\tau)$.

\clearpage
\subsection{Example}
\label{sec:noise_example}
Using the results from the previous sections, it is possible to simulate a typical measurement sample containing white noise, flicker noise and random walk behaviour. The simulation was written in Python using the \textit{AllanTools} library \cite{allantools} to generate the time domain data, which was then converted to a power spectrum using the algorithm of \citeauthor{welch} \cite{welch}. The Allan deviation was calculated using the \textit{AllanTools}. The full Python source is available at \cite{}. The time domain data shown here was downsampled from $2^{25}$ data points to \num{2000} points for faster plotting, using the Largest-Triangle-Three-Buckets (LTTB) algorithm created by \citeauthor{lttb} \cite{lttb}. The downsampling algorithm chosen is optimal for this application, because it aims to visually keep the result the same by favouring parts of the data, where there is more change. The only difference noticable to the author is, that the edges of the white noise plot are a little rougher. The full data set can be obtained using the source code given above if one desires. The power spectrum and the Allan deviation were always calculated from the full dataset. The data of the power spectrum was additionally binned to be evenly spaced on a logarithmic scale. This considerably reduced the high frequency noise and made the plot easier while not negatively impacting the shape.

\begin{figure}[ht]
    \centering
    \begin{subfigure}{0.32\linewidth}
        \centering
        \scalebox{0.75}{%
            %% Creator: Matplotlib, PGF backend
%%
%% To include the figure in your LaTeX document, write
%%   \input{<filename>.pgf}
%%
%% Make sure the required packages are loaded in your preamble
%%   \usepackage{pgf}
%%
%% Also ensure that all the required font packages are loaded; for instance,
%% the lmodern package is sometimes necessary when using math font.
%%   \usepackage{lmodern}
%%
%% Figures using additional raster images can only be included by \input if
%% they are in the same directory as the main LaTeX file. For loading figures
%% from other directories you can use the `import` package
%%   \usepackage{import}
%%
%% and then include the figures with
%%   \import{<path to file>}{<filename>.pgf}
%%
%% Matplotlib used the following preamble
%%   \usepackage{siunitx}
%%   \usepackage{fontspec}
%%   \makeatletter\@ifpackageloaded{underscore}{}{\usepackage[strings]{underscore}}\makeatother
%%
\begingroup%
\makeatletter%
\begin{pgfpicture}%
\pgfpathrectangle{\pgfpointorigin}{\pgfqpoint{2.440000in}{2.440000in}}%
\pgfusepath{use as bounding box, clip}%
\begin{pgfscope}%
\pgfsetbuttcap%
\pgfsetmiterjoin%
\definecolor{currentfill}{rgb}{1.000000,1.000000,1.000000}%
\pgfsetfillcolor{currentfill}%
\pgfsetlinewidth{0.000000pt}%
\definecolor{currentstroke}{rgb}{1.000000,1.000000,1.000000}%
\pgfsetstrokecolor{currentstroke}%
\pgfsetdash{}{0pt}%
\pgfpathmoveto{\pgfqpoint{0.000000in}{0.000000in}}%
\pgfpathlineto{\pgfqpoint{2.440000in}{0.000000in}}%
\pgfpathlineto{\pgfqpoint{2.440000in}{2.440000in}}%
\pgfpathlineto{\pgfqpoint{0.000000in}{2.440000in}}%
\pgfpathlineto{\pgfqpoint{0.000000in}{0.000000in}}%
\pgfpathclose%
\pgfusepath{fill}%
\end{pgfscope}%
\begin{pgfscope}%
\pgfsetbuttcap%
\pgfsetmiterjoin%
\definecolor{currentfill}{rgb}{1.000000,1.000000,1.000000}%
\pgfsetfillcolor{currentfill}%
\pgfsetlinewidth{0.000000pt}%
\definecolor{currentstroke}{rgb}{0.000000,0.000000,0.000000}%
\pgfsetstrokecolor{currentstroke}%
\pgfsetstrokeopacity{0.000000}%
\pgfsetdash{}{0pt}%
\pgfpathmoveto{\pgfqpoint{0.589745in}{0.416447in}}%
\pgfpathlineto{\pgfqpoint{2.398330in}{0.416447in}}%
\pgfpathlineto{\pgfqpoint{2.398330in}{2.398330in}}%
\pgfpathlineto{\pgfqpoint{0.589745in}{2.398330in}}%
\pgfpathlineto{\pgfqpoint{0.589745in}{0.416447in}}%
\pgfpathclose%
\pgfusepath{fill}%
\end{pgfscope}%
\begin{pgfscope}%
\pgfpathrectangle{\pgfqpoint{0.589745in}{0.416447in}}{\pgfqpoint{1.808585in}{1.981882in}}%
\pgfusepath{clip}%
\pgfsetrectcap%
\pgfsetroundjoin%
\pgfsetlinewidth{0.803000pt}%
\definecolor{currentstroke}{rgb}{0.450000,0.450000,0.450000}%
\pgfsetstrokecolor{currentstroke}%
\pgfsetdash{}{0pt}%
\pgfpathmoveto{\pgfqpoint{0.671953in}{0.416447in}}%
\pgfpathlineto{\pgfqpoint{0.671953in}{2.398330in}}%
\pgfusepath{stroke}%
\end{pgfscope}%
\begin{pgfscope}%
\pgfsetbuttcap%
\pgfsetroundjoin%
\definecolor{currentfill}{rgb}{0.000000,0.000000,0.000000}%
\pgfsetfillcolor{currentfill}%
\pgfsetlinewidth{0.803000pt}%
\definecolor{currentstroke}{rgb}{0.000000,0.000000,0.000000}%
\pgfsetstrokecolor{currentstroke}%
\pgfsetdash{}{0pt}%
\pgfsys@defobject{currentmarker}{\pgfqpoint{0.000000in}{-0.048611in}}{\pgfqpoint{0.000000in}{0.000000in}}{%
\pgfpathmoveto{\pgfqpoint{0.000000in}{0.000000in}}%
\pgfpathlineto{\pgfqpoint{0.000000in}{-0.048611in}}%
\pgfusepath{stroke,fill}%
}%
\begin{pgfscope}%
\pgfsys@transformshift{0.671953in}{0.416447in}%
\pgfsys@useobject{currentmarker}{}%
\end{pgfscope}%
\end{pgfscope}%
\begin{pgfscope}%
\definecolor{textcolor}{rgb}{0.000000,0.000000,0.000000}%
\pgfsetstrokecolor{textcolor}%
\pgfsetfillcolor{textcolor}%
\pgftext[x=0.671953in,y=0.319225in,,top]{\color{textcolor}\rmfamily\fontsize{8.000000}{9.600000}\selectfont \(\displaystyle {0}\)}%
\end{pgfscope}%
\begin{pgfscope}%
\pgfpathrectangle{\pgfqpoint{0.589745in}{0.416447in}}{\pgfqpoint{1.808585in}{1.981882in}}%
\pgfusepath{clip}%
\pgfsetrectcap%
\pgfsetroundjoin%
\pgfsetlinewidth{0.803000pt}%
\definecolor{currentstroke}{rgb}{0.450000,0.450000,0.450000}%
\pgfsetstrokecolor{currentstroke}%
\pgfsetdash{}{0pt}%
\pgfpathmoveto{\pgfqpoint{1.161954in}{0.416447in}}%
\pgfpathlineto{\pgfqpoint{1.161954in}{2.398330in}}%
\pgfusepath{stroke}%
\end{pgfscope}%
\begin{pgfscope}%
\pgfsetbuttcap%
\pgfsetroundjoin%
\definecolor{currentfill}{rgb}{0.000000,0.000000,0.000000}%
\pgfsetfillcolor{currentfill}%
\pgfsetlinewidth{0.803000pt}%
\definecolor{currentstroke}{rgb}{0.000000,0.000000,0.000000}%
\pgfsetstrokecolor{currentstroke}%
\pgfsetdash{}{0pt}%
\pgfsys@defobject{currentmarker}{\pgfqpoint{0.000000in}{-0.048611in}}{\pgfqpoint{0.000000in}{0.000000in}}{%
\pgfpathmoveto{\pgfqpoint{0.000000in}{0.000000in}}%
\pgfpathlineto{\pgfqpoint{0.000000in}{-0.048611in}}%
\pgfusepath{stroke,fill}%
}%
\begin{pgfscope}%
\pgfsys@transformshift{1.161954in}{0.416447in}%
\pgfsys@useobject{currentmarker}{}%
\end{pgfscope}%
\end{pgfscope}%
\begin{pgfscope}%
\definecolor{textcolor}{rgb}{0.000000,0.000000,0.000000}%
\pgfsetstrokecolor{textcolor}%
\pgfsetfillcolor{textcolor}%
\pgftext[x=1.161954in,y=0.319225in,,top]{\color{textcolor}\rmfamily\fontsize{8.000000}{9.600000}\selectfont \(\displaystyle {10}\)}%
\end{pgfscope}%
\begin{pgfscope}%
\pgfpathrectangle{\pgfqpoint{0.589745in}{0.416447in}}{\pgfqpoint{1.808585in}{1.981882in}}%
\pgfusepath{clip}%
\pgfsetrectcap%
\pgfsetroundjoin%
\pgfsetlinewidth{0.803000pt}%
\definecolor{currentstroke}{rgb}{0.450000,0.450000,0.450000}%
\pgfsetstrokecolor{currentstroke}%
\pgfsetdash{}{0pt}%
\pgfpathmoveto{\pgfqpoint{1.651954in}{0.416447in}}%
\pgfpathlineto{\pgfqpoint{1.651954in}{2.398330in}}%
\pgfusepath{stroke}%
\end{pgfscope}%
\begin{pgfscope}%
\pgfsetbuttcap%
\pgfsetroundjoin%
\definecolor{currentfill}{rgb}{0.000000,0.000000,0.000000}%
\pgfsetfillcolor{currentfill}%
\pgfsetlinewidth{0.803000pt}%
\definecolor{currentstroke}{rgb}{0.000000,0.000000,0.000000}%
\pgfsetstrokecolor{currentstroke}%
\pgfsetdash{}{0pt}%
\pgfsys@defobject{currentmarker}{\pgfqpoint{0.000000in}{-0.048611in}}{\pgfqpoint{0.000000in}{0.000000in}}{%
\pgfpathmoveto{\pgfqpoint{0.000000in}{0.000000in}}%
\pgfpathlineto{\pgfqpoint{0.000000in}{-0.048611in}}%
\pgfusepath{stroke,fill}%
}%
\begin{pgfscope}%
\pgfsys@transformshift{1.651954in}{0.416447in}%
\pgfsys@useobject{currentmarker}{}%
\end{pgfscope}%
\end{pgfscope}%
\begin{pgfscope}%
\definecolor{textcolor}{rgb}{0.000000,0.000000,0.000000}%
\pgfsetstrokecolor{textcolor}%
\pgfsetfillcolor{textcolor}%
\pgftext[x=1.651954in,y=0.319225in,,top]{\color{textcolor}\rmfamily\fontsize{8.000000}{9.600000}\selectfont \(\displaystyle {20}\)}%
\end{pgfscope}%
\begin{pgfscope}%
\pgfpathrectangle{\pgfqpoint{0.589745in}{0.416447in}}{\pgfqpoint{1.808585in}{1.981882in}}%
\pgfusepath{clip}%
\pgfsetrectcap%
\pgfsetroundjoin%
\pgfsetlinewidth{0.803000pt}%
\definecolor{currentstroke}{rgb}{0.450000,0.450000,0.450000}%
\pgfsetstrokecolor{currentstroke}%
\pgfsetdash{}{0pt}%
\pgfpathmoveto{\pgfqpoint{2.141954in}{0.416447in}}%
\pgfpathlineto{\pgfqpoint{2.141954in}{2.398330in}}%
\pgfusepath{stroke}%
\end{pgfscope}%
\begin{pgfscope}%
\pgfsetbuttcap%
\pgfsetroundjoin%
\definecolor{currentfill}{rgb}{0.000000,0.000000,0.000000}%
\pgfsetfillcolor{currentfill}%
\pgfsetlinewidth{0.803000pt}%
\definecolor{currentstroke}{rgb}{0.000000,0.000000,0.000000}%
\pgfsetstrokecolor{currentstroke}%
\pgfsetdash{}{0pt}%
\pgfsys@defobject{currentmarker}{\pgfqpoint{0.000000in}{-0.048611in}}{\pgfqpoint{0.000000in}{0.000000in}}{%
\pgfpathmoveto{\pgfqpoint{0.000000in}{0.000000in}}%
\pgfpathlineto{\pgfqpoint{0.000000in}{-0.048611in}}%
\pgfusepath{stroke,fill}%
}%
\begin{pgfscope}%
\pgfsys@transformshift{2.141954in}{0.416447in}%
\pgfsys@useobject{currentmarker}{}%
\end{pgfscope}%
\end{pgfscope}%
\begin{pgfscope}%
\definecolor{textcolor}{rgb}{0.000000,0.000000,0.000000}%
\pgfsetstrokecolor{textcolor}%
\pgfsetfillcolor{textcolor}%
\pgftext[x=2.141954in,y=0.319225in,,top]{\color{textcolor}\rmfamily\fontsize{8.000000}{9.600000}\selectfont \(\displaystyle {30}\)}%
\end{pgfscope}%
\begin{pgfscope}%
\definecolor{textcolor}{rgb}{0.000000,0.000000,0.000000}%
\pgfsetstrokecolor{textcolor}%
\pgfsetfillcolor{textcolor}%
\pgftext[x=1.494037in,y=0.165003in,,top]{\color{textcolor}\rmfamily\fontsize{10.000000}{12.000000}\selectfont Time in \(\displaystyle \unit{\second}\)}%
\end{pgfscope}%
\begin{pgfscope}%
\pgfpathrectangle{\pgfqpoint{0.589745in}{0.416447in}}{\pgfqpoint{1.808585in}{1.981882in}}%
\pgfusepath{clip}%
\pgfsetrectcap%
\pgfsetroundjoin%
\pgfsetlinewidth{0.803000pt}%
\definecolor{currentstroke}{rgb}{0.450000,0.450000,0.450000}%
\pgfsetstrokecolor{currentstroke}%
\pgfsetdash{}{0pt}%
\pgfpathmoveto{\pgfqpoint{0.589745in}{0.614636in}}%
\pgfpathlineto{\pgfqpoint{2.398330in}{0.614636in}}%
\pgfusepath{stroke}%
\end{pgfscope}%
\begin{pgfscope}%
\pgfsetbuttcap%
\pgfsetroundjoin%
\definecolor{currentfill}{rgb}{0.000000,0.000000,0.000000}%
\pgfsetfillcolor{currentfill}%
\pgfsetlinewidth{0.803000pt}%
\definecolor{currentstroke}{rgb}{0.000000,0.000000,0.000000}%
\pgfsetstrokecolor{currentstroke}%
\pgfsetdash{}{0pt}%
\pgfsys@defobject{currentmarker}{\pgfqpoint{-0.048611in}{0.000000in}}{\pgfqpoint{-0.000000in}{0.000000in}}{%
\pgfpathmoveto{\pgfqpoint{-0.000000in}{0.000000in}}%
\pgfpathlineto{\pgfqpoint{-0.048611in}{0.000000in}}%
\pgfusepath{stroke,fill}%
}%
\begin{pgfscope}%
\pgfsys@transformshift{0.589745in}{0.614636in}%
\pgfsys@useobject{currentmarker}{}%
\end{pgfscope}%
\end{pgfscope}%
\begin{pgfscope}%
\definecolor{textcolor}{rgb}{0.000000,0.000000,0.000000}%
\pgfsetstrokecolor{textcolor}%
\pgfsetfillcolor{textcolor}%
\pgftext[x=0.223614in, y=0.576080in, left, base]{\color{textcolor}\rmfamily\fontsize{8.000000}{9.600000}\selectfont \(\displaystyle {\ensuremath{-}200}\)}%
\end{pgfscope}%
\begin{pgfscope}%
\pgfpathrectangle{\pgfqpoint{0.589745in}{0.416447in}}{\pgfqpoint{1.808585in}{1.981882in}}%
\pgfusepath{clip}%
\pgfsetrectcap%
\pgfsetroundjoin%
\pgfsetlinewidth{0.803000pt}%
\definecolor{currentstroke}{rgb}{0.450000,0.450000,0.450000}%
\pgfsetstrokecolor{currentstroke}%
\pgfsetdash{}{0pt}%
\pgfpathmoveto{\pgfqpoint{0.589745in}{1.011012in}}%
\pgfpathlineto{\pgfqpoint{2.398330in}{1.011012in}}%
\pgfusepath{stroke}%
\end{pgfscope}%
\begin{pgfscope}%
\pgfsetbuttcap%
\pgfsetroundjoin%
\definecolor{currentfill}{rgb}{0.000000,0.000000,0.000000}%
\pgfsetfillcolor{currentfill}%
\pgfsetlinewidth{0.803000pt}%
\definecolor{currentstroke}{rgb}{0.000000,0.000000,0.000000}%
\pgfsetstrokecolor{currentstroke}%
\pgfsetdash{}{0pt}%
\pgfsys@defobject{currentmarker}{\pgfqpoint{-0.048611in}{0.000000in}}{\pgfqpoint{-0.000000in}{0.000000in}}{%
\pgfpathmoveto{\pgfqpoint{-0.000000in}{0.000000in}}%
\pgfpathlineto{\pgfqpoint{-0.048611in}{0.000000in}}%
\pgfusepath{stroke,fill}%
}%
\begin{pgfscope}%
\pgfsys@transformshift{0.589745in}{1.011012in}%
\pgfsys@useobject{currentmarker}{}%
\end{pgfscope}%
\end{pgfscope}%
\begin{pgfscope}%
\definecolor{textcolor}{rgb}{0.000000,0.000000,0.000000}%
\pgfsetstrokecolor{textcolor}%
\pgfsetfillcolor{textcolor}%
\pgftext[x=0.223614in, y=0.972457in, left, base]{\color{textcolor}\rmfamily\fontsize{8.000000}{9.600000}\selectfont \(\displaystyle {\ensuremath{-}100}\)}%
\end{pgfscope}%
\begin{pgfscope}%
\pgfpathrectangle{\pgfqpoint{0.589745in}{0.416447in}}{\pgfqpoint{1.808585in}{1.981882in}}%
\pgfusepath{clip}%
\pgfsetrectcap%
\pgfsetroundjoin%
\pgfsetlinewidth{0.803000pt}%
\definecolor{currentstroke}{rgb}{0.450000,0.450000,0.450000}%
\pgfsetstrokecolor{currentstroke}%
\pgfsetdash{}{0pt}%
\pgfpathmoveto{\pgfqpoint{0.589745in}{1.407389in}}%
\pgfpathlineto{\pgfqpoint{2.398330in}{1.407389in}}%
\pgfusepath{stroke}%
\end{pgfscope}%
\begin{pgfscope}%
\pgfsetbuttcap%
\pgfsetroundjoin%
\definecolor{currentfill}{rgb}{0.000000,0.000000,0.000000}%
\pgfsetfillcolor{currentfill}%
\pgfsetlinewidth{0.803000pt}%
\definecolor{currentstroke}{rgb}{0.000000,0.000000,0.000000}%
\pgfsetstrokecolor{currentstroke}%
\pgfsetdash{}{0pt}%
\pgfsys@defobject{currentmarker}{\pgfqpoint{-0.048611in}{0.000000in}}{\pgfqpoint{-0.000000in}{0.000000in}}{%
\pgfpathmoveto{\pgfqpoint{-0.000000in}{0.000000in}}%
\pgfpathlineto{\pgfqpoint{-0.048611in}{0.000000in}}%
\pgfusepath{stroke,fill}%
}%
\begin{pgfscope}%
\pgfsys@transformshift{0.589745in}{1.407389in}%
\pgfsys@useobject{currentmarker}{}%
\end{pgfscope}%
\end{pgfscope}%
\begin{pgfscope}%
\definecolor{textcolor}{rgb}{0.000000,0.000000,0.000000}%
\pgfsetstrokecolor{textcolor}%
\pgfsetfillcolor{textcolor}%
\pgftext[x=0.433494in, y=1.368833in, left, base]{\color{textcolor}\rmfamily\fontsize{8.000000}{9.600000}\selectfont \(\displaystyle {0}\)}%
\end{pgfscope}%
\begin{pgfscope}%
\pgfpathrectangle{\pgfqpoint{0.589745in}{0.416447in}}{\pgfqpoint{1.808585in}{1.981882in}}%
\pgfusepath{clip}%
\pgfsetrectcap%
\pgfsetroundjoin%
\pgfsetlinewidth{0.803000pt}%
\definecolor{currentstroke}{rgb}{0.450000,0.450000,0.450000}%
\pgfsetstrokecolor{currentstroke}%
\pgfsetdash{}{0pt}%
\pgfpathmoveto{\pgfqpoint{0.589745in}{1.803765in}}%
\pgfpathlineto{\pgfqpoint{2.398330in}{1.803765in}}%
\pgfusepath{stroke}%
\end{pgfscope}%
\begin{pgfscope}%
\pgfsetbuttcap%
\pgfsetroundjoin%
\definecolor{currentfill}{rgb}{0.000000,0.000000,0.000000}%
\pgfsetfillcolor{currentfill}%
\pgfsetlinewidth{0.803000pt}%
\definecolor{currentstroke}{rgb}{0.000000,0.000000,0.000000}%
\pgfsetstrokecolor{currentstroke}%
\pgfsetdash{}{0pt}%
\pgfsys@defobject{currentmarker}{\pgfqpoint{-0.048611in}{0.000000in}}{\pgfqpoint{-0.000000in}{0.000000in}}{%
\pgfpathmoveto{\pgfqpoint{-0.000000in}{0.000000in}}%
\pgfpathlineto{\pgfqpoint{-0.048611in}{0.000000in}}%
\pgfusepath{stroke,fill}%
}%
\begin{pgfscope}%
\pgfsys@transformshift{0.589745in}{1.803765in}%
\pgfsys@useobject{currentmarker}{}%
\end{pgfscope}%
\end{pgfscope}%
\begin{pgfscope}%
\definecolor{textcolor}{rgb}{0.000000,0.000000,0.000000}%
\pgfsetstrokecolor{textcolor}%
\pgfsetfillcolor{textcolor}%
\pgftext[x=0.315437in, y=1.765210in, left, base]{\color{textcolor}\rmfamily\fontsize{8.000000}{9.600000}\selectfont \(\displaystyle {100}\)}%
\end{pgfscope}%
\begin{pgfscope}%
\pgfpathrectangle{\pgfqpoint{0.589745in}{0.416447in}}{\pgfqpoint{1.808585in}{1.981882in}}%
\pgfusepath{clip}%
\pgfsetrectcap%
\pgfsetroundjoin%
\pgfsetlinewidth{0.803000pt}%
\definecolor{currentstroke}{rgb}{0.450000,0.450000,0.450000}%
\pgfsetstrokecolor{currentstroke}%
\pgfsetdash{}{0pt}%
\pgfpathmoveto{\pgfqpoint{0.589745in}{2.200142in}}%
\pgfpathlineto{\pgfqpoint{2.398330in}{2.200142in}}%
\pgfusepath{stroke}%
\end{pgfscope}%
\begin{pgfscope}%
\pgfsetbuttcap%
\pgfsetroundjoin%
\definecolor{currentfill}{rgb}{0.000000,0.000000,0.000000}%
\pgfsetfillcolor{currentfill}%
\pgfsetlinewidth{0.803000pt}%
\definecolor{currentstroke}{rgb}{0.000000,0.000000,0.000000}%
\pgfsetstrokecolor{currentstroke}%
\pgfsetdash{}{0pt}%
\pgfsys@defobject{currentmarker}{\pgfqpoint{-0.048611in}{0.000000in}}{\pgfqpoint{-0.000000in}{0.000000in}}{%
\pgfpathmoveto{\pgfqpoint{-0.000000in}{0.000000in}}%
\pgfpathlineto{\pgfqpoint{-0.048611in}{0.000000in}}%
\pgfusepath{stroke,fill}%
}%
\begin{pgfscope}%
\pgfsys@transformshift{0.589745in}{2.200142in}%
\pgfsys@useobject{currentmarker}{}%
\end{pgfscope}%
\end{pgfscope}%
\begin{pgfscope}%
\definecolor{textcolor}{rgb}{0.000000,0.000000,0.000000}%
\pgfsetstrokecolor{textcolor}%
\pgfsetfillcolor{textcolor}%
\pgftext[x=0.315437in, y=2.161586in, left, base]{\color{textcolor}\rmfamily\fontsize{8.000000}{9.600000}\selectfont \(\displaystyle {200}\)}%
\end{pgfscope}%
\begin{pgfscope}%
\definecolor{textcolor}{rgb}{0.000000,0.000000,0.000000}%
\pgfsetstrokecolor{textcolor}%
\pgfsetfillcolor{textcolor}%
\pgftext[x=0.168059in,y=1.407389in,,bottom,rotate=90.000000]{\color{textcolor}\rmfamily\fontsize{10.000000}{12.000000}\selectfont Ampl. in arb. unit}%
\end{pgfscope}%
\begin{pgfscope}%
\pgfpathrectangle{\pgfqpoint{0.589745in}{0.416447in}}{\pgfqpoint{1.808585in}{1.981882in}}%
\pgfusepath{clip}%
\pgfsetrectcap%
\pgfsetroundjoin%
\pgfsetlinewidth{1.505625pt}%
\definecolor{currentstroke}{rgb}{0.000000,0.447059,0.698039}%
\pgfsetstrokecolor{currentstroke}%
\pgfsetdash{}{0pt}%
\pgfpathmoveto{\pgfqpoint{0.671953in}{1.390058in}}%
\pgfpathlineto{\pgfqpoint{0.672729in}{1.968821in}}%
\pgfpathlineto{\pgfqpoint{0.672917in}{0.924010in}}%
\pgfpathlineto{\pgfqpoint{0.673617in}{1.825449in}}%
\pgfpathlineto{\pgfqpoint{0.674881in}{0.886237in}}%
\pgfpathlineto{\pgfqpoint{0.675366in}{1.850616in}}%
\pgfpathlineto{\pgfqpoint{0.676077in}{0.992823in}}%
\pgfpathlineto{\pgfqpoint{0.677195in}{1.934093in}}%
\pgfpathlineto{\pgfqpoint{0.677721in}{0.943473in}}%
\pgfpathlineto{\pgfqpoint{0.678898in}{1.979228in}}%
\pgfpathlineto{\pgfqpoint{0.679515in}{0.953329in}}%
\pgfpathlineto{\pgfqpoint{0.680224in}{1.861907in}}%
\pgfpathlineto{\pgfqpoint{0.681088in}{0.990862in}}%
\pgfpathlineto{\pgfqpoint{0.681875in}{1.864185in}}%
\pgfpathlineto{\pgfqpoint{0.682995in}{0.854129in}}%
\pgfpathlineto{\pgfqpoint{0.683525in}{1.818747in}}%
\pgfpathlineto{\pgfqpoint{0.684304in}{0.930747in}}%
\pgfpathlineto{\pgfqpoint{0.685198in}{1.858381in}}%
\pgfpathlineto{\pgfqpoint{0.686001in}{0.970812in}}%
\pgfpathlineto{\pgfqpoint{0.686838in}{1.836966in}}%
\pgfpathlineto{\pgfqpoint{0.687644in}{0.974603in}}%
\pgfpathlineto{\pgfqpoint{0.688490in}{1.898515in}}%
\pgfpathlineto{\pgfqpoint{0.689443in}{0.959955in}}%
\pgfpathlineto{\pgfqpoint{0.690131in}{1.852103in}}%
\pgfpathlineto{\pgfqpoint{0.690964in}{0.905464in}}%
\pgfpathlineto{\pgfqpoint{0.691764in}{1.868832in}}%
\pgfpathlineto{\pgfqpoint{0.692567in}{1.003429in}}%
\pgfpathlineto{\pgfqpoint{0.693404in}{1.839082in}}%
\pgfpathlineto{\pgfqpoint{0.694246in}{0.953775in}}%
\pgfpathlineto{\pgfqpoint{0.695440in}{1.911204in}}%
\pgfpathlineto{\pgfqpoint{0.695882in}{0.942527in}}%
\pgfpathlineto{\pgfqpoint{0.696768in}{1.921432in}}%
\pgfpathlineto{\pgfqpoint{0.697666in}{0.921535in}}%
\pgfpathlineto{\pgfqpoint{0.698291in}{1.849825in}}%
\pgfpathlineto{\pgfqpoint{0.699173in}{0.994292in}}%
\pgfpathlineto{\pgfqpoint{0.700541in}{1.993873in}}%
\pgfpathlineto{\pgfqpoint{0.700906in}{0.908926in}}%
\pgfpathlineto{\pgfqpoint{0.701616in}{1.896744in}}%
\pgfpathlineto{\pgfqpoint{0.702838in}{0.882288in}}%
\pgfpathlineto{\pgfqpoint{0.703238in}{1.853847in}}%
\pgfpathlineto{\pgfqpoint{0.704091in}{0.909742in}}%
\pgfpathlineto{\pgfqpoint{0.705007in}{1.874277in}}%
\pgfpathlineto{\pgfqpoint{0.705979in}{0.870524in}}%
\pgfpathlineto{\pgfqpoint{0.706527in}{1.822620in}}%
\pgfpathlineto{\pgfqpoint{0.707571in}{0.961696in}}%
\pgfpathlineto{\pgfqpoint{0.708252in}{1.812295in}}%
\pgfpathlineto{\pgfqpoint{0.709114in}{0.921924in}}%
\pgfpathlineto{\pgfqpoint{0.709909in}{1.886792in}}%
\pgfpathlineto{\pgfqpoint{0.710697in}{0.995482in}}%
\pgfpathlineto{\pgfqpoint{0.711490in}{1.801445in}}%
\pgfpathlineto{\pgfqpoint{0.712352in}{0.919597in}}%
\pgfpathlineto{\pgfqpoint{0.713118in}{1.824321in}}%
\pgfpathlineto{\pgfqpoint{0.714049in}{0.933093in}}%
\pgfpathlineto{\pgfqpoint{0.714857in}{1.814853in}}%
\pgfpathlineto{\pgfqpoint{0.715684in}{0.986263in}}%
\pgfpathlineto{\pgfqpoint{0.716918in}{1.986031in}}%
\pgfpathlineto{\pgfqpoint{0.717252in}{0.983770in}}%
\pgfpathlineto{\pgfqpoint{0.718127in}{1.849425in}}%
\pgfpathlineto{\pgfqpoint{0.718900in}{0.935696in}}%
\pgfpathlineto{\pgfqpoint{0.719694in}{1.862124in}}%
\pgfpathlineto{\pgfqpoint{0.720659in}{0.961175in}}%
\pgfpathlineto{\pgfqpoint{0.721341in}{1.901537in}}%
\pgfpathlineto{\pgfqpoint{0.722301in}{0.969724in}}%
\pgfpathlineto{\pgfqpoint{0.723126in}{1.926806in}}%
\pgfpathlineto{\pgfqpoint{0.723798in}{0.950144in}}%
\pgfpathlineto{\pgfqpoint{0.724633in}{1.906360in}}%
\pgfpathlineto{\pgfqpoint{0.725480in}{0.825233in}}%
\pgfpathlineto{\pgfqpoint{0.726354in}{1.867020in}}%
\pgfpathlineto{\pgfqpoint{0.727227in}{0.960734in}}%
\pgfpathlineto{\pgfqpoint{0.728092in}{1.843145in}}%
\pgfpathlineto{\pgfqpoint{0.728789in}{0.927657in}}%
\pgfpathlineto{\pgfqpoint{0.729623in}{1.842379in}}%
\pgfpathlineto{\pgfqpoint{0.730407in}{0.912226in}}%
\pgfpathlineto{\pgfqpoint{0.731563in}{1.910858in}}%
\pgfpathlineto{\pgfqpoint{0.732219in}{0.942011in}}%
\pgfpathlineto{\pgfqpoint{0.732902in}{1.854422in}}%
\pgfpathlineto{\pgfqpoint{0.733700in}{1.022015in}}%
\pgfpathlineto{\pgfqpoint{0.734561in}{1.904973in}}%
\pgfpathlineto{\pgfqpoint{0.735442in}{0.947451in}}%
\pgfpathlineto{\pgfqpoint{0.736195in}{1.835741in}}%
\pgfpathlineto{\pgfqpoint{0.737413in}{0.853966in}}%
\pgfpathlineto{\pgfqpoint{0.738025in}{1.899703in}}%
\pgfpathlineto{\pgfqpoint{0.738670in}{1.019403in}}%
\pgfpathlineto{\pgfqpoint{0.739647in}{1.868035in}}%
\pgfpathlineto{\pgfqpoint{0.740446in}{0.916401in}}%
\pgfpathlineto{\pgfqpoint{0.741115in}{1.859487in}}%
\pgfpathlineto{\pgfqpoint{0.741980in}{0.908306in}}%
\pgfpathlineto{\pgfqpoint{0.742907in}{1.909470in}}%
\pgfpathlineto{\pgfqpoint{0.743635in}{0.979621in}}%
\pgfpathlineto{\pgfqpoint{0.744429in}{1.860718in}}%
\pgfpathlineto{\pgfqpoint{0.745280in}{0.936093in}}%
\pgfpathlineto{\pgfqpoint{0.746082in}{1.813341in}}%
\pgfpathlineto{\pgfqpoint{0.747099in}{0.863681in}}%
\pgfpathlineto{\pgfqpoint{0.747797in}{1.812285in}}%
\pgfpathlineto{\pgfqpoint{0.748516in}{0.949032in}}%
\pgfpathlineto{\pgfqpoint{0.749373in}{1.839578in}}%
\pgfpathlineto{\pgfqpoint{0.750487in}{0.859796in}}%
\pgfpathlineto{\pgfqpoint{0.751097in}{1.839316in}}%
\pgfpathlineto{\pgfqpoint{0.751875in}{0.980661in}}%
\pgfpathlineto{\pgfqpoint{0.752681in}{1.896141in}}%
\pgfpathlineto{\pgfqpoint{0.753487in}{0.950754in}}%
\pgfpathlineto{\pgfqpoint{0.754246in}{1.825487in}}%
\pgfpathlineto{\pgfqpoint{0.755227in}{0.910598in}}%
\pgfpathlineto{\pgfqpoint{0.756003in}{1.897060in}}%
\pgfpathlineto{\pgfqpoint{0.756730in}{0.987383in}}%
\pgfpathlineto{\pgfqpoint{0.757538in}{1.878718in}}%
\pgfpathlineto{\pgfqpoint{0.758411in}{0.916747in}}%
\pgfpathlineto{\pgfqpoint{0.759290in}{1.900809in}}%
\pgfpathlineto{\pgfqpoint{0.760084in}{0.932070in}}%
\pgfpathlineto{\pgfqpoint{0.760902in}{1.819554in}}%
\pgfpathlineto{\pgfqpoint{0.761797in}{0.965610in}}%
\pgfpathlineto{\pgfqpoint{0.762699in}{1.878590in}}%
\pgfpathlineto{\pgfqpoint{0.763431in}{0.943700in}}%
\pgfpathlineto{\pgfqpoint{0.764153in}{1.808347in}}%
\pgfpathlineto{\pgfqpoint{0.765318in}{0.870883in}}%
\pgfpathlineto{\pgfqpoint{0.765983in}{1.954536in}}%
\pgfpathlineto{\pgfqpoint{0.766633in}{1.006652in}}%
\pgfpathlineto{\pgfqpoint{0.767778in}{1.938822in}}%
\pgfpathlineto{\pgfqpoint{0.768362in}{0.928780in}}%
\pgfpathlineto{\pgfqpoint{0.769256in}{1.884059in}}%
\pgfpathlineto{\pgfqpoint{0.770002in}{0.965876in}}%
\pgfpathlineto{\pgfqpoint{0.770720in}{1.838987in}}%
\pgfpathlineto{\pgfqpoint{0.771652in}{0.958845in}}%
\pgfpathlineto{\pgfqpoint{0.772397in}{1.840191in}}%
\pgfpathlineto{\pgfqpoint{0.773222in}{0.972942in}}%
\pgfpathlineto{\pgfqpoint{0.773998in}{1.853990in}}%
\pgfpathlineto{\pgfqpoint{0.774887in}{0.912386in}}%
\pgfpathlineto{\pgfqpoint{0.775650in}{1.877630in}}%
\pgfpathlineto{\pgfqpoint{0.776635in}{0.840716in}}%
\pgfpathlineto{\pgfqpoint{0.777397in}{1.832099in}}%
\pgfpathlineto{\pgfqpoint{0.778338in}{0.925450in}}%
\pgfpathlineto{\pgfqpoint{0.779043in}{1.884550in}}%
\pgfpathlineto{\pgfqpoint{0.779775in}{0.991865in}}%
\pgfpathlineto{\pgfqpoint{0.780620in}{1.842735in}}%
\pgfpathlineto{\pgfqpoint{0.781603in}{0.954340in}}%
\pgfpathlineto{\pgfqpoint{0.782339in}{1.859449in}}%
\pgfpathlineto{\pgfqpoint{0.783060in}{0.993204in}}%
\pgfpathlineto{\pgfqpoint{0.783991in}{1.985297in}}%
\pgfpathlineto{\pgfqpoint{0.784992in}{0.853161in}}%
\pgfpathlineto{\pgfqpoint{0.785573in}{1.885643in}}%
\pgfpathlineto{\pgfqpoint{0.786395in}{0.839072in}}%
\pgfpathlineto{\pgfqpoint{0.787250in}{1.849323in}}%
\pgfpathlineto{\pgfqpoint{0.788088in}{0.961822in}}%
\pgfpathlineto{\pgfqpoint{0.788981in}{1.831331in}}%
\pgfpathlineto{\pgfqpoint{0.789824in}{0.931567in}}%
\pgfpathlineto{\pgfqpoint{0.790615in}{1.828641in}}%
\pgfpathlineto{\pgfqpoint{0.791300in}{0.961258in}}%
\pgfpathlineto{\pgfqpoint{0.792208in}{1.939693in}}%
\pgfpathlineto{\pgfqpoint{0.792950in}{0.977455in}}%
\pgfpathlineto{\pgfqpoint{0.794214in}{1.918346in}}%
\pgfpathlineto{\pgfqpoint{0.794712in}{0.952942in}}%
\pgfpathlineto{\pgfqpoint{0.795755in}{1.991434in}}%
\pgfpathlineto{\pgfqpoint{0.796254in}{0.913540in}}%
\pgfpathlineto{\pgfqpoint{0.797170in}{1.928414in}}%
\pgfpathlineto{\pgfqpoint{0.797979in}{0.923051in}}%
\pgfpathlineto{\pgfqpoint{0.798790in}{1.918474in}}%
\pgfpathlineto{\pgfqpoint{0.799702in}{0.940094in}}%
\pgfpathlineto{\pgfqpoint{0.800635in}{1.918799in}}%
\pgfpathlineto{\pgfqpoint{0.801367in}{0.783151in}}%
\pgfpathlineto{\pgfqpoint{0.802097in}{1.912950in}}%
\pgfpathlineto{\pgfqpoint{0.802830in}{0.952135in}}%
\pgfpathlineto{\pgfqpoint{0.803653in}{1.853935in}}%
\pgfpathlineto{\pgfqpoint{0.804525in}{0.907575in}}%
\pgfpathlineto{\pgfqpoint{0.805327in}{1.877443in}}%
\pgfpathlineto{\pgfqpoint{0.806184in}{0.937874in}}%
\pgfpathlineto{\pgfqpoint{0.807133in}{1.910826in}}%
\pgfpathlineto{\pgfqpoint{0.807765in}{0.956287in}}%
\pgfpathlineto{\pgfqpoint{0.809177in}{1.972494in}}%
\pgfpathlineto{\pgfqpoint{0.809446in}{0.963914in}}%
\pgfpathlineto{\pgfqpoint{0.810630in}{1.953303in}}%
\pgfpathlineto{\pgfqpoint{0.811106in}{0.911859in}}%
\pgfpathlineto{\pgfqpoint{0.812020in}{1.925010in}}%
\pgfpathlineto{\pgfqpoint{0.812962in}{0.915107in}}%
\pgfpathlineto{\pgfqpoint{0.813566in}{1.853003in}}%
\pgfpathlineto{\pgfqpoint{0.814453in}{0.926265in}}%
\pgfpathlineto{\pgfqpoint{0.815169in}{1.804152in}}%
\pgfpathlineto{\pgfqpoint{0.816016in}{0.936228in}}%
\pgfpathlineto{\pgfqpoint{0.817165in}{1.891131in}}%
\pgfpathlineto{\pgfqpoint{0.817706in}{0.964511in}}%
\pgfpathlineto{\pgfqpoint{0.818578in}{1.913216in}}%
\pgfpathlineto{\pgfqpoint{0.819434in}{0.924802in}}%
\pgfpathlineto{\pgfqpoint{0.820242in}{1.908786in}}%
\pgfpathlineto{\pgfqpoint{0.820921in}{0.943002in}}%
\pgfpathlineto{\pgfqpoint{0.821845in}{1.859820in}}%
\pgfpathlineto{\pgfqpoint{0.822557in}{0.965993in}}%
\pgfpathlineto{\pgfqpoint{0.823428in}{1.839890in}}%
\pgfpathlineto{\pgfqpoint{0.824504in}{0.919132in}}%
\pgfpathlineto{\pgfqpoint{0.825097in}{1.876305in}}%
\pgfpathlineto{\pgfqpoint{0.825863in}{0.976644in}}%
\pgfpathlineto{\pgfqpoint{0.826703in}{1.866248in}}%
\pgfpathlineto{\pgfqpoint{0.827833in}{0.927111in}}%
\pgfpathlineto{\pgfqpoint{0.828321in}{1.821228in}}%
\pgfpathlineto{\pgfqpoint{0.829199in}{0.949860in}}%
\pgfpathlineto{\pgfqpoint{0.829990in}{1.788585in}}%
\pgfpathlineto{\pgfqpoint{0.830881in}{0.935437in}}%
\pgfpathlineto{\pgfqpoint{0.831811in}{1.859709in}}%
\pgfpathlineto{\pgfqpoint{0.832471in}{0.999807in}}%
\pgfpathlineto{\pgfqpoint{0.833250in}{1.829851in}}%
\pgfpathlineto{\pgfqpoint{0.834078in}{0.875258in}}%
\pgfpathlineto{\pgfqpoint{0.835113in}{1.898662in}}%
\pgfpathlineto{\pgfqpoint{0.835754in}{0.932281in}}%
\pgfpathlineto{\pgfqpoint{0.836664in}{1.918016in}}%
\pgfpathlineto{\pgfqpoint{0.837460in}{0.956705in}}%
\pgfpathlineto{\pgfqpoint{0.838221in}{1.799903in}}%
\pgfpathlineto{\pgfqpoint{0.839113in}{1.013171in}}%
\pgfpathlineto{\pgfqpoint{0.839945in}{1.935962in}}%
\pgfpathlineto{\pgfqpoint{0.840744in}{0.928934in}}%
\pgfpathlineto{\pgfqpoint{0.841966in}{1.975067in}}%
\pgfpathlineto{\pgfqpoint{0.842404in}{0.969952in}}%
\pgfpathlineto{\pgfqpoint{0.843197in}{1.831167in}}%
\pgfpathlineto{\pgfqpoint{0.843971in}{0.983666in}}%
\pgfpathlineto{\pgfqpoint{0.845106in}{1.883622in}}%
\pgfpathlineto{\pgfqpoint{0.845680in}{0.965107in}}%
\pgfpathlineto{\pgfqpoint{0.846544in}{1.846699in}}%
\pgfpathlineto{\pgfqpoint{0.847456in}{0.940398in}}%
\pgfpathlineto{\pgfqpoint{0.848177in}{1.842464in}}%
\pgfpathlineto{\pgfqpoint{0.849116in}{0.913497in}}%
\pgfpathlineto{\pgfqpoint{0.849984in}{1.876954in}}%
\pgfpathlineto{\pgfqpoint{0.850821in}{0.898109in}}%
\pgfpathlineto{\pgfqpoint{0.851353in}{1.807150in}}%
\pgfpathlineto{\pgfqpoint{0.852565in}{0.911759in}}%
\pgfpathlineto{\pgfqpoint{0.853008in}{1.891832in}}%
\pgfpathlineto{\pgfqpoint{0.853846in}{0.997227in}}%
\pgfpathlineto{\pgfqpoint{0.854751in}{1.857390in}}%
\pgfpathlineto{\pgfqpoint{0.855514in}{0.955199in}}%
\pgfpathlineto{\pgfqpoint{0.856431in}{1.906378in}}%
\pgfpathlineto{\pgfqpoint{0.857162in}{0.989701in}}%
\pgfpathlineto{\pgfqpoint{0.858558in}{1.928829in}}%
\pgfpathlineto{\pgfqpoint{0.858767in}{0.932100in}}%
\pgfpathlineto{\pgfqpoint{0.859771in}{1.842806in}}%
\pgfpathlineto{\pgfqpoint{0.860497in}{0.947346in}}%
\pgfpathlineto{\pgfqpoint{0.861262in}{1.859595in}}%
\pgfpathlineto{\pgfqpoint{0.862071in}{0.858575in}}%
\pgfpathlineto{\pgfqpoint{0.862919in}{1.847251in}}%
\pgfpathlineto{\pgfqpoint{0.863717in}{0.936615in}}%
\pgfpathlineto{\pgfqpoint{0.864595in}{1.865887in}}%
\pgfpathlineto{\pgfqpoint{0.865705in}{0.903003in}}%
\pgfpathlineto{\pgfqpoint{0.866647in}{2.024127in}}%
\pgfpathlineto{\pgfqpoint{0.867021in}{0.994802in}}%
\pgfpathlineto{\pgfqpoint{0.868172in}{1.915046in}}%
\pgfpathlineto{\pgfqpoint{0.868754in}{0.900977in}}%
\pgfpathlineto{\pgfqpoint{0.869468in}{1.839651in}}%
\pgfpathlineto{\pgfqpoint{0.870278in}{0.966647in}}%
\pgfpathlineto{\pgfqpoint{0.871130in}{1.801544in}}%
\pgfpathlineto{\pgfqpoint{0.871983in}{0.916570in}}%
\pgfpathlineto{\pgfqpoint{0.872830in}{1.860126in}}%
\pgfpathlineto{\pgfqpoint{0.873726in}{0.975216in}}%
\pgfpathlineto{\pgfqpoint{0.874682in}{1.910150in}}%
\pgfpathlineto{\pgfqpoint{0.875261in}{0.995994in}}%
\pgfpathlineto{\pgfqpoint{0.876055in}{1.924162in}}%
\pgfpathlineto{\pgfqpoint{0.876888in}{1.004656in}}%
\pgfpathlineto{\pgfqpoint{0.877735in}{1.829588in}}%
\pgfpathlineto{\pgfqpoint{0.878562in}{0.968527in}}%
\pgfpathlineto{\pgfqpoint{0.879332in}{1.910136in}}%
\pgfpathlineto{\pgfqpoint{0.880244in}{0.896575in}}%
\pgfpathlineto{\pgfqpoint{0.881492in}{1.954122in}}%
\pgfpathlineto{\pgfqpoint{0.881869in}{0.836526in}}%
\pgfpathlineto{\pgfqpoint{0.882737in}{1.935734in}}%
\pgfpathlineto{\pgfqpoint{0.883681in}{0.934157in}}%
\pgfpathlineto{\pgfqpoint{0.884300in}{1.917465in}}%
\pgfpathlineto{\pgfqpoint{0.885336in}{0.899955in}}%
\pgfpathlineto{\pgfqpoint{0.885921in}{1.945908in}}%
\pgfpathlineto{\pgfqpoint{0.886950in}{0.951583in}}%
\pgfpathlineto{\pgfqpoint{0.887620in}{1.888583in}}%
\pgfpathlineto{\pgfqpoint{0.888628in}{0.935774in}}%
\pgfpathlineto{\pgfqpoint{0.889277in}{1.821069in}}%
\pgfpathlineto{\pgfqpoint{0.890069in}{0.973241in}}%
\pgfpathlineto{\pgfqpoint{0.890925in}{1.891866in}}%
\pgfpathlineto{\pgfqpoint{0.891690in}{0.944985in}}%
\pgfpathlineto{\pgfqpoint{0.892505in}{1.838362in}}%
\pgfpathlineto{\pgfqpoint{0.893566in}{0.929005in}}%
\pgfpathlineto{\pgfqpoint{0.894166in}{1.846158in}}%
\pgfpathlineto{\pgfqpoint{0.895052in}{0.909420in}}%
\pgfpathlineto{\pgfqpoint{0.896111in}{1.908136in}}%
\pgfpathlineto{\pgfqpoint{0.896636in}{1.032220in}}%
\pgfpathlineto{\pgfqpoint{0.897448in}{1.826730in}}%
\pgfpathlineto{\pgfqpoint{0.898274in}{0.947559in}}%
\pgfpathlineto{\pgfqpoint{0.899351in}{1.873340in}}%
\pgfpathlineto{\pgfqpoint{0.900055in}{0.955764in}}%
\pgfpathlineto{\pgfqpoint{0.900780in}{1.868620in}}%
\pgfpathlineto{\pgfqpoint{0.901581in}{0.981084in}}%
\pgfpathlineto{\pgfqpoint{0.902387in}{1.815577in}}%
\pgfpathlineto{\pgfqpoint{0.903511in}{0.942104in}}%
\pgfpathlineto{\pgfqpoint{0.904249in}{1.892713in}}%
\pgfpathlineto{\pgfqpoint{0.905031in}{0.910773in}}%
\pgfpathlineto{\pgfqpoint{0.905868in}{1.921574in}}%
\pgfpathlineto{\pgfqpoint{0.906533in}{0.967817in}}%
\pgfpathlineto{\pgfqpoint{0.907435in}{1.974369in}}%
\pgfpathlineto{\pgfqpoint{0.908360in}{0.931451in}}%
\pgfpathlineto{\pgfqpoint{0.908987in}{1.782586in}}%
\pgfpathlineto{\pgfqpoint{0.909969in}{0.839765in}}%
\pgfpathlineto{\pgfqpoint{0.910610in}{1.865420in}}%
\pgfpathlineto{\pgfqpoint{0.911716in}{0.921522in}}%
\pgfpathlineto{\pgfqpoint{0.912459in}{1.953497in}}%
\pgfpathlineto{\pgfqpoint{0.913256in}{0.942119in}}%
\pgfpathlineto{\pgfqpoint{0.914220in}{1.919305in}}%
\pgfpathlineto{\pgfqpoint{0.914773in}{0.996490in}}%
\pgfpathlineto{\pgfqpoint{0.915640in}{1.880612in}}%
\pgfpathlineto{\pgfqpoint{0.916444in}{0.953390in}}%
\pgfpathlineto{\pgfqpoint{0.917182in}{1.788220in}}%
\pgfpathlineto{\pgfqpoint{0.918334in}{0.915645in}}%
\pgfpathlineto{\pgfqpoint{0.918841in}{1.809948in}}%
\pgfpathlineto{\pgfqpoint{0.919852in}{0.800493in}}%
\pgfpathlineto{\pgfqpoint{0.920704in}{1.901483in}}%
\pgfpathlineto{\pgfqpoint{0.921338in}{0.982846in}}%
\pgfpathlineto{\pgfqpoint{0.922121in}{1.852851in}}%
\pgfpathlineto{\pgfqpoint{0.923505in}{0.830227in}}%
\pgfpathlineto{\pgfqpoint{0.923776in}{1.856446in}}%
\pgfpathlineto{\pgfqpoint{0.924739in}{0.975453in}}%
\pgfpathlineto{\pgfqpoint{0.925696in}{1.974853in}}%
\pgfpathlineto{\pgfqpoint{0.926517in}{0.897205in}}%
\pgfpathlineto{\pgfqpoint{0.927170in}{1.859341in}}%
\pgfpathlineto{\pgfqpoint{0.927908in}{0.983060in}}%
\pgfpathlineto{\pgfqpoint{0.928707in}{1.873349in}}%
\pgfpathlineto{\pgfqpoint{0.929904in}{0.887355in}}%
\pgfpathlineto{\pgfqpoint{0.930446in}{1.856216in}}%
\pgfpathlineto{\pgfqpoint{0.931480in}{0.899779in}}%
\pgfpathlineto{\pgfqpoint{0.932079in}{1.881969in}}%
\pgfpathlineto{\pgfqpoint{0.932849in}{0.997098in}}%
\pgfpathlineto{\pgfqpoint{0.933839in}{1.863773in}}%
\pgfpathlineto{\pgfqpoint{0.934503in}{1.011224in}}%
\pgfpathlineto{\pgfqpoint{0.935553in}{1.870375in}}%
\pgfpathlineto{\pgfqpoint{0.936202in}{0.975855in}}%
\pgfpathlineto{\pgfqpoint{0.936935in}{1.831416in}}%
\pgfpathlineto{\pgfqpoint{0.937854in}{0.979440in}}%
\pgfpathlineto{\pgfqpoint{0.938611in}{1.948745in}}%
\pgfpathlineto{\pgfqpoint{0.939441in}{0.932947in}}%
\pgfpathlineto{\pgfqpoint{0.940447in}{1.873377in}}%
\pgfpathlineto{\pgfqpoint{0.941072in}{0.956613in}}%
\pgfpathlineto{\pgfqpoint{0.942206in}{1.949430in}}%
\pgfpathlineto{\pgfqpoint{0.942776in}{0.958605in}}%
\pgfpathlineto{\pgfqpoint{0.943601in}{1.937305in}}%
\pgfpathlineto{\pgfqpoint{0.944377in}{0.944466in}}%
\pgfpathlineto{\pgfqpoint{0.945215in}{1.867946in}}%
\pgfpathlineto{\pgfqpoint{0.946150in}{0.930644in}}%
\pgfpathlineto{\pgfqpoint{0.946807in}{1.841073in}}%
\pgfpathlineto{\pgfqpoint{0.947639in}{0.889335in}}%
\pgfpathlineto{\pgfqpoint{0.948750in}{1.915814in}}%
\pgfpathlineto{\pgfqpoint{0.949318in}{0.941388in}}%
\pgfpathlineto{\pgfqpoint{0.950326in}{1.852700in}}%
\pgfpathlineto{\pgfqpoint{0.951146in}{0.918533in}}%
\pgfpathlineto{\pgfqpoint{0.951816in}{1.916378in}}%
\pgfpathlineto{\pgfqpoint{0.952570in}{0.976912in}}%
\pgfpathlineto{\pgfqpoint{0.953529in}{1.862099in}}%
\pgfpathlineto{\pgfqpoint{0.954278in}{0.864489in}}%
\pgfpathlineto{\pgfqpoint{0.955312in}{1.929327in}}%
\pgfpathlineto{\pgfqpoint{0.955903in}{0.816075in}}%
\pgfpathlineto{\pgfqpoint{0.956867in}{1.950221in}}%
\pgfpathlineto{\pgfqpoint{0.957672in}{0.897614in}}%
\pgfpathlineto{\pgfqpoint{0.958362in}{1.818784in}}%
\pgfpathlineto{\pgfqpoint{0.959314in}{0.945048in}}%
\pgfpathlineto{\pgfqpoint{0.960036in}{1.794925in}}%
\pgfpathlineto{\pgfqpoint{0.961051in}{0.943806in}}%
\pgfpathlineto{\pgfqpoint{0.961658in}{1.868501in}}%
\pgfpathlineto{\pgfqpoint{0.962469in}{0.930532in}}%
\pgfpathlineto{\pgfqpoint{0.963355in}{1.862916in}}%
\pgfpathlineto{\pgfqpoint{0.964094in}{0.917431in}}%
\pgfpathlineto{\pgfqpoint{0.964911in}{1.827272in}}%
\pgfpathlineto{\pgfqpoint{0.965861in}{0.913737in}}%
\pgfpathlineto{\pgfqpoint{0.966611in}{1.802471in}}%
\pgfpathlineto{\pgfqpoint{0.967532in}{0.888776in}}%
\pgfpathlineto{\pgfqpoint{0.968584in}{1.940575in}}%
\pgfpathlineto{\pgfqpoint{0.969056in}{0.986223in}}%
\pgfpathlineto{\pgfqpoint{0.969879in}{1.835854in}}%
\pgfpathlineto{\pgfqpoint{0.970776in}{0.933819in}}%
\pgfpathlineto{\pgfqpoint{0.971798in}{1.944776in}}%
\pgfpathlineto{\pgfqpoint{0.972366in}{0.947668in}}%
\pgfpathlineto{\pgfqpoint{0.973380in}{1.900666in}}%
\pgfpathlineto{\pgfqpoint{0.974212in}{0.857394in}}%
\pgfpathlineto{\pgfqpoint{0.974917in}{1.908773in}}%
\pgfpathlineto{\pgfqpoint{0.975646in}{0.952821in}}%
\pgfpathlineto{\pgfqpoint{0.976452in}{1.782702in}}%
\pgfpathlineto{\pgfqpoint{0.977397in}{0.895873in}}%
\pgfpathlineto{\pgfqpoint{0.978139in}{1.855442in}}%
\pgfpathlineto{\pgfqpoint{0.979476in}{0.841761in}}%
\pgfpathlineto{\pgfqpoint{0.979776in}{1.881598in}}%
\pgfpathlineto{\pgfqpoint{0.980800in}{0.900177in}}%
\pgfpathlineto{\pgfqpoint{0.981627in}{1.854632in}}%
\pgfpathlineto{\pgfqpoint{0.982485in}{0.925992in}}%
\pgfpathlineto{\pgfqpoint{0.983131in}{1.858845in}}%
\pgfpathlineto{\pgfqpoint{0.983867in}{0.946859in}}%
\pgfpathlineto{\pgfqpoint{0.984689in}{1.864660in}}%
\pgfpathlineto{\pgfqpoint{0.985501in}{0.996232in}}%
\pgfpathlineto{\pgfqpoint{0.986539in}{1.947542in}}%
\pgfpathlineto{\pgfqpoint{0.987291in}{0.835195in}}%
\pgfpathlineto{\pgfqpoint{0.988034in}{1.858824in}}%
\pgfpathlineto{\pgfqpoint{0.988792in}{0.959897in}}%
\pgfpathlineto{\pgfqpoint{0.989604in}{2.001426in}}%
\pgfpathlineto{\pgfqpoint{0.990555in}{0.924408in}}%
\pgfpathlineto{\pgfqpoint{0.991307in}{1.940406in}}%
\pgfpathlineto{\pgfqpoint{0.992094in}{0.920862in}}%
\pgfpathlineto{\pgfqpoint{0.992946in}{1.969269in}}%
\pgfpathlineto{\pgfqpoint{0.993791in}{0.953612in}}%
\pgfpathlineto{\pgfqpoint{0.994556in}{1.827249in}}%
\pgfpathlineto{\pgfqpoint{0.995554in}{0.930453in}}%
\pgfpathlineto{\pgfqpoint{0.996547in}{1.934025in}}%
\pgfpathlineto{\pgfqpoint{0.997018in}{0.980186in}}%
\pgfpathlineto{\pgfqpoint{0.998054in}{1.920400in}}%
\pgfpathlineto{\pgfqpoint{0.998794in}{0.957463in}}%
\pgfpathlineto{\pgfqpoint{0.999481in}{2.036878in}}%
\pgfpathlineto{\pgfqpoint{1.000346in}{0.961627in}}%
\pgfpathlineto{\pgfqpoint{1.001368in}{1.880781in}}%
\pgfpathlineto{\pgfqpoint{1.002167in}{0.934899in}}%
\pgfpathlineto{\pgfqpoint{1.002868in}{1.881968in}}%
\pgfpathlineto{\pgfqpoint{1.003775in}{0.978487in}}%
\pgfpathlineto{\pgfqpoint{1.004583in}{1.930446in}}%
\pgfpathlineto{\pgfqpoint{1.005259in}{0.964250in}}%
\pgfpathlineto{\pgfqpoint{1.006090in}{1.836504in}}%
\pgfpathlineto{\pgfqpoint{1.007028in}{0.976324in}}%
\pgfpathlineto{\pgfqpoint{1.007819in}{1.909598in}}%
\pgfpathlineto{\pgfqpoint{1.008652in}{0.974114in}}%
\pgfpathlineto{\pgfqpoint{1.009525in}{1.886892in}}%
\pgfpathlineto{\pgfqpoint{1.010662in}{0.888121in}}%
\pgfpathlineto{\pgfqpoint{1.011110in}{1.875728in}}%
\pgfpathlineto{\pgfqpoint{1.011889in}{0.994959in}}%
\pgfpathlineto{\pgfqpoint{1.012718in}{1.950309in}}%
\pgfpathlineto{\pgfqpoint{1.013618in}{0.824112in}}%
\pgfpathlineto{\pgfqpoint{1.014308in}{1.824737in}}%
\pgfpathlineto{\pgfqpoint{1.015427in}{0.917662in}}%
\pgfpathlineto{\pgfqpoint{1.015960in}{1.826491in}}%
\pgfpathlineto{\pgfqpoint{1.016870in}{0.965144in}}%
\pgfpathlineto{\pgfqpoint{1.017625in}{1.943877in}}%
\pgfpathlineto{\pgfqpoint{1.018654in}{0.925720in}}%
\pgfpathlineto{\pgfqpoint{1.019390in}{1.878702in}}%
\pgfpathlineto{\pgfqpoint{1.020084in}{0.987823in}}%
\pgfpathlineto{\pgfqpoint{1.020953in}{1.869664in}}%
\pgfpathlineto{\pgfqpoint{1.021876in}{0.922716in}}%
\pgfpathlineto{\pgfqpoint{1.022649in}{1.946487in}}%
\pgfpathlineto{\pgfqpoint{1.023376in}{0.999263in}}%
\pgfpathlineto{\pgfqpoint{1.024307in}{1.858748in}}%
\pgfpathlineto{\pgfqpoint{1.025085in}{0.927571in}}%
\pgfpathlineto{\pgfqpoint{1.025971in}{1.904437in}}%
\pgfpathlineto{\pgfqpoint{1.026655in}{0.933176in}}%
\pgfpathlineto{\pgfqpoint{1.027624in}{1.949315in}}%
\pgfpathlineto{\pgfqpoint{1.028620in}{0.911187in}}%
\pgfpathlineto{\pgfqpoint{1.029128in}{1.834519in}}%
\pgfpathlineto{\pgfqpoint{1.030036in}{0.891842in}}%
\pgfpathlineto{\pgfqpoint{1.030871in}{1.948467in}}%
\pgfpathlineto{\pgfqpoint{1.031789in}{0.894160in}}%
\pgfpathlineto{\pgfqpoint{1.032756in}{1.895186in}}%
\pgfpathlineto{\pgfqpoint{1.033227in}{1.007060in}}%
\pgfpathlineto{\pgfqpoint{1.034248in}{1.879433in}}%
\pgfpathlineto{\pgfqpoint{1.034947in}{1.021877in}}%
\pgfpathlineto{\pgfqpoint{1.035886in}{1.885724in}}%
\pgfpathlineto{\pgfqpoint{1.036528in}{0.978898in}}%
\pgfpathlineto{\pgfqpoint{1.037508in}{1.888315in}}%
\pgfpathlineto{\pgfqpoint{1.038156in}{1.033792in}}%
\pgfpathlineto{\pgfqpoint{1.039024in}{1.901940in}}%
\pgfpathlineto{\pgfqpoint{1.039892in}{0.937945in}}%
\pgfpathlineto{\pgfqpoint{1.040954in}{1.869001in}}%
\pgfpathlineto{\pgfqpoint{1.041631in}{0.872977in}}%
\pgfpathlineto{\pgfqpoint{1.042294in}{1.864400in}}%
\pgfpathlineto{\pgfqpoint{1.043178in}{0.937488in}}%
\pgfpathlineto{\pgfqpoint{1.043968in}{1.836261in}}%
\pgfpathlineto{\pgfqpoint{1.044845in}{0.912365in}}%
\pgfpathlineto{\pgfqpoint{1.045704in}{1.851276in}}%
\pgfpathlineto{\pgfqpoint{1.046418in}{0.926715in}}%
\pgfpathlineto{\pgfqpoint{1.047448in}{1.903917in}}%
\pgfpathlineto{\pgfqpoint{1.048051in}{0.914358in}}%
\pgfpathlineto{\pgfqpoint{1.048880in}{1.931334in}}%
\pgfpathlineto{\pgfqpoint{1.049711in}{1.012311in}}%
\pgfpathlineto{\pgfqpoint{1.050597in}{1.829933in}}%
\pgfpathlineto{\pgfqpoint{1.051323in}{0.944522in}}%
\pgfpathlineto{\pgfqpoint{1.052280in}{1.876502in}}%
\pgfpathlineto{\pgfqpoint{1.053054in}{1.007435in}}%
\pgfpathlineto{\pgfqpoint{1.053919in}{1.827578in}}%
\pgfpathlineto{\pgfqpoint{1.054676in}{0.964274in}}%
\pgfpathlineto{\pgfqpoint{1.055568in}{1.875377in}}%
\pgfpathlineto{\pgfqpoint{1.056255in}{0.905686in}}%
\pgfpathlineto{\pgfqpoint{1.057089in}{1.848011in}}%
\pgfpathlineto{\pgfqpoint{1.058071in}{0.786896in}}%
\pgfpathlineto{\pgfqpoint{1.058806in}{1.807567in}}%
\pgfpathlineto{\pgfqpoint{1.059632in}{0.997363in}}%
\pgfpathlineto{\pgfqpoint{1.060424in}{1.900758in}}%
\pgfpathlineto{\pgfqpoint{1.061323in}{0.951698in}}%
\pgfpathlineto{\pgfqpoint{1.062072in}{1.864918in}}%
\pgfpathlineto{\pgfqpoint{1.063104in}{0.941079in}}%
\pgfpathlineto{\pgfqpoint{1.063719in}{1.943079in}}%
\pgfpathlineto{\pgfqpoint{1.064485in}{0.948131in}}%
\pgfpathlineto{\pgfqpoint{1.065309in}{1.821880in}}%
\pgfpathlineto{\pgfqpoint{1.066448in}{0.942999in}}%
\pgfpathlineto{\pgfqpoint{1.067113in}{1.851704in}}%
\pgfpathlineto{\pgfqpoint{1.067831in}{0.924133in}}%
\pgfpathlineto{\pgfqpoint{1.068890in}{2.008441in}}%
\pgfpathlineto{\pgfqpoint{1.069466in}{0.998968in}}%
\pgfpathlineto{\pgfqpoint{1.070620in}{1.938726in}}%
\pgfpathlineto{\pgfqpoint{1.071156in}{0.908493in}}%
\pgfpathlineto{\pgfqpoint{1.071919in}{1.931548in}}%
\pgfpathlineto{\pgfqpoint{1.072815in}{0.964615in}}%
\pgfpathlineto{\pgfqpoint{1.073549in}{1.837764in}}%
\pgfpathlineto{\pgfqpoint{1.074359in}{1.028163in}}%
\pgfpathlineto{\pgfqpoint{1.075196in}{1.939959in}}%
\pgfpathlineto{\pgfqpoint{1.076456in}{0.865203in}}%
\pgfpathlineto{\pgfqpoint{1.076932in}{1.940698in}}%
\pgfpathlineto{\pgfqpoint{1.077655in}{0.953265in}}%
\pgfpathlineto{\pgfqpoint{1.078653in}{1.873809in}}%
\pgfpathlineto{\pgfqpoint{1.079601in}{0.881150in}}%
\pgfpathlineto{\pgfqpoint{1.080117in}{1.942272in}}%
\pgfpathlineto{\pgfqpoint{1.081127in}{0.920678in}}%
\pgfpathlineto{\pgfqpoint{1.081765in}{1.863401in}}%
\pgfpathlineto{\pgfqpoint{1.082852in}{0.909065in}}%
\pgfpathlineto{\pgfqpoint{1.083452in}{1.916314in}}%
\pgfpathlineto{\pgfqpoint{1.084298in}{0.852558in}}%
\pgfpathlineto{\pgfqpoint{1.085105in}{1.829058in}}%
\pgfpathlineto{\pgfqpoint{1.086050in}{0.919632in}}%
\pgfpathlineto{\pgfqpoint{1.086766in}{1.979659in}}%
\pgfpathlineto{\pgfqpoint{1.087644in}{0.971216in}}%
\pgfpathlineto{\pgfqpoint{1.088402in}{1.845712in}}%
\pgfpathlineto{\pgfqpoint{1.089187in}{0.967573in}}%
\pgfpathlineto{\pgfqpoint{1.090303in}{1.950956in}}%
\pgfpathlineto{\pgfqpoint{1.090839in}{0.974458in}}%
\pgfpathlineto{\pgfqpoint{1.091644in}{1.889084in}}%
\pgfpathlineto{\pgfqpoint{1.092684in}{0.934198in}}%
\pgfpathlineto{\pgfqpoint{1.093425in}{1.878181in}}%
\pgfpathlineto{\pgfqpoint{1.094158in}{0.869170in}}%
\pgfpathlineto{\pgfqpoint{1.094960in}{1.777946in}}%
\pgfpathlineto{\pgfqpoint{1.095930in}{0.951090in}}%
\pgfpathlineto{\pgfqpoint{1.096746in}{1.921961in}}%
\pgfpathlineto{\pgfqpoint{1.097527in}{0.952075in}}%
\pgfpathlineto{\pgfqpoint{1.098236in}{1.899088in}}%
\pgfpathlineto{\pgfqpoint{1.099071in}{1.000413in}}%
\pgfpathlineto{\pgfqpoint{1.099908in}{1.838915in}}%
\pgfpathlineto{\pgfqpoint{1.100782in}{0.971551in}}%
\pgfpathlineto{\pgfqpoint{1.101566in}{1.862371in}}%
\pgfpathlineto{\pgfqpoint{1.102337in}{0.951583in}}%
\pgfpathlineto{\pgfqpoint{1.103265in}{1.898583in}}%
\pgfpathlineto{\pgfqpoint{1.103985in}{0.966560in}}%
\pgfpathlineto{\pgfqpoint{1.104833in}{1.836054in}}%
\pgfpathlineto{\pgfqpoint{1.105831in}{0.889052in}}%
\pgfpathlineto{\pgfqpoint{1.106460in}{1.796155in}}%
\pgfpathlineto{\pgfqpoint{1.107420in}{0.929485in}}%
\pgfpathlineto{\pgfqpoint{1.108157in}{1.875975in}}%
\pgfpathlineto{\pgfqpoint{1.109055in}{0.911203in}}%
\pgfpathlineto{\pgfqpoint{1.110079in}{1.923684in}}%
\pgfpathlineto{\pgfqpoint{1.110571in}{0.962391in}}%
\pgfpathlineto{\pgfqpoint{1.111399in}{1.870648in}}%
\pgfpathlineto{\pgfqpoint{1.112250in}{0.951023in}}%
\pgfpathlineto{\pgfqpoint{1.113092in}{1.866469in}}%
\pgfpathlineto{\pgfqpoint{1.114000in}{0.921416in}}%
\pgfpathlineto{\pgfqpoint{1.114694in}{1.839079in}}%
\pgfpathlineto{\pgfqpoint{1.115745in}{0.947215in}}%
\pgfpathlineto{\pgfqpoint{1.116421in}{1.842394in}}%
\pgfpathlineto{\pgfqpoint{1.117297in}{0.858612in}}%
\pgfpathlineto{\pgfqpoint{1.118168in}{1.848415in}}%
\pgfpathlineto{\pgfqpoint{1.118826in}{0.947497in}}%
\pgfpathlineto{\pgfqpoint{1.119786in}{1.878342in}}%
\pgfpathlineto{\pgfqpoint{1.120510in}{0.934920in}}%
\pgfpathlineto{\pgfqpoint{1.121322in}{1.887029in}}%
\pgfpathlineto{\pgfqpoint{1.122123in}{0.853663in}}%
\pgfpathlineto{\pgfqpoint{1.122931in}{1.831485in}}%
\pgfpathlineto{\pgfqpoint{1.123761in}{0.951090in}}%
\pgfpathlineto{\pgfqpoint{1.124681in}{1.896780in}}%
\pgfpathlineto{\pgfqpoint{1.125773in}{0.866871in}}%
\pgfpathlineto{\pgfqpoint{1.126379in}{1.886518in}}%
\pgfpathlineto{\pgfqpoint{1.127194in}{0.938202in}}%
\pgfpathlineto{\pgfqpoint{1.128049in}{1.860474in}}%
\pgfpathlineto{\pgfqpoint{1.128786in}{0.868156in}}%
\pgfpathlineto{\pgfqpoint{1.129493in}{1.856836in}}%
\pgfpathlineto{\pgfqpoint{1.130489in}{0.960726in}}%
\pgfpathlineto{\pgfqpoint{1.131320in}{1.869323in}}%
\pgfpathlineto{\pgfqpoint{1.131978in}{0.982989in}}%
\pgfpathlineto{\pgfqpoint{1.133086in}{1.903635in}}%
\pgfpathlineto{\pgfqpoint{1.133856in}{0.861757in}}%
\pgfpathlineto{\pgfqpoint{1.134555in}{1.835722in}}%
\pgfpathlineto{\pgfqpoint{1.135485in}{0.991542in}}%
\pgfpathlineto{\pgfqpoint{1.136238in}{1.881432in}}%
\pgfpathlineto{\pgfqpoint{1.136902in}{0.990356in}}%
\pgfpathlineto{\pgfqpoint{1.137765in}{1.912173in}}%
\pgfpathlineto{\pgfqpoint{1.138888in}{0.865190in}}%
\pgfpathlineto{\pgfqpoint{1.139376in}{1.847432in}}%
\pgfpathlineto{\pgfqpoint{1.140290in}{0.927398in}}%
\pgfpathlineto{\pgfqpoint{1.141111in}{1.891210in}}%
\pgfpathlineto{\pgfqpoint{1.142012in}{0.901695in}}%
\pgfpathlineto{\pgfqpoint{1.142736in}{1.864733in}}%
\pgfpathlineto{\pgfqpoint{1.143554in}{0.969740in}}%
\pgfpathlineto{\pgfqpoint{1.144702in}{1.979212in}}%
\pgfpathlineto{\pgfqpoint{1.145266in}{0.983327in}}%
\pgfpathlineto{\pgfqpoint{1.145964in}{1.876253in}}%
\pgfpathlineto{\pgfqpoint{1.146938in}{0.946770in}}%
\pgfpathlineto{\pgfqpoint{1.147856in}{1.898882in}}%
\pgfpathlineto{\pgfqpoint{1.148679in}{0.910798in}}%
\pgfpathlineto{\pgfqpoint{1.149467in}{1.866954in}}%
\pgfpathlineto{\pgfqpoint{1.150098in}{0.953051in}}%
\pgfpathlineto{\pgfqpoint{1.151030in}{1.849328in}}%
\pgfpathlineto{\pgfqpoint{1.151722in}{0.964660in}}%
\pgfpathlineto{\pgfqpoint{1.152599in}{1.848290in}}%
\pgfpathlineto{\pgfqpoint{1.153642in}{0.905327in}}%
\pgfpathlineto{\pgfqpoint{1.154305in}{1.937473in}}%
\pgfpathlineto{\pgfqpoint{1.155300in}{0.866231in}}%
\pgfpathlineto{\pgfqpoint{1.156042in}{1.855845in}}%
\pgfpathlineto{\pgfqpoint{1.156654in}{0.965440in}}%
\pgfpathlineto{\pgfqpoint{1.157659in}{1.910776in}}%
\pgfpathlineto{\pgfqpoint{1.158311in}{0.976488in}}%
\pgfpathlineto{\pgfqpoint{1.159139in}{1.824588in}}%
\pgfpathlineto{\pgfqpoint{1.159997in}{0.960220in}}%
\pgfpathlineto{\pgfqpoint{1.160917in}{1.872936in}}%
\pgfpathlineto{\pgfqpoint{1.161718in}{0.858554in}}%
\pgfpathlineto{\pgfqpoint{1.162427in}{1.918735in}}%
\pgfpathlineto{\pgfqpoint{1.163285in}{0.950870in}}%
\pgfpathlineto{\pgfqpoint{1.164204in}{1.899257in}}%
\pgfpathlineto{\pgfqpoint{1.164877in}{0.994891in}}%
\pgfpathlineto{\pgfqpoint{1.165754in}{1.870946in}}%
\pgfpathlineto{\pgfqpoint{1.166552in}{0.899499in}}%
\pgfpathlineto{\pgfqpoint{1.167410in}{1.834458in}}%
\pgfpathlineto{\pgfqpoint{1.168587in}{0.845813in}}%
\pgfpathlineto{\pgfqpoint{1.169088in}{1.896937in}}%
\pgfpathlineto{\pgfqpoint{1.169868in}{0.886505in}}%
\pgfpathlineto{\pgfqpoint{1.170658in}{1.831432in}}%
\pgfpathlineto{\pgfqpoint{1.171485in}{0.986606in}}%
\pgfpathlineto{\pgfqpoint{1.172486in}{1.853588in}}%
\pgfpathlineto{\pgfqpoint{1.173215in}{0.886229in}}%
\pgfpathlineto{\pgfqpoint{1.174076in}{1.878325in}}%
\pgfpathlineto{\pgfqpoint{1.174789in}{0.964919in}}%
\pgfpathlineto{\pgfqpoint{1.175642in}{1.800946in}}%
\pgfpathlineto{\pgfqpoint{1.177017in}{0.853167in}}%
\pgfpathlineto{\pgfqpoint{1.177305in}{1.901475in}}%
\pgfpathlineto{\pgfqpoint{1.178249in}{0.947481in}}%
\pgfpathlineto{\pgfqpoint{1.178989in}{1.813654in}}%
\pgfpathlineto{\pgfqpoint{1.179809in}{0.978280in}}%
\pgfpathlineto{\pgfqpoint{1.180549in}{1.842963in}}%
\pgfpathlineto{\pgfqpoint{1.181393in}{0.982913in}}%
\pgfpathlineto{\pgfqpoint{1.182263in}{1.894624in}}%
\pgfpathlineto{\pgfqpoint{1.183006in}{0.953425in}}%
\pgfpathlineto{\pgfqpoint{1.183808in}{1.880532in}}%
\pgfpathlineto{\pgfqpoint{1.185058in}{0.862992in}}%
\pgfpathlineto{\pgfqpoint{1.185887in}{1.974264in}}%
\pgfpathlineto{\pgfqpoint{1.186295in}{0.953757in}}%
\pgfpathlineto{\pgfqpoint{1.187160in}{1.891388in}}%
\pgfpathlineto{\pgfqpoint{1.188043in}{0.933731in}}%
\pgfpathlineto{\pgfqpoint{1.188877in}{1.850812in}}%
\pgfpathlineto{\pgfqpoint{1.189649in}{0.960600in}}%
\pgfpathlineto{\pgfqpoint{1.190434in}{1.881523in}}%
\pgfpathlineto{\pgfqpoint{1.191300in}{0.911256in}}%
\pgfpathlineto{\pgfqpoint{1.192207in}{1.881107in}}%
\pgfpathlineto{\pgfqpoint{1.193045in}{0.965056in}}%
\pgfpathlineto{\pgfqpoint{1.193816in}{1.894738in}}%
\pgfpathlineto{\pgfqpoint{1.194510in}{1.019037in}}%
\pgfpathlineto{\pgfqpoint{1.195936in}{1.981488in}}%
\pgfpathlineto{\pgfqpoint{1.196163in}{0.977978in}}%
\pgfpathlineto{\pgfqpoint{1.196979in}{1.861549in}}%
\pgfpathlineto{\pgfqpoint{1.197834in}{0.959206in}}%
\pgfpathlineto{\pgfqpoint{1.198937in}{1.896963in}}%
\pgfpathlineto{\pgfqpoint{1.199499in}{0.973849in}}%
\pgfpathlineto{\pgfqpoint{1.200333in}{1.807918in}}%
\pgfpathlineto{\pgfqpoint{1.201110in}{0.987127in}}%
\pgfpathlineto{\pgfqpoint{1.202380in}{1.927327in}}%
\pgfpathlineto{\pgfqpoint{1.202809in}{0.962885in}}%
\pgfpathlineto{\pgfqpoint{1.203571in}{1.914621in}}%
\pgfpathlineto{\pgfqpoint{1.204426in}{0.974147in}}%
\pgfpathlineto{\pgfqpoint{1.205200in}{1.798490in}}%
\pgfpathlineto{\pgfqpoint{1.206326in}{0.915482in}}%
\pgfpathlineto{\pgfqpoint{1.207309in}{1.917270in}}%
\pgfpathlineto{\pgfqpoint{1.207787in}{0.928207in}}%
\pgfpathlineto{\pgfqpoint{1.208690in}{1.886971in}}%
\pgfpathlineto{\pgfqpoint{1.209575in}{0.902663in}}%
\pgfpathlineto{\pgfqpoint{1.210174in}{1.904951in}}%
\pgfpathlineto{\pgfqpoint{1.210964in}{1.010502in}}%
\pgfpathlineto{\pgfqpoint{1.211826in}{1.860244in}}%
\pgfpathlineto{\pgfqpoint{1.212837in}{0.889499in}}%
\pgfpathlineto{\pgfqpoint{1.213430in}{1.832052in}}%
\pgfpathlineto{\pgfqpoint{1.214288in}{0.974583in}}%
\pgfpathlineto{\pgfqpoint{1.215222in}{1.893066in}}%
\pgfpathlineto{\pgfqpoint{1.215992in}{0.992055in}}%
\pgfpathlineto{\pgfqpoint{1.216770in}{1.835041in}}%
\pgfpathlineto{\pgfqpoint{1.217749in}{0.863188in}}%
\pgfpathlineto{\pgfqpoint{1.218369in}{1.859496in}}%
\pgfpathlineto{\pgfqpoint{1.219427in}{0.926951in}}%
\pgfpathlineto{\pgfqpoint{1.220335in}{1.957477in}}%
\pgfpathlineto{\pgfqpoint{1.220923in}{1.002082in}}%
\pgfpathlineto{\pgfqpoint{1.221795in}{1.951000in}}%
\pgfpathlineto{\pgfqpoint{1.222493in}{0.980578in}}%
\pgfpathlineto{\pgfqpoint{1.223307in}{1.889615in}}%
\pgfpathlineto{\pgfqpoint{1.224377in}{0.947006in}}%
\pgfpathlineto{\pgfqpoint{1.224965in}{1.859111in}}%
\pgfpathlineto{\pgfqpoint{1.225972in}{0.951100in}}%
\pgfpathlineto{\pgfqpoint{1.226740in}{1.919869in}}%
\pgfpathlineto{\pgfqpoint{1.227416in}{0.971016in}}%
\pgfpathlineto{\pgfqpoint{1.228277in}{1.966905in}}%
\pgfpathlineto{\pgfqpoint{1.229315in}{0.941336in}}%
\pgfpathlineto{\pgfqpoint{1.230153in}{1.865463in}}%
\pgfpathlineto{\pgfqpoint{1.230723in}{0.997048in}}%
\pgfpathlineto{\pgfqpoint{1.231555in}{1.875947in}}%
\pgfpathlineto{\pgfqpoint{1.232359in}{0.933145in}}%
\pgfpathlineto{\pgfqpoint{1.233251in}{1.879851in}}%
\pgfpathlineto{\pgfqpoint{1.234092in}{0.981762in}}%
\pgfpathlineto{\pgfqpoint{1.234858in}{1.923212in}}%
\pgfpathlineto{\pgfqpoint{1.235942in}{0.914275in}}%
\pgfpathlineto{\pgfqpoint{1.236573in}{1.841437in}}%
\pgfpathlineto{\pgfqpoint{1.237595in}{0.924644in}}%
\pgfpathlineto{\pgfqpoint{1.238220in}{1.820725in}}%
\pgfpathlineto{\pgfqpoint{1.238961in}{0.921503in}}%
\pgfpathlineto{\pgfqpoint{1.239855in}{1.884637in}}%
\pgfpathlineto{\pgfqpoint{1.240607in}{1.021500in}}%
\pgfpathlineto{\pgfqpoint{1.241468in}{1.992413in}}%
\pgfpathlineto{\pgfqpoint{1.242379in}{0.963495in}}%
\pgfpathlineto{\pgfqpoint{1.243090in}{1.894443in}}%
\pgfpathlineto{\pgfqpoint{1.244009in}{0.963752in}}%
\pgfpathlineto{\pgfqpoint{1.244708in}{1.901535in}}%
\pgfpathlineto{\pgfqpoint{1.245585in}{0.904545in}}%
\pgfpathlineto{\pgfqpoint{1.246359in}{1.895568in}}%
\pgfpathlineto{\pgfqpoint{1.247507in}{0.825508in}}%
\pgfpathlineto{\pgfqpoint{1.248013in}{1.796018in}}%
\pgfpathlineto{\pgfqpoint{1.248909in}{0.935858in}}%
\pgfpathlineto{\pgfqpoint{1.249967in}{1.921679in}}%
\pgfpathlineto{\pgfqpoint{1.250572in}{0.939715in}}%
\pgfpathlineto{\pgfqpoint{1.251517in}{1.944285in}}%
\pgfpathlineto{\pgfqpoint{1.252156in}{0.949982in}}%
\pgfpathlineto{\pgfqpoint{1.253027in}{2.007045in}}%
\pgfpathlineto{\pgfqpoint{1.253881in}{0.951451in}}%
\pgfpathlineto{\pgfqpoint{1.254589in}{1.879205in}}%
\pgfpathlineto{\pgfqpoint{1.255497in}{0.887253in}}%
\pgfpathlineto{\pgfqpoint{1.256378in}{1.893316in}}%
\pgfpathlineto{\pgfqpoint{1.257055in}{1.000399in}}%
\pgfpathlineto{\pgfqpoint{1.257928in}{1.821947in}}%
\pgfpathlineto{\pgfqpoint{1.258687in}{1.007435in}}%
\pgfpathlineto{\pgfqpoint{1.259574in}{1.816938in}}%
\pgfpathlineto{\pgfqpoint{1.260566in}{0.915229in}}%
\pgfpathlineto{\pgfqpoint{1.261269in}{1.889195in}}%
\pgfpathlineto{\pgfqpoint{1.261980in}{0.904820in}}%
\pgfpathlineto{\pgfqpoint{1.262858in}{1.806096in}}%
\pgfpathlineto{\pgfqpoint{1.263863in}{0.851989in}}%
\pgfpathlineto{\pgfqpoint{1.264509in}{1.886477in}}%
\pgfpathlineto{\pgfqpoint{1.265334in}{0.923469in}}%
\pgfpathlineto{\pgfqpoint{1.266115in}{1.868280in}}%
\pgfpathlineto{\pgfqpoint{1.266953in}{0.907384in}}%
\pgfpathlineto{\pgfqpoint{1.267797in}{1.803340in}}%
\pgfpathlineto{\pgfqpoint{1.268575in}{0.909992in}}%
\pgfpathlineto{\pgfqpoint{1.269418in}{1.862310in}}%
\pgfpathlineto{\pgfqpoint{1.270249in}{0.999626in}}%
\pgfpathlineto{\pgfqpoint{1.271239in}{1.887390in}}%
\pgfpathlineto{\pgfqpoint{1.271918in}{0.873704in}}%
\pgfpathlineto{\pgfqpoint{1.272760in}{1.877890in}}%
\pgfpathlineto{\pgfqpoint{1.273553in}{0.971879in}}%
\pgfpathlineto{\pgfqpoint{1.274446in}{1.852103in}}%
\pgfpathlineto{\pgfqpoint{1.275432in}{0.895076in}}%
\pgfpathlineto{\pgfqpoint{1.276033in}{1.829386in}}%
\pgfpathlineto{\pgfqpoint{1.276817in}{0.929191in}}%
\pgfpathlineto{\pgfqpoint{1.277682in}{1.885911in}}%
\pgfpathlineto{\pgfqpoint{1.278540in}{0.915347in}}%
\pgfpathlineto{\pgfqpoint{1.279306in}{1.901149in}}%
\pgfpathlineto{\pgfqpoint{1.280120in}{0.976097in}}%
\pgfpathlineto{\pgfqpoint{1.280917in}{1.825243in}}%
\pgfpathlineto{\pgfqpoint{1.281902in}{0.963294in}}%
\pgfpathlineto{\pgfqpoint{1.282753in}{1.875041in}}%
\pgfpathlineto{\pgfqpoint{1.283390in}{1.027344in}}%
\pgfpathlineto{\pgfqpoint{1.284248in}{1.879295in}}%
\pgfpathlineto{\pgfqpoint{1.285194in}{0.946102in}}%
\pgfpathlineto{\pgfqpoint{1.286026in}{1.901073in}}%
\pgfpathlineto{\pgfqpoint{1.286671in}{0.955314in}}%
\pgfpathlineto{\pgfqpoint{1.287772in}{1.921068in}}%
\pgfpathlineto{\pgfqpoint{1.288314in}{1.001050in}}%
\pgfpathlineto{\pgfqpoint{1.289296in}{1.906448in}}%
\pgfpathlineto{\pgfqpoint{1.290028in}{0.975968in}}%
\pgfpathlineto{\pgfqpoint{1.290804in}{1.869364in}}%
\pgfpathlineto{\pgfqpoint{1.291766in}{0.968715in}}%
\pgfpathlineto{\pgfqpoint{1.292774in}{1.973940in}}%
\pgfpathlineto{\pgfqpoint{1.293271in}{0.981487in}}%
\pgfpathlineto{\pgfqpoint{1.294173in}{1.856619in}}%
\pgfpathlineto{\pgfqpoint{1.294972in}{0.915883in}}%
\pgfpathlineto{\pgfqpoint{1.295853in}{1.887979in}}%
\pgfpathlineto{\pgfqpoint{1.296598in}{0.859700in}}%
\pgfpathlineto{\pgfqpoint{1.297395in}{1.848840in}}%
\pgfpathlineto{\pgfqpoint{1.298278in}{0.953005in}}%
\pgfpathlineto{\pgfqpoint{1.299136in}{1.956441in}}%
\pgfpathlineto{\pgfqpoint{1.299869in}{1.031515in}}%
\pgfpathlineto{\pgfqpoint{1.300740in}{1.856035in}}%
\pgfpathlineto{\pgfqpoint{1.301560in}{0.914517in}}%
\pgfpathlineto{\pgfqpoint{1.302327in}{1.822561in}}%
\pgfpathlineto{\pgfqpoint{1.303502in}{0.828037in}}%
\pgfpathlineto{\pgfqpoint{1.304015in}{1.847101in}}%
\pgfpathlineto{\pgfqpoint{1.304821in}{0.969415in}}%
\pgfpathlineto{\pgfqpoint{1.305888in}{1.873932in}}%
\pgfpathlineto{\pgfqpoint{1.306500in}{0.953758in}}%
\pgfpathlineto{\pgfqpoint{1.307254in}{1.856073in}}%
\pgfpathlineto{\pgfqpoint{1.308138in}{0.917864in}}%
\pgfpathlineto{\pgfqpoint{1.309314in}{1.910684in}}%
\pgfpathlineto{\pgfqpoint{1.309840in}{0.880099in}}%
\pgfpathlineto{\pgfqpoint{1.310531in}{1.875721in}}%
\pgfpathlineto{\pgfqpoint{1.311436in}{0.942457in}}%
\pgfpathlineto{\pgfqpoint{1.312335in}{1.858647in}}%
\pgfpathlineto{\pgfqpoint{1.313018in}{1.035197in}}%
\pgfpathlineto{\pgfqpoint{1.313863in}{1.861108in}}%
\pgfpathlineto{\pgfqpoint{1.314803in}{0.929041in}}%
\pgfpathlineto{\pgfqpoint{1.315519in}{1.866458in}}%
\pgfpathlineto{\pgfqpoint{1.316487in}{0.920992in}}%
\pgfpathlineto{\pgfqpoint{1.317366in}{1.892515in}}%
\pgfpathlineto{\pgfqpoint{1.318162in}{0.856966in}}%
\pgfpathlineto{\pgfqpoint{1.318777in}{1.916302in}}%
\pgfpathlineto{\pgfqpoint{1.319599in}{0.989639in}}%
\pgfpathlineto{\pgfqpoint{1.320466in}{1.849372in}}%
\pgfpathlineto{\pgfqpoint{1.321439in}{0.945237in}}%
\pgfpathlineto{\pgfqpoint{1.322266in}{1.967647in}}%
\pgfpathlineto{\pgfqpoint{1.323135in}{0.933812in}}%
\pgfpathlineto{\pgfqpoint{1.323716in}{1.889714in}}%
\pgfpathlineto{\pgfqpoint{1.324555in}{0.976982in}}%
\pgfpathlineto{\pgfqpoint{1.325648in}{1.864980in}}%
\pgfpathlineto{\pgfqpoint{1.326178in}{0.943825in}}%
\pgfpathlineto{\pgfqpoint{1.327043in}{1.876597in}}%
\pgfpathlineto{\pgfqpoint{1.327833in}{0.979456in}}%
\pgfpathlineto{\pgfqpoint{1.328841in}{1.929790in}}%
\pgfpathlineto{\pgfqpoint{1.329710in}{0.922829in}}%
\pgfpathlineto{\pgfqpoint{1.330425in}{1.892845in}}%
\pgfpathlineto{\pgfqpoint{1.331111in}{0.995993in}}%
\pgfpathlineto{\pgfqpoint{1.332053in}{1.873762in}}%
\pgfpathlineto{\pgfqpoint{1.333110in}{0.883229in}}%
\pgfpathlineto{\pgfqpoint{1.333638in}{1.912781in}}%
\pgfpathlineto{\pgfqpoint{1.334573in}{0.858840in}}%
\pgfpathlineto{\pgfqpoint{1.335383in}{1.854315in}}%
\pgfpathlineto{\pgfqpoint{1.336175in}{0.924151in}}%
\pgfpathlineto{\pgfqpoint{1.337279in}{1.943278in}}%
\pgfpathlineto{\pgfqpoint{1.337753in}{0.969261in}}%
\pgfpathlineto{\pgfqpoint{1.339008in}{1.935669in}}%
\pgfpathlineto{\pgfqpoint{1.339361in}{1.026058in}}%
\pgfpathlineto{\pgfqpoint{1.340450in}{1.917379in}}%
\pgfpathlineto{\pgfqpoint{1.341040in}{0.883527in}}%
\pgfpathlineto{\pgfqpoint{1.341820in}{1.855810in}}%
\pgfpathlineto{\pgfqpoint{1.342702in}{0.921526in}}%
\pgfpathlineto{\pgfqpoint{1.343672in}{1.939965in}}%
\pgfpathlineto{\pgfqpoint{1.344346in}{0.918783in}}%
\pgfpathlineto{\pgfqpoint{1.345396in}{1.880082in}}%
\pgfpathlineto{\pgfqpoint{1.345954in}{0.939865in}}%
\pgfpathlineto{\pgfqpoint{1.346912in}{1.876810in}}%
\pgfpathlineto{\pgfqpoint{1.347562in}{1.023936in}}%
\pgfpathlineto{\pgfqpoint{1.348408in}{1.832466in}}%
\pgfpathlineto{\pgfqpoint{1.349235in}{0.998527in}}%
\pgfpathlineto{\pgfqpoint{1.350221in}{1.861057in}}%
\pgfpathlineto{\pgfqpoint{1.351090in}{0.864700in}}%
\pgfpathlineto{\pgfqpoint{1.351737in}{1.805355in}}%
\pgfpathlineto{\pgfqpoint{1.352755in}{0.920206in}}%
\pgfpathlineto{\pgfqpoint{1.353512in}{1.897522in}}%
\pgfpathlineto{\pgfqpoint{1.354191in}{0.962596in}}%
\pgfpathlineto{\pgfqpoint{1.355029in}{1.861403in}}%
\pgfpathlineto{\pgfqpoint{1.355971in}{0.902557in}}%
\pgfpathlineto{\pgfqpoint{1.356616in}{1.803598in}}%
\pgfpathlineto{\pgfqpoint{1.357782in}{0.911220in}}%
\pgfpathlineto{\pgfqpoint{1.358285in}{1.898375in}}%
\pgfpathlineto{\pgfqpoint{1.359340in}{0.919967in}}%
\pgfpathlineto{\pgfqpoint{1.359937in}{1.810196in}}%
\pgfpathlineto{\pgfqpoint{1.360917in}{0.935891in}}%
\pgfpathlineto{\pgfqpoint{1.361578in}{1.871366in}}%
\pgfpathlineto{\pgfqpoint{1.362596in}{0.890955in}}%
\pgfpathlineto{\pgfqpoint{1.363274in}{1.868869in}}%
\pgfpathlineto{\pgfqpoint{1.364306in}{0.896138in}}%
\pgfpathlineto{\pgfqpoint{1.364892in}{1.815266in}}%
\pgfpathlineto{\pgfqpoint{1.365708in}{0.929000in}}%
\pgfpathlineto{\pgfqpoint{1.366621in}{1.988175in}}%
\pgfpathlineto{\pgfqpoint{1.367432in}{0.906602in}}%
\pgfpathlineto{\pgfqpoint{1.368294in}{1.863340in}}%
\pgfpathlineto{\pgfqpoint{1.369114in}{0.944718in}}%
\pgfpathlineto{\pgfqpoint{1.369790in}{1.880110in}}%
\pgfpathlineto{\pgfqpoint{1.370732in}{0.910682in}}%
\pgfpathlineto{\pgfqpoint{1.371544in}{1.891047in}}%
\pgfpathlineto{\pgfqpoint{1.372716in}{0.856668in}}%
\pgfpathlineto{\pgfqpoint{1.373173in}{1.847055in}}%
\pgfpathlineto{\pgfqpoint{1.374242in}{0.885500in}}%
\pgfpathlineto{\pgfqpoint{1.374818in}{1.880778in}}%
\pgfpathlineto{\pgfqpoint{1.375669in}{1.000058in}}%
\pgfpathlineto{\pgfqpoint{1.376371in}{1.845650in}}%
\pgfpathlineto{\pgfqpoint{1.377358in}{0.925514in}}%
\pgfpathlineto{\pgfqpoint{1.378023in}{1.864982in}}%
\pgfpathlineto{\pgfqpoint{1.378865in}{0.979031in}}%
\pgfpathlineto{\pgfqpoint{1.379743in}{1.868278in}}%
\pgfpathlineto{\pgfqpoint{1.380503in}{0.869016in}}%
\pgfpathlineto{\pgfqpoint{1.381451in}{1.874065in}}%
\pgfpathlineto{\pgfqpoint{1.382126in}{0.968639in}}%
\pgfpathlineto{\pgfqpoint{1.383026in}{1.908605in}}%
\pgfpathlineto{\pgfqpoint{1.383921in}{0.964758in}}%
\pgfpathlineto{\pgfqpoint{1.384663in}{1.882017in}}%
\pgfpathlineto{\pgfqpoint{1.385664in}{0.899790in}}%
\pgfpathlineto{\pgfqpoint{1.386397in}{1.890455in}}%
\pgfpathlineto{\pgfqpoint{1.387088in}{0.958696in}}%
\pgfpathlineto{\pgfqpoint{1.388174in}{1.927852in}}%
\pgfpathlineto{\pgfqpoint{1.388947in}{0.919355in}}%
\pgfpathlineto{\pgfqpoint{1.389993in}{1.992877in}}%
\pgfpathlineto{\pgfqpoint{1.390492in}{0.896884in}}%
\pgfpathlineto{\pgfqpoint{1.391345in}{1.925244in}}%
\pgfpathlineto{\pgfqpoint{1.392112in}{0.893695in}}%
\pgfpathlineto{\pgfqpoint{1.393158in}{1.927275in}}%
\pgfpathlineto{\pgfqpoint{1.393653in}{0.966824in}}%
\pgfpathlineto{\pgfqpoint{1.394502in}{1.821324in}}%
\pgfpathlineto{\pgfqpoint{1.395452in}{0.875706in}}%
\pgfpathlineto{\pgfqpoint{1.396502in}{1.967454in}}%
\pgfpathlineto{\pgfqpoint{1.396952in}{0.940950in}}%
\pgfpathlineto{\pgfqpoint{1.397932in}{1.848095in}}%
\pgfpathlineto{\pgfqpoint{1.398585in}{0.908595in}}%
\pgfpathlineto{\pgfqpoint{1.399480in}{1.901109in}}%
\pgfpathlineto{\pgfqpoint{1.400382in}{0.904024in}}%
\pgfpathlineto{\pgfqpoint{1.401060in}{1.863517in}}%
\pgfpathlineto{\pgfqpoint{1.401916in}{0.960186in}}%
\pgfpathlineto{\pgfqpoint{1.402833in}{1.942660in}}%
\pgfpathlineto{\pgfqpoint{1.403542in}{0.976794in}}%
\pgfpathlineto{\pgfqpoint{1.404441in}{1.904739in}}%
\pgfpathlineto{\pgfqpoint{1.405196in}{0.939625in}}%
\pgfpathlineto{\pgfqpoint{1.406578in}{1.953124in}}%
\pgfpathlineto{\pgfqpoint{1.406846in}{0.915301in}}%
\pgfpathlineto{\pgfqpoint{1.407772in}{1.839493in}}%
\pgfpathlineto{\pgfqpoint{1.408720in}{0.954392in}}%
\pgfpathlineto{\pgfqpoint{1.409380in}{1.841776in}}%
\pgfpathlineto{\pgfqpoint{1.410139in}{0.938143in}}%
\pgfpathlineto{\pgfqpoint{1.411097in}{1.856704in}}%
\pgfpathlineto{\pgfqpoint{1.411789in}{0.894626in}}%
\pgfpathlineto{\pgfqpoint{1.412619in}{1.862757in}}%
\pgfpathlineto{\pgfqpoint{1.413570in}{0.828045in}}%
\pgfpathlineto{\pgfqpoint{1.414550in}{1.916458in}}%
\pgfpathlineto{\pgfqpoint{1.415091in}{0.992502in}}%
\pgfpathlineto{\pgfqpoint{1.415980in}{1.820921in}}%
\pgfpathlineto{\pgfqpoint{1.416732in}{0.934650in}}%
\pgfpathlineto{\pgfqpoint{1.417542in}{1.946686in}}%
\pgfpathlineto{\pgfqpoint{1.418537in}{0.907682in}}%
\pgfpathlineto{\pgfqpoint{1.419162in}{1.830596in}}%
\pgfpathlineto{\pgfqpoint{1.419980in}{0.969624in}}%
\pgfpathlineto{\pgfqpoint{1.420910in}{1.865543in}}%
\pgfpathlineto{\pgfqpoint{1.421680in}{0.915863in}}%
\pgfpathlineto{\pgfqpoint{1.422455in}{1.847338in}}%
\pgfpathlineto{\pgfqpoint{1.423273in}{0.954725in}}%
\pgfpathlineto{\pgfqpoint{1.424210in}{1.874398in}}%
\pgfpathlineto{\pgfqpoint{1.425157in}{0.853159in}}%
\pgfpathlineto{\pgfqpoint{1.425826in}{1.867218in}}%
\pgfpathlineto{\pgfqpoint{1.426562in}{0.972543in}}%
\pgfpathlineto{\pgfqpoint{1.427392in}{1.815258in}}%
\pgfpathlineto{\pgfqpoint{1.428357in}{0.933476in}}%
\pgfpathlineto{\pgfqpoint{1.429301in}{1.933303in}}%
\pgfpathlineto{\pgfqpoint{1.429952in}{0.974864in}}%
\pgfpathlineto{\pgfqpoint{1.430772in}{1.951939in}}%
\pgfpathlineto{\pgfqpoint{1.431639in}{0.891010in}}%
\pgfpathlineto{\pgfqpoint{1.432376in}{1.994416in}}%
\pgfpathlineto{\pgfqpoint{1.433247in}{0.829500in}}%
\pgfpathlineto{\pgfqpoint{1.434019in}{1.786701in}}%
\pgfpathlineto{\pgfqpoint{1.434826in}{0.971679in}}%
\pgfpathlineto{\pgfqpoint{1.435735in}{1.861455in}}%
\pgfpathlineto{\pgfqpoint{1.436476in}{1.003507in}}%
\pgfpathlineto{\pgfqpoint{1.437423in}{1.856431in}}%
\pgfpathlineto{\pgfqpoint{1.438414in}{0.849521in}}%
\pgfpathlineto{\pgfqpoint{1.438929in}{1.843823in}}%
\pgfpathlineto{\pgfqpoint{1.439729in}{0.855483in}}%
\pgfpathlineto{\pgfqpoint{1.440632in}{1.828003in}}%
\pgfpathlineto{\pgfqpoint{1.441625in}{0.946666in}}%
\pgfpathlineto{\pgfqpoint{1.442200in}{1.855318in}}%
\pgfpathlineto{\pgfqpoint{1.443104in}{0.974994in}}%
\pgfpathlineto{\pgfqpoint{1.444370in}{1.898341in}}%
\pgfpathlineto{\pgfqpoint{1.444763in}{0.926411in}}%
\pgfpathlineto{\pgfqpoint{1.445734in}{1.900878in}}%
\pgfpathlineto{\pgfqpoint{1.446323in}{0.979793in}}%
\pgfpathlineto{\pgfqpoint{1.447143in}{1.815595in}}%
\pgfpathlineto{\pgfqpoint{1.447995in}{0.933485in}}%
\pgfpathlineto{\pgfqpoint{1.448820in}{1.893570in}}%
\pgfpathlineto{\pgfqpoint{1.450038in}{0.802785in}}%
\pgfpathlineto{\pgfqpoint{1.450521in}{1.864432in}}%
\pgfpathlineto{\pgfqpoint{1.451253in}{0.948271in}}%
\pgfpathlineto{\pgfqpoint{1.452346in}{1.949242in}}%
\pgfpathlineto{\pgfqpoint{1.452931in}{0.996316in}}%
\pgfpathlineto{\pgfqpoint{1.454004in}{1.899456in}}%
\pgfpathlineto{\pgfqpoint{1.454555in}{1.000307in}}%
\pgfpathlineto{\pgfqpoint{1.455523in}{1.827979in}}%
\pgfpathlineto{\pgfqpoint{1.456190in}{0.998134in}}%
\pgfpathlineto{\pgfqpoint{1.457012in}{1.864162in}}%
\pgfpathlineto{\pgfqpoint{1.457851in}{0.942774in}}%
\pgfpathlineto{\pgfqpoint{1.458962in}{1.880421in}}%
\pgfpathlineto{\pgfqpoint{1.459630in}{0.828704in}}%
\pgfpathlineto{\pgfqpoint{1.460807in}{2.008836in}}%
\pgfpathlineto{\pgfqpoint{1.461587in}{0.850789in}}%
\pgfpathlineto{\pgfqpoint{1.461945in}{1.796746in}}%
\pgfpathlineto{\pgfqpoint{1.462844in}{0.944200in}}%
\pgfpathlineto{\pgfqpoint{1.463625in}{1.832567in}}%
\pgfpathlineto{\pgfqpoint{1.464467in}{0.992934in}}%
\pgfpathlineto{\pgfqpoint{1.465534in}{1.913884in}}%
\pgfpathlineto{\pgfqpoint{1.466200in}{0.969686in}}%
\pgfpathlineto{\pgfqpoint{1.466910in}{1.804456in}}%
\pgfpathlineto{\pgfqpoint{1.467777in}{0.886812in}}%
\pgfpathlineto{\pgfqpoint{1.468647in}{1.916629in}}%
\pgfpathlineto{\pgfqpoint{1.469454in}{0.960928in}}%
\pgfpathlineto{\pgfqpoint{1.470354in}{1.954652in}}%
\pgfpathlineto{\pgfqpoint{1.471047in}{0.940725in}}%
\pgfpathlineto{\pgfqpoint{1.471904in}{1.890846in}}%
\pgfpathlineto{\pgfqpoint{1.472756in}{0.940675in}}%
\pgfpathlineto{\pgfqpoint{1.473552in}{1.812860in}}%
\pgfpathlineto{\pgfqpoint{1.474449in}{0.931847in}}%
\pgfpathlineto{\pgfqpoint{1.475161in}{1.855370in}}%
\pgfpathlineto{\pgfqpoint{1.476121in}{0.916726in}}%
\pgfpathlineto{\pgfqpoint{1.476793in}{1.818853in}}%
\pgfpathlineto{\pgfqpoint{1.477695in}{0.924468in}}%
\pgfpathlineto{\pgfqpoint{1.478465in}{1.868419in}}%
\pgfpathlineto{\pgfqpoint{1.479274in}{0.945970in}}%
\pgfpathlineto{\pgfqpoint{1.480082in}{1.927730in}}%
\pgfpathlineto{\pgfqpoint{1.480896in}{0.867930in}}%
\pgfpathlineto{\pgfqpoint{1.481754in}{1.822276in}}%
\pgfpathlineto{\pgfqpoint{1.482597in}{0.999445in}}%
\pgfpathlineto{\pgfqpoint{1.483429in}{1.868242in}}%
\pgfpathlineto{\pgfqpoint{1.484201in}{0.907958in}}%
\pgfpathlineto{\pgfqpoint{1.485037in}{1.840717in}}%
\pgfpathlineto{\pgfqpoint{1.486193in}{0.846677in}}%
\pgfpathlineto{\pgfqpoint{1.486668in}{1.800399in}}%
\pgfpathlineto{\pgfqpoint{1.487552in}{0.947356in}}%
\pgfpathlineto{\pgfqpoint{1.488318in}{1.832381in}}%
\pgfpathlineto{\pgfqpoint{1.489160in}{0.939851in}}%
\pgfpathlineto{\pgfqpoint{1.489976in}{1.800479in}}%
\pgfpathlineto{\pgfqpoint{1.490766in}{1.006025in}}%
\pgfpathlineto{\pgfqpoint{1.491792in}{1.868080in}}%
\pgfpathlineto{\pgfqpoint{1.492928in}{0.853971in}}%
\pgfpathlineto{\pgfqpoint{1.493292in}{1.907211in}}%
\pgfpathlineto{\pgfqpoint{1.494040in}{0.952070in}}%
\pgfpathlineto{\pgfqpoint{1.495166in}{1.939562in}}%
\pgfpathlineto{\pgfqpoint{1.495732in}{0.903285in}}%
\pgfpathlineto{\pgfqpoint{1.496512in}{1.820000in}}%
\pgfpathlineto{\pgfqpoint{1.497472in}{0.990181in}}%
\pgfpathlineto{\pgfqpoint{1.498156in}{1.878063in}}%
\pgfpathlineto{\pgfqpoint{1.498980in}{0.978772in}}%
\pgfpathlineto{\pgfqpoint{1.500413in}{1.948143in}}%
\pgfpathlineto{\pgfqpoint{1.500634in}{1.018767in}}%
\pgfpathlineto{\pgfqpoint{1.501451in}{1.880171in}}%
\pgfpathlineto{\pgfqpoint{1.502308in}{0.931968in}}%
\pgfpathlineto{\pgfqpoint{1.503233in}{1.877247in}}%
\pgfpathlineto{\pgfqpoint{1.504069in}{0.881370in}}%
\pgfpathlineto{\pgfqpoint{1.504757in}{1.819496in}}%
\pgfpathlineto{\pgfqpoint{1.505663in}{0.982740in}}%
\pgfpathlineto{\pgfqpoint{1.506630in}{1.880271in}}%
\pgfpathlineto{\pgfqpoint{1.507208in}{0.951671in}}%
\pgfpathlineto{\pgfqpoint{1.508131in}{1.936267in}}%
\pgfpathlineto{\pgfqpoint{1.508900in}{0.984017in}}%
\pgfpathlineto{\pgfqpoint{1.509847in}{1.889281in}}%
\pgfpathlineto{\pgfqpoint{1.510711in}{0.882812in}}%
\pgfpathlineto{\pgfqpoint{1.511329in}{1.854556in}}%
\pgfpathlineto{\pgfqpoint{1.512365in}{0.970675in}}%
\pgfpathlineto{\pgfqpoint{1.513137in}{1.872292in}}%
\pgfpathlineto{\pgfqpoint{1.514027in}{0.915750in}}%
\pgfpathlineto{\pgfqpoint{1.514624in}{1.872980in}}%
\pgfpathlineto{\pgfqpoint{1.515602in}{0.971979in}}%
\pgfpathlineto{\pgfqpoint{1.516260in}{1.947795in}}%
\pgfpathlineto{\pgfqpoint{1.517181in}{0.985762in}}%
\pgfpathlineto{\pgfqpoint{1.518118in}{1.899192in}}%
\pgfpathlineto{\pgfqpoint{1.519051in}{0.838944in}}%
\pgfpathlineto{\pgfqpoint{1.519813in}{1.901793in}}%
\pgfpathlineto{\pgfqpoint{1.520423in}{0.943590in}}%
\pgfpathlineto{\pgfqpoint{1.521274in}{1.844270in}}%
\pgfpathlineto{\pgfqpoint{1.522032in}{0.999250in}}%
\pgfpathlineto{\pgfqpoint{1.523037in}{1.977176in}}%
\pgfpathlineto{\pgfqpoint{1.523743in}{1.011568in}}%
\pgfpathlineto{\pgfqpoint{1.524628in}{1.915263in}}%
\pgfpathlineto{\pgfqpoint{1.525359in}{0.989045in}}%
\pgfpathlineto{\pgfqpoint{1.526176in}{1.878452in}}%
\pgfpathlineto{\pgfqpoint{1.526959in}{0.988811in}}%
\pgfpathlineto{\pgfqpoint{1.528153in}{1.891643in}}%
\pgfpathlineto{\pgfqpoint{1.528644in}{0.959898in}}%
\pgfpathlineto{\pgfqpoint{1.529561in}{1.846630in}}%
\pgfpathlineto{\pgfqpoint{1.530353in}{0.952573in}}%
\pgfpathlineto{\pgfqpoint{1.531167in}{1.845349in}}%
\pgfpathlineto{\pgfqpoint{1.532146in}{0.921349in}}%
\pgfpathlineto{\pgfqpoint{1.532774in}{1.806732in}}%
\pgfpathlineto{\pgfqpoint{1.533604in}{0.993542in}}%
\pgfpathlineto{\pgfqpoint{1.534560in}{1.916789in}}%
\pgfpathlineto{\pgfqpoint{1.535263in}{0.968733in}}%
\pgfpathlineto{\pgfqpoint{1.536305in}{1.939826in}}%
\pgfpathlineto{\pgfqpoint{1.537361in}{0.861600in}}%
\pgfpathlineto{\pgfqpoint{1.537846in}{1.980837in}}%
\pgfpathlineto{\pgfqpoint{1.538505in}{0.971290in}}%
\pgfpathlineto{\pgfqpoint{1.539357in}{1.877834in}}%
\pgfpathlineto{\pgfqpoint{1.540155in}{0.994518in}}%
\pgfpathlineto{\pgfqpoint{1.541190in}{1.929016in}}%
\pgfpathlineto{\pgfqpoint{1.542005in}{0.929513in}}%
\pgfpathlineto{\pgfqpoint{1.542825in}{1.907040in}}%
\pgfpathlineto{\pgfqpoint{1.543514in}{0.955515in}}%
\pgfpathlineto{\pgfqpoint{1.544336in}{1.816249in}}%
\pgfpathlineto{\pgfqpoint{1.545242in}{0.978272in}}%
\pgfpathlineto{\pgfqpoint{1.546163in}{1.962021in}}%
\pgfpathlineto{\pgfqpoint{1.546732in}{0.985031in}}%
\pgfpathlineto{\pgfqpoint{1.547831in}{1.977063in}}%
\pgfpathlineto{\pgfqpoint{1.548411in}{0.925395in}}%
\pgfpathlineto{\pgfqpoint{1.549276in}{1.839100in}}%
\pgfpathlineto{\pgfqpoint{1.550152in}{0.980066in}}%
\pgfpathlineto{\pgfqpoint{1.550850in}{1.837160in}}%
\pgfpathlineto{\pgfqpoint{1.551644in}{0.972118in}}%
\pgfpathlineto{\pgfqpoint{1.552524in}{1.846642in}}%
\pgfpathlineto{\pgfqpoint{1.553332in}{0.999184in}}%
\pgfpathlineto{\pgfqpoint{1.554120in}{1.828847in}}%
\pgfpathlineto{\pgfqpoint{1.554988in}{0.960115in}}%
\pgfpathlineto{\pgfqpoint{1.556076in}{1.950887in}}%
\pgfpathlineto{\pgfqpoint{1.556615in}{0.991002in}}%
\pgfpathlineto{\pgfqpoint{1.557424in}{1.837324in}}%
\pgfpathlineto{\pgfqpoint{1.558302in}{0.982376in}}%
\pgfpathlineto{\pgfqpoint{1.559297in}{1.926931in}}%
\pgfpathlineto{\pgfqpoint{1.559905in}{0.996618in}}%
\pgfpathlineto{\pgfqpoint{1.560699in}{1.864367in}}%
\pgfpathlineto{\pgfqpoint{1.561543in}{0.936185in}}%
\pgfpathlineto{\pgfqpoint{1.562538in}{1.835641in}}%
\pgfpathlineto{\pgfqpoint{1.563324in}{0.913436in}}%
\pgfpathlineto{\pgfqpoint{1.564437in}{1.984464in}}%
\pgfpathlineto{\pgfqpoint{1.564842in}{0.925021in}}%
\pgfpathlineto{\pgfqpoint{1.565714in}{1.866008in}}%
\pgfpathlineto{\pgfqpoint{1.566824in}{0.841756in}}%
\pgfpathlineto{\pgfqpoint{1.567290in}{1.845003in}}%
\pgfpathlineto{\pgfqpoint{1.568245in}{0.951484in}}%
\pgfpathlineto{\pgfqpoint{1.569177in}{1.878670in}}%
\pgfpathlineto{\pgfqpoint{1.569754in}{1.000162in}}%
\pgfpathlineto{\pgfqpoint{1.570679in}{1.884829in}}%
\pgfpathlineto{\pgfqpoint{1.571494in}{0.982306in}}%
\pgfpathlineto{\pgfqpoint{1.572258in}{1.839836in}}%
\pgfpathlineto{\pgfqpoint{1.573062in}{0.890982in}}%
\pgfpathlineto{\pgfqpoint{1.574008in}{1.915187in}}%
\pgfpathlineto{\pgfqpoint{1.575061in}{0.872037in}}%
\pgfpathlineto{\pgfqpoint{1.575766in}{1.878171in}}%
\pgfpathlineto{\pgfqpoint{1.576421in}{0.965219in}}%
\pgfpathlineto{\pgfqpoint{1.577199in}{1.923542in}}%
\pgfpathlineto{\pgfqpoint{1.578279in}{0.903250in}}%
\pgfpathlineto{\pgfqpoint{1.579004in}{1.853946in}}%
\pgfpathlineto{\pgfqpoint{1.579703in}{0.957179in}}%
\pgfpathlineto{\pgfqpoint{1.580680in}{1.867882in}}%
\pgfpathlineto{\pgfqpoint{1.581595in}{0.885236in}}%
\pgfpathlineto{\pgfqpoint{1.582100in}{1.840900in}}%
\pgfpathlineto{\pgfqpoint{1.582979in}{0.926315in}}%
\pgfpathlineto{\pgfqpoint{1.583873in}{1.851222in}}%
\pgfpathlineto{\pgfqpoint{1.584664in}{0.935832in}}%
\pgfpathlineto{\pgfqpoint{1.585508in}{1.912248in}}%
\pgfpathlineto{\pgfqpoint{1.586235in}{0.992727in}}%
\pgfpathlineto{\pgfqpoint{1.587139in}{1.851321in}}%
\pgfpathlineto{\pgfqpoint{1.588158in}{0.796722in}}%
\pgfpathlineto{\pgfqpoint{1.588842in}{1.902402in}}%
\pgfpathlineto{\pgfqpoint{1.589508in}{0.968068in}}%
\pgfpathlineto{\pgfqpoint{1.590424in}{1.889549in}}%
\pgfpathlineto{\pgfqpoint{1.591166in}{0.955797in}}%
\pgfpathlineto{\pgfqpoint{1.591984in}{1.889203in}}%
\pgfpathlineto{\pgfqpoint{1.592821in}{0.993330in}}%
\pgfpathlineto{\pgfqpoint{1.593781in}{1.836611in}}%
\pgfpathlineto{\pgfqpoint{1.594658in}{0.939186in}}%
\pgfpathlineto{\pgfqpoint{1.595256in}{1.845461in}}%
\pgfpathlineto{\pgfqpoint{1.596379in}{0.891931in}}%
\pgfpathlineto{\pgfqpoint{1.597080in}{1.891885in}}%
\pgfpathlineto{\pgfqpoint{1.597874in}{0.982131in}}%
\pgfpathlineto{\pgfqpoint{1.598657in}{1.818918in}}%
\pgfpathlineto{\pgfqpoint{1.599400in}{0.925601in}}%
\pgfpathlineto{\pgfqpoint{1.600268in}{1.934672in}}%
\pgfpathlineto{\pgfqpoint{1.601047in}{0.996086in}}%
\pgfpathlineto{\pgfqpoint{1.602384in}{1.968656in}}%
\pgfpathlineto{\pgfqpoint{1.602747in}{0.959816in}}%
\pgfpathlineto{\pgfqpoint{1.603582in}{1.880499in}}%
\pgfpathlineto{\pgfqpoint{1.604319in}{0.887347in}}%
\pgfpathlineto{\pgfqpoint{1.605167in}{1.857378in}}%
\pgfpathlineto{\pgfqpoint{1.606036in}{0.990775in}}%
\pgfpathlineto{\pgfqpoint{1.606942in}{1.933781in}}%
\pgfpathlineto{\pgfqpoint{1.607947in}{0.878551in}}%
\pgfpathlineto{\pgfqpoint{1.608481in}{1.808380in}}%
\pgfpathlineto{\pgfqpoint{1.609758in}{0.882192in}}%
\pgfpathlineto{\pgfqpoint{1.610270in}{1.945182in}}%
\pgfpathlineto{\pgfqpoint{1.611101in}{0.826812in}}%
\pgfpathlineto{\pgfqpoint{1.611806in}{1.822404in}}%
\pgfpathlineto{\pgfqpoint{1.612561in}{1.019269in}}%
\pgfpathlineto{\pgfqpoint{1.613537in}{1.853856in}}%
\pgfpathlineto{\pgfqpoint{1.614186in}{0.977069in}}%
\pgfpathlineto{\pgfqpoint{1.615073in}{1.828613in}}%
\pgfpathlineto{\pgfqpoint{1.615907in}{0.875409in}}%
\pgfpathlineto{\pgfqpoint{1.616673in}{1.875010in}}%
\pgfpathlineto{\pgfqpoint{1.617514in}{0.985773in}}%
\pgfpathlineto{\pgfqpoint{1.618599in}{1.942546in}}%
\pgfpathlineto{\pgfqpoint{1.619248in}{0.973979in}}%
\pgfpathlineto{\pgfqpoint{1.620156in}{1.875530in}}%
\pgfpathlineto{\pgfqpoint{1.620885in}{0.908863in}}%
\pgfpathlineto{\pgfqpoint{1.621623in}{1.894764in}}%
\pgfpathlineto{\pgfqpoint{1.622491in}{0.934399in}}%
\pgfpathlineto{\pgfqpoint{1.623729in}{1.934595in}}%
\pgfpathlineto{\pgfqpoint{1.624165in}{0.923037in}}%
\pgfpathlineto{\pgfqpoint{1.625122in}{1.944508in}}%
\pgfpathlineto{\pgfqpoint{1.625802in}{0.991216in}}%
\pgfpathlineto{\pgfqpoint{1.626610in}{1.885670in}}%
\pgfpathlineto{\pgfqpoint{1.627472in}{0.959795in}}%
\pgfpathlineto{\pgfqpoint{1.628283in}{1.938630in}}%
\pgfpathlineto{\pgfqpoint{1.629082in}{0.959549in}}%
\pgfpathlineto{\pgfqpoint{1.629922in}{1.847037in}}%
\pgfpathlineto{\pgfqpoint{1.630641in}{0.964192in}}%
\pgfpathlineto{\pgfqpoint{1.631467in}{1.806613in}}%
\pgfpathlineto{\pgfqpoint{1.632670in}{0.883566in}}%
\pgfpathlineto{\pgfqpoint{1.633151in}{1.894788in}}%
\pgfpathlineto{\pgfqpoint{1.633933in}{1.005182in}}%
\pgfpathlineto{\pgfqpoint{1.634916in}{1.890244in}}%
\pgfpathlineto{\pgfqpoint{1.635659in}{0.971993in}}%
\pgfpathlineto{\pgfqpoint{1.636402in}{1.800784in}}%
\pgfpathlineto{\pgfqpoint{1.637374in}{0.965314in}}%
\pgfpathlineto{\pgfqpoint{1.638614in}{2.010762in}}%
\pgfpathlineto{\pgfqpoint{1.638997in}{0.960054in}}%
\pgfpathlineto{\pgfqpoint{1.639766in}{1.869390in}}%
\pgfpathlineto{\pgfqpoint{1.640596in}{0.939040in}}%
\pgfpathlineto{\pgfqpoint{1.641433in}{1.874396in}}%
\pgfpathlineto{\pgfqpoint{1.642234in}{0.924181in}}%
\pgfpathlineto{\pgfqpoint{1.643234in}{1.909221in}}%
\pgfpathlineto{\pgfqpoint{1.644106in}{0.905825in}}%
\pgfpathlineto{\pgfqpoint{1.644674in}{1.837137in}}%
\pgfpathlineto{\pgfqpoint{1.645571in}{0.877088in}}%
\pgfpathlineto{\pgfqpoint{1.646312in}{1.784527in}}%
\pgfpathlineto{\pgfqpoint{1.647248in}{0.961783in}}%
\pgfpathlineto{\pgfqpoint{1.648017in}{1.821900in}}%
\pgfpathlineto{\pgfqpoint{1.649121in}{0.917725in}}%
\pgfpathlineto{\pgfqpoint{1.649734in}{1.886641in}}%
\pgfpathlineto{\pgfqpoint{1.650526in}{0.856964in}}%
\pgfpathlineto{\pgfqpoint{1.651253in}{1.887885in}}%
\pgfpathlineto{\pgfqpoint{1.652066in}{0.981513in}}%
\pgfpathlineto{\pgfqpoint{1.653303in}{1.948749in}}%
\pgfpathlineto{\pgfqpoint{1.653901in}{0.859435in}}%
\pgfpathlineto{\pgfqpoint{1.654517in}{1.825800in}}%
\pgfpathlineto{\pgfqpoint{1.655718in}{0.921417in}}%
\pgfpathlineto{\pgfqpoint{1.656301in}{1.929015in}}%
\pgfpathlineto{\pgfqpoint{1.657343in}{0.881773in}}%
\pgfpathlineto{\pgfqpoint{1.657803in}{1.781030in}}%
\pgfpathlineto{\pgfqpoint{1.658999in}{0.830793in}}%
\pgfpathlineto{\pgfqpoint{1.659517in}{1.831252in}}%
\pgfpathlineto{\pgfqpoint{1.660305in}{0.944198in}}%
\pgfpathlineto{\pgfqpoint{1.661402in}{1.951344in}}%
\pgfpathlineto{\pgfqpoint{1.661917in}{0.936882in}}%
\pgfpathlineto{\pgfqpoint{1.663277in}{2.021110in}}%
\pgfpathlineto{\pgfqpoint{1.663581in}{0.989050in}}%
\pgfpathlineto{\pgfqpoint{1.664407in}{1.851902in}}%
\pgfpathlineto{\pgfqpoint{1.665740in}{0.856285in}}%
\pgfpathlineto{\pgfqpoint{1.666045in}{1.918272in}}%
\pgfpathlineto{\pgfqpoint{1.666853in}{0.935022in}}%
\pgfpathlineto{\pgfqpoint{1.667763in}{1.841497in}}%
\pgfpathlineto{\pgfqpoint{1.668503in}{0.966925in}}%
\pgfpathlineto{\pgfqpoint{1.669345in}{1.909837in}}%
\pgfpathlineto{\pgfqpoint{1.670397in}{0.963082in}}%
\pgfpathlineto{\pgfqpoint{1.670963in}{1.899573in}}%
\pgfpathlineto{\pgfqpoint{1.672023in}{0.957483in}}%
\pgfpathlineto{\pgfqpoint{1.672938in}{1.887414in}}%
\pgfpathlineto{\pgfqpoint{1.673439in}{1.013732in}}%
\pgfpathlineto{\pgfqpoint{1.674281in}{1.841109in}}%
\pgfpathlineto{\pgfqpoint{1.675188in}{0.902313in}}%
\pgfpathlineto{\pgfqpoint{1.675951in}{1.855567in}}%
\pgfpathlineto{\pgfqpoint{1.676874in}{0.944152in}}%
\pgfpathlineto{\pgfqpoint{1.677767in}{1.900222in}}%
\pgfpathlineto{\pgfqpoint{1.678528in}{0.866151in}}%
\pgfpathlineto{\pgfqpoint{1.679303in}{1.849874in}}%
\pgfpathlineto{\pgfqpoint{1.680053in}{0.920077in}}%
\pgfpathlineto{\pgfqpoint{1.680921in}{1.940776in}}%
\pgfpathlineto{\pgfqpoint{1.681686in}{0.988902in}}%
\pgfpathlineto{\pgfqpoint{1.682550in}{1.809741in}}%
\pgfpathlineto{\pgfqpoint{1.683317in}{0.960322in}}%
\pgfpathlineto{\pgfqpoint{1.684138in}{1.855485in}}%
\pgfpathlineto{\pgfqpoint{1.685639in}{0.828523in}}%
\pgfpathlineto{\pgfqpoint{1.685792in}{1.824441in}}%
\pgfpathlineto{\pgfqpoint{1.687272in}{0.863344in}}%
\pgfpathlineto{\pgfqpoint{1.687498in}{1.847011in}}%
\pgfpathlineto{\pgfqpoint{1.688260in}{0.985761in}}%
\pgfpathlineto{\pgfqpoint{1.689304in}{1.863816in}}%
\pgfpathlineto{\pgfqpoint{1.690082in}{0.867663in}}%
\pgfpathlineto{\pgfqpoint{1.690755in}{1.866531in}}%
\pgfpathlineto{\pgfqpoint{1.691555in}{0.992824in}}%
\pgfpathlineto{\pgfqpoint{1.692401in}{1.989777in}}%
\pgfpathlineto{\pgfqpoint{1.693306in}{0.895187in}}%
\pgfpathlineto{\pgfqpoint{1.694246in}{1.871768in}}%
\pgfpathlineto{\pgfqpoint{1.694929in}{0.926060in}}%
\pgfpathlineto{\pgfqpoint{1.695775in}{1.923790in}}%
\pgfpathlineto{\pgfqpoint{1.696542in}{0.962071in}}%
\pgfpathlineto{\pgfqpoint{1.697367in}{1.971627in}}%
\pgfpathlineto{\pgfqpoint{1.698382in}{0.896694in}}%
\pgfpathlineto{\pgfqpoint{1.698959in}{1.837402in}}%
\pgfpathlineto{\pgfqpoint{1.699881in}{0.979377in}}%
\pgfpathlineto{\pgfqpoint{1.700612in}{1.906080in}}%
\pgfpathlineto{\pgfqpoint{1.701428in}{0.861688in}}%
\pgfpathlineto{\pgfqpoint{1.702291in}{1.901618in}}%
\pgfpathlineto{\pgfqpoint{1.703135in}{0.910339in}}%
\pgfpathlineto{\pgfqpoint{1.703966in}{1.915074in}}%
\pgfpathlineto{\pgfqpoint{1.704861in}{0.966625in}}%
\pgfpathlineto{\pgfqpoint{1.706297in}{2.016555in}}%
\pgfpathlineto{\pgfqpoint{1.706392in}{1.043345in}}%
\pgfpathlineto{\pgfqpoint{1.707184in}{1.831446in}}%
\pgfpathlineto{\pgfqpoint{1.708117in}{0.966883in}}%
\pgfpathlineto{\pgfqpoint{1.709088in}{1.891497in}}%
\pgfpathlineto{\pgfqpoint{1.709766in}{0.959719in}}%
\pgfpathlineto{\pgfqpoint{1.710721in}{1.885989in}}%
\pgfpathlineto{\pgfqpoint{1.711579in}{0.872460in}}%
\pgfpathlineto{\pgfqpoint{1.712336in}{1.899003in}}%
\pgfpathlineto{\pgfqpoint{1.713108in}{0.964057in}}%
\pgfpathlineto{\pgfqpoint{1.713996in}{1.938405in}}%
\pgfpathlineto{\pgfqpoint{1.714612in}{0.985316in}}%
\pgfpathlineto{\pgfqpoint{1.715416in}{1.849845in}}%
\pgfpathlineto{\pgfqpoint{1.716539in}{0.875168in}}%
\pgfpathlineto{\pgfqpoint{1.717048in}{1.891574in}}%
\pgfpathlineto{\pgfqpoint{1.717949in}{0.985876in}}%
\pgfpathlineto{\pgfqpoint{1.718716in}{1.856454in}}%
\pgfpathlineto{\pgfqpoint{1.719602in}{0.933449in}}%
\pgfpathlineto{\pgfqpoint{1.720719in}{1.955600in}}%
\pgfpathlineto{\pgfqpoint{1.721243in}{0.948660in}}%
\pgfpathlineto{\pgfqpoint{1.722028in}{1.868094in}}%
\pgfpathlineto{\pgfqpoint{1.723046in}{0.769476in}}%
\pgfpathlineto{\pgfqpoint{1.723766in}{1.856219in}}%
\pgfpathlineto{\pgfqpoint{1.724635in}{0.897689in}}%
\pgfpathlineto{\pgfqpoint{1.725650in}{1.992217in}}%
\pgfpathlineto{\pgfqpoint{1.726257in}{0.941695in}}%
\pgfpathlineto{\pgfqpoint{1.727158in}{1.884910in}}%
\pgfpathlineto{\pgfqpoint{1.727926in}{0.971629in}}%
\pgfpathlineto{\pgfqpoint{1.728621in}{1.861158in}}%
\pgfpathlineto{\pgfqpoint{1.729497in}{0.965160in}}%
\pgfpathlineto{\pgfqpoint{1.730515in}{1.902926in}}%
\pgfpathlineto{\pgfqpoint{1.731042in}{1.013350in}}%
\pgfpathlineto{\pgfqpoint{1.731947in}{1.875073in}}%
\pgfpathlineto{\pgfqpoint{1.732715in}{1.000196in}}%
\pgfpathlineto{\pgfqpoint{1.733577in}{1.880516in}}%
\pgfpathlineto{\pgfqpoint{1.734684in}{0.945946in}}%
\pgfpathlineto{\pgfqpoint{1.735168in}{1.816859in}}%
\pgfpathlineto{\pgfqpoint{1.736017in}{0.939694in}}%
\pgfpathlineto{\pgfqpoint{1.736849in}{1.861117in}}%
\pgfpathlineto{\pgfqpoint{1.738099in}{0.889686in}}%
\pgfpathlineto{\pgfqpoint{1.738864in}{2.014690in}}%
\pgfpathlineto{\pgfqpoint{1.739432in}{0.874568in}}%
\pgfpathlineto{\pgfqpoint{1.740342in}{1.895759in}}%
\pgfpathlineto{\pgfqpoint{1.741183in}{0.958260in}}%
\pgfpathlineto{\pgfqpoint{1.741771in}{1.821314in}}%
\pgfpathlineto{\pgfqpoint{1.742846in}{0.836092in}}%
\pgfpathlineto{\pgfqpoint{1.743380in}{1.824769in}}%
\pgfpathlineto{\pgfqpoint{1.744284in}{0.993327in}}%
\pgfpathlineto{\pgfqpoint{1.745132in}{1.863478in}}%
\pgfpathlineto{\pgfqpoint{1.745947in}{0.945369in}}%
\pgfpathlineto{\pgfqpoint{1.746734in}{1.923211in}}%
\pgfpathlineto{\pgfqpoint{1.747575in}{0.881652in}}%
\pgfpathlineto{\pgfqpoint{1.748616in}{1.916153in}}%
\pgfpathlineto{\pgfqpoint{1.749245in}{0.983460in}}%
\pgfpathlineto{\pgfqpoint{1.750080in}{1.889241in}}%
\pgfpathlineto{\pgfqpoint{1.751063in}{0.922114in}}%
\pgfpathlineto{\pgfqpoint{1.751731in}{1.895246in}}%
\pgfpathlineto{\pgfqpoint{1.752504in}{0.989678in}}%
\pgfpathlineto{\pgfqpoint{1.753364in}{1.906170in}}%
\pgfpathlineto{\pgfqpoint{1.754111in}{0.959489in}}%
\pgfpathlineto{\pgfqpoint{1.755186in}{1.903382in}}%
\pgfpathlineto{\pgfqpoint{1.755787in}{0.988943in}}%
\pgfpathlineto{\pgfqpoint{1.756645in}{1.882508in}}%
\pgfpathlineto{\pgfqpoint{1.757408in}{0.938434in}}%
\pgfpathlineto{\pgfqpoint{1.758251in}{1.868732in}}%
\pgfpathlineto{\pgfqpoint{1.759052in}{0.936868in}}%
\pgfpathlineto{\pgfqpoint{1.759857in}{1.925166in}}%
\pgfpathlineto{\pgfqpoint{1.760827in}{0.891945in}}%
\pgfpathlineto{\pgfqpoint{1.761526in}{1.848174in}}%
\pgfpathlineto{\pgfqpoint{1.762807in}{0.862881in}}%
\pgfpathlineto{\pgfqpoint{1.763187in}{1.960879in}}%
\pgfpathlineto{\pgfqpoint{1.764082in}{0.985963in}}%
\pgfpathlineto{\pgfqpoint{1.764857in}{1.831440in}}%
\pgfpathlineto{\pgfqpoint{1.765676in}{0.965404in}}%
\pgfpathlineto{\pgfqpoint{1.766843in}{1.896121in}}%
\pgfpathlineto{\pgfqpoint{1.767455in}{0.933932in}}%
\pgfpathlineto{\pgfqpoint{1.768134in}{1.851845in}}%
\pgfpathlineto{\pgfqpoint{1.769370in}{0.887990in}}%
\pgfpathlineto{\pgfqpoint{1.769877in}{1.859196in}}%
\pgfpathlineto{\pgfqpoint{1.770553in}{0.934789in}}%
\pgfpathlineto{\pgfqpoint{1.771458in}{1.816239in}}%
\pgfpathlineto{\pgfqpoint{1.772184in}{0.966743in}}%
\pgfpathlineto{\pgfqpoint{1.773254in}{1.838169in}}%
\pgfpathlineto{\pgfqpoint{1.773906in}{1.002299in}}%
\pgfpathlineto{\pgfqpoint{1.774816in}{1.865697in}}%
\pgfpathlineto{\pgfqpoint{1.775889in}{0.876857in}}%
\pgfpathlineto{\pgfqpoint{1.776565in}{1.888424in}}%
\pgfpathlineto{\pgfqpoint{1.777177in}{1.003838in}}%
\pgfpathlineto{\pgfqpoint{1.778089in}{1.868147in}}%
\pgfpathlineto{\pgfqpoint{1.778776in}{0.949282in}}%
\pgfpathlineto{\pgfqpoint{1.779712in}{1.885126in}}%
\pgfpathlineto{\pgfqpoint{1.780506in}{0.971985in}}%
\pgfpathlineto{\pgfqpoint{1.781401in}{1.901425in}}%
\pgfpathlineto{\pgfqpoint{1.782220in}{0.956808in}}%
\pgfpathlineto{\pgfqpoint{1.783056in}{1.899741in}}%
\pgfpathlineto{\pgfqpoint{1.783937in}{0.957352in}}%
\pgfpathlineto{\pgfqpoint{1.784695in}{1.854673in}}%
\pgfpathlineto{\pgfqpoint{1.785709in}{0.843260in}}%
\pgfpathlineto{\pgfqpoint{1.786497in}{2.007952in}}%
\pgfpathlineto{\pgfqpoint{1.787175in}{0.901322in}}%
\pgfpathlineto{\pgfqpoint{1.787855in}{1.823070in}}%
\pgfpathlineto{\pgfqpoint{1.788761in}{0.865166in}}%
\pgfpathlineto{\pgfqpoint{1.789834in}{1.909883in}}%
\pgfpathlineto{\pgfqpoint{1.790286in}{0.952886in}}%
\pgfpathlineto{\pgfqpoint{1.791425in}{1.909084in}}%
\pgfpathlineto{\pgfqpoint{1.791959in}{0.964912in}}%
\pgfpathlineto{\pgfqpoint{1.792823in}{1.823381in}}%
\pgfpathlineto{\pgfqpoint{1.793614in}{0.998060in}}%
\pgfpathlineto{\pgfqpoint{1.794574in}{1.878477in}}%
\pgfpathlineto{\pgfqpoint{1.795238in}{0.999570in}}%
\pgfpathlineto{\pgfqpoint{1.796430in}{1.903984in}}%
\pgfpathlineto{\pgfqpoint{1.797374in}{0.815366in}}%
\pgfpathlineto{\pgfqpoint{1.797826in}{1.851522in}}%
\pgfpathlineto{\pgfqpoint{1.798644in}{0.872023in}}%
\pgfpathlineto{\pgfqpoint{1.799433in}{1.893042in}}%
\pgfpathlineto{\pgfqpoint{1.800222in}{0.966578in}}%
\pgfpathlineto{\pgfqpoint{1.800995in}{1.871771in}}%
\pgfpathlineto{\pgfqpoint{1.801835in}{0.948261in}}%
\pgfpathlineto{\pgfqpoint{1.802699in}{1.862610in}}%
\pgfpathlineto{\pgfqpoint{1.803483in}{0.925621in}}%
\pgfpathlineto{\pgfqpoint{1.804521in}{1.917848in}}%
\pgfpathlineto{\pgfqpoint{1.805242in}{0.973726in}}%
\pgfpathlineto{\pgfqpoint{1.806018in}{1.923728in}}%
\pgfpathlineto{\pgfqpoint{1.806784in}{0.994754in}}%
\pgfpathlineto{\pgfqpoint{1.807632in}{1.872168in}}%
\pgfpathlineto{\pgfqpoint{1.808451in}{0.952839in}}%
\pgfpathlineto{\pgfqpoint{1.809483in}{1.900742in}}%
\pgfpathlineto{\pgfqpoint{1.810045in}{0.889797in}}%
\pgfpathlineto{\pgfqpoint{1.811125in}{1.887353in}}%
\pgfpathlineto{\pgfqpoint{1.811796in}{0.996054in}}%
\pgfpathlineto{\pgfqpoint{1.812527in}{1.838683in}}%
\pgfpathlineto{\pgfqpoint{1.813442in}{1.014212in}}%
\pgfpathlineto{\pgfqpoint{1.814348in}{2.002545in}}%
\pgfpathlineto{\pgfqpoint{1.815153in}{0.883977in}}%
\pgfpathlineto{\pgfqpoint{1.815943in}{1.871121in}}%
\pgfpathlineto{\pgfqpoint{1.816744in}{0.941005in}}%
\pgfpathlineto{\pgfqpoint{1.817507in}{1.862785in}}%
\pgfpathlineto{\pgfqpoint{1.818284in}{0.908921in}}%
\pgfpathlineto{\pgfqpoint{1.819233in}{1.882888in}}%
\pgfpathlineto{\pgfqpoint{1.819945in}{0.947852in}}%
\pgfpathlineto{\pgfqpoint{1.820794in}{1.901676in}}%
\pgfpathlineto{\pgfqpoint{1.821603in}{0.939137in}}%
\pgfpathlineto{\pgfqpoint{1.822475in}{1.887538in}}%
\pgfpathlineto{\pgfqpoint{1.823416in}{0.861476in}}%
\pgfpathlineto{\pgfqpoint{1.824152in}{1.867391in}}%
\pgfpathlineto{\pgfqpoint{1.824852in}{0.985484in}}%
\pgfpathlineto{\pgfqpoint{1.825720in}{1.892607in}}%
\pgfpathlineto{\pgfqpoint{1.826521in}{0.947272in}}%
\pgfpathlineto{\pgfqpoint{1.827323in}{1.842663in}}%
\pgfpathlineto{\pgfqpoint{1.828220in}{0.849490in}}%
\pgfpathlineto{\pgfqpoint{1.829068in}{1.878823in}}%
\pgfpathlineto{\pgfqpoint{1.829862in}{0.972942in}}%
\pgfpathlineto{\pgfqpoint{1.830774in}{1.903180in}}%
\pgfpathlineto{\pgfqpoint{1.831549in}{0.856140in}}%
\pgfpathlineto{\pgfqpoint{1.832403in}{1.871444in}}%
\pgfpathlineto{\pgfqpoint{1.833410in}{0.866941in}}%
\pgfpathlineto{\pgfqpoint{1.834000in}{1.917110in}}%
\pgfpathlineto{\pgfqpoint{1.834791in}{0.926469in}}%
\pgfpathlineto{\pgfqpoint{1.835695in}{1.885730in}}%
\pgfpathlineto{\pgfqpoint{1.836433in}{0.983148in}}%
\pgfpathlineto{\pgfqpoint{1.837377in}{1.841927in}}%
\pgfpathlineto{\pgfqpoint{1.838125in}{0.963646in}}%
\pgfpathlineto{\pgfqpoint{1.838882in}{1.876067in}}%
\pgfpathlineto{\pgfqpoint{1.839661in}{0.961999in}}%
\pgfpathlineto{\pgfqpoint{1.840772in}{1.991329in}}%
\pgfpathlineto{\pgfqpoint{1.841549in}{0.912587in}}%
\pgfpathlineto{\pgfqpoint{1.842280in}{1.896458in}}%
\pgfpathlineto{\pgfqpoint{1.843039in}{0.989259in}}%
\pgfpathlineto{\pgfqpoint{1.843871in}{1.920741in}}%
\pgfpathlineto{\pgfqpoint{1.844976in}{0.923070in}}%
\pgfpathlineto{\pgfqpoint{1.845448in}{1.913073in}}%
\pgfpathlineto{\pgfqpoint{1.846278in}{0.974632in}}%
\pgfpathlineto{\pgfqpoint{1.847143in}{1.822496in}}%
\pgfpathlineto{\pgfqpoint{1.847984in}{0.867730in}}%
\pgfpathlineto{\pgfqpoint{1.848821in}{1.822149in}}%
\pgfpathlineto{\pgfqpoint{1.849533in}{0.981897in}}%
\pgfpathlineto{\pgfqpoint{1.850527in}{1.841739in}}%
\pgfpathlineto{\pgfqpoint{1.851362in}{0.914040in}}%
\pgfpathlineto{\pgfqpoint{1.852055in}{1.958092in}}%
\pgfpathlineto{\pgfqpoint{1.853039in}{0.944597in}}%
\pgfpathlineto{\pgfqpoint{1.853673in}{1.809892in}}%
\pgfpathlineto{\pgfqpoint{1.854544in}{0.994054in}}%
\pgfpathlineto{\pgfqpoint{1.855336in}{1.826394in}}%
\pgfpathlineto{\pgfqpoint{1.856426in}{0.920691in}}%
\pgfpathlineto{\pgfqpoint{1.856994in}{1.820766in}}%
\pgfpathlineto{\pgfqpoint{1.857856in}{0.923605in}}%
\pgfpathlineto{\pgfqpoint{1.858684in}{1.900596in}}%
\pgfpathlineto{\pgfqpoint{1.859619in}{0.920320in}}%
\pgfpathlineto{\pgfqpoint{1.860395in}{1.888878in}}%
\pgfpathlineto{\pgfqpoint{1.861099in}{0.924756in}}%
\pgfpathlineto{\pgfqpoint{1.861910in}{1.852863in}}%
\pgfpathlineto{\pgfqpoint{1.862994in}{0.881497in}}%
\pgfpathlineto{\pgfqpoint{1.863570in}{1.847022in}}%
\pgfpathlineto{\pgfqpoint{1.864527in}{0.938182in}}%
\pgfpathlineto{\pgfqpoint{1.865279in}{1.860568in}}%
\pgfpathlineto{\pgfqpoint{1.866093in}{0.946815in}}%
\pgfpathlineto{\pgfqpoint{1.866927in}{1.864378in}}%
\pgfpathlineto{\pgfqpoint{1.868123in}{0.916687in}}%
\pgfpathlineto{\pgfqpoint{1.868507in}{1.859604in}}%
\pgfpathlineto{\pgfqpoint{1.869466in}{0.922916in}}%
\pgfpathlineto{\pgfqpoint{1.870194in}{1.850016in}}%
\pgfpathlineto{\pgfqpoint{1.871158in}{0.937526in}}%
\pgfpathlineto{\pgfqpoint{1.871832in}{1.857938in}}%
\pgfpathlineto{\pgfqpoint{1.872619in}{0.919698in}}%
\pgfpathlineto{\pgfqpoint{1.873504in}{1.853710in}}%
\pgfpathlineto{\pgfqpoint{1.874271in}{1.006680in}}%
\pgfpathlineto{\pgfqpoint{1.875100in}{1.869091in}}%
\pgfpathlineto{\pgfqpoint{1.875890in}{0.951443in}}%
\pgfpathlineto{\pgfqpoint{1.876783in}{1.858968in}}%
\pgfpathlineto{\pgfqpoint{1.877610in}{0.901341in}}%
\pgfpathlineto{\pgfqpoint{1.878584in}{1.934207in}}%
\pgfpathlineto{\pgfqpoint{1.879279in}{0.918559in}}%
\pgfpathlineto{\pgfqpoint{1.880010in}{1.826960in}}%
\pgfpathlineto{\pgfqpoint{1.880806in}{0.964589in}}%
\pgfpathlineto{\pgfqpoint{1.881800in}{1.869607in}}%
\pgfpathlineto{\pgfqpoint{1.882596in}{0.916901in}}%
\pgfpathlineto{\pgfqpoint{1.883289in}{1.838527in}}%
\pgfpathlineto{\pgfqpoint{1.884560in}{0.900276in}}%
\pgfpathlineto{\pgfqpoint{1.884995in}{1.847401in}}%
\pgfpathlineto{\pgfqpoint{1.885782in}{0.983790in}}%
\pgfpathlineto{\pgfqpoint{1.886652in}{1.924814in}}%
\pgfpathlineto{\pgfqpoint{1.887585in}{0.826252in}}%
\pgfpathlineto{\pgfqpoint{1.888237in}{1.791151in}}%
\pgfpathlineto{\pgfqpoint{1.889241in}{0.829434in}}%
\pgfpathlineto{\pgfqpoint{1.889869in}{1.777811in}}%
\pgfpathlineto{\pgfqpoint{1.890789in}{0.993644in}}%
\pgfpathlineto{\pgfqpoint{1.891911in}{1.930644in}}%
\pgfpathlineto{\pgfqpoint{1.892478in}{0.956500in}}%
\pgfpathlineto{\pgfqpoint{1.893354in}{1.863895in}}%
\pgfpathlineto{\pgfqpoint{1.894519in}{0.881553in}}%
\pgfpathlineto{\pgfqpoint{1.894795in}{1.837924in}}%
\pgfpathlineto{\pgfqpoint{1.895617in}{1.017467in}}%
\pgfpathlineto{\pgfqpoint{1.896579in}{1.818057in}}%
\pgfpathlineto{\pgfqpoint{1.897550in}{0.903912in}}%
\pgfpathlineto{\pgfqpoint{1.898232in}{1.893882in}}%
\pgfpathlineto{\pgfqpoint{1.899066in}{0.954466in}}%
\pgfpathlineto{\pgfqpoint{1.899915in}{1.906149in}}%
\pgfpathlineto{\pgfqpoint{1.900827in}{0.909494in}}%
\pgfpathlineto{\pgfqpoint{1.901732in}{1.957849in}}%
\pgfpathlineto{\pgfqpoint{1.902321in}{0.921607in}}%
\pgfpathlineto{\pgfqpoint{1.903057in}{1.867177in}}%
\pgfpathlineto{\pgfqpoint{1.903998in}{0.953177in}}%
\pgfpathlineto{\pgfqpoint{1.904706in}{1.894895in}}%
\pgfpathlineto{\pgfqpoint{1.905708in}{0.983823in}}%
\pgfpathlineto{\pgfqpoint{1.906633in}{1.943869in}}%
\pgfpathlineto{\pgfqpoint{1.907195in}{0.906943in}}%
\pgfpathlineto{\pgfqpoint{1.908021in}{1.869527in}}%
\pgfpathlineto{\pgfqpoint{1.908983in}{0.966518in}}%
\pgfpathlineto{\pgfqpoint{1.909733in}{1.866132in}}%
\pgfpathlineto{\pgfqpoint{1.910664in}{0.881764in}}%
\pgfpathlineto{\pgfqpoint{1.911355in}{1.841690in}}%
\pgfpathlineto{\pgfqpoint{1.912407in}{0.939732in}}%
\pgfpathlineto{\pgfqpoint{1.912965in}{1.888509in}}%
\pgfpathlineto{\pgfqpoint{1.913833in}{0.937832in}}%
\pgfpathlineto{\pgfqpoint{1.914557in}{1.858091in}}%
\pgfpathlineto{\pgfqpoint{1.915437in}{0.926347in}}%
\pgfpathlineto{\pgfqpoint{1.916225in}{1.862290in}}%
\pgfpathlineto{\pgfqpoint{1.917073in}{0.955791in}}%
\pgfpathlineto{\pgfqpoint{1.917901in}{1.865130in}}%
\pgfpathlineto{\pgfqpoint{1.918845in}{0.943853in}}%
\pgfpathlineto{\pgfqpoint{1.919778in}{1.928282in}}%
\pgfpathlineto{\pgfqpoint{1.920374in}{0.925788in}}%
\pgfpathlineto{\pgfqpoint{1.921158in}{1.810934in}}%
\pgfpathlineto{\pgfqpoint{1.922069in}{0.959835in}}%
\pgfpathlineto{\pgfqpoint{1.922796in}{1.839856in}}%
\pgfpathlineto{\pgfqpoint{1.923763in}{0.947897in}}%
\pgfpathlineto{\pgfqpoint{1.924423in}{1.814519in}}%
\pgfpathlineto{\pgfqpoint{1.925308in}{0.870824in}}%
\pgfpathlineto{\pgfqpoint{1.926191in}{1.849315in}}%
\pgfpathlineto{\pgfqpoint{1.926893in}{0.938642in}}%
\pgfpathlineto{\pgfqpoint{1.927782in}{1.917445in}}%
\pgfpathlineto{\pgfqpoint{1.928735in}{0.851639in}}%
\pgfpathlineto{\pgfqpoint{1.929659in}{1.946976in}}%
\pgfpathlineto{\pgfqpoint{1.930213in}{0.972781in}}%
\pgfpathlineto{\pgfqpoint{1.931227in}{1.934806in}}%
\pgfpathlineto{\pgfqpoint{1.931841in}{0.977645in}}%
\pgfpathlineto{\pgfqpoint{1.932747in}{1.825097in}}%
\pgfpathlineto{\pgfqpoint{1.933479in}{0.858200in}}%
\pgfpathlineto{\pgfqpoint{1.934422in}{1.867971in}}%
\pgfpathlineto{\pgfqpoint{1.935454in}{0.850252in}}%
\pgfpathlineto{\pgfqpoint{1.936352in}{1.933470in}}%
\pgfpathlineto{\pgfqpoint{1.936822in}{0.971295in}}%
\pgfpathlineto{\pgfqpoint{1.937681in}{1.924110in}}%
\pgfpathlineto{\pgfqpoint{1.938447in}{0.898008in}}%
\pgfpathlineto{\pgfqpoint{1.939304in}{1.862899in}}%
\pgfpathlineto{\pgfqpoint{1.940164in}{0.944781in}}%
\pgfpathlineto{\pgfqpoint{1.941063in}{1.879333in}}%
\pgfpathlineto{\pgfqpoint{1.941768in}{0.960478in}}%
\pgfpathlineto{\pgfqpoint{1.942750in}{1.919537in}}%
\pgfpathlineto{\pgfqpoint{1.943464in}{0.918971in}}%
\pgfpathlineto{\pgfqpoint{1.944273in}{1.946846in}}%
\pgfpathlineto{\pgfqpoint{1.945098in}{0.988017in}}%
\pgfpathlineto{\pgfqpoint{1.945932in}{1.857676in}}%
\pgfpathlineto{\pgfqpoint{1.946643in}{0.959172in}}%
\pgfpathlineto{\pgfqpoint{1.947664in}{1.889650in}}%
\pgfpathlineto{\pgfqpoint{1.948320in}{0.897243in}}%
\pgfpathlineto{\pgfqpoint{1.949172in}{1.830342in}}%
\pgfpathlineto{\pgfqpoint{1.950600in}{0.830298in}}%
\pgfpathlineto{\pgfqpoint{1.950782in}{1.850133in}}%
\pgfpathlineto{\pgfqpoint{1.951580in}{0.900085in}}%
\pgfpathlineto{\pgfqpoint{1.952431in}{1.890901in}}%
\pgfpathlineto{\pgfqpoint{1.953281in}{0.907575in}}%
\pgfpathlineto{\pgfqpoint{1.954557in}{1.959204in}}%
\pgfpathlineto{\pgfqpoint{1.954875in}{0.921892in}}%
\pgfpathlineto{\pgfqpoint{1.955741in}{1.854375in}}%
\pgfpathlineto{\pgfqpoint{1.956599in}{0.945133in}}%
\pgfpathlineto{\pgfqpoint{1.957340in}{1.841195in}}%
\pgfpathlineto{\pgfqpoint{1.958258in}{0.917562in}}%
\pgfpathlineto{\pgfqpoint{1.958997in}{1.876314in}}%
\pgfpathlineto{\pgfqpoint{1.960020in}{0.884397in}}%
\pgfpathlineto{\pgfqpoint{1.960644in}{1.807889in}}%
\pgfpathlineto{\pgfqpoint{1.961478in}{0.953253in}}%
\pgfpathlineto{\pgfqpoint{1.962558in}{1.895668in}}%
\pgfpathlineto{\pgfqpoint{1.963186in}{0.899854in}}%
\pgfpathlineto{\pgfqpoint{1.963957in}{1.806457in}}%
\pgfpathlineto{\pgfqpoint{1.964809in}{0.923649in}}%
\pgfpathlineto{\pgfqpoint{1.965612in}{1.851825in}}%
\pgfpathlineto{\pgfqpoint{1.966453in}{0.949254in}}%
\pgfpathlineto{\pgfqpoint{1.967395in}{1.890233in}}%
\pgfpathlineto{\pgfqpoint{1.968426in}{0.906444in}}%
\pgfpathlineto{\pgfqpoint{1.968858in}{1.804455in}}%
\pgfpathlineto{\pgfqpoint{1.969689in}{0.940510in}}%
\pgfpathlineto{\pgfqpoint{1.970535in}{1.815719in}}%
\pgfpathlineto{\pgfqpoint{1.971340in}{0.999371in}}%
\pgfpathlineto{\pgfqpoint{1.972276in}{1.852040in}}%
\pgfpathlineto{\pgfqpoint{1.973080in}{0.970996in}}%
\pgfpathlineto{\pgfqpoint{1.974085in}{2.001523in}}%
\pgfpathlineto{\pgfqpoint{1.974690in}{1.006241in}}%
\pgfpathlineto{\pgfqpoint{1.975518in}{1.825071in}}%
\pgfpathlineto{\pgfqpoint{1.976294in}{0.950722in}}%
\pgfpathlineto{\pgfqpoint{1.977146in}{1.896397in}}%
\pgfpathlineto{\pgfqpoint{1.978166in}{0.949150in}}%
\pgfpathlineto{\pgfqpoint{1.978781in}{1.830965in}}%
\pgfpathlineto{\pgfqpoint{1.979605in}{0.919804in}}%
\pgfpathlineto{\pgfqpoint{1.980643in}{1.883833in}}%
\pgfpathlineto{\pgfqpoint{1.981234in}{0.946792in}}%
\pgfpathlineto{\pgfqpoint{1.982080in}{1.896428in}}%
\pgfpathlineto{\pgfqpoint{1.982980in}{0.788022in}}%
\pgfpathlineto{\pgfqpoint{1.983805in}{1.924446in}}%
\pgfpathlineto{\pgfqpoint{1.984491in}{0.943915in}}%
\pgfpathlineto{\pgfqpoint{1.985493in}{1.978185in}}%
\pgfpathlineto{\pgfqpoint{1.986348in}{0.883973in}}%
\pgfpathlineto{\pgfqpoint{1.986980in}{1.881776in}}%
\pgfpathlineto{\pgfqpoint{1.987791in}{0.979544in}}%
\pgfpathlineto{\pgfqpoint{1.988730in}{1.923338in}}%
\pgfpathlineto{\pgfqpoint{1.989601in}{0.963483in}}%
\pgfpathlineto{\pgfqpoint{1.990293in}{1.876016in}}%
\pgfpathlineto{\pgfqpoint{1.991387in}{0.916767in}}%
\pgfpathlineto{\pgfqpoint{1.991947in}{1.869899in}}%
\pgfpathlineto{\pgfqpoint{1.992741in}{0.949150in}}%
\pgfpathlineto{\pgfqpoint{1.993963in}{1.892184in}}%
\pgfpathlineto{\pgfqpoint{1.994371in}{1.005325in}}%
\pgfpathlineto{\pgfqpoint{1.995320in}{1.855942in}}%
\pgfpathlineto{\pgfqpoint{1.996015in}{0.966562in}}%
\pgfpathlineto{\pgfqpoint{1.996859in}{1.835679in}}%
\pgfpathlineto{\pgfqpoint{1.998199in}{0.809376in}}%
\pgfpathlineto{\pgfqpoint{1.998609in}{1.913244in}}%
\pgfpathlineto{\pgfqpoint{1.999375in}{0.913693in}}%
\pgfpathlineto{\pgfqpoint{2.000299in}{1.858168in}}%
\pgfpathlineto{\pgfqpoint{2.000993in}{0.927131in}}%
\pgfpathlineto{\pgfqpoint{2.001904in}{1.811409in}}%
\pgfpathlineto{\pgfqpoint{2.002650in}{0.874330in}}%
\pgfpathlineto{\pgfqpoint{2.003429in}{1.867060in}}%
\pgfpathlineto{\pgfqpoint{2.004267in}{0.982212in}}%
\pgfpathlineto{\pgfqpoint{2.005105in}{1.829197in}}%
\pgfpathlineto{\pgfqpoint{2.005920in}{0.927993in}}%
\pgfpathlineto{\pgfqpoint{2.006787in}{1.927509in}}%
\pgfpathlineto{\pgfqpoint{2.007600in}{0.929508in}}%
\pgfpathlineto{\pgfqpoint{2.008485in}{1.883530in}}%
\pgfpathlineto{\pgfqpoint{2.009387in}{0.954583in}}%
\pgfpathlineto{\pgfqpoint{2.010124in}{1.967908in}}%
\pgfpathlineto{\pgfqpoint{2.011119in}{0.911123in}}%
\pgfpathlineto{\pgfqpoint{2.011696in}{1.947972in}}%
\pgfpathlineto{\pgfqpoint{2.012529in}{0.965660in}}%
\pgfpathlineto{\pgfqpoint{2.013786in}{1.912052in}}%
\pgfpathlineto{\pgfqpoint{2.014140in}{0.998996in}}%
\pgfpathlineto{\pgfqpoint{2.015061in}{1.883181in}}%
\pgfpathlineto{\pgfqpoint{2.015987in}{0.920823in}}%
\pgfpathlineto{\pgfqpoint{2.016929in}{1.893754in}}%
\pgfpathlineto{\pgfqpoint{2.017969in}{0.820555in}}%
\pgfpathlineto{\pgfqpoint{2.018281in}{1.792449in}}%
\pgfpathlineto{\pgfqpoint{2.019370in}{0.889513in}}%
\pgfpathlineto{\pgfqpoint{2.020081in}{1.856640in}}%
\pgfpathlineto{\pgfqpoint{2.020820in}{0.895263in}}%
\pgfpathlineto{\pgfqpoint{2.021607in}{1.845670in}}%
\pgfpathlineto{\pgfqpoint{2.022637in}{0.948520in}}%
\pgfpathlineto{\pgfqpoint{2.023272in}{1.872169in}}%
\pgfpathlineto{\pgfqpoint{2.024002in}{0.863177in}}%
\pgfpathlineto{\pgfqpoint{2.024853in}{1.906089in}}%
\pgfpathlineto{\pgfqpoint{2.025689in}{0.850208in}}%
\pgfpathlineto{\pgfqpoint{2.026464in}{1.786159in}}%
\pgfpathlineto{\pgfqpoint{2.027296in}{1.005688in}}%
\pgfpathlineto{\pgfqpoint{2.028749in}{1.941886in}}%
\pgfpathlineto{\pgfqpoint{2.029040in}{0.947784in}}%
\pgfpathlineto{\pgfqpoint{2.029781in}{1.808882in}}%
\pgfpathlineto{\pgfqpoint{2.030666in}{0.969065in}}%
\pgfpathlineto{\pgfqpoint{2.031421in}{1.853866in}}%
\pgfpathlineto{\pgfqpoint{2.032334in}{0.793138in}}%
\pgfpathlineto{\pgfqpoint{2.033130in}{1.882749in}}%
\pgfpathlineto{\pgfqpoint{2.033887in}{0.980358in}}%
\pgfpathlineto{\pgfqpoint{2.034894in}{1.989535in}}%
\pgfpathlineto{\pgfqpoint{2.035538in}{0.946214in}}%
\pgfpathlineto{\pgfqpoint{2.036457in}{1.852208in}}%
\pgfpathlineto{\pgfqpoint{2.037374in}{0.834766in}}%
\pgfpathlineto{\pgfqpoint{2.038110in}{1.892965in}}%
\pgfpathlineto{\pgfqpoint{2.038804in}{0.959383in}}%
\pgfpathlineto{\pgfqpoint{2.040062in}{1.921132in}}%
\pgfpathlineto{\pgfqpoint{2.040563in}{0.992386in}}%
\pgfpathlineto{\pgfqpoint{2.041592in}{1.918864in}}%
\pgfpathlineto{\pgfqpoint{2.042111in}{0.990963in}}%
\pgfpathlineto{\pgfqpoint{2.043330in}{1.915329in}}%
\pgfpathlineto{\pgfqpoint{2.043775in}{0.923626in}}%
\pgfpathlineto{\pgfqpoint{2.044777in}{1.903886in}}%
\pgfpathlineto{\pgfqpoint{2.045387in}{0.921935in}}%
\pgfpathlineto{\pgfqpoint{2.046249in}{1.842908in}}%
\pgfpathlineto{\pgfqpoint{2.047057in}{0.923356in}}%
\pgfpathlineto{\pgfqpoint{2.048066in}{1.877839in}}%
\pgfpathlineto{\pgfqpoint{2.048713in}{0.907701in}}%
\pgfpathlineto{\pgfqpoint{2.049536in}{1.835846in}}%
\pgfpathlineto{\pgfqpoint{2.050522in}{0.921527in}}%
\pgfpathlineto{\pgfqpoint{2.051212in}{1.850682in}}%
\pgfpathlineto{\pgfqpoint{2.052027in}{0.998155in}}%
\pgfpathlineto{\pgfqpoint{2.053022in}{1.833433in}}%
\pgfpathlineto{\pgfqpoint{2.053621in}{0.990859in}}%
\pgfpathlineto{\pgfqpoint{2.054441in}{1.886744in}}%
\pgfpathlineto{\pgfqpoint{2.055275in}{1.008050in}}%
\pgfpathlineto{\pgfqpoint{2.056114in}{1.823872in}}%
\pgfpathlineto{\pgfqpoint{2.056921in}{0.874640in}}%
\pgfpathlineto{\pgfqpoint{2.057896in}{1.872685in}}%
\pgfpathlineto{\pgfqpoint{2.058608in}{0.948966in}}%
\pgfpathlineto{\pgfqpoint{2.059515in}{1.883426in}}%
\pgfpathlineto{\pgfqpoint{2.060212in}{0.919312in}}%
\pgfpathlineto{\pgfqpoint{2.061076in}{1.927565in}}%
\pgfpathlineto{\pgfqpoint{2.061885in}{0.874781in}}%
\pgfpathlineto{\pgfqpoint{2.062691in}{1.825864in}}%
\pgfpathlineto{\pgfqpoint{2.063574in}{1.002022in}}%
\pgfpathlineto{\pgfqpoint{2.064405in}{1.846659in}}%
\pgfpathlineto{\pgfqpoint{2.065533in}{0.933443in}}%
\pgfpathlineto{\pgfqpoint{2.065962in}{1.834335in}}%
\pgfpathlineto{\pgfqpoint{2.067050in}{0.862159in}}%
\pgfpathlineto{\pgfqpoint{2.067922in}{1.962726in}}%
\pgfpathlineto{\pgfqpoint{2.068521in}{0.962352in}}%
\pgfpathlineto{\pgfqpoint{2.069377in}{1.848612in}}%
\pgfpathlineto{\pgfqpoint{2.070091in}{0.952363in}}%
\pgfpathlineto{\pgfqpoint{2.071128in}{1.890038in}}%
\pgfpathlineto{\pgfqpoint{2.071724in}{0.916248in}}%
\pgfpathlineto{\pgfqpoint{2.072687in}{1.848265in}}%
\pgfpathlineto{\pgfqpoint{2.073407in}{0.948687in}}%
\pgfpathlineto{\pgfqpoint{2.074241in}{1.835408in}}%
\pgfpathlineto{\pgfqpoint{2.075103in}{0.942205in}}%
\pgfpathlineto{\pgfqpoint{2.076163in}{1.879151in}}%
\pgfpathlineto{\pgfqpoint{2.076704in}{0.980427in}}%
\pgfpathlineto{\pgfqpoint{2.078056in}{1.963383in}}%
\pgfpathlineto{\pgfqpoint{2.078662in}{0.882096in}}%
\pgfpathlineto{\pgfqpoint{2.079176in}{1.867339in}}%
\pgfpathlineto{\pgfqpoint{2.079982in}{0.971804in}}%
\pgfpathlineto{\pgfqpoint{2.080855in}{1.895066in}}%
\pgfpathlineto{\pgfqpoint{2.081736in}{0.943770in}}%
\pgfpathlineto{\pgfqpoint{2.082618in}{1.918070in}}%
\pgfpathlineto{\pgfqpoint{2.083382in}{0.952567in}}%
\pgfpathlineto{\pgfqpoint{2.084066in}{1.799430in}}%
\pgfpathlineto{\pgfqpoint{2.085456in}{0.884441in}}%
\pgfpathlineto{\pgfqpoint{2.085744in}{1.863759in}}%
\pgfpathlineto{\pgfqpoint{2.086629in}{0.971537in}}%
\pgfpathlineto{\pgfqpoint{2.087399in}{1.879913in}}%
\pgfpathlineto{\pgfqpoint{2.088464in}{0.849004in}}%
\pgfpathlineto{\pgfqpoint{2.089251in}{1.881516in}}%
\pgfpathlineto{\pgfqpoint{2.089908in}{0.991090in}}%
\pgfpathlineto{\pgfqpoint{2.090913in}{1.894438in}}%
\pgfpathlineto{\pgfqpoint{2.092035in}{0.826565in}}%
\pgfpathlineto{\pgfqpoint{2.092313in}{1.832316in}}%
\pgfpathlineto{\pgfqpoint{2.093298in}{0.845125in}}%
\pgfpathlineto{\pgfqpoint{2.094129in}{1.856372in}}%
\pgfpathlineto{\pgfqpoint{2.095030in}{0.873292in}}%
\pgfpathlineto{\pgfqpoint{2.095648in}{1.912704in}}%
\pgfpathlineto{\pgfqpoint{2.096455in}{0.935736in}}%
\pgfpathlineto{\pgfqpoint{2.097370in}{1.878680in}}%
\pgfpathlineto{\pgfqpoint{2.098350in}{0.888997in}}%
\pgfpathlineto{\pgfqpoint{2.099122in}{1.909388in}}%
\pgfpathlineto{\pgfqpoint{2.099934in}{0.878869in}}%
\pgfpathlineto{\pgfqpoint{2.100623in}{1.893302in}}%
\pgfpathlineto{\pgfqpoint{2.101345in}{1.007647in}}%
\pgfpathlineto{\pgfqpoint{2.102175in}{1.871850in}}%
\pgfpathlineto{\pgfqpoint{2.102994in}{0.960096in}}%
\pgfpathlineto{\pgfqpoint{2.103980in}{1.870850in}}%
\pgfpathlineto{\pgfqpoint{2.104720in}{0.980899in}}%
\pgfpathlineto{\pgfqpoint{2.105481in}{1.861792in}}%
\pgfpathlineto{\pgfqpoint{2.106309in}{1.004508in}}%
\pgfpathlineto{\pgfqpoint{2.107283in}{1.961368in}}%
\pgfpathlineto{\pgfqpoint{2.108323in}{0.887458in}}%
\pgfpathlineto{\pgfqpoint{2.108825in}{1.867682in}}%
\pgfpathlineto{\pgfqpoint{2.109618in}{0.961412in}}%
\pgfpathlineto{\pgfqpoint{2.110487in}{1.845474in}}%
\pgfpathlineto{\pgfqpoint{2.111381in}{0.974401in}}%
\pgfpathlineto{\pgfqpoint{2.112587in}{1.960854in}}%
\pgfpathlineto{\pgfqpoint{2.112865in}{0.981598in}}%
\pgfpathlineto{\pgfqpoint{2.113899in}{1.902911in}}%
\pgfpathlineto{\pgfqpoint{2.114552in}{0.982842in}}%
\pgfpathlineto{\pgfqpoint{2.115443in}{1.858870in}}%
\pgfpathlineto{\pgfqpoint{2.116180in}{0.961525in}}%
\pgfpathlineto{\pgfqpoint{2.117338in}{1.881974in}}%
\pgfpathlineto{\pgfqpoint{2.117985in}{0.956760in}}%
\pgfpathlineto{\pgfqpoint{2.118801in}{1.869287in}}%
\pgfpathlineto{\pgfqpoint{2.119547in}{0.916234in}}%
\pgfpathlineto{\pgfqpoint{2.120579in}{2.020325in}}%
\pgfpathlineto{\pgfqpoint{2.121120in}{1.002792in}}%
\pgfpathlineto{\pgfqpoint{2.122184in}{1.915418in}}%
\pgfpathlineto{\pgfqpoint{2.122918in}{0.970620in}}%
\pgfpathlineto{\pgfqpoint{2.123755in}{1.879913in}}%
\pgfpathlineto{\pgfqpoint{2.124513in}{0.970498in}}%
\pgfpathlineto{\pgfqpoint{2.125323in}{1.886077in}}%
\pgfpathlineto{\pgfqpoint{2.126069in}{0.983202in}}%
\pgfpathlineto{\pgfqpoint{2.126880in}{1.891900in}}%
\pgfpathlineto{\pgfqpoint{2.127937in}{0.859042in}}%
\pgfpathlineto{\pgfqpoint{2.128560in}{1.830189in}}%
\pgfpathlineto{\pgfqpoint{2.129360in}{0.919663in}}%
\pgfpathlineto{\pgfqpoint{2.130316in}{1.885256in}}%
\pgfpathlineto{\pgfqpoint{2.130997in}{0.943111in}}%
\pgfpathlineto{\pgfqpoint{2.132051in}{1.928521in}}%
\pgfpathlineto{\pgfqpoint{2.132714in}{0.956668in}}%
\pgfpathlineto{\pgfqpoint{2.133637in}{2.015120in}}%
\pgfpathlineto{\pgfqpoint{2.134352in}{0.891785in}}%
\pgfpathlineto{\pgfqpoint{2.135171in}{1.881947in}}%
\pgfpathlineto{\pgfqpoint{2.136275in}{0.889134in}}%
\pgfpathlineto{\pgfqpoint{2.136747in}{1.876436in}}%
\pgfpathlineto{\pgfqpoint{2.138148in}{0.856223in}}%
\pgfpathlineto{\pgfqpoint{2.138458in}{1.847701in}}%
\pgfpathlineto{\pgfqpoint{2.139250in}{0.951875in}}%
\pgfpathlineto{\pgfqpoint{2.140232in}{1.873010in}}%
\pgfpathlineto{\pgfqpoint{2.140989in}{0.937129in}}%
\pgfpathlineto{\pgfqpoint{2.141783in}{1.847775in}}%
\pgfpathlineto{\pgfqpoint{2.142582in}{0.952974in}}%
\pgfpathlineto{\pgfqpoint{2.143428in}{2.022161in}}%
\pgfpathlineto{\pgfqpoint{2.144293in}{0.965031in}}%
\pgfpathlineto{\pgfqpoint{2.144973in}{1.896295in}}%
\pgfpathlineto{\pgfqpoint{2.146143in}{0.942559in}}%
\pgfpathlineto{\pgfqpoint{2.146669in}{1.947757in}}%
\pgfpathlineto{\pgfqpoint{2.147479in}{0.883565in}}%
\pgfpathlineto{\pgfqpoint{2.148261in}{1.847198in}}%
\pgfpathlineto{\pgfqpoint{2.149250in}{0.917316in}}%
\pgfpathlineto{\pgfqpoint{2.150218in}{1.867066in}}%
\pgfpathlineto{\pgfqpoint{2.150739in}{0.980804in}}%
\pgfpathlineto{\pgfqpoint{2.151629in}{1.841238in}}%
\pgfpathlineto{\pgfqpoint{2.152394in}{0.896374in}}%
\pgfpathlineto{\pgfqpoint{2.153201in}{1.850680in}}%
\pgfpathlineto{\pgfqpoint{2.154013in}{0.924508in}}%
\pgfpathlineto{\pgfqpoint{2.154880in}{1.862766in}}%
\pgfpathlineto{\pgfqpoint{2.155692in}{0.939333in}}%
\pgfpathlineto{\pgfqpoint{2.156656in}{1.865996in}}%
\pgfpathlineto{\pgfqpoint{2.157485in}{0.882593in}}%
\pgfpathlineto{\pgfqpoint{2.158424in}{1.900914in}}%
\pgfpathlineto{\pgfqpoint{2.159080in}{0.818716in}}%
\pgfpathlineto{\pgfqpoint{2.159827in}{1.825944in}}%
\pgfpathlineto{\pgfqpoint{2.160602in}{0.969944in}}%
\pgfpathlineto{\pgfqpoint{2.161578in}{1.893197in}}%
\pgfpathlineto{\pgfqpoint{2.162365in}{0.820089in}}%
\pgfpathlineto{\pgfqpoint{2.163096in}{1.843787in}}%
\pgfpathlineto{\pgfqpoint{2.164607in}{0.780421in}}%
\pgfpathlineto{\pgfqpoint{2.164739in}{1.890808in}}%
\pgfpathlineto{\pgfqpoint{2.165589in}{0.968176in}}%
\pgfpathlineto{\pgfqpoint{2.166391in}{1.899915in}}%
\pgfpathlineto{\pgfqpoint{2.167190in}{0.960748in}}%
\pgfpathlineto{\pgfqpoint{2.168181in}{1.971816in}}%
\pgfpathlineto{\pgfqpoint{2.168899in}{0.978814in}}%
\pgfpathlineto{\pgfqpoint{2.169724in}{1.957480in}}%
\pgfpathlineto{\pgfqpoint{2.170687in}{0.883683in}}%
\pgfpathlineto{\pgfqpoint{2.171319in}{1.808270in}}%
\pgfpathlineto{\pgfqpoint{2.172188in}{0.790505in}}%
\pgfpathlineto{\pgfqpoint{2.172944in}{1.820605in}}%
\pgfpathlineto{\pgfqpoint{2.173848in}{0.880397in}}%
\pgfpathlineto{\pgfqpoint{2.174762in}{1.887806in}}%
\pgfpathlineto{\pgfqpoint{2.175501in}{0.985837in}}%
\pgfpathlineto{\pgfqpoint{2.176580in}{1.936903in}}%
\pgfpathlineto{\pgfqpoint{2.177106in}{0.965835in}}%
\pgfpathlineto{\pgfqpoint{2.178204in}{1.922266in}}%
\pgfpathlineto{\pgfqpoint{2.178723in}{0.977398in}}%
\pgfpathlineto{\pgfqpoint{2.179566in}{1.880844in}}%
\pgfpathlineto{\pgfqpoint{2.180377in}{0.937331in}}%
\pgfpathlineto{\pgfqpoint{2.181166in}{1.832200in}}%
\pgfpathlineto{\pgfqpoint{2.182080in}{0.974489in}}%
\pgfpathlineto{\pgfqpoint{2.182951in}{1.883259in}}%
\pgfpathlineto{\pgfqpoint{2.183676in}{0.933264in}}%
\pgfpathlineto{\pgfqpoint{2.184703in}{1.914786in}}%
\pgfpathlineto{\pgfqpoint{2.185345in}{0.926756in}}%
\pgfpathlineto{\pgfqpoint{2.186803in}{1.965687in}}%
\pgfpathlineto{\pgfqpoint{2.186953in}{0.984224in}}%
\pgfpathlineto{\pgfqpoint{2.187814in}{1.879938in}}%
\pgfpathlineto{\pgfqpoint{2.188861in}{0.883163in}}%
\pgfpathlineto{\pgfqpoint{2.189433in}{1.896434in}}%
\pgfpathlineto{\pgfqpoint{2.190552in}{0.838629in}}%
\pgfpathlineto{\pgfqpoint{2.191083in}{1.796079in}}%
\pgfpathlineto{\pgfqpoint{2.192261in}{0.916956in}}%
\pgfpathlineto{\pgfqpoint{2.192712in}{1.878294in}}%
\pgfpathlineto{\pgfqpoint{2.193663in}{0.933669in}}%
\pgfpathlineto{\pgfqpoint{2.194433in}{1.851122in}}%
\pgfpathlineto{\pgfqpoint{2.195192in}{0.969625in}}%
\pgfpathlineto{\pgfqpoint{2.196186in}{1.855285in}}%
\pgfpathlineto{\pgfqpoint{2.196977in}{0.965778in}}%
\pgfpathlineto{\pgfqpoint{2.197646in}{1.988226in}}%
\pgfpathlineto{\pgfqpoint{2.198580in}{0.922251in}}%
\pgfpathlineto{\pgfqpoint{2.199344in}{1.850133in}}%
\pgfpathlineto{\pgfqpoint{2.200128in}{0.956832in}}%
\pgfpathlineto{\pgfqpoint{2.201102in}{1.878922in}}%
\pgfpathlineto{\pgfqpoint{2.201798in}{0.973523in}}%
\pgfpathlineto{\pgfqpoint{2.202648in}{1.797479in}}%
\pgfpathlineto{\pgfqpoint{2.203557in}{0.935437in}}%
\pgfpathlineto{\pgfqpoint{2.204560in}{1.970994in}}%
\pgfpathlineto{\pgfqpoint{2.205047in}{0.965523in}}%
\pgfpathlineto{\pgfqpoint{2.206071in}{1.898982in}}%
\pgfpathlineto{\pgfqpoint{2.206788in}{0.992761in}}%
\pgfpathlineto{\pgfqpoint{2.207519in}{1.846977in}}%
\pgfpathlineto{\pgfqpoint{2.208880in}{0.830709in}}%
\pgfpathlineto{\pgfqpoint{2.209179in}{1.840075in}}%
\pgfpathlineto{\pgfqpoint{2.210105in}{0.932308in}}%
\pgfpathlineto{\pgfqpoint{2.211076in}{1.910888in}}%
\pgfpathlineto{\pgfqpoint{2.211796in}{0.857511in}}%
\pgfpathlineto{\pgfqpoint{2.212485in}{1.848629in}}%
\pgfpathlineto{\pgfqpoint{2.213300in}{1.000287in}}%
\pgfpathlineto{\pgfqpoint{2.214110in}{1.843114in}}%
\pgfpathlineto{\pgfqpoint{2.214952in}{0.911744in}}%
\pgfpathlineto{\pgfqpoint{2.215797in}{1.828436in}}%
\pgfpathlineto{\pgfqpoint{2.216555in}{0.948295in}}%
\pgfpathlineto{\pgfqpoint{2.217411in}{1.912268in}}%
\pgfpathlineto{\pgfqpoint{2.218312in}{0.946008in}}%
\pgfpathlineto{\pgfqpoint{2.219174in}{1.958992in}}%
\pgfpathlineto{\pgfqpoint{2.220100in}{0.944548in}}%
\pgfpathlineto{\pgfqpoint{2.220669in}{1.804434in}}%
\pgfpathlineto{\pgfqpoint{2.221767in}{0.860091in}}%
\pgfpathlineto{\pgfqpoint{2.222458in}{1.906041in}}%
\pgfpathlineto{\pgfqpoint{2.223136in}{0.921664in}}%
\pgfpathlineto{\pgfqpoint{2.224205in}{1.915558in}}%
\pgfpathlineto{\pgfqpoint{2.224842in}{0.968506in}}%
\pgfpathlineto{\pgfqpoint{2.225608in}{1.909737in}}%
\pgfpathlineto{\pgfqpoint{2.226457in}{0.948829in}}%
\pgfpathlineto{\pgfqpoint{2.227661in}{1.947901in}}%
\pgfpathlineto{\pgfqpoint{2.228091in}{0.941792in}}%
\pgfpathlineto{\pgfqpoint{2.229080in}{1.869540in}}%
\pgfpathlineto{\pgfqpoint{2.229770in}{0.917962in}}%
\pgfpathlineto{\pgfqpoint{2.230846in}{1.880855in}}%
\pgfpathlineto{\pgfqpoint{2.231569in}{0.943499in}}%
\pgfpathlineto{\pgfqpoint{2.232263in}{1.871213in}}%
\pgfpathlineto{\pgfqpoint{2.233200in}{0.930591in}}%
\pgfpathlineto{\pgfqpoint{2.233905in}{1.794738in}}%
\pgfpathlineto{\pgfqpoint{2.234758in}{0.990876in}}%
\pgfpathlineto{\pgfqpoint{2.235594in}{1.875845in}}%
\pgfpathlineto{\pgfqpoint{2.236415in}{0.970341in}}%
\pgfpathlineto{\pgfqpoint{2.237184in}{1.902489in}}%
\pgfpathlineto{\pgfqpoint{2.238199in}{0.944765in}}%
\pgfpathlineto{\pgfqpoint{2.238783in}{1.994020in}}%
\pgfpathlineto{\pgfqpoint{2.239716in}{0.955593in}}%
\pgfpathlineto{\pgfqpoint{2.240708in}{1.891969in}}%
\pgfpathlineto{\pgfqpoint{2.241370in}{0.969664in}}%
\pgfpathlineto{\pgfqpoint{2.242424in}{1.938565in}}%
\pgfpathlineto{\pgfqpoint{2.242885in}{1.016984in}}%
\pgfpathlineto{\pgfqpoint{2.243720in}{1.842275in}}%
\pgfpathlineto{\pgfqpoint{2.244529in}{0.997434in}}%
\pgfpathlineto{\pgfqpoint{2.245357in}{1.877536in}}%
\pgfpathlineto{\pgfqpoint{2.246432in}{0.905731in}}%
\pgfpathlineto{\pgfqpoint{2.247006in}{1.818897in}}%
\pgfpathlineto{\pgfqpoint{2.248290in}{0.888734in}}%
\pgfpathlineto{\pgfqpoint{2.248669in}{1.853920in}}%
\pgfpathlineto{\pgfqpoint{2.249525in}{0.967121in}}%
\pgfpathlineto{\pgfqpoint{2.250357in}{1.846898in}}%
\pgfpathlineto{\pgfqpoint{2.251120in}{0.929412in}}%
\pgfpathlineto{\pgfqpoint{2.252353in}{1.931450in}}%
\pgfpathlineto{\pgfqpoint{2.252772in}{0.988661in}}%
\pgfpathlineto{\pgfqpoint{2.253607in}{1.889046in}}%
\pgfpathlineto{\pgfqpoint{2.254713in}{0.909252in}}%
\pgfpathlineto{\pgfqpoint{2.255375in}{1.855294in}}%
\pgfpathlineto{\pgfqpoint{2.256139in}{0.964220in}}%
\pgfpathlineto{\pgfqpoint{2.257154in}{1.903518in}}%
\pgfpathlineto{\pgfqpoint{2.257730in}{0.937220in}}%
\pgfpathlineto{\pgfqpoint{2.258555in}{1.861714in}}%
\pgfpathlineto{\pgfqpoint{2.259403in}{0.879271in}}%
\pgfpathlineto{\pgfqpoint{2.260269in}{1.853060in}}%
\pgfpathlineto{\pgfqpoint{2.261031in}{0.942300in}}%
\pgfpathlineto{\pgfqpoint{2.261967in}{1.850065in}}%
\pgfpathlineto{\pgfqpoint{2.262783in}{0.951436in}}%
\pgfpathlineto{\pgfqpoint{2.263683in}{1.939118in}}%
\pgfpathlineto{\pgfqpoint{2.264301in}{0.980121in}}%
\pgfpathlineto{\pgfqpoint{2.265266in}{1.908763in}}%
\pgfpathlineto{\pgfqpoint{2.265986in}{0.948806in}}%
\pgfpathlineto{\pgfqpoint{2.266772in}{1.892035in}}%
\pgfpathlineto{\pgfqpoint{2.267604in}{0.955760in}}%
\pgfpathlineto{\pgfqpoint{2.268478in}{1.852067in}}%
\pgfpathlineto{\pgfqpoint{2.269377in}{0.981203in}}%
\pgfpathlineto{\pgfqpoint{2.270143in}{1.842347in}}%
\pgfpathlineto{\pgfqpoint{2.270963in}{0.959201in}}%
\pgfpathlineto{\pgfqpoint{2.271809in}{1.919263in}}%
\pgfpathlineto{\pgfqpoint{2.272801in}{0.934442in}}%
\pgfpathlineto{\pgfqpoint{2.273404in}{1.833429in}}%
\pgfpathlineto{\pgfqpoint{2.274211in}{0.935385in}}%
\pgfpathlineto{\pgfqpoint{2.275007in}{1.804334in}}%
\pgfpathlineto{\pgfqpoint{2.276005in}{0.926275in}}%
\pgfpathlineto{\pgfqpoint{2.276741in}{1.925482in}}%
\pgfpathlineto{\pgfqpoint{2.277600in}{0.944379in}}%
\pgfpathlineto{\pgfqpoint{2.278318in}{1.886928in}}%
\pgfpathlineto{\pgfqpoint{2.279105in}{0.943949in}}%
\pgfpathlineto{\pgfqpoint{2.280057in}{1.832033in}}%
\pgfpathlineto{\pgfqpoint{2.280929in}{0.924480in}}%
\pgfpathlineto{\pgfqpoint{2.281934in}{1.961646in}}%
\pgfpathlineto{\pgfqpoint{2.282597in}{0.930442in}}%
\pgfpathlineto{\pgfqpoint{2.283244in}{1.791219in}}%
\pgfpathlineto{\pgfqpoint{2.284091in}{0.969984in}}%
\pgfpathlineto{\pgfqpoint{2.284859in}{1.865249in}}%
\pgfpathlineto{\pgfqpoint{2.285975in}{0.873143in}}%
\pgfpathlineto{\pgfqpoint{2.286502in}{1.845904in}}%
\pgfpathlineto{\pgfqpoint{2.287344in}{1.000486in}}%
\pgfpathlineto{\pgfqpoint{2.288195in}{1.822813in}}%
\pgfpathlineto{\pgfqpoint{2.289102in}{0.962214in}}%
\pgfpathlineto{\pgfqpoint{2.289797in}{1.824089in}}%
\pgfpathlineto{\pgfqpoint{2.290890in}{0.861164in}}%
\pgfpathlineto{\pgfqpoint{2.291495in}{1.851732in}}%
\pgfpathlineto{\pgfqpoint{2.292541in}{0.825499in}}%
\pgfpathlineto{\pgfqpoint{2.293266in}{1.821592in}}%
\pgfpathlineto{\pgfqpoint{2.294030in}{0.933734in}}%
\pgfpathlineto{\pgfqpoint{2.294789in}{1.870184in}}%
\pgfpathlineto{\pgfqpoint{2.295778in}{0.870116in}}%
\pgfpathlineto{\pgfqpoint{2.296481in}{1.875273in}}%
\pgfpathlineto{\pgfqpoint{2.297229in}{0.922325in}}%
\pgfpathlineto{\pgfqpoint{2.298410in}{1.914122in}}%
\pgfpathlineto{\pgfqpoint{2.299103in}{0.939797in}}%
\pgfpathlineto{\pgfqpoint{2.299701in}{1.841558in}}%
\pgfpathlineto{\pgfqpoint{2.300570in}{0.967657in}}%
\pgfpathlineto{\pgfqpoint{2.301402in}{1.926833in}}%
\pgfpathlineto{\pgfqpoint{2.302353in}{0.878452in}}%
\pgfpathlineto{\pgfqpoint{2.303060in}{1.840934in}}%
\pgfpathlineto{\pgfqpoint{2.303864in}{0.933827in}}%
\pgfpathlineto{\pgfqpoint{2.304802in}{1.855261in}}%
\pgfpathlineto{\pgfqpoint{2.305520in}{0.935676in}}%
\pgfpathlineto{\pgfqpoint{2.306373in}{1.811187in}}%
\pgfpathlineto{\pgfqpoint{2.307219in}{0.938646in}}%
\pgfpathlineto{\pgfqpoint{2.308307in}{1.992764in}}%
\pgfpathlineto{\pgfqpoint{2.308738in}{0.935822in}}%
\pgfpathlineto{\pgfqpoint{2.309903in}{1.953722in}}%
\pgfpathlineto{\pgfqpoint{2.310368in}{0.980665in}}%
\pgfpathlineto{\pgfqpoint{2.311476in}{1.896877in}}%
\pgfpathlineto{\pgfqpoint{2.312009in}{1.000217in}}%
\pgfpathlineto{\pgfqpoint{2.312931in}{1.865934in}}%
\pgfpathlineto{\pgfqpoint{2.313728in}{0.931863in}}%
\pgfpathlineto{\pgfqpoint{2.314588in}{1.849419in}}%
\pgfpathlineto{\pgfqpoint{2.315306in}{0.973516in}}%
\pgfpathlineto{\pgfqpoint{2.316122in}{1.223945in}}%
\pgfusepath{stroke}%
\end{pgfscope}%
\begin{pgfscope}%
\pgfsetrectcap%
\pgfsetmiterjoin%
\pgfsetlinewidth{0.803000pt}%
\definecolor{currentstroke}{rgb}{0.000000,0.000000,0.000000}%
\pgfsetstrokecolor{currentstroke}%
\pgfsetdash{}{0pt}%
\pgfpathmoveto{\pgfqpoint{0.589745in}{0.416447in}}%
\pgfpathlineto{\pgfqpoint{0.589745in}{2.398330in}}%
\pgfusepath{stroke}%
\end{pgfscope}%
\begin{pgfscope}%
\pgfsetrectcap%
\pgfsetmiterjoin%
\pgfsetlinewidth{0.803000pt}%
\definecolor{currentstroke}{rgb}{0.000000,0.000000,0.000000}%
\pgfsetstrokecolor{currentstroke}%
\pgfsetdash{}{0pt}%
\pgfpathmoveto{\pgfqpoint{2.398330in}{0.416447in}}%
\pgfpathlineto{\pgfqpoint{2.398330in}{2.398330in}}%
\pgfusepath{stroke}%
\end{pgfscope}%
\begin{pgfscope}%
\pgfsetrectcap%
\pgfsetmiterjoin%
\pgfsetlinewidth{0.803000pt}%
\definecolor{currentstroke}{rgb}{0.000000,0.000000,0.000000}%
\pgfsetstrokecolor{currentstroke}%
\pgfsetdash{}{0pt}%
\pgfpathmoveto{\pgfqpoint{0.589745in}{0.416447in}}%
\pgfpathlineto{\pgfqpoint{2.398330in}{0.416447in}}%
\pgfusepath{stroke}%
\end{pgfscope}%
\begin{pgfscope}%
\pgfsetrectcap%
\pgfsetmiterjoin%
\pgfsetlinewidth{0.803000pt}%
\definecolor{currentstroke}{rgb}{0.000000,0.000000,0.000000}%
\pgfsetstrokecolor{currentstroke}%
\pgfsetdash{}{0pt}%
\pgfpathmoveto{\pgfqpoint{0.589745in}{2.398330in}}%
\pgfpathlineto{\pgfqpoint{2.398330in}{2.398330in}}%
\pgfusepath{stroke}%
\end{pgfscope}%
\begin{pgfscope}%
\pgfsetbuttcap%
\pgfsetmiterjoin%
\definecolor{currentfill}{rgb}{1.000000,1.000000,1.000000}%
\pgfsetfillcolor{currentfill}%
\pgfsetfillopacity{0.800000}%
\pgfsetlinewidth{1.003750pt}%
\definecolor{currentstroke}{rgb}{0.800000,0.800000,0.800000}%
\pgfsetstrokecolor{currentstroke}%
\pgfsetstrokeopacity{0.800000}%
\pgfsetdash{}{0pt}%
\pgfpathmoveto{\pgfqpoint{0.667523in}{2.154552in}}%
\pgfpathlineto{\pgfqpoint{1.636634in}{2.154552in}}%
\pgfpathquadraticcurveto{\pgfqpoint{1.658856in}{2.154552in}}{\pgfqpoint{1.658856in}{2.176775in}}%
\pgfpathlineto{\pgfqpoint{1.658856in}{2.320552in}}%
\pgfpathquadraticcurveto{\pgfqpoint{1.658856in}{2.342774in}}{\pgfqpoint{1.636634in}{2.342774in}}%
\pgfpathlineto{\pgfqpoint{0.667523in}{2.342774in}}%
\pgfpathquadraticcurveto{\pgfqpoint{0.645300in}{2.342774in}}{\pgfqpoint{0.645300in}{2.320552in}}%
\pgfpathlineto{\pgfqpoint{0.645300in}{2.176775in}}%
\pgfpathquadraticcurveto{\pgfqpoint{0.645300in}{2.154552in}}{\pgfqpoint{0.667523in}{2.154552in}}%
\pgfpathlineto{\pgfqpoint{0.667523in}{2.154552in}}%
\pgfpathclose%
\pgfusepath{stroke,fill}%
\end{pgfscope}%
\begin{pgfscope}%
\pgfsetrectcap%
\pgfsetroundjoin%
\pgfsetlinewidth{1.505625pt}%
\definecolor{currentstroke}{rgb}{0.000000,0.447059,0.698039}%
\pgfsetstrokecolor{currentstroke}%
\pgfsetdash{}{0pt}%
\pgfpathmoveto{\pgfqpoint{0.689745in}{2.259441in}}%
\pgfpathlineto{\pgfqpoint{0.800856in}{2.259441in}}%
\pgfpathlineto{\pgfqpoint{0.911967in}{2.259441in}}%
\pgfusepath{stroke}%
\end{pgfscope}%
\begin{pgfscope}%
\definecolor{textcolor}{rgb}{0.000000,0.000000,0.000000}%
\pgfsetstrokecolor{textcolor}%
\pgfsetfillcolor{textcolor}%
\pgftext[x=1.000856in,y=2.220552in,left,base]{\color{textcolor}\rmfamily\fontsize{8.000000}{9.600000}\selectfont White noise}%
\end{pgfscope}%
\end{pgfpicture}%
\makeatother%
\endgroup%

        } % scalebox
        \caption{White noise}
    \end{subfigure}
    \begin{subfigure}{0.32\linewidth}
        \centering
        \scalebox{0.75}{%
            %% Creator: Matplotlib, PGF backend
%%
%% To include the figure in your LaTeX document, write
%%   \input{<filename>.pgf}
%%
%% Make sure the required packages are loaded in your preamble
%%   \usepackage{pgf}
%%
%% Also ensure that all the required font packages are loaded; for instance,
%% the lmodern package is sometimes necessary when using math font.
%%   \usepackage{lmodern}
%%
%% Figures using additional raster images can only be included by \input if
%% they are in the same directory as the main LaTeX file. For loading figures
%% from other directories you can use the `import` package
%%   \usepackage{import}
%%
%% and then include the figures with
%%   \import{<path to file>}{<filename>.pgf}
%%
%% Matplotlib used the following preamble
%%   \usepackage{siunitx}
%%   \usepackage{fontspec}
%%   \makeatletter\@ifpackageloaded{underscore}{}{\usepackage[strings]{underscore}}\makeatother
%%
\begingroup%
\makeatletter%
\begin{pgfpicture}%
\pgfpathrectangle{\pgfpointorigin}{\pgfqpoint{2.440000in}{2.440000in}}%
\pgfusepath{use as bounding box, clip}%
\begin{pgfscope}%
\pgfsetbuttcap%
\pgfsetmiterjoin%
\definecolor{currentfill}{rgb}{1.000000,1.000000,1.000000}%
\pgfsetfillcolor{currentfill}%
\pgfsetlinewidth{0.000000pt}%
\definecolor{currentstroke}{rgb}{1.000000,1.000000,1.000000}%
\pgfsetstrokecolor{currentstroke}%
\pgfsetdash{}{0pt}%
\pgfpathmoveto{\pgfqpoint{0.000000in}{0.000000in}}%
\pgfpathlineto{\pgfqpoint{2.440000in}{0.000000in}}%
\pgfpathlineto{\pgfqpoint{2.440000in}{2.440000in}}%
\pgfpathlineto{\pgfqpoint{0.000000in}{2.440000in}}%
\pgfpathlineto{\pgfqpoint{0.000000in}{0.000000in}}%
\pgfpathclose%
\pgfusepath{fill}%
\end{pgfscope}%
\begin{pgfscope}%
\pgfsetbuttcap%
\pgfsetmiterjoin%
\definecolor{currentfill}{rgb}{1.000000,1.000000,1.000000}%
\pgfsetfillcolor{currentfill}%
\pgfsetlinewidth{0.000000pt}%
\definecolor{currentstroke}{rgb}{0.000000,0.000000,0.000000}%
\pgfsetstrokecolor{currentstroke}%
\pgfsetstrokeopacity{0.000000}%
\pgfsetdash{}{0pt}%
\pgfpathmoveto{\pgfqpoint{0.530716in}{0.416447in}}%
\pgfpathlineto{\pgfqpoint{2.398330in}{0.416447in}}%
\pgfpathlineto{\pgfqpoint{2.398330in}{2.398330in}}%
\pgfpathlineto{\pgfqpoint{0.530716in}{2.398330in}}%
\pgfpathlineto{\pgfqpoint{0.530716in}{0.416447in}}%
\pgfpathclose%
\pgfusepath{fill}%
\end{pgfscope}%
\begin{pgfscope}%
\pgfpathrectangle{\pgfqpoint{0.530716in}{0.416447in}}{\pgfqpoint{1.867614in}{1.981883in}}%
\pgfusepath{clip}%
\pgfsetrectcap%
\pgfsetroundjoin%
\pgfsetlinewidth{0.803000pt}%
\definecolor{currentstroke}{rgb}{0.450000,0.450000,0.450000}%
\pgfsetstrokecolor{currentstroke}%
\pgfsetdash{}{0pt}%
\pgfpathmoveto{\pgfqpoint{0.615608in}{0.416447in}}%
\pgfpathlineto{\pgfqpoint{0.615608in}{2.398330in}}%
\pgfusepath{stroke}%
\end{pgfscope}%
\begin{pgfscope}%
\pgfsetbuttcap%
\pgfsetroundjoin%
\definecolor{currentfill}{rgb}{0.000000,0.000000,0.000000}%
\pgfsetfillcolor{currentfill}%
\pgfsetlinewidth{0.803000pt}%
\definecolor{currentstroke}{rgb}{0.000000,0.000000,0.000000}%
\pgfsetstrokecolor{currentstroke}%
\pgfsetdash{}{0pt}%
\pgfsys@defobject{currentmarker}{\pgfqpoint{0.000000in}{-0.048611in}}{\pgfqpoint{0.000000in}{0.000000in}}{%
\pgfpathmoveto{\pgfqpoint{0.000000in}{0.000000in}}%
\pgfpathlineto{\pgfqpoint{0.000000in}{-0.048611in}}%
\pgfusepath{stroke,fill}%
}%
\begin{pgfscope}%
\pgfsys@transformshift{0.615608in}{0.416447in}%
\pgfsys@useobject{currentmarker}{}%
\end{pgfscope}%
\end{pgfscope}%
\begin{pgfscope}%
\definecolor{textcolor}{rgb}{0.000000,0.000000,0.000000}%
\pgfsetstrokecolor{textcolor}%
\pgfsetfillcolor{textcolor}%
\pgftext[x=0.615608in,y=0.319225in,,top]{\color{textcolor}\rmfamily\fontsize{8.000000}{9.600000}\selectfont \(\displaystyle {0}\)}%
\end{pgfscope}%
\begin{pgfscope}%
\pgfpathrectangle{\pgfqpoint{0.530716in}{0.416447in}}{\pgfqpoint{1.867614in}{1.981883in}}%
\pgfusepath{clip}%
\pgfsetrectcap%
\pgfsetroundjoin%
\pgfsetlinewidth{0.803000pt}%
\definecolor{currentstroke}{rgb}{0.450000,0.450000,0.450000}%
\pgfsetstrokecolor{currentstroke}%
\pgfsetdash{}{0pt}%
\pgfpathmoveto{\pgfqpoint{1.121601in}{0.416447in}}%
\pgfpathlineto{\pgfqpoint{1.121601in}{2.398330in}}%
\pgfusepath{stroke}%
\end{pgfscope}%
\begin{pgfscope}%
\pgfsetbuttcap%
\pgfsetroundjoin%
\definecolor{currentfill}{rgb}{0.000000,0.000000,0.000000}%
\pgfsetfillcolor{currentfill}%
\pgfsetlinewidth{0.803000pt}%
\definecolor{currentstroke}{rgb}{0.000000,0.000000,0.000000}%
\pgfsetstrokecolor{currentstroke}%
\pgfsetdash{}{0pt}%
\pgfsys@defobject{currentmarker}{\pgfqpoint{0.000000in}{-0.048611in}}{\pgfqpoint{0.000000in}{0.000000in}}{%
\pgfpathmoveto{\pgfqpoint{0.000000in}{0.000000in}}%
\pgfpathlineto{\pgfqpoint{0.000000in}{-0.048611in}}%
\pgfusepath{stroke,fill}%
}%
\begin{pgfscope}%
\pgfsys@transformshift{1.121601in}{0.416447in}%
\pgfsys@useobject{currentmarker}{}%
\end{pgfscope}%
\end{pgfscope}%
\begin{pgfscope}%
\definecolor{textcolor}{rgb}{0.000000,0.000000,0.000000}%
\pgfsetstrokecolor{textcolor}%
\pgfsetfillcolor{textcolor}%
\pgftext[x=1.121601in,y=0.319225in,,top]{\color{textcolor}\rmfamily\fontsize{8.000000}{9.600000}\selectfont \(\displaystyle {10}\)}%
\end{pgfscope}%
\begin{pgfscope}%
\pgfpathrectangle{\pgfqpoint{0.530716in}{0.416447in}}{\pgfqpoint{1.867614in}{1.981883in}}%
\pgfusepath{clip}%
\pgfsetrectcap%
\pgfsetroundjoin%
\pgfsetlinewidth{0.803000pt}%
\definecolor{currentstroke}{rgb}{0.450000,0.450000,0.450000}%
\pgfsetstrokecolor{currentstroke}%
\pgfsetdash{}{0pt}%
\pgfpathmoveto{\pgfqpoint{1.627594in}{0.416447in}}%
\pgfpathlineto{\pgfqpoint{1.627594in}{2.398330in}}%
\pgfusepath{stroke}%
\end{pgfscope}%
\begin{pgfscope}%
\pgfsetbuttcap%
\pgfsetroundjoin%
\definecolor{currentfill}{rgb}{0.000000,0.000000,0.000000}%
\pgfsetfillcolor{currentfill}%
\pgfsetlinewidth{0.803000pt}%
\definecolor{currentstroke}{rgb}{0.000000,0.000000,0.000000}%
\pgfsetstrokecolor{currentstroke}%
\pgfsetdash{}{0pt}%
\pgfsys@defobject{currentmarker}{\pgfqpoint{0.000000in}{-0.048611in}}{\pgfqpoint{0.000000in}{0.000000in}}{%
\pgfpathmoveto{\pgfqpoint{0.000000in}{0.000000in}}%
\pgfpathlineto{\pgfqpoint{0.000000in}{-0.048611in}}%
\pgfusepath{stroke,fill}%
}%
\begin{pgfscope}%
\pgfsys@transformshift{1.627594in}{0.416447in}%
\pgfsys@useobject{currentmarker}{}%
\end{pgfscope}%
\end{pgfscope}%
\begin{pgfscope}%
\definecolor{textcolor}{rgb}{0.000000,0.000000,0.000000}%
\pgfsetstrokecolor{textcolor}%
\pgfsetfillcolor{textcolor}%
\pgftext[x=1.627594in,y=0.319225in,,top]{\color{textcolor}\rmfamily\fontsize{8.000000}{9.600000}\selectfont \(\displaystyle {20}\)}%
\end{pgfscope}%
\begin{pgfscope}%
\pgfpathrectangle{\pgfqpoint{0.530716in}{0.416447in}}{\pgfqpoint{1.867614in}{1.981883in}}%
\pgfusepath{clip}%
\pgfsetrectcap%
\pgfsetroundjoin%
\pgfsetlinewidth{0.803000pt}%
\definecolor{currentstroke}{rgb}{0.450000,0.450000,0.450000}%
\pgfsetstrokecolor{currentstroke}%
\pgfsetdash{}{0pt}%
\pgfpathmoveto{\pgfqpoint{2.133587in}{0.416447in}}%
\pgfpathlineto{\pgfqpoint{2.133587in}{2.398330in}}%
\pgfusepath{stroke}%
\end{pgfscope}%
\begin{pgfscope}%
\pgfsetbuttcap%
\pgfsetroundjoin%
\definecolor{currentfill}{rgb}{0.000000,0.000000,0.000000}%
\pgfsetfillcolor{currentfill}%
\pgfsetlinewidth{0.803000pt}%
\definecolor{currentstroke}{rgb}{0.000000,0.000000,0.000000}%
\pgfsetstrokecolor{currentstroke}%
\pgfsetdash{}{0pt}%
\pgfsys@defobject{currentmarker}{\pgfqpoint{0.000000in}{-0.048611in}}{\pgfqpoint{0.000000in}{0.000000in}}{%
\pgfpathmoveto{\pgfqpoint{0.000000in}{0.000000in}}%
\pgfpathlineto{\pgfqpoint{0.000000in}{-0.048611in}}%
\pgfusepath{stroke,fill}%
}%
\begin{pgfscope}%
\pgfsys@transformshift{2.133587in}{0.416447in}%
\pgfsys@useobject{currentmarker}{}%
\end{pgfscope}%
\end{pgfscope}%
\begin{pgfscope}%
\definecolor{textcolor}{rgb}{0.000000,0.000000,0.000000}%
\pgfsetstrokecolor{textcolor}%
\pgfsetfillcolor{textcolor}%
\pgftext[x=2.133587in,y=0.319225in,,top]{\color{textcolor}\rmfamily\fontsize{8.000000}{9.600000}\selectfont \(\displaystyle {30}\)}%
\end{pgfscope}%
\begin{pgfscope}%
\definecolor{textcolor}{rgb}{0.000000,0.000000,0.000000}%
\pgfsetstrokecolor{textcolor}%
\pgfsetfillcolor{textcolor}%
\pgftext[x=1.464523in,y=0.165003in,,top]{\color{textcolor}\rmfamily\fontsize{10.000000}{12.000000}\selectfont Time in \(\displaystyle \unit{\second}\)}%
\end{pgfscope}%
\begin{pgfscope}%
\pgfpathrectangle{\pgfqpoint{0.530716in}{0.416447in}}{\pgfqpoint{1.867614in}{1.981883in}}%
\pgfusepath{clip}%
\pgfsetrectcap%
\pgfsetroundjoin%
\pgfsetlinewidth{0.803000pt}%
\definecolor{currentstroke}{rgb}{0.450000,0.450000,0.450000}%
\pgfsetstrokecolor{currentstroke}%
\pgfsetdash{}{0pt}%
\pgfpathmoveto{\pgfqpoint{0.530716in}{0.416447in}}%
\pgfpathlineto{\pgfqpoint{2.398330in}{0.416447in}}%
\pgfusepath{stroke}%
\end{pgfscope}%
\begin{pgfscope}%
\pgfsetbuttcap%
\pgfsetroundjoin%
\definecolor{currentfill}{rgb}{0.000000,0.000000,0.000000}%
\pgfsetfillcolor{currentfill}%
\pgfsetlinewidth{0.803000pt}%
\definecolor{currentstroke}{rgb}{0.000000,0.000000,0.000000}%
\pgfsetstrokecolor{currentstroke}%
\pgfsetdash{}{0pt}%
\pgfsys@defobject{currentmarker}{\pgfqpoint{-0.048611in}{0.000000in}}{\pgfqpoint{-0.000000in}{0.000000in}}{%
\pgfpathmoveto{\pgfqpoint{-0.000000in}{0.000000in}}%
\pgfpathlineto{\pgfqpoint{-0.048611in}{0.000000in}}%
\pgfusepath{stroke,fill}%
}%
\begin{pgfscope}%
\pgfsys@transformshift{0.530716in}{0.416447in}%
\pgfsys@useobject{currentmarker}{}%
\end{pgfscope}%
\end{pgfscope}%
\begin{pgfscope}%
\definecolor{textcolor}{rgb}{0.000000,0.000000,0.000000}%
\pgfsetstrokecolor{textcolor}%
\pgfsetfillcolor{textcolor}%
\pgftext[x=0.223614in, y=0.377892in, left, base]{\color{textcolor}\rmfamily\fontsize{8.000000}{9.600000}\selectfont \(\displaystyle {\ensuremath{-}60}\)}%
\end{pgfscope}%
\begin{pgfscope}%
\pgfpathrectangle{\pgfqpoint{0.530716in}{0.416447in}}{\pgfqpoint{1.867614in}{1.981883in}}%
\pgfusepath{clip}%
\pgfsetrectcap%
\pgfsetroundjoin%
\pgfsetlinewidth{0.803000pt}%
\definecolor{currentstroke}{rgb}{0.450000,0.450000,0.450000}%
\pgfsetstrokecolor{currentstroke}%
\pgfsetdash{}{0pt}%
\pgfpathmoveto{\pgfqpoint{0.530716in}{0.776790in}}%
\pgfpathlineto{\pgfqpoint{2.398330in}{0.776790in}}%
\pgfusepath{stroke}%
\end{pgfscope}%
\begin{pgfscope}%
\pgfsetbuttcap%
\pgfsetroundjoin%
\definecolor{currentfill}{rgb}{0.000000,0.000000,0.000000}%
\pgfsetfillcolor{currentfill}%
\pgfsetlinewidth{0.803000pt}%
\definecolor{currentstroke}{rgb}{0.000000,0.000000,0.000000}%
\pgfsetstrokecolor{currentstroke}%
\pgfsetdash{}{0pt}%
\pgfsys@defobject{currentmarker}{\pgfqpoint{-0.048611in}{0.000000in}}{\pgfqpoint{-0.000000in}{0.000000in}}{%
\pgfpathmoveto{\pgfqpoint{-0.000000in}{0.000000in}}%
\pgfpathlineto{\pgfqpoint{-0.048611in}{0.000000in}}%
\pgfusepath{stroke,fill}%
}%
\begin{pgfscope}%
\pgfsys@transformshift{0.530716in}{0.776790in}%
\pgfsys@useobject{currentmarker}{}%
\end{pgfscope}%
\end{pgfscope}%
\begin{pgfscope}%
\definecolor{textcolor}{rgb}{0.000000,0.000000,0.000000}%
\pgfsetstrokecolor{textcolor}%
\pgfsetfillcolor{textcolor}%
\pgftext[x=0.223614in, y=0.738234in, left, base]{\color{textcolor}\rmfamily\fontsize{8.000000}{9.600000}\selectfont \(\displaystyle {\ensuremath{-}40}\)}%
\end{pgfscope}%
\begin{pgfscope}%
\pgfpathrectangle{\pgfqpoint{0.530716in}{0.416447in}}{\pgfqpoint{1.867614in}{1.981883in}}%
\pgfusepath{clip}%
\pgfsetrectcap%
\pgfsetroundjoin%
\pgfsetlinewidth{0.803000pt}%
\definecolor{currentstroke}{rgb}{0.450000,0.450000,0.450000}%
\pgfsetstrokecolor{currentstroke}%
\pgfsetdash{}{0pt}%
\pgfpathmoveto{\pgfqpoint{0.530716in}{1.137132in}}%
\pgfpathlineto{\pgfqpoint{2.398330in}{1.137132in}}%
\pgfusepath{stroke}%
\end{pgfscope}%
\begin{pgfscope}%
\pgfsetbuttcap%
\pgfsetroundjoin%
\definecolor{currentfill}{rgb}{0.000000,0.000000,0.000000}%
\pgfsetfillcolor{currentfill}%
\pgfsetlinewidth{0.803000pt}%
\definecolor{currentstroke}{rgb}{0.000000,0.000000,0.000000}%
\pgfsetstrokecolor{currentstroke}%
\pgfsetdash{}{0pt}%
\pgfsys@defobject{currentmarker}{\pgfqpoint{-0.048611in}{0.000000in}}{\pgfqpoint{-0.000000in}{0.000000in}}{%
\pgfpathmoveto{\pgfqpoint{-0.000000in}{0.000000in}}%
\pgfpathlineto{\pgfqpoint{-0.048611in}{0.000000in}}%
\pgfusepath{stroke,fill}%
}%
\begin{pgfscope}%
\pgfsys@transformshift{0.530716in}{1.137132in}%
\pgfsys@useobject{currentmarker}{}%
\end{pgfscope}%
\end{pgfscope}%
\begin{pgfscope}%
\definecolor{textcolor}{rgb}{0.000000,0.000000,0.000000}%
\pgfsetstrokecolor{textcolor}%
\pgfsetfillcolor{textcolor}%
\pgftext[x=0.223614in, y=1.098576in, left, base]{\color{textcolor}\rmfamily\fontsize{8.000000}{9.600000}\selectfont \(\displaystyle {\ensuremath{-}20}\)}%
\end{pgfscope}%
\begin{pgfscope}%
\pgfpathrectangle{\pgfqpoint{0.530716in}{0.416447in}}{\pgfqpoint{1.867614in}{1.981883in}}%
\pgfusepath{clip}%
\pgfsetrectcap%
\pgfsetroundjoin%
\pgfsetlinewidth{0.803000pt}%
\definecolor{currentstroke}{rgb}{0.450000,0.450000,0.450000}%
\pgfsetstrokecolor{currentstroke}%
\pgfsetdash{}{0pt}%
\pgfpathmoveto{\pgfqpoint{0.530716in}{1.497474in}}%
\pgfpathlineto{\pgfqpoint{2.398330in}{1.497474in}}%
\pgfusepath{stroke}%
\end{pgfscope}%
\begin{pgfscope}%
\pgfsetbuttcap%
\pgfsetroundjoin%
\definecolor{currentfill}{rgb}{0.000000,0.000000,0.000000}%
\pgfsetfillcolor{currentfill}%
\pgfsetlinewidth{0.803000pt}%
\definecolor{currentstroke}{rgb}{0.000000,0.000000,0.000000}%
\pgfsetstrokecolor{currentstroke}%
\pgfsetdash{}{0pt}%
\pgfsys@defobject{currentmarker}{\pgfqpoint{-0.048611in}{0.000000in}}{\pgfqpoint{-0.000000in}{0.000000in}}{%
\pgfpathmoveto{\pgfqpoint{-0.000000in}{0.000000in}}%
\pgfpathlineto{\pgfqpoint{-0.048611in}{0.000000in}}%
\pgfusepath{stroke,fill}%
}%
\begin{pgfscope}%
\pgfsys@transformshift{0.530716in}{1.497474in}%
\pgfsys@useobject{currentmarker}{}%
\end{pgfscope}%
\end{pgfscope}%
\begin{pgfscope}%
\definecolor{textcolor}{rgb}{0.000000,0.000000,0.000000}%
\pgfsetstrokecolor{textcolor}%
\pgfsetfillcolor{textcolor}%
\pgftext[x=0.374465in, y=1.458919in, left, base]{\color{textcolor}\rmfamily\fontsize{8.000000}{9.600000}\selectfont \(\displaystyle {0}\)}%
\end{pgfscope}%
\begin{pgfscope}%
\pgfpathrectangle{\pgfqpoint{0.530716in}{0.416447in}}{\pgfqpoint{1.867614in}{1.981883in}}%
\pgfusepath{clip}%
\pgfsetrectcap%
\pgfsetroundjoin%
\pgfsetlinewidth{0.803000pt}%
\definecolor{currentstroke}{rgb}{0.450000,0.450000,0.450000}%
\pgfsetstrokecolor{currentstroke}%
\pgfsetdash{}{0pt}%
\pgfpathmoveto{\pgfqpoint{0.530716in}{1.857817in}}%
\pgfpathlineto{\pgfqpoint{2.398330in}{1.857817in}}%
\pgfusepath{stroke}%
\end{pgfscope}%
\begin{pgfscope}%
\pgfsetbuttcap%
\pgfsetroundjoin%
\definecolor{currentfill}{rgb}{0.000000,0.000000,0.000000}%
\pgfsetfillcolor{currentfill}%
\pgfsetlinewidth{0.803000pt}%
\definecolor{currentstroke}{rgb}{0.000000,0.000000,0.000000}%
\pgfsetstrokecolor{currentstroke}%
\pgfsetdash{}{0pt}%
\pgfsys@defobject{currentmarker}{\pgfqpoint{-0.048611in}{0.000000in}}{\pgfqpoint{-0.000000in}{0.000000in}}{%
\pgfpathmoveto{\pgfqpoint{-0.000000in}{0.000000in}}%
\pgfpathlineto{\pgfqpoint{-0.048611in}{0.000000in}}%
\pgfusepath{stroke,fill}%
}%
\begin{pgfscope}%
\pgfsys@transformshift{0.530716in}{1.857817in}%
\pgfsys@useobject{currentmarker}{}%
\end{pgfscope}%
\end{pgfscope}%
\begin{pgfscope}%
\definecolor{textcolor}{rgb}{0.000000,0.000000,0.000000}%
\pgfsetstrokecolor{textcolor}%
\pgfsetfillcolor{textcolor}%
\pgftext[x=0.315437in, y=1.819261in, left, base]{\color{textcolor}\rmfamily\fontsize{8.000000}{9.600000}\selectfont \(\displaystyle {20}\)}%
\end{pgfscope}%
\begin{pgfscope}%
\pgfpathrectangle{\pgfqpoint{0.530716in}{0.416447in}}{\pgfqpoint{1.867614in}{1.981883in}}%
\pgfusepath{clip}%
\pgfsetrectcap%
\pgfsetroundjoin%
\pgfsetlinewidth{0.803000pt}%
\definecolor{currentstroke}{rgb}{0.450000,0.450000,0.450000}%
\pgfsetstrokecolor{currentstroke}%
\pgfsetdash{}{0pt}%
\pgfpathmoveto{\pgfqpoint{0.530716in}{2.218159in}}%
\pgfpathlineto{\pgfqpoint{2.398330in}{2.218159in}}%
\pgfusepath{stroke}%
\end{pgfscope}%
\begin{pgfscope}%
\pgfsetbuttcap%
\pgfsetroundjoin%
\definecolor{currentfill}{rgb}{0.000000,0.000000,0.000000}%
\pgfsetfillcolor{currentfill}%
\pgfsetlinewidth{0.803000pt}%
\definecolor{currentstroke}{rgb}{0.000000,0.000000,0.000000}%
\pgfsetstrokecolor{currentstroke}%
\pgfsetdash{}{0pt}%
\pgfsys@defobject{currentmarker}{\pgfqpoint{-0.048611in}{0.000000in}}{\pgfqpoint{-0.000000in}{0.000000in}}{%
\pgfpathmoveto{\pgfqpoint{-0.000000in}{0.000000in}}%
\pgfpathlineto{\pgfqpoint{-0.048611in}{0.000000in}}%
\pgfusepath{stroke,fill}%
}%
\begin{pgfscope}%
\pgfsys@transformshift{0.530716in}{2.218159in}%
\pgfsys@useobject{currentmarker}{}%
\end{pgfscope}%
\end{pgfscope}%
\begin{pgfscope}%
\definecolor{textcolor}{rgb}{0.000000,0.000000,0.000000}%
\pgfsetstrokecolor{textcolor}%
\pgfsetfillcolor{textcolor}%
\pgftext[x=0.315437in, y=2.179603in, left, base]{\color{textcolor}\rmfamily\fontsize{8.000000}{9.600000}\selectfont \(\displaystyle {40}\)}%
\end{pgfscope}%
\begin{pgfscope}%
\definecolor{textcolor}{rgb}{0.000000,0.000000,0.000000}%
\pgfsetstrokecolor{textcolor}%
\pgfsetfillcolor{textcolor}%
\pgftext[x=0.168059in,y=1.407389in,,bottom,rotate=90.000000]{\color{textcolor}\rmfamily\fontsize{10.000000}{12.000000}\selectfont Ampl. in arb. unit}%
\end{pgfscope}%
\begin{pgfscope}%
\pgfpathrectangle{\pgfqpoint{0.530716in}{0.416447in}}{\pgfqpoint{1.867614in}{1.981883in}}%
\pgfusepath{clip}%
\pgfsetrectcap%
\pgfsetroundjoin%
\pgfsetlinewidth{1.505625pt}%
\definecolor{currentstroke}{rgb}{0.000000,0.619608,0.450980}%
\pgfsetstrokecolor{currentstroke}%
\pgfsetdash{}{0pt}%
\pgfpathmoveto{\pgfqpoint{0.615608in}{1.485487in}}%
\pgfpathlineto{\pgfqpoint{0.616389in}{1.909141in}}%
\pgfpathlineto{\pgfqpoint{0.617033in}{0.948391in}}%
\pgfpathlineto{\pgfqpoint{0.617963in}{1.764174in}}%
\pgfpathlineto{\pgfqpoint{0.618250in}{0.904025in}}%
\pgfpathlineto{\pgfqpoint{0.619525in}{1.879703in}}%
\pgfpathlineto{\pgfqpoint{0.620117in}{1.190003in}}%
\pgfpathlineto{\pgfqpoint{0.621218in}{1.866556in}}%
\pgfpathlineto{\pgfqpoint{0.621729in}{1.169699in}}%
\pgfpathlineto{\pgfqpoint{0.622473in}{1.798642in}}%
\pgfpathlineto{\pgfqpoint{0.623274in}{1.038293in}}%
\pgfpathlineto{\pgfqpoint{0.624306in}{1.755055in}}%
\pgfpathlineto{\pgfqpoint{0.625008in}{0.959144in}}%
\pgfpathlineto{\pgfqpoint{0.625817in}{1.728093in}}%
\pgfpathlineto{\pgfqpoint{0.626751in}{1.001231in}}%
\pgfpathlineto{\pgfqpoint{0.627644in}{1.764671in}}%
\pgfpathlineto{\pgfqpoint{0.628513in}{1.086840in}}%
\pgfpathlineto{\pgfqpoint{0.629405in}{1.901976in}}%
\pgfpathlineto{\pgfqpoint{0.630469in}{1.097899in}}%
\pgfpathlineto{\pgfqpoint{0.631116in}{1.891516in}}%
\pgfpathlineto{\pgfqpoint{0.631869in}{1.080155in}}%
\pgfpathlineto{\pgfqpoint{0.633002in}{1.959529in}}%
\pgfpathlineto{\pgfqpoint{0.633559in}{1.106609in}}%
\pgfpathlineto{\pgfqpoint{0.634752in}{1.924812in}}%
\pgfpathlineto{\pgfqpoint{0.635636in}{1.069654in}}%
\pgfpathlineto{\pgfqpoint{0.636011in}{1.930277in}}%
\pgfpathlineto{\pgfqpoint{0.637564in}{1.001664in}}%
\pgfpathlineto{\pgfqpoint{0.637976in}{1.697895in}}%
\pgfpathlineto{\pgfqpoint{0.638880in}{1.031478in}}%
\pgfpathlineto{\pgfqpoint{0.639618in}{1.864967in}}%
\pgfpathlineto{\pgfqpoint{0.640644in}{1.042401in}}%
\pgfpathlineto{\pgfqpoint{0.641200in}{1.919377in}}%
\pgfpathlineto{\pgfqpoint{0.641987in}{1.125201in}}%
\pgfpathlineto{\pgfqpoint{0.643189in}{1.914067in}}%
\pgfpathlineto{\pgfqpoint{0.644303in}{1.004544in}}%
\pgfpathlineto{\pgfqpoint{0.644610in}{1.810873in}}%
\pgfpathlineto{\pgfqpoint{0.645412in}{1.143979in}}%
\pgfpathlineto{\pgfqpoint{0.646324in}{1.830669in}}%
\pgfpathlineto{\pgfqpoint{0.647349in}{1.014457in}}%
\pgfpathlineto{\pgfqpoint{0.647935in}{1.725601in}}%
\pgfpathlineto{\pgfqpoint{0.649111in}{1.049597in}}%
\pgfpathlineto{\pgfqpoint{0.649885in}{1.728612in}}%
\pgfpathlineto{\pgfqpoint{0.650539in}{1.045009in}}%
\pgfpathlineto{\pgfqpoint{0.651404in}{1.847042in}}%
\pgfpathlineto{\pgfqpoint{0.652829in}{1.035418in}}%
\pgfpathlineto{\pgfqpoint{0.653076in}{1.858303in}}%
\pgfpathlineto{\pgfqpoint{0.654086in}{1.215227in}}%
\pgfpathlineto{\pgfqpoint{0.654822in}{1.851482in}}%
\pgfpathlineto{\pgfqpoint{0.655813in}{1.170465in}}%
\pgfpathlineto{\pgfqpoint{0.656676in}{2.050599in}}%
\pgfpathlineto{\pgfqpoint{0.657520in}{1.121789in}}%
\pgfpathlineto{\pgfqpoint{0.658195in}{1.831591in}}%
\pgfpathlineto{\pgfqpoint{0.658982in}{1.072855in}}%
\pgfpathlineto{\pgfqpoint{0.659878in}{1.750598in}}%
\pgfpathlineto{\pgfqpoint{0.660818in}{0.992206in}}%
\pgfpathlineto{\pgfqpoint{0.661507in}{1.730892in}}%
\pgfpathlineto{\pgfqpoint{0.662535in}{1.123673in}}%
\pgfpathlineto{\pgfqpoint{0.663433in}{1.875986in}}%
\pgfpathlineto{\pgfqpoint{0.664161in}{1.002215in}}%
\pgfpathlineto{\pgfqpoint{0.665063in}{1.693846in}}%
\pgfpathlineto{\pgfqpoint{0.665788in}{1.122375in}}%
\pgfpathlineto{\pgfqpoint{0.666848in}{1.886786in}}%
\pgfpathlineto{\pgfqpoint{0.667464in}{1.144632in}}%
\pgfpathlineto{\pgfqpoint{0.668715in}{1.846849in}}%
\pgfpathlineto{\pgfqpoint{0.669176in}{1.057006in}}%
\pgfpathlineto{\pgfqpoint{0.670189in}{1.879206in}}%
\pgfpathlineto{\pgfqpoint{0.670865in}{1.085301in}}%
\pgfpathlineto{\pgfqpoint{0.671697in}{1.852854in}}%
\pgfpathlineto{\pgfqpoint{0.673266in}{1.949738in}}%
\pgfpathlineto{\pgfqpoint{0.673515in}{1.022005in}}%
\pgfpathlineto{\pgfqpoint{0.674259in}{1.733740in}}%
\pgfpathlineto{\pgfqpoint{0.675123in}{0.953353in}}%
\pgfpathlineto{\pgfqpoint{0.676176in}{1.866137in}}%
\pgfpathlineto{\pgfqpoint{0.677024in}{1.048099in}}%
\pgfpathlineto{\pgfqpoint{0.678065in}{1.828449in}}%
\pgfpathlineto{\pgfqpoint{0.678690in}{1.058248in}}%
\pgfpathlineto{\pgfqpoint{0.679375in}{1.804847in}}%
\pgfpathlineto{\pgfqpoint{0.680790in}{0.997170in}}%
\pgfpathlineto{\pgfqpoint{0.681107in}{1.807746in}}%
\pgfpathlineto{\pgfqpoint{0.682039in}{1.121442in}}%
\pgfpathlineto{\pgfqpoint{0.682929in}{1.865206in}}%
\pgfpathlineto{\pgfqpoint{0.683757in}{1.097940in}}%
\pgfpathlineto{\pgfqpoint{0.684688in}{1.801675in}}%
\pgfpathlineto{\pgfqpoint{0.685487in}{1.028131in}}%
\pgfpathlineto{\pgfqpoint{0.686248in}{1.877328in}}%
\pgfpathlineto{\pgfqpoint{0.687345in}{1.144156in}}%
\pgfpathlineto{\pgfqpoint{0.687895in}{1.852149in}}%
\pgfpathlineto{\pgfqpoint{0.689152in}{1.133254in}}%
\pgfpathlineto{\pgfqpoint{0.689711in}{1.877241in}}%
\pgfpathlineto{\pgfqpoint{0.690471in}{1.052021in}}%
\pgfpathlineto{\pgfqpoint{0.691260in}{1.806044in}}%
\pgfpathlineto{\pgfqpoint{0.692187in}{1.173042in}}%
\pgfpathlineto{\pgfqpoint{0.693537in}{1.985615in}}%
\pgfpathlineto{\pgfqpoint{0.694038in}{0.952525in}}%
\pgfpathlineto{\pgfqpoint{0.694659in}{1.717914in}}%
\pgfpathlineto{\pgfqpoint{0.695494in}{1.059364in}}%
\pgfpathlineto{\pgfqpoint{0.696583in}{1.979335in}}%
\pgfpathlineto{\pgfqpoint{0.697196in}{1.179068in}}%
\pgfpathlineto{\pgfqpoint{0.698297in}{1.800880in}}%
\pgfpathlineto{\pgfqpoint{0.699117in}{1.067784in}}%
\pgfpathlineto{\pgfqpoint{0.700040in}{1.749230in}}%
\pgfpathlineto{\pgfqpoint{0.700646in}{1.014687in}}%
\pgfpathlineto{\pgfqpoint{0.701690in}{1.818126in}}%
\pgfpathlineto{\pgfqpoint{0.702980in}{0.955989in}}%
\pgfpathlineto{\pgfqpoint{0.703181in}{1.712852in}}%
\pgfpathlineto{\pgfqpoint{0.704331in}{1.011309in}}%
\pgfpathlineto{\pgfqpoint{0.704881in}{1.635466in}}%
\pgfpathlineto{\pgfqpoint{0.705828in}{0.985761in}}%
\pgfpathlineto{\pgfqpoint{0.707079in}{1.769500in}}%
\pgfpathlineto{\pgfqpoint{0.707592in}{1.044082in}}%
\pgfpathlineto{\pgfqpoint{0.708309in}{1.767160in}}%
\pgfpathlineto{\pgfqpoint{0.709098in}{1.060046in}}%
\pgfpathlineto{\pgfqpoint{0.710153in}{1.981751in}}%
\pgfpathlineto{\pgfqpoint{0.711174in}{1.091500in}}%
\pgfpathlineto{\pgfqpoint{0.711635in}{1.740557in}}%
\pgfpathlineto{\pgfqpoint{0.712580in}{1.015049in}}%
\pgfpathlineto{\pgfqpoint{0.713621in}{1.926177in}}%
\pgfpathlineto{\pgfqpoint{0.714464in}{1.017466in}}%
\pgfpathlineto{\pgfqpoint{0.715057in}{1.814099in}}%
\pgfpathlineto{\pgfqpoint{0.716240in}{1.091770in}}%
\pgfpathlineto{\pgfqpoint{0.716901in}{2.002395in}}%
\pgfpathlineto{\pgfqpoint{0.717594in}{1.131319in}}%
\pgfpathlineto{\pgfqpoint{0.718526in}{1.839655in}}%
\pgfpathlineto{\pgfqpoint{0.719304in}{1.159978in}}%
\pgfpathlineto{\pgfqpoint{0.720261in}{1.922236in}}%
\pgfpathlineto{\pgfqpoint{0.721081in}{1.192096in}}%
\pgfpathlineto{\pgfqpoint{0.721898in}{1.900919in}}%
\pgfpathlineto{\pgfqpoint{0.722849in}{1.166111in}}%
\pgfpathlineto{\pgfqpoint{0.723618in}{1.875358in}}%
\pgfpathlineto{\pgfqpoint{0.724761in}{1.034061in}}%
\pgfpathlineto{\pgfqpoint{0.725297in}{1.947089in}}%
\pgfpathlineto{\pgfqpoint{0.726131in}{1.253872in}}%
\pgfpathlineto{\pgfqpoint{0.726995in}{1.911148in}}%
\pgfpathlineto{\pgfqpoint{0.728049in}{1.159152in}}%
\pgfpathlineto{\pgfqpoint{0.728650in}{1.911836in}}%
\pgfpathlineto{\pgfqpoint{0.729784in}{1.158513in}}%
\pgfpathlineto{\pgfqpoint{0.730354in}{1.795225in}}%
\pgfpathlineto{\pgfqpoint{0.731501in}{0.948541in}}%
\pgfpathlineto{\pgfqpoint{0.732143in}{1.743059in}}%
\pgfpathlineto{\pgfqpoint{0.732956in}{1.091888in}}%
\pgfpathlineto{\pgfqpoint{0.733994in}{1.863039in}}%
\pgfpathlineto{\pgfqpoint{0.734626in}{0.998558in}}%
\pgfpathlineto{\pgfqpoint{0.735715in}{1.823018in}}%
\pgfpathlineto{\pgfqpoint{0.736405in}{0.956140in}}%
\pgfpathlineto{\pgfqpoint{0.737329in}{1.787004in}}%
\pgfpathlineto{\pgfqpoint{0.738460in}{1.028043in}}%
\pgfpathlineto{\pgfqpoint{0.739377in}{1.930025in}}%
\pgfpathlineto{\pgfqpoint{0.739945in}{0.910240in}}%
\pgfpathlineto{\pgfqpoint{0.740598in}{1.756048in}}%
\pgfpathlineto{\pgfqpoint{0.741623in}{0.990877in}}%
\pgfpathlineto{\pgfqpoint{0.742241in}{1.749395in}}%
\pgfpathlineto{\pgfqpoint{0.743492in}{0.940449in}}%
\pgfpathlineto{\pgfqpoint{0.743972in}{1.668923in}}%
\pgfpathlineto{\pgfqpoint{0.745041in}{1.091491in}}%
\pgfpathlineto{\pgfqpoint{0.745629in}{1.821354in}}%
\pgfpathlineto{\pgfqpoint{0.746532in}{0.938855in}}%
\pgfpathlineto{\pgfqpoint{0.748004in}{1.828111in}}%
\pgfpathlineto{\pgfqpoint{0.748177in}{1.063154in}}%
\pgfpathlineto{\pgfqpoint{0.749041in}{1.757636in}}%
\pgfpathlineto{\pgfqpoint{0.750116in}{1.093138in}}%
\pgfpathlineto{\pgfqpoint{0.750816in}{1.708506in}}%
\pgfpathlineto{\pgfqpoint{0.751626in}{0.979340in}}%
\pgfpathlineto{\pgfqpoint{0.752578in}{1.683446in}}%
\pgfpathlineto{\pgfqpoint{0.753612in}{0.976416in}}%
\pgfpathlineto{\pgfqpoint{0.754150in}{1.806064in}}%
\pgfpathlineto{\pgfqpoint{0.755274in}{0.994307in}}%
\pgfpathlineto{\pgfqpoint{0.755863in}{1.838760in}}%
\pgfpathlineto{\pgfqpoint{0.756757in}{1.144290in}}%
\pgfpathlineto{\pgfqpoint{0.757612in}{1.865093in}}%
\pgfpathlineto{\pgfqpoint{0.758565in}{1.018533in}}%
\pgfpathlineto{\pgfqpoint{0.759335in}{1.973696in}}%
\pgfpathlineto{\pgfqpoint{0.760095in}{1.073093in}}%
\pgfpathlineto{\pgfqpoint{0.761711in}{1.985310in}}%
\pgfpathlineto{\pgfqpoint{0.761812in}{1.269055in}}%
\pgfpathlineto{\pgfqpoint{0.762826in}{1.851812in}}%
\pgfpathlineto{\pgfqpoint{0.763729in}{1.122423in}}%
\pgfpathlineto{\pgfqpoint{0.764390in}{1.949994in}}%
\pgfpathlineto{\pgfqpoint{0.765793in}{0.931277in}}%
\pgfpathlineto{\pgfqpoint{0.766052in}{1.918207in}}%
\pgfpathlineto{\pgfqpoint{0.767034in}{1.249742in}}%
\pgfpathlineto{\pgfqpoint{0.767831in}{1.905778in}}%
\pgfpathlineto{\pgfqpoint{0.768783in}{1.101147in}}%
\pgfpathlineto{\pgfqpoint{0.769581in}{2.019940in}}%
\pgfpathlineto{\pgfqpoint{0.770333in}{1.086298in}}%
\pgfpathlineto{\pgfqpoint{0.771224in}{1.992518in}}%
\pgfpathlineto{\pgfqpoint{0.772075in}{1.262976in}}%
\pgfpathlineto{\pgfqpoint{0.773563in}{2.029100in}}%
\pgfpathlineto{\pgfqpoint{0.773992in}{1.126857in}}%
\pgfpathlineto{\pgfqpoint{0.774550in}{2.018086in}}%
\pgfpathlineto{\pgfqpoint{0.775590in}{1.105534in}}%
\pgfpathlineto{\pgfqpoint{0.776407in}{1.963667in}}%
\pgfpathlineto{\pgfqpoint{0.777301in}{1.055729in}}%
\pgfpathlineto{\pgfqpoint{0.778102in}{1.986373in}}%
\pgfpathlineto{\pgfqpoint{0.778879in}{1.070629in}}%
\pgfpathlineto{\pgfqpoint{0.779669in}{1.820512in}}%
\pgfpathlineto{\pgfqpoint{0.780468in}{1.007059in}}%
\pgfpathlineto{\pgfqpoint{0.781413in}{1.852889in}}%
\pgfpathlineto{\pgfqpoint{0.782790in}{0.957770in}}%
\pgfpathlineto{\pgfqpoint{0.783114in}{1.820707in}}%
\pgfpathlineto{\pgfqpoint{0.784357in}{1.090384in}}%
\pgfpathlineto{\pgfqpoint{0.784850in}{1.965411in}}%
\pgfpathlineto{\pgfqpoint{0.786048in}{1.056542in}}%
\pgfpathlineto{\pgfqpoint{0.786777in}{1.896312in}}%
\pgfpathlineto{\pgfqpoint{0.787358in}{1.191847in}}%
\pgfpathlineto{\pgfqpoint{0.788293in}{1.961772in}}%
\pgfpathlineto{\pgfqpoint{0.789404in}{1.134134in}}%
\pgfpathlineto{\pgfqpoint{0.789868in}{1.766459in}}%
\pgfpathlineto{\pgfqpoint{0.790982in}{1.068900in}}%
\pgfpathlineto{\pgfqpoint{0.791689in}{1.894225in}}%
\pgfpathlineto{\pgfqpoint{0.792561in}{1.171756in}}%
\pgfpathlineto{\pgfqpoint{0.793210in}{1.816320in}}%
\pgfpathlineto{\pgfqpoint{0.794286in}{1.065972in}}%
\pgfpathlineto{\pgfqpoint{0.795106in}{1.954971in}}%
\pgfpathlineto{\pgfqpoint{0.795850in}{1.167790in}}%
\pgfpathlineto{\pgfqpoint{0.796732in}{1.671300in}}%
\pgfpathlineto{\pgfqpoint{0.797592in}{0.843681in}}%
\pgfpathlineto{\pgfqpoint{0.798332in}{1.648414in}}%
\pgfpathlineto{\pgfqpoint{0.799209in}{1.039412in}}%
\pgfpathlineto{\pgfqpoint{0.800038in}{1.749631in}}%
\pgfpathlineto{\pgfqpoint{0.800992in}{0.900221in}}%
\pgfpathlineto{\pgfqpoint{0.801800in}{1.691781in}}%
\pgfpathlineto{\pgfqpoint{0.802959in}{1.040097in}}%
\pgfpathlineto{\pgfqpoint{0.803548in}{1.756590in}}%
\pgfpathlineto{\pgfqpoint{0.804615in}{0.909191in}}%
\pgfpathlineto{\pgfqpoint{0.805197in}{1.717431in}}%
\pgfpathlineto{\pgfqpoint{0.806070in}{0.974538in}}%
\pgfpathlineto{\pgfqpoint{0.807252in}{1.820778in}}%
\pgfpathlineto{\pgfqpoint{0.808178in}{0.928652in}}%
\pgfpathlineto{\pgfqpoint{0.808511in}{1.667442in}}%
\pgfpathlineto{\pgfqpoint{0.810060in}{1.032029in}}%
\pgfpathlineto{\pgfqpoint{0.810524in}{1.917549in}}%
\pgfpathlineto{\pgfqpoint{0.811319in}{0.922950in}}%
\pgfpathlineto{\pgfqpoint{0.812285in}{1.857960in}}%
\pgfpathlineto{\pgfqpoint{0.812756in}{1.102178in}}%
\pgfpathlineto{\pgfqpoint{0.814306in}{1.887609in}}%
\pgfpathlineto{\pgfqpoint{0.814535in}{1.158017in}}%
\pgfpathlineto{\pgfqpoint{0.815512in}{1.871434in}}%
\pgfpathlineto{\pgfqpoint{0.816688in}{0.971406in}}%
\pgfpathlineto{\pgfqpoint{0.817169in}{1.807899in}}%
\pgfpathlineto{\pgfqpoint{0.817921in}{0.990464in}}%
\pgfpathlineto{\pgfqpoint{0.819070in}{1.798077in}}%
\pgfpathlineto{\pgfqpoint{0.819891in}{0.987988in}}%
\pgfpathlineto{\pgfqpoint{0.820581in}{1.805210in}}%
\pgfpathlineto{\pgfqpoint{0.821693in}{0.994336in}}%
\pgfpathlineto{\pgfqpoint{0.822893in}{1.923101in}}%
\pgfpathlineto{\pgfqpoint{0.823255in}{0.988370in}}%
\pgfpathlineto{\pgfqpoint{0.823813in}{1.744059in}}%
\pgfpathlineto{\pgfqpoint{0.825126in}{0.890259in}}%
\pgfpathlineto{\pgfqpoint{0.825824in}{1.869193in}}%
\pgfpathlineto{\pgfqpoint{0.826362in}{0.960276in}}%
\pgfpathlineto{\pgfqpoint{0.827229in}{1.650025in}}%
\pgfpathlineto{\pgfqpoint{0.828187in}{0.990233in}}%
\pgfpathlineto{\pgfqpoint{0.828992in}{1.726321in}}%
\pgfpathlineto{\pgfqpoint{0.830075in}{1.084455in}}%
\pgfpathlineto{\pgfqpoint{0.831014in}{1.941454in}}%
\pgfpathlineto{\pgfqpoint{0.831707in}{1.149357in}}%
\pgfpathlineto{\pgfqpoint{0.832317in}{1.782889in}}%
\pgfpathlineto{\pgfqpoint{0.833187in}{1.202709in}}%
\pgfpathlineto{\pgfqpoint{0.834022in}{1.844514in}}%
\pgfpathlineto{\pgfqpoint{0.834863in}{1.062576in}}%
\pgfpathlineto{\pgfqpoint{0.835866in}{1.802124in}}%
\pgfpathlineto{\pgfqpoint{0.836753in}{0.988447in}}%
\pgfpathlineto{\pgfqpoint{0.837491in}{1.794097in}}%
\pgfpathlineto{\pgfqpoint{0.838463in}{1.045344in}}%
\pgfpathlineto{\pgfqpoint{0.839306in}{1.978786in}}%
\pgfpathlineto{\pgfqpoint{0.840094in}{1.128307in}}%
\pgfpathlineto{\pgfqpoint{0.840915in}{1.830858in}}%
\pgfpathlineto{\pgfqpoint{0.841671in}{1.141764in}}%
\pgfpathlineto{\pgfqpoint{0.843096in}{1.885123in}}%
\pgfpathlineto{\pgfqpoint{0.843458in}{1.102839in}}%
\pgfpathlineto{\pgfqpoint{0.844413in}{1.798879in}}%
\pgfpathlineto{\pgfqpoint{0.845290in}{1.074744in}}%
\pgfpathlineto{\pgfqpoint{0.845947in}{1.810240in}}%
\pgfpathlineto{\pgfqpoint{0.846778in}{1.062367in}}%
\pgfpathlineto{\pgfqpoint{0.848081in}{1.830728in}}%
\pgfpathlineto{\pgfqpoint{0.848518in}{0.922558in}}%
\pgfpathlineto{\pgfqpoint{0.849377in}{1.789719in}}%
\pgfpathlineto{\pgfqpoint{0.850485in}{1.087826in}}%
\pgfpathlineto{\pgfqpoint{0.851157in}{1.801351in}}%
\pgfpathlineto{\pgfqpoint{0.852097in}{0.990926in}}%
\pgfpathlineto{\pgfqpoint{0.852833in}{1.790648in}}%
\pgfpathlineto{\pgfqpoint{0.853701in}{0.979739in}}%
\pgfpathlineto{\pgfqpoint{0.854399in}{1.792572in}}%
\pgfpathlineto{\pgfqpoint{0.855555in}{1.115171in}}%
\pgfpathlineto{\pgfqpoint{0.856533in}{1.977237in}}%
\pgfpathlineto{\pgfqpoint{0.857228in}{1.110526in}}%
\pgfpathlineto{\pgfqpoint{0.858141in}{1.979444in}}%
\pgfpathlineto{\pgfqpoint{0.858684in}{1.172618in}}%
\pgfpathlineto{\pgfqpoint{0.859623in}{1.967067in}}%
\pgfpathlineto{\pgfqpoint{0.860992in}{0.981161in}}%
\pgfpathlineto{\pgfqpoint{0.861320in}{1.817340in}}%
\pgfpathlineto{\pgfqpoint{0.862085in}{1.192748in}}%
\pgfpathlineto{\pgfqpoint{0.862895in}{1.949224in}}%
\pgfpathlineto{\pgfqpoint{0.863812in}{1.204140in}}%
\pgfpathlineto{\pgfqpoint{0.864663in}{1.833968in}}%
\pgfpathlineto{\pgfqpoint{0.865883in}{0.937855in}}%
\pgfpathlineto{\pgfqpoint{0.866308in}{1.767816in}}%
\pgfpathlineto{\pgfqpoint{0.867680in}{1.127841in}}%
\pgfpathlineto{\pgfqpoint{0.868123in}{1.894274in}}%
\pgfpathlineto{\pgfqpoint{0.868928in}{0.981381in}}%
\pgfpathlineto{\pgfqpoint{0.869959in}{1.864318in}}%
\pgfpathlineto{\pgfqpoint{0.870616in}{1.187849in}}%
\pgfpathlineto{\pgfqpoint{0.871396in}{1.798964in}}%
\pgfpathlineto{\pgfqpoint{0.872553in}{1.113078in}}%
\pgfpathlineto{\pgfqpoint{0.873257in}{1.837865in}}%
\pgfpathlineto{\pgfqpoint{0.874072in}{0.975199in}}%
\pgfpathlineto{\pgfqpoint{0.874844in}{1.924923in}}%
\pgfpathlineto{\pgfqpoint{0.875909in}{1.074210in}}%
\pgfpathlineto{\pgfqpoint{0.876951in}{1.912138in}}%
\pgfpathlineto{\pgfqpoint{0.877649in}{1.079988in}}%
\pgfpathlineto{\pgfqpoint{0.878982in}{1.913677in}}%
\pgfpathlineto{\pgfqpoint{0.879068in}{1.262213in}}%
\pgfpathlineto{\pgfqpoint{0.880171in}{1.949709in}}%
\pgfpathlineto{\pgfqpoint{0.880739in}{1.251880in}}%
\pgfpathlineto{\pgfqpoint{0.881587in}{1.897722in}}%
\pgfpathlineto{\pgfqpoint{0.883098in}{1.097569in}}%
\pgfpathlineto{\pgfqpoint{0.883571in}{1.992210in}}%
\pgfpathlineto{\pgfqpoint{0.884291in}{1.113557in}}%
\pgfpathlineto{\pgfqpoint{0.885122in}{1.785866in}}%
\pgfpathlineto{\pgfqpoint{0.885928in}{0.919211in}}%
\pgfpathlineto{\pgfqpoint{0.886689in}{1.827729in}}%
\pgfpathlineto{\pgfqpoint{0.887597in}{0.973519in}}%
\pgfpathlineto{\pgfqpoint{0.888397in}{1.681720in}}%
\pgfpathlineto{\pgfqpoint{0.889586in}{1.145876in}}%
\pgfpathlineto{\pgfqpoint{0.890091in}{1.933185in}}%
\pgfpathlineto{\pgfqpoint{0.891219in}{1.114141in}}%
\pgfpathlineto{\pgfqpoint{0.891845in}{1.925473in}}%
\pgfpathlineto{\pgfqpoint{0.892729in}{1.120930in}}%
\pgfpathlineto{\pgfqpoint{0.893887in}{1.988610in}}%
\pgfpathlineto{\pgfqpoint{0.894801in}{1.226741in}}%
\pgfpathlineto{\pgfqpoint{0.895452in}{2.000226in}}%
\pgfpathlineto{\pgfqpoint{0.896081in}{1.298507in}}%
\pgfpathlineto{\pgfqpoint{0.896951in}{1.828617in}}%
\pgfpathlineto{\pgfqpoint{0.897991in}{1.095978in}}%
\pgfpathlineto{\pgfqpoint{0.898657in}{2.041574in}}%
\pgfpathlineto{\pgfqpoint{0.899521in}{1.254728in}}%
\pgfpathlineto{\pgfqpoint{0.900510in}{1.969895in}}%
\pgfpathlineto{\pgfqpoint{0.901600in}{1.076471in}}%
\pgfpathlineto{\pgfqpoint{0.902274in}{1.897999in}}%
\pgfpathlineto{\pgfqpoint{0.903408in}{1.126441in}}%
\pgfpathlineto{\pgfqpoint{0.904006in}{1.964832in}}%
\pgfpathlineto{\pgfqpoint{0.904656in}{1.149664in}}%
\pgfpathlineto{\pgfqpoint{0.905798in}{1.984566in}}%
\pgfpathlineto{\pgfqpoint{0.906614in}{1.125456in}}%
\pgfpathlineto{\pgfqpoint{0.907098in}{1.854249in}}%
\pgfpathlineto{\pgfqpoint{0.907967in}{1.185363in}}%
\pgfpathlineto{\pgfqpoint{0.909253in}{2.052736in}}%
\pgfpathlineto{\pgfqpoint{0.909830in}{1.207097in}}%
\pgfpathlineto{\pgfqpoint{0.911080in}{2.072833in}}%
\pgfpathlineto{\pgfqpoint{0.911633in}{1.262770in}}%
\pgfpathlineto{\pgfqpoint{0.912405in}{2.045773in}}%
\pgfpathlineto{\pgfqpoint{0.913060in}{1.285226in}}%
\pgfpathlineto{\pgfqpoint{0.913930in}{1.846878in}}%
\pgfpathlineto{\pgfqpoint{0.915267in}{0.984115in}}%
\pgfpathlineto{\pgfqpoint{0.915599in}{1.801504in}}%
\pgfpathlineto{\pgfqpoint{0.916819in}{1.111489in}}%
\pgfpathlineto{\pgfqpoint{0.917821in}{1.966423in}}%
\pgfpathlineto{\pgfqpoint{0.918490in}{1.040068in}}%
\pgfpathlineto{\pgfqpoint{0.919295in}{1.947879in}}%
\pgfpathlineto{\pgfqpoint{0.919912in}{1.134994in}}%
\pgfpathlineto{\pgfqpoint{0.921306in}{2.029680in}}%
\pgfpathlineto{\pgfqpoint{0.921597in}{1.185892in}}%
\pgfpathlineto{\pgfqpoint{0.922392in}{1.889818in}}%
\pgfpathlineto{\pgfqpoint{0.923457in}{1.191549in}}%
\pgfpathlineto{\pgfqpoint{0.924119in}{1.842257in}}%
\pgfpathlineto{\pgfqpoint{0.925058in}{1.221022in}}%
\pgfpathlineto{\pgfqpoint{0.926032in}{1.854886in}}%
\pgfpathlineto{\pgfqpoint{0.926655in}{1.135222in}}%
\pgfpathlineto{\pgfqpoint{0.927947in}{1.849460in}}%
\pgfpathlineto{\pgfqpoint{0.928325in}{1.052273in}}%
\pgfpathlineto{\pgfqpoint{0.929333in}{1.897166in}}%
\pgfpathlineto{\pgfqpoint{0.930073in}{1.160307in}}%
\pgfpathlineto{\pgfqpoint{0.931155in}{1.974871in}}%
\pgfpathlineto{\pgfqpoint{0.932241in}{1.024521in}}%
\pgfpathlineto{\pgfqpoint{0.932572in}{1.881097in}}%
\pgfpathlineto{\pgfqpoint{0.933540in}{1.261711in}}%
\pgfpathlineto{\pgfqpoint{0.934284in}{1.931705in}}%
\pgfpathlineto{\pgfqpoint{0.935216in}{1.253165in}}%
\pgfpathlineto{\pgfqpoint{0.936498in}{1.924508in}}%
\pgfpathlineto{\pgfqpoint{0.936910in}{1.039181in}}%
\pgfpathlineto{\pgfqpoint{0.937717in}{1.887937in}}%
\pgfpathlineto{\pgfqpoint{0.938590in}{1.030566in}}%
\pgfpathlineto{\pgfqpoint{0.939536in}{1.788535in}}%
\pgfpathlineto{\pgfqpoint{0.940537in}{1.069179in}}%
\pgfpathlineto{\pgfqpoint{0.941238in}{1.839115in}}%
\pgfpathlineto{\pgfqpoint{0.942275in}{1.120660in}}%
\pgfpathlineto{\pgfqpoint{0.943457in}{2.045879in}}%
\pgfpathlineto{\pgfqpoint{0.943936in}{1.179554in}}%
\pgfpathlineto{\pgfqpoint{0.944627in}{1.763704in}}%
\pgfpathlineto{\pgfqpoint{0.945345in}{1.246525in}}%
\pgfpathlineto{\pgfqpoint{0.946863in}{1.859905in}}%
\pgfpathlineto{\pgfqpoint{0.947315in}{1.007940in}}%
\pgfpathlineto{\pgfqpoint{0.948046in}{1.889897in}}%
\pgfpathlineto{\pgfqpoint{0.949240in}{1.052943in}}%
\pgfpathlineto{\pgfqpoint{0.949627in}{1.756918in}}%
\pgfpathlineto{\pgfqpoint{0.950612in}{1.071995in}}%
\pgfpathlineto{\pgfqpoint{0.951327in}{1.818563in}}%
\pgfpathlineto{\pgfqpoint{0.952220in}{1.145946in}}%
\pgfpathlineto{\pgfqpoint{0.953208in}{1.829940in}}%
\pgfpathlineto{\pgfqpoint{0.953933in}{0.957653in}}%
\pgfpathlineto{\pgfqpoint{0.954665in}{1.864947in}}%
\pgfpathlineto{\pgfqpoint{0.955632in}{1.075104in}}%
\pgfpathlineto{\pgfqpoint{0.956544in}{1.864569in}}%
\pgfpathlineto{\pgfqpoint{0.957405in}{1.138263in}}%
\pgfpathlineto{\pgfqpoint{0.958661in}{1.921434in}}%
\pgfpathlineto{\pgfqpoint{0.958983in}{1.058062in}}%
\pgfpathlineto{\pgfqpoint{0.959830in}{1.845567in}}%
\pgfpathlineto{\pgfqpoint{0.960991in}{0.875117in}}%
\pgfpathlineto{\pgfqpoint{0.961557in}{1.718996in}}%
\pgfpathlineto{\pgfqpoint{0.962824in}{1.164750in}}%
\pgfpathlineto{\pgfqpoint{0.963527in}{1.970816in}}%
\pgfpathlineto{\pgfqpoint{0.964098in}{1.191117in}}%
\pgfpathlineto{\pgfqpoint{0.965483in}{1.941558in}}%
\pgfpathlineto{\pgfqpoint{0.965792in}{1.137274in}}%
\pgfpathlineto{\pgfqpoint{0.966635in}{2.029528in}}%
\pgfpathlineto{\pgfqpoint{0.967522in}{1.194858in}}%
\pgfpathlineto{\pgfqpoint{0.968514in}{1.820638in}}%
\pgfpathlineto{\pgfqpoint{0.969111in}{1.042259in}}%
\pgfpathlineto{\pgfqpoint{0.970048in}{1.878741in}}%
\pgfpathlineto{\pgfqpoint{0.970926in}{1.115357in}}%
\pgfpathlineto{\pgfqpoint{0.971955in}{1.907986in}}%
\pgfpathlineto{\pgfqpoint{0.972602in}{1.074840in}}%
\pgfpathlineto{\pgfqpoint{0.973792in}{1.845346in}}%
\pgfpathlineto{\pgfqpoint{0.974270in}{0.985680in}}%
\pgfpathlineto{\pgfqpoint{0.975581in}{1.661080in}}%
\pgfpathlineto{\pgfqpoint{0.976037in}{0.911889in}}%
\pgfpathlineto{\pgfqpoint{0.976848in}{1.764085in}}%
\pgfpathlineto{\pgfqpoint{0.977702in}{0.901448in}}%
\pgfpathlineto{\pgfqpoint{0.978607in}{1.641372in}}%
\pgfpathlineto{\pgfqpoint{0.979919in}{0.883105in}}%
\pgfpathlineto{\pgfqpoint{0.980302in}{1.627585in}}%
\pgfpathlineto{\pgfqpoint{0.981025in}{1.066795in}}%
\pgfpathlineto{\pgfqpoint{0.981880in}{1.764934in}}%
\pgfpathlineto{\pgfqpoint{0.982967in}{1.075512in}}%
\pgfpathlineto{\pgfqpoint{0.983631in}{1.664064in}}%
\pgfpathlineto{\pgfqpoint{0.984781in}{0.957546in}}%
\pgfpathlineto{\pgfqpoint{0.985448in}{1.835516in}}%
\pgfpathlineto{\pgfqpoint{0.986325in}{1.074889in}}%
\pgfpathlineto{\pgfqpoint{0.987043in}{1.783712in}}%
\pgfpathlineto{\pgfqpoint{0.988318in}{0.947834in}}%
\pgfpathlineto{\pgfqpoint{0.988819in}{1.824236in}}%
\pgfpathlineto{\pgfqpoint{0.989964in}{1.041174in}}%
\pgfpathlineto{\pgfqpoint{0.991024in}{1.900659in}}%
\pgfpathlineto{\pgfqpoint{0.991298in}{1.054288in}}%
\pgfpathlineto{\pgfqpoint{0.992305in}{1.783436in}}%
\pgfpathlineto{\pgfqpoint{0.993053in}{0.931287in}}%
\pgfpathlineto{\pgfqpoint{0.993780in}{1.735517in}}%
\pgfpathlineto{\pgfqpoint{0.994914in}{1.092685in}}%
\pgfpathlineto{\pgfqpoint{0.995533in}{1.741546in}}%
\pgfpathlineto{\pgfqpoint{0.996600in}{0.890472in}}%
\pgfpathlineto{\pgfqpoint{0.997266in}{1.751419in}}%
\pgfpathlineto{\pgfqpoint{0.998025in}{1.028674in}}%
\pgfpathlineto{\pgfqpoint{0.999197in}{1.745645in}}%
\pgfpathlineto{\pgfqpoint{1.000007in}{0.904836in}}%
\pgfpathlineto{\pgfqpoint{1.000708in}{1.712132in}}%
\pgfpathlineto{\pgfqpoint{1.001410in}{0.978356in}}%
\pgfpathlineto{\pgfqpoint{1.002355in}{1.832911in}}%
\pgfpathlineto{\pgfqpoint{1.003110in}{1.062861in}}%
\pgfpathlineto{\pgfqpoint{1.003969in}{1.777727in}}%
\pgfpathlineto{\pgfqpoint{1.005413in}{0.985473in}}%
\pgfpathlineto{\pgfqpoint{1.005756in}{1.782209in}}%
\pgfpathlineto{\pgfqpoint{1.006603in}{1.029232in}}%
\pgfpathlineto{\pgfqpoint{1.007371in}{1.925186in}}%
\pgfpathlineto{\pgfqpoint{1.008399in}{1.145769in}}%
\pgfpathlineto{\pgfqpoint{1.009135in}{1.884702in}}%
\pgfpathlineto{\pgfqpoint{1.010048in}{1.061595in}}%
\pgfpathlineto{\pgfqpoint{1.010808in}{1.792001in}}%
\pgfpathlineto{\pgfqpoint{1.011891in}{1.103837in}}%
\pgfpathlineto{\pgfqpoint{1.012708in}{1.921912in}}%
\pgfpathlineto{\pgfqpoint{1.013510in}{0.991913in}}%
\pgfpathlineto{\pgfqpoint{1.014279in}{1.823036in}}%
\pgfpathlineto{\pgfqpoint{1.015172in}{1.052349in}}%
\pgfpathlineto{\pgfqpoint{1.016103in}{1.836014in}}%
\pgfpathlineto{\pgfqpoint{1.016738in}{1.140515in}}%
\pgfpathlineto{\pgfqpoint{1.017736in}{1.781432in}}%
\pgfpathlineto{\pgfqpoint{1.018900in}{0.900362in}}%
\pgfpathlineto{\pgfqpoint{1.019256in}{1.681912in}}%
\pgfpathlineto{\pgfqpoint{1.020809in}{1.004907in}}%
\pgfpathlineto{\pgfqpoint{1.021058in}{1.826830in}}%
\pgfpathlineto{\pgfqpoint{1.022243in}{1.003696in}}%
\pgfpathlineto{\pgfqpoint{1.022646in}{1.815043in}}%
\pgfpathlineto{\pgfqpoint{1.023764in}{1.059708in}}%
\pgfpathlineto{\pgfqpoint{1.024525in}{1.818509in}}%
\pgfpathlineto{\pgfqpoint{1.025384in}{0.981069in}}%
\pgfpathlineto{\pgfqpoint{1.026049in}{1.785220in}}%
\pgfpathlineto{\pgfqpoint{1.027582in}{1.175781in}}%
\pgfpathlineto{\pgfqpoint{1.027862in}{1.963483in}}%
\pgfpathlineto{\pgfqpoint{1.028626in}{1.124442in}}%
\pgfpathlineto{\pgfqpoint{1.029773in}{1.885669in}}%
\pgfpathlineto{\pgfqpoint{1.030360in}{1.182130in}}%
\pgfpathlineto{\pgfqpoint{1.031821in}{1.938933in}}%
\pgfpathlineto{\pgfqpoint{1.031996in}{1.199020in}}%
\pgfpathlineto{\pgfqpoint{1.032984in}{1.928994in}}%
\pgfpathlineto{\pgfqpoint{1.033910in}{1.151320in}}%
\pgfpathlineto{\pgfqpoint{1.034544in}{1.871597in}}%
\pgfpathlineto{\pgfqpoint{1.035584in}{1.306230in}}%
\pgfpathlineto{\pgfqpoint{1.036685in}{1.952306in}}%
\pgfpathlineto{\pgfqpoint{1.037249in}{1.105013in}}%
\pgfpathlineto{\pgfqpoint{1.038132in}{1.938596in}}%
\pgfpathlineto{\pgfqpoint{1.038852in}{1.205721in}}%
\pgfpathlineto{\pgfqpoint{1.039949in}{1.832099in}}%
\pgfpathlineto{\pgfqpoint{1.040521in}{1.170502in}}%
\pgfpathlineto{\pgfqpoint{1.041727in}{1.804723in}}%
\pgfpathlineto{\pgfqpoint{1.042506in}{1.137365in}}%
\pgfpathlineto{\pgfqpoint{1.043111in}{1.850976in}}%
\pgfpathlineto{\pgfqpoint{1.043957in}{1.078382in}}%
\pgfpathlineto{\pgfqpoint{1.044867in}{1.918939in}}%
\pgfpathlineto{\pgfqpoint{1.045653in}{1.156901in}}%
\pgfpathlineto{\pgfqpoint{1.046509in}{1.920322in}}%
\pgfpathlineto{\pgfqpoint{1.047376in}{1.101506in}}%
\pgfpathlineto{\pgfqpoint{1.048207in}{1.905055in}}%
\pgfpathlineto{\pgfqpoint{1.049062in}{1.018705in}}%
\pgfpathlineto{\pgfqpoint{1.050129in}{1.928496in}}%
\pgfpathlineto{\pgfqpoint{1.050827in}{1.106261in}}%
\pgfpathlineto{\pgfqpoint{1.051729in}{1.853797in}}%
\pgfpathlineto{\pgfqpoint{1.052564in}{0.997646in}}%
\pgfpathlineto{\pgfqpoint{1.053283in}{1.804453in}}%
\pgfpathlineto{\pgfqpoint{1.054466in}{0.937400in}}%
\pgfpathlineto{\pgfqpoint{1.055074in}{1.844943in}}%
\pgfpathlineto{\pgfqpoint{1.056104in}{1.016261in}}%
\pgfpathlineto{\pgfqpoint{1.057185in}{1.771246in}}%
\pgfpathlineto{\pgfqpoint{1.057642in}{1.015131in}}%
\pgfpathlineto{\pgfqpoint{1.058343in}{1.820999in}}%
\pgfpathlineto{\pgfqpoint{1.059519in}{0.980505in}}%
\pgfpathlineto{\pgfqpoint{1.060071in}{1.798861in}}%
\pgfpathlineto{\pgfqpoint{1.061173in}{1.031310in}}%
\pgfpathlineto{\pgfqpoint{1.061798in}{1.711662in}}%
\pgfpathlineto{\pgfqpoint{1.062593in}{1.059258in}}%
\pgfpathlineto{\pgfqpoint{1.063508in}{1.836880in}}%
\pgfpathlineto{\pgfqpoint{1.064324in}{1.015117in}}%
\pgfpathlineto{\pgfqpoint{1.065275in}{1.823268in}}%
\pgfpathlineto{\pgfqpoint{1.066011in}{0.948415in}}%
\pgfpathlineto{\pgfqpoint{1.067041in}{1.820442in}}%
\pgfpathlineto{\pgfqpoint{1.067764in}{0.969966in}}%
\pgfpathlineto{\pgfqpoint{1.068565in}{1.659407in}}%
\pgfpathlineto{\pgfqpoint{1.069663in}{0.933484in}}%
\pgfpathlineto{\pgfqpoint{1.070236in}{1.633418in}}%
\pgfpathlineto{\pgfqpoint{1.071737in}{0.845241in}}%
\pgfpathlineto{\pgfqpoint{1.072047in}{1.571927in}}%
\pgfpathlineto{\pgfqpoint{1.073024in}{0.890434in}}%
\pgfpathlineto{\pgfqpoint{1.073669in}{1.702189in}}%
\pgfpathlineto{\pgfqpoint{1.074662in}{0.867254in}}%
\pgfpathlineto{\pgfqpoint{1.075432in}{1.682314in}}%
\pgfpathlineto{\pgfqpoint{1.076373in}{0.976113in}}%
\pgfpathlineto{\pgfqpoint{1.077209in}{1.724140in}}%
\pgfpathlineto{\pgfqpoint{1.078444in}{0.876992in}}%
\pgfpathlineto{\pgfqpoint{1.078861in}{1.765999in}}%
\pgfpathlineto{\pgfqpoint{1.079614in}{0.938837in}}%
\pgfpathlineto{\pgfqpoint{1.080676in}{1.726553in}}%
\pgfpathlineto{\pgfqpoint{1.081340in}{0.891502in}}%
\pgfpathlineto{\pgfqpoint{1.082555in}{1.820084in}}%
\pgfpathlineto{\pgfqpoint{1.083102in}{1.057895in}}%
\pgfpathlineto{\pgfqpoint{1.083924in}{1.787622in}}%
\pgfpathlineto{\pgfqpoint{1.084733in}{1.036654in}}%
\pgfpathlineto{\pgfqpoint{1.085973in}{2.064104in}}%
\pgfpathlineto{\pgfqpoint{1.086647in}{1.115670in}}%
\pgfpathlineto{\pgfqpoint{1.087312in}{1.709034in}}%
\pgfpathlineto{\pgfqpoint{1.088361in}{0.922502in}}%
\pgfpathlineto{\pgfqpoint{1.089230in}{1.784945in}}%
\pgfpathlineto{\pgfqpoint{1.090041in}{1.039070in}}%
\pgfpathlineto{\pgfqpoint{1.091016in}{1.866615in}}%
\pgfpathlineto{\pgfqpoint{1.092027in}{1.096087in}}%
\pgfpathlineto{\pgfqpoint{1.092378in}{1.912510in}}%
\pgfpathlineto{\pgfqpoint{1.093424in}{1.039851in}}%
\pgfpathlineto{\pgfqpoint{1.094036in}{1.906331in}}%
\pgfpathlineto{\pgfqpoint{1.095141in}{1.222368in}}%
\pgfpathlineto{\pgfqpoint{1.095782in}{1.992712in}}%
\pgfpathlineto{\pgfqpoint{1.096587in}{1.075009in}}%
\pgfpathlineto{\pgfqpoint{1.097505in}{1.759200in}}%
\pgfpathlineto{\pgfqpoint{1.098525in}{1.007573in}}%
\pgfpathlineto{\pgfqpoint{1.099192in}{1.949337in}}%
\pgfpathlineto{\pgfqpoint{1.100291in}{1.104176in}}%
\pgfpathlineto{\pgfqpoint{1.100927in}{1.824060in}}%
\pgfpathlineto{\pgfqpoint{1.101910in}{1.114734in}}%
\pgfpathlineto{\pgfqpoint{1.102851in}{1.869613in}}%
\pgfpathlineto{\pgfqpoint{1.103375in}{1.050259in}}%
\pgfpathlineto{\pgfqpoint{1.104406in}{1.936097in}}%
\pgfpathlineto{\pgfqpoint{1.105274in}{1.153375in}}%
\pgfpathlineto{\pgfqpoint{1.105967in}{1.864720in}}%
\pgfpathlineto{\pgfqpoint{1.107000in}{1.149085in}}%
\pgfpathlineto{\pgfqpoint{1.107627in}{1.904324in}}%
\pgfpathlineto{\pgfqpoint{1.108587in}{1.071541in}}%
\pgfpathlineto{\pgfqpoint{1.109435in}{1.703684in}}%
\pgfpathlineto{\pgfqpoint{1.110276in}{0.895537in}}%
\pgfpathlineto{\pgfqpoint{1.111613in}{1.885798in}}%
\pgfpathlineto{\pgfqpoint{1.111953in}{0.979029in}}%
\pgfpathlineto{\pgfqpoint{1.113484in}{1.887602in}}%
\pgfpathlineto{\pgfqpoint{1.113934in}{1.101967in}}%
\pgfpathlineto{\pgfqpoint{1.114789in}{1.798718in}}%
\pgfpathlineto{\pgfqpoint{1.115664in}{1.001108in}}%
\pgfpathlineto{\pgfqpoint{1.116308in}{1.836872in}}%
\pgfpathlineto{\pgfqpoint{1.117244in}{0.978511in}}%
\pgfpathlineto{\pgfqpoint{1.117832in}{1.766684in}}%
\pgfpathlineto{\pgfqpoint{1.118684in}{1.091900in}}%
\pgfpathlineto{\pgfqpoint{1.119865in}{1.744420in}}%
\pgfpathlineto{\pgfqpoint{1.120374in}{0.978901in}}%
\pgfpathlineto{\pgfqpoint{1.121734in}{1.901795in}}%
\pgfpathlineto{\pgfqpoint{1.122485in}{1.067534in}}%
\pgfpathlineto{\pgfqpoint{1.123016in}{1.901377in}}%
\pgfpathlineto{\pgfqpoint{1.123920in}{1.108673in}}%
\pgfpathlineto{\pgfqpoint{1.124912in}{1.773052in}}%
\pgfpathlineto{\pgfqpoint{1.125898in}{0.876613in}}%
\pgfpathlineto{\pgfqpoint{1.126319in}{1.826735in}}%
\pgfpathlineto{\pgfqpoint{1.127434in}{1.042971in}}%
\pgfpathlineto{\pgfqpoint{1.128206in}{1.881997in}}%
\pgfpathlineto{\pgfqpoint{1.129510in}{0.908563in}}%
\pgfpathlineto{\pgfqpoint{1.129873in}{1.699081in}}%
\pgfpathlineto{\pgfqpoint{1.130791in}{0.989125in}}%
\pgfpathlineto{\pgfqpoint{1.131487in}{1.767144in}}%
\pgfpathlineto{\pgfqpoint{1.132274in}{1.012675in}}%
\pgfpathlineto{\pgfqpoint{1.133210in}{1.815281in}}%
\pgfpathlineto{\pgfqpoint{1.134307in}{1.042890in}}%
\pgfpathlineto{\pgfqpoint{1.134826in}{1.772231in}}%
\pgfpathlineto{\pgfqpoint{1.135922in}{1.011259in}}%
\pgfpathlineto{\pgfqpoint{1.136545in}{1.841432in}}%
\pgfpathlineto{\pgfqpoint{1.137401in}{1.089839in}}%
\pgfpathlineto{\pgfqpoint{1.138652in}{1.848010in}}%
\pgfpathlineto{\pgfqpoint{1.139205in}{1.123362in}}%
\pgfpathlineto{\pgfqpoint{1.140222in}{1.931629in}}%
\pgfpathlineto{\pgfqpoint{1.141152in}{1.148100in}}%
\pgfpathlineto{\pgfqpoint{1.141781in}{1.921477in}}%
\pgfpathlineto{\pgfqpoint{1.142984in}{0.884330in}}%
\pgfpathlineto{\pgfqpoint{1.143373in}{1.705849in}}%
\pgfpathlineto{\pgfqpoint{1.144169in}{1.063133in}}%
\pgfpathlineto{\pgfqpoint{1.145015in}{1.859308in}}%
\pgfpathlineto{\pgfqpoint{1.146067in}{1.223191in}}%
\pgfpathlineto{\pgfqpoint{1.146859in}{1.964628in}}%
\pgfpathlineto{\pgfqpoint{1.147921in}{1.203116in}}%
\pgfpathlineto{\pgfqpoint{1.148421in}{1.916873in}}%
\pgfpathlineto{\pgfqpoint{1.149522in}{1.092435in}}%
\pgfpathlineto{\pgfqpoint{1.150164in}{1.998968in}}%
\pgfpathlineto{\pgfqpoint{1.151053in}{1.218499in}}%
\pgfpathlineto{\pgfqpoint{1.151846in}{2.031168in}}%
\pgfpathlineto{\pgfqpoint{1.152791in}{1.217094in}}%
\pgfpathlineto{\pgfqpoint{1.153517in}{1.866887in}}%
\pgfpathlineto{\pgfqpoint{1.154455in}{1.155977in}}%
\pgfpathlineto{\pgfqpoint{1.155636in}{2.027622in}}%
\pgfpathlineto{\pgfqpoint{1.156474in}{1.183812in}}%
\pgfpathlineto{\pgfqpoint{1.157002in}{1.916365in}}%
\pgfpathlineto{\pgfqpoint{1.157876in}{1.059193in}}%
\pgfpathlineto{\pgfqpoint{1.158701in}{1.976333in}}%
\pgfpathlineto{\pgfqpoint{1.159477in}{1.160503in}}%
\pgfpathlineto{\pgfqpoint{1.160510in}{1.872921in}}%
\pgfpathlineto{\pgfqpoint{1.161296in}{1.091020in}}%
\pgfpathlineto{\pgfqpoint{1.162241in}{1.940546in}}%
\pgfpathlineto{\pgfqpoint{1.162911in}{1.180884in}}%
\pgfpathlineto{\pgfqpoint{1.163764in}{1.903630in}}%
\pgfpathlineto{\pgfqpoint{1.164665in}{1.240864in}}%
\pgfpathlineto{\pgfqpoint{1.165673in}{1.871420in}}%
\pgfpathlineto{\pgfqpoint{1.166273in}{1.258497in}}%
\pgfpathlineto{\pgfqpoint{1.167378in}{1.923428in}}%
\pgfpathlineto{\pgfqpoint{1.168163in}{1.098329in}}%
\pgfpathlineto{\pgfqpoint{1.169109in}{2.048385in}}%
\pgfpathlineto{\pgfqpoint{1.169748in}{1.157377in}}%
\pgfpathlineto{\pgfqpoint{1.170521in}{1.817185in}}%
\pgfpathlineto{\pgfqpoint{1.171711in}{1.021144in}}%
\pgfpathlineto{\pgfqpoint{1.172264in}{1.853467in}}%
\pgfpathlineto{\pgfqpoint{1.173133in}{1.137839in}}%
\pgfpathlineto{\pgfqpoint{1.174316in}{1.868744in}}%
\pgfpathlineto{\pgfqpoint{1.175271in}{0.970112in}}%
\pgfpathlineto{\pgfqpoint{1.175609in}{1.800873in}}%
\pgfpathlineto{\pgfqpoint{1.176472in}{1.089074in}}%
\pgfpathlineto{\pgfqpoint{1.177891in}{1.878269in}}%
\pgfpathlineto{\pgfqpoint{1.178396in}{1.004742in}}%
\pgfpathlineto{\pgfqpoint{1.179136in}{1.781457in}}%
\pgfpathlineto{\pgfqpoint{1.179887in}{1.106206in}}%
\pgfpathlineto{\pgfqpoint{1.181114in}{1.858975in}}%
\pgfpathlineto{\pgfqpoint{1.181556in}{1.113700in}}%
\pgfpathlineto{\pgfqpoint{1.182892in}{1.905604in}}%
\pgfpathlineto{\pgfqpoint{1.183462in}{1.005890in}}%
\pgfpathlineto{\pgfqpoint{1.184140in}{1.694126in}}%
\pgfpathlineto{\pgfqpoint{1.185433in}{0.932039in}}%
\pgfpathlineto{\pgfqpoint{1.186006in}{1.669316in}}%
\pgfpathlineto{\pgfqpoint{1.186884in}{0.905834in}}%
\pgfpathlineto{\pgfqpoint{1.187609in}{1.728854in}}%
\pgfpathlineto{\pgfqpoint{1.188519in}{0.902894in}}%
\pgfpathlineto{\pgfqpoint{1.189209in}{1.773782in}}%
\pgfpathlineto{\pgfqpoint{1.190521in}{0.933349in}}%
\pgfpathlineto{\pgfqpoint{1.191042in}{1.742158in}}%
\pgfpathlineto{\pgfqpoint{1.191976in}{1.168236in}}%
\pgfpathlineto{\pgfqpoint{1.192629in}{1.772567in}}%
\pgfpathlineto{\pgfqpoint{1.193457in}{1.116208in}}%
\pgfpathlineto{\pgfqpoint{1.194530in}{1.795905in}}%
\pgfpathlineto{\pgfqpoint{1.195522in}{0.986082in}}%
\pgfpathlineto{\pgfqpoint{1.196090in}{1.768558in}}%
\pgfpathlineto{\pgfqpoint{1.197109in}{0.971948in}}%
\pgfpathlineto{\pgfqpoint{1.197743in}{1.802753in}}%
\pgfpathlineto{\pgfqpoint{1.198554in}{1.093467in}}%
\pgfpathlineto{\pgfqpoint{1.199825in}{1.806895in}}%
\pgfpathlineto{\pgfqpoint{1.200357in}{1.000702in}}%
\pgfpathlineto{\pgfqpoint{1.201292in}{1.695095in}}%
\pgfpathlineto{\pgfqpoint{1.201952in}{1.038874in}}%
\pgfpathlineto{\pgfqpoint{1.202808in}{1.930222in}}%
\pgfpathlineto{\pgfqpoint{1.204466in}{1.142998in}}%
\pgfpathlineto{\pgfqpoint{1.204518in}{1.879874in}}%
\pgfpathlineto{\pgfqpoint{1.205530in}{1.187635in}}%
\pgfpathlineto{\pgfqpoint{1.206856in}{1.955845in}}%
\pgfpathlineto{\pgfqpoint{1.207113in}{1.226245in}}%
\pgfpathlineto{\pgfqpoint{1.207946in}{1.874934in}}%
\pgfpathlineto{\pgfqpoint{1.208869in}{1.216848in}}%
\pgfpathlineto{\pgfqpoint{1.210147in}{2.053347in}}%
\pgfpathlineto{\pgfqpoint{1.210530in}{1.095896in}}%
\pgfpathlineto{\pgfqpoint{1.211343in}{1.940748in}}%
\pgfpathlineto{\pgfqpoint{1.212523in}{0.967466in}}%
\pgfpathlineto{\pgfqpoint{1.213116in}{1.697995in}}%
\pgfpathlineto{\pgfqpoint{1.214482in}{1.937737in}}%
\pgfpathlineto{\pgfqpoint{1.214735in}{1.109936in}}%
\pgfpathlineto{\pgfqpoint{1.215904in}{1.902931in}}%
\pgfpathlineto{\pgfqpoint{1.216423in}{1.155309in}}%
\pgfpathlineto{\pgfqpoint{1.217448in}{1.767300in}}%
\pgfpathlineto{\pgfqpoint{1.218289in}{1.003821in}}%
\pgfpathlineto{\pgfqpoint{1.219278in}{1.756962in}}%
\pgfpathlineto{\pgfqpoint{1.220309in}{1.104343in}}%
\pgfpathlineto{\pgfqpoint{1.221439in}{2.054883in}}%
\pgfpathlineto{\pgfqpoint{1.221853in}{1.091731in}}%
\pgfpathlineto{\pgfqpoint{1.222434in}{1.909667in}}%
\pgfpathlineto{\pgfqpoint{1.223347in}{1.039534in}}%
\pgfpathlineto{\pgfqpoint{1.224234in}{1.879193in}}%
\pgfpathlineto{\pgfqpoint{1.224975in}{1.211304in}}%
\pgfpathlineto{\pgfqpoint{1.225862in}{1.928544in}}%
\pgfpathlineto{\pgfqpoint{1.227401in}{1.128311in}}%
\pgfpathlineto{\pgfqpoint{1.227916in}{2.033834in}}%
\pgfpathlineto{\pgfqpoint{1.228349in}{1.245279in}}%
\pgfpathlineto{\pgfqpoint{1.229673in}{1.891870in}}%
\pgfpathlineto{\pgfqpoint{1.230322in}{1.013395in}}%
\pgfpathlineto{\pgfqpoint{1.231069in}{1.965357in}}%
\pgfpathlineto{\pgfqpoint{1.231766in}{1.108149in}}%
\pgfpathlineto{\pgfqpoint{1.232919in}{1.852738in}}%
\pgfpathlineto{\pgfqpoint{1.233628in}{1.150219in}}%
\pgfpathlineto{\pgfqpoint{1.234562in}{1.876277in}}%
\pgfpathlineto{\pgfqpoint{1.235329in}{1.109403in}}%
\pgfpathlineto{\pgfqpoint{1.236667in}{2.038995in}}%
\pgfpathlineto{\pgfqpoint{1.237116in}{1.256541in}}%
\pgfpathlineto{\pgfqpoint{1.237810in}{2.000576in}}%
\pgfpathlineto{\pgfqpoint{1.238726in}{1.245101in}}%
\pgfpathlineto{\pgfqpoint{1.239607in}{1.975099in}}%
\pgfpathlineto{\pgfqpoint{1.240320in}{1.203905in}}%
\pgfpathlineto{\pgfqpoint{1.241134in}{1.924714in}}%
\pgfpathlineto{\pgfqpoint{1.242195in}{1.173463in}}%
\pgfpathlineto{\pgfqpoint{1.242748in}{2.002533in}}%
\pgfpathlineto{\pgfqpoint{1.243750in}{1.210997in}}%
\pgfpathlineto{\pgfqpoint{1.244530in}{1.916751in}}%
\pgfpathlineto{\pgfqpoint{1.245353in}{1.249993in}}%
\pgfpathlineto{\pgfqpoint{1.246316in}{1.954637in}}%
\pgfpathlineto{\pgfqpoint{1.247617in}{1.166601in}}%
\pgfpathlineto{\pgfqpoint{1.248218in}{2.047440in}}%
\pgfpathlineto{\pgfqpoint{1.248823in}{1.168467in}}%
\pgfpathlineto{\pgfqpoint{1.249835in}{1.981646in}}%
\pgfpathlineto{\pgfqpoint{1.250519in}{1.079472in}}%
\pgfpathlineto{\pgfqpoint{1.251288in}{1.888946in}}%
\pgfpathlineto{\pgfqpoint{1.252200in}{1.238716in}}%
\pgfpathlineto{\pgfqpoint{1.253085in}{1.958454in}}%
\pgfpathlineto{\pgfqpoint{1.254090in}{1.202377in}}%
\pgfpathlineto{\pgfqpoint{1.254748in}{1.949699in}}%
\pgfpathlineto{\pgfqpoint{1.256120in}{1.115200in}}%
\pgfpathlineto{\pgfqpoint{1.256755in}{2.074119in}}%
\pgfpathlineto{\pgfqpoint{1.257192in}{1.152322in}}%
\pgfpathlineto{\pgfqpoint{1.258038in}{1.962448in}}%
\pgfpathlineto{\pgfqpoint{1.259019in}{1.111363in}}%
\pgfpathlineto{\pgfqpoint{1.260409in}{2.057945in}}%
\pgfpathlineto{\pgfqpoint{1.260743in}{1.176541in}}%
\pgfpathlineto{\pgfqpoint{1.261504in}{1.912938in}}%
\pgfpathlineto{\pgfqpoint{1.262329in}{1.154280in}}%
\pgfpathlineto{\pgfqpoint{1.263580in}{1.897443in}}%
\pgfpathlineto{\pgfqpoint{1.264138in}{0.993707in}}%
\pgfpathlineto{\pgfqpoint{1.264922in}{1.875239in}}%
\pgfpathlineto{\pgfqpoint{1.266312in}{0.969421in}}%
\pgfpathlineto{\pgfqpoint{1.266606in}{1.848495in}}%
\pgfpathlineto{\pgfqpoint{1.267404in}{1.136512in}}%
\pgfpathlineto{\pgfqpoint{1.268250in}{1.762616in}}%
\pgfpathlineto{\pgfqpoint{1.269408in}{1.011949in}}%
\pgfpathlineto{\pgfqpoint{1.269982in}{1.889160in}}%
\pgfpathlineto{\pgfqpoint{1.270824in}{1.065207in}}%
\pgfpathlineto{\pgfqpoint{1.271850in}{1.867406in}}%
\pgfpathlineto{\pgfqpoint{1.272555in}{1.162292in}}%
\pgfpathlineto{\pgfqpoint{1.273605in}{1.864295in}}%
\pgfpathlineto{\pgfqpoint{1.274199in}{0.963743in}}%
\pgfpathlineto{\pgfqpoint{1.275079in}{1.736438in}}%
\pgfpathlineto{\pgfqpoint{1.276155in}{1.012789in}}%
\pgfpathlineto{\pgfqpoint{1.276825in}{1.700726in}}%
\pgfpathlineto{\pgfqpoint{1.277619in}{0.867882in}}%
\pgfpathlineto{\pgfqpoint{1.278897in}{1.739185in}}%
\pgfpathlineto{\pgfqpoint{1.279303in}{0.930489in}}%
\pgfpathlineto{\pgfqpoint{1.280214in}{1.742989in}}%
\pgfpathlineto{\pgfqpoint{1.281245in}{0.965874in}}%
\pgfpathlineto{\pgfqpoint{1.281834in}{1.590709in}}%
\pgfpathlineto{\pgfqpoint{1.282705in}{1.053107in}}%
\pgfpathlineto{\pgfqpoint{1.284000in}{1.806425in}}%
\pgfpathlineto{\pgfqpoint{1.284463in}{0.935460in}}%
\pgfpathlineto{\pgfqpoint{1.285328in}{1.731231in}}%
\pgfpathlineto{\pgfqpoint{1.286074in}{1.061914in}}%
\pgfpathlineto{\pgfqpoint{1.287212in}{1.795053in}}%
\pgfpathlineto{\pgfqpoint{1.287785in}{1.183291in}}%
\pgfpathlineto{\pgfqpoint{1.289007in}{1.942206in}}%
\pgfpathlineto{\pgfqpoint{1.289544in}{1.162767in}}%
\pgfpathlineto{\pgfqpoint{1.290381in}{1.775456in}}%
\pgfpathlineto{\pgfqpoint{1.291319in}{0.978131in}}%
\pgfpathlineto{\pgfqpoint{1.292132in}{1.862679in}}%
\pgfpathlineto{\pgfqpoint{1.293077in}{1.163543in}}%
\pgfpathlineto{\pgfqpoint{1.293743in}{1.854524in}}%
\pgfpathlineto{\pgfqpoint{1.294626in}{1.039500in}}%
\pgfpathlineto{\pgfqpoint{1.295650in}{1.768505in}}%
\pgfpathlineto{\pgfqpoint{1.296570in}{0.899835in}}%
\pgfpathlineto{\pgfqpoint{1.297413in}{1.963413in}}%
\pgfpathlineto{\pgfqpoint{1.298011in}{0.983215in}}%
\pgfpathlineto{\pgfqpoint{1.298999in}{1.901536in}}%
\pgfpathlineto{\pgfqpoint{1.299702in}{1.008398in}}%
\pgfpathlineto{\pgfqpoint{1.301156in}{0.876766in}}%
\pgfpathlineto{\pgfqpoint{1.301548in}{1.839817in}}%
\pgfpathlineto{\pgfqpoint{1.302250in}{1.080157in}}%
\pgfpathlineto{\pgfqpoint{1.303077in}{1.761206in}}%
\pgfpathlineto{\pgfqpoint{1.304727in}{0.971472in}}%
\pgfpathlineto{\pgfqpoint{1.304841in}{1.732302in}}%
\pgfpathlineto{\pgfqpoint{1.305816in}{0.982217in}}%
\pgfpathlineto{\pgfqpoint{1.306472in}{1.865965in}}%
\pgfpathlineto{\pgfqpoint{1.307590in}{0.963626in}}%
\pgfpathlineto{\pgfqpoint{1.308243in}{1.727210in}}%
\pgfpathlineto{\pgfqpoint{1.309024in}{1.013797in}}%
\pgfpathlineto{\pgfqpoint{1.309919in}{1.891844in}}%
\pgfpathlineto{\pgfqpoint{1.310962in}{1.057229in}}%
\pgfpathlineto{\pgfqpoint{1.311687in}{1.805506in}}%
\pgfpathlineto{\pgfqpoint{1.312461in}{1.102079in}}%
\pgfpathlineto{\pgfqpoint{1.313582in}{1.922898in}}%
\pgfpathlineto{\pgfqpoint{1.314161in}{1.067162in}}%
\pgfpathlineto{\pgfqpoint{1.315340in}{1.892444in}}%
\pgfpathlineto{\pgfqpoint{1.315916in}{1.179191in}}%
\pgfpathlineto{\pgfqpoint{1.317087in}{1.900528in}}%
\pgfpathlineto{\pgfqpoint{1.317553in}{1.121359in}}%
\pgfpathlineto{\pgfqpoint{1.318738in}{1.957137in}}%
\pgfpathlineto{\pgfqpoint{1.319520in}{1.194964in}}%
\pgfpathlineto{\pgfqpoint{1.320144in}{1.957453in}}%
\pgfpathlineto{\pgfqpoint{1.321499in}{1.074310in}}%
\pgfpathlineto{\pgfqpoint{1.321999in}{1.819980in}}%
\pgfpathlineto{\pgfqpoint{1.322729in}{1.069407in}}%
\pgfpathlineto{\pgfqpoint{1.323789in}{1.901340in}}%
\pgfpathlineto{\pgfqpoint{1.324328in}{1.131498in}}%
\pgfpathlineto{\pgfqpoint{1.325297in}{1.847770in}}%
\pgfpathlineto{\pgfqpoint{1.326030in}{0.956398in}}%
\pgfpathlineto{\pgfqpoint{1.326994in}{1.813551in}}%
\pgfpathlineto{\pgfqpoint{1.327713in}{1.012683in}}%
\pgfpathlineto{\pgfqpoint{1.328576in}{1.810927in}}%
\pgfpathlineto{\pgfqpoint{1.329830in}{0.988222in}}%
\pgfpathlineto{\pgfqpoint{1.330288in}{1.767800in}}%
\pgfpathlineto{\pgfqpoint{1.331361in}{0.997457in}}%
\pgfpathlineto{\pgfqpoint{1.332204in}{1.825025in}}%
\pgfpathlineto{\pgfqpoint{1.332812in}{1.007857in}}%
\pgfpathlineto{\pgfqpoint{1.333828in}{1.825128in}}%
\pgfpathlineto{\pgfqpoint{1.334673in}{1.134619in}}%
\pgfpathlineto{\pgfqpoint{1.335571in}{1.842198in}}%
\pgfpathlineto{\pgfqpoint{1.336561in}{1.054197in}}%
\pgfpathlineto{\pgfqpoint{1.337084in}{1.840816in}}%
\pgfpathlineto{\pgfqpoint{1.337988in}{1.171876in}}%
\pgfpathlineto{\pgfqpoint{1.339172in}{1.803850in}}%
\pgfpathlineto{\pgfqpoint{1.340000in}{0.923545in}}%
\pgfpathlineto{\pgfqpoint{1.340498in}{1.815377in}}%
\pgfpathlineto{\pgfqpoint{1.341983in}{1.043248in}}%
\pgfpathlineto{\pgfqpoint{1.342180in}{1.842015in}}%
\pgfpathlineto{\pgfqpoint{1.343019in}{1.093087in}}%
\pgfpathlineto{\pgfqpoint{1.344133in}{1.947557in}}%
\pgfpathlineto{\pgfqpoint{1.345079in}{1.048818in}}%
\pgfpathlineto{\pgfqpoint{1.345658in}{1.781535in}}%
\pgfpathlineto{\pgfqpoint{1.346508in}{1.156578in}}%
\pgfpathlineto{\pgfqpoint{1.347318in}{1.755962in}}%
\pgfpathlineto{\pgfqpoint{1.348160in}{1.068760in}}%
\pgfpathlineto{\pgfqpoint{1.349003in}{1.692701in}}%
\pgfpathlineto{\pgfqpoint{1.349945in}{1.110336in}}%
\pgfpathlineto{\pgfqpoint{1.350715in}{1.817551in}}%
\pgfpathlineto{\pgfqpoint{1.352037in}{0.994010in}}%
\pgfpathlineto{\pgfqpoint{1.352766in}{1.852411in}}%
\pgfpathlineto{\pgfqpoint{1.353527in}{1.156481in}}%
\pgfpathlineto{\pgfqpoint{1.354706in}{1.875894in}}%
\pgfpathlineto{\pgfqpoint{1.354913in}{1.129031in}}%
\pgfpathlineto{\pgfqpoint{1.355836in}{1.939831in}}%
\pgfpathlineto{\pgfqpoint{1.356901in}{1.065759in}}%
\pgfpathlineto{\pgfqpoint{1.357494in}{1.888568in}}%
\pgfpathlineto{\pgfqpoint{1.358993in}{1.015916in}}%
\pgfpathlineto{\pgfqpoint{1.360859in}{1.893450in}}%
\pgfpathlineto{\pgfqpoint{1.362328in}{1.006588in}}%
\pgfpathlineto{\pgfqpoint{1.362961in}{1.832989in}}%
\pgfpathlineto{\pgfqpoint{1.363429in}{1.017855in}}%
\pgfpathlineto{\pgfqpoint{1.364777in}{1.879339in}}%
\pgfpathlineto{\pgfqpoint{1.365501in}{1.072783in}}%
\pgfpathlineto{\pgfqpoint{1.366215in}{1.890865in}}%
\pgfpathlineto{\pgfqpoint{1.367460in}{1.090133in}}%
\pgfpathlineto{\pgfqpoint{1.367715in}{1.784040in}}%
\pgfpathlineto{\pgfqpoint{1.368505in}{1.098430in}}%
\pgfpathlineto{\pgfqpoint{1.369446in}{1.867928in}}%
\pgfpathlineto{\pgfqpoint{1.370259in}{1.009900in}}%
\pgfpathlineto{\pgfqpoint{1.371263in}{1.786368in}}%
\pgfpathlineto{\pgfqpoint{1.373174in}{0.798598in}}%
\pgfpathlineto{\pgfqpoint{1.373637in}{1.715140in}}%
\pgfpathlineto{\pgfqpoint{1.374552in}{1.020555in}}%
\pgfpathlineto{\pgfqpoint{1.375702in}{1.849044in}}%
\pgfpathlineto{\pgfqpoint{1.376179in}{0.934482in}}%
\pgfpathlineto{\pgfqpoint{1.377024in}{1.561678in}}%
\pgfpathlineto{\pgfqpoint{1.377862in}{0.767867in}}%
\pgfpathlineto{\pgfqpoint{1.379001in}{1.664326in}}%
\pgfpathlineto{\pgfqpoint{1.379666in}{0.823162in}}%
\pgfpathlineto{\pgfqpoint{1.380399in}{1.543296in}}%
\pgfpathlineto{\pgfqpoint{1.381480in}{0.962276in}}%
\pgfpathlineto{\pgfqpoint{1.382248in}{1.727112in}}%
\pgfpathlineto{\pgfqpoint{1.383118in}{0.960233in}}%
\pgfpathlineto{\pgfqpoint{1.383957in}{1.636958in}}%
\pgfpathlineto{\pgfqpoint{1.385128in}{0.816355in}}%
\pgfpathlineto{\pgfqpoint{1.385571in}{1.702466in}}%
\pgfpathlineto{\pgfqpoint{1.386643in}{0.845261in}}%
\pgfpathlineto{\pgfqpoint{1.387662in}{1.731889in}}%
\pgfpathlineto{\pgfqpoint{1.388057in}{1.055667in}}%
\pgfpathlineto{\pgfqpoint{1.389158in}{1.874749in}}%
\pgfpathlineto{\pgfqpoint{1.390043in}{0.955782in}}%
\pgfpathlineto{\pgfqpoint{1.390802in}{1.806133in}}%
\pgfpathlineto{\pgfqpoint{1.391597in}{1.052238in}}%
\pgfpathlineto{\pgfqpoint{1.392563in}{1.918629in}}%
\pgfpathlineto{\pgfqpoint{1.393332in}{0.948490in}}%
\pgfpathlineto{\pgfqpoint{1.394252in}{1.823393in}}%
\pgfpathlineto{\pgfqpoint{1.394864in}{1.024684in}}%
\pgfpathlineto{\pgfqpoint{1.395940in}{1.852493in}}%
\pgfpathlineto{\pgfqpoint{1.397185in}{0.885907in}}%
\pgfpathlineto{\pgfqpoint{1.397554in}{1.660624in}}%
\pgfpathlineto{\pgfqpoint{1.398255in}{0.976511in}}%
\pgfpathlineto{\pgfqpoint{1.399254in}{1.845231in}}%
\pgfpathlineto{\pgfqpoint{1.400041in}{1.145746in}}%
\pgfpathlineto{\pgfqpoint{1.400907in}{1.811215in}}%
\pgfpathlineto{\pgfqpoint{1.401958in}{1.089896in}}%
\pgfpathlineto{\pgfqpoint{1.402517in}{1.808765in}}%
\pgfpathlineto{\pgfqpoint{1.403354in}{1.101008in}}%
\pgfpathlineto{\pgfqpoint{1.404989in}{1.811484in}}%
\pgfpathlineto{\pgfqpoint{1.405646in}{0.824005in}}%
\pgfpathlineto{\pgfqpoint{1.405986in}{1.788096in}}%
\pgfpathlineto{\pgfqpoint{1.406751in}{1.019216in}}%
\pgfpathlineto{\pgfqpoint{1.407653in}{1.845621in}}%
\pgfpathlineto{\pgfqpoint{1.408455in}{1.150163in}}%
\pgfpathlineto{\pgfqpoint{1.409569in}{1.806669in}}%
\pgfpathlineto{\pgfqpoint{1.410142in}{1.082087in}}%
\pgfpathlineto{\pgfqpoint{1.411439in}{1.887366in}}%
\pgfpathlineto{\pgfqpoint{1.411851in}{1.110627in}}%
\pgfpathlineto{\pgfqpoint{1.412867in}{1.835360in}}%
\pgfpathlineto{\pgfqpoint{1.414075in}{1.105617in}}%
\pgfpathlineto{\pgfqpoint{1.414400in}{2.013860in}}%
\pgfpathlineto{\pgfqpoint{1.415324in}{1.240007in}}%
\pgfpathlineto{\pgfqpoint{1.416808in}{1.858692in}}%
\pgfpathlineto{\pgfqpoint{1.417184in}{1.055290in}}%
\pgfpathlineto{\pgfqpoint{1.418572in}{1.887424in}}%
\pgfpathlineto{\pgfqpoint{1.418843in}{1.044600in}}%
\pgfpathlineto{\pgfqpoint{1.419637in}{1.835206in}}%
\pgfpathlineto{\pgfqpoint{1.420459in}{0.980368in}}%
\pgfpathlineto{\pgfqpoint{1.421516in}{1.822461in}}%
\pgfpathlineto{\pgfqpoint{1.422763in}{0.981300in}}%
\pgfpathlineto{\pgfqpoint{1.422911in}{1.771959in}}%
\pgfpathlineto{\pgfqpoint{1.423803in}{1.050494in}}%
\pgfpathlineto{\pgfqpoint{1.424850in}{1.833017in}}%
\pgfpathlineto{\pgfqpoint{1.425765in}{1.084848in}}%
\pgfpathlineto{\pgfqpoint{1.426368in}{1.734638in}}%
\pgfpathlineto{\pgfqpoint{1.427181in}{1.039737in}}%
\pgfpathlineto{\pgfqpoint{1.428488in}{1.771951in}}%
\pgfpathlineto{\pgfqpoint{1.428915in}{0.948547in}}%
\pgfpathlineto{\pgfqpoint{1.429945in}{1.777438in}}%
\pgfpathlineto{\pgfqpoint{1.430809in}{1.014462in}}%
\pgfpathlineto{\pgfqpoint{1.431581in}{1.729500in}}%
\pgfpathlineto{\pgfqpoint{1.432261in}{0.915924in}}%
\pgfpathlineto{\pgfqpoint{1.433119in}{1.779759in}}%
\pgfpathlineto{\pgfqpoint{1.434003in}{0.941569in}}%
\pgfpathlineto{\pgfqpoint{1.434887in}{1.694851in}}%
\pgfpathlineto{\pgfqpoint{1.435866in}{0.979688in}}%
\pgfpathlineto{\pgfqpoint{1.436721in}{1.866825in}}%
\pgfpathlineto{\pgfqpoint{1.437359in}{0.891867in}}%
\pgfpathlineto{\pgfqpoint{1.438292in}{1.833130in}}%
\pgfpathlineto{\pgfqpoint{1.439195in}{0.999713in}}%
\pgfpathlineto{\pgfqpoint{1.440337in}{1.913141in}}%
\pgfpathlineto{\pgfqpoint{1.440916in}{1.090719in}}%
\pgfpathlineto{\pgfqpoint{1.441654in}{1.869336in}}%
\pgfpathlineto{\pgfqpoint{1.442608in}{1.032115in}}%
\pgfpathlineto{\pgfqpoint{1.443523in}{1.905820in}}%
\pgfpathlineto{\pgfqpoint{1.444524in}{0.887793in}}%
\pgfpathlineto{\pgfqpoint{1.445121in}{1.728438in}}%
\pgfpathlineto{\pgfqpoint{1.446326in}{0.984124in}}%
\pgfpathlineto{\pgfqpoint{1.447118in}{1.796409in}}%
\pgfpathlineto{\pgfqpoint{1.447805in}{0.896470in}}%
\pgfpathlineto{\pgfqpoint{1.448389in}{1.753330in}}%
\pgfpathlineto{\pgfqpoint{1.449280in}{0.991490in}}%
\pgfpathlineto{\pgfqpoint{1.450146in}{1.693241in}}%
\pgfpathlineto{\pgfqpoint{1.451175in}{0.816516in}}%
\pgfpathlineto{\pgfqpoint{1.451941in}{1.779568in}}%
\pgfpathlineto{\pgfqpoint{1.452740in}{0.954949in}}%
\pgfpathlineto{\pgfqpoint{1.454095in}{1.842673in}}%
\pgfpathlineto{\pgfqpoint{1.454524in}{0.957918in}}%
\pgfpathlineto{\pgfqpoint{1.455207in}{1.711811in}}%
\pgfpathlineto{\pgfqpoint{1.456059in}{1.003504in}}%
\pgfpathlineto{\pgfqpoint{1.456955in}{1.797809in}}%
\pgfpathlineto{\pgfqpoint{1.457750in}{0.936873in}}%
\pgfpathlineto{\pgfqpoint{1.458595in}{1.593389in}}%
\pgfpathlineto{\pgfqpoint{1.459681in}{0.958086in}}%
\pgfpathlineto{\pgfqpoint{1.460779in}{1.791164in}}%
\pgfpathlineto{\pgfqpoint{1.461125in}{1.114511in}}%
\pgfpathlineto{\pgfqpoint{1.462203in}{1.661128in}}%
\pgfpathlineto{\pgfqpoint{1.462979in}{0.995446in}}%
\pgfpathlineto{\pgfqpoint{1.463767in}{1.594824in}}%
\pgfpathlineto{\pgfqpoint{1.464683in}{0.932331in}}%
\pgfpathlineto{\pgfqpoint{1.465491in}{1.754598in}}%
\pgfpathlineto{\pgfqpoint{1.466247in}{1.115722in}}%
\pgfpathlineto{\pgfqpoint{1.467084in}{1.773735in}}%
\pgfpathlineto{\pgfqpoint{1.468057in}{1.003674in}}%
\pgfpathlineto{\pgfqpoint{1.468824in}{1.713653in}}%
\pgfpathlineto{\pgfqpoint{1.469694in}{0.974888in}}%
\pgfpathlineto{\pgfqpoint{1.470871in}{1.688856in}}%
\pgfpathlineto{\pgfqpoint{1.471460in}{0.955673in}}%
\pgfpathlineto{\pgfqpoint{1.472864in}{1.767625in}}%
\pgfpathlineto{\pgfqpoint{1.473148in}{0.951069in}}%
\pgfpathlineto{\pgfqpoint{1.474196in}{1.772682in}}%
\pgfpathlineto{\pgfqpoint{1.474999in}{1.014336in}}%
\pgfpathlineto{\pgfqpoint{1.475631in}{1.674842in}}%
\pgfpathlineto{\pgfqpoint{1.476454in}{0.994908in}}%
\pgfpathlineto{\pgfqpoint{1.477630in}{1.760262in}}%
\pgfpathlineto{\pgfqpoint{1.478268in}{0.871380in}}%
\pgfpathlineto{\pgfqpoint{1.478977in}{1.672563in}}%
\pgfpathlineto{\pgfqpoint{1.479976in}{0.848541in}}%
\pgfpathlineto{\pgfqpoint{1.480680in}{1.629696in}}%
\pgfpathlineto{\pgfqpoint{1.481945in}{0.829913in}}%
\pgfpathlineto{\pgfqpoint{1.482489in}{1.683131in}}%
\pgfpathlineto{\pgfqpoint{1.483219in}{0.992635in}}%
\pgfpathlineto{\pgfqpoint{1.484275in}{1.758278in}}%
\pgfpathlineto{\pgfqpoint{1.485035in}{0.895017in}}%
\pgfpathlineto{\pgfqpoint{1.485824in}{1.678474in}}%
\pgfpathlineto{\pgfqpoint{1.487036in}{0.822762in}}%
\pgfpathlineto{\pgfqpoint{1.487524in}{1.742166in}}%
\pgfpathlineto{\pgfqpoint{1.488602in}{0.988725in}}%
\pgfpathlineto{\pgfqpoint{1.489268in}{1.730539in}}%
\pgfpathlineto{\pgfqpoint{1.490059in}{1.025384in}}%
\pgfpathlineto{\pgfqpoint{1.490962in}{1.661862in}}%
\pgfpathlineto{\pgfqpoint{1.491929in}{0.951893in}}%
\pgfpathlineto{\pgfqpoint{1.492875in}{1.694523in}}%
\pgfpathlineto{\pgfqpoint{1.493440in}{1.079936in}}%
\pgfpathlineto{\pgfqpoint{1.494595in}{1.804496in}}%
\pgfpathlineto{\pgfqpoint{1.495278in}{1.048284in}}%
\pgfpathlineto{\pgfqpoint{1.496239in}{1.783555in}}%
\pgfpathlineto{\pgfqpoint{1.496819in}{1.058551in}}%
\pgfpathlineto{\pgfqpoint{1.497999in}{1.710686in}}%
\pgfpathlineto{\pgfqpoint{1.498791in}{0.945735in}}%
\pgfpathlineto{\pgfqpoint{1.499811in}{1.830764in}}%
\pgfpathlineto{\pgfqpoint{1.500634in}{1.015040in}}%
\pgfpathlineto{\pgfqpoint{1.501326in}{1.852493in}}%
\pgfpathlineto{\pgfqpoint{1.502023in}{0.962776in}}%
\pgfpathlineto{\pgfqpoint{1.502819in}{1.811321in}}%
\pgfpathlineto{\pgfqpoint{1.503683in}{1.004451in}}%
\pgfpathlineto{\pgfqpoint{1.504775in}{1.668219in}}%
\pgfpathlineto{\pgfqpoint{1.505879in}{0.841808in}}%
\pgfpathlineto{\pgfqpoint{1.506213in}{1.607402in}}%
\pgfpathlineto{\pgfqpoint{1.507102in}{0.891903in}}%
\pgfpathlineto{\pgfqpoint{1.507876in}{1.723704in}}%
\pgfpathlineto{\pgfqpoint{1.508840in}{1.019190in}}%
\pgfpathlineto{\pgfqpoint{1.509762in}{1.810558in}}%
\pgfpathlineto{\pgfqpoint{1.510562in}{0.872885in}}%
\pgfpathlineto{\pgfqpoint{1.511422in}{2.006752in}}%
\pgfpathlineto{\pgfqpoint{1.512146in}{0.990651in}}%
\pgfpathlineto{\pgfqpoint{1.513173in}{1.825757in}}%
\pgfpathlineto{\pgfqpoint{1.514178in}{0.901927in}}%
\pgfpathlineto{\pgfqpoint{1.514700in}{1.630512in}}%
\pgfpathlineto{\pgfqpoint{1.515677in}{1.044522in}}%
\pgfpathlineto{\pgfqpoint{1.516548in}{1.765465in}}%
\pgfpathlineto{\pgfqpoint{1.517227in}{0.947542in}}%
\pgfpathlineto{\pgfqpoint{1.518073in}{1.712435in}}%
\pgfpathlineto{\pgfqpoint{1.519016in}{0.976717in}}%
\pgfpathlineto{\pgfqpoint{1.519841in}{1.700222in}}%
\pgfpathlineto{\pgfqpoint{1.520629in}{0.915715in}}%
\pgfpathlineto{\pgfqpoint{1.521576in}{1.883783in}}%
\pgfpathlineto{\pgfqpoint{1.522312in}{1.118189in}}%
\pgfpathlineto{\pgfqpoint{1.523198in}{1.861087in}}%
\pgfpathlineto{\pgfqpoint{1.524292in}{1.028711in}}%
\pgfpathlineto{\pgfqpoint{1.525126in}{1.898766in}}%
\pgfpathlineto{\pgfqpoint{1.525861in}{0.996756in}}%
\pgfpathlineto{\pgfqpoint{1.526616in}{1.681944in}}%
\pgfpathlineto{\pgfqpoint{1.527994in}{0.907299in}}%
\pgfpathlineto{\pgfqpoint{1.528292in}{1.790826in}}%
\pgfpathlineto{\pgfqpoint{1.529325in}{0.870884in}}%
\pgfpathlineto{\pgfqpoint{1.530314in}{1.694813in}}%
\pgfpathlineto{\pgfqpoint{1.530875in}{0.974103in}}%
\pgfpathlineto{\pgfqpoint{1.531677in}{1.731490in}}%
\pgfpathlineto{\pgfqpoint{1.532567in}{0.902464in}}%
\pgfpathlineto{\pgfqpoint{1.533459in}{1.686826in}}%
\pgfpathlineto{\pgfqpoint{1.534392in}{0.894583in}}%
\pgfpathlineto{\pgfqpoint{1.535274in}{1.891278in}}%
\pgfpathlineto{\pgfqpoint{1.535936in}{1.068673in}}%
\pgfpathlineto{\pgfqpoint{1.536755in}{1.795508in}}%
\pgfpathlineto{\pgfqpoint{1.537872in}{0.981127in}}%
\pgfpathlineto{\pgfqpoint{1.538623in}{1.656949in}}%
\pgfpathlineto{\pgfqpoint{1.539322in}{0.866814in}}%
\pgfpathlineto{\pgfqpoint{1.540322in}{1.707742in}}%
\pgfpathlineto{\pgfqpoint{1.541114in}{0.962579in}}%
\pgfpathlineto{\pgfqpoint{1.542016in}{1.769026in}}%
\pgfpathlineto{\pgfqpoint{1.543099in}{0.956607in}}%
\pgfpathlineto{\pgfqpoint{1.543620in}{1.878723in}}%
\pgfpathlineto{\pgfqpoint{1.544595in}{1.049797in}}%
\pgfpathlineto{\pgfqpoint{1.545360in}{1.732772in}}%
\pgfpathlineto{\pgfqpoint{1.546187in}{1.062945in}}%
\pgfpathlineto{\pgfqpoint{1.547132in}{1.757944in}}%
\pgfpathlineto{\pgfqpoint{1.547817in}{1.132902in}}%
\pgfpathlineto{\pgfqpoint{1.548883in}{1.995572in}}%
\pgfpathlineto{\pgfqpoint{1.549611in}{1.183255in}}%
\pgfpathlineto{\pgfqpoint{1.550398in}{1.866981in}}%
\pgfpathlineto{\pgfqpoint{1.551243in}{1.152357in}}%
\pgfpathlineto{\pgfqpoint{1.552065in}{1.851902in}}%
\pgfpathlineto{\pgfqpoint{1.552918in}{0.987876in}}%
\pgfpathlineto{\pgfqpoint{1.554119in}{1.740903in}}%
\pgfpathlineto{\pgfqpoint{1.554741in}{0.911679in}}%
\pgfpathlineto{\pgfqpoint{1.555571in}{1.627959in}}%
\pgfpathlineto{\pgfqpoint{1.556453in}{0.998359in}}%
\pgfpathlineto{\pgfqpoint{1.557319in}{1.787075in}}%
\pgfpathlineto{\pgfqpoint{1.558430in}{0.767833in}}%
\pgfpathlineto{\pgfqpoint{1.559012in}{1.645752in}}%
\pgfpathlineto{\pgfqpoint{1.559762in}{0.846062in}}%
\pgfpathlineto{\pgfqpoint{1.560638in}{1.621472in}}%
\pgfpathlineto{\pgfqpoint{1.561584in}{0.802362in}}%
\pgfpathlineto{\pgfqpoint{1.562251in}{1.599253in}}%
\pgfpathlineto{\pgfqpoint{1.563280in}{0.791676in}}%
\pgfpathlineto{\pgfqpoint{1.564020in}{1.518127in}}%
\pgfpathlineto{\pgfqpoint{1.564832in}{1.022925in}}%
\pgfpathlineto{\pgfqpoint{1.565700in}{1.701577in}}%
\pgfpathlineto{\pgfqpoint{1.566729in}{1.000901in}}%
\pgfpathlineto{\pgfqpoint{1.567370in}{1.801086in}}%
\pgfpathlineto{\pgfqpoint{1.568210in}{1.026623in}}%
\pgfpathlineto{\pgfqpoint{1.569634in}{1.711988in}}%
\pgfpathlineto{\pgfqpoint{1.570168in}{0.928735in}}%
\pgfpathlineto{\pgfqpoint{1.571173in}{1.803908in}}%
\pgfpathlineto{\pgfqpoint{1.571756in}{1.049816in}}%
\pgfpathlineto{\pgfqpoint{1.572992in}{1.692386in}}%
\pgfpathlineto{\pgfqpoint{1.573382in}{0.998826in}}%
\pgfpathlineto{\pgfqpoint{1.574552in}{1.677892in}}%
\pgfpathlineto{\pgfqpoint{1.575050in}{0.923637in}}%
\pgfpathlineto{\pgfqpoint{1.576266in}{1.749182in}}%
\pgfpathlineto{\pgfqpoint{1.576696in}{0.897398in}}%
\pgfpathlineto{\pgfqpoint{1.577616in}{1.761802in}}%
\pgfpathlineto{\pgfqpoint{1.578550in}{1.001721in}}%
\pgfpathlineto{\pgfqpoint{1.579518in}{1.876334in}}%
\pgfpathlineto{\pgfqpoint{1.580469in}{1.038237in}}%
\pgfpathlineto{\pgfqpoint{1.580951in}{1.732838in}}%
\pgfpathlineto{\pgfqpoint{1.581827in}{1.067114in}}%
\pgfpathlineto{\pgfqpoint{1.582663in}{1.735775in}}%
\pgfpathlineto{\pgfqpoint{1.583961in}{0.907978in}}%
\pgfpathlineto{\pgfqpoint{1.584460in}{1.684929in}}%
\pgfpathlineto{\pgfqpoint{1.585495in}{0.880276in}}%
\pgfpathlineto{\pgfqpoint{1.586041in}{1.703116in}}%
\pgfpathlineto{\pgfqpoint{1.587121in}{0.938425in}}%
\pgfpathlineto{\pgfqpoint{1.587863in}{1.681253in}}%
\pgfpathlineto{\pgfqpoint{1.588720in}{0.942892in}}%
\pgfpathlineto{\pgfqpoint{1.589442in}{1.598548in}}%
\pgfpathlineto{\pgfqpoint{1.590340in}{0.843288in}}%
\pgfpathlineto{\pgfqpoint{1.591203in}{1.791093in}}%
\pgfpathlineto{\pgfqpoint{1.592372in}{0.972408in}}%
\pgfpathlineto{\pgfqpoint{1.592929in}{1.666744in}}%
\pgfpathlineto{\pgfqpoint{1.594153in}{0.778691in}}%
\pgfpathlineto{\pgfqpoint{1.594602in}{1.530109in}}%
\pgfpathlineto{\pgfqpoint{1.595694in}{0.874441in}}%
\pgfpathlineto{\pgfqpoint{1.596387in}{1.654167in}}%
\pgfpathlineto{\pgfqpoint{1.597216in}{0.909133in}}%
\pgfpathlineto{\pgfqpoint{1.598105in}{1.608109in}}%
\pgfpathlineto{\pgfqpoint{1.599609in}{0.870115in}}%
\pgfpathlineto{\pgfqpoint{1.599895in}{1.706109in}}%
\pgfpathlineto{\pgfqpoint{1.600615in}{0.841744in}}%
\pgfpathlineto{\pgfqpoint{1.601485in}{1.572648in}}%
\pgfpathlineto{\pgfqpoint{1.602815in}{0.736183in}}%
\pgfpathlineto{\pgfqpoint{1.603051in}{1.418133in}}%
\pgfpathlineto{\pgfqpoint{1.603993in}{0.733110in}}%
\pgfpathlineto{\pgfqpoint{1.604909in}{1.620535in}}%
\pgfpathlineto{\pgfqpoint{1.605591in}{0.949626in}}%
\pgfpathlineto{\pgfqpoint{1.606866in}{1.646765in}}%
\pgfpathlineto{\pgfqpoint{1.607287in}{0.943537in}}%
\pgfpathlineto{\pgfqpoint{1.608151in}{1.480179in}}%
\pgfpathlineto{\pgfqpoint{1.609300in}{0.740289in}}%
\pgfpathlineto{\pgfqpoint{1.609991in}{1.678424in}}%
\pgfpathlineto{\pgfqpoint{1.611413in}{0.657082in}}%
\pgfpathlineto{\pgfqpoint{1.611901in}{1.622677in}}%
\pgfpathlineto{\pgfqpoint{1.612427in}{0.956402in}}%
\pgfpathlineto{\pgfqpoint{1.613236in}{1.570551in}}%
\pgfpathlineto{\pgfqpoint{1.614656in}{0.841546in}}%
\pgfpathlineto{\pgfqpoint{1.615129in}{1.719613in}}%
\pgfpathlineto{\pgfqpoint{1.616600in}{0.766005in}}%
\pgfpathlineto{\pgfqpoint{1.616688in}{1.622546in}}%
\pgfpathlineto{\pgfqpoint{1.617575in}{0.829194in}}%
\pgfpathlineto{\pgfqpoint{1.618999in}{1.671852in}}%
\pgfpathlineto{\pgfqpoint{1.619335in}{0.921055in}}%
\pgfpathlineto{\pgfqpoint{1.620077in}{1.608881in}}%
\pgfpathlineto{\pgfqpoint{1.620914in}{0.853847in}}%
\pgfpathlineto{\pgfqpoint{1.621735in}{1.608809in}}%
\pgfpathlineto{\pgfqpoint{1.622690in}{0.754409in}}%
\pgfpathlineto{\pgfqpoint{1.623814in}{1.567486in}}%
\pgfpathlineto{\pgfqpoint{1.624310in}{0.683159in}}%
\pgfpathlineto{\pgfqpoint{1.625206in}{1.479441in}}%
\pgfpathlineto{\pgfqpoint{1.626128in}{0.805740in}}%
\pgfpathlineto{\pgfqpoint{1.626844in}{1.652008in}}%
\pgfpathlineto{\pgfqpoint{1.627933in}{0.951183in}}%
\pgfpathlineto{\pgfqpoint{1.628537in}{1.642251in}}%
\pgfpathlineto{\pgfqpoint{1.629569in}{1.024795in}}%
\pgfpathlineto{\pgfqpoint{1.630754in}{1.907442in}}%
\pgfpathlineto{\pgfqpoint{1.631251in}{0.851143in}}%
\pgfpathlineto{\pgfqpoint{1.631953in}{1.461549in}}%
\pgfpathlineto{\pgfqpoint{1.633480in}{1.650550in}}%
\pgfpathlineto{\pgfqpoint{1.634126in}{0.742533in}}%
\pgfpathlineto{\pgfqpoint{1.634917in}{1.584566in}}%
\pgfpathlineto{\pgfqpoint{1.635328in}{0.763666in}}%
\pgfpathlineto{\pgfqpoint{1.636470in}{1.788904in}}%
\pgfpathlineto{\pgfqpoint{1.637039in}{0.930004in}}%
\pgfpathlineto{\pgfqpoint{1.638086in}{1.616064in}}%
\pgfpathlineto{\pgfqpoint{1.638789in}{0.893169in}}%
\pgfpathlineto{\pgfqpoint{1.639624in}{1.568123in}}%
\pgfpathlineto{\pgfqpoint{1.640430in}{0.806687in}}%
\pgfpathlineto{\pgfqpoint{1.641454in}{1.490861in}}%
\pgfpathlineto{\pgfqpoint{1.642262in}{0.845284in}}%
\pgfpathlineto{\pgfqpoint{1.643171in}{1.559334in}}%
\pgfpathlineto{\pgfqpoint{1.644321in}{0.858508in}}%
\pgfpathlineto{\pgfqpoint{1.645032in}{1.580904in}}%
\pgfpathlineto{\pgfqpoint{1.645767in}{0.735470in}}%
\pgfpathlineto{\pgfqpoint{1.646550in}{1.582633in}}%
\pgfpathlineto{\pgfqpoint{1.647280in}{0.835971in}}%
\pgfpathlineto{\pgfqpoint{1.648176in}{1.612171in}}%
\pgfpathlineto{\pgfqpoint{1.649105in}{0.724993in}}%
\pgfpathlineto{\pgfqpoint{1.650250in}{1.696883in}}%
\pgfpathlineto{\pgfqpoint{1.650645in}{0.955917in}}%
\pgfpathlineto{\pgfqpoint{1.651503in}{1.568748in}}%
\pgfpathlineto{\pgfqpoint{1.652419in}{0.849989in}}%
\pgfpathlineto{\pgfqpoint{1.653226in}{1.619712in}}%
\pgfpathlineto{\pgfqpoint{1.654158in}{0.921330in}}%
\pgfpathlineto{\pgfqpoint{1.654916in}{1.606864in}}%
\pgfpathlineto{\pgfqpoint{1.655735in}{0.875634in}}%
\pgfpathlineto{\pgfqpoint{1.656618in}{1.534899in}}%
\pgfpathlineto{\pgfqpoint{1.657430in}{0.881546in}}%
\pgfpathlineto{\pgfqpoint{1.658330in}{1.855032in}}%
\pgfpathlineto{\pgfqpoint{1.659577in}{0.963141in}}%
\pgfpathlineto{\pgfqpoint{1.659997in}{1.745401in}}%
\pgfpathlineto{\pgfqpoint{1.660880in}{1.014493in}}%
\pgfpathlineto{\pgfqpoint{1.661724in}{1.824102in}}%
\pgfpathlineto{\pgfqpoint{1.662614in}{0.936995in}}%
\pgfpathlineto{\pgfqpoint{1.663419in}{1.766195in}}%
\pgfpathlineto{\pgfqpoint{1.664474in}{0.901451in}}%
\pgfpathlineto{\pgfqpoint{1.665075in}{1.665246in}}%
\pgfpathlineto{\pgfqpoint{1.665945in}{1.122113in}}%
\pgfpathlineto{\pgfqpoint{1.667467in}{1.687457in}}%
\pgfpathlineto{\pgfqpoint{1.667918in}{0.760847in}}%
\pgfpathlineto{\pgfqpoint{1.668901in}{1.757537in}}%
\pgfpathlineto{\pgfqpoint{1.669319in}{1.082879in}}%
\pgfpathlineto{\pgfqpoint{1.670923in}{1.661623in}}%
\pgfpathlineto{\pgfqpoint{1.671264in}{0.911842in}}%
\pgfpathlineto{\pgfqpoint{1.671977in}{1.614311in}}%
\pgfpathlineto{\pgfqpoint{1.672892in}{0.970473in}}%
\pgfpathlineto{\pgfqpoint{1.673653in}{1.662121in}}%
\pgfpathlineto{\pgfqpoint{1.674460in}{0.858153in}}%
\pgfpathlineto{\pgfqpoint{1.675438in}{1.687194in}}%
\pgfpathlineto{\pgfqpoint{1.676571in}{0.852159in}}%
\pgfpathlineto{\pgfqpoint{1.676965in}{1.704041in}}%
\pgfpathlineto{\pgfqpoint{1.678153in}{0.815310in}}%
\pgfpathlineto{\pgfqpoint{1.678775in}{1.749875in}}%
\pgfpathlineto{\pgfqpoint{1.679572in}{0.852938in}}%
\pgfpathlineto{\pgfqpoint{1.681087in}{1.679635in}}%
\pgfpathlineto{\pgfqpoint{1.681420in}{0.856817in}}%
\pgfpathlineto{\pgfqpoint{1.682074in}{1.719342in}}%
\pgfpathlineto{\pgfqpoint{1.683058in}{0.891273in}}%
\pgfpathlineto{\pgfqpoint{1.683823in}{1.617793in}}%
\pgfpathlineto{\pgfqpoint{1.684855in}{0.898924in}}%
\pgfpathlineto{\pgfqpoint{1.685669in}{1.661708in}}%
\pgfpathlineto{\pgfqpoint{1.686415in}{0.923736in}}%
\pgfpathlineto{\pgfqpoint{1.687408in}{1.736404in}}%
\pgfpathlineto{\pgfqpoint{1.688079in}{0.916291in}}%
\pgfpathlineto{\pgfqpoint{1.689147in}{1.671280in}}%
\pgfpathlineto{\pgfqpoint{1.689943in}{0.935081in}}%
\pgfpathlineto{\pgfqpoint{1.690690in}{1.786047in}}%
\pgfpathlineto{\pgfqpoint{1.691427in}{0.969927in}}%
\pgfpathlineto{\pgfqpoint{1.692489in}{1.653311in}}%
\pgfpathlineto{\pgfqpoint{1.693216in}{0.836988in}}%
\pgfpathlineto{\pgfqpoint{1.693979in}{1.589011in}}%
\pgfpathlineto{\pgfqpoint{1.695155in}{0.897388in}}%
\pgfpathlineto{\pgfqpoint{1.695905in}{1.699316in}}%
\pgfpathlineto{\pgfqpoint{1.696726in}{0.899802in}}%
\pgfpathlineto{\pgfqpoint{1.697897in}{1.741830in}}%
\pgfpathlineto{\pgfqpoint{1.698311in}{0.922520in}}%
\pgfpathlineto{\pgfqpoint{1.699435in}{1.642316in}}%
\pgfpathlineto{\pgfqpoint{1.700018in}{0.986575in}}%
\pgfpathlineto{\pgfqpoint{1.700796in}{1.516802in}}%
\pgfpathlineto{\pgfqpoint{1.702235in}{0.723922in}}%
\pgfpathlineto{\pgfqpoint{1.702482in}{1.629523in}}%
\pgfpathlineto{\pgfqpoint{1.703544in}{0.919426in}}%
\pgfpathlineto{\pgfqpoint{1.704535in}{1.787822in}}%
\pgfpathlineto{\pgfqpoint{1.705096in}{0.959588in}}%
\pgfpathlineto{\pgfqpoint{1.705981in}{1.677502in}}%
\pgfpathlineto{\pgfqpoint{1.706878in}{0.933357in}}%
\pgfpathlineto{\pgfqpoint{1.708401in}{1.894804in}}%
\pgfpathlineto{\pgfqpoint{1.708530in}{0.975177in}}%
\pgfpathlineto{\pgfqpoint{1.709766in}{1.744886in}}%
\pgfpathlineto{\pgfqpoint{1.710258in}{0.937385in}}%
\pgfpathlineto{\pgfqpoint{1.711256in}{1.773384in}}%
\pgfpathlineto{\pgfqpoint{1.712188in}{0.942079in}}%
\pgfpathlineto{\pgfqpoint{1.712674in}{1.618764in}}%
\pgfpathlineto{\pgfqpoint{1.713664in}{0.909364in}}%
\pgfpathlineto{\pgfqpoint{1.714364in}{1.772787in}}%
\pgfpathlineto{\pgfqpoint{1.715473in}{0.982835in}}%
\pgfpathlineto{\pgfqpoint{1.716302in}{1.798696in}}%
\pgfpathlineto{\pgfqpoint{1.716970in}{0.859624in}}%
\pgfpathlineto{\pgfqpoint{1.717858in}{1.674658in}}%
\pgfpathlineto{\pgfqpoint{1.718800in}{0.898574in}}%
\pgfpathlineto{\pgfqpoint{1.719628in}{1.644978in}}%
\pgfpathlineto{\pgfqpoint{1.720773in}{0.798940in}}%
\pgfpathlineto{\pgfqpoint{1.721189in}{1.666121in}}%
\pgfpathlineto{\pgfqpoint{1.722846in}{0.907346in}}%
\pgfpathlineto{\pgfqpoint{1.722936in}{1.762860in}}%
\pgfpathlineto{\pgfqpoint{1.723977in}{0.905269in}}%
\pgfpathlineto{\pgfqpoint{1.724868in}{1.695880in}}%
\pgfpathlineto{\pgfqpoint{1.725491in}{0.958775in}}%
\pgfpathlineto{\pgfqpoint{1.726689in}{1.750621in}}%
\pgfpathlineto{\pgfqpoint{1.727636in}{0.864087in}}%
\pgfpathlineto{\pgfqpoint{1.728018in}{1.691439in}}%
\pgfpathlineto{\pgfqpoint{1.729055in}{0.814867in}}%
\pgfpathlineto{\pgfqpoint{1.729674in}{1.601640in}}%
\pgfpathlineto{\pgfqpoint{1.730572in}{0.880505in}}%
\pgfpathlineto{\pgfqpoint{1.731827in}{1.608869in}}%
\pgfpathlineto{\pgfqpoint{1.732201in}{0.819501in}}%
\pgfpathlineto{\pgfqpoint{1.733869in}{1.678188in}}%
\pgfpathlineto{\pgfqpoint{1.733989in}{0.793603in}}%
\pgfpathlineto{\pgfqpoint{1.735196in}{1.698081in}}%
\pgfpathlineto{\pgfqpoint{1.735624in}{0.869668in}}%
\pgfpathlineto{\pgfqpoint{1.736539in}{1.557113in}}%
\pgfpathlineto{\pgfqpoint{1.737433in}{0.782695in}}%
\pgfpathlineto{\pgfqpoint{1.738156in}{1.457188in}}%
\pgfpathlineto{\pgfqpoint{1.739291in}{0.846141in}}%
\pgfpathlineto{\pgfqpoint{1.740204in}{1.546557in}}%
\pgfpathlineto{\pgfqpoint{1.740781in}{0.832042in}}%
\pgfpathlineto{\pgfqpoint{1.741563in}{1.677226in}}%
\pgfpathlineto{\pgfqpoint{1.742597in}{0.979880in}}%
\pgfpathlineto{\pgfqpoint{1.743391in}{1.720136in}}%
\pgfpathlineto{\pgfqpoint{1.744200in}{0.966471in}}%
\pgfpathlineto{\pgfqpoint{1.745145in}{1.629966in}}%
\pgfpathlineto{\pgfqpoint{1.745992in}{0.861751in}}%
\pgfpathlineto{\pgfqpoint{1.746856in}{1.551650in}}%
\pgfpathlineto{\pgfqpoint{1.747594in}{0.799284in}}%
\pgfpathlineto{\pgfqpoint{1.748761in}{1.727900in}}%
\pgfpathlineto{\pgfqpoint{1.749200in}{0.904272in}}%
\pgfpathlineto{\pgfqpoint{1.750318in}{1.684685in}}%
\pgfpathlineto{\pgfqpoint{1.750894in}{0.869788in}}%
\pgfpathlineto{\pgfqpoint{1.752009in}{1.691967in}}%
\pgfpathlineto{\pgfqpoint{1.752595in}{0.879279in}}%
\pgfpathlineto{\pgfqpoint{1.753474in}{1.545210in}}%
\pgfpathlineto{\pgfqpoint{1.754336in}{0.795484in}}%
\pgfpathlineto{\pgfqpoint{1.755376in}{1.585293in}}%
\pgfpathlineto{\pgfqpoint{1.756235in}{0.818519in}}%
\pgfpathlineto{\pgfqpoint{1.757516in}{1.545504in}}%
\pgfpathlineto{\pgfqpoint{1.758019in}{0.824061in}}%
\pgfpathlineto{\pgfqpoint{1.758767in}{1.558197in}}%
\pgfpathlineto{\pgfqpoint{1.759587in}{0.813332in}}%
\pgfpathlineto{\pgfqpoint{1.760306in}{1.583296in}}%
\pgfpathlineto{\pgfqpoint{1.761101in}{0.832199in}}%
\pgfpathlineto{\pgfqpoint{1.762041in}{1.559966in}}%
\pgfpathlineto{\pgfqpoint{1.762860in}{0.833047in}}%
\pgfpathlineto{\pgfqpoint{1.763743in}{1.537024in}}%
\pgfpathlineto{\pgfqpoint{1.764547in}{0.940505in}}%
\pgfpathlineto{\pgfqpoint{1.765989in}{1.644627in}}%
\pgfpathlineto{\pgfqpoint{1.766324in}{0.775738in}}%
\pgfpathlineto{\pgfqpoint{1.767054in}{1.633908in}}%
\pgfpathlineto{\pgfqpoint{1.768189in}{0.826890in}}%
\pgfpathlineto{\pgfqpoint{1.768896in}{1.564636in}}%
\pgfpathlineto{\pgfqpoint{1.770276in}{0.815382in}}%
\pgfpathlineto{\pgfqpoint{1.770532in}{1.562461in}}%
\pgfpathlineto{\pgfqpoint{1.771396in}{0.990541in}}%
\pgfpathlineto{\pgfqpoint{1.772289in}{1.621856in}}%
\pgfpathlineto{\pgfqpoint{1.772993in}{0.962064in}}%
\pgfpathlineto{\pgfqpoint{1.774012in}{1.699895in}}%
\pgfpathlineto{\pgfqpoint{1.774933in}{0.818495in}}%
\pgfpathlineto{\pgfqpoint{1.775673in}{1.502115in}}%
\pgfpathlineto{\pgfqpoint{1.776523in}{0.786922in}}%
\pgfpathlineto{\pgfqpoint{1.777359in}{1.615681in}}%
\pgfpathlineto{\pgfqpoint{1.778172in}{0.732261in}}%
\pgfpathlineto{\pgfqpoint{1.779177in}{1.516131in}}%
\pgfpathlineto{\pgfqpoint{1.779849in}{0.703251in}}%
\pgfpathlineto{\pgfqpoint{1.781040in}{1.559218in}}%
\pgfpathlineto{\pgfqpoint{1.781681in}{0.841387in}}%
\pgfpathlineto{\pgfqpoint{1.782362in}{1.550463in}}%
\pgfpathlineto{\pgfqpoint{1.783354in}{0.745388in}}%
\pgfpathlineto{\pgfqpoint{1.784120in}{1.561561in}}%
\pgfpathlineto{\pgfqpoint{1.785547in}{0.759421in}}%
\pgfpathlineto{\pgfqpoint{1.785859in}{1.562352in}}%
\pgfpathlineto{\pgfqpoint{1.786803in}{0.843661in}}%
\pgfpathlineto{\pgfqpoint{1.787987in}{1.615703in}}%
\pgfpathlineto{\pgfqpoint{1.788880in}{0.698739in}}%
\pgfpathlineto{\pgfqpoint{1.789331in}{1.456308in}}%
\pgfpathlineto{\pgfqpoint{1.790025in}{0.830126in}}%
\pgfpathlineto{\pgfqpoint{1.790869in}{1.591949in}}%
\pgfpathlineto{\pgfqpoint{1.791998in}{0.860204in}}%
\pgfpathlineto{\pgfqpoint{1.792618in}{1.653213in}}%
\pgfpathlineto{\pgfqpoint{1.794034in}{0.747057in}}%
\pgfpathlineto{\pgfqpoint{1.794658in}{1.584706in}}%
\pgfpathlineto{\pgfqpoint{1.796859in}{0.701655in}}%
\pgfpathlineto{\pgfqpoint{1.797738in}{1.476214in}}%
\pgfpathlineto{\pgfqpoint{1.798543in}{0.740148in}}%
\pgfpathlineto{\pgfqpoint{1.799421in}{1.483183in}}%
\pgfpathlineto{\pgfqpoint{1.800270in}{0.649176in}}%
\pgfpathlineto{\pgfqpoint{1.801072in}{1.398328in}}%
\pgfpathlineto{\pgfqpoint{1.802381in}{0.716292in}}%
\pgfpathlineto{\pgfqpoint{1.802829in}{1.552862in}}%
\pgfpathlineto{\pgfqpoint{1.803889in}{0.810539in}}%
\pgfpathlineto{\pgfqpoint{1.804454in}{1.506694in}}%
\pgfpathlineto{\pgfqpoint{1.805298in}{0.934529in}}%
\pgfpathlineto{\pgfqpoint{1.806682in}{1.610154in}}%
\pgfpathlineto{\pgfqpoint{1.807237in}{0.701946in}}%
\pgfpathlineto{\pgfqpoint{1.807849in}{1.526856in}}%
\pgfpathlineto{\pgfqpoint{1.808680in}{0.661488in}}%
\pgfpathlineto{\pgfqpoint{1.809621in}{1.483567in}}%
\pgfpathlineto{\pgfqpoint{1.810430in}{0.824786in}}%
\pgfpathlineto{\pgfqpoint{1.811711in}{1.581688in}}%
\pgfpathlineto{\pgfqpoint{1.812313in}{0.753230in}}%
\pgfpathlineto{\pgfqpoint{1.812976in}{1.539405in}}%
\pgfpathlineto{\pgfqpoint{1.814133in}{0.732984in}}%
\pgfpathlineto{\pgfqpoint{1.814644in}{1.555604in}}%
\pgfpathlineto{\pgfqpoint{1.815591in}{0.783669in}}%
\pgfpathlineto{\pgfqpoint{1.816438in}{1.420526in}}%
\pgfpathlineto{\pgfqpoint{1.817193in}{0.828168in}}%
\pgfpathlineto{\pgfqpoint{1.818535in}{1.639316in}}%
\pgfpathlineto{\pgfqpoint{1.818879in}{0.888252in}}%
\pgfpathlineto{\pgfqpoint{1.819837in}{1.604721in}}%
\pgfpathlineto{\pgfqpoint{1.820817in}{0.857353in}}%
\pgfpathlineto{\pgfqpoint{1.822098in}{1.568294in}}%
\pgfpathlineto{\pgfqpoint{1.822681in}{0.712964in}}%
\pgfpathlineto{\pgfqpoint{1.823253in}{1.543654in}}%
\pgfpathlineto{\pgfqpoint{1.824514in}{0.743025in}}%
\pgfpathlineto{\pgfqpoint{1.824839in}{1.444569in}}%
\pgfpathlineto{\pgfqpoint{1.825889in}{0.919711in}}%
\pgfpathlineto{\pgfqpoint{1.826817in}{1.604836in}}%
\pgfpathlineto{\pgfqpoint{1.827828in}{0.828812in}}%
\pgfpathlineto{\pgfqpoint{1.828250in}{1.668858in}}%
\pgfpathlineto{\pgfqpoint{1.829088in}{0.763750in}}%
\pgfpathlineto{\pgfqpoint{1.830003in}{1.556678in}}%
\pgfpathlineto{\pgfqpoint{1.830780in}{0.792174in}}%
\pgfpathlineto{\pgfqpoint{1.831700in}{1.704413in}}%
\pgfpathlineto{\pgfqpoint{1.832499in}{0.891555in}}%
\pgfpathlineto{\pgfqpoint{1.833397in}{1.508743in}}%
\pgfpathlineto{\pgfqpoint{1.834182in}{0.821768in}}%
\pgfpathlineto{\pgfqpoint{1.835225in}{1.621351in}}%
\pgfpathlineto{\pgfqpoint{1.835939in}{0.896369in}}%
\pgfpathlineto{\pgfqpoint{1.836985in}{1.658447in}}%
\pgfpathlineto{\pgfqpoint{1.837579in}{1.022229in}}%
\pgfpathlineto{\pgfqpoint{1.838952in}{1.715496in}}%
\pgfpathlineto{\pgfqpoint{1.839319in}{0.969896in}}%
\pgfpathlineto{\pgfqpoint{1.840182in}{1.661621in}}%
\pgfpathlineto{\pgfqpoint{1.841017in}{1.021383in}}%
\pgfpathlineto{\pgfqpoint{1.841895in}{1.677415in}}%
\pgfpathlineto{\pgfqpoint{1.843069in}{0.876644in}}%
\pgfpathlineto{\pgfqpoint{1.843924in}{1.768995in}}%
\pgfpathlineto{\pgfqpoint{1.844526in}{0.907920in}}%
\pgfpathlineto{\pgfqpoint{1.845277in}{1.560771in}}%
\pgfpathlineto{\pgfqpoint{1.846296in}{0.880323in}}%
\pgfpathlineto{\pgfqpoint{1.846919in}{1.777368in}}%
\pgfpathlineto{\pgfqpoint{1.848033in}{0.939017in}}%
\pgfpathlineto{\pgfqpoint{1.848808in}{1.719538in}}%
\pgfpathlineto{\pgfqpoint{1.849848in}{0.946542in}}%
\pgfpathlineto{\pgfqpoint{1.850430in}{1.638055in}}%
\pgfpathlineto{\pgfqpoint{1.851448in}{0.922509in}}%
\pgfpathlineto{\pgfqpoint{1.852032in}{1.681832in}}%
\pgfpathlineto{\pgfqpoint{1.852889in}{0.932780in}}%
\pgfpathlineto{\pgfqpoint{1.854171in}{1.648369in}}%
\pgfpathlineto{\pgfqpoint{1.855213in}{0.886600in}}%
\pgfpathlineto{\pgfqpoint{1.855440in}{1.657182in}}%
\pgfpathlineto{\pgfqpoint{1.856471in}{0.977522in}}%
\pgfpathlineto{\pgfqpoint{1.857339in}{1.646856in}}%
\pgfpathlineto{\pgfqpoint{1.858020in}{0.908090in}}%
\pgfpathlineto{\pgfqpoint{1.858928in}{1.800151in}}%
\pgfpathlineto{\pgfqpoint{1.859665in}{1.036627in}}%
\pgfpathlineto{\pgfqpoint{1.860569in}{1.585294in}}%
\pgfpathlineto{\pgfqpoint{1.861443in}{0.735797in}}%
\pgfpathlineto{\pgfqpoint{1.862342in}{1.699777in}}%
\pgfpathlineto{\pgfqpoint{1.863113in}{0.970379in}}%
\pgfpathlineto{\pgfqpoint{1.864056in}{1.638776in}}%
\pgfpathlineto{\pgfqpoint{1.865057in}{0.914736in}}%
\pgfpathlineto{\pgfqpoint{1.865661in}{1.747711in}}%
\pgfpathlineto{\pgfqpoint{1.866587in}{0.821716in}}%
\pgfpathlineto{\pgfqpoint{1.867702in}{1.705447in}}%
\pgfpathlineto{\pgfqpoint{1.868168in}{1.051514in}}%
\pgfpathlineto{\pgfqpoint{1.869108in}{1.667584in}}%
\pgfpathlineto{\pgfqpoint{1.870112in}{0.907523in}}%
\pgfpathlineto{\pgfqpoint{1.870985in}{1.744353in}}%
\pgfpathlineto{\pgfqpoint{1.871722in}{0.916308in}}%
\pgfpathlineto{\pgfqpoint{1.872713in}{1.651499in}}%
\pgfpathlineto{\pgfqpoint{1.873335in}{0.822364in}}%
\pgfpathlineto{\pgfqpoint{1.874655in}{1.735660in}}%
\pgfpathlineto{\pgfqpoint{1.874999in}{0.963200in}}%
\pgfpathlineto{\pgfqpoint{1.876373in}{1.861542in}}%
\pgfpathlineto{\pgfqpoint{1.876798in}{0.808887in}}%
\pgfpathlineto{\pgfqpoint{1.877652in}{1.698520in}}%
\pgfpathlineto{\pgfqpoint{1.878394in}{0.938001in}}%
\pgfpathlineto{\pgfqpoint{1.879230in}{1.618203in}}%
\pgfpathlineto{\pgfqpoint{1.880289in}{0.851743in}}%
\pgfpathlineto{\pgfqpoint{1.881272in}{1.562023in}}%
\pgfpathlineto{\pgfqpoint{1.881862in}{0.807002in}}%
\pgfpathlineto{\pgfqpoint{1.883008in}{1.661285in}}%
\pgfpathlineto{\pgfqpoint{1.883658in}{0.900029in}}%
\pgfpathlineto{\pgfqpoint{1.884369in}{1.801120in}}%
\pgfpathlineto{\pgfqpoint{1.885423in}{0.965009in}}%
\pgfpathlineto{\pgfqpoint{1.886127in}{1.827514in}}%
\pgfpathlineto{\pgfqpoint{1.886990in}{0.925413in}}%
\pgfpathlineto{\pgfqpoint{1.888003in}{1.757512in}}%
\pgfpathlineto{\pgfqpoint{1.888699in}{0.969312in}}%
\pgfpathlineto{\pgfqpoint{1.889481in}{1.671919in}}%
\pgfpathlineto{\pgfqpoint{1.890567in}{0.927280in}}%
\pgfpathlineto{\pgfqpoint{1.891173in}{1.839109in}}%
\pgfpathlineto{\pgfqpoint{1.892237in}{0.878502in}}%
\pgfpathlineto{\pgfqpoint{1.893152in}{1.776528in}}%
\pgfpathlineto{\pgfqpoint{1.893795in}{0.893429in}}%
\pgfpathlineto{\pgfqpoint{1.894583in}{1.701184in}}%
\pgfpathlineto{\pgfqpoint{1.895540in}{0.957592in}}%
\pgfpathlineto{\pgfqpoint{1.896223in}{1.608461in}}%
\pgfpathlineto{\pgfqpoint{1.897724in}{1.046283in}}%
\pgfpathlineto{\pgfqpoint{1.897955in}{1.814685in}}%
\pgfpathlineto{\pgfqpoint{1.898826in}{0.973533in}}%
\pgfpathlineto{\pgfqpoint{1.899618in}{1.796328in}}%
\pgfpathlineto{\pgfqpoint{1.900550in}{1.029864in}}%
\pgfpathlineto{\pgfqpoint{1.901916in}{1.903448in}}%
\pgfpathlineto{\pgfqpoint{1.902162in}{1.037775in}}%
\pgfpathlineto{\pgfqpoint{1.903527in}{1.883520in}}%
\pgfpathlineto{\pgfqpoint{1.903940in}{0.994263in}}%
\pgfpathlineto{\pgfqpoint{1.904952in}{1.812108in}}%
\pgfpathlineto{\pgfqpoint{1.905799in}{0.856994in}}%
\pgfpathlineto{\pgfqpoint{1.906463in}{1.596984in}}%
\pgfpathlineto{\pgfqpoint{1.907663in}{0.986084in}}%
\pgfpathlineto{\pgfqpoint{1.908367in}{1.750128in}}%
\pgfpathlineto{\pgfqpoint{1.909683in}{0.857625in}}%
\pgfpathlineto{\pgfqpoint{1.909872in}{1.653087in}}%
\pgfpathlineto{\pgfqpoint{1.910921in}{0.955592in}}%
\pgfpathlineto{\pgfqpoint{1.911788in}{1.666485in}}%
\pgfpathlineto{\pgfqpoint{1.912457in}{1.013166in}}%
\pgfpathlineto{\pgfqpoint{1.913305in}{1.659347in}}%
\pgfpathlineto{\pgfqpoint{1.914070in}{0.990457in}}%
\pgfpathlineto{\pgfqpoint{1.915469in}{1.806312in}}%
\pgfpathlineto{\pgfqpoint{1.915820in}{1.038988in}}%
\pgfpathlineto{\pgfqpoint{1.916884in}{1.829305in}}%
\pgfpathlineto{\pgfqpoint{1.918019in}{0.929197in}}%
\pgfpathlineto{\pgfqpoint{1.918432in}{1.776650in}}%
\pgfpathlineto{\pgfqpoint{1.919210in}{1.061272in}}%
\pgfpathlineto{\pgfqpoint{1.920040in}{1.781195in}}%
\pgfpathlineto{\pgfqpoint{1.921642in}{1.012324in}}%
\pgfpathlineto{\pgfqpoint{1.921864in}{1.805302in}}%
\pgfpathlineto{\pgfqpoint{1.922935in}{0.962918in}}%
\pgfpathlineto{\pgfqpoint{1.923507in}{1.602492in}}%
\pgfpathlineto{\pgfqpoint{1.924729in}{0.904831in}}%
\pgfpathlineto{\pgfqpoint{1.925361in}{1.743913in}}%
\pgfpathlineto{\pgfqpoint{1.926179in}{0.885865in}}%
\pgfpathlineto{\pgfqpoint{1.927034in}{1.661472in}}%
\pgfpathlineto{\pgfqpoint{1.927977in}{0.904255in}}%
\pgfpathlineto{\pgfqpoint{1.929008in}{1.697802in}}%
\pgfpathlineto{\pgfqpoint{1.929544in}{0.923218in}}%
\pgfpathlineto{\pgfqpoint{1.930607in}{1.714970in}}%
\pgfpathlineto{\pgfqpoint{1.931092in}{0.949501in}}%
\pgfpathlineto{\pgfqpoint{1.932113in}{1.584941in}}%
\pgfpathlineto{\pgfqpoint{1.932917in}{0.896793in}}%
\pgfpathlineto{\pgfqpoint{1.933919in}{1.709676in}}%
\pgfpathlineto{\pgfqpoint{1.934472in}{0.927015in}}%
\pgfpathlineto{\pgfqpoint{1.935493in}{1.627333in}}%
\pgfpathlineto{\pgfqpoint{1.936927in}{0.809457in}}%
\pgfpathlineto{\pgfqpoint{1.937067in}{1.646791in}}%
\pgfpathlineto{\pgfqpoint{1.938146in}{0.981150in}}%
\pgfpathlineto{\pgfqpoint{1.939335in}{1.827141in}}%
\pgfpathlineto{\pgfqpoint{1.939636in}{1.075842in}}%
\pgfpathlineto{\pgfqpoint{1.940395in}{1.696186in}}%
\pgfpathlineto{\pgfqpoint{1.941922in}{0.880543in}}%
\pgfpathlineto{\pgfqpoint{1.942109in}{1.512196in}}%
\pgfpathlineto{\pgfqpoint{1.943309in}{0.760435in}}%
\pgfpathlineto{\pgfqpoint{1.943796in}{1.592132in}}%
\pgfpathlineto{\pgfqpoint{1.944806in}{0.964905in}}%
\pgfpathlineto{\pgfqpoint{1.945517in}{1.637463in}}%
\pgfpathlineto{\pgfqpoint{1.946964in}{0.838017in}}%
\pgfpathlineto{\pgfqpoint{1.947322in}{1.689313in}}%
\pgfpathlineto{\pgfqpoint{1.948046in}{1.099299in}}%
\pgfpathlineto{\pgfqpoint{1.948900in}{1.720334in}}%
\pgfpathlineto{\pgfqpoint{1.949982in}{0.966018in}}%
\pgfpathlineto{\pgfqpoint{1.950726in}{1.815746in}}%
\pgfpathlineto{\pgfqpoint{1.951462in}{1.123124in}}%
\pgfpathlineto{\pgfqpoint{1.952450in}{1.892067in}}%
\pgfpathlineto{\pgfqpoint{1.953201in}{0.863751in}}%
\pgfpathlineto{\pgfqpoint{1.954410in}{1.777953in}}%
\pgfpathlineto{\pgfqpoint{1.954895in}{1.127480in}}%
\pgfpathlineto{\pgfqpoint{1.955901in}{1.794889in}}%
\pgfpathlineto{\pgfqpoint{1.956914in}{1.039867in}}%
\pgfpathlineto{\pgfqpoint{1.957497in}{1.730160in}}%
\pgfpathlineto{\pgfqpoint{1.958304in}{1.021923in}}%
\pgfpathlineto{\pgfqpoint{1.959097in}{1.844587in}}%
\pgfpathlineto{\pgfqpoint{1.960239in}{0.992201in}}%
\pgfpathlineto{\pgfqpoint{1.960847in}{1.825340in}}%
\pgfpathlineto{\pgfqpoint{1.961782in}{0.941603in}}%
\pgfpathlineto{\pgfqpoint{1.962554in}{1.746596in}}%
\pgfpathlineto{\pgfqpoint{1.963394in}{0.927191in}}%
\pgfpathlineto{\pgfqpoint{1.964244in}{1.797909in}}%
\pgfpathlineto{\pgfqpoint{1.965222in}{0.906219in}}%
\pgfpathlineto{\pgfqpoint{1.966116in}{1.696181in}}%
\pgfpathlineto{\pgfqpoint{1.966760in}{1.059144in}}%
\pgfpathlineto{\pgfqpoint{1.968048in}{1.847421in}}%
\pgfpathlineto{\pgfqpoint{1.968722in}{1.046063in}}%
\pgfpathlineto{\pgfqpoint{1.969797in}{1.863956in}}%
\pgfpathlineto{\pgfqpoint{1.970149in}{1.009915in}}%
\pgfpathlineto{\pgfqpoint{1.971352in}{1.738333in}}%
\pgfpathlineto{\pgfqpoint{1.971883in}{0.973807in}}%
\pgfpathlineto{\pgfqpoint{1.972697in}{1.774405in}}%
\pgfpathlineto{\pgfqpoint{1.974219in}{0.974951in}}%
\pgfpathlineto{\pgfqpoint{1.974400in}{1.690621in}}%
\pgfpathlineto{\pgfqpoint{1.975640in}{0.998085in}}%
\pgfpathlineto{\pgfqpoint{1.976140in}{1.764191in}}%
\pgfpathlineto{\pgfqpoint{1.977305in}{0.910246in}}%
\pgfpathlineto{\pgfqpoint{1.977794in}{1.785647in}}%
\pgfpathlineto{\pgfqpoint{1.978653in}{1.135852in}}%
\pgfpathlineto{\pgfqpoint{1.979601in}{1.770325in}}%
\pgfpathlineto{\pgfqpoint{1.980625in}{0.979400in}}%
\pgfpathlineto{\pgfqpoint{1.981343in}{1.815699in}}%
\pgfpathlineto{\pgfqpoint{1.982529in}{0.860838in}}%
\pgfpathlineto{\pgfqpoint{1.983143in}{1.780650in}}%
\pgfpathlineto{\pgfqpoint{1.983733in}{1.016664in}}%
\pgfpathlineto{\pgfqpoint{1.985149in}{1.721993in}}%
\pgfpathlineto{\pgfqpoint{1.985526in}{0.935143in}}%
\pgfpathlineto{\pgfqpoint{1.986568in}{1.789986in}}%
\pgfpathlineto{\pgfqpoint{1.987148in}{0.759397in}}%
\pgfpathlineto{\pgfqpoint{1.988157in}{1.680180in}}%
\pgfpathlineto{\pgfqpoint{1.989148in}{0.836725in}}%
\pgfpathlineto{\pgfqpoint{1.989858in}{1.613955in}}%
\pgfpathlineto{\pgfqpoint{1.990576in}{0.879802in}}%
\pgfpathlineto{\pgfqpoint{1.991433in}{1.557173in}}%
\pgfpathlineto{\pgfqpoint{1.992325in}{0.864762in}}%
\pgfpathlineto{\pgfqpoint{1.993319in}{1.657605in}}%
\pgfpathlineto{\pgfqpoint{1.994200in}{0.657238in}}%
\pgfpathlineto{\pgfqpoint{1.994831in}{1.550268in}}%
\pgfpathlineto{\pgfqpoint{1.995858in}{0.908335in}}%
\pgfpathlineto{\pgfqpoint{1.996525in}{1.683408in}}%
\pgfpathlineto{\pgfqpoint{1.997330in}{1.060409in}}%
\pgfpathlineto{\pgfqpoint{1.998188in}{1.641633in}}%
\pgfpathlineto{\pgfqpoint{1.999172in}{0.850625in}}%
\pgfpathlineto{\pgfqpoint{1.999998in}{1.655154in}}%
\pgfpathlineto{\pgfqpoint{2.000981in}{0.848974in}}%
\pgfpathlineto{\pgfqpoint{2.001863in}{1.622809in}}%
\pgfpathlineto{\pgfqpoint{2.002585in}{0.963991in}}%
\pgfpathlineto{\pgfqpoint{2.003353in}{1.678375in}}%
\pgfpathlineto{\pgfqpoint{2.004409in}{0.891706in}}%
\pgfpathlineto{\pgfqpoint{2.005016in}{1.710430in}}%
\pgfpathlineto{\pgfqpoint{2.005854in}{0.928887in}}%
\pgfpathlineto{\pgfqpoint{2.007156in}{1.739577in}}%
\pgfpathlineto{\pgfqpoint{2.007604in}{0.910698in}}%
\pgfpathlineto{\pgfqpoint{2.008410in}{1.598066in}}%
\pgfpathlineto{\pgfqpoint{2.009303in}{0.915098in}}%
\pgfpathlineto{\pgfqpoint{2.010604in}{1.630755in}}%
\pgfpathlineto{\pgfqpoint{2.011086in}{0.986112in}}%
\pgfpathlineto{\pgfqpoint{2.011975in}{1.856148in}}%
\pgfpathlineto{\pgfqpoint{2.012909in}{1.012582in}}%
\pgfpathlineto{\pgfqpoint{2.013665in}{1.664905in}}%
\pgfpathlineto{\pgfqpoint{2.014399in}{1.081051in}}%
\pgfpathlineto{\pgfqpoint{2.015305in}{1.701489in}}%
\pgfpathlineto{\pgfqpoint{2.016327in}{0.782194in}}%
\pgfpathlineto{\pgfqpoint{2.016922in}{1.594869in}}%
\pgfpathlineto{\pgfqpoint{2.018537in}{0.825956in}}%
\pgfpathlineto{\pgfqpoint{2.019122in}{1.810525in}}%
\pgfpathlineto{\pgfqpoint{2.019430in}{0.980928in}}%
\pgfpathlineto{\pgfqpoint{2.020465in}{1.695691in}}%
\pgfpathlineto{\pgfqpoint{2.021401in}{0.847019in}}%
\pgfpathlineto{\pgfqpoint{2.022405in}{1.706288in}}%
\pgfpathlineto{\pgfqpoint{2.023032in}{1.006507in}}%
\pgfpathlineto{\pgfqpoint{2.023682in}{1.733162in}}%
\pgfpathlineto{\pgfqpoint{2.025120in}{0.878973in}}%
\pgfpathlineto{\pgfqpoint{2.025562in}{1.792581in}}%
\pgfpathlineto{\pgfqpoint{2.026355in}{1.005867in}}%
\pgfpathlineto{\pgfqpoint{2.027132in}{1.743026in}}%
\pgfpathlineto{\pgfqpoint{2.028529in}{0.864178in}}%
\pgfpathlineto{\pgfqpoint{2.029021in}{1.840568in}}%
\pgfpathlineto{\pgfqpoint{2.029630in}{1.076776in}}%
\pgfpathlineto{\pgfqpoint{2.030913in}{1.659419in}}%
\pgfpathlineto{\pgfqpoint{2.031716in}{0.956192in}}%
\pgfpathlineto{\pgfqpoint{2.032682in}{1.715998in}}%
\pgfpathlineto{\pgfqpoint{2.033098in}{0.916361in}}%
\pgfpathlineto{\pgfqpoint{2.034113in}{1.830194in}}%
\pgfpathlineto{\pgfqpoint{2.034933in}{0.997746in}}%
\pgfpathlineto{\pgfqpoint{2.036152in}{1.662889in}}%
\pgfpathlineto{\pgfqpoint{2.036612in}{0.863110in}}%
\pgfpathlineto{\pgfqpoint{2.037570in}{1.731400in}}%
\pgfpathlineto{\pgfqpoint{2.038156in}{1.046560in}}%
\pgfpathlineto{\pgfqpoint{2.039022in}{1.739129in}}%
\pgfpathlineto{\pgfqpoint{2.040018in}{1.022360in}}%
\pgfpathlineto{\pgfqpoint{2.040858in}{1.884658in}}%
\pgfpathlineto{\pgfqpoint{2.041992in}{0.855511in}}%
\pgfpathlineto{\pgfqpoint{2.042409in}{1.560541in}}%
\pgfpathlineto{\pgfqpoint{2.043281in}{0.864347in}}%
\pgfpathlineto{\pgfqpoint{2.044166in}{1.678900in}}%
\pgfpathlineto{\pgfqpoint{2.045146in}{0.763183in}}%
\pgfpathlineto{\pgfqpoint{2.046476in}{1.800079in}}%
\pgfpathlineto{\pgfqpoint{2.046700in}{0.968768in}}%
\pgfpathlineto{\pgfqpoint{2.047471in}{1.627586in}}%
\pgfpathlineto{\pgfqpoint{2.048441in}{0.956784in}}%
\pgfpathlineto{\pgfqpoint{2.049411in}{1.730769in}}%
\pgfpathlineto{\pgfqpoint{2.050080in}{1.003120in}}%
\pgfpathlineto{\pgfqpoint{2.050904in}{1.832637in}}%
\pgfpathlineto{\pgfqpoint{2.051927in}{1.089640in}}%
\pgfpathlineto{\pgfqpoint{2.052563in}{1.843231in}}%
\pgfpathlineto{\pgfqpoint{2.053543in}{0.905889in}}%
\pgfpathlineto{\pgfqpoint{2.054450in}{1.704341in}}%
\pgfpathlineto{\pgfqpoint{2.055391in}{0.835999in}}%
\pgfpathlineto{\pgfqpoint{2.056501in}{1.717752in}}%
\pgfpathlineto{\pgfqpoint{2.056965in}{0.942092in}}%
\pgfpathlineto{\pgfqpoint{2.058237in}{1.734419in}}%
\pgfpathlineto{\pgfqpoint{2.058526in}{1.020505in}}%
\pgfpathlineto{\pgfqpoint{2.059870in}{1.715639in}}%
\pgfpathlineto{\pgfqpoint{2.060224in}{0.957547in}}%
\pgfpathlineto{\pgfqpoint{2.061180in}{1.677361in}}%
\pgfpathlineto{\pgfqpoint{2.062032in}{0.925018in}}%
\pgfpathlineto{\pgfqpoint{2.062900in}{1.686154in}}%
\pgfpathlineto{\pgfqpoint{2.063819in}{0.875511in}}%
\pgfpathlineto{\pgfqpoint{2.064482in}{1.610520in}}%
\pgfpathlineto{\pgfqpoint{2.065375in}{0.953381in}}%
\pgfpathlineto{\pgfqpoint{2.066406in}{1.807023in}}%
\pgfpathlineto{\pgfqpoint{2.067057in}{0.902980in}}%
\pgfpathlineto{\pgfqpoint{2.068189in}{1.739158in}}%
\pgfpathlineto{\pgfqpoint{2.068963in}{0.943976in}}%
\pgfpathlineto{\pgfqpoint{2.070048in}{1.786276in}}%
\pgfpathlineto{\pgfqpoint{2.070502in}{1.073689in}}%
\pgfpathlineto{\pgfqpoint{2.071663in}{1.737055in}}%
\pgfpathlineto{\pgfqpoint{2.072296in}{0.870633in}}%
\pgfpathlineto{\pgfqpoint{2.072979in}{1.584047in}}%
\pgfpathlineto{\pgfqpoint{2.074149in}{0.900290in}}%
\pgfpathlineto{\pgfqpoint{2.074830in}{1.624560in}}%
\pgfpathlineto{\pgfqpoint{2.075758in}{0.934860in}}%
\pgfpathlineto{\pgfqpoint{2.076468in}{1.717912in}}%
\pgfpathlineto{\pgfqpoint{2.077509in}{0.857773in}}%
\pgfpathlineto{\pgfqpoint{2.078083in}{1.771620in}}%
\pgfpathlineto{\pgfqpoint{2.079379in}{1.113831in}}%
\pgfpathlineto{\pgfqpoint{2.079825in}{1.790544in}}%
\pgfpathlineto{\pgfqpoint{2.080772in}{0.964032in}}%
\pgfpathlineto{\pgfqpoint{2.082189in}{1.825998in}}%
\pgfpathlineto{\pgfqpoint{2.082342in}{1.119717in}}%
\pgfpathlineto{\pgfqpoint{2.083376in}{1.790755in}}%
\pgfpathlineto{\pgfqpoint{2.084177in}{0.967902in}}%
\pgfpathlineto{\pgfqpoint{2.085048in}{1.709259in}}%
\pgfpathlineto{\pgfqpoint{2.085780in}{0.880714in}}%
\pgfpathlineto{\pgfqpoint{2.086613in}{1.590801in}}%
\pgfpathlineto{\pgfqpoint{2.087505in}{0.974426in}}%
\pgfpathlineto{\pgfqpoint{2.089031in}{1.795457in}}%
\pgfpathlineto{\pgfqpoint{2.089136in}{0.976198in}}%
\pgfpathlineto{\pgfqpoint{2.089972in}{1.645408in}}%
\pgfpathlineto{\pgfqpoint{2.090820in}{0.896664in}}%
\pgfpathlineto{\pgfqpoint{2.091908in}{1.764431in}}%
\pgfpathlineto{\pgfqpoint{2.092566in}{0.954251in}}%
\pgfpathlineto{\pgfqpoint{2.093352in}{1.682862in}}%
\pgfpathlineto{\pgfqpoint{2.094459in}{0.760966in}}%
\pgfpathlineto{\pgfqpoint{2.095050in}{1.546085in}}%
\pgfpathlineto{\pgfqpoint{2.096138in}{0.778234in}}%
\pgfpathlineto{\pgfqpoint{2.096899in}{1.562730in}}%
\pgfpathlineto{\pgfqpoint{2.097922in}{0.887259in}}%
\pgfpathlineto{\pgfqpoint{2.098470in}{1.730154in}}%
\pgfpathlineto{\pgfqpoint{2.099337in}{0.949314in}}%
\pgfpathlineto{\pgfqpoint{2.100173in}{1.725711in}}%
\pgfpathlineto{\pgfqpoint{2.101236in}{0.820420in}}%
\pgfpathlineto{\pgfqpoint{2.101861in}{1.527951in}}%
\pgfpathlineto{\pgfqpoint{2.102985in}{0.789783in}}%
\pgfpathlineto{\pgfqpoint{2.103659in}{1.650168in}}%
\pgfpathlineto{\pgfqpoint{2.104452in}{0.967281in}}%
\pgfpathlineto{\pgfqpoint{2.105508in}{1.622212in}}%
\pgfpathlineto{\pgfqpoint{2.106311in}{0.786874in}}%
\pgfpathlineto{\pgfqpoint{2.106973in}{1.652985in}}%
\pgfpathlineto{\pgfqpoint{2.107864in}{0.922782in}}%
\pgfpathlineto{\pgfqpoint{2.108863in}{1.641815in}}%
\pgfpathlineto{\pgfqpoint{2.109845in}{0.914470in}}%
\pgfpathlineto{\pgfqpoint{2.110709in}{1.725593in}}%
\pgfpathlineto{\pgfqpoint{2.111221in}{0.892446in}}%
\pgfpathlineto{\pgfqpoint{2.112720in}{1.652962in}}%
\pgfpathlineto{\pgfqpoint{2.112943in}{0.919167in}}%
\pgfpathlineto{\pgfqpoint{2.114216in}{1.635841in}}%
\pgfpathlineto{\pgfqpoint{2.114594in}{0.921835in}}%
\pgfpathlineto{\pgfqpoint{2.115554in}{1.744910in}}%
\pgfpathlineto{\pgfqpoint{2.116875in}{0.863870in}}%
\pgfpathlineto{\pgfqpoint{2.117215in}{1.612184in}}%
\pgfpathlineto{\pgfqpoint{2.118007in}{0.896472in}}%
\pgfpathlineto{\pgfqpoint{2.119680in}{1.705865in}}%
\pgfpathlineto{\pgfqpoint{2.120079in}{0.868097in}}%
\pgfpathlineto{\pgfqpoint{2.120600in}{1.688804in}}%
\pgfpathlineto{\pgfqpoint{2.121445in}{0.799278in}}%
\pgfpathlineto{\pgfqpoint{2.122297in}{1.545119in}}%
\pgfpathlineto{\pgfqpoint{2.123501in}{0.774617in}}%
\pgfpathlineto{\pgfqpoint{2.124079in}{1.562130in}}%
\pgfpathlineto{\pgfqpoint{2.125184in}{0.747519in}}%
\pgfpathlineto{\pgfqpoint{2.125833in}{1.507340in}}%
\pgfpathlineto{\pgfqpoint{2.126583in}{0.741794in}}%
\pgfpathlineto{\pgfqpoint{2.127690in}{1.602450in}}%
\pgfpathlineto{\pgfqpoint{2.128338in}{0.785305in}}%
\pgfpathlineto{\pgfqpoint{2.129136in}{1.624103in}}%
\pgfpathlineto{\pgfqpoint{2.130399in}{0.714824in}}%
\pgfpathlineto{\pgfqpoint{2.130913in}{1.570649in}}%
\pgfpathlineto{\pgfqpoint{2.132111in}{0.835290in}}%
\pgfpathlineto{\pgfqpoint{2.132763in}{1.719955in}}%
\pgfpathlineto{\pgfqpoint{2.133366in}{0.941731in}}%
\pgfpathlineto{\pgfqpoint{2.134442in}{1.660233in}}%
\pgfpathlineto{\pgfqpoint{2.135227in}{0.707125in}}%
\pgfpathlineto{\pgfqpoint{2.136041in}{1.693128in}}%
\pgfpathlineto{\pgfqpoint{2.136968in}{0.885798in}}%
\pgfpathlineto{\pgfqpoint{2.137557in}{1.573838in}}%
\pgfpathlineto{\pgfqpoint{2.138500in}{0.819508in}}%
\pgfpathlineto{\pgfqpoint{2.139238in}{1.646215in}}%
\pgfpathlineto{\pgfqpoint{2.140188in}{0.800885in}}%
\pgfpathlineto{\pgfqpoint{2.141235in}{1.669782in}}%
\pgfpathlineto{\pgfqpoint{2.141912in}{0.851425in}}%
\pgfpathlineto{\pgfqpoint{2.142666in}{1.611129in}}%
\pgfpathlineto{\pgfqpoint{2.143819in}{0.990915in}}%
\pgfpathlineto{\pgfqpoint{2.144523in}{1.770498in}}%
\pgfpathlineto{\pgfqpoint{2.145435in}{0.938009in}}%
\pgfpathlineto{\pgfqpoint{2.146066in}{1.671282in}}%
\pgfpathlineto{\pgfqpoint{2.146959in}{1.080604in}}%
\pgfpathlineto{\pgfqpoint{2.147839in}{1.898531in}}%
\pgfpathlineto{\pgfqpoint{2.148625in}{1.067320in}}%
\pgfpathlineto{\pgfqpoint{2.149581in}{1.617441in}}%
\pgfpathlineto{\pgfqpoint{2.150295in}{0.940418in}}%
\pgfpathlineto{\pgfqpoint{2.151663in}{1.656705in}}%
\pgfpathlineto{\pgfqpoint{2.152063in}{0.958930in}}%
\pgfpathlineto{\pgfqpoint{2.152879in}{1.630757in}}%
\pgfpathlineto{\pgfqpoint{2.154063in}{0.789410in}}%
\pgfpathlineto{\pgfqpoint{2.155097in}{1.693968in}}%
\pgfpathlineto{\pgfqpoint{2.155466in}{0.914720in}}%
\pgfpathlineto{\pgfqpoint{2.156281in}{1.550785in}}%
\pgfpathlineto{\pgfqpoint{2.157178in}{0.836698in}}%
\pgfpathlineto{\pgfqpoint{2.158198in}{1.690469in}}%
\pgfpathlineto{\pgfqpoint{2.158814in}{0.788420in}}%
\pgfpathlineto{\pgfqpoint{2.159768in}{1.650230in}}%
\pgfpathlineto{\pgfqpoint{2.160543in}{1.033881in}}%
\pgfpathlineto{\pgfqpoint{2.161344in}{1.667132in}}%
\pgfpathlineto{\pgfqpoint{2.162314in}{0.834110in}}%
\pgfpathlineto{\pgfqpoint{2.163558in}{1.676370in}}%
\pgfpathlineto{\pgfqpoint{2.164069in}{0.965806in}}%
\pgfpathlineto{\pgfqpoint{2.164759in}{1.696304in}}%
\pgfpathlineto{\pgfqpoint{2.165997in}{0.761752in}}%
\pgfpathlineto{\pgfqpoint{2.166520in}{1.582079in}}%
\pgfpathlineto{\pgfqpoint{2.167638in}{0.740116in}}%
\pgfpathlineto{\pgfqpoint{2.168440in}{1.759148in}}%
\pgfpathlineto{\pgfqpoint{2.169389in}{0.776785in}}%
\pgfpathlineto{\pgfqpoint{2.170035in}{1.602176in}}%
\pgfpathlineto{\pgfqpoint{2.170754in}{0.850193in}}%
\pgfpathlineto{\pgfqpoint{2.171610in}{1.559292in}}%
\pgfpathlineto{\pgfqpoint{2.173075in}{0.801315in}}%
\pgfpathlineto{\pgfqpoint{2.173514in}{1.541734in}}%
\pgfpathlineto{\pgfqpoint{2.174152in}{0.784922in}}%
\pgfpathlineto{\pgfqpoint{2.175663in}{1.549266in}}%
\pgfpathlineto{\pgfqpoint{2.175923in}{0.875629in}}%
\pgfpathlineto{\pgfqpoint{2.176810in}{1.513038in}}%
\pgfpathlineto{\pgfqpoint{2.178764in}{0.630770in}}%
\pgfpathlineto{\pgfqpoint{2.179435in}{1.578304in}}%
\pgfpathlineto{\pgfqpoint{2.180192in}{0.826406in}}%
\pgfpathlineto{\pgfqpoint{2.181010in}{1.552807in}}%
\pgfpathlineto{\pgfqpoint{2.181808in}{0.843180in}}%
\pgfpathlineto{\pgfqpoint{2.182811in}{1.617396in}}%
\pgfpathlineto{\pgfqpoint{2.183443in}{0.827426in}}%
\pgfpathlineto{\pgfqpoint{2.184888in}{1.697082in}}%
\pgfpathlineto{\pgfqpoint{2.185464in}{0.797899in}}%
\pgfpathlineto{\pgfqpoint{2.186138in}{1.529043in}}%
\pgfpathlineto{\pgfqpoint{2.187233in}{0.739644in}}%
\pgfpathlineto{\pgfqpoint{2.187886in}{1.537333in}}%
\pgfpathlineto{\pgfqpoint{2.188888in}{0.825978in}}%
\pgfpathlineto{\pgfqpoint{2.189398in}{1.614085in}}%
\pgfpathlineto{\pgfqpoint{2.190890in}{0.742085in}}%
\pgfpathlineto{\pgfqpoint{2.191246in}{1.750685in}}%
\pgfpathlineto{\pgfqpoint{2.192190in}{0.820051in}}%
\pgfpathlineto{\pgfqpoint{2.192825in}{1.523683in}}%
\pgfpathlineto{\pgfqpoint{2.194099in}{0.769611in}}%
\pgfpathlineto{\pgfqpoint{2.194489in}{1.433047in}}%
\pgfpathlineto{\pgfqpoint{2.195353in}{0.680822in}}%
\pgfpathlineto{\pgfqpoint{2.196274in}{1.495305in}}%
\pgfpathlineto{\pgfqpoint{2.197351in}{0.715248in}}%
\pgfpathlineto{\pgfqpoint{2.198576in}{1.669165in}}%
\pgfpathlineto{\pgfqpoint{2.198794in}{0.868125in}}%
\pgfpathlineto{\pgfqpoint{2.199683in}{1.643221in}}%
\pgfpathlineto{\pgfqpoint{2.200475in}{0.845242in}}%
\pgfpathlineto{\pgfqpoint{2.201406in}{1.641740in}}%
\pgfpathlineto{\pgfqpoint{2.202132in}{0.939768in}}%
\pgfpathlineto{\pgfqpoint{2.203456in}{1.754872in}}%
\pgfpathlineto{\pgfqpoint{2.203858in}{1.021143in}}%
\pgfpathlineto{\pgfqpoint{2.205430in}{1.711562in}}%
\pgfpathlineto{\pgfqpoint{2.205575in}{0.813530in}}%
\pgfpathlineto{\pgfqpoint{2.206873in}{1.711051in}}%
\pgfpathlineto{\pgfqpoint{2.207437in}{0.794396in}}%
\pgfpathlineto{\pgfqpoint{2.208071in}{1.569987in}}%
\pgfpathlineto{\pgfqpoint{2.208919in}{0.898602in}}%
\pgfpathlineto{\pgfqpoint{2.209921in}{1.595624in}}%
\pgfpathlineto{\pgfqpoint{2.210695in}{0.805138in}}%
\pgfpathlineto{\pgfqpoint{2.211786in}{1.557843in}}%
\pgfpathlineto{\pgfqpoint{2.212463in}{0.705545in}}%
\pgfpathlineto{\pgfqpoint{2.213194in}{1.568389in}}%
\pgfpathlineto{\pgfqpoint{2.214054in}{0.782375in}}%
\pgfpathlineto{\pgfqpoint{2.215216in}{1.684620in}}%
\pgfpathlineto{\pgfqpoint{2.215775in}{0.773361in}}%
\pgfpathlineto{\pgfqpoint{2.216879in}{1.627170in}}%
\pgfpathlineto{\pgfqpoint{2.217430in}{0.734387in}}%
\pgfpathlineto{\pgfqpoint{2.218999in}{1.649103in}}%
\pgfpathlineto{\pgfqpoint{2.219223in}{0.813738in}}%
\pgfpathlineto{\pgfqpoint{2.220729in}{1.623979in}}%
\pgfpathlineto{\pgfqpoint{2.221000in}{0.836615in}}%
\pgfpathlineto{\pgfqpoint{2.221812in}{1.538824in}}%
\pgfpathlineto{\pgfqpoint{2.222845in}{0.718834in}}%
\pgfpathlineto{\pgfqpoint{2.223585in}{1.575260in}}%
\pgfpathlineto{\pgfqpoint{2.224228in}{0.655946in}}%
\pgfpathlineto{\pgfqpoint{2.225064in}{1.552560in}}%
\pgfpathlineto{\pgfqpoint{2.226002in}{0.777608in}}%
\pgfpathlineto{\pgfqpoint{2.226816in}{1.681117in}}%
\pgfpathlineto{\pgfqpoint{2.228396in}{0.805991in}}%
\pgfpathlineto{\pgfqpoint{2.228493in}{1.484773in}}%
\pgfpathlineto{\pgfqpoint{2.229695in}{0.887471in}}%
\pgfpathlineto{\pgfqpoint{2.230487in}{1.700244in}}%
\pgfpathlineto{\pgfqpoint{2.231030in}{0.918654in}}%
\pgfpathlineto{\pgfqpoint{2.232138in}{1.607500in}}%
\pgfpathlineto{\pgfqpoint{2.232743in}{0.838075in}}%
\pgfpathlineto{\pgfqpoint{2.233757in}{1.561361in}}%
\pgfpathlineto{\pgfqpoint{2.234474in}{0.811322in}}%
\pgfpathlineto{\pgfqpoint{2.235263in}{1.555473in}}%
\pgfpathlineto{\pgfqpoint{2.236630in}{0.732142in}}%
\pgfpathlineto{\pgfqpoint{2.237006in}{1.567636in}}%
\pgfpathlineto{\pgfqpoint{2.238327in}{0.777376in}}%
\pgfpathlineto{\pgfqpoint{2.238694in}{1.579009in}}%
\pgfpathlineto{\pgfqpoint{2.239529in}{0.860555in}}%
\pgfpathlineto{\pgfqpoint{2.241041in}{1.602070in}}%
\pgfpathlineto{\pgfqpoint{2.241323in}{0.757656in}}%
\pgfpathlineto{\pgfqpoint{2.242059in}{1.641807in}}%
\pgfpathlineto{\pgfqpoint{2.242951in}{0.934487in}}%
\pgfpathlineto{\pgfqpoint{2.244059in}{1.753433in}}%
\pgfpathlineto{\pgfqpoint{2.244615in}{0.891839in}}%
\pgfpathlineto{\pgfqpoint{2.245676in}{1.653298in}}%
\pgfpathlineto{\pgfqpoint{2.246394in}{0.842160in}}%
\pgfpathlineto{\pgfqpoint{2.247168in}{1.576473in}}%
\pgfpathlineto{\pgfqpoint{2.248641in}{0.721575in}}%
\pgfpathlineto{\pgfqpoint{2.248887in}{1.490111in}}%
\pgfpathlineto{\pgfqpoint{2.249959in}{0.787962in}}%
\pgfpathlineto{\pgfqpoint{2.250567in}{1.601838in}}%
\pgfpathlineto{\pgfqpoint{2.252064in}{0.697897in}}%
\pgfpathlineto{\pgfqpoint{2.252321in}{1.515282in}}%
\pgfpathlineto{\pgfqpoint{2.253478in}{0.640075in}}%
\pgfpathlineto{\pgfqpoint{2.254091in}{1.408194in}}%
\pgfpathlineto{\pgfqpoint{2.255095in}{0.815249in}}%
\pgfpathlineto{\pgfqpoint{2.256324in}{1.508366in}}%
\pgfpathlineto{\pgfqpoint{2.256504in}{0.794286in}}%
\pgfpathlineto{\pgfqpoint{2.257362in}{1.478473in}}%
\pgfpathlineto{\pgfqpoint{2.258521in}{0.727203in}}%
\pgfpathlineto{\pgfqpoint{2.259311in}{1.514681in}}%
\pgfpathlineto{\pgfqpoint{2.260322in}{0.683210in}}%
\pgfpathlineto{\pgfqpoint{2.260973in}{1.664113in}}%
\pgfpathlineto{\pgfqpoint{2.261951in}{0.800218in}}%
\pgfpathlineto{\pgfqpoint{2.262550in}{1.577167in}}%
\pgfpathlineto{\pgfqpoint{2.263569in}{0.787929in}}%
\pgfpathlineto{\pgfqpoint{2.264437in}{1.555797in}}%
\pgfpathlineto{\pgfqpoint{2.265208in}{0.785645in}}%
\pgfpathlineto{\pgfqpoint{2.266208in}{1.580763in}}%
\pgfpathlineto{\pgfqpoint{2.267156in}{0.868107in}}%
\pgfpathlineto{\pgfqpoint{2.267947in}{1.598878in}}%
\pgfpathlineto{\pgfqpoint{2.268438in}{0.755764in}}%
\pgfpathlineto{\pgfqpoint{2.269367in}{1.770263in}}%
\pgfpathlineto{\pgfqpoint{2.270361in}{0.846278in}}%
\pgfpathlineto{\pgfqpoint{2.271074in}{1.456471in}}%
\pgfpathlineto{\pgfqpoint{2.271856in}{0.804210in}}%
\pgfpathlineto{\pgfqpoint{2.272875in}{1.591725in}}%
\pgfpathlineto{\pgfqpoint{2.273577in}{0.766899in}}%
\pgfpathlineto{\pgfqpoint{2.274378in}{1.630332in}}%
\pgfpathlineto{\pgfqpoint{2.275313in}{0.816936in}}%
\pgfpathlineto{\pgfqpoint{2.276357in}{1.582700in}}%
\pgfpathlineto{\pgfqpoint{2.277421in}{0.701827in}}%
\pgfpathlineto{\pgfqpoint{2.277874in}{1.472976in}}%
\pgfpathlineto{\pgfqpoint{2.278631in}{0.750038in}}%
\pgfpathlineto{\pgfqpoint{2.279463in}{1.388666in}}%
\pgfpathlineto{\pgfqpoint{2.280816in}{0.751986in}}%
\pgfpathlineto{\pgfqpoint{2.281210in}{1.509672in}}%
\pgfpathlineto{\pgfqpoint{2.282313in}{0.818262in}}%
\pgfpathlineto{\pgfqpoint{2.282986in}{1.526237in}}%
\pgfpathlineto{\pgfqpoint{2.283905in}{0.766631in}}%
\pgfpathlineto{\pgfqpoint{2.285073in}{1.613030in}}%
\pgfpathlineto{\pgfqpoint{2.285628in}{0.763952in}}%
\pgfpathlineto{\pgfqpoint{2.286849in}{1.811664in}}%
\pgfpathlineto{\pgfqpoint{2.287163in}{0.950754in}}%
\pgfpathlineto{\pgfqpoint{2.288054in}{1.528641in}}%
\pgfpathlineto{\pgfqpoint{2.288855in}{0.814885in}}%
\pgfpathlineto{\pgfqpoint{2.290257in}{1.495844in}}%
\pgfpathlineto{\pgfqpoint{2.290518in}{0.756501in}}%
\pgfpathlineto{\pgfqpoint{2.291779in}{1.629791in}}%
\pgfpathlineto{\pgfqpoint{2.292257in}{0.814656in}}%
\pgfpathlineto{\pgfqpoint{2.293190in}{1.533882in}}%
\pgfpathlineto{\pgfqpoint{2.293991in}{0.726970in}}%
\pgfpathlineto{\pgfqpoint{2.294805in}{1.466843in}}%
\pgfpathlineto{\pgfqpoint{2.295607in}{0.773689in}}%
\pgfpathlineto{\pgfqpoint{2.296482in}{1.487017in}}%
\pgfpathlineto{\pgfqpoint{2.297733in}{0.712636in}}%
\pgfpathlineto{\pgfqpoint{2.298168in}{1.511449in}}%
\pgfpathlineto{\pgfqpoint{2.299070in}{0.742894in}}%
\pgfpathlineto{\pgfqpoint{2.299941in}{1.592274in}}%
\pgfpathlineto{\pgfqpoint{2.300760in}{0.892264in}}%
\pgfpathlineto{\pgfqpoint{2.301584in}{1.504441in}}%
\pgfpathlineto{\pgfqpoint{2.302404in}{0.752247in}}%
\pgfpathlineto{\pgfqpoint{2.303598in}{1.483258in}}%
\pgfpathlineto{\pgfqpoint{2.304364in}{0.745064in}}%
\pgfpathlineto{\pgfqpoint{2.305204in}{1.497899in}}%
\pgfpathlineto{\pgfqpoint{2.305803in}{0.646837in}}%
\pgfpathlineto{\pgfqpoint{2.306677in}{1.561958in}}%
\pgfpathlineto{\pgfqpoint{2.307544in}{0.803699in}}%
\pgfpathlineto{\pgfqpoint{2.308713in}{1.597399in}}%
\pgfpathlineto{\pgfqpoint{2.309367in}{0.817270in}}%
\pgfpathlineto{\pgfqpoint{2.310220in}{1.575583in}}%
\pgfpathlineto{\pgfqpoint{2.311133in}{0.721365in}}%
\pgfpathlineto{\pgfqpoint{2.311755in}{1.508310in}}%
\pgfpathlineto{\pgfqpoint{2.313322in}{0.780787in}}%
\pgfpathlineto{\pgfqpoint{2.313438in}{1.250256in}}%
\pgfpathlineto{\pgfqpoint{2.313438in}{1.250256in}}%
\pgfusepath{stroke}%
\end{pgfscope}%
\begin{pgfscope}%
\pgfsetrectcap%
\pgfsetmiterjoin%
\pgfsetlinewidth{0.803000pt}%
\definecolor{currentstroke}{rgb}{0.000000,0.000000,0.000000}%
\pgfsetstrokecolor{currentstroke}%
\pgfsetdash{}{0pt}%
\pgfpathmoveto{\pgfqpoint{0.530716in}{0.416447in}}%
\pgfpathlineto{\pgfqpoint{0.530716in}{2.398330in}}%
\pgfusepath{stroke}%
\end{pgfscope}%
\begin{pgfscope}%
\pgfsetrectcap%
\pgfsetmiterjoin%
\pgfsetlinewidth{0.803000pt}%
\definecolor{currentstroke}{rgb}{0.000000,0.000000,0.000000}%
\pgfsetstrokecolor{currentstroke}%
\pgfsetdash{}{0pt}%
\pgfpathmoveto{\pgfqpoint{2.398330in}{0.416447in}}%
\pgfpathlineto{\pgfqpoint{2.398330in}{2.398330in}}%
\pgfusepath{stroke}%
\end{pgfscope}%
\begin{pgfscope}%
\pgfsetrectcap%
\pgfsetmiterjoin%
\pgfsetlinewidth{0.803000pt}%
\definecolor{currentstroke}{rgb}{0.000000,0.000000,0.000000}%
\pgfsetstrokecolor{currentstroke}%
\pgfsetdash{}{0pt}%
\pgfpathmoveto{\pgfqpoint{0.530716in}{0.416447in}}%
\pgfpathlineto{\pgfqpoint{2.398330in}{0.416447in}}%
\pgfusepath{stroke}%
\end{pgfscope}%
\begin{pgfscope}%
\pgfsetrectcap%
\pgfsetmiterjoin%
\pgfsetlinewidth{0.803000pt}%
\definecolor{currentstroke}{rgb}{0.000000,0.000000,0.000000}%
\pgfsetstrokecolor{currentstroke}%
\pgfsetdash{}{0pt}%
\pgfpathmoveto{\pgfqpoint{0.530716in}{2.398330in}}%
\pgfpathlineto{\pgfqpoint{2.398330in}{2.398330in}}%
\pgfusepath{stroke}%
\end{pgfscope}%
\begin{pgfscope}%
\pgfsetbuttcap%
\pgfsetmiterjoin%
\definecolor{currentfill}{rgb}{1.000000,1.000000,1.000000}%
\pgfsetfillcolor{currentfill}%
\pgfsetfillopacity{0.800000}%
\pgfsetlinewidth{1.003750pt}%
\definecolor{currentstroke}{rgb}{0.800000,0.800000,0.800000}%
\pgfsetstrokecolor{currentstroke}%
\pgfsetstrokeopacity{0.800000}%
\pgfsetdash{}{0pt}%
\pgfpathmoveto{\pgfqpoint{0.608494in}{2.154552in}}%
\pgfpathlineto{\pgfqpoint{1.608827in}{2.154552in}}%
\pgfpathquadraticcurveto{\pgfqpoint{1.631049in}{2.154552in}}{\pgfqpoint{1.631049in}{2.176775in}}%
\pgfpathlineto{\pgfqpoint{1.631049in}{2.320552in}}%
\pgfpathquadraticcurveto{\pgfqpoint{1.631049in}{2.342774in}}{\pgfqpoint{1.608827in}{2.342774in}}%
\pgfpathlineto{\pgfqpoint{0.608494in}{2.342774in}}%
\pgfpathquadraticcurveto{\pgfqpoint{0.586272in}{2.342774in}}{\pgfqpoint{0.586272in}{2.320552in}}%
\pgfpathlineto{\pgfqpoint{0.586272in}{2.176775in}}%
\pgfpathquadraticcurveto{\pgfqpoint{0.586272in}{2.154552in}}{\pgfqpoint{0.608494in}{2.154552in}}%
\pgfpathlineto{\pgfqpoint{0.608494in}{2.154552in}}%
\pgfpathclose%
\pgfusepath{stroke,fill}%
\end{pgfscope}%
\begin{pgfscope}%
\pgfsetrectcap%
\pgfsetroundjoin%
\pgfsetlinewidth{1.505625pt}%
\definecolor{currentstroke}{rgb}{0.000000,0.619608,0.450980}%
\pgfsetstrokecolor{currentstroke}%
\pgfsetdash{}{0pt}%
\pgfpathmoveto{\pgfqpoint{0.630716in}{2.259441in}}%
\pgfpathlineto{\pgfqpoint{0.741827in}{2.259441in}}%
\pgfpathlineto{\pgfqpoint{0.852938in}{2.259441in}}%
\pgfusepath{stroke}%
\end{pgfscope}%
\begin{pgfscope}%
\definecolor{textcolor}{rgb}{0.000000,0.000000,0.000000}%
\pgfsetstrokecolor{textcolor}%
\pgfsetfillcolor{textcolor}%
\pgftext[x=0.941827in,y=2.220552in,left,base]{\color{textcolor}\rmfamily\fontsize{8.000000}{9.600000}\selectfont Flicker noise}%
\end{pgfscope}%
\end{pgfpicture}%
\makeatother%
\endgroup%

        } % scalebox
        \caption{Flicker noise}
    \end{subfigure}
    \begin{subfigure}{0.32\linewidth}
        \centering
        \scalebox{0.75}{%
            %% Creator: Matplotlib, PGF backend
%%
%% To include the figure in your LaTeX document, write
%%   \input{<filename>.pgf}
%%
%% Make sure the required packages are loaded in your preamble
%%   \usepackage{pgf}
%%
%% Also ensure that all the required font packages are loaded; for instance,
%% the lmodern package is sometimes necessary when using math font.
%%   \usepackage{lmodern}
%%
%% Figures using additional raster images can only be included by \input if
%% they are in the same directory as the main LaTeX file. For loading figures
%% from other directories you can use the `import` package
%%   \usepackage{import}
%%
%% and then include the figures with
%%   \import{<path to file>}{<filename>.pgf}
%%
%% Matplotlib used the following preamble
%%   \usepackage{siunitx}
%%   \usepackage{fontspec}
%%   \makeatletter\@ifpackageloaded{underscore}{}{\usepackage[strings]{underscore}}\makeatother
%%
\begingroup%
\makeatletter%
\begin{pgfpicture}%
\pgfpathrectangle{\pgfpointorigin}{\pgfqpoint{2.440000in}{2.440000in}}%
\pgfusepath{use as bounding box, clip}%
\begin{pgfscope}%
\pgfsetbuttcap%
\pgfsetmiterjoin%
\definecolor{currentfill}{rgb}{1.000000,1.000000,1.000000}%
\pgfsetfillcolor{currentfill}%
\pgfsetlinewidth{0.000000pt}%
\definecolor{currentstroke}{rgb}{1.000000,1.000000,1.000000}%
\pgfsetstrokecolor{currentstroke}%
\pgfsetdash{}{0pt}%
\pgfpathmoveto{\pgfqpoint{0.000000in}{0.000000in}}%
\pgfpathlineto{\pgfqpoint{2.440000in}{0.000000in}}%
\pgfpathlineto{\pgfqpoint{2.440000in}{2.440000in}}%
\pgfpathlineto{\pgfqpoint{0.000000in}{2.440000in}}%
\pgfpathlineto{\pgfqpoint{0.000000in}{0.000000in}}%
\pgfpathclose%
\pgfusepath{fill}%
\end{pgfscope}%
\begin{pgfscope}%
\pgfsetbuttcap%
\pgfsetmiterjoin%
\definecolor{currentfill}{rgb}{1.000000,1.000000,1.000000}%
\pgfsetfillcolor{currentfill}%
\pgfsetlinewidth{0.000000pt}%
\definecolor{currentstroke}{rgb}{0.000000,0.000000,0.000000}%
\pgfsetstrokecolor{currentstroke}%
\pgfsetstrokeopacity{0.000000}%
\pgfsetdash{}{0pt}%
\pgfpathmoveto{\pgfqpoint{0.530716in}{0.416447in}}%
\pgfpathlineto{\pgfqpoint{2.398330in}{0.416447in}}%
\pgfpathlineto{\pgfqpoint{2.398330in}{2.398330in}}%
\pgfpathlineto{\pgfqpoint{0.530716in}{2.398330in}}%
\pgfpathlineto{\pgfqpoint{0.530716in}{0.416447in}}%
\pgfpathclose%
\pgfusepath{fill}%
\end{pgfscope}%
\begin{pgfscope}%
\pgfpathrectangle{\pgfqpoint{0.530716in}{0.416447in}}{\pgfqpoint{1.867614in}{1.981883in}}%
\pgfusepath{clip}%
\pgfsetrectcap%
\pgfsetroundjoin%
\pgfsetlinewidth{0.803000pt}%
\definecolor{currentstroke}{rgb}{0.450000,0.450000,0.450000}%
\pgfsetstrokecolor{currentstroke}%
\pgfsetdash{}{0pt}%
\pgfpathmoveto{\pgfqpoint{0.615608in}{0.416447in}}%
\pgfpathlineto{\pgfqpoint{0.615608in}{2.398330in}}%
\pgfusepath{stroke}%
\end{pgfscope}%
\begin{pgfscope}%
\pgfsetbuttcap%
\pgfsetroundjoin%
\definecolor{currentfill}{rgb}{0.000000,0.000000,0.000000}%
\pgfsetfillcolor{currentfill}%
\pgfsetlinewidth{0.803000pt}%
\definecolor{currentstroke}{rgb}{0.000000,0.000000,0.000000}%
\pgfsetstrokecolor{currentstroke}%
\pgfsetdash{}{0pt}%
\pgfsys@defobject{currentmarker}{\pgfqpoint{0.000000in}{-0.048611in}}{\pgfqpoint{0.000000in}{0.000000in}}{%
\pgfpathmoveto{\pgfqpoint{0.000000in}{0.000000in}}%
\pgfpathlineto{\pgfqpoint{0.000000in}{-0.048611in}}%
\pgfusepath{stroke,fill}%
}%
\begin{pgfscope}%
\pgfsys@transformshift{0.615608in}{0.416447in}%
\pgfsys@useobject{currentmarker}{}%
\end{pgfscope}%
\end{pgfscope}%
\begin{pgfscope}%
\definecolor{textcolor}{rgb}{0.000000,0.000000,0.000000}%
\pgfsetstrokecolor{textcolor}%
\pgfsetfillcolor{textcolor}%
\pgftext[x=0.615608in,y=0.319225in,,top]{\color{textcolor}\rmfamily\fontsize{8.000000}{9.600000}\selectfont \(\displaystyle {0}\)}%
\end{pgfscope}%
\begin{pgfscope}%
\pgfpathrectangle{\pgfqpoint{0.530716in}{0.416447in}}{\pgfqpoint{1.867614in}{1.981883in}}%
\pgfusepath{clip}%
\pgfsetrectcap%
\pgfsetroundjoin%
\pgfsetlinewidth{0.803000pt}%
\definecolor{currentstroke}{rgb}{0.450000,0.450000,0.450000}%
\pgfsetstrokecolor{currentstroke}%
\pgfsetdash{}{0pt}%
\pgfpathmoveto{\pgfqpoint{1.121601in}{0.416447in}}%
\pgfpathlineto{\pgfqpoint{1.121601in}{2.398330in}}%
\pgfusepath{stroke}%
\end{pgfscope}%
\begin{pgfscope}%
\pgfsetbuttcap%
\pgfsetroundjoin%
\definecolor{currentfill}{rgb}{0.000000,0.000000,0.000000}%
\pgfsetfillcolor{currentfill}%
\pgfsetlinewidth{0.803000pt}%
\definecolor{currentstroke}{rgb}{0.000000,0.000000,0.000000}%
\pgfsetstrokecolor{currentstroke}%
\pgfsetdash{}{0pt}%
\pgfsys@defobject{currentmarker}{\pgfqpoint{0.000000in}{-0.048611in}}{\pgfqpoint{0.000000in}{0.000000in}}{%
\pgfpathmoveto{\pgfqpoint{0.000000in}{0.000000in}}%
\pgfpathlineto{\pgfqpoint{0.000000in}{-0.048611in}}%
\pgfusepath{stroke,fill}%
}%
\begin{pgfscope}%
\pgfsys@transformshift{1.121601in}{0.416447in}%
\pgfsys@useobject{currentmarker}{}%
\end{pgfscope}%
\end{pgfscope}%
\begin{pgfscope}%
\definecolor{textcolor}{rgb}{0.000000,0.000000,0.000000}%
\pgfsetstrokecolor{textcolor}%
\pgfsetfillcolor{textcolor}%
\pgftext[x=1.121601in,y=0.319225in,,top]{\color{textcolor}\rmfamily\fontsize{8.000000}{9.600000}\selectfont \(\displaystyle {10}\)}%
\end{pgfscope}%
\begin{pgfscope}%
\pgfpathrectangle{\pgfqpoint{0.530716in}{0.416447in}}{\pgfqpoint{1.867614in}{1.981883in}}%
\pgfusepath{clip}%
\pgfsetrectcap%
\pgfsetroundjoin%
\pgfsetlinewidth{0.803000pt}%
\definecolor{currentstroke}{rgb}{0.450000,0.450000,0.450000}%
\pgfsetstrokecolor{currentstroke}%
\pgfsetdash{}{0pt}%
\pgfpathmoveto{\pgfqpoint{1.627594in}{0.416447in}}%
\pgfpathlineto{\pgfqpoint{1.627594in}{2.398330in}}%
\pgfusepath{stroke}%
\end{pgfscope}%
\begin{pgfscope}%
\pgfsetbuttcap%
\pgfsetroundjoin%
\definecolor{currentfill}{rgb}{0.000000,0.000000,0.000000}%
\pgfsetfillcolor{currentfill}%
\pgfsetlinewidth{0.803000pt}%
\definecolor{currentstroke}{rgb}{0.000000,0.000000,0.000000}%
\pgfsetstrokecolor{currentstroke}%
\pgfsetdash{}{0pt}%
\pgfsys@defobject{currentmarker}{\pgfqpoint{0.000000in}{-0.048611in}}{\pgfqpoint{0.000000in}{0.000000in}}{%
\pgfpathmoveto{\pgfqpoint{0.000000in}{0.000000in}}%
\pgfpathlineto{\pgfqpoint{0.000000in}{-0.048611in}}%
\pgfusepath{stroke,fill}%
}%
\begin{pgfscope}%
\pgfsys@transformshift{1.627594in}{0.416447in}%
\pgfsys@useobject{currentmarker}{}%
\end{pgfscope}%
\end{pgfscope}%
\begin{pgfscope}%
\definecolor{textcolor}{rgb}{0.000000,0.000000,0.000000}%
\pgfsetstrokecolor{textcolor}%
\pgfsetfillcolor{textcolor}%
\pgftext[x=1.627594in,y=0.319225in,,top]{\color{textcolor}\rmfamily\fontsize{8.000000}{9.600000}\selectfont \(\displaystyle {20}\)}%
\end{pgfscope}%
\begin{pgfscope}%
\pgfpathrectangle{\pgfqpoint{0.530716in}{0.416447in}}{\pgfqpoint{1.867614in}{1.981883in}}%
\pgfusepath{clip}%
\pgfsetrectcap%
\pgfsetroundjoin%
\pgfsetlinewidth{0.803000pt}%
\definecolor{currentstroke}{rgb}{0.450000,0.450000,0.450000}%
\pgfsetstrokecolor{currentstroke}%
\pgfsetdash{}{0pt}%
\pgfpathmoveto{\pgfqpoint{2.133587in}{0.416447in}}%
\pgfpathlineto{\pgfqpoint{2.133587in}{2.398330in}}%
\pgfusepath{stroke}%
\end{pgfscope}%
\begin{pgfscope}%
\pgfsetbuttcap%
\pgfsetroundjoin%
\definecolor{currentfill}{rgb}{0.000000,0.000000,0.000000}%
\pgfsetfillcolor{currentfill}%
\pgfsetlinewidth{0.803000pt}%
\definecolor{currentstroke}{rgb}{0.000000,0.000000,0.000000}%
\pgfsetstrokecolor{currentstroke}%
\pgfsetdash{}{0pt}%
\pgfsys@defobject{currentmarker}{\pgfqpoint{0.000000in}{-0.048611in}}{\pgfqpoint{0.000000in}{0.000000in}}{%
\pgfpathmoveto{\pgfqpoint{0.000000in}{0.000000in}}%
\pgfpathlineto{\pgfqpoint{0.000000in}{-0.048611in}}%
\pgfusepath{stroke,fill}%
}%
\begin{pgfscope}%
\pgfsys@transformshift{2.133587in}{0.416447in}%
\pgfsys@useobject{currentmarker}{}%
\end{pgfscope}%
\end{pgfscope}%
\begin{pgfscope}%
\definecolor{textcolor}{rgb}{0.000000,0.000000,0.000000}%
\pgfsetstrokecolor{textcolor}%
\pgfsetfillcolor{textcolor}%
\pgftext[x=2.133587in,y=0.319225in,,top]{\color{textcolor}\rmfamily\fontsize{8.000000}{9.600000}\selectfont \(\displaystyle {30}\)}%
\end{pgfscope}%
\begin{pgfscope}%
\definecolor{textcolor}{rgb}{0.000000,0.000000,0.000000}%
\pgfsetstrokecolor{textcolor}%
\pgfsetfillcolor{textcolor}%
\pgftext[x=1.464523in,y=0.165003in,,top]{\color{textcolor}\rmfamily\fontsize{10.000000}{12.000000}\selectfont Time in \(\displaystyle \unit{\second}\)}%
\end{pgfscope}%
\begin{pgfscope}%
\pgfpathrectangle{\pgfqpoint{0.530716in}{0.416447in}}{\pgfqpoint{1.867614in}{1.981883in}}%
\pgfusepath{clip}%
\pgfsetrectcap%
\pgfsetroundjoin%
\pgfsetlinewidth{0.803000pt}%
\definecolor{currentstroke}{rgb}{0.450000,0.450000,0.450000}%
\pgfsetstrokecolor{currentstroke}%
\pgfsetdash{}{0pt}%
\pgfpathmoveto{\pgfqpoint{0.530716in}{0.416447in}}%
\pgfpathlineto{\pgfqpoint{2.398330in}{0.416447in}}%
\pgfusepath{stroke}%
\end{pgfscope}%
\begin{pgfscope}%
\pgfsetbuttcap%
\pgfsetroundjoin%
\definecolor{currentfill}{rgb}{0.000000,0.000000,0.000000}%
\pgfsetfillcolor{currentfill}%
\pgfsetlinewidth{0.803000pt}%
\definecolor{currentstroke}{rgb}{0.000000,0.000000,0.000000}%
\pgfsetstrokecolor{currentstroke}%
\pgfsetdash{}{0pt}%
\pgfsys@defobject{currentmarker}{\pgfqpoint{-0.048611in}{0.000000in}}{\pgfqpoint{-0.000000in}{0.000000in}}{%
\pgfpathmoveto{\pgfqpoint{-0.000000in}{0.000000in}}%
\pgfpathlineto{\pgfqpoint{-0.048611in}{0.000000in}}%
\pgfusepath{stroke,fill}%
}%
\begin{pgfscope}%
\pgfsys@transformshift{0.530716in}{0.416447in}%
\pgfsys@useobject{currentmarker}{}%
\end{pgfscope}%
\end{pgfscope}%
\begin{pgfscope}%
\definecolor{textcolor}{rgb}{0.000000,0.000000,0.000000}%
\pgfsetstrokecolor{textcolor}%
\pgfsetfillcolor{textcolor}%
\pgftext[x=0.223614in, y=0.377892in, left, base]{\color{textcolor}\rmfamily\fontsize{8.000000}{9.600000}\selectfont \(\displaystyle {\ensuremath{-}50}\)}%
\end{pgfscope}%
\begin{pgfscope}%
\pgfpathrectangle{\pgfqpoint{0.530716in}{0.416447in}}{\pgfqpoint{1.867614in}{1.981883in}}%
\pgfusepath{clip}%
\pgfsetrectcap%
\pgfsetroundjoin%
\pgfsetlinewidth{0.803000pt}%
\definecolor{currentstroke}{rgb}{0.450000,0.450000,0.450000}%
\pgfsetstrokecolor{currentstroke}%
\pgfsetdash{}{0pt}%
\pgfpathmoveto{\pgfqpoint{0.530716in}{0.847292in}}%
\pgfpathlineto{\pgfqpoint{2.398330in}{0.847292in}}%
\pgfusepath{stroke}%
\end{pgfscope}%
\begin{pgfscope}%
\pgfsetbuttcap%
\pgfsetroundjoin%
\definecolor{currentfill}{rgb}{0.000000,0.000000,0.000000}%
\pgfsetfillcolor{currentfill}%
\pgfsetlinewidth{0.803000pt}%
\definecolor{currentstroke}{rgb}{0.000000,0.000000,0.000000}%
\pgfsetstrokecolor{currentstroke}%
\pgfsetdash{}{0pt}%
\pgfsys@defobject{currentmarker}{\pgfqpoint{-0.048611in}{0.000000in}}{\pgfqpoint{-0.000000in}{0.000000in}}{%
\pgfpathmoveto{\pgfqpoint{-0.000000in}{0.000000in}}%
\pgfpathlineto{\pgfqpoint{-0.048611in}{0.000000in}}%
\pgfusepath{stroke,fill}%
}%
\begin{pgfscope}%
\pgfsys@transformshift{0.530716in}{0.847292in}%
\pgfsys@useobject{currentmarker}{}%
\end{pgfscope}%
\end{pgfscope}%
\begin{pgfscope}%
\definecolor{textcolor}{rgb}{0.000000,0.000000,0.000000}%
\pgfsetstrokecolor{textcolor}%
\pgfsetfillcolor{textcolor}%
\pgftext[x=0.374465in, y=0.808736in, left, base]{\color{textcolor}\rmfamily\fontsize{8.000000}{9.600000}\selectfont \(\displaystyle {0}\)}%
\end{pgfscope}%
\begin{pgfscope}%
\pgfpathrectangle{\pgfqpoint{0.530716in}{0.416447in}}{\pgfqpoint{1.867614in}{1.981883in}}%
\pgfusepath{clip}%
\pgfsetrectcap%
\pgfsetroundjoin%
\pgfsetlinewidth{0.803000pt}%
\definecolor{currentstroke}{rgb}{0.450000,0.450000,0.450000}%
\pgfsetstrokecolor{currentstroke}%
\pgfsetdash{}{0pt}%
\pgfpathmoveto{\pgfqpoint{0.530716in}{1.278136in}}%
\pgfpathlineto{\pgfqpoint{2.398330in}{1.278136in}}%
\pgfusepath{stroke}%
\end{pgfscope}%
\begin{pgfscope}%
\pgfsetbuttcap%
\pgfsetroundjoin%
\definecolor{currentfill}{rgb}{0.000000,0.000000,0.000000}%
\pgfsetfillcolor{currentfill}%
\pgfsetlinewidth{0.803000pt}%
\definecolor{currentstroke}{rgb}{0.000000,0.000000,0.000000}%
\pgfsetstrokecolor{currentstroke}%
\pgfsetdash{}{0pt}%
\pgfsys@defobject{currentmarker}{\pgfqpoint{-0.048611in}{0.000000in}}{\pgfqpoint{-0.000000in}{0.000000in}}{%
\pgfpathmoveto{\pgfqpoint{-0.000000in}{0.000000in}}%
\pgfpathlineto{\pgfqpoint{-0.048611in}{0.000000in}}%
\pgfusepath{stroke,fill}%
}%
\begin{pgfscope}%
\pgfsys@transformshift{0.530716in}{1.278136in}%
\pgfsys@useobject{currentmarker}{}%
\end{pgfscope}%
\end{pgfscope}%
\begin{pgfscope}%
\definecolor{textcolor}{rgb}{0.000000,0.000000,0.000000}%
\pgfsetstrokecolor{textcolor}%
\pgfsetfillcolor{textcolor}%
\pgftext[x=0.315437in, y=1.239580in, left, base]{\color{textcolor}\rmfamily\fontsize{8.000000}{9.600000}\selectfont \(\displaystyle {50}\)}%
\end{pgfscope}%
\begin{pgfscope}%
\pgfpathrectangle{\pgfqpoint{0.530716in}{0.416447in}}{\pgfqpoint{1.867614in}{1.981883in}}%
\pgfusepath{clip}%
\pgfsetrectcap%
\pgfsetroundjoin%
\pgfsetlinewidth{0.803000pt}%
\definecolor{currentstroke}{rgb}{0.450000,0.450000,0.450000}%
\pgfsetstrokecolor{currentstroke}%
\pgfsetdash{}{0pt}%
\pgfpathmoveto{\pgfqpoint{0.530716in}{1.708980in}}%
\pgfpathlineto{\pgfqpoint{2.398330in}{1.708980in}}%
\pgfusepath{stroke}%
\end{pgfscope}%
\begin{pgfscope}%
\pgfsetbuttcap%
\pgfsetroundjoin%
\definecolor{currentfill}{rgb}{0.000000,0.000000,0.000000}%
\pgfsetfillcolor{currentfill}%
\pgfsetlinewidth{0.803000pt}%
\definecolor{currentstroke}{rgb}{0.000000,0.000000,0.000000}%
\pgfsetstrokecolor{currentstroke}%
\pgfsetdash{}{0pt}%
\pgfsys@defobject{currentmarker}{\pgfqpoint{-0.048611in}{0.000000in}}{\pgfqpoint{-0.000000in}{0.000000in}}{%
\pgfpathmoveto{\pgfqpoint{-0.000000in}{0.000000in}}%
\pgfpathlineto{\pgfqpoint{-0.048611in}{0.000000in}}%
\pgfusepath{stroke,fill}%
}%
\begin{pgfscope}%
\pgfsys@transformshift{0.530716in}{1.708980in}%
\pgfsys@useobject{currentmarker}{}%
\end{pgfscope}%
\end{pgfscope}%
\begin{pgfscope}%
\definecolor{textcolor}{rgb}{0.000000,0.000000,0.000000}%
\pgfsetstrokecolor{textcolor}%
\pgfsetfillcolor{textcolor}%
\pgftext[x=0.256408in, y=1.670424in, left, base]{\color{textcolor}\rmfamily\fontsize{8.000000}{9.600000}\selectfont \(\displaystyle {100}\)}%
\end{pgfscope}%
\begin{pgfscope}%
\pgfpathrectangle{\pgfqpoint{0.530716in}{0.416447in}}{\pgfqpoint{1.867614in}{1.981883in}}%
\pgfusepath{clip}%
\pgfsetrectcap%
\pgfsetroundjoin%
\pgfsetlinewidth{0.803000pt}%
\definecolor{currentstroke}{rgb}{0.450000,0.450000,0.450000}%
\pgfsetstrokecolor{currentstroke}%
\pgfsetdash{}{0pt}%
\pgfpathmoveto{\pgfqpoint{0.530716in}{2.139824in}}%
\pgfpathlineto{\pgfqpoint{2.398330in}{2.139824in}}%
\pgfusepath{stroke}%
\end{pgfscope}%
\begin{pgfscope}%
\pgfsetbuttcap%
\pgfsetroundjoin%
\definecolor{currentfill}{rgb}{0.000000,0.000000,0.000000}%
\pgfsetfillcolor{currentfill}%
\pgfsetlinewidth{0.803000pt}%
\definecolor{currentstroke}{rgb}{0.000000,0.000000,0.000000}%
\pgfsetstrokecolor{currentstroke}%
\pgfsetdash{}{0pt}%
\pgfsys@defobject{currentmarker}{\pgfqpoint{-0.048611in}{0.000000in}}{\pgfqpoint{-0.000000in}{0.000000in}}{%
\pgfpathmoveto{\pgfqpoint{-0.000000in}{0.000000in}}%
\pgfpathlineto{\pgfqpoint{-0.048611in}{0.000000in}}%
\pgfusepath{stroke,fill}%
}%
\begin{pgfscope}%
\pgfsys@transformshift{0.530716in}{2.139824in}%
\pgfsys@useobject{currentmarker}{}%
\end{pgfscope}%
\end{pgfscope}%
\begin{pgfscope}%
\definecolor{textcolor}{rgb}{0.000000,0.000000,0.000000}%
\pgfsetstrokecolor{textcolor}%
\pgfsetfillcolor{textcolor}%
\pgftext[x=0.256408in, y=2.101268in, left, base]{\color{textcolor}\rmfamily\fontsize{8.000000}{9.600000}\selectfont \(\displaystyle {150}\)}%
\end{pgfscope}%
\begin{pgfscope}%
\definecolor{textcolor}{rgb}{0.000000,0.000000,0.000000}%
\pgfsetstrokecolor{textcolor}%
\pgfsetfillcolor{textcolor}%
\pgftext[x=0.168059in,y=1.407389in,,bottom,rotate=90.000000]{\color{textcolor}\rmfamily\fontsize{10.000000}{12.000000}\selectfont Ampl. in arb. unit}%
\end{pgfscope}%
\begin{pgfscope}%
\pgfpathrectangle{\pgfqpoint{0.530716in}{0.416447in}}{\pgfqpoint{1.867614in}{1.981883in}}%
\pgfusepath{clip}%
\pgfsetrectcap%
\pgfsetroundjoin%
\pgfsetlinewidth{1.505625pt}%
\definecolor{currentstroke}{rgb}{0.835294,0.368627,0.000000}%
\pgfsetstrokecolor{currentstroke}%
\pgfsetdash{}{0pt}%
\pgfpathmoveto{\pgfqpoint{0.615608in}{0.847192in}}%
\pgfpathlineto{\pgfqpoint{0.616991in}{0.912399in}}%
\pgfpathlineto{\pgfqpoint{0.617593in}{0.858162in}}%
\pgfpathlineto{\pgfqpoint{0.618780in}{0.893125in}}%
\pgfpathlineto{\pgfqpoint{0.619008in}{0.873298in}}%
\pgfpathlineto{\pgfqpoint{0.620091in}{0.853533in}}%
\pgfpathlineto{\pgfqpoint{0.621240in}{0.782582in}}%
\pgfpathlineto{\pgfqpoint{0.621975in}{0.820809in}}%
\pgfpathlineto{\pgfqpoint{0.623764in}{0.774618in}}%
\pgfpathlineto{\pgfqpoint{0.624498in}{0.803780in}}%
\pgfpathlineto{\pgfqpoint{0.625199in}{0.777324in}}%
\pgfpathlineto{\pgfqpoint{0.625897in}{0.828094in}}%
\pgfpathlineto{\pgfqpoint{0.627611in}{0.776936in}}%
\pgfpathlineto{\pgfqpoint{0.628531in}{0.815778in}}%
\pgfpathlineto{\pgfqpoint{0.630033in}{0.782712in}}%
\pgfpathlineto{\pgfqpoint{0.630089in}{0.795000in}}%
\pgfpathlineto{\pgfqpoint{0.631393in}{0.844249in}}%
\pgfpathlineto{\pgfqpoint{0.632195in}{0.816375in}}%
\pgfpathlineto{\pgfqpoint{0.634574in}{0.879697in}}%
\pgfpathlineto{\pgfqpoint{0.635492in}{0.830932in}}%
\pgfpathlineto{\pgfqpoint{0.636547in}{0.853672in}}%
\pgfpathlineto{\pgfqpoint{0.637623in}{0.813494in}}%
\pgfpathlineto{\pgfqpoint{0.637899in}{0.844635in}}%
\pgfpathlineto{\pgfqpoint{0.638562in}{0.820054in}}%
\pgfpathlineto{\pgfqpoint{0.639827in}{0.856492in}}%
\pgfpathlineto{\pgfqpoint{0.640363in}{0.827168in}}%
\pgfpathlineto{\pgfqpoint{0.641675in}{0.852242in}}%
\pgfpathlineto{\pgfqpoint{0.642527in}{0.816547in}}%
\pgfpathlineto{\pgfqpoint{0.644370in}{0.883923in}}%
\pgfpathlineto{\pgfqpoint{0.644797in}{0.846332in}}%
\pgfpathlineto{\pgfqpoint{0.645503in}{0.897508in}}%
\pgfpathlineto{\pgfqpoint{0.646796in}{0.860632in}}%
\pgfpathlineto{\pgfqpoint{0.647640in}{0.899646in}}%
\pgfpathlineto{\pgfqpoint{0.648542in}{0.887443in}}%
\pgfpathlineto{\pgfqpoint{0.649071in}{0.915719in}}%
\pgfpathlineto{\pgfqpoint{0.649830in}{0.908180in}}%
\pgfpathlineto{\pgfqpoint{0.652149in}{0.959887in}}%
\pgfpathlineto{\pgfqpoint{0.653224in}{0.965133in}}%
\pgfpathlineto{\pgfqpoint{0.654419in}{0.915033in}}%
\pgfpathlineto{\pgfqpoint{0.656000in}{1.016863in}}%
\pgfpathlineto{\pgfqpoint{0.656417in}{0.983166in}}%
\pgfpathlineto{\pgfqpoint{0.657633in}{0.984984in}}%
\pgfpathlineto{\pgfqpoint{0.659748in}{0.948536in}}%
\pgfpathlineto{\pgfqpoint{0.660378in}{0.991782in}}%
\pgfpathlineto{\pgfqpoint{0.661242in}{0.971668in}}%
\pgfpathlineto{\pgfqpoint{0.661558in}{1.000535in}}%
\pgfpathlineto{\pgfqpoint{0.662393in}{0.986134in}}%
\pgfpathlineto{\pgfqpoint{0.664237in}{1.076909in}}%
\pgfpathlineto{\pgfqpoint{0.665926in}{0.985134in}}%
\pgfpathlineto{\pgfqpoint{0.668069in}{1.022575in}}%
\pgfpathlineto{\pgfqpoint{0.668662in}{1.018731in}}%
\pgfpathlineto{\pgfqpoint{0.671074in}{0.909524in}}%
\pgfpathlineto{\pgfqpoint{0.671986in}{0.950198in}}%
\pgfpathlineto{\pgfqpoint{0.672608in}{0.906671in}}%
\pgfpathlineto{\pgfqpoint{0.674070in}{0.949617in}}%
\pgfpathlineto{\pgfqpoint{0.674608in}{0.911952in}}%
\pgfpathlineto{\pgfqpoint{0.675209in}{0.931977in}}%
\pgfpathlineto{\pgfqpoint{0.676312in}{0.898927in}}%
\pgfpathlineto{\pgfqpoint{0.677896in}{0.967199in}}%
\pgfpathlineto{\pgfqpoint{0.679084in}{0.934528in}}%
\pgfpathlineto{\pgfqpoint{0.679712in}{0.984113in}}%
\pgfpathlineto{\pgfqpoint{0.682113in}{0.904295in}}%
\pgfpathlineto{\pgfqpoint{0.683060in}{0.919990in}}%
\pgfpathlineto{\pgfqpoint{0.683784in}{0.874514in}}%
\pgfpathlineto{\pgfqpoint{0.685016in}{0.900272in}}%
\pgfpathlineto{\pgfqpoint{0.685426in}{0.847532in}}%
\pgfpathlineto{\pgfqpoint{0.686609in}{0.867313in}}%
\pgfpathlineto{\pgfqpoint{0.687488in}{0.836697in}}%
\pgfpathlineto{\pgfqpoint{0.688561in}{0.860995in}}%
\pgfpathlineto{\pgfqpoint{0.690642in}{0.835640in}}%
\pgfpathlineto{\pgfqpoint{0.691253in}{0.850849in}}%
\pgfpathlineto{\pgfqpoint{0.692864in}{0.794726in}}%
\pgfpathlineto{\pgfqpoint{0.692943in}{0.806030in}}%
\pgfpathlineto{\pgfqpoint{0.694544in}{0.790664in}}%
\pgfpathlineto{\pgfqpoint{0.694895in}{0.813634in}}%
\pgfpathlineto{\pgfqpoint{0.696297in}{0.782340in}}%
\pgfpathlineto{\pgfqpoint{0.696772in}{0.803331in}}%
\pgfpathlineto{\pgfqpoint{0.697731in}{0.784892in}}%
\pgfpathlineto{\pgfqpoint{0.698296in}{0.741562in}}%
\pgfpathlineto{\pgfqpoint{0.698924in}{0.758005in}}%
\pgfpathlineto{\pgfqpoint{0.700254in}{0.725883in}}%
\pgfpathlineto{\pgfqpoint{0.702204in}{0.653762in}}%
\pgfpathlineto{\pgfqpoint{0.702519in}{0.691547in}}%
\pgfpathlineto{\pgfqpoint{0.703939in}{0.659841in}}%
\pgfpathlineto{\pgfqpoint{0.704221in}{0.691215in}}%
\pgfpathlineto{\pgfqpoint{0.705149in}{0.656076in}}%
\pgfpathlineto{\pgfqpoint{0.706206in}{0.714214in}}%
\pgfpathlineto{\pgfqpoint{0.706609in}{0.682318in}}%
\pgfpathlineto{\pgfqpoint{0.708149in}{0.653515in}}%
\pgfpathlineto{\pgfqpoint{0.709300in}{0.716763in}}%
\pgfpathlineto{\pgfqpoint{0.710262in}{0.679371in}}%
\pgfpathlineto{\pgfqpoint{0.711307in}{0.688778in}}%
\pgfpathlineto{\pgfqpoint{0.711752in}{0.636750in}}%
\pgfpathlineto{\pgfqpoint{0.713185in}{0.618630in}}%
\pgfpathlineto{\pgfqpoint{0.713686in}{0.662821in}}%
\pgfpathlineto{\pgfqpoint{0.714185in}{0.644266in}}%
\pgfpathlineto{\pgfqpoint{0.715313in}{0.635155in}}%
\pgfpathlineto{\pgfqpoint{0.717592in}{0.706642in}}%
\pgfpathlineto{\pgfqpoint{0.718520in}{0.675316in}}%
\pgfpathlineto{\pgfqpoint{0.719621in}{0.719760in}}%
\pgfpathlineto{\pgfqpoint{0.720408in}{0.691915in}}%
\pgfpathlineto{\pgfqpoint{0.721879in}{0.775179in}}%
\pgfpathlineto{\pgfqpoint{0.723972in}{0.717894in}}%
\pgfpathlineto{\pgfqpoint{0.724622in}{0.764033in}}%
\pgfpathlineto{\pgfqpoint{0.725352in}{0.731121in}}%
\pgfpathlineto{\pgfqpoint{0.728343in}{0.832260in}}%
\pgfpathlineto{\pgfqpoint{0.729363in}{0.788544in}}%
\pgfpathlineto{\pgfqpoint{0.729504in}{0.804959in}}%
\pgfpathlineto{\pgfqpoint{0.730593in}{0.793220in}}%
\pgfpathlineto{\pgfqpoint{0.733517in}{0.877900in}}%
\pgfpathlineto{\pgfqpoint{0.735125in}{0.807415in}}%
\pgfpathlineto{\pgfqpoint{0.735715in}{0.815771in}}%
\pgfpathlineto{\pgfqpoint{0.737768in}{0.901398in}}%
\pgfpathlineto{\pgfqpoint{0.738079in}{0.886067in}}%
\pgfpathlineto{\pgfqpoint{0.739223in}{0.908566in}}%
\pgfpathlineto{\pgfqpoint{0.739715in}{0.880215in}}%
\pgfpathlineto{\pgfqpoint{0.741259in}{0.908472in}}%
\pgfpathlineto{\pgfqpoint{0.742170in}{0.886027in}}%
\pgfpathlineto{\pgfqpoint{0.743016in}{0.827968in}}%
\pgfpathlineto{\pgfqpoint{0.743123in}{0.845073in}}%
\pgfpathlineto{\pgfqpoint{0.744356in}{0.839929in}}%
\pgfpathlineto{\pgfqpoint{0.744825in}{0.810783in}}%
\pgfpathlineto{\pgfqpoint{0.746263in}{0.800502in}}%
\pgfpathlineto{\pgfqpoint{0.746621in}{0.822832in}}%
\pgfpathlineto{\pgfqpoint{0.747428in}{0.802224in}}%
\pgfpathlineto{\pgfqpoint{0.748318in}{0.830871in}}%
\pgfpathlineto{\pgfqpoint{0.749025in}{0.809481in}}%
\pgfpathlineto{\pgfqpoint{0.750386in}{0.844052in}}%
\pgfpathlineto{\pgfqpoint{0.751559in}{0.816836in}}%
\pgfpathlineto{\pgfqpoint{0.752695in}{0.865878in}}%
\pgfpathlineto{\pgfqpoint{0.753283in}{0.841231in}}%
\pgfpathlineto{\pgfqpoint{0.754950in}{0.853483in}}%
\pgfpathlineto{\pgfqpoint{0.755791in}{0.897239in}}%
\pgfpathlineto{\pgfqpoint{0.756450in}{0.850795in}}%
\pgfpathlineto{\pgfqpoint{0.757929in}{0.902505in}}%
\pgfpathlineto{\pgfqpoint{0.758923in}{0.863575in}}%
\pgfpathlineto{\pgfqpoint{0.759997in}{0.925621in}}%
\pgfpathlineto{\pgfqpoint{0.760709in}{0.876287in}}%
\pgfpathlineto{\pgfqpoint{0.761258in}{0.899479in}}%
\pgfpathlineto{\pgfqpoint{0.762083in}{0.913711in}}%
\pgfpathlineto{\pgfqpoint{0.763105in}{0.867991in}}%
\pgfpathlineto{\pgfqpoint{0.763565in}{0.902850in}}%
\pgfpathlineto{\pgfqpoint{0.765080in}{0.869552in}}%
\pgfpathlineto{\pgfqpoint{0.766476in}{0.942640in}}%
\pgfpathlineto{\pgfqpoint{0.767170in}{0.906816in}}%
\pgfpathlineto{\pgfqpoint{0.768724in}{0.951637in}}%
\pgfpathlineto{\pgfqpoint{0.769783in}{0.934906in}}%
\pgfpathlineto{\pgfqpoint{0.770995in}{0.870832in}}%
\pgfpathlineto{\pgfqpoint{0.771780in}{0.903127in}}%
\pgfpathlineto{\pgfqpoint{0.773620in}{0.837812in}}%
\pgfpathlineto{\pgfqpoint{0.774394in}{0.891472in}}%
\pgfpathlineto{\pgfqpoint{0.774999in}{0.864073in}}%
\pgfpathlineto{\pgfqpoint{0.775799in}{0.891254in}}%
\pgfpathlineto{\pgfqpoint{0.776811in}{0.843441in}}%
\pgfpathlineto{\pgfqpoint{0.777877in}{0.864350in}}%
\pgfpathlineto{\pgfqpoint{0.779303in}{0.944281in}}%
\pgfpathlineto{\pgfqpoint{0.780101in}{0.926143in}}%
\pgfpathlineto{\pgfqpoint{0.780963in}{0.944864in}}%
\pgfpathlineto{\pgfqpoint{0.781568in}{0.998410in}}%
\pgfpathlineto{\pgfqpoint{0.782919in}{0.971073in}}%
\pgfpathlineto{\pgfqpoint{0.783549in}{1.010125in}}%
\pgfpathlineto{\pgfqpoint{0.784144in}{0.980323in}}%
\pgfpathlineto{\pgfqpoint{0.784858in}{1.004495in}}%
\pgfpathlineto{\pgfqpoint{0.785583in}{0.963252in}}%
\pgfpathlineto{\pgfqpoint{0.787176in}{0.983982in}}%
\pgfpathlineto{\pgfqpoint{0.787643in}{0.959070in}}%
\pgfpathlineto{\pgfqpoint{0.788165in}{0.982877in}}%
\pgfpathlineto{\pgfqpoint{0.789083in}{0.963326in}}%
\pgfpathlineto{\pgfqpoint{0.790047in}{1.008555in}}%
\pgfpathlineto{\pgfqpoint{0.791395in}{0.975936in}}%
\pgfpathlineto{\pgfqpoint{0.793317in}{1.039856in}}%
\pgfpathlineto{\pgfqpoint{0.794253in}{1.014694in}}%
\pgfpathlineto{\pgfqpoint{0.796301in}{1.053157in}}%
\pgfpathlineto{\pgfqpoint{0.797216in}{1.021701in}}%
\pgfpathlineto{\pgfqpoint{0.798155in}{1.068852in}}%
\pgfpathlineto{\pgfqpoint{0.798834in}{1.054150in}}%
\pgfpathlineto{\pgfqpoint{0.799866in}{1.056099in}}%
\pgfpathlineto{\pgfqpoint{0.800472in}{1.127294in}}%
\pgfpathlineto{\pgfqpoint{0.801353in}{1.113917in}}%
\pgfpathlineto{\pgfqpoint{0.802263in}{1.156860in}}%
\pgfpathlineto{\pgfqpoint{0.802661in}{1.155418in}}%
\pgfpathlineto{\pgfqpoint{0.803805in}{1.102080in}}%
\pgfpathlineto{\pgfqpoint{0.804981in}{1.103986in}}%
\pgfpathlineto{\pgfqpoint{0.805858in}{1.169853in}}%
\pgfpathlineto{\pgfqpoint{0.806228in}{1.140875in}}%
\pgfpathlineto{\pgfqpoint{0.806942in}{1.152975in}}%
\pgfpathlineto{\pgfqpoint{0.808346in}{1.088150in}}%
\pgfpathlineto{\pgfqpoint{0.808506in}{1.104384in}}%
\pgfpathlineto{\pgfqpoint{0.809525in}{1.093777in}}%
\pgfpathlineto{\pgfqpoint{0.810345in}{1.134227in}}%
\pgfpathlineto{\pgfqpoint{0.812221in}{1.105351in}}%
\pgfpathlineto{\pgfqpoint{0.812990in}{1.141158in}}%
\pgfpathlineto{\pgfqpoint{0.813618in}{1.133407in}}%
\pgfpathlineto{\pgfqpoint{0.815298in}{1.175722in}}%
\pgfpathlineto{\pgfqpoint{0.815686in}{1.145439in}}%
\pgfpathlineto{\pgfqpoint{0.816907in}{1.138504in}}%
\pgfpathlineto{\pgfqpoint{0.818313in}{1.196348in}}%
\pgfpathlineto{\pgfqpoint{0.819318in}{1.133786in}}%
\pgfpathlineto{\pgfqpoint{0.820378in}{1.143580in}}%
\pgfpathlineto{\pgfqpoint{0.820845in}{1.119763in}}%
\pgfpathlineto{\pgfqpoint{0.822438in}{1.162022in}}%
\pgfpathlineto{\pgfqpoint{0.823301in}{1.119600in}}%
\pgfpathlineto{\pgfqpoint{0.824062in}{1.119721in}}%
\pgfpathlineto{\pgfqpoint{0.825474in}{1.170847in}}%
\pgfpathlineto{\pgfqpoint{0.827208in}{1.116500in}}%
\pgfpathlineto{\pgfqpoint{0.828569in}{1.152964in}}%
\pgfpathlineto{\pgfqpoint{0.829288in}{1.137651in}}%
\pgfpathlineto{\pgfqpoint{0.830502in}{1.143906in}}%
\pgfpathlineto{\pgfqpoint{0.831386in}{1.178065in}}%
\pgfpathlineto{\pgfqpoint{0.831928in}{1.155571in}}%
\pgfpathlineto{\pgfqpoint{0.832934in}{1.178264in}}%
\pgfpathlineto{\pgfqpoint{0.833946in}{1.164854in}}%
\pgfpathlineto{\pgfqpoint{0.835085in}{1.092464in}}%
\pgfpathlineto{\pgfqpoint{0.835706in}{1.117098in}}%
\pgfpathlineto{\pgfqpoint{0.836677in}{1.075848in}}%
\pgfpathlineto{\pgfqpoint{0.837532in}{1.109766in}}%
\pgfpathlineto{\pgfqpoint{0.838649in}{1.087650in}}%
\pgfpathlineto{\pgfqpoint{0.839333in}{1.117795in}}%
\pgfpathlineto{\pgfqpoint{0.840029in}{1.097583in}}%
\pgfpathlineto{\pgfqpoint{0.840818in}{1.138856in}}%
\pgfpathlineto{\pgfqpoint{0.843467in}{1.042479in}}%
\pgfpathlineto{\pgfqpoint{0.844291in}{1.061153in}}%
\pgfpathlineto{\pgfqpoint{0.845805in}{0.990600in}}%
\pgfpathlineto{\pgfqpoint{0.846731in}{1.014793in}}%
\pgfpathlineto{\pgfqpoint{0.847181in}{0.982341in}}%
\pgfpathlineto{\pgfqpoint{0.847647in}{1.023463in}}%
\pgfpathlineto{\pgfqpoint{0.848626in}{1.016799in}}%
\pgfpathlineto{\pgfqpoint{0.851444in}{1.075896in}}%
\pgfpathlineto{\pgfqpoint{0.852293in}{1.048712in}}%
\pgfpathlineto{\pgfqpoint{0.852776in}{1.076041in}}%
\pgfpathlineto{\pgfqpoint{0.853644in}{1.039569in}}%
\pgfpathlineto{\pgfqpoint{0.854708in}{1.074067in}}%
\pgfpathlineto{\pgfqpoint{0.855901in}{1.029989in}}%
\pgfpathlineto{\pgfqpoint{0.856892in}{1.068001in}}%
\pgfpathlineto{\pgfqpoint{0.857143in}{1.043996in}}%
\pgfpathlineto{\pgfqpoint{0.857799in}{1.055411in}}%
\pgfpathlineto{\pgfqpoint{0.859122in}{1.014566in}}%
\pgfpathlineto{\pgfqpoint{0.859585in}{1.054366in}}%
\pgfpathlineto{\pgfqpoint{0.860787in}{1.029087in}}%
\pgfpathlineto{\pgfqpoint{0.861994in}{1.056358in}}%
\pgfpathlineto{\pgfqpoint{0.862873in}{1.015756in}}%
\pgfpathlineto{\pgfqpoint{0.863596in}{1.037385in}}%
\pgfpathlineto{\pgfqpoint{0.863893in}{1.011282in}}%
\pgfpathlineto{\pgfqpoint{0.865136in}{1.019269in}}%
\pgfpathlineto{\pgfqpoint{0.865457in}{0.987721in}}%
\pgfpathlineto{\pgfqpoint{0.866503in}{0.990131in}}%
\pgfpathlineto{\pgfqpoint{0.867151in}{0.948704in}}%
\pgfpathlineto{\pgfqpoint{0.868503in}{0.953229in}}%
\pgfpathlineto{\pgfqpoint{0.870670in}{0.877357in}}%
\pgfpathlineto{\pgfqpoint{0.872073in}{0.910482in}}%
\pgfpathlineto{\pgfqpoint{0.872361in}{0.888315in}}%
\pgfpathlineto{\pgfqpoint{0.873962in}{0.933817in}}%
\pgfpathlineto{\pgfqpoint{0.875297in}{0.846194in}}%
\pgfpathlineto{\pgfqpoint{0.876039in}{0.856185in}}%
\pgfpathlineto{\pgfqpoint{0.877234in}{0.905523in}}%
\pgfpathlineto{\pgfqpoint{0.877904in}{0.844296in}}%
\pgfpathlineto{\pgfqpoint{0.878834in}{0.878015in}}%
\pgfpathlineto{\pgfqpoint{0.879070in}{0.860772in}}%
\pgfpathlineto{\pgfqpoint{0.879988in}{0.877213in}}%
\pgfpathlineto{\pgfqpoint{0.880895in}{0.848690in}}%
\pgfpathlineto{\pgfqpoint{0.882341in}{0.894522in}}%
\pgfpathlineto{\pgfqpoint{0.882990in}{0.842333in}}%
\pgfpathlineto{\pgfqpoint{0.883638in}{0.861555in}}%
\pgfpathlineto{\pgfqpoint{0.884882in}{0.832833in}}%
\pgfpathlineto{\pgfqpoint{0.886514in}{0.873601in}}%
\pgfpathlineto{\pgfqpoint{0.887111in}{0.835338in}}%
\pgfpathlineto{\pgfqpoint{0.887572in}{0.843972in}}%
\pgfpathlineto{\pgfqpoint{0.888887in}{0.808377in}}%
\pgfpathlineto{\pgfqpoint{0.889647in}{0.844867in}}%
\pgfpathlineto{\pgfqpoint{0.890477in}{0.822134in}}%
\pgfpathlineto{\pgfqpoint{0.891017in}{0.868532in}}%
\pgfpathlineto{\pgfqpoint{0.891790in}{0.839619in}}%
\pgfpathlineto{\pgfqpoint{0.892936in}{0.833071in}}%
\pgfpathlineto{\pgfqpoint{0.893533in}{0.803030in}}%
\pgfpathlineto{\pgfqpoint{0.895134in}{0.778878in}}%
\pgfpathlineto{\pgfqpoint{0.895816in}{0.816312in}}%
\pgfpathlineto{\pgfqpoint{0.897622in}{0.764917in}}%
\pgfpathlineto{\pgfqpoint{0.898032in}{0.797442in}}%
\pgfpathlineto{\pgfqpoint{0.898689in}{0.790419in}}%
\pgfpathlineto{\pgfqpoint{0.900146in}{0.850904in}}%
\pgfpathlineto{\pgfqpoint{0.901384in}{0.780857in}}%
\pgfpathlineto{\pgfqpoint{0.902347in}{0.808350in}}%
\pgfpathlineto{\pgfqpoint{0.902831in}{0.768389in}}%
\pgfpathlineto{\pgfqpoint{0.904270in}{0.771297in}}%
\pgfpathlineto{\pgfqpoint{0.904587in}{0.720840in}}%
\pgfpathlineto{\pgfqpoint{0.905635in}{0.742573in}}%
\pgfpathlineto{\pgfqpoint{0.907244in}{0.673266in}}%
\pgfpathlineto{\pgfqpoint{0.908373in}{0.705893in}}%
\pgfpathlineto{\pgfqpoint{0.909521in}{0.692711in}}%
\pgfpathlineto{\pgfqpoint{0.910338in}{0.728487in}}%
\pgfpathlineto{\pgfqpoint{0.910794in}{0.709571in}}%
\pgfpathlineto{\pgfqpoint{0.912172in}{0.758031in}}%
\pgfpathlineto{\pgfqpoint{0.913023in}{0.730181in}}%
\pgfpathlineto{\pgfqpoint{0.913499in}{0.756304in}}%
\pgfpathlineto{\pgfqpoint{0.914000in}{0.712657in}}%
\pgfpathlineto{\pgfqpoint{0.914760in}{0.735028in}}%
\pgfpathlineto{\pgfqpoint{0.916003in}{0.711202in}}%
\pgfpathlineto{\pgfqpoint{0.916484in}{0.746084in}}%
\pgfpathlineto{\pgfqpoint{0.917275in}{0.716819in}}%
\pgfpathlineto{\pgfqpoint{0.918601in}{0.767077in}}%
\pgfpathlineto{\pgfqpoint{0.919235in}{0.737946in}}%
\pgfpathlineto{\pgfqpoint{0.919992in}{0.780244in}}%
\pgfpathlineto{\pgfqpoint{0.920724in}{0.766810in}}%
\pgfpathlineto{\pgfqpoint{0.922284in}{0.772590in}}%
\pgfpathlineto{\pgfqpoint{0.922828in}{0.811081in}}%
\pgfpathlineto{\pgfqpoint{0.923908in}{0.770323in}}%
\pgfpathlineto{\pgfqpoint{0.924643in}{0.811184in}}%
\pgfpathlineto{\pgfqpoint{0.926805in}{0.719243in}}%
\pgfpathlineto{\pgfqpoint{0.927897in}{0.785517in}}%
\pgfpathlineto{\pgfqpoint{0.928419in}{0.753694in}}%
\pgfpathlineto{\pgfqpoint{0.929890in}{0.752445in}}%
\pgfpathlineto{\pgfqpoint{0.930506in}{0.805185in}}%
\pgfpathlineto{\pgfqpoint{0.931693in}{0.761133in}}%
\pgfpathlineto{\pgfqpoint{0.934224in}{0.822428in}}%
\pgfpathlineto{\pgfqpoint{0.935093in}{0.803935in}}%
\pgfpathlineto{\pgfqpoint{0.935941in}{0.834224in}}%
\pgfpathlineto{\pgfqpoint{0.936338in}{0.793264in}}%
\pgfpathlineto{\pgfqpoint{0.936971in}{0.831791in}}%
\pgfpathlineto{\pgfqpoint{0.938195in}{0.781282in}}%
\pgfpathlineto{\pgfqpoint{0.940121in}{0.860196in}}%
\pgfpathlineto{\pgfqpoint{0.940601in}{0.825465in}}%
\pgfpathlineto{\pgfqpoint{0.941599in}{0.868643in}}%
\pgfpathlineto{\pgfqpoint{0.942153in}{0.837179in}}%
\pgfpathlineto{\pgfqpoint{0.942824in}{0.849787in}}%
\pgfpathlineto{\pgfqpoint{0.944153in}{0.789255in}}%
\pgfpathlineto{\pgfqpoint{0.945161in}{0.798383in}}%
\pgfpathlineto{\pgfqpoint{0.945817in}{0.835199in}}%
\pgfpathlineto{\pgfqpoint{0.946627in}{0.793158in}}%
\pgfpathlineto{\pgfqpoint{0.947242in}{0.830981in}}%
\pgfpathlineto{\pgfqpoint{0.948645in}{0.831275in}}%
\pgfpathlineto{\pgfqpoint{0.949328in}{0.800211in}}%
\pgfpathlineto{\pgfqpoint{0.949931in}{0.818872in}}%
\pgfpathlineto{\pgfqpoint{0.951067in}{0.793808in}}%
\pgfpathlineto{\pgfqpoint{0.951630in}{0.820370in}}%
\pgfpathlineto{\pgfqpoint{0.952958in}{0.816917in}}%
\pgfpathlineto{\pgfqpoint{0.955040in}{0.734237in}}%
\pgfpathlineto{\pgfqpoint{0.956066in}{0.702966in}}%
\pgfpathlineto{\pgfqpoint{0.957481in}{0.752113in}}%
\pgfpathlineto{\pgfqpoint{0.958620in}{0.723297in}}%
\pgfpathlineto{\pgfqpoint{0.958949in}{0.747302in}}%
\pgfpathlineto{\pgfqpoint{0.960258in}{0.760413in}}%
\pgfpathlineto{\pgfqpoint{0.961877in}{0.692085in}}%
\pgfpathlineto{\pgfqpoint{0.963120in}{0.711397in}}%
\pgfpathlineto{\pgfqpoint{0.963536in}{0.671198in}}%
\pgfpathlineto{\pgfqpoint{0.964861in}{0.663003in}}%
\pgfpathlineto{\pgfqpoint{0.965633in}{0.703241in}}%
\pgfpathlineto{\pgfqpoint{0.966280in}{0.666964in}}%
\pgfpathlineto{\pgfqpoint{0.967401in}{0.704754in}}%
\pgfpathlineto{\pgfqpoint{0.968163in}{0.672898in}}%
\pgfpathlineto{\pgfqpoint{0.968901in}{0.696973in}}%
\pgfpathlineto{\pgfqpoint{0.970810in}{0.643812in}}%
\pgfpathlineto{\pgfqpoint{0.971304in}{0.677965in}}%
\pgfpathlineto{\pgfqpoint{0.971791in}{0.641334in}}%
\pgfpathlineto{\pgfqpoint{0.972832in}{0.670497in}}%
\pgfpathlineto{\pgfqpoint{0.973620in}{0.627575in}}%
\pgfpathlineto{\pgfqpoint{0.974531in}{0.662754in}}%
\pgfpathlineto{\pgfqpoint{0.975862in}{0.622844in}}%
\pgfpathlineto{\pgfqpoint{0.976479in}{0.658469in}}%
\pgfpathlineto{\pgfqpoint{0.977214in}{0.631608in}}%
\pgfpathlineto{\pgfqpoint{0.978446in}{0.674082in}}%
\pgfpathlineto{\pgfqpoint{0.979786in}{0.632986in}}%
\pgfpathlineto{\pgfqpoint{0.980166in}{0.677427in}}%
\pgfpathlineto{\pgfqpoint{0.981371in}{0.685473in}}%
\pgfpathlineto{\pgfqpoint{0.982982in}{0.654053in}}%
\pgfpathlineto{\pgfqpoint{0.983706in}{0.682482in}}%
\pgfpathlineto{\pgfqpoint{0.984517in}{0.654485in}}%
\pgfpathlineto{\pgfqpoint{0.985429in}{0.683787in}}%
\pgfpathlineto{\pgfqpoint{0.986822in}{0.665620in}}%
\pgfpathlineto{\pgfqpoint{0.988408in}{0.696069in}}%
\pgfpathlineto{\pgfqpoint{0.989066in}{0.632515in}}%
\pgfpathlineto{\pgfqpoint{0.990625in}{0.662952in}}%
\pgfpathlineto{\pgfqpoint{0.991466in}{0.609428in}}%
\pgfpathlineto{\pgfqpoint{0.992695in}{0.615990in}}%
\pgfpathlineto{\pgfqpoint{0.993519in}{0.577495in}}%
\pgfpathlineto{\pgfqpoint{0.994586in}{0.591915in}}%
\pgfpathlineto{\pgfqpoint{0.995983in}{0.562479in}}%
\pgfpathlineto{\pgfqpoint{0.997710in}{0.657782in}}%
\pgfpathlineto{\pgfqpoint{0.998009in}{0.635517in}}%
\pgfpathlineto{\pgfqpoint{0.999169in}{0.635481in}}%
\pgfpathlineto{\pgfqpoint{0.999950in}{0.665023in}}%
\pgfpathlineto{\pgfqpoint{1.000997in}{0.648673in}}%
\pgfpathlineto{\pgfqpoint{1.001811in}{0.685947in}}%
\pgfpathlineto{\pgfqpoint{1.002324in}{0.676509in}}%
\pgfpathlineto{\pgfqpoint{1.003637in}{0.718145in}}%
\pgfpathlineto{\pgfqpoint{1.004124in}{0.687414in}}%
\pgfpathlineto{\pgfqpoint{1.005383in}{0.734706in}}%
\pgfpathlineto{\pgfqpoint{1.006713in}{0.660998in}}%
\pgfpathlineto{\pgfqpoint{1.007452in}{0.701998in}}%
\pgfpathlineto{\pgfqpoint{1.008480in}{0.684300in}}%
\pgfpathlineto{\pgfqpoint{1.009147in}{0.691271in}}%
\pgfpathlineto{\pgfqpoint{1.010362in}{0.772292in}}%
\pgfpathlineto{\pgfqpoint{1.011252in}{0.735609in}}%
\pgfpathlineto{\pgfqpoint{1.012111in}{0.763297in}}%
\pgfpathlineto{\pgfqpoint{1.014144in}{0.741911in}}%
\pgfpathlineto{\pgfqpoint{1.014431in}{0.743098in}}%
\pgfpathlineto{\pgfqpoint{1.015519in}{0.847218in}}%
\pgfpathlineto{\pgfqpoint{1.016167in}{0.825397in}}%
\pgfpathlineto{\pgfqpoint{1.016699in}{0.833507in}}%
\pgfpathlineto{\pgfqpoint{1.018220in}{0.818514in}}%
\pgfpathlineto{\pgfqpoint{1.020264in}{0.767364in}}%
\pgfpathlineto{\pgfqpoint{1.022153in}{0.803245in}}%
\pgfpathlineto{\pgfqpoint{1.022789in}{0.734656in}}%
\pgfpathlineto{\pgfqpoint{1.023524in}{0.760958in}}%
\pgfpathlineto{\pgfqpoint{1.024734in}{0.766902in}}%
\pgfpathlineto{\pgfqpoint{1.026042in}{0.694935in}}%
\pgfpathlineto{\pgfqpoint{1.026822in}{0.722689in}}%
\pgfpathlineto{\pgfqpoint{1.027337in}{0.671581in}}%
\pgfpathlineto{\pgfqpoint{1.029751in}{0.739051in}}%
\pgfpathlineto{\pgfqpoint{1.030472in}{0.722009in}}%
\pgfpathlineto{\pgfqpoint{1.031176in}{0.743286in}}%
\pgfpathlineto{\pgfqpoint{1.032696in}{0.723589in}}%
\pgfpathlineto{\pgfqpoint{1.034544in}{0.785814in}}%
\pgfpathlineto{\pgfqpoint{1.035972in}{0.804132in}}%
\pgfpathlineto{\pgfqpoint{1.037761in}{0.731541in}}%
\pgfpathlineto{\pgfqpoint{1.038197in}{0.762015in}}%
\pgfpathlineto{\pgfqpoint{1.039326in}{0.753665in}}%
\pgfpathlineto{\pgfqpoint{1.040061in}{0.692244in}}%
\pgfpathlineto{\pgfqpoint{1.040982in}{0.703811in}}%
\pgfpathlineto{\pgfqpoint{1.042375in}{0.785898in}}%
\pgfpathlineto{\pgfqpoint{1.043225in}{0.764288in}}%
\pgfpathlineto{\pgfqpoint{1.045045in}{0.817705in}}%
\pgfpathlineto{\pgfqpoint{1.047064in}{0.762504in}}%
\pgfpathlineto{\pgfqpoint{1.047965in}{0.802885in}}%
\pgfpathlineto{\pgfqpoint{1.049442in}{0.741158in}}%
\pgfpathlineto{\pgfqpoint{1.049982in}{0.780398in}}%
\pgfpathlineto{\pgfqpoint{1.051370in}{0.763711in}}%
\pgfpathlineto{\pgfqpoint{1.054819in}{0.906739in}}%
\pgfpathlineto{\pgfqpoint{1.055061in}{0.879609in}}%
\pgfpathlineto{\pgfqpoint{1.057980in}{0.950144in}}%
\pgfpathlineto{\pgfqpoint{1.059622in}{0.874328in}}%
\pgfpathlineto{\pgfqpoint{1.060472in}{0.880857in}}%
\pgfpathlineto{\pgfqpoint{1.061012in}{0.841558in}}%
\pgfpathlineto{\pgfqpoint{1.062953in}{0.903689in}}%
\pgfpathlineto{\pgfqpoint{1.064432in}{0.833555in}}%
\pgfpathlineto{\pgfqpoint{1.065695in}{0.833104in}}%
\pgfpathlineto{\pgfqpoint{1.066096in}{0.859784in}}%
\pgfpathlineto{\pgfqpoint{1.068307in}{0.878972in}}%
\pgfpathlineto{\pgfqpoint{1.068974in}{0.842707in}}%
\pgfpathlineto{\pgfqpoint{1.070224in}{0.877847in}}%
\pgfpathlineto{\pgfqpoint{1.070389in}{0.858217in}}%
\pgfpathlineto{\pgfqpoint{1.071284in}{0.823409in}}%
\pgfpathlineto{\pgfqpoint{1.071935in}{0.850511in}}%
\pgfpathlineto{\pgfqpoint{1.072936in}{0.825297in}}%
\pgfpathlineto{\pgfqpoint{1.073969in}{0.858901in}}%
\pgfpathlineto{\pgfqpoint{1.075009in}{0.814499in}}%
\pgfpathlineto{\pgfqpoint{1.075530in}{0.845391in}}%
\pgfpathlineto{\pgfqpoint{1.076708in}{0.812582in}}%
\pgfpathlineto{\pgfqpoint{1.077251in}{0.844574in}}%
\pgfpathlineto{\pgfqpoint{1.078107in}{0.810187in}}%
\pgfpathlineto{\pgfqpoint{1.079762in}{0.866533in}}%
\pgfpathlineto{\pgfqpoint{1.080618in}{0.816653in}}%
\pgfpathlineto{\pgfqpoint{1.081336in}{0.836158in}}%
\pgfpathlineto{\pgfqpoint{1.082563in}{0.819954in}}%
\pgfpathlineto{\pgfqpoint{1.084848in}{0.729974in}}%
\pgfpathlineto{\pgfqpoint{1.086008in}{0.754956in}}%
\pgfpathlineto{\pgfqpoint{1.086882in}{0.808456in}}%
\pgfpathlineto{\pgfqpoint{1.089251in}{0.746092in}}%
\pgfpathlineto{\pgfqpoint{1.090109in}{0.775061in}}%
\pgfpathlineto{\pgfqpoint{1.091042in}{0.735635in}}%
\pgfpathlineto{\pgfqpoint{1.092326in}{0.778410in}}%
\pgfpathlineto{\pgfqpoint{1.093584in}{0.719790in}}%
\pgfpathlineto{\pgfqpoint{1.094610in}{0.752258in}}%
\pgfpathlineto{\pgfqpoint{1.095139in}{0.718604in}}%
\pgfpathlineto{\pgfqpoint{1.095788in}{0.741767in}}%
\pgfpathlineto{\pgfqpoint{1.096585in}{0.721061in}}%
\pgfpathlineto{\pgfqpoint{1.097541in}{0.759006in}}%
\pgfpathlineto{\pgfqpoint{1.098792in}{0.720190in}}%
\pgfpathlineto{\pgfqpoint{1.099162in}{0.742991in}}%
\pgfpathlineto{\pgfqpoint{1.099975in}{0.730728in}}%
\pgfpathlineto{\pgfqpoint{1.102018in}{0.803965in}}%
\pgfpathlineto{\pgfqpoint{1.103206in}{0.759891in}}%
\pgfpathlineto{\pgfqpoint{1.104676in}{0.791466in}}%
\pgfpathlineto{\pgfqpoint{1.105465in}{0.770953in}}%
\pgfpathlineto{\pgfqpoint{1.106569in}{0.797913in}}%
\pgfpathlineto{\pgfqpoint{1.108425in}{0.700011in}}%
\pgfpathlineto{\pgfqpoint{1.108818in}{0.729323in}}%
\pgfpathlineto{\pgfqpoint{1.109330in}{0.692234in}}%
\pgfpathlineto{\pgfqpoint{1.110254in}{0.711700in}}%
\pgfpathlineto{\pgfqpoint{1.111414in}{0.689032in}}%
\pgfpathlineto{\pgfqpoint{1.112106in}{0.720786in}}%
\pgfpathlineto{\pgfqpoint{1.113243in}{0.683607in}}%
\pgfpathlineto{\pgfqpoint{1.114115in}{0.704406in}}%
\pgfpathlineto{\pgfqpoint{1.114961in}{0.665471in}}%
\pgfpathlineto{\pgfqpoint{1.115702in}{0.711765in}}%
\pgfpathlineto{\pgfqpoint{1.116537in}{0.679151in}}%
\pgfpathlineto{\pgfqpoint{1.116970in}{0.715797in}}%
\pgfpathlineto{\pgfqpoint{1.118468in}{0.695800in}}%
\pgfpathlineto{\pgfqpoint{1.118888in}{0.736347in}}%
\pgfpathlineto{\pgfqpoint{1.119526in}{0.729672in}}%
\pgfpathlineto{\pgfqpoint{1.120421in}{0.763755in}}%
\pgfpathlineto{\pgfqpoint{1.121317in}{0.747153in}}%
\pgfpathlineto{\pgfqpoint{1.122263in}{0.777657in}}%
\pgfpathlineto{\pgfqpoint{1.123203in}{0.733753in}}%
\pgfpathlineto{\pgfqpoint{1.124113in}{0.776788in}}%
\pgfpathlineto{\pgfqpoint{1.125921in}{0.701311in}}%
\pgfpathlineto{\pgfqpoint{1.127442in}{0.754962in}}%
\pgfpathlineto{\pgfqpoint{1.128727in}{0.750078in}}%
\pgfpathlineto{\pgfqpoint{1.129368in}{0.708796in}}%
\pgfpathlineto{\pgfqpoint{1.130024in}{0.734917in}}%
\pgfpathlineto{\pgfqpoint{1.130763in}{0.709230in}}%
\pgfpathlineto{\pgfqpoint{1.132084in}{0.741089in}}%
\pgfpathlineto{\pgfqpoint{1.132332in}{0.718391in}}%
\pgfpathlineto{\pgfqpoint{1.133117in}{0.745836in}}%
\pgfpathlineto{\pgfqpoint{1.134489in}{0.709490in}}%
\pgfpathlineto{\pgfqpoint{1.134833in}{0.735792in}}%
\pgfpathlineto{\pgfqpoint{1.135977in}{0.712132in}}%
\pgfpathlineto{\pgfqpoint{1.137164in}{0.755415in}}%
\pgfpathlineto{\pgfqpoint{1.138120in}{0.732757in}}%
\pgfpathlineto{\pgfqpoint{1.139727in}{0.775114in}}%
\pgfpathlineto{\pgfqpoint{1.140035in}{0.747078in}}%
\pgfpathlineto{\pgfqpoint{1.140773in}{0.776717in}}%
\pgfpathlineto{\pgfqpoint{1.142290in}{0.784903in}}%
\pgfpathlineto{\pgfqpoint{1.143654in}{0.829343in}}%
\pgfpathlineto{\pgfqpoint{1.144890in}{0.794436in}}%
\pgfpathlineto{\pgfqpoint{1.145246in}{0.829653in}}%
\pgfpathlineto{\pgfqpoint{1.146601in}{0.846443in}}%
\pgfpathlineto{\pgfqpoint{1.146911in}{0.817925in}}%
\pgfpathlineto{\pgfqpoint{1.148205in}{0.854983in}}%
\pgfpathlineto{\pgfqpoint{1.148849in}{0.837919in}}%
\pgfpathlineto{\pgfqpoint{1.149332in}{0.880222in}}%
\pgfpathlineto{\pgfqpoint{1.150201in}{0.836555in}}%
\pgfpathlineto{\pgfqpoint{1.152425in}{0.907345in}}%
\pgfpathlineto{\pgfqpoint{1.153738in}{0.841266in}}%
\pgfpathlineto{\pgfqpoint{1.155026in}{0.830662in}}%
\pgfpathlineto{\pgfqpoint{1.155585in}{0.862156in}}%
\pgfpathlineto{\pgfqpoint{1.156127in}{0.819800in}}%
\pgfpathlineto{\pgfqpoint{1.156940in}{0.837431in}}%
\pgfpathlineto{\pgfqpoint{1.158142in}{0.806924in}}%
\pgfpathlineto{\pgfqpoint{1.159033in}{0.844107in}}%
\pgfpathlineto{\pgfqpoint{1.160031in}{0.800521in}}%
\pgfpathlineto{\pgfqpoint{1.161142in}{0.843738in}}%
\pgfpathlineto{\pgfqpoint{1.161262in}{0.824027in}}%
\pgfpathlineto{\pgfqpoint{1.162192in}{0.800042in}}%
\pgfpathlineto{\pgfqpoint{1.163521in}{0.804427in}}%
\pgfpathlineto{\pgfqpoint{1.164064in}{0.866945in}}%
\pgfpathlineto{\pgfqpoint{1.165090in}{0.804678in}}%
\pgfpathlineto{\pgfqpoint{1.165455in}{0.843702in}}%
\pgfpathlineto{\pgfqpoint{1.167070in}{0.811414in}}%
\pgfpathlineto{\pgfqpoint{1.169319in}{0.861984in}}%
\pgfpathlineto{\pgfqpoint{1.170336in}{0.838289in}}%
\pgfpathlineto{\pgfqpoint{1.171317in}{0.862944in}}%
\pgfpathlineto{\pgfqpoint{1.172204in}{0.838499in}}%
\pgfpathlineto{\pgfqpoint{1.173759in}{0.874172in}}%
\pgfpathlineto{\pgfqpoint{1.175320in}{0.799870in}}%
\pgfpathlineto{\pgfqpoint{1.177590in}{0.912106in}}%
\pgfpathlineto{\pgfqpoint{1.178160in}{0.907114in}}%
\pgfpathlineto{\pgfqpoint{1.179522in}{0.967292in}}%
\pgfpathlineto{\pgfqpoint{1.180543in}{0.923939in}}%
\pgfpathlineto{\pgfqpoint{1.180776in}{0.955486in}}%
\pgfpathlineto{\pgfqpoint{1.182321in}{0.946062in}}%
\pgfpathlineto{\pgfqpoint{1.182679in}{0.903915in}}%
\pgfpathlineto{\pgfqpoint{1.184034in}{0.942191in}}%
\pgfpathlineto{\pgfqpoint{1.184224in}{0.924957in}}%
\pgfpathlineto{\pgfqpoint{1.185993in}{0.961904in}}%
\pgfpathlineto{\pgfqpoint{1.186657in}{0.915501in}}%
\pgfpathlineto{\pgfqpoint{1.187712in}{0.942764in}}%
\pgfpathlineto{\pgfqpoint{1.188427in}{0.914904in}}%
\pgfpathlineto{\pgfqpoint{1.190514in}{0.951705in}}%
\pgfpathlineto{\pgfqpoint{1.191174in}{0.895907in}}%
\pgfpathlineto{\pgfqpoint{1.192597in}{0.882609in}}%
\pgfpathlineto{\pgfqpoint{1.193439in}{0.927636in}}%
\pgfpathlineto{\pgfqpoint{1.194179in}{0.900856in}}%
\pgfpathlineto{\pgfqpoint{1.196390in}{0.988153in}}%
\pgfpathlineto{\pgfqpoint{1.197297in}{0.962874in}}%
\pgfpathlineto{\pgfqpoint{1.198821in}{1.002715in}}%
\pgfpathlineto{\pgfqpoint{1.200239in}{0.959304in}}%
\pgfpathlineto{\pgfqpoint{1.200785in}{0.983301in}}%
\pgfpathlineto{\pgfqpoint{1.201899in}{0.980222in}}%
\pgfpathlineto{\pgfqpoint{1.203554in}{0.922038in}}%
\pgfpathlineto{\pgfqpoint{1.204470in}{0.921415in}}%
\pgfpathlineto{\pgfqpoint{1.204946in}{0.961621in}}%
\pgfpathlineto{\pgfqpoint{1.205345in}{0.943653in}}%
\pgfpathlineto{\pgfqpoint{1.206287in}{0.976697in}}%
\pgfpathlineto{\pgfqpoint{1.207062in}{0.963694in}}%
\pgfpathlineto{\pgfqpoint{1.208445in}{1.003947in}}%
\pgfpathlineto{\pgfqpoint{1.208831in}{0.969816in}}%
\pgfpathlineto{\pgfqpoint{1.209855in}{1.008429in}}%
\pgfpathlineto{\pgfqpoint{1.210908in}{0.982514in}}%
\pgfpathlineto{\pgfqpoint{1.211539in}{1.026800in}}%
\pgfpathlineto{\pgfqpoint{1.212796in}{1.027754in}}%
\pgfpathlineto{\pgfqpoint{1.213981in}{0.970682in}}%
\pgfpathlineto{\pgfqpoint{1.215537in}{1.012528in}}%
\pgfpathlineto{\pgfqpoint{1.216213in}{0.977828in}}%
\pgfpathlineto{\pgfqpoint{1.217144in}{0.998699in}}%
\pgfpathlineto{\pgfqpoint{1.217482in}{0.954504in}}%
\pgfpathlineto{\pgfqpoint{1.218248in}{0.977261in}}%
\pgfpathlineto{\pgfqpoint{1.219199in}{0.908384in}}%
\pgfpathlineto{\pgfqpoint{1.219809in}{0.945637in}}%
\pgfpathlineto{\pgfqpoint{1.220825in}{0.911512in}}%
\pgfpathlineto{\pgfqpoint{1.222147in}{0.909012in}}%
\pgfpathlineto{\pgfqpoint{1.222967in}{0.857244in}}%
\pgfpathlineto{\pgfqpoint{1.224922in}{0.912191in}}%
\pgfpathlineto{\pgfqpoint{1.226567in}{0.886454in}}%
\pgfpathlineto{\pgfqpoint{1.227381in}{0.929652in}}%
\pgfpathlineto{\pgfqpoint{1.228109in}{0.925157in}}%
\pgfpathlineto{\pgfqpoint{1.230233in}{0.828128in}}%
\pgfpathlineto{\pgfqpoint{1.231495in}{0.804387in}}%
\pgfpathlineto{\pgfqpoint{1.231802in}{0.829301in}}%
\pgfpathlineto{\pgfqpoint{1.232929in}{0.810413in}}%
\pgfpathlineto{\pgfqpoint{1.233570in}{0.845746in}}%
\pgfpathlineto{\pgfqpoint{1.234694in}{0.840208in}}%
\pgfpathlineto{\pgfqpoint{1.236308in}{0.763586in}}%
\pgfpathlineto{\pgfqpoint{1.236937in}{0.791980in}}%
\pgfpathlineto{\pgfqpoint{1.237725in}{0.747864in}}%
\pgfpathlineto{\pgfqpoint{1.238656in}{0.776745in}}%
\pgfpathlineto{\pgfqpoint{1.239358in}{0.755473in}}%
\pgfpathlineto{\pgfqpoint{1.240186in}{0.780944in}}%
\pgfpathlineto{\pgfqpoint{1.241392in}{0.753739in}}%
\pgfpathlineto{\pgfqpoint{1.243851in}{0.852177in}}%
\pgfpathlineto{\pgfqpoint{1.245877in}{0.811828in}}%
\pgfpathlineto{\pgfqpoint{1.246410in}{0.855146in}}%
\pgfpathlineto{\pgfqpoint{1.247139in}{0.818518in}}%
\pgfpathlineto{\pgfqpoint{1.249628in}{0.926997in}}%
\pgfpathlineto{\pgfqpoint{1.250799in}{0.913575in}}%
\pgfpathlineto{\pgfqpoint{1.251665in}{0.958188in}}%
\pgfpathlineto{\pgfqpoint{1.252399in}{0.940180in}}%
\pgfpathlineto{\pgfqpoint{1.253544in}{0.928044in}}%
\pgfpathlineto{\pgfqpoint{1.254532in}{0.959173in}}%
\pgfpathlineto{\pgfqpoint{1.255360in}{0.949511in}}%
\pgfpathlineto{\pgfqpoint{1.256904in}{0.885320in}}%
\pgfpathlineto{\pgfqpoint{1.257212in}{0.911345in}}%
\pgfpathlineto{\pgfqpoint{1.258630in}{0.926751in}}%
\pgfpathlineto{\pgfqpoint{1.259004in}{0.890168in}}%
\pgfpathlineto{\pgfqpoint{1.260048in}{0.909083in}}%
\pgfpathlineto{\pgfqpoint{1.260631in}{0.862159in}}%
\pgfpathlineto{\pgfqpoint{1.263780in}{0.995311in}}%
\pgfpathlineto{\pgfqpoint{1.265541in}{0.914545in}}%
\pgfpathlineto{\pgfqpoint{1.266221in}{0.932010in}}%
\pgfpathlineto{\pgfqpoint{1.267124in}{0.887223in}}%
\pgfpathlineto{\pgfqpoint{1.267620in}{0.911910in}}%
\pgfpathlineto{\pgfqpoint{1.268518in}{0.877603in}}%
\pgfpathlineto{\pgfqpoint{1.269769in}{0.876576in}}%
\pgfpathlineto{\pgfqpoint{1.270268in}{0.924298in}}%
\pgfpathlineto{\pgfqpoint{1.271283in}{0.888027in}}%
\pgfpathlineto{\pgfqpoint{1.271638in}{0.916274in}}%
\pgfpathlineto{\pgfqpoint{1.272941in}{0.906888in}}%
\pgfpathlineto{\pgfqpoint{1.273327in}{0.878692in}}%
\pgfpathlineto{\pgfqpoint{1.274957in}{0.851508in}}%
\pgfpathlineto{\pgfqpoint{1.275678in}{0.896168in}}%
\pgfpathlineto{\pgfqpoint{1.276291in}{0.852697in}}%
\pgfpathlineto{\pgfqpoint{1.277548in}{0.862499in}}%
\pgfpathlineto{\pgfqpoint{1.277821in}{0.888505in}}%
\pgfpathlineto{\pgfqpoint{1.279660in}{0.841084in}}%
\pgfpathlineto{\pgfqpoint{1.280779in}{0.885848in}}%
\pgfpathlineto{\pgfqpoint{1.281059in}{0.856259in}}%
\pgfpathlineto{\pgfqpoint{1.282251in}{0.870498in}}%
\pgfpathlineto{\pgfqpoint{1.284373in}{0.794860in}}%
\pgfpathlineto{\pgfqpoint{1.285730in}{0.834465in}}%
\pgfpathlineto{\pgfqpoint{1.286421in}{0.831021in}}%
\pgfpathlineto{\pgfqpoint{1.287511in}{0.755048in}}%
\pgfpathlineto{\pgfqpoint{1.287982in}{0.782140in}}%
\pgfpathlineto{\pgfqpoint{1.288946in}{0.783011in}}%
\pgfpathlineto{\pgfqpoint{1.290151in}{0.809492in}}%
\pgfpathlineto{\pgfqpoint{1.291997in}{0.877152in}}%
\pgfpathlineto{\pgfqpoint{1.293625in}{0.812960in}}%
\pgfpathlineto{\pgfqpoint{1.294277in}{0.840201in}}%
\pgfpathlineto{\pgfqpoint{1.295326in}{0.802173in}}%
\pgfpathlineto{\pgfqpoint{1.295948in}{0.824937in}}%
\pgfpathlineto{\pgfqpoint{1.296373in}{0.789393in}}%
\pgfpathlineto{\pgfqpoint{1.297247in}{0.807320in}}%
\pgfpathlineto{\pgfqpoint{1.297979in}{0.776188in}}%
\pgfpathlineto{\pgfqpoint{1.299512in}{0.761110in}}%
\pgfpathlineto{\pgfqpoint{1.300928in}{0.819110in}}%
\pgfpathlineto{\pgfqpoint{1.301372in}{0.778369in}}%
\pgfpathlineto{\pgfqpoint{1.302336in}{0.803027in}}%
\pgfpathlineto{\pgfqpoint{1.303870in}{0.756338in}}%
\pgfpathlineto{\pgfqpoint{1.304456in}{0.760921in}}%
\pgfpathlineto{\pgfqpoint{1.304806in}{0.799827in}}%
\pgfpathlineto{\pgfqpoint{1.306421in}{0.852287in}}%
\pgfpathlineto{\pgfqpoint{1.307214in}{0.809797in}}%
\pgfpathlineto{\pgfqpoint{1.307744in}{0.849563in}}%
\pgfpathlineto{\pgfqpoint{1.308675in}{0.844470in}}%
\pgfpathlineto{\pgfqpoint{1.309510in}{0.894004in}}%
\pgfpathlineto{\pgfqpoint{1.310889in}{0.836540in}}%
\pgfpathlineto{\pgfqpoint{1.312738in}{0.908276in}}%
\pgfpathlineto{\pgfqpoint{1.313266in}{0.884940in}}%
\pgfpathlineto{\pgfqpoint{1.314126in}{0.901130in}}%
\pgfpathlineto{\pgfqpoint{1.315520in}{0.860879in}}%
\pgfpathlineto{\pgfqpoint{1.316214in}{0.904689in}}%
\pgfpathlineto{\pgfqpoint{1.316720in}{0.866915in}}%
\pgfpathlineto{\pgfqpoint{1.317855in}{0.878054in}}%
\pgfpathlineto{\pgfqpoint{1.318401in}{0.847671in}}%
\pgfpathlineto{\pgfqpoint{1.319466in}{0.867771in}}%
\pgfpathlineto{\pgfqpoint{1.320380in}{0.843764in}}%
\pgfpathlineto{\pgfqpoint{1.321032in}{0.863092in}}%
\pgfpathlineto{\pgfqpoint{1.322261in}{0.841116in}}%
\pgfpathlineto{\pgfqpoint{1.323091in}{0.876225in}}%
\pgfpathlineto{\pgfqpoint{1.323647in}{0.842696in}}%
\pgfpathlineto{\pgfqpoint{1.324847in}{0.885930in}}%
\pgfpathlineto{\pgfqpoint{1.325811in}{0.831262in}}%
\pgfpathlineto{\pgfqpoint{1.326859in}{0.850194in}}%
\pgfpathlineto{\pgfqpoint{1.327545in}{0.808219in}}%
\pgfpathlineto{\pgfqpoint{1.329200in}{0.871240in}}%
\pgfpathlineto{\pgfqpoint{1.329699in}{0.846674in}}%
\pgfpathlineto{\pgfqpoint{1.330467in}{0.874445in}}%
\pgfpathlineto{\pgfqpoint{1.332953in}{0.827936in}}%
\pgfpathlineto{\pgfqpoint{1.334443in}{0.860997in}}%
\pgfpathlineto{\pgfqpoint{1.334558in}{0.845384in}}%
\pgfpathlineto{\pgfqpoint{1.335495in}{0.836515in}}%
\pgfpathlineto{\pgfqpoint{1.337161in}{0.890256in}}%
\pgfpathlineto{\pgfqpoint{1.338250in}{0.861420in}}%
\pgfpathlineto{\pgfqpoint{1.338799in}{0.889252in}}%
\pgfpathlineto{\pgfqpoint{1.339884in}{0.835293in}}%
\pgfpathlineto{\pgfqpoint{1.341741in}{0.903082in}}%
\pgfpathlineto{\pgfqpoint{1.342346in}{0.879220in}}%
\pgfpathlineto{\pgfqpoint{1.343508in}{0.911733in}}%
\pgfpathlineto{\pgfqpoint{1.344773in}{0.852606in}}%
\pgfpathlineto{\pgfqpoint{1.345621in}{0.891131in}}%
\pgfpathlineto{\pgfqpoint{1.347244in}{0.841612in}}%
\pgfpathlineto{\pgfqpoint{1.347693in}{0.867808in}}%
\pgfpathlineto{\pgfqpoint{1.348862in}{0.836266in}}%
\pgfpathlineto{\pgfqpoint{1.349210in}{0.861811in}}%
\pgfpathlineto{\pgfqpoint{1.350067in}{0.834197in}}%
\pgfpathlineto{\pgfqpoint{1.350804in}{0.858831in}}%
\pgfpathlineto{\pgfqpoint{1.351647in}{0.826897in}}%
\pgfpathlineto{\pgfqpoint{1.352355in}{0.853314in}}%
\pgfpathlineto{\pgfqpoint{1.355141in}{0.778115in}}%
\pgfpathlineto{\pgfqpoint{1.356601in}{0.841678in}}%
\pgfpathlineto{\pgfqpoint{1.358412in}{0.781918in}}%
\pgfpathlineto{\pgfqpoint{1.359187in}{0.814736in}}%
\pgfpathlineto{\pgfqpoint{1.360267in}{0.744928in}}%
\pgfpathlineto{\pgfqpoint{1.361700in}{0.726574in}}%
\pgfpathlineto{\pgfqpoint{1.363359in}{0.809687in}}%
\pgfpathlineto{\pgfqpoint{1.363734in}{0.785399in}}%
\pgfpathlineto{\pgfqpoint{1.364710in}{0.807885in}}%
\pgfpathlineto{\pgfqpoint{1.365127in}{0.764938in}}%
\pgfpathlineto{\pgfqpoint{1.366430in}{0.771025in}}%
\pgfpathlineto{\pgfqpoint{1.367147in}{0.737247in}}%
\pgfpathlineto{\pgfqpoint{1.367657in}{0.765224in}}%
\pgfpathlineto{\pgfqpoint{1.368766in}{0.747181in}}%
\pgfpathlineto{\pgfqpoint{1.370404in}{0.809258in}}%
\pgfpathlineto{\pgfqpoint{1.371815in}{0.758607in}}%
\pgfpathlineto{\pgfqpoint{1.372023in}{0.775497in}}%
\pgfpathlineto{\pgfqpoint{1.372859in}{0.786168in}}%
\pgfpathlineto{\pgfqpoint{1.374159in}{0.722228in}}%
\pgfpathlineto{\pgfqpoint{1.374902in}{0.738427in}}%
\pgfpathlineto{\pgfqpoint{1.375946in}{0.732196in}}%
\pgfpathlineto{\pgfqpoint{1.376335in}{0.687256in}}%
\pgfpathlineto{\pgfqpoint{1.377837in}{0.722416in}}%
\pgfpathlineto{\pgfqpoint{1.378477in}{0.705722in}}%
\pgfpathlineto{\pgfqpoint{1.378987in}{0.749570in}}%
\pgfpathlineto{\pgfqpoint{1.380397in}{0.765956in}}%
\pgfpathlineto{\pgfqpoint{1.381239in}{0.757223in}}%
\pgfpathlineto{\pgfqpoint{1.381824in}{0.714343in}}%
\pgfpathlineto{\pgfqpoint{1.382238in}{0.741887in}}%
\pgfpathlineto{\pgfqpoint{1.383524in}{0.711027in}}%
\pgfpathlineto{\pgfqpoint{1.384578in}{0.721852in}}%
\pgfpathlineto{\pgfqpoint{1.385036in}{0.777844in}}%
\pgfpathlineto{\pgfqpoint{1.385964in}{0.752970in}}%
\pgfpathlineto{\pgfqpoint{1.386642in}{0.756174in}}%
\pgfpathlineto{\pgfqpoint{1.388819in}{0.822893in}}%
\pgfpathlineto{\pgfqpoint{1.390135in}{0.752246in}}%
\pgfpathlineto{\pgfqpoint{1.391310in}{0.775994in}}%
\pgfpathlineto{\pgfqpoint{1.391862in}{0.745632in}}%
\pgfpathlineto{\pgfqpoint{1.393064in}{0.765102in}}%
\pgfpathlineto{\pgfqpoint{1.395197in}{0.704033in}}%
\pgfpathlineto{\pgfqpoint{1.395693in}{0.711945in}}%
\pgfpathlineto{\pgfqpoint{1.397740in}{0.656514in}}%
\pgfpathlineto{\pgfqpoint{1.398243in}{0.696926in}}%
\pgfpathlineto{\pgfqpoint{1.399204in}{0.686013in}}%
\pgfpathlineto{\pgfqpoint{1.400403in}{0.728205in}}%
\pgfpathlineto{\pgfqpoint{1.401086in}{0.680771in}}%
\pgfpathlineto{\pgfqpoint{1.401642in}{0.706140in}}%
\pgfpathlineto{\pgfqpoint{1.404205in}{0.648064in}}%
\pgfpathlineto{\pgfqpoint{1.405512in}{0.676972in}}%
\pgfpathlineto{\pgfqpoint{1.406175in}{0.653121in}}%
\pgfpathlineto{\pgfqpoint{1.408926in}{0.754958in}}%
\pgfpathlineto{\pgfqpoint{1.409824in}{0.778435in}}%
\pgfpathlineto{\pgfqpoint{1.410378in}{0.732291in}}%
\pgfpathlineto{\pgfqpoint{1.410992in}{0.759623in}}%
\pgfpathlineto{\pgfqpoint{1.412736in}{0.700327in}}%
\pgfpathlineto{\pgfqpoint{1.413569in}{0.733148in}}%
\pgfpathlineto{\pgfqpoint{1.414834in}{0.705231in}}%
\pgfpathlineto{\pgfqpoint{1.415694in}{0.731251in}}%
\pgfpathlineto{\pgfqpoint{1.416926in}{0.685873in}}%
\pgfpathlineto{\pgfqpoint{1.417469in}{0.690000in}}%
\pgfpathlineto{\pgfqpoint{1.420082in}{0.809142in}}%
\pgfpathlineto{\pgfqpoint{1.420532in}{0.793901in}}%
\pgfpathlineto{\pgfqpoint{1.421318in}{0.819214in}}%
\pgfpathlineto{\pgfqpoint{1.422057in}{0.799775in}}%
\pgfpathlineto{\pgfqpoint{1.423650in}{0.818371in}}%
\pgfpathlineto{\pgfqpoint{1.424477in}{0.783799in}}%
\pgfpathlineto{\pgfqpoint{1.424868in}{0.807981in}}%
\pgfpathlineto{\pgfqpoint{1.425625in}{0.802014in}}%
\pgfpathlineto{\pgfqpoint{1.427724in}{0.704627in}}%
\pgfpathlineto{\pgfqpoint{1.428218in}{0.726954in}}%
\pgfpathlineto{\pgfqpoint{1.430185in}{0.675299in}}%
\pgfpathlineto{\pgfqpoint{1.430640in}{0.706055in}}%
\pgfpathlineto{\pgfqpoint{1.431702in}{0.662014in}}%
\pgfpathlineto{\pgfqpoint{1.433058in}{0.658256in}}%
\pgfpathlineto{\pgfqpoint{1.433594in}{0.690033in}}%
\pgfpathlineto{\pgfqpoint{1.434464in}{0.687567in}}%
\pgfpathlineto{\pgfqpoint{1.435450in}{0.647564in}}%
\pgfpathlineto{\pgfqpoint{1.436232in}{0.682758in}}%
\pgfpathlineto{\pgfqpoint{1.436549in}{0.647567in}}%
\pgfpathlineto{\pgfqpoint{1.437390in}{0.661415in}}%
\pgfpathlineto{\pgfqpoint{1.439531in}{0.581414in}}%
\pgfpathlineto{\pgfqpoint{1.439903in}{0.622783in}}%
\pgfpathlineto{\pgfqpoint{1.441189in}{0.628322in}}%
\pgfpathlineto{\pgfqpoint{1.442602in}{0.562781in}}%
\pgfpathlineto{\pgfqpoint{1.444292in}{0.604050in}}%
\pgfpathlineto{\pgfqpoint{1.445053in}{0.574310in}}%
\pgfpathlineto{\pgfqpoint{1.446500in}{0.580004in}}%
\pgfpathlineto{\pgfqpoint{1.447167in}{0.557232in}}%
\pgfpathlineto{\pgfqpoint{1.449307in}{0.637455in}}%
\pgfpathlineto{\pgfqpoint{1.451480in}{0.585268in}}%
\pgfpathlineto{\pgfqpoint{1.451826in}{0.624915in}}%
\pgfpathlineto{\pgfqpoint{1.453046in}{0.584042in}}%
\pgfpathlineto{\pgfqpoint{1.453496in}{0.587899in}}%
\pgfpathlineto{\pgfqpoint{1.460116in}{0.809181in}}%
\pgfpathlineto{\pgfqpoint{1.460720in}{0.774006in}}%
\pgfpathlineto{\pgfqpoint{1.461951in}{0.791846in}}%
\pgfpathlineto{\pgfqpoint{1.462215in}{0.759451in}}%
\pgfpathlineto{\pgfqpoint{1.464253in}{0.819390in}}%
\pgfpathlineto{\pgfqpoint{1.464797in}{0.785214in}}%
\pgfpathlineto{\pgfqpoint{1.465830in}{0.794819in}}%
\pgfpathlineto{\pgfqpoint{1.466455in}{0.767376in}}%
\pgfpathlineto{\pgfqpoint{1.467480in}{0.791788in}}%
\pgfpathlineto{\pgfqpoint{1.469379in}{0.762156in}}%
\pgfpathlineto{\pgfqpoint{1.469706in}{0.788462in}}%
\pgfpathlineto{\pgfqpoint{1.471030in}{0.777736in}}%
\pgfpathlineto{\pgfqpoint{1.471440in}{0.805700in}}%
\pgfpathlineto{\pgfqpoint{1.472174in}{0.782823in}}%
\pgfpathlineto{\pgfqpoint{1.473279in}{0.768233in}}%
\pgfpathlineto{\pgfqpoint{1.473872in}{0.803100in}}%
\pgfpathlineto{\pgfqpoint{1.474994in}{0.777180in}}%
\pgfpathlineto{\pgfqpoint{1.475597in}{0.803391in}}%
\pgfpathlineto{\pgfqpoint{1.476424in}{0.789503in}}%
\pgfpathlineto{\pgfqpoint{1.477378in}{0.833664in}}%
\pgfpathlineto{\pgfqpoint{1.478333in}{0.814330in}}%
\pgfpathlineto{\pgfqpoint{1.480443in}{0.868872in}}%
\pgfpathlineto{\pgfqpoint{1.481867in}{0.805386in}}%
\pgfpathlineto{\pgfqpoint{1.482548in}{0.833549in}}%
\pgfpathlineto{\pgfqpoint{1.483580in}{0.814849in}}%
\pgfpathlineto{\pgfqpoint{1.484073in}{0.839345in}}%
\pgfpathlineto{\pgfqpoint{1.485542in}{0.825087in}}%
\pgfpathlineto{\pgfqpoint{1.486403in}{0.865468in}}%
\pgfpathlineto{\pgfqpoint{1.486858in}{0.834262in}}%
\pgfpathlineto{\pgfqpoint{1.488573in}{0.882820in}}%
\pgfpathlineto{\pgfqpoint{1.489746in}{0.826105in}}%
\pgfpathlineto{\pgfqpoint{1.490663in}{0.860227in}}%
\pgfpathlineto{\pgfqpoint{1.491321in}{0.819061in}}%
\pgfpathlineto{\pgfqpoint{1.491839in}{0.845421in}}%
\pgfpathlineto{\pgfqpoint{1.492810in}{0.806146in}}%
\pgfpathlineto{\pgfqpoint{1.493449in}{0.860668in}}%
\pgfpathlineto{\pgfqpoint{1.494318in}{0.835295in}}%
\pgfpathlineto{\pgfqpoint{1.495351in}{0.902493in}}%
\pgfpathlineto{\pgfqpoint{1.496138in}{0.857005in}}%
\pgfpathlineto{\pgfqpoint{1.496894in}{0.870304in}}%
\pgfpathlineto{\pgfqpoint{1.497980in}{0.830608in}}%
\pgfpathlineto{\pgfqpoint{1.498593in}{0.851657in}}%
\pgfpathlineto{\pgfqpoint{1.499602in}{0.817005in}}%
\pgfpathlineto{\pgfqpoint{1.500599in}{0.852019in}}%
\pgfpathlineto{\pgfqpoint{1.501065in}{0.824574in}}%
\pgfpathlineto{\pgfqpoint{1.502016in}{0.850770in}}%
\pgfpathlineto{\pgfqpoint{1.503300in}{0.852027in}}%
\pgfpathlineto{\pgfqpoint{1.503671in}{0.803504in}}%
\pgfpathlineto{\pgfqpoint{1.504695in}{0.820287in}}%
\pgfpathlineto{\pgfqpoint{1.505443in}{0.767848in}}%
\pgfpathlineto{\pgfqpoint{1.506303in}{0.802565in}}%
\pgfpathlineto{\pgfqpoint{1.507250in}{0.791396in}}%
\pgfpathlineto{\pgfqpoint{1.509254in}{0.684476in}}%
\pgfpathlineto{\pgfqpoint{1.510061in}{0.713207in}}%
\pgfpathlineto{\pgfqpoint{1.510618in}{0.691370in}}%
\pgfpathlineto{\pgfqpoint{1.511283in}{0.733207in}}%
\pgfpathlineto{\pgfqpoint{1.512879in}{0.746031in}}%
\pgfpathlineto{\pgfqpoint{1.513595in}{0.796605in}}%
\pgfpathlineto{\pgfqpoint{1.514342in}{0.754153in}}%
\pgfpathlineto{\pgfqpoint{1.515087in}{0.806021in}}%
\pgfpathlineto{\pgfqpoint{1.515783in}{0.788291in}}%
\pgfpathlineto{\pgfqpoint{1.516420in}{0.818640in}}%
\pgfpathlineto{\pgfqpoint{1.517607in}{0.788685in}}%
\pgfpathlineto{\pgfqpoint{1.518653in}{0.844588in}}%
\pgfpathlineto{\pgfqpoint{1.520065in}{0.785248in}}%
\pgfpathlineto{\pgfqpoint{1.521406in}{0.775088in}}%
\pgfpathlineto{\pgfqpoint{1.523208in}{0.843335in}}%
\pgfpathlineto{\pgfqpoint{1.524191in}{0.828070in}}%
\pgfpathlineto{\pgfqpoint{1.524985in}{0.859470in}}%
\pgfpathlineto{\pgfqpoint{1.525985in}{0.841390in}}%
\pgfpathlineto{\pgfqpoint{1.526562in}{0.871536in}}%
\pgfpathlineto{\pgfqpoint{1.527440in}{0.863278in}}%
\pgfpathlineto{\pgfqpoint{1.528392in}{0.901278in}}%
\pgfpathlineto{\pgfqpoint{1.529810in}{0.899505in}}%
\pgfpathlineto{\pgfqpoint{1.531952in}{0.959268in}}%
\pgfpathlineto{\pgfqpoint{1.532808in}{0.924404in}}%
\pgfpathlineto{\pgfqpoint{1.534061in}{1.011719in}}%
\pgfpathlineto{\pgfqpoint{1.534279in}{0.992251in}}%
\pgfpathlineto{\pgfqpoint{1.535846in}{1.035807in}}%
\pgfpathlineto{\pgfqpoint{1.537020in}{0.981351in}}%
\pgfpathlineto{\pgfqpoint{1.538343in}{0.971953in}}%
\pgfpathlineto{\pgfqpoint{1.539291in}{1.021483in}}%
\pgfpathlineto{\pgfqpoint{1.539758in}{0.996823in}}%
\pgfpathlineto{\pgfqpoint{1.540807in}{1.017904in}}%
\pgfpathlineto{\pgfqpoint{1.541347in}{0.978130in}}%
\pgfpathlineto{\pgfqpoint{1.541996in}{0.999391in}}%
\pgfpathlineto{\pgfqpoint{1.544827in}{0.921486in}}%
\pgfpathlineto{\pgfqpoint{1.545352in}{0.941538in}}%
\pgfpathlineto{\pgfqpoint{1.546637in}{0.939114in}}%
\pgfpathlineto{\pgfqpoint{1.547383in}{0.898242in}}%
\pgfpathlineto{\pgfqpoint{1.548035in}{0.917169in}}%
\pgfpathlineto{\pgfqpoint{1.548731in}{0.895971in}}%
\pgfpathlineto{\pgfqpoint{1.550894in}{0.932148in}}%
\pgfpathlineto{\pgfqpoint{1.551449in}{0.878970in}}%
\pgfpathlineto{\pgfqpoint{1.552197in}{0.914131in}}%
\pgfpathlineto{\pgfqpoint{1.553347in}{0.893873in}}%
\pgfpathlineto{\pgfqpoint{1.554470in}{0.951666in}}%
\pgfpathlineto{\pgfqpoint{1.554801in}{0.924170in}}%
\pgfpathlineto{\pgfqpoint{1.555847in}{0.925887in}}%
\pgfpathlineto{\pgfqpoint{1.556738in}{0.934473in}}%
\pgfpathlineto{\pgfqpoint{1.557585in}{0.989585in}}%
\pgfpathlineto{\pgfqpoint{1.558674in}{0.945326in}}%
\pgfpathlineto{\pgfqpoint{1.560299in}{1.005683in}}%
\pgfpathlineto{\pgfqpoint{1.560709in}{0.962159in}}%
\pgfpathlineto{\pgfqpoint{1.562011in}{1.013229in}}%
\pgfpathlineto{\pgfqpoint{1.562671in}{0.989235in}}%
\pgfpathlineto{\pgfqpoint{1.563923in}{1.029365in}}%
\pgfpathlineto{\pgfqpoint{1.563955in}{1.017566in}}%
\pgfpathlineto{\pgfqpoint{1.565536in}{1.015669in}}%
\pgfpathlineto{\pgfqpoint{1.567212in}{0.949552in}}%
\pgfpathlineto{\pgfqpoint{1.567780in}{0.966901in}}%
\pgfpathlineto{\pgfqpoint{1.568411in}{0.936849in}}%
\pgfpathlineto{\pgfqpoint{1.570362in}{0.991474in}}%
\pgfpathlineto{\pgfqpoint{1.570904in}{0.944867in}}%
\pgfpathlineto{\pgfqpoint{1.571610in}{0.974981in}}%
\pgfpathlineto{\pgfqpoint{1.572517in}{0.949879in}}%
\pgfpathlineto{\pgfqpoint{1.574802in}{1.015517in}}%
\pgfpathlineto{\pgfqpoint{1.577054in}{0.928605in}}%
\pgfpathlineto{\pgfqpoint{1.578175in}{0.922377in}}%
\pgfpathlineto{\pgfqpoint{1.579206in}{0.960378in}}%
\pgfpathlineto{\pgfqpoint{1.579871in}{0.914385in}}%
\pgfpathlineto{\pgfqpoint{1.580523in}{0.960358in}}%
\pgfpathlineto{\pgfqpoint{1.580943in}{0.940174in}}%
\pgfpathlineto{\pgfqpoint{1.581846in}{0.955919in}}%
\pgfpathlineto{\pgfqpoint{1.582758in}{0.906535in}}%
\pgfpathlineto{\pgfqpoint{1.584318in}{0.936738in}}%
\pgfpathlineto{\pgfqpoint{1.584955in}{0.916625in}}%
\pgfpathlineto{\pgfqpoint{1.585527in}{0.954191in}}%
\pgfpathlineto{\pgfqpoint{1.586464in}{0.904054in}}%
\pgfpathlineto{\pgfqpoint{1.587166in}{0.946919in}}%
\pgfpathlineto{\pgfqpoint{1.588084in}{0.901331in}}%
\pgfpathlineto{\pgfqpoint{1.588773in}{0.927874in}}%
\pgfpathlineto{\pgfqpoint{1.590203in}{0.903158in}}%
\pgfpathlineto{\pgfqpoint{1.591819in}{0.826010in}}%
\pgfpathlineto{\pgfqpoint{1.592529in}{0.864363in}}%
\pgfpathlineto{\pgfqpoint{1.593212in}{0.831545in}}%
\pgfpathlineto{\pgfqpoint{1.593971in}{0.851723in}}%
\pgfpathlineto{\pgfqpoint{1.595324in}{0.808524in}}%
\pgfpathlineto{\pgfqpoint{1.596857in}{0.869276in}}%
\pgfpathlineto{\pgfqpoint{1.597621in}{0.842511in}}%
\pgfpathlineto{\pgfqpoint{1.598214in}{0.871108in}}%
\pgfpathlineto{\pgfqpoint{1.599511in}{0.868716in}}%
\pgfpathlineto{\pgfqpoint{1.600214in}{0.904543in}}%
\pgfpathlineto{\pgfqpoint{1.600589in}{0.864875in}}%
\pgfpathlineto{\pgfqpoint{1.601452in}{0.888504in}}%
\pgfpathlineto{\pgfqpoint{1.602232in}{0.854310in}}%
\pgfpathlineto{\pgfqpoint{1.603403in}{0.871186in}}%
\pgfpathlineto{\pgfqpoint{1.604496in}{0.826207in}}%
\pgfpathlineto{\pgfqpoint{1.604741in}{0.840569in}}%
\pgfpathlineto{\pgfqpoint{1.605959in}{0.818425in}}%
\pgfpathlineto{\pgfqpoint{1.606931in}{0.765397in}}%
\pgfpathlineto{\pgfqpoint{1.607424in}{0.803795in}}%
\pgfpathlineto{\pgfqpoint{1.608263in}{0.803533in}}%
\pgfpathlineto{\pgfqpoint{1.609737in}{0.860382in}}%
\pgfpathlineto{\pgfqpoint{1.610776in}{0.803024in}}%
\pgfpathlineto{\pgfqpoint{1.611876in}{0.792202in}}%
\pgfpathlineto{\pgfqpoint{1.612436in}{0.833913in}}%
\pgfpathlineto{\pgfqpoint{1.614079in}{0.839333in}}%
\pgfpathlineto{\pgfqpoint{1.614800in}{0.881943in}}%
\pgfpathlineto{\pgfqpoint{1.615023in}{0.852159in}}%
\pgfpathlineto{\pgfqpoint{1.615874in}{0.879313in}}%
\pgfpathlineto{\pgfqpoint{1.617266in}{0.846549in}}%
\pgfpathlineto{\pgfqpoint{1.618111in}{0.884921in}}%
\pgfpathlineto{\pgfqpoint{1.621496in}{0.790193in}}%
\pgfpathlineto{\pgfqpoint{1.621915in}{0.822011in}}%
\pgfpathlineto{\pgfqpoint{1.623607in}{0.773116in}}%
\pgfpathlineto{\pgfqpoint{1.625423in}{0.820455in}}%
\pgfpathlineto{\pgfqpoint{1.627656in}{0.735530in}}%
\pgfpathlineto{\pgfqpoint{1.628874in}{0.807104in}}%
\pgfpathlineto{\pgfqpoint{1.629667in}{0.793350in}}%
\pgfpathlineto{\pgfqpoint{1.630344in}{0.824848in}}%
\pgfpathlineto{\pgfqpoint{1.631881in}{0.769153in}}%
\pgfpathlineto{\pgfqpoint{1.632582in}{0.774371in}}%
\pgfpathlineto{\pgfqpoint{1.633489in}{0.823568in}}%
\pgfpathlineto{\pgfqpoint{1.633950in}{0.790229in}}%
\pgfpathlineto{\pgfqpoint{1.634919in}{0.772498in}}%
\pgfpathlineto{\pgfqpoint{1.636081in}{0.784155in}}%
\pgfpathlineto{\pgfqpoint{1.637578in}{0.853603in}}%
\pgfpathlineto{\pgfqpoint{1.638342in}{0.840785in}}%
\pgfpathlineto{\pgfqpoint{1.639347in}{0.861693in}}%
\pgfpathlineto{\pgfqpoint{1.639778in}{0.912740in}}%
\pgfpathlineto{\pgfqpoint{1.642351in}{0.821166in}}%
\pgfpathlineto{\pgfqpoint{1.643231in}{0.838410in}}%
\pgfpathlineto{\pgfqpoint{1.644645in}{0.780312in}}%
\pgfpathlineto{\pgfqpoint{1.646374in}{0.823099in}}%
\pgfpathlineto{\pgfqpoint{1.647813in}{0.760625in}}%
\pgfpathlineto{\pgfqpoint{1.649729in}{0.798172in}}%
\pgfpathlineto{\pgfqpoint{1.651700in}{0.711557in}}%
\pgfpathlineto{\pgfqpoint{1.653000in}{0.743022in}}%
\pgfpathlineto{\pgfqpoint{1.655062in}{0.704121in}}%
\pgfpathlineto{\pgfqpoint{1.656053in}{0.722680in}}%
\pgfpathlineto{\pgfqpoint{1.657906in}{0.661382in}}%
\pgfpathlineto{\pgfqpoint{1.658837in}{0.686942in}}%
\pgfpathlineto{\pgfqpoint{1.660823in}{0.626321in}}%
\pgfpathlineto{\pgfqpoint{1.661843in}{0.617389in}}%
\pgfpathlineto{\pgfqpoint{1.662628in}{0.574711in}}%
\pgfpathlineto{\pgfqpoint{1.663950in}{0.577737in}}%
\pgfpathlineto{\pgfqpoint{1.666062in}{0.628841in}}%
\pgfpathlineto{\pgfqpoint{1.666975in}{0.603786in}}%
\pgfpathlineto{\pgfqpoint{1.668382in}{0.616082in}}%
\pgfpathlineto{\pgfqpoint{1.669247in}{0.659085in}}%
\pgfpathlineto{\pgfqpoint{1.669605in}{0.623252in}}%
\pgfpathlineto{\pgfqpoint{1.670457in}{0.612767in}}%
\pgfpathlineto{\pgfqpoint{1.671745in}{0.650541in}}%
\pgfpathlineto{\pgfqpoint{1.672369in}{0.619940in}}%
\pgfpathlineto{\pgfqpoint{1.672804in}{0.645589in}}%
\pgfpathlineto{\pgfqpoint{1.674176in}{0.620295in}}%
\pgfpathlineto{\pgfqpoint{1.674825in}{0.652034in}}%
\pgfpathlineto{\pgfqpoint{1.675265in}{0.630696in}}%
\pgfpathlineto{\pgfqpoint{1.676802in}{0.609778in}}%
\pgfpathlineto{\pgfqpoint{1.678495in}{0.707073in}}%
\pgfpathlineto{\pgfqpoint{1.679004in}{0.718472in}}%
\pgfpathlineto{\pgfqpoint{1.680084in}{0.670348in}}%
\pgfpathlineto{\pgfqpoint{1.680728in}{0.697714in}}%
\pgfpathlineto{\pgfqpoint{1.681466in}{0.656088in}}%
\pgfpathlineto{\pgfqpoint{1.682734in}{0.657312in}}%
\pgfpathlineto{\pgfqpoint{1.684405in}{0.595316in}}%
\pgfpathlineto{\pgfqpoint{1.685873in}{0.663315in}}%
\pgfpathlineto{\pgfqpoint{1.686628in}{0.621586in}}%
\pgfpathlineto{\pgfqpoint{1.687162in}{0.651788in}}%
\pgfpathlineto{\pgfqpoint{1.688775in}{0.652135in}}%
\pgfpathlineto{\pgfqpoint{1.688959in}{0.630587in}}%
\pgfpathlineto{\pgfqpoint{1.689867in}{0.649545in}}%
\pgfpathlineto{\pgfqpoint{1.690952in}{0.649894in}}%
\pgfpathlineto{\pgfqpoint{1.691785in}{0.617065in}}%
\pgfpathlineto{\pgfqpoint{1.692567in}{0.633305in}}%
\pgfpathlineto{\pgfqpoint{1.694417in}{0.587948in}}%
\pgfpathlineto{\pgfqpoint{1.695631in}{0.549340in}}%
\pgfpathlineto{\pgfqpoint{1.696478in}{0.596595in}}%
\pgfpathlineto{\pgfqpoint{1.696864in}{0.576360in}}%
\pgfpathlineto{\pgfqpoint{1.697851in}{0.572178in}}%
\pgfpathlineto{\pgfqpoint{1.698370in}{0.626522in}}%
\pgfpathlineto{\pgfqpoint{1.699447in}{0.586204in}}%
\pgfpathlineto{\pgfqpoint{1.700220in}{0.613393in}}%
\pgfpathlineto{\pgfqpoint{1.701503in}{0.571212in}}%
\pgfpathlineto{\pgfqpoint{1.701838in}{0.608181in}}%
\pgfpathlineto{\pgfqpoint{1.702460in}{0.569683in}}%
\pgfpathlineto{\pgfqpoint{1.703365in}{0.600766in}}%
\pgfpathlineto{\pgfqpoint{1.704894in}{0.572629in}}%
\pgfpathlineto{\pgfqpoint{1.705658in}{0.604375in}}%
\pgfpathlineto{\pgfqpoint{1.706565in}{0.564320in}}%
\pgfpathlineto{\pgfqpoint{1.706978in}{0.591605in}}%
\pgfpathlineto{\pgfqpoint{1.708402in}{0.590391in}}%
\pgfpathlineto{\pgfqpoint{1.709153in}{0.628814in}}%
\pgfpathlineto{\pgfqpoint{1.709410in}{0.598027in}}%
\pgfpathlineto{\pgfqpoint{1.710149in}{0.615049in}}%
\pgfpathlineto{\pgfqpoint{1.711288in}{0.587098in}}%
\pgfpathlineto{\pgfqpoint{1.712516in}{0.596207in}}%
\pgfpathlineto{\pgfqpoint{1.713006in}{0.632559in}}%
\pgfpathlineto{\pgfqpoint{1.714715in}{0.587724in}}%
\pgfpathlineto{\pgfqpoint{1.715599in}{0.650782in}}%
\pgfpathlineto{\pgfqpoint{1.716070in}{0.616583in}}%
\pgfpathlineto{\pgfqpoint{1.717744in}{0.592574in}}%
\pgfpathlineto{\pgfqpoint{1.718457in}{0.638378in}}%
\pgfpathlineto{\pgfqpoint{1.719042in}{0.594357in}}%
\pgfpathlineto{\pgfqpoint{1.719535in}{0.618055in}}%
\pgfpathlineto{\pgfqpoint{1.720486in}{0.592002in}}%
\pgfpathlineto{\pgfqpoint{1.722665in}{0.662623in}}%
\pgfpathlineto{\pgfqpoint{1.723416in}{0.644865in}}%
\pgfpathlineto{\pgfqpoint{1.724521in}{0.708244in}}%
\pgfpathlineto{\pgfqpoint{1.724803in}{0.677404in}}%
\pgfpathlineto{\pgfqpoint{1.725525in}{0.683896in}}%
\pgfpathlineto{\pgfqpoint{1.726305in}{0.621101in}}%
\pgfpathlineto{\pgfqpoint{1.727154in}{0.652619in}}%
\pgfpathlineto{\pgfqpoint{1.727970in}{0.609338in}}%
\pgfpathlineto{\pgfqpoint{1.729035in}{0.611333in}}%
\pgfpathlineto{\pgfqpoint{1.729680in}{0.659727in}}%
\pgfpathlineto{\pgfqpoint{1.730819in}{0.649460in}}%
\pgfpathlineto{\pgfqpoint{1.731897in}{0.696779in}}%
\pgfpathlineto{\pgfqpoint{1.732989in}{0.670865in}}%
\pgfpathlineto{\pgfqpoint{1.733057in}{0.686135in}}%
\pgfpathlineto{\pgfqpoint{1.734706in}{0.716435in}}%
\pgfpathlineto{\pgfqpoint{1.735014in}{0.685042in}}%
\pgfpathlineto{\pgfqpoint{1.736290in}{0.703475in}}%
\pgfpathlineto{\pgfqpoint{1.736839in}{0.666897in}}%
\pgfpathlineto{\pgfqpoint{1.737302in}{0.693330in}}%
\pgfpathlineto{\pgfqpoint{1.738351in}{0.678021in}}%
\pgfpathlineto{\pgfqpoint{1.739226in}{0.709243in}}%
\pgfpathlineto{\pgfqpoint{1.740626in}{0.678572in}}%
\pgfpathlineto{\pgfqpoint{1.742379in}{0.733517in}}%
\pgfpathlineto{\pgfqpoint{1.742726in}{0.700314in}}%
\pgfpathlineto{\pgfqpoint{1.743369in}{0.727386in}}%
\pgfpathlineto{\pgfqpoint{1.744752in}{0.715961in}}%
\pgfpathlineto{\pgfqpoint{1.745144in}{0.739110in}}%
\pgfpathlineto{\pgfqpoint{1.746156in}{0.717536in}}%
\pgfpathlineto{\pgfqpoint{1.746795in}{0.768689in}}%
\pgfpathlineto{\pgfqpoint{1.747719in}{0.720958in}}%
\pgfpathlineto{\pgfqpoint{1.748837in}{0.773487in}}%
\pgfpathlineto{\pgfqpoint{1.749569in}{0.742338in}}%
\pgfpathlineto{\pgfqpoint{1.750239in}{0.778549in}}%
\pgfpathlineto{\pgfqpoint{1.750968in}{0.765177in}}%
\pgfpathlineto{\pgfqpoint{1.752531in}{0.833182in}}%
\pgfpathlineto{\pgfqpoint{1.753114in}{0.808477in}}%
\pgfpathlineto{\pgfqpoint{1.753523in}{0.832044in}}%
\pgfpathlineto{\pgfqpoint{1.754382in}{0.811031in}}%
\pgfpathlineto{\pgfqpoint{1.755849in}{0.840933in}}%
\pgfpathlineto{\pgfqpoint{1.756503in}{0.798148in}}%
\pgfpathlineto{\pgfqpoint{1.757439in}{0.833149in}}%
\pgfpathlineto{\pgfqpoint{1.758440in}{0.795812in}}%
\pgfpathlineto{\pgfqpoint{1.759278in}{0.824367in}}%
\pgfpathlineto{\pgfqpoint{1.760523in}{0.775533in}}%
\pgfpathlineto{\pgfqpoint{1.761111in}{0.813990in}}%
\pgfpathlineto{\pgfqpoint{1.762147in}{0.786617in}}%
\pgfpathlineto{\pgfqpoint{1.763640in}{0.826837in}}%
\pgfpathlineto{\pgfqpoint{1.764378in}{0.803642in}}%
\pgfpathlineto{\pgfqpoint{1.766482in}{0.885427in}}%
\pgfpathlineto{\pgfqpoint{1.767714in}{0.935080in}}%
\pgfpathlineto{\pgfqpoint{1.770047in}{0.864404in}}%
\pgfpathlineto{\pgfqpoint{1.770851in}{0.912636in}}%
\pgfpathlineto{\pgfqpoint{1.771420in}{0.892648in}}%
\pgfpathlineto{\pgfqpoint{1.772550in}{0.917064in}}%
\pgfpathlineto{\pgfqpoint{1.773534in}{0.857535in}}%
\pgfpathlineto{\pgfqpoint{1.774005in}{0.869097in}}%
\pgfpathlineto{\pgfqpoint{1.775088in}{0.834296in}}%
\pgfpathlineto{\pgfqpoint{1.776040in}{0.887571in}}%
\pgfpathlineto{\pgfqpoint{1.776390in}{0.868703in}}%
\pgfpathlineto{\pgfqpoint{1.778065in}{0.927095in}}%
\pgfpathlineto{\pgfqpoint{1.778439in}{0.909893in}}%
\pgfpathlineto{\pgfqpoint{1.779740in}{0.893872in}}%
\pgfpathlineto{\pgfqpoint{1.780501in}{0.950274in}}%
\pgfpathlineto{\pgfqpoint{1.781067in}{0.906806in}}%
\pgfpathlineto{\pgfqpoint{1.781689in}{0.946455in}}%
\pgfpathlineto{\pgfqpoint{1.783531in}{0.868801in}}%
\pgfpathlineto{\pgfqpoint{1.785228in}{0.924310in}}%
\pgfpathlineto{\pgfqpoint{1.786398in}{0.856720in}}%
\pgfpathlineto{\pgfqpoint{1.789071in}{0.934546in}}%
\pgfpathlineto{\pgfqpoint{1.789576in}{0.902717in}}%
\pgfpathlineto{\pgfqpoint{1.790491in}{0.906172in}}%
\pgfpathlineto{\pgfqpoint{1.791159in}{0.940572in}}%
\pgfpathlineto{\pgfqpoint{1.792365in}{0.919402in}}%
\pgfpathlineto{\pgfqpoint{1.793767in}{0.994981in}}%
\pgfpathlineto{\pgfqpoint{1.794599in}{0.958408in}}%
\pgfpathlineto{\pgfqpoint{1.795801in}{0.964953in}}%
\pgfpathlineto{\pgfqpoint{1.796569in}{0.995618in}}%
\pgfpathlineto{\pgfqpoint{1.796994in}{0.965336in}}%
\pgfpathlineto{\pgfqpoint{1.798058in}{1.013882in}}%
\pgfpathlineto{\pgfqpoint{1.798828in}{0.999153in}}%
\pgfpathlineto{\pgfqpoint{1.799441in}{1.024620in}}%
\pgfpathlineto{\pgfqpoint{1.800642in}{1.002922in}}%
\pgfpathlineto{\pgfqpoint{1.801894in}{1.028734in}}%
\pgfpathlineto{\pgfqpoint{1.803068in}{0.992495in}}%
\pgfpathlineto{\pgfqpoint{1.805054in}{1.081204in}}%
\pgfpathlineto{\pgfqpoint{1.805382in}{1.053354in}}%
\pgfpathlineto{\pgfqpoint{1.806772in}{1.079006in}}%
\pgfpathlineto{\pgfqpoint{1.807561in}{1.055049in}}%
\pgfpathlineto{\pgfqpoint{1.808540in}{1.069921in}}%
\pgfpathlineto{\pgfqpoint{1.810117in}{1.121065in}}%
\pgfpathlineto{\pgfqpoint{1.810381in}{1.093366in}}%
\pgfpathlineto{\pgfqpoint{1.811330in}{1.098533in}}%
\pgfpathlineto{\pgfqpoint{1.812519in}{1.158931in}}%
\pgfpathlineto{\pgfqpoint{1.813775in}{1.104584in}}%
\pgfpathlineto{\pgfqpoint{1.815620in}{1.143415in}}%
\pgfpathlineto{\pgfqpoint{1.816451in}{1.117300in}}%
\pgfpathlineto{\pgfqpoint{1.818552in}{1.163244in}}%
\pgfpathlineto{\pgfqpoint{1.819151in}{1.116813in}}%
\pgfpathlineto{\pgfqpoint{1.820370in}{1.144072in}}%
\pgfpathlineto{\pgfqpoint{1.820846in}{1.103011in}}%
\pgfpathlineto{\pgfqpoint{1.822210in}{1.147060in}}%
\pgfpathlineto{\pgfqpoint{1.824020in}{1.072952in}}%
\pgfpathlineto{\pgfqpoint{1.825565in}{1.045275in}}%
\pgfpathlineto{\pgfqpoint{1.826437in}{1.069204in}}%
\pgfpathlineto{\pgfqpoint{1.827470in}{1.027022in}}%
\pgfpathlineto{\pgfqpoint{1.830388in}{1.106837in}}%
\pgfpathlineto{\pgfqpoint{1.830824in}{1.085684in}}%
\pgfpathlineto{\pgfqpoint{1.832086in}{1.102574in}}%
\pgfpathlineto{\pgfqpoint{1.832540in}{1.154785in}}%
\pgfpathlineto{\pgfqpoint{1.833981in}{1.145120in}}%
\pgfpathlineto{\pgfqpoint{1.835176in}{1.178175in}}%
\pgfpathlineto{\pgfqpoint{1.836888in}{1.264253in}}%
\pgfpathlineto{\pgfqpoint{1.838718in}{1.241180in}}%
\pgfpathlineto{\pgfqpoint{1.840403in}{1.349676in}}%
\pgfpathlineto{\pgfqpoint{1.841583in}{1.317710in}}%
\pgfpathlineto{\pgfqpoint{1.842468in}{1.378295in}}%
\pgfpathlineto{\pgfqpoint{1.843444in}{1.356292in}}%
\pgfpathlineto{\pgfqpoint{1.844628in}{1.425223in}}%
\pgfpathlineto{\pgfqpoint{1.845417in}{1.392823in}}%
\pgfpathlineto{\pgfqpoint{1.846795in}{1.406775in}}%
\pgfpathlineto{\pgfqpoint{1.847407in}{1.450531in}}%
\pgfpathlineto{\pgfqpoint{1.847829in}{1.432896in}}%
\pgfpathlineto{\pgfqpoint{1.849109in}{1.474449in}}%
\pgfpathlineto{\pgfqpoint{1.849814in}{1.451489in}}%
\pgfpathlineto{\pgfqpoint{1.850444in}{1.470561in}}%
\pgfpathlineto{\pgfqpoint{1.851489in}{1.425516in}}%
\pgfpathlineto{\pgfqpoint{1.853341in}{1.481812in}}%
\pgfpathlineto{\pgfqpoint{1.855188in}{1.437374in}}%
\pgfpathlineto{\pgfqpoint{1.855432in}{1.466749in}}%
\pgfpathlineto{\pgfqpoint{1.856695in}{1.428905in}}%
\pgfpathlineto{\pgfqpoint{1.857406in}{1.430205in}}%
\pgfpathlineto{\pgfqpoint{1.857975in}{1.483927in}}%
\pgfpathlineto{\pgfqpoint{1.858825in}{1.478956in}}%
\pgfpathlineto{\pgfqpoint{1.860437in}{1.513436in}}%
\pgfpathlineto{\pgfqpoint{1.861958in}{1.453496in}}%
\pgfpathlineto{\pgfqpoint{1.862385in}{1.472443in}}%
\pgfpathlineto{\pgfqpoint{1.863541in}{1.428877in}}%
\pgfpathlineto{\pgfqpoint{1.864191in}{1.482308in}}%
\pgfpathlineto{\pgfqpoint{1.864770in}{1.453140in}}%
\pgfpathlineto{\pgfqpoint{1.865993in}{1.466355in}}%
\pgfpathlineto{\pgfqpoint{1.868541in}{1.554002in}}%
\pgfpathlineto{\pgfqpoint{1.869065in}{1.538865in}}%
\pgfpathlineto{\pgfqpoint{1.870546in}{1.585577in}}%
\pgfpathlineto{\pgfqpoint{1.870725in}{1.567747in}}%
\pgfpathlineto{\pgfqpoint{1.872086in}{1.587135in}}%
\pgfpathlineto{\pgfqpoint{1.873240in}{1.556447in}}%
\pgfpathlineto{\pgfqpoint{1.875338in}{1.637453in}}%
\pgfpathlineto{\pgfqpoint{1.876126in}{1.659654in}}%
\pgfpathlineto{\pgfqpoint{1.876716in}{1.635422in}}%
\pgfpathlineto{\pgfqpoint{1.877948in}{1.676163in}}%
\pgfpathlineto{\pgfqpoint{1.878448in}{1.660216in}}%
\pgfpathlineto{\pgfqpoint{1.879563in}{1.636223in}}%
\pgfpathlineto{\pgfqpoint{1.880171in}{1.690552in}}%
\pgfpathlineto{\pgfqpoint{1.880945in}{1.637363in}}%
\pgfpathlineto{\pgfqpoint{1.881997in}{1.666191in}}%
\pgfpathlineto{\pgfqpoint{1.882684in}{1.638971in}}%
\pgfpathlineto{\pgfqpoint{1.884189in}{1.625123in}}%
\pgfpathlineto{\pgfqpoint{1.884310in}{1.643881in}}%
\pgfpathlineto{\pgfqpoint{1.885715in}{1.662025in}}%
\pgfpathlineto{\pgfqpoint{1.886054in}{1.638928in}}%
\pgfpathlineto{\pgfqpoint{1.886960in}{1.644832in}}%
\pgfpathlineto{\pgfqpoint{1.888291in}{1.589083in}}%
\pgfpathlineto{\pgfqpoint{1.889309in}{1.634663in}}%
\pgfpathlineto{\pgfqpoint{1.890070in}{1.603896in}}%
\pgfpathlineto{\pgfqpoint{1.890282in}{1.634298in}}%
\pgfpathlineto{\pgfqpoint{1.891110in}{1.617354in}}%
\pgfpathlineto{\pgfqpoint{1.891972in}{1.643702in}}%
\pgfpathlineto{\pgfqpoint{1.893511in}{1.625146in}}%
\pgfpathlineto{\pgfqpoint{1.894300in}{1.651522in}}%
\pgfpathlineto{\pgfqpoint{1.894703in}{1.608021in}}%
\pgfpathlineto{\pgfqpoint{1.896120in}{1.659497in}}%
\pgfpathlineto{\pgfqpoint{1.897340in}{1.592123in}}%
\pgfpathlineto{\pgfqpoint{1.898104in}{1.622917in}}%
\pgfpathlineto{\pgfqpoint{1.899110in}{1.583634in}}%
\pgfpathlineto{\pgfqpoint{1.900751in}{1.629097in}}%
\pgfpathlineto{\pgfqpoint{1.901570in}{1.597795in}}%
\pgfpathlineto{\pgfqpoint{1.902976in}{1.624617in}}%
\pgfpathlineto{\pgfqpoint{1.904291in}{1.584390in}}%
\pgfpathlineto{\pgfqpoint{1.906273in}{1.660269in}}%
\pgfpathlineto{\pgfqpoint{1.907251in}{1.626966in}}%
\pgfpathlineto{\pgfqpoint{1.907348in}{1.643568in}}%
\pgfpathlineto{\pgfqpoint{1.908774in}{1.623736in}}%
\pgfpathlineto{\pgfqpoint{1.908963in}{1.657202in}}%
\pgfpathlineto{\pgfqpoint{1.909918in}{1.624904in}}%
\pgfpathlineto{\pgfqpoint{1.911083in}{1.665070in}}%
\pgfpathlineto{\pgfqpoint{1.911597in}{1.616096in}}%
\pgfpathlineto{\pgfqpoint{1.912543in}{1.635169in}}%
\pgfpathlineto{\pgfqpoint{1.913515in}{1.616100in}}%
\pgfpathlineto{\pgfqpoint{1.914408in}{1.653316in}}%
\pgfpathlineto{\pgfqpoint{1.915150in}{1.604466in}}%
\pgfpathlineto{\pgfqpoint{1.917209in}{1.665102in}}%
\pgfpathlineto{\pgfqpoint{1.917685in}{1.625670in}}%
\pgfpathlineto{\pgfqpoint{1.919126in}{1.688979in}}%
\pgfpathlineto{\pgfqpoint{1.919348in}{1.669765in}}%
\pgfpathlineto{\pgfqpoint{1.920208in}{1.660006in}}%
\pgfpathlineto{\pgfqpoint{1.921275in}{1.705835in}}%
\pgfpathlineto{\pgfqpoint{1.923345in}{1.672905in}}%
\pgfpathlineto{\pgfqpoint{1.925190in}{1.784595in}}%
\pgfpathlineto{\pgfqpoint{1.926468in}{1.788711in}}%
\pgfpathlineto{\pgfqpoint{1.928541in}{1.707968in}}%
\pgfpathlineto{\pgfqpoint{1.930147in}{1.669230in}}%
\pgfpathlineto{\pgfqpoint{1.930807in}{1.693076in}}%
\pgfpathlineto{\pgfqpoint{1.931415in}{1.668157in}}%
\pgfpathlineto{\pgfqpoint{1.933271in}{1.729433in}}%
\pgfpathlineto{\pgfqpoint{1.935835in}{1.624233in}}%
\pgfpathlineto{\pgfqpoint{1.937487in}{1.625738in}}%
\pgfpathlineto{\pgfqpoint{1.938254in}{1.660132in}}%
\pgfpathlineto{\pgfqpoint{1.939463in}{1.633215in}}%
\pgfpathlineto{\pgfqpoint{1.939756in}{1.649033in}}%
\pgfpathlineto{\pgfqpoint{1.941026in}{1.637206in}}%
\pgfpathlineto{\pgfqpoint{1.941290in}{1.666234in}}%
\pgfpathlineto{\pgfqpoint{1.942093in}{1.657010in}}%
\pgfpathlineto{\pgfqpoint{1.943493in}{1.684919in}}%
\pgfpathlineto{\pgfqpoint{1.945241in}{1.754531in}}%
\pgfpathlineto{\pgfqpoint{1.945534in}{1.724252in}}%
\pgfpathlineto{\pgfqpoint{1.947034in}{1.782670in}}%
\pgfpathlineto{\pgfqpoint{1.947651in}{1.732399in}}%
\pgfpathlineto{\pgfqpoint{1.948056in}{1.758422in}}%
\pgfpathlineto{\pgfqpoint{1.949586in}{1.773886in}}%
\pgfpathlineto{\pgfqpoint{1.950357in}{1.770523in}}%
\pgfpathlineto{\pgfqpoint{1.950987in}{1.724575in}}%
\pgfpathlineto{\pgfqpoint{1.951770in}{1.747388in}}%
\pgfpathlineto{\pgfqpoint{1.952453in}{1.721432in}}%
\pgfpathlineto{\pgfqpoint{1.953440in}{1.729540in}}%
\pgfpathlineto{\pgfqpoint{1.954194in}{1.688862in}}%
\pgfpathlineto{\pgfqpoint{1.955586in}{1.737781in}}%
\pgfpathlineto{\pgfqpoint{1.955798in}{1.722542in}}%
\pgfpathlineto{\pgfqpoint{1.957041in}{1.698714in}}%
\pgfpathlineto{\pgfqpoint{1.958320in}{1.751803in}}%
\pgfpathlineto{\pgfqpoint{1.959521in}{1.716973in}}%
\pgfpathlineto{\pgfqpoint{1.961051in}{1.776785in}}%
\pgfpathlineto{\pgfqpoint{1.961816in}{1.725635in}}%
\pgfpathlineto{\pgfqpoint{1.963331in}{1.710494in}}%
\pgfpathlineto{\pgfqpoint{1.964030in}{1.759763in}}%
\pgfpathlineto{\pgfqpoint{1.964828in}{1.733581in}}%
\pgfpathlineto{\pgfqpoint{1.965358in}{1.756488in}}%
\pgfpathlineto{\pgfqpoint{1.967125in}{1.706424in}}%
\pgfpathlineto{\pgfqpoint{1.969169in}{1.770638in}}%
\pgfpathlineto{\pgfqpoint{1.969664in}{1.727020in}}%
\pgfpathlineto{\pgfqpoint{1.970929in}{1.762430in}}%
\pgfpathlineto{\pgfqpoint{1.971278in}{1.735724in}}%
\pgfpathlineto{\pgfqpoint{1.972373in}{1.793398in}}%
\pgfpathlineto{\pgfqpoint{1.975022in}{1.691335in}}%
\pgfpathlineto{\pgfqpoint{1.975376in}{1.707938in}}%
\pgfpathlineto{\pgfqpoint{1.976749in}{1.680699in}}%
\pgfpathlineto{\pgfqpoint{1.977325in}{1.702920in}}%
\pgfpathlineto{\pgfqpoint{1.977885in}{1.671491in}}%
\pgfpathlineto{\pgfqpoint{1.979109in}{1.720128in}}%
\pgfpathlineto{\pgfqpoint{1.980094in}{1.695641in}}%
\pgfpathlineto{\pgfqpoint{1.980944in}{1.721822in}}%
\pgfpathlineto{\pgfqpoint{1.981547in}{1.698062in}}%
\pgfpathlineto{\pgfqpoint{1.983044in}{1.748922in}}%
\pgfpathlineto{\pgfqpoint{1.983915in}{1.718772in}}%
\pgfpathlineto{\pgfqpoint{1.985513in}{1.753353in}}%
\pgfpathlineto{\pgfqpoint{1.987697in}{1.694930in}}%
\pgfpathlineto{\pgfqpoint{1.988984in}{1.762188in}}%
\pgfpathlineto{\pgfqpoint{1.989992in}{1.721011in}}%
\pgfpathlineto{\pgfqpoint{1.990774in}{1.732680in}}%
\pgfpathlineto{\pgfqpoint{1.992152in}{1.823608in}}%
\pgfpathlineto{\pgfqpoint{1.992518in}{1.806720in}}%
\pgfpathlineto{\pgfqpoint{1.993358in}{1.806226in}}%
\pgfpathlineto{\pgfqpoint{1.994645in}{1.858579in}}%
\pgfpathlineto{\pgfqpoint{1.995431in}{1.811336in}}%
\pgfpathlineto{\pgfqpoint{1.995769in}{1.822931in}}%
\pgfpathlineto{\pgfqpoint{1.997324in}{1.770706in}}%
\pgfpathlineto{\pgfqpoint{1.998030in}{1.805393in}}%
\pgfpathlineto{\pgfqpoint{1.998279in}{1.782248in}}%
\pgfpathlineto{\pgfqpoint{1.999423in}{1.807561in}}%
\pgfpathlineto{\pgfqpoint{2.000693in}{1.804302in}}%
\pgfpathlineto{\pgfqpoint{2.001265in}{1.758957in}}%
\pgfpathlineto{\pgfqpoint{2.001841in}{1.799350in}}%
\pgfpathlineto{\pgfqpoint{2.002926in}{1.825035in}}%
\pgfpathlineto{\pgfqpoint{2.004492in}{1.757005in}}%
\pgfpathlineto{\pgfqpoint{2.005677in}{1.790198in}}%
\pgfpathlineto{\pgfqpoint{2.009450in}{1.683210in}}%
\pgfpathlineto{\pgfqpoint{2.011746in}{1.728376in}}%
\pgfpathlineto{\pgfqpoint{2.012243in}{1.700158in}}%
\pgfpathlineto{\pgfqpoint{2.012973in}{1.729855in}}%
\pgfpathlineto{\pgfqpoint{2.013908in}{1.694494in}}%
\pgfpathlineto{\pgfqpoint{2.014492in}{1.744636in}}%
\pgfpathlineto{\pgfqpoint{2.015881in}{1.678367in}}%
\pgfpathlineto{\pgfqpoint{2.016527in}{1.711868in}}%
\pgfpathlineto{\pgfqpoint{2.016881in}{1.692975in}}%
\pgfpathlineto{\pgfqpoint{2.018413in}{1.686266in}}%
\pgfpathlineto{\pgfqpoint{2.019833in}{1.724470in}}%
\pgfpathlineto{\pgfqpoint{2.020577in}{1.690537in}}%
\pgfpathlineto{\pgfqpoint{2.021434in}{1.725781in}}%
\pgfpathlineto{\pgfqpoint{2.022345in}{1.685268in}}%
\pgfpathlineto{\pgfqpoint{2.023069in}{1.714234in}}%
\pgfpathlineto{\pgfqpoint{2.024427in}{1.702515in}}%
\pgfpathlineto{\pgfqpoint{2.024990in}{1.735603in}}%
\pgfpathlineto{\pgfqpoint{2.026056in}{1.670851in}}%
\pgfpathlineto{\pgfqpoint{2.026256in}{1.691322in}}%
\pgfpathlineto{\pgfqpoint{2.027563in}{1.655360in}}%
\pgfpathlineto{\pgfqpoint{2.028760in}{1.730205in}}%
\pgfpathlineto{\pgfqpoint{2.028938in}{1.703114in}}%
\pgfpathlineto{\pgfqpoint{2.030243in}{1.681975in}}%
\pgfpathlineto{\pgfqpoint{2.030653in}{1.701065in}}%
\pgfpathlineto{\pgfqpoint{2.031538in}{1.681441in}}%
\pgfpathlineto{\pgfqpoint{2.032433in}{1.681862in}}%
\pgfpathlineto{\pgfqpoint{2.033018in}{1.718538in}}%
\pgfpathlineto{\pgfqpoint{2.034431in}{1.697592in}}%
\pgfpathlineto{\pgfqpoint{2.035533in}{1.725110in}}%
\pgfpathlineto{\pgfqpoint{2.036032in}{1.705668in}}%
\pgfpathlineto{\pgfqpoint{2.037144in}{1.753298in}}%
\pgfpathlineto{\pgfqpoint{2.037677in}{1.732123in}}%
\pgfpathlineto{\pgfqpoint{2.038831in}{1.747534in}}%
\pgfpathlineto{\pgfqpoint{2.039024in}{1.721279in}}%
\pgfpathlineto{\pgfqpoint{2.039816in}{1.746771in}}%
\pgfpathlineto{\pgfqpoint{2.041200in}{1.705736in}}%
\pgfpathlineto{\pgfqpoint{2.041517in}{1.749925in}}%
\pgfpathlineto{\pgfqpoint{2.042366in}{1.712916in}}%
\pgfpathlineto{\pgfqpoint{2.043956in}{1.710053in}}%
\pgfpathlineto{\pgfqpoint{2.045421in}{1.770141in}}%
\pgfpathlineto{\pgfqpoint{2.046052in}{1.725569in}}%
\pgfpathlineto{\pgfqpoint{2.046650in}{1.743182in}}%
\pgfpathlineto{\pgfqpoint{2.047671in}{1.705140in}}%
\pgfpathlineto{\pgfqpoint{2.048374in}{1.730741in}}%
\pgfpathlineto{\pgfqpoint{2.049309in}{1.704324in}}%
\pgfpathlineto{\pgfqpoint{2.050783in}{1.721516in}}%
\pgfpathlineto{\pgfqpoint{2.051495in}{1.698122in}}%
\pgfpathlineto{\pgfqpoint{2.052353in}{1.745185in}}%
\pgfpathlineto{\pgfqpoint{2.053037in}{1.726961in}}%
\pgfpathlineto{\pgfqpoint{2.055693in}{1.837793in}}%
\pgfpathlineto{\pgfqpoint{2.056155in}{1.822869in}}%
\pgfpathlineto{\pgfqpoint{2.057407in}{1.829822in}}%
\pgfpathlineto{\pgfqpoint{2.058160in}{1.869709in}}%
\pgfpathlineto{\pgfqpoint{2.058944in}{1.820080in}}%
\pgfpathlineto{\pgfqpoint{2.059693in}{1.850419in}}%
\pgfpathlineto{\pgfqpoint{2.060547in}{1.823327in}}%
\pgfpathlineto{\pgfqpoint{2.061496in}{1.847844in}}%
\pgfpathlineto{\pgfqpoint{2.061974in}{1.822074in}}%
\pgfpathlineto{\pgfqpoint{2.062870in}{1.846094in}}%
\pgfpathlineto{\pgfqpoint{2.063624in}{1.816204in}}%
\pgfpathlineto{\pgfqpoint{2.064509in}{1.852702in}}%
\pgfpathlineto{\pgfqpoint{2.065344in}{1.817389in}}%
\pgfpathlineto{\pgfqpoint{2.066310in}{1.851786in}}%
\pgfpathlineto{\pgfqpoint{2.067477in}{1.836448in}}%
\pgfpathlineto{\pgfqpoint{2.067932in}{1.874570in}}%
\pgfpathlineto{\pgfqpoint{2.068711in}{1.842021in}}%
\pgfpathlineto{\pgfqpoint{2.070155in}{1.883888in}}%
\pgfpathlineto{\pgfqpoint{2.070519in}{1.846855in}}%
\pgfpathlineto{\pgfqpoint{2.071325in}{1.859473in}}%
\pgfpathlineto{\pgfqpoint{2.073852in}{1.799896in}}%
\pgfpathlineto{\pgfqpoint{2.075386in}{1.782742in}}%
\pgfpathlineto{\pgfqpoint{2.076203in}{1.823222in}}%
\pgfpathlineto{\pgfqpoint{2.077796in}{1.771079in}}%
\pgfpathlineto{\pgfqpoint{2.078399in}{1.799221in}}%
\pgfpathlineto{\pgfqpoint{2.079206in}{1.772194in}}%
\pgfpathlineto{\pgfqpoint{2.079976in}{1.813812in}}%
\pgfpathlineto{\pgfqpoint{2.081702in}{1.802772in}}%
\pgfpathlineto{\pgfqpoint{2.082612in}{1.866865in}}%
\pgfpathlineto{\pgfqpoint{2.083558in}{1.821232in}}%
\pgfpathlineto{\pgfqpoint{2.084494in}{1.869907in}}%
\pgfpathlineto{\pgfqpoint{2.085588in}{1.841952in}}%
\pgfpathlineto{\pgfqpoint{2.086185in}{1.861524in}}%
\pgfpathlineto{\pgfqpoint{2.086576in}{1.824031in}}%
\pgfpathlineto{\pgfqpoint{2.089001in}{1.865604in}}%
\pgfpathlineto{\pgfqpoint{2.089869in}{1.815855in}}%
\pgfpathlineto{\pgfqpoint{2.090803in}{1.822326in}}%
\pgfpathlineto{\pgfqpoint{2.092292in}{1.853868in}}%
\pgfpathlineto{\pgfqpoint{2.093446in}{1.798982in}}%
\pgfpathlineto{\pgfqpoint{2.095036in}{1.832801in}}%
\pgfpathlineto{\pgfqpoint{2.095872in}{1.803836in}}%
\pgfpathlineto{\pgfqpoint{2.096188in}{1.837173in}}%
\pgfpathlineto{\pgfqpoint{2.096974in}{1.813981in}}%
\pgfpathlineto{\pgfqpoint{2.098171in}{1.884898in}}%
\pgfpathlineto{\pgfqpoint{2.099279in}{1.821608in}}%
\pgfpathlineto{\pgfqpoint{2.099615in}{1.839305in}}%
\pgfpathlineto{\pgfqpoint{2.100779in}{1.839972in}}%
\pgfpathlineto{\pgfqpoint{2.102160in}{1.802501in}}%
\pgfpathlineto{\pgfqpoint{2.103872in}{1.837687in}}%
\pgfpathlineto{\pgfqpoint{2.105884in}{1.764956in}}%
\pgfpathlineto{\pgfqpoint{2.106277in}{1.799654in}}%
\pgfpathlineto{\pgfqpoint{2.106988in}{1.769361in}}%
\pgfpathlineto{\pgfqpoint{2.108024in}{1.800630in}}%
\pgfpathlineto{\pgfqpoint{2.108856in}{1.777241in}}%
\pgfpathlineto{\pgfqpoint{2.110850in}{1.843478in}}%
\pgfpathlineto{\pgfqpoint{2.112815in}{1.786815in}}%
\pgfpathlineto{\pgfqpoint{2.113384in}{1.816509in}}%
\pgfpathlineto{\pgfqpoint{2.113776in}{1.804240in}}%
\pgfpathlineto{\pgfqpoint{2.114745in}{1.822643in}}%
\pgfpathlineto{\pgfqpoint{2.115836in}{1.792548in}}%
\pgfpathlineto{\pgfqpoint{2.116849in}{1.867339in}}%
\pgfpathlineto{\pgfqpoint{2.117343in}{1.828604in}}%
\pgfpathlineto{\pgfqpoint{2.118482in}{1.812336in}}%
\pgfpathlineto{\pgfqpoint{2.119288in}{1.838006in}}%
\pgfpathlineto{\pgfqpoint{2.119720in}{1.802383in}}%
\pgfpathlineto{\pgfqpoint{2.120553in}{1.806624in}}%
\pgfpathlineto{\pgfqpoint{2.121835in}{1.756721in}}%
\pgfpathlineto{\pgfqpoint{2.122242in}{1.780080in}}%
\pgfpathlineto{\pgfqpoint{2.123381in}{1.750540in}}%
\pgfpathlineto{\pgfqpoint{2.124879in}{1.817518in}}%
\pgfpathlineto{\pgfqpoint{2.126366in}{1.811375in}}%
\pgfpathlineto{\pgfqpoint{2.127338in}{1.754845in}}%
\pgfpathlineto{\pgfqpoint{2.127948in}{1.793934in}}%
\pgfpathlineto{\pgfqpoint{2.128213in}{1.765655in}}%
\pgfpathlineto{\pgfqpoint{2.129889in}{1.726506in}}%
\pgfpathlineto{\pgfqpoint{2.131284in}{1.770514in}}%
\pgfpathlineto{\pgfqpoint{2.132045in}{1.772863in}}%
\pgfpathlineto{\pgfqpoint{2.133435in}{1.702592in}}%
\pgfpathlineto{\pgfqpoint{2.134424in}{1.697365in}}%
\pgfpathlineto{\pgfqpoint{2.135761in}{1.712490in}}%
\pgfpathlineto{\pgfqpoint{2.136524in}{1.750573in}}%
\pgfpathlineto{\pgfqpoint{2.137313in}{1.710020in}}%
\pgfpathlineto{\pgfqpoint{2.137681in}{1.739253in}}%
\pgfpathlineto{\pgfqpoint{2.138704in}{1.732678in}}%
\pgfpathlineto{\pgfqpoint{2.141133in}{1.814456in}}%
\pgfpathlineto{\pgfqpoint{2.141924in}{1.790337in}}%
\pgfpathlineto{\pgfqpoint{2.143300in}{1.790884in}}%
\pgfpathlineto{\pgfqpoint{2.143825in}{1.784782in}}%
\pgfpathlineto{\pgfqpoint{2.144883in}{1.712033in}}%
\pgfpathlineto{\pgfqpoint{2.145312in}{1.736768in}}%
\pgfpathlineto{\pgfqpoint{2.146418in}{1.696969in}}%
\pgfpathlineto{\pgfqpoint{2.147507in}{1.729659in}}%
\pgfpathlineto{\pgfqpoint{2.148077in}{1.681326in}}%
\pgfpathlineto{\pgfqpoint{2.148720in}{1.711134in}}%
\pgfpathlineto{\pgfqpoint{2.150164in}{1.746876in}}%
\pgfpathlineto{\pgfqpoint{2.150762in}{1.717843in}}%
\pgfpathlineto{\pgfqpoint{2.151321in}{1.743735in}}%
\pgfpathlineto{\pgfqpoint{2.152198in}{1.748688in}}%
\pgfpathlineto{\pgfqpoint{2.153488in}{1.703546in}}%
\pgfpathlineto{\pgfqpoint{2.154225in}{1.754307in}}%
\pgfpathlineto{\pgfqpoint{2.154590in}{1.730596in}}%
\pgfpathlineto{\pgfqpoint{2.155854in}{1.682153in}}%
\pgfpathlineto{\pgfqpoint{2.156247in}{1.704989in}}%
\pgfpathlineto{\pgfqpoint{2.158167in}{1.652289in}}%
\pgfpathlineto{\pgfqpoint{2.158798in}{1.684123in}}%
\pgfpathlineto{\pgfqpoint{2.160455in}{1.664400in}}%
\pgfpathlineto{\pgfqpoint{2.161168in}{1.677311in}}%
\pgfpathlineto{\pgfqpoint{2.162533in}{1.629712in}}%
\pgfpathlineto{\pgfqpoint{2.164583in}{1.700413in}}%
\pgfpathlineto{\pgfqpoint{2.165574in}{1.648646in}}%
\pgfpathlineto{\pgfqpoint{2.165633in}{1.659524in}}%
\pgfpathlineto{\pgfqpoint{2.167641in}{1.750228in}}%
\pgfpathlineto{\pgfqpoint{2.168765in}{1.704198in}}%
\pgfpathlineto{\pgfqpoint{2.169006in}{1.718322in}}%
\pgfpathlineto{\pgfqpoint{2.170612in}{1.714440in}}%
\pgfpathlineto{\pgfqpoint{2.172282in}{1.794214in}}%
\pgfpathlineto{\pgfqpoint{2.173177in}{1.755098in}}%
\pgfpathlineto{\pgfqpoint{2.173794in}{1.791424in}}%
\pgfpathlineto{\pgfqpoint{2.176008in}{1.706243in}}%
\pgfpathlineto{\pgfqpoint{2.177382in}{1.716002in}}%
\pgfpathlineto{\pgfqpoint{2.178893in}{1.643815in}}%
\pgfpathlineto{\pgfqpoint{2.179484in}{1.682388in}}%
\pgfpathlineto{\pgfqpoint{2.180613in}{1.663525in}}%
\pgfpathlineto{\pgfqpoint{2.181218in}{1.620574in}}%
\pgfpathlineto{\pgfqpoint{2.182354in}{1.626012in}}%
\pgfpathlineto{\pgfqpoint{2.182815in}{1.662113in}}%
\pgfpathlineto{\pgfqpoint{2.183431in}{1.651223in}}%
\pgfpathlineto{\pgfqpoint{2.184906in}{1.699076in}}%
\pgfpathlineto{\pgfqpoint{2.185248in}{1.679001in}}%
\pgfpathlineto{\pgfqpoint{2.186670in}{1.705381in}}%
\pgfpathlineto{\pgfqpoint{2.188489in}{1.808783in}}%
\pgfpathlineto{\pgfqpoint{2.190809in}{1.713890in}}%
\pgfpathlineto{\pgfqpoint{2.191607in}{1.726461in}}%
\pgfpathlineto{\pgfqpoint{2.193477in}{1.802301in}}%
\pgfpathlineto{\pgfqpoint{2.193751in}{1.765574in}}%
\pgfpathlineto{\pgfqpoint{2.194510in}{1.796329in}}%
\pgfpathlineto{\pgfqpoint{2.195916in}{1.741377in}}%
\pgfpathlineto{\pgfqpoint{2.197012in}{1.770920in}}%
\pgfpathlineto{\pgfqpoint{2.197609in}{1.742954in}}%
\pgfpathlineto{\pgfqpoint{2.198037in}{1.783608in}}%
\pgfpathlineto{\pgfqpoint{2.199311in}{1.779424in}}%
\pgfpathlineto{\pgfqpoint{2.200634in}{1.849903in}}%
\pgfpathlineto{\pgfqpoint{2.202846in}{1.801853in}}%
\pgfpathlineto{\pgfqpoint{2.203475in}{1.798664in}}%
\pgfpathlineto{\pgfqpoint{2.204813in}{1.850900in}}%
\pgfpathlineto{\pgfqpoint{2.206026in}{1.801414in}}%
\pgfpathlineto{\pgfqpoint{2.207594in}{1.869730in}}%
\pgfpathlineto{\pgfqpoint{2.208473in}{1.847434in}}%
\pgfpathlineto{\pgfqpoint{2.209303in}{1.888518in}}%
\pgfpathlineto{\pgfqpoint{2.210061in}{1.870123in}}%
\pgfpathlineto{\pgfqpoint{2.211420in}{1.903378in}}%
\pgfpathlineto{\pgfqpoint{2.213394in}{1.855157in}}%
\pgfpathlineto{\pgfqpoint{2.214375in}{1.852390in}}%
\pgfpathlineto{\pgfqpoint{2.215529in}{1.914084in}}%
\pgfpathlineto{\pgfqpoint{2.216105in}{1.871150in}}%
\pgfpathlineto{\pgfqpoint{2.217147in}{1.898004in}}%
\pgfpathlineto{\pgfqpoint{2.218117in}{1.880195in}}%
\pgfpathlineto{\pgfqpoint{2.218982in}{1.925898in}}%
\pgfpathlineto{\pgfqpoint{2.219636in}{1.887888in}}%
\pgfpathlineto{\pgfqpoint{2.220786in}{1.905936in}}%
\pgfpathlineto{\pgfqpoint{2.221064in}{1.881065in}}%
\pgfpathlineto{\pgfqpoint{2.222029in}{1.899376in}}%
\pgfpathlineto{\pgfqpoint{2.222858in}{1.875271in}}%
\pgfpathlineto{\pgfqpoint{2.224830in}{1.947232in}}%
\pgfpathlineto{\pgfqpoint{2.225133in}{1.915134in}}%
\pgfpathlineto{\pgfqpoint{2.225936in}{1.928391in}}%
\pgfpathlineto{\pgfqpoint{2.226765in}{1.897661in}}%
\pgfpathlineto{\pgfqpoint{2.228051in}{1.910863in}}%
\pgfpathlineto{\pgfqpoint{2.230079in}{1.993094in}}%
\pgfpathlineto{\pgfqpoint{2.230587in}{1.957964in}}%
\pgfpathlineto{\pgfqpoint{2.231648in}{2.014523in}}%
\pgfpathlineto{\pgfqpoint{2.232459in}{1.965499in}}%
\pgfpathlineto{\pgfqpoint{2.232731in}{2.004862in}}%
\pgfpathlineto{\pgfqpoint{2.233739in}{1.982845in}}%
\pgfpathlineto{\pgfqpoint{2.235195in}{1.989717in}}%
\pgfpathlineto{\pgfqpoint{2.236426in}{2.046586in}}%
\pgfpathlineto{\pgfqpoint{2.237575in}{2.012170in}}%
\pgfpathlineto{\pgfqpoint{2.238482in}{2.065749in}}%
\pgfpathlineto{\pgfqpoint{2.238802in}{2.040024in}}%
\pgfpathlineto{\pgfqpoint{2.239855in}{2.027256in}}%
\pgfpathlineto{\pgfqpoint{2.240868in}{2.072444in}}%
\pgfpathlineto{\pgfqpoint{2.241279in}{2.058465in}}%
\pgfpathlineto{\pgfqpoint{2.242762in}{2.103538in}}%
\pgfpathlineto{\pgfqpoint{2.243288in}{2.090352in}}%
\pgfpathlineto{\pgfqpoint{2.244204in}{2.111700in}}%
\pgfpathlineto{\pgfqpoint{2.245270in}{2.165231in}}%
\pgfpathlineto{\pgfqpoint{2.246197in}{2.122178in}}%
\pgfpathlineto{\pgfqpoint{2.246613in}{2.150901in}}%
\pgfpathlineto{\pgfqpoint{2.247157in}{2.136621in}}%
\pgfpathlineto{\pgfqpoint{2.248568in}{2.185545in}}%
\pgfpathlineto{\pgfqpoint{2.250550in}{2.138481in}}%
\pgfpathlineto{\pgfqpoint{2.251278in}{2.164476in}}%
\pgfpathlineto{\pgfqpoint{2.251461in}{2.131029in}}%
\pgfpathlineto{\pgfqpoint{2.252721in}{2.169677in}}%
\pgfpathlineto{\pgfqpoint{2.253533in}{2.145993in}}%
\pgfpathlineto{\pgfqpoint{2.254594in}{2.166902in}}%
\pgfpathlineto{\pgfqpoint{2.255220in}{2.109572in}}%
\pgfpathlineto{\pgfqpoint{2.255777in}{2.129634in}}%
\pgfpathlineto{\pgfqpoint{2.256527in}{2.104874in}}%
\pgfpathlineto{\pgfqpoint{2.257484in}{2.135544in}}%
\pgfpathlineto{\pgfqpoint{2.258821in}{2.086863in}}%
\pgfpathlineto{\pgfqpoint{2.259817in}{2.113572in}}%
\pgfpathlineto{\pgfqpoint{2.260420in}{2.075081in}}%
\pgfpathlineto{\pgfqpoint{2.261602in}{2.150374in}}%
\pgfpathlineto{\pgfqpoint{2.262033in}{2.123197in}}%
\pgfpathlineto{\pgfqpoint{2.262906in}{2.110727in}}%
\pgfpathlineto{\pgfqpoint{2.263518in}{2.146921in}}%
\pgfpathlineto{\pgfqpoint{2.264815in}{2.123340in}}%
\pgfpathlineto{\pgfqpoint{2.265626in}{2.164946in}}%
\pgfpathlineto{\pgfqpoint{2.265924in}{2.144159in}}%
\pgfpathlineto{\pgfqpoint{2.267354in}{2.100149in}}%
\pgfpathlineto{\pgfqpoint{2.267855in}{2.143801in}}%
\pgfpathlineto{\pgfqpoint{2.270330in}{2.089311in}}%
\pgfpathlineto{\pgfqpoint{2.271376in}{2.111659in}}%
\pgfpathlineto{\pgfqpoint{2.272411in}{2.078892in}}%
\pgfpathlineto{\pgfqpoint{2.273762in}{2.126734in}}%
\pgfpathlineto{\pgfqpoint{2.275050in}{2.132382in}}%
\pgfpathlineto{\pgfqpoint{2.275591in}{2.104589in}}%
\pgfpathlineto{\pgfqpoint{2.277954in}{2.187357in}}%
\pgfpathlineto{\pgfqpoint{2.278823in}{2.212275in}}%
\pgfpathlineto{\pgfqpoint{2.279803in}{2.167926in}}%
\pgfpathlineto{\pgfqpoint{2.280582in}{2.188158in}}%
\pgfpathlineto{\pgfqpoint{2.282892in}{2.093924in}}%
\pgfpathlineto{\pgfqpoint{2.283701in}{2.113477in}}%
\pgfpathlineto{\pgfqpoint{2.285194in}{2.092957in}}%
\pgfpathlineto{\pgfqpoint{2.286896in}{2.017675in}}%
\pgfpathlineto{\pgfqpoint{2.287198in}{2.043458in}}%
\pgfpathlineto{\pgfqpoint{2.288099in}{2.005668in}}%
\pgfpathlineto{\pgfqpoint{2.289425in}{2.017888in}}%
\pgfpathlineto{\pgfqpoint{2.290490in}{1.962851in}}%
\pgfpathlineto{\pgfqpoint{2.290706in}{1.978115in}}%
\pgfpathlineto{\pgfqpoint{2.291788in}{1.980541in}}%
\pgfpathlineto{\pgfqpoint{2.293787in}{1.927724in}}%
\pgfpathlineto{\pgfqpoint{2.294339in}{1.954834in}}%
\pgfpathlineto{\pgfqpoint{2.296584in}{1.884649in}}%
\pgfpathlineto{\pgfqpoint{2.298501in}{1.958637in}}%
\pgfpathlineto{\pgfqpoint{2.299685in}{1.936251in}}%
\pgfpathlineto{\pgfqpoint{2.300080in}{1.972479in}}%
\pgfpathlineto{\pgfqpoint{2.301259in}{1.976462in}}%
\pgfpathlineto{\pgfqpoint{2.302232in}{2.021676in}}%
\pgfpathlineto{\pgfqpoint{2.303363in}{1.979659in}}%
\pgfpathlineto{\pgfqpoint{2.304789in}{1.991189in}}%
\pgfpathlineto{\pgfqpoint{2.306184in}{2.054864in}}%
\pgfpathlineto{\pgfqpoint{2.307212in}{2.001851in}}%
\pgfpathlineto{\pgfqpoint{2.307761in}{2.026047in}}%
\pgfpathlineto{\pgfqpoint{2.308342in}{2.008755in}}%
\pgfpathlineto{\pgfqpoint{2.309205in}{2.027782in}}%
\pgfpathlineto{\pgfqpoint{2.310106in}{1.990745in}}%
\pgfpathlineto{\pgfqpoint{2.310984in}{2.009463in}}%
\pgfpathlineto{\pgfqpoint{2.311750in}{1.974025in}}%
\pgfpathlineto{\pgfqpoint{2.312666in}{1.998155in}}%
\pgfpathlineto{\pgfqpoint{2.313438in}{1.963366in}}%
\pgfpathlineto{\pgfqpoint{2.313438in}{1.963366in}}%
\pgfusepath{stroke}%
\end{pgfscope}%
\begin{pgfscope}%
\pgfsetrectcap%
\pgfsetmiterjoin%
\pgfsetlinewidth{0.803000pt}%
\definecolor{currentstroke}{rgb}{0.000000,0.000000,0.000000}%
\pgfsetstrokecolor{currentstroke}%
\pgfsetdash{}{0pt}%
\pgfpathmoveto{\pgfqpoint{0.530716in}{0.416447in}}%
\pgfpathlineto{\pgfqpoint{0.530716in}{2.398330in}}%
\pgfusepath{stroke}%
\end{pgfscope}%
\begin{pgfscope}%
\pgfsetrectcap%
\pgfsetmiterjoin%
\pgfsetlinewidth{0.803000pt}%
\definecolor{currentstroke}{rgb}{0.000000,0.000000,0.000000}%
\pgfsetstrokecolor{currentstroke}%
\pgfsetdash{}{0pt}%
\pgfpathmoveto{\pgfqpoint{2.398330in}{0.416447in}}%
\pgfpathlineto{\pgfqpoint{2.398330in}{2.398330in}}%
\pgfusepath{stroke}%
\end{pgfscope}%
\begin{pgfscope}%
\pgfsetrectcap%
\pgfsetmiterjoin%
\pgfsetlinewidth{0.803000pt}%
\definecolor{currentstroke}{rgb}{0.000000,0.000000,0.000000}%
\pgfsetstrokecolor{currentstroke}%
\pgfsetdash{}{0pt}%
\pgfpathmoveto{\pgfqpoint{0.530716in}{0.416447in}}%
\pgfpathlineto{\pgfqpoint{2.398330in}{0.416447in}}%
\pgfusepath{stroke}%
\end{pgfscope}%
\begin{pgfscope}%
\pgfsetrectcap%
\pgfsetmiterjoin%
\pgfsetlinewidth{0.803000pt}%
\definecolor{currentstroke}{rgb}{0.000000,0.000000,0.000000}%
\pgfsetstrokecolor{currentstroke}%
\pgfsetdash{}{0pt}%
\pgfpathmoveto{\pgfqpoint{0.530716in}{2.398330in}}%
\pgfpathlineto{\pgfqpoint{2.398330in}{2.398330in}}%
\pgfusepath{stroke}%
\end{pgfscope}%
\begin{pgfscope}%
\pgfsetbuttcap%
\pgfsetmiterjoin%
\definecolor{currentfill}{rgb}{1.000000,1.000000,1.000000}%
\pgfsetfillcolor{currentfill}%
\pgfsetfillopacity{0.800000}%
\pgfsetlinewidth{1.003750pt}%
\definecolor{currentstroke}{rgb}{0.800000,0.800000,0.800000}%
\pgfsetstrokecolor{currentstroke}%
\pgfsetstrokeopacity{0.800000}%
\pgfsetdash{}{0pt}%
\pgfpathmoveto{\pgfqpoint{0.608494in}{2.154552in}}%
\pgfpathlineto{\pgfqpoint{1.673716in}{2.154552in}}%
\pgfpathquadraticcurveto{\pgfqpoint{1.695938in}{2.154552in}}{\pgfqpoint{1.695938in}{2.176775in}}%
\pgfpathlineto{\pgfqpoint{1.695938in}{2.320552in}}%
\pgfpathquadraticcurveto{\pgfqpoint{1.695938in}{2.342774in}}{\pgfqpoint{1.673716in}{2.342774in}}%
\pgfpathlineto{\pgfqpoint{0.608494in}{2.342774in}}%
\pgfpathquadraticcurveto{\pgfqpoint{0.586272in}{2.342774in}}{\pgfqpoint{0.586272in}{2.320552in}}%
\pgfpathlineto{\pgfqpoint{0.586272in}{2.176775in}}%
\pgfpathquadraticcurveto{\pgfqpoint{0.586272in}{2.154552in}}{\pgfqpoint{0.608494in}{2.154552in}}%
\pgfpathlineto{\pgfqpoint{0.608494in}{2.154552in}}%
\pgfpathclose%
\pgfusepath{stroke,fill}%
\end{pgfscope}%
\begin{pgfscope}%
\pgfsetrectcap%
\pgfsetroundjoin%
\pgfsetlinewidth{1.505625pt}%
\definecolor{currentstroke}{rgb}{0.835294,0.368627,0.000000}%
\pgfsetstrokecolor{currentstroke}%
\pgfsetdash{}{0pt}%
\pgfpathmoveto{\pgfqpoint{0.630716in}{2.259441in}}%
\pgfpathlineto{\pgfqpoint{0.741827in}{2.259441in}}%
\pgfpathlineto{\pgfqpoint{0.852938in}{2.259441in}}%
\pgfusepath{stroke}%
\end{pgfscope}%
\begin{pgfscope}%
\definecolor{textcolor}{rgb}{0.000000,0.000000,0.000000}%
\pgfsetstrokecolor{textcolor}%
\pgfsetfillcolor{textcolor}%
\pgftext[x=0.941827in,y=2.220552in,left,base]{\color{textcolor}\rmfamily\fontsize{8.000000}{9.600000}\selectfont Random walk}%
\end{pgfscope}%
\end{pgfpicture}%
\makeatother%
\endgroup%

        } % scalebox
        \caption{Random walk}
    \end{subfigure}
    \caption{Three sperate noise components, that were summed together to simulate a typical noise source.}
    \label{fig:adev_example_noise_types}
\end{figure}

The three time series shown in figure \ref{fig:adev_example_noise_types} were sequentially generated using a fixed seed for the random number generator to ensure repeatability as long as the order of creation is kept the same. For generting the noise, the algorithm presented by \citeauthor{noise_generation} \cite{noise_generation} and implemented in the \textit{AllanTools} library was used. The noise strength parameters were deliberately chosen in such a way, that both the white noise and the random walk part have more noise power than the flicker noise. This allows to distinguish them in the plots at both extremes of the frequency scale. Finally, the three types of noise data were summed together to give the combined signal. The time series is shown in figure \ref{fig:adev_example_time}, again downsampled using LTTB. The summed series clearly shows the white noise content and it is possible to deduce some flicker or random walk noise, but it is highly obscured due to the amount of white noise. Using only the time domain plot makes it very hard to distinguish the type of noise present, let alone estimate the individual noise power of the three sources. Therefore, a different analysis tool is called for.

\begin{figure}[ht]
    \centering
    %% Creator: Matplotlib, PGF backend
%%
%% To include the figure in your LaTeX document, write
%%   \input{<filename>.pgf}
%%
%% Make sure the required packages are loaded in your preamble
%%   \usepackage{pgf}
%%
%% Also ensure that all the required font packages are loaded; for instance,
%% the lmodern package is sometimes necessary when using math font.
%%   \usepackage{lmodern}
%%
%% Figures using additional raster images can only be included by \input if
%% they are in the same directory as the main LaTeX file. For loading figures
%% from other directories you can use the `import` package
%%   \usepackage{import}
%%
%% and then include the figures with
%%   \import{<path to file>}{<filename>.pgf}
%%
%% Matplotlib used the following preamble
%%   \usepackage{siunitx}
%%   \usepackage{fontspec}
%%   \makeatletter\@ifpackageloaded{underscore}{}{\usepackage[strings]{underscore}}\makeatother
%%
\begingroup%
\makeatletter%
\begin{pgfpicture}%
\pgfpathrectangle{\pgfpointorigin}{\pgfqpoint{4.060000in}{2.510000in}}%
\pgfusepath{use as bounding box, clip}%
\begin{pgfscope}%
\pgfsetbuttcap%
\pgfsetmiterjoin%
\definecolor{currentfill}{rgb}{1.000000,1.000000,1.000000}%
\pgfsetfillcolor{currentfill}%
\pgfsetlinewidth{0.000000pt}%
\definecolor{currentstroke}{rgb}{1.000000,1.000000,1.000000}%
\pgfsetstrokecolor{currentstroke}%
\pgfsetdash{}{0pt}%
\pgfpathmoveto{\pgfqpoint{0.000000in}{0.000000in}}%
\pgfpathlineto{\pgfqpoint{4.060000in}{0.000000in}}%
\pgfpathlineto{\pgfqpoint{4.060000in}{2.510000in}}%
\pgfpathlineto{\pgfqpoint{0.000000in}{2.510000in}}%
\pgfpathlineto{\pgfqpoint{0.000000in}{0.000000in}}%
\pgfpathclose%
\pgfusepath{fill}%
\end{pgfscope}%
\begin{pgfscope}%
\pgfsetbuttcap%
\pgfsetmiterjoin%
\definecolor{currentfill}{rgb}{1.000000,1.000000,1.000000}%
\pgfsetfillcolor{currentfill}%
\pgfsetlinewidth{0.000000pt}%
\definecolor{currentstroke}{rgb}{0.000000,0.000000,0.000000}%
\pgfsetstrokecolor{currentstroke}%
\pgfsetstrokeopacity{0.000000}%
\pgfsetdash{}{0pt}%
\pgfpathmoveto{\pgfqpoint{0.589745in}{0.416447in}}%
\pgfpathlineto{\pgfqpoint{3.980629in}{0.416447in}}%
\pgfpathlineto{\pgfqpoint{3.980629in}{2.468330in}}%
\pgfpathlineto{\pgfqpoint{0.589745in}{2.468330in}}%
\pgfpathlineto{\pgfqpoint{0.589745in}{0.416447in}}%
\pgfpathclose%
\pgfusepath{fill}%
\end{pgfscope}%
\begin{pgfscope}%
\pgfpathrectangle{\pgfqpoint{0.589745in}{0.416447in}}{\pgfqpoint{3.390884in}{2.051883in}}%
\pgfusepath{clip}%
\pgfsetrectcap%
\pgfsetroundjoin%
\pgfsetlinewidth{0.803000pt}%
\definecolor{currentstroke}{rgb}{0.450000,0.450000,0.450000}%
\pgfsetstrokecolor{currentstroke}%
\pgfsetdash{}{0pt}%
\pgfpathmoveto{\pgfqpoint{0.743876in}{0.416447in}}%
\pgfpathlineto{\pgfqpoint{0.743876in}{2.468330in}}%
\pgfusepath{stroke}%
\end{pgfscope}%
\begin{pgfscope}%
\pgfsetbuttcap%
\pgfsetroundjoin%
\definecolor{currentfill}{rgb}{0.000000,0.000000,0.000000}%
\pgfsetfillcolor{currentfill}%
\pgfsetlinewidth{0.803000pt}%
\definecolor{currentstroke}{rgb}{0.000000,0.000000,0.000000}%
\pgfsetstrokecolor{currentstroke}%
\pgfsetdash{}{0pt}%
\pgfsys@defobject{currentmarker}{\pgfqpoint{0.000000in}{-0.048611in}}{\pgfqpoint{0.000000in}{0.000000in}}{%
\pgfpathmoveto{\pgfqpoint{0.000000in}{0.000000in}}%
\pgfpathlineto{\pgfqpoint{0.000000in}{-0.048611in}}%
\pgfusepath{stroke,fill}%
}%
\begin{pgfscope}%
\pgfsys@transformshift{0.743876in}{0.416447in}%
\pgfsys@useobject{currentmarker}{}%
\end{pgfscope}%
\end{pgfscope}%
\begin{pgfscope}%
\definecolor{textcolor}{rgb}{0.000000,0.000000,0.000000}%
\pgfsetstrokecolor{textcolor}%
\pgfsetfillcolor{textcolor}%
\pgftext[x=0.743876in,y=0.319225in,,top]{\color{textcolor}\rmfamily\fontsize{8.000000}{9.600000}\selectfont \(\displaystyle {0}\)}%
\end{pgfscope}%
\begin{pgfscope}%
\pgfpathrectangle{\pgfqpoint{0.589745in}{0.416447in}}{\pgfqpoint{3.390884in}{2.051883in}}%
\pgfusepath{clip}%
\pgfsetrectcap%
\pgfsetroundjoin%
\pgfsetlinewidth{0.803000pt}%
\definecolor{currentstroke}{rgb}{0.450000,0.450000,0.450000}%
\pgfsetstrokecolor{currentstroke}%
\pgfsetdash{}{0pt}%
\pgfpathmoveto{\pgfqpoint{1.203222in}{0.416447in}}%
\pgfpathlineto{\pgfqpoint{1.203222in}{2.468330in}}%
\pgfusepath{stroke}%
\end{pgfscope}%
\begin{pgfscope}%
\pgfsetbuttcap%
\pgfsetroundjoin%
\definecolor{currentfill}{rgb}{0.000000,0.000000,0.000000}%
\pgfsetfillcolor{currentfill}%
\pgfsetlinewidth{0.803000pt}%
\definecolor{currentstroke}{rgb}{0.000000,0.000000,0.000000}%
\pgfsetstrokecolor{currentstroke}%
\pgfsetdash{}{0pt}%
\pgfsys@defobject{currentmarker}{\pgfqpoint{0.000000in}{-0.048611in}}{\pgfqpoint{0.000000in}{0.000000in}}{%
\pgfpathmoveto{\pgfqpoint{0.000000in}{0.000000in}}%
\pgfpathlineto{\pgfqpoint{0.000000in}{-0.048611in}}%
\pgfusepath{stroke,fill}%
}%
\begin{pgfscope}%
\pgfsys@transformshift{1.203222in}{0.416447in}%
\pgfsys@useobject{currentmarker}{}%
\end{pgfscope}%
\end{pgfscope}%
\begin{pgfscope}%
\definecolor{textcolor}{rgb}{0.000000,0.000000,0.000000}%
\pgfsetstrokecolor{textcolor}%
\pgfsetfillcolor{textcolor}%
\pgftext[x=1.203222in,y=0.319225in,,top]{\color{textcolor}\rmfamily\fontsize{8.000000}{9.600000}\selectfont \(\displaystyle {5}\)}%
\end{pgfscope}%
\begin{pgfscope}%
\pgfpathrectangle{\pgfqpoint{0.589745in}{0.416447in}}{\pgfqpoint{3.390884in}{2.051883in}}%
\pgfusepath{clip}%
\pgfsetrectcap%
\pgfsetroundjoin%
\pgfsetlinewidth{0.803000pt}%
\definecolor{currentstroke}{rgb}{0.450000,0.450000,0.450000}%
\pgfsetstrokecolor{currentstroke}%
\pgfsetdash{}{0pt}%
\pgfpathmoveto{\pgfqpoint{1.662569in}{0.416447in}}%
\pgfpathlineto{\pgfqpoint{1.662569in}{2.468330in}}%
\pgfusepath{stroke}%
\end{pgfscope}%
\begin{pgfscope}%
\pgfsetbuttcap%
\pgfsetroundjoin%
\definecolor{currentfill}{rgb}{0.000000,0.000000,0.000000}%
\pgfsetfillcolor{currentfill}%
\pgfsetlinewidth{0.803000pt}%
\definecolor{currentstroke}{rgb}{0.000000,0.000000,0.000000}%
\pgfsetstrokecolor{currentstroke}%
\pgfsetdash{}{0pt}%
\pgfsys@defobject{currentmarker}{\pgfqpoint{0.000000in}{-0.048611in}}{\pgfqpoint{0.000000in}{0.000000in}}{%
\pgfpathmoveto{\pgfqpoint{0.000000in}{0.000000in}}%
\pgfpathlineto{\pgfqpoint{0.000000in}{-0.048611in}}%
\pgfusepath{stroke,fill}%
}%
\begin{pgfscope}%
\pgfsys@transformshift{1.662569in}{0.416447in}%
\pgfsys@useobject{currentmarker}{}%
\end{pgfscope}%
\end{pgfscope}%
\begin{pgfscope}%
\definecolor{textcolor}{rgb}{0.000000,0.000000,0.000000}%
\pgfsetstrokecolor{textcolor}%
\pgfsetfillcolor{textcolor}%
\pgftext[x=1.662569in,y=0.319225in,,top]{\color{textcolor}\rmfamily\fontsize{8.000000}{9.600000}\selectfont \(\displaystyle {10}\)}%
\end{pgfscope}%
\begin{pgfscope}%
\pgfpathrectangle{\pgfqpoint{0.589745in}{0.416447in}}{\pgfqpoint{3.390884in}{2.051883in}}%
\pgfusepath{clip}%
\pgfsetrectcap%
\pgfsetroundjoin%
\pgfsetlinewidth{0.803000pt}%
\definecolor{currentstroke}{rgb}{0.450000,0.450000,0.450000}%
\pgfsetstrokecolor{currentstroke}%
\pgfsetdash{}{0pt}%
\pgfpathmoveto{\pgfqpoint{2.121915in}{0.416447in}}%
\pgfpathlineto{\pgfqpoint{2.121915in}{2.468330in}}%
\pgfusepath{stroke}%
\end{pgfscope}%
\begin{pgfscope}%
\pgfsetbuttcap%
\pgfsetroundjoin%
\definecolor{currentfill}{rgb}{0.000000,0.000000,0.000000}%
\pgfsetfillcolor{currentfill}%
\pgfsetlinewidth{0.803000pt}%
\definecolor{currentstroke}{rgb}{0.000000,0.000000,0.000000}%
\pgfsetstrokecolor{currentstroke}%
\pgfsetdash{}{0pt}%
\pgfsys@defobject{currentmarker}{\pgfqpoint{0.000000in}{-0.048611in}}{\pgfqpoint{0.000000in}{0.000000in}}{%
\pgfpathmoveto{\pgfqpoint{0.000000in}{0.000000in}}%
\pgfpathlineto{\pgfqpoint{0.000000in}{-0.048611in}}%
\pgfusepath{stroke,fill}%
}%
\begin{pgfscope}%
\pgfsys@transformshift{2.121915in}{0.416447in}%
\pgfsys@useobject{currentmarker}{}%
\end{pgfscope}%
\end{pgfscope}%
\begin{pgfscope}%
\definecolor{textcolor}{rgb}{0.000000,0.000000,0.000000}%
\pgfsetstrokecolor{textcolor}%
\pgfsetfillcolor{textcolor}%
\pgftext[x=2.121915in,y=0.319225in,,top]{\color{textcolor}\rmfamily\fontsize{8.000000}{9.600000}\selectfont \(\displaystyle {15}\)}%
\end{pgfscope}%
\begin{pgfscope}%
\pgfpathrectangle{\pgfqpoint{0.589745in}{0.416447in}}{\pgfqpoint{3.390884in}{2.051883in}}%
\pgfusepath{clip}%
\pgfsetrectcap%
\pgfsetroundjoin%
\pgfsetlinewidth{0.803000pt}%
\definecolor{currentstroke}{rgb}{0.450000,0.450000,0.450000}%
\pgfsetstrokecolor{currentstroke}%
\pgfsetdash{}{0pt}%
\pgfpathmoveto{\pgfqpoint{2.581262in}{0.416447in}}%
\pgfpathlineto{\pgfqpoint{2.581262in}{2.468330in}}%
\pgfusepath{stroke}%
\end{pgfscope}%
\begin{pgfscope}%
\pgfsetbuttcap%
\pgfsetroundjoin%
\definecolor{currentfill}{rgb}{0.000000,0.000000,0.000000}%
\pgfsetfillcolor{currentfill}%
\pgfsetlinewidth{0.803000pt}%
\definecolor{currentstroke}{rgb}{0.000000,0.000000,0.000000}%
\pgfsetstrokecolor{currentstroke}%
\pgfsetdash{}{0pt}%
\pgfsys@defobject{currentmarker}{\pgfqpoint{0.000000in}{-0.048611in}}{\pgfqpoint{0.000000in}{0.000000in}}{%
\pgfpathmoveto{\pgfqpoint{0.000000in}{0.000000in}}%
\pgfpathlineto{\pgfqpoint{0.000000in}{-0.048611in}}%
\pgfusepath{stroke,fill}%
}%
\begin{pgfscope}%
\pgfsys@transformshift{2.581262in}{0.416447in}%
\pgfsys@useobject{currentmarker}{}%
\end{pgfscope}%
\end{pgfscope}%
\begin{pgfscope}%
\definecolor{textcolor}{rgb}{0.000000,0.000000,0.000000}%
\pgfsetstrokecolor{textcolor}%
\pgfsetfillcolor{textcolor}%
\pgftext[x=2.581262in,y=0.319225in,,top]{\color{textcolor}\rmfamily\fontsize{8.000000}{9.600000}\selectfont \(\displaystyle {20}\)}%
\end{pgfscope}%
\begin{pgfscope}%
\pgfpathrectangle{\pgfqpoint{0.589745in}{0.416447in}}{\pgfqpoint{3.390884in}{2.051883in}}%
\pgfusepath{clip}%
\pgfsetrectcap%
\pgfsetroundjoin%
\pgfsetlinewidth{0.803000pt}%
\definecolor{currentstroke}{rgb}{0.450000,0.450000,0.450000}%
\pgfsetstrokecolor{currentstroke}%
\pgfsetdash{}{0pt}%
\pgfpathmoveto{\pgfqpoint{3.040609in}{0.416447in}}%
\pgfpathlineto{\pgfqpoint{3.040609in}{2.468330in}}%
\pgfusepath{stroke}%
\end{pgfscope}%
\begin{pgfscope}%
\pgfsetbuttcap%
\pgfsetroundjoin%
\definecolor{currentfill}{rgb}{0.000000,0.000000,0.000000}%
\pgfsetfillcolor{currentfill}%
\pgfsetlinewidth{0.803000pt}%
\definecolor{currentstroke}{rgb}{0.000000,0.000000,0.000000}%
\pgfsetstrokecolor{currentstroke}%
\pgfsetdash{}{0pt}%
\pgfsys@defobject{currentmarker}{\pgfqpoint{0.000000in}{-0.048611in}}{\pgfqpoint{0.000000in}{0.000000in}}{%
\pgfpathmoveto{\pgfqpoint{0.000000in}{0.000000in}}%
\pgfpathlineto{\pgfqpoint{0.000000in}{-0.048611in}}%
\pgfusepath{stroke,fill}%
}%
\begin{pgfscope}%
\pgfsys@transformshift{3.040609in}{0.416447in}%
\pgfsys@useobject{currentmarker}{}%
\end{pgfscope}%
\end{pgfscope}%
\begin{pgfscope}%
\definecolor{textcolor}{rgb}{0.000000,0.000000,0.000000}%
\pgfsetstrokecolor{textcolor}%
\pgfsetfillcolor{textcolor}%
\pgftext[x=3.040609in,y=0.319225in,,top]{\color{textcolor}\rmfamily\fontsize{8.000000}{9.600000}\selectfont \(\displaystyle {25}\)}%
\end{pgfscope}%
\begin{pgfscope}%
\pgfpathrectangle{\pgfqpoint{0.589745in}{0.416447in}}{\pgfqpoint{3.390884in}{2.051883in}}%
\pgfusepath{clip}%
\pgfsetrectcap%
\pgfsetroundjoin%
\pgfsetlinewidth{0.803000pt}%
\definecolor{currentstroke}{rgb}{0.450000,0.450000,0.450000}%
\pgfsetstrokecolor{currentstroke}%
\pgfsetdash{}{0pt}%
\pgfpathmoveto{\pgfqpoint{3.499955in}{0.416447in}}%
\pgfpathlineto{\pgfqpoint{3.499955in}{2.468330in}}%
\pgfusepath{stroke}%
\end{pgfscope}%
\begin{pgfscope}%
\pgfsetbuttcap%
\pgfsetroundjoin%
\definecolor{currentfill}{rgb}{0.000000,0.000000,0.000000}%
\pgfsetfillcolor{currentfill}%
\pgfsetlinewidth{0.803000pt}%
\definecolor{currentstroke}{rgb}{0.000000,0.000000,0.000000}%
\pgfsetstrokecolor{currentstroke}%
\pgfsetdash{}{0pt}%
\pgfsys@defobject{currentmarker}{\pgfqpoint{0.000000in}{-0.048611in}}{\pgfqpoint{0.000000in}{0.000000in}}{%
\pgfpathmoveto{\pgfqpoint{0.000000in}{0.000000in}}%
\pgfpathlineto{\pgfqpoint{0.000000in}{-0.048611in}}%
\pgfusepath{stroke,fill}%
}%
\begin{pgfscope}%
\pgfsys@transformshift{3.499955in}{0.416447in}%
\pgfsys@useobject{currentmarker}{}%
\end{pgfscope}%
\end{pgfscope}%
\begin{pgfscope}%
\definecolor{textcolor}{rgb}{0.000000,0.000000,0.000000}%
\pgfsetstrokecolor{textcolor}%
\pgfsetfillcolor{textcolor}%
\pgftext[x=3.499955in,y=0.319225in,,top]{\color{textcolor}\rmfamily\fontsize{8.000000}{9.600000}\selectfont \(\displaystyle {30}\)}%
\end{pgfscope}%
\begin{pgfscope}%
\pgfpathrectangle{\pgfqpoint{0.589745in}{0.416447in}}{\pgfqpoint{3.390884in}{2.051883in}}%
\pgfusepath{clip}%
\pgfsetrectcap%
\pgfsetroundjoin%
\pgfsetlinewidth{0.803000pt}%
\definecolor{currentstroke}{rgb}{0.450000,0.450000,0.450000}%
\pgfsetstrokecolor{currentstroke}%
\pgfsetdash{}{0pt}%
\pgfpathmoveto{\pgfqpoint{3.959302in}{0.416447in}}%
\pgfpathlineto{\pgfqpoint{3.959302in}{2.468330in}}%
\pgfusepath{stroke}%
\end{pgfscope}%
\begin{pgfscope}%
\pgfsetbuttcap%
\pgfsetroundjoin%
\definecolor{currentfill}{rgb}{0.000000,0.000000,0.000000}%
\pgfsetfillcolor{currentfill}%
\pgfsetlinewidth{0.803000pt}%
\definecolor{currentstroke}{rgb}{0.000000,0.000000,0.000000}%
\pgfsetstrokecolor{currentstroke}%
\pgfsetdash{}{0pt}%
\pgfsys@defobject{currentmarker}{\pgfqpoint{0.000000in}{-0.048611in}}{\pgfqpoint{0.000000in}{0.000000in}}{%
\pgfpathmoveto{\pgfqpoint{0.000000in}{0.000000in}}%
\pgfpathlineto{\pgfqpoint{0.000000in}{-0.048611in}}%
\pgfusepath{stroke,fill}%
}%
\begin{pgfscope}%
\pgfsys@transformshift{3.959302in}{0.416447in}%
\pgfsys@useobject{currentmarker}{}%
\end{pgfscope}%
\end{pgfscope}%
\begin{pgfscope}%
\definecolor{textcolor}{rgb}{0.000000,0.000000,0.000000}%
\pgfsetstrokecolor{textcolor}%
\pgfsetfillcolor{textcolor}%
\pgftext[x=3.959302in,y=0.319225in,,top]{\color{textcolor}\rmfamily\fontsize{8.000000}{9.600000}\selectfont \(\displaystyle {35}\)}%
\end{pgfscope}%
\begin{pgfscope}%
\definecolor{textcolor}{rgb}{0.000000,0.000000,0.000000}%
\pgfsetstrokecolor{textcolor}%
\pgfsetfillcolor{textcolor}%
\pgftext[x=2.285187in,y=0.165003in,,top]{\color{textcolor}\rmfamily\fontsize{10.000000}{12.000000}\selectfont Time in \(\displaystyle \unit{\second}\)}%
\end{pgfscope}%
\begin{pgfscope}%
\pgfpathrectangle{\pgfqpoint{0.589745in}{0.416447in}}{\pgfqpoint{3.390884in}{2.051883in}}%
\pgfusepath{clip}%
\pgfsetrectcap%
\pgfsetroundjoin%
\pgfsetlinewidth{0.803000pt}%
\definecolor{currentstroke}{rgb}{0.450000,0.450000,0.450000}%
\pgfsetstrokecolor{currentstroke}%
\pgfsetdash{}{0pt}%
\pgfpathmoveto{\pgfqpoint{0.589745in}{0.543913in}}%
\pgfpathlineto{\pgfqpoint{3.980629in}{0.543913in}}%
\pgfusepath{stroke}%
\end{pgfscope}%
\begin{pgfscope}%
\pgfsetbuttcap%
\pgfsetroundjoin%
\definecolor{currentfill}{rgb}{0.000000,0.000000,0.000000}%
\pgfsetfillcolor{currentfill}%
\pgfsetlinewidth{0.803000pt}%
\definecolor{currentstroke}{rgb}{0.000000,0.000000,0.000000}%
\pgfsetstrokecolor{currentstroke}%
\pgfsetdash{}{0pt}%
\pgfsys@defobject{currentmarker}{\pgfqpoint{-0.048611in}{0.000000in}}{\pgfqpoint{-0.000000in}{0.000000in}}{%
\pgfpathmoveto{\pgfqpoint{-0.000000in}{0.000000in}}%
\pgfpathlineto{\pgfqpoint{-0.048611in}{0.000000in}}%
\pgfusepath{stroke,fill}%
}%
\begin{pgfscope}%
\pgfsys@transformshift{0.589745in}{0.543913in}%
\pgfsys@useobject{currentmarker}{}%
\end{pgfscope}%
\end{pgfscope}%
\begin{pgfscope}%
\definecolor{textcolor}{rgb}{0.000000,0.000000,0.000000}%
\pgfsetstrokecolor{textcolor}%
\pgfsetfillcolor{textcolor}%
\pgftext[x=0.223614in, y=0.505357in, left, base]{\color{textcolor}\rmfamily\fontsize{8.000000}{9.600000}\selectfont \(\displaystyle {\ensuremath{-}200}\)}%
\end{pgfscope}%
\begin{pgfscope}%
\pgfpathrectangle{\pgfqpoint{0.589745in}{0.416447in}}{\pgfqpoint{3.390884in}{2.051883in}}%
\pgfusepath{clip}%
\pgfsetrectcap%
\pgfsetroundjoin%
\pgfsetlinewidth{0.803000pt}%
\definecolor{currentstroke}{rgb}{0.450000,0.450000,0.450000}%
\pgfsetstrokecolor{currentstroke}%
\pgfsetdash{}{0pt}%
\pgfpathmoveto{\pgfqpoint{0.589745in}{0.930619in}}%
\pgfpathlineto{\pgfqpoint{3.980629in}{0.930619in}}%
\pgfusepath{stroke}%
\end{pgfscope}%
\begin{pgfscope}%
\pgfsetbuttcap%
\pgfsetroundjoin%
\definecolor{currentfill}{rgb}{0.000000,0.000000,0.000000}%
\pgfsetfillcolor{currentfill}%
\pgfsetlinewidth{0.803000pt}%
\definecolor{currentstroke}{rgb}{0.000000,0.000000,0.000000}%
\pgfsetstrokecolor{currentstroke}%
\pgfsetdash{}{0pt}%
\pgfsys@defobject{currentmarker}{\pgfqpoint{-0.048611in}{0.000000in}}{\pgfqpoint{-0.000000in}{0.000000in}}{%
\pgfpathmoveto{\pgfqpoint{-0.000000in}{0.000000in}}%
\pgfpathlineto{\pgfqpoint{-0.048611in}{0.000000in}}%
\pgfusepath{stroke,fill}%
}%
\begin{pgfscope}%
\pgfsys@transformshift{0.589745in}{0.930619in}%
\pgfsys@useobject{currentmarker}{}%
\end{pgfscope}%
\end{pgfscope}%
\begin{pgfscope}%
\definecolor{textcolor}{rgb}{0.000000,0.000000,0.000000}%
\pgfsetstrokecolor{textcolor}%
\pgfsetfillcolor{textcolor}%
\pgftext[x=0.223614in, y=0.892063in, left, base]{\color{textcolor}\rmfamily\fontsize{8.000000}{9.600000}\selectfont \(\displaystyle {\ensuremath{-}100}\)}%
\end{pgfscope}%
\begin{pgfscope}%
\pgfpathrectangle{\pgfqpoint{0.589745in}{0.416447in}}{\pgfqpoint{3.390884in}{2.051883in}}%
\pgfusepath{clip}%
\pgfsetrectcap%
\pgfsetroundjoin%
\pgfsetlinewidth{0.803000pt}%
\definecolor{currentstroke}{rgb}{0.450000,0.450000,0.450000}%
\pgfsetstrokecolor{currentstroke}%
\pgfsetdash{}{0pt}%
\pgfpathmoveto{\pgfqpoint{0.589745in}{1.317324in}}%
\pgfpathlineto{\pgfqpoint{3.980629in}{1.317324in}}%
\pgfusepath{stroke}%
\end{pgfscope}%
\begin{pgfscope}%
\pgfsetbuttcap%
\pgfsetroundjoin%
\definecolor{currentfill}{rgb}{0.000000,0.000000,0.000000}%
\pgfsetfillcolor{currentfill}%
\pgfsetlinewidth{0.803000pt}%
\definecolor{currentstroke}{rgb}{0.000000,0.000000,0.000000}%
\pgfsetstrokecolor{currentstroke}%
\pgfsetdash{}{0pt}%
\pgfsys@defobject{currentmarker}{\pgfqpoint{-0.048611in}{0.000000in}}{\pgfqpoint{-0.000000in}{0.000000in}}{%
\pgfpathmoveto{\pgfqpoint{-0.000000in}{0.000000in}}%
\pgfpathlineto{\pgfqpoint{-0.048611in}{0.000000in}}%
\pgfusepath{stroke,fill}%
}%
\begin{pgfscope}%
\pgfsys@transformshift{0.589745in}{1.317324in}%
\pgfsys@useobject{currentmarker}{}%
\end{pgfscope}%
\end{pgfscope}%
\begin{pgfscope}%
\definecolor{textcolor}{rgb}{0.000000,0.000000,0.000000}%
\pgfsetstrokecolor{textcolor}%
\pgfsetfillcolor{textcolor}%
\pgftext[x=0.433494in, y=1.278768in, left, base]{\color{textcolor}\rmfamily\fontsize{8.000000}{9.600000}\selectfont \(\displaystyle {0}\)}%
\end{pgfscope}%
\begin{pgfscope}%
\pgfpathrectangle{\pgfqpoint{0.589745in}{0.416447in}}{\pgfqpoint{3.390884in}{2.051883in}}%
\pgfusepath{clip}%
\pgfsetrectcap%
\pgfsetroundjoin%
\pgfsetlinewidth{0.803000pt}%
\definecolor{currentstroke}{rgb}{0.450000,0.450000,0.450000}%
\pgfsetstrokecolor{currentstroke}%
\pgfsetdash{}{0pt}%
\pgfpathmoveto{\pgfqpoint{0.589745in}{1.704029in}}%
\pgfpathlineto{\pgfqpoint{3.980629in}{1.704029in}}%
\pgfusepath{stroke}%
\end{pgfscope}%
\begin{pgfscope}%
\pgfsetbuttcap%
\pgfsetroundjoin%
\definecolor{currentfill}{rgb}{0.000000,0.000000,0.000000}%
\pgfsetfillcolor{currentfill}%
\pgfsetlinewidth{0.803000pt}%
\definecolor{currentstroke}{rgb}{0.000000,0.000000,0.000000}%
\pgfsetstrokecolor{currentstroke}%
\pgfsetdash{}{0pt}%
\pgfsys@defobject{currentmarker}{\pgfqpoint{-0.048611in}{0.000000in}}{\pgfqpoint{-0.000000in}{0.000000in}}{%
\pgfpathmoveto{\pgfqpoint{-0.000000in}{0.000000in}}%
\pgfpathlineto{\pgfqpoint{-0.048611in}{0.000000in}}%
\pgfusepath{stroke,fill}%
}%
\begin{pgfscope}%
\pgfsys@transformshift{0.589745in}{1.704029in}%
\pgfsys@useobject{currentmarker}{}%
\end{pgfscope}%
\end{pgfscope}%
\begin{pgfscope}%
\definecolor{textcolor}{rgb}{0.000000,0.000000,0.000000}%
\pgfsetstrokecolor{textcolor}%
\pgfsetfillcolor{textcolor}%
\pgftext[x=0.315437in, y=1.665474in, left, base]{\color{textcolor}\rmfamily\fontsize{8.000000}{9.600000}\selectfont \(\displaystyle {100}\)}%
\end{pgfscope}%
\begin{pgfscope}%
\pgfpathrectangle{\pgfqpoint{0.589745in}{0.416447in}}{\pgfqpoint{3.390884in}{2.051883in}}%
\pgfusepath{clip}%
\pgfsetrectcap%
\pgfsetroundjoin%
\pgfsetlinewidth{0.803000pt}%
\definecolor{currentstroke}{rgb}{0.450000,0.450000,0.450000}%
\pgfsetstrokecolor{currentstroke}%
\pgfsetdash{}{0pt}%
\pgfpathmoveto{\pgfqpoint{0.589745in}{2.090735in}}%
\pgfpathlineto{\pgfqpoint{3.980629in}{2.090735in}}%
\pgfusepath{stroke}%
\end{pgfscope}%
\begin{pgfscope}%
\pgfsetbuttcap%
\pgfsetroundjoin%
\definecolor{currentfill}{rgb}{0.000000,0.000000,0.000000}%
\pgfsetfillcolor{currentfill}%
\pgfsetlinewidth{0.803000pt}%
\definecolor{currentstroke}{rgb}{0.000000,0.000000,0.000000}%
\pgfsetstrokecolor{currentstroke}%
\pgfsetdash{}{0pt}%
\pgfsys@defobject{currentmarker}{\pgfqpoint{-0.048611in}{0.000000in}}{\pgfqpoint{-0.000000in}{0.000000in}}{%
\pgfpathmoveto{\pgfqpoint{-0.000000in}{0.000000in}}%
\pgfpathlineto{\pgfqpoint{-0.048611in}{0.000000in}}%
\pgfusepath{stroke,fill}%
}%
\begin{pgfscope}%
\pgfsys@transformshift{0.589745in}{2.090735in}%
\pgfsys@useobject{currentmarker}{}%
\end{pgfscope}%
\end{pgfscope}%
\begin{pgfscope}%
\definecolor{textcolor}{rgb}{0.000000,0.000000,0.000000}%
\pgfsetstrokecolor{textcolor}%
\pgfsetfillcolor{textcolor}%
\pgftext[x=0.315437in, y=2.052179in, left, base]{\color{textcolor}\rmfamily\fontsize{8.000000}{9.600000}\selectfont \(\displaystyle {200}\)}%
\end{pgfscope}%
\begin{pgfscope}%
\definecolor{textcolor}{rgb}{0.000000,0.000000,0.000000}%
\pgfsetstrokecolor{textcolor}%
\pgfsetfillcolor{textcolor}%
\pgftext[x=0.168059in,y=1.442389in,,bottom,rotate=90.000000]{\color{textcolor}\rmfamily\fontsize{10.000000}{12.000000}\selectfont Ampl. in arb. unit}%
\end{pgfscope}%
\begin{pgfscope}%
\pgfpathrectangle{\pgfqpoint{0.589745in}{0.416447in}}{\pgfqpoint{3.390884in}{2.051883in}}%
\pgfusepath{clip}%
\pgfsetrectcap%
\pgfsetroundjoin%
\pgfsetlinewidth{1.505625pt}%
\definecolor{currentstroke}{rgb}{0.337255,0.705882,0.913725}%
\pgfsetstrokecolor{currentstroke}%
\pgfsetdash{}{0pt}%
\pgfpathmoveto{\pgfqpoint{0.743876in}{1.297799in}}%
\pgfpathlineto{\pgfqpoint{0.745331in}{1.874121in}}%
\pgfpathlineto{\pgfqpoint{0.745683in}{0.859778in}}%
\pgfpathlineto{\pgfqpoint{0.746995in}{1.664025in}}%
\pgfpathlineto{\pgfqpoint{0.748963in}{0.830471in}}%
\pgfpathlineto{\pgfqpoint{0.750274in}{1.762984in}}%
\pgfpathlineto{\pgfqpoint{0.751608in}{0.908221in}}%
\pgfpathlineto{\pgfqpoint{0.753341in}{1.759531in}}%
\pgfpathlineto{\pgfqpoint{0.754691in}{0.818889in}}%
\pgfpathlineto{\pgfqpoint{0.756896in}{1.839884in}}%
\pgfpathlineto{\pgfqpoint{0.757793in}{0.876218in}}%
\pgfpathlineto{\pgfqpoint{0.759383in}{1.712915in}}%
\pgfpathlineto{\pgfqpoint{0.761003in}{0.807165in}}%
\pgfpathlineto{\pgfqpoint{0.762478in}{1.762420in}}%
\pgfpathlineto{\pgfqpoint{0.764579in}{0.739193in}}%
\pgfpathlineto{\pgfqpoint{0.765556in}{1.706135in}}%
\pgfpathlineto{\pgfqpoint{0.767033in}{0.795975in}}%
\pgfpathlineto{\pgfqpoint{0.768708in}{1.723526in}}%
\pgfpathlineto{\pgfqpoint{0.770213in}{0.858140in}}%
\pgfpathlineto{\pgfqpoint{0.772225in}{1.800953in}}%
\pgfpathlineto{\pgfqpoint{0.773586in}{0.821581in}}%
\pgfpathlineto{\pgfqpoint{0.774880in}{1.811499in}}%
\pgfpathlineto{\pgfqpoint{0.776345in}{0.934310in}}%
\pgfpathlineto{\pgfqpoint{0.777957in}{1.780823in}}%
\pgfpathlineto{\pgfqpoint{0.779518in}{0.845331in}}%
\pgfpathlineto{\pgfqpoint{0.781019in}{1.791155in}}%
\pgfpathlineto{\pgfqpoint{0.782524in}{0.890867in}}%
\pgfpathlineto{\pgfqpoint{0.784094in}{1.694465in}}%
\pgfpathlineto{\pgfqpoint{0.785761in}{0.853835in}}%
\pgfpathlineto{\pgfqpoint{0.787334in}{1.743791in}}%
\pgfpathlineto{\pgfqpoint{0.788740in}{0.846893in}}%
\pgfpathlineto{\pgfqpoint{0.790401in}{1.796028in}}%
\pgfpathlineto{\pgfqpoint{0.791976in}{0.852017in}}%
\pgfpathlineto{\pgfqpoint{0.793255in}{1.806011in}}%
\pgfpathlineto{\pgfqpoint{0.795163in}{0.903563in}}%
\pgfpathlineto{\pgfqpoint{0.796339in}{1.739169in}}%
\pgfpathlineto{\pgfqpoint{0.798004in}{0.884500in}}%
\pgfpathlineto{\pgfqpoint{0.799490in}{1.854530in}}%
\pgfpathlineto{\pgfqpoint{0.800965in}{0.926065in}}%
\pgfpathlineto{\pgfqpoint{0.802531in}{1.745330in}}%
\pgfpathlineto{\pgfqpoint{0.804131in}{0.826368in}}%
\pgfpathlineto{\pgfqpoint{0.805848in}{1.775138in}}%
\pgfpathlineto{\pgfqpoint{0.807714in}{0.813058in}}%
\pgfpathlineto{\pgfqpoint{0.808697in}{1.770097in}}%
\pgfpathlineto{\pgfqpoint{0.811189in}{0.886305in}}%
\pgfpathlineto{\pgfqpoint{0.811809in}{1.781075in}}%
\pgfpathlineto{\pgfqpoint{0.813548in}{0.930014in}}%
\pgfpathlineto{\pgfqpoint{0.815039in}{1.857290in}}%
\pgfpathlineto{\pgfqpoint{0.816593in}{0.977023in}}%
\pgfpathlineto{\pgfqpoint{0.818003in}{1.821673in}}%
\pgfpathlineto{\pgfqpoint{0.819620in}{0.930904in}}%
\pgfpathlineto{\pgfqpoint{0.821055in}{1.788775in}}%
\pgfpathlineto{\pgfqpoint{0.822801in}{0.914643in}}%
\pgfpathlineto{\pgfqpoint{0.824316in}{1.756955in}}%
\pgfpathlineto{\pgfqpoint{0.825866in}{0.920444in}}%
\pgfpathlineto{\pgfqpoint{0.828179in}{1.969586in}}%
\pgfpathlineto{\pgfqpoint{0.828806in}{0.950286in}}%
\pgfpathlineto{\pgfqpoint{0.830445in}{1.832830in}}%
\pgfpathlineto{\pgfqpoint{0.831896in}{0.963551in}}%
\pgfpathlineto{\pgfqpoint{0.833384in}{1.803781in}}%
\pgfpathlineto{\pgfqpoint{0.835090in}{0.955823in}}%
\pgfpathlineto{\pgfqpoint{0.836473in}{1.893128in}}%
\pgfpathlineto{\pgfqpoint{0.838026in}{0.966476in}}%
\pgfpathlineto{\pgfqpoint{0.839570in}{1.850274in}}%
\pgfpathlineto{\pgfqpoint{0.841079in}{0.904578in}}%
\pgfpathlineto{\pgfqpoint{0.842644in}{1.903032in}}%
\pgfpathlineto{\pgfqpoint{0.844232in}{0.772813in}}%
\pgfpathlineto{\pgfqpoint{0.846169in}{1.875663in}}%
\pgfpathlineto{\pgfqpoint{0.847507in}{0.928535in}}%
\pgfpathlineto{\pgfqpoint{0.848881in}{1.733728in}}%
\pgfpathlineto{\pgfqpoint{0.850437in}{0.834089in}}%
\pgfpathlineto{\pgfqpoint{0.852000in}{1.729618in}}%
\pgfpathlineto{\pgfqpoint{0.853469in}{0.804240in}}%
\pgfpathlineto{\pgfqpoint{0.855090in}{1.787235in}}%
\pgfpathlineto{\pgfqpoint{0.856867in}{0.899194in}}%
\pgfpathlineto{\pgfqpoint{0.858532in}{1.817113in}}%
\pgfpathlineto{\pgfqpoint{0.859634in}{0.971042in}}%
\pgfpathlineto{\pgfqpoint{0.861259in}{1.864864in}}%
\pgfpathlineto{\pgfqpoint{0.862909in}{0.881256in}}%
\pgfpathlineto{\pgfqpoint{0.864484in}{1.762204in}}%
\pgfpathlineto{\pgfqpoint{0.865862in}{0.837951in}}%
\pgfpathlineto{\pgfqpoint{0.867544in}{1.776973in}}%
\pgfpathlineto{\pgfqpoint{0.869414in}{0.865320in}}%
\pgfpathlineto{\pgfqpoint{0.870455in}{1.764743in}}%
\pgfpathlineto{\pgfqpoint{0.872450in}{0.821411in}}%
\pgfpathlineto{\pgfqpoint{0.873546in}{1.786066in}}%
\pgfpathlineto{\pgfqpoint{0.875289in}{0.804681in}}%
\pgfpathlineto{\pgfqpoint{0.876906in}{1.782800in}}%
\pgfpathlineto{\pgfqpoint{0.878306in}{0.895620in}}%
\pgfpathlineto{\pgfqpoint{0.879760in}{1.754396in}}%
\pgfpathlineto{\pgfqpoint{0.881356in}{0.851536in}}%
\pgfpathlineto{\pgfqpoint{0.883185in}{1.793764in}}%
\pgfpathlineto{\pgfqpoint{0.884766in}{0.745105in}}%
\pgfpathlineto{\pgfqpoint{0.886048in}{1.673104in}}%
\pgfpathlineto{\pgfqpoint{0.888448in}{0.677683in}}%
\pgfpathlineto{\pgfqpoint{0.889028in}{1.734404in}}%
\pgfpathlineto{\pgfqpoint{0.891117in}{0.764823in}}%
\pgfpathlineto{\pgfqpoint{0.892201in}{1.719667in}}%
\pgfpathlineto{\pgfqpoint{0.893720in}{0.872943in}}%
\pgfpathlineto{\pgfqpoint{0.895231in}{1.733110in}}%
\pgfpathlineto{\pgfqpoint{0.896739in}{0.763979in}}%
\pgfpathlineto{\pgfqpoint{0.898634in}{1.709061in}}%
\pgfpathlineto{\pgfqpoint{0.899710in}{0.801714in}}%
\pgfpathlineto{\pgfqpoint{0.901460in}{1.731501in}}%
\pgfpathlineto{\pgfqpoint{0.902838in}{0.806694in}}%
\pgfpathlineto{\pgfqpoint{0.904337in}{1.704759in}}%
\pgfpathlineto{\pgfqpoint{0.905974in}{0.730998in}}%
\pgfpathlineto{\pgfqpoint{0.907622in}{1.682169in}}%
\pgfpathlineto{\pgfqpoint{0.909112in}{0.755601in}}%
\pgfpathlineto{\pgfqpoint{0.910644in}{1.604921in}}%
\pgfpathlineto{\pgfqpoint{0.912322in}{0.737617in}}%
\pgfpathlineto{\pgfqpoint{0.914014in}{1.737162in}}%
\pgfpathlineto{\pgfqpoint{0.915387in}{0.759134in}}%
\pgfpathlineto{\pgfqpoint{0.916740in}{1.671603in}}%
\pgfpathlineto{\pgfqpoint{0.918923in}{0.636516in}}%
\pgfpathlineto{\pgfqpoint{0.920172in}{1.787725in}}%
\pgfpathlineto{\pgfqpoint{0.921390in}{0.819113in}}%
\pgfpathlineto{\pgfqpoint{0.923536in}{1.738818in}}%
\pgfpathlineto{\pgfqpoint{0.924630in}{0.739239in}}%
\pgfpathlineto{\pgfqpoint{0.925945in}{1.663831in}}%
\pgfpathlineto{\pgfqpoint{0.927512in}{0.839605in}}%
\pgfpathlineto{\pgfqpoint{0.929659in}{1.751647in}}%
\pgfpathlineto{\pgfqpoint{0.930872in}{0.826787in}}%
\pgfpathlineto{\pgfqpoint{0.932196in}{1.696667in}}%
\pgfpathlineto{\pgfqpoint{0.933743in}{0.799316in}}%
\pgfpathlineto{\pgfqpoint{0.935198in}{1.714145in}}%
\pgfpathlineto{\pgfqpoint{0.936864in}{0.846102in}}%
\pgfpathlineto{\pgfqpoint{0.938296in}{1.755656in}}%
\pgfpathlineto{\pgfqpoint{0.940142in}{0.705848in}}%
\pgfpathlineto{\pgfqpoint{0.941571in}{1.702212in}}%
\pgfpathlineto{\pgfqpoint{0.943335in}{0.802625in}}%
\pgfpathlineto{\pgfqpoint{0.944699in}{1.771485in}}%
\pgfpathlineto{\pgfqpoint{0.946893in}{0.759365in}}%
\pgfpathlineto{\pgfqpoint{0.947614in}{1.738941in}}%
\pgfpathlineto{\pgfqpoint{0.949457in}{0.874325in}}%
\pgfpathlineto{\pgfqpoint{0.950836in}{1.741433in}}%
\pgfpathlineto{\pgfqpoint{0.952181in}{0.855947in}}%
\pgfpathlineto{\pgfqpoint{0.953934in}{1.874134in}}%
\pgfpathlineto{\pgfqpoint{0.955810in}{0.721658in}}%
\pgfpathlineto{\pgfqpoint{0.956900in}{1.769561in}}%
\pgfpathlineto{\pgfqpoint{0.958441in}{0.787771in}}%
\pgfpathlineto{\pgfqpoint{0.960044in}{1.702919in}}%
\pgfpathlineto{\pgfqpoint{0.961567in}{0.830476in}}%
\pgfpathlineto{\pgfqpoint{0.962998in}{1.698178in}}%
\pgfpathlineto{\pgfqpoint{0.964894in}{0.848038in}}%
\pgfpathlineto{\pgfqpoint{0.966116in}{1.735536in}}%
\pgfpathlineto{\pgfqpoint{0.967637in}{0.860217in}}%
\pgfpathlineto{\pgfqpoint{0.969340in}{1.793605in}}%
\pgfpathlineto{\pgfqpoint{0.971437in}{0.819299in}}%
\pgfpathlineto{\pgfqpoint{0.972270in}{1.710175in}}%
\pgfpathlineto{\pgfqpoint{0.974034in}{0.859809in}}%
\pgfpathlineto{\pgfqpoint{0.975989in}{1.851597in}}%
\pgfpathlineto{\pgfqpoint{0.976926in}{0.808385in}}%
\pgfpathlineto{\pgfqpoint{0.978643in}{1.785623in}}%
\pgfpathlineto{\pgfqpoint{0.980433in}{0.764456in}}%
\pgfpathlineto{\pgfqpoint{0.981681in}{1.775202in}}%
\pgfpathlineto{\pgfqpoint{0.983391in}{0.801619in}}%
\pgfpathlineto{\pgfqpoint{0.985138in}{1.762647in}}%
\pgfpathlineto{\pgfqpoint{0.986511in}{0.693028in}}%
\pgfpathlineto{\pgfqpoint{0.987749in}{1.778127in}}%
\pgfpathlineto{\pgfqpoint{0.989255in}{0.868304in}}%
\pgfpathlineto{\pgfqpoint{0.990797in}{1.713347in}}%
\pgfpathlineto{\pgfqpoint{0.992433in}{0.778055in}}%
\pgfpathlineto{\pgfqpoint{0.994734in}{1.851912in}}%
\pgfpathlineto{\pgfqpoint{0.995543in}{0.819371in}}%
\pgfpathlineto{\pgfqpoint{0.996968in}{1.757246in}}%
\pgfpathlineto{\pgfqpoint{0.998508in}{0.882814in}}%
\pgfpathlineto{\pgfqpoint{1.000634in}{1.846031in}}%
\pgfpathlineto{\pgfqpoint{1.002027in}{0.787407in}}%
\pgfpathlineto{\pgfqpoint{1.003878in}{1.813862in}}%
\pgfpathlineto{\pgfqpoint{1.004670in}{0.881645in}}%
\pgfpathlineto{\pgfqpoint{1.006484in}{1.872637in}}%
\pgfpathlineto{\pgfqpoint{1.007802in}{0.931889in}}%
\pgfpathlineto{\pgfqpoint{1.009282in}{1.786894in}}%
\pgfpathlineto{\pgfqpoint{1.011047in}{0.885349in}}%
\pgfpathlineto{\pgfqpoint{1.012388in}{1.787340in}}%
\pgfpathlineto{\pgfqpoint{1.013978in}{0.919111in}}%
\pgfpathlineto{\pgfqpoint{1.015811in}{1.751559in}}%
\pgfpathlineto{\pgfqpoint{1.017146in}{0.914133in}}%
\pgfpathlineto{\pgfqpoint{1.018592in}{1.826968in}}%
\pgfpathlineto{\pgfqpoint{1.020385in}{0.865492in}}%
\pgfpathlineto{\pgfqpoint{1.022101in}{1.893985in}}%
\pgfpathlineto{\pgfqpoint{1.023173in}{0.944317in}}%
\pgfpathlineto{\pgfqpoint{1.024906in}{1.793568in}}%
\pgfpathlineto{\pgfqpoint{1.026241in}{0.911866in}}%
\pgfpathlineto{\pgfqpoint{1.027771in}{1.799493in}}%
\pgfpathlineto{\pgfqpoint{1.029349in}{0.907704in}}%
\pgfpathlineto{\pgfqpoint{1.031002in}{1.794471in}}%
\pgfpathlineto{\pgfqpoint{1.032439in}{0.962462in}}%
\pgfpathlineto{\pgfqpoint{1.034151in}{1.784787in}}%
\pgfpathlineto{\pgfqpoint{1.036132in}{0.855147in}}%
\pgfpathlineto{\pgfqpoint{1.037047in}{1.725094in}}%
\pgfpathlineto{\pgfqpoint{1.038693in}{0.849480in}}%
\pgfpathlineto{\pgfqpoint{1.040572in}{1.776346in}}%
\pgfpathlineto{\pgfqpoint{1.041847in}{0.881842in}}%
\pgfpathlineto{\pgfqpoint{1.043590in}{1.787761in}}%
\pgfpathlineto{\pgfqpoint{1.045391in}{0.882651in}}%
\pgfpathlineto{\pgfqpoint{1.046289in}{1.800342in}}%
\pgfpathlineto{\pgfqpoint{1.047840in}{0.810546in}}%
\pgfpathlineto{\pgfqpoint{1.049781in}{1.870698in}}%
\pgfpathlineto{\pgfqpoint{1.050984in}{0.926825in}}%
\pgfpathlineto{\pgfqpoint{1.052689in}{1.877091in}}%
\pgfpathlineto{\pgfqpoint{1.054070in}{0.902777in}}%
\pgfpathlineto{\pgfqpoint{1.055543in}{1.779458in}}%
\pgfpathlineto{\pgfqpoint{1.057282in}{0.981959in}}%
\pgfpathlineto{\pgfqpoint{1.058841in}{1.938760in}}%
\pgfpathlineto{\pgfqpoint{1.060339in}{0.880449in}}%
\pgfpathlineto{\pgfqpoint{1.062630in}{1.962319in}}%
\pgfpathlineto{\pgfqpoint{1.063451in}{0.955846in}}%
\pgfpathlineto{\pgfqpoint{1.064937in}{1.850285in}}%
\pgfpathlineto{\pgfqpoint{1.066389in}{0.965798in}}%
\pgfpathlineto{\pgfqpoint{1.067945in}{1.855949in}}%
\pgfpathlineto{\pgfqpoint{1.069593in}{0.971215in}}%
\pgfpathlineto{\pgfqpoint{1.071213in}{1.889938in}}%
\pgfpathlineto{\pgfqpoint{1.072553in}{0.948848in}}%
\pgfpathlineto{\pgfqpoint{1.074275in}{1.781720in}}%
\pgfpathlineto{\pgfqpoint{1.076036in}{0.891635in}}%
\pgfpathlineto{\pgfqpoint{1.077662in}{1.868859in}}%
\pgfpathlineto{\pgfqpoint{1.078814in}{0.953082in}}%
\pgfpathlineto{\pgfqpoint{1.080229in}{1.800109in}}%
\pgfpathlineto{\pgfqpoint{1.081941in}{1.004090in}}%
\pgfpathlineto{\pgfqpoint{1.083333in}{1.933886in}}%
\pgfpathlineto{\pgfqpoint{1.084904in}{1.001981in}}%
\pgfpathlineto{\pgfqpoint{1.086601in}{1.843029in}}%
\pgfpathlineto{\pgfqpoint{1.088254in}{0.946492in}}%
\pgfpathlineto{\pgfqpoint{1.089750in}{1.925079in}}%
\pgfpathlineto{\pgfqpoint{1.091121in}{1.029144in}}%
\pgfpathlineto{\pgfqpoint{1.092895in}{1.866499in}}%
\pgfpathlineto{\pgfqpoint{1.094129in}{0.951413in}}%
\pgfpathlineto{\pgfqpoint{1.096060in}{1.879047in}}%
\pgfpathlineto{\pgfqpoint{1.097374in}{0.947273in}}%
\pgfpathlineto{\pgfqpoint{1.098807in}{1.867669in}}%
\pgfpathlineto{\pgfqpoint{1.100324in}{0.883421in}}%
\pgfpathlineto{\pgfqpoint{1.101914in}{1.870719in}}%
\pgfpathlineto{\pgfqpoint{1.103410in}{0.945636in}}%
\pgfpathlineto{\pgfqpoint{1.105316in}{1.958863in}}%
\pgfpathlineto{\pgfqpoint{1.107137in}{0.954540in}}%
\pgfpathlineto{\pgfqpoint{1.108229in}{1.958877in}}%
\pgfpathlineto{\pgfqpoint{1.109886in}{0.985918in}}%
\pgfpathlineto{\pgfqpoint{1.111762in}{1.962337in}}%
\pgfpathlineto{\pgfqpoint{1.112853in}{0.913294in}}%
\pgfpathlineto{\pgfqpoint{1.114194in}{1.898409in}}%
\pgfpathlineto{\pgfqpoint{1.115711in}{1.005053in}}%
\pgfpathlineto{\pgfqpoint{1.117674in}{1.893678in}}%
\pgfpathlineto{\pgfqpoint{1.118907in}{0.921409in}}%
\pgfpathlineto{\pgfqpoint{1.120496in}{1.900269in}}%
\pgfpathlineto{\pgfqpoint{1.122034in}{0.990631in}}%
\pgfpathlineto{\pgfqpoint{1.123522in}{1.848414in}}%
\pgfpathlineto{\pgfqpoint{1.126443in}{0.858417in}}%
\pgfpathlineto{\pgfqpoint{1.126543in}{1.928091in}}%
\pgfpathlineto{\pgfqpoint{1.128104in}{1.019972in}}%
\pgfpathlineto{\pgfqpoint{1.129678in}{1.845369in}}%
\pgfpathlineto{\pgfqpoint{1.131243in}{0.993641in}}%
\pgfpathlineto{\pgfqpoint{1.132687in}{1.939966in}}%
\pgfpathlineto{\pgfqpoint{1.134396in}{0.945624in}}%
\pgfpathlineto{\pgfqpoint{1.136102in}{1.947518in}}%
\pgfpathlineto{\pgfqpoint{1.137444in}{0.861593in}}%
\pgfpathlineto{\pgfqpoint{1.139071in}{1.952592in}}%
\pgfpathlineto{\pgfqpoint{1.140840in}{1.001289in}}%
\pgfpathlineto{\pgfqpoint{1.142001in}{1.905718in}}%
\pgfpathlineto{\pgfqpoint{1.143944in}{0.923849in}}%
\pgfpathlineto{\pgfqpoint{1.145040in}{1.903227in}}%
\pgfpathlineto{\pgfqpoint{1.146686in}{0.997069in}}%
\pgfpathlineto{\pgfqpoint{1.148311in}{1.881432in}}%
\pgfpathlineto{\pgfqpoint{1.150115in}{0.976182in}}%
\pgfpathlineto{\pgfqpoint{1.151332in}{1.839205in}}%
\pgfpathlineto{\pgfqpoint{1.152817in}{1.002849in}}%
\pgfpathlineto{\pgfqpoint{1.154423in}{1.883839in}}%
\pgfpathlineto{\pgfqpoint{1.155857in}{0.952066in}}%
\pgfpathlineto{\pgfqpoint{1.157384in}{1.828289in}}%
\pgfpathlineto{\pgfqpoint{1.159374in}{0.909811in}}%
\pgfpathlineto{\pgfqpoint{1.160498in}{1.827163in}}%
\pgfpathlineto{\pgfqpoint{1.162160in}{0.885989in}}%
\pgfpathlineto{\pgfqpoint{1.163545in}{1.786244in}}%
\pgfpathlineto{\pgfqpoint{1.165437in}{0.940672in}}%
\pgfpathlineto{\pgfqpoint{1.166651in}{1.766710in}}%
\pgfpathlineto{\pgfqpoint{1.168201in}{0.952322in}}%
\pgfpathlineto{\pgfqpoint{1.170220in}{1.855006in}}%
\pgfpathlineto{\pgfqpoint{1.171791in}{0.915818in}}%
\pgfpathlineto{\pgfqpoint{1.172900in}{1.843719in}}%
\pgfpathlineto{\pgfqpoint{1.174371in}{0.998381in}}%
\pgfpathlineto{\pgfqpoint{1.176273in}{1.824088in}}%
\pgfpathlineto{\pgfqpoint{1.177690in}{0.936713in}}%
\pgfpathlineto{\pgfqpoint{1.179403in}{1.856823in}}%
\pgfpathlineto{\pgfqpoint{1.180870in}{0.931680in}}%
\pgfpathlineto{\pgfqpoint{1.182438in}{1.929533in}}%
\pgfpathlineto{\pgfqpoint{1.183596in}{0.936902in}}%
\pgfpathlineto{\pgfqpoint{1.185377in}{1.938412in}}%
\pgfpathlineto{\pgfqpoint{1.187111in}{0.937958in}}%
\pgfpathlineto{\pgfqpoint{1.188286in}{1.798947in}}%
\pgfpathlineto{\pgfqpoint{1.190128in}{0.818761in}}%
\pgfpathlineto{\pgfqpoint{1.191329in}{1.829272in}}%
\pgfpathlineto{\pgfqpoint{1.193404in}{0.957889in}}%
\pgfpathlineto{\pgfqpoint{1.194795in}{1.937541in}}%
\pgfpathlineto{\pgfqpoint{1.196290in}{0.933384in}}%
\pgfpathlineto{\pgfqpoint{1.198097in}{1.864858in}}%
\pgfpathlineto{\pgfqpoint{1.199125in}{0.924160in}}%
\pgfpathlineto{\pgfqpoint{1.200761in}{1.827621in}}%
\pgfpathlineto{\pgfqpoint{1.202267in}{0.889088in}}%
\pgfpathlineto{\pgfqpoint{1.204261in}{1.768953in}}%
\pgfpathlineto{\pgfqpoint{1.205811in}{0.828690in}}%
\pgfpathlineto{\pgfqpoint{1.206853in}{1.737813in}}%
\pgfpathlineto{\pgfqpoint{1.208656in}{0.762050in}}%
\pgfpathlineto{\pgfqpoint{1.210254in}{1.821707in}}%
\pgfpathlineto{\pgfqpoint{1.211444in}{0.938176in}}%
\pgfpathlineto{\pgfqpoint{1.212911in}{1.760461in}}%
\pgfpathlineto{\pgfqpoint{1.215507in}{0.731518in}}%
\pgfpathlineto{\pgfqpoint{1.216014in}{1.732873in}}%
\pgfpathlineto{\pgfqpoint{1.217819in}{0.849476in}}%
\pgfpathlineto{\pgfqpoint{1.219614in}{1.877895in}}%
\pgfpathlineto{\pgfqpoint{1.221154in}{0.770239in}}%
\pgfpathlineto{\pgfqpoint{1.222378in}{1.778592in}}%
\pgfpathlineto{\pgfqpoint{1.223761in}{0.924167in}}%
\pgfpathlineto{\pgfqpoint{1.225259in}{1.774460in}}%
\pgfpathlineto{\pgfqpoint{1.226820in}{0.922315in}}%
\pgfpathlineto{\pgfqpoint{1.228675in}{1.795622in}}%
\pgfpathlineto{\pgfqpoint{1.230174in}{0.863846in}}%
\pgfpathlineto{\pgfqpoint{1.231582in}{1.792926in}}%
\pgfpathlineto{\pgfqpoint{1.232987in}{0.892776in}}%
\pgfpathlineto{\pgfqpoint{1.235007in}{1.761847in}}%
\pgfpathlineto{\pgfqpoint{1.236662in}{0.835453in}}%
\pgfpathlineto{\pgfqpoint{1.238095in}{1.779391in}}%
\pgfpathlineto{\pgfqpoint{1.239312in}{0.843763in}}%
\pgfpathlineto{\pgfqpoint{1.240686in}{1.751214in}}%
\pgfpathlineto{\pgfqpoint{1.242409in}{0.908557in}}%
\pgfpathlineto{\pgfqpoint{1.243828in}{1.847710in}}%
\pgfpathlineto{\pgfqpoint{1.245385in}{0.860178in}}%
\pgfpathlineto{\pgfqpoint{1.247271in}{1.767465in}}%
\pgfpathlineto{\pgfqpoint{1.248479in}{0.822999in}}%
\pgfpathlineto{\pgfqpoint{1.250092in}{1.781491in}}%
\pgfpathlineto{\pgfqpoint{1.251636in}{0.870764in}}%
\pgfpathlineto{\pgfqpoint{1.253184in}{1.822058in}}%
\pgfpathlineto{\pgfqpoint{1.254639in}{0.834059in}}%
\pgfpathlineto{\pgfqpoint{1.256210in}{1.728041in}}%
\pgfpathlineto{\pgfqpoint{1.257836in}{0.848851in}}%
\pgfpathlineto{\pgfqpoint{1.259195in}{1.774860in}}%
\pgfpathlineto{\pgfqpoint{1.260753in}{0.815851in}}%
\pgfpathlineto{\pgfqpoint{1.262837in}{1.784926in}}%
\pgfpathlineto{\pgfqpoint{1.263903in}{0.852457in}}%
\pgfpathlineto{\pgfqpoint{1.265793in}{1.785727in}}%
\pgfpathlineto{\pgfqpoint{1.267329in}{0.794622in}}%
\pgfpathlineto{\pgfqpoint{1.268586in}{1.735302in}}%
\pgfpathlineto{\pgfqpoint{1.269999in}{0.826257in}}%
\pgfpathlineto{\pgfqpoint{1.271797in}{1.712469in}}%
\pgfpathlineto{\pgfqpoint{1.273202in}{0.713994in}}%
\pgfpathlineto{\pgfqpoint{1.275583in}{1.868194in}}%
\pgfpathlineto{\pgfqpoint{1.276248in}{0.675410in}}%
\pgfpathlineto{\pgfqpoint{1.278056in}{1.780956in}}%
\pgfpathlineto{\pgfqpoint{1.279566in}{0.739526in}}%
\pgfpathlineto{\pgfqpoint{1.281067in}{1.729907in}}%
\pgfpathlineto{\pgfqpoint{1.282353in}{0.881408in}}%
\pgfpathlineto{\pgfqpoint{1.284821in}{1.805204in}}%
\pgfpathlineto{\pgfqpoint{1.285585in}{0.823278in}}%
\pgfpathlineto{\pgfqpoint{1.287052in}{1.729088in}}%
\pgfpathlineto{\pgfqpoint{1.289046in}{0.709524in}}%
\pgfpathlineto{\pgfqpoint{1.290220in}{1.731008in}}%
\pgfpathlineto{\pgfqpoint{1.291606in}{0.782224in}}%
\pgfpathlineto{\pgfqpoint{1.293233in}{1.701306in}}%
\pgfpathlineto{\pgfqpoint{1.294919in}{0.775570in}}%
\pgfpathlineto{\pgfqpoint{1.296507in}{1.741101in}}%
\pgfpathlineto{\pgfqpoint{1.298051in}{0.790285in}}%
\pgfpathlineto{\pgfqpoint{1.299927in}{1.811718in}}%
\pgfpathlineto{\pgfqpoint{1.300933in}{0.888952in}}%
\pgfpathlineto{\pgfqpoint{1.302451in}{1.733385in}}%
\pgfpathlineto{\pgfqpoint{1.304133in}{0.835493in}}%
\pgfpathlineto{\pgfqpoint{1.306050in}{1.807328in}}%
\pgfpathlineto{\pgfqpoint{1.307115in}{0.777513in}}%
\pgfpathlineto{\pgfqpoint{1.309015in}{1.774119in}}%
\pgfpathlineto{\pgfqpoint{1.310576in}{0.719911in}}%
\pgfpathlineto{\pgfqpoint{1.311897in}{1.740124in}}%
\pgfpathlineto{\pgfqpoint{1.313264in}{0.819157in}}%
\pgfpathlineto{\pgfqpoint{1.315645in}{1.770804in}}%
\pgfpathlineto{\pgfqpoint{1.316546in}{0.789323in}}%
\pgfpathlineto{\pgfqpoint{1.317938in}{1.741547in}}%
\pgfpathlineto{\pgfqpoint{1.320446in}{0.720938in}}%
\pgfpathlineto{\pgfqpoint{1.321007in}{1.792272in}}%
\pgfpathlineto{\pgfqpoint{1.322928in}{0.804990in}}%
\pgfpathlineto{\pgfqpoint{1.324076in}{1.729372in}}%
\pgfpathlineto{\pgfqpoint{1.326087in}{0.812008in}}%
\pgfpathlineto{\pgfqpoint{1.327297in}{1.720478in}}%
\pgfpathlineto{\pgfqpoint{1.328677in}{0.884461in}}%
\pgfpathlineto{\pgfqpoint{1.330219in}{1.692196in}}%
\pgfpathlineto{\pgfqpoint{1.331741in}{0.836414in}}%
\pgfpathlineto{\pgfqpoint{1.333687in}{1.886431in}}%
\pgfpathlineto{\pgfqpoint{1.335139in}{0.796608in}}%
\pgfpathlineto{\pgfqpoint{1.336662in}{1.799506in}}%
\pgfpathlineto{\pgfqpoint{1.337912in}{0.906182in}}%
\pgfpathlineto{\pgfqpoint{1.339433in}{1.898612in}}%
\pgfpathlineto{\pgfqpoint{1.340994in}{0.825811in}}%
\pgfpathlineto{\pgfqpoint{1.342626in}{1.850045in}}%
\pgfpathlineto{\pgfqpoint{1.344103in}{0.804065in}}%
\pgfpathlineto{\pgfqpoint{1.345699in}{1.873514in}}%
\pgfpathlineto{\pgfqpoint{1.347248in}{0.831878in}}%
\pgfpathlineto{\pgfqpoint{1.348807in}{1.719160in}}%
\pgfpathlineto{\pgfqpoint{1.350589in}{0.795976in}}%
\pgfpathlineto{\pgfqpoint{1.352108in}{1.775959in}}%
\pgfpathlineto{\pgfqpoint{1.353335in}{0.867214in}}%
\pgfpathlineto{\pgfqpoint{1.355276in}{1.774397in}}%
\pgfpathlineto{\pgfqpoint{1.356664in}{0.856095in}}%
\pgfpathlineto{\pgfqpoint{1.357953in}{1.872409in}}%
\pgfpathlineto{\pgfqpoint{1.359574in}{0.830200in}}%
\pgfpathlineto{\pgfqpoint{1.361489in}{1.721525in}}%
\pgfpathlineto{\pgfqpoint{1.362988in}{0.752846in}}%
\pgfpathlineto{\pgfqpoint{1.364302in}{1.751939in}}%
\pgfpathlineto{\pgfqpoint{1.366004in}{0.826096in}}%
\pgfpathlineto{\pgfqpoint{1.367753in}{1.792705in}}%
\pgfpathlineto{\pgfqpoint{1.368786in}{0.811845in}}%
\pgfpathlineto{\pgfqpoint{1.370343in}{1.662971in}}%
\pgfpathlineto{\pgfqpoint{1.371907in}{0.837775in}}%
\pgfpathlineto{\pgfqpoint{1.373584in}{1.764336in}}%
\pgfpathlineto{\pgfqpoint{1.375147in}{0.824491in}}%
\pgfpathlineto{\pgfqpoint{1.376784in}{1.736480in}}%
\pgfpathlineto{\pgfqpoint{1.378344in}{0.792556in}}%
\pgfpathlineto{\pgfqpoint{1.379988in}{1.740689in}}%
\pgfpathlineto{\pgfqpoint{1.381461in}{0.812366in}}%
\pgfpathlineto{\pgfqpoint{1.382770in}{1.781723in}}%
\pgfpathlineto{\pgfqpoint{1.384458in}{0.660663in}}%
\pgfpathlineto{\pgfqpoint{1.386192in}{1.693496in}}%
\pgfpathlineto{\pgfqpoint{1.387282in}{0.794443in}}%
\pgfpathlineto{\pgfqpoint{1.389205in}{1.693760in}}%
\pgfpathlineto{\pgfqpoint{1.390555in}{0.808630in}}%
\pgfpathlineto{\pgfqpoint{1.391969in}{1.732458in}}%
\pgfpathlineto{\pgfqpoint{1.393900in}{0.743944in}}%
\pgfpathlineto{\pgfqpoint{1.395366in}{1.693346in}}%
\pgfpathlineto{\pgfqpoint{1.396697in}{0.765612in}}%
\pgfpathlineto{\pgfqpoint{1.398210in}{1.616355in}}%
\pgfpathlineto{\pgfqpoint{1.399885in}{0.736770in}}%
\pgfpathlineto{\pgfqpoint{1.401389in}{1.684448in}}%
\pgfpathlineto{\pgfqpoint{1.402795in}{0.750338in}}%
\pgfpathlineto{\pgfqpoint{1.404231in}{1.558950in}}%
\pgfpathlineto{\pgfqpoint{1.405957in}{0.714079in}}%
\pgfpathlineto{\pgfqpoint{1.407618in}{1.751231in}}%
\pgfpathlineto{\pgfqpoint{1.408900in}{0.754167in}}%
\pgfpathlineto{\pgfqpoint{1.410717in}{1.766312in}}%
\pgfpathlineto{\pgfqpoint{1.412584in}{0.703697in}}%
\pgfpathlineto{\pgfqpoint{1.413547in}{1.638356in}}%
\pgfpathlineto{\pgfqpoint{1.415239in}{0.701813in}}%
\pgfpathlineto{\pgfqpoint{1.416805in}{1.763857in}}%
\pgfpathlineto{\pgfqpoint{1.418526in}{0.722942in}}%
\pgfpathlineto{\pgfqpoint{1.419750in}{1.612161in}}%
\pgfpathlineto{\pgfqpoint{1.421710in}{0.764046in}}%
\pgfpathlineto{\pgfqpoint{1.423135in}{1.690289in}}%
\pgfpathlineto{\pgfqpoint{1.424447in}{0.780934in}}%
\pgfpathlineto{\pgfqpoint{1.426976in}{1.725156in}}%
\pgfpathlineto{\pgfqpoint{1.427602in}{0.737281in}}%
\pgfpathlineto{\pgfqpoint{1.429248in}{1.665454in}}%
\pgfpathlineto{\pgfqpoint{1.430796in}{0.704254in}}%
\pgfpathlineto{\pgfqpoint{1.432090in}{1.689808in}}%
\pgfpathlineto{\pgfqpoint{1.433719in}{0.743279in}}%
\pgfpathlineto{\pgfqpoint{1.435710in}{1.658670in}}%
\pgfpathlineto{\pgfqpoint{1.436626in}{0.751729in}}%
\pgfpathlineto{\pgfqpoint{1.438510in}{1.643671in}}%
\pgfpathlineto{\pgfqpoint{1.439879in}{0.681186in}}%
\pgfpathlineto{\pgfqpoint{1.441259in}{1.620361in}}%
\pgfpathlineto{\pgfqpoint{1.443004in}{0.719619in}}%
\pgfpathlineto{\pgfqpoint{1.444614in}{1.643769in}}%
\pgfpathlineto{\pgfqpoint{1.445953in}{0.756158in}}%
\pgfpathlineto{\pgfqpoint{1.447449in}{1.648895in}}%
\pgfpathlineto{\pgfqpoint{1.449015in}{0.750550in}}%
\pgfpathlineto{\pgfqpoint{1.450569in}{1.750427in}}%
\pgfpathlineto{\pgfqpoint{1.452127in}{0.830482in}}%
\pgfpathlineto{\pgfqpoint{1.454163in}{1.684888in}}%
\pgfpathlineto{\pgfqpoint{1.455150in}{0.829568in}}%
\pgfpathlineto{\pgfqpoint{1.456943in}{1.735389in}}%
\pgfpathlineto{\pgfqpoint{1.458241in}{0.866832in}}%
\pgfpathlineto{\pgfqpoint{1.460017in}{1.660245in}}%
\pgfpathlineto{\pgfqpoint{1.461437in}{0.831529in}}%
\pgfpathlineto{\pgfqpoint{1.463108in}{1.750741in}}%
\pgfpathlineto{\pgfqpoint{1.464396in}{0.792538in}}%
\pgfpathlineto{\pgfqpoint{1.465961in}{1.695849in}}%
\pgfpathlineto{\pgfqpoint{1.467802in}{0.639496in}}%
\pgfpathlineto{\pgfqpoint{1.469436in}{1.726587in}}%
\pgfpathlineto{\pgfqpoint{1.470835in}{0.867074in}}%
\pgfpathlineto{\pgfqpoint{1.472214in}{1.770077in}}%
\pgfpathlineto{\pgfqpoint{1.473671in}{0.864934in}}%
\pgfpathlineto{\pgfqpoint{1.475303in}{1.703946in}}%
\pgfpathlineto{\pgfqpoint{1.477377in}{0.801252in}}%
\pgfpathlineto{\pgfqpoint{1.478390in}{1.813790in}}%
\pgfpathlineto{\pgfqpoint{1.479827in}{0.832763in}}%
\pgfpathlineto{\pgfqpoint{1.481370in}{1.644739in}}%
\pgfpathlineto{\pgfqpoint{1.482976in}{0.880277in}}%
\pgfpathlineto{\pgfqpoint{1.485267in}{1.761361in}}%
\pgfpathlineto{\pgfqpoint{1.486101in}{0.778362in}}%
\pgfpathlineto{\pgfqpoint{1.488085in}{1.865326in}}%
\pgfpathlineto{\pgfqpoint{1.489166in}{0.813756in}}%
\pgfpathlineto{\pgfqpoint{1.490732in}{1.741227in}}%
\pgfpathlineto{\pgfqpoint{1.492333in}{0.782458in}}%
\pgfpathlineto{\pgfqpoint{1.493765in}{1.743487in}}%
\pgfpathlineto{\pgfqpoint{1.495272in}{0.818174in}}%
\pgfpathlineto{\pgfqpoint{1.496820in}{1.646883in}}%
\pgfpathlineto{\pgfqpoint{1.498652in}{0.824852in}}%
\pgfpathlineto{\pgfqpoint{1.499908in}{1.780079in}}%
\pgfpathlineto{\pgfqpoint{1.502271in}{0.707728in}}%
\pgfpathlineto{\pgfqpoint{1.503163in}{1.797495in}}%
\pgfpathlineto{\pgfqpoint{1.504519in}{0.847365in}}%
\pgfpathlineto{\pgfqpoint{1.506067in}{1.737340in}}%
\pgfpathlineto{\pgfqpoint{1.508168in}{0.734374in}}%
\pgfpathlineto{\pgfqpoint{1.509135in}{1.852406in}}%
\pgfpathlineto{\pgfqpoint{1.511029in}{0.792745in}}%
\pgfpathlineto{\pgfqpoint{1.512232in}{1.722164in}}%
\pgfpathlineto{\pgfqpoint{1.514263in}{0.766822in}}%
\pgfpathlineto{\pgfqpoint{1.515389in}{1.764152in}}%
\pgfpathlineto{\pgfqpoint{1.516974in}{0.723092in}}%
\pgfpathlineto{\pgfqpoint{1.518487in}{1.670140in}}%
\pgfpathlineto{\pgfqpoint{1.520258in}{0.776336in}}%
\pgfpathlineto{\pgfqpoint{1.521787in}{1.843766in}}%
\pgfpathlineto{\pgfqpoint{1.523037in}{0.902028in}}%
\pgfpathlineto{\pgfqpoint{1.525144in}{1.795080in}}%
\pgfpathlineto{\pgfqpoint{1.526110in}{0.900369in}}%
\pgfpathlineto{\pgfqpoint{1.528232in}{1.829554in}}%
\pgfpathlineto{\pgfqpoint{1.529247in}{0.855852in}}%
\pgfpathlineto{\pgfqpoint{1.530746in}{1.731863in}}%
\pgfpathlineto{\pgfqpoint{1.532696in}{0.823600in}}%
\pgfpathlineto{\pgfqpoint{1.533952in}{1.714022in}}%
\pgfpathlineto{\pgfqpoint{1.535459in}{0.737557in}}%
\pgfpathlineto{\pgfqpoint{1.537050in}{1.706819in}}%
\pgfpathlineto{\pgfqpoint{1.538782in}{0.833764in}}%
\pgfpathlineto{\pgfqpoint{1.540312in}{1.796672in}}%
\pgfpathlineto{\pgfqpoint{1.542179in}{0.802465in}}%
\pgfpathlineto{\pgfqpoint{1.543105in}{1.776976in}}%
\pgfpathlineto{\pgfqpoint{1.544671in}{0.909276in}}%
\pgfpathlineto{\pgfqpoint{1.546242in}{1.787703in}}%
\pgfpathlineto{\pgfqpoint{1.548215in}{0.849271in}}%
\pgfpathlineto{\pgfqpoint{1.549349in}{1.761437in}}%
\pgfpathlineto{\pgfqpoint{1.550794in}{0.869348in}}%
\pgfpathlineto{\pgfqpoint{1.552396in}{1.769770in}}%
\pgfpathlineto{\pgfqpoint{1.553885in}{0.856651in}}%
\pgfpathlineto{\pgfqpoint{1.555509in}{1.742653in}}%
\pgfpathlineto{\pgfqpoint{1.557345in}{0.809485in}}%
\pgfpathlineto{\pgfqpoint{1.559469in}{1.845728in}}%
\pgfpathlineto{\pgfqpoint{1.560325in}{0.813837in}}%
\pgfpathlineto{\pgfqpoint{1.561707in}{1.751278in}}%
\pgfpathlineto{\pgfqpoint{1.563261in}{0.838126in}}%
\pgfpathlineto{\pgfqpoint{1.564800in}{1.721863in}}%
\pgfpathlineto{\pgfqpoint{1.566736in}{0.779188in}}%
\pgfpathlineto{\pgfqpoint{1.567784in}{1.732339in}}%
\pgfpathlineto{\pgfqpoint{1.569380in}{0.845449in}}%
\pgfpathlineto{\pgfqpoint{1.570960in}{1.741745in}}%
\pgfpathlineto{\pgfqpoint{1.572662in}{0.794371in}}%
\pgfpathlineto{\pgfqpoint{1.573963in}{1.725378in}}%
\pgfpathlineto{\pgfqpoint{1.575491in}{0.883657in}}%
\pgfpathlineto{\pgfqpoint{1.577268in}{1.720342in}}%
\pgfpathlineto{\pgfqpoint{1.578658in}{0.770239in}}%
\pgfpathlineto{\pgfqpoint{1.580477in}{1.726341in}}%
\pgfpathlineto{\pgfqpoint{1.581710in}{0.809558in}}%
\pgfpathlineto{\pgfqpoint{1.583509in}{1.740724in}}%
\pgfpathlineto{\pgfqpoint{1.584867in}{0.798377in}}%
\pgfpathlineto{\pgfqpoint{1.586389in}{1.742231in}}%
\pgfpathlineto{\pgfqpoint{1.587892in}{0.702516in}}%
\pgfpathlineto{\pgfqpoint{1.589405in}{1.690159in}}%
\pgfpathlineto{\pgfqpoint{1.590962in}{0.790866in}}%
\pgfpathlineto{\pgfqpoint{1.592688in}{1.731083in}}%
\pgfpathlineto{\pgfqpoint{1.594735in}{0.719649in}}%
\pgfpathlineto{\pgfqpoint{1.595870in}{1.751595in}}%
\pgfpathlineto{\pgfqpoint{1.597398in}{0.817812in}}%
\pgfpathlineto{\pgfqpoint{1.598635in}{1.672541in}}%
\pgfpathlineto{\pgfqpoint{1.600383in}{0.739973in}}%
\pgfpathlineto{\pgfqpoint{1.601709in}{1.685057in}}%
\pgfpathlineto{\pgfqpoint{1.603398in}{0.813787in}}%
\pgfpathlineto{\pgfqpoint{1.605135in}{1.731812in}}%
\pgfpathlineto{\pgfqpoint{1.606368in}{0.814759in}}%
\pgfpathlineto{\pgfqpoint{1.608155in}{1.755737in}}%
\pgfpathlineto{\pgfqpoint{1.609889in}{0.709380in}}%
\pgfpathlineto{\pgfqpoint{1.611916in}{1.824182in}}%
\pgfpathlineto{\pgfqpoint{1.612566in}{0.908843in}}%
\pgfpathlineto{\pgfqpoint{1.614355in}{1.751212in}}%
\pgfpathlineto{\pgfqpoint{1.615843in}{0.828023in}}%
\pgfpathlineto{\pgfqpoint{1.617217in}{1.749554in}}%
\pgfpathlineto{\pgfqpoint{1.619324in}{0.708429in}}%
\pgfpathlineto{\pgfqpoint{1.620238in}{1.699889in}}%
\pgfpathlineto{\pgfqpoint{1.621953in}{0.818242in}}%
\pgfpathlineto{\pgfqpoint{1.623491in}{1.738902in}}%
\pgfpathlineto{\pgfqpoint{1.625181in}{0.783883in}}%
\pgfpathlineto{\pgfqpoint{1.626538in}{1.786066in}}%
\pgfpathlineto{\pgfqpoint{1.627978in}{0.889637in}}%
\pgfpathlineto{\pgfqpoint{1.630225in}{1.857044in}}%
\pgfpathlineto{\pgfqpoint{1.631281in}{0.867115in}}%
\pgfpathlineto{\pgfqpoint{1.632590in}{1.739946in}}%
\pgfpathlineto{\pgfqpoint{1.634416in}{0.838631in}}%
\pgfpathlineto{\pgfqpoint{1.636715in}{1.836055in}}%
\pgfpathlineto{\pgfqpoint{1.637681in}{0.789427in}}%
\pgfpathlineto{\pgfqpoint{1.639157in}{1.677359in}}%
\pgfpathlineto{\pgfqpoint{1.640341in}{0.775065in}}%
\pgfpathlineto{\pgfqpoint{1.642089in}{1.655231in}}%
\pgfpathlineto{\pgfqpoint{1.643386in}{0.760549in}}%
\pgfpathlineto{\pgfqpoint{1.645030in}{1.653462in}}%
\pgfpathlineto{\pgfqpoint{1.646987in}{0.709665in}}%
\pgfpathlineto{\pgfqpoint{1.648228in}{1.774573in}}%
\pgfpathlineto{\pgfqpoint{1.650094in}{0.676849in}}%
\pgfpathlineto{\pgfqpoint{1.651242in}{1.630567in}}%
\pgfpathlineto{\pgfqpoint{1.652634in}{0.808889in}}%
\pgfpathlineto{\pgfqpoint{1.654262in}{1.715196in}}%
\pgfpathlineto{\pgfqpoint{1.655774in}{0.804606in}}%
\pgfpathlineto{\pgfqpoint{1.657753in}{1.730545in}}%
\pgfpathlineto{\pgfqpoint{1.659021in}{0.736352in}}%
\pgfpathlineto{\pgfqpoint{1.660549in}{1.688307in}}%
\pgfpathlineto{\pgfqpoint{1.662128in}{0.720061in}}%
\pgfpathlineto{\pgfqpoint{1.663456in}{1.803272in}}%
\pgfpathlineto{\pgfqpoint{1.665009in}{0.806175in}}%
\pgfpathlineto{\pgfqpoint{1.666513in}{1.718269in}}%
\pgfpathlineto{\pgfqpoint{1.668154in}{0.843381in}}%
\pgfpathlineto{\pgfqpoint{1.669695in}{1.725325in}}%
\pgfpathlineto{\pgfqpoint{1.671190in}{0.766444in}}%
\pgfpathlineto{\pgfqpoint{1.672738in}{1.647837in}}%
\pgfpathlineto{\pgfqpoint{1.675006in}{0.701413in}}%
\pgfpathlineto{\pgfqpoint{1.675946in}{1.720970in}}%
\pgfpathlineto{\pgfqpoint{1.677407in}{0.661599in}}%
\pgfpathlineto{\pgfqpoint{1.678888in}{1.650850in}}%
\pgfpathlineto{\pgfqpoint{1.680440in}{0.818796in}}%
\pgfpathlineto{\pgfqpoint{1.682991in}{1.778624in}}%
\pgfpathlineto{\pgfqpoint{1.683682in}{0.758296in}}%
\pgfpathlineto{\pgfqpoint{1.685298in}{1.687302in}}%
\pgfpathlineto{\pgfqpoint{1.686727in}{0.832855in}}%
\pgfpathlineto{\pgfqpoint{1.688583in}{1.661031in}}%
\pgfpathlineto{\pgfqpoint{1.690108in}{0.802042in}}%
\pgfpathlineto{\pgfqpoint{1.691351in}{1.762802in}}%
\pgfpathlineto{\pgfqpoint{1.692988in}{0.815036in}}%
\pgfpathlineto{\pgfqpoint{1.694285in}{1.650314in}}%
\pgfpathlineto{\pgfqpoint{1.695968in}{0.850662in}}%
\pgfpathlineto{\pgfqpoint{1.697432in}{1.721438in}}%
\pgfpathlineto{\pgfqpoint{1.699170in}{0.852130in}}%
\pgfpathlineto{\pgfqpoint{1.700646in}{1.743298in}}%
\pgfpathlineto{\pgfqpoint{1.702039in}{0.835444in}}%
\pgfpathlineto{\pgfqpoint{1.703543in}{1.744870in}}%
\pgfpathlineto{\pgfqpoint{1.705887in}{0.761176in}}%
\pgfpathlineto{\pgfqpoint{1.707442in}{1.903936in}}%
\pgfpathlineto{\pgfqpoint{1.708207in}{0.870809in}}%
\pgfpathlineto{\pgfqpoint{1.709829in}{1.851877in}}%
\pgfpathlineto{\pgfqpoint{1.711483in}{0.846569in}}%
\pgfpathlineto{\pgfqpoint{1.713048in}{1.806870in}}%
\pgfpathlineto{\pgfqpoint{1.714495in}{0.893452in}}%
\pgfpathlineto{\pgfqpoint{1.715966in}{1.811081in}}%
\pgfpathlineto{\pgfqpoint{1.717590in}{0.880217in}}%
\pgfpathlineto{\pgfqpoint{1.719674in}{1.901461in}}%
\pgfpathlineto{\pgfqpoint{1.720992in}{0.854673in}}%
\pgfpathlineto{\pgfqpoint{1.722308in}{1.833943in}}%
\pgfpathlineto{\pgfqpoint{1.724017in}{0.905832in}}%
\pgfpathlineto{\pgfqpoint{1.725149in}{1.782433in}}%
\pgfpathlineto{\pgfqpoint{1.727034in}{0.874421in}}%
\pgfpathlineto{\pgfqpoint{1.728237in}{1.778460in}}%
\pgfpathlineto{\pgfqpoint{1.729841in}{0.895682in}}%
\pgfpathlineto{\pgfqpoint{1.731547in}{1.739563in}}%
\pgfpathlineto{\pgfqpoint{1.732962in}{0.865978in}}%
\pgfpathlineto{\pgfqpoint{1.734525in}{1.718486in}}%
\pgfpathlineto{\pgfqpoint{1.736095in}{0.874812in}}%
\pgfpathlineto{\pgfqpoint{1.738364in}{1.817273in}}%
\pgfpathlineto{\pgfqpoint{1.739263in}{0.898183in}}%
\pgfpathlineto{\pgfqpoint{1.740596in}{1.830602in}}%
\pgfpathlineto{\pgfqpoint{1.742199in}{0.915889in}}%
\pgfpathlineto{\pgfqpoint{1.743650in}{1.709617in}}%
\pgfpathlineto{\pgfqpoint{1.745211in}{0.886689in}}%
\pgfpathlineto{\pgfqpoint{1.747606in}{1.841286in}}%
\pgfpathlineto{\pgfqpoint{1.748502in}{0.863265in}}%
\pgfpathlineto{\pgfqpoint{1.750194in}{1.770925in}}%
\pgfpathlineto{\pgfqpoint{1.751628in}{0.841434in}}%
\pgfpathlineto{\pgfqpoint{1.752976in}{1.831985in}}%
\pgfpathlineto{\pgfqpoint{1.754459in}{0.928331in}}%
\pgfpathlineto{\pgfqpoint{1.756074in}{1.768940in}}%
\pgfpathlineto{\pgfqpoint{1.757970in}{0.813173in}}%
\pgfpathlineto{\pgfqpoint{1.759314in}{1.756386in}}%
\pgfpathlineto{\pgfqpoint{1.760689in}{0.874931in}}%
\pgfpathlineto{\pgfqpoint{1.762441in}{1.786390in}}%
\pgfpathlineto{\pgfqpoint{1.763812in}{0.909052in}}%
\pgfpathlineto{\pgfqpoint{1.765520in}{1.755746in}}%
\pgfpathlineto{\pgfqpoint{1.767170in}{0.803855in}}%
\pgfpathlineto{\pgfqpoint{1.768342in}{1.767895in}}%
\pgfpathlineto{\pgfqpoint{1.770038in}{0.908653in}}%
\pgfpathlineto{\pgfqpoint{1.772027in}{1.907201in}}%
\pgfpathlineto{\pgfqpoint{1.773130in}{0.932771in}}%
\pgfpathlineto{\pgfqpoint{1.774867in}{1.882583in}}%
\pgfpathlineto{\pgfqpoint{1.776842in}{0.785138in}}%
\pgfpathlineto{\pgfqpoint{1.777625in}{1.791345in}}%
\pgfpathlineto{\pgfqpoint{1.779159in}{0.908573in}}%
\pgfpathlineto{\pgfqpoint{1.780707in}{1.770531in}}%
\pgfpathlineto{\pgfqpoint{1.782618in}{0.899349in}}%
\pgfpathlineto{\pgfqpoint{1.784036in}{1.831181in}}%
\pgfpathlineto{\pgfqpoint{1.785766in}{0.840639in}}%
\pgfpathlineto{\pgfqpoint{1.786918in}{1.893170in}}%
\pgfpathlineto{\pgfqpoint{1.788400in}{0.911691in}}%
\pgfpathlineto{\pgfqpoint{1.790165in}{1.772172in}}%
\pgfpathlineto{\pgfqpoint{1.791696in}{0.907355in}}%
\pgfpathlineto{\pgfqpoint{1.793063in}{1.841855in}}%
\pgfpathlineto{\pgfqpoint{1.794571in}{0.853612in}}%
\pgfpathlineto{\pgfqpoint{1.796134in}{1.798651in}}%
\pgfpathlineto{\pgfqpoint{1.798215in}{0.889114in}}%
\pgfpathlineto{\pgfqpoint{1.799256in}{1.863362in}}%
\pgfpathlineto{\pgfqpoint{1.801289in}{0.860499in}}%
\pgfpathlineto{\pgfqpoint{1.802472in}{1.821609in}}%
\pgfpathlineto{\pgfqpoint{1.804387in}{0.859906in}}%
\pgfpathlineto{\pgfqpoint{1.805744in}{1.799209in}}%
\pgfpathlineto{\pgfqpoint{1.806949in}{0.869982in}}%
\pgfpathlineto{\pgfqpoint{1.808625in}{1.821016in}}%
\pgfpathlineto{\pgfqpoint{1.810343in}{0.956951in}}%
\pgfpathlineto{\pgfqpoint{1.811648in}{1.950137in}}%
\pgfpathlineto{\pgfqpoint{1.813356in}{0.908650in}}%
\pgfpathlineto{\pgfqpoint{1.814690in}{1.829252in}}%
\pgfpathlineto{\pgfqpoint{1.816413in}{0.929614in}}%
\pgfpathlineto{\pgfqpoint{1.817723in}{1.888346in}}%
\pgfpathlineto{\pgfqpoint{1.819368in}{0.909240in}}%
\pgfpathlineto{\pgfqpoint{1.820819in}{1.862333in}}%
\pgfpathlineto{\pgfqpoint{1.822972in}{0.767257in}}%
\pgfpathlineto{\pgfqpoint{1.823921in}{1.774524in}}%
\pgfpathlineto{\pgfqpoint{1.825599in}{0.907133in}}%
\pgfpathlineto{\pgfqpoint{1.827584in}{1.882190in}}%
\pgfpathlineto{\pgfqpoint{1.828523in}{0.951154in}}%
\pgfpathlineto{\pgfqpoint{1.830489in}{1.915042in}}%
\pgfpathlineto{\pgfqpoint{1.831687in}{0.895785in}}%
\pgfpathlineto{\pgfqpoint{1.833321in}{1.970494in}}%
\pgfpathlineto{\pgfqpoint{1.834922in}{0.923232in}}%
\pgfpathlineto{\pgfqpoint{1.836249in}{1.852508in}}%
\pgfpathlineto{\pgfqpoint{1.837952in}{0.849702in}}%
\pgfpathlineto{\pgfqpoint{1.839603in}{1.789281in}}%
\pgfpathlineto{\pgfqpoint{1.841542in}{0.862878in}}%
\pgfpathlineto{\pgfqpoint{1.842509in}{1.757851in}}%
\pgfpathlineto{\pgfqpoint{1.844647in}{0.879789in}}%
\pgfpathlineto{\pgfqpoint{1.845600in}{1.749746in}}%
\pgfpathlineto{\pgfqpoint{1.847455in}{0.840129in}}%
\pgfpathlineto{\pgfqpoint{1.848773in}{1.792922in}}%
\pgfpathlineto{\pgfqpoint{1.850107in}{0.884994in}}%
\pgfpathlineto{\pgfqpoint{1.851865in}{1.801328in}}%
\pgfpathlineto{\pgfqpoint{1.853638in}{0.852325in}}%
\pgfpathlineto{\pgfqpoint{1.854848in}{1.840203in}}%
\pgfpathlineto{\pgfqpoint{1.856744in}{0.830428in}}%
\pgfpathlineto{\pgfqpoint{1.857859in}{1.790736in}}%
\pgfpathlineto{\pgfqpoint{1.859430in}{0.818701in}}%
\pgfpathlineto{\pgfqpoint{1.861179in}{1.740442in}}%
\pgfpathlineto{\pgfqpoint{1.862450in}{0.798324in}}%
\pgfpathlineto{\pgfqpoint{1.864052in}{1.729259in}}%
\pgfpathlineto{\pgfqpoint{1.865610in}{0.878711in}}%
\pgfpathlineto{\pgfqpoint{1.867487in}{1.772277in}}%
\pgfpathlineto{\pgfqpoint{1.868739in}{0.780944in}}%
\pgfpathlineto{\pgfqpoint{1.870317in}{1.767464in}}%
\pgfpathlineto{\pgfqpoint{1.871804in}{0.878496in}}%
\pgfpathlineto{\pgfqpoint{1.873479in}{1.732243in}}%
\pgfpathlineto{\pgfqpoint{1.875327in}{0.771658in}}%
\pgfpathlineto{\pgfqpoint{1.876682in}{1.748879in}}%
\pgfpathlineto{\pgfqpoint{1.877924in}{0.824766in}}%
\pgfpathlineto{\pgfqpoint{1.879547in}{1.747014in}}%
\pgfpathlineto{\pgfqpoint{1.881154in}{0.789706in}}%
\pgfpathlineto{\pgfqpoint{1.882592in}{1.840432in}}%
\pgfpathlineto{\pgfqpoint{1.884687in}{0.809687in}}%
\pgfpathlineto{\pgfqpoint{1.885611in}{1.740208in}}%
\pgfpathlineto{\pgfqpoint{1.887226in}{0.906881in}}%
\pgfpathlineto{\pgfqpoint{1.889065in}{1.790326in}}%
\pgfpathlineto{\pgfqpoint{1.890247in}{0.948736in}}%
\pgfpathlineto{\pgfqpoint{1.891857in}{1.820091in}}%
\pgfpathlineto{\pgfqpoint{1.893629in}{0.898019in}}%
\pgfpathlineto{\pgfqpoint{1.895189in}{1.895332in}}%
\pgfpathlineto{\pgfqpoint{1.896519in}{0.917484in}}%
\pgfpathlineto{\pgfqpoint{1.898463in}{1.865603in}}%
\pgfpathlineto{\pgfqpoint{1.899480in}{0.973861in}}%
\pgfpathlineto{\pgfqpoint{1.901321in}{1.863562in}}%
\pgfpathlineto{\pgfqpoint{1.902694in}{0.950452in}}%
\pgfpathlineto{\pgfqpoint{1.904148in}{1.818345in}}%
\pgfpathlineto{\pgfqpoint{1.906735in}{0.834336in}}%
\pgfpathlineto{\pgfqpoint{1.907842in}{1.900131in}}%
\pgfpathlineto{\pgfqpoint{1.908774in}{0.908504in}}%
\pgfpathlineto{\pgfqpoint{1.910464in}{1.827262in}}%
\pgfpathlineto{\pgfqpoint{1.912027in}{0.852045in}}%
\pgfpathlineto{\pgfqpoint{1.913615in}{1.782357in}}%
\pgfpathlineto{\pgfqpoint{1.915012in}{0.769796in}}%
\pgfpathlineto{\pgfqpoint{1.916587in}{1.793442in}}%
\pgfpathlineto{\pgfqpoint{1.918160in}{0.910376in}}%
\pgfpathlineto{\pgfqpoint{1.919770in}{1.888013in}}%
\pgfpathlineto{\pgfqpoint{1.921821in}{0.836673in}}%
\pgfpathlineto{\pgfqpoint{1.922777in}{1.839165in}}%
\pgfpathlineto{\pgfqpoint{1.924315in}{0.856038in}}%
\pgfpathlineto{\pgfqpoint{1.925813in}{1.741070in}}%
\pgfpathlineto{\pgfqpoint{1.927956in}{0.793198in}}%
\pgfpathlineto{\pgfqpoint{1.928832in}{1.786302in}}%
\pgfpathlineto{\pgfqpoint{1.930899in}{0.805938in}}%
\pgfpathlineto{\pgfqpoint{1.932017in}{1.725334in}}%
\pgfpathlineto{\pgfqpoint{1.933435in}{0.901312in}}%
\pgfpathlineto{\pgfqpoint{1.934990in}{1.785768in}}%
\pgfpathlineto{\pgfqpoint{1.936648in}{0.827610in}}%
\pgfpathlineto{\pgfqpoint{1.938852in}{1.826944in}}%
\pgfpathlineto{\pgfqpoint{1.939838in}{0.745122in}}%
\pgfpathlineto{\pgfqpoint{1.941135in}{1.769608in}}%
\pgfpathlineto{\pgfqpoint{1.942970in}{0.832042in}}%
\pgfpathlineto{\pgfqpoint{1.944516in}{1.723248in}}%
\pgfpathlineto{\pgfqpoint{1.945939in}{0.848839in}}%
\pgfpathlineto{\pgfqpoint{1.947382in}{1.730637in}}%
\pgfpathlineto{\pgfqpoint{1.949144in}{0.843168in}}%
\pgfpathlineto{\pgfqpoint{1.950485in}{1.763735in}}%
\pgfpathlineto{\pgfqpoint{1.952301in}{0.848246in}}%
\pgfpathlineto{\pgfqpoint{1.953949in}{1.788760in}}%
\pgfpathlineto{\pgfqpoint{1.955442in}{0.751315in}}%
\pgfpathlineto{\pgfqpoint{1.956594in}{1.822970in}}%
\pgfpathlineto{\pgfqpoint{1.958295in}{0.826268in}}%
\pgfpathlineto{\pgfqpoint{1.959760in}{1.728872in}}%
\pgfpathlineto{\pgfqpoint{1.961190in}{0.859740in}}%
\pgfpathlineto{\pgfqpoint{1.963137in}{1.792741in}}%
\pgfpathlineto{\pgfqpoint{1.964766in}{0.843285in}}%
\pgfpathlineto{\pgfqpoint{1.965854in}{1.743170in}}%
\pgfpathlineto{\pgfqpoint{1.967489in}{0.863480in}}%
\pgfpathlineto{\pgfqpoint{1.969125in}{1.747306in}}%
\pgfpathlineto{\pgfqpoint{1.970471in}{0.864120in}}%
\pgfpathlineto{\pgfqpoint{1.972091in}{1.778502in}}%
\pgfpathlineto{\pgfqpoint{1.973573in}{0.888373in}}%
\pgfpathlineto{\pgfqpoint{1.975093in}{1.766866in}}%
\pgfpathlineto{\pgfqpoint{1.976672in}{0.865997in}}%
\pgfpathlineto{\pgfqpoint{1.978432in}{1.753285in}}%
\pgfpathlineto{\pgfqpoint{1.979720in}{0.847900in}}%
\pgfpathlineto{\pgfqpoint{1.981486in}{1.734511in}}%
\pgfpathlineto{\pgfqpoint{1.982790in}{0.836657in}}%
\pgfpathlineto{\pgfqpoint{1.984457in}{1.807071in}}%
\pgfpathlineto{\pgfqpoint{1.986209in}{0.737156in}}%
\pgfpathlineto{\pgfqpoint{1.987728in}{1.718017in}}%
\pgfpathlineto{\pgfqpoint{1.989214in}{0.800412in}}%
\pgfpathlineto{\pgfqpoint{1.991283in}{1.806679in}}%
\pgfpathlineto{\pgfqpoint{1.992173in}{0.854242in}}%
\pgfpathlineto{\pgfqpoint{1.993824in}{1.739779in}}%
\pgfpathlineto{\pgfqpoint{1.995170in}{0.903690in}}%
\pgfpathlineto{\pgfqpoint{1.997229in}{1.836723in}}%
\pgfpathlineto{\pgfqpoint{1.998336in}{0.770199in}}%
\pgfpathlineto{\pgfqpoint{1.999798in}{1.742757in}}%
\pgfpathlineto{\pgfqpoint{2.001450in}{0.833793in}}%
\pgfpathlineto{\pgfqpoint{2.003269in}{1.844674in}}%
\pgfpathlineto{\pgfqpoint{2.004532in}{0.816624in}}%
\pgfpathlineto{\pgfqpoint{2.006502in}{1.787615in}}%
\pgfpathlineto{\pgfqpoint{2.007650in}{0.853986in}}%
\pgfpathlineto{\pgfqpoint{2.009344in}{1.785748in}}%
\pgfpathlineto{\pgfqpoint{2.010564in}{0.925695in}}%
\pgfpathlineto{\pgfqpoint{2.012799in}{1.819617in}}%
\pgfpathlineto{\pgfqpoint{2.013699in}{0.939739in}}%
\pgfpathlineto{\pgfqpoint{2.015548in}{1.794669in}}%
\pgfpathlineto{\pgfqpoint{2.017178in}{0.789993in}}%
\pgfpathlineto{\pgfqpoint{2.018355in}{1.715527in}}%
\pgfpathlineto{\pgfqpoint{2.020299in}{0.853600in}}%
\pgfpathlineto{\pgfqpoint{2.021719in}{1.833507in}}%
\pgfpathlineto{\pgfqpoint{2.022992in}{0.921187in}}%
\pgfpathlineto{\pgfqpoint{2.024562in}{1.782707in}}%
\pgfpathlineto{\pgfqpoint{2.026329in}{0.814122in}}%
\pgfpathlineto{\pgfqpoint{2.027885in}{1.735089in}}%
\pgfpathlineto{\pgfqpoint{2.029724in}{0.800616in}}%
\pgfpathlineto{\pgfqpoint{2.030668in}{1.801084in}}%
\pgfpathlineto{\pgfqpoint{2.032645in}{0.811369in}}%
\pgfpathlineto{\pgfqpoint{2.034142in}{1.763406in}}%
\pgfpathlineto{\pgfqpoint{2.035377in}{0.877556in}}%
\pgfpathlineto{\pgfqpoint{2.036841in}{1.726391in}}%
\pgfpathlineto{\pgfqpoint{2.038751in}{0.764815in}}%
\pgfpathlineto{\pgfqpoint{2.040020in}{1.797602in}}%
\pgfpathlineto{\pgfqpoint{2.041956in}{0.812025in}}%
\pgfpathlineto{\pgfqpoint{2.043055in}{1.727033in}}%
\pgfpathlineto{\pgfqpoint{2.044584in}{0.826550in}}%
\pgfpathlineto{\pgfqpoint{2.046297in}{1.849434in}}%
\pgfpathlineto{\pgfqpoint{2.047817in}{0.794803in}}%
\pgfpathlineto{\pgfqpoint{2.049522in}{1.765474in}}%
\pgfpathlineto{\pgfqpoint{2.050742in}{0.868763in}}%
\pgfpathlineto{\pgfqpoint{2.052237in}{1.778065in}}%
\pgfpathlineto{\pgfqpoint{2.054004in}{0.830779in}}%
\pgfpathlineto{\pgfqpoint{2.055526in}{1.809062in}}%
\pgfpathlineto{\pgfqpoint{2.056851in}{0.893857in}}%
\pgfpathlineto{\pgfqpoint{2.058729in}{1.790738in}}%
\pgfpathlineto{\pgfqpoint{2.060222in}{0.853604in}}%
\pgfpathlineto{\pgfqpoint{2.061664in}{1.814151in}}%
\pgfpathlineto{\pgfqpoint{2.063227in}{0.935340in}}%
\pgfpathlineto{\pgfqpoint{2.064576in}{1.762150in}}%
\pgfpathlineto{\pgfqpoint{2.066426in}{0.874226in}}%
\pgfpathlineto{\pgfqpoint{2.067695in}{1.779744in}}%
\pgfpathlineto{\pgfqpoint{2.069253in}{0.915177in}}%
\pgfpathlineto{\pgfqpoint{2.070898in}{1.739920in}}%
\pgfpathlineto{\pgfqpoint{2.072324in}{0.784000in}}%
\pgfpathlineto{\pgfqpoint{2.074101in}{1.739077in}}%
\pgfpathlineto{\pgfqpoint{2.075618in}{0.831238in}}%
\pgfpathlineto{\pgfqpoint{2.077054in}{1.807966in}}%
\pgfpathlineto{\pgfqpoint{2.079610in}{0.748528in}}%
\pgfpathlineto{\pgfqpoint{2.080122in}{1.785186in}}%
\pgfpathlineto{\pgfqpoint{2.082000in}{0.758215in}}%
\pgfpathlineto{\pgfqpoint{2.083117in}{1.740751in}}%
\pgfpathlineto{\pgfqpoint{2.084669in}{0.805806in}}%
\pgfpathlineto{\pgfqpoint{2.086705in}{1.818771in}}%
\pgfpathlineto{\pgfqpoint{2.088156in}{0.806478in}}%
\pgfpathlineto{\pgfqpoint{2.090116in}{1.887424in}}%
\pgfpathlineto{\pgfqpoint{2.091051in}{0.813766in}}%
\pgfpathlineto{\pgfqpoint{2.092651in}{1.816667in}}%
\pgfpathlineto{\pgfqpoint{2.094088in}{0.773857in}}%
\pgfpathlineto{\pgfqpoint{2.096050in}{1.783247in}}%
\pgfpathlineto{\pgfqpoint{2.096978in}{0.867049in}}%
\pgfpathlineto{\pgfqpoint{2.098571in}{1.679074in}}%
\pgfpathlineto{\pgfqpoint{2.100351in}{0.710927in}}%
\pgfpathlineto{\pgfqpoint{2.102319in}{1.829791in}}%
\pgfpathlineto{\pgfqpoint{2.103163in}{0.844210in}}%
\pgfpathlineto{\pgfqpoint{2.105024in}{1.735026in}}%
\pgfpathlineto{\pgfqpoint{2.106225in}{0.803562in}}%
\pgfpathlineto{\pgfqpoint{2.107903in}{1.818583in}}%
\pgfpathlineto{\pgfqpoint{2.109594in}{0.776691in}}%
\pgfpathlineto{\pgfqpoint{2.110866in}{1.706439in}}%
\pgfpathlineto{\pgfqpoint{2.112656in}{0.807266in}}%
\pgfpathlineto{\pgfqpoint{2.114189in}{1.798732in}}%
\pgfpathlineto{\pgfqpoint{2.115519in}{0.838264in}}%
\pgfpathlineto{\pgfqpoint{2.117204in}{1.746380in}}%
\pgfpathlineto{\pgfqpoint{2.118620in}{0.774537in}}%
\pgfpathlineto{\pgfqpoint{2.120170in}{1.658074in}}%
\pgfpathlineto{\pgfqpoint{2.121713in}{0.758828in}}%
\pgfpathlineto{\pgfqpoint{2.123449in}{1.726067in}}%
\pgfpathlineto{\pgfqpoint{2.125227in}{0.782104in}}%
\pgfpathlineto{\pgfqpoint{2.127294in}{1.744536in}}%
\pgfpathlineto{\pgfqpoint{2.127887in}{0.755202in}}%
\pgfpathlineto{\pgfqpoint{2.129683in}{1.661630in}}%
\pgfpathlineto{\pgfqpoint{2.130981in}{0.730193in}}%
\pgfpathlineto{\pgfqpoint{2.132619in}{1.685561in}}%
\pgfpathlineto{\pgfqpoint{2.134320in}{0.652826in}}%
\pgfpathlineto{\pgfqpoint{2.136158in}{1.742416in}}%
\pgfpathlineto{\pgfqpoint{2.137302in}{0.778640in}}%
\pgfpathlineto{\pgfqpoint{2.138991in}{1.656785in}}%
\pgfpathlineto{\pgfqpoint{2.140249in}{0.715727in}}%
\pgfpathlineto{\pgfqpoint{2.141768in}{1.780556in}}%
\pgfpathlineto{\pgfqpoint{2.143632in}{0.746595in}}%
\pgfpathlineto{\pgfqpoint{2.144804in}{1.655903in}}%
\pgfpathlineto{\pgfqpoint{2.146338in}{0.835918in}}%
\pgfpathlineto{\pgfqpoint{2.148082in}{1.720140in}}%
\pgfpathlineto{\pgfqpoint{2.149525in}{0.780517in}}%
\pgfpathlineto{\pgfqpoint{2.151165in}{1.700240in}}%
\pgfpathlineto{\pgfqpoint{2.152638in}{0.800149in}}%
\pgfpathlineto{\pgfqpoint{2.154118in}{1.703754in}}%
\pgfpathlineto{\pgfqpoint{2.156045in}{0.690233in}}%
\pgfpathlineto{\pgfqpoint{2.157298in}{1.708238in}}%
\pgfpathlineto{\pgfqpoint{2.158679in}{0.816103in}}%
\pgfpathlineto{\pgfqpoint{2.160389in}{1.682991in}}%
\pgfpathlineto{\pgfqpoint{2.162044in}{0.780598in}}%
\pgfpathlineto{\pgfqpoint{2.163691in}{1.678756in}}%
\pgfpathlineto{\pgfqpoint{2.165035in}{0.783520in}}%
\pgfpathlineto{\pgfqpoint{2.166572in}{1.766267in}}%
\pgfpathlineto{\pgfqpoint{2.168197in}{0.766404in}}%
\pgfpathlineto{\pgfqpoint{2.169578in}{1.872053in}}%
\pgfpathlineto{\pgfqpoint{2.171211in}{0.698223in}}%
\pgfpathlineto{\pgfqpoint{2.172660in}{1.625831in}}%
\pgfpathlineto{\pgfqpoint{2.174173in}{0.791497in}}%
\pgfpathlineto{\pgfqpoint{2.176331in}{1.683589in}}%
\pgfpathlineto{\pgfqpoint{2.177266in}{0.787125in}}%
\pgfpathlineto{\pgfqpoint{2.179813in}{1.746471in}}%
\pgfpathlineto{\pgfqpoint{2.180899in}{0.689964in}}%
\pgfpathlineto{\pgfqpoint{2.181863in}{1.706070in}}%
\pgfpathlineto{\pgfqpoint{2.183365in}{0.695625in}}%
\pgfpathlineto{\pgfqpoint{2.185162in}{1.695415in}}%
\pgfpathlineto{\pgfqpoint{2.186794in}{0.852414in}}%
\pgfpathlineto{\pgfqpoint{2.188001in}{1.714552in}}%
\pgfpathlineto{\pgfqpoint{2.189789in}{0.796714in}}%
\pgfpathlineto{\pgfqpoint{2.191424in}{1.683124in}}%
\pgfpathlineto{\pgfqpoint{2.192758in}{0.792675in}}%
\pgfpathlineto{\pgfqpoint{2.194623in}{1.756425in}}%
\pgfpathlineto{\pgfqpoint{2.195729in}{0.864824in}}%
\pgfpathlineto{\pgfqpoint{2.197453in}{1.697451in}}%
\pgfpathlineto{\pgfqpoint{2.198862in}{0.750466in}}%
\pgfpathlineto{\pgfqpoint{2.200410in}{1.705124in}}%
\pgfpathlineto{\pgfqpoint{2.202693in}{0.654039in}}%
\pgfpathlineto{\pgfqpoint{2.203598in}{1.738882in}}%
\pgfpathlineto{\pgfqpoint{2.205483in}{0.758575in}}%
\pgfpathlineto{\pgfqpoint{2.207021in}{1.828063in}}%
\pgfpathlineto{\pgfqpoint{2.208313in}{0.863369in}}%
\pgfpathlineto{\pgfqpoint{2.210339in}{1.798779in}}%
\pgfpathlineto{\pgfqpoint{2.211163in}{0.848429in}}%
\pgfpathlineto{\pgfqpoint{2.212889in}{1.694433in}}%
\pgfpathlineto{\pgfqpoint{2.214227in}{0.866802in}}%
\pgfpathlineto{\pgfqpoint{2.216116in}{1.727732in}}%
\pgfpathlineto{\pgfqpoint{2.217342in}{0.795909in}}%
\pgfpathlineto{\pgfqpoint{2.219164in}{1.670065in}}%
\pgfpathlineto{\pgfqpoint{2.220677in}{0.721965in}}%
\pgfpathlineto{\pgfqpoint{2.222884in}{1.789543in}}%
\pgfpathlineto{\pgfqpoint{2.224347in}{0.669933in}}%
\pgfpathlineto{\pgfqpoint{2.225372in}{1.669980in}}%
\pgfpathlineto{\pgfqpoint{2.226703in}{0.750358in}}%
\pgfpathlineto{\pgfqpoint{2.228168in}{1.648390in}}%
\pgfpathlineto{\pgfqpoint{2.229786in}{0.795386in}}%
\pgfpathlineto{\pgfqpoint{2.231746in}{1.738145in}}%
\pgfpathlineto{\pgfqpoint{2.233003in}{0.770178in}}%
\pgfpathlineto{\pgfqpoint{2.234759in}{1.679915in}}%
\pgfpathlineto{\pgfqpoint{2.235952in}{0.691807in}}%
\pgfpathlineto{\pgfqpoint{2.237583in}{1.733985in}}%
\pgfpathlineto{\pgfqpoint{2.239096in}{0.756892in}}%
\pgfpathlineto{\pgfqpoint{2.240566in}{1.684691in}}%
\pgfpathlineto{\pgfqpoint{2.242691in}{0.664232in}}%
\pgfpathlineto{\pgfqpoint{2.243690in}{1.661145in}}%
\pgfpathlineto{\pgfqpoint{2.245286in}{0.707649in}}%
\pgfpathlineto{\pgfqpoint{2.246922in}{1.625836in}}%
\pgfpathlineto{\pgfqpoint{2.248460in}{0.688634in}}%
\pgfpathlineto{\pgfqpoint{2.250039in}{1.615520in}}%
\pgfpathlineto{\pgfqpoint{2.251596in}{0.685385in}}%
\pgfpathlineto{\pgfqpoint{2.253753in}{1.681157in}}%
\pgfpathlineto{\pgfqpoint{2.254473in}{0.731890in}}%
\pgfpathlineto{\pgfqpoint{2.256227in}{1.624556in}}%
\pgfpathlineto{\pgfqpoint{2.257508in}{0.722553in}}%
\pgfpathlineto{\pgfqpoint{2.259022in}{1.688617in}}%
\pgfpathlineto{\pgfqpoint{2.260548in}{0.601479in}}%
\pgfpathlineto{\pgfqpoint{2.262156in}{1.607881in}}%
\pgfpathlineto{\pgfqpoint{2.263621in}{0.782887in}}%
\pgfpathlineto{\pgfqpoint{2.265297in}{1.614640in}}%
\pgfpathlineto{\pgfqpoint{2.266744in}{0.741103in}}%
\pgfpathlineto{\pgfqpoint{2.268313in}{1.647548in}}%
\pgfpathlineto{\pgfqpoint{2.270480in}{0.690262in}}%
\pgfpathlineto{\pgfqpoint{2.271370in}{1.627903in}}%
\pgfpathlineto{\pgfqpoint{2.273028in}{0.763641in}}%
\pgfpathlineto{\pgfqpoint{2.274486in}{1.657778in}}%
\pgfpathlineto{\pgfqpoint{2.276042in}{0.818906in}}%
\pgfpathlineto{\pgfqpoint{2.278342in}{1.729592in}}%
\pgfpathlineto{\pgfqpoint{2.279053in}{0.871900in}}%
\pgfpathlineto{\pgfqpoint{2.280976in}{1.716586in}}%
\pgfpathlineto{\pgfqpoint{2.282918in}{0.733174in}}%
\pgfpathlineto{\pgfqpoint{2.283789in}{1.734505in}}%
\pgfpathlineto{\pgfqpoint{2.285192in}{0.790925in}}%
\pgfpathlineto{\pgfqpoint{2.286771in}{1.736863in}}%
\pgfpathlineto{\pgfqpoint{2.288365in}{0.741984in}}%
\pgfpathlineto{\pgfqpoint{2.289827in}{1.685555in}}%
\pgfpathlineto{\pgfqpoint{2.291714in}{0.847478in}}%
\pgfpathlineto{\pgfqpoint{2.292908in}{1.756993in}}%
\pgfpathlineto{\pgfqpoint{2.294454in}{0.826801in}}%
\pgfpathlineto{\pgfqpoint{2.297140in}{1.782126in}}%
\pgfpathlineto{\pgfqpoint{2.297678in}{0.843370in}}%
\pgfpathlineto{\pgfqpoint{2.299086in}{1.674510in}}%
\pgfpathlineto{\pgfqpoint{2.300895in}{0.755836in}}%
\pgfpathlineto{\pgfqpoint{2.302427in}{1.695247in}}%
\pgfpathlineto{\pgfqpoint{2.303995in}{0.755843in}}%
\pgfpathlineto{\pgfqpoint{2.305285in}{1.679578in}}%
\pgfpathlineto{\pgfqpoint{2.306984in}{0.833709in}}%
\pgfpathlineto{\pgfqpoint{2.308494in}{1.719493in}}%
\pgfpathlineto{\pgfqpoint{2.310665in}{0.729118in}}%
\pgfpathlineto{\pgfqpoint{2.311610in}{1.779948in}}%
\pgfpathlineto{\pgfqpoint{2.313053in}{0.829521in}}%
\pgfpathlineto{\pgfqpoint{2.314829in}{1.749324in}}%
\pgfpathlineto{\pgfqpoint{2.316449in}{0.716566in}}%
\pgfpathlineto{\pgfqpoint{2.317606in}{1.724587in}}%
\pgfpathlineto{\pgfqpoint{2.319800in}{0.815749in}}%
\pgfpathlineto{\pgfqpoint{2.320997in}{1.731010in}}%
\pgfpathlineto{\pgfqpoint{2.322423in}{0.799258in}}%
\pgfpathlineto{\pgfqpoint{2.323784in}{1.743891in}}%
\pgfpathlineto{\pgfqpoint{2.325360in}{0.877570in}}%
\pgfpathlineto{\pgfqpoint{2.326853in}{1.798010in}}%
\pgfpathlineto{\pgfqpoint{2.328762in}{0.885606in}}%
\pgfpathlineto{\pgfqpoint{2.330273in}{1.780199in}}%
\pgfpathlineto{\pgfqpoint{2.332084in}{0.722490in}}%
\pgfpathlineto{\pgfqpoint{2.333049in}{1.690233in}}%
\pgfpathlineto{\pgfqpoint{2.335053in}{0.760126in}}%
\pgfpathlineto{\pgfqpoint{2.336253in}{1.724107in}}%
\pgfpathlineto{\pgfqpoint{2.337798in}{0.881149in}}%
\pgfpathlineto{\pgfqpoint{2.339558in}{1.824708in}}%
\pgfpathlineto{\pgfqpoint{2.340882in}{0.927475in}}%
\pgfpathlineto{\pgfqpoint{2.342542in}{1.828956in}}%
\pgfpathlineto{\pgfqpoint{2.343911in}{0.871075in}}%
\pgfpathlineto{\pgfqpoint{2.345516in}{1.727044in}}%
\pgfpathlineto{\pgfqpoint{2.346910in}{0.890509in}}%
\pgfpathlineto{\pgfqpoint{2.349150in}{1.762354in}}%
\pgfpathlineto{\pgfqpoint{2.350066in}{0.863382in}}%
\pgfpathlineto{\pgfqpoint{2.352007in}{1.753387in}}%
\pgfpathlineto{\pgfqpoint{2.353275in}{0.843578in}}%
\pgfpathlineto{\pgfqpoint{2.354689in}{1.731292in}}%
\pgfpathlineto{\pgfqpoint{2.356637in}{0.771068in}}%
\pgfpathlineto{\pgfqpoint{2.357813in}{1.643119in}}%
\pgfpathlineto{\pgfqpoint{2.360074in}{0.790948in}}%
\pgfpathlineto{\pgfqpoint{2.360793in}{1.685667in}}%
\pgfpathlineto{\pgfqpoint{2.362491in}{0.803644in}}%
\pgfpathlineto{\pgfqpoint{2.363956in}{1.680878in}}%
\pgfpathlineto{\pgfqpoint{2.365970in}{0.741874in}}%
\pgfpathlineto{\pgfqpoint{2.367322in}{1.842780in}}%
\pgfpathlineto{\pgfqpoint{2.368558in}{0.774366in}}%
\pgfpathlineto{\pgfqpoint{2.370343in}{1.734597in}}%
\pgfpathlineto{\pgfqpoint{2.371651in}{0.813269in}}%
\pgfpathlineto{\pgfqpoint{2.373592in}{1.782178in}}%
\pgfpathlineto{\pgfqpoint{2.375120in}{0.740063in}}%
\pgfpathlineto{\pgfqpoint{2.376427in}{1.723467in}}%
\pgfpathlineto{\pgfqpoint{2.378167in}{0.763461in}}%
\pgfpathlineto{\pgfqpoint{2.379329in}{1.653166in}}%
\pgfpathlineto{\pgfqpoint{2.380976in}{0.835364in}}%
\pgfpathlineto{\pgfqpoint{2.382546in}{1.756612in}}%
\pgfpathlineto{\pgfqpoint{2.383983in}{0.835565in}}%
\pgfpathlineto{\pgfqpoint{2.386044in}{1.831244in}}%
\pgfpathlineto{\pgfqpoint{2.387131in}{0.755458in}}%
\pgfpathlineto{\pgfqpoint{2.388826in}{1.750138in}}%
\pgfpathlineto{\pgfqpoint{2.390395in}{0.863189in}}%
\pgfpathlineto{\pgfqpoint{2.391703in}{1.784204in}}%
\pgfpathlineto{\pgfqpoint{2.393193in}{0.887185in}}%
\pgfpathlineto{\pgfqpoint{2.394842in}{1.756339in}}%
\pgfpathlineto{\pgfqpoint{2.396357in}{0.891590in}}%
\pgfpathlineto{\pgfqpoint{2.397835in}{1.707144in}}%
\pgfpathlineto{\pgfqpoint{2.399831in}{0.793620in}}%
\pgfpathlineto{\pgfqpoint{2.401502in}{1.870452in}}%
\pgfpathlineto{\pgfqpoint{2.402840in}{0.838124in}}%
\pgfpathlineto{\pgfqpoint{2.404030in}{1.735207in}}%
\pgfpathlineto{\pgfqpoint{2.405789in}{0.869925in}}%
\pgfpathlineto{\pgfqpoint{2.407542in}{1.865879in}}%
\pgfpathlineto{\pgfqpoint{2.408681in}{0.895814in}}%
\pgfpathlineto{\pgfqpoint{2.410844in}{1.877847in}}%
\pgfpathlineto{\pgfqpoint{2.411753in}{0.852610in}}%
\pgfpathlineto{\pgfqpoint{2.413250in}{1.757360in}}%
\pgfpathlineto{\pgfqpoint{2.415092in}{0.892420in}}%
\pgfpathlineto{\pgfqpoint{2.416367in}{1.818138in}}%
\pgfpathlineto{\pgfqpoint{2.417937in}{0.885150in}}%
\pgfpathlineto{\pgfqpoint{2.419748in}{1.806298in}}%
\pgfpathlineto{\pgfqpoint{2.421653in}{0.786119in}}%
\pgfpathlineto{\pgfqpoint{2.422528in}{1.742358in}}%
\pgfpathlineto{\pgfqpoint{2.425093in}{0.797104in}}%
\pgfpathlineto{\pgfqpoint{2.425615in}{1.765824in}}%
\pgfpathlineto{\pgfqpoint{2.427146in}{0.952579in}}%
\pgfpathlineto{\pgfqpoint{2.428881in}{1.833156in}}%
\pgfpathlineto{\pgfqpoint{2.430394in}{0.939739in}}%
\pgfpathlineto{\pgfqpoint{2.431842in}{1.800130in}}%
\pgfpathlineto{\pgfqpoint{2.433348in}{0.802698in}}%
\pgfpathlineto{\pgfqpoint{2.435122in}{1.809002in}}%
\pgfpathlineto{\pgfqpoint{2.436610in}{0.882275in}}%
\pgfpathlineto{\pgfqpoint{2.438419in}{1.829667in}}%
\pgfpathlineto{\pgfqpoint{2.439647in}{0.894875in}}%
\pgfpathlineto{\pgfqpoint{2.441105in}{1.877818in}}%
\pgfpathlineto{\pgfqpoint{2.443129in}{0.853597in}}%
\pgfpathlineto{\pgfqpoint{2.444128in}{1.740309in}}%
\pgfpathlineto{\pgfqpoint{2.445799in}{0.855623in}}%
\pgfpathlineto{\pgfqpoint{2.447683in}{1.794956in}}%
\pgfpathlineto{\pgfqpoint{2.448805in}{0.877869in}}%
\pgfpathlineto{\pgfqpoint{2.450294in}{1.738610in}}%
\pgfpathlineto{\pgfqpoint{2.451942in}{0.842177in}}%
\pgfpathlineto{\pgfqpoint{2.453618in}{1.806978in}}%
\pgfpathlineto{\pgfqpoint{2.455259in}{0.842121in}}%
\pgfpathlineto{\pgfqpoint{2.456684in}{1.846577in}}%
\pgfpathlineto{\pgfqpoint{2.458046in}{0.918644in}}%
\pgfpathlineto{\pgfqpoint{2.460162in}{1.804553in}}%
\pgfpathlineto{\pgfqpoint{2.461652in}{0.744884in}}%
\pgfpathlineto{\pgfqpoint{2.462935in}{1.820516in}}%
\pgfpathlineto{\pgfqpoint{2.464182in}{0.909572in}}%
\pgfpathlineto{\pgfqpoint{2.465901in}{1.829477in}}%
\pgfpathlineto{\pgfqpoint{2.467280in}{0.901659in}}%
\pgfpathlineto{\pgfqpoint{2.468826in}{1.823392in}}%
\pgfpathlineto{\pgfqpoint{2.470380in}{0.922256in}}%
\pgfpathlineto{\pgfqpoint{2.472194in}{1.779523in}}%
\pgfpathlineto{\pgfqpoint{2.473438in}{0.924713in}}%
\pgfpathlineto{\pgfqpoint{2.474960in}{1.789907in}}%
\pgfpathlineto{\pgfqpoint{2.477066in}{0.790634in}}%
\pgfpathlineto{\pgfqpoint{2.478379in}{1.830857in}}%
\pgfpathlineto{\pgfqpoint{2.479868in}{0.899452in}}%
\pgfpathlineto{\pgfqpoint{2.481336in}{1.727162in}}%
\pgfpathlineto{\pgfqpoint{2.482730in}{0.906268in}}%
\pgfpathlineto{\pgfqpoint{2.484357in}{1.882814in}}%
\pgfpathlineto{\pgfqpoint{2.485817in}{0.896514in}}%
\pgfpathlineto{\pgfqpoint{2.487807in}{1.812762in}}%
\pgfpathlineto{\pgfqpoint{2.488865in}{0.895289in}}%
\pgfpathlineto{\pgfqpoint{2.490570in}{1.770402in}}%
\pgfpathlineto{\pgfqpoint{2.491953in}{0.796121in}}%
\pgfpathlineto{\pgfqpoint{2.493542in}{1.776544in}}%
\pgfpathlineto{\pgfqpoint{2.495082in}{0.943616in}}%
\pgfpathlineto{\pgfqpoint{2.496871in}{1.874825in}}%
\pgfpathlineto{\pgfqpoint{2.498754in}{0.825504in}}%
\pgfpathlineto{\pgfqpoint{2.499682in}{1.745761in}}%
\pgfpathlineto{\pgfqpoint{2.502149in}{0.751538in}}%
\pgfpathlineto{\pgfqpoint{2.503109in}{1.826097in}}%
\pgfpathlineto{\pgfqpoint{2.504668in}{0.715518in}}%
\pgfpathlineto{\pgfqpoint{2.505856in}{1.724663in}}%
\pgfpathlineto{\pgfqpoint{2.507389in}{0.955106in}}%
\pgfpathlineto{\pgfqpoint{2.509234in}{1.746717in}}%
\pgfpathlineto{\pgfqpoint{2.510451in}{0.879006in}}%
\pgfpathlineto{\pgfqpoint{2.512017in}{1.684758in}}%
\pgfpathlineto{\pgfqpoint{2.513679in}{0.736120in}}%
\pgfpathlineto{\pgfqpoint{2.515115in}{1.769408in}}%
\pgfpathlineto{\pgfqpoint{2.516691in}{0.834439in}}%
\pgfpathlineto{\pgfqpoint{2.518276in}{1.740380in}}%
\pgfpathlineto{\pgfqpoint{2.520067in}{0.821852in}}%
\pgfpathlineto{\pgfqpoint{2.521345in}{1.673451in}}%
\pgfpathlineto{\pgfqpoint{2.523011in}{0.774704in}}%
\pgfpathlineto{\pgfqpoint{2.524395in}{1.775922in}}%
\pgfpathlineto{\pgfqpoint{2.526022in}{0.823320in}}%
\pgfpathlineto{\pgfqpoint{2.527803in}{1.717657in}}%
\pgfpathlineto{\pgfqpoint{2.529161in}{0.850128in}}%
\pgfpathlineto{\pgfqpoint{2.530956in}{1.815251in}}%
\pgfpathlineto{\pgfqpoint{2.532230in}{0.846583in}}%
\pgfpathlineto{\pgfqpoint{2.533745in}{1.737678in}}%
\pgfpathlineto{\pgfqpoint{2.535360in}{0.790883in}}%
\pgfpathlineto{\pgfqpoint{2.536882in}{1.780642in}}%
\pgfpathlineto{\pgfqpoint{2.538288in}{0.772685in}}%
\pgfpathlineto{\pgfqpoint{2.539955in}{1.695805in}}%
\pgfpathlineto{\pgfqpoint{2.541302in}{0.838681in}}%
\pgfpathlineto{\pgfqpoint{2.543916in}{1.726409in}}%
\pgfpathlineto{\pgfqpoint{2.544526in}{0.831093in}}%
\pgfpathlineto{\pgfqpoint{2.546009in}{1.685962in}}%
\pgfpathlineto{\pgfqpoint{2.547474in}{0.880729in}}%
\pgfpathlineto{\pgfqpoint{2.549118in}{1.733991in}}%
\pgfpathlineto{\pgfqpoint{2.550710in}{0.846602in}}%
\pgfpathlineto{\pgfqpoint{2.552372in}{1.626794in}}%
\pgfpathlineto{\pgfqpoint{2.553927in}{0.820102in}}%
\pgfpathlineto{\pgfqpoint{2.556252in}{1.834787in}}%
\pgfpathlineto{\pgfqpoint{2.556737in}{0.885833in}}%
\pgfpathlineto{\pgfqpoint{2.558410in}{1.710233in}}%
\pgfpathlineto{\pgfqpoint{2.559967in}{0.804682in}}%
\pgfpathlineto{\pgfqpoint{2.561537in}{1.717419in}}%
\pgfpathlineto{\pgfqpoint{2.563039in}{0.739296in}}%
\pgfpathlineto{\pgfqpoint{2.564445in}{1.692889in}}%
\pgfpathlineto{\pgfqpoint{2.566548in}{0.760595in}}%
\pgfpathlineto{\pgfqpoint{2.567614in}{1.722613in}}%
\pgfpathlineto{\pgfqpoint{2.569921in}{0.661154in}}%
\pgfpathlineto{\pgfqpoint{2.570625in}{1.673253in}}%
\pgfpathlineto{\pgfqpoint{2.572439in}{0.775362in}}%
\pgfpathlineto{\pgfqpoint{2.573784in}{1.634862in}}%
\pgfpathlineto{\pgfqpoint{2.575378in}{0.771948in}}%
\pgfpathlineto{\pgfqpoint{2.577100in}{1.679412in}}%
\pgfpathlineto{\pgfqpoint{2.578585in}{0.719283in}}%
\pgfpathlineto{\pgfqpoint{2.579947in}{1.706466in}}%
\pgfpathlineto{\pgfqpoint{2.581472in}{0.809536in}}%
\pgfpathlineto{\pgfqpoint{2.582982in}{1.714785in}}%
\pgfpathlineto{\pgfqpoint{2.584549in}{0.794862in}}%
\pgfpathlineto{\pgfqpoint{2.586067in}{1.731493in}}%
\pgfpathlineto{\pgfqpoint{2.587942in}{0.774476in}}%
\pgfpathlineto{\pgfqpoint{2.589413in}{1.704945in}}%
\pgfpathlineto{\pgfqpoint{2.591365in}{0.745226in}}%
\pgfpathlineto{\pgfqpoint{2.592262in}{1.639411in}}%
\pgfpathlineto{\pgfqpoint{2.594470in}{0.661123in}}%
\pgfpathlineto{\pgfqpoint{2.595441in}{1.653515in}}%
\pgfpathlineto{\pgfqpoint{2.596919in}{0.732379in}}%
\pgfpathlineto{\pgfqpoint{2.598975in}{1.755802in}}%
\pgfpathlineto{\pgfqpoint{2.599942in}{0.782592in}}%
\pgfpathlineto{\pgfqpoint{2.602491in}{1.870300in}}%
\pgfpathlineto{\pgfqpoint{2.603311in}{0.843786in}}%
\pgfpathlineto{\pgfqpoint{2.604610in}{1.692745in}}%
\pgfpathlineto{\pgfqpoint{2.606245in}{0.814871in}}%
\pgfpathlineto{\pgfqpoint{2.607682in}{1.738453in}}%
\pgfpathlineto{\pgfqpoint{2.609196in}{0.784303in}}%
\pgfpathlineto{\pgfqpoint{2.610903in}{1.671883in}}%
\pgfpathlineto{\pgfqpoint{2.612290in}{0.796254in}}%
\pgfpathlineto{\pgfqpoint{2.613869in}{1.731862in}}%
\pgfpathlineto{\pgfqpoint{2.615840in}{0.762426in}}%
\pgfpathlineto{\pgfqpoint{2.616902in}{1.695864in}}%
\pgfpathlineto{\pgfqpoint{2.618890in}{0.738376in}}%
\pgfpathlineto{\pgfqpoint{2.620605in}{1.674971in}}%
\pgfpathlineto{\pgfqpoint{2.621544in}{0.802891in}}%
\pgfpathlineto{\pgfqpoint{2.623122in}{1.656094in}}%
\pgfpathlineto{\pgfqpoint{2.625057in}{0.648239in}}%
\pgfpathlineto{\pgfqpoint{2.626253in}{1.628852in}}%
\pgfpathlineto{\pgfqpoint{2.627727in}{0.776050in}}%
\pgfpathlineto{\pgfqpoint{2.629586in}{1.739052in}}%
\pgfpathlineto{\pgfqpoint{2.631086in}{0.694375in}}%
\pgfpathlineto{\pgfqpoint{2.632538in}{1.662413in}}%
\pgfpathlineto{\pgfqpoint{2.633944in}{0.699580in}}%
\pgfpathlineto{\pgfqpoint{2.635573in}{1.690658in}}%
\pgfpathlineto{\pgfqpoint{2.637679in}{0.722385in}}%
\pgfpathlineto{\pgfqpoint{2.638544in}{1.603496in}}%
\pgfpathlineto{\pgfqpoint{2.640063in}{0.776078in}}%
\pgfpathlineto{\pgfqpoint{2.641602in}{1.642609in}}%
\pgfpathlineto{\pgfqpoint{2.644418in}{0.601333in}}%
\pgfpathlineto{\pgfqpoint{2.644704in}{1.578702in}}%
\pgfpathlineto{\pgfqpoint{2.646708in}{0.696819in}}%
\pgfpathlineto{\pgfqpoint{2.647904in}{1.633382in}}%
\pgfpathlineto{\pgfqpoint{2.649332in}{0.731107in}}%
\pgfpathlineto{\pgfqpoint{2.651174in}{1.654753in}}%
\pgfpathlineto{\pgfqpoint{2.652748in}{0.655188in}}%
\pgfpathlineto{\pgfqpoint{2.654010in}{1.622738in}}%
\pgfpathlineto{\pgfqpoint{2.655510in}{0.754469in}}%
\pgfpathlineto{\pgfqpoint{2.657096in}{1.755245in}}%
\pgfpathlineto{\pgfqpoint{2.658792in}{0.650053in}}%
\pgfpathlineto{\pgfqpoint{2.660554in}{1.622818in}}%
\pgfpathlineto{\pgfqpoint{2.661835in}{0.740979in}}%
\pgfpathlineto{\pgfqpoint{2.663236in}{1.643016in}}%
\pgfpathlineto{\pgfqpoint{2.664834in}{0.746396in}}%
\pgfpathlineto{\pgfqpoint{2.666405in}{1.706001in}}%
\pgfpathlineto{\pgfqpoint{2.668309in}{0.654846in}}%
\pgfpathlineto{\pgfqpoint{2.669392in}{1.604781in}}%
\pgfpathlineto{\pgfqpoint{2.671120in}{0.755707in}}%
\pgfpathlineto{\pgfqpoint{2.672491in}{1.711299in}}%
\pgfpathlineto{\pgfqpoint{2.674020in}{0.703854in}}%
\pgfpathlineto{\pgfqpoint{2.675637in}{1.605474in}}%
\pgfpathlineto{\pgfqpoint{2.677220in}{0.726340in}}%
\pgfpathlineto{\pgfqpoint{2.678778in}{1.693821in}}%
\pgfpathlineto{\pgfqpoint{2.680834in}{0.699245in}}%
\pgfpathlineto{\pgfqpoint{2.681710in}{1.594229in}}%
\pgfpathlineto{\pgfqpoint{2.683954in}{0.692714in}}%
\pgfpathlineto{\pgfqpoint{2.684963in}{1.605485in}}%
\pgfpathlineto{\pgfqpoint{2.686539in}{0.747030in}}%
\pgfpathlineto{\pgfqpoint{2.687975in}{1.631696in}}%
\pgfpathlineto{\pgfqpoint{2.689653in}{0.736415in}}%
\pgfpathlineto{\pgfqpoint{2.691443in}{1.654882in}}%
\pgfpathlineto{\pgfqpoint{2.692517in}{0.728619in}}%
\pgfpathlineto{\pgfqpoint{2.694471in}{1.652960in}}%
\pgfpathlineto{\pgfqpoint{2.695609in}{0.813098in}}%
\pgfpathlineto{\pgfqpoint{2.697583in}{1.709625in}}%
\pgfpathlineto{\pgfqpoint{2.698739in}{0.764978in}}%
\pgfpathlineto{\pgfqpoint{2.700245in}{1.573899in}}%
\pgfpathlineto{\pgfqpoint{2.702351in}{0.618607in}}%
\pgfpathlineto{\pgfqpoint{2.703305in}{1.583942in}}%
\pgfpathlineto{\pgfqpoint{2.704848in}{0.701818in}}%
\pgfpathlineto{\pgfqpoint{2.706433in}{1.606039in}}%
\pgfpathlineto{\pgfqpoint{2.708094in}{0.673521in}}%
\pgfpathlineto{\pgfqpoint{2.710189in}{1.735500in}}%
\pgfpathlineto{\pgfqpoint{2.711171in}{0.712503in}}%
\pgfpathlineto{\pgfqpoint{2.712643in}{1.652573in}}%
\pgfpathlineto{\pgfqpoint{2.714551in}{0.509715in}}%
\pgfpathlineto{\pgfqpoint{2.715943in}{1.617094in}}%
\pgfpathlineto{\pgfqpoint{2.717531in}{0.610355in}}%
\pgfpathlineto{\pgfqpoint{2.719434in}{1.697240in}}%
\pgfpathlineto{\pgfqpoint{2.720600in}{0.712042in}}%
\pgfpathlineto{\pgfqpoint{2.721831in}{1.572530in}}%
\pgfpathlineto{\pgfqpoint{2.723701in}{0.680210in}}%
\pgfpathlineto{\pgfqpoint{2.725003in}{1.627153in}}%
\pgfpathlineto{\pgfqpoint{2.726521in}{0.765425in}}%
\pgfpathlineto{\pgfqpoint{2.728555in}{1.698087in}}%
\pgfpathlineto{\pgfqpoint{2.729810in}{0.763677in}}%
\pgfpathlineto{\pgfqpoint{2.731240in}{1.630674in}}%
\pgfpathlineto{\pgfqpoint{2.732679in}{0.768135in}}%
\pgfpathlineto{\pgfqpoint{2.734295in}{1.659277in}}%
\pgfpathlineto{\pgfqpoint{2.736370in}{0.670484in}}%
\pgfpathlineto{\pgfqpoint{2.737279in}{1.558851in}}%
\pgfpathlineto{\pgfqpoint{2.738871in}{0.702689in}}%
\pgfpathlineto{\pgfqpoint{2.740431in}{1.608722in}}%
\pgfpathlineto{\pgfqpoint{2.742774in}{0.688396in}}%
\pgfpathlineto{\pgfqpoint{2.744208in}{1.764515in}}%
\pgfpathlineto{\pgfqpoint{2.745273in}{0.633261in}}%
\pgfpathlineto{\pgfqpoint{2.746980in}{1.654156in}}%
\pgfpathlineto{\pgfqpoint{2.748731in}{0.687864in}}%
\pgfpathlineto{\pgfqpoint{2.749659in}{1.599817in}}%
\pgfpathlineto{\pgfqpoint{2.751673in}{0.608919in}}%
\pgfpathlineto{\pgfqpoint{2.752910in}{1.634026in}}%
\pgfpathlineto{\pgfqpoint{2.754503in}{0.793632in}}%
\pgfpathlineto{\pgfqpoint{2.756017in}{1.650601in}}%
\pgfpathlineto{\pgfqpoint{2.757487in}{0.741896in}}%
\pgfpathlineto{\pgfqpoint{2.758963in}{1.737907in}}%
\pgfpathlineto{\pgfqpoint{2.760540in}{0.671455in}}%
\pgfpathlineto{\pgfqpoint{2.762491in}{1.674836in}}%
\pgfpathlineto{\pgfqpoint{2.763584in}{0.773960in}}%
\pgfpathlineto{\pgfqpoint{2.765236in}{1.608352in}}%
\pgfpathlineto{\pgfqpoint{2.767079in}{0.707020in}}%
\pgfpathlineto{\pgfqpoint{2.768332in}{1.625602in}}%
\pgfpathlineto{\pgfqpoint{2.769782in}{0.773682in}}%
\pgfpathlineto{\pgfqpoint{2.771394in}{1.664535in}}%
\pgfpathlineto{\pgfqpoint{2.772793in}{0.733087in}}%
\pgfpathlineto{\pgfqpoint{2.774890in}{1.688714in}}%
\pgfpathlineto{\pgfqpoint{2.775987in}{0.748851in}}%
\pgfpathlineto{\pgfqpoint{2.777545in}{1.655186in}}%
\pgfpathlineto{\pgfqpoint{2.778976in}{0.714294in}}%
\pgfpathlineto{\pgfqpoint{2.780556in}{1.609394in}}%
\pgfpathlineto{\pgfqpoint{2.782057in}{0.693204in}}%
\pgfpathlineto{\pgfqpoint{2.783568in}{1.680204in}}%
\pgfpathlineto{\pgfqpoint{2.785386in}{0.672505in}}%
\pgfpathlineto{\pgfqpoint{2.786696in}{1.606494in}}%
\pgfpathlineto{\pgfqpoint{2.788705in}{0.723928in}}%
\pgfpathlineto{\pgfqpoint{2.789810in}{1.772710in}}%
\pgfpathlineto{\pgfqpoint{2.791489in}{0.793629in}}%
\pgfpathlineto{\pgfqpoint{2.792948in}{1.678050in}}%
\pgfpathlineto{\pgfqpoint{2.794477in}{0.765202in}}%
\pgfpathlineto{\pgfqpoint{2.796666in}{1.726169in}}%
\pgfpathlineto{\pgfqpoint{2.797596in}{0.811526in}}%
\pgfpathlineto{\pgfqpoint{2.800009in}{1.767204in}}%
\pgfpathlineto{\pgfqpoint{2.800536in}{0.811850in}}%
\pgfpathlineto{\pgfqpoint{2.802354in}{1.696469in}}%
\pgfpathlineto{\pgfqpoint{2.803620in}{0.788058in}}%
\pgfpathlineto{\pgfqpoint{2.805318in}{1.664693in}}%
\pgfpathlineto{\pgfqpoint{2.806680in}{0.838489in}}%
\pgfpathlineto{\pgfqpoint{2.809503in}{1.771405in}}%
\pgfpathlineto{\pgfqpoint{2.809908in}{0.859568in}}%
\pgfpathlineto{\pgfqpoint{2.811614in}{1.666725in}}%
\pgfpathlineto{\pgfqpoint{2.813626in}{0.727499in}}%
\pgfpathlineto{\pgfqpoint{2.814492in}{1.643586in}}%
\pgfpathlineto{\pgfqpoint{2.816724in}{0.745994in}}%
\pgfpathlineto{\pgfqpoint{2.817751in}{1.684779in}}%
\pgfpathlineto{\pgfqpoint{2.819039in}{0.751430in}}%
\pgfpathlineto{\pgfqpoint{2.820793in}{1.719338in}}%
\pgfpathlineto{\pgfqpoint{2.822598in}{0.766103in}}%
\pgfpathlineto{\pgfqpoint{2.823961in}{1.716382in}}%
\pgfpathlineto{\pgfqpoint{2.825620in}{0.757759in}}%
\pgfpathlineto{\pgfqpoint{2.827062in}{1.702038in}}%
\pgfpathlineto{\pgfqpoint{2.828714in}{0.768669in}}%
\pgfpathlineto{\pgfqpoint{2.829835in}{1.679913in}}%
\pgfpathlineto{\pgfqpoint{2.832038in}{0.687696in}}%
\pgfpathlineto{\pgfqpoint{2.833514in}{1.867145in}}%
\pgfpathlineto{\pgfqpoint{2.834786in}{0.776254in}}%
\pgfpathlineto{\pgfqpoint{2.836060in}{1.745938in}}%
\pgfpathlineto{\pgfqpoint{2.837759in}{0.755053in}}%
\pgfpathlineto{\pgfqpoint{2.839429in}{1.732171in}}%
\pgfpathlineto{\pgfqpoint{2.840619in}{0.806834in}}%
\pgfpathlineto{\pgfqpoint{2.842186in}{1.721762in}}%
\pgfpathlineto{\pgfqpoint{2.844515in}{0.778407in}}%
\pgfpathlineto{\pgfqpoint{2.845344in}{1.700341in}}%
\pgfpathlineto{\pgfqpoint{2.846858in}{0.841372in}}%
\pgfpathlineto{\pgfqpoint{2.848658in}{1.785828in}}%
\pgfpathlineto{\pgfqpoint{2.850705in}{0.743457in}}%
\pgfpathlineto{\pgfqpoint{2.851525in}{1.669699in}}%
\pgfpathlineto{\pgfqpoint{2.853908in}{0.683455in}}%
\pgfpathlineto{\pgfqpoint{2.854755in}{1.714577in}}%
\pgfpathlineto{\pgfqpoint{2.856288in}{0.745545in}}%
\pgfpathlineto{\pgfqpoint{2.857769in}{1.693834in}}%
\pgfpathlineto{\pgfqpoint{2.859246in}{0.816100in}}%
\pgfpathlineto{\pgfqpoint{2.860696in}{1.751694in}}%
\pgfpathlineto{\pgfqpoint{2.862271in}{0.829443in}}%
\pgfpathlineto{\pgfqpoint{2.863762in}{1.682943in}}%
\pgfpathlineto{\pgfqpoint{2.866274in}{0.697362in}}%
\pgfpathlineto{\pgfqpoint{2.867307in}{1.800993in}}%
\pgfpathlineto{\pgfqpoint{2.868495in}{0.865142in}}%
\pgfpathlineto{\pgfqpoint{2.870114in}{1.776513in}}%
\pgfpathlineto{\pgfqpoint{2.871551in}{0.806004in}}%
\pgfpathlineto{\pgfqpoint{2.873139in}{1.763692in}}%
\pgfpathlineto{\pgfqpoint{2.874675in}{0.820374in}}%
\pgfpathlineto{\pgfqpoint{2.876610in}{1.764240in}}%
\pgfpathlineto{\pgfqpoint{2.877663in}{0.829929in}}%
\pgfpathlineto{\pgfqpoint{2.879915in}{1.811043in}}%
\pgfpathlineto{\pgfqpoint{2.880947in}{0.881243in}}%
\pgfpathlineto{\pgfqpoint{2.882318in}{1.769995in}}%
\pgfpathlineto{\pgfqpoint{2.883815in}{0.901972in}}%
\pgfpathlineto{\pgfqpoint{2.885731in}{1.899629in}}%
\pgfpathlineto{\pgfqpoint{2.887008in}{0.817155in}}%
\pgfpathlineto{\pgfqpoint{2.888868in}{1.776055in}}%
\pgfpathlineto{\pgfqpoint{2.890224in}{0.818427in}}%
\pgfpathlineto{\pgfqpoint{2.891654in}{1.735264in}}%
\pgfpathlineto{\pgfqpoint{2.893111in}{0.838381in}}%
\pgfpathlineto{\pgfqpoint{2.894666in}{1.745082in}}%
\pgfpathlineto{\pgfqpoint{2.896225in}{0.836904in}}%
\pgfpathlineto{\pgfqpoint{2.897818in}{1.785503in}}%
\pgfpathlineto{\pgfqpoint{2.899334in}{0.876494in}}%
\pgfpathlineto{\pgfqpoint{2.900969in}{1.835752in}}%
\pgfpathlineto{\pgfqpoint{2.902733in}{0.806903in}}%
\pgfpathlineto{\pgfqpoint{2.904001in}{1.807617in}}%
\pgfpathlineto{\pgfqpoint{2.905815in}{0.892346in}}%
\pgfpathlineto{\pgfqpoint{2.907052in}{1.822068in}}%
\pgfpathlineto{\pgfqpoint{2.908684in}{0.884912in}}%
\pgfpathlineto{\pgfqpoint{2.910660in}{1.853577in}}%
\pgfpathlineto{\pgfqpoint{2.911739in}{0.815663in}}%
\pgfpathlineto{\pgfqpoint{2.913331in}{1.831277in}}%
\pgfpathlineto{\pgfqpoint{2.914819in}{0.905398in}}%
\pgfpathlineto{\pgfqpoint{2.916346in}{1.827471in}}%
\pgfpathlineto{\pgfqpoint{2.917982in}{0.853704in}}%
\pgfpathlineto{\pgfqpoint{2.919583in}{1.806836in}}%
\pgfpathlineto{\pgfqpoint{2.921471in}{0.808999in}}%
\pgfpathlineto{\pgfqpoint{2.922577in}{1.867734in}}%
\pgfpathlineto{\pgfqpoint{2.924059in}{0.930302in}}%
\pgfpathlineto{\pgfqpoint{2.926164in}{1.903367in}}%
\pgfpathlineto{\pgfqpoint{2.927139in}{0.960756in}}%
\pgfpathlineto{\pgfqpoint{2.928908in}{1.824012in}}%
\pgfpathlineto{\pgfqpoint{2.930199in}{0.980468in}}%
\pgfpathlineto{\pgfqpoint{2.932512in}{1.942298in}}%
\pgfpathlineto{\pgfqpoint{2.933190in}{0.918282in}}%
\pgfpathlineto{\pgfqpoint{2.935274in}{1.943973in}}%
\pgfpathlineto{\pgfqpoint{2.936730in}{0.843468in}}%
\pgfpathlineto{\pgfqpoint{2.937954in}{1.793854in}}%
\pgfpathlineto{\pgfqpoint{2.939884in}{0.884951in}}%
\pgfpathlineto{\pgfqpoint{2.941085in}{1.851640in}}%
\pgfpathlineto{\pgfqpoint{2.942449in}{0.981821in}}%
\pgfpathlineto{\pgfqpoint{2.944040in}{1.821167in}}%
\pgfpathlineto{\pgfqpoint{2.945597in}{0.932728in}}%
\pgfpathlineto{\pgfqpoint{2.947219in}{1.792153in}}%
\pgfpathlineto{\pgfqpoint{2.948795in}{0.858792in}}%
\pgfpathlineto{\pgfqpoint{2.950590in}{1.793295in}}%
\pgfpathlineto{\pgfqpoint{2.951700in}{0.968716in}}%
\pgfpathlineto{\pgfqpoint{2.953360in}{1.819925in}}%
\pgfpathlineto{\pgfqpoint{2.955129in}{0.909903in}}%
\pgfpathlineto{\pgfqpoint{2.956428in}{1.893420in}}%
\pgfpathlineto{\pgfqpoint{2.958384in}{0.914269in}}%
\pgfpathlineto{\pgfqpoint{2.959672in}{1.866640in}}%
\pgfpathlineto{\pgfqpoint{2.960993in}{1.057827in}}%
\pgfpathlineto{\pgfqpoint{2.962579in}{1.905419in}}%
\pgfpathlineto{\pgfqpoint{2.964623in}{0.948430in}}%
\pgfpathlineto{\pgfqpoint{2.965791in}{1.885999in}}%
\pgfpathlineto{\pgfqpoint{2.967304in}{1.045284in}}%
\pgfpathlineto{\pgfqpoint{2.968857in}{2.002087in}}%
\pgfpathlineto{\pgfqpoint{2.970609in}{1.025765in}}%
\pgfpathlineto{\pgfqpoint{2.972064in}{2.013112in}}%
\pgfpathlineto{\pgfqpoint{2.973383in}{1.028392in}}%
\pgfpathlineto{\pgfqpoint{2.974904in}{1.970967in}}%
\pgfpathlineto{\pgfqpoint{2.976937in}{0.973543in}}%
\pgfpathlineto{\pgfqpoint{2.978016in}{1.975146in}}%
\pgfpathlineto{\pgfqpoint{2.979531in}{1.132170in}}%
\pgfpathlineto{\pgfqpoint{2.981221in}{2.062243in}}%
\pgfpathlineto{\pgfqpoint{2.982747in}{1.084300in}}%
\pgfpathlineto{\pgfqpoint{2.984311in}{2.022846in}}%
\pgfpathlineto{\pgfqpoint{2.986066in}{1.140036in}}%
\pgfpathlineto{\pgfqpoint{2.987361in}{2.019727in}}%
\pgfpathlineto{\pgfqpoint{2.989071in}{1.065044in}}%
\pgfpathlineto{\pgfqpoint{2.990437in}{1.976427in}}%
\pgfpathlineto{\pgfqpoint{2.991907in}{1.097218in}}%
\pgfpathlineto{\pgfqpoint{2.993508in}{1.986582in}}%
\pgfpathlineto{\pgfqpoint{2.994983in}{1.045529in}}%
\pgfpathlineto{\pgfqpoint{2.996754in}{2.009772in}}%
\pgfpathlineto{\pgfqpoint{2.998099in}{1.143241in}}%
\pgfpathlineto{\pgfqpoint{2.999635in}{1.997718in}}%
\pgfpathlineto{\pgfqpoint{3.001115in}{1.125232in}}%
\pgfpathlineto{\pgfqpoint{3.002790in}{2.058799in}}%
\pgfpathlineto{\pgfqpoint{3.004341in}{1.085277in}}%
\pgfpathlineto{\pgfqpoint{3.006166in}{2.044313in}}%
\pgfpathlineto{\pgfqpoint{3.007470in}{1.098714in}}%
\pgfpathlineto{\pgfqpoint{3.008829in}{1.970480in}}%
\pgfpathlineto{\pgfqpoint{3.010391in}{1.107818in}}%
\pgfpathlineto{\pgfqpoint{3.011960in}{2.009867in}}%
\pgfpathlineto{\pgfqpoint{3.013688in}{1.073269in}}%
\pgfpathlineto{\pgfqpoint{3.014988in}{2.002005in}}%
\pgfpathlineto{\pgfqpoint{3.016581in}{1.169539in}}%
\pgfpathlineto{\pgfqpoint{3.018186in}{2.051001in}}%
\pgfpathlineto{\pgfqpoint{3.019613in}{1.170251in}}%
\pgfpathlineto{\pgfqpoint{3.021252in}{2.099881in}}%
\pgfpathlineto{\pgfqpoint{3.023042in}{1.037454in}}%
\pgfpathlineto{\pgfqpoint{3.024460in}{2.012247in}}%
\pgfpathlineto{\pgfqpoint{3.026147in}{1.069980in}}%
\pgfpathlineto{\pgfqpoint{3.027945in}{2.091425in}}%
\pgfpathlineto{\pgfqpoint{3.029093in}{1.166225in}}%
\pgfpathlineto{\pgfqpoint{3.030427in}{2.028194in}}%
\pgfpathlineto{\pgfqpoint{3.032260in}{1.143069in}}%
\pgfpathlineto{\pgfqpoint{3.033860in}{2.059538in}}%
\pgfpathlineto{\pgfqpoint{3.035031in}{1.239157in}}%
\pgfpathlineto{\pgfqpoint{3.036560in}{2.066777in}}%
\pgfpathlineto{\pgfqpoint{3.038102in}{1.246942in}}%
\pgfpathlineto{\pgfqpoint{3.039905in}{2.074318in}}%
\pgfpathlineto{\pgfqpoint{3.041299in}{1.170712in}}%
\pgfpathlineto{\pgfqpoint{3.043004in}{2.080630in}}%
\pgfpathlineto{\pgfqpoint{3.044446in}{1.180003in}}%
\pgfpathlineto{\pgfqpoint{3.046161in}{2.111650in}}%
\pgfpathlineto{\pgfqpoint{3.047870in}{1.157832in}}%
\pgfpathlineto{\pgfqpoint{3.049566in}{2.143465in}}%
\pgfpathlineto{\pgfqpoint{3.051148in}{1.070597in}}%
\pgfpathlineto{\pgfqpoint{3.052051in}{2.080556in}}%
\pgfpathlineto{\pgfqpoint{3.053815in}{1.124352in}}%
\pgfpathlineto{\pgfqpoint{3.055142in}{2.120618in}}%
\pgfpathlineto{\pgfqpoint{3.056915in}{1.225544in}}%
\pgfpathlineto{\pgfqpoint{3.058203in}{2.084275in}}%
\pgfpathlineto{\pgfqpoint{3.059808in}{1.177147in}}%
\pgfpathlineto{\pgfqpoint{3.061853in}{2.126582in}}%
\pgfpathlineto{\pgfqpoint{3.063090in}{1.190003in}}%
\pgfpathlineto{\pgfqpoint{3.064331in}{2.043616in}}%
\pgfpathlineto{\pgfqpoint{3.066264in}{1.126306in}}%
\pgfpathlineto{\pgfqpoint{3.067608in}{2.085802in}}%
\pgfpathlineto{\pgfqpoint{3.069581in}{1.146957in}}%
\pgfpathlineto{\pgfqpoint{3.070626in}{2.094156in}}%
\pgfpathlineto{\pgfqpoint{3.072254in}{1.185324in}}%
\pgfpathlineto{\pgfqpoint{3.073612in}{2.103940in}}%
\pgfpathlineto{\pgfqpoint{3.075262in}{1.203242in}}%
\pgfpathlineto{\pgfqpoint{3.077236in}{2.174184in}}%
\pgfpathlineto{\pgfqpoint{3.078330in}{1.201732in}}%
\pgfpathlineto{\pgfqpoint{3.080018in}{2.096407in}}%
\pgfpathlineto{\pgfqpoint{3.081651in}{1.158616in}}%
\pgfpathlineto{\pgfqpoint{3.083401in}{2.134054in}}%
\pgfpathlineto{\pgfqpoint{3.084518in}{1.157202in}}%
\pgfpathlineto{\pgfqpoint{3.085988in}{2.052843in}}%
\pgfpathlineto{\pgfqpoint{3.087950in}{1.173044in}}%
\pgfpathlineto{\pgfqpoint{3.089281in}{2.111696in}}%
\pgfpathlineto{\pgfqpoint{3.090872in}{1.194003in}}%
\pgfpathlineto{\pgfqpoint{3.092635in}{2.148442in}}%
\pgfpathlineto{\pgfqpoint{3.093769in}{1.133346in}}%
\pgfpathlineto{\pgfqpoint{3.095425in}{2.069326in}}%
\pgfpathlineto{\pgfqpoint{3.096953in}{1.172515in}}%
\pgfpathlineto{\pgfqpoint{3.098408in}{2.120678in}}%
\pgfpathlineto{\pgfqpoint{3.100194in}{1.075312in}}%
\pgfpathlineto{\pgfqpoint{3.101548in}{2.115848in}}%
\pgfpathlineto{\pgfqpoint{3.102966in}{1.222686in}}%
\pgfpathlineto{\pgfqpoint{3.104865in}{2.175775in}}%
\pgfpathlineto{\pgfqpoint{3.106018in}{1.250296in}}%
\pgfpathlineto{\pgfqpoint{3.107600in}{2.055254in}}%
\pgfpathlineto{\pgfqpoint{3.109088in}{1.108032in}}%
\pgfpathlineto{\pgfqpoint{3.111364in}{2.214964in}}%
\pgfpathlineto{\pgfqpoint{3.112791in}{1.107221in}}%
\pgfpathlineto{\pgfqpoint{3.113923in}{2.101023in}}%
\pgfpathlineto{\pgfqpoint{3.115356in}{1.223415in}}%
\pgfpathlineto{\pgfqpoint{3.116966in}{2.151034in}}%
\pgfpathlineto{\pgfqpoint{3.118403in}{1.167011in}}%
\pgfpathlineto{\pgfqpoint{3.120009in}{2.127000in}}%
\pgfpathlineto{\pgfqpoint{3.121623in}{1.246904in}}%
\pgfpathlineto{\pgfqpoint{3.123031in}{2.092338in}}%
\pgfpathlineto{\pgfqpoint{3.124520in}{1.236838in}}%
\pgfpathlineto{\pgfqpoint{3.126470in}{2.214885in}}%
\pgfpathlineto{\pgfqpoint{3.127808in}{1.184156in}}%
\pgfpathlineto{\pgfqpoint{3.129326in}{2.184771in}}%
\pgfpathlineto{\pgfqpoint{3.130686in}{1.225349in}}%
\pgfpathlineto{\pgfqpoint{3.132435in}{2.099137in}}%
\pgfpathlineto{\pgfqpoint{3.133940in}{1.182927in}}%
\pgfpathlineto{\pgfqpoint{3.135683in}{2.134257in}}%
\pgfpathlineto{\pgfqpoint{3.136914in}{1.133262in}}%
\pgfpathlineto{\pgfqpoint{3.138510in}{2.066182in}}%
\pgfpathlineto{\pgfqpoint{3.141188in}{1.036666in}}%
\pgfpathlineto{\pgfqpoint{3.141530in}{2.110513in}}%
\pgfpathlineto{\pgfqpoint{3.143026in}{1.159053in}}%
\pgfpathlineto{\pgfqpoint{3.144621in}{2.121399in}}%
\pgfpathlineto{\pgfqpoint{3.146215in}{1.142295in}}%
\pgfpathlineto{\pgfqpoint{3.147939in}{2.124616in}}%
\pgfpathlineto{\pgfqpoint{3.149203in}{1.120998in}}%
\pgfpathlineto{\pgfqpoint{3.150771in}{2.054430in}}%
\pgfpathlineto{\pgfqpoint{3.152435in}{1.176224in}}%
\pgfpathlineto{\pgfqpoint{3.153855in}{2.043567in}}%
\pgfpathlineto{\pgfqpoint{3.155545in}{1.202724in}}%
\pgfpathlineto{\pgfqpoint{3.156931in}{2.135469in}}%
\pgfpathlineto{\pgfqpoint{3.158850in}{1.157804in}}%
\pgfpathlineto{\pgfqpoint{3.160195in}{2.088339in}}%
\pgfpathlineto{\pgfqpoint{3.161582in}{1.228630in}}%
\pgfpathlineto{\pgfqpoint{3.163609in}{2.174843in}}%
\pgfpathlineto{\pgfqpoint{3.164785in}{1.203917in}}%
\pgfpathlineto{\pgfqpoint{3.166300in}{2.095468in}}%
\pgfpathlineto{\pgfqpoint{3.167905in}{1.202197in}}%
\pgfpathlineto{\pgfqpoint{3.169334in}{2.148301in}}%
\pgfpathlineto{\pgfqpoint{3.170910in}{1.271165in}}%
\pgfpathlineto{\pgfqpoint{3.172677in}{2.144353in}}%
\pgfpathlineto{\pgfqpoint{3.173905in}{1.247725in}}%
\pgfpathlineto{\pgfqpoint{3.175682in}{2.163916in}}%
\pgfpathlineto{\pgfqpoint{3.176978in}{1.210531in}}%
\pgfpathlineto{\pgfqpoint{3.178565in}{2.114706in}}%
\pgfpathlineto{\pgfqpoint{3.180073in}{1.272561in}}%
\pgfpathlineto{\pgfqpoint{3.181828in}{2.137345in}}%
\pgfpathlineto{\pgfqpoint{3.183336in}{1.288298in}}%
\pgfpathlineto{\pgfqpoint{3.185220in}{2.268162in}}%
\pgfpathlineto{\pgfqpoint{3.186355in}{1.334927in}}%
\pgfpathlineto{\pgfqpoint{3.187906in}{2.138364in}}%
\pgfpathlineto{\pgfqpoint{3.189361in}{1.236055in}}%
\pgfpathlineto{\pgfqpoint{3.191072in}{2.124641in}}%
\pgfpathlineto{\pgfqpoint{3.192507in}{1.304938in}}%
\pgfpathlineto{\pgfqpoint{3.194025in}{2.138492in}}%
\pgfpathlineto{\pgfqpoint{3.195568in}{1.183907in}}%
\pgfpathlineto{\pgfqpoint{3.197085in}{2.127156in}}%
\pgfpathlineto{\pgfqpoint{3.198811in}{1.204660in}}%
\pgfpathlineto{\pgfqpoint{3.200209in}{2.199708in}}%
\pgfpathlineto{\pgfqpoint{3.201897in}{1.080388in}}%
\pgfpathlineto{\pgfqpoint{3.203443in}{2.186676in}}%
\pgfpathlineto{\pgfqpoint{3.204871in}{1.192041in}}%
\pgfpathlineto{\pgfqpoint{3.206608in}{2.295936in}}%
\pgfpathlineto{\pgfqpoint{3.208211in}{1.233612in}}%
\pgfpathlineto{\pgfqpoint{3.209782in}{2.241284in}}%
\pgfpathlineto{\pgfqpoint{3.211119in}{1.238823in}}%
\pgfpathlineto{\pgfqpoint{3.212678in}{2.215964in}}%
\pgfpathlineto{\pgfqpoint{3.214310in}{1.221857in}}%
\pgfpathlineto{\pgfqpoint{3.215608in}{2.120096in}}%
\pgfpathlineto{\pgfqpoint{3.217213in}{1.237787in}}%
\pgfpathlineto{\pgfqpoint{3.218709in}{2.178188in}}%
\pgfpathlineto{\pgfqpoint{3.220197in}{1.254526in}}%
\pgfpathlineto{\pgfqpoint{3.221896in}{2.083271in}}%
\pgfpathlineto{\pgfqpoint{3.223786in}{1.214224in}}%
\pgfpathlineto{\pgfqpoint{3.225033in}{2.139048in}}%
\pgfpathlineto{\pgfqpoint{3.226478in}{1.268751in}}%
\pgfpathlineto{\pgfqpoint{3.227918in}{2.121255in}}%
\pgfpathlineto{\pgfqpoint{3.229697in}{1.205226in}}%
\pgfpathlineto{\pgfqpoint{3.231199in}{2.171485in}}%
\pgfpathlineto{\pgfqpoint{3.232635in}{1.167943in}}%
\pgfpathlineto{\pgfqpoint{3.234057in}{2.049370in}}%
\pgfpathlineto{\pgfqpoint{3.235712in}{1.201634in}}%
\pgfpathlineto{\pgfqpoint{3.237377in}{2.071239in}}%
\pgfpathlineto{\pgfqpoint{3.238775in}{1.087530in}}%
\pgfpathlineto{\pgfqpoint{3.240236in}{2.100082in}}%
\pgfpathlineto{\pgfqpoint{3.241807in}{1.255096in}}%
\pgfpathlineto{\pgfqpoint{3.243575in}{2.139990in}}%
\pgfpathlineto{\pgfqpoint{3.244908in}{1.200254in}}%
\pgfpathlineto{\pgfqpoint{3.246533in}{2.219574in}}%
\pgfpathlineto{\pgfqpoint{3.248056in}{1.254044in}}%
\pgfpathlineto{\pgfqpoint{3.250189in}{2.219348in}}%
\pgfpathlineto{\pgfqpoint{3.251453in}{1.254127in}}%
\pgfpathlineto{\pgfqpoint{3.253010in}{2.265397in}}%
\pgfpathlineto{\pgfqpoint{3.254654in}{1.143340in}}%
\pgfpathlineto{\pgfqpoint{3.255736in}{2.213554in}}%
\pgfpathlineto{\pgfqpoint{3.258160in}{1.151041in}}%
\pgfpathlineto{\pgfqpoint{3.258752in}{2.060193in}}%
\pgfpathlineto{\pgfqpoint{3.260319in}{1.276229in}}%
\pgfpathlineto{\pgfqpoint{3.262044in}{2.177000in}}%
\pgfpathlineto{\pgfqpoint{3.263677in}{1.233753in}}%
\pgfpathlineto{\pgfqpoint{3.265548in}{2.153947in}}%
\pgfpathlineto{\pgfqpoint{3.267497in}{1.092613in}}%
\pgfpathlineto{\pgfqpoint{3.268082in}{2.074735in}}%
\pgfpathlineto{\pgfqpoint{3.270124in}{1.150982in}}%
\pgfpathlineto{\pgfqpoint{3.271458in}{2.079877in}}%
\pgfpathlineto{\pgfqpoint{3.272843in}{1.105169in}}%
\pgfpathlineto{\pgfqpoint{3.274319in}{2.077019in}}%
\pgfpathlineto{\pgfqpoint{3.275845in}{1.228181in}}%
\pgfpathlineto{\pgfqpoint{3.277439in}{2.145248in}}%
\pgfpathlineto{\pgfqpoint{3.278809in}{1.195574in}}%
\pgfpathlineto{\pgfqpoint{3.280404in}{2.207483in}}%
\pgfpathlineto{\pgfqpoint{3.281989in}{1.138557in}}%
\pgfpathlineto{\pgfqpoint{3.283424in}{2.103408in}}%
\pgfpathlineto{\pgfqpoint{3.285675in}{1.183877in}}%
\pgfpathlineto{\pgfqpoint{3.286529in}{2.016287in}}%
\pgfpathlineto{\pgfqpoint{3.288255in}{1.183214in}}%
\pgfpathlineto{\pgfqpoint{3.289938in}{2.103097in}}%
\pgfpathlineto{\pgfqpoint{3.291173in}{1.225050in}}%
\pgfpathlineto{\pgfqpoint{3.292770in}{2.120856in}}%
\pgfpathlineto{\pgfqpoint{3.294430in}{1.092186in}}%
\pgfpathlineto{\pgfqpoint{3.295790in}{2.096857in}}%
\pgfpathlineto{\pgfqpoint{3.297721in}{1.183919in}}%
\pgfpathlineto{\pgfqpoint{3.299229in}{2.177867in}}%
\pgfpathlineto{\pgfqpoint{3.300437in}{1.215181in}}%
\pgfpathlineto{\pgfqpoint{3.302159in}{2.123750in}}%
\pgfpathlineto{\pgfqpoint{3.303879in}{1.114881in}}%
\pgfpathlineto{\pgfqpoint{3.305259in}{2.142905in}}%
\pgfpathlineto{\pgfqpoint{3.306560in}{1.210915in}}%
\pgfpathlineto{\pgfqpoint{3.308284in}{2.073256in}}%
\pgfpathlineto{\pgfqpoint{3.309858in}{1.267001in}}%
\pgfpathlineto{\pgfqpoint{3.312130in}{2.201394in}}%
\pgfpathlineto{\pgfqpoint{3.312760in}{1.248343in}}%
\pgfpathlineto{\pgfqpoint{3.315046in}{2.161520in}}%
\pgfpathlineto{\pgfqpoint{3.315880in}{1.169335in}}%
\pgfpathlineto{\pgfqpoint{3.317758in}{2.124972in}}%
\pgfpathlineto{\pgfqpoint{3.318903in}{1.183066in}}%
\pgfpathlineto{\pgfqpoint{3.320595in}{2.100555in}}%
\pgfpathlineto{\pgfqpoint{3.322033in}{1.184032in}}%
\pgfpathlineto{\pgfqpoint{3.323906in}{2.106473in}}%
\pgfpathlineto{\pgfqpoint{3.325139in}{1.146837in}}%
\pgfpathlineto{\pgfqpoint{3.326715in}{2.104476in}}%
\pgfpathlineto{\pgfqpoint{3.328530in}{1.193413in}}%
\pgfpathlineto{\pgfqpoint{3.329825in}{2.141257in}}%
\pgfpathlineto{\pgfqpoint{3.331320in}{1.325865in}}%
\pgfpathlineto{\pgfqpoint{3.333089in}{2.120415in}}%
\pgfpathlineto{\pgfqpoint{3.334341in}{1.245228in}}%
\pgfpathlineto{\pgfqpoint{3.335878in}{2.120129in}}%
\pgfpathlineto{\pgfqpoint{3.337441in}{1.272532in}}%
\pgfpathlineto{\pgfqpoint{3.339014in}{2.076281in}}%
\pgfpathlineto{\pgfqpoint{3.340527in}{1.144406in}}%
\pgfpathlineto{\pgfqpoint{3.342270in}{2.138727in}}%
\pgfpathlineto{\pgfqpoint{3.343690in}{1.220133in}}%
\pgfpathlineto{\pgfqpoint{3.345450in}{2.136180in}}%
\pgfpathlineto{\pgfqpoint{3.346898in}{1.198846in}}%
\pgfpathlineto{\pgfqpoint{3.348318in}{2.164051in}}%
\pgfpathlineto{\pgfqpoint{3.349834in}{1.221372in}}%
\pgfpathlineto{\pgfqpoint{3.351310in}{2.110502in}}%
\pgfpathlineto{\pgfqpoint{3.353261in}{1.244554in}}%
\pgfpathlineto{\pgfqpoint{3.354438in}{2.116235in}}%
\pgfpathlineto{\pgfqpoint{3.357062in}{1.175240in}}%
\pgfpathlineto{\pgfqpoint{3.357478in}{2.112895in}}%
\pgfpathlineto{\pgfqpoint{3.359517in}{1.217973in}}%
\pgfpathlineto{\pgfqpoint{3.361152in}{2.270385in}}%
\pgfpathlineto{\pgfqpoint{3.362277in}{1.279118in}}%
\pgfpathlineto{\pgfqpoint{3.363881in}{2.182229in}}%
\pgfpathlineto{\pgfqpoint{3.365220in}{1.263774in}}%
\pgfpathlineto{\pgfqpoint{3.367163in}{2.178063in}}%
\pgfpathlineto{\pgfqpoint{3.368281in}{1.224816in}}%
\pgfpathlineto{\pgfqpoint{3.369931in}{2.127351in}}%
\pgfpathlineto{\pgfqpoint{3.371437in}{1.270419in}}%
\pgfpathlineto{\pgfqpoint{3.373001in}{2.140455in}}%
\pgfpathlineto{\pgfqpoint{3.374999in}{1.225180in}}%
\pgfpathlineto{\pgfqpoint{3.376604in}{2.216261in}}%
\pgfpathlineto{\pgfqpoint{3.377619in}{1.317751in}}%
\pgfpathlineto{\pgfqpoint{3.380154in}{2.284326in}}%
\pgfpathlineto{\pgfqpoint{3.380930in}{1.261818in}}%
\pgfpathlineto{\pgfqpoint{3.382252in}{2.215775in}}%
\pgfpathlineto{\pgfqpoint{3.383764in}{1.315060in}}%
\pgfpathlineto{\pgfqpoint{3.385401in}{2.181995in}}%
\pgfpathlineto{\pgfqpoint{3.387053in}{1.247332in}}%
\pgfpathlineto{\pgfqpoint{3.388450in}{2.193227in}}%
\pgfpathlineto{\pgfqpoint{3.390138in}{1.237974in}}%
\pgfpathlineto{\pgfqpoint{3.391493in}{2.114437in}}%
\pgfpathlineto{\pgfqpoint{3.393173in}{1.217974in}}%
\pgfpathlineto{\pgfqpoint{3.394567in}{2.147258in}}%
\pgfpathlineto{\pgfqpoint{3.396227in}{1.330899in}}%
\pgfpathlineto{\pgfqpoint{3.397671in}{2.151158in}}%
\pgfpathlineto{\pgfqpoint{3.399667in}{1.143455in}}%
\pgfpathlineto{\pgfqpoint{3.401143in}{2.208645in}}%
\pgfpathlineto{\pgfqpoint{3.402374in}{1.327016in}}%
\pgfpathlineto{\pgfqpoint{3.404258in}{2.237920in}}%
\pgfpathlineto{\pgfqpoint{3.406362in}{1.174528in}}%
\pgfpathlineto{\pgfqpoint{3.406883in}{2.228827in}}%
\pgfpathlineto{\pgfqpoint{3.408731in}{1.159936in}}%
\pgfpathlineto{\pgfqpoint{3.409931in}{2.131716in}}%
\pgfpathlineto{\pgfqpoint{3.411978in}{1.229259in}}%
\pgfpathlineto{\pgfqpoint{3.413135in}{2.212295in}}%
\pgfpathlineto{\pgfqpoint{3.414564in}{1.211327in}}%
\pgfpathlineto{\pgfqpoint{3.416365in}{2.194441in}}%
\pgfpathlineto{\pgfqpoint{3.418202in}{1.215376in}}%
\pgfpathlineto{\pgfqpoint{3.419650in}{2.234813in}}%
\pgfpathlineto{\pgfqpoint{3.421171in}{1.216385in}}%
\pgfpathlineto{\pgfqpoint{3.422463in}{2.243183in}}%
\pgfpathlineto{\pgfqpoint{3.423889in}{1.294511in}}%
\pgfpathlineto{\pgfqpoint{3.425374in}{2.206291in}}%
\pgfpathlineto{\pgfqpoint{3.427972in}{1.144807in}}%
\pgfpathlineto{\pgfqpoint{3.428458in}{2.100797in}}%
\pgfpathlineto{\pgfqpoint{3.430323in}{1.282846in}}%
\pgfpathlineto{\pgfqpoint{3.431571in}{2.191456in}}%
\pgfpathlineto{\pgfqpoint{3.433226in}{1.277356in}}%
\pgfpathlineto{\pgfqpoint{3.434951in}{2.258356in}}%
\pgfpathlineto{\pgfqpoint{3.436900in}{1.204374in}}%
\pgfpathlineto{\pgfqpoint{3.437842in}{2.172370in}}%
\pgfpathlineto{\pgfqpoint{3.439328in}{1.298101in}}%
\pgfpathlineto{\pgfqpoint{3.440957in}{2.161347in}}%
\pgfpathlineto{\pgfqpoint{3.442648in}{1.227912in}}%
\pgfpathlineto{\pgfqpoint{3.443962in}{2.120698in}}%
\pgfpathlineto{\pgfqpoint{3.445417in}{1.257187in}}%
\pgfpathlineto{\pgfqpoint{3.447245in}{2.168657in}}%
\pgfpathlineto{\pgfqpoint{3.448946in}{1.216144in}}%
\pgfpathlineto{\pgfqpoint{3.450249in}{2.078741in}}%
\pgfpathlineto{\pgfqpoint{3.451631in}{1.272865in}}%
\pgfpathlineto{\pgfqpoint{3.453831in}{2.149700in}}%
\pgfpathlineto{\pgfqpoint{3.454838in}{1.256438in}}%
\pgfpathlineto{\pgfqpoint{3.456546in}{2.172501in}}%
\pgfpathlineto{\pgfqpoint{3.457944in}{1.197417in}}%
\pgfpathlineto{\pgfqpoint{3.459878in}{2.289394in}}%
\pgfpathlineto{\pgfqpoint{3.460894in}{1.303585in}}%
\pgfpathlineto{\pgfqpoint{3.462414in}{2.127552in}}%
\pgfpathlineto{\pgfqpoint{3.464053in}{1.288933in}}%
\pgfpathlineto{\pgfqpoint{3.465833in}{2.184823in}}%
\pgfpathlineto{\pgfqpoint{3.467254in}{1.288220in}}%
\pgfpathlineto{\pgfqpoint{3.468774in}{2.182156in}}%
\pgfpathlineto{\pgfqpoint{3.470417in}{1.251581in}}%
\pgfpathlineto{\pgfqpoint{3.471692in}{2.208406in}}%
\pgfpathlineto{\pgfqpoint{3.473674in}{1.156238in}}%
\pgfpathlineto{\pgfqpoint{3.474841in}{2.132466in}}%
\pgfpathlineto{\pgfqpoint{3.476343in}{1.231156in}}%
\pgfpathlineto{\pgfqpoint{3.478134in}{2.137827in}}%
\pgfpathlineto{\pgfqpoint{3.479411in}{1.180090in}}%
\pgfpathlineto{\pgfqpoint{3.481387in}{2.130541in}}%
\pgfpathlineto{\pgfqpoint{3.482619in}{1.248974in}}%
\pgfpathlineto{\pgfqpoint{3.484361in}{2.273776in}}%
\pgfpathlineto{\pgfqpoint{3.485702in}{1.138452in}}%
\pgfpathlineto{\pgfqpoint{3.487794in}{2.257243in}}%
\pgfpathlineto{\pgfqpoint{3.488744in}{1.200246in}}%
\pgfpathlineto{\pgfqpoint{3.490192in}{2.122417in}}%
\pgfpathlineto{\pgfqpoint{3.492819in}{1.086567in}}%
\pgfpathlineto{\pgfqpoint{3.493400in}{2.093165in}}%
\pgfpathlineto{\pgfqpoint{3.494884in}{1.191045in}}%
\pgfpathlineto{\pgfqpoint{3.496725in}{2.189203in}}%
\pgfpathlineto{\pgfqpoint{3.497895in}{1.232915in}}%
\pgfpathlineto{\pgfqpoint{3.499633in}{2.083492in}}%
\pgfpathlineto{\pgfqpoint{3.501083in}{1.196522in}}%
\pgfpathlineto{\pgfqpoint{3.502718in}{2.212286in}}%
\pgfpathlineto{\pgfqpoint{3.504927in}{1.116477in}}%
\pgfpathlineto{\pgfqpoint{3.505614in}{2.143232in}}%
\pgfpathlineto{\pgfqpoint{3.507500in}{1.220441in}}%
\pgfpathlineto{\pgfqpoint{3.508795in}{2.223174in}}%
\pgfpathlineto{\pgfqpoint{3.510393in}{1.173506in}}%
\pgfpathlineto{\pgfqpoint{3.511778in}{2.086554in}}%
\pgfpathlineto{\pgfqpoint{3.513633in}{1.225103in}}%
\pgfpathlineto{\pgfqpoint{3.515691in}{2.216949in}}%
\pgfpathlineto{\pgfqpoint{3.516425in}{1.276763in}}%
\pgfpathlineto{\pgfqpoint{3.518094in}{2.149265in}}%
\pgfpathlineto{\pgfqpoint{3.519528in}{1.201712in}}%
\pgfpathlineto{\pgfqpoint{3.521040in}{2.132251in}}%
\pgfpathlineto{\pgfqpoint{3.522567in}{1.199492in}}%
\pgfpathlineto{\pgfqpoint{3.524118in}{2.157260in}}%
\pgfpathlineto{\pgfqpoint{3.525711in}{1.270325in}}%
\pgfpathlineto{\pgfqpoint{3.527518in}{2.170475in}}%
\pgfpathlineto{\pgfqpoint{3.529073in}{1.132925in}}%
\pgfpathlineto{\pgfqpoint{3.530343in}{2.072264in}}%
\pgfpathlineto{\pgfqpoint{3.532064in}{1.105665in}}%
\pgfpathlineto{\pgfqpoint{3.533464in}{2.118974in}}%
\pgfpathlineto{\pgfqpoint{3.534918in}{1.273832in}}%
\pgfpathlineto{\pgfqpoint{3.536748in}{2.148208in}}%
\pgfpathlineto{\pgfqpoint{3.538222in}{1.075516in}}%
\pgfpathlineto{\pgfqpoint{3.539615in}{2.068260in}}%
\pgfpathlineto{\pgfqpoint{3.542425in}{1.020567in}}%
\pgfpathlineto{\pgfqpoint{3.542674in}{2.084593in}}%
\pgfpathlineto{\pgfqpoint{3.544320in}{1.201876in}}%
\pgfpathlineto{\pgfqpoint{3.545770in}{2.097300in}}%
\pgfpathlineto{\pgfqpoint{3.547269in}{1.209100in}}%
\pgfpathlineto{\pgfqpoint{3.549127in}{2.215438in}}%
\pgfpathlineto{\pgfqpoint{3.550472in}{1.204566in}}%
\pgfpathlineto{\pgfqpoint{3.552020in}{2.143549in}}%
\pgfpathlineto{\pgfqpoint{3.553825in}{1.101673in}}%
\pgfpathlineto{\pgfqpoint{3.555166in}{2.088656in}}%
\pgfpathlineto{\pgfqpoint{3.556639in}{1.067270in}}%
\pgfpathlineto{\pgfqpoint{3.558206in}{2.079854in}}%
\pgfpathlineto{\pgfqpoint{3.559751in}{1.105151in}}%
\pgfpathlineto{\pgfqpoint{3.561465in}{2.126558in}}%
\pgfpathlineto{\pgfqpoint{3.562784in}{1.240278in}}%
\pgfpathlineto{\pgfqpoint{3.564365in}{2.090211in}}%
\pgfpathlineto{\pgfqpoint{3.566667in}{1.131270in}}%
\pgfpathlineto{\pgfqpoint{3.567919in}{2.160818in}}%
\pgfpathlineto{\pgfqpoint{3.568985in}{1.258819in}}%
\pgfpathlineto{\pgfqpoint{3.570648in}{2.193182in}}%
\pgfpathlineto{\pgfqpoint{3.571993in}{1.145094in}}%
\pgfpathlineto{\pgfqpoint{3.574155in}{2.144345in}}%
\pgfpathlineto{\pgfqpoint{3.575186in}{1.198767in}}%
\pgfpathlineto{\pgfqpoint{3.576953in}{2.158293in}}%
\pgfpathlineto{\pgfqpoint{3.578513in}{1.123859in}}%
\pgfpathlineto{\pgfqpoint{3.580104in}{2.165945in}}%
\pgfpathlineto{\pgfqpoint{3.581307in}{1.139594in}}%
\pgfpathlineto{\pgfqpoint{3.584041in}{2.196323in}}%
\pgfpathlineto{\pgfqpoint{3.584322in}{1.204983in}}%
\pgfpathlineto{\pgfqpoint{3.586097in}{2.062818in}}%
\pgfpathlineto{\pgfqpoint{3.587900in}{1.119537in}}%
\pgfpathlineto{\pgfqpoint{3.588972in}{2.090156in}}%
\pgfpathlineto{\pgfqpoint{3.591070in}{1.054024in}}%
\pgfpathlineto{\pgfqpoint{3.592065in}{2.031656in}}%
\pgfpathlineto{\pgfqpoint{3.593824in}{1.187117in}}%
\pgfpathlineto{\pgfqpoint{3.595120in}{2.059725in}}%
\pgfpathlineto{\pgfqpoint{3.596903in}{1.154236in}}%
\pgfpathlineto{\pgfqpoint{3.598347in}{2.085266in}}%
\pgfpathlineto{\pgfqpoint{3.599770in}{1.257053in}}%
\pgfpathlineto{\pgfqpoint{3.601404in}{2.106825in}}%
\pgfpathlineto{\pgfqpoint{3.604164in}{1.113726in}}%
\pgfpathlineto{\pgfqpoint{3.604370in}{2.224679in}}%
\pgfpathlineto{\pgfqpoint{3.605979in}{1.233942in}}%
\pgfpathlineto{\pgfqpoint{3.607553in}{2.105859in}}%
\pgfpathlineto{\pgfqpoint{3.609025in}{1.194925in}}%
\pgfpathlineto{\pgfqpoint{3.610851in}{2.130587in}}%
\pgfpathlineto{\pgfqpoint{3.612071in}{1.210090in}}%
\pgfpathlineto{\pgfqpoint{3.613885in}{2.069979in}}%
\pgfpathlineto{\pgfqpoint{3.615453in}{1.199818in}}%
\pgfpathlineto{\pgfqpoint{3.617333in}{2.198523in}}%
\pgfpathlineto{\pgfqpoint{3.618283in}{1.227399in}}%
\pgfpathlineto{\pgfqpoint{3.620166in}{2.168817in}}%
\pgfpathlineto{\pgfqpoint{3.621510in}{1.238684in}}%
\pgfpathlineto{\pgfqpoint{3.623365in}{2.216508in}}%
\pgfpathlineto{\pgfqpoint{3.624437in}{1.247662in}}%
\pgfpathlineto{\pgfqpoint{3.626222in}{2.124850in}}%
\pgfpathlineto{\pgfqpoint{3.627729in}{1.250554in}}%
\pgfpathlineto{\pgfqpoint{3.629036in}{2.124389in}}%
\pgfpathlineto{\pgfqpoint{3.630901in}{1.166975in}}%
\pgfpathlineto{\pgfqpoint{3.632191in}{2.143852in}}%
\pgfpathlineto{\pgfqpoint{3.633720in}{1.296588in}}%
\pgfpathlineto{\pgfqpoint{3.635238in}{2.150514in}}%
\pgfpathlineto{\pgfqpoint{3.636799in}{1.271535in}}%
\pgfpathlineto{\pgfqpoint{3.638401in}{2.131484in}}%
\pgfpathlineto{\pgfqpoint{3.639823in}{1.260539in}}%
\pgfpathlineto{\pgfqpoint{3.641427in}{2.185503in}}%
\pgfpathlineto{\pgfqpoint{3.643116in}{1.245872in}}%
\pgfpathlineto{\pgfqpoint{3.644732in}{2.250836in}}%
\pgfpathlineto{\pgfqpoint{3.646468in}{1.243274in}}%
\pgfpathlineto{\pgfqpoint{3.647822in}{2.146320in}}%
\pgfpathlineto{\pgfqpoint{3.649593in}{1.192826in}}%
\pgfpathlineto{\pgfqpoint{3.650891in}{2.219665in}}%
\pgfpathlineto{\pgfqpoint{3.652596in}{1.145366in}}%
\pgfpathlineto{\pgfqpoint{3.654165in}{2.206411in}}%
\pgfpathlineto{\pgfqpoint{3.655622in}{1.240680in}}%
\pgfpathlineto{\pgfqpoint{3.656796in}{2.193100in}}%
\pgfpathlineto{\pgfqpoint{3.658387in}{1.265000in}}%
\pgfpathlineto{\pgfqpoint{3.659957in}{2.171000in}}%
\pgfpathlineto{\pgfqpoint{3.661452in}{1.284243in}}%
\pgfpathlineto{\pgfqpoint{3.663305in}{2.215221in}}%
\pgfpathlineto{\pgfqpoint{3.664600in}{1.223046in}}%
\pgfpathlineto{\pgfqpoint{3.666337in}{2.203275in}}%
\pgfpathlineto{\pgfqpoint{3.667707in}{1.297984in}}%
\pgfpathlineto{\pgfqpoint{3.669273in}{2.207036in}}%
\pgfpathlineto{\pgfqpoint{3.671029in}{1.248641in}}%
\pgfpathlineto{\pgfqpoint{3.672260in}{2.113614in}}%
\pgfpathlineto{\pgfqpoint{3.674491in}{1.332680in}}%
\pgfpathlineto{\pgfqpoint{3.675519in}{2.283421in}}%
\pgfpathlineto{\pgfqpoint{3.677058in}{1.351097in}}%
\pgfpathlineto{\pgfqpoint{3.678500in}{2.260639in}}%
\pgfpathlineto{\pgfqpoint{3.679994in}{1.365739in}}%
\pgfpathlineto{\pgfqpoint{3.681498in}{2.341004in}}%
\pgfpathlineto{\pgfqpoint{3.683246in}{1.304115in}}%
\pgfpathlineto{\pgfqpoint{3.685107in}{2.243846in}}%
\pgfpathlineto{\pgfqpoint{3.686349in}{1.325593in}}%
\pgfpathlineto{\pgfqpoint{3.688324in}{2.325406in}}%
\pgfpathlineto{\pgfqpoint{3.689997in}{1.281763in}}%
\pgfpathlineto{\pgfqpoint{3.690753in}{2.254173in}}%
\pgfpathlineto{\pgfqpoint{3.692270in}{1.396748in}}%
\pgfpathlineto{\pgfqpoint{3.694009in}{2.247902in}}%
\pgfpathlineto{\pgfqpoint{3.695460in}{1.346674in}}%
\pgfpathlineto{\pgfqpoint{3.696915in}{2.260929in}}%
\pgfpathlineto{\pgfqpoint{3.698512in}{1.390246in}}%
\pgfpathlineto{\pgfqpoint{3.700032in}{2.283492in}}%
\pgfpathlineto{\pgfqpoint{3.701637in}{1.438115in}}%
\pgfpathlineto{\pgfqpoint{3.703198in}{2.315154in}}%
\pgfpathlineto{\pgfqpoint{3.705383in}{1.294145in}}%
\pgfpathlineto{\pgfqpoint{3.706238in}{2.260867in}}%
\pgfpathlineto{\pgfqpoint{3.707725in}{1.409605in}}%
\pgfpathlineto{\pgfqpoint{3.709290in}{2.303221in}}%
\pgfpathlineto{\pgfqpoint{3.711364in}{1.317316in}}%
\pgfpathlineto{\pgfqpoint{3.712398in}{2.237371in}}%
\pgfpathlineto{\pgfqpoint{3.714309in}{1.368175in}}%
\pgfpathlineto{\pgfqpoint{3.715528in}{2.258171in}}%
\pgfpathlineto{\pgfqpoint{3.717020in}{1.387429in}}%
\pgfpathlineto{\pgfqpoint{3.718555in}{2.235371in}}%
\pgfpathlineto{\pgfqpoint{3.720158in}{1.299212in}}%
\pgfpathlineto{\pgfqpoint{3.722264in}{2.309873in}}%
\pgfpathlineto{\pgfqpoint{3.723209in}{1.345975in}}%
\pgfpathlineto{\pgfqpoint{3.724964in}{2.284701in}}%
\pgfpathlineto{\pgfqpoint{3.726632in}{1.351236in}}%
\pgfpathlineto{\pgfqpoint{3.728181in}{2.328627in}}%
\pgfpathlineto{\pgfqpoint{3.729875in}{1.314531in}}%
\pgfpathlineto{\pgfqpoint{3.731150in}{2.362166in}}%
\pgfpathlineto{\pgfqpoint{3.732501in}{1.374891in}}%
\pgfpathlineto{\pgfqpoint{3.733974in}{2.325569in}}%
\pgfpathlineto{\pgfqpoint{3.735501in}{1.375365in}}%
\pgfpathlineto{\pgfqpoint{3.737172in}{2.243686in}}%
\pgfpathlineto{\pgfqpoint{3.738777in}{1.423612in}}%
\pgfpathlineto{\pgfqpoint{3.740293in}{2.248182in}}%
\pgfpathlineto{\pgfqpoint{3.741845in}{1.406947in}}%
\pgfpathlineto{\pgfqpoint{3.743417in}{2.304008in}}%
\pgfpathlineto{\pgfqpoint{3.744745in}{1.413105in}}%
\pgfpathlineto{\pgfqpoint{3.746380in}{2.262543in}}%
\pgfpathlineto{\pgfqpoint{3.747920in}{1.391557in}}%
\pgfpathlineto{\pgfqpoint{3.749831in}{2.234174in}}%
\pgfpathlineto{\pgfqpoint{3.751285in}{1.333978in}}%
\pgfpathlineto{\pgfqpoint{3.752663in}{2.319332in}}%
\pgfpathlineto{\pgfqpoint{3.754204in}{1.387385in}}%
\pgfpathlineto{\pgfqpoint{3.755621in}{2.334575in}}%
\pgfpathlineto{\pgfqpoint{3.757096in}{1.380500in}}%
\pgfpathlineto{\pgfqpoint{3.758925in}{2.247400in}}%
\pgfpathlineto{\pgfqpoint{3.760516in}{1.350836in}}%
\pgfpathlineto{\pgfqpoint{3.762400in}{2.375063in}}%
\pgfpathlineto{\pgfqpoint{3.764150in}{1.297406in}}%
\pgfpathlineto{\pgfqpoint{3.764942in}{2.232505in}}%
\pgfpathlineto{\pgfqpoint{3.766445in}{1.427648in}}%
\pgfpathlineto{\pgfqpoint{3.767884in}{2.273797in}}%
\pgfpathlineto{\pgfqpoint{3.770039in}{1.280323in}}%
\pgfpathlineto{\pgfqpoint{3.770964in}{2.239234in}}%
\pgfpathlineto{\pgfqpoint{3.773323in}{1.300100in}}%
\pgfpathlineto{\pgfqpoint{3.774046in}{2.153522in}}%
\pgfpathlineto{\pgfqpoint{3.775840in}{1.352354in}}%
\pgfpathlineto{\pgfqpoint{3.777990in}{2.330104in}}%
\pgfpathlineto{\pgfqpoint{3.779192in}{1.293141in}}%
\pgfpathlineto{\pgfqpoint{3.780325in}{2.213110in}}%
\pgfpathlineto{\pgfqpoint{3.782288in}{1.169441in}}%
\pgfpathlineto{\pgfqpoint{3.783357in}{2.109297in}}%
\pgfpathlineto{\pgfqpoint{3.785080in}{1.247674in}}%
\pgfpathlineto{\pgfqpoint{3.786502in}{2.195388in}}%
\pgfpathlineto{\pgfqpoint{3.788356in}{1.221052in}}%
\pgfpathlineto{\pgfqpoint{3.789500in}{2.160060in}}%
\pgfpathlineto{\pgfqpoint{3.791076in}{1.239776in}}%
\pgfpathlineto{\pgfqpoint{3.792573in}{2.121095in}}%
\pgfpathlineto{\pgfqpoint{3.794138in}{1.311398in}}%
\pgfpathlineto{\pgfqpoint{3.795712in}{2.154571in}}%
\pgfpathlineto{\pgfqpoint{3.797613in}{1.258777in}}%
\pgfpathlineto{\pgfqpoint{3.798901in}{2.241410in}}%
\pgfpathlineto{\pgfqpoint{3.800684in}{1.265534in}}%
\pgfpathlineto{\pgfqpoint{3.802008in}{2.192406in}}%
\pgfpathlineto{\pgfqpoint{3.803516in}{1.297792in}}%
\pgfpathlineto{\pgfqpoint{3.804979in}{2.186403in}}%
\pgfpathlineto{\pgfqpoint{3.806621in}{1.298804in}}%
\pgfpathlineto{\pgfqpoint{3.808921in}{2.240051in}}%
\pgfpathlineto{\pgfqpoint{3.809807in}{1.263924in}}%
\pgfpathlineto{\pgfqpoint{3.811847in}{2.370877in}}%
\pgfpathlineto{\pgfqpoint{3.812770in}{1.283106in}}%
\pgfpathlineto{\pgfqpoint{3.814840in}{2.339935in}}%
\pgfpathlineto{\pgfqpoint{3.815814in}{1.287909in}}%
\pgfpathlineto{\pgfqpoint{3.817788in}{2.249330in}}%
\pgfpathlineto{\pgfqpoint{3.818806in}{1.372487in}}%
\pgfpathlineto{\pgfqpoint{3.820516in}{2.222409in}}%
\pgfpathlineto{\pgfqpoint{3.821935in}{1.317560in}}%
\pgfpathlineto{\pgfqpoint{3.823622in}{2.210711in}}%
\pgfpathlineto{\pgfqpoint{3.824968in}{1.338661in}}%
\pgfpathlineto{\pgfqpoint{3.826498in}{1.586163in}}%
\pgfpathlineto{\pgfqpoint{3.826498in}{1.586163in}}%
\pgfusepath{stroke}%
\end{pgfscope}%
\begin{pgfscope}%
\pgfsetrectcap%
\pgfsetmiterjoin%
\pgfsetlinewidth{0.803000pt}%
\definecolor{currentstroke}{rgb}{0.000000,0.000000,0.000000}%
\pgfsetstrokecolor{currentstroke}%
\pgfsetdash{}{0pt}%
\pgfpathmoveto{\pgfqpoint{0.589745in}{0.416448in}}%
\pgfpathlineto{\pgfqpoint{0.589745in}{2.468330in}}%
\pgfusepath{stroke}%
\end{pgfscope}%
\begin{pgfscope}%
\pgfsetrectcap%
\pgfsetmiterjoin%
\pgfsetlinewidth{0.803000pt}%
\definecolor{currentstroke}{rgb}{0.000000,0.000000,0.000000}%
\pgfsetstrokecolor{currentstroke}%
\pgfsetdash{}{0pt}%
\pgfpathmoveto{\pgfqpoint{3.980629in}{0.416448in}}%
\pgfpathlineto{\pgfqpoint{3.980629in}{2.468330in}}%
\pgfusepath{stroke}%
\end{pgfscope}%
\begin{pgfscope}%
\pgfsetrectcap%
\pgfsetmiterjoin%
\pgfsetlinewidth{0.803000pt}%
\definecolor{currentstroke}{rgb}{0.000000,0.000000,0.000000}%
\pgfsetstrokecolor{currentstroke}%
\pgfsetdash{}{0pt}%
\pgfpathmoveto{\pgfqpoint{0.589745in}{0.416447in}}%
\pgfpathlineto{\pgfqpoint{3.980629in}{0.416447in}}%
\pgfusepath{stroke}%
\end{pgfscope}%
\begin{pgfscope}%
\pgfsetrectcap%
\pgfsetmiterjoin%
\pgfsetlinewidth{0.803000pt}%
\definecolor{currentstroke}{rgb}{0.000000,0.000000,0.000000}%
\pgfsetstrokecolor{currentstroke}%
\pgfsetdash{}{0pt}%
\pgfpathmoveto{\pgfqpoint{0.589745in}{2.468330in}}%
\pgfpathlineto{\pgfqpoint{3.980629in}{2.468330in}}%
\pgfusepath{stroke}%
\end{pgfscope}%
\end{pgfpicture}%
\makeatother%
\endgroup%

    \caption{A simulated time series containing white noise, flicker noise and random walk behaviour.}
    \label{fig:adev_example_time}
\end{figure}

A common approach to identify noise sources is the power spectrum. It is easily accessible, even in real-time, using spectrum analyzers and, utilizing the computational power of modern computers, large time-domain data sets can be converted making this the method of choice in the lab. The power spectrum of figure \ref{fig:adev_example_time} is shown in figure \ref{fig:adev_example_psd}. It allows to clearly separate the white noise part from the other $f^{\alpha}$ components. The dashed lines representing the individual components were plotted using the $h_\alpha$ values calculated from the input parameters of the simulation. The noise spectral density $h_0$ of the white noise signal can be easily extracted even by hand without resorting to a fit. This yields $h_{0} = \qty{2e-3}{\per \Hz} $. $h_{-1}$ and $h_{-2}$ can be exctracted as well using a fit to
\begin{equation}
    S(f) = \sum_{\alpha = -2}^0 h_\alpha f^\alpha \, .
\end{equation}
The noise corner frequency $f_c$ can either be calculated from $h_0$ and $h_{-1}$ using equation \ref{eqn:corner_frequency} or determined graphically by contructing a tangent with a slope of $-1$ to the spectral density. From the intersection of the blue $h_0$ line and the green $h_{-1}$ line the corner frequncy is found to be $f_c \approx \qty{1.8}{\kHz}$.

\begin{figure}[hb]
    \centering
    %% Creator: Matplotlib, PGF backend
%%
%% To include the figure in your LaTeX document, write
%%   \input{<filename>.pgf}
%%
%% Make sure the required packages are loaded in your preamble
%%   \usepackage{pgf}
%%
%% Also ensure that all the required font packages are loaded; for instance,
%% the lmodern package is sometimes necessary when using math font.
%%   \usepackage{lmodern}
%%
%% Figures using additional raster images can only be included by \input if
%% they are in the same directory as the main LaTeX file. For loading figures
%% from other directories you can use the `import` package
%%   \usepackage{import}
%%
%% and then include the figures with
%%   \import{<path to file>}{<filename>.pgf}
%%
%% Matplotlib used the following preamble
%%   \usepackage{siunitx}
%%   \usepackage{fontspec}
%%   \makeatletter\@ifpackageloaded{underscore}{}{\usepackage[strings]{underscore}}\makeatother
%%
\begingroup%
\makeatletter%
\begin{pgfpicture}%
\pgfpathrectangle{\pgfpointorigin}{\pgfqpoint{4.060000in}{2.510000in}}%
\pgfusepath{use as bounding box, clip}%
\begin{pgfscope}%
\pgfsetbuttcap%
\pgfsetmiterjoin%
\definecolor{currentfill}{rgb}{1.000000,1.000000,1.000000}%
\pgfsetfillcolor{currentfill}%
\pgfsetlinewidth{0.000000pt}%
\definecolor{currentstroke}{rgb}{1.000000,1.000000,1.000000}%
\pgfsetstrokecolor{currentstroke}%
\pgfsetdash{}{0pt}%
\pgfpathmoveto{\pgfqpoint{0.000000in}{0.000000in}}%
\pgfpathlineto{\pgfqpoint{4.060000in}{0.000000in}}%
\pgfpathlineto{\pgfqpoint{4.060000in}{2.510000in}}%
\pgfpathlineto{\pgfqpoint{0.000000in}{2.510000in}}%
\pgfpathlineto{\pgfqpoint{0.000000in}{0.000000in}}%
\pgfpathclose%
\pgfusepath{fill}%
\end{pgfscope}%
\begin{pgfscope}%
\pgfsetbuttcap%
\pgfsetmiterjoin%
\definecolor{currentfill}{rgb}{1.000000,1.000000,1.000000}%
\pgfsetfillcolor{currentfill}%
\pgfsetlinewidth{0.000000pt}%
\definecolor{currentstroke}{rgb}{0.000000,0.000000,0.000000}%
\pgfsetstrokecolor{currentstroke}%
\pgfsetstrokeopacity{0.000000}%
\pgfsetdash{}{0pt}%
\pgfpathmoveto{\pgfqpoint{0.594525in}{0.417642in}}%
\pgfpathlineto{\pgfqpoint{3.940488in}{0.417642in}}%
\pgfpathlineto{\pgfqpoint{3.940488in}{2.468330in}}%
\pgfpathlineto{\pgfqpoint{0.594525in}{2.468330in}}%
\pgfpathlineto{\pgfqpoint{0.594525in}{0.417642in}}%
\pgfpathclose%
\pgfusepath{fill}%
\end{pgfscope}%
\begin{pgfscope}%
\pgfpathrectangle{\pgfqpoint{0.594525in}{0.417642in}}{\pgfqpoint{3.345963in}{2.050688in}}%
\pgfusepath{clip}%
\pgfsetrectcap%
\pgfsetroundjoin%
\pgfsetlinewidth{0.803000pt}%
\definecolor{currentstroke}{rgb}{0.450000,0.450000,0.450000}%
\pgfsetstrokecolor{currentstroke}%
\pgfsetdash{}{0pt}%
\pgfpathmoveto{\pgfqpoint{0.711524in}{0.417642in}}%
\pgfpathlineto{\pgfqpoint{0.711524in}{2.468330in}}%
\pgfusepath{stroke}%
\end{pgfscope}%
\begin{pgfscope}%
\pgfsetbuttcap%
\pgfsetroundjoin%
\definecolor{currentfill}{rgb}{0.000000,0.000000,0.000000}%
\pgfsetfillcolor{currentfill}%
\pgfsetlinewidth{0.803000pt}%
\definecolor{currentstroke}{rgb}{0.000000,0.000000,0.000000}%
\pgfsetstrokecolor{currentstroke}%
\pgfsetdash{}{0pt}%
\pgfsys@defobject{currentmarker}{\pgfqpoint{0.000000in}{-0.048611in}}{\pgfqpoint{0.000000in}{0.000000in}}{%
\pgfpathmoveto{\pgfqpoint{0.000000in}{0.000000in}}%
\pgfpathlineto{\pgfqpoint{0.000000in}{-0.048611in}}%
\pgfusepath{stroke,fill}%
}%
\begin{pgfscope}%
\pgfsys@transformshift{0.711524in}{0.417642in}%
\pgfsys@useobject{currentmarker}{}%
\end{pgfscope}%
\end{pgfscope}%
\begin{pgfscope}%
\definecolor{textcolor}{rgb}{0.000000,0.000000,0.000000}%
\pgfsetstrokecolor{textcolor}%
\pgfsetfillcolor{textcolor}%
\pgftext[x=0.711524in,y=0.320420in,,top]{\color{textcolor}\rmfamily\fontsize{8.000000}{9.600000}\selectfont \(\displaystyle {10^{-1}}\)}%
\end{pgfscope}%
\begin{pgfscope}%
\pgfpathrectangle{\pgfqpoint{0.594525in}{0.417642in}}{\pgfqpoint{3.345963in}{2.050688in}}%
\pgfusepath{clip}%
\pgfsetrectcap%
\pgfsetroundjoin%
\pgfsetlinewidth{0.803000pt}%
\definecolor{currentstroke}{rgb}{0.450000,0.450000,0.450000}%
\pgfsetstrokecolor{currentstroke}%
\pgfsetdash{}{0pt}%
\pgfpathmoveto{\pgfqpoint{1.171359in}{0.417642in}}%
\pgfpathlineto{\pgfqpoint{1.171359in}{2.468330in}}%
\pgfusepath{stroke}%
\end{pgfscope}%
\begin{pgfscope}%
\pgfsetbuttcap%
\pgfsetroundjoin%
\definecolor{currentfill}{rgb}{0.000000,0.000000,0.000000}%
\pgfsetfillcolor{currentfill}%
\pgfsetlinewidth{0.803000pt}%
\definecolor{currentstroke}{rgb}{0.000000,0.000000,0.000000}%
\pgfsetstrokecolor{currentstroke}%
\pgfsetdash{}{0pt}%
\pgfsys@defobject{currentmarker}{\pgfqpoint{0.000000in}{-0.048611in}}{\pgfqpoint{0.000000in}{0.000000in}}{%
\pgfpathmoveto{\pgfqpoint{0.000000in}{0.000000in}}%
\pgfpathlineto{\pgfqpoint{0.000000in}{-0.048611in}}%
\pgfusepath{stroke,fill}%
}%
\begin{pgfscope}%
\pgfsys@transformshift{1.171359in}{0.417642in}%
\pgfsys@useobject{currentmarker}{}%
\end{pgfscope}%
\end{pgfscope}%
\begin{pgfscope}%
\definecolor{textcolor}{rgb}{0.000000,0.000000,0.000000}%
\pgfsetstrokecolor{textcolor}%
\pgfsetfillcolor{textcolor}%
\pgftext[x=1.171359in,y=0.320420in,,top]{\color{textcolor}\rmfamily\fontsize{8.000000}{9.600000}\selectfont \(\displaystyle {10^{0}}\)}%
\end{pgfscope}%
\begin{pgfscope}%
\pgfpathrectangle{\pgfqpoint{0.594525in}{0.417642in}}{\pgfqpoint{3.345963in}{2.050688in}}%
\pgfusepath{clip}%
\pgfsetrectcap%
\pgfsetroundjoin%
\pgfsetlinewidth{0.803000pt}%
\definecolor{currentstroke}{rgb}{0.450000,0.450000,0.450000}%
\pgfsetstrokecolor{currentstroke}%
\pgfsetdash{}{0pt}%
\pgfpathmoveto{\pgfqpoint{1.631193in}{0.417642in}}%
\pgfpathlineto{\pgfqpoint{1.631193in}{2.468330in}}%
\pgfusepath{stroke}%
\end{pgfscope}%
\begin{pgfscope}%
\pgfsetbuttcap%
\pgfsetroundjoin%
\definecolor{currentfill}{rgb}{0.000000,0.000000,0.000000}%
\pgfsetfillcolor{currentfill}%
\pgfsetlinewidth{0.803000pt}%
\definecolor{currentstroke}{rgb}{0.000000,0.000000,0.000000}%
\pgfsetstrokecolor{currentstroke}%
\pgfsetdash{}{0pt}%
\pgfsys@defobject{currentmarker}{\pgfqpoint{0.000000in}{-0.048611in}}{\pgfqpoint{0.000000in}{0.000000in}}{%
\pgfpathmoveto{\pgfqpoint{0.000000in}{0.000000in}}%
\pgfpathlineto{\pgfqpoint{0.000000in}{-0.048611in}}%
\pgfusepath{stroke,fill}%
}%
\begin{pgfscope}%
\pgfsys@transformshift{1.631193in}{0.417642in}%
\pgfsys@useobject{currentmarker}{}%
\end{pgfscope}%
\end{pgfscope}%
\begin{pgfscope}%
\definecolor{textcolor}{rgb}{0.000000,0.000000,0.000000}%
\pgfsetstrokecolor{textcolor}%
\pgfsetfillcolor{textcolor}%
\pgftext[x=1.631193in,y=0.320420in,,top]{\color{textcolor}\rmfamily\fontsize{8.000000}{9.600000}\selectfont \(\displaystyle {10^{1}}\)}%
\end{pgfscope}%
\begin{pgfscope}%
\pgfpathrectangle{\pgfqpoint{0.594525in}{0.417642in}}{\pgfqpoint{3.345963in}{2.050688in}}%
\pgfusepath{clip}%
\pgfsetrectcap%
\pgfsetroundjoin%
\pgfsetlinewidth{0.803000pt}%
\definecolor{currentstroke}{rgb}{0.450000,0.450000,0.450000}%
\pgfsetstrokecolor{currentstroke}%
\pgfsetdash{}{0pt}%
\pgfpathmoveto{\pgfqpoint{2.091028in}{0.417642in}}%
\pgfpathlineto{\pgfqpoint{2.091028in}{2.468330in}}%
\pgfusepath{stroke}%
\end{pgfscope}%
\begin{pgfscope}%
\pgfsetbuttcap%
\pgfsetroundjoin%
\definecolor{currentfill}{rgb}{0.000000,0.000000,0.000000}%
\pgfsetfillcolor{currentfill}%
\pgfsetlinewidth{0.803000pt}%
\definecolor{currentstroke}{rgb}{0.000000,0.000000,0.000000}%
\pgfsetstrokecolor{currentstroke}%
\pgfsetdash{}{0pt}%
\pgfsys@defobject{currentmarker}{\pgfqpoint{0.000000in}{-0.048611in}}{\pgfqpoint{0.000000in}{0.000000in}}{%
\pgfpathmoveto{\pgfqpoint{0.000000in}{0.000000in}}%
\pgfpathlineto{\pgfqpoint{0.000000in}{-0.048611in}}%
\pgfusepath{stroke,fill}%
}%
\begin{pgfscope}%
\pgfsys@transformshift{2.091028in}{0.417642in}%
\pgfsys@useobject{currentmarker}{}%
\end{pgfscope}%
\end{pgfscope}%
\begin{pgfscope}%
\definecolor{textcolor}{rgb}{0.000000,0.000000,0.000000}%
\pgfsetstrokecolor{textcolor}%
\pgfsetfillcolor{textcolor}%
\pgftext[x=2.091028in,y=0.320420in,,top]{\color{textcolor}\rmfamily\fontsize{8.000000}{9.600000}\selectfont \(\displaystyle {10^{2}}\)}%
\end{pgfscope}%
\begin{pgfscope}%
\pgfpathrectangle{\pgfqpoint{0.594525in}{0.417642in}}{\pgfqpoint{3.345963in}{2.050688in}}%
\pgfusepath{clip}%
\pgfsetrectcap%
\pgfsetroundjoin%
\pgfsetlinewidth{0.803000pt}%
\definecolor{currentstroke}{rgb}{0.450000,0.450000,0.450000}%
\pgfsetstrokecolor{currentstroke}%
\pgfsetdash{}{0pt}%
\pgfpathmoveto{\pgfqpoint{2.550863in}{0.417642in}}%
\pgfpathlineto{\pgfqpoint{2.550863in}{2.468330in}}%
\pgfusepath{stroke}%
\end{pgfscope}%
\begin{pgfscope}%
\pgfsetbuttcap%
\pgfsetroundjoin%
\definecolor{currentfill}{rgb}{0.000000,0.000000,0.000000}%
\pgfsetfillcolor{currentfill}%
\pgfsetlinewidth{0.803000pt}%
\definecolor{currentstroke}{rgb}{0.000000,0.000000,0.000000}%
\pgfsetstrokecolor{currentstroke}%
\pgfsetdash{}{0pt}%
\pgfsys@defobject{currentmarker}{\pgfqpoint{0.000000in}{-0.048611in}}{\pgfqpoint{0.000000in}{0.000000in}}{%
\pgfpathmoveto{\pgfqpoint{0.000000in}{0.000000in}}%
\pgfpathlineto{\pgfqpoint{0.000000in}{-0.048611in}}%
\pgfusepath{stroke,fill}%
}%
\begin{pgfscope}%
\pgfsys@transformshift{2.550863in}{0.417642in}%
\pgfsys@useobject{currentmarker}{}%
\end{pgfscope}%
\end{pgfscope}%
\begin{pgfscope}%
\definecolor{textcolor}{rgb}{0.000000,0.000000,0.000000}%
\pgfsetstrokecolor{textcolor}%
\pgfsetfillcolor{textcolor}%
\pgftext[x=2.550863in,y=0.320420in,,top]{\color{textcolor}\rmfamily\fontsize{8.000000}{9.600000}\selectfont \(\displaystyle {10^{3}}\)}%
\end{pgfscope}%
\begin{pgfscope}%
\pgfpathrectangle{\pgfqpoint{0.594525in}{0.417642in}}{\pgfqpoint{3.345963in}{2.050688in}}%
\pgfusepath{clip}%
\pgfsetrectcap%
\pgfsetroundjoin%
\pgfsetlinewidth{0.803000pt}%
\definecolor{currentstroke}{rgb}{0.450000,0.450000,0.450000}%
\pgfsetstrokecolor{currentstroke}%
\pgfsetdash{}{0pt}%
\pgfpathmoveto{\pgfqpoint{3.010697in}{0.417642in}}%
\pgfpathlineto{\pgfqpoint{3.010697in}{2.468330in}}%
\pgfusepath{stroke}%
\end{pgfscope}%
\begin{pgfscope}%
\pgfsetbuttcap%
\pgfsetroundjoin%
\definecolor{currentfill}{rgb}{0.000000,0.000000,0.000000}%
\pgfsetfillcolor{currentfill}%
\pgfsetlinewidth{0.803000pt}%
\definecolor{currentstroke}{rgb}{0.000000,0.000000,0.000000}%
\pgfsetstrokecolor{currentstroke}%
\pgfsetdash{}{0pt}%
\pgfsys@defobject{currentmarker}{\pgfqpoint{0.000000in}{-0.048611in}}{\pgfqpoint{0.000000in}{0.000000in}}{%
\pgfpathmoveto{\pgfqpoint{0.000000in}{0.000000in}}%
\pgfpathlineto{\pgfqpoint{0.000000in}{-0.048611in}}%
\pgfusepath{stroke,fill}%
}%
\begin{pgfscope}%
\pgfsys@transformshift{3.010697in}{0.417642in}%
\pgfsys@useobject{currentmarker}{}%
\end{pgfscope}%
\end{pgfscope}%
\begin{pgfscope}%
\definecolor{textcolor}{rgb}{0.000000,0.000000,0.000000}%
\pgfsetstrokecolor{textcolor}%
\pgfsetfillcolor{textcolor}%
\pgftext[x=3.010697in,y=0.320420in,,top]{\color{textcolor}\rmfamily\fontsize{8.000000}{9.600000}\selectfont \(\displaystyle {10^{4}}\)}%
\end{pgfscope}%
\begin{pgfscope}%
\pgfpathrectangle{\pgfqpoint{0.594525in}{0.417642in}}{\pgfqpoint{3.345963in}{2.050688in}}%
\pgfusepath{clip}%
\pgfsetrectcap%
\pgfsetroundjoin%
\pgfsetlinewidth{0.803000pt}%
\definecolor{currentstroke}{rgb}{0.450000,0.450000,0.450000}%
\pgfsetstrokecolor{currentstroke}%
\pgfsetdash{}{0pt}%
\pgfpathmoveto{\pgfqpoint{3.470532in}{0.417642in}}%
\pgfpathlineto{\pgfqpoint{3.470532in}{2.468330in}}%
\pgfusepath{stroke}%
\end{pgfscope}%
\begin{pgfscope}%
\pgfsetbuttcap%
\pgfsetroundjoin%
\definecolor{currentfill}{rgb}{0.000000,0.000000,0.000000}%
\pgfsetfillcolor{currentfill}%
\pgfsetlinewidth{0.803000pt}%
\definecolor{currentstroke}{rgb}{0.000000,0.000000,0.000000}%
\pgfsetstrokecolor{currentstroke}%
\pgfsetdash{}{0pt}%
\pgfsys@defobject{currentmarker}{\pgfqpoint{0.000000in}{-0.048611in}}{\pgfqpoint{0.000000in}{0.000000in}}{%
\pgfpathmoveto{\pgfqpoint{0.000000in}{0.000000in}}%
\pgfpathlineto{\pgfqpoint{0.000000in}{-0.048611in}}%
\pgfusepath{stroke,fill}%
}%
\begin{pgfscope}%
\pgfsys@transformshift{3.470532in}{0.417642in}%
\pgfsys@useobject{currentmarker}{}%
\end{pgfscope}%
\end{pgfscope}%
\begin{pgfscope}%
\definecolor{textcolor}{rgb}{0.000000,0.000000,0.000000}%
\pgfsetstrokecolor{textcolor}%
\pgfsetfillcolor{textcolor}%
\pgftext[x=3.470532in,y=0.320420in,,top]{\color{textcolor}\rmfamily\fontsize{8.000000}{9.600000}\selectfont \(\displaystyle {10^{5}}\)}%
\end{pgfscope}%
\begin{pgfscope}%
\pgfpathrectangle{\pgfqpoint{0.594525in}{0.417642in}}{\pgfqpoint{3.345963in}{2.050688in}}%
\pgfusepath{clip}%
\pgfsetrectcap%
\pgfsetroundjoin%
\pgfsetlinewidth{0.803000pt}%
\definecolor{currentstroke}{rgb}{0.450000,0.450000,0.450000}%
\pgfsetstrokecolor{currentstroke}%
\pgfsetdash{}{0pt}%
\pgfpathmoveto{\pgfqpoint{3.930367in}{0.417642in}}%
\pgfpathlineto{\pgfqpoint{3.930367in}{2.468330in}}%
\pgfusepath{stroke}%
\end{pgfscope}%
\begin{pgfscope}%
\pgfsetbuttcap%
\pgfsetroundjoin%
\definecolor{currentfill}{rgb}{0.000000,0.000000,0.000000}%
\pgfsetfillcolor{currentfill}%
\pgfsetlinewidth{0.803000pt}%
\definecolor{currentstroke}{rgb}{0.000000,0.000000,0.000000}%
\pgfsetstrokecolor{currentstroke}%
\pgfsetdash{}{0pt}%
\pgfsys@defobject{currentmarker}{\pgfqpoint{0.000000in}{-0.048611in}}{\pgfqpoint{0.000000in}{0.000000in}}{%
\pgfpathmoveto{\pgfqpoint{0.000000in}{0.000000in}}%
\pgfpathlineto{\pgfqpoint{0.000000in}{-0.048611in}}%
\pgfusepath{stroke,fill}%
}%
\begin{pgfscope}%
\pgfsys@transformshift{3.930367in}{0.417642in}%
\pgfsys@useobject{currentmarker}{}%
\end{pgfscope}%
\end{pgfscope}%
\begin{pgfscope}%
\definecolor{textcolor}{rgb}{0.000000,0.000000,0.000000}%
\pgfsetstrokecolor{textcolor}%
\pgfsetfillcolor{textcolor}%
\pgftext[x=3.930367in,y=0.320420in,,top]{\color{textcolor}\rmfamily\fontsize{8.000000}{9.600000}\selectfont \(\displaystyle {10^{6}}\)}%
\end{pgfscope}%
\begin{pgfscope}%
\pgfpathrectangle{\pgfqpoint{0.594525in}{0.417642in}}{\pgfqpoint{3.345963in}{2.050688in}}%
\pgfusepath{clip}%
\pgfsetrectcap%
\pgfsetroundjoin%
\pgfsetlinewidth{0.803000pt}%
\definecolor{currentstroke}{rgb}{0.850000,0.850000,0.850000}%
\pgfsetstrokecolor{currentstroke}%
\pgfsetdash{}{0pt}%
\pgfpathmoveto{\pgfqpoint{0.609510in}{0.417642in}}%
\pgfpathlineto{\pgfqpoint{0.609510in}{2.468330in}}%
\pgfusepath{stroke}%
\end{pgfscope}%
\begin{pgfscope}%
\pgfsetbuttcap%
\pgfsetroundjoin%
\definecolor{currentfill}{rgb}{0.000000,0.000000,0.000000}%
\pgfsetfillcolor{currentfill}%
\pgfsetlinewidth{0.602250pt}%
\definecolor{currentstroke}{rgb}{0.000000,0.000000,0.000000}%
\pgfsetstrokecolor{currentstroke}%
\pgfsetdash{}{0pt}%
\pgfsys@defobject{currentmarker}{\pgfqpoint{0.000000in}{-0.027778in}}{\pgfqpoint{0.000000in}{0.000000in}}{%
\pgfpathmoveto{\pgfqpoint{0.000000in}{0.000000in}}%
\pgfpathlineto{\pgfqpoint{0.000000in}{-0.027778in}}%
\pgfusepath{stroke,fill}%
}%
\begin{pgfscope}%
\pgfsys@transformshift{0.609510in}{0.417642in}%
\pgfsys@useobject{currentmarker}{}%
\end{pgfscope}%
\end{pgfscope}%
\begin{pgfscope}%
\pgfpathrectangle{\pgfqpoint{0.594525in}{0.417642in}}{\pgfqpoint{3.345963in}{2.050688in}}%
\pgfusepath{clip}%
\pgfsetrectcap%
\pgfsetroundjoin%
\pgfsetlinewidth{0.803000pt}%
\definecolor{currentstroke}{rgb}{0.850000,0.850000,0.850000}%
\pgfsetstrokecolor{currentstroke}%
\pgfsetdash{}{0pt}%
\pgfpathmoveto{\pgfqpoint{0.640295in}{0.417642in}}%
\pgfpathlineto{\pgfqpoint{0.640295in}{2.468330in}}%
\pgfusepath{stroke}%
\end{pgfscope}%
\begin{pgfscope}%
\pgfsetbuttcap%
\pgfsetroundjoin%
\definecolor{currentfill}{rgb}{0.000000,0.000000,0.000000}%
\pgfsetfillcolor{currentfill}%
\pgfsetlinewidth{0.602250pt}%
\definecolor{currentstroke}{rgb}{0.000000,0.000000,0.000000}%
\pgfsetstrokecolor{currentstroke}%
\pgfsetdash{}{0pt}%
\pgfsys@defobject{currentmarker}{\pgfqpoint{0.000000in}{-0.027778in}}{\pgfqpoint{0.000000in}{0.000000in}}{%
\pgfpathmoveto{\pgfqpoint{0.000000in}{0.000000in}}%
\pgfpathlineto{\pgfqpoint{0.000000in}{-0.027778in}}%
\pgfusepath{stroke,fill}%
}%
\begin{pgfscope}%
\pgfsys@transformshift{0.640295in}{0.417642in}%
\pgfsys@useobject{currentmarker}{}%
\end{pgfscope}%
\end{pgfscope}%
\begin{pgfscope}%
\pgfpathrectangle{\pgfqpoint{0.594525in}{0.417642in}}{\pgfqpoint{3.345963in}{2.050688in}}%
\pgfusepath{clip}%
\pgfsetrectcap%
\pgfsetroundjoin%
\pgfsetlinewidth{0.803000pt}%
\definecolor{currentstroke}{rgb}{0.850000,0.850000,0.850000}%
\pgfsetstrokecolor{currentstroke}%
\pgfsetdash{}{0pt}%
\pgfpathmoveto{\pgfqpoint{0.666961in}{0.417642in}}%
\pgfpathlineto{\pgfqpoint{0.666961in}{2.468330in}}%
\pgfusepath{stroke}%
\end{pgfscope}%
\begin{pgfscope}%
\pgfsetbuttcap%
\pgfsetroundjoin%
\definecolor{currentfill}{rgb}{0.000000,0.000000,0.000000}%
\pgfsetfillcolor{currentfill}%
\pgfsetlinewidth{0.602250pt}%
\definecolor{currentstroke}{rgb}{0.000000,0.000000,0.000000}%
\pgfsetstrokecolor{currentstroke}%
\pgfsetdash{}{0pt}%
\pgfsys@defobject{currentmarker}{\pgfqpoint{0.000000in}{-0.027778in}}{\pgfqpoint{0.000000in}{0.000000in}}{%
\pgfpathmoveto{\pgfqpoint{0.000000in}{0.000000in}}%
\pgfpathlineto{\pgfqpoint{0.000000in}{-0.027778in}}%
\pgfusepath{stroke,fill}%
}%
\begin{pgfscope}%
\pgfsys@transformshift{0.666961in}{0.417642in}%
\pgfsys@useobject{currentmarker}{}%
\end{pgfscope}%
\end{pgfscope}%
\begin{pgfscope}%
\pgfpathrectangle{\pgfqpoint{0.594525in}{0.417642in}}{\pgfqpoint{3.345963in}{2.050688in}}%
\pgfusepath{clip}%
\pgfsetrectcap%
\pgfsetroundjoin%
\pgfsetlinewidth{0.803000pt}%
\definecolor{currentstroke}{rgb}{0.850000,0.850000,0.850000}%
\pgfsetstrokecolor{currentstroke}%
\pgfsetdash{}{0pt}%
\pgfpathmoveto{\pgfqpoint{0.690483in}{0.417642in}}%
\pgfpathlineto{\pgfqpoint{0.690483in}{2.468330in}}%
\pgfusepath{stroke}%
\end{pgfscope}%
\begin{pgfscope}%
\pgfsetbuttcap%
\pgfsetroundjoin%
\definecolor{currentfill}{rgb}{0.000000,0.000000,0.000000}%
\pgfsetfillcolor{currentfill}%
\pgfsetlinewidth{0.602250pt}%
\definecolor{currentstroke}{rgb}{0.000000,0.000000,0.000000}%
\pgfsetstrokecolor{currentstroke}%
\pgfsetdash{}{0pt}%
\pgfsys@defobject{currentmarker}{\pgfqpoint{0.000000in}{-0.027778in}}{\pgfqpoint{0.000000in}{0.000000in}}{%
\pgfpathmoveto{\pgfqpoint{0.000000in}{0.000000in}}%
\pgfpathlineto{\pgfqpoint{0.000000in}{-0.027778in}}%
\pgfusepath{stroke,fill}%
}%
\begin{pgfscope}%
\pgfsys@transformshift{0.690483in}{0.417642in}%
\pgfsys@useobject{currentmarker}{}%
\end{pgfscope}%
\end{pgfscope}%
\begin{pgfscope}%
\pgfpathrectangle{\pgfqpoint{0.594525in}{0.417642in}}{\pgfqpoint{3.345963in}{2.050688in}}%
\pgfusepath{clip}%
\pgfsetrectcap%
\pgfsetroundjoin%
\pgfsetlinewidth{0.803000pt}%
\definecolor{currentstroke}{rgb}{0.850000,0.850000,0.850000}%
\pgfsetstrokecolor{currentstroke}%
\pgfsetdash{}{0pt}%
\pgfpathmoveto{\pgfqpoint{0.849948in}{0.417642in}}%
\pgfpathlineto{\pgfqpoint{0.849948in}{2.468330in}}%
\pgfusepath{stroke}%
\end{pgfscope}%
\begin{pgfscope}%
\pgfsetbuttcap%
\pgfsetroundjoin%
\definecolor{currentfill}{rgb}{0.000000,0.000000,0.000000}%
\pgfsetfillcolor{currentfill}%
\pgfsetlinewidth{0.602250pt}%
\definecolor{currentstroke}{rgb}{0.000000,0.000000,0.000000}%
\pgfsetstrokecolor{currentstroke}%
\pgfsetdash{}{0pt}%
\pgfsys@defobject{currentmarker}{\pgfqpoint{0.000000in}{-0.027778in}}{\pgfqpoint{0.000000in}{0.000000in}}{%
\pgfpathmoveto{\pgfqpoint{0.000000in}{0.000000in}}%
\pgfpathlineto{\pgfqpoint{0.000000in}{-0.027778in}}%
\pgfusepath{stroke,fill}%
}%
\begin{pgfscope}%
\pgfsys@transformshift{0.849948in}{0.417642in}%
\pgfsys@useobject{currentmarker}{}%
\end{pgfscope}%
\end{pgfscope}%
\begin{pgfscope}%
\pgfpathrectangle{\pgfqpoint{0.594525in}{0.417642in}}{\pgfqpoint{3.345963in}{2.050688in}}%
\pgfusepath{clip}%
\pgfsetrectcap%
\pgfsetroundjoin%
\pgfsetlinewidth{0.803000pt}%
\definecolor{currentstroke}{rgb}{0.850000,0.850000,0.850000}%
\pgfsetstrokecolor{currentstroke}%
\pgfsetdash{}{0pt}%
\pgfpathmoveto{\pgfqpoint{0.930921in}{0.417642in}}%
\pgfpathlineto{\pgfqpoint{0.930921in}{2.468330in}}%
\pgfusepath{stroke}%
\end{pgfscope}%
\begin{pgfscope}%
\pgfsetbuttcap%
\pgfsetroundjoin%
\definecolor{currentfill}{rgb}{0.000000,0.000000,0.000000}%
\pgfsetfillcolor{currentfill}%
\pgfsetlinewidth{0.602250pt}%
\definecolor{currentstroke}{rgb}{0.000000,0.000000,0.000000}%
\pgfsetstrokecolor{currentstroke}%
\pgfsetdash{}{0pt}%
\pgfsys@defobject{currentmarker}{\pgfqpoint{0.000000in}{-0.027778in}}{\pgfqpoint{0.000000in}{0.000000in}}{%
\pgfpathmoveto{\pgfqpoint{0.000000in}{0.000000in}}%
\pgfpathlineto{\pgfqpoint{0.000000in}{-0.027778in}}%
\pgfusepath{stroke,fill}%
}%
\begin{pgfscope}%
\pgfsys@transformshift{0.930921in}{0.417642in}%
\pgfsys@useobject{currentmarker}{}%
\end{pgfscope}%
\end{pgfscope}%
\begin{pgfscope}%
\pgfpathrectangle{\pgfqpoint{0.594525in}{0.417642in}}{\pgfqpoint{3.345963in}{2.050688in}}%
\pgfusepath{clip}%
\pgfsetrectcap%
\pgfsetroundjoin%
\pgfsetlinewidth{0.803000pt}%
\definecolor{currentstroke}{rgb}{0.850000,0.850000,0.850000}%
\pgfsetstrokecolor{currentstroke}%
\pgfsetdash{}{0pt}%
\pgfpathmoveto{\pgfqpoint{0.988372in}{0.417642in}}%
\pgfpathlineto{\pgfqpoint{0.988372in}{2.468330in}}%
\pgfusepath{stroke}%
\end{pgfscope}%
\begin{pgfscope}%
\pgfsetbuttcap%
\pgfsetroundjoin%
\definecolor{currentfill}{rgb}{0.000000,0.000000,0.000000}%
\pgfsetfillcolor{currentfill}%
\pgfsetlinewidth{0.602250pt}%
\definecolor{currentstroke}{rgb}{0.000000,0.000000,0.000000}%
\pgfsetstrokecolor{currentstroke}%
\pgfsetdash{}{0pt}%
\pgfsys@defobject{currentmarker}{\pgfqpoint{0.000000in}{-0.027778in}}{\pgfqpoint{0.000000in}{0.000000in}}{%
\pgfpathmoveto{\pgfqpoint{0.000000in}{0.000000in}}%
\pgfpathlineto{\pgfqpoint{0.000000in}{-0.027778in}}%
\pgfusepath{stroke,fill}%
}%
\begin{pgfscope}%
\pgfsys@transformshift{0.988372in}{0.417642in}%
\pgfsys@useobject{currentmarker}{}%
\end{pgfscope}%
\end{pgfscope}%
\begin{pgfscope}%
\pgfpathrectangle{\pgfqpoint{0.594525in}{0.417642in}}{\pgfqpoint{3.345963in}{2.050688in}}%
\pgfusepath{clip}%
\pgfsetrectcap%
\pgfsetroundjoin%
\pgfsetlinewidth{0.803000pt}%
\definecolor{currentstroke}{rgb}{0.850000,0.850000,0.850000}%
\pgfsetstrokecolor{currentstroke}%
\pgfsetdash{}{0pt}%
\pgfpathmoveto{\pgfqpoint{1.032935in}{0.417642in}}%
\pgfpathlineto{\pgfqpoint{1.032935in}{2.468330in}}%
\pgfusepath{stroke}%
\end{pgfscope}%
\begin{pgfscope}%
\pgfsetbuttcap%
\pgfsetroundjoin%
\definecolor{currentfill}{rgb}{0.000000,0.000000,0.000000}%
\pgfsetfillcolor{currentfill}%
\pgfsetlinewidth{0.602250pt}%
\definecolor{currentstroke}{rgb}{0.000000,0.000000,0.000000}%
\pgfsetstrokecolor{currentstroke}%
\pgfsetdash{}{0pt}%
\pgfsys@defobject{currentmarker}{\pgfqpoint{0.000000in}{-0.027778in}}{\pgfqpoint{0.000000in}{0.000000in}}{%
\pgfpathmoveto{\pgfqpoint{0.000000in}{0.000000in}}%
\pgfpathlineto{\pgfqpoint{0.000000in}{-0.027778in}}%
\pgfusepath{stroke,fill}%
}%
\begin{pgfscope}%
\pgfsys@transformshift{1.032935in}{0.417642in}%
\pgfsys@useobject{currentmarker}{}%
\end{pgfscope}%
\end{pgfscope}%
\begin{pgfscope}%
\pgfpathrectangle{\pgfqpoint{0.594525in}{0.417642in}}{\pgfqpoint{3.345963in}{2.050688in}}%
\pgfusepath{clip}%
\pgfsetrectcap%
\pgfsetroundjoin%
\pgfsetlinewidth{0.803000pt}%
\definecolor{currentstroke}{rgb}{0.850000,0.850000,0.850000}%
\pgfsetstrokecolor{currentstroke}%
\pgfsetdash{}{0pt}%
\pgfpathmoveto{\pgfqpoint{1.069345in}{0.417642in}}%
\pgfpathlineto{\pgfqpoint{1.069345in}{2.468330in}}%
\pgfusepath{stroke}%
\end{pgfscope}%
\begin{pgfscope}%
\pgfsetbuttcap%
\pgfsetroundjoin%
\definecolor{currentfill}{rgb}{0.000000,0.000000,0.000000}%
\pgfsetfillcolor{currentfill}%
\pgfsetlinewidth{0.602250pt}%
\definecolor{currentstroke}{rgb}{0.000000,0.000000,0.000000}%
\pgfsetstrokecolor{currentstroke}%
\pgfsetdash{}{0pt}%
\pgfsys@defobject{currentmarker}{\pgfqpoint{0.000000in}{-0.027778in}}{\pgfqpoint{0.000000in}{0.000000in}}{%
\pgfpathmoveto{\pgfqpoint{0.000000in}{0.000000in}}%
\pgfpathlineto{\pgfqpoint{0.000000in}{-0.027778in}}%
\pgfusepath{stroke,fill}%
}%
\begin{pgfscope}%
\pgfsys@transformshift{1.069345in}{0.417642in}%
\pgfsys@useobject{currentmarker}{}%
\end{pgfscope}%
\end{pgfscope}%
\begin{pgfscope}%
\pgfpathrectangle{\pgfqpoint{0.594525in}{0.417642in}}{\pgfqpoint{3.345963in}{2.050688in}}%
\pgfusepath{clip}%
\pgfsetrectcap%
\pgfsetroundjoin%
\pgfsetlinewidth{0.803000pt}%
\definecolor{currentstroke}{rgb}{0.850000,0.850000,0.850000}%
\pgfsetstrokecolor{currentstroke}%
\pgfsetdash{}{0pt}%
\pgfpathmoveto{\pgfqpoint{1.100129in}{0.417642in}}%
\pgfpathlineto{\pgfqpoint{1.100129in}{2.468330in}}%
\pgfusepath{stroke}%
\end{pgfscope}%
\begin{pgfscope}%
\pgfsetbuttcap%
\pgfsetroundjoin%
\definecolor{currentfill}{rgb}{0.000000,0.000000,0.000000}%
\pgfsetfillcolor{currentfill}%
\pgfsetlinewidth{0.602250pt}%
\definecolor{currentstroke}{rgb}{0.000000,0.000000,0.000000}%
\pgfsetstrokecolor{currentstroke}%
\pgfsetdash{}{0pt}%
\pgfsys@defobject{currentmarker}{\pgfqpoint{0.000000in}{-0.027778in}}{\pgfqpoint{0.000000in}{0.000000in}}{%
\pgfpathmoveto{\pgfqpoint{0.000000in}{0.000000in}}%
\pgfpathlineto{\pgfqpoint{0.000000in}{-0.027778in}}%
\pgfusepath{stroke,fill}%
}%
\begin{pgfscope}%
\pgfsys@transformshift{1.100129in}{0.417642in}%
\pgfsys@useobject{currentmarker}{}%
\end{pgfscope}%
\end{pgfscope}%
\begin{pgfscope}%
\pgfpathrectangle{\pgfqpoint{0.594525in}{0.417642in}}{\pgfqpoint{3.345963in}{2.050688in}}%
\pgfusepath{clip}%
\pgfsetrectcap%
\pgfsetroundjoin%
\pgfsetlinewidth{0.803000pt}%
\definecolor{currentstroke}{rgb}{0.850000,0.850000,0.850000}%
\pgfsetstrokecolor{currentstroke}%
\pgfsetdash{}{0pt}%
\pgfpathmoveto{\pgfqpoint{1.126796in}{0.417642in}}%
\pgfpathlineto{\pgfqpoint{1.126796in}{2.468330in}}%
\pgfusepath{stroke}%
\end{pgfscope}%
\begin{pgfscope}%
\pgfsetbuttcap%
\pgfsetroundjoin%
\definecolor{currentfill}{rgb}{0.000000,0.000000,0.000000}%
\pgfsetfillcolor{currentfill}%
\pgfsetlinewidth{0.602250pt}%
\definecolor{currentstroke}{rgb}{0.000000,0.000000,0.000000}%
\pgfsetstrokecolor{currentstroke}%
\pgfsetdash{}{0pt}%
\pgfsys@defobject{currentmarker}{\pgfqpoint{0.000000in}{-0.027778in}}{\pgfqpoint{0.000000in}{0.000000in}}{%
\pgfpathmoveto{\pgfqpoint{0.000000in}{0.000000in}}%
\pgfpathlineto{\pgfqpoint{0.000000in}{-0.027778in}}%
\pgfusepath{stroke,fill}%
}%
\begin{pgfscope}%
\pgfsys@transformshift{1.126796in}{0.417642in}%
\pgfsys@useobject{currentmarker}{}%
\end{pgfscope}%
\end{pgfscope}%
\begin{pgfscope}%
\pgfpathrectangle{\pgfqpoint{0.594525in}{0.417642in}}{\pgfqpoint{3.345963in}{2.050688in}}%
\pgfusepath{clip}%
\pgfsetrectcap%
\pgfsetroundjoin%
\pgfsetlinewidth{0.803000pt}%
\definecolor{currentstroke}{rgb}{0.850000,0.850000,0.850000}%
\pgfsetstrokecolor{currentstroke}%
\pgfsetdash{}{0pt}%
\pgfpathmoveto{\pgfqpoint{1.150318in}{0.417642in}}%
\pgfpathlineto{\pgfqpoint{1.150318in}{2.468330in}}%
\pgfusepath{stroke}%
\end{pgfscope}%
\begin{pgfscope}%
\pgfsetbuttcap%
\pgfsetroundjoin%
\definecolor{currentfill}{rgb}{0.000000,0.000000,0.000000}%
\pgfsetfillcolor{currentfill}%
\pgfsetlinewidth{0.602250pt}%
\definecolor{currentstroke}{rgb}{0.000000,0.000000,0.000000}%
\pgfsetstrokecolor{currentstroke}%
\pgfsetdash{}{0pt}%
\pgfsys@defobject{currentmarker}{\pgfqpoint{0.000000in}{-0.027778in}}{\pgfqpoint{0.000000in}{0.000000in}}{%
\pgfpathmoveto{\pgfqpoint{0.000000in}{0.000000in}}%
\pgfpathlineto{\pgfqpoint{0.000000in}{-0.027778in}}%
\pgfusepath{stroke,fill}%
}%
\begin{pgfscope}%
\pgfsys@transformshift{1.150318in}{0.417642in}%
\pgfsys@useobject{currentmarker}{}%
\end{pgfscope}%
\end{pgfscope}%
\begin{pgfscope}%
\pgfpathrectangle{\pgfqpoint{0.594525in}{0.417642in}}{\pgfqpoint{3.345963in}{2.050688in}}%
\pgfusepath{clip}%
\pgfsetrectcap%
\pgfsetroundjoin%
\pgfsetlinewidth{0.803000pt}%
\definecolor{currentstroke}{rgb}{0.850000,0.850000,0.850000}%
\pgfsetstrokecolor{currentstroke}%
\pgfsetdash{}{0pt}%
\pgfpathmoveto{\pgfqpoint{1.309783in}{0.417642in}}%
\pgfpathlineto{\pgfqpoint{1.309783in}{2.468330in}}%
\pgfusepath{stroke}%
\end{pgfscope}%
\begin{pgfscope}%
\pgfsetbuttcap%
\pgfsetroundjoin%
\definecolor{currentfill}{rgb}{0.000000,0.000000,0.000000}%
\pgfsetfillcolor{currentfill}%
\pgfsetlinewidth{0.602250pt}%
\definecolor{currentstroke}{rgb}{0.000000,0.000000,0.000000}%
\pgfsetstrokecolor{currentstroke}%
\pgfsetdash{}{0pt}%
\pgfsys@defobject{currentmarker}{\pgfqpoint{0.000000in}{-0.027778in}}{\pgfqpoint{0.000000in}{0.000000in}}{%
\pgfpathmoveto{\pgfqpoint{0.000000in}{0.000000in}}%
\pgfpathlineto{\pgfqpoint{0.000000in}{-0.027778in}}%
\pgfusepath{stroke,fill}%
}%
\begin{pgfscope}%
\pgfsys@transformshift{1.309783in}{0.417642in}%
\pgfsys@useobject{currentmarker}{}%
\end{pgfscope}%
\end{pgfscope}%
\begin{pgfscope}%
\pgfpathrectangle{\pgfqpoint{0.594525in}{0.417642in}}{\pgfqpoint{3.345963in}{2.050688in}}%
\pgfusepath{clip}%
\pgfsetrectcap%
\pgfsetroundjoin%
\pgfsetlinewidth{0.803000pt}%
\definecolor{currentstroke}{rgb}{0.850000,0.850000,0.850000}%
\pgfsetstrokecolor{currentstroke}%
\pgfsetdash{}{0pt}%
\pgfpathmoveto{\pgfqpoint{1.390755in}{0.417642in}}%
\pgfpathlineto{\pgfqpoint{1.390755in}{2.468330in}}%
\pgfusepath{stroke}%
\end{pgfscope}%
\begin{pgfscope}%
\pgfsetbuttcap%
\pgfsetroundjoin%
\definecolor{currentfill}{rgb}{0.000000,0.000000,0.000000}%
\pgfsetfillcolor{currentfill}%
\pgfsetlinewidth{0.602250pt}%
\definecolor{currentstroke}{rgb}{0.000000,0.000000,0.000000}%
\pgfsetstrokecolor{currentstroke}%
\pgfsetdash{}{0pt}%
\pgfsys@defobject{currentmarker}{\pgfqpoint{0.000000in}{-0.027778in}}{\pgfqpoint{0.000000in}{0.000000in}}{%
\pgfpathmoveto{\pgfqpoint{0.000000in}{0.000000in}}%
\pgfpathlineto{\pgfqpoint{0.000000in}{-0.027778in}}%
\pgfusepath{stroke,fill}%
}%
\begin{pgfscope}%
\pgfsys@transformshift{1.390755in}{0.417642in}%
\pgfsys@useobject{currentmarker}{}%
\end{pgfscope}%
\end{pgfscope}%
\begin{pgfscope}%
\pgfpathrectangle{\pgfqpoint{0.594525in}{0.417642in}}{\pgfqpoint{3.345963in}{2.050688in}}%
\pgfusepath{clip}%
\pgfsetrectcap%
\pgfsetroundjoin%
\pgfsetlinewidth{0.803000pt}%
\definecolor{currentstroke}{rgb}{0.850000,0.850000,0.850000}%
\pgfsetstrokecolor{currentstroke}%
\pgfsetdash{}{0pt}%
\pgfpathmoveto{\pgfqpoint{1.448207in}{0.417642in}}%
\pgfpathlineto{\pgfqpoint{1.448207in}{2.468330in}}%
\pgfusepath{stroke}%
\end{pgfscope}%
\begin{pgfscope}%
\pgfsetbuttcap%
\pgfsetroundjoin%
\definecolor{currentfill}{rgb}{0.000000,0.000000,0.000000}%
\pgfsetfillcolor{currentfill}%
\pgfsetlinewidth{0.602250pt}%
\definecolor{currentstroke}{rgb}{0.000000,0.000000,0.000000}%
\pgfsetstrokecolor{currentstroke}%
\pgfsetdash{}{0pt}%
\pgfsys@defobject{currentmarker}{\pgfqpoint{0.000000in}{-0.027778in}}{\pgfqpoint{0.000000in}{0.000000in}}{%
\pgfpathmoveto{\pgfqpoint{0.000000in}{0.000000in}}%
\pgfpathlineto{\pgfqpoint{0.000000in}{-0.027778in}}%
\pgfusepath{stroke,fill}%
}%
\begin{pgfscope}%
\pgfsys@transformshift{1.448207in}{0.417642in}%
\pgfsys@useobject{currentmarker}{}%
\end{pgfscope}%
\end{pgfscope}%
\begin{pgfscope}%
\pgfpathrectangle{\pgfqpoint{0.594525in}{0.417642in}}{\pgfqpoint{3.345963in}{2.050688in}}%
\pgfusepath{clip}%
\pgfsetrectcap%
\pgfsetroundjoin%
\pgfsetlinewidth{0.803000pt}%
\definecolor{currentstroke}{rgb}{0.850000,0.850000,0.850000}%
\pgfsetstrokecolor{currentstroke}%
\pgfsetdash{}{0pt}%
\pgfpathmoveto{\pgfqpoint{1.492769in}{0.417642in}}%
\pgfpathlineto{\pgfqpoint{1.492769in}{2.468330in}}%
\pgfusepath{stroke}%
\end{pgfscope}%
\begin{pgfscope}%
\pgfsetbuttcap%
\pgfsetroundjoin%
\definecolor{currentfill}{rgb}{0.000000,0.000000,0.000000}%
\pgfsetfillcolor{currentfill}%
\pgfsetlinewidth{0.602250pt}%
\definecolor{currentstroke}{rgb}{0.000000,0.000000,0.000000}%
\pgfsetstrokecolor{currentstroke}%
\pgfsetdash{}{0pt}%
\pgfsys@defobject{currentmarker}{\pgfqpoint{0.000000in}{-0.027778in}}{\pgfqpoint{0.000000in}{0.000000in}}{%
\pgfpathmoveto{\pgfqpoint{0.000000in}{0.000000in}}%
\pgfpathlineto{\pgfqpoint{0.000000in}{-0.027778in}}%
\pgfusepath{stroke,fill}%
}%
\begin{pgfscope}%
\pgfsys@transformshift{1.492769in}{0.417642in}%
\pgfsys@useobject{currentmarker}{}%
\end{pgfscope}%
\end{pgfscope}%
\begin{pgfscope}%
\pgfpathrectangle{\pgfqpoint{0.594525in}{0.417642in}}{\pgfqpoint{3.345963in}{2.050688in}}%
\pgfusepath{clip}%
\pgfsetrectcap%
\pgfsetroundjoin%
\pgfsetlinewidth{0.803000pt}%
\definecolor{currentstroke}{rgb}{0.850000,0.850000,0.850000}%
\pgfsetstrokecolor{currentstroke}%
\pgfsetdash{}{0pt}%
\pgfpathmoveto{\pgfqpoint{1.529180in}{0.417642in}}%
\pgfpathlineto{\pgfqpoint{1.529180in}{2.468330in}}%
\pgfusepath{stroke}%
\end{pgfscope}%
\begin{pgfscope}%
\pgfsetbuttcap%
\pgfsetroundjoin%
\definecolor{currentfill}{rgb}{0.000000,0.000000,0.000000}%
\pgfsetfillcolor{currentfill}%
\pgfsetlinewidth{0.602250pt}%
\definecolor{currentstroke}{rgb}{0.000000,0.000000,0.000000}%
\pgfsetstrokecolor{currentstroke}%
\pgfsetdash{}{0pt}%
\pgfsys@defobject{currentmarker}{\pgfqpoint{0.000000in}{-0.027778in}}{\pgfqpoint{0.000000in}{0.000000in}}{%
\pgfpathmoveto{\pgfqpoint{0.000000in}{0.000000in}}%
\pgfpathlineto{\pgfqpoint{0.000000in}{-0.027778in}}%
\pgfusepath{stroke,fill}%
}%
\begin{pgfscope}%
\pgfsys@transformshift{1.529180in}{0.417642in}%
\pgfsys@useobject{currentmarker}{}%
\end{pgfscope}%
\end{pgfscope}%
\begin{pgfscope}%
\pgfpathrectangle{\pgfqpoint{0.594525in}{0.417642in}}{\pgfqpoint{3.345963in}{2.050688in}}%
\pgfusepath{clip}%
\pgfsetrectcap%
\pgfsetroundjoin%
\pgfsetlinewidth{0.803000pt}%
\definecolor{currentstroke}{rgb}{0.850000,0.850000,0.850000}%
\pgfsetstrokecolor{currentstroke}%
\pgfsetdash{}{0pt}%
\pgfpathmoveto{\pgfqpoint{1.559964in}{0.417642in}}%
\pgfpathlineto{\pgfqpoint{1.559964in}{2.468330in}}%
\pgfusepath{stroke}%
\end{pgfscope}%
\begin{pgfscope}%
\pgfsetbuttcap%
\pgfsetroundjoin%
\definecolor{currentfill}{rgb}{0.000000,0.000000,0.000000}%
\pgfsetfillcolor{currentfill}%
\pgfsetlinewidth{0.602250pt}%
\definecolor{currentstroke}{rgb}{0.000000,0.000000,0.000000}%
\pgfsetstrokecolor{currentstroke}%
\pgfsetdash{}{0pt}%
\pgfsys@defobject{currentmarker}{\pgfqpoint{0.000000in}{-0.027778in}}{\pgfqpoint{0.000000in}{0.000000in}}{%
\pgfpathmoveto{\pgfqpoint{0.000000in}{0.000000in}}%
\pgfpathlineto{\pgfqpoint{0.000000in}{-0.027778in}}%
\pgfusepath{stroke,fill}%
}%
\begin{pgfscope}%
\pgfsys@transformshift{1.559964in}{0.417642in}%
\pgfsys@useobject{currentmarker}{}%
\end{pgfscope}%
\end{pgfscope}%
\begin{pgfscope}%
\pgfpathrectangle{\pgfqpoint{0.594525in}{0.417642in}}{\pgfqpoint{3.345963in}{2.050688in}}%
\pgfusepath{clip}%
\pgfsetrectcap%
\pgfsetroundjoin%
\pgfsetlinewidth{0.803000pt}%
\definecolor{currentstroke}{rgb}{0.850000,0.850000,0.850000}%
\pgfsetstrokecolor{currentstroke}%
\pgfsetdash{}{0pt}%
\pgfpathmoveto{\pgfqpoint{1.586631in}{0.417642in}}%
\pgfpathlineto{\pgfqpoint{1.586631in}{2.468330in}}%
\pgfusepath{stroke}%
\end{pgfscope}%
\begin{pgfscope}%
\pgfsetbuttcap%
\pgfsetroundjoin%
\definecolor{currentfill}{rgb}{0.000000,0.000000,0.000000}%
\pgfsetfillcolor{currentfill}%
\pgfsetlinewidth{0.602250pt}%
\definecolor{currentstroke}{rgb}{0.000000,0.000000,0.000000}%
\pgfsetstrokecolor{currentstroke}%
\pgfsetdash{}{0pt}%
\pgfsys@defobject{currentmarker}{\pgfqpoint{0.000000in}{-0.027778in}}{\pgfqpoint{0.000000in}{0.000000in}}{%
\pgfpathmoveto{\pgfqpoint{0.000000in}{0.000000in}}%
\pgfpathlineto{\pgfqpoint{0.000000in}{-0.027778in}}%
\pgfusepath{stroke,fill}%
}%
\begin{pgfscope}%
\pgfsys@transformshift{1.586631in}{0.417642in}%
\pgfsys@useobject{currentmarker}{}%
\end{pgfscope}%
\end{pgfscope}%
\begin{pgfscope}%
\pgfpathrectangle{\pgfqpoint{0.594525in}{0.417642in}}{\pgfqpoint{3.345963in}{2.050688in}}%
\pgfusepath{clip}%
\pgfsetrectcap%
\pgfsetroundjoin%
\pgfsetlinewidth{0.803000pt}%
\definecolor{currentstroke}{rgb}{0.850000,0.850000,0.850000}%
\pgfsetstrokecolor{currentstroke}%
\pgfsetdash{}{0pt}%
\pgfpathmoveto{\pgfqpoint{1.610152in}{0.417642in}}%
\pgfpathlineto{\pgfqpoint{1.610152in}{2.468330in}}%
\pgfusepath{stroke}%
\end{pgfscope}%
\begin{pgfscope}%
\pgfsetbuttcap%
\pgfsetroundjoin%
\definecolor{currentfill}{rgb}{0.000000,0.000000,0.000000}%
\pgfsetfillcolor{currentfill}%
\pgfsetlinewidth{0.602250pt}%
\definecolor{currentstroke}{rgb}{0.000000,0.000000,0.000000}%
\pgfsetstrokecolor{currentstroke}%
\pgfsetdash{}{0pt}%
\pgfsys@defobject{currentmarker}{\pgfqpoint{0.000000in}{-0.027778in}}{\pgfqpoint{0.000000in}{0.000000in}}{%
\pgfpathmoveto{\pgfqpoint{0.000000in}{0.000000in}}%
\pgfpathlineto{\pgfqpoint{0.000000in}{-0.027778in}}%
\pgfusepath{stroke,fill}%
}%
\begin{pgfscope}%
\pgfsys@transformshift{1.610152in}{0.417642in}%
\pgfsys@useobject{currentmarker}{}%
\end{pgfscope}%
\end{pgfscope}%
\begin{pgfscope}%
\pgfpathrectangle{\pgfqpoint{0.594525in}{0.417642in}}{\pgfqpoint{3.345963in}{2.050688in}}%
\pgfusepath{clip}%
\pgfsetrectcap%
\pgfsetroundjoin%
\pgfsetlinewidth{0.803000pt}%
\definecolor{currentstroke}{rgb}{0.850000,0.850000,0.850000}%
\pgfsetstrokecolor{currentstroke}%
\pgfsetdash{}{0pt}%
\pgfpathmoveto{\pgfqpoint{1.769617in}{0.417642in}}%
\pgfpathlineto{\pgfqpoint{1.769617in}{2.468330in}}%
\pgfusepath{stroke}%
\end{pgfscope}%
\begin{pgfscope}%
\pgfsetbuttcap%
\pgfsetroundjoin%
\definecolor{currentfill}{rgb}{0.000000,0.000000,0.000000}%
\pgfsetfillcolor{currentfill}%
\pgfsetlinewidth{0.602250pt}%
\definecolor{currentstroke}{rgb}{0.000000,0.000000,0.000000}%
\pgfsetstrokecolor{currentstroke}%
\pgfsetdash{}{0pt}%
\pgfsys@defobject{currentmarker}{\pgfqpoint{0.000000in}{-0.027778in}}{\pgfqpoint{0.000000in}{0.000000in}}{%
\pgfpathmoveto{\pgfqpoint{0.000000in}{0.000000in}}%
\pgfpathlineto{\pgfqpoint{0.000000in}{-0.027778in}}%
\pgfusepath{stroke,fill}%
}%
\begin{pgfscope}%
\pgfsys@transformshift{1.769617in}{0.417642in}%
\pgfsys@useobject{currentmarker}{}%
\end{pgfscope}%
\end{pgfscope}%
\begin{pgfscope}%
\pgfpathrectangle{\pgfqpoint{0.594525in}{0.417642in}}{\pgfqpoint{3.345963in}{2.050688in}}%
\pgfusepath{clip}%
\pgfsetrectcap%
\pgfsetroundjoin%
\pgfsetlinewidth{0.803000pt}%
\definecolor{currentstroke}{rgb}{0.850000,0.850000,0.850000}%
\pgfsetstrokecolor{currentstroke}%
\pgfsetdash{}{0pt}%
\pgfpathmoveto{\pgfqpoint{1.850590in}{0.417642in}}%
\pgfpathlineto{\pgfqpoint{1.850590in}{2.468330in}}%
\pgfusepath{stroke}%
\end{pgfscope}%
\begin{pgfscope}%
\pgfsetbuttcap%
\pgfsetroundjoin%
\definecolor{currentfill}{rgb}{0.000000,0.000000,0.000000}%
\pgfsetfillcolor{currentfill}%
\pgfsetlinewidth{0.602250pt}%
\definecolor{currentstroke}{rgb}{0.000000,0.000000,0.000000}%
\pgfsetstrokecolor{currentstroke}%
\pgfsetdash{}{0pt}%
\pgfsys@defobject{currentmarker}{\pgfqpoint{0.000000in}{-0.027778in}}{\pgfqpoint{0.000000in}{0.000000in}}{%
\pgfpathmoveto{\pgfqpoint{0.000000in}{0.000000in}}%
\pgfpathlineto{\pgfqpoint{0.000000in}{-0.027778in}}%
\pgfusepath{stroke,fill}%
}%
\begin{pgfscope}%
\pgfsys@transformshift{1.850590in}{0.417642in}%
\pgfsys@useobject{currentmarker}{}%
\end{pgfscope}%
\end{pgfscope}%
\begin{pgfscope}%
\pgfpathrectangle{\pgfqpoint{0.594525in}{0.417642in}}{\pgfqpoint{3.345963in}{2.050688in}}%
\pgfusepath{clip}%
\pgfsetrectcap%
\pgfsetroundjoin%
\pgfsetlinewidth{0.803000pt}%
\definecolor{currentstroke}{rgb}{0.850000,0.850000,0.850000}%
\pgfsetstrokecolor{currentstroke}%
\pgfsetdash{}{0pt}%
\pgfpathmoveto{\pgfqpoint{1.908041in}{0.417642in}}%
\pgfpathlineto{\pgfqpoint{1.908041in}{2.468330in}}%
\pgfusepath{stroke}%
\end{pgfscope}%
\begin{pgfscope}%
\pgfsetbuttcap%
\pgfsetroundjoin%
\definecolor{currentfill}{rgb}{0.000000,0.000000,0.000000}%
\pgfsetfillcolor{currentfill}%
\pgfsetlinewidth{0.602250pt}%
\definecolor{currentstroke}{rgb}{0.000000,0.000000,0.000000}%
\pgfsetstrokecolor{currentstroke}%
\pgfsetdash{}{0pt}%
\pgfsys@defobject{currentmarker}{\pgfqpoint{0.000000in}{-0.027778in}}{\pgfqpoint{0.000000in}{0.000000in}}{%
\pgfpathmoveto{\pgfqpoint{0.000000in}{0.000000in}}%
\pgfpathlineto{\pgfqpoint{0.000000in}{-0.027778in}}%
\pgfusepath{stroke,fill}%
}%
\begin{pgfscope}%
\pgfsys@transformshift{1.908041in}{0.417642in}%
\pgfsys@useobject{currentmarker}{}%
\end{pgfscope}%
\end{pgfscope}%
\begin{pgfscope}%
\pgfpathrectangle{\pgfqpoint{0.594525in}{0.417642in}}{\pgfqpoint{3.345963in}{2.050688in}}%
\pgfusepath{clip}%
\pgfsetrectcap%
\pgfsetroundjoin%
\pgfsetlinewidth{0.803000pt}%
\definecolor{currentstroke}{rgb}{0.850000,0.850000,0.850000}%
\pgfsetstrokecolor{currentstroke}%
\pgfsetdash{}{0pt}%
\pgfpathmoveto{\pgfqpoint{1.952604in}{0.417642in}}%
\pgfpathlineto{\pgfqpoint{1.952604in}{2.468330in}}%
\pgfusepath{stroke}%
\end{pgfscope}%
\begin{pgfscope}%
\pgfsetbuttcap%
\pgfsetroundjoin%
\definecolor{currentfill}{rgb}{0.000000,0.000000,0.000000}%
\pgfsetfillcolor{currentfill}%
\pgfsetlinewidth{0.602250pt}%
\definecolor{currentstroke}{rgb}{0.000000,0.000000,0.000000}%
\pgfsetstrokecolor{currentstroke}%
\pgfsetdash{}{0pt}%
\pgfsys@defobject{currentmarker}{\pgfqpoint{0.000000in}{-0.027778in}}{\pgfqpoint{0.000000in}{0.000000in}}{%
\pgfpathmoveto{\pgfqpoint{0.000000in}{0.000000in}}%
\pgfpathlineto{\pgfqpoint{0.000000in}{-0.027778in}}%
\pgfusepath{stroke,fill}%
}%
\begin{pgfscope}%
\pgfsys@transformshift{1.952604in}{0.417642in}%
\pgfsys@useobject{currentmarker}{}%
\end{pgfscope}%
\end{pgfscope}%
\begin{pgfscope}%
\pgfpathrectangle{\pgfqpoint{0.594525in}{0.417642in}}{\pgfqpoint{3.345963in}{2.050688in}}%
\pgfusepath{clip}%
\pgfsetrectcap%
\pgfsetroundjoin%
\pgfsetlinewidth{0.803000pt}%
\definecolor{currentstroke}{rgb}{0.850000,0.850000,0.850000}%
\pgfsetstrokecolor{currentstroke}%
\pgfsetdash{}{0pt}%
\pgfpathmoveto{\pgfqpoint{1.989014in}{0.417642in}}%
\pgfpathlineto{\pgfqpoint{1.989014in}{2.468330in}}%
\pgfusepath{stroke}%
\end{pgfscope}%
\begin{pgfscope}%
\pgfsetbuttcap%
\pgfsetroundjoin%
\definecolor{currentfill}{rgb}{0.000000,0.000000,0.000000}%
\pgfsetfillcolor{currentfill}%
\pgfsetlinewidth{0.602250pt}%
\definecolor{currentstroke}{rgb}{0.000000,0.000000,0.000000}%
\pgfsetstrokecolor{currentstroke}%
\pgfsetdash{}{0pt}%
\pgfsys@defobject{currentmarker}{\pgfqpoint{0.000000in}{-0.027778in}}{\pgfqpoint{0.000000in}{0.000000in}}{%
\pgfpathmoveto{\pgfqpoint{0.000000in}{0.000000in}}%
\pgfpathlineto{\pgfqpoint{0.000000in}{-0.027778in}}%
\pgfusepath{stroke,fill}%
}%
\begin{pgfscope}%
\pgfsys@transformshift{1.989014in}{0.417642in}%
\pgfsys@useobject{currentmarker}{}%
\end{pgfscope}%
\end{pgfscope}%
\begin{pgfscope}%
\pgfpathrectangle{\pgfqpoint{0.594525in}{0.417642in}}{\pgfqpoint{3.345963in}{2.050688in}}%
\pgfusepath{clip}%
\pgfsetrectcap%
\pgfsetroundjoin%
\pgfsetlinewidth{0.803000pt}%
\definecolor{currentstroke}{rgb}{0.850000,0.850000,0.850000}%
\pgfsetstrokecolor{currentstroke}%
\pgfsetdash{}{0pt}%
\pgfpathmoveto{\pgfqpoint{2.019799in}{0.417642in}}%
\pgfpathlineto{\pgfqpoint{2.019799in}{2.468330in}}%
\pgfusepath{stroke}%
\end{pgfscope}%
\begin{pgfscope}%
\pgfsetbuttcap%
\pgfsetroundjoin%
\definecolor{currentfill}{rgb}{0.000000,0.000000,0.000000}%
\pgfsetfillcolor{currentfill}%
\pgfsetlinewidth{0.602250pt}%
\definecolor{currentstroke}{rgb}{0.000000,0.000000,0.000000}%
\pgfsetstrokecolor{currentstroke}%
\pgfsetdash{}{0pt}%
\pgfsys@defobject{currentmarker}{\pgfqpoint{0.000000in}{-0.027778in}}{\pgfqpoint{0.000000in}{0.000000in}}{%
\pgfpathmoveto{\pgfqpoint{0.000000in}{0.000000in}}%
\pgfpathlineto{\pgfqpoint{0.000000in}{-0.027778in}}%
\pgfusepath{stroke,fill}%
}%
\begin{pgfscope}%
\pgfsys@transformshift{2.019799in}{0.417642in}%
\pgfsys@useobject{currentmarker}{}%
\end{pgfscope}%
\end{pgfscope}%
\begin{pgfscope}%
\pgfpathrectangle{\pgfqpoint{0.594525in}{0.417642in}}{\pgfqpoint{3.345963in}{2.050688in}}%
\pgfusepath{clip}%
\pgfsetrectcap%
\pgfsetroundjoin%
\pgfsetlinewidth{0.803000pt}%
\definecolor{currentstroke}{rgb}{0.850000,0.850000,0.850000}%
\pgfsetstrokecolor{currentstroke}%
\pgfsetdash{}{0pt}%
\pgfpathmoveto{\pgfqpoint{2.046465in}{0.417642in}}%
\pgfpathlineto{\pgfqpoint{2.046465in}{2.468330in}}%
\pgfusepath{stroke}%
\end{pgfscope}%
\begin{pgfscope}%
\pgfsetbuttcap%
\pgfsetroundjoin%
\definecolor{currentfill}{rgb}{0.000000,0.000000,0.000000}%
\pgfsetfillcolor{currentfill}%
\pgfsetlinewidth{0.602250pt}%
\definecolor{currentstroke}{rgb}{0.000000,0.000000,0.000000}%
\pgfsetstrokecolor{currentstroke}%
\pgfsetdash{}{0pt}%
\pgfsys@defobject{currentmarker}{\pgfqpoint{0.000000in}{-0.027778in}}{\pgfqpoint{0.000000in}{0.000000in}}{%
\pgfpathmoveto{\pgfqpoint{0.000000in}{0.000000in}}%
\pgfpathlineto{\pgfqpoint{0.000000in}{-0.027778in}}%
\pgfusepath{stroke,fill}%
}%
\begin{pgfscope}%
\pgfsys@transformshift{2.046465in}{0.417642in}%
\pgfsys@useobject{currentmarker}{}%
\end{pgfscope}%
\end{pgfscope}%
\begin{pgfscope}%
\pgfpathrectangle{\pgfqpoint{0.594525in}{0.417642in}}{\pgfqpoint{3.345963in}{2.050688in}}%
\pgfusepath{clip}%
\pgfsetrectcap%
\pgfsetroundjoin%
\pgfsetlinewidth{0.803000pt}%
\definecolor{currentstroke}{rgb}{0.850000,0.850000,0.850000}%
\pgfsetstrokecolor{currentstroke}%
\pgfsetdash{}{0pt}%
\pgfpathmoveto{\pgfqpoint{2.069987in}{0.417642in}}%
\pgfpathlineto{\pgfqpoint{2.069987in}{2.468330in}}%
\pgfusepath{stroke}%
\end{pgfscope}%
\begin{pgfscope}%
\pgfsetbuttcap%
\pgfsetroundjoin%
\definecolor{currentfill}{rgb}{0.000000,0.000000,0.000000}%
\pgfsetfillcolor{currentfill}%
\pgfsetlinewidth{0.602250pt}%
\definecolor{currentstroke}{rgb}{0.000000,0.000000,0.000000}%
\pgfsetstrokecolor{currentstroke}%
\pgfsetdash{}{0pt}%
\pgfsys@defobject{currentmarker}{\pgfqpoint{0.000000in}{-0.027778in}}{\pgfqpoint{0.000000in}{0.000000in}}{%
\pgfpathmoveto{\pgfqpoint{0.000000in}{0.000000in}}%
\pgfpathlineto{\pgfqpoint{0.000000in}{-0.027778in}}%
\pgfusepath{stroke,fill}%
}%
\begin{pgfscope}%
\pgfsys@transformshift{2.069987in}{0.417642in}%
\pgfsys@useobject{currentmarker}{}%
\end{pgfscope}%
\end{pgfscope}%
\begin{pgfscope}%
\pgfpathrectangle{\pgfqpoint{0.594525in}{0.417642in}}{\pgfqpoint{3.345963in}{2.050688in}}%
\pgfusepath{clip}%
\pgfsetrectcap%
\pgfsetroundjoin%
\pgfsetlinewidth{0.803000pt}%
\definecolor{currentstroke}{rgb}{0.850000,0.850000,0.850000}%
\pgfsetstrokecolor{currentstroke}%
\pgfsetdash{}{0pt}%
\pgfpathmoveto{\pgfqpoint{2.229452in}{0.417642in}}%
\pgfpathlineto{\pgfqpoint{2.229452in}{2.468330in}}%
\pgfusepath{stroke}%
\end{pgfscope}%
\begin{pgfscope}%
\pgfsetbuttcap%
\pgfsetroundjoin%
\definecolor{currentfill}{rgb}{0.000000,0.000000,0.000000}%
\pgfsetfillcolor{currentfill}%
\pgfsetlinewidth{0.602250pt}%
\definecolor{currentstroke}{rgb}{0.000000,0.000000,0.000000}%
\pgfsetstrokecolor{currentstroke}%
\pgfsetdash{}{0pt}%
\pgfsys@defobject{currentmarker}{\pgfqpoint{0.000000in}{-0.027778in}}{\pgfqpoint{0.000000in}{0.000000in}}{%
\pgfpathmoveto{\pgfqpoint{0.000000in}{0.000000in}}%
\pgfpathlineto{\pgfqpoint{0.000000in}{-0.027778in}}%
\pgfusepath{stroke,fill}%
}%
\begin{pgfscope}%
\pgfsys@transformshift{2.229452in}{0.417642in}%
\pgfsys@useobject{currentmarker}{}%
\end{pgfscope}%
\end{pgfscope}%
\begin{pgfscope}%
\pgfpathrectangle{\pgfqpoint{0.594525in}{0.417642in}}{\pgfqpoint{3.345963in}{2.050688in}}%
\pgfusepath{clip}%
\pgfsetrectcap%
\pgfsetroundjoin%
\pgfsetlinewidth{0.803000pt}%
\definecolor{currentstroke}{rgb}{0.850000,0.850000,0.850000}%
\pgfsetstrokecolor{currentstroke}%
\pgfsetdash{}{0pt}%
\pgfpathmoveto{\pgfqpoint{2.310425in}{0.417642in}}%
\pgfpathlineto{\pgfqpoint{2.310425in}{2.468330in}}%
\pgfusepath{stroke}%
\end{pgfscope}%
\begin{pgfscope}%
\pgfsetbuttcap%
\pgfsetroundjoin%
\definecolor{currentfill}{rgb}{0.000000,0.000000,0.000000}%
\pgfsetfillcolor{currentfill}%
\pgfsetlinewidth{0.602250pt}%
\definecolor{currentstroke}{rgb}{0.000000,0.000000,0.000000}%
\pgfsetstrokecolor{currentstroke}%
\pgfsetdash{}{0pt}%
\pgfsys@defobject{currentmarker}{\pgfqpoint{0.000000in}{-0.027778in}}{\pgfqpoint{0.000000in}{0.000000in}}{%
\pgfpathmoveto{\pgfqpoint{0.000000in}{0.000000in}}%
\pgfpathlineto{\pgfqpoint{0.000000in}{-0.027778in}}%
\pgfusepath{stroke,fill}%
}%
\begin{pgfscope}%
\pgfsys@transformshift{2.310425in}{0.417642in}%
\pgfsys@useobject{currentmarker}{}%
\end{pgfscope}%
\end{pgfscope}%
\begin{pgfscope}%
\pgfpathrectangle{\pgfqpoint{0.594525in}{0.417642in}}{\pgfqpoint{3.345963in}{2.050688in}}%
\pgfusepath{clip}%
\pgfsetrectcap%
\pgfsetroundjoin%
\pgfsetlinewidth{0.803000pt}%
\definecolor{currentstroke}{rgb}{0.850000,0.850000,0.850000}%
\pgfsetstrokecolor{currentstroke}%
\pgfsetdash{}{0pt}%
\pgfpathmoveto{\pgfqpoint{2.367876in}{0.417642in}}%
\pgfpathlineto{\pgfqpoint{2.367876in}{2.468330in}}%
\pgfusepath{stroke}%
\end{pgfscope}%
\begin{pgfscope}%
\pgfsetbuttcap%
\pgfsetroundjoin%
\definecolor{currentfill}{rgb}{0.000000,0.000000,0.000000}%
\pgfsetfillcolor{currentfill}%
\pgfsetlinewidth{0.602250pt}%
\definecolor{currentstroke}{rgb}{0.000000,0.000000,0.000000}%
\pgfsetstrokecolor{currentstroke}%
\pgfsetdash{}{0pt}%
\pgfsys@defobject{currentmarker}{\pgfqpoint{0.000000in}{-0.027778in}}{\pgfqpoint{0.000000in}{0.000000in}}{%
\pgfpathmoveto{\pgfqpoint{0.000000in}{0.000000in}}%
\pgfpathlineto{\pgfqpoint{0.000000in}{-0.027778in}}%
\pgfusepath{stroke,fill}%
}%
\begin{pgfscope}%
\pgfsys@transformshift{2.367876in}{0.417642in}%
\pgfsys@useobject{currentmarker}{}%
\end{pgfscope}%
\end{pgfscope}%
\begin{pgfscope}%
\pgfpathrectangle{\pgfqpoint{0.594525in}{0.417642in}}{\pgfqpoint{3.345963in}{2.050688in}}%
\pgfusepath{clip}%
\pgfsetrectcap%
\pgfsetroundjoin%
\pgfsetlinewidth{0.803000pt}%
\definecolor{currentstroke}{rgb}{0.850000,0.850000,0.850000}%
\pgfsetstrokecolor{currentstroke}%
\pgfsetdash{}{0pt}%
\pgfpathmoveto{\pgfqpoint{2.412439in}{0.417642in}}%
\pgfpathlineto{\pgfqpoint{2.412439in}{2.468330in}}%
\pgfusepath{stroke}%
\end{pgfscope}%
\begin{pgfscope}%
\pgfsetbuttcap%
\pgfsetroundjoin%
\definecolor{currentfill}{rgb}{0.000000,0.000000,0.000000}%
\pgfsetfillcolor{currentfill}%
\pgfsetlinewidth{0.602250pt}%
\definecolor{currentstroke}{rgb}{0.000000,0.000000,0.000000}%
\pgfsetstrokecolor{currentstroke}%
\pgfsetdash{}{0pt}%
\pgfsys@defobject{currentmarker}{\pgfqpoint{0.000000in}{-0.027778in}}{\pgfqpoint{0.000000in}{0.000000in}}{%
\pgfpathmoveto{\pgfqpoint{0.000000in}{0.000000in}}%
\pgfpathlineto{\pgfqpoint{0.000000in}{-0.027778in}}%
\pgfusepath{stroke,fill}%
}%
\begin{pgfscope}%
\pgfsys@transformshift{2.412439in}{0.417642in}%
\pgfsys@useobject{currentmarker}{}%
\end{pgfscope}%
\end{pgfscope}%
\begin{pgfscope}%
\pgfpathrectangle{\pgfqpoint{0.594525in}{0.417642in}}{\pgfqpoint{3.345963in}{2.050688in}}%
\pgfusepath{clip}%
\pgfsetrectcap%
\pgfsetroundjoin%
\pgfsetlinewidth{0.803000pt}%
\definecolor{currentstroke}{rgb}{0.850000,0.850000,0.850000}%
\pgfsetstrokecolor{currentstroke}%
\pgfsetdash{}{0pt}%
\pgfpathmoveto{\pgfqpoint{2.448849in}{0.417642in}}%
\pgfpathlineto{\pgfqpoint{2.448849in}{2.468330in}}%
\pgfusepath{stroke}%
\end{pgfscope}%
\begin{pgfscope}%
\pgfsetbuttcap%
\pgfsetroundjoin%
\definecolor{currentfill}{rgb}{0.000000,0.000000,0.000000}%
\pgfsetfillcolor{currentfill}%
\pgfsetlinewidth{0.602250pt}%
\definecolor{currentstroke}{rgb}{0.000000,0.000000,0.000000}%
\pgfsetstrokecolor{currentstroke}%
\pgfsetdash{}{0pt}%
\pgfsys@defobject{currentmarker}{\pgfqpoint{0.000000in}{-0.027778in}}{\pgfqpoint{0.000000in}{0.000000in}}{%
\pgfpathmoveto{\pgfqpoint{0.000000in}{0.000000in}}%
\pgfpathlineto{\pgfqpoint{0.000000in}{-0.027778in}}%
\pgfusepath{stroke,fill}%
}%
\begin{pgfscope}%
\pgfsys@transformshift{2.448849in}{0.417642in}%
\pgfsys@useobject{currentmarker}{}%
\end{pgfscope}%
\end{pgfscope}%
\begin{pgfscope}%
\pgfpathrectangle{\pgfqpoint{0.594525in}{0.417642in}}{\pgfqpoint{3.345963in}{2.050688in}}%
\pgfusepath{clip}%
\pgfsetrectcap%
\pgfsetroundjoin%
\pgfsetlinewidth{0.803000pt}%
\definecolor{currentstroke}{rgb}{0.850000,0.850000,0.850000}%
\pgfsetstrokecolor{currentstroke}%
\pgfsetdash{}{0pt}%
\pgfpathmoveto{\pgfqpoint{2.479633in}{0.417642in}}%
\pgfpathlineto{\pgfqpoint{2.479633in}{2.468330in}}%
\pgfusepath{stroke}%
\end{pgfscope}%
\begin{pgfscope}%
\pgfsetbuttcap%
\pgfsetroundjoin%
\definecolor{currentfill}{rgb}{0.000000,0.000000,0.000000}%
\pgfsetfillcolor{currentfill}%
\pgfsetlinewidth{0.602250pt}%
\definecolor{currentstroke}{rgb}{0.000000,0.000000,0.000000}%
\pgfsetstrokecolor{currentstroke}%
\pgfsetdash{}{0pt}%
\pgfsys@defobject{currentmarker}{\pgfqpoint{0.000000in}{-0.027778in}}{\pgfqpoint{0.000000in}{0.000000in}}{%
\pgfpathmoveto{\pgfqpoint{0.000000in}{0.000000in}}%
\pgfpathlineto{\pgfqpoint{0.000000in}{-0.027778in}}%
\pgfusepath{stroke,fill}%
}%
\begin{pgfscope}%
\pgfsys@transformshift{2.479633in}{0.417642in}%
\pgfsys@useobject{currentmarker}{}%
\end{pgfscope}%
\end{pgfscope}%
\begin{pgfscope}%
\pgfpathrectangle{\pgfqpoint{0.594525in}{0.417642in}}{\pgfqpoint{3.345963in}{2.050688in}}%
\pgfusepath{clip}%
\pgfsetrectcap%
\pgfsetroundjoin%
\pgfsetlinewidth{0.803000pt}%
\definecolor{currentstroke}{rgb}{0.850000,0.850000,0.850000}%
\pgfsetstrokecolor{currentstroke}%
\pgfsetdash{}{0pt}%
\pgfpathmoveto{\pgfqpoint{2.506300in}{0.417642in}}%
\pgfpathlineto{\pgfqpoint{2.506300in}{2.468330in}}%
\pgfusepath{stroke}%
\end{pgfscope}%
\begin{pgfscope}%
\pgfsetbuttcap%
\pgfsetroundjoin%
\definecolor{currentfill}{rgb}{0.000000,0.000000,0.000000}%
\pgfsetfillcolor{currentfill}%
\pgfsetlinewidth{0.602250pt}%
\definecolor{currentstroke}{rgb}{0.000000,0.000000,0.000000}%
\pgfsetstrokecolor{currentstroke}%
\pgfsetdash{}{0pt}%
\pgfsys@defobject{currentmarker}{\pgfqpoint{0.000000in}{-0.027778in}}{\pgfqpoint{0.000000in}{0.000000in}}{%
\pgfpathmoveto{\pgfqpoint{0.000000in}{0.000000in}}%
\pgfpathlineto{\pgfqpoint{0.000000in}{-0.027778in}}%
\pgfusepath{stroke,fill}%
}%
\begin{pgfscope}%
\pgfsys@transformshift{2.506300in}{0.417642in}%
\pgfsys@useobject{currentmarker}{}%
\end{pgfscope}%
\end{pgfscope}%
\begin{pgfscope}%
\pgfpathrectangle{\pgfqpoint{0.594525in}{0.417642in}}{\pgfqpoint{3.345963in}{2.050688in}}%
\pgfusepath{clip}%
\pgfsetrectcap%
\pgfsetroundjoin%
\pgfsetlinewidth{0.803000pt}%
\definecolor{currentstroke}{rgb}{0.850000,0.850000,0.850000}%
\pgfsetstrokecolor{currentstroke}%
\pgfsetdash{}{0pt}%
\pgfpathmoveto{\pgfqpoint{2.529822in}{0.417642in}}%
\pgfpathlineto{\pgfqpoint{2.529822in}{2.468330in}}%
\pgfusepath{stroke}%
\end{pgfscope}%
\begin{pgfscope}%
\pgfsetbuttcap%
\pgfsetroundjoin%
\definecolor{currentfill}{rgb}{0.000000,0.000000,0.000000}%
\pgfsetfillcolor{currentfill}%
\pgfsetlinewidth{0.602250pt}%
\definecolor{currentstroke}{rgb}{0.000000,0.000000,0.000000}%
\pgfsetstrokecolor{currentstroke}%
\pgfsetdash{}{0pt}%
\pgfsys@defobject{currentmarker}{\pgfqpoint{0.000000in}{-0.027778in}}{\pgfqpoint{0.000000in}{0.000000in}}{%
\pgfpathmoveto{\pgfqpoint{0.000000in}{0.000000in}}%
\pgfpathlineto{\pgfqpoint{0.000000in}{-0.027778in}}%
\pgfusepath{stroke,fill}%
}%
\begin{pgfscope}%
\pgfsys@transformshift{2.529822in}{0.417642in}%
\pgfsys@useobject{currentmarker}{}%
\end{pgfscope}%
\end{pgfscope}%
\begin{pgfscope}%
\pgfpathrectangle{\pgfqpoint{0.594525in}{0.417642in}}{\pgfqpoint{3.345963in}{2.050688in}}%
\pgfusepath{clip}%
\pgfsetrectcap%
\pgfsetroundjoin%
\pgfsetlinewidth{0.803000pt}%
\definecolor{currentstroke}{rgb}{0.850000,0.850000,0.850000}%
\pgfsetstrokecolor{currentstroke}%
\pgfsetdash{}{0pt}%
\pgfpathmoveto{\pgfqpoint{2.689287in}{0.417642in}}%
\pgfpathlineto{\pgfqpoint{2.689287in}{2.468330in}}%
\pgfusepath{stroke}%
\end{pgfscope}%
\begin{pgfscope}%
\pgfsetbuttcap%
\pgfsetroundjoin%
\definecolor{currentfill}{rgb}{0.000000,0.000000,0.000000}%
\pgfsetfillcolor{currentfill}%
\pgfsetlinewidth{0.602250pt}%
\definecolor{currentstroke}{rgb}{0.000000,0.000000,0.000000}%
\pgfsetstrokecolor{currentstroke}%
\pgfsetdash{}{0pt}%
\pgfsys@defobject{currentmarker}{\pgfqpoint{0.000000in}{-0.027778in}}{\pgfqpoint{0.000000in}{0.000000in}}{%
\pgfpathmoveto{\pgfqpoint{0.000000in}{0.000000in}}%
\pgfpathlineto{\pgfqpoint{0.000000in}{-0.027778in}}%
\pgfusepath{stroke,fill}%
}%
\begin{pgfscope}%
\pgfsys@transformshift{2.689287in}{0.417642in}%
\pgfsys@useobject{currentmarker}{}%
\end{pgfscope}%
\end{pgfscope}%
\begin{pgfscope}%
\pgfpathrectangle{\pgfqpoint{0.594525in}{0.417642in}}{\pgfqpoint{3.345963in}{2.050688in}}%
\pgfusepath{clip}%
\pgfsetrectcap%
\pgfsetroundjoin%
\pgfsetlinewidth{0.803000pt}%
\definecolor{currentstroke}{rgb}{0.850000,0.850000,0.850000}%
\pgfsetstrokecolor{currentstroke}%
\pgfsetdash{}{0pt}%
\pgfpathmoveto{\pgfqpoint{2.770260in}{0.417642in}}%
\pgfpathlineto{\pgfqpoint{2.770260in}{2.468330in}}%
\pgfusepath{stroke}%
\end{pgfscope}%
\begin{pgfscope}%
\pgfsetbuttcap%
\pgfsetroundjoin%
\definecolor{currentfill}{rgb}{0.000000,0.000000,0.000000}%
\pgfsetfillcolor{currentfill}%
\pgfsetlinewidth{0.602250pt}%
\definecolor{currentstroke}{rgb}{0.000000,0.000000,0.000000}%
\pgfsetstrokecolor{currentstroke}%
\pgfsetdash{}{0pt}%
\pgfsys@defobject{currentmarker}{\pgfqpoint{0.000000in}{-0.027778in}}{\pgfqpoint{0.000000in}{0.000000in}}{%
\pgfpathmoveto{\pgfqpoint{0.000000in}{0.000000in}}%
\pgfpathlineto{\pgfqpoint{0.000000in}{-0.027778in}}%
\pgfusepath{stroke,fill}%
}%
\begin{pgfscope}%
\pgfsys@transformshift{2.770260in}{0.417642in}%
\pgfsys@useobject{currentmarker}{}%
\end{pgfscope}%
\end{pgfscope}%
\begin{pgfscope}%
\pgfpathrectangle{\pgfqpoint{0.594525in}{0.417642in}}{\pgfqpoint{3.345963in}{2.050688in}}%
\pgfusepath{clip}%
\pgfsetrectcap%
\pgfsetroundjoin%
\pgfsetlinewidth{0.803000pt}%
\definecolor{currentstroke}{rgb}{0.850000,0.850000,0.850000}%
\pgfsetstrokecolor{currentstroke}%
\pgfsetdash{}{0pt}%
\pgfpathmoveto{\pgfqpoint{2.827711in}{0.417642in}}%
\pgfpathlineto{\pgfqpoint{2.827711in}{2.468330in}}%
\pgfusepath{stroke}%
\end{pgfscope}%
\begin{pgfscope}%
\pgfsetbuttcap%
\pgfsetroundjoin%
\definecolor{currentfill}{rgb}{0.000000,0.000000,0.000000}%
\pgfsetfillcolor{currentfill}%
\pgfsetlinewidth{0.602250pt}%
\definecolor{currentstroke}{rgb}{0.000000,0.000000,0.000000}%
\pgfsetstrokecolor{currentstroke}%
\pgfsetdash{}{0pt}%
\pgfsys@defobject{currentmarker}{\pgfqpoint{0.000000in}{-0.027778in}}{\pgfqpoint{0.000000in}{0.000000in}}{%
\pgfpathmoveto{\pgfqpoint{0.000000in}{0.000000in}}%
\pgfpathlineto{\pgfqpoint{0.000000in}{-0.027778in}}%
\pgfusepath{stroke,fill}%
}%
\begin{pgfscope}%
\pgfsys@transformshift{2.827711in}{0.417642in}%
\pgfsys@useobject{currentmarker}{}%
\end{pgfscope}%
\end{pgfscope}%
\begin{pgfscope}%
\pgfpathrectangle{\pgfqpoint{0.594525in}{0.417642in}}{\pgfqpoint{3.345963in}{2.050688in}}%
\pgfusepath{clip}%
\pgfsetrectcap%
\pgfsetroundjoin%
\pgfsetlinewidth{0.803000pt}%
\definecolor{currentstroke}{rgb}{0.850000,0.850000,0.850000}%
\pgfsetstrokecolor{currentstroke}%
\pgfsetdash{}{0pt}%
\pgfpathmoveto{\pgfqpoint{2.872273in}{0.417642in}}%
\pgfpathlineto{\pgfqpoint{2.872273in}{2.468330in}}%
\pgfusepath{stroke}%
\end{pgfscope}%
\begin{pgfscope}%
\pgfsetbuttcap%
\pgfsetroundjoin%
\definecolor{currentfill}{rgb}{0.000000,0.000000,0.000000}%
\pgfsetfillcolor{currentfill}%
\pgfsetlinewidth{0.602250pt}%
\definecolor{currentstroke}{rgb}{0.000000,0.000000,0.000000}%
\pgfsetstrokecolor{currentstroke}%
\pgfsetdash{}{0pt}%
\pgfsys@defobject{currentmarker}{\pgfqpoint{0.000000in}{-0.027778in}}{\pgfqpoint{0.000000in}{0.000000in}}{%
\pgfpathmoveto{\pgfqpoint{0.000000in}{0.000000in}}%
\pgfpathlineto{\pgfqpoint{0.000000in}{-0.027778in}}%
\pgfusepath{stroke,fill}%
}%
\begin{pgfscope}%
\pgfsys@transformshift{2.872273in}{0.417642in}%
\pgfsys@useobject{currentmarker}{}%
\end{pgfscope}%
\end{pgfscope}%
\begin{pgfscope}%
\pgfpathrectangle{\pgfqpoint{0.594525in}{0.417642in}}{\pgfqpoint{3.345963in}{2.050688in}}%
\pgfusepath{clip}%
\pgfsetrectcap%
\pgfsetroundjoin%
\pgfsetlinewidth{0.803000pt}%
\definecolor{currentstroke}{rgb}{0.850000,0.850000,0.850000}%
\pgfsetstrokecolor{currentstroke}%
\pgfsetdash{}{0pt}%
\pgfpathmoveto{\pgfqpoint{2.908684in}{0.417642in}}%
\pgfpathlineto{\pgfqpoint{2.908684in}{2.468330in}}%
\pgfusepath{stroke}%
\end{pgfscope}%
\begin{pgfscope}%
\pgfsetbuttcap%
\pgfsetroundjoin%
\definecolor{currentfill}{rgb}{0.000000,0.000000,0.000000}%
\pgfsetfillcolor{currentfill}%
\pgfsetlinewidth{0.602250pt}%
\definecolor{currentstroke}{rgb}{0.000000,0.000000,0.000000}%
\pgfsetstrokecolor{currentstroke}%
\pgfsetdash{}{0pt}%
\pgfsys@defobject{currentmarker}{\pgfqpoint{0.000000in}{-0.027778in}}{\pgfqpoint{0.000000in}{0.000000in}}{%
\pgfpathmoveto{\pgfqpoint{0.000000in}{0.000000in}}%
\pgfpathlineto{\pgfqpoint{0.000000in}{-0.027778in}}%
\pgfusepath{stroke,fill}%
}%
\begin{pgfscope}%
\pgfsys@transformshift{2.908684in}{0.417642in}%
\pgfsys@useobject{currentmarker}{}%
\end{pgfscope}%
\end{pgfscope}%
\begin{pgfscope}%
\pgfpathrectangle{\pgfqpoint{0.594525in}{0.417642in}}{\pgfqpoint{3.345963in}{2.050688in}}%
\pgfusepath{clip}%
\pgfsetrectcap%
\pgfsetroundjoin%
\pgfsetlinewidth{0.803000pt}%
\definecolor{currentstroke}{rgb}{0.850000,0.850000,0.850000}%
\pgfsetstrokecolor{currentstroke}%
\pgfsetdash{}{0pt}%
\pgfpathmoveto{\pgfqpoint{2.939468in}{0.417642in}}%
\pgfpathlineto{\pgfqpoint{2.939468in}{2.468330in}}%
\pgfusepath{stroke}%
\end{pgfscope}%
\begin{pgfscope}%
\pgfsetbuttcap%
\pgfsetroundjoin%
\definecolor{currentfill}{rgb}{0.000000,0.000000,0.000000}%
\pgfsetfillcolor{currentfill}%
\pgfsetlinewidth{0.602250pt}%
\definecolor{currentstroke}{rgb}{0.000000,0.000000,0.000000}%
\pgfsetstrokecolor{currentstroke}%
\pgfsetdash{}{0pt}%
\pgfsys@defobject{currentmarker}{\pgfqpoint{0.000000in}{-0.027778in}}{\pgfqpoint{0.000000in}{0.000000in}}{%
\pgfpathmoveto{\pgfqpoint{0.000000in}{0.000000in}}%
\pgfpathlineto{\pgfqpoint{0.000000in}{-0.027778in}}%
\pgfusepath{stroke,fill}%
}%
\begin{pgfscope}%
\pgfsys@transformshift{2.939468in}{0.417642in}%
\pgfsys@useobject{currentmarker}{}%
\end{pgfscope}%
\end{pgfscope}%
\begin{pgfscope}%
\pgfpathrectangle{\pgfqpoint{0.594525in}{0.417642in}}{\pgfqpoint{3.345963in}{2.050688in}}%
\pgfusepath{clip}%
\pgfsetrectcap%
\pgfsetroundjoin%
\pgfsetlinewidth{0.803000pt}%
\definecolor{currentstroke}{rgb}{0.850000,0.850000,0.850000}%
\pgfsetstrokecolor{currentstroke}%
\pgfsetdash{}{0pt}%
\pgfpathmoveto{\pgfqpoint{2.966135in}{0.417642in}}%
\pgfpathlineto{\pgfqpoint{2.966135in}{2.468330in}}%
\pgfusepath{stroke}%
\end{pgfscope}%
\begin{pgfscope}%
\pgfsetbuttcap%
\pgfsetroundjoin%
\definecolor{currentfill}{rgb}{0.000000,0.000000,0.000000}%
\pgfsetfillcolor{currentfill}%
\pgfsetlinewidth{0.602250pt}%
\definecolor{currentstroke}{rgb}{0.000000,0.000000,0.000000}%
\pgfsetstrokecolor{currentstroke}%
\pgfsetdash{}{0pt}%
\pgfsys@defobject{currentmarker}{\pgfqpoint{0.000000in}{-0.027778in}}{\pgfqpoint{0.000000in}{0.000000in}}{%
\pgfpathmoveto{\pgfqpoint{0.000000in}{0.000000in}}%
\pgfpathlineto{\pgfqpoint{0.000000in}{-0.027778in}}%
\pgfusepath{stroke,fill}%
}%
\begin{pgfscope}%
\pgfsys@transformshift{2.966135in}{0.417642in}%
\pgfsys@useobject{currentmarker}{}%
\end{pgfscope}%
\end{pgfscope}%
\begin{pgfscope}%
\pgfpathrectangle{\pgfqpoint{0.594525in}{0.417642in}}{\pgfqpoint{3.345963in}{2.050688in}}%
\pgfusepath{clip}%
\pgfsetrectcap%
\pgfsetroundjoin%
\pgfsetlinewidth{0.803000pt}%
\definecolor{currentstroke}{rgb}{0.850000,0.850000,0.850000}%
\pgfsetstrokecolor{currentstroke}%
\pgfsetdash{}{0pt}%
\pgfpathmoveto{\pgfqpoint{2.989656in}{0.417642in}}%
\pgfpathlineto{\pgfqpoint{2.989656in}{2.468330in}}%
\pgfusepath{stroke}%
\end{pgfscope}%
\begin{pgfscope}%
\pgfsetbuttcap%
\pgfsetroundjoin%
\definecolor{currentfill}{rgb}{0.000000,0.000000,0.000000}%
\pgfsetfillcolor{currentfill}%
\pgfsetlinewidth{0.602250pt}%
\definecolor{currentstroke}{rgb}{0.000000,0.000000,0.000000}%
\pgfsetstrokecolor{currentstroke}%
\pgfsetdash{}{0pt}%
\pgfsys@defobject{currentmarker}{\pgfqpoint{0.000000in}{-0.027778in}}{\pgfqpoint{0.000000in}{0.000000in}}{%
\pgfpathmoveto{\pgfqpoint{0.000000in}{0.000000in}}%
\pgfpathlineto{\pgfqpoint{0.000000in}{-0.027778in}}%
\pgfusepath{stroke,fill}%
}%
\begin{pgfscope}%
\pgfsys@transformshift{2.989656in}{0.417642in}%
\pgfsys@useobject{currentmarker}{}%
\end{pgfscope}%
\end{pgfscope}%
\begin{pgfscope}%
\pgfpathrectangle{\pgfqpoint{0.594525in}{0.417642in}}{\pgfqpoint{3.345963in}{2.050688in}}%
\pgfusepath{clip}%
\pgfsetrectcap%
\pgfsetroundjoin%
\pgfsetlinewidth{0.803000pt}%
\definecolor{currentstroke}{rgb}{0.850000,0.850000,0.850000}%
\pgfsetstrokecolor{currentstroke}%
\pgfsetdash{}{0pt}%
\pgfpathmoveto{\pgfqpoint{3.149121in}{0.417642in}}%
\pgfpathlineto{\pgfqpoint{3.149121in}{2.468330in}}%
\pgfusepath{stroke}%
\end{pgfscope}%
\begin{pgfscope}%
\pgfsetbuttcap%
\pgfsetroundjoin%
\definecolor{currentfill}{rgb}{0.000000,0.000000,0.000000}%
\pgfsetfillcolor{currentfill}%
\pgfsetlinewidth{0.602250pt}%
\definecolor{currentstroke}{rgb}{0.000000,0.000000,0.000000}%
\pgfsetstrokecolor{currentstroke}%
\pgfsetdash{}{0pt}%
\pgfsys@defobject{currentmarker}{\pgfqpoint{0.000000in}{-0.027778in}}{\pgfqpoint{0.000000in}{0.000000in}}{%
\pgfpathmoveto{\pgfqpoint{0.000000in}{0.000000in}}%
\pgfpathlineto{\pgfqpoint{0.000000in}{-0.027778in}}%
\pgfusepath{stroke,fill}%
}%
\begin{pgfscope}%
\pgfsys@transformshift{3.149121in}{0.417642in}%
\pgfsys@useobject{currentmarker}{}%
\end{pgfscope}%
\end{pgfscope}%
\begin{pgfscope}%
\pgfpathrectangle{\pgfqpoint{0.594525in}{0.417642in}}{\pgfqpoint{3.345963in}{2.050688in}}%
\pgfusepath{clip}%
\pgfsetrectcap%
\pgfsetroundjoin%
\pgfsetlinewidth{0.803000pt}%
\definecolor{currentstroke}{rgb}{0.850000,0.850000,0.850000}%
\pgfsetstrokecolor{currentstroke}%
\pgfsetdash{}{0pt}%
\pgfpathmoveto{\pgfqpoint{3.230094in}{0.417642in}}%
\pgfpathlineto{\pgfqpoint{3.230094in}{2.468330in}}%
\pgfusepath{stroke}%
\end{pgfscope}%
\begin{pgfscope}%
\pgfsetbuttcap%
\pgfsetroundjoin%
\definecolor{currentfill}{rgb}{0.000000,0.000000,0.000000}%
\pgfsetfillcolor{currentfill}%
\pgfsetlinewidth{0.602250pt}%
\definecolor{currentstroke}{rgb}{0.000000,0.000000,0.000000}%
\pgfsetstrokecolor{currentstroke}%
\pgfsetdash{}{0pt}%
\pgfsys@defobject{currentmarker}{\pgfqpoint{0.000000in}{-0.027778in}}{\pgfqpoint{0.000000in}{0.000000in}}{%
\pgfpathmoveto{\pgfqpoint{0.000000in}{0.000000in}}%
\pgfpathlineto{\pgfqpoint{0.000000in}{-0.027778in}}%
\pgfusepath{stroke,fill}%
}%
\begin{pgfscope}%
\pgfsys@transformshift{3.230094in}{0.417642in}%
\pgfsys@useobject{currentmarker}{}%
\end{pgfscope}%
\end{pgfscope}%
\begin{pgfscope}%
\pgfpathrectangle{\pgfqpoint{0.594525in}{0.417642in}}{\pgfqpoint{3.345963in}{2.050688in}}%
\pgfusepath{clip}%
\pgfsetrectcap%
\pgfsetroundjoin%
\pgfsetlinewidth{0.803000pt}%
\definecolor{currentstroke}{rgb}{0.850000,0.850000,0.850000}%
\pgfsetstrokecolor{currentstroke}%
\pgfsetdash{}{0pt}%
\pgfpathmoveto{\pgfqpoint{3.287545in}{0.417642in}}%
\pgfpathlineto{\pgfqpoint{3.287545in}{2.468330in}}%
\pgfusepath{stroke}%
\end{pgfscope}%
\begin{pgfscope}%
\pgfsetbuttcap%
\pgfsetroundjoin%
\definecolor{currentfill}{rgb}{0.000000,0.000000,0.000000}%
\pgfsetfillcolor{currentfill}%
\pgfsetlinewidth{0.602250pt}%
\definecolor{currentstroke}{rgb}{0.000000,0.000000,0.000000}%
\pgfsetstrokecolor{currentstroke}%
\pgfsetdash{}{0pt}%
\pgfsys@defobject{currentmarker}{\pgfqpoint{0.000000in}{-0.027778in}}{\pgfqpoint{0.000000in}{0.000000in}}{%
\pgfpathmoveto{\pgfqpoint{0.000000in}{0.000000in}}%
\pgfpathlineto{\pgfqpoint{0.000000in}{-0.027778in}}%
\pgfusepath{stroke,fill}%
}%
\begin{pgfscope}%
\pgfsys@transformshift{3.287545in}{0.417642in}%
\pgfsys@useobject{currentmarker}{}%
\end{pgfscope}%
\end{pgfscope}%
\begin{pgfscope}%
\pgfpathrectangle{\pgfqpoint{0.594525in}{0.417642in}}{\pgfqpoint{3.345963in}{2.050688in}}%
\pgfusepath{clip}%
\pgfsetrectcap%
\pgfsetroundjoin%
\pgfsetlinewidth{0.803000pt}%
\definecolor{currentstroke}{rgb}{0.850000,0.850000,0.850000}%
\pgfsetstrokecolor{currentstroke}%
\pgfsetdash{}{0pt}%
\pgfpathmoveto{\pgfqpoint{3.332108in}{0.417642in}}%
\pgfpathlineto{\pgfqpoint{3.332108in}{2.468330in}}%
\pgfusepath{stroke}%
\end{pgfscope}%
\begin{pgfscope}%
\pgfsetbuttcap%
\pgfsetroundjoin%
\definecolor{currentfill}{rgb}{0.000000,0.000000,0.000000}%
\pgfsetfillcolor{currentfill}%
\pgfsetlinewidth{0.602250pt}%
\definecolor{currentstroke}{rgb}{0.000000,0.000000,0.000000}%
\pgfsetstrokecolor{currentstroke}%
\pgfsetdash{}{0pt}%
\pgfsys@defobject{currentmarker}{\pgfqpoint{0.000000in}{-0.027778in}}{\pgfqpoint{0.000000in}{0.000000in}}{%
\pgfpathmoveto{\pgfqpoint{0.000000in}{0.000000in}}%
\pgfpathlineto{\pgfqpoint{0.000000in}{-0.027778in}}%
\pgfusepath{stroke,fill}%
}%
\begin{pgfscope}%
\pgfsys@transformshift{3.332108in}{0.417642in}%
\pgfsys@useobject{currentmarker}{}%
\end{pgfscope}%
\end{pgfscope}%
\begin{pgfscope}%
\pgfpathrectangle{\pgfqpoint{0.594525in}{0.417642in}}{\pgfqpoint{3.345963in}{2.050688in}}%
\pgfusepath{clip}%
\pgfsetrectcap%
\pgfsetroundjoin%
\pgfsetlinewidth{0.803000pt}%
\definecolor{currentstroke}{rgb}{0.850000,0.850000,0.850000}%
\pgfsetstrokecolor{currentstroke}%
\pgfsetdash{}{0pt}%
\pgfpathmoveto{\pgfqpoint{3.368518in}{0.417642in}}%
\pgfpathlineto{\pgfqpoint{3.368518in}{2.468330in}}%
\pgfusepath{stroke}%
\end{pgfscope}%
\begin{pgfscope}%
\pgfsetbuttcap%
\pgfsetroundjoin%
\definecolor{currentfill}{rgb}{0.000000,0.000000,0.000000}%
\pgfsetfillcolor{currentfill}%
\pgfsetlinewidth{0.602250pt}%
\definecolor{currentstroke}{rgb}{0.000000,0.000000,0.000000}%
\pgfsetstrokecolor{currentstroke}%
\pgfsetdash{}{0pt}%
\pgfsys@defobject{currentmarker}{\pgfqpoint{0.000000in}{-0.027778in}}{\pgfqpoint{0.000000in}{0.000000in}}{%
\pgfpathmoveto{\pgfqpoint{0.000000in}{0.000000in}}%
\pgfpathlineto{\pgfqpoint{0.000000in}{-0.027778in}}%
\pgfusepath{stroke,fill}%
}%
\begin{pgfscope}%
\pgfsys@transformshift{3.368518in}{0.417642in}%
\pgfsys@useobject{currentmarker}{}%
\end{pgfscope}%
\end{pgfscope}%
\begin{pgfscope}%
\pgfpathrectangle{\pgfqpoint{0.594525in}{0.417642in}}{\pgfqpoint{3.345963in}{2.050688in}}%
\pgfusepath{clip}%
\pgfsetrectcap%
\pgfsetroundjoin%
\pgfsetlinewidth{0.803000pt}%
\definecolor{currentstroke}{rgb}{0.850000,0.850000,0.850000}%
\pgfsetstrokecolor{currentstroke}%
\pgfsetdash{}{0pt}%
\pgfpathmoveto{\pgfqpoint{3.399303in}{0.417642in}}%
\pgfpathlineto{\pgfqpoint{3.399303in}{2.468330in}}%
\pgfusepath{stroke}%
\end{pgfscope}%
\begin{pgfscope}%
\pgfsetbuttcap%
\pgfsetroundjoin%
\definecolor{currentfill}{rgb}{0.000000,0.000000,0.000000}%
\pgfsetfillcolor{currentfill}%
\pgfsetlinewidth{0.602250pt}%
\definecolor{currentstroke}{rgb}{0.000000,0.000000,0.000000}%
\pgfsetstrokecolor{currentstroke}%
\pgfsetdash{}{0pt}%
\pgfsys@defobject{currentmarker}{\pgfqpoint{0.000000in}{-0.027778in}}{\pgfqpoint{0.000000in}{0.000000in}}{%
\pgfpathmoveto{\pgfqpoint{0.000000in}{0.000000in}}%
\pgfpathlineto{\pgfqpoint{0.000000in}{-0.027778in}}%
\pgfusepath{stroke,fill}%
}%
\begin{pgfscope}%
\pgfsys@transformshift{3.399303in}{0.417642in}%
\pgfsys@useobject{currentmarker}{}%
\end{pgfscope}%
\end{pgfscope}%
\begin{pgfscope}%
\pgfpathrectangle{\pgfqpoint{0.594525in}{0.417642in}}{\pgfqpoint{3.345963in}{2.050688in}}%
\pgfusepath{clip}%
\pgfsetrectcap%
\pgfsetroundjoin%
\pgfsetlinewidth{0.803000pt}%
\definecolor{currentstroke}{rgb}{0.850000,0.850000,0.850000}%
\pgfsetstrokecolor{currentstroke}%
\pgfsetdash{}{0pt}%
\pgfpathmoveto{\pgfqpoint{3.425969in}{0.417642in}}%
\pgfpathlineto{\pgfqpoint{3.425969in}{2.468330in}}%
\pgfusepath{stroke}%
\end{pgfscope}%
\begin{pgfscope}%
\pgfsetbuttcap%
\pgfsetroundjoin%
\definecolor{currentfill}{rgb}{0.000000,0.000000,0.000000}%
\pgfsetfillcolor{currentfill}%
\pgfsetlinewidth{0.602250pt}%
\definecolor{currentstroke}{rgb}{0.000000,0.000000,0.000000}%
\pgfsetstrokecolor{currentstroke}%
\pgfsetdash{}{0pt}%
\pgfsys@defobject{currentmarker}{\pgfqpoint{0.000000in}{-0.027778in}}{\pgfqpoint{0.000000in}{0.000000in}}{%
\pgfpathmoveto{\pgfqpoint{0.000000in}{0.000000in}}%
\pgfpathlineto{\pgfqpoint{0.000000in}{-0.027778in}}%
\pgfusepath{stroke,fill}%
}%
\begin{pgfscope}%
\pgfsys@transformshift{3.425969in}{0.417642in}%
\pgfsys@useobject{currentmarker}{}%
\end{pgfscope}%
\end{pgfscope}%
\begin{pgfscope}%
\pgfpathrectangle{\pgfqpoint{0.594525in}{0.417642in}}{\pgfqpoint{3.345963in}{2.050688in}}%
\pgfusepath{clip}%
\pgfsetrectcap%
\pgfsetroundjoin%
\pgfsetlinewidth{0.803000pt}%
\definecolor{currentstroke}{rgb}{0.850000,0.850000,0.850000}%
\pgfsetstrokecolor{currentstroke}%
\pgfsetdash{}{0pt}%
\pgfpathmoveto{\pgfqpoint{3.449491in}{0.417642in}}%
\pgfpathlineto{\pgfqpoint{3.449491in}{2.468330in}}%
\pgfusepath{stroke}%
\end{pgfscope}%
\begin{pgfscope}%
\pgfsetbuttcap%
\pgfsetroundjoin%
\definecolor{currentfill}{rgb}{0.000000,0.000000,0.000000}%
\pgfsetfillcolor{currentfill}%
\pgfsetlinewidth{0.602250pt}%
\definecolor{currentstroke}{rgb}{0.000000,0.000000,0.000000}%
\pgfsetstrokecolor{currentstroke}%
\pgfsetdash{}{0pt}%
\pgfsys@defobject{currentmarker}{\pgfqpoint{0.000000in}{-0.027778in}}{\pgfqpoint{0.000000in}{0.000000in}}{%
\pgfpathmoveto{\pgfqpoint{0.000000in}{0.000000in}}%
\pgfpathlineto{\pgfqpoint{0.000000in}{-0.027778in}}%
\pgfusepath{stroke,fill}%
}%
\begin{pgfscope}%
\pgfsys@transformshift{3.449491in}{0.417642in}%
\pgfsys@useobject{currentmarker}{}%
\end{pgfscope}%
\end{pgfscope}%
\begin{pgfscope}%
\pgfpathrectangle{\pgfqpoint{0.594525in}{0.417642in}}{\pgfqpoint{3.345963in}{2.050688in}}%
\pgfusepath{clip}%
\pgfsetrectcap%
\pgfsetroundjoin%
\pgfsetlinewidth{0.803000pt}%
\definecolor{currentstroke}{rgb}{0.850000,0.850000,0.850000}%
\pgfsetstrokecolor{currentstroke}%
\pgfsetdash{}{0pt}%
\pgfpathmoveto{\pgfqpoint{3.608956in}{0.417642in}}%
\pgfpathlineto{\pgfqpoint{3.608956in}{2.468330in}}%
\pgfusepath{stroke}%
\end{pgfscope}%
\begin{pgfscope}%
\pgfsetbuttcap%
\pgfsetroundjoin%
\definecolor{currentfill}{rgb}{0.000000,0.000000,0.000000}%
\pgfsetfillcolor{currentfill}%
\pgfsetlinewidth{0.602250pt}%
\definecolor{currentstroke}{rgb}{0.000000,0.000000,0.000000}%
\pgfsetstrokecolor{currentstroke}%
\pgfsetdash{}{0pt}%
\pgfsys@defobject{currentmarker}{\pgfqpoint{0.000000in}{-0.027778in}}{\pgfqpoint{0.000000in}{0.000000in}}{%
\pgfpathmoveto{\pgfqpoint{0.000000in}{0.000000in}}%
\pgfpathlineto{\pgfqpoint{0.000000in}{-0.027778in}}%
\pgfusepath{stroke,fill}%
}%
\begin{pgfscope}%
\pgfsys@transformshift{3.608956in}{0.417642in}%
\pgfsys@useobject{currentmarker}{}%
\end{pgfscope}%
\end{pgfscope}%
\begin{pgfscope}%
\pgfpathrectangle{\pgfqpoint{0.594525in}{0.417642in}}{\pgfqpoint{3.345963in}{2.050688in}}%
\pgfusepath{clip}%
\pgfsetrectcap%
\pgfsetroundjoin%
\pgfsetlinewidth{0.803000pt}%
\definecolor{currentstroke}{rgb}{0.850000,0.850000,0.850000}%
\pgfsetstrokecolor{currentstroke}%
\pgfsetdash{}{0pt}%
\pgfpathmoveto{\pgfqpoint{3.689929in}{0.417642in}}%
\pgfpathlineto{\pgfqpoint{3.689929in}{2.468330in}}%
\pgfusepath{stroke}%
\end{pgfscope}%
\begin{pgfscope}%
\pgfsetbuttcap%
\pgfsetroundjoin%
\definecolor{currentfill}{rgb}{0.000000,0.000000,0.000000}%
\pgfsetfillcolor{currentfill}%
\pgfsetlinewidth{0.602250pt}%
\definecolor{currentstroke}{rgb}{0.000000,0.000000,0.000000}%
\pgfsetstrokecolor{currentstroke}%
\pgfsetdash{}{0pt}%
\pgfsys@defobject{currentmarker}{\pgfqpoint{0.000000in}{-0.027778in}}{\pgfqpoint{0.000000in}{0.000000in}}{%
\pgfpathmoveto{\pgfqpoint{0.000000in}{0.000000in}}%
\pgfpathlineto{\pgfqpoint{0.000000in}{-0.027778in}}%
\pgfusepath{stroke,fill}%
}%
\begin{pgfscope}%
\pgfsys@transformshift{3.689929in}{0.417642in}%
\pgfsys@useobject{currentmarker}{}%
\end{pgfscope}%
\end{pgfscope}%
\begin{pgfscope}%
\pgfpathrectangle{\pgfqpoint{0.594525in}{0.417642in}}{\pgfqpoint{3.345963in}{2.050688in}}%
\pgfusepath{clip}%
\pgfsetrectcap%
\pgfsetroundjoin%
\pgfsetlinewidth{0.803000pt}%
\definecolor{currentstroke}{rgb}{0.850000,0.850000,0.850000}%
\pgfsetstrokecolor{currentstroke}%
\pgfsetdash{}{0pt}%
\pgfpathmoveto{\pgfqpoint{3.747380in}{0.417642in}}%
\pgfpathlineto{\pgfqpoint{3.747380in}{2.468330in}}%
\pgfusepath{stroke}%
\end{pgfscope}%
\begin{pgfscope}%
\pgfsetbuttcap%
\pgfsetroundjoin%
\definecolor{currentfill}{rgb}{0.000000,0.000000,0.000000}%
\pgfsetfillcolor{currentfill}%
\pgfsetlinewidth{0.602250pt}%
\definecolor{currentstroke}{rgb}{0.000000,0.000000,0.000000}%
\pgfsetstrokecolor{currentstroke}%
\pgfsetdash{}{0pt}%
\pgfsys@defobject{currentmarker}{\pgfqpoint{0.000000in}{-0.027778in}}{\pgfqpoint{0.000000in}{0.000000in}}{%
\pgfpathmoveto{\pgfqpoint{0.000000in}{0.000000in}}%
\pgfpathlineto{\pgfqpoint{0.000000in}{-0.027778in}}%
\pgfusepath{stroke,fill}%
}%
\begin{pgfscope}%
\pgfsys@transformshift{3.747380in}{0.417642in}%
\pgfsys@useobject{currentmarker}{}%
\end{pgfscope}%
\end{pgfscope}%
\begin{pgfscope}%
\pgfpathrectangle{\pgfqpoint{0.594525in}{0.417642in}}{\pgfqpoint{3.345963in}{2.050688in}}%
\pgfusepath{clip}%
\pgfsetrectcap%
\pgfsetroundjoin%
\pgfsetlinewidth{0.803000pt}%
\definecolor{currentstroke}{rgb}{0.850000,0.850000,0.850000}%
\pgfsetstrokecolor{currentstroke}%
\pgfsetdash{}{0pt}%
\pgfpathmoveto{\pgfqpoint{3.791943in}{0.417642in}}%
\pgfpathlineto{\pgfqpoint{3.791943in}{2.468330in}}%
\pgfusepath{stroke}%
\end{pgfscope}%
\begin{pgfscope}%
\pgfsetbuttcap%
\pgfsetroundjoin%
\definecolor{currentfill}{rgb}{0.000000,0.000000,0.000000}%
\pgfsetfillcolor{currentfill}%
\pgfsetlinewidth{0.602250pt}%
\definecolor{currentstroke}{rgb}{0.000000,0.000000,0.000000}%
\pgfsetstrokecolor{currentstroke}%
\pgfsetdash{}{0pt}%
\pgfsys@defobject{currentmarker}{\pgfqpoint{0.000000in}{-0.027778in}}{\pgfqpoint{0.000000in}{0.000000in}}{%
\pgfpathmoveto{\pgfqpoint{0.000000in}{0.000000in}}%
\pgfpathlineto{\pgfqpoint{0.000000in}{-0.027778in}}%
\pgfusepath{stroke,fill}%
}%
\begin{pgfscope}%
\pgfsys@transformshift{3.791943in}{0.417642in}%
\pgfsys@useobject{currentmarker}{}%
\end{pgfscope}%
\end{pgfscope}%
\begin{pgfscope}%
\pgfpathrectangle{\pgfqpoint{0.594525in}{0.417642in}}{\pgfqpoint{3.345963in}{2.050688in}}%
\pgfusepath{clip}%
\pgfsetrectcap%
\pgfsetroundjoin%
\pgfsetlinewidth{0.803000pt}%
\definecolor{currentstroke}{rgb}{0.850000,0.850000,0.850000}%
\pgfsetstrokecolor{currentstroke}%
\pgfsetdash{}{0pt}%
\pgfpathmoveto{\pgfqpoint{3.828353in}{0.417642in}}%
\pgfpathlineto{\pgfqpoint{3.828353in}{2.468330in}}%
\pgfusepath{stroke}%
\end{pgfscope}%
\begin{pgfscope}%
\pgfsetbuttcap%
\pgfsetroundjoin%
\definecolor{currentfill}{rgb}{0.000000,0.000000,0.000000}%
\pgfsetfillcolor{currentfill}%
\pgfsetlinewidth{0.602250pt}%
\definecolor{currentstroke}{rgb}{0.000000,0.000000,0.000000}%
\pgfsetstrokecolor{currentstroke}%
\pgfsetdash{}{0pt}%
\pgfsys@defobject{currentmarker}{\pgfqpoint{0.000000in}{-0.027778in}}{\pgfqpoint{0.000000in}{0.000000in}}{%
\pgfpathmoveto{\pgfqpoint{0.000000in}{0.000000in}}%
\pgfpathlineto{\pgfqpoint{0.000000in}{-0.027778in}}%
\pgfusepath{stroke,fill}%
}%
\begin{pgfscope}%
\pgfsys@transformshift{3.828353in}{0.417642in}%
\pgfsys@useobject{currentmarker}{}%
\end{pgfscope}%
\end{pgfscope}%
\begin{pgfscope}%
\pgfpathrectangle{\pgfqpoint{0.594525in}{0.417642in}}{\pgfqpoint{3.345963in}{2.050688in}}%
\pgfusepath{clip}%
\pgfsetrectcap%
\pgfsetroundjoin%
\pgfsetlinewidth{0.803000pt}%
\definecolor{currentstroke}{rgb}{0.850000,0.850000,0.850000}%
\pgfsetstrokecolor{currentstroke}%
\pgfsetdash{}{0pt}%
\pgfpathmoveto{\pgfqpoint{3.859137in}{0.417642in}}%
\pgfpathlineto{\pgfqpoint{3.859137in}{2.468330in}}%
\pgfusepath{stroke}%
\end{pgfscope}%
\begin{pgfscope}%
\pgfsetbuttcap%
\pgfsetroundjoin%
\definecolor{currentfill}{rgb}{0.000000,0.000000,0.000000}%
\pgfsetfillcolor{currentfill}%
\pgfsetlinewidth{0.602250pt}%
\definecolor{currentstroke}{rgb}{0.000000,0.000000,0.000000}%
\pgfsetstrokecolor{currentstroke}%
\pgfsetdash{}{0pt}%
\pgfsys@defobject{currentmarker}{\pgfqpoint{0.000000in}{-0.027778in}}{\pgfqpoint{0.000000in}{0.000000in}}{%
\pgfpathmoveto{\pgfqpoint{0.000000in}{0.000000in}}%
\pgfpathlineto{\pgfqpoint{0.000000in}{-0.027778in}}%
\pgfusepath{stroke,fill}%
}%
\begin{pgfscope}%
\pgfsys@transformshift{3.859137in}{0.417642in}%
\pgfsys@useobject{currentmarker}{}%
\end{pgfscope}%
\end{pgfscope}%
\begin{pgfscope}%
\pgfpathrectangle{\pgfqpoint{0.594525in}{0.417642in}}{\pgfqpoint{3.345963in}{2.050688in}}%
\pgfusepath{clip}%
\pgfsetrectcap%
\pgfsetroundjoin%
\pgfsetlinewidth{0.803000pt}%
\definecolor{currentstroke}{rgb}{0.850000,0.850000,0.850000}%
\pgfsetstrokecolor{currentstroke}%
\pgfsetdash{}{0pt}%
\pgfpathmoveto{\pgfqpoint{3.885804in}{0.417642in}}%
\pgfpathlineto{\pgfqpoint{3.885804in}{2.468330in}}%
\pgfusepath{stroke}%
\end{pgfscope}%
\begin{pgfscope}%
\pgfsetbuttcap%
\pgfsetroundjoin%
\definecolor{currentfill}{rgb}{0.000000,0.000000,0.000000}%
\pgfsetfillcolor{currentfill}%
\pgfsetlinewidth{0.602250pt}%
\definecolor{currentstroke}{rgb}{0.000000,0.000000,0.000000}%
\pgfsetstrokecolor{currentstroke}%
\pgfsetdash{}{0pt}%
\pgfsys@defobject{currentmarker}{\pgfqpoint{0.000000in}{-0.027778in}}{\pgfqpoint{0.000000in}{0.000000in}}{%
\pgfpathmoveto{\pgfqpoint{0.000000in}{0.000000in}}%
\pgfpathlineto{\pgfqpoint{0.000000in}{-0.027778in}}%
\pgfusepath{stroke,fill}%
}%
\begin{pgfscope}%
\pgfsys@transformshift{3.885804in}{0.417642in}%
\pgfsys@useobject{currentmarker}{}%
\end{pgfscope}%
\end{pgfscope}%
\begin{pgfscope}%
\pgfpathrectangle{\pgfqpoint{0.594525in}{0.417642in}}{\pgfqpoint{3.345963in}{2.050688in}}%
\pgfusepath{clip}%
\pgfsetrectcap%
\pgfsetroundjoin%
\pgfsetlinewidth{0.803000pt}%
\definecolor{currentstroke}{rgb}{0.850000,0.850000,0.850000}%
\pgfsetstrokecolor{currentstroke}%
\pgfsetdash{}{0pt}%
\pgfpathmoveto{\pgfqpoint{3.909326in}{0.417642in}}%
\pgfpathlineto{\pgfqpoint{3.909326in}{2.468330in}}%
\pgfusepath{stroke}%
\end{pgfscope}%
\begin{pgfscope}%
\pgfsetbuttcap%
\pgfsetroundjoin%
\definecolor{currentfill}{rgb}{0.000000,0.000000,0.000000}%
\pgfsetfillcolor{currentfill}%
\pgfsetlinewidth{0.602250pt}%
\definecolor{currentstroke}{rgb}{0.000000,0.000000,0.000000}%
\pgfsetstrokecolor{currentstroke}%
\pgfsetdash{}{0pt}%
\pgfsys@defobject{currentmarker}{\pgfqpoint{0.000000in}{-0.027778in}}{\pgfqpoint{0.000000in}{0.000000in}}{%
\pgfpathmoveto{\pgfqpoint{0.000000in}{0.000000in}}%
\pgfpathlineto{\pgfqpoint{0.000000in}{-0.027778in}}%
\pgfusepath{stroke,fill}%
}%
\begin{pgfscope}%
\pgfsys@transformshift{3.909326in}{0.417642in}%
\pgfsys@useobject{currentmarker}{}%
\end{pgfscope}%
\end{pgfscope}%
\begin{pgfscope}%
\definecolor{textcolor}{rgb}{0.000000,0.000000,0.000000}%
\pgfsetstrokecolor{textcolor}%
\pgfsetfillcolor{textcolor}%
\pgftext[x=2.267506in,y=0.165003in,,top]{\color{textcolor}\rmfamily\fontsize{10.000000}{12.000000}\selectfont Frequency in \(\displaystyle \unit{\Hz}\)}%
\end{pgfscope}%
\begin{pgfscope}%
\pgfpathrectangle{\pgfqpoint{0.594525in}{0.417642in}}{\pgfqpoint{3.345963in}{2.050688in}}%
\pgfusepath{clip}%
\pgfsetrectcap%
\pgfsetroundjoin%
\pgfsetlinewidth{0.803000pt}%
\definecolor{currentstroke}{rgb}{0.450000,0.450000,0.450000}%
\pgfsetstrokecolor{currentstroke}%
\pgfsetdash{}{0pt}%
\pgfpathmoveto{\pgfqpoint{0.594525in}{0.627420in}}%
\pgfpathlineto{\pgfqpoint{3.940488in}{0.627420in}}%
\pgfusepath{stroke}%
\end{pgfscope}%
\begin{pgfscope}%
\pgfsetbuttcap%
\pgfsetroundjoin%
\definecolor{currentfill}{rgb}{0.000000,0.000000,0.000000}%
\pgfsetfillcolor{currentfill}%
\pgfsetlinewidth{0.803000pt}%
\definecolor{currentstroke}{rgb}{0.000000,0.000000,0.000000}%
\pgfsetstrokecolor{currentstroke}%
\pgfsetdash{}{0pt}%
\pgfsys@defobject{currentmarker}{\pgfqpoint{-0.048611in}{0.000000in}}{\pgfqpoint{-0.000000in}{0.000000in}}{%
\pgfpathmoveto{\pgfqpoint{-0.000000in}{0.000000in}}%
\pgfpathlineto{\pgfqpoint{-0.048611in}{0.000000in}}%
\pgfusepath{stroke,fill}%
}%
\begin{pgfscope}%
\pgfsys@transformshift{0.594525in}{0.627420in}%
\pgfsys@useobject{currentmarker}{}%
\end{pgfscope}%
\end{pgfscope}%
\begin{pgfscope}%
\definecolor{textcolor}{rgb}{0.000000,0.000000,0.000000}%
\pgfsetstrokecolor{textcolor}%
\pgfsetfillcolor{textcolor}%
\pgftext[x=0.241129in, y=0.588267in, left, base]{\color{textcolor}\rmfamily\fontsize{8.000000}{9.600000}\selectfont \(\displaystyle {10^{-3}}\)}%
\end{pgfscope}%
\begin{pgfscope}%
\pgfpathrectangle{\pgfqpoint{0.594525in}{0.417642in}}{\pgfqpoint{3.345963in}{2.050688in}}%
\pgfusepath{clip}%
\pgfsetrectcap%
\pgfsetroundjoin%
\pgfsetlinewidth{0.803000pt}%
\definecolor{currentstroke}{rgb}{0.450000,0.450000,0.450000}%
\pgfsetstrokecolor{currentstroke}%
\pgfsetdash{}{0pt}%
\pgfpathmoveto{\pgfqpoint{0.594525in}{0.904540in}}%
\pgfpathlineto{\pgfqpoint{3.940488in}{0.904540in}}%
\pgfusepath{stroke}%
\end{pgfscope}%
\begin{pgfscope}%
\pgfsetbuttcap%
\pgfsetroundjoin%
\definecolor{currentfill}{rgb}{0.000000,0.000000,0.000000}%
\pgfsetfillcolor{currentfill}%
\pgfsetlinewidth{0.803000pt}%
\definecolor{currentstroke}{rgb}{0.000000,0.000000,0.000000}%
\pgfsetstrokecolor{currentstroke}%
\pgfsetdash{}{0pt}%
\pgfsys@defobject{currentmarker}{\pgfqpoint{-0.048611in}{0.000000in}}{\pgfqpoint{-0.000000in}{0.000000in}}{%
\pgfpathmoveto{\pgfqpoint{-0.000000in}{0.000000in}}%
\pgfpathlineto{\pgfqpoint{-0.048611in}{0.000000in}}%
\pgfusepath{stroke,fill}%
}%
\begin{pgfscope}%
\pgfsys@transformshift{0.594525in}{0.904540in}%
\pgfsys@useobject{currentmarker}{}%
\end{pgfscope}%
\end{pgfscope}%
\begin{pgfscope}%
\definecolor{textcolor}{rgb}{0.000000,0.000000,0.000000}%
\pgfsetstrokecolor{textcolor}%
\pgfsetfillcolor{textcolor}%
\pgftext[x=0.241129in, y=0.865388in, left, base]{\color{textcolor}\rmfamily\fontsize{8.000000}{9.600000}\selectfont \(\displaystyle {10^{-2}}\)}%
\end{pgfscope}%
\begin{pgfscope}%
\pgfpathrectangle{\pgfqpoint{0.594525in}{0.417642in}}{\pgfqpoint{3.345963in}{2.050688in}}%
\pgfusepath{clip}%
\pgfsetrectcap%
\pgfsetroundjoin%
\pgfsetlinewidth{0.803000pt}%
\definecolor{currentstroke}{rgb}{0.450000,0.450000,0.450000}%
\pgfsetstrokecolor{currentstroke}%
\pgfsetdash{}{0pt}%
\pgfpathmoveto{\pgfqpoint{0.594525in}{1.181661in}}%
\pgfpathlineto{\pgfqpoint{3.940488in}{1.181661in}}%
\pgfusepath{stroke}%
\end{pgfscope}%
\begin{pgfscope}%
\pgfsetbuttcap%
\pgfsetroundjoin%
\definecolor{currentfill}{rgb}{0.000000,0.000000,0.000000}%
\pgfsetfillcolor{currentfill}%
\pgfsetlinewidth{0.803000pt}%
\definecolor{currentstroke}{rgb}{0.000000,0.000000,0.000000}%
\pgfsetstrokecolor{currentstroke}%
\pgfsetdash{}{0pt}%
\pgfsys@defobject{currentmarker}{\pgfqpoint{-0.048611in}{0.000000in}}{\pgfqpoint{-0.000000in}{0.000000in}}{%
\pgfpathmoveto{\pgfqpoint{-0.000000in}{0.000000in}}%
\pgfpathlineto{\pgfqpoint{-0.048611in}{0.000000in}}%
\pgfusepath{stroke,fill}%
}%
\begin{pgfscope}%
\pgfsys@transformshift{0.594525in}{1.181661in}%
\pgfsys@useobject{currentmarker}{}%
\end{pgfscope}%
\end{pgfscope}%
\begin{pgfscope}%
\definecolor{textcolor}{rgb}{0.000000,0.000000,0.000000}%
\pgfsetstrokecolor{textcolor}%
\pgfsetfillcolor{textcolor}%
\pgftext[x=0.241129in, y=1.142508in, left, base]{\color{textcolor}\rmfamily\fontsize{8.000000}{9.600000}\selectfont \(\displaystyle {10^{-1}}\)}%
\end{pgfscope}%
\begin{pgfscope}%
\pgfpathrectangle{\pgfqpoint{0.594525in}{0.417642in}}{\pgfqpoint{3.345963in}{2.050688in}}%
\pgfusepath{clip}%
\pgfsetrectcap%
\pgfsetroundjoin%
\pgfsetlinewidth{0.803000pt}%
\definecolor{currentstroke}{rgb}{0.450000,0.450000,0.450000}%
\pgfsetstrokecolor{currentstroke}%
\pgfsetdash{}{0pt}%
\pgfpathmoveto{\pgfqpoint{0.594525in}{1.458781in}}%
\pgfpathlineto{\pgfqpoint{3.940488in}{1.458781in}}%
\pgfusepath{stroke}%
\end{pgfscope}%
\begin{pgfscope}%
\pgfsetbuttcap%
\pgfsetroundjoin%
\definecolor{currentfill}{rgb}{0.000000,0.000000,0.000000}%
\pgfsetfillcolor{currentfill}%
\pgfsetlinewidth{0.803000pt}%
\definecolor{currentstroke}{rgb}{0.000000,0.000000,0.000000}%
\pgfsetstrokecolor{currentstroke}%
\pgfsetdash{}{0pt}%
\pgfsys@defobject{currentmarker}{\pgfqpoint{-0.048611in}{0.000000in}}{\pgfqpoint{-0.000000in}{0.000000in}}{%
\pgfpathmoveto{\pgfqpoint{-0.000000in}{0.000000in}}%
\pgfpathlineto{\pgfqpoint{-0.048611in}{0.000000in}}%
\pgfusepath{stroke,fill}%
}%
\begin{pgfscope}%
\pgfsys@transformshift{0.594525in}{1.458781in}%
\pgfsys@useobject{currentmarker}{}%
\end{pgfscope}%
\end{pgfscope}%
\begin{pgfscope}%
\definecolor{textcolor}{rgb}{0.000000,0.000000,0.000000}%
\pgfsetstrokecolor{textcolor}%
\pgfsetfillcolor{textcolor}%
\pgftext[x=0.321376in, y=1.419628in, left, base]{\color{textcolor}\rmfamily\fontsize{8.000000}{9.600000}\selectfont \(\displaystyle {10^{0}}\)}%
\end{pgfscope}%
\begin{pgfscope}%
\pgfpathrectangle{\pgfqpoint{0.594525in}{0.417642in}}{\pgfqpoint{3.345963in}{2.050688in}}%
\pgfusepath{clip}%
\pgfsetrectcap%
\pgfsetroundjoin%
\pgfsetlinewidth{0.803000pt}%
\definecolor{currentstroke}{rgb}{0.450000,0.450000,0.450000}%
\pgfsetstrokecolor{currentstroke}%
\pgfsetdash{}{0pt}%
\pgfpathmoveto{\pgfqpoint{0.594525in}{1.735901in}}%
\pgfpathlineto{\pgfqpoint{3.940488in}{1.735901in}}%
\pgfusepath{stroke}%
\end{pgfscope}%
\begin{pgfscope}%
\pgfsetbuttcap%
\pgfsetroundjoin%
\definecolor{currentfill}{rgb}{0.000000,0.000000,0.000000}%
\pgfsetfillcolor{currentfill}%
\pgfsetlinewidth{0.803000pt}%
\definecolor{currentstroke}{rgb}{0.000000,0.000000,0.000000}%
\pgfsetstrokecolor{currentstroke}%
\pgfsetdash{}{0pt}%
\pgfsys@defobject{currentmarker}{\pgfqpoint{-0.048611in}{0.000000in}}{\pgfqpoint{-0.000000in}{0.000000in}}{%
\pgfpathmoveto{\pgfqpoint{-0.000000in}{0.000000in}}%
\pgfpathlineto{\pgfqpoint{-0.048611in}{0.000000in}}%
\pgfusepath{stroke,fill}%
}%
\begin{pgfscope}%
\pgfsys@transformshift{0.594525in}{1.735901in}%
\pgfsys@useobject{currentmarker}{}%
\end{pgfscope}%
\end{pgfscope}%
\begin{pgfscope}%
\definecolor{textcolor}{rgb}{0.000000,0.000000,0.000000}%
\pgfsetstrokecolor{textcolor}%
\pgfsetfillcolor{textcolor}%
\pgftext[x=0.321376in, y=1.696749in, left, base]{\color{textcolor}\rmfamily\fontsize{8.000000}{9.600000}\selectfont \(\displaystyle {10^{1}}\)}%
\end{pgfscope}%
\begin{pgfscope}%
\pgfpathrectangle{\pgfqpoint{0.594525in}{0.417642in}}{\pgfqpoint{3.345963in}{2.050688in}}%
\pgfusepath{clip}%
\pgfsetrectcap%
\pgfsetroundjoin%
\pgfsetlinewidth{0.803000pt}%
\definecolor{currentstroke}{rgb}{0.450000,0.450000,0.450000}%
\pgfsetstrokecolor{currentstroke}%
\pgfsetdash{}{0pt}%
\pgfpathmoveto{\pgfqpoint{0.594525in}{2.013022in}}%
\pgfpathlineto{\pgfqpoint{3.940488in}{2.013022in}}%
\pgfusepath{stroke}%
\end{pgfscope}%
\begin{pgfscope}%
\pgfsetbuttcap%
\pgfsetroundjoin%
\definecolor{currentfill}{rgb}{0.000000,0.000000,0.000000}%
\pgfsetfillcolor{currentfill}%
\pgfsetlinewidth{0.803000pt}%
\definecolor{currentstroke}{rgb}{0.000000,0.000000,0.000000}%
\pgfsetstrokecolor{currentstroke}%
\pgfsetdash{}{0pt}%
\pgfsys@defobject{currentmarker}{\pgfqpoint{-0.048611in}{0.000000in}}{\pgfqpoint{-0.000000in}{0.000000in}}{%
\pgfpathmoveto{\pgfqpoint{-0.000000in}{0.000000in}}%
\pgfpathlineto{\pgfqpoint{-0.048611in}{0.000000in}}%
\pgfusepath{stroke,fill}%
}%
\begin{pgfscope}%
\pgfsys@transformshift{0.594525in}{2.013022in}%
\pgfsys@useobject{currentmarker}{}%
\end{pgfscope}%
\end{pgfscope}%
\begin{pgfscope}%
\definecolor{textcolor}{rgb}{0.000000,0.000000,0.000000}%
\pgfsetstrokecolor{textcolor}%
\pgfsetfillcolor{textcolor}%
\pgftext[x=0.321376in, y=1.973869in, left, base]{\color{textcolor}\rmfamily\fontsize{8.000000}{9.600000}\selectfont \(\displaystyle {10^{2}}\)}%
\end{pgfscope}%
\begin{pgfscope}%
\pgfpathrectangle{\pgfqpoint{0.594525in}{0.417642in}}{\pgfqpoint{3.345963in}{2.050688in}}%
\pgfusepath{clip}%
\pgfsetrectcap%
\pgfsetroundjoin%
\pgfsetlinewidth{0.803000pt}%
\definecolor{currentstroke}{rgb}{0.450000,0.450000,0.450000}%
\pgfsetstrokecolor{currentstroke}%
\pgfsetdash{}{0pt}%
\pgfpathmoveto{\pgfqpoint{0.594525in}{2.290142in}}%
\pgfpathlineto{\pgfqpoint{3.940488in}{2.290142in}}%
\pgfusepath{stroke}%
\end{pgfscope}%
\begin{pgfscope}%
\pgfsetbuttcap%
\pgfsetroundjoin%
\definecolor{currentfill}{rgb}{0.000000,0.000000,0.000000}%
\pgfsetfillcolor{currentfill}%
\pgfsetlinewidth{0.803000pt}%
\definecolor{currentstroke}{rgb}{0.000000,0.000000,0.000000}%
\pgfsetstrokecolor{currentstroke}%
\pgfsetdash{}{0pt}%
\pgfsys@defobject{currentmarker}{\pgfqpoint{-0.048611in}{0.000000in}}{\pgfqpoint{-0.000000in}{0.000000in}}{%
\pgfpathmoveto{\pgfqpoint{-0.000000in}{0.000000in}}%
\pgfpathlineto{\pgfqpoint{-0.048611in}{0.000000in}}%
\pgfusepath{stroke,fill}%
}%
\begin{pgfscope}%
\pgfsys@transformshift{0.594525in}{2.290142in}%
\pgfsys@useobject{currentmarker}{}%
\end{pgfscope}%
\end{pgfscope}%
\begin{pgfscope}%
\definecolor{textcolor}{rgb}{0.000000,0.000000,0.000000}%
\pgfsetstrokecolor{textcolor}%
\pgfsetfillcolor{textcolor}%
\pgftext[x=0.321376in, y=2.250989in, left, base]{\color{textcolor}\rmfamily\fontsize{8.000000}{9.600000}\selectfont \(\displaystyle {10^{3}}\)}%
\end{pgfscope}%
\begin{pgfscope}%
\pgfpathrectangle{\pgfqpoint{0.594525in}{0.417642in}}{\pgfqpoint{3.345963in}{2.050688in}}%
\pgfusepath{clip}%
\pgfsetrectcap%
\pgfsetroundjoin%
\pgfsetlinewidth{0.803000pt}%
\definecolor{currentstroke}{rgb}{0.850000,0.850000,0.850000}%
\pgfsetstrokecolor{currentstroke}%
\pgfsetdash{}{0pt}%
\pgfpathmoveto{\pgfqpoint{0.594525in}{0.433721in}}%
\pgfpathlineto{\pgfqpoint{3.940488in}{0.433721in}}%
\pgfusepath{stroke}%
\end{pgfscope}%
\begin{pgfscope}%
\pgfsetbuttcap%
\pgfsetroundjoin%
\definecolor{currentfill}{rgb}{0.000000,0.000000,0.000000}%
\pgfsetfillcolor{currentfill}%
\pgfsetlinewidth{0.602250pt}%
\definecolor{currentstroke}{rgb}{0.000000,0.000000,0.000000}%
\pgfsetstrokecolor{currentstroke}%
\pgfsetdash{}{0pt}%
\pgfsys@defobject{currentmarker}{\pgfqpoint{-0.027778in}{0.000000in}}{\pgfqpoint{-0.000000in}{0.000000in}}{%
\pgfpathmoveto{\pgfqpoint{-0.000000in}{0.000000in}}%
\pgfpathlineto{\pgfqpoint{-0.027778in}{0.000000in}}%
\pgfusepath{stroke,fill}%
}%
\begin{pgfscope}%
\pgfsys@transformshift{0.594525in}{0.433721in}%
\pgfsys@useobject{currentmarker}{}%
\end{pgfscope}%
\end{pgfscope}%
\begin{pgfscope}%
\pgfpathrectangle{\pgfqpoint{0.594525in}{0.417642in}}{\pgfqpoint{3.345963in}{2.050688in}}%
\pgfusepath{clip}%
\pgfsetrectcap%
\pgfsetroundjoin%
\pgfsetlinewidth{0.803000pt}%
\definecolor{currentstroke}{rgb}{0.850000,0.850000,0.850000}%
\pgfsetstrokecolor{currentstroke}%
\pgfsetdash{}{0pt}%
\pgfpathmoveto{\pgfqpoint{0.594525in}{0.482520in}}%
\pgfpathlineto{\pgfqpoint{3.940488in}{0.482520in}}%
\pgfusepath{stroke}%
\end{pgfscope}%
\begin{pgfscope}%
\pgfsetbuttcap%
\pgfsetroundjoin%
\definecolor{currentfill}{rgb}{0.000000,0.000000,0.000000}%
\pgfsetfillcolor{currentfill}%
\pgfsetlinewidth{0.602250pt}%
\definecolor{currentstroke}{rgb}{0.000000,0.000000,0.000000}%
\pgfsetstrokecolor{currentstroke}%
\pgfsetdash{}{0pt}%
\pgfsys@defobject{currentmarker}{\pgfqpoint{-0.027778in}{0.000000in}}{\pgfqpoint{-0.000000in}{0.000000in}}{%
\pgfpathmoveto{\pgfqpoint{-0.000000in}{0.000000in}}%
\pgfpathlineto{\pgfqpoint{-0.027778in}{0.000000in}}%
\pgfusepath{stroke,fill}%
}%
\begin{pgfscope}%
\pgfsys@transformshift{0.594525in}{0.482520in}%
\pgfsys@useobject{currentmarker}{}%
\end{pgfscope}%
\end{pgfscope}%
\begin{pgfscope}%
\pgfpathrectangle{\pgfqpoint{0.594525in}{0.417642in}}{\pgfqpoint{3.345963in}{2.050688in}}%
\pgfusepath{clip}%
\pgfsetrectcap%
\pgfsetroundjoin%
\pgfsetlinewidth{0.803000pt}%
\definecolor{currentstroke}{rgb}{0.850000,0.850000,0.850000}%
\pgfsetstrokecolor{currentstroke}%
\pgfsetdash{}{0pt}%
\pgfpathmoveto{\pgfqpoint{0.594525in}{0.517143in}}%
\pgfpathlineto{\pgfqpoint{3.940488in}{0.517143in}}%
\pgfusepath{stroke}%
\end{pgfscope}%
\begin{pgfscope}%
\pgfsetbuttcap%
\pgfsetroundjoin%
\definecolor{currentfill}{rgb}{0.000000,0.000000,0.000000}%
\pgfsetfillcolor{currentfill}%
\pgfsetlinewidth{0.602250pt}%
\definecolor{currentstroke}{rgb}{0.000000,0.000000,0.000000}%
\pgfsetstrokecolor{currentstroke}%
\pgfsetdash{}{0pt}%
\pgfsys@defobject{currentmarker}{\pgfqpoint{-0.027778in}{0.000000in}}{\pgfqpoint{-0.000000in}{0.000000in}}{%
\pgfpathmoveto{\pgfqpoint{-0.000000in}{0.000000in}}%
\pgfpathlineto{\pgfqpoint{-0.027778in}{0.000000in}}%
\pgfusepath{stroke,fill}%
}%
\begin{pgfscope}%
\pgfsys@transformshift{0.594525in}{0.517143in}%
\pgfsys@useobject{currentmarker}{}%
\end{pgfscope}%
\end{pgfscope}%
\begin{pgfscope}%
\pgfpathrectangle{\pgfqpoint{0.594525in}{0.417642in}}{\pgfqpoint{3.345963in}{2.050688in}}%
\pgfusepath{clip}%
\pgfsetrectcap%
\pgfsetroundjoin%
\pgfsetlinewidth{0.803000pt}%
\definecolor{currentstroke}{rgb}{0.850000,0.850000,0.850000}%
\pgfsetstrokecolor{currentstroke}%
\pgfsetdash{}{0pt}%
\pgfpathmoveto{\pgfqpoint{0.594525in}{0.543998in}}%
\pgfpathlineto{\pgfqpoint{3.940488in}{0.543998in}}%
\pgfusepath{stroke}%
\end{pgfscope}%
\begin{pgfscope}%
\pgfsetbuttcap%
\pgfsetroundjoin%
\definecolor{currentfill}{rgb}{0.000000,0.000000,0.000000}%
\pgfsetfillcolor{currentfill}%
\pgfsetlinewidth{0.602250pt}%
\definecolor{currentstroke}{rgb}{0.000000,0.000000,0.000000}%
\pgfsetstrokecolor{currentstroke}%
\pgfsetdash{}{0pt}%
\pgfsys@defobject{currentmarker}{\pgfqpoint{-0.027778in}{0.000000in}}{\pgfqpoint{-0.000000in}{0.000000in}}{%
\pgfpathmoveto{\pgfqpoint{-0.000000in}{0.000000in}}%
\pgfpathlineto{\pgfqpoint{-0.027778in}{0.000000in}}%
\pgfusepath{stroke,fill}%
}%
\begin{pgfscope}%
\pgfsys@transformshift{0.594525in}{0.543998in}%
\pgfsys@useobject{currentmarker}{}%
\end{pgfscope}%
\end{pgfscope}%
\begin{pgfscope}%
\pgfpathrectangle{\pgfqpoint{0.594525in}{0.417642in}}{\pgfqpoint{3.345963in}{2.050688in}}%
\pgfusepath{clip}%
\pgfsetrectcap%
\pgfsetroundjoin%
\pgfsetlinewidth{0.803000pt}%
\definecolor{currentstroke}{rgb}{0.850000,0.850000,0.850000}%
\pgfsetstrokecolor{currentstroke}%
\pgfsetdash{}{0pt}%
\pgfpathmoveto{\pgfqpoint{0.594525in}{0.565941in}}%
\pgfpathlineto{\pgfqpoint{3.940488in}{0.565941in}}%
\pgfusepath{stroke}%
\end{pgfscope}%
\begin{pgfscope}%
\pgfsetbuttcap%
\pgfsetroundjoin%
\definecolor{currentfill}{rgb}{0.000000,0.000000,0.000000}%
\pgfsetfillcolor{currentfill}%
\pgfsetlinewidth{0.602250pt}%
\definecolor{currentstroke}{rgb}{0.000000,0.000000,0.000000}%
\pgfsetstrokecolor{currentstroke}%
\pgfsetdash{}{0pt}%
\pgfsys@defobject{currentmarker}{\pgfqpoint{-0.027778in}{0.000000in}}{\pgfqpoint{-0.000000in}{0.000000in}}{%
\pgfpathmoveto{\pgfqpoint{-0.000000in}{0.000000in}}%
\pgfpathlineto{\pgfqpoint{-0.027778in}{0.000000in}}%
\pgfusepath{stroke,fill}%
}%
\begin{pgfscope}%
\pgfsys@transformshift{0.594525in}{0.565941in}%
\pgfsys@useobject{currentmarker}{}%
\end{pgfscope}%
\end{pgfscope}%
\begin{pgfscope}%
\pgfpathrectangle{\pgfqpoint{0.594525in}{0.417642in}}{\pgfqpoint{3.345963in}{2.050688in}}%
\pgfusepath{clip}%
\pgfsetrectcap%
\pgfsetroundjoin%
\pgfsetlinewidth{0.803000pt}%
\definecolor{currentstroke}{rgb}{0.850000,0.850000,0.850000}%
\pgfsetstrokecolor{currentstroke}%
\pgfsetdash{}{0pt}%
\pgfpathmoveto{\pgfqpoint{0.594525in}{0.584494in}}%
\pgfpathlineto{\pgfqpoint{3.940488in}{0.584494in}}%
\pgfusepath{stroke}%
\end{pgfscope}%
\begin{pgfscope}%
\pgfsetbuttcap%
\pgfsetroundjoin%
\definecolor{currentfill}{rgb}{0.000000,0.000000,0.000000}%
\pgfsetfillcolor{currentfill}%
\pgfsetlinewidth{0.602250pt}%
\definecolor{currentstroke}{rgb}{0.000000,0.000000,0.000000}%
\pgfsetstrokecolor{currentstroke}%
\pgfsetdash{}{0pt}%
\pgfsys@defobject{currentmarker}{\pgfqpoint{-0.027778in}{0.000000in}}{\pgfqpoint{-0.000000in}{0.000000in}}{%
\pgfpathmoveto{\pgfqpoint{-0.000000in}{0.000000in}}%
\pgfpathlineto{\pgfqpoint{-0.027778in}{0.000000in}}%
\pgfusepath{stroke,fill}%
}%
\begin{pgfscope}%
\pgfsys@transformshift{0.594525in}{0.584494in}%
\pgfsys@useobject{currentmarker}{}%
\end{pgfscope}%
\end{pgfscope}%
\begin{pgfscope}%
\pgfpathrectangle{\pgfqpoint{0.594525in}{0.417642in}}{\pgfqpoint{3.345963in}{2.050688in}}%
\pgfusepath{clip}%
\pgfsetrectcap%
\pgfsetroundjoin%
\pgfsetlinewidth{0.803000pt}%
\definecolor{currentstroke}{rgb}{0.850000,0.850000,0.850000}%
\pgfsetstrokecolor{currentstroke}%
\pgfsetdash{}{0pt}%
\pgfpathmoveto{\pgfqpoint{0.594525in}{0.600564in}}%
\pgfpathlineto{\pgfqpoint{3.940488in}{0.600564in}}%
\pgfusepath{stroke}%
\end{pgfscope}%
\begin{pgfscope}%
\pgfsetbuttcap%
\pgfsetroundjoin%
\definecolor{currentfill}{rgb}{0.000000,0.000000,0.000000}%
\pgfsetfillcolor{currentfill}%
\pgfsetlinewidth{0.602250pt}%
\definecolor{currentstroke}{rgb}{0.000000,0.000000,0.000000}%
\pgfsetstrokecolor{currentstroke}%
\pgfsetdash{}{0pt}%
\pgfsys@defobject{currentmarker}{\pgfqpoint{-0.027778in}{0.000000in}}{\pgfqpoint{-0.000000in}{0.000000in}}{%
\pgfpathmoveto{\pgfqpoint{-0.000000in}{0.000000in}}%
\pgfpathlineto{\pgfqpoint{-0.027778in}{0.000000in}}%
\pgfusepath{stroke,fill}%
}%
\begin{pgfscope}%
\pgfsys@transformshift{0.594525in}{0.600564in}%
\pgfsys@useobject{currentmarker}{}%
\end{pgfscope}%
\end{pgfscope}%
\begin{pgfscope}%
\pgfpathrectangle{\pgfqpoint{0.594525in}{0.417642in}}{\pgfqpoint{3.345963in}{2.050688in}}%
\pgfusepath{clip}%
\pgfsetrectcap%
\pgfsetroundjoin%
\pgfsetlinewidth{0.803000pt}%
\definecolor{currentstroke}{rgb}{0.850000,0.850000,0.850000}%
\pgfsetstrokecolor{currentstroke}%
\pgfsetdash{}{0pt}%
\pgfpathmoveto{\pgfqpoint{0.594525in}{0.614740in}}%
\pgfpathlineto{\pgfqpoint{3.940488in}{0.614740in}}%
\pgfusepath{stroke}%
\end{pgfscope}%
\begin{pgfscope}%
\pgfsetbuttcap%
\pgfsetroundjoin%
\definecolor{currentfill}{rgb}{0.000000,0.000000,0.000000}%
\pgfsetfillcolor{currentfill}%
\pgfsetlinewidth{0.602250pt}%
\definecolor{currentstroke}{rgb}{0.000000,0.000000,0.000000}%
\pgfsetstrokecolor{currentstroke}%
\pgfsetdash{}{0pt}%
\pgfsys@defobject{currentmarker}{\pgfqpoint{-0.027778in}{0.000000in}}{\pgfqpoint{-0.000000in}{0.000000in}}{%
\pgfpathmoveto{\pgfqpoint{-0.000000in}{0.000000in}}%
\pgfpathlineto{\pgfqpoint{-0.027778in}{0.000000in}}%
\pgfusepath{stroke,fill}%
}%
\begin{pgfscope}%
\pgfsys@transformshift{0.594525in}{0.614740in}%
\pgfsys@useobject{currentmarker}{}%
\end{pgfscope}%
\end{pgfscope}%
\begin{pgfscope}%
\pgfpathrectangle{\pgfqpoint{0.594525in}{0.417642in}}{\pgfqpoint{3.345963in}{2.050688in}}%
\pgfusepath{clip}%
\pgfsetrectcap%
\pgfsetroundjoin%
\pgfsetlinewidth{0.803000pt}%
\definecolor{currentstroke}{rgb}{0.850000,0.850000,0.850000}%
\pgfsetstrokecolor{currentstroke}%
\pgfsetdash{}{0pt}%
\pgfpathmoveto{\pgfqpoint{0.594525in}{0.710842in}}%
\pgfpathlineto{\pgfqpoint{3.940488in}{0.710842in}}%
\pgfusepath{stroke}%
\end{pgfscope}%
\begin{pgfscope}%
\pgfsetbuttcap%
\pgfsetroundjoin%
\definecolor{currentfill}{rgb}{0.000000,0.000000,0.000000}%
\pgfsetfillcolor{currentfill}%
\pgfsetlinewidth{0.602250pt}%
\definecolor{currentstroke}{rgb}{0.000000,0.000000,0.000000}%
\pgfsetstrokecolor{currentstroke}%
\pgfsetdash{}{0pt}%
\pgfsys@defobject{currentmarker}{\pgfqpoint{-0.027778in}{0.000000in}}{\pgfqpoint{-0.000000in}{0.000000in}}{%
\pgfpathmoveto{\pgfqpoint{-0.000000in}{0.000000in}}%
\pgfpathlineto{\pgfqpoint{-0.027778in}{0.000000in}}%
\pgfusepath{stroke,fill}%
}%
\begin{pgfscope}%
\pgfsys@transformshift{0.594525in}{0.710842in}%
\pgfsys@useobject{currentmarker}{}%
\end{pgfscope}%
\end{pgfscope}%
\begin{pgfscope}%
\pgfpathrectangle{\pgfqpoint{0.594525in}{0.417642in}}{\pgfqpoint{3.345963in}{2.050688in}}%
\pgfusepath{clip}%
\pgfsetrectcap%
\pgfsetroundjoin%
\pgfsetlinewidth{0.803000pt}%
\definecolor{currentstroke}{rgb}{0.850000,0.850000,0.850000}%
\pgfsetstrokecolor{currentstroke}%
\pgfsetdash{}{0pt}%
\pgfpathmoveto{\pgfqpoint{0.594525in}{0.759640in}}%
\pgfpathlineto{\pgfqpoint{3.940488in}{0.759640in}}%
\pgfusepath{stroke}%
\end{pgfscope}%
\begin{pgfscope}%
\pgfsetbuttcap%
\pgfsetroundjoin%
\definecolor{currentfill}{rgb}{0.000000,0.000000,0.000000}%
\pgfsetfillcolor{currentfill}%
\pgfsetlinewidth{0.602250pt}%
\definecolor{currentstroke}{rgb}{0.000000,0.000000,0.000000}%
\pgfsetstrokecolor{currentstroke}%
\pgfsetdash{}{0pt}%
\pgfsys@defobject{currentmarker}{\pgfqpoint{-0.027778in}{0.000000in}}{\pgfqpoint{-0.000000in}{0.000000in}}{%
\pgfpathmoveto{\pgfqpoint{-0.000000in}{0.000000in}}%
\pgfpathlineto{\pgfqpoint{-0.027778in}{0.000000in}}%
\pgfusepath{stroke,fill}%
}%
\begin{pgfscope}%
\pgfsys@transformshift{0.594525in}{0.759640in}%
\pgfsys@useobject{currentmarker}{}%
\end{pgfscope}%
\end{pgfscope}%
\begin{pgfscope}%
\pgfpathrectangle{\pgfqpoint{0.594525in}{0.417642in}}{\pgfqpoint{3.345963in}{2.050688in}}%
\pgfusepath{clip}%
\pgfsetrectcap%
\pgfsetroundjoin%
\pgfsetlinewidth{0.803000pt}%
\definecolor{currentstroke}{rgb}{0.850000,0.850000,0.850000}%
\pgfsetstrokecolor{currentstroke}%
\pgfsetdash{}{0pt}%
\pgfpathmoveto{\pgfqpoint{0.594525in}{0.794263in}}%
\pgfpathlineto{\pgfqpoint{3.940488in}{0.794263in}}%
\pgfusepath{stroke}%
\end{pgfscope}%
\begin{pgfscope}%
\pgfsetbuttcap%
\pgfsetroundjoin%
\definecolor{currentfill}{rgb}{0.000000,0.000000,0.000000}%
\pgfsetfillcolor{currentfill}%
\pgfsetlinewidth{0.602250pt}%
\definecolor{currentstroke}{rgb}{0.000000,0.000000,0.000000}%
\pgfsetstrokecolor{currentstroke}%
\pgfsetdash{}{0pt}%
\pgfsys@defobject{currentmarker}{\pgfqpoint{-0.027778in}{0.000000in}}{\pgfqpoint{-0.000000in}{0.000000in}}{%
\pgfpathmoveto{\pgfqpoint{-0.000000in}{0.000000in}}%
\pgfpathlineto{\pgfqpoint{-0.027778in}{0.000000in}}%
\pgfusepath{stroke,fill}%
}%
\begin{pgfscope}%
\pgfsys@transformshift{0.594525in}{0.794263in}%
\pgfsys@useobject{currentmarker}{}%
\end{pgfscope}%
\end{pgfscope}%
\begin{pgfscope}%
\pgfpathrectangle{\pgfqpoint{0.594525in}{0.417642in}}{\pgfqpoint{3.345963in}{2.050688in}}%
\pgfusepath{clip}%
\pgfsetrectcap%
\pgfsetroundjoin%
\pgfsetlinewidth{0.803000pt}%
\definecolor{currentstroke}{rgb}{0.850000,0.850000,0.850000}%
\pgfsetstrokecolor{currentstroke}%
\pgfsetdash{}{0pt}%
\pgfpathmoveto{\pgfqpoint{0.594525in}{0.821119in}}%
\pgfpathlineto{\pgfqpoint{3.940488in}{0.821119in}}%
\pgfusepath{stroke}%
\end{pgfscope}%
\begin{pgfscope}%
\pgfsetbuttcap%
\pgfsetroundjoin%
\definecolor{currentfill}{rgb}{0.000000,0.000000,0.000000}%
\pgfsetfillcolor{currentfill}%
\pgfsetlinewidth{0.602250pt}%
\definecolor{currentstroke}{rgb}{0.000000,0.000000,0.000000}%
\pgfsetstrokecolor{currentstroke}%
\pgfsetdash{}{0pt}%
\pgfsys@defobject{currentmarker}{\pgfqpoint{-0.027778in}{0.000000in}}{\pgfqpoint{-0.000000in}{0.000000in}}{%
\pgfpathmoveto{\pgfqpoint{-0.000000in}{0.000000in}}%
\pgfpathlineto{\pgfqpoint{-0.027778in}{0.000000in}}%
\pgfusepath{stroke,fill}%
}%
\begin{pgfscope}%
\pgfsys@transformshift{0.594525in}{0.821119in}%
\pgfsys@useobject{currentmarker}{}%
\end{pgfscope}%
\end{pgfscope}%
\begin{pgfscope}%
\pgfpathrectangle{\pgfqpoint{0.594525in}{0.417642in}}{\pgfqpoint{3.345963in}{2.050688in}}%
\pgfusepath{clip}%
\pgfsetrectcap%
\pgfsetroundjoin%
\pgfsetlinewidth{0.803000pt}%
\definecolor{currentstroke}{rgb}{0.850000,0.850000,0.850000}%
\pgfsetstrokecolor{currentstroke}%
\pgfsetdash{}{0pt}%
\pgfpathmoveto{\pgfqpoint{0.594525in}{0.843062in}}%
\pgfpathlineto{\pgfqpoint{3.940488in}{0.843062in}}%
\pgfusepath{stroke}%
\end{pgfscope}%
\begin{pgfscope}%
\pgfsetbuttcap%
\pgfsetroundjoin%
\definecolor{currentfill}{rgb}{0.000000,0.000000,0.000000}%
\pgfsetfillcolor{currentfill}%
\pgfsetlinewidth{0.602250pt}%
\definecolor{currentstroke}{rgb}{0.000000,0.000000,0.000000}%
\pgfsetstrokecolor{currentstroke}%
\pgfsetdash{}{0pt}%
\pgfsys@defobject{currentmarker}{\pgfqpoint{-0.027778in}{0.000000in}}{\pgfqpoint{-0.000000in}{0.000000in}}{%
\pgfpathmoveto{\pgfqpoint{-0.000000in}{0.000000in}}%
\pgfpathlineto{\pgfqpoint{-0.027778in}{0.000000in}}%
\pgfusepath{stroke,fill}%
}%
\begin{pgfscope}%
\pgfsys@transformshift{0.594525in}{0.843062in}%
\pgfsys@useobject{currentmarker}{}%
\end{pgfscope}%
\end{pgfscope}%
\begin{pgfscope}%
\pgfpathrectangle{\pgfqpoint{0.594525in}{0.417642in}}{\pgfqpoint{3.345963in}{2.050688in}}%
\pgfusepath{clip}%
\pgfsetrectcap%
\pgfsetroundjoin%
\pgfsetlinewidth{0.803000pt}%
\definecolor{currentstroke}{rgb}{0.850000,0.850000,0.850000}%
\pgfsetstrokecolor{currentstroke}%
\pgfsetdash{}{0pt}%
\pgfpathmoveto{\pgfqpoint{0.594525in}{0.861614in}}%
\pgfpathlineto{\pgfqpoint{3.940488in}{0.861614in}}%
\pgfusepath{stroke}%
\end{pgfscope}%
\begin{pgfscope}%
\pgfsetbuttcap%
\pgfsetroundjoin%
\definecolor{currentfill}{rgb}{0.000000,0.000000,0.000000}%
\pgfsetfillcolor{currentfill}%
\pgfsetlinewidth{0.602250pt}%
\definecolor{currentstroke}{rgb}{0.000000,0.000000,0.000000}%
\pgfsetstrokecolor{currentstroke}%
\pgfsetdash{}{0pt}%
\pgfsys@defobject{currentmarker}{\pgfqpoint{-0.027778in}{0.000000in}}{\pgfqpoint{-0.000000in}{0.000000in}}{%
\pgfpathmoveto{\pgfqpoint{-0.000000in}{0.000000in}}%
\pgfpathlineto{\pgfqpoint{-0.027778in}{0.000000in}}%
\pgfusepath{stroke,fill}%
}%
\begin{pgfscope}%
\pgfsys@transformshift{0.594525in}{0.861614in}%
\pgfsys@useobject{currentmarker}{}%
\end{pgfscope}%
\end{pgfscope}%
\begin{pgfscope}%
\pgfpathrectangle{\pgfqpoint{0.594525in}{0.417642in}}{\pgfqpoint{3.345963in}{2.050688in}}%
\pgfusepath{clip}%
\pgfsetrectcap%
\pgfsetroundjoin%
\pgfsetlinewidth{0.803000pt}%
\definecolor{currentstroke}{rgb}{0.850000,0.850000,0.850000}%
\pgfsetstrokecolor{currentstroke}%
\pgfsetdash{}{0pt}%
\pgfpathmoveto{\pgfqpoint{0.594525in}{0.877685in}}%
\pgfpathlineto{\pgfqpoint{3.940488in}{0.877685in}}%
\pgfusepath{stroke}%
\end{pgfscope}%
\begin{pgfscope}%
\pgfsetbuttcap%
\pgfsetroundjoin%
\definecolor{currentfill}{rgb}{0.000000,0.000000,0.000000}%
\pgfsetfillcolor{currentfill}%
\pgfsetlinewidth{0.602250pt}%
\definecolor{currentstroke}{rgb}{0.000000,0.000000,0.000000}%
\pgfsetstrokecolor{currentstroke}%
\pgfsetdash{}{0pt}%
\pgfsys@defobject{currentmarker}{\pgfqpoint{-0.027778in}{0.000000in}}{\pgfqpoint{-0.000000in}{0.000000in}}{%
\pgfpathmoveto{\pgfqpoint{-0.000000in}{0.000000in}}%
\pgfpathlineto{\pgfqpoint{-0.027778in}{0.000000in}}%
\pgfusepath{stroke,fill}%
}%
\begin{pgfscope}%
\pgfsys@transformshift{0.594525in}{0.877685in}%
\pgfsys@useobject{currentmarker}{}%
\end{pgfscope}%
\end{pgfscope}%
\begin{pgfscope}%
\pgfpathrectangle{\pgfqpoint{0.594525in}{0.417642in}}{\pgfqpoint{3.345963in}{2.050688in}}%
\pgfusepath{clip}%
\pgfsetrectcap%
\pgfsetroundjoin%
\pgfsetlinewidth{0.803000pt}%
\definecolor{currentstroke}{rgb}{0.850000,0.850000,0.850000}%
\pgfsetstrokecolor{currentstroke}%
\pgfsetdash{}{0pt}%
\pgfpathmoveto{\pgfqpoint{0.594525in}{0.891860in}}%
\pgfpathlineto{\pgfqpoint{3.940488in}{0.891860in}}%
\pgfusepath{stroke}%
\end{pgfscope}%
\begin{pgfscope}%
\pgfsetbuttcap%
\pgfsetroundjoin%
\definecolor{currentfill}{rgb}{0.000000,0.000000,0.000000}%
\pgfsetfillcolor{currentfill}%
\pgfsetlinewidth{0.602250pt}%
\definecolor{currentstroke}{rgb}{0.000000,0.000000,0.000000}%
\pgfsetstrokecolor{currentstroke}%
\pgfsetdash{}{0pt}%
\pgfsys@defobject{currentmarker}{\pgfqpoint{-0.027778in}{0.000000in}}{\pgfqpoint{-0.000000in}{0.000000in}}{%
\pgfpathmoveto{\pgfqpoint{-0.000000in}{0.000000in}}%
\pgfpathlineto{\pgfqpoint{-0.027778in}{0.000000in}}%
\pgfusepath{stroke,fill}%
}%
\begin{pgfscope}%
\pgfsys@transformshift{0.594525in}{0.891860in}%
\pgfsys@useobject{currentmarker}{}%
\end{pgfscope}%
\end{pgfscope}%
\begin{pgfscope}%
\pgfpathrectangle{\pgfqpoint{0.594525in}{0.417642in}}{\pgfqpoint{3.345963in}{2.050688in}}%
\pgfusepath{clip}%
\pgfsetrectcap%
\pgfsetroundjoin%
\pgfsetlinewidth{0.803000pt}%
\definecolor{currentstroke}{rgb}{0.850000,0.850000,0.850000}%
\pgfsetstrokecolor{currentstroke}%
\pgfsetdash{}{0pt}%
\pgfpathmoveto{\pgfqpoint{0.594525in}{0.987962in}}%
\pgfpathlineto{\pgfqpoint{3.940488in}{0.987962in}}%
\pgfusepath{stroke}%
\end{pgfscope}%
\begin{pgfscope}%
\pgfsetbuttcap%
\pgfsetroundjoin%
\definecolor{currentfill}{rgb}{0.000000,0.000000,0.000000}%
\pgfsetfillcolor{currentfill}%
\pgfsetlinewidth{0.602250pt}%
\definecolor{currentstroke}{rgb}{0.000000,0.000000,0.000000}%
\pgfsetstrokecolor{currentstroke}%
\pgfsetdash{}{0pt}%
\pgfsys@defobject{currentmarker}{\pgfqpoint{-0.027778in}{0.000000in}}{\pgfqpoint{-0.000000in}{0.000000in}}{%
\pgfpathmoveto{\pgfqpoint{-0.000000in}{0.000000in}}%
\pgfpathlineto{\pgfqpoint{-0.027778in}{0.000000in}}%
\pgfusepath{stroke,fill}%
}%
\begin{pgfscope}%
\pgfsys@transformshift{0.594525in}{0.987962in}%
\pgfsys@useobject{currentmarker}{}%
\end{pgfscope}%
\end{pgfscope}%
\begin{pgfscope}%
\pgfpathrectangle{\pgfqpoint{0.594525in}{0.417642in}}{\pgfqpoint{3.345963in}{2.050688in}}%
\pgfusepath{clip}%
\pgfsetrectcap%
\pgfsetroundjoin%
\pgfsetlinewidth{0.803000pt}%
\definecolor{currentstroke}{rgb}{0.850000,0.850000,0.850000}%
\pgfsetstrokecolor{currentstroke}%
\pgfsetdash{}{0pt}%
\pgfpathmoveto{\pgfqpoint{0.594525in}{1.036760in}}%
\pgfpathlineto{\pgfqpoint{3.940488in}{1.036760in}}%
\pgfusepath{stroke}%
\end{pgfscope}%
\begin{pgfscope}%
\pgfsetbuttcap%
\pgfsetroundjoin%
\definecolor{currentfill}{rgb}{0.000000,0.000000,0.000000}%
\pgfsetfillcolor{currentfill}%
\pgfsetlinewidth{0.602250pt}%
\definecolor{currentstroke}{rgb}{0.000000,0.000000,0.000000}%
\pgfsetstrokecolor{currentstroke}%
\pgfsetdash{}{0pt}%
\pgfsys@defobject{currentmarker}{\pgfqpoint{-0.027778in}{0.000000in}}{\pgfqpoint{-0.000000in}{0.000000in}}{%
\pgfpathmoveto{\pgfqpoint{-0.000000in}{0.000000in}}%
\pgfpathlineto{\pgfqpoint{-0.027778in}{0.000000in}}%
\pgfusepath{stroke,fill}%
}%
\begin{pgfscope}%
\pgfsys@transformshift{0.594525in}{1.036760in}%
\pgfsys@useobject{currentmarker}{}%
\end{pgfscope}%
\end{pgfscope}%
\begin{pgfscope}%
\pgfpathrectangle{\pgfqpoint{0.594525in}{0.417642in}}{\pgfqpoint{3.345963in}{2.050688in}}%
\pgfusepath{clip}%
\pgfsetrectcap%
\pgfsetroundjoin%
\pgfsetlinewidth{0.803000pt}%
\definecolor{currentstroke}{rgb}{0.850000,0.850000,0.850000}%
\pgfsetstrokecolor{currentstroke}%
\pgfsetdash{}{0pt}%
\pgfpathmoveto{\pgfqpoint{0.594525in}{1.071383in}}%
\pgfpathlineto{\pgfqpoint{3.940488in}{1.071383in}}%
\pgfusepath{stroke}%
\end{pgfscope}%
\begin{pgfscope}%
\pgfsetbuttcap%
\pgfsetroundjoin%
\definecolor{currentfill}{rgb}{0.000000,0.000000,0.000000}%
\pgfsetfillcolor{currentfill}%
\pgfsetlinewidth{0.602250pt}%
\definecolor{currentstroke}{rgb}{0.000000,0.000000,0.000000}%
\pgfsetstrokecolor{currentstroke}%
\pgfsetdash{}{0pt}%
\pgfsys@defobject{currentmarker}{\pgfqpoint{-0.027778in}{0.000000in}}{\pgfqpoint{-0.000000in}{0.000000in}}{%
\pgfpathmoveto{\pgfqpoint{-0.000000in}{0.000000in}}%
\pgfpathlineto{\pgfqpoint{-0.027778in}{0.000000in}}%
\pgfusepath{stroke,fill}%
}%
\begin{pgfscope}%
\pgfsys@transformshift{0.594525in}{1.071383in}%
\pgfsys@useobject{currentmarker}{}%
\end{pgfscope}%
\end{pgfscope}%
\begin{pgfscope}%
\pgfpathrectangle{\pgfqpoint{0.594525in}{0.417642in}}{\pgfqpoint{3.345963in}{2.050688in}}%
\pgfusepath{clip}%
\pgfsetrectcap%
\pgfsetroundjoin%
\pgfsetlinewidth{0.803000pt}%
\definecolor{currentstroke}{rgb}{0.850000,0.850000,0.850000}%
\pgfsetstrokecolor{currentstroke}%
\pgfsetdash{}{0pt}%
\pgfpathmoveto{\pgfqpoint{0.594525in}{1.098239in}}%
\pgfpathlineto{\pgfqpoint{3.940488in}{1.098239in}}%
\pgfusepath{stroke}%
\end{pgfscope}%
\begin{pgfscope}%
\pgfsetbuttcap%
\pgfsetroundjoin%
\definecolor{currentfill}{rgb}{0.000000,0.000000,0.000000}%
\pgfsetfillcolor{currentfill}%
\pgfsetlinewidth{0.602250pt}%
\definecolor{currentstroke}{rgb}{0.000000,0.000000,0.000000}%
\pgfsetstrokecolor{currentstroke}%
\pgfsetdash{}{0pt}%
\pgfsys@defobject{currentmarker}{\pgfqpoint{-0.027778in}{0.000000in}}{\pgfqpoint{-0.000000in}{0.000000in}}{%
\pgfpathmoveto{\pgfqpoint{-0.000000in}{0.000000in}}%
\pgfpathlineto{\pgfqpoint{-0.027778in}{0.000000in}}%
\pgfusepath{stroke,fill}%
}%
\begin{pgfscope}%
\pgfsys@transformshift{0.594525in}{1.098239in}%
\pgfsys@useobject{currentmarker}{}%
\end{pgfscope}%
\end{pgfscope}%
\begin{pgfscope}%
\pgfpathrectangle{\pgfqpoint{0.594525in}{0.417642in}}{\pgfqpoint{3.345963in}{2.050688in}}%
\pgfusepath{clip}%
\pgfsetrectcap%
\pgfsetroundjoin%
\pgfsetlinewidth{0.803000pt}%
\definecolor{currentstroke}{rgb}{0.850000,0.850000,0.850000}%
\pgfsetstrokecolor{currentstroke}%
\pgfsetdash{}{0pt}%
\pgfpathmoveto{\pgfqpoint{0.594525in}{1.120182in}}%
\pgfpathlineto{\pgfqpoint{3.940488in}{1.120182in}}%
\pgfusepath{stroke}%
\end{pgfscope}%
\begin{pgfscope}%
\pgfsetbuttcap%
\pgfsetroundjoin%
\definecolor{currentfill}{rgb}{0.000000,0.000000,0.000000}%
\pgfsetfillcolor{currentfill}%
\pgfsetlinewidth{0.602250pt}%
\definecolor{currentstroke}{rgb}{0.000000,0.000000,0.000000}%
\pgfsetstrokecolor{currentstroke}%
\pgfsetdash{}{0pt}%
\pgfsys@defobject{currentmarker}{\pgfqpoint{-0.027778in}{0.000000in}}{\pgfqpoint{-0.000000in}{0.000000in}}{%
\pgfpathmoveto{\pgfqpoint{-0.000000in}{0.000000in}}%
\pgfpathlineto{\pgfqpoint{-0.027778in}{0.000000in}}%
\pgfusepath{stroke,fill}%
}%
\begin{pgfscope}%
\pgfsys@transformshift{0.594525in}{1.120182in}%
\pgfsys@useobject{currentmarker}{}%
\end{pgfscope}%
\end{pgfscope}%
\begin{pgfscope}%
\pgfpathrectangle{\pgfqpoint{0.594525in}{0.417642in}}{\pgfqpoint{3.345963in}{2.050688in}}%
\pgfusepath{clip}%
\pgfsetrectcap%
\pgfsetroundjoin%
\pgfsetlinewidth{0.803000pt}%
\definecolor{currentstroke}{rgb}{0.850000,0.850000,0.850000}%
\pgfsetstrokecolor{currentstroke}%
\pgfsetdash{}{0pt}%
\pgfpathmoveto{\pgfqpoint{0.594525in}{1.138734in}}%
\pgfpathlineto{\pgfqpoint{3.940488in}{1.138734in}}%
\pgfusepath{stroke}%
\end{pgfscope}%
\begin{pgfscope}%
\pgfsetbuttcap%
\pgfsetroundjoin%
\definecolor{currentfill}{rgb}{0.000000,0.000000,0.000000}%
\pgfsetfillcolor{currentfill}%
\pgfsetlinewidth{0.602250pt}%
\definecolor{currentstroke}{rgb}{0.000000,0.000000,0.000000}%
\pgfsetstrokecolor{currentstroke}%
\pgfsetdash{}{0pt}%
\pgfsys@defobject{currentmarker}{\pgfqpoint{-0.027778in}{0.000000in}}{\pgfqpoint{-0.000000in}{0.000000in}}{%
\pgfpathmoveto{\pgfqpoint{-0.000000in}{0.000000in}}%
\pgfpathlineto{\pgfqpoint{-0.027778in}{0.000000in}}%
\pgfusepath{stroke,fill}%
}%
\begin{pgfscope}%
\pgfsys@transformshift{0.594525in}{1.138734in}%
\pgfsys@useobject{currentmarker}{}%
\end{pgfscope}%
\end{pgfscope}%
\begin{pgfscope}%
\pgfpathrectangle{\pgfqpoint{0.594525in}{0.417642in}}{\pgfqpoint{3.345963in}{2.050688in}}%
\pgfusepath{clip}%
\pgfsetrectcap%
\pgfsetroundjoin%
\pgfsetlinewidth{0.803000pt}%
\definecolor{currentstroke}{rgb}{0.850000,0.850000,0.850000}%
\pgfsetstrokecolor{currentstroke}%
\pgfsetdash{}{0pt}%
\pgfpathmoveto{\pgfqpoint{0.594525in}{1.154805in}}%
\pgfpathlineto{\pgfqpoint{3.940488in}{1.154805in}}%
\pgfusepath{stroke}%
\end{pgfscope}%
\begin{pgfscope}%
\pgfsetbuttcap%
\pgfsetroundjoin%
\definecolor{currentfill}{rgb}{0.000000,0.000000,0.000000}%
\pgfsetfillcolor{currentfill}%
\pgfsetlinewidth{0.602250pt}%
\definecolor{currentstroke}{rgb}{0.000000,0.000000,0.000000}%
\pgfsetstrokecolor{currentstroke}%
\pgfsetdash{}{0pt}%
\pgfsys@defobject{currentmarker}{\pgfqpoint{-0.027778in}{0.000000in}}{\pgfqpoint{-0.000000in}{0.000000in}}{%
\pgfpathmoveto{\pgfqpoint{-0.000000in}{0.000000in}}%
\pgfpathlineto{\pgfqpoint{-0.027778in}{0.000000in}}%
\pgfusepath{stroke,fill}%
}%
\begin{pgfscope}%
\pgfsys@transformshift{0.594525in}{1.154805in}%
\pgfsys@useobject{currentmarker}{}%
\end{pgfscope}%
\end{pgfscope}%
\begin{pgfscope}%
\pgfpathrectangle{\pgfqpoint{0.594525in}{0.417642in}}{\pgfqpoint{3.345963in}{2.050688in}}%
\pgfusepath{clip}%
\pgfsetrectcap%
\pgfsetroundjoin%
\pgfsetlinewidth{0.803000pt}%
\definecolor{currentstroke}{rgb}{0.850000,0.850000,0.850000}%
\pgfsetstrokecolor{currentstroke}%
\pgfsetdash{}{0pt}%
\pgfpathmoveto{\pgfqpoint{0.594525in}{1.168980in}}%
\pgfpathlineto{\pgfqpoint{3.940488in}{1.168980in}}%
\pgfusepath{stroke}%
\end{pgfscope}%
\begin{pgfscope}%
\pgfsetbuttcap%
\pgfsetroundjoin%
\definecolor{currentfill}{rgb}{0.000000,0.000000,0.000000}%
\pgfsetfillcolor{currentfill}%
\pgfsetlinewidth{0.602250pt}%
\definecolor{currentstroke}{rgb}{0.000000,0.000000,0.000000}%
\pgfsetstrokecolor{currentstroke}%
\pgfsetdash{}{0pt}%
\pgfsys@defobject{currentmarker}{\pgfqpoint{-0.027778in}{0.000000in}}{\pgfqpoint{-0.000000in}{0.000000in}}{%
\pgfpathmoveto{\pgfqpoint{-0.000000in}{0.000000in}}%
\pgfpathlineto{\pgfqpoint{-0.027778in}{0.000000in}}%
\pgfusepath{stroke,fill}%
}%
\begin{pgfscope}%
\pgfsys@transformshift{0.594525in}{1.168980in}%
\pgfsys@useobject{currentmarker}{}%
\end{pgfscope}%
\end{pgfscope}%
\begin{pgfscope}%
\pgfpathrectangle{\pgfqpoint{0.594525in}{0.417642in}}{\pgfqpoint{3.345963in}{2.050688in}}%
\pgfusepath{clip}%
\pgfsetrectcap%
\pgfsetroundjoin%
\pgfsetlinewidth{0.803000pt}%
\definecolor{currentstroke}{rgb}{0.850000,0.850000,0.850000}%
\pgfsetstrokecolor{currentstroke}%
\pgfsetdash{}{0pt}%
\pgfpathmoveto{\pgfqpoint{0.594525in}{1.265082in}}%
\pgfpathlineto{\pgfqpoint{3.940488in}{1.265082in}}%
\pgfusepath{stroke}%
\end{pgfscope}%
\begin{pgfscope}%
\pgfsetbuttcap%
\pgfsetroundjoin%
\definecolor{currentfill}{rgb}{0.000000,0.000000,0.000000}%
\pgfsetfillcolor{currentfill}%
\pgfsetlinewidth{0.602250pt}%
\definecolor{currentstroke}{rgb}{0.000000,0.000000,0.000000}%
\pgfsetstrokecolor{currentstroke}%
\pgfsetdash{}{0pt}%
\pgfsys@defobject{currentmarker}{\pgfqpoint{-0.027778in}{0.000000in}}{\pgfqpoint{-0.000000in}{0.000000in}}{%
\pgfpathmoveto{\pgfqpoint{-0.000000in}{0.000000in}}%
\pgfpathlineto{\pgfqpoint{-0.027778in}{0.000000in}}%
\pgfusepath{stroke,fill}%
}%
\begin{pgfscope}%
\pgfsys@transformshift{0.594525in}{1.265082in}%
\pgfsys@useobject{currentmarker}{}%
\end{pgfscope}%
\end{pgfscope}%
\begin{pgfscope}%
\pgfpathrectangle{\pgfqpoint{0.594525in}{0.417642in}}{\pgfqpoint{3.345963in}{2.050688in}}%
\pgfusepath{clip}%
\pgfsetrectcap%
\pgfsetroundjoin%
\pgfsetlinewidth{0.803000pt}%
\definecolor{currentstroke}{rgb}{0.850000,0.850000,0.850000}%
\pgfsetstrokecolor{currentstroke}%
\pgfsetdash{}{0pt}%
\pgfpathmoveto{\pgfqpoint{0.594525in}{1.313881in}}%
\pgfpathlineto{\pgfqpoint{3.940488in}{1.313881in}}%
\pgfusepath{stroke}%
\end{pgfscope}%
\begin{pgfscope}%
\pgfsetbuttcap%
\pgfsetroundjoin%
\definecolor{currentfill}{rgb}{0.000000,0.000000,0.000000}%
\pgfsetfillcolor{currentfill}%
\pgfsetlinewidth{0.602250pt}%
\definecolor{currentstroke}{rgb}{0.000000,0.000000,0.000000}%
\pgfsetstrokecolor{currentstroke}%
\pgfsetdash{}{0pt}%
\pgfsys@defobject{currentmarker}{\pgfqpoint{-0.027778in}{0.000000in}}{\pgfqpoint{-0.000000in}{0.000000in}}{%
\pgfpathmoveto{\pgfqpoint{-0.000000in}{0.000000in}}%
\pgfpathlineto{\pgfqpoint{-0.027778in}{0.000000in}}%
\pgfusepath{stroke,fill}%
}%
\begin{pgfscope}%
\pgfsys@transformshift{0.594525in}{1.313881in}%
\pgfsys@useobject{currentmarker}{}%
\end{pgfscope}%
\end{pgfscope}%
\begin{pgfscope}%
\pgfpathrectangle{\pgfqpoint{0.594525in}{0.417642in}}{\pgfqpoint{3.345963in}{2.050688in}}%
\pgfusepath{clip}%
\pgfsetrectcap%
\pgfsetroundjoin%
\pgfsetlinewidth{0.803000pt}%
\definecolor{currentstroke}{rgb}{0.850000,0.850000,0.850000}%
\pgfsetstrokecolor{currentstroke}%
\pgfsetdash{}{0pt}%
\pgfpathmoveto{\pgfqpoint{0.594525in}{1.348504in}}%
\pgfpathlineto{\pgfqpoint{3.940488in}{1.348504in}}%
\pgfusepath{stroke}%
\end{pgfscope}%
\begin{pgfscope}%
\pgfsetbuttcap%
\pgfsetroundjoin%
\definecolor{currentfill}{rgb}{0.000000,0.000000,0.000000}%
\pgfsetfillcolor{currentfill}%
\pgfsetlinewidth{0.602250pt}%
\definecolor{currentstroke}{rgb}{0.000000,0.000000,0.000000}%
\pgfsetstrokecolor{currentstroke}%
\pgfsetdash{}{0pt}%
\pgfsys@defobject{currentmarker}{\pgfqpoint{-0.027778in}{0.000000in}}{\pgfqpoint{-0.000000in}{0.000000in}}{%
\pgfpathmoveto{\pgfqpoint{-0.000000in}{0.000000in}}%
\pgfpathlineto{\pgfqpoint{-0.027778in}{0.000000in}}%
\pgfusepath{stroke,fill}%
}%
\begin{pgfscope}%
\pgfsys@transformshift{0.594525in}{1.348504in}%
\pgfsys@useobject{currentmarker}{}%
\end{pgfscope}%
\end{pgfscope}%
\begin{pgfscope}%
\pgfpathrectangle{\pgfqpoint{0.594525in}{0.417642in}}{\pgfqpoint{3.345963in}{2.050688in}}%
\pgfusepath{clip}%
\pgfsetrectcap%
\pgfsetroundjoin%
\pgfsetlinewidth{0.803000pt}%
\definecolor{currentstroke}{rgb}{0.850000,0.850000,0.850000}%
\pgfsetstrokecolor{currentstroke}%
\pgfsetdash{}{0pt}%
\pgfpathmoveto{\pgfqpoint{0.594525in}{1.375360in}}%
\pgfpathlineto{\pgfqpoint{3.940488in}{1.375360in}}%
\pgfusepath{stroke}%
\end{pgfscope}%
\begin{pgfscope}%
\pgfsetbuttcap%
\pgfsetroundjoin%
\definecolor{currentfill}{rgb}{0.000000,0.000000,0.000000}%
\pgfsetfillcolor{currentfill}%
\pgfsetlinewidth{0.602250pt}%
\definecolor{currentstroke}{rgb}{0.000000,0.000000,0.000000}%
\pgfsetstrokecolor{currentstroke}%
\pgfsetdash{}{0pt}%
\pgfsys@defobject{currentmarker}{\pgfqpoint{-0.027778in}{0.000000in}}{\pgfqpoint{-0.000000in}{0.000000in}}{%
\pgfpathmoveto{\pgfqpoint{-0.000000in}{0.000000in}}%
\pgfpathlineto{\pgfqpoint{-0.027778in}{0.000000in}}%
\pgfusepath{stroke,fill}%
}%
\begin{pgfscope}%
\pgfsys@transformshift{0.594525in}{1.375360in}%
\pgfsys@useobject{currentmarker}{}%
\end{pgfscope}%
\end{pgfscope}%
\begin{pgfscope}%
\pgfpathrectangle{\pgfqpoint{0.594525in}{0.417642in}}{\pgfqpoint{3.345963in}{2.050688in}}%
\pgfusepath{clip}%
\pgfsetrectcap%
\pgfsetroundjoin%
\pgfsetlinewidth{0.803000pt}%
\definecolor{currentstroke}{rgb}{0.850000,0.850000,0.850000}%
\pgfsetstrokecolor{currentstroke}%
\pgfsetdash{}{0pt}%
\pgfpathmoveto{\pgfqpoint{0.594525in}{1.397302in}}%
\pgfpathlineto{\pgfqpoint{3.940488in}{1.397302in}}%
\pgfusepath{stroke}%
\end{pgfscope}%
\begin{pgfscope}%
\pgfsetbuttcap%
\pgfsetroundjoin%
\definecolor{currentfill}{rgb}{0.000000,0.000000,0.000000}%
\pgfsetfillcolor{currentfill}%
\pgfsetlinewidth{0.602250pt}%
\definecolor{currentstroke}{rgb}{0.000000,0.000000,0.000000}%
\pgfsetstrokecolor{currentstroke}%
\pgfsetdash{}{0pt}%
\pgfsys@defobject{currentmarker}{\pgfqpoint{-0.027778in}{0.000000in}}{\pgfqpoint{-0.000000in}{0.000000in}}{%
\pgfpathmoveto{\pgfqpoint{-0.000000in}{0.000000in}}%
\pgfpathlineto{\pgfqpoint{-0.027778in}{0.000000in}}%
\pgfusepath{stroke,fill}%
}%
\begin{pgfscope}%
\pgfsys@transformshift{0.594525in}{1.397302in}%
\pgfsys@useobject{currentmarker}{}%
\end{pgfscope}%
\end{pgfscope}%
\begin{pgfscope}%
\pgfpathrectangle{\pgfqpoint{0.594525in}{0.417642in}}{\pgfqpoint{3.345963in}{2.050688in}}%
\pgfusepath{clip}%
\pgfsetrectcap%
\pgfsetroundjoin%
\pgfsetlinewidth{0.803000pt}%
\definecolor{currentstroke}{rgb}{0.850000,0.850000,0.850000}%
\pgfsetstrokecolor{currentstroke}%
\pgfsetdash{}{0pt}%
\pgfpathmoveto{\pgfqpoint{0.594525in}{1.415855in}}%
\pgfpathlineto{\pgfqpoint{3.940488in}{1.415855in}}%
\pgfusepath{stroke}%
\end{pgfscope}%
\begin{pgfscope}%
\pgfsetbuttcap%
\pgfsetroundjoin%
\definecolor{currentfill}{rgb}{0.000000,0.000000,0.000000}%
\pgfsetfillcolor{currentfill}%
\pgfsetlinewidth{0.602250pt}%
\definecolor{currentstroke}{rgb}{0.000000,0.000000,0.000000}%
\pgfsetstrokecolor{currentstroke}%
\pgfsetdash{}{0pt}%
\pgfsys@defobject{currentmarker}{\pgfqpoint{-0.027778in}{0.000000in}}{\pgfqpoint{-0.000000in}{0.000000in}}{%
\pgfpathmoveto{\pgfqpoint{-0.000000in}{0.000000in}}%
\pgfpathlineto{\pgfqpoint{-0.027778in}{0.000000in}}%
\pgfusepath{stroke,fill}%
}%
\begin{pgfscope}%
\pgfsys@transformshift{0.594525in}{1.415855in}%
\pgfsys@useobject{currentmarker}{}%
\end{pgfscope}%
\end{pgfscope}%
\begin{pgfscope}%
\pgfpathrectangle{\pgfqpoint{0.594525in}{0.417642in}}{\pgfqpoint{3.345963in}{2.050688in}}%
\pgfusepath{clip}%
\pgfsetrectcap%
\pgfsetroundjoin%
\pgfsetlinewidth{0.803000pt}%
\definecolor{currentstroke}{rgb}{0.850000,0.850000,0.850000}%
\pgfsetstrokecolor{currentstroke}%
\pgfsetdash{}{0pt}%
\pgfpathmoveto{\pgfqpoint{0.594525in}{1.431925in}}%
\pgfpathlineto{\pgfqpoint{3.940488in}{1.431925in}}%
\pgfusepath{stroke}%
\end{pgfscope}%
\begin{pgfscope}%
\pgfsetbuttcap%
\pgfsetroundjoin%
\definecolor{currentfill}{rgb}{0.000000,0.000000,0.000000}%
\pgfsetfillcolor{currentfill}%
\pgfsetlinewidth{0.602250pt}%
\definecolor{currentstroke}{rgb}{0.000000,0.000000,0.000000}%
\pgfsetstrokecolor{currentstroke}%
\pgfsetdash{}{0pt}%
\pgfsys@defobject{currentmarker}{\pgfqpoint{-0.027778in}{0.000000in}}{\pgfqpoint{-0.000000in}{0.000000in}}{%
\pgfpathmoveto{\pgfqpoint{-0.000000in}{0.000000in}}%
\pgfpathlineto{\pgfqpoint{-0.027778in}{0.000000in}}%
\pgfusepath{stroke,fill}%
}%
\begin{pgfscope}%
\pgfsys@transformshift{0.594525in}{1.431925in}%
\pgfsys@useobject{currentmarker}{}%
\end{pgfscope}%
\end{pgfscope}%
\begin{pgfscope}%
\pgfpathrectangle{\pgfqpoint{0.594525in}{0.417642in}}{\pgfqpoint{3.345963in}{2.050688in}}%
\pgfusepath{clip}%
\pgfsetrectcap%
\pgfsetroundjoin%
\pgfsetlinewidth{0.803000pt}%
\definecolor{currentstroke}{rgb}{0.850000,0.850000,0.850000}%
\pgfsetstrokecolor{currentstroke}%
\pgfsetdash{}{0pt}%
\pgfpathmoveto{\pgfqpoint{0.594525in}{1.446101in}}%
\pgfpathlineto{\pgfqpoint{3.940488in}{1.446101in}}%
\pgfusepath{stroke}%
\end{pgfscope}%
\begin{pgfscope}%
\pgfsetbuttcap%
\pgfsetroundjoin%
\definecolor{currentfill}{rgb}{0.000000,0.000000,0.000000}%
\pgfsetfillcolor{currentfill}%
\pgfsetlinewidth{0.602250pt}%
\definecolor{currentstroke}{rgb}{0.000000,0.000000,0.000000}%
\pgfsetstrokecolor{currentstroke}%
\pgfsetdash{}{0pt}%
\pgfsys@defobject{currentmarker}{\pgfqpoint{-0.027778in}{0.000000in}}{\pgfqpoint{-0.000000in}{0.000000in}}{%
\pgfpathmoveto{\pgfqpoint{-0.000000in}{0.000000in}}%
\pgfpathlineto{\pgfqpoint{-0.027778in}{0.000000in}}%
\pgfusepath{stroke,fill}%
}%
\begin{pgfscope}%
\pgfsys@transformshift{0.594525in}{1.446101in}%
\pgfsys@useobject{currentmarker}{}%
\end{pgfscope}%
\end{pgfscope}%
\begin{pgfscope}%
\pgfpathrectangle{\pgfqpoint{0.594525in}{0.417642in}}{\pgfqpoint{3.345963in}{2.050688in}}%
\pgfusepath{clip}%
\pgfsetrectcap%
\pgfsetroundjoin%
\pgfsetlinewidth{0.803000pt}%
\definecolor{currentstroke}{rgb}{0.850000,0.850000,0.850000}%
\pgfsetstrokecolor{currentstroke}%
\pgfsetdash{}{0pt}%
\pgfpathmoveto{\pgfqpoint{0.594525in}{1.542203in}}%
\pgfpathlineto{\pgfqpoint{3.940488in}{1.542203in}}%
\pgfusepath{stroke}%
\end{pgfscope}%
\begin{pgfscope}%
\pgfsetbuttcap%
\pgfsetroundjoin%
\definecolor{currentfill}{rgb}{0.000000,0.000000,0.000000}%
\pgfsetfillcolor{currentfill}%
\pgfsetlinewidth{0.602250pt}%
\definecolor{currentstroke}{rgb}{0.000000,0.000000,0.000000}%
\pgfsetstrokecolor{currentstroke}%
\pgfsetdash{}{0pt}%
\pgfsys@defobject{currentmarker}{\pgfqpoint{-0.027778in}{0.000000in}}{\pgfqpoint{-0.000000in}{0.000000in}}{%
\pgfpathmoveto{\pgfqpoint{-0.000000in}{0.000000in}}%
\pgfpathlineto{\pgfqpoint{-0.027778in}{0.000000in}}%
\pgfusepath{stroke,fill}%
}%
\begin{pgfscope}%
\pgfsys@transformshift{0.594525in}{1.542203in}%
\pgfsys@useobject{currentmarker}{}%
\end{pgfscope}%
\end{pgfscope}%
\begin{pgfscope}%
\pgfpathrectangle{\pgfqpoint{0.594525in}{0.417642in}}{\pgfqpoint{3.345963in}{2.050688in}}%
\pgfusepath{clip}%
\pgfsetrectcap%
\pgfsetroundjoin%
\pgfsetlinewidth{0.803000pt}%
\definecolor{currentstroke}{rgb}{0.850000,0.850000,0.850000}%
\pgfsetstrokecolor{currentstroke}%
\pgfsetdash{}{0pt}%
\pgfpathmoveto{\pgfqpoint{0.594525in}{1.591001in}}%
\pgfpathlineto{\pgfqpoint{3.940488in}{1.591001in}}%
\pgfusepath{stroke}%
\end{pgfscope}%
\begin{pgfscope}%
\pgfsetbuttcap%
\pgfsetroundjoin%
\definecolor{currentfill}{rgb}{0.000000,0.000000,0.000000}%
\pgfsetfillcolor{currentfill}%
\pgfsetlinewidth{0.602250pt}%
\definecolor{currentstroke}{rgb}{0.000000,0.000000,0.000000}%
\pgfsetstrokecolor{currentstroke}%
\pgfsetdash{}{0pt}%
\pgfsys@defobject{currentmarker}{\pgfqpoint{-0.027778in}{0.000000in}}{\pgfqpoint{-0.000000in}{0.000000in}}{%
\pgfpathmoveto{\pgfqpoint{-0.000000in}{0.000000in}}%
\pgfpathlineto{\pgfqpoint{-0.027778in}{0.000000in}}%
\pgfusepath{stroke,fill}%
}%
\begin{pgfscope}%
\pgfsys@transformshift{0.594525in}{1.591001in}%
\pgfsys@useobject{currentmarker}{}%
\end{pgfscope}%
\end{pgfscope}%
\begin{pgfscope}%
\pgfpathrectangle{\pgfqpoint{0.594525in}{0.417642in}}{\pgfqpoint{3.345963in}{2.050688in}}%
\pgfusepath{clip}%
\pgfsetrectcap%
\pgfsetroundjoin%
\pgfsetlinewidth{0.803000pt}%
\definecolor{currentstroke}{rgb}{0.850000,0.850000,0.850000}%
\pgfsetstrokecolor{currentstroke}%
\pgfsetdash{}{0pt}%
\pgfpathmoveto{\pgfqpoint{0.594525in}{1.625624in}}%
\pgfpathlineto{\pgfqpoint{3.940488in}{1.625624in}}%
\pgfusepath{stroke}%
\end{pgfscope}%
\begin{pgfscope}%
\pgfsetbuttcap%
\pgfsetroundjoin%
\definecolor{currentfill}{rgb}{0.000000,0.000000,0.000000}%
\pgfsetfillcolor{currentfill}%
\pgfsetlinewidth{0.602250pt}%
\definecolor{currentstroke}{rgb}{0.000000,0.000000,0.000000}%
\pgfsetstrokecolor{currentstroke}%
\pgfsetdash{}{0pt}%
\pgfsys@defobject{currentmarker}{\pgfqpoint{-0.027778in}{0.000000in}}{\pgfqpoint{-0.000000in}{0.000000in}}{%
\pgfpathmoveto{\pgfqpoint{-0.000000in}{0.000000in}}%
\pgfpathlineto{\pgfqpoint{-0.027778in}{0.000000in}}%
\pgfusepath{stroke,fill}%
}%
\begin{pgfscope}%
\pgfsys@transformshift{0.594525in}{1.625624in}%
\pgfsys@useobject{currentmarker}{}%
\end{pgfscope}%
\end{pgfscope}%
\begin{pgfscope}%
\pgfpathrectangle{\pgfqpoint{0.594525in}{0.417642in}}{\pgfqpoint{3.345963in}{2.050688in}}%
\pgfusepath{clip}%
\pgfsetrectcap%
\pgfsetroundjoin%
\pgfsetlinewidth{0.803000pt}%
\definecolor{currentstroke}{rgb}{0.850000,0.850000,0.850000}%
\pgfsetstrokecolor{currentstroke}%
\pgfsetdash{}{0pt}%
\pgfpathmoveto{\pgfqpoint{0.594525in}{1.652480in}}%
\pgfpathlineto{\pgfqpoint{3.940488in}{1.652480in}}%
\pgfusepath{stroke}%
\end{pgfscope}%
\begin{pgfscope}%
\pgfsetbuttcap%
\pgfsetroundjoin%
\definecolor{currentfill}{rgb}{0.000000,0.000000,0.000000}%
\pgfsetfillcolor{currentfill}%
\pgfsetlinewidth{0.602250pt}%
\definecolor{currentstroke}{rgb}{0.000000,0.000000,0.000000}%
\pgfsetstrokecolor{currentstroke}%
\pgfsetdash{}{0pt}%
\pgfsys@defobject{currentmarker}{\pgfqpoint{-0.027778in}{0.000000in}}{\pgfqpoint{-0.000000in}{0.000000in}}{%
\pgfpathmoveto{\pgfqpoint{-0.000000in}{0.000000in}}%
\pgfpathlineto{\pgfqpoint{-0.027778in}{0.000000in}}%
\pgfusepath{stroke,fill}%
}%
\begin{pgfscope}%
\pgfsys@transformshift{0.594525in}{1.652480in}%
\pgfsys@useobject{currentmarker}{}%
\end{pgfscope}%
\end{pgfscope}%
\begin{pgfscope}%
\pgfpathrectangle{\pgfqpoint{0.594525in}{0.417642in}}{\pgfqpoint{3.345963in}{2.050688in}}%
\pgfusepath{clip}%
\pgfsetrectcap%
\pgfsetroundjoin%
\pgfsetlinewidth{0.803000pt}%
\definecolor{currentstroke}{rgb}{0.850000,0.850000,0.850000}%
\pgfsetstrokecolor{currentstroke}%
\pgfsetdash{}{0pt}%
\pgfpathmoveto{\pgfqpoint{0.594525in}{1.674423in}}%
\pgfpathlineto{\pgfqpoint{3.940488in}{1.674423in}}%
\pgfusepath{stroke}%
\end{pgfscope}%
\begin{pgfscope}%
\pgfsetbuttcap%
\pgfsetroundjoin%
\definecolor{currentfill}{rgb}{0.000000,0.000000,0.000000}%
\pgfsetfillcolor{currentfill}%
\pgfsetlinewidth{0.602250pt}%
\definecolor{currentstroke}{rgb}{0.000000,0.000000,0.000000}%
\pgfsetstrokecolor{currentstroke}%
\pgfsetdash{}{0pt}%
\pgfsys@defobject{currentmarker}{\pgfqpoint{-0.027778in}{0.000000in}}{\pgfqpoint{-0.000000in}{0.000000in}}{%
\pgfpathmoveto{\pgfqpoint{-0.000000in}{0.000000in}}%
\pgfpathlineto{\pgfqpoint{-0.027778in}{0.000000in}}%
\pgfusepath{stroke,fill}%
}%
\begin{pgfscope}%
\pgfsys@transformshift{0.594525in}{1.674423in}%
\pgfsys@useobject{currentmarker}{}%
\end{pgfscope}%
\end{pgfscope}%
\begin{pgfscope}%
\pgfpathrectangle{\pgfqpoint{0.594525in}{0.417642in}}{\pgfqpoint{3.345963in}{2.050688in}}%
\pgfusepath{clip}%
\pgfsetrectcap%
\pgfsetroundjoin%
\pgfsetlinewidth{0.803000pt}%
\definecolor{currentstroke}{rgb}{0.850000,0.850000,0.850000}%
\pgfsetstrokecolor{currentstroke}%
\pgfsetdash{}{0pt}%
\pgfpathmoveto{\pgfqpoint{0.594525in}{1.692975in}}%
\pgfpathlineto{\pgfqpoint{3.940488in}{1.692975in}}%
\pgfusepath{stroke}%
\end{pgfscope}%
\begin{pgfscope}%
\pgfsetbuttcap%
\pgfsetroundjoin%
\definecolor{currentfill}{rgb}{0.000000,0.000000,0.000000}%
\pgfsetfillcolor{currentfill}%
\pgfsetlinewidth{0.602250pt}%
\definecolor{currentstroke}{rgb}{0.000000,0.000000,0.000000}%
\pgfsetstrokecolor{currentstroke}%
\pgfsetdash{}{0pt}%
\pgfsys@defobject{currentmarker}{\pgfqpoint{-0.027778in}{0.000000in}}{\pgfqpoint{-0.000000in}{0.000000in}}{%
\pgfpathmoveto{\pgfqpoint{-0.000000in}{0.000000in}}%
\pgfpathlineto{\pgfqpoint{-0.027778in}{0.000000in}}%
\pgfusepath{stroke,fill}%
}%
\begin{pgfscope}%
\pgfsys@transformshift{0.594525in}{1.692975in}%
\pgfsys@useobject{currentmarker}{}%
\end{pgfscope}%
\end{pgfscope}%
\begin{pgfscope}%
\pgfpathrectangle{\pgfqpoint{0.594525in}{0.417642in}}{\pgfqpoint{3.345963in}{2.050688in}}%
\pgfusepath{clip}%
\pgfsetrectcap%
\pgfsetroundjoin%
\pgfsetlinewidth{0.803000pt}%
\definecolor{currentstroke}{rgb}{0.850000,0.850000,0.850000}%
\pgfsetstrokecolor{currentstroke}%
\pgfsetdash{}{0pt}%
\pgfpathmoveto{\pgfqpoint{0.594525in}{1.709046in}}%
\pgfpathlineto{\pgfqpoint{3.940488in}{1.709046in}}%
\pgfusepath{stroke}%
\end{pgfscope}%
\begin{pgfscope}%
\pgfsetbuttcap%
\pgfsetroundjoin%
\definecolor{currentfill}{rgb}{0.000000,0.000000,0.000000}%
\pgfsetfillcolor{currentfill}%
\pgfsetlinewidth{0.602250pt}%
\definecolor{currentstroke}{rgb}{0.000000,0.000000,0.000000}%
\pgfsetstrokecolor{currentstroke}%
\pgfsetdash{}{0pt}%
\pgfsys@defobject{currentmarker}{\pgfqpoint{-0.027778in}{0.000000in}}{\pgfqpoint{-0.000000in}{0.000000in}}{%
\pgfpathmoveto{\pgfqpoint{-0.000000in}{0.000000in}}%
\pgfpathlineto{\pgfqpoint{-0.027778in}{0.000000in}}%
\pgfusepath{stroke,fill}%
}%
\begin{pgfscope}%
\pgfsys@transformshift{0.594525in}{1.709046in}%
\pgfsys@useobject{currentmarker}{}%
\end{pgfscope}%
\end{pgfscope}%
\begin{pgfscope}%
\pgfpathrectangle{\pgfqpoint{0.594525in}{0.417642in}}{\pgfqpoint{3.345963in}{2.050688in}}%
\pgfusepath{clip}%
\pgfsetrectcap%
\pgfsetroundjoin%
\pgfsetlinewidth{0.803000pt}%
\definecolor{currentstroke}{rgb}{0.850000,0.850000,0.850000}%
\pgfsetstrokecolor{currentstroke}%
\pgfsetdash{}{0pt}%
\pgfpathmoveto{\pgfqpoint{0.594525in}{1.723221in}}%
\pgfpathlineto{\pgfqpoint{3.940488in}{1.723221in}}%
\pgfusepath{stroke}%
\end{pgfscope}%
\begin{pgfscope}%
\pgfsetbuttcap%
\pgfsetroundjoin%
\definecolor{currentfill}{rgb}{0.000000,0.000000,0.000000}%
\pgfsetfillcolor{currentfill}%
\pgfsetlinewidth{0.602250pt}%
\definecolor{currentstroke}{rgb}{0.000000,0.000000,0.000000}%
\pgfsetstrokecolor{currentstroke}%
\pgfsetdash{}{0pt}%
\pgfsys@defobject{currentmarker}{\pgfqpoint{-0.027778in}{0.000000in}}{\pgfqpoint{-0.000000in}{0.000000in}}{%
\pgfpathmoveto{\pgfqpoint{-0.000000in}{0.000000in}}%
\pgfpathlineto{\pgfqpoint{-0.027778in}{0.000000in}}%
\pgfusepath{stroke,fill}%
}%
\begin{pgfscope}%
\pgfsys@transformshift{0.594525in}{1.723221in}%
\pgfsys@useobject{currentmarker}{}%
\end{pgfscope}%
\end{pgfscope}%
\begin{pgfscope}%
\pgfpathrectangle{\pgfqpoint{0.594525in}{0.417642in}}{\pgfqpoint{3.345963in}{2.050688in}}%
\pgfusepath{clip}%
\pgfsetrectcap%
\pgfsetroundjoin%
\pgfsetlinewidth{0.803000pt}%
\definecolor{currentstroke}{rgb}{0.850000,0.850000,0.850000}%
\pgfsetstrokecolor{currentstroke}%
\pgfsetdash{}{0pt}%
\pgfpathmoveto{\pgfqpoint{0.594525in}{1.819323in}}%
\pgfpathlineto{\pgfqpoint{3.940488in}{1.819323in}}%
\pgfusepath{stroke}%
\end{pgfscope}%
\begin{pgfscope}%
\pgfsetbuttcap%
\pgfsetroundjoin%
\definecolor{currentfill}{rgb}{0.000000,0.000000,0.000000}%
\pgfsetfillcolor{currentfill}%
\pgfsetlinewidth{0.602250pt}%
\definecolor{currentstroke}{rgb}{0.000000,0.000000,0.000000}%
\pgfsetstrokecolor{currentstroke}%
\pgfsetdash{}{0pt}%
\pgfsys@defobject{currentmarker}{\pgfqpoint{-0.027778in}{0.000000in}}{\pgfqpoint{-0.000000in}{0.000000in}}{%
\pgfpathmoveto{\pgfqpoint{-0.000000in}{0.000000in}}%
\pgfpathlineto{\pgfqpoint{-0.027778in}{0.000000in}}%
\pgfusepath{stroke,fill}%
}%
\begin{pgfscope}%
\pgfsys@transformshift{0.594525in}{1.819323in}%
\pgfsys@useobject{currentmarker}{}%
\end{pgfscope}%
\end{pgfscope}%
\begin{pgfscope}%
\pgfpathrectangle{\pgfqpoint{0.594525in}{0.417642in}}{\pgfqpoint{3.345963in}{2.050688in}}%
\pgfusepath{clip}%
\pgfsetrectcap%
\pgfsetroundjoin%
\pgfsetlinewidth{0.803000pt}%
\definecolor{currentstroke}{rgb}{0.850000,0.850000,0.850000}%
\pgfsetstrokecolor{currentstroke}%
\pgfsetdash{}{0pt}%
\pgfpathmoveto{\pgfqpoint{0.594525in}{1.868121in}}%
\pgfpathlineto{\pgfqpoint{3.940488in}{1.868121in}}%
\pgfusepath{stroke}%
\end{pgfscope}%
\begin{pgfscope}%
\pgfsetbuttcap%
\pgfsetroundjoin%
\definecolor{currentfill}{rgb}{0.000000,0.000000,0.000000}%
\pgfsetfillcolor{currentfill}%
\pgfsetlinewidth{0.602250pt}%
\definecolor{currentstroke}{rgb}{0.000000,0.000000,0.000000}%
\pgfsetstrokecolor{currentstroke}%
\pgfsetdash{}{0pt}%
\pgfsys@defobject{currentmarker}{\pgfqpoint{-0.027778in}{0.000000in}}{\pgfqpoint{-0.000000in}{0.000000in}}{%
\pgfpathmoveto{\pgfqpoint{-0.000000in}{0.000000in}}%
\pgfpathlineto{\pgfqpoint{-0.027778in}{0.000000in}}%
\pgfusepath{stroke,fill}%
}%
\begin{pgfscope}%
\pgfsys@transformshift{0.594525in}{1.868121in}%
\pgfsys@useobject{currentmarker}{}%
\end{pgfscope}%
\end{pgfscope}%
\begin{pgfscope}%
\pgfpathrectangle{\pgfqpoint{0.594525in}{0.417642in}}{\pgfqpoint{3.345963in}{2.050688in}}%
\pgfusepath{clip}%
\pgfsetrectcap%
\pgfsetroundjoin%
\pgfsetlinewidth{0.803000pt}%
\definecolor{currentstroke}{rgb}{0.850000,0.850000,0.850000}%
\pgfsetstrokecolor{currentstroke}%
\pgfsetdash{}{0pt}%
\pgfpathmoveto{\pgfqpoint{0.594525in}{1.902745in}}%
\pgfpathlineto{\pgfqpoint{3.940488in}{1.902745in}}%
\pgfusepath{stroke}%
\end{pgfscope}%
\begin{pgfscope}%
\pgfsetbuttcap%
\pgfsetroundjoin%
\definecolor{currentfill}{rgb}{0.000000,0.000000,0.000000}%
\pgfsetfillcolor{currentfill}%
\pgfsetlinewidth{0.602250pt}%
\definecolor{currentstroke}{rgb}{0.000000,0.000000,0.000000}%
\pgfsetstrokecolor{currentstroke}%
\pgfsetdash{}{0pt}%
\pgfsys@defobject{currentmarker}{\pgfqpoint{-0.027778in}{0.000000in}}{\pgfqpoint{-0.000000in}{0.000000in}}{%
\pgfpathmoveto{\pgfqpoint{-0.000000in}{0.000000in}}%
\pgfpathlineto{\pgfqpoint{-0.027778in}{0.000000in}}%
\pgfusepath{stroke,fill}%
}%
\begin{pgfscope}%
\pgfsys@transformshift{0.594525in}{1.902745in}%
\pgfsys@useobject{currentmarker}{}%
\end{pgfscope}%
\end{pgfscope}%
\begin{pgfscope}%
\pgfpathrectangle{\pgfqpoint{0.594525in}{0.417642in}}{\pgfqpoint{3.345963in}{2.050688in}}%
\pgfusepath{clip}%
\pgfsetrectcap%
\pgfsetroundjoin%
\pgfsetlinewidth{0.803000pt}%
\definecolor{currentstroke}{rgb}{0.850000,0.850000,0.850000}%
\pgfsetstrokecolor{currentstroke}%
\pgfsetdash{}{0pt}%
\pgfpathmoveto{\pgfqpoint{0.594525in}{1.929600in}}%
\pgfpathlineto{\pgfqpoint{3.940488in}{1.929600in}}%
\pgfusepath{stroke}%
\end{pgfscope}%
\begin{pgfscope}%
\pgfsetbuttcap%
\pgfsetroundjoin%
\definecolor{currentfill}{rgb}{0.000000,0.000000,0.000000}%
\pgfsetfillcolor{currentfill}%
\pgfsetlinewidth{0.602250pt}%
\definecolor{currentstroke}{rgb}{0.000000,0.000000,0.000000}%
\pgfsetstrokecolor{currentstroke}%
\pgfsetdash{}{0pt}%
\pgfsys@defobject{currentmarker}{\pgfqpoint{-0.027778in}{0.000000in}}{\pgfqpoint{-0.000000in}{0.000000in}}{%
\pgfpathmoveto{\pgfqpoint{-0.000000in}{0.000000in}}%
\pgfpathlineto{\pgfqpoint{-0.027778in}{0.000000in}}%
\pgfusepath{stroke,fill}%
}%
\begin{pgfscope}%
\pgfsys@transformshift{0.594525in}{1.929600in}%
\pgfsys@useobject{currentmarker}{}%
\end{pgfscope}%
\end{pgfscope}%
\begin{pgfscope}%
\pgfpathrectangle{\pgfqpoint{0.594525in}{0.417642in}}{\pgfqpoint{3.345963in}{2.050688in}}%
\pgfusepath{clip}%
\pgfsetrectcap%
\pgfsetroundjoin%
\pgfsetlinewidth{0.803000pt}%
\definecolor{currentstroke}{rgb}{0.850000,0.850000,0.850000}%
\pgfsetstrokecolor{currentstroke}%
\pgfsetdash{}{0pt}%
\pgfpathmoveto{\pgfqpoint{0.594525in}{1.951543in}}%
\pgfpathlineto{\pgfqpoint{3.940488in}{1.951543in}}%
\pgfusepath{stroke}%
\end{pgfscope}%
\begin{pgfscope}%
\pgfsetbuttcap%
\pgfsetroundjoin%
\definecolor{currentfill}{rgb}{0.000000,0.000000,0.000000}%
\pgfsetfillcolor{currentfill}%
\pgfsetlinewidth{0.602250pt}%
\definecolor{currentstroke}{rgb}{0.000000,0.000000,0.000000}%
\pgfsetstrokecolor{currentstroke}%
\pgfsetdash{}{0pt}%
\pgfsys@defobject{currentmarker}{\pgfqpoint{-0.027778in}{0.000000in}}{\pgfqpoint{-0.000000in}{0.000000in}}{%
\pgfpathmoveto{\pgfqpoint{-0.000000in}{0.000000in}}%
\pgfpathlineto{\pgfqpoint{-0.027778in}{0.000000in}}%
\pgfusepath{stroke,fill}%
}%
\begin{pgfscope}%
\pgfsys@transformshift{0.594525in}{1.951543in}%
\pgfsys@useobject{currentmarker}{}%
\end{pgfscope}%
\end{pgfscope}%
\begin{pgfscope}%
\pgfpathrectangle{\pgfqpoint{0.594525in}{0.417642in}}{\pgfqpoint{3.345963in}{2.050688in}}%
\pgfusepath{clip}%
\pgfsetrectcap%
\pgfsetroundjoin%
\pgfsetlinewidth{0.803000pt}%
\definecolor{currentstroke}{rgb}{0.850000,0.850000,0.850000}%
\pgfsetstrokecolor{currentstroke}%
\pgfsetdash{}{0pt}%
\pgfpathmoveto{\pgfqpoint{0.594525in}{1.970095in}}%
\pgfpathlineto{\pgfqpoint{3.940488in}{1.970095in}}%
\pgfusepath{stroke}%
\end{pgfscope}%
\begin{pgfscope}%
\pgfsetbuttcap%
\pgfsetroundjoin%
\definecolor{currentfill}{rgb}{0.000000,0.000000,0.000000}%
\pgfsetfillcolor{currentfill}%
\pgfsetlinewidth{0.602250pt}%
\definecolor{currentstroke}{rgb}{0.000000,0.000000,0.000000}%
\pgfsetstrokecolor{currentstroke}%
\pgfsetdash{}{0pt}%
\pgfsys@defobject{currentmarker}{\pgfqpoint{-0.027778in}{0.000000in}}{\pgfqpoint{-0.000000in}{0.000000in}}{%
\pgfpathmoveto{\pgfqpoint{-0.000000in}{0.000000in}}%
\pgfpathlineto{\pgfqpoint{-0.027778in}{0.000000in}}%
\pgfusepath{stroke,fill}%
}%
\begin{pgfscope}%
\pgfsys@transformshift{0.594525in}{1.970095in}%
\pgfsys@useobject{currentmarker}{}%
\end{pgfscope}%
\end{pgfscope}%
\begin{pgfscope}%
\pgfpathrectangle{\pgfqpoint{0.594525in}{0.417642in}}{\pgfqpoint{3.345963in}{2.050688in}}%
\pgfusepath{clip}%
\pgfsetrectcap%
\pgfsetroundjoin%
\pgfsetlinewidth{0.803000pt}%
\definecolor{currentstroke}{rgb}{0.850000,0.850000,0.850000}%
\pgfsetstrokecolor{currentstroke}%
\pgfsetdash{}{0pt}%
\pgfpathmoveto{\pgfqpoint{0.594525in}{1.986166in}}%
\pgfpathlineto{\pgfqpoint{3.940488in}{1.986166in}}%
\pgfusepath{stroke}%
\end{pgfscope}%
\begin{pgfscope}%
\pgfsetbuttcap%
\pgfsetroundjoin%
\definecolor{currentfill}{rgb}{0.000000,0.000000,0.000000}%
\pgfsetfillcolor{currentfill}%
\pgfsetlinewidth{0.602250pt}%
\definecolor{currentstroke}{rgb}{0.000000,0.000000,0.000000}%
\pgfsetstrokecolor{currentstroke}%
\pgfsetdash{}{0pt}%
\pgfsys@defobject{currentmarker}{\pgfqpoint{-0.027778in}{0.000000in}}{\pgfqpoint{-0.000000in}{0.000000in}}{%
\pgfpathmoveto{\pgfqpoint{-0.000000in}{0.000000in}}%
\pgfpathlineto{\pgfqpoint{-0.027778in}{0.000000in}}%
\pgfusepath{stroke,fill}%
}%
\begin{pgfscope}%
\pgfsys@transformshift{0.594525in}{1.986166in}%
\pgfsys@useobject{currentmarker}{}%
\end{pgfscope}%
\end{pgfscope}%
\begin{pgfscope}%
\pgfpathrectangle{\pgfqpoint{0.594525in}{0.417642in}}{\pgfqpoint{3.345963in}{2.050688in}}%
\pgfusepath{clip}%
\pgfsetrectcap%
\pgfsetroundjoin%
\pgfsetlinewidth{0.803000pt}%
\definecolor{currentstroke}{rgb}{0.850000,0.850000,0.850000}%
\pgfsetstrokecolor{currentstroke}%
\pgfsetdash{}{0pt}%
\pgfpathmoveto{\pgfqpoint{0.594525in}{2.000341in}}%
\pgfpathlineto{\pgfqpoint{3.940488in}{2.000341in}}%
\pgfusepath{stroke}%
\end{pgfscope}%
\begin{pgfscope}%
\pgfsetbuttcap%
\pgfsetroundjoin%
\definecolor{currentfill}{rgb}{0.000000,0.000000,0.000000}%
\pgfsetfillcolor{currentfill}%
\pgfsetlinewidth{0.602250pt}%
\definecolor{currentstroke}{rgb}{0.000000,0.000000,0.000000}%
\pgfsetstrokecolor{currentstroke}%
\pgfsetdash{}{0pt}%
\pgfsys@defobject{currentmarker}{\pgfqpoint{-0.027778in}{0.000000in}}{\pgfqpoint{-0.000000in}{0.000000in}}{%
\pgfpathmoveto{\pgfqpoint{-0.000000in}{0.000000in}}%
\pgfpathlineto{\pgfqpoint{-0.027778in}{0.000000in}}%
\pgfusepath{stroke,fill}%
}%
\begin{pgfscope}%
\pgfsys@transformshift{0.594525in}{2.000341in}%
\pgfsys@useobject{currentmarker}{}%
\end{pgfscope}%
\end{pgfscope}%
\begin{pgfscope}%
\pgfpathrectangle{\pgfqpoint{0.594525in}{0.417642in}}{\pgfqpoint{3.345963in}{2.050688in}}%
\pgfusepath{clip}%
\pgfsetrectcap%
\pgfsetroundjoin%
\pgfsetlinewidth{0.803000pt}%
\definecolor{currentstroke}{rgb}{0.850000,0.850000,0.850000}%
\pgfsetstrokecolor{currentstroke}%
\pgfsetdash{}{0pt}%
\pgfpathmoveto{\pgfqpoint{0.594525in}{2.096443in}}%
\pgfpathlineto{\pgfqpoint{3.940488in}{2.096443in}}%
\pgfusepath{stroke}%
\end{pgfscope}%
\begin{pgfscope}%
\pgfsetbuttcap%
\pgfsetroundjoin%
\definecolor{currentfill}{rgb}{0.000000,0.000000,0.000000}%
\pgfsetfillcolor{currentfill}%
\pgfsetlinewidth{0.602250pt}%
\definecolor{currentstroke}{rgb}{0.000000,0.000000,0.000000}%
\pgfsetstrokecolor{currentstroke}%
\pgfsetdash{}{0pt}%
\pgfsys@defobject{currentmarker}{\pgfqpoint{-0.027778in}{0.000000in}}{\pgfqpoint{-0.000000in}{0.000000in}}{%
\pgfpathmoveto{\pgfqpoint{-0.000000in}{0.000000in}}%
\pgfpathlineto{\pgfqpoint{-0.027778in}{0.000000in}}%
\pgfusepath{stroke,fill}%
}%
\begin{pgfscope}%
\pgfsys@transformshift{0.594525in}{2.096443in}%
\pgfsys@useobject{currentmarker}{}%
\end{pgfscope}%
\end{pgfscope}%
\begin{pgfscope}%
\pgfpathrectangle{\pgfqpoint{0.594525in}{0.417642in}}{\pgfqpoint{3.345963in}{2.050688in}}%
\pgfusepath{clip}%
\pgfsetrectcap%
\pgfsetroundjoin%
\pgfsetlinewidth{0.803000pt}%
\definecolor{currentstroke}{rgb}{0.850000,0.850000,0.850000}%
\pgfsetstrokecolor{currentstroke}%
\pgfsetdash{}{0pt}%
\pgfpathmoveto{\pgfqpoint{0.594525in}{2.145242in}}%
\pgfpathlineto{\pgfqpoint{3.940488in}{2.145242in}}%
\pgfusepath{stroke}%
\end{pgfscope}%
\begin{pgfscope}%
\pgfsetbuttcap%
\pgfsetroundjoin%
\definecolor{currentfill}{rgb}{0.000000,0.000000,0.000000}%
\pgfsetfillcolor{currentfill}%
\pgfsetlinewidth{0.602250pt}%
\definecolor{currentstroke}{rgb}{0.000000,0.000000,0.000000}%
\pgfsetstrokecolor{currentstroke}%
\pgfsetdash{}{0pt}%
\pgfsys@defobject{currentmarker}{\pgfqpoint{-0.027778in}{0.000000in}}{\pgfqpoint{-0.000000in}{0.000000in}}{%
\pgfpathmoveto{\pgfqpoint{-0.000000in}{0.000000in}}%
\pgfpathlineto{\pgfqpoint{-0.027778in}{0.000000in}}%
\pgfusepath{stroke,fill}%
}%
\begin{pgfscope}%
\pgfsys@transformshift{0.594525in}{2.145242in}%
\pgfsys@useobject{currentmarker}{}%
\end{pgfscope}%
\end{pgfscope}%
\begin{pgfscope}%
\pgfpathrectangle{\pgfqpoint{0.594525in}{0.417642in}}{\pgfqpoint{3.345963in}{2.050688in}}%
\pgfusepath{clip}%
\pgfsetrectcap%
\pgfsetroundjoin%
\pgfsetlinewidth{0.803000pt}%
\definecolor{currentstroke}{rgb}{0.850000,0.850000,0.850000}%
\pgfsetstrokecolor{currentstroke}%
\pgfsetdash{}{0pt}%
\pgfpathmoveto{\pgfqpoint{0.594525in}{2.179865in}}%
\pgfpathlineto{\pgfqpoint{3.940488in}{2.179865in}}%
\pgfusepath{stroke}%
\end{pgfscope}%
\begin{pgfscope}%
\pgfsetbuttcap%
\pgfsetroundjoin%
\definecolor{currentfill}{rgb}{0.000000,0.000000,0.000000}%
\pgfsetfillcolor{currentfill}%
\pgfsetlinewidth{0.602250pt}%
\definecolor{currentstroke}{rgb}{0.000000,0.000000,0.000000}%
\pgfsetstrokecolor{currentstroke}%
\pgfsetdash{}{0pt}%
\pgfsys@defobject{currentmarker}{\pgfqpoint{-0.027778in}{0.000000in}}{\pgfqpoint{-0.000000in}{0.000000in}}{%
\pgfpathmoveto{\pgfqpoint{-0.000000in}{0.000000in}}%
\pgfpathlineto{\pgfqpoint{-0.027778in}{0.000000in}}%
\pgfusepath{stroke,fill}%
}%
\begin{pgfscope}%
\pgfsys@transformshift{0.594525in}{2.179865in}%
\pgfsys@useobject{currentmarker}{}%
\end{pgfscope}%
\end{pgfscope}%
\begin{pgfscope}%
\pgfpathrectangle{\pgfqpoint{0.594525in}{0.417642in}}{\pgfqpoint{3.345963in}{2.050688in}}%
\pgfusepath{clip}%
\pgfsetrectcap%
\pgfsetroundjoin%
\pgfsetlinewidth{0.803000pt}%
\definecolor{currentstroke}{rgb}{0.850000,0.850000,0.850000}%
\pgfsetstrokecolor{currentstroke}%
\pgfsetdash{}{0pt}%
\pgfpathmoveto{\pgfqpoint{0.594525in}{2.206721in}}%
\pgfpathlineto{\pgfqpoint{3.940488in}{2.206721in}}%
\pgfusepath{stroke}%
\end{pgfscope}%
\begin{pgfscope}%
\pgfsetbuttcap%
\pgfsetroundjoin%
\definecolor{currentfill}{rgb}{0.000000,0.000000,0.000000}%
\pgfsetfillcolor{currentfill}%
\pgfsetlinewidth{0.602250pt}%
\definecolor{currentstroke}{rgb}{0.000000,0.000000,0.000000}%
\pgfsetstrokecolor{currentstroke}%
\pgfsetdash{}{0pt}%
\pgfsys@defobject{currentmarker}{\pgfqpoint{-0.027778in}{0.000000in}}{\pgfqpoint{-0.000000in}{0.000000in}}{%
\pgfpathmoveto{\pgfqpoint{-0.000000in}{0.000000in}}%
\pgfpathlineto{\pgfqpoint{-0.027778in}{0.000000in}}%
\pgfusepath{stroke,fill}%
}%
\begin{pgfscope}%
\pgfsys@transformshift{0.594525in}{2.206721in}%
\pgfsys@useobject{currentmarker}{}%
\end{pgfscope}%
\end{pgfscope}%
\begin{pgfscope}%
\pgfpathrectangle{\pgfqpoint{0.594525in}{0.417642in}}{\pgfqpoint{3.345963in}{2.050688in}}%
\pgfusepath{clip}%
\pgfsetrectcap%
\pgfsetroundjoin%
\pgfsetlinewidth{0.803000pt}%
\definecolor{currentstroke}{rgb}{0.850000,0.850000,0.850000}%
\pgfsetstrokecolor{currentstroke}%
\pgfsetdash{}{0pt}%
\pgfpathmoveto{\pgfqpoint{0.594525in}{2.228663in}}%
\pgfpathlineto{\pgfqpoint{3.940488in}{2.228663in}}%
\pgfusepath{stroke}%
\end{pgfscope}%
\begin{pgfscope}%
\pgfsetbuttcap%
\pgfsetroundjoin%
\definecolor{currentfill}{rgb}{0.000000,0.000000,0.000000}%
\pgfsetfillcolor{currentfill}%
\pgfsetlinewidth{0.602250pt}%
\definecolor{currentstroke}{rgb}{0.000000,0.000000,0.000000}%
\pgfsetstrokecolor{currentstroke}%
\pgfsetdash{}{0pt}%
\pgfsys@defobject{currentmarker}{\pgfqpoint{-0.027778in}{0.000000in}}{\pgfqpoint{-0.000000in}{0.000000in}}{%
\pgfpathmoveto{\pgfqpoint{-0.000000in}{0.000000in}}%
\pgfpathlineto{\pgfqpoint{-0.027778in}{0.000000in}}%
\pgfusepath{stroke,fill}%
}%
\begin{pgfscope}%
\pgfsys@transformshift{0.594525in}{2.228663in}%
\pgfsys@useobject{currentmarker}{}%
\end{pgfscope}%
\end{pgfscope}%
\begin{pgfscope}%
\pgfpathrectangle{\pgfqpoint{0.594525in}{0.417642in}}{\pgfqpoint{3.345963in}{2.050688in}}%
\pgfusepath{clip}%
\pgfsetrectcap%
\pgfsetroundjoin%
\pgfsetlinewidth{0.803000pt}%
\definecolor{currentstroke}{rgb}{0.850000,0.850000,0.850000}%
\pgfsetstrokecolor{currentstroke}%
\pgfsetdash{}{0pt}%
\pgfpathmoveto{\pgfqpoint{0.594525in}{2.247216in}}%
\pgfpathlineto{\pgfqpoint{3.940488in}{2.247216in}}%
\pgfusepath{stroke}%
\end{pgfscope}%
\begin{pgfscope}%
\pgfsetbuttcap%
\pgfsetroundjoin%
\definecolor{currentfill}{rgb}{0.000000,0.000000,0.000000}%
\pgfsetfillcolor{currentfill}%
\pgfsetlinewidth{0.602250pt}%
\definecolor{currentstroke}{rgb}{0.000000,0.000000,0.000000}%
\pgfsetstrokecolor{currentstroke}%
\pgfsetdash{}{0pt}%
\pgfsys@defobject{currentmarker}{\pgfqpoint{-0.027778in}{0.000000in}}{\pgfqpoint{-0.000000in}{0.000000in}}{%
\pgfpathmoveto{\pgfqpoint{-0.000000in}{0.000000in}}%
\pgfpathlineto{\pgfqpoint{-0.027778in}{0.000000in}}%
\pgfusepath{stroke,fill}%
}%
\begin{pgfscope}%
\pgfsys@transformshift{0.594525in}{2.247216in}%
\pgfsys@useobject{currentmarker}{}%
\end{pgfscope}%
\end{pgfscope}%
\begin{pgfscope}%
\pgfpathrectangle{\pgfqpoint{0.594525in}{0.417642in}}{\pgfqpoint{3.345963in}{2.050688in}}%
\pgfusepath{clip}%
\pgfsetrectcap%
\pgfsetroundjoin%
\pgfsetlinewidth{0.803000pt}%
\definecolor{currentstroke}{rgb}{0.850000,0.850000,0.850000}%
\pgfsetstrokecolor{currentstroke}%
\pgfsetdash{}{0pt}%
\pgfpathmoveto{\pgfqpoint{0.594525in}{2.263286in}}%
\pgfpathlineto{\pgfqpoint{3.940488in}{2.263286in}}%
\pgfusepath{stroke}%
\end{pgfscope}%
\begin{pgfscope}%
\pgfsetbuttcap%
\pgfsetroundjoin%
\definecolor{currentfill}{rgb}{0.000000,0.000000,0.000000}%
\pgfsetfillcolor{currentfill}%
\pgfsetlinewidth{0.602250pt}%
\definecolor{currentstroke}{rgb}{0.000000,0.000000,0.000000}%
\pgfsetstrokecolor{currentstroke}%
\pgfsetdash{}{0pt}%
\pgfsys@defobject{currentmarker}{\pgfqpoint{-0.027778in}{0.000000in}}{\pgfqpoint{-0.000000in}{0.000000in}}{%
\pgfpathmoveto{\pgfqpoint{-0.000000in}{0.000000in}}%
\pgfpathlineto{\pgfqpoint{-0.027778in}{0.000000in}}%
\pgfusepath{stroke,fill}%
}%
\begin{pgfscope}%
\pgfsys@transformshift{0.594525in}{2.263286in}%
\pgfsys@useobject{currentmarker}{}%
\end{pgfscope}%
\end{pgfscope}%
\begin{pgfscope}%
\pgfpathrectangle{\pgfqpoint{0.594525in}{0.417642in}}{\pgfqpoint{3.345963in}{2.050688in}}%
\pgfusepath{clip}%
\pgfsetrectcap%
\pgfsetroundjoin%
\pgfsetlinewidth{0.803000pt}%
\definecolor{currentstroke}{rgb}{0.850000,0.850000,0.850000}%
\pgfsetstrokecolor{currentstroke}%
\pgfsetdash{}{0pt}%
\pgfpathmoveto{\pgfqpoint{0.594525in}{2.277462in}}%
\pgfpathlineto{\pgfqpoint{3.940488in}{2.277462in}}%
\pgfusepath{stroke}%
\end{pgfscope}%
\begin{pgfscope}%
\pgfsetbuttcap%
\pgfsetroundjoin%
\definecolor{currentfill}{rgb}{0.000000,0.000000,0.000000}%
\pgfsetfillcolor{currentfill}%
\pgfsetlinewidth{0.602250pt}%
\definecolor{currentstroke}{rgb}{0.000000,0.000000,0.000000}%
\pgfsetstrokecolor{currentstroke}%
\pgfsetdash{}{0pt}%
\pgfsys@defobject{currentmarker}{\pgfqpoint{-0.027778in}{0.000000in}}{\pgfqpoint{-0.000000in}{0.000000in}}{%
\pgfpathmoveto{\pgfqpoint{-0.000000in}{0.000000in}}%
\pgfpathlineto{\pgfqpoint{-0.027778in}{0.000000in}}%
\pgfusepath{stroke,fill}%
}%
\begin{pgfscope}%
\pgfsys@transformshift{0.594525in}{2.277462in}%
\pgfsys@useobject{currentmarker}{}%
\end{pgfscope}%
\end{pgfscope}%
\begin{pgfscope}%
\pgfpathrectangle{\pgfqpoint{0.594525in}{0.417642in}}{\pgfqpoint{3.345963in}{2.050688in}}%
\pgfusepath{clip}%
\pgfsetrectcap%
\pgfsetroundjoin%
\pgfsetlinewidth{0.803000pt}%
\definecolor{currentstroke}{rgb}{0.850000,0.850000,0.850000}%
\pgfsetstrokecolor{currentstroke}%
\pgfsetdash{}{0pt}%
\pgfpathmoveto{\pgfqpoint{0.594525in}{2.373564in}}%
\pgfpathlineto{\pgfqpoint{3.940488in}{2.373564in}}%
\pgfusepath{stroke}%
\end{pgfscope}%
\begin{pgfscope}%
\pgfsetbuttcap%
\pgfsetroundjoin%
\definecolor{currentfill}{rgb}{0.000000,0.000000,0.000000}%
\pgfsetfillcolor{currentfill}%
\pgfsetlinewidth{0.602250pt}%
\definecolor{currentstroke}{rgb}{0.000000,0.000000,0.000000}%
\pgfsetstrokecolor{currentstroke}%
\pgfsetdash{}{0pt}%
\pgfsys@defobject{currentmarker}{\pgfqpoint{-0.027778in}{0.000000in}}{\pgfqpoint{-0.000000in}{0.000000in}}{%
\pgfpathmoveto{\pgfqpoint{-0.000000in}{0.000000in}}%
\pgfpathlineto{\pgfqpoint{-0.027778in}{0.000000in}}%
\pgfusepath{stroke,fill}%
}%
\begin{pgfscope}%
\pgfsys@transformshift{0.594525in}{2.373564in}%
\pgfsys@useobject{currentmarker}{}%
\end{pgfscope}%
\end{pgfscope}%
\begin{pgfscope}%
\pgfpathrectangle{\pgfqpoint{0.594525in}{0.417642in}}{\pgfqpoint{3.345963in}{2.050688in}}%
\pgfusepath{clip}%
\pgfsetrectcap%
\pgfsetroundjoin%
\pgfsetlinewidth{0.803000pt}%
\definecolor{currentstroke}{rgb}{0.850000,0.850000,0.850000}%
\pgfsetstrokecolor{currentstroke}%
\pgfsetdash{}{0pt}%
\pgfpathmoveto{\pgfqpoint{0.594525in}{2.422362in}}%
\pgfpathlineto{\pgfqpoint{3.940488in}{2.422362in}}%
\pgfusepath{stroke}%
\end{pgfscope}%
\begin{pgfscope}%
\pgfsetbuttcap%
\pgfsetroundjoin%
\definecolor{currentfill}{rgb}{0.000000,0.000000,0.000000}%
\pgfsetfillcolor{currentfill}%
\pgfsetlinewidth{0.602250pt}%
\definecolor{currentstroke}{rgb}{0.000000,0.000000,0.000000}%
\pgfsetstrokecolor{currentstroke}%
\pgfsetdash{}{0pt}%
\pgfsys@defobject{currentmarker}{\pgfqpoint{-0.027778in}{0.000000in}}{\pgfqpoint{-0.000000in}{0.000000in}}{%
\pgfpathmoveto{\pgfqpoint{-0.000000in}{0.000000in}}%
\pgfpathlineto{\pgfqpoint{-0.027778in}{0.000000in}}%
\pgfusepath{stroke,fill}%
}%
\begin{pgfscope}%
\pgfsys@transformshift{0.594525in}{2.422362in}%
\pgfsys@useobject{currentmarker}{}%
\end{pgfscope}%
\end{pgfscope}%
\begin{pgfscope}%
\pgfpathrectangle{\pgfqpoint{0.594525in}{0.417642in}}{\pgfqpoint{3.345963in}{2.050688in}}%
\pgfusepath{clip}%
\pgfsetrectcap%
\pgfsetroundjoin%
\pgfsetlinewidth{0.803000pt}%
\definecolor{currentstroke}{rgb}{0.850000,0.850000,0.850000}%
\pgfsetstrokecolor{currentstroke}%
\pgfsetdash{}{0pt}%
\pgfpathmoveto{\pgfqpoint{0.594525in}{2.456985in}}%
\pgfpathlineto{\pgfqpoint{3.940488in}{2.456985in}}%
\pgfusepath{stroke}%
\end{pgfscope}%
\begin{pgfscope}%
\pgfsetbuttcap%
\pgfsetroundjoin%
\definecolor{currentfill}{rgb}{0.000000,0.000000,0.000000}%
\pgfsetfillcolor{currentfill}%
\pgfsetlinewidth{0.602250pt}%
\definecolor{currentstroke}{rgb}{0.000000,0.000000,0.000000}%
\pgfsetstrokecolor{currentstroke}%
\pgfsetdash{}{0pt}%
\pgfsys@defobject{currentmarker}{\pgfqpoint{-0.027778in}{0.000000in}}{\pgfqpoint{-0.000000in}{0.000000in}}{%
\pgfpathmoveto{\pgfqpoint{-0.000000in}{0.000000in}}%
\pgfpathlineto{\pgfqpoint{-0.027778in}{0.000000in}}%
\pgfusepath{stroke,fill}%
}%
\begin{pgfscope}%
\pgfsys@transformshift{0.594525in}{2.456985in}%
\pgfsys@useobject{currentmarker}{}%
\end{pgfscope}%
\end{pgfscope}%
\begin{pgfscope}%
\definecolor{textcolor}{rgb}{0.000000,0.000000,0.000000}%
\pgfsetstrokecolor{textcolor}%
\pgfsetfillcolor{textcolor}%
\pgftext[x=0.185574in,y=1.442986in,,bottom,rotate=90.000000]{\color{textcolor}\rmfamily\fontsize{10.000000}{12.000000}\selectfont  \(\displaystyle S_y(f)\) in \(\displaystyle \unit{1 \per \Hz}\)}%
\end{pgfscope}%
\begin{pgfscope}%
\pgfpathrectangle{\pgfqpoint{0.594525in}{0.417642in}}{\pgfqpoint{3.345963in}{2.050688in}}%
\pgfusepath{clip}%
\pgfsetbuttcap%
\pgfsetroundjoin%
\definecolor{currentfill}{rgb}{0.337255,0.705882,0.913725}%
\pgfsetfillcolor{currentfill}%
\pgfsetlinewidth{1.003750pt}%
\definecolor{currentstroke}{rgb}{0.337255,0.705882,0.913725}%
\pgfsetstrokecolor{currentstroke}%
\pgfsetdash{}{0pt}%
\pgfsys@defobject{currentmarker}{\pgfqpoint{-0.013889in}{-0.013889in}}{\pgfqpoint{0.013889in}{0.013889in}}{%
\pgfpathmoveto{\pgfqpoint{0.000000in}{-0.013889in}}%
\pgfpathcurveto{\pgfqpoint{0.003683in}{-0.013889in}}{\pgfqpoint{0.007216in}{-0.012425in}}{\pgfqpoint{0.009821in}{-0.009821in}}%
\pgfpathcurveto{\pgfqpoint{0.012425in}{-0.007216in}}{\pgfqpoint{0.013889in}{-0.003683in}}{\pgfqpoint{0.013889in}{0.000000in}}%
\pgfpathcurveto{\pgfqpoint{0.013889in}{0.003683in}}{\pgfqpoint{0.012425in}{0.007216in}}{\pgfqpoint{0.009821in}{0.009821in}}%
\pgfpathcurveto{\pgfqpoint{0.007216in}{0.012425in}}{\pgfqpoint{0.003683in}{0.013889in}}{\pgfqpoint{0.000000in}{0.013889in}}%
\pgfpathcurveto{\pgfqpoint{-0.003683in}{0.013889in}}{\pgfqpoint{-0.007216in}{0.012425in}}{\pgfqpoint{-0.009821in}{0.009821in}}%
\pgfpathcurveto{\pgfqpoint{-0.012425in}{0.007216in}}{\pgfqpoint{-0.013889in}{0.003683in}}{\pgfqpoint{-0.013889in}{0.000000in}}%
\pgfpathcurveto{\pgfqpoint{-0.013889in}{-0.003683in}}{\pgfqpoint{-0.012425in}{-0.007216in}}{\pgfqpoint{-0.009821in}{-0.009821in}}%
\pgfpathcurveto{\pgfqpoint{-0.007216in}{-0.012425in}}{\pgfqpoint{-0.003683in}{-0.013889in}}{\pgfqpoint{0.000000in}{-0.013889in}}%
\pgfpathlineto{\pgfqpoint{0.000000in}{-0.013889in}}%
\pgfpathclose%
\pgfusepath{stroke,fill}%
}%
\begin{pgfscope}%
\pgfsys@transformshift{0.746614in}{2.375117in}%
\pgfsys@useobject{currentmarker}{}%
\end{pgfscope}%
\begin{pgfscope}%
\pgfsys@transformshift{0.885038in}{2.239725in}%
\pgfsys@useobject{currentmarker}{}%
\end{pgfscope}%
\begin{pgfscope}%
\pgfsys@transformshift{0.966011in}{2.053354in}%
\pgfsys@useobject{currentmarker}{}%
\end{pgfscope}%
\begin{pgfscope}%
\pgfsys@transformshift{1.023462in}{1.895763in}%
\pgfsys@useobject{currentmarker}{}%
\end{pgfscope}%
\begin{pgfscope}%
\pgfsys@transformshift{1.068025in}{1.853367in}%
\pgfsys@useobject{currentmarker}{}%
\end{pgfscope}%
\begin{pgfscope}%
\pgfsys@transformshift{1.104435in}{1.861200in}%
\pgfsys@useobject{currentmarker}{}%
\end{pgfscope}%
\begin{pgfscope}%
\pgfsys@transformshift{1.135219in}{1.824069in}%
\pgfsys@useobject{currentmarker}{}%
\end{pgfscope}%
\begin{pgfscope}%
\pgfsys@transformshift{1.161886in}{1.823847in}%
\pgfsys@useobject{currentmarker}{}%
\end{pgfscope}%
\begin{pgfscope}%
\pgfsys@transformshift{1.185408in}{1.763107in}%
\pgfsys@useobject{currentmarker}{}%
\end{pgfscope}%
\begin{pgfscope}%
\pgfsys@transformshift{1.206449in}{1.744003in}%
\pgfsys@useobject{currentmarker}{}%
\end{pgfscope}%
\begin{pgfscope}%
\pgfsys@transformshift{1.225482in}{1.787582in}%
\pgfsys@useobject{currentmarker}{}%
\end{pgfscope}%
\begin{pgfscope}%
\pgfsys@transformshift{1.242859in}{1.781036in}%
\pgfsys@useobject{currentmarker}{}%
\end{pgfscope}%
\begin{pgfscope}%
\pgfsys@transformshift{1.258844in}{1.706404in}%
\pgfsys@useobject{currentmarker}{}%
\end{pgfscope}%
\begin{pgfscope}%
\pgfsys@transformshift{1.273643in}{1.604472in}%
\pgfsys@useobject{currentmarker}{}%
\end{pgfscope}%
\begin{pgfscope}%
\pgfsys@transformshift{1.287422in}{1.660339in}%
\pgfsys@useobject{currentmarker}{}%
\end{pgfscope}%
\begin{pgfscope}%
\pgfsys@transformshift{1.300310in}{1.702685in}%
\pgfsys@useobject{currentmarker}{}%
\end{pgfscope}%
\begin{pgfscope}%
\pgfsys@transformshift{1.318206in}{1.681106in}%
\pgfsys@useobject{currentmarker}{}%
\end{pgfscope}%
\begin{pgfscope}%
\pgfsys@transformshift{1.334629in}{1.668544in}%
\pgfsys@useobject{currentmarker}{}%
\end{pgfscope}%
\begin{pgfscope}%
\pgfsys@transformshift{1.349804in}{1.599029in}%
\pgfsys@useobject{currentmarker}{}%
\end{pgfscope}%
\begin{pgfscope}%
\pgfsys@transformshift{1.368394in}{1.540970in}%
\pgfsys@useobject{currentmarker}{}%
\end{pgfscope}%
\begin{pgfscope}%
\pgfsys@transformshift{1.385401in}{1.486048in}%
\pgfsys@useobject{currentmarker}{}%
\end{pgfscope}%
\begin{pgfscope}%
\pgfsys@transformshift{1.401072in}{1.547171in}%
\pgfsys@useobject{currentmarker}{}%
\end{pgfscope}%
\begin{pgfscope}%
\pgfsys@transformshift{1.415602in}{1.603633in}%
\pgfsys@useobject{currentmarker}{}%
\end{pgfscope}%
\begin{pgfscope}%
\pgfsys@transformshift{1.432394in}{1.447854in}%
\pgfsys@useobject{currentmarker}{}%
\end{pgfscope}%
\begin{pgfscope}%
\pgfsys@transformshift{1.447882in}{1.468855in}%
\pgfsys@useobject{currentmarker}{}%
\end{pgfscope}%
\begin{pgfscope}%
\pgfsys@transformshift{1.462256in}{1.529581in}%
\pgfsys@useobject{currentmarker}{}%
\end{pgfscope}%
\begin{pgfscope}%
\pgfsys@transformshift{1.478241in}{1.500083in}%
\pgfsys@useobject{currentmarker}{}%
\end{pgfscope}%
\begin{pgfscope}%
\pgfsys@transformshift{1.495404in}{1.561244in}%
\pgfsys@useobject{currentmarker}{}%
\end{pgfscope}%
\begin{pgfscope}%
\pgfsys@transformshift{1.513367in}{1.499754in}%
\pgfsys@useobject{currentmarker}{}%
\end{pgfscope}%
\begin{pgfscope}%
\pgfsys@transformshift{1.529846in}{1.430393in}%
\pgfsys@useobject{currentmarker}{}%
\end{pgfscope}%
\begin{pgfscope}%
\pgfsys@transformshift{1.545069in}{1.393581in}%
\pgfsys@useobject{currentmarker}{}%
\end{pgfscope}%
\begin{pgfscope}%
\pgfsys@transformshift{1.560913in}{1.414033in}%
\pgfsys@useobject{currentmarker}{}%
\end{pgfscope}%
\begin{pgfscope}%
\pgfsys@transformshift{1.577158in}{1.443734in}%
\pgfsys@useobject{currentmarker}{}%
\end{pgfscope}%
\begin{pgfscope}%
\pgfsys@transformshift{1.593622in}{1.386438in}%
\pgfsys@useobject{currentmarker}{}%
\end{pgfscope}%
\begin{pgfscope}%
\pgfsys@transformshift{1.610159in}{1.420531in}%
\pgfsys@useobject{currentmarker}{}%
\end{pgfscope}%
\begin{pgfscope}%
\pgfsys@transformshift{1.625431in}{1.372785in}%
\pgfsys@useobject{currentmarker}{}%
\end{pgfscope}%
\begin{pgfscope}%
\pgfsys@transformshift{1.641886in}{1.333705in}%
\pgfsys@useobject{currentmarker}{}%
\end{pgfscope}%
\begin{pgfscope}%
\pgfsys@transformshift{1.658131in}{1.375466in}%
\pgfsys@useobject{currentmarker}{}%
\end{pgfscope}%
\begin{pgfscope}%
\pgfsys@transformshift{1.674116in}{1.338869in}%
\pgfsys@useobject{currentmarker}{}%
\end{pgfscope}%
\begin{pgfscope}%
\pgfsys@transformshift{1.690691in}{1.367063in}%
\pgfsys@useobject{currentmarker}{}%
\end{pgfscope}%
\begin{pgfscope}%
\pgfsys@transformshift{1.706811in}{1.323170in}%
\pgfsys@useobject{currentmarker}{}%
\end{pgfscope}%
\begin{pgfscope}%
\pgfsys@transformshift{1.722482in}{1.325661in}%
\pgfsys@useobject{currentmarker}{}%
\end{pgfscope}%
\begin{pgfscope}%
\pgfsys@transformshift{1.738409in}{1.280024in}%
\pgfsys@useobject{currentmarker}{}%
\end{pgfscope}%
\begin{pgfscope}%
\pgfsys@transformshift{1.755089in}{1.301084in}%
\pgfsys@useobject{currentmarker}{}%
\end{pgfscope}%
\begin{pgfscope}%
\pgfsys@transformshift{1.771074in}{1.275488in}%
\pgfsys@useobject{currentmarker}{}%
\end{pgfscope}%
\begin{pgfscope}%
\pgfsys@transformshift{1.786967in}{1.284826in}%
\pgfsys@useobject{currentmarker}{}%
\end{pgfscope}%
\begin{pgfscope}%
\pgfsys@transformshift{1.803204in}{1.241295in}%
\pgfsys@useobject{currentmarker}{}%
\end{pgfscope}%
\begin{pgfscope}%
\pgfsys@transformshift{1.819614in}{1.237664in}%
\pgfsys@useobject{currentmarker}{}%
\end{pgfscope}%
\begin{pgfscope}%
\pgfsys@transformshift{1.836062in}{1.234113in}%
\pgfsys@useobject{currentmarker}{}%
\end{pgfscope}%
\begin{pgfscope}%
\pgfsys@transformshift{1.852046in}{1.220522in}%
\pgfsys@useobject{currentmarker}{}%
\end{pgfscope}%
\begin{pgfscope}%
\pgfsys@transformshift{1.867940in}{1.213939in}%
\pgfsys@useobject{currentmarker}{}%
\end{pgfscope}%
\begin{pgfscope}%
\pgfsys@transformshift{1.884344in}{1.212882in}%
\pgfsys@useobject{currentmarker}{}%
\end{pgfscope}%
\begin{pgfscope}%
\pgfsys@transformshift{1.900741in}{1.204359in}%
\pgfsys@useobject{currentmarker}{}%
\end{pgfscope}%
\begin{pgfscope}%
\pgfsys@transformshift{1.916750in}{1.171839in}%
\pgfsys@useobject{currentmarker}{}%
\end{pgfscope}%
\begin{pgfscope}%
\pgfsys@transformshift{1.932888in}{1.167180in}%
\pgfsys@useobject{currentmarker}{}%
\end{pgfscope}%
\begin{pgfscope}%
\pgfsys@transformshift{1.949034in}{1.168335in}%
\pgfsys@useobject{currentmarker}{}%
\end{pgfscope}%
\begin{pgfscope}%
\pgfsys@transformshift{1.965094in}{1.174513in}%
\pgfsys@useobject{currentmarker}{}%
\end{pgfscope}%
\begin{pgfscope}%
\pgfsys@transformshift{1.981405in}{1.134048in}%
\pgfsys@useobject{currentmarker}{}%
\end{pgfscope}%
\begin{pgfscope}%
\pgfsys@transformshift{1.997628in}{1.115160in}%
\pgfsys@useobject{currentmarker}{}%
\end{pgfscope}%
\begin{pgfscope}%
\pgfsys@transformshift{2.013686in}{1.122954in}%
\pgfsys@useobject{currentmarker}{}%
\end{pgfscope}%
\begin{pgfscope}%
\pgfsys@transformshift{2.029846in}{1.090919in}%
\pgfsys@useobject{currentmarker}{}%
\end{pgfscope}%
\begin{pgfscope}%
\pgfsys@transformshift{2.046141in}{1.109949in}%
\pgfsys@useobject{currentmarker}{}%
\end{pgfscope}%
\begin{pgfscope}%
\pgfsys@transformshift{2.062309in}{1.071037in}%
\pgfsys@useobject{currentmarker}{}%
\end{pgfscope}%
\begin{pgfscope}%
\pgfsys@transformshift{2.078410in}{1.077589in}%
\pgfsys@useobject{currentmarker}{}%
\end{pgfscope}%
\begin{pgfscope}%
\pgfsys@transformshift{2.094600in}{1.063153in}%
\pgfsys@useobject{currentmarker}{}%
\end{pgfscope}%
\begin{pgfscope}%
\pgfsys@transformshift{2.110765in}{1.062281in}%
\pgfsys@useobject{currentmarker}{}%
\end{pgfscope}%
\begin{pgfscope}%
\pgfsys@transformshift{2.127015in}{1.041432in}%
\pgfsys@useobject{currentmarker}{}%
\end{pgfscope}%
\begin{pgfscope}%
\pgfsys@transformshift{2.143236in}{1.051214in}%
\pgfsys@useobject{currentmarker}{}%
\end{pgfscope}%
\begin{pgfscope}%
\pgfsys@transformshift{2.159341in}{1.024474in}%
\pgfsys@useobject{currentmarker}{}%
\end{pgfscope}%
\begin{pgfscope}%
\pgfsys@transformshift{2.175495in}{1.018667in}%
\pgfsys@useobject{currentmarker}{}%
\end{pgfscope}%
\begin{pgfscope}%
\pgfsys@transformshift{2.191737in}{1.012299in}%
\pgfsys@useobject{currentmarker}{}%
\end{pgfscope}%
\begin{pgfscope}%
\pgfsys@transformshift{2.207888in}{1.006323in}%
\pgfsys@useobject{currentmarker}{}%
\end{pgfscope}%
\begin{pgfscope}%
\pgfsys@transformshift{2.224056in}{0.990798in}%
\pgfsys@useobject{currentmarker}{}%
\end{pgfscope}%
\begin{pgfscope}%
\pgfsys@transformshift{2.240258in}{0.977054in}%
\pgfsys@useobject{currentmarker}{}%
\end{pgfscope}%
\begin{pgfscope}%
\pgfsys@transformshift{2.256390in}{0.978527in}%
\pgfsys@useobject{currentmarker}{}%
\end{pgfscope}%
\begin{pgfscope}%
\pgfsys@transformshift{2.272567in}{0.970068in}%
\pgfsys@useobject{currentmarker}{}%
\end{pgfscope}%
\begin{pgfscope}%
\pgfsys@transformshift{2.288773in}{0.957301in}%
\pgfsys@useobject{currentmarker}{}%
\end{pgfscope}%
\begin{pgfscope}%
\pgfsys@transformshift{2.304948in}{0.948956in}%
\pgfsys@useobject{currentmarker}{}%
\end{pgfscope}%
\begin{pgfscope}%
\pgfsys@transformshift{2.321118in}{0.942986in}%
\pgfsys@useobject{currentmarker}{}%
\end{pgfscope}%
\begin{pgfscope}%
\pgfsys@transformshift{2.337293in}{0.929624in}%
\pgfsys@useobject{currentmarker}{}%
\end{pgfscope}%
\begin{pgfscope}%
\pgfsys@transformshift{2.353475in}{0.928504in}%
\pgfsys@useobject{currentmarker}{}%
\end{pgfscope}%
\begin{pgfscope}%
\pgfsys@transformshift{2.369657in}{0.917565in}%
\pgfsys@useobject{currentmarker}{}%
\end{pgfscope}%
\begin{pgfscope}%
\pgfsys@transformshift{2.385825in}{0.916485in}%
\pgfsys@useobject{currentmarker}{}%
\end{pgfscope}%
\begin{pgfscope}%
\pgfsys@transformshift{2.401990in}{0.902735in}%
\pgfsys@useobject{currentmarker}{}%
\end{pgfscope}%
\begin{pgfscope}%
\pgfsys@transformshift{2.418150in}{0.890090in}%
\pgfsys@useobject{currentmarker}{}%
\end{pgfscope}%
\begin{pgfscope}%
\pgfsys@transformshift{2.434320in}{0.888688in}%
\pgfsys@useobject{currentmarker}{}%
\end{pgfscope}%
\begin{pgfscope}%
\pgfsys@transformshift{2.450502in}{0.877712in}%
\pgfsys@useobject{currentmarker}{}%
\end{pgfscope}%
\begin{pgfscope}%
\pgfsys@transformshift{2.466690in}{0.867780in}%
\pgfsys@useobject{currentmarker}{}%
\end{pgfscope}%
\begin{pgfscope}%
\pgfsys@transformshift{2.482871in}{0.864670in}%
\pgfsys@useobject{currentmarker}{}%
\end{pgfscope}%
\begin{pgfscope}%
\pgfsys@transformshift{2.499046in}{0.857490in}%
\pgfsys@useobject{currentmarker}{}%
\end{pgfscope}%
\begin{pgfscope}%
\pgfsys@transformshift{2.515222in}{0.853585in}%
\pgfsys@useobject{currentmarker}{}%
\end{pgfscope}%
\begin{pgfscope}%
\pgfsys@transformshift{2.531396in}{0.841882in}%
\pgfsys@useobject{currentmarker}{}%
\end{pgfscope}%
\begin{pgfscope}%
\pgfsys@transformshift{2.547560in}{0.836538in}%
\pgfsys@useobject{currentmarker}{}%
\end{pgfscope}%
\begin{pgfscope}%
\pgfsys@transformshift{2.563732in}{0.830884in}%
\pgfsys@useobject{currentmarker}{}%
\end{pgfscope}%
\begin{pgfscope}%
\pgfsys@transformshift{2.579911in}{0.822677in}%
\pgfsys@useobject{currentmarker}{}%
\end{pgfscope}%
\begin{pgfscope}%
\pgfsys@transformshift{2.596086in}{0.819284in}%
\pgfsys@useobject{currentmarker}{}%
\end{pgfscope}%
\begin{pgfscope}%
\pgfsys@transformshift{2.612260in}{0.814107in}%
\pgfsys@useobject{currentmarker}{}%
\end{pgfscope}%
\begin{pgfscope}%
\pgfsys@transformshift{2.628436in}{0.807084in}%
\pgfsys@useobject{currentmarker}{}%
\end{pgfscope}%
\begin{pgfscope}%
\pgfsys@transformshift{2.644616in}{0.803285in}%
\pgfsys@useobject{currentmarker}{}%
\end{pgfscope}%
\begin{pgfscope}%
\pgfsys@transformshift{2.660788in}{0.797194in}%
\pgfsys@useobject{currentmarker}{}%
\end{pgfscope}%
\begin{pgfscope}%
\pgfsys@transformshift{2.676961in}{0.792695in}%
\pgfsys@useobject{currentmarker}{}%
\end{pgfscope}%
\begin{pgfscope}%
\pgfsys@transformshift{2.693139in}{0.786555in}%
\pgfsys@useobject{currentmarker}{}%
\end{pgfscope}%
\begin{pgfscope}%
\pgfsys@transformshift{2.709314in}{0.780396in}%
\pgfsys@useobject{currentmarker}{}%
\end{pgfscope}%
\begin{pgfscope}%
\pgfsys@transformshift{2.725487in}{0.781004in}%
\pgfsys@useobject{currentmarker}{}%
\end{pgfscope}%
\begin{pgfscope}%
\pgfsys@transformshift{2.741660in}{0.773270in}%
\pgfsys@useobject{currentmarker}{}%
\end{pgfscope}%
\begin{pgfscope}%
\pgfsys@transformshift{2.757836in}{0.769307in}%
\pgfsys@useobject{currentmarker}{}%
\end{pgfscope}%
\begin{pgfscope}%
\pgfsys@transformshift{2.774015in}{0.765017in}%
\pgfsys@useobject{currentmarker}{}%
\end{pgfscope}%
\begin{pgfscope}%
\pgfsys@transformshift{2.790187in}{0.762884in}%
\pgfsys@useobject{currentmarker}{}%
\end{pgfscope}%
\begin{pgfscope}%
\pgfsys@transformshift{2.806359in}{0.756767in}%
\pgfsys@useobject{currentmarker}{}%
\end{pgfscope}%
\begin{pgfscope}%
\pgfsys@transformshift{2.822535in}{0.753893in}%
\pgfsys@useobject{currentmarker}{}%
\end{pgfscope}%
\begin{pgfscope}%
\pgfsys@transformshift{2.838711in}{0.753737in}%
\pgfsys@useobject{currentmarker}{}%
\end{pgfscope}%
\begin{pgfscope}%
\pgfsys@transformshift{2.854888in}{0.749431in}%
\pgfsys@useobject{currentmarker}{}%
\end{pgfscope}%
\begin{pgfscope}%
\pgfsys@transformshift{2.871062in}{0.747507in}%
\pgfsys@useobject{currentmarker}{}%
\end{pgfscope}%
\begin{pgfscope}%
\pgfsys@transformshift{2.887237in}{0.747416in}%
\pgfsys@useobject{currentmarker}{}%
\end{pgfscope}%
\begin{pgfscope}%
\pgfsys@transformshift{2.903412in}{0.744217in}%
\pgfsys@useobject{currentmarker}{}%
\end{pgfscope}%
\begin{pgfscope}%
\pgfsys@transformshift{2.919587in}{0.739510in}%
\pgfsys@useobject{currentmarker}{}%
\end{pgfscope}%
\begin{pgfscope}%
\pgfsys@transformshift{2.935762in}{0.739205in}%
\pgfsys@useobject{currentmarker}{}%
\end{pgfscope}%
\begin{pgfscope}%
\pgfsys@transformshift{2.951937in}{0.737664in}%
\pgfsys@useobject{currentmarker}{}%
\end{pgfscope}%
\begin{pgfscope}%
\pgfsys@transformshift{2.968113in}{0.735149in}%
\pgfsys@useobject{currentmarker}{}%
\end{pgfscope}%
\begin{pgfscope}%
\pgfsys@transformshift{2.984288in}{0.731992in}%
\pgfsys@useobject{currentmarker}{}%
\end{pgfscope}%
\begin{pgfscope}%
\pgfsys@transformshift{3.000463in}{0.731346in}%
\pgfsys@useobject{currentmarker}{}%
\end{pgfscope}%
\begin{pgfscope}%
\pgfsys@transformshift{3.016638in}{0.729415in}%
\pgfsys@useobject{currentmarker}{}%
\end{pgfscope}%
\begin{pgfscope}%
\pgfsys@transformshift{3.032813in}{0.728433in}%
\pgfsys@useobject{currentmarker}{}%
\end{pgfscope}%
\begin{pgfscope}%
\pgfsys@transformshift{3.048988in}{0.727884in}%
\pgfsys@useobject{currentmarker}{}%
\end{pgfscope}%
\begin{pgfscope}%
\pgfsys@transformshift{3.065163in}{0.726141in}%
\pgfsys@useobject{currentmarker}{}%
\end{pgfscope}%
\begin{pgfscope}%
\pgfsys@transformshift{3.081338in}{0.724961in}%
\pgfsys@useobject{currentmarker}{}%
\end{pgfscope}%
\begin{pgfscope}%
\pgfsys@transformshift{3.097513in}{0.724569in}%
\pgfsys@useobject{currentmarker}{}%
\end{pgfscope}%
\begin{pgfscope}%
\pgfsys@transformshift{3.113688in}{0.723232in}%
\pgfsys@useobject{currentmarker}{}%
\end{pgfscope}%
\begin{pgfscope}%
\pgfsys@transformshift{3.129863in}{0.722611in}%
\pgfsys@useobject{currentmarker}{}%
\end{pgfscope}%
\begin{pgfscope}%
\pgfsys@transformshift{3.146038in}{0.721029in}%
\pgfsys@useobject{currentmarker}{}%
\end{pgfscope}%
\begin{pgfscope}%
\pgfsys@transformshift{3.162214in}{0.721151in}%
\pgfsys@useobject{currentmarker}{}%
\end{pgfscope}%
\begin{pgfscope}%
\pgfsys@transformshift{3.178389in}{0.719183in}%
\pgfsys@useobject{currentmarker}{}%
\end{pgfscope}%
\begin{pgfscope}%
\pgfsys@transformshift{3.194564in}{0.719238in}%
\pgfsys@useobject{currentmarker}{}%
\end{pgfscope}%
\begin{pgfscope}%
\pgfsys@transformshift{3.210739in}{0.718981in}%
\pgfsys@useobject{currentmarker}{}%
\end{pgfscope}%
\begin{pgfscope}%
\pgfsys@transformshift{3.226914in}{0.718102in}%
\pgfsys@useobject{currentmarker}{}%
\end{pgfscope}%
\begin{pgfscope}%
\pgfsys@transformshift{3.243089in}{0.717109in}%
\pgfsys@useobject{currentmarker}{}%
\end{pgfscope}%
\begin{pgfscope}%
\pgfsys@transformshift{3.259264in}{0.717185in}%
\pgfsys@useobject{currentmarker}{}%
\end{pgfscope}%
\begin{pgfscope}%
\pgfsys@transformshift{3.275439in}{0.716098in}%
\pgfsys@useobject{currentmarker}{}%
\end{pgfscope}%
\begin{pgfscope}%
\pgfsys@transformshift{3.291615in}{0.716084in}%
\pgfsys@useobject{currentmarker}{}%
\end{pgfscope}%
\begin{pgfscope}%
\pgfsys@transformshift{3.307790in}{0.715935in}%
\pgfsys@useobject{currentmarker}{}%
\end{pgfscope}%
\begin{pgfscope}%
\pgfsys@transformshift{3.323965in}{0.715840in}%
\pgfsys@useobject{currentmarker}{}%
\end{pgfscope}%
\begin{pgfscope}%
\pgfsys@transformshift{3.340140in}{0.715462in}%
\pgfsys@useobject{currentmarker}{}%
\end{pgfscope}%
\begin{pgfscope}%
\pgfsys@transformshift{3.356315in}{0.714178in}%
\pgfsys@useobject{currentmarker}{}%
\end{pgfscope}%
\begin{pgfscope}%
\pgfsys@transformshift{3.372490in}{0.714182in}%
\pgfsys@useobject{currentmarker}{}%
\end{pgfscope}%
\begin{pgfscope}%
\pgfsys@transformshift{3.388665in}{0.713769in}%
\pgfsys@useobject{currentmarker}{}%
\end{pgfscope}%
\begin{pgfscope}%
\pgfsys@transformshift{3.404840in}{0.713832in}%
\pgfsys@useobject{currentmarker}{}%
\end{pgfscope}%
\begin{pgfscope}%
\pgfsys@transformshift{3.421015in}{0.713027in}%
\pgfsys@useobject{currentmarker}{}%
\end{pgfscope}%
\begin{pgfscope}%
\pgfsys@transformshift{3.437190in}{0.713270in}%
\pgfsys@useobject{currentmarker}{}%
\end{pgfscope}%
\begin{pgfscope}%
\pgfsys@transformshift{3.453365in}{0.712901in}%
\pgfsys@useobject{currentmarker}{}%
\end{pgfscope}%
\begin{pgfscope}%
\pgfsys@transformshift{3.469540in}{0.712786in}%
\pgfsys@useobject{currentmarker}{}%
\end{pgfscope}%
\begin{pgfscope}%
\pgfsys@transformshift{3.485715in}{0.712741in}%
\pgfsys@useobject{currentmarker}{}%
\end{pgfscope}%
\begin{pgfscope}%
\pgfsys@transformshift{3.501891in}{0.712751in}%
\pgfsys@useobject{currentmarker}{}%
\end{pgfscope}%
\begin{pgfscope}%
\pgfsys@transformshift{3.518066in}{0.712821in}%
\pgfsys@useobject{currentmarker}{}%
\end{pgfscope}%
\begin{pgfscope}%
\pgfsys@transformshift{3.534241in}{0.712598in}%
\pgfsys@useobject{currentmarker}{}%
\end{pgfscope}%
\begin{pgfscope}%
\pgfsys@transformshift{3.550416in}{0.712817in}%
\pgfsys@useobject{currentmarker}{}%
\end{pgfscope}%
\begin{pgfscope}%
\pgfsys@transformshift{3.566591in}{0.712159in}%
\pgfsys@useobject{currentmarker}{}%
\end{pgfscope}%
\begin{pgfscope}%
\pgfsys@transformshift{3.582766in}{0.712229in}%
\pgfsys@useobject{currentmarker}{}%
\end{pgfscope}%
\begin{pgfscope}%
\pgfsys@transformshift{3.598941in}{0.712188in}%
\pgfsys@useobject{currentmarker}{}%
\end{pgfscope}%
\begin{pgfscope}%
\pgfsys@transformshift{3.615116in}{0.712108in}%
\pgfsys@useobject{currentmarker}{}%
\end{pgfscope}%
\begin{pgfscope}%
\pgfsys@transformshift{3.631291in}{0.712030in}%
\pgfsys@useobject{currentmarker}{}%
\end{pgfscope}%
\begin{pgfscope}%
\pgfsys@transformshift{3.647466in}{0.711970in}%
\pgfsys@useobject{currentmarker}{}%
\end{pgfscope}%
\begin{pgfscope}%
\pgfsys@transformshift{3.663641in}{0.711685in}%
\pgfsys@useobject{currentmarker}{}%
\end{pgfscope}%
\begin{pgfscope}%
\pgfsys@transformshift{3.679817in}{0.711956in}%
\pgfsys@useobject{currentmarker}{}%
\end{pgfscope}%
\begin{pgfscope}%
\pgfsys@transformshift{3.695992in}{0.711792in}%
\pgfsys@useobject{currentmarker}{}%
\end{pgfscope}%
\begin{pgfscope}%
\pgfsys@transformshift{3.712167in}{0.711846in}%
\pgfsys@useobject{currentmarker}{}%
\end{pgfscope}%
\begin{pgfscope}%
\pgfsys@transformshift{3.728342in}{0.711723in}%
\pgfsys@useobject{currentmarker}{}%
\end{pgfscope}%
\begin{pgfscope}%
\pgfsys@transformshift{3.744517in}{0.711745in}%
\pgfsys@useobject{currentmarker}{}%
\end{pgfscope}%
\begin{pgfscope}%
\pgfsys@transformshift{3.760692in}{0.711602in}%
\pgfsys@useobject{currentmarker}{}%
\end{pgfscope}%
\begin{pgfscope}%
\pgfsys@transformshift{3.776867in}{0.711559in}%
\pgfsys@useobject{currentmarker}{}%
\end{pgfscope}%
\begin{pgfscope}%
\pgfsys@transformshift{3.788399in}{0.711360in}%
\pgfsys@useobject{currentmarker}{}%
\end{pgfscope}%
\end{pgfscope}%
\begin{pgfscope}%
\pgfpathrectangle{\pgfqpoint{0.594525in}{0.417642in}}{\pgfqpoint{3.345963in}{2.050688in}}%
\pgfusepath{clip}%
\pgfsetbuttcap%
\pgfsetroundjoin%
\pgfsetlinewidth{1.505625pt}%
\definecolor{currentstroke}{rgb}{0.000000,0.447059,0.698039}%
\pgfsetstrokecolor{currentstroke}%
\pgfsetdash{{5.550000pt}{2.400000pt}}{0.000000pt}%
\pgfpathmoveto{\pgfqpoint{0.746614in}{0.710842in}}%
\pgfpathlineto{\pgfqpoint{3.788399in}{0.710842in}}%
\pgfpathlineto{\pgfqpoint{3.788399in}{0.710842in}}%
\pgfusepath{stroke}%
\end{pgfscope}%
\begin{pgfscope}%
\pgfpathrectangle{\pgfqpoint{0.594525in}{0.417642in}}{\pgfqpoint{3.345963in}{2.050688in}}%
\pgfusepath{clip}%
\pgfsetbuttcap%
\pgfsetroundjoin%
\pgfsetlinewidth{1.505625pt}%
\definecolor{currentstroke}{rgb}{0.000000,0.619608,0.450980}%
\pgfsetstrokecolor{currentstroke}%
\pgfsetdash{{5.550000pt}{2.400000pt}}{0.000000pt}%
\pgfpathmoveto{\pgfqpoint{0.746614in}{1.869142in}}%
\pgfpathlineto{\pgfqpoint{0.885038in}{1.785721in}}%
\pgfpathlineto{\pgfqpoint{0.966011in}{1.736922in}}%
\pgfpathlineto{\pgfqpoint{1.023462in}{1.702299in}}%
\pgfpathlineto{\pgfqpoint{1.068025in}{1.675443in}}%
\pgfpathlineto{\pgfqpoint{1.104435in}{1.653501in}}%
\pgfpathlineto{\pgfqpoint{1.135219in}{1.634948in}}%
\pgfpathlineto{\pgfqpoint{1.161886in}{1.618878in}}%
\pgfpathlineto{\pgfqpoint{1.185408in}{1.604702in}}%
\pgfpathlineto{\pgfqpoint{1.206449in}{1.592022in}}%
\pgfpathlineto{\pgfqpoint{1.225482in}{1.580551in}}%
\pgfpathlineto{\pgfqpoint{1.242859in}{1.570079in}}%
\pgfpathlineto{\pgfqpoint{1.258844in}{1.560446in}}%
\pgfpathlineto{\pgfqpoint{1.273643in}{1.551527in}}%
\pgfpathlineto{\pgfqpoint{1.287422in}{1.543223in}}%
\pgfpathlineto{\pgfqpoint{1.300310in}{1.535456in}}%
\pgfpathlineto{\pgfqpoint{1.318206in}{1.524671in}}%
\pgfpathlineto{\pgfqpoint{1.334629in}{1.514773in}}%
\pgfpathlineto{\pgfqpoint{1.349804in}{1.505628in}}%
\pgfpathlineto{\pgfqpoint{1.368394in}{1.494425in}}%
\pgfpathlineto{\pgfqpoint{1.385401in}{1.484176in}}%
\pgfpathlineto{\pgfqpoint{1.401072in}{1.474732in}}%
\pgfpathlineto{\pgfqpoint{1.415602in}{1.465975in}}%
\pgfpathlineto{\pgfqpoint{1.432394in}{1.455855in}}%
\pgfpathlineto{\pgfqpoint{1.447882in}{1.446521in}}%
\pgfpathlineto{\pgfqpoint{1.462256in}{1.437859in}}%
\pgfpathlineto{\pgfqpoint{1.478241in}{1.428226in}}%
\pgfpathlineto{\pgfqpoint{1.495404in}{1.417882in}}%
\pgfpathlineto{\pgfqpoint{1.513367in}{1.407057in}}%
\pgfpathlineto{\pgfqpoint{1.529846in}{1.397125in}}%
\pgfpathlineto{\pgfqpoint{1.545069in}{1.387951in}}%
\pgfpathlineto{\pgfqpoint{1.560913in}{1.378403in}}%
\pgfpathlineto{\pgfqpoint{1.577158in}{1.368613in}}%
\pgfpathlineto{\pgfqpoint{1.593622in}{1.358691in}}%
\pgfpathlineto{\pgfqpoint{1.610159in}{1.348725in}}%
\pgfpathlineto{\pgfqpoint{1.625431in}{1.339521in}}%
\pgfpathlineto{\pgfqpoint{1.641886in}{1.329605in}}%
\pgfpathlineto{\pgfqpoint{1.658131in}{1.319814in}}%
\pgfpathlineto{\pgfqpoint{1.674116in}{1.310181in}}%
\pgfpathlineto{\pgfqpoint{1.690691in}{1.300192in}}%
\pgfpathlineto{\pgfqpoint{1.706811in}{1.290477in}}%
\pgfpathlineto{\pgfqpoint{1.722482in}{1.281033in}}%
\pgfpathlineto{\pgfqpoint{1.738409in}{1.271435in}}%
\pgfpathlineto{\pgfqpoint{1.755089in}{1.261383in}}%
\pgfpathlineto{\pgfqpoint{1.771074in}{1.251749in}}%
\pgfpathlineto{\pgfqpoint{1.786967in}{1.242171in}}%
\pgfpathlineto{\pgfqpoint{1.803204in}{1.232386in}}%
\pgfpathlineto{\pgfqpoint{1.819614in}{1.222496in}}%
\pgfpathlineto{\pgfqpoint{1.836062in}{1.212584in}}%
\pgfpathlineto{\pgfqpoint{1.852046in}{1.202951in}}%
\pgfpathlineto{\pgfqpoint{1.867940in}{1.193372in}}%
\pgfpathlineto{\pgfqpoint{1.884344in}{1.183486in}}%
\pgfpathlineto{\pgfqpoint{1.900741in}{1.173605in}}%
\pgfpathlineto{\pgfqpoint{1.916750in}{1.163957in}}%
\pgfpathlineto{\pgfqpoint{1.932888in}{1.154232in}}%
\pgfpathlineto{\pgfqpoint{1.949034in}{1.144501in}}%
\pgfpathlineto{\pgfqpoint{1.965094in}{1.134823in}}%
\pgfpathlineto{\pgfqpoint{1.981405in}{1.124993in}}%
\pgfpathlineto{\pgfqpoint{1.997628in}{1.115216in}}%
\pgfpathlineto{\pgfqpoint{2.013686in}{1.105539in}}%
\pgfpathlineto{\pgfqpoint{2.029846in}{1.095800in}}%
\pgfpathlineto{\pgfqpoint{2.046141in}{1.085979in}}%
\pgfpathlineto{\pgfqpoint{2.062309in}{1.076235in}}%
\pgfpathlineto{\pgfqpoint{2.078410in}{1.066532in}}%
\pgfpathlineto{\pgfqpoint{2.094600in}{1.056775in}}%
\pgfpathlineto{\pgfqpoint{2.110765in}{1.047034in}}%
\pgfpathlineto{\pgfqpoint{2.127015in}{1.037241in}}%
\pgfpathlineto{\pgfqpoint{2.143236in}{1.027465in}}%
\pgfpathlineto{\pgfqpoint{2.159341in}{1.017759in}}%
\pgfpathlineto{\pgfqpoint{2.175495in}{1.008024in}}%
\pgfpathlineto{\pgfqpoint{2.191737in}{0.998235in}}%
\pgfpathlineto{\pgfqpoint{2.207888in}{0.988502in}}%
\pgfpathlineto{\pgfqpoint{2.224056in}{0.978758in}}%
\pgfpathlineto{\pgfqpoint{2.240258in}{0.968995in}}%
\pgfpathlineto{\pgfqpoint{2.256390in}{0.959273in}}%
\pgfpathlineto{\pgfqpoint{2.272567in}{0.949524in}}%
\pgfpathlineto{\pgfqpoint{2.288773in}{0.939757in}}%
\pgfpathlineto{\pgfqpoint{2.304948in}{0.930009in}}%
\pgfpathlineto{\pgfqpoint{2.321118in}{0.920264in}}%
\pgfpathlineto{\pgfqpoint{2.337293in}{0.910516in}}%
\pgfpathlineto{\pgfqpoint{2.353475in}{0.900764in}}%
\pgfpathlineto{\pgfqpoint{2.369657in}{0.891012in}}%
\pgfpathlineto{\pgfqpoint{2.385825in}{0.881268in}}%
\pgfpathlineto{\pgfqpoint{2.401990in}{0.871526in}}%
\pgfpathlineto{\pgfqpoint{2.418150in}{0.861787in}}%
\pgfpathlineto{\pgfqpoint{2.434320in}{0.852042in}}%
\pgfpathlineto{\pgfqpoint{2.450502in}{0.842290in}}%
\pgfpathlineto{\pgfqpoint{2.466690in}{0.832535in}}%
\pgfpathlineto{\pgfqpoint{2.482871in}{0.822783in}}%
\pgfpathlineto{\pgfqpoint{2.499046in}{0.813035in}}%
\pgfpathlineto{\pgfqpoint{2.515222in}{0.803287in}}%
\pgfpathlineto{\pgfqpoint{2.531396in}{0.793539in}}%
\pgfpathlineto{\pgfqpoint{2.547560in}{0.783798in}}%
\pgfpathlineto{\pgfqpoint{2.563732in}{0.774052in}}%
\pgfpathlineto{\pgfqpoint{2.579911in}{0.764302in}}%
\pgfpathlineto{\pgfqpoint{2.596086in}{0.754554in}}%
\pgfpathlineto{\pgfqpoint{2.612260in}{0.744807in}}%
\pgfpathlineto{\pgfqpoint{2.628436in}{0.735058in}}%
\pgfpathlineto{\pgfqpoint{2.644616in}{0.725307in}}%
\pgfpathlineto{\pgfqpoint{2.660788in}{0.715561in}}%
\pgfpathlineto{\pgfqpoint{2.676961in}{0.705815in}}%
\pgfpathlineto{\pgfqpoint{2.693139in}{0.696065in}}%
\pgfpathlineto{\pgfqpoint{2.709314in}{0.686317in}}%
\pgfpathlineto{\pgfqpoint{2.725487in}{0.676570in}}%
\pgfpathlineto{\pgfqpoint{2.741660in}{0.666823in}}%
\pgfpathlineto{\pgfqpoint{2.757836in}{0.657075in}}%
\pgfpathlineto{\pgfqpoint{2.774015in}{0.647325in}}%
\pgfpathlineto{\pgfqpoint{2.790187in}{0.637579in}}%
\pgfpathlineto{\pgfqpoint{2.806359in}{0.627832in}}%
\pgfpathlineto{\pgfqpoint{2.822535in}{0.618084in}}%
\pgfpathlineto{\pgfqpoint{2.838711in}{0.608336in}}%
\pgfpathlineto{\pgfqpoint{2.854888in}{0.598587in}}%
\pgfpathlineto{\pgfqpoint{2.871062in}{0.588839in}}%
\pgfpathlineto{\pgfqpoint{2.887237in}{0.579091in}}%
\pgfpathlineto{\pgfqpoint{2.903412in}{0.569343in}}%
\pgfpathlineto{\pgfqpoint{2.919587in}{0.559595in}}%
\pgfpathlineto{\pgfqpoint{2.935762in}{0.549848in}}%
\pgfpathlineto{\pgfqpoint{2.951937in}{0.540099in}}%
\pgfpathlineto{\pgfqpoint{2.968113in}{0.530351in}}%
\pgfpathlineto{\pgfqpoint{2.984288in}{0.520603in}}%
\pgfpathlineto{\pgfqpoint{3.000463in}{0.510855in}}%
\pgfusepath{stroke}%
\end{pgfscope}%
\begin{pgfscope}%
\pgfpathrectangle{\pgfqpoint{0.594525in}{0.417642in}}{\pgfqpoint{3.345963in}{2.050688in}}%
\pgfusepath{clip}%
\pgfsetbuttcap%
\pgfsetroundjoin%
\pgfsetlinewidth{1.505625pt}%
\definecolor{currentstroke}{rgb}{0.835294,0.368627,0.000000}%
\pgfsetstrokecolor{currentstroke}%
\pgfsetdash{{5.550000pt}{2.400000pt}}{0.000000pt}%
\pgfpathmoveto{\pgfqpoint{0.746614in}{2.298226in}}%
\pgfpathlineto{\pgfqpoint{0.885038in}{2.131383in}}%
\pgfpathlineto{\pgfqpoint{0.966011in}{2.033786in}}%
\pgfpathlineto{\pgfqpoint{1.023462in}{1.964540in}}%
\pgfpathlineto{\pgfqpoint{1.068025in}{1.910828in}}%
\pgfpathlineto{\pgfqpoint{1.104435in}{1.866943in}}%
\pgfpathlineto{\pgfqpoint{1.135219in}{1.829838in}}%
\pgfpathlineto{\pgfqpoint{1.161886in}{1.797697in}}%
\pgfpathlineto{\pgfqpoint{1.185408in}{1.769346in}}%
\pgfpathlineto{\pgfqpoint{1.206449in}{1.743985in}}%
\pgfpathlineto{\pgfqpoint{1.225482in}{1.721044in}}%
\pgfpathlineto{\pgfqpoint{1.242859in}{1.700100in}}%
\pgfpathlineto{\pgfqpoint{1.258844in}{1.680833in}}%
\pgfpathlineto{\pgfqpoint{1.273643in}{1.662995in}}%
\pgfpathlineto{\pgfqpoint{1.287422in}{1.646388in}}%
\pgfpathlineto{\pgfqpoint{1.300310in}{1.630854in}}%
\pgfpathlineto{\pgfqpoint{1.318206in}{1.609284in}}%
\pgfpathlineto{\pgfqpoint{1.334629in}{1.589489in}}%
\pgfpathlineto{\pgfqpoint{1.349804in}{1.571199in}}%
\pgfpathlineto{\pgfqpoint{1.368394in}{1.548792in}}%
\pgfpathlineto{\pgfqpoint{1.385401in}{1.528294in}}%
\pgfpathlineto{\pgfqpoint{1.401072in}{1.509405in}}%
\pgfpathlineto{\pgfqpoint{1.415602in}{1.491892in}}%
\pgfpathlineto{\pgfqpoint{1.432394in}{1.471653in}}%
\pgfpathlineto{\pgfqpoint{1.447882in}{1.452984in}}%
\pgfpathlineto{\pgfqpoint{1.462256in}{1.435660in}}%
\pgfpathlineto{\pgfqpoint{1.478241in}{1.416393in}}%
\pgfpathlineto{\pgfqpoint{1.495404in}{1.395707in}}%
\pgfpathlineto{\pgfqpoint{1.513367in}{1.374056in}}%
\pgfpathlineto{\pgfqpoint{1.529846in}{1.354193in}}%
\pgfpathlineto{\pgfqpoint{1.545069in}{1.335844in}}%
\pgfpathlineto{\pgfqpoint{1.560913in}{1.316748in}}%
\pgfpathlineto{\pgfqpoint{1.577158in}{1.297168in}}%
\pgfpathlineto{\pgfqpoint{1.593622in}{1.277323in}}%
\pgfpathlineto{\pgfqpoint{1.610159in}{1.257391in}}%
\pgfpathlineto{\pgfqpoint{1.625431in}{1.238985in}}%
\pgfpathlineto{\pgfqpoint{1.641886in}{1.219151in}}%
\pgfpathlineto{\pgfqpoint{1.658131in}{1.199571in}}%
\pgfpathlineto{\pgfqpoint{1.674116in}{1.180304in}}%
\pgfpathlineto{\pgfqpoint{1.690691in}{1.160326in}}%
\pgfpathlineto{\pgfqpoint{1.706811in}{1.140896in}}%
\pgfpathlineto{\pgfqpoint{1.722482in}{1.122008in}}%
\pgfpathlineto{\pgfqpoint{1.738409in}{1.102811in}}%
\pgfpathlineto{\pgfqpoint{1.755089in}{1.082707in}}%
\pgfpathlineto{\pgfqpoint{1.771074in}{1.063441in}}%
\pgfpathlineto{\pgfqpoint{1.786967in}{1.044284in}}%
\pgfpathlineto{\pgfqpoint{1.803204in}{1.024714in}}%
\pgfpathlineto{\pgfqpoint{1.819614in}{1.004935in}}%
\pgfpathlineto{\pgfqpoint{1.836062in}{0.985110in}}%
\pgfpathlineto{\pgfqpoint{1.852046in}{0.965844in}}%
\pgfpathlineto{\pgfqpoint{1.867940in}{0.946687in}}%
\pgfpathlineto{\pgfqpoint{1.884344in}{0.926915in}}%
\pgfpathlineto{\pgfqpoint{1.900741in}{0.907152in}}%
\pgfpathlineto{\pgfqpoint{1.916750in}{0.887856in}}%
\pgfpathlineto{\pgfqpoint{1.932888in}{0.868405in}}%
\pgfpathlineto{\pgfqpoint{1.949034in}{0.848944in}}%
\pgfpathlineto{\pgfqpoint{1.965094in}{0.829587in}}%
\pgfpathlineto{\pgfqpoint{1.981405in}{0.809927in}}%
\pgfpathlineto{\pgfqpoint{1.997628in}{0.790374in}}%
\pgfpathlineto{\pgfqpoint{2.013686in}{0.771019in}}%
\pgfpathlineto{\pgfqpoint{2.029846in}{0.751541in}}%
\pgfpathlineto{\pgfqpoint{2.046141in}{0.731900in}}%
\pgfpathlineto{\pgfqpoint{2.062309in}{0.712413in}}%
\pgfpathlineto{\pgfqpoint{2.078410in}{0.693006in}}%
\pgfusepath{stroke}%
\end{pgfscope}%
\begin{pgfscope}%
\pgfsetrectcap%
\pgfsetmiterjoin%
\pgfsetlinewidth{0.803000pt}%
\definecolor{currentstroke}{rgb}{0.000000,0.000000,0.000000}%
\pgfsetstrokecolor{currentstroke}%
\pgfsetdash{}{0pt}%
\pgfpathmoveto{\pgfqpoint{0.594525in}{0.417642in}}%
\pgfpathlineto{\pgfqpoint{0.594525in}{2.468330in}}%
\pgfusepath{stroke}%
\end{pgfscope}%
\begin{pgfscope}%
\pgfsetrectcap%
\pgfsetmiterjoin%
\pgfsetlinewidth{0.803000pt}%
\definecolor{currentstroke}{rgb}{0.000000,0.000000,0.000000}%
\pgfsetstrokecolor{currentstroke}%
\pgfsetdash{}{0pt}%
\pgfpathmoveto{\pgfqpoint{3.940488in}{0.417642in}}%
\pgfpathlineto{\pgfqpoint{3.940488in}{2.468330in}}%
\pgfusepath{stroke}%
\end{pgfscope}%
\begin{pgfscope}%
\pgfsetrectcap%
\pgfsetmiterjoin%
\pgfsetlinewidth{0.803000pt}%
\definecolor{currentstroke}{rgb}{0.000000,0.000000,0.000000}%
\pgfsetstrokecolor{currentstroke}%
\pgfsetdash{}{0pt}%
\pgfpathmoveto{\pgfqpoint{0.594525in}{0.417642in}}%
\pgfpathlineto{\pgfqpoint{3.940488in}{0.417642in}}%
\pgfusepath{stroke}%
\end{pgfscope}%
\begin{pgfscope}%
\pgfsetrectcap%
\pgfsetmiterjoin%
\pgfsetlinewidth{0.803000pt}%
\definecolor{currentstroke}{rgb}{0.000000,0.000000,0.000000}%
\pgfsetstrokecolor{currentstroke}%
\pgfsetdash{}{0pt}%
\pgfpathmoveto{\pgfqpoint{0.594525in}{2.468330in}}%
\pgfpathlineto{\pgfqpoint{3.940488in}{2.468330in}}%
\pgfusepath{stroke}%
\end{pgfscope}%
\begin{pgfscope}%
\pgfsetbuttcap%
\pgfsetmiterjoin%
\definecolor{currentfill}{rgb}{1.000000,1.000000,1.000000}%
\pgfsetfillcolor{currentfill}%
\pgfsetfillopacity{0.800000}%
\pgfsetlinewidth{1.003750pt}%
\definecolor{currentstroke}{rgb}{0.800000,0.800000,0.800000}%
\pgfsetstrokecolor{currentstroke}%
\pgfsetstrokeopacity{0.800000}%
\pgfsetdash{}{0pt}%
\pgfpathmoveto{\pgfqpoint{2.344930in}{1.837575in}}%
\pgfpathlineto{\pgfqpoint{3.862710in}{1.837575in}}%
\pgfpathquadraticcurveto{\pgfqpoint{3.884932in}{1.837575in}}{\pgfqpoint{3.884932in}{1.859797in}}%
\pgfpathlineto{\pgfqpoint{3.884932in}{2.390552in}}%
\pgfpathquadraticcurveto{\pgfqpoint{3.884932in}{2.412774in}}{\pgfqpoint{3.862710in}{2.412774in}}%
\pgfpathlineto{\pgfqpoint{2.344930in}{2.412774in}}%
\pgfpathquadraticcurveto{\pgfqpoint{2.322707in}{2.412774in}}{\pgfqpoint{2.322707in}{2.390552in}}%
\pgfpathlineto{\pgfqpoint{2.322707in}{1.859797in}}%
\pgfpathquadraticcurveto{\pgfqpoint{2.322707in}{1.837575in}}{\pgfqpoint{2.344930in}{1.837575in}}%
\pgfpathlineto{\pgfqpoint{2.344930in}{1.837575in}}%
\pgfpathclose%
\pgfusepath{stroke,fill}%
\end{pgfscope}%
\begin{pgfscope}%
\pgfsetbuttcap%
\pgfsetroundjoin%
\pgfsetlinewidth{1.505625pt}%
\definecolor{currentstroke}{rgb}{0.000000,0.447059,0.698039}%
\pgfsetstrokecolor{currentstroke}%
\pgfsetdash{{5.550000pt}{2.400000pt}}{0.000000pt}%
\pgfpathmoveto{\pgfqpoint{2.367152in}{2.307358in}}%
\pgfpathlineto{\pgfqpoint{2.478263in}{2.307358in}}%
\pgfpathlineto{\pgfqpoint{2.589374in}{2.307358in}}%
\pgfusepath{stroke}%
\end{pgfscope}%
\begin{pgfscope}%
\definecolor{textcolor}{rgb}{0.000000,0.000000,0.000000}%
\pgfsetstrokecolor{textcolor}%
\pgfsetfillcolor{textcolor}%
\pgftext[x=2.678263in,y=2.268469in,left,base]{\color{textcolor}\rmfamily\fontsize{8.000000}{9.600000}\selectfont White noise \(\displaystyle h_{0}f^{0}\)}%
\end{pgfscope}%
\begin{pgfscope}%
\pgfsetbuttcap%
\pgfsetroundjoin%
\pgfsetlinewidth{1.505625pt}%
\definecolor{currentstroke}{rgb}{0.000000,0.619608,0.450980}%
\pgfsetstrokecolor{currentstroke}%
\pgfsetdash{{5.550000pt}{2.400000pt}}{0.000000pt}%
\pgfpathmoveto{\pgfqpoint{2.367152in}{2.130336in}}%
\pgfpathlineto{\pgfqpoint{2.478263in}{2.130336in}}%
\pgfpathlineto{\pgfqpoint{2.589374in}{2.130336in}}%
\pgfusepath{stroke}%
\end{pgfscope}%
\begin{pgfscope}%
\definecolor{textcolor}{rgb}{0.000000,0.000000,0.000000}%
\pgfsetstrokecolor{textcolor}%
\pgfsetfillcolor{textcolor}%
\pgftext[x=2.678263in,y=2.091447in,left,base]{\color{textcolor}\rmfamily\fontsize{8.000000}{9.600000}\selectfont Flicker noise \(\displaystyle h_{-1}f^{-1}\)}%
\end{pgfscope}%
\begin{pgfscope}%
\pgfsetbuttcap%
\pgfsetroundjoin%
\pgfsetlinewidth{1.505625pt}%
\definecolor{currentstroke}{rgb}{0.835294,0.368627,0.000000}%
\pgfsetstrokecolor{currentstroke}%
\pgfsetdash{{5.550000pt}{2.400000pt}}{0.000000pt}%
\pgfpathmoveto{\pgfqpoint{2.367152in}{1.947914in}}%
\pgfpathlineto{\pgfqpoint{2.478263in}{1.947914in}}%
\pgfpathlineto{\pgfqpoint{2.589374in}{1.947914in}}%
\pgfusepath{stroke}%
\end{pgfscope}%
\begin{pgfscope}%
\definecolor{textcolor}{rgb}{0.000000,0.000000,0.000000}%
\pgfsetstrokecolor{textcolor}%
\pgfsetfillcolor{textcolor}%
\pgftext[x=2.678263in,y=1.909025in,left,base]{\color{textcolor}\rmfamily\fontsize{8.000000}{9.600000}\selectfont Random walk \(\displaystyle h_{-2}f^{-2}\)}%
\end{pgfscope}%
\end{pgfpicture}%
\makeatother%
\endgroup%

    \caption{A simulated power spectrum containing white noise, flicker noise and random walk behaviour.}
    \label{fig:adev_example_psd}
\end{figure}

To get an even better representation of the individual noise contributions, the Allan variance or Allan deviation can be used. The Allan deviation plot shown in figure \ref{fig:adev_example_adev} gives very clean results and all noise components can be clearly identified. The individual components were plotted using dashed lines as well.

\begin{figure}[hb]
    \centering
    \input{images/example_adev.pgf}
    \caption{A simulated Allan deviation containing white noise, flicker noise and random walk behaviour.}
    \label{fig:adev_example_adev}
\end{figure}

The Allan variance was calculated using the overlapping Allan variance algorithm \cite{oadev_definition} and only Allan deviation values for frequency vales of $(1, 2, 4)$ per decade were plotted. The overlapping Allan variance gives a better confidence at longer intervals or lower frequencies, allowing to identify very low frequency noise like the random walk shown here. Reference \cite{oadev_definition} also gives a very good comparison of other algorithms to identify even more noise types in data sets, like phase noise. Plotting only three values per decade improves the clarity of the plot, because at longer taus, even though the overlapping Allan variance is used, some oscillations inevitably show up. Using fewer values of $\tau$ causes less distractions in this case.
From the figure \ref{fig:adev_example_adev}, the Allan deviation of the flicker noise can be estimated from the flat minimum to be around $2.3$ or $\sqrt{5}$. Using table \ref{tab:adev_alpha} the Allan variance can be converted to
\begin{equation}
    h_{-1} = \frac{5}{2 \ln 2} \approx 3.6 \nonumber
\end{equation}

Using the previously found $h_0$, this corner frequency is calculated using equation \ref{eqn:corner_frequency} to be:
\begin{equation}
    f_c = \frac{5}{\qty{2e-3}{\per \Hz} \cdot 2 \ln 2} \approx \qty{1.8}{\kHz} \nonumber
\end{equation}

This is obviously the same result as the one from the geometric approach above.

\clearpage
\section{Autozeroing}
\label{sec:autozero}
Autozeroing (AZ), sometimes called zero-drift or dynamic offset compensation, is such an important concept, that it must be discussed in its own right. The need for autozeroing comes from the typical behaviour of amplifiers. Every amplifier has some offset, be it small or large, and especially at high gains, this offset becomes a problem for high precision measurements. To make matters worse, this offset is not stable over time and drifts with both time and temperature. It can therefore not be calibrated out once, it must be permanently adjusted during operation, depending on environmental conditions. This procedure is called autozeroing.

There are many different ways to implement autozeroing and regarding operational amplifiers a good overview can be found in \cite{horowitz1989}. As an example, the autozero cycle for the Keithley \device{Model 2002} and the Keysight \device{3458A} Mulimeter is shown in figure \ref{fig:dmm_autozero_comparison}. Keithley uses a more complex and slower algorithm, while HP implemented a simpler, but faster algorithm. The most simple (digital) approach is to regularly switch the input from the signal to zero, take a reading, then subtract this reading from all subsequent readings until a new zero reading is taken. The other second approach adds another measurement of the reference voltage to apply a gain correction. This is done by the Keithley \device{Model 2002} and works very well to suppress gain drift in the input amplifier due to temperature changes, but increases the time between samples by another \qty{50}{\percent}. This is the reason, why on the \device{Model 2002}, there is another mode to strech the time between autozero cycles, but the price to pay is a non-uniform sampling rate. This makes post-processing a lot more complicated as most alorithms assume a (near) constant sampling rate. The Keysight \device{3458A} calculates gain corrections only during the manual autocalibration routine to maintain a higher throughput.

\begin{figure}[hb]
    \centering
    %\resizebox {0.8\textwidth} {!} {
        \import{figures/}{dmm_autozero.tex}
    %} % resizebox
    \caption{Auto-zero phases of the Keysight \device{3458A} and Keithley \device{Model 2002}.}
    \label{fig:dmm_autozero_comparison}
\end{figure}

\subsection{Offset-Nulling}
Offset-nulling is the most basic approach to autozeroing. It aims to remove the offset drift of an amplifier. Especially at high gains, the offset, which is multiplied by the gain, can be substantial. In order to explain how offset-nulling works and how it shapes the spectrum, it is best to discuss it based on an example. While this technique can also be found in many integrated circuits, it is more noticable in DMMs, because is a switchable option. Therefore, the example data set simulated is based on the parameters of the aforementioned Keysight \device{3458A} multimeter. The corner frequency and the white noise floor is modeled after the \qty{10}{\V} range of the \device{3458A} \cite{3458A_noise_floor, sampling_with_3458A} with the values given below. Do note that both references \cite{3458A_noise_floor, sampling_with_3458A} contain a typographical error. The corner frequency of the noise floor is erroneously given as \qty{0.5}{\Hz}, but should be \qty{1.5}{\Hz}. This can be seen in figure 2.35 in \cite{sampling_with_3458A}, where the noise spectral density is plotted and it was also confirmed with the author \cite{lapuh_email_corner_frequency}. The sample is generated using the Python \textit{AllanTools} library \cite{allantools}.

\begin{figure}[ht]
    \centering
    \scalebox{1}{%
        \import{figures/}{offset_nulling_definitions.tex}
    } % scalebox
    \caption{Integration sequences of the offset-nulling algorithm. Solid lines denote sampled data. Red is the input signal, green is the zero reading and blue is the dead time required for switching inputs.}
    \label{fig:dmm_autozer_offset_nulling}
\end{figure}

For this simulation, a noise-free and arbitrarily chosen \qty{10}{\V} input is assumed to be sampled by the device at a sampling rate of \qty{10}{\plc} at \qty{50}{\Hz}, the same rate discussed previously on page \pageref{sec:dead_time}. As it will be shown, the actual mean value of the input signal has no bearing on the outcome of the calculation when considering offset-nulling, but its value must be considered for other types of autozeroing as discussed in section \ref{sec:autozero_gain} and is included here only for the sake of completeness.

Figure \ref{fig:dmm_autozer_offset_nulling} shows the individual sequences of the offset-nulling algorithm. First, the source is sampled for $\tau_s = \qty{10}{\plc}$, then the input is switched to the LO terminal. While this operation is very fast and takes less than \qty{1}{\ms} \cite{article_3458A_input_mpedance}, if the instrument is synchronized to the line frequency the zero measurement will nontheless be delayed until the next zero crossing, hence the dead-time $\theta = \qty{1}{\plc}$. Finally, the zero reference is measured for another $\tau_r = \qty{10}{\plc}$ and then the instrument switches back to the HI terminal.

The data is simulated in the following way. First, two sets of noise data are generated, a white noise spectrum with a noise spectral density of \qty[power-half-as-sqrt, per-mode=symbol]{165}{\nV \Hz\tothe{-0.5}} and a flicker noise spectrum with an intesity scaled to result in a final spectrum with a corner frequency of \qty{1.5}{\Hz}. The required flicker noise intensity is calculated using equation \ref{eqn:corner_frequency}. To get a good low frequency estimate, $2^{20} \approx 10^{6}$ values were generated. Finally, the two noise data sets are summed with the noise-free input source to give the final result. Other effects, such as power-line hum are neglected in this simple simulation, because it would needlessly overcomplicate the example and limit the educational value. The same goes for higher order random-walk $f^{-2}$ noise components, which can be introduced by temperature fluctions and other environmental effects and would be present in a real measurement.

\begin{figure}[ht]
    \centering
    %% Creator: Matplotlib, PGF backend
%%
%% To include the figure in your LaTeX document, write
%%   \input{<filename>.pgf}
%%
%% Make sure the required packages are loaded in your preamble
%%   \usepackage{pgf}
%%
%% Also ensure that all the required font packages are loaded; for instance,
%% the lmodern package is sometimes necessary when using math font.
%%   \usepackage{lmodern}
%%
%% Figures using additional raster images can only be included by \input if
%% they are in the same directory as the main LaTeX file. For loading figures
%% from other directories you can use the `import` package
%%   \usepackage{import}
%%
%% and then include the figures with
%%   \import{<path to file>}{<filename>.pgf}
%%
%% Matplotlib used the following preamble
%%   \usepackage{siunitx}
%%   \usepackage{fontspec}
%%   \makeatletter\@ifpackageloaded{underscore}{}{\usepackage[strings]{underscore}}\makeatother
%%
\begingroup%
\makeatletter%
\begin{pgfpicture}%
\pgfpathrectangle{\pgfpointorigin}{\pgfqpoint{4.060000in}{2.510000in}}%
\pgfusepath{use as bounding box, clip}%
\begin{pgfscope}%
\pgfsetbuttcap%
\pgfsetmiterjoin%
\definecolor{currentfill}{rgb}{1.000000,1.000000,1.000000}%
\pgfsetfillcolor{currentfill}%
\pgfsetlinewidth{0.000000pt}%
\definecolor{currentstroke}{rgb}{1.000000,1.000000,1.000000}%
\pgfsetstrokecolor{currentstroke}%
\pgfsetdash{}{0pt}%
\pgfpathmoveto{\pgfqpoint{0.000000in}{0.000000in}}%
\pgfpathlineto{\pgfqpoint{4.060000in}{0.000000in}}%
\pgfpathlineto{\pgfqpoint{4.060000in}{2.510000in}}%
\pgfpathlineto{\pgfqpoint{0.000000in}{2.510000in}}%
\pgfpathlineto{\pgfqpoint{0.000000in}{0.000000in}}%
\pgfpathclose%
\pgfusepath{fill}%
\end{pgfscope}%
\begin{pgfscope}%
\pgfsetbuttcap%
\pgfsetmiterjoin%
\definecolor{currentfill}{rgb}{1.000000,1.000000,1.000000}%
\pgfsetfillcolor{currentfill}%
\pgfsetlinewidth{0.000000pt}%
\definecolor{currentstroke}{rgb}{0.000000,0.000000,0.000000}%
\pgfsetstrokecolor{currentstroke}%
\pgfsetstrokeopacity{0.000000}%
\pgfsetdash{}{0pt}%
\pgfpathmoveto{\pgfqpoint{0.471687in}{0.416447in}}%
\pgfpathlineto{\pgfqpoint{3.935174in}{0.416447in}}%
\pgfpathlineto{\pgfqpoint{3.935174in}{2.336783in}}%
\pgfpathlineto{\pgfqpoint{0.471687in}{2.336783in}}%
\pgfpathlineto{\pgfqpoint{0.471687in}{0.416447in}}%
\pgfpathclose%
\pgfusepath{fill}%
\end{pgfscope}%
\begin{pgfscope}%
\pgfpathrectangle{\pgfqpoint{0.471687in}{0.416447in}}{\pgfqpoint{3.463487in}{1.920336in}}%
\pgfusepath{clip}%
\pgfsetrectcap%
\pgfsetroundjoin%
\pgfsetlinewidth{0.803000pt}%
\definecolor{currentstroke}{rgb}{0.450000,0.450000,0.450000}%
\pgfsetstrokecolor{currentstroke}%
\pgfsetdash{}{0pt}%
\pgfpathmoveto{\pgfqpoint{0.629119in}{0.416447in}}%
\pgfpathlineto{\pgfqpoint{0.629119in}{2.336783in}}%
\pgfusepath{stroke}%
\end{pgfscope}%
\begin{pgfscope}%
\pgfsetbuttcap%
\pgfsetroundjoin%
\definecolor{currentfill}{rgb}{0.000000,0.000000,0.000000}%
\pgfsetfillcolor{currentfill}%
\pgfsetlinewidth{0.803000pt}%
\definecolor{currentstroke}{rgb}{0.000000,0.000000,0.000000}%
\pgfsetstrokecolor{currentstroke}%
\pgfsetdash{}{0pt}%
\pgfsys@defobject{currentmarker}{\pgfqpoint{0.000000in}{-0.048611in}}{\pgfqpoint{0.000000in}{0.000000in}}{%
\pgfpathmoveto{\pgfqpoint{0.000000in}{0.000000in}}%
\pgfpathlineto{\pgfqpoint{0.000000in}{-0.048611in}}%
\pgfusepath{stroke,fill}%
}%
\begin{pgfscope}%
\pgfsys@transformshift{0.629119in}{0.416447in}%
\pgfsys@useobject{currentmarker}{}%
\end{pgfscope}%
\end{pgfscope}%
\begin{pgfscope}%
\definecolor{textcolor}{rgb}{0.000000,0.000000,0.000000}%
\pgfsetstrokecolor{textcolor}%
\pgfsetfillcolor{textcolor}%
\pgftext[x=0.629119in,y=0.319225in,,top]{\color{textcolor}\rmfamily\fontsize{8.000000}{9.600000}\selectfont \(\displaystyle {0.00}\)}%
\end{pgfscope}%
\begin{pgfscope}%
\pgfpathrectangle{\pgfqpoint{0.471687in}{0.416447in}}{\pgfqpoint{3.463487in}{1.920336in}}%
\pgfusepath{clip}%
\pgfsetrectcap%
\pgfsetroundjoin%
\pgfsetlinewidth{0.803000pt}%
\definecolor{currentstroke}{rgb}{0.450000,0.450000,0.450000}%
\pgfsetstrokecolor{currentstroke}%
\pgfsetdash{}{0pt}%
\pgfpathmoveto{\pgfqpoint{1.098300in}{0.416447in}}%
\pgfpathlineto{\pgfqpoint{1.098300in}{2.336783in}}%
\pgfusepath{stroke}%
\end{pgfscope}%
\begin{pgfscope}%
\pgfsetbuttcap%
\pgfsetroundjoin%
\definecolor{currentfill}{rgb}{0.000000,0.000000,0.000000}%
\pgfsetfillcolor{currentfill}%
\pgfsetlinewidth{0.803000pt}%
\definecolor{currentstroke}{rgb}{0.000000,0.000000,0.000000}%
\pgfsetstrokecolor{currentstroke}%
\pgfsetdash{}{0pt}%
\pgfsys@defobject{currentmarker}{\pgfqpoint{0.000000in}{-0.048611in}}{\pgfqpoint{0.000000in}{0.000000in}}{%
\pgfpathmoveto{\pgfqpoint{0.000000in}{0.000000in}}%
\pgfpathlineto{\pgfqpoint{0.000000in}{-0.048611in}}%
\pgfusepath{stroke,fill}%
}%
\begin{pgfscope}%
\pgfsys@transformshift{1.098300in}{0.416447in}%
\pgfsys@useobject{currentmarker}{}%
\end{pgfscope}%
\end{pgfscope}%
\begin{pgfscope}%
\definecolor{textcolor}{rgb}{0.000000,0.000000,0.000000}%
\pgfsetstrokecolor{textcolor}%
\pgfsetfillcolor{textcolor}%
\pgftext[x=1.098300in,y=0.319225in,,top]{\color{textcolor}\rmfamily\fontsize{8.000000}{9.600000}\selectfont \(\displaystyle {0.25}\)}%
\end{pgfscope}%
\begin{pgfscope}%
\pgfpathrectangle{\pgfqpoint{0.471687in}{0.416447in}}{\pgfqpoint{3.463487in}{1.920336in}}%
\pgfusepath{clip}%
\pgfsetrectcap%
\pgfsetroundjoin%
\pgfsetlinewidth{0.803000pt}%
\definecolor{currentstroke}{rgb}{0.450000,0.450000,0.450000}%
\pgfsetstrokecolor{currentstroke}%
\pgfsetdash{}{0pt}%
\pgfpathmoveto{\pgfqpoint{1.567482in}{0.416447in}}%
\pgfpathlineto{\pgfqpoint{1.567482in}{2.336783in}}%
\pgfusepath{stroke}%
\end{pgfscope}%
\begin{pgfscope}%
\pgfsetbuttcap%
\pgfsetroundjoin%
\definecolor{currentfill}{rgb}{0.000000,0.000000,0.000000}%
\pgfsetfillcolor{currentfill}%
\pgfsetlinewidth{0.803000pt}%
\definecolor{currentstroke}{rgb}{0.000000,0.000000,0.000000}%
\pgfsetstrokecolor{currentstroke}%
\pgfsetdash{}{0pt}%
\pgfsys@defobject{currentmarker}{\pgfqpoint{0.000000in}{-0.048611in}}{\pgfqpoint{0.000000in}{0.000000in}}{%
\pgfpathmoveto{\pgfqpoint{0.000000in}{0.000000in}}%
\pgfpathlineto{\pgfqpoint{0.000000in}{-0.048611in}}%
\pgfusepath{stroke,fill}%
}%
\begin{pgfscope}%
\pgfsys@transformshift{1.567482in}{0.416447in}%
\pgfsys@useobject{currentmarker}{}%
\end{pgfscope}%
\end{pgfscope}%
\begin{pgfscope}%
\definecolor{textcolor}{rgb}{0.000000,0.000000,0.000000}%
\pgfsetstrokecolor{textcolor}%
\pgfsetfillcolor{textcolor}%
\pgftext[x=1.567482in,y=0.319225in,,top]{\color{textcolor}\rmfamily\fontsize{8.000000}{9.600000}\selectfont \(\displaystyle {0.50}\)}%
\end{pgfscope}%
\begin{pgfscope}%
\pgfpathrectangle{\pgfqpoint{0.471687in}{0.416447in}}{\pgfqpoint{3.463487in}{1.920336in}}%
\pgfusepath{clip}%
\pgfsetrectcap%
\pgfsetroundjoin%
\pgfsetlinewidth{0.803000pt}%
\definecolor{currentstroke}{rgb}{0.450000,0.450000,0.450000}%
\pgfsetstrokecolor{currentstroke}%
\pgfsetdash{}{0pt}%
\pgfpathmoveto{\pgfqpoint{2.036664in}{0.416447in}}%
\pgfpathlineto{\pgfqpoint{2.036664in}{2.336783in}}%
\pgfusepath{stroke}%
\end{pgfscope}%
\begin{pgfscope}%
\pgfsetbuttcap%
\pgfsetroundjoin%
\definecolor{currentfill}{rgb}{0.000000,0.000000,0.000000}%
\pgfsetfillcolor{currentfill}%
\pgfsetlinewidth{0.803000pt}%
\definecolor{currentstroke}{rgb}{0.000000,0.000000,0.000000}%
\pgfsetstrokecolor{currentstroke}%
\pgfsetdash{}{0pt}%
\pgfsys@defobject{currentmarker}{\pgfqpoint{0.000000in}{-0.048611in}}{\pgfqpoint{0.000000in}{0.000000in}}{%
\pgfpathmoveto{\pgfqpoint{0.000000in}{0.000000in}}%
\pgfpathlineto{\pgfqpoint{0.000000in}{-0.048611in}}%
\pgfusepath{stroke,fill}%
}%
\begin{pgfscope}%
\pgfsys@transformshift{2.036664in}{0.416447in}%
\pgfsys@useobject{currentmarker}{}%
\end{pgfscope}%
\end{pgfscope}%
\begin{pgfscope}%
\definecolor{textcolor}{rgb}{0.000000,0.000000,0.000000}%
\pgfsetstrokecolor{textcolor}%
\pgfsetfillcolor{textcolor}%
\pgftext[x=2.036664in,y=0.319225in,,top]{\color{textcolor}\rmfamily\fontsize{8.000000}{9.600000}\selectfont \(\displaystyle {0.75}\)}%
\end{pgfscope}%
\begin{pgfscope}%
\pgfpathrectangle{\pgfqpoint{0.471687in}{0.416447in}}{\pgfqpoint{3.463487in}{1.920336in}}%
\pgfusepath{clip}%
\pgfsetrectcap%
\pgfsetroundjoin%
\pgfsetlinewidth{0.803000pt}%
\definecolor{currentstroke}{rgb}{0.450000,0.450000,0.450000}%
\pgfsetstrokecolor{currentstroke}%
\pgfsetdash{}{0pt}%
\pgfpathmoveto{\pgfqpoint{2.505845in}{0.416447in}}%
\pgfpathlineto{\pgfqpoint{2.505845in}{2.336783in}}%
\pgfusepath{stroke}%
\end{pgfscope}%
\begin{pgfscope}%
\pgfsetbuttcap%
\pgfsetroundjoin%
\definecolor{currentfill}{rgb}{0.000000,0.000000,0.000000}%
\pgfsetfillcolor{currentfill}%
\pgfsetlinewidth{0.803000pt}%
\definecolor{currentstroke}{rgb}{0.000000,0.000000,0.000000}%
\pgfsetstrokecolor{currentstroke}%
\pgfsetdash{}{0pt}%
\pgfsys@defobject{currentmarker}{\pgfqpoint{0.000000in}{-0.048611in}}{\pgfqpoint{0.000000in}{0.000000in}}{%
\pgfpathmoveto{\pgfqpoint{0.000000in}{0.000000in}}%
\pgfpathlineto{\pgfqpoint{0.000000in}{-0.048611in}}%
\pgfusepath{stroke,fill}%
}%
\begin{pgfscope}%
\pgfsys@transformshift{2.505845in}{0.416447in}%
\pgfsys@useobject{currentmarker}{}%
\end{pgfscope}%
\end{pgfscope}%
\begin{pgfscope}%
\definecolor{textcolor}{rgb}{0.000000,0.000000,0.000000}%
\pgfsetstrokecolor{textcolor}%
\pgfsetfillcolor{textcolor}%
\pgftext[x=2.505845in,y=0.319225in,,top]{\color{textcolor}\rmfamily\fontsize{8.000000}{9.600000}\selectfont \(\displaystyle {1.00}\)}%
\end{pgfscope}%
\begin{pgfscope}%
\pgfpathrectangle{\pgfqpoint{0.471687in}{0.416447in}}{\pgfqpoint{3.463487in}{1.920336in}}%
\pgfusepath{clip}%
\pgfsetrectcap%
\pgfsetroundjoin%
\pgfsetlinewidth{0.803000pt}%
\definecolor{currentstroke}{rgb}{0.450000,0.450000,0.450000}%
\pgfsetstrokecolor{currentstroke}%
\pgfsetdash{}{0pt}%
\pgfpathmoveto{\pgfqpoint{2.975027in}{0.416447in}}%
\pgfpathlineto{\pgfqpoint{2.975027in}{2.336783in}}%
\pgfusepath{stroke}%
\end{pgfscope}%
\begin{pgfscope}%
\pgfsetbuttcap%
\pgfsetroundjoin%
\definecolor{currentfill}{rgb}{0.000000,0.000000,0.000000}%
\pgfsetfillcolor{currentfill}%
\pgfsetlinewidth{0.803000pt}%
\definecolor{currentstroke}{rgb}{0.000000,0.000000,0.000000}%
\pgfsetstrokecolor{currentstroke}%
\pgfsetdash{}{0pt}%
\pgfsys@defobject{currentmarker}{\pgfqpoint{0.000000in}{-0.048611in}}{\pgfqpoint{0.000000in}{0.000000in}}{%
\pgfpathmoveto{\pgfqpoint{0.000000in}{0.000000in}}%
\pgfpathlineto{\pgfqpoint{0.000000in}{-0.048611in}}%
\pgfusepath{stroke,fill}%
}%
\begin{pgfscope}%
\pgfsys@transformshift{2.975027in}{0.416447in}%
\pgfsys@useobject{currentmarker}{}%
\end{pgfscope}%
\end{pgfscope}%
\begin{pgfscope}%
\definecolor{textcolor}{rgb}{0.000000,0.000000,0.000000}%
\pgfsetstrokecolor{textcolor}%
\pgfsetfillcolor{textcolor}%
\pgftext[x=2.975027in,y=0.319225in,,top]{\color{textcolor}\rmfamily\fontsize{8.000000}{9.600000}\selectfont \(\displaystyle {1.25}\)}%
\end{pgfscope}%
\begin{pgfscope}%
\pgfpathrectangle{\pgfqpoint{0.471687in}{0.416447in}}{\pgfqpoint{3.463487in}{1.920336in}}%
\pgfusepath{clip}%
\pgfsetrectcap%
\pgfsetroundjoin%
\pgfsetlinewidth{0.803000pt}%
\definecolor{currentstroke}{rgb}{0.450000,0.450000,0.450000}%
\pgfsetstrokecolor{currentstroke}%
\pgfsetdash{}{0pt}%
\pgfpathmoveto{\pgfqpoint{3.444209in}{0.416447in}}%
\pgfpathlineto{\pgfqpoint{3.444209in}{2.336783in}}%
\pgfusepath{stroke}%
\end{pgfscope}%
\begin{pgfscope}%
\pgfsetbuttcap%
\pgfsetroundjoin%
\definecolor{currentfill}{rgb}{0.000000,0.000000,0.000000}%
\pgfsetfillcolor{currentfill}%
\pgfsetlinewidth{0.803000pt}%
\definecolor{currentstroke}{rgb}{0.000000,0.000000,0.000000}%
\pgfsetstrokecolor{currentstroke}%
\pgfsetdash{}{0pt}%
\pgfsys@defobject{currentmarker}{\pgfqpoint{0.000000in}{-0.048611in}}{\pgfqpoint{0.000000in}{0.000000in}}{%
\pgfpathmoveto{\pgfqpoint{0.000000in}{0.000000in}}%
\pgfpathlineto{\pgfqpoint{0.000000in}{-0.048611in}}%
\pgfusepath{stroke,fill}%
}%
\begin{pgfscope}%
\pgfsys@transformshift{3.444209in}{0.416447in}%
\pgfsys@useobject{currentmarker}{}%
\end{pgfscope}%
\end{pgfscope}%
\begin{pgfscope}%
\definecolor{textcolor}{rgb}{0.000000,0.000000,0.000000}%
\pgfsetstrokecolor{textcolor}%
\pgfsetfillcolor{textcolor}%
\pgftext[x=3.444209in,y=0.319225in,,top]{\color{textcolor}\rmfamily\fontsize{8.000000}{9.600000}\selectfont \(\displaystyle {1.50}\)}%
\end{pgfscope}%
\begin{pgfscope}%
\pgfpathrectangle{\pgfqpoint{0.471687in}{0.416447in}}{\pgfqpoint{3.463487in}{1.920336in}}%
\pgfusepath{clip}%
\pgfsetrectcap%
\pgfsetroundjoin%
\pgfsetlinewidth{0.803000pt}%
\definecolor{currentstroke}{rgb}{0.450000,0.450000,0.450000}%
\pgfsetstrokecolor{currentstroke}%
\pgfsetdash{}{0pt}%
\pgfpathmoveto{\pgfqpoint{3.913390in}{0.416447in}}%
\pgfpathlineto{\pgfqpoint{3.913390in}{2.336783in}}%
\pgfusepath{stroke}%
\end{pgfscope}%
\begin{pgfscope}%
\pgfsetbuttcap%
\pgfsetroundjoin%
\definecolor{currentfill}{rgb}{0.000000,0.000000,0.000000}%
\pgfsetfillcolor{currentfill}%
\pgfsetlinewidth{0.803000pt}%
\definecolor{currentstroke}{rgb}{0.000000,0.000000,0.000000}%
\pgfsetstrokecolor{currentstroke}%
\pgfsetdash{}{0pt}%
\pgfsys@defobject{currentmarker}{\pgfqpoint{0.000000in}{-0.048611in}}{\pgfqpoint{0.000000in}{0.000000in}}{%
\pgfpathmoveto{\pgfqpoint{0.000000in}{0.000000in}}%
\pgfpathlineto{\pgfqpoint{0.000000in}{-0.048611in}}%
\pgfusepath{stroke,fill}%
}%
\begin{pgfscope}%
\pgfsys@transformshift{3.913390in}{0.416447in}%
\pgfsys@useobject{currentmarker}{}%
\end{pgfscope}%
\end{pgfscope}%
\begin{pgfscope}%
\definecolor{textcolor}{rgb}{0.000000,0.000000,0.000000}%
\pgfsetstrokecolor{textcolor}%
\pgfsetfillcolor{textcolor}%
\pgftext[x=3.913390in,y=0.319225in,,top]{\color{textcolor}\rmfamily\fontsize{8.000000}{9.600000}\selectfont \(\displaystyle {1.75}\)}%
\end{pgfscope}%
\begin{pgfscope}%
\definecolor{textcolor}{rgb}{0.000000,0.000000,0.000000}%
\pgfsetstrokecolor{textcolor}%
\pgfsetfillcolor{textcolor}%
\pgftext[x=2.203431in,y=0.165003in,,top]{\color{textcolor}\rmfamily\fontsize{10.000000}{12.000000}\selectfont Time in \(\displaystyle \unit{\second}\)}%
\end{pgfscope}%
\begin{pgfscope}%
\definecolor{textcolor}{rgb}{0.000000,0.000000,0.000000}%
\pgfsetstrokecolor{textcolor}%
\pgfsetfillcolor{textcolor}%
\pgftext[x=3.935174in,y=0.178892in,right,top]{\color{textcolor}\rmfamily\fontsize{8.000000}{9.600000}\selectfont \(\displaystyle \times{10^{6}}{}\)}%
\end{pgfscope}%
\begin{pgfscope}%
\pgfpathrectangle{\pgfqpoint{0.471687in}{0.416447in}}{\pgfqpoint{3.463487in}{1.920336in}}%
\pgfusepath{clip}%
\pgfsetrectcap%
\pgfsetroundjoin%
\pgfsetlinewidth{0.803000pt}%
\definecolor{currentstroke}{rgb}{0.450000,0.450000,0.450000}%
\pgfsetstrokecolor{currentstroke}%
\pgfsetdash{}{0pt}%
\pgfpathmoveto{\pgfqpoint{0.471687in}{0.469719in}}%
\pgfpathlineto{\pgfqpoint{3.935174in}{0.469719in}}%
\pgfusepath{stroke}%
\end{pgfscope}%
\begin{pgfscope}%
\pgfsetbuttcap%
\pgfsetroundjoin%
\definecolor{currentfill}{rgb}{0.000000,0.000000,0.000000}%
\pgfsetfillcolor{currentfill}%
\pgfsetlinewidth{0.803000pt}%
\definecolor{currentstroke}{rgb}{0.000000,0.000000,0.000000}%
\pgfsetstrokecolor{currentstroke}%
\pgfsetdash{}{0pt}%
\pgfsys@defobject{currentmarker}{\pgfqpoint{-0.048611in}{0.000000in}}{\pgfqpoint{-0.000000in}{0.000000in}}{%
\pgfpathmoveto{\pgfqpoint{-0.000000in}{0.000000in}}%
\pgfpathlineto{\pgfqpoint{-0.048611in}{0.000000in}}%
\pgfusepath{stroke,fill}%
}%
\begin{pgfscope}%
\pgfsys@transformshift{0.471687in}{0.469719in}%
\pgfsys@useobject{currentmarker}{}%
\end{pgfscope}%
\end{pgfscope}%
\begin{pgfscope}%
\definecolor{textcolor}{rgb}{0.000000,0.000000,0.000000}%
\pgfsetstrokecolor{textcolor}%
\pgfsetfillcolor{textcolor}%
\pgftext[x=0.223614in, y=0.431163in, left, base]{\color{textcolor}\rmfamily\fontsize{8.000000}{9.600000}\selectfont \(\displaystyle {\ensuremath{-}3}\)}%
\end{pgfscope}%
\begin{pgfscope}%
\pgfpathrectangle{\pgfqpoint{0.471687in}{0.416447in}}{\pgfqpoint{3.463487in}{1.920336in}}%
\pgfusepath{clip}%
\pgfsetrectcap%
\pgfsetroundjoin%
\pgfsetlinewidth{0.803000pt}%
\definecolor{currentstroke}{rgb}{0.450000,0.450000,0.450000}%
\pgfsetstrokecolor{currentstroke}%
\pgfsetdash{}{0pt}%
\pgfpathmoveto{\pgfqpoint{0.471687in}{0.768082in}}%
\pgfpathlineto{\pgfqpoint{3.935174in}{0.768082in}}%
\pgfusepath{stroke}%
\end{pgfscope}%
\begin{pgfscope}%
\pgfsetbuttcap%
\pgfsetroundjoin%
\definecolor{currentfill}{rgb}{0.000000,0.000000,0.000000}%
\pgfsetfillcolor{currentfill}%
\pgfsetlinewidth{0.803000pt}%
\definecolor{currentstroke}{rgb}{0.000000,0.000000,0.000000}%
\pgfsetstrokecolor{currentstroke}%
\pgfsetdash{}{0pt}%
\pgfsys@defobject{currentmarker}{\pgfqpoint{-0.048611in}{0.000000in}}{\pgfqpoint{-0.000000in}{0.000000in}}{%
\pgfpathmoveto{\pgfqpoint{-0.000000in}{0.000000in}}%
\pgfpathlineto{\pgfqpoint{-0.048611in}{0.000000in}}%
\pgfusepath{stroke,fill}%
}%
\begin{pgfscope}%
\pgfsys@transformshift{0.471687in}{0.768082in}%
\pgfsys@useobject{currentmarker}{}%
\end{pgfscope}%
\end{pgfscope}%
\begin{pgfscope}%
\definecolor{textcolor}{rgb}{0.000000,0.000000,0.000000}%
\pgfsetstrokecolor{textcolor}%
\pgfsetfillcolor{textcolor}%
\pgftext[x=0.223614in, y=0.729527in, left, base]{\color{textcolor}\rmfamily\fontsize{8.000000}{9.600000}\selectfont \(\displaystyle {\ensuremath{-}2}\)}%
\end{pgfscope}%
\begin{pgfscope}%
\pgfpathrectangle{\pgfqpoint{0.471687in}{0.416447in}}{\pgfqpoint{3.463487in}{1.920336in}}%
\pgfusepath{clip}%
\pgfsetrectcap%
\pgfsetroundjoin%
\pgfsetlinewidth{0.803000pt}%
\definecolor{currentstroke}{rgb}{0.450000,0.450000,0.450000}%
\pgfsetstrokecolor{currentstroke}%
\pgfsetdash{}{0pt}%
\pgfpathmoveto{\pgfqpoint{0.471687in}{1.066446in}}%
\pgfpathlineto{\pgfqpoint{3.935174in}{1.066446in}}%
\pgfusepath{stroke}%
\end{pgfscope}%
\begin{pgfscope}%
\pgfsetbuttcap%
\pgfsetroundjoin%
\definecolor{currentfill}{rgb}{0.000000,0.000000,0.000000}%
\pgfsetfillcolor{currentfill}%
\pgfsetlinewidth{0.803000pt}%
\definecolor{currentstroke}{rgb}{0.000000,0.000000,0.000000}%
\pgfsetstrokecolor{currentstroke}%
\pgfsetdash{}{0pt}%
\pgfsys@defobject{currentmarker}{\pgfqpoint{-0.048611in}{0.000000in}}{\pgfqpoint{-0.000000in}{0.000000in}}{%
\pgfpathmoveto{\pgfqpoint{-0.000000in}{0.000000in}}%
\pgfpathlineto{\pgfqpoint{-0.048611in}{0.000000in}}%
\pgfusepath{stroke,fill}%
}%
\begin{pgfscope}%
\pgfsys@transformshift{0.471687in}{1.066446in}%
\pgfsys@useobject{currentmarker}{}%
\end{pgfscope}%
\end{pgfscope}%
\begin{pgfscope}%
\definecolor{textcolor}{rgb}{0.000000,0.000000,0.000000}%
\pgfsetstrokecolor{textcolor}%
\pgfsetfillcolor{textcolor}%
\pgftext[x=0.223614in, y=1.027891in, left, base]{\color{textcolor}\rmfamily\fontsize{8.000000}{9.600000}\selectfont \(\displaystyle {\ensuremath{-}1}\)}%
\end{pgfscope}%
\begin{pgfscope}%
\pgfpathrectangle{\pgfqpoint{0.471687in}{0.416447in}}{\pgfqpoint{3.463487in}{1.920336in}}%
\pgfusepath{clip}%
\pgfsetrectcap%
\pgfsetroundjoin%
\pgfsetlinewidth{0.803000pt}%
\definecolor{currentstroke}{rgb}{0.450000,0.450000,0.450000}%
\pgfsetstrokecolor{currentstroke}%
\pgfsetdash{}{0pt}%
\pgfpathmoveto{\pgfqpoint{0.471687in}{1.364810in}}%
\pgfpathlineto{\pgfqpoint{3.935174in}{1.364810in}}%
\pgfusepath{stroke}%
\end{pgfscope}%
\begin{pgfscope}%
\pgfsetbuttcap%
\pgfsetroundjoin%
\definecolor{currentfill}{rgb}{0.000000,0.000000,0.000000}%
\pgfsetfillcolor{currentfill}%
\pgfsetlinewidth{0.803000pt}%
\definecolor{currentstroke}{rgb}{0.000000,0.000000,0.000000}%
\pgfsetstrokecolor{currentstroke}%
\pgfsetdash{}{0pt}%
\pgfsys@defobject{currentmarker}{\pgfqpoint{-0.048611in}{0.000000in}}{\pgfqpoint{-0.000000in}{0.000000in}}{%
\pgfpathmoveto{\pgfqpoint{-0.000000in}{0.000000in}}%
\pgfpathlineto{\pgfqpoint{-0.048611in}{0.000000in}}%
\pgfusepath{stroke,fill}%
}%
\begin{pgfscope}%
\pgfsys@transformshift{0.471687in}{1.364810in}%
\pgfsys@useobject{currentmarker}{}%
\end{pgfscope}%
\end{pgfscope}%
\begin{pgfscope}%
\definecolor{textcolor}{rgb}{0.000000,0.000000,0.000000}%
\pgfsetstrokecolor{textcolor}%
\pgfsetfillcolor{textcolor}%
\pgftext[x=0.315437in, y=1.326254in, left, base]{\color{textcolor}\rmfamily\fontsize{8.000000}{9.600000}\selectfont \(\displaystyle {0}\)}%
\end{pgfscope}%
\begin{pgfscope}%
\pgfpathrectangle{\pgfqpoint{0.471687in}{0.416447in}}{\pgfqpoint{3.463487in}{1.920336in}}%
\pgfusepath{clip}%
\pgfsetrectcap%
\pgfsetroundjoin%
\pgfsetlinewidth{0.803000pt}%
\definecolor{currentstroke}{rgb}{0.450000,0.450000,0.450000}%
\pgfsetstrokecolor{currentstroke}%
\pgfsetdash{}{0pt}%
\pgfpathmoveto{\pgfqpoint{0.471687in}{1.663173in}}%
\pgfpathlineto{\pgfqpoint{3.935174in}{1.663173in}}%
\pgfusepath{stroke}%
\end{pgfscope}%
\begin{pgfscope}%
\pgfsetbuttcap%
\pgfsetroundjoin%
\definecolor{currentfill}{rgb}{0.000000,0.000000,0.000000}%
\pgfsetfillcolor{currentfill}%
\pgfsetlinewidth{0.803000pt}%
\definecolor{currentstroke}{rgb}{0.000000,0.000000,0.000000}%
\pgfsetstrokecolor{currentstroke}%
\pgfsetdash{}{0pt}%
\pgfsys@defobject{currentmarker}{\pgfqpoint{-0.048611in}{0.000000in}}{\pgfqpoint{-0.000000in}{0.000000in}}{%
\pgfpathmoveto{\pgfqpoint{-0.000000in}{0.000000in}}%
\pgfpathlineto{\pgfqpoint{-0.048611in}{0.000000in}}%
\pgfusepath{stroke,fill}%
}%
\begin{pgfscope}%
\pgfsys@transformshift{0.471687in}{1.663173in}%
\pgfsys@useobject{currentmarker}{}%
\end{pgfscope}%
\end{pgfscope}%
\begin{pgfscope}%
\definecolor{textcolor}{rgb}{0.000000,0.000000,0.000000}%
\pgfsetstrokecolor{textcolor}%
\pgfsetfillcolor{textcolor}%
\pgftext[x=0.315437in, y=1.624618in, left, base]{\color{textcolor}\rmfamily\fontsize{8.000000}{9.600000}\selectfont \(\displaystyle {1}\)}%
\end{pgfscope}%
\begin{pgfscope}%
\pgfpathrectangle{\pgfqpoint{0.471687in}{0.416447in}}{\pgfqpoint{3.463487in}{1.920336in}}%
\pgfusepath{clip}%
\pgfsetrectcap%
\pgfsetroundjoin%
\pgfsetlinewidth{0.803000pt}%
\definecolor{currentstroke}{rgb}{0.450000,0.450000,0.450000}%
\pgfsetstrokecolor{currentstroke}%
\pgfsetdash{}{0pt}%
\pgfpathmoveto{\pgfqpoint{0.471687in}{1.961537in}}%
\pgfpathlineto{\pgfqpoint{3.935174in}{1.961537in}}%
\pgfusepath{stroke}%
\end{pgfscope}%
\begin{pgfscope}%
\pgfsetbuttcap%
\pgfsetroundjoin%
\definecolor{currentfill}{rgb}{0.000000,0.000000,0.000000}%
\pgfsetfillcolor{currentfill}%
\pgfsetlinewidth{0.803000pt}%
\definecolor{currentstroke}{rgb}{0.000000,0.000000,0.000000}%
\pgfsetstrokecolor{currentstroke}%
\pgfsetdash{}{0pt}%
\pgfsys@defobject{currentmarker}{\pgfqpoint{-0.048611in}{0.000000in}}{\pgfqpoint{-0.000000in}{0.000000in}}{%
\pgfpathmoveto{\pgfqpoint{-0.000000in}{0.000000in}}%
\pgfpathlineto{\pgfqpoint{-0.048611in}{0.000000in}}%
\pgfusepath{stroke,fill}%
}%
\begin{pgfscope}%
\pgfsys@transformshift{0.471687in}{1.961537in}%
\pgfsys@useobject{currentmarker}{}%
\end{pgfscope}%
\end{pgfscope}%
\begin{pgfscope}%
\definecolor{textcolor}{rgb}{0.000000,0.000000,0.000000}%
\pgfsetstrokecolor{textcolor}%
\pgfsetfillcolor{textcolor}%
\pgftext[x=0.315437in, y=1.922982in, left, base]{\color{textcolor}\rmfamily\fontsize{8.000000}{9.600000}\selectfont \(\displaystyle {2}\)}%
\end{pgfscope}%
\begin{pgfscope}%
\pgfpathrectangle{\pgfqpoint{0.471687in}{0.416447in}}{\pgfqpoint{3.463487in}{1.920336in}}%
\pgfusepath{clip}%
\pgfsetrectcap%
\pgfsetroundjoin%
\pgfsetlinewidth{0.803000pt}%
\definecolor{currentstroke}{rgb}{0.450000,0.450000,0.450000}%
\pgfsetstrokecolor{currentstroke}%
\pgfsetdash{}{0pt}%
\pgfpathmoveto{\pgfqpoint{0.471687in}{2.259901in}}%
\pgfpathlineto{\pgfqpoint{3.935174in}{2.259901in}}%
\pgfusepath{stroke}%
\end{pgfscope}%
\begin{pgfscope}%
\pgfsetbuttcap%
\pgfsetroundjoin%
\definecolor{currentfill}{rgb}{0.000000,0.000000,0.000000}%
\pgfsetfillcolor{currentfill}%
\pgfsetlinewidth{0.803000pt}%
\definecolor{currentstroke}{rgb}{0.000000,0.000000,0.000000}%
\pgfsetstrokecolor{currentstroke}%
\pgfsetdash{}{0pt}%
\pgfsys@defobject{currentmarker}{\pgfqpoint{-0.048611in}{0.000000in}}{\pgfqpoint{-0.000000in}{0.000000in}}{%
\pgfpathmoveto{\pgfqpoint{-0.000000in}{0.000000in}}%
\pgfpathlineto{\pgfqpoint{-0.048611in}{0.000000in}}%
\pgfusepath{stroke,fill}%
}%
\begin{pgfscope}%
\pgfsys@transformshift{0.471687in}{2.259901in}%
\pgfsys@useobject{currentmarker}{}%
\end{pgfscope}%
\end{pgfscope}%
\begin{pgfscope}%
\definecolor{textcolor}{rgb}{0.000000,0.000000,0.000000}%
\pgfsetstrokecolor{textcolor}%
\pgfsetfillcolor{textcolor}%
\pgftext[x=0.315437in, y=2.221345in, left, base]{\color{textcolor}\rmfamily\fontsize{8.000000}{9.600000}\selectfont \(\displaystyle {3}\)}%
\end{pgfscope}%
\begin{pgfscope}%
\definecolor{textcolor}{rgb}{0.000000,0.000000,0.000000}%
\pgfsetstrokecolor{textcolor}%
\pgfsetfillcolor{textcolor}%
\pgftext[x=0.168059in,y=1.376615in,,bottom,rotate=90.000000]{\color{textcolor}\rmfamily\fontsize{10.000000}{12.000000}\selectfont Amplitude in \(\displaystyle \unit{\V}\)}%
\end{pgfscope}%
\begin{pgfscope}%
\definecolor{textcolor}{rgb}{0.000000,0.000000,0.000000}%
\pgfsetstrokecolor{textcolor}%
\pgfsetfillcolor{textcolor}%
\pgftext[x=0.471687in,y=2.378450in,left,base]{\color{textcolor}\rmfamily\fontsize{8.000000}{9.600000}\selectfont \(\displaystyle \times{10^{\ensuremath{-}6}}{+10^{1}}\)}%
\end{pgfscope}%
\begin{pgfscope}%
\pgfpathrectangle{\pgfqpoint{0.471687in}{0.416447in}}{\pgfqpoint{3.463487in}{1.920336in}}%
\pgfusepath{clip}%
\pgfsetrectcap%
\pgfsetroundjoin%
\pgfsetlinewidth{1.505625pt}%
\definecolor{currentstroke}{rgb}{0.337255,0.705882,0.913725}%
\pgfsetstrokecolor{currentstroke}%
\pgfsetdash{}{0pt}%
\pgfpathmoveto{\pgfqpoint{0.629119in}{1.313736in}}%
\pgfpathlineto{\pgfqpoint{0.629662in}{2.030033in}}%
\pgfpathlineto{\pgfqpoint{0.630726in}{0.782787in}}%
\pgfpathlineto{\pgfqpoint{0.632407in}{1.861178in}}%
\pgfpathlineto{\pgfqpoint{0.634496in}{0.836707in}}%
\pgfpathlineto{\pgfqpoint{0.635425in}{1.922822in}}%
\pgfpathlineto{\pgfqpoint{0.637006in}{1.008572in}}%
\pgfpathlineto{\pgfqpoint{0.638606in}{1.950448in}}%
\pgfpathlineto{\pgfqpoint{0.640302in}{0.902775in}}%
\pgfpathlineto{\pgfqpoint{0.641796in}{1.929426in}}%
\pgfpathlineto{\pgfqpoint{0.643304in}{1.019038in}}%
\pgfpathlineto{\pgfqpoint{0.645218in}{2.067371in}}%
\pgfpathlineto{\pgfqpoint{0.646731in}{0.939700in}}%
\pgfpathlineto{\pgfqpoint{0.648515in}{2.007468in}}%
\pgfpathlineto{\pgfqpoint{0.650369in}{0.690786in}}%
\pgfpathlineto{\pgfqpoint{0.651781in}{1.953278in}}%
\pgfpathlineto{\pgfqpoint{0.652850in}{0.913723in}}%
\pgfpathlineto{\pgfqpoint{0.654772in}{1.833213in}}%
\pgfpathlineto{\pgfqpoint{0.656095in}{0.996667in}}%
\pgfpathlineto{\pgfqpoint{0.657632in}{2.004090in}}%
\pgfpathlineto{\pgfqpoint{0.659544in}{0.787830in}}%
\pgfpathlineto{\pgfqpoint{0.660678in}{1.872153in}}%
\pgfpathlineto{\pgfqpoint{0.662248in}{0.776356in}}%
\pgfpathlineto{\pgfqpoint{0.663857in}{1.825062in}}%
\pgfpathlineto{\pgfqpoint{0.665666in}{0.869605in}}%
\pgfpathlineto{\pgfqpoint{0.666957in}{1.912805in}}%
\pgfpathlineto{\pgfqpoint{0.668846in}{0.759987in}}%
\pgfpathlineto{\pgfqpoint{0.670830in}{2.049471in}}%
\pgfpathlineto{\pgfqpoint{0.672313in}{0.664558in}}%
\pgfpathlineto{\pgfqpoint{0.673346in}{1.820120in}}%
\pgfpathlineto{\pgfqpoint{0.675104in}{0.826844in}}%
\pgfpathlineto{\pgfqpoint{0.677434in}{1.871141in}}%
\pgfpathlineto{\pgfqpoint{0.678004in}{0.745766in}}%
\pgfpathlineto{\pgfqpoint{0.679987in}{1.874275in}}%
\pgfpathlineto{\pgfqpoint{0.681997in}{0.700209in}}%
\pgfpathlineto{\pgfqpoint{0.683220in}{2.030017in}}%
\pgfpathlineto{\pgfqpoint{0.684352in}{0.869446in}}%
\pgfpathlineto{\pgfqpoint{0.685895in}{1.945112in}}%
\pgfpathlineto{\pgfqpoint{0.687496in}{0.893721in}}%
\pgfpathlineto{\pgfqpoint{0.689206in}{2.032614in}}%
\pgfpathlineto{\pgfqpoint{0.690639in}{0.953598in}}%
\pgfpathlineto{\pgfqpoint{0.692520in}{1.927976in}}%
\pgfpathlineto{\pgfqpoint{0.693872in}{0.969550in}}%
\pgfpathlineto{\pgfqpoint{0.695607in}{1.882447in}}%
\pgfpathlineto{\pgfqpoint{0.697168in}{0.707795in}}%
\pgfpathlineto{\pgfqpoint{0.698495in}{1.964045in}}%
\pgfpathlineto{\pgfqpoint{0.700102in}{0.881147in}}%
\pgfpathlineto{\pgfqpoint{0.701702in}{1.875228in}}%
\pgfpathlineto{\pgfqpoint{0.703514in}{0.853379in}}%
\pgfpathlineto{\pgfqpoint{0.704831in}{1.991519in}}%
\pgfpathlineto{\pgfqpoint{0.706386in}{0.940503in}}%
\pgfpathlineto{\pgfqpoint{0.708090in}{1.924455in}}%
\pgfpathlineto{\pgfqpoint{0.710409in}{0.841407in}}%
\pgfpathlineto{\pgfqpoint{0.711122in}{1.898912in}}%
\pgfpathlineto{\pgfqpoint{0.713121in}{0.811683in}}%
\pgfpathlineto{\pgfqpoint{0.714410in}{1.900313in}}%
\pgfpathlineto{\pgfqpoint{0.716644in}{0.956332in}}%
\pgfpathlineto{\pgfqpoint{0.717450in}{2.053094in}}%
\pgfpathlineto{\pgfqpoint{0.719499in}{0.902671in}}%
\pgfpathlineto{\pgfqpoint{0.721557in}{2.050395in}}%
\pgfpathlineto{\pgfqpoint{0.722639in}{0.780246in}}%
\pgfpathlineto{\pgfqpoint{0.723744in}{1.831673in}}%
\pgfpathlineto{\pgfqpoint{0.726411in}{0.834272in}}%
\pgfpathlineto{\pgfqpoint{0.726892in}{1.955248in}}%
\pgfpathlineto{\pgfqpoint{0.728427in}{0.981763in}}%
\pgfpathlineto{\pgfqpoint{0.730554in}{1.904818in}}%
\pgfpathlineto{\pgfqpoint{0.731819in}{0.991252in}}%
\pgfpathlineto{\pgfqpoint{0.733194in}{1.907159in}}%
\pgfpathlineto{\pgfqpoint{0.735088in}{0.900566in}}%
\pgfpathlineto{\pgfqpoint{0.736625in}{1.986058in}}%
\pgfpathlineto{\pgfqpoint{0.738353in}{0.764520in}}%
\pgfpathlineto{\pgfqpoint{0.739668in}{1.942065in}}%
\pgfpathlineto{\pgfqpoint{0.741032in}{0.803080in}}%
\pgfpathlineto{\pgfqpoint{0.742640in}{1.786283in}}%
\pgfpathlineto{\pgfqpoint{0.744498in}{0.900714in}}%
\pgfpathlineto{\pgfqpoint{0.745935in}{1.904088in}}%
\pgfpathlineto{\pgfqpoint{0.747408in}{0.693320in}}%
\pgfpathlineto{\pgfqpoint{0.749851in}{1.941743in}}%
\pgfpathlineto{\pgfqpoint{0.750539in}{0.886235in}}%
\pgfpathlineto{\pgfqpoint{0.752099in}{1.882152in}}%
\pgfpathlineto{\pgfqpoint{0.754299in}{0.875249in}}%
\pgfpathlineto{\pgfqpoint{0.756068in}{1.975772in}}%
\pgfpathlineto{\pgfqpoint{0.756843in}{0.951062in}}%
\pgfpathlineto{\pgfqpoint{0.758884in}{1.878273in}}%
\pgfpathlineto{\pgfqpoint{0.760107in}{0.800813in}}%
\pgfpathlineto{\pgfqpoint{0.761511in}{1.788562in}}%
\pgfpathlineto{\pgfqpoint{0.763231in}{0.830823in}}%
\pgfpathlineto{\pgfqpoint{0.764739in}{1.864036in}}%
\pgfpathlineto{\pgfqpoint{0.766670in}{0.771477in}}%
\pgfpathlineto{\pgfqpoint{0.768185in}{1.835404in}}%
\pgfpathlineto{\pgfqpoint{0.770682in}{0.600226in}}%
\pgfpathlineto{\pgfqpoint{0.771128in}{1.781654in}}%
\pgfpathlineto{\pgfqpoint{0.772571in}{0.861958in}}%
\pgfpathlineto{\pgfqpoint{0.774270in}{1.962688in}}%
\pgfpathlineto{\pgfqpoint{0.775797in}{0.914775in}}%
\pgfpathlineto{\pgfqpoint{0.777572in}{1.938045in}}%
\pgfpathlineto{\pgfqpoint{0.779319in}{0.845212in}}%
\pgfpathlineto{\pgfqpoint{0.781610in}{1.941898in}}%
\pgfpathlineto{\pgfqpoint{0.782240in}{0.854221in}}%
\pgfpathlineto{\pgfqpoint{0.784323in}{1.898510in}}%
\pgfpathlineto{\pgfqpoint{0.785381in}{0.729523in}}%
\pgfpathlineto{\pgfqpoint{0.786782in}{1.955807in}}%
\pgfpathlineto{\pgfqpoint{0.788896in}{0.880891in}}%
\pgfpathlineto{\pgfqpoint{0.789917in}{1.741109in}}%
\pgfpathlineto{\pgfqpoint{0.791439in}{0.946724in}}%
\pgfpathlineto{\pgfqpoint{0.793257in}{1.875185in}}%
\pgfpathlineto{\pgfqpoint{0.794822in}{0.804798in}}%
\pgfpathlineto{\pgfqpoint{0.796173in}{1.818341in}}%
\pgfpathlineto{\pgfqpoint{0.797756in}{0.800446in}}%
\pgfpathlineto{\pgfqpoint{0.799325in}{1.825880in}}%
\pgfpathlineto{\pgfqpoint{0.800901in}{0.726576in}}%
\pgfpathlineto{\pgfqpoint{0.802822in}{1.791218in}}%
\pgfpathlineto{\pgfqpoint{0.804293in}{0.702364in}}%
\pgfpathlineto{\pgfqpoint{0.806102in}{1.737292in}}%
\pgfpathlineto{\pgfqpoint{0.807739in}{0.719402in}}%
\pgfpathlineto{\pgfqpoint{0.808795in}{1.668075in}}%
\pgfpathlineto{\pgfqpoint{0.810535in}{0.655990in}}%
\pgfpathlineto{\pgfqpoint{0.812720in}{1.917325in}}%
\pgfpathlineto{\pgfqpoint{0.813543in}{0.779689in}}%
\pgfpathlineto{\pgfqpoint{0.815085in}{1.874914in}}%
\pgfpathlineto{\pgfqpoint{0.816972in}{0.836045in}}%
\pgfpathlineto{\pgfqpoint{0.818521in}{1.877102in}}%
\pgfpathlineto{\pgfqpoint{0.820695in}{0.550361in}}%
\pgfpathlineto{\pgfqpoint{0.821449in}{1.800938in}}%
\pgfpathlineto{\pgfqpoint{0.824263in}{0.544727in}}%
\pgfpathlineto{\pgfqpoint{0.824705in}{1.724606in}}%
\pgfpathlineto{\pgfqpoint{0.826129in}{0.665962in}}%
\pgfpathlineto{\pgfqpoint{0.827789in}{1.599333in}}%
\pgfpathlineto{\pgfqpoint{0.829482in}{0.628716in}}%
\pgfpathlineto{\pgfqpoint{0.830885in}{1.779081in}}%
\pgfpathlineto{\pgfqpoint{0.833073in}{0.727896in}}%
\pgfpathlineto{\pgfqpoint{0.834234in}{1.769388in}}%
\pgfpathlineto{\pgfqpoint{0.835687in}{0.696354in}}%
\pgfpathlineto{\pgfqpoint{0.837267in}{1.649467in}}%
\pgfpathlineto{\pgfqpoint{0.838990in}{0.826655in}}%
\pgfpathlineto{\pgfqpoint{0.840680in}{1.901756in}}%
\pgfpathlineto{\pgfqpoint{0.842406in}{0.779389in}}%
\pgfpathlineto{\pgfqpoint{0.843656in}{1.809251in}}%
\pgfpathlineto{\pgfqpoint{0.845070in}{0.844060in}}%
\pgfpathlineto{\pgfqpoint{0.846634in}{1.899510in}}%
\pgfpathlineto{\pgfqpoint{0.848348in}{0.871415in}}%
\pgfpathlineto{\pgfqpoint{0.850556in}{1.920712in}}%
\pgfpathlineto{\pgfqpoint{0.851918in}{0.743132in}}%
\pgfpathlineto{\pgfqpoint{0.852923in}{1.935583in}}%
\pgfpathlineto{\pgfqpoint{0.854501in}{0.847278in}}%
\pgfpathlineto{\pgfqpoint{0.856177in}{1.846931in}}%
\pgfpathlineto{\pgfqpoint{0.858686in}{0.735809in}}%
\pgfpathlineto{\pgfqpoint{0.859261in}{1.811311in}}%
\pgfpathlineto{\pgfqpoint{0.860917in}{0.677966in}}%
\pgfpathlineto{\pgfqpoint{0.862500in}{1.736532in}}%
\pgfpathlineto{\pgfqpoint{0.864186in}{0.770882in}}%
\pgfpathlineto{\pgfqpoint{0.865626in}{1.948116in}}%
\pgfpathlineto{\pgfqpoint{0.867384in}{0.698437in}}%
\pgfpathlineto{\pgfqpoint{0.868663in}{1.651586in}}%
\pgfpathlineto{\pgfqpoint{0.870245in}{0.892557in}}%
\pgfpathlineto{\pgfqpoint{0.872343in}{1.920199in}}%
\pgfpathlineto{\pgfqpoint{0.873648in}{0.680566in}}%
\pgfpathlineto{\pgfqpoint{0.875354in}{1.847336in}}%
\pgfpathlineto{\pgfqpoint{0.876637in}{0.813999in}}%
\pgfpathlineto{\pgfqpoint{0.878521in}{1.917577in}}%
\pgfpathlineto{\pgfqpoint{0.879775in}{0.893854in}}%
\pgfpathlineto{\pgfqpoint{0.882178in}{2.002268in}}%
\pgfpathlineto{\pgfqpoint{0.883082in}{0.936964in}}%
\pgfpathlineto{\pgfqpoint{0.884828in}{1.951268in}}%
\pgfpathlineto{\pgfqpoint{0.886547in}{0.863759in}}%
\pgfpathlineto{\pgfqpoint{0.887710in}{1.979088in}}%
\pgfpathlineto{\pgfqpoint{0.889167in}{0.940212in}}%
\pgfpathlineto{\pgfqpoint{0.891284in}{1.964144in}}%
\pgfpathlineto{\pgfqpoint{0.892427in}{0.890527in}}%
\pgfpathlineto{\pgfqpoint{0.894037in}{1.822359in}}%
\pgfpathlineto{\pgfqpoint{0.895793in}{0.720583in}}%
\pgfpathlineto{\pgfqpoint{0.897088in}{1.866959in}}%
\pgfpathlineto{\pgfqpoint{0.898664in}{0.835383in}}%
\pgfpathlineto{\pgfqpoint{0.900222in}{1.876274in}}%
\pgfpathlineto{\pgfqpoint{0.901778in}{1.020045in}}%
\pgfpathlineto{\pgfqpoint{0.904701in}{2.030014in}}%
\pgfpathlineto{\pgfqpoint{0.905332in}{1.055922in}}%
\pgfpathlineto{\pgfqpoint{0.906680in}{1.914905in}}%
\pgfpathlineto{\pgfqpoint{0.908430in}{0.963881in}}%
\pgfpathlineto{\pgfqpoint{0.909736in}{1.930045in}}%
\pgfpathlineto{\pgfqpoint{0.911214in}{0.918010in}}%
\pgfpathlineto{\pgfqpoint{0.912840in}{1.943757in}}%
\pgfpathlineto{\pgfqpoint{0.914554in}{0.954247in}}%
\pgfpathlineto{\pgfqpoint{0.916868in}{2.001914in}}%
\pgfpathlineto{\pgfqpoint{0.917573in}{1.023125in}}%
\pgfpathlineto{\pgfqpoint{0.919507in}{1.878708in}}%
\pgfpathlineto{\pgfqpoint{0.921243in}{0.830330in}}%
\pgfpathlineto{\pgfqpoint{0.922364in}{1.738094in}}%
\pgfpathlineto{\pgfqpoint{0.924454in}{0.874039in}}%
\pgfpathlineto{\pgfqpoint{0.925711in}{1.835736in}}%
\pgfpathlineto{\pgfqpoint{0.927739in}{2.217505in}}%
\pgfpathlineto{\pgfqpoint{0.928898in}{0.872566in}}%
\pgfpathlineto{\pgfqpoint{0.930289in}{2.044368in}}%
\pgfpathlineto{\pgfqpoint{0.931929in}{0.939247in}}%
\pgfpathlineto{\pgfqpoint{0.933340in}{2.058880in}}%
\pgfpathlineto{\pgfqpoint{0.935355in}{0.946566in}}%
\pgfpathlineto{\pgfqpoint{0.936794in}{2.019576in}}%
\pgfpathlineto{\pgfqpoint{0.937996in}{0.872840in}}%
\pgfpathlineto{\pgfqpoint{0.940106in}{1.947383in}}%
\pgfpathlineto{\pgfqpoint{0.941317in}{1.001347in}}%
\pgfpathlineto{\pgfqpoint{0.942758in}{1.920423in}}%
\pgfpathlineto{\pgfqpoint{0.944584in}{0.926824in}}%
\pgfpathlineto{\pgfqpoint{0.945949in}{1.940164in}}%
\pgfpathlineto{\pgfqpoint{0.947678in}{0.871784in}}%
\pgfpathlineto{\pgfqpoint{0.949522in}{1.949246in}}%
\pgfpathlineto{\pgfqpoint{0.950761in}{0.987740in}}%
\pgfpathlineto{\pgfqpoint{0.952472in}{1.941010in}}%
\pgfpathlineto{\pgfqpoint{0.953909in}{0.918467in}}%
\pgfpathlineto{\pgfqpoint{0.955398in}{2.029326in}}%
\pgfpathlineto{\pgfqpoint{0.957494in}{0.908128in}}%
\pgfpathlineto{\pgfqpoint{0.958816in}{2.108411in}}%
\pgfpathlineto{\pgfqpoint{0.960560in}{0.832095in}}%
\pgfpathlineto{\pgfqpoint{0.961759in}{1.802686in}}%
\pgfpathlineto{\pgfqpoint{0.964026in}{0.716225in}}%
\pgfpathlineto{\pgfqpoint{0.964930in}{1.709862in}}%
\pgfpathlineto{\pgfqpoint{0.967342in}{0.769438in}}%
\pgfpathlineto{\pgfqpoint{0.968171in}{1.871745in}}%
\pgfpathlineto{\pgfqpoint{0.969786in}{0.839723in}}%
\pgfpathlineto{\pgfqpoint{0.971631in}{1.851603in}}%
\pgfpathlineto{\pgfqpoint{0.972699in}{0.841209in}}%
\pgfpathlineto{\pgfqpoint{0.974414in}{1.859033in}}%
\pgfpathlineto{\pgfqpoint{0.976274in}{0.724427in}}%
\pgfpathlineto{\pgfqpoint{0.977530in}{1.908576in}}%
\pgfpathlineto{\pgfqpoint{0.979478in}{0.759891in}}%
\pgfpathlineto{\pgfqpoint{0.980641in}{1.720189in}}%
\pgfpathlineto{\pgfqpoint{0.982875in}{0.720251in}}%
\pgfpathlineto{\pgfqpoint{0.983822in}{1.890408in}}%
\pgfpathlineto{\pgfqpoint{0.985388in}{0.795666in}}%
\pgfpathlineto{\pgfqpoint{0.987318in}{1.812907in}}%
\pgfpathlineto{\pgfqpoint{0.988575in}{0.747272in}}%
\pgfpathlineto{\pgfqpoint{0.990358in}{1.956012in}}%
\pgfpathlineto{\pgfqpoint{0.991580in}{0.932461in}}%
\pgfpathlineto{\pgfqpoint{0.993450in}{2.047572in}}%
\pgfpathlineto{\pgfqpoint{0.994778in}{0.860864in}}%
\pgfpathlineto{\pgfqpoint{0.996330in}{1.716798in}}%
\pgfpathlineto{\pgfqpoint{0.998220in}{0.893812in}}%
\pgfpathlineto{\pgfqpoint{0.999889in}{1.997210in}}%
\pgfpathlineto{\pgfqpoint{1.001462in}{0.651399in}}%
\pgfpathlineto{\pgfqpoint{1.002962in}{1.924314in}}%
\pgfpathlineto{\pgfqpoint{1.004214in}{0.801641in}}%
\pgfpathlineto{\pgfqpoint{1.006044in}{1.829529in}}%
\pgfpathlineto{\pgfqpoint{1.007587in}{0.821979in}}%
\pgfpathlineto{\pgfqpoint{1.008920in}{1.923653in}}%
\pgfpathlineto{\pgfqpoint{1.010532in}{0.891967in}}%
\pgfpathlineto{\pgfqpoint{1.012220in}{1.868855in}}%
\pgfpathlineto{\pgfqpoint{1.013665in}{0.895457in}}%
\pgfpathlineto{\pgfqpoint{1.015420in}{1.876537in}}%
\pgfpathlineto{\pgfqpoint{1.018314in}{0.844087in}}%
\pgfpathlineto{\pgfqpoint{1.018411in}{1.920520in}}%
\pgfpathlineto{\pgfqpoint{1.020001in}{0.825228in}}%
\pgfpathlineto{\pgfqpoint{1.022029in}{1.949719in}}%
\pgfpathlineto{\pgfqpoint{1.023366in}{0.841394in}}%
\pgfpathlineto{\pgfqpoint{1.024919in}{1.888344in}}%
\pgfpathlineto{\pgfqpoint{1.026526in}{0.827851in}}%
\pgfpathlineto{\pgfqpoint{1.028035in}{1.858461in}}%
\pgfpathlineto{\pgfqpoint{1.029610in}{0.820022in}}%
\pgfpathlineto{\pgfqpoint{1.031160in}{2.025829in}}%
\pgfpathlineto{\pgfqpoint{1.032663in}{0.805145in}}%
\pgfpathlineto{\pgfqpoint{1.034265in}{1.967621in}}%
\pgfpathlineto{\pgfqpoint{1.036102in}{0.914391in}}%
\pgfpathlineto{\pgfqpoint{1.037457in}{1.967193in}}%
\pgfpathlineto{\pgfqpoint{1.039015in}{0.916445in}}%
\pgfpathlineto{\pgfqpoint{1.040702in}{1.882199in}}%
\pgfpathlineto{\pgfqpoint{1.042138in}{1.026526in}}%
\pgfpathlineto{\pgfqpoint{1.043647in}{1.902451in}}%
\pgfpathlineto{\pgfqpoint{1.045481in}{0.854159in}}%
\pgfpathlineto{\pgfqpoint{1.046975in}{1.837058in}}%
\pgfpathlineto{\pgfqpoint{1.048333in}{0.946800in}}%
\pgfpathlineto{\pgfqpoint{1.049918in}{1.881631in}}%
\pgfpathlineto{\pgfqpoint{1.051923in}{0.789639in}}%
\pgfpathlineto{\pgfqpoint{1.053106in}{1.760216in}}%
\pgfpathlineto{\pgfqpoint{1.055055in}{0.730531in}}%
\pgfpathlineto{\pgfqpoint{1.056357in}{2.042609in}}%
\pgfpathlineto{\pgfqpoint{1.057949in}{0.914034in}}%
\pgfpathlineto{\pgfqpoint{1.059742in}{1.929347in}}%
\pgfpathlineto{\pgfqpoint{1.061008in}{0.860186in}}%
\pgfpathlineto{\pgfqpoint{1.063349in}{1.924626in}}%
\pgfpathlineto{\pgfqpoint{1.064399in}{0.824598in}}%
\pgfpathlineto{\pgfqpoint{1.065652in}{1.844889in}}%
\pgfpathlineto{\pgfqpoint{1.067436in}{0.660774in}}%
\pgfpathlineto{\pgfqpoint{1.069187in}{1.976450in}}%
\pgfpathlineto{\pgfqpoint{1.070736in}{0.828000in}}%
\pgfpathlineto{\pgfqpoint{1.072592in}{1.990424in}}%
\pgfpathlineto{\pgfqpoint{1.073526in}{0.890630in}}%
\pgfpathlineto{\pgfqpoint{1.075169in}{1.932148in}}%
\pgfpathlineto{\pgfqpoint{1.076874in}{0.915492in}}%
\pgfpathlineto{\pgfqpoint{1.078584in}{1.895241in}}%
\pgfpathlineto{\pgfqpoint{1.079945in}{0.677113in}}%
\pgfpathlineto{\pgfqpoint{1.081638in}{1.746207in}}%
\pgfpathlineto{\pgfqpoint{1.083157in}{0.712924in}}%
\pgfpathlineto{\pgfqpoint{1.084705in}{1.685279in}}%
\pgfpathlineto{\pgfqpoint{1.086373in}{0.861023in}}%
\pgfpathlineto{\pgfqpoint{1.087897in}{1.924734in}}%
\pgfpathlineto{\pgfqpoint{1.089329in}{0.885061in}}%
\pgfpathlineto{\pgfqpoint{1.091112in}{1.909412in}}%
\pgfpathlineto{\pgfqpoint{1.092775in}{0.837061in}}%
\pgfpathlineto{\pgfqpoint{1.094440in}{1.838738in}}%
\pgfpathlineto{\pgfqpoint{1.095658in}{0.813919in}}%
\pgfpathlineto{\pgfqpoint{1.097382in}{1.995472in}}%
\pgfpathlineto{\pgfqpoint{1.099291in}{0.915024in}}%
\pgfpathlineto{\pgfqpoint{1.100997in}{1.798446in}}%
\pgfpathlineto{\pgfqpoint{1.102040in}{0.781405in}}%
\pgfpathlineto{\pgfqpoint{1.103854in}{2.016831in}}%
\pgfpathlineto{\pgfqpoint{1.105216in}{0.858247in}}%
\pgfpathlineto{\pgfqpoint{1.106689in}{1.797435in}}%
\pgfpathlineto{\pgfqpoint{1.108430in}{0.822865in}}%
\pgfpathlineto{\pgfqpoint{1.109871in}{1.859048in}}%
\pgfpathlineto{\pgfqpoint{1.111804in}{0.756265in}}%
\pgfpathlineto{\pgfqpoint{1.113607in}{1.803851in}}%
\pgfpathlineto{\pgfqpoint{1.115820in}{0.742733in}}%
\pgfpathlineto{\pgfqpoint{1.116449in}{1.796112in}}%
\pgfpathlineto{\pgfqpoint{1.117705in}{0.837556in}}%
\pgfpathlineto{\pgfqpoint{1.119257in}{2.016038in}}%
\pgfpathlineto{\pgfqpoint{1.121251in}{0.865656in}}%
\pgfpathlineto{\pgfqpoint{1.123039in}{2.085473in}}%
\pgfpathlineto{\pgfqpoint{1.124021in}{0.901372in}}%
\pgfpathlineto{\pgfqpoint{1.125832in}{2.081843in}}%
\pgfpathlineto{\pgfqpoint{1.127295in}{0.984305in}}%
\pgfpathlineto{\pgfqpoint{1.128825in}{1.894024in}}%
\pgfpathlineto{\pgfqpoint{1.130771in}{1.030701in}}%
\pgfpathlineto{\pgfqpoint{1.132125in}{2.063401in}}%
\pgfpathlineto{\pgfqpoint{1.133605in}{1.006170in}}%
\pgfpathlineto{\pgfqpoint{1.134985in}{1.941159in}}%
\pgfpathlineto{\pgfqpoint{1.136623in}{0.837615in}}%
\pgfpathlineto{\pgfqpoint{1.139241in}{2.190580in}}%
\pgfpathlineto{\pgfqpoint{1.139713in}{1.008640in}}%
\pgfpathlineto{\pgfqpoint{1.142492in}{2.035589in}}%
\pgfpathlineto{\pgfqpoint{1.143460in}{0.959821in}}%
\pgfpathlineto{\pgfqpoint{1.144881in}{1.966181in}}%
\pgfpathlineto{\pgfqpoint{1.146058in}{1.019930in}}%
\pgfpathlineto{\pgfqpoint{1.148948in}{2.067244in}}%
\pgfpathlineto{\pgfqpoint{1.149865in}{0.853242in}}%
\pgfpathlineto{\pgfqpoint{1.151152in}{1.896213in}}%
\pgfpathlineto{\pgfqpoint{1.152704in}{0.857674in}}%
\pgfpathlineto{\pgfqpoint{1.154692in}{2.059110in}}%
\pgfpathlineto{\pgfqpoint{1.155548in}{0.987060in}}%
\pgfpathlineto{\pgfqpoint{1.158337in}{2.203317in}}%
\pgfpathlineto{\pgfqpoint{1.158975in}{0.974368in}}%
\pgfpathlineto{\pgfqpoint{1.161171in}{2.142918in}}%
\pgfpathlineto{\pgfqpoint{1.161953in}{0.956478in}}%
\pgfpathlineto{\pgfqpoint{1.163372in}{1.882982in}}%
\pgfpathlineto{\pgfqpoint{1.165253in}{0.915043in}}%
\pgfpathlineto{\pgfqpoint{1.166757in}{1.912704in}}%
\pgfpathlineto{\pgfqpoint{1.168521in}{0.831828in}}%
\pgfpathlineto{\pgfqpoint{1.169916in}{1.802895in}}%
\pgfpathlineto{\pgfqpoint{1.171715in}{0.798924in}}%
\pgfpathlineto{\pgfqpoint{1.173002in}{1.910986in}}%
\pgfpathlineto{\pgfqpoint{1.174489in}{0.900658in}}%
\pgfpathlineto{\pgfqpoint{1.176659in}{2.026741in}}%
\pgfpathlineto{\pgfqpoint{1.177973in}{0.963725in}}%
\pgfpathlineto{\pgfqpoint{1.179653in}{1.916045in}}%
\pgfpathlineto{\pgfqpoint{1.180686in}{0.925250in}}%
\pgfpathlineto{\pgfqpoint{1.182386in}{2.043949in}}%
\pgfpathlineto{\pgfqpoint{1.183965in}{0.852373in}}%
\pgfpathlineto{\pgfqpoint{1.185469in}{1.782736in}}%
\pgfpathlineto{\pgfqpoint{1.187290in}{0.737110in}}%
\pgfpathlineto{\pgfqpoint{1.189039in}{1.894000in}}%
\pgfpathlineto{\pgfqpoint{1.190216in}{0.906934in}}%
\pgfpathlineto{\pgfqpoint{1.191738in}{1.855330in}}%
\pgfpathlineto{\pgfqpoint{1.193777in}{0.846392in}}%
\pgfpathlineto{\pgfqpoint{1.194937in}{1.976236in}}%
\pgfpathlineto{\pgfqpoint{1.196540in}{0.782555in}}%
\pgfpathlineto{\pgfqpoint{1.198242in}{1.788626in}}%
\pgfpathlineto{\pgfqpoint{1.199673in}{0.748393in}}%
\pgfpathlineto{\pgfqpoint{1.202085in}{1.996055in}}%
\pgfpathlineto{\pgfqpoint{1.202890in}{0.903578in}}%
\pgfpathlineto{\pgfqpoint{1.205065in}{1.805288in}}%
\pgfpathlineto{\pgfqpoint{1.206160in}{0.846370in}}%
\pgfpathlineto{\pgfqpoint{1.208607in}{1.983788in}}%
\pgfpathlineto{\pgfqpoint{1.209521in}{0.733329in}}%
\pgfpathlineto{\pgfqpoint{1.210956in}{1.805011in}}%
\pgfpathlineto{\pgfqpoint{1.212642in}{0.792022in}}%
\pgfpathlineto{\pgfqpoint{1.214199in}{1.825147in}}%
\pgfpathlineto{\pgfqpoint{1.215484in}{0.790858in}}%
\pgfpathlineto{\pgfqpoint{1.216968in}{1.838871in}}%
\pgfpathlineto{\pgfqpoint{1.218551in}{0.806810in}}%
\pgfpathlineto{\pgfqpoint{1.220283in}{1.861646in}}%
\pgfpathlineto{\pgfqpoint{1.222407in}{0.679510in}}%
\pgfpathlineto{\pgfqpoint{1.223455in}{1.755641in}}%
\pgfpathlineto{\pgfqpoint{1.224851in}{0.872267in}}%
\pgfpathlineto{\pgfqpoint{1.227240in}{1.977841in}}%
\pgfpathlineto{\pgfqpoint{1.228478in}{0.887915in}}%
\pgfpathlineto{\pgfqpoint{1.230255in}{2.001532in}}%
\pgfpathlineto{\pgfqpoint{1.231768in}{0.811179in}}%
\pgfpathlineto{\pgfqpoint{1.232959in}{1.928726in}}%
\pgfpathlineto{\pgfqpoint{1.234734in}{0.780148in}}%
\pgfpathlineto{\pgfqpoint{1.235897in}{1.884190in}}%
\pgfpathlineto{\pgfqpoint{1.238273in}{0.854851in}}%
\pgfpathlineto{\pgfqpoint{1.239161in}{2.168947in}}%
\pgfpathlineto{\pgfqpoint{1.240948in}{0.778802in}}%
\pgfpathlineto{\pgfqpoint{1.242433in}{1.902737in}}%
\pgfpathlineto{\pgfqpoint{1.243910in}{0.893503in}}%
\pgfpathlineto{\pgfqpoint{1.246161in}{1.937236in}}%
\pgfpathlineto{\pgfqpoint{1.246979in}{0.877795in}}%
\pgfpathlineto{\pgfqpoint{1.248880in}{1.807555in}}%
\pgfpathlineto{\pgfqpoint{1.250064in}{0.851521in}}%
\pgfpathlineto{\pgfqpoint{1.252295in}{1.895622in}}%
\pgfpathlineto{\pgfqpoint{1.254060in}{0.851306in}}%
\pgfpathlineto{\pgfqpoint{1.255286in}{1.979699in}}%
\pgfpathlineto{\pgfqpoint{1.256329in}{0.985662in}}%
\pgfpathlineto{\pgfqpoint{1.258228in}{1.995205in}}%
\pgfpathlineto{\pgfqpoint{1.259923in}{0.889980in}}%
\pgfpathlineto{\pgfqpoint{1.261408in}{1.916411in}}%
\pgfpathlineto{\pgfqpoint{1.262662in}{0.992239in}}%
\pgfpathlineto{\pgfqpoint{1.264458in}{1.866590in}}%
\pgfpathlineto{\pgfqpoint{1.265848in}{0.937471in}}%
\pgfpathlineto{\pgfqpoint{1.267418in}{1.862439in}}%
\pgfpathlineto{\pgfqpoint{1.269302in}{0.832309in}}%
\pgfpathlineto{\pgfqpoint{1.271525in}{2.017333in}}%
\pgfpathlineto{\pgfqpoint{1.272452in}{0.868345in}}%
\pgfpathlineto{\pgfqpoint{1.274308in}{1.976805in}}%
\pgfpathlineto{\pgfqpoint{1.275597in}{0.739296in}}%
\pgfpathlineto{\pgfqpoint{1.276874in}{1.927837in}}%
\pgfpathlineto{\pgfqpoint{1.278726in}{0.777249in}}%
\pgfpathlineto{\pgfqpoint{1.280522in}{2.038270in}}%
\pgfpathlineto{\pgfqpoint{1.281696in}{0.840196in}}%
\pgfpathlineto{\pgfqpoint{1.283607in}{2.023055in}}%
\pgfpathlineto{\pgfqpoint{1.284731in}{0.951826in}}%
\pgfpathlineto{\pgfqpoint{1.286991in}{1.944344in}}%
\pgfpathlineto{\pgfqpoint{1.287871in}{0.912405in}}%
\pgfpathlineto{\pgfqpoint{1.289727in}{1.986721in}}%
\pgfpathlineto{\pgfqpoint{1.291708in}{0.824718in}}%
\pgfpathlineto{\pgfqpoint{1.292714in}{1.909854in}}%
\pgfpathlineto{\pgfqpoint{1.294908in}{0.879358in}}%
\pgfpathlineto{\pgfqpoint{1.296167in}{1.907983in}}%
\pgfpathlineto{\pgfqpoint{1.297718in}{0.717483in}}%
\pgfpathlineto{\pgfqpoint{1.299062in}{1.764024in}}%
\pgfpathlineto{\pgfqpoint{1.301117in}{0.820406in}}%
\pgfpathlineto{\pgfqpoint{1.302038in}{1.711566in}}%
\pgfpathlineto{\pgfqpoint{1.304109in}{0.743417in}}%
\pgfpathlineto{\pgfqpoint{1.305660in}{1.942476in}}%
\pgfpathlineto{\pgfqpoint{1.307663in}{0.792991in}}%
\pgfpathlineto{\pgfqpoint{1.308339in}{1.953903in}}%
\pgfpathlineto{\pgfqpoint{1.309927in}{0.928886in}}%
\pgfpathlineto{\pgfqpoint{1.312133in}{2.058617in}}%
\pgfpathlineto{\pgfqpoint{1.313156in}{0.856368in}}%
\pgfpathlineto{\pgfqpoint{1.315319in}{1.856194in}}%
\pgfpathlineto{\pgfqpoint{1.316358in}{0.737148in}}%
\pgfpathlineto{\pgfqpoint{1.318043in}{1.862729in}}%
\pgfpathlineto{\pgfqpoint{1.319446in}{0.687640in}}%
\pgfpathlineto{\pgfqpoint{1.321028in}{1.811937in}}%
\pgfpathlineto{\pgfqpoint{1.322512in}{0.740314in}}%
\pgfpathlineto{\pgfqpoint{1.324260in}{1.750430in}}%
\pgfpathlineto{\pgfqpoint{1.325769in}{0.684926in}}%
\pgfpathlineto{\pgfqpoint{1.327769in}{1.894518in}}%
\pgfpathlineto{\pgfqpoint{1.328816in}{0.778531in}}%
\pgfpathlineto{\pgfqpoint{1.331901in}{1.997257in}}%
\pgfpathlineto{\pgfqpoint{1.332166in}{0.893254in}}%
\pgfpathlineto{\pgfqpoint{1.333596in}{1.895249in}}%
\pgfpathlineto{\pgfqpoint{1.335540in}{0.924850in}}%
\pgfpathlineto{\pgfqpoint{1.336803in}{1.915452in}}%
\pgfpathlineto{\pgfqpoint{1.338479in}{0.758581in}}%
\pgfpathlineto{\pgfqpoint{1.340103in}{1.883960in}}%
\pgfpathlineto{\pgfqpoint{1.341627in}{0.851561in}}%
\pgfpathlineto{\pgfqpoint{1.343216in}{1.816755in}}%
\pgfpathlineto{\pgfqpoint{1.344877in}{0.942153in}}%
\pgfpathlineto{\pgfqpoint{1.346982in}{1.977534in}}%
\pgfpathlineto{\pgfqpoint{1.347805in}{0.900102in}}%
\pgfpathlineto{\pgfqpoint{1.349828in}{2.003137in}}%
\pgfpathlineto{\pgfqpoint{1.351061in}{0.917538in}}%
\pgfpathlineto{\pgfqpoint{1.352631in}{1.721769in}}%
\pgfpathlineto{\pgfqpoint{1.355068in}{0.778153in}}%
\pgfpathlineto{\pgfqpoint{1.356296in}{1.915507in}}%
\pgfpathlineto{\pgfqpoint{1.357279in}{0.780638in}}%
\pgfpathlineto{\pgfqpoint{1.358756in}{1.810905in}}%
\pgfpathlineto{\pgfqpoint{1.360435in}{0.965735in}}%
\pgfpathlineto{\pgfqpoint{1.363256in}{1.957477in}}%
\pgfpathlineto{\pgfqpoint{1.363829in}{0.773381in}}%
\pgfpathlineto{\pgfqpoint{1.365832in}{2.113878in}}%
\pgfpathlineto{\pgfqpoint{1.366747in}{0.777140in}}%
\pgfpathlineto{\pgfqpoint{1.368283in}{1.752727in}}%
\pgfpathlineto{\pgfqpoint{1.369975in}{0.912657in}}%
\pgfpathlineto{\pgfqpoint{1.372229in}{1.935359in}}%
\pgfpathlineto{\pgfqpoint{1.372972in}{1.018107in}}%
\pgfpathlineto{\pgfqpoint{1.374805in}{1.901782in}}%
\pgfpathlineto{\pgfqpoint{1.376261in}{0.734374in}}%
\pgfpathlineto{\pgfqpoint{1.377875in}{1.981721in}}%
\pgfpathlineto{\pgfqpoint{1.380075in}{0.878378in}}%
\pgfpathlineto{\pgfqpoint{1.381136in}{1.802154in}}%
\pgfpathlineto{\pgfqpoint{1.383413in}{0.772306in}}%
\pgfpathlineto{\pgfqpoint{1.384122in}{1.723722in}}%
\pgfpathlineto{\pgfqpoint{1.385577in}{0.874458in}}%
\pgfpathlineto{\pgfqpoint{1.388274in}{1.817781in}}%
\pgfpathlineto{\pgfqpoint{1.388710in}{0.691186in}}%
\pgfpathlineto{\pgfqpoint{1.390283in}{1.715837in}}%
\pgfpathlineto{\pgfqpoint{1.391979in}{0.816419in}}%
\pgfpathlineto{\pgfqpoint{1.393625in}{1.735352in}}%
\pgfpathlineto{\pgfqpoint{1.395242in}{0.632244in}}%
\pgfpathlineto{\pgfqpoint{1.396626in}{1.726239in}}%
\pgfpathlineto{\pgfqpoint{1.398906in}{0.770896in}}%
\pgfpathlineto{\pgfqpoint{1.399939in}{1.765181in}}%
\pgfpathlineto{\pgfqpoint{1.402038in}{0.832596in}}%
\pgfpathlineto{\pgfqpoint{1.403274in}{1.961354in}}%
\pgfpathlineto{\pgfqpoint{1.404656in}{0.955392in}}%
\pgfpathlineto{\pgfqpoint{1.406649in}{1.875591in}}%
\pgfpathlineto{\pgfqpoint{1.407686in}{0.937496in}}%
\pgfpathlineto{\pgfqpoint{1.409399in}{1.913508in}}%
\pgfpathlineto{\pgfqpoint{1.411355in}{0.811820in}}%
\pgfpathlineto{\pgfqpoint{1.413600in}{2.025946in}}%
\pgfpathlineto{\pgfqpoint{1.413949in}{0.946285in}}%
\pgfpathlineto{\pgfqpoint{1.415788in}{1.756340in}}%
\pgfpathlineto{\pgfqpoint{1.417363in}{0.811228in}}%
\pgfpathlineto{\pgfqpoint{1.418683in}{1.861304in}}%
\pgfpathlineto{\pgfqpoint{1.420216in}{0.836906in}}%
\pgfpathlineto{\pgfqpoint{1.421865in}{1.753511in}}%
\pgfpathlineto{\pgfqpoint{1.423684in}{0.670969in}}%
\pgfpathlineto{\pgfqpoint{1.425198in}{1.752600in}}%
\pgfpathlineto{\pgfqpoint{1.426961in}{0.805383in}}%
\pgfpathlineto{\pgfqpoint{1.428114in}{1.716881in}}%
\pgfpathlineto{\pgfqpoint{1.431020in}{0.582326in}}%
\pgfpathlineto{\pgfqpoint{1.431334in}{1.629585in}}%
\pgfpathlineto{\pgfqpoint{1.432966in}{0.706617in}}%
\pgfpathlineto{\pgfqpoint{1.435037in}{1.746040in}}%
\pgfpathlineto{\pgfqpoint{1.436759in}{0.503735in}}%
\pgfpathlineto{\pgfqpoint{1.438056in}{1.899941in}}%
\pgfpathlineto{\pgfqpoint{1.439131in}{0.681114in}}%
\pgfpathlineto{\pgfqpoint{1.441321in}{1.724703in}}%
\pgfpathlineto{\pgfqpoint{1.442513in}{0.830810in}}%
\pgfpathlineto{\pgfqpoint{1.444705in}{1.959332in}}%
\pgfpathlineto{\pgfqpoint{1.445570in}{0.880140in}}%
\pgfpathlineto{\pgfqpoint{1.447615in}{1.824801in}}%
\pgfpathlineto{\pgfqpoint{1.448630in}{0.868631in}}%
\pgfpathlineto{\pgfqpoint{1.450277in}{1.818783in}}%
\pgfpathlineto{\pgfqpoint{1.453246in}{0.574823in}}%
\pgfpathlineto{\pgfqpoint{1.453488in}{1.788452in}}%
\pgfpathlineto{\pgfqpoint{1.455613in}{0.711390in}}%
\pgfpathlineto{\pgfqpoint{1.456646in}{1.681151in}}%
\pgfpathlineto{\pgfqpoint{1.458254in}{0.785220in}}%
\pgfpathlineto{\pgfqpoint{1.461068in}{1.809901in}}%
\pgfpathlineto{\pgfqpoint{1.461355in}{0.689119in}}%
\pgfpathlineto{\pgfqpoint{1.462906in}{1.813857in}}%
\pgfpathlineto{\pgfqpoint{1.464349in}{0.781881in}}%
\pgfpathlineto{\pgfqpoint{1.466926in}{1.935465in}}%
\pgfpathlineto{\pgfqpoint{1.467756in}{0.796321in}}%
\pgfpathlineto{\pgfqpoint{1.469118in}{1.695470in}}%
\pgfpathlineto{\pgfqpoint{1.471083in}{0.755797in}}%
\pgfpathlineto{\pgfqpoint{1.472968in}{1.917520in}}%
\pgfpathlineto{\pgfqpoint{1.473820in}{0.686599in}}%
\pgfpathlineto{\pgfqpoint{1.475518in}{1.823559in}}%
\pgfpathlineto{\pgfqpoint{1.477196in}{0.699398in}}%
\pgfpathlineto{\pgfqpoint{1.479048in}{1.873124in}}%
\pgfpathlineto{\pgfqpoint{1.480112in}{0.899094in}}%
\pgfpathlineto{\pgfqpoint{1.482320in}{1.906868in}}%
\pgfpathlineto{\pgfqpoint{1.483252in}{0.803460in}}%
\pgfpathlineto{\pgfqpoint{1.485062in}{2.027574in}}%
\pgfpathlineto{\pgfqpoint{1.486678in}{0.566256in}}%
\pgfpathlineto{\pgfqpoint{1.488611in}{1.959496in}}%
\pgfpathlineto{\pgfqpoint{1.489569in}{0.895882in}}%
\pgfpathlineto{\pgfqpoint{1.491705in}{1.947704in}}%
\pgfpathlineto{\pgfqpoint{1.493136in}{0.804561in}}%
\pgfpathlineto{\pgfqpoint{1.495038in}{1.938731in}}%
\pgfpathlineto{\pgfqpoint{1.496229in}{0.660155in}}%
\pgfpathlineto{\pgfqpoint{1.497793in}{1.741419in}}%
\pgfpathlineto{\pgfqpoint{1.499629in}{0.822172in}}%
\pgfpathlineto{\pgfqpoint{1.500774in}{1.676404in}}%
\pgfpathlineto{\pgfqpoint{1.502882in}{0.670825in}}%
\pgfpathlineto{\pgfqpoint{1.504240in}{1.756402in}}%
\pgfpathlineto{\pgfqpoint{1.505781in}{0.773398in}}%
\pgfpathlineto{\pgfqpoint{1.506979in}{1.866412in}}%
\pgfpathlineto{\pgfqpoint{1.508598in}{0.712020in}}%
\pgfpathlineto{\pgfqpoint{1.510287in}{1.805715in}}%
\pgfpathlineto{\pgfqpoint{1.511858in}{0.893757in}}%
\pgfpathlineto{\pgfqpoint{1.513222in}{1.882056in}}%
\pgfpathlineto{\pgfqpoint{1.514784in}{0.961690in}}%
\pgfpathlineto{\pgfqpoint{1.516474in}{1.882670in}}%
\pgfpathlineto{\pgfqpoint{1.518591in}{0.867507in}}%
\pgfpathlineto{\pgfqpoint{1.519600in}{2.033072in}}%
\pgfpathlineto{\pgfqpoint{1.521074in}{0.750432in}}%
\pgfpathlineto{\pgfqpoint{1.523091in}{1.890270in}}%
\pgfpathlineto{\pgfqpoint{1.524287in}{0.700998in}}%
\pgfpathlineto{\pgfqpoint{1.525875in}{1.809143in}}%
\pgfpathlineto{\pgfqpoint{1.527659in}{0.777917in}}%
\pgfpathlineto{\pgfqpoint{1.529358in}{1.774819in}}%
\pgfpathlineto{\pgfqpoint{1.530585in}{0.833481in}}%
\pgfpathlineto{\pgfqpoint{1.532250in}{1.826279in}}%
\pgfpathlineto{\pgfqpoint{1.534196in}{0.744682in}}%
\pgfpathlineto{\pgfqpoint{1.535825in}{1.924813in}}%
\pgfpathlineto{\pgfqpoint{1.536883in}{0.747754in}}%
\pgfpathlineto{\pgfqpoint{1.538537in}{1.730926in}}%
\pgfpathlineto{\pgfqpoint{1.540733in}{0.757252in}}%
\pgfpathlineto{\pgfqpoint{1.541712in}{1.820703in}}%
\pgfpathlineto{\pgfqpoint{1.544001in}{0.733605in}}%
\pgfpathlineto{\pgfqpoint{1.545139in}{1.890694in}}%
\pgfpathlineto{\pgfqpoint{1.546814in}{0.876385in}}%
\pgfpathlineto{\pgfqpoint{1.548745in}{2.047162in}}%
\pgfpathlineto{\pgfqpoint{1.549470in}{1.018351in}}%
\pgfpathlineto{\pgfqpoint{1.551109in}{2.001236in}}%
\pgfpathlineto{\pgfqpoint{1.553864in}{0.735843in}}%
\pgfpathlineto{\pgfqpoint{1.554341in}{1.923704in}}%
\pgfpathlineto{\pgfqpoint{1.555819in}{0.883072in}}%
\pgfpathlineto{\pgfqpoint{1.558326in}{2.023559in}}%
\pgfpathlineto{\pgfqpoint{1.559040in}{0.890239in}}%
\pgfpathlineto{\pgfqpoint{1.560722in}{1.960933in}}%
\pgfpathlineto{\pgfqpoint{1.562216in}{0.946172in}}%
\pgfpathlineto{\pgfqpoint{1.564041in}{1.837432in}}%
\pgfpathlineto{\pgfqpoint{1.566260in}{0.606302in}}%
\pgfpathlineto{\pgfqpoint{1.566843in}{1.744427in}}%
\pgfpathlineto{\pgfqpoint{1.568449in}{0.822797in}}%
\pgfpathlineto{\pgfqpoint{1.571012in}{1.855081in}}%
\pgfpathlineto{\pgfqpoint{1.571607in}{0.786284in}}%
\pgfpathlineto{\pgfqpoint{1.574359in}{1.897350in}}%
\pgfpathlineto{\pgfqpoint{1.575110in}{0.726616in}}%
\pgfpathlineto{\pgfqpoint{1.576246in}{1.749186in}}%
\pgfpathlineto{\pgfqpoint{1.578227in}{0.782677in}}%
\pgfpathlineto{\pgfqpoint{1.579396in}{1.774097in}}%
\pgfpathlineto{\pgfqpoint{1.581730in}{0.761618in}}%
\pgfpathlineto{\pgfqpoint{1.583444in}{1.943467in}}%
\pgfpathlineto{\pgfqpoint{1.584277in}{0.891548in}}%
\pgfpathlineto{\pgfqpoint{1.586263in}{1.933719in}}%
\pgfpathlineto{\pgfqpoint{1.588099in}{0.799343in}}%
\pgfpathlineto{\pgfqpoint{1.588975in}{1.860128in}}%
\pgfpathlineto{\pgfqpoint{1.590929in}{0.800849in}}%
\pgfpathlineto{\pgfqpoint{1.592328in}{1.774448in}}%
\pgfpathlineto{\pgfqpoint{1.593992in}{0.622193in}}%
\pgfpathlineto{\pgfqpoint{1.595431in}{1.788734in}}%
\pgfpathlineto{\pgfqpoint{1.597148in}{0.779794in}}%
\pgfpathlineto{\pgfqpoint{1.598398in}{1.801956in}}%
\pgfpathlineto{\pgfqpoint{1.601025in}{0.801539in}}%
\pgfpathlineto{\pgfqpoint{1.601620in}{1.881345in}}%
\pgfpathlineto{\pgfqpoint{1.603320in}{0.880894in}}%
\pgfpathlineto{\pgfqpoint{1.604874in}{1.806273in}}%
\pgfpathlineto{\pgfqpoint{1.606240in}{0.854759in}}%
\pgfpathlineto{\pgfqpoint{1.607873in}{1.904783in}}%
\pgfpathlineto{\pgfqpoint{1.609579in}{0.807394in}}%
\pgfpathlineto{\pgfqpoint{1.610981in}{1.831907in}}%
\pgfpathlineto{\pgfqpoint{1.612540in}{0.722820in}}%
\pgfpathlineto{\pgfqpoint{1.614570in}{1.919128in}}%
\pgfpathlineto{\pgfqpoint{1.615858in}{0.838974in}}%
\pgfpathlineto{\pgfqpoint{1.617488in}{1.874266in}}%
\pgfpathlineto{\pgfqpoint{1.618844in}{0.663033in}}%
\pgfpathlineto{\pgfqpoint{1.620554in}{1.871538in}}%
\pgfpathlineto{\pgfqpoint{1.622326in}{0.815414in}}%
\pgfpathlineto{\pgfqpoint{1.624413in}{1.900665in}}%
\pgfpathlineto{\pgfqpoint{1.625246in}{0.708736in}}%
\pgfpathlineto{\pgfqpoint{1.626688in}{1.791258in}}%
\pgfpathlineto{\pgfqpoint{1.628285in}{0.765861in}}%
\pgfpathlineto{\pgfqpoint{1.630059in}{1.849392in}}%
\pgfpathlineto{\pgfqpoint{1.631399in}{0.924366in}}%
\pgfpathlineto{\pgfqpoint{1.633159in}{1.860035in}}%
\pgfpathlineto{\pgfqpoint{1.634579in}{0.896299in}}%
\pgfpathlineto{\pgfqpoint{1.636395in}{1.968363in}}%
\pgfpathlineto{\pgfqpoint{1.638966in}{0.848562in}}%
\pgfpathlineto{\pgfqpoint{1.639669in}{1.939346in}}%
\pgfpathlineto{\pgfqpoint{1.641713in}{0.894973in}}%
\pgfpathlineto{\pgfqpoint{1.643599in}{2.122798in}}%
\pgfpathlineto{\pgfqpoint{1.644046in}{0.949118in}}%
\pgfpathlineto{\pgfqpoint{1.645795in}{2.018929in}}%
\pgfpathlineto{\pgfqpoint{1.647371in}{1.023978in}}%
\pgfpathlineto{\pgfqpoint{1.649473in}{2.033192in}}%
\pgfpathlineto{\pgfqpoint{1.650434in}{0.833749in}}%
\pgfpathlineto{\pgfqpoint{1.651918in}{1.932717in}}%
\pgfpathlineto{\pgfqpoint{1.653453in}{0.955235in}}%
\pgfpathlineto{\pgfqpoint{1.656300in}{2.249495in}}%
\pgfpathlineto{\pgfqpoint{1.657798in}{0.856187in}}%
\pgfpathlineto{\pgfqpoint{1.658297in}{2.032110in}}%
\pgfpathlineto{\pgfqpoint{1.660495in}{0.955138in}}%
\pgfpathlineto{\pgfqpoint{1.661656in}{1.998931in}}%
\pgfpathlineto{\pgfqpoint{1.663009in}{0.917452in}}%
\pgfpathlineto{\pgfqpoint{1.664752in}{1.843938in}}%
\pgfpathlineto{\pgfqpoint{1.666229in}{0.944425in}}%
\pgfpathlineto{\pgfqpoint{1.667979in}{1.966996in}}%
\pgfpathlineto{\pgfqpoint{1.669516in}{0.695386in}}%
\pgfpathlineto{\pgfqpoint{1.671021in}{1.998584in}}%
\pgfpathlineto{\pgfqpoint{1.672492in}{0.800052in}}%
\pgfpathlineto{\pgfqpoint{1.674969in}{1.847082in}}%
\pgfpathlineto{\pgfqpoint{1.675974in}{0.633572in}}%
\pgfpathlineto{\pgfqpoint{1.677110in}{1.766770in}}%
\pgfpathlineto{\pgfqpoint{1.678729in}{0.799972in}}%
\pgfpathlineto{\pgfqpoint{1.680288in}{1.792492in}}%
\pgfpathlineto{\pgfqpoint{1.681814in}{0.840943in}}%
\pgfpathlineto{\pgfqpoint{1.683769in}{1.805305in}}%
\pgfpathlineto{\pgfqpoint{1.685305in}{0.825468in}}%
\pgfpathlineto{\pgfqpoint{1.686707in}{1.774427in}}%
\pgfpathlineto{\pgfqpoint{1.688244in}{0.675017in}}%
\pgfpathlineto{\pgfqpoint{1.690656in}{1.986576in}}%
\pgfpathlineto{\pgfqpoint{1.691600in}{0.852312in}}%
\pgfpathlineto{\pgfqpoint{1.692961in}{1.821111in}}%
\pgfpathlineto{\pgfqpoint{1.695032in}{0.824533in}}%
\pgfpathlineto{\pgfqpoint{1.696100in}{1.909043in}}%
\pgfpathlineto{\pgfqpoint{1.698441in}{0.828106in}}%
\pgfpathlineto{\pgfqpoint{1.700024in}{1.969790in}}%
\pgfpathlineto{\pgfqpoint{1.700817in}{0.950451in}}%
\pgfpathlineto{\pgfqpoint{1.702946in}{1.796673in}}%
\pgfpathlineto{\pgfqpoint{1.704424in}{0.691714in}}%
\pgfpathlineto{\pgfqpoint{1.705612in}{1.884923in}}%
\pgfpathlineto{\pgfqpoint{1.707053in}{0.855350in}}%
\pgfpathlineto{\pgfqpoint{1.708664in}{1.747826in}}%
\pgfpathlineto{\pgfqpoint{1.711520in}{0.716337in}}%
\pgfpathlineto{\pgfqpoint{1.711932in}{1.847556in}}%
\pgfpathlineto{\pgfqpoint{1.714063in}{0.822280in}}%
\pgfpathlineto{\pgfqpoint{1.714936in}{1.816153in}}%
\pgfpathlineto{\pgfqpoint{1.716661in}{0.813493in}}%
\pgfpathlineto{\pgfqpoint{1.718672in}{1.796068in}}%
\pgfpathlineto{\pgfqpoint{1.719683in}{0.761865in}}%
\pgfpathlineto{\pgfqpoint{1.721869in}{1.821047in}}%
\pgfpathlineto{\pgfqpoint{1.722841in}{0.838045in}}%
\pgfpathlineto{\pgfqpoint{1.724799in}{1.829435in}}%
\pgfpathlineto{\pgfqpoint{1.726067in}{0.807064in}}%
\pgfpathlineto{\pgfqpoint{1.727646in}{1.840986in}}%
\pgfpathlineto{\pgfqpoint{1.729437in}{0.909929in}}%
\pgfpathlineto{\pgfqpoint{1.730777in}{1.872942in}}%
\pgfpathlineto{\pgfqpoint{1.732275in}{0.750120in}}%
\pgfpathlineto{\pgfqpoint{1.733923in}{1.794718in}}%
\pgfpathlineto{\pgfqpoint{1.735479in}{0.870065in}}%
\pgfpathlineto{\pgfqpoint{1.736998in}{1.700687in}}%
\pgfpathlineto{\pgfqpoint{1.739377in}{0.793862in}}%
\pgfpathlineto{\pgfqpoint{1.740504in}{1.783995in}}%
\pgfpathlineto{\pgfqpoint{1.742029in}{0.623822in}}%
\pgfpathlineto{\pgfqpoint{1.743947in}{1.763074in}}%
\pgfpathlineto{\pgfqpoint{1.745045in}{0.723361in}}%
\pgfpathlineto{\pgfqpoint{1.747017in}{1.815461in}}%
\pgfpathlineto{\pgfqpoint{1.748034in}{0.776824in}}%
\pgfpathlineto{\pgfqpoint{1.749671in}{1.903750in}}%
\pgfpathlineto{\pgfqpoint{1.751646in}{0.893538in}}%
\pgfpathlineto{\pgfqpoint{1.752808in}{1.775652in}}%
\pgfpathlineto{\pgfqpoint{1.754437in}{0.742337in}}%
\pgfpathlineto{\pgfqpoint{1.756300in}{1.816385in}}%
\pgfpathlineto{\pgfqpoint{1.757556in}{0.527967in}}%
\pgfpathlineto{\pgfqpoint{1.759105in}{1.705680in}}%
\pgfpathlineto{\pgfqpoint{1.760865in}{0.679279in}}%
\pgfpathlineto{\pgfqpoint{1.762236in}{1.767933in}}%
\pgfpathlineto{\pgfqpoint{1.764161in}{0.782227in}}%
\pgfpathlineto{\pgfqpoint{1.765627in}{1.852856in}}%
\pgfpathlineto{\pgfqpoint{1.767088in}{0.862864in}}%
\pgfpathlineto{\pgfqpoint{1.768598in}{1.844974in}}%
\pgfpathlineto{\pgfqpoint{1.770259in}{0.873510in}}%
\pgfpathlineto{\pgfqpoint{1.771656in}{1.908073in}}%
\pgfpathlineto{\pgfqpoint{1.773288in}{0.918472in}}%
\pgfpathlineto{\pgfqpoint{1.775674in}{1.974014in}}%
\pgfpathlineto{\pgfqpoint{1.776930in}{0.805935in}}%
\pgfpathlineto{\pgfqpoint{1.778598in}{1.998770in}}%
\pgfpathlineto{\pgfqpoint{1.779549in}{0.950131in}}%
\pgfpathlineto{\pgfqpoint{1.781589in}{1.952396in}}%
\pgfpathlineto{\pgfqpoint{1.782732in}{0.842927in}}%
\pgfpathlineto{\pgfqpoint{1.784384in}{1.958900in}}%
\pgfpathlineto{\pgfqpoint{1.785953in}{0.904875in}}%
\pgfpathlineto{\pgfqpoint{1.787726in}{1.975008in}}%
\pgfpathlineto{\pgfqpoint{1.789185in}{0.922324in}}%
\pgfpathlineto{\pgfqpoint{1.790653in}{1.891627in}}%
\pgfpathlineto{\pgfqpoint{1.793126in}{0.824462in}}%
\pgfpathlineto{\pgfqpoint{1.794247in}{1.968825in}}%
\pgfpathlineto{\pgfqpoint{1.795420in}{0.879146in}}%
\pgfpathlineto{\pgfqpoint{1.796881in}{1.894621in}}%
\pgfpathlineto{\pgfqpoint{1.798692in}{0.777695in}}%
\pgfpathlineto{\pgfqpoint{1.800230in}{1.885634in}}%
\pgfpathlineto{\pgfqpoint{1.801820in}{0.701946in}}%
\pgfpathlineto{\pgfqpoint{1.803356in}{1.991422in}}%
\pgfpathlineto{\pgfqpoint{1.805031in}{0.741897in}}%
\pgfpathlineto{\pgfqpoint{1.806450in}{1.938304in}}%
\pgfpathlineto{\pgfqpoint{1.808410in}{0.800050in}}%
\pgfpathlineto{\pgfqpoint{1.809509in}{1.896080in}}%
\pgfpathlineto{\pgfqpoint{1.811293in}{0.791523in}}%
\pgfpathlineto{\pgfqpoint{1.813076in}{1.896254in}}%
\pgfpathlineto{\pgfqpoint{1.815017in}{0.787439in}}%
\pgfpathlineto{\pgfqpoint{1.816364in}{1.907700in}}%
\pgfpathlineto{\pgfqpoint{1.817586in}{0.860888in}}%
\pgfpathlineto{\pgfqpoint{1.819296in}{1.912395in}}%
\pgfpathlineto{\pgfqpoint{1.820582in}{0.961216in}}%
\pgfpathlineto{\pgfqpoint{1.822661in}{1.928296in}}%
\pgfpathlineto{\pgfqpoint{1.823923in}{0.753581in}}%
\pgfpathlineto{\pgfqpoint{1.825416in}{1.731558in}}%
\pgfpathlineto{\pgfqpoint{1.826930in}{0.815027in}}%
\pgfpathlineto{\pgfqpoint{1.828611in}{1.820568in}}%
\pgfpathlineto{\pgfqpoint{1.830665in}{0.756344in}}%
\pgfpathlineto{\pgfqpoint{1.831696in}{1.880940in}}%
\pgfpathlineto{\pgfqpoint{1.833347in}{0.606567in}}%
\pgfpathlineto{\pgfqpoint{1.835031in}{1.828728in}}%
\pgfpathlineto{\pgfqpoint{1.836880in}{0.664543in}}%
\pgfpathlineto{\pgfqpoint{1.838002in}{1.738591in}}%
\pgfpathlineto{\pgfqpoint{1.839402in}{0.901868in}}%
\pgfpathlineto{\pgfqpoint{1.841083in}{1.679947in}}%
\pgfpathlineto{\pgfqpoint{1.842837in}{0.758634in}}%
\pgfpathlineto{\pgfqpoint{1.844164in}{1.786667in}}%
\pgfpathlineto{\pgfqpoint{1.846233in}{0.783125in}}%
\pgfpathlineto{\pgfqpoint{1.847406in}{1.882752in}}%
\pgfpathlineto{\pgfqpoint{1.848880in}{0.889059in}}%
\pgfpathlineto{\pgfqpoint{1.850677in}{1.912759in}}%
\pgfpathlineto{\pgfqpoint{1.852202in}{0.687741in}}%
\pgfpathlineto{\pgfqpoint{1.853677in}{1.777390in}}%
\pgfpathlineto{\pgfqpoint{1.855449in}{0.879178in}}%
\pgfpathlineto{\pgfqpoint{1.857589in}{1.811074in}}%
\pgfpathlineto{\pgfqpoint{1.858607in}{0.793450in}}%
\pgfpathlineto{\pgfqpoint{1.860147in}{1.692219in}}%
\pgfpathlineto{\pgfqpoint{1.862457in}{0.549535in}}%
\pgfpathlineto{\pgfqpoint{1.863180in}{1.799617in}}%
\pgfpathlineto{\pgfqpoint{1.864957in}{0.839390in}}%
\pgfpathlineto{\pgfqpoint{1.866223in}{1.892176in}}%
\pgfpathlineto{\pgfqpoint{1.867849in}{0.779054in}}%
\pgfpathlineto{\pgfqpoint{1.870339in}{1.814231in}}%
\pgfpathlineto{\pgfqpoint{1.871548in}{0.610906in}}%
\pgfpathlineto{\pgfqpoint{1.872528in}{1.754108in}}%
\pgfpathlineto{\pgfqpoint{1.874649in}{0.789229in}}%
\pgfpathlineto{\pgfqpoint{1.875763in}{1.964462in}}%
\pgfpathlineto{\pgfqpoint{1.877400in}{0.962867in}}%
\pgfpathlineto{\pgfqpoint{1.879442in}{1.881064in}}%
\pgfpathlineto{\pgfqpoint{1.880590in}{0.809670in}}%
\pgfpathlineto{\pgfqpoint{1.881959in}{1.730116in}}%
\pgfpathlineto{\pgfqpoint{1.883530in}{0.781878in}}%
\pgfpathlineto{\pgfqpoint{1.885571in}{1.812566in}}%
\pgfpathlineto{\pgfqpoint{1.886898in}{0.842379in}}%
\pgfpathlineto{\pgfqpoint{1.889741in}{1.974186in}}%
\pgfpathlineto{\pgfqpoint{1.889851in}{0.967233in}}%
\pgfpathlineto{\pgfqpoint{1.892317in}{1.782198in}}%
\pgfpathlineto{\pgfqpoint{1.893296in}{0.722460in}}%
\pgfpathlineto{\pgfqpoint{1.895001in}{1.871690in}}%
\pgfpathlineto{\pgfqpoint{1.896275in}{0.671062in}}%
\pgfpathlineto{\pgfqpoint{1.898104in}{1.785928in}}%
\pgfpathlineto{\pgfqpoint{1.899388in}{0.739553in}}%
\pgfpathlineto{\pgfqpoint{1.901194in}{1.983040in}}%
\pgfpathlineto{\pgfqpoint{1.902715in}{0.812580in}}%
\pgfpathlineto{\pgfqpoint{1.904475in}{1.879767in}}%
\pgfpathlineto{\pgfqpoint{1.905838in}{0.840787in}}%
\pgfpathlineto{\pgfqpoint{1.907765in}{1.776755in}}%
\pgfpathlineto{\pgfqpoint{1.908964in}{0.727353in}}%
\pgfpathlineto{\pgfqpoint{1.910355in}{1.808431in}}%
\pgfpathlineto{\pgfqpoint{1.912474in}{0.859544in}}%
\pgfpathlineto{\pgfqpoint{1.914554in}{2.175986in}}%
\pgfpathlineto{\pgfqpoint{1.915268in}{0.983861in}}%
\pgfpathlineto{\pgfqpoint{1.917390in}{1.993038in}}%
\pgfpathlineto{\pgfqpoint{1.918824in}{0.880237in}}%
\pgfpathlineto{\pgfqpoint{1.920036in}{1.968811in}}%
\pgfpathlineto{\pgfqpoint{1.921789in}{0.781695in}}%
\pgfpathlineto{\pgfqpoint{1.923456in}{2.017152in}}%
\pgfpathlineto{\pgfqpoint{1.924635in}{0.898720in}}%
\pgfpathlineto{\pgfqpoint{1.926343in}{1.825386in}}%
\pgfpathlineto{\pgfqpoint{1.927725in}{0.792009in}}%
\pgfpathlineto{\pgfqpoint{1.929406in}{1.802211in}}%
\pgfpathlineto{\pgfqpoint{1.930917in}{0.872013in}}%
\pgfpathlineto{\pgfqpoint{1.932522in}{1.859251in}}%
\pgfpathlineto{\pgfqpoint{1.934105in}{0.899585in}}%
\pgfpathlineto{\pgfqpoint{1.935650in}{2.015920in}}%
\pgfpathlineto{\pgfqpoint{1.937111in}{0.893049in}}%
\pgfpathlineto{\pgfqpoint{1.938691in}{1.846367in}}%
\pgfpathlineto{\pgfqpoint{1.940550in}{0.902705in}}%
\pgfpathlineto{\pgfqpoint{1.942116in}{1.932707in}}%
\pgfpathlineto{\pgfqpoint{1.943684in}{0.776912in}}%
\pgfpathlineto{\pgfqpoint{1.945853in}{1.875176in}}%
\pgfpathlineto{\pgfqpoint{1.946580in}{0.939502in}}%
\pgfpathlineto{\pgfqpoint{1.948495in}{1.931785in}}%
\pgfpathlineto{\pgfqpoint{1.949832in}{0.881738in}}%
\pgfpathlineto{\pgfqpoint{1.951326in}{1.921549in}}%
\pgfpathlineto{\pgfqpoint{1.953080in}{0.913792in}}%
\pgfpathlineto{\pgfqpoint{1.955130in}{1.916155in}}%
\pgfpathlineto{\pgfqpoint{1.956022in}{0.762946in}}%
\pgfpathlineto{\pgfqpoint{1.957626in}{1.741814in}}%
\pgfpathlineto{\pgfqpoint{1.959538in}{0.706109in}}%
\pgfpathlineto{\pgfqpoint{1.960889in}{1.739799in}}%
\pgfpathlineto{\pgfqpoint{1.963031in}{0.890354in}}%
\pgfpathlineto{\pgfqpoint{1.963922in}{1.859940in}}%
\pgfpathlineto{\pgfqpoint{1.965915in}{0.726477in}}%
\pgfpathlineto{\pgfqpoint{1.967986in}{1.842991in}}%
\pgfpathlineto{\pgfqpoint{1.968852in}{0.661125in}}%
\pgfpathlineto{\pgfqpoint{1.970313in}{1.853603in}}%
\pgfpathlineto{\pgfqpoint{1.971892in}{0.844551in}}%
\pgfpathlineto{\pgfqpoint{1.974489in}{1.868523in}}%
\pgfpathlineto{\pgfqpoint{1.975162in}{0.727956in}}%
\pgfpathlineto{\pgfqpoint{1.976959in}{1.908037in}}%
\pgfpathlineto{\pgfqpoint{1.978758in}{0.748498in}}%
\pgfpathlineto{\pgfqpoint{1.979910in}{1.810368in}}%
\pgfpathlineto{\pgfqpoint{1.981796in}{0.661667in}}%
\pgfpathlineto{\pgfqpoint{1.982814in}{1.882298in}}%
\pgfpathlineto{\pgfqpoint{1.984502in}{0.899076in}}%
\pgfpathlineto{\pgfqpoint{1.986758in}{1.796490in}}%
\pgfpathlineto{\pgfqpoint{1.988132in}{0.646228in}}%
\pgfpathlineto{\pgfqpoint{1.989661in}{1.760980in}}%
\pgfpathlineto{\pgfqpoint{1.991160in}{0.735209in}}%
\pgfpathlineto{\pgfqpoint{1.992481in}{1.780781in}}%
\pgfpathlineto{\pgfqpoint{1.994168in}{0.699548in}}%
\pgfpathlineto{\pgfqpoint{1.995509in}{1.752808in}}%
\pgfpathlineto{\pgfqpoint{1.997481in}{0.713063in}}%
\pgfpathlineto{\pgfqpoint{1.998833in}{1.749512in}}%
\pgfpathlineto{\pgfqpoint{2.000594in}{0.661876in}}%
\pgfpathlineto{\pgfqpoint{2.001979in}{1.783880in}}%
\pgfpathlineto{\pgfqpoint{2.003818in}{0.698414in}}%
\pgfpathlineto{\pgfqpoint{2.004971in}{1.822549in}}%
\pgfpathlineto{\pgfqpoint{2.006639in}{0.894533in}}%
\pgfpathlineto{\pgfqpoint{2.008481in}{1.859646in}}%
\pgfpathlineto{\pgfqpoint{2.009703in}{0.766003in}}%
\pgfpathlineto{\pgfqpoint{2.011890in}{1.804141in}}%
\pgfpathlineto{\pgfqpoint{2.012761in}{0.809290in}}%
\pgfpathlineto{\pgfqpoint{2.014370in}{1.766144in}}%
\pgfpathlineto{\pgfqpoint{2.016560in}{0.864192in}}%
\pgfpathlineto{\pgfqpoint{2.017511in}{1.980913in}}%
\pgfpathlineto{\pgfqpoint{2.019069in}{0.903169in}}%
\pgfpathlineto{\pgfqpoint{2.020649in}{1.852942in}}%
\pgfpathlineto{\pgfqpoint{2.022258in}{0.847958in}}%
\pgfpathlineto{\pgfqpoint{2.024571in}{1.810423in}}%
\pgfpathlineto{\pgfqpoint{2.025747in}{0.560162in}}%
\pgfpathlineto{\pgfqpoint{2.027728in}{1.731171in}}%
\pgfpathlineto{\pgfqpoint{2.028897in}{0.648227in}}%
\pgfpathlineto{\pgfqpoint{2.031541in}{1.782285in}}%
\pgfpathlineto{\pgfqpoint{2.031668in}{0.800789in}}%
\pgfpathlineto{\pgfqpoint{2.033468in}{1.749055in}}%
\pgfpathlineto{\pgfqpoint{2.034813in}{0.704511in}}%
\pgfpathlineto{\pgfqpoint{2.037132in}{1.801870in}}%
\pgfpathlineto{\pgfqpoint{2.038350in}{0.835884in}}%
\pgfpathlineto{\pgfqpoint{2.040256in}{1.815464in}}%
\pgfpathlineto{\pgfqpoint{2.041388in}{0.708210in}}%
\pgfpathlineto{\pgfqpoint{2.043028in}{1.746113in}}%
\pgfpathlineto{\pgfqpoint{2.044798in}{0.666718in}}%
\pgfpathlineto{\pgfqpoint{2.045887in}{1.720516in}}%
\pgfpathlineto{\pgfqpoint{2.047507in}{0.691616in}}%
\pgfpathlineto{\pgfqpoint{2.049139in}{1.752452in}}%
\pgfpathlineto{\pgfqpoint{2.051586in}{0.667280in}}%
\pgfpathlineto{\pgfqpoint{2.052174in}{1.805757in}}%
\pgfpathlineto{\pgfqpoint{2.054789in}{0.717825in}}%
\pgfpathlineto{\pgfqpoint{2.055397in}{1.869903in}}%
\pgfpathlineto{\pgfqpoint{2.056959in}{0.797677in}}%
\pgfpathlineto{\pgfqpoint{2.059542in}{1.797373in}}%
\pgfpathlineto{\pgfqpoint{2.060200in}{0.732709in}}%
\pgfpathlineto{\pgfqpoint{2.061763in}{1.734032in}}%
\pgfpathlineto{\pgfqpoint{2.063337in}{0.586695in}}%
\pgfpathlineto{\pgfqpoint{2.065176in}{1.828189in}}%
\pgfpathlineto{\pgfqpoint{2.066341in}{0.706746in}}%
\pgfpathlineto{\pgfqpoint{2.068179in}{1.695737in}}%
\pgfpathlineto{\pgfqpoint{2.069688in}{0.690694in}}%
\pgfpathlineto{\pgfqpoint{2.071250in}{1.725666in}}%
\pgfpathlineto{\pgfqpoint{2.072650in}{0.792339in}}%
\pgfpathlineto{\pgfqpoint{2.074453in}{1.766531in}}%
\pgfpathlineto{\pgfqpoint{2.075896in}{0.813077in}}%
\pgfpathlineto{\pgfqpoint{2.077544in}{1.771234in}}%
\pgfpathlineto{\pgfqpoint{2.079288in}{0.632059in}}%
\pgfpathlineto{\pgfqpoint{2.080610in}{1.804002in}}%
\pgfpathlineto{\pgfqpoint{2.082480in}{0.800550in}}%
\pgfpathlineto{\pgfqpoint{2.084454in}{1.797561in}}%
\pgfpathlineto{\pgfqpoint{2.085854in}{0.710961in}}%
\pgfpathlineto{\pgfqpoint{2.086986in}{1.779511in}}%
\pgfpathlineto{\pgfqpoint{2.089308in}{0.726583in}}%
\pgfpathlineto{\pgfqpoint{2.090679in}{1.834410in}}%
\pgfpathlineto{\pgfqpoint{2.091632in}{0.612328in}}%
\pgfpathlineto{\pgfqpoint{2.093927in}{1.744584in}}%
\pgfpathlineto{\pgfqpoint{2.094906in}{0.718374in}}%
\pgfpathlineto{\pgfqpoint{2.096408in}{1.667398in}}%
\pgfpathlineto{\pgfqpoint{2.098188in}{0.775381in}}%
\pgfpathlineto{\pgfqpoint{2.099558in}{1.760887in}}%
\pgfpathlineto{\pgfqpoint{2.101014in}{0.794343in}}%
\pgfpathlineto{\pgfqpoint{2.103401in}{1.947618in}}%
\pgfpathlineto{\pgfqpoint{2.104297in}{0.871215in}}%
\pgfpathlineto{\pgfqpoint{2.105855in}{1.855166in}}%
\pgfpathlineto{\pgfqpoint{2.107315in}{0.877023in}}%
\pgfpathlineto{\pgfqpoint{2.109837in}{1.938543in}}%
\pgfpathlineto{\pgfqpoint{2.110574in}{0.728083in}}%
\pgfpathlineto{\pgfqpoint{2.112127in}{1.716590in}}%
\pgfpathlineto{\pgfqpoint{2.113706in}{0.850447in}}%
\pgfpathlineto{\pgfqpoint{2.115198in}{1.777831in}}%
\pgfpathlineto{\pgfqpoint{2.117444in}{0.782523in}}%
\pgfpathlineto{\pgfqpoint{2.119348in}{1.849050in}}%
\pgfpathlineto{\pgfqpoint{2.120110in}{0.891636in}}%
\pgfpathlineto{\pgfqpoint{2.121590in}{1.880412in}}%
\pgfpathlineto{\pgfqpoint{2.123075in}{0.675135in}}%
\pgfpathlineto{\pgfqpoint{2.125638in}{1.866370in}}%
\pgfpathlineto{\pgfqpoint{2.126543in}{0.690626in}}%
\pgfpathlineto{\pgfqpoint{2.128123in}{1.948179in}}%
\pgfpathlineto{\pgfqpoint{2.129853in}{0.898085in}}%
\pgfpathlineto{\pgfqpoint{2.131035in}{1.901538in}}%
\pgfpathlineto{\pgfqpoint{2.132537in}{0.866555in}}%
\pgfpathlineto{\pgfqpoint{2.134117in}{1.802235in}}%
\pgfpathlineto{\pgfqpoint{2.135869in}{0.711891in}}%
\pgfpathlineto{\pgfqpoint{2.137252in}{1.853062in}}%
\pgfpathlineto{\pgfqpoint{2.139327in}{0.834983in}}%
\pgfpathlineto{\pgfqpoint{2.140468in}{1.835131in}}%
\pgfpathlineto{\pgfqpoint{2.142332in}{0.868001in}}%
\pgfpathlineto{\pgfqpoint{2.143630in}{1.952593in}}%
\pgfpathlineto{\pgfqpoint{2.145238in}{0.724745in}}%
\pgfpathlineto{\pgfqpoint{2.146782in}{1.862453in}}%
\pgfpathlineto{\pgfqpoint{2.148345in}{0.806904in}}%
\pgfpathlineto{\pgfqpoint{2.150115in}{1.869903in}}%
\pgfpathlineto{\pgfqpoint{2.152217in}{0.587611in}}%
\pgfpathlineto{\pgfqpoint{2.153203in}{1.812108in}}%
\pgfpathlineto{\pgfqpoint{2.154647in}{0.953174in}}%
\pgfpathlineto{\pgfqpoint{2.156213in}{1.981117in}}%
\pgfpathlineto{\pgfqpoint{2.157888in}{0.620821in}}%
\pgfpathlineto{\pgfqpoint{2.159335in}{1.834408in}}%
\pgfpathlineto{\pgfqpoint{2.162325in}{0.700998in}}%
\pgfpathlineto{\pgfqpoint{2.162478in}{1.757406in}}%
\pgfpathlineto{\pgfqpoint{2.164164in}{0.740487in}}%
\pgfpathlineto{\pgfqpoint{2.165820in}{1.688856in}}%
\pgfpathlineto{\pgfqpoint{2.167636in}{0.722798in}}%
\pgfpathlineto{\pgfqpoint{2.168956in}{1.671496in}}%
\pgfpathlineto{\pgfqpoint{2.171048in}{0.700921in}}%
\pgfpathlineto{\pgfqpoint{2.171979in}{1.665848in}}%
\pgfpathlineto{\pgfqpoint{2.173747in}{0.736217in}}%
\pgfpathlineto{\pgfqpoint{2.175743in}{1.654914in}}%
\pgfpathlineto{\pgfqpoint{2.176663in}{0.713974in}}%
\pgfpathlineto{\pgfqpoint{2.178278in}{1.724544in}}%
\pgfpathlineto{\pgfqpoint{2.180228in}{0.626882in}}%
\pgfpathlineto{\pgfqpoint{2.182188in}{1.830378in}}%
\pgfpathlineto{\pgfqpoint{2.183118in}{0.763529in}}%
\pgfpathlineto{\pgfqpoint{2.184609in}{1.775002in}}%
\pgfpathlineto{\pgfqpoint{2.186967in}{0.700086in}}%
\pgfpathlineto{\pgfqpoint{2.188134in}{1.730854in}}%
\pgfpathlineto{\pgfqpoint{2.189845in}{0.643746in}}%
\pgfpathlineto{\pgfqpoint{2.190990in}{1.703756in}}%
\pgfpathlineto{\pgfqpoint{2.192903in}{0.633200in}}%
\pgfpathlineto{\pgfqpoint{2.193984in}{1.724325in}}%
\pgfpathlineto{\pgfqpoint{2.195642in}{0.767551in}}%
\pgfpathlineto{\pgfqpoint{2.197252in}{1.840183in}}%
\pgfpathlineto{\pgfqpoint{2.198838in}{0.719374in}}%
\pgfpathlineto{\pgfqpoint{2.200356in}{1.833459in}}%
\pgfpathlineto{\pgfqpoint{2.202319in}{0.643152in}}%
\pgfpathlineto{\pgfqpoint{2.203629in}{1.846192in}}%
\pgfpathlineto{\pgfqpoint{2.205229in}{0.824220in}}%
\pgfpathlineto{\pgfqpoint{2.206712in}{1.773210in}}%
\pgfpathlineto{\pgfqpoint{2.208285in}{0.698656in}}%
\pgfpathlineto{\pgfqpoint{2.210427in}{1.747835in}}%
\pgfpathlineto{\pgfqpoint{2.212457in}{0.569722in}}%
\pgfpathlineto{\pgfqpoint{2.213696in}{1.895229in}}%
\pgfpathlineto{\pgfqpoint{2.214515in}{0.767437in}}%
\pgfpathlineto{\pgfqpoint{2.216456in}{1.906142in}}%
\pgfpathlineto{\pgfqpoint{2.217635in}{0.909433in}}%
\pgfpathlineto{\pgfqpoint{2.219636in}{1.882316in}}%
\pgfpathlineto{\pgfqpoint{2.221109in}{0.890447in}}%
\pgfpathlineto{\pgfqpoint{2.222343in}{1.777691in}}%
\pgfpathlineto{\pgfqpoint{2.224493in}{0.714213in}}%
\pgfpathlineto{\pgfqpoint{2.225613in}{1.820070in}}%
\pgfpathlineto{\pgfqpoint{2.227818in}{0.862185in}}%
\pgfpathlineto{\pgfqpoint{2.228671in}{1.795384in}}%
\pgfpathlineto{\pgfqpoint{2.230738in}{0.740398in}}%
\pgfpathlineto{\pgfqpoint{2.231991in}{1.803933in}}%
\pgfpathlineto{\pgfqpoint{2.233652in}{0.677074in}}%
\pgfpathlineto{\pgfqpoint{2.235019in}{1.653896in}}%
\pgfpathlineto{\pgfqpoint{2.236574in}{0.699601in}}%
\pgfpathlineto{\pgfqpoint{2.238646in}{1.685737in}}%
\pgfpathlineto{\pgfqpoint{2.239910in}{0.670211in}}%
\pgfpathlineto{\pgfqpoint{2.241319in}{1.597599in}}%
\pgfpathlineto{\pgfqpoint{2.243067in}{0.759102in}}%
\pgfpathlineto{\pgfqpoint{2.244726in}{1.749423in}}%
\pgfpathlineto{\pgfqpoint{2.246183in}{0.741906in}}%
\pgfpathlineto{\pgfqpoint{2.247729in}{1.789142in}}%
\pgfpathlineto{\pgfqpoint{2.249206in}{0.777537in}}%
\pgfpathlineto{\pgfqpoint{2.250779in}{1.738636in}}%
\pgfpathlineto{\pgfqpoint{2.252658in}{0.561905in}}%
\pgfpathlineto{\pgfqpoint{2.253862in}{1.623240in}}%
\pgfpathlineto{\pgfqpoint{2.256164in}{0.756721in}}%
\pgfpathlineto{\pgfqpoint{2.257430in}{1.788236in}}%
\pgfpathlineto{\pgfqpoint{2.258630in}{0.739963in}}%
\pgfpathlineto{\pgfqpoint{2.261355in}{1.810350in}}%
\pgfpathlineto{\pgfqpoint{2.263241in}{0.544346in}}%
\pgfpathlineto{\pgfqpoint{2.263528in}{1.581184in}}%
\pgfpathlineto{\pgfqpoint{2.265002in}{0.795326in}}%
\pgfpathlineto{\pgfqpoint{2.266868in}{1.787379in}}%
\pgfpathlineto{\pgfqpoint{2.269115in}{0.751598in}}%
\pgfpathlineto{\pgfqpoint{2.270274in}{1.909999in}}%
\pgfpathlineto{\pgfqpoint{2.271332in}{0.743440in}}%
\pgfpathlineto{\pgfqpoint{2.273145in}{1.822481in}}%
\pgfpathlineto{\pgfqpoint{2.274591in}{0.811812in}}%
\pgfpathlineto{\pgfqpoint{2.275986in}{1.744530in}}%
\pgfpathlineto{\pgfqpoint{2.278032in}{0.806613in}}%
\pgfpathlineto{\pgfqpoint{2.279297in}{1.840017in}}%
\pgfpathlineto{\pgfqpoint{2.280795in}{0.756464in}}%
\pgfpathlineto{\pgfqpoint{2.282242in}{1.813015in}}%
\pgfpathlineto{\pgfqpoint{2.284004in}{0.703054in}}%
\pgfpathlineto{\pgfqpoint{2.285916in}{1.883292in}}%
\pgfpathlineto{\pgfqpoint{2.287111in}{0.775065in}}%
\pgfpathlineto{\pgfqpoint{2.288669in}{1.707930in}}%
\pgfpathlineto{\pgfqpoint{2.290173in}{0.517696in}}%
\pgfpathlineto{\pgfqpoint{2.291799in}{1.791983in}}%
\pgfpathlineto{\pgfqpoint{2.293580in}{0.780011in}}%
\pgfpathlineto{\pgfqpoint{2.295200in}{1.792905in}}%
\pgfpathlineto{\pgfqpoint{2.296465in}{0.737005in}}%
\pgfpathlineto{\pgfqpoint{2.298573in}{1.701254in}}%
\pgfpathlineto{\pgfqpoint{2.299794in}{0.640506in}}%
\pgfpathlineto{\pgfqpoint{2.301689in}{1.748803in}}%
\pgfpathlineto{\pgfqpoint{2.303074in}{0.633382in}}%
\pgfpathlineto{\pgfqpoint{2.304745in}{1.905837in}}%
\pgfpathlineto{\pgfqpoint{2.306007in}{0.706975in}}%
\pgfpathlineto{\pgfqpoint{2.307602in}{1.785375in}}%
\pgfpathlineto{\pgfqpoint{2.309080in}{0.748482in}}%
\pgfpathlineto{\pgfqpoint{2.311227in}{1.718946in}}%
\pgfpathlineto{\pgfqpoint{2.312815in}{0.645308in}}%
\pgfpathlineto{\pgfqpoint{2.313971in}{1.811802in}}%
\pgfpathlineto{\pgfqpoint{2.315713in}{0.844113in}}%
\pgfpathlineto{\pgfqpoint{2.317082in}{1.737039in}}%
\pgfpathlineto{\pgfqpoint{2.318872in}{0.669881in}}%
\pgfpathlineto{\pgfqpoint{2.320127in}{1.938058in}}%
\pgfpathlineto{\pgfqpoint{2.321888in}{0.743325in}}%
\pgfpathlineto{\pgfqpoint{2.323466in}{1.786497in}}%
\pgfpathlineto{\pgfqpoint{2.325251in}{0.826282in}}%
\pgfpathlineto{\pgfqpoint{2.326665in}{1.734642in}}%
\pgfpathlineto{\pgfqpoint{2.328419in}{0.726358in}}%
\pgfpathlineto{\pgfqpoint{2.329754in}{1.696607in}}%
\pgfpathlineto{\pgfqpoint{2.331094in}{0.721138in}}%
\pgfpathlineto{\pgfqpoint{2.333655in}{1.797268in}}%
\pgfpathlineto{\pgfqpoint{2.334644in}{0.598171in}}%
\pgfpathlineto{\pgfqpoint{2.336352in}{1.699264in}}%
\pgfpathlineto{\pgfqpoint{2.337744in}{0.535951in}}%
\pgfpathlineto{\pgfqpoint{2.339273in}{1.771718in}}%
\pgfpathlineto{\pgfqpoint{2.340589in}{0.758514in}}%
\pgfpathlineto{\pgfqpoint{2.343047in}{1.879138in}}%
\pgfpathlineto{\pgfqpoint{2.344189in}{0.662703in}}%
\pgfpathlineto{\pgfqpoint{2.345543in}{1.907914in}}%
\pgfpathlineto{\pgfqpoint{2.346897in}{0.746740in}}%
\pgfpathlineto{\pgfqpoint{2.348631in}{1.741263in}}%
\pgfpathlineto{\pgfqpoint{2.350547in}{0.677409in}}%
\pgfpathlineto{\pgfqpoint{2.351838in}{1.806847in}}%
\pgfpathlineto{\pgfqpoint{2.353294in}{0.832784in}}%
\pgfpathlineto{\pgfqpoint{2.355395in}{1.827095in}}%
\pgfpathlineto{\pgfqpoint{2.357040in}{0.675584in}}%
\pgfpathlineto{\pgfqpoint{2.358016in}{1.823951in}}%
\pgfpathlineto{\pgfqpoint{2.359447in}{0.789607in}}%
\pgfpathlineto{\pgfqpoint{2.361144in}{1.956807in}}%
\pgfpathlineto{\pgfqpoint{2.363048in}{0.755476in}}%
\pgfpathlineto{\pgfqpoint{2.364287in}{1.868536in}}%
\pgfpathlineto{\pgfqpoint{2.365831in}{0.942820in}}%
\pgfpathlineto{\pgfqpoint{2.368018in}{1.868632in}}%
\pgfpathlineto{\pgfqpoint{2.369182in}{0.779408in}}%
\pgfpathlineto{\pgfqpoint{2.370563in}{1.773972in}}%
\pgfpathlineto{\pgfqpoint{2.372150in}{0.845869in}}%
\pgfpathlineto{\pgfqpoint{2.374136in}{1.922917in}}%
\pgfpathlineto{\pgfqpoint{2.375485in}{0.880265in}}%
\pgfpathlineto{\pgfqpoint{2.377315in}{2.004554in}}%
\pgfpathlineto{\pgfqpoint{2.379021in}{0.763534in}}%
\pgfpathlineto{\pgfqpoint{2.379955in}{1.829007in}}%
\pgfpathlineto{\pgfqpoint{2.381724in}{0.806209in}}%
\pgfpathlineto{\pgfqpoint{2.383765in}{1.825768in}}%
\pgfpathlineto{\pgfqpoint{2.384673in}{0.733782in}}%
\pgfpathlineto{\pgfqpoint{2.386440in}{1.836888in}}%
\pgfpathlineto{\pgfqpoint{2.388023in}{0.774906in}}%
\pgfpathlineto{\pgfqpoint{2.389837in}{1.938467in}}%
\pgfpathlineto{\pgfqpoint{2.391015in}{0.866390in}}%
\pgfpathlineto{\pgfqpoint{2.392583in}{1.750847in}}%
\pgfpathlineto{\pgfqpoint{2.394502in}{0.797130in}}%
\pgfpathlineto{\pgfqpoint{2.396038in}{1.791509in}}%
\pgfpathlineto{\pgfqpoint{2.397664in}{0.638569in}}%
\pgfpathlineto{\pgfqpoint{2.398856in}{1.635981in}}%
\pgfpathlineto{\pgfqpoint{2.400781in}{0.633072in}}%
\pgfpathlineto{\pgfqpoint{2.402144in}{1.699083in}}%
\pgfpathlineto{\pgfqpoint{2.404417in}{0.688214in}}%
\pgfpathlineto{\pgfqpoint{2.405490in}{1.727162in}}%
\pgfpathlineto{\pgfqpoint{2.406742in}{0.758718in}}%
\pgfpathlineto{\pgfqpoint{2.408335in}{1.633547in}}%
\pgfpathlineto{\pgfqpoint{2.409956in}{0.716917in}}%
\pgfpathlineto{\pgfqpoint{2.411732in}{1.684772in}}%
\pgfpathlineto{\pgfqpoint{2.413858in}{0.726025in}}%
\pgfpathlineto{\pgfqpoint{2.414677in}{1.816917in}}%
\pgfpathlineto{\pgfqpoint{2.416220in}{0.769854in}}%
\pgfpathlineto{\pgfqpoint{2.418625in}{1.827126in}}%
\pgfpathlineto{\pgfqpoint{2.419477in}{0.670491in}}%
\pgfpathlineto{\pgfqpoint{2.421226in}{1.746567in}}%
\pgfpathlineto{\pgfqpoint{2.422654in}{0.621867in}}%
\pgfpathlineto{\pgfqpoint{2.424547in}{1.752024in}}%
\pgfpathlineto{\pgfqpoint{2.425804in}{0.733087in}}%
\pgfpathlineto{\pgfqpoint{2.427363in}{1.747543in}}%
\pgfpathlineto{\pgfqpoint{2.428811in}{0.839098in}}%
\pgfpathlineto{\pgfqpoint{2.430792in}{1.708564in}}%
\pgfpathlineto{\pgfqpoint{2.432993in}{0.626256in}}%
\pgfpathlineto{\pgfqpoint{2.433620in}{1.744573in}}%
\pgfpathlineto{\pgfqpoint{2.435181in}{0.696625in}}%
\pgfpathlineto{\pgfqpoint{2.437265in}{1.794758in}}%
\pgfpathlineto{\pgfqpoint{2.438386in}{0.847022in}}%
\pgfpathlineto{\pgfqpoint{2.440562in}{1.858383in}}%
\pgfpathlineto{\pgfqpoint{2.441697in}{0.805367in}}%
\pgfpathlineto{\pgfqpoint{2.443216in}{1.816316in}}%
\pgfpathlineto{\pgfqpoint{2.445634in}{0.718614in}}%
\pgfpathlineto{\pgfqpoint{2.446376in}{1.707729in}}%
\pgfpathlineto{\pgfqpoint{2.448335in}{0.700669in}}%
\pgfpathlineto{\pgfqpoint{2.449399in}{1.790923in}}%
\pgfpathlineto{\pgfqpoint{2.451098in}{0.667267in}}%
\pgfpathlineto{\pgfqpoint{2.452473in}{1.754247in}}%
\pgfpathlineto{\pgfqpoint{2.454763in}{0.787167in}}%
\pgfpathlineto{\pgfqpoint{2.455592in}{1.753376in}}%
\pgfpathlineto{\pgfqpoint{2.457279in}{0.575548in}}%
\pgfpathlineto{\pgfqpoint{2.459437in}{1.870859in}}%
\pgfpathlineto{\pgfqpoint{2.460331in}{0.863504in}}%
\pgfpathlineto{\pgfqpoint{2.463076in}{1.924681in}}%
\pgfpathlineto{\pgfqpoint{2.463544in}{0.830366in}}%
\pgfpathlineto{\pgfqpoint{2.465440in}{1.694651in}}%
\pgfpathlineto{\pgfqpoint{2.467376in}{0.639440in}}%
\pgfpathlineto{\pgfqpoint{2.469117in}{1.960690in}}%
\pgfpathlineto{\pgfqpoint{2.469847in}{0.856532in}}%
\pgfpathlineto{\pgfqpoint{2.471340in}{1.811391in}}%
\pgfpathlineto{\pgfqpoint{2.473087in}{0.757628in}}%
\pgfpathlineto{\pgfqpoint{2.474709in}{1.751182in}}%
\pgfpathlineto{\pgfqpoint{2.476675in}{0.701285in}}%
\pgfpathlineto{\pgfqpoint{2.477791in}{1.806670in}}%
\pgfpathlineto{\pgfqpoint{2.479427in}{0.733556in}}%
\pgfpathlineto{\pgfqpoint{2.481012in}{1.830296in}}%
\pgfpathlineto{\pgfqpoint{2.482680in}{0.760309in}}%
\pgfpathlineto{\pgfqpoint{2.484202in}{1.871104in}}%
\pgfpathlineto{\pgfqpoint{2.485928in}{0.778180in}}%
\pgfpathlineto{\pgfqpoint{2.487100in}{1.798724in}}%
\pgfpathlineto{\pgfqpoint{2.488855in}{0.880888in}}%
\pgfpathlineto{\pgfqpoint{2.490276in}{1.731695in}}%
\pgfpathlineto{\pgfqpoint{2.493222in}{0.547766in}}%
\pgfpathlineto{\pgfqpoint{2.493491in}{1.688704in}}%
\pgfpathlineto{\pgfqpoint{2.495377in}{0.748813in}}%
\pgfpathlineto{\pgfqpoint{2.497064in}{1.776994in}}%
\pgfpathlineto{\pgfqpoint{2.498974in}{0.675487in}}%
\pgfpathlineto{\pgfqpoint{2.500258in}{1.864152in}}%
\pgfpathlineto{\pgfqpoint{2.501565in}{0.804317in}}%
\pgfpathlineto{\pgfqpoint{2.502951in}{1.807421in}}%
\pgfpathlineto{\pgfqpoint{2.505685in}{0.683755in}}%
\pgfpathlineto{\pgfqpoint{2.506323in}{1.921726in}}%
\pgfpathlineto{\pgfqpoint{2.508223in}{0.667097in}}%
\pgfpathlineto{\pgfqpoint{2.509330in}{2.003355in}}%
\pgfpathlineto{\pgfqpoint{2.510863in}{0.798188in}}%
\pgfpathlineto{\pgfqpoint{2.512511in}{1.810562in}}%
\pgfpathlineto{\pgfqpoint{2.513922in}{0.847603in}}%
\pgfpathlineto{\pgfqpoint{2.515861in}{1.934886in}}%
\pgfpathlineto{\pgfqpoint{2.517425in}{0.823746in}}%
\pgfpathlineto{\pgfqpoint{2.518775in}{1.807554in}}%
\pgfpathlineto{\pgfqpoint{2.520253in}{0.929596in}}%
\pgfpathlineto{\pgfqpoint{2.522732in}{1.987362in}}%
\pgfpathlineto{\pgfqpoint{2.523503in}{0.909723in}}%
\pgfpathlineto{\pgfqpoint{2.524924in}{1.889438in}}%
\pgfpathlineto{\pgfqpoint{2.526746in}{0.879357in}}%
\pgfpathlineto{\pgfqpoint{2.528260in}{1.787584in}}%
\pgfpathlineto{\pgfqpoint{2.530283in}{0.804106in}}%
\pgfpathlineto{\pgfqpoint{2.531314in}{1.792639in}}%
\pgfpathlineto{\pgfqpoint{2.533514in}{0.708199in}}%
\pgfpathlineto{\pgfqpoint{2.534483in}{1.764706in}}%
\pgfpathlineto{\pgfqpoint{2.535958in}{0.751052in}}%
\pgfpathlineto{\pgfqpoint{2.537599in}{1.877886in}}%
\pgfpathlineto{\pgfqpoint{2.539251in}{0.742161in}}%
\pgfpathlineto{\pgfqpoint{2.540786in}{1.713452in}}%
\pgfpathlineto{\pgfqpoint{2.542594in}{0.752498in}}%
\pgfpathlineto{\pgfqpoint{2.544312in}{1.975217in}}%
\pgfpathlineto{\pgfqpoint{2.545519in}{0.604986in}}%
\pgfpathlineto{\pgfqpoint{2.546991in}{1.855326in}}%
\pgfpathlineto{\pgfqpoint{2.548976in}{0.720903in}}%
\pgfpathlineto{\pgfqpoint{2.550169in}{1.765331in}}%
\pgfpathlineto{\pgfqpoint{2.552302in}{0.836147in}}%
\pgfpathlineto{\pgfqpoint{2.553617in}{1.787548in}}%
\pgfpathlineto{\pgfqpoint{2.555051in}{0.574456in}}%
\pgfpathlineto{\pgfqpoint{2.556965in}{1.796102in}}%
\pgfpathlineto{\pgfqpoint{2.558037in}{0.831216in}}%
\pgfpathlineto{\pgfqpoint{2.559716in}{1.742552in}}%
\pgfpathlineto{\pgfqpoint{2.561376in}{0.586287in}}%
\pgfpathlineto{\pgfqpoint{2.563566in}{1.676887in}}%
\pgfpathlineto{\pgfqpoint{2.564493in}{0.558908in}}%
\pgfpathlineto{\pgfqpoint{2.566227in}{1.632156in}}%
\pgfpathlineto{\pgfqpoint{2.567477in}{0.676292in}}%
\pgfpathlineto{\pgfqpoint{2.569619in}{1.685992in}}%
\pgfpathlineto{\pgfqpoint{2.570975in}{0.544607in}}%
\pgfpathlineto{\pgfqpoint{2.572814in}{1.746417in}}%
\pgfpathlineto{\pgfqpoint{2.573862in}{0.761455in}}%
\pgfpathlineto{\pgfqpoint{2.575917in}{1.835031in}}%
\pgfpathlineto{\pgfqpoint{2.577996in}{0.643766in}}%
\pgfpathlineto{\pgfqpoint{2.578904in}{1.905405in}}%
\pgfpathlineto{\pgfqpoint{2.580135in}{0.829372in}}%
\pgfpathlineto{\pgfqpoint{2.582512in}{1.840965in}}%
\pgfpathlineto{\pgfqpoint{2.583438in}{0.645416in}}%
\pgfpathlineto{\pgfqpoint{2.584971in}{1.725707in}}%
\pgfpathlineto{\pgfqpoint{2.586476in}{0.518875in}}%
\pgfpathlineto{\pgfqpoint{2.588170in}{1.784425in}}%
\pgfpathlineto{\pgfqpoint{2.589698in}{0.780520in}}%
\pgfpathlineto{\pgfqpoint{2.591131in}{1.665977in}}%
\pgfpathlineto{\pgfqpoint{2.592791in}{0.710307in}}%
\pgfpathlineto{\pgfqpoint{2.594539in}{1.804510in}}%
\pgfpathlineto{\pgfqpoint{2.595831in}{0.842876in}}%
\pgfpathlineto{\pgfqpoint{2.597513in}{1.841864in}}%
\pgfpathlineto{\pgfqpoint{2.599274in}{0.723193in}}%
\pgfpathlineto{\pgfqpoint{2.600578in}{1.753347in}}%
\pgfpathlineto{\pgfqpoint{2.602156in}{0.869126in}}%
\pgfpathlineto{\pgfqpoint{2.604566in}{1.860298in}}%
\pgfpathlineto{\pgfqpoint{2.605543in}{0.768293in}}%
\pgfpathlineto{\pgfqpoint{2.606905in}{1.791671in}}%
\pgfpathlineto{\pgfqpoint{2.608874in}{0.864731in}}%
\pgfpathlineto{\pgfqpoint{2.610236in}{1.854719in}}%
\pgfpathlineto{\pgfqpoint{2.611696in}{0.897755in}}%
\pgfpathlineto{\pgfqpoint{2.613354in}{1.843576in}}%
\pgfpathlineto{\pgfqpoint{2.614919in}{0.810620in}}%
\pgfpathlineto{\pgfqpoint{2.617158in}{1.782902in}}%
\pgfpathlineto{\pgfqpoint{2.618179in}{0.781534in}}%
\pgfpathlineto{\pgfqpoint{2.620170in}{1.845780in}}%
\pgfpathlineto{\pgfqpoint{2.621059in}{0.797061in}}%
\pgfpathlineto{\pgfqpoint{2.622791in}{1.959676in}}%
\pgfpathlineto{\pgfqpoint{2.624623in}{0.875542in}}%
\pgfpathlineto{\pgfqpoint{2.625859in}{1.873929in}}%
\pgfpathlineto{\pgfqpoint{2.627390in}{0.831628in}}%
\pgfpathlineto{\pgfqpoint{2.629241in}{1.872405in}}%
\pgfpathlineto{\pgfqpoint{2.630582in}{0.785343in}}%
\pgfpathlineto{\pgfqpoint{2.632077in}{1.698261in}}%
\pgfpathlineto{\pgfqpoint{2.633701in}{0.785654in}}%
\pgfpathlineto{\pgfqpoint{2.635999in}{1.774351in}}%
\pgfpathlineto{\pgfqpoint{2.636958in}{0.726400in}}%
\pgfpathlineto{\pgfqpoint{2.639137in}{1.868649in}}%
\pgfpathlineto{\pgfqpoint{2.640180in}{0.700850in}}%
\pgfpathlineto{\pgfqpoint{2.641760in}{1.966695in}}%
\pgfpathlineto{\pgfqpoint{2.643106in}{0.875789in}}%
\pgfpathlineto{\pgfqpoint{2.645740in}{1.823637in}}%
\pgfpathlineto{\pgfqpoint{2.646489in}{0.668783in}}%
\pgfpathlineto{\pgfqpoint{2.648063in}{1.828354in}}%
\pgfpathlineto{\pgfqpoint{2.649646in}{0.843641in}}%
\pgfpathlineto{\pgfqpoint{2.651033in}{1.831684in}}%
\pgfpathlineto{\pgfqpoint{2.653069in}{0.586438in}}%
\pgfpathlineto{\pgfqpoint{2.654150in}{1.702862in}}%
\pgfpathlineto{\pgfqpoint{2.655815in}{0.674494in}}%
\pgfpathlineto{\pgfqpoint{2.657406in}{1.934451in}}%
\pgfpathlineto{\pgfqpoint{2.658899in}{0.799796in}}%
\pgfpathlineto{\pgfqpoint{2.660543in}{1.673956in}}%
\pgfpathlineto{\pgfqpoint{2.662326in}{0.705845in}}%
\pgfpathlineto{\pgfqpoint{2.663635in}{1.726717in}}%
\pgfpathlineto{\pgfqpoint{2.665240in}{0.680172in}}%
\pgfpathlineto{\pgfqpoint{2.667269in}{1.865462in}}%
\pgfpathlineto{\pgfqpoint{2.668824in}{0.660336in}}%
\pgfpathlineto{\pgfqpoint{2.670937in}{1.931813in}}%
\pgfpathlineto{\pgfqpoint{2.671696in}{0.751590in}}%
\pgfpathlineto{\pgfqpoint{2.673048in}{1.730188in}}%
\pgfpathlineto{\pgfqpoint{2.675227in}{0.562705in}}%
\pgfpathlineto{\pgfqpoint{2.676759in}{1.760498in}}%
\pgfpathlineto{\pgfqpoint{2.677836in}{0.682198in}}%
\pgfpathlineto{\pgfqpoint{2.679648in}{1.852214in}}%
\pgfpathlineto{\pgfqpoint{2.680986in}{0.816509in}}%
\pgfpathlineto{\pgfqpoint{2.683117in}{1.820568in}}%
\pgfpathlineto{\pgfqpoint{2.684499in}{0.612133in}}%
\pgfpathlineto{\pgfqpoint{2.686242in}{1.781771in}}%
\pgfpathlineto{\pgfqpoint{2.687290in}{0.797188in}}%
\pgfpathlineto{\pgfqpoint{2.689627in}{1.947011in}}%
\pgfpathlineto{\pgfqpoint{2.690390in}{0.958344in}}%
\pgfpathlineto{\pgfqpoint{2.691984in}{1.763940in}}%
\pgfpathlineto{\pgfqpoint{2.693683in}{0.806346in}}%
\pgfpathlineto{\pgfqpoint{2.695176in}{1.933666in}}%
\pgfpathlineto{\pgfqpoint{2.696783in}{0.803936in}}%
\pgfpathlineto{\pgfqpoint{2.698856in}{1.869983in}}%
\pgfpathlineto{\pgfqpoint{2.699948in}{0.820675in}}%
\pgfpathlineto{\pgfqpoint{2.701556in}{1.892150in}}%
\pgfpathlineto{\pgfqpoint{2.703691in}{0.859558in}}%
\pgfpathlineto{\pgfqpoint{2.704655in}{1.873191in}}%
\pgfpathlineto{\pgfqpoint{2.706246in}{0.877079in}}%
\pgfpathlineto{\pgfqpoint{2.708370in}{1.855069in}}%
\pgfpathlineto{\pgfqpoint{2.709332in}{0.907335in}}%
\pgfpathlineto{\pgfqpoint{2.711037in}{1.821155in}}%
\pgfpathlineto{\pgfqpoint{2.712845in}{0.885336in}}%
\pgfpathlineto{\pgfqpoint{2.714320in}{1.918441in}}%
\pgfpathlineto{\pgfqpoint{2.715752in}{0.806075in}}%
\pgfpathlineto{\pgfqpoint{2.717828in}{1.939335in}}%
\pgfpathlineto{\pgfqpoint{2.719382in}{0.805928in}}%
\pgfpathlineto{\pgfqpoint{2.720496in}{1.893800in}}%
\pgfpathlineto{\pgfqpoint{2.722133in}{0.794957in}}%
\pgfpathlineto{\pgfqpoint{2.723663in}{1.821384in}}%
\pgfpathlineto{\pgfqpoint{2.725194in}{0.872376in}}%
\pgfpathlineto{\pgfqpoint{2.726910in}{1.885371in}}%
\pgfpathlineto{\pgfqpoint{2.728378in}{0.917546in}}%
\pgfpathlineto{\pgfqpoint{2.729868in}{1.833870in}}%
\pgfpathlineto{\pgfqpoint{2.731475in}{0.816915in}}%
\pgfpathlineto{\pgfqpoint{2.732995in}{1.863475in}}%
\pgfpathlineto{\pgfqpoint{2.734646in}{0.881949in}}%
\pgfpathlineto{\pgfqpoint{2.736808in}{1.923535in}}%
\pgfpathlineto{\pgfqpoint{2.737869in}{0.691681in}}%
\pgfpathlineto{\pgfqpoint{2.739292in}{1.731584in}}%
\pgfpathlineto{\pgfqpoint{2.741442in}{0.679831in}}%
\pgfpathlineto{\pgfqpoint{2.742937in}{1.828375in}}%
\pgfpathlineto{\pgfqpoint{2.744080in}{0.756928in}}%
\pgfpathlineto{\pgfqpoint{2.745764in}{1.868965in}}%
\pgfpathlineto{\pgfqpoint{2.747850in}{0.857335in}}%
\pgfpathlineto{\pgfqpoint{2.749010in}{1.787534in}}%
\pgfpathlineto{\pgfqpoint{2.751061in}{0.728652in}}%
\pgfpathlineto{\pgfqpoint{2.751926in}{2.041155in}}%
\pgfpathlineto{\pgfqpoint{2.753555in}{0.737133in}}%
\pgfpathlineto{\pgfqpoint{2.755007in}{1.853826in}}%
\pgfpathlineto{\pgfqpoint{2.756840in}{0.885513in}}%
\pgfpathlineto{\pgfqpoint{2.758838in}{1.832008in}}%
\pgfpathlineto{\pgfqpoint{2.759785in}{0.868194in}}%
\pgfpathlineto{\pgfqpoint{2.761478in}{1.814644in}}%
\pgfpathlineto{\pgfqpoint{2.763511in}{0.711895in}}%
\pgfpathlineto{\pgfqpoint{2.764679in}{1.780208in}}%
\pgfpathlineto{\pgfqpoint{2.766493in}{0.753833in}}%
\pgfpathlineto{\pgfqpoint{2.768843in}{2.062974in}}%
\pgfpathlineto{\pgfqpoint{2.769265in}{0.759175in}}%
\pgfpathlineto{\pgfqpoint{2.770922in}{1.731734in}}%
\pgfpathlineto{\pgfqpoint{2.772528in}{0.690582in}}%
\pgfpathlineto{\pgfqpoint{2.773926in}{1.760760in}}%
\pgfpathlineto{\pgfqpoint{2.775909in}{0.553081in}}%
\pgfpathlineto{\pgfqpoint{2.777116in}{1.677071in}}%
\pgfpathlineto{\pgfqpoint{2.778795in}{0.816429in}}%
\pgfpathlineto{\pgfqpoint{2.780270in}{1.857461in}}%
\pgfpathlineto{\pgfqpoint{2.781838in}{0.837156in}}%
\pgfpathlineto{\pgfqpoint{2.784025in}{1.920312in}}%
\pgfpathlineto{\pgfqpoint{2.785316in}{0.813555in}}%
\pgfpathlineto{\pgfqpoint{2.787081in}{1.794967in}}%
\pgfpathlineto{\pgfqpoint{2.788329in}{0.750916in}}%
\pgfpathlineto{\pgfqpoint{2.790981in}{1.934752in}}%
\pgfpathlineto{\pgfqpoint{2.791241in}{0.816964in}}%
\pgfpathlineto{\pgfqpoint{2.792986in}{1.773469in}}%
\pgfpathlineto{\pgfqpoint{2.794615in}{0.732276in}}%
\pgfpathlineto{\pgfqpoint{2.795988in}{1.767739in}}%
\pgfpathlineto{\pgfqpoint{2.797649in}{0.659434in}}%
\pgfpathlineto{\pgfqpoint{2.799416in}{1.694836in}}%
\pgfpathlineto{\pgfqpoint{2.800800in}{0.654197in}}%
\pgfpathlineto{\pgfqpoint{2.802942in}{1.825180in}}%
\pgfpathlineto{\pgfqpoint{2.804404in}{0.697199in}}%
\pgfpathlineto{\pgfqpoint{2.805444in}{1.718579in}}%
\pgfpathlineto{\pgfqpoint{2.807439in}{0.768192in}}%
\pgfpathlineto{\pgfqpoint{2.808709in}{1.815470in}}%
\pgfpathlineto{\pgfqpoint{2.810309in}{0.766980in}}%
\pgfpathlineto{\pgfqpoint{2.812877in}{1.938108in}}%
\pgfpathlineto{\pgfqpoint{2.814354in}{0.710293in}}%
\pgfpathlineto{\pgfqpoint{2.815200in}{1.805023in}}%
\pgfpathlineto{\pgfqpoint{2.816483in}{0.847781in}}%
\pgfpathlineto{\pgfqpoint{2.818754in}{1.872069in}}%
\pgfpathlineto{\pgfqpoint{2.819628in}{0.852654in}}%
\pgfpathlineto{\pgfqpoint{2.821976in}{1.787735in}}%
\pgfpathlineto{\pgfqpoint{2.823432in}{0.721361in}}%
\pgfpathlineto{\pgfqpoint{2.824488in}{1.987050in}}%
\pgfpathlineto{\pgfqpoint{2.826031in}{0.816737in}}%
\pgfpathlineto{\pgfqpoint{2.827520in}{1.827347in}}%
\pgfpathlineto{\pgfqpoint{2.829599in}{0.874460in}}%
\pgfpathlineto{\pgfqpoint{2.830865in}{1.983881in}}%
\pgfpathlineto{\pgfqpoint{2.832689in}{0.995212in}}%
\pgfpathlineto{\pgfqpoint{2.833817in}{1.861594in}}%
\pgfpathlineto{\pgfqpoint{2.835466in}{0.761906in}}%
\pgfpathlineto{\pgfqpoint{2.836951in}{1.845656in}}%
\pgfpathlineto{\pgfqpoint{2.838638in}{0.780767in}}%
\pgfpathlineto{\pgfqpoint{2.840330in}{1.958217in}}%
\pgfpathlineto{\pgfqpoint{2.842320in}{0.881678in}}%
\pgfpathlineto{\pgfqpoint{2.843307in}{1.841713in}}%
\pgfpathlineto{\pgfqpoint{2.845281in}{0.665197in}}%
\pgfpathlineto{\pgfqpoint{2.847326in}{1.942035in}}%
\pgfpathlineto{\pgfqpoint{2.848027in}{0.959924in}}%
\pgfpathlineto{\pgfqpoint{2.849724in}{1.855666in}}%
\pgfpathlineto{\pgfqpoint{2.851491in}{0.722209in}}%
\pgfpathlineto{\pgfqpoint{2.853347in}{1.938320in}}%
\pgfpathlineto{\pgfqpoint{2.854349in}{0.906344in}}%
\pgfpathlineto{\pgfqpoint{2.857384in}{1.973337in}}%
\pgfpathlineto{\pgfqpoint{2.857985in}{0.732211in}}%
\pgfpathlineto{\pgfqpoint{2.859260in}{1.961500in}}%
\pgfpathlineto{\pgfqpoint{2.860579in}{0.829920in}}%
\pgfpathlineto{\pgfqpoint{2.862276in}{1.720501in}}%
\pgfpathlineto{\pgfqpoint{2.863766in}{0.680650in}}%
\pgfpathlineto{\pgfqpoint{2.865379in}{1.862256in}}%
\pgfpathlineto{\pgfqpoint{2.866889in}{0.858672in}}%
\pgfpathlineto{\pgfqpoint{2.869529in}{2.053283in}}%
\pgfpathlineto{\pgfqpoint{2.870040in}{0.949189in}}%
\pgfpathlineto{\pgfqpoint{2.872803in}{1.961345in}}%
\pgfpathlineto{\pgfqpoint{2.873621in}{0.791531in}}%
\pgfpathlineto{\pgfqpoint{2.874789in}{1.915646in}}%
\pgfpathlineto{\pgfqpoint{2.876973in}{0.839737in}}%
\pgfpathlineto{\pgfqpoint{2.878071in}{1.746433in}}%
\pgfpathlineto{\pgfqpoint{2.879937in}{0.791207in}}%
\pgfpathlineto{\pgfqpoint{2.881139in}{1.807416in}}%
\pgfpathlineto{\pgfqpoint{2.882907in}{0.859651in}}%
\pgfpathlineto{\pgfqpoint{2.884300in}{1.915645in}}%
\pgfpathlineto{\pgfqpoint{2.885824in}{0.846218in}}%
\pgfpathlineto{\pgfqpoint{2.887574in}{1.782942in}}%
\pgfpathlineto{\pgfqpoint{2.888953in}{0.744088in}}%
\pgfpathlineto{\pgfqpoint{2.890632in}{1.902908in}}%
\pgfpathlineto{\pgfqpoint{2.892743in}{0.921590in}}%
\pgfpathlineto{\pgfqpoint{2.895234in}{1.971622in}}%
\pgfpathlineto{\pgfqpoint{2.895824in}{0.781094in}}%
\pgfpathlineto{\pgfqpoint{2.896895in}{1.796182in}}%
\pgfpathlineto{\pgfqpoint{2.898400in}{0.764917in}}%
\pgfpathlineto{\pgfqpoint{2.900357in}{1.839686in}}%
\pgfpathlineto{\pgfqpoint{2.901631in}{0.851660in}}%
\pgfpathlineto{\pgfqpoint{2.903364in}{1.755428in}}%
\pgfpathlineto{\pgfqpoint{2.905689in}{0.730292in}}%
\pgfpathlineto{\pgfqpoint{2.906370in}{1.820888in}}%
\pgfpathlineto{\pgfqpoint{2.908281in}{0.756506in}}%
\pgfpathlineto{\pgfqpoint{2.909575in}{1.898404in}}%
\pgfpathlineto{\pgfqpoint{2.911915in}{0.691009in}}%
\pgfpathlineto{\pgfqpoint{2.912639in}{1.740263in}}%
\pgfpathlineto{\pgfqpoint{2.914808in}{0.792011in}}%
\pgfpathlineto{\pgfqpoint{2.915802in}{1.851133in}}%
\pgfpathlineto{\pgfqpoint{2.917709in}{0.729283in}}%
\pgfpathlineto{\pgfqpoint{2.919006in}{1.915687in}}%
\pgfpathlineto{\pgfqpoint{2.920622in}{0.884582in}}%
\pgfpathlineto{\pgfqpoint{2.922047in}{1.828097in}}%
\pgfpathlineto{\pgfqpoint{2.923965in}{0.750375in}}%
\pgfpathlineto{\pgfqpoint{2.925368in}{1.932448in}}%
\pgfpathlineto{\pgfqpoint{2.927015in}{0.779756in}}%
\pgfpathlineto{\pgfqpoint{2.928583in}{1.849186in}}%
\pgfpathlineto{\pgfqpoint{2.929946in}{0.762788in}}%
\pgfpathlineto{\pgfqpoint{2.932935in}{1.958896in}}%
\pgfpathlineto{\pgfqpoint{2.933142in}{0.872702in}}%
\pgfpathlineto{\pgfqpoint{2.935055in}{1.845593in}}%
\pgfpathlineto{\pgfqpoint{2.936409in}{0.860406in}}%
\pgfpathlineto{\pgfqpoint{2.938103in}{1.792762in}}%
\pgfpathlineto{\pgfqpoint{2.939455in}{0.786907in}}%
\pgfpathlineto{\pgfqpoint{2.941032in}{1.813028in}}%
\pgfpathlineto{\pgfqpoint{2.942707in}{0.749076in}}%
\pgfpathlineto{\pgfqpoint{2.944144in}{1.714546in}}%
\pgfpathlineto{\pgfqpoint{2.946481in}{0.619357in}}%
\pgfpathlineto{\pgfqpoint{2.947511in}{1.835505in}}%
\pgfpathlineto{\pgfqpoint{2.948944in}{0.778409in}}%
\pgfpathlineto{\pgfqpoint{2.951444in}{1.896995in}}%
\pgfpathlineto{\pgfqpoint{2.952377in}{0.768450in}}%
\pgfpathlineto{\pgfqpoint{2.953670in}{2.001969in}}%
\pgfpathlineto{\pgfqpoint{2.955168in}{0.918942in}}%
\pgfpathlineto{\pgfqpoint{2.957749in}{2.008098in}}%
\pgfpathlineto{\pgfqpoint{2.958385in}{0.893161in}}%
\pgfpathlineto{\pgfqpoint{2.960165in}{1.702449in}}%
\pgfpathlineto{\pgfqpoint{2.961454in}{0.815014in}}%
\pgfpathlineto{\pgfqpoint{2.963021in}{1.760583in}}%
\pgfpathlineto{\pgfqpoint{2.965241in}{0.854097in}}%
\pgfpathlineto{\pgfqpoint{2.966297in}{1.881260in}}%
\pgfpathlineto{\pgfqpoint{2.967835in}{0.767203in}}%
\pgfpathlineto{\pgfqpoint{2.969425in}{1.842273in}}%
\pgfpathlineto{\pgfqpoint{2.971716in}{0.742177in}}%
\pgfpathlineto{\pgfqpoint{2.972640in}{1.776407in}}%
\pgfpathlineto{\pgfqpoint{2.974270in}{0.802580in}}%
\pgfpathlineto{\pgfqpoint{2.976630in}{1.885184in}}%
\pgfpathlineto{\pgfqpoint{2.977297in}{0.767348in}}%
\pgfpathlineto{\pgfqpoint{2.979136in}{1.922502in}}%
\pgfpathlineto{\pgfqpoint{2.981706in}{0.651666in}}%
\pgfpathlineto{\pgfqpoint{2.981928in}{1.940785in}}%
\pgfpathlineto{\pgfqpoint{2.983625in}{0.999973in}}%
\pgfpathlineto{\pgfqpoint{2.985089in}{1.924879in}}%
\pgfpathlineto{\pgfqpoint{2.986655in}{0.944882in}}%
\pgfpathlineto{\pgfqpoint{2.989133in}{2.045629in}}%
\pgfpathlineto{\pgfqpoint{2.990069in}{0.879407in}}%
\pgfpathlineto{\pgfqpoint{2.991383in}{1.757948in}}%
\pgfpathlineto{\pgfqpoint{2.994458in}{0.775661in}}%
\pgfpathlineto{\pgfqpoint{2.994566in}{1.836202in}}%
\pgfpathlineto{\pgfqpoint{2.996378in}{0.707173in}}%
\pgfpathlineto{\pgfqpoint{2.998391in}{1.869116in}}%
\pgfpathlineto{\pgfqpoint{2.999413in}{0.806221in}}%
\pgfpathlineto{\pgfqpoint{3.001128in}{1.838701in}}%
\pgfpathlineto{\pgfqpoint{3.003033in}{0.795182in}}%
\pgfpathlineto{\pgfqpoint{3.004059in}{1.851951in}}%
\pgfpathlineto{\pgfqpoint{3.006222in}{0.755614in}}%
\pgfpathlineto{\pgfqpoint{3.007532in}{1.892869in}}%
\pgfpathlineto{\pgfqpoint{3.009295in}{0.706101in}}%
\pgfpathlineto{\pgfqpoint{3.010357in}{1.673614in}}%
\pgfpathlineto{\pgfqpoint{3.011901in}{0.770420in}}%
\pgfpathlineto{\pgfqpoint{3.013728in}{1.907193in}}%
\pgfpathlineto{\pgfqpoint{3.016172in}{0.750259in}}%
\pgfpathlineto{\pgfqpoint{3.016713in}{1.738033in}}%
\pgfpathlineto{\pgfqpoint{3.019408in}{0.730714in}}%
\pgfpathlineto{\pgfqpoint{3.019858in}{1.827796in}}%
\pgfpathlineto{\pgfqpoint{3.021884in}{0.825970in}}%
\pgfpathlineto{\pgfqpoint{3.022923in}{1.841280in}}%
\pgfpathlineto{\pgfqpoint{3.024637in}{0.989979in}}%
\pgfpathlineto{\pgfqpoint{3.026081in}{2.016569in}}%
\pgfpathlineto{\pgfqpoint{3.027853in}{0.983422in}}%
\pgfpathlineto{\pgfqpoint{3.029272in}{1.953004in}}%
\pgfpathlineto{\pgfqpoint{3.031391in}{0.880405in}}%
\pgfpathlineto{\pgfqpoint{3.032575in}{1.840337in}}%
\pgfpathlineto{\pgfqpoint{3.034663in}{0.781803in}}%
\pgfpathlineto{\pgfqpoint{3.035664in}{1.931889in}}%
\pgfpathlineto{\pgfqpoint{3.038058in}{0.761806in}}%
\pgfpathlineto{\pgfqpoint{3.038787in}{1.809297in}}%
\pgfpathlineto{\pgfqpoint{3.040849in}{0.721277in}}%
\pgfpathlineto{\pgfqpoint{3.041945in}{1.770691in}}%
\pgfpathlineto{\pgfqpoint{3.043534in}{0.832599in}}%
\pgfpathlineto{\pgfqpoint{3.045251in}{1.851001in}}%
\pgfpathlineto{\pgfqpoint{3.046580in}{0.816661in}}%
\pgfpathlineto{\pgfqpoint{3.048109in}{1.726349in}}%
\pgfpathlineto{\pgfqpoint{3.049818in}{0.800168in}}%
\pgfpathlineto{\pgfqpoint{3.052661in}{2.042039in}}%
\pgfpathlineto{\pgfqpoint{3.052835in}{1.017660in}}%
\pgfpathlineto{\pgfqpoint{3.055252in}{1.875329in}}%
\pgfpathlineto{\pgfqpoint{3.056587in}{0.848354in}}%
\pgfpathlineto{\pgfqpoint{3.057940in}{2.017757in}}%
\pgfpathlineto{\pgfqpoint{3.059365in}{0.813552in}}%
\pgfpathlineto{\pgfqpoint{3.060776in}{1.911441in}}%
\pgfpathlineto{\pgfqpoint{3.062803in}{0.903613in}}%
\pgfpathlineto{\pgfqpoint{3.064276in}{1.862553in}}%
\pgfpathlineto{\pgfqpoint{3.066138in}{0.721097in}}%
\pgfpathlineto{\pgfqpoint{3.067218in}{1.942072in}}%
\pgfpathlineto{\pgfqpoint{3.068958in}{0.827853in}}%
\pgfpathlineto{\pgfqpoint{3.070225in}{1.891218in}}%
\pgfpathlineto{\pgfqpoint{3.072256in}{0.861544in}}%
\pgfpathlineto{\pgfqpoint{3.073347in}{1.845964in}}%
\pgfpathlineto{\pgfqpoint{3.075298in}{0.991484in}}%
\pgfpathlineto{\pgfqpoint{3.076854in}{2.065940in}}%
\pgfpathlineto{\pgfqpoint{3.078132in}{1.029725in}}%
\pgfpathlineto{\pgfqpoint{3.079717in}{1.998221in}}%
\pgfpathlineto{\pgfqpoint{3.081456in}{0.902908in}}%
\pgfpathlineto{\pgfqpoint{3.084074in}{2.048893in}}%
\pgfpathlineto{\pgfqpoint{3.084573in}{1.021124in}}%
\pgfpathlineto{\pgfqpoint{3.086028in}{1.917137in}}%
\pgfpathlineto{\pgfqpoint{3.087840in}{0.725023in}}%
\pgfpathlineto{\pgfqpoint{3.089460in}{1.949845in}}%
\pgfpathlineto{\pgfqpoint{3.090708in}{0.849944in}}%
\pgfpathlineto{\pgfqpoint{3.092260in}{1.833000in}}%
\pgfpathlineto{\pgfqpoint{3.093830in}{0.863470in}}%
\pgfpathlineto{\pgfqpoint{3.095508in}{1.813112in}}%
\pgfpathlineto{\pgfqpoint{3.097228in}{0.885099in}}%
\pgfpathlineto{\pgfqpoint{3.098641in}{1.790460in}}%
\pgfpathlineto{\pgfqpoint{3.100316in}{0.815032in}}%
\pgfpathlineto{\pgfqpoint{3.101723in}{1.710369in}}%
\pgfpathlineto{\pgfqpoint{3.103974in}{0.694057in}}%
\pgfpathlineto{\pgfqpoint{3.104973in}{1.836361in}}%
\pgfpathlineto{\pgfqpoint{3.107564in}{0.708349in}}%
\pgfpathlineto{\pgfqpoint{3.108177in}{1.848703in}}%
\pgfpathlineto{\pgfqpoint{3.109611in}{0.887098in}}%
\pgfpathlineto{\pgfqpoint{3.111420in}{1.887141in}}%
\pgfpathlineto{\pgfqpoint{3.112901in}{0.813876in}}%
\pgfpathlineto{\pgfqpoint{3.114714in}{1.889072in}}%
\pgfpathlineto{\pgfqpoint{3.116622in}{0.713609in}}%
\pgfpathlineto{\pgfqpoint{3.117790in}{1.928306in}}%
\pgfpathlineto{\pgfqpoint{3.119110in}{0.821246in}}%
\pgfpathlineto{\pgfqpoint{3.120669in}{1.908253in}}%
\pgfpathlineto{\pgfqpoint{3.123315in}{0.769197in}}%
\pgfpathlineto{\pgfqpoint{3.124416in}{1.919863in}}%
\pgfpathlineto{\pgfqpoint{3.125627in}{0.776254in}}%
\pgfpathlineto{\pgfqpoint{3.128199in}{1.990649in}}%
\pgfpathlineto{\pgfqpoint{3.128588in}{0.903192in}}%
\pgfpathlineto{\pgfqpoint{3.130880in}{1.934947in}}%
\pgfpathlineto{\pgfqpoint{3.131633in}{1.001331in}}%
\pgfpathlineto{\pgfqpoint{3.133345in}{1.995647in}}%
\pgfpathlineto{\pgfqpoint{3.134835in}{0.769280in}}%
\pgfpathlineto{\pgfqpoint{3.136673in}{1.900509in}}%
\pgfpathlineto{\pgfqpoint{3.138806in}{0.834726in}}%
\pgfpathlineto{\pgfqpoint{3.139702in}{1.906512in}}%
\pgfpathlineto{\pgfqpoint{3.141745in}{0.705341in}}%
\pgfpathlineto{\pgfqpoint{3.143125in}{1.875211in}}%
\pgfpathlineto{\pgfqpoint{3.144255in}{0.834046in}}%
\pgfpathlineto{\pgfqpoint{3.146017in}{1.816428in}}%
\pgfpathlineto{\pgfqpoint{3.147657in}{0.825195in}}%
\pgfpathlineto{\pgfqpoint{3.150083in}{2.070829in}}%
\pgfpathlineto{\pgfqpoint{3.150993in}{0.956217in}}%
\pgfpathlineto{\pgfqpoint{3.152661in}{1.884329in}}%
\pgfpathlineto{\pgfqpoint{3.153729in}{0.979740in}}%
\pgfpathlineto{\pgfqpoint{3.156131in}{1.983257in}}%
\pgfpathlineto{\pgfqpoint{3.157432in}{0.781610in}}%
\pgfpathlineto{\pgfqpoint{3.158585in}{1.854771in}}%
\pgfpathlineto{\pgfqpoint{3.160377in}{0.559546in}}%
\pgfpathlineto{\pgfqpoint{3.161838in}{1.908470in}}%
\pgfpathlineto{\pgfqpoint{3.164042in}{0.825445in}}%
\pgfpathlineto{\pgfqpoint{3.164912in}{1.743922in}}%
\pgfpathlineto{\pgfqpoint{3.166640in}{0.803237in}}%
\pgfpathlineto{\pgfqpoint{3.168818in}{2.141759in}}%
\pgfpathlineto{\pgfqpoint{3.169480in}{0.974410in}}%
\pgfpathlineto{\pgfqpoint{3.171035in}{1.828767in}}%
\pgfpathlineto{\pgfqpoint{3.172751in}{0.884084in}}%
\pgfpathlineto{\pgfqpoint{3.174368in}{2.006026in}}%
\pgfpathlineto{\pgfqpoint{3.175966in}{0.848428in}}%
\pgfpathlineto{\pgfqpoint{3.177351in}{1.999893in}}%
\pgfpathlineto{\pgfqpoint{3.178914in}{0.883057in}}%
\pgfpathlineto{\pgfqpoint{3.180923in}{2.033710in}}%
\pgfpathlineto{\pgfqpoint{3.182142in}{0.873076in}}%
\pgfpathlineto{\pgfqpoint{3.184263in}{1.805555in}}%
\pgfpathlineto{\pgfqpoint{3.185824in}{0.753034in}}%
\pgfpathlineto{\pgfqpoint{3.186846in}{1.853678in}}%
\pgfpathlineto{\pgfqpoint{3.188384in}{0.895993in}}%
\pgfpathlineto{\pgfqpoint{3.190908in}{2.040952in}}%
\pgfpathlineto{\pgfqpoint{3.191534in}{0.962005in}}%
\pgfpathlineto{\pgfqpoint{3.193341in}{1.940125in}}%
\pgfpathlineto{\pgfqpoint{3.195692in}{0.700107in}}%
\pgfpathlineto{\pgfqpoint{3.196732in}{1.874598in}}%
\pgfpathlineto{\pgfqpoint{3.198451in}{0.850884in}}%
\pgfpathlineto{\pgfqpoint{3.200119in}{1.913047in}}%
\pgfpathlineto{\pgfqpoint{3.201090in}{0.854301in}}%
\pgfpathlineto{\pgfqpoint{3.202961in}{1.937054in}}%
\pgfpathlineto{\pgfqpoint{3.204490in}{0.874366in}}%
\pgfpathlineto{\pgfqpoint{3.206283in}{1.967670in}}%
\pgfpathlineto{\pgfqpoint{3.207293in}{0.902749in}}%
\pgfpathlineto{\pgfqpoint{3.209177in}{1.892292in}}%
\pgfpathlineto{\pgfqpoint{3.210432in}{0.981128in}}%
\pgfpathlineto{\pgfqpoint{3.212644in}{2.097264in}}%
\pgfpathlineto{\pgfqpoint{3.214413in}{0.863753in}}%
\pgfpathlineto{\pgfqpoint{3.215302in}{1.948300in}}%
\pgfpathlineto{\pgfqpoint{3.216753in}{0.994721in}}%
\pgfpathlineto{\pgfqpoint{3.218359in}{1.919007in}}%
\pgfpathlineto{\pgfqpoint{3.219896in}{0.970879in}}%
\pgfpathlineto{\pgfqpoint{3.221633in}{1.836385in}}%
\pgfpathlineto{\pgfqpoint{3.223221in}{0.884594in}}%
\pgfpathlineto{\pgfqpoint{3.225885in}{2.086755in}}%
\pgfpathlineto{\pgfqpoint{3.226649in}{0.810767in}}%
\pgfpathlineto{\pgfqpoint{3.227784in}{1.831742in}}%
\pgfpathlineto{\pgfqpoint{3.230012in}{0.788917in}}%
\pgfpathlineto{\pgfqpoint{3.231005in}{1.942914in}}%
\pgfpathlineto{\pgfqpoint{3.232827in}{0.757860in}}%
\pgfpathlineto{\pgfqpoint{3.234263in}{1.940028in}}%
\pgfpathlineto{\pgfqpoint{3.235762in}{0.944856in}}%
\pgfpathlineto{\pgfqpoint{3.237231in}{1.932461in}}%
\pgfpathlineto{\pgfqpoint{3.238945in}{0.917563in}}%
\pgfpathlineto{\pgfqpoint{3.240720in}{2.109269in}}%
\pgfpathlineto{\pgfqpoint{3.242125in}{0.945373in}}%
\pgfpathlineto{\pgfqpoint{3.244348in}{2.165151in}}%
\pgfpathlineto{\pgfqpoint{3.245448in}{0.915898in}}%
\pgfpathlineto{\pgfqpoint{3.246864in}{1.874913in}}%
\pgfpathlineto{\pgfqpoint{3.248760in}{0.805963in}}%
\pgfpathlineto{\pgfqpoint{3.251341in}{2.042577in}}%
\pgfpathlineto{\pgfqpoint{3.251443in}{0.869861in}}%
\pgfpathlineto{\pgfqpoint{3.252998in}{1.792198in}}%
\pgfpathlineto{\pgfqpoint{3.255878in}{0.825213in}}%
\pgfpathlineto{\pgfqpoint{3.256287in}{1.736858in}}%
\pgfpathlineto{\pgfqpoint{3.257785in}{0.737628in}}%
\pgfpathlineto{\pgfqpoint{3.259968in}{1.793492in}}%
\pgfpathlineto{\pgfqpoint{3.261470in}{0.758178in}}%
\pgfpathlineto{\pgfqpoint{3.263171in}{1.999438in}}%
\pgfpathlineto{\pgfqpoint{3.264071in}{0.987025in}}%
\pgfpathlineto{\pgfqpoint{3.265636in}{1.898030in}}%
\pgfpathlineto{\pgfqpoint{3.267785in}{0.769598in}}%
\pgfpathlineto{\pgfqpoint{3.269194in}{1.837542in}}%
\pgfpathlineto{\pgfqpoint{3.271708in}{0.775656in}}%
\pgfpathlineto{\pgfqpoint{3.272658in}{1.946052in}}%
\pgfpathlineto{\pgfqpoint{3.273519in}{0.603856in}}%
\pgfpathlineto{\pgfqpoint{3.275158in}{1.729419in}}%
\pgfpathlineto{\pgfqpoint{3.277190in}{0.755228in}}%
\pgfpathlineto{\pgfqpoint{3.278242in}{1.838404in}}%
\pgfpathlineto{\pgfqpoint{3.280324in}{0.824973in}}%
\pgfpathlineto{\pgfqpoint{3.281388in}{1.800457in}}%
\pgfpathlineto{\pgfqpoint{3.283017in}{0.844567in}}%
\pgfpathlineto{\pgfqpoint{3.284698in}{1.932247in}}%
\pgfpathlineto{\pgfqpoint{3.287077in}{0.781800in}}%
\pgfpathlineto{\pgfqpoint{3.287650in}{1.792063in}}%
\pgfpathlineto{\pgfqpoint{3.289339in}{0.694708in}}%
\pgfpathlineto{\pgfqpoint{3.291985in}{2.076585in}}%
\pgfpathlineto{\pgfqpoint{3.292413in}{0.877531in}}%
\pgfpathlineto{\pgfqpoint{3.293947in}{1.778533in}}%
\pgfpathlineto{\pgfqpoint{3.296586in}{0.711752in}}%
\pgfpathlineto{\pgfqpoint{3.297193in}{1.747756in}}%
\pgfpathlineto{\pgfqpoint{3.299621in}{0.759987in}}%
\pgfpathlineto{\pgfqpoint{3.300259in}{1.794446in}}%
\pgfpathlineto{\pgfqpoint{3.301990in}{0.749851in}}%
\pgfpathlineto{\pgfqpoint{3.303534in}{1.980924in}}%
\pgfpathlineto{\pgfqpoint{3.306245in}{0.879655in}}%
\pgfpathlineto{\pgfqpoint{3.306813in}{1.969859in}}%
\pgfpathlineto{\pgfqpoint{3.308148in}{0.802782in}}%
\pgfpathlineto{\pgfqpoint{3.310094in}{1.811213in}}%
\pgfpathlineto{\pgfqpoint{3.311648in}{0.791640in}}%
\pgfpathlineto{\pgfqpoint{3.313135in}{1.777436in}}%
\pgfpathlineto{\pgfqpoint{3.314445in}{0.863750in}}%
\pgfpathlineto{\pgfqpoint{3.316204in}{1.826286in}}%
\pgfpathlineto{\pgfqpoint{3.317794in}{0.844247in}}%
\pgfpathlineto{\pgfqpoint{3.319234in}{1.754855in}}%
\pgfpathlineto{\pgfqpoint{3.320790in}{0.913134in}}%
\pgfpathlineto{\pgfqpoint{3.322577in}{1.831953in}}%
\pgfpathlineto{\pgfqpoint{3.324398in}{0.848811in}}%
\pgfpathlineto{\pgfqpoint{3.325899in}{1.922607in}}%
\pgfpathlineto{\pgfqpoint{3.327073in}{0.858974in}}%
\pgfpathlineto{\pgfqpoint{3.328676in}{1.866267in}}%
\pgfpathlineto{\pgfqpoint{3.331282in}{0.803974in}}%
\pgfpathlineto{\pgfqpoint{3.332147in}{1.762881in}}%
\pgfpathlineto{\pgfqpoint{3.333424in}{0.803989in}}%
\pgfpathlineto{\pgfqpoint{3.335036in}{1.815230in}}%
\pgfpathlineto{\pgfqpoint{3.336629in}{0.707895in}}%
\pgfpathlineto{\pgfqpoint{3.338479in}{1.800333in}}%
\pgfpathlineto{\pgfqpoint{3.339715in}{0.684626in}}%
\pgfpathlineto{\pgfqpoint{3.341725in}{1.842821in}}%
\pgfpathlineto{\pgfqpoint{3.342818in}{0.608030in}}%
\pgfpathlineto{\pgfqpoint{3.344911in}{1.915491in}}%
\pgfpathlineto{\pgfqpoint{3.346072in}{0.874550in}}%
\pgfpathlineto{\pgfqpoint{3.347533in}{1.689909in}}%
\pgfpathlineto{\pgfqpoint{3.349840in}{0.682863in}}%
\pgfpathlineto{\pgfqpoint{3.350938in}{1.816210in}}%
\pgfpathlineto{\pgfqpoint{3.352328in}{0.754881in}}%
\pgfpathlineto{\pgfqpoint{3.354335in}{1.737014in}}%
\pgfpathlineto{\pgfqpoint{3.355527in}{0.782015in}}%
\pgfpathlineto{\pgfqpoint{3.358001in}{2.016302in}}%
\pgfpathlineto{\pgfqpoint{3.358620in}{0.826616in}}%
\pgfpathlineto{\pgfqpoint{3.361102in}{2.007732in}}%
\pgfpathlineto{\pgfqpoint{3.361715in}{0.893856in}}%
\pgfpathlineto{\pgfqpoint{3.363544in}{1.832126in}}%
\pgfpathlineto{\pgfqpoint{3.365006in}{0.757821in}}%
\pgfpathlineto{\pgfqpoint{3.367226in}{1.812281in}}%
\pgfpathlineto{\pgfqpoint{3.368080in}{0.874526in}}%
\pgfpathlineto{\pgfqpoint{3.370355in}{1.814303in}}%
\pgfpathlineto{\pgfqpoint{3.371648in}{0.511872in}}%
\pgfpathlineto{\pgfqpoint{3.373716in}{1.936989in}}%
\pgfpathlineto{\pgfqpoint{3.374662in}{0.636709in}}%
\pgfpathlineto{\pgfqpoint{3.376133in}{1.774754in}}%
\pgfpathlineto{\pgfqpoint{3.377469in}{0.682167in}}%
\pgfpathlineto{\pgfqpoint{3.379249in}{1.626161in}}%
\pgfpathlineto{\pgfqpoint{3.381158in}{0.665089in}}%
\pgfpathlineto{\pgfqpoint{3.382270in}{1.665891in}}%
\pgfpathlineto{\pgfqpoint{3.383933in}{0.526977in}}%
\pgfpathlineto{\pgfqpoint{3.385376in}{1.664146in}}%
\pgfpathlineto{\pgfqpoint{3.386945in}{0.775551in}}%
\pgfpathlineto{\pgfqpoint{3.388667in}{1.711887in}}%
\pgfpathlineto{\pgfqpoint{3.390101in}{0.762895in}}%
\pgfpathlineto{\pgfqpoint{3.392106in}{1.842254in}}%
\pgfpathlineto{\pgfqpoint{3.393247in}{0.754376in}}%
\pgfpathlineto{\pgfqpoint{3.394944in}{1.762402in}}%
\pgfpathlineto{\pgfqpoint{3.396740in}{0.674964in}}%
\pgfpathlineto{\pgfqpoint{3.398077in}{1.617474in}}%
\pgfpathlineto{\pgfqpoint{3.400491in}{0.624038in}}%
\pgfpathlineto{\pgfqpoint{3.401895in}{1.776807in}}%
\pgfpathlineto{\pgfqpoint{3.403172in}{0.762313in}}%
\pgfpathlineto{\pgfqpoint{3.404648in}{1.806624in}}%
\pgfpathlineto{\pgfqpoint{3.405884in}{0.798709in}}%
\pgfpathlineto{\pgfqpoint{3.407758in}{1.850522in}}%
\pgfpathlineto{\pgfqpoint{3.409483in}{0.724827in}}%
\pgfpathlineto{\pgfqpoint{3.410958in}{1.752827in}}%
\pgfpathlineto{\pgfqpoint{3.412254in}{0.835996in}}%
\pgfpathlineto{\pgfqpoint{3.413853in}{1.815233in}}%
\pgfpathlineto{\pgfqpoint{3.415345in}{0.880965in}}%
\pgfpathlineto{\pgfqpoint{3.417088in}{1.885034in}}%
\pgfpathlineto{\pgfqpoint{3.418461in}{0.757393in}}%
\pgfpathlineto{\pgfqpoint{3.420041in}{1.730951in}}%
\pgfpathlineto{\pgfqpoint{3.422890in}{0.759283in}}%
\pgfpathlineto{\pgfqpoint{3.423355in}{1.801055in}}%
\pgfpathlineto{\pgfqpoint{3.425059in}{0.817116in}}%
\pgfpathlineto{\pgfqpoint{3.426789in}{1.909031in}}%
\pgfpathlineto{\pgfqpoint{3.427923in}{0.936633in}}%
\pgfpathlineto{\pgfqpoint{3.430272in}{2.082028in}}%
\pgfpathlineto{\pgfqpoint{3.431124in}{0.966094in}}%
\pgfpathlineto{\pgfqpoint{3.432789in}{1.987023in}}%
\pgfpathlineto{\pgfqpoint{3.434264in}{0.942550in}}%
\pgfpathlineto{\pgfqpoint{3.435902in}{1.777072in}}%
\pgfpathlineto{\pgfqpoint{3.437478in}{0.627000in}}%
\pgfpathlineto{\pgfqpoint{3.439286in}{1.988374in}}%
\pgfpathlineto{\pgfqpoint{3.440692in}{0.879554in}}%
\pgfpathlineto{\pgfqpoint{3.442771in}{1.999178in}}%
\pgfpathlineto{\pgfqpoint{3.443918in}{0.776575in}}%
\pgfpathlineto{\pgfqpoint{3.445252in}{1.853528in}}%
\pgfpathlineto{\pgfqpoint{3.447081in}{0.740357in}}%
\pgfpathlineto{\pgfqpoint{3.448433in}{1.704325in}}%
\pgfpathlineto{\pgfqpoint{3.450924in}{0.775807in}}%
\pgfpathlineto{\pgfqpoint{3.452047in}{1.849269in}}%
\pgfpathlineto{\pgfqpoint{3.453158in}{1.002563in}}%
\pgfpathlineto{\pgfqpoint{3.454745in}{1.861293in}}%
\pgfpathlineto{\pgfqpoint{3.457566in}{0.717118in}}%
\pgfpathlineto{\pgfqpoint{3.457844in}{1.807566in}}%
\pgfpathlineto{\pgfqpoint{3.459850in}{0.618627in}}%
\pgfpathlineto{\pgfqpoint{3.461298in}{1.769466in}}%
\pgfpathlineto{\pgfqpoint{3.462589in}{0.711877in}}%
\pgfpathlineto{\pgfqpoint{3.464498in}{1.619798in}}%
\pgfpathlineto{\pgfqpoint{3.466632in}{0.610629in}}%
\pgfpathlineto{\pgfqpoint{3.467865in}{1.777083in}}%
\pgfpathlineto{\pgfqpoint{3.469464in}{0.654094in}}%
\pgfpathlineto{\pgfqpoint{3.471098in}{1.927826in}}%
\pgfpathlineto{\pgfqpoint{3.472194in}{0.595302in}}%
\pgfpathlineto{\pgfqpoint{3.473613in}{1.740578in}}%
\pgfpathlineto{\pgfqpoint{3.475716in}{0.678478in}}%
\pgfpathlineto{\pgfqpoint{3.477486in}{1.702071in}}%
\pgfpathlineto{\pgfqpoint{3.478601in}{0.695224in}}%
\pgfpathlineto{\pgfqpoint{3.479988in}{1.764006in}}%
\pgfpathlineto{\pgfqpoint{3.481771in}{0.638639in}}%
\pgfpathlineto{\pgfqpoint{3.483232in}{1.584218in}}%
\pgfpathlineto{\pgfqpoint{3.484668in}{0.600316in}}%
\pgfpathlineto{\pgfqpoint{3.486594in}{1.717337in}}%
\pgfpathlineto{\pgfqpoint{3.487956in}{0.676985in}}%
\pgfpathlineto{\pgfqpoint{3.490653in}{1.909095in}}%
\pgfpathlineto{\pgfqpoint{3.490954in}{0.876098in}}%
\pgfpathlineto{\pgfqpoint{3.492978in}{1.874664in}}%
\pgfpathlineto{\pgfqpoint{3.494237in}{0.887612in}}%
\pgfpathlineto{\pgfqpoint{3.496468in}{1.917081in}}%
\pgfpathlineto{\pgfqpoint{3.497262in}{0.819191in}}%
\pgfpathlineto{\pgfqpoint{3.498816in}{1.804801in}}%
\pgfpathlineto{\pgfqpoint{3.500611in}{0.597553in}}%
\pgfpathlineto{\pgfqpoint{3.502306in}{1.805603in}}%
\pgfpathlineto{\pgfqpoint{3.504499in}{0.758701in}}%
\pgfpathlineto{\pgfqpoint{3.505371in}{1.904895in}}%
\pgfpathlineto{\pgfqpoint{3.507362in}{0.660025in}}%
\pgfpathlineto{\pgfqpoint{3.508612in}{1.953496in}}%
\pgfpathlineto{\pgfqpoint{3.510179in}{0.844798in}}%
\pgfpathlineto{\pgfqpoint{3.512668in}{2.039174in}}%
\pgfpathlineto{\pgfqpoint{3.513829in}{0.598965in}}%
\pgfpathlineto{\pgfqpoint{3.514590in}{1.669022in}}%
\pgfpathlineto{\pgfqpoint{3.516485in}{0.848793in}}%
\pgfpathlineto{\pgfqpoint{3.517879in}{1.924467in}}%
\pgfpathlineto{\pgfqpoint{3.519985in}{0.703549in}}%
\pgfpathlineto{\pgfqpoint{3.521085in}{1.843856in}}%
\pgfpathlineto{\pgfqpoint{3.522974in}{0.759373in}}%
\pgfpathlineto{\pgfqpoint{3.524132in}{2.111371in}}%
\pgfpathlineto{\pgfqpoint{3.525622in}{0.914869in}}%
\pgfpathlineto{\pgfqpoint{3.527215in}{1.898141in}}%
\pgfpathlineto{\pgfqpoint{3.528788in}{0.888470in}}%
\pgfpathlineto{\pgfqpoint{3.530624in}{1.877753in}}%
\pgfpathlineto{\pgfqpoint{3.532285in}{0.727260in}}%
\pgfpathlineto{\pgfqpoint{3.533839in}{1.860795in}}%
\pgfpathlineto{\pgfqpoint{3.535109in}{0.807085in}}%
\pgfpathlineto{\pgfqpoint{3.536782in}{1.925348in}}%
\pgfpathlineto{\pgfqpoint{3.538448in}{0.897228in}}%
\pgfpathlineto{\pgfqpoint{3.540435in}{1.953660in}}%
\pgfpathlineto{\pgfqpoint{3.541403in}{0.813709in}}%
\pgfpathlineto{\pgfqpoint{3.543115in}{1.729272in}}%
\pgfpathlineto{\pgfqpoint{3.544912in}{0.658147in}}%
\pgfpathlineto{\pgfqpoint{3.546418in}{1.923413in}}%
\pgfpathlineto{\pgfqpoint{3.548210in}{0.752077in}}%
\pgfpathlineto{\pgfqpoint{3.549865in}{1.750133in}}%
\pgfpathlineto{\pgfqpoint{3.551492in}{0.646483in}}%
\pgfpathlineto{\pgfqpoint{3.552687in}{1.870864in}}%
\pgfpathlineto{\pgfqpoint{3.554141in}{0.790195in}}%
\pgfpathlineto{\pgfqpoint{3.555561in}{1.853612in}}%
\pgfpathlineto{\pgfqpoint{3.557130in}{0.954050in}}%
\pgfpathlineto{\pgfqpoint{3.558736in}{1.882755in}}%
\pgfpathlineto{\pgfqpoint{3.560286in}{0.717654in}}%
\pgfpathlineto{\pgfqpoint{3.562044in}{1.745368in}}%
\pgfpathlineto{\pgfqpoint{3.564151in}{0.732123in}}%
\pgfpathlineto{\pgfqpoint{3.565086in}{1.669565in}}%
\pgfpathlineto{\pgfqpoint{3.566706in}{0.730799in}}%
\pgfpathlineto{\pgfqpoint{3.568389in}{1.776737in}}%
\pgfpathlineto{\pgfqpoint{3.570200in}{0.855482in}}%
\pgfpathlineto{\pgfqpoint{3.571557in}{1.812535in}}%
\pgfpathlineto{\pgfqpoint{3.573376in}{0.668073in}}%
\pgfpathlineto{\pgfqpoint{3.574533in}{1.709480in}}%
\pgfpathlineto{\pgfqpoint{3.577001in}{0.644982in}}%
\pgfpathlineto{\pgfqpoint{3.577699in}{1.730001in}}%
\pgfpathlineto{\pgfqpoint{3.579229in}{0.855901in}}%
\pgfpathlineto{\pgfqpoint{3.580828in}{1.709228in}}%
\pgfpathlineto{\pgfqpoint{3.582394in}{0.766898in}}%
\pgfpathlineto{\pgfqpoint{3.584532in}{1.801936in}}%
\pgfpathlineto{\pgfqpoint{3.585578in}{0.749279in}}%
\pgfpathlineto{\pgfqpoint{3.587309in}{1.753454in}}%
\pgfpathlineto{\pgfqpoint{3.589164in}{0.698792in}}%
\pgfpathlineto{\pgfqpoint{3.590369in}{1.735078in}}%
\pgfpathlineto{\pgfqpoint{3.592176in}{0.852065in}}%
\pgfpathlineto{\pgfqpoint{3.593807in}{1.879119in}}%
\pgfpathlineto{\pgfqpoint{3.595026in}{0.821055in}}%
\pgfpathlineto{\pgfqpoint{3.597171in}{1.924054in}}%
\pgfpathlineto{\pgfqpoint{3.598715in}{0.718971in}}%
\pgfpathlineto{\pgfqpoint{3.599982in}{1.786317in}}%
\pgfpathlineto{\pgfqpoint{3.601909in}{0.826698in}}%
\pgfpathlineto{\pgfqpoint{3.603248in}{1.856633in}}%
\pgfpathlineto{\pgfqpoint{3.604607in}{0.682113in}}%
\pgfpathlineto{\pgfqpoint{3.606135in}{1.739424in}}%
\pgfpathlineto{\pgfqpoint{3.607662in}{0.810904in}}%
\pgfpathlineto{\pgfqpoint{3.609476in}{1.884698in}}%
\pgfpathlineto{\pgfqpoint{3.610736in}{0.856562in}}%
\pgfpathlineto{\pgfqpoint{3.613051in}{1.843262in}}%
\pgfpathlineto{\pgfqpoint{3.613982in}{0.828223in}}%
\pgfpathlineto{\pgfqpoint{3.616184in}{1.885736in}}%
\pgfpathlineto{\pgfqpoint{3.617092in}{0.869725in}}%
\pgfpathlineto{\pgfqpoint{3.618645in}{1.853531in}}%
\pgfpathlineto{\pgfqpoint{3.620245in}{0.825099in}}%
\pgfpathlineto{\pgfqpoint{3.622580in}{1.804500in}}%
\pgfpathlineto{\pgfqpoint{3.623480in}{0.605291in}}%
\pgfpathlineto{\pgfqpoint{3.625071in}{1.643902in}}%
\pgfpathlineto{\pgfqpoint{3.627148in}{0.558035in}}%
\pgfpathlineto{\pgfqpoint{3.628290in}{1.690480in}}%
\pgfpathlineto{\pgfqpoint{3.629665in}{0.822559in}}%
\pgfpathlineto{\pgfqpoint{3.631363in}{1.801384in}}%
\pgfpathlineto{\pgfqpoint{3.632842in}{0.818415in}}%
\pgfpathlineto{\pgfqpoint{3.634918in}{1.880373in}}%
\pgfpathlineto{\pgfqpoint{3.636142in}{0.843521in}}%
\pgfpathlineto{\pgfqpoint{3.637933in}{2.000365in}}%
\pgfpathlineto{\pgfqpoint{3.639172in}{0.792819in}}%
\pgfpathlineto{\pgfqpoint{3.640762in}{1.875808in}}%
\pgfpathlineto{\pgfqpoint{3.642365in}{0.804213in}}%
\pgfpathlineto{\pgfqpoint{3.644296in}{2.044834in}}%
\pgfpathlineto{\pgfqpoint{3.645880in}{0.810513in}}%
\pgfpathlineto{\pgfqpoint{3.646961in}{1.816287in}}%
\pgfpathlineto{\pgfqpoint{3.649071in}{0.583349in}}%
\pgfpathlineto{\pgfqpoint{3.650470in}{1.848550in}}%
\pgfpathlineto{\pgfqpoint{3.651718in}{0.794739in}}%
\pgfpathlineto{\pgfqpoint{3.653584in}{1.900363in}}%
\pgfpathlineto{\pgfqpoint{3.654899in}{0.882963in}}%
\pgfpathlineto{\pgfqpoint{3.656695in}{1.862171in}}%
\pgfpathlineto{\pgfqpoint{3.658077in}{0.843984in}}%
\pgfpathlineto{\pgfqpoint{3.659975in}{1.760005in}}%
\pgfpathlineto{\pgfqpoint{3.661198in}{0.634185in}}%
\pgfpathlineto{\pgfqpoint{3.662743in}{1.707098in}}%
\pgfpathlineto{\pgfqpoint{3.664511in}{0.804010in}}%
\pgfpathlineto{\pgfqpoint{3.666010in}{1.710327in}}%
\pgfpathlineto{\pgfqpoint{3.667618in}{0.823267in}}%
\pgfpathlineto{\pgfqpoint{3.669848in}{1.825940in}}%
\pgfpathlineto{\pgfqpoint{3.670591in}{0.697989in}}%
\pgfpathlineto{\pgfqpoint{3.672756in}{1.883773in}}%
\pgfpathlineto{\pgfqpoint{3.674076in}{0.537296in}}%
\pgfpathlineto{\pgfqpoint{3.675409in}{1.710748in}}%
\pgfpathlineto{\pgfqpoint{3.678211in}{0.648686in}}%
\pgfpathlineto{\pgfqpoint{3.678747in}{1.839172in}}%
\pgfpathlineto{\pgfqpoint{3.680063in}{0.829010in}}%
\pgfpathlineto{\pgfqpoint{3.681644in}{1.943929in}}%
\pgfpathlineto{\pgfqpoint{3.684220in}{0.857797in}}%
\pgfpathlineto{\pgfqpoint{3.685034in}{1.901253in}}%
\pgfpathlineto{\pgfqpoint{3.686582in}{0.761569in}}%
\pgfpathlineto{\pgfqpoint{3.687995in}{1.836297in}}%
\pgfpathlineto{\pgfqpoint{3.689824in}{0.756343in}}%
\pgfpathlineto{\pgfqpoint{3.691071in}{1.810710in}}%
\pgfpathlineto{\pgfqpoint{3.692706in}{0.777843in}}%
\pgfpathlineto{\pgfqpoint{3.695097in}{1.851365in}}%
\pgfpathlineto{\pgfqpoint{3.695815in}{0.838932in}}%
\pgfpathlineto{\pgfqpoint{3.697391in}{1.776810in}}%
\pgfpathlineto{\pgfqpoint{3.698980in}{0.932427in}}%
\pgfpathlineto{\pgfqpoint{3.700781in}{1.864178in}}%
\pgfpathlineto{\pgfqpoint{3.702306in}{0.892343in}}%
\pgfpathlineto{\pgfqpoint{3.704245in}{1.889774in}}%
\pgfpathlineto{\pgfqpoint{3.705499in}{0.844337in}}%
\pgfpathlineto{\pgfqpoint{3.706831in}{1.810168in}}%
\pgfpathlineto{\pgfqpoint{3.708704in}{0.814659in}}%
\pgfpathlineto{\pgfqpoint{3.710452in}{1.830068in}}%
\pgfpathlineto{\pgfqpoint{3.712106in}{0.806766in}}%
\pgfpathlineto{\pgfqpoint{3.713226in}{1.833440in}}%
\pgfpathlineto{\pgfqpoint{3.715115in}{0.807980in}}%
\pgfpathlineto{\pgfqpoint{3.716462in}{1.666892in}}%
\pgfpathlineto{\pgfqpoint{3.718039in}{0.819517in}}%
\pgfpathlineto{\pgfqpoint{3.720631in}{1.887773in}}%
\pgfpathlineto{\pgfqpoint{3.721252in}{0.805672in}}%
\pgfpathlineto{\pgfqpoint{3.723001in}{1.895179in}}%
\pgfpathlineto{\pgfqpoint{3.724170in}{0.877452in}}%
\pgfpathlineto{\pgfqpoint{3.725946in}{1.778258in}}%
\pgfpathlineto{\pgfqpoint{3.728116in}{0.677521in}}%
\pgfpathlineto{\pgfqpoint{3.728915in}{1.750486in}}%
\pgfpathlineto{\pgfqpoint{3.730479in}{0.847645in}}%
\pgfpathlineto{\pgfqpoint{3.732063in}{1.730008in}}%
\pgfpathlineto{\pgfqpoint{3.733712in}{0.740403in}}%
\pgfpathlineto{\pgfqpoint{3.735297in}{1.868586in}}%
\pgfpathlineto{\pgfqpoint{3.736871in}{0.702598in}}%
\pgfpathlineto{\pgfqpoint{3.738443in}{1.741604in}}%
\pgfpathlineto{\pgfqpoint{3.741025in}{0.615297in}}%
\pgfpathlineto{\pgfqpoint{3.741614in}{1.813666in}}%
\pgfpathlineto{\pgfqpoint{3.743245in}{0.539107in}}%
\pgfpathlineto{\pgfqpoint{3.745064in}{1.676986in}}%
\pgfpathlineto{\pgfqpoint{3.746339in}{0.854380in}}%
\pgfpathlineto{\pgfqpoint{3.747979in}{1.871414in}}%
\pgfpathlineto{\pgfqpoint{3.749728in}{0.740480in}}%
\pgfpathlineto{\pgfqpoint{3.751170in}{1.802038in}}%
\pgfpathlineto{\pgfqpoint{3.752620in}{0.845670in}}%
\pgfpathlineto{\pgfqpoint{3.754340in}{1.738447in}}%
\pgfpathlineto{\pgfqpoint{3.755994in}{0.842491in}}%
\pgfpathlineto{\pgfqpoint{3.757343in}{1.838429in}}%
\pgfpathlineto{\pgfqpoint{3.758910in}{0.682301in}}%
\pgfpathlineto{\pgfqpoint{3.761142in}{1.838346in}}%
\pgfpathlineto{\pgfqpoint{3.762446in}{0.624070in}}%
\pgfpathlineto{\pgfqpoint{3.764262in}{1.811551in}}%
\pgfpathlineto{\pgfqpoint{3.765373in}{0.670880in}}%
\pgfpathlineto{\pgfqpoint{3.767335in}{1.782328in}}%
\pgfpathlineto{\pgfqpoint{3.768309in}{0.804008in}}%
\pgfpathlineto{\pgfqpoint{3.770448in}{1.728287in}}%
\pgfpathlineto{\pgfqpoint{3.771740in}{0.711231in}}%
\pgfpathlineto{\pgfqpoint{3.773288in}{1.806708in}}%
\pgfpathlineto{\pgfqpoint{3.776088in}{0.565778in}}%
\pgfpathlineto{\pgfqpoint{3.776213in}{1.677674in}}%
\pgfpathlineto{\pgfqpoint{3.777743in}{1.164265in}}%
\pgfpathlineto{\pgfqpoint{3.777743in}{1.164265in}}%
\pgfusepath{stroke}%
\end{pgfscope}%
\begin{pgfscope}%
\pgfsetrectcap%
\pgfsetmiterjoin%
\pgfsetlinewidth{0.803000pt}%
\definecolor{currentstroke}{rgb}{0.000000,0.000000,0.000000}%
\pgfsetstrokecolor{currentstroke}%
\pgfsetdash{}{0pt}%
\pgfpathmoveto{\pgfqpoint{0.471687in}{0.416447in}}%
\pgfpathlineto{\pgfqpoint{0.471687in}{2.336783in}}%
\pgfusepath{stroke}%
\end{pgfscope}%
\begin{pgfscope}%
\pgfsetrectcap%
\pgfsetmiterjoin%
\pgfsetlinewidth{0.803000pt}%
\definecolor{currentstroke}{rgb}{0.000000,0.000000,0.000000}%
\pgfsetstrokecolor{currentstroke}%
\pgfsetdash{}{0pt}%
\pgfpathmoveto{\pgfqpoint{3.935174in}{0.416447in}}%
\pgfpathlineto{\pgfqpoint{3.935174in}{2.336783in}}%
\pgfusepath{stroke}%
\end{pgfscope}%
\begin{pgfscope}%
\pgfsetrectcap%
\pgfsetmiterjoin%
\pgfsetlinewidth{0.803000pt}%
\definecolor{currentstroke}{rgb}{0.000000,0.000000,0.000000}%
\pgfsetstrokecolor{currentstroke}%
\pgfsetdash{}{0pt}%
\pgfpathmoveto{\pgfqpoint{0.471687in}{0.416447in}}%
\pgfpathlineto{\pgfqpoint{3.935174in}{0.416447in}}%
\pgfusepath{stroke}%
\end{pgfscope}%
\begin{pgfscope}%
\pgfsetrectcap%
\pgfsetmiterjoin%
\pgfsetlinewidth{0.803000pt}%
\definecolor{currentstroke}{rgb}{0.000000,0.000000,0.000000}%
\pgfsetstrokecolor{currentstroke}%
\pgfsetdash{}{0pt}%
\pgfpathmoveto{\pgfqpoint{0.471687in}{2.336783in}}%
\pgfpathlineto{\pgfqpoint{3.935174in}{2.336783in}}%
\pgfusepath{stroke}%
\end{pgfscope}%
\end{pgfpicture}%
\makeatother%
\endgroup%

    \caption{Time series data with white noise and flicker noise.}
    \label{fig:autozero_raw_time}
\end{figure}

The time domain plot of the simulation is shown in figure \ref{fig:autozero_raw_time}. The white noise component is clearly visible, while the $f^{-1}$ flicker noise can be recognized, but its strength can hardly be estimated. It was already shown in section \ref{sec:noise_example}, different types of noise have different frequency components and can be be distinguished in the frequency domain, which leads to the next approach.

The noise power spectral density shown in figure \ref{fig:autozero_raw_psd} is calculated from the time series and confirms the flicker and white noise content. The theoretical white noise floor is shown as a horizontal dashed blue line and the flicker noise as a dashed green line. The \qty{1.5}{\Hz} corner frequency, which is defined as the intersection between the $f^{-1}$ noise and the white noise floor easily identified using the those lines. It is evident, that the \qty{5}{\Hz} sampling frequency with a \qty{2.5}{\Hz} bandwidth does not allow the spectral density to fully settle to the noise floor.

\begin{figure}[hb]
    \centering
    %% Creator: Matplotlib, PGF backend
%%
%% To include the figure in your LaTeX document, write
%%   \input{<filename>.pgf}
%%
%% Make sure the required packages are loaded in your preamble
%%   \usepackage{pgf}
%%
%% Also ensure that all the required font packages are loaded; for instance,
%% the lmodern package is sometimes necessary when using math font.
%%   \usepackage{lmodern}
%%
%% Figures using additional raster images can only be included by \input if
%% they are in the same directory as the main LaTeX file. For loading figures
%% from other directories you can use the `import` package
%%   \usepackage{import}
%%
%% and then include the figures with
%%   \import{<path to file>}{<filename>.pgf}
%%
%% Matplotlib used the following preamble
%%   \usepackage{siunitx}
%%   \usepackage{fontspec}
%%   \makeatletter\@ifpackageloaded{underscore}{}{\usepackage[strings]{underscore}}\makeatother
%%
\begingroup%
\makeatletter%
\begin{pgfpicture}%
\pgfpathrectangle{\pgfpointorigin}{\pgfqpoint{4.060000in}{2.510000in}}%
\pgfusepath{use as bounding box, clip}%
\begin{pgfscope}%
\pgfsetbuttcap%
\pgfsetmiterjoin%
\definecolor{currentfill}{rgb}{1.000000,1.000000,1.000000}%
\pgfsetfillcolor{currentfill}%
\pgfsetlinewidth{0.000000pt}%
\definecolor{currentstroke}{rgb}{1.000000,1.000000,1.000000}%
\pgfsetstrokecolor{currentstroke}%
\pgfsetdash{}{0pt}%
\pgfpathmoveto{\pgfqpoint{0.000000in}{0.000000in}}%
\pgfpathlineto{\pgfqpoint{4.060000in}{0.000000in}}%
\pgfpathlineto{\pgfqpoint{4.060000in}{2.510000in}}%
\pgfpathlineto{\pgfqpoint{0.000000in}{2.510000in}}%
\pgfpathlineto{\pgfqpoint{0.000000in}{0.000000in}}%
\pgfpathclose%
\pgfusepath{fill}%
\end{pgfscope}%
\begin{pgfscope}%
\pgfsetbuttcap%
\pgfsetmiterjoin%
\definecolor{currentfill}{rgb}{1.000000,1.000000,1.000000}%
\pgfsetfillcolor{currentfill}%
\pgfsetlinewidth{0.000000pt}%
\definecolor{currentstroke}{rgb}{0.000000,0.000000,0.000000}%
\pgfsetstrokecolor{currentstroke}%
\pgfsetstrokeopacity{0.000000}%
\pgfsetdash{}{0pt}%
\pgfpathmoveto{\pgfqpoint{0.645450in}{0.417642in}}%
\pgfpathlineto{\pgfqpoint{4.018330in}{0.417642in}}%
\pgfpathlineto{\pgfqpoint{4.018330in}{2.468330in}}%
\pgfpathlineto{\pgfqpoint{0.645450in}{2.468330in}}%
\pgfpathlineto{\pgfqpoint{0.645450in}{0.417642in}}%
\pgfpathclose%
\pgfusepath{fill}%
\end{pgfscope}%
\begin{pgfscope}%
\pgfpathrectangle{\pgfqpoint{0.645450in}{0.417642in}}{\pgfqpoint{3.372880in}{2.050688in}}%
\pgfusepath{clip}%
\pgfsetrectcap%
\pgfsetroundjoin%
\pgfsetlinewidth{0.803000pt}%
\definecolor{currentstroke}{rgb}{0.450000,0.450000,0.450000}%
\pgfsetstrokecolor{currentstroke}%
\pgfsetdash{}{0pt}%
\pgfpathmoveto{\pgfqpoint{0.790086in}{0.417642in}}%
\pgfpathlineto{\pgfqpoint{0.790086in}{2.468330in}}%
\pgfusepath{stroke}%
\end{pgfscope}%
\begin{pgfscope}%
\pgfsetbuttcap%
\pgfsetroundjoin%
\definecolor{currentfill}{rgb}{0.000000,0.000000,0.000000}%
\pgfsetfillcolor{currentfill}%
\pgfsetlinewidth{0.803000pt}%
\definecolor{currentstroke}{rgb}{0.000000,0.000000,0.000000}%
\pgfsetstrokecolor{currentstroke}%
\pgfsetdash{}{0pt}%
\pgfsys@defobject{currentmarker}{\pgfqpoint{0.000000in}{-0.048611in}}{\pgfqpoint{0.000000in}{0.000000in}}{%
\pgfpathmoveto{\pgfqpoint{0.000000in}{0.000000in}}%
\pgfpathlineto{\pgfqpoint{0.000000in}{-0.048611in}}%
\pgfusepath{stroke,fill}%
}%
\begin{pgfscope}%
\pgfsys@transformshift{0.790086in}{0.417642in}%
\pgfsys@useobject{currentmarker}{}%
\end{pgfscope}%
\end{pgfscope}%
\begin{pgfscope}%
\definecolor{textcolor}{rgb}{0.000000,0.000000,0.000000}%
\pgfsetstrokecolor{textcolor}%
\pgfsetfillcolor{textcolor}%
\pgftext[x=0.790086in,y=0.320420in,,top]{\color{textcolor}\rmfamily\fontsize{8.000000}{9.600000}\selectfont \(\displaystyle {10^{-2}}\)}%
\end{pgfscope}%
\begin{pgfscope}%
\pgfpathrectangle{\pgfqpoint{0.645450in}{0.417642in}}{\pgfqpoint{3.372880in}{2.050688in}}%
\pgfusepath{clip}%
\pgfsetrectcap%
\pgfsetroundjoin%
\pgfsetlinewidth{0.803000pt}%
\definecolor{currentstroke}{rgb}{0.450000,0.450000,0.450000}%
\pgfsetstrokecolor{currentstroke}%
\pgfsetdash{}{0pt}%
\pgfpathmoveto{\pgfqpoint{1.929485in}{0.417642in}}%
\pgfpathlineto{\pgfqpoint{1.929485in}{2.468330in}}%
\pgfusepath{stroke}%
\end{pgfscope}%
\begin{pgfscope}%
\pgfsetbuttcap%
\pgfsetroundjoin%
\definecolor{currentfill}{rgb}{0.000000,0.000000,0.000000}%
\pgfsetfillcolor{currentfill}%
\pgfsetlinewidth{0.803000pt}%
\definecolor{currentstroke}{rgb}{0.000000,0.000000,0.000000}%
\pgfsetstrokecolor{currentstroke}%
\pgfsetdash{}{0pt}%
\pgfsys@defobject{currentmarker}{\pgfqpoint{0.000000in}{-0.048611in}}{\pgfqpoint{0.000000in}{0.000000in}}{%
\pgfpathmoveto{\pgfqpoint{0.000000in}{0.000000in}}%
\pgfpathlineto{\pgfqpoint{0.000000in}{-0.048611in}}%
\pgfusepath{stroke,fill}%
}%
\begin{pgfscope}%
\pgfsys@transformshift{1.929485in}{0.417642in}%
\pgfsys@useobject{currentmarker}{}%
\end{pgfscope}%
\end{pgfscope}%
\begin{pgfscope}%
\definecolor{textcolor}{rgb}{0.000000,0.000000,0.000000}%
\pgfsetstrokecolor{textcolor}%
\pgfsetfillcolor{textcolor}%
\pgftext[x=1.929485in,y=0.320420in,,top]{\color{textcolor}\rmfamily\fontsize{8.000000}{9.600000}\selectfont \(\displaystyle {10^{-1}}\)}%
\end{pgfscope}%
\begin{pgfscope}%
\pgfpathrectangle{\pgfqpoint{0.645450in}{0.417642in}}{\pgfqpoint{3.372880in}{2.050688in}}%
\pgfusepath{clip}%
\pgfsetrectcap%
\pgfsetroundjoin%
\pgfsetlinewidth{0.803000pt}%
\definecolor{currentstroke}{rgb}{0.450000,0.450000,0.450000}%
\pgfsetstrokecolor{currentstroke}%
\pgfsetdash{}{0pt}%
\pgfpathmoveto{\pgfqpoint{3.068884in}{0.417642in}}%
\pgfpathlineto{\pgfqpoint{3.068884in}{2.468330in}}%
\pgfusepath{stroke}%
\end{pgfscope}%
\begin{pgfscope}%
\pgfsetbuttcap%
\pgfsetroundjoin%
\definecolor{currentfill}{rgb}{0.000000,0.000000,0.000000}%
\pgfsetfillcolor{currentfill}%
\pgfsetlinewidth{0.803000pt}%
\definecolor{currentstroke}{rgb}{0.000000,0.000000,0.000000}%
\pgfsetstrokecolor{currentstroke}%
\pgfsetdash{}{0pt}%
\pgfsys@defobject{currentmarker}{\pgfqpoint{0.000000in}{-0.048611in}}{\pgfqpoint{0.000000in}{0.000000in}}{%
\pgfpathmoveto{\pgfqpoint{0.000000in}{0.000000in}}%
\pgfpathlineto{\pgfqpoint{0.000000in}{-0.048611in}}%
\pgfusepath{stroke,fill}%
}%
\begin{pgfscope}%
\pgfsys@transformshift{3.068884in}{0.417642in}%
\pgfsys@useobject{currentmarker}{}%
\end{pgfscope}%
\end{pgfscope}%
\begin{pgfscope}%
\definecolor{textcolor}{rgb}{0.000000,0.000000,0.000000}%
\pgfsetstrokecolor{textcolor}%
\pgfsetfillcolor{textcolor}%
\pgftext[x=3.068884in,y=0.320420in,,top]{\color{textcolor}\rmfamily\fontsize{8.000000}{9.600000}\selectfont \(\displaystyle {10^{0}}\)}%
\end{pgfscope}%
\begin{pgfscope}%
\pgfpathrectangle{\pgfqpoint{0.645450in}{0.417642in}}{\pgfqpoint{3.372880in}{2.050688in}}%
\pgfusepath{clip}%
\pgfsetrectcap%
\pgfsetroundjoin%
\pgfsetlinewidth{0.803000pt}%
\definecolor{currentstroke}{rgb}{0.850000,0.850000,0.850000}%
\pgfsetstrokecolor{currentstroke}%
\pgfsetdash{}{0pt}%
\pgfpathmoveto{\pgfqpoint{0.679666in}{0.417642in}}%
\pgfpathlineto{\pgfqpoint{0.679666in}{2.468330in}}%
\pgfusepath{stroke}%
\end{pgfscope}%
\begin{pgfscope}%
\pgfsetbuttcap%
\pgfsetroundjoin%
\definecolor{currentfill}{rgb}{0.000000,0.000000,0.000000}%
\pgfsetfillcolor{currentfill}%
\pgfsetlinewidth{0.602250pt}%
\definecolor{currentstroke}{rgb}{0.000000,0.000000,0.000000}%
\pgfsetstrokecolor{currentstroke}%
\pgfsetdash{}{0pt}%
\pgfsys@defobject{currentmarker}{\pgfqpoint{0.000000in}{-0.027778in}}{\pgfqpoint{0.000000in}{0.000000in}}{%
\pgfpathmoveto{\pgfqpoint{0.000000in}{0.000000in}}%
\pgfpathlineto{\pgfqpoint{0.000000in}{-0.027778in}}%
\pgfusepath{stroke,fill}%
}%
\begin{pgfscope}%
\pgfsys@transformshift{0.679666in}{0.417642in}%
\pgfsys@useobject{currentmarker}{}%
\end{pgfscope}%
\end{pgfscope}%
\begin{pgfscope}%
\pgfpathrectangle{\pgfqpoint{0.645450in}{0.417642in}}{\pgfqpoint{3.372880in}{2.050688in}}%
\pgfusepath{clip}%
\pgfsetrectcap%
\pgfsetroundjoin%
\pgfsetlinewidth{0.803000pt}%
\definecolor{currentstroke}{rgb}{0.850000,0.850000,0.850000}%
\pgfsetstrokecolor{currentstroke}%
\pgfsetdash{}{0pt}%
\pgfpathmoveto{\pgfqpoint{0.737949in}{0.417642in}}%
\pgfpathlineto{\pgfqpoint{0.737949in}{2.468330in}}%
\pgfusepath{stroke}%
\end{pgfscope}%
\begin{pgfscope}%
\pgfsetbuttcap%
\pgfsetroundjoin%
\definecolor{currentfill}{rgb}{0.000000,0.000000,0.000000}%
\pgfsetfillcolor{currentfill}%
\pgfsetlinewidth{0.602250pt}%
\definecolor{currentstroke}{rgb}{0.000000,0.000000,0.000000}%
\pgfsetstrokecolor{currentstroke}%
\pgfsetdash{}{0pt}%
\pgfsys@defobject{currentmarker}{\pgfqpoint{0.000000in}{-0.027778in}}{\pgfqpoint{0.000000in}{0.000000in}}{%
\pgfpathmoveto{\pgfqpoint{0.000000in}{0.000000in}}%
\pgfpathlineto{\pgfqpoint{0.000000in}{-0.027778in}}%
\pgfusepath{stroke,fill}%
}%
\begin{pgfscope}%
\pgfsys@transformshift{0.737949in}{0.417642in}%
\pgfsys@useobject{currentmarker}{}%
\end{pgfscope}%
\end{pgfscope}%
\begin{pgfscope}%
\pgfpathrectangle{\pgfqpoint{0.645450in}{0.417642in}}{\pgfqpoint{3.372880in}{2.050688in}}%
\pgfusepath{clip}%
\pgfsetrectcap%
\pgfsetroundjoin%
\pgfsetlinewidth{0.803000pt}%
\definecolor{currentstroke}{rgb}{0.850000,0.850000,0.850000}%
\pgfsetstrokecolor{currentstroke}%
\pgfsetdash{}{0pt}%
\pgfpathmoveto{\pgfqpoint{1.133079in}{0.417642in}}%
\pgfpathlineto{\pgfqpoint{1.133079in}{2.468330in}}%
\pgfusepath{stroke}%
\end{pgfscope}%
\begin{pgfscope}%
\pgfsetbuttcap%
\pgfsetroundjoin%
\definecolor{currentfill}{rgb}{0.000000,0.000000,0.000000}%
\pgfsetfillcolor{currentfill}%
\pgfsetlinewidth{0.602250pt}%
\definecolor{currentstroke}{rgb}{0.000000,0.000000,0.000000}%
\pgfsetstrokecolor{currentstroke}%
\pgfsetdash{}{0pt}%
\pgfsys@defobject{currentmarker}{\pgfqpoint{0.000000in}{-0.027778in}}{\pgfqpoint{0.000000in}{0.000000in}}{%
\pgfpathmoveto{\pgfqpoint{0.000000in}{0.000000in}}%
\pgfpathlineto{\pgfqpoint{0.000000in}{-0.027778in}}%
\pgfusepath{stroke,fill}%
}%
\begin{pgfscope}%
\pgfsys@transformshift{1.133079in}{0.417642in}%
\pgfsys@useobject{currentmarker}{}%
\end{pgfscope}%
\end{pgfscope}%
\begin{pgfscope}%
\pgfpathrectangle{\pgfqpoint{0.645450in}{0.417642in}}{\pgfqpoint{3.372880in}{2.050688in}}%
\pgfusepath{clip}%
\pgfsetrectcap%
\pgfsetroundjoin%
\pgfsetlinewidth{0.803000pt}%
\definecolor{currentstroke}{rgb}{0.850000,0.850000,0.850000}%
\pgfsetstrokecolor{currentstroke}%
\pgfsetdash{}{0pt}%
\pgfpathmoveto{\pgfqpoint{1.333717in}{0.417642in}}%
\pgfpathlineto{\pgfqpoint{1.333717in}{2.468330in}}%
\pgfusepath{stroke}%
\end{pgfscope}%
\begin{pgfscope}%
\pgfsetbuttcap%
\pgfsetroundjoin%
\definecolor{currentfill}{rgb}{0.000000,0.000000,0.000000}%
\pgfsetfillcolor{currentfill}%
\pgfsetlinewidth{0.602250pt}%
\definecolor{currentstroke}{rgb}{0.000000,0.000000,0.000000}%
\pgfsetstrokecolor{currentstroke}%
\pgfsetdash{}{0pt}%
\pgfsys@defobject{currentmarker}{\pgfqpoint{0.000000in}{-0.027778in}}{\pgfqpoint{0.000000in}{0.000000in}}{%
\pgfpathmoveto{\pgfqpoint{0.000000in}{0.000000in}}%
\pgfpathlineto{\pgfqpoint{0.000000in}{-0.027778in}}%
\pgfusepath{stroke,fill}%
}%
\begin{pgfscope}%
\pgfsys@transformshift{1.333717in}{0.417642in}%
\pgfsys@useobject{currentmarker}{}%
\end{pgfscope}%
\end{pgfscope}%
\begin{pgfscope}%
\pgfpathrectangle{\pgfqpoint{0.645450in}{0.417642in}}{\pgfqpoint{3.372880in}{2.050688in}}%
\pgfusepath{clip}%
\pgfsetrectcap%
\pgfsetroundjoin%
\pgfsetlinewidth{0.803000pt}%
\definecolor{currentstroke}{rgb}{0.850000,0.850000,0.850000}%
\pgfsetstrokecolor{currentstroke}%
\pgfsetdash{}{0pt}%
\pgfpathmoveto{\pgfqpoint{1.476072in}{0.417642in}}%
\pgfpathlineto{\pgfqpoint{1.476072in}{2.468330in}}%
\pgfusepath{stroke}%
\end{pgfscope}%
\begin{pgfscope}%
\pgfsetbuttcap%
\pgfsetroundjoin%
\definecolor{currentfill}{rgb}{0.000000,0.000000,0.000000}%
\pgfsetfillcolor{currentfill}%
\pgfsetlinewidth{0.602250pt}%
\definecolor{currentstroke}{rgb}{0.000000,0.000000,0.000000}%
\pgfsetstrokecolor{currentstroke}%
\pgfsetdash{}{0pt}%
\pgfsys@defobject{currentmarker}{\pgfqpoint{0.000000in}{-0.027778in}}{\pgfqpoint{0.000000in}{0.000000in}}{%
\pgfpathmoveto{\pgfqpoint{0.000000in}{0.000000in}}%
\pgfpathlineto{\pgfqpoint{0.000000in}{-0.027778in}}%
\pgfusepath{stroke,fill}%
}%
\begin{pgfscope}%
\pgfsys@transformshift{1.476072in}{0.417642in}%
\pgfsys@useobject{currentmarker}{}%
\end{pgfscope}%
\end{pgfscope}%
\begin{pgfscope}%
\pgfpathrectangle{\pgfqpoint{0.645450in}{0.417642in}}{\pgfqpoint{3.372880in}{2.050688in}}%
\pgfusepath{clip}%
\pgfsetrectcap%
\pgfsetroundjoin%
\pgfsetlinewidth{0.803000pt}%
\definecolor{currentstroke}{rgb}{0.850000,0.850000,0.850000}%
\pgfsetstrokecolor{currentstroke}%
\pgfsetdash{}{0pt}%
\pgfpathmoveto{\pgfqpoint{1.586491in}{0.417642in}}%
\pgfpathlineto{\pgfqpoint{1.586491in}{2.468330in}}%
\pgfusepath{stroke}%
\end{pgfscope}%
\begin{pgfscope}%
\pgfsetbuttcap%
\pgfsetroundjoin%
\definecolor{currentfill}{rgb}{0.000000,0.000000,0.000000}%
\pgfsetfillcolor{currentfill}%
\pgfsetlinewidth{0.602250pt}%
\definecolor{currentstroke}{rgb}{0.000000,0.000000,0.000000}%
\pgfsetstrokecolor{currentstroke}%
\pgfsetdash{}{0pt}%
\pgfsys@defobject{currentmarker}{\pgfqpoint{0.000000in}{-0.027778in}}{\pgfqpoint{0.000000in}{0.000000in}}{%
\pgfpathmoveto{\pgfqpoint{0.000000in}{0.000000in}}%
\pgfpathlineto{\pgfqpoint{0.000000in}{-0.027778in}}%
\pgfusepath{stroke,fill}%
}%
\begin{pgfscope}%
\pgfsys@transformshift{1.586491in}{0.417642in}%
\pgfsys@useobject{currentmarker}{}%
\end{pgfscope}%
\end{pgfscope}%
\begin{pgfscope}%
\pgfpathrectangle{\pgfqpoint{0.645450in}{0.417642in}}{\pgfqpoint{3.372880in}{2.050688in}}%
\pgfusepath{clip}%
\pgfsetrectcap%
\pgfsetroundjoin%
\pgfsetlinewidth{0.803000pt}%
\definecolor{currentstroke}{rgb}{0.850000,0.850000,0.850000}%
\pgfsetstrokecolor{currentstroke}%
\pgfsetdash{}{0pt}%
\pgfpathmoveto{\pgfqpoint{1.676710in}{0.417642in}}%
\pgfpathlineto{\pgfqpoint{1.676710in}{2.468330in}}%
\pgfusepath{stroke}%
\end{pgfscope}%
\begin{pgfscope}%
\pgfsetbuttcap%
\pgfsetroundjoin%
\definecolor{currentfill}{rgb}{0.000000,0.000000,0.000000}%
\pgfsetfillcolor{currentfill}%
\pgfsetlinewidth{0.602250pt}%
\definecolor{currentstroke}{rgb}{0.000000,0.000000,0.000000}%
\pgfsetstrokecolor{currentstroke}%
\pgfsetdash{}{0pt}%
\pgfsys@defobject{currentmarker}{\pgfqpoint{0.000000in}{-0.027778in}}{\pgfqpoint{0.000000in}{0.000000in}}{%
\pgfpathmoveto{\pgfqpoint{0.000000in}{0.000000in}}%
\pgfpathlineto{\pgfqpoint{0.000000in}{-0.027778in}}%
\pgfusepath{stroke,fill}%
}%
\begin{pgfscope}%
\pgfsys@transformshift{1.676710in}{0.417642in}%
\pgfsys@useobject{currentmarker}{}%
\end{pgfscope}%
\end{pgfscope}%
\begin{pgfscope}%
\pgfpathrectangle{\pgfqpoint{0.645450in}{0.417642in}}{\pgfqpoint{3.372880in}{2.050688in}}%
\pgfusepath{clip}%
\pgfsetrectcap%
\pgfsetroundjoin%
\pgfsetlinewidth{0.803000pt}%
\definecolor{currentstroke}{rgb}{0.850000,0.850000,0.850000}%
\pgfsetstrokecolor{currentstroke}%
\pgfsetdash{}{0pt}%
\pgfpathmoveto{\pgfqpoint{1.752989in}{0.417642in}}%
\pgfpathlineto{\pgfqpoint{1.752989in}{2.468330in}}%
\pgfusepath{stroke}%
\end{pgfscope}%
\begin{pgfscope}%
\pgfsetbuttcap%
\pgfsetroundjoin%
\definecolor{currentfill}{rgb}{0.000000,0.000000,0.000000}%
\pgfsetfillcolor{currentfill}%
\pgfsetlinewidth{0.602250pt}%
\definecolor{currentstroke}{rgb}{0.000000,0.000000,0.000000}%
\pgfsetstrokecolor{currentstroke}%
\pgfsetdash{}{0pt}%
\pgfsys@defobject{currentmarker}{\pgfqpoint{0.000000in}{-0.027778in}}{\pgfqpoint{0.000000in}{0.000000in}}{%
\pgfpathmoveto{\pgfqpoint{0.000000in}{0.000000in}}%
\pgfpathlineto{\pgfqpoint{0.000000in}{-0.027778in}}%
\pgfusepath{stroke,fill}%
}%
\begin{pgfscope}%
\pgfsys@transformshift{1.752989in}{0.417642in}%
\pgfsys@useobject{currentmarker}{}%
\end{pgfscope}%
\end{pgfscope}%
\begin{pgfscope}%
\pgfpathrectangle{\pgfqpoint{0.645450in}{0.417642in}}{\pgfqpoint{3.372880in}{2.050688in}}%
\pgfusepath{clip}%
\pgfsetrectcap%
\pgfsetroundjoin%
\pgfsetlinewidth{0.803000pt}%
\definecolor{currentstroke}{rgb}{0.850000,0.850000,0.850000}%
\pgfsetstrokecolor{currentstroke}%
\pgfsetdash{}{0pt}%
\pgfpathmoveto{\pgfqpoint{1.819065in}{0.417642in}}%
\pgfpathlineto{\pgfqpoint{1.819065in}{2.468330in}}%
\pgfusepath{stroke}%
\end{pgfscope}%
\begin{pgfscope}%
\pgfsetbuttcap%
\pgfsetroundjoin%
\definecolor{currentfill}{rgb}{0.000000,0.000000,0.000000}%
\pgfsetfillcolor{currentfill}%
\pgfsetlinewidth{0.602250pt}%
\definecolor{currentstroke}{rgb}{0.000000,0.000000,0.000000}%
\pgfsetstrokecolor{currentstroke}%
\pgfsetdash{}{0pt}%
\pgfsys@defobject{currentmarker}{\pgfqpoint{0.000000in}{-0.027778in}}{\pgfqpoint{0.000000in}{0.000000in}}{%
\pgfpathmoveto{\pgfqpoint{0.000000in}{0.000000in}}%
\pgfpathlineto{\pgfqpoint{0.000000in}{-0.027778in}}%
\pgfusepath{stroke,fill}%
}%
\begin{pgfscope}%
\pgfsys@transformshift{1.819065in}{0.417642in}%
\pgfsys@useobject{currentmarker}{}%
\end{pgfscope}%
\end{pgfscope}%
\begin{pgfscope}%
\pgfpathrectangle{\pgfqpoint{0.645450in}{0.417642in}}{\pgfqpoint{3.372880in}{2.050688in}}%
\pgfusepath{clip}%
\pgfsetrectcap%
\pgfsetroundjoin%
\pgfsetlinewidth{0.803000pt}%
\definecolor{currentstroke}{rgb}{0.850000,0.850000,0.850000}%
\pgfsetstrokecolor{currentstroke}%
\pgfsetdash{}{0pt}%
\pgfpathmoveto{\pgfqpoint{1.877348in}{0.417642in}}%
\pgfpathlineto{\pgfqpoint{1.877348in}{2.468330in}}%
\pgfusepath{stroke}%
\end{pgfscope}%
\begin{pgfscope}%
\pgfsetbuttcap%
\pgfsetroundjoin%
\definecolor{currentfill}{rgb}{0.000000,0.000000,0.000000}%
\pgfsetfillcolor{currentfill}%
\pgfsetlinewidth{0.602250pt}%
\definecolor{currentstroke}{rgb}{0.000000,0.000000,0.000000}%
\pgfsetstrokecolor{currentstroke}%
\pgfsetdash{}{0pt}%
\pgfsys@defobject{currentmarker}{\pgfqpoint{0.000000in}{-0.027778in}}{\pgfqpoint{0.000000in}{0.000000in}}{%
\pgfpathmoveto{\pgfqpoint{0.000000in}{0.000000in}}%
\pgfpathlineto{\pgfqpoint{0.000000in}{-0.027778in}}%
\pgfusepath{stroke,fill}%
}%
\begin{pgfscope}%
\pgfsys@transformshift{1.877348in}{0.417642in}%
\pgfsys@useobject{currentmarker}{}%
\end{pgfscope}%
\end{pgfscope}%
\begin{pgfscope}%
\pgfpathrectangle{\pgfqpoint{0.645450in}{0.417642in}}{\pgfqpoint{3.372880in}{2.050688in}}%
\pgfusepath{clip}%
\pgfsetrectcap%
\pgfsetroundjoin%
\pgfsetlinewidth{0.803000pt}%
\definecolor{currentstroke}{rgb}{0.850000,0.850000,0.850000}%
\pgfsetstrokecolor{currentstroke}%
\pgfsetdash{}{0pt}%
\pgfpathmoveto{\pgfqpoint{2.272478in}{0.417642in}}%
\pgfpathlineto{\pgfqpoint{2.272478in}{2.468330in}}%
\pgfusepath{stroke}%
\end{pgfscope}%
\begin{pgfscope}%
\pgfsetbuttcap%
\pgfsetroundjoin%
\definecolor{currentfill}{rgb}{0.000000,0.000000,0.000000}%
\pgfsetfillcolor{currentfill}%
\pgfsetlinewidth{0.602250pt}%
\definecolor{currentstroke}{rgb}{0.000000,0.000000,0.000000}%
\pgfsetstrokecolor{currentstroke}%
\pgfsetdash{}{0pt}%
\pgfsys@defobject{currentmarker}{\pgfqpoint{0.000000in}{-0.027778in}}{\pgfqpoint{0.000000in}{0.000000in}}{%
\pgfpathmoveto{\pgfqpoint{0.000000in}{0.000000in}}%
\pgfpathlineto{\pgfqpoint{0.000000in}{-0.027778in}}%
\pgfusepath{stroke,fill}%
}%
\begin{pgfscope}%
\pgfsys@transformshift{2.272478in}{0.417642in}%
\pgfsys@useobject{currentmarker}{}%
\end{pgfscope}%
\end{pgfscope}%
\begin{pgfscope}%
\pgfpathrectangle{\pgfqpoint{0.645450in}{0.417642in}}{\pgfqpoint{3.372880in}{2.050688in}}%
\pgfusepath{clip}%
\pgfsetrectcap%
\pgfsetroundjoin%
\pgfsetlinewidth{0.803000pt}%
\definecolor{currentstroke}{rgb}{0.850000,0.850000,0.850000}%
\pgfsetstrokecolor{currentstroke}%
\pgfsetdash{}{0pt}%
\pgfpathmoveto{\pgfqpoint{2.473116in}{0.417642in}}%
\pgfpathlineto{\pgfqpoint{2.473116in}{2.468330in}}%
\pgfusepath{stroke}%
\end{pgfscope}%
\begin{pgfscope}%
\pgfsetbuttcap%
\pgfsetroundjoin%
\definecolor{currentfill}{rgb}{0.000000,0.000000,0.000000}%
\pgfsetfillcolor{currentfill}%
\pgfsetlinewidth{0.602250pt}%
\definecolor{currentstroke}{rgb}{0.000000,0.000000,0.000000}%
\pgfsetstrokecolor{currentstroke}%
\pgfsetdash{}{0pt}%
\pgfsys@defobject{currentmarker}{\pgfqpoint{0.000000in}{-0.027778in}}{\pgfqpoint{0.000000in}{0.000000in}}{%
\pgfpathmoveto{\pgfqpoint{0.000000in}{0.000000in}}%
\pgfpathlineto{\pgfqpoint{0.000000in}{-0.027778in}}%
\pgfusepath{stroke,fill}%
}%
\begin{pgfscope}%
\pgfsys@transformshift{2.473116in}{0.417642in}%
\pgfsys@useobject{currentmarker}{}%
\end{pgfscope}%
\end{pgfscope}%
\begin{pgfscope}%
\pgfpathrectangle{\pgfqpoint{0.645450in}{0.417642in}}{\pgfqpoint{3.372880in}{2.050688in}}%
\pgfusepath{clip}%
\pgfsetrectcap%
\pgfsetroundjoin%
\pgfsetlinewidth{0.803000pt}%
\definecolor{currentstroke}{rgb}{0.850000,0.850000,0.850000}%
\pgfsetstrokecolor{currentstroke}%
\pgfsetdash{}{0pt}%
\pgfpathmoveto{\pgfqpoint{2.615471in}{0.417642in}}%
\pgfpathlineto{\pgfqpoint{2.615471in}{2.468330in}}%
\pgfusepath{stroke}%
\end{pgfscope}%
\begin{pgfscope}%
\pgfsetbuttcap%
\pgfsetroundjoin%
\definecolor{currentfill}{rgb}{0.000000,0.000000,0.000000}%
\pgfsetfillcolor{currentfill}%
\pgfsetlinewidth{0.602250pt}%
\definecolor{currentstroke}{rgb}{0.000000,0.000000,0.000000}%
\pgfsetstrokecolor{currentstroke}%
\pgfsetdash{}{0pt}%
\pgfsys@defobject{currentmarker}{\pgfqpoint{0.000000in}{-0.027778in}}{\pgfqpoint{0.000000in}{0.000000in}}{%
\pgfpathmoveto{\pgfqpoint{0.000000in}{0.000000in}}%
\pgfpathlineto{\pgfqpoint{0.000000in}{-0.027778in}}%
\pgfusepath{stroke,fill}%
}%
\begin{pgfscope}%
\pgfsys@transformshift{2.615471in}{0.417642in}%
\pgfsys@useobject{currentmarker}{}%
\end{pgfscope}%
\end{pgfscope}%
\begin{pgfscope}%
\pgfpathrectangle{\pgfqpoint{0.645450in}{0.417642in}}{\pgfqpoint{3.372880in}{2.050688in}}%
\pgfusepath{clip}%
\pgfsetrectcap%
\pgfsetroundjoin%
\pgfsetlinewidth{0.803000pt}%
\definecolor{currentstroke}{rgb}{0.850000,0.850000,0.850000}%
\pgfsetstrokecolor{currentstroke}%
\pgfsetdash{}{0pt}%
\pgfpathmoveto{\pgfqpoint{2.725890in}{0.417642in}}%
\pgfpathlineto{\pgfqpoint{2.725890in}{2.468330in}}%
\pgfusepath{stroke}%
\end{pgfscope}%
\begin{pgfscope}%
\pgfsetbuttcap%
\pgfsetroundjoin%
\definecolor{currentfill}{rgb}{0.000000,0.000000,0.000000}%
\pgfsetfillcolor{currentfill}%
\pgfsetlinewidth{0.602250pt}%
\definecolor{currentstroke}{rgb}{0.000000,0.000000,0.000000}%
\pgfsetstrokecolor{currentstroke}%
\pgfsetdash{}{0pt}%
\pgfsys@defobject{currentmarker}{\pgfqpoint{0.000000in}{-0.027778in}}{\pgfqpoint{0.000000in}{0.000000in}}{%
\pgfpathmoveto{\pgfqpoint{0.000000in}{0.000000in}}%
\pgfpathlineto{\pgfqpoint{0.000000in}{-0.027778in}}%
\pgfusepath{stroke,fill}%
}%
\begin{pgfscope}%
\pgfsys@transformshift{2.725890in}{0.417642in}%
\pgfsys@useobject{currentmarker}{}%
\end{pgfscope}%
\end{pgfscope}%
\begin{pgfscope}%
\pgfpathrectangle{\pgfqpoint{0.645450in}{0.417642in}}{\pgfqpoint{3.372880in}{2.050688in}}%
\pgfusepath{clip}%
\pgfsetrectcap%
\pgfsetroundjoin%
\pgfsetlinewidth{0.803000pt}%
\definecolor{currentstroke}{rgb}{0.850000,0.850000,0.850000}%
\pgfsetstrokecolor{currentstroke}%
\pgfsetdash{}{0pt}%
\pgfpathmoveto{\pgfqpoint{2.816109in}{0.417642in}}%
\pgfpathlineto{\pgfqpoint{2.816109in}{2.468330in}}%
\pgfusepath{stroke}%
\end{pgfscope}%
\begin{pgfscope}%
\pgfsetbuttcap%
\pgfsetroundjoin%
\definecolor{currentfill}{rgb}{0.000000,0.000000,0.000000}%
\pgfsetfillcolor{currentfill}%
\pgfsetlinewidth{0.602250pt}%
\definecolor{currentstroke}{rgb}{0.000000,0.000000,0.000000}%
\pgfsetstrokecolor{currentstroke}%
\pgfsetdash{}{0pt}%
\pgfsys@defobject{currentmarker}{\pgfqpoint{0.000000in}{-0.027778in}}{\pgfqpoint{0.000000in}{0.000000in}}{%
\pgfpathmoveto{\pgfqpoint{0.000000in}{0.000000in}}%
\pgfpathlineto{\pgfqpoint{0.000000in}{-0.027778in}}%
\pgfusepath{stroke,fill}%
}%
\begin{pgfscope}%
\pgfsys@transformshift{2.816109in}{0.417642in}%
\pgfsys@useobject{currentmarker}{}%
\end{pgfscope}%
\end{pgfscope}%
\begin{pgfscope}%
\pgfpathrectangle{\pgfqpoint{0.645450in}{0.417642in}}{\pgfqpoint{3.372880in}{2.050688in}}%
\pgfusepath{clip}%
\pgfsetrectcap%
\pgfsetroundjoin%
\pgfsetlinewidth{0.803000pt}%
\definecolor{currentstroke}{rgb}{0.850000,0.850000,0.850000}%
\pgfsetstrokecolor{currentstroke}%
\pgfsetdash{}{0pt}%
\pgfpathmoveto{\pgfqpoint{2.892388in}{0.417642in}}%
\pgfpathlineto{\pgfqpoint{2.892388in}{2.468330in}}%
\pgfusepath{stroke}%
\end{pgfscope}%
\begin{pgfscope}%
\pgfsetbuttcap%
\pgfsetroundjoin%
\definecolor{currentfill}{rgb}{0.000000,0.000000,0.000000}%
\pgfsetfillcolor{currentfill}%
\pgfsetlinewidth{0.602250pt}%
\definecolor{currentstroke}{rgb}{0.000000,0.000000,0.000000}%
\pgfsetstrokecolor{currentstroke}%
\pgfsetdash{}{0pt}%
\pgfsys@defobject{currentmarker}{\pgfqpoint{0.000000in}{-0.027778in}}{\pgfqpoint{0.000000in}{0.000000in}}{%
\pgfpathmoveto{\pgfqpoint{0.000000in}{0.000000in}}%
\pgfpathlineto{\pgfqpoint{0.000000in}{-0.027778in}}%
\pgfusepath{stroke,fill}%
}%
\begin{pgfscope}%
\pgfsys@transformshift{2.892388in}{0.417642in}%
\pgfsys@useobject{currentmarker}{}%
\end{pgfscope}%
\end{pgfscope}%
\begin{pgfscope}%
\pgfpathrectangle{\pgfqpoint{0.645450in}{0.417642in}}{\pgfqpoint{3.372880in}{2.050688in}}%
\pgfusepath{clip}%
\pgfsetrectcap%
\pgfsetroundjoin%
\pgfsetlinewidth{0.803000pt}%
\definecolor{currentstroke}{rgb}{0.850000,0.850000,0.850000}%
\pgfsetstrokecolor{currentstroke}%
\pgfsetdash{}{0pt}%
\pgfpathmoveto{\pgfqpoint{2.958464in}{0.417642in}}%
\pgfpathlineto{\pgfqpoint{2.958464in}{2.468330in}}%
\pgfusepath{stroke}%
\end{pgfscope}%
\begin{pgfscope}%
\pgfsetbuttcap%
\pgfsetroundjoin%
\definecolor{currentfill}{rgb}{0.000000,0.000000,0.000000}%
\pgfsetfillcolor{currentfill}%
\pgfsetlinewidth{0.602250pt}%
\definecolor{currentstroke}{rgb}{0.000000,0.000000,0.000000}%
\pgfsetstrokecolor{currentstroke}%
\pgfsetdash{}{0pt}%
\pgfsys@defobject{currentmarker}{\pgfqpoint{0.000000in}{-0.027778in}}{\pgfqpoint{0.000000in}{0.000000in}}{%
\pgfpathmoveto{\pgfqpoint{0.000000in}{0.000000in}}%
\pgfpathlineto{\pgfqpoint{0.000000in}{-0.027778in}}%
\pgfusepath{stroke,fill}%
}%
\begin{pgfscope}%
\pgfsys@transformshift{2.958464in}{0.417642in}%
\pgfsys@useobject{currentmarker}{}%
\end{pgfscope}%
\end{pgfscope}%
\begin{pgfscope}%
\pgfpathrectangle{\pgfqpoint{0.645450in}{0.417642in}}{\pgfqpoint{3.372880in}{2.050688in}}%
\pgfusepath{clip}%
\pgfsetrectcap%
\pgfsetroundjoin%
\pgfsetlinewidth{0.803000pt}%
\definecolor{currentstroke}{rgb}{0.850000,0.850000,0.850000}%
\pgfsetstrokecolor{currentstroke}%
\pgfsetdash{}{0pt}%
\pgfpathmoveto{\pgfqpoint{3.016748in}{0.417642in}}%
\pgfpathlineto{\pgfqpoint{3.016748in}{2.468330in}}%
\pgfusepath{stroke}%
\end{pgfscope}%
\begin{pgfscope}%
\pgfsetbuttcap%
\pgfsetroundjoin%
\definecolor{currentfill}{rgb}{0.000000,0.000000,0.000000}%
\pgfsetfillcolor{currentfill}%
\pgfsetlinewidth{0.602250pt}%
\definecolor{currentstroke}{rgb}{0.000000,0.000000,0.000000}%
\pgfsetstrokecolor{currentstroke}%
\pgfsetdash{}{0pt}%
\pgfsys@defobject{currentmarker}{\pgfqpoint{0.000000in}{-0.027778in}}{\pgfqpoint{0.000000in}{0.000000in}}{%
\pgfpathmoveto{\pgfqpoint{0.000000in}{0.000000in}}%
\pgfpathlineto{\pgfqpoint{0.000000in}{-0.027778in}}%
\pgfusepath{stroke,fill}%
}%
\begin{pgfscope}%
\pgfsys@transformshift{3.016748in}{0.417642in}%
\pgfsys@useobject{currentmarker}{}%
\end{pgfscope}%
\end{pgfscope}%
\begin{pgfscope}%
\pgfpathrectangle{\pgfqpoint{0.645450in}{0.417642in}}{\pgfqpoint{3.372880in}{2.050688in}}%
\pgfusepath{clip}%
\pgfsetrectcap%
\pgfsetroundjoin%
\pgfsetlinewidth{0.803000pt}%
\definecolor{currentstroke}{rgb}{0.850000,0.850000,0.850000}%
\pgfsetstrokecolor{currentstroke}%
\pgfsetdash{}{0pt}%
\pgfpathmoveto{\pgfqpoint{3.411877in}{0.417642in}}%
\pgfpathlineto{\pgfqpoint{3.411877in}{2.468330in}}%
\pgfusepath{stroke}%
\end{pgfscope}%
\begin{pgfscope}%
\pgfsetbuttcap%
\pgfsetroundjoin%
\definecolor{currentfill}{rgb}{0.000000,0.000000,0.000000}%
\pgfsetfillcolor{currentfill}%
\pgfsetlinewidth{0.602250pt}%
\definecolor{currentstroke}{rgb}{0.000000,0.000000,0.000000}%
\pgfsetstrokecolor{currentstroke}%
\pgfsetdash{}{0pt}%
\pgfsys@defobject{currentmarker}{\pgfqpoint{0.000000in}{-0.027778in}}{\pgfqpoint{0.000000in}{0.000000in}}{%
\pgfpathmoveto{\pgfqpoint{0.000000in}{0.000000in}}%
\pgfpathlineto{\pgfqpoint{0.000000in}{-0.027778in}}%
\pgfusepath{stroke,fill}%
}%
\begin{pgfscope}%
\pgfsys@transformshift{3.411877in}{0.417642in}%
\pgfsys@useobject{currentmarker}{}%
\end{pgfscope}%
\end{pgfscope}%
\begin{pgfscope}%
\pgfpathrectangle{\pgfqpoint{0.645450in}{0.417642in}}{\pgfqpoint{3.372880in}{2.050688in}}%
\pgfusepath{clip}%
\pgfsetrectcap%
\pgfsetroundjoin%
\pgfsetlinewidth{0.803000pt}%
\definecolor{currentstroke}{rgb}{0.850000,0.850000,0.850000}%
\pgfsetstrokecolor{currentstroke}%
\pgfsetdash{}{0pt}%
\pgfpathmoveto{\pgfqpoint{3.612515in}{0.417642in}}%
\pgfpathlineto{\pgfqpoint{3.612515in}{2.468330in}}%
\pgfusepath{stroke}%
\end{pgfscope}%
\begin{pgfscope}%
\pgfsetbuttcap%
\pgfsetroundjoin%
\definecolor{currentfill}{rgb}{0.000000,0.000000,0.000000}%
\pgfsetfillcolor{currentfill}%
\pgfsetlinewidth{0.602250pt}%
\definecolor{currentstroke}{rgb}{0.000000,0.000000,0.000000}%
\pgfsetstrokecolor{currentstroke}%
\pgfsetdash{}{0pt}%
\pgfsys@defobject{currentmarker}{\pgfqpoint{0.000000in}{-0.027778in}}{\pgfqpoint{0.000000in}{0.000000in}}{%
\pgfpathmoveto{\pgfqpoint{0.000000in}{0.000000in}}%
\pgfpathlineto{\pgfqpoint{0.000000in}{-0.027778in}}%
\pgfusepath{stroke,fill}%
}%
\begin{pgfscope}%
\pgfsys@transformshift{3.612515in}{0.417642in}%
\pgfsys@useobject{currentmarker}{}%
\end{pgfscope}%
\end{pgfscope}%
\begin{pgfscope}%
\pgfpathrectangle{\pgfqpoint{0.645450in}{0.417642in}}{\pgfqpoint{3.372880in}{2.050688in}}%
\pgfusepath{clip}%
\pgfsetrectcap%
\pgfsetroundjoin%
\pgfsetlinewidth{0.803000pt}%
\definecolor{currentstroke}{rgb}{0.850000,0.850000,0.850000}%
\pgfsetstrokecolor{currentstroke}%
\pgfsetdash{}{0pt}%
\pgfpathmoveto{\pgfqpoint{3.754870in}{0.417642in}}%
\pgfpathlineto{\pgfqpoint{3.754870in}{2.468330in}}%
\pgfusepath{stroke}%
\end{pgfscope}%
\begin{pgfscope}%
\pgfsetbuttcap%
\pgfsetroundjoin%
\definecolor{currentfill}{rgb}{0.000000,0.000000,0.000000}%
\pgfsetfillcolor{currentfill}%
\pgfsetlinewidth{0.602250pt}%
\definecolor{currentstroke}{rgb}{0.000000,0.000000,0.000000}%
\pgfsetstrokecolor{currentstroke}%
\pgfsetdash{}{0pt}%
\pgfsys@defobject{currentmarker}{\pgfqpoint{0.000000in}{-0.027778in}}{\pgfqpoint{0.000000in}{0.000000in}}{%
\pgfpathmoveto{\pgfqpoint{0.000000in}{0.000000in}}%
\pgfpathlineto{\pgfqpoint{0.000000in}{-0.027778in}}%
\pgfusepath{stroke,fill}%
}%
\begin{pgfscope}%
\pgfsys@transformshift{3.754870in}{0.417642in}%
\pgfsys@useobject{currentmarker}{}%
\end{pgfscope}%
\end{pgfscope}%
\begin{pgfscope}%
\pgfpathrectangle{\pgfqpoint{0.645450in}{0.417642in}}{\pgfqpoint{3.372880in}{2.050688in}}%
\pgfusepath{clip}%
\pgfsetrectcap%
\pgfsetroundjoin%
\pgfsetlinewidth{0.803000pt}%
\definecolor{currentstroke}{rgb}{0.850000,0.850000,0.850000}%
\pgfsetstrokecolor{currentstroke}%
\pgfsetdash{}{0pt}%
\pgfpathmoveto{\pgfqpoint{3.865289in}{0.417642in}}%
\pgfpathlineto{\pgfqpoint{3.865289in}{2.468330in}}%
\pgfusepath{stroke}%
\end{pgfscope}%
\begin{pgfscope}%
\pgfsetbuttcap%
\pgfsetroundjoin%
\definecolor{currentfill}{rgb}{0.000000,0.000000,0.000000}%
\pgfsetfillcolor{currentfill}%
\pgfsetlinewidth{0.602250pt}%
\definecolor{currentstroke}{rgb}{0.000000,0.000000,0.000000}%
\pgfsetstrokecolor{currentstroke}%
\pgfsetdash{}{0pt}%
\pgfsys@defobject{currentmarker}{\pgfqpoint{0.000000in}{-0.027778in}}{\pgfqpoint{0.000000in}{0.000000in}}{%
\pgfpathmoveto{\pgfqpoint{0.000000in}{0.000000in}}%
\pgfpathlineto{\pgfqpoint{0.000000in}{-0.027778in}}%
\pgfusepath{stroke,fill}%
}%
\begin{pgfscope}%
\pgfsys@transformshift{3.865289in}{0.417642in}%
\pgfsys@useobject{currentmarker}{}%
\end{pgfscope}%
\end{pgfscope}%
\begin{pgfscope}%
\pgfpathrectangle{\pgfqpoint{0.645450in}{0.417642in}}{\pgfqpoint{3.372880in}{2.050688in}}%
\pgfusepath{clip}%
\pgfsetrectcap%
\pgfsetroundjoin%
\pgfsetlinewidth{0.803000pt}%
\definecolor{currentstroke}{rgb}{0.850000,0.850000,0.850000}%
\pgfsetstrokecolor{currentstroke}%
\pgfsetdash{}{0pt}%
\pgfpathmoveto{\pgfqpoint{3.955508in}{0.417642in}}%
\pgfpathlineto{\pgfqpoint{3.955508in}{2.468330in}}%
\pgfusepath{stroke}%
\end{pgfscope}%
\begin{pgfscope}%
\pgfsetbuttcap%
\pgfsetroundjoin%
\definecolor{currentfill}{rgb}{0.000000,0.000000,0.000000}%
\pgfsetfillcolor{currentfill}%
\pgfsetlinewidth{0.602250pt}%
\definecolor{currentstroke}{rgb}{0.000000,0.000000,0.000000}%
\pgfsetstrokecolor{currentstroke}%
\pgfsetdash{}{0pt}%
\pgfsys@defobject{currentmarker}{\pgfqpoint{0.000000in}{-0.027778in}}{\pgfqpoint{0.000000in}{0.000000in}}{%
\pgfpathmoveto{\pgfqpoint{0.000000in}{0.000000in}}%
\pgfpathlineto{\pgfqpoint{0.000000in}{-0.027778in}}%
\pgfusepath{stroke,fill}%
}%
\begin{pgfscope}%
\pgfsys@transformshift{3.955508in}{0.417642in}%
\pgfsys@useobject{currentmarker}{}%
\end{pgfscope}%
\end{pgfscope}%
\begin{pgfscope}%
\definecolor{textcolor}{rgb}{0.000000,0.000000,0.000000}%
\pgfsetstrokecolor{textcolor}%
\pgfsetfillcolor{textcolor}%
\pgftext[x=2.331890in,y=0.165003in,,top]{\color{textcolor}\rmfamily\fontsize{10.000000}{12.000000}\selectfont Frequency in \(\displaystyle \unit{\Hz}\)}%
\end{pgfscope}%
\begin{pgfscope}%
\pgfpathrectangle{\pgfqpoint{0.645450in}{0.417642in}}{\pgfqpoint{3.372880in}{2.050688in}}%
\pgfusepath{clip}%
\pgfsetrectcap%
\pgfsetroundjoin%
\pgfsetlinewidth{0.803000pt}%
\definecolor{currentstroke}{rgb}{0.450000,0.450000,0.450000}%
\pgfsetstrokecolor{currentstroke}%
\pgfsetdash{}{0pt}%
\pgfpathmoveto{\pgfqpoint{0.645450in}{1.340600in}}%
\pgfpathlineto{\pgfqpoint{4.018330in}{1.340600in}}%
\pgfusepath{stroke}%
\end{pgfscope}%
\begin{pgfscope}%
\pgfsetbuttcap%
\pgfsetroundjoin%
\definecolor{currentfill}{rgb}{0.000000,0.000000,0.000000}%
\pgfsetfillcolor{currentfill}%
\pgfsetlinewidth{0.803000pt}%
\definecolor{currentstroke}{rgb}{0.000000,0.000000,0.000000}%
\pgfsetstrokecolor{currentstroke}%
\pgfsetdash{}{0pt}%
\pgfsys@defobject{currentmarker}{\pgfqpoint{-0.048611in}{0.000000in}}{\pgfqpoint{-0.000000in}{0.000000in}}{%
\pgfpathmoveto{\pgfqpoint{-0.000000in}{0.000000in}}%
\pgfpathlineto{\pgfqpoint{-0.048611in}{0.000000in}}%
\pgfusepath{stroke,fill}%
}%
\begin{pgfscope}%
\pgfsys@transformshift{0.645450in}{1.340600in}%
\pgfsys@useobject{currentmarker}{}%
\end{pgfscope}%
\end{pgfscope}%
\begin{pgfscope}%
\definecolor{textcolor}{rgb}{0.000000,0.000000,0.000000}%
\pgfsetstrokecolor{textcolor}%
\pgfsetfillcolor{textcolor}%
\pgftext[x=0.241129in, y=1.301448in, left, base]{\color{textcolor}\rmfamily\fontsize{8.000000}{9.600000}\selectfont \(\displaystyle {10^{-13}}\)}%
\end{pgfscope}%
\begin{pgfscope}%
\pgfpathrectangle{\pgfqpoint{0.645450in}{0.417642in}}{\pgfqpoint{3.372880in}{2.050688in}}%
\pgfusepath{clip}%
\pgfsetrectcap%
\pgfsetroundjoin%
\pgfsetlinewidth{0.803000pt}%
\definecolor{currentstroke}{rgb}{0.450000,0.450000,0.450000}%
\pgfsetstrokecolor{currentstroke}%
\pgfsetdash{}{0pt}%
\pgfpathmoveto{\pgfqpoint{0.645450in}{2.277445in}}%
\pgfpathlineto{\pgfqpoint{4.018330in}{2.277445in}}%
\pgfusepath{stroke}%
\end{pgfscope}%
\begin{pgfscope}%
\pgfsetbuttcap%
\pgfsetroundjoin%
\definecolor{currentfill}{rgb}{0.000000,0.000000,0.000000}%
\pgfsetfillcolor{currentfill}%
\pgfsetlinewidth{0.803000pt}%
\definecolor{currentstroke}{rgb}{0.000000,0.000000,0.000000}%
\pgfsetstrokecolor{currentstroke}%
\pgfsetdash{}{0pt}%
\pgfsys@defobject{currentmarker}{\pgfqpoint{-0.048611in}{0.000000in}}{\pgfqpoint{-0.000000in}{0.000000in}}{%
\pgfpathmoveto{\pgfqpoint{-0.000000in}{0.000000in}}%
\pgfpathlineto{\pgfqpoint{-0.048611in}{0.000000in}}%
\pgfusepath{stroke,fill}%
}%
\begin{pgfscope}%
\pgfsys@transformshift{0.645450in}{2.277445in}%
\pgfsys@useobject{currentmarker}{}%
\end{pgfscope}%
\end{pgfscope}%
\begin{pgfscope}%
\definecolor{textcolor}{rgb}{0.000000,0.000000,0.000000}%
\pgfsetstrokecolor{textcolor}%
\pgfsetfillcolor{textcolor}%
\pgftext[x=0.241129in, y=2.238292in, left, base]{\color{textcolor}\rmfamily\fontsize{8.000000}{9.600000}\selectfont \(\displaystyle {10^{-12}}\)}%
\end{pgfscope}%
\begin{pgfscope}%
\pgfpathrectangle{\pgfqpoint{0.645450in}{0.417642in}}{\pgfqpoint{3.372880in}{2.050688in}}%
\pgfusepath{clip}%
\pgfsetrectcap%
\pgfsetroundjoin%
\pgfsetlinewidth{0.803000pt}%
\definecolor{currentstroke}{rgb}{0.850000,0.850000,0.850000}%
\pgfsetstrokecolor{currentstroke}%
\pgfsetdash{}{0pt}%
\pgfpathmoveto{\pgfqpoint{0.645450in}{0.685774in}}%
\pgfpathlineto{\pgfqpoint{4.018330in}{0.685774in}}%
\pgfusepath{stroke}%
\end{pgfscope}%
\begin{pgfscope}%
\pgfsetbuttcap%
\pgfsetroundjoin%
\definecolor{currentfill}{rgb}{0.000000,0.000000,0.000000}%
\pgfsetfillcolor{currentfill}%
\pgfsetlinewidth{0.602250pt}%
\definecolor{currentstroke}{rgb}{0.000000,0.000000,0.000000}%
\pgfsetstrokecolor{currentstroke}%
\pgfsetdash{}{0pt}%
\pgfsys@defobject{currentmarker}{\pgfqpoint{-0.027778in}{0.000000in}}{\pgfqpoint{-0.000000in}{0.000000in}}{%
\pgfpathmoveto{\pgfqpoint{-0.000000in}{0.000000in}}%
\pgfpathlineto{\pgfqpoint{-0.027778in}{0.000000in}}%
\pgfusepath{stroke,fill}%
}%
\begin{pgfscope}%
\pgfsys@transformshift{0.645450in}{0.685774in}%
\pgfsys@useobject{currentmarker}{}%
\end{pgfscope}%
\end{pgfscope}%
\begin{pgfscope}%
\pgfpathrectangle{\pgfqpoint{0.645450in}{0.417642in}}{\pgfqpoint{3.372880in}{2.050688in}}%
\pgfusepath{clip}%
\pgfsetrectcap%
\pgfsetroundjoin%
\pgfsetlinewidth{0.803000pt}%
\definecolor{currentstroke}{rgb}{0.850000,0.850000,0.850000}%
\pgfsetstrokecolor{currentstroke}%
\pgfsetdash{}{0pt}%
\pgfpathmoveto{\pgfqpoint{0.645450in}{0.850744in}}%
\pgfpathlineto{\pgfqpoint{4.018330in}{0.850744in}}%
\pgfusepath{stroke}%
\end{pgfscope}%
\begin{pgfscope}%
\pgfsetbuttcap%
\pgfsetroundjoin%
\definecolor{currentfill}{rgb}{0.000000,0.000000,0.000000}%
\pgfsetfillcolor{currentfill}%
\pgfsetlinewidth{0.602250pt}%
\definecolor{currentstroke}{rgb}{0.000000,0.000000,0.000000}%
\pgfsetstrokecolor{currentstroke}%
\pgfsetdash{}{0pt}%
\pgfsys@defobject{currentmarker}{\pgfqpoint{-0.027778in}{0.000000in}}{\pgfqpoint{-0.000000in}{0.000000in}}{%
\pgfpathmoveto{\pgfqpoint{-0.000000in}{0.000000in}}%
\pgfpathlineto{\pgfqpoint{-0.027778in}{0.000000in}}%
\pgfusepath{stroke,fill}%
}%
\begin{pgfscope}%
\pgfsys@transformshift{0.645450in}{0.850744in}%
\pgfsys@useobject{currentmarker}{}%
\end{pgfscope}%
\end{pgfscope}%
\begin{pgfscope}%
\pgfpathrectangle{\pgfqpoint{0.645450in}{0.417642in}}{\pgfqpoint{3.372880in}{2.050688in}}%
\pgfusepath{clip}%
\pgfsetrectcap%
\pgfsetroundjoin%
\pgfsetlinewidth{0.803000pt}%
\definecolor{currentstroke}{rgb}{0.850000,0.850000,0.850000}%
\pgfsetstrokecolor{currentstroke}%
\pgfsetdash{}{0pt}%
\pgfpathmoveto{\pgfqpoint{0.645450in}{0.967792in}}%
\pgfpathlineto{\pgfqpoint{4.018330in}{0.967792in}}%
\pgfusepath{stroke}%
\end{pgfscope}%
\begin{pgfscope}%
\pgfsetbuttcap%
\pgfsetroundjoin%
\definecolor{currentfill}{rgb}{0.000000,0.000000,0.000000}%
\pgfsetfillcolor{currentfill}%
\pgfsetlinewidth{0.602250pt}%
\definecolor{currentstroke}{rgb}{0.000000,0.000000,0.000000}%
\pgfsetstrokecolor{currentstroke}%
\pgfsetdash{}{0pt}%
\pgfsys@defobject{currentmarker}{\pgfqpoint{-0.027778in}{0.000000in}}{\pgfqpoint{-0.000000in}{0.000000in}}{%
\pgfpathmoveto{\pgfqpoint{-0.000000in}{0.000000in}}%
\pgfpathlineto{\pgfqpoint{-0.027778in}{0.000000in}}%
\pgfusepath{stroke,fill}%
}%
\begin{pgfscope}%
\pgfsys@transformshift{0.645450in}{0.967792in}%
\pgfsys@useobject{currentmarker}{}%
\end{pgfscope}%
\end{pgfscope}%
\begin{pgfscope}%
\pgfpathrectangle{\pgfqpoint{0.645450in}{0.417642in}}{\pgfqpoint{3.372880in}{2.050688in}}%
\pgfusepath{clip}%
\pgfsetrectcap%
\pgfsetroundjoin%
\pgfsetlinewidth{0.803000pt}%
\definecolor{currentstroke}{rgb}{0.850000,0.850000,0.850000}%
\pgfsetstrokecolor{currentstroke}%
\pgfsetdash{}{0pt}%
\pgfpathmoveto{\pgfqpoint{0.645450in}{1.058582in}}%
\pgfpathlineto{\pgfqpoint{4.018330in}{1.058582in}}%
\pgfusepath{stroke}%
\end{pgfscope}%
\begin{pgfscope}%
\pgfsetbuttcap%
\pgfsetroundjoin%
\definecolor{currentfill}{rgb}{0.000000,0.000000,0.000000}%
\pgfsetfillcolor{currentfill}%
\pgfsetlinewidth{0.602250pt}%
\definecolor{currentstroke}{rgb}{0.000000,0.000000,0.000000}%
\pgfsetstrokecolor{currentstroke}%
\pgfsetdash{}{0pt}%
\pgfsys@defobject{currentmarker}{\pgfqpoint{-0.027778in}{0.000000in}}{\pgfqpoint{-0.000000in}{0.000000in}}{%
\pgfpathmoveto{\pgfqpoint{-0.000000in}{0.000000in}}%
\pgfpathlineto{\pgfqpoint{-0.027778in}{0.000000in}}%
\pgfusepath{stroke,fill}%
}%
\begin{pgfscope}%
\pgfsys@transformshift{0.645450in}{1.058582in}%
\pgfsys@useobject{currentmarker}{}%
\end{pgfscope}%
\end{pgfscope}%
\begin{pgfscope}%
\pgfpathrectangle{\pgfqpoint{0.645450in}{0.417642in}}{\pgfqpoint{3.372880in}{2.050688in}}%
\pgfusepath{clip}%
\pgfsetrectcap%
\pgfsetroundjoin%
\pgfsetlinewidth{0.803000pt}%
\definecolor{currentstroke}{rgb}{0.850000,0.850000,0.850000}%
\pgfsetstrokecolor{currentstroke}%
\pgfsetdash{}{0pt}%
\pgfpathmoveto{\pgfqpoint{0.645450in}{1.132763in}}%
\pgfpathlineto{\pgfqpoint{4.018330in}{1.132763in}}%
\pgfusepath{stroke}%
\end{pgfscope}%
\begin{pgfscope}%
\pgfsetbuttcap%
\pgfsetroundjoin%
\definecolor{currentfill}{rgb}{0.000000,0.000000,0.000000}%
\pgfsetfillcolor{currentfill}%
\pgfsetlinewidth{0.602250pt}%
\definecolor{currentstroke}{rgb}{0.000000,0.000000,0.000000}%
\pgfsetstrokecolor{currentstroke}%
\pgfsetdash{}{0pt}%
\pgfsys@defobject{currentmarker}{\pgfqpoint{-0.027778in}{0.000000in}}{\pgfqpoint{-0.000000in}{0.000000in}}{%
\pgfpathmoveto{\pgfqpoint{-0.000000in}{0.000000in}}%
\pgfpathlineto{\pgfqpoint{-0.027778in}{0.000000in}}%
\pgfusepath{stroke,fill}%
}%
\begin{pgfscope}%
\pgfsys@transformshift{0.645450in}{1.132763in}%
\pgfsys@useobject{currentmarker}{}%
\end{pgfscope}%
\end{pgfscope}%
\begin{pgfscope}%
\pgfpathrectangle{\pgfqpoint{0.645450in}{0.417642in}}{\pgfqpoint{3.372880in}{2.050688in}}%
\pgfusepath{clip}%
\pgfsetrectcap%
\pgfsetroundjoin%
\pgfsetlinewidth{0.803000pt}%
\definecolor{currentstroke}{rgb}{0.850000,0.850000,0.850000}%
\pgfsetstrokecolor{currentstroke}%
\pgfsetdash{}{0pt}%
\pgfpathmoveto{\pgfqpoint{0.645450in}{1.195481in}}%
\pgfpathlineto{\pgfqpoint{4.018330in}{1.195481in}}%
\pgfusepath{stroke}%
\end{pgfscope}%
\begin{pgfscope}%
\pgfsetbuttcap%
\pgfsetroundjoin%
\definecolor{currentfill}{rgb}{0.000000,0.000000,0.000000}%
\pgfsetfillcolor{currentfill}%
\pgfsetlinewidth{0.602250pt}%
\definecolor{currentstroke}{rgb}{0.000000,0.000000,0.000000}%
\pgfsetstrokecolor{currentstroke}%
\pgfsetdash{}{0pt}%
\pgfsys@defobject{currentmarker}{\pgfqpoint{-0.027778in}{0.000000in}}{\pgfqpoint{-0.000000in}{0.000000in}}{%
\pgfpathmoveto{\pgfqpoint{-0.000000in}{0.000000in}}%
\pgfpathlineto{\pgfqpoint{-0.027778in}{0.000000in}}%
\pgfusepath{stroke,fill}%
}%
\begin{pgfscope}%
\pgfsys@transformshift{0.645450in}{1.195481in}%
\pgfsys@useobject{currentmarker}{}%
\end{pgfscope}%
\end{pgfscope}%
\begin{pgfscope}%
\pgfpathrectangle{\pgfqpoint{0.645450in}{0.417642in}}{\pgfqpoint{3.372880in}{2.050688in}}%
\pgfusepath{clip}%
\pgfsetrectcap%
\pgfsetroundjoin%
\pgfsetlinewidth{0.803000pt}%
\definecolor{currentstroke}{rgb}{0.850000,0.850000,0.850000}%
\pgfsetstrokecolor{currentstroke}%
\pgfsetdash{}{0pt}%
\pgfpathmoveto{\pgfqpoint{0.645450in}{1.249811in}}%
\pgfpathlineto{\pgfqpoint{4.018330in}{1.249811in}}%
\pgfusepath{stroke}%
\end{pgfscope}%
\begin{pgfscope}%
\pgfsetbuttcap%
\pgfsetroundjoin%
\definecolor{currentfill}{rgb}{0.000000,0.000000,0.000000}%
\pgfsetfillcolor{currentfill}%
\pgfsetlinewidth{0.602250pt}%
\definecolor{currentstroke}{rgb}{0.000000,0.000000,0.000000}%
\pgfsetstrokecolor{currentstroke}%
\pgfsetdash{}{0pt}%
\pgfsys@defobject{currentmarker}{\pgfqpoint{-0.027778in}{0.000000in}}{\pgfqpoint{-0.000000in}{0.000000in}}{%
\pgfpathmoveto{\pgfqpoint{-0.000000in}{0.000000in}}%
\pgfpathlineto{\pgfqpoint{-0.027778in}{0.000000in}}%
\pgfusepath{stroke,fill}%
}%
\begin{pgfscope}%
\pgfsys@transformshift{0.645450in}{1.249811in}%
\pgfsys@useobject{currentmarker}{}%
\end{pgfscope}%
\end{pgfscope}%
\begin{pgfscope}%
\pgfpathrectangle{\pgfqpoint{0.645450in}{0.417642in}}{\pgfqpoint{3.372880in}{2.050688in}}%
\pgfusepath{clip}%
\pgfsetrectcap%
\pgfsetroundjoin%
\pgfsetlinewidth{0.803000pt}%
\definecolor{currentstroke}{rgb}{0.850000,0.850000,0.850000}%
\pgfsetstrokecolor{currentstroke}%
\pgfsetdash{}{0pt}%
\pgfpathmoveto{\pgfqpoint{0.645450in}{1.297733in}}%
\pgfpathlineto{\pgfqpoint{4.018330in}{1.297733in}}%
\pgfusepath{stroke}%
\end{pgfscope}%
\begin{pgfscope}%
\pgfsetbuttcap%
\pgfsetroundjoin%
\definecolor{currentfill}{rgb}{0.000000,0.000000,0.000000}%
\pgfsetfillcolor{currentfill}%
\pgfsetlinewidth{0.602250pt}%
\definecolor{currentstroke}{rgb}{0.000000,0.000000,0.000000}%
\pgfsetstrokecolor{currentstroke}%
\pgfsetdash{}{0pt}%
\pgfsys@defobject{currentmarker}{\pgfqpoint{-0.027778in}{0.000000in}}{\pgfqpoint{-0.000000in}{0.000000in}}{%
\pgfpathmoveto{\pgfqpoint{-0.000000in}{0.000000in}}%
\pgfpathlineto{\pgfqpoint{-0.027778in}{0.000000in}}%
\pgfusepath{stroke,fill}%
}%
\begin{pgfscope}%
\pgfsys@transformshift{0.645450in}{1.297733in}%
\pgfsys@useobject{currentmarker}{}%
\end{pgfscope}%
\end{pgfscope}%
\begin{pgfscope}%
\pgfpathrectangle{\pgfqpoint{0.645450in}{0.417642in}}{\pgfqpoint{3.372880in}{2.050688in}}%
\pgfusepath{clip}%
\pgfsetrectcap%
\pgfsetroundjoin%
\pgfsetlinewidth{0.803000pt}%
\definecolor{currentstroke}{rgb}{0.850000,0.850000,0.850000}%
\pgfsetstrokecolor{currentstroke}%
\pgfsetdash{}{0pt}%
\pgfpathmoveto{\pgfqpoint{0.645450in}{1.622619in}}%
\pgfpathlineto{\pgfqpoint{4.018330in}{1.622619in}}%
\pgfusepath{stroke}%
\end{pgfscope}%
\begin{pgfscope}%
\pgfsetbuttcap%
\pgfsetroundjoin%
\definecolor{currentfill}{rgb}{0.000000,0.000000,0.000000}%
\pgfsetfillcolor{currentfill}%
\pgfsetlinewidth{0.602250pt}%
\definecolor{currentstroke}{rgb}{0.000000,0.000000,0.000000}%
\pgfsetstrokecolor{currentstroke}%
\pgfsetdash{}{0pt}%
\pgfsys@defobject{currentmarker}{\pgfqpoint{-0.027778in}{0.000000in}}{\pgfqpoint{-0.000000in}{0.000000in}}{%
\pgfpathmoveto{\pgfqpoint{-0.000000in}{0.000000in}}%
\pgfpathlineto{\pgfqpoint{-0.027778in}{0.000000in}}%
\pgfusepath{stroke,fill}%
}%
\begin{pgfscope}%
\pgfsys@transformshift{0.645450in}{1.622619in}%
\pgfsys@useobject{currentmarker}{}%
\end{pgfscope}%
\end{pgfscope}%
\begin{pgfscope}%
\pgfpathrectangle{\pgfqpoint{0.645450in}{0.417642in}}{\pgfqpoint{3.372880in}{2.050688in}}%
\pgfusepath{clip}%
\pgfsetrectcap%
\pgfsetroundjoin%
\pgfsetlinewidth{0.803000pt}%
\definecolor{currentstroke}{rgb}{0.850000,0.850000,0.850000}%
\pgfsetstrokecolor{currentstroke}%
\pgfsetdash{}{0pt}%
\pgfpathmoveto{\pgfqpoint{0.645450in}{1.787589in}}%
\pgfpathlineto{\pgfqpoint{4.018330in}{1.787589in}}%
\pgfusepath{stroke}%
\end{pgfscope}%
\begin{pgfscope}%
\pgfsetbuttcap%
\pgfsetroundjoin%
\definecolor{currentfill}{rgb}{0.000000,0.000000,0.000000}%
\pgfsetfillcolor{currentfill}%
\pgfsetlinewidth{0.602250pt}%
\definecolor{currentstroke}{rgb}{0.000000,0.000000,0.000000}%
\pgfsetstrokecolor{currentstroke}%
\pgfsetdash{}{0pt}%
\pgfsys@defobject{currentmarker}{\pgfqpoint{-0.027778in}{0.000000in}}{\pgfqpoint{-0.000000in}{0.000000in}}{%
\pgfpathmoveto{\pgfqpoint{-0.000000in}{0.000000in}}%
\pgfpathlineto{\pgfqpoint{-0.027778in}{0.000000in}}%
\pgfusepath{stroke,fill}%
}%
\begin{pgfscope}%
\pgfsys@transformshift{0.645450in}{1.787589in}%
\pgfsys@useobject{currentmarker}{}%
\end{pgfscope}%
\end{pgfscope}%
\begin{pgfscope}%
\pgfpathrectangle{\pgfqpoint{0.645450in}{0.417642in}}{\pgfqpoint{3.372880in}{2.050688in}}%
\pgfusepath{clip}%
\pgfsetrectcap%
\pgfsetroundjoin%
\pgfsetlinewidth{0.803000pt}%
\definecolor{currentstroke}{rgb}{0.850000,0.850000,0.850000}%
\pgfsetstrokecolor{currentstroke}%
\pgfsetdash{}{0pt}%
\pgfpathmoveto{\pgfqpoint{0.645450in}{1.904637in}}%
\pgfpathlineto{\pgfqpoint{4.018330in}{1.904637in}}%
\pgfusepath{stroke}%
\end{pgfscope}%
\begin{pgfscope}%
\pgfsetbuttcap%
\pgfsetroundjoin%
\definecolor{currentfill}{rgb}{0.000000,0.000000,0.000000}%
\pgfsetfillcolor{currentfill}%
\pgfsetlinewidth{0.602250pt}%
\definecolor{currentstroke}{rgb}{0.000000,0.000000,0.000000}%
\pgfsetstrokecolor{currentstroke}%
\pgfsetdash{}{0pt}%
\pgfsys@defobject{currentmarker}{\pgfqpoint{-0.027778in}{0.000000in}}{\pgfqpoint{-0.000000in}{0.000000in}}{%
\pgfpathmoveto{\pgfqpoint{-0.000000in}{0.000000in}}%
\pgfpathlineto{\pgfqpoint{-0.027778in}{0.000000in}}%
\pgfusepath{stroke,fill}%
}%
\begin{pgfscope}%
\pgfsys@transformshift{0.645450in}{1.904637in}%
\pgfsys@useobject{currentmarker}{}%
\end{pgfscope}%
\end{pgfscope}%
\begin{pgfscope}%
\pgfpathrectangle{\pgfqpoint{0.645450in}{0.417642in}}{\pgfqpoint{3.372880in}{2.050688in}}%
\pgfusepath{clip}%
\pgfsetrectcap%
\pgfsetroundjoin%
\pgfsetlinewidth{0.803000pt}%
\definecolor{currentstroke}{rgb}{0.850000,0.850000,0.850000}%
\pgfsetstrokecolor{currentstroke}%
\pgfsetdash{}{0pt}%
\pgfpathmoveto{\pgfqpoint{0.645450in}{1.995427in}}%
\pgfpathlineto{\pgfqpoint{4.018330in}{1.995427in}}%
\pgfusepath{stroke}%
\end{pgfscope}%
\begin{pgfscope}%
\pgfsetbuttcap%
\pgfsetroundjoin%
\definecolor{currentfill}{rgb}{0.000000,0.000000,0.000000}%
\pgfsetfillcolor{currentfill}%
\pgfsetlinewidth{0.602250pt}%
\definecolor{currentstroke}{rgb}{0.000000,0.000000,0.000000}%
\pgfsetstrokecolor{currentstroke}%
\pgfsetdash{}{0pt}%
\pgfsys@defobject{currentmarker}{\pgfqpoint{-0.027778in}{0.000000in}}{\pgfqpoint{-0.000000in}{0.000000in}}{%
\pgfpathmoveto{\pgfqpoint{-0.000000in}{0.000000in}}%
\pgfpathlineto{\pgfqpoint{-0.027778in}{0.000000in}}%
\pgfusepath{stroke,fill}%
}%
\begin{pgfscope}%
\pgfsys@transformshift{0.645450in}{1.995427in}%
\pgfsys@useobject{currentmarker}{}%
\end{pgfscope}%
\end{pgfscope}%
\begin{pgfscope}%
\pgfpathrectangle{\pgfqpoint{0.645450in}{0.417642in}}{\pgfqpoint{3.372880in}{2.050688in}}%
\pgfusepath{clip}%
\pgfsetrectcap%
\pgfsetroundjoin%
\pgfsetlinewidth{0.803000pt}%
\definecolor{currentstroke}{rgb}{0.850000,0.850000,0.850000}%
\pgfsetstrokecolor{currentstroke}%
\pgfsetdash{}{0pt}%
\pgfpathmoveto{\pgfqpoint{0.645450in}{2.069607in}}%
\pgfpathlineto{\pgfqpoint{4.018330in}{2.069607in}}%
\pgfusepath{stroke}%
\end{pgfscope}%
\begin{pgfscope}%
\pgfsetbuttcap%
\pgfsetroundjoin%
\definecolor{currentfill}{rgb}{0.000000,0.000000,0.000000}%
\pgfsetfillcolor{currentfill}%
\pgfsetlinewidth{0.602250pt}%
\definecolor{currentstroke}{rgb}{0.000000,0.000000,0.000000}%
\pgfsetstrokecolor{currentstroke}%
\pgfsetdash{}{0pt}%
\pgfsys@defobject{currentmarker}{\pgfqpoint{-0.027778in}{0.000000in}}{\pgfqpoint{-0.000000in}{0.000000in}}{%
\pgfpathmoveto{\pgfqpoint{-0.000000in}{0.000000in}}%
\pgfpathlineto{\pgfqpoint{-0.027778in}{0.000000in}}%
\pgfusepath{stroke,fill}%
}%
\begin{pgfscope}%
\pgfsys@transformshift{0.645450in}{2.069607in}%
\pgfsys@useobject{currentmarker}{}%
\end{pgfscope}%
\end{pgfscope}%
\begin{pgfscope}%
\pgfpathrectangle{\pgfqpoint{0.645450in}{0.417642in}}{\pgfqpoint{3.372880in}{2.050688in}}%
\pgfusepath{clip}%
\pgfsetrectcap%
\pgfsetroundjoin%
\pgfsetlinewidth{0.803000pt}%
\definecolor{currentstroke}{rgb}{0.850000,0.850000,0.850000}%
\pgfsetstrokecolor{currentstroke}%
\pgfsetdash{}{0pt}%
\pgfpathmoveto{\pgfqpoint{0.645450in}{2.132326in}}%
\pgfpathlineto{\pgfqpoint{4.018330in}{2.132326in}}%
\pgfusepath{stroke}%
\end{pgfscope}%
\begin{pgfscope}%
\pgfsetbuttcap%
\pgfsetroundjoin%
\definecolor{currentfill}{rgb}{0.000000,0.000000,0.000000}%
\pgfsetfillcolor{currentfill}%
\pgfsetlinewidth{0.602250pt}%
\definecolor{currentstroke}{rgb}{0.000000,0.000000,0.000000}%
\pgfsetstrokecolor{currentstroke}%
\pgfsetdash{}{0pt}%
\pgfsys@defobject{currentmarker}{\pgfqpoint{-0.027778in}{0.000000in}}{\pgfqpoint{-0.000000in}{0.000000in}}{%
\pgfpathmoveto{\pgfqpoint{-0.000000in}{0.000000in}}%
\pgfpathlineto{\pgfqpoint{-0.027778in}{0.000000in}}%
\pgfusepath{stroke,fill}%
}%
\begin{pgfscope}%
\pgfsys@transformshift{0.645450in}{2.132326in}%
\pgfsys@useobject{currentmarker}{}%
\end{pgfscope}%
\end{pgfscope}%
\begin{pgfscope}%
\pgfpathrectangle{\pgfqpoint{0.645450in}{0.417642in}}{\pgfqpoint{3.372880in}{2.050688in}}%
\pgfusepath{clip}%
\pgfsetrectcap%
\pgfsetroundjoin%
\pgfsetlinewidth{0.803000pt}%
\definecolor{currentstroke}{rgb}{0.850000,0.850000,0.850000}%
\pgfsetstrokecolor{currentstroke}%
\pgfsetdash{}{0pt}%
\pgfpathmoveto{\pgfqpoint{0.645450in}{2.186655in}}%
\pgfpathlineto{\pgfqpoint{4.018330in}{2.186655in}}%
\pgfusepath{stroke}%
\end{pgfscope}%
\begin{pgfscope}%
\pgfsetbuttcap%
\pgfsetroundjoin%
\definecolor{currentfill}{rgb}{0.000000,0.000000,0.000000}%
\pgfsetfillcolor{currentfill}%
\pgfsetlinewidth{0.602250pt}%
\definecolor{currentstroke}{rgb}{0.000000,0.000000,0.000000}%
\pgfsetstrokecolor{currentstroke}%
\pgfsetdash{}{0pt}%
\pgfsys@defobject{currentmarker}{\pgfqpoint{-0.027778in}{0.000000in}}{\pgfqpoint{-0.000000in}{0.000000in}}{%
\pgfpathmoveto{\pgfqpoint{-0.000000in}{0.000000in}}%
\pgfpathlineto{\pgfqpoint{-0.027778in}{0.000000in}}%
\pgfusepath{stroke,fill}%
}%
\begin{pgfscope}%
\pgfsys@transformshift{0.645450in}{2.186655in}%
\pgfsys@useobject{currentmarker}{}%
\end{pgfscope}%
\end{pgfscope}%
\begin{pgfscope}%
\pgfpathrectangle{\pgfqpoint{0.645450in}{0.417642in}}{\pgfqpoint{3.372880in}{2.050688in}}%
\pgfusepath{clip}%
\pgfsetrectcap%
\pgfsetroundjoin%
\pgfsetlinewidth{0.803000pt}%
\definecolor{currentstroke}{rgb}{0.850000,0.850000,0.850000}%
\pgfsetstrokecolor{currentstroke}%
\pgfsetdash{}{0pt}%
\pgfpathmoveto{\pgfqpoint{0.645450in}{2.234577in}}%
\pgfpathlineto{\pgfqpoint{4.018330in}{2.234577in}}%
\pgfusepath{stroke}%
\end{pgfscope}%
\begin{pgfscope}%
\pgfsetbuttcap%
\pgfsetroundjoin%
\definecolor{currentfill}{rgb}{0.000000,0.000000,0.000000}%
\pgfsetfillcolor{currentfill}%
\pgfsetlinewidth{0.602250pt}%
\definecolor{currentstroke}{rgb}{0.000000,0.000000,0.000000}%
\pgfsetstrokecolor{currentstroke}%
\pgfsetdash{}{0pt}%
\pgfsys@defobject{currentmarker}{\pgfqpoint{-0.027778in}{0.000000in}}{\pgfqpoint{-0.000000in}{0.000000in}}{%
\pgfpathmoveto{\pgfqpoint{-0.000000in}{0.000000in}}%
\pgfpathlineto{\pgfqpoint{-0.027778in}{0.000000in}}%
\pgfusepath{stroke,fill}%
}%
\begin{pgfscope}%
\pgfsys@transformshift{0.645450in}{2.234577in}%
\pgfsys@useobject{currentmarker}{}%
\end{pgfscope}%
\end{pgfscope}%
\begin{pgfscope}%
\definecolor{textcolor}{rgb}{0.000000,0.000000,0.000000}%
\pgfsetstrokecolor{textcolor}%
\pgfsetfillcolor{textcolor}%
\pgftext[x=0.185574in,y=1.442986in,,bottom,rotate=90.000000]{\color{textcolor}\rmfamily\fontsize{10.000000}{12.000000}\selectfont  \(\displaystyle S_y(f)\) in \(\displaystyle \unit{1 \per \Hz}\)}%
\end{pgfscope}%
\begin{pgfscope}%
\pgfpathrectangle{\pgfqpoint{0.645450in}{0.417642in}}{\pgfqpoint{3.372880in}{2.050688in}}%
\pgfusepath{clip}%
\pgfsetbuttcap%
\pgfsetroundjoin%
\definecolor{currentfill}{rgb}{0.337255,0.705882,0.913725}%
\pgfsetfillcolor{currentfill}%
\pgfsetlinewidth{1.003750pt}%
\definecolor{currentstroke}{rgb}{0.337255,0.705882,0.913725}%
\pgfsetstrokecolor{currentstroke}%
\pgfsetdash{}{0pt}%
\pgfsys@defobject{currentmarker}{\pgfqpoint{-0.013889in}{-0.013889in}}{\pgfqpoint{0.013889in}{0.013889in}}{%
\pgfpathmoveto{\pgfqpoint{0.000000in}{-0.013889in}}%
\pgfpathcurveto{\pgfqpoint{0.003683in}{-0.013889in}}{\pgfqpoint{0.007216in}{-0.012425in}}{\pgfqpoint{0.009821in}{-0.009821in}}%
\pgfpathcurveto{\pgfqpoint{0.012425in}{-0.007216in}}{\pgfqpoint{0.013889in}{-0.003683in}}{\pgfqpoint{0.013889in}{0.000000in}}%
\pgfpathcurveto{\pgfqpoint{0.013889in}{0.003683in}}{\pgfqpoint{0.012425in}{0.007216in}}{\pgfqpoint{0.009821in}{0.009821in}}%
\pgfpathcurveto{\pgfqpoint{0.007216in}{0.012425in}}{\pgfqpoint{0.003683in}{0.013889in}}{\pgfqpoint{0.000000in}{0.013889in}}%
\pgfpathcurveto{\pgfqpoint{-0.003683in}{0.013889in}}{\pgfqpoint{-0.007216in}{0.012425in}}{\pgfqpoint{-0.009821in}{0.009821in}}%
\pgfpathcurveto{\pgfqpoint{-0.012425in}{0.007216in}}{\pgfqpoint{-0.013889in}{0.003683in}}{\pgfqpoint{-0.013889in}{0.000000in}}%
\pgfpathcurveto{\pgfqpoint{-0.013889in}{-0.003683in}}{\pgfqpoint{-0.012425in}{-0.007216in}}{\pgfqpoint{-0.009821in}{-0.009821in}}%
\pgfpathcurveto{\pgfqpoint{-0.007216in}{-0.012425in}}{\pgfqpoint{-0.003683in}{-0.013889in}}{\pgfqpoint{0.000000in}{-0.013889in}}%
\pgfpathlineto{\pgfqpoint{0.000000in}{-0.013889in}}%
\pgfpathclose%
\pgfusepath{stroke,fill}%
}%
\begin{pgfscope}%
\pgfsys@transformshift{0.798763in}{2.369539in}%
\pgfsys@useobject{currentmarker}{}%
\end{pgfscope}%
\begin{pgfscope}%
\pgfsys@transformshift{0.815909in}{2.355388in}%
\pgfsys@useobject{currentmarker}{}%
\end{pgfscope}%
\begin{pgfscope}%
\pgfsys@transformshift{0.833077in}{2.337119in}%
\pgfsys@useobject{currentmarker}{}%
\end{pgfscope}%
\begin{pgfscope}%
\pgfsys@transformshift{0.850295in}{2.345512in}%
\pgfsys@useobject{currentmarker}{}%
\end{pgfscope}%
\begin{pgfscope}%
\pgfsys@transformshift{0.867491in}{2.339817in}%
\pgfsys@useobject{currentmarker}{}%
\end{pgfscope}%
\begin{pgfscope}%
\pgfsys@transformshift{0.884644in}{2.301971in}%
\pgfsys@useobject{currentmarker}{}%
\end{pgfscope}%
\begin{pgfscope}%
\pgfsys@transformshift{0.901789in}{2.292868in}%
\pgfsys@useobject{currentmarker}{}%
\end{pgfscope}%
\begin{pgfscope}%
\pgfsys@transformshift{0.918950in}{2.259504in}%
\pgfsys@useobject{currentmarker}{}%
\end{pgfscope}%
\begin{pgfscope}%
\pgfsys@transformshift{0.936151in}{2.279087in}%
\pgfsys@useobject{currentmarker}{}%
\end{pgfscope}%
\begin{pgfscope}%
\pgfsys@transformshift{0.953327in}{2.244986in}%
\pgfsys@useobject{currentmarker}{}%
\end{pgfscope}%
\begin{pgfscope}%
\pgfsys@transformshift{0.970499in}{2.228562in}%
\pgfsys@useobject{currentmarker}{}%
\end{pgfscope}%
\begin{pgfscope}%
\pgfsys@transformshift{0.987690in}{2.218829in}%
\pgfsys@useobject{currentmarker}{}%
\end{pgfscope}%
\begin{pgfscope}%
\pgfsys@transformshift{1.004877in}{2.228582in}%
\pgfsys@useobject{currentmarker}{}%
\end{pgfscope}%
\begin{pgfscope}%
\pgfsys@transformshift{1.022041in}{2.209100in}%
\pgfsys@useobject{currentmarker}{}%
\end{pgfscope}%
\begin{pgfscope}%
\pgfsys@transformshift{1.039200in}{2.190229in}%
\pgfsys@useobject{currentmarker}{}%
\end{pgfscope}%
\begin{pgfscope}%
\pgfsys@transformshift{1.056405in}{2.170675in}%
\pgfsys@useobject{currentmarker}{}%
\end{pgfscope}%
\begin{pgfscope}%
\pgfsys@transformshift{1.073597in}{2.154113in}%
\pgfsys@useobject{currentmarker}{}%
\end{pgfscope}%
\begin{pgfscope}%
\pgfsys@transformshift{1.090758in}{2.148187in}%
\pgfsys@useobject{currentmarker}{}%
\end{pgfscope}%
\begin{pgfscope}%
\pgfsys@transformshift{1.107935in}{2.135436in}%
\pgfsys@useobject{currentmarker}{}%
\end{pgfscope}%
\begin{pgfscope}%
\pgfsys@transformshift{1.125104in}{2.128805in}%
\pgfsys@useobject{currentmarker}{}%
\end{pgfscope}%
\begin{pgfscope}%
\pgfsys@transformshift{1.142277in}{2.127870in}%
\pgfsys@useobject{currentmarker}{}%
\end{pgfscope}%
\begin{pgfscope}%
\pgfsys@transformshift{1.159462in}{2.103854in}%
\pgfsys@useobject{currentmarker}{}%
\end{pgfscope}%
\begin{pgfscope}%
\pgfsys@transformshift{1.176637in}{2.054866in}%
\pgfsys@useobject{currentmarker}{}%
\end{pgfscope}%
\begin{pgfscope}%
\pgfsys@transformshift{1.193811in}{2.038818in}%
\pgfsys@useobject{currentmarker}{}%
\end{pgfscope}%
\begin{pgfscope}%
\pgfsys@transformshift{1.210988in}{2.069198in}%
\pgfsys@useobject{currentmarker}{}%
\end{pgfscope}%
\begin{pgfscope}%
\pgfsys@transformshift{1.228173in}{2.049803in}%
\pgfsys@useobject{currentmarker}{}%
\end{pgfscope}%
\begin{pgfscope}%
\pgfsys@transformshift{1.245346in}{2.046012in}%
\pgfsys@useobject{currentmarker}{}%
\end{pgfscope}%
\begin{pgfscope}%
\pgfsys@transformshift{1.262511in}{2.010404in}%
\pgfsys@useobject{currentmarker}{}%
\end{pgfscope}%
\begin{pgfscope}%
\pgfsys@transformshift{1.279693in}{1.994390in}%
\pgfsys@useobject{currentmarker}{}%
\end{pgfscope}%
\begin{pgfscope}%
\pgfsys@transformshift{1.296871in}{1.997541in}%
\pgfsys@useobject{currentmarker}{}%
\end{pgfscope}%
\begin{pgfscope}%
\pgfsys@transformshift{1.314045in}{1.964752in}%
\pgfsys@useobject{currentmarker}{}%
\end{pgfscope}%
\begin{pgfscope}%
\pgfsys@transformshift{1.331237in}{1.941037in}%
\pgfsys@useobject{currentmarker}{}%
\end{pgfscope}%
\begin{pgfscope}%
\pgfsys@transformshift{1.348424in}{1.941346in}%
\pgfsys@useobject{currentmarker}{}%
\end{pgfscope}%
\begin{pgfscope}%
\pgfsys@transformshift{1.365588in}{1.942168in}%
\pgfsys@useobject{currentmarker}{}%
\end{pgfscope}%
\begin{pgfscope}%
\pgfsys@transformshift{1.382764in}{1.933287in}%
\pgfsys@useobject{currentmarker}{}%
\end{pgfscope}%
\begin{pgfscope}%
\pgfsys@transformshift{1.399949in}{1.925064in}%
\pgfsys@useobject{currentmarker}{}%
\end{pgfscope}%
\begin{pgfscope}%
\pgfsys@transformshift{1.417122in}{1.906971in}%
\pgfsys@useobject{currentmarker}{}%
\end{pgfscope}%
\begin{pgfscope}%
\pgfsys@transformshift{1.434298in}{1.874060in}%
\pgfsys@useobject{currentmarker}{}%
\end{pgfscope}%
\begin{pgfscope}%
\pgfsys@transformshift{1.451471in}{1.863300in}%
\pgfsys@useobject{currentmarker}{}%
\end{pgfscope}%
\begin{pgfscope}%
\pgfsys@transformshift{1.468652in}{1.846963in}%
\pgfsys@useobject{currentmarker}{}%
\end{pgfscope}%
\begin{pgfscope}%
\pgfsys@transformshift{1.485835in}{1.842656in}%
\pgfsys@useobject{currentmarker}{}%
\end{pgfscope}%
\begin{pgfscope}%
\pgfsys@transformshift{1.503000in}{1.828376in}%
\pgfsys@useobject{currentmarker}{}%
\end{pgfscope}%
\begin{pgfscope}%
\pgfsys@transformshift{1.520170in}{1.810330in}%
\pgfsys@useobject{currentmarker}{}%
\end{pgfscope}%
\begin{pgfscope}%
\pgfsys@transformshift{1.537352in}{1.787311in}%
\pgfsys@useobject{currentmarker}{}%
\end{pgfscope}%
\begin{pgfscope}%
\pgfsys@transformshift{1.554535in}{1.788856in}%
\pgfsys@useobject{currentmarker}{}%
\end{pgfscope}%
\begin{pgfscope}%
\pgfsys@transformshift{1.571714in}{1.775845in}%
\pgfsys@useobject{currentmarker}{}%
\end{pgfscope}%
\begin{pgfscope}%
\pgfsys@transformshift{1.588880in}{1.763342in}%
\pgfsys@useobject{currentmarker}{}%
\end{pgfscope}%
\begin{pgfscope}%
\pgfsys@transformshift{1.606061in}{1.757766in}%
\pgfsys@useobject{currentmarker}{}%
\end{pgfscope}%
\begin{pgfscope}%
\pgfsys@transformshift{1.623246in}{1.739553in}%
\pgfsys@useobject{currentmarker}{}%
\end{pgfscope}%
\begin{pgfscope}%
\pgfsys@transformshift{1.640415in}{1.736345in}%
\pgfsys@useobject{currentmarker}{}%
\end{pgfscope}%
\begin{pgfscope}%
\pgfsys@transformshift{1.657591in}{1.708728in}%
\pgfsys@useobject{currentmarker}{}%
\end{pgfscope}%
\begin{pgfscope}%
\pgfsys@transformshift{1.674773in}{1.694707in}%
\pgfsys@useobject{currentmarker}{}%
\end{pgfscope}%
\begin{pgfscope}%
\pgfsys@transformshift{1.691951in}{1.676768in}%
\pgfsys@useobject{currentmarker}{}%
\end{pgfscope}%
\begin{pgfscope}%
\pgfsys@transformshift{1.709125in}{1.665896in}%
\pgfsys@useobject{currentmarker}{}%
\end{pgfscope}%
\begin{pgfscope}%
\pgfsys@transformshift{1.726300in}{1.661451in}%
\pgfsys@useobject{currentmarker}{}%
\end{pgfscope}%
\begin{pgfscope}%
\pgfsys@transformshift{1.743475in}{1.642215in}%
\pgfsys@useobject{currentmarker}{}%
\end{pgfscope}%
\begin{pgfscope}%
\pgfsys@transformshift{1.760647in}{1.648606in}%
\pgfsys@useobject{currentmarker}{}%
\end{pgfscope}%
\begin{pgfscope}%
\pgfsys@transformshift{1.777828in}{1.618853in}%
\pgfsys@useobject{currentmarker}{}%
\end{pgfscope}%
\begin{pgfscope}%
\pgfsys@transformshift{1.795006in}{1.616543in}%
\pgfsys@useobject{currentmarker}{}%
\end{pgfscope}%
\begin{pgfscope}%
\pgfsys@transformshift{1.812184in}{1.608467in}%
\pgfsys@useobject{currentmarker}{}%
\end{pgfscope}%
\begin{pgfscope}%
\pgfsys@transformshift{1.829363in}{1.582848in}%
\pgfsys@useobject{currentmarker}{}%
\end{pgfscope}%
\begin{pgfscope}%
\pgfsys@transformshift{1.846538in}{1.569411in}%
\pgfsys@useobject{currentmarker}{}%
\end{pgfscope}%
\begin{pgfscope}%
\pgfsys@transformshift{1.863717in}{1.552584in}%
\pgfsys@useobject{currentmarker}{}%
\end{pgfscope}%
\begin{pgfscope}%
\pgfsys@transformshift{1.880892in}{1.561243in}%
\pgfsys@useobject{currentmarker}{}%
\end{pgfscope}%
\begin{pgfscope}%
\pgfsys@transformshift{1.898069in}{1.544903in}%
\pgfsys@useobject{currentmarker}{}%
\end{pgfscope}%
\begin{pgfscope}%
\pgfsys@transformshift{1.915248in}{1.530007in}%
\pgfsys@useobject{currentmarker}{}%
\end{pgfscope}%
\begin{pgfscope}%
\pgfsys@transformshift{1.932419in}{1.514105in}%
\pgfsys@useobject{currentmarker}{}%
\end{pgfscope}%
\begin{pgfscope}%
\pgfsys@transformshift{1.949598in}{1.498392in}%
\pgfsys@useobject{currentmarker}{}%
\end{pgfscope}%
\begin{pgfscope}%
\pgfsys@transformshift{1.966775in}{1.496286in}%
\pgfsys@useobject{currentmarker}{}%
\end{pgfscope}%
\begin{pgfscope}%
\pgfsys@transformshift{1.983952in}{1.482095in}%
\pgfsys@useobject{currentmarker}{}%
\end{pgfscope}%
\begin{pgfscope}%
\pgfsys@transformshift{2.001130in}{1.465992in}%
\pgfsys@useobject{currentmarker}{}%
\end{pgfscope}%
\begin{pgfscope}%
\pgfsys@transformshift{2.018304in}{1.466003in}%
\pgfsys@useobject{currentmarker}{}%
\end{pgfscope}%
\begin{pgfscope}%
\pgfsys@transformshift{2.035483in}{1.451519in}%
\pgfsys@useobject{currentmarker}{}%
\end{pgfscope}%
\begin{pgfscope}%
\pgfsys@transformshift{2.052661in}{1.442309in}%
\pgfsys@useobject{currentmarker}{}%
\end{pgfscope}%
\begin{pgfscope}%
\pgfsys@transformshift{2.069835in}{1.424957in}%
\pgfsys@useobject{currentmarker}{}%
\end{pgfscope}%
\begin{pgfscope}%
\pgfsys@transformshift{2.087014in}{1.420765in}%
\pgfsys@useobject{currentmarker}{}%
\end{pgfscope}%
\begin{pgfscope}%
\pgfsys@transformshift{2.104192in}{1.411034in}%
\pgfsys@useobject{currentmarker}{}%
\end{pgfscope}%
\begin{pgfscope}%
\pgfsys@transformshift{2.121366in}{1.397100in}%
\pgfsys@useobject{currentmarker}{}%
\end{pgfscope}%
\begin{pgfscope}%
\pgfsys@transformshift{2.138545in}{1.378209in}%
\pgfsys@useobject{currentmarker}{}%
\end{pgfscope}%
\begin{pgfscope}%
\pgfsys@transformshift{2.155722in}{1.375356in}%
\pgfsys@useobject{currentmarker}{}%
\end{pgfscope}%
\begin{pgfscope}%
\pgfsys@transformshift{2.172897in}{1.361141in}%
\pgfsys@useobject{currentmarker}{}%
\end{pgfscope}%
\begin{pgfscope}%
\pgfsys@transformshift{2.190075in}{1.345863in}%
\pgfsys@useobject{currentmarker}{}%
\end{pgfscope}%
\begin{pgfscope}%
\pgfsys@transformshift{2.207252in}{1.342483in}%
\pgfsys@useobject{currentmarker}{}%
\end{pgfscope}%
\begin{pgfscope}%
\pgfsys@transformshift{2.224428in}{1.328964in}%
\pgfsys@useobject{currentmarker}{}%
\end{pgfscope}%
\begin{pgfscope}%
\pgfsys@transformshift{2.241606in}{1.322628in}%
\pgfsys@useobject{currentmarker}{}%
\end{pgfscope}%
\begin{pgfscope}%
\pgfsys@transformshift{2.258784in}{1.314357in}%
\pgfsys@useobject{currentmarker}{}%
\end{pgfscope}%
\begin{pgfscope}%
\pgfsys@transformshift{2.275960in}{1.291284in}%
\pgfsys@useobject{currentmarker}{}%
\end{pgfscope}%
\begin{pgfscope}%
\pgfsys@transformshift{2.293137in}{1.292678in}%
\pgfsys@useobject{currentmarker}{}%
\end{pgfscope}%
\begin{pgfscope}%
\pgfsys@transformshift{2.310315in}{1.282975in}%
\pgfsys@useobject{currentmarker}{}%
\end{pgfscope}%
\begin{pgfscope}%
\pgfsys@transformshift{2.327492in}{1.266017in}%
\pgfsys@useobject{currentmarker}{}%
\end{pgfscope}%
\begin{pgfscope}%
\pgfsys@transformshift{2.344667in}{1.250564in}%
\pgfsys@useobject{currentmarker}{}%
\end{pgfscope}%
\begin{pgfscope}%
\pgfsys@transformshift{2.361842in}{1.251848in}%
\pgfsys@useobject{currentmarker}{}%
\end{pgfscope}%
\begin{pgfscope}%
\pgfsys@transformshift{2.379019in}{1.239491in}%
\pgfsys@useobject{currentmarker}{}%
\end{pgfscope}%
\begin{pgfscope}%
\pgfsys@transformshift{2.396196in}{1.228690in}%
\pgfsys@useobject{currentmarker}{}%
\end{pgfscope}%
\begin{pgfscope}%
\pgfsys@transformshift{2.413375in}{1.215646in}%
\pgfsys@useobject{currentmarker}{}%
\end{pgfscope}%
\begin{pgfscope}%
\pgfsys@transformshift{2.430551in}{1.212407in}%
\pgfsys@useobject{currentmarker}{}%
\end{pgfscope}%
\begin{pgfscope}%
\pgfsys@transformshift{2.447728in}{1.203781in}%
\pgfsys@useobject{currentmarker}{}%
\end{pgfscope}%
\begin{pgfscope}%
\pgfsys@transformshift{2.464905in}{1.189132in}%
\pgfsys@useobject{currentmarker}{}%
\end{pgfscope}%
\begin{pgfscope}%
\pgfsys@transformshift{2.482083in}{1.188539in}%
\pgfsys@useobject{currentmarker}{}%
\end{pgfscope}%
\begin{pgfscope}%
\pgfsys@transformshift{2.499259in}{1.175722in}%
\pgfsys@useobject{currentmarker}{}%
\end{pgfscope}%
\begin{pgfscope}%
\pgfsys@transformshift{2.516435in}{1.169161in}%
\pgfsys@useobject{currentmarker}{}%
\end{pgfscope}%
\begin{pgfscope}%
\pgfsys@transformshift{2.533611in}{1.160645in}%
\pgfsys@useobject{currentmarker}{}%
\end{pgfscope}%
\begin{pgfscope}%
\pgfsys@transformshift{2.550788in}{1.151083in}%
\pgfsys@useobject{currentmarker}{}%
\end{pgfscope}%
\begin{pgfscope}%
\pgfsys@transformshift{2.567966in}{1.147466in}%
\pgfsys@useobject{currentmarker}{}%
\end{pgfscope}%
\begin{pgfscope}%
\pgfsys@transformshift{2.585142in}{1.130067in}%
\pgfsys@useobject{currentmarker}{}%
\end{pgfscope}%
\begin{pgfscope}%
\pgfsys@transformshift{2.602320in}{1.126331in}%
\pgfsys@useobject{currentmarker}{}%
\end{pgfscope}%
\begin{pgfscope}%
\pgfsys@transformshift{2.619497in}{1.119151in}%
\pgfsys@useobject{currentmarker}{}%
\end{pgfscope}%
\begin{pgfscope}%
\pgfsys@transformshift{2.636674in}{1.110803in}%
\pgfsys@useobject{currentmarker}{}%
\end{pgfscope}%
\begin{pgfscope}%
\pgfsys@transformshift{2.653851in}{1.106339in}%
\pgfsys@useobject{currentmarker}{}%
\end{pgfscope}%
\begin{pgfscope}%
\pgfsys@transformshift{2.671026in}{1.099218in}%
\pgfsys@useobject{currentmarker}{}%
\end{pgfscope}%
\begin{pgfscope}%
\pgfsys@transformshift{2.688204in}{1.090124in}%
\pgfsys@useobject{currentmarker}{}%
\end{pgfscope}%
\begin{pgfscope}%
\pgfsys@transformshift{2.705381in}{1.081622in}%
\pgfsys@useobject{currentmarker}{}%
\end{pgfscope}%
\begin{pgfscope}%
\pgfsys@transformshift{2.722558in}{1.079393in}%
\pgfsys@useobject{currentmarker}{}%
\end{pgfscope}%
\begin{pgfscope}%
\pgfsys@transformshift{2.739735in}{1.069525in}%
\pgfsys@useobject{currentmarker}{}%
\end{pgfscope}%
\begin{pgfscope}%
\pgfsys@transformshift{2.756912in}{1.061630in}%
\pgfsys@useobject{currentmarker}{}%
\end{pgfscope}%
\begin{pgfscope}%
\pgfsys@transformshift{2.774089in}{1.054536in}%
\pgfsys@useobject{currentmarker}{}%
\end{pgfscope}%
\begin{pgfscope}%
\pgfsys@transformshift{2.791265in}{1.045623in}%
\pgfsys@useobject{currentmarker}{}%
\end{pgfscope}%
\begin{pgfscope}%
\pgfsys@transformshift{2.808442in}{1.044915in}%
\pgfsys@useobject{currentmarker}{}%
\end{pgfscope}%
\begin{pgfscope}%
\pgfsys@transformshift{2.825619in}{1.034193in}%
\pgfsys@useobject{currentmarker}{}%
\end{pgfscope}%
\begin{pgfscope}%
\pgfsys@transformshift{2.842796in}{1.032889in}%
\pgfsys@useobject{currentmarker}{}%
\end{pgfscope}%
\begin{pgfscope}%
\pgfsys@transformshift{2.859973in}{1.023181in}%
\pgfsys@useobject{currentmarker}{}%
\end{pgfscope}%
\begin{pgfscope}%
\pgfsys@transformshift{2.877150in}{1.015690in}%
\pgfsys@useobject{currentmarker}{}%
\end{pgfscope}%
\begin{pgfscope}%
\pgfsys@transformshift{2.894327in}{1.008549in}%
\pgfsys@useobject{currentmarker}{}%
\end{pgfscope}%
\begin{pgfscope}%
\pgfsys@transformshift{2.911503in}{1.004377in}%
\pgfsys@useobject{currentmarker}{}%
\end{pgfscope}%
\begin{pgfscope}%
\pgfsys@transformshift{2.928680in}{0.999273in}%
\pgfsys@useobject{currentmarker}{}%
\end{pgfscope}%
\begin{pgfscope}%
\pgfsys@transformshift{2.945857in}{0.993265in}%
\pgfsys@useobject{currentmarker}{}%
\end{pgfscope}%
\begin{pgfscope}%
\pgfsys@transformshift{2.963034in}{0.987821in}%
\pgfsys@useobject{currentmarker}{}%
\end{pgfscope}%
\begin{pgfscope}%
\pgfsys@transformshift{2.980211in}{0.983847in}%
\pgfsys@useobject{currentmarker}{}%
\end{pgfscope}%
\begin{pgfscope}%
\pgfsys@transformshift{2.997388in}{0.976047in}%
\pgfsys@useobject{currentmarker}{}%
\end{pgfscope}%
\begin{pgfscope}%
\pgfsys@transformshift{3.014564in}{0.970799in}%
\pgfsys@useobject{currentmarker}{}%
\end{pgfscope}%
\begin{pgfscope}%
\pgfsys@transformshift{3.031741in}{0.964278in}%
\pgfsys@useobject{currentmarker}{}%
\end{pgfscope}%
\begin{pgfscope}%
\pgfsys@transformshift{3.048919in}{0.964355in}%
\pgfsys@useobject{currentmarker}{}%
\end{pgfscope}%
\begin{pgfscope}%
\pgfsys@transformshift{3.066095in}{0.957361in}%
\pgfsys@useobject{currentmarker}{}%
\end{pgfscope}%
\begin{pgfscope}%
\pgfsys@transformshift{3.083272in}{0.955115in}%
\pgfsys@useobject{currentmarker}{}%
\end{pgfscope}%
\begin{pgfscope}%
\pgfsys@transformshift{3.100449in}{0.947834in}%
\pgfsys@useobject{currentmarker}{}%
\end{pgfscope}%
\begin{pgfscope}%
\pgfsys@transformshift{3.117625in}{0.947402in}%
\pgfsys@useobject{currentmarker}{}%
\end{pgfscope}%
\begin{pgfscope}%
\pgfsys@transformshift{3.134803in}{0.940558in}%
\pgfsys@useobject{currentmarker}{}%
\end{pgfscope}%
\begin{pgfscope}%
\pgfsys@transformshift{3.151980in}{0.938391in}%
\pgfsys@useobject{currentmarker}{}%
\end{pgfscope}%
\begin{pgfscope}%
\pgfsys@transformshift{3.169157in}{0.932671in}%
\pgfsys@useobject{currentmarker}{}%
\end{pgfscope}%
\begin{pgfscope}%
\pgfsys@transformshift{3.186334in}{0.929200in}%
\pgfsys@useobject{currentmarker}{}%
\end{pgfscope}%
\begin{pgfscope}%
\pgfsys@transformshift{3.203510in}{0.926423in}%
\pgfsys@useobject{currentmarker}{}%
\end{pgfscope}%
\begin{pgfscope}%
\pgfsys@transformshift{3.220687in}{0.921929in}%
\pgfsys@useobject{currentmarker}{}%
\end{pgfscope}%
\begin{pgfscope}%
\pgfsys@transformshift{3.237864in}{0.917920in}%
\pgfsys@useobject{currentmarker}{}%
\end{pgfscope}%
\begin{pgfscope}%
\pgfsys@transformshift{3.255041in}{0.915457in}%
\pgfsys@useobject{currentmarker}{}%
\end{pgfscope}%
\begin{pgfscope}%
\pgfsys@transformshift{3.272218in}{0.912856in}%
\pgfsys@useobject{currentmarker}{}%
\end{pgfscope}%
\begin{pgfscope}%
\pgfsys@transformshift{3.289395in}{0.908592in}%
\pgfsys@useobject{currentmarker}{}%
\end{pgfscope}%
\begin{pgfscope}%
\pgfsys@transformshift{3.306572in}{0.905919in}%
\pgfsys@useobject{currentmarker}{}%
\end{pgfscope}%
\begin{pgfscope}%
\pgfsys@transformshift{3.323748in}{0.901082in}%
\pgfsys@useobject{currentmarker}{}%
\end{pgfscope}%
\begin{pgfscope}%
\pgfsys@transformshift{3.340925in}{0.900104in}%
\pgfsys@useobject{currentmarker}{}%
\end{pgfscope}%
\begin{pgfscope}%
\pgfsys@transformshift{3.358102in}{0.898832in}%
\pgfsys@useobject{currentmarker}{}%
\end{pgfscope}%
\begin{pgfscope}%
\pgfsys@transformshift{3.375279in}{0.893306in}%
\pgfsys@useobject{currentmarker}{}%
\end{pgfscope}%
\begin{pgfscope}%
\pgfsys@transformshift{3.392456in}{0.892284in}%
\pgfsys@useobject{currentmarker}{}%
\end{pgfscope}%
\begin{pgfscope}%
\pgfsys@transformshift{3.409632in}{0.888080in}%
\pgfsys@useobject{currentmarker}{}%
\end{pgfscope}%
\begin{pgfscope}%
\pgfsys@transformshift{3.426809in}{0.886720in}%
\pgfsys@useobject{currentmarker}{}%
\end{pgfscope}%
\begin{pgfscope}%
\pgfsys@transformshift{3.443986in}{0.884715in}%
\pgfsys@useobject{currentmarker}{}%
\end{pgfscope}%
\begin{pgfscope}%
\pgfsys@transformshift{3.461163in}{0.881883in}%
\pgfsys@useobject{currentmarker}{}%
\end{pgfscope}%
\begin{pgfscope}%
\pgfsys@transformshift{3.478340in}{0.878892in}%
\pgfsys@useobject{currentmarker}{}%
\end{pgfscope}%
\begin{pgfscope}%
\pgfsys@transformshift{3.495517in}{0.875331in}%
\pgfsys@useobject{currentmarker}{}%
\end{pgfscope}%
\begin{pgfscope}%
\pgfsys@transformshift{3.512694in}{0.874344in}%
\pgfsys@useobject{currentmarker}{}%
\end{pgfscope}%
\begin{pgfscope}%
\pgfsys@transformshift{3.529871in}{0.873259in}%
\pgfsys@useobject{currentmarker}{}%
\end{pgfscope}%
\begin{pgfscope}%
\pgfsys@transformshift{3.547048in}{0.868436in}%
\pgfsys@useobject{currentmarker}{}%
\end{pgfscope}%
\begin{pgfscope}%
\pgfsys@transformshift{3.564225in}{0.868001in}%
\pgfsys@useobject{currentmarker}{}%
\end{pgfscope}%
\begin{pgfscope}%
\pgfsys@transformshift{3.581401in}{0.866962in}%
\pgfsys@useobject{currentmarker}{}%
\end{pgfscope}%
\begin{pgfscope}%
\pgfsys@transformshift{3.598578in}{0.866481in}%
\pgfsys@useobject{currentmarker}{}%
\end{pgfscope}%
\begin{pgfscope}%
\pgfsys@transformshift{3.615755in}{0.862275in}%
\pgfsys@useobject{currentmarker}{}%
\end{pgfscope}%
\begin{pgfscope}%
\pgfsys@transformshift{3.632932in}{0.863571in}%
\pgfsys@useobject{currentmarker}{}%
\end{pgfscope}%
\begin{pgfscope}%
\pgfsys@transformshift{3.650109in}{0.859779in}%
\pgfsys@useobject{currentmarker}{}%
\end{pgfscope}%
\begin{pgfscope}%
\pgfsys@transformshift{3.667286in}{0.860135in}%
\pgfsys@useobject{currentmarker}{}%
\end{pgfscope}%
\begin{pgfscope}%
\pgfsys@transformshift{3.684463in}{0.859492in}%
\pgfsys@useobject{currentmarker}{}%
\end{pgfscope}%
\begin{pgfscope}%
\pgfsys@transformshift{3.701639in}{0.857949in}%
\pgfsys@useobject{currentmarker}{}%
\end{pgfscope}%
\begin{pgfscope}%
\pgfsys@transformshift{3.718816in}{0.852877in}%
\pgfsys@useobject{currentmarker}{}%
\end{pgfscope}%
\begin{pgfscope}%
\pgfsys@transformshift{3.735993in}{0.854421in}%
\pgfsys@useobject{currentmarker}{}%
\end{pgfscope}%
\begin{pgfscope}%
\pgfsys@transformshift{3.753170in}{0.854307in}%
\pgfsys@useobject{currentmarker}{}%
\end{pgfscope}%
\begin{pgfscope}%
\pgfsys@transformshift{3.770347in}{0.854438in}%
\pgfsys@useobject{currentmarker}{}%
\end{pgfscope}%
\begin{pgfscope}%
\pgfsys@transformshift{3.787524in}{0.851531in}%
\pgfsys@useobject{currentmarker}{}%
\end{pgfscope}%
\begin{pgfscope}%
\pgfsys@transformshift{3.804701in}{0.851269in}%
\pgfsys@useobject{currentmarker}{}%
\end{pgfscope}%
\begin{pgfscope}%
\pgfsys@transformshift{3.821878in}{0.850276in}%
\pgfsys@useobject{currentmarker}{}%
\end{pgfscope}%
\begin{pgfscope}%
\pgfsys@transformshift{3.839054in}{0.850664in}%
\pgfsys@useobject{currentmarker}{}%
\end{pgfscope}%
\begin{pgfscope}%
\pgfsys@transformshift{3.856231in}{0.850346in}%
\pgfsys@useobject{currentmarker}{}%
\end{pgfscope}%
\begin{pgfscope}%
\pgfsys@transformshift{3.865017in}{0.846571in}%
\pgfsys@useobject{currentmarker}{}%
\end{pgfscope}%
\end{pgfscope}%
\begin{pgfscope}%
\pgfpathrectangle{\pgfqpoint{0.645450in}{0.417642in}}{\pgfqpoint{3.372880in}{2.050688in}}%
\pgfusepath{clip}%
\pgfsetbuttcap%
\pgfsetroundjoin%
\pgfsetlinewidth{1.505625pt}%
\definecolor{currentstroke}{rgb}{0.000000,0.447059,0.698039}%
\pgfsetstrokecolor{currentstroke}%
\pgfsetdash{{5.550000pt}{2.400000pt}}{0.000000pt}%
\pgfpathmoveto{\pgfqpoint{0.798763in}{0.791268in}}%
\pgfpathlineto{\pgfqpoint{3.865017in}{0.791268in}}%
\pgfpathlineto{\pgfqpoint{3.865017in}{0.791268in}}%
\pgfusepath{stroke}%
\end{pgfscope}%
\begin{pgfscope}%
\pgfpathrectangle{\pgfqpoint{0.645450in}{0.417642in}}{\pgfqpoint{3.372880in}{2.050688in}}%
\pgfusepath{clip}%
\pgfsetbuttcap%
\pgfsetroundjoin%
\pgfsetlinewidth{1.505625pt}%
\definecolor{currentstroke}{rgb}{0.000000,0.619608,0.450980}%
\pgfsetstrokecolor{currentstroke}%
\pgfsetdash{{5.550000pt}{2.400000pt}}{0.000000pt}%
\pgfpathmoveto{\pgfqpoint{0.798763in}{2.375117in}}%
\pgfpathlineto{\pgfqpoint{3.066095in}{0.510855in}}%
\pgfpathlineto{\pgfqpoint{3.066095in}{0.510855in}}%
\pgfusepath{stroke}%
\end{pgfscope}%
\begin{pgfscope}%
\pgfsetrectcap%
\pgfsetmiterjoin%
\pgfsetlinewidth{0.803000pt}%
\definecolor{currentstroke}{rgb}{0.000000,0.000000,0.000000}%
\pgfsetstrokecolor{currentstroke}%
\pgfsetdash{}{0pt}%
\pgfpathmoveto{\pgfqpoint{0.645450in}{0.417642in}}%
\pgfpathlineto{\pgfqpoint{0.645450in}{2.468330in}}%
\pgfusepath{stroke}%
\end{pgfscope}%
\begin{pgfscope}%
\pgfsetrectcap%
\pgfsetmiterjoin%
\pgfsetlinewidth{0.803000pt}%
\definecolor{currentstroke}{rgb}{0.000000,0.000000,0.000000}%
\pgfsetstrokecolor{currentstroke}%
\pgfsetdash{}{0pt}%
\pgfpathmoveto{\pgfqpoint{4.018330in}{0.417642in}}%
\pgfpathlineto{\pgfqpoint{4.018330in}{2.468330in}}%
\pgfusepath{stroke}%
\end{pgfscope}%
\begin{pgfscope}%
\pgfsetrectcap%
\pgfsetmiterjoin%
\pgfsetlinewidth{0.803000pt}%
\definecolor{currentstroke}{rgb}{0.000000,0.000000,0.000000}%
\pgfsetstrokecolor{currentstroke}%
\pgfsetdash{}{0pt}%
\pgfpathmoveto{\pgfqpoint{0.645450in}{0.417642in}}%
\pgfpathlineto{\pgfqpoint{4.018330in}{0.417642in}}%
\pgfusepath{stroke}%
\end{pgfscope}%
\begin{pgfscope}%
\pgfsetrectcap%
\pgfsetmiterjoin%
\pgfsetlinewidth{0.803000pt}%
\definecolor{currentstroke}{rgb}{0.000000,0.000000,0.000000}%
\pgfsetstrokecolor{currentstroke}%
\pgfsetdash{}{0pt}%
\pgfpathmoveto{\pgfqpoint{0.645450in}{2.468330in}}%
\pgfpathlineto{\pgfqpoint{4.018330in}{2.468330in}}%
\pgfusepath{stroke}%
\end{pgfscope}%
\begin{pgfscope}%
\pgfsetbuttcap%
\pgfsetmiterjoin%
\definecolor{currentfill}{rgb}{1.000000,1.000000,1.000000}%
\pgfsetfillcolor{currentfill}%
\pgfsetfillopacity{0.800000}%
\pgfsetlinewidth{1.003750pt}%
\definecolor{currentstroke}{rgb}{0.800000,0.800000,0.800000}%
\pgfsetstrokecolor{currentstroke}%
\pgfsetstrokeopacity{0.800000}%
\pgfsetdash{}{0pt}%
\pgfpathmoveto{\pgfqpoint{2.940219in}{2.069664in}}%
\pgfpathlineto{\pgfqpoint{3.940552in}{2.069664in}}%
\pgfpathquadraticcurveto{\pgfqpoint{3.962774in}{2.069664in}}{\pgfqpoint{3.962774in}{2.091886in}}%
\pgfpathlineto{\pgfqpoint{3.962774in}{2.390552in}}%
\pgfpathquadraticcurveto{\pgfqpoint{3.962774in}{2.412774in}}{\pgfqpoint{3.940552in}{2.412774in}}%
\pgfpathlineto{\pgfqpoint{2.940219in}{2.412774in}}%
\pgfpathquadraticcurveto{\pgfqpoint{2.917997in}{2.412774in}}{\pgfqpoint{2.917997in}{2.390552in}}%
\pgfpathlineto{\pgfqpoint{2.917997in}{2.091886in}}%
\pgfpathquadraticcurveto{\pgfqpoint{2.917997in}{2.069664in}}{\pgfqpoint{2.940219in}{2.069664in}}%
\pgfpathlineto{\pgfqpoint{2.940219in}{2.069664in}}%
\pgfpathclose%
\pgfusepath{stroke,fill}%
\end{pgfscope}%
\begin{pgfscope}%
\pgfsetbuttcap%
\pgfsetroundjoin%
\pgfsetlinewidth{1.505625pt}%
\definecolor{currentstroke}{rgb}{0.000000,0.447059,0.698039}%
\pgfsetstrokecolor{currentstroke}%
\pgfsetdash{{5.550000pt}{2.400000pt}}{0.000000pt}%
\pgfpathmoveto{\pgfqpoint{2.962441in}{2.329441in}}%
\pgfpathlineto{\pgfqpoint{3.073552in}{2.329441in}}%
\pgfpathlineto{\pgfqpoint{3.184663in}{2.329441in}}%
\pgfusepath{stroke}%
\end{pgfscope}%
\begin{pgfscope}%
\definecolor{textcolor}{rgb}{0.000000,0.000000,0.000000}%
\pgfsetstrokecolor{textcolor}%
\pgfsetfillcolor{textcolor}%
\pgftext[x=3.273552in,y=2.290552in,left,base]{\color{textcolor}\rmfamily\fontsize{8.000000}{9.600000}\selectfont White noise}%
\end{pgfscope}%
\begin{pgfscope}%
\pgfsetbuttcap%
\pgfsetroundjoin%
\pgfsetlinewidth{1.505625pt}%
\definecolor{currentstroke}{rgb}{0.000000,0.619608,0.450980}%
\pgfsetstrokecolor{currentstroke}%
\pgfsetdash{{5.550000pt}{2.400000pt}}{0.000000pt}%
\pgfpathmoveto{\pgfqpoint{2.962441in}{2.174552in}}%
\pgfpathlineto{\pgfqpoint{3.073552in}{2.174552in}}%
\pgfpathlineto{\pgfqpoint{3.184663in}{2.174552in}}%
\pgfusepath{stroke}%
\end{pgfscope}%
\begin{pgfscope}%
\definecolor{textcolor}{rgb}{0.000000,0.000000,0.000000}%
\pgfsetstrokecolor{textcolor}%
\pgfsetfillcolor{textcolor}%
\pgftext[x=3.273552in,y=2.135663in,left,base]{\color{textcolor}\rmfamily\fontsize{8.000000}{9.600000}\selectfont Flicker noise}%
\end{pgfscope}%
\end{pgfpicture}%
\makeatother%
\endgroup%

    \caption{Simulated power spectrum of a Keysight \device{3458A} containing white noise and flicker noise.}
    \label{fig:autozero_raw_psd}
\end{figure}

From the power spectral density is can be seeen, that higher frequencies have a significantly lower noise spectral density than the low frequencies. It is therefore most beneficial, to so measurements a higher frequencies. To discuss the optimal measurement interval, the Allan deviation is an excellent tool.

\begin{figure}[ht]
    \centering
    %% Creator: Matplotlib, PGF backend
%%
%% To include the figure in your LaTeX document, write
%%   \input{<filename>.pgf}
%%
%% Make sure the required packages are loaded in your preamble
%%   \usepackage{pgf}
%%
%% Also ensure that all the required font packages are loaded; for instance,
%% the lmodern package is sometimes necessary when using math font.
%%   \usepackage{lmodern}
%%
%% Figures using additional raster images can only be included by \input if
%% they are in the same directory as the main LaTeX file. For loading figures
%% from other directories you can use the `import` package
%%   \usepackage{import}
%%
%% and then include the figures with
%%   \import{<path to file>}{<filename>.pgf}
%%
%% Matplotlib used the following preamble
%%   \usepackage{siunitx}
%%   \usepackage{fontspec}
%%   \makeatletter\@ifpackageloaded{underscore}{}{\usepackage[strings]{underscore}}\makeatother
%%
\begingroup%
\makeatletter%
\begin{pgfpicture}%
\pgfpathrectangle{\pgfpointorigin}{\pgfqpoint{4.060000in}{2.510000in}}%
\pgfusepath{use as bounding box, clip}%
\begin{pgfscope}%
\pgfsetbuttcap%
\pgfsetmiterjoin%
\definecolor{currentfill}{rgb}{1.000000,1.000000,1.000000}%
\pgfsetfillcolor{currentfill}%
\pgfsetlinewidth{0.000000pt}%
\definecolor{currentstroke}{rgb}{1.000000,1.000000,1.000000}%
\pgfsetstrokecolor{currentstroke}%
\pgfsetdash{}{0pt}%
\pgfpathmoveto{\pgfqpoint{0.000000in}{0.000000in}}%
\pgfpathlineto{\pgfqpoint{4.060000in}{0.000000in}}%
\pgfpathlineto{\pgfqpoint{4.060000in}{2.510000in}}%
\pgfpathlineto{\pgfqpoint{0.000000in}{2.510000in}}%
\pgfpathlineto{\pgfqpoint{0.000000in}{0.000000in}}%
\pgfpathclose%
\pgfusepath{fill}%
\end{pgfscope}%
\begin{pgfscope}%
\pgfsetbuttcap%
\pgfsetmiterjoin%
\definecolor{currentfill}{rgb}{1.000000,1.000000,1.000000}%
\pgfsetfillcolor{currentfill}%
\pgfsetlinewidth{0.000000pt}%
\definecolor{currentstroke}{rgb}{0.000000,0.000000,0.000000}%
\pgfsetstrokecolor{currentstroke}%
\pgfsetstrokeopacity{0.000000}%
\pgfsetdash{}{0pt}%
\pgfpathmoveto{\pgfqpoint{0.770608in}{0.417642in}}%
\pgfpathlineto{\pgfqpoint{3.933156in}{0.417642in}}%
\pgfpathlineto{\pgfqpoint{3.933156in}{2.441829in}}%
\pgfpathlineto{\pgfqpoint{0.770608in}{2.441829in}}%
\pgfpathlineto{\pgfqpoint{0.770608in}{0.417642in}}%
\pgfpathclose%
\pgfusepath{fill}%
\end{pgfscope}%
\begin{pgfscope}%
\pgfpathrectangle{\pgfqpoint{0.770608in}{0.417642in}}{\pgfqpoint{3.162547in}{2.024187in}}%
\pgfusepath{clip}%
\pgfsetrectcap%
\pgfsetroundjoin%
\pgfsetlinewidth{0.803000pt}%
\definecolor{currentstroke}{rgb}{0.450000,0.450000,0.450000}%
\pgfsetstrokecolor{currentstroke}%
\pgfsetdash{}{0pt}%
\pgfpathmoveto{\pgfqpoint{0.914360in}{0.417642in}}%
\pgfpathlineto{\pgfqpoint{0.914360in}{2.441829in}}%
\pgfusepath{stroke}%
\end{pgfscope}%
\begin{pgfscope}%
\pgfsetbuttcap%
\pgfsetroundjoin%
\definecolor{currentfill}{rgb}{0.000000,0.000000,0.000000}%
\pgfsetfillcolor{currentfill}%
\pgfsetlinewidth{0.803000pt}%
\definecolor{currentstroke}{rgb}{0.000000,0.000000,0.000000}%
\pgfsetstrokecolor{currentstroke}%
\pgfsetdash{}{0pt}%
\pgfsys@defobject{currentmarker}{\pgfqpoint{0.000000in}{-0.048611in}}{\pgfqpoint{0.000000in}{0.000000in}}{%
\pgfpathmoveto{\pgfqpoint{0.000000in}{0.000000in}}%
\pgfpathlineto{\pgfqpoint{0.000000in}{-0.048611in}}%
\pgfusepath{stroke,fill}%
}%
\begin{pgfscope}%
\pgfsys@transformshift{0.914360in}{0.417642in}%
\pgfsys@useobject{currentmarker}{}%
\end{pgfscope}%
\end{pgfscope}%
\begin{pgfscope}%
\definecolor{textcolor}{rgb}{0.000000,0.000000,0.000000}%
\pgfsetstrokecolor{textcolor}%
\pgfsetfillcolor{textcolor}%
\pgftext[x=0.914360in,y=0.320420in,,top]{\color{textcolor}\rmfamily\fontsize{8.000000}{9.600000}\selectfont \(\displaystyle {10^{-1}}\)}%
\end{pgfscope}%
\begin{pgfscope}%
\pgfpathrectangle{\pgfqpoint{0.770608in}{0.417642in}}{\pgfqpoint{3.162547in}{2.024187in}}%
\pgfusepath{clip}%
\pgfsetrectcap%
\pgfsetroundjoin%
\pgfsetlinewidth{0.803000pt}%
\definecolor{currentstroke}{rgb}{0.450000,0.450000,0.450000}%
\pgfsetstrokecolor{currentstroke}%
\pgfsetdash{}{0pt}%
\pgfpathmoveto{\pgfqpoint{1.417028in}{0.417642in}}%
\pgfpathlineto{\pgfqpoint{1.417028in}{2.441829in}}%
\pgfusepath{stroke}%
\end{pgfscope}%
\begin{pgfscope}%
\pgfsetbuttcap%
\pgfsetroundjoin%
\definecolor{currentfill}{rgb}{0.000000,0.000000,0.000000}%
\pgfsetfillcolor{currentfill}%
\pgfsetlinewidth{0.803000pt}%
\definecolor{currentstroke}{rgb}{0.000000,0.000000,0.000000}%
\pgfsetstrokecolor{currentstroke}%
\pgfsetdash{}{0pt}%
\pgfsys@defobject{currentmarker}{\pgfqpoint{0.000000in}{-0.048611in}}{\pgfqpoint{0.000000in}{0.000000in}}{%
\pgfpathmoveto{\pgfqpoint{0.000000in}{0.000000in}}%
\pgfpathlineto{\pgfqpoint{0.000000in}{-0.048611in}}%
\pgfusepath{stroke,fill}%
}%
\begin{pgfscope}%
\pgfsys@transformshift{1.417028in}{0.417642in}%
\pgfsys@useobject{currentmarker}{}%
\end{pgfscope}%
\end{pgfscope}%
\begin{pgfscope}%
\definecolor{textcolor}{rgb}{0.000000,0.000000,0.000000}%
\pgfsetstrokecolor{textcolor}%
\pgfsetfillcolor{textcolor}%
\pgftext[x=1.417028in,y=0.320420in,,top]{\color{textcolor}\rmfamily\fontsize{8.000000}{9.600000}\selectfont \(\displaystyle {10^{0}}\)}%
\end{pgfscope}%
\begin{pgfscope}%
\pgfpathrectangle{\pgfqpoint{0.770608in}{0.417642in}}{\pgfqpoint{3.162547in}{2.024187in}}%
\pgfusepath{clip}%
\pgfsetrectcap%
\pgfsetroundjoin%
\pgfsetlinewidth{0.803000pt}%
\definecolor{currentstroke}{rgb}{0.450000,0.450000,0.450000}%
\pgfsetstrokecolor{currentstroke}%
\pgfsetdash{}{0pt}%
\pgfpathmoveto{\pgfqpoint{1.919696in}{0.417642in}}%
\pgfpathlineto{\pgfqpoint{1.919696in}{2.441829in}}%
\pgfusepath{stroke}%
\end{pgfscope}%
\begin{pgfscope}%
\pgfsetbuttcap%
\pgfsetroundjoin%
\definecolor{currentfill}{rgb}{0.000000,0.000000,0.000000}%
\pgfsetfillcolor{currentfill}%
\pgfsetlinewidth{0.803000pt}%
\definecolor{currentstroke}{rgb}{0.000000,0.000000,0.000000}%
\pgfsetstrokecolor{currentstroke}%
\pgfsetdash{}{0pt}%
\pgfsys@defobject{currentmarker}{\pgfqpoint{0.000000in}{-0.048611in}}{\pgfqpoint{0.000000in}{0.000000in}}{%
\pgfpathmoveto{\pgfqpoint{0.000000in}{0.000000in}}%
\pgfpathlineto{\pgfqpoint{0.000000in}{-0.048611in}}%
\pgfusepath{stroke,fill}%
}%
\begin{pgfscope}%
\pgfsys@transformshift{1.919696in}{0.417642in}%
\pgfsys@useobject{currentmarker}{}%
\end{pgfscope}%
\end{pgfscope}%
\begin{pgfscope}%
\definecolor{textcolor}{rgb}{0.000000,0.000000,0.000000}%
\pgfsetstrokecolor{textcolor}%
\pgfsetfillcolor{textcolor}%
\pgftext[x=1.919696in,y=0.320420in,,top]{\color{textcolor}\rmfamily\fontsize{8.000000}{9.600000}\selectfont \(\displaystyle {10^{1}}\)}%
\end{pgfscope}%
\begin{pgfscope}%
\pgfpathrectangle{\pgfqpoint{0.770608in}{0.417642in}}{\pgfqpoint{3.162547in}{2.024187in}}%
\pgfusepath{clip}%
\pgfsetrectcap%
\pgfsetroundjoin%
\pgfsetlinewidth{0.803000pt}%
\definecolor{currentstroke}{rgb}{0.450000,0.450000,0.450000}%
\pgfsetstrokecolor{currentstroke}%
\pgfsetdash{}{0pt}%
\pgfpathmoveto{\pgfqpoint{2.422364in}{0.417642in}}%
\pgfpathlineto{\pgfqpoint{2.422364in}{2.441829in}}%
\pgfusepath{stroke}%
\end{pgfscope}%
\begin{pgfscope}%
\pgfsetbuttcap%
\pgfsetroundjoin%
\definecolor{currentfill}{rgb}{0.000000,0.000000,0.000000}%
\pgfsetfillcolor{currentfill}%
\pgfsetlinewidth{0.803000pt}%
\definecolor{currentstroke}{rgb}{0.000000,0.000000,0.000000}%
\pgfsetstrokecolor{currentstroke}%
\pgfsetdash{}{0pt}%
\pgfsys@defobject{currentmarker}{\pgfqpoint{0.000000in}{-0.048611in}}{\pgfqpoint{0.000000in}{0.000000in}}{%
\pgfpathmoveto{\pgfqpoint{0.000000in}{0.000000in}}%
\pgfpathlineto{\pgfqpoint{0.000000in}{-0.048611in}}%
\pgfusepath{stroke,fill}%
}%
\begin{pgfscope}%
\pgfsys@transformshift{2.422364in}{0.417642in}%
\pgfsys@useobject{currentmarker}{}%
\end{pgfscope}%
\end{pgfscope}%
\begin{pgfscope}%
\definecolor{textcolor}{rgb}{0.000000,0.000000,0.000000}%
\pgfsetstrokecolor{textcolor}%
\pgfsetfillcolor{textcolor}%
\pgftext[x=2.422364in,y=0.320420in,,top]{\color{textcolor}\rmfamily\fontsize{8.000000}{9.600000}\selectfont \(\displaystyle {10^{2}}\)}%
\end{pgfscope}%
\begin{pgfscope}%
\pgfpathrectangle{\pgfqpoint{0.770608in}{0.417642in}}{\pgfqpoint{3.162547in}{2.024187in}}%
\pgfusepath{clip}%
\pgfsetrectcap%
\pgfsetroundjoin%
\pgfsetlinewidth{0.803000pt}%
\definecolor{currentstroke}{rgb}{0.450000,0.450000,0.450000}%
\pgfsetstrokecolor{currentstroke}%
\pgfsetdash{}{0pt}%
\pgfpathmoveto{\pgfqpoint{2.925031in}{0.417642in}}%
\pgfpathlineto{\pgfqpoint{2.925031in}{2.441829in}}%
\pgfusepath{stroke}%
\end{pgfscope}%
\begin{pgfscope}%
\pgfsetbuttcap%
\pgfsetroundjoin%
\definecolor{currentfill}{rgb}{0.000000,0.000000,0.000000}%
\pgfsetfillcolor{currentfill}%
\pgfsetlinewidth{0.803000pt}%
\definecolor{currentstroke}{rgb}{0.000000,0.000000,0.000000}%
\pgfsetstrokecolor{currentstroke}%
\pgfsetdash{}{0pt}%
\pgfsys@defobject{currentmarker}{\pgfqpoint{0.000000in}{-0.048611in}}{\pgfqpoint{0.000000in}{0.000000in}}{%
\pgfpathmoveto{\pgfqpoint{0.000000in}{0.000000in}}%
\pgfpathlineto{\pgfqpoint{0.000000in}{-0.048611in}}%
\pgfusepath{stroke,fill}%
}%
\begin{pgfscope}%
\pgfsys@transformshift{2.925031in}{0.417642in}%
\pgfsys@useobject{currentmarker}{}%
\end{pgfscope}%
\end{pgfscope}%
\begin{pgfscope}%
\definecolor{textcolor}{rgb}{0.000000,0.000000,0.000000}%
\pgfsetstrokecolor{textcolor}%
\pgfsetfillcolor{textcolor}%
\pgftext[x=2.925031in,y=0.320420in,,top]{\color{textcolor}\rmfamily\fontsize{8.000000}{9.600000}\selectfont \(\displaystyle {10^{3}}\)}%
\end{pgfscope}%
\begin{pgfscope}%
\pgfpathrectangle{\pgfqpoint{0.770608in}{0.417642in}}{\pgfqpoint{3.162547in}{2.024187in}}%
\pgfusepath{clip}%
\pgfsetrectcap%
\pgfsetroundjoin%
\pgfsetlinewidth{0.803000pt}%
\definecolor{currentstroke}{rgb}{0.450000,0.450000,0.450000}%
\pgfsetstrokecolor{currentstroke}%
\pgfsetdash{}{0pt}%
\pgfpathmoveto{\pgfqpoint{3.427699in}{0.417642in}}%
\pgfpathlineto{\pgfqpoint{3.427699in}{2.441829in}}%
\pgfusepath{stroke}%
\end{pgfscope}%
\begin{pgfscope}%
\pgfsetbuttcap%
\pgfsetroundjoin%
\definecolor{currentfill}{rgb}{0.000000,0.000000,0.000000}%
\pgfsetfillcolor{currentfill}%
\pgfsetlinewidth{0.803000pt}%
\definecolor{currentstroke}{rgb}{0.000000,0.000000,0.000000}%
\pgfsetstrokecolor{currentstroke}%
\pgfsetdash{}{0pt}%
\pgfsys@defobject{currentmarker}{\pgfqpoint{0.000000in}{-0.048611in}}{\pgfqpoint{0.000000in}{0.000000in}}{%
\pgfpathmoveto{\pgfqpoint{0.000000in}{0.000000in}}%
\pgfpathlineto{\pgfqpoint{0.000000in}{-0.048611in}}%
\pgfusepath{stroke,fill}%
}%
\begin{pgfscope}%
\pgfsys@transformshift{3.427699in}{0.417642in}%
\pgfsys@useobject{currentmarker}{}%
\end{pgfscope}%
\end{pgfscope}%
\begin{pgfscope}%
\definecolor{textcolor}{rgb}{0.000000,0.000000,0.000000}%
\pgfsetstrokecolor{textcolor}%
\pgfsetfillcolor{textcolor}%
\pgftext[x=3.427699in,y=0.320420in,,top]{\color{textcolor}\rmfamily\fontsize{8.000000}{9.600000}\selectfont \(\displaystyle {10^{4}}\)}%
\end{pgfscope}%
\begin{pgfscope}%
\pgfpathrectangle{\pgfqpoint{0.770608in}{0.417642in}}{\pgfqpoint{3.162547in}{2.024187in}}%
\pgfusepath{clip}%
\pgfsetrectcap%
\pgfsetroundjoin%
\pgfsetlinewidth{0.803000pt}%
\definecolor{currentstroke}{rgb}{0.450000,0.450000,0.450000}%
\pgfsetstrokecolor{currentstroke}%
\pgfsetdash{}{0pt}%
\pgfpathmoveto{\pgfqpoint{3.930367in}{0.417642in}}%
\pgfpathlineto{\pgfqpoint{3.930367in}{2.441829in}}%
\pgfusepath{stroke}%
\end{pgfscope}%
\begin{pgfscope}%
\pgfsetbuttcap%
\pgfsetroundjoin%
\definecolor{currentfill}{rgb}{0.000000,0.000000,0.000000}%
\pgfsetfillcolor{currentfill}%
\pgfsetlinewidth{0.803000pt}%
\definecolor{currentstroke}{rgb}{0.000000,0.000000,0.000000}%
\pgfsetstrokecolor{currentstroke}%
\pgfsetdash{}{0pt}%
\pgfsys@defobject{currentmarker}{\pgfqpoint{0.000000in}{-0.048611in}}{\pgfqpoint{0.000000in}{0.000000in}}{%
\pgfpathmoveto{\pgfqpoint{0.000000in}{0.000000in}}%
\pgfpathlineto{\pgfqpoint{0.000000in}{-0.048611in}}%
\pgfusepath{stroke,fill}%
}%
\begin{pgfscope}%
\pgfsys@transformshift{3.930367in}{0.417642in}%
\pgfsys@useobject{currentmarker}{}%
\end{pgfscope}%
\end{pgfscope}%
\begin{pgfscope}%
\definecolor{textcolor}{rgb}{0.000000,0.000000,0.000000}%
\pgfsetstrokecolor{textcolor}%
\pgfsetfillcolor{textcolor}%
\pgftext[x=3.930367in,y=0.320420in,,top]{\color{textcolor}\rmfamily\fontsize{8.000000}{9.600000}\selectfont \(\displaystyle {10^{5}}\)}%
\end{pgfscope}%
\begin{pgfscope}%
\pgfpathrectangle{\pgfqpoint{0.770608in}{0.417642in}}{\pgfqpoint{3.162547in}{2.024187in}}%
\pgfusepath{clip}%
\pgfsetrectcap%
\pgfsetroundjoin%
\pgfsetlinewidth{0.803000pt}%
\definecolor{currentstroke}{rgb}{0.850000,0.850000,0.850000}%
\pgfsetstrokecolor{currentstroke}%
\pgfsetdash{}{0pt}%
\pgfpathmoveto{\pgfqpoint{0.802844in}{0.417642in}}%
\pgfpathlineto{\pgfqpoint{0.802844in}{2.441829in}}%
\pgfusepath{stroke}%
\end{pgfscope}%
\begin{pgfscope}%
\pgfsetbuttcap%
\pgfsetroundjoin%
\definecolor{currentfill}{rgb}{0.000000,0.000000,0.000000}%
\pgfsetfillcolor{currentfill}%
\pgfsetlinewidth{0.602250pt}%
\definecolor{currentstroke}{rgb}{0.000000,0.000000,0.000000}%
\pgfsetstrokecolor{currentstroke}%
\pgfsetdash{}{0pt}%
\pgfsys@defobject{currentmarker}{\pgfqpoint{0.000000in}{-0.027778in}}{\pgfqpoint{0.000000in}{0.000000in}}{%
\pgfpathmoveto{\pgfqpoint{0.000000in}{0.000000in}}%
\pgfpathlineto{\pgfqpoint{0.000000in}{-0.027778in}}%
\pgfusepath{stroke,fill}%
}%
\begin{pgfscope}%
\pgfsys@transformshift{0.802844in}{0.417642in}%
\pgfsys@useobject{currentmarker}{}%
\end{pgfscope}%
\end{pgfscope}%
\begin{pgfscope}%
\pgfpathrectangle{\pgfqpoint{0.770608in}{0.417642in}}{\pgfqpoint{3.162547in}{2.024187in}}%
\pgfusepath{clip}%
\pgfsetrectcap%
\pgfsetroundjoin%
\pgfsetlinewidth{0.803000pt}%
\definecolor{currentstroke}{rgb}{0.850000,0.850000,0.850000}%
\pgfsetstrokecolor{currentstroke}%
\pgfsetdash{}{0pt}%
\pgfpathmoveto{\pgfqpoint{0.836496in}{0.417642in}}%
\pgfpathlineto{\pgfqpoint{0.836496in}{2.441829in}}%
\pgfusepath{stroke}%
\end{pgfscope}%
\begin{pgfscope}%
\pgfsetbuttcap%
\pgfsetroundjoin%
\definecolor{currentfill}{rgb}{0.000000,0.000000,0.000000}%
\pgfsetfillcolor{currentfill}%
\pgfsetlinewidth{0.602250pt}%
\definecolor{currentstroke}{rgb}{0.000000,0.000000,0.000000}%
\pgfsetstrokecolor{currentstroke}%
\pgfsetdash{}{0pt}%
\pgfsys@defobject{currentmarker}{\pgfqpoint{0.000000in}{-0.027778in}}{\pgfqpoint{0.000000in}{0.000000in}}{%
\pgfpathmoveto{\pgfqpoint{0.000000in}{0.000000in}}%
\pgfpathlineto{\pgfqpoint{0.000000in}{-0.027778in}}%
\pgfusepath{stroke,fill}%
}%
\begin{pgfscope}%
\pgfsys@transformshift{0.836496in}{0.417642in}%
\pgfsys@useobject{currentmarker}{}%
\end{pgfscope}%
\end{pgfscope}%
\begin{pgfscope}%
\pgfpathrectangle{\pgfqpoint{0.770608in}{0.417642in}}{\pgfqpoint{3.162547in}{2.024187in}}%
\pgfusepath{clip}%
\pgfsetrectcap%
\pgfsetroundjoin%
\pgfsetlinewidth{0.803000pt}%
\definecolor{currentstroke}{rgb}{0.850000,0.850000,0.850000}%
\pgfsetstrokecolor{currentstroke}%
\pgfsetdash{}{0pt}%
\pgfpathmoveto{\pgfqpoint{0.865647in}{0.417642in}}%
\pgfpathlineto{\pgfqpoint{0.865647in}{2.441829in}}%
\pgfusepath{stroke}%
\end{pgfscope}%
\begin{pgfscope}%
\pgfsetbuttcap%
\pgfsetroundjoin%
\definecolor{currentfill}{rgb}{0.000000,0.000000,0.000000}%
\pgfsetfillcolor{currentfill}%
\pgfsetlinewidth{0.602250pt}%
\definecolor{currentstroke}{rgb}{0.000000,0.000000,0.000000}%
\pgfsetstrokecolor{currentstroke}%
\pgfsetdash{}{0pt}%
\pgfsys@defobject{currentmarker}{\pgfqpoint{0.000000in}{-0.027778in}}{\pgfqpoint{0.000000in}{0.000000in}}{%
\pgfpathmoveto{\pgfqpoint{0.000000in}{0.000000in}}%
\pgfpathlineto{\pgfqpoint{0.000000in}{-0.027778in}}%
\pgfusepath{stroke,fill}%
}%
\begin{pgfscope}%
\pgfsys@transformshift{0.865647in}{0.417642in}%
\pgfsys@useobject{currentmarker}{}%
\end{pgfscope}%
\end{pgfscope}%
\begin{pgfscope}%
\pgfpathrectangle{\pgfqpoint{0.770608in}{0.417642in}}{\pgfqpoint{3.162547in}{2.024187in}}%
\pgfusepath{clip}%
\pgfsetrectcap%
\pgfsetroundjoin%
\pgfsetlinewidth{0.803000pt}%
\definecolor{currentstroke}{rgb}{0.850000,0.850000,0.850000}%
\pgfsetstrokecolor{currentstroke}%
\pgfsetdash{}{0pt}%
\pgfpathmoveto{\pgfqpoint{0.891360in}{0.417642in}}%
\pgfpathlineto{\pgfqpoint{0.891360in}{2.441829in}}%
\pgfusepath{stroke}%
\end{pgfscope}%
\begin{pgfscope}%
\pgfsetbuttcap%
\pgfsetroundjoin%
\definecolor{currentfill}{rgb}{0.000000,0.000000,0.000000}%
\pgfsetfillcolor{currentfill}%
\pgfsetlinewidth{0.602250pt}%
\definecolor{currentstroke}{rgb}{0.000000,0.000000,0.000000}%
\pgfsetstrokecolor{currentstroke}%
\pgfsetdash{}{0pt}%
\pgfsys@defobject{currentmarker}{\pgfqpoint{0.000000in}{-0.027778in}}{\pgfqpoint{0.000000in}{0.000000in}}{%
\pgfpathmoveto{\pgfqpoint{0.000000in}{0.000000in}}%
\pgfpathlineto{\pgfqpoint{0.000000in}{-0.027778in}}%
\pgfusepath{stroke,fill}%
}%
\begin{pgfscope}%
\pgfsys@transformshift{0.891360in}{0.417642in}%
\pgfsys@useobject{currentmarker}{}%
\end{pgfscope}%
\end{pgfscope}%
\begin{pgfscope}%
\pgfpathrectangle{\pgfqpoint{0.770608in}{0.417642in}}{\pgfqpoint{3.162547in}{2.024187in}}%
\pgfusepath{clip}%
\pgfsetrectcap%
\pgfsetroundjoin%
\pgfsetlinewidth{0.803000pt}%
\definecolor{currentstroke}{rgb}{0.850000,0.850000,0.850000}%
\pgfsetstrokecolor{currentstroke}%
\pgfsetdash{}{0pt}%
\pgfpathmoveto{\pgfqpoint{1.065679in}{0.417642in}}%
\pgfpathlineto{\pgfqpoint{1.065679in}{2.441829in}}%
\pgfusepath{stroke}%
\end{pgfscope}%
\begin{pgfscope}%
\pgfsetbuttcap%
\pgfsetroundjoin%
\definecolor{currentfill}{rgb}{0.000000,0.000000,0.000000}%
\pgfsetfillcolor{currentfill}%
\pgfsetlinewidth{0.602250pt}%
\definecolor{currentstroke}{rgb}{0.000000,0.000000,0.000000}%
\pgfsetstrokecolor{currentstroke}%
\pgfsetdash{}{0pt}%
\pgfsys@defobject{currentmarker}{\pgfqpoint{0.000000in}{-0.027778in}}{\pgfqpoint{0.000000in}{0.000000in}}{%
\pgfpathmoveto{\pgfqpoint{0.000000in}{0.000000in}}%
\pgfpathlineto{\pgfqpoint{0.000000in}{-0.027778in}}%
\pgfusepath{stroke,fill}%
}%
\begin{pgfscope}%
\pgfsys@transformshift{1.065679in}{0.417642in}%
\pgfsys@useobject{currentmarker}{}%
\end{pgfscope}%
\end{pgfscope}%
\begin{pgfscope}%
\pgfpathrectangle{\pgfqpoint{0.770608in}{0.417642in}}{\pgfqpoint{3.162547in}{2.024187in}}%
\pgfusepath{clip}%
\pgfsetrectcap%
\pgfsetroundjoin%
\pgfsetlinewidth{0.803000pt}%
\definecolor{currentstroke}{rgb}{0.850000,0.850000,0.850000}%
\pgfsetstrokecolor{currentstroke}%
\pgfsetdash{}{0pt}%
\pgfpathmoveto{\pgfqpoint{1.154194in}{0.417642in}}%
\pgfpathlineto{\pgfqpoint{1.154194in}{2.441829in}}%
\pgfusepath{stroke}%
\end{pgfscope}%
\begin{pgfscope}%
\pgfsetbuttcap%
\pgfsetroundjoin%
\definecolor{currentfill}{rgb}{0.000000,0.000000,0.000000}%
\pgfsetfillcolor{currentfill}%
\pgfsetlinewidth{0.602250pt}%
\definecolor{currentstroke}{rgb}{0.000000,0.000000,0.000000}%
\pgfsetstrokecolor{currentstroke}%
\pgfsetdash{}{0pt}%
\pgfsys@defobject{currentmarker}{\pgfqpoint{0.000000in}{-0.027778in}}{\pgfqpoint{0.000000in}{0.000000in}}{%
\pgfpathmoveto{\pgfqpoint{0.000000in}{0.000000in}}%
\pgfpathlineto{\pgfqpoint{0.000000in}{-0.027778in}}%
\pgfusepath{stroke,fill}%
}%
\begin{pgfscope}%
\pgfsys@transformshift{1.154194in}{0.417642in}%
\pgfsys@useobject{currentmarker}{}%
\end{pgfscope}%
\end{pgfscope}%
\begin{pgfscope}%
\pgfpathrectangle{\pgfqpoint{0.770608in}{0.417642in}}{\pgfqpoint{3.162547in}{2.024187in}}%
\pgfusepath{clip}%
\pgfsetrectcap%
\pgfsetroundjoin%
\pgfsetlinewidth{0.803000pt}%
\definecolor{currentstroke}{rgb}{0.850000,0.850000,0.850000}%
\pgfsetstrokecolor{currentstroke}%
\pgfsetdash{}{0pt}%
\pgfpathmoveto{\pgfqpoint{1.216997in}{0.417642in}}%
\pgfpathlineto{\pgfqpoint{1.216997in}{2.441829in}}%
\pgfusepath{stroke}%
\end{pgfscope}%
\begin{pgfscope}%
\pgfsetbuttcap%
\pgfsetroundjoin%
\definecolor{currentfill}{rgb}{0.000000,0.000000,0.000000}%
\pgfsetfillcolor{currentfill}%
\pgfsetlinewidth{0.602250pt}%
\definecolor{currentstroke}{rgb}{0.000000,0.000000,0.000000}%
\pgfsetstrokecolor{currentstroke}%
\pgfsetdash{}{0pt}%
\pgfsys@defobject{currentmarker}{\pgfqpoint{0.000000in}{-0.027778in}}{\pgfqpoint{0.000000in}{0.000000in}}{%
\pgfpathmoveto{\pgfqpoint{0.000000in}{0.000000in}}%
\pgfpathlineto{\pgfqpoint{0.000000in}{-0.027778in}}%
\pgfusepath{stroke,fill}%
}%
\begin{pgfscope}%
\pgfsys@transformshift{1.216997in}{0.417642in}%
\pgfsys@useobject{currentmarker}{}%
\end{pgfscope}%
\end{pgfscope}%
\begin{pgfscope}%
\pgfpathrectangle{\pgfqpoint{0.770608in}{0.417642in}}{\pgfqpoint{3.162547in}{2.024187in}}%
\pgfusepath{clip}%
\pgfsetrectcap%
\pgfsetroundjoin%
\pgfsetlinewidth{0.803000pt}%
\definecolor{currentstroke}{rgb}{0.850000,0.850000,0.850000}%
\pgfsetstrokecolor{currentstroke}%
\pgfsetdash{}{0pt}%
\pgfpathmoveto{\pgfqpoint{1.265710in}{0.417642in}}%
\pgfpathlineto{\pgfqpoint{1.265710in}{2.441829in}}%
\pgfusepath{stroke}%
\end{pgfscope}%
\begin{pgfscope}%
\pgfsetbuttcap%
\pgfsetroundjoin%
\definecolor{currentfill}{rgb}{0.000000,0.000000,0.000000}%
\pgfsetfillcolor{currentfill}%
\pgfsetlinewidth{0.602250pt}%
\definecolor{currentstroke}{rgb}{0.000000,0.000000,0.000000}%
\pgfsetstrokecolor{currentstroke}%
\pgfsetdash{}{0pt}%
\pgfsys@defobject{currentmarker}{\pgfqpoint{0.000000in}{-0.027778in}}{\pgfqpoint{0.000000in}{0.000000in}}{%
\pgfpathmoveto{\pgfqpoint{0.000000in}{0.000000in}}%
\pgfpathlineto{\pgfqpoint{0.000000in}{-0.027778in}}%
\pgfusepath{stroke,fill}%
}%
\begin{pgfscope}%
\pgfsys@transformshift{1.265710in}{0.417642in}%
\pgfsys@useobject{currentmarker}{}%
\end{pgfscope}%
\end{pgfscope}%
\begin{pgfscope}%
\pgfpathrectangle{\pgfqpoint{0.770608in}{0.417642in}}{\pgfqpoint{3.162547in}{2.024187in}}%
\pgfusepath{clip}%
\pgfsetrectcap%
\pgfsetroundjoin%
\pgfsetlinewidth{0.803000pt}%
\definecolor{currentstroke}{rgb}{0.850000,0.850000,0.850000}%
\pgfsetstrokecolor{currentstroke}%
\pgfsetdash{}{0pt}%
\pgfpathmoveto{\pgfqpoint{1.305512in}{0.417642in}}%
\pgfpathlineto{\pgfqpoint{1.305512in}{2.441829in}}%
\pgfusepath{stroke}%
\end{pgfscope}%
\begin{pgfscope}%
\pgfsetbuttcap%
\pgfsetroundjoin%
\definecolor{currentfill}{rgb}{0.000000,0.000000,0.000000}%
\pgfsetfillcolor{currentfill}%
\pgfsetlinewidth{0.602250pt}%
\definecolor{currentstroke}{rgb}{0.000000,0.000000,0.000000}%
\pgfsetstrokecolor{currentstroke}%
\pgfsetdash{}{0pt}%
\pgfsys@defobject{currentmarker}{\pgfqpoint{0.000000in}{-0.027778in}}{\pgfqpoint{0.000000in}{0.000000in}}{%
\pgfpathmoveto{\pgfqpoint{0.000000in}{0.000000in}}%
\pgfpathlineto{\pgfqpoint{0.000000in}{-0.027778in}}%
\pgfusepath{stroke,fill}%
}%
\begin{pgfscope}%
\pgfsys@transformshift{1.305512in}{0.417642in}%
\pgfsys@useobject{currentmarker}{}%
\end{pgfscope}%
\end{pgfscope}%
\begin{pgfscope}%
\pgfpathrectangle{\pgfqpoint{0.770608in}{0.417642in}}{\pgfqpoint{3.162547in}{2.024187in}}%
\pgfusepath{clip}%
\pgfsetrectcap%
\pgfsetroundjoin%
\pgfsetlinewidth{0.803000pt}%
\definecolor{currentstroke}{rgb}{0.850000,0.850000,0.850000}%
\pgfsetstrokecolor{currentstroke}%
\pgfsetdash{}{0pt}%
\pgfpathmoveto{\pgfqpoint{1.339164in}{0.417642in}}%
\pgfpathlineto{\pgfqpoint{1.339164in}{2.441829in}}%
\pgfusepath{stroke}%
\end{pgfscope}%
\begin{pgfscope}%
\pgfsetbuttcap%
\pgfsetroundjoin%
\definecolor{currentfill}{rgb}{0.000000,0.000000,0.000000}%
\pgfsetfillcolor{currentfill}%
\pgfsetlinewidth{0.602250pt}%
\definecolor{currentstroke}{rgb}{0.000000,0.000000,0.000000}%
\pgfsetstrokecolor{currentstroke}%
\pgfsetdash{}{0pt}%
\pgfsys@defobject{currentmarker}{\pgfqpoint{0.000000in}{-0.027778in}}{\pgfqpoint{0.000000in}{0.000000in}}{%
\pgfpathmoveto{\pgfqpoint{0.000000in}{0.000000in}}%
\pgfpathlineto{\pgfqpoint{0.000000in}{-0.027778in}}%
\pgfusepath{stroke,fill}%
}%
\begin{pgfscope}%
\pgfsys@transformshift{1.339164in}{0.417642in}%
\pgfsys@useobject{currentmarker}{}%
\end{pgfscope}%
\end{pgfscope}%
\begin{pgfscope}%
\pgfpathrectangle{\pgfqpoint{0.770608in}{0.417642in}}{\pgfqpoint{3.162547in}{2.024187in}}%
\pgfusepath{clip}%
\pgfsetrectcap%
\pgfsetroundjoin%
\pgfsetlinewidth{0.803000pt}%
\definecolor{currentstroke}{rgb}{0.850000,0.850000,0.850000}%
\pgfsetstrokecolor{currentstroke}%
\pgfsetdash{}{0pt}%
\pgfpathmoveto{\pgfqpoint{1.368315in}{0.417642in}}%
\pgfpathlineto{\pgfqpoint{1.368315in}{2.441829in}}%
\pgfusepath{stroke}%
\end{pgfscope}%
\begin{pgfscope}%
\pgfsetbuttcap%
\pgfsetroundjoin%
\definecolor{currentfill}{rgb}{0.000000,0.000000,0.000000}%
\pgfsetfillcolor{currentfill}%
\pgfsetlinewidth{0.602250pt}%
\definecolor{currentstroke}{rgb}{0.000000,0.000000,0.000000}%
\pgfsetstrokecolor{currentstroke}%
\pgfsetdash{}{0pt}%
\pgfsys@defobject{currentmarker}{\pgfqpoint{0.000000in}{-0.027778in}}{\pgfqpoint{0.000000in}{0.000000in}}{%
\pgfpathmoveto{\pgfqpoint{0.000000in}{0.000000in}}%
\pgfpathlineto{\pgfqpoint{0.000000in}{-0.027778in}}%
\pgfusepath{stroke,fill}%
}%
\begin{pgfscope}%
\pgfsys@transformshift{1.368315in}{0.417642in}%
\pgfsys@useobject{currentmarker}{}%
\end{pgfscope}%
\end{pgfscope}%
\begin{pgfscope}%
\pgfpathrectangle{\pgfqpoint{0.770608in}{0.417642in}}{\pgfqpoint{3.162547in}{2.024187in}}%
\pgfusepath{clip}%
\pgfsetrectcap%
\pgfsetroundjoin%
\pgfsetlinewidth{0.803000pt}%
\definecolor{currentstroke}{rgb}{0.850000,0.850000,0.850000}%
\pgfsetstrokecolor{currentstroke}%
\pgfsetdash{}{0pt}%
\pgfpathmoveto{\pgfqpoint{1.394027in}{0.417642in}}%
\pgfpathlineto{\pgfqpoint{1.394027in}{2.441829in}}%
\pgfusepath{stroke}%
\end{pgfscope}%
\begin{pgfscope}%
\pgfsetbuttcap%
\pgfsetroundjoin%
\definecolor{currentfill}{rgb}{0.000000,0.000000,0.000000}%
\pgfsetfillcolor{currentfill}%
\pgfsetlinewidth{0.602250pt}%
\definecolor{currentstroke}{rgb}{0.000000,0.000000,0.000000}%
\pgfsetstrokecolor{currentstroke}%
\pgfsetdash{}{0pt}%
\pgfsys@defobject{currentmarker}{\pgfqpoint{0.000000in}{-0.027778in}}{\pgfqpoint{0.000000in}{0.000000in}}{%
\pgfpathmoveto{\pgfqpoint{0.000000in}{0.000000in}}%
\pgfpathlineto{\pgfqpoint{0.000000in}{-0.027778in}}%
\pgfusepath{stroke,fill}%
}%
\begin{pgfscope}%
\pgfsys@transformshift{1.394027in}{0.417642in}%
\pgfsys@useobject{currentmarker}{}%
\end{pgfscope}%
\end{pgfscope}%
\begin{pgfscope}%
\pgfpathrectangle{\pgfqpoint{0.770608in}{0.417642in}}{\pgfqpoint{3.162547in}{2.024187in}}%
\pgfusepath{clip}%
\pgfsetrectcap%
\pgfsetroundjoin%
\pgfsetlinewidth{0.803000pt}%
\definecolor{currentstroke}{rgb}{0.850000,0.850000,0.850000}%
\pgfsetstrokecolor{currentstroke}%
\pgfsetdash{}{0pt}%
\pgfpathmoveto{\pgfqpoint{1.568346in}{0.417642in}}%
\pgfpathlineto{\pgfqpoint{1.568346in}{2.441829in}}%
\pgfusepath{stroke}%
\end{pgfscope}%
\begin{pgfscope}%
\pgfsetbuttcap%
\pgfsetroundjoin%
\definecolor{currentfill}{rgb}{0.000000,0.000000,0.000000}%
\pgfsetfillcolor{currentfill}%
\pgfsetlinewidth{0.602250pt}%
\definecolor{currentstroke}{rgb}{0.000000,0.000000,0.000000}%
\pgfsetstrokecolor{currentstroke}%
\pgfsetdash{}{0pt}%
\pgfsys@defobject{currentmarker}{\pgfqpoint{0.000000in}{-0.027778in}}{\pgfqpoint{0.000000in}{0.000000in}}{%
\pgfpathmoveto{\pgfqpoint{0.000000in}{0.000000in}}%
\pgfpathlineto{\pgfqpoint{0.000000in}{-0.027778in}}%
\pgfusepath{stroke,fill}%
}%
\begin{pgfscope}%
\pgfsys@transformshift{1.568346in}{0.417642in}%
\pgfsys@useobject{currentmarker}{}%
\end{pgfscope}%
\end{pgfscope}%
\begin{pgfscope}%
\pgfpathrectangle{\pgfqpoint{0.770608in}{0.417642in}}{\pgfqpoint{3.162547in}{2.024187in}}%
\pgfusepath{clip}%
\pgfsetrectcap%
\pgfsetroundjoin%
\pgfsetlinewidth{0.803000pt}%
\definecolor{currentstroke}{rgb}{0.850000,0.850000,0.850000}%
\pgfsetstrokecolor{currentstroke}%
\pgfsetdash{}{0pt}%
\pgfpathmoveto{\pgfqpoint{1.656862in}{0.417642in}}%
\pgfpathlineto{\pgfqpoint{1.656862in}{2.441829in}}%
\pgfusepath{stroke}%
\end{pgfscope}%
\begin{pgfscope}%
\pgfsetbuttcap%
\pgfsetroundjoin%
\definecolor{currentfill}{rgb}{0.000000,0.000000,0.000000}%
\pgfsetfillcolor{currentfill}%
\pgfsetlinewidth{0.602250pt}%
\definecolor{currentstroke}{rgb}{0.000000,0.000000,0.000000}%
\pgfsetstrokecolor{currentstroke}%
\pgfsetdash{}{0pt}%
\pgfsys@defobject{currentmarker}{\pgfqpoint{0.000000in}{-0.027778in}}{\pgfqpoint{0.000000in}{0.000000in}}{%
\pgfpathmoveto{\pgfqpoint{0.000000in}{0.000000in}}%
\pgfpathlineto{\pgfqpoint{0.000000in}{-0.027778in}}%
\pgfusepath{stroke,fill}%
}%
\begin{pgfscope}%
\pgfsys@transformshift{1.656862in}{0.417642in}%
\pgfsys@useobject{currentmarker}{}%
\end{pgfscope}%
\end{pgfscope}%
\begin{pgfscope}%
\pgfpathrectangle{\pgfqpoint{0.770608in}{0.417642in}}{\pgfqpoint{3.162547in}{2.024187in}}%
\pgfusepath{clip}%
\pgfsetrectcap%
\pgfsetroundjoin%
\pgfsetlinewidth{0.803000pt}%
\definecolor{currentstroke}{rgb}{0.850000,0.850000,0.850000}%
\pgfsetstrokecolor{currentstroke}%
\pgfsetdash{}{0pt}%
\pgfpathmoveto{\pgfqpoint{1.719664in}{0.417642in}}%
\pgfpathlineto{\pgfqpoint{1.719664in}{2.441829in}}%
\pgfusepath{stroke}%
\end{pgfscope}%
\begin{pgfscope}%
\pgfsetbuttcap%
\pgfsetroundjoin%
\definecolor{currentfill}{rgb}{0.000000,0.000000,0.000000}%
\pgfsetfillcolor{currentfill}%
\pgfsetlinewidth{0.602250pt}%
\definecolor{currentstroke}{rgb}{0.000000,0.000000,0.000000}%
\pgfsetstrokecolor{currentstroke}%
\pgfsetdash{}{0pt}%
\pgfsys@defobject{currentmarker}{\pgfqpoint{0.000000in}{-0.027778in}}{\pgfqpoint{0.000000in}{0.000000in}}{%
\pgfpathmoveto{\pgfqpoint{0.000000in}{0.000000in}}%
\pgfpathlineto{\pgfqpoint{0.000000in}{-0.027778in}}%
\pgfusepath{stroke,fill}%
}%
\begin{pgfscope}%
\pgfsys@transformshift{1.719664in}{0.417642in}%
\pgfsys@useobject{currentmarker}{}%
\end{pgfscope}%
\end{pgfscope}%
\begin{pgfscope}%
\pgfpathrectangle{\pgfqpoint{0.770608in}{0.417642in}}{\pgfqpoint{3.162547in}{2.024187in}}%
\pgfusepath{clip}%
\pgfsetrectcap%
\pgfsetroundjoin%
\pgfsetlinewidth{0.803000pt}%
\definecolor{currentstroke}{rgb}{0.850000,0.850000,0.850000}%
\pgfsetstrokecolor{currentstroke}%
\pgfsetdash{}{0pt}%
\pgfpathmoveto{\pgfqpoint{1.768378in}{0.417642in}}%
\pgfpathlineto{\pgfqpoint{1.768378in}{2.441829in}}%
\pgfusepath{stroke}%
\end{pgfscope}%
\begin{pgfscope}%
\pgfsetbuttcap%
\pgfsetroundjoin%
\definecolor{currentfill}{rgb}{0.000000,0.000000,0.000000}%
\pgfsetfillcolor{currentfill}%
\pgfsetlinewidth{0.602250pt}%
\definecolor{currentstroke}{rgb}{0.000000,0.000000,0.000000}%
\pgfsetstrokecolor{currentstroke}%
\pgfsetdash{}{0pt}%
\pgfsys@defobject{currentmarker}{\pgfqpoint{0.000000in}{-0.027778in}}{\pgfqpoint{0.000000in}{0.000000in}}{%
\pgfpathmoveto{\pgfqpoint{0.000000in}{0.000000in}}%
\pgfpathlineto{\pgfqpoint{0.000000in}{-0.027778in}}%
\pgfusepath{stroke,fill}%
}%
\begin{pgfscope}%
\pgfsys@transformshift{1.768378in}{0.417642in}%
\pgfsys@useobject{currentmarker}{}%
\end{pgfscope}%
\end{pgfscope}%
\begin{pgfscope}%
\pgfpathrectangle{\pgfqpoint{0.770608in}{0.417642in}}{\pgfqpoint{3.162547in}{2.024187in}}%
\pgfusepath{clip}%
\pgfsetrectcap%
\pgfsetroundjoin%
\pgfsetlinewidth{0.803000pt}%
\definecolor{currentstroke}{rgb}{0.850000,0.850000,0.850000}%
\pgfsetstrokecolor{currentstroke}%
\pgfsetdash{}{0pt}%
\pgfpathmoveto{\pgfqpoint{1.808180in}{0.417642in}}%
\pgfpathlineto{\pgfqpoint{1.808180in}{2.441829in}}%
\pgfusepath{stroke}%
\end{pgfscope}%
\begin{pgfscope}%
\pgfsetbuttcap%
\pgfsetroundjoin%
\definecolor{currentfill}{rgb}{0.000000,0.000000,0.000000}%
\pgfsetfillcolor{currentfill}%
\pgfsetlinewidth{0.602250pt}%
\definecolor{currentstroke}{rgb}{0.000000,0.000000,0.000000}%
\pgfsetstrokecolor{currentstroke}%
\pgfsetdash{}{0pt}%
\pgfsys@defobject{currentmarker}{\pgfqpoint{0.000000in}{-0.027778in}}{\pgfqpoint{0.000000in}{0.000000in}}{%
\pgfpathmoveto{\pgfqpoint{0.000000in}{0.000000in}}%
\pgfpathlineto{\pgfqpoint{0.000000in}{-0.027778in}}%
\pgfusepath{stroke,fill}%
}%
\begin{pgfscope}%
\pgfsys@transformshift{1.808180in}{0.417642in}%
\pgfsys@useobject{currentmarker}{}%
\end{pgfscope}%
\end{pgfscope}%
\begin{pgfscope}%
\pgfpathrectangle{\pgfqpoint{0.770608in}{0.417642in}}{\pgfqpoint{3.162547in}{2.024187in}}%
\pgfusepath{clip}%
\pgfsetrectcap%
\pgfsetroundjoin%
\pgfsetlinewidth{0.803000pt}%
\definecolor{currentstroke}{rgb}{0.850000,0.850000,0.850000}%
\pgfsetstrokecolor{currentstroke}%
\pgfsetdash{}{0pt}%
\pgfpathmoveto{\pgfqpoint{1.841832in}{0.417642in}}%
\pgfpathlineto{\pgfqpoint{1.841832in}{2.441829in}}%
\pgfusepath{stroke}%
\end{pgfscope}%
\begin{pgfscope}%
\pgfsetbuttcap%
\pgfsetroundjoin%
\definecolor{currentfill}{rgb}{0.000000,0.000000,0.000000}%
\pgfsetfillcolor{currentfill}%
\pgfsetlinewidth{0.602250pt}%
\definecolor{currentstroke}{rgb}{0.000000,0.000000,0.000000}%
\pgfsetstrokecolor{currentstroke}%
\pgfsetdash{}{0pt}%
\pgfsys@defobject{currentmarker}{\pgfqpoint{0.000000in}{-0.027778in}}{\pgfqpoint{0.000000in}{0.000000in}}{%
\pgfpathmoveto{\pgfqpoint{0.000000in}{0.000000in}}%
\pgfpathlineto{\pgfqpoint{0.000000in}{-0.027778in}}%
\pgfusepath{stroke,fill}%
}%
\begin{pgfscope}%
\pgfsys@transformshift{1.841832in}{0.417642in}%
\pgfsys@useobject{currentmarker}{}%
\end{pgfscope}%
\end{pgfscope}%
\begin{pgfscope}%
\pgfpathrectangle{\pgfqpoint{0.770608in}{0.417642in}}{\pgfqpoint{3.162547in}{2.024187in}}%
\pgfusepath{clip}%
\pgfsetrectcap%
\pgfsetroundjoin%
\pgfsetlinewidth{0.803000pt}%
\definecolor{currentstroke}{rgb}{0.850000,0.850000,0.850000}%
\pgfsetstrokecolor{currentstroke}%
\pgfsetdash{}{0pt}%
\pgfpathmoveto{\pgfqpoint{1.870982in}{0.417642in}}%
\pgfpathlineto{\pgfqpoint{1.870982in}{2.441829in}}%
\pgfusepath{stroke}%
\end{pgfscope}%
\begin{pgfscope}%
\pgfsetbuttcap%
\pgfsetroundjoin%
\definecolor{currentfill}{rgb}{0.000000,0.000000,0.000000}%
\pgfsetfillcolor{currentfill}%
\pgfsetlinewidth{0.602250pt}%
\definecolor{currentstroke}{rgb}{0.000000,0.000000,0.000000}%
\pgfsetstrokecolor{currentstroke}%
\pgfsetdash{}{0pt}%
\pgfsys@defobject{currentmarker}{\pgfqpoint{0.000000in}{-0.027778in}}{\pgfqpoint{0.000000in}{0.000000in}}{%
\pgfpathmoveto{\pgfqpoint{0.000000in}{0.000000in}}%
\pgfpathlineto{\pgfqpoint{0.000000in}{-0.027778in}}%
\pgfusepath{stroke,fill}%
}%
\begin{pgfscope}%
\pgfsys@transformshift{1.870982in}{0.417642in}%
\pgfsys@useobject{currentmarker}{}%
\end{pgfscope}%
\end{pgfscope}%
\begin{pgfscope}%
\pgfpathrectangle{\pgfqpoint{0.770608in}{0.417642in}}{\pgfqpoint{3.162547in}{2.024187in}}%
\pgfusepath{clip}%
\pgfsetrectcap%
\pgfsetroundjoin%
\pgfsetlinewidth{0.803000pt}%
\definecolor{currentstroke}{rgb}{0.850000,0.850000,0.850000}%
\pgfsetstrokecolor{currentstroke}%
\pgfsetdash{}{0pt}%
\pgfpathmoveto{\pgfqpoint{1.896695in}{0.417642in}}%
\pgfpathlineto{\pgfqpoint{1.896695in}{2.441829in}}%
\pgfusepath{stroke}%
\end{pgfscope}%
\begin{pgfscope}%
\pgfsetbuttcap%
\pgfsetroundjoin%
\definecolor{currentfill}{rgb}{0.000000,0.000000,0.000000}%
\pgfsetfillcolor{currentfill}%
\pgfsetlinewidth{0.602250pt}%
\definecolor{currentstroke}{rgb}{0.000000,0.000000,0.000000}%
\pgfsetstrokecolor{currentstroke}%
\pgfsetdash{}{0pt}%
\pgfsys@defobject{currentmarker}{\pgfqpoint{0.000000in}{-0.027778in}}{\pgfqpoint{0.000000in}{0.000000in}}{%
\pgfpathmoveto{\pgfqpoint{0.000000in}{0.000000in}}%
\pgfpathlineto{\pgfqpoint{0.000000in}{-0.027778in}}%
\pgfusepath{stroke,fill}%
}%
\begin{pgfscope}%
\pgfsys@transformshift{1.896695in}{0.417642in}%
\pgfsys@useobject{currentmarker}{}%
\end{pgfscope}%
\end{pgfscope}%
\begin{pgfscope}%
\pgfpathrectangle{\pgfqpoint{0.770608in}{0.417642in}}{\pgfqpoint{3.162547in}{2.024187in}}%
\pgfusepath{clip}%
\pgfsetrectcap%
\pgfsetroundjoin%
\pgfsetlinewidth{0.803000pt}%
\definecolor{currentstroke}{rgb}{0.850000,0.850000,0.850000}%
\pgfsetstrokecolor{currentstroke}%
\pgfsetdash{}{0pt}%
\pgfpathmoveto{\pgfqpoint{2.071014in}{0.417642in}}%
\pgfpathlineto{\pgfqpoint{2.071014in}{2.441829in}}%
\pgfusepath{stroke}%
\end{pgfscope}%
\begin{pgfscope}%
\pgfsetbuttcap%
\pgfsetroundjoin%
\definecolor{currentfill}{rgb}{0.000000,0.000000,0.000000}%
\pgfsetfillcolor{currentfill}%
\pgfsetlinewidth{0.602250pt}%
\definecolor{currentstroke}{rgb}{0.000000,0.000000,0.000000}%
\pgfsetstrokecolor{currentstroke}%
\pgfsetdash{}{0pt}%
\pgfsys@defobject{currentmarker}{\pgfqpoint{0.000000in}{-0.027778in}}{\pgfqpoint{0.000000in}{0.000000in}}{%
\pgfpathmoveto{\pgfqpoint{0.000000in}{0.000000in}}%
\pgfpathlineto{\pgfqpoint{0.000000in}{-0.027778in}}%
\pgfusepath{stroke,fill}%
}%
\begin{pgfscope}%
\pgfsys@transformshift{2.071014in}{0.417642in}%
\pgfsys@useobject{currentmarker}{}%
\end{pgfscope}%
\end{pgfscope}%
\begin{pgfscope}%
\pgfpathrectangle{\pgfqpoint{0.770608in}{0.417642in}}{\pgfqpoint{3.162547in}{2.024187in}}%
\pgfusepath{clip}%
\pgfsetrectcap%
\pgfsetroundjoin%
\pgfsetlinewidth{0.803000pt}%
\definecolor{currentstroke}{rgb}{0.850000,0.850000,0.850000}%
\pgfsetstrokecolor{currentstroke}%
\pgfsetdash{}{0pt}%
\pgfpathmoveto{\pgfqpoint{2.159529in}{0.417642in}}%
\pgfpathlineto{\pgfqpoint{2.159529in}{2.441829in}}%
\pgfusepath{stroke}%
\end{pgfscope}%
\begin{pgfscope}%
\pgfsetbuttcap%
\pgfsetroundjoin%
\definecolor{currentfill}{rgb}{0.000000,0.000000,0.000000}%
\pgfsetfillcolor{currentfill}%
\pgfsetlinewidth{0.602250pt}%
\definecolor{currentstroke}{rgb}{0.000000,0.000000,0.000000}%
\pgfsetstrokecolor{currentstroke}%
\pgfsetdash{}{0pt}%
\pgfsys@defobject{currentmarker}{\pgfqpoint{0.000000in}{-0.027778in}}{\pgfqpoint{0.000000in}{0.000000in}}{%
\pgfpathmoveto{\pgfqpoint{0.000000in}{0.000000in}}%
\pgfpathlineto{\pgfqpoint{0.000000in}{-0.027778in}}%
\pgfusepath{stroke,fill}%
}%
\begin{pgfscope}%
\pgfsys@transformshift{2.159529in}{0.417642in}%
\pgfsys@useobject{currentmarker}{}%
\end{pgfscope}%
\end{pgfscope}%
\begin{pgfscope}%
\pgfpathrectangle{\pgfqpoint{0.770608in}{0.417642in}}{\pgfqpoint{3.162547in}{2.024187in}}%
\pgfusepath{clip}%
\pgfsetrectcap%
\pgfsetroundjoin%
\pgfsetlinewidth{0.803000pt}%
\definecolor{currentstroke}{rgb}{0.850000,0.850000,0.850000}%
\pgfsetstrokecolor{currentstroke}%
\pgfsetdash{}{0pt}%
\pgfpathmoveto{\pgfqpoint{2.222332in}{0.417642in}}%
\pgfpathlineto{\pgfqpoint{2.222332in}{2.441829in}}%
\pgfusepath{stroke}%
\end{pgfscope}%
\begin{pgfscope}%
\pgfsetbuttcap%
\pgfsetroundjoin%
\definecolor{currentfill}{rgb}{0.000000,0.000000,0.000000}%
\pgfsetfillcolor{currentfill}%
\pgfsetlinewidth{0.602250pt}%
\definecolor{currentstroke}{rgb}{0.000000,0.000000,0.000000}%
\pgfsetstrokecolor{currentstroke}%
\pgfsetdash{}{0pt}%
\pgfsys@defobject{currentmarker}{\pgfqpoint{0.000000in}{-0.027778in}}{\pgfqpoint{0.000000in}{0.000000in}}{%
\pgfpathmoveto{\pgfqpoint{0.000000in}{0.000000in}}%
\pgfpathlineto{\pgfqpoint{0.000000in}{-0.027778in}}%
\pgfusepath{stroke,fill}%
}%
\begin{pgfscope}%
\pgfsys@transformshift{2.222332in}{0.417642in}%
\pgfsys@useobject{currentmarker}{}%
\end{pgfscope}%
\end{pgfscope}%
\begin{pgfscope}%
\pgfpathrectangle{\pgfqpoint{0.770608in}{0.417642in}}{\pgfqpoint{3.162547in}{2.024187in}}%
\pgfusepath{clip}%
\pgfsetrectcap%
\pgfsetroundjoin%
\pgfsetlinewidth{0.803000pt}%
\definecolor{currentstroke}{rgb}{0.850000,0.850000,0.850000}%
\pgfsetstrokecolor{currentstroke}%
\pgfsetdash{}{0pt}%
\pgfpathmoveto{\pgfqpoint{2.271045in}{0.417642in}}%
\pgfpathlineto{\pgfqpoint{2.271045in}{2.441829in}}%
\pgfusepath{stroke}%
\end{pgfscope}%
\begin{pgfscope}%
\pgfsetbuttcap%
\pgfsetroundjoin%
\definecolor{currentfill}{rgb}{0.000000,0.000000,0.000000}%
\pgfsetfillcolor{currentfill}%
\pgfsetlinewidth{0.602250pt}%
\definecolor{currentstroke}{rgb}{0.000000,0.000000,0.000000}%
\pgfsetstrokecolor{currentstroke}%
\pgfsetdash{}{0pt}%
\pgfsys@defobject{currentmarker}{\pgfqpoint{0.000000in}{-0.027778in}}{\pgfqpoint{0.000000in}{0.000000in}}{%
\pgfpathmoveto{\pgfqpoint{0.000000in}{0.000000in}}%
\pgfpathlineto{\pgfqpoint{0.000000in}{-0.027778in}}%
\pgfusepath{stroke,fill}%
}%
\begin{pgfscope}%
\pgfsys@transformshift{2.271045in}{0.417642in}%
\pgfsys@useobject{currentmarker}{}%
\end{pgfscope}%
\end{pgfscope}%
\begin{pgfscope}%
\pgfpathrectangle{\pgfqpoint{0.770608in}{0.417642in}}{\pgfqpoint{3.162547in}{2.024187in}}%
\pgfusepath{clip}%
\pgfsetrectcap%
\pgfsetroundjoin%
\pgfsetlinewidth{0.803000pt}%
\definecolor{currentstroke}{rgb}{0.850000,0.850000,0.850000}%
\pgfsetstrokecolor{currentstroke}%
\pgfsetdash{}{0pt}%
\pgfpathmoveto{\pgfqpoint{2.310847in}{0.417642in}}%
\pgfpathlineto{\pgfqpoint{2.310847in}{2.441829in}}%
\pgfusepath{stroke}%
\end{pgfscope}%
\begin{pgfscope}%
\pgfsetbuttcap%
\pgfsetroundjoin%
\definecolor{currentfill}{rgb}{0.000000,0.000000,0.000000}%
\pgfsetfillcolor{currentfill}%
\pgfsetlinewidth{0.602250pt}%
\definecolor{currentstroke}{rgb}{0.000000,0.000000,0.000000}%
\pgfsetstrokecolor{currentstroke}%
\pgfsetdash{}{0pt}%
\pgfsys@defobject{currentmarker}{\pgfqpoint{0.000000in}{-0.027778in}}{\pgfqpoint{0.000000in}{0.000000in}}{%
\pgfpathmoveto{\pgfqpoint{0.000000in}{0.000000in}}%
\pgfpathlineto{\pgfqpoint{0.000000in}{-0.027778in}}%
\pgfusepath{stroke,fill}%
}%
\begin{pgfscope}%
\pgfsys@transformshift{2.310847in}{0.417642in}%
\pgfsys@useobject{currentmarker}{}%
\end{pgfscope}%
\end{pgfscope}%
\begin{pgfscope}%
\pgfpathrectangle{\pgfqpoint{0.770608in}{0.417642in}}{\pgfqpoint{3.162547in}{2.024187in}}%
\pgfusepath{clip}%
\pgfsetrectcap%
\pgfsetroundjoin%
\pgfsetlinewidth{0.803000pt}%
\definecolor{currentstroke}{rgb}{0.850000,0.850000,0.850000}%
\pgfsetstrokecolor{currentstroke}%
\pgfsetdash{}{0pt}%
\pgfpathmoveto{\pgfqpoint{2.344499in}{0.417642in}}%
\pgfpathlineto{\pgfqpoint{2.344499in}{2.441829in}}%
\pgfusepath{stroke}%
\end{pgfscope}%
\begin{pgfscope}%
\pgfsetbuttcap%
\pgfsetroundjoin%
\definecolor{currentfill}{rgb}{0.000000,0.000000,0.000000}%
\pgfsetfillcolor{currentfill}%
\pgfsetlinewidth{0.602250pt}%
\definecolor{currentstroke}{rgb}{0.000000,0.000000,0.000000}%
\pgfsetstrokecolor{currentstroke}%
\pgfsetdash{}{0pt}%
\pgfsys@defobject{currentmarker}{\pgfqpoint{0.000000in}{-0.027778in}}{\pgfqpoint{0.000000in}{0.000000in}}{%
\pgfpathmoveto{\pgfqpoint{0.000000in}{0.000000in}}%
\pgfpathlineto{\pgfqpoint{0.000000in}{-0.027778in}}%
\pgfusepath{stroke,fill}%
}%
\begin{pgfscope}%
\pgfsys@transformshift{2.344499in}{0.417642in}%
\pgfsys@useobject{currentmarker}{}%
\end{pgfscope}%
\end{pgfscope}%
\begin{pgfscope}%
\pgfpathrectangle{\pgfqpoint{0.770608in}{0.417642in}}{\pgfqpoint{3.162547in}{2.024187in}}%
\pgfusepath{clip}%
\pgfsetrectcap%
\pgfsetroundjoin%
\pgfsetlinewidth{0.803000pt}%
\definecolor{currentstroke}{rgb}{0.850000,0.850000,0.850000}%
\pgfsetstrokecolor{currentstroke}%
\pgfsetdash{}{0pt}%
\pgfpathmoveto{\pgfqpoint{2.373650in}{0.417642in}}%
\pgfpathlineto{\pgfqpoint{2.373650in}{2.441829in}}%
\pgfusepath{stroke}%
\end{pgfscope}%
\begin{pgfscope}%
\pgfsetbuttcap%
\pgfsetroundjoin%
\definecolor{currentfill}{rgb}{0.000000,0.000000,0.000000}%
\pgfsetfillcolor{currentfill}%
\pgfsetlinewidth{0.602250pt}%
\definecolor{currentstroke}{rgb}{0.000000,0.000000,0.000000}%
\pgfsetstrokecolor{currentstroke}%
\pgfsetdash{}{0pt}%
\pgfsys@defobject{currentmarker}{\pgfqpoint{0.000000in}{-0.027778in}}{\pgfqpoint{0.000000in}{0.000000in}}{%
\pgfpathmoveto{\pgfqpoint{0.000000in}{0.000000in}}%
\pgfpathlineto{\pgfqpoint{0.000000in}{-0.027778in}}%
\pgfusepath{stroke,fill}%
}%
\begin{pgfscope}%
\pgfsys@transformshift{2.373650in}{0.417642in}%
\pgfsys@useobject{currentmarker}{}%
\end{pgfscope}%
\end{pgfscope}%
\begin{pgfscope}%
\pgfpathrectangle{\pgfqpoint{0.770608in}{0.417642in}}{\pgfqpoint{3.162547in}{2.024187in}}%
\pgfusepath{clip}%
\pgfsetrectcap%
\pgfsetroundjoin%
\pgfsetlinewidth{0.803000pt}%
\definecolor{currentstroke}{rgb}{0.850000,0.850000,0.850000}%
\pgfsetstrokecolor{currentstroke}%
\pgfsetdash{}{0pt}%
\pgfpathmoveto{\pgfqpoint{2.399363in}{0.417642in}}%
\pgfpathlineto{\pgfqpoint{2.399363in}{2.441829in}}%
\pgfusepath{stroke}%
\end{pgfscope}%
\begin{pgfscope}%
\pgfsetbuttcap%
\pgfsetroundjoin%
\definecolor{currentfill}{rgb}{0.000000,0.000000,0.000000}%
\pgfsetfillcolor{currentfill}%
\pgfsetlinewidth{0.602250pt}%
\definecolor{currentstroke}{rgb}{0.000000,0.000000,0.000000}%
\pgfsetstrokecolor{currentstroke}%
\pgfsetdash{}{0pt}%
\pgfsys@defobject{currentmarker}{\pgfqpoint{0.000000in}{-0.027778in}}{\pgfqpoint{0.000000in}{0.000000in}}{%
\pgfpathmoveto{\pgfqpoint{0.000000in}{0.000000in}}%
\pgfpathlineto{\pgfqpoint{0.000000in}{-0.027778in}}%
\pgfusepath{stroke,fill}%
}%
\begin{pgfscope}%
\pgfsys@transformshift{2.399363in}{0.417642in}%
\pgfsys@useobject{currentmarker}{}%
\end{pgfscope}%
\end{pgfscope}%
\begin{pgfscope}%
\pgfpathrectangle{\pgfqpoint{0.770608in}{0.417642in}}{\pgfqpoint{3.162547in}{2.024187in}}%
\pgfusepath{clip}%
\pgfsetrectcap%
\pgfsetroundjoin%
\pgfsetlinewidth{0.803000pt}%
\definecolor{currentstroke}{rgb}{0.850000,0.850000,0.850000}%
\pgfsetstrokecolor{currentstroke}%
\pgfsetdash{}{0pt}%
\pgfpathmoveto{\pgfqpoint{2.573682in}{0.417642in}}%
\pgfpathlineto{\pgfqpoint{2.573682in}{2.441829in}}%
\pgfusepath{stroke}%
\end{pgfscope}%
\begin{pgfscope}%
\pgfsetbuttcap%
\pgfsetroundjoin%
\definecolor{currentfill}{rgb}{0.000000,0.000000,0.000000}%
\pgfsetfillcolor{currentfill}%
\pgfsetlinewidth{0.602250pt}%
\definecolor{currentstroke}{rgb}{0.000000,0.000000,0.000000}%
\pgfsetstrokecolor{currentstroke}%
\pgfsetdash{}{0pt}%
\pgfsys@defobject{currentmarker}{\pgfqpoint{0.000000in}{-0.027778in}}{\pgfqpoint{0.000000in}{0.000000in}}{%
\pgfpathmoveto{\pgfqpoint{0.000000in}{0.000000in}}%
\pgfpathlineto{\pgfqpoint{0.000000in}{-0.027778in}}%
\pgfusepath{stroke,fill}%
}%
\begin{pgfscope}%
\pgfsys@transformshift{2.573682in}{0.417642in}%
\pgfsys@useobject{currentmarker}{}%
\end{pgfscope}%
\end{pgfscope}%
\begin{pgfscope}%
\pgfpathrectangle{\pgfqpoint{0.770608in}{0.417642in}}{\pgfqpoint{3.162547in}{2.024187in}}%
\pgfusepath{clip}%
\pgfsetrectcap%
\pgfsetroundjoin%
\pgfsetlinewidth{0.803000pt}%
\definecolor{currentstroke}{rgb}{0.850000,0.850000,0.850000}%
\pgfsetstrokecolor{currentstroke}%
\pgfsetdash{}{0pt}%
\pgfpathmoveto{\pgfqpoint{2.662197in}{0.417642in}}%
\pgfpathlineto{\pgfqpoint{2.662197in}{2.441829in}}%
\pgfusepath{stroke}%
\end{pgfscope}%
\begin{pgfscope}%
\pgfsetbuttcap%
\pgfsetroundjoin%
\definecolor{currentfill}{rgb}{0.000000,0.000000,0.000000}%
\pgfsetfillcolor{currentfill}%
\pgfsetlinewidth{0.602250pt}%
\definecolor{currentstroke}{rgb}{0.000000,0.000000,0.000000}%
\pgfsetstrokecolor{currentstroke}%
\pgfsetdash{}{0pt}%
\pgfsys@defobject{currentmarker}{\pgfqpoint{0.000000in}{-0.027778in}}{\pgfqpoint{0.000000in}{0.000000in}}{%
\pgfpathmoveto{\pgfqpoint{0.000000in}{0.000000in}}%
\pgfpathlineto{\pgfqpoint{0.000000in}{-0.027778in}}%
\pgfusepath{stroke,fill}%
}%
\begin{pgfscope}%
\pgfsys@transformshift{2.662197in}{0.417642in}%
\pgfsys@useobject{currentmarker}{}%
\end{pgfscope}%
\end{pgfscope}%
\begin{pgfscope}%
\pgfpathrectangle{\pgfqpoint{0.770608in}{0.417642in}}{\pgfqpoint{3.162547in}{2.024187in}}%
\pgfusepath{clip}%
\pgfsetrectcap%
\pgfsetroundjoin%
\pgfsetlinewidth{0.803000pt}%
\definecolor{currentstroke}{rgb}{0.850000,0.850000,0.850000}%
\pgfsetstrokecolor{currentstroke}%
\pgfsetdash{}{0pt}%
\pgfpathmoveto{\pgfqpoint{2.725000in}{0.417642in}}%
\pgfpathlineto{\pgfqpoint{2.725000in}{2.441829in}}%
\pgfusepath{stroke}%
\end{pgfscope}%
\begin{pgfscope}%
\pgfsetbuttcap%
\pgfsetroundjoin%
\definecolor{currentfill}{rgb}{0.000000,0.000000,0.000000}%
\pgfsetfillcolor{currentfill}%
\pgfsetlinewidth{0.602250pt}%
\definecolor{currentstroke}{rgb}{0.000000,0.000000,0.000000}%
\pgfsetstrokecolor{currentstroke}%
\pgfsetdash{}{0pt}%
\pgfsys@defobject{currentmarker}{\pgfqpoint{0.000000in}{-0.027778in}}{\pgfqpoint{0.000000in}{0.000000in}}{%
\pgfpathmoveto{\pgfqpoint{0.000000in}{0.000000in}}%
\pgfpathlineto{\pgfqpoint{0.000000in}{-0.027778in}}%
\pgfusepath{stroke,fill}%
}%
\begin{pgfscope}%
\pgfsys@transformshift{2.725000in}{0.417642in}%
\pgfsys@useobject{currentmarker}{}%
\end{pgfscope}%
\end{pgfscope}%
\begin{pgfscope}%
\pgfpathrectangle{\pgfqpoint{0.770608in}{0.417642in}}{\pgfqpoint{3.162547in}{2.024187in}}%
\pgfusepath{clip}%
\pgfsetrectcap%
\pgfsetroundjoin%
\pgfsetlinewidth{0.803000pt}%
\definecolor{currentstroke}{rgb}{0.850000,0.850000,0.850000}%
\pgfsetstrokecolor{currentstroke}%
\pgfsetdash{}{0pt}%
\pgfpathmoveto{\pgfqpoint{2.773713in}{0.417642in}}%
\pgfpathlineto{\pgfqpoint{2.773713in}{2.441829in}}%
\pgfusepath{stroke}%
\end{pgfscope}%
\begin{pgfscope}%
\pgfsetbuttcap%
\pgfsetroundjoin%
\definecolor{currentfill}{rgb}{0.000000,0.000000,0.000000}%
\pgfsetfillcolor{currentfill}%
\pgfsetlinewidth{0.602250pt}%
\definecolor{currentstroke}{rgb}{0.000000,0.000000,0.000000}%
\pgfsetstrokecolor{currentstroke}%
\pgfsetdash{}{0pt}%
\pgfsys@defobject{currentmarker}{\pgfqpoint{0.000000in}{-0.027778in}}{\pgfqpoint{0.000000in}{0.000000in}}{%
\pgfpathmoveto{\pgfqpoint{0.000000in}{0.000000in}}%
\pgfpathlineto{\pgfqpoint{0.000000in}{-0.027778in}}%
\pgfusepath{stroke,fill}%
}%
\begin{pgfscope}%
\pgfsys@transformshift{2.773713in}{0.417642in}%
\pgfsys@useobject{currentmarker}{}%
\end{pgfscope}%
\end{pgfscope}%
\begin{pgfscope}%
\pgfpathrectangle{\pgfqpoint{0.770608in}{0.417642in}}{\pgfqpoint{3.162547in}{2.024187in}}%
\pgfusepath{clip}%
\pgfsetrectcap%
\pgfsetroundjoin%
\pgfsetlinewidth{0.803000pt}%
\definecolor{currentstroke}{rgb}{0.850000,0.850000,0.850000}%
\pgfsetstrokecolor{currentstroke}%
\pgfsetdash{}{0pt}%
\pgfpathmoveto{\pgfqpoint{2.813515in}{0.417642in}}%
\pgfpathlineto{\pgfqpoint{2.813515in}{2.441829in}}%
\pgfusepath{stroke}%
\end{pgfscope}%
\begin{pgfscope}%
\pgfsetbuttcap%
\pgfsetroundjoin%
\definecolor{currentfill}{rgb}{0.000000,0.000000,0.000000}%
\pgfsetfillcolor{currentfill}%
\pgfsetlinewidth{0.602250pt}%
\definecolor{currentstroke}{rgb}{0.000000,0.000000,0.000000}%
\pgfsetstrokecolor{currentstroke}%
\pgfsetdash{}{0pt}%
\pgfsys@defobject{currentmarker}{\pgfqpoint{0.000000in}{-0.027778in}}{\pgfqpoint{0.000000in}{0.000000in}}{%
\pgfpathmoveto{\pgfqpoint{0.000000in}{0.000000in}}%
\pgfpathlineto{\pgfqpoint{0.000000in}{-0.027778in}}%
\pgfusepath{stroke,fill}%
}%
\begin{pgfscope}%
\pgfsys@transformshift{2.813515in}{0.417642in}%
\pgfsys@useobject{currentmarker}{}%
\end{pgfscope}%
\end{pgfscope}%
\begin{pgfscope}%
\pgfpathrectangle{\pgfqpoint{0.770608in}{0.417642in}}{\pgfqpoint{3.162547in}{2.024187in}}%
\pgfusepath{clip}%
\pgfsetrectcap%
\pgfsetroundjoin%
\pgfsetlinewidth{0.803000pt}%
\definecolor{currentstroke}{rgb}{0.850000,0.850000,0.850000}%
\pgfsetstrokecolor{currentstroke}%
\pgfsetdash{}{0pt}%
\pgfpathmoveto{\pgfqpoint{2.847167in}{0.417642in}}%
\pgfpathlineto{\pgfqpoint{2.847167in}{2.441829in}}%
\pgfusepath{stroke}%
\end{pgfscope}%
\begin{pgfscope}%
\pgfsetbuttcap%
\pgfsetroundjoin%
\definecolor{currentfill}{rgb}{0.000000,0.000000,0.000000}%
\pgfsetfillcolor{currentfill}%
\pgfsetlinewidth{0.602250pt}%
\definecolor{currentstroke}{rgb}{0.000000,0.000000,0.000000}%
\pgfsetstrokecolor{currentstroke}%
\pgfsetdash{}{0pt}%
\pgfsys@defobject{currentmarker}{\pgfqpoint{0.000000in}{-0.027778in}}{\pgfqpoint{0.000000in}{0.000000in}}{%
\pgfpathmoveto{\pgfqpoint{0.000000in}{0.000000in}}%
\pgfpathlineto{\pgfqpoint{0.000000in}{-0.027778in}}%
\pgfusepath{stroke,fill}%
}%
\begin{pgfscope}%
\pgfsys@transformshift{2.847167in}{0.417642in}%
\pgfsys@useobject{currentmarker}{}%
\end{pgfscope}%
\end{pgfscope}%
\begin{pgfscope}%
\pgfpathrectangle{\pgfqpoint{0.770608in}{0.417642in}}{\pgfqpoint{3.162547in}{2.024187in}}%
\pgfusepath{clip}%
\pgfsetrectcap%
\pgfsetroundjoin%
\pgfsetlinewidth{0.803000pt}%
\definecolor{currentstroke}{rgb}{0.850000,0.850000,0.850000}%
\pgfsetstrokecolor{currentstroke}%
\pgfsetdash{}{0pt}%
\pgfpathmoveto{\pgfqpoint{2.876318in}{0.417642in}}%
\pgfpathlineto{\pgfqpoint{2.876318in}{2.441829in}}%
\pgfusepath{stroke}%
\end{pgfscope}%
\begin{pgfscope}%
\pgfsetbuttcap%
\pgfsetroundjoin%
\definecolor{currentfill}{rgb}{0.000000,0.000000,0.000000}%
\pgfsetfillcolor{currentfill}%
\pgfsetlinewidth{0.602250pt}%
\definecolor{currentstroke}{rgb}{0.000000,0.000000,0.000000}%
\pgfsetstrokecolor{currentstroke}%
\pgfsetdash{}{0pt}%
\pgfsys@defobject{currentmarker}{\pgfqpoint{0.000000in}{-0.027778in}}{\pgfqpoint{0.000000in}{0.000000in}}{%
\pgfpathmoveto{\pgfqpoint{0.000000in}{0.000000in}}%
\pgfpathlineto{\pgfqpoint{0.000000in}{-0.027778in}}%
\pgfusepath{stroke,fill}%
}%
\begin{pgfscope}%
\pgfsys@transformshift{2.876318in}{0.417642in}%
\pgfsys@useobject{currentmarker}{}%
\end{pgfscope}%
\end{pgfscope}%
\begin{pgfscope}%
\pgfpathrectangle{\pgfqpoint{0.770608in}{0.417642in}}{\pgfqpoint{3.162547in}{2.024187in}}%
\pgfusepath{clip}%
\pgfsetrectcap%
\pgfsetroundjoin%
\pgfsetlinewidth{0.803000pt}%
\definecolor{currentstroke}{rgb}{0.850000,0.850000,0.850000}%
\pgfsetstrokecolor{currentstroke}%
\pgfsetdash{}{0pt}%
\pgfpathmoveto{\pgfqpoint{2.902030in}{0.417642in}}%
\pgfpathlineto{\pgfqpoint{2.902030in}{2.441829in}}%
\pgfusepath{stroke}%
\end{pgfscope}%
\begin{pgfscope}%
\pgfsetbuttcap%
\pgfsetroundjoin%
\definecolor{currentfill}{rgb}{0.000000,0.000000,0.000000}%
\pgfsetfillcolor{currentfill}%
\pgfsetlinewidth{0.602250pt}%
\definecolor{currentstroke}{rgb}{0.000000,0.000000,0.000000}%
\pgfsetstrokecolor{currentstroke}%
\pgfsetdash{}{0pt}%
\pgfsys@defobject{currentmarker}{\pgfqpoint{0.000000in}{-0.027778in}}{\pgfqpoint{0.000000in}{0.000000in}}{%
\pgfpathmoveto{\pgfqpoint{0.000000in}{0.000000in}}%
\pgfpathlineto{\pgfqpoint{0.000000in}{-0.027778in}}%
\pgfusepath{stroke,fill}%
}%
\begin{pgfscope}%
\pgfsys@transformshift{2.902030in}{0.417642in}%
\pgfsys@useobject{currentmarker}{}%
\end{pgfscope}%
\end{pgfscope}%
\begin{pgfscope}%
\pgfpathrectangle{\pgfqpoint{0.770608in}{0.417642in}}{\pgfqpoint{3.162547in}{2.024187in}}%
\pgfusepath{clip}%
\pgfsetrectcap%
\pgfsetroundjoin%
\pgfsetlinewidth{0.803000pt}%
\definecolor{currentstroke}{rgb}{0.850000,0.850000,0.850000}%
\pgfsetstrokecolor{currentstroke}%
\pgfsetdash{}{0pt}%
\pgfpathmoveto{\pgfqpoint{3.076349in}{0.417642in}}%
\pgfpathlineto{\pgfqpoint{3.076349in}{2.441829in}}%
\pgfusepath{stroke}%
\end{pgfscope}%
\begin{pgfscope}%
\pgfsetbuttcap%
\pgfsetroundjoin%
\definecolor{currentfill}{rgb}{0.000000,0.000000,0.000000}%
\pgfsetfillcolor{currentfill}%
\pgfsetlinewidth{0.602250pt}%
\definecolor{currentstroke}{rgb}{0.000000,0.000000,0.000000}%
\pgfsetstrokecolor{currentstroke}%
\pgfsetdash{}{0pt}%
\pgfsys@defobject{currentmarker}{\pgfqpoint{0.000000in}{-0.027778in}}{\pgfqpoint{0.000000in}{0.000000in}}{%
\pgfpathmoveto{\pgfqpoint{0.000000in}{0.000000in}}%
\pgfpathlineto{\pgfqpoint{0.000000in}{-0.027778in}}%
\pgfusepath{stroke,fill}%
}%
\begin{pgfscope}%
\pgfsys@transformshift{3.076349in}{0.417642in}%
\pgfsys@useobject{currentmarker}{}%
\end{pgfscope}%
\end{pgfscope}%
\begin{pgfscope}%
\pgfpathrectangle{\pgfqpoint{0.770608in}{0.417642in}}{\pgfqpoint{3.162547in}{2.024187in}}%
\pgfusepath{clip}%
\pgfsetrectcap%
\pgfsetroundjoin%
\pgfsetlinewidth{0.803000pt}%
\definecolor{currentstroke}{rgb}{0.850000,0.850000,0.850000}%
\pgfsetstrokecolor{currentstroke}%
\pgfsetdash{}{0pt}%
\pgfpathmoveto{\pgfqpoint{3.164865in}{0.417642in}}%
\pgfpathlineto{\pgfqpoint{3.164865in}{2.441829in}}%
\pgfusepath{stroke}%
\end{pgfscope}%
\begin{pgfscope}%
\pgfsetbuttcap%
\pgfsetroundjoin%
\definecolor{currentfill}{rgb}{0.000000,0.000000,0.000000}%
\pgfsetfillcolor{currentfill}%
\pgfsetlinewidth{0.602250pt}%
\definecolor{currentstroke}{rgb}{0.000000,0.000000,0.000000}%
\pgfsetstrokecolor{currentstroke}%
\pgfsetdash{}{0pt}%
\pgfsys@defobject{currentmarker}{\pgfqpoint{0.000000in}{-0.027778in}}{\pgfqpoint{0.000000in}{0.000000in}}{%
\pgfpathmoveto{\pgfqpoint{0.000000in}{0.000000in}}%
\pgfpathlineto{\pgfqpoint{0.000000in}{-0.027778in}}%
\pgfusepath{stroke,fill}%
}%
\begin{pgfscope}%
\pgfsys@transformshift{3.164865in}{0.417642in}%
\pgfsys@useobject{currentmarker}{}%
\end{pgfscope}%
\end{pgfscope}%
\begin{pgfscope}%
\pgfpathrectangle{\pgfqpoint{0.770608in}{0.417642in}}{\pgfqpoint{3.162547in}{2.024187in}}%
\pgfusepath{clip}%
\pgfsetrectcap%
\pgfsetroundjoin%
\pgfsetlinewidth{0.803000pt}%
\definecolor{currentstroke}{rgb}{0.850000,0.850000,0.850000}%
\pgfsetstrokecolor{currentstroke}%
\pgfsetdash{}{0pt}%
\pgfpathmoveto{\pgfqpoint{3.227667in}{0.417642in}}%
\pgfpathlineto{\pgfqpoint{3.227667in}{2.441829in}}%
\pgfusepath{stroke}%
\end{pgfscope}%
\begin{pgfscope}%
\pgfsetbuttcap%
\pgfsetroundjoin%
\definecolor{currentfill}{rgb}{0.000000,0.000000,0.000000}%
\pgfsetfillcolor{currentfill}%
\pgfsetlinewidth{0.602250pt}%
\definecolor{currentstroke}{rgb}{0.000000,0.000000,0.000000}%
\pgfsetstrokecolor{currentstroke}%
\pgfsetdash{}{0pt}%
\pgfsys@defobject{currentmarker}{\pgfqpoint{0.000000in}{-0.027778in}}{\pgfqpoint{0.000000in}{0.000000in}}{%
\pgfpathmoveto{\pgfqpoint{0.000000in}{0.000000in}}%
\pgfpathlineto{\pgfqpoint{0.000000in}{-0.027778in}}%
\pgfusepath{stroke,fill}%
}%
\begin{pgfscope}%
\pgfsys@transformshift{3.227667in}{0.417642in}%
\pgfsys@useobject{currentmarker}{}%
\end{pgfscope}%
\end{pgfscope}%
\begin{pgfscope}%
\pgfpathrectangle{\pgfqpoint{0.770608in}{0.417642in}}{\pgfqpoint{3.162547in}{2.024187in}}%
\pgfusepath{clip}%
\pgfsetrectcap%
\pgfsetroundjoin%
\pgfsetlinewidth{0.803000pt}%
\definecolor{currentstroke}{rgb}{0.850000,0.850000,0.850000}%
\pgfsetstrokecolor{currentstroke}%
\pgfsetdash{}{0pt}%
\pgfpathmoveto{\pgfqpoint{3.276381in}{0.417642in}}%
\pgfpathlineto{\pgfqpoint{3.276381in}{2.441829in}}%
\pgfusepath{stroke}%
\end{pgfscope}%
\begin{pgfscope}%
\pgfsetbuttcap%
\pgfsetroundjoin%
\definecolor{currentfill}{rgb}{0.000000,0.000000,0.000000}%
\pgfsetfillcolor{currentfill}%
\pgfsetlinewidth{0.602250pt}%
\definecolor{currentstroke}{rgb}{0.000000,0.000000,0.000000}%
\pgfsetstrokecolor{currentstroke}%
\pgfsetdash{}{0pt}%
\pgfsys@defobject{currentmarker}{\pgfqpoint{0.000000in}{-0.027778in}}{\pgfqpoint{0.000000in}{0.000000in}}{%
\pgfpathmoveto{\pgfqpoint{0.000000in}{0.000000in}}%
\pgfpathlineto{\pgfqpoint{0.000000in}{-0.027778in}}%
\pgfusepath{stroke,fill}%
}%
\begin{pgfscope}%
\pgfsys@transformshift{3.276381in}{0.417642in}%
\pgfsys@useobject{currentmarker}{}%
\end{pgfscope}%
\end{pgfscope}%
\begin{pgfscope}%
\pgfpathrectangle{\pgfqpoint{0.770608in}{0.417642in}}{\pgfqpoint{3.162547in}{2.024187in}}%
\pgfusepath{clip}%
\pgfsetrectcap%
\pgfsetroundjoin%
\pgfsetlinewidth{0.803000pt}%
\definecolor{currentstroke}{rgb}{0.850000,0.850000,0.850000}%
\pgfsetstrokecolor{currentstroke}%
\pgfsetdash{}{0pt}%
\pgfpathmoveto{\pgfqpoint{3.316183in}{0.417642in}}%
\pgfpathlineto{\pgfqpoint{3.316183in}{2.441829in}}%
\pgfusepath{stroke}%
\end{pgfscope}%
\begin{pgfscope}%
\pgfsetbuttcap%
\pgfsetroundjoin%
\definecolor{currentfill}{rgb}{0.000000,0.000000,0.000000}%
\pgfsetfillcolor{currentfill}%
\pgfsetlinewidth{0.602250pt}%
\definecolor{currentstroke}{rgb}{0.000000,0.000000,0.000000}%
\pgfsetstrokecolor{currentstroke}%
\pgfsetdash{}{0pt}%
\pgfsys@defobject{currentmarker}{\pgfqpoint{0.000000in}{-0.027778in}}{\pgfqpoint{0.000000in}{0.000000in}}{%
\pgfpathmoveto{\pgfqpoint{0.000000in}{0.000000in}}%
\pgfpathlineto{\pgfqpoint{0.000000in}{-0.027778in}}%
\pgfusepath{stroke,fill}%
}%
\begin{pgfscope}%
\pgfsys@transformshift{3.316183in}{0.417642in}%
\pgfsys@useobject{currentmarker}{}%
\end{pgfscope}%
\end{pgfscope}%
\begin{pgfscope}%
\pgfpathrectangle{\pgfqpoint{0.770608in}{0.417642in}}{\pgfqpoint{3.162547in}{2.024187in}}%
\pgfusepath{clip}%
\pgfsetrectcap%
\pgfsetroundjoin%
\pgfsetlinewidth{0.803000pt}%
\definecolor{currentstroke}{rgb}{0.850000,0.850000,0.850000}%
\pgfsetstrokecolor{currentstroke}%
\pgfsetdash{}{0pt}%
\pgfpathmoveto{\pgfqpoint{3.349835in}{0.417642in}}%
\pgfpathlineto{\pgfqpoint{3.349835in}{2.441829in}}%
\pgfusepath{stroke}%
\end{pgfscope}%
\begin{pgfscope}%
\pgfsetbuttcap%
\pgfsetroundjoin%
\definecolor{currentfill}{rgb}{0.000000,0.000000,0.000000}%
\pgfsetfillcolor{currentfill}%
\pgfsetlinewidth{0.602250pt}%
\definecolor{currentstroke}{rgb}{0.000000,0.000000,0.000000}%
\pgfsetstrokecolor{currentstroke}%
\pgfsetdash{}{0pt}%
\pgfsys@defobject{currentmarker}{\pgfqpoint{0.000000in}{-0.027778in}}{\pgfqpoint{0.000000in}{0.000000in}}{%
\pgfpathmoveto{\pgfqpoint{0.000000in}{0.000000in}}%
\pgfpathlineto{\pgfqpoint{0.000000in}{-0.027778in}}%
\pgfusepath{stroke,fill}%
}%
\begin{pgfscope}%
\pgfsys@transformshift{3.349835in}{0.417642in}%
\pgfsys@useobject{currentmarker}{}%
\end{pgfscope}%
\end{pgfscope}%
\begin{pgfscope}%
\pgfpathrectangle{\pgfqpoint{0.770608in}{0.417642in}}{\pgfqpoint{3.162547in}{2.024187in}}%
\pgfusepath{clip}%
\pgfsetrectcap%
\pgfsetroundjoin%
\pgfsetlinewidth{0.803000pt}%
\definecolor{currentstroke}{rgb}{0.850000,0.850000,0.850000}%
\pgfsetstrokecolor{currentstroke}%
\pgfsetdash{}{0pt}%
\pgfpathmoveto{\pgfqpoint{3.378985in}{0.417642in}}%
\pgfpathlineto{\pgfqpoint{3.378985in}{2.441829in}}%
\pgfusepath{stroke}%
\end{pgfscope}%
\begin{pgfscope}%
\pgfsetbuttcap%
\pgfsetroundjoin%
\definecolor{currentfill}{rgb}{0.000000,0.000000,0.000000}%
\pgfsetfillcolor{currentfill}%
\pgfsetlinewidth{0.602250pt}%
\definecolor{currentstroke}{rgb}{0.000000,0.000000,0.000000}%
\pgfsetstrokecolor{currentstroke}%
\pgfsetdash{}{0pt}%
\pgfsys@defobject{currentmarker}{\pgfqpoint{0.000000in}{-0.027778in}}{\pgfqpoint{0.000000in}{0.000000in}}{%
\pgfpathmoveto{\pgfqpoint{0.000000in}{0.000000in}}%
\pgfpathlineto{\pgfqpoint{0.000000in}{-0.027778in}}%
\pgfusepath{stroke,fill}%
}%
\begin{pgfscope}%
\pgfsys@transformshift{3.378985in}{0.417642in}%
\pgfsys@useobject{currentmarker}{}%
\end{pgfscope}%
\end{pgfscope}%
\begin{pgfscope}%
\pgfpathrectangle{\pgfqpoint{0.770608in}{0.417642in}}{\pgfqpoint{3.162547in}{2.024187in}}%
\pgfusepath{clip}%
\pgfsetrectcap%
\pgfsetroundjoin%
\pgfsetlinewidth{0.803000pt}%
\definecolor{currentstroke}{rgb}{0.850000,0.850000,0.850000}%
\pgfsetstrokecolor{currentstroke}%
\pgfsetdash{}{0pt}%
\pgfpathmoveto{\pgfqpoint{3.404698in}{0.417642in}}%
\pgfpathlineto{\pgfqpoint{3.404698in}{2.441829in}}%
\pgfusepath{stroke}%
\end{pgfscope}%
\begin{pgfscope}%
\pgfsetbuttcap%
\pgfsetroundjoin%
\definecolor{currentfill}{rgb}{0.000000,0.000000,0.000000}%
\pgfsetfillcolor{currentfill}%
\pgfsetlinewidth{0.602250pt}%
\definecolor{currentstroke}{rgb}{0.000000,0.000000,0.000000}%
\pgfsetstrokecolor{currentstroke}%
\pgfsetdash{}{0pt}%
\pgfsys@defobject{currentmarker}{\pgfqpoint{0.000000in}{-0.027778in}}{\pgfqpoint{0.000000in}{0.000000in}}{%
\pgfpathmoveto{\pgfqpoint{0.000000in}{0.000000in}}%
\pgfpathlineto{\pgfqpoint{0.000000in}{-0.027778in}}%
\pgfusepath{stroke,fill}%
}%
\begin{pgfscope}%
\pgfsys@transformshift{3.404698in}{0.417642in}%
\pgfsys@useobject{currentmarker}{}%
\end{pgfscope}%
\end{pgfscope}%
\begin{pgfscope}%
\pgfpathrectangle{\pgfqpoint{0.770608in}{0.417642in}}{\pgfqpoint{3.162547in}{2.024187in}}%
\pgfusepath{clip}%
\pgfsetrectcap%
\pgfsetroundjoin%
\pgfsetlinewidth{0.803000pt}%
\definecolor{currentstroke}{rgb}{0.850000,0.850000,0.850000}%
\pgfsetstrokecolor{currentstroke}%
\pgfsetdash{}{0pt}%
\pgfpathmoveto{\pgfqpoint{3.579017in}{0.417642in}}%
\pgfpathlineto{\pgfqpoint{3.579017in}{2.441829in}}%
\pgfusepath{stroke}%
\end{pgfscope}%
\begin{pgfscope}%
\pgfsetbuttcap%
\pgfsetroundjoin%
\definecolor{currentfill}{rgb}{0.000000,0.000000,0.000000}%
\pgfsetfillcolor{currentfill}%
\pgfsetlinewidth{0.602250pt}%
\definecolor{currentstroke}{rgb}{0.000000,0.000000,0.000000}%
\pgfsetstrokecolor{currentstroke}%
\pgfsetdash{}{0pt}%
\pgfsys@defobject{currentmarker}{\pgfqpoint{0.000000in}{-0.027778in}}{\pgfqpoint{0.000000in}{0.000000in}}{%
\pgfpathmoveto{\pgfqpoint{0.000000in}{0.000000in}}%
\pgfpathlineto{\pgfqpoint{0.000000in}{-0.027778in}}%
\pgfusepath{stroke,fill}%
}%
\begin{pgfscope}%
\pgfsys@transformshift{3.579017in}{0.417642in}%
\pgfsys@useobject{currentmarker}{}%
\end{pgfscope}%
\end{pgfscope}%
\begin{pgfscope}%
\pgfpathrectangle{\pgfqpoint{0.770608in}{0.417642in}}{\pgfqpoint{3.162547in}{2.024187in}}%
\pgfusepath{clip}%
\pgfsetrectcap%
\pgfsetroundjoin%
\pgfsetlinewidth{0.803000pt}%
\definecolor{currentstroke}{rgb}{0.850000,0.850000,0.850000}%
\pgfsetstrokecolor{currentstroke}%
\pgfsetdash{}{0pt}%
\pgfpathmoveto{\pgfqpoint{3.667532in}{0.417642in}}%
\pgfpathlineto{\pgfqpoint{3.667532in}{2.441829in}}%
\pgfusepath{stroke}%
\end{pgfscope}%
\begin{pgfscope}%
\pgfsetbuttcap%
\pgfsetroundjoin%
\definecolor{currentfill}{rgb}{0.000000,0.000000,0.000000}%
\pgfsetfillcolor{currentfill}%
\pgfsetlinewidth{0.602250pt}%
\definecolor{currentstroke}{rgb}{0.000000,0.000000,0.000000}%
\pgfsetstrokecolor{currentstroke}%
\pgfsetdash{}{0pt}%
\pgfsys@defobject{currentmarker}{\pgfqpoint{0.000000in}{-0.027778in}}{\pgfqpoint{0.000000in}{0.000000in}}{%
\pgfpathmoveto{\pgfqpoint{0.000000in}{0.000000in}}%
\pgfpathlineto{\pgfqpoint{0.000000in}{-0.027778in}}%
\pgfusepath{stroke,fill}%
}%
\begin{pgfscope}%
\pgfsys@transformshift{3.667532in}{0.417642in}%
\pgfsys@useobject{currentmarker}{}%
\end{pgfscope}%
\end{pgfscope}%
\begin{pgfscope}%
\pgfpathrectangle{\pgfqpoint{0.770608in}{0.417642in}}{\pgfqpoint{3.162547in}{2.024187in}}%
\pgfusepath{clip}%
\pgfsetrectcap%
\pgfsetroundjoin%
\pgfsetlinewidth{0.803000pt}%
\definecolor{currentstroke}{rgb}{0.850000,0.850000,0.850000}%
\pgfsetstrokecolor{currentstroke}%
\pgfsetdash{}{0pt}%
\pgfpathmoveto{\pgfqpoint{3.730335in}{0.417642in}}%
\pgfpathlineto{\pgfqpoint{3.730335in}{2.441829in}}%
\pgfusepath{stroke}%
\end{pgfscope}%
\begin{pgfscope}%
\pgfsetbuttcap%
\pgfsetroundjoin%
\definecolor{currentfill}{rgb}{0.000000,0.000000,0.000000}%
\pgfsetfillcolor{currentfill}%
\pgfsetlinewidth{0.602250pt}%
\definecolor{currentstroke}{rgb}{0.000000,0.000000,0.000000}%
\pgfsetstrokecolor{currentstroke}%
\pgfsetdash{}{0pt}%
\pgfsys@defobject{currentmarker}{\pgfqpoint{0.000000in}{-0.027778in}}{\pgfqpoint{0.000000in}{0.000000in}}{%
\pgfpathmoveto{\pgfqpoint{0.000000in}{0.000000in}}%
\pgfpathlineto{\pgfqpoint{0.000000in}{-0.027778in}}%
\pgfusepath{stroke,fill}%
}%
\begin{pgfscope}%
\pgfsys@transformshift{3.730335in}{0.417642in}%
\pgfsys@useobject{currentmarker}{}%
\end{pgfscope}%
\end{pgfscope}%
\begin{pgfscope}%
\pgfpathrectangle{\pgfqpoint{0.770608in}{0.417642in}}{\pgfqpoint{3.162547in}{2.024187in}}%
\pgfusepath{clip}%
\pgfsetrectcap%
\pgfsetroundjoin%
\pgfsetlinewidth{0.803000pt}%
\definecolor{currentstroke}{rgb}{0.850000,0.850000,0.850000}%
\pgfsetstrokecolor{currentstroke}%
\pgfsetdash{}{0pt}%
\pgfpathmoveto{\pgfqpoint{3.779049in}{0.417642in}}%
\pgfpathlineto{\pgfqpoint{3.779049in}{2.441829in}}%
\pgfusepath{stroke}%
\end{pgfscope}%
\begin{pgfscope}%
\pgfsetbuttcap%
\pgfsetroundjoin%
\definecolor{currentfill}{rgb}{0.000000,0.000000,0.000000}%
\pgfsetfillcolor{currentfill}%
\pgfsetlinewidth{0.602250pt}%
\definecolor{currentstroke}{rgb}{0.000000,0.000000,0.000000}%
\pgfsetstrokecolor{currentstroke}%
\pgfsetdash{}{0pt}%
\pgfsys@defobject{currentmarker}{\pgfqpoint{0.000000in}{-0.027778in}}{\pgfqpoint{0.000000in}{0.000000in}}{%
\pgfpathmoveto{\pgfqpoint{0.000000in}{0.000000in}}%
\pgfpathlineto{\pgfqpoint{0.000000in}{-0.027778in}}%
\pgfusepath{stroke,fill}%
}%
\begin{pgfscope}%
\pgfsys@transformshift{3.779049in}{0.417642in}%
\pgfsys@useobject{currentmarker}{}%
\end{pgfscope}%
\end{pgfscope}%
\begin{pgfscope}%
\pgfpathrectangle{\pgfqpoint{0.770608in}{0.417642in}}{\pgfqpoint{3.162547in}{2.024187in}}%
\pgfusepath{clip}%
\pgfsetrectcap%
\pgfsetroundjoin%
\pgfsetlinewidth{0.803000pt}%
\definecolor{currentstroke}{rgb}{0.850000,0.850000,0.850000}%
\pgfsetstrokecolor{currentstroke}%
\pgfsetdash{}{0pt}%
\pgfpathmoveto{\pgfqpoint{3.818850in}{0.417642in}}%
\pgfpathlineto{\pgfqpoint{3.818850in}{2.441829in}}%
\pgfusepath{stroke}%
\end{pgfscope}%
\begin{pgfscope}%
\pgfsetbuttcap%
\pgfsetroundjoin%
\definecolor{currentfill}{rgb}{0.000000,0.000000,0.000000}%
\pgfsetfillcolor{currentfill}%
\pgfsetlinewidth{0.602250pt}%
\definecolor{currentstroke}{rgb}{0.000000,0.000000,0.000000}%
\pgfsetstrokecolor{currentstroke}%
\pgfsetdash{}{0pt}%
\pgfsys@defobject{currentmarker}{\pgfqpoint{0.000000in}{-0.027778in}}{\pgfqpoint{0.000000in}{0.000000in}}{%
\pgfpathmoveto{\pgfqpoint{0.000000in}{0.000000in}}%
\pgfpathlineto{\pgfqpoint{0.000000in}{-0.027778in}}%
\pgfusepath{stroke,fill}%
}%
\begin{pgfscope}%
\pgfsys@transformshift{3.818850in}{0.417642in}%
\pgfsys@useobject{currentmarker}{}%
\end{pgfscope}%
\end{pgfscope}%
\begin{pgfscope}%
\pgfpathrectangle{\pgfqpoint{0.770608in}{0.417642in}}{\pgfqpoint{3.162547in}{2.024187in}}%
\pgfusepath{clip}%
\pgfsetrectcap%
\pgfsetroundjoin%
\pgfsetlinewidth{0.803000pt}%
\definecolor{currentstroke}{rgb}{0.850000,0.850000,0.850000}%
\pgfsetstrokecolor{currentstroke}%
\pgfsetdash{}{0pt}%
\pgfpathmoveto{\pgfqpoint{3.852502in}{0.417642in}}%
\pgfpathlineto{\pgfqpoint{3.852502in}{2.441829in}}%
\pgfusepath{stroke}%
\end{pgfscope}%
\begin{pgfscope}%
\pgfsetbuttcap%
\pgfsetroundjoin%
\definecolor{currentfill}{rgb}{0.000000,0.000000,0.000000}%
\pgfsetfillcolor{currentfill}%
\pgfsetlinewidth{0.602250pt}%
\definecolor{currentstroke}{rgb}{0.000000,0.000000,0.000000}%
\pgfsetstrokecolor{currentstroke}%
\pgfsetdash{}{0pt}%
\pgfsys@defobject{currentmarker}{\pgfqpoint{0.000000in}{-0.027778in}}{\pgfqpoint{0.000000in}{0.000000in}}{%
\pgfpathmoveto{\pgfqpoint{0.000000in}{0.000000in}}%
\pgfpathlineto{\pgfqpoint{0.000000in}{-0.027778in}}%
\pgfusepath{stroke,fill}%
}%
\begin{pgfscope}%
\pgfsys@transformshift{3.852502in}{0.417642in}%
\pgfsys@useobject{currentmarker}{}%
\end{pgfscope}%
\end{pgfscope}%
\begin{pgfscope}%
\pgfpathrectangle{\pgfqpoint{0.770608in}{0.417642in}}{\pgfqpoint{3.162547in}{2.024187in}}%
\pgfusepath{clip}%
\pgfsetrectcap%
\pgfsetroundjoin%
\pgfsetlinewidth{0.803000pt}%
\definecolor{currentstroke}{rgb}{0.850000,0.850000,0.850000}%
\pgfsetstrokecolor{currentstroke}%
\pgfsetdash{}{0pt}%
\pgfpathmoveto{\pgfqpoint{3.881653in}{0.417642in}}%
\pgfpathlineto{\pgfqpoint{3.881653in}{2.441829in}}%
\pgfusepath{stroke}%
\end{pgfscope}%
\begin{pgfscope}%
\pgfsetbuttcap%
\pgfsetroundjoin%
\definecolor{currentfill}{rgb}{0.000000,0.000000,0.000000}%
\pgfsetfillcolor{currentfill}%
\pgfsetlinewidth{0.602250pt}%
\definecolor{currentstroke}{rgb}{0.000000,0.000000,0.000000}%
\pgfsetstrokecolor{currentstroke}%
\pgfsetdash{}{0pt}%
\pgfsys@defobject{currentmarker}{\pgfqpoint{0.000000in}{-0.027778in}}{\pgfqpoint{0.000000in}{0.000000in}}{%
\pgfpathmoveto{\pgfqpoint{0.000000in}{0.000000in}}%
\pgfpathlineto{\pgfqpoint{0.000000in}{-0.027778in}}%
\pgfusepath{stroke,fill}%
}%
\begin{pgfscope}%
\pgfsys@transformshift{3.881653in}{0.417642in}%
\pgfsys@useobject{currentmarker}{}%
\end{pgfscope}%
\end{pgfscope}%
\begin{pgfscope}%
\pgfpathrectangle{\pgfqpoint{0.770608in}{0.417642in}}{\pgfqpoint{3.162547in}{2.024187in}}%
\pgfusepath{clip}%
\pgfsetrectcap%
\pgfsetroundjoin%
\pgfsetlinewidth{0.803000pt}%
\definecolor{currentstroke}{rgb}{0.850000,0.850000,0.850000}%
\pgfsetstrokecolor{currentstroke}%
\pgfsetdash{}{0pt}%
\pgfpathmoveto{\pgfqpoint{3.907366in}{0.417642in}}%
\pgfpathlineto{\pgfqpoint{3.907366in}{2.441829in}}%
\pgfusepath{stroke}%
\end{pgfscope}%
\begin{pgfscope}%
\pgfsetbuttcap%
\pgfsetroundjoin%
\definecolor{currentfill}{rgb}{0.000000,0.000000,0.000000}%
\pgfsetfillcolor{currentfill}%
\pgfsetlinewidth{0.602250pt}%
\definecolor{currentstroke}{rgb}{0.000000,0.000000,0.000000}%
\pgfsetstrokecolor{currentstroke}%
\pgfsetdash{}{0pt}%
\pgfsys@defobject{currentmarker}{\pgfqpoint{0.000000in}{-0.027778in}}{\pgfqpoint{0.000000in}{0.000000in}}{%
\pgfpathmoveto{\pgfqpoint{0.000000in}{0.000000in}}%
\pgfpathlineto{\pgfqpoint{0.000000in}{-0.027778in}}%
\pgfusepath{stroke,fill}%
}%
\begin{pgfscope}%
\pgfsys@transformshift{3.907366in}{0.417642in}%
\pgfsys@useobject{currentmarker}{}%
\end{pgfscope}%
\end{pgfscope}%
\begin{pgfscope}%
\definecolor{textcolor}{rgb}{0.000000,0.000000,0.000000}%
\pgfsetstrokecolor{textcolor}%
\pgfsetfillcolor{textcolor}%
\pgftext[x=2.351882in,y=0.165003in,,top]{\color{textcolor}\rmfamily\fontsize{10.000000}{12.000000}\selectfont \(\displaystyle \tau\) in \unit{\second}}%
\end{pgfscope}%
\begin{pgfscope}%
\pgfpathrectangle{\pgfqpoint{0.770608in}{0.417642in}}{\pgfqpoint{3.162547in}{2.024187in}}%
\pgfusepath{clip}%
\pgfsetrectcap%
\pgfsetroundjoin%
\pgfsetlinewidth{0.803000pt}%
\definecolor{currentstroke}{rgb}{0.850000,0.850000,0.850000}%
\pgfsetstrokecolor{currentstroke}%
\pgfsetdash{}{0pt}%
\pgfpathmoveto{\pgfqpoint{0.770608in}{0.859723in}}%
\pgfpathlineto{\pgfqpoint{3.933156in}{0.859723in}}%
\pgfusepath{stroke}%
\end{pgfscope}%
\begin{pgfscope}%
\pgfsetbuttcap%
\pgfsetroundjoin%
\definecolor{currentfill}{rgb}{0.000000,0.000000,0.000000}%
\pgfsetfillcolor{currentfill}%
\pgfsetlinewidth{0.602250pt}%
\definecolor{currentstroke}{rgb}{0.000000,0.000000,0.000000}%
\pgfsetstrokecolor{currentstroke}%
\pgfsetdash{}{0pt}%
\pgfsys@defobject{currentmarker}{\pgfqpoint{-0.027778in}{0.000000in}}{\pgfqpoint{-0.000000in}{0.000000in}}{%
\pgfpathmoveto{\pgfqpoint{-0.000000in}{0.000000in}}%
\pgfpathlineto{\pgfqpoint{-0.027778in}{0.000000in}}%
\pgfusepath{stroke,fill}%
}%
\begin{pgfscope}%
\pgfsys@transformshift{0.770608in}{0.859723in}%
\pgfsys@useobject{currentmarker}{}%
\end{pgfscope}%
\end{pgfscope}%
\begin{pgfscope}%
\definecolor{textcolor}{rgb}{0.000000,0.000000,0.000000}%
\pgfsetstrokecolor{textcolor}%
\pgfsetfillcolor{textcolor}%
\pgftext[x=0.236114in, y=0.814783in, left, base]{\color{textcolor}\rmfamily\fontsize{8.000000}{9.600000}\selectfont \(\displaystyle {2\times10^{-7}}\)}%
\end{pgfscope}%
\begin{pgfscope}%
\pgfpathrectangle{\pgfqpoint{0.770608in}{0.417642in}}{\pgfqpoint{3.162547in}{2.024187in}}%
\pgfusepath{clip}%
\pgfsetrectcap%
\pgfsetroundjoin%
\pgfsetlinewidth{0.803000pt}%
\definecolor{currentstroke}{rgb}{0.850000,0.850000,0.850000}%
\pgfsetstrokecolor{currentstroke}%
\pgfsetdash{}{0pt}%
\pgfpathmoveto{\pgfqpoint{0.770608in}{1.774410in}}%
\pgfpathlineto{\pgfqpoint{3.933156in}{1.774410in}}%
\pgfusepath{stroke}%
\end{pgfscope}%
\begin{pgfscope}%
\pgfsetbuttcap%
\pgfsetroundjoin%
\definecolor{currentfill}{rgb}{0.000000,0.000000,0.000000}%
\pgfsetfillcolor{currentfill}%
\pgfsetlinewidth{0.602250pt}%
\definecolor{currentstroke}{rgb}{0.000000,0.000000,0.000000}%
\pgfsetstrokecolor{currentstroke}%
\pgfsetdash{}{0pt}%
\pgfsys@defobject{currentmarker}{\pgfqpoint{-0.027778in}{0.000000in}}{\pgfqpoint{-0.000000in}{0.000000in}}{%
\pgfpathmoveto{\pgfqpoint{-0.000000in}{0.000000in}}%
\pgfpathlineto{\pgfqpoint{-0.027778in}{0.000000in}}%
\pgfusepath{stroke,fill}%
}%
\begin{pgfscope}%
\pgfsys@transformshift{0.770608in}{1.774410in}%
\pgfsys@useobject{currentmarker}{}%
\end{pgfscope}%
\end{pgfscope}%
\begin{pgfscope}%
\definecolor{textcolor}{rgb}{0.000000,0.000000,0.000000}%
\pgfsetstrokecolor{textcolor}%
\pgfsetfillcolor{textcolor}%
\pgftext[x=0.236114in, y=1.729470in, left, base]{\color{textcolor}\rmfamily\fontsize{8.000000}{9.600000}\selectfont \(\displaystyle {3\times10^{-7}}\)}%
\end{pgfscope}%
\begin{pgfscope}%
\pgfpathrectangle{\pgfqpoint{0.770608in}{0.417642in}}{\pgfqpoint{3.162547in}{2.024187in}}%
\pgfusepath{clip}%
\pgfsetrectcap%
\pgfsetroundjoin%
\pgfsetlinewidth{0.803000pt}%
\definecolor{currentstroke}{rgb}{0.850000,0.850000,0.850000}%
\pgfsetstrokecolor{currentstroke}%
\pgfsetdash{}{0pt}%
\pgfpathmoveto{\pgfqpoint{0.770608in}{2.423390in}}%
\pgfpathlineto{\pgfqpoint{3.933156in}{2.423390in}}%
\pgfusepath{stroke}%
\end{pgfscope}%
\begin{pgfscope}%
\pgfsetbuttcap%
\pgfsetroundjoin%
\definecolor{currentfill}{rgb}{0.000000,0.000000,0.000000}%
\pgfsetfillcolor{currentfill}%
\pgfsetlinewidth{0.602250pt}%
\definecolor{currentstroke}{rgb}{0.000000,0.000000,0.000000}%
\pgfsetstrokecolor{currentstroke}%
\pgfsetdash{}{0pt}%
\pgfsys@defobject{currentmarker}{\pgfqpoint{-0.027778in}{0.000000in}}{\pgfqpoint{-0.000000in}{0.000000in}}{%
\pgfpathmoveto{\pgfqpoint{-0.000000in}{0.000000in}}%
\pgfpathlineto{\pgfqpoint{-0.027778in}{0.000000in}}%
\pgfusepath{stroke,fill}%
}%
\begin{pgfscope}%
\pgfsys@transformshift{0.770608in}{2.423390in}%
\pgfsys@useobject{currentmarker}{}%
\end{pgfscope}%
\end{pgfscope}%
\begin{pgfscope}%
\definecolor{textcolor}{rgb}{0.000000,0.000000,0.000000}%
\pgfsetstrokecolor{textcolor}%
\pgfsetfillcolor{textcolor}%
\pgftext[x=0.236114in, y=2.378450in, left, base]{\color{textcolor}\rmfamily\fontsize{8.000000}{9.600000}\selectfont \(\displaystyle {4\times10^{-7}}\)}%
\end{pgfscope}%
\begin{pgfscope}%
\definecolor{textcolor}{rgb}{0.000000,0.000000,0.000000}%
\pgfsetstrokecolor{textcolor}%
\pgfsetfillcolor{textcolor}%
\pgftext[x=0.180559in,y=1.429735in,,bottom,rotate=90.000000]{\color{textcolor}\rmfamily\fontsize{10.000000}{12.000000}\selectfont ADEV \(\displaystyle \sigma_A(\tau)\)}%
\end{pgfscope}%
\begin{pgfscope}%
\pgfpathrectangle{\pgfqpoint{0.770608in}{0.417642in}}{\pgfqpoint{3.162547in}{2.024187in}}%
\pgfusepath{clip}%
\pgfsetbuttcap%
\pgfsetroundjoin%
\definecolor{currentfill}{rgb}{0.835294,0.368627,0.000000}%
\pgfsetfillcolor{currentfill}%
\pgfsetlinewidth{1.003750pt}%
\definecolor{currentstroke}{rgb}{0.835294,0.368627,0.000000}%
\pgfsetstrokecolor{currentstroke}%
\pgfsetdash{}{0pt}%
\pgfsys@defobject{currentmarker}{\pgfqpoint{-0.020833in}{-0.020833in}}{\pgfqpoint{0.020833in}{0.020833in}}{%
\pgfpathmoveto{\pgfqpoint{0.000000in}{-0.020833in}}%
\pgfpathcurveto{\pgfqpoint{0.005525in}{-0.020833in}}{\pgfqpoint{0.010825in}{-0.018638in}}{\pgfqpoint{0.014731in}{-0.014731in}}%
\pgfpathcurveto{\pgfqpoint{0.018638in}{-0.010825in}}{\pgfqpoint{0.020833in}{-0.005525in}}{\pgfqpoint{0.020833in}{0.000000in}}%
\pgfpathcurveto{\pgfqpoint{0.020833in}{0.005525in}}{\pgfqpoint{0.018638in}{0.010825in}}{\pgfqpoint{0.014731in}{0.014731in}}%
\pgfpathcurveto{\pgfqpoint{0.010825in}{0.018638in}}{\pgfqpoint{0.005525in}{0.020833in}}{\pgfqpoint{0.000000in}{0.020833in}}%
\pgfpathcurveto{\pgfqpoint{-0.005525in}{0.020833in}}{\pgfqpoint{-0.010825in}{0.018638in}}{\pgfqpoint{-0.014731in}{0.014731in}}%
\pgfpathcurveto{\pgfqpoint{-0.018638in}{0.010825in}}{\pgfqpoint{-0.020833in}{0.005525in}}{\pgfqpoint{-0.020833in}{0.000000in}}%
\pgfpathcurveto{\pgfqpoint{-0.020833in}{-0.005525in}}{\pgfqpoint{-0.018638in}{-0.010825in}}{\pgfqpoint{-0.014731in}{-0.014731in}}%
\pgfpathcurveto{\pgfqpoint{-0.010825in}{-0.018638in}}{\pgfqpoint{-0.005525in}{-0.020833in}}{\pgfqpoint{0.000000in}{-0.020833in}}%
\pgfpathlineto{\pgfqpoint{0.000000in}{-0.020833in}}%
\pgfpathclose%
\pgfusepath{stroke,fill}%
}%
\begin{pgfscope}%
\pgfsys@transformshift{1.065679in}{2.349820in}%
\pgfsys@useobject{currentmarker}{}%
\end{pgfscope}%
\begin{pgfscope}%
\pgfsys@transformshift{1.216997in}{1.887192in}%
\pgfsys@useobject{currentmarker}{}%
\end{pgfscope}%
\begin{pgfscope}%
\pgfsys@transformshift{1.368315in}{1.590536in}%
\pgfsys@useobject{currentmarker}{}%
\end{pgfscope}%
\begin{pgfscope}%
\pgfsys@transformshift{1.519633in}{1.426472in}%
\pgfsys@useobject{currentmarker}{}%
\end{pgfscope}%
\begin{pgfscope}%
\pgfsys@transformshift{1.670951in}{1.340547in}%
\pgfsys@useobject{currentmarker}{}%
\end{pgfscope}%
\begin{pgfscope}%
\pgfsys@transformshift{1.822269in}{1.290767in}%
\pgfsys@useobject{currentmarker}{}%
\end{pgfscope}%
\begin{pgfscope}%
\pgfsys@transformshift{1.973587in}{1.278402in}%
\pgfsys@useobject{currentmarker}{}%
\end{pgfscope}%
\begin{pgfscope}%
\pgfsys@transformshift{2.124905in}{1.249216in}%
\pgfsys@useobject{currentmarker}{}%
\end{pgfscope}%
\begin{pgfscope}%
\pgfsys@transformshift{2.276223in}{1.193297in}%
\pgfsys@useobject{currentmarker}{}%
\end{pgfscope}%
\begin{pgfscope}%
\pgfsys@transformshift{2.427541in}{1.208473in}%
\pgfsys@useobject{currentmarker}{}%
\end{pgfscope}%
\begin{pgfscope}%
\pgfsys@transformshift{2.578859in}{1.197027in}%
\pgfsys@useobject{currentmarker}{}%
\end{pgfscope}%
\begin{pgfscope}%
\pgfsys@transformshift{2.730177in}{1.197157in}%
\pgfsys@useobject{currentmarker}{}%
\end{pgfscope}%
\begin{pgfscope}%
\pgfsys@transformshift{2.881495in}{1.197295in}%
\pgfsys@useobject{currentmarker}{}%
\end{pgfscope}%
\begin{pgfscope}%
\pgfsys@transformshift{3.032813in}{1.150942in}%
\pgfsys@useobject{currentmarker}{}%
\end{pgfscope}%
\begin{pgfscope}%
\pgfsys@transformshift{3.184131in}{1.249516in}%
\pgfsys@useobject{currentmarker}{}%
\end{pgfscope}%
\begin{pgfscope}%
\pgfsys@transformshift{3.335449in}{1.374248in}%
\pgfsys@useobject{currentmarker}{}%
\end{pgfscope}%
\begin{pgfscope}%
\pgfsys@transformshift{3.486767in}{1.434432in}%
\pgfsys@useobject{currentmarker}{}%
\end{pgfscope}%
\begin{pgfscope}%
\pgfsys@transformshift{3.638085in}{1.689817in}%
\pgfsys@useobject{currentmarker}{}%
\end{pgfscope}%
\begin{pgfscope}%
\pgfsys@transformshift{3.789403in}{0.964402in}%
\pgfsys@useobject{currentmarker}{}%
\end{pgfscope}%
\end{pgfscope}%
\begin{pgfscope}%
\pgfpathrectangle{\pgfqpoint{0.770608in}{0.417642in}}{\pgfqpoint{3.162547in}{2.024187in}}%
\pgfusepath{clip}%
\pgfsetbuttcap%
\pgfsetroundjoin%
\pgfsetlinewidth{1.505625pt}%
\definecolor{currentstroke}{rgb}{0.003922,0.450980,0.698039}%
\pgfsetstrokecolor{currentstroke}%
\pgfsetdash{{5.550000pt}{2.400000pt}}{0.000000pt}%
\pgfpathmoveto{\pgfqpoint{0.914360in}{2.241114in}}%
\pgfpathlineto{\pgfqpoint{0.970212in}{1.952537in}}%
\pgfpathlineto{\pgfqpoint{1.026064in}{1.663959in}}%
\pgfpathlineto{\pgfqpoint{1.081916in}{1.375382in}}%
\pgfpathlineto{\pgfqpoint{1.137768in}{1.086805in}}%
\pgfpathlineto{\pgfqpoint{1.193620in}{0.798228in}}%
\pgfpathlineto{\pgfqpoint{1.249472in}{0.509650in}}%
\pgfusepath{stroke}%
\end{pgfscope}%
\begin{pgfscope}%
\pgfpathrectangle{\pgfqpoint{0.770608in}{0.417642in}}{\pgfqpoint{3.162547in}{2.024187in}}%
\pgfusepath{clip}%
\pgfsetbuttcap%
\pgfsetroundjoin%
\pgfsetlinewidth{1.505625pt}%
\definecolor{currentstroke}{rgb}{0.007843,0.619608,0.450980}%
\pgfsetstrokecolor{currentstroke}%
\pgfsetdash{{5.550000pt}{2.400000pt}}{0.000000pt}%
\pgfpathmoveto{\pgfqpoint{1.065679in}{1.251522in}}%
\pgfpathlineto{\pgfqpoint{1.216997in}{1.251522in}}%
\pgfpathlineto{\pgfqpoint{1.368315in}{1.251522in}}%
\pgfpathlineto{\pgfqpoint{1.519633in}{1.251522in}}%
\pgfpathlineto{\pgfqpoint{1.670951in}{1.251522in}}%
\pgfpathlineto{\pgfqpoint{1.822269in}{1.251522in}}%
\pgfpathlineto{\pgfqpoint{1.973587in}{1.251522in}}%
\pgfpathlineto{\pgfqpoint{2.124905in}{1.251522in}}%
\pgfpathlineto{\pgfqpoint{2.276223in}{1.251522in}}%
\pgfpathlineto{\pgfqpoint{2.427541in}{1.251522in}}%
\pgfpathlineto{\pgfqpoint{2.578859in}{1.251522in}}%
\pgfpathlineto{\pgfqpoint{2.730177in}{1.251522in}}%
\pgfpathlineto{\pgfqpoint{2.881495in}{1.251522in}}%
\pgfpathlineto{\pgfqpoint{3.032813in}{1.251522in}}%
\pgfpathlineto{\pgfqpoint{3.184131in}{1.251522in}}%
\pgfpathlineto{\pgfqpoint{3.335449in}{1.251522in}}%
\pgfpathlineto{\pgfqpoint{3.486767in}{1.251522in}}%
\pgfpathlineto{\pgfqpoint{3.638085in}{1.251522in}}%
\pgfpathlineto{\pgfqpoint{3.789403in}{1.251522in}}%
\pgfusepath{stroke}%
\end{pgfscope}%
\begin{pgfscope}%
\pgfsetrectcap%
\pgfsetmiterjoin%
\pgfsetlinewidth{0.803000pt}%
\definecolor{currentstroke}{rgb}{0.000000,0.000000,0.000000}%
\pgfsetstrokecolor{currentstroke}%
\pgfsetdash{}{0pt}%
\pgfpathmoveto{\pgfqpoint{0.770608in}{0.417642in}}%
\pgfpathlineto{\pgfqpoint{0.770608in}{2.441829in}}%
\pgfusepath{stroke}%
\end{pgfscope}%
\begin{pgfscope}%
\pgfsetrectcap%
\pgfsetmiterjoin%
\pgfsetlinewidth{0.803000pt}%
\definecolor{currentstroke}{rgb}{0.000000,0.000000,0.000000}%
\pgfsetstrokecolor{currentstroke}%
\pgfsetdash{}{0pt}%
\pgfpathmoveto{\pgfqpoint{3.933156in}{0.417642in}}%
\pgfpathlineto{\pgfqpoint{3.933156in}{2.441829in}}%
\pgfusepath{stroke}%
\end{pgfscope}%
\begin{pgfscope}%
\pgfsetrectcap%
\pgfsetmiterjoin%
\pgfsetlinewidth{0.803000pt}%
\definecolor{currentstroke}{rgb}{0.000000,0.000000,0.000000}%
\pgfsetstrokecolor{currentstroke}%
\pgfsetdash{}{0pt}%
\pgfpathmoveto{\pgfqpoint{0.770608in}{0.417642in}}%
\pgfpathlineto{\pgfqpoint{3.933156in}{0.417642in}}%
\pgfusepath{stroke}%
\end{pgfscope}%
\begin{pgfscope}%
\pgfsetrectcap%
\pgfsetmiterjoin%
\pgfsetlinewidth{0.803000pt}%
\definecolor{currentstroke}{rgb}{0.000000,0.000000,0.000000}%
\pgfsetstrokecolor{currentstroke}%
\pgfsetdash{}{0pt}%
\pgfpathmoveto{\pgfqpoint{0.770608in}{2.441829in}}%
\pgfpathlineto{\pgfqpoint{3.933156in}{2.441829in}}%
\pgfusepath{stroke}%
\end{pgfscope}%
\begin{pgfscope}%
\pgfsetbuttcap%
\pgfsetmiterjoin%
\definecolor{currentfill}{rgb}{1.000000,1.000000,1.000000}%
\pgfsetfillcolor{currentfill}%
\pgfsetfillopacity{0.800000}%
\pgfsetlinewidth{1.003750pt}%
\definecolor{currentstroke}{rgb}{0.800000,0.800000,0.800000}%
\pgfsetstrokecolor{currentstroke}%
\pgfsetstrokeopacity{0.800000}%
\pgfsetdash{}{0pt}%
\pgfpathmoveto{\pgfqpoint{2.855045in}{2.043162in}}%
\pgfpathlineto{\pgfqpoint{3.855378in}{2.043162in}}%
\pgfpathquadraticcurveto{\pgfqpoint{3.877600in}{2.043162in}}{\pgfqpoint{3.877600in}{2.065385in}}%
\pgfpathlineto{\pgfqpoint{3.877600in}{2.364051in}}%
\pgfpathquadraticcurveto{\pgfqpoint{3.877600in}{2.386273in}}{\pgfqpoint{3.855378in}{2.386273in}}%
\pgfpathlineto{\pgfqpoint{2.855045in}{2.386273in}}%
\pgfpathquadraticcurveto{\pgfqpoint{2.832822in}{2.386273in}}{\pgfqpoint{2.832822in}{2.364051in}}%
\pgfpathlineto{\pgfqpoint{2.832822in}{2.065385in}}%
\pgfpathquadraticcurveto{\pgfqpoint{2.832822in}{2.043162in}}{\pgfqpoint{2.855045in}{2.043162in}}%
\pgfpathlineto{\pgfqpoint{2.855045in}{2.043162in}}%
\pgfpathclose%
\pgfusepath{stroke,fill}%
\end{pgfscope}%
\begin{pgfscope}%
\pgfsetbuttcap%
\pgfsetroundjoin%
\pgfsetlinewidth{1.505625pt}%
\definecolor{currentstroke}{rgb}{0.003922,0.450980,0.698039}%
\pgfsetstrokecolor{currentstroke}%
\pgfsetdash{{5.550000pt}{2.400000pt}}{0.000000pt}%
\pgfpathmoveto{\pgfqpoint{2.877267in}{2.302940in}}%
\pgfpathlineto{\pgfqpoint{2.988378in}{2.302940in}}%
\pgfpathlineto{\pgfqpoint{3.099489in}{2.302940in}}%
\pgfusepath{stroke}%
\end{pgfscope}%
\begin{pgfscope}%
\definecolor{textcolor}{rgb}{0.000000,0.000000,0.000000}%
\pgfsetstrokecolor{textcolor}%
\pgfsetfillcolor{textcolor}%
\pgftext[x=3.188378in,y=2.264051in,left,base]{\color{textcolor}\rmfamily\fontsize{8.000000}{9.600000}\selectfont White noise}%
\end{pgfscope}%
\begin{pgfscope}%
\pgfsetbuttcap%
\pgfsetroundjoin%
\pgfsetlinewidth{1.505625pt}%
\definecolor{currentstroke}{rgb}{0.007843,0.619608,0.450980}%
\pgfsetstrokecolor{currentstroke}%
\pgfsetdash{{5.550000pt}{2.400000pt}}{0.000000pt}%
\pgfpathmoveto{\pgfqpoint{2.877267in}{2.148051in}}%
\pgfpathlineto{\pgfqpoint{2.988378in}{2.148051in}}%
\pgfpathlineto{\pgfqpoint{3.099489in}{2.148051in}}%
\pgfusepath{stroke}%
\end{pgfscope}%
\begin{pgfscope}%
\definecolor{textcolor}{rgb}{0.000000,0.000000,0.000000}%
\pgfsetstrokecolor{textcolor}%
\pgfsetfillcolor{textcolor}%
\pgftext[x=3.188378in,y=2.109162in,left,base]{\color{textcolor}\rmfamily\fontsize{8.000000}{9.600000}\selectfont Flicker noise}%
\end{pgfscope}%
\end{pgfpicture}%
\makeatother%
\endgroup%

    \caption{Simulated Allan deviation of the input amplifier of a Keysight \device{3458A} containing white noise and flicker noise.}
    \label{fig:autozero_raw_adev}
\end{figure}

The Allan deviation it plotted in figure \ref{fig:autozero_raw_adev} and shows two distinct regions. Short $\tau$ display an asymptotic behaviour towards white noise with a $\tau^{−0.5}$ dependence and at longer $\tau$ the constant flicker noise region can be identified. At very long $\tau$ typical end-of-data oscillations can be seen, which are the result of the limited confidence of the Allan deviation estimator as previously discussed and can be safely ignored. The Allan deviation clearly demonstrates the performance of the device at longer integration times and it is obvious, that an beyond integration time of about \qty{1}{\second} or \qty{50}{\plc} no additional information can be extracted from the measurement and the variance is constant. This leads to the need to autozeroing to remove the flicker noise. It can be shown \cite{autozero_with_dead_time}, that subtracting a reference measurement from the actual measurement data removes all correlated effects. Since flicker noise is autocorrelated, it can be removed by subtracting a zero measurement.

To demonstrate autozeroing, two cases will be discussed. Going back to figure \ref{fig:dmm_autozer_offset_nulling} it can be seen, that between switching inputs, a dead time $\theta$ is added. For a first discussion, this dead time neglected and then the effect of adding a dead time is discussed.

Using figure \ref{fig:autozero_raw_adev} it was shown, that integrating over flicker noise, does not reduce the variance. In order to have as little flicker noise content in the final measurement value it is clear that the autozeroing should be done as fast possible to keep the flicker noise content out. This allows to calculate the expected variance of the autozeroed measurement. The noise of the input measurement $x$ and the reference measurement $y$ are the same, because in this model the only noise source comes from the input amplifier, as the input signal is assumed to be noise-free. The zero level is, by definition, noise-free. As dicussed above, the autozero interval is chosen, so that its variance is dominated by white noise The variance $\sigma^2$ of the combined measurement of $x-y$ can then be calculated using equation \ref{eqn:adding_white_noise}:
\begin{equation}
    \sigma_{x-y}^2 = \sigma_x^2 + \sigma_y^2 \label{eqn:autozeroing}
\end{equation}

By subtracting the zero reading, the amplifier noise is effectively added twice to the final result, once for the input measurement and once for the zero measurement. Additional noise from the input signal noise would simply be added to this as it is uncorrelated as well.

Do note, that the number of samples is now half the number before applying autozeroing. This leads to an interesting effect. Imagine a data set containing only white noise with a variance $\sigma^2$. Removing half the samples, obviously does not change the variance as white noise is not correlated, but subtracting the samples is effectively decimating the data set and since the sampling rate is halfed, the Nyquist band is halfed as well. Unfortunately the input noise bandwdith stays the same. The second Nyquist band is then folded back into the first, thus doubling the noise power density.

To conclude, it is expected, that the variance doubles and the power spectral density quadruples!

These consideration can be compared to the simulated data. Applying the autozeroing algorithm to the simulated data set, the constant \qty{10}{\V} input signal was nulled for every odd value and then the residual noise was subtracted from the signal value. The result in the time domain is shown in figure \ref{fig:autozero_time}.

\begin{figure}[hb]
    \centering
    %% Creator: Matplotlib, PGF backend
%%
%% To include the figure in your LaTeX document, write
%%   \input{<filename>.pgf}
%%
%% Make sure the required packages are loaded in your preamble
%%   \usepackage{pgf}
%%
%% Also ensure that all the required font packages are loaded; for instance,
%% the lmodern package is sometimes necessary when using math font.
%%   \usepackage{lmodern}
%%
%% Figures using additional raster images can only be included by \input if
%% they are in the same directory as the main LaTeX file. For loading figures
%% from other directories you can use the `import` package
%%   \usepackage{import}
%%
%% and then include the figures with
%%   \import{<path to file>}{<filename>.pgf}
%%
%% Matplotlib used the following preamble
%%   \usepackage{siunitx}
%%   \usepackage{fontspec}
%%
\begingroup%
\makeatletter%
\begin{pgfpicture}%
\pgfpathrectangle{\pgfpointorigin}{\pgfqpoint{4.068242in}{2.514312in}}%
\pgfusepath{use as bounding box, clip}%
\begin{pgfscope}%
\pgfsetbuttcap%
\pgfsetmiterjoin%
\definecolor{currentfill}{rgb}{1.000000,1.000000,1.000000}%
\pgfsetfillcolor{currentfill}%
\pgfsetlinewidth{0.000000pt}%
\definecolor{currentstroke}{rgb}{1.000000,1.000000,1.000000}%
\pgfsetstrokecolor{currentstroke}%
\pgfsetdash{}{0pt}%
\pgfpathmoveto{\pgfqpoint{0.000000in}{0.000000in}}%
\pgfpathlineto{\pgfqpoint{4.068242in}{0.000000in}}%
\pgfpathlineto{\pgfqpoint{4.068242in}{2.514312in}}%
\pgfpathlineto{\pgfqpoint{0.000000in}{2.514312in}}%
\pgfpathlineto{\pgfqpoint{0.000000in}{0.000000in}}%
\pgfpathclose%
\pgfusepath{fill}%
\end{pgfscope}%
\begin{pgfscope}%
\pgfsetbuttcap%
\pgfsetmiterjoin%
\definecolor{currentfill}{rgb}{1.000000,1.000000,1.000000}%
\pgfsetfillcolor{currentfill}%
\pgfsetlinewidth{0.000000pt}%
\definecolor{currentstroke}{rgb}{0.000000,0.000000,0.000000}%
\pgfsetstrokecolor{currentstroke}%
\pgfsetstrokeopacity{0.000000}%
\pgfsetdash{}{0pt}%
\pgfpathmoveto{\pgfqpoint{0.471687in}{0.416447in}}%
\pgfpathlineto{\pgfqpoint{4.009530in}{0.416447in}}%
\pgfpathlineto{\pgfqpoint{4.009530in}{2.341095in}}%
\pgfpathlineto{\pgfqpoint{0.471687in}{2.341095in}}%
\pgfpathlineto{\pgfqpoint{0.471687in}{0.416447in}}%
\pgfpathclose%
\pgfusepath{fill}%
\end{pgfscope}%
\begin{pgfscope}%
\pgfpathrectangle{\pgfqpoint{0.471687in}{0.416447in}}{\pgfqpoint{3.537842in}{1.924647in}}%
\pgfusepath{clip}%
\pgfsetrectcap%
\pgfsetroundjoin%
\pgfsetlinewidth{0.803000pt}%
\definecolor{currentstroke}{rgb}{0.450000,0.450000,0.450000}%
\pgfsetstrokecolor{currentstroke}%
\pgfsetdash{}{0pt}%
\pgfpathmoveto{\pgfqpoint{0.632499in}{0.416447in}}%
\pgfpathlineto{\pgfqpoint{0.632499in}{2.341095in}}%
\pgfusepath{stroke}%
\end{pgfscope}%
\begin{pgfscope}%
\pgfsetbuttcap%
\pgfsetroundjoin%
\definecolor{currentfill}{rgb}{0.000000,0.000000,0.000000}%
\pgfsetfillcolor{currentfill}%
\pgfsetlinewidth{0.803000pt}%
\definecolor{currentstroke}{rgb}{0.000000,0.000000,0.000000}%
\pgfsetstrokecolor{currentstroke}%
\pgfsetdash{}{0pt}%
\pgfsys@defobject{currentmarker}{\pgfqpoint{0.000000in}{-0.048611in}}{\pgfqpoint{0.000000in}{0.000000in}}{%
\pgfpathmoveto{\pgfqpoint{0.000000in}{0.000000in}}%
\pgfpathlineto{\pgfqpoint{0.000000in}{-0.048611in}}%
\pgfusepath{stroke,fill}%
}%
\begin{pgfscope}%
\pgfsys@transformshift{0.632499in}{0.416447in}%
\pgfsys@useobject{currentmarker}{}%
\end{pgfscope}%
\end{pgfscope}%
\begin{pgfscope}%
\definecolor{textcolor}{rgb}{0.000000,0.000000,0.000000}%
\pgfsetstrokecolor{textcolor}%
\pgfsetfillcolor{textcolor}%
\pgftext[x=0.632499in,y=0.319225in,,top]{\color{textcolor}\rmfamily\fontsize{8.000000}{9.600000}\selectfont \(\displaystyle {0}\)}%
\end{pgfscope}%
\begin{pgfscope}%
\pgfpathrectangle{\pgfqpoint{0.471687in}{0.416447in}}{\pgfqpoint{3.537842in}{1.924647in}}%
\pgfusepath{clip}%
\pgfsetrectcap%
\pgfsetroundjoin%
\pgfsetlinewidth{0.803000pt}%
\definecolor{currentstroke}{rgb}{0.450000,0.450000,0.450000}%
\pgfsetstrokecolor{currentstroke}%
\pgfsetdash{}{0pt}%
\pgfpathmoveto{\pgfqpoint{1.034527in}{0.416447in}}%
\pgfpathlineto{\pgfqpoint{1.034527in}{2.341095in}}%
\pgfusepath{stroke}%
\end{pgfscope}%
\begin{pgfscope}%
\pgfsetbuttcap%
\pgfsetroundjoin%
\definecolor{currentfill}{rgb}{0.000000,0.000000,0.000000}%
\pgfsetfillcolor{currentfill}%
\pgfsetlinewidth{0.803000pt}%
\definecolor{currentstroke}{rgb}{0.000000,0.000000,0.000000}%
\pgfsetstrokecolor{currentstroke}%
\pgfsetdash{}{0pt}%
\pgfsys@defobject{currentmarker}{\pgfqpoint{0.000000in}{-0.048611in}}{\pgfqpoint{0.000000in}{0.000000in}}{%
\pgfpathmoveto{\pgfqpoint{0.000000in}{0.000000in}}%
\pgfpathlineto{\pgfqpoint{0.000000in}{-0.048611in}}%
\pgfusepath{stroke,fill}%
}%
\begin{pgfscope}%
\pgfsys@transformshift{1.034527in}{0.416447in}%
\pgfsys@useobject{currentmarker}{}%
\end{pgfscope}%
\end{pgfscope}%
\begin{pgfscope}%
\definecolor{textcolor}{rgb}{0.000000,0.000000,0.000000}%
\pgfsetstrokecolor{textcolor}%
\pgfsetfillcolor{textcolor}%
\pgftext[x=1.034527in,y=0.319225in,,top]{\color{textcolor}\rmfamily\fontsize{8.000000}{9.600000}\selectfont \(\displaystyle {25000}\)}%
\end{pgfscope}%
\begin{pgfscope}%
\pgfpathrectangle{\pgfqpoint{0.471687in}{0.416447in}}{\pgfqpoint{3.537842in}{1.924647in}}%
\pgfusepath{clip}%
\pgfsetrectcap%
\pgfsetroundjoin%
\pgfsetlinewidth{0.803000pt}%
\definecolor{currentstroke}{rgb}{0.450000,0.450000,0.450000}%
\pgfsetstrokecolor{currentstroke}%
\pgfsetdash{}{0pt}%
\pgfpathmoveto{\pgfqpoint{1.436555in}{0.416447in}}%
\pgfpathlineto{\pgfqpoint{1.436555in}{2.341095in}}%
\pgfusepath{stroke}%
\end{pgfscope}%
\begin{pgfscope}%
\pgfsetbuttcap%
\pgfsetroundjoin%
\definecolor{currentfill}{rgb}{0.000000,0.000000,0.000000}%
\pgfsetfillcolor{currentfill}%
\pgfsetlinewidth{0.803000pt}%
\definecolor{currentstroke}{rgb}{0.000000,0.000000,0.000000}%
\pgfsetstrokecolor{currentstroke}%
\pgfsetdash{}{0pt}%
\pgfsys@defobject{currentmarker}{\pgfqpoint{0.000000in}{-0.048611in}}{\pgfqpoint{0.000000in}{0.000000in}}{%
\pgfpathmoveto{\pgfqpoint{0.000000in}{0.000000in}}%
\pgfpathlineto{\pgfqpoint{0.000000in}{-0.048611in}}%
\pgfusepath{stroke,fill}%
}%
\begin{pgfscope}%
\pgfsys@transformshift{1.436555in}{0.416447in}%
\pgfsys@useobject{currentmarker}{}%
\end{pgfscope}%
\end{pgfscope}%
\begin{pgfscope}%
\definecolor{textcolor}{rgb}{0.000000,0.000000,0.000000}%
\pgfsetstrokecolor{textcolor}%
\pgfsetfillcolor{textcolor}%
\pgftext[x=1.436555in,y=0.319225in,,top]{\color{textcolor}\rmfamily\fontsize{8.000000}{9.600000}\selectfont \(\displaystyle {50000}\)}%
\end{pgfscope}%
\begin{pgfscope}%
\pgfpathrectangle{\pgfqpoint{0.471687in}{0.416447in}}{\pgfqpoint{3.537842in}{1.924647in}}%
\pgfusepath{clip}%
\pgfsetrectcap%
\pgfsetroundjoin%
\pgfsetlinewidth{0.803000pt}%
\definecolor{currentstroke}{rgb}{0.450000,0.450000,0.450000}%
\pgfsetstrokecolor{currentstroke}%
\pgfsetdash{}{0pt}%
\pgfpathmoveto{\pgfqpoint{1.838583in}{0.416447in}}%
\pgfpathlineto{\pgfqpoint{1.838583in}{2.341095in}}%
\pgfusepath{stroke}%
\end{pgfscope}%
\begin{pgfscope}%
\pgfsetbuttcap%
\pgfsetroundjoin%
\definecolor{currentfill}{rgb}{0.000000,0.000000,0.000000}%
\pgfsetfillcolor{currentfill}%
\pgfsetlinewidth{0.803000pt}%
\definecolor{currentstroke}{rgb}{0.000000,0.000000,0.000000}%
\pgfsetstrokecolor{currentstroke}%
\pgfsetdash{}{0pt}%
\pgfsys@defobject{currentmarker}{\pgfqpoint{0.000000in}{-0.048611in}}{\pgfqpoint{0.000000in}{0.000000in}}{%
\pgfpathmoveto{\pgfqpoint{0.000000in}{0.000000in}}%
\pgfpathlineto{\pgfqpoint{0.000000in}{-0.048611in}}%
\pgfusepath{stroke,fill}%
}%
\begin{pgfscope}%
\pgfsys@transformshift{1.838583in}{0.416447in}%
\pgfsys@useobject{currentmarker}{}%
\end{pgfscope}%
\end{pgfscope}%
\begin{pgfscope}%
\definecolor{textcolor}{rgb}{0.000000,0.000000,0.000000}%
\pgfsetstrokecolor{textcolor}%
\pgfsetfillcolor{textcolor}%
\pgftext[x=1.838583in,y=0.319225in,,top]{\color{textcolor}\rmfamily\fontsize{8.000000}{9.600000}\selectfont \(\displaystyle {75000}\)}%
\end{pgfscope}%
\begin{pgfscope}%
\pgfpathrectangle{\pgfqpoint{0.471687in}{0.416447in}}{\pgfqpoint{3.537842in}{1.924647in}}%
\pgfusepath{clip}%
\pgfsetrectcap%
\pgfsetroundjoin%
\pgfsetlinewidth{0.803000pt}%
\definecolor{currentstroke}{rgb}{0.450000,0.450000,0.450000}%
\pgfsetstrokecolor{currentstroke}%
\pgfsetdash{}{0pt}%
\pgfpathmoveto{\pgfqpoint{2.240612in}{0.416447in}}%
\pgfpathlineto{\pgfqpoint{2.240612in}{2.341095in}}%
\pgfusepath{stroke}%
\end{pgfscope}%
\begin{pgfscope}%
\pgfsetbuttcap%
\pgfsetroundjoin%
\definecolor{currentfill}{rgb}{0.000000,0.000000,0.000000}%
\pgfsetfillcolor{currentfill}%
\pgfsetlinewidth{0.803000pt}%
\definecolor{currentstroke}{rgb}{0.000000,0.000000,0.000000}%
\pgfsetstrokecolor{currentstroke}%
\pgfsetdash{}{0pt}%
\pgfsys@defobject{currentmarker}{\pgfqpoint{0.000000in}{-0.048611in}}{\pgfqpoint{0.000000in}{0.000000in}}{%
\pgfpathmoveto{\pgfqpoint{0.000000in}{0.000000in}}%
\pgfpathlineto{\pgfqpoint{0.000000in}{-0.048611in}}%
\pgfusepath{stroke,fill}%
}%
\begin{pgfscope}%
\pgfsys@transformshift{2.240612in}{0.416447in}%
\pgfsys@useobject{currentmarker}{}%
\end{pgfscope}%
\end{pgfscope}%
\begin{pgfscope}%
\definecolor{textcolor}{rgb}{0.000000,0.000000,0.000000}%
\pgfsetstrokecolor{textcolor}%
\pgfsetfillcolor{textcolor}%
\pgftext[x=2.240612in,y=0.319225in,,top]{\color{textcolor}\rmfamily\fontsize{8.000000}{9.600000}\selectfont \(\displaystyle {100000}\)}%
\end{pgfscope}%
\begin{pgfscope}%
\pgfpathrectangle{\pgfqpoint{0.471687in}{0.416447in}}{\pgfqpoint{3.537842in}{1.924647in}}%
\pgfusepath{clip}%
\pgfsetrectcap%
\pgfsetroundjoin%
\pgfsetlinewidth{0.803000pt}%
\definecolor{currentstroke}{rgb}{0.450000,0.450000,0.450000}%
\pgfsetstrokecolor{currentstroke}%
\pgfsetdash{}{0pt}%
\pgfpathmoveto{\pgfqpoint{2.642640in}{0.416447in}}%
\pgfpathlineto{\pgfqpoint{2.642640in}{2.341095in}}%
\pgfusepath{stroke}%
\end{pgfscope}%
\begin{pgfscope}%
\pgfsetbuttcap%
\pgfsetroundjoin%
\definecolor{currentfill}{rgb}{0.000000,0.000000,0.000000}%
\pgfsetfillcolor{currentfill}%
\pgfsetlinewidth{0.803000pt}%
\definecolor{currentstroke}{rgb}{0.000000,0.000000,0.000000}%
\pgfsetstrokecolor{currentstroke}%
\pgfsetdash{}{0pt}%
\pgfsys@defobject{currentmarker}{\pgfqpoint{0.000000in}{-0.048611in}}{\pgfqpoint{0.000000in}{0.000000in}}{%
\pgfpathmoveto{\pgfqpoint{0.000000in}{0.000000in}}%
\pgfpathlineto{\pgfqpoint{0.000000in}{-0.048611in}}%
\pgfusepath{stroke,fill}%
}%
\begin{pgfscope}%
\pgfsys@transformshift{2.642640in}{0.416447in}%
\pgfsys@useobject{currentmarker}{}%
\end{pgfscope}%
\end{pgfscope}%
\begin{pgfscope}%
\definecolor{textcolor}{rgb}{0.000000,0.000000,0.000000}%
\pgfsetstrokecolor{textcolor}%
\pgfsetfillcolor{textcolor}%
\pgftext[x=2.642640in,y=0.319225in,,top]{\color{textcolor}\rmfamily\fontsize{8.000000}{9.600000}\selectfont \(\displaystyle {125000}\)}%
\end{pgfscope}%
\begin{pgfscope}%
\pgfpathrectangle{\pgfqpoint{0.471687in}{0.416447in}}{\pgfqpoint{3.537842in}{1.924647in}}%
\pgfusepath{clip}%
\pgfsetrectcap%
\pgfsetroundjoin%
\pgfsetlinewidth{0.803000pt}%
\definecolor{currentstroke}{rgb}{0.450000,0.450000,0.450000}%
\pgfsetstrokecolor{currentstroke}%
\pgfsetdash{}{0pt}%
\pgfpathmoveto{\pgfqpoint{3.044668in}{0.416447in}}%
\pgfpathlineto{\pgfqpoint{3.044668in}{2.341095in}}%
\pgfusepath{stroke}%
\end{pgfscope}%
\begin{pgfscope}%
\pgfsetbuttcap%
\pgfsetroundjoin%
\definecolor{currentfill}{rgb}{0.000000,0.000000,0.000000}%
\pgfsetfillcolor{currentfill}%
\pgfsetlinewidth{0.803000pt}%
\definecolor{currentstroke}{rgb}{0.000000,0.000000,0.000000}%
\pgfsetstrokecolor{currentstroke}%
\pgfsetdash{}{0pt}%
\pgfsys@defobject{currentmarker}{\pgfqpoint{0.000000in}{-0.048611in}}{\pgfqpoint{0.000000in}{0.000000in}}{%
\pgfpathmoveto{\pgfqpoint{0.000000in}{0.000000in}}%
\pgfpathlineto{\pgfqpoint{0.000000in}{-0.048611in}}%
\pgfusepath{stroke,fill}%
}%
\begin{pgfscope}%
\pgfsys@transformshift{3.044668in}{0.416447in}%
\pgfsys@useobject{currentmarker}{}%
\end{pgfscope}%
\end{pgfscope}%
\begin{pgfscope}%
\definecolor{textcolor}{rgb}{0.000000,0.000000,0.000000}%
\pgfsetstrokecolor{textcolor}%
\pgfsetfillcolor{textcolor}%
\pgftext[x=3.044668in,y=0.319225in,,top]{\color{textcolor}\rmfamily\fontsize{8.000000}{9.600000}\selectfont \(\displaystyle {150000}\)}%
\end{pgfscope}%
\begin{pgfscope}%
\pgfpathrectangle{\pgfqpoint{0.471687in}{0.416447in}}{\pgfqpoint{3.537842in}{1.924647in}}%
\pgfusepath{clip}%
\pgfsetrectcap%
\pgfsetroundjoin%
\pgfsetlinewidth{0.803000pt}%
\definecolor{currentstroke}{rgb}{0.450000,0.450000,0.450000}%
\pgfsetstrokecolor{currentstroke}%
\pgfsetdash{}{0pt}%
\pgfpathmoveto{\pgfqpoint{3.446697in}{0.416447in}}%
\pgfpathlineto{\pgfqpoint{3.446697in}{2.341095in}}%
\pgfusepath{stroke}%
\end{pgfscope}%
\begin{pgfscope}%
\pgfsetbuttcap%
\pgfsetroundjoin%
\definecolor{currentfill}{rgb}{0.000000,0.000000,0.000000}%
\pgfsetfillcolor{currentfill}%
\pgfsetlinewidth{0.803000pt}%
\definecolor{currentstroke}{rgb}{0.000000,0.000000,0.000000}%
\pgfsetstrokecolor{currentstroke}%
\pgfsetdash{}{0pt}%
\pgfsys@defobject{currentmarker}{\pgfqpoint{0.000000in}{-0.048611in}}{\pgfqpoint{0.000000in}{0.000000in}}{%
\pgfpathmoveto{\pgfqpoint{0.000000in}{0.000000in}}%
\pgfpathlineto{\pgfqpoint{0.000000in}{-0.048611in}}%
\pgfusepath{stroke,fill}%
}%
\begin{pgfscope}%
\pgfsys@transformshift{3.446697in}{0.416447in}%
\pgfsys@useobject{currentmarker}{}%
\end{pgfscope}%
\end{pgfscope}%
\begin{pgfscope}%
\definecolor{textcolor}{rgb}{0.000000,0.000000,0.000000}%
\pgfsetstrokecolor{textcolor}%
\pgfsetfillcolor{textcolor}%
\pgftext[x=3.446697in,y=0.319225in,,top]{\color{textcolor}\rmfamily\fontsize{8.000000}{9.600000}\selectfont \(\displaystyle {175000}\)}%
\end{pgfscope}%
\begin{pgfscope}%
\pgfpathrectangle{\pgfqpoint{0.471687in}{0.416447in}}{\pgfqpoint{3.537842in}{1.924647in}}%
\pgfusepath{clip}%
\pgfsetrectcap%
\pgfsetroundjoin%
\pgfsetlinewidth{0.803000pt}%
\definecolor{currentstroke}{rgb}{0.450000,0.450000,0.450000}%
\pgfsetstrokecolor{currentstroke}%
\pgfsetdash{}{0pt}%
\pgfpathmoveto{\pgfqpoint{3.848725in}{0.416447in}}%
\pgfpathlineto{\pgfqpoint{3.848725in}{2.341095in}}%
\pgfusepath{stroke}%
\end{pgfscope}%
\begin{pgfscope}%
\pgfsetbuttcap%
\pgfsetroundjoin%
\definecolor{currentfill}{rgb}{0.000000,0.000000,0.000000}%
\pgfsetfillcolor{currentfill}%
\pgfsetlinewidth{0.803000pt}%
\definecolor{currentstroke}{rgb}{0.000000,0.000000,0.000000}%
\pgfsetstrokecolor{currentstroke}%
\pgfsetdash{}{0pt}%
\pgfsys@defobject{currentmarker}{\pgfqpoint{0.000000in}{-0.048611in}}{\pgfqpoint{0.000000in}{0.000000in}}{%
\pgfpathmoveto{\pgfqpoint{0.000000in}{0.000000in}}%
\pgfpathlineto{\pgfqpoint{0.000000in}{-0.048611in}}%
\pgfusepath{stroke,fill}%
}%
\begin{pgfscope}%
\pgfsys@transformshift{3.848725in}{0.416447in}%
\pgfsys@useobject{currentmarker}{}%
\end{pgfscope}%
\end{pgfscope}%
\begin{pgfscope}%
\definecolor{textcolor}{rgb}{0.000000,0.000000,0.000000}%
\pgfsetstrokecolor{textcolor}%
\pgfsetfillcolor{textcolor}%
\pgftext[x=3.848725in,y=0.319225in,,top]{\color{textcolor}\rmfamily\fontsize{8.000000}{9.600000}\selectfont \(\displaystyle {200000}\)}%
\end{pgfscope}%
\begin{pgfscope}%
\definecolor{textcolor}{rgb}{0.000000,0.000000,0.000000}%
\pgfsetstrokecolor{textcolor}%
\pgfsetfillcolor{textcolor}%
\pgftext[x=2.240609in,y=0.165003in,,top]{\color{textcolor}\rmfamily\fontsize{10.000000}{12.000000}\selectfont Time in \(\displaystyle \unit{\second}\)}%
\end{pgfscope}%
\begin{pgfscope}%
\pgfpathrectangle{\pgfqpoint{0.471687in}{0.416447in}}{\pgfqpoint{3.537842in}{1.924647in}}%
\pgfusepath{clip}%
\pgfsetrectcap%
\pgfsetroundjoin%
\pgfsetlinewidth{0.803000pt}%
\definecolor{currentstroke}{rgb}{0.450000,0.450000,0.450000}%
\pgfsetstrokecolor{currentstroke}%
\pgfsetdash{}{0pt}%
\pgfpathmoveto{\pgfqpoint{0.471687in}{0.523372in}}%
\pgfpathlineto{\pgfqpoint{4.009530in}{0.523372in}}%
\pgfusepath{stroke}%
\end{pgfscope}%
\begin{pgfscope}%
\pgfsetbuttcap%
\pgfsetroundjoin%
\definecolor{currentfill}{rgb}{0.000000,0.000000,0.000000}%
\pgfsetfillcolor{currentfill}%
\pgfsetlinewidth{0.803000pt}%
\definecolor{currentstroke}{rgb}{0.000000,0.000000,0.000000}%
\pgfsetstrokecolor{currentstroke}%
\pgfsetdash{}{0pt}%
\pgfsys@defobject{currentmarker}{\pgfqpoint{-0.048611in}{0.000000in}}{\pgfqpoint{-0.000000in}{0.000000in}}{%
\pgfpathmoveto{\pgfqpoint{-0.000000in}{0.000000in}}%
\pgfpathlineto{\pgfqpoint{-0.048611in}{0.000000in}}%
\pgfusepath{stroke,fill}%
}%
\begin{pgfscope}%
\pgfsys@transformshift{0.471687in}{0.523372in}%
\pgfsys@useobject{currentmarker}{}%
\end{pgfscope}%
\end{pgfscope}%
\begin{pgfscope}%
\definecolor{textcolor}{rgb}{0.000000,0.000000,0.000000}%
\pgfsetstrokecolor{textcolor}%
\pgfsetfillcolor{textcolor}%
\pgftext[x=0.223614in, y=0.484817in, left, base]{\color{textcolor}\rmfamily\fontsize{8.000000}{9.600000}\selectfont \(\displaystyle {\ensuremath{-}4}\)}%
\end{pgfscope}%
\begin{pgfscope}%
\pgfpathrectangle{\pgfqpoint{0.471687in}{0.416447in}}{\pgfqpoint{3.537842in}{1.924647in}}%
\pgfusepath{clip}%
\pgfsetrectcap%
\pgfsetroundjoin%
\pgfsetlinewidth{0.803000pt}%
\definecolor{currentstroke}{rgb}{0.450000,0.450000,0.450000}%
\pgfsetstrokecolor{currentstroke}%
\pgfsetdash{}{0pt}%
\pgfpathmoveto{\pgfqpoint{0.471687in}{0.951072in}}%
\pgfpathlineto{\pgfqpoint{4.009530in}{0.951072in}}%
\pgfusepath{stroke}%
\end{pgfscope}%
\begin{pgfscope}%
\pgfsetbuttcap%
\pgfsetroundjoin%
\definecolor{currentfill}{rgb}{0.000000,0.000000,0.000000}%
\pgfsetfillcolor{currentfill}%
\pgfsetlinewidth{0.803000pt}%
\definecolor{currentstroke}{rgb}{0.000000,0.000000,0.000000}%
\pgfsetstrokecolor{currentstroke}%
\pgfsetdash{}{0pt}%
\pgfsys@defobject{currentmarker}{\pgfqpoint{-0.048611in}{0.000000in}}{\pgfqpoint{-0.000000in}{0.000000in}}{%
\pgfpathmoveto{\pgfqpoint{-0.000000in}{0.000000in}}%
\pgfpathlineto{\pgfqpoint{-0.048611in}{0.000000in}}%
\pgfusepath{stroke,fill}%
}%
\begin{pgfscope}%
\pgfsys@transformshift{0.471687in}{0.951072in}%
\pgfsys@useobject{currentmarker}{}%
\end{pgfscope}%
\end{pgfscope}%
\begin{pgfscope}%
\definecolor{textcolor}{rgb}{0.000000,0.000000,0.000000}%
\pgfsetstrokecolor{textcolor}%
\pgfsetfillcolor{textcolor}%
\pgftext[x=0.223614in, y=0.912516in, left, base]{\color{textcolor}\rmfamily\fontsize{8.000000}{9.600000}\selectfont \(\displaystyle {\ensuremath{-}2}\)}%
\end{pgfscope}%
\begin{pgfscope}%
\pgfpathrectangle{\pgfqpoint{0.471687in}{0.416447in}}{\pgfqpoint{3.537842in}{1.924647in}}%
\pgfusepath{clip}%
\pgfsetrectcap%
\pgfsetroundjoin%
\pgfsetlinewidth{0.803000pt}%
\definecolor{currentstroke}{rgb}{0.450000,0.450000,0.450000}%
\pgfsetstrokecolor{currentstroke}%
\pgfsetdash{}{0pt}%
\pgfpathmoveto{\pgfqpoint{0.471687in}{1.378771in}}%
\pgfpathlineto{\pgfqpoint{4.009530in}{1.378771in}}%
\pgfusepath{stroke}%
\end{pgfscope}%
\begin{pgfscope}%
\pgfsetbuttcap%
\pgfsetroundjoin%
\definecolor{currentfill}{rgb}{0.000000,0.000000,0.000000}%
\pgfsetfillcolor{currentfill}%
\pgfsetlinewidth{0.803000pt}%
\definecolor{currentstroke}{rgb}{0.000000,0.000000,0.000000}%
\pgfsetstrokecolor{currentstroke}%
\pgfsetdash{}{0pt}%
\pgfsys@defobject{currentmarker}{\pgfqpoint{-0.048611in}{0.000000in}}{\pgfqpoint{-0.000000in}{0.000000in}}{%
\pgfpathmoveto{\pgfqpoint{-0.000000in}{0.000000in}}%
\pgfpathlineto{\pgfqpoint{-0.048611in}{0.000000in}}%
\pgfusepath{stroke,fill}%
}%
\begin{pgfscope}%
\pgfsys@transformshift{0.471687in}{1.378771in}%
\pgfsys@useobject{currentmarker}{}%
\end{pgfscope}%
\end{pgfscope}%
\begin{pgfscope}%
\definecolor{textcolor}{rgb}{0.000000,0.000000,0.000000}%
\pgfsetstrokecolor{textcolor}%
\pgfsetfillcolor{textcolor}%
\pgftext[x=0.315437in, y=1.340216in, left, base]{\color{textcolor}\rmfamily\fontsize{8.000000}{9.600000}\selectfont \(\displaystyle {0}\)}%
\end{pgfscope}%
\begin{pgfscope}%
\pgfpathrectangle{\pgfqpoint{0.471687in}{0.416447in}}{\pgfqpoint{3.537842in}{1.924647in}}%
\pgfusepath{clip}%
\pgfsetrectcap%
\pgfsetroundjoin%
\pgfsetlinewidth{0.803000pt}%
\definecolor{currentstroke}{rgb}{0.450000,0.450000,0.450000}%
\pgfsetstrokecolor{currentstroke}%
\pgfsetdash{}{0pt}%
\pgfpathmoveto{\pgfqpoint{0.471687in}{1.806471in}}%
\pgfpathlineto{\pgfqpoint{4.009530in}{1.806471in}}%
\pgfusepath{stroke}%
\end{pgfscope}%
\begin{pgfscope}%
\pgfsetbuttcap%
\pgfsetroundjoin%
\definecolor{currentfill}{rgb}{0.000000,0.000000,0.000000}%
\pgfsetfillcolor{currentfill}%
\pgfsetlinewidth{0.803000pt}%
\definecolor{currentstroke}{rgb}{0.000000,0.000000,0.000000}%
\pgfsetstrokecolor{currentstroke}%
\pgfsetdash{}{0pt}%
\pgfsys@defobject{currentmarker}{\pgfqpoint{-0.048611in}{0.000000in}}{\pgfqpoint{-0.000000in}{0.000000in}}{%
\pgfpathmoveto{\pgfqpoint{-0.000000in}{0.000000in}}%
\pgfpathlineto{\pgfqpoint{-0.048611in}{0.000000in}}%
\pgfusepath{stroke,fill}%
}%
\begin{pgfscope}%
\pgfsys@transformshift{0.471687in}{1.806471in}%
\pgfsys@useobject{currentmarker}{}%
\end{pgfscope}%
\end{pgfscope}%
\begin{pgfscope}%
\definecolor{textcolor}{rgb}{0.000000,0.000000,0.000000}%
\pgfsetstrokecolor{textcolor}%
\pgfsetfillcolor{textcolor}%
\pgftext[x=0.315437in, y=1.767915in, left, base]{\color{textcolor}\rmfamily\fontsize{8.000000}{9.600000}\selectfont \(\displaystyle {2}\)}%
\end{pgfscope}%
\begin{pgfscope}%
\pgfpathrectangle{\pgfqpoint{0.471687in}{0.416447in}}{\pgfqpoint{3.537842in}{1.924647in}}%
\pgfusepath{clip}%
\pgfsetrectcap%
\pgfsetroundjoin%
\pgfsetlinewidth{0.803000pt}%
\definecolor{currentstroke}{rgb}{0.450000,0.450000,0.450000}%
\pgfsetstrokecolor{currentstroke}%
\pgfsetdash{}{0pt}%
\pgfpathmoveto{\pgfqpoint{0.471687in}{2.234170in}}%
\pgfpathlineto{\pgfqpoint{4.009530in}{2.234170in}}%
\pgfusepath{stroke}%
\end{pgfscope}%
\begin{pgfscope}%
\pgfsetbuttcap%
\pgfsetroundjoin%
\definecolor{currentfill}{rgb}{0.000000,0.000000,0.000000}%
\pgfsetfillcolor{currentfill}%
\pgfsetlinewidth{0.803000pt}%
\definecolor{currentstroke}{rgb}{0.000000,0.000000,0.000000}%
\pgfsetstrokecolor{currentstroke}%
\pgfsetdash{}{0pt}%
\pgfsys@defobject{currentmarker}{\pgfqpoint{-0.048611in}{0.000000in}}{\pgfqpoint{-0.000000in}{0.000000in}}{%
\pgfpathmoveto{\pgfqpoint{-0.000000in}{0.000000in}}%
\pgfpathlineto{\pgfqpoint{-0.048611in}{0.000000in}}%
\pgfusepath{stroke,fill}%
}%
\begin{pgfscope}%
\pgfsys@transformshift{0.471687in}{2.234170in}%
\pgfsys@useobject{currentmarker}{}%
\end{pgfscope}%
\end{pgfscope}%
\begin{pgfscope}%
\definecolor{textcolor}{rgb}{0.000000,0.000000,0.000000}%
\pgfsetstrokecolor{textcolor}%
\pgfsetfillcolor{textcolor}%
\pgftext[x=0.315437in, y=2.195614in, left, base]{\color{textcolor}\rmfamily\fontsize{8.000000}{9.600000}\selectfont \(\displaystyle {4}\)}%
\end{pgfscope}%
\begin{pgfscope}%
\definecolor{textcolor}{rgb}{0.000000,0.000000,0.000000}%
\pgfsetstrokecolor{textcolor}%
\pgfsetfillcolor{textcolor}%
\pgftext[x=0.168059in,y=1.378771in,,bottom,rotate=90.000000]{\color{textcolor}\rmfamily\fontsize{10.000000}{12.000000}\selectfont Amplitude in \(\displaystyle \unit{\V}\)}%
\end{pgfscope}%
\begin{pgfscope}%
\definecolor{textcolor}{rgb}{0.000000,0.000000,0.000000}%
\pgfsetstrokecolor{textcolor}%
\pgfsetfillcolor{textcolor}%
\pgftext[x=0.471687in,y=2.382761in,left,base]{\color{textcolor}\rmfamily\fontsize{8.000000}{9.600000}\selectfont \(\displaystyle \times{10^{\ensuremath{-}6}}{+10^{1}}\)}%
\end{pgfscope}%
\begin{pgfscope}%
\pgfpathrectangle{\pgfqpoint{0.471687in}{0.416447in}}{\pgfqpoint{3.537842in}{1.924647in}}%
\pgfusepath{clip}%
\pgfsetrectcap%
\pgfsetroundjoin%
\pgfsetlinewidth{1.505625pt}%
\definecolor{currentstroke}{rgb}{0.925490,0.882353,0.200000}%
\pgfsetstrokecolor{currentstroke}%
\pgfsetdash{}{0pt}%
\pgfpathmoveto{\pgfqpoint{0.632499in}{1.287433in}}%
\pgfpathlineto{\pgfqpoint{0.633296in}{1.710792in}}%
\pgfpathlineto{\pgfqpoint{0.635908in}{1.021405in}}%
\pgfpathlineto{\pgfqpoint{0.641806in}{1.842120in}}%
\pgfpathlineto{\pgfqpoint{0.642314in}{1.004790in}}%
\pgfpathlineto{\pgfqpoint{0.646534in}{1.726777in}}%
\pgfpathlineto{\pgfqpoint{0.648650in}{1.115606in}}%
\pgfpathlineto{\pgfqpoint{0.654658in}{1.806626in}}%
\pgfpathlineto{\pgfqpoint{0.655713in}{0.975518in}}%
\pgfpathlineto{\pgfqpoint{0.658338in}{1.656328in}}%
\pgfpathlineto{\pgfqpoint{0.662416in}{1.015333in}}%
\pgfpathlineto{\pgfqpoint{0.664982in}{1.686220in}}%
\pgfpathlineto{\pgfqpoint{0.668597in}{1.079152in}}%
\pgfpathlineto{\pgfqpoint{0.671904in}{1.712408in}}%
\pgfpathlineto{\pgfqpoint{0.674933in}{1.096655in}}%
\pgfpathlineto{\pgfqpoint{0.677783in}{1.677530in}}%
\pgfpathlineto{\pgfqpoint{0.681752in}{1.040439in}}%
\pgfpathlineto{\pgfqpoint{0.685367in}{1.773041in}}%
\pgfpathlineto{\pgfqpoint{0.687335in}{0.932432in}}%
\pgfpathlineto{\pgfqpoint{0.691259in}{1.790099in}}%
\pgfpathlineto{\pgfqpoint{0.693986in}{1.092348in}}%
\pgfpathlineto{\pgfqpoint{0.697080in}{1.657695in}}%
\pgfpathlineto{\pgfqpoint{0.702027in}{0.919470in}}%
\pgfpathlineto{\pgfqpoint{0.704967in}{1.802841in}}%
\pgfpathlineto{\pgfqpoint{0.706864in}{1.101068in}}%
\pgfpathlineto{\pgfqpoint{0.710254in}{1.723504in}}%
\pgfpathlineto{\pgfqpoint{0.715812in}{0.978217in}}%
\pgfpathlineto{\pgfqpoint{0.716912in}{1.646696in}}%
\pgfpathlineto{\pgfqpoint{0.720417in}{0.969045in}}%
\pgfpathlineto{\pgfqpoint{0.722778in}{1.684075in}}%
\pgfpathlineto{\pgfqpoint{0.726432in}{1.039160in}}%
\pgfpathlineto{\pgfqpoint{0.730966in}{1.812285in}}%
\pgfpathlineto{\pgfqpoint{0.732440in}{1.002605in}}%
\pgfpathlineto{\pgfqpoint{0.735759in}{1.659799in}}%
\pgfpathlineto{\pgfqpoint{0.742069in}{0.891826in}}%
\pgfpathlineto{\pgfqpoint{0.742230in}{1.712848in}}%
\pgfpathlineto{\pgfqpoint{0.745414in}{1.087560in}}%
\pgfpathlineto{\pgfqpoint{0.750052in}{1.725644in}}%
\pgfpathlineto{\pgfqpoint{0.752020in}{1.087099in}}%
\pgfpathlineto{\pgfqpoint{0.755069in}{1.677292in}}%
\pgfpathlineto{\pgfqpoint{0.758755in}{1.013170in}}%
\pgfpathlineto{\pgfqpoint{0.761636in}{1.701235in}}%
\pgfpathlineto{\pgfqpoint{0.764975in}{1.026044in}}%
\pgfpathlineto{\pgfqpoint{0.769079in}{1.824066in}}%
\pgfpathlineto{\pgfqpoint{0.771234in}{1.091928in}}%
\pgfpathlineto{\pgfqpoint{0.776058in}{1.709336in}}%
\pgfpathlineto{\pgfqpoint{0.777531in}{1.049007in}}%
\pgfpathlineto{\pgfqpoint{0.781307in}{1.838824in}}%
\pgfpathlineto{\pgfqpoint{0.784568in}{1.037171in}}%
\pgfpathlineto{\pgfqpoint{0.787810in}{1.729629in}}%
\pgfpathlineto{\pgfqpoint{0.790486in}{1.028312in}}%
\pgfpathlineto{\pgfqpoint{0.794661in}{1.759188in}}%
\pgfpathlineto{\pgfqpoint{0.797465in}{1.124859in}}%
\pgfpathlineto{\pgfqpoint{0.800669in}{1.707749in}}%
\pgfpathlineto{\pgfqpoint{0.806020in}{0.951056in}}%
\pgfpathlineto{\pgfqpoint{0.806683in}{1.660959in}}%
\pgfpathlineto{\pgfqpoint{0.810780in}{1.029262in}}%
\pgfpathlineto{\pgfqpoint{0.813077in}{1.835055in}}%
\pgfpathlineto{\pgfqpoint{0.818133in}{1.009550in}}%
\pgfpathlineto{\pgfqpoint{0.819947in}{1.698485in}}%
\pgfpathlineto{\pgfqpoint{0.824237in}{1.090256in}}%
\pgfpathlineto{\pgfqpoint{0.825974in}{1.743554in}}%
\pgfpathlineto{\pgfqpoint{0.829267in}{1.007824in}}%
\pgfpathlineto{\pgfqpoint{0.832998in}{1.823798in}}%
\pgfpathlineto{\pgfqpoint{0.836620in}{1.002207in}}%
\pgfpathlineto{\pgfqpoint{0.839572in}{1.759965in}}%
\pgfpathlineto{\pgfqpoint{0.843612in}{1.026064in}}%
\pgfpathlineto{\pgfqpoint{0.845721in}{1.677591in}}%
\pgfpathlineto{\pgfqpoint{0.848751in}{1.036036in}}%
\pgfpathlineto{\pgfqpoint{0.851813in}{1.752759in}}%
\pgfpathlineto{\pgfqpoint{0.857486in}{1.007899in}}%
\pgfpathlineto{\pgfqpoint{0.859590in}{1.758885in}}%
\pgfpathlineto{\pgfqpoint{0.863288in}{1.018189in}}%
\pgfpathlineto{\pgfqpoint{0.864678in}{1.861767in}}%
\pgfpathlineto{\pgfqpoint{0.867868in}{1.004540in}}%
\pgfpathlineto{\pgfqpoint{0.872172in}{1.727079in}}%
\pgfpathlineto{\pgfqpoint{0.877253in}{0.905479in}}%
\pgfpathlineto{\pgfqpoint{0.877704in}{1.682164in}}%
\pgfpathlineto{\pgfqpoint{0.880997in}{1.083885in}}%
\pgfpathlineto{\pgfqpoint{0.885345in}{1.725641in}}%
\pgfpathlineto{\pgfqpoint{0.887372in}{1.121101in}}%
\pgfpathlineto{\pgfqpoint{0.890324in}{1.698718in}}%
\pgfpathlineto{\pgfqpoint{0.893708in}{1.105986in}}%
\pgfpathlineto{\pgfqpoint{0.897599in}{1.692155in}}%
\pgfpathlineto{\pgfqpoint{0.900185in}{1.120828in}}%
\pgfpathlineto{\pgfqpoint{0.904109in}{1.698380in}}%
\pgfpathlineto{\pgfqpoint{0.907910in}{0.987772in}}%
\pgfpathlineto{\pgfqpoint{0.909737in}{1.698417in}}%
\pgfpathlineto{\pgfqpoint{0.913976in}{1.037763in}}%
\pgfpathlineto{\pgfqpoint{0.916099in}{1.632923in}}%
\pgfpathlineto{\pgfqpoint{0.920081in}{1.057637in}}%
\pgfpathlineto{\pgfqpoint{0.923799in}{1.683956in}}%
\pgfpathlineto{\pgfqpoint{0.925896in}{1.003338in}}%
\pgfpathlineto{\pgfqpoint{0.930334in}{1.709928in}}%
\pgfpathlineto{\pgfqpoint{0.932373in}{1.067450in}}%
\pgfpathlineto{\pgfqpoint{0.936413in}{1.721890in}}%
\pgfpathlineto{\pgfqpoint{0.939860in}{1.040447in}}%
\pgfpathlineto{\pgfqpoint{0.943424in}{1.720759in}}%
\pgfpathlineto{\pgfqpoint{0.945116in}{1.059008in}}%
\pgfpathlineto{\pgfqpoint{0.949194in}{1.728121in}}%
\pgfpathlineto{\pgfqpoint{0.952539in}{1.028415in}}%
\pgfpathlineto{\pgfqpoint{0.955035in}{1.693352in}}%
\pgfpathlineto{\pgfqpoint{0.958836in}{1.094524in}}%
\pgfpathlineto{\pgfqpoint{0.961602in}{1.689254in}}%
\pgfpathlineto{\pgfqpoint{0.965455in}{1.128690in}}%
\pgfpathlineto{\pgfqpoint{0.968485in}{1.694511in}}%
\pgfpathlineto{\pgfqpoint{0.971161in}{0.972445in}}%
\pgfpathlineto{\pgfqpoint{0.975567in}{1.694261in}}%
\pgfpathlineto{\pgfqpoint{0.977625in}{0.980983in}}%
\pgfpathlineto{\pgfqpoint{0.980893in}{1.713750in}}%
\pgfpathlineto{\pgfqpoint{0.984347in}{1.026048in}}%
\pgfpathlineto{\pgfqpoint{0.987731in}{1.756009in}}%
\pgfpathlineto{\pgfqpoint{0.990394in}{1.032513in}}%
\pgfpathlineto{\pgfqpoint{0.993957in}{1.747289in}}%
\pgfpathlineto{\pgfqpoint{0.997431in}{0.906413in}}%
\pgfpathlineto{\pgfqpoint{1.001438in}{1.766895in}}%
\pgfpathlineto{\pgfqpoint{1.003169in}{1.125614in}}%
\pgfpathlineto{\pgfqpoint{1.008488in}{1.709350in}}%
\pgfpathlineto{\pgfqpoint{1.009607in}{1.081095in}}%
\pgfpathlineto{\pgfqpoint{1.013017in}{1.730557in}}%
\pgfpathlineto{\pgfqpoint{1.016477in}{1.012690in}}%
\pgfpathlineto{\pgfqpoint{1.019353in}{1.632734in}}%
\pgfpathlineto{\pgfqpoint{1.023971in}{0.895441in}}%
\pgfpathlineto{\pgfqpoint{1.025740in}{1.582666in}}%
\pgfpathlineto{\pgfqpoint{1.030088in}{0.979639in}}%
\pgfpathlineto{\pgfqpoint{1.033523in}{1.737372in}}%
\pgfpathlineto{\pgfqpoint{1.036090in}{1.087960in}}%
\pgfpathlineto{\pgfqpoint{1.039866in}{1.739275in}}%
\pgfpathlineto{\pgfqpoint{1.041828in}{1.079902in}}%
\pgfpathlineto{\pgfqpoint{1.046163in}{1.697898in}}%
\pgfpathlineto{\pgfqpoint{1.048826in}{1.079810in}}%
\pgfpathlineto{\pgfqpoint{1.052428in}{1.750116in}}%
\pgfpathlineto{\pgfqpoint{1.054873in}{1.083934in}}%
\pgfpathlineto{\pgfqpoint{1.059022in}{1.751654in}}%
\pgfpathlineto{\pgfqpoint{1.061151in}{1.098829in}}%
\pgfpathlineto{\pgfqpoint{1.064631in}{1.694007in}}%
\pgfpathlineto{\pgfqpoint{1.068992in}{0.921340in}}%
\pgfpathlineto{\pgfqpoint{1.072594in}{1.764964in}}%
\pgfpathlineto{\pgfqpoint{1.074029in}{0.971943in}}%
\pgfpathlineto{\pgfqpoint{1.077238in}{1.865404in}}%
\pgfpathlineto{\pgfqpoint{1.080937in}{0.947445in}}%
\pgfpathlineto{\pgfqpoint{1.083774in}{1.698554in}}%
\pgfpathlineto{\pgfqpoint{1.088064in}{0.965725in}}%
\pgfpathlineto{\pgfqpoint{1.091293in}{1.714397in}}%
\pgfpathlineto{\pgfqpoint{1.095268in}{0.941174in}}%
\pgfpathlineto{\pgfqpoint{1.097353in}{1.689092in}}%
\pgfpathlineto{\pgfqpoint{1.100209in}{1.076193in}}%
\pgfpathlineto{\pgfqpoint{1.105226in}{1.873081in}}%
\pgfpathlineto{\pgfqpoint{1.106281in}{1.104453in}}%
\pgfpathlineto{\pgfqpoint{1.109555in}{1.704374in}}%
\pgfpathlineto{\pgfqpoint{1.114630in}{1.037519in}}%
\pgfpathlineto{\pgfqpoint{1.117113in}{1.687282in}}%
\pgfpathlineto{\pgfqpoint{1.119474in}{1.044910in}}%
\pgfpathlineto{\pgfqpoint{1.124067in}{1.796341in}}%
\pgfpathlineto{\pgfqpoint{1.127746in}{0.979387in}}%
\pgfpathlineto{\pgfqpoint{1.129836in}{1.684247in}}%
\pgfpathlineto{\pgfqpoint{1.132924in}{1.086616in}}%
\pgfpathlineto{\pgfqpoint{1.137787in}{1.728482in}}%
\pgfpathlineto{\pgfqpoint{1.139369in}{1.069420in}}%
\pgfpathlineto{\pgfqpoint{1.141788in}{1.691017in}}%
\pgfpathlineto{\pgfqpoint{1.145133in}{1.070916in}}%
\pgfpathlineto{\pgfqpoint{1.149404in}{1.694255in}}%
\pgfpathlineto{\pgfqpoint{1.152537in}{1.019894in}}%
\pgfpathlineto{\pgfqpoint{1.154576in}{1.601672in}}%
\pgfpathlineto{\pgfqpoint{1.157940in}{1.056826in}}%
\pgfpathlineto{\pgfqpoint{1.162076in}{1.782408in}}%
\pgfpathlineto{\pgfqpoint{1.164314in}{1.057121in}}%
\pgfpathlineto{\pgfqpoint{1.167589in}{1.697024in}}%
\pgfpathlineto{\pgfqpoint{1.171345in}{1.099707in}}%
\pgfpathlineto{\pgfqpoint{1.174066in}{1.748616in}}%
\pgfpathlineto{\pgfqpoint{1.178504in}{1.016102in}}%
\pgfpathlineto{\pgfqpoint{1.180543in}{1.719853in}}%
\pgfpathlineto{\pgfqpoint{1.184178in}{1.053330in}}%
\pgfpathlineto{\pgfqpoint{1.186834in}{1.777559in}}%
\pgfpathlineto{\pgfqpoint{1.190752in}{1.048605in}}%
\pgfpathlineto{\pgfqpoint{1.194348in}{1.643124in}}%
\pgfpathlineto{\pgfqpoint{1.197345in}{1.081377in}}%
\pgfpathlineto{\pgfqpoint{1.200298in}{1.679423in}}%
\pgfpathlineto{\pgfqpoint{1.203012in}{1.147414in}}%
\pgfpathlineto{\pgfqpoint{1.206357in}{1.631527in}}%
\pgfpathlineto{\pgfqpoint{1.212397in}{0.986601in}}%
\pgfpathlineto{\pgfqpoint{1.212609in}{1.776572in}}%
\pgfpathlineto{\pgfqpoint{1.216063in}{1.082487in}}%
\pgfpathlineto{\pgfqpoint{1.220380in}{1.698489in}}%
\pgfpathlineto{\pgfqpoint{1.224149in}{0.986721in}}%
\pgfpathlineto{\pgfqpoint{1.225886in}{1.698095in}}%
\pgfpathlineto{\pgfqpoint{1.228871in}{1.052301in}}%
\pgfpathlineto{\pgfqpoint{1.232164in}{1.744315in}}%
\pgfpathlineto{\pgfqpoint{1.236596in}{1.007865in}}%
\pgfpathlineto{\pgfqpoint{1.238423in}{1.719021in}}%
\pgfpathlineto{\pgfqpoint{1.242668in}{0.968124in}}%
\pgfpathlineto{\pgfqpoint{1.244984in}{1.613648in}}%
\pgfpathlineto{\pgfqpoint{1.249557in}{0.957862in}}%
\pgfpathlineto{\pgfqpoint{1.251468in}{1.688941in}}%
\pgfpathlineto{\pgfqpoint{1.254652in}{0.987658in}}%
\pgfpathlineto{\pgfqpoint{1.257759in}{1.637700in}}%
\pgfpathlineto{\pgfqpoint{1.262911in}{1.029704in}}%
\pgfpathlineto{\pgfqpoint{1.264185in}{1.612969in}}%
\pgfpathlineto{\pgfqpoint{1.268758in}{1.051192in}}%
\pgfpathlineto{\pgfqpoint{1.270746in}{1.651709in}}%
\pgfpathlineto{\pgfqpoint{1.275023in}{1.019818in}}%
\pgfpathlineto{\pgfqpoint{1.278497in}{1.730637in}}%
\pgfpathlineto{\pgfqpoint{1.280510in}{1.074182in}}%
\pgfpathlineto{\pgfqpoint{1.283842in}{1.726622in}}%
\pgfpathlineto{\pgfqpoint{1.287644in}{1.080721in}}%
\pgfpathlineto{\pgfqpoint{1.291600in}{1.734733in}}%
\pgfpathlineto{\pgfqpoint{1.293394in}{1.073014in}}%
\pgfpathlineto{\pgfqpoint{1.298386in}{1.772280in}}%
\pgfpathlineto{\pgfqpoint{1.299949in}{1.100471in}}%
\pgfpathlineto{\pgfqpoint{1.304497in}{1.717193in}}%
\pgfpathlineto{\pgfqpoint{1.306967in}{0.917906in}}%
\pgfpathlineto{\pgfqpoint{1.310042in}{1.722505in}}%
\pgfpathlineto{\pgfqpoint{1.313547in}{0.934635in}}%
\pgfpathlineto{\pgfqpoint{1.316384in}{1.714465in}}%
\pgfpathlineto{\pgfqpoint{1.320765in}{1.036808in}}%
\pgfpathlineto{\pgfqpoint{1.322418in}{1.693686in}}%
\pgfpathlineto{\pgfqpoint{1.327499in}{0.996929in}}%
\pgfpathlineto{\pgfqpoint{1.329712in}{1.680006in}}%
\pgfpathlineto{\pgfqpoint{1.332324in}{1.092049in}}%
\pgfpathlineto{\pgfqpoint{1.337476in}{1.811466in}}%
\pgfpathlineto{\pgfqpoint{1.338332in}{1.029015in}}%
\pgfpathlineto{\pgfqpoint{1.342050in}{1.615097in}}%
\pgfpathlineto{\pgfqpoint{1.345324in}{1.073357in}}%
\pgfpathlineto{\pgfqpoint{1.348945in}{1.750824in}}%
\pgfpathlineto{\pgfqpoint{1.351608in}{1.150853in}}%
\pgfpathlineto{\pgfqpoint{1.354914in}{1.723853in}}%
\pgfpathlineto{\pgfqpoint{1.357764in}{1.037251in}}%
\pgfpathlineto{\pgfqpoint{1.360935in}{1.676405in}}%
\pgfpathlineto{\pgfqpoint{1.365046in}{1.089913in}}%
\pgfpathlineto{\pgfqpoint{1.367535in}{1.738079in}}%
\pgfpathlineto{\pgfqpoint{1.370764in}{1.068512in}}%
\pgfpathlineto{\pgfqpoint{1.374360in}{1.650361in}}%
\pgfpathlineto{\pgfqpoint{1.377344in}{1.000850in}}%
\pgfpathlineto{\pgfqpoint{1.381030in}{1.820204in}}%
\pgfpathlineto{\pgfqpoint{1.383590in}{1.046812in}}%
\pgfpathlineto{\pgfqpoint{1.386864in}{1.737731in}}%
\pgfpathlineto{\pgfqpoint{1.391760in}{1.018526in}}%
\pgfpathlineto{\pgfqpoint{1.394583in}{1.758213in}}%
\pgfpathlineto{\pgfqpoint{1.397304in}{1.012696in}}%
\pgfpathlineto{\pgfqpoint{1.399575in}{1.651315in}}%
\pgfpathlineto{\pgfqpoint{1.404136in}{0.985194in}}%
\pgfpathlineto{\pgfqpoint{1.406085in}{1.736685in}}%
\pgfpathlineto{\pgfqpoint{1.410549in}{1.044281in}}%
\pgfpathlineto{\pgfqpoint{1.413366in}{1.719072in}}%
\pgfpathlineto{\pgfqpoint{1.415817in}{1.051409in}}%
\pgfpathlineto{\pgfqpoint{1.420622in}{1.817249in}}%
\pgfpathlineto{\pgfqpoint{1.422860in}{1.060688in}}%
\pgfpathlineto{\pgfqpoint{1.425543in}{1.658240in}}%
\pgfpathlineto{\pgfqpoint{1.430232in}{1.043025in}}%
\pgfpathlineto{\pgfqpoint{1.431757in}{1.666235in}}%
\pgfpathlineto{\pgfqpoint{1.434986in}{1.074570in}}%
\pgfpathlineto{\pgfqpoint{1.438626in}{1.694186in}}%
\pgfpathlineto{\pgfqpoint{1.442627in}{1.029084in}}%
\pgfpathlineto{\pgfqpoint{1.445123in}{1.771999in}}%
\pgfpathlineto{\pgfqpoint{1.448101in}{1.012768in}}%
\pgfpathlineto{\pgfqpoint{1.451620in}{1.681373in}}%
\pgfpathlineto{\pgfqpoint{1.455499in}{1.086662in}}%
\pgfpathlineto{\pgfqpoint{1.457782in}{1.666179in}}%
\pgfpathlineto{\pgfqpoint{1.460973in}{0.984866in}}%
\pgfpathlineto{\pgfqpoint{1.466453in}{1.755599in}}%
\pgfpathlineto{\pgfqpoint{1.467688in}{1.043588in}}%
\pgfpathlineto{\pgfqpoint{1.470518in}{1.700088in}}%
\pgfpathlineto{\pgfqpoint{1.473992in}{1.057940in}}%
\pgfpathlineto{\pgfqpoint{1.477002in}{1.680953in}}%
\pgfpathlineto{\pgfqpoint{1.480952in}{0.990789in}}%
\pgfpathlineto{\pgfqpoint{1.483859in}{1.723901in}}%
\pgfpathlineto{\pgfqpoint{1.487056in}{1.079111in}}%
\pgfpathlineto{\pgfqpoint{1.489809in}{1.699458in}}%
\pgfpathlineto{\pgfqpoint{1.494196in}{1.026919in}}%
\pgfpathlineto{\pgfqpoint{1.497142in}{1.750842in}}%
\pgfpathlineto{\pgfqpoint{1.499677in}{1.054516in}}%
\pgfpathlineto{\pgfqpoint{1.502803in}{1.731890in}}%
\pgfpathlineto{\pgfqpoint{1.505871in}{0.997256in}}%
\pgfpathlineto{\pgfqpoint{1.509621in}{1.732776in}}%
\pgfpathlineto{\pgfqpoint{1.513738in}{1.013596in}}%
\pgfpathlineto{\pgfqpoint{1.515784in}{1.687130in}}%
\pgfpathlineto{\pgfqpoint{1.519006in}{1.030255in}}%
\pgfpathlineto{\pgfqpoint{1.522190in}{1.862694in}}%
\pgfpathlineto{\pgfqpoint{1.525355in}{0.991856in}}%
\pgfpathlineto{\pgfqpoint{1.528977in}{1.695608in}}%
\pgfpathlineto{\pgfqpoint{1.533807in}{1.024631in}}%
\pgfpathlineto{\pgfqpoint{1.535911in}{1.658778in}}%
\pgfpathlineto{\pgfqpoint{1.538838in}{1.028650in}}%
\pgfpathlineto{\pgfqpoint{1.541584in}{1.718313in}}%
\pgfpathlineto{\pgfqpoint{1.544646in}{1.056032in}}%
\pgfpathlineto{\pgfqpoint{1.548949in}{1.776379in}}%
\pgfpathlineto{\pgfqpoint{1.551561in}{1.050844in}}%
\pgfpathlineto{\pgfqpoint{1.554423in}{1.702553in}}%
\pgfpathlineto{\pgfqpoint{1.557530in}{1.127453in}}%
\pgfpathlineto{\pgfqpoint{1.560894in}{1.668934in}}%
\pgfpathlineto{\pgfqpoint{1.564625in}{1.103523in}}%
\pgfpathlineto{\pgfqpoint{1.567179in}{1.654155in}}%
\pgfpathlineto{\pgfqpoint{1.570344in}{1.047267in}}%
\pgfpathlineto{\pgfqpoint{1.573746in}{1.704602in}}%
\pgfpathlineto{\pgfqpoint{1.577291in}{1.013436in}}%
\pgfpathlineto{\pgfqpoint{1.580320in}{1.760947in}}%
\pgfpathlineto{\pgfqpoint{1.583344in}{1.144265in}}%
\pgfpathlineto{\pgfqpoint{1.586457in}{1.666995in}}%
\pgfpathlineto{\pgfqpoint{1.590066in}{1.080091in}}%
\pgfpathlineto{\pgfqpoint{1.594420in}{1.755200in}}%
\pgfpathlineto{\pgfqpoint{1.596890in}{1.060880in}}%
\pgfpathlineto{\pgfqpoint{1.600859in}{1.758513in}}%
\pgfpathlineto{\pgfqpoint{1.602680in}{1.078936in}}%
\pgfpathlineto{\pgfqpoint{1.607395in}{1.754808in}}%
\pgfpathlineto{\pgfqpoint{1.609003in}{1.044972in}}%
\pgfpathlineto{\pgfqpoint{1.612952in}{1.699934in}}%
\pgfpathlineto{\pgfqpoint{1.616445in}{1.060455in}}%
\pgfpathlineto{\pgfqpoint{1.619198in}{1.671192in}}%
\pgfpathlineto{\pgfqpoint{1.621906in}{1.067587in}}%
\pgfpathlineto{\pgfqpoint{1.625303in}{1.716161in}}%
\pgfpathlineto{\pgfqpoint{1.630873in}{0.989747in}}%
\pgfpathlineto{\pgfqpoint{1.632179in}{1.728743in}}%
\pgfpathlineto{\pgfqpoint{1.635369in}{0.976480in}}%
\pgfpathlineto{\pgfqpoint{1.638290in}{1.646435in}}%
\pgfpathlineto{\pgfqpoint{1.641197in}{1.005907in}}%
\pgfpathlineto{\pgfqpoint{1.645410in}{1.744916in}}%
\pgfpathlineto{\pgfqpoint{1.648723in}{1.068852in}}%
\pgfpathlineto{\pgfqpoint{1.651264in}{1.757923in}}%
\pgfpathlineto{\pgfqpoint{1.655085in}{1.037903in}}%
\pgfpathlineto{\pgfqpoint{1.657439in}{1.774435in}}%
\pgfpathlineto{\pgfqpoint{1.660900in}{1.066340in}}%
\pgfpathlineto{\pgfqpoint{1.664077in}{1.694483in}}%
\pgfpathlineto{\pgfqpoint{1.667261in}{1.112385in}}%
\pgfpathlineto{\pgfqpoint{1.670240in}{1.667197in}}%
\pgfpathlineto{\pgfqpoint{1.674582in}{0.978437in}}%
\pgfpathlineto{\pgfqpoint{1.678711in}{1.779933in}}%
\pgfpathlineto{\pgfqpoint{1.680313in}{1.079889in}}%
\pgfpathlineto{\pgfqpoint{1.683581in}{1.754525in}}%
\pgfpathlineto{\pgfqpoint{1.686649in}{1.100288in}}%
\pgfpathlineto{\pgfqpoint{1.689724in}{1.645567in}}%
\pgfpathlineto{\pgfqpoint{1.694741in}{0.904573in}}%
\pgfpathlineto{\pgfqpoint{1.695989in}{1.672558in}}%
\pgfpathlineto{\pgfqpoint{1.700125in}{1.055348in}}%
\pgfpathlineto{\pgfqpoint{1.702904in}{1.731354in}}%
\pgfpathlineto{\pgfqpoint{1.707837in}{0.954341in}}%
\pgfpathlineto{\pgfqpoint{1.709388in}{1.708244in}}%
\pgfpathlineto{\pgfqpoint{1.713131in}{1.072819in}}%
\pgfpathlineto{\pgfqpoint{1.715634in}{1.706967in}}%
\pgfpathlineto{\pgfqpoint{1.719146in}{1.016052in}}%
\pgfpathlineto{\pgfqpoint{1.722021in}{1.647776in}}%
\pgfpathlineto{\pgfqpoint{1.727842in}{0.964247in}}%
\pgfpathlineto{\pgfqpoint{1.729148in}{1.814330in}}%
\pgfpathlineto{\pgfqpoint{1.732094in}{1.098019in}}%
\pgfpathlineto{\pgfqpoint{1.735394in}{1.770304in}}%
\pgfpathlineto{\pgfqpoint{1.738205in}{0.982454in}}%
\pgfpathlineto{\pgfqpoint{1.741994in}{1.717171in}}%
\pgfpathlineto{\pgfqpoint{1.745634in}{1.038337in}}%
\pgfpathlineto{\pgfqpoint{1.748060in}{1.694708in}}%
\pgfpathlineto{\pgfqpoint{1.751089in}{1.008748in}}%
\pgfpathlineto{\pgfqpoint{1.754235in}{1.688121in}}%
\pgfpathlineto{\pgfqpoint{1.758705in}{1.010516in}}%
\pgfpathlineto{\pgfqpoint{1.760564in}{1.707939in}}%
\pgfpathlineto{\pgfqpoint{1.766386in}{0.982634in}}%
\pgfpathlineto{\pgfqpoint{1.767293in}{1.643282in}}%
\pgfpathlineto{\pgfqpoint{1.771326in}{1.001775in}}%
\pgfpathlineto{\pgfqpoint{1.774420in}{1.691661in}}%
\pgfpathlineto{\pgfqpoint{1.777681in}{0.993386in}}%
\pgfpathlineto{\pgfqpoint{1.780035in}{1.694648in}}%
\pgfpathlineto{\pgfqpoint{1.784506in}{0.980769in}}%
\pgfpathlineto{\pgfqpoint{1.786551in}{1.636501in}}%
\pgfpathlineto{\pgfqpoint{1.790694in}{1.023071in}}%
\pgfpathlineto{\pgfqpoint{1.793022in}{1.684629in}}%
\pgfpathlineto{\pgfqpoint{1.796509in}{1.040947in}}%
\pgfpathlineto{\pgfqpoint{1.799397in}{1.746873in}}%
\pgfpathlineto{\pgfqpoint{1.802491in}{1.050896in}}%
\pgfpathlineto{\pgfqpoint{1.805823in}{1.875203in}}%
\pgfpathlineto{\pgfqpoint{1.809534in}{1.020329in}}%
\pgfpathlineto{\pgfqpoint{1.813420in}{1.727086in}}%
\pgfpathlineto{\pgfqpoint{1.815221in}{1.134247in}}%
\pgfpathlineto{\pgfqpoint{1.818823in}{1.783109in}}%
\pgfpathlineto{\pgfqpoint{1.822702in}{1.025516in}}%
\pgfpathlineto{\pgfqpoint{1.825847in}{1.868272in}}%
\pgfpathlineto{\pgfqpoint{1.828890in}{1.027466in}}%
\pgfpathlineto{\pgfqpoint{1.831373in}{1.761167in}}%
\pgfpathlineto{\pgfqpoint{1.835329in}{1.083258in}}%
\pgfpathlineto{\pgfqpoint{1.837863in}{1.706374in}}%
\pgfpathlineto{\pgfqpoint{1.841414in}{0.965669in}}%
\pgfpathlineto{\pgfqpoint{1.844990in}{1.707238in}}%
\pgfpathlineto{\pgfqpoint{1.848959in}{1.051305in}}%
\pgfpathlineto{\pgfqpoint{1.851474in}{1.706287in}}%
\pgfpathlineto{\pgfqpoint{1.854099in}{1.094402in}}%
\pgfpathlineto{\pgfqpoint{1.857649in}{1.680098in}}%
\pgfpathlineto{\pgfqpoint{1.861148in}{0.997127in}}%
\pgfpathlineto{\pgfqpoint{1.863567in}{1.680462in}}%
\pgfpathlineto{\pgfqpoint{1.867163in}{1.055860in}}%
\pgfpathlineto{\pgfqpoint{1.870077in}{1.741682in}}%
\pgfpathlineto{\pgfqpoint{1.873434in}{0.964419in}}%
\pgfpathlineto{\pgfqpoint{1.877493in}{1.717346in}}%
\pgfpathlineto{\pgfqpoint{1.880298in}{0.900514in}}%
\pgfpathlineto{\pgfqpoint{1.883122in}{1.661398in}}%
\pgfpathlineto{\pgfqpoint{1.886788in}{0.994794in}}%
\pgfpathlineto{\pgfqpoint{1.889702in}{1.626603in}}%
\pgfpathlineto{\pgfqpoint{1.892815in}{1.125985in}}%
\pgfpathlineto{\pgfqpoint{1.896598in}{1.707556in}}%
\pgfpathlineto{\pgfqpoint{1.899113in}{1.055031in}}%
\pgfpathlineto{\pgfqpoint{1.903448in}{1.715055in}}%
\pgfpathlineto{\pgfqpoint{1.905944in}{1.081278in}}%
\pgfpathlineto{\pgfqpoint{1.908909in}{1.628979in}}%
\pgfpathlineto{\pgfqpoint{1.912499in}{1.077481in}}%
\pgfpathlineto{\pgfqpoint{1.915284in}{1.671403in}}%
\pgfpathlineto{\pgfqpoint{1.919761in}{1.068185in}}%
\pgfpathlineto{\pgfqpoint{1.922077in}{1.670612in}}%
\pgfpathlineto{\pgfqpoint{1.926927in}{0.913578in}}%
\pgfpathlineto{\pgfqpoint{1.928618in}{1.793778in}}%
\pgfpathlineto{\pgfqpoint{1.932819in}{1.022210in}}%
\pgfpathlineto{\pgfqpoint{1.934948in}{1.633031in}}%
\pgfpathlineto{\pgfqpoint{1.937759in}{1.049061in}}%
\pgfpathlineto{\pgfqpoint{1.943246in}{1.820656in}}%
\pgfpathlineto{\pgfqpoint{1.945857in}{0.959979in}}%
\pgfpathlineto{\pgfqpoint{1.947858in}{1.682445in}}%
\pgfpathlineto{\pgfqpoint{1.950849in}{1.100444in}}%
\pgfpathlineto{\pgfqpoint{1.953943in}{1.638548in}}%
\pgfpathlineto{\pgfqpoint{1.957719in}{0.960335in}}%
\pgfpathlineto{\pgfqpoint{1.960511in}{1.947774in}}%
\pgfpathlineto{\pgfqpoint{1.964647in}{1.039808in}}%
\pgfpathlineto{\pgfqpoint{1.966686in}{1.678991in}}%
\pgfpathlineto{\pgfqpoint{1.970011in}{1.016091in}}%
\pgfpathlineto{\pgfqpoint{1.973530in}{1.811810in}}%
\pgfpathlineto{\pgfqpoint{1.977203in}{1.028244in}}%
\pgfpathlineto{\pgfqpoint{1.979789in}{1.677622in}}%
\pgfpathlineto{\pgfqpoint{1.983732in}{1.085845in}}%
\pgfpathlineto{\pgfqpoint{1.986601in}{1.730023in}}%
\pgfpathlineto{\pgfqpoint{1.991065in}{0.973951in}}%
\pgfpathlineto{\pgfqpoint{1.992827in}{1.685812in}}%
\pgfpathlineto{\pgfqpoint{1.996545in}{1.102913in}}%
\pgfpathlineto{\pgfqpoint{1.999768in}{1.732266in}}%
\pgfpathlineto{\pgfqpoint{2.002791in}{1.056217in}}%
\pgfpathlineto{\pgfqpoint{2.006290in}{1.723855in}}%
\pgfpathlineto{\pgfqpoint{2.008651in}{1.015265in}}%
\pgfpathlineto{\pgfqpoint{2.012639in}{1.774132in}}%
\pgfpathlineto{\pgfqpoint{2.015122in}{1.022672in}}%
\pgfpathlineto{\pgfqpoint{2.018428in}{1.670106in}}%
\pgfpathlineto{\pgfqpoint{2.022236in}{0.997439in}}%
\pgfpathlineto{\pgfqpoint{2.026823in}{1.754252in}}%
\pgfpathlineto{\pgfqpoint{2.028000in}{1.126671in}}%
\pgfpathlineto{\pgfqpoint{2.032155in}{1.673254in}}%
\pgfpathlineto{\pgfqpoint{2.035500in}{1.062531in}}%
\pgfpathlineto{\pgfqpoint{2.037751in}{1.678140in}}%
\pgfpathlineto{\pgfqpoint{2.040839in}{1.081894in}}%
\pgfpathlineto{\pgfqpoint{2.044396in}{1.788377in}}%
\pgfpathlineto{\pgfqpoint{2.047638in}{1.080240in}}%
\pgfpathlineto{\pgfqpoint{2.050591in}{1.646668in}}%
\pgfpathlineto{\pgfqpoint{2.054309in}{0.988560in}}%
\pgfpathlineto{\pgfqpoint{2.057473in}{1.771910in}}%
\pgfpathlineto{\pgfqpoint{2.061281in}{1.041154in}}%
\pgfpathlineto{\pgfqpoint{2.063861in}{1.666433in}}%
\pgfpathlineto{\pgfqpoint{2.069348in}{0.980817in}}%
\pgfpathlineto{\pgfqpoint{2.069811in}{1.623705in}}%
\pgfpathlineto{\pgfqpoint{2.073464in}{1.026053in}}%
\pgfpathlineto{\pgfqpoint{2.076301in}{1.616043in}}%
\pgfpathlineto{\pgfqpoint{2.079685in}{1.085432in}}%
\pgfpathlineto{\pgfqpoint{2.083235in}{1.730784in}}%
\pgfpathlineto{\pgfqpoint{2.086156in}{1.110115in}}%
\pgfpathlineto{\pgfqpoint{2.090009in}{1.739015in}}%
\pgfpathlineto{\pgfqpoint{2.092427in}{1.115791in}}%
\pgfpathlineto{\pgfqpoint{2.096750in}{1.767449in}}%
\pgfpathlineto{\pgfqpoint{2.098892in}{1.063212in}}%
\pgfpathlineto{\pgfqpoint{2.102874in}{1.742336in}}%
\pgfpathlineto{\pgfqpoint{2.105672in}{1.098602in}}%
\pgfpathlineto{\pgfqpoint{2.108528in}{1.653116in}}%
\pgfpathlineto{\pgfqpoint{2.112696in}{1.032823in}}%
\pgfpathlineto{\pgfqpoint{2.115095in}{1.626489in}}%
\pgfpathlineto{\pgfqpoint{2.118582in}{1.090799in}}%
\pgfpathlineto{\pgfqpoint{2.122267in}{1.837597in}}%
\pgfpathlineto{\pgfqpoint{2.124802in}{1.076921in}}%
\pgfpathlineto{\pgfqpoint{2.128147in}{1.731559in}}%
\pgfpathlineto{\pgfqpoint{2.131588in}{1.060040in}}%
\pgfpathlineto{\pgfqpoint{2.134406in}{1.658820in}}%
\pgfpathlineto{\pgfqpoint{2.138690in}{1.019800in}}%
\pgfpathlineto{\pgfqpoint{2.142420in}{1.732543in}}%
\pgfpathlineto{\pgfqpoint{2.145032in}{1.065351in}}%
\pgfpathlineto{\pgfqpoint{2.147322in}{1.684383in}}%
\pgfpathlineto{\pgfqpoint{2.152481in}{0.872943in}}%
\pgfpathlineto{\pgfqpoint{2.153896in}{1.680875in}}%
\pgfpathlineto{\pgfqpoint{2.158714in}{0.994883in}}%
\pgfpathlineto{\pgfqpoint{2.160586in}{1.728221in}}%
\pgfpathlineto{\pgfqpoint{2.163918in}{0.972175in}}%
\pgfpathlineto{\pgfqpoint{2.166510in}{1.660251in}}%
\pgfpathlineto{\pgfqpoint{2.170228in}{1.100548in}}%
\pgfpathlineto{\pgfqpoint{2.173399in}{1.685156in}}%
\pgfpathlineto{\pgfqpoint{2.177394in}{0.955533in}}%
\pgfpathlineto{\pgfqpoint{2.180732in}{1.733711in}}%
\pgfpathlineto{\pgfqpoint{2.182823in}{1.060375in}}%
\pgfpathlineto{\pgfqpoint{2.186097in}{1.682178in}}%
\pgfpathlineto{\pgfqpoint{2.189107in}{1.076165in}}%
\pgfpathlineto{\pgfqpoint{2.192799in}{1.668635in}}%
\pgfpathlineto{\pgfqpoint{2.195713in}{1.129483in}}%
\pgfpathlineto{\pgfqpoint{2.201618in}{1.741373in}}%
\pgfpathlineto{\pgfqpoint{2.202146in}{1.045370in}}%
\pgfpathlineto{\pgfqpoint{2.205259in}{1.650958in}}%
\pgfpathlineto{\pgfqpoint{2.210932in}{0.916212in}}%
\pgfpathlineto{\pgfqpoint{2.212502in}{1.703987in}}%
\pgfpathlineto{\pgfqpoint{2.215068in}{1.036087in}}%
\pgfpathlineto{\pgfqpoint{2.219989in}{1.802478in}}%
\pgfpathlineto{\pgfqpoint{2.221308in}{1.104401in}}%
\pgfpathlineto{\pgfqpoint{2.226100in}{1.767899in}}%
\pgfpathlineto{\pgfqpoint{2.227991in}{1.053653in}}%
\pgfpathlineto{\pgfqpoint{2.231548in}{1.707349in}}%
\pgfpathlineto{\pgfqpoint{2.234848in}{1.064228in}}%
\pgfpathlineto{\pgfqpoint{2.237473in}{1.717969in}}%
\pgfpathlineto{\pgfqpoint{2.242683in}{1.019270in}}%
\pgfpathlineto{\pgfqpoint{2.244265in}{1.727336in}}%
\pgfpathlineto{\pgfqpoint{2.247289in}{1.056474in}}%
\pgfpathlineto{\pgfqpoint{2.250724in}{1.760882in}}%
\pgfpathlineto{\pgfqpoint{2.253528in}{1.047621in}}%
\pgfpathlineto{\pgfqpoint{2.256725in}{1.630592in}}%
\pgfpathlineto{\pgfqpoint{2.260424in}{1.014447in}}%
\pgfpathlineto{\pgfqpoint{2.263884in}{1.761710in}}%
\pgfpathlineto{\pgfqpoint{2.266965in}{1.113270in}}%
\pgfpathlineto{\pgfqpoint{2.271661in}{1.734161in}}%
\pgfpathlineto{\pgfqpoint{2.273662in}{1.032666in}}%
\pgfpathlineto{\pgfqpoint{2.276563in}{1.737323in}}%
\pgfpathlineto{\pgfqpoint{2.279580in}{1.103879in}}%
\pgfpathlineto{\pgfqpoint{2.283574in}{1.736643in}}%
\pgfpathlineto{\pgfqpoint{2.285780in}{1.013346in}}%
\pgfpathlineto{\pgfqpoint{2.289460in}{1.697539in}}%
\pgfpathlineto{\pgfqpoint{2.294181in}{0.973262in}}%
\pgfpathlineto{\pgfqpoint{2.296085in}{1.685866in}}%
\pgfpathlineto{\pgfqpoint{2.299771in}{0.970853in}}%
\pgfpathlineto{\pgfqpoint{2.301990in}{1.754538in}}%
\pgfpathlineto{\pgfqpoint{2.305187in}{0.997491in}}%
\pgfpathlineto{\pgfqpoint{2.310211in}{1.803744in}}%
\pgfpathlineto{\pgfqpoint{2.312462in}{1.049681in}}%
\pgfpathlineto{\pgfqpoint{2.314971in}{1.637390in}}%
\pgfpathlineto{\pgfqpoint{2.318888in}{1.092223in}}%
\pgfpathlineto{\pgfqpoint{2.322793in}{1.765593in}}%
\pgfpathlineto{\pgfqpoint{2.325070in}{1.081224in}}%
\pgfpathlineto{\pgfqpoint{2.327675in}{1.685935in}}%
\pgfpathlineto{\pgfqpoint{2.331296in}{1.104301in}}%
\pgfpathlineto{\pgfqpoint{2.334879in}{1.680239in}}%
\pgfpathlineto{\pgfqpoint{2.337587in}{1.005403in}}%
\pgfpathlineto{\pgfqpoint{2.340598in}{1.657421in}}%
\pgfpathlineto{\pgfqpoint{2.344071in}{1.051064in}}%
\pgfpathlineto{\pgfqpoint{2.347551in}{1.756920in}}%
\pgfpathlineto{\pgfqpoint{2.350491in}{1.027486in}}%
\pgfpathlineto{\pgfqpoint{2.354164in}{1.720583in}}%
\pgfpathlineto{\pgfqpoint{2.356775in}{1.051428in}}%
\pgfpathlineto{\pgfqpoint{2.360088in}{1.849528in}}%
\pgfpathlineto{\pgfqpoint{2.365453in}{0.953773in}}%
\pgfpathlineto{\pgfqpoint{2.366617in}{1.726026in}}%
\pgfpathlineto{\pgfqpoint{2.369640in}{1.092284in}}%
\pgfpathlineto{\pgfqpoint{2.372947in}{1.666959in}}%
\pgfpathlineto{\pgfqpoint{2.376465in}{0.976550in}}%
\pgfpathlineto{\pgfqpoint{2.380099in}{1.704649in}}%
\pgfpathlineto{\pgfqpoint{2.382647in}{1.052461in}}%
\pgfpathlineto{\pgfqpoint{2.386667in}{1.663344in}}%
\pgfpathlineto{\pgfqpoint{2.389446in}{1.105309in}}%
\pgfpathlineto{\pgfqpoint{2.392379in}{1.650447in}}%
\pgfpathlineto{\pgfqpoint{2.396026in}{0.979072in}}%
\pgfpathlineto{\pgfqpoint{2.398747in}{1.686516in}}%
\pgfpathlineto{\pgfqpoint{2.402478in}{1.064779in}}%
\pgfpathlineto{\pgfqpoint{2.405604in}{1.757573in}}%
\pgfpathlineto{\pgfqpoint{2.408402in}{1.066478in}}%
\pgfpathlineto{\pgfqpoint{2.411709in}{1.696068in}}%
\pgfpathlineto{\pgfqpoint{2.415201in}{0.998299in}}%
\pgfpathlineto{\pgfqpoint{2.418302in}{1.674986in}}%
\pgfpathlineto{\pgfqpoint{2.421653in}{1.115589in}}%
\pgfpathlineto{\pgfqpoint{2.425152in}{1.690816in}}%
\pgfpathlineto{\pgfqpoint{2.427616in}{1.014172in}}%
\pgfpathlineto{\pgfqpoint{2.432781in}{1.830374in}}%
\pgfpathlineto{\pgfqpoint{2.434222in}{1.112868in}}%
\pgfpathlineto{\pgfqpoint{2.438371in}{1.748726in}}%
\pgfpathlineto{\pgfqpoint{2.440790in}{1.056741in}}%
\pgfpathlineto{\pgfqpoint{2.445222in}{1.825243in}}%
\pgfpathlineto{\pgfqpoint{2.447003in}{1.118029in}}%
\pgfpathlineto{\pgfqpoint{2.452233in}{1.676063in}}%
\pgfpathlineto{\pgfqpoint{2.454047in}{0.963376in}}%
\pgfpathlineto{\pgfqpoint{2.456678in}{1.647299in}}%
\pgfpathlineto{\pgfqpoint{2.460383in}{1.062019in}}%
\pgfpathlineto{\pgfqpoint{2.464236in}{1.751301in}}%
\pgfpathlineto{\pgfqpoint{2.466307in}{1.073578in}}%
\pgfpathlineto{\pgfqpoint{2.471903in}{1.727528in}}%
\pgfpathlineto{\pgfqpoint{2.473203in}{1.077234in}}%
\pgfpathlineto{\pgfqpoint{2.476040in}{1.669492in}}%
\pgfpathlineto{\pgfqpoint{2.481571in}{0.941586in}}%
\pgfpathlineto{\pgfqpoint{2.482716in}{1.687168in}}%
\pgfpathlineto{\pgfqpoint{2.486100in}{1.038278in}}%
\pgfpathlineto{\pgfqpoint{2.489149in}{1.716565in}}%
\pgfpathlineto{\pgfqpoint{2.492166in}{1.057482in}}%
\pgfpathlineto{\pgfqpoint{2.495253in}{1.738459in}}%
\pgfpathlineto{\pgfqpoint{2.498830in}{1.075619in}}%
\pgfpathlineto{\pgfqpoint{2.503159in}{1.707566in}}%
\pgfpathlineto{\pgfqpoint{2.507932in}{0.932652in}}%
\pgfpathlineto{\pgfqpoint{2.508176in}{1.702261in}}%
\pgfpathlineto{\pgfqpoint{2.513856in}{1.026501in}}%
\pgfpathlineto{\pgfqpoint{2.516339in}{1.756979in}}%
\pgfpathlineto{\pgfqpoint{2.518114in}{1.093921in}}%
\pgfpathlineto{\pgfqpoint{2.521761in}{1.762727in}}%
\pgfpathlineto{\pgfqpoint{2.524373in}{1.064609in}}%
\pgfpathlineto{\pgfqpoint{2.528065in}{1.734702in}}%
\pgfpathlineto{\pgfqpoint{2.532999in}{0.995076in}}%
\pgfpathlineto{\pgfqpoint{2.534002in}{1.665870in}}%
\pgfpathlineto{\pgfqpoint{2.537791in}{1.051706in}}%
\pgfpathlineto{\pgfqpoint{2.541548in}{1.664855in}}%
\pgfpathlineto{\pgfqpoint{2.544236in}{1.090172in}}%
\pgfpathlineto{\pgfqpoint{2.547054in}{1.709043in}}%
\pgfpathlineto{\pgfqpoint{2.550289in}{1.022711in}}%
\pgfpathlineto{\pgfqpoint{2.553403in}{1.675269in}}%
\pgfpathlineto{\pgfqpoint{2.556452in}{1.025900in}}%
\pgfpathlineto{\pgfqpoint{2.560420in}{1.698929in}}%
\pgfpathlineto{\pgfqpoint{2.563797in}{1.077088in}}%
\pgfpathlineto{\pgfqpoint{2.566924in}{1.695170in}}%
\pgfpathlineto{\pgfqpoint{2.569426in}{1.029593in}}%
\pgfpathlineto{\pgfqpoint{2.573105in}{1.746534in}}%
\pgfpathlineto{\pgfqpoint{2.576321in}{1.078539in}}%
\pgfpathlineto{\pgfqpoint{2.580734in}{1.732503in}}%
\pgfpathlineto{\pgfqpoint{2.583442in}{0.933559in}}%
\pgfpathlineto{\pgfqpoint{2.585449in}{1.595901in}}%
\pgfpathlineto{\pgfqpoint{2.589933in}{1.021707in}}%
\pgfpathlineto{\pgfqpoint{2.593033in}{1.722210in}}%
\pgfpathlineto{\pgfqpoint{2.597092in}{1.024379in}}%
\pgfpathlineto{\pgfqpoint{2.598423in}{1.639352in}}%
\pgfpathlineto{\pgfqpoint{2.601575in}{1.076573in}}%
\pgfpathlineto{\pgfqpoint{2.605229in}{1.790035in}}%
\pgfpathlineto{\pgfqpoint{2.609030in}{1.026846in}}%
\pgfpathlineto{\pgfqpoint{2.611346in}{1.656296in}}%
\pgfpathlineto{\pgfqpoint{2.615694in}{1.006017in}}%
\pgfpathlineto{\pgfqpoint{2.617669in}{1.662642in}}%
\pgfpathlineto{\pgfqpoint{2.620963in}{1.008933in}}%
\pgfpathlineto{\pgfqpoint{2.624932in}{1.697785in}}%
\pgfpathlineto{\pgfqpoint{2.627550in}{1.001115in}}%
\pgfpathlineto{\pgfqpoint{2.632843in}{1.824192in}}%
\pgfpathlineto{\pgfqpoint{2.634117in}{1.023567in}}%
\pgfpathlineto{\pgfqpoint{2.637636in}{1.671156in}}%
\pgfpathlineto{\pgfqpoint{2.641225in}{1.041392in}}%
\pgfpathlineto{\pgfqpoint{2.643592in}{1.699752in}}%
\pgfpathlineto{\pgfqpoint{2.647690in}{0.959744in}}%
\pgfpathlineto{\pgfqpoint{2.649947in}{1.697753in}}%
\pgfpathlineto{\pgfqpoint{2.653742in}{1.066607in}}%
\pgfpathlineto{\pgfqpoint{2.657538in}{1.774060in}}%
\pgfpathlineto{\pgfqpoint{2.659570in}{1.095146in}}%
\pgfpathlineto{\pgfqpoint{2.663372in}{1.836417in}}%
\pgfpathlineto{\pgfqpoint{2.667540in}{1.021507in}}%
\pgfpathlineto{\pgfqpoint{2.669322in}{1.638758in}}%
\pgfpathlineto{\pgfqpoint{2.674609in}{0.984548in}}%
\pgfpathlineto{\pgfqpoint{2.675838in}{1.653392in}}%
\pgfpathlineto{\pgfqpoint{2.680064in}{1.036074in}}%
\pgfpathlineto{\pgfqpoint{2.682418in}{1.667701in}}%
\pgfpathlineto{\pgfqpoint{2.685699in}{1.001607in}}%
\pgfpathlineto{\pgfqpoint{2.689031in}{1.751981in}}%
\pgfpathlineto{\pgfqpoint{2.693977in}{0.863521in}}%
\pgfpathlineto{\pgfqpoint{2.695232in}{1.726652in}}%
\pgfpathlineto{\pgfqpoint{2.698866in}{1.002623in}}%
\pgfpathlineto{\pgfqpoint{2.701838in}{1.735450in}}%
\pgfpathlineto{\pgfqpoint{2.706894in}{0.975634in}}%
\pgfpathlineto{\pgfqpoint{2.708663in}{1.701857in}}%
\pgfpathlineto{\pgfqpoint{2.712419in}{1.032729in}}%
\pgfpathlineto{\pgfqpoint{2.716935in}{1.790945in}}%
\pgfpathlineto{\pgfqpoint{2.718318in}{1.021533in}}%
\pgfpathlineto{\pgfqpoint{2.721405in}{1.689192in}}%
\pgfpathlineto{\pgfqpoint{2.726461in}{1.049866in}}%
\pgfpathlineto{\pgfqpoint{2.727343in}{1.756016in}}%
\pgfpathlineto{\pgfqpoint{2.731845in}{1.041449in}}%
\pgfpathlineto{\pgfqpoint{2.733679in}{1.800535in}}%
\pgfpathlineto{\pgfqpoint{2.737853in}{0.960786in}}%
\pgfpathlineto{\pgfqpoint{2.740439in}{1.741966in}}%
\pgfpathlineto{\pgfqpoint{2.743527in}{1.108610in}}%
\pgfpathlineto{\pgfqpoint{2.747984in}{1.716518in}}%
\pgfpathlineto{\pgfqpoint{2.749824in}{1.105360in}}%
\pgfpathlineto{\pgfqpoint{2.754076in}{1.669388in}}%
\pgfpathlineto{\pgfqpoint{2.757125in}{1.030016in}}%
\pgfpathlineto{\pgfqpoint{2.759672in}{1.687574in}}%
\pgfpathlineto{\pgfqpoint{2.762946in}{1.013724in}}%
\pgfpathlineto{\pgfqpoint{2.766375in}{1.823340in}}%
\pgfpathlineto{\pgfqpoint{2.769411in}{1.093776in}}%
\pgfpathlineto{\pgfqpoint{2.772801in}{1.809590in}}%
\pgfpathlineto{\pgfqpoint{2.776782in}{1.034428in}}%
\pgfpathlineto{\pgfqpoint{2.779979in}{1.712715in}}%
\pgfpathlineto{\pgfqpoint{2.782366in}{1.093019in}}%
\pgfpathlineto{\pgfqpoint{2.786142in}{1.667318in}}%
\pgfpathlineto{\pgfqpoint{2.788978in}{1.071478in}}%
\pgfpathlineto{\pgfqpoint{2.791847in}{1.765720in}}%
\pgfpathlineto{\pgfqpoint{2.796266in}{1.014833in}}%
\pgfpathlineto{\pgfqpoint{2.798730in}{1.753926in}}%
\pgfpathlineto{\pgfqpoint{2.801670in}{1.034629in}}%
\pgfpathlineto{\pgfqpoint{2.805111in}{1.684145in}}%
\pgfpathlineto{\pgfqpoint{2.807980in}{0.968022in}}%
\pgfpathlineto{\pgfqpoint{2.811601in}{1.889899in}}%
\pgfpathlineto{\pgfqpoint{2.815815in}{1.037236in}}%
\pgfpathlineto{\pgfqpoint{2.817963in}{1.724750in}}%
\pgfpathlineto{\pgfqpoint{2.820780in}{1.058328in}}%
\pgfpathlineto{\pgfqpoint{2.824177in}{1.676183in}}%
\pgfpathlineto{\pgfqpoint{2.827650in}{1.035453in}}%
\pgfpathlineto{\pgfqpoint{2.831342in}{1.664177in}}%
\pgfpathlineto{\pgfqpoint{2.833613in}{1.099268in}}%
\pgfpathlineto{\pgfqpoint{2.837022in}{1.744259in}}%
\pgfpathlineto{\pgfqpoint{2.841171in}{0.855418in}}%
\pgfpathlineto{\pgfqpoint{2.843294in}{1.611215in}}%
\pgfpathlineto{\pgfqpoint{2.847462in}{0.956037in}}%
\pgfpathlineto{\pgfqpoint{2.850704in}{1.714463in}}%
\pgfpathlineto{\pgfqpoint{2.853657in}{1.061238in}}%
\pgfpathlineto{\pgfqpoint{2.856345in}{1.747986in}}%
\pgfpathlineto{\pgfqpoint{2.859581in}{1.090105in}}%
\pgfpathlineto{\pgfqpoint{2.863846in}{1.784938in}}%
\pgfpathlineto{\pgfqpoint{2.866354in}{1.003894in}}%
\pgfpathlineto{\pgfqpoint{2.870986in}{1.770678in}}%
\pgfpathlineto{\pgfqpoint{2.873070in}{1.054058in}}%
\pgfpathlineto{\pgfqpoint{2.876350in}{1.718228in}}%
\pgfpathlineto{\pgfqpoint{2.878904in}{1.044586in}}%
\pgfpathlineto{\pgfqpoint{2.883149in}{1.772013in}}%
\pgfpathlineto{\pgfqpoint{2.886121in}{1.050942in}}%
\pgfpathlineto{\pgfqpoint{2.889511in}{1.782941in}}%
\pgfpathlineto{\pgfqpoint{2.894496in}{0.924324in}}%
\pgfpathlineto{\pgfqpoint{2.895307in}{1.582453in}}%
\pgfpathlineto{\pgfqpoint{2.899115in}{1.009359in}}%
\pgfpathlineto{\pgfqpoint{2.901276in}{1.655236in}}%
\pgfpathlineto{\pgfqpoint{2.905599in}{0.989713in}}%
\pgfpathlineto{\pgfqpoint{2.907889in}{1.707504in}}%
\pgfpathlineto{\pgfqpoint{2.911517in}{1.059486in}}%
\pgfpathlineto{\pgfqpoint{2.914668in}{1.673703in}}%
\pgfpathlineto{\pgfqpoint{2.917544in}{1.071388in}}%
\pgfpathlineto{\pgfqpoint{2.920998in}{1.781848in}}%
\pgfpathlineto{\pgfqpoint{2.925475in}{1.006602in}}%
\pgfpathlineto{\pgfqpoint{2.927906in}{1.711819in}}%
\pgfpathlineto{\pgfqpoint{2.930415in}{1.128479in}}%
\pgfpathlineto{\pgfqpoint{2.934577in}{1.722809in}}%
\pgfpathlineto{\pgfqpoint{2.937459in}{1.007735in}}%
\pgfpathlineto{\pgfqpoint{2.940411in}{1.730940in}}%
\pgfpathlineto{\pgfqpoint{2.943434in}{1.121114in}}%
\pgfpathlineto{\pgfqpoint{2.946471in}{1.633855in}}%
\pgfpathlineto{\pgfqpoint{2.950568in}{1.062151in}}%
\pgfpathlineto{\pgfqpoint{2.953025in}{1.784029in}}%
\pgfpathlineto{\pgfqpoint{2.957335in}{1.070535in}}%
\pgfpathlineto{\pgfqpoint{2.959432in}{1.655773in}}%
\pgfpathlineto{\pgfqpoint{2.963291in}{1.053001in}}%
\pgfpathlineto{\pgfqpoint{2.968367in}{1.783011in}}%
\pgfpathlineto{\pgfqpoint{2.969602in}{1.049625in}}%
\pgfpathlineto{\pgfqpoint{2.972593in}{1.757808in}}%
\pgfpathlineto{\pgfqpoint{2.975507in}{1.072149in}}%
\pgfpathlineto{\pgfqpoint{2.979675in}{1.743777in}}%
\pgfpathlineto{\pgfqpoint{2.982029in}{1.085018in}}%
\pgfpathlineto{\pgfqpoint{2.985348in}{1.656212in}}%
\pgfpathlineto{\pgfqpoint{2.989626in}{1.067700in}}%
\pgfpathlineto{\pgfqpoint{2.991671in}{1.802546in}}%
\pgfpathlineto{\pgfqpoint{2.995183in}{1.039862in}}%
\pgfpathlineto{\pgfqpoint{2.997975in}{1.682744in}}%
\pgfpathlineto{\pgfqpoint{3.002857in}{1.082314in}}%
\pgfpathlineto{\pgfqpoint{3.005302in}{1.700011in}}%
\pgfpathlineto{\pgfqpoint{3.007907in}{1.076142in}}%
\pgfpathlineto{\pgfqpoint{3.011355in}{1.711187in}}%
\pgfpathlineto{\pgfqpoint{3.014359in}{1.043097in}}%
\pgfpathlineto{\pgfqpoint{3.018514in}{1.760539in}}%
\pgfpathlineto{\pgfqpoint{3.020836in}{1.011372in}}%
\pgfpathlineto{\pgfqpoint{3.023962in}{1.810936in}}%
\pgfpathlineto{\pgfqpoint{3.027770in}{1.044660in}}%
\pgfpathlineto{\pgfqpoint{3.030440in}{1.657536in}}%
\pgfpathlineto{\pgfqpoint{3.034100in}{1.021869in}}%
\pgfpathlineto{\pgfqpoint{3.037342in}{1.762652in}}%
\pgfpathlineto{\pgfqpoint{3.039928in}{1.125788in}}%
\pgfpathlineto{\pgfqpoint{3.043684in}{1.690410in}}%
\pgfpathlineto{\pgfqpoint{3.047119in}{1.083345in}}%
\pgfpathlineto{\pgfqpoint{3.049499in}{1.655792in}}%
\pgfpathlineto{\pgfqpoint{3.054156in}{0.988006in}}%
\pgfpathlineto{\pgfqpoint{3.056800in}{1.709857in}}%
\pgfpathlineto{\pgfqpoint{3.059489in}{1.037103in}}%
\pgfpathlineto{\pgfqpoint{3.063149in}{1.699223in}}%
\pgfpathlineto{\pgfqpoint{3.066088in}{1.044810in}}%
\pgfpathlineto{\pgfqpoint{3.069755in}{1.722534in}}%
\pgfpathlineto{\pgfqpoint{3.073749in}{1.033597in}}%
\pgfpathlineto{\pgfqpoint{3.075358in}{1.628300in}}%
\pgfpathlineto{\pgfqpoint{3.079288in}{1.075838in}}%
\pgfpathlineto{\pgfqpoint{3.083784in}{1.723418in}}%
\pgfpathlineto{\pgfqpoint{3.085328in}{1.099324in}}%
\pgfpathlineto{\pgfqpoint{3.089490in}{1.672158in}}%
\pgfpathlineto{\pgfqpoint{3.092654in}{1.005450in}}%
\pgfpathlineto{\pgfqpoint{3.095028in}{1.777837in}}%
\pgfpathlineto{\pgfqpoint{3.098502in}{1.061272in}}%
\pgfpathlineto{\pgfqpoint{3.101924in}{1.803839in}}%
\pgfpathlineto{\pgfqpoint{3.106806in}{0.925364in}}%
\pgfpathlineto{\pgfqpoint{3.107552in}{1.644708in}}%
\pgfpathlineto{\pgfqpoint{3.110929in}{1.019849in}}%
\pgfpathlineto{\pgfqpoint{3.115310in}{1.689756in}}%
\pgfpathlineto{\pgfqpoint{3.117259in}{0.949696in}}%
\pgfpathlineto{\pgfqpoint{3.120713in}{1.715638in}}%
\pgfpathlineto{\pgfqpoint{3.123839in}{1.052217in}}%
\pgfpathlineto{\pgfqpoint{3.128901in}{1.779101in}}%
\pgfpathlineto{\pgfqpoint{3.131095in}{0.996417in}}%
\pgfpathlineto{\pgfqpoint{3.133854in}{1.712008in}}%
\pgfpathlineto{\pgfqpoint{3.136807in}{1.066784in}}%
\pgfpathlineto{\pgfqpoint{3.139785in}{1.765972in}}%
\pgfpathlineto{\pgfqpoint{3.143863in}{1.077777in}}%
\pgfpathlineto{\pgfqpoint{3.146308in}{1.672419in}}%
\pgfpathlineto{\pgfqpoint{3.150463in}{1.099078in}}%
\pgfpathlineto{\pgfqpoint{3.152663in}{1.681492in}}%
\pgfpathlineto{\pgfqpoint{3.156664in}{0.989599in}}%
\pgfpathlineto{\pgfqpoint{3.159610in}{1.759912in}}%
\pgfpathlineto{\pgfqpoint{3.163669in}{1.055651in}}%
\pgfpathlineto{\pgfqpoint{3.166274in}{1.642031in}}%
\pgfpathlineto{\pgfqpoint{3.170371in}{0.991995in}}%
\pgfpathlineto{\pgfqpoint{3.172057in}{1.677317in}}%
\pgfpathlineto{\pgfqpoint{3.175775in}{1.110950in}}%
\pgfpathlineto{\pgfqpoint{3.178727in}{1.737489in}}%
\pgfpathlineto{\pgfqpoint{3.181821in}{1.067468in}}%
\pgfpathlineto{\pgfqpoint{3.185243in}{1.702592in}}%
\pgfpathlineto{\pgfqpoint{3.188356in}{1.101777in}}%
\pgfpathlineto{\pgfqpoint{3.192010in}{1.709892in}}%
\pgfpathlineto{\pgfqpoint{3.194873in}{1.086568in}}%
\pgfpathlineto{\pgfqpoint{3.199362in}{1.764492in}}%
\pgfpathlineto{\pgfqpoint{3.201980in}{1.069325in}}%
\pgfpathlineto{\pgfqpoint{3.205962in}{1.758812in}}%
\pgfpathlineto{\pgfqpoint{3.207635in}{1.104761in}}%
\pgfpathlineto{\pgfqpoint{3.211108in}{1.690293in}}%
\pgfpathlineto{\pgfqpoint{3.217052in}{0.976898in}}%
\pgfpathlineto{\pgfqpoint{3.217914in}{1.737093in}}%
\pgfpathlineto{\pgfqpoint{3.220397in}{1.097878in}}%
\pgfpathlineto{\pgfqpoint{3.223703in}{1.759405in}}%
\pgfpathlineto{\pgfqpoint{3.227627in}{1.019766in}}%
\pgfpathlineto{\pgfqpoint{3.231673in}{1.722190in}}%
\pgfpathlineto{\pgfqpoint{3.233982in}{1.011480in}}%
\pgfpathlineto{\pgfqpoint{3.237693in}{1.792475in}}%
\pgfpathlineto{\pgfqpoint{3.239655in}{1.154855in}}%
\pgfpathlineto{\pgfqpoint{3.243032in}{1.617066in}}%
\pgfpathlineto{\pgfqpoint{3.246873in}{1.069162in}}%
\pgfpathlineto{\pgfqpoint{3.250256in}{1.666556in}}%
\pgfpathlineto{\pgfqpoint{3.253228in}{1.105010in}}%
\pgfpathlineto{\pgfqpoint{3.258174in}{1.734685in}}%
\pgfpathlineto{\pgfqpoint{3.259435in}{1.083085in}}%
\pgfpathlineto{\pgfqpoint{3.264735in}{1.775132in}}%
\pgfpathlineto{\pgfqpoint{3.265855in}{1.084633in}}%
\pgfpathlineto{\pgfqpoint{3.268756in}{1.691167in}}%
\pgfpathlineto{\pgfqpoint{3.272017in}{0.991167in}}%
\pgfpathlineto{\pgfqpoint{3.275400in}{1.697113in}}%
\pgfpathlineto{\pgfqpoint{3.279530in}{0.989636in}}%
\pgfpathlineto{\pgfqpoint{3.282978in}{1.792538in}}%
\pgfpathlineto{\pgfqpoint{3.284985in}{1.072114in}}%
\pgfpathlineto{\pgfqpoint{3.288870in}{1.707650in}}%
\pgfpathlineto{\pgfqpoint{3.293199in}{0.995041in}}%
\pgfpathlineto{\pgfqpoint{3.294569in}{1.721006in}}%
\pgfpathlineto{\pgfqpoint{3.298133in}{1.100231in}}%
\pgfpathlineto{\pgfqpoint{3.301002in}{1.722447in}}%
\pgfpathlineto{\pgfqpoint{3.304868in}{1.040918in}}%
\pgfpathlineto{\pgfqpoint{3.310142in}{1.816055in}}%
\pgfpathlineto{\pgfqpoint{3.311010in}{1.148409in}}%
\pgfpathlineto{\pgfqpoint{3.314799in}{1.738707in}}%
\pgfpathlineto{\pgfqpoint{3.317263in}{0.991245in}}%
\pgfpathlineto{\pgfqpoint{3.321412in}{1.752259in}}%
\pgfpathlineto{\pgfqpoint{3.324403in}{0.929654in}}%
\pgfpathlineto{\pgfqpoint{3.326989in}{1.705928in}}%
\pgfpathlineto{\pgfqpoint{3.330546in}{1.043806in}}%
\pgfpathlineto{\pgfqpoint{3.333305in}{1.746657in}}%
\pgfpathlineto{\pgfqpoint{3.337043in}{1.069229in}}%
\pgfpathlineto{\pgfqpoint{3.339661in}{1.690732in}}%
\pgfpathlineto{\pgfqpoint{3.342883in}{1.087626in}}%
\pgfpathlineto{\pgfqpoint{3.346556in}{1.785941in}}%
\pgfpathlineto{\pgfqpoint{3.349341in}{1.091107in}}%
\pgfpathlineto{\pgfqpoint{3.353909in}{1.719077in}}%
\pgfpathlineto{\pgfqpoint{3.356018in}{1.081927in}}%
\pgfpathlineto{\pgfqpoint{3.360347in}{1.734646in}}%
\pgfpathlineto{\pgfqpoint{3.363197in}{1.023517in}}%
\pgfpathlineto{\pgfqpoint{3.365622in}{1.710975in}}%
\pgfpathlineto{\pgfqpoint{3.368735in}{1.098905in}}%
\pgfpathlineto{\pgfqpoint{3.372408in}{1.707342in}}%
\pgfpathlineto{\pgfqpoint{3.375097in}{1.042990in}}%
\pgfpathlineto{\pgfqpoint{3.379098in}{1.739864in}}%
\pgfpathlineto{\pgfqpoint{3.381439in}{1.088801in}}%
\pgfpathlineto{\pgfqpoint{3.385157in}{1.780475in}}%
\pgfpathlineto{\pgfqpoint{3.388239in}{1.074176in}}%
\pgfpathlineto{\pgfqpoint{3.392748in}{1.722384in}}%
\pgfpathlineto{\pgfqpoint{3.394967in}{1.015812in}}%
\pgfpathlineto{\pgfqpoint{3.398164in}{1.731401in}}%
\pgfpathlineto{\pgfqpoint{3.400853in}{1.038693in}}%
\pgfpathlineto{\pgfqpoint{3.404230in}{1.642334in}}%
\pgfpathlineto{\pgfqpoint{3.409324in}{1.034631in}}%
\pgfpathlineto{\pgfqpoint{3.410566in}{1.751663in}}%
\pgfpathlineto{\pgfqpoint{3.415744in}{1.048743in}}%
\pgfpathlineto{\pgfqpoint{3.417242in}{1.680309in}}%
\pgfpathlineto{\pgfqpoint{3.421095in}{1.085528in}}%
\pgfpathlineto{\pgfqpoint{3.424087in}{1.813433in}}%
\pgfpathlineto{\pgfqpoint{3.426859in}{0.999460in}}%
\pgfpathlineto{\pgfqpoint{3.430802in}{1.786779in}}%
\pgfpathlineto{\pgfqpoint{3.433388in}{1.118567in}}%
\pgfpathlineto{\pgfqpoint{3.437415in}{1.727936in}}%
\pgfpathlineto{\pgfqpoint{3.441705in}{0.944675in}}%
\pgfpathlineto{\pgfqpoint{3.443030in}{1.801888in}}%
\pgfpathlineto{\pgfqpoint{3.446066in}{1.003633in}}%
\pgfpathlineto{\pgfqpoint{3.451875in}{1.841777in}}%
\pgfpathlineto{\pgfqpoint{3.452415in}{1.144302in}}%
\pgfpathlineto{\pgfqpoint{3.455715in}{1.618312in}}%
\pgfpathlineto{\pgfqpoint{3.459124in}{1.054276in}}%
\pgfpathlineto{\pgfqpoint{3.462469in}{1.702744in}}%
\pgfpathlineto{\pgfqpoint{3.468014in}{0.967476in}}%
\pgfpathlineto{\pgfqpoint{3.469249in}{1.707144in}}%
\pgfpathlineto{\pgfqpoint{3.472343in}{1.034022in}}%
\pgfpathlineto{\pgfqpoint{3.476389in}{1.702924in}}%
\pgfpathlineto{\pgfqpoint{3.478891in}{1.012826in}}%
\pgfpathlineto{\pgfqpoint{3.481638in}{1.716113in}}%
\pgfpathlineto{\pgfqpoint{3.485812in}{1.062577in}}%
\pgfpathlineto{\pgfqpoint{3.487884in}{1.630581in}}%
\pgfpathlineto{\pgfqpoint{3.492399in}{1.128687in}}%
\pgfpathlineto{\pgfqpoint{3.494548in}{1.715954in}}%
\pgfpathlineto{\pgfqpoint{3.497577in}{1.119990in}}%
\pgfpathlineto{\pgfqpoint{3.501630in}{1.698806in}}%
\pgfpathlineto{\pgfqpoint{3.504364in}{1.047757in}}%
\pgfpathlineto{\pgfqpoint{3.507361in}{1.705934in}}%
\pgfpathlineto{\pgfqpoint{3.511362in}{0.996455in}}%
\pgfpathlineto{\pgfqpoint{3.515672in}{1.756648in}}%
\pgfpathlineto{\pgfqpoint{3.517216in}{0.977767in}}%
\pgfpathlineto{\pgfqpoint{3.520419in}{1.748743in}}%
\pgfpathlineto{\pgfqpoint{3.523397in}{0.973939in}}%
\pgfpathlineto{\pgfqpoint{3.526742in}{1.632029in}}%
\pgfpathlineto{\pgfqpoint{3.529913in}{1.024750in}}%
\pgfpathlineto{\pgfqpoint{3.533554in}{1.722254in}}%
\pgfpathlineto{\pgfqpoint{3.536359in}{0.909104in}}%
\pgfpathlineto{\pgfqpoint{3.539652in}{1.717383in}}%
\pgfpathlineto{\pgfqpoint{3.544599in}{0.992108in}}%
\pgfpathlineto{\pgfqpoint{3.546593in}{1.737043in}}%
\pgfpathlineto{\pgfqpoint{3.549635in}{1.044420in}}%
\pgfpathlineto{\pgfqpoint{3.552453in}{1.677772in}}%
\pgfpathlineto{\pgfqpoint{3.556261in}{0.998467in}}%
\pgfpathlineto{\pgfqpoint{3.558801in}{1.685640in}}%
\pgfpathlineto{\pgfqpoint{3.562609in}{1.055108in}}%
\pgfpathlineto{\pgfqpoint{3.565626in}{1.673295in}}%
\pgfpathlineto{\pgfqpoint{3.568759in}{1.127016in}}%
\pgfpathlineto{\pgfqpoint{3.571583in}{1.686821in}}%
\pgfpathlineto{\pgfqpoint{3.575204in}{1.073902in}}%
\pgfpathlineto{\pgfqpoint{3.578890in}{1.702938in}}%
\pgfpathlineto{\pgfqpoint{3.581251in}{0.994069in}}%
\pgfpathlineto{\pgfqpoint{3.585657in}{1.842380in}}%
\pgfpathlineto{\pgfqpoint{3.588153in}{1.104301in}}%
\pgfpathlineto{\pgfqpoint{3.591491in}{1.750942in}}%
\pgfpathlineto{\pgfqpoint{3.594418in}{1.112896in}}%
\pgfpathlineto{\pgfqpoint{3.597711in}{1.701869in}}%
\pgfpathlineto{\pgfqpoint{3.600760in}{1.003573in}}%
\pgfpathlineto{\pgfqpoint{3.604459in}{1.704761in}}%
\pgfpathlineto{\pgfqpoint{3.607939in}{1.059055in}}%
\pgfpathlineto{\pgfqpoint{3.612036in}{1.759855in}}%
\pgfpathlineto{\pgfqpoint{3.613715in}{1.124068in}}%
\pgfpathlineto{\pgfqpoint{3.618179in}{1.699424in}}%
\pgfpathlineto{\pgfqpoint{3.620495in}{1.115241in}}%
\pgfpathlineto{\pgfqpoint{3.623435in}{1.727609in}}%
\pgfpathlineto{\pgfqpoint{3.626529in}{1.166179in}}%
\pgfpathlineto{\pgfqpoint{3.630388in}{1.692283in}}%
\pgfpathlineto{\pgfqpoint{3.633173in}{1.065661in}}%
\pgfpathlineto{\pgfqpoint{3.636641in}{1.757566in}}%
\pgfpathlineto{\pgfqpoint{3.639259in}{1.075380in}}%
\pgfpathlineto{\pgfqpoint{3.642713in}{1.699389in}}%
\pgfpathlineto{\pgfqpoint{3.646334in}{1.071149in}}%
\pgfpathlineto{\pgfqpoint{3.648959in}{1.680073in}}%
\pgfpathlineto{\pgfqpoint{3.654182in}{1.022953in}}%
\pgfpathlineto{\pgfqpoint{3.656131in}{1.734149in}}%
\pgfpathlineto{\pgfqpoint{3.659862in}{0.999664in}}%
\pgfpathlineto{\pgfqpoint{3.662261in}{1.766728in}}%
\pgfpathlineto{\pgfqpoint{3.665690in}{1.041139in}}%
\pgfpathlineto{\pgfqpoint{3.668513in}{1.640082in}}%
\pgfpathlineto{\pgfqpoint{3.672315in}{1.035534in}}%
\pgfpathlineto{\pgfqpoint{3.674817in}{1.705390in}}%
\pgfpathlineto{\pgfqpoint{3.679153in}{0.995280in}}%
\pgfpathlineto{\pgfqpoint{3.681230in}{1.665942in}}%
\pgfpathlineto{\pgfqpoint{3.684504in}{1.067573in}}%
\pgfpathlineto{\pgfqpoint{3.688525in}{1.713221in}}%
\pgfpathlineto{\pgfqpoint{3.691239in}{0.980591in}}%
\pgfpathlineto{\pgfqpoint{3.694070in}{1.684161in}}%
\pgfpathlineto{\pgfqpoint{3.698186in}{1.063184in}}%
\pgfpathlineto{\pgfqpoint{3.700869in}{1.773994in}}%
\pgfpathlineto{\pgfqpoint{3.704162in}{1.073184in}}%
\pgfpathlineto{\pgfqpoint{3.708716in}{1.721017in}}%
\pgfpathlineto{\pgfqpoint{3.710672in}{1.010154in}}%
\pgfpathlineto{\pgfqpoint{3.715046in}{1.727325in}}%
\pgfpathlineto{\pgfqpoint{3.716622in}{1.122434in}}%
\pgfpathlineto{\pgfqpoint{3.720307in}{1.708249in}}%
\pgfpathlineto{\pgfqpoint{3.724598in}{1.008650in}}%
\pgfpathlineto{\pgfqpoint{3.726734in}{1.734248in}}%
\pgfpathlineto{\pgfqpoint{3.729776in}{1.019961in}}%
\pgfpathlineto{\pgfqpoint{3.733314in}{1.771625in}}%
\pgfpathlineto{\pgfqpoint{3.736601in}{1.031083in}}%
\pgfpathlineto{\pgfqpoint{3.739354in}{1.680053in}}%
\pgfpathlineto{\pgfqpoint{3.744075in}{1.006185in}}%
\pgfpathlineto{\pgfqpoint{3.745922in}{1.681025in}}%
\pgfpathlineto{\pgfqpoint{3.749099in}{1.072105in}}%
\pgfpathlineto{\pgfqpoint{3.754219in}{1.765227in}}%
\pgfpathlineto{\pgfqpoint{3.757140in}{0.980769in}}%
\pgfpathlineto{\pgfqpoint{3.759037in}{1.659036in}}%
\pgfpathlineto{\pgfqpoint{3.762273in}{1.101355in}}%
\pgfpathlineto{\pgfqpoint{3.765991in}{1.758950in}}%
\pgfpathlineto{\pgfqpoint{3.770474in}{0.974925in}}%
\pgfpathlineto{\pgfqpoint{3.771703in}{1.691693in}}%
\pgfpathlineto{\pgfqpoint{3.774983in}{1.060253in}}%
\pgfpathlineto{\pgfqpoint{3.778187in}{1.703440in}}%
\pgfpathlineto{\pgfqpoint{3.781827in}{1.093088in}}%
\pgfpathlineto{\pgfqpoint{3.784381in}{1.673610in}}%
\pgfpathlineto{\pgfqpoint{3.787777in}{1.058421in}}%
\pgfpathlineto{\pgfqpoint{3.790743in}{1.801085in}}%
\pgfpathlineto{\pgfqpoint{3.794956in}{1.038766in}}%
\pgfpathlineto{\pgfqpoint{3.798031in}{1.771641in}}%
\pgfpathlineto{\pgfqpoint{3.801749in}{1.012056in}}%
\pgfpathlineto{\pgfqpoint{3.804592in}{1.664721in}}%
\pgfpathlineto{\pgfqpoint{3.808329in}{0.954320in}}%
\pgfpathlineto{\pgfqpoint{3.810883in}{1.866207in}}%
\pgfpathlineto{\pgfqpoint{3.813880in}{1.052700in}}%
\pgfpathlineto{\pgfqpoint{3.816717in}{1.712802in}}%
\pgfpathlineto{\pgfqpoint{3.819978in}{1.075895in}}%
\pgfpathlineto{\pgfqpoint{3.824777in}{1.813769in}}%
\pgfpathlineto{\pgfqpoint{3.826231in}{0.961081in}}%
\pgfpathlineto{\pgfqpoint{3.829415in}{1.608505in}}%
\pgfpathlineto{\pgfqpoint{3.833892in}{0.952199in}}%
\pgfpathlineto{\pgfqpoint{3.835937in}{1.602145in}}%
\pgfpathlineto{\pgfqpoint{3.840581in}{0.996266in}}%
\pgfpathlineto{\pgfqpoint{3.842447in}{1.678111in}}%
\pgfpathlineto{\pgfqpoint{3.845901in}{0.987191in}}%
\pgfpathlineto{\pgfqpoint{3.848719in}{1.384537in}}%
\pgfpathlineto{\pgfqpoint{3.848719in}{1.384537in}}%
\pgfusepath{stroke}%
\end{pgfscope}%
\begin{pgfscope}%
\pgfsetrectcap%
\pgfsetmiterjoin%
\pgfsetlinewidth{0.803000pt}%
\definecolor{currentstroke}{rgb}{0.000000,0.000000,0.000000}%
\pgfsetstrokecolor{currentstroke}%
\pgfsetdash{}{0pt}%
\pgfpathmoveto{\pgfqpoint{0.471687in}{0.416447in}}%
\pgfpathlineto{\pgfqpoint{0.471687in}{2.341095in}}%
\pgfusepath{stroke}%
\end{pgfscope}%
\begin{pgfscope}%
\pgfsetrectcap%
\pgfsetmiterjoin%
\pgfsetlinewidth{0.803000pt}%
\definecolor{currentstroke}{rgb}{0.000000,0.000000,0.000000}%
\pgfsetstrokecolor{currentstroke}%
\pgfsetdash{}{0pt}%
\pgfpathmoveto{\pgfqpoint{4.009530in}{0.416447in}}%
\pgfpathlineto{\pgfqpoint{4.009530in}{2.341095in}}%
\pgfusepath{stroke}%
\end{pgfscope}%
\begin{pgfscope}%
\pgfsetrectcap%
\pgfsetmiterjoin%
\pgfsetlinewidth{0.803000pt}%
\definecolor{currentstroke}{rgb}{0.000000,0.000000,0.000000}%
\pgfsetstrokecolor{currentstroke}%
\pgfsetdash{}{0pt}%
\pgfpathmoveto{\pgfqpoint{0.471687in}{0.416447in}}%
\pgfpathlineto{\pgfqpoint{4.009530in}{0.416447in}}%
\pgfusepath{stroke}%
\end{pgfscope}%
\begin{pgfscope}%
\pgfsetrectcap%
\pgfsetmiterjoin%
\pgfsetlinewidth{0.803000pt}%
\definecolor{currentstroke}{rgb}{0.000000,0.000000,0.000000}%
\pgfsetstrokecolor{currentstroke}%
\pgfsetdash{}{0pt}%
\pgfpathmoveto{\pgfqpoint{0.471687in}{2.341095in}}%
\pgfpathlineto{\pgfqpoint{4.009530in}{2.341095in}}%
\pgfusepath{stroke}%
\end{pgfscope}%
\end{pgfpicture}%
\makeatother%
\endgroup%

    \caption{Simulated measurement with auto-zeroing applied.}
    \label{fig:autozero_time}
\end{figure}

When comparing to figure \ref{fig:autozero_raw_time} it is immediately evident, that the $f^{-1}$ component is no longer present. The difference in white noise strength is difficult to compare and it must be turned gain to the power spectral density. When calculating the spectral density it is important to remember, that the sampling rate is now halved. The result is shown in figure \ref{fig:autozero_psd} along with dashed lines showing the noise content prior to applying the autozero algorithm as before in figure \ref{fig:autozero_raw_psd}.

\begin{figure}[ht]
    \centering
    %% Creator: Matplotlib, PGF backend
%%
%% To include the figure in your LaTeX document, write
%%   \input{<filename>.pgf}
%%
%% Make sure the required packages are loaded in your preamble
%%   \usepackage{pgf}
%%
%% Also ensure that all the required font packages are loaded; for instance,
%% the lmodern package is sometimes necessary when using math font.
%%   \usepackage{lmodern}
%%
%% Figures using additional raster images can only be included by \input if
%% they are in the same directory as the main LaTeX file. For loading figures
%% from other directories you can use the `import` package
%%   \usepackage{import}
%%
%% and then include the figures with
%%   \import{<path to file>}{<filename>.pgf}
%%
%% Matplotlib used the following preamble
%%   \usepackage{siunitx}
%%   \usepackage{fontspec}
%%   \makeatletter\@ifpackageloaded{underscore}{}{\usepackage[strings]{underscore}}\makeatother
%%
\begingroup%
\makeatletter%
\begin{pgfpicture}%
\pgfpathrectangle{\pgfpointorigin}{\pgfqpoint{4.068242in}{2.514312in}}%
\pgfusepath{use as bounding box, clip}%
\begin{pgfscope}%
\pgfsetbuttcap%
\pgfsetmiterjoin%
\definecolor{currentfill}{rgb}{1.000000,1.000000,1.000000}%
\pgfsetfillcolor{currentfill}%
\pgfsetlinewidth{0.000000pt}%
\definecolor{currentstroke}{rgb}{1.000000,1.000000,1.000000}%
\pgfsetstrokecolor{currentstroke}%
\pgfsetdash{}{0pt}%
\pgfpathmoveto{\pgfqpoint{0.000000in}{0.000000in}}%
\pgfpathlineto{\pgfqpoint{4.068242in}{0.000000in}}%
\pgfpathlineto{\pgfqpoint{4.068242in}{2.514312in}}%
\pgfpathlineto{\pgfqpoint{0.000000in}{2.514312in}}%
\pgfpathlineto{\pgfqpoint{0.000000in}{0.000000in}}%
\pgfpathclose%
\pgfusepath{fill}%
\end{pgfscope}%
\begin{pgfscope}%
\pgfsetbuttcap%
\pgfsetmiterjoin%
\definecolor{currentfill}{rgb}{1.000000,1.000000,1.000000}%
\pgfsetfillcolor{currentfill}%
\pgfsetlinewidth{0.000000pt}%
\definecolor{currentstroke}{rgb}{0.000000,0.000000,0.000000}%
\pgfsetstrokecolor{currentstroke}%
\pgfsetstrokeopacity{0.000000}%
\pgfsetdash{}{0pt}%
\pgfpathmoveto{\pgfqpoint{0.661284in}{0.417642in}}%
\pgfpathlineto{\pgfqpoint{4.026572in}{0.417642in}}%
\pgfpathlineto{\pgfqpoint{4.026572in}{2.472642in}}%
\pgfpathlineto{\pgfqpoint{0.661284in}{2.472642in}}%
\pgfpathlineto{\pgfqpoint{0.661284in}{0.417642in}}%
\pgfpathclose%
\pgfusepath{fill}%
\end{pgfscope}%
\begin{pgfscope}%
\pgfpathrectangle{\pgfqpoint{0.661284in}{0.417642in}}{\pgfqpoint{3.365288in}{2.055000in}}%
\pgfusepath{clip}%
\pgfsetrectcap%
\pgfsetroundjoin%
\pgfsetlinewidth{0.803000pt}%
\definecolor{currentstroke}{rgb}{0.450000,0.450000,0.450000}%
\pgfsetstrokecolor{currentstroke}%
\pgfsetdash{}{0pt}%
\pgfpathmoveto{\pgfqpoint{0.704506in}{0.417642in}}%
\pgfpathlineto{\pgfqpoint{0.704506in}{2.472642in}}%
\pgfusepath{stroke}%
\end{pgfscope}%
\begin{pgfscope}%
\pgfsetbuttcap%
\pgfsetroundjoin%
\definecolor{currentfill}{rgb}{0.000000,0.000000,0.000000}%
\pgfsetfillcolor{currentfill}%
\pgfsetlinewidth{0.803000pt}%
\definecolor{currentstroke}{rgb}{0.000000,0.000000,0.000000}%
\pgfsetstrokecolor{currentstroke}%
\pgfsetdash{}{0pt}%
\pgfsys@defobject{currentmarker}{\pgfqpoint{0.000000in}{-0.048611in}}{\pgfqpoint{0.000000in}{0.000000in}}{%
\pgfpathmoveto{\pgfqpoint{0.000000in}{0.000000in}}%
\pgfpathlineto{\pgfqpoint{0.000000in}{-0.048611in}}%
\pgfusepath{stroke,fill}%
}%
\begin{pgfscope}%
\pgfsys@transformshift{0.704506in}{0.417642in}%
\pgfsys@useobject{currentmarker}{}%
\end{pgfscope}%
\end{pgfscope}%
\begin{pgfscope}%
\definecolor{textcolor}{rgb}{0.000000,0.000000,0.000000}%
\pgfsetstrokecolor{textcolor}%
\pgfsetfillcolor{textcolor}%
\pgftext[x=0.704506in,y=0.320420in,,top]{\color{textcolor}\rmfamily\fontsize{8.000000}{9.600000}\selectfont \(\displaystyle {10^{-3}}\)}%
\end{pgfscope}%
\begin{pgfscope}%
\pgfpathrectangle{\pgfqpoint{0.661284in}{0.417642in}}{\pgfqpoint{3.365288in}{2.055000in}}%
\pgfusepath{clip}%
\pgfsetrectcap%
\pgfsetroundjoin%
\pgfsetlinewidth{0.803000pt}%
\definecolor{currentstroke}{rgb}{0.450000,0.450000,0.450000}%
\pgfsetstrokecolor{currentstroke}%
\pgfsetdash{}{0pt}%
\pgfpathmoveto{\pgfqpoint{1.727889in}{0.417642in}}%
\pgfpathlineto{\pgfqpoint{1.727889in}{2.472642in}}%
\pgfusepath{stroke}%
\end{pgfscope}%
\begin{pgfscope}%
\pgfsetbuttcap%
\pgfsetroundjoin%
\definecolor{currentfill}{rgb}{0.000000,0.000000,0.000000}%
\pgfsetfillcolor{currentfill}%
\pgfsetlinewidth{0.803000pt}%
\definecolor{currentstroke}{rgb}{0.000000,0.000000,0.000000}%
\pgfsetstrokecolor{currentstroke}%
\pgfsetdash{}{0pt}%
\pgfsys@defobject{currentmarker}{\pgfqpoint{0.000000in}{-0.048611in}}{\pgfqpoint{0.000000in}{0.000000in}}{%
\pgfpathmoveto{\pgfqpoint{0.000000in}{0.000000in}}%
\pgfpathlineto{\pgfqpoint{0.000000in}{-0.048611in}}%
\pgfusepath{stroke,fill}%
}%
\begin{pgfscope}%
\pgfsys@transformshift{1.727889in}{0.417642in}%
\pgfsys@useobject{currentmarker}{}%
\end{pgfscope}%
\end{pgfscope}%
\begin{pgfscope}%
\definecolor{textcolor}{rgb}{0.000000,0.000000,0.000000}%
\pgfsetstrokecolor{textcolor}%
\pgfsetfillcolor{textcolor}%
\pgftext[x=1.727889in,y=0.320420in,,top]{\color{textcolor}\rmfamily\fontsize{8.000000}{9.600000}\selectfont \(\displaystyle {10^{-2}}\)}%
\end{pgfscope}%
\begin{pgfscope}%
\pgfpathrectangle{\pgfqpoint{0.661284in}{0.417642in}}{\pgfqpoint{3.365288in}{2.055000in}}%
\pgfusepath{clip}%
\pgfsetrectcap%
\pgfsetroundjoin%
\pgfsetlinewidth{0.803000pt}%
\definecolor{currentstroke}{rgb}{0.450000,0.450000,0.450000}%
\pgfsetstrokecolor{currentstroke}%
\pgfsetdash{}{0pt}%
\pgfpathmoveto{\pgfqpoint{2.751272in}{0.417642in}}%
\pgfpathlineto{\pgfqpoint{2.751272in}{2.472642in}}%
\pgfusepath{stroke}%
\end{pgfscope}%
\begin{pgfscope}%
\pgfsetbuttcap%
\pgfsetroundjoin%
\definecolor{currentfill}{rgb}{0.000000,0.000000,0.000000}%
\pgfsetfillcolor{currentfill}%
\pgfsetlinewidth{0.803000pt}%
\definecolor{currentstroke}{rgb}{0.000000,0.000000,0.000000}%
\pgfsetstrokecolor{currentstroke}%
\pgfsetdash{}{0pt}%
\pgfsys@defobject{currentmarker}{\pgfqpoint{0.000000in}{-0.048611in}}{\pgfqpoint{0.000000in}{0.000000in}}{%
\pgfpathmoveto{\pgfqpoint{0.000000in}{0.000000in}}%
\pgfpathlineto{\pgfqpoint{0.000000in}{-0.048611in}}%
\pgfusepath{stroke,fill}%
}%
\begin{pgfscope}%
\pgfsys@transformshift{2.751272in}{0.417642in}%
\pgfsys@useobject{currentmarker}{}%
\end{pgfscope}%
\end{pgfscope}%
\begin{pgfscope}%
\definecolor{textcolor}{rgb}{0.000000,0.000000,0.000000}%
\pgfsetstrokecolor{textcolor}%
\pgfsetfillcolor{textcolor}%
\pgftext[x=2.751272in,y=0.320420in,,top]{\color{textcolor}\rmfamily\fontsize{8.000000}{9.600000}\selectfont \(\displaystyle {10^{-1}}\)}%
\end{pgfscope}%
\begin{pgfscope}%
\pgfpathrectangle{\pgfqpoint{0.661284in}{0.417642in}}{\pgfqpoint{3.365288in}{2.055000in}}%
\pgfusepath{clip}%
\pgfsetrectcap%
\pgfsetroundjoin%
\pgfsetlinewidth{0.803000pt}%
\definecolor{currentstroke}{rgb}{0.450000,0.450000,0.450000}%
\pgfsetstrokecolor{currentstroke}%
\pgfsetdash{}{0pt}%
\pgfpathmoveto{\pgfqpoint{3.774656in}{0.417642in}}%
\pgfpathlineto{\pgfqpoint{3.774656in}{2.472642in}}%
\pgfusepath{stroke}%
\end{pgfscope}%
\begin{pgfscope}%
\pgfsetbuttcap%
\pgfsetroundjoin%
\definecolor{currentfill}{rgb}{0.000000,0.000000,0.000000}%
\pgfsetfillcolor{currentfill}%
\pgfsetlinewidth{0.803000pt}%
\definecolor{currentstroke}{rgb}{0.000000,0.000000,0.000000}%
\pgfsetstrokecolor{currentstroke}%
\pgfsetdash{}{0pt}%
\pgfsys@defobject{currentmarker}{\pgfqpoint{0.000000in}{-0.048611in}}{\pgfqpoint{0.000000in}{0.000000in}}{%
\pgfpathmoveto{\pgfqpoint{0.000000in}{0.000000in}}%
\pgfpathlineto{\pgfqpoint{0.000000in}{-0.048611in}}%
\pgfusepath{stroke,fill}%
}%
\begin{pgfscope}%
\pgfsys@transformshift{3.774656in}{0.417642in}%
\pgfsys@useobject{currentmarker}{}%
\end{pgfscope}%
\end{pgfscope}%
\begin{pgfscope}%
\definecolor{textcolor}{rgb}{0.000000,0.000000,0.000000}%
\pgfsetstrokecolor{textcolor}%
\pgfsetfillcolor{textcolor}%
\pgftext[x=3.774656in,y=0.320420in,,top]{\color{textcolor}\rmfamily\fontsize{8.000000}{9.600000}\selectfont \(\displaystyle {10^{0}}\)}%
\end{pgfscope}%
\begin{pgfscope}%
\pgfpathrectangle{\pgfqpoint{0.661284in}{0.417642in}}{\pgfqpoint{3.365288in}{2.055000in}}%
\pgfusepath{clip}%
\pgfsetrectcap%
\pgfsetroundjoin%
\pgfsetlinewidth{0.803000pt}%
\definecolor{currentstroke}{rgb}{0.850000,0.850000,0.850000}%
\pgfsetstrokecolor{currentstroke}%
\pgfsetdash{}{0pt}%
\pgfpathmoveto{\pgfqpoint{1.012575in}{0.417642in}}%
\pgfpathlineto{\pgfqpoint{1.012575in}{2.472642in}}%
\pgfusepath{stroke}%
\end{pgfscope}%
\begin{pgfscope}%
\pgfsetbuttcap%
\pgfsetroundjoin%
\definecolor{currentfill}{rgb}{0.000000,0.000000,0.000000}%
\pgfsetfillcolor{currentfill}%
\pgfsetlinewidth{0.602250pt}%
\definecolor{currentstroke}{rgb}{0.000000,0.000000,0.000000}%
\pgfsetstrokecolor{currentstroke}%
\pgfsetdash{}{0pt}%
\pgfsys@defobject{currentmarker}{\pgfqpoint{0.000000in}{-0.027778in}}{\pgfqpoint{0.000000in}{0.000000in}}{%
\pgfpathmoveto{\pgfqpoint{0.000000in}{0.000000in}}%
\pgfpathlineto{\pgfqpoint{0.000000in}{-0.027778in}}%
\pgfusepath{stroke,fill}%
}%
\begin{pgfscope}%
\pgfsys@transformshift{1.012575in}{0.417642in}%
\pgfsys@useobject{currentmarker}{}%
\end{pgfscope}%
\end{pgfscope}%
\begin{pgfscope}%
\pgfpathrectangle{\pgfqpoint{0.661284in}{0.417642in}}{\pgfqpoint{3.365288in}{2.055000in}}%
\pgfusepath{clip}%
\pgfsetrectcap%
\pgfsetroundjoin%
\pgfsetlinewidth{0.803000pt}%
\definecolor{currentstroke}{rgb}{0.850000,0.850000,0.850000}%
\pgfsetstrokecolor{currentstroke}%
\pgfsetdash{}{0pt}%
\pgfpathmoveto{\pgfqpoint{1.192784in}{0.417642in}}%
\pgfpathlineto{\pgfqpoint{1.192784in}{2.472642in}}%
\pgfusepath{stroke}%
\end{pgfscope}%
\begin{pgfscope}%
\pgfsetbuttcap%
\pgfsetroundjoin%
\definecolor{currentfill}{rgb}{0.000000,0.000000,0.000000}%
\pgfsetfillcolor{currentfill}%
\pgfsetlinewidth{0.602250pt}%
\definecolor{currentstroke}{rgb}{0.000000,0.000000,0.000000}%
\pgfsetstrokecolor{currentstroke}%
\pgfsetdash{}{0pt}%
\pgfsys@defobject{currentmarker}{\pgfqpoint{0.000000in}{-0.027778in}}{\pgfqpoint{0.000000in}{0.000000in}}{%
\pgfpathmoveto{\pgfqpoint{0.000000in}{0.000000in}}%
\pgfpathlineto{\pgfqpoint{0.000000in}{-0.027778in}}%
\pgfusepath{stroke,fill}%
}%
\begin{pgfscope}%
\pgfsys@transformshift{1.192784in}{0.417642in}%
\pgfsys@useobject{currentmarker}{}%
\end{pgfscope}%
\end{pgfscope}%
\begin{pgfscope}%
\pgfpathrectangle{\pgfqpoint{0.661284in}{0.417642in}}{\pgfqpoint{3.365288in}{2.055000in}}%
\pgfusepath{clip}%
\pgfsetrectcap%
\pgfsetroundjoin%
\pgfsetlinewidth{0.803000pt}%
\definecolor{currentstroke}{rgb}{0.850000,0.850000,0.850000}%
\pgfsetstrokecolor{currentstroke}%
\pgfsetdash{}{0pt}%
\pgfpathmoveto{\pgfqpoint{1.320644in}{0.417642in}}%
\pgfpathlineto{\pgfqpoint{1.320644in}{2.472642in}}%
\pgfusepath{stroke}%
\end{pgfscope}%
\begin{pgfscope}%
\pgfsetbuttcap%
\pgfsetroundjoin%
\definecolor{currentfill}{rgb}{0.000000,0.000000,0.000000}%
\pgfsetfillcolor{currentfill}%
\pgfsetlinewidth{0.602250pt}%
\definecolor{currentstroke}{rgb}{0.000000,0.000000,0.000000}%
\pgfsetstrokecolor{currentstroke}%
\pgfsetdash{}{0pt}%
\pgfsys@defobject{currentmarker}{\pgfqpoint{0.000000in}{-0.027778in}}{\pgfqpoint{0.000000in}{0.000000in}}{%
\pgfpathmoveto{\pgfqpoint{0.000000in}{0.000000in}}%
\pgfpathlineto{\pgfqpoint{0.000000in}{-0.027778in}}%
\pgfusepath{stroke,fill}%
}%
\begin{pgfscope}%
\pgfsys@transformshift{1.320644in}{0.417642in}%
\pgfsys@useobject{currentmarker}{}%
\end{pgfscope}%
\end{pgfscope}%
\begin{pgfscope}%
\pgfpathrectangle{\pgfqpoint{0.661284in}{0.417642in}}{\pgfqpoint{3.365288in}{2.055000in}}%
\pgfusepath{clip}%
\pgfsetrectcap%
\pgfsetroundjoin%
\pgfsetlinewidth{0.803000pt}%
\definecolor{currentstroke}{rgb}{0.850000,0.850000,0.850000}%
\pgfsetstrokecolor{currentstroke}%
\pgfsetdash{}{0pt}%
\pgfpathmoveto{\pgfqpoint{1.419820in}{0.417642in}}%
\pgfpathlineto{\pgfqpoint{1.419820in}{2.472642in}}%
\pgfusepath{stroke}%
\end{pgfscope}%
\begin{pgfscope}%
\pgfsetbuttcap%
\pgfsetroundjoin%
\definecolor{currentfill}{rgb}{0.000000,0.000000,0.000000}%
\pgfsetfillcolor{currentfill}%
\pgfsetlinewidth{0.602250pt}%
\definecolor{currentstroke}{rgb}{0.000000,0.000000,0.000000}%
\pgfsetstrokecolor{currentstroke}%
\pgfsetdash{}{0pt}%
\pgfsys@defobject{currentmarker}{\pgfqpoint{0.000000in}{-0.027778in}}{\pgfqpoint{0.000000in}{0.000000in}}{%
\pgfpathmoveto{\pgfqpoint{0.000000in}{0.000000in}}%
\pgfpathlineto{\pgfqpoint{0.000000in}{-0.027778in}}%
\pgfusepath{stroke,fill}%
}%
\begin{pgfscope}%
\pgfsys@transformshift{1.419820in}{0.417642in}%
\pgfsys@useobject{currentmarker}{}%
\end{pgfscope}%
\end{pgfscope}%
\begin{pgfscope}%
\pgfpathrectangle{\pgfqpoint{0.661284in}{0.417642in}}{\pgfqpoint{3.365288in}{2.055000in}}%
\pgfusepath{clip}%
\pgfsetrectcap%
\pgfsetroundjoin%
\pgfsetlinewidth{0.803000pt}%
\definecolor{currentstroke}{rgb}{0.850000,0.850000,0.850000}%
\pgfsetstrokecolor{currentstroke}%
\pgfsetdash{}{0pt}%
\pgfpathmoveto{\pgfqpoint{1.500853in}{0.417642in}}%
\pgfpathlineto{\pgfqpoint{1.500853in}{2.472642in}}%
\pgfusepath{stroke}%
\end{pgfscope}%
\begin{pgfscope}%
\pgfsetbuttcap%
\pgfsetroundjoin%
\definecolor{currentfill}{rgb}{0.000000,0.000000,0.000000}%
\pgfsetfillcolor{currentfill}%
\pgfsetlinewidth{0.602250pt}%
\definecolor{currentstroke}{rgb}{0.000000,0.000000,0.000000}%
\pgfsetstrokecolor{currentstroke}%
\pgfsetdash{}{0pt}%
\pgfsys@defobject{currentmarker}{\pgfqpoint{0.000000in}{-0.027778in}}{\pgfqpoint{0.000000in}{0.000000in}}{%
\pgfpathmoveto{\pgfqpoint{0.000000in}{0.000000in}}%
\pgfpathlineto{\pgfqpoint{0.000000in}{-0.027778in}}%
\pgfusepath{stroke,fill}%
}%
\begin{pgfscope}%
\pgfsys@transformshift{1.500853in}{0.417642in}%
\pgfsys@useobject{currentmarker}{}%
\end{pgfscope}%
\end{pgfscope}%
\begin{pgfscope}%
\pgfpathrectangle{\pgfqpoint{0.661284in}{0.417642in}}{\pgfqpoint{3.365288in}{2.055000in}}%
\pgfusepath{clip}%
\pgfsetrectcap%
\pgfsetroundjoin%
\pgfsetlinewidth{0.803000pt}%
\definecolor{currentstroke}{rgb}{0.850000,0.850000,0.850000}%
\pgfsetstrokecolor{currentstroke}%
\pgfsetdash{}{0pt}%
\pgfpathmoveto{\pgfqpoint{1.569365in}{0.417642in}}%
\pgfpathlineto{\pgfqpoint{1.569365in}{2.472642in}}%
\pgfusepath{stroke}%
\end{pgfscope}%
\begin{pgfscope}%
\pgfsetbuttcap%
\pgfsetroundjoin%
\definecolor{currentfill}{rgb}{0.000000,0.000000,0.000000}%
\pgfsetfillcolor{currentfill}%
\pgfsetlinewidth{0.602250pt}%
\definecolor{currentstroke}{rgb}{0.000000,0.000000,0.000000}%
\pgfsetstrokecolor{currentstroke}%
\pgfsetdash{}{0pt}%
\pgfsys@defobject{currentmarker}{\pgfqpoint{0.000000in}{-0.027778in}}{\pgfqpoint{0.000000in}{0.000000in}}{%
\pgfpathmoveto{\pgfqpoint{0.000000in}{0.000000in}}%
\pgfpathlineto{\pgfqpoint{0.000000in}{-0.027778in}}%
\pgfusepath{stroke,fill}%
}%
\begin{pgfscope}%
\pgfsys@transformshift{1.569365in}{0.417642in}%
\pgfsys@useobject{currentmarker}{}%
\end{pgfscope}%
\end{pgfscope}%
\begin{pgfscope}%
\pgfpathrectangle{\pgfqpoint{0.661284in}{0.417642in}}{\pgfqpoint{3.365288in}{2.055000in}}%
\pgfusepath{clip}%
\pgfsetrectcap%
\pgfsetroundjoin%
\pgfsetlinewidth{0.803000pt}%
\definecolor{currentstroke}{rgb}{0.850000,0.850000,0.850000}%
\pgfsetstrokecolor{currentstroke}%
\pgfsetdash{}{0pt}%
\pgfpathmoveto{\pgfqpoint{1.628713in}{0.417642in}}%
\pgfpathlineto{\pgfqpoint{1.628713in}{2.472642in}}%
\pgfusepath{stroke}%
\end{pgfscope}%
\begin{pgfscope}%
\pgfsetbuttcap%
\pgfsetroundjoin%
\definecolor{currentfill}{rgb}{0.000000,0.000000,0.000000}%
\pgfsetfillcolor{currentfill}%
\pgfsetlinewidth{0.602250pt}%
\definecolor{currentstroke}{rgb}{0.000000,0.000000,0.000000}%
\pgfsetstrokecolor{currentstroke}%
\pgfsetdash{}{0pt}%
\pgfsys@defobject{currentmarker}{\pgfqpoint{0.000000in}{-0.027778in}}{\pgfqpoint{0.000000in}{0.000000in}}{%
\pgfpathmoveto{\pgfqpoint{0.000000in}{0.000000in}}%
\pgfpathlineto{\pgfqpoint{0.000000in}{-0.027778in}}%
\pgfusepath{stroke,fill}%
}%
\begin{pgfscope}%
\pgfsys@transformshift{1.628713in}{0.417642in}%
\pgfsys@useobject{currentmarker}{}%
\end{pgfscope}%
\end{pgfscope}%
\begin{pgfscope}%
\pgfpathrectangle{\pgfqpoint{0.661284in}{0.417642in}}{\pgfqpoint{3.365288in}{2.055000in}}%
\pgfusepath{clip}%
\pgfsetrectcap%
\pgfsetroundjoin%
\pgfsetlinewidth{0.803000pt}%
\definecolor{currentstroke}{rgb}{0.850000,0.850000,0.850000}%
\pgfsetstrokecolor{currentstroke}%
\pgfsetdash{}{0pt}%
\pgfpathmoveto{\pgfqpoint{1.681062in}{0.417642in}}%
\pgfpathlineto{\pgfqpoint{1.681062in}{2.472642in}}%
\pgfusepath{stroke}%
\end{pgfscope}%
\begin{pgfscope}%
\pgfsetbuttcap%
\pgfsetroundjoin%
\definecolor{currentfill}{rgb}{0.000000,0.000000,0.000000}%
\pgfsetfillcolor{currentfill}%
\pgfsetlinewidth{0.602250pt}%
\definecolor{currentstroke}{rgb}{0.000000,0.000000,0.000000}%
\pgfsetstrokecolor{currentstroke}%
\pgfsetdash{}{0pt}%
\pgfsys@defobject{currentmarker}{\pgfqpoint{0.000000in}{-0.027778in}}{\pgfqpoint{0.000000in}{0.000000in}}{%
\pgfpathmoveto{\pgfqpoint{0.000000in}{0.000000in}}%
\pgfpathlineto{\pgfqpoint{0.000000in}{-0.027778in}}%
\pgfusepath{stroke,fill}%
}%
\begin{pgfscope}%
\pgfsys@transformshift{1.681062in}{0.417642in}%
\pgfsys@useobject{currentmarker}{}%
\end{pgfscope}%
\end{pgfscope}%
\begin{pgfscope}%
\pgfpathrectangle{\pgfqpoint{0.661284in}{0.417642in}}{\pgfqpoint{3.365288in}{2.055000in}}%
\pgfusepath{clip}%
\pgfsetrectcap%
\pgfsetroundjoin%
\pgfsetlinewidth{0.803000pt}%
\definecolor{currentstroke}{rgb}{0.850000,0.850000,0.850000}%
\pgfsetstrokecolor{currentstroke}%
\pgfsetdash{}{0pt}%
\pgfpathmoveto{\pgfqpoint{2.035958in}{0.417642in}}%
\pgfpathlineto{\pgfqpoint{2.035958in}{2.472642in}}%
\pgfusepath{stroke}%
\end{pgfscope}%
\begin{pgfscope}%
\pgfsetbuttcap%
\pgfsetroundjoin%
\definecolor{currentfill}{rgb}{0.000000,0.000000,0.000000}%
\pgfsetfillcolor{currentfill}%
\pgfsetlinewidth{0.602250pt}%
\definecolor{currentstroke}{rgb}{0.000000,0.000000,0.000000}%
\pgfsetstrokecolor{currentstroke}%
\pgfsetdash{}{0pt}%
\pgfsys@defobject{currentmarker}{\pgfqpoint{0.000000in}{-0.027778in}}{\pgfqpoint{0.000000in}{0.000000in}}{%
\pgfpathmoveto{\pgfqpoint{0.000000in}{0.000000in}}%
\pgfpathlineto{\pgfqpoint{0.000000in}{-0.027778in}}%
\pgfusepath{stroke,fill}%
}%
\begin{pgfscope}%
\pgfsys@transformshift{2.035958in}{0.417642in}%
\pgfsys@useobject{currentmarker}{}%
\end{pgfscope}%
\end{pgfscope}%
\begin{pgfscope}%
\pgfpathrectangle{\pgfqpoint{0.661284in}{0.417642in}}{\pgfqpoint{3.365288in}{2.055000in}}%
\pgfusepath{clip}%
\pgfsetrectcap%
\pgfsetroundjoin%
\pgfsetlinewidth{0.803000pt}%
\definecolor{currentstroke}{rgb}{0.850000,0.850000,0.850000}%
\pgfsetstrokecolor{currentstroke}%
\pgfsetdash{}{0pt}%
\pgfpathmoveto{\pgfqpoint{2.216167in}{0.417642in}}%
\pgfpathlineto{\pgfqpoint{2.216167in}{2.472642in}}%
\pgfusepath{stroke}%
\end{pgfscope}%
\begin{pgfscope}%
\pgfsetbuttcap%
\pgfsetroundjoin%
\definecolor{currentfill}{rgb}{0.000000,0.000000,0.000000}%
\pgfsetfillcolor{currentfill}%
\pgfsetlinewidth{0.602250pt}%
\definecolor{currentstroke}{rgb}{0.000000,0.000000,0.000000}%
\pgfsetstrokecolor{currentstroke}%
\pgfsetdash{}{0pt}%
\pgfsys@defobject{currentmarker}{\pgfqpoint{0.000000in}{-0.027778in}}{\pgfqpoint{0.000000in}{0.000000in}}{%
\pgfpathmoveto{\pgfqpoint{0.000000in}{0.000000in}}%
\pgfpathlineto{\pgfqpoint{0.000000in}{-0.027778in}}%
\pgfusepath{stroke,fill}%
}%
\begin{pgfscope}%
\pgfsys@transformshift{2.216167in}{0.417642in}%
\pgfsys@useobject{currentmarker}{}%
\end{pgfscope}%
\end{pgfscope}%
\begin{pgfscope}%
\pgfpathrectangle{\pgfqpoint{0.661284in}{0.417642in}}{\pgfqpoint{3.365288in}{2.055000in}}%
\pgfusepath{clip}%
\pgfsetrectcap%
\pgfsetroundjoin%
\pgfsetlinewidth{0.803000pt}%
\definecolor{currentstroke}{rgb}{0.850000,0.850000,0.850000}%
\pgfsetstrokecolor{currentstroke}%
\pgfsetdash{}{0pt}%
\pgfpathmoveto{\pgfqpoint{2.344027in}{0.417642in}}%
\pgfpathlineto{\pgfqpoint{2.344027in}{2.472642in}}%
\pgfusepath{stroke}%
\end{pgfscope}%
\begin{pgfscope}%
\pgfsetbuttcap%
\pgfsetroundjoin%
\definecolor{currentfill}{rgb}{0.000000,0.000000,0.000000}%
\pgfsetfillcolor{currentfill}%
\pgfsetlinewidth{0.602250pt}%
\definecolor{currentstroke}{rgb}{0.000000,0.000000,0.000000}%
\pgfsetstrokecolor{currentstroke}%
\pgfsetdash{}{0pt}%
\pgfsys@defobject{currentmarker}{\pgfqpoint{0.000000in}{-0.027778in}}{\pgfqpoint{0.000000in}{0.000000in}}{%
\pgfpathmoveto{\pgfqpoint{0.000000in}{0.000000in}}%
\pgfpathlineto{\pgfqpoint{0.000000in}{-0.027778in}}%
\pgfusepath{stroke,fill}%
}%
\begin{pgfscope}%
\pgfsys@transformshift{2.344027in}{0.417642in}%
\pgfsys@useobject{currentmarker}{}%
\end{pgfscope}%
\end{pgfscope}%
\begin{pgfscope}%
\pgfpathrectangle{\pgfqpoint{0.661284in}{0.417642in}}{\pgfqpoint{3.365288in}{2.055000in}}%
\pgfusepath{clip}%
\pgfsetrectcap%
\pgfsetroundjoin%
\pgfsetlinewidth{0.803000pt}%
\definecolor{currentstroke}{rgb}{0.850000,0.850000,0.850000}%
\pgfsetstrokecolor{currentstroke}%
\pgfsetdash{}{0pt}%
\pgfpathmoveto{\pgfqpoint{2.443203in}{0.417642in}}%
\pgfpathlineto{\pgfqpoint{2.443203in}{2.472642in}}%
\pgfusepath{stroke}%
\end{pgfscope}%
\begin{pgfscope}%
\pgfsetbuttcap%
\pgfsetroundjoin%
\definecolor{currentfill}{rgb}{0.000000,0.000000,0.000000}%
\pgfsetfillcolor{currentfill}%
\pgfsetlinewidth{0.602250pt}%
\definecolor{currentstroke}{rgb}{0.000000,0.000000,0.000000}%
\pgfsetstrokecolor{currentstroke}%
\pgfsetdash{}{0pt}%
\pgfsys@defobject{currentmarker}{\pgfqpoint{0.000000in}{-0.027778in}}{\pgfqpoint{0.000000in}{0.000000in}}{%
\pgfpathmoveto{\pgfqpoint{0.000000in}{0.000000in}}%
\pgfpathlineto{\pgfqpoint{0.000000in}{-0.027778in}}%
\pgfusepath{stroke,fill}%
}%
\begin{pgfscope}%
\pgfsys@transformshift{2.443203in}{0.417642in}%
\pgfsys@useobject{currentmarker}{}%
\end{pgfscope}%
\end{pgfscope}%
\begin{pgfscope}%
\pgfpathrectangle{\pgfqpoint{0.661284in}{0.417642in}}{\pgfqpoint{3.365288in}{2.055000in}}%
\pgfusepath{clip}%
\pgfsetrectcap%
\pgfsetroundjoin%
\pgfsetlinewidth{0.803000pt}%
\definecolor{currentstroke}{rgb}{0.850000,0.850000,0.850000}%
\pgfsetstrokecolor{currentstroke}%
\pgfsetdash{}{0pt}%
\pgfpathmoveto{\pgfqpoint{2.524236in}{0.417642in}}%
\pgfpathlineto{\pgfqpoint{2.524236in}{2.472642in}}%
\pgfusepath{stroke}%
\end{pgfscope}%
\begin{pgfscope}%
\pgfsetbuttcap%
\pgfsetroundjoin%
\definecolor{currentfill}{rgb}{0.000000,0.000000,0.000000}%
\pgfsetfillcolor{currentfill}%
\pgfsetlinewidth{0.602250pt}%
\definecolor{currentstroke}{rgb}{0.000000,0.000000,0.000000}%
\pgfsetstrokecolor{currentstroke}%
\pgfsetdash{}{0pt}%
\pgfsys@defobject{currentmarker}{\pgfqpoint{0.000000in}{-0.027778in}}{\pgfqpoint{0.000000in}{0.000000in}}{%
\pgfpathmoveto{\pgfqpoint{0.000000in}{0.000000in}}%
\pgfpathlineto{\pgfqpoint{0.000000in}{-0.027778in}}%
\pgfusepath{stroke,fill}%
}%
\begin{pgfscope}%
\pgfsys@transformshift{2.524236in}{0.417642in}%
\pgfsys@useobject{currentmarker}{}%
\end{pgfscope}%
\end{pgfscope}%
\begin{pgfscope}%
\pgfpathrectangle{\pgfqpoint{0.661284in}{0.417642in}}{\pgfqpoint{3.365288in}{2.055000in}}%
\pgfusepath{clip}%
\pgfsetrectcap%
\pgfsetroundjoin%
\pgfsetlinewidth{0.803000pt}%
\definecolor{currentstroke}{rgb}{0.850000,0.850000,0.850000}%
\pgfsetstrokecolor{currentstroke}%
\pgfsetdash{}{0pt}%
\pgfpathmoveto{\pgfqpoint{2.592748in}{0.417642in}}%
\pgfpathlineto{\pgfqpoint{2.592748in}{2.472642in}}%
\pgfusepath{stroke}%
\end{pgfscope}%
\begin{pgfscope}%
\pgfsetbuttcap%
\pgfsetroundjoin%
\definecolor{currentfill}{rgb}{0.000000,0.000000,0.000000}%
\pgfsetfillcolor{currentfill}%
\pgfsetlinewidth{0.602250pt}%
\definecolor{currentstroke}{rgb}{0.000000,0.000000,0.000000}%
\pgfsetstrokecolor{currentstroke}%
\pgfsetdash{}{0pt}%
\pgfsys@defobject{currentmarker}{\pgfqpoint{0.000000in}{-0.027778in}}{\pgfqpoint{0.000000in}{0.000000in}}{%
\pgfpathmoveto{\pgfqpoint{0.000000in}{0.000000in}}%
\pgfpathlineto{\pgfqpoint{0.000000in}{-0.027778in}}%
\pgfusepath{stroke,fill}%
}%
\begin{pgfscope}%
\pgfsys@transformshift{2.592748in}{0.417642in}%
\pgfsys@useobject{currentmarker}{}%
\end{pgfscope}%
\end{pgfscope}%
\begin{pgfscope}%
\pgfpathrectangle{\pgfqpoint{0.661284in}{0.417642in}}{\pgfqpoint{3.365288in}{2.055000in}}%
\pgfusepath{clip}%
\pgfsetrectcap%
\pgfsetroundjoin%
\pgfsetlinewidth{0.803000pt}%
\definecolor{currentstroke}{rgb}{0.850000,0.850000,0.850000}%
\pgfsetstrokecolor{currentstroke}%
\pgfsetdash{}{0pt}%
\pgfpathmoveto{\pgfqpoint{2.652096in}{0.417642in}}%
\pgfpathlineto{\pgfqpoint{2.652096in}{2.472642in}}%
\pgfusepath{stroke}%
\end{pgfscope}%
\begin{pgfscope}%
\pgfsetbuttcap%
\pgfsetroundjoin%
\definecolor{currentfill}{rgb}{0.000000,0.000000,0.000000}%
\pgfsetfillcolor{currentfill}%
\pgfsetlinewidth{0.602250pt}%
\definecolor{currentstroke}{rgb}{0.000000,0.000000,0.000000}%
\pgfsetstrokecolor{currentstroke}%
\pgfsetdash{}{0pt}%
\pgfsys@defobject{currentmarker}{\pgfqpoint{0.000000in}{-0.027778in}}{\pgfqpoint{0.000000in}{0.000000in}}{%
\pgfpathmoveto{\pgfqpoint{0.000000in}{0.000000in}}%
\pgfpathlineto{\pgfqpoint{0.000000in}{-0.027778in}}%
\pgfusepath{stroke,fill}%
}%
\begin{pgfscope}%
\pgfsys@transformshift{2.652096in}{0.417642in}%
\pgfsys@useobject{currentmarker}{}%
\end{pgfscope}%
\end{pgfscope}%
\begin{pgfscope}%
\pgfpathrectangle{\pgfqpoint{0.661284in}{0.417642in}}{\pgfqpoint{3.365288in}{2.055000in}}%
\pgfusepath{clip}%
\pgfsetrectcap%
\pgfsetroundjoin%
\pgfsetlinewidth{0.803000pt}%
\definecolor{currentstroke}{rgb}{0.850000,0.850000,0.850000}%
\pgfsetstrokecolor{currentstroke}%
\pgfsetdash{}{0pt}%
\pgfpathmoveto{\pgfqpoint{2.704445in}{0.417642in}}%
\pgfpathlineto{\pgfqpoint{2.704445in}{2.472642in}}%
\pgfusepath{stroke}%
\end{pgfscope}%
\begin{pgfscope}%
\pgfsetbuttcap%
\pgfsetroundjoin%
\definecolor{currentfill}{rgb}{0.000000,0.000000,0.000000}%
\pgfsetfillcolor{currentfill}%
\pgfsetlinewidth{0.602250pt}%
\definecolor{currentstroke}{rgb}{0.000000,0.000000,0.000000}%
\pgfsetstrokecolor{currentstroke}%
\pgfsetdash{}{0pt}%
\pgfsys@defobject{currentmarker}{\pgfqpoint{0.000000in}{-0.027778in}}{\pgfqpoint{0.000000in}{0.000000in}}{%
\pgfpathmoveto{\pgfqpoint{0.000000in}{0.000000in}}%
\pgfpathlineto{\pgfqpoint{0.000000in}{-0.027778in}}%
\pgfusepath{stroke,fill}%
}%
\begin{pgfscope}%
\pgfsys@transformshift{2.704445in}{0.417642in}%
\pgfsys@useobject{currentmarker}{}%
\end{pgfscope}%
\end{pgfscope}%
\begin{pgfscope}%
\pgfpathrectangle{\pgfqpoint{0.661284in}{0.417642in}}{\pgfqpoint{3.365288in}{2.055000in}}%
\pgfusepath{clip}%
\pgfsetrectcap%
\pgfsetroundjoin%
\pgfsetlinewidth{0.803000pt}%
\definecolor{currentstroke}{rgb}{0.850000,0.850000,0.850000}%
\pgfsetstrokecolor{currentstroke}%
\pgfsetdash{}{0pt}%
\pgfpathmoveto{\pgfqpoint{3.059341in}{0.417642in}}%
\pgfpathlineto{\pgfqpoint{3.059341in}{2.472642in}}%
\pgfusepath{stroke}%
\end{pgfscope}%
\begin{pgfscope}%
\pgfsetbuttcap%
\pgfsetroundjoin%
\definecolor{currentfill}{rgb}{0.000000,0.000000,0.000000}%
\pgfsetfillcolor{currentfill}%
\pgfsetlinewidth{0.602250pt}%
\definecolor{currentstroke}{rgb}{0.000000,0.000000,0.000000}%
\pgfsetstrokecolor{currentstroke}%
\pgfsetdash{}{0pt}%
\pgfsys@defobject{currentmarker}{\pgfqpoint{0.000000in}{-0.027778in}}{\pgfqpoint{0.000000in}{0.000000in}}{%
\pgfpathmoveto{\pgfqpoint{0.000000in}{0.000000in}}%
\pgfpathlineto{\pgfqpoint{0.000000in}{-0.027778in}}%
\pgfusepath{stroke,fill}%
}%
\begin{pgfscope}%
\pgfsys@transformshift{3.059341in}{0.417642in}%
\pgfsys@useobject{currentmarker}{}%
\end{pgfscope}%
\end{pgfscope}%
\begin{pgfscope}%
\pgfpathrectangle{\pgfqpoint{0.661284in}{0.417642in}}{\pgfqpoint{3.365288in}{2.055000in}}%
\pgfusepath{clip}%
\pgfsetrectcap%
\pgfsetroundjoin%
\pgfsetlinewidth{0.803000pt}%
\definecolor{currentstroke}{rgb}{0.850000,0.850000,0.850000}%
\pgfsetstrokecolor{currentstroke}%
\pgfsetdash{}{0pt}%
\pgfpathmoveto{\pgfqpoint{3.239550in}{0.417642in}}%
\pgfpathlineto{\pgfqpoint{3.239550in}{2.472642in}}%
\pgfusepath{stroke}%
\end{pgfscope}%
\begin{pgfscope}%
\pgfsetbuttcap%
\pgfsetroundjoin%
\definecolor{currentfill}{rgb}{0.000000,0.000000,0.000000}%
\pgfsetfillcolor{currentfill}%
\pgfsetlinewidth{0.602250pt}%
\definecolor{currentstroke}{rgb}{0.000000,0.000000,0.000000}%
\pgfsetstrokecolor{currentstroke}%
\pgfsetdash{}{0pt}%
\pgfsys@defobject{currentmarker}{\pgfqpoint{0.000000in}{-0.027778in}}{\pgfqpoint{0.000000in}{0.000000in}}{%
\pgfpathmoveto{\pgfqpoint{0.000000in}{0.000000in}}%
\pgfpathlineto{\pgfqpoint{0.000000in}{-0.027778in}}%
\pgfusepath{stroke,fill}%
}%
\begin{pgfscope}%
\pgfsys@transformshift{3.239550in}{0.417642in}%
\pgfsys@useobject{currentmarker}{}%
\end{pgfscope}%
\end{pgfscope}%
\begin{pgfscope}%
\pgfpathrectangle{\pgfqpoint{0.661284in}{0.417642in}}{\pgfqpoint{3.365288in}{2.055000in}}%
\pgfusepath{clip}%
\pgfsetrectcap%
\pgfsetroundjoin%
\pgfsetlinewidth{0.803000pt}%
\definecolor{currentstroke}{rgb}{0.850000,0.850000,0.850000}%
\pgfsetstrokecolor{currentstroke}%
\pgfsetdash{}{0pt}%
\pgfpathmoveto{\pgfqpoint{3.367410in}{0.417642in}}%
\pgfpathlineto{\pgfqpoint{3.367410in}{2.472642in}}%
\pgfusepath{stroke}%
\end{pgfscope}%
\begin{pgfscope}%
\pgfsetbuttcap%
\pgfsetroundjoin%
\definecolor{currentfill}{rgb}{0.000000,0.000000,0.000000}%
\pgfsetfillcolor{currentfill}%
\pgfsetlinewidth{0.602250pt}%
\definecolor{currentstroke}{rgb}{0.000000,0.000000,0.000000}%
\pgfsetstrokecolor{currentstroke}%
\pgfsetdash{}{0pt}%
\pgfsys@defobject{currentmarker}{\pgfqpoint{0.000000in}{-0.027778in}}{\pgfqpoint{0.000000in}{0.000000in}}{%
\pgfpathmoveto{\pgfqpoint{0.000000in}{0.000000in}}%
\pgfpathlineto{\pgfqpoint{0.000000in}{-0.027778in}}%
\pgfusepath{stroke,fill}%
}%
\begin{pgfscope}%
\pgfsys@transformshift{3.367410in}{0.417642in}%
\pgfsys@useobject{currentmarker}{}%
\end{pgfscope}%
\end{pgfscope}%
\begin{pgfscope}%
\pgfpathrectangle{\pgfqpoint{0.661284in}{0.417642in}}{\pgfqpoint{3.365288in}{2.055000in}}%
\pgfusepath{clip}%
\pgfsetrectcap%
\pgfsetroundjoin%
\pgfsetlinewidth{0.803000pt}%
\definecolor{currentstroke}{rgb}{0.850000,0.850000,0.850000}%
\pgfsetstrokecolor{currentstroke}%
\pgfsetdash{}{0pt}%
\pgfpathmoveto{\pgfqpoint{3.466586in}{0.417642in}}%
\pgfpathlineto{\pgfqpoint{3.466586in}{2.472642in}}%
\pgfusepath{stroke}%
\end{pgfscope}%
\begin{pgfscope}%
\pgfsetbuttcap%
\pgfsetroundjoin%
\definecolor{currentfill}{rgb}{0.000000,0.000000,0.000000}%
\pgfsetfillcolor{currentfill}%
\pgfsetlinewidth{0.602250pt}%
\definecolor{currentstroke}{rgb}{0.000000,0.000000,0.000000}%
\pgfsetstrokecolor{currentstroke}%
\pgfsetdash{}{0pt}%
\pgfsys@defobject{currentmarker}{\pgfqpoint{0.000000in}{-0.027778in}}{\pgfqpoint{0.000000in}{0.000000in}}{%
\pgfpathmoveto{\pgfqpoint{0.000000in}{0.000000in}}%
\pgfpathlineto{\pgfqpoint{0.000000in}{-0.027778in}}%
\pgfusepath{stroke,fill}%
}%
\begin{pgfscope}%
\pgfsys@transformshift{3.466586in}{0.417642in}%
\pgfsys@useobject{currentmarker}{}%
\end{pgfscope}%
\end{pgfscope}%
\begin{pgfscope}%
\pgfpathrectangle{\pgfqpoint{0.661284in}{0.417642in}}{\pgfqpoint{3.365288in}{2.055000in}}%
\pgfusepath{clip}%
\pgfsetrectcap%
\pgfsetroundjoin%
\pgfsetlinewidth{0.803000pt}%
\definecolor{currentstroke}{rgb}{0.850000,0.850000,0.850000}%
\pgfsetstrokecolor{currentstroke}%
\pgfsetdash{}{0pt}%
\pgfpathmoveto{\pgfqpoint{3.547619in}{0.417642in}}%
\pgfpathlineto{\pgfqpoint{3.547619in}{2.472642in}}%
\pgfusepath{stroke}%
\end{pgfscope}%
\begin{pgfscope}%
\pgfsetbuttcap%
\pgfsetroundjoin%
\definecolor{currentfill}{rgb}{0.000000,0.000000,0.000000}%
\pgfsetfillcolor{currentfill}%
\pgfsetlinewidth{0.602250pt}%
\definecolor{currentstroke}{rgb}{0.000000,0.000000,0.000000}%
\pgfsetstrokecolor{currentstroke}%
\pgfsetdash{}{0pt}%
\pgfsys@defobject{currentmarker}{\pgfqpoint{0.000000in}{-0.027778in}}{\pgfqpoint{0.000000in}{0.000000in}}{%
\pgfpathmoveto{\pgfqpoint{0.000000in}{0.000000in}}%
\pgfpathlineto{\pgfqpoint{0.000000in}{-0.027778in}}%
\pgfusepath{stroke,fill}%
}%
\begin{pgfscope}%
\pgfsys@transformshift{3.547619in}{0.417642in}%
\pgfsys@useobject{currentmarker}{}%
\end{pgfscope}%
\end{pgfscope}%
\begin{pgfscope}%
\pgfpathrectangle{\pgfqpoint{0.661284in}{0.417642in}}{\pgfqpoint{3.365288in}{2.055000in}}%
\pgfusepath{clip}%
\pgfsetrectcap%
\pgfsetroundjoin%
\pgfsetlinewidth{0.803000pt}%
\definecolor{currentstroke}{rgb}{0.850000,0.850000,0.850000}%
\pgfsetstrokecolor{currentstroke}%
\pgfsetdash{}{0pt}%
\pgfpathmoveto{\pgfqpoint{3.616131in}{0.417642in}}%
\pgfpathlineto{\pgfqpoint{3.616131in}{2.472642in}}%
\pgfusepath{stroke}%
\end{pgfscope}%
\begin{pgfscope}%
\pgfsetbuttcap%
\pgfsetroundjoin%
\definecolor{currentfill}{rgb}{0.000000,0.000000,0.000000}%
\pgfsetfillcolor{currentfill}%
\pgfsetlinewidth{0.602250pt}%
\definecolor{currentstroke}{rgb}{0.000000,0.000000,0.000000}%
\pgfsetstrokecolor{currentstroke}%
\pgfsetdash{}{0pt}%
\pgfsys@defobject{currentmarker}{\pgfqpoint{0.000000in}{-0.027778in}}{\pgfqpoint{0.000000in}{0.000000in}}{%
\pgfpathmoveto{\pgfqpoint{0.000000in}{0.000000in}}%
\pgfpathlineto{\pgfqpoint{0.000000in}{-0.027778in}}%
\pgfusepath{stroke,fill}%
}%
\begin{pgfscope}%
\pgfsys@transformshift{3.616131in}{0.417642in}%
\pgfsys@useobject{currentmarker}{}%
\end{pgfscope}%
\end{pgfscope}%
\begin{pgfscope}%
\pgfpathrectangle{\pgfqpoint{0.661284in}{0.417642in}}{\pgfqpoint{3.365288in}{2.055000in}}%
\pgfusepath{clip}%
\pgfsetrectcap%
\pgfsetroundjoin%
\pgfsetlinewidth{0.803000pt}%
\definecolor{currentstroke}{rgb}{0.850000,0.850000,0.850000}%
\pgfsetstrokecolor{currentstroke}%
\pgfsetdash{}{0pt}%
\pgfpathmoveto{\pgfqpoint{3.675479in}{0.417642in}}%
\pgfpathlineto{\pgfqpoint{3.675479in}{2.472642in}}%
\pgfusepath{stroke}%
\end{pgfscope}%
\begin{pgfscope}%
\pgfsetbuttcap%
\pgfsetroundjoin%
\definecolor{currentfill}{rgb}{0.000000,0.000000,0.000000}%
\pgfsetfillcolor{currentfill}%
\pgfsetlinewidth{0.602250pt}%
\definecolor{currentstroke}{rgb}{0.000000,0.000000,0.000000}%
\pgfsetstrokecolor{currentstroke}%
\pgfsetdash{}{0pt}%
\pgfsys@defobject{currentmarker}{\pgfqpoint{0.000000in}{-0.027778in}}{\pgfqpoint{0.000000in}{0.000000in}}{%
\pgfpathmoveto{\pgfqpoint{0.000000in}{0.000000in}}%
\pgfpathlineto{\pgfqpoint{0.000000in}{-0.027778in}}%
\pgfusepath{stroke,fill}%
}%
\begin{pgfscope}%
\pgfsys@transformshift{3.675479in}{0.417642in}%
\pgfsys@useobject{currentmarker}{}%
\end{pgfscope}%
\end{pgfscope}%
\begin{pgfscope}%
\pgfpathrectangle{\pgfqpoint{0.661284in}{0.417642in}}{\pgfqpoint{3.365288in}{2.055000in}}%
\pgfusepath{clip}%
\pgfsetrectcap%
\pgfsetroundjoin%
\pgfsetlinewidth{0.803000pt}%
\definecolor{currentstroke}{rgb}{0.850000,0.850000,0.850000}%
\pgfsetstrokecolor{currentstroke}%
\pgfsetdash{}{0pt}%
\pgfpathmoveto{\pgfqpoint{3.727828in}{0.417642in}}%
\pgfpathlineto{\pgfqpoint{3.727828in}{2.472642in}}%
\pgfusepath{stroke}%
\end{pgfscope}%
\begin{pgfscope}%
\pgfsetbuttcap%
\pgfsetroundjoin%
\definecolor{currentfill}{rgb}{0.000000,0.000000,0.000000}%
\pgfsetfillcolor{currentfill}%
\pgfsetlinewidth{0.602250pt}%
\definecolor{currentstroke}{rgb}{0.000000,0.000000,0.000000}%
\pgfsetstrokecolor{currentstroke}%
\pgfsetdash{}{0pt}%
\pgfsys@defobject{currentmarker}{\pgfqpoint{0.000000in}{-0.027778in}}{\pgfqpoint{0.000000in}{0.000000in}}{%
\pgfpathmoveto{\pgfqpoint{0.000000in}{0.000000in}}%
\pgfpathlineto{\pgfqpoint{0.000000in}{-0.027778in}}%
\pgfusepath{stroke,fill}%
}%
\begin{pgfscope}%
\pgfsys@transformshift{3.727828in}{0.417642in}%
\pgfsys@useobject{currentmarker}{}%
\end{pgfscope}%
\end{pgfscope}%
\begin{pgfscope}%
\definecolor{textcolor}{rgb}{0.000000,0.000000,0.000000}%
\pgfsetstrokecolor{textcolor}%
\pgfsetfillcolor{textcolor}%
\pgftext[x=2.343928in,y=0.165003in,,top]{\color{textcolor}\rmfamily\fontsize{10.000000}{12.000000}\selectfont Frequency in \(\displaystyle \unit{\Hz}\)}%
\end{pgfscope}%
\begin{pgfscope}%
\pgfpathrectangle{\pgfqpoint{0.661284in}{0.417642in}}{\pgfqpoint{3.365288in}{2.055000in}}%
\pgfusepath{clip}%
\pgfsetrectcap%
\pgfsetroundjoin%
\pgfsetlinewidth{0.803000pt}%
\definecolor{currentstroke}{rgb}{0.450000,0.450000,0.450000}%
\pgfsetstrokecolor{currentstroke}%
\pgfsetdash{}{0pt}%
\pgfpathmoveto{\pgfqpoint{0.661284in}{0.855017in}}%
\pgfpathlineto{\pgfqpoint{4.026572in}{0.855017in}}%
\pgfusepath{stroke}%
\end{pgfscope}%
\begin{pgfscope}%
\pgfsetbuttcap%
\pgfsetroundjoin%
\definecolor{currentfill}{rgb}{0.000000,0.000000,0.000000}%
\pgfsetfillcolor{currentfill}%
\pgfsetlinewidth{0.803000pt}%
\definecolor{currentstroke}{rgb}{0.000000,0.000000,0.000000}%
\pgfsetstrokecolor{currentstroke}%
\pgfsetdash{}{0pt}%
\pgfsys@defobject{currentmarker}{\pgfqpoint{-0.048611in}{0.000000in}}{\pgfqpoint{-0.000000in}{0.000000in}}{%
\pgfpathmoveto{\pgfqpoint{-0.000000in}{0.000000in}}%
\pgfpathlineto{\pgfqpoint{-0.048611in}{0.000000in}}%
\pgfusepath{stroke,fill}%
}%
\begin{pgfscope}%
\pgfsys@transformshift{0.661284in}{0.855017in}%
\pgfsys@useobject{currentmarker}{}%
\end{pgfscope}%
\end{pgfscope}%
\begin{pgfscope}%
\definecolor{textcolor}{rgb}{0.000000,0.000000,0.000000}%
\pgfsetstrokecolor{textcolor}%
\pgfsetfillcolor{textcolor}%
\pgftext[x=0.256963in, y=0.815865in, left, base]{\color{textcolor}\rmfamily\fontsize{8.000000}{9.600000}\selectfont \(\displaystyle {10^{-13}}\)}%
\end{pgfscope}%
\begin{pgfscope}%
\pgfpathrectangle{\pgfqpoint{0.661284in}{0.417642in}}{\pgfqpoint{3.365288in}{2.055000in}}%
\pgfusepath{clip}%
\pgfsetrectcap%
\pgfsetroundjoin%
\pgfsetlinewidth{0.803000pt}%
\definecolor{currentstroke}{rgb}{0.450000,0.450000,0.450000}%
\pgfsetstrokecolor{currentstroke}%
\pgfsetdash{}{0pt}%
\pgfpathmoveto{\pgfqpoint{0.661284in}{1.463773in}}%
\pgfpathlineto{\pgfqpoint{4.026572in}{1.463773in}}%
\pgfusepath{stroke}%
\end{pgfscope}%
\begin{pgfscope}%
\pgfsetbuttcap%
\pgfsetroundjoin%
\definecolor{currentfill}{rgb}{0.000000,0.000000,0.000000}%
\pgfsetfillcolor{currentfill}%
\pgfsetlinewidth{0.803000pt}%
\definecolor{currentstroke}{rgb}{0.000000,0.000000,0.000000}%
\pgfsetstrokecolor{currentstroke}%
\pgfsetdash{}{0pt}%
\pgfsys@defobject{currentmarker}{\pgfqpoint{-0.048611in}{0.000000in}}{\pgfqpoint{-0.000000in}{0.000000in}}{%
\pgfpathmoveto{\pgfqpoint{-0.000000in}{0.000000in}}%
\pgfpathlineto{\pgfqpoint{-0.048611in}{0.000000in}}%
\pgfusepath{stroke,fill}%
}%
\begin{pgfscope}%
\pgfsys@transformshift{0.661284in}{1.463773in}%
\pgfsys@useobject{currentmarker}{}%
\end{pgfscope}%
\end{pgfscope}%
\begin{pgfscope}%
\definecolor{textcolor}{rgb}{0.000000,0.000000,0.000000}%
\pgfsetstrokecolor{textcolor}%
\pgfsetfillcolor{textcolor}%
\pgftext[x=0.256963in, y=1.424620in, left, base]{\color{textcolor}\rmfamily\fontsize{8.000000}{9.600000}\selectfont \(\displaystyle {10^{-12}}\)}%
\end{pgfscope}%
\begin{pgfscope}%
\pgfpathrectangle{\pgfqpoint{0.661284in}{0.417642in}}{\pgfqpoint{3.365288in}{2.055000in}}%
\pgfusepath{clip}%
\pgfsetrectcap%
\pgfsetroundjoin%
\pgfsetlinewidth{0.803000pt}%
\definecolor{currentstroke}{rgb}{0.450000,0.450000,0.450000}%
\pgfsetstrokecolor{currentstroke}%
\pgfsetdash{}{0pt}%
\pgfpathmoveto{\pgfqpoint{0.661284in}{2.072528in}}%
\pgfpathlineto{\pgfqpoint{4.026572in}{2.072528in}}%
\pgfusepath{stroke}%
\end{pgfscope}%
\begin{pgfscope}%
\pgfsetbuttcap%
\pgfsetroundjoin%
\definecolor{currentfill}{rgb}{0.000000,0.000000,0.000000}%
\pgfsetfillcolor{currentfill}%
\pgfsetlinewidth{0.803000pt}%
\definecolor{currentstroke}{rgb}{0.000000,0.000000,0.000000}%
\pgfsetstrokecolor{currentstroke}%
\pgfsetdash{}{0pt}%
\pgfsys@defobject{currentmarker}{\pgfqpoint{-0.048611in}{0.000000in}}{\pgfqpoint{-0.000000in}{0.000000in}}{%
\pgfpathmoveto{\pgfqpoint{-0.000000in}{0.000000in}}%
\pgfpathlineto{\pgfqpoint{-0.048611in}{0.000000in}}%
\pgfusepath{stroke,fill}%
}%
\begin{pgfscope}%
\pgfsys@transformshift{0.661284in}{2.072528in}%
\pgfsys@useobject{currentmarker}{}%
\end{pgfscope}%
\end{pgfscope}%
\begin{pgfscope}%
\definecolor{textcolor}{rgb}{0.000000,0.000000,0.000000}%
\pgfsetstrokecolor{textcolor}%
\pgfsetfillcolor{textcolor}%
\pgftext[x=0.256963in, y=2.033376in, left, base]{\color{textcolor}\rmfamily\fontsize{8.000000}{9.600000}\selectfont \(\displaystyle {10^{-11}}\)}%
\end{pgfscope}%
\begin{pgfscope}%
\pgfpathrectangle{\pgfqpoint{0.661284in}{0.417642in}}{\pgfqpoint{3.365288in}{2.055000in}}%
\pgfusepath{clip}%
\pgfsetrectcap%
\pgfsetroundjoin%
\pgfsetlinewidth{0.803000pt}%
\definecolor{currentstroke}{rgb}{0.850000,0.850000,0.850000}%
\pgfsetstrokecolor{currentstroke}%
\pgfsetdash{}{0pt}%
\pgfpathmoveto{\pgfqpoint{0.661284in}{0.429516in}}%
\pgfpathlineto{\pgfqpoint{4.026572in}{0.429516in}}%
\pgfusepath{stroke}%
\end{pgfscope}%
\begin{pgfscope}%
\pgfsetbuttcap%
\pgfsetroundjoin%
\definecolor{currentfill}{rgb}{0.000000,0.000000,0.000000}%
\pgfsetfillcolor{currentfill}%
\pgfsetlinewidth{0.602250pt}%
\definecolor{currentstroke}{rgb}{0.000000,0.000000,0.000000}%
\pgfsetstrokecolor{currentstroke}%
\pgfsetdash{}{0pt}%
\pgfsys@defobject{currentmarker}{\pgfqpoint{-0.027778in}{0.000000in}}{\pgfqpoint{-0.000000in}{0.000000in}}{%
\pgfpathmoveto{\pgfqpoint{-0.000000in}{0.000000in}}%
\pgfpathlineto{\pgfqpoint{-0.027778in}{0.000000in}}%
\pgfusepath{stroke,fill}%
}%
\begin{pgfscope}%
\pgfsys@transformshift{0.661284in}{0.429516in}%
\pgfsys@useobject{currentmarker}{}%
\end{pgfscope}%
\end{pgfscope}%
\begin{pgfscope}%
\pgfpathrectangle{\pgfqpoint{0.661284in}{0.417642in}}{\pgfqpoint{3.365288in}{2.055000in}}%
\pgfusepath{clip}%
\pgfsetrectcap%
\pgfsetroundjoin%
\pgfsetlinewidth{0.803000pt}%
\definecolor{currentstroke}{rgb}{0.850000,0.850000,0.850000}%
\pgfsetstrokecolor{currentstroke}%
\pgfsetdash{}{0pt}%
\pgfpathmoveto{\pgfqpoint{0.661284in}{0.536712in}}%
\pgfpathlineto{\pgfqpoint{4.026572in}{0.536712in}}%
\pgfusepath{stroke}%
\end{pgfscope}%
\begin{pgfscope}%
\pgfsetbuttcap%
\pgfsetroundjoin%
\definecolor{currentfill}{rgb}{0.000000,0.000000,0.000000}%
\pgfsetfillcolor{currentfill}%
\pgfsetlinewidth{0.602250pt}%
\definecolor{currentstroke}{rgb}{0.000000,0.000000,0.000000}%
\pgfsetstrokecolor{currentstroke}%
\pgfsetdash{}{0pt}%
\pgfsys@defobject{currentmarker}{\pgfqpoint{-0.027778in}{0.000000in}}{\pgfqpoint{-0.000000in}{0.000000in}}{%
\pgfpathmoveto{\pgfqpoint{-0.000000in}{0.000000in}}%
\pgfpathlineto{\pgfqpoint{-0.027778in}{0.000000in}}%
\pgfusepath{stroke,fill}%
}%
\begin{pgfscope}%
\pgfsys@transformshift{0.661284in}{0.536712in}%
\pgfsys@useobject{currentmarker}{}%
\end{pgfscope}%
\end{pgfscope}%
\begin{pgfscope}%
\pgfpathrectangle{\pgfqpoint{0.661284in}{0.417642in}}{\pgfqpoint{3.365288in}{2.055000in}}%
\pgfusepath{clip}%
\pgfsetrectcap%
\pgfsetroundjoin%
\pgfsetlinewidth{0.803000pt}%
\definecolor{currentstroke}{rgb}{0.850000,0.850000,0.850000}%
\pgfsetstrokecolor{currentstroke}%
\pgfsetdash{}{0pt}%
\pgfpathmoveto{\pgfqpoint{0.661284in}{0.612769in}}%
\pgfpathlineto{\pgfqpoint{4.026572in}{0.612769in}}%
\pgfusepath{stroke}%
\end{pgfscope}%
\begin{pgfscope}%
\pgfsetbuttcap%
\pgfsetroundjoin%
\definecolor{currentfill}{rgb}{0.000000,0.000000,0.000000}%
\pgfsetfillcolor{currentfill}%
\pgfsetlinewidth{0.602250pt}%
\definecolor{currentstroke}{rgb}{0.000000,0.000000,0.000000}%
\pgfsetstrokecolor{currentstroke}%
\pgfsetdash{}{0pt}%
\pgfsys@defobject{currentmarker}{\pgfqpoint{-0.027778in}{0.000000in}}{\pgfqpoint{-0.000000in}{0.000000in}}{%
\pgfpathmoveto{\pgfqpoint{-0.000000in}{0.000000in}}%
\pgfpathlineto{\pgfqpoint{-0.027778in}{0.000000in}}%
\pgfusepath{stroke,fill}%
}%
\begin{pgfscope}%
\pgfsys@transformshift{0.661284in}{0.612769in}%
\pgfsys@useobject{currentmarker}{}%
\end{pgfscope}%
\end{pgfscope}%
\begin{pgfscope}%
\pgfpathrectangle{\pgfqpoint{0.661284in}{0.417642in}}{\pgfqpoint{3.365288in}{2.055000in}}%
\pgfusepath{clip}%
\pgfsetrectcap%
\pgfsetroundjoin%
\pgfsetlinewidth{0.803000pt}%
\definecolor{currentstroke}{rgb}{0.850000,0.850000,0.850000}%
\pgfsetstrokecolor{currentstroke}%
\pgfsetdash{}{0pt}%
\pgfpathmoveto{\pgfqpoint{0.661284in}{0.671764in}}%
\pgfpathlineto{\pgfqpoint{4.026572in}{0.671764in}}%
\pgfusepath{stroke}%
\end{pgfscope}%
\begin{pgfscope}%
\pgfsetbuttcap%
\pgfsetroundjoin%
\definecolor{currentfill}{rgb}{0.000000,0.000000,0.000000}%
\pgfsetfillcolor{currentfill}%
\pgfsetlinewidth{0.602250pt}%
\definecolor{currentstroke}{rgb}{0.000000,0.000000,0.000000}%
\pgfsetstrokecolor{currentstroke}%
\pgfsetdash{}{0pt}%
\pgfsys@defobject{currentmarker}{\pgfqpoint{-0.027778in}{0.000000in}}{\pgfqpoint{-0.000000in}{0.000000in}}{%
\pgfpathmoveto{\pgfqpoint{-0.000000in}{0.000000in}}%
\pgfpathlineto{\pgfqpoint{-0.027778in}{0.000000in}}%
\pgfusepath{stroke,fill}%
}%
\begin{pgfscope}%
\pgfsys@transformshift{0.661284in}{0.671764in}%
\pgfsys@useobject{currentmarker}{}%
\end{pgfscope}%
\end{pgfscope}%
\begin{pgfscope}%
\pgfpathrectangle{\pgfqpoint{0.661284in}{0.417642in}}{\pgfqpoint{3.365288in}{2.055000in}}%
\pgfusepath{clip}%
\pgfsetrectcap%
\pgfsetroundjoin%
\pgfsetlinewidth{0.803000pt}%
\definecolor{currentstroke}{rgb}{0.850000,0.850000,0.850000}%
\pgfsetstrokecolor{currentstroke}%
\pgfsetdash{}{0pt}%
\pgfpathmoveto{\pgfqpoint{0.661284in}{0.719966in}}%
\pgfpathlineto{\pgfqpoint{4.026572in}{0.719966in}}%
\pgfusepath{stroke}%
\end{pgfscope}%
\begin{pgfscope}%
\pgfsetbuttcap%
\pgfsetroundjoin%
\definecolor{currentfill}{rgb}{0.000000,0.000000,0.000000}%
\pgfsetfillcolor{currentfill}%
\pgfsetlinewidth{0.602250pt}%
\definecolor{currentstroke}{rgb}{0.000000,0.000000,0.000000}%
\pgfsetstrokecolor{currentstroke}%
\pgfsetdash{}{0pt}%
\pgfsys@defobject{currentmarker}{\pgfqpoint{-0.027778in}{0.000000in}}{\pgfqpoint{-0.000000in}{0.000000in}}{%
\pgfpathmoveto{\pgfqpoint{-0.000000in}{0.000000in}}%
\pgfpathlineto{\pgfqpoint{-0.027778in}{0.000000in}}%
\pgfusepath{stroke,fill}%
}%
\begin{pgfscope}%
\pgfsys@transformshift{0.661284in}{0.719966in}%
\pgfsys@useobject{currentmarker}{}%
\end{pgfscope}%
\end{pgfscope}%
\begin{pgfscope}%
\pgfpathrectangle{\pgfqpoint{0.661284in}{0.417642in}}{\pgfqpoint{3.365288in}{2.055000in}}%
\pgfusepath{clip}%
\pgfsetrectcap%
\pgfsetroundjoin%
\pgfsetlinewidth{0.803000pt}%
\definecolor{currentstroke}{rgb}{0.850000,0.850000,0.850000}%
\pgfsetstrokecolor{currentstroke}%
\pgfsetdash{}{0pt}%
\pgfpathmoveto{\pgfqpoint{0.661284in}{0.760720in}}%
\pgfpathlineto{\pgfqpoint{4.026572in}{0.760720in}}%
\pgfusepath{stroke}%
\end{pgfscope}%
\begin{pgfscope}%
\pgfsetbuttcap%
\pgfsetroundjoin%
\definecolor{currentfill}{rgb}{0.000000,0.000000,0.000000}%
\pgfsetfillcolor{currentfill}%
\pgfsetlinewidth{0.602250pt}%
\definecolor{currentstroke}{rgb}{0.000000,0.000000,0.000000}%
\pgfsetstrokecolor{currentstroke}%
\pgfsetdash{}{0pt}%
\pgfsys@defobject{currentmarker}{\pgfqpoint{-0.027778in}{0.000000in}}{\pgfqpoint{-0.000000in}{0.000000in}}{%
\pgfpathmoveto{\pgfqpoint{-0.000000in}{0.000000in}}%
\pgfpathlineto{\pgfqpoint{-0.027778in}{0.000000in}}%
\pgfusepath{stroke,fill}%
}%
\begin{pgfscope}%
\pgfsys@transformshift{0.661284in}{0.760720in}%
\pgfsys@useobject{currentmarker}{}%
\end{pgfscope}%
\end{pgfscope}%
\begin{pgfscope}%
\pgfpathrectangle{\pgfqpoint{0.661284in}{0.417642in}}{\pgfqpoint{3.365288in}{2.055000in}}%
\pgfusepath{clip}%
\pgfsetrectcap%
\pgfsetroundjoin%
\pgfsetlinewidth{0.803000pt}%
\definecolor{currentstroke}{rgb}{0.850000,0.850000,0.850000}%
\pgfsetstrokecolor{currentstroke}%
\pgfsetdash{}{0pt}%
\pgfpathmoveto{\pgfqpoint{0.661284in}{0.796023in}}%
\pgfpathlineto{\pgfqpoint{4.026572in}{0.796023in}}%
\pgfusepath{stroke}%
\end{pgfscope}%
\begin{pgfscope}%
\pgfsetbuttcap%
\pgfsetroundjoin%
\definecolor{currentfill}{rgb}{0.000000,0.000000,0.000000}%
\pgfsetfillcolor{currentfill}%
\pgfsetlinewidth{0.602250pt}%
\definecolor{currentstroke}{rgb}{0.000000,0.000000,0.000000}%
\pgfsetstrokecolor{currentstroke}%
\pgfsetdash{}{0pt}%
\pgfsys@defobject{currentmarker}{\pgfqpoint{-0.027778in}{0.000000in}}{\pgfqpoint{-0.000000in}{0.000000in}}{%
\pgfpathmoveto{\pgfqpoint{-0.000000in}{0.000000in}}%
\pgfpathlineto{\pgfqpoint{-0.027778in}{0.000000in}}%
\pgfusepath{stroke,fill}%
}%
\begin{pgfscope}%
\pgfsys@transformshift{0.661284in}{0.796023in}%
\pgfsys@useobject{currentmarker}{}%
\end{pgfscope}%
\end{pgfscope}%
\begin{pgfscope}%
\pgfpathrectangle{\pgfqpoint{0.661284in}{0.417642in}}{\pgfqpoint{3.365288in}{2.055000in}}%
\pgfusepath{clip}%
\pgfsetrectcap%
\pgfsetroundjoin%
\pgfsetlinewidth{0.803000pt}%
\definecolor{currentstroke}{rgb}{0.850000,0.850000,0.850000}%
\pgfsetstrokecolor{currentstroke}%
\pgfsetdash{}{0pt}%
\pgfpathmoveto{\pgfqpoint{0.661284in}{0.827162in}}%
\pgfpathlineto{\pgfqpoint{4.026572in}{0.827162in}}%
\pgfusepath{stroke}%
\end{pgfscope}%
\begin{pgfscope}%
\pgfsetbuttcap%
\pgfsetroundjoin%
\definecolor{currentfill}{rgb}{0.000000,0.000000,0.000000}%
\pgfsetfillcolor{currentfill}%
\pgfsetlinewidth{0.602250pt}%
\definecolor{currentstroke}{rgb}{0.000000,0.000000,0.000000}%
\pgfsetstrokecolor{currentstroke}%
\pgfsetdash{}{0pt}%
\pgfsys@defobject{currentmarker}{\pgfqpoint{-0.027778in}{0.000000in}}{\pgfqpoint{-0.000000in}{0.000000in}}{%
\pgfpathmoveto{\pgfqpoint{-0.000000in}{0.000000in}}%
\pgfpathlineto{\pgfqpoint{-0.027778in}{0.000000in}}%
\pgfusepath{stroke,fill}%
}%
\begin{pgfscope}%
\pgfsys@transformshift{0.661284in}{0.827162in}%
\pgfsys@useobject{currentmarker}{}%
\end{pgfscope}%
\end{pgfscope}%
\begin{pgfscope}%
\pgfpathrectangle{\pgfqpoint{0.661284in}{0.417642in}}{\pgfqpoint{3.365288in}{2.055000in}}%
\pgfusepath{clip}%
\pgfsetrectcap%
\pgfsetroundjoin%
\pgfsetlinewidth{0.803000pt}%
\definecolor{currentstroke}{rgb}{0.850000,0.850000,0.850000}%
\pgfsetstrokecolor{currentstroke}%
\pgfsetdash{}{0pt}%
\pgfpathmoveto{\pgfqpoint{0.661284in}{1.038271in}}%
\pgfpathlineto{\pgfqpoint{4.026572in}{1.038271in}}%
\pgfusepath{stroke}%
\end{pgfscope}%
\begin{pgfscope}%
\pgfsetbuttcap%
\pgfsetroundjoin%
\definecolor{currentfill}{rgb}{0.000000,0.000000,0.000000}%
\pgfsetfillcolor{currentfill}%
\pgfsetlinewidth{0.602250pt}%
\definecolor{currentstroke}{rgb}{0.000000,0.000000,0.000000}%
\pgfsetstrokecolor{currentstroke}%
\pgfsetdash{}{0pt}%
\pgfsys@defobject{currentmarker}{\pgfqpoint{-0.027778in}{0.000000in}}{\pgfqpoint{-0.000000in}{0.000000in}}{%
\pgfpathmoveto{\pgfqpoint{-0.000000in}{0.000000in}}%
\pgfpathlineto{\pgfqpoint{-0.027778in}{0.000000in}}%
\pgfusepath{stroke,fill}%
}%
\begin{pgfscope}%
\pgfsys@transformshift{0.661284in}{1.038271in}%
\pgfsys@useobject{currentmarker}{}%
\end{pgfscope}%
\end{pgfscope}%
\begin{pgfscope}%
\pgfpathrectangle{\pgfqpoint{0.661284in}{0.417642in}}{\pgfqpoint{3.365288in}{2.055000in}}%
\pgfusepath{clip}%
\pgfsetrectcap%
\pgfsetroundjoin%
\pgfsetlinewidth{0.803000pt}%
\definecolor{currentstroke}{rgb}{0.850000,0.850000,0.850000}%
\pgfsetstrokecolor{currentstroke}%
\pgfsetdash{}{0pt}%
\pgfpathmoveto{\pgfqpoint{0.661284in}{1.145468in}}%
\pgfpathlineto{\pgfqpoint{4.026572in}{1.145468in}}%
\pgfusepath{stroke}%
\end{pgfscope}%
\begin{pgfscope}%
\pgfsetbuttcap%
\pgfsetroundjoin%
\definecolor{currentfill}{rgb}{0.000000,0.000000,0.000000}%
\pgfsetfillcolor{currentfill}%
\pgfsetlinewidth{0.602250pt}%
\definecolor{currentstroke}{rgb}{0.000000,0.000000,0.000000}%
\pgfsetstrokecolor{currentstroke}%
\pgfsetdash{}{0pt}%
\pgfsys@defobject{currentmarker}{\pgfqpoint{-0.027778in}{0.000000in}}{\pgfqpoint{-0.000000in}{0.000000in}}{%
\pgfpathmoveto{\pgfqpoint{-0.000000in}{0.000000in}}%
\pgfpathlineto{\pgfqpoint{-0.027778in}{0.000000in}}%
\pgfusepath{stroke,fill}%
}%
\begin{pgfscope}%
\pgfsys@transformshift{0.661284in}{1.145468in}%
\pgfsys@useobject{currentmarker}{}%
\end{pgfscope}%
\end{pgfscope}%
\begin{pgfscope}%
\pgfpathrectangle{\pgfqpoint{0.661284in}{0.417642in}}{\pgfqpoint{3.365288in}{2.055000in}}%
\pgfusepath{clip}%
\pgfsetrectcap%
\pgfsetroundjoin%
\pgfsetlinewidth{0.803000pt}%
\definecolor{currentstroke}{rgb}{0.850000,0.850000,0.850000}%
\pgfsetstrokecolor{currentstroke}%
\pgfsetdash{}{0pt}%
\pgfpathmoveto{\pgfqpoint{0.661284in}{1.221525in}}%
\pgfpathlineto{\pgfqpoint{4.026572in}{1.221525in}}%
\pgfusepath{stroke}%
\end{pgfscope}%
\begin{pgfscope}%
\pgfsetbuttcap%
\pgfsetroundjoin%
\definecolor{currentfill}{rgb}{0.000000,0.000000,0.000000}%
\pgfsetfillcolor{currentfill}%
\pgfsetlinewidth{0.602250pt}%
\definecolor{currentstroke}{rgb}{0.000000,0.000000,0.000000}%
\pgfsetstrokecolor{currentstroke}%
\pgfsetdash{}{0pt}%
\pgfsys@defobject{currentmarker}{\pgfqpoint{-0.027778in}{0.000000in}}{\pgfqpoint{-0.000000in}{0.000000in}}{%
\pgfpathmoveto{\pgfqpoint{-0.000000in}{0.000000in}}%
\pgfpathlineto{\pgfqpoint{-0.027778in}{0.000000in}}%
\pgfusepath{stroke,fill}%
}%
\begin{pgfscope}%
\pgfsys@transformshift{0.661284in}{1.221525in}%
\pgfsys@useobject{currentmarker}{}%
\end{pgfscope}%
\end{pgfscope}%
\begin{pgfscope}%
\pgfpathrectangle{\pgfqpoint{0.661284in}{0.417642in}}{\pgfqpoint{3.365288in}{2.055000in}}%
\pgfusepath{clip}%
\pgfsetrectcap%
\pgfsetroundjoin%
\pgfsetlinewidth{0.803000pt}%
\definecolor{currentstroke}{rgb}{0.850000,0.850000,0.850000}%
\pgfsetstrokecolor{currentstroke}%
\pgfsetdash{}{0pt}%
\pgfpathmoveto{\pgfqpoint{0.661284in}{1.280519in}}%
\pgfpathlineto{\pgfqpoint{4.026572in}{1.280519in}}%
\pgfusepath{stroke}%
\end{pgfscope}%
\begin{pgfscope}%
\pgfsetbuttcap%
\pgfsetroundjoin%
\definecolor{currentfill}{rgb}{0.000000,0.000000,0.000000}%
\pgfsetfillcolor{currentfill}%
\pgfsetlinewidth{0.602250pt}%
\definecolor{currentstroke}{rgb}{0.000000,0.000000,0.000000}%
\pgfsetstrokecolor{currentstroke}%
\pgfsetdash{}{0pt}%
\pgfsys@defobject{currentmarker}{\pgfqpoint{-0.027778in}{0.000000in}}{\pgfqpoint{-0.000000in}{0.000000in}}{%
\pgfpathmoveto{\pgfqpoint{-0.000000in}{0.000000in}}%
\pgfpathlineto{\pgfqpoint{-0.027778in}{0.000000in}}%
\pgfusepath{stroke,fill}%
}%
\begin{pgfscope}%
\pgfsys@transformshift{0.661284in}{1.280519in}%
\pgfsys@useobject{currentmarker}{}%
\end{pgfscope}%
\end{pgfscope}%
\begin{pgfscope}%
\pgfpathrectangle{\pgfqpoint{0.661284in}{0.417642in}}{\pgfqpoint{3.365288in}{2.055000in}}%
\pgfusepath{clip}%
\pgfsetrectcap%
\pgfsetroundjoin%
\pgfsetlinewidth{0.803000pt}%
\definecolor{currentstroke}{rgb}{0.850000,0.850000,0.850000}%
\pgfsetstrokecolor{currentstroke}%
\pgfsetdash{}{0pt}%
\pgfpathmoveto{\pgfqpoint{0.661284in}{1.328721in}}%
\pgfpathlineto{\pgfqpoint{4.026572in}{1.328721in}}%
\pgfusepath{stroke}%
\end{pgfscope}%
\begin{pgfscope}%
\pgfsetbuttcap%
\pgfsetroundjoin%
\definecolor{currentfill}{rgb}{0.000000,0.000000,0.000000}%
\pgfsetfillcolor{currentfill}%
\pgfsetlinewidth{0.602250pt}%
\definecolor{currentstroke}{rgb}{0.000000,0.000000,0.000000}%
\pgfsetstrokecolor{currentstroke}%
\pgfsetdash{}{0pt}%
\pgfsys@defobject{currentmarker}{\pgfqpoint{-0.027778in}{0.000000in}}{\pgfqpoint{-0.000000in}{0.000000in}}{%
\pgfpathmoveto{\pgfqpoint{-0.000000in}{0.000000in}}%
\pgfpathlineto{\pgfqpoint{-0.027778in}{0.000000in}}%
\pgfusepath{stroke,fill}%
}%
\begin{pgfscope}%
\pgfsys@transformshift{0.661284in}{1.328721in}%
\pgfsys@useobject{currentmarker}{}%
\end{pgfscope}%
\end{pgfscope}%
\begin{pgfscope}%
\pgfpathrectangle{\pgfqpoint{0.661284in}{0.417642in}}{\pgfqpoint{3.365288in}{2.055000in}}%
\pgfusepath{clip}%
\pgfsetrectcap%
\pgfsetroundjoin%
\pgfsetlinewidth{0.803000pt}%
\definecolor{currentstroke}{rgb}{0.850000,0.850000,0.850000}%
\pgfsetstrokecolor{currentstroke}%
\pgfsetdash{}{0pt}%
\pgfpathmoveto{\pgfqpoint{0.661284in}{1.369476in}}%
\pgfpathlineto{\pgfqpoint{4.026572in}{1.369476in}}%
\pgfusepath{stroke}%
\end{pgfscope}%
\begin{pgfscope}%
\pgfsetbuttcap%
\pgfsetroundjoin%
\definecolor{currentfill}{rgb}{0.000000,0.000000,0.000000}%
\pgfsetfillcolor{currentfill}%
\pgfsetlinewidth{0.602250pt}%
\definecolor{currentstroke}{rgb}{0.000000,0.000000,0.000000}%
\pgfsetstrokecolor{currentstroke}%
\pgfsetdash{}{0pt}%
\pgfsys@defobject{currentmarker}{\pgfqpoint{-0.027778in}{0.000000in}}{\pgfqpoint{-0.000000in}{0.000000in}}{%
\pgfpathmoveto{\pgfqpoint{-0.000000in}{0.000000in}}%
\pgfpathlineto{\pgfqpoint{-0.027778in}{0.000000in}}%
\pgfusepath{stroke,fill}%
}%
\begin{pgfscope}%
\pgfsys@transformshift{0.661284in}{1.369476in}%
\pgfsys@useobject{currentmarker}{}%
\end{pgfscope}%
\end{pgfscope}%
\begin{pgfscope}%
\pgfpathrectangle{\pgfqpoint{0.661284in}{0.417642in}}{\pgfqpoint{3.365288in}{2.055000in}}%
\pgfusepath{clip}%
\pgfsetrectcap%
\pgfsetroundjoin%
\pgfsetlinewidth{0.803000pt}%
\definecolor{currentstroke}{rgb}{0.850000,0.850000,0.850000}%
\pgfsetstrokecolor{currentstroke}%
\pgfsetdash{}{0pt}%
\pgfpathmoveto{\pgfqpoint{0.661284in}{1.404778in}}%
\pgfpathlineto{\pgfqpoint{4.026572in}{1.404778in}}%
\pgfusepath{stroke}%
\end{pgfscope}%
\begin{pgfscope}%
\pgfsetbuttcap%
\pgfsetroundjoin%
\definecolor{currentfill}{rgb}{0.000000,0.000000,0.000000}%
\pgfsetfillcolor{currentfill}%
\pgfsetlinewidth{0.602250pt}%
\definecolor{currentstroke}{rgb}{0.000000,0.000000,0.000000}%
\pgfsetstrokecolor{currentstroke}%
\pgfsetdash{}{0pt}%
\pgfsys@defobject{currentmarker}{\pgfqpoint{-0.027778in}{0.000000in}}{\pgfqpoint{-0.000000in}{0.000000in}}{%
\pgfpathmoveto{\pgfqpoint{-0.000000in}{0.000000in}}%
\pgfpathlineto{\pgfqpoint{-0.027778in}{0.000000in}}%
\pgfusepath{stroke,fill}%
}%
\begin{pgfscope}%
\pgfsys@transformshift{0.661284in}{1.404778in}%
\pgfsys@useobject{currentmarker}{}%
\end{pgfscope}%
\end{pgfscope}%
\begin{pgfscope}%
\pgfpathrectangle{\pgfqpoint{0.661284in}{0.417642in}}{\pgfqpoint{3.365288in}{2.055000in}}%
\pgfusepath{clip}%
\pgfsetrectcap%
\pgfsetroundjoin%
\pgfsetlinewidth{0.803000pt}%
\definecolor{currentstroke}{rgb}{0.850000,0.850000,0.850000}%
\pgfsetstrokecolor{currentstroke}%
\pgfsetdash{}{0pt}%
\pgfpathmoveto{\pgfqpoint{0.661284in}{1.435918in}}%
\pgfpathlineto{\pgfqpoint{4.026572in}{1.435918in}}%
\pgfusepath{stroke}%
\end{pgfscope}%
\begin{pgfscope}%
\pgfsetbuttcap%
\pgfsetroundjoin%
\definecolor{currentfill}{rgb}{0.000000,0.000000,0.000000}%
\pgfsetfillcolor{currentfill}%
\pgfsetlinewidth{0.602250pt}%
\definecolor{currentstroke}{rgb}{0.000000,0.000000,0.000000}%
\pgfsetstrokecolor{currentstroke}%
\pgfsetdash{}{0pt}%
\pgfsys@defobject{currentmarker}{\pgfqpoint{-0.027778in}{0.000000in}}{\pgfqpoint{-0.000000in}{0.000000in}}{%
\pgfpathmoveto{\pgfqpoint{-0.000000in}{0.000000in}}%
\pgfpathlineto{\pgfqpoint{-0.027778in}{0.000000in}}%
\pgfusepath{stroke,fill}%
}%
\begin{pgfscope}%
\pgfsys@transformshift{0.661284in}{1.435918in}%
\pgfsys@useobject{currentmarker}{}%
\end{pgfscope}%
\end{pgfscope}%
\begin{pgfscope}%
\pgfpathrectangle{\pgfqpoint{0.661284in}{0.417642in}}{\pgfqpoint{3.365288in}{2.055000in}}%
\pgfusepath{clip}%
\pgfsetrectcap%
\pgfsetroundjoin%
\pgfsetlinewidth{0.803000pt}%
\definecolor{currentstroke}{rgb}{0.850000,0.850000,0.850000}%
\pgfsetstrokecolor{currentstroke}%
\pgfsetdash{}{0pt}%
\pgfpathmoveto{\pgfqpoint{0.661284in}{1.647027in}}%
\pgfpathlineto{\pgfqpoint{4.026572in}{1.647027in}}%
\pgfusepath{stroke}%
\end{pgfscope}%
\begin{pgfscope}%
\pgfsetbuttcap%
\pgfsetroundjoin%
\definecolor{currentfill}{rgb}{0.000000,0.000000,0.000000}%
\pgfsetfillcolor{currentfill}%
\pgfsetlinewidth{0.602250pt}%
\definecolor{currentstroke}{rgb}{0.000000,0.000000,0.000000}%
\pgfsetstrokecolor{currentstroke}%
\pgfsetdash{}{0pt}%
\pgfsys@defobject{currentmarker}{\pgfqpoint{-0.027778in}{0.000000in}}{\pgfqpoint{-0.000000in}{0.000000in}}{%
\pgfpathmoveto{\pgfqpoint{-0.000000in}{0.000000in}}%
\pgfpathlineto{\pgfqpoint{-0.027778in}{0.000000in}}%
\pgfusepath{stroke,fill}%
}%
\begin{pgfscope}%
\pgfsys@transformshift{0.661284in}{1.647027in}%
\pgfsys@useobject{currentmarker}{}%
\end{pgfscope}%
\end{pgfscope}%
\begin{pgfscope}%
\pgfpathrectangle{\pgfqpoint{0.661284in}{0.417642in}}{\pgfqpoint{3.365288in}{2.055000in}}%
\pgfusepath{clip}%
\pgfsetrectcap%
\pgfsetroundjoin%
\pgfsetlinewidth{0.803000pt}%
\definecolor{currentstroke}{rgb}{0.850000,0.850000,0.850000}%
\pgfsetstrokecolor{currentstroke}%
\pgfsetdash{}{0pt}%
\pgfpathmoveto{\pgfqpoint{0.661284in}{1.754223in}}%
\pgfpathlineto{\pgfqpoint{4.026572in}{1.754223in}}%
\pgfusepath{stroke}%
\end{pgfscope}%
\begin{pgfscope}%
\pgfsetbuttcap%
\pgfsetroundjoin%
\definecolor{currentfill}{rgb}{0.000000,0.000000,0.000000}%
\pgfsetfillcolor{currentfill}%
\pgfsetlinewidth{0.602250pt}%
\definecolor{currentstroke}{rgb}{0.000000,0.000000,0.000000}%
\pgfsetstrokecolor{currentstroke}%
\pgfsetdash{}{0pt}%
\pgfsys@defobject{currentmarker}{\pgfqpoint{-0.027778in}{0.000000in}}{\pgfqpoint{-0.000000in}{0.000000in}}{%
\pgfpathmoveto{\pgfqpoint{-0.000000in}{0.000000in}}%
\pgfpathlineto{\pgfqpoint{-0.027778in}{0.000000in}}%
\pgfusepath{stroke,fill}%
}%
\begin{pgfscope}%
\pgfsys@transformshift{0.661284in}{1.754223in}%
\pgfsys@useobject{currentmarker}{}%
\end{pgfscope}%
\end{pgfscope}%
\begin{pgfscope}%
\pgfpathrectangle{\pgfqpoint{0.661284in}{0.417642in}}{\pgfqpoint{3.365288in}{2.055000in}}%
\pgfusepath{clip}%
\pgfsetrectcap%
\pgfsetroundjoin%
\pgfsetlinewidth{0.803000pt}%
\definecolor{currentstroke}{rgb}{0.850000,0.850000,0.850000}%
\pgfsetstrokecolor{currentstroke}%
\pgfsetdash{}{0pt}%
\pgfpathmoveto{\pgfqpoint{0.661284in}{1.830280in}}%
\pgfpathlineto{\pgfqpoint{4.026572in}{1.830280in}}%
\pgfusepath{stroke}%
\end{pgfscope}%
\begin{pgfscope}%
\pgfsetbuttcap%
\pgfsetroundjoin%
\definecolor{currentfill}{rgb}{0.000000,0.000000,0.000000}%
\pgfsetfillcolor{currentfill}%
\pgfsetlinewidth{0.602250pt}%
\definecolor{currentstroke}{rgb}{0.000000,0.000000,0.000000}%
\pgfsetstrokecolor{currentstroke}%
\pgfsetdash{}{0pt}%
\pgfsys@defobject{currentmarker}{\pgfqpoint{-0.027778in}{0.000000in}}{\pgfqpoint{-0.000000in}{0.000000in}}{%
\pgfpathmoveto{\pgfqpoint{-0.000000in}{0.000000in}}%
\pgfpathlineto{\pgfqpoint{-0.027778in}{0.000000in}}%
\pgfusepath{stroke,fill}%
}%
\begin{pgfscope}%
\pgfsys@transformshift{0.661284in}{1.830280in}%
\pgfsys@useobject{currentmarker}{}%
\end{pgfscope}%
\end{pgfscope}%
\begin{pgfscope}%
\pgfpathrectangle{\pgfqpoint{0.661284in}{0.417642in}}{\pgfqpoint{3.365288in}{2.055000in}}%
\pgfusepath{clip}%
\pgfsetrectcap%
\pgfsetroundjoin%
\pgfsetlinewidth{0.803000pt}%
\definecolor{currentstroke}{rgb}{0.850000,0.850000,0.850000}%
\pgfsetstrokecolor{currentstroke}%
\pgfsetdash{}{0pt}%
\pgfpathmoveto{\pgfqpoint{0.661284in}{1.889275in}}%
\pgfpathlineto{\pgfqpoint{4.026572in}{1.889275in}}%
\pgfusepath{stroke}%
\end{pgfscope}%
\begin{pgfscope}%
\pgfsetbuttcap%
\pgfsetroundjoin%
\definecolor{currentfill}{rgb}{0.000000,0.000000,0.000000}%
\pgfsetfillcolor{currentfill}%
\pgfsetlinewidth{0.602250pt}%
\definecolor{currentstroke}{rgb}{0.000000,0.000000,0.000000}%
\pgfsetstrokecolor{currentstroke}%
\pgfsetdash{}{0pt}%
\pgfsys@defobject{currentmarker}{\pgfqpoint{-0.027778in}{0.000000in}}{\pgfqpoint{-0.000000in}{0.000000in}}{%
\pgfpathmoveto{\pgfqpoint{-0.000000in}{0.000000in}}%
\pgfpathlineto{\pgfqpoint{-0.027778in}{0.000000in}}%
\pgfusepath{stroke,fill}%
}%
\begin{pgfscope}%
\pgfsys@transformshift{0.661284in}{1.889275in}%
\pgfsys@useobject{currentmarker}{}%
\end{pgfscope}%
\end{pgfscope}%
\begin{pgfscope}%
\pgfpathrectangle{\pgfqpoint{0.661284in}{0.417642in}}{\pgfqpoint{3.365288in}{2.055000in}}%
\pgfusepath{clip}%
\pgfsetrectcap%
\pgfsetroundjoin%
\pgfsetlinewidth{0.803000pt}%
\definecolor{currentstroke}{rgb}{0.850000,0.850000,0.850000}%
\pgfsetstrokecolor{currentstroke}%
\pgfsetdash{}{0pt}%
\pgfpathmoveto{\pgfqpoint{0.661284in}{1.937477in}}%
\pgfpathlineto{\pgfqpoint{4.026572in}{1.937477in}}%
\pgfusepath{stroke}%
\end{pgfscope}%
\begin{pgfscope}%
\pgfsetbuttcap%
\pgfsetroundjoin%
\definecolor{currentfill}{rgb}{0.000000,0.000000,0.000000}%
\pgfsetfillcolor{currentfill}%
\pgfsetlinewidth{0.602250pt}%
\definecolor{currentstroke}{rgb}{0.000000,0.000000,0.000000}%
\pgfsetstrokecolor{currentstroke}%
\pgfsetdash{}{0pt}%
\pgfsys@defobject{currentmarker}{\pgfqpoint{-0.027778in}{0.000000in}}{\pgfqpoint{-0.000000in}{0.000000in}}{%
\pgfpathmoveto{\pgfqpoint{-0.000000in}{0.000000in}}%
\pgfpathlineto{\pgfqpoint{-0.027778in}{0.000000in}}%
\pgfusepath{stroke,fill}%
}%
\begin{pgfscope}%
\pgfsys@transformshift{0.661284in}{1.937477in}%
\pgfsys@useobject{currentmarker}{}%
\end{pgfscope}%
\end{pgfscope}%
\begin{pgfscope}%
\pgfpathrectangle{\pgfqpoint{0.661284in}{0.417642in}}{\pgfqpoint{3.365288in}{2.055000in}}%
\pgfusepath{clip}%
\pgfsetrectcap%
\pgfsetroundjoin%
\pgfsetlinewidth{0.803000pt}%
\definecolor{currentstroke}{rgb}{0.850000,0.850000,0.850000}%
\pgfsetstrokecolor{currentstroke}%
\pgfsetdash{}{0pt}%
\pgfpathmoveto{\pgfqpoint{0.661284in}{1.978231in}}%
\pgfpathlineto{\pgfqpoint{4.026572in}{1.978231in}}%
\pgfusepath{stroke}%
\end{pgfscope}%
\begin{pgfscope}%
\pgfsetbuttcap%
\pgfsetroundjoin%
\definecolor{currentfill}{rgb}{0.000000,0.000000,0.000000}%
\pgfsetfillcolor{currentfill}%
\pgfsetlinewidth{0.602250pt}%
\definecolor{currentstroke}{rgb}{0.000000,0.000000,0.000000}%
\pgfsetstrokecolor{currentstroke}%
\pgfsetdash{}{0pt}%
\pgfsys@defobject{currentmarker}{\pgfqpoint{-0.027778in}{0.000000in}}{\pgfqpoint{-0.000000in}{0.000000in}}{%
\pgfpathmoveto{\pgfqpoint{-0.000000in}{0.000000in}}%
\pgfpathlineto{\pgfqpoint{-0.027778in}{0.000000in}}%
\pgfusepath{stroke,fill}%
}%
\begin{pgfscope}%
\pgfsys@transformshift{0.661284in}{1.978231in}%
\pgfsys@useobject{currentmarker}{}%
\end{pgfscope}%
\end{pgfscope}%
\begin{pgfscope}%
\pgfpathrectangle{\pgfqpoint{0.661284in}{0.417642in}}{\pgfqpoint{3.365288in}{2.055000in}}%
\pgfusepath{clip}%
\pgfsetrectcap%
\pgfsetroundjoin%
\pgfsetlinewidth{0.803000pt}%
\definecolor{currentstroke}{rgb}{0.850000,0.850000,0.850000}%
\pgfsetstrokecolor{currentstroke}%
\pgfsetdash{}{0pt}%
\pgfpathmoveto{\pgfqpoint{0.661284in}{2.013534in}}%
\pgfpathlineto{\pgfqpoint{4.026572in}{2.013534in}}%
\pgfusepath{stroke}%
\end{pgfscope}%
\begin{pgfscope}%
\pgfsetbuttcap%
\pgfsetroundjoin%
\definecolor{currentfill}{rgb}{0.000000,0.000000,0.000000}%
\pgfsetfillcolor{currentfill}%
\pgfsetlinewidth{0.602250pt}%
\definecolor{currentstroke}{rgb}{0.000000,0.000000,0.000000}%
\pgfsetstrokecolor{currentstroke}%
\pgfsetdash{}{0pt}%
\pgfsys@defobject{currentmarker}{\pgfqpoint{-0.027778in}{0.000000in}}{\pgfqpoint{-0.000000in}{0.000000in}}{%
\pgfpathmoveto{\pgfqpoint{-0.000000in}{0.000000in}}%
\pgfpathlineto{\pgfqpoint{-0.027778in}{0.000000in}}%
\pgfusepath{stroke,fill}%
}%
\begin{pgfscope}%
\pgfsys@transformshift{0.661284in}{2.013534in}%
\pgfsys@useobject{currentmarker}{}%
\end{pgfscope}%
\end{pgfscope}%
\begin{pgfscope}%
\pgfpathrectangle{\pgfqpoint{0.661284in}{0.417642in}}{\pgfqpoint{3.365288in}{2.055000in}}%
\pgfusepath{clip}%
\pgfsetrectcap%
\pgfsetroundjoin%
\pgfsetlinewidth{0.803000pt}%
\definecolor{currentstroke}{rgb}{0.850000,0.850000,0.850000}%
\pgfsetstrokecolor{currentstroke}%
\pgfsetdash{}{0pt}%
\pgfpathmoveto{\pgfqpoint{0.661284in}{2.044673in}}%
\pgfpathlineto{\pgfqpoint{4.026572in}{2.044673in}}%
\pgfusepath{stroke}%
\end{pgfscope}%
\begin{pgfscope}%
\pgfsetbuttcap%
\pgfsetroundjoin%
\definecolor{currentfill}{rgb}{0.000000,0.000000,0.000000}%
\pgfsetfillcolor{currentfill}%
\pgfsetlinewidth{0.602250pt}%
\definecolor{currentstroke}{rgb}{0.000000,0.000000,0.000000}%
\pgfsetstrokecolor{currentstroke}%
\pgfsetdash{}{0pt}%
\pgfsys@defobject{currentmarker}{\pgfqpoint{-0.027778in}{0.000000in}}{\pgfqpoint{-0.000000in}{0.000000in}}{%
\pgfpathmoveto{\pgfqpoint{-0.000000in}{0.000000in}}%
\pgfpathlineto{\pgfqpoint{-0.027778in}{0.000000in}}%
\pgfusepath{stroke,fill}%
}%
\begin{pgfscope}%
\pgfsys@transformshift{0.661284in}{2.044673in}%
\pgfsys@useobject{currentmarker}{}%
\end{pgfscope}%
\end{pgfscope}%
\begin{pgfscope}%
\pgfpathrectangle{\pgfqpoint{0.661284in}{0.417642in}}{\pgfqpoint{3.365288in}{2.055000in}}%
\pgfusepath{clip}%
\pgfsetrectcap%
\pgfsetroundjoin%
\pgfsetlinewidth{0.803000pt}%
\definecolor{currentstroke}{rgb}{0.850000,0.850000,0.850000}%
\pgfsetstrokecolor{currentstroke}%
\pgfsetdash{}{0pt}%
\pgfpathmoveto{\pgfqpoint{0.661284in}{2.255782in}}%
\pgfpathlineto{\pgfqpoint{4.026572in}{2.255782in}}%
\pgfusepath{stroke}%
\end{pgfscope}%
\begin{pgfscope}%
\pgfsetbuttcap%
\pgfsetroundjoin%
\definecolor{currentfill}{rgb}{0.000000,0.000000,0.000000}%
\pgfsetfillcolor{currentfill}%
\pgfsetlinewidth{0.602250pt}%
\definecolor{currentstroke}{rgb}{0.000000,0.000000,0.000000}%
\pgfsetstrokecolor{currentstroke}%
\pgfsetdash{}{0pt}%
\pgfsys@defobject{currentmarker}{\pgfqpoint{-0.027778in}{0.000000in}}{\pgfqpoint{-0.000000in}{0.000000in}}{%
\pgfpathmoveto{\pgfqpoint{-0.000000in}{0.000000in}}%
\pgfpathlineto{\pgfqpoint{-0.027778in}{0.000000in}}%
\pgfusepath{stroke,fill}%
}%
\begin{pgfscope}%
\pgfsys@transformshift{0.661284in}{2.255782in}%
\pgfsys@useobject{currentmarker}{}%
\end{pgfscope}%
\end{pgfscope}%
\begin{pgfscope}%
\pgfpathrectangle{\pgfqpoint{0.661284in}{0.417642in}}{\pgfqpoint{3.365288in}{2.055000in}}%
\pgfusepath{clip}%
\pgfsetrectcap%
\pgfsetroundjoin%
\pgfsetlinewidth{0.803000pt}%
\definecolor{currentstroke}{rgb}{0.850000,0.850000,0.850000}%
\pgfsetstrokecolor{currentstroke}%
\pgfsetdash{}{0pt}%
\pgfpathmoveto{\pgfqpoint{0.661284in}{2.362979in}}%
\pgfpathlineto{\pgfqpoint{4.026572in}{2.362979in}}%
\pgfusepath{stroke}%
\end{pgfscope}%
\begin{pgfscope}%
\pgfsetbuttcap%
\pgfsetroundjoin%
\definecolor{currentfill}{rgb}{0.000000,0.000000,0.000000}%
\pgfsetfillcolor{currentfill}%
\pgfsetlinewidth{0.602250pt}%
\definecolor{currentstroke}{rgb}{0.000000,0.000000,0.000000}%
\pgfsetstrokecolor{currentstroke}%
\pgfsetdash{}{0pt}%
\pgfsys@defobject{currentmarker}{\pgfqpoint{-0.027778in}{0.000000in}}{\pgfqpoint{-0.000000in}{0.000000in}}{%
\pgfpathmoveto{\pgfqpoint{-0.000000in}{0.000000in}}%
\pgfpathlineto{\pgfqpoint{-0.027778in}{0.000000in}}%
\pgfusepath{stroke,fill}%
}%
\begin{pgfscope}%
\pgfsys@transformshift{0.661284in}{2.362979in}%
\pgfsys@useobject{currentmarker}{}%
\end{pgfscope}%
\end{pgfscope}%
\begin{pgfscope}%
\pgfpathrectangle{\pgfqpoint{0.661284in}{0.417642in}}{\pgfqpoint{3.365288in}{2.055000in}}%
\pgfusepath{clip}%
\pgfsetrectcap%
\pgfsetroundjoin%
\pgfsetlinewidth{0.803000pt}%
\definecolor{currentstroke}{rgb}{0.850000,0.850000,0.850000}%
\pgfsetstrokecolor{currentstroke}%
\pgfsetdash{}{0pt}%
\pgfpathmoveto{\pgfqpoint{0.661284in}{2.439036in}}%
\pgfpathlineto{\pgfqpoint{4.026572in}{2.439036in}}%
\pgfusepath{stroke}%
\end{pgfscope}%
\begin{pgfscope}%
\pgfsetbuttcap%
\pgfsetroundjoin%
\definecolor{currentfill}{rgb}{0.000000,0.000000,0.000000}%
\pgfsetfillcolor{currentfill}%
\pgfsetlinewidth{0.602250pt}%
\definecolor{currentstroke}{rgb}{0.000000,0.000000,0.000000}%
\pgfsetstrokecolor{currentstroke}%
\pgfsetdash{}{0pt}%
\pgfsys@defobject{currentmarker}{\pgfqpoint{-0.027778in}{0.000000in}}{\pgfqpoint{-0.000000in}{0.000000in}}{%
\pgfpathmoveto{\pgfqpoint{-0.000000in}{0.000000in}}%
\pgfpathlineto{\pgfqpoint{-0.027778in}{0.000000in}}%
\pgfusepath{stroke,fill}%
}%
\begin{pgfscope}%
\pgfsys@transformshift{0.661284in}{2.439036in}%
\pgfsys@useobject{currentmarker}{}%
\end{pgfscope}%
\end{pgfscope}%
\begin{pgfscope}%
\definecolor{textcolor}{rgb}{0.000000,0.000000,0.000000}%
\pgfsetstrokecolor{textcolor}%
\pgfsetfillcolor{textcolor}%
\pgftext[x=0.201408in,y=1.445142in,,bottom,rotate=90.000000]{\color{textcolor}\rmfamily\fontsize{10.000000}{12.000000}\selectfont  \(\displaystyle S_y(f)\) in \(\displaystyle \unit{\V^2 \per \Hz}\)}%
\end{pgfscope}%
\begin{pgfscope}%
\pgfpathrectangle{\pgfqpoint{0.661284in}{0.417642in}}{\pgfqpoint{3.365288in}{2.055000in}}%
\pgfusepath{clip}%
\pgfsetbuttcap%
\pgfsetroundjoin%
\definecolor{currentfill}{rgb}{0.925490,0.882353,0.200000}%
\pgfsetfillcolor{currentfill}%
\pgfsetlinewidth{1.003750pt}%
\definecolor{currentstroke}{rgb}{0.925490,0.882353,0.200000}%
\pgfsetstrokecolor{currentstroke}%
\pgfsetdash{}{0pt}%
\pgfsys@defobject{currentmarker}{\pgfqpoint{-0.013889in}{-0.013889in}}{\pgfqpoint{0.013889in}{0.013889in}}{%
\pgfpathmoveto{\pgfqpoint{0.000000in}{-0.013889in}}%
\pgfpathcurveto{\pgfqpoint{0.003683in}{-0.013889in}}{\pgfqpoint{0.007216in}{-0.012425in}}{\pgfqpoint{0.009821in}{-0.009821in}}%
\pgfpathcurveto{\pgfqpoint{0.012425in}{-0.007216in}}{\pgfqpoint{0.013889in}{-0.003683in}}{\pgfqpoint{0.013889in}{0.000000in}}%
\pgfpathcurveto{\pgfqpoint{0.013889in}{0.003683in}}{\pgfqpoint{0.012425in}{0.007216in}}{\pgfqpoint{0.009821in}{0.009821in}}%
\pgfpathcurveto{\pgfqpoint{0.007216in}{0.012425in}}{\pgfqpoint{0.003683in}{0.013889in}}{\pgfqpoint{0.000000in}{0.013889in}}%
\pgfpathcurveto{\pgfqpoint{-0.003683in}{0.013889in}}{\pgfqpoint{-0.007216in}{0.012425in}}{\pgfqpoint{-0.009821in}{0.009821in}}%
\pgfpathcurveto{\pgfqpoint{-0.012425in}{0.007216in}}{\pgfqpoint{-0.013889in}{0.003683in}}{\pgfqpoint{-0.013889in}{0.000000in}}%
\pgfpathcurveto{\pgfqpoint{-0.013889in}{-0.003683in}}{\pgfqpoint{-0.012425in}{-0.007216in}}{\pgfqpoint{-0.009821in}{-0.009821in}}%
\pgfpathcurveto{\pgfqpoint{-0.007216in}{-0.012425in}}{\pgfqpoint{-0.003683in}{-0.013889in}}{\pgfqpoint{0.000000in}{-0.013889in}}%
\pgfpathlineto{\pgfqpoint{0.000000in}{-0.013889in}}%
\pgfpathclose%
\pgfusepath{stroke,fill}%
}%
\begin{pgfscope}%
\pgfsys@transformshift{-226.701573in}{0.917118in}%
\pgfsys@useobject{currentmarker}{}%
\end{pgfscope}%
\begin{pgfscope}%
\pgfsys@transformshift{0.814251in}{1.014274in}%
\pgfsys@useobject{currentmarker}{}%
\end{pgfscope}%
\begin{pgfscope}%
\pgfsys@transformshift{1.122320in}{1.068466in}%
\pgfsys@useobject{currentmarker}{}%
\end{pgfscope}%
\begin{pgfscope}%
\pgfsys@transformshift{1.302529in}{1.066637in}%
\pgfsys@useobject{currentmarker}{}%
\end{pgfscope}%
\begin{pgfscope}%
\pgfsys@transformshift{1.430389in}{1.052853in}%
\pgfsys@useobject{currentmarker}{}%
\end{pgfscope}%
\begin{pgfscope}%
\pgfsys@transformshift{1.529565in}{1.053667in}%
\pgfsys@useobject{currentmarker}{}%
\end{pgfscope}%
\begin{pgfscope}%
\pgfsys@transformshift{1.610598in}{1.062301in}%
\pgfsys@useobject{currentmarker}{}%
\end{pgfscope}%
\begin{pgfscope}%
\pgfsys@transformshift{1.679110in}{1.064726in}%
\pgfsys@useobject{currentmarker}{}%
\end{pgfscope}%
\begin{pgfscope}%
\pgfsys@transformshift{1.738458in}{1.058176in}%
\pgfsys@useobject{currentmarker}{}%
\end{pgfscope}%
\begin{pgfscope}%
\pgfsys@transformshift{1.790807in}{1.068691in}%
\pgfsys@useobject{currentmarker}{}%
\end{pgfscope}%
\begin{pgfscope}%
\pgfsys@transformshift{1.837634in}{1.073089in}%
\pgfsys@useobject{currentmarker}{}%
\end{pgfscope}%
\begin{pgfscope}%
\pgfsys@transformshift{1.879995in}{1.042711in}%
\pgfsys@useobject{currentmarker}{}%
\end{pgfscope}%
\begin{pgfscope}%
\pgfsys@transformshift{1.918667in}{1.037694in}%
\pgfsys@useobject{currentmarker}{}%
\end{pgfscope}%
\begin{pgfscope}%
\pgfsys@transformshift{1.954242in}{1.036308in}%
\pgfsys@useobject{currentmarker}{}%
\end{pgfscope}%
\begin{pgfscope}%
\pgfsys@transformshift{1.987179in}{1.036246in}%
\pgfsys@useobject{currentmarker}{}%
\end{pgfscope}%
\begin{pgfscope}%
\pgfsys@transformshift{2.017843in}{1.035211in}%
\pgfsys@useobject{currentmarker}{}%
\end{pgfscope}%
\begin{pgfscope}%
\pgfsys@transformshift{2.046527in}{1.047195in}%
\pgfsys@useobject{currentmarker}{}%
\end{pgfscope}%
\begin{pgfscope}%
\pgfsys@transformshift{2.073472in}{1.054803in}%
\pgfsys@useobject{currentmarker}{}%
\end{pgfscope}%
\begin{pgfscope}%
\pgfsys@transformshift{2.098876in}{1.057923in}%
\pgfsys@useobject{currentmarker}{}%
\end{pgfscope}%
\begin{pgfscope}%
\pgfsys@transformshift{2.122906in}{1.056611in}%
\pgfsys@useobject{currentmarker}{}%
\end{pgfscope}%
\begin{pgfscope}%
\pgfsys@transformshift{2.145703in}{1.055027in}%
\pgfsys@useobject{currentmarker}{}%
\end{pgfscope}%
\begin{pgfscope}%
\pgfsys@transformshift{2.167388in}{1.057073in}%
\pgfsys@useobject{currentmarker}{}%
\end{pgfscope}%
\begin{pgfscope}%
\pgfsys@transformshift{2.188064in}{1.057267in}%
\pgfsys@useobject{currentmarker}{}%
\end{pgfscope}%
\begin{pgfscope}%
\pgfsys@transformshift{2.207821in}{1.062110in}%
\pgfsys@useobject{currentmarker}{}%
\end{pgfscope}%
\begin{pgfscope}%
\pgfsys@transformshift{2.226736in}{1.049090in}%
\pgfsys@useobject{currentmarker}{}%
\end{pgfscope}%
\begin{pgfscope}%
\pgfsys@transformshift{2.244880in}{1.048645in}%
\pgfsys@useobject{currentmarker}{}%
\end{pgfscope}%
\begin{pgfscope}%
\pgfsys@transformshift{2.262311in}{1.039161in}%
\pgfsys@useobject{currentmarker}{}%
\end{pgfscope}%
\begin{pgfscope}%
\pgfsys@transformshift{2.279085in}{1.058768in}%
\pgfsys@useobject{currentmarker}{}%
\end{pgfscope}%
\begin{pgfscope}%
\pgfsys@transformshift{2.295248in}{1.065334in}%
\pgfsys@useobject{currentmarker}{}%
\end{pgfscope}%
\begin{pgfscope}%
\pgfsys@transformshift{2.310845in}{1.080489in}%
\pgfsys@useobject{currentmarker}{}%
\end{pgfscope}%
\begin{pgfscope}%
\pgfsys@transformshift{2.325912in}{1.061885in}%
\pgfsys@useobject{currentmarker}{}%
\end{pgfscope}%
\begin{pgfscope}%
\pgfsys@transformshift{2.340486in}{1.047184in}%
\pgfsys@useobject{currentmarker}{}%
\end{pgfscope}%
\begin{pgfscope}%
\pgfsys@transformshift{2.354596in}{1.067414in}%
\pgfsys@useobject{currentmarker}{}%
\end{pgfscope}%
\begin{pgfscope}%
\pgfsys@transformshift{2.368273in}{1.075823in}%
\pgfsys@useobject{currentmarker}{}%
\end{pgfscope}%
\begin{pgfscope}%
\pgfsys@transformshift{2.381541in}{1.076550in}%
\pgfsys@useobject{currentmarker}{}%
\end{pgfscope}%
\begin{pgfscope}%
\pgfsys@transformshift{2.394425in}{1.073870in}%
\pgfsys@useobject{currentmarker}{}%
\end{pgfscope}%
\begin{pgfscope}%
\pgfsys@transformshift{2.406945in}{1.049516in}%
\pgfsys@useobject{currentmarker}{}%
\end{pgfscope}%
\begin{pgfscope}%
\pgfsys@transformshift{2.419123in}{1.041498in}%
\pgfsys@useobject{currentmarker}{}%
\end{pgfscope}%
\begin{pgfscope}%
\pgfsys@transformshift{2.430975in}{1.045516in}%
\pgfsys@useobject{currentmarker}{}%
\end{pgfscope}%
\begin{pgfscope}%
\pgfsys@transformshift{2.442520in}{1.061690in}%
\pgfsys@useobject{currentmarker}{}%
\end{pgfscope}%
\begin{pgfscope}%
\pgfsys@transformshift{2.453773in}{1.052828in}%
\pgfsys@useobject{currentmarker}{}%
\end{pgfscope}%
\begin{pgfscope}%
\pgfsys@transformshift{2.464747in}{1.044787in}%
\pgfsys@useobject{currentmarker}{}%
\end{pgfscope}%
\begin{pgfscope}%
\pgfsys@transformshift{2.475457in}{1.063407in}%
\pgfsys@useobject{currentmarker}{}%
\end{pgfscope}%
\begin{pgfscope}%
\pgfsys@transformshift{2.485915in}{1.071297in}%
\pgfsys@useobject{currentmarker}{}%
\end{pgfscope}%
\begin{pgfscope}%
\pgfsys@transformshift{2.496133in}{1.062101in}%
\pgfsys@useobject{currentmarker}{}%
\end{pgfscope}%
\begin{pgfscope}%
\pgfsys@transformshift{2.506121in}{1.060626in}%
\pgfsys@useobject{currentmarker}{}%
\end{pgfscope}%
\begin{pgfscope}%
\pgfsys@transformshift{2.515890in}{1.066415in}%
\pgfsys@useobject{currentmarker}{}%
\end{pgfscope}%
\begin{pgfscope}%
\pgfsys@transformshift{2.525448in}{1.061172in}%
\pgfsys@useobject{currentmarker}{}%
\end{pgfscope}%
\begin{pgfscope}%
\pgfsys@transformshift{2.534805in}{1.046259in}%
\pgfsys@useobject{currentmarker}{}%
\end{pgfscope}%
\begin{pgfscope}%
\pgfsys@transformshift{2.543970in}{1.044821in}%
\pgfsys@useobject{currentmarker}{}%
\end{pgfscope}%
\begin{pgfscope}%
\pgfsys@transformshift{2.552949in}{1.045016in}%
\pgfsys@useobject{currentmarker}{}%
\end{pgfscope}%
\begin{pgfscope}%
\pgfsys@transformshift{2.561750in}{1.053191in}%
\pgfsys@useobject{currentmarker}{}%
\end{pgfscope}%
\begin{pgfscope}%
\pgfsys@transformshift{2.570380in}{1.052134in}%
\pgfsys@useobject{currentmarker}{}%
\end{pgfscope}%
\begin{pgfscope}%
\pgfsys@transformshift{2.578846in}{1.056648in}%
\pgfsys@useobject{currentmarker}{}%
\end{pgfscope}%
\begin{pgfscope}%
\pgfsys@transformshift{2.587154in}{1.076210in}%
\pgfsys@useobject{currentmarker}{}%
\end{pgfscope}%
\begin{pgfscope}%
\pgfsys@transformshift{2.595309in}{1.084467in}%
\pgfsys@useobject{currentmarker}{}%
\end{pgfscope}%
\begin{pgfscope}%
\pgfsys@transformshift{2.603317in}{1.073348in}%
\pgfsys@useobject{currentmarker}{}%
\end{pgfscope}%
\begin{pgfscope}%
\pgfsys@transformshift{2.611184in}{1.064785in}%
\pgfsys@useobject{currentmarker}{}%
\end{pgfscope}%
\begin{pgfscope}%
\pgfsys@transformshift{2.618914in}{1.048245in}%
\pgfsys@useobject{currentmarker}{}%
\end{pgfscope}%
\begin{pgfscope}%
\pgfsys@transformshift{2.626511in}{1.052333in}%
\pgfsys@useobject{currentmarker}{}%
\end{pgfscope}%
\begin{pgfscope}%
\pgfsys@transformshift{2.633981in}{1.051798in}%
\pgfsys@useobject{currentmarker}{}%
\end{pgfscope}%
\begin{pgfscope}%
\pgfsys@transformshift{2.641328in}{1.054553in}%
\pgfsys@useobject{currentmarker}{}%
\end{pgfscope}%
\begin{pgfscope}%
\pgfsys@transformshift{2.648555in}{1.048005in}%
\pgfsys@useobject{currentmarker}{}%
\end{pgfscope}%
\begin{pgfscope}%
\pgfsys@transformshift{2.655666in}{1.067959in}%
\pgfsys@useobject{currentmarker}{}%
\end{pgfscope}%
\begin{pgfscope}%
\pgfsys@transformshift{2.662665in}{1.047972in}%
\pgfsys@useobject{currentmarker}{}%
\end{pgfscope}%
\begin{pgfscope}%
\pgfsys@transformshift{2.669556in}{1.062115in}%
\pgfsys@useobject{currentmarker}{}%
\end{pgfscope}%
\begin{pgfscope}%
\pgfsys@transformshift{2.676342in}{1.063700in}%
\pgfsys@useobject{currentmarker}{}%
\end{pgfscope}%
\begin{pgfscope}%
\pgfsys@transformshift{2.683026in}{1.065577in}%
\pgfsys@useobject{currentmarker}{}%
\end{pgfscope}%
\begin{pgfscope}%
\pgfsys@transformshift{2.689610in}{1.047080in}%
\pgfsys@useobject{currentmarker}{}%
\end{pgfscope}%
\begin{pgfscope}%
\pgfsys@transformshift{2.696099in}{1.062420in}%
\pgfsys@useobject{currentmarker}{}%
\end{pgfscope}%
\begin{pgfscope}%
\pgfsys@transformshift{2.702494in}{1.061566in}%
\pgfsys@useobject{currentmarker}{}%
\end{pgfscope}%
\begin{pgfscope}%
\pgfsys@transformshift{2.708798in}{1.049075in}%
\pgfsys@useobject{currentmarker}{}%
\end{pgfscope}%
\begin{pgfscope}%
\pgfsys@transformshift{2.715014in}{1.060910in}%
\pgfsys@useobject{currentmarker}{}%
\end{pgfscope}%
\begin{pgfscope}%
\pgfsys@transformshift{2.721145in}{1.059531in}%
\pgfsys@useobject{currentmarker}{}%
\end{pgfscope}%
\begin{pgfscope}%
\pgfsys@transformshift{2.727192in}{1.061459in}%
\pgfsys@useobject{currentmarker}{}%
\end{pgfscope}%
\begin{pgfscope}%
\pgfsys@transformshift{2.733157in}{1.063544in}%
\pgfsys@useobject{currentmarker}{}%
\end{pgfscope}%
\begin{pgfscope}%
\pgfsys@transformshift{2.739044in}{1.044163in}%
\pgfsys@useobject{currentmarker}{}%
\end{pgfscope}%
\begin{pgfscope}%
\pgfsys@transformshift{2.744854in}{1.073665in}%
\pgfsys@useobject{currentmarker}{}%
\end{pgfscope}%
\begin{pgfscope}%
\pgfsys@transformshift{2.750589in}{1.051730in}%
\pgfsys@useobject{currentmarker}{}%
\end{pgfscope}%
\begin{pgfscope}%
\pgfsys@transformshift{2.756251in}{1.033932in}%
\pgfsys@useobject{currentmarker}{}%
\end{pgfscope}%
\begin{pgfscope}%
\pgfsys@transformshift{2.761842in}{1.058967in}%
\pgfsys@useobject{currentmarker}{}%
\end{pgfscope}%
\begin{pgfscope}%
\pgfsys@transformshift{2.767363in}{1.065441in}%
\pgfsys@useobject{currentmarker}{}%
\end{pgfscope}%
\begin{pgfscope}%
\pgfsys@transformshift{2.772816in}{1.072727in}%
\pgfsys@useobject{currentmarker}{}%
\end{pgfscope}%
\begin{pgfscope}%
\pgfsys@transformshift{2.778204in}{1.055902in}%
\pgfsys@useobject{currentmarker}{}%
\end{pgfscope}%
\begin{pgfscope}%
\pgfsys@transformshift{2.783526in}{1.051872in}%
\pgfsys@useobject{currentmarker}{}%
\end{pgfscope}%
\begin{pgfscope}%
\pgfsys@transformshift{2.788786in}{1.069382in}%
\pgfsys@useobject{currentmarker}{}%
\end{pgfscope}%
\begin{pgfscope}%
\pgfsys@transformshift{2.793984in}{1.075077in}%
\pgfsys@useobject{currentmarker}{}%
\end{pgfscope}%
\begin{pgfscope}%
\pgfsys@transformshift{2.799123in}{1.081030in}%
\pgfsys@useobject{currentmarker}{}%
\end{pgfscope}%
\begin{pgfscope}%
\pgfsys@transformshift{2.804202in}{1.068756in}%
\pgfsys@useobject{currentmarker}{}%
\end{pgfscope}%
\begin{pgfscope}%
\pgfsys@transformshift{2.809224in}{1.072286in}%
\pgfsys@useobject{currentmarker}{}%
\end{pgfscope}%
\begin{pgfscope}%
\pgfsys@transformshift{2.814190in}{1.076191in}%
\pgfsys@useobject{currentmarker}{}%
\end{pgfscope}%
\begin{pgfscope}%
\pgfsys@transformshift{2.819101in}{1.055897in}%
\pgfsys@useobject{currentmarker}{}%
\end{pgfscope}%
\begin{pgfscope}%
\pgfsys@transformshift{2.823959in}{1.054285in}%
\pgfsys@useobject{currentmarker}{}%
\end{pgfscope}%
\begin{pgfscope}%
\pgfsys@transformshift{2.828764in}{1.077242in}%
\pgfsys@useobject{currentmarker}{}%
\end{pgfscope}%
\begin{pgfscope}%
\pgfsys@transformshift{2.833517in}{1.071546in}%
\pgfsys@useobject{currentmarker}{}%
\end{pgfscope}%
\begin{pgfscope}%
\pgfsys@transformshift{2.838220in}{1.063512in}%
\pgfsys@useobject{currentmarker}{}%
\end{pgfscope}%
\begin{pgfscope}%
\pgfsys@transformshift{2.842874in}{1.071746in}%
\pgfsys@useobject{currentmarker}{}%
\end{pgfscope}%
\begin{pgfscope}%
\pgfsys@transformshift{2.847480in}{1.076255in}%
\pgfsys@useobject{currentmarker}{}%
\end{pgfscope}%
\begin{pgfscope}%
\pgfsys@transformshift{2.852039in}{1.061833in}%
\pgfsys@useobject{currentmarker}{}%
\end{pgfscope}%
\begin{pgfscope}%
\pgfsys@transformshift{2.856551in}{1.055450in}%
\pgfsys@useobject{currentmarker}{}%
\end{pgfscope}%
\begin{pgfscope}%
\pgfsys@transformshift{2.861018in}{1.064027in}%
\pgfsys@useobject{currentmarker}{}%
\end{pgfscope}%
\begin{pgfscope}%
\pgfsys@transformshift{2.865440in}{1.075302in}%
\pgfsys@useobject{currentmarker}{}%
\end{pgfscope}%
\begin{pgfscope}%
\pgfsys@transformshift{2.869819in}{1.073672in}%
\pgfsys@useobject{currentmarker}{}%
\end{pgfscope}%
\begin{pgfscope}%
\pgfsys@transformshift{2.874155in}{1.078225in}%
\pgfsys@useobject{currentmarker}{}%
\end{pgfscope}%
\begin{pgfscope}%
\pgfsys@transformshift{2.878449in}{1.067332in}%
\pgfsys@useobject{currentmarker}{}%
\end{pgfscope}%
\begin{pgfscope}%
\pgfsys@transformshift{2.882702in}{1.051291in}%
\pgfsys@useobject{currentmarker}{}%
\end{pgfscope}%
\begin{pgfscope}%
\pgfsys@transformshift{2.886915in}{1.061893in}%
\pgfsys@useobject{currentmarker}{}%
\end{pgfscope}%
\begin{pgfscope}%
\pgfsys@transformshift{2.891089in}{1.057770in}%
\pgfsys@useobject{currentmarker}{}%
\end{pgfscope}%
\begin{pgfscope}%
\pgfsys@transformshift{2.895223in}{1.047599in}%
\pgfsys@useobject{currentmarker}{}%
\end{pgfscope}%
\begin{pgfscope}%
\pgfsys@transformshift{2.899319in}{1.055634in}%
\pgfsys@useobject{currentmarker}{}%
\end{pgfscope}%
\begin{pgfscope}%
\pgfsys@transformshift{2.903378in}{1.069063in}%
\pgfsys@useobject{currentmarker}{}%
\end{pgfscope}%
\begin{pgfscope}%
\pgfsys@transformshift{2.907400in}{1.075029in}%
\pgfsys@useobject{currentmarker}{}%
\end{pgfscope}%
\begin{pgfscope}%
\pgfsys@transformshift{2.911387in}{1.071409in}%
\pgfsys@useobject{currentmarker}{}%
\end{pgfscope}%
\begin{pgfscope}%
\pgfsys@transformshift{2.915337in}{1.072798in}%
\pgfsys@useobject{currentmarker}{}%
\end{pgfscope}%
\begin{pgfscope}%
\pgfsys@transformshift{2.919253in}{1.061453in}%
\pgfsys@useobject{currentmarker}{}%
\end{pgfscope}%
\begin{pgfscope}%
\pgfsys@transformshift{2.923135in}{1.059197in}%
\pgfsys@useobject{currentmarker}{}%
\end{pgfscope}%
\begin{pgfscope}%
\pgfsys@transformshift{2.926983in}{1.051366in}%
\pgfsys@useobject{currentmarker}{}%
\end{pgfscope}%
\begin{pgfscope}%
\pgfsys@transformshift{2.930798in}{1.056414in}%
\pgfsys@useobject{currentmarker}{}%
\end{pgfscope}%
\begin{pgfscope}%
\pgfsys@transformshift{2.934580in}{1.059224in}%
\pgfsys@useobject{currentmarker}{}%
\end{pgfscope}%
\begin{pgfscope}%
\pgfsys@transformshift{2.938331in}{1.069699in}%
\pgfsys@useobject{currentmarker}{}%
\end{pgfscope}%
\begin{pgfscope}%
\pgfsys@transformshift{2.942050in}{1.067595in}%
\pgfsys@useobject{currentmarker}{}%
\end{pgfscope}%
\begin{pgfscope}%
\pgfsys@transformshift{2.945739in}{1.074243in}%
\pgfsys@useobject{currentmarker}{}%
\end{pgfscope}%
\begin{pgfscope}%
\pgfsys@transformshift{2.949397in}{1.089225in}%
\pgfsys@useobject{currentmarker}{}%
\end{pgfscope}%
\begin{pgfscope}%
\pgfsys@transformshift{2.953025in}{1.075117in}%
\pgfsys@useobject{currentmarker}{}%
\end{pgfscope}%
\begin{pgfscope}%
\pgfsys@transformshift{2.956624in}{1.066562in}%
\pgfsys@useobject{currentmarker}{}%
\end{pgfscope}%
\begin{pgfscope}%
\pgfsys@transformshift{2.960194in}{1.057137in}%
\pgfsys@useobject{currentmarker}{}%
\end{pgfscope}%
\begin{pgfscope}%
\pgfsys@transformshift{2.963735in}{1.054922in}%
\pgfsys@useobject{currentmarker}{}%
\end{pgfscope}%
\begin{pgfscope}%
\pgfsys@transformshift{2.967249in}{1.071743in}%
\pgfsys@useobject{currentmarker}{}%
\end{pgfscope}%
\begin{pgfscope}%
\pgfsys@transformshift{2.970735in}{1.052476in}%
\pgfsys@useobject{currentmarker}{}%
\end{pgfscope}%
\begin{pgfscope}%
\pgfsys@transformshift{2.974193in}{1.058056in}%
\pgfsys@useobject{currentmarker}{}%
\end{pgfscope}%
\begin{pgfscope}%
\pgfsys@transformshift{2.977625in}{1.040740in}%
\pgfsys@useobject{currentmarker}{}%
\end{pgfscope}%
\begin{pgfscope}%
\pgfsys@transformshift{2.981031in}{1.048299in}%
\pgfsys@useobject{currentmarker}{}%
\end{pgfscope}%
\begin{pgfscope}%
\pgfsys@transformshift{2.984411in}{1.065221in}%
\pgfsys@useobject{currentmarker}{}%
\end{pgfscope}%
\begin{pgfscope}%
\pgfsys@transformshift{2.987765in}{1.078460in}%
\pgfsys@useobject{currentmarker}{}%
\end{pgfscope}%
\begin{pgfscope}%
\pgfsys@transformshift{2.991095in}{1.084598in}%
\pgfsys@useobject{currentmarker}{}%
\end{pgfscope}%
\begin{pgfscope}%
\pgfsys@transformshift{2.994399in}{1.065699in}%
\pgfsys@useobject{currentmarker}{}%
\end{pgfscope}%
\begin{pgfscope}%
\pgfsys@transformshift{2.997679in}{1.030369in}%
\pgfsys@useobject{currentmarker}{}%
\end{pgfscope}%
\begin{pgfscope}%
\pgfsys@transformshift{3.000935in}{1.063405in}%
\pgfsys@useobject{currentmarker}{}%
\end{pgfscope}%
\begin{pgfscope}%
\pgfsys@transformshift{3.004168in}{1.057613in}%
\pgfsys@useobject{currentmarker}{}%
\end{pgfscope}%
\begin{pgfscope}%
\pgfsys@transformshift{3.007377in}{1.065702in}%
\pgfsys@useobject{currentmarker}{}%
\end{pgfscope}%
\begin{pgfscope}%
\pgfsys@transformshift{3.010563in}{1.066944in}%
\pgfsys@useobject{currentmarker}{}%
\end{pgfscope}%
\begin{pgfscope}%
\pgfsys@transformshift{3.013726in}{1.053117in}%
\pgfsys@useobject{currentmarker}{}%
\end{pgfscope}%
\begin{pgfscope}%
\pgfsys@transformshift{3.016867in}{1.052756in}%
\pgfsys@useobject{currentmarker}{}%
\end{pgfscope}%
\begin{pgfscope}%
\pgfsys@transformshift{3.019986in}{1.048333in}%
\pgfsys@useobject{currentmarker}{}%
\end{pgfscope}%
\begin{pgfscope}%
\pgfsys@transformshift{3.023083in}{1.057898in}%
\pgfsys@useobject{currentmarker}{}%
\end{pgfscope}%
\begin{pgfscope}%
\pgfsys@transformshift{3.026159in}{1.071922in}%
\pgfsys@useobject{currentmarker}{}%
\end{pgfscope}%
\begin{pgfscope}%
\pgfsys@transformshift{3.029214in}{1.043931in}%
\pgfsys@useobject{currentmarker}{}%
\end{pgfscope}%
\begin{pgfscope}%
\pgfsys@transformshift{3.032247in}{1.038918in}%
\pgfsys@useobject{currentmarker}{}%
\end{pgfscope}%
\begin{pgfscope}%
\pgfsys@transformshift{3.035261in}{1.061098in}%
\pgfsys@useobject{currentmarker}{}%
\end{pgfscope}%
\begin{pgfscope}%
\pgfsys@transformshift{3.038254in}{1.055273in}%
\pgfsys@useobject{currentmarker}{}%
\end{pgfscope}%
\begin{pgfscope}%
\pgfsys@transformshift{3.041226in}{1.064357in}%
\pgfsys@useobject{currentmarker}{}%
\end{pgfscope}%
\begin{pgfscope}%
\pgfsys@transformshift{3.044180in}{1.086191in}%
\pgfsys@useobject{currentmarker}{}%
\end{pgfscope}%
\begin{pgfscope}%
\pgfsys@transformshift{3.047113in}{1.073221in}%
\pgfsys@useobject{currentmarker}{}%
\end{pgfscope}%
\begin{pgfscope}%
\pgfsys@transformshift{3.050028in}{1.080876in}%
\pgfsys@useobject{currentmarker}{}%
\end{pgfscope}%
\begin{pgfscope}%
\pgfsys@transformshift{3.052923in}{1.077483in}%
\pgfsys@useobject{currentmarker}{}%
\end{pgfscope}%
\begin{pgfscope}%
\pgfsys@transformshift{3.055800in}{1.052379in}%
\pgfsys@useobject{currentmarker}{}%
\end{pgfscope}%
\begin{pgfscope}%
\pgfsys@transformshift{3.058658in}{1.063229in}%
\pgfsys@useobject{currentmarker}{}%
\end{pgfscope}%
\begin{pgfscope}%
\pgfsys@transformshift{3.061498in}{1.049668in}%
\pgfsys@useobject{currentmarker}{}%
\end{pgfscope}%
\begin{pgfscope}%
\pgfsys@transformshift{3.064320in}{1.041062in}%
\pgfsys@useobject{currentmarker}{}%
\end{pgfscope}%
\begin{pgfscope}%
\pgfsys@transformshift{3.067124in}{1.062297in}%
\pgfsys@useobject{currentmarker}{}%
\end{pgfscope}%
\begin{pgfscope}%
\pgfsys@transformshift{3.069911in}{1.073231in}%
\pgfsys@useobject{currentmarker}{}%
\end{pgfscope}%
\begin{pgfscope}%
\pgfsys@transformshift{3.072680in}{1.070836in}%
\pgfsys@useobject{currentmarker}{}%
\end{pgfscope}%
\begin{pgfscope}%
\pgfsys@transformshift{3.075432in}{1.053913in}%
\pgfsys@useobject{currentmarker}{}%
\end{pgfscope}%
\begin{pgfscope}%
\pgfsys@transformshift{3.078167in}{1.052484in}%
\pgfsys@useobject{currentmarker}{}%
\end{pgfscope}%
\begin{pgfscope}%
\pgfsys@transformshift{3.080885in}{1.050885in}%
\pgfsys@useobject{currentmarker}{}%
\end{pgfscope}%
\begin{pgfscope}%
\pgfsys@transformshift{3.083587in}{1.058893in}%
\pgfsys@useobject{currentmarker}{}%
\end{pgfscope}%
\begin{pgfscope}%
\pgfsys@transformshift{3.086273in}{1.080966in}%
\pgfsys@useobject{currentmarker}{}%
\end{pgfscope}%
\begin{pgfscope}%
\pgfsys@transformshift{3.088942in}{1.071027in}%
\pgfsys@useobject{currentmarker}{}%
\end{pgfscope}%
\begin{pgfscope}%
\pgfsys@transformshift{3.091595in}{1.088250in}%
\pgfsys@useobject{currentmarker}{}%
\end{pgfscope}%
\begin{pgfscope}%
\pgfsys@transformshift{3.094233in}{1.092900in}%
\pgfsys@useobject{currentmarker}{}%
\end{pgfscope}%
\begin{pgfscope}%
\pgfsys@transformshift{3.096855in}{1.057005in}%
\pgfsys@useobject{currentmarker}{}%
\end{pgfscope}%
\begin{pgfscope}%
\pgfsys@transformshift{3.099462in}{1.072022in}%
\pgfsys@useobject{currentmarker}{}%
\end{pgfscope}%
\begin{pgfscope}%
\pgfsys@transformshift{3.102053in}{1.059525in}%
\pgfsys@useobject{currentmarker}{}%
\end{pgfscope}%
\begin{pgfscope}%
\pgfsys@transformshift{3.104630in}{1.064927in}%
\pgfsys@useobject{currentmarker}{}%
\end{pgfscope}%
\begin{pgfscope}%
\pgfsys@transformshift{3.107192in}{1.069123in}%
\pgfsys@useobject{currentmarker}{}%
\end{pgfscope}%
\begin{pgfscope}%
\pgfsys@transformshift{3.109739in}{1.067920in}%
\pgfsys@useobject{currentmarker}{}%
\end{pgfscope}%
\begin{pgfscope}%
\pgfsys@transformshift{3.112271in}{1.061654in}%
\pgfsys@useobject{currentmarker}{}%
\end{pgfscope}%
\begin{pgfscope}%
\pgfsys@transformshift{3.114789in}{1.070740in}%
\pgfsys@useobject{currentmarker}{}%
\end{pgfscope}%
\begin{pgfscope}%
\pgfsys@transformshift{3.117293in}{1.076792in}%
\pgfsys@useobject{currentmarker}{}%
\end{pgfscope}%
\begin{pgfscope}%
\pgfsys@transformshift{3.119783in}{1.071877in}%
\pgfsys@useobject{currentmarker}{}%
\end{pgfscope}%
\begin{pgfscope}%
\pgfsys@transformshift{3.122259in}{1.077535in}%
\pgfsys@useobject{currentmarker}{}%
\end{pgfscope}%
\begin{pgfscope}%
\pgfsys@transformshift{3.124722in}{1.091536in}%
\pgfsys@useobject{currentmarker}{}%
\end{pgfscope}%
\begin{pgfscope}%
\pgfsys@transformshift{3.127170in}{1.065970in}%
\pgfsys@useobject{currentmarker}{}%
\end{pgfscope}%
\begin{pgfscope}%
\pgfsys@transformshift{3.129606in}{1.060287in}%
\pgfsys@useobject{currentmarker}{}%
\end{pgfscope}%
\begin{pgfscope}%
\pgfsys@transformshift{3.132028in}{1.078817in}%
\pgfsys@useobject{currentmarker}{}%
\end{pgfscope}%
\begin{pgfscope}%
\pgfsys@transformshift{3.134437in}{1.071045in}%
\pgfsys@useobject{currentmarker}{}%
\end{pgfscope}%
\begin{pgfscope}%
\pgfsys@transformshift{3.136833in}{1.047356in}%
\pgfsys@useobject{currentmarker}{}%
\end{pgfscope}%
\begin{pgfscope}%
\pgfsys@transformshift{3.139216in}{1.052235in}%
\pgfsys@useobject{currentmarker}{}%
\end{pgfscope}%
\begin{pgfscope}%
\pgfsys@transformshift{3.141586in}{1.070348in}%
\pgfsys@useobject{currentmarker}{}%
\end{pgfscope}%
\begin{pgfscope}%
\pgfsys@transformshift{3.143944in}{1.073614in}%
\pgfsys@useobject{currentmarker}{}%
\end{pgfscope}%
\begin{pgfscope}%
\pgfsys@transformshift{3.146289in}{1.066914in}%
\pgfsys@useobject{currentmarker}{}%
\end{pgfscope}%
\begin{pgfscope}%
\pgfsys@transformshift{3.148622in}{1.059874in}%
\pgfsys@useobject{currentmarker}{}%
\end{pgfscope}%
\begin{pgfscope}%
\pgfsys@transformshift{3.150943in}{1.043048in}%
\pgfsys@useobject{currentmarker}{}%
\end{pgfscope}%
\begin{pgfscope}%
\pgfsys@transformshift{3.153252in}{1.052797in}%
\pgfsys@useobject{currentmarker}{}%
\end{pgfscope}%
\begin{pgfscope}%
\pgfsys@transformshift{3.155549in}{1.061105in}%
\pgfsys@useobject{currentmarker}{}%
\end{pgfscope}%
\begin{pgfscope}%
\pgfsys@transformshift{3.157834in}{1.067530in}%
\pgfsys@useobject{currentmarker}{}%
\end{pgfscope}%
\begin{pgfscope}%
\pgfsys@transformshift{3.160108in}{1.063675in}%
\pgfsys@useobject{currentmarker}{}%
\end{pgfscope}%
\begin{pgfscope}%
\pgfsys@transformshift{3.162369in}{1.073291in}%
\pgfsys@useobject{currentmarker}{}%
\end{pgfscope}%
\begin{pgfscope}%
\pgfsys@transformshift{3.164620in}{1.078887in}%
\pgfsys@useobject{currentmarker}{}%
\end{pgfscope}%
\begin{pgfscope}%
\pgfsys@transformshift{3.166859in}{1.079391in}%
\pgfsys@useobject{currentmarker}{}%
\end{pgfscope}%
\begin{pgfscope}%
\pgfsys@transformshift{3.169087in}{1.080432in}%
\pgfsys@useobject{currentmarker}{}%
\end{pgfscope}%
\begin{pgfscope}%
\pgfsys@transformshift{3.171303in}{1.083726in}%
\pgfsys@useobject{currentmarker}{}%
\end{pgfscope}%
\begin{pgfscope}%
\pgfsys@transformshift{3.173509in}{1.066528in}%
\pgfsys@useobject{currentmarker}{}%
\end{pgfscope}%
\begin{pgfscope}%
\pgfsys@transformshift{3.175704in}{1.065441in}%
\pgfsys@useobject{currentmarker}{}%
\end{pgfscope}%
\begin{pgfscope}%
\pgfsys@transformshift{3.177888in}{1.078700in}%
\pgfsys@useobject{currentmarker}{}%
\end{pgfscope}%
\begin{pgfscope}%
\pgfsys@transformshift{3.180061in}{1.065950in}%
\pgfsys@useobject{currentmarker}{}%
\end{pgfscope}%
\begin{pgfscope}%
\pgfsys@transformshift{3.182224in}{1.079501in}%
\pgfsys@useobject{currentmarker}{}%
\end{pgfscope}%
\begin{pgfscope}%
\pgfsys@transformshift{3.184376in}{1.086760in}%
\pgfsys@useobject{currentmarker}{}%
\end{pgfscope}%
\begin{pgfscope}%
\pgfsys@transformshift{3.186518in}{1.070897in}%
\pgfsys@useobject{currentmarker}{}%
\end{pgfscope}%
\begin{pgfscope}%
\pgfsys@transformshift{3.188650in}{1.067716in}%
\pgfsys@useobject{currentmarker}{}%
\end{pgfscope}%
\begin{pgfscope}%
\pgfsys@transformshift{3.190771in}{1.078477in}%
\pgfsys@useobject{currentmarker}{}%
\end{pgfscope}%
\begin{pgfscope}%
\pgfsys@transformshift{3.192883in}{1.078303in}%
\pgfsys@useobject{currentmarker}{}%
\end{pgfscope}%
\begin{pgfscope}%
\pgfsys@transformshift{3.194984in}{1.090304in}%
\pgfsys@useobject{currentmarker}{}%
\end{pgfscope}%
\begin{pgfscope}%
\pgfsys@transformshift{3.197076in}{1.065057in}%
\pgfsys@useobject{currentmarker}{}%
\end{pgfscope}%
\begin{pgfscope}%
\pgfsys@transformshift{3.199158in}{1.042604in}%
\pgfsys@useobject{currentmarker}{}%
\end{pgfscope}%
\begin{pgfscope}%
\pgfsys@transformshift{3.201230in}{1.067517in}%
\pgfsys@useobject{currentmarker}{}%
\end{pgfscope}%
\begin{pgfscope}%
\pgfsys@transformshift{3.203292in}{1.073800in}%
\pgfsys@useobject{currentmarker}{}%
\end{pgfscope}%
\begin{pgfscope}%
\pgfsys@transformshift{3.205345in}{1.055721in}%
\pgfsys@useobject{currentmarker}{}%
\end{pgfscope}%
\begin{pgfscope}%
\pgfsys@transformshift{3.207388in}{1.053353in}%
\pgfsys@useobject{currentmarker}{}%
\end{pgfscope}%
\begin{pgfscope}%
\pgfsys@transformshift{3.209422in}{1.081432in}%
\pgfsys@useobject{currentmarker}{}%
\end{pgfscope}%
\begin{pgfscope}%
\pgfsys@transformshift{3.211447in}{1.089080in}%
\pgfsys@useobject{currentmarker}{}%
\end{pgfscope}%
\begin{pgfscope}%
\pgfsys@transformshift{3.213463in}{1.082655in}%
\pgfsys@useobject{currentmarker}{}%
\end{pgfscope}%
\begin{pgfscope}%
\pgfsys@transformshift{3.215469in}{1.069421in}%
\pgfsys@useobject{currentmarker}{}%
\end{pgfscope}%
\begin{pgfscope}%
\pgfsys@transformshift{3.217467in}{1.058531in}%
\pgfsys@useobject{currentmarker}{}%
\end{pgfscope}%
\begin{pgfscope}%
\pgfsys@transformshift{3.219456in}{1.047272in}%
\pgfsys@useobject{currentmarker}{}%
\end{pgfscope}%
\begin{pgfscope}%
\pgfsys@transformshift{3.221435in}{1.053869in}%
\pgfsys@useobject{currentmarker}{}%
\end{pgfscope}%
\begin{pgfscope}%
\pgfsys@transformshift{3.223406in}{1.074933in}%
\pgfsys@useobject{currentmarker}{}%
\end{pgfscope}%
\begin{pgfscope}%
\pgfsys@transformshift{3.225369in}{1.086394in}%
\pgfsys@useobject{currentmarker}{}%
\end{pgfscope}%
\begin{pgfscope}%
\pgfsys@transformshift{3.227322in}{1.069965in}%
\pgfsys@useobject{currentmarker}{}%
\end{pgfscope}%
\begin{pgfscope}%
\pgfsys@transformshift{3.229267in}{1.054223in}%
\pgfsys@useobject{currentmarker}{}%
\end{pgfscope}%
\begin{pgfscope}%
\pgfsys@transformshift{3.231204in}{1.054107in}%
\pgfsys@useobject{currentmarker}{}%
\end{pgfscope}%
\begin{pgfscope}%
\pgfsys@transformshift{3.233132in}{1.069064in}%
\pgfsys@useobject{currentmarker}{}%
\end{pgfscope}%
\begin{pgfscope}%
\pgfsys@transformshift{3.235052in}{1.079486in}%
\pgfsys@useobject{currentmarker}{}%
\end{pgfscope}%
\begin{pgfscope}%
\pgfsys@transformshift{3.236964in}{1.083747in}%
\pgfsys@useobject{currentmarker}{}%
\end{pgfscope}%
\begin{pgfscope}%
\pgfsys@transformshift{3.238867in}{1.064904in}%
\pgfsys@useobject{currentmarker}{}%
\end{pgfscope}%
\begin{pgfscope}%
\pgfsys@transformshift{3.240762in}{1.073115in}%
\pgfsys@useobject{currentmarker}{}%
\end{pgfscope}%
\begin{pgfscope}%
\pgfsys@transformshift{3.242650in}{1.076281in}%
\pgfsys@useobject{currentmarker}{}%
\end{pgfscope}%
\begin{pgfscope}%
\pgfsys@transformshift{3.244529in}{1.059475in}%
\pgfsys@useobject{currentmarker}{}%
\end{pgfscope}%
\begin{pgfscope}%
\pgfsys@transformshift{3.246400in}{1.054172in}%
\pgfsys@useobject{currentmarker}{}%
\end{pgfscope}%
\begin{pgfscope}%
\pgfsys@transformshift{3.248264in}{1.057365in}%
\pgfsys@useobject{currentmarker}{}%
\end{pgfscope}%
\begin{pgfscope}%
\pgfsys@transformshift{3.250119in}{1.059039in}%
\pgfsys@useobject{currentmarker}{}%
\end{pgfscope}%
\begin{pgfscope}%
\pgfsys@transformshift{3.251967in}{1.057469in}%
\pgfsys@useobject{currentmarker}{}%
\end{pgfscope}%
\begin{pgfscope}%
\pgfsys@transformshift{3.253808in}{1.078845in}%
\pgfsys@useobject{currentmarker}{}%
\end{pgfscope}%
\begin{pgfscope}%
\pgfsys@transformshift{3.255641in}{1.069802in}%
\pgfsys@useobject{currentmarker}{}%
\end{pgfscope}%
\begin{pgfscope}%
\pgfsys@transformshift{3.257466in}{1.074777in}%
\pgfsys@useobject{currentmarker}{}%
\end{pgfscope}%
\begin{pgfscope}%
\pgfsys@transformshift{3.259284in}{1.075518in}%
\pgfsys@useobject{currentmarker}{}%
\end{pgfscope}%
\begin{pgfscope}%
\pgfsys@transformshift{3.261094in}{1.077168in}%
\pgfsys@useobject{currentmarker}{}%
\end{pgfscope}%
\begin{pgfscope}%
\pgfsys@transformshift{3.262897in}{1.073036in}%
\pgfsys@useobject{currentmarker}{}%
\end{pgfscope}%
\begin{pgfscope}%
\pgfsys@transformshift{3.264693in}{1.093484in}%
\pgfsys@useobject{currentmarker}{}%
\end{pgfscope}%
\begin{pgfscope}%
\pgfsys@transformshift{3.266481in}{1.070063in}%
\pgfsys@useobject{currentmarker}{}%
\end{pgfscope}%
\begin{pgfscope}%
\pgfsys@transformshift{3.268263in}{1.049574in}%
\pgfsys@useobject{currentmarker}{}%
\end{pgfscope}%
\begin{pgfscope}%
\pgfsys@transformshift{3.270037in}{1.068124in}%
\pgfsys@useobject{currentmarker}{}%
\end{pgfscope}%
\begin{pgfscope}%
\pgfsys@transformshift{3.271804in}{1.089683in}%
\pgfsys@useobject{currentmarker}{}%
\end{pgfscope}%
\begin{pgfscope}%
\pgfsys@transformshift{3.273564in}{1.088001in}%
\pgfsys@useobject{currentmarker}{}%
\end{pgfscope}%
\begin{pgfscope}%
\pgfsys@transformshift{3.275318in}{1.075663in}%
\pgfsys@useobject{currentmarker}{}%
\end{pgfscope}%
\begin{pgfscope}%
\pgfsys@transformshift{3.277064in}{1.086642in}%
\pgfsys@useobject{currentmarker}{}%
\end{pgfscope}%
\begin{pgfscope}%
\pgfsys@transformshift{3.278804in}{1.085310in}%
\pgfsys@useobject{currentmarker}{}%
\end{pgfscope}%
\begin{pgfscope}%
\pgfsys@transformshift{3.280536in}{1.078689in}%
\pgfsys@useobject{currentmarker}{}%
\end{pgfscope}%
\begin{pgfscope}%
\pgfsys@transformshift{3.282262in}{1.083052in}%
\pgfsys@useobject{currentmarker}{}%
\end{pgfscope}%
\begin{pgfscope}%
\pgfsys@transformshift{3.283982in}{1.069254in}%
\pgfsys@useobject{currentmarker}{}%
\end{pgfscope}%
\begin{pgfscope}%
\pgfsys@transformshift{3.285694in}{1.056010in}%
\pgfsys@useobject{currentmarker}{}%
\end{pgfscope}%
\begin{pgfscope}%
\pgfsys@transformshift{3.287401in}{1.083810in}%
\pgfsys@useobject{currentmarker}{}%
\end{pgfscope}%
\begin{pgfscope}%
\pgfsys@transformshift{3.289100in}{1.081412in}%
\pgfsys@useobject{currentmarker}{}%
\end{pgfscope}%
\begin{pgfscope}%
\pgfsys@transformshift{3.290793in}{1.074903in}%
\pgfsys@useobject{currentmarker}{}%
\end{pgfscope}%
\begin{pgfscope}%
\pgfsys@transformshift{3.292480in}{1.062981in}%
\pgfsys@useobject{currentmarker}{}%
\end{pgfscope}%
\begin{pgfscope}%
\pgfsys@transformshift{3.294160in}{1.069559in}%
\pgfsys@useobject{currentmarker}{}%
\end{pgfscope}%
\begin{pgfscope}%
\pgfsys@transformshift{3.295834in}{1.082690in}%
\pgfsys@useobject{currentmarker}{}%
\end{pgfscope}%
\begin{pgfscope}%
\pgfsys@transformshift{3.297502in}{1.088799in}%
\pgfsys@useobject{currentmarker}{}%
\end{pgfscope}%
\begin{pgfscope}%
\pgfsys@transformshift{3.299164in}{1.077406in}%
\pgfsys@useobject{currentmarker}{}%
\end{pgfscope}%
\begin{pgfscope}%
\pgfsys@transformshift{3.300819in}{1.064103in}%
\pgfsys@useobject{currentmarker}{}%
\end{pgfscope}%
\begin{pgfscope}%
\pgfsys@transformshift{3.302468in}{1.075986in}%
\pgfsys@useobject{currentmarker}{}%
\end{pgfscope}%
\begin{pgfscope}%
\pgfsys@transformshift{3.304111in}{1.071644in}%
\pgfsys@useobject{currentmarker}{}%
\end{pgfscope}%
\begin{pgfscope}%
\pgfsys@transformshift{3.305748in}{1.054429in}%
\pgfsys@useobject{currentmarker}{}%
\end{pgfscope}%
\begin{pgfscope}%
\pgfsys@transformshift{3.307379in}{1.069546in}%
\pgfsys@useobject{currentmarker}{}%
\end{pgfscope}%
\begin{pgfscope}%
\pgfsys@transformshift{3.309004in}{1.090408in}%
\pgfsys@useobject{currentmarker}{}%
\end{pgfscope}%
\begin{pgfscope}%
\pgfsys@transformshift{3.310623in}{1.074086in}%
\pgfsys@useobject{currentmarker}{}%
\end{pgfscope}%
\begin{pgfscope}%
\pgfsys@transformshift{3.312237in}{1.087238in}%
\pgfsys@useobject{currentmarker}{}%
\end{pgfscope}%
\begin{pgfscope}%
\pgfsys@transformshift{3.313844in}{1.073904in}%
\pgfsys@useobject{currentmarker}{}%
\end{pgfscope}%
\begin{pgfscope}%
\pgfsys@transformshift{3.315446in}{1.069369in}%
\pgfsys@useobject{currentmarker}{}%
\end{pgfscope}%
\begin{pgfscope}%
\pgfsys@transformshift{3.317041in}{1.079843in}%
\pgfsys@useobject{currentmarker}{}%
\end{pgfscope}%
\begin{pgfscope}%
\pgfsys@transformshift{3.318632in}{1.071094in}%
\pgfsys@useobject{currentmarker}{}%
\end{pgfscope}%
\begin{pgfscope}%
\pgfsys@transformshift{3.320216in}{1.066889in}%
\pgfsys@useobject{currentmarker}{}%
\end{pgfscope}%
\begin{pgfscope}%
\pgfsys@transformshift{3.321795in}{1.069746in}%
\pgfsys@useobject{currentmarker}{}%
\end{pgfscope}%
\begin{pgfscope}%
\pgfsys@transformshift{3.323368in}{1.077153in}%
\pgfsys@useobject{currentmarker}{}%
\end{pgfscope}%
\begin{pgfscope}%
\pgfsys@transformshift{3.324936in}{1.050740in}%
\pgfsys@useobject{currentmarker}{}%
\end{pgfscope}%
\begin{pgfscope}%
\pgfsys@transformshift{3.326498in}{1.056639in}%
\pgfsys@useobject{currentmarker}{}%
\end{pgfscope}%
\begin{pgfscope}%
\pgfsys@transformshift{3.328055in}{1.073771in}%
\pgfsys@useobject{currentmarker}{}%
\end{pgfscope}%
\begin{pgfscope}%
\pgfsys@transformshift{3.329606in}{1.096709in}%
\pgfsys@useobject{currentmarker}{}%
\end{pgfscope}%
\begin{pgfscope}%
\pgfsys@transformshift{3.331152in}{1.092825in}%
\pgfsys@useobject{currentmarker}{}%
\end{pgfscope}%
\begin{pgfscope}%
\pgfsys@transformshift{3.332693in}{1.070548in}%
\pgfsys@useobject{currentmarker}{}%
\end{pgfscope}%
\begin{pgfscope}%
\pgfsys@transformshift{3.334228in}{1.049256in}%
\pgfsys@useobject{currentmarker}{}%
\end{pgfscope}%
\begin{pgfscope}%
\pgfsys@transformshift{3.335758in}{1.051374in}%
\pgfsys@useobject{currentmarker}{}%
\end{pgfscope}%
\begin{pgfscope}%
\pgfsys@transformshift{3.337283in}{1.071089in}%
\pgfsys@useobject{currentmarker}{}%
\end{pgfscope}%
\begin{pgfscope}%
\pgfsys@transformshift{3.338802in}{1.079345in}%
\pgfsys@useobject{currentmarker}{}%
\end{pgfscope}%
\begin{pgfscope}%
\pgfsys@transformshift{3.340316in}{1.075126in}%
\pgfsys@useobject{currentmarker}{}%
\end{pgfscope}%
\begin{pgfscope}%
\pgfsys@transformshift{3.341826in}{1.088937in}%
\pgfsys@useobject{currentmarker}{}%
\end{pgfscope}%
\begin{pgfscope}%
\pgfsys@transformshift{3.343330in}{1.098460in}%
\pgfsys@useobject{currentmarker}{}%
\end{pgfscope}%
\begin{pgfscope}%
\pgfsys@transformshift{3.344829in}{1.064093in}%
\pgfsys@useobject{currentmarker}{}%
\end{pgfscope}%
\begin{pgfscope}%
\pgfsys@transformshift{3.346323in}{1.065261in}%
\pgfsys@useobject{currentmarker}{}%
\end{pgfscope}%
\begin{pgfscope}%
\pgfsys@transformshift{3.347812in}{1.086919in}%
\pgfsys@useobject{currentmarker}{}%
\end{pgfscope}%
\begin{pgfscope}%
\pgfsys@transformshift{3.349296in}{1.091304in}%
\pgfsys@useobject{currentmarker}{}%
\end{pgfscope}%
\begin{pgfscope}%
\pgfsys@transformshift{3.350775in}{1.069802in}%
\pgfsys@useobject{currentmarker}{}%
\end{pgfscope}%
\begin{pgfscope}%
\pgfsys@transformshift{3.352249in}{1.061367in}%
\pgfsys@useobject{currentmarker}{}%
\end{pgfscope}%
\begin{pgfscope}%
\pgfsys@transformshift{3.353718in}{1.069377in}%
\pgfsys@useobject{currentmarker}{}%
\end{pgfscope}%
\begin{pgfscope}%
\pgfsys@transformshift{3.355182in}{1.062582in}%
\pgfsys@useobject{currentmarker}{}%
\end{pgfscope}%
\begin{pgfscope}%
\pgfsys@transformshift{3.356642in}{1.054048in}%
\pgfsys@useobject{currentmarker}{}%
\end{pgfscope}%
\begin{pgfscope}%
\pgfsys@transformshift{3.358097in}{1.079792in}%
\pgfsys@useobject{currentmarker}{}%
\end{pgfscope}%
\begin{pgfscope}%
\pgfsys@transformshift{3.359547in}{1.071958in}%
\pgfsys@useobject{currentmarker}{}%
\end{pgfscope}%
\begin{pgfscope}%
\pgfsys@transformshift{3.360992in}{1.080122in}%
\pgfsys@useobject{currentmarker}{}%
\end{pgfscope}%
\begin{pgfscope}%
\pgfsys@transformshift{3.362433in}{1.085473in}%
\pgfsys@useobject{currentmarker}{}%
\end{pgfscope}%
\begin{pgfscope}%
\pgfsys@transformshift{3.363869in}{1.081741in}%
\pgfsys@useobject{currentmarker}{}%
\end{pgfscope}%
\begin{pgfscope}%
\pgfsys@transformshift{3.365300in}{1.081584in}%
\pgfsys@useobject{currentmarker}{}%
\end{pgfscope}%
\begin{pgfscope}%
\pgfsys@transformshift{3.366727in}{1.079071in}%
\pgfsys@useobject{currentmarker}{}%
\end{pgfscope}%
\begin{pgfscope}%
\pgfsys@transformshift{3.368149in}{1.075899in}%
\pgfsys@useobject{currentmarker}{}%
\end{pgfscope}%
\begin{pgfscope}%
\pgfsys@transformshift{3.369567in}{1.076957in}%
\pgfsys@useobject{currentmarker}{}%
\end{pgfscope}%
\begin{pgfscope}%
\pgfsys@transformshift{3.370980in}{1.071287in}%
\pgfsys@useobject{currentmarker}{}%
\end{pgfscope}%
\begin{pgfscope}%
\pgfsys@transformshift{3.372389in}{1.072021in}%
\pgfsys@useobject{currentmarker}{}%
\end{pgfscope}%
\begin{pgfscope}%
\pgfsys@transformshift{3.373793in}{1.067413in}%
\pgfsys@useobject{currentmarker}{}%
\end{pgfscope}%
\begin{pgfscope}%
\pgfsys@transformshift{3.375193in}{1.069900in}%
\pgfsys@useobject{currentmarker}{}%
\end{pgfscope}%
\begin{pgfscope}%
\pgfsys@transformshift{3.376589in}{1.092196in}%
\pgfsys@useobject{currentmarker}{}%
\end{pgfscope}%
\begin{pgfscope}%
\pgfsys@transformshift{3.377980in}{1.087043in}%
\pgfsys@useobject{currentmarker}{}%
\end{pgfscope}%
\begin{pgfscope}%
\pgfsys@transformshift{3.379366in}{1.094171in}%
\pgfsys@useobject{currentmarker}{}%
\end{pgfscope}%
\begin{pgfscope}%
\pgfsys@transformshift{3.380749in}{1.073175in}%
\pgfsys@useobject{currentmarker}{}%
\end{pgfscope}%
\begin{pgfscope}%
\pgfsys@transformshift{3.382127in}{1.071143in}%
\pgfsys@useobject{currentmarker}{}%
\end{pgfscope}%
\begin{pgfscope}%
\pgfsys@transformshift{3.383501in}{1.086396in}%
\pgfsys@useobject{currentmarker}{}%
\end{pgfscope}%
\begin{pgfscope}%
\pgfsys@transformshift{3.384870in}{1.078553in}%
\pgfsys@useobject{currentmarker}{}%
\end{pgfscope}%
\begin{pgfscope}%
\pgfsys@transformshift{3.386236in}{1.081220in}%
\pgfsys@useobject{currentmarker}{}%
\end{pgfscope}%
\begin{pgfscope}%
\pgfsys@transformshift{3.387597in}{1.089642in}%
\pgfsys@useobject{currentmarker}{}%
\end{pgfscope}%
\begin{pgfscope}%
\pgfsys@transformshift{3.388954in}{1.086653in}%
\pgfsys@useobject{currentmarker}{}%
\end{pgfscope}%
\begin{pgfscope}%
\pgfsys@transformshift{3.390307in}{1.082322in}%
\pgfsys@useobject{currentmarker}{}%
\end{pgfscope}%
\begin{pgfscope}%
\pgfsys@transformshift{3.391656in}{1.076711in}%
\pgfsys@useobject{currentmarker}{}%
\end{pgfscope}%
\begin{pgfscope}%
\pgfsys@transformshift{3.393001in}{1.098930in}%
\pgfsys@useobject{currentmarker}{}%
\end{pgfscope}%
\begin{pgfscope}%
\pgfsys@transformshift{3.394342in}{1.096172in}%
\pgfsys@useobject{currentmarker}{}%
\end{pgfscope}%
\begin{pgfscope}%
\pgfsys@transformshift{3.395678in}{1.070520in}%
\pgfsys@useobject{currentmarker}{}%
\end{pgfscope}%
\begin{pgfscope}%
\pgfsys@transformshift{3.397011in}{1.066631in}%
\pgfsys@useobject{currentmarker}{}%
\end{pgfscope}%
\begin{pgfscope}%
\pgfsys@transformshift{3.398340in}{1.093580in}%
\pgfsys@useobject{currentmarker}{}%
\end{pgfscope}%
\begin{pgfscope}%
\pgfsys@transformshift{3.399664in}{1.096701in}%
\pgfsys@useobject{currentmarker}{}%
\end{pgfscope}%
\begin{pgfscope}%
\pgfsys@transformshift{3.400985in}{1.093715in}%
\pgfsys@useobject{currentmarker}{}%
\end{pgfscope}%
\begin{pgfscope}%
\pgfsys@transformshift{3.402302in}{1.078462in}%
\pgfsys@useobject{currentmarker}{}%
\end{pgfscope}%
\begin{pgfscope}%
\pgfsys@transformshift{3.403615in}{1.086110in}%
\pgfsys@useobject{currentmarker}{}%
\end{pgfscope}%
\begin{pgfscope}%
\pgfsys@transformshift{3.404924in}{1.059847in}%
\pgfsys@useobject{currentmarker}{}%
\end{pgfscope}%
\begin{pgfscope}%
\pgfsys@transformshift{3.406230in}{1.094497in}%
\pgfsys@useobject{currentmarker}{}%
\end{pgfscope}%
\begin{pgfscope}%
\pgfsys@transformshift{3.407531in}{1.104935in}%
\pgfsys@useobject{currentmarker}{}%
\end{pgfscope}%
\begin{pgfscope}%
\pgfsys@transformshift{3.408829in}{1.080823in}%
\pgfsys@useobject{currentmarker}{}%
\end{pgfscope}%
\begin{pgfscope}%
\pgfsys@transformshift{3.410123in}{1.079410in}%
\pgfsys@useobject{currentmarker}{}%
\end{pgfscope}%
\begin{pgfscope}%
\pgfsys@transformshift{3.411413in}{1.086027in}%
\pgfsys@useobject{currentmarker}{}%
\end{pgfscope}%
\begin{pgfscope}%
\pgfsys@transformshift{3.412699in}{1.090103in}%
\pgfsys@useobject{currentmarker}{}%
\end{pgfscope}%
\begin{pgfscope}%
\pgfsys@transformshift{3.413982in}{1.086858in}%
\pgfsys@useobject{currentmarker}{}%
\end{pgfscope}%
\begin{pgfscope}%
\pgfsys@transformshift{3.415261in}{1.091246in}%
\pgfsys@useobject{currentmarker}{}%
\end{pgfscope}%
\begin{pgfscope}%
\pgfsys@transformshift{3.416536in}{1.094424in}%
\pgfsys@useobject{currentmarker}{}%
\end{pgfscope}%
\begin{pgfscope}%
\pgfsys@transformshift{3.417808in}{1.092591in}%
\pgfsys@useobject{currentmarker}{}%
\end{pgfscope}%
\begin{pgfscope}%
\pgfsys@transformshift{3.419076in}{1.082992in}%
\pgfsys@useobject{currentmarker}{}%
\end{pgfscope}%
\begin{pgfscope}%
\pgfsys@transformshift{3.420340in}{1.083335in}%
\pgfsys@useobject{currentmarker}{}%
\end{pgfscope}%
\begin{pgfscope}%
\pgfsys@transformshift{3.421601in}{1.079011in}%
\pgfsys@useobject{currentmarker}{}%
\end{pgfscope}%
\begin{pgfscope}%
\pgfsys@transformshift{3.422858in}{1.082217in}%
\pgfsys@useobject{currentmarker}{}%
\end{pgfscope}%
\begin{pgfscope}%
\pgfsys@transformshift{3.424112in}{1.086408in}%
\pgfsys@useobject{currentmarker}{}%
\end{pgfscope}%
\begin{pgfscope}%
\pgfsys@transformshift{3.425362in}{1.086111in}%
\pgfsys@useobject{currentmarker}{}%
\end{pgfscope}%
\begin{pgfscope}%
\pgfsys@transformshift{3.426609in}{1.076304in}%
\pgfsys@useobject{currentmarker}{}%
\end{pgfscope}%
\begin{pgfscope}%
\pgfsys@transformshift{3.427852in}{1.075781in}%
\pgfsys@useobject{currentmarker}{}%
\end{pgfscope}%
\begin{pgfscope}%
\pgfsys@transformshift{3.429092in}{1.083604in}%
\pgfsys@useobject{currentmarker}{}%
\end{pgfscope}%
\begin{pgfscope}%
\pgfsys@transformshift{3.430328in}{1.091751in}%
\pgfsys@useobject{currentmarker}{}%
\end{pgfscope}%
\begin{pgfscope}%
\pgfsys@transformshift{3.431561in}{1.088680in}%
\pgfsys@useobject{currentmarker}{}%
\end{pgfscope}%
\begin{pgfscope}%
\pgfsys@transformshift{3.432791in}{1.071067in}%
\pgfsys@useobject{currentmarker}{}%
\end{pgfscope}%
\begin{pgfscope}%
\pgfsys@transformshift{3.434017in}{1.060325in}%
\pgfsys@useobject{currentmarker}{}%
\end{pgfscope}%
\begin{pgfscope}%
\pgfsys@transformshift{3.435239in}{1.054339in}%
\pgfsys@useobject{currentmarker}{}%
\end{pgfscope}%
\begin{pgfscope}%
\pgfsys@transformshift{3.436459in}{1.075601in}%
\pgfsys@useobject{currentmarker}{}%
\end{pgfscope}%
\begin{pgfscope}%
\pgfsys@transformshift{3.437675in}{1.088673in}%
\pgfsys@useobject{currentmarker}{}%
\end{pgfscope}%
\begin{pgfscope}%
\pgfsys@transformshift{3.438887in}{1.078736in}%
\pgfsys@useobject{currentmarker}{}%
\end{pgfscope}%
\begin{pgfscope}%
\pgfsys@transformshift{3.440097in}{1.088966in}%
\pgfsys@useobject{currentmarker}{}%
\end{pgfscope}%
\begin{pgfscope}%
\pgfsys@transformshift{3.441303in}{1.095519in}%
\pgfsys@useobject{currentmarker}{}%
\end{pgfscope}%
\begin{pgfscope}%
\pgfsys@transformshift{3.442506in}{1.084769in}%
\pgfsys@useobject{currentmarker}{}%
\end{pgfscope}%
\begin{pgfscope}%
\pgfsys@transformshift{3.443705in}{1.066442in}%
\pgfsys@useobject{currentmarker}{}%
\end{pgfscope}%
\begin{pgfscope}%
\pgfsys@transformshift{3.444902in}{1.074969in}%
\pgfsys@useobject{currentmarker}{}%
\end{pgfscope}%
\begin{pgfscope}%
\pgfsys@transformshift{3.446095in}{1.074447in}%
\pgfsys@useobject{currentmarker}{}%
\end{pgfscope}%
\begin{pgfscope}%
\pgfsys@transformshift{3.447285in}{1.074558in}%
\pgfsys@useobject{currentmarker}{}%
\end{pgfscope}%
\begin{pgfscope}%
\pgfsys@transformshift{3.448472in}{1.098666in}%
\pgfsys@useobject{currentmarker}{}%
\end{pgfscope}%
\begin{pgfscope}%
\pgfsys@transformshift{3.449655in}{1.103434in}%
\pgfsys@useobject{currentmarker}{}%
\end{pgfscope}%
\begin{pgfscope}%
\pgfsys@transformshift{3.450836in}{1.099655in}%
\pgfsys@useobject{currentmarker}{}%
\end{pgfscope}%
\begin{pgfscope}%
\pgfsys@transformshift{3.452013in}{1.102298in}%
\pgfsys@useobject{currentmarker}{}%
\end{pgfscope}%
\begin{pgfscope}%
\pgfsys@transformshift{3.453187in}{1.091114in}%
\pgfsys@useobject{currentmarker}{}%
\end{pgfscope}%
\begin{pgfscope}%
\pgfsys@transformshift{3.454358in}{1.080221in}%
\pgfsys@useobject{currentmarker}{}%
\end{pgfscope}%
\begin{pgfscope}%
\pgfsys@transformshift{3.455526in}{1.084127in}%
\pgfsys@useobject{currentmarker}{}%
\end{pgfscope}%
\begin{pgfscope}%
\pgfsys@transformshift{3.456692in}{1.086007in}%
\pgfsys@useobject{currentmarker}{}%
\end{pgfscope}%
\begin{pgfscope}%
\pgfsys@transformshift{3.457853in}{1.084952in}%
\pgfsys@useobject{currentmarker}{}%
\end{pgfscope}%
\begin{pgfscope}%
\pgfsys@transformshift{3.459012in}{1.086300in}%
\pgfsys@useobject{currentmarker}{}%
\end{pgfscope}%
\begin{pgfscope}%
\pgfsys@transformshift{3.460168in}{1.083148in}%
\pgfsys@useobject{currentmarker}{}%
\end{pgfscope}%
\begin{pgfscope}%
\pgfsys@transformshift{3.461321in}{1.071855in}%
\pgfsys@useobject{currentmarker}{}%
\end{pgfscope}%
\begin{pgfscope}%
\pgfsys@transformshift{3.462471in}{1.072876in}%
\pgfsys@useobject{currentmarker}{}%
\end{pgfscope}%
\begin{pgfscope}%
\pgfsys@transformshift{3.463618in}{1.067858in}%
\pgfsys@useobject{currentmarker}{}%
\end{pgfscope}%
\begin{pgfscope}%
\pgfsys@transformshift{3.464762in}{1.067344in}%
\pgfsys@useobject{currentmarker}{}%
\end{pgfscope}%
\begin{pgfscope}%
\pgfsys@transformshift{3.465903in}{1.074237in}%
\pgfsys@useobject{currentmarker}{}%
\end{pgfscope}%
\begin{pgfscope}%
\pgfsys@transformshift{3.467041in}{1.062809in}%
\pgfsys@useobject{currentmarker}{}%
\end{pgfscope}%
\begin{pgfscope}%
\pgfsys@transformshift{3.468177in}{1.071326in}%
\pgfsys@useobject{currentmarker}{}%
\end{pgfscope}%
\begin{pgfscope}%
\pgfsys@transformshift{3.469309in}{1.088411in}%
\pgfsys@useobject{currentmarker}{}%
\end{pgfscope}%
\begin{pgfscope}%
\pgfsys@transformshift{3.470438in}{1.070934in}%
\pgfsys@useobject{currentmarker}{}%
\end{pgfscope}%
\begin{pgfscope}%
\pgfsys@transformshift{3.471565in}{1.081956in}%
\pgfsys@useobject{currentmarker}{}%
\end{pgfscope}%
\begin{pgfscope}%
\pgfsys@transformshift{3.472689in}{1.081412in}%
\pgfsys@useobject{currentmarker}{}%
\end{pgfscope}%
\begin{pgfscope}%
\pgfsys@transformshift{3.473810in}{1.079050in}%
\pgfsys@useobject{currentmarker}{}%
\end{pgfscope}%
\begin{pgfscope}%
\pgfsys@transformshift{3.474928in}{1.093675in}%
\pgfsys@useobject{currentmarker}{}%
\end{pgfscope}%
\begin{pgfscope}%
\pgfsys@transformshift{3.476043in}{1.081191in}%
\pgfsys@useobject{currentmarker}{}%
\end{pgfscope}%
\begin{pgfscope}%
\pgfsys@transformshift{3.477156in}{1.080566in}%
\pgfsys@useobject{currentmarker}{}%
\end{pgfscope}%
\begin{pgfscope}%
\pgfsys@transformshift{3.478265in}{1.079687in}%
\pgfsys@useobject{currentmarker}{}%
\end{pgfscope}%
\begin{pgfscope}%
\pgfsys@transformshift{3.479372in}{1.078564in}%
\pgfsys@useobject{currentmarker}{}%
\end{pgfscope}%
\begin{pgfscope}%
\pgfsys@transformshift{3.480477in}{1.077200in}%
\pgfsys@useobject{currentmarker}{}%
\end{pgfscope}%
\begin{pgfscope}%
\pgfsys@transformshift{3.481578in}{1.069689in}%
\pgfsys@useobject{currentmarker}{}%
\end{pgfscope}%
\begin{pgfscope}%
\pgfsys@transformshift{3.482677in}{1.088707in}%
\pgfsys@useobject{currentmarker}{}%
\end{pgfscope}%
\begin{pgfscope}%
\pgfsys@transformshift{3.483773in}{1.094342in}%
\pgfsys@useobject{currentmarker}{}%
\end{pgfscope}%
\begin{pgfscope}%
\pgfsys@transformshift{3.484866in}{1.093502in}%
\pgfsys@useobject{currentmarker}{}%
\end{pgfscope}%
\begin{pgfscope}%
\pgfsys@transformshift{3.485957in}{1.097751in}%
\pgfsys@useobject{currentmarker}{}%
\end{pgfscope}%
\begin{pgfscope}%
\pgfsys@transformshift{3.487045in}{1.094804in}%
\pgfsys@useobject{currentmarker}{}%
\end{pgfscope}%
\begin{pgfscope}%
\pgfsys@transformshift{3.488130in}{1.080227in}%
\pgfsys@useobject{currentmarker}{}%
\end{pgfscope}%
\begin{pgfscope}%
\pgfsys@transformshift{3.489213in}{1.081441in}%
\pgfsys@useobject{currentmarker}{}%
\end{pgfscope}%
\begin{pgfscope}%
\pgfsys@transformshift{3.490293in}{1.097150in}%
\pgfsys@useobject{currentmarker}{}%
\end{pgfscope}%
\begin{pgfscope}%
\pgfsys@transformshift{3.491371in}{1.088591in}%
\pgfsys@useobject{currentmarker}{}%
\end{pgfscope}%
\begin{pgfscope}%
\pgfsys@transformshift{3.492445in}{1.077715in}%
\pgfsys@useobject{currentmarker}{}%
\end{pgfscope}%
\begin{pgfscope}%
\pgfsys@transformshift{3.493518in}{1.076269in}%
\pgfsys@useobject{currentmarker}{}%
\end{pgfscope}%
\begin{pgfscope}%
\pgfsys@transformshift{3.494587in}{1.071289in}%
\pgfsys@useobject{currentmarker}{}%
\end{pgfscope}%
\begin{pgfscope}%
\pgfsys@transformshift{3.495654in}{1.085352in}%
\pgfsys@useobject{currentmarker}{}%
\end{pgfscope}%
\begin{pgfscope}%
\pgfsys@transformshift{3.496719in}{1.104507in}%
\pgfsys@useobject{currentmarker}{}%
\end{pgfscope}%
\begin{pgfscope}%
\pgfsys@transformshift{3.497781in}{1.110329in}%
\pgfsys@useobject{currentmarker}{}%
\end{pgfscope}%
\begin{pgfscope}%
\pgfsys@transformshift{3.498840in}{1.097423in}%
\pgfsys@useobject{currentmarker}{}%
\end{pgfscope}%
\begin{pgfscope}%
\pgfsys@transformshift{3.499897in}{1.082210in}%
\pgfsys@useobject{currentmarker}{}%
\end{pgfscope}%
\begin{pgfscope}%
\pgfsys@transformshift{3.500952in}{1.074419in}%
\pgfsys@useobject{currentmarker}{}%
\end{pgfscope}%
\begin{pgfscope}%
\pgfsys@transformshift{3.502004in}{1.066287in}%
\pgfsys@useobject{currentmarker}{}%
\end{pgfscope}%
\begin{pgfscope}%
\pgfsys@transformshift{3.503053in}{1.081627in}%
\pgfsys@useobject{currentmarker}{}%
\end{pgfscope}%
\begin{pgfscope}%
\pgfsys@transformshift{3.504100in}{1.100744in}%
\pgfsys@useobject{currentmarker}{}%
\end{pgfscope}%
\begin{pgfscope}%
\pgfsys@transformshift{3.505145in}{1.078901in}%
\pgfsys@useobject{currentmarker}{}%
\end{pgfscope}%
\begin{pgfscope}%
\pgfsys@transformshift{3.506187in}{1.077510in}%
\pgfsys@useobject{currentmarker}{}%
\end{pgfscope}%
\begin{pgfscope}%
\pgfsys@transformshift{3.507227in}{1.089262in}%
\pgfsys@useobject{currentmarker}{}%
\end{pgfscope}%
\begin{pgfscope}%
\pgfsys@transformshift{3.508264in}{1.091713in}%
\pgfsys@useobject{currentmarker}{}%
\end{pgfscope}%
\begin{pgfscope}%
\pgfsys@transformshift{3.509299in}{1.089804in}%
\pgfsys@useobject{currentmarker}{}%
\end{pgfscope}%
\begin{pgfscope}%
\pgfsys@transformshift{3.510331in}{1.087294in}%
\pgfsys@useobject{currentmarker}{}%
\end{pgfscope}%
\begin{pgfscope}%
\pgfsys@transformshift{3.511361in}{1.085622in}%
\pgfsys@useobject{currentmarker}{}%
\end{pgfscope}%
\begin{pgfscope}%
\pgfsys@transformshift{3.512389in}{1.061440in}%
\pgfsys@useobject{currentmarker}{}%
\end{pgfscope}%
\begin{pgfscope}%
\pgfsys@transformshift{3.513414in}{1.083904in}%
\pgfsys@useobject{currentmarker}{}%
\end{pgfscope}%
\begin{pgfscope}%
\pgfsys@transformshift{3.514437in}{1.116193in}%
\pgfsys@useobject{currentmarker}{}%
\end{pgfscope}%
\begin{pgfscope}%
\pgfsys@transformshift{3.515457in}{1.118154in}%
\pgfsys@useobject{currentmarker}{}%
\end{pgfscope}%
\begin{pgfscope}%
\pgfsys@transformshift{3.516476in}{1.100832in}%
\pgfsys@useobject{currentmarker}{}%
\end{pgfscope}%
\begin{pgfscope}%
\pgfsys@transformshift{3.517491in}{1.100490in}%
\pgfsys@useobject{currentmarker}{}%
\end{pgfscope}%
\begin{pgfscope}%
\pgfsys@transformshift{3.518505in}{1.078514in}%
\pgfsys@useobject{currentmarker}{}%
\end{pgfscope}%
\begin{pgfscope}%
\pgfsys@transformshift{3.519516in}{1.073058in}%
\pgfsys@useobject{currentmarker}{}%
\end{pgfscope}%
\begin{pgfscope}%
\pgfsys@transformshift{3.520525in}{1.091306in}%
\pgfsys@useobject{currentmarker}{}%
\end{pgfscope}%
\begin{pgfscope}%
\pgfsys@transformshift{3.521532in}{1.097558in}%
\pgfsys@useobject{currentmarker}{}%
\end{pgfscope}%
\begin{pgfscope}%
\pgfsys@transformshift{3.522536in}{1.083094in}%
\pgfsys@useobject{currentmarker}{}%
\end{pgfscope}%
\begin{pgfscope}%
\pgfsys@transformshift{3.523538in}{1.100913in}%
\pgfsys@useobject{currentmarker}{}%
\end{pgfscope}%
\begin{pgfscope}%
\pgfsys@transformshift{3.524538in}{1.108558in}%
\pgfsys@useobject{currentmarker}{}%
\end{pgfscope}%
\begin{pgfscope}%
\pgfsys@transformshift{3.525536in}{1.068547in}%
\pgfsys@useobject{currentmarker}{}%
\end{pgfscope}%
\begin{pgfscope}%
\pgfsys@transformshift{3.526531in}{1.071305in}%
\pgfsys@useobject{currentmarker}{}%
\end{pgfscope}%
\begin{pgfscope}%
\pgfsys@transformshift{3.527525in}{1.088856in}%
\pgfsys@useobject{currentmarker}{}%
\end{pgfscope}%
\begin{pgfscope}%
\pgfsys@transformshift{3.528516in}{1.080762in}%
\pgfsys@useobject{currentmarker}{}%
\end{pgfscope}%
\begin{pgfscope}%
\pgfsys@transformshift{3.529504in}{1.099633in}%
\pgfsys@useobject{currentmarker}{}%
\end{pgfscope}%
\begin{pgfscope}%
\pgfsys@transformshift{3.530491in}{1.107418in}%
\pgfsys@useobject{currentmarker}{}%
\end{pgfscope}%
\begin{pgfscope}%
\pgfsys@transformshift{3.531475in}{1.098676in}%
\pgfsys@useobject{currentmarker}{}%
\end{pgfscope}%
\begin{pgfscope}%
\pgfsys@transformshift{3.532458in}{1.089576in}%
\pgfsys@useobject{currentmarker}{}%
\end{pgfscope}%
\begin{pgfscope}%
\pgfsys@transformshift{3.533438in}{1.065166in}%
\pgfsys@useobject{currentmarker}{}%
\end{pgfscope}%
\begin{pgfscope}%
\pgfsys@transformshift{3.534415in}{1.076467in}%
\pgfsys@useobject{currentmarker}{}%
\end{pgfscope}%
\begin{pgfscope}%
\pgfsys@transformshift{3.535391in}{1.096564in}%
\pgfsys@useobject{currentmarker}{}%
\end{pgfscope}%
\begin{pgfscope}%
\pgfsys@transformshift{3.536365in}{1.103463in}%
\pgfsys@useobject{currentmarker}{}%
\end{pgfscope}%
\begin{pgfscope}%
\pgfsys@transformshift{3.537336in}{1.103359in}%
\pgfsys@useobject{currentmarker}{}%
\end{pgfscope}%
\begin{pgfscope}%
\pgfsys@transformshift{3.538306in}{1.102743in}%
\pgfsys@useobject{currentmarker}{}%
\end{pgfscope}%
\begin{pgfscope}%
\pgfsys@transformshift{3.539273in}{1.108879in}%
\pgfsys@useobject{currentmarker}{}%
\end{pgfscope}%
\begin{pgfscope}%
\pgfsys@transformshift{3.540238in}{1.094798in}%
\pgfsys@useobject{currentmarker}{}%
\end{pgfscope}%
\begin{pgfscope}%
\pgfsys@transformshift{3.541201in}{1.087244in}%
\pgfsys@useobject{currentmarker}{}%
\end{pgfscope}%
\begin{pgfscope}%
\pgfsys@transformshift{3.542162in}{1.082806in}%
\pgfsys@useobject{currentmarker}{}%
\end{pgfscope}%
\begin{pgfscope}%
\pgfsys@transformshift{3.543121in}{1.076256in}%
\pgfsys@useobject{currentmarker}{}%
\end{pgfscope}%
\begin{pgfscope}%
\pgfsys@transformshift{3.544078in}{1.080793in}%
\pgfsys@useobject{currentmarker}{}%
\end{pgfscope}%
\begin{pgfscope}%
\pgfsys@transformshift{3.545033in}{1.099618in}%
\pgfsys@useobject{currentmarker}{}%
\end{pgfscope}%
\begin{pgfscope}%
\pgfsys@transformshift{3.545985in}{1.096411in}%
\pgfsys@useobject{currentmarker}{}%
\end{pgfscope}%
\begin{pgfscope}%
\pgfsys@transformshift{3.546936in}{1.079066in}%
\pgfsys@useobject{currentmarker}{}%
\end{pgfscope}%
\begin{pgfscope}%
\pgfsys@transformshift{3.547885in}{1.095567in}%
\pgfsys@useobject{currentmarker}{}%
\end{pgfscope}%
\begin{pgfscope}%
\pgfsys@transformshift{3.548831in}{1.092909in}%
\pgfsys@useobject{currentmarker}{}%
\end{pgfscope}%
\begin{pgfscope}%
\pgfsys@transformshift{3.549776in}{1.095385in}%
\pgfsys@useobject{currentmarker}{}%
\end{pgfscope}%
\begin{pgfscope}%
\pgfsys@transformshift{3.550719in}{1.089889in}%
\pgfsys@useobject{currentmarker}{}%
\end{pgfscope}%
\begin{pgfscope}%
\pgfsys@transformshift{3.551659in}{1.095531in}%
\pgfsys@useobject{currentmarker}{}%
\end{pgfscope}%
\begin{pgfscope}%
\pgfsys@transformshift{3.552598in}{1.095240in}%
\pgfsys@useobject{currentmarker}{}%
\end{pgfscope}%
\begin{pgfscope}%
\pgfsys@transformshift{3.553535in}{1.098437in}%
\pgfsys@useobject{currentmarker}{}%
\end{pgfscope}%
\begin{pgfscope}%
\pgfsys@transformshift{3.554469in}{1.082725in}%
\pgfsys@useobject{currentmarker}{}%
\end{pgfscope}%
\begin{pgfscope}%
\pgfsys@transformshift{3.555402in}{1.103166in}%
\pgfsys@useobject{currentmarker}{}%
\end{pgfscope}%
\begin{pgfscope}%
\pgfsys@transformshift{3.556333in}{1.095126in}%
\pgfsys@useobject{currentmarker}{}%
\end{pgfscope}%
\begin{pgfscope}%
\pgfsys@transformshift{3.557262in}{1.093239in}%
\pgfsys@useobject{currentmarker}{}%
\end{pgfscope}%
\begin{pgfscope}%
\pgfsys@transformshift{3.558188in}{1.109098in}%
\pgfsys@useobject{currentmarker}{}%
\end{pgfscope}%
\begin{pgfscope}%
\pgfsys@transformshift{3.559113in}{1.112710in}%
\pgfsys@useobject{currentmarker}{}%
\end{pgfscope}%
\begin{pgfscope}%
\pgfsys@transformshift{3.560037in}{1.103872in}%
\pgfsys@useobject{currentmarker}{}%
\end{pgfscope}%
\begin{pgfscope}%
\pgfsys@transformshift{3.560958in}{1.078163in}%
\pgfsys@useobject{currentmarker}{}%
\end{pgfscope}%
\begin{pgfscope}%
\pgfsys@transformshift{3.561877in}{1.066492in}%
\pgfsys@useobject{currentmarker}{}%
\end{pgfscope}%
\begin{pgfscope}%
\pgfsys@transformshift{3.562794in}{1.061318in}%
\pgfsys@useobject{currentmarker}{}%
\end{pgfscope}%
\begin{pgfscope}%
\pgfsys@transformshift{3.563710in}{1.065983in}%
\pgfsys@useobject{currentmarker}{}%
\end{pgfscope}%
\begin{pgfscope}%
\pgfsys@transformshift{3.564623in}{1.093821in}%
\pgfsys@useobject{currentmarker}{}%
\end{pgfscope}%
\begin{pgfscope}%
\pgfsys@transformshift{3.565535in}{1.094783in}%
\pgfsys@useobject{currentmarker}{}%
\end{pgfscope}%
\begin{pgfscope}%
\pgfsys@transformshift{3.566445in}{1.093937in}%
\pgfsys@useobject{currentmarker}{}%
\end{pgfscope}%
\begin{pgfscope}%
\pgfsys@transformshift{3.567353in}{1.091733in}%
\pgfsys@useobject{currentmarker}{}%
\end{pgfscope}%
\begin{pgfscope}%
\pgfsys@transformshift{3.568259in}{1.100143in}%
\pgfsys@useobject{currentmarker}{}%
\end{pgfscope}%
\begin{pgfscope}%
\pgfsys@transformshift{3.569163in}{1.102326in}%
\pgfsys@useobject{currentmarker}{}%
\end{pgfscope}%
\begin{pgfscope}%
\pgfsys@transformshift{3.570066in}{1.077102in}%
\pgfsys@useobject{currentmarker}{}%
\end{pgfscope}%
\begin{pgfscope}%
\pgfsys@transformshift{3.570966in}{1.065675in}%
\pgfsys@useobject{currentmarker}{}%
\end{pgfscope}%
\begin{pgfscope}%
\pgfsys@transformshift{3.571865in}{1.079477in}%
\pgfsys@useobject{currentmarker}{}%
\end{pgfscope}%
\begin{pgfscope}%
\pgfsys@transformshift{3.572762in}{1.079356in}%
\pgfsys@useobject{currentmarker}{}%
\end{pgfscope}%
\begin{pgfscope}%
\pgfsys@transformshift{3.573657in}{1.075789in}%
\pgfsys@useobject{currentmarker}{}%
\end{pgfscope}%
\begin{pgfscope}%
\pgfsys@transformshift{3.574550in}{1.072771in}%
\pgfsys@useobject{currentmarker}{}%
\end{pgfscope}%
\begin{pgfscope}%
\pgfsys@transformshift{3.575442in}{1.075804in}%
\pgfsys@useobject{currentmarker}{}%
\end{pgfscope}%
\begin{pgfscope}%
\pgfsys@transformshift{3.576332in}{1.093588in}%
\pgfsys@useobject{currentmarker}{}%
\end{pgfscope}%
\begin{pgfscope}%
\pgfsys@transformshift{3.577220in}{1.083733in}%
\pgfsys@useobject{currentmarker}{}%
\end{pgfscope}%
\begin{pgfscope}%
\pgfsys@transformshift{3.578106in}{1.096679in}%
\pgfsys@useobject{currentmarker}{}%
\end{pgfscope}%
\begin{pgfscope}%
\pgfsys@transformshift{3.578991in}{1.104411in}%
\pgfsys@useobject{currentmarker}{}%
\end{pgfscope}%
\begin{pgfscope}%
\pgfsys@transformshift{3.579873in}{1.085639in}%
\pgfsys@useobject{currentmarker}{}%
\end{pgfscope}%
\begin{pgfscope}%
\pgfsys@transformshift{3.580754in}{1.078045in}%
\pgfsys@useobject{currentmarker}{}%
\end{pgfscope}%
\begin{pgfscope}%
\pgfsys@transformshift{3.581633in}{1.087525in}%
\pgfsys@useobject{currentmarker}{}%
\end{pgfscope}%
\begin{pgfscope}%
\pgfsys@transformshift{3.582511in}{1.088296in}%
\pgfsys@useobject{currentmarker}{}%
\end{pgfscope}%
\begin{pgfscope}%
\pgfsys@transformshift{3.583387in}{1.082297in}%
\pgfsys@useobject{currentmarker}{}%
\end{pgfscope}%
\begin{pgfscope}%
\pgfsys@transformshift{3.584261in}{1.086997in}%
\pgfsys@useobject{currentmarker}{}%
\end{pgfscope}%
\begin{pgfscope}%
\pgfsys@transformshift{3.585133in}{1.081620in}%
\pgfsys@useobject{currentmarker}{}%
\end{pgfscope}%
\begin{pgfscope}%
\pgfsys@transformshift{3.586004in}{1.100529in}%
\pgfsys@useobject{currentmarker}{}%
\end{pgfscope}%
\begin{pgfscope}%
\pgfsys@transformshift{3.586873in}{1.097329in}%
\pgfsys@useobject{currentmarker}{}%
\end{pgfscope}%
\begin{pgfscope}%
\pgfsys@transformshift{3.587740in}{1.100936in}%
\pgfsys@useobject{currentmarker}{}%
\end{pgfscope}%
\begin{pgfscope}%
\pgfsys@transformshift{3.588605in}{1.099626in}%
\pgfsys@useobject{currentmarker}{}%
\end{pgfscope}%
\begin{pgfscope}%
\pgfsys@transformshift{3.589469in}{1.083383in}%
\pgfsys@useobject{currentmarker}{}%
\end{pgfscope}%
\begin{pgfscope}%
\pgfsys@transformshift{3.590331in}{1.085777in}%
\pgfsys@useobject{currentmarker}{}%
\end{pgfscope}%
\begin{pgfscope}%
\pgfsys@transformshift{3.591192in}{1.095035in}%
\pgfsys@useobject{currentmarker}{}%
\end{pgfscope}%
\begin{pgfscope}%
\pgfsys@transformshift{3.592051in}{1.086408in}%
\pgfsys@useobject{currentmarker}{}%
\end{pgfscope}%
\begin{pgfscope}%
\pgfsys@transformshift{3.592908in}{1.075028in}%
\pgfsys@useobject{currentmarker}{}%
\end{pgfscope}%
\begin{pgfscope}%
\pgfsys@transformshift{3.593763in}{1.083978in}%
\pgfsys@useobject{currentmarker}{}%
\end{pgfscope}%
\begin{pgfscope}%
\pgfsys@transformshift{3.594617in}{1.076995in}%
\pgfsys@useobject{currentmarker}{}%
\end{pgfscope}%
\begin{pgfscope}%
\pgfsys@transformshift{3.595470in}{1.104639in}%
\pgfsys@useobject{currentmarker}{}%
\end{pgfscope}%
\begin{pgfscope}%
\pgfsys@transformshift{3.596320in}{1.106533in}%
\pgfsys@useobject{currentmarker}{}%
\end{pgfscope}%
\begin{pgfscope}%
\pgfsys@transformshift{3.597169in}{1.103506in}%
\pgfsys@useobject{currentmarker}{}%
\end{pgfscope}%
\begin{pgfscope}%
\pgfsys@transformshift{3.598017in}{1.108192in}%
\pgfsys@useobject{currentmarker}{}%
\end{pgfscope}%
\begin{pgfscope}%
\pgfsys@transformshift{3.598862in}{1.102877in}%
\pgfsys@useobject{currentmarker}{}%
\end{pgfscope}%
\begin{pgfscope}%
\pgfsys@transformshift{3.599706in}{1.079966in}%
\pgfsys@useobject{currentmarker}{}%
\end{pgfscope}%
\begin{pgfscope}%
\pgfsys@transformshift{3.600549in}{1.086353in}%
\pgfsys@useobject{currentmarker}{}%
\end{pgfscope}%
\begin{pgfscope}%
\pgfsys@transformshift{3.601390in}{1.089048in}%
\pgfsys@useobject{currentmarker}{}%
\end{pgfscope}%
\begin{pgfscope}%
\pgfsys@transformshift{3.602229in}{1.095510in}%
\pgfsys@useobject{currentmarker}{}%
\end{pgfscope}%
\begin{pgfscope}%
\pgfsys@transformshift{3.603067in}{1.102471in}%
\pgfsys@useobject{currentmarker}{}%
\end{pgfscope}%
\begin{pgfscope}%
\pgfsys@transformshift{3.603903in}{1.104856in}%
\pgfsys@useobject{currentmarker}{}%
\end{pgfscope}%
\begin{pgfscope}%
\pgfsys@transformshift{3.604738in}{1.092503in}%
\pgfsys@useobject{currentmarker}{}%
\end{pgfscope}%
\begin{pgfscope}%
\pgfsys@transformshift{3.605571in}{1.084620in}%
\pgfsys@useobject{currentmarker}{}%
\end{pgfscope}%
\begin{pgfscope}%
\pgfsys@transformshift{3.606403in}{1.069393in}%
\pgfsys@useobject{currentmarker}{}%
\end{pgfscope}%
\begin{pgfscope}%
\pgfsys@transformshift{3.607233in}{1.071821in}%
\pgfsys@useobject{currentmarker}{}%
\end{pgfscope}%
\begin{pgfscope}%
\pgfsys@transformshift{3.608061in}{1.067014in}%
\pgfsys@useobject{currentmarker}{}%
\end{pgfscope}%
\begin{pgfscope}%
\pgfsys@transformshift{3.608888in}{1.082511in}%
\pgfsys@useobject{currentmarker}{}%
\end{pgfscope}%
\begin{pgfscope}%
\pgfsys@transformshift{3.609713in}{1.065288in}%
\pgfsys@useobject{currentmarker}{}%
\end{pgfscope}%
\begin{pgfscope}%
\pgfsys@transformshift{3.610537in}{1.076707in}%
\pgfsys@useobject{currentmarker}{}%
\end{pgfscope}%
\begin{pgfscope}%
\pgfsys@transformshift{3.611359in}{1.082279in}%
\pgfsys@useobject{currentmarker}{}%
\end{pgfscope}%
\begin{pgfscope}%
\pgfsys@transformshift{3.612180in}{1.097203in}%
\pgfsys@useobject{currentmarker}{}%
\end{pgfscope}%
\begin{pgfscope}%
\pgfsys@transformshift{3.612999in}{1.097008in}%
\pgfsys@useobject{currentmarker}{}%
\end{pgfscope}%
\begin{pgfscope}%
\pgfsys@transformshift{3.613817in}{1.096009in}%
\pgfsys@useobject{currentmarker}{}%
\end{pgfscope}%
\begin{pgfscope}%
\pgfsys@transformshift{3.614633in}{1.094591in}%
\pgfsys@useobject{currentmarker}{}%
\end{pgfscope}%
\begin{pgfscope}%
\pgfsys@transformshift{3.615448in}{1.099515in}%
\pgfsys@useobject{currentmarker}{}%
\end{pgfscope}%
\begin{pgfscope}%
\pgfsys@transformshift{3.616261in}{1.090555in}%
\pgfsys@useobject{currentmarker}{}%
\end{pgfscope}%
\begin{pgfscope}%
\pgfsys@transformshift{3.617073in}{1.100263in}%
\pgfsys@useobject{currentmarker}{}%
\end{pgfscope}%
\begin{pgfscope}%
\pgfsys@transformshift{3.617884in}{1.095127in}%
\pgfsys@useobject{currentmarker}{}%
\end{pgfscope}%
\begin{pgfscope}%
\pgfsys@transformshift{3.618692in}{1.114810in}%
\pgfsys@useobject{currentmarker}{}%
\end{pgfscope}%
\begin{pgfscope}%
\pgfsys@transformshift{3.619500in}{1.132853in}%
\pgfsys@useobject{currentmarker}{}%
\end{pgfscope}%
\begin{pgfscope}%
\pgfsys@transformshift{3.620306in}{1.127936in}%
\pgfsys@useobject{currentmarker}{}%
\end{pgfscope}%
\begin{pgfscope}%
\pgfsys@transformshift{3.621110in}{1.097408in}%
\pgfsys@useobject{currentmarker}{}%
\end{pgfscope}%
\begin{pgfscope}%
\pgfsys@transformshift{3.621913in}{1.091156in}%
\pgfsys@useobject{currentmarker}{}%
\end{pgfscope}%
\begin{pgfscope}%
\pgfsys@transformshift{3.622715in}{1.106270in}%
\pgfsys@useobject{currentmarker}{}%
\end{pgfscope}%
\begin{pgfscope}%
\pgfsys@transformshift{3.623515in}{1.107635in}%
\pgfsys@useobject{currentmarker}{}%
\end{pgfscope}%
\begin{pgfscope}%
\pgfsys@transformshift{3.624313in}{1.100284in}%
\pgfsys@useobject{currentmarker}{}%
\end{pgfscope}%
\begin{pgfscope}%
\pgfsys@transformshift{3.625111in}{1.093804in}%
\pgfsys@useobject{currentmarker}{}%
\end{pgfscope}%
\begin{pgfscope}%
\pgfsys@transformshift{3.625906in}{1.100331in}%
\pgfsys@useobject{currentmarker}{}%
\end{pgfscope}%
\begin{pgfscope}%
\pgfsys@transformshift{3.626701in}{1.091660in}%
\pgfsys@useobject{currentmarker}{}%
\end{pgfscope}%
\begin{pgfscope}%
\pgfsys@transformshift{3.627494in}{1.106912in}%
\pgfsys@useobject{currentmarker}{}%
\end{pgfscope}%
\begin{pgfscope}%
\pgfsys@transformshift{3.628285in}{1.101280in}%
\pgfsys@useobject{currentmarker}{}%
\end{pgfscope}%
\begin{pgfscope}%
\pgfsys@transformshift{3.629075in}{1.084204in}%
\pgfsys@useobject{currentmarker}{}%
\end{pgfscope}%
\begin{pgfscope}%
\pgfsys@transformshift{3.629864in}{1.099520in}%
\pgfsys@useobject{currentmarker}{}%
\end{pgfscope}%
\begin{pgfscope}%
\pgfsys@transformshift{3.630651in}{1.102112in}%
\pgfsys@useobject{currentmarker}{}%
\end{pgfscope}%
\begin{pgfscope}%
\pgfsys@transformshift{3.631437in}{1.090524in}%
\pgfsys@useobject{currentmarker}{}%
\end{pgfscope}%
\begin{pgfscope}%
\pgfsys@transformshift{3.632222in}{1.088079in}%
\pgfsys@useobject{currentmarker}{}%
\end{pgfscope}%
\begin{pgfscope}%
\pgfsys@transformshift{3.633005in}{1.088419in}%
\pgfsys@useobject{currentmarker}{}%
\end{pgfscope}%
\begin{pgfscope}%
\pgfsys@transformshift{3.633787in}{1.092057in}%
\pgfsys@useobject{currentmarker}{}%
\end{pgfscope}%
\begin{pgfscope}%
\pgfsys@transformshift{3.634567in}{1.092458in}%
\pgfsys@useobject{currentmarker}{}%
\end{pgfscope}%
\begin{pgfscope}%
\pgfsys@transformshift{3.635346in}{1.089778in}%
\pgfsys@useobject{currentmarker}{}%
\end{pgfscope}%
\begin{pgfscope}%
\pgfsys@transformshift{3.636124in}{1.099423in}%
\pgfsys@useobject{currentmarker}{}%
\end{pgfscope}%
\begin{pgfscope}%
\pgfsys@transformshift{3.636900in}{1.092092in}%
\pgfsys@useobject{currentmarker}{}%
\end{pgfscope}%
\begin{pgfscope}%
\pgfsys@transformshift{3.637675in}{1.091483in}%
\pgfsys@useobject{currentmarker}{}%
\end{pgfscope}%
\begin{pgfscope}%
\pgfsys@transformshift{3.638449in}{1.091084in}%
\pgfsys@useobject{currentmarker}{}%
\end{pgfscope}%
\begin{pgfscope}%
\pgfsys@transformshift{3.639221in}{1.089557in}%
\pgfsys@useobject{currentmarker}{}%
\end{pgfscope}%
\begin{pgfscope}%
\pgfsys@transformshift{3.639992in}{1.091414in}%
\pgfsys@useobject{currentmarker}{}%
\end{pgfscope}%
\begin{pgfscope}%
\pgfsys@transformshift{3.640762in}{1.088810in}%
\pgfsys@useobject{currentmarker}{}%
\end{pgfscope}%
\begin{pgfscope}%
\pgfsys@transformshift{3.641530in}{1.100765in}%
\pgfsys@useobject{currentmarker}{}%
\end{pgfscope}%
\begin{pgfscope}%
\pgfsys@transformshift{3.642297in}{1.102728in}%
\pgfsys@useobject{currentmarker}{}%
\end{pgfscope}%
\begin{pgfscope}%
\pgfsys@transformshift{3.643063in}{1.095768in}%
\pgfsys@useobject{currentmarker}{}%
\end{pgfscope}%
\begin{pgfscope}%
\pgfsys@transformshift{3.643827in}{1.086083in}%
\pgfsys@useobject{currentmarker}{}%
\end{pgfscope}%
\begin{pgfscope}%
\pgfsys@transformshift{3.644590in}{1.101626in}%
\pgfsys@useobject{currentmarker}{}%
\end{pgfscope}%
\begin{pgfscope}%
\pgfsys@transformshift{3.645352in}{1.114790in}%
\pgfsys@useobject{currentmarker}{}%
\end{pgfscope}%
\begin{pgfscope}%
\pgfsys@transformshift{3.646112in}{1.102889in}%
\pgfsys@useobject{currentmarker}{}%
\end{pgfscope}%
\begin{pgfscope}%
\pgfsys@transformshift{3.646871in}{1.091159in}%
\pgfsys@useobject{currentmarker}{}%
\end{pgfscope}%
\begin{pgfscope}%
\pgfsys@transformshift{3.647629in}{1.084598in}%
\pgfsys@useobject{currentmarker}{}%
\end{pgfscope}%
\begin{pgfscope}%
\pgfsys@transformshift{3.648385in}{1.080873in}%
\pgfsys@useobject{currentmarker}{}%
\end{pgfscope}%
\begin{pgfscope}%
\pgfsys@transformshift{3.649141in}{1.096081in}%
\pgfsys@useobject{currentmarker}{}%
\end{pgfscope}%
\begin{pgfscope}%
\pgfsys@transformshift{3.649895in}{1.088973in}%
\pgfsys@useobject{currentmarker}{}%
\end{pgfscope}%
\begin{pgfscope}%
\pgfsys@transformshift{3.650647in}{1.106697in}%
\pgfsys@useobject{currentmarker}{}%
\end{pgfscope}%
\begin{pgfscope}%
\pgfsys@transformshift{3.651399in}{1.104754in}%
\pgfsys@useobject{currentmarker}{}%
\end{pgfscope}%
\begin{pgfscope}%
\pgfsys@transformshift{3.652149in}{1.111953in}%
\pgfsys@useobject{currentmarker}{}%
\end{pgfscope}%
\begin{pgfscope}%
\pgfsys@transformshift{3.652898in}{1.107109in}%
\pgfsys@useobject{currentmarker}{}%
\end{pgfscope}%
\begin{pgfscope}%
\pgfsys@transformshift{3.653645in}{1.106404in}%
\pgfsys@useobject{currentmarker}{}%
\end{pgfscope}%
\begin{pgfscope}%
\pgfsys@transformshift{3.654392in}{1.114146in}%
\pgfsys@useobject{currentmarker}{}%
\end{pgfscope}%
\begin{pgfscope}%
\pgfsys@transformshift{3.655137in}{1.086845in}%
\pgfsys@useobject{currentmarker}{}%
\end{pgfscope}%
\begin{pgfscope}%
\pgfsys@transformshift{3.655881in}{1.085007in}%
\pgfsys@useobject{currentmarker}{}%
\end{pgfscope}%
\begin{pgfscope}%
\pgfsys@transformshift{3.656623in}{1.085051in}%
\pgfsys@useobject{currentmarker}{}%
\end{pgfscope}%
\begin{pgfscope}%
\pgfsys@transformshift{3.657365in}{1.110253in}%
\pgfsys@useobject{currentmarker}{}%
\end{pgfscope}%
\begin{pgfscope}%
\pgfsys@transformshift{3.658105in}{1.105170in}%
\pgfsys@useobject{currentmarker}{}%
\end{pgfscope}%
\begin{pgfscope}%
\pgfsys@transformshift{3.658844in}{1.110003in}%
\pgfsys@useobject{currentmarker}{}%
\end{pgfscope}%
\begin{pgfscope}%
\pgfsys@transformshift{3.659581in}{1.116032in}%
\pgfsys@useobject{currentmarker}{}%
\end{pgfscope}%
\begin{pgfscope}%
\pgfsys@transformshift{3.660318in}{1.106151in}%
\pgfsys@useobject{currentmarker}{}%
\end{pgfscope}%
\begin{pgfscope}%
\pgfsys@transformshift{3.661053in}{1.104216in}%
\pgfsys@useobject{currentmarker}{}%
\end{pgfscope}%
\begin{pgfscope}%
\pgfsys@transformshift{3.661787in}{1.104730in}%
\pgfsys@useobject{currentmarker}{}%
\end{pgfscope}%
\begin{pgfscope}%
\pgfsys@transformshift{3.662520in}{1.102888in}%
\pgfsys@useobject{currentmarker}{}%
\end{pgfscope}%
\begin{pgfscope}%
\pgfsys@transformshift{3.663251in}{1.085819in}%
\pgfsys@useobject{currentmarker}{}%
\end{pgfscope}%
\begin{pgfscope}%
\pgfsys@transformshift{3.663982in}{1.072822in}%
\pgfsys@useobject{currentmarker}{}%
\end{pgfscope}%
\begin{pgfscope}%
\pgfsys@transformshift{3.664711in}{1.087902in}%
\pgfsys@useobject{currentmarker}{}%
\end{pgfscope}%
\begin{pgfscope}%
\pgfsys@transformshift{3.665439in}{1.111314in}%
\pgfsys@useobject{currentmarker}{}%
\end{pgfscope}%
\begin{pgfscope}%
\pgfsys@transformshift{3.666166in}{1.105178in}%
\pgfsys@useobject{currentmarker}{}%
\end{pgfscope}%
\begin{pgfscope}%
\pgfsys@transformshift{3.666891in}{1.115902in}%
\pgfsys@useobject{currentmarker}{}%
\end{pgfscope}%
\begin{pgfscope}%
\pgfsys@transformshift{3.667616in}{1.122452in}%
\pgfsys@useobject{currentmarker}{}%
\end{pgfscope}%
\begin{pgfscope}%
\pgfsys@transformshift{3.668339in}{1.115263in}%
\pgfsys@useobject{currentmarker}{}%
\end{pgfscope}%
\begin{pgfscope}%
\pgfsys@transformshift{3.669061in}{1.110317in}%
\pgfsys@useobject{currentmarker}{}%
\end{pgfscope}%
\begin{pgfscope}%
\pgfsys@transformshift{3.669782in}{1.107569in}%
\pgfsys@useobject{currentmarker}{}%
\end{pgfscope}%
\begin{pgfscope}%
\pgfsys@transformshift{3.670502in}{1.103941in}%
\pgfsys@useobject{currentmarker}{}%
\end{pgfscope}%
\begin{pgfscope}%
\pgfsys@transformshift{3.671221in}{1.087242in}%
\pgfsys@useobject{currentmarker}{}%
\end{pgfscope}%
\begin{pgfscope}%
\pgfsys@transformshift{3.671938in}{1.085376in}%
\pgfsys@useobject{currentmarker}{}%
\end{pgfscope}%
\begin{pgfscope}%
\pgfsys@transformshift{3.672654in}{1.098662in}%
\pgfsys@useobject{currentmarker}{}%
\end{pgfscope}%
\begin{pgfscope}%
\pgfsys@transformshift{3.673369in}{1.100369in}%
\pgfsys@useobject{currentmarker}{}%
\end{pgfscope}%
\begin{pgfscope}%
\pgfsys@transformshift{3.674083in}{1.090903in}%
\pgfsys@useobject{currentmarker}{}%
\end{pgfscope}%
\begin{pgfscope}%
\pgfsys@transformshift{3.674796in}{1.079550in}%
\pgfsys@useobject{currentmarker}{}%
\end{pgfscope}%
\begin{pgfscope}%
\pgfsys@transformshift{3.675508in}{1.093728in}%
\pgfsys@useobject{currentmarker}{}%
\end{pgfscope}%
\begin{pgfscope}%
\pgfsys@transformshift{3.676218in}{1.127618in}%
\pgfsys@useobject{currentmarker}{}%
\end{pgfscope}%
\begin{pgfscope}%
\pgfsys@transformshift{3.676928in}{1.111886in}%
\pgfsys@useobject{currentmarker}{}%
\end{pgfscope}%
\begin{pgfscope}%
\pgfsys@transformshift{3.677636in}{1.116112in}%
\pgfsys@useobject{currentmarker}{}%
\end{pgfscope}%
\begin{pgfscope}%
\pgfsys@transformshift{3.678343in}{1.110387in}%
\pgfsys@useobject{currentmarker}{}%
\end{pgfscope}%
\begin{pgfscope}%
\pgfsys@transformshift{3.679049in}{1.099625in}%
\pgfsys@useobject{currentmarker}{}%
\end{pgfscope}%
\begin{pgfscope}%
\pgfsys@transformshift{3.679754in}{1.107570in}%
\pgfsys@useobject{currentmarker}{}%
\end{pgfscope}%
\begin{pgfscope}%
\pgfsys@transformshift{3.680458in}{1.109057in}%
\pgfsys@useobject{currentmarker}{}%
\end{pgfscope}%
\begin{pgfscope}%
\pgfsys@transformshift{3.681161in}{1.112267in}%
\pgfsys@useobject{currentmarker}{}%
\end{pgfscope}%
\begin{pgfscope}%
\pgfsys@transformshift{3.681862in}{1.100500in}%
\pgfsys@useobject{currentmarker}{}%
\end{pgfscope}%
\begin{pgfscope}%
\pgfsys@transformshift{3.682563in}{1.099597in}%
\pgfsys@useobject{currentmarker}{}%
\end{pgfscope}%
\begin{pgfscope}%
\pgfsys@transformshift{3.683262in}{1.107796in}%
\pgfsys@useobject{currentmarker}{}%
\end{pgfscope}%
\begin{pgfscope}%
\pgfsys@transformshift{3.683960in}{1.079260in}%
\pgfsys@useobject{currentmarker}{}%
\end{pgfscope}%
\begin{pgfscope}%
\pgfsys@transformshift{3.684658in}{1.077655in}%
\pgfsys@useobject{currentmarker}{}%
\end{pgfscope}%
\begin{pgfscope}%
\pgfsys@transformshift{3.685354in}{1.084691in}%
\pgfsys@useobject{currentmarker}{}%
\end{pgfscope}%
\begin{pgfscope}%
\pgfsys@transformshift{3.686049in}{1.084020in}%
\pgfsys@useobject{currentmarker}{}%
\end{pgfscope}%
\begin{pgfscope}%
\pgfsys@transformshift{3.686743in}{1.099861in}%
\pgfsys@useobject{currentmarker}{}%
\end{pgfscope}%
\begin{pgfscope}%
\pgfsys@transformshift{3.687435in}{1.102212in}%
\pgfsys@useobject{currentmarker}{}%
\end{pgfscope}%
\begin{pgfscope}%
\pgfsys@transformshift{3.688127in}{1.099991in}%
\pgfsys@useobject{currentmarker}{}%
\end{pgfscope}%
\begin{pgfscope}%
\pgfsys@transformshift{3.688818in}{1.114436in}%
\pgfsys@useobject{currentmarker}{}%
\end{pgfscope}%
\begin{pgfscope}%
\pgfsys@transformshift{3.689507in}{1.086963in}%
\pgfsys@useobject{currentmarker}{}%
\end{pgfscope}%
\begin{pgfscope}%
\pgfsys@transformshift{3.690196in}{1.085149in}%
\pgfsys@useobject{currentmarker}{}%
\end{pgfscope}%
\begin{pgfscope}%
\pgfsys@transformshift{3.690883in}{1.107773in}%
\pgfsys@useobject{currentmarker}{}%
\end{pgfscope}%
\begin{pgfscope}%
\pgfsys@transformshift{3.691570in}{1.096746in}%
\pgfsys@useobject{currentmarker}{}%
\end{pgfscope}%
\begin{pgfscope}%
\pgfsys@transformshift{3.692255in}{1.096372in}%
\pgfsys@useobject{currentmarker}{}%
\end{pgfscope}%
\begin{pgfscope}%
\pgfsys@transformshift{3.692940in}{1.102106in}%
\pgfsys@useobject{currentmarker}{}%
\end{pgfscope}%
\begin{pgfscope}%
\pgfsys@transformshift{3.693623in}{1.094440in}%
\pgfsys@useobject{currentmarker}{}%
\end{pgfscope}%
\begin{pgfscope}%
\pgfsys@transformshift{3.694305in}{1.102080in}%
\pgfsys@useobject{currentmarker}{}%
\end{pgfscope}%
\begin{pgfscope}%
\pgfsys@transformshift{3.694986in}{1.112939in}%
\pgfsys@useobject{currentmarker}{}%
\end{pgfscope}%
\begin{pgfscope}%
\pgfsys@transformshift{3.695666in}{1.104457in}%
\pgfsys@useobject{currentmarker}{}%
\end{pgfscope}%
\begin{pgfscope}%
\pgfsys@transformshift{3.696345in}{1.101061in}%
\pgfsys@useobject{currentmarker}{}%
\end{pgfscope}%
\begin{pgfscope}%
\pgfsys@transformshift{3.697023in}{1.102125in}%
\pgfsys@useobject{currentmarker}{}%
\end{pgfscope}%
\begin{pgfscope}%
\pgfsys@transformshift{3.697700in}{1.098444in}%
\pgfsys@useobject{currentmarker}{}%
\end{pgfscope}%
\begin{pgfscope}%
\pgfsys@transformshift{3.698376in}{1.107702in}%
\pgfsys@useobject{currentmarker}{}%
\end{pgfscope}%
\begin{pgfscope}%
\pgfsys@transformshift{3.699051in}{1.104963in}%
\pgfsys@useobject{currentmarker}{}%
\end{pgfscope}%
\begin{pgfscope}%
\pgfsys@transformshift{3.699725in}{1.096745in}%
\pgfsys@useobject{currentmarker}{}%
\end{pgfscope}%
\begin{pgfscope}%
\pgfsys@transformshift{3.700398in}{1.097847in}%
\pgfsys@useobject{currentmarker}{}%
\end{pgfscope}%
\begin{pgfscope}%
\pgfsys@transformshift{3.701070in}{1.106602in}%
\pgfsys@useobject{currentmarker}{}%
\end{pgfscope}%
\begin{pgfscope}%
\pgfsys@transformshift{3.701741in}{1.096362in}%
\pgfsys@useobject{currentmarker}{}%
\end{pgfscope}%
\begin{pgfscope}%
\pgfsys@transformshift{3.702411in}{1.072264in}%
\pgfsys@useobject{currentmarker}{}%
\end{pgfscope}%
\begin{pgfscope}%
\pgfsys@transformshift{3.703079in}{1.086617in}%
\pgfsys@useobject{currentmarker}{}%
\end{pgfscope}%
\begin{pgfscope}%
\pgfsys@transformshift{3.703747in}{1.095401in}%
\pgfsys@useobject{currentmarker}{}%
\end{pgfscope}%
\begin{pgfscope}%
\pgfsys@transformshift{3.704414in}{1.090434in}%
\pgfsys@useobject{currentmarker}{}%
\end{pgfscope}%
\begin{pgfscope}%
\pgfsys@transformshift{3.705080in}{1.098199in}%
\pgfsys@useobject{currentmarker}{}%
\end{pgfscope}%
\begin{pgfscope}%
\pgfsys@transformshift{3.705745in}{1.107136in}%
\pgfsys@useobject{currentmarker}{}%
\end{pgfscope}%
\begin{pgfscope}%
\pgfsys@transformshift{3.706409in}{1.116610in}%
\pgfsys@useobject{currentmarker}{}%
\end{pgfscope}%
\begin{pgfscope}%
\pgfsys@transformshift{3.707072in}{1.112842in}%
\pgfsys@useobject{currentmarker}{}%
\end{pgfscope}%
\begin{pgfscope}%
\pgfsys@transformshift{3.707733in}{1.103528in}%
\pgfsys@useobject{currentmarker}{}%
\end{pgfscope}%
\begin{pgfscope}%
\pgfsys@transformshift{3.708394in}{1.082636in}%
\pgfsys@useobject{currentmarker}{}%
\end{pgfscope}%
\begin{pgfscope}%
\pgfsys@transformshift{3.709054in}{1.083650in}%
\pgfsys@useobject{currentmarker}{}%
\end{pgfscope}%
\begin{pgfscope}%
\pgfsys@transformshift{3.709713in}{1.076769in}%
\pgfsys@useobject{currentmarker}{}%
\end{pgfscope}%
\begin{pgfscope}%
\pgfsys@transformshift{3.710371in}{1.095775in}%
\pgfsys@useobject{currentmarker}{}%
\end{pgfscope}%
\begin{pgfscope}%
\pgfsys@transformshift{3.711028in}{1.099722in}%
\pgfsys@useobject{currentmarker}{}%
\end{pgfscope}%
\begin{pgfscope}%
\pgfsys@transformshift{3.711684in}{1.102856in}%
\pgfsys@useobject{currentmarker}{}%
\end{pgfscope}%
\begin{pgfscope}%
\pgfsys@transformshift{3.712339in}{1.098029in}%
\pgfsys@useobject{currentmarker}{}%
\end{pgfscope}%
\begin{pgfscope}%
\pgfsys@transformshift{3.712993in}{1.099484in}%
\pgfsys@useobject{currentmarker}{}%
\end{pgfscope}%
\begin{pgfscope}%
\pgfsys@transformshift{3.713646in}{1.094790in}%
\pgfsys@useobject{currentmarker}{}%
\end{pgfscope}%
\begin{pgfscope}%
\pgfsys@transformshift{3.714299in}{1.082649in}%
\pgfsys@useobject{currentmarker}{}%
\end{pgfscope}%
\begin{pgfscope}%
\pgfsys@transformshift{3.714950in}{1.075168in}%
\pgfsys@useobject{currentmarker}{}%
\end{pgfscope}%
\begin{pgfscope}%
\pgfsys@transformshift{3.715600in}{1.079227in}%
\pgfsys@useobject{currentmarker}{}%
\end{pgfscope}%
\begin{pgfscope}%
\pgfsys@transformshift{3.716249in}{1.075441in}%
\pgfsys@useobject{currentmarker}{}%
\end{pgfscope}%
\begin{pgfscope}%
\pgfsys@transformshift{3.716898in}{1.080529in}%
\pgfsys@useobject{currentmarker}{}%
\end{pgfscope}%
\begin{pgfscope}%
\pgfsys@transformshift{3.717545in}{1.082637in}%
\pgfsys@useobject{currentmarker}{}%
\end{pgfscope}%
\begin{pgfscope}%
\pgfsys@transformshift{3.718192in}{1.089724in}%
\pgfsys@useobject{currentmarker}{}%
\end{pgfscope}%
\begin{pgfscope}%
\pgfsys@transformshift{3.718837in}{1.093345in}%
\pgfsys@useobject{currentmarker}{}%
\end{pgfscope}%
\begin{pgfscope}%
\pgfsys@transformshift{3.719482in}{1.093051in}%
\pgfsys@useobject{currentmarker}{}%
\end{pgfscope}%
\begin{pgfscope}%
\pgfsys@transformshift{3.720125in}{1.081157in}%
\pgfsys@useobject{currentmarker}{}%
\end{pgfscope}%
\begin{pgfscope}%
\pgfsys@transformshift{3.720768in}{1.078670in}%
\pgfsys@useobject{currentmarker}{}%
\end{pgfscope}%
\begin{pgfscope}%
\pgfsys@transformshift{3.721410in}{1.103660in}%
\pgfsys@useobject{currentmarker}{}%
\end{pgfscope}%
\begin{pgfscope}%
\pgfsys@transformshift{3.722051in}{1.094849in}%
\pgfsys@useobject{currentmarker}{}%
\end{pgfscope}%
\begin{pgfscope}%
\pgfsys@transformshift{3.722691in}{1.094982in}%
\pgfsys@useobject{currentmarker}{}%
\end{pgfscope}%
\begin{pgfscope}%
\pgfsys@transformshift{3.723330in}{1.108391in}%
\pgfsys@useobject{currentmarker}{}%
\end{pgfscope}%
\begin{pgfscope}%
\pgfsys@transformshift{3.723968in}{1.093654in}%
\pgfsys@useobject{currentmarker}{}%
\end{pgfscope}%
\begin{pgfscope}%
\pgfsys@transformshift{3.724605in}{1.097862in}%
\pgfsys@useobject{currentmarker}{}%
\end{pgfscope}%
\begin{pgfscope}%
\pgfsys@transformshift{3.725241in}{1.091051in}%
\pgfsys@useobject{currentmarker}{}%
\end{pgfscope}%
\begin{pgfscope}%
\pgfsys@transformshift{3.725877in}{1.104220in}%
\pgfsys@useobject{currentmarker}{}%
\end{pgfscope}%
\begin{pgfscope}%
\pgfsys@transformshift{3.726511in}{1.090184in}%
\pgfsys@useobject{currentmarker}{}%
\end{pgfscope}%
\begin{pgfscope}%
\pgfsys@transformshift{3.727145in}{1.090871in}%
\pgfsys@useobject{currentmarker}{}%
\end{pgfscope}%
\begin{pgfscope}%
\pgfsys@transformshift{3.727777in}{1.093851in}%
\pgfsys@useobject{currentmarker}{}%
\end{pgfscope}%
\begin{pgfscope}%
\pgfsys@transformshift{3.728409in}{1.097608in}%
\pgfsys@useobject{currentmarker}{}%
\end{pgfscope}%
\begin{pgfscope}%
\pgfsys@transformshift{3.729040in}{1.082360in}%
\pgfsys@useobject{currentmarker}{}%
\end{pgfscope}%
\begin{pgfscope}%
\pgfsys@transformshift{3.729670in}{1.102733in}%
\pgfsys@useobject{currentmarker}{}%
\end{pgfscope}%
\begin{pgfscope}%
\pgfsys@transformshift{3.730299in}{1.100331in}%
\pgfsys@useobject{currentmarker}{}%
\end{pgfscope}%
\begin{pgfscope}%
\pgfsys@transformshift{3.730927in}{1.084415in}%
\pgfsys@useobject{currentmarker}{}%
\end{pgfscope}%
\begin{pgfscope}%
\pgfsys@transformshift{3.731555in}{1.074947in}%
\pgfsys@useobject{currentmarker}{}%
\end{pgfscope}%
\begin{pgfscope}%
\pgfsys@transformshift{3.732181in}{1.080266in}%
\pgfsys@useobject{currentmarker}{}%
\end{pgfscope}%
\begin{pgfscope}%
\pgfsys@transformshift{3.732807in}{1.078689in}%
\pgfsys@useobject{currentmarker}{}%
\end{pgfscope}%
\begin{pgfscope}%
\pgfsys@transformshift{3.733431in}{1.083515in}%
\pgfsys@useobject{currentmarker}{}%
\end{pgfscope}%
\begin{pgfscope}%
\pgfsys@transformshift{3.734055in}{1.099121in}%
\pgfsys@useobject{currentmarker}{}%
\end{pgfscope}%
\begin{pgfscope}%
\pgfsys@transformshift{3.734678in}{1.102719in}%
\pgfsys@useobject{currentmarker}{}%
\end{pgfscope}%
\begin{pgfscope}%
\pgfsys@transformshift{3.735300in}{1.119766in}%
\pgfsys@useobject{currentmarker}{}%
\end{pgfscope}%
\begin{pgfscope}%
\pgfsys@transformshift{3.735921in}{1.111447in}%
\pgfsys@useobject{currentmarker}{}%
\end{pgfscope}%
\begin{pgfscope}%
\pgfsys@transformshift{3.736542in}{1.084313in}%
\pgfsys@useobject{currentmarker}{}%
\end{pgfscope}%
\begin{pgfscope}%
\pgfsys@transformshift{3.737161in}{1.098586in}%
\pgfsys@useobject{currentmarker}{}%
\end{pgfscope}%
\begin{pgfscope}%
\pgfsys@transformshift{3.737780in}{1.095535in}%
\pgfsys@useobject{currentmarker}{}%
\end{pgfscope}%
\begin{pgfscope}%
\pgfsys@transformshift{3.738397in}{1.120266in}%
\pgfsys@useobject{currentmarker}{}%
\end{pgfscope}%
\begin{pgfscope}%
\pgfsys@transformshift{3.739014in}{1.109924in}%
\pgfsys@useobject{currentmarker}{}%
\end{pgfscope}%
\begin{pgfscope}%
\pgfsys@transformshift{3.739630in}{1.106543in}%
\pgfsys@useobject{currentmarker}{}%
\end{pgfscope}%
\begin{pgfscope}%
\pgfsys@transformshift{3.740245in}{1.122894in}%
\pgfsys@useobject{currentmarker}{}%
\end{pgfscope}%
\begin{pgfscope}%
\pgfsys@transformshift{3.740860in}{1.104464in}%
\pgfsys@useobject{currentmarker}{}%
\end{pgfscope}%
\begin{pgfscope}%
\pgfsys@transformshift{3.741473in}{1.109443in}%
\pgfsys@useobject{currentmarker}{}%
\end{pgfscope}%
\begin{pgfscope}%
\pgfsys@transformshift{3.742086in}{1.102555in}%
\pgfsys@useobject{currentmarker}{}%
\end{pgfscope}%
\begin{pgfscope}%
\pgfsys@transformshift{3.742697in}{1.096553in}%
\pgfsys@useobject{currentmarker}{}%
\end{pgfscope}%
\begin{pgfscope}%
\pgfsys@transformshift{3.743308in}{1.089164in}%
\pgfsys@useobject{currentmarker}{}%
\end{pgfscope}%
\begin{pgfscope}%
\pgfsys@transformshift{3.743918in}{1.089794in}%
\pgfsys@useobject{currentmarker}{}%
\end{pgfscope}%
\begin{pgfscope}%
\pgfsys@transformshift{3.744528in}{1.098503in}%
\pgfsys@useobject{currentmarker}{}%
\end{pgfscope}%
\begin{pgfscope}%
\pgfsys@transformshift{3.745136in}{1.082363in}%
\pgfsys@useobject{currentmarker}{}%
\end{pgfscope}%
\begin{pgfscope}%
\pgfsys@transformshift{3.745744in}{1.063729in}%
\pgfsys@useobject{currentmarker}{}%
\end{pgfscope}%
\begin{pgfscope}%
\pgfsys@transformshift{3.746351in}{1.079496in}%
\pgfsys@useobject{currentmarker}{}%
\end{pgfscope}%
\begin{pgfscope}%
\pgfsys@transformshift{3.746956in}{1.089165in}%
\pgfsys@useobject{currentmarker}{}%
\end{pgfscope}%
\begin{pgfscope}%
\pgfsys@transformshift{3.747562in}{1.085522in}%
\pgfsys@useobject{currentmarker}{}%
\end{pgfscope}%
\begin{pgfscope}%
\pgfsys@transformshift{3.748166in}{1.094977in}%
\pgfsys@useobject{currentmarker}{}%
\end{pgfscope}%
\begin{pgfscope}%
\pgfsys@transformshift{3.748769in}{1.100392in}%
\pgfsys@useobject{currentmarker}{}%
\end{pgfscope}%
\begin{pgfscope}%
\pgfsys@transformshift{3.749372in}{1.098957in}%
\pgfsys@useobject{currentmarker}{}%
\end{pgfscope}%
\begin{pgfscope}%
\pgfsys@transformshift{3.749974in}{1.096974in}%
\pgfsys@useobject{currentmarker}{}%
\end{pgfscope}%
\begin{pgfscope}%
\pgfsys@transformshift{3.750575in}{1.107468in}%
\pgfsys@useobject{currentmarker}{}%
\end{pgfscope}%
\begin{pgfscope}%
\pgfsys@transformshift{3.751175in}{1.117348in}%
\pgfsys@useobject{currentmarker}{}%
\end{pgfscope}%
\begin{pgfscope}%
\pgfsys@transformshift{3.751774in}{1.105988in}%
\pgfsys@useobject{currentmarker}{}%
\end{pgfscope}%
\begin{pgfscope}%
\pgfsys@transformshift{3.752373in}{1.099467in}%
\pgfsys@useobject{currentmarker}{}%
\end{pgfscope}%
\begin{pgfscope}%
\pgfsys@transformshift{3.752971in}{1.097415in}%
\pgfsys@useobject{currentmarker}{}%
\end{pgfscope}%
\begin{pgfscope}%
\pgfsys@transformshift{3.753568in}{1.085916in}%
\pgfsys@useobject{currentmarker}{}%
\end{pgfscope}%
\begin{pgfscope}%
\pgfsys@transformshift{3.754164in}{1.069640in}%
\pgfsys@useobject{currentmarker}{}%
\end{pgfscope}%
\begin{pgfscope}%
\pgfsys@transformshift{3.754759in}{1.101184in}%
\pgfsys@useobject{currentmarker}{}%
\end{pgfscope}%
\begin{pgfscope}%
\pgfsys@transformshift{3.755354in}{1.108104in}%
\pgfsys@useobject{currentmarker}{}%
\end{pgfscope}%
\begin{pgfscope}%
\pgfsys@transformshift{3.755948in}{1.094531in}%
\pgfsys@useobject{currentmarker}{}%
\end{pgfscope}%
\begin{pgfscope}%
\pgfsys@transformshift{3.756541in}{1.088791in}%
\pgfsys@useobject{currentmarker}{}%
\end{pgfscope}%
\begin{pgfscope}%
\pgfsys@transformshift{3.757133in}{1.088731in}%
\pgfsys@useobject{currentmarker}{}%
\end{pgfscope}%
\begin{pgfscope}%
\pgfsys@transformshift{3.757724in}{1.093292in}%
\pgfsys@useobject{currentmarker}{}%
\end{pgfscope}%
\begin{pgfscope}%
\pgfsys@transformshift{3.758315in}{1.091527in}%
\pgfsys@useobject{currentmarker}{}%
\end{pgfscope}%
\begin{pgfscope}%
\pgfsys@transformshift{3.758905in}{1.108381in}%
\pgfsys@useobject{currentmarker}{}%
\end{pgfscope}%
\begin{pgfscope}%
\pgfsys@transformshift{3.759494in}{1.103095in}%
\pgfsys@useobject{currentmarker}{}%
\end{pgfscope}%
\begin{pgfscope}%
\pgfsys@transformshift{3.760082in}{1.094050in}%
\pgfsys@useobject{currentmarker}{}%
\end{pgfscope}%
\begin{pgfscope}%
\pgfsys@transformshift{3.760670in}{1.092184in}%
\pgfsys@useobject{currentmarker}{}%
\end{pgfscope}%
\begin{pgfscope}%
\pgfsys@transformshift{3.761256in}{1.096964in}%
\pgfsys@useobject{currentmarker}{}%
\end{pgfscope}%
\begin{pgfscope}%
\pgfsys@transformshift{3.761842in}{1.094704in}%
\pgfsys@useobject{currentmarker}{}%
\end{pgfscope}%
\begin{pgfscope}%
\pgfsys@transformshift{3.762427in}{1.098403in}%
\pgfsys@useobject{currentmarker}{}%
\end{pgfscope}%
\begin{pgfscope}%
\pgfsys@transformshift{3.763012in}{1.108920in}%
\pgfsys@useobject{currentmarker}{}%
\end{pgfscope}%
\begin{pgfscope}%
\pgfsys@transformshift{3.763596in}{1.086550in}%
\pgfsys@useobject{currentmarker}{}%
\end{pgfscope}%
\begin{pgfscope}%
\pgfsys@transformshift{3.764178in}{1.095611in}%
\pgfsys@useobject{currentmarker}{}%
\end{pgfscope}%
\begin{pgfscope}%
\pgfsys@transformshift{3.764761in}{1.104846in}%
\pgfsys@useobject{currentmarker}{}%
\end{pgfscope}%
\begin{pgfscope}%
\pgfsys@transformshift{3.765342in}{1.117375in}%
\pgfsys@useobject{currentmarker}{}%
\end{pgfscope}%
\begin{pgfscope}%
\pgfsys@transformshift{3.765923in}{1.098579in}%
\pgfsys@useobject{currentmarker}{}%
\end{pgfscope}%
\begin{pgfscope}%
\pgfsys@transformshift{3.766502in}{1.074205in}%
\pgfsys@useobject{currentmarker}{}%
\end{pgfscope}%
\begin{pgfscope}%
\pgfsys@transformshift{3.767081in}{1.082078in}%
\pgfsys@useobject{currentmarker}{}%
\end{pgfscope}%
\begin{pgfscope}%
\pgfsys@transformshift{3.767660in}{1.096937in}%
\pgfsys@useobject{currentmarker}{}%
\end{pgfscope}%
\begin{pgfscope}%
\pgfsys@transformshift{3.768237in}{1.090474in}%
\pgfsys@useobject{currentmarker}{}%
\end{pgfscope}%
\begin{pgfscope}%
\pgfsys@transformshift{3.768814in}{1.091681in}%
\pgfsys@useobject{currentmarker}{}%
\end{pgfscope}%
\begin{pgfscope}%
\pgfsys@transformshift{3.769390in}{1.102287in}%
\pgfsys@useobject{currentmarker}{}%
\end{pgfscope}%
\begin{pgfscope}%
\pgfsys@transformshift{3.769966in}{1.101896in}%
\pgfsys@useobject{currentmarker}{}%
\end{pgfscope}%
\begin{pgfscope}%
\pgfsys@transformshift{3.770540in}{1.089394in}%
\pgfsys@useobject{currentmarker}{}%
\end{pgfscope}%
\begin{pgfscope}%
\pgfsys@transformshift{3.771114in}{1.086525in}%
\pgfsys@useobject{currentmarker}{}%
\end{pgfscope}%
\begin{pgfscope}%
\pgfsys@transformshift{3.771687in}{1.068468in}%
\pgfsys@useobject{currentmarker}{}%
\end{pgfscope}%
\begin{pgfscope}%
\pgfsys@transformshift{3.772260in}{1.108637in}%
\pgfsys@useobject{currentmarker}{}%
\end{pgfscope}%
\begin{pgfscope}%
\pgfsys@transformshift{3.772831in}{1.106282in}%
\pgfsys@useobject{currentmarker}{}%
\end{pgfscope}%
\begin{pgfscope}%
\pgfsys@transformshift{3.773402in}{1.090616in}%
\pgfsys@useobject{currentmarker}{}%
\end{pgfscope}%
\begin{pgfscope}%
\pgfsys@transformshift{3.773972in}{1.091593in}%
\pgfsys@useobject{currentmarker}{}%
\end{pgfscope}%
\begin{pgfscope}%
\pgfsys@transformshift{3.774542in}{1.082005in}%
\pgfsys@useobject{currentmarker}{}%
\end{pgfscope}%
\begin{pgfscope}%
\pgfsys@transformshift{3.775110in}{1.090067in}%
\pgfsys@useobject{currentmarker}{}%
\end{pgfscope}%
\begin{pgfscope}%
\pgfsys@transformshift{3.775678in}{1.096624in}%
\pgfsys@useobject{currentmarker}{}%
\end{pgfscope}%
\begin{pgfscope}%
\pgfsys@transformshift{3.776246in}{1.087415in}%
\pgfsys@useobject{currentmarker}{}%
\end{pgfscope}%
\begin{pgfscope}%
\pgfsys@transformshift{3.776812in}{1.106387in}%
\pgfsys@useobject{currentmarker}{}%
\end{pgfscope}%
\begin{pgfscope}%
\pgfsys@transformshift{3.777378in}{1.118836in}%
\pgfsys@useobject{currentmarker}{}%
\end{pgfscope}%
\begin{pgfscope}%
\pgfsys@transformshift{3.777943in}{1.104990in}%
\pgfsys@useobject{currentmarker}{}%
\end{pgfscope}%
\begin{pgfscope}%
\pgfsys@transformshift{3.778508in}{1.101232in}%
\pgfsys@useobject{currentmarker}{}%
\end{pgfscope}%
\begin{pgfscope}%
\pgfsys@transformshift{3.779071in}{1.123068in}%
\pgfsys@useobject{currentmarker}{}%
\end{pgfscope}%
\begin{pgfscope}%
\pgfsys@transformshift{3.779634in}{1.095677in}%
\pgfsys@useobject{currentmarker}{}%
\end{pgfscope}%
\begin{pgfscope}%
\pgfsys@transformshift{3.780196in}{1.099428in}%
\pgfsys@useobject{currentmarker}{}%
\end{pgfscope}%
\begin{pgfscope}%
\pgfsys@transformshift{3.780758in}{1.105008in}%
\pgfsys@useobject{currentmarker}{}%
\end{pgfscope}%
\begin{pgfscope}%
\pgfsys@transformshift{3.781319in}{1.090579in}%
\pgfsys@useobject{currentmarker}{}%
\end{pgfscope}%
\begin{pgfscope}%
\pgfsys@transformshift{3.781879in}{1.081539in}%
\pgfsys@useobject{currentmarker}{}%
\end{pgfscope}%
\begin{pgfscope}%
\pgfsys@transformshift{3.782438in}{1.092657in}%
\pgfsys@useobject{currentmarker}{}%
\end{pgfscope}%
\begin{pgfscope}%
\pgfsys@transformshift{3.782997in}{1.095744in}%
\pgfsys@useobject{currentmarker}{}%
\end{pgfscope}%
\begin{pgfscope}%
\pgfsys@transformshift{3.783555in}{1.090700in}%
\pgfsys@useobject{currentmarker}{}%
\end{pgfscope}%
\begin{pgfscope}%
\pgfsys@transformshift{3.784112in}{1.099852in}%
\pgfsys@useobject{currentmarker}{}%
\end{pgfscope}%
\begin{pgfscope}%
\pgfsys@transformshift{3.784669in}{1.108002in}%
\pgfsys@useobject{currentmarker}{}%
\end{pgfscope}%
\begin{pgfscope}%
\pgfsys@transformshift{3.785225in}{1.100782in}%
\pgfsys@useobject{currentmarker}{}%
\end{pgfscope}%
\begin{pgfscope}%
\pgfsys@transformshift{3.785780in}{1.078500in}%
\pgfsys@useobject{currentmarker}{}%
\end{pgfscope}%
\begin{pgfscope}%
\pgfsys@transformshift{3.786334in}{1.099632in}%
\pgfsys@useobject{currentmarker}{}%
\end{pgfscope}%
\begin{pgfscope}%
\pgfsys@transformshift{3.786888in}{1.099557in}%
\pgfsys@useobject{currentmarker}{}%
\end{pgfscope}%
\begin{pgfscope}%
\pgfsys@transformshift{3.787441in}{1.094793in}%
\pgfsys@useobject{currentmarker}{}%
\end{pgfscope}%
\begin{pgfscope}%
\pgfsys@transformshift{3.787994in}{1.090850in}%
\pgfsys@useobject{currentmarker}{}%
\end{pgfscope}%
\begin{pgfscope}%
\pgfsys@transformshift{3.788546in}{1.103354in}%
\pgfsys@useobject{currentmarker}{}%
\end{pgfscope}%
\begin{pgfscope}%
\pgfsys@transformshift{3.789097in}{1.120457in}%
\pgfsys@useobject{currentmarker}{}%
\end{pgfscope}%
\begin{pgfscope}%
\pgfsys@transformshift{3.789647in}{1.105718in}%
\pgfsys@useobject{currentmarker}{}%
\end{pgfscope}%
\begin{pgfscope}%
\pgfsys@transformshift{3.790197in}{1.092641in}%
\pgfsys@useobject{currentmarker}{}%
\end{pgfscope}%
\begin{pgfscope}%
\pgfsys@transformshift{3.790746in}{1.108105in}%
\pgfsys@useobject{currentmarker}{}%
\end{pgfscope}%
\begin{pgfscope}%
\pgfsys@transformshift{3.791294in}{1.101946in}%
\pgfsys@useobject{currentmarker}{}%
\end{pgfscope}%
\begin{pgfscope}%
\pgfsys@transformshift{3.791842in}{1.089538in}%
\pgfsys@useobject{currentmarker}{}%
\end{pgfscope}%
\begin{pgfscope}%
\pgfsys@transformshift{3.792389in}{1.118539in}%
\pgfsys@useobject{currentmarker}{}%
\end{pgfscope}%
\begin{pgfscope}%
\pgfsys@transformshift{3.792935in}{1.123131in}%
\pgfsys@useobject{currentmarker}{}%
\end{pgfscope}%
\begin{pgfscope}%
\pgfsys@transformshift{3.793481in}{1.097190in}%
\pgfsys@useobject{currentmarker}{}%
\end{pgfscope}%
\begin{pgfscope}%
\pgfsys@transformshift{3.794026in}{1.099456in}%
\pgfsys@useobject{currentmarker}{}%
\end{pgfscope}%
\begin{pgfscope}%
\pgfsys@transformshift{3.794570in}{1.092658in}%
\pgfsys@useobject{currentmarker}{}%
\end{pgfscope}%
\begin{pgfscope}%
\pgfsys@transformshift{3.795114in}{1.113757in}%
\pgfsys@useobject{currentmarker}{}%
\end{pgfscope}%
\begin{pgfscope}%
\pgfsys@transformshift{3.795657in}{1.095905in}%
\pgfsys@useobject{currentmarker}{}%
\end{pgfscope}%
\begin{pgfscope}%
\pgfsys@transformshift{3.796199in}{1.095981in}%
\pgfsys@useobject{currentmarker}{}%
\end{pgfscope}%
\begin{pgfscope}%
\pgfsys@transformshift{3.796741in}{1.118569in}%
\pgfsys@useobject{currentmarker}{}%
\end{pgfscope}%
\begin{pgfscope}%
\pgfsys@transformshift{3.797282in}{1.105482in}%
\pgfsys@useobject{currentmarker}{}%
\end{pgfscope}%
\begin{pgfscope}%
\pgfsys@transformshift{3.797822in}{1.097790in}%
\pgfsys@useobject{currentmarker}{}%
\end{pgfscope}%
\begin{pgfscope}%
\pgfsys@transformshift{3.798362in}{1.089894in}%
\pgfsys@useobject{currentmarker}{}%
\end{pgfscope}%
\begin{pgfscope}%
\pgfsys@transformshift{3.798901in}{1.096322in}%
\pgfsys@useobject{currentmarker}{}%
\end{pgfscope}%
\begin{pgfscope}%
\pgfsys@transformshift{3.799440in}{1.100211in}%
\pgfsys@useobject{currentmarker}{}%
\end{pgfscope}%
\begin{pgfscope}%
\pgfsys@transformshift{3.799977in}{1.098846in}%
\pgfsys@useobject{currentmarker}{}%
\end{pgfscope}%
\begin{pgfscope}%
\pgfsys@transformshift{3.800514in}{1.104055in}%
\pgfsys@useobject{currentmarker}{}%
\end{pgfscope}%
\begin{pgfscope}%
\pgfsys@transformshift{3.801051in}{1.102983in}%
\pgfsys@useobject{currentmarker}{}%
\end{pgfscope}%
\begin{pgfscope}%
\pgfsys@transformshift{3.801587in}{1.089319in}%
\pgfsys@useobject{currentmarker}{}%
\end{pgfscope}%
\begin{pgfscope}%
\pgfsys@transformshift{3.802122in}{1.085387in}%
\pgfsys@useobject{currentmarker}{}%
\end{pgfscope}%
\begin{pgfscope}%
\pgfsys@transformshift{3.802656in}{1.103806in}%
\pgfsys@useobject{currentmarker}{}%
\end{pgfscope}%
\begin{pgfscope}%
\pgfsys@transformshift{3.803190in}{1.113758in}%
\pgfsys@useobject{currentmarker}{}%
\end{pgfscope}%
\begin{pgfscope}%
\pgfsys@transformshift{3.803723in}{1.110853in}%
\pgfsys@useobject{currentmarker}{}%
\end{pgfscope}%
\begin{pgfscope}%
\pgfsys@transformshift{3.804256in}{1.098607in}%
\pgfsys@useobject{currentmarker}{}%
\end{pgfscope}%
\begin{pgfscope}%
\pgfsys@transformshift{3.804788in}{1.111258in}%
\pgfsys@useobject{currentmarker}{}%
\end{pgfscope}%
\begin{pgfscope}%
\pgfsys@transformshift{3.805319in}{1.124047in}%
\pgfsys@useobject{currentmarker}{}%
\end{pgfscope}%
\begin{pgfscope}%
\pgfsys@transformshift{3.805850in}{1.120708in}%
\pgfsys@useobject{currentmarker}{}%
\end{pgfscope}%
\begin{pgfscope}%
\pgfsys@transformshift{3.806380in}{1.119374in}%
\pgfsys@useobject{currentmarker}{}%
\end{pgfscope}%
\begin{pgfscope}%
\pgfsys@transformshift{3.806910in}{1.101597in}%
\pgfsys@useobject{currentmarker}{}%
\end{pgfscope}%
\begin{pgfscope}%
\pgfsys@transformshift{3.807438in}{1.093211in}%
\pgfsys@useobject{currentmarker}{}%
\end{pgfscope}%
\begin{pgfscope}%
\pgfsys@transformshift{3.807966in}{1.096472in}%
\pgfsys@useobject{currentmarker}{}%
\end{pgfscope}%
\begin{pgfscope}%
\pgfsys@transformshift{3.808494in}{1.106836in}%
\pgfsys@useobject{currentmarker}{}%
\end{pgfscope}%
\begin{pgfscope}%
\pgfsys@transformshift{3.809021in}{1.083971in}%
\pgfsys@useobject{currentmarker}{}%
\end{pgfscope}%
\begin{pgfscope}%
\pgfsys@transformshift{3.809547in}{1.092714in}%
\pgfsys@useobject{currentmarker}{}%
\end{pgfscope}%
\begin{pgfscope}%
\pgfsys@transformshift{3.810073in}{1.092319in}%
\pgfsys@useobject{currentmarker}{}%
\end{pgfscope}%
\begin{pgfscope}%
\pgfsys@transformshift{3.810598in}{1.097520in}%
\pgfsys@useobject{currentmarker}{}%
\end{pgfscope}%
\begin{pgfscope}%
\pgfsys@transformshift{3.811122in}{1.103590in}%
\pgfsys@useobject{currentmarker}{}%
\end{pgfscope}%
\begin{pgfscope}%
\pgfsys@transformshift{3.811646in}{1.093275in}%
\pgfsys@useobject{currentmarker}{}%
\end{pgfscope}%
\begin{pgfscope}%
\pgfsys@transformshift{3.812169in}{1.095590in}%
\pgfsys@useobject{currentmarker}{}%
\end{pgfscope}%
\begin{pgfscope}%
\pgfsys@transformshift{3.812692in}{1.110786in}%
\pgfsys@useobject{currentmarker}{}%
\end{pgfscope}%
\begin{pgfscope}%
\pgfsys@transformshift{3.813214in}{1.105838in}%
\pgfsys@useobject{currentmarker}{}%
\end{pgfscope}%
\begin{pgfscope}%
\pgfsys@transformshift{3.813735in}{1.118291in}%
\pgfsys@useobject{currentmarker}{}%
\end{pgfscope}%
\begin{pgfscope}%
\pgfsys@transformshift{3.814256in}{1.127127in}%
\pgfsys@useobject{currentmarker}{}%
\end{pgfscope}%
\begin{pgfscope}%
\pgfsys@transformshift{3.814776in}{1.104572in}%
\pgfsys@useobject{currentmarker}{}%
\end{pgfscope}%
\begin{pgfscope}%
\pgfsys@transformshift{3.815296in}{1.108914in}%
\pgfsys@useobject{currentmarker}{}%
\end{pgfscope}%
\begin{pgfscope}%
\pgfsys@transformshift{3.815815in}{1.122463in}%
\pgfsys@useobject{currentmarker}{}%
\end{pgfscope}%
\begin{pgfscope}%
\pgfsys@transformshift{3.816333in}{1.112906in}%
\pgfsys@useobject{currentmarker}{}%
\end{pgfscope}%
\begin{pgfscope}%
\pgfsys@transformshift{3.816851in}{1.110644in}%
\pgfsys@useobject{currentmarker}{}%
\end{pgfscope}%
\begin{pgfscope}%
\pgfsys@transformshift{3.817368in}{1.108711in}%
\pgfsys@useobject{currentmarker}{}%
\end{pgfscope}%
\begin{pgfscope}%
\pgfsys@transformshift{3.817884in}{1.117479in}%
\pgfsys@useobject{currentmarker}{}%
\end{pgfscope}%
\begin{pgfscope}%
\pgfsys@transformshift{3.818400in}{1.118471in}%
\pgfsys@useobject{currentmarker}{}%
\end{pgfscope}%
\begin{pgfscope}%
\pgfsys@transformshift{3.818915in}{1.121967in}%
\pgfsys@useobject{currentmarker}{}%
\end{pgfscope}%
\begin{pgfscope}%
\pgfsys@transformshift{3.819430in}{1.095930in}%
\pgfsys@useobject{currentmarker}{}%
\end{pgfscope}%
\begin{pgfscope}%
\pgfsys@transformshift{3.819944in}{1.106524in}%
\pgfsys@useobject{currentmarker}{}%
\end{pgfscope}%
\begin{pgfscope}%
\pgfsys@transformshift{3.820458in}{1.126092in}%
\pgfsys@useobject{currentmarker}{}%
\end{pgfscope}%
\begin{pgfscope}%
\pgfsys@transformshift{3.820971in}{1.099725in}%
\pgfsys@useobject{currentmarker}{}%
\end{pgfscope}%
\begin{pgfscope}%
\pgfsys@transformshift{3.821483in}{1.097731in}%
\pgfsys@useobject{currentmarker}{}%
\end{pgfscope}%
\begin{pgfscope}%
\pgfsys@transformshift{3.821995in}{1.102844in}%
\pgfsys@useobject{currentmarker}{}%
\end{pgfscope}%
\begin{pgfscope}%
\pgfsys@transformshift{3.822506in}{1.109219in}%
\pgfsys@useobject{currentmarker}{}%
\end{pgfscope}%
\begin{pgfscope}%
\pgfsys@transformshift{3.823016in}{1.122062in}%
\pgfsys@useobject{currentmarker}{}%
\end{pgfscope}%
\begin{pgfscope}%
\pgfsys@transformshift{3.823526in}{1.109979in}%
\pgfsys@useobject{currentmarker}{}%
\end{pgfscope}%
\begin{pgfscope}%
\pgfsys@transformshift{3.824036in}{1.095933in}%
\pgfsys@useobject{currentmarker}{}%
\end{pgfscope}%
\begin{pgfscope}%
\pgfsys@transformshift{3.824545in}{1.102362in}%
\pgfsys@useobject{currentmarker}{}%
\end{pgfscope}%
\begin{pgfscope}%
\pgfsys@transformshift{3.825053in}{1.102142in}%
\pgfsys@useobject{currentmarker}{}%
\end{pgfscope}%
\begin{pgfscope}%
\pgfsys@transformshift{3.825561in}{1.086996in}%
\pgfsys@useobject{currentmarker}{}%
\end{pgfscope}%
\begin{pgfscope}%
\pgfsys@transformshift{3.826068in}{1.075483in}%
\pgfsys@useobject{currentmarker}{}%
\end{pgfscope}%
\begin{pgfscope}%
\pgfsys@transformshift{3.826574in}{1.084488in}%
\pgfsys@useobject{currentmarker}{}%
\end{pgfscope}%
\begin{pgfscope}%
\pgfsys@transformshift{3.827080in}{1.088628in}%
\pgfsys@useobject{currentmarker}{}%
\end{pgfscope}%
\begin{pgfscope}%
\pgfsys@transformshift{3.827585in}{1.090466in}%
\pgfsys@useobject{currentmarker}{}%
\end{pgfscope}%
\begin{pgfscope}%
\pgfsys@transformshift{3.828090in}{1.084618in}%
\pgfsys@useobject{currentmarker}{}%
\end{pgfscope}%
\begin{pgfscope}%
\pgfsys@transformshift{3.828594in}{1.081628in}%
\pgfsys@useobject{currentmarker}{}%
\end{pgfscope}%
\begin{pgfscope}%
\pgfsys@transformshift{3.829098in}{1.108597in}%
\pgfsys@useobject{currentmarker}{}%
\end{pgfscope}%
\begin{pgfscope}%
\pgfsys@transformshift{3.829601in}{1.105705in}%
\pgfsys@useobject{currentmarker}{}%
\end{pgfscope}%
\begin{pgfscope}%
\pgfsys@transformshift{3.830103in}{1.108406in}%
\pgfsys@useobject{currentmarker}{}%
\end{pgfscope}%
\begin{pgfscope}%
\pgfsys@transformshift{3.830605in}{1.107674in}%
\pgfsys@useobject{currentmarker}{}%
\end{pgfscope}%
\begin{pgfscope}%
\pgfsys@transformshift{3.831107in}{1.111510in}%
\pgfsys@useobject{currentmarker}{}%
\end{pgfscope}%
\begin{pgfscope}%
\pgfsys@transformshift{3.831608in}{1.098904in}%
\pgfsys@useobject{currentmarker}{}%
\end{pgfscope}%
\begin{pgfscope}%
\pgfsys@transformshift{3.832108in}{1.091779in}%
\pgfsys@useobject{currentmarker}{}%
\end{pgfscope}%
\begin{pgfscope}%
\pgfsys@transformshift{3.832607in}{1.104769in}%
\pgfsys@useobject{currentmarker}{}%
\end{pgfscope}%
\begin{pgfscope}%
\pgfsys@transformshift{3.833107in}{1.099650in}%
\pgfsys@useobject{currentmarker}{}%
\end{pgfscope}%
\begin{pgfscope}%
\pgfsys@transformshift{3.833605in}{1.094131in}%
\pgfsys@useobject{currentmarker}{}%
\end{pgfscope}%
\begin{pgfscope}%
\pgfsys@transformshift{3.834103in}{1.079360in}%
\pgfsys@useobject{currentmarker}{}%
\end{pgfscope}%
\begin{pgfscope}%
\pgfsys@transformshift{3.834600in}{1.103740in}%
\pgfsys@useobject{currentmarker}{}%
\end{pgfscope}%
\begin{pgfscope}%
\pgfsys@transformshift{3.835097in}{1.106436in}%
\pgfsys@useobject{currentmarker}{}%
\end{pgfscope}%
\begin{pgfscope}%
\pgfsys@transformshift{3.835594in}{1.114751in}%
\pgfsys@useobject{currentmarker}{}%
\end{pgfscope}%
\begin{pgfscope}%
\pgfsys@transformshift{3.836089in}{1.119675in}%
\pgfsys@useobject{currentmarker}{}%
\end{pgfscope}%
\begin{pgfscope}%
\pgfsys@transformshift{3.836585in}{1.076014in}%
\pgfsys@useobject{currentmarker}{}%
\end{pgfscope}%
\begin{pgfscope}%
\pgfsys@transformshift{3.837079in}{1.081611in}%
\pgfsys@useobject{currentmarker}{}%
\end{pgfscope}%
\begin{pgfscope}%
\pgfsys@transformshift{3.837573in}{1.098453in}%
\pgfsys@useobject{currentmarker}{}%
\end{pgfscope}%
\begin{pgfscope}%
\pgfsys@transformshift{3.838067in}{1.101780in}%
\pgfsys@useobject{currentmarker}{}%
\end{pgfscope}%
\begin{pgfscope}%
\pgfsys@transformshift{3.838560in}{1.088923in}%
\pgfsys@useobject{currentmarker}{}%
\end{pgfscope}%
\begin{pgfscope}%
\pgfsys@transformshift{3.839052in}{1.098270in}%
\pgfsys@useobject{currentmarker}{}%
\end{pgfscope}%
\begin{pgfscope}%
\pgfsys@transformshift{3.839544in}{1.090799in}%
\pgfsys@useobject{currentmarker}{}%
\end{pgfscope}%
\begin{pgfscope}%
\pgfsys@transformshift{3.840036in}{1.091480in}%
\pgfsys@useobject{currentmarker}{}%
\end{pgfscope}%
\begin{pgfscope}%
\pgfsys@transformshift{3.840527in}{1.090484in}%
\pgfsys@useobject{currentmarker}{}%
\end{pgfscope}%
\begin{pgfscope}%
\pgfsys@transformshift{3.841017in}{1.093611in}%
\pgfsys@useobject{currentmarker}{}%
\end{pgfscope}%
\begin{pgfscope}%
\pgfsys@transformshift{3.841507in}{1.105547in}%
\pgfsys@useobject{currentmarker}{}%
\end{pgfscope}%
\begin{pgfscope}%
\pgfsys@transformshift{3.841996in}{1.098672in}%
\pgfsys@useobject{currentmarker}{}%
\end{pgfscope}%
\begin{pgfscope}%
\pgfsys@transformshift{3.842484in}{1.115088in}%
\pgfsys@useobject{currentmarker}{}%
\end{pgfscope}%
\begin{pgfscope}%
\pgfsys@transformshift{3.842973in}{1.094881in}%
\pgfsys@useobject{currentmarker}{}%
\end{pgfscope}%
\begin{pgfscope}%
\pgfsys@transformshift{3.843460in}{1.085364in}%
\pgfsys@useobject{currentmarker}{}%
\end{pgfscope}%
\begin{pgfscope}%
\pgfsys@transformshift{3.843947in}{1.086976in}%
\pgfsys@useobject{currentmarker}{}%
\end{pgfscope}%
\begin{pgfscope}%
\pgfsys@transformshift{3.844434in}{1.081474in}%
\pgfsys@useobject{currentmarker}{}%
\end{pgfscope}%
\begin{pgfscope}%
\pgfsys@transformshift{3.844920in}{1.065761in}%
\pgfsys@useobject{currentmarker}{}%
\end{pgfscope}%
\begin{pgfscope}%
\pgfsys@transformshift{3.845405in}{1.082295in}%
\pgfsys@useobject{currentmarker}{}%
\end{pgfscope}%
\begin{pgfscope}%
\pgfsys@transformshift{3.845890in}{1.113339in}%
\pgfsys@useobject{currentmarker}{}%
\end{pgfscope}%
\begin{pgfscope}%
\pgfsys@transformshift{3.846375in}{1.097626in}%
\pgfsys@useobject{currentmarker}{}%
\end{pgfscope}%
\begin{pgfscope}%
\pgfsys@transformshift{3.846859in}{1.092324in}%
\pgfsys@useobject{currentmarker}{}%
\end{pgfscope}%
\begin{pgfscope}%
\pgfsys@transformshift{3.847342in}{1.096311in}%
\pgfsys@useobject{currentmarker}{}%
\end{pgfscope}%
\begin{pgfscope}%
\pgfsys@transformshift{3.847825in}{1.111614in}%
\pgfsys@useobject{currentmarker}{}%
\end{pgfscope}%
\begin{pgfscope}%
\pgfsys@transformshift{3.848307in}{1.113147in}%
\pgfsys@useobject{currentmarker}{}%
\end{pgfscope}%
\begin{pgfscope}%
\pgfsys@transformshift{3.848789in}{1.106901in}%
\pgfsys@useobject{currentmarker}{}%
\end{pgfscope}%
\begin{pgfscope}%
\pgfsys@transformshift{3.849270in}{1.099027in}%
\pgfsys@useobject{currentmarker}{}%
\end{pgfscope}%
\begin{pgfscope}%
\pgfsys@transformshift{3.849751in}{1.093626in}%
\pgfsys@useobject{currentmarker}{}%
\end{pgfscope}%
\begin{pgfscope}%
\pgfsys@transformshift{3.850231in}{1.094475in}%
\pgfsys@useobject{currentmarker}{}%
\end{pgfscope}%
\begin{pgfscope}%
\pgfsys@transformshift{3.850711in}{1.101835in}%
\pgfsys@useobject{currentmarker}{}%
\end{pgfscope}%
\begin{pgfscope}%
\pgfsys@transformshift{3.851190in}{1.106929in}%
\pgfsys@useobject{currentmarker}{}%
\end{pgfscope}%
\begin{pgfscope}%
\pgfsys@transformshift{3.851669in}{1.123303in}%
\pgfsys@useobject{currentmarker}{}%
\end{pgfscope}%
\begin{pgfscope}%
\pgfsys@transformshift{3.852147in}{1.104698in}%
\pgfsys@useobject{currentmarker}{}%
\end{pgfscope}%
\begin{pgfscope}%
\pgfsys@transformshift{3.852624in}{1.084616in}%
\pgfsys@useobject{currentmarker}{}%
\end{pgfscope}%
\begin{pgfscope}%
\pgfsys@transformshift{3.853102in}{1.090089in}%
\pgfsys@useobject{currentmarker}{}%
\end{pgfscope}%
\begin{pgfscope}%
\pgfsys@transformshift{3.853578in}{1.091688in}%
\pgfsys@useobject{currentmarker}{}%
\end{pgfscope}%
\begin{pgfscope}%
\pgfsys@transformshift{3.854054in}{1.071386in}%
\pgfsys@useobject{currentmarker}{}%
\end{pgfscope}%
\begin{pgfscope}%
\pgfsys@transformshift{3.854530in}{1.106825in}%
\pgfsys@useobject{currentmarker}{}%
\end{pgfscope}%
\begin{pgfscope}%
\pgfsys@transformshift{3.855005in}{1.116553in}%
\pgfsys@useobject{currentmarker}{}%
\end{pgfscope}%
\begin{pgfscope}%
\pgfsys@transformshift{3.855480in}{1.098649in}%
\pgfsys@useobject{currentmarker}{}%
\end{pgfscope}%
\begin{pgfscope}%
\pgfsys@transformshift{3.855954in}{1.105729in}%
\pgfsys@useobject{currentmarker}{}%
\end{pgfscope}%
\begin{pgfscope}%
\pgfsys@transformshift{3.856427in}{1.106422in}%
\pgfsys@useobject{currentmarker}{}%
\end{pgfscope}%
\begin{pgfscope}%
\pgfsys@transformshift{3.856900in}{1.094751in}%
\pgfsys@useobject{currentmarker}{}%
\end{pgfscope}%
\begin{pgfscope}%
\pgfsys@transformshift{3.857373in}{1.103940in}%
\pgfsys@useobject{currentmarker}{}%
\end{pgfscope}%
\begin{pgfscope}%
\pgfsys@transformshift{3.857845in}{1.097197in}%
\pgfsys@useobject{currentmarker}{}%
\end{pgfscope}%
\begin{pgfscope}%
\pgfsys@transformshift{3.858317in}{1.111346in}%
\pgfsys@useobject{currentmarker}{}%
\end{pgfscope}%
\begin{pgfscope}%
\pgfsys@transformshift{3.858788in}{1.114129in}%
\pgfsys@useobject{currentmarker}{}%
\end{pgfscope}%
\begin{pgfscope}%
\pgfsys@transformshift{3.859258in}{1.113475in}%
\pgfsys@useobject{currentmarker}{}%
\end{pgfscope}%
\begin{pgfscope}%
\pgfsys@transformshift{3.859728in}{1.112247in}%
\pgfsys@useobject{currentmarker}{}%
\end{pgfscope}%
\begin{pgfscope}%
\pgfsys@transformshift{3.860198in}{1.119220in}%
\pgfsys@useobject{currentmarker}{}%
\end{pgfscope}%
\begin{pgfscope}%
\pgfsys@transformshift{3.860667in}{1.129651in}%
\pgfsys@useobject{currentmarker}{}%
\end{pgfscope}%
\begin{pgfscope}%
\pgfsys@transformshift{3.861135in}{1.132022in}%
\pgfsys@useobject{currentmarker}{}%
\end{pgfscope}%
\begin{pgfscope}%
\pgfsys@transformshift{3.861604in}{1.106821in}%
\pgfsys@useobject{currentmarker}{}%
\end{pgfscope}%
\begin{pgfscope}%
\pgfsys@transformshift{3.862071in}{1.106051in}%
\pgfsys@useobject{currentmarker}{}%
\end{pgfscope}%
\begin{pgfscope}%
\pgfsys@transformshift{3.862538in}{1.135693in}%
\pgfsys@useobject{currentmarker}{}%
\end{pgfscope}%
\begin{pgfscope}%
\pgfsys@transformshift{3.863005in}{1.130692in}%
\pgfsys@useobject{currentmarker}{}%
\end{pgfscope}%
\begin{pgfscope}%
\pgfsys@transformshift{3.863471in}{1.118678in}%
\pgfsys@useobject{currentmarker}{}%
\end{pgfscope}%
\begin{pgfscope}%
\pgfsys@transformshift{3.863937in}{1.109265in}%
\pgfsys@useobject{currentmarker}{}%
\end{pgfscope}%
\begin{pgfscope}%
\pgfsys@transformshift{3.864402in}{1.106742in}%
\pgfsys@useobject{currentmarker}{}%
\end{pgfscope}%
\begin{pgfscope}%
\pgfsys@transformshift{3.864866in}{1.104303in}%
\pgfsys@useobject{currentmarker}{}%
\end{pgfscope}%
\begin{pgfscope}%
\pgfsys@transformshift{3.865331in}{1.094973in}%
\pgfsys@useobject{currentmarker}{}%
\end{pgfscope}%
\begin{pgfscope}%
\pgfsys@transformshift{3.865794in}{1.067794in}%
\pgfsys@useobject{currentmarker}{}%
\end{pgfscope}%
\begin{pgfscope}%
\pgfsys@transformshift{3.866258in}{1.070222in}%
\pgfsys@useobject{currentmarker}{}%
\end{pgfscope}%
\begin{pgfscope}%
\pgfsys@transformshift{3.866720in}{1.081085in}%
\pgfsys@useobject{currentmarker}{}%
\end{pgfscope}%
\begin{pgfscope}%
\pgfsys@transformshift{3.867182in}{1.099068in}%
\pgfsys@useobject{currentmarker}{}%
\end{pgfscope}%
\begin{pgfscope}%
\pgfsys@transformshift{3.867644in}{1.109055in}%
\pgfsys@useobject{currentmarker}{}%
\end{pgfscope}%
\begin{pgfscope}%
\pgfsys@transformshift{3.868106in}{1.095793in}%
\pgfsys@useobject{currentmarker}{}%
\end{pgfscope}%
\begin{pgfscope}%
\pgfsys@transformshift{3.868566in}{1.109330in}%
\pgfsys@useobject{currentmarker}{}%
\end{pgfscope}%
\begin{pgfscope}%
\pgfsys@transformshift{3.869027in}{1.116224in}%
\pgfsys@useobject{currentmarker}{}%
\end{pgfscope}%
\begin{pgfscope}%
\pgfsys@transformshift{3.869487in}{1.109574in}%
\pgfsys@useobject{currentmarker}{}%
\end{pgfscope}%
\begin{pgfscope}%
\pgfsys@transformshift{3.869946in}{1.108950in}%
\pgfsys@useobject{currentmarker}{}%
\end{pgfscope}%
\begin{pgfscope}%
\pgfsys@transformshift{3.870405in}{1.117998in}%
\pgfsys@useobject{currentmarker}{}%
\end{pgfscope}%
\begin{pgfscope}%
\pgfsys@transformshift{3.870863in}{1.114290in}%
\pgfsys@useobject{currentmarker}{}%
\end{pgfscope}%
\begin{pgfscope}%
\pgfsys@transformshift{3.871321in}{1.106152in}%
\pgfsys@useobject{currentmarker}{}%
\end{pgfscope}%
\begin{pgfscope}%
\pgfsys@transformshift{3.871779in}{1.097245in}%
\pgfsys@useobject{currentmarker}{}%
\end{pgfscope}%
\begin{pgfscope}%
\pgfsys@transformshift{3.872236in}{1.110419in}%
\pgfsys@useobject{currentmarker}{}%
\end{pgfscope}%
\begin{pgfscope}%
\pgfsys@transformshift{3.872692in}{1.118900in}%
\pgfsys@useobject{currentmarker}{}%
\end{pgfscope}%
\begin{pgfscope}%
\pgfsys@transformshift{3.873148in}{1.109477in}%
\pgfsys@useobject{currentmarker}{}%
\end{pgfscope}%
\begin{pgfscope}%
\pgfsys@transformshift{3.873604in}{1.091539in}%
\pgfsys@useobject{currentmarker}{}%
\end{pgfscope}%
\end{pgfscope}%
\begin{pgfscope}%
\pgfpathrectangle{\pgfqpoint{0.661284in}{0.417642in}}{\pgfqpoint{3.365288in}{2.055000in}}%
\pgfusepath{clip}%
\pgfsetbuttcap%
\pgfsetroundjoin%
\pgfsetlinewidth{1.505625pt}%
\definecolor{currentstroke}{rgb}{0.003922,0.450980,0.698039}%
\pgfsetstrokecolor{currentstroke}%
\pgfsetdash{{5.550000pt}{2.400000pt}}{0.000000pt}%
\pgfpathmoveto{\pgfqpoint{0.814251in}{0.511051in}}%
\pgfpathlineto{\pgfqpoint{3.873604in}{0.511051in}}%
\pgfpathlineto{\pgfqpoint{3.873604in}{0.511051in}}%
\pgfusepath{stroke}%
\end{pgfscope}%
\begin{pgfscope}%
\pgfpathrectangle{\pgfqpoint{0.661284in}{0.417642in}}{\pgfqpoint{3.365288in}{2.055000in}}%
\pgfusepath{clip}%
\pgfsetbuttcap%
\pgfsetroundjoin%
\pgfsetlinewidth{1.505625pt}%
\definecolor{currentstroke}{rgb}{0.007843,0.619608,0.450980}%
\pgfsetstrokecolor{currentstroke}%
\pgfsetdash{{5.550000pt}{2.400000pt}}{0.000000pt}%
\pgfpathmoveto{\pgfqpoint{0.814251in}{2.379233in}}%
\pgfpathlineto{\pgfqpoint{3.873604in}{0.559388in}}%
\pgfpathlineto{\pgfqpoint{3.873604in}{0.559388in}}%
\pgfusepath{stroke}%
\end{pgfscope}%
\begin{pgfscope}%
\pgfsetrectcap%
\pgfsetmiterjoin%
\pgfsetlinewidth{0.803000pt}%
\definecolor{currentstroke}{rgb}{0.000000,0.000000,0.000000}%
\pgfsetstrokecolor{currentstroke}%
\pgfsetdash{}{0pt}%
\pgfpathmoveto{\pgfqpoint{0.661284in}{0.417642in}}%
\pgfpathlineto{\pgfqpoint{0.661284in}{2.472642in}}%
\pgfusepath{stroke}%
\end{pgfscope}%
\begin{pgfscope}%
\pgfsetrectcap%
\pgfsetmiterjoin%
\pgfsetlinewidth{0.803000pt}%
\definecolor{currentstroke}{rgb}{0.000000,0.000000,0.000000}%
\pgfsetstrokecolor{currentstroke}%
\pgfsetdash{}{0pt}%
\pgfpathmoveto{\pgfqpoint{4.026572in}{0.417642in}}%
\pgfpathlineto{\pgfqpoint{4.026572in}{2.472642in}}%
\pgfusepath{stroke}%
\end{pgfscope}%
\begin{pgfscope}%
\pgfsetrectcap%
\pgfsetmiterjoin%
\pgfsetlinewidth{0.803000pt}%
\definecolor{currentstroke}{rgb}{0.000000,0.000000,0.000000}%
\pgfsetstrokecolor{currentstroke}%
\pgfsetdash{}{0pt}%
\pgfpathmoveto{\pgfqpoint{0.661284in}{0.417642in}}%
\pgfpathlineto{\pgfqpoint{4.026572in}{0.417642in}}%
\pgfusepath{stroke}%
\end{pgfscope}%
\begin{pgfscope}%
\pgfsetrectcap%
\pgfsetmiterjoin%
\pgfsetlinewidth{0.803000pt}%
\definecolor{currentstroke}{rgb}{0.000000,0.000000,0.000000}%
\pgfsetstrokecolor{currentstroke}%
\pgfsetdash{}{0pt}%
\pgfpathmoveto{\pgfqpoint{0.661284in}{2.472642in}}%
\pgfpathlineto{\pgfqpoint{4.026572in}{2.472642in}}%
\pgfusepath{stroke}%
\end{pgfscope}%
\begin{pgfscope}%
\pgfsetbuttcap%
\pgfsetmiterjoin%
\definecolor{currentfill}{rgb}{1.000000,1.000000,1.000000}%
\pgfsetfillcolor{currentfill}%
\pgfsetfillopacity{0.800000}%
\pgfsetlinewidth{1.003750pt}%
\definecolor{currentstroke}{rgb}{0.800000,0.800000,0.800000}%
\pgfsetstrokecolor{currentstroke}%
\pgfsetstrokeopacity{0.800000}%
\pgfsetdash{}{0pt}%
\pgfpathmoveto{\pgfqpoint{2.948460in}{2.073975in}}%
\pgfpathlineto{\pgfqpoint{3.948794in}{2.073975in}}%
\pgfpathquadraticcurveto{\pgfqpoint{3.971016in}{2.073975in}}{\pgfqpoint{3.971016in}{2.096197in}}%
\pgfpathlineto{\pgfqpoint{3.971016in}{2.394864in}}%
\pgfpathquadraticcurveto{\pgfqpoint{3.971016in}{2.417086in}}{\pgfqpoint{3.948794in}{2.417086in}}%
\pgfpathlineto{\pgfqpoint{2.948460in}{2.417086in}}%
\pgfpathquadraticcurveto{\pgfqpoint{2.926238in}{2.417086in}}{\pgfqpoint{2.926238in}{2.394864in}}%
\pgfpathlineto{\pgfqpoint{2.926238in}{2.096197in}}%
\pgfpathquadraticcurveto{\pgfqpoint{2.926238in}{2.073975in}}{\pgfqpoint{2.948460in}{2.073975in}}%
\pgfpathlineto{\pgfqpoint{2.948460in}{2.073975in}}%
\pgfpathclose%
\pgfusepath{stroke,fill}%
\end{pgfscope}%
\begin{pgfscope}%
\pgfsetbuttcap%
\pgfsetroundjoin%
\pgfsetlinewidth{1.505625pt}%
\definecolor{currentstroke}{rgb}{0.003922,0.450980,0.698039}%
\pgfsetstrokecolor{currentstroke}%
\pgfsetdash{{5.550000pt}{2.400000pt}}{0.000000pt}%
\pgfpathmoveto{\pgfqpoint{2.970683in}{2.333753in}}%
\pgfpathlineto{\pgfqpoint{3.081794in}{2.333753in}}%
\pgfpathlineto{\pgfqpoint{3.192905in}{2.333753in}}%
\pgfusepath{stroke}%
\end{pgfscope}%
\begin{pgfscope}%
\definecolor{textcolor}{rgb}{0.000000,0.000000,0.000000}%
\pgfsetstrokecolor{textcolor}%
\pgfsetfillcolor{textcolor}%
\pgftext[x=3.281794in,y=2.294864in,left,base]{\color{textcolor}\rmfamily\fontsize{8.000000}{9.600000}\selectfont White noise}%
\end{pgfscope}%
\begin{pgfscope}%
\pgfsetbuttcap%
\pgfsetroundjoin%
\pgfsetlinewidth{1.505625pt}%
\definecolor{currentstroke}{rgb}{0.007843,0.619608,0.450980}%
\pgfsetstrokecolor{currentstroke}%
\pgfsetdash{{5.550000pt}{2.400000pt}}{0.000000pt}%
\pgfpathmoveto{\pgfqpoint{2.970683in}{2.178864in}}%
\pgfpathlineto{\pgfqpoint{3.081794in}{2.178864in}}%
\pgfpathlineto{\pgfqpoint{3.192905in}{2.178864in}}%
\pgfusepath{stroke}%
\end{pgfscope}%
\begin{pgfscope}%
\definecolor{textcolor}{rgb}{0.000000,0.000000,0.000000}%
\pgfsetstrokecolor{textcolor}%
\pgfsetfillcolor{textcolor}%
\pgftext[x=3.281794in,y=2.139975in,left,base]{\color{textcolor}\rmfamily\fontsize{8.000000}{9.600000}\selectfont Flicker noise}%
\end{pgfscope}%
\end{pgfpicture}%
\makeatother%
\endgroup%

    \caption{Simulated power spectrum of a Keysight \device{3458A} with autozeroing applied. The dashed lines denote the noise present prior to applying the autozero algorithm.}
    \label{fig:autozero_psd}
\end{figure}

The power spectral density in figure \ref{fig:autozero_psd} confirms an increase in the white noise power as discussed above and it can be determined that the white noise power $\sqrt{h_{-1}}$ has increased from \qty[power-half-as-sqrt, per-mode=symbol]{165}{\nV \Hz\tothe{-0.5}} to \qty[power-half-as-sqrt, per-mode=symbol]{489}{\nV \Hz\tothe{-0.5}}, an increase by a factor of $\sqrt{8.8}$, which is more than estimated by \ref{eqn:autozeroing}, including the factor of $2$ for the decimation, which gauged the increase of $\sqrt{h_{-1}}$ to be $\sqrt{4}$ . The reason for the additional noise was already mentioned above. There is still some substantial $f^{-1}$ noise present at the autozero frequency of \qty{5}{\Hz}. This type of noise is not uncorrelated and therefore the covariance is not zero, hence equation \ref{eqn:adding_white_noise} does not strictly hold. The hypothesis can be confirmed, by moving the corner frequency in the simulation from \qty{1.5}{\Hz} a decade lower to \qty{0.15}{\Hz}. The white noise floor would then only increase by a factor of $\sqrt{4.5}$.

Nonetheless, down to very low frequencies the $f^{-1}$ noise is effectively suppressed and the spectral density is almost perfectly flat.

The Allan deviation plot in figure \ref{fig:autozero_adev} also confirms that white noise is the only component and shows a $\tau^{-\frac 1 2}$ dependence for the full range of integration times.

\begin{figure}[hb]
    \centering
    %% Creator: Matplotlib, PGF backend
%%
%% To include the figure in your LaTeX document, write
%%   \input{<filename>.pgf}
%%
%% Make sure the required packages are loaded in your preamble
%%   \usepackage{pgf}
%%
%% Also ensure that all the required font packages are loaded; for instance,
%% the lmodern package is sometimes necessary when using math font.
%%   \usepackage{lmodern}
%%
%% Figures using additional raster images can only be included by \input if
%% they are in the same directory as the main LaTeX file. For loading figures
%% from other directories you can use the `import` package
%%   \usepackage{import}
%%
%% and then include the figures with
%%   \import{<path to file>}{<filename>.pgf}
%%
%% Matplotlib used the following preamble
%%   \usepackage{siunitx}
%%   \sisetup{per-mode = symbol}%
%%   \usepackage{fontspec}
%%   \makeatletter\@ifpackageloaded{underscore}{}{\usepackage[strings]{underscore}}\makeatother
%%
\begingroup%
\makeatletter%
\begin{pgfpicture}%
\pgfpathrectangle{\pgfpointorigin}{\pgfqpoint{4.068242in}{2.514312in}}%
\pgfusepath{use as bounding box, clip}%
\begin{pgfscope}%
\pgfsetbuttcap%
\pgfsetmiterjoin%
\definecolor{currentfill}{rgb}{1.000000,1.000000,1.000000}%
\pgfsetfillcolor{currentfill}%
\pgfsetlinewidth{0.000000pt}%
\definecolor{currentstroke}{rgb}{1.000000,1.000000,1.000000}%
\pgfsetstrokecolor{currentstroke}%
\pgfsetdash{}{0pt}%
\pgfpathmoveto{\pgfqpoint{0.000000in}{0.000000in}}%
\pgfpathlineto{\pgfqpoint{4.068242in}{0.000000in}}%
\pgfpathlineto{\pgfqpoint{4.068242in}{2.514312in}}%
\pgfpathlineto{\pgfqpoint{0.000000in}{2.514312in}}%
\pgfpathlineto{\pgfqpoint{0.000000in}{0.000000in}}%
\pgfpathclose%
\pgfusepath{fill}%
\end{pgfscope}%
\begin{pgfscope}%
\pgfsetbuttcap%
\pgfsetmiterjoin%
\definecolor{currentfill}{rgb}{1.000000,1.000000,1.000000}%
\pgfsetfillcolor{currentfill}%
\pgfsetlinewidth{0.000000pt}%
\definecolor{currentstroke}{rgb}{0.000000,0.000000,0.000000}%
\pgfsetstrokecolor{currentstroke}%
\pgfsetstrokeopacity{0.000000}%
\pgfsetdash{}{0pt}%
\pgfpathmoveto{\pgfqpoint{0.589510in}{0.417642in}}%
\pgfpathlineto{\pgfqpoint{4.026572in}{0.417642in}}%
\pgfpathlineto{\pgfqpoint{4.026572in}{2.472642in}}%
\pgfpathlineto{\pgfqpoint{0.589510in}{2.472642in}}%
\pgfpathlineto{\pgfqpoint{0.589510in}{0.417642in}}%
\pgfpathclose%
\pgfusepath{fill}%
\end{pgfscope}%
\begin{pgfscope}%
\pgfpathrectangle{\pgfqpoint{0.589510in}{0.417642in}}{\pgfqpoint{3.437062in}{2.055000in}}%
\pgfusepath{clip}%
\pgfsetrectcap%
\pgfsetroundjoin%
\pgfsetlinewidth{0.803000pt}%
\definecolor{currentstroke}{rgb}{0.450000,0.450000,0.450000}%
\pgfsetstrokecolor{currentstroke}%
\pgfsetdash{}{0pt}%
\pgfpathmoveto{\pgfqpoint{0.988710in}{0.417642in}}%
\pgfpathlineto{\pgfqpoint{0.988710in}{2.472642in}}%
\pgfusepath{stroke}%
\end{pgfscope}%
\begin{pgfscope}%
\pgfsetbuttcap%
\pgfsetroundjoin%
\definecolor{currentfill}{rgb}{0.000000,0.000000,0.000000}%
\pgfsetfillcolor{currentfill}%
\pgfsetlinewidth{0.803000pt}%
\definecolor{currentstroke}{rgb}{0.000000,0.000000,0.000000}%
\pgfsetstrokecolor{currentstroke}%
\pgfsetdash{}{0pt}%
\pgfsys@defobject{currentmarker}{\pgfqpoint{0.000000in}{-0.048611in}}{\pgfqpoint{0.000000in}{0.000000in}}{%
\pgfpathmoveto{\pgfqpoint{0.000000in}{0.000000in}}%
\pgfpathlineto{\pgfqpoint{0.000000in}{-0.048611in}}%
\pgfusepath{stroke,fill}%
}%
\begin{pgfscope}%
\pgfsys@transformshift{0.988710in}{0.417642in}%
\pgfsys@useobject{currentmarker}{}%
\end{pgfscope}%
\end{pgfscope}%
\begin{pgfscope}%
\definecolor{textcolor}{rgb}{0.000000,0.000000,0.000000}%
\pgfsetstrokecolor{textcolor}%
\pgfsetfillcolor{textcolor}%
\pgftext[x=0.988710in,y=0.320420in,,top]{\color{textcolor}\rmfamily\fontsize{8.000000}{9.600000}\selectfont \(\displaystyle {10^{0}}\)}%
\end{pgfscope}%
\begin{pgfscope}%
\pgfpathrectangle{\pgfqpoint{0.589510in}{0.417642in}}{\pgfqpoint{3.437062in}{2.055000in}}%
\pgfusepath{clip}%
\pgfsetrectcap%
\pgfsetroundjoin%
\pgfsetlinewidth{0.803000pt}%
\definecolor{currentstroke}{rgb}{0.450000,0.450000,0.450000}%
\pgfsetstrokecolor{currentstroke}%
\pgfsetdash{}{0pt}%
\pgfpathmoveto{\pgfqpoint{1.599281in}{0.417642in}}%
\pgfpathlineto{\pgfqpoint{1.599281in}{2.472642in}}%
\pgfusepath{stroke}%
\end{pgfscope}%
\begin{pgfscope}%
\pgfsetbuttcap%
\pgfsetroundjoin%
\definecolor{currentfill}{rgb}{0.000000,0.000000,0.000000}%
\pgfsetfillcolor{currentfill}%
\pgfsetlinewidth{0.803000pt}%
\definecolor{currentstroke}{rgb}{0.000000,0.000000,0.000000}%
\pgfsetstrokecolor{currentstroke}%
\pgfsetdash{}{0pt}%
\pgfsys@defobject{currentmarker}{\pgfqpoint{0.000000in}{-0.048611in}}{\pgfqpoint{0.000000in}{0.000000in}}{%
\pgfpathmoveto{\pgfqpoint{0.000000in}{0.000000in}}%
\pgfpathlineto{\pgfqpoint{0.000000in}{-0.048611in}}%
\pgfusepath{stroke,fill}%
}%
\begin{pgfscope}%
\pgfsys@transformshift{1.599281in}{0.417642in}%
\pgfsys@useobject{currentmarker}{}%
\end{pgfscope}%
\end{pgfscope}%
\begin{pgfscope}%
\definecolor{textcolor}{rgb}{0.000000,0.000000,0.000000}%
\pgfsetstrokecolor{textcolor}%
\pgfsetfillcolor{textcolor}%
\pgftext[x=1.599281in,y=0.320420in,,top]{\color{textcolor}\rmfamily\fontsize{8.000000}{9.600000}\selectfont \(\displaystyle {10^{1}}\)}%
\end{pgfscope}%
\begin{pgfscope}%
\pgfpathrectangle{\pgfqpoint{0.589510in}{0.417642in}}{\pgfqpoint{3.437062in}{2.055000in}}%
\pgfusepath{clip}%
\pgfsetrectcap%
\pgfsetroundjoin%
\pgfsetlinewidth{0.803000pt}%
\definecolor{currentstroke}{rgb}{0.450000,0.450000,0.450000}%
\pgfsetstrokecolor{currentstroke}%
\pgfsetdash{}{0pt}%
\pgfpathmoveto{\pgfqpoint{2.209852in}{0.417642in}}%
\pgfpathlineto{\pgfqpoint{2.209852in}{2.472642in}}%
\pgfusepath{stroke}%
\end{pgfscope}%
\begin{pgfscope}%
\pgfsetbuttcap%
\pgfsetroundjoin%
\definecolor{currentfill}{rgb}{0.000000,0.000000,0.000000}%
\pgfsetfillcolor{currentfill}%
\pgfsetlinewidth{0.803000pt}%
\definecolor{currentstroke}{rgb}{0.000000,0.000000,0.000000}%
\pgfsetstrokecolor{currentstroke}%
\pgfsetdash{}{0pt}%
\pgfsys@defobject{currentmarker}{\pgfqpoint{0.000000in}{-0.048611in}}{\pgfqpoint{0.000000in}{0.000000in}}{%
\pgfpathmoveto{\pgfqpoint{0.000000in}{0.000000in}}%
\pgfpathlineto{\pgfqpoint{0.000000in}{-0.048611in}}%
\pgfusepath{stroke,fill}%
}%
\begin{pgfscope}%
\pgfsys@transformshift{2.209852in}{0.417642in}%
\pgfsys@useobject{currentmarker}{}%
\end{pgfscope}%
\end{pgfscope}%
\begin{pgfscope}%
\definecolor{textcolor}{rgb}{0.000000,0.000000,0.000000}%
\pgfsetstrokecolor{textcolor}%
\pgfsetfillcolor{textcolor}%
\pgftext[x=2.209852in,y=0.320420in,,top]{\color{textcolor}\rmfamily\fontsize{8.000000}{9.600000}\selectfont \(\displaystyle {10^{2}}\)}%
\end{pgfscope}%
\begin{pgfscope}%
\pgfpathrectangle{\pgfqpoint{0.589510in}{0.417642in}}{\pgfqpoint{3.437062in}{2.055000in}}%
\pgfusepath{clip}%
\pgfsetrectcap%
\pgfsetroundjoin%
\pgfsetlinewidth{0.803000pt}%
\definecolor{currentstroke}{rgb}{0.450000,0.450000,0.450000}%
\pgfsetstrokecolor{currentstroke}%
\pgfsetdash{}{0pt}%
\pgfpathmoveto{\pgfqpoint{2.820422in}{0.417642in}}%
\pgfpathlineto{\pgfqpoint{2.820422in}{2.472642in}}%
\pgfusepath{stroke}%
\end{pgfscope}%
\begin{pgfscope}%
\pgfsetbuttcap%
\pgfsetroundjoin%
\definecolor{currentfill}{rgb}{0.000000,0.000000,0.000000}%
\pgfsetfillcolor{currentfill}%
\pgfsetlinewidth{0.803000pt}%
\definecolor{currentstroke}{rgb}{0.000000,0.000000,0.000000}%
\pgfsetstrokecolor{currentstroke}%
\pgfsetdash{}{0pt}%
\pgfsys@defobject{currentmarker}{\pgfqpoint{0.000000in}{-0.048611in}}{\pgfqpoint{0.000000in}{0.000000in}}{%
\pgfpathmoveto{\pgfqpoint{0.000000in}{0.000000in}}%
\pgfpathlineto{\pgfqpoint{0.000000in}{-0.048611in}}%
\pgfusepath{stroke,fill}%
}%
\begin{pgfscope}%
\pgfsys@transformshift{2.820422in}{0.417642in}%
\pgfsys@useobject{currentmarker}{}%
\end{pgfscope}%
\end{pgfscope}%
\begin{pgfscope}%
\definecolor{textcolor}{rgb}{0.000000,0.000000,0.000000}%
\pgfsetstrokecolor{textcolor}%
\pgfsetfillcolor{textcolor}%
\pgftext[x=2.820422in,y=0.320420in,,top]{\color{textcolor}\rmfamily\fontsize{8.000000}{9.600000}\selectfont \(\displaystyle {10^{3}}\)}%
\end{pgfscope}%
\begin{pgfscope}%
\pgfpathrectangle{\pgfqpoint{0.589510in}{0.417642in}}{\pgfqpoint{3.437062in}{2.055000in}}%
\pgfusepath{clip}%
\pgfsetrectcap%
\pgfsetroundjoin%
\pgfsetlinewidth{0.803000pt}%
\definecolor{currentstroke}{rgb}{0.450000,0.450000,0.450000}%
\pgfsetstrokecolor{currentstroke}%
\pgfsetdash{}{0pt}%
\pgfpathmoveto{\pgfqpoint{3.430993in}{0.417642in}}%
\pgfpathlineto{\pgfqpoint{3.430993in}{2.472642in}}%
\pgfusepath{stroke}%
\end{pgfscope}%
\begin{pgfscope}%
\pgfsetbuttcap%
\pgfsetroundjoin%
\definecolor{currentfill}{rgb}{0.000000,0.000000,0.000000}%
\pgfsetfillcolor{currentfill}%
\pgfsetlinewidth{0.803000pt}%
\definecolor{currentstroke}{rgb}{0.000000,0.000000,0.000000}%
\pgfsetstrokecolor{currentstroke}%
\pgfsetdash{}{0pt}%
\pgfsys@defobject{currentmarker}{\pgfqpoint{0.000000in}{-0.048611in}}{\pgfqpoint{0.000000in}{0.000000in}}{%
\pgfpathmoveto{\pgfqpoint{0.000000in}{0.000000in}}%
\pgfpathlineto{\pgfqpoint{0.000000in}{-0.048611in}}%
\pgfusepath{stroke,fill}%
}%
\begin{pgfscope}%
\pgfsys@transformshift{3.430993in}{0.417642in}%
\pgfsys@useobject{currentmarker}{}%
\end{pgfscope}%
\end{pgfscope}%
\begin{pgfscope}%
\definecolor{textcolor}{rgb}{0.000000,0.000000,0.000000}%
\pgfsetstrokecolor{textcolor}%
\pgfsetfillcolor{textcolor}%
\pgftext[x=3.430993in,y=0.320420in,,top]{\color{textcolor}\rmfamily\fontsize{8.000000}{9.600000}\selectfont \(\displaystyle {10^{4}}\)}%
\end{pgfscope}%
\begin{pgfscope}%
\pgfpathrectangle{\pgfqpoint{0.589510in}{0.417642in}}{\pgfqpoint{3.437062in}{2.055000in}}%
\pgfusepath{clip}%
\pgfsetrectcap%
\pgfsetroundjoin%
\pgfsetlinewidth{0.803000pt}%
\definecolor{currentstroke}{rgb}{0.850000,0.850000,0.850000}%
\pgfsetstrokecolor{currentstroke}%
\pgfsetdash{}{0pt}%
\pgfpathmoveto{\pgfqpoint{0.669456in}{0.417642in}}%
\pgfpathlineto{\pgfqpoint{0.669456in}{2.472642in}}%
\pgfusepath{stroke}%
\end{pgfscope}%
\begin{pgfscope}%
\pgfsetbuttcap%
\pgfsetroundjoin%
\definecolor{currentfill}{rgb}{0.000000,0.000000,0.000000}%
\pgfsetfillcolor{currentfill}%
\pgfsetlinewidth{0.602250pt}%
\definecolor{currentstroke}{rgb}{0.000000,0.000000,0.000000}%
\pgfsetstrokecolor{currentstroke}%
\pgfsetdash{}{0pt}%
\pgfsys@defobject{currentmarker}{\pgfqpoint{0.000000in}{-0.027778in}}{\pgfqpoint{0.000000in}{0.000000in}}{%
\pgfpathmoveto{\pgfqpoint{0.000000in}{0.000000in}}%
\pgfpathlineto{\pgfqpoint{0.000000in}{-0.027778in}}%
\pgfusepath{stroke,fill}%
}%
\begin{pgfscope}%
\pgfsys@transformshift{0.669456in}{0.417642in}%
\pgfsys@useobject{currentmarker}{}%
\end{pgfscope}%
\end{pgfscope}%
\begin{pgfscope}%
\pgfpathrectangle{\pgfqpoint{0.589510in}{0.417642in}}{\pgfqpoint{3.437062in}{2.055000in}}%
\pgfusepath{clip}%
\pgfsetrectcap%
\pgfsetroundjoin%
\pgfsetlinewidth{0.803000pt}%
\definecolor{currentstroke}{rgb}{0.850000,0.850000,0.850000}%
\pgfsetstrokecolor{currentstroke}%
\pgfsetdash{}{0pt}%
\pgfpathmoveto{\pgfqpoint{0.745740in}{0.417642in}}%
\pgfpathlineto{\pgfqpoint{0.745740in}{2.472642in}}%
\pgfusepath{stroke}%
\end{pgfscope}%
\begin{pgfscope}%
\pgfsetbuttcap%
\pgfsetroundjoin%
\definecolor{currentfill}{rgb}{0.000000,0.000000,0.000000}%
\pgfsetfillcolor{currentfill}%
\pgfsetlinewidth{0.602250pt}%
\definecolor{currentstroke}{rgb}{0.000000,0.000000,0.000000}%
\pgfsetstrokecolor{currentstroke}%
\pgfsetdash{}{0pt}%
\pgfsys@defobject{currentmarker}{\pgfqpoint{0.000000in}{-0.027778in}}{\pgfqpoint{0.000000in}{0.000000in}}{%
\pgfpathmoveto{\pgfqpoint{0.000000in}{0.000000in}}%
\pgfpathlineto{\pgfqpoint{0.000000in}{-0.027778in}}%
\pgfusepath{stroke,fill}%
}%
\begin{pgfscope}%
\pgfsys@transformshift{0.745740in}{0.417642in}%
\pgfsys@useobject{currentmarker}{}%
\end{pgfscope}%
\end{pgfscope}%
\begin{pgfscope}%
\pgfpathrectangle{\pgfqpoint{0.589510in}{0.417642in}}{\pgfqpoint{3.437062in}{2.055000in}}%
\pgfusepath{clip}%
\pgfsetrectcap%
\pgfsetroundjoin%
\pgfsetlinewidth{0.803000pt}%
\definecolor{currentstroke}{rgb}{0.850000,0.850000,0.850000}%
\pgfsetstrokecolor{currentstroke}%
\pgfsetdash{}{0pt}%
\pgfpathmoveto{\pgfqpoint{0.804910in}{0.417642in}}%
\pgfpathlineto{\pgfqpoint{0.804910in}{2.472642in}}%
\pgfusepath{stroke}%
\end{pgfscope}%
\begin{pgfscope}%
\pgfsetbuttcap%
\pgfsetroundjoin%
\definecolor{currentfill}{rgb}{0.000000,0.000000,0.000000}%
\pgfsetfillcolor{currentfill}%
\pgfsetlinewidth{0.602250pt}%
\definecolor{currentstroke}{rgb}{0.000000,0.000000,0.000000}%
\pgfsetstrokecolor{currentstroke}%
\pgfsetdash{}{0pt}%
\pgfsys@defobject{currentmarker}{\pgfqpoint{0.000000in}{-0.027778in}}{\pgfqpoint{0.000000in}{0.000000in}}{%
\pgfpathmoveto{\pgfqpoint{0.000000in}{0.000000in}}%
\pgfpathlineto{\pgfqpoint{0.000000in}{-0.027778in}}%
\pgfusepath{stroke,fill}%
}%
\begin{pgfscope}%
\pgfsys@transformshift{0.804910in}{0.417642in}%
\pgfsys@useobject{currentmarker}{}%
\end{pgfscope}%
\end{pgfscope}%
\begin{pgfscope}%
\pgfpathrectangle{\pgfqpoint{0.589510in}{0.417642in}}{\pgfqpoint{3.437062in}{2.055000in}}%
\pgfusepath{clip}%
\pgfsetrectcap%
\pgfsetroundjoin%
\pgfsetlinewidth{0.803000pt}%
\definecolor{currentstroke}{rgb}{0.850000,0.850000,0.850000}%
\pgfsetstrokecolor{currentstroke}%
\pgfsetdash{}{0pt}%
\pgfpathmoveto{\pgfqpoint{0.853256in}{0.417642in}}%
\pgfpathlineto{\pgfqpoint{0.853256in}{2.472642in}}%
\pgfusepath{stroke}%
\end{pgfscope}%
\begin{pgfscope}%
\pgfsetbuttcap%
\pgfsetroundjoin%
\definecolor{currentfill}{rgb}{0.000000,0.000000,0.000000}%
\pgfsetfillcolor{currentfill}%
\pgfsetlinewidth{0.602250pt}%
\definecolor{currentstroke}{rgb}{0.000000,0.000000,0.000000}%
\pgfsetstrokecolor{currentstroke}%
\pgfsetdash{}{0pt}%
\pgfsys@defobject{currentmarker}{\pgfqpoint{0.000000in}{-0.027778in}}{\pgfqpoint{0.000000in}{0.000000in}}{%
\pgfpathmoveto{\pgfqpoint{0.000000in}{0.000000in}}%
\pgfpathlineto{\pgfqpoint{0.000000in}{-0.027778in}}%
\pgfusepath{stroke,fill}%
}%
\begin{pgfscope}%
\pgfsys@transformshift{0.853256in}{0.417642in}%
\pgfsys@useobject{currentmarker}{}%
\end{pgfscope}%
\end{pgfscope}%
\begin{pgfscope}%
\pgfpathrectangle{\pgfqpoint{0.589510in}{0.417642in}}{\pgfqpoint{3.437062in}{2.055000in}}%
\pgfusepath{clip}%
\pgfsetrectcap%
\pgfsetroundjoin%
\pgfsetlinewidth{0.803000pt}%
\definecolor{currentstroke}{rgb}{0.850000,0.850000,0.850000}%
\pgfsetstrokecolor{currentstroke}%
\pgfsetdash{}{0pt}%
\pgfpathmoveto{\pgfqpoint{0.894132in}{0.417642in}}%
\pgfpathlineto{\pgfqpoint{0.894132in}{2.472642in}}%
\pgfusepath{stroke}%
\end{pgfscope}%
\begin{pgfscope}%
\pgfsetbuttcap%
\pgfsetroundjoin%
\definecolor{currentfill}{rgb}{0.000000,0.000000,0.000000}%
\pgfsetfillcolor{currentfill}%
\pgfsetlinewidth{0.602250pt}%
\definecolor{currentstroke}{rgb}{0.000000,0.000000,0.000000}%
\pgfsetstrokecolor{currentstroke}%
\pgfsetdash{}{0pt}%
\pgfsys@defobject{currentmarker}{\pgfqpoint{0.000000in}{-0.027778in}}{\pgfqpoint{0.000000in}{0.000000in}}{%
\pgfpathmoveto{\pgfqpoint{0.000000in}{0.000000in}}%
\pgfpathlineto{\pgfqpoint{0.000000in}{-0.027778in}}%
\pgfusepath{stroke,fill}%
}%
\begin{pgfscope}%
\pgfsys@transformshift{0.894132in}{0.417642in}%
\pgfsys@useobject{currentmarker}{}%
\end{pgfscope}%
\end{pgfscope}%
\begin{pgfscope}%
\pgfpathrectangle{\pgfqpoint{0.589510in}{0.417642in}}{\pgfqpoint{3.437062in}{2.055000in}}%
\pgfusepath{clip}%
\pgfsetrectcap%
\pgfsetroundjoin%
\pgfsetlinewidth{0.803000pt}%
\definecolor{currentstroke}{rgb}{0.850000,0.850000,0.850000}%
\pgfsetstrokecolor{currentstroke}%
\pgfsetdash{}{0pt}%
\pgfpathmoveto{\pgfqpoint{0.929540in}{0.417642in}}%
\pgfpathlineto{\pgfqpoint{0.929540in}{2.472642in}}%
\pgfusepath{stroke}%
\end{pgfscope}%
\begin{pgfscope}%
\pgfsetbuttcap%
\pgfsetroundjoin%
\definecolor{currentfill}{rgb}{0.000000,0.000000,0.000000}%
\pgfsetfillcolor{currentfill}%
\pgfsetlinewidth{0.602250pt}%
\definecolor{currentstroke}{rgb}{0.000000,0.000000,0.000000}%
\pgfsetstrokecolor{currentstroke}%
\pgfsetdash{}{0pt}%
\pgfsys@defobject{currentmarker}{\pgfqpoint{0.000000in}{-0.027778in}}{\pgfqpoint{0.000000in}{0.000000in}}{%
\pgfpathmoveto{\pgfqpoint{0.000000in}{0.000000in}}%
\pgfpathlineto{\pgfqpoint{0.000000in}{-0.027778in}}%
\pgfusepath{stroke,fill}%
}%
\begin{pgfscope}%
\pgfsys@transformshift{0.929540in}{0.417642in}%
\pgfsys@useobject{currentmarker}{}%
\end{pgfscope}%
\end{pgfscope}%
\begin{pgfscope}%
\pgfpathrectangle{\pgfqpoint{0.589510in}{0.417642in}}{\pgfqpoint{3.437062in}{2.055000in}}%
\pgfusepath{clip}%
\pgfsetrectcap%
\pgfsetroundjoin%
\pgfsetlinewidth{0.803000pt}%
\definecolor{currentstroke}{rgb}{0.850000,0.850000,0.850000}%
\pgfsetstrokecolor{currentstroke}%
\pgfsetdash{}{0pt}%
\pgfpathmoveto{\pgfqpoint{0.960772in}{0.417642in}}%
\pgfpathlineto{\pgfqpoint{0.960772in}{2.472642in}}%
\pgfusepath{stroke}%
\end{pgfscope}%
\begin{pgfscope}%
\pgfsetbuttcap%
\pgfsetroundjoin%
\definecolor{currentfill}{rgb}{0.000000,0.000000,0.000000}%
\pgfsetfillcolor{currentfill}%
\pgfsetlinewidth{0.602250pt}%
\definecolor{currentstroke}{rgb}{0.000000,0.000000,0.000000}%
\pgfsetstrokecolor{currentstroke}%
\pgfsetdash{}{0pt}%
\pgfsys@defobject{currentmarker}{\pgfqpoint{0.000000in}{-0.027778in}}{\pgfqpoint{0.000000in}{0.000000in}}{%
\pgfpathmoveto{\pgfqpoint{0.000000in}{0.000000in}}%
\pgfpathlineto{\pgfqpoint{0.000000in}{-0.027778in}}%
\pgfusepath{stroke,fill}%
}%
\begin{pgfscope}%
\pgfsys@transformshift{0.960772in}{0.417642in}%
\pgfsys@useobject{currentmarker}{}%
\end{pgfscope}%
\end{pgfscope}%
\begin{pgfscope}%
\pgfpathrectangle{\pgfqpoint{0.589510in}{0.417642in}}{\pgfqpoint{3.437062in}{2.055000in}}%
\pgfusepath{clip}%
\pgfsetrectcap%
\pgfsetroundjoin%
\pgfsetlinewidth{0.803000pt}%
\definecolor{currentstroke}{rgb}{0.850000,0.850000,0.850000}%
\pgfsetstrokecolor{currentstroke}%
\pgfsetdash{}{0pt}%
\pgfpathmoveto{\pgfqpoint{1.172510in}{0.417642in}}%
\pgfpathlineto{\pgfqpoint{1.172510in}{2.472642in}}%
\pgfusepath{stroke}%
\end{pgfscope}%
\begin{pgfscope}%
\pgfsetbuttcap%
\pgfsetroundjoin%
\definecolor{currentfill}{rgb}{0.000000,0.000000,0.000000}%
\pgfsetfillcolor{currentfill}%
\pgfsetlinewidth{0.602250pt}%
\definecolor{currentstroke}{rgb}{0.000000,0.000000,0.000000}%
\pgfsetstrokecolor{currentstroke}%
\pgfsetdash{}{0pt}%
\pgfsys@defobject{currentmarker}{\pgfqpoint{0.000000in}{-0.027778in}}{\pgfqpoint{0.000000in}{0.000000in}}{%
\pgfpathmoveto{\pgfqpoint{0.000000in}{0.000000in}}%
\pgfpathlineto{\pgfqpoint{0.000000in}{-0.027778in}}%
\pgfusepath{stroke,fill}%
}%
\begin{pgfscope}%
\pgfsys@transformshift{1.172510in}{0.417642in}%
\pgfsys@useobject{currentmarker}{}%
\end{pgfscope}%
\end{pgfscope}%
\begin{pgfscope}%
\pgfpathrectangle{\pgfqpoint{0.589510in}{0.417642in}}{\pgfqpoint{3.437062in}{2.055000in}}%
\pgfusepath{clip}%
\pgfsetrectcap%
\pgfsetroundjoin%
\pgfsetlinewidth{0.803000pt}%
\definecolor{currentstroke}{rgb}{0.850000,0.850000,0.850000}%
\pgfsetstrokecolor{currentstroke}%
\pgfsetdash{}{0pt}%
\pgfpathmoveto{\pgfqpoint{1.280027in}{0.417642in}}%
\pgfpathlineto{\pgfqpoint{1.280027in}{2.472642in}}%
\pgfusepath{stroke}%
\end{pgfscope}%
\begin{pgfscope}%
\pgfsetbuttcap%
\pgfsetroundjoin%
\definecolor{currentfill}{rgb}{0.000000,0.000000,0.000000}%
\pgfsetfillcolor{currentfill}%
\pgfsetlinewidth{0.602250pt}%
\definecolor{currentstroke}{rgb}{0.000000,0.000000,0.000000}%
\pgfsetstrokecolor{currentstroke}%
\pgfsetdash{}{0pt}%
\pgfsys@defobject{currentmarker}{\pgfqpoint{0.000000in}{-0.027778in}}{\pgfqpoint{0.000000in}{0.000000in}}{%
\pgfpathmoveto{\pgfqpoint{0.000000in}{0.000000in}}%
\pgfpathlineto{\pgfqpoint{0.000000in}{-0.027778in}}%
\pgfusepath{stroke,fill}%
}%
\begin{pgfscope}%
\pgfsys@transformshift{1.280027in}{0.417642in}%
\pgfsys@useobject{currentmarker}{}%
\end{pgfscope}%
\end{pgfscope}%
\begin{pgfscope}%
\pgfpathrectangle{\pgfqpoint{0.589510in}{0.417642in}}{\pgfqpoint{3.437062in}{2.055000in}}%
\pgfusepath{clip}%
\pgfsetrectcap%
\pgfsetroundjoin%
\pgfsetlinewidth{0.803000pt}%
\definecolor{currentstroke}{rgb}{0.850000,0.850000,0.850000}%
\pgfsetstrokecolor{currentstroke}%
\pgfsetdash{}{0pt}%
\pgfpathmoveto{\pgfqpoint{1.356311in}{0.417642in}}%
\pgfpathlineto{\pgfqpoint{1.356311in}{2.472642in}}%
\pgfusepath{stroke}%
\end{pgfscope}%
\begin{pgfscope}%
\pgfsetbuttcap%
\pgfsetroundjoin%
\definecolor{currentfill}{rgb}{0.000000,0.000000,0.000000}%
\pgfsetfillcolor{currentfill}%
\pgfsetlinewidth{0.602250pt}%
\definecolor{currentstroke}{rgb}{0.000000,0.000000,0.000000}%
\pgfsetstrokecolor{currentstroke}%
\pgfsetdash{}{0pt}%
\pgfsys@defobject{currentmarker}{\pgfqpoint{0.000000in}{-0.027778in}}{\pgfqpoint{0.000000in}{0.000000in}}{%
\pgfpathmoveto{\pgfqpoint{0.000000in}{0.000000in}}%
\pgfpathlineto{\pgfqpoint{0.000000in}{-0.027778in}}%
\pgfusepath{stroke,fill}%
}%
\begin{pgfscope}%
\pgfsys@transformshift{1.356311in}{0.417642in}%
\pgfsys@useobject{currentmarker}{}%
\end{pgfscope}%
\end{pgfscope}%
\begin{pgfscope}%
\pgfpathrectangle{\pgfqpoint{0.589510in}{0.417642in}}{\pgfqpoint{3.437062in}{2.055000in}}%
\pgfusepath{clip}%
\pgfsetrectcap%
\pgfsetroundjoin%
\pgfsetlinewidth{0.803000pt}%
\definecolor{currentstroke}{rgb}{0.850000,0.850000,0.850000}%
\pgfsetstrokecolor{currentstroke}%
\pgfsetdash{}{0pt}%
\pgfpathmoveto{\pgfqpoint{1.415481in}{0.417642in}}%
\pgfpathlineto{\pgfqpoint{1.415481in}{2.472642in}}%
\pgfusepath{stroke}%
\end{pgfscope}%
\begin{pgfscope}%
\pgfsetbuttcap%
\pgfsetroundjoin%
\definecolor{currentfill}{rgb}{0.000000,0.000000,0.000000}%
\pgfsetfillcolor{currentfill}%
\pgfsetlinewidth{0.602250pt}%
\definecolor{currentstroke}{rgb}{0.000000,0.000000,0.000000}%
\pgfsetstrokecolor{currentstroke}%
\pgfsetdash{}{0pt}%
\pgfsys@defobject{currentmarker}{\pgfqpoint{0.000000in}{-0.027778in}}{\pgfqpoint{0.000000in}{0.000000in}}{%
\pgfpathmoveto{\pgfqpoint{0.000000in}{0.000000in}}%
\pgfpathlineto{\pgfqpoint{0.000000in}{-0.027778in}}%
\pgfusepath{stroke,fill}%
}%
\begin{pgfscope}%
\pgfsys@transformshift{1.415481in}{0.417642in}%
\pgfsys@useobject{currentmarker}{}%
\end{pgfscope}%
\end{pgfscope}%
\begin{pgfscope}%
\pgfpathrectangle{\pgfqpoint{0.589510in}{0.417642in}}{\pgfqpoint{3.437062in}{2.055000in}}%
\pgfusepath{clip}%
\pgfsetrectcap%
\pgfsetroundjoin%
\pgfsetlinewidth{0.803000pt}%
\definecolor{currentstroke}{rgb}{0.850000,0.850000,0.850000}%
\pgfsetstrokecolor{currentstroke}%
\pgfsetdash{}{0pt}%
\pgfpathmoveto{\pgfqpoint{1.463827in}{0.417642in}}%
\pgfpathlineto{\pgfqpoint{1.463827in}{2.472642in}}%
\pgfusepath{stroke}%
\end{pgfscope}%
\begin{pgfscope}%
\pgfsetbuttcap%
\pgfsetroundjoin%
\definecolor{currentfill}{rgb}{0.000000,0.000000,0.000000}%
\pgfsetfillcolor{currentfill}%
\pgfsetlinewidth{0.602250pt}%
\definecolor{currentstroke}{rgb}{0.000000,0.000000,0.000000}%
\pgfsetstrokecolor{currentstroke}%
\pgfsetdash{}{0pt}%
\pgfsys@defobject{currentmarker}{\pgfqpoint{0.000000in}{-0.027778in}}{\pgfqpoint{0.000000in}{0.000000in}}{%
\pgfpathmoveto{\pgfqpoint{0.000000in}{0.000000in}}%
\pgfpathlineto{\pgfqpoint{0.000000in}{-0.027778in}}%
\pgfusepath{stroke,fill}%
}%
\begin{pgfscope}%
\pgfsys@transformshift{1.463827in}{0.417642in}%
\pgfsys@useobject{currentmarker}{}%
\end{pgfscope}%
\end{pgfscope}%
\begin{pgfscope}%
\pgfpathrectangle{\pgfqpoint{0.589510in}{0.417642in}}{\pgfqpoint{3.437062in}{2.055000in}}%
\pgfusepath{clip}%
\pgfsetrectcap%
\pgfsetroundjoin%
\pgfsetlinewidth{0.803000pt}%
\definecolor{currentstroke}{rgb}{0.850000,0.850000,0.850000}%
\pgfsetstrokecolor{currentstroke}%
\pgfsetdash{}{0pt}%
\pgfpathmoveto{\pgfqpoint{1.504702in}{0.417642in}}%
\pgfpathlineto{\pgfqpoint{1.504702in}{2.472642in}}%
\pgfusepath{stroke}%
\end{pgfscope}%
\begin{pgfscope}%
\pgfsetbuttcap%
\pgfsetroundjoin%
\definecolor{currentfill}{rgb}{0.000000,0.000000,0.000000}%
\pgfsetfillcolor{currentfill}%
\pgfsetlinewidth{0.602250pt}%
\definecolor{currentstroke}{rgb}{0.000000,0.000000,0.000000}%
\pgfsetstrokecolor{currentstroke}%
\pgfsetdash{}{0pt}%
\pgfsys@defobject{currentmarker}{\pgfqpoint{0.000000in}{-0.027778in}}{\pgfqpoint{0.000000in}{0.000000in}}{%
\pgfpathmoveto{\pgfqpoint{0.000000in}{0.000000in}}%
\pgfpathlineto{\pgfqpoint{0.000000in}{-0.027778in}}%
\pgfusepath{stroke,fill}%
}%
\begin{pgfscope}%
\pgfsys@transformshift{1.504702in}{0.417642in}%
\pgfsys@useobject{currentmarker}{}%
\end{pgfscope}%
\end{pgfscope}%
\begin{pgfscope}%
\pgfpathrectangle{\pgfqpoint{0.589510in}{0.417642in}}{\pgfqpoint{3.437062in}{2.055000in}}%
\pgfusepath{clip}%
\pgfsetrectcap%
\pgfsetroundjoin%
\pgfsetlinewidth{0.803000pt}%
\definecolor{currentstroke}{rgb}{0.850000,0.850000,0.850000}%
\pgfsetstrokecolor{currentstroke}%
\pgfsetdash{}{0pt}%
\pgfpathmoveto{\pgfqpoint{1.540111in}{0.417642in}}%
\pgfpathlineto{\pgfqpoint{1.540111in}{2.472642in}}%
\pgfusepath{stroke}%
\end{pgfscope}%
\begin{pgfscope}%
\pgfsetbuttcap%
\pgfsetroundjoin%
\definecolor{currentfill}{rgb}{0.000000,0.000000,0.000000}%
\pgfsetfillcolor{currentfill}%
\pgfsetlinewidth{0.602250pt}%
\definecolor{currentstroke}{rgb}{0.000000,0.000000,0.000000}%
\pgfsetstrokecolor{currentstroke}%
\pgfsetdash{}{0pt}%
\pgfsys@defobject{currentmarker}{\pgfqpoint{0.000000in}{-0.027778in}}{\pgfqpoint{0.000000in}{0.000000in}}{%
\pgfpathmoveto{\pgfqpoint{0.000000in}{0.000000in}}%
\pgfpathlineto{\pgfqpoint{0.000000in}{-0.027778in}}%
\pgfusepath{stroke,fill}%
}%
\begin{pgfscope}%
\pgfsys@transformshift{1.540111in}{0.417642in}%
\pgfsys@useobject{currentmarker}{}%
\end{pgfscope}%
\end{pgfscope}%
\begin{pgfscope}%
\pgfpathrectangle{\pgfqpoint{0.589510in}{0.417642in}}{\pgfqpoint{3.437062in}{2.055000in}}%
\pgfusepath{clip}%
\pgfsetrectcap%
\pgfsetroundjoin%
\pgfsetlinewidth{0.803000pt}%
\definecolor{currentstroke}{rgb}{0.850000,0.850000,0.850000}%
\pgfsetstrokecolor{currentstroke}%
\pgfsetdash{}{0pt}%
\pgfpathmoveto{\pgfqpoint{1.571343in}{0.417642in}}%
\pgfpathlineto{\pgfqpoint{1.571343in}{2.472642in}}%
\pgfusepath{stroke}%
\end{pgfscope}%
\begin{pgfscope}%
\pgfsetbuttcap%
\pgfsetroundjoin%
\definecolor{currentfill}{rgb}{0.000000,0.000000,0.000000}%
\pgfsetfillcolor{currentfill}%
\pgfsetlinewidth{0.602250pt}%
\definecolor{currentstroke}{rgb}{0.000000,0.000000,0.000000}%
\pgfsetstrokecolor{currentstroke}%
\pgfsetdash{}{0pt}%
\pgfsys@defobject{currentmarker}{\pgfqpoint{0.000000in}{-0.027778in}}{\pgfqpoint{0.000000in}{0.000000in}}{%
\pgfpathmoveto{\pgfqpoint{0.000000in}{0.000000in}}%
\pgfpathlineto{\pgfqpoint{0.000000in}{-0.027778in}}%
\pgfusepath{stroke,fill}%
}%
\begin{pgfscope}%
\pgfsys@transformshift{1.571343in}{0.417642in}%
\pgfsys@useobject{currentmarker}{}%
\end{pgfscope}%
\end{pgfscope}%
\begin{pgfscope}%
\pgfpathrectangle{\pgfqpoint{0.589510in}{0.417642in}}{\pgfqpoint{3.437062in}{2.055000in}}%
\pgfusepath{clip}%
\pgfsetrectcap%
\pgfsetroundjoin%
\pgfsetlinewidth{0.803000pt}%
\definecolor{currentstroke}{rgb}{0.850000,0.850000,0.850000}%
\pgfsetstrokecolor{currentstroke}%
\pgfsetdash{}{0pt}%
\pgfpathmoveto{\pgfqpoint{1.783081in}{0.417642in}}%
\pgfpathlineto{\pgfqpoint{1.783081in}{2.472642in}}%
\pgfusepath{stroke}%
\end{pgfscope}%
\begin{pgfscope}%
\pgfsetbuttcap%
\pgfsetroundjoin%
\definecolor{currentfill}{rgb}{0.000000,0.000000,0.000000}%
\pgfsetfillcolor{currentfill}%
\pgfsetlinewidth{0.602250pt}%
\definecolor{currentstroke}{rgb}{0.000000,0.000000,0.000000}%
\pgfsetstrokecolor{currentstroke}%
\pgfsetdash{}{0pt}%
\pgfsys@defobject{currentmarker}{\pgfqpoint{0.000000in}{-0.027778in}}{\pgfqpoint{0.000000in}{0.000000in}}{%
\pgfpathmoveto{\pgfqpoint{0.000000in}{0.000000in}}%
\pgfpathlineto{\pgfqpoint{0.000000in}{-0.027778in}}%
\pgfusepath{stroke,fill}%
}%
\begin{pgfscope}%
\pgfsys@transformshift{1.783081in}{0.417642in}%
\pgfsys@useobject{currentmarker}{}%
\end{pgfscope}%
\end{pgfscope}%
\begin{pgfscope}%
\pgfpathrectangle{\pgfqpoint{0.589510in}{0.417642in}}{\pgfqpoint{3.437062in}{2.055000in}}%
\pgfusepath{clip}%
\pgfsetrectcap%
\pgfsetroundjoin%
\pgfsetlinewidth{0.803000pt}%
\definecolor{currentstroke}{rgb}{0.850000,0.850000,0.850000}%
\pgfsetstrokecolor{currentstroke}%
\pgfsetdash{}{0pt}%
\pgfpathmoveto{\pgfqpoint{1.890597in}{0.417642in}}%
\pgfpathlineto{\pgfqpoint{1.890597in}{2.472642in}}%
\pgfusepath{stroke}%
\end{pgfscope}%
\begin{pgfscope}%
\pgfsetbuttcap%
\pgfsetroundjoin%
\definecolor{currentfill}{rgb}{0.000000,0.000000,0.000000}%
\pgfsetfillcolor{currentfill}%
\pgfsetlinewidth{0.602250pt}%
\definecolor{currentstroke}{rgb}{0.000000,0.000000,0.000000}%
\pgfsetstrokecolor{currentstroke}%
\pgfsetdash{}{0pt}%
\pgfsys@defobject{currentmarker}{\pgfqpoint{0.000000in}{-0.027778in}}{\pgfqpoint{0.000000in}{0.000000in}}{%
\pgfpathmoveto{\pgfqpoint{0.000000in}{0.000000in}}%
\pgfpathlineto{\pgfqpoint{0.000000in}{-0.027778in}}%
\pgfusepath{stroke,fill}%
}%
\begin{pgfscope}%
\pgfsys@transformshift{1.890597in}{0.417642in}%
\pgfsys@useobject{currentmarker}{}%
\end{pgfscope}%
\end{pgfscope}%
\begin{pgfscope}%
\pgfpathrectangle{\pgfqpoint{0.589510in}{0.417642in}}{\pgfqpoint{3.437062in}{2.055000in}}%
\pgfusepath{clip}%
\pgfsetrectcap%
\pgfsetroundjoin%
\pgfsetlinewidth{0.803000pt}%
\definecolor{currentstroke}{rgb}{0.850000,0.850000,0.850000}%
\pgfsetstrokecolor{currentstroke}%
\pgfsetdash{}{0pt}%
\pgfpathmoveto{\pgfqpoint{1.966881in}{0.417642in}}%
\pgfpathlineto{\pgfqpoint{1.966881in}{2.472642in}}%
\pgfusepath{stroke}%
\end{pgfscope}%
\begin{pgfscope}%
\pgfsetbuttcap%
\pgfsetroundjoin%
\definecolor{currentfill}{rgb}{0.000000,0.000000,0.000000}%
\pgfsetfillcolor{currentfill}%
\pgfsetlinewidth{0.602250pt}%
\definecolor{currentstroke}{rgb}{0.000000,0.000000,0.000000}%
\pgfsetstrokecolor{currentstroke}%
\pgfsetdash{}{0pt}%
\pgfsys@defobject{currentmarker}{\pgfqpoint{0.000000in}{-0.027778in}}{\pgfqpoint{0.000000in}{0.000000in}}{%
\pgfpathmoveto{\pgfqpoint{0.000000in}{0.000000in}}%
\pgfpathlineto{\pgfqpoint{0.000000in}{-0.027778in}}%
\pgfusepath{stroke,fill}%
}%
\begin{pgfscope}%
\pgfsys@transformshift{1.966881in}{0.417642in}%
\pgfsys@useobject{currentmarker}{}%
\end{pgfscope}%
\end{pgfscope}%
\begin{pgfscope}%
\pgfpathrectangle{\pgfqpoint{0.589510in}{0.417642in}}{\pgfqpoint{3.437062in}{2.055000in}}%
\pgfusepath{clip}%
\pgfsetrectcap%
\pgfsetroundjoin%
\pgfsetlinewidth{0.803000pt}%
\definecolor{currentstroke}{rgb}{0.850000,0.850000,0.850000}%
\pgfsetstrokecolor{currentstroke}%
\pgfsetdash{}{0pt}%
\pgfpathmoveto{\pgfqpoint{2.026052in}{0.417642in}}%
\pgfpathlineto{\pgfqpoint{2.026052in}{2.472642in}}%
\pgfusepath{stroke}%
\end{pgfscope}%
\begin{pgfscope}%
\pgfsetbuttcap%
\pgfsetroundjoin%
\definecolor{currentfill}{rgb}{0.000000,0.000000,0.000000}%
\pgfsetfillcolor{currentfill}%
\pgfsetlinewidth{0.602250pt}%
\definecolor{currentstroke}{rgb}{0.000000,0.000000,0.000000}%
\pgfsetstrokecolor{currentstroke}%
\pgfsetdash{}{0pt}%
\pgfsys@defobject{currentmarker}{\pgfqpoint{0.000000in}{-0.027778in}}{\pgfqpoint{0.000000in}{0.000000in}}{%
\pgfpathmoveto{\pgfqpoint{0.000000in}{0.000000in}}%
\pgfpathlineto{\pgfqpoint{0.000000in}{-0.027778in}}%
\pgfusepath{stroke,fill}%
}%
\begin{pgfscope}%
\pgfsys@transformshift{2.026052in}{0.417642in}%
\pgfsys@useobject{currentmarker}{}%
\end{pgfscope}%
\end{pgfscope}%
\begin{pgfscope}%
\pgfpathrectangle{\pgfqpoint{0.589510in}{0.417642in}}{\pgfqpoint{3.437062in}{2.055000in}}%
\pgfusepath{clip}%
\pgfsetrectcap%
\pgfsetroundjoin%
\pgfsetlinewidth{0.803000pt}%
\definecolor{currentstroke}{rgb}{0.850000,0.850000,0.850000}%
\pgfsetstrokecolor{currentstroke}%
\pgfsetdash{}{0pt}%
\pgfpathmoveto{\pgfqpoint{2.074397in}{0.417642in}}%
\pgfpathlineto{\pgfqpoint{2.074397in}{2.472642in}}%
\pgfusepath{stroke}%
\end{pgfscope}%
\begin{pgfscope}%
\pgfsetbuttcap%
\pgfsetroundjoin%
\definecolor{currentfill}{rgb}{0.000000,0.000000,0.000000}%
\pgfsetfillcolor{currentfill}%
\pgfsetlinewidth{0.602250pt}%
\definecolor{currentstroke}{rgb}{0.000000,0.000000,0.000000}%
\pgfsetstrokecolor{currentstroke}%
\pgfsetdash{}{0pt}%
\pgfsys@defobject{currentmarker}{\pgfqpoint{0.000000in}{-0.027778in}}{\pgfqpoint{0.000000in}{0.000000in}}{%
\pgfpathmoveto{\pgfqpoint{0.000000in}{0.000000in}}%
\pgfpathlineto{\pgfqpoint{0.000000in}{-0.027778in}}%
\pgfusepath{stroke,fill}%
}%
\begin{pgfscope}%
\pgfsys@transformshift{2.074397in}{0.417642in}%
\pgfsys@useobject{currentmarker}{}%
\end{pgfscope}%
\end{pgfscope}%
\begin{pgfscope}%
\pgfpathrectangle{\pgfqpoint{0.589510in}{0.417642in}}{\pgfqpoint{3.437062in}{2.055000in}}%
\pgfusepath{clip}%
\pgfsetrectcap%
\pgfsetroundjoin%
\pgfsetlinewidth{0.803000pt}%
\definecolor{currentstroke}{rgb}{0.850000,0.850000,0.850000}%
\pgfsetstrokecolor{currentstroke}%
\pgfsetdash{}{0pt}%
\pgfpathmoveto{\pgfqpoint{2.115273in}{0.417642in}}%
\pgfpathlineto{\pgfqpoint{2.115273in}{2.472642in}}%
\pgfusepath{stroke}%
\end{pgfscope}%
\begin{pgfscope}%
\pgfsetbuttcap%
\pgfsetroundjoin%
\definecolor{currentfill}{rgb}{0.000000,0.000000,0.000000}%
\pgfsetfillcolor{currentfill}%
\pgfsetlinewidth{0.602250pt}%
\definecolor{currentstroke}{rgb}{0.000000,0.000000,0.000000}%
\pgfsetstrokecolor{currentstroke}%
\pgfsetdash{}{0pt}%
\pgfsys@defobject{currentmarker}{\pgfqpoint{0.000000in}{-0.027778in}}{\pgfqpoint{0.000000in}{0.000000in}}{%
\pgfpathmoveto{\pgfqpoint{0.000000in}{0.000000in}}%
\pgfpathlineto{\pgfqpoint{0.000000in}{-0.027778in}}%
\pgfusepath{stroke,fill}%
}%
\begin{pgfscope}%
\pgfsys@transformshift{2.115273in}{0.417642in}%
\pgfsys@useobject{currentmarker}{}%
\end{pgfscope}%
\end{pgfscope}%
\begin{pgfscope}%
\pgfpathrectangle{\pgfqpoint{0.589510in}{0.417642in}}{\pgfqpoint{3.437062in}{2.055000in}}%
\pgfusepath{clip}%
\pgfsetrectcap%
\pgfsetroundjoin%
\pgfsetlinewidth{0.803000pt}%
\definecolor{currentstroke}{rgb}{0.850000,0.850000,0.850000}%
\pgfsetstrokecolor{currentstroke}%
\pgfsetdash{}{0pt}%
\pgfpathmoveto{\pgfqpoint{2.150681in}{0.417642in}}%
\pgfpathlineto{\pgfqpoint{2.150681in}{2.472642in}}%
\pgfusepath{stroke}%
\end{pgfscope}%
\begin{pgfscope}%
\pgfsetbuttcap%
\pgfsetroundjoin%
\definecolor{currentfill}{rgb}{0.000000,0.000000,0.000000}%
\pgfsetfillcolor{currentfill}%
\pgfsetlinewidth{0.602250pt}%
\definecolor{currentstroke}{rgb}{0.000000,0.000000,0.000000}%
\pgfsetstrokecolor{currentstroke}%
\pgfsetdash{}{0pt}%
\pgfsys@defobject{currentmarker}{\pgfqpoint{0.000000in}{-0.027778in}}{\pgfqpoint{0.000000in}{0.000000in}}{%
\pgfpathmoveto{\pgfqpoint{0.000000in}{0.000000in}}%
\pgfpathlineto{\pgfqpoint{0.000000in}{-0.027778in}}%
\pgfusepath{stroke,fill}%
}%
\begin{pgfscope}%
\pgfsys@transformshift{2.150681in}{0.417642in}%
\pgfsys@useobject{currentmarker}{}%
\end{pgfscope}%
\end{pgfscope}%
\begin{pgfscope}%
\pgfpathrectangle{\pgfqpoint{0.589510in}{0.417642in}}{\pgfqpoint{3.437062in}{2.055000in}}%
\pgfusepath{clip}%
\pgfsetrectcap%
\pgfsetroundjoin%
\pgfsetlinewidth{0.803000pt}%
\definecolor{currentstroke}{rgb}{0.850000,0.850000,0.850000}%
\pgfsetstrokecolor{currentstroke}%
\pgfsetdash{}{0pt}%
\pgfpathmoveto{\pgfqpoint{2.181914in}{0.417642in}}%
\pgfpathlineto{\pgfqpoint{2.181914in}{2.472642in}}%
\pgfusepath{stroke}%
\end{pgfscope}%
\begin{pgfscope}%
\pgfsetbuttcap%
\pgfsetroundjoin%
\definecolor{currentfill}{rgb}{0.000000,0.000000,0.000000}%
\pgfsetfillcolor{currentfill}%
\pgfsetlinewidth{0.602250pt}%
\definecolor{currentstroke}{rgb}{0.000000,0.000000,0.000000}%
\pgfsetstrokecolor{currentstroke}%
\pgfsetdash{}{0pt}%
\pgfsys@defobject{currentmarker}{\pgfqpoint{0.000000in}{-0.027778in}}{\pgfqpoint{0.000000in}{0.000000in}}{%
\pgfpathmoveto{\pgfqpoint{0.000000in}{0.000000in}}%
\pgfpathlineto{\pgfqpoint{0.000000in}{-0.027778in}}%
\pgfusepath{stroke,fill}%
}%
\begin{pgfscope}%
\pgfsys@transformshift{2.181914in}{0.417642in}%
\pgfsys@useobject{currentmarker}{}%
\end{pgfscope}%
\end{pgfscope}%
\begin{pgfscope}%
\pgfpathrectangle{\pgfqpoint{0.589510in}{0.417642in}}{\pgfqpoint{3.437062in}{2.055000in}}%
\pgfusepath{clip}%
\pgfsetrectcap%
\pgfsetroundjoin%
\pgfsetlinewidth{0.803000pt}%
\definecolor{currentstroke}{rgb}{0.850000,0.850000,0.850000}%
\pgfsetstrokecolor{currentstroke}%
\pgfsetdash{}{0pt}%
\pgfpathmoveto{\pgfqpoint{2.393652in}{0.417642in}}%
\pgfpathlineto{\pgfqpoint{2.393652in}{2.472642in}}%
\pgfusepath{stroke}%
\end{pgfscope}%
\begin{pgfscope}%
\pgfsetbuttcap%
\pgfsetroundjoin%
\definecolor{currentfill}{rgb}{0.000000,0.000000,0.000000}%
\pgfsetfillcolor{currentfill}%
\pgfsetlinewidth{0.602250pt}%
\definecolor{currentstroke}{rgb}{0.000000,0.000000,0.000000}%
\pgfsetstrokecolor{currentstroke}%
\pgfsetdash{}{0pt}%
\pgfsys@defobject{currentmarker}{\pgfqpoint{0.000000in}{-0.027778in}}{\pgfqpoint{0.000000in}{0.000000in}}{%
\pgfpathmoveto{\pgfqpoint{0.000000in}{0.000000in}}%
\pgfpathlineto{\pgfqpoint{0.000000in}{-0.027778in}}%
\pgfusepath{stroke,fill}%
}%
\begin{pgfscope}%
\pgfsys@transformshift{2.393652in}{0.417642in}%
\pgfsys@useobject{currentmarker}{}%
\end{pgfscope}%
\end{pgfscope}%
\begin{pgfscope}%
\pgfpathrectangle{\pgfqpoint{0.589510in}{0.417642in}}{\pgfqpoint{3.437062in}{2.055000in}}%
\pgfusepath{clip}%
\pgfsetrectcap%
\pgfsetroundjoin%
\pgfsetlinewidth{0.803000pt}%
\definecolor{currentstroke}{rgb}{0.850000,0.850000,0.850000}%
\pgfsetstrokecolor{currentstroke}%
\pgfsetdash{}{0pt}%
\pgfpathmoveto{\pgfqpoint{2.501168in}{0.417642in}}%
\pgfpathlineto{\pgfqpoint{2.501168in}{2.472642in}}%
\pgfusepath{stroke}%
\end{pgfscope}%
\begin{pgfscope}%
\pgfsetbuttcap%
\pgfsetroundjoin%
\definecolor{currentfill}{rgb}{0.000000,0.000000,0.000000}%
\pgfsetfillcolor{currentfill}%
\pgfsetlinewidth{0.602250pt}%
\definecolor{currentstroke}{rgb}{0.000000,0.000000,0.000000}%
\pgfsetstrokecolor{currentstroke}%
\pgfsetdash{}{0pt}%
\pgfsys@defobject{currentmarker}{\pgfqpoint{0.000000in}{-0.027778in}}{\pgfqpoint{0.000000in}{0.000000in}}{%
\pgfpathmoveto{\pgfqpoint{0.000000in}{0.000000in}}%
\pgfpathlineto{\pgfqpoint{0.000000in}{-0.027778in}}%
\pgfusepath{stroke,fill}%
}%
\begin{pgfscope}%
\pgfsys@transformshift{2.501168in}{0.417642in}%
\pgfsys@useobject{currentmarker}{}%
\end{pgfscope}%
\end{pgfscope}%
\begin{pgfscope}%
\pgfpathrectangle{\pgfqpoint{0.589510in}{0.417642in}}{\pgfqpoint{3.437062in}{2.055000in}}%
\pgfusepath{clip}%
\pgfsetrectcap%
\pgfsetroundjoin%
\pgfsetlinewidth{0.803000pt}%
\definecolor{currentstroke}{rgb}{0.850000,0.850000,0.850000}%
\pgfsetstrokecolor{currentstroke}%
\pgfsetdash{}{0pt}%
\pgfpathmoveto{\pgfqpoint{2.577452in}{0.417642in}}%
\pgfpathlineto{\pgfqpoint{2.577452in}{2.472642in}}%
\pgfusepath{stroke}%
\end{pgfscope}%
\begin{pgfscope}%
\pgfsetbuttcap%
\pgfsetroundjoin%
\definecolor{currentfill}{rgb}{0.000000,0.000000,0.000000}%
\pgfsetfillcolor{currentfill}%
\pgfsetlinewidth{0.602250pt}%
\definecolor{currentstroke}{rgb}{0.000000,0.000000,0.000000}%
\pgfsetstrokecolor{currentstroke}%
\pgfsetdash{}{0pt}%
\pgfsys@defobject{currentmarker}{\pgfqpoint{0.000000in}{-0.027778in}}{\pgfqpoint{0.000000in}{0.000000in}}{%
\pgfpathmoveto{\pgfqpoint{0.000000in}{0.000000in}}%
\pgfpathlineto{\pgfqpoint{0.000000in}{-0.027778in}}%
\pgfusepath{stroke,fill}%
}%
\begin{pgfscope}%
\pgfsys@transformshift{2.577452in}{0.417642in}%
\pgfsys@useobject{currentmarker}{}%
\end{pgfscope}%
\end{pgfscope}%
\begin{pgfscope}%
\pgfpathrectangle{\pgfqpoint{0.589510in}{0.417642in}}{\pgfqpoint{3.437062in}{2.055000in}}%
\pgfusepath{clip}%
\pgfsetrectcap%
\pgfsetroundjoin%
\pgfsetlinewidth{0.803000pt}%
\definecolor{currentstroke}{rgb}{0.850000,0.850000,0.850000}%
\pgfsetstrokecolor{currentstroke}%
\pgfsetdash{}{0pt}%
\pgfpathmoveto{\pgfqpoint{2.636622in}{0.417642in}}%
\pgfpathlineto{\pgfqpoint{2.636622in}{2.472642in}}%
\pgfusepath{stroke}%
\end{pgfscope}%
\begin{pgfscope}%
\pgfsetbuttcap%
\pgfsetroundjoin%
\definecolor{currentfill}{rgb}{0.000000,0.000000,0.000000}%
\pgfsetfillcolor{currentfill}%
\pgfsetlinewidth{0.602250pt}%
\definecolor{currentstroke}{rgb}{0.000000,0.000000,0.000000}%
\pgfsetstrokecolor{currentstroke}%
\pgfsetdash{}{0pt}%
\pgfsys@defobject{currentmarker}{\pgfqpoint{0.000000in}{-0.027778in}}{\pgfqpoint{0.000000in}{0.000000in}}{%
\pgfpathmoveto{\pgfqpoint{0.000000in}{0.000000in}}%
\pgfpathlineto{\pgfqpoint{0.000000in}{-0.027778in}}%
\pgfusepath{stroke,fill}%
}%
\begin{pgfscope}%
\pgfsys@transformshift{2.636622in}{0.417642in}%
\pgfsys@useobject{currentmarker}{}%
\end{pgfscope}%
\end{pgfscope}%
\begin{pgfscope}%
\pgfpathrectangle{\pgfqpoint{0.589510in}{0.417642in}}{\pgfqpoint{3.437062in}{2.055000in}}%
\pgfusepath{clip}%
\pgfsetrectcap%
\pgfsetroundjoin%
\pgfsetlinewidth{0.803000pt}%
\definecolor{currentstroke}{rgb}{0.850000,0.850000,0.850000}%
\pgfsetstrokecolor{currentstroke}%
\pgfsetdash{}{0pt}%
\pgfpathmoveto{\pgfqpoint{2.684968in}{0.417642in}}%
\pgfpathlineto{\pgfqpoint{2.684968in}{2.472642in}}%
\pgfusepath{stroke}%
\end{pgfscope}%
\begin{pgfscope}%
\pgfsetbuttcap%
\pgfsetroundjoin%
\definecolor{currentfill}{rgb}{0.000000,0.000000,0.000000}%
\pgfsetfillcolor{currentfill}%
\pgfsetlinewidth{0.602250pt}%
\definecolor{currentstroke}{rgb}{0.000000,0.000000,0.000000}%
\pgfsetstrokecolor{currentstroke}%
\pgfsetdash{}{0pt}%
\pgfsys@defobject{currentmarker}{\pgfqpoint{0.000000in}{-0.027778in}}{\pgfqpoint{0.000000in}{0.000000in}}{%
\pgfpathmoveto{\pgfqpoint{0.000000in}{0.000000in}}%
\pgfpathlineto{\pgfqpoint{0.000000in}{-0.027778in}}%
\pgfusepath{stroke,fill}%
}%
\begin{pgfscope}%
\pgfsys@transformshift{2.684968in}{0.417642in}%
\pgfsys@useobject{currentmarker}{}%
\end{pgfscope}%
\end{pgfscope}%
\begin{pgfscope}%
\pgfpathrectangle{\pgfqpoint{0.589510in}{0.417642in}}{\pgfqpoint{3.437062in}{2.055000in}}%
\pgfusepath{clip}%
\pgfsetrectcap%
\pgfsetroundjoin%
\pgfsetlinewidth{0.803000pt}%
\definecolor{currentstroke}{rgb}{0.850000,0.850000,0.850000}%
\pgfsetstrokecolor{currentstroke}%
\pgfsetdash{}{0pt}%
\pgfpathmoveto{\pgfqpoint{2.725844in}{0.417642in}}%
\pgfpathlineto{\pgfqpoint{2.725844in}{2.472642in}}%
\pgfusepath{stroke}%
\end{pgfscope}%
\begin{pgfscope}%
\pgfsetbuttcap%
\pgfsetroundjoin%
\definecolor{currentfill}{rgb}{0.000000,0.000000,0.000000}%
\pgfsetfillcolor{currentfill}%
\pgfsetlinewidth{0.602250pt}%
\definecolor{currentstroke}{rgb}{0.000000,0.000000,0.000000}%
\pgfsetstrokecolor{currentstroke}%
\pgfsetdash{}{0pt}%
\pgfsys@defobject{currentmarker}{\pgfqpoint{0.000000in}{-0.027778in}}{\pgfqpoint{0.000000in}{0.000000in}}{%
\pgfpathmoveto{\pgfqpoint{0.000000in}{0.000000in}}%
\pgfpathlineto{\pgfqpoint{0.000000in}{-0.027778in}}%
\pgfusepath{stroke,fill}%
}%
\begin{pgfscope}%
\pgfsys@transformshift{2.725844in}{0.417642in}%
\pgfsys@useobject{currentmarker}{}%
\end{pgfscope}%
\end{pgfscope}%
\begin{pgfscope}%
\pgfpathrectangle{\pgfqpoint{0.589510in}{0.417642in}}{\pgfqpoint{3.437062in}{2.055000in}}%
\pgfusepath{clip}%
\pgfsetrectcap%
\pgfsetroundjoin%
\pgfsetlinewidth{0.803000pt}%
\definecolor{currentstroke}{rgb}{0.850000,0.850000,0.850000}%
\pgfsetstrokecolor{currentstroke}%
\pgfsetdash{}{0pt}%
\pgfpathmoveto{\pgfqpoint{2.761252in}{0.417642in}}%
\pgfpathlineto{\pgfqpoint{2.761252in}{2.472642in}}%
\pgfusepath{stroke}%
\end{pgfscope}%
\begin{pgfscope}%
\pgfsetbuttcap%
\pgfsetroundjoin%
\definecolor{currentfill}{rgb}{0.000000,0.000000,0.000000}%
\pgfsetfillcolor{currentfill}%
\pgfsetlinewidth{0.602250pt}%
\definecolor{currentstroke}{rgb}{0.000000,0.000000,0.000000}%
\pgfsetstrokecolor{currentstroke}%
\pgfsetdash{}{0pt}%
\pgfsys@defobject{currentmarker}{\pgfqpoint{0.000000in}{-0.027778in}}{\pgfqpoint{0.000000in}{0.000000in}}{%
\pgfpathmoveto{\pgfqpoint{0.000000in}{0.000000in}}%
\pgfpathlineto{\pgfqpoint{0.000000in}{-0.027778in}}%
\pgfusepath{stroke,fill}%
}%
\begin{pgfscope}%
\pgfsys@transformshift{2.761252in}{0.417642in}%
\pgfsys@useobject{currentmarker}{}%
\end{pgfscope}%
\end{pgfscope}%
\begin{pgfscope}%
\pgfpathrectangle{\pgfqpoint{0.589510in}{0.417642in}}{\pgfqpoint{3.437062in}{2.055000in}}%
\pgfusepath{clip}%
\pgfsetrectcap%
\pgfsetroundjoin%
\pgfsetlinewidth{0.803000pt}%
\definecolor{currentstroke}{rgb}{0.850000,0.850000,0.850000}%
\pgfsetstrokecolor{currentstroke}%
\pgfsetdash{}{0pt}%
\pgfpathmoveto{\pgfqpoint{2.792484in}{0.417642in}}%
\pgfpathlineto{\pgfqpoint{2.792484in}{2.472642in}}%
\pgfusepath{stroke}%
\end{pgfscope}%
\begin{pgfscope}%
\pgfsetbuttcap%
\pgfsetroundjoin%
\definecolor{currentfill}{rgb}{0.000000,0.000000,0.000000}%
\pgfsetfillcolor{currentfill}%
\pgfsetlinewidth{0.602250pt}%
\definecolor{currentstroke}{rgb}{0.000000,0.000000,0.000000}%
\pgfsetstrokecolor{currentstroke}%
\pgfsetdash{}{0pt}%
\pgfsys@defobject{currentmarker}{\pgfqpoint{0.000000in}{-0.027778in}}{\pgfqpoint{0.000000in}{0.000000in}}{%
\pgfpathmoveto{\pgfqpoint{0.000000in}{0.000000in}}%
\pgfpathlineto{\pgfqpoint{0.000000in}{-0.027778in}}%
\pgfusepath{stroke,fill}%
}%
\begin{pgfscope}%
\pgfsys@transformshift{2.792484in}{0.417642in}%
\pgfsys@useobject{currentmarker}{}%
\end{pgfscope}%
\end{pgfscope}%
\begin{pgfscope}%
\pgfpathrectangle{\pgfqpoint{0.589510in}{0.417642in}}{\pgfqpoint{3.437062in}{2.055000in}}%
\pgfusepath{clip}%
\pgfsetrectcap%
\pgfsetroundjoin%
\pgfsetlinewidth{0.803000pt}%
\definecolor{currentstroke}{rgb}{0.850000,0.850000,0.850000}%
\pgfsetstrokecolor{currentstroke}%
\pgfsetdash{}{0pt}%
\pgfpathmoveto{\pgfqpoint{3.004223in}{0.417642in}}%
\pgfpathlineto{\pgfqpoint{3.004223in}{2.472642in}}%
\pgfusepath{stroke}%
\end{pgfscope}%
\begin{pgfscope}%
\pgfsetbuttcap%
\pgfsetroundjoin%
\definecolor{currentfill}{rgb}{0.000000,0.000000,0.000000}%
\pgfsetfillcolor{currentfill}%
\pgfsetlinewidth{0.602250pt}%
\definecolor{currentstroke}{rgb}{0.000000,0.000000,0.000000}%
\pgfsetstrokecolor{currentstroke}%
\pgfsetdash{}{0pt}%
\pgfsys@defobject{currentmarker}{\pgfqpoint{0.000000in}{-0.027778in}}{\pgfqpoint{0.000000in}{0.000000in}}{%
\pgfpathmoveto{\pgfqpoint{0.000000in}{0.000000in}}%
\pgfpathlineto{\pgfqpoint{0.000000in}{-0.027778in}}%
\pgfusepath{stroke,fill}%
}%
\begin{pgfscope}%
\pgfsys@transformshift{3.004223in}{0.417642in}%
\pgfsys@useobject{currentmarker}{}%
\end{pgfscope}%
\end{pgfscope}%
\begin{pgfscope}%
\pgfpathrectangle{\pgfqpoint{0.589510in}{0.417642in}}{\pgfqpoint{3.437062in}{2.055000in}}%
\pgfusepath{clip}%
\pgfsetrectcap%
\pgfsetroundjoin%
\pgfsetlinewidth{0.803000pt}%
\definecolor{currentstroke}{rgb}{0.850000,0.850000,0.850000}%
\pgfsetstrokecolor{currentstroke}%
\pgfsetdash{}{0pt}%
\pgfpathmoveto{\pgfqpoint{3.111739in}{0.417642in}}%
\pgfpathlineto{\pgfqpoint{3.111739in}{2.472642in}}%
\pgfusepath{stroke}%
\end{pgfscope}%
\begin{pgfscope}%
\pgfsetbuttcap%
\pgfsetroundjoin%
\definecolor{currentfill}{rgb}{0.000000,0.000000,0.000000}%
\pgfsetfillcolor{currentfill}%
\pgfsetlinewidth{0.602250pt}%
\definecolor{currentstroke}{rgb}{0.000000,0.000000,0.000000}%
\pgfsetstrokecolor{currentstroke}%
\pgfsetdash{}{0pt}%
\pgfsys@defobject{currentmarker}{\pgfqpoint{0.000000in}{-0.027778in}}{\pgfqpoint{0.000000in}{0.000000in}}{%
\pgfpathmoveto{\pgfqpoint{0.000000in}{0.000000in}}%
\pgfpathlineto{\pgfqpoint{0.000000in}{-0.027778in}}%
\pgfusepath{stroke,fill}%
}%
\begin{pgfscope}%
\pgfsys@transformshift{3.111739in}{0.417642in}%
\pgfsys@useobject{currentmarker}{}%
\end{pgfscope}%
\end{pgfscope}%
\begin{pgfscope}%
\pgfpathrectangle{\pgfqpoint{0.589510in}{0.417642in}}{\pgfqpoint{3.437062in}{2.055000in}}%
\pgfusepath{clip}%
\pgfsetrectcap%
\pgfsetroundjoin%
\pgfsetlinewidth{0.803000pt}%
\definecolor{currentstroke}{rgb}{0.850000,0.850000,0.850000}%
\pgfsetstrokecolor{currentstroke}%
\pgfsetdash{}{0pt}%
\pgfpathmoveto{\pgfqpoint{3.188023in}{0.417642in}}%
\pgfpathlineto{\pgfqpoint{3.188023in}{2.472642in}}%
\pgfusepath{stroke}%
\end{pgfscope}%
\begin{pgfscope}%
\pgfsetbuttcap%
\pgfsetroundjoin%
\definecolor{currentfill}{rgb}{0.000000,0.000000,0.000000}%
\pgfsetfillcolor{currentfill}%
\pgfsetlinewidth{0.602250pt}%
\definecolor{currentstroke}{rgb}{0.000000,0.000000,0.000000}%
\pgfsetstrokecolor{currentstroke}%
\pgfsetdash{}{0pt}%
\pgfsys@defobject{currentmarker}{\pgfqpoint{0.000000in}{-0.027778in}}{\pgfqpoint{0.000000in}{0.000000in}}{%
\pgfpathmoveto{\pgfqpoint{0.000000in}{0.000000in}}%
\pgfpathlineto{\pgfqpoint{0.000000in}{-0.027778in}}%
\pgfusepath{stroke,fill}%
}%
\begin{pgfscope}%
\pgfsys@transformshift{3.188023in}{0.417642in}%
\pgfsys@useobject{currentmarker}{}%
\end{pgfscope}%
\end{pgfscope}%
\begin{pgfscope}%
\pgfpathrectangle{\pgfqpoint{0.589510in}{0.417642in}}{\pgfqpoint{3.437062in}{2.055000in}}%
\pgfusepath{clip}%
\pgfsetrectcap%
\pgfsetroundjoin%
\pgfsetlinewidth{0.803000pt}%
\definecolor{currentstroke}{rgb}{0.850000,0.850000,0.850000}%
\pgfsetstrokecolor{currentstroke}%
\pgfsetdash{}{0pt}%
\pgfpathmoveto{\pgfqpoint{3.247193in}{0.417642in}}%
\pgfpathlineto{\pgfqpoint{3.247193in}{2.472642in}}%
\pgfusepath{stroke}%
\end{pgfscope}%
\begin{pgfscope}%
\pgfsetbuttcap%
\pgfsetroundjoin%
\definecolor{currentfill}{rgb}{0.000000,0.000000,0.000000}%
\pgfsetfillcolor{currentfill}%
\pgfsetlinewidth{0.602250pt}%
\definecolor{currentstroke}{rgb}{0.000000,0.000000,0.000000}%
\pgfsetstrokecolor{currentstroke}%
\pgfsetdash{}{0pt}%
\pgfsys@defobject{currentmarker}{\pgfqpoint{0.000000in}{-0.027778in}}{\pgfqpoint{0.000000in}{0.000000in}}{%
\pgfpathmoveto{\pgfqpoint{0.000000in}{0.000000in}}%
\pgfpathlineto{\pgfqpoint{0.000000in}{-0.027778in}}%
\pgfusepath{stroke,fill}%
}%
\begin{pgfscope}%
\pgfsys@transformshift{3.247193in}{0.417642in}%
\pgfsys@useobject{currentmarker}{}%
\end{pgfscope}%
\end{pgfscope}%
\begin{pgfscope}%
\pgfpathrectangle{\pgfqpoint{0.589510in}{0.417642in}}{\pgfqpoint{3.437062in}{2.055000in}}%
\pgfusepath{clip}%
\pgfsetrectcap%
\pgfsetroundjoin%
\pgfsetlinewidth{0.803000pt}%
\definecolor{currentstroke}{rgb}{0.850000,0.850000,0.850000}%
\pgfsetstrokecolor{currentstroke}%
\pgfsetdash{}{0pt}%
\pgfpathmoveto{\pgfqpoint{3.295539in}{0.417642in}}%
\pgfpathlineto{\pgfqpoint{3.295539in}{2.472642in}}%
\pgfusepath{stroke}%
\end{pgfscope}%
\begin{pgfscope}%
\pgfsetbuttcap%
\pgfsetroundjoin%
\definecolor{currentfill}{rgb}{0.000000,0.000000,0.000000}%
\pgfsetfillcolor{currentfill}%
\pgfsetlinewidth{0.602250pt}%
\definecolor{currentstroke}{rgb}{0.000000,0.000000,0.000000}%
\pgfsetstrokecolor{currentstroke}%
\pgfsetdash{}{0pt}%
\pgfsys@defobject{currentmarker}{\pgfqpoint{0.000000in}{-0.027778in}}{\pgfqpoint{0.000000in}{0.000000in}}{%
\pgfpathmoveto{\pgfqpoint{0.000000in}{0.000000in}}%
\pgfpathlineto{\pgfqpoint{0.000000in}{-0.027778in}}%
\pgfusepath{stroke,fill}%
}%
\begin{pgfscope}%
\pgfsys@transformshift{3.295539in}{0.417642in}%
\pgfsys@useobject{currentmarker}{}%
\end{pgfscope}%
\end{pgfscope}%
\begin{pgfscope}%
\pgfpathrectangle{\pgfqpoint{0.589510in}{0.417642in}}{\pgfqpoint{3.437062in}{2.055000in}}%
\pgfusepath{clip}%
\pgfsetrectcap%
\pgfsetroundjoin%
\pgfsetlinewidth{0.803000pt}%
\definecolor{currentstroke}{rgb}{0.850000,0.850000,0.850000}%
\pgfsetstrokecolor{currentstroke}%
\pgfsetdash{}{0pt}%
\pgfpathmoveto{\pgfqpoint{3.336415in}{0.417642in}}%
\pgfpathlineto{\pgfqpoint{3.336415in}{2.472642in}}%
\pgfusepath{stroke}%
\end{pgfscope}%
\begin{pgfscope}%
\pgfsetbuttcap%
\pgfsetroundjoin%
\definecolor{currentfill}{rgb}{0.000000,0.000000,0.000000}%
\pgfsetfillcolor{currentfill}%
\pgfsetlinewidth{0.602250pt}%
\definecolor{currentstroke}{rgb}{0.000000,0.000000,0.000000}%
\pgfsetstrokecolor{currentstroke}%
\pgfsetdash{}{0pt}%
\pgfsys@defobject{currentmarker}{\pgfqpoint{0.000000in}{-0.027778in}}{\pgfqpoint{0.000000in}{0.000000in}}{%
\pgfpathmoveto{\pgfqpoint{0.000000in}{0.000000in}}%
\pgfpathlineto{\pgfqpoint{0.000000in}{-0.027778in}}%
\pgfusepath{stroke,fill}%
}%
\begin{pgfscope}%
\pgfsys@transformshift{3.336415in}{0.417642in}%
\pgfsys@useobject{currentmarker}{}%
\end{pgfscope}%
\end{pgfscope}%
\begin{pgfscope}%
\pgfpathrectangle{\pgfqpoint{0.589510in}{0.417642in}}{\pgfqpoint{3.437062in}{2.055000in}}%
\pgfusepath{clip}%
\pgfsetrectcap%
\pgfsetroundjoin%
\pgfsetlinewidth{0.803000pt}%
\definecolor{currentstroke}{rgb}{0.850000,0.850000,0.850000}%
\pgfsetstrokecolor{currentstroke}%
\pgfsetdash{}{0pt}%
\pgfpathmoveto{\pgfqpoint{3.371823in}{0.417642in}}%
\pgfpathlineto{\pgfqpoint{3.371823in}{2.472642in}}%
\pgfusepath{stroke}%
\end{pgfscope}%
\begin{pgfscope}%
\pgfsetbuttcap%
\pgfsetroundjoin%
\definecolor{currentfill}{rgb}{0.000000,0.000000,0.000000}%
\pgfsetfillcolor{currentfill}%
\pgfsetlinewidth{0.602250pt}%
\definecolor{currentstroke}{rgb}{0.000000,0.000000,0.000000}%
\pgfsetstrokecolor{currentstroke}%
\pgfsetdash{}{0pt}%
\pgfsys@defobject{currentmarker}{\pgfqpoint{0.000000in}{-0.027778in}}{\pgfqpoint{0.000000in}{0.000000in}}{%
\pgfpathmoveto{\pgfqpoint{0.000000in}{0.000000in}}%
\pgfpathlineto{\pgfqpoint{0.000000in}{-0.027778in}}%
\pgfusepath{stroke,fill}%
}%
\begin{pgfscope}%
\pgfsys@transformshift{3.371823in}{0.417642in}%
\pgfsys@useobject{currentmarker}{}%
\end{pgfscope}%
\end{pgfscope}%
\begin{pgfscope}%
\pgfpathrectangle{\pgfqpoint{0.589510in}{0.417642in}}{\pgfqpoint{3.437062in}{2.055000in}}%
\pgfusepath{clip}%
\pgfsetrectcap%
\pgfsetroundjoin%
\pgfsetlinewidth{0.803000pt}%
\definecolor{currentstroke}{rgb}{0.850000,0.850000,0.850000}%
\pgfsetstrokecolor{currentstroke}%
\pgfsetdash{}{0pt}%
\pgfpathmoveto{\pgfqpoint{3.403055in}{0.417642in}}%
\pgfpathlineto{\pgfqpoint{3.403055in}{2.472642in}}%
\pgfusepath{stroke}%
\end{pgfscope}%
\begin{pgfscope}%
\pgfsetbuttcap%
\pgfsetroundjoin%
\definecolor{currentfill}{rgb}{0.000000,0.000000,0.000000}%
\pgfsetfillcolor{currentfill}%
\pgfsetlinewidth{0.602250pt}%
\definecolor{currentstroke}{rgb}{0.000000,0.000000,0.000000}%
\pgfsetstrokecolor{currentstroke}%
\pgfsetdash{}{0pt}%
\pgfsys@defobject{currentmarker}{\pgfqpoint{0.000000in}{-0.027778in}}{\pgfqpoint{0.000000in}{0.000000in}}{%
\pgfpathmoveto{\pgfqpoint{0.000000in}{0.000000in}}%
\pgfpathlineto{\pgfqpoint{0.000000in}{-0.027778in}}%
\pgfusepath{stroke,fill}%
}%
\begin{pgfscope}%
\pgfsys@transformshift{3.403055in}{0.417642in}%
\pgfsys@useobject{currentmarker}{}%
\end{pgfscope}%
\end{pgfscope}%
\begin{pgfscope}%
\pgfpathrectangle{\pgfqpoint{0.589510in}{0.417642in}}{\pgfqpoint{3.437062in}{2.055000in}}%
\pgfusepath{clip}%
\pgfsetrectcap%
\pgfsetroundjoin%
\pgfsetlinewidth{0.803000pt}%
\definecolor{currentstroke}{rgb}{0.850000,0.850000,0.850000}%
\pgfsetstrokecolor{currentstroke}%
\pgfsetdash{}{0pt}%
\pgfpathmoveto{\pgfqpoint{3.614793in}{0.417642in}}%
\pgfpathlineto{\pgfqpoint{3.614793in}{2.472642in}}%
\pgfusepath{stroke}%
\end{pgfscope}%
\begin{pgfscope}%
\pgfsetbuttcap%
\pgfsetroundjoin%
\definecolor{currentfill}{rgb}{0.000000,0.000000,0.000000}%
\pgfsetfillcolor{currentfill}%
\pgfsetlinewidth{0.602250pt}%
\definecolor{currentstroke}{rgb}{0.000000,0.000000,0.000000}%
\pgfsetstrokecolor{currentstroke}%
\pgfsetdash{}{0pt}%
\pgfsys@defobject{currentmarker}{\pgfqpoint{0.000000in}{-0.027778in}}{\pgfqpoint{0.000000in}{0.000000in}}{%
\pgfpathmoveto{\pgfqpoint{0.000000in}{0.000000in}}%
\pgfpathlineto{\pgfqpoint{0.000000in}{-0.027778in}}%
\pgfusepath{stroke,fill}%
}%
\begin{pgfscope}%
\pgfsys@transformshift{3.614793in}{0.417642in}%
\pgfsys@useobject{currentmarker}{}%
\end{pgfscope}%
\end{pgfscope}%
\begin{pgfscope}%
\pgfpathrectangle{\pgfqpoint{0.589510in}{0.417642in}}{\pgfqpoint{3.437062in}{2.055000in}}%
\pgfusepath{clip}%
\pgfsetrectcap%
\pgfsetroundjoin%
\pgfsetlinewidth{0.803000pt}%
\definecolor{currentstroke}{rgb}{0.850000,0.850000,0.850000}%
\pgfsetstrokecolor{currentstroke}%
\pgfsetdash{}{0pt}%
\pgfpathmoveto{\pgfqpoint{3.722309in}{0.417642in}}%
\pgfpathlineto{\pgfqpoint{3.722309in}{2.472642in}}%
\pgfusepath{stroke}%
\end{pgfscope}%
\begin{pgfscope}%
\pgfsetbuttcap%
\pgfsetroundjoin%
\definecolor{currentfill}{rgb}{0.000000,0.000000,0.000000}%
\pgfsetfillcolor{currentfill}%
\pgfsetlinewidth{0.602250pt}%
\definecolor{currentstroke}{rgb}{0.000000,0.000000,0.000000}%
\pgfsetstrokecolor{currentstroke}%
\pgfsetdash{}{0pt}%
\pgfsys@defobject{currentmarker}{\pgfqpoint{0.000000in}{-0.027778in}}{\pgfqpoint{0.000000in}{0.000000in}}{%
\pgfpathmoveto{\pgfqpoint{0.000000in}{0.000000in}}%
\pgfpathlineto{\pgfqpoint{0.000000in}{-0.027778in}}%
\pgfusepath{stroke,fill}%
}%
\begin{pgfscope}%
\pgfsys@transformshift{3.722309in}{0.417642in}%
\pgfsys@useobject{currentmarker}{}%
\end{pgfscope}%
\end{pgfscope}%
\begin{pgfscope}%
\pgfpathrectangle{\pgfqpoint{0.589510in}{0.417642in}}{\pgfqpoint{3.437062in}{2.055000in}}%
\pgfusepath{clip}%
\pgfsetrectcap%
\pgfsetroundjoin%
\pgfsetlinewidth{0.803000pt}%
\definecolor{currentstroke}{rgb}{0.850000,0.850000,0.850000}%
\pgfsetstrokecolor{currentstroke}%
\pgfsetdash{}{0pt}%
\pgfpathmoveto{\pgfqpoint{3.798593in}{0.417642in}}%
\pgfpathlineto{\pgfqpoint{3.798593in}{2.472642in}}%
\pgfusepath{stroke}%
\end{pgfscope}%
\begin{pgfscope}%
\pgfsetbuttcap%
\pgfsetroundjoin%
\definecolor{currentfill}{rgb}{0.000000,0.000000,0.000000}%
\pgfsetfillcolor{currentfill}%
\pgfsetlinewidth{0.602250pt}%
\definecolor{currentstroke}{rgb}{0.000000,0.000000,0.000000}%
\pgfsetstrokecolor{currentstroke}%
\pgfsetdash{}{0pt}%
\pgfsys@defobject{currentmarker}{\pgfqpoint{0.000000in}{-0.027778in}}{\pgfqpoint{0.000000in}{0.000000in}}{%
\pgfpathmoveto{\pgfqpoint{0.000000in}{0.000000in}}%
\pgfpathlineto{\pgfqpoint{0.000000in}{-0.027778in}}%
\pgfusepath{stroke,fill}%
}%
\begin{pgfscope}%
\pgfsys@transformshift{3.798593in}{0.417642in}%
\pgfsys@useobject{currentmarker}{}%
\end{pgfscope}%
\end{pgfscope}%
\begin{pgfscope}%
\pgfpathrectangle{\pgfqpoint{0.589510in}{0.417642in}}{\pgfqpoint{3.437062in}{2.055000in}}%
\pgfusepath{clip}%
\pgfsetrectcap%
\pgfsetroundjoin%
\pgfsetlinewidth{0.803000pt}%
\definecolor{currentstroke}{rgb}{0.850000,0.850000,0.850000}%
\pgfsetstrokecolor{currentstroke}%
\pgfsetdash{}{0pt}%
\pgfpathmoveto{\pgfqpoint{3.857764in}{0.417642in}}%
\pgfpathlineto{\pgfqpoint{3.857764in}{2.472642in}}%
\pgfusepath{stroke}%
\end{pgfscope}%
\begin{pgfscope}%
\pgfsetbuttcap%
\pgfsetroundjoin%
\definecolor{currentfill}{rgb}{0.000000,0.000000,0.000000}%
\pgfsetfillcolor{currentfill}%
\pgfsetlinewidth{0.602250pt}%
\definecolor{currentstroke}{rgb}{0.000000,0.000000,0.000000}%
\pgfsetstrokecolor{currentstroke}%
\pgfsetdash{}{0pt}%
\pgfsys@defobject{currentmarker}{\pgfqpoint{0.000000in}{-0.027778in}}{\pgfqpoint{0.000000in}{0.000000in}}{%
\pgfpathmoveto{\pgfqpoint{0.000000in}{0.000000in}}%
\pgfpathlineto{\pgfqpoint{0.000000in}{-0.027778in}}%
\pgfusepath{stroke,fill}%
}%
\begin{pgfscope}%
\pgfsys@transformshift{3.857764in}{0.417642in}%
\pgfsys@useobject{currentmarker}{}%
\end{pgfscope}%
\end{pgfscope}%
\begin{pgfscope}%
\pgfpathrectangle{\pgfqpoint{0.589510in}{0.417642in}}{\pgfqpoint{3.437062in}{2.055000in}}%
\pgfusepath{clip}%
\pgfsetrectcap%
\pgfsetroundjoin%
\pgfsetlinewidth{0.803000pt}%
\definecolor{currentstroke}{rgb}{0.850000,0.850000,0.850000}%
\pgfsetstrokecolor{currentstroke}%
\pgfsetdash{}{0pt}%
\pgfpathmoveto{\pgfqpoint{3.906110in}{0.417642in}}%
\pgfpathlineto{\pgfqpoint{3.906110in}{2.472642in}}%
\pgfusepath{stroke}%
\end{pgfscope}%
\begin{pgfscope}%
\pgfsetbuttcap%
\pgfsetroundjoin%
\definecolor{currentfill}{rgb}{0.000000,0.000000,0.000000}%
\pgfsetfillcolor{currentfill}%
\pgfsetlinewidth{0.602250pt}%
\definecolor{currentstroke}{rgb}{0.000000,0.000000,0.000000}%
\pgfsetstrokecolor{currentstroke}%
\pgfsetdash{}{0pt}%
\pgfsys@defobject{currentmarker}{\pgfqpoint{0.000000in}{-0.027778in}}{\pgfqpoint{0.000000in}{0.000000in}}{%
\pgfpathmoveto{\pgfqpoint{0.000000in}{0.000000in}}%
\pgfpathlineto{\pgfqpoint{0.000000in}{-0.027778in}}%
\pgfusepath{stroke,fill}%
}%
\begin{pgfscope}%
\pgfsys@transformshift{3.906110in}{0.417642in}%
\pgfsys@useobject{currentmarker}{}%
\end{pgfscope}%
\end{pgfscope}%
\begin{pgfscope}%
\pgfpathrectangle{\pgfqpoint{0.589510in}{0.417642in}}{\pgfqpoint{3.437062in}{2.055000in}}%
\pgfusepath{clip}%
\pgfsetrectcap%
\pgfsetroundjoin%
\pgfsetlinewidth{0.803000pt}%
\definecolor{currentstroke}{rgb}{0.850000,0.850000,0.850000}%
\pgfsetstrokecolor{currentstroke}%
\pgfsetdash{}{0pt}%
\pgfpathmoveto{\pgfqpoint{3.946985in}{0.417642in}}%
\pgfpathlineto{\pgfqpoint{3.946985in}{2.472642in}}%
\pgfusepath{stroke}%
\end{pgfscope}%
\begin{pgfscope}%
\pgfsetbuttcap%
\pgfsetroundjoin%
\definecolor{currentfill}{rgb}{0.000000,0.000000,0.000000}%
\pgfsetfillcolor{currentfill}%
\pgfsetlinewidth{0.602250pt}%
\definecolor{currentstroke}{rgb}{0.000000,0.000000,0.000000}%
\pgfsetstrokecolor{currentstroke}%
\pgfsetdash{}{0pt}%
\pgfsys@defobject{currentmarker}{\pgfqpoint{0.000000in}{-0.027778in}}{\pgfqpoint{0.000000in}{0.000000in}}{%
\pgfpathmoveto{\pgfqpoint{0.000000in}{0.000000in}}%
\pgfpathlineto{\pgfqpoint{0.000000in}{-0.027778in}}%
\pgfusepath{stroke,fill}%
}%
\begin{pgfscope}%
\pgfsys@transformshift{3.946985in}{0.417642in}%
\pgfsys@useobject{currentmarker}{}%
\end{pgfscope}%
\end{pgfscope}%
\begin{pgfscope}%
\pgfpathrectangle{\pgfqpoint{0.589510in}{0.417642in}}{\pgfqpoint{3.437062in}{2.055000in}}%
\pgfusepath{clip}%
\pgfsetrectcap%
\pgfsetroundjoin%
\pgfsetlinewidth{0.803000pt}%
\definecolor{currentstroke}{rgb}{0.850000,0.850000,0.850000}%
\pgfsetstrokecolor{currentstroke}%
\pgfsetdash{}{0pt}%
\pgfpathmoveto{\pgfqpoint{3.982393in}{0.417642in}}%
\pgfpathlineto{\pgfqpoint{3.982393in}{2.472642in}}%
\pgfusepath{stroke}%
\end{pgfscope}%
\begin{pgfscope}%
\pgfsetbuttcap%
\pgfsetroundjoin%
\definecolor{currentfill}{rgb}{0.000000,0.000000,0.000000}%
\pgfsetfillcolor{currentfill}%
\pgfsetlinewidth{0.602250pt}%
\definecolor{currentstroke}{rgb}{0.000000,0.000000,0.000000}%
\pgfsetstrokecolor{currentstroke}%
\pgfsetdash{}{0pt}%
\pgfsys@defobject{currentmarker}{\pgfqpoint{0.000000in}{-0.027778in}}{\pgfqpoint{0.000000in}{0.000000in}}{%
\pgfpathmoveto{\pgfqpoint{0.000000in}{0.000000in}}%
\pgfpathlineto{\pgfqpoint{0.000000in}{-0.027778in}}%
\pgfusepath{stroke,fill}%
}%
\begin{pgfscope}%
\pgfsys@transformshift{3.982393in}{0.417642in}%
\pgfsys@useobject{currentmarker}{}%
\end{pgfscope}%
\end{pgfscope}%
\begin{pgfscope}%
\pgfpathrectangle{\pgfqpoint{0.589510in}{0.417642in}}{\pgfqpoint{3.437062in}{2.055000in}}%
\pgfusepath{clip}%
\pgfsetrectcap%
\pgfsetroundjoin%
\pgfsetlinewidth{0.803000pt}%
\definecolor{currentstroke}{rgb}{0.850000,0.850000,0.850000}%
\pgfsetstrokecolor{currentstroke}%
\pgfsetdash{}{0pt}%
\pgfpathmoveto{\pgfqpoint{4.013626in}{0.417642in}}%
\pgfpathlineto{\pgfqpoint{4.013626in}{2.472642in}}%
\pgfusepath{stroke}%
\end{pgfscope}%
\begin{pgfscope}%
\pgfsetbuttcap%
\pgfsetroundjoin%
\definecolor{currentfill}{rgb}{0.000000,0.000000,0.000000}%
\pgfsetfillcolor{currentfill}%
\pgfsetlinewidth{0.602250pt}%
\definecolor{currentstroke}{rgb}{0.000000,0.000000,0.000000}%
\pgfsetstrokecolor{currentstroke}%
\pgfsetdash{}{0pt}%
\pgfsys@defobject{currentmarker}{\pgfqpoint{0.000000in}{-0.027778in}}{\pgfqpoint{0.000000in}{0.000000in}}{%
\pgfpathmoveto{\pgfqpoint{0.000000in}{0.000000in}}%
\pgfpathlineto{\pgfqpoint{0.000000in}{-0.027778in}}%
\pgfusepath{stroke,fill}%
}%
\begin{pgfscope}%
\pgfsys@transformshift{4.013626in}{0.417642in}%
\pgfsys@useobject{currentmarker}{}%
\end{pgfscope}%
\end{pgfscope}%
\begin{pgfscope}%
\definecolor{textcolor}{rgb}{0.000000,0.000000,0.000000}%
\pgfsetstrokecolor{textcolor}%
\pgfsetfillcolor{textcolor}%
\pgftext[x=2.308041in,y=0.165003in,,top]{\color{textcolor}\rmfamily\fontsize{10.000000}{12.000000}\selectfont \(\displaystyle \tau\) in \unit{\second}}%
\end{pgfscope}%
\begin{pgfscope}%
\pgfpathrectangle{\pgfqpoint{0.589510in}{0.417642in}}{\pgfqpoint{3.437062in}{2.055000in}}%
\pgfusepath{clip}%
\pgfsetrectcap%
\pgfsetroundjoin%
\pgfsetlinewidth{0.803000pt}%
\definecolor{currentstroke}{rgb}{0.450000,0.450000,0.450000}%
\pgfsetstrokecolor{currentstroke}%
\pgfsetdash{}{0pt}%
\pgfpathmoveto{\pgfqpoint{0.589510in}{0.505447in}}%
\pgfpathlineto{\pgfqpoint{4.026572in}{0.505447in}}%
\pgfusepath{stroke}%
\end{pgfscope}%
\begin{pgfscope}%
\pgfsetbuttcap%
\pgfsetroundjoin%
\definecolor{currentfill}{rgb}{0.000000,0.000000,0.000000}%
\pgfsetfillcolor{currentfill}%
\pgfsetlinewidth{0.803000pt}%
\definecolor{currentstroke}{rgb}{0.000000,0.000000,0.000000}%
\pgfsetstrokecolor{currentstroke}%
\pgfsetdash{}{0pt}%
\pgfsys@defobject{currentmarker}{\pgfqpoint{-0.048611in}{0.000000in}}{\pgfqpoint{-0.000000in}{0.000000in}}{%
\pgfpathmoveto{\pgfqpoint{-0.000000in}{0.000000in}}%
\pgfpathlineto{\pgfqpoint{-0.048611in}{0.000000in}}%
\pgfusepath{stroke,fill}%
}%
\begin{pgfscope}%
\pgfsys@transformshift{0.589510in}{0.505447in}%
\pgfsys@useobject{currentmarker}{}%
\end{pgfscope}%
\end{pgfscope}%
\begin{pgfscope}%
\definecolor{textcolor}{rgb}{0.000000,0.000000,0.000000}%
\pgfsetstrokecolor{textcolor}%
\pgfsetfillcolor{textcolor}%
\pgftext[x=0.236114in, y=0.466294in, left, base]{\color{textcolor}\rmfamily\fontsize{8.000000}{9.600000}\selectfont \(\displaystyle {10^{-9}}\)}%
\end{pgfscope}%
\begin{pgfscope}%
\pgfpathrectangle{\pgfqpoint{0.589510in}{0.417642in}}{\pgfqpoint{3.437062in}{2.055000in}}%
\pgfusepath{clip}%
\pgfsetrectcap%
\pgfsetroundjoin%
\pgfsetlinewidth{0.803000pt}%
\definecolor{currentstroke}{rgb}{0.450000,0.450000,0.450000}%
\pgfsetstrokecolor{currentstroke}%
\pgfsetdash{}{0pt}%
\pgfpathmoveto{\pgfqpoint{0.589510in}{1.187752in}}%
\pgfpathlineto{\pgfqpoint{4.026572in}{1.187752in}}%
\pgfusepath{stroke}%
\end{pgfscope}%
\begin{pgfscope}%
\pgfsetbuttcap%
\pgfsetroundjoin%
\definecolor{currentfill}{rgb}{0.000000,0.000000,0.000000}%
\pgfsetfillcolor{currentfill}%
\pgfsetlinewidth{0.803000pt}%
\definecolor{currentstroke}{rgb}{0.000000,0.000000,0.000000}%
\pgfsetstrokecolor{currentstroke}%
\pgfsetdash{}{0pt}%
\pgfsys@defobject{currentmarker}{\pgfqpoint{-0.048611in}{0.000000in}}{\pgfqpoint{-0.000000in}{0.000000in}}{%
\pgfpathmoveto{\pgfqpoint{-0.000000in}{0.000000in}}%
\pgfpathlineto{\pgfqpoint{-0.048611in}{0.000000in}}%
\pgfusepath{stroke,fill}%
}%
\begin{pgfscope}%
\pgfsys@transformshift{0.589510in}{1.187752in}%
\pgfsys@useobject{currentmarker}{}%
\end{pgfscope}%
\end{pgfscope}%
\begin{pgfscope}%
\definecolor{textcolor}{rgb}{0.000000,0.000000,0.000000}%
\pgfsetstrokecolor{textcolor}%
\pgfsetfillcolor{textcolor}%
\pgftext[x=0.236114in, y=1.148599in, left, base]{\color{textcolor}\rmfamily\fontsize{8.000000}{9.600000}\selectfont \(\displaystyle {10^{-8}}\)}%
\end{pgfscope}%
\begin{pgfscope}%
\pgfpathrectangle{\pgfqpoint{0.589510in}{0.417642in}}{\pgfqpoint{3.437062in}{2.055000in}}%
\pgfusepath{clip}%
\pgfsetrectcap%
\pgfsetroundjoin%
\pgfsetlinewidth{0.803000pt}%
\definecolor{currentstroke}{rgb}{0.450000,0.450000,0.450000}%
\pgfsetstrokecolor{currentstroke}%
\pgfsetdash{}{0pt}%
\pgfpathmoveto{\pgfqpoint{0.589510in}{1.870057in}}%
\pgfpathlineto{\pgfqpoint{4.026572in}{1.870057in}}%
\pgfusepath{stroke}%
\end{pgfscope}%
\begin{pgfscope}%
\pgfsetbuttcap%
\pgfsetroundjoin%
\definecolor{currentfill}{rgb}{0.000000,0.000000,0.000000}%
\pgfsetfillcolor{currentfill}%
\pgfsetlinewidth{0.803000pt}%
\definecolor{currentstroke}{rgb}{0.000000,0.000000,0.000000}%
\pgfsetstrokecolor{currentstroke}%
\pgfsetdash{}{0pt}%
\pgfsys@defobject{currentmarker}{\pgfqpoint{-0.048611in}{0.000000in}}{\pgfqpoint{-0.000000in}{0.000000in}}{%
\pgfpathmoveto{\pgfqpoint{-0.000000in}{0.000000in}}%
\pgfpathlineto{\pgfqpoint{-0.048611in}{0.000000in}}%
\pgfusepath{stroke,fill}%
}%
\begin{pgfscope}%
\pgfsys@transformshift{0.589510in}{1.870057in}%
\pgfsys@useobject{currentmarker}{}%
\end{pgfscope}%
\end{pgfscope}%
\begin{pgfscope}%
\definecolor{textcolor}{rgb}{0.000000,0.000000,0.000000}%
\pgfsetstrokecolor{textcolor}%
\pgfsetfillcolor{textcolor}%
\pgftext[x=0.236114in, y=1.830904in, left, base]{\color{textcolor}\rmfamily\fontsize{8.000000}{9.600000}\selectfont \(\displaystyle {10^{-7}}\)}%
\end{pgfscope}%
\begin{pgfscope}%
\pgfpathrectangle{\pgfqpoint{0.589510in}{0.417642in}}{\pgfqpoint{3.437062in}{2.055000in}}%
\pgfusepath{clip}%
\pgfsetrectcap%
\pgfsetroundjoin%
\pgfsetlinewidth{0.803000pt}%
\definecolor{currentstroke}{rgb}{0.850000,0.850000,0.850000}%
\pgfsetstrokecolor{currentstroke}%
\pgfsetdash{}{0pt}%
\pgfpathmoveto{\pgfqpoint{0.589510in}{0.439324in}}%
\pgfpathlineto{\pgfqpoint{4.026572in}{0.439324in}}%
\pgfusepath{stroke}%
\end{pgfscope}%
\begin{pgfscope}%
\pgfsetbuttcap%
\pgfsetroundjoin%
\definecolor{currentfill}{rgb}{0.000000,0.000000,0.000000}%
\pgfsetfillcolor{currentfill}%
\pgfsetlinewidth{0.602250pt}%
\definecolor{currentstroke}{rgb}{0.000000,0.000000,0.000000}%
\pgfsetstrokecolor{currentstroke}%
\pgfsetdash{}{0pt}%
\pgfsys@defobject{currentmarker}{\pgfqpoint{-0.027778in}{0.000000in}}{\pgfqpoint{-0.000000in}{0.000000in}}{%
\pgfpathmoveto{\pgfqpoint{-0.000000in}{0.000000in}}%
\pgfpathlineto{\pgfqpoint{-0.027778in}{0.000000in}}%
\pgfusepath{stroke,fill}%
}%
\begin{pgfscope}%
\pgfsys@transformshift{0.589510in}{0.439324in}%
\pgfsys@useobject{currentmarker}{}%
\end{pgfscope}%
\end{pgfscope}%
\begin{pgfscope}%
\pgfpathrectangle{\pgfqpoint{0.589510in}{0.417642in}}{\pgfqpoint{3.437062in}{2.055000in}}%
\pgfusepath{clip}%
\pgfsetrectcap%
\pgfsetroundjoin%
\pgfsetlinewidth{0.803000pt}%
\definecolor{currentstroke}{rgb}{0.850000,0.850000,0.850000}%
\pgfsetstrokecolor{currentstroke}%
\pgfsetdash{}{0pt}%
\pgfpathmoveto{\pgfqpoint{0.589510in}{0.474226in}}%
\pgfpathlineto{\pgfqpoint{4.026572in}{0.474226in}}%
\pgfusepath{stroke}%
\end{pgfscope}%
\begin{pgfscope}%
\pgfsetbuttcap%
\pgfsetroundjoin%
\definecolor{currentfill}{rgb}{0.000000,0.000000,0.000000}%
\pgfsetfillcolor{currentfill}%
\pgfsetlinewidth{0.602250pt}%
\definecolor{currentstroke}{rgb}{0.000000,0.000000,0.000000}%
\pgfsetstrokecolor{currentstroke}%
\pgfsetdash{}{0pt}%
\pgfsys@defobject{currentmarker}{\pgfqpoint{-0.027778in}{0.000000in}}{\pgfqpoint{-0.000000in}{0.000000in}}{%
\pgfpathmoveto{\pgfqpoint{-0.000000in}{0.000000in}}%
\pgfpathlineto{\pgfqpoint{-0.027778in}{0.000000in}}%
\pgfusepath{stroke,fill}%
}%
\begin{pgfscope}%
\pgfsys@transformshift{0.589510in}{0.474226in}%
\pgfsys@useobject{currentmarker}{}%
\end{pgfscope}%
\end{pgfscope}%
\begin{pgfscope}%
\pgfpathrectangle{\pgfqpoint{0.589510in}{0.417642in}}{\pgfqpoint{3.437062in}{2.055000in}}%
\pgfusepath{clip}%
\pgfsetrectcap%
\pgfsetroundjoin%
\pgfsetlinewidth{0.803000pt}%
\definecolor{currentstroke}{rgb}{0.850000,0.850000,0.850000}%
\pgfsetstrokecolor{currentstroke}%
\pgfsetdash{}{0pt}%
\pgfpathmoveto{\pgfqpoint{0.589510in}{0.710841in}}%
\pgfpathlineto{\pgfqpoint{4.026572in}{0.710841in}}%
\pgfusepath{stroke}%
\end{pgfscope}%
\begin{pgfscope}%
\pgfsetbuttcap%
\pgfsetroundjoin%
\definecolor{currentfill}{rgb}{0.000000,0.000000,0.000000}%
\pgfsetfillcolor{currentfill}%
\pgfsetlinewidth{0.602250pt}%
\definecolor{currentstroke}{rgb}{0.000000,0.000000,0.000000}%
\pgfsetstrokecolor{currentstroke}%
\pgfsetdash{}{0pt}%
\pgfsys@defobject{currentmarker}{\pgfqpoint{-0.027778in}{0.000000in}}{\pgfqpoint{-0.000000in}{0.000000in}}{%
\pgfpathmoveto{\pgfqpoint{-0.000000in}{0.000000in}}%
\pgfpathlineto{\pgfqpoint{-0.027778in}{0.000000in}}%
\pgfusepath{stroke,fill}%
}%
\begin{pgfscope}%
\pgfsys@transformshift{0.589510in}{0.710841in}%
\pgfsys@useobject{currentmarker}{}%
\end{pgfscope}%
\end{pgfscope}%
\begin{pgfscope}%
\pgfpathrectangle{\pgfqpoint{0.589510in}{0.417642in}}{\pgfqpoint{3.437062in}{2.055000in}}%
\pgfusepath{clip}%
\pgfsetrectcap%
\pgfsetroundjoin%
\pgfsetlinewidth{0.803000pt}%
\definecolor{currentstroke}{rgb}{0.850000,0.850000,0.850000}%
\pgfsetstrokecolor{currentstroke}%
\pgfsetdash{}{0pt}%
\pgfpathmoveto{\pgfqpoint{0.589510in}{0.830989in}}%
\pgfpathlineto{\pgfqpoint{4.026572in}{0.830989in}}%
\pgfusepath{stroke}%
\end{pgfscope}%
\begin{pgfscope}%
\pgfsetbuttcap%
\pgfsetroundjoin%
\definecolor{currentfill}{rgb}{0.000000,0.000000,0.000000}%
\pgfsetfillcolor{currentfill}%
\pgfsetlinewidth{0.602250pt}%
\definecolor{currentstroke}{rgb}{0.000000,0.000000,0.000000}%
\pgfsetstrokecolor{currentstroke}%
\pgfsetdash{}{0pt}%
\pgfsys@defobject{currentmarker}{\pgfqpoint{-0.027778in}{0.000000in}}{\pgfqpoint{-0.000000in}{0.000000in}}{%
\pgfpathmoveto{\pgfqpoint{-0.000000in}{0.000000in}}%
\pgfpathlineto{\pgfqpoint{-0.027778in}{0.000000in}}%
\pgfusepath{stroke,fill}%
}%
\begin{pgfscope}%
\pgfsys@transformshift{0.589510in}{0.830989in}%
\pgfsys@useobject{currentmarker}{}%
\end{pgfscope}%
\end{pgfscope}%
\begin{pgfscope}%
\pgfpathrectangle{\pgfqpoint{0.589510in}{0.417642in}}{\pgfqpoint{3.437062in}{2.055000in}}%
\pgfusepath{clip}%
\pgfsetrectcap%
\pgfsetroundjoin%
\pgfsetlinewidth{0.803000pt}%
\definecolor{currentstroke}{rgb}{0.850000,0.850000,0.850000}%
\pgfsetstrokecolor{currentstroke}%
\pgfsetdash{}{0pt}%
\pgfpathmoveto{\pgfqpoint{0.589510in}{0.916235in}}%
\pgfpathlineto{\pgfqpoint{4.026572in}{0.916235in}}%
\pgfusepath{stroke}%
\end{pgfscope}%
\begin{pgfscope}%
\pgfsetbuttcap%
\pgfsetroundjoin%
\definecolor{currentfill}{rgb}{0.000000,0.000000,0.000000}%
\pgfsetfillcolor{currentfill}%
\pgfsetlinewidth{0.602250pt}%
\definecolor{currentstroke}{rgb}{0.000000,0.000000,0.000000}%
\pgfsetstrokecolor{currentstroke}%
\pgfsetdash{}{0pt}%
\pgfsys@defobject{currentmarker}{\pgfqpoint{-0.027778in}{0.000000in}}{\pgfqpoint{-0.000000in}{0.000000in}}{%
\pgfpathmoveto{\pgfqpoint{-0.000000in}{0.000000in}}%
\pgfpathlineto{\pgfqpoint{-0.027778in}{0.000000in}}%
\pgfusepath{stroke,fill}%
}%
\begin{pgfscope}%
\pgfsys@transformshift{0.589510in}{0.916235in}%
\pgfsys@useobject{currentmarker}{}%
\end{pgfscope}%
\end{pgfscope}%
\begin{pgfscope}%
\pgfpathrectangle{\pgfqpoint{0.589510in}{0.417642in}}{\pgfqpoint{3.437062in}{2.055000in}}%
\pgfusepath{clip}%
\pgfsetrectcap%
\pgfsetroundjoin%
\pgfsetlinewidth{0.803000pt}%
\definecolor{currentstroke}{rgb}{0.850000,0.850000,0.850000}%
\pgfsetstrokecolor{currentstroke}%
\pgfsetdash{}{0pt}%
\pgfpathmoveto{\pgfqpoint{0.589510in}{0.982357in}}%
\pgfpathlineto{\pgfqpoint{4.026572in}{0.982357in}}%
\pgfusepath{stroke}%
\end{pgfscope}%
\begin{pgfscope}%
\pgfsetbuttcap%
\pgfsetroundjoin%
\definecolor{currentfill}{rgb}{0.000000,0.000000,0.000000}%
\pgfsetfillcolor{currentfill}%
\pgfsetlinewidth{0.602250pt}%
\definecolor{currentstroke}{rgb}{0.000000,0.000000,0.000000}%
\pgfsetstrokecolor{currentstroke}%
\pgfsetdash{}{0pt}%
\pgfsys@defobject{currentmarker}{\pgfqpoint{-0.027778in}{0.000000in}}{\pgfqpoint{-0.000000in}{0.000000in}}{%
\pgfpathmoveto{\pgfqpoint{-0.000000in}{0.000000in}}%
\pgfpathlineto{\pgfqpoint{-0.027778in}{0.000000in}}%
\pgfusepath{stroke,fill}%
}%
\begin{pgfscope}%
\pgfsys@transformshift{0.589510in}{0.982357in}%
\pgfsys@useobject{currentmarker}{}%
\end{pgfscope}%
\end{pgfscope}%
\begin{pgfscope}%
\pgfpathrectangle{\pgfqpoint{0.589510in}{0.417642in}}{\pgfqpoint{3.437062in}{2.055000in}}%
\pgfusepath{clip}%
\pgfsetrectcap%
\pgfsetroundjoin%
\pgfsetlinewidth{0.803000pt}%
\definecolor{currentstroke}{rgb}{0.850000,0.850000,0.850000}%
\pgfsetstrokecolor{currentstroke}%
\pgfsetdash{}{0pt}%
\pgfpathmoveto{\pgfqpoint{0.589510in}{1.036383in}}%
\pgfpathlineto{\pgfqpoint{4.026572in}{1.036383in}}%
\pgfusepath{stroke}%
\end{pgfscope}%
\begin{pgfscope}%
\pgfsetbuttcap%
\pgfsetroundjoin%
\definecolor{currentfill}{rgb}{0.000000,0.000000,0.000000}%
\pgfsetfillcolor{currentfill}%
\pgfsetlinewidth{0.602250pt}%
\definecolor{currentstroke}{rgb}{0.000000,0.000000,0.000000}%
\pgfsetstrokecolor{currentstroke}%
\pgfsetdash{}{0pt}%
\pgfsys@defobject{currentmarker}{\pgfqpoint{-0.027778in}{0.000000in}}{\pgfqpoint{-0.000000in}{0.000000in}}{%
\pgfpathmoveto{\pgfqpoint{-0.000000in}{0.000000in}}%
\pgfpathlineto{\pgfqpoint{-0.027778in}{0.000000in}}%
\pgfusepath{stroke,fill}%
}%
\begin{pgfscope}%
\pgfsys@transformshift{0.589510in}{1.036383in}%
\pgfsys@useobject{currentmarker}{}%
\end{pgfscope}%
\end{pgfscope}%
\begin{pgfscope}%
\pgfpathrectangle{\pgfqpoint{0.589510in}{0.417642in}}{\pgfqpoint{3.437062in}{2.055000in}}%
\pgfusepath{clip}%
\pgfsetrectcap%
\pgfsetroundjoin%
\pgfsetlinewidth{0.803000pt}%
\definecolor{currentstroke}{rgb}{0.850000,0.850000,0.850000}%
\pgfsetstrokecolor{currentstroke}%
\pgfsetdash{}{0pt}%
\pgfpathmoveto{\pgfqpoint{0.589510in}{1.082061in}}%
\pgfpathlineto{\pgfqpoint{4.026572in}{1.082061in}}%
\pgfusepath{stroke}%
\end{pgfscope}%
\begin{pgfscope}%
\pgfsetbuttcap%
\pgfsetroundjoin%
\definecolor{currentfill}{rgb}{0.000000,0.000000,0.000000}%
\pgfsetfillcolor{currentfill}%
\pgfsetlinewidth{0.602250pt}%
\definecolor{currentstroke}{rgb}{0.000000,0.000000,0.000000}%
\pgfsetstrokecolor{currentstroke}%
\pgfsetdash{}{0pt}%
\pgfsys@defobject{currentmarker}{\pgfqpoint{-0.027778in}{0.000000in}}{\pgfqpoint{-0.000000in}{0.000000in}}{%
\pgfpathmoveto{\pgfqpoint{-0.000000in}{0.000000in}}%
\pgfpathlineto{\pgfqpoint{-0.027778in}{0.000000in}}%
\pgfusepath{stroke,fill}%
}%
\begin{pgfscope}%
\pgfsys@transformshift{0.589510in}{1.082061in}%
\pgfsys@useobject{currentmarker}{}%
\end{pgfscope}%
\end{pgfscope}%
\begin{pgfscope}%
\pgfpathrectangle{\pgfqpoint{0.589510in}{0.417642in}}{\pgfqpoint{3.437062in}{2.055000in}}%
\pgfusepath{clip}%
\pgfsetrectcap%
\pgfsetroundjoin%
\pgfsetlinewidth{0.803000pt}%
\definecolor{currentstroke}{rgb}{0.850000,0.850000,0.850000}%
\pgfsetstrokecolor{currentstroke}%
\pgfsetdash{}{0pt}%
\pgfpathmoveto{\pgfqpoint{0.589510in}{1.121630in}}%
\pgfpathlineto{\pgfqpoint{4.026572in}{1.121630in}}%
\pgfusepath{stroke}%
\end{pgfscope}%
\begin{pgfscope}%
\pgfsetbuttcap%
\pgfsetroundjoin%
\definecolor{currentfill}{rgb}{0.000000,0.000000,0.000000}%
\pgfsetfillcolor{currentfill}%
\pgfsetlinewidth{0.602250pt}%
\definecolor{currentstroke}{rgb}{0.000000,0.000000,0.000000}%
\pgfsetstrokecolor{currentstroke}%
\pgfsetdash{}{0pt}%
\pgfsys@defobject{currentmarker}{\pgfqpoint{-0.027778in}{0.000000in}}{\pgfqpoint{-0.000000in}{0.000000in}}{%
\pgfpathmoveto{\pgfqpoint{-0.000000in}{0.000000in}}%
\pgfpathlineto{\pgfqpoint{-0.027778in}{0.000000in}}%
\pgfusepath{stroke,fill}%
}%
\begin{pgfscope}%
\pgfsys@transformshift{0.589510in}{1.121630in}%
\pgfsys@useobject{currentmarker}{}%
\end{pgfscope}%
\end{pgfscope}%
\begin{pgfscope}%
\pgfpathrectangle{\pgfqpoint{0.589510in}{0.417642in}}{\pgfqpoint{3.437062in}{2.055000in}}%
\pgfusepath{clip}%
\pgfsetrectcap%
\pgfsetroundjoin%
\pgfsetlinewidth{0.803000pt}%
\definecolor{currentstroke}{rgb}{0.850000,0.850000,0.850000}%
\pgfsetstrokecolor{currentstroke}%
\pgfsetdash{}{0pt}%
\pgfpathmoveto{\pgfqpoint{0.589510in}{1.156531in}}%
\pgfpathlineto{\pgfqpoint{4.026572in}{1.156531in}}%
\pgfusepath{stroke}%
\end{pgfscope}%
\begin{pgfscope}%
\pgfsetbuttcap%
\pgfsetroundjoin%
\definecolor{currentfill}{rgb}{0.000000,0.000000,0.000000}%
\pgfsetfillcolor{currentfill}%
\pgfsetlinewidth{0.602250pt}%
\definecolor{currentstroke}{rgb}{0.000000,0.000000,0.000000}%
\pgfsetstrokecolor{currentstroke}%
\pgfsetdash{}{0pt}%
\pgfsys@defobject{currentmarker}{\pgfqpoint{-0.027778in}{0.000000in}}{\pgfqpoint{-0.000000in}{0.000000in}}{%
\pgfpathmoveto{\pgfqpoint{-0.000000in}{0.000000in}}%
\pgfpathlineto{\pgfqpoint{-0.027778in}{0.000000in}}%
\pgfusepath{stroke,fill}%
}%
\begin{pgfscope}%
\pgfsys@transformshift{0.589510in}{1.156531in}%
\pgfsys@useobject{currentmarker}{}%
\end{pgfscope}%
\end{pgfscope}%
\begin{pgfscope}%
\pgfpathrectangle{\pgfqpoint{0.589510in}{0.417642in}}{\pgfqpoint{3.437062in}{2.055000in}}%
\pgfusepath{clip}%
\pgfsetrectcap%
\pgfsetroundjoin%
\pgfsetlinewidth{0.803000pt}%
\definecolor{currentstroke}{rgb}{0.850000,0.850000,0.850000}%
\pgfsetstrokecolor{currentstroke}%
\pgfsetdash{}{0pt}%
\pgfpathmoveto{\pgfqpoint{0.589510in}{1.393146in}}%
\pgfpathlineto{\pgfqpoint{4.026572in}{1.393146in}}%
\pgfusepath{stroke}%
\end{pgfscope}%
\begin{pgfscope}%
\pgfsetbuttcap%
\pgfsetroundjoin%
\definecolor{currentfill}{rgb}{0.000000,0.000000,0.000000}%
\pgfsetfillcolor{currentfill}%
\pgfsetlinewidth{0.602250pt}%
\definecolor{currentstroke}{rgb}{0.000000,0.000000,0.000000}%
\pgfsetstrokecolor{currentstroke}%
\pgfsetdash{}{0pt}%
\pgfsys@defobject{currentmarker}{\pgfqpoint{-0.027778in}{0.000000in}}{\pgfqpoint{-0.000000in}{0.000000in}}{%
\pgfpathmoveto{\pgfqpoint{-0.000000in}{0.000000in}}%
\pgfpathlineto{\pgfqpoint{-0.027778in}{0.000000in}}%
\pgfusepath{stroke,fill}%
}%
\begin{pgfscope}%
\pgfsys@transformshift{0.589510in}{1.393146in}%
\pgfsys@useobject{currentmarker}{}%
\end{pgfscope}%
\end{pgfscope}%
\begin{pgfscope}%
\pgfpathrectangle{\pgfqpoint{0.589510in}{0.417642in}}{\pgfqpoint{3.437062in}{2.055000in}}%
\pgfusepath{clip}%
\pgfsetrectcap%
\pgfsetroundjoin%
\pgfsetlinewidth{0.803000pt}%
\definecolor{currentstroke}{rgb}{0.850000,0.850000,0.850000}%
\pgfsetstrokecolor{currentstroke}%
\pgfsetdash{}{0pt}%
\pgfpathmoveto{\pgfqpoint{0.589510in}{1.513294in}}%
\pgfpathlineto{\pgfqpoint{4.026572in}{1.513294in}}%
\pgfusepath{stroke}%
\end{pgfscope}%
\begin{pgfscope}%
\pgfsetbuttcap%
\pgfsetroundjoin%
\definecolor{currentfill}{rgb}{0.000000,0.000000,0.000000}%
\pgfsetfillcolor{currentfill}%
\pgfsetlinewidth{0.602250pt}%
\definecolor{currentstroke}{rgb}{0.000000,0.000000,0.000000}%
\pgfsetstrokecolor{currentstroke}%
\pgfsetdash{}{0pt}%
\pgfsys@defobject{currentmarker}{\pgfqpoint{-0.027778in}{0.000000in}}{\pgfqpoint{-0.000000in}{0.000000in}}{%
\pgfpathmoveto{\pgfqpoint{-0.000000in}{0.000000in}}%
\pgfpathlineto{\pgfqpoint{-0.027778in}{0.000000in}}%
\pgfusepath{stroke,fill}%
}%
\begin{pgfscope}%
\pgfsys@transformshift{0.589510in}{1.513294in}%
\pgfsys@useobject{currentmarker}{}%
\end{pgfscope}%
\end{pgfscope}%
\begin{pgfscope}%
\pgfpathrectangle{\pgfqpoint{0.589510in}{0.417642in}}{\pgfqpoint{3.437062in}{2.055000in}}%
\pgfusepath{clip}%
\pgfsetrectcap%
\pgfsetroundjoin%
\pgfsetlinewidth{0.803000pt}%
\definecolor{currentstroke}{rgb}{0.850000,0.850000,0.850000}%
\pgfsetstrokecolor{currentstroke}%
\pgfsetdash{}{0pt}%
\pgfpathmoveto{\pgfqpoint{0.589510in}{1.598541in}}%
\pgfpathlineto{\pgfqpoint{4.026572in}{1.598541in}}%
\pgfusepath{stroke}%
\end{pgfscope}%
\begin{pgfscope}%
\pgfsetbuttcap%
\pgfsetroundjoin%
\definecolor{currentfill}{rgb}{0.000000,0.000000,0.000000}%
\pgfsetfillcolor{currentfill}%
\pgfsetlinewidth{0.602250pt}%
\definecolor{currentstroke}{rgb}{0.000000,0.000000,0.000000}%
\pgfsetstrokecolor{currentstroke}%
\pgfsetdash{}{0pt}%
\pgfsys@defobject{currentmarker}{\pgfqpoint{-0.027778in}{0.000000in}}{\pgfqpoint{-0.000000in}{0.000000in}}{%
\pgfpathmoveto{\pgfqpoint{-0.000000in}{0.000000in}}%
\pgfpathlineto{\pgfqpoint{-0.027778in}{0.000000in}}%
\pgfusepath{stroke,fill}%
}%
\begin{pgfscope}%
\pgfsys@transformshift{0.589510in}{1.598541in}%
\pgfsys@useobject{currentmarker}{}%
\end{pgfscope}%
\end{pgfscope}%
\begin{pgfscope}%
\pgfpathrectangle{\pgfqpoint{0.589510in}{0.417642in}}{\pgfqpoint{3.437062in}{2.055000in}}%
\pgfusepath{clip}%
\pgfsetrectcap%
\pgfsetroundjoin%
\pgfsetlinewidth{0.803000pt}%
\definecolor{currentstroke}{rgb}{0.850000,0.850000,0.850000}%
\pgfsetstrokecolor{currentstroke}%
\pgfsetdash{}{0pt}%
\pgfpathmoveto{\pgfqpoint{0.589510in}{1.664663in}}%
\pgfpathlineto{\pgfqpoint{4.026572in}{1.664663in}}%
\pgfusepath{stroke}%
\end{pgfscope}%
\begin{pgfscope}%
\pgfsetbuttcap%
\pgfsetroundjoin%
\definecolor{currentfill}{rgb}{0.000000,0.000000,0.000000}%
\pgfsetfillcolor{currentfill}%
\pgfsetlinewidth{0.602250pt}%
\definecolor{currentstroke}{rgb}{0.000000,0.000000,0.000000}%
\pgfsetstrokecolor{currentstroke}%
\pgfsetdash{}{0pt}%
\pgfsys@defobject{currentmarker}{\pgfqpoint{-0.027778in}{0.000000in}}{\pgfqpoint{-0.000000in}{0.000000in}}{%
\pgfpathmoveto{\pgfqpoint{-0.000000in}{0.000000in}}%
\pgfpathlineto{\pgfqpoint{-0.027778in}{0.000000in}}%
\pgfusepath{stroke,fill}%
}%
\begin{pgfscope}%
\pgfsys@transformshift{0.589510in}{1.664663in}%
\pgfsys@useobject{currentmarker}{}%
\end{pgfscope}%
\end{pgfscope}%
\begin{pgfscope}%
\pgfpathrectangle{\pgfqpoint{0.589510in}{0.417642in}}{\pgfqpoint{3.437062in}{2.055000in}}%
\pgfusepath{clip}%
\pgfsetrectcap%
\pgfsetroundjoin%
\pgfsetlinewidth{0.803000pt}%
\definecolor{currentstroke}{rgb}{0.850000,0.850000,0.850000}%
\pgfsetstrokecolor{currentstroke}%
\pgfsetdash{}{0pt}%
\pgfpathmoveto{\pgfqpoint{0.589510in}{1.718689in}}%
\pgfpathlineto{\pgfqpoint{4.026572in}{1.718689in}}%
\pgfusepath{stroke}%
\end{pgfscope}%
\begin{pgfscope}%
\pgfsetbuttcap%
\pgfsetroundjoin%
\definecolor{currentfill}{rgb}{0.000000,0.000000,0.000000}%
\pgfsetfillcolor{currentfill}%
\pgfsetlinewidth{0.602250pt}%
\definecolor{currentstroke}{rgb}{0.000000,0.000000,0.000000}%
\pgfsetstrokecolor{currentstroke}%
\pgfsetdash{}{0pt}%
\pgfsys@defobject{currentmarker}{\pgfqpoint{-0.027778in}{0.000000in}}{\pgfqpoint{-0.000000in}{0.000000in}}{%
\pgfpathmoveto{\pgfqpoint{-0.000000in}{0.000000in}}%
\pgfpathlineto{\pgfqpoint{-0.027778in}{0.000000in}}%
\pgfusepath{stroke,fill}%
}%
\begin{pgfscope}%
\pgfsys@transformshift{0.589510in}{1.718689in}%
\pgfsys@useobject{currentmarker}{}%
\end{pgfscope}%
\end{pgfscope}%
\begin{pgfscope}%
\pgfpathrectangle{\pgfqpoint{0.589510in}{0.417642in}}{\pgfqpoint{3.437062in}{2.055000in}}%
\pgfusepath{clip}%
\pgfsetrectcap%
\pgfsetroundjoin%
\pgfsetlinewidth{0.803000pt}%
\definecolor{currentstroke}{rgb}{0.850000,0.850000,0.850000}%
\pgfsetstrokecolor{currentstroke}%
\pgfsetdash{}{0pt}%
\pgfpathmoveto{\pgfqpoint{0.589510in}{1.764367in}}%
\pgfpathlineto{\pgfqpoint{4.026572in}{1.764367in}}%
\pgfusepath{stroke}%
\end{pgfscope}%
\begin{pgfscope}%
\pgfsetbuttcap%
\pgfsetroundjoin%
\definecolor{currentfill}{rgb}{0.000000,0.000000,0.000000}%
\pgfsetfillcolor{currentfill}%
\pgfsetlinewidth{0.602250pt}%
\definecolor{currentstroke}{rgb}{0.000000,0.000000,0.000000}%
\pgfsetstrokecolor{currentstroke}%
\pgfsetdash{}{0pt}%
\pgfsys@defobject{currentmarker}{\pgfqpoint{-0.027778in}{0.000000in}}{\pgfqpoint{-0.000000in}{0.000000in}}{%
\pgfpathmoveto{\pgfqpoint{-0.000000in}{0.000000in}}%
\pgfpathlineto{\pgfqpoint{-0.027778in}{0.000000in}}%
\pgfusepath{stroke,fill}%
}%
\begin{pgfscope}%
\pgfsys@transformshift{0.589510in}{1.764367in}%
\pgfsys@useobject{currentmarker}{}%
\end{pgfscope}%
\end{pgfscope}%
\begin{pgfscope}%
\pgfpathrectangle{\pgfqpoint{0.589510in}{0.417642in}}{\pgfqpoint{3.437062in}{2.055000in}}%
\pgfusepath{clip}%
\pgfsetrectcap%
\pgfsetroundjoin%
\pgfsetlinewidth{0.803000pt}%
\definecolor{currentstroke}{rgb}{0.850000,0.850000,0.850000}%
\pgfsetstrokecolor{currentstroke}%
\pgfsetdash{}{0pt}%
\pgfpathmoveto{\pgfqpoint{0.589510in}{1.803935in}}%
\pgfpathlineto{\pgfqpoint{4.026572in}{1.803935in}}%
\pgfusepath{stroke}%
\end{pgfscope}%
\begin{pgfscope}%
\pgfsetbuttcap%
\pgfsetroundjoin%
\definecolor{currentfill}{rgb}{0.000000,0.000000,0.000000}%
\pgfsetfillcolor{currentfill}%
\pgfsetlinewidth{0.602250pt}%
\definecolor{currentstroke}{rgb}{0.000000,0.000000,0.000000}%
\pgfsetstrokecolor{currentstroke}%
\pgfsetdash{}{0pt}%
\pgfsys@defobject{currentmarker}{\pgfqpoint{-0.027778in}{0.000000in}}{\pgfqpoint{-0.000000in}{0.000000in}}{%
\pgfpathmoveto{\pgfqpoint{-0.000000in}{0.000000in}}%
\pgfpathlineto{\pgfqpoint{-0.027778in}{0.000000in}}%
\pgfusepath{stroke,fill}%
}%
\begin{pgfscope}%
\pgfsys@transformshift{0.589510in}{1.803935in}%
\pgfsys@useobject{currentmarker}{}%
\end{pgfscope}%
\end{pgfscope}%
\begin{pgfscope}%
\pgfpathrectangle{\pgfqpoint{0.589510in}{0.417642in}}{\pgfqpoint{3.437062in}{2.055000in}}%
\pgfusepath{clip}%
\pgfsetrectcap%
\pgfsetroundjoin%
\pgfsetlinewidth{0.803000pt}%
\definecolor{currentstroke}{rgb}{0.850000,0.850000,0.850000}%
\pgfsetstrokecolor{currentstroke}%
\pgfsetdash{}{0pt}%
\pgfpathmoveto{\pgfqpoint{0.589510in}{1.838837in}}%
\pgfpathlineto{\pgfqpoint{4.026572in}{1.838837in}}%
\pgfusepath{stroke}%
\end{pgfscope}%
\begin{pgfscope}%
\pgfsetbuttcap%
\pgfsetroundjoin%
\definecolor{currentfill}{rgb}{0.000000,0.000000,0.000000}%
\pgfsetfillcolor{currentfill}%
\pgfsetlinewidth{0.602250pt}%
\definecolor{currentstroke}{rgb}{0.000000,0.000000,0.000000}%
\pgfsetstrokecolor{currentstroke}%
\pgfsetdash{}{0pt}%
\pgfsys@defobject{currentmarker}{\pgfqpoint{-0.027778in}{0.000000in}}{\pgfqpoint{-0.000000in}{0.000000in}}{%
\pgfpathmoveto{\pgfqpoint{-0.000000in}{0.000000in}}%
\pgfpathlineto{\pgfqpoint{-0.027778in}{0.000000in}}%
\pgfusepath{stroke,fill}%
}%
\begin{pgfscope}%
\pgfsys@transformshift{0.589510in}{1.838837in}%
\pgfsys@useobject{currentmarker}{}%
\end{pgfscope}%
\end{pgfscope}%
\begin{pgfscope}%
\pgfpathrectangle{\pgfqpoint{0.589510in}{0.417642in}}{\pgfqpoint{3.437062in}{2.055000in}}%
\pgfusepath{clip}%
\pgfsetrectcap%
\pgfsetroundjoin%
\pgfsetlinewidth{0.803000pt}%
\definecolor{currentstroke}{rgb}{0.850000,0.850000,0.850000}%
\pgfsetstrokecolor{currentstroke}%
\pgfsetdash{}{0pt}%
\pgfpathmoveto{\pgfqpoint{0.589510in}{2.075451in}}%
\pgfpathlineto{\pgfqpoint{4.026572in}{2.075451in}}%
\pgfusepath{stroke}%
\end{pgfscope}%
\begin{pgfscope}%
\pgfsetbuttcap%
\pgfsetroundjoin%
\definecolor{currentfill}{rgb}{0.000000,0.000000,0.000000}%
\pgfsetfillcolor{currentfill}%
\pgfsetlinewidth{0.602250pt}%
\definecolor{currentstroke}{rgb}{0.000000,0.000000,0.000000}%
\pgfsetstrokecolor{currentstroke}%
\pgfsetdash{}{0pt}%
\pgfsys@defobject{currentmarker}{\pgfqpoint{-0.027778in}{0.000000in}}{\pgfqpoint{-0.000000in}{0.000000in}}{%
\pgfpathmoveto{\pgfqpoint{-0.000000in}{0.000000in}}%
\pgfpathlineto{\pgfqpoint{-0.027778in}{0.000000in}}%
\pgfusepath{stroke,fill}%
}%
\begin{pgfscope}%
\pgfsys@transformshift{0.589510in}{2.075451in}%
\pgfsys@useobject{currentmarker}{}%
\end{pgfscope}%
\end{pgfscope}%
\begin{pgfscope}%
\pgfpathrectangle{\pgfqpoint{0.589510in}{0.417642in}}{\pgfqpoint{3.437062in}{2.055000in}}%
\pgfusepath{clip}%
\pgfsetrectcap%
\pgfsetroundjoin%
\pgfsetlinewidth{0.803000pt}%
\definecolor{currentstroke}{rgb}{0.850000,0.850000,0.850000}%
\pgfsetstrokecolor{currentstroke}%
\pgfsetdash{}{0pt}%
\pgfpathmoveto{\pgfqpoint{0.589510in}{2.195599in}}%
\pgfpathlineto{\pgfqpoint{4.026572in}{2.195599in}}%
\pgfusepath{stroke}%
\end{pgfscope}%
\begin{pgfscope}%
\pgfsetbuttcap%
\pgfsetroundjoin%
\definecolor{currentfill}{rgb}{0.000000,0.000000,0.000000}%
\pgfsetfillcolor{currentfill}%
\pgfsetlinewidth{0.602250pt}%
\definecolor{currentstroke}{rgb}{0.000000,0.000000,0.000000}%
\pgfsetstrokecolor{currentstroke}%
\pgfsetdash{}{0pt}%
\pgfsys@defobject{currentmarker}{\pgfqpoint{-0.027778in}{0.000000in}}{\pgfqpoint{-0.000000in}{0.000000in}}{%
\pgfpathmoveto{\pgfqpoint{-0.000000in}{0.000000in}}%
\pgfpathlineto{\pgfqpoint{-0.027778in}{0.000000in}}%
\pgfusepath{stroke,fill}%
}%
\begin{pgfscope}%
\pgfsys@transformshift{0.589510in}{2.195599in}%
\pgfsys@useobject{currentmarker}{}%
\end{pgfscope}%
\end{pgfscope}%
\begin{pgfscope}%
\pgfpathrectangle{\pgfqpoint{0.589510in}{0.417642in}}{\pgfqpoint{3.437062in}{2.055000in}}%
\pgfusepath{clip}%
\pgfsetrectcap%
\pgfsetroundjoin%
\pgfsetlinewidth{0.803000pt}%
\definecolor{currentstroke}{rgb}{0.850000,0.850000,0.850000}%
\pgfsetstrokecolor{currentstroke}%
\pgfsetdash{}{0pt}%
\pgfpathmoveto{\pgfqpoint{0.589510in}{2.280846in}}%
\pgfpathlineto{\pgfqpoint{4.026572in}{2.280846in}}%
\pgfusepath{stroke}%
\end{pgfscope}%
\begin{pgfscope}%
\pgfsetbuttcap%
\pgfsetroundjoin%
\definecolor{currentfill}{rgb}{0.000000,0.000000,0.000000}%
\pgfsetfillcolor{currentfill}%
\pgfsetlinewidth{0.602250pt}%
\definecolor{currentstroke}{rgb}{0.000000,0.000000,0.000000}%
\pgfsetstrokecolor{currentstroke}%
\pgfsetdash{}{0pt}%
\pgfsys@defobject{currentmarker}{\pgfqpoint{-0.027778in}{0.000000in}}{\pgfqpoint{-0.000000in}{0.000000in}}{%
\pgfpathmoveto{\pgfqpoint{-0.000000in}{0.000000in}}%
\pgfpathlineto{\pgfqpoint{-0.027778in}{0.000000in}}%
\pgfusepath{stroke,fill}%
}%
\begin{pgfscope}%
\pgfsys@transformshift{0.589510in}{2.280846in}%
\pgfsys@useobject{currentmarker}{}%
\end{pgfscope}%
\end{pgfscope}%
\begin{pgfscope}%
\pgfpathrectangle{\pgfqpoint{0.589510in}{0.417642in}}{\pgfqpoint{3.437062in}{2.055000in}}%
\pgfusepath{clip}%
\pgfsetrectcap%
\pgfsetroundjoin%
\pgfsetlinewidth{0.803000pt}%
\definecolor{currentstroke}{rgb}{0.850000,0.850000,0.850000}%
\pgfsetstrokecolor{currentstroke}%
\pgfsetdash{}{0pt}%
\pgfpathmoveto{\pgfqpoint{0.589510in}{2.346968in}}%
\pgfpathlineto{\pgfqpoint{4.026572in}{2.346968in}}%
\pgfusepath{stroke}%
\end{pgfscope}%
\begin{pgfscope}%
\pgfsetbuttcap%
\pgfsetroundjoin%
\definecolor{currentfill}{rgb}{0.000000,0.000000,0.000000}%
\pgfsetfillcolor{currentfill}%
\pgfsetlinewidth{0.602250pt}%
\definecolor{currentstroke}{rgb}{0.000000,0.000000,0.000000}%
\pgfsetstrokecolor{currentstroke}%
\pgfsetdash{}{0pt}%
\pgfsys@defobject{currentmarker}{\pgfqpoint{-0.027778in}{0.000000in}}{\pgfqpoint{-0.000000in}{0.000000in}}{%
\pgfpathmoveto{\pgfqpoint{-0.000000in}{0.000000in}}%
\pgfpathlineto{\pgfqpoint{-0.027778in}{0.000000in}}%
\pgfusepath{stroke,fill}%
}%
\begin{pgfscope}%
\pgfsys@transformshift{0.589510in}{2.346968in}%
\pgfsys@useobject{currentmarker}{}%
\end{pgfscope}%
\end{pgfscope}%
\begin{pgfscope}%
\pgfpathrectangle{\pgfqpoint{0.589510in}{0.417642in}}{\pgfqpoint{3.437062in}{2.055000in}}%
\pgfusepath{clip}%
\pgfsetrectcap%
\pgfsetroundjoin%
\pgfsetlinewidth{0.803000pt}%
\definecolor{currentstroke}{rgb}{0.850000,0.850000,0.850000}%
\pgfsetstrokecolor{currentstroke}%
\pgfsetdash{}{0pt}%
\pgfpathmoveto{\pgfqpoint{0.589510in}{2.400994in}}%
\pgfpathlineto{\pgfqpoint{4.026572in}{2.400994in}}%
\pgfusepath{stroke}%
\end{pgfscope}%
\begin{pgfscope}%
\pgfsetbuttcap%
\pgfsetroundjoin%
\definecolor{currentfill}{rgb}{0.000000,0.000000,0.000000}%
\pgfsetfillcolor{currentfill}%
\pgfsetlinewidth{0.602250pt}%
\definecolor{currentstroke}{rgb}{0.000000,0.000000,0.000000}%
\pgfsetstrokecolor{currentstroke}%
\pgfsetdash{}{0pt}%
\pgfsys@defobject{currentmarker}{\pgfqpoint{-0.027778in}{0.000000in}}{\pgfqpoint{-0.000000in}{0.000000in}}{%
\pgfpathmoveto{\pgfqpoint{-0.000000in}{0.000000in}}%
\pgfpathlineto{\pgfqpoint{-0.027778in}{0.000000in}}%
\pgfusepath{stroke,fill}%
}%
\begin{pgfscope}%
\pgfsys@transformshift{0.589510in}{2.400994in}%
\pgfsys@useobject{currentmarker}{}%
\end{pgfscope}%
\end{pgfscope}%
\begin{pgfscope}%
\pgfpathrectangle{\pgfqpoint{0.589510in}{0.417642in}}{\pgfqpoint{3.437062in}{2.055000in}}%
\pgfusepath{clip}%
\pgfsetrectcap%
\pgfsetroundjoin%
\pgfsetlinewidth{0.803000pt}%
\definecolor{currentstroke}{rgb}{0.850000,0.850000,0.850000}%
\pgfsetstrokecolor{currentstroke}%
\pgfsetdash{}{0pt}%
\pgfpathmoveto{\pgfqpoint{0.589510in}{2.446672in}}%
\pgfpathlineto{\pgfqpoint{4.026572in}{2.446672in}}%
\pgfusepath{stroke}%
\end{pgfscope}%
\begin{pgfscope}%
\pgfsetbuttcap%
\pgfsetroundjoin%
\definecolor{currentfill}{rgb}{0.000000,0.000000,0.000000}%
\pgfsetfillcolor{currentfill}%
\pgfsetlinewidth{0.602250pt}%
\definecolor{currentstroke}{rgb}{0.000000,0.000000,0.000000}%
\pgfsetstrokecolor{currentstroke}%
\pgfsetdash{}{0pt}%
\pgfsys@defobject{currentmarker}{\pgfqpoint{-0.027778in}{0.000000in}}{\pgfqpoint{-0.000000in}{0.000000in}}{%
\pgfpathmoveto{\pgfqpoint{-0.000000in}{0.000000in}}%
\pgfpathlineto{\pgfqpoint{-0.027778in}{0.000000in}}%
\pgfusepath{stroke,fill}%
}%
\begin{pgfscope}%
\pgfsys@transformshift{0.589510in}{2.446672in}%
\pgfsys@useobject{currentmarker}{}%
\end{pgfscope}%
\end{pgfscope}%
\begin{pgfscope}%
\definecolor{textcolor}{rgb}{0.000000,0.000000,0.000000}%
\pgfsetstrokecolor{textcolor}%
\pgfsetfillcolor{textcolor}%
\pgftext[x=0.180559in,y=1.445142in,,bottom,rotate=90.000000]{\color{textcolor}\rmfamily\fontsize{10.000000}{12.000000}\selectfont ADEV \(\displaystyle \sigma_A(\tau)\) in \unit{\V}}%
\end{pgfscope}%
\begin{pgfscope}%
\pgfpathrectangle{\pgfqpoint{0.589510in}{0.417642in}}{\pgfqpoint{3.437062in}{2.055000in}}%
\pgfusepath{clip}%
\pgfsetbuttcap%
\pgfsetroundjoin%
\definecolor{currentfill}{rgb}{0.925490,0.882353,0.200000}%
\pgfsetfillcolor{currentfill}%
\pgfsetlinewidth{1.003750pt}%
\definecolor{currentstroke}{rgb}{0.925490,0.882353,0.200000}%
\pgfsetstrokecolor{currentstroke}%
\pgfsetdash{}{0pt}%
\pgfsys@defobject{currentmarker}{\pgfqpoint{-0.020833in}{-0.020833in}}{\pgfqpoint{0.020833in}{0.020833in}}{%
\pgfpathmoveto{\pgfqpoint{0.000000in}{-0.020833in}}%
\pgfpathcurveto{\pgfqpoint{0.005525in}{-0.020833in}}{\pgfqpoint{0.010825in}{-0.018638in}}{\pgfqpoint{0.014731in}{-0.014731in}}%
\pgfpathcurveto{\pgfqpoint{0.018638in}{-0.010825in}}{\pgfqpoint{0.020833in}{-0.005525in}}{\pgfqpoint{0.020833in}{0.000000in}}%
\pgfpathcurveto{\pgfqpoint{0.020833in}{0.005525in}}{\pgfqpoint{0.018638in}{0.010825in}}{\pgfqpoint{0.014731in}{0.014731in}}%
\pgfpathcurveto{\pgfqpoint{0.010825in}{0.018638in}}{\pgfqpoint{0.005525in}{0.020833in}}{\pgfqpoint{0.000000in}{0.020833in}}%
\pgfpathcurveto{\pgfqpoint{-0.005525in}{0.020833in}}{\pgfqpoint{-0.010825in}{0.018638in}}{\pgfqpoint{-0.014731in}{0.014731in}}%
\pgfpathcurveto{\pgfqpoint{-0.018638in}{0.010825in}}{\pgfqpoint{-0.020833in}{0.005525in}}{\pgfqpoint{-0.020833in}{0.000000in}}%
\pgfpathcurveto{\pgfqpoint{-0.020833in}{-0.005525in}}{\pgfqpoint{-0.018638in}{-0.010825in}}{\pgfqpoint{-0.014731in}{-0.014731in}}%
\pgfpathcurveto{\pgfqpoint{-0.010825in}{-0.018638in}}{\pgfqpoint{-0.005525in}{-0.020833in}}{\pgfqpoint{0.000000in}{-0.020833in}}%
\pgfpathlineto{\pgfqpoint{0.000000in}{-0.020833in}}%
\pgfpathclose%
\pgfusepath{stroke,fill}%
}%
\begin{pgfscope}%
\pgfsys@transformshift{0.745740in}{2.379233in}%
\pgfsys@useobject{currentmarker}{}%
\end{pgfscope}%
\begin{pgfscope}%
\pgfsys@transformshift{0.929540in}{2.269764in}%
\pgfsys@useobject{currentmarker}{}%
\end{pgfscope}%
\begin{pgfscope}%
\pgfsys@transformshift{1.113340in}{2.160671in}%
\pgfsys@useobject{currentmarker}{}%
\end{pgfscope}%
\begin{pgfscope}%
\pgfsys@transformshift{1.297140in}{2.054742in}%
\pgfsys@useobject{currentmarker}{}%
\end{pgfscope}%
\begin{pgfscope}%
\pgfsys@transformshift{1.480940in}{1.949170in}%
\pgfsys@useobject{currentmarker}{}%
\end{pgfscope}%
\begin{pgfscope}%
\pgfsys@transformshift{1.664740in}{1.844707in}%
\pgfsys@useobject{currentmarker}{}%
\end{pgfscope}%
\begin{pgfscope}%
\pgfsys@transformshift{1.848540in}{1.739518in}%
\pgfsys@useobject{currentmarker}{}%
\end{pgfscope}%
\begin{pgfscope}%
\pgfsys@transformshift{2.032341in}{1.640852in}%
\pgfsys@useobject{currentmarker}{}%
\end{pgfscope}%
\begin{pgfscope}%
\pgfsys@transformshift{2.216141in}{1.539100in}%
\pgfsys@useobject{currentmarker}{}%
\end{pgfscope}%
\begin{pgfscope}%
\pgfsys@transformshift{2.399941in}{1.440819in}%
\pgfsys@useobject{currentmarker}{}%
\end{pgfscope}%
\begin{pgfscope}%
\pgfsys@transformshift{2.583741in}{1.331430in}%
\pgfsys@useobject{currentmarker}{}%
\end{pgfscope}%
\begin{pgfscope}%
\pgfsys@transformshift{2.767541in}{1.237471in}%
\pgfsys@useobject{currentmarker}{}%
\end{pgfscope}%
\begin{pgfscope}%
\pgfsys@transformshift{2.951341in}{1.138768in}%
\pgfsys@useobject{currentmarker}{}%
\end{pgfscope}%
\begin{pgfscope}%
\pgfsys@transformshift{3.135141in}{1.015049in}%
\pgfsys@useobject{currentmarker}{}%
\end{pgfscope}%
\begin{pgfscope}%
\pgfsys@transformshift{3.318941in}{0.900614in}%
\pgfsys@useobject{currentmarker}{}%
\end{pgfscope}%
\begin{pgfscope}%
\pgfsys@transformshift{3.502741in}{0.763555in}%
\pgfsys@useobject{currentmarker}{}%
\end{pgfscope}%
\begin{pgfscope}%
\pgfsys@transformshift{3.686541in}{0.686051in}%
\pgfsys@useobject{currentmarker}{}%
\end{pgfscope}%
\begin{pgfscope}%
\pgfsys@transformshift{3.870342in}{0.571408in}%
\pgfsys@useobject{currentmarker}{}%
\end{pgfscope}%
\end{pgfscope}%
\begin{pgfscope}%
\pgfpathrectangle{\pgfqpoint{0.589510in}{0.417642in}}{\pgfqpoint{3.437062in}{2.055000in}}%
\pgfusepath{clip}%
\pgfsetbuttcap%
\pgfsetroundjoin%
\pgfsetlinewidth{1.505625pt}%
\definecolor{currentstroke}{rgb}{0.003922,0.450980,0.698039}%
\pgfsetstrokecolor{currentstroke}%
\pgfsetdash{{5.550000pt}{2.400000pt}}{0.000000pt}%
\pgfpathmoveto{\pgfqpoint{0.745740in}{2.051509in}}%
\pgfpathlineto{\pgfqpoint{0.929540in}{1.948811in}}%
\pgfpathlineto{\pgfqpoint{1.113340in}{1.846114in}}%
\pgfpathlineto{\pgfqpoint{1.297140in}{1.743417in}}%
\pgfpathlineto{\pgfqpoint{1.480940in}{1.640720in}}%
\pgfpathlineto{\pgfqpoint{1.664740in}{1.538023in}}%
\pgfpathlineto{\pgfqpoint{1.848540in}{1.435326in}}%
\pgfpathlineto{\pgfqpoint{2.032341in}{1.332628in}}%
\pgfpathlineto{\pgfqpoint{2.216141in}{1.229931in}}%
\pgfpathlineto{\pgfqpoint{2.399941in}{1.127234in}}%
\pgfpathlineto{\pgfqpoint{2.583741in}{1.024537in}}%
\pgfpathlineto{\pgfqpoint{2.767541in}{0.921840in}}%
\pgfpathlineto{\pgfqpoint{2.951341in}{0.819143in}}%
\pgfpathlineto{\pgfqpoint{3.135141in}{0.716445in}}%
\pgfpathlineto{\pgfqpoint{3.318941in}{0.613748in}}%
\pgfpathlineto{\pgfqpoint{3.502741in}{0.511051in}}%
\pgfusepath{stroke}%
\end{pgfscope}%
\begin{pgfscope}%
\pgfpathrectangle{\pgfqpoint{0.589510in}{0.417642in}}{\pgfqpoint{3.437062in}{2.055000in}}%
\pgfusepath{clip}%
\pgfsetbuttcap%
\pgfsetroundjoin%
\pgfsetlinewidth{1.505625pt}%
\definecolor{currentstroke}{rgb}{0.007843,0.619608,0.450980}%
\pgfsetstrokecolor{currentstroke}%
\pgfsetdash{{5.550000pt}{2.400000pt}}{0.000000pt}%
\pgfpathmoveto{\pgfqpoint{0.745740in}{2.126916in}}%
\pgfpathlineto{\pgfqpoint{0.929540in}{2.126916in}}%
\pgfpathlineto{\pgfqpoint{1.113340in}{2.126916in}}%
\pgfpathlineto{\pgfqpoint{1.297140in}{2.126916in}}%
\pgfpathlineto{\pgfqpoint{1.480940in}{2.126916in}}%
\pgfpathlineto{\pgfqpoint{1.664740in}{2.126916in}}%
\pgfpathlineto{\pgfqpoint{1.848540in}{2.126916in}}%
\pgfpathlineto{\pgfqpoint{2.032341in}{2.126916in}}%
\pgfpathlineto{\pgfqpoint{2.216141in}{2.126916in}}%
\pgfpathlineto{\pgfqpoint{2.399941in}{2.126916in}}%
\pgfpathlineto{\pgfqpoint{2.583741in}{2.126916in}}%
\pgfpathlineto{\pgfqpoint{2.767541in}{2.126916in}}%
\pgfpathlineto{\pgfqpoint{2.951341in}{2.126916in}}%
\pgfpathlineto{\pgfqpoint{3.135141in}{2.126916in}}%
\pgfpathlineto{\pgfqpoint{3.318941in}{2.126916in}}%
\pgfpathlineto{\pgfqpoint{3.502741in}{2.126916in}}%
\pgfpathlineto{\pgfqpoint{3.686541in}{2.126916in}}%
\pgfpathlineto{\pgfqpoint{3.870342in}{2.126916in}}%
\pgfusepath{stroke}%
\end{pgfscope}%
\begin{pgfscope}%
\pgfsetrectcap%
\pgfsetmiterjoin%
\pgfsetlinewidth{0.803000pt}%
\definecolor{currentstroke}{rgb}{0.000000,0.000000,0.000000}%
\pgfsetstrokecolor{currentstroke}%
\pgfsetdash{}{0pt}%
\pgfpathmoveto{\pgfqpoint{0.589510in}{0.417642in}}%
\pgfpathlineto{\pgfqpoint{0.589510in}{2.472642in}}%
\pgfusepath{stroke}%
\end{pgfscope}%
\begin{pgfscope}%
\pgfsetrectcap%
\pgfsetmiterjoin%
\pgfsetlinewidth{0.803000pt}%
\definecolor{currentstroke}{rgb}{0.000000,0.000000,0.000000}%
\pgfsetstrokecolor{currentstroke}%
\pgfsetdash{}{0pt}%
\pgfpathmoveto{\pgfqpoint{4.026572in}{0.417642in}}%
\pgfpathlineto{\pgfqpoint{4.026572in}{2.472642in}}%
\pgfusepath{stroke}%
\end{pgfscope}%
\begin{pgfscope}%
\pgfsetrectcap%
\pgfsetmiterjoin%
\pgfsetlinewidth{0.803000pt}%
\definecolor{currentstroke}{rgb}{0.000000,0.000000,0.000000}%
\pgfsetstrokecolor{currentstroke}%
\pgfsetdash{}{0pt}%
\pgfpathmoveto{\pgfqpoint{0.589510in}{0.417642in}}%
\pgfpathlineto{\pgfqpoint{4.026572in}{0.417642in}}%
\pgfusepath{stroke}%
\end{pgfscope}%
\begin{pgfscope}%
\pgfsetrectcap%
\pgfsetmiterjoin%
\pgfsetlinewidth{0.803000pt}%
\definecolor{currentstroke}{rgb}{0.000000,0.000000,0.000000}%
\pgfsetstrokecolor{currentstroke}%
\pgfsetdash{}{0pt}%
\pgfpathmoveto{\pgfqpoint{0.589510in}{2.472642in}}%
\pgfpathlineto{\pgfqpoint{4.026572in}{2.472642in}}%
\pgfusepath{stroke}%
\end{pgfscope}%
\begin{pgfscope}%
\pgfsetbuttcap%
\pgfsetmiterjoin%
\definecolor{currentfill}{rgb}{1.000000,1.000000,1.000000}%
\pgfsetfillcolor{currentfill}%
\pgfsetfillopacity{0.800000}%
\pgfsetlinewidth{1.003750pt}%
\definecolor{currentstroke}{rgb}{0.800000,0.800000,0.800000}%
\pgfsetstrokecolor{currentstroke}%
\pgfsetstrokeopacity{0.800000}%
\pgfsetdash{}{0pt}%
\pgfpathmoveto{\pgfqpoint{2.948460in}{2.073975in}}%
\pgfpathlineto{\pgfqpoint{3.948794in}{2.073975in}}%
\pgfpathquadraticcurveto{\pgfqpoint{3.971016in}{2.073975in}}{\pgfqpoint{3.971016in}{2.096197in}}%
\pgfpathlineto{\pgfqpoint{3.971016in}{2.394864in}}%
\pgfpathquadraticcurveto{\pgfqpoint{3.971016in}{2.417086in}}{\pgfqpoint{3.948794in}{2.417086in}}%
\pgfpathlineto{\pgfqpoint{2.948460in}{2.417086in}}%
\pgfpathquadraticcurveto{\pgfqpoint{2.926238in}{2.417086in}}{\pgfqpoint{2.926238in}{2.394864in}}%
\pgfpathlineto{\pgfqpoint{2.926238in}{2.096197in}}%
\pgfpathquadraticcurveto{\pgfqpoint{2.926238in}{2.073975in}}{\pgfqpoint{2.948460in}{2.073975in}}%
\pgfpathlineto{\pgfqpoint{2.948460in}{2.073975in}}%
\pgfpathclose%
\pgfusepath{stroke,fill}%
\end{pgfscope}%
\begin{pgfscope}%
\pgfsetbuttcap%
\pgfsetroundjoin%
\pgfsetlinewidth{1.505625pt}%
\definecolor{currentstroke}{rgb}{0.003922,0.450980,0.698039}%
\pgfsetstrokecolor{currentstroke}%
\pgfsetdash{{5.550000pt}{2.400000pt}}{0.000000pt}%
\pgfpathmoveto{\pgfqpoint{2.970683in}{2.333753in}}%
\pgfpathlineto{\pgfqpoint{3.081794in}{2.333753in}}%
\pgfpathlineto{\pgfqpoint{3.192905in}{2.333753in}}%
\pgfusepath{stroke}%
\end{pgfscope}%
\begin{pgfscope}%
\definecolor{textcolor}{rgb}{0.000000,0.000000,0.000000}%
\pgfsetstrokecolor{textcolor}%
\pgfsetfillcolor{textcolor}%
\pgftext[x=3.281794in,y=2.294864in,left,base]{\color{textcolor}\rmfamily\fontsize{8.000000}{9.600000}\selectfont White noise}%
\end{pgfscope}%
\begin{pgfscope}%
\pgfsetbuttcap%
\pgfsetroundjoin%
\pgfsetlinewidth{1.505625pt}%
\definecolor{currentstroke}{rgb}{0.007843,0.619608,0.450980}%
\pgfsetstrokecolor{currentstroke}%
\pgfsetdash{{5.550000pt}{2.400000pt}}{0.000000pt}%
\pgfpathmoveto{\pgfqpoint{2.970683in}{2.178864in}}%
\pgfpathlineto{\pgfqpoint{3.081794in}{2.178864in}}%
\pgfpathlineto{\pgfqpoint{3.192905in}{2.178864in}}%
\pgfusepath{stroke}%
\end{pgfscope}%
\begin{pgfscope}%
\definecolor{textcolor}{rgb}{0.000000,0.000000,0.000000}%
\pgfsetstrokecolor{textcolor}%
\pgfsetfillcolor{textcolor}%
\pgftext[x=3.281794in,y=2.139975in,left,base]{\color{textcolor}\rmfamily\fontsize{8.000000}{9.600000}\selectfont Flicker noise}%
\end{pgfscope}%
\end{pgfpicture}%
\makeatother%
\endgroup%

    \caption{Simulated Allan deviation of a Keysight \device{3458A} with autozeroing applied. The dashed lines denote the deviation prior to applying the autozero algorithm.}
    \label{fig:autozero_adev}
\end{figure}

From this plot it can be seen, that for measurement times longer than about \qty{2}{\s} or \qty{100}{\plc} autozeroing has a clear benefit over a measurement without autozeroing. It must be noted though, that judging from this simulation, the device would reach a noise floor of \qty[per-mode = symbol]{0.01}{\V \per \V} only at integration times of slighly more than \qty{10}{\s}, while the datasheet claims \qty{2}{\s}. It is therefore likely, that the noise parameters of a real device are be better than the numbers used in the simulation. Additionally, the datasheet likely refers to an instrument, that is synced to a \qty{60}{\Hz} power line frequency which shifts the sampling frequency up by \qty{20}{\percent} and, as discussed, reduces the noise floor, because more noise content is white noise at the autozero interval. In this simulation the \qty{0.01}{\V \per \V} noise level would be reached at exactly \qty{10}{\s} when using a line frequency of \qty{60}{\Hz}. For the purpose of demonstrating the autozeroing algorithms these subtleties are irrelavant.

For the comparison of different intregation times before applying autozeroing figure \ref{fig:autozero_nplcs_adev} can be consulted. Using the Allan deviation makes it is very simple to compare noise figures for identical measurement times $\tau$, but different integration times before autozeroing is applied.

\begin{figure}[ht]
    \centering
    %% Creator: Matplotlib, PGF backend
%%
%% To include the figure in your LaTeX document, write
%%   \input{<filename>.pgf}
%%
%% Make sure the required packages are loaded in your preamble
%%   \usepackage{pgf}
%%
%% Also ensure that all the required font packages are loaded; for instance,
%% the lmodern package is sometimes necessary when using math font.
%%   \usepackage{lmodern}
%%
%% Figures using additional raster images can only be included by \input if
%% they are in the same directory as the main LaTeX file. For loading figures
%% from other directories you can use the `import` package
%%   \usepackage{import}
%%
%% and then include the figures with
%%   \import{<path to file>}{<filename>.pgf}
%%
%% Matplotlib used the following preamble
%%   \usepackage{siunitx}
%%   \usepackage{fontspec}
%%   \makeatletter\@ifpackageloaded{underscore}{}{\usepackage[strings]{underscore}}\makeatother
%%
\begingroup%
\makeatletter%
\begin{pgfpicture}%
\pgfpathrectangle{\pgfpointorigin}{\pgfqpoint{4.060000in}{2.510000in}}%
\pgfusepath{use as bounding box, clip}%
\begin{pgfscope}%
\pgfsetbuttcap%
\pgfsetmiterjoin%
\definecolor{currentfill}{rgb}{1.000000,1.000000,1.000000}%
\pgfsetfillcolor{currentfill}%
\pgfsetlinewidth{0.000000pt}%
\definecolor{currentstroke}{rgb}{1.000000,1.000000,1.000000}%
\pgfsetstrokecolor{currentstroke}%
\pgfsetdash{}{0pt}%
\pgfpathmoveto{\pgfqpoint{0.000000in}{0.000000in}}%
\pgfpathlineto{\pgfqpoint{4.060000in}{0.000000in}}%
\pgfpathlineto{\pgfqpoint{4.060000in}{2.510000in}}%
\pgfpathlineto{\pgfqpoint{0.000000in}{2.510000in}}%
\pgfpathlineto{\pgfqpoint{0.000000in}{0.000000in}}%
\pgfpathclose%
\pgfusepath{fill}%
\end{pgfscope}%
\begin{pgfscope}%
\pgfsetbuttcap%
\pgfsetmiterjoin%
\definecolor{currentfill}{rgb}{1.000000,1.000000,1.000000}%
\pgfsetfillcolor{currentfill}%
\pgfsetlinewidth{0.000000pt}%
\definecolor{currentstroke}{rgb}{0.000000,0.000000,0.000000}%
\pgfsetstrokecolor{currentstroke}%
\pgfsetstrokeopacity{0.000000}%
\pgfsetdash{}{0pt}%
\pgfpathmoveto{\pgfqpoint{0.589510in}{0.417642in}}%
\pgfpathlineto{\pgfqpoint{4.018330in}{0.417642in}}%
\pgfpathlineto{\pgfqpoint{4.018330in}{2.468330in}}%
\pgfpathlineto{\pgfqpoint{0.589510in}{2.468330in}}%
\pgfpathlineto{\pgfqpoint{0.589510in}{0.417642in}}%
\pgfpathclose%
\pgfusepath{fill}%
\end{pgfscope}%
\begin{pgfscope}%
\pgfpathrectangle{\pgfqpoint{0.589510in}{0.417642in}}{\pgfqpoint{3.428820in}{2.050688in}}%
\pgfusepath{clip}%
\pgfsetrectcap%
\pgfsetroundjoin%
\pgfsetlinewidth{0.803000pt}%
\definecolor{currentstroke}{rgb}{0.450000,0.450000,0.450000}%
\pgfsetstrokecolor{currentstroke}%
\pgfsetdash{}{0pt}%
\pgfpathmoveto{\pgfqpoint{0.745365in}{0.417642in}}%
\pgfpathlineto{\pgfqpoint{0.745365in}{2.468330in}}%
\pgfusepath{stroke}%
\end{pgfscope}%
\begin{pgfscope}%
\pgfsetbuttcap%
\pgfsetroundjoin%
\definecolor{currentfill}{rgb}{0.000000,0.000000,0.000000}%
\pgfsetfillcolor{currentfill}%
\pgfsetlinewidth{0.803000pt}%
\definecolor{currentstroke}{rgb}{0.000000,0.000000,0.000000}%
\pgfsetstrokecolor{currentstroke}%
\pgfsetdash{}{0pt}%
\pgfsys@defobject{currentmarker}{\pgfqpoint{0.000000in}{-0.048611in}}{\pgfqpoint{0.000000in}{0.000000in}}{%
\pgfpathmoveto{\pgfqpoint{0.000000in}{0.000000in}}%
\pgfpathlineto{\pgfqpoint{0.000000in}{-0.048611in}}%
\pgfusepath{stroke,fill}%
}%
\begin{pgfscope}%
\pgfsys@transformshift{0.745365in}{0.417642in}%
\pgfsys@useobject{currentmarker}{}%
\end{pgfscope}%
\end{pgfscope}%
\begin{pgfscope}%
\definecolor{textcolor}{rgb}{0.000000,0.000000,0.000000}%
\pgfsetstrokecolor{textcolor}%
\pgfsetfillcolor{textcolor}%
\pgftext[x=0.745365in,y=0.320420in,,top]{\color{textcolor}\rmfamily\fontsize{8.000000}{9.600000}\selectfont \(\displaystyle {10^{0}}\)}%
\end{pgfscope}%
\begin{pgfscope}%
\pgfpathrectangle{\pgfqpoint{0.589510in}{0.417642in}}{\pgfqpoint{3.428820in}{2.050688in}}%
\pgfusepath{clip}%
\pgfsetrectcap%
\pgfsetroundjoin%
\pgfsetlinewidth{0.803000pt}%
\definecolor{currentstroke}{rgb}{0.450000,0.450000,0.450000}%
\pgfsetstrokecolor{currentstroke}%
\pgfsetdash{}{0pt}%
\pgfpathmoveto{\pgfqpoint{1.524643in}{0.417642in}}%
\pgfpathlineto{\pgfqpoint{1.524643in}{2.468330in}}%
\pgfusepath{stroke}%
\end{pgfscope}%
\begin{pgfscope}%
\pgfsetbuttcap%
\pgfsetroundjoin%
\definecolor{currentfill}{rgb}{0.000000,0.000000,0.000000}%
\pgfsetfillcolor{currentfill}%
\pgfsetlinewidth{0.803000pt}%
\definecolor{currentstroke}{rgb}{0.000000,0.000000,0.000000}%
\pgfsetstrokecolor{currentstroke}%
\pgfsetdash{}{0pt}%
\pgfsys@defobject{currentmarker}{\pgfqpoint{0.000000in}{-0.048611in}}{\pgfqpoint{0.000000in}{0.000000in}}{%
\pgfpathmoveto{\pgfqpoint{0.000000in}{0.000000in}}%
\pgfpathlineto{\pgfqpoint{0.000000in}{-0.048611in}}%
\pgfusepath{stroke,fill}%
}%
\begin{pgfscope}%
\pgfsys@transformshift{1.524643in}{0.417642in}%
\pgfsys@useobject{currentmarker}{}%
\end{pgfscope}%
\end{pgfscope}%
\begin{pgfscope}%
\definecolor{textcolor}{rgb}{0.000000,0.000000,0.000000}%
\pgfsetstrokecolor{textcolor}%
\pgfsetfillcolor{textcolor}%
\pgftext[x=1.524643in,y=0.320420in,,top]{\color{textcolor}\rmfamily\fontsize{8.000000}{9.600000}\selectfont \(\displaystyle {10^{1}}\)}%
\end{pgfscope}%
\begin{pgfscope}%
\pgfpathrectangle{\pgfqpoint{0.589510in}{0.417642in}}{\pgfqpoint{3.428820in}{2.050688in}}%
\pgfusepath{clip}%
\pgfsetrectcap%
\pgfsetroundjoin%
\pgfsetlinewidth{0.803000pt}%
\definecolor{currentstroke}{rgb}{0.450000,0.450000,0.450000}%
\pgfsetstrokecolor{currentstroke}%
\pgfsetdash{}{0pt}%
\pgfpathmoveto{\pgfqpoint{2.303920in}{0.417642in}}%
\pgfpathlineto{\pgfqpoint{2.303920in}{2.468330in}}%
\pgfusepath{stroke}%
\end{pgfscope}%
\begin{pgfscope}%
\pgfsetbuttcap%
\pgfsetroundjoin%
\definecolor{currentfill}{rgb}{0.000000,0.000000,0.000000}%
\pgfsetfillcolor{currentfill}%
\pgfsetlinewidth{0.803000pt}%
\definecolor{currentstroke}{rgb}{0.000000,0.000000,0.000000}%
\pgfsetstrokecolor{currentstroke}%
\pgfsetdash{}{0pt}%
\pgfsys@defobject{currentmarker}{\pgfqpoint{0.000000in}{-0.048611in}}{\pgfqpoint{0.000000in}{0.000000in}}{%
\pgfpathmoveto{\pgfqpoint{0.000000in}{0.000000in}}%
\pgfpathlineto{\pgfqpoint{0.000000in}{-0.048611in}}%
\pgfusepath{stroke,fill}%
}%
\begin{pgfscope}%
\pgfsys@transformshift{2.303920in}{0.417642in}%
\pgfsys@useobject{currentmarker}{}%
\end{pgfscope}%
\end{pgfscope}%
\begin{pgfscope}%
\definecolor{textcolor}{rgb}{0.000000,0.000000,0.000000}%
\pgfsetstrokecolor{textcolor}%
\pgfsetfillcolor{textcolor}%
\pgftext[x=2.303920in,y=0.320420in,,top]{\color{textcolor}\rmfamily\fontsize{8.000000}{9.600000}\selectfont \(\displaystyle {10^{2}}\)}%
\end{pgfscope}%
\begin{pgfscope}%
\pgfpathrectangle{\pgfqpoint{0.589510in}{0.417642in}}{\pgfqpoint{3.428820in}{2.050688in}}%
\pgfusepath{clip}%
\pgfsetrectcap%
\pgfsetroundjoin%
\pgfsetlinewidth{0.803000pt}%
\definecolor{currentstroke}{rgb}{0.450000,0.450000,0.450000}%
\pgfsetstrokecolor{currentstroke}%
\pgfsetdash{}{0pt}%
\pgfpathmoveto{\pgfqpoint{3.083197in}{0.417642in}}%
\pgfpathlineto{\pgfqpoint{3.083197in}{2.468330in}}%
\pgfusepath{stroke}%
\end{pgfscope}%
\begin{pgfscope}%
\pgfsetbuttcap%
\pgfsetroundjoin%
\definecolor{currentfill}{rgb}{0.000000,0.000000,0.000000}%
\pgfsetfillcolor{currentfill}%
\pgfsetlinewidth{0.803000pt}%
\definecolor{currentstroke}{rgb}{0.000000,0.000000,0.000000}%
\pgfsetstrokecolor{currentstroke}%
\pgfsetdash{}{0pt}%
\pgfsys@defobject{currentmarker}{\pgfqpoint{0.000000in}{-0.048611in}}{\pgfqpoint{0.000000in}{0.000000in}}{%
\pgfpathmoveto{\pgfqpoint{0.000000in}{0.000000in}}%
\pgfpathlineto{\pgfqpoint{0.000000in}{-0.048611in}}%
\pgfusepath{stroke,fill}%
}%
\begin{pgfscope}%
\pgfsys@transformshift{3.083197in}{0.417642in}%
\pgfsys@useobject{currentmarker}{}%
\end{pgfscope}%
\end{pgfscope}%
\begin{pgfscope}%
\definecolor{textcolor}{rgb}{0.000000,0.000000,0.000000}%
\pgfsetstrokecolor{textcolor}%
\pgfsetfillcolor{textcolor}%
\pgftext[x=3.083197in,y=0.320420in,,top]{\color{textcolor}\rmfamily\fontsize{8.000000}{9.600000}\selectfont \(\displaystyle {10^{3}}\)}%
\end{pgfscope}%
\begin{pgfscope}%
\pgfpathrectangle{\pgfqpoint{0.589510in}{0.417642in}}{\pgfqpoint{3.428820in}{2.050688in}}%
\pgfusepath{clip}%
\pgfsetrectcap%
\pgfsetroundjoin%
\pgfsetlinewidth{0.803000pt}%
\definecolor{currentstroke}{rgb}{0.450000,0.450000,0.450000}%
\pgfsetstrokecolor{currentstroke}%
\pgfsetdash{}{0pt}%
\pgfpathmoveto{\pgfqpoint{3.862475in}{0.417642in}}%
\pgfpathlineto{\pgfqpoint{3.862475in}{2.468330in}}%
\pgfusepath{stroke}%
\end{pgfscope}%
\begin{pgfscope}%
\pgfsetbuttcap%
\pgfsetroundjoin%
\definecolor{currentfill}{rgb}{0.000000,0.000000,0.000000}%
\pgfsetfillcolor{currentfill}%
\pgfsetlinewidth{0.803000pt}%
\definecolor{currentstroke}{rgb}{0.000000,0.000000,0.000000}%
\pgfsetstrokecolor{currentstroke}%
\pgfsetdash{}{0pt}%
\pgfsys@defobject{currentmarker}{\pgfqpoint{0.000000in}{-0.048611in}}{\pgfqpoint{0.000000in}{0.000000in}}{%
\pgfpathmoveto{\pgfqpoint{0.000000in}{0.000000in}}%
\pgfpathlineto{\pgfqpoint{0.000000in}{-0.048611in}}%
\pgfusepath{stroke,fill}%
}%
\begin{pgfscope}%
\pgfsys@transformshift{3.862475in}{0.417642in}%
\pgfsys@useobject{currentmarker}{}%
\end{pgfscope}%
\end{pgfscope}%
\begin{pgfscope}%
\definecolor{textcolor}{rgb}{0.000000,0.000000,0.000000}%
\pgfsetstrokecolor{textcolor}%
\pgfsetfillcolor{textcolor}%
\pgftext[x=3.862475in,y=0.320420in,,top]{\color{textcolor}\rmfamily\fontsize{8.000000}{9.600000}\selectfont \(\displaystyle {10^{4}}\)}%
\end{pgfscope}%
\begin{pgfscope}%
\pgfpathrectangle{\pgfqpoint{0.589510in}{0.417642in}}{\pgfqpoint{3.428820in}{2.050688in}}%
\pgfusepath{clip}%
\pgfsetrectcap%
\pgfsetroundjoin%
\pgfsetlinewidth{0.803000pt}%
\definecolor{currentstroke}{rgb}{0.850000,0.850000,0.850000}%
\pgfsetstrokecolor{currentstroke}%
\pgfsetdash{}{0pt}%
\pgfpathmoveto{\pgfqpoint{0.624654in}{0.417642in}}%
\pgfpathlineto{\pgfqpoint{0.624654in}{2.468330in}}%
\pgfusepath{stroke}%
\end{pgfscope}%
\begin{pgfscope}%
\pgfsetbuttcap%
\pgfsetroundjoin%
\definecolor{currentfill}{rgb}{0.000000,0.000000,0.000000}%
\pgfsetfillcolor{currentfill}%
\pgfsetlinewidth{0.602250pt}%
\definecolor{currentstroke}{rgb}{0.000000,0.000000,0.000000}%
\pgfsetstrokecolor{currentstroke}%
\pgfsetdash{}{0pt}%
\pgfsys@defobject{currentmarker}{\pgfqpoint{0.000000in}{-0.027778in}}{\pgfqpoint{0.000000in}{0.000000in}}{%
\pgfpathmoveto{\pgfqpoint{0.000000in}{0.000000in}}%
\pgfpathlineto{\pgfqpoint{0.000000in}{-0.027778in}}%
\pgfusepath{stroke,fill}%
}%
\begin{pgfscope}%
\pgfsys@transformshift{0.624654in}{0.417642in}%
\pgfsys@useobject{currentmarker}{}%
\end{pgfscope}%
\end{pgfscope}%
\begin{pgfscope}%
\pgfpathrectangle{\pgfqpoint{0.589510in}{0.417642in}}{\pgfqpoint{3.428820in}{2.050688in}}%
\pgfusepath{clip}%
\pgfsetrectcap%
\pgfsetroundjoin%
\pgfsetlinewidth{0.803000pt}%
\definecolor{currentstroke}{rgb}{0.850000,0.850000,0.850000}%
\pgfsetstrokecolor{currentstroke}%
\pgfsetdash{}{0pt}%
\pgfpathmoveto{\pgfqpoint{0.669845in}{0.417642in}}%
\pgfpathlineto{\pgfqpoint{0.669845in}{2.468330in}}%
\pgfusepath{stroke}%
\end{pgfscope}%
\begin{pgfscope}%
\pgfsetbuttcap%
\pgfsetroundjoin%
\definecolor{currentfill}{rgb}{0.000000,0.000000,0.000000}%
\pgfsetfillcolor{currentfill}%
\pgfsetlinewidth{0.602250pt}%
\definecolor{currentstroke}{rgb}{0.000000,0.000000,0.000000}%
\pgfsetstrokecolor{currentstroke}%
\pgfsetdash{}{0pt}%
\pgfsys@defobject{currentmarker}{\pgfqpoint{0.000000in}{-0.027778in}}{\pgfqpoint{0.000000in}{0.000000in}}{%
\pgfpathmoveto{\pgfqpoint{0.000000in}{0.000000in}}%
\pgfpathlineto{\pgfqpoint{0.000000in}{-0.027778in}}%
\pgfusepath{stroke,fill}%
}%
\begin{pgfscope}%
\pgfsys@transformshift{0.669845in}{0.417642in}%
\pgfsys@useobject{currentmarker}{}%
\end{pgfscope}%
\end{pgfscope}%
\begin{pgfscope}%
\pgfpathrectangle{\pgfqpoint{0.589510in}{0.417642in}}{\pgfqpoint{3.428820in}{2.050688in}}%
\pgfusepath{clip}%
\pgfsetrectcap%
\pgfsetroundjoin%
\pgfsetlinewidth{0.803000pt}%
\definecolor{currentstroke}{rgb}{0.850000,0.850000,0.850000}%
\pgfsetstrokecolor{currentstroke}%
\pgfsetdash{}{0pt}%
\pgfpathmoveto{\pgfqpoint{0.709707in}{0.417642in}}%
\pgfpathlineto{\pgfqpoint{0.709707in}{2.468330in}}%
\pgfusepath{stroke}%
\end{pgfscope}%
\begin{pgfscope}%
\pgfsetbuttcap%
\pgfsetroundjoin%
\definecolor{currentfill}{rgb}{0.000000,0.000000,0.000000}%
\pgfsetfillcolor{currentfill}%
\pgfsetlinewidth{0.602250pt}%
\definecolor{currentstroke}{rgb}{0.000000,0.000000,0.000000}%
\pgfsetstrokecolor{currentstroke}%
\pgfsetdash{}{0pt}%
\pgfsys@defobject{currentmarker}{\pgfqpoint{0.000000in}{-0.027778in}}{\pgfqpoint{0.000000in}{0.000000in}}{%
\pgfpathmoveto{\pgfqpoint{0.000000in}{0.000000in}}%
\pgfpathlineto{\pgfqpoint{0.000000in}{-0.027778in}}%
\pgfusepath{stroke,fill}%
}%
\begin{pgfscope}%
\pgfsys@transformshift{0.709707in}{0.417642in}%
\pgfsys@useobject{currentmarker}{}%
\end{pgfscope}%
\end{pgfscope}%
\begin{pgfscope}%
\pgfpathrectangle{\pgfqpoint{0.589510in}{0.417642in}}{\pgfqpoint{3.428820in}{2.050688in}}%
\pgfusepath{clip}%
\pgfsetrectcap%
\pgfsetroundjoin%
\pgfsetlinewidth{0.803000pt}%
\definecolor{currentstroke}{rgb}{0.850000,0.850000,0.850000}%
\pgfsetstrokecolor{currentstroke}%
\pgfsetdash{}{0pt}%
\pgfpathmoveto{\pgfqpoint{0.979951in}{0.417642in}}%
\pgfpathlineto{\pgfqpoint{0.979951in}{2.468330in}}%
\pgfusepath{stroke}%
\end{pgfscope}%
\begin{pgfscope}%
\pgfsetbuttcap%
\pgfsetroundjoin%
\definecolor{currentfill}{rgb}{0.000000,0.000000,0.000000}%
\pgfsetfillcolor{currentfill}%
\pgfsetlinewidth{0.602250pt}%
\definecolor{currentstroke}{rgb}{0.000000,0.000000,0.000000}%
\pgfsetstrokecolor{currentstroke}%
\pgfsetdash{}{0pt}%
\pgfsys@defobject{currentmarker}{\pgfqpoint{0.000000in}{-0.027778in}}{\pgfqpoint{0.000000in}{0.000000in}}{%
\pgfpathmoveto{\pgfqpoint{0.000000in}{0.000000in}}%
\pgfpathlineto{\pgfqpoint{0.000000in}{-0.027778in}}%
\pgfusepath{stroke,fill}%
}%
\begin{pgfscope}%
\pgfsys@transformshift{0.979951in}{0.417642in}%
\pgfsys@useobject{currentmarker}{}%
\end{pgfscope}%
\end{pgfscope}%
\begin{pgfscope}%
\pgfpathrectangle{\pgfqpoint{0.589510in}{0.417642in}}{\pgfqpoint{3.428820in}{2.050688in}}%
\pgfusepath{clip}%
\pgfsetrectcap%
\pgfsetroundjoin%
\pgfsetlinewidth{0.803000pt}%
\definecolor{currentstroke}{rgb}{0.850000,0.850000,0.850000}%
\pgfsetstrokecolor{currentstroke}%
\pgfsetdash{}{0pt}%
\pgfpathmoveto{\pgfqpoint{1.117175in}{0.417642in}}%
\pgfpathlineto{\pgfqpoint{1.117175in}{2.468330in}}%
\pgfusepath{stroke}%
\end{pgfscope}%
\begin{pgfscope}%
\pgfsetbuttcap%
\pgfsetroundjoin%
\definecolor{currentfill}{rgb}{0.000000,0.000000,0.000000}%
\pgfsetfillcolor{currentfill}%
\pgfsetlinewidth{0.602250pt}%
\definecolor{currentstroke}{rgb}{0.000000,0.000000,0.000000}%
\pgfsetstrokecolor{currentstroke}%
\pgfsetdash{}{0pt}%
\pgfsys@defobject{currentmarker}{\pgfqpoint{0.000000in}{-0.027778in}}{\pgfqpoint{0.000000in}{0.000000in}}{%
\pgfpathmoveto{\pgfqpoint{0.000000in}{0.000000in}}%
\pgfpathlineto{\pgfqpoint{0.000000in}{-0.027778in}}%
\pgfusepath{stroke,fill}%
}%
\begin{pgfscope}%
\pgfsys@transformshift{1.117175in}{0.417642in}%
\pgfsys@useobject{currentmarker}{}%
\end{pgfscope}%
\end{pgfscope}%
\begin{pgfscope}%
\pgfpathrectangle{\pgfqpoint{0.589510in}{0.417642in}}{\pgfqpoint{3.428820in}{2.050688in}}%
\pgfusepath{clip}%
\pgfsetrectcap%
\pgfsetroundjoin%
\pgfsetlinewidth{0.803000pt}%
\definecolor{currentstroke}{rgb}{0.850000,0.850000,0.850000}%
\pgfsetstrokecolor{currentstroke}%
\pgfsetdash{}{0pt}%
\pgfpathmoveto{\pgfqpoint{1.214537in}{0.417642in}}%
\pgfpathlineto{\pgfqpoint{1.214537in}{2.468330in}}%
\pgfusepath{stroke}%
\end{pgfscope}%
\begin{pgfscope}%
\pgfsetbuttcap%
\pgfsetroundjoin%
\definecolor{currentfill}{rgb}{0.000000,0.000000,0.000000}%
\pgfsetfillcolor{currentfill}%
\pgfsetlinewidth{0.602250pt}%
\definecolor{currentstroke}{rgb}{0.000000,0.000000,0.000000}%
\pgfsetstrokecolor{currentstroke}%
\pgfsetdash{}{0pt}%
\pgfsys@defobject{currentmarker}{\pgfqpoint{0.000000in}{-0.027778in}}{\pgfqpoint{0.000000in}{0.000000in}}{%
\pgfpathmoveto{\pgfqpoint{0.000000in}{0.000000in}}%
\pgfpathlineto{\pgfqpoint{0.000000in}{-0.027778in}}%
\pgfusepath{stroke,fill}%
}%
\begin{pgfscope}%
\pgfsys@transformshift{1.214537in}{0.417642in}%
\pgfsys@useobject{currentmarker}{}%
\end{pgfscope}%
\end{pgfscope}%
\begin{pgfscope}%
\pgfpathrectangle{\pgfqpoint{0.589510in}{0.417642in}}{\pgfqpoint{3.428820in}{2.050688in}}%
\pgfusepath{clip}%
\pgfsetrectcap%
\pgfsetroundjoin%
\pgfsetlinewidth{0.803000pt}%
\definecolor{currentstroke}{rgb}{0.850000,0.850000,0.850000}%
\pgfsetstrokecolor{currentstroke}%
\pgfsetdash{}{0pt}%
\pgfpathmoveto{\pgfqpoint{1.290057in}{0.417642in}}%
\pgfpathlineto{\pgfqpoint{1.290057in}{2.468330in}}%
\pgfusepath{stroke}%
\end{pgfscope}%
\begin{pgfscope}%
\pgfsetbuttcap%
\pgfsetroundjoin%
\definecolor{currentfill}{rgb}{0.000000,0.000000,0.000000}%
\pgfsetfillcolor{currentfill}%
\pgfsetlinewidth{0.602250pt}%
\definecolor{currentstroke}{rgb}{0.000000,0.000000,0.000000}%
\pgfsetstrokecolor{currentstroke}%
\pgfsetdash{}{0pt}%
\pgfsys@defobject{currentmarker}{\pgfqpoint{0.000000in}{-0.027778in}}{\pgfqpoint{0.000000in}{0.000000in}}{%
\pgfpathmoveto{\pgfqpoint{0.000000in}{0.000000in}}%
\pgfpathlineto{\pgfqpoint{0.000000in}{-0.027778in}}%
\pgfusepath{stroke,fill}%
}%
\begin{pgfscope}%
\pgfsys@transformshift{1.290057in}{0.417642in}%
\pgfsys@useobject{currentmarker}{}%
\end{pgfscope}%
\end{pgfscope}%
\begin{pgfscope}%
\pgfpathrectangle{\pgfqpoint{0.589510in}{0.417642in}}{\pgfqpoint{3.428820in}{2.050688in}}%
\pgfusepath{clip}%
\pgfsetrectcap%
\pgfsetroundjoin%
\pgfsetlinewidth{0.803000pt}%
\definecolor{currentstroke}{rgb}{0.850000,0.850000,0.850000}%
\pgfsetstrokecolor{currentstroke}%
\pgfsetdash{}{0pt}%
\pgfpathmoveto{\pgfqpoint{1.351761in}{0.417642in}}%
\pgfpathlineto{\pgfqpoint{1.351761in}{2.468330in}}%
\pgfusepath{stroke}%
\end{pgfscope}%
\begin{pgfscope}%
\pgfsetbuttcap%
\pgfsetroundjoin%
\definecolor{currentfill}{rgb}{0.000000,0.000000,0.000000}%
\pgfsetfillcolor{currentfill}%
\pgfsetlinewidth{0.602250pt}%
\definecolor{currentstroke}{rgb}{0.000000,0.000000,0.000000}%
\pgfsetstrokecolor{currentstroke}%
\pgfsetdash{}{0pt}%
\pgfsys@defobject{currentmarker}{\pgfqpoint{0.000000in}{-0.027778in}}{\pgfqpoint{0.000000in}{0.000000in}}{%
\pgfpathmoveto{\pgfqpoint{0.000000in}{0.000000in}}%
\pgfpathlineto{\pgfqpoint{0.000000in}{-0.027778in}}%
\pgfusepath{stroke,fill}%
}%
\begin{pgfscope}%
\pgfsys@transformshift{1.351761in}{0.417642in}%
\pgfsys@useobject{currentmarker}{}%
\end{pgfscope}%
\end{pgfscope}%
\begin{pgfscope}%
\pgfpathrectangle{\pgfqpoint{0.589510in}{0.417642in}}{\pgfqpoint{3.428820in}{2.050688in}}%
\pgfusepath{clip}%
\pgfsetrectcap%
\pgfsetroundjoin%
\pgfsetlinewidth{0.803000pt}%
\definecolor{currentstroke}{rgb}{0.850000,0.850000,0.850000}%
\pgfsetstrokecolor{currentstroke}%
\pgfsetdash{}{0pt}%
\pgfpathmoveto{\pgfqpoint{1.403931in}{0.417642in}}%
\pgfpathlineto{\pgfqpoint{1.403931in}{2.468330in}}%
\pgfusepath{stroke}%
\end{pgfscope}%
\begin{pgfscope}%
\pgfsetbuttcap%
\pgfsetroundjoin%
\definecolor{currentfill}{rgb}{0.000000,0.000000,0.000000}%
\pgfsetfillcolor{currentfill}%
\pgfsetlinewidth{0.602250pt}%
\definecolor{currentstroke}{rgb}{0.000000,0.000000,0.000000}%
\pgfsetstrokecolor{currentstroke}%
\pgfsetdash{}{0pt}%
\pgfsys@defobject{currentmarker}{\pgfqpoint{0.000000in}{-0.027778in}}{\pgfqpoint{0.000000in}{0.000000in}}{%
\pgfpathmoveto{\pgfqpoint{0.000000in}{0.000000in}}%
\pgfpathlineto{\pgfqpoint{0.000000in}{-0.027778in}}%
\pgfusepath{stroke,fill}%
}%
\begin{pgfscope}%
\pgfsys@transformshift{1.403931in}{0.417642in}%
\pgfsys@useobject{currentmarker}{}%
\end{pgfscope}%
\end{pgfscope}%
\begin{pgfscope}%
\pgfpathrectangle{\pgfqpoint{0.589510in}{0.417642in}}{\pgfqpoint{3.428820in}{2.050688in}}%
\pgfusepath{clip}%
\pgfsetrectcap%
\pgfsetroundjoin%
\pgfsetlinewidth{0.803000pt}%
\definecolor{currentstroke}{rgb}{0.850000,0.850000,0.850000}%
\pgfsetstrokecolor{currentstroke}%
\pgfsetdash{}{0pt}%
\pgfpathmoveto{\pgfqpoint{1.449123in}{0.417642in}}%
\pgfpathlineto{\pgfqpoint{1.449123in}{2.468330in}}%
\pgfusepath{stroke}%
\end{pgfscope}%
\begin{pgfscope}%
\pgfsetbuttcap%
\pgfsetroundjoin%
\definecolor{currentfill}{rgb}{0.000000,0.000000,0.000000}%
\pgfsetfillcolor{currentfill}%
\pgfsetlinewidth{0.602250pt}%
\definecolor{currentstroke}{rgb}{0.000000,0.000000,0.000000}%
\pgfsetstrokecolor{currentstroke}%
\pgfsetdash{}{0pt}%
\pgfsys@defobject{currentmarker}{\pgfqpoint{0.000000in}{-0.027778in}}{\pgfqpoint{0.000000in}{0.000000in}}{%
\pgfpathmoveto{\pgfqpoint{0.000000in}{0.000000in}}%
\pgfpathlineto{\pgfqpoint{0.000000in}{-0.027778in}}%
\pgfusepath{stroke,fill}%
}%
\begin{pgfscope}%
\pgfsys@transformshift{1.449123in}{0.417642in}%
\pgfsys@useobject{currentmarker}{}%
\end{pgfscope}%
\end{pgfscope}%
\begin{pgfscope}%
\pgfpathrectangle{\pgfqpoint{0.589510in}{0.417642in}}{\pgfqpoint{3.428820in}{2.050688in}}%
\pgfusepath{clip}%
\pgfsetrectcap%
\pgfsetroundjoin%
\pgfsetlinewidth{0.803000pt}%
\definecolor{currentstroke}{rgb}{0.850000,0.850000,0.850000}%
\pgfsetstrokecolor{currentstroke}%
\pgfsetdash{}{0pt}%
\pgfpathmoveto{\pgfqpoint{1.488985in}{0.417642in}}%
\pgfpathlineto{\pgfqpoint{1.488985in}{2.468330in}}%
\pgfusepath{stroke}%
\end{pgfscope}%
\begin{pgfscope}%
\pgfsetbuttcap%
\pgfsetroundjoin%
\definecolor{currentfill}{rgb}{0.000000,0.000000,0.000000}%
\pgfsetfillcolor{currentfill}%
\pgfsetlinewidth{0.602250pt}%
\definecolor{currentstroke}{rgb}{0.000000,0.000000,0.000000}%
\pgfsetstrokecolor{currentstroke}%
\pgfsetdash{}{0pt}%
\pgfsys@defobject{currentmarker}{\pgfqpoint{0.000000in}{-0.027778in}}{\pgfqpoint{0.000000in}{0.000000in}}{%
\pgfpathmoveto{\pgfqpoint{0.000000in}{0.000000in}}%
\pgfpathlineto{\pgfqpoint{0.000000in}{-0.027778in}}%
\pgfusepath{stroke,fill}%
}%
\begin{pgfscope}%
\pgfsys@transformshift{1.488985in}{0.417642in}%
\pgfsys@useobject{currentmarker}{}%
\end{pgfscope}%
\end{pgfscope}%
\begin{pgfscope}%
\pgfpathrectangle{\pgfqpoint{0.589510in}{0.417642in}}{\pgfqpoint{3.428820in}{2.050688in}}%
\pgfusepath{clip}%
\pgfsetrectcap%
\pgfsetroundjoin%
\pgfsetlinewidth{0.803000pt}%
\definecolor{currentstroke}{rgb}{0.850000,0.850000,0.850000}%
\pgfsetstrokecolor{currentstroke}%
\pgfsetdash{}{0pt}%
\pgfpathmoveto{\pgfqpoint{1.759228in}{0.417642in}}%
\pgfpathlineto{\pgfqpoint{1.759228in}{2.468330in}}%
\pgfusepath{stroke}%
\end{pgfscope}%
\begin{pgfscope}%
\pgfsetbuttcap%
\pgfsetroundjoin%
\definecolor{currentfill}{rgb}{0.000000,0.000000,0.000000}%
\pgfsetfillcolor{currentfill}%
\pgfsetlinewidth{0.602250pt}%
\definecolor{currentstroke}{rgb}{0.000000,0.000000,0.000000}%
\pgfsetstrokecolor{currentstroke}%
\pgfsetdash{}{0pt}%
\pgfsys@defobject{currentmarker}{\pgfqpoint{0.000000in}{-0.027778in}}{\pgfqpoint{0.000000in}{0.000000in}}{%
\pgfpathmoveto{\pgfqpoint{0.000000in}{0.000000in}}%
\pgfpathlineto{\pgfqpoint{0.000000in}{-0.027778in}}%
\pgfusepath{stroke,fill}%
}%
\begin{pgfscope}%
\pgfsys@transformshift{1.759228in}{0.417642in}%
\pgfsys@useobject{currentmarker}{}%
\end{pgfscope}%
\end{pgfscope}%
\begin{pgfscope}%
\pgfpathrectangle{\pgfqpoint{0.589510in}{0.417642in}}{\pgfqpoint{3.428820in}{2.050688in}}%
\pgfusepath{clip}%
\pgfsetrectcap%
\pgfsetroundjoin%
\pgfsetlinewidth{0.803000pt}%
\definecolor{currentstroke}{rgb}{0.850000,0.850000,0.850000}%
\pgfsetstrokecolor{currentstroke}%
\pgfsetdash{}{0pt}%
\pgfpathmoveto{\pgfqpoint{1.896452in}{0.417642in}}%
\pgfpathlineto{\pgfqpoint{1.896452in}{2.468330in}}%
\pgfusepath{stroke}%
\end{pgfscope}%
\begin{pgfscope}%
\pgfsetbuttcap%
\pgfsetroundjoin%
\definecolor{currentfill}{rgb}{0.000000,0.000000,0.000000}%
\pgfsetfillcolor{currentfill}%
\pgfsetlinewidth{0.602250pt}%
\definecolor{currentstroke}{rgb}{0.000000,0.000000,0.000000}%
\pgfsetstrokecolor{currentstroke}%
\pgfsetdash{}{0pt}%
\pgfsys@defobject{currentmarker}{\pgfqpoint{0.000000in}{-0.027778in}}{\pgfqpoint{0.000000in}{0.000000in}}{%
\pgfpathmoveto{\pgfqpoint{0.000000in}{0.000000in}}%
\pgfpathlineto{\pgfqpoint{0.000000in}{-0.027778in}}%
\pgfusepath{stroke,fill}%
}%
\begin{pgfscope}%
\pgfsys@transformshift{1.896452in}{0.417642in}%
\pgfsys@useobject{currentmarker}{}%
\end{pgfscope}%
\end{pgfscope}%
\begin{pgfscope}%
\pgfpathrectangle{\pgfqpoint{0.589510in}{0.417642in}}{\pgfqpoint{3.428820in}{2.050688in}}%
\pgfusepath{clip}%
\pgfsetrectcap%
\pgfsetroundjoin%
\pgfsetlinewidth{0.803000pt}%
\definecolor{currentstroke}{rgb}{0.850000,0.850000,0.850000}%
\pgfsetstrokecolor{currentstroke}%
\pgfsetdash{}{0pt}%
\pgfpathmoveto{\pgfqpoint{1.993814in}{0.417642in}}%
\pgfpathlineto{\pgfqpoint{1.993814in}{2.468330in}}%
\pgfusepath{stroke}%
\end{pgfscope}%
\begin{pgfscope}%
\pgfsetbuttcap%
\pgfsetroundjoin%
\definecolor{currentfill}{rgb}{0.000000,0.000000,0.000000}%
\pgfsetfillcolor{currentfill}%
\pgfsetlinewidth{0.602250pt}%
\definecolor{currentstroke}{rgb}{0.000000,0.000000,0.000000}%
\pgfsetstrokecolor{currentstroke}%
\pgfsetdash{}{0pt}%
\pgfsys@defobject{currentmarker}{\pgfqpoint{0.000000in}{-0.027778in}}{\pgfqpoint{0.000000in}{0.000000in}}{%
\pgfpathmoveto{\pgfqpoint{0.000000in}{0.000000in}}%
\pgfpathlineto{\pgfqpoint{0.000000in}{-0.027778in}}%
\pgfusepath{stroke,fill}%
}%
\begin{pgfscope}%
\pgfsys@transformshift{1.993814in}{0.417642in}%
\pgfsys@useobject{currentmarker}{}%
\end{pgfscope}%
\end{pgfscope}%
\begin{pgfscope}%
\pgfpathrectangle{\pgfqpoint{0.589510in}{0.417642in}}{\pgfqpoint{3.428820in}{2.050688in}}%
\pgfusepath{clip}%
\pgfsetrectcap%
\pgfsetroundjoin%
\pgfsetlinewidth{0.803000pt}%
\definecolor{currentstroke}{rgb}{0.850000,0.850000,0.850000}%
\pgfsetstrokecolor{currentstroke}%
\pgfsetdash{}{0pt}%
\pgfpathmoveto{\pgfqpoint{2.069334in}{0.417642in}}%
\pgfpathlineto{\pgfqpoint{2.069334in}{2.468330in}}%
\pgfusepath{stroke}%
\end{pgfscope}%
\begin{pgfscope}%
\pgfsetbuttcap%
\pgfsetroundjoin%
\definecolor{currentfill}{rgb}{0.000000,0.000000,0.000000}%
\pgfsetfillcolor{currentfill}%
\pgfsetlinewidth{0.602250pt}%
\definecolor{currentstroke}{rgb}{0.000000,0.000000,0.000000}%
\pgfsetstrokecolor{currentstroke}%
\pgfsetdash{}{0pt}%
\pgfsys@defobject{currentmarker}{\pgfqpoint{0.000000in}{-0.027778in}}{\pgfqpoint{0.000000in}{0.000000in}}{%
\pgfpathmoveto{\pgfqpoint{0.000000in}{0.000000in}}%
\pgfpathlineto{\pgfqpoint{0.000000in}{-0.027778in}}%
\pgfusepath{stroke,fill}%
}%
\begin{pgfscope}%
\pgfsys@transformshift{2.069334in}{0.417642in}%
\pgfsys@useobject{currentmarker}{}%
\end{pgfscope}%
\end{pgfscope}%
\begin{pgfscope}%
\pgfpathrectangle{\pgfqpoint{0.589510in}{0.417642in}}{\pgfqpoint{3.428820in}{2.050688in}}%
\pgfusepath{clip}%
\pgfsetrectcap%
\pgfsetroundjoin%
\pgfsetlinewidth{0.803000pt}%
\definecolor{currentstroke}{rgb}{0.850000,0.850000,0.850000}%
\pgfsetstrokecolor{currentstroke}%
\pgfsetdash{}{0pt}%
\pgfpathmoveto{\pgfqpoint{2.131038in}{0.417642in}}%
\pgfpathlineto{\pgfqpoint{2.131038in}{2.468330in}}%
\pgfusepath{stroke}%
\end{pgfscope}%
\begin{pgfscope}%
\pgfsetbuttcap%
\pgfsetroundjoin%
\definecolor{currentfill}{rgb}{0.000000,0.000000,0.000000}%
\pgfsetfillcolor{currentfill}%
\pgfsetlinewidth{0.602250pt}%
\definecolor{currentstroke}{rgb}{0.000000,0.000000,0.000000}%
\pgfsetstrokecolor{currentstroke}%
\pgfsetdash{}{0pt}%
\pgfsys@defobject{currentmarker}{\pgfqpoint{0.000000in}{-0.027778in}}{\pgfqpoint{0.000000in}{0.000000in}}{%
\pgfpathmoveto{\pgfqpoint{0.000000in}{0.000000in}}%
\pgfpathlineto{\pgfqpoint{0.000000in}{-0.027778in}}%
\pgfusepath{stroke,fill}%
}%
\begin{pgfscope}%
\pgfsys@transformshift{2.131038in}{0.417642in}%
\pgfsys@useobject{currentmarker}{}%
\end{pgfscope}%
\end{pgfscope}%
\begin{pgfscope}%
\pgfpathrectangle{\pgfqpoint{0.589510in}{0.417642in}}{\pgfqpoint{3.428820in}{2.050688in}}%
\pgfusepath{clip}%
\pgfsetrectcap%
\pgfsetroundjoin%
\pgfsetlinewidth{0.803000pt}%
\definecolor{currentstroke}{rgb}{0.850000,0.850000,0.850000}%
\pgfsetstrokecolor{currentstroke}%
\pgfsetdash{}{0pt}%
\pgfpathmoveto{\pgfqpoint{2.183208in}{0.417642in}}%
\pgfpathlineto{\pgfqpoint{2.183208in}{2.468330in}}%
\pgfusepath{stroke}%
\end{pgfscope}%
\begin{pgfscope}%
\pgfsetbuttcap%
\pgfsetroundjoin%
\definecolor{currentfill}{rgb}{0.000000,0.000000,0.000000}%
\pgfsetfillcolor{currentfill}%
\pgfsetlinewidth{0.602250pt}%
\definecolor{currentstroke}{rgb}{0.000000,0.000000,0.000000}%
\pgfsetstrokecolor{currentstroke}%
\pgfsetdash{}{0pt}%
\pgfsys@defobject{currentmarker}{\pgfqpoint{0.000000in}{-0.027778in}}{\pgfqpoint{0.000000in}{0.000000in}}{%
\pgfpathmoveto{\pgfqpoint{0.000000in}{0.000000in}}%
\pgfpathlineto{\pgfqpoint{0.000000in}{-0.027778in}}%
\pgfusepath{stroke,fill}%
}%
\begin{pgfscope}%
\pgfsys@transformshift{2.183208in}{0.417642in}%
\pgfsys@useobject{currentmarker}{}%
\end{pgfscope}%
\end{pgfscope}%
\begin{pgfscope}%
\pgfpathrectangle{\pgfqpoint{0.589510in}{0.417642in}}{\pgfqpoint{3.428820in}{2.050688in}}%
\pgfusepath{clip}%
\pgfsetrectcap%
\pgfsetroundjoin%
\pgfsetlinewidth{0.803000pt}%
\definecolor{currentstroke}{rgb}{0.850000,0.850000,0.850000}%
\pgfsetstrokecolor{currentstroke}%
\pgfsetdash{}{0pt}%
\pgfpathmoveto{\pgfqpoint{2.228400in}{0.417642in}}%
\pgfpathlineto{\pgfqpoint{2.228400in}{2.468330in}}%
\pgfusepath{stroke}%
\end{pgfscope}%
\begin{pgfscope}%
\pgfsetbuttcap%
\pgfsetroundjoin%
\definecolor{currentfill}{rgb}{0.000000,0.000000,0.000000}%
\pgfsetfillcolor{currentfill}%
\pgfsetlinewidth{0.602250pt}%
\definecolor{currentstroke}{rgb}{0.000000,0.000000,0.000000}%
\pgfsetstrokecolor{currentstroke}%
\pgfsetdash{}{0pt}%
\pgfsys@defobject{currentmarker}{\pgfqpoint{0.000000in}{-0.027778in}}{\pgfqpoint{0.000000in}{0.000000in}}{%
\pgfpathmoveto{\pgfqpoint{0.000000in}{0.000000in}}%
\pgfpathlineto{\pgfqpoint{0.000000in}{-0.027778in}}%
\pgfusepath{stroke,fill}%
}%
\begin{pgfscope}%
\pgfsys@transformshift{2.228400in}{0.417642in}%
\pgfsys@useobject{currentmarker}{}%
\end{pgfscope}%
\end{pgfscope}%
\begin{pgfscope}%
\pgfpathrectangle{\pgfqpoint{0.589510in}{0.417642in}}{\pgfqpoint{3.428820in}{2.050688in}}%
\pgfusepath{clip}%
\pgfsetrectcap%
\pgfsetroundjoin%
\pgfsetlinewidth{0.803000pt}%
\definecolor{currentstroke}{rgb}{0.850000,0.850000,0.850000}%
\pgfsetstrokecolor{currentstroke}%
\pgfsetdash{}{0pt}%
\pgfpathmoveto{\pgfqpoint{2.268262in}{0.417642in}}%
\pgfpathlineto{\pgfqpoint{2.268262in}{2.468330in}}%
\pgfusepath{stroke}%
\end{pgfscope}%
\begin{pgfscope}%
\pgfsetbuttcap%
\pgfsetroundjoin%
\definecolor{currentfill}{rgb}{0.000000,0.000000,0.000000}%
\pgfsetfillcolor{currentfill}%
\pgfsetlinewidth{0.602250pt}%
\definecolor{currentstroke}{rgb}{0.000000,0.000000,0.000000}%
\pgfsetstrokecolor{currentstroke}%
\pgfsetdash{}{0pt}%
\pgfsys@defobject{currentmarker}{\pgfqpoint{0.000000in}{-0.027778in}}{\pgfqpoint{0.000000in}{0.000000in}}{%
\pgfpathmoveto{\pgfqpoint{0.000000in}{0.000000in}}%
\pgfpathlineto{\pgfqpoint{0.000000in}{-0.027778in}}%
\pgfusepath{stroke,fill}%
}%
\begin{pgfscope}%
\pgfsys@transformshift{2.268262in}{0.417642in}%
\pgfsys@useobject{currentmarker}{}%
\end{pgfscope}%
\end{pgfscope}%
\begin{pgfscope}%
\pgfpathrectangle{\pgfqpoint{0.589510in}{0.417642in}}{\pgfqpoint{3.428820in}{2.050688in}}%
\pgfusepath{clip}%
\pgfsetrectcap%
\pgfsetroundjoin%
\pgfsetlinewidth{0.803000pt}%
\definecolor{currentstroke}{rgb}{0.850000,0.850000,0.850000}%
\pgfsetstrokecolor{currentstroke}%
\pgfsetdash{}{0pt}%
\pgfpathmoveto{\pgfqpoint{2.538506in}{0.417642in}}%
\pgfpathlineto{\pgfqpoint{2.538506in}{2.468330in}}%
\pgfusepath{stroke}%
\end{pgfscope}%
\begin{pgfscope}%
\pgfsetbuttcap%
\pgfsetroundjoin%
\definecolor{currentfill}{rgb}{0.000000,0.000000,0.000000}%
\pgfsetfillcolor{currentfill}%
\pgfsetlinewidth{0.602250pt}%
\definecolor{currentstroke}{rgb}{0.000000,0.000000,0.000000}%
\pgfsetstrokecolor{currentstroke}%
\pgfsetdash{}{0pt}%
\pgfsys@defobject{currentmarker}{\pgfqpoint{0.000000in}{-0.027778in}}{\pgfqpoint{0.000000in}{0.000000in}}{%
\pgfpathmoveto{\pgfqpoint{0.000000in}{0.000000in}}%
\pgfpathlineto{\pgfqpoint{0.000000in}{-0.027778in}}%
\pgfusepath{stroke,fill}%
}%
\begin{pgfscope}%
\pgfsys@transformshift{2.538506in}{0.417642in}%
\pgfsys@useobject{currentmarker}{}%
\end{pgfscope}%
\end{pgfscope}%
\begin{pgfscope}%
\pgfpathrectangle{\pgfqpoint{0.589510in}{0.417642in}}{\pgfqpoint{3.428820in}{2.050688in}}%
\pgfusepath{clip}%
\pgfsetrectcap%
\pgfsetroundjoin%
\pgfsetlinewidth{0.803000pt}%
\definecolor{currentstroke}{rgb}{0.850000,0.850000,0.850000}%
\pgfsetstrokecolor{currentstroke}%
\pgfsetdash{}{0pt}%
\pgfpathmoveto{\pgfqpoint{2.675730in}{0.417642in}}%
\pgfpathlineto{\pgfqpoint{2.675730in}{2.468330in}}%
\pgfusepath{stroke}%
\end{pgfscope}%
\begin{pgfscope}%
\pgfsetbuttcap%
\pgfsetroundjoin%
\definecolor{currentfill}{rgb}{0.000000,0.000000,0.000000}%
\pgfsetfillcolor{currentfill}%
\pgfsetlinewidth{0.602250pt}%
\definecolor{currentstroke}{rgb}{0.000000,0.000000,0.000000}%
\pgfsetstrokecolor{currentstroke}%
\pgfsetdash{}{0pt}%
\pgfsys@defobject{currentmarker}{\pgfqpoint{0.000000in}{-0.027778in}}{\pgfqpoint{0.000000in}{0.000000in}}{%
\pgfpathmoveto{\pgfqpoint{0.000000in}{0.000000in}}%
\pgfpathlineto{\pgfqpoint{0.000000in}{-0.027778in}}%
\pgfusepath{stroke,fill}%
}%
\begin{pgfscope}%
\pgfsys@transformshift{2.675730in}{0.417642in}%
\pgfsys@useobject{currentmarker}{}%
\end{pgfscope}%
\end{pgfscope}%
\begin{pgfscope}%
\pgfpathrectangle{\pgfqpoint{0.589510in}{0.417642in}}{\pgfqpoint{3.428820in}{2.050688in}}%
\pgfusepath{clip}%
\pgfsetrectcap%
\pgfsetroundjoin%
\pgfsetlinewidth{0.803000pt}%
\definecolor{currentstroke}{rgb}{0.850000,0.850000,0.850000}%
\pgfsetstrokecolor{currentstroke}%
\pgfsetdash{}{0pt}%
\pgfpathmoveto{\pgfqpoint{2.773092in}{0.417642in}}%
\pgfpathlineto{\pgfqpoint{2.773092in}{2.468330in}}%
\pgfusepath{stroke}%
\end{pgfscope}%
\begin{pgfscope}%
\pgfsetbuttcap%
\pgfsetroundjoin%
\definecolor{currentfill}{rgb}{0.000000,0.000000,0.000000}%
\pgfsetfillcolor{currentfill}%
\pgfsetlinewidth{0.602250pt}%
\definecolor{currentstroke}{rgb}{0.000000,0.000000,0.000000}%
\pgfsetstrokecolor{currentstroke}%
\pgfsetdash{}{0pt}%
\pgfsys@defobject{currentmarker}{\pgfqpoint{0.000000in}{-0.027778in}}{\pgfqpoint{0.000000in}{0.000000in}}{%
\pgfpathmoveto{\pgfqpoint{0.000000in}{0.000000in}}%
\pgfpathlineto{\pgfqpoint{0.000000in}{-0.027778in}}%
\pgfusepath{stroke,fill}%
}%
\begin{pgfscope}%
\pgfsys@transformshift{2.773092in}{0.417642in}%
\pgfsys@useobject{currentmarker}{}%
\end{pgfscope}%
\end{pgfscope}%
\begin{pgfscope}%
\pgfpathrectangle{\pgfqpoint{0.589510in}{0.417642in}}{\pgfqpoint{3.428820in}{2.050688in}}%
\pgfusepath{clip}%
\pgfsetrectcap%
\pgfsetroundjoin%
\pgfsetlinewidth{0.803000pt}%
\definecolor{currentstroke}{rgb}{0.850000,0.850000,0.850000}%
\pgfsetstrokecolor{currentstroke}%
\pgfsetdash{}{0pt}%
\pgfpathmoveto{\pgfqpoint{2.848611in}{0.417642in}}%
\pgfpathlineto{\pgfqpoint{2.848611in}{2.468330in}}%
\pgfusepath{stroke}%
\end{pgfscope}%
\begin{pgfscope}%
\pgfsetbuttcap%
\pgfsetroundjoin%
\definecolor{currentfill}{rgb}{0.000000,0.000000,0.000000}%
\pgfsetfillcolor{currentfill}%
\pgfsetlinewidth{0.602250pt}%
\definecolor{currentstroke}{rgb}{0.000000,0.000000,0.000000}%
\pgfsetstrokecolor{currentstroke}%
\pgfsetdash{}{0pt}%
\pgfsys@defobject{currentmarker}{\pgfqpoint{0.000000in}{-0.027778in}}{\pgfqpoint{0.000000in}{0.000000in}}{%
\pgfpathmoveto{\pgfqpoint{0.000000in}{0.000000in}}%
\pgfpathlineto{\pgfqpoint{0.000000in}{-0.027778in}}%
\pgfusepath{stroke,fill}%
}%
\begin{pgfscope}%
\pgfsys@transformshift{2.848611in}{0.417642in}%
\pgfsys@useobject{currentmarker}{}%
\end{pgfscope}%
\end{pgfscope}%
\begin{pgfscope}%
\pgfpathrectangle{\pgfqpoint{0.589510in}{0.417642in}}{\pgfqpoint{3.428820in}{2.050688in}}%
\pgfusepath{clip}%
\pgfsetrectcap%
\pgfsetroundjoin%
\pgfsetlinewidth{0.803000pt}%
\definecolor{currentstroke}{rgb}{0.850000,0.850000,0.850000}%
\pgfsetstrokecolor{currentstroke}%
\pgfsetdash{}{0pt}%
\pgfpathmoveto{\pgfqpoint{2.910315in}{0.417642in}}%
\pgfpathlineto{\pgfqpoint{2.910315in}{2.468330in}}%
\pgfusepath{stroke}%
\end{pgfscope}%
\begin{pgfscope}%
\pgfsetbuttcap%
\pgfsetroundjoin%
\definecolor{currentfill}{rgb}{0.000000,0.000000,0.000000}%
\pgfsetfillcolor{currentfill}%
\pgfsetlinewidth{0.602250pt}%
\definecolor{currentstroke}{rgb}{0.000000,0.000000,0.000000}%
\pgfsetstrokecolor{currentstroke}%
\pgfsetdash{}{0pt}%
\pgfsys@defobject{currentmarker}{\pgfqpoint{0.000000in}{-0.027778in}}{\pgfqpoint{0.000000in}{0.000000in}}{%
\pgfpathmoveto{\pgfqpoint{0.000000in}{0.000000in}}%
\pgfpathlineto{\pgfqpoint{0.000000in}{-0.027778in}}%
\pgfusepath{stroke,fill}%
}%
\begin{pgfscope}%
\pgfsys@transformshift{2.910315in}{0.417642in}%
\pgfsys@useobject{currentmarker}{}%
\end{pgfscope}%
\end{pgfscope}%
\begin{pgfscope}%
\pgfpathrectangle{\pgfqpoint{0.589510in}{0.417642in}}{\pgfqpoint{3.428820in}{2.050688in}}%
\pgfusepath{clip}%
\pgfsetrectcap%
\pgfsetroundjoin%
\pgfsetlinewidth{0.803000pt}%
\definecolor{currentstroke}{rgb}{0.850000,0.850000,0.850000}%
\pgfsetstrokecolor{currentstroke}%
\pgfsetdash{}{0pt}%
\pgfpathmoveto{\pgfqpoint{2.962486in}{0.417642in}}%
\pgfpathlineto{\pgfqpoint{2.962486in}{2.468330in}}%
\pgfusepath{stroke}%
\end{pgfscope}%
\begin{pgfscope}%
\pgfsetbuttcap%
\pgfsetroundjoin%
\definecolor{currentfill}{rgb}{0.000000,0.000000,0.000000}%
\pgfsetfillcolor{currentfill}%
\pgfsetlinewidth{0.602250pt}%
\definecolor{currentstroke}{rgb}{0.000000,0.000000,0.000000}%
\pgfsetstrokecolor{currentstroke}%
\pgfsetdash{}{0pt}%
\pgfsys@defobject{currentmarker}{\pgfqpoint{0.000000in}{-0.027778in}}{\pgfqpoint{0.000000in}{0.000000in}}{%
\pgfpathmoveto{\pgfqpoint{0.000000in}{0.000000in}}%
\pgfpathlineto{\pgfqpoint{0.000000in}{-0.027778in}}%
\pgfusepath{stroke,fill}%
}%
\begin{pgfscope}%
\pgfsys@transformshift{2.962486in}{0.417642in}%
\pgfsys@useobject{currentmarker}{}%
\end{pgfscope}%
\end{pgfscope}%
\begin{pgfscope}%
\pgfpathrectangle{\pgfqpoint{0.589510in}{0.417642in}}{\pgfqpoint{3.428820in}{2.050688in}}%
\pgfusepath{clip}%
\pgfsetrectcap%
\pgfsetroundjoin%
\pgfsetlinewidth{0.803000pt}%
\definecolor{currentstroke}{rgb}{0.850000,0.850000,0.850000}%
\pgfsetstrokecolor{currentstroke}%
\pgfsetdash{}{0pt}%
\pgfpathmoveto{\pgfqpoint{3.007677in}{0.417642in}}%
\pgfpathlineto{\pgfqpoint{3.007677in}{2.468330in}}%
\pgfusepath{stroke}%
\end{pgfscope}%
\begin{pgfscope}%
\pgfsetbuttcap%
\pgfsetroundjoin%
\definecolor{currentfill}{rgb}{0.000000,0.000000,0.000000}%
\pgfsetfillcolor{currentfill}%
\pgfsetlinewidth{0.602250pt}%
\definecolor{currentstroke}{rgb}{0.000000,0.000000,0.000000}%
\pgfsetstrokecolor{currentstroke}%
\pgfsetdash{}{0pt}%
\pgfsys@defobject{currentmarker}{\pgfqpoint{0.000000in}{-0.027778in}}{\pgfqpoint{0.000000in}{0.000000in}}{%
\pgfpathmoveto{\pgfqpoint{0.000000in}{0.000000in}}%
\pgfpathlineto{\pgfqpoint{0.000000in}{-0.027778in}}%
\pgfusepath{stroke,fill}%
}%
\begin{pgfscope}%
\pgfsys@transformshift{3.007677in}{0.417642in}%
\pgfsys@useobject{currentmarker}{}%
\end{pgfscope}%
\end{pgfscope}%
\begin{pgfscope}%
\pgfpathrectangle{\pgfqpoint{0.589510in}{0.417642in}}{\pgfqpoint{3.428820in}{2.050688in}}%
\pgfusepath{clip}%
\pgfsetrectcap%
\pgfsetroundjoin%
\pgfsetlinewidth{0.803000pt}%
\definecolor{currentstroke}{rgb}{0.850000,0.850000,0.850000}%
\pgfsetstrokecolor{currentstroke}%
\pgfsetdash{}{0pt}%
\pgfpathmoveto{\pgfqpoint{3.047539in}{0.417642in}}%
\pgfpathlineto{\pgfqpoint{3.047539in}{2.468330in}}%
\pgfusepath{stroke}%
\end{pgfscope}%
\begin{pgfscope}%
\pgfsetbuttcap%
\pgfsetroundjoin%
\definecolor{currentfill}{rgb}{0.000000,0.000000,0.000000}%
\pgfsetfillcolor{currentfill}%
\pgfsetlinewidth{0.602250pt}%
\definecolor{currentstroke}{rgb}{0.000000,0.000000,0.000000}%
\pgfsetstrokecolor{currentstroke}%
\pgfsetdash{}{0pt}%
\pgfsys@defobject{currentmarker}{\pgfqpoint{0.000000in}{-0.027778in}}{\pgfqpoint{0.000000in}{0.000000in}}{%
\pgfpathmoveto{\pgfqpoint{0.000000in}{0.000000in}}%
\pgfpathlineto{\pgfqpoint{0.000000in}{-0.027778in}}%
\pgfusepath{stroke,fill}%
}%
\begin{pgfscope}%
\pgfsys@transformshift{3.047539in}{0.417642in}%
\pgfsys@useobject{currentmarker}{}%
\end{pgfscope}%
\end{pgfscope}%
\begin{pgfscope}%
\pgfpathrectangle{\pgfqpoint{0.589510in}{0.417642in}}{\pgfqpoint{3.428820in}{2.050688in}}%
\pgfusepath{clip}%
\pgfsetrectcap%
\pgfsetroundjoin%
\pgfsetlinewidth{0.803000pt}%
\definecolor{currentstroke}{rgb}{0.850000,0.850000,0.850000}%
\pgfsetstrokecolor{currentstroke}%
\pgfsetdash{}{0pt}%
\pgfpathmoveto{\pgfqpoint{3.317783in}{0.417642in}}%
\pgfpathlineto{\pgfqpoint{3.317783in}{2.468330in}}%
\pgfusepath{stroke}%
\end{pgfscope}%
\begin{pgfscope}%
\pgfsetbuttcap%
\pgfsetroundjoin%
\definecolor{currentfill}{rgb}{0.000000,0.000000,0.000000}%
\pgfsetfillcolor{currentfill}%
\pgfsetlinewidth{0.602250pt}%
\definecolor{currentstroke}{rgb}{0.000000,0.000000,0.000000}%
\pgfsetstrokecolor{currentstroke}%
\pgfsetdash{}{0pt}%
\pgfsys@defobject{currentmarker}{\pgfqpoint{0.000000in}{-0.027778in}}{\pgfqpoint{0.000000in}{0.000000in}}{%
\pgfpathmoveto{\pgfqpoint{0.000000in}{0.000000in}}%
\pgfpathlineto{\pgfqpoint{0.000000in}{-0.027778in}}%
\pgfusepath{stroke,fill}%
}%
\begin{pgfscope}%
\pgfsys@transformshift{3.317783in}{0.417642in}%
\pgfsys@useobject{currentmarker}{}%
\end{pgfscope}%
\end{pgfscope}%
\begin{pgfscope}%
\pgfpathrectangle{\pgfqpoint{0.589510in}{0.417642in}}{\pgfqpoint{3.428820in}{2.050688in}}%
\pgfusepath{clip}%
\pgfsetrectcap%
\pgfsetroundjoin%
\pgfsetlinewidth{0.803000pt}%
\definecolor{currentstroke}{rgb}{0.850000,0.850000,0.850000}%
\pgfsetstrokecolor{currentstroke}%
\pgfsetdash{}{0pt}%
\pgfpathmoveto{\pgfqpoint{3.455007in}{0.417642in}}%
\pgfpathlineto{\pgfqpoint{3.455007in}{2.468330in}}%
\pgfusepath{stroke}%
\end{pgfscope}%
\begin{pgfscope}%
\pgfsetbuttcap%
\pgfsetroundjoin%
\definecolor{currentfill}{rgb}{0.000000,0.000000,0.000000}%
\pgfsetfillcolor{currentfill}%
\pgfsetlinewidth{0.602250pt}%
\definecolor{currentstroke}{rgb}{0.000000,0.000000,0.000000}%
\pgfsetstrokecolor{currentstroke}%
\pgfsetdash{}{0pt}%
\pgfsys@defobject{currentmarker}{\pgfqpoint{0.000000in}{-0.027778in}}{\pgfqpoint{0.000000in}{0.000000in}}{%
\pgfpathmoveto{\pgfqpoint{0.000000in}{0.000000in}}%
\pgfpathlineto{\pgfqpoint{0.000000in}{-0.027778in}}%
\pgfusepath{stroke,fill}%
}%
\begin{pgfscope}%
\pgfsys@transformshift{3.455007in}{0.417642in}%
\pgfsys@useobject{currentmarker}{}%
\end{pgfscope}%
\end{pgfscope}%
\begin{pgfscope}%
\pgfpathrectangle{\pgfqpoint{0.589510in}{0.417642in}}{\pgfqpoint{3.428820in}{2.050688in}}%
\pgfusepath{clip}%
\pgfsetrectcap%
\pgfsetroundjoin%
\pgfsetlinewidth{0.803000pt}%
\definecolor{currentstroke}{rgb}{0.850000,0.850000,0.850000}%
\pgfsetstrokecolor{currentstroke}%
\pgfsetdash{}{0pt}%
\pgfpathmoveto{\pgfqpoint{3.552369in}{0.417642in}}%
\pgfpathlineto{\pgfqpoint{3.552369in}{2.468330in}}%
\pgfusepath{stroke}%
\end{pgfscope}%
\begin{pgfscope}%
\pgfsetbuttcap%
\pgfsetroundjoin%
\definecolor{currentfill}{rgb}{0.000000,0.000000,0.000000}%
\pgfsetfillcolor{currentfill}%
\pgfsetlinewidth{0.602250pt}%
\definecolor{currentstroke}{rgb}{0.000000,0.000000,0.000000}%
\pgfsetstrokecolor{currentstroke}%
\pgfsetdash{}{0pt}%
\pgfsys@defobject{currentmarker}{\pgfqpoint{0.000000in}{-0.027778in}}{\pgfqpoint{0.000000in}{0.000000in}}{%
\pgfpathmoveto{\pgfqpoint{0.000000in}{0.000000in}}%
\pgfpathlineto{\pgfqpoint{0.000000in}{-0.027778in}}%
\pgfusepath{stroke,fill}%
}%
\begin{pgfscope}%
\pgfsys@transformshift{3.552369in}{0.417642in}%
\pgfsys@useobject{currentmarker}{}%
\end{pgfscope}%
\end{pgfscope}%
\begin{pgfscope}%
\pgfpathrectangle{\pgfqpoint{0.589510in}{0.417642in}}{\pgfqpoint{3.428820in}{2.050688in}}%
\pgfusepath{clip}%
\pgfsetrectcap%
\pgfsetroundjoin%
\pgfsetlinewidth{0.803000pt}%
\definecolor{currentstroke}{rgb}{0.850000,0.850000,0.850000}%
\pgfsetstrokecolor{currentstroke}%
\pgfsetdash{}{0pt}%
\pgfpathmoveto{\pgfqpoint{3.627889in}{0.417642in}}%
\pgfpathlineto{\pgfqpoint{3.627889in}{2.468330in}}%
\pgfusepath{stroke}%
\end{pgfscope}%
\begin{pgfscope}%
\pgfsetbuttcap%
\pgfsetroundjoin%
\definecolor{currentfill}{rgb}{0.000000,0.000000,0.000000}%
\pgfsetfillcolor{currentfill}%
\pgfsetlinewidth{0.602250pt}%
\definecolor{currentstroke}{rgb}{0.000000,0.000000,0.000000}%
\pgfsetstrokecolor{currentstroke}%
\pgfsetdash{}{0pt}%
\pgfsys@defobject{currentmarker}{\pgfqpoint{0.000000in}{-0.027778in}}{\pgfqpoint{0.000000in}{0.000000in}}{%
\pgfpathmoveto{\pgfqpoint{0.000000in}{0.000000in}}%
\pgfpathlineto{\pgfqpoint{0.000000in}{-0.027778in}}%
\pgfusepath{stroke,fill}%
}%
\begin{pgfscope}%
\pgfsys@transformshift{3.627889in}{0.417642in}%
\pgfsys@useobject{currentmarker}{}%
\end{pgfscope}%
\end{pgfscope}%
\begin{pgfscope}%
\pgfpathrectangle{\pgfqpoint{0.589510in}{0.417642in}}{\pgfqpoint{3.428820in}{2.050688in}}%
\pgfusepath{clip}%
\pgfsetrectcap%
\pgfsetroundjoin%
\pgfsetlinewidth{0.803000pt}%
\definecolor{currentstroke}{rgb}{0.850000,0.850000,0.850000}%
\pgfsetstrokecolor{currentstroke}%
\pgfsetdash{}{0pt}%
\pgfpathmoveto{\pgfqpoint{3.689593in}{0.417642in}}%
\pgfpathlineto{\pgfqpoint{3.689593in}{2.468330in}}%
\pgfusepath{stroke}%
\end{pgfscope}%
\begin{pgfscope}%
\pgfsetbuttcap%
\pgfsetroundjoin%
\definecolor{currentfill}{rgb}{0.000000,0.000000,0.000000}%
\pgfsetfillcolor{currentfill}%
\pgfsetlinewidth{0.602250pt}%
\definecolor{currentstroke}{rgb}{0.000000,0.000000,0.000000}%
\pgfsetstrokecolor{currentstroke}%
\pgfsetdash{}{0pt}%
\pgfsys@defobject{currentmarker}{\pgfqpoint{0.000000in}{-0.027778in}}{\pgfqpoint{0.000000in}{0.000000in}}{%
\pgfpathmoveto{\pgfqpoint{0.000000in}{0.000000in}}%
\pgfpathlineto{\pgfqpoint{0.000000in}{-0.027778in}}%
\pgfusepath{stroke,fill}%
}%
\begin{pgfscope}%
\pgfsys@transformshift{3.689593in}{0.417642in}%
\pgfsys@useobject{currentmarker}{}%
\end{pgfscope}%
\end{pgfscope}%
\begin{pgfscope}%
\pgfpathrectangle{\pgfqpoint{0.589510in}{0.417642in}}{\pgfqpoint{3.428820in}{2.050688in}}%
\pgfusepath{clip}%
\pgfsetrectcap%
\pgfsetroundjoin%
\pgfsetlinewidth{0.803000pt}%
\definecolor{currentstroke}{rgb}{0.850000,0.850000,0.850000}%
\pgfsetstrokecolor{currentstroke}%
\pgfsetdash{}{0pt}%
\pgfpathmoveto{\pgfqpoint{3.741763in}{0.417642in}}%
\pgfpathlineto{\pgfqpoint{3.741763in}{2.468330in}}%
\pgfusepath{stroke}%
\end{pgfscope}%
\begin{pgfscope}%
\pgfsetbuttcap%
\pgfsetroundjoin%
\definecolor{currentfill}{rgb}{0.000000,0.000000,0.000000}%
\pgfsetfillcolor{currentfill}%
\pgfsetlinewidth{0.602250pt}%
\definecolor{currentstroke}{rgb}{0.000000,0.000000,0.000000}%
\pgfsetstrokecolor{currentstroke}%
\pgfsetdash{}{0pt}%
\pgfsys@defobject{currentmarker}{\pgfqpoint{0.000000in}{-0.027778in}}{\pgfqpoint{0.000000in}{0.000000in}}{%
\pgfpathmoveto{\pgfqpoint{0.000000in}{0.000000in}}%
\pgfpathlineto{\pgfqpoint{0.000000in}{-0.027778in}}%
\pgfusepath{stroke,fill}%
}%
\begin{pgfscope}%
\pgfsys@transformshift{3.741763in}{0.417642in}%
\pgfsys@useobject{currentmarker}{}%
\end{pgfscope}%
\end{pgfscope}%
\begin{pgfscope}%
\pgfpathrectangle{\pgfqpoint{0.589510in}{0.417642in}}{\pgfqpoint{3.428820in}{2.050688in}}%
\pgfusepath{clip}%
\pgfsetrectcap%
\pgfsetroundjoin%
\pgfsetlinewidth{0.803000pt}%
\definecolor{currentstroke}{rgb}{0.850000,0.850000,0.850000}%
\pgfsetstrokecolor{currentstroke}%
\pgfsetdash{}{0pt}%
\pgfpathmoveto{\pgfqpoint{3.786955in}{0.417642in}}%
\pgfpathlineto{\pgfqpoint{3.786955in}{2.468330in}}%
\pgfusepath{stroke}%
\end{pgfscope}%
\begin{pgfscope}%
\pgfsetbuttcap%
\pgfsetroundjoin%
\definecolor{currentfill}{rgb}{0.000000,0.000000,0.000000}%
\pgfsetfillcolor{currentfill}%
\pgfsetlinewidth{0.602250pt}%
\definecolor{currentstroke}{rgb}{0.000000,0.000000,0.000000}%
\pgfsetstrokecolor{currentstroke}%
\pgfsetdash{}{0pt}%
\pgfsys@defobject{currentmarker}{\pgfqpoint{0.000000in}{-0.027778in}}{\pgfqpoint{0.000000in}{0.000000in}}{%
\pgfpathmoveto{\pgfqpoint{0.000000in}{0.000000in}}%
\pgfpathlineto{\pgfqpoint{0.000000in}{-0.027778in}}%
\pgfusepath{stroke,fill}%
}%
\begin{pgfscope}%
\pgfsys@transformshift{3.786955in}{0.417642in}%
\pgfsys@useobject{currentmarker}{}%
\end{pgfscope}%
\end{pgfscope}%
\begin{pgfscope}%
\pgfpathrectangle{\pgfqpoint{0.589510in}{0.417642in}}{\pgfqpoint{3.428820in}{2.050688in}}%
\pgfusepath{clip}%
\pgfsetrectcap%
\pgfsetroundjoin%
\pgfsetlinewidth{0.803000pt}%
\definecolor{currentstroke}{rgb}{0.850000,0.850000,0.850000}%
\pgfsetstrokecolor{currentstroke}%
\pgfsetdash{}{0pt}%
\pgfpathmoveto{\pgfqpoint{3.826817in}{0.417642in}}%
\pgfpathlineto{\pgfqpoint{3.826817in}{2.468330in}}%
\pgfusepath{stroke}%
\end{pgfscope}%
\begin{pgfscope}%
\pgfsetbuttcap%
\pgfsetroundjoin%
\definecolor{currentfill}{rgb}{0.000000,0.000000,0.000000}%
\pgfsetfillcolor{currentfill}%
\pgfsetlinewidth{0.602250pt}%
\definecolor{currentstroke}{rgb}{0.000000,0.000000,0.000000}%
\pgfsetstrokecolor{currentstroke}%
\pgfsetdash{}{0pt}%
\pgfsys@defobject{currentmarker}{\pgfqpoint{0.000000in}{-0.027778in}}{\pgfqpoint{0.000000in}{0.000000in}}{%
\pgfpathmoveto{\pgfqpoint{0.000000in}{0.000000in}}%
\pgfpathlineto{\pgfqpoint{0.000000in}{-0.027778in}}%
\pgfusepath{stroke,fill}%
}%
\begin{pgfscope}%
\pgfsys@transformshift{3.826817in}{0.417642in}%
\pgfsys@useobject{currentmarker}{}%
\end{pgfscope}%
\end{pgfscope}%
\begin{pgfscope}%
\definecolor{textcolor}{rgb}{0.000000,0.000000,0.000000}%
\pgfsetstrokecolor{textcolor}%
\pgfsetfillcolor{textcolor}%
\pgftext[x=2.303920in,y=0.165003in,,top]{\color{textcolor}\rmfamily\fontsize{10.000000}{12.000000}\selectfont \(\displaystyle \tau\) in \unit{\second}}%
\end{pgfscope}%
\begin{pgfscope}%
\pgfpathrectangle{\pgfqpoint{0.589510in}{0.417642in}}{\pgfqpoint{3.428820in}{2.050688in}}%
\pgfusepath{clip}%
\pgfsetrectcap%
\pgfsetroundjoin%
\pgfsetlinewidth{0.803000pt}%
\definecolor{currentstroke}{rgb}{0.450000,0.450000,0.450000}%
\pgfsetstrokecolor{currentstroke}%
\pgfsetdash{}{0pt}%
\pgfpathmoveto{\pgfqpoint{0.589510in}{1.278155in}}%
\pgfpathlineto{\pgfqpoint{4.018330in}{1.278155in}}%
\pgfusepath{stroke}%
\end{pgfscope}%
\begin{pgfscope}%
\pgfsetbuttcap%
\pgfsetroundjoin%
\definecolor{currentfill}{rgb}{0.000000,0.000000,0.000000}%
\pgfsetfillcolor{currentfill}%
\pgfsetlinewidth{0.803000pt}%
\definecolor{currentstroke}{rgb}{0.000000,0.000000,0.000000}%
\pgfsetstrokecolor{currentstroke}%
\pgfsetdash{}{0pt}%
\pgfsys@defobject{currentmarker}{\pgfqpoint{-0.048611in}{0.000000in}}{\pgfqpoint{-0.000000in}{0.000000in}}{%
\pgfpathmoveto{\pgfqpoint{-0.000000in}{0.000000in}}%
\pgfpathlineto{\pgfqpoint{-0.048611in}{0.000000in}}%
\pgfusepath{stroke,fill}%
}%
\begin{pgfscope}%
\pgfsys@transformshift{0.589510in}{1.278155in}%
\pgfsys@useobject{currentmarker}{}%
\end{pgfscope}%
\end{pgfscope}%
\begin{pgfscope}%
\definecolor{textcolor}{rgb}{0.000000,0.000000,0.000000}%
\pgfsetstrokecolor{textcolor}%
\pgfsetfillcolor{textcolor}%
\pgftext[x=0.236114in, y=1.239002in, left, base]{\color{textcolor}\rmfamily\fontsize{8.000000}{9.600000}\selectfont \(\displaystyle {10^{-7}}\)}%
\end{pgfscope}%
\begin{pgfscope}%
\pgfpathrectangle{\pgfqpoint{0.589510in}{0.417642in}}{\pgfqpoint{3.428820in}{2.050688in}}%
\pgfusepath{clip}%
\pgfsetrectcap%
\pgfsetroundjoin%
\pgfsetlinewidth{0.803000pt}%
\definecolor{currentstroke}{rgb}{0.450000,0.450000,0.450000}%
\pgfsetstrokecolor{currentstroke}%
\pgfsetdash{}{0pt}%
\pgfpathmoveto{\pgfqpoint{0.589510in}{2.281322in}}%
\pgfpathlineto{\pgfqpoint{4.018330in}{2.281322in}}%
\pgfusepath{stroke}%
\end{pgfscope}%
\begin{pgfscope}%
\pgfsetbuttcap%
\pgfsetroundjoin%
\definecolor{currentfill}{rgb}{0.000000,0.000000,0.000000}%
\pgfsetfillcolor{currentfill}%
\pgfsetlinewidth{0.803000pt}%
\definecolor{currentstroke}{rgb}{0.000000,0.000000,0.000000}%
\pgfsetstrokecolor{currentstroke}%
\pgfsetdash{}{0pt}%
\pgfsys@defobject{currentmarker}{\pgfqpoint{-0.048611in}{0.000000in}}{\pgfqpoint{-0.000000in}{0.000000in}}{%
\pgfpathmoveto{\pgfqpoint{-0.000000in}{0.000000in}}%
\pgfpathlineto{\pgfqpoint{-0.048611in}{0.000000in}}%
\pgfusepath{stroke,fill}%
}%
\begin{pgfscope}%
\pgfsys@transformshift{0.589510in}{2.281322in}%
\pgfsys@useobject{currentmarker}{}%
\end{pgfscope}%
\end{pgfscope}%
\begin{pgfscope}%
\definecolor{textcolor}{rgb}{0.000000,0.000000,0.000000}%
\pgfsetstrokecolor{textcolor}%
\pgfsetfillcolor{textcolor}%
\pgftext[x=0.236114in, y=2.242170in, left, base]{\color{textcolor}\rmfamily\fontsize{8.000000}{9.600000}\selectfont \(\displaystyle {10^{-6}}\)}%
\end{pgfscope}%
\begin{pgfscope}%
\pgfpathrectangle{\pgfqpoint{0.589510in}{0.417642in}}{\pgfqpoint{3.428820in}{2.050688in}}%
\pgfusepath{clip}%
\pgfsetrectcap%
\pgfsetroundjoin%
\pgfsetlinewidth{0.803000pt}%
\definecolor{currentstroke}{rgb}{0.850000,0.850000,0.850000}%
\pgfsetstrokecolor{currentstroke}%
\pgfsetdash{}{0pt}%
\pgfpathmoveto{\pgfqpoint{0.589510in}{0.576971in}}%
\pgfpathlineto{\pgfqpoint{4.018330in}{0.576971in}}%
\pgfusepath{stroke}%
\end{pgfscope}%
\begin{pgfscope}%
\pgfsetbuttcap%
\pgfsetroundjoin%
\definecolor{currentfill}{rgb}{0.000000,0.000000,0.000000}%
\pgfsetfillcolor{currentfill}%
\pgfsetlinewidth{0.602250pt}%
\definecolor{currentstroke}{rgb}{0.000000,0.000000,0.000000}%
\pgfsetstrokecolor{currentstroke}%
\pgfsetdash{}{0pt}%
\pgfsys@defobject{currentmarker}{\pgfqpoint{-0.027778in}{0.000000in}}{\pgfqpoint{-0.000000in}{0.000000in}}{%
\pgfpathmoveto{\pgfqpoint{-0.000000in}{0.000000in}}%
\pgfpathlineto{\pgfqpoint{-0.027778in}{0.000000in}}%
\pgfusepath{stroke,fill}%
}%
\begin{pgfscope}%
\pgfsys@transformshift{0.589510in}{0.576971in}%
\pgfsys@useobject{currentmarker}{}%
\end{pgfscope}%
\end{pgfscope}%
\begin{pgfscope}%
\pgfpathrectangle{\pgfqpoint{0.589510in}{0.417642in}}{\pgfqpoint{3.428820in}{2.050688in}}%
\pgfusepath{clip}%
\pgfsetrectcap%
\pgfsetroundjoin%
\pgfsetlinewidth{0.803000pt}%
\definecolor{currentstroke}{rgb}{0.850000,0.850000,0.850000}%
\pgfsetstrokecolor{currentstroke}%
\pgfsetdash{}{0pt}%
\pgfpathmoveto{\pgfqpoint{0.589510in}{0.753620in}}%
\pgfpathlineto{\pgfqpoint{4.018330in}{0.753620in}}%
\pgfusepath{stroke}%
\end{pgfscope}%
\begin{pgfscope}%
\pgfsetbuttcap%
\pgfsetroundjoin%
\definecolor{currentfill}{rgb}{0.000000,0.000000,0.000000}%
\pgfsetfillcolor{currentfill}%
\pgfsetlinewidth{0.602250pt}%
\definecolor{currentstroke}{rgb}{0.000000,0.000000,0.000000}%
\pgfsetstrokecolor{currentstroke}%
\pgfsetdash{}{0pt}%
\pgfsys@defobject{currentmarker}{\pgfqpoint{-0.027778in}{0.000000in}}{\pgfqpoint{-0.000000in}{0.000000in}}{%
\pgfpathmoveto{\pgfqpoint{-0.000000in}{0.000000in}}%
\pgfpathlineto{\pgfqpoint{-0.027778in}{0.000000in}}%
\pgfusepath{stroke,fill}%
}%
\begin{pgfscope}%
\pgfsys@transformshift{0.589510in}{0.753620in}%
\pgfsys@useobject{currentmarker}{}%
\end{pgfscope}%
\end{pgfscope}%
\begin{pgfscope}%
\pgfpathrectangle{\pgfqpoint{0.589510in}{0.417642in}}{\pgfqpoint{3.428820in}{2.050688in}}%
\pgfusepath{clip}%
\pgfsetrectcap%
\pgfsetroundjoin%
\pgfsetlinewidth{0.803000pt}%
\definecolor{currentstroke}{rgb}{0.850000,0.850000,0.850000}%
\pgfsetstrokecolor{currentstroke}%
\pgfsetdash{}{0pt}%
\pgfpathmoveto{\pgfqpoint{0.589510in}{0.878955in}}%
\pgfpathlineto{\pgfqpoint{4.018330in}{0.878955in}}%
\pgfusepath{stroke}%
\end{pgfscope}%
\begin{pgfscope}%
\pgfsetbuttcap%
\pgfsetroundjoin%
\definecolor{currentfill}{rgb}{0.000000,0.000000,0.000000}%
\pgfsetfillcolor{currentfill}%
\pgfsetlinewidth{0.602250pt}%
\definecolor{currentstroke}{rgb}{0.000000,0.000000,0.000000}%
\pgfsetstrokecolor{currentstroke}%
\pgfsetdash{}{0pt}%
\pgfsys@defobject{currentmarker}{\pgfqpoint{-0.027778in}{0.000000in}}{\pgfqpoint{-0.000000in}{0.000000in}}{%
\pgfpathmoveto{\pgfqpoint{-0.000000in}{0.000000in}}%
\pgfpathlineto{\pgfqpoint{-0.027778in}{0.000000in}}%
\pgfusepath{stroke,fill}%
}%
\begin{pgfscope}%
\pgfsys@transformshift{0.589510in}{0.878955in}%
\pgfsys@useobject{currentmarker}{}%
\end{pgfscope}%
\end{pgfscope}%
\begin{pgfscope}%
\pgfpathrectangle{\pgfqpoint{0.589510in}{0.417642in}}{\pgfqpoint{3.428820in}{2.050688in}}%
\pgfusepath{clip}%
\pgfsetrectcap%
\pgfsetroundjoin%
\pgfsetlinewidth{0.803000pt}%
\definecolor{currentstroke}{rgb}{0.850000,0.850000,0.850000}%
\pgfsetstrokecolor{currentstroke}%
\pgfsetdash{}{0pt}%
\pgfpathmoveto{\pgfqpoint{0.589510in}{0.976172in}}%
\pgfpathlineto{\pgfqpoint{4.018330in}{0.976172in}}%
\pgfusepath{stroke}%
\end{pgfscope}%
\begin{pgfscope}%
\pgfsetbuttcap%
\pgfsetroundjoin%
\definecolor{currentfill}{rgb}{0.000000,0.000000,0.000000}%
\pgfsetfillcolor{currentfill}%
\pgfsetlinewidth{0.602250pt}%
\definecolor{currentstroke}{rgb}{0.000000,0.000000,0.000000}%
\pgfsetstrokecolor{currentstroke}%
\pgfsetdash{}{0pt}%
\pgfsys@defobject{currentmarker}{\pgfqpoint{-0.027778in}{0.000000in}}{\pgfqpoint{-0.000000in}{0.000000in}}{%
\pgfpathmoveto{\pgfqpoint{-0.000000in}{0.000000in}}%
\pgfpathlineto{\pgfqpoint{-0.027778in}{0.000000in}}%
\pgfusepath{stroke,fill}%
}%
\begin{pgfscope}%
\pgfsys@transformshift{0.589510in}{0.976172in}%
\pgfsys@useobject{currentmarker}{}%
\end{pgfscope}%
\end{pgfscope}%
\begin{pgfscope}%
\pgfpathrectangle{\pgfqpoint{0.589510in}{0.417642in}}{\pgfqpoint{3.428820in}{2.050688in}}%
\pgfusepath{clip}%
\pgfsetrectcap%
\pgfsetroundjoin%
\pgfsetlinewidth{0.803000pt}%
\definecolor{currentstroke}{rgb}{0.850000,0.850000,0.850000}%
\pgfsetstrokecolor{currentstroke}%
\pgfsetdash{}{0pt}%
\pgfpathmoveto{\pgfqpoint{0.589510in}{1.055604in}}%
\pgfpathlineto{\pgfqpoint{4.018330in}{1.055604in}}%
\pgfusepath{stroke}%
\end{pgfscope}%
\begin{pgfscope}%
\pgfsetbuttcap%
\pgfsetroundjoin%
\definecolor{currentfill}{rgb}{0.000000,0.000000,0.000000}%
\pgfsetfillcolor{currentfill}%
\pgfsetlinewidth{0.602250pt}%
\definecolor{currentstroke}{rgb}{0.000000,0.000000,0.000000}%
\pgfsetstrokecolor{currentstroke}%
\pgfsetdash{}{0pt}%
\pgfsys@defobject{currentmarker}{\pgfqpoint{-0.027778in}{0.000000in}}{\pgfqpoint{-0.000000in}{0.000000in}}{%
\pgfpathmoveto{\pgfqpoint{-0.000000in}{0.000000in}}%
\pgfpathlineto{\pgfqpoint{-0.027778in}{0.000000in}}%
\pgfusepath{stroke,fill}%
}%
\begin{pgfscope}%
\pgfsys@transformshift{0.589510in}{1.055604in}%
\pgfsys@useobject{currentmarker}{}%
\end{pgfscope}%
\end{pgfscope}%
\begin{pgfscope}%
\pgfpathrectangle{\pgfqpoint{0.589510in}{0.417642in}}{\pgfqpoint{3.428820in}{2.050688in}}%
\pgfusepath{clip}%
\pgfsetrectcap%
\pgfsetroundjoin%
\pgfsetlinewidth{0.803000pt}%
\definecolor{currentstroke}{rgb}{0.850000,0.850000,0.850000}%
\pgfsetstrokecolor{currentstroke}%
\pgfsetdash{}{0pt}%
\pgfpathmoveto{\pgfqpoint{0.589510in}{1.122763in}}%
\pgfpathlineto{\pgfqpoint{4.018330in}{1.122763in}}%
\pgfusepath{stroke}%
\end{pgfscope}%
\begin{pgfscope}%
\pgfsetbuttcap%
\pgfsetroundjoin%
\definecolor{currentfill}{rgb}{0.000000,0.000000,0.000000}%
\pgfsetfillcolor{currentfill}%
\pgfsetlinewidth{0.602250pt}%
\definecolor{currentstroke}{rgb}{0.000000,0.000000,0.000000}%
\pgfsetstrokecolor{currentstroke}%
\pgfsetdash{}{0pt}%
\pgfsys@defobject{currentmarker}{\pgfqpoint{-0.027778in}{0.000000in}}{\pgfqpoint{-0.000000in}{0.000000in}}{%
\pgfpathmoveto{\pgfqpoint{-0.000000in}{0.000000in}}%
\pgfpathlineto{\pgfqpoint{-0.027778in}{0.000000in}}%
\pgfusepath{stroke,fill}%
}%
\begin{pgfscope}%
\pgfsys@transformshift{0.589510in}{1.122763in}%
\pgfsys@useobject{currentmarker}{}%
\end{pgfscope}%
\end{pgfscope}%
\begin{pgfscope}%
\pgfpathrectangle{\pgfqpoint{0.589510in}{0.417642in}}{\pgfqpoint{3.428820in}{2.050688in}}%
\pgfusepath{clip}%
\pgfsetrectcap%
\pgfsetroundjoin%
\pgfsetlinewidth{0.803000pt}%
\definecolor{currentstroke}{rgb}{0.850000,0.850000,0.850000}%
\pgfsetstrokecolor{currentstroke}%
\pgfsetdash{}{0pt}%
\pgfpathmoveto{\pgfqpoint{0.589510in}{1.180938in}}%
\pgfpathlineto{\pgfqpoint{4.018330in}{1.180938in}}%
\pgfusepath{stroke}%
\end{pgfscope}%
\begin{pgfscope}%
\pgfsetbuttcap%
\pgfsetroundjoin%
\definecolor{currentfill}{rgb}{0.000000,0.000000,0.000000}%
\pgfsetfillcolor{currentfill}%
\pgfsetlinewidth{0.602250pt}%
\definecolor{currentstroke}{rgb}{0.000000,0.000000,0.000000}%
\pgfsetstrokecolor{currentstroke}%
\pgfsetdash{}{0pt}%
\pgfsys@defobject{currentmarker}{\pgfqpoint{-0.027778in}{0.000000in}}{\pgfqpoint{-0.000000in}{0.000000in}}{%
\pgfpathmoveto{\pgfqpoint{-0.000000in}{0.000000in}}%
\pgfpathlineto{\pgfqpoint{-0.027778in}{0.000000in}}%
\pgfusepath{stroke,fill}%
}%
\begin{pgfscope}%
\pgfsys@transformshift{0.589510in}{1.180938in}%
\pgfsys@useobject{currentmarker}{}%
\end{pgfscope}%
\end{pgfscope}%
\begin{pgfscope}%
\pgfpathrectangle{\pgfqpoint{0.589510in}{0.417642in}}{\pgfqpoint{3.428820in}{2.050688in}}%
\pgfusepath{clip}%
\pgfsetrectcap%
\pgfsetroundjoin%
\pgfsetlinewidth{0.803000pt}%
\definecolor{currentstroke}{rgb}{0.850000,0.850000,0.850000}%
\pgfsetstrokecolor{currentstroke}%
\pgfsetdash{}{0pt}%
\pgfpathmoveto{\pgfqpoint{0.589510in}{1.232253in}}%
\pgfpathlineto{\pgfqpoint{4.018330in}{1.232253in}}%
\pgfusepath{stroke}%
\end{pgfscope}%
\begin{pgfscope}%
\pgfsetbuttcap%
\pgfsetroundjoin%
\definecolor{currentfill}{rgb}{0.000000,0.000000,0.000000}%
\pgfsetfillcolor{currentfill}%
\pgfsetlinewidth{0.602250pt}%
\definecolor{currentstroke}{rgb}{0.000000,0.000000,0.000000}%
\pgfsetstrokecolor{currentstroke}%
\pgfsetdash{}{0pt}%
\pgfsys@defobject{currentmarker}{\pgfqpoint{-0.027778in}{0.000000in}}{\pgfqpoint{-0.000000in}{0.000000in}}{%
\pgfpathmoveto{\pgfqpoint{-0.000000in}{0.000000in}}%
\pgfpathlineto{\pgfqpoint{-0.027778in}{0.000000in}}%
\pgfusepath{stroke,fill}%
}%
\begin{pgfscope}%
\pgfsys@transformshift{0.589510in}{1.232253in}%
\pgfsys@useobject{currentmarker}{}%
\end{pgfscope}%
\end{pgfscope}%
\begin{pgfscope}%
\pgfpathrectangle{\pgfqpoint{0.589510in}{0.417642in}}{\pgfqpoint{3.428820in}{2.050688in}}%
\pgfusepath{clip}%
\pgfsetrectcap%
\pgfsetroundjoin%
\pgfsetlinewidth{0.803000pt}%
\definecolor{currentstroke}{rgb}{0.850000,0.850000,0.850000}%
\pgfsetstrokecolor{currentstroke}%
\pgfsetdash{}{0pt}%
\pgfpathmoveto{\pgfqpoint{0.589510in}{1.580139in}}%
\pgfpathlineto{\pgfqpoint{4.018330in}{1.580139in}}%
\pgfusepath{stroke}%
\end{pgfscope}%
\begin{pgfscope}%
\pgfsetbuttcap%
\pgfsetroundjoin%
\definecolor{currentfill}{rgb}{0.000000,0.000000,0.000000}%
\pgfsetfillcolor{currentfill}%
\pgfsetlinewidth{0.602250pt}%
\definecolor{currentstroke}{rgb}{0.000000,0.000000,0.000000}%
\pgfsetstrokecolor{currentstroke}%
\pgfsetdash{}{0pt}%
\pgfsys@defobject{currentmarker}{\pgfqpoint{-0.027778in}{0.000000in}}{\pgfqpoint{-0.000000in}{0.000000in}}{%
\pgfpathmoveto{\pgfqpoint{-0.000000in}{0.000000in}}%
\pgfpathlineto{\pgfqpoint{-0.027778in}{0.000000in}}%
\pgfusepath{stroke,fill}%
}%
\begin{pgfscope}%
\pgfsys@transformshift{0.589510in}{1.580139in}%
\pgfsys@useobject{currentmarker}{}%
\end{pgfscope}%
\end{pgfscope}%
\begin{pgfscope}%
\pgfpathrectangle{\pgfqpoint{0.589510in}{0.417642in}}{\pgfqpoint{3.428820in}{2.050688in}}%
\pgfusepath{clip}%
\pgfsetrectcap%
\pgfsetroundjoin%
\pgfsetlinewidth{0.803000pt}%
\definecolor{currentstroke}{rgb}{0.850000,0.850000,0.850000}%
\pgfsetstrokecolor{currentstroke}%
\pgfsetdash{}{0pt}%
\pgfpathmoveto{\pgfqpoint{0.589510in}{1.756788in}}%
\pgfpathlineto{\pgfqpoint{4.018330in}{1.756788in}}%
\pgfusepath{stroke}%
\end{pgfscope}%
\begin{pgfscope}%
\pgfsetbuttcap%
\pgfsetroundjoin%
\definecolor{currentfill}{rgb}{0.000000,0.000000,0.000000}%
\pgfsetfillcolor{currentfill}%
\pgfsetlinewidth{0.602250pt}%
\definecolor{currentstroke}{rgb}{0.000000,0.000000,0.000000}%
\pgfsetstrokecolor{currentstroke}%
\pgfsetdash{}{0pt}%
\pgfsys@defobject{currentmarker}{\pgfqpoint{-0.027778in}{0.000000in}}{\pgfqpoint{-0.000000in}{0.000000in}}{%
\pgfpathmoveto{\pgfqpoint{-0.000000in}{0.000000in}}%
\pgfpathlineto{\pgfqpoint{-0.027778in}{0.000000in}}%
\pgfusepath{stroke,fill}%
}%
\begin{pgfscope}%
\pgfsys@transformshift{0.589510in}{1.756788in}%
\pgfsys@useobject{currentmarker}{}%
\end{pgfscope}%
\end{pgfscope}%
\begin{pgfscope}%
\pgfpathrectangle{\pgfqpoint{0.589510in}{0.417642in}}{\pgfqpoint{3.428820in}{2.050688in}}%
\pgfusepath{clip}%
\pgfsetrectcap%
\pgfsetroundjoin%
\pgfsetlinewidth{0.803000pt}%
\definecolor{currentstroke}{rgb}{0.850000,0.850000,0.850000}%
\pgfsetstrokecolor{currentstroke}%
\pgfsetdash{}{0pt}%
\pgfpathmoveto{\pgfqpoint{0.589510in}{1.882122in}}%
\pgfpathlineto{\pgfqpoint{4.018330in}{1.882122in}}%
\pgfusepath{stroke}%
\end{pgfscope}%
\begin{pgfscope}%
\pgfsetbuttcap%
\pgfsetroundjoin%
\definecolor{currentfill}{rgb}{0.000000,0.000000,0.000000}%
\pgfsetfillcolor{currentfill}%
\pgfsetlinewidth{0.602250pt}%
\definecolor{currentstroke}{rgb}{0.000000,0.000000,0.000000}%
\pgfsetstrokecolor{currentstroke}%
\pgfsetdash{}{0pt}%
\pgfsys@defobject{currentmarker}{\pgfqpoint{-0.027778in}{0.000000in}}{\pgfqpoint{-0.000000in}{0.000000in}}{%
\pgfpathmoveto{\pgfqpoint{-0.000000in}{0.000000in}}%
\pgfpathlineto{\pgfqpoint{-0.027778in}{0.000000in}}%
\pgfusepath{stroke,fill}%
}%
\begin{pgfscope}%
\pgfsys@transformshift{0.589510in}{1.882122in}%
\pgfsys@useobject{currentmarker}{}%
\end{pgfscope}%
\end{pgfscope}%
\begin{pgfscope}%
\pgfpathrectangle{\pgfqpoint{0.589510in}{0.417642in}}{\pgfqpoint{3.428820in}{2.050688in}}%
\pgfusepath{clip}%
\pgfsetrectcap%
\pgfsetroundjoin%
\pgfsetlinewidth{0.803000pt}%
\definecolor{currentstroke}{rgb}{0.850000,0.850000,0.850000}%
\pgfsetstrokecolor{currentstroke}%
\pgfsetdash{}{0pt}%
\pgfpathmoveto{\pgfqpoint{0.589510in}{1.979339in}}%
\pgfpathlineto{\pgfqpoint{4.018330in}{1.979339in}}%
\pgfusepath{stroke}%
\end{pgfscope}%
\begin{pgfscope}%
\pgfsetbuttcap%
\pgfsetroundjoin%
\definecolor{currentfill}{rgb}{0.000000,0.000000,0.000000}%
\pgfsetfillcolor{currentfill}%
\pgfsetlinewidth{0.602250pt}%
\definecolor{currentstroke}{rgb}{0.000000,0.000000,0.000000}%
\pgfsetstrokecolor{currentstroke}%
\pgfsetdash{}{0pt}%
\pgfsys@defobject{currentmarker}{\pgfqpoint{-0.027778in}{0.000000in}}{\pgfqpoint{-0.000000in}{0.000000in}}{%
\pgfpathmoveto{\pgfqpoint{-0.000000in}{0.000000in}}%
\pgfpathlineto{\pgfqpoint{-0.027778in}{0.000000in}}%
\pgfusepath{stroke,fill}%
}%
\begin{pgfscope}%
\pgfsys@transformshift{0.589510in}{1.979339in}%
\pgfsys@useobject{currentmarker}{}%
\end{pgfscope}%
\end{pgfscope}%
\begin{pgfscope}%
\pgfpathrectangle{\pgfqpoint{0.589510in}{0.417642in}}{\pgfqpoint{3.428820in}{2.050688in}}%
\pgfusepath{clip}%
\pgfsetrectcap%
\pgfsetroundjoin%
\pgfsetlinewidth{0.803000pt}%
\definecolor{currentstroke}{rgb}{0.850000,0.850000,0.850000}%
\pgfsetstrokecolor{currentstroke}%
\pgfsetdash{}{0pt}%
\pgfpathmoveto{\pgfqpoint{0.589510in}{2.058771in}}%
\pgfpathlineto{\pgfqpoint{4.018330in}{2.058771in}}%
\pgfusepath{stroke}%
\end{pgfscope}%
\begin{pgfscope}%
\pgfsetbuttcap%
\pgfsetroundjoin%
\definecolor{currentfill}{rgb}{0.000000,0.000000,0.000000}%
\pgfsetfillcolor{currentfill}%
\pgfsetlinewidth{0.602250pt}%
\definecolor{currentstroke}{rgb}{0.000000,0.000000,0.000000}%
\pgfsetstrokecolor{currentstroke}%
\pgfsetdash{}{0pt}%
\pgfsys@defobject{currentmarker}{\pgfqpoint{-0.027778in}{0.000000in}}{\pgfqpoint{-0.000000in}{0.000000in}}{%
\pgfpathmoveto{\pgfqpoint{-0.000000in}{0.000000in}}%
\pgfpathlineto{\pgfqpoint{-0.027778in}{0.000000in}}%
\pgfusepath{stroke,fill}%
}%
\begin{pgfscope}%
\pgfsys@transformshift{0.589510in}{2.058771in}%
\pgfsys@useobject{currentmarker}{}%
\end{pgfscope}%
\end{pgfscope}%
\begin{pgfscope}%
\pgfpathrectangle{\pgfqpoint{0.589510in}{0.417642in}}{\pgfqpoint{3.428820in}{2.050688in}}%
\pgfusepath{clip}%
\pgfsetrectcap%
\pgfsetroundjoin%
\pgfsetlinewidth{0.803000pt}%
\definecolor{currentstroke}{rgb}{0.850000,0.850000,0.850000}%
\pgfsetstrokecolor{currentstroke}%
\pgfsetdash{}{0pt}%
\pgfpathmoveto{\pgfqpoint{0.589510in}{2.125930in}}%
\pgfpathlineto{\pgfqpoint{4.018330in}{2.125930in}}%
\pgfusepath{stroke}%
\end{pgfscope}%
\begin{pgfscope}%
\pgfsetbuttcap%
\pgfsetroundjoin%
\definecolor{currentfill}{rgb}{0.000000,0.000000,0.000000}%
\pgfsetfillcolor{currentfill}%
\pgfsetlinewidth{0.602250pt}%
\definecolor{currentstroke}{rgb}{0.000000,0.000000,0.000000}%
\pgfsetstrokecolor{currentstroke}%
\pgfsetdash{}{0pt}%
\pgfsys@defobject{currentmarker}{\pgfqpoint{-0.027778in}{0.000000in}}{\pgfqpoint{-0.000000in}{0.000000in}}{%
\pgfpathmoveto{\pgfqpoint{-0.000000in}{0.000000in}}%
\pgfpathlineto{\pgfqpoint{-0.027778in}{0.000000in}}%
\pgfusepath{stroke,fill}%
}%
\begin{pgfscope}%
\pgfsys@transformshift{0.589510in}{2.125930in}%
\pgfsys@useobject{currentmarker}{}%
\end{pgfscope}%
\end{pgfscope}%
\begin{pgfscope}%
\pgfpathrectangle{\pgfqpoint{0.589510in}{0.417642in}}{\pgfqpoint{3.428820in}{2.050688in}}%
\pgfusepath{clip}%
\pgfsetrectcap%
\pgfsetroundjoin%
\pgfsetlinewidth{0.803000pt}%
\definecolor{currentstroke}{rgb}{0.850000,0.850000,0.850000}%
\pgfsetstrokecolor{currentstroke}%
\pgfsetdash{}{0pt}%
\pgfpathmoveto{\pgfqpoint{0.589510in}{2.184105in}}%
\pgfpathlineto{\pgfqpoint{4.018330in}{2.184105in}}%
\pgfusepath{stroke}%
\end{pgfscope}%
\begin{pgfscope}%
\pgfsetbuttcap%
\pgfsetroundjoin%
\definecolor{currentfill}{rgb}{0.000000,0.000000,0.000000}%
\pgfsetfillcolor{currentfill}%
\pgfsetlinewidth{0.602250pt}%
\definecolor{currentstroke}{rgb}{0.000000,0.000000,0.000000}%
\pgfsetstrokecolor{currentstroke}%
\pgfsetdash{}{0pt}%
\pgfsys@defobject{currentmarker}{\pgfqpoint{-0.027778in}{0.000000in}}{\pgfqpoint{-0.000000in}{0.000000in}}{%
\pgfpathmoveto{\pgfqpoint{-0.000000in}{0.000000in}}%
\pgfpathlineto{\pgfqpoint{-0.027778in}{0.000000in}}%
\pgfusepath{stroke,fill}%
}%
\begin{pgfscope}%
\pgfsys@transformshift{0.589510in}{2.184105in}%
\pgfsys@useobject{currentmarker}{}%
\end{pgfscope}%
\end{pgfscope}%
\begin{pgfscope}%
\pgfpathrectangle{\pgfqpoint{0.589510in}{0.417642in}}{\pgfqpoint{3.428820in}{2.050688in}}%
\pgfusepath{clip}%
\pgfsetrectcap%
\pgfsetroundjoin%
\pgfsetlinewidth{0.803000pt}%
\definecolor{currentstroke}{rgb}{0.850000,0.850000,0.850000}%
\pgfsetstrokecolor{currentstroke}%
\pgfsetdash{}{0pt}%
\pgfpathmoveto{\pgfqpoint{0.589510in}{2.235420in}}%
\pgfpathlineto{\pgfqpoint{4.018330in}{2.235420in}}%
\pgfusepath{stroke}%
\end{pgfscope}%
\begin{pgfscope}%
\pgfsetbuttcap%
\pgfsetroundjoin%
\definecolor{currentfill}{rgb}{0.000000,0.000000,0.000000}%
\pgfsetfillcolor{currentfill}%
\pgfsetlinewidth{0.602250pt}%
\definecolor{currentstroke}{rgb}{0.000000,0.000000,0.000000}%
\pgfsetstrokecolor{currentstroke}%
\pgfsetdash{}{0pt}%
\pgfsys@defobject{currentmarker}{\pgfqpoint{-0.027778in}{0.000000in}}{\pgfqpoint{-0.000000in}{0.000000in}}{%
\pgfpathmoveto{\pgfqpoint{-0.000000in}{0.000000in}}%
\pgfpathlineto{\pgfqpoint{-0.027778in}{0.000000in}}%
\pgfusepath{stroke,fill}%
}%
\begin{pgfscope}%
\pgfsys@transformshift{0.589510in}{2.235420in}%
\pgfsys@useobject{currentmarker}{}%
\end{pgfscope}%
\end{pgfscope}%
\begin{pgfscope}%
\definecolor{textcolor}{rgb}{0.000000,0.000000,0.000000}%
\pgfsetstrokecolor{textcolor}%
\pgfsetfillcolor{textcolor}%
\pgftext[x=0.180559in,y=1.442986in,,bottom,rotate=90.000000]{\color{textcolor}\rmfamily\fontsize{10.000000}{12.000000}\selectfont ADEV \(\displaystyle \sigma_A(\tau)\)}%
\end{pgfscope}%
\begin{pgfscope}%
\pgfpathrectangle{\pgfqpoint{0.589510in}{0.417642in}}{\pgfqpoint{3.428820in}{2.050688in}}%
\pgfusepath{clip}%
\pgfsetrectcap%
\pgfsetroundjoin%
\pgfsetlinewidth{1.505625pt}%
\definecolor{currentstroke}{rgb}{0.121569,0.466667,0.705882}%
\pgfsetstrokecolor{currentstroke}%
\pgfsetdash{}{0pt}%
\pgfpathmoveto{\pgfqpoint{0.979951in}{2.375117in}}%
\pgfpathlineto{\pgfqpoint{1.214537in}{2.222556in}}%
\pgfpathlineto{\pgfqpoint{1.351761in}{2.132710in}}%
\pgfpathlineto{\pgfqpoint{1.449123in}{2.068984in}}%
\pgfpathlineto{\pgfqpoint{1.524643in}{2.020110in}}%
\pgfpathlineto{\pgfqpoint{1.586347in}{1.980120in}}%
\pgfpathlineto{\pgfqpoint{1.683709in}{1.917580in}}%
\pgfpathlineto{\pgfqpoint{1.759228in}{1.868612in}}%
\pgfpathlineto{\pgfqpoint{1.820933in}{1.828915in}}%
\pgfpathlineto{\pgfqpoint{1.873103in}{1.794910in}}%
\pgfpathlineto{\pgfqpoint{1.938812in}{1.751927in}}%
\pgfpathlineto{\pgfqpoint{2.010327in}{1.705348in}}%
\pgfpathlineto{\pgfqpoint{2.069334in}{1.666798in}}%
\pgfpathlineto{\pgfqpoint{2.142135in}{1.619602in}}%
\pgfpathlineto{\pgfqpoint{2.202015in}{1.581055in}}%
\pgfpathlineto{\pgfqpoint{2.268262in}{1.538940in}}%
\pgfpathlineto{\pgfqpoint{2.329966in}{1.500262in}}%
\pgfpathlineto{\pgfqpoint{2.397880in}{1.457355in}}%
\pgfpathlineto{\pgfqpoint{2.458729in}{1.418998in}}%
\pgfpathlineto{\pgfqpoint{2.524690in}{1.377505in}}%
\pgfpathlineto{\pgfqpoint{2.588736in}{1.337005in}}%
\pgfpathlineto{\pgfqpoint{2.652380in}{1.296474in}}%
\pgfpathlineto{\pgfqpoint{2.716093in}{1.255259in}}%
\pgfpathlineto{\pgfqpoint{2.779793in}{1.213622in}}%
\pgfpathlineto{\pgfqpoint{2.844526in}{1.171818in}}%
\pgfpathlineto{\pgfqpoint{2.908052in}{1.132632in}}%
\pgfpathlineto{\pgfqpoint{2.971078in}{1.092825in}}%
\pgfpathlineto{\pgfqpoint{3.035287in}{1.051806in}}%
\pgfpathlineto{\pgfqpoint{3.099064in}{1.008828in}}%
\pgfpathlineto{\pgfqpoint{3.162486in}{0.963426in}}%
\pgfpathlineto{\pgfqpoint{3.226237in}{0.918537in}}%
\pgfpathlineto{\pgfqpoint{3.289931in}{0.876555in}}%
\pgfpathlineto{\pgfqpoint{3.353407in}{0.838101in}}%
\pgfpathlineto{\pgfqpoint{3.417085in}{0.800367in}}%
\pgfpathlineto{\pgfqpoint{3.480635in}{0.762053in}}%
\pgfpathlineto{\pgfqpoint{3.544321in}{0.724786in}}%
\pgfpathlineto{\pgfqpoint{3.607954in}{0.688225in}}%
\pgfpathlineto{\pgfqpoint{3.671520in}{0.650632in}}%
\pgfpathlineto{\pgfqpoint{3.735222in}{0.609517in}}%
\pgfpathlineto{\pgfqpoint{3.798843in}{0.561654in}}%
\pgfpathlineto{\pgfqpoint{3.862475in}{0.510855in}}%
\pgfusepath{stroke}%
\end{pgfscope}%
\begin{pgfscope}%
\pgfpathrectangle{\pgfqpoint{0.589510in}{0.417642in}}{\pgfqpoint{3.428820in}{2.050688in}}%
\pgfusepath{clip}%
\pgfsetrectcap%
\pgfsetroundjoin%
\pgfsetlinewidth{1.505625pt}%
\definecolor{currentstroke}{rgb}{1.000000,0.498039,0.054902}%
\pgfsetstrokecolor{currentstroke}%
\pgfsetdash{}{0pt}%
\pgfpathmoveto{\pgfqpoint{1.214537in}{2.240269in}}%
\pgfpathlineto{\pgfqpoint{1.449123in}{2.085354in}}%
\pgfpathlineto{\pgfqpoint{1.586347in}{1.994375in}}%
\pgfpathlineto{\pgfqpoint{1.683709in}{1.929436in}}%
\pgfpathlineto{\pgfqpoint{1.759228in}{1.879193in}}%
\pgfpathlineto{\pgfqpoint{1.820933in}{1.838468in}}%
\pgfpathlineto{\pgfqpoint{1.873103in}{1.803935in}}%
\pgfpathlineto{\pgfqpoint{1.918294in}{1.774372in}}%
\pgfpathlineto{\pgfqpoint{1.993814in}{1.725097in}}%
\pgfpathlineto{\pgfqpoint{2.055518in}{1.684793in}}%
\pgfpathlineto{\pgfqpoint{2.131038in}{1.635733in}}%
\pgfpathlineto{\pgfqpoint{2.192742in}{1.595764in}}%
\pgfpathlineto{\pgfqpoint{2.260656in}{1.551868in}}%
\pgfpathlineto{\pgfqpoint{2.329966in}{1.507503in}}%
\pgfpathlineto{\pgfqpoint{2.397880in}{1.463899in}}%
\pgfpathlineto{\pgfqpoint{2.454417in}{1.427445in}}%
\pgfpathlineto{\pgfqpoint{2.524690in}{1.381381in}}%
\pgfpathlineto{\pgfqpoint{2.588736in}{1.338633in}}%
\pgfpathlineto{\pgfqpoint{2.652380in}{1.296093in}}%
\pgfpathlineto{\pgfqpoint{2.714084in}{1.254395in}}%
\pgfpathlineto{\pgfqpoint{2.779793in}{1.210689in}}%
\pgfpathlineto{\pgfqpoint{2.843153in}{1.170768in}}%
\pgfpathlineto{\pgfqpoint{2.908052in}{1.129786in}}%
\pgfpathlineto{\pgfqpoint{2.970134in}{1.091358in}}%
\pgfpathlineto{\pgfqpoint{3.035287in}{1.050254in}}%
\pgfpathlineto{\pgfqpoint{3.099064in}{1.007434in}}%
\pgfpathlineto{\pgfqpoint{3.162486in}{0.967436in}}%
\pgfpathlineto{\pgfqpoint{3.225793in}{0.926937in}}%
\pgfpathlineto{\pgfqpoint{3.289564in}{0.886703in}}%
\pgfpathlineto{\pgfqpoint{3.353102in}{0.845938in}}%
\pgfpathlineto{\pgfqpoint{3.416833in}{0.804030in}}%
\pgfpathlineto{\pgfqpoint{3.480635in}{0.761844in}}%
\pgfpathlineto{\pgfqpoint{3.544147in}{0.718314in}}%
\pgfpathlineto{\pgfqpoint{3.607811in}{0.672329in}}%
\pgfpathlineto{\pgfqpoint{3.671520in}{0.627802in}}%
\pgfpathlineto{\pgfqpoint{3.735123in}{0.585864in}}%
\pgfpathlineto{\pgfqpoint{3.798761in}{0.549202in}}%
\pgfpathlineto{\pgfqpoint{3.862475in}{0.513493in}}%
\pgfusepath{stroke}%
\end{pgfscope}%
\begin{pgfscope}%
\pgfpathrectangle{\pgfqpoint{0.589510in}{0.417642in}}{\pgfqpoint{3.428820in}{2.050688in}}%
\pgfusepath{clip}%
\pgfsetrectcap%
\pgfsetroundjoin%
\pgfsetlinewidth{1.505625pt}%
\definecolor{currentstroke}{rgb}{0.172549,0.627451,0.172549}%
\pgfsetstrokecolor{currentstroke}%
\pgfsetdash{}{0pt}%
\pgfpathmoveto{\pgfqpoint{1.524643in}{2.083793in}}%
\pgfpathlineto{\pgfqpoint{1.759228in}{1.922646in}}%
\pgfpathlineto{\pgfqpoint{1.896452in}{1.829063in}}%
\pgfpathlineto{\pgfqpoint{1.993814in}{1.763103in}}%
\pgfpathlineto{\pgfqpoint{2.069334in}{1.712750in}}%
\pgfpathlineto{\pgfqpoint{2.131038in}{1.671906in}}%
\pgfpathlineto{\pgfqpoint{2.183208in}{1.637527in}}%
\pgfpathlineto{\pgfqpoint{2.268262in}{1.580371in}}%
\pgfpathlineto{\pgfqpoint{2.303920in}{1.556700in}}%
\pgfpathlineto{\pgfqpoint{2.392713in}{1.497375in}}%
\pgfpathlineto{\pgfqpoint{2.441144in}{1.465166in}}%
\pgfpathlineto{\pgfqpoint{2.521146in}{1.411614in}}%
\pgfpathlineto{\pgfqpoint{2.585806in}{1.367952in}}%
\pgfpathlineto{\pgfqpoint{2.652380in}{1.323221in}}%
\pgfpathlineto{\pgfqpoint{2.707986in}{1.286929in}}%
\pgfpathlineto{\pgfqpoint{2.773092in}{1.245214in}}%
\pgfpathlineto{\pgfqpoint{2.841774in}{1.201062in}}%
\pgfpathlineto{\pgfqpoint{2.904627in}{1.160838in}}%
\pgfpathlineto{\pgfqpoint{2.967286in}{1.120240in}}%
\pgfpathlineto{\pgfqpoint{3.032153in}{1.079762in}}%
\pgfpathlineto{\pgfqpoint{3.096471in}{1.038495in}}%
\pgfpathlineto{\pgfqpoint{3.161414in}{0.995285in}}%
\pgfpathlineto{\pgfqpoint{3.224904in}{0.950579in}}%
\pgfpathlineto{\pgfqpoint{3.289564in}{0.902799in}}%
\pgfpathlineto{\pgfqpoint{3.353102in}{0.859109in}}%
\pgfpathlineto{\pgfqpoint{3.416833in}{0.818226in}}%
\pgfpathlineto{\pgfqpoint{3.480007in}{0.779794in}}%
\pgfpathlineto{\pgfqpoint{3.543800in}{0.741619in}}%
\pgfpathlineto{\pgfqpoint{3.607667in}{0.697866in}}%
\pgfpathlineto{\pgfqpoint{3.671044in}{0.655941in}}%
\pgfpathlineto{\pgfqpoint{3.734926in}{0.615985in}}%
\pgfpathlineto{\pgfqpoint{3.798597in}{0.574889in}}%
\pgfpathlineto{\pgfqpoint{3.862475in}{0.533284in}}%
\pgfusepath{stroke}%
\end{pgfscope}%
\begin{pgfscope}%
\pgfpathrectangle{\pgfqpoint{0.589510in}{0.417642in}}{\pgfqpoint{3.428820in}{2.050688in}}%
\pgfusepath{clip}%
\pgfsetrectcap%
\pgfsetroundjoin%
\pgfsetlinewidth{1.505625pt}%
\definecolor{currentstroke}{rgb}{0.839216,0.152941,0.156863}%
\pgfsetstrokecolor{currentstroke}%
\pgfsetdash{}{0pt}%
\pgfpathmoveto{\pgfqpoint{1.759228in}{1.992968in}}%
\pgfpathlineto{\pgfqpoint{1.993814in}{1.825371in}}%
\pgfpathlineto{\pgfqpoint{2.131038in}{1.728027in}}%
\pgfpathlineto{\pgfqpoint{2.228400in}{1.660768in}}%
\pgfpathlineto{\pgfqpoint{2.303920in}{1.610179in}}%
\pgfpathlineto{\pgfqpoint{2.365624in}{1.568076in}}%
\pgfpathlineto{\pgfqpoint{2.417794in}{1.532972in}}%
\pgfpathlineto{\pgfqpoint{2.502848in}{1.476067in}}%
\pgfpathlineto{\pgfqpoint{2.570762in}{1.430346in}}%
\pgfpathlineto{\pgfqpoint{2.652380in}{1.375399in}}%
\pgfpathlineto{\pgfqpoint{2.697572in}{1.344883in}}%
\pgfpathlineto{\pgfqpoint{2.773092in}{1.294736in}}%
\pgfpathlineto{\pgfqpoint{2.834796in}{1.253997in}}%
\pgfpathlineto{\pgfqpoint{2.898842in}{1.212219in}}%
\pgfpathlineto{\pgfqpoint{2.962486in}{1.170977in}}%
\pgfpathlineto{\pgfqpoint{3.032153in}{1.126893in}}%
\pgfpathlineto{\pgfqpoint{3.096471in}{1.085988in}}%
\pgfpathlineto{\pgfqpoint{3.161414in}{1.044413in}}%
\pgfpathlineto{\pgfqpoint{3.224904in}{1.002373in}}%
\pgfpathlineto{\pgfqpoint{3.289564in}{0.959627in}}%
\pgfpathlineto{\pgfqpoint{3.353102in}{0.914048in}}%
\pgfpathlineto{\pgfqpoint{3.416833in}{0.866264in}}%
\pgfpathlineto{\pgfqpoint{3.478958in}{0.819126in}}%
\pgfpathlineto{\pgfqpoint{3.543800in}{0.770756in}}%
\pgfpathlineto{\pgfqpoint{3.606948in}{0.730012in}}%
\pgfpathlineto{\pgfqpoint{3.671044in}{0.690351in}}%
\pgfpathlineto{\pgfqpoint{3.734926in}{0.659054in}}%
\pgfpathlineto{\pgfqpoint{3.798597in}{0.623408in}}%
\pgfpathlineto{\pgfqpoint{3.862475in}{0.585598in}}%
\pgfusepath{stroke}%
\end{pgfscope}%
\begin{pgfscope}%
\pgfpathrectangle{\pgfqpoint{0.589510in}{0.417642in}}{\pgfqpoint{3.428820in}{2.050688in}}%
\pgfusepath{clip}%
\pgfsetrectcap%
\pgfsetroundjoin%
\pgfsetlinewidth{1.505625pt}%
\definecolor{currentstroke}{rgb}{0.580392,0.403922,0.741176}%
\pgfsetstrokecolor{currentstroke}%
\pgfsetdash{}{0pt}%
\pgfpathmoveto{\pgfqpoint{1.993814in}{1.925895in}}%
\pgfpathlineto{\pgfqpoint{2.228400in}{1.753695in}}%
\pgfpathlineto{\pgfqpoint{2.365624in}{1.653693in}}%
\pgfpathlineto{\pgfqpoint{2.462986in}{1.583685in}}%
\pgfpathlineto{\pgfqpoint{2.538506in}{1.531561in}}%
\pgfpathlineto{\pgfqpoint{2.652380in}{1.450529in}}%
\pgfpathlineto{\pgfqpoint{2.697572in}{1.419484in}}%
\pgfpathlineto{\pgfqpoint{2.773092in}{1.367909in}}%
\pgfpathlineto{\pgfqpoint{2.834796in}{1.326866in}}%
\pgfpathlineto{\pgfqpoint{2.886966in}{1.292635in}}%
\pgfpathlineto{\pgfqpoint{2.952675in}{1.251344in}}%
\pgfpathlineto{\pgfqpoint{3.024190in}{1.205871in}}%
\pgfpathlineto{\pgfqpoint{3.096471in}{1.159927in}}%
\pgfpathlineto{\pgfqpoint{3.155999in}{1.118765in}}%
\pgfpathlineto{\pgfqpoint{3.224904in}{1.072803in}}%
\pgfpathlineto{\pgfqpoint{3.289564in}{1.030425in}}%
\pgfpathlineto{\pgfqpoint{3.350039in}{0.989776in}}%
\pgfpathlineto{\pgfqpoint{3.416833in}{0.943834in}}%
\pgfpathlineto{\pgfqpoint{3.476849in}{0.901071in}}%
\pgfpathlineto{\pgfqpoint{3.542060in}{0.856391in}}%
\pgfpathlineto{\pgfqpoint{3.605505in}{0.816256in}}%
\pgfpathlineto{\pgfqpoint{3.671044in}{0.770716in}}%
\pgfpathlineto{\pgfqpoint{3.733937in}{0.724962in}}%
\pgfpathlineto{\pgfqpoint{3.798597in}{0.679742in}}%
\pgfpathlineto{\pgfqpoint{3.862475in}{0.637345in}}%
\pgfusepath{stroke}%
\end{pgfscope}%
\begin{pgfscope}%
\pgfpathrectangle{\pgfqpoint{0.589510in}{0.417642in}}{\pgfqpoint{3.428820in}{2.050688in}}%
\pgfusepath{clip}%
\pgfsetrectcap%
\pgfsetroundjoin%
\pgfsetlinewidth{1.505625pt}%
\definecolor{currentstroke}{rgb}{0.549020,0.337255,0.294118}%
\pgfsetstrokecolor{currentstroke}%
\pgfsetdash{}{0pt}%
\pgfpathmoveto{\pgfqpoint{2.303920in}{1.873531in}}%
\pgfpathlineto{\pgfqpoint{2.538506in}{1.693356in}}%
\pgfpathlineto{\pgfqpoint{2.675730in}{1.591473in}}%
\pgfpathlineto{\pgfqpoint{2.773092in}{1.522910in}}%
\pgfpathlineto{\pgfqpoint{2.848611in}{1.468820in}}%
\pgfpathlineto{\pgfqpoint{2.962486in}{1.380223in}}%
\pgfpathlineto{\pgfqpoint{3.007677in}{1.346924in}}%
\pgfpathlineto{\pgfqpoint{3.083197in}{1.288141in}}%
\pgfpathlineto{\pgfqpoint{3.144901in}{1.243164in}}%
\pgfpathlineto{\pgfqpoint{3.220421in}{1.190698in}}%
\pgfpathlineto{\pgfqpoint{3.282125in}{1.149962in}}%
\pgfpathlineto{\pgfqpoint{3.350039in}{1.106735in}}%
\pgfpathlineto{\pgfqpoint{3.406577in}{1.065758in}}%
\pgfpathlineto{\pgfqpoint{3.476849in}{1.014195in}}%
\pgfpathlineto{\pgfqpoint{3.543800in}{0.966556in}}%
\pgfpathlineto{\pgfqpoint{3.606948in}{0.924687in}}%
\pgfpathlineto{\pgfqpoint{3.666243in}{0.887864in}}%
\pgfpathlineto{\pgfqpoint{3.731953in}{0.841151in}}%
\pgfpathlineto{\pgfqpoint{3.795312in}{0.799190in}}%
\pgfpathlineto{\pgfqpoint{3.862475in}{0.756411in}}%
\pgfusepath{stroke}%
\end{pgfscope}%
\begin{pgfscope}%
\pgfpathrectangle{\pgfqpoint{0.589510in}{0.417642in}}{\pgfqpoint{3.428820in}{2.050688in}}%
\pgfusepath{clip}%
\pgfsetbuttcap%
\pgfsetroundjoin%
\pgfsetlinewidth{1.505625pt}%
\definecolor{currentstroke}{rgb}{0.890196,0.466667,0.760784}%
\pgfsetstrokecolor{currentstroke}%
\pgfsetdash{{5.550000pt}{2.400000pt}}{0.000000pt}%
\pgfpathmoveto{\pgfqpoint{0.745365in}{2.222570in}}%
\pgfpathlineto{\pgfqpoint{0.979951in}{2.084834in}}%
\pgfpathlineto{\pgfqpoint{1.117175in}{2.010220in}}%
\pgfpathlineto{\pgfqpoint{1.214537in}{1.960921in}}%
\pgfpathlineto{\pgfqpoint{1.290057in}{1.924980in}}%
\pgfpathlineto{\pgfqpoint{1.351761in}{1.897328in}}%
\pgfpathlineto{\pgfqpoint{1.403931in}{1.875283in}}%
\pgfpathlineto{\pgfqpoint{1.488985in}{1.841889in}}%
\pgfpathlineto{\pgfqpoint{1.556899in}{1.817597in}}%
\pgfpathlineto{\pgfqpoint{1.613436in}{1.799093in}}%
\pgfpathlineto{\pgfqpoint{1.683709in}{1.778206in}}%
\pgfpathlineto{\pgfqpoint{1.759228in}{1.758495in}}%
\pgfpathlineto{\pgfqpoint{1.820933in}{1.744216in}}%
\pgfpathlineto{\pgfqpoint{1.884979in}{1.730871in}}%
\pgfpathlineto{\pgfqpoint{1.948622in}{1.719070in}}%
\pgfpathlineto{\pgfqpoint{2.010327in}{1.708939in}}%
\pgfpathlineto{\pgfqpoint{2.076036in}{1.699696in}}%
\pgfpathlineto{\pgfqpoint{2.142135in}{1.691906in}}%
\pgfpathlineto{\pgfqpoint{2.206558in}{1.685376in}}%
\pgfpathlineto{\pgfqpoint{2.272002in}{1.679547in}}%
\pgfpathlineto{\pgfqpoint{2.333085in}{1.674914in}}%
\pgfpathlineto{\pgfqpoint{2.397880in}{1.670388in}}%
\pgfpathlineto{\pgfqpoint{2.460864in}{1.666573in}}%
\pgfpathlineto{\pgfqpoint{2.526448in}{1.663500in}}%
\pgfpathlineto{\pgfqpoint{2.588736in}{1.661233in}}%
\pgfpathlineto{\pgfqpoint{2.653587in}{1.659775in}}%
\pgfpathlineto{\pgfqpoint{2.717092in}{1.658648in}}%
\pgfpathlineto{\pgfqpoint{2.780622in}{1.657636in}}%
\pgfpathlineto{\pgfqpoint{2.844526in}{1.657176in}}%
\pgfpathlineto{\pgfqpoint{2.908052in}{1.657078in}}%
\pgfpathlineto{\pgfqpoint{2.971549in}{1.656483in}}%
\pgfpathlineto{\pgfqpoint{3.035287in}{1.655301in}}%
\pgfpathlineto{\pgfqpoint{3.099064in}{1.654076in}}%
\pgfpathlineto{\pgfqpoint{3.162486in}{1.653046in}}%
\pgfpathlineto{\pgfqpoint{3.226237in}{1.652506in}}%
\pgfpathlineto{\pgfqpoint{3.289931in}{1.652609in}}%
\pgfpathlineto{\pgfqpoint{3.353407in}{1.652533in}}%
\pgfpathlineto{\pgfqpoint{3.417085in}{1.652406in}}%
\pgfpathlineto{\pgfqpoint{3.480740in}{1.651375in}}%
\pgfpathlineto{\pgfqpoint{3.544321in}{1.649895in}}%
\pgfpathlineto{\pgfqpoint{3.607954in}{1.649227in}}%
\pgfpathlineto{\pgfqpoint{3.671580in}{1.648581in}}%
\pgfpathlineto{\pgfqpoint{3.735222in}{1.648575in}}%
\pgfpathlineto{\pgfqpoint{3.798843in}{1.649737in}}%
\pgfpathlineto{\pgfqpoint{3.862475in}{1.651719in}}%
\pgfusepath{stroke}%
\end{pgfscope}%
\begin{pgfscope}%
\pgfsetrectcap%
\pgfsetmiterjoin%
\pgfsetlinewidth{0.803000pt}%
\definecolor{currentstroke}{rgb}{0.000000,0.000000,0.000000}%
\pgfsetstrokecolor{currentstroke}%
\pgfsetdash{}{0pt}%
\pgfpathmoveto{\pgfqpoint{0.589510in}{0.417642in}}%
\pgfpathlineto{\pgfqpoint{0.589510in}{2.468330in}}%
\pgfusepath{stroke}%
\end{pgfscope}%
\begin{pgfscope}%
\pgfsetrectcap%
\pgfsetmiterjoin%
\pgfsetlinewidth{0.803000pt}%
\definecolor{currentstroke}{rgb}{0.000000,0.000000,0.000000}%
\pgfsetstrokecolor{currentstroke}%
\pgfsetdash{}{0pt}%
\pgfpathmoveto{\pgfqpoint{4.018330in}{0.417642in}}%
\pgfpathlineto{\pgfqpoint{4.018330in}{2.468330in}}%
\pgfusepath{stroke}%
\end{pgfscope}%
\begin{pgfscope}%
\pgfsetrectcap%
\pgfsetmiterjoin%
\pgfsetlinewidth{0.803000pt}%
\definecolor{currentstroke}{rgb}{0.000000,0.000000,0.000000}%
\pgfsetstrokecolor{currentstroke}%
\pgfsetdash{}{0pt}%
\pgfpathmoveto{\pgfqpoint{0.589510in}{0.417642in}}%
\pgfpathlineto{\pgfqpoint{4.018330in}{0.417642in}}%
\pgfusepath{stroke}%
\end{pgfscope}%
\begin{pgfscope}%
\pgfsetrectcap%
\pgfsetmiterjoin%
\pgfsetlinewidth{0.803000pt}%
\definecolor{currentstroke}{rgb}{0.000000,0.000000,0.000000}%
\pgfsetstrokecolor{currentstroke}%
\pgfsetdash{}{0pt}%
\pgfpathmoveto{\pgfqpoint{0.589510in}{2.468330in}}%
\pgfpathlineto{\pgfqpoint{4.018330in}{2.468330in}}%
\pgfusepath{stroke}%
\end{pgfscope}%
\begin{pgfscope}%
\pgfsetbuttcap%
\pgfsetmiterjoin%
\definecolor{currentfill}{rgb}{1.000000,1.000000,1.000000}%
\pgfsetfillcolor{currentfill}%
\pgfsetfillopacity{0.800000}%
\pgfsetlinewidth{1.003750pt}%
\definecolor{currentstroke}{rgb}{0.800000,0.800000,0.800000}%
\pgfsetstrokecolor{currentstroke}%
\pgfsetstrokeopacity{0.800000}%
\pgfsetdash{}{0pt}%
\pgfpathmoveto{\pgfqpoint{3.099885in}{1.450109in}}%
\pgfpathlineto{\pgfqpoint{3.940552in}{1.450109in}}%
\pgfpathquadraticcurveto{\pgfqpoint{3.962774in}{1.450109in}}{\pgfqpoint{3.962774in}{1.472331in}}%
\pgfpathlineto{\pgfqpoint{3.962774in}{2.390552in}}%
\pgfpathquadraticcurveto{\pgfqpoint{3.962774in}{2.412774in}}{\pgfqpoint{3.940552in}{2.412774in}}%
\pgfpathlineto{\pgfqpoint{3.099885in}{2.412774in}}%
\pgfpathquadraticcurveto{\pgfqpoint{3.077663in}{2.412774in}}{\pgfqpoint{3.077663in}{2.390552in}}%
\pgfpathlineto{\pgfqpoint{3.077663in}{1.472331in}}%
\pgfpathquadraticcurveto{\pgfqpoint{3.077663in}{1.450109in}}{\pgfqpoint{3.099885in}{1.450109in}}%
\pgfpathlineto{\pgfqpoint{3.099885in}{1.450109in}}%
\pgfpathclose%
\pgfusepath{stroke,fill}%
\end{pgfscope}%
\begin{pgfscope}%
\pgfsetrectcap%
\pgfsetroundjoin%
\pgfsetlinewidth{1.505625pt}%
\definecolor{currentstroke}{rgb}{0.121569,0.466667,0.705882}%
\pgfsetstrokecolor{currentstroke}%
\pgfsetdash{}{0pt}%
\pgfpathmoveto{\pgfqpoint{3.122108in}{2.329441in}}%
\pgfpathlineto{\pgfqpoint{3.233219in}{2.329441in}}%
\pgfpathlineto{\pgfqpoint{3.344330in}{2.329441in}}%
\pgfusepath{stroke}%
\end{pgfscope}%
\begin{pgfscope}%
\definecolor{textcolor}{rgb}{0.000000,0.000000,0.000000}%
\pgfsetstrokecolor{textcolor}%
\pgfsetfillcolor{textcolor}%
\pgftext[x=3.433219in,y=2.290552in,left,base]{\color{textcolor}\rmfamily\fontsize{8.000000}{9.600000}\selectfont NPLC 1}%
\end{pgfscope}%
\begin{pgfscope}%
\pgfsetrectcap%
\pgfsetroundjoin%
\pgfsetlinewidth{1.505625pt}%
\definecolor{currentstroke}{rgb}{1.000000,0.498039,0.054902}%
\pgfsetstrokecolor{currentstroke}%
\pgfsetdash{}{0pt}%
\pgfpathmoveto{\pgfqpoint{3.122108in}{2.174552in}}%
\pgfpathlineto{\pgfqpoint{3.233219in}{2.174552in}}%
\pgfpathlineto{\pgfqpoint{3.344330in}{2.174552in}}%
\pgfusepath{stroke}%
\end{pgfscope}%
\begin{pgfscope}%
\definecolor{textcolor}{rgb}{0.000000,0.000000,0.000000}%
\pgfsetstrokecolor{textcolor}%
\pgfsetfillcolor{textcolor}%
\pgftext[x=3.433219in,y=2.135663in,left,base]{\color{textcolor}\rmfamily\fontsize{8.000000}{9.600000}\selectfont NPLC 2}%
\end{pgfscope}%
\begin{pgfscope}%
\pgfsetrectcap%
\pgfsetroundjoin%
\pgfsetlinewidth{1.505625pt}%
\definecolor{currentstroke}{rgb}{0.172549,0.627451,0.172549}%
\pgfsetstrokecolor{currentstroke}%
\pgfsetdash{}{0pt}%
\pgfpathmoveto{\pgfqpoint{3.122108in}{2.019664in}}%
\pgfpathlineto{\pgfqpoint{3.233219in}{2.019664in}}%
\pgfpathlineto{\pgfqpoint{3.344330in}{2.019664in}}%
\pgfusepath{stroke}%
\end{pgfscope}%
\begin{pgfscope}%
\definecolor{textcolor}{rgb}{0.000000,0.000000,0.000000}%
\pgfsetstrokecolor{textcolor}%
\pgfsetfillcolor{textcolor}%
\pgftext[x=3.433219in,y=1.980775in,left,base]{\color{textcolor}\rmfamily\fontsize{8.000000}{9.600000}\selectfont NPLC 5}%
\end{pgfscope}%
\begin{pgfscope}%
\pgfsetrectcap%
\pgfsetroundjoin%
\pgfsetlinewidth{1.505625pt}%
\definecolor{currentstroke}{rgb}{0.839216,0.152941,0.156863}%
\pgfsetstrokecolor{currentstroke}%
\pgfsetdash{}{0pt}%
\pgfpathmoveto{\pgfqpoint{3.122108in}{1.864775in}}%
\pgfpathlineto{\pgfqpoint{3.233219in}{1.864775in}}%
\pgfpathlineto{\pgfqpoint{3.344330in}{1.864775in}}%
\pgfusepath{stroke}%
\end{pgfscope}%
\begin{pgfscope}%
\definecolor{textcolor}{rgb}{0.000000,0.000000,0.000000}%
\pgfsetstrokecolor{textcolor}%
\pgfsetfillcolor{textcolor}%
\pgftext[x=3.433219in,y=1.825886in,left,base]{\color{textcolor}\rmfamily\fontsize{8.000000}{9.600000}\selectfont NPLC 10}%
\end{pgfscope}%
\begin{pgfscope}%
\pgfsetrectcap%
\pgfsetroundjoin%
\pgfsetlinewidth{1.505625pt}%
\definecolor{currentstroke}{rgb}{0.580392,0.403922,0.741176}%
\pgfsetstrokecolor{currentstroke}%
\pgfsetdash{}{0pt}%
\pgfpathmoveto{\pgfqpoint{3.122108in}{1.709886in}}%
\pgfpathlineto{\pgfqpoint{3.233219in}{1.709886in}}%
\pgfpathlineto{\pgfqpoint{3.344330in}{1.709886in}}%
\pgfusepath{stroke}%
\end{pgfscope}%
\begin{pgfscope}%
\definecolor{textcolor}{rgb}{0.000000,0.000000,0.000000}%
\pgfsetstrokecolor{textcolor}%
\pgfsetfillcolor{textcolor}%
\pgftext[x=3.433219in,y=1.670997in,left,base]{\color{textcolor}\rmfamily\fontsize{8.000000}{9.600000}\selectfont NPLC 20}%
\end{pgfscope}%
\begin{pgfscope}%
\pgfsetrectcap%
\pgfsetroundjoin%
\pgfsetlinewidth{1.505625pt}%
\definecolor{currentstroke}{rgb}{0.549020,0.337255,0.294118}%
\pgfsetstrokecolor{currentstroke}%
\pgfsetdash{}{0pt}%
\pgfpathmoveto{\pgfqpoint{3.122108in}{1.554997in}}%
\pgfpathlineto{\pgfqpoint{3.233219in}{1.554997in}}%
\pgfpathlineto{\pgfqpoint{3.344330in}{1.554997in}}%
\pgfusepath{stroke}%
\end{pgfscope}%
\begin{pgfscope}%
\definecolor{textcolor}{rgb}{0.000000,0.000000,0.000000}%
\pgfsetstrokecolor{textcolor}%
\pgfsetfillcolor{textcolor}%
\pgftext[x=3.433219in,y=1.516108in,left,base]{\color{textcolor}\rmfamily\fontsize{8.000000}{9.600000}\selectfont NPLC 50}%
\end{pgfscope}%
\end{pgfpicture}%
\makeatother%
\endgroup%

    \caption{Allan deviation for different integration times before applying the AZ algorithm. Deadtime $\theta = 0$. The dashed line denotes the Allan variance without AZ.}
    \label{fig:autozero_nplcs_adev}
\end{figure}

It can be seen, that with increasing integration times before applying the AZ algorithm more uncertainty is accumulated due to the $f^{-1}$ content, which cannot be filtered. As a result, after removing the $f^{-1}$ content using autozeroing more time is required for filtering until the same Allan deviation can be reached. From these simulations it can be said, that when there is negligible dead time $\theta$ involved when switching the inputs it is advantageous to switch to switch early, while white noise is still dominating the noise.

Finally, the case of a non-negligible dead time shall be treated. When the dead time has to be considered, it is clear, that the autozero frequency cannot be arbitrarily increased, because the proportion of sampling time lost, in comparison to the time spent sampling, increases. This effective loss in sampling time then increases the noise spectral density due to aliasing as discussed above. To show this effect the simulation above is modified to include a dead time of \qty{1}{\plc} as detailed in figure \ref{fig:dmm_autozer_offset_nulling}. The measurement sequence now includes a dead time after switching each input. \citeauthor{autozero_with_dead_time} proposes \cite{autozero_with_dead_time} splitting the measurement interval in two and instead of measuring HI-LO-HI-LO, to measure HI-LO-LO-HI. This scheme is a mixed bag, because the $f^{-1}$ noise is correlated and its autocorrelation function decays with $e^{-t}$, therefore constantly changing the order of subtracted samples is not as efficient in removing the noise as the normal autozero procedure. Only when the dead time is large in comparison to the measurement time, this method yields an advantage. Therefore only the simplest case of a HI-LO-HI-LO measurement is treated here. The Allan deviation for different integration times is evaluated in same way as in figure \ref{fig:autozero_nplcs_adev}. The results are shown in figure \ref{fig:autozero_deadtime_nplcs_adev}.

Figure \ref{fig:autozero_deadtime_nplcs_adev} demonstrates, that the effectiveness of the AZ scheme no longer increases with an increasing switching frequency and there is an optimal autozero interval. For the parameters chosen for this simulation ($f_c = \qty{1.5}{\Hz}$ and \qty[power-half-as-sqrt, per-mode=symbol]{165}{\nV \Hz\tothe{-0.5}}), \qty{5}{\plc} is the optimal interval. If the corner frequency is shifted to a lower frequency, the optimum shifts more towards \qty{10}{\plc}. The same goes for a higher line frequency of \qty{60}{\Hz}. This explains, why HP chose \qty{10}{\plc} as the maximum integration time. For integration times higher than that, software averaging is used, therefore delivering the perfomance shown in figure \ref{fig:autozero_deadtime_nplcs_adev} along the \qty{10}{\plc} line.

\begin{figure}[ht]
    \centering
    %% Creator: Matplotlib, PGF backend
%%
%% To include the figure in your LaTeX document, write
%%   \input{<filename>.pgf}
%%
%% Make sure the required packages are loaded in your preamble
%%   \usepackage{pgf}
%%
%% Also ensure that all the required font packages are loaded; for instance,
%% the lmodern package is sometimes necessary when using math font.
%%   \usepackage{lmodern}
%%
%% Figures using additional raster images can only be included by \input if
%% they are in the same directory as the main LaTeX file. For loading figures
%% from other directories you can use the `import` package
%%   \usepackage{import}
%%
%% and then include the figures with
%%   \import{<path to file>}{<filename>.pgf}
%%
%% Matplotlib used the following preamble
%%   \usepackage{siunitx}
%%   \usepackage{fontspec}
%%   \makeatletter\@ifpackageloaded{underscore}{}{\usepackage[strings]{underscore}}\makeatother
%%
\begingroup%
\makeatletter%
\begin{pgfpicture}%
\pgfpathrectangle{\pgfpointorigin}{\pgfqpoint{4.190000in}{2.510000in}}%
\pgfusepath{use as bounding box, clip}%
\begin{pgfscope}%
\pgfsetbuttcap%
\pgfsetmiterjoin%
\definecolor{currentfill}{rgb}{1.000000,1.000000,1.000000}%
\pgfsetfillcolor{currentfill}%
\pgfsetlinewidth{0.000000pt}%
\definecolor{currentstroke}{rgb}{1.000000,1.000000,1.000000}%
\pgfsetstrokecolor{currentstroke}%
\pgfsetdash{}{0pt}%
\pgfpathmoveto{\pgfqpoint{0.000000in}{0.000000in}}%
\pgfpathlineto{\pgfqpoint{4.190000in}{0.000000in}}%
\pgfpathlineto{\pgfqpoint{4.190000in}{2.510000in}}%
\pgfpathlineto{\pgfqpoint{0.000000in}{2.510000in}}%
\pgfpathlineto{\pgfqpoint{0.000000in}{0.000000in}}%
\pgfpathclose%
\pgfusepath{fill}%
\end{pgfscope}%
\begin{pgfscope}%
\pgfsetbuttcap%
\pgfsetmiterjoin%
\definecolor{currentfill}{rgb}{1.000000,1.000000,1.000000}%
\pgfsetfillcolor{currentfill}%
\pgfsetlinewidth{0.000000pt}%
\definecolor{currentstroke}{rgb}{0.000000,0.000000,0.000000}%
\pgfsetstrokecolor{currentstroke}%
\pgfsetstrokeopacity{0.000000}%
\pgfsetdash{}{0pt}%
\pgfpathmoveto{\pgfqpoint{0.589510in}{0.417642in}}%
\pgfpathlineto{\pgfqpoint{4.148330in}{0.417642in}}%
\pgfpathlineto{\pgfqpoint{4.148330in}{2.468330in}}%
\pgfpathlineto{\pgfqpoint{0.589510in}{2.468330in}}%
\pgfpathlineto{\pgfqpoint{0.589510in}{0.417642in}}%
\pgfpathclose%
\pgfusepath{fill}%
\end{pgfscope}%
\begin{pgfscope}%
\pgfpathrectangle{\pgfqpoint{0.589510in}{0.417642in}}{\pgfqpoint{3.558820in}{2.050688in}}%
\pgfusepath{clip}%
\pgfsetrectcap%
\pgfsetroundjoin%
\pgfsetlinewidth{0.803000pt}%
\definecolor{currentstroke}{rgb}{0.450000,0.450000,0.450000}%
\pgfsetstrokecolor{currentstroke}%
\pgfsetdash{}{0pt}%
\pgfpathmoveto{\pgfqpoint{0.751274in}{0.417642in}}%
\pgfpathlineto{\pgfqpoint{0.751274in}{2.468330in}}%
\pgfusepath{stroke}%
\end{pgfscope}%
\begin{pgfscope}%
\pgfsetbuttcap%
\pgfsetroundjoin%
\definecolor{currentfill}{rgb}{0.000000,0.000000,0.000000}%
\pgfsetfillcolor{currentfill}%
\pgfsetlinewidth{0.803000pt}%
\definecolor{currentstroke}{rgb}{0.000000,0.000000,0.000000}%
\pgfsetstrokecolor{currentstroke}%
\pgfsetdash{}{0pt}%
\pgfsys@defobject{currentmarker}{\pgfqpoint{0.000000in}{-0.048611in}}{\pgfqpoint{0.000000in}{0.000000in}}{%
\pgfpathmoveto{\pgfqpoint{0.000000in}{0.000000in}}%
\pgfpathlineto{\pgfqpoint{0.000000in}{-0.048611in}}%
\pgfusepath{stroke,fill}%
}%
\begin{pgfscope}%
\pgfsys@transformshift{0.751274in}{0.417642in}%
\pgfsys@useobject{currentmarker}{}%
\end{pgfscope}%
\end{pgfscope}%
\begin{pgfscope}%
\definecolor{textcolor}{rgb}{0.000000,0.000000,0.000000}%
\pgfsetstrokecolor{textcolor}%
\pgfsetfillcolor{textcolor}%
\pgftext[x=0.751274in,y=0.320420in,,top]{\color{textcolor}\rmfamily\fontsize{8.000000}{9.600000}\selectfont \(\displaystyle {10^{0}}\)}%
\end{pgfscope}%
\begin{pgfscope}%
\pgfpathrectangle{\pgfqpoint{0.589510in}{0.417642in}}{\pgfqpoint{3.558820in}{2.050688in}}%
\pgfusepath{clip}%
\pgfsetrectcap%
\pgfsetroundjoin%
\pgfsetlinewidth{0.803000pt}%
\definecolor{currentstroke}{rgb}{0.450000,0.450000,0.450000}%
\pgfsetstrokecolor{currentstroke}%
\pgfsetdash{}{0pt}%
\pgfpathmoveto{\pgfqpoint{1.560097in}{0.417642in}}%
\pgfpathlineto{\pgfqpoint{1.560097in}{2.468330in}}%
\pgfusepath{stroke}%
\end{pgfscope}%
\begin{pgfscope}%
\pgfsetbuttcap%
\pgfsetroundjoin%
\definecolor{currentfill}{rgb}{0.000000,0.000000,0.000000}%
\pgfsetfillcolor{currentfill}%
\pgfsetlinewidth{0.803000pt}%
\definecolor{currentstroke}{rgb}{0.000000,0.000000,0.000000}%
\pgfsetstrokecolor{currentstroke}%
\pgfsetdash{}{0pt}%
\pgfsys@defobject{currentmarker}{\pgfqpoint{0.000000in}{-0.048611in}}{\pgfqpoint{0.000000in}{0.000000in}}{%
\pgfpathmoveto{\pgfqpoint{0.000000in}{0.000000in}}%
\pgfpathlineto{\pgfqpoint{0.000000in}{-0.048611in}}%
\pgfusepath{stroke,fill}%
}%
\begin{pgfscope}%
\pgfsys@transformshift{1.560097in}{0.417642in}%
\pgfsys@useobject{currentmarker}{}%
\end{pgfscope}%
\end{pgfscope}%
\begin{pgfscope}%
\definecolor{textcolor}{rgb}{0.000000,0.000000,0.000000}%
\pgfsetstrokecolor{textcolor}%
\pgfsetfillcolor{textcolor}%
\pgftext[x=1.560097in,y=0.320420in,,top]{\color{textcolor}\rmfamily\fontsize{8.000000}{9.600000}\selectfont \(\displaystyle {10^{1}}\)}%
\end{pgfscope}%
\begin{pgfscope}%
\pgfpathrectangle{\pgfqpoint{0.589510in}{0.417642in}}{\pgfqpoint{3.558820in}{2.050688in}}%
\pgfusepath{clip}%
\pgfsetrectcap%
\pgfsetroundjoin%
\pgfsetlinewidth{0.803000pt}%
\definecolor{currentstroke}{rgb}{0.450000,0.450000,0.450000}%
\pgfsetstrokecolor{currentstroke}%
\pgfsetdash{}{0pt}%
\pgfpathmoveto{\pgfqpoint{2.368920in}{0.417642in}}%
\pgfpathlineto{\pgfqpoint{2.368920in}{2.468330in}}%
\pgfusepath{stroke}%
\end{pgfscope}%
\begin{pgfscope}%
\pgfsetbuttcap%
\pgfsetroundjoin%
\definecolor{currentfill}{rgb}{0.000000,0.000000,0.000000}%
\pgfsetfillcolor{currentfill}%
\pgfsetlinewidth{0.803000pt}%
\definecolor{currentstroke}{rgb}{0.000000,0.000000,0.000000}%
\pgfsetstrokecolor{currentstroke}%
\pgfsetdash{}{0pt}%
\pgfsys@defobject{currentmarker}{\pgfqpoint{0.000000in}{-0.048611in}}{\pgfqpoint{0.000000in}{0.000000in}}{%
\pgfpathmoveto{\pgfqpoint{0.000000in}{0.000000in}}%
\pgfpathlineto{\pgfqpoint{0.000000in}{-0.048611in}}%
\pgfusepath{stroke,fill}%
}%
\begin{pgfscope}%
\pgfsys@transformshift{2.368920in}{0.417642in}%
\pgfsys@useobject{currentmarker}{}%
\end{pgfscope}%
\end{pgfscope}%
\begin{pgfscope}%
\definecolor{textcolor}{rgb}{0.000000,0.000000,0.000000}%
\pgfsetstrokecolor{textcolor}%
\pgfsetfillcolor{textcolor}%
\pgftext[x=2.368920in,y=0.320420in,,top]{\color{textcolor}\rmfamily\fontsize{8.000000}{9.600000}\selectfont \(\displaystyle {10^{2}}\)}%
\end{pgfscope}%
\begin{pgfscope}%
\pgfpathrectangle{\pgfqpoint{0.589510in}{0.417642in}}{\pgfqpoint{3.558820in}{2.050688in}}%
\pgfusepath{clip}%
\pgfsetrectcap%
\pgfsetroundjoin%
\pgfsetlinewidth{0.803000pt}%
\definecolor{currentstroke}{rgb}{0.450000,0.450000,0.450000}%
\pgfsetstrokecolor{currentstroke}%
\pgfsetdash{}{0pt}%
\pgfpathmoveto{\pgfqpoint{3.177743in}{0.417642in}}%
\pgfpathlineto{\pgfqpoint{3.177743in}{2.468330in}}%
\pgfusepath{stroke}%
\end{pgfscope}%
\begin{pgfscope}%
\pgfsetbuttcap%
\pgfsetroundjoin%
\definecolor{currentfill}{rgb}{0.000000,0.000000,0.000000}%
\pgfsetfillcolor{currentfill}%
\pgfsetlinewidth{0.803000pt}%
\definecolor{currentstroke}{rgb}{0.000000,0.000000,0.000000}%
\pgfsetstrokecolor{currentstroke}%
\pgfsetdash{}{0pt}%
\pgfsys@defobject{currentmarker}{\pgfqpoint{0.000000in}{-0.048611in}}{\pgfqpoint{0.000000in}{0.000000in}}{%
\pgfpathmoveto{\pgfqpoint{0.000000in}{0.000000in}}%
\pgfpathlineto{\pgfqpoint{0.000000in}{-0.048611in}}%
\pgfusepath{stroke,fill}%
}%
\begin{pgfscope}%
\pgfsys@transformshift{3.177743in}{0.417642in}%
\pgfsys@useobject{currentmarker}{}%
\end{pgfscope}%
\end{pgfscope}%
\begin{pgfscope}%
\definecolor{textcolor}{rgb}{0.000000,0.000000,0.000000}%
\pgfsetstrokecolor{textcolor}%
\pgfsetfillcolor{textcolor}%
\pgftext[x=3.177743in,y=0.320420in,,top]{\color{textcolor}\rmfamily\fontsize{8.000000}{9.600000}\selectfont \(\displaystyle {10^{3}}\)}%
\end{pgfscope}%
\begin{pgfscope}%
\pgfpathrectangle{\pgfqpoint{0.589510in}{0.417642in}}{\pgfqpoint{3.558820in}{2.050688in}}%
\pgfusepath{clip}%
\pgfsetrectcap%
\pgfsetroundjoin%
\pgfsetlinewidth{0.803000pt}%
\definecolor{currentstroke}{rgb}{0.450000,0.450000,0.450000}%
\pgfsetstrokecolor{currentstroke}%
\pgfsetdash{}{0pt}%
\pgfpathmoveto{\pgfqpoint{3.986565in}{0.417642in}}%
\pgfpathlineto{\pgfqpoint{3.986565in}{2.468330in}}%
\pgfusepath{stroke}%
\end{pgfscope}%
\begin{pgfscope}%
\pgfsetbuttcap%
\pgfsetroundjoin%
\definecolor{currentfill}{rgb}{0.000000,0.000000,0.000000}%
\pgfsetfillcolor{currentfill}%
\pgfsetlinewidth{0.803000pt}%
\definecolor{currentstroke}{rgb}{0.000000,0.000000,0.000000}%
\pgfsetstrokecolor{currentstroke}%
\pgfsetdash{}{0pt}%
\pgfsys@defobject{currentmarker}{\pgfqpoint{0.000000in}{-0.048611in}}{\pgfqpoint{0.000000in}{0.000000in}}{%
\pgfpathmoveto{\pgfqpoint{0.000000in}{0.000000in}}%
\pgfpathlineto{\pgfqpoint{0.000000in}{-0.048611in}}%
\pgfusepath{stroke,fill}%
}%
\begin{pgfscope}%
\pgfsys@transformshift{3.986565in}{0.417642in}%
\pgfsys@useobject{currentmarker}{}%
\end{pgfscope}%
\end{pgfscope}%
\begin{pgfscope}%
\definecolor{textcolor}{rgb}{0.000000,0.000000,0.000000}%
\pgfsetstrokecolor{textcolor}%
\pgfsetfillcolor{textcolor}%
\pgftext[x=3.986565in,y=0.320420in,,top]{\color{textcolor}\rmfamily\fontsize{8.000000}{9.600000}\selectfont \(\displaystyle {10^{4}}\)}%
\end{pgfscope}%
\begin{pgfscope}%
\pgfpathrectangle{\pgfqpoint{0.589510in}{0.417642in}}{\pgfqpoint{3.558820in}{2.050688in}}%
\pgfusepath{clip}%
\pgfsetrectcap%
\pgfsetroundjoin%
\pgfsetlinewidth{0.803000pt}%
\definecolor{currentstroke}{rgb}{0.850000,0.850000,0.850000}%
\pgfsetstrokecolor{currentstroke}%
\pgfsetdash{}{0pt}%
\pgfpathmoveto{\pgfqpoint{0.625986in}{0.417642in}}%
\pgfpathlineto{\pgfqpoint{0.625986in}{2.468330in}}%
\pgfusepath{stroke}%
\end{pgfscope}%
\begin{pgfscope}%
\pgfsetbuttcap%
\pgfsetroundjoin%
\definecolor{currentfill}{rgb}{0.000000,0.000000,0.000000}%
\pgfsetfillcolor{currentfill}%
\pgfsetlinewidth{0.602250pt}%
\definecolor{currentstroke}{rgb}{0.000000,0.000000,0.000000}%
\pgfsetstrokecolor{currentstroke}%
\pgfsetdash{}{0pt}%
\pgfsys@defobject{currentmarker}{\pgfqpoint{0.000000in}{-0.027778in}}{\pgfqpoint{0.000000in}{0.000000in}}{%
\pgfpathmoveto{\pgfqpoint{0.000000in}{0.000000in}}%
\pgfpathlineto{\pgfqpoint{0.000000in}{-0.027778in}}%
\pgfusepath{stroke,fill}%
}%
\begin{pgfscope}%
\pgfsys@transformshift{0.625986in}{0.417642in}%
\pgfsys@useobject{currentmarker}{}%
\end{pgfscope}%
\end{pgfscope}%
\begin{pgfscope}%
\pgfpathrectangle{\pgfqpoint{0.589510in}{0.417642in}}{\pgfqpoint{3.558820in}{2.050688in}}%
\pgfusepath{clip}%
\pgfsetrectcap%
\pgfsetroundjoin%
\pgfsetlinewidth{0.803000pt}%
\definecolor{currentstroke}{rgb}{0.850000,0.850000,0.850000}%
\pgfsetstrokecolor{currentstroke}%
\pgfsetdash{}{0pt}%
\pgfpathmoveto{\pgfqpoint{0.672891in}{0.417642in}}%
\pgfpathlineto{\pgfqpoint{0.672891in}{2.468330in}}%
\pgfusepath{stroke}%
\end{pgfscope}%
\begin{pgfscope}%
\pgfsetbuttcap%
\pgfsetroundjoin%
\definecolor{currentfill}{rgb}{0.000000,0.000000,0.000000}%
\pgfsetfillcolor{currentfill}%
\pgfsetlinewidth{0.602250pt}%
\definecolor{currentstroke}{rgb}{0.000000,0.000000,0.000000}%
\pgfsetstrokecolor{currentstroke}%
\pgfsetdash{}{0pt}%
\pgfsys@defobject{currentmarker}{\pgfqpoint{0.000000in}{-0.027778in}}{\pgfqpoint{0.000000in}{0.000000in}}{%
\pgfpathmoveto{\pgfqpoint{0.000000in}{0.000000in}}%
\pgfpathlineto{\pgfqpoint{0.000000in}{-0.027778in}}%
\pgfusepath{stroke,fill}%
}%
\begin{pgfscope}%
\pgfsys@transformshift{0.672891in}{0.417642in}%
\pgfsys@useobject{currentmarker}{}%
\end{pgfscope}%
\end{pgfscope}%
\begin{pgfscope}%
\pgfpathrectangle{\pgfqpoint{0.589510in}{0.417642in}}{\pgfqpoint{3.558820in}{2.050688in}}%
\pgfusepath{clip}%
\pgfsetrectcap%
\pgfsetroundjoin%
\pgfsetlinewidth{0.803000pt}%
\definecolor{currentstroke}{rgb}{0.850000,0.850000,0.850000}%
\pgfsetstrokecolor{currentstroke}%
\pgfsetdash{}{0pt}%
\pgfpathmoveto{\pgfqpoint{0.714265in}{0.417642in}}%
\pgfpathlineto{\pgfqpoint{0.714265in}{2.468330in}}%
\pgfusepath{stroke}%
\end{pgfscope}%
\begin{pgfscope}%
\pgfsetbuttcap%
\pgfsetroundjoin%
\definecolor{currentfill}{rgb}{0.000000,0.000000,0.000000}%
\pgfsetfillcolor{currentfill}%
\pgfsetlinewidth{0.602250pt}%
\definecolor{currentstroke}{rgb}{0.000000,0.000000,0.000000}%
\pgfsetstrokecolor{currentstroke}%
\pgfsetdash{}{0pt}%
\pgfsys@defobject{currentmarker}{\pgfqpoint{0.000000in}{-0.027778in}}{\pgfqpoint{0.000000in}{0.000000in}}{%
\pgfpathmoveto{\pgfqpoint{0.000000in}{0.000000in}}%
\pgfpathlineto{\pgfqpoint{0.000000in}{-0.027778in}}%
\pgfusepath{stroke,fill}%
}%
\begin{pgfscope}%
\pgfsys@transformshift{0.714265in}{0.417642in}%
\pgfsys@useobject{currentmarker}{}%
\end{pgfscope}%
\end{pgfscope}%
\begin{pgfscope}%
\pgfpathrectangle{\pgfqpoint{0.589510in}{0.417642in}}{\pgfqpoint{3.558820in}{2.050688in}}%
\pgfusepath{clip}%
\pgfsetrectcap%
\pgfsetroundjoin%
\pgfsetlinewidth{0.803000pt}%
\definecolor{currentstroke}{rgb}{0.850000,0.850000,0.850000}%
\pgfsetstrokecolor{currentstroke}%
\pgfsetdash{}{0pt}%
\pgfpathmoveto{\pgfqpoint{0.994754in}{0.417642in}}%
\pgfpathlineto{\pgfqpoint{0.994754in}{2.468330in}}%
\pgfusepath{stroke}%
\end{pgfscope}%
\begin{pgfscope}%
\pgfsetbuttcap%
\pgfsetroundjoin%
\definecolor{currentfill}{rgb}{0.000000,0.000000,0.000000}%
\pgfsetfillcolor{currentfill}%
\pgfsetlinewidth{0.602250pt}%
\definecolor{currentstroke}{rgb}{0.000000,0.000000,0.000000}%
\pgfsetstrokecolor{currentstroke}%
\pgfsetdash{}{0pt}%
\pgfsys@defobject{currentmarker}{\pgfqpoint{0.000000in}{-0.027778in}}{\pgfqpoint{0.000000in}{0.000000in}}{%
\pgfpathmoveto{\pgfqpoint{0.000000in}{0.000000in}}%
\pgfpathlineto{\pgfqpoint{0.000000in}{-0.027778in}}%
\pgfusepath{stroke,fill}%
}%
\begin{pgfscope}%
\pgfsys@transformshift{0.994754in}{0.417642in}%
\pgfsys@useobject{currentmarker}{}%
\end{pgfscope}%
\end{pgfscope}%
\begin{pgfscope}%
\pgfpathrectangle{\pgfqpoint{0.589510in}{0.417642in}}{\pgfqpoint{3.558820in}{2.050688in}}%
\pgfusepath{clip}%
\pgfsetrectcap%
\pgfsetroundjoin%
\pgfsetlinewidth{0.803000pt}%
\definecolor{currentstroke}{rgb}{0.850000,0.850000,0.850000}%
\pgfsetstrokecolor{currentstroke}%
\pgfsetdash{}{0pt}%
\pgfpathmoveto{\pgfqpoint{1.137181in}{0.417642in}}%
\pgfpathlineto{\pgfqpoint{1.137181in}{2.468330in}}%
\pgfusepath{stroke}%
\end{pgfscope}%
\begin{pgfscope}%
\pgfsetbuttcap%
\pgfsetroundjoin%
\definecolor{currentfill}{rgb}{0.000000,0.000000,0.000000}%
\pgfsetfillcolor{currentfill}%
\pgfsetlinewidth{0.602250pt}%
\definecolor{currentstroke}{rgb}{0.000000,0.000000,0.000000}%
\pgfsetstrokecolor{currentstroke}%
\pgfsetdash{}{0pt}%
\pgfsys@defobject{currentmarker}{\pgfqpoint{0.000000in}{-0.027778in}}{\pgfqpoint{0.000000in}{0.000000in}}{%
\pgfpathmoveto{\pgfqpoint{0.000000in}{0.000000in}}%
\pgfpathlineto{\pgfqpoint{0.000000in}{-0.027778in}}%
\pgfusepath{stroke,fill}%
}%
\begin{pgfscope}%
\pgfsys@transformshift{1.137181in}{0.417642in}%
\pgfsys@useobject{currentmarker}{}%
\end{pgfscope}%
\end{pgfscope}%
\begin{pgfscope}%
\pgfpathrectangle{\pgfqpoint{0.589510in}{0.417642in}}{\pgfqpoint{3.558820in}{2.050688in}}%
\pgfusepath{clip}%
\pgfsetrectcap%
\pgfsetroundjoin%
\pgfsetlinewidth{0.803000pt}%
\definecolor{currentstroke}{rgb}{0.850000,0.850000,0.850000}%
\pgfsetstrokecolor{currentstroke}%
\pgfsetdash{}{0pt}%
\pgfpathmoveto{\pgfqpoint{1.238234in}{0.417642in}}%
\pgfpathlineto{\pgfqpoint{1.238234in}{2.468330in}}%
\pgfusepath{stroke}%
\end{pgfscope}%
\begin{pgfscope}%
\pgfsetbuttcap%
\pgfsetroundjoin%
\definecolor{currentfill}{rgb}{0.000000,0.000000,0.000000}%
\pgfsetfillcolor{currentfill}%
\pgfsetlinewidth{0.602250pt}%
\definecolor{currentstroke}{rgb}{0.000000,0.000000,0.000000}%
\pgfsetstrokecolor{currentstroke}%
\pgfsetdash{}{0pt}%
\pgfsys@defobject{currentmarker}{\pgfqpoint{0.000000in}{-0.027778in}}{\pgfqpoint{0.000000in}{0.000000in}}{%
\pgfpathmoveto{\pgfqpoint{0.000000in}{0.000000in}}%
\pgfpathlineto{\pgfqpoint{0.000000in}{-0.027778in}}%
\pgfusepath{stroke,fill}%
}%
\begin{pgfscope}%
\pgfsys@transformshift{1.238234in}{0.417642in}%
\pgfsys@useobject{currentmarker}{}%
\end{pgfscope}%
\end{pgfscope}%
\begin{pgfscope}%
\pgfpathrectangle{\pgfqpoint{0.589510in}{0.417642in}}{\pgfqpoint{3.558820in}{2.050688in}}%
\pgfusepath{clip}%
\pgfsetrectcap%
\pgfsetroundjoin%
\pgfsetlinewidth{0.803000pt}%
\definecolor{currentstroke}{rgb}{0.850000,0.850000,0.850000}%
\pgfsetstrokecolor{currentstroke}%
\pgfsetdash{}{0pt}%
\pgfpathmoveto{\pgfqpoint{1.316617in}{0.417642in}}%
\pgfpathlineto{\pgfqpoint{1.316617in}{2.468330in}}%
\pgfusepath{stroke}%
\end{pgfscope}%
\begin{pgfscope}%
\pgfsetbuttcap%
\pgfsetroundjoin%
\definecolor{currentfill}{rgb}{0.000000,0.000000,0.000000}%
\pgfsetfillcolor{currentfill}%
\pgfsetlinewidth{0.602250pt}%
\definecolor{currentstroke}{rgb}{0.000000,0.000000,0.000000}%
\pgfsetstrokecolor{currentstroke}%
\pgfsetdash{}{0pt}%
\pgfsys@defobject{currentmarker}{\pgfqpoint{0.000000in}{-0.027778in}}{\pgfqpoint{0.000000in}{0.000000in}}{%
\pgfpathmoveto{\pgfqpoint{0.000000in}{0.000000in}}%
\pgfpathlineto{\pgfqpoint{0.000000in}{-0.027778in}}%
\pgfusepath{stroke,fill}%
}%
\begin{pgfscope}%
\pgfsys@transformshift{1.316617in}{0.417642in}%
\pgfsys@useobject{currentmarker}{}%
\end{pgfscope}%
\end{pgfscope}%
\begin{pgfscope}%
\pgfpathrectangle{\pgfqpoint{0.589510in}{0.417642in}}{\pgfqpoint{3.558820in}{2.050688in}}%
\pgfusepath{clip}%
\pgfsetrectcap%
\pgfsetroundjoin%
\pgfsetlinewidth{0.803000pt}%
\definecolor{currentstroke}{rgb}{0.850000,0.850000,0.850000}%
\pgfsetstrokecolor{currentstroke}%
\pgfsetdash{}{0pt}%
\pgfpathmoveto{\pgfqpoint{1.380661in}{0.417642in}}%
\pgfpathlineto{\pgfqpoint{1.380661in}{2.468330in}}%
\pgfusepath{stroke}%
\end{pgfscope}%
\begin{pgfscope}%
\pgfsetbuttcap%
\pgfsetroundjoin%
\definecolor{currentfill}{rgb}{0.000000,0.000000,0.000000}%
\pgfsetfillcolor{currentfill}%
\pgfsetlinewidth{0.602250pt}%
\definecolor{currentstroke}{rgb}{0.000000,0.000000,0.000000}%
\pgfsetstrokecolor{currentstroke}%
\pgfsetdash{}{0pt}%
\pgfsys@defobject{currentmarker}{\pgfqpoint{0.000000in}{-0.027778in}}{\pgfqpoint{0.000000in}{0.000000in}}{%
\pgfpathmoveto{\pgfqpoint{0.000000in}{0.000000in}}%
\pgfpathlineto{\pgfqpoint{0.000000in}{-0.027778in}}%
\pgfusepath{stroke,fill}%
}%
\begin{pgfscope}%
\pgfsys@transformshift{1.380661in}{0.417642in}%
\pgfsys@useobject{currentmarker}{}%
\end{pgfscope}%
\end{pgfscope}%
\begin{pgfscope}%
\pgfpathrectangle{\pgfqpoint{0.589510in}{0.417642in}}{\pgfqpoint{3.558820in}{2.050688in}}%
\pgfusepath{clip}%
\pgfsetrectcap%
\pgfsetroundjoin%
\pgfsetlinewidth{0.803000pt}%
\definecolor{currentstroke}{rgb}{0.850000,0.850000,0.850000}%
\pgfsetstrokecolor{currentstroke}%
\pgfsetdash{}{0pt}%
\pgfpathmoveto{\pgfqpoint{1.434809in}{0.417642in}}%
\pgfpathlineto{\pgfqpoint{1.434809in}{2.468330in}}%
\pgfusepath{stroke}%
\end{pgfscope}%
\begin{pgfscope}%
\pgfsetbuttcap%
\pgfsetroundjoin%
\definecolor{currentfill}{rgb}{0.000000,0.000000,0.000000}%
\pgfsetfillcolor{currentfill}%
\pgfsetlinewidth{0.602250pt}%
\definecolor{currentstroke}{rgb}{0.000000,0.000000,0.000000}%
\pgfsetstrokecolor{currentstroke}%
\pgfsetdash{}{0pt}%
\pgfsys@defobject{currentmarker}{\pgfqpoint{0.000000in}{-0.027778in}}{\pgfqpoint{0.000000in}{0.000000in}}{%
\pgfpathmoveto{\pgfqpoint{0.000000in}{0.000000in}}%
\pgfpathlineto{\pgfqpoint{0.000000in}{-0.027778in}}%
\pgfusepath{stroke,fill}%
}%
\begin{pgfscope}%
\pgfsys@transformshift{1.434809in}{0.417642in}%
\pgfsys@useobject{currentmarker}{}%
\end{pgfscope}%
\end{pgfscope}%
\begin{pgfscope}%
\pgfpathrectangle{\pgfqpoint{0.589510in}{0.417642in}}{\pgfqpoint{3.558820in}{2.050688in}}%
\pgfusepath{clip}%
\pgfsetrectcap%
\pgfsetroundjoin%
\pgfsetlinewidth{0.803000pt}%
\definecolor{currentstroke}{rgb}{0.850000,0.850000,0.850000}%
\pgfsetstrokecolor{currentstroke}%
\pgfsetdash{}{0pt}%
\pgfpathmoveto{\pgfqpoint{1.481714in}{0.417642in}}%
\pgfpathlineto{\pgfqpoint{1.481714in}{2.468330in}}%
\pgfusepath{stroke}%
\end{pgfscope}%
\begin{pgfscope}%
\pgfsetbuttcap%
\pgfsetroundjoin%
\definecolor{currentfill}{rgb}{0.000000,0.000000,0.000000}%
\pgfsetfillcolor{currentfill}%
\pgfsetlinewidth{0.602250pt}%
\definecolor{currentstroke}{rgb}{0.000000,0.000000,0.000000}%
\pgfsetstrokecolor{currentstroke}%
\pgfsetdash{}{0pt}%
\pgfsys@defobject{currentmarker}{\pgfqpoint{0.000000in}{-0.027778in}}{\pgfqpoint{0.000000in}{0.000000in}}{%
\pgfpathmoveto{\pgfqpoint{0.000000in}{0.000000in}}%
\pgfpathlineto{\pgfqpoint{0.000000in}{-0.027778in}}%
\pgfusepath{stroke,fill}%
}%
\begin{pgfscope}%
\pgfsys@transformshift{1.481714in}{0.417642in}%
\pgfsys@useobject{currentmarker}{}%
\end{pgfscope}%
\end{pgfscope}%
\begin{pgfscope}%
\pgfpathrectangle{\pgfqpoint{0.589510in}{0.417642in}}{\pgfqpoint{3.558820in}{2.050688in}}%
\pgfusepath{clip}%
\pgfsetrectcap%
\pgfsetroundjoin%
\pgfsetlinewidth{0.803000pt}%
\definecolor{currentstroke}{rgb}{0.850000,0.850000,0.850000}%
\pgfsetstrokecolor{currentstroke}%
\pgfsetdash{}{0pt}%
\pgfpathmoveto{\pgfqpoint{1.523087in}{0.417642in}}%
\pgfpathlineto{\pgfqpoint{1.523087in}{2.468330in}}%
\pgfusepath{stroke}%
\end{pgfscope}%
\begin{pgfscope}%
\pgfsetbuttcap%
\pgfsetroundjoin%
\definecolor{currentfill}{rgb}{0.000000,0.000000,0.000000}%
\pgfsetfillcolor{currentfill}%
\pgfsetlinewidth{0.602250pt}%
\definecolor{currentstroke}{rgb}{0.000000,0.000000,0.000000}%
\pgfsetstrokecolor{currentstroke}%
\pgfsetdash{}{0pt}%
\pgfsys@defobject{currentmarker}{\pgfqpoint{0.000000in}{-0.027778in}}{\pgfqpoint{0.000000in}{0.000000in}}{%
\pgfpathmoveto{\pgfqpoint{0.000000in}{0.000000in}}%
\pgfpathlineto{\pgfqpoint{0.000000in}{-0.027778in}}%
\pgfusepath{stroke,fill}%
}%
\begin{pgfscope}%
\pgfsys@transformshift{1.523087in}{0.417642in}%
\pgfsys@useobject{currentmarker}{}%
\end{pgfscope}%
\end{pgfscope}%
\begin{pgfscope}%
\pgfpathrectangle{\pgfqpoint{0.589510in}{0.417642in}}{\pgfqpoint{3.558820in}{2.050688in}}%
\pgfusepath{clip}%
\pgfsetrectcap%
\pgfsetroundjoin%
\pgfsetlinewidth{0.803000pt}%
\definecolor{currentstroke}{rgb}{0.850000,0.850000,0.850000}%
\pgfsetstrokecolor{currentstroke}%
\pgfsetdash{}{0pt}%
\pgfpathmoveto{\pgfqpoint{1.803577in}{0.417642in}}%
\pgfpathlineto{\pgfqpoint{1.803577in}{2.468330in}}%
\pgfusepath{stroke}%
\end{pgfscope}%
\begin{pgfscope}%
\pgfsetbuttcap%
\pgfsetroundjoin%
\definecolor{currentfill}{rgb}{0.000000,0.000000,0.000000}%
\pgfsetfillcolor{currentfill}%
\pgfsetlinewidth{0.602250pt}%
\definecolor{currentstroke}{rgb}{0.000000,0.000000,0.000000}%
\pgfsetstrokecolor{currentstroke}%
\pgfsetdash{}{0pt}%
\pgfsys@defobject{currentmarker}{\pgfqpoint{0.000000in}{-0.027778in}}{\pgfqpoint{0.000000in}{0.000000in}}{%
\pgfpathmoveto{\pgfqpoint{0.000000in}{0.000000in}}%
\pgfpathlineto{\pgfqpoint{0.000000in}{-0.027778in}}%
\pgfusepath{stroke,fill}%
}%
\begin{pgfscope}%
\pgfsys@transformshift{1.803577in}{0.417642in}%
\pgfsys@useobject{currentmarker}{}%
\end{pgfscope}%
\end{pgfscope}%
\begin{pgfscope}%
\pgfpathrectangle{\pgfqpoint{0.589510in}{0.417642in}}{\pgfqpoint{3.558820in}{2.050688in}}%
\pgfusepath{clip}%
\pgfsetrectcap%
\pgfsetroundjoin%
\pgfsetlinewidth{0.803000pt}%
\definecolor{currentstroke}{rgb}{0.850000,0.850000,0.850000}%
\pgfsetstrokecolor{currentstroke}%
\pgfsetdash{}{0pt}%
\pgfpathmoveto{\pgfqpoint{1.946004in}{0.417642in}}%
\pgfpathlineto{\pgfqpoint{1.946004in}{2.468330in}}%
\pgfusepath{stroke}%
\end{pgfscope}%
\begin{pgfscope}%
\pgfsetbuttcap%
\pgfsetroundjoin%
\definecolor{currentfill}{rgb}{0.000000,0.000000,0.000000}%
\pgfsetfillcolor{currentfill}%
\pgfsetlinewidth{0.602250pt}%
\definecolor{currentstroke}{rgb}{0.000000,0.000000,0.000000}%
\pgfsetstrokecolor{currentstroke}%
\pgfsetdash{}{0pt}%
\pgfsys@defobject{currentmarker}{\pgfqpoint{0.000000in}{-0.027778in}}{\pgfqpoint{0.000000in}{0.000000in}}{%
\pgfpathmoveto{\pgfqpoint{0.000000in}{0.000000in}}%
\pgfpathlineto{\pgfqpoint{0.000000in}{-0.027778in}}%
\pgfusepath{stroke,fill}%
}%
\begin{pgfscope}%
\pgfsys@transformshift{1.946004in}{0.417642in}%
\pgfsys@useobject{currentmarker}{}%
\end{pgfscope}%
\end{pgfscope}%
\begin{pgfscope}%
\pgfpathrectangle{\pgfqpoint{0.589510in}{0.417642in}}{\pgfqpoint{3.558820in}{2.050688in}}%
\pgfusepath{clip}%
\pgfsetrectcap%
\pgfsetroundjoin%
\pgfsetlinewidth{0.803000pt}%
\definecolor{currentstroke}{rgb}{0.850000,0.850000,0.850000}%
\pgfsetstrokecolor{currentstroke}%
\pgfsetdash{}{0pt}%
\pgfpathmoveto{\pgfqpoint{2.047057in}{0.417642in}}%
\pgfpathlineto{\pgfqpoint{2.047057in}{2.468330in}}%
\pgfusepath{stroke}%
\end{pgfscope}%
\begin{pgfscope}%
\pgfsetbuttcap%
\pgfsetroundjoin%
\definecolor{currentfill}{rgb}{0.000000,0.000000,0.000000}%
\pgfsetfillcolor{currentfill}%
\pgfsetlinewidth{0.602250pt}%
\definecolor{currentstroke}{rgb}{0.000000,0.000000,0.000000}%
\pgfsetstrokecolor{currentstroke}%
\pgfsetdash{}{0pt}%
\pgfsys@defobject{currentmarker}{\pgfqpoint{0.000000in}{-0.027778in}}{\pgfqpoint{0.000000in}{0.000000in}}{%
\pgfpathmoveto{\pgfqpoint{0.000000in}{0.000000in}}%
\pgfpathlineto{\pgfqpoint{0.000000in}{-0.027778in}}%
\pgfusepath{stroke,fill}%
}%
\begin{pgfscope}%
\pgfsys@transformshift{2.047057in}{0.417642in}%
\pgfsys@useobject{currentmarker}{}%
\end{pgfscope}%
\end{pgfscope}%
\begin{pgfscope}%
\pgfpathrectangle{\pgfqpoint{0.589510in}{0.417642in}}{\pgfqpoint{3.558820in}{2.050688in}}%
\pgfusepath{clip}%
\pgfsetrectcap%
\pgfsetroundjoin%
\pgfsetlinewidth{0.803000pt}%
\definecolor{currentstroke}{rgb}{0.850000,0.850000,0.850000}%
\pgfsetstrokecolor{currentstroke}%
\pgfsetdash{}{0pt}%
\pgfpathmoveto{\pgfqpoint{2.125440in}{0.417642in}}%
\pgfpathlineto{\pgfqpoint{2.125440in}{2.468330in}}%
\pgfusepath{stroke}%
\end{pgfscope}%
\begin{pgfscope}%
\pgfsetbuttcap%
\pgfsetroundjoin%
\definecolor{currentfill}{rgb}{0.000000,0.000000,0.000000}%
\pgfsetfillcolor{currentfill}%
\pgfsetlinewidth{0.602250pt}%
\definecolor{currentstroke}{rgb}{0.000000,0.000000,0.000000}%
\pgfsetstrokecolor{currentstroke}%
\pgfsetdash{}{0pt}%
\pgfsys@defobject{currentmarker}{\pgfqpoint{0.000000in}{-0.027778in}}{\pgfqpoint{0.000000in}{0.000000in}}{%
\pgfpathmoveto{\pgfqpoint{0.000000in}{0.000000in}}%
\pgfpathlineto{\pgfqpoint{0.000000in}{-0.027778in}}%
\pgfusepath{stroke,fill}%
}%
\begin{pgfscope}%
\pgfsys@transformshift{2.125440in}{0.417642in}%
\pgfsys@useobject{currentmarker}{}%
\end{pgfscope}%
\end{pgfscope}%
\begin{pgfscope}%
\pgfpathrectangle{\pgfqpoint{0.589510in}{0.417642in}}{\pgfqpoint{3.558820in}{2.050688in}}%
\pgfusepath{clip}%
\pgfsetrectcap%
\pgfsetroundjoin%
\pgfsetlinewidth{0.803000pt}%
\definecolor{currentstroke}{rgb}{0.850000,0.850000,0.850000}%
\pgfsetstrokecolor{currentstroke}%
\pgfsetdash{}{0pt}%
\pgfpathmoveto{\pgfqpoint{2.189484in}{0.417642in}}%
\pgfpathlineto{\pgfqpoint{2.189484in}{2.468330in}}%
\pgfusepath{stroke}%
\end{pgfscope}%
\begin{pgfscope}%
\pgfsetbuttcap%
\pgfsetroundjoin%
\definecolor{currentfill}{rgb}{0.000000,0.000000,0.000000}%
\pgfsetfillcolor{currentfill}%
\pgfsetlinewidth{0.602250pt}%
\definecolor{currentstroke}{rgb}{0.000000,0.000000,0.000000}%
\pgfsetstrokecolor{currentstroke}%
\pgfsetdash{}{0pt}%
\pgfsys@defobject{currentmarker}{\pgfqpoint{0.000000in}{-0.027778in}}{\pgfqpoint{0.000000in}{0.000000in}}{%
\pgfpathmoveto{\pgfqpoint{0.000000in}{0.000000in}}%
\pgfpathlineto{\pgfqpoint{0.000000in}{-0.027778in}}%
\pgfusepath{stroke,fill}%
}%
\begin{pgfscope}%
\pgfsys@transformshift{2.189484in}{0.417642in}%
\pgfsys@useobject{currentmarker}{}%
\end{pgfscope}%
\end{pgfscope}%
\begin{pgfscope}%
\pgfpathrectangle{\pgfqpoint{0.589510in}{0.417642in}}{\pgfqpoint{3.558820in}{2.050688in}}%
\pgfusepath{clip}%
\pgfsetrectcap%
\pgfsetroundjoin%
\pgfsetlinewidth{0.803000pt}%
\definecolor{currentstroke}{rgb}{0.850000,0.850000,0.850000}%
\pgfsetstrokecolor{currentstroke}%
\pgfsetdash{}{0pt}%
\pgfpathmoveto{\pgfqpoint{2.243632in}{0.417642in}}%
\pgfpathlineto{\pgfqpoint{2.243632in}{2.468330in}}%
\pgfusepath{stroke}%
\end{pgfscope}%
\begin{pgfscope}%
\pgfsetbuttcap%
\pgfsetroundjoin%
\definecolor{currentfill}{rgb}{0.000000,0.000000,0.000000}%
\pgfsetfillcolor{currentfill}%
\pgfsetlinewidth{0.602250pt}%
\definecolor{currentstroke}{rgb}{0.000000,0.000000,0.000000}%
\pgfsetstrokecolor{currentstroke}%
\pgfsetdash{}{0pt}%
\pgfsys@defobject{currentmarker}{\pgfqpoint{0.000000in}{-0.027778in}}{\pgfqpoint{0.000000in}{0.000000in}}{%
\pgfpathmoveto{\pgfqpoint{0.000000in}{0.000000in}}%
\pgfpathlineto{\pgfqpoint{0.000000in}{-0.027778in}}%
\pgfusepath{stroke,fill}%
}%
\begin{pgfscope}%
\pgfsys@transformshift{2.243632in}{0.417642in}%
\pgfsys@useobject{currentmarker}{}%
\end{pgfscope}%
\end{pgfscope}%
\begin{pgfscope}%
\pgfpathrectangle{\pgfqpoint{0.589510in}{0.417642in}}{\pgfqpoint{3.558820in}{2.050688in}}%
\pgfusepath{clip}%
\pgfsetrectcap%
\pgfsetroundjoin%
\pgfsetlinewidth{0.803000pt}%
\definecolor{currentstroke}{rgb}{0.850000,0.850000,0.850000}%
\pgfsetstrokecolor{currentstroke}%
\pgfsetdash{}{0pt}%
\pgfpathmoveto{\pgfqpoint{2.290537in}{0.417642in}}%
\pgfpathlineto{\pgfqpoint{2.290537in}{2.468330in}}%
\pgfusepath{stroke}%
\end{pgfscope}%
\begin{pgfscope}%
\pgfsetbuttcap%
\pgfsetroundjoin%
\definecolor{currentfill}{rgb}{0.000000,0.000000,0.000000}%
\pgfsetfillcolor{currentfill}%
\pgfsetlinewidth{0.602250pt}%
\definecolor{currentstroke}{rgb}{0.000000,0.000000,0.000000}%
\pgfsetstrokecolor{currentstroke}%
\pgfsetdash{}{0pt}%
\pgfsys@defobject{currentmarker}{\pgfqpoint{0.000000in}{-0.027778in}}{\pgfqpoint{0.000000in}{0.000000in}}{%
\pgfpathmoveto{\pgfqpoint{0.000000in}{0.000000in}}%
\pgfpathlineto{\pgfqpoint{0.000000in}{-0.027778in}}%
\pgfusepath{stroke,fill}%
}%
\begin{pgfscope}%
\pgfsys@transformshift{2.290537in}{0.417642in}%
\pgfsys@useobject{currentmarker}{}%
\end{pgfscope}%
\end{pgfscope}%
\begin{pgfscope}%
\pgfpathrectangle{\pgfqpoint{0.589510in}{0.417642in}}{\pgfqpoint{3.558820in}{2.050688in}}%
\pgfusepath{clip}%
\pgfsetrectcap%
\pgfsetroundjoin%
\pgfsetlinewidth{0.803000pt}%
\definecolor{currentstroke}{rgb}{0.850000,0.850000,0.850000}%
\pgfsetstrokecolor{currentstroke}%
\pgfsetdash{}{0pt}%
\pgfpathmoveto{\pgfqpoint{2.331910in}{0.417642in}}%
\pgfpathlineto{\pgfqpoint{2.331910in}{2.468330in}}%
\pgfusepath{stroke}%
\end{pgfscope}%
\begin{pgfscope}%
\pgfsetbuttcap%
\pgfsetroundjoin%
\definecolor{currentfill}{rgb}{0.000000,0.000000,0.000000}%
\pgfsetfillcolor{currentfill}%
\pgfsetlinewidth{0.602250pt}%
\definecolor{currentstroke}{rgb}{0.000000,0.000000,0.000000}%
\pgfsetstrokecolor{currentstroke}%
\pgfsetdash{}{0pt}%
\pgfsys@defobject{currentmarker}{\pgfqpoint{0.000000in}{-0.027778in}}{\pgfqpoint{0.000000in}{0.000000in}}{%
\pgfpathmoveto{\pgfqpoint{0.000000in}{0.000000in}}%
\pgfpathlineto{\pgfqpoint{0.000000in}{-0.027778in}}%
\pgfusepath{stroke,fill}%
}%
\begin{pgfscope}%
\pgfsys@transformshift{2.331910in}{0.417642in}%
\pgfsys@useobject{currentmarker}{}%
\end{pgfscope}%
\end{pgfscope}%
\begin{pgfscope}%
\pgfpathrectangle{\pgfqpoint{0.589510in}{0.417642in}}{\pgfqpoint{3.558820in}{2.050688in}}%
\pgfusepath{clip}%
\pgfsetrectcap%
\pgfsetroundjoin%
\pgfsetlinewidth{0.803000pt}%
\definecolor{currentstroke}{rgb}{0.850000,0.850000,0.850000}%
\pgfsetstrokecolor{currentstroke}%
\pgfsetdash{}{0pt}%
\pgfpathmoveto{\pgfqpoint{2.612400in}{0.417642in}}%
\pgfpathlineto{\pgfqpoint{2.612400in}{2.468330in}}%
\pgfusepath{stroke}%
\end{pgfscope}%
\begin{pgfscope}%
\pgfsetbuttcap%
\pgfsetroundjoin%
\definecolor{currentfill}{rgb}{0.000000,0.000000,0.000000}%
\pgfsetfillcolor{currentfill}%
\pgfsetlinewidth{0.602250pt}%
\definecolor{currentstroke}{rgb}{0.000000,0.000000,0.000000}%
\pgfsetstrokecolor{currentstroke}%
\pgfsetdash{}{0pt}%
\pgfsys@defobject{currentmarker}{\pgfqpoint{0.000000in}{-0.027778in}}{\pgfqpoint{0.000000in}{0.000000in}}{%
\pgfpathmoveto{\pgfqpoint{0.000000in}{0.000000in}}%
\pgfpathlineto{\pgfqpoint{0.000000in}{-0.027778in}}%
\pgfusepath{stroke,fill}%
}%
\begin{pgfscope}%
\pgfsys@transformshift{2.612400in}{0.417642in}%
\pgfsys@useobject{currentmarker}{}%
\end{pgfscope}%
\end{pgfscope}%
\begin{pgfscope}%
\pgfpathrectangle{\pgfqpoint{0.589510in}{0.417642in}}{\pgfqpoint{3.558820in}{2.050688in}}%
\pgfusepath{clip}%
\pgfsetrectcap%
\pgfsetroundjoin%
\pgfsetlinewidth{0.803000pt}%
\definecolor{currentstroke}{rgb}{0.850000,0.850000,0.850000}%
\pgfsetstrokecolor{currentstroke}%
\pgfsetdash{}{0pt}%
\pgfpathmoveto{\pgfqpoint{2.754826in}{0.417642in}}%
\pgfpathlineto{\pgfqpoint{2.754826in}{2.468330in}}%
\pgfusepath{stroke}%
\end{pgfscope}%
\begin{pgfscope}%
\pgfsetbuttcap%
\pgfsetroundjoin%
\definecolor{currentfill}{rgb}{0.000000,0.000000,0.000000}%
\pgfsetfillcolor{currentfill}%
\pgfsetlinewidth{0.602250pt}%
\definecolor{currentstroke}{rgb}{0.000000,0.000000,0.000000}%
\pgfsetstrokecolor{currentstroke}%
\pgfsetdash{}{0pt}%
\pgfsys@defobject{currentmarker}{\pgfqpoint{0.000000in}{-0.027778in}}{\pgfqpoint{0.000000in}{0.000000in}}{%
\pgfpathmoveto{\pgfqpoint{0.000000in}{0.000000in}}%
\pgfpathlineto{\pgfqpoint{0.000000in}{-0.027778in}}%
\pgfusepath{stroke,fill}%
}%
\begin{pgfscope}%
\pgfsys@transformshift{2.754826in}{0.417642in}%
\pgfsys@useobject{currentmarker}{}%
\end{pgfscope}%
\end{pgfscope}%
\begin{pgfscope}%
\pgfpathrectangle{\pgfqpoint{0.589510in}{0.417642in}}{\pgfqpoint{3.558820in}{2.050688in}}%
\pgfusepath{clip}%
\pgfsetrectcap%
\pgfsetroundjoin%
\pgfsetlinewidth{0.803000pt}%
\definecolor{currentstroke}{rgb}{0.850000,0.850000,0.850000}%
\pgfsetstrokecolor{currentstroke}%
\pgfsetdash{}{0pt}%
\pgfpathmoveto{\pgfqpoint{2.855880in}{0.417642in}}%
\pgfpathlineto{\pgfqpoint{2.855880in}{2.468330in}}%
\pgfusepath{stroke}%
\end{pgfscope}%
\begin{pgfscope}%
\pgfsetbuttcap%
\pgfsetroundjoin%
\definecolor{currentfill}{rgb}{0.000000,0.000000,0.000000}%
\pgfsetfillcolor{currentfill}%
\pgfsetlinewidth{0.602250pt}%
\definecolor{currentstroke}{rgb}{0.000000,0.000000,0.000000}%
\pgfsetstrokecolor{currentstroke}%
\pgfsetdash{}{0pt}%
\pgfsys@defobject{currentmarker}{\pgfqpoint{0.000000in}{-0.027778in}}{\pgfqpoint{0.000000in}{0.000000in}}{%
\pgfpathmoveto{\pgfqpoint{0.000000in}{0.000000in}}%
\pgfpathlineto{\pgfqpoint{0.000000in}{-0.027778in}}%
\pgfusepath{stroke,fill}%
}%
\begin{pgfscope}%
\pgfsys@transformshift{2.855880in}{0.417642in}%
\pgfsys@useobject{currentmarker}{}%
\end{pgfscope}%
\end{pgfscope}%
\begin{pgfscope}%
\pgfpathrectangle{\pgfqpoint{0.589510in}{0.417642in}}{\pgfqpoint{3.558820in}{2.050688in}}%
\pgfusepath{clip}%
\pgfsetrectcap%
\pgfsetroundjoin%
\pgfsetlinewidth{0.803000pt}%
\definecolor{currentstroke}{rgb}{0.850000,0.850000,0.850000}%
\pgfsetstrokecolor{currentstroke}%
\pgfsetdash{}{0pt}%
\pgfpathmoveto{\pgfqpoint{2.934263in}{0.417642in}}%
\pgfpathlineto{\pgfqpoint{2.934263in}{2.468330in}}%
\pgfusepath{stroke}%
\end{pgfscope}%
\begin{pgfscope}%
\pgfsetbuttcap%
\pgfsetroundjoin%
\definecolor{currentfill}{rgb}{0.000000,0.000000,0.000000}%
\pgfsetfillcolor{currentfill}%
\pgfsetlinewidth{0.602250pt}%
\definecolor{currentstroke}{rgb}{0.000000,0.000000,0.000000}%
\pgfsetstrokecolor{currentstroke}%
\pgfsetdash{}{0pt}%
\pgfsys@defobject{currentmarker}{\pgfqpoint{0.000000in}{-0.027778in}}{\pgfqpoint{0.000000in}{0.000000in}}{%
\pgfpathmoveto{\pgfqpoint{0.000000in}{0.000000in}}%
\pgfpathlineto{\pgfqpoint{0.000000in}{-0.027778in}}%
\pgfusepath{stroke,fill}%
}%
\begin{pgfscope}%
\pgfsys@transformshift{2.934263in}{0.417642in}%
\pgfsys@useobject{currentmarker}{}%
\end{pgfscope}%
\end{pgfscope}%
\begin{pgfscope}%
\pgfpathrectangle{\pgfqpoint{0.589510in}{0.417642in}}{\pgfqpoint{3.558820in}{2.050688in}}%
\pgfusepath{clip}%
\pgfsetrectcap%
\pgfsetroundjoin%
\pgfsetlinewidth{0.803000pt}%
\definecolor{currentstroke}{rgb}{0.850000,0.850000,0.850000}%
\pgfsetstrokecolor{currentstroke}%
\pgfsetdash{}{0pt}%
\pgfpathmoveto{\pgfqpoint{2.998306in}{0.417642in}}%
\pgfpathlineto{\pgfqpoint{2.998306in}{2.468330in}}%
\pgfusepath{stroke}%
\end{pgfscope}%
\begin{pgfscope}%
\pgfsetbuttcap%
\pgfsetroundjoin%
\definecolor{currentfill}{rgb}{0.000000,0.000000,0.000000}%
\pgfsetfillcolor{currentfill}%
\pgfsetlinewidth{0.602250pt}%
\definecolor{currentstroke}{rgb}{0.000000,0.000000,0.000000}%
\pgfsetstrokecolor{currentstroke}%
\pgfsetdash{}{0pt}%
\pgfsys@defobject{currentmarker}{\pgfqpoint{0.000000in}{-0.027778in}}{\pgfqpoint{0.000000in}{0.000000in}}{%
\pgfpathmoveto{\pgfqpoint{0.000000in}{0.000000in}}%
\pgfpathlineto{\pgfqpoint{0.000000in}{-0.027778in}}%
\pgfusepath{stroke,fill}%
}%
\begin{pgfscope}%
\pgfsys@transformshift{2.998306in}{0.417642in}%
\pgfsys@useobject{currentmarker}{}%
\end{pgfscope}%
\end{pgfscope}%
\begin{pgfscope}%
\pgfpathrectangle{\pgfqpoint{0.589510in}{0.417642in}}{\pgfqpoint{3.558820in}{2.050688in}}%
\pgfusepath{clip}%
\pgfsetrectcap%
\pgfsetroundjoin%
\pgfsetlinewidth{0.803000pt}%
\definecolor{currentstroke}{rgb}{0.850000,0.850000,0.850000}%
\pgfsetstrokecolor{currentstroke}%
\pgfsetdash{}{0pt}%
\pgfpathmoveto{\pgfqpoint{3.052454in}{0.417642in}}%
\pgfpathlineto{\pgfqpoint{3.052454in}{2.468330in}}%
\pgfusepath{stroke}%
\end{pgfscope}%
\begin{pgfscope}%
\pgfsetbuttcap%
\pgfsetroundjoin%
\definecolor{currentfill}{rgb}{0.000000,0.000000,0.000000}%
\pgfsetfillcolor{currentfill}%
\pgfsetlinewidth{0.602250pt}%
\definecolor{currentstroke}{rgb}{0.000000,0.000000,0.000000}%
\pgfsetstrokecolor{currentstroke}%
\pgfsetdash{}{0pt}%
\pgfsys@defobject{currentmarker}{\pgfqpoint{0.000000in}{-0.027778in}}{\pgfqpoint{0.000000in}{0.000000in}}{%
\pgfpathmoveto{\pgfqpoint{0.000000in}{0.000000in}}%
\pgfpathlineto{\pgfqpoint{0.000000in}{-0.027778in}}%
\pgfusepath{stroke,fill}%
}%
\begin{pgfscope}%
\pgfsys@transformshift{3.052454in}{0.417642in}%
\pgfsys@useobject{currentmarker}{}%
\end{pgfscope}%
\end{pgfscope}%
\begin{pgfscope}%
\pgfpathrectangle{\pgfqpoint{0.589510in}{0.417642in}}{\pgfqpoint{3.558820in}{2.050688in}}%
\pgfusepath{clip}%
\pgfsetrectcap%
\pgfsetroundjoin%
\pgfsetlinewidth{0.803000pt}%
\definecolor{currentstroke}{rgb}{0.850000,0.850000,0.850000}%
\pgfsetstrokecolor{currentstroke}%
\pgfsetdash{}{0pt}%
\pgfpathmoveto{\pgfqpoint{3.099360in}{0.417642in}}%
\pgfpathlineto{\pgfqpoint{3.099360in}{2.468330in}}%
\pgfusepath{stroke}%
\end{pgfscope}%
\begin{pgfscope}%
\pgfsetbuttcap%
\pgfsetroundjoin%
\definecolor{currentfill}{rgb}{0.000000,0.000000,0.000000}%
\pgfsetfillcolor{currentfill}%
\pgfsetlinewidth{0.602250pt}%
\definecolor{currentstroke}{rgb}{0.000000,0.000000,0.000000}%
\pgfsetstrokecolor{currentstroke}%
\pgfsetdash{}{0pt}%
\pgfsys@defobject{currentmarker}{\pgfqpoint{0.000000in}{-0.027778in}}{\pgfqpoint{0.000000in}{0.000000in}}{%
\pgfpathmoveto{\pgfqpoint{0.000000in}{0.000000in}}%
\pgfpathlineto{\pgfqpoint{0.000000in}{-0.027778in}}%
\pgfusepath{stroke,fill}%
}%
\begin{pgfscope}%
\pgfsys@transformshift{3.099360in}{0.417642in}%
\pgfsys@useobject{currentmarker}{}%
\end{pgfscope}%
\end{pgfscope}%
\begin{pgfscope}%
\pgfpathrectangle{\pgfqpoint{0.589510in}{0.417642in}}{\pgfqpoint{3.558820in}{2.050688in}}%
\pgfusepath{clip}%
\pgfsetrectcap%
\pgfsetroundjoin%
\pgfsetlinewidth{0.803000pt}%
\definecolor{currentstroke}{rgb}{0.850000,0.850000,0.850000}%
\pgfsetstrokecolor{currentstroke}%
\pgfsetdash{}{0pt}%
\pgfpathmoveto{\pgfqpoint{3.140733in}{0.417642in}}%
\pgfpathlineto{\pgfqpoint{3.140733in}{2.468330in}}%
\pgfusepath{stroke}%
\end{pgfscope}%
\begin{pgfscope}%
\pgfsetbuttcap%
\pgfsetroundjoin%
\definecolor{currentfill}{rgb}{0.000000,0.000000,0.000000}%
\pgfsetfillcolor{currentfill}%
\pgfsetlinewidth{0.602250pt}%
\definecolor{currentstroke}{rgb}{0.000000,0.000000,0.000000}%
\pgfsetstrokecolor{currentstroke}%
\pgfsetdash{}{0pt}%
\pgfsys@defobject{currentmarker}{\pgfqpoint{0.000000in}{-0.027778in}}{\pgfqpoint{0.000000in}{0.000000in}}{%
\pgfpathmoveto{\pgfqpoint{0.000000in}{0.000000in}}%
\pgfpathlineto{\pgfqpoint{0.000000in}{-0.027778in}}%
\pgfusepath{stroke,fill}%
}%
\begin{pgfscope}%
\pgfsys@transformshift{3.140733in}{0.417642in}%
\pgfsys@useobject{currentmarker}{}%
\end{pgfscope}%
\end{pgfscope}%
\begin{pgfscope}%
\pgfpathrectangle{\pgfqpoint{0.589510in}{0.417642in}}{\pgfqpoint{3.558820in}{2.050688in}}%
\pgfusepath{clip}%
\pgfsetrectcap%
\pgfsetroundjoin%
\pgfsetlinewidth{0.803000pt}%
\definecolor{currentstroke}{rgb}{0.850000,0.850000,0.850000}%
\pgfsetstrokecolor{currentstroke}%
\pgfsetdash{}{0pt}%
\pgfpathmoveto{\pgfqpoint{3.421223in}{0.417642in}}%
\pgfpathlineto{\pgfqpoint{3.421223in}{2.468330in}}%
\pgfusepath{stroke}%
\end{pgfscope}%
\begin{pgfscope}%
\pgfsetbuttcap%
\pgfsetroundjoin%
\definecolor{currentfill}{rgb}{0.000000,0.000000,0.000000}%
\pgfsetfillcolor{currentfill}%
\pgfsetlinewidth{0.602250pt}%
\definecolor{currentstroke}{rgb}{0.000000,0.000000,0.000000}%
\pgfsetstrokecolor{currentstroke}%
\pgfsetdash{}{0pt}%
\pgfsys@defobject{currentmarker}{\pgfqpoint{0.000000in}{-0.027778in}}{\pgfqpoint{0.000000in}{0.000000in}}{%
\pgfpathmoveto{\pgfqpoint{0.000000in}{0.000000in}}%
\pgfpathlineto{\pgfqpoint{0.000000in}{-0.027778in}}%
\pgfusepath{stroke,fill}%
}%
\begin{pgfscope}%
\pgfsys@transformshift{3.421223in}{0.417642in}%
\pgfsys@useobject{currentmarker}{}%
\end{pgfscope}%
\end{pgfscope}%
\begin{pgfscope}%
\pgfpathrectangle{\pgfqpoint{0.589510in}{0.417642in}}{\pgfqpoint{3.558820in}{2.050688in}}%
\pgfusepath{clip}%
\pgfsetrectcap%
\pgfsetroundjoin%
\pgfsetlinewidth{0.803000pt}%
\definecolor{currentstroke}{rgb}{0.850000,0.850000,0.850000}%
\pgfsetstrokecolor{currentstroke}%
\pgfsetdash{}{0pt}%
\pgfpathmoveto{\pgfqpoint{3.563649in}{0.417642in}}%
\pgfpathlineto{\pgfqpoint{3.563649in}{2.468330in}}%
\pgfusepath{stroke}%
\end{pgfscope}%
\begin{pgfscope}%
\pgfsetbuttcap%
\pgfsetroundjoin%
\definecolor{currentfill}{rgb}{0.000000,0.000000,0.000000}%
\pgfsetfillcolor{currentfill}%
\pgfsetlinewidth{0.602250pt}%
\definecolor{currentstroke}{rgb}{0.000000,0.000000,0.000000}%
\pgfsetstrokecolor{currentstroke}%
\pgfsetdash{}{0pt}%
\pgfsys@defobject{currentmarker}{\pgfqpoint{0.000000in}{-0.027778in}}{\pgfqpoint{0.000000in}{0.000000in}}{%
\pgfpathmoveto{\pgfqpoint{0.000000in}{0.000000in}}%
\pgfpathlineto{\pgfqpoint{0.000000in}{-0.027778in}}%
\pgfusepath{stroke,fill}%
}%
\begin{pgfscope}%
\pgfsys@transformshift{3.563649in}{0.417642in}%
\pgfsys@useobject{currentmarker}{}%
\end{pgfscope}%
\end{pgfscope}%
\begin{pgfscope}%
\pgfpathrectangle{\pgfqpoint{0.589510in}{0.417642in}}{\pgfqpoint{3.558820in}{2.050688in}}%
\pgfusepath{clip}%
\pgfsetrectcap%
\pgfsetroundjoin%
\pgfsetlinewidth{0.803000pt}%
\definecolor{currentstroke}{rgb}{0.850000,0.850000,0.850000}%
\pgfsetstrokecolor{currentstroke}%
\pgfsetdash{}{0pt}%
\pgfpathmoveto{\pgfqpoint{3.664702in}{0.417642in}}%
\pgfpathlineto{\pgfqpoint{3.664702in}{2.468330in}}%
\pgfusepath{stroke}%
\end{pgfscope}%
\begin{pgfscope}%
\pgfsetbuttcap%
\pgfsetroundjoin%
\definecolor{currentfill}{rgb}{0.000000,0.000000,0.000000}%
\pgfsetfillcolor{currentfill}%
\pgfsetlinewidth{0.602250pt}%
\definecolor{currentstroke}{rgb}{0.000000,0.000000,0.000000}%
\pgfsetstrokecolor{currentstroke}%
\pgfsetdash{}{0pt}%
\pgfsys@defobject{currentmarker}{\pgfqpoint{0.000000in}{-0.027778in}}{\pgfqpoint{0.000000in}{0.000000in}}{%
\pgfpathmoveto{\pgfqpoint{0.000000in}{0.000000in}}%
\pgfpathlineto{\pgfqpoint{0.000000in}{-0.027778in}}%
\pgfusepath{stroke,fill}%
}%
\begin{pgfscope}%
\pgfsys@transformshift{3.664702in}{0.417642in}%
\pgfsys@useobject{currentmarker}{}%
\end{pgfscope}%
\end{pgfscope}%
\begin{pgfscope}%
\pgfpathrectangle{\pgfqpoint{0.589510in}{0.417642in}}{\pgfqpoint{3.558820in}{2.050688in}}%
\pgfusepath{clip}%
\pgfsetrectcap%
\pgfsetroundjoin%
\pgfsetlinewidth{0.803000pt}%
\definecolor{currentstroke}{rgb}{0.850000,0.850000,0.850000}%
\pgfsetstrokecolor{currentstroke}%
\pgfsetdash{}{0pt}%
\pgfpathmoveto{\pgfqpoint{3.743086in}{0.417642in}}%
\pgfpathlineto{\pgfqpoint{3.743086in}{2.468330in}}%
\pgfusepath{stroke}%
\end{pgfscope}%
\begin{pgfscope}%
\pgfsetbuttcap%
\pgfsetroundjoin%
\definecolor{currentfill}{rgb}{0.000000,0.000000,0.000000}%
\pgfsetfillcolor{currentfill}%
\pgfsetlinewidth{0.602250pt}%
\definecolor{currentstroke}{rgb}{0.000000,0.000000,0.000000}%
\pgfsetstrokecolor{currentstroke}%
\pgfsetdash{}{0pt}%
\pgfsys@defobject{currentmarker}{\pgfqpoint{0.000000in}{-0.027778in}}{\pgfqpoint{0.000000in}{0.000000in}}{%
\pgfpathmoveto{\pgfqpoint{0.000000in}{0.000000in}}%
\pgfpathlineto{\pgfqpoint{0.000000in}{-0.027778in}}%
\pgfusepath{stroke,fill}%
}%
\begin{pgfscope}%
\pgfsys@transformshift{3.743086in}{0.417642in}%
\pgfsys@useobject{currentmarker}{}%
\end{pgfscope}%
\end{pgfscope}%
\begin{pgfscope}%
\pgfpathrectangle{\pgfqpoint{0.589510in}{0.417642in}}{\pgfqpoint{3.558820in}{2.050688in}}%
\pgfusepath{clip}%
\pgfsetrectcap%
\pgfsetroundjoin%
\pgfsetlinewidth{0.803000pt}%
\definecolor{currentstroke}{rgb}{0.850000,0.850000,0.850000}%
\pgfsetstrokecolor{currentstroke}%
\pgfsetdash{}{0pt}%
\pgfpathmoveto{\pgfqpoint{3.807129in}{0.417642in}}%
\pgfpathlineto{\pgfqpoint{3.807129in}{2.468330in}}%
\pgfusepath{stroke}%
\end{pgfscope}%
\begin{pgfscope}%
\pgfsetbuttcap%
\pgfsetroundjoin%
\definecolor{currentfill}{rgb}{0.000000,0.000000,0.000000}%
\pgfsetfillcolor{currentfill}%
\pgfsetlinewidth{0.602250pt}%
\definecolor{currentstroke}{rgb}{0.000000,0.000000,0.000000}%
\pgfsetstrokecolor{currentstroke}%
\pgfsetdash{}{0pt}%
\pgfsys@defobject{currentmarker}{\pgfqpoint{0.000000in}{-0.027778in}}{\pgfqpoint{0.000000in}{0.000000in}}{%
\pgfpathmoveto{\pgfqpoint{0.000000in}{0.000000in}}%
\pgfpathlineto{\pgfqpoint{0.000000in}{-0.027778in}}%
\pgfusepath{stroke,fill}%
}%
\begin{pgfscope}%
\pgfsys@transformshift{3.807129in}{0.417642in}%
\pgfsys@useobject{currentmarker}{}%
\end{pgfscope}%
\end{pgfscope}%
\begin{pgfscope}%
\pgfpathrectangle{\pgfqpoint{0.589510in}{0.417642in}}{\pgfqpoint{3.558820in}{2.050688in}}%
\pgfusepath{clip}%
\pgfsetrectcap%
\pgfsetroundjoin%
\pgfsetlinewidth{0.803000pt}%
\definecolor{currentstroke}{rgb}{0.850000,0.850000,0.850000}%
\pgfsetstrokecolor{currentstroke}%
\pgfsetdash{}{0pt}%
\pgfpathmoveto{\pgfqpoint{3.861277in}{0.417642in}}%
\pgfpathlineto{\pgfqpoint{3.861277in}{2.468330in}}%
\pgfusepath{stroke}%
\end{pgfscope}%
\begin{pgfscope}%
\pgfsetbuttcap%
\pgfsetroundjoin%
\definecolor{currentfill}{rgb}{0.000000,0.000000,0.000000}%
\pgfsetfillcolor{currentfill}%
\pgfsetlinewidth{0.602250pt}%
\definecolor{currentstroke}{rgb}{0.000000,0.000000,0.000000}%
\pgfsetstrokecolor{currentstroke}%
\pgfsetdash{}{0pt}%
\pgfsys@defobject{currentmarker}{\pgfqpoint{0.000000in}{-0.027778in}}{\pgfqpoint{0.000000in}{0.000000in}}{%
\pgfpathmoveto{\pgfqpoint{0.000000in}{0.000000in}}%
\pgfpathlineto{\pgfqpoint{0.000000in}{-0.027778in}}%
\pgfusepath{stroke,fill}%
}%
\begin{pgfscope}%
\pgfsys@transformshift{3.861277in}{0.417642in}%
\pgfsys@useobject{currentmarker}{}%
\end{pgfscope}%
\end{pgfscope}%
\begin{pgfscope}%
\pgfpathrectangle{\pgfqpoint{0.589510in}{0.417642in}}{\pgfqpoint{3.558820in}{2.050688in}}%
\pgfusepath{clip}%
\pgfsetrectcap%
\pgfsetroundjoin%
\pgfsetlinewidth{0.803000pt}%
\definecolor{currentstroke}{rgb}{0.850000,0.850000,0.850000}%
\pgfsetstrokecolor{currentstroke}%
\pgfsetdash{}{0pt}%
\pgfpathmoveto{\pgfqpoint{3.908182in}{0.417642in}}%
\pgfpathlineto{\pgfqpoint{3.908182in}{2.468330in}}%
\pgfusepath{stroke}%
\end{pgfscope}%
\begin{pgfscope}%
\pgfsetbuttcap%
\pgfsetroundjoin%
\definecolor{currentfill}{rgb}{0.000000,0.000000,0.000000}%
\pgfsetfillcolor{currentfill}%
\pgfsetlinewidth{0.602250pt}%
\definecolor{currentstroke}{rgb}{0.000000,0.000000,0.000000}%
\pgfsetstrokecolor{currentstroke}%
\pgfsetdash{}{0pt}%
\pgfsys@defobject{currentmarker}{\pgfqpoint{0.000000in}{-0.027778in}}{\pgfqpoint{0.000000in}{0.000000in}}{%
\pgfpathmoveto{\pgfqpoint{0.000000in}{0.000000in}}%
\pgfpathlineto{\pgfqpoint{0.000000in}{-0.027778in}}%
\pgfusepath{stroke,fill}%
}%
\begin{pgfscope}%
\pgfsys@transformshift{3.908182in}{0.417642in}%
\pgfsys@useobject{currentmarker}{}%
\end{pgfscope}%
\end{pgfscope}%
\begin{pgfscope}%
\pgfpathrectangle{\pgfqpoint{0.589510in}{0.417642in}}{\pgfqpoint{3.558820in}{2.050688in}}%
\pgfusepath{clip}%
\pgfsetrectcap%
\pgfsetroundjoin%
\pgfsetlinewidth{0.803000pt}%
\definecolor{currentstroke}{rgb}{0.850000,0.850000,0.850000}%
\pgfsetstrokecolor{currentstroke}%
\pgfsetdash{}{0pt}%
\pgfpathmoveto{\pgfqpoint{3.949556in}{0.417642in}}%
\pgfpathlineto{\pgfqpoint{3.949556in}{2.468330in}}%
\pgfusepath{stroke}%
\end{pgfscope}%
\begin{pgfscope}%
\pgfsetbuttcap%
\pgfsetroundjoin%
\definecolor{currentfill}{rgb}{0.000000,0.000000,0.000000}%
\pgfsetfillcolor{currentfill}%
\pgfsetlinewidth{0.602250pt}%
\definecolor{currentstroke}{rgb}{0.000000,0.000000,0.000000}%
\pgfsetstrokecolor{currentstroke}%
\pgfsetdash{}{0pt}%
\pgfsys@defobject{currentmarker}{\pgfqpoint{0.000000in}{-0.027778in}}{\pgfqpoint{0.000000in}{0.000000in}}{%
\pgfpathmoveto{\pgfqpoint{0.000000in}{0.000000in}}%
\pgfpathlineto{\pgfqpoint{0.000000in}{-0.027778in}}%
\pgfusepath{stroke,fill}%
}%
\begin{pgfscope}%
\pgfsys@transformshift{3.949556in}{0.417642in}%
\pgfsys@useobject{currentmarker}{}%
\end{pgfscope}%
\end{pgfscope}%
\begin{pgfscope}%
\definecolor{textcolor}{rgb}{0.000000,0.000000,0.000000}%
\pgfsetstrokecolor{textcolor}%
\pgfsetfillcolor{textcolor}%
\pgftext[x=2.368920in,y=0.165003in,,top]{\color{textcolor}\rmfamily\fontsize{10.000000}{12.000000}\selectfont \(\displaystyle \tau\) in \unit{\second}}%
\end{pgfscope}%
\begin{pgfscope}%
\pgfpathrectangle{\pgfqpoint{0.589510in}{0.417642in}}{\pgfqpoint{3.558820in}{2.050688in}}%
\pgfusepath{clip}%
\pgfsetrectcap%
\pgfsetroundjoin%
\pgfsetlinewidth{0.803000pt}%
\definecolor{currentstroke}{rgb}{0.450000,0.450000,0.450000}%
\pgfsetstrokecolor{currentstroke}%
\pgfsetdash{}{0pt}%
\pgfpathmoveto{\pgfqpoint{0.589510in}{1.220333in}}%
\pgfpathlineto{\pgfqpoint{4.148330in}{1.220333in}}%
\pgfusepath{stroke}%
\end{pgfscope}%
\begin{pgfscope}%
\pgfsetbuttcap%
\pgfsetroundjoin%
\definecolor{currentfill}{rgb}{0.000000,0.000000,0.000000}%
\pgfsetfillcolor{currentfill}%
\pgfsetlinewidth{0.803000pt}%
\definecolor{currentstroke}{rgb}{0.000000,0.000000,0.000000}%
\pgfsetstrokecolor{currentstroke}%
\pgfsetdash{}{0pt}%
\pgfsys@defobject{currentmarker}{\pgfqpoint{-0.048611in}{0.000000in}}{\pgfqpoint{-0.000000in}{0.000000in}}{%
\pgfpathmoveto{\pgfqpoint{-0.000000in}{0.000000in}}%
\pgfpathlineto{\pgfqpoint{-0.048611in}{0.000000in}}%
\pgfusepath{stroke,fill}%
}%
\begin{pgfscope}%
\pgfsys@transformshift{0.589510in}{1.220333in}%
\pgfsys@useobject{currentmarker}{}%
\end{pgfscope}%
\end{pgfscope}%
\begin{pgfscope}%
\definecolor{textcolor}{rgb}{0.000000,0.000000,0.000000}%
\pgfsetstrokecolor{textcolor}%
\pgfsetfillcolor{textcolor}%
\pgftext[x=0.236114in, y=1.181180in, left, base]{\color{textcolor}\rmfamily\fontsize{8.000000}{9.600000}\selectfont \(\displaystyle {10^{-7}}\)}%
\end{pgfscope}%
\begin{pgfscope}%
\pgfpathrectangle{\pgfqpoint{0.589510in}{0.417642in}}{\pgfqpoint{3.558820in}{2.050688in}}%
\pgfusepath{clip}%
\pgfsetrectcap%
\pgfsetroundjoin%
\pgfsetlinewidth{0.803000pt}%
\definecolor{currentstroke}{rgb}{0.450000,0.450000,0.450000}%
\pgfsetstrokecolor{currentstroke}%
\pgfsetdash{}{0pt}%
\pgfpathmoveto{\pgfqpoint{0.589510in}{2.269035in}}%
\pgfpathlineto{\pgfqpoint{4.148330in}{2.269035in}}%
\pgfusepath{stroke}%
\end{pgfscope}%
\begin{pgfscope}%
\pgfsetbuttcap%
\pgfsetroundjoin%
\definecolor{currentfill}{rgb}{0.000000,0.000000,0.000000}%
\pgfsetfillcolor{currentfill}%
\pgfsetlinewidth{0.803000pt}%
\definecolor{currentstroke}{rgb}{0.000000,0.000000,0.000000}%
\pgfsetstrokecolor{currentstroke}%
\pgfsetdash{}{0pt}%
\pgfsys@defobject{currentmarker}{\pgfqpoint{-0.048611in}{0.000000in}}{\pgfqpoint{-0.000000in}{0.000000in}}{%
\pgfpathmoveto{\pgfqpoint{-0.000000in}{0.000000in}}%
\pgfpathlineto{\pgfqpoint{-0.048611in}{0.000000in}}%
\pgfusepath{stroke,fill}%
}%
\begin{pgfscope}%
\pgfsys@transformshift{0.589510in}{2.269035in}%
\pgfsys@useobject{currentmarker}{}%
\end{pgfscope}%
\end{pgfscope}%
\begin{pgfscope}%
\definecolor{textcolor}{rgb}{0.000000,0.000000,0.000000}%
\pgfsetstrokecolor{textcolor}%
\pgfsetfillcolor{textcolor}%
\pgftext[x=0.236114in, y=2.229882in, left, base]{\color{textcolor}\rmfamily\fontsize{8.000000}{9.600000}\selectfont \(\displaystyle {10^{-6}}\)}%
\end{pgfscope}%
\begin{pgfscope}%
\pgfpathrectangle{\pgfqpoint{0.589510in}{0.417642in}}{\pgfqpoint{3.558820in}{2.050688in}}%
\pgfusepath{clip}%
\pgfsetrectcap%
\pgfsetroundjoin%
\pgfsetlinewidth{0.803000pt}%
\definecolor{currentstroke}{rgb}{0.850000,0.850000,0.850000}%
\pgfsetstrokecolor{currentstroke}%
\pgfsetdash{}{0pt}%
\pgfpathmoveto{\pgfqpoint{0.589510in}{0.487322in}}%
\pgfpathlineto{\pgfqpoint{4.148330in}{0.487322in}}%
\pgfusepath{stroke}%
\end{pgfscope}%
\begin{pgfscope}%
\pgfsetbuttcap%
\pgfsetroundjoin%
\definecolor{currentfill}{rgb}{0.000000,0.000000,0.000000}%
\pgfsetfillcolor{currentfill}%
\pgfsetlinewidth{0.602250pt}%
\definecolor{currentstroke}{rgb}{0.000000,0.000000,0.000000}%
\pgfsetstrokecolor{currentstroke}%
\pgfsetdash{}{0pt}%
\pgfsys@defobject{currentmarker}{\pgfqpoint{-0.027778in}{0.000000in}}{\pgfqpoint{-0.000000in}{0.000000in}}{%
\pgfpathmoveto{\pgfqpoint{-0.000000in}{0.000000in}}%
\pgfpathlineto{\pgfqpoint{-0.027778in}{0.000000in}}%
\pgfusepath{stroke,fill}%
}%
\begin{pgfscope}%
\pgfsys@transformshift{0.589510in}{0.487322in}%
\pgfsys@useobject{currentmarker}{}%
\end{pgfscope}%
\end{pgfscope}%
\begin{pgfscope}%
\pgfpathrectangle{\pgfqpoint{0.589510in}{0.417642in}}{\pgfqpoint{3.558820in}{2.050688in}}%
\pgfusepath{clip}%
\pgfsetrectcap%
\pgfsetroundjoin%
\pgfsetlinewidth{0.803000pt}%
\definecolor{currentstroke}{rgb}{0.850000,0.850000,0.850000}%
\pgfsetstrokecolor{currentstroke}%
\pgfsetdash{}{0pt}%
\pgfpathmoveto{\pgfqpoint{0.589510in}{0.671989in}}%
\pgfpathlineto{\pgfqpoint{4.148330in}{0.671989in}}%
\pgfusepath{stroke}%
\end{pgfscope}%
\begin{pgfscope}%
\pgfsetbuttcap%
\pgfsetroundjoin%
\definecolor{currentfill}{rgb}{0.000000,0.000000,0.000000}%
\pgfsetfillcolor{currentfill}%
\pgfsetlinewidth{0.602250pt}%
\definecolor{currentstroke}{rgb}{0.000000,0.000000,0.000000}%
\pgfsetstrokecolor{currentstroke}%
\pgfsetdash{}{0pt}%
\pgfsys@defobject{currentmarker}{\pgfqpoint{-0.027778in}{0.000000in}}{\pgfqpoint{-0.000000in}{0.000000in}}{%
\pgfpathmoveto{\pgfqpoint{-0.000000in}{0.000000in}}%
\pgfpathlineto{\pgfqpoint{-0.027778in}{0.000000in}}%
\pgfusepath{stroke,fill}%
}%
\begin{pgfscope}%
\pgfsys@transformshift{0.589510in}{0.671989in}%
\pgfsys@useobject{currentmarker}{}%
\end{pgfscope}%
\end{pgfscope}%
\begin{pgfscope}%
\pgfpathrectangle{\pgfqpoint{0.589510in}{0.417642in}}{\pgfqpoint{3.558820in}{2.050688in}}%
\pgfusepath{clip}%
\pgfsetrectcap%
\pgfsetroundjoin%
\pgfsetlinewidth{0.803000pt}%
\definecolor{currentstroke}{rgb}{0.850000,0.850000,0.850000}%
\pgfsetstrokecolor{currentstroke}%
\pgfsetdash{}{0pt}%
\pgfpathmoveto{\pgfqpoint{0.589510in}{0.803012in}}%
\pgfpathlineto{\pgfqpoint{4.148330in}{0.803012in}}%
\pgfusepath{stroke}%
\end{pgfscope}%
\begin{pgfscope}%
\pgfsetbuttcap%
\pgfsetroundjoin%
\definecolor{currentfill}{rgb}{0.000000,0.000000,0.000000}%
\pgfsetfillcolor{currentfill}%
\pgfsetlinewidth{0.602250pt}%
\definecolor{currentstroke}{rgb}{0.000000,0.000000,0.000000}%
\pgfsetstrokecolor{currentstroke}%
\pgfsetdash{}{0pt}%
\pgfsys@defobject{currentmarker}{\pgfqpoint{-0.027778in}{0.000000in}}{\pgfqpoint{-0.000000in}{0.000000in}}{%
\pgfpathmoveto{\pgfqpoint{-0.000000in}{0.000000in}}%
\pgfpathlineto{\pgfqpoint{-0.027778in}{0.000000in}}%
\pgfusepath{stroke,fill}%
}%
\begin{pgfscope}%
\pgfsys@transformshift{0.589510in}{0.803012in}%
\pgfsys@useobject{currentmarker}{}%
\end{pgfscope}%
\end{pgfscope}%
\begin{pgfscope}%
\pgfpathrectangle{\pgfqpoint{0.589510in}{0.417642in}}{\pgfqpoint{3.558820in}{2.050688in}}%
\pgfusepath{clip}%
\pgfsetrectcap%
\pgfsetroundjoin%
\pgfsetlinewidth{0.803000pt}%
\definecolor{currentstroke}{rgb}{0.850000,0.850000,0.850000}%
\pgfsetstrokecolor{currentstroke}%
\pgfsetdash{}{0pt}%
\pgfpathmoveto{\pgfqpoint{0.589510in}{0.904642in}}%
\pgfpathlineto{\pgfqpoint{4.148330in}{0.904642in}}%
\pgfusepath{stroke}%
\end{pgfscope}%
\begin{pgfscope}%
\pgfsetbuttcap%
\pgfsetroundjoin%
\definecolor{currentfill}{rgb}{0.000000,0.000000,0.000000}%
\pgfsetfillcolor{currentfill}%
\pgfsetlinewidth{0.602250pt}%
\definecolor{currentstroke}{rgb}{0.000000,0.000000,0.000000}%
\pgfsetstrokecolor{currentstroke}%
\pgfsetdash{}{0pt}%
\pgfsys@defobject{currentmarker}{\pgfqpoint{-0.027778in}{0.000000in}}{\pgfqpoint{-0.000000in}{0.000000in}}{%
\pgfpathmoveto{\pgfqpoint{-0.000000in}{0.000000in}}%
\pgfpathlineto{\pgfqpoint{-0.027778in}{0.000000in}}%
\pgfusepath{stroke,fill}%
}%
\begin{pgfscope}%
\pgfsys@transformshift{0.589510in}{0.904642in}%
\pgfsys@useobject{currentmarker}{}%
\end{pgfscope}%
\end{pgfscope}%
\begin{pgfscope}%
\pgfpathrectangle{\pgfqpoint{0.589510in}{0.417642in}}{\pgfqpoint{3.558820in}{2.050688in}}%
\pgfusepath{clip}%
\pgfsetrectcap%
\pgfsetroundjoin%
\pgfsetlinewidth{0.803000pt}%
\definecolor{currentstroke}{rgb}{0.850000,0.850000,0.850000}%
\pgfsetstrokecolor{currentstroke}%
\pgfsetdash{}{0pt}%
\pgfpathmoveto{\pgfqpoint{0.589510in}{0.987680in}}%
\pgfpathlineto{\pgfqpoint{4.148330in}{0.987680in}}%
\pgfusepath{stroke}%
\end{pgfscope}%
\begin{pgfscope}%
\pgfsetbuttcap%
\pgfsetroundjoin%
\definecolor{currentfill}{rgb}{0.000000,0.000000,0.000000}%
\pgfsetfillcolor{currentfill}%
\pgfsetlinewidth{0.602250pt}%
\definecolor{currentstroke}{rgb}{0.000000,0.000000,0.000000}%
\pgfsetstrokecolor{currentstroke}%
\pgfsetdash{}{0pt}%
\pgfsys@defobject{currentmarker}{\pgfqpoint{-0.027778in}{0.000000in}}{\pgfqpoint{-0.000000in}{0.000000in}}{%
\pgfpathmoveto{\pgfqpoint{-0.000000in}{0.000000in}}%
\pgfpathlineto{\pgfqpoint{-0.027778in}{0.000000in}}%
\pgfusepath{stroke,fill}%
}%
\begin{pgfscope}%
\pgfsys@transformshift{0.589510in}{0.987680in}%
\pgfsys@useobject{currentmarker}{}%
\end{pgfscope}%
\end{pgfscope}%
\begin{pgfscope}%
\pgfpathrectangle{\pgfqpoint{0.589510in}{0.417642in}}{\pgfqpoint{3.558820in}{2.050688in}}%
\pgfusepath{clip}%
\pgfsetrectcap%
\pgfsetroundjoin%
\pgfsetlinewidth{0.803000pt}%
\definecolor{currentstroke}{rgb}{0.850000,0.850000,0.850000}%
\pgfsetstrokecolor{currentstroke}%
\pgfsetdash{}{0pt}%
\pgfpathmoveto{\pgfqpoint{0.589510in}{1.057887in}}%
\pgfpathlineto{\pgfqpoint{4.148330in}{1.057887in}}%
\pgfusepath{stroke}%
\end{pgfscope}%
\begin{pgfscope}%
\pgfsetbuttcap%
\pgfsetroundjoin%
\definecolor{currentfill}{rgb}{0.000000,0.000000,0.000000}%
\pgfsetfillcolor{currentfill}%
\pgfsetlinewidth{0.602250pt}%
\definecolor{currentstroke}{rgb}{0.000000,0.000000,0.000000}%
\pgfsetstrokecolor{currentstroke}%
\pgfsetdash{}{0pt}%
\pgfsys@defobject{currentmarker}{\pgfqpoint{-0.027778in}{0.000000in}}{\pgfqpoint{-0.000000in}{0.000000in}}{%
\pgfpathmoveto{\pgfqpoint{-0.000000in}{0.000000in}}%
\pgfpathlineto{\pgfqpoint{-0.027778in}{0.000000in}}%
\pgfusepath{stroke,fill}%
}%
\begin{pgfscope}%
\pgfsys@transformshift{0.589510in}{1.057887in}%
\pgfsys@useobject{currentmarker}{}%
\end{pgfscope}%
\end{pgfscope}%
\begin{pgfscope}%
\pgfpathrectangle{\pgfqpoint{0.589510in}{0.417642in}}{\pgfqpoint{3.558820in}{2.050688in}}%
\pgfusepath{clip}%
\pgfsetrectcap%
\pgfsetroundjoin%
\pgfsetlinewidth{0.803000pt}%
\definecolor{currentstroke}{rgb}{0.850000,0.850000,0.850000}%
\pgfsetstrokecolor{currentstroke}%
\pgfsetdash{}{0pt}%
\pgfpathmoveto{\pgfqpoint{0.589510in}{1.118703in}}%
\pgfpathlineto{\pgfqpoint{4.148330in}{1.118703in}}%
\pgfusepath{stroke}%
\end{pgfscope}%
\begin{pgfscope}%
\pgfsetbuttcap%
\pgfsetroundjoin%
\definecolor{currentfill}{rgb}{0.000000,0.000000,0.000000}%
\pgfsetfillcolor{currentfill}%
\pgfsetlinewidth{0.602250pt}%
\definecolor{currentstroke}{rgb}{0.000000,0.000000,0.000000}%
\pgfsetstrokecolor{currentstroke}%
\pgfsetdash{}{0pt}%
\pgfsys@defobject{currentmarker}{\pgfqpoint{-0.027778in}{0.000000in}}{\pgfqpoint{-0.000000in}{0.000000in}}{%
\pgfpathmoveto{\pgfqpoint{-0.000000in}{0.000000in}}%
\pgfpathlineto{\pgfqpoint{-0.027778in}{0.000000in}}%
\pgfusepath{stroke,fill}%
}%
\begin{pgfscope}%
\pgfsys@transformshift{0.589510in}{1.118703in}%
\pgfsys@useobject{currentmarker}{}%
\end{pgfscope}%
\end{pgfscope}%
\begin{pgfscope}%
\pgfpathrectangle{\pgfqpoint{0.589510in}{0.417642in}}{\pgfqpoint{3.558820in}{2.050688in}}%
\pgfusepath{clip}%
\pgfsetrectcap%
\pgfsetroundjoin%
\pgfsetlinewidth{0.803000pt}%
\definecolor{currentstroke}{rgb}{0.850000,0.850000,0.850000}%
\pgfsetstrokecolor{currentstroke}%
\pgfsetdash{}{0pt}%
\pgfpathmoveto{\pgfqpoint{0.589510in}{1.172347in}}%
\pgfpathlineto{\pgfqpoint{4.148330in}{1.172347in}}%
\pgfusepath{stroke}%
\end{pgfscope}%
\begin{pgfscope}%
\pgfsetbuttcap%
\pgfsetroundjoin%
\definecolor{currentfill}{rgb}{0.000000,0.000000,0.000000}%
\pgfsetfillcolor{currentfill}%
\pgfsetlinewidth{0.602250pt}%
\definecolor{currentstroke}{rgb}{0.000000,0.000000,0.000000}%
\pgfsetstrokecolor{currentstroke}%
\pgfsetdash{}{0pt}%
\pgfsys@defobject{currentmarker}{\pgfqpoint{-0.027778in}{0.000000in}}{\pgfqpoint{-0.000000in}{0.000000in}}{%
\pgfpathmoveto{\pgfqpoint{-0.000000in}{0.000000in}}%
\pgfpathlineto{\pgfqpoint{-0.027778in}{0.000000in}}%
\pgfusepath{stroke,fill}%
}%
\begin{pgfscope}%
\pgfsys@transformshift{0.589510in}{1.172347in}%
\pgfsys@useobject{currentmarker}{}%
\end{pgfscope}%
\end{pgfscope}%
\begin{pgfscope}%
\pgfpathrectangle{\pgfqpoint{0.589510in}{0.417642in}}{\pgfqpoint{3.558820in}{2.050688in}}%
\pgfusepath{clip}%
\pgfsetrectcap%
\pgfsetroundjoin%
\pgfsetlinewidth{0.803000pt}%
\definecolor{currentstroke}{rgb}{0.850000,0.850000,0.850000}%
\pgfsetstrokecolor{currentstroke}%
\pgfsetdash{}{0pt}%
\pgfpathmoveto{\pgfqpoint{0.589510in}{1.536024in}}%
\pgfpathlineto{\pgfqpoint{4.148330in}{1.536024in}}%
\pgfusepath{stroke}%
\end{pgfscope}%
\begin{pgfscope}%
\pgfsetbuttcap%
\pgfsetroundjoin%
\definecolor{currentfill}{rgb}{0.000000,0.000000,0.000000}%
\pgfsetfillcolor{currentfill}%
\pgfsetlinewidth{0.602250pt}%
\definecolor{currentstroke}{rgb}{0.000000,0.000000,0.000000}%
\pgfsetstrokecolor{currentstroke}%
\pgfsetdash{}{0pt}%
\pgfsys@defobject{currentmarker}{\pgfqpoint{-0.027778in}{0.000000in}}{\pgfqpoint{-0.000000in}{0.000000in}}{%
\pgfpathmoveto{\pgfqpoint{-0.000000in}{0.000000in}}%
\pgfpathlineto{\pgfqpoint{-0.027778in}{0.000000in}}%
\pgfusepath{stroke,fill}%
}%
\begin{pgfscope}%
\pgfsys@transformshift{0.589510in}{1.536024in}%
\pgfsys@useobject{currentmarker}{}%
\end{pgfscope}%
\end{pgfscope}%
\begin{pgfscope}%
\pgfpathrectangle{\pgfqpoint{0.589510in}{0.417642in}}{\pgfqpoint{3.558820in}{2.050688in}}%
\pgfusepath{clip}%
\pgfsetrectcap%
\pgfsetroundjoin%
\pgfsetlinewidth{0.803000pt}%
\definecolor{currentstroke}{rgb}{0.850000,0.850000,0.850000}%
\pgfsetstrokecolor{currentstroke}%
\pgfsetdash{}{0pt}%
\pgfpathmoveto{\pgfqpoint{0.589510in}{1.720691in}}%
\pgfpathlineto{\pgfqpoint{4.148330in}{1.720691in}}%
\pgfusepath{stroke}%
\end{pgfscope}%
\begin{pgfscope}%
\pgfsetbuttcap%
\pgfsetroundjoin%
\definecolor{currentfill}{rgb}{0.000000,0.000000,0.000000}%
\pgfsetfillcolor{currentfill}%
\pgfsetlinewidth{0.602250pt}%
\definecolor{currentstroke}{rgb}{0.000000,0.000000,0.000000}%
\pgfsetstrokecolor{currentstroke}%
\pgfsetdash{}{0pt}%
\pgfsys@defobject{currentmarker}{\pgfqpoint{-0.027778in}{0.000000in}}{\pgfqpoint{-0.000000in}{0.000000in}}{%
\pgfpathmoveto{\pgfqpoint{-0.000000in}{0.000000in}}%
\pgfpathlineto{\pgfqpoint{-0.027778in}{0.000000in}}%
\pgfusepath{stroke,fill}%
}%
\begin{pgfscope}%
\pgfsys@transformshift{0.589510in}{1.720691in}%
\pgfsys@useobject{currentmarker}{}%
\end{pgfscope}%
\end{pgfscope}%
\begin{pgfscope}%
\pgfpathrectangle{\pgfqpoint{0.589510in}{0.417642in}}{\pgfqpoint{3.558820in}{2.050688in}}%
\pgfusepath{clip}%
\pgfsetrectcap%
\pgfsetroundjoin%
\pgfsetlinewidth{0.803000pt}%
\definecolor{currentstroke}{rgb}{0.850000,0.850000,0.850000}%
\pgfsetstrokecolor{currentstroke}%
\pgfsetdash{}{0pt}%
\pgfpathmoveto{\pgfqpoint{0.589510in}{1.851714in}}%
\pgfpathlineto{\pgfqpoint{4.148330in}{1.851714in}}%
\pgfusepath{stroke}%
\end{pgfscope}%
\begin{pgfscope}%
\pgfsetbuttcap%
\pgfsetroundjoin%
\definecolor{currentfill}{rgb}{0.000000,0.000000,0.000000}%
\pgfsetfillcolor{currentfill}%
\pgfsetlinewidth{0.602250pt}%
\definecolor{currentstroke}{rgb}{0.000000,0.000000,0.000000}%
\pgfsetstrokecolor{currentstroke}%
\pgfsetdash{}{0pt}%
\pgfsys@defobject{currentmarker}{\pgfqpoint{-0.027778in}{0.000000in}}{\pgfqpoint{-0.000000in}{0.000000in}}{%
\pgfpathmoveto{\pgfqpoint{-0.000000in}{0.000000in}}%
\pgfpathlineto{\pgfqpoint{-0.027778in}{0.000000in}}%
\pgfusepath{stroke,fill}%
}%
\begin{pgfscope}%
\pgfsys@transformshift{0.589510in}{1.851714in}%
\pgfsys@useobject{currentmarker}{}%
\end{pgfscope}%
\end{pgfscope}%
\begin{pgfscope}%
\pgfpathrectangle{\pgfqpoint{0.589510in}{0.417642in}}{\pgfqpoint{3.558820in}{2.050688in}}%
\pgfusepath{clip}%
\pgfsetrectcap%
\pgfsetroundjoin%
\pgfsetlinewidth{0.803000pt}%
\definecolor{currentstroke}{rgb}{0.850000,0.850000,0.850000}%
\pgfsetstrokecolor{currentstroke}%
\pgfsetdash{}{0pt}%
\pgfpathmoveto{\pgfqpoint{0.589510in}{1.953344in}}%
\pgfpathlineto{\pgfqpoint{4.148330in}{1.953344in}}%
\pgfusepath{stroke}%
\end{pgfscope}%
\begin{pgfscope}%
\pgfsetbuttcap%
\pgfsetroundjoin%
\definecolor{currentfill}{rgb}{0.000000,0.000000,0.000000}%
\pgfsetfillcolor{currentfill}%
\pgfsetlinewidth{0.602250pt}%
\definecolor{currentstroke}{rgb}{0.000000,0.000000,0.000000}%
\pgfsetstrokecolor{currentstroke}%
\pgfsetdash{}{0pt}%
\pgfsys@defobject{currentmarker}{\pgfqpoint{-0.027778in}{0.000000in}}{\pgfqpoint{-0.000000in}{0.000000in}}{%
\pgfpathmoveto{\pgfqpoint{-0.000000in}{0.000000in}}%
\pgfpathlineto{\pgfqpoint{-0.027778in}{0.000000in}}%
\pgfusepath{stroke,fill}%
}%
\begin{pgfscope}%
\pgfsys@transformshift{0.589510in}{1.953344in}%
\pgfsys@useobject{currentmarker}{}%
\end{pgfscope}%
\end{pgfscope}%
\begin{pgfscope}%
\pgfpathrectangle{\pgfqpoint{0.589510in}{0.417642in}}{\pgfqpoint{3.558820in}{2.050688in}}%
\pgfusepath{clip}%
\pgfsetrectcap%
\pgfsetroundjoin%
\pgfsetlinewidth{0.803000pt}%
\definecolor{currentstroke}{rgb}{0.850000,0.850000,0.850000}%
\pgfsetstrokecolor{currentstroke}%
\pgfsetdash{}{0pt}%
\pgfpathmoveto{\pgfqpoint{0.589510in}{2.036382in}}%
\pgfpathlineto{\pgfqpoint{4.148330in}{2.036382in}}%
\pgfusepath{stroke}%
\end{pgfscope}%
\begin{pgfscope}%
\pgfsetbuttcap%
\pgfsetroundjoin%
\definecolor{currentfill}{rgb}{0.000000,0.000000,0.000000}%
\pgfsetfillcolor{currentfill}%
\pgfsetlinewidth{0.602250pt}%
\definecolor{currentstroke}{rgb}{0.000000,0.000000,0.000000}%
\pgfsetstrokecolor{currentstroke}%
\pgfsetdash{}{0pt}%
\pgfsys@defobject{currentmarker}{\pgfqpoint{-0.027778in}{0.000000in}}{\pgfqpoint{-0.000000in}{0.000000in}}{%
\pgfpathmoveto{\pgfqpoint{-0.000000in}{0.000000in}}%
\pgfpathlineto{\pgfqpoint{-0.027778in}{0.000000in}}%
\pgfusepath{stroke,fill}%
}%
\begin{pgfscope}%
\pgfsys@transformshift{0.589510in}{2.036382in}%
\pgfsys@useobject{currentmarker}{}%
\end{pgfscope}%
\end{pgfscope}%
\begin{pgfscope}%
\pgfpathrectangle{\pgfqpoint{0.589510in}{0.417642in}}{\pgfqpoint{3.558820in}{2.050688in}}%
\pgfusepath{clip}%
\pgfsetrectcap%
\pgfsetroundjoin%
\pgfsetlinewidth{0.803000pt}%
\definecolor{currentstroke}{rgb}{0.850000,0.850000,0.850000}%
\pgfsetstrokecolor{currentstroke}%
\pgfsetdash{}{0pt}%
\pgfpathmoveto{\pgfqpoint{0.589510in}{2.106589in}}%
\pgfpathlineto{\pgfqpoint{4.148330in}{2.106589in}}%
\pgfusepath{stroke}%
\end{pgfscope}%
\begin{pgfscope}%
\pgfsetbuttcap%
\pgfsetroundjoin%
\definecolor{currentfill}{rgb}{0.000000,0.000000,0.000000}%
\pgfsetfillcolor{currentfill}%
\pgfsetlinewidth{0.602250pt}%
\definecolor{currentstroke}{rgb}{0.000000,0.000000,0.000000}%
\pgfsetstrokecolor{currentstroke}%
\pgfsetdash{}{0pt}%
\pgfsys@defobject{currentmarker}{\pgfqpoint{-0.027778in}{0.000000in}}{\pgfqpoint{-0.000000in}{0.000000in}}{%
\pgfpathmoveto{\pgfqpoint{-0.000000in}{0.000000in}}%
\pgfpathlineto{\pgfqpoint{-0.027778in}{0.000000in}}%
\pgfusepath{stroke,fill}%
}%
\begin{pgfscope}%
\pgfsys@transformshift{0.589510in}{2.106589in}%
\pgfsys@useobject{currentmarker}{}%
\end{pgfscope}%
\end{pgfscope}%
\begin{pgfscope}%
\pgfpathrectangle{\pgfqpoint{0.589510in}{0.417642in}}{\pgfqpoint{3.558820in}{2.050688in}}%
\pgfusepath{clip}%
\pgfsetrectcap%
\pgfsetroundjoin%
\pgfsetlinewidth{0.803000pt}%
\definecolor{currentstroke}{rgb}{0.850000,0.850000,0.850000}%
\pgfsetstrokecolor{currentstroke}%
\pgfsetdash{}{0pt}%
\pgfpathmoveto{\pgfqpoint{0.589510in}{2.167405in}}%
\pgfpathlineto{\pgfqpoint{4.148330in}{2.167405in}}%
\pgfusepath{stroke}%
\end{pgfscope}%
\begin{pgfscope}%
\pgfsetbuttcap%
\pgfsetroundjoin%
\definecolor{currentfill}{rgb}{0.000000,0.000000,0.000000}%
\pgfsetfillcolor{currentfill}%
\pgfsetlinewidth{0.602250pt}%
\definecolor{currentstroke}{rgb}{0.000000,0.000000,0.000000}%
\pgfsetstrokecolor{currentstroke}%
\pgfsetdash{}{0pt}%
\pgfsys@defobject{currentmarker}{\pgfqpoint{-0.027778in}{0.000000in}}{\pgfqpoint{-0.000000in}{0.000000in}}{%
\pgfpathmoveto{\pgfqpoint{-0.000000in}{0.000000in}}%
\pgfpathlineto{\pgfqpoint{-0.027778in}{0.000000in}}%
\pgfusepath{stroke,fill}%
}%
\begin{pgfscope}%
\pgfsys@transformshift{0.589510in}{2.167405in}%
\pgfsys@useobject{currentmarker}{}%
\end{pgfscope}%
\end{pgfscope}%
\begin{pgfscope}%
\pgfpathrectangle{\pgfqpoint{0.589510in}{0.417642in}}{\pgfqpoint{3.558820in}{2.050688in}}%
\pgfusepath{clip}%
\pgfsetrectcap%
\pgfsetroundjoin%
\pgfsetlinewidth{0.803000pt}%
\definecolor{currentstroke}{rgb}{0.850000,0.850000,0.850000}%
\pgfsetstrokecolor{currentstroke}%
\pgfsetdash{}{0pt}%
\pgfpathmoveto{\pgfqpoint{0.589510in}{2.221049in}}%
\pgfpathlineto{\pgfqpoint{4.148330in}{2.221049in}}%
\pgfusepath{stroke}%
\end{pgfscope}%
\begin{pgfscope}%
\pgfsetbuttcap%
\pgfsetroundjoin%
\definecolor{currentfill}{rgb}{0.000000,0.000000,0.000000}%
\pgfsetfillcolor{currentfill}%
\pgfsetlinewidth{0.602250pt}%
\definecolor{currentstroke}{rgb}{0.000000,0.000000,0.000000}%
\pgfsetstrokecolor{currentstroke}%
\pgfsetdash{}{0pt}%
\pgfsys@defobject{currentmarker}{\pgfqpoint{-0.027778in}{0.000000in}}{\pgfqpoint{-0.000000in}{0.000000in}}{%
\pgfpathmoveto{\pgfqpoint{-0.000000in}{0.000000in}}%
\pgfpathlineto{\pgfqpoint{-0.027778in}{0.000000in}}%
\pgfusepath{stroke,fill}%
}%
\begin{pgfscope}%
\pgfsys@transformshift{0.589510in}{2.221049in}%
\pgfsys@useobject{currentmarker}{}%
\end{pgfscope}%
\end{pgfscope}%
\begin{pgfscope}%
\definecolor{textcolor}{rgb}{0.000000,0.000000,0.000000}%
\pgfsetstrokecolor{textcolor}%
\pgfsetfillcolor{textcolor}%
\pgftext[x=0.180559in,y=1.442986in,,bottom,rotate=90.000000]{\color{textcolor}\rmfamily\fontsize{10.000000}{12.000000}\selectfont ADEV \(\displaystyle \sigma_A(\tau)\)}%
\end{pgfscope}%
\begin{pgfscope}%
\pgfpathrectangle{\pgfqpoint{0.589510in}{0.417642in}}{\pgfqpoint{3.558820in}{2.050688in}}%
\pgfusepath{clip}%
\pgfsetrectcap%
\pgfsetroundjoin%
\pgfsetlinewidth{1.505625pt}%
\definecolor{currentstroke}{rgb}{0.121569,0.466667,0.705882}%
\pgfsetstrokecolor{currentstroke}%
\pgfsetdash{}{0pt}%
\pgfpathmoveto{\pgfqpoint{1.238234in}{2.375117in}}%
\pgfpathlineto{\pgfqpoint{1.481714in}{2.215135in}}%
\pgfpathlineto{\pgfqpoint{1.624141in}{2.121675in}}%
\pgfpathlineto{\pgfqpoint{1.725194in}{2.055351in}}%
\pgfpathlineto{\pgfqpoint{1.803577in}{2.004091in}}%
\pgfpathlineto{\pgfqpoint{1.867621in}{1.962366in}}%
\pgfpathlineto{\pgfqpoint{1.921769in}{1.927137in}}%
\pgfpathlineto{\pgfqpoint{1.968674in}{1.896660in}}%
\pgfpathlineto{\pgfqpoint{2.047057in}{1.845639in}}%
\pgfpathlineto{\pgfqpoint{2.111101in}{1.803997in}}%
\pgfpathlineto{\pgfqpoint{2.189484in}{1.752851in}}%
\pgfpathlineto{\pgfqpoint{2.253527in}{1.710983in}}%
\pgfpathlineto{\pgfqpoint{2.324016in}{1.665004in}}%
\pgfpathlineto{\pgfqpoint{2.395954in}{1.618281in}}%
\pgfpathlineto{\pgfqpoint{2.466443in}{1.572855in}}%
\pgfpathlineto{\pgfqpoint{2.525123in}{1.534998in}}%
\pgfpathlineto{\pgfqpoint{2.598060in}{1.487871in}}%
\pgfpathlineto{\pgfqpoint{2.664535in}{1.444553in}}%
\pgfpathlineto{\pgfqpoint{2.730591in}{1.401296in}}%
\pgfpathlineto{\pgfqpoint{2.794635in}{1.359348in}}%
\pgfpathlineto{\pgfqpoint{2.862836in}{1.314734in}}%
\pgfpathlineto{\pgfqpoint{2.928597in}{1.272099in}}%
\pgfpathlineto{\pgfqpoint{2.995957in}{1.228376in}}%
\pgfpathlineto{\pgfqpoint{3.060393in}{1.186746in}}%
\pgfpathlineto{\pgfqpoint{3.128016in}{1.142924in}}%
\pgfpathlineto{\pgfqpoint{3.194211in}{1.099648in}}%
\pgfpathlineto{\pgfqpoint{3.260038in}{1.056396in}}%
\pgfpathlineto{\pgfqpoint{3.325745in}{1.012620in}}%
\pgfpathlineto{\pgfqpoint{3.391933in}{0.969355in}}%
\pgfpathlineto{\pgfqpoint{3.457881in}{0.926450in}}%
\pgfpathlineto{\pgfqpoint{3.524028in}{0.883532in}}%
\pgfpathlineto{\pgfqpoint{3.590249in}{0.840276in}}%
\pgfpathlineto{\pgfqpoint{3.656169in}{0.796273in}}%
\pgfpathlineto{\pgfqpoint{3.722246in}{0.752617in}}%
\pgfpathlineto{\pgfqpoint{3.788371in}{0.709864in}}%
\pgfpathlineto{\pgfqpoint{3.854385in}{0.667163in}}%
\pgfpathlineto{\pgfqpoint{3.920436in}{0.625681in}}%
\pgfpathlineto{\pgfqpoint{3.986565in}{0.584100in}}%
\pgfusepath{stroke}%
\end{pgfscope}%
\begin{pgfscope}%
\pgfpathrectangle{\pgfqpoint{0.589510in}{0.417642in}}{\pgfqpoint{3.558820in}{2.050688in}}%
\pgfusepath{clip}%
\pgfsetrectcap%
\pgfsetroundjoin%
\pgfsetlinewidth{1.505625pt}%
\definecolor{currentstroke}{rgb}{1.000000,0.498039,0.054902}%
\pgfsetstrokecolor{currentstroke}%
\pgfsetdash{}{0pt}%
\pgfpathmoveto{\pgfqpoint{1.380661in}{2.234627in}}%
\pgfpathlineto{\pgfqpoint{1.624141in}{2.073029in}}%
\pgfpathlineto{\pgfqpoint{1.766567in}{1.978780in}}%
\pgfpathlineto{\pgfqpoint{1.867621in}{1.911958in}}%
\pgfpathlineto{\pgfqpoint{1.946004in}{1.860075in}}%
\pgfpathlineto{\pgfqpoint{2.064195in}{1.781872in}}%
\pgfpathlineto{\pgfqpoint{2.111101in}{1.750982in}}%
\pgfpathlineto{\pgfqpoint{2.189484in}{1.699450in}}%
\pgfpathlineto{\pgfqpoint{2.253527in}{1.657428in}}%
\pgfpathlineto{\pgfqpoint{2.331910in}{1.606179in}}%
\pgfpathlineto{\pgfqpoint{2.395954in}{1.564262in}}%
\pgfpathlineto{\pgfqpoint{2.466443in}{1.518196in}}%
\pgfpathlineto{\pgfqpoint{2.525123in}{1.480225in}}%
\pgfpathlineto{\pgfqpoint{2.598060in}{1.433030in}}%
\pgfpathlineto{\pgfqpoint{2.658426in}{1.394115in}}%
\pgfpathlineto{\pgfqpoint{2.725537in}{1.351103in}}%
\pgfpathlineto{\pgfqpoint{2.794635in}{1.306515in}}%
\pgfpathlineto{\pgfqpoint{2.862836in}{1.262232in}}%
\pgfpathlineto{\pgfqpoint{2.928597in}{1.218739in}}%
\pgfpathlineto{\pgfqpoint{2.994776in}{1.174157in}}%
\pgfpathlineto{\pgfqpoint{3.059410in}{1.130988in}}%
\pgfpathlineto{\pgfqpoint{3.126394in}{1.087552in}}%
\pgfpathlineto{\pgfqpoint{3.192868in}{1.045616in}}%
\pgfpathlineto{\pgfqpoint{3.258925in}{1.004293in}}%
\pgfpathlineto{\pgfqpoint{3.325745in}{0.960900in}}%
\pgfpathlineto{\pgfqpoint{3.392315in}{0.916418in}}%
\pgfpathlineto{\pgfqpoint{3.457881in}{0.873040in}}%
\pgfpathlineto{\pgfqpoint{3.524290in}{0.829733in}}%
\pgfpathlineto{\pgfqpoint{3.590032in}{0.788062in}}%
\pgfpathlineto{\pgfqpoint{3.656349in}{0.747707in}}%
\pgfpathlineto{\pgfqpoint{3.722097in}{0.707272in}}%
\pgfpathlineto{\pgfqpoint{3.788371in}{0.668206in}}%
\pgfpathlineto{\pgfqpoint{3.854385in}{0.628274in}}%
\pgfpathlineto{\pgfqpoint{3.920521in}{0.588333in}}%
\pgfpathlineto{\pgfqpoint{3.986425in}{0.547299in}}%
\pgfusepath{stroke}%
\end{pgfscope}%
\begin{pgfscope}%
\pgfpathrectangle{\pgfqpoint{0.589510in}{0.417642in}}{\pgfqpoint{3.558820in}{2.050688in}}%
\pgfusepath{clip}%
\pgfsetrectcap%
\pgfsetroundjoin%
\pgfsetlinewidth{1.505625pt}%
\definecolor{currentstroke}{rgb}{0.172549,0.627451,0.172549}%
\pgfsetstrokecolor{currentstroke}%
\pgfsetdash{}{0pt}%
\pgfpathmoveto{\pgfqpoint{1.624141in}{2.071775in}}%
\pgfpathlineto{\pgfqpoint{1.867621in}{1.904442in}}%
\pgfpathlineto{\pgfqpoint{2.010047in}{1.807239in}}%
\pgfpathlineto{\pgfqpoint{2.111101in}{1.738907in}}%
\pgfpathlineto{\pgfqpoint{2.189484in}{1.685970in}}%
\pgfpathlineto{\pgfqpoint{2.253527in}{1.642753in}}%
\pgfpathlineto{\pgfqpoint{2.307675in}{1.606249in}}%
\pgfpathlineto{\pgfqpoint{2.395954in}{1.547295in}}%
\pgfpathlineto{\pgfqpoint{2.466443in}{1.500402in}}%
\pgfpathlineto{\pgfqpoint{2.525123in}{1.461967in}}%
\pgfpathlineto{\pgfqpoint{2.598060in}{1.414517in}}%
\pgfpathlineto{\pgfqpoint{2.658426in}{1.375169in}}%
\pgfpathlineto{\pgfqpoint{2.725537in}{1.331228in}}%
\pgfpathlineto{\pgfqpoint{2.794635in}{1.285221in}}%
\pgfpathlineto{\pgfqpoint{2.862836in}{1.239319in}}%
\pgfpathlineto{\pgfqpoint{2.928597in}{1.195288in}}%
\pgfpathlineto{\pgfqpoint{2.991210in}{1.153860in}}%
\pgfpathlineto{\pgfqpoint{3.056446in}{1.110513in}}%
\pgfpathlineto{\pgfqpoint{3.126394in}{1.065377in}}%
\pgfpathlineto{\pgfqpoint{3.192868in}{1.022557in}}%
\pgfpathlineto{\pgfqpoint{3.258925in}{0.980354in}}%
\pgfpathlineto{\pgfqpoint{3.325745in}{0.937935in}}%
\pgfpathlineto{\pgfqpoint{3.391169in}{0.895843in}}%
\pgfpathlineto{\pgfqpoint{3.457881in}{0.852925in}}%
\pgfpathlineto{\pgfqpoint{3.523503in}{0.810610in}}%
\pgfpathlineto{\pgfqpoint{3.589380in}{0.767052in}}%
\pgfpathlineto{\pgfqpoint{3.655809in}{0.723612in}}%
\pgfpathlineto{\pgfqpoint{3.721650in}{0.679728in}}%
\pgfpathlineto{\pgfqpoint{3.788371in}{0.636288in}}%
\pgfpathlineto{\pgfqpoint{3.854385in}{0.594932in}}%
\pgfpathlineto{\pgfqpoint{3.920267in}{0.553152in}}%
\pgfpathlineto{\pgfqpoint{3.986425in}{0.510855in}}%
\pgfusepath{stroke}%
\end{pgfscope}%
\begin{pgfscope}%
\pgfpathrectangle{\pgfqpoint{0.589510in}{0.417642in}}{\pgfqpoint{3.558820in}{2.050688in}}%
\pgfusepath{clip}%
\pgfsetrectcap%
\pgfsetroundjoin%
\pgfsetlinewidth{1.505625pt}%
\definecolor{currentstroke}{rgb}{0.839216,0.152941,0.156863}%
\pgfsetstrokecolor{currentstroke}%
\pgfsetdash{}{0pt}%
\pgfpathmoveto{\pgfqpoint{1.837056in}{1.974334in}}%
\pgfpathlineto{\pgfqpoint{2.080536in}{1.800374in}}%
\pgfpathlineto{\pgfqpoint{2.222963in}{1.700411in}}%
\pgfpathlineto{\pgfqpoint{2.324016in}{1.630732in}}%
\pgfpathlineto{\pgfqpoint{2.466443in}{1.533101in}}%
\pgfpathlineto{\pgfqpoint{2.520591in}{1.496058in}}%
\pgfpathlineto{\pgfqpoint{2.567496in}{1.464236in}}%
\pgfpathlineto{\pgfqpoint{2.645879in}{1.411361in}}%
\pgfpathlineto{\pgfqpoint{2.709923in}{1.368336in}}%
\pgfpathlineto{\pgfqpoint{2.788306in}{1.315937in}}%
\pgfpathlineto{\pgfqpoint{2.852349in}{1.273211in}}%
\pgfpathlineto{\pgfqpoint{2.922838in}{1.226380in}}%
\pgfpathlineto{\pgfqpoint{2.994776in}{1.179009in}}%
\pgfpathlineto{\pgfqpoint{3.054456in}{1.139691in}}%
\pgfpathlineto{\pgfqpoint{3.123946in}{1.093184in}}%
\pgfpathlineto{\pgfqpoint{3.189487in}{1.049831in}}%
\pgfpathlineto{\pgfqpoint{3.257248in}{1.004769in}}%
\pgfpathlineto{\pgfqpoint{3.324359in}{0.961301in}}%
\pgfpathlineto{\pgfqpoint{3.389250in}{0.919346in}}%
\pgfpathlineto{\pgfqpoint{3.458197in}{0.873797in}}%
\pgfpathlineto{\pgfqpoint{3.521661in}{0.832417in}}%
\pgfpathlineto{\pgfqpoint{3.590032in}{0.788556in}}%
\pgfpathlineto{\pgfqpoint{3.655268in}{0.747114in}}%
\pgfpathlineto{\pgfqpoint{3.721948in}{0.705607in}}%
\pgfpathlineto{\pgfqpoint{3.787629in}{0.664167in}}%
\pgfpathlineto{\pgfqpoint{3.854385in}{0.621367in}}%
\pgfpathlineto{\pgfqpoint{3.919927in}{0.580988in}}%
\pgfpathlineto{\pgfqpoint{3.986144in}{0.538149in}}%
\pgfusepath{stroke}%
\end{pgfscope}%
\begin{pgfscope}%
\pgfpathrectangle{\pgfqpoint{0.589510in}{0.417642in}}{\pgfqpoint{3.558820in}{2.050688in}}%
\pgfusepath{clip}%
\pgfsetrectcap%
\pgfsetroundjoin%
\pgfsetlinewidth{1.505625pt}%
\definecolor{currentstroke}{rgb}{0.580392,0.403922,0.741176}%
\pgfsetstrokecolor{currentstroke}%
\pgfsetdash{}{0pt}%
\pgfpathmoveto{\pgfqpoint{2.064195in}{1.904741in}}%
\pgfpathlineto{\pgfqpoint{2.307675in}{1.725488in}}%
\pgfpathlineto{\pgfqpoint{2.450102in}{1.621672in}}%
\pgfpathlineto{\pgfqpoint{2.551155in}{1.549319in}}%
\pgfpathlineto{\pgfqpoint{2.629538in}{1.493388in}}%
\pgfpathlineto{\pgfqpoint{2.693582in}{1.448688in}}%
\pgfpathlineto{\pgfqpoint{2.794635in}{1.377410in}}%
\pgfpathlineto{\pgfqpoint{2.836008in}{1.348386in}}%
\pgfpathlineto{\pgfqpoint{2.906497in}{1.299435in}}%
\pgfpathlineto{\pgfqpoint{2.991210in}{1.241334in}}%
\pgfpathlineto{\pgfqpoint{3.059410in}{1.194659in}}%
\pgfpathlineto{\pgfqpoint{3.116498in}{1.156427in}}%
\pgfpathlineto{\pgfqpoint{3.180542in}{1.113930in}}%
\pgfpathlineto{\pgfqpoint{3.258925in}{1.061252in}}%
\pgfpathlineto{\pgfqpoint{3.322968in}{1.018904in}}%
\pgfpathlineto{\pgfqpoint{3.385382in}{0.977204in}}%
\pgfpathlineto{\pgfqpoint{3.452138in}{0.933016in}}%
\pgfpathlineto{\pgfqpoint{3.519543in}{0.888740in}}%
\pgfpathlineto{\pgfqpoint{3.590032in}{0.841767in}}%
\pgfpathlineto{\pgfqpoint{3.656349in}{0.797542in}}%
\pgfpathlineto{\pgfqpoint{3.721650in}{0.753789in}}%
\pgfpathlineto{\pgfqpoint{3.787258in}{0.708113in}}%
\pgfpathlineto{\pgfqpoint{3.853463in}{0.663005in}}%
\pgfpathlineto{\pgfqpoint{3.920012in}{0.618549in}}%
\pgfpathlineto{\pgfqpoint{3.986425in}{0.577533in}}%
\pgfusepath{stroke}%
\end{pgfscope}%
\begin{pgfscope}%
\pgfpathrectangle{\pgfqpoint{0.589510in}{0.417642in}}{\pgfqpoint{3.558820in}{2.050688in}}%
\pgfusepath{clip}%
\pgfsetrectcap%
\pgfsetroundjoin%
\pgfsetlinewidth{1.505625pt}%
\definecolor{currentstroke}{rgb}{0.549020,0.337255,0.294118}%
\pgfsetstrokecolor{currentstroke}%
\pgfsetdash{}{0pt}%
\pgfpathmoveto{\pgfqpoint{2.375876in}{1.846309in}}%
\pgfpathlineto{\pgfqpoint{2.619356in}{1.659467in}}%
\pgfpathlineto{\pgfqpoint{2.761782in}{1.551069in}}%
\pgfpathlineto{\pgfqpoint{2.862836in}{1.476160in}}%
\pgfpathlineto{\pgfqpoint{2.941219in}{1.420684in}}%
\pgfpathlineto{\pgfqpoint{3.059410in}{1.337779in}}%
\pgfpathlineto{\pgfqpoint{3.106316in}{1.304855in}}%
\pgfpathlineto{\pgfqpoint{3.184699in}{1.248824in}}%
\pgfpathlineto{\pgfqpoint{3.248742in}{1.203995in}}%
\pgfpathlineto{\pgfqpoint{3.302890in}{1.165309in}}%
\pgfpathlineto{\pgfqpoint{3.391169in}{1.102856in}}%
\pgfpathlineto{\pgfqpoint{3.445317in}{1.065292in}}%
\pgfpathlineto{\pgfqpoint{3.520339in}{1.014947in}}%
\pgfpathlineto{\pgfqpoint{3.582123in}{0.974435in}}%
\pgfpathlineto{\pgfqpoint{3.653641in}{0.926194in}}%
\pgfpathlineto{\pgfqpoint{3.720752in}{0.880174in}}%
\pgfpathlineto{\pgfqpoint{3.783521in}{0.836117in}}%
\pgfpathlineto{\pgfqpoint{3.852847in}{0.789123in}}%
\pgfpathlineto{\pgfqpoint{3.919502in}{0.745224in}}%
\pgfpathlineto{\pgfqpoint{3.986425in}{0.700586in}}%
\pgfusepath{stroke}%
\end{pgfscope}%
\begin{pgfscope}%
\pgfpathrectangle{\pgfqpoint{0.589510in}{0.417642in}}{\pgfqpoint{3.558820in}{2.050688in}}%
\pgfusepath{clip}%
\pgfsetbuttcap%
\pgfsetroundjoin%
\pgfsetlinewidth{1.505625pt}%
\definecolor{currentstroke}{rgb}{0.890196,0.466667,0.760784}%
\pgfsetstrokecolor{currentstroke}%
\pgfsetdash{{5.550000pt}{2.400000pt}}{0.000000pt}%
\pgfpathmoveto{\pgfqpoint{0.751274in}{2.207344in}}%
\pgfpathlineto{\pgfqpoint{0.994754in}{2.063713in}}%
\pgfpathlineto{\pgfqpoint{1.137181in}{1.985641in}}%
\pgfpathlineto{\pgfqpoint{1.238234in}{1.933944in}}%
\pgfpathlineto{\pgfqpoint{1.316617in}{1.896299in}}%
\pgfpathlineto{\pgfqpoint{1.380661in}{1.867413in}}%
\pgfpathlineto{\pgfqpoint{1.434809in}{1.844384in}}%
\pgfpathlineto{\pgfqpoint{1.523087in}{1.809611in}}%
\pgfpathlineto{\pgfqpoint{1.593576in}{1.784439in}}%
\pgfpathlineto{\pgfqpoint{1.652257in}{1.765209in}}%
\pgfpathlineto{\pgfqpoint{1.725194in}{1.743572in}}%
\pgfpathlineto{\pgfqpoint{1.803577in}{1.723088in}}%
\pgfpathlineto{\pgfqpoint{1.867621in}{1.708252in}}%
\pgfpathlineto{\pgfqpoint{1.934095in}{1.694536in}}%
\pgfpathlineto{\pgfqpoint{2.000152in}{1.682440in}}%
\pgfpathlineto{\pgfqpoint{2.064195in}{1.672147in}}%
\pgfpathlineto{\pgfqpoint{2.132396in}{1.662778in}}%
\pgfpathlineto{\pgfqpoint{2.201002in}{1.654845in}}%
\pgfpathlineto{\pgfqpoint{2.267867in}{1.648357in}}%
\pgfpathlineto{\pgfqpoint{2.335792in}{1.642737in}}%
\pgfpathlineto{\pgfqpoint{2.399191in}{1.638187in}}%
\pgfpathlineto{\pgfqpoint{2.466443in}{1.633919in}}%
\pgfpathlineto{\pgfqpoint{2.531814in}{1.630324in}}%
\pgfpathlineto{\pgfqpoint{2.599885in}{1.626969in}}%
\pgfpathlineto{\pgfqpoint{2.664535in}{1.624096in}}%
\pgfpathlineto{\pgfqpoint{2.731844in}{1.621635in}}%
\pgfpathlineto{\pgfqpoint{2.797757in}{1.619706in}}%
\pgfpathlineto{\pgfqpoint{2.863696in}{1.618496in}}%
\pgfpathlineto{\pgfqpoint{2.930022in}{1.617727in}}%
\pgfpathlineto{\pgfqpoint{2.995957in}{1.617060in}}%
\pgfpathlineto{\pgfqpoint{3.061862in}{1.616303in}}%
\pgfpathlineto{\pgfqpoint{3.128016in}{1.615473in}}%
\pgfpathlineto{\pgfqpoint{3.194211in}{1.614837in}}%
\pgfpathlineto{\pgfqpoint{3.260038in}{1.614665in}}%
\pgfpathlineto{\pgfqpoint{3.326206in}{1.614826in}}%
\pgfpathlineto{\pgfqpoint{3.392315in}{1.615245in}}%
\pgfpathlineto{\pgfqpoint{3.458197in}{1.615827in}}%
\pgfpathlineto{\pgfqpoint{3.524290in}{1.616559in}}%
\pgfpathlineto{\pgfqpoint{3.590358in}{1.617833in}}%
\pgfpathlineto{\pgfqpoint{3.656349in}{1.618945in}}%
\pgfpathlineto{\pgfqpoint{3.722395in}{1.619275in}}%
\pgfpathlineto{\pgfqpoint{3.788433in}{1.618841in}}%
\pgfpathlineto{\pgfqpoint{3.854488in}{1.617314in}}%
\pgfpathlineto{\pgfqpoint{3.920521in}{1.615837in}}%
\pgfpathlineto{\pgfqpoint{3.986565in}{1.614663in}}%
\pgfusepath{stroke}%
\end{pgfscope}%
\begin{pgfscope}%
\pgfsetrectcap%
\pgfsetmiterjoin%
\pgfsetlinewidth{0.803000pt}%
\definecolor{currentstroke}{rgb}{0.000000,0.000000,0.000000}%
\pgfsetstrokecolor{currentstroke}%
\pgfsetdash{}{0pt}%
\pgfpathmoveto{\pgfqpoint{0.589510in}{0.417642in}}%
\pgfpathlineto{\pgfqpoint{0.589510in}{2.468330in}}%
\pgfusepath{stroke}%
\end{pgfscope}%
\begin{pgfscope}%
\pgfsetrectcap%
\pgfsetmiterjoin%
\pgfsetlinewidth{0.803000pt}%
\definecolor{currentstroke}{rgb}{0.000000,0.000000,0.000000}%
\pgfsetstrokecolor{currentstroke}%
\pgfsetdash{}{0pt}%
\pgfpathmoveto{\pgfqpoint{4.148330in}{0.417642in}}%
\pgfpathlineto{\pgfqpoint{4.148330in}{2.468330in}}%
\pgfusepath{stroke}%
\end{pgfscope}%
\begin{pgfscope}%
\pgfsetrectcap%
\pgfsetmiterjoin%
\pgfsetlinewidth{0.803000pt}%
\definecolor{currentstroke}{rgb}{0.000000,0.000000,0.000000}%
\pgfsetstrokecolor{currentstroke}%
\pgfsetdash{}{0pt}%
\pgfpathmoveto{\pgfqpoint{0.589510in}{0.417642in}}%
\pgfpathlineto{\pgfqpoint{4.148330in}{0.417642in}}%
\pgfusepath{stroke}%
\end{pgfscope}%
\begin{pgfscope}%
\pgfsetrectcap%
\pgfsetmiterjoin%
\pgfsetlinewidth{0.803000pt}%
\definecolor{currentstroke}{rgb}{0.000000,0.000000,0.000000}%
\pgfsetstrokecolor{currentstroke}%
\pgfsetdash{}{0pt}%
\pgfpathmoveto{\pgfqpoint{0.589510in}{2.468330in}}%
\pgfpathlineto{\pgfqpoint{4.148330in}{2.468330in}}%
\pgfusepath{stroke}%
\end{pgfscope}%
\begin{pgfscope}%
\pgfsetbuttcap%
\pgfsetmiterjoin%
\definecolor{currentfill}{rgb}{1.000000,1.000000,1.000000}%
\pgfsetfillcolor{currentfill}%
\pgfsetfillopacity{0.800000}%
\pgfsetlinewidth{1.003750pt}%
\definecolor{currentstroke}{rgb}{0.800000,0.800000,0.800000}%
\pgfsetstrokecolor{currentstroke}%
\pgfsetstrokeopacity{0.800000}%
\pgfsetdash{}{0pt}%
\pgfpathmoveto{\pgfqpoint{3.229885in}{1.450109in}}%
\pgfpathlineto{\pgfqpoint{4.070552in}{1.450109in}}%
\pgfpathquadraticcurveto{\pgfqpoint{4.092774in}{1.450109in}}{\pgfqpoint{4.092774in}{1.472331in}}%
\pgfpathlineto{\pgfqpoint{4.092774in}{2.390552in}}%
\pgfpathquadraticcurveto{\pgfqpoint{4.092774in}{2.412774in}}{\pgfqpoint{4.070552in}{2.412774in}}%
\pgfpathlineto{\pgfqpoint{3.229885in}{2.412774in}}%
\pgfpathquadraticcurveto{\pgfqpoint{3.207663in}{2.412774in}}{\pgfqpoint{3.207663in}{2.390552in}}%
\pgfpathlineto{\pgfqpoint{3.207663in}{1.472331in}}%
\pgfpathquadraticcurveto{\pgfqpoint{3.207663in}{1.450109in}}{\pgfqpoint{3.229885in}{1.450109in}}%
\pgfpathlineto{\pgfqpoint{3.229885in}{1.450109in}}%
\pgfpathclose%
\pgfusepath{stroke,fill}%
\end{pgfscope}%
\begin{pgfscope}%
\pgfsetrectcap%
\pgfsetroundjoin%
\pgfsetlinewidth{1.505625pt}%
\definecolor{currentstroke}{rgb}{0.121569,0.466667,0.705882}%
\pgfsetstrokecolor{currentstroke}%
\pgfsetdash{}{0pt}%
\pgfpathmoveto{\pgfqpoint{3.252108in}{2.329441in}}%
\pgfpathlineto{\pgfqpoint{3.363219in}{2.329441in}}%
\pgfpathlineto{\pgfqpoint{3.474330in}{2.329441in}}%
\pgfusepath{stroke}%
\end{pgfscope}%
\begin{pgfscope}%
\definecolor{textcolor}{rgb}{0.000000,0.000000,0.000000}%
\pgfsetstrokecolor{textcolor}%
\pgfsetfillcolor{textcolor}%
\pgftext[x=3.563219in,y=2.290552in,left,base]{\color{textcolor}\rmfamily\fontsize{8.000000}{9.600000}\selectfont NPLC 1}%
\end{pgfscope}%
\begin{pgfscope}%
\pgfsetrectcap%
\pgfsetroundjoin%
\pgfsetlinewidth{1.505625pt}%
\definecolor{currentstroke}{rgb}{1.000000,0.498039,0.054902}%
\pgfsetstrokecolor{currentstroke}%
\pgfsetdash{}{0pt}%
\pgfpathmoveto{\pgfqpoint{3.252108in}{2.174552in}}%
\pgfpathlineto{\pgfqpoint{3.363219in}{2.174552in}}%
\pgfpathlineto{\pgfqpoint{3.474330in}{2.174552in}}%
\pgfusepath{stroke}%
\end{pgfscope}%
\begin{pgfscope}%
\definecolor{textcolor}{rgb}{0.000000,0.000000,0.000000}%
\pgfsetstrokecolor{textcolor}%
\pgfsetfillcolor{textcolor}%
\pgftext[x=3.563219in,y=2.135663in,left,base]{\color{textcolor}\rmfamily\fontsize{8.000000}{9.600000}\selectfont NPLC 2}%
\end{pgfscope}%
\begin{pgfscope}%
\pgfsetrectcap%
\pgfsetroundjoin%
\pgfsetlinewidth{1.505625pt}%
\definecolor{currentstroke}{rgb}{0.172549,0.627451,0.172549}%
\pgfsetstrokecolor{currentstroke}%
\pgfsetdash{}{0pt}%
\pgfpathmoveto{\pgfqpoint{3.252108in}{2.019664in}}%
\pgfpathlineto{\pgfqpoint{3.363219in}{2.019664in}}%
\pgfpathlineto{\pgfqpoint{3.474330in}{2.019664in}}%
\pgfusepath{stroke}%
\end{pgfscope}%
\begin{pgfscope}%
\definecolor{textcolor}{rgb}{0.000000,0.000000,0.000000}%
\pgfsetstrokecolor{textcolor}%
\pgfsetfillcolor{textcolor}%
\pgftext[x=3.563219in,y=1.980775in,left,base]{\color{textcolor}\rmfamily\fontsize{8.000000}{9.600000}\selectfont NPLC 5}%
\end{pgfscope}%
\begin{pgfscope}%
\pgfsetrectcap%
\pgfsetroundjoin%
\pgfsetlinewidth{1.505625pt}%
\definecolor{currentstroke}{rgb}{0.839216,0.152941,0.156863}%
\pgfsetstrokecolor{currentstroke}%
\pgfsetdash{}{0pt}%
\pgfpathmoveto{\pgfqpoint{3.252108in}{1.864775in}}%
\pgfpathlineto{\pgfqpoint{3.363219in}{1.864775in}}%
\pgfpathlineto{\pgfqpoint{3.474330in}{1.864775in}}%
\pgfusepath{stroke}%
\end{pgfscope}%
\begin{pgfscope}%
\definecolor{textcolor}{rgb}{0.000000,0.000000,0.000000}%
\pgfsetstrokecolor{textcolor}%
\pgfsetfillcolor{textcolor}%
\pgftext[x=3.563219in,y=1.825886in,left,base]{\color{textcolor}\rmfamily\fontsize{8.000000}{9.600000}\selectfont NPLC 10}%
\end{pgfscope}%
\begin{pgfscope}%
\pgfsetrectcap%
\pgfsetroundjoin%
\pgfsetlinewidth{1.505625pt}%
\definecolor{currentstroke}{rgb}{0.580392,0.403922,0.741176}%
\pgfsetstrokecolor{currentstroke}%
\pgfsetdash{}{0pt}%
\pgfpathmoveto{\pgfqpoint{3.252108in}{1.709886in}}%
\pgfpathlineto{\pgfqpoint{3.363219in}{1.709886in}}%
\pgfpathlineto{\pgfqpoint{3.474330in}{1.709886in}}%
\pgfusepath{stroke}%
\end{pgfscope}%
\begin{pgfscope}%
\definecolor{textcolor}{rgb}{0.000000,0.000000,0.000000}%
\pgfsetstrokecolor{textcolor}%
\pgfsetfillcolor{textcolor}%
\pgftext[x=3.563219in,y=1.670997in,left,base]{\color{textcolor}\rmfamily\fontsize{8.000000}{9.600000}\selectfont NPLC 20}%
\end{pgfscope}%
\begin{pgfscope}%
\pgfsetrectcap%
\pgfsetroundjoin%
\pgfsetlinewidth{1.505625pt}%
\definecolor{currentstroke}{rgb}{0.549020,0.337255,0.294118}%
\pgfsetstrokecolor{currentstroke}%
\pgfsetdash{}{0pt}%
\pgfpathmoveto{\pgfqpoint{3.252108in}{1.554997in}}%
\pgfpathlineto{\pgfqpoint{3.363219in}{1.554997in}}%
\pgfpathlineto{\pgfqpoint{3.474330in}{1.554997in}}%
\pgfusepath{stroke}%
\end{pgfscope}%
\begin{pgfscope}%
\definecolor{textcolor}{rgb}{0.000000,0.000000,0.000000}%
\pgfsetstrokecolor{textcolor}%
\pgfsetfillcolor{textcolor}%
\pgftext[x=3.563219in,y=1.516108in,left,base]{\color{textcolor}\rmfamily\fontsize{8.000000}{9.600000}\selectfont NPLC 50}%
\end{pgfscope}%
\end{pgfpicture}%
\makeatother%
\endgroup%

    \caption{Allan deviation for different integration times before applying the AZ algorithm. Deadtime $\theta = \qty{1}{\s}$. The dashed line denotes the Allan variance without AZ.}
    \label{fig:autozero_deadtime_nplcs_adev}
\end{figure}

It should be stressed here, that the dead time is not be the only factor to consider when chossing the autozero interval. For example, in case of an amplifier, switching the input also adds an error current due to the charge injection of the switching transistors. This may negatively impact the measurement of a high impedance source. These additional drawbacks are implementation specific and must considered during the design phase.

\clearpage
\subsection{Gain Correction}
\label{sec:autozero_gain}
The effect of the gain correction, where the input value $x$ is scaled by a scaling factor $y$ to adjust the changing, can be calculated, assuming white noise, as follows:
\begin{align}
    \sigma_{x \cdot y}^2 &= \langle x^2 y^2 \rangle - \langle x y \rangle^2 \nonumber\\
    &= \langle x^2 \rangle \langle y^2 \rangle + \underbrace{2\,\mathrm{Cov}\left(x^2,y^2\right)}_{\text{uncorrelated} \, = \, 0} - \left( \langle x \rangle \langle y \rangle + \underbrace{2\,\mathrm{Cov}\left(x,y\right)}_{=\, 0} \right)^2 \nonumber\\
    &= \left(\sigma_x^2 + \langle x \rangle^2\right) \cdot \left(\sigma_y^2 + \langle y \rangle^2\right) - \langle x \rangle \langle y \rangle \nonumber\\
    &= \sigma_x^2 \sigma_y^2 + \sigma_x^2 \langle y \rangle^2 + \sigma_y^2 \langle x \rangle^2 \label{eqn:variance_multiplied}\\
\end{align}

With respect to the gain correction, equation \ref{eqn:variance_multiplied} can be further reduced. The scaling factor is derived from the reference voltage $V_{ref}$ and normalized using $\frac{V_{ref, meas}}{V_{ref}}$. The expected value, therefore is $\langle y \rangle \approx 1$, as the ADC should not drift much from its calibrated value. Furthermore, $\sigma_y^2$ is scaled by the constant $1/V_{ref}$ and $\sigma_x^2 \sigma_y^2 \ll \sigma_x^2$. The latter should be true for any measurement of significance.

\begin{equation}
    \sigma_{x \cdot y}^2 \approx \sigma_x^2 + \sigma_y^2 \langle x \rangle^2
\end{equation}

The gain correction noise therefore behaves similar to the offset correction case, except, that it scales with the input voltage and has no effect with a shorted input, while fully introducing its noise for a full scale input.


% check \cite{psd_to_adev} Appendix II for details on dead time
% Compare PSD in Generation-Recombination Noise, Allan Variance, and Low-Frequency Gain Instabilities in Microwave Amplifiers to our controller. The hump look similar. Due to popcorn noise

\clearpage
\section{Current Sources}
% TODO: The FET Constant-Current Source/Limiter
% TODO: Cable choice https://www.gore.com/resources/search?f[]=product:24241&f[]=content_type:6&f[]=language:en or http://www.jenving.com/products/view/hd5-hdmi-dvi-dp-blue-b75-1001000064
Throughout this work the concept of current sources is widely used, for example section \ref{sec:laser_current_driver} discusses a current source to drive laser diodes and the temperature controller discussed in section \ref{sec:temperature_controller} uses a current source to measure the resistance of a temperature sensitive resistor. While there are many more use cases, this section will limit the discussion to a few examples used by the devices presented in this work. Namely, this is a unidirectional transconductance amplifier with an operational-amplifier and a field-effect transistor and a bidirectional Howland current pump invented by Bradford Howland in 1962 and first published in 1964 by \citeauthor{howland_current_source} \cite{howland_current_source}. The discussion will start with the properties of the ideal current source and, based on that, develop a more accurate model. The models developed typically represent the static, time-independent case unless explicitely stated. First, the unidirection current source is treated, then the bidirectional Howland current pump is discussed.

\subsection{Current Sink and Current Source}
%\begin{chapquote}{Adapted from William Shakespeare, \textit{Hamlet}}
%``To sink, or not to sink, that is the question.''
%\end{chapquote}

The question whether to use a current source or a current sink is elemental for the design of a laser driver. Figure \ref{fig:current_sink_source} shows different configuration of current sinks and sources with respect to the laser diode.

\begin{figure}[ht]
    \centering
    \begin{subfigure}{0.225\linewidth}
        \centering
        \import{figures/}{current_source_high.tex}
        \caption{Source with\protect\\grounded LD.}
        \label{fig:current_source_high}
    \end{subfigure}
    \begin{subfigure}{0.225\linewidth}
        \centering
        \import{figures/}{current_source_low.tex}
        \caption{Source with\protect\\floating LD.}
        \label{fig:current_source_low}
    \end{subfigure}
    \begin{subfigure}{0.225\linewidth}
        \centering
        \import{figures/}{current_sink_high.tex}
        \caption{Sink with\protect\\grounded LD.}
        \label{fig:current_sink_high}
    \end{subfigure}
    \begin{subfigure}{0.225\linewidth}
        \centering
        \import{figures/}{current_sink_low.tex}
        \caption{Sink with\protect\\floating LD.}
        \label{fig:current_sink_low}
    \end{subfigure}
    \caption{Different configuration of current sinks and sources with respect to the laser diode. A green checkmark denotes a fail-safe configuration when accidentally shorting one or more pins of the diode to the laser chassis, illustrated by a dashed connection.}
    \label{fig:current_sink_source}
\end{figure}

The most practical configuration depends on the laser diode and safety aspects in terms of protecting the laser diode. The protection of the laser diode is discussed first. The laser resonator is assumed grounded in our setup. While not intended, there are numerous ways to accidently short the diode to ground and since there are no immediate consequences arrising from it, when the controller is disconnected, it might easily be overlooked. This blunder should not bear the risk of destroying an expensive laser diode. To ensure this, a configuration where the laser diode is shorted out instead of the current source or sink must be chosen. That way, the laser diode is automatically removed from the circuit in case of an error condition.
Choosing between a current sink and a current source is more subtle. If the can of the laser diode is connected to anode, a current sink can be considered, to keep the can at ground potential. This is not an issue with our laser design though, because the laser diode mount is floating. Another aspect is the electronics side. A current source is typically implemented using p-channel field-effect transistors, while current sinks are using n-channel transistors and additionally the input of a current source is referenced to the positive supply, while the sink is referenced to the negative supply. Using the negative supply as a reference for control signals brings more challenges than vice versa, because typically integrated components like digital-to-analog converters prefer working with positive voltages and would need additional support to be floated to a negative reference. This makes a current source simpler to implement in this scenario and this work focusses on the current source, but in principle all methods derived can be applied to a current sink as well.

\subsection{Ideal Current Source}
The ideal current source as shown in figure \ref{fig:ideal_current_source} has two major properties besides the output current $I_{out}$, the output impedance and the compliance voltage, which are best understood when looking at the two equivalent representations of a current source separately. On the left in figure \ref{fig:ideal_current_source_norton}, the Norton representation can be seen. Norton's theorem reduces any linear circuit to a current source, shown in green, with a parallel resistance $R_{out}$, usually called output resistance or impedance. On the right, the Thévenin representation can be see, which simplifies a circuit as a voltage source, also shown in green, with a series resistance.

\begin{figure}[ht]
    \centering
    \begin{subfigure}{0.4\linewidth}
        \centering
        \import{figures/}{current_source_norton.tex}
        \caption{Norton representation.}
        \label{fig:ideal_current_source_norton}
    \end{subfigure}
    \begin{subfigure}{0.4\linewidth}
        \centering
        \import{figures/}{current_source_thevenin.tex}
        \caption{Thévenin representation.}
        \label{fig:ideal_current_source_thevenin}
    \end{subfigure}
    \caption{An ideal current source with output impedance $R_{out}$ and noise $e_n$.}
    \label{fig:ideal_current_source}
\end{figure}

First, the output impedance is discussed. Ideally, $R_{out}$ is infinite and all current is forced to flow through the load. Given a finite output impedance leads to a decreased accuracy of $I_{out}$, because it is influenced by the load impedance as
\begin{equation}
    I_{out} = I_{set} \cdot \frac{R_{out}}{R_{load} + R_{out}} \, .
\end{equation}

In addition to a decreased accuracy, inserting a noise voltage source between the current source and the load as shown in figure \ref{fig:ideal_current_source} in orange, has the same effect as a changing load resistance and due to the finite output impedance $R_{out}$, any voltage noise $e_n$ translates to current noise $i_n$ through the load as
\begin{equation}
    i_n = \frac{e_n}{R_{load} + R_{out}} \approx \frac{e_n}{R_{out}} \, ,
\end{equation}

again making a high output impedance desirable to supress noise sources between the current source and the load.

Going to figure \ref{fig:ideal_current_source_thevenin} of a current source in Thévenin representation allows to discuss the compliance voltage property. As it was said above, the output impedance of an ideal current source is infinite and so is the maximum output voltage of said current source. A finite output impedance immediately implies a finite supply voltage to keep the current to a finite limit, which dictates a maximum output voltage. This is called the compliance voltage.

\subsection{The Field-Effect Transistor Current Source}
\label{sec:mosfet_current_source}
% Good slides can be found here: https://www.ittc.ku.edu/~jstiles/312/handouts/
Given the limited supply voltage of a real current source drives the need for a resistive element that has a finite resitance and infinite, or very high, frequency dependent dynamic impedance to react to load changes. One such pass element, having these properties, is a field-effect-transisistor (FET). A junction-gate field-effect transistor (JFET) or metal–oxide–semiconductor field-effect transistor (MOSFET) can be used either as a current source or sink, depending on its doping. A p-channel FET, which uses a positve doping of the channel, is a current source, while an n-channel FET works as a current sink. This discussin is focussing on the p-channel FET with MOSFETs at its center, because it covers the bulk of the laser current driver design in section \ref{sec:laser_current_driver}.

\begin{figure}[hb]
    \centering
    \begin{subfigure}{0.4\linewidth}
        \centering
        \import{figures/}{p-channel_jfet.tex}
        \caption{P-Channel JFET.}
        \label{fig:pjfet}
    \end{subfigure}
    \begin{subfigure}{0.4\linewidth}
        \centering
        \import{figures/}{p-channel_mosfet.tex}
        \caption{P-Channel MOSFET.}
        \label{fig:pmos}
    \end{subfigure}
    \caption{The simplified semiconductor structure of a JFET and a MOSFET.}
    \label{fig:FETs}
\end{figure}

The difference between a JFET and a MOSFET, is the gate structure as illustrated in figure \ref{fig:FETs}. While a MOSFET has an insulated gate, the JFET does not. This reduces the gate leakage current, typically by about three orders of magnitude, and allows to forward bias the device since there is no diode, resulting in larger current handling capacity. So for low currents up to a few \unit{\mA} or low noise applications, JFETS are preferred, while MOSFETs can handle several hundred ampere. The same mathematical approach can be applied to both types of FETs though. The other difference between a JFET and a MOSFET is the fact that JFETs are only available as depletion-mode (normally-on) devices, while MOSFETs are available as both depletion and enhancement (normally-off) devices. The reason is the gate structure as mentioned above. An enhancement-mode device does not conduct, when the gate-to-source voltage $V_{GS} = \qty{0}{\V}$, so $|V_{GS}|$ must be increased or enhanced for the device to allow conduction. This is not possible with an uninsulated gate like a simple p-n junction of a JFET, which would then start conducting start leaking. A p-channel depletion-mode device on the other hand conducts at $V_{GS} = \qty{0}{\V}$ and $|V_{GS}|$ must be decreased or depleted to reduce the current, which is not possible with the uninsulated gate, because the p-n junction is reverse biased. The annotated circuit symbol and the quantities used to discuss the device properties are shown in figure \ref{fig:fet_symbols}.

\begin{figure}[ht]
    \centering
    \begin{subfigure}{0.4\linewidth}
        \centering
        \import{figures/}{jfet_pins.tex}
        \caption{P-channel JFET.}
        \label{fig:fet_symbols_jfet}
    \end{subfigure}
    \begin{subfigure}{0.4\linewidth}
        \centering
        \import{figures/}{pmos_pins.tex}
        \caption{P-channel MOSFET.}
        \label{fig:fet_symbols_mosfet}
    \end{subfigure}
    \caption{Basic p-channel FET circuit.}
    \label{fig:fet_symbols}
\end{figure}

A p-channel FET has its source (S) connected to the positive supply and the drain (D) is connected to a more negative voltage, typically the load. For the MOSFET the gate (G) is biased below the source to allow conduction. The source is is usually connected to the substrate for solitary devices as shown in figure \ref{fig:pmos}. This will be assumed in all further discussions and the consequences of a substrate, that is biased differently are omitted here. The interested reader may look up these details in \cite{mosfet_details}.

As it was mentioned above, if appropriately biased, a FET can be considered a voltage controlled current source. This property can be seen in figure \ref{fig:fet_curret_gate_bias}.

\begin{figure}[hb]
    \centering
    %% Creator: Matplotlib, PGF backend
%%
%% To include the figure in your LaTeX document, write
%%   \input{<filename>.pgf}
%%
%% Make sure the required packages are loaded in your preamble
%%   \usepackage{pgf}
%%
%% Also ensure that all the required font packages are loaded; for instance,
%% the lmodern package is sometimes necessary when using math font.
%%   \usepackage{lmodern}
%%
%% Figures using additional raster images can only be included by \input if
%% they are in the same directory as the main LaTeX file. For loading figures
%% from other directories you can use the `import` package
%%   \usepackage{import}
%%
%% and then include the figures with
%%   \import{<path to file>}{<filename>.pgf}
%%
%% Matplotlib used the following preamble
%%   \usepackage{siunitx}
%%   \usepackage{fontspec}
%%
\begingroup%
\makeatletter%
\begin{pgfpicture}%
\pgfpathrectangle{\pgfpointorigin}{\pgfqpoint{5.492126in}{3.394321in}}%
\pgfusepath{use as bounding box, clip}%
\begin{pgfscope}%
\pgfsetbuttcap%
\pgfsetmiterjoin%
\definecolor{currentfill}{rgb}{1.000000,1.000000,1.000000}%
\pgfsetfillcolor{currentfill}%
\pgfsetlinewidth{0.000000pt}%
\definecolor{currentstroke}{rgb}{1.000000,1.000000,1.000000}%
\pgfsetstrokecolor{currentstroke}%
\pgfsetdash{}{0pt}%
\pgfpathmoveto{\pgfqpoint{0.000000in}{0.000000in}}%
\pgfpathlineto{\pgfqpoint{5.492126in}{0.000000in}}%
\pgfpathlineto{\pgfqpoint{5.492126in}{3.394321in}}%
\pgfpathlineto{\pgfqpoint{0.000000in}{3.394321in}}%
\pgfpathlineto{\pgfqpoint{0.000000in}{0.000000in}}%
\pgfpathclose%
\pgfusepath{fill}%
\end{pgfscope}%
\begin{pgfscope}%
\pgfsetbuttcap%
\pgfsetmiterjoin%
\definecolor{currentfill}{rgb}{1.000000,1.000000,1.000000}%
\pgfsetfillcolor{currentfill}%
\pgfsetlinewidth{0.000000pt}%
\definecolor{currentstroke}{rgb}{0.000000,0.000000,0.000000}%
\pgfsetstrokecolor{currentstroke}%
\pgfsetstrokeopacity{0.000000}%
\pgfsetdash{}{0pt}%
\pgfpathmoveto{\pgfqpoint{0.693677in}{0.524170in}}%
\pgfpathlineto{\pgfqpoint{5.342126in}{0.524170in}}%
\pgfpathlineto{\pgfqpoint{5.342126in}{3.120077in}}%
\pgfpathlineto{\pgfqpoint{0.693677in}{3.120077in}}%
\pgfpathlineto{\pgfqpoint{0.693677in}{0.524170in}}%
\pgfpathclose%
\pgfusepath{fill}%
\end{pgfscope}%
\begin{pgfscope}%
\pgfpathrectangle{\pgfqpoint{0.693677in}{0.524170in}}{\pgfqpoint{4.648449in}{2.595908in}}%
\pgfusepath{clip}%
\pgfsetrectcap%
\pgfsetroundjoin%
\pgfsetlinewidth{0.803000pt}%
\definecolor{currentstroke}{rgb}{0.450000,0.450000,0.450000}%
\pgfsetstrokecolor{currentstroke}%
\pgfsetdash{}{0pt}%
\pgfpathmoveto{\pgfqpoint{5.130833in}{0.524170in}}%
\pgfpathlineto{\pgfqpoint{5.130833in}{3.120077in}}%
\pgfusepath{stroke}%
\end{pgfscope}%
\begin{pgfscope}%
\pgfsetbuttcap%
\pgfsetroundjoin%
\definecolor{currentfill}{rgb}{0.000000,0.000000,0.000000}%
\pgfsetfillcolor{currentfill}%
\pgfsetlinewidth{0.803000pt}%
\definecolor{currentstroke}{rgb}{0.000000,0.000000,0.000000}%
\pgfsetstrokecolor{currentstroke}%
\pgfsetdash{}{0pt}%
\pgfsys@defobject{currentmarker}{\pgfqpoint{0.000000in}{-0.048611in}}{\pgfqpoint{0.000000in}{0.000000in}}{%
\pgfpathmoveto{\pgfqpoint{0.000000in}{0.000000in}}%
\pgfpathlineto{\pgfqpoint{0.000000in}{-0.048611in}}%
\pgfusepath{stroke,fill}%
}%
\begin{pgfscope}%
\pgfsys@transformshift{5.130833in}{0.524170in}%
\pgfsys@useobject{currentmarker}{}%
\end{pgfscope}%
\end{pgfscope}%
\begin{pgfscope}%
\definecolor{textcolor}{rgb}{0.000000,0.000000,0.000000}%
\pgfsetstrokecolor{textcolor}%
\pgfsetfillcolor{textcolor}%
\pgftext[x=5.130833in,y=0.426948in,,top]{\color{textcolor}\rmfamily\fontsize{8.000000}{9.600000}\selectfont \(\displaystyle {\ensuremath{-}10}\)}%
\end{pgfscope}%
\begin{pgfscope}%
\pgfpathrectangle{\pgfqpoint{0.693677in}{0.524170in}}{\pgfqpoint{4.648449in}{2.595908in}}%
\pgfusepath{clip}%
\pgfsetrectcap%
\pgfsetroundjoin%
\pgfsetlinewidth{0.803000pt}%
\definecolor{currentstroke}{rgb}{0.450000,0.450000,0.450000}%
\pgfsetstrokecolor{currentstroke}%
\pgfsetdash{}{0pt}%
\pgfpathmoveto{\pgfqpoint{4.285660in}{0.524170in}}%
\pgfpathlineto{\pgfqpoint{4.285660in}{3.120077in}}%
\pgfusepath{stroke}%
\end{pgfscope}%
\begin{pgfscope}%
\pgfsetbuttcap%
\pgfsetroundjoin%
\definecolor{currentfill}{rgb}{0.000000,0.000000,0.000000}%
\pgfsetfillcolor{currentfill}%
\pgfsetlinewidth{0.803000pt}%
\definecolor{currentstroke}{rgb}{0.000000,0.000000,0.000000}%
\pgfsetstrokecolor{currentstroke}%
\pgfsetdash{}{0pt}%
\pgfsys@defobject{currentmarker}{\pgfqpoint{0.000000in}{-0.048611in}}{\pgfqpoint{0.000000in}{0.000000in}}{%
\pgfpathmoveto{\pgfqpoint{0.000000in}{0.000000in}}%
\pgfpathlineto{\pgfqpoint{0.000000in}{-0.048611in}}%
\pgfusepath{stroke,fill}%
}%
\begin{pgfscope}%
\pgfsys@transformshift{4.285660in}{0.524170in}%
\pgfsys@useobject{currentmarker}{}%
\end{pgfscope}%
\end{pgfscope}%
\begin{pgfscope}%
\definecolor{textcolor}{rgb}{0.000000,0.000000,0.000000}%
\pgfsetstrokecolor{textcolor}%
\pgfsetfillcolor{textcolor}%
\pgftext[x=4.285660in,y=0.426948in,,top]{\color{textcolor}\rmfamily\fontsize{8.000000}{9.600000}\selectfont \(\displaystyle {\ensuremath{-}8}\)}%
\end{pgfscope}%
\begin{pgfscope}%
\pgfpathrectangle{\pgfqpoint{0.693677in}{0.524170in}}{\pgfqpoint{4.648449in}{2.595908in}}%
\pgfusepath{clip}%
\pgfsetrectcap%
\pgfsetroundjoin%
\pgfsetlinewidth{0.803000pt}%
\definecolor{currentstroke}{rgb}{0.450000,0.450000,0.450000}%
\pgfsetstrokecolor{currentstroke}%
\pgfsetdash{}{0pt}%
\pgfpathmoveto{\pgfqpoint{3.440488in}{0.524170in}}%
\pgfpathlineto{\pgfqpoint{3.440488in}{3.120077in}}%
\pgfusepath{stroke}%
\end{pgfscope}%
\begin{pgfscope}%
\pgfsetbuttcap%
\pgfsetroundjoin%
\definecolor{currentfill}{rgb}{0.000000,0.000000,0.000000}%
\pgfsetfillcolor{currentfill}%
\pgfsetlinewidth{0.803000pt}%
\definecolor{currentstroke}{rgb}{0.000000,0.000000,0.000000}%
\pgfsetstrokecolor{currentstroke}%
\pgfsetdash{}{0pt}%
\pgfsys@defobject{currentmarker}{\pgfqpoint{0.000000in}{-0.048611in}}{\pgfqpoint{0.000000in}{0.000000in}}{%
\pgfpathmoveto{\pgfqpoint{0.000000in}{0.000000in}}%
\pgfpathlineto{\pgfqpoint{0.000000in}{-0.048611in}}%
\pgfusepath{stroke,fill}%
}%
\begin{pgfscope}%
\pgfsys@transformshift{3.440488in}{0.524170in}%
\pgfsys@useobject{currentmarker}{}%
\end{pgfscope}%
\end{pgfscope}%
\begin{pgfscope}%
\definecolor{textcolor}{rgb}{0.000000,0.000000,0.000000}%
\pgfsetstrokecolor{textcolor}%
\pgfsetfillcolor{textcolor}%
\pgftext[x=3.440488in,y=0.426948in,,top]{\color{textcolor}\rmfamily\fontsize{8.000000}{9.600000}\selectfont \(\displaystyle {\ensuremath{-}6}\)}%
\end{pgfscope}%
\begin{pgfscope}%
\pgfpathrectangle{\pgfqpoint{0.693677in}{0.524170in}}{\pgfqpoint{4.648449in}{2.595908in}}%
\pgfusepath{clip}%
\pgfsetrectcap%
\pgfsetroundjoin%
\pgfsetlinewidth{0.803000pt}%
\definecolor{currentstroke}{rgb}{0.450000,0.450000,0.450000}%
\pgfsetstrokecolor{currentstroke}%
\pgfsetdash{}{0pt}%
\pgfpathmoveto{\pgfqpoint{2.595315in}{0.524170in}}%
\pgfpathlineto{\pgfqpoint{2.595315in}{3.120077in}}%
\pgfusepath{stroke}%
\end{pgfscope}%
\begin{pgfscope}%
\pgfsetbuttcap%
\pgfsetroundjoin%
\definecolor{currentfill}{rgb}{0.000000,0.000000,0.000000}%
\pgfsetfillcolor{currentfill}%
\pgfsetlinewidth{0.803000pt}%
\definecolor{currentstroke}{rgb}{0.000000,0.000000,0.000000}%
\pgfsetstrokecolor{currentstroke}%
\pgfsetdash{}{0pt}%
\pgfsys@defobject{currentmarker}{\pgfqpoint{0.000000in}{-0.048611in}}{\pgfqpoint{0.000000in}{0.000000in}}{%
\pgfpathmoveto{\pgfqpoint{0.000000in}{0.000000in}}%
\pgfpathlineto{\pgfqpoint{0.000000in}{-0.048611in}}%
\pgfusepath{stroke,fill}%
}%
\begin{pgfscope}%
\pgfsys@transformshift{2.595315in}{0.524170in}%
\pgfsys@useobject{currentmarker}{}%
\end{pgfscope}%
\end{pgfscope}%
\begin{pgfscope}%
\definecolor{textcolor}{rgb}{0.000000,0.000000,0.000000}%
\pgfsetstrokecolor{textcolor}%
\pgfsetfillcolor{textcolor}%
\pgftext[x=2.595315in,y=0.426948in,,top]{\color{textcolor}\rmfamily\fontsize{8.000000}{9.600000}\selectfont \(\displaystyle {\ensuremath{-}4}\)}%
\end{pgfscope}%
\begin{pgfscope}%
\pgfpathrectangle{\pgfqpoint{0.693677in}{0.524170in}}{\pgfqpoint{4.648449in}{2.595908in}}%
\pgfusepath{clip}%
\pgfsetrectcap%
\pgfsetroundjoin%
\pgfsetlinewidth{0.803000pt}%
\definecolor{currentstroke}{rgb}{0.450000,0.450000,0.450000}%
\pgfsetstrokecolor{currentstroke}%
\pgfsetdash{}{0pt}%
\pgfpathmoveto{\pgfqpoint{1.750143in}{0.524170in}}%
\pgfpathlineto{\pgfqpoint{1.750143in}{3.120077in}}%
\pgfusepath{stroke}%
\end{pgfscope}%
\begin{pgfscope}%
\pgfsetbuttcap%
\pgfsetroundjoin%
\definecolor{currentfill}{rgb}{0.000000,0.000000,0.000000}%
\pgfsetfillcolor{currentfill}%
\pgfsetlinewidth{0.803000pt}%
\definecolor{currentstroke}{rgb}{0.000000,0.000000,0.000000}%
\pgfsetstrokecolor{currentstroke}%
\pgfsetdash{}{0pt}%
\pgfsys@defobject{currentmarker}{\pgfqpoint{0.000000in}{-0.048611in}}{\pgfqpoint{0.000000in}{0.000000in}}{%
\pgfpathmoveto{\pgfqpoint{0.000000in}{0.000000in}}%
\pgfpathlineto{\pgfqpoint{0.000000in}{-0.048611in}}%
\pgfusepath{stroke,fill}%
}%
\begin{pgfscope}%
\pgfsys@transformshift{1.750143in}{0.524170in}%
\pgfsys@useobject{currentmarker}{}%
\end{pgfscope}%
\end{pgfscope}%
\begin{pgfscope}%
\definecolor{textcolor}{rgb}{0.000000,0.000000,0.000000}%
\pgfsetstrokecolor{textcolor}%
\pgfsetfillcolor{textcolor}%
\pgftext[x=1.750143in,y=0.426948in,,top]{\color{textcolor}\rmfamily\fontsize{8.000000}{9.600000}\selectfont \(\displaystyle {\ensuremath{-}2}\)}%
\end{pgfscope}%
\begin{pgfscope}%
\pgfpathrectangle{\pgfqpoint{0.693677in}{0.524170in}}{\pgfqpoint{4.648449in}{2.595908in}}%
\pgfusepath{clip}%
\pgfsetrectcap%
\pgfsetroundjoin%
\pgfsetlinewidth{0.803000pt}%
\definecolor{currentstroke}{rgb}{0.450000,0.450000,0.450000}%
\pgfsetstrokecolor{currentstroke}%
\pgfsetdash{}{0pt}%
\pgfpathmoveto{\pgfqpoint{0.904970in}{0.524170in}}%
\pgfpathlineto{\pgfqpoint{0.904970in}{3.120077in}}%
\pgfusepath{stroke}%
\end{pgfscope}%
\begin{pgfscope}%
\pgfsetbuttcap%
\pgfsetroundjoin%
\definecolor{currentfill}{rgb}{0.000000,0.000000,0.000000}%
\pgfsetfillcolor{currentfill}%
\pgfsetlinewidth{0.803000pt}%
\definecolor{currentstroke}{rgb}{0.000000,0.000000,0.000000}%
\pgfsetstrokecolor{currentstroke}%
\pgfsetdash{}{0pt}%
\pgfsys@defobject{currentmarker}{\pgfqpoint{0.000000in}{-0.048611in}}{\pgfqpoint{0.000000in}{0.000000in}}{%
\pgfpathmoveto{\pgfqpoint{0.000000in}{0.000000in}}%
\pgfpathlineto{\pgfqpoint{0.000000in}{-0.048611in}}%
\pgfusepath{stroke,fill}%
}%
\begin{pgfscope}%
\pgfsys@transformshift{0.904970in}{0.524170in}%
\pgfsys@useobject{currentmarker}{}%
\end{pgfscope}%
\end{pgfscope}%
\begin{pgfscope}%
\definecolor{textcolor}{rgb}{0.000000,0.000000,0.000000}%
\pgfsetstrokecolor{textcolor}%
\pgfsetfillcolor{textcolor}%
\pgftext[x=0.904970in,y=0.426948in,,top]{\color{textcolor}\rmfamily\fontsize{8.000000}{9.600000}\selectfont \(\displaystyle {0}\)}%
\end{pgfscope}%
\begin{pgfscope}%
\definecolor{textcolor}{rgb}{0.000000,0.000000,0.000000}%
\pgfsetstrokecolor{textcolor}%
\pgfsetfillcolor{textcolor}%
\pgftext[x=3.017901in,y=0.272725in,,top]{\color{textcolor}\rmfamily\fontsize{10.000000}{12.000000}\selectfont Drain-Source Voltage \(\displaystyle V_{DS}\) in \unit{\V}}%
\end{pgfscope}%
\begin{pgfscope}%
\pgfpathrectangle{\pgfqpoint{0.693677in}{0.524170in}}{\pgfqpoint{4.648449in}{2.595908in}}%
\pgfusepath{clip}%
\pgfsetrectcap%
\pgfsetroundjoin%
\pgfsetlinewidth{0.803000pt}%
\definecolor{currentstroke}{rgb}{0.450000,0.450000,0.450000}%
\pgfsetstrokecolor{currentstroke}%
\pgfsetdash{}{0pt}%
\pgfpathmoveto{\pgfqpoint{0.693677in}{2.979278in}}%
\pgfpathlineto{\pgfqpoint{5.342126in}{2.979278in}}%
\pgfusepath{stroke}%
\end{pgfscope}%
\begin{pgfscope}%
\pgfsetbuttcap%
\pgfsetroundjoin%
\definecolor{currentfill}{rgb}{0.000000,0.000000,0.000000}%
\pgfsetfillcolor{currentfill}%
\pgfsetlinewidth{0.803000pt}%
\definecolor{currentstroke}{rgb}{0.000000,0.000000,0.000000}%
\pgfsetstrokecolor{currentstroke}%
\pgfsetdash{}{0pt}%
\pgfsys@defobject{currentmarker}{\pgfqpoint{-0.048611in}{0.000000in}}{\pgfqpoint{-0.000000in}{0.000000in}}{%
\pgfpathmoveto{\pgfqpoint{-0.000000in}{0.000000in}}%
\pgfpathlineto{\pgfqpoint{-0.048611in}{0.000000in}}%
\pgfusepath{stroke,fill}%
}%
\begin{pgfscope}%
\pgfsys@transformshift{0.693677in}{2.979278in}%
\pgfsys@useobject{currentmarker}{}%
\end{pgfscope}%
\end{pgfscope}%
\begin{pgfscope}%
\definecolor{textcolor}{rgb}{0.000000,0.000000,0.000000}%
\pgfsetstrokecolor{textcolor}%
\pgfsetfillcolor{textcolor}%
\pgftext[x=0.327546in, y=2.940722in, left, base]{\color{textcolor}\rmfamily\fontsize{8.000000}{9.600000}\selectfont \(\displaystyle {\ensuremath{-}500}\)}%
\end{pgfscope}%
\begin{pgfscope}%
\pgfpathrectangle{\pgfqpoint{0.693677in}{0.524170in}}{\pgfqpoint{4.648449in}{2.595908in}}%
\pgfusepath{clip}%
\pgfsetrectcap%
\pgfsetroundjoin%
\pgfsetlinewidth{0.803000pt}%
\definecolor{currentstroke}{rgb}{0.450000,0.450000,0.450000}%
\pgfsetstrokecolor{currentstroke}%
\pgfsetdash{}{0pt}%
\pgfpathmoveto{\pgfqpoint{0.693677in}{2.511855in}}%
\pgfpathlineto{\pgfqpoint{5.342126in}{2.511855in}}%
\pgfusepath{stroke}%
\end{pgfscope}%
\begin{pgfscope}%
\pgfsetbuttcap%
\pgfsetroundjoin%
\definecolor{currentfill}{rgb}{0.000000,0.000000,0.000000}%
\pgfsetfillcolor{currentfill}%
\pgfsetlinewidth{0.803000pt}%
\definecolor{currentstroke}{rgb}{0.000000,0.000000,0.000000}%
\pgfsetstrokecolor{currentstroke}%
\pgfsetdash{}{0pt}%
\pgfsys@defobject{currentmarker}{\pgfqpoint{-0.048611in}{0.000000in}}{\pgfqpoint{-0.000000in}{0.000000in}}{%
\pgfpathmoveto{\pgfqpoint{-0.000000in}{0.000000in}}%
\pgfpathlineto{\pgfqpoint{-0.048611in}{0.000000in}}%
\pgfusepath{stroke,fill}%
}%
\begin{pgfscope}%
\pgfsys@transformshift{0.693677in}{2.511855in}%
\pgfsys@useobject{currentmarker}{}%
\end{pgfscope}%
\end{pgfscope}%
\begin{pgfscope}%
\definecolor{textcolor}{rgb}{0.000000,0.000000,0.000000}%
\pgfsetstrokecolor{textcolor}%
\pgfsetfillcolor{textcolor}%
\pgftext[x=0.327546in, y=2.473300in, left, base]{\color{textcolor}\rmfamily\fontsize{8.000000}{9.600000}\selectfont \(\displaystyle {\ensuremath{-}400}\)}%
\end{pgfscope}%
\begin{pgfscope}%
\pgfpathrectangle{\pgfqpoint{0.693677in}{0.524170in}}{\pgfqpoint{4.648449in}{2.595908in}}%
\pgfusepath{clip}%
\pgfsetrectcap%
\pgfsetroundjoin%
\pgfsetlinewidth{0.803000pt}%
\definecolor{currentstroke}{rgb}{0.450000,0.450000,0.450000}%
\pgfsetstrokecolor{currentstroke}%
\pgfsetdash{}{0pt}%
\pgfpathmoveto{\pgfqpoint{0.693677in}{2.044433in}}%
\pgfpathlineto{\pgfqpoint{5.342126in}{2.044433in}}%
\pgfusepath{stroke}%
\end{pgfscope}%
\begin{pgfscope}%
\pgfsetbuttcap%
\pgfsetroundjoin%
\definecolor{currentfill}{rgb}{0.000000,0.000000,0.000000}%
\pgfsetfillcolor{currentfill}%
\pgfsetlinewidth{0.803000pt}%
\definecolor{currentstroke}{rgb}{0.000000,0.000000,0.000000}%
\pgfsetstrokecolor{currentstroke}%
\pgfsetdash{}{0pt}%
\pgfsys@defobject{currentmarker}{\pgfqpoint{-0.048611in}{0.000000in}}{\pgfqpoint{-0.000000in}{0.000000in}}{%
\pgfpathmoveto{\pgfqpoint{-0.000000in}{0.000000in}}%
\pgfpathlineto{\pgfqpoint{-0.048611in}{0.000000in}}%
\pgfusepath{stroke,fill}%
}%
\begin{pgfscope}%
\pgfsys@transformshift{0.693677in}{2.044433in}%
\pgfsys@useobject{currentmarker}{}%
\end{pgfscope}%
\end{pgfscope}%
\begin{pgfscope}%
\definecolor{textcolor}{rgb}{0.000000,0.000000,0.000000}%
\pgfsetstrokecolor{textcolor}%
\pgfsetfillcolor{textcolor}%
\pgftext[x=0.327546in, y=2.005877in, left, base]{\color{textcolor}\rmfamily\fontsize{8.000000}{9.600000}\selectfont \(\displaystyle {\ensuremath{-}300}\)}%
\end{pgfscope}%
\begin{pgfscope}%
\pgfpathrectangle{\pgfqpoint{0.693677in}{0.524170in}}{\pgfqpoint{4.648449in}{2.595908in}}%
\pgfusepath{clip}%
\pgfsetrectcap%
\pgfsetroundjoin%
\pgfsetlinewidth{0.803000pt}%
\definecolor{currentstroke}{rgb}{0.450000,0.450000,0.450000}%
\pgfsetstrokecolor{currentstroke}%
\pgfsetdash{}{0pt}%
\pgfpathmoveto{\pgfqpoint{0.693677in}{1.577010in}}%
\pgfpathlineto{\pgfqpoint{5.342126in}{1.577010in}}%
\pgfusepath{stroke}%
\end{pgfscope}%
\begin{pgfscope}%
\pgfsetbuttcap%
\pgfsetroundjoin%
\definecolor{currentfill}{rgb}{0.000000,0.000000,0.000000}%
\pgfsetfillcolor{currentfill}%
\pgfsetlinewidth{0.803000pt}%
\definecolor{currentstroke}{rgb}{0.000000,0.000000,0.000000}%
\pgfsetstrokecolor{currentstroke}%
\pgfsetdash{}{0pt}%
\pgfsys@defobject{currentmarker}{\pgfqpoint{-0.048611in}{0.000000in}}{\pgfqpoint{-0.000000in}{0.000000in}}{%
\pgfpathmoveto{\pgfqpoint{-0.000000in}{0.000000in}}%
\pgfpathlineto{\pgfqpoint{-0.048611in}{0.000000in}}%
\pgfusepath{stroke,fill}%
}%
\begin{pgfscope}%
\pgfsys@transformshift{0.693677in}{1.577010in}%
\pgfsys@useobject{currentmarker}{}%
\end{pgfscope}%
\end{pgfscope}%
\begin{pgfscope}%
\definecolor{textcolor}{rgb}{0.000000,0.000000,0.000000}%
\pgfsetstrokecolor{textcolor}%
\pgfsetfillcolor{textcolor}%
\pgftext[x=0.327546in, y=1.538455in, left, base]{\color{textcolor}\rmfamily\fontsize{8.000000}{9.600000}\selectfont \(\displaystyle {\ensuremath{-}200}\)}%
\end{pgfscope}%
\begin{pgfscope}%
\pgfpathrectangle{\pgfqpoint{0.693677in}{0.524170in}}{\pgfqpoint{4.648449in}{2.595908in}}%
\pgfusepath{clip}%
\pgfsetrectcap%
\pgfsetroundjoin%
\pgfsetlinewidth{0.803000pt}%
\definecolor{currentstroke}{rgb}{0.450000,0.450000,0.450000}%
\pgfsetstrokecolor{currentstroke}%
\pgfsetdash{}{0pt}%
\pgfpathmoveto{\pgfqpoint{0.693677in}{1.109588in}}%
\pgfpathlineto{\pgfqpoint{5.342126in}{1.109588in}}%
\pgfusepath{stroke}%
\end{pgfscope}%
\begin{pgfscope}%
\pgfsetbuttcap%
\pgfsetroundjoin%
\definecolor{currentfill}{rgb}{0.000000,0.000000,0.000000}%
\pgfsetfillcolor{currentfill}%
\pgfsetlinewidth{0.803000pt}%
\definecolor{currentstroke}{rgb}{0.000000,0.000000,0.000000}%
\pgfsetstrokecolor{currentstroke}%
\pgfsetdash{}{0pt}%
\pgfsys@defobject{currentmarker}{\pgfqpoint{-0.048611in}{0.000000in}}{\pgfqpoint{-0.000000in}{0.000000in}}{%
\pgfpathmoveto{\pgfqpoint{-0.000000in}{0.000000in}}%
\pgfpathlineto{\pgfqpoint{-0.048611in}{0.000000in}}%
\pgfusepath{stroke,fill}%
}%
\begin{pgfscope}%
\pgfsys@transformshift{0.693677in}{1.109588in}%
\pgfsys@useobject{currentmarker}{}%
\end{pgfscope}%
\end{pgfscope}%
\begin{pgfscope}%
\definecolor{textcolor}{rgb}{0.000000,0.000000,0.000000}%
\pgfsetstrokecolor{textcolor}%
\pgfsetfillcolor{textcolor}%
\pgftext[x=0.327546in, y=1.071032in, left, base]{\color{textcolor}\rmfamily\fontsize{8.000000}{9.600000}\selectfont \(\displaystyle {\ensuremath{-}100}\)}%
\end{pgfscope}%
\begin{pgfscope}%
\pgfpathrectangle{\pgfqpoint{0.693677in}{0.524170in}}{\pgfqpoint{4.648449in}{2.595908in}}%
\pgfusepath{clip}%
\pgfsetrectcap%
\pgfsetroundjoin%
\pgfsetlinewidth{0.803000pt}%
\definecolor{currentstroke}{rgb}{0.450000,0.450000,0.450000}%
\pgfsetstrokecolor{currentstroke}%
\pgfsetdash{}{0pt}%
\pgfpathmoveto{\pgfqpoint{0.693677in}{0.642166in}}%
\pgfpathlineto{\pgfqpoint{5.342126in}{0.642166in}}%
\pgfusepath{stroke}%
\end{pgfscope}%
\begin{pgfscope}%
\pgfsetbuttcap%
\pgfsetroundjoin%
\definecolor{currentfill}{rgb}{0.000000,0.000000,0.000000}%
\pgfsetfillcolor{currentfill}%
\pgfsetlinewidth{0.803000pt}%
\definecolor{currentstroke}{rgb}{0.000000,0.000000,0.000000}%
\pgfsetstrokecolor{currentstroke}%
\pgfsetdash{}{0pt}%
\pgfsys@defobject{currentmarker}{\pgfqpoint{-0.048611in}{0.000000in}}{\pgfqpoint{-0.000000in}{0.000000in}}{%
\pgfpathmoveto{\pgfqpoint{-0.000000in}{0.000000in}}%
\pgfpathlineto{\pgfqpoint{-0.048611in}{0.000000in}}%
\pgfusepath{stroke,fill}%
}%
\begin{pgfscope}%
\pgfsys@transformshift{0.693677in}{0.642166in}%
\pgfsys@useobject{currentmarker}{}%
\end{pgfscope}%
\end{pgfscope}%
\begin{pgfscope}%
\definecolor{textcolor}{rgb}{0.000000,0.000000,0.000000}%
\pgfsetstrokecolor{textcolor}%
\pgfsetfillcolor{textcolor}%
\pgftext[x=0.537426in, y=0.603610in, left, base]{\color{textcolor}\rmfamily\fontsize{8.000000}{9.600000}\selectfont \(\displaystyle {0}\)}%
\end{pgfscope}%
\begin{pgfscope}%
\definecolor{textcolor}{rgb}{0.000000,0.000000,0.000000}%
\pgfsetstrokecolor{textcolor}%
\pgfsetfillcolor{textcolor}%
\pgftext[x=0.271991in,y=1.822124in,,bottom,rotate=90.000000]{\color{textcolor}\rmfamily\fontsize{10.000000}{12.000000}\selectfont Drain Current \(\displaystyle I_D\) in \unit{\A}}%
\end{pgfscope}%
\begin{pgfscope}%
\definecolor{textcolor}{rgb}{0.000000,0.000000,0.000000}%
\pgfsetstrokecolor{textcolor}%
\pgfsetfillcolor{textcolor}%
\pgftext[x=0.693677in,y=3.161744in,left,base]{\color{textcolor}\rmfamily\fontsize{8.000000}{9.600000}\selectfont \(\displaystyle \times{10^{\ensuremath{-}3}}{}\)}%
\end{pgfscope}%
\begin{pgfscope}%
\pgfpathrectangle{\pgfqpoint{0.693677in}{0.524170in}}{\pgfqpoint{4.648449in}{2.595908in}}%
\pgfusepath{clip}%
\pgfsetrectcap%
\pgfsetroundjoin%
\pgfsetlinewidth{1.003750pt}%
\definecolor{currentstroke}{rgb}{0.003922,0.450980,0.698039}%
\pgfsetstrokecolor{currentstroke}%
\pgfsetstrokeopacity{0.700000}%
\pgfsetdash{}{0pt}%
\pgfpathmoveto{\pgfqpoint{0.904970in}{0.642166in}}%
\pgfpathlineto{\pgfqpoint{0.947229in}{0.650885in}}%
\pgfpathlineto{\pgfqpoint{0.989487in}{0.651962in}}%
\pgfpathlineto{\pgfqpoint{1.031746in}{0.651966in}}%
\pgfpathlineto{\pgfqpoint{1.074005in}{0.651970in}}%
\pgfpathlineto{\pgfqpoint{1.116263in}{0.651974in}}%
\pgfpathlineto{\pgfqpoint{1.158522in}{0.651978in}}%
\pgfpathlineto{\pgfqpoint{1.200780in}{0.651982in}}%
\pgfpathlineto{\pgfqpoint{1.243039in}{0.651986in}}%
\pgfpathlineto{\pgfqpoint{1.285298in}{0.651989in}}%
\pgfpathlineto{\pgfqpoint{1.327556in}{0.651993in}}%
\pgfpathlineto{\pgfqpoint{1.369815in}{0.651997in}}%
\pgfpathlineto{\pgfqpoint{1.412074in}{0.652001in}}%
\pgfpathlineto{\pgfqpoint{1.454332in}{0.652005in}}%
\pgfpathlineto{\pgfqpoint{1.496591in}{0.652009in}}%
\pgfpathlineto{\pgfqpoint{1.538849in}{0.652013in}}%
\pgfpathlineto{\pgfqpoint{1.581108in}{0.652017in}}%
\pgfpathlineto{\pgfqpoint{1.623367in}{0.652020in}}%
\pgfpathlineto{\pgfqpoint{1.665625in}{0.652024in}}%
\pgfpathlineto{\pgfqpoint{1.707884in}{0.652028in}}%
\pgfpathlineto{\pgfqpoint{1.750143in}{0.652032in}}%
\pgfpathlineto{\pgfqpoint{1.792401in}{0.652036in}}%
\pgfpathlineto{\pgfqpoint{1.834660in}{0.652040in}}%
\pgfpathlineto{\pgfqpoint{1.876918in}{0.652044in}}%
\pgfpathlineto{\pgfqpoint{1.919177in}{0.652048in}}%
\pgfpathlineto{\pgfqpoint{1.961436in}{0.652052in}}%
\pgfpathlineto{\pgfqpoint{2.003694in}{0.652055in}}%
\pgfpathlineto{\pgfqpoint{2.045953in}{0.652059in}}%
\pgfpathlineto{\pgfqpoint{2.088212in}{0.652063in}}%
\pgfpathlineto{\pgfqpoint{2.130470in}{0.652067in}}%
\pgfpathlineto{\pgfqpoint{2.172729in}{0.652071in}}%
\pgfpathlineto{\pgfqpoint{2.214988in}{0.652075in}}%
\pgfpathlineto{\pgfqpoint{2.257246in}{0.652079in}}%
\pgfpathlineto{\pgfqpoint{2.299505in}{0.652083in}}%
\pgfpathlineto{\pgfqpoint{2.341763in}{0.652086in}}%
\pgfpathlineto{\pgfqpoint{2.384022in}{0.652090in}}%
\pgfpathlineto{\pgfqpoint{2.426281in}{0.652094in}}%
\pgfpathlineto{\pgfqpoint{2.468539in}{0.652098in}}%
\pgfpathlineto{\pgfqpoint{2.510798in}{0.652102in}}%
\pgfpathlineto{\pgfqpoint{2.553057in}{0.652106in}}%
\pgfpathlineto{\pgfqpoint{2.595315in}{0.652110in}}%
\pgfpathlineto{\pgfqpoint{2.637574in}{0.652114in}}%
\pgfpathlineto{\pgfqpoint{2.679832in}{0.652118in}}%
\pgfpathlineto{\pgfqpoint{2.722091in}{0.652121in}}%
\pgfpathlineto{\pgfqpoint{2.764350in}{0.652125in}}%
\pgfpathlineto{\pgfqpoint{2.806608in}{0.652129in}}%
\pgfpathlineto{\pgfqpoint{2.848867in}{0.652133in}}%
\pgfpathlineto{\pgfqpoint{2.891126in}{0.652137in}}%
\pgfpathlineto{\pgfqpoint{2.933384in}{0.652141in}}%
\pgfpathlineto{\pgfqpoint{2.975643in}{0.652145in}}%
\pgfpathlineto{\pgfqpoint{3.017901in}{0.652149in}}%
\pgfpathlineto{\pgfqpoint{3.060160in}{0.652152in}}%
\pgfpathlineto{\pgfqpoint{3.102419in}{0.652156in}}%
\pgfpathlineto{\pgfqpoint{3.144677in}{0.652160in}}%
\pgfpathlineto{\pgfqpoint{3.186936in}{0.652164in}}%
\pgfpathlineto{\pgfqpoint{3.229195in}{0.652168in}}%
\pgfpathlineto{\pgfqpoint{3.271453in}{0.652172in}}%
\pgfpathlineto{\pgfqpoint{3.313712in}{0.652176in}}%
\pgfpathlineto{\pgfqpoint{3.355971in}{0.652180in}}%
\pgfpathlineto{\pgfqpoint{3.398229in}{0.652184in}}%
\pgfpathlineto{\pgfqpoint{3.440488in}{0.652187in}}%
\pgfpathlineto{\pgfqpoint{3.482746in}{0.652191in}}%
\pgfpathlineto{\pgfqpoint{3.525005in}{0.652195in}}%
\pgfpathlineto{\pgfqpoint{3.567264in}{0.652199in}}%
\pgfpathlineto{\pgfqpoint{3.609522in}{0.652203in}}%
\pgfpathlineto{\pgfqpoint{3.651781in}{0.652207in}}%
\pgfpathlineto{\pgfqpoint{3.694040in}{0.652211in}}%
\pgfpathlineto{\pgfqpoint{3.736298in}{0.652215in}}%
\pgfpathlineto{\pgfqpoint{3.778557in}{0.652218in}}%
\pgfpathlineto{\pgfqpoint{3.820815in}{0.652222in}}%
\pgfpathlineto{\pgfqpoint{3.863074in}{0.652226in}}%
\pgfpathlineto{\pgfqpoint{3.905333in}{0.652230in}}%
\pgfpathlineto{\pgfqpoint{3.947591in}{0.652234in}}%
\pgfpathlineto{\pgfqpoint{3.989850in}{0.652238in}}%
\pgfpathlineto{\pgfqpoint{4.032109in}{0.652242in}}%
\pgfpathlineto{\pgfqpoint{4.074367in}{0.652246in}}%
\pgfpathlineto{\pgfqpoint{4.116626in}{0.652250in}}%
\pgfpathlineto{\pgfqpoint{4.158885in}{0.652253in}}%
\pgfpathlineto{\pgfqpoint{4.201143in}{0.652257in}}%
\pgfpathlineto{\pgfqpoint{4.243402in}{0.652261in}}%
\pgfpathlineto{\pgfqpoint{4.285660in}{0.652265in}}%
\pgfpathlineto{\pgfqpoint{4.327919in}{0.652269in}}%
\pgfpathlineto{\pgfqpoint{4.370178in}{0.652273in}}%
\pgfpathlineto{\pgfqpoint{4.412436in}{0.652277in}}%
\pgfpathlineto{\pgfqpoint{4.454695in}{0.652281in}}%
\pgfpathlineto{\pgfqpoint{4.496954in}{0.652284in}}%
\pgfpathlineto{\pgfqpoint{4.539212in}{0.652288in}}%
\pgfpathlineto{\pgfqpoint{4.581471in}{0.652292in}}%
\pgfpathlineto{\pgfqpoint{4.623729in}{0.652296in}}%
\pgfpathlineto{\pgfqpoint{4.665988in}{0.652300in}}%
\pgfpathlineto{\pgfqpoint{4.708247in}{0.652304in}}%
\pgfpathlineto{\pgfqpoint{4.750505in}{0.652308in}}%
\pgfpathlineto{\pgfqpoint{4.792764in}{0.652312in}}%
\pgfpathlineto{\pgfqpoint{4.835023in}{0.652315in}}%
\pgfpathlineto{\pgfqpoint{4.877281in}{0.652319in}}%
\pgfpathlineto{\pgfqpoint{4.919540in}{0.652323in}}%
\pgfpathlineto{\pgfqpoint{4.961798in}{0.652327in}}%
\pgfpathlineto{\pgfqpoint{5.004057in}{0.652331in}}%
\pgfpathlineto{\pgfqpoint{5.046316in}{0.652335in}}%
\pgfpathlineto{\pgfqpoint{5.088574in}{0.652339in}}%
\pgfpathlineto{\pgfqpoint{5.130833in}{0.652343in}}%
\pgfusepath{stroke}%
\end{pgfscope}%
\begin{pgfscope}%
\pgfpathrectangle{\pgfqpoint{0.693677in}{0.524170in}}{\pgfqpoint{4.648449in}{2.595908in}}%
\pgfusepath{clip}%
\pgfsetrectcap%
\pgfsetroundjoin%
\pgfsetlinewidth{1.003750pt}%
\definecolor{currentstroke}{rgb}{0.870588,0.560784,0.019608}%
\pgfsetstrokecolor{currentstroke}%
\pgfsetstrokeopacity{0.700000}%
\pgfsetdash{}{0pt}%
\pgfpathmoveto{\pgfqpoint{0.904970in}{0.642166in}}%
\pgfpathlineto{\pgfqpoint{0.947229in}{0.692166in}}%
\pgfpathlineto{\pgfqpoint{0.989487in}{0.736888in}}%
\pgfpathlineto{\pgfqpoint{1.031746in}{0.776003in}}%
\pgfpathlineto{\pgfqpoint{1.074005in}{0.809146in}}%
\pgfpathlineto{\pgfqpoint{1.116263in}{0.835912in}}%
\pgfpathlineto{\pgfqpoint{1.158522in}{0.855850in}}%
\pgfpathlineto{\pgfqpoint{1.200780in}{0.868453in}}%
\pgfpathlineto{\pgfqpoint{1.243039in}{0.873155in}}%
\pgfpathlineto{\pgfqpoint{1.285298in}{0.873251in}}%
\pgfpathlineto{\pgfqpoint{1.327556in}{0.873334in}}%
\pgfpathlineto{\pgfqpoint{1.369815in}{0.873416in}}%
\pgfpathlineto{\pgfqpoint{1.412074in}{0.873499in}}%
\pgfpathlineto{\pgfqpoint{1.454332in}{0.873582in}}%
\pgfpathlineto{\pgfqpoint{1.496591in}{0.873665in}}%
\pgfpathlineto{\pgfqpoint{1.538849in}{0.873748in}}%
\pgfpathlineto{\pgfqpoint{1.581108in}{0.873831in}}%
\pgfpathlineto{\pgfqpoint{1.623367in}{0.873914in}}%
\pgfpathlineto{\pgfqpoint{1.665625in}{0.873997in}}%
\pgfpathlineto{\pgfqpoint{1.707884in}{0.874080in}}%
\pgfpathlineto{\pgfqpoint{1.750143in}{0.874163in}}%
\pgfpathlineto{\pgfqpoint{1.792401in}{0.874245in}}%
\pgfpathlineto{\pgfqpoint{1.834660in}{0.874328in}}%
\pgfpathlineto{\pgfqpoint{1.876918in}{0.874411in}}%
\pgfpathlineto{\pgfqpoint{1.919177in}{0.874494in}}%
\pgfpathlineto{\pgfqpoint{1.961436in}{0.874577in}}%
\pgfpathlineto{\pgfqpoint{2.003694in}{0.874660in}}%
\pgfpathlineto{\pgfqpoint{2.045953in}{0.874742in}}%
\pgfpathlineto{\pgfqpoint{2.088212in}{0.874825in}}%
\pgfpathlineto{\pgfqpoint{2.130470in}{0.874908in}}%
\pgfpathlineto{\pgfqpoint{2.172729in}{0.874991in}}%
\pgfpathlineto{\pgfqpoint{2.214988in}{0.875074in}}%
\pgfpathlineto{\pgfqpoint{2.257246in}{0.875156in}}%
\pgfpathlineto{\pgfqpoint{2.299505in}{0.875239in}}%
\pgfpathlineto{\pgfqpoint{2.341763in}{0.875322in}}%
\pgfpathlineto{\pgfqpoint{2.384022in}{0.875405in}}%
\pgfpathlineto{\pgfqpoint{2.426281in}{0.875487in}}%
\pgfpathlineto{\pgfqpoint{2.468539in}{0.875570in}}%
\pgfpathlineto{\pgfqpoint{2.510798in}{0.875653in}}%
\pgfpathlineto{\pgfqpoint{2.553057in}{0.875736in}}%
\pgfpathlineto{\pgfqpoint{2.595315in}{0.875818in}}%
\pgfpathlineto{\pgfqpoint{2.637574in}{0.875901in}}%
\pgfpathlineto{\pgfqpoint{2.679832in}{0.875984in}}%
\pgfpathlineto{\pgfqpoint{2.722091in}{0.876067in}}%
\pgfpathlineto{\pgfqpoint{2.764350in}{0.876149in}}%
\pgfpathlineto{\pgfqpoint{2.806608in}{0.876232in}}%
\pgfpathlineto{\pgfqpoint{2.848867in}{0.876315in}}%
\pgfpathlineto{\pgfqpoint{2.891126in}{0.876397in}}%
\pgfpathlineto{\pgfqpoint{2.933384in}{0.876480in}}%
\pgfpathlineto{\pgfqpoint{2.975643in}{0.876563in}}%
\pgfpathlineto{\pgfqpoint{3.017901in}{0.876645in}}%
\pgfpathlineto{\pgfqpoint{3.060160in}{0.876728in}}%
\pgfpathlineto{\pgfqpoint{3.102419in}{0.876811in}}%
\pgfpathlineto{\pgfqpoint{3.144677in}{0.876893in}}%
\pgfpathlineto{\pgfqpoint{3.186936in}{0.876976in}}%
\pgfpathlineto{\pgfqpoint{3.229195in}{0.877059in}}%
\pgfpathlineto{\pgfqpoint{3.271453in}{0.877141in}}%
\pgfpathlineto{\pgfqpoint{3.313712in}{0.877224in}}%
\pgfpathlineto{\pgfqpoint{3.355971in}{0.877306in}}%
\pgfpathlineto{\pgfqpoint{3.398229in}{0.877389in}}%
\pgfpathlineto{\pgfqpoint{3.440488in}{0.877472in}}%
\pgfpathlineto{\pgfqpoint{3.482746in}{0.877554in}}%
\pgfpathlineto{\pgfqpoint{3.525005in}{0.877637in}}%
\pgfpathlineto{\pgfqpoint{3.567264in}{0.877719in}}%
\pgfpathlineto{\pgfqpoint{3.609522in}{0.877802in}}%
\pgfpathlineto{\pgfqpoint{3.651781in}{0.877885in}}%
\pgfpathlineto{\pgfqpoint{3.694040in}{0.877967in}}%
\pgfpathlineto{\pgfqpoint{3.736298in}{0.878050in}}%
\pgfpathlineto{\pgfqpoint{3.778557in}{0.878132in}}%
\pgfpathlineto{\pgfqpoint{3.820815in}{0.878215in}}%
\pgfpathlineto{\pgfqpoint{3.863074in}{0.878297in}}%
\pgfpathlineto{\pgfqpoint{3.905333in}{0.878380in}}%
\pgfpathlineto{\pgfqpoint{3.947591in}{0.878462in}}%
\pgfpathlineto{\pgfqpoint{3.989850in}{0.878545in}}%
\pgfpathlineto{\pgfqpoint{4.032109in}{0.878627in}}%
\pgfpathlineto{\pgfqpoint{4.074367in}{0.878710in}}%
\pgfpathlineto{\pgfqpoint{4.116626in}{0.878793in}}%
\pgfpathlineto{\pgfqpoint{4.158885in}{0.878875in}}%
\pgfpathlineto{\pgfqpoint{4.201143in}{0.878958in}}%
\pgfpathlineto{\pgfqpoint{4.243402in}{0.879040in}}%
\pgfpathlineto{\pgfqpoint{4.285660in}{0.879122in}}%
\pgfpathlineto{\pgfqpoint{4.327919in}{0.879205in}}%
\pgfpathlineto{\pgfqpoint{4.370178in}{0.879287in}}%
\pgfpathlineto{\pgfqpoint{4.412436in}{0.879370in}}%
\pgfpathlineto{\pgfqpoint{4.454695in}{0.879452in}}%
\pgfpathlineto{\pgfqpoint{4.496954in}{0.879535in}}%
\pgfpathlineto{\pgfqpoint{4.539212in}{0.879617in}}%
\pgfpathlineto{\pgfqpoint{4.581471in}{0.879700in}}%
\pgfpathlineto{\pgfqpoint{4.623729in}{0.879782in}}%
\pgfpathlineto{\pgfqpoint{4.665988in}{0.879864in}}%
\pgfpathlineto{\pgfqpoint{4.708247in}{0.879947in}}%
\pgfpathlineto{\pgfqpoint{4.750505in}{0.880029in}}%
\pgfpathlineto{\pgfqpoint{4.792764in}{0.880112in}}%
\pgfpathlineto{\pgfqpoint{4.835023in}{0.880194in}}%
\pgfpathlineto{\pgfqpoint{4.877281in}{0.880277in}}%
\pgfpathlineto{\pgfqpoint{4.919540in}{0.880359in}}%
\pgfpathlineto{\pgfqpoint{4.961798in}{0.880441in}}%
\pgfpathlineto{\pgfqpoint{5.004057in}{0.880524in}}%
\pgfpathlineto{\pgfqpoint{5.046316in}{0.880606in}}%
\pgfpathlineto{\pgfqpoint{5.088574in}{0.880688in}}%
\pgfpathlineto{\pgfqpoint{5.130833in}{0.880771in}}%
\pgfusepath{stroke}%
\end{pgfscope}%
\begin{pgfscope}%
\pgfpathrectangle{\pgfqpoint{0.693677in}{0.524170in}}{\pgfqpoint{4.648449in}{2.595908in}}%
\pgfusepath{clip}%
\pgfsetrectcap%
\pgfsetroundjoin%
\pgfsetlinewidth{1.003750pt}%
\definecolor{currentstroke}{rgb}{0.007843,0.619608,0.450980}%
\pgfsetstrokecolor{currentstroke}%
\pgfsetstrokeopacity{0.700000}%
\pgfsetdash{}{0pt}%
\pgfpathmoveto{\pgfqpoint{0.904970in}{0.642166in}}%
\pgfpathlineto{\pgfqpoint{0.947229in}{0.731506in}}%
\pgfpathlineto{\pgfqpoint{0.989487in}{0.817858in}}%
\pgfpathlineto{\pgfqpoint{1.031746in}{0.901100in}}%
\pgfpathlineto{\pgfqpoint{1.074005in}{0.981101in}}%
\pgfpathlineto{\pgfqpoint{1.116263in}{1.057724in}}%
\pgfpathlineto{\pgfqpoint{1.158522in}{1.130821in}}%
\pgfpathlineto{\pgfqpoint{1.200780in}{1.200236in}}%
\pgfpathlineto{\pgfqpoint{1.243039in}{1.265801in}}%
\pgfpathlineto{\pgfqpoint{1.285298in}{1.327338in}}%
\pgfpathlineto{\pgfqpoint{1.327556in}{1.384654in}}%
\pgfpathlineto{\pgfqpoint{1.369815in}{1.437543in}}%
\pgfpathlineto{\pgfqpoint{1.412074in}{1.485785in}}%
\pgfpathlineto{\pgfqpoint{1.454332in}{1.529140in}}%
\pgfpathlineto{\pgfqpoint{1.496591in}{1.567352in}}%
\pgfpathlineto{\pgfqpoint{1.538849in}{1.600142in}}%
\pgfpathlineto{\pgfqpoint{1.581108in}{1.627209in}}%
\pgfpathlineto{\pgfqpoint{1.623367in}{1.648225in}}%
\pgfpathlineto{\pgfqpoint{1.665625in}{1.662835in}}%
\pgfpathlineto{\pgfqpoint{1.707884in}{1.670650in}}%
\pgfpathlineto{\pgfqpoint{1.750143in}{1.672045in}}%
\pgfpathlineto{\pgfqpoint{1.792401in}{1.672365in}}%
\pgfpathlineto{\pgfqpoint{1.834660in}{1.672685in}}%
\pgfpathlineto{\pgfqpoint{1.876918in}{1.673005in}}%
\pgfpathlineto{\pgfqpoint{1.919177in}{1.673326in}}%
\pgfpathlineto{\pgfqpoint{1.961436in}{1.673646in}}%
\pgfpathlineto{\pgfqpoint{2.003694in}{1.673965in}}%
\pgfpathlineto{\pgfqpoint{2.045953in}{1.674286in}}%
\pgfpathlineto{\pgfqpoint{2.088212in}{1.674605in}}%
\pgfpathlineto{\pgfqpoint{2.130470in}{1.674925in}}%
\pgfpathlineto{\pgfqpoint{2.172729in}{1.675245in}}%
\pgfpathlineto{\pgfqpoint{2.214988in}{1.675565in}}%
\pgfpathlineto{\pgfqpoint{2.257246in}{1.675885in}}%
\pgfpathlineto{\pgfqpoint{2.299505in}{1.676204in}}%
\pgfpathlineto{\pgfqpoint{2.341763in}{1.676524in}}%
\pgfpathlineto{\pgfqpoint{2.384022in}{1.676843in}}%
\pgfpathlineto{\pgfqpoint{2.426281in}{1.677163in}}%
\pgfpathlineto{\pgfqpoint{2.468539in}{1.677482in}}%
\pgfpathlineto{\pgfqpoint{2.510798in}{1.677802in}}%
\pgfpathlineto{\pgfqpoint{2.553057in}{1.678121in}}%
\pgfpathlineto{\pgfqpoint{2.595315in}{1.678440in}}%
\pgfpathlineto{\pgfqpoint{2.637574in}{1.678759in}}%
\pgfpathlineto{\pgfqpoint{2.679832in}{1.679079in}}%
\pgfpathlineto{\pgfqpoint{2.722091in}{1.679397in}}%
\pgfpathlineto{\pgfqpoint{2.764350in}{1.679717in}}%
\pgfpathlineto{\pgfqpoint{2.806608in}{1.680035in}}%
\pgfpathlineto{\pgfqpoint{2.848867in}{1.680355in}}%
\pgfpathlineto{\pgfqpoint{2.891126in}{1.680673in}}%
\pgfpathlineto{\pgfqpoint{2.933384in}{1.680992in}}%
\pgfpathlineto{\pgfqpoint{2.975643in}{1.681311in}}%
\pgfpathlineto{\pgfqpoint{3.017901in}{1.681630in}}%
\pgfpathlineto{\pgfqpoint{3.060160in}{1.681949in}}%
\pgfpathlineto{\pgfqpoint{3.102419in}{1.682267in}}%
\pgfpathlineto{\pgfqpoint{3.144677in}{1.682586in}}%
\pgfpathlineto{\pgfqpoint{3.186936in}{1.682904in}}%
\pgfpathlineto{\pgfqpoint{3.229195in}{1.683223in}}%
\pgfpathlineto{\pgfqpoint{3.271453in}{1.683542in}}%
\pgfpathlineto{\pgfqpoint{3.313712in}{1.683860in}}%
\pgfpathlineto{\pgfqpoint{3.355971in}{1.684178in}}%
\pgfpathlineto{\pgfqpoint{3.398229in}{1.684496in}}%
\pgfpathlineto{\pgfqpoint{3.440488in}{1.684815in}}%
\pgfpathlineto{\pgfqpoint{3.482746in}{1.685133in}}%
\pgfpathlineto{\pgfqpoint{3.525005in}{1.685451in}}%
\pgfpathlineto{\pgfqpoint{3.567264in}{1.685769in}}%
\pgfpathlineto{\pgfqpoint{3.609522in}{1.686087in}}%
\pgfpathlineto{\pgfqpoint{3.651781in}{1.686405in}}%
\pgfpathlineto{\pgfqpoint{3.694040in}{1.686723in}}%
\pgfpathlineto{\pgfqpoint{3.736298in}{1.687041in}}%
\pgfpathlineto{\pgfqpoint{3.778557in}{1.687359in}}%
\pgfpathlineto{\pgfqpoint{3.820815in}{1.687676in}}%
\pgfpathlineto{\pgfqpoint{3.863074in}{1.687994in}}%
\pgfpathlineto{\pgfqpoint{3.905333in}{1.688312in}}%
\pgfpathlineto{\pgfqpoint{3.947591in}{1.688629in}}%
\pgfpathlineto{\pgfqpoint{3.989850in}{1.688947in}}%
\pgfpathlineto{\pgfqpoint{4.032109in}{1.689265in}}%
\pgfpathlineto{\pgfqpoint{4.074367in}{1.689582in}}%
\pgfpathlineto{\pgfqpoint{4.116626in}{1.689899in}}%
\pgfpathlineto{\pgfqpoint{4.158885in}{1.690217in}}%
\pgfpathlineto{\pgfqpoint{4.201143in}{1.690534in}}%
\pgfpathlineto{\pgfqpoint{4.243402in}{1.690852in}}%
\pgfpathlineto{\pgfqpoint{4.285660in}{1.691168in}}%
\pgfpathlineto{\pgfqpoint{4.327919in}{1.691486in}}%
\pgfpathlineto{\pgfqpoint{4.370178in}{1.691803in}}%
\pgfpathlineto{\pgfqpoint{4.412436in}{1.692120in}}%
\pgfpathlineto{\pgfqpoint{4.454695in}{1.692437in}}%
\pgfpathlineto{\pgfqpoint{4.496954in}{1.692754in}}%
\pgfpathlineto{\pgfqpoint{4.539212in}{1.693071in}}%
\pgfpathlineto{\pgfqpoint{4.581471in}{1.693388in}}%
\pgfpathlineto{\pgfqpoint{4.623729in}{1.693705in}}%
\pgfpathlineto{\pgfqpoint{4.665988in}{1.694022in}}%
\pgfpathlineto{\pgfqpoint{4.708247in}{1.694338in}}%
\pgfpathlineto{\pgfqpoint{4.750505in}{1.694654in}}%
\pgfpathlineto{\pgfqpoint{4.792764in}{1.694971in}}%
\pgfpathlineto{\pgfqpoint{4.835023in}{1.695288in}}%
\pgfpathlineto{\pgfqpoint{4.877281in}{1.695604in}}%
\pgfpathlineto{\pgfqpoint{4.919540in}{1.695921in}}%
\pgfpathlineto{\pgfqpoint{4.961798in}{1.696237in}}%
\pgfpathlineto{\pgfqpoint{5.004057in}{1.696554in}}%
\pgfpathlineto{\pgfqpoint{5.046316in}{1.696870in}}%
\pgfpathlineto{\pgfqpoint{5.088574in}{1.697186in}}%
\pgfpathlineto{\pgfqpoint{5.130833in}{1.697503in}}%
\pgfusepath{stroke}%
\end{pgfscope}%
\begin{pgfscope}%
\pgfpathrectangle{\pgfqpoint{0.693677in}{0.524170in}}{\pgfqpoint{4.648449in}{2.595908in}}%
\pgfusepath{clip}%
\pgfsetrectcap%
\pgfsetroundjoin%
\pgfsetlinewidth{1.003750pt}%
\definecolor{currentstroke}{rgb}{0.835294,0.368627,0.000000}%
\pgfsetstrokecolor{currentstroke}%
\pgfsetstrokeopacity{0.700000}%
\pgfsetdash{}{0pt}%
\pgfpathmoveto{\pgfqpoint{0.904970in}{0.642166in}}%
\pgfpathlineto{\pgfqpoint{0.947229in}{0.756646in}}%
\pgfpathlineto{\pgfqpoint{0.989487in}{0.869273in}}%
\pgfpathlineto{\pgfqpoint{1.031746in}{0.979989in}}%
\pgfpathlineto{\pgfqpoint{1.074005in}{1.088734in}}%
\pgfpathlineto{\pgfqpoint{1.116263in}{1.195444in}}%
\pgfpathlineto{\pgfqpoint{1.158522in}{1.300054in}}%
\pgfpathlineto{\pgfqpoint{1.200780in}{1.402493in}}%
\pgfpathlineto{\pgfqpoint{1.243039in}{1.502689in}}%
\pgfpathlineto{\pgfqpoint{1.285298in}{1.600567in}}%
\pgfpathlineto{\pgfqpoint{1.327556in}{1.696046in}}%
\pgfpathlineto{\pgfqpoint{1.369815in}{1.789041in}}%
\pgfpathlineto{\pgfqpoint{1.412074in}{1.879466in}}%
\pgfpathlineto{\pgfqpoint{1.454332in}{1.967226in}}%
\pgfpathlineto{\pgfqpoint{1.496591in}{2.052224in}}%
\pgfpathlineto{\pgfqpoint{1.538849in}{2.134357in}}%
\pgfpathlineto{\pgfqpoint{1.581108in}{2.213516in}}%
\pgfpathlineto{\pgfqpoint{1.623367in}{2.289586in}}%
\pgfpathlineto{\pgfqpoint{1.665625in}{2.362445in}}%
\pgfpathlineto{\pgfqpoint{1.707884in}{2.431964in}}%
\pgfpathlineto{\pgfqpoint{1.750143in}{2.498007in}}%
\pgfpathlineto{\pgfqpoint{1.792401in}{2.560430in}}%
\pgfpathlineto{\pgfqpoint{1.834660in}{2.619078in}}%
\pgfpathlineto{\pgfqpoint{1.876918in}{2.673788in}}%
\pgfpathlineto{\pgfqpoint{1.919177in}{2.724386in}}%
\pgfpathlineto{\pgfqpoint{1.961436in}{2.770686in}}%
\pgfpathlineto{\pgfqpoint{2.003694in}{2.812490in}}%
\pgfpathlineto{\pgfqpoint{2.045953in}{2.849586in}}%
\pgfpathlineto{\pgfqpoint{2.088212in}{2.881745in}}%
\pgfpathlineto{\pgfqpoint{2.130470in}{2.908726in}}%
\pgfpathlineto{\pgfqpoint{2.172729in}{2.930264in}}%
\pgfpathlineto{\pgfqpoint{2.214988in}{2.946079in}}%
\pgfpathlineto{\pgfqpoint{2.257246in}{2.955865in}}%
\pgfpathlineto{\pgfqpoint{2.299505in}{2.959313in}}%
\pgfpathlineto{\pgfqpoint{2.341763in}{2.959955in}}%
\pgfpathlineto{\pgfqpoint{2.384022in}{2.960598in}}%
\pgfpathlineto{\pgfqpoint{2.426281in}{2.961241in}}%
\pgfpathlineto{\pgfqpoint{2.468539in}{2.961883in}}%
\pgfpathlineto{\pgfqpoint{2.510798in}{2.962526in}}%
\pgfpathlineto{\pgfqpoint{2.553057in}{2.963168in}}%
\pgfpathlineto{\pgfqpoint{2.595315in}{2.963810in}}%
\pgfpathlineto{\pgfqpoint{2.637574in}{2.964452in}}%
\pgfpathlineto{\pgfqpoint{2.679832in}{2.965094in}}%
\pgfpathlineto{\pgfqpoint{2.722091in}{2.965735in}}%
\pgfpathlineto{\pgfqpoint{2.764350in}{2.966377in}}%
\pgfpathlineto{\pgfqpoint{2.806608in}{2.967018in}}%
\pgfpathlineto{\pgfqpoint{2.848867in}{2.967659in}}%
\pgfpathlineto{\pgfqpoint{2.891126in}{2.968301in}}%
\pgfpathlineto{\pgfqpoint{2.933384in}{2.968941in}}%
\pgfpathlineto{\pgfqpoint{2.975643in}{2.969582in}}%
\pgfpathlineto{\pgfqpoint{3.017901in}{2.970223in}}%
\pgfpathlineto{\pgfqpoint{3.060160in}{2.970863in}}%
\pgfpathlineto{\pgfqpoint{3.102419in}{2.971504in}}%
\pgfpathlineto{\pgfqpoint{3.144677in}{2.972144in}}%
\pgfpathlineto{\pgfqpoint{3.186936in}{2.972784in}}%
\pgfpathlineto{\pgfqpoint{3.229195in}{2.973424in}}%
\pgfpathlineto{\pgfqpoint{3.271453in}{2.974064in}}%
\pgfpathlineto{\pgfqpoint{3.313712in}{2.974704in}}%
\pgfpathlineto{\pgfqpoint{3.355971in}{2.975344in}}%
\pgfpathlineto{\pgfqpoint{3.398229in}{2.975983in}}%
\pgfpathlineto{\pgfqpoint{3.440488in}{2.976622in}}%
\pgfpathlineto{\pgfqpoint{3.482746in}{2.977261in}}%
\pgfpathlineto{\pgfqpoint{3.525005in}{2.977900in}}%
\pgfpathlineto{\pgfqpoint{3.567264in}{2.978539in}}%
\pgfpathlineto{\pgfqpoint{3.609522in}{2.979178in}}%
\pgfpathlineto{\pgfqpoint{3.651781in}{2.979816in}}%
\pgfpathlineto{\pgfqpoint{3.694040in}{2.980455in}}%
\pgfpathlineto{\pgfqpoint{3.736298in}{2.981093in}}%
\pgfpathlineto{\pgfqpoint{3.778557in}{2.981732in}}%
\pgfpathlineto{\pgfqpoint{3.820815in}{2.982370in}}%
\pgfpathlineto{\pgfqpoint{3.863074in}{2.983008in}}%
\pgfpathlineto{\pgfqpoint{3.905333in}{2.983646in}}%
\pgfpathlineto{\pgfqpoint{3.947591in}{2.984283in}}%
\pgfpathlineto{\pgfqpoint{3.989850in}{2.984921in}}%
\pgfpathlineto{\pgfqpoint{4.032109in}{2.985558in}}%
\pgfpathlineto{\pgfqpoint{4.074367in}{2.986195in}}%
\pgfpathlineto{\pgfqpoint{4.116626in}{2.986832in}}%
\pgfpathlineto{\pgfqpoint{4.158885in}{2.987470in}}%
\pgfpathlineto{\pgfqpoint{4.201143in}{2.988106in}}%
\pgfpathlineto{\pgfqpoint{4.243402in}{2.988743in}}%
\pgfpathlineto{\pgfqpoint{4.285660in}{2.989379in}}%
\pgfpathlineto{\pgfqpoint{4.327919in}{2.990016in}}%
\pgfpathlineto{\pgfqpoint{4.370178in}{2.990652in}}%
\pgfpathlineto{\pgfqpoint{4.412436in}{2.991288in}}%
\pgfpathlineto{\pgfqpoint{4.454695in}{2.991924in}}%
\pgfpathlineto{\pgfqpoint{4.496954in}{2.992560in}}%
\pgfpathlineto{\pgfqpoint{4.539212in}{2.993196in}}%
\pgfpathlineto{\pgfqpoint{4.581471in}{2.993831in}}%
\pgfpathlineto{\pgfqpoint{4.623729in}{2.994467in}}%
\pgfpathlineto{\pgfqpoint{4.665988in}{2.995102in}}%
\pgfpathlineto{\pgfqpoint{4.708247in}{2.995737in}}%
\pgfpathlineto{\pgfqpoint{4.750505in}{2.996373in}}%
\pgfpathlineto{\pgfqpoint{4.792764in}{2.997007in}}%
\pgfpathlineto{\pgfqpoint{4.835023in}{2.997642in}}%
\pgfpathlineto{\pgfqpoint{4.877281in}{2.998277in}}%
\pgfpathlineto{\pgfqpoint{4.919540in}{2.998911in}}%
\pgfpathlineto{\pgfqpoint{4.961798in}{2.999545in}}%
\pgfpathlineto{\pgfqpoint{5.004057in}{3.000180in}}%
\pgfpathlineto{\pgfqpoint{5.046316in}{3.000814in}}%
\pgfpathlineto{\pgfqpoint{5.088574in}{3.001448in}}%
\pgfpathlineto{\pgfqpoint{5.130833in}{3.002082in}}%
\pgfusepath{stroke}%
\end{pgfscope}%
\begin{pgfscope}%
\pgfsetrectcap%
\pgfsetmiterjoin%
\pgfsetlinewidth{0.803000pt}%
\definecolor{currentstroke}{rgb}{0.000000,0.000000,0.000000}%
\pgfsetstrokecolor{currentstroke}%
\pgfsetdash{}{0pt}%
\pgfpathmoveto{\pgfqpoint{0.693677in}{0.524170in}}%
\pgfpathlineto{\pgfqpoint{0.693677in}{3.120077in}}%
\pgfusepath{stroke}%
\end{pgfscope}%
\begin{pgfscope}%
\pgfsetrectcap%
\pgfsetmiterjoin%
\pgfsetlinewidth{0.803000pt}%
\definecolor{currentstroke}{rgb}{0.000000,0.000000,0.000000}%
\pgfsetstrokecolor{currentstroke}%
\pgfsetdash{}{0pt}%
\pgfpathmoveto{\pgfqpoint{5.342126in}{0.524170in}}%
\pgfpathlineto{\pgfqpoint{5.342126in}{3.120077in}}%
\pgfusepath{stroke}%
\end{pgfscope}%
\begin{pgfscope}%
\pgfsetrectcap%
\pgfsetmiterjoin%
\pgfsetlinewidth{0.803000pt}%
\definecolor{currentstroke}{rgb}{0.000000,0.000000,0.000000}%
\pgfsetstrokecolor{currentstroke}%
\pgfsetdash{}{0pt}%
\pgfpathmoveto{\pgfqpoint{0.693677in}{0.524170in}}%
\pgfpathlineto{\pgfqpoint{5.342126in}{0.524170in}}%
\pgfusepath{stroke}%
\end{pgfscope}%
\begin{pgfscope}%
\pgfsetrectcap%
\pgfsetmiterjoin%
\pgfsetlinewidth{0.803000pt}%
\definecolor{currentstroke}{rgb}{0.000000,0.000000,0.000000}%
\pgfsetstrokecolor{currentstroke}%
\pgfsetdash{}{0pt}%
\pgfpathmoveto{\pgfqpoint{0.693677in}{3.120077in}}%
\pgfpathlineto{\pgfqpoint{5.342126in}{3.120077in}}%
\pgfusepath{stroke}%
\end{pgfscope}%
\begin{pgfscope}%
\pgfsetbuttcap%
\pgfsetmiterjoin%
\definecolor{currentfill}{rgb}{1.000000,1.000000,1.000000}%
\pgfsetfillcolor{currentfill}%
\pgfsetfillopacity{0.800000}%
\pgfsetlinewidth{1.003750pt}%
\definecolor{currentstroke}{rgb}{0.800000,0.800000,0.800000}%
\pgfsetstrokecolor{currentstroke}%
\pgfsetstrokeopacity{0.800000}%
\pgfsetdash{}{0pt}%
\pgfpathmoveto{\pgfqpoint{0.771455in}{2.411634in}}%
\pgfpathlineto{\pgfqpoint{1.857795in}{2.411634in}}%
\pgfpathquadraticcurveto{\pgfqpoint{1.880017in}{2.411634in}}{\pgfqpoint{1.880017in}{2.433856in}}%
\pgfpathlineto{\pgfqpoint{1.880017in}{3.042300in}}%
\pgfpathquadraticcurveto{\pgfqpoint{1.880017in}{3.064522in}}{\pgfqpoint{1.857795in}{3.064522in}}%
\pgfpathlineto{\pgfqpoint{0.771455in}{3.064522in}}%
\pgfpathquadraticcurveto{\pgfqpoint{0.749232in}{3.064522in}}{\pgfqpoint{0.749232in}{3.042300in}}%
\pgfpathlineto{\pgfqpoint{0.749232in}{2.433856in}}%
\pgfpathquadraticcurveto{\pgfqpoint{0.749232in}{2.411634in}}{\pgfqpoint{0.771455in}{2.411634in}}%
\pgfpathlineto{\pgfqpoint{0.771455in}{2.411634in}}%
\pgfpathclose%
\pgfusepath{stroke,fill}%
\end{pgfscope}%
\begin{pgfscope}%
\pgfsetrectcap%
\pgfsetroundjoin%
\pgfsetlinewidth{1.003750pt}%
\definecolor{currentstroke}{rgb}{0.003922,0.450980,0.698039}%
\pgfsetstrokecolor{currentstroke}%
\pgfsetstrokeopacity{0.700000}%
\pgfsetdash{}{0pt}%
\pgfpathmoveto{\pgfqpoint{0.793677in}{2.981189in}}%
\pgfpathlineto{\pgfqpoint{0.904788in}{2.981189in}}%
\pgfpathlineto{\pgfqpoint{1.015899in}{2.981189in}}%
\pgfusepath{stroke}%
\end{pgfscope}%
\begin{pgfscope}%
\definecolor{textcolor}{rgb}{0.000000,0.000000,0.000000}%
\pgfsetstrokecolor{textcolor}%
\pgfsetfillcolor{textcolor}%
\pgftext[x=1.104788in,y=2.942300in,left,base]{\color{textcolor}\rmfamily\fontsize{8.000000}{9.600000}\selectfont \(\displaystyle V_{GS} = \qty{-3.5}{\V}\)}%
\end{pgfscope}%
\begin{pgfscope}%
\pgfsetrectcap%
\pgfsetroundjoin%
\pgfsetlinewidth{1.003750pt}%
\definecolor{currentstroke}{rgb}{0.870588,0.560784,0.019608}%
\pgfsetstrokecolor{currentstroke}%
\pgfsetstrokeopacity{0.700000}%
\pgfsetdash{}{0pt}%
\pgfpathmoveto{\pgfqpoint{0.793677in}{2.826300in}}%
\pgfpathlineto{\pgfqpoint{0.904788in}{2.826300in}}%
\pgfpathlineto{\pgfqpoint{1.015899in}{2.826300in}}%
\pgfusepath{stroke}%
\end{pgfscope}%
\begin{pgfscope}%
\definecolor{textcolor}{rgb}{0.000000,0.000000,0.000000}%
\pgfsetstrokecolor{textcolor}%
\pgfsetfillcolor{textcolor}%
\pgftext[x=1.104788in,y=2.787411in,left,base]{\color{textcolor}\rmfamily\fontsize{8.000000}{9.600000}\selectfont \(\displaystyle V_{GS} = \qty{-4}{\V}\)}%
\end{pgfscope}%
\begin{pgfscope}%
\pgfsetrectcap%
\pgfsetroundjoin%
\pgfsetlinewidth{1.003750pt}%
\definecolor{currentstroke}{rgb}{0.007843,0.619608,0.450980}%
\pgfsetstrokecolor{currentstroke}%
\pgfsetstrokeopacity{0.700000}%
\pgfsetdash{}{0pt}%
\pgfpathmoveto{\pgfqpoint{0.793677in}{2.671411in}}%
\pgfpathlineto{\pgfqpoint{0.904788in}{2.671411in}}%
\pgfpathlineto{\pgfqpoint{1.015899in}{2.671411in}}%
\pgfusepath{stroke}%
\end{pgfscope}%
\begin{pgfscope}%
\definecolor{textcolor}{rgb}{0.000000,0.000000,0.000000}%
\pgfsetstrokecolor{textcolor}%
\pgfsetfillcolor{textcolor}%
\pgftext[x=1.104788in,y=2.632522in,left,base]{\color{textcolor}\rmfamily\fontsize{8.000000}{9.600000}\selectfont \(\displaystyle V_{GS} = \qty{-4.5}{\V}\)}%
\end{pgfscope}%
\begin{pgfscope}%
\pgfsetrectcap%
\pgfsetroundjoin%
\pgfsetlinewidth{1.003750pt}%
\definecolor{currentstroke}{rgb}{0.835294,0.368627,0.000000}%
\pgfsetstrokecolor{currentstroke}%
\pgfsetstrokeopacity{0.700000}%
\pgfsetdash{}{0pt}%
\pgfpathmoveto{\pgfqpoint{0.793677in}{2.516522in}}%
\pgfpathlineto{\pgfqpoint{0.904788in}{2.516522in}}%
\pgfpathlineto{\pgfqpoint{1.015899in}{2.516522in}}%
\pgfusepath{stroke}%
\end{pgfscope}%
\begin{pgfscope}%
\definecolor{textcolor}{rgb}{0.000000,0.000000,0.000000}%
\pgfsetstrokecolor{textcolor}%
\pgfsetfillcolor{textcolor}%
\pgftext[x=1.104788in,y=2.477633in,left,base]{\color{textcolor}\rmfamily\fontsize{8.000000}{9.600000}\selectfont \(\displaystyle V_{GS} = \qty{-5}{\V}\)}%
\end{pgfscope}%
\end{pgfpicture}%
\makeatother%
\endgroup%

    \caption{Simulated drain current for different gate bias voltages of an \device{IRF9610} p-channel MOSFET.}
    \label{fig:fet_curret_gate_bias}
\end{figure}

Figure \ref{fig:fet_curret_gate_bias} shows the current $I_D$ flowing out of the drain of a p-channel MOSFET over the drain-to-source voltage $V_{DS}$ that is applied accross the FET. For illustrative purposes an example p-channel MOSFET was chosen and its \textit{Simulation Program with Integrated Circuit Emphasis} (SPICE) model \cite{irf9610_spice,irf9610_spice_better} was used to generate the data, yet the overall shape is the same for all FETs. For more information on modelling MOSFETs in SPICE, \citep[p. 442]{spice_mosfets} can be consulted. There are two regions, the first region, where $V_{DS} > V_{GS} - V_{th}$, demonstrates an almost linear linear correlation of the channel current and the voltage across the device. This is called the ohmic region, where the MOSFET behaves much like a (gate-) voltage controlled resistor and can be described \cite{shockley_fet_equations} as
\begin{equation}
    I_{D,ohmic} = \underbrace{\kappa (V_{GS} - V_{th}) V_{DS}}_{\text{ohmic}} - \underbrace{\frac 1 2 \kappa V_{DS}^2}_{\text{pinch off}} \, .
\end{equation}
For small voltages $V_{DS}$ the output current is proportional to the applied voltage $V_{DS}$ accross the channel, just like a normal resistor, hence calling its name: ohmic region. As the voltage increases further $I_D$ starts leveling off, because $V_{DS}$ starts affecting the channel conductivity. The channel is slowly getting pinched off at one end and becomes tapered. The reason is, that the voltage $V_{DS}$ is dropped accross the length of the channel. This voltage drop is linear with $V_{DS}$, resulting in a $-V_{DS}^2$ dependency of the current, reducing the conductivity of the channel. $V_{th}$ is called the threshold voltage of a MOSFET or pinch-off voltage $V_p$ in case of a JFET and is the voltage at which a current starts flowing.

The parameter $\kappa$ is a device specific parameter and depends on process parameters and the geometry of the device.
\begin{equation}
    \kappa = \kappa' \frac W L = \mu C_{ox} \frac W L
\end{equation}
$\mu$ is the electron mobility, which is about \qty{1350}{\square \cm \per \V} for n-channel MOSFETs and about \qty{540}{\square \cm \per \V} for p-channel MOSFETs \cite{fet_equations}. $C_{ox}$ is the gate-oxide capacitance per unit area and determined by the thickness $t_{ox}$ of the silicon dioxide layer of the gate
\begin{equation}
    C_{ox} = \frac{\epsilon_{ox}}{t_{ox}} \approx \frac{3.9 \cdot \epsilon_{0}}{t_{ox}} \approx \frac{\qty{3.45e-11}{\F \per \m}}{t_{ox}}\,,
\end{equation}
W is the width of the channel, and L is the length of the channel.

The letter $\kappa$ is used here instead of the usual $k$ as it is used by \citeauthor{fet_equations} \cite{fet_equations} to avoid confusion with the Boltzmann constant $k_B$. Unfortunately, $\kappa$ is not well controlled \cite{horowitz1989}, because it is not just determined by the size, but also the doping of the material. While the size of the structure can be well controlled to within a few \unit{\nm} using lithography masks, the doping is a matter of temperature and time in a diffusion furnace. The ohmic mode of operation is, for example, used in linear voltage regulators to control the output voltage of the regulator, forming a low impedance voltage source, not the desired current source. This brings up the next region to discuss.

Once the voltage $V_{DS}$ has reached $V_{GS} - V_{th}$, the channel is fully pinched off, any further increase in $V_{DS}$ will not lead to an increase in $I_D$, in other words the output resistance becomes infinite. The MOSFET is said to be pinched-off or in saturation. In practice there still is a small influence of $V_{DS}$ on the channel. While the depth can no longer decrease as its length is \num{0} at one end already, the channel will retract a small amount in length with increasing $V_{DS}$. This is taken into account by the factor $\lambda$, called channel-length modulation. The drain current in saturation can now be described \cite{shockley_fet_equations} as
\begin{equation}
    I_{D,sat} = \underbrace{\frac 1 2 \kappa \left(V_{GS} - V_{th} \right)^2}_{\text{ideal FET}} (1 + \lambda V_{DS}) \, . \label{eqn:mosfet_saturation}
\end{equation}

The parameter $\lambda$ is the first order Taylor exapansion of the length dependence of $\kappa$ and typically is small and on the order of \qtyrange[per-mode=power]{0.01}{0.05}{\per \volt} for p-channel MOSFETs \citep[p. 23]{mosfet_flicker_noise}. It mainly depends on the length of the channel to which it is inversely proportional, since the channel length defines the slope of the tapered channel. Sometimes the value $\frac{1}{\lambda}$ is also referred to as the Early voltage $V_A$. It is noteworthy, that more modern processes choose a smaller channel length to reduce the on-state resistance of the MOSFET, because the main application of a MOSFET nowadays is as a switch. The reduced channel length makes the MOSFET more susceptible to the channel length modulation effect. This will be discussed in more detail in section \ref{sec:component_selection}, when choosing a suitable MOSFET.

Going back to figure \ref{fig:fet_curret_gate_bias} the effect of the channel-length modulation can be seen as a small slope of $I_D$ in the saturation region.

Combining the previous equations, the FET drain current behaviour can be summed up as
\begin{equation}
    I_D = \begin{cases}
        0 & \text{if } V_{GS} - V_{th} < 0\\
        \kappa (V_{GS} - V_{th}) V_{DS} - \frac 1 2 \kappa V_{DS}^2 & \text{if } V_{GS} - V_{th} >= 0 \text{ and } V_{DS} < V_{GS} - V_{th}\\
        \frac 1 2 \kappa \left(V_{GS} - V_{th} \right)^2 (1 + \lambda V_{DS}) & \text{if } V_{GS} - V_{th} >= 0 \text{ and } V_{DS} \geq V_{GS} - V_{th}
    \end{cases}
    \label{eqn:mosfet_id_large_signal}
\end{equation}

The saturation region is the region of interest for building a high output impedance current source, because for a wide range of $V_{DS}$, the current remains almost constant and can be adjusted using the gate voltage $V_{GS}$. As a reminder, for the p-channel MOSFET, all voltages are reversed. $V_{GS}$, $V_{th}$, $V_{DS}$, $\kappa$ and $I_D$ are negative. Some datasheets therefore only give the magnitude of those quantities. The important aspect to remember, is that for the p-channel enhancement-mode MOSFET the gate must be biased negative with respect to the source pin by a least the threshold voltage ($V_{GS} < V_{th}$ or $|V_{GS}| > |V_{th}|$) to turn the transistor on and allow current to flow.

Before proceeding to the precision current source in section \ref{sec:precision_current_source}, the concept of conductance and transconductance must be explored. The transconductance describes the relationship of the input voltage with the output currrent. The conductance is a measure for how well current flows from input to output. The transconductance $g_m$ and the channel conductance $g_{DS}$ are defined as
\begin{align}
    g_{m, sat} &\coloneqq \left. \frac{\partial I_{D,sat}}{\partial V_{GS}} \right|_{V_{DS} = const} = \kappa \left(V_{GS} - V_{th} \right) (1 + \lambda V_{DS}) \, , \label{eqn:mosfet_gm}\\
    &= \sqrt{2 \kappa I_D \left(1+ \lambda V_{DS}\right)} \approx \sqrt{2 \kappa I_D} \label{eqn:mosfet_gm_approximation} \\
    g_{DS, sat} &\coloneqq \left. \frac{\partial I_{D,sat}}{\partial V_{DS}} \right|_{V_{GS} = const} = \frac{1}{2} \kappa \left(V_{GS} - V_{th} \right)^2 \lambda\\
    &= \frac{I_D}{\frac{1}{\lambda} + V_{DS}} = \frac{1}{R_o} \approx I_D \lambda \label{eqn:mosfet_gds}\,.
\end{align}
The transconductance $g_m$, as a measure of the current gain with respect to the gate-source voltage of the MOSFET, is proportional to the square root of the drain current $I_D$. The inverse of the channel conductance $g_{DS}$ is called output resistance $R_o$ and discussed below. Typically the $V_{DS}$ term in the denominator of the output resistance in equation \ref{eqn:mosfet_gds} can be neglected.

The meaning of $g_{m}$ and $g_{GS}$ can be best understood, when looking at a mathematical model of the MOSFET. These models come in varying complexity and either as a large-signal or small-signal model. Only the latter is used here. The small-signal model, is a first-order Taylor approximation around the working point, for a constant gate-source voltage $V_{GS}$ and constant drain-source $V_{DS}$, hence both $g_{m}$ and $g_{GS}$ are constants.
\begin{align}
    I_D &\approx \frac{\partial I_D}{\partial V_{GS}} \Delta V_{GS} + \frac{\partial I_D}{\partial V_{DS}} \Delta V_{DS}\\
    &= g_{m} \Delta V_{GS} + g_{DS} \Delta V_{DS}\\
    &= g_{m} v_{GS} + \frac{1}{R_o} v_{DS} = i_D \label{eqn:mosfet_id_small_signal}
\end{align}
The lower case letters denote the variables of the small-signal model as they only change very little compared to the working point parameters.
From \ref{eqn:mosfet_id_small_signal} it can be seen, that the $g_{DS}$ term adds to the output current and is proportional to $v_{DS}$. Comparing with figure \ref{fig:ideal_current_source_norton}, this the proportionality constant can be identified as $\frac{1}{R_o}$ like proposed above. Just like the ideal current source in figure \ref{fig:ideal_current_source}, the model can be given in the Norton or Thévenin representation both shown in figure \ref{fig:mostfet_small_signa_model}.

\begin{figure}[hb]
    \centering
    \begin{subfigure}{0.4\linewidth}
        \centering
        \import{figures/}{mosfet_small_signal.tex}
        \caption{Small signal model of a saturated MOSFET including the output resistance. The output resistance models the channel-length modulation as given by equation \ref{eqn:mosfet_id_small_signal}.}
        \label{fig:mostfet_small_signa_model_model_norton}
    \end{subfigure}
    \begin{subfigure}{0.4\linewidth}
        \centering
        \import{figures/}{mosfet_small_signal_t-model.tex}
        \caption{MOSFET model in Thévenin representation.}
        \label{fig:mostfet_small_signa_model_thevenin}
    \end{subfigure}
    \caption{Equivalent MOSFET models in Norton and Thévenin representations.}
    \label{fig:mostfet_small_signa_model}
\end{figure}

A detailed graphic derivation of the Thévenin representation can be found in \cite{fet_equations}. The Thévenin representation will prove especially valuable, when treating circuits with a resistance in the source leg.
The small-signal model now shows, that the output impedance is dependent on the channel-length modulation $\lambda$ and $v_{DS}$. Typically, $\frac{1}{\lambda} \gg v_{DS}$, so $\lambda$ is the most important factor governing the output impedance of a MOSFET.

To give an example of the output impedance of a MOSFET, parameters were taken from the aforementioned SPICE model of the \device{IRF9610}. Do note, that these parameters of a model are tuned to match certain operating conditions by their creatros and only present an estimation of the real MOSFET. Using the example parameters from \ref{tab:current_source_parameters} $I_D=\qty{250}{\mA}$, $\lambda = \qty[per-mode=power]{4}{\per \milli \volt}$, $V_{DS}=\qty{3.5}{\V}$ yields
\begin{equation}
    R_{out} = R_{o}\left(I_D=\qty{250}{\mA}, \lambda = \qty[per-mode=power]{4}{\per \milli \volt}\right) = \qty{1014}{\ohm} \overset{V_{DS} = 0}{\approx} \qty{1}{\kilo \ohm} \, , \label{eqn:mosfet_rout_irf9610}
\end{equation}
which is not very convincing as a current source. The small impact of $V_{DS}$ on the output impedance can be seen when dropping the $V_{GS}$ term, which leads to an output impedance of \qty{1}{\kilo \ohm}. Usually, in textbooks, this dependence is therefore neglected. To improve $R_{out}$, the focus thus lies on the $\lambda$ dependence. The model derived from equation \ref{eqn:mosfet_id_small_signal} can be used to do so, leading to the precision current source presented next.

% This can be demonstrated building a simple current source and then cascoding it. A very simple current source can be built using a JFET. As mentioned above, a JFET is a depletion-mode device and is already turned on at $V_{GS} = \qty{0}{\V}$. To turn it off the gate voltage must be increased above the source leg. For an illustration, refer to figure \ref{fig:jfet_curret_gate_bias}, which is very similar to the MOSFET behaviour.
%
% \begin{figure}[ht]
%     \centering
%     \input{images/jfet_current_gate_bias.pgf}
%     \caption{Simulated drain current for different gate bias voltages of a \device{2N5460} p-channel JSFET.}
%     \label{fig:jfet_curret_gate_bias}
% \end{figure}
%
% The topmost curve of figure \ref{fig:jfet_curret_gate_bias} is the case with a direct connection of the gate to the source. Above about $V_{DS}=\qty{4}{\V}$, the JFET works as a current source, although the effect of the Early voltage given in equation \ref{eqn:mosfet_saturation} can be clearly seen with a slight dependence of the output current on $V_{DS}$. With increasing $V_{GS}$, it can be observed, that the slope of $I_D$ flattens.
%
%
% An example for a cascode is shown in figure \ref{fig:current_source_jfet_cascode}.
%
% \begin{figure}[ht]
%     \centering
%     \begin{subfigure}[t]{0.3\linewidth}
%         \centering
%         \import{figures/}{current_source_fet_no_bias.tex}
%         \caption{JFET current source.}
%         \label{fig:current_source_jfet_no_bias}
%     \end{subfigure}%
%     %\hfill%
%     \begin{subfigure}[t]{0.3\linewidth}
%         \centering
%         \import{figures/}{current_source_fet_bias.tex}
%         \caption{Self-biased JFET current source.}
%         \label{fig:current_source_jfet_bias}
%     \end{subfigure}%
%     %\hfill%
%     \begin{subfigure}[t]{0.3\linewidth}
%         \centering
%         \import{figures/}{current_source_fet_cascode.tex}
%         \caption{Cascoded JFET current source.}
%         \label{fig:current_source_jfet_cascode}
%     \end{subfigure}
%     \caption{Different types of JFET current sources with increasing output impedance.}
%     \label{fig:current_source_jfet}
% \end{figure}

\clearpage
\subsection{Precision Current Source}
\label{sec:precision_current_source}
In the previous section \ref{sec:mosfet_current_source} it was shown in equation \ref{eqn:mosfet_id_small_signal}, that the output impedance of a MOSFET depends on the channel-length modulation $\lambda$ and is too low for practical purposes. On the quest to improve the output impedance of the MOSFET circuit \ref{fig:mostfet_small_signa_model_model_norton}, the most obvious solution would be to simply add a source resistor $R_S$ into the circuit as shown in in figure \ref{fig:pmos_current_source_resistor}. At first glance, this may seem to only add a series resistance to $R_o$, but the attempt is more interesting and will lead to an even better solution.

\begin{figure}[ht]
    \centering
    \begin{subfigure}[t]{0.45\linewidth}
        \centering
        \import{figures/}{current_source_resistor.tex}
        \caption{P-channel MOSFET with source resistor $R_S$ to improve the output impedance $R_{out}$.}
        \label{fig:pmos_current_source_resistor}
    \end{subfigure}%
    \begin{subfigure}[t]{0.45\linewidth}
         \centering
         \import{figures/}{pmos_small_signal_resistor.tex}
         \caption{Equivalent small-signal Thévenin model.}
         \label{fig:pmos_current_source_resistor_small_signal}
     \end{subfigure}%
     \caption{Circuit of a MOSFET with source degeneration resistor and equivalent Thévenin model.}
\end{figure}

Before calculating the output impedance, we shall have a look at $v_{GS}$ and the input signal $v_i$ derived from it. With the introduction of the sense resistor $R_S$, $v_i$ no longer equals $v_{GS}$, because $\frac{1}{g_m}$ now forms a voltage divider with $R_S$ and it follows
\begin{equation}
    v_{GS} = v_i \frac{\frac{1}{g_m}}{R_S + \frac{1}{g_m}} = v_i \frac{1}{1 + g_m R_S} \,.
\end{equation}
This implies a reduction in gain, by the factor $\frac{1}{1 + R_S g_m}$ compared to the previously discussed approach. The cause of this reduction is negative feedback. To understand this, imagine, that with a constant $v_i$ and hence a constant current $I_D$ flowing, a changing load resistance is trying to modulate $I_D$. Any increase in $I_D$ will cause the voltage across $R_S$ to rise, reducing $v_{GS}$, because $v_i$ is still constant. The decreasing $v_{GS}$ will then reduce $I_D$, thus introducing negative feedback. Having realized there is negative feedback present, it can be postulated, that the reduction in input sensitivity, or effective transconductance, will passed on to the output impedance. This very interesting relationship will now be derived.

To calculate the output impedance, figure \ref{fig:pmos_current_source_resistor_small_signal} can be simplified by grounding $v_i$, because there is no AC component as there is no current flowing through the insulated MOSFET gate and is not modulated. The load $R_{load}$ resistance must is replaced by an AC test voltage $v_{load}$ to modulate $I_D$. These changes result in the small signal model shown in figure \ref{fig:pmos_common_gate_amplifier}. As a sidenote, this configuration is also called a common-gate amplifier.

\begin{figure}[ht]
    \centering
    \import{figures/}{mosfet_small_signal_cg.tex}
    \caption{Small signal model of the common-gate amplifier with source resistance $R_S$.}
    \label{fig:pmos_common_gate_amplifier}
\end{figure}

The (dynamic) output impedance is given by
\begin{equation}
    R_{out,cg} = \frac{v_{load}}{i_D}\,, \label{eqn:mosfet_rout}
\end{equation}
with $i_D = i_S$, since there is no gate current. $v_{load}$ can easily be calculated by looking at figure \ref{fig:pmos_common_gate_amplifier} and is the total voltage across $R_o$ and $R_S$. $v_{GS}$ can also be found, because the gate is grounded. With the resistance $\frac{1}{g_m}$ at one end, the voltage at the source pin must be $-v_{GS}$.
\begin{align}
    v_{load} &= \left(i_D - i\right) R_o + i_S R_S \nonumber\\
    &= \left(i_D - g_m v_{gs}\right) R_o + i_D R_S \nonumber\\
    &= \left(i_D + g_m i_D R_S\right) R_o + i_D R_S \label{eqn:mosfet_cg_vout}
\end{align}
Using equations \ref{eqn:mosfet_rout} and \ref{eqn:mosfet_cg_vout} gives
\begin{equation}
    R_{out,cg} = \left(1 + g_m R_S\right) R_o + R_S \label{eqn:mosfet_cg_rout}
\end{equation}
for the output impedance.

This result is interesting, as it can be be immediately seen, that the output impedance scales very quickly with the transconductance $g_m$ and $R_S$. As it was already speculated above, the reduction in the transconductance $\frac{1}{1 + g_m R_S}$ of the MOSFET is transfered to the output impedance, which is increasing by the inverse of the loss in transconductance.

Going back to the quest for increased output impedance, it is appararent, that increasing $R_S$ quickly raises the output impedance, as it scales with $gm_m R_o$, but it would come at the cost of a significantly reduced compliance voltage. So, other means need to be explored. As we have seen, the scale factor $gm_m R_o$ is explained by feedback and this brings up another solution. The amount of feedback can be increased further using an operational amplifier (op-amp) as shown in figure \ref{fig:precision_current_source}.

\begin{figure}[ht]
    \centering
    \import{figures/}{precision_current_source.tex}
    \caption{Transconductance amplifier with a p-channel MOSFET.}
    \label{fig:precision_current_source}
\end{figure}

The output impedance of this transconductance amplifier is amplified by the open-loop gain of the op-amp as shown in appendix \ref{sec:transfer_function_transconductance}, while the transfer function greatly simplified and found to be
\begin{align}
    R_{out} &\approx A_{ol} \left(g_m R_o R_S + R_o + R_S \right) \nonumber\\
    I_{out} &\approx \frac{V_{ref}}{R_S} \label{eqn:current_source_transfer_function}
\end{align}

In addition to the increased output impedance, the current $I_D = I_{out}$ can now steered by adjusting $V_{ref}$ and is, given sufficient loop gain of the op-amp, no longer dependent on the MOSFET, but rather only on the sense resistor $R_S$.

This has the added benefit, that it is possible to leverage the tight accuracy and precision of a resistor, over the poor specifications of a MOSFET. Resistors can be manufactured with tolerances of less than \qty{100}{\micro \ohm \per \ohm}, which is orders of magnitude better than FETs, which can be matched to low \unit{\percent} values with patience.

Using the example parameters from table \ref{tab:current_source_parameters}, the output impedance in staturation can now be calculated again for $I_{out}=\qty{250}{\mA}$ and the ideal \device{IRF9610} model with the addition of an idealized \device{AD797} op-amp using the worst-case specifications.
\begin{equation}
    R_{out} \approx \qty[per-mode=power]{2}{\volt \per \uV} \left(\qty{0.64}{\siemens}\cdot \qty{1014}{\ohm} \cdot \qty{30}{\ohm} + \qty{1014}{\ohm} + \qty{30}{\ohm} \right) \approx \qty{40}{\giga\ohm}
\end{equation}

From these consideration, it can be seen, that the open-loop gain and the unity-gain bandwidth of the op-amp essentially determine the properties of the current source, given that $R_{id} \gg R_S$ and $R_o \gg R_S$. This will be important for selecting an operational amplifier later.

The next section will focus on the MOSFET and discuss the compliance voltage of the current source, which was only briefly touched during the introdution. It will give rise to criteria for selecting a MOSFET for the precision current source.

\clearpage
\subsection{Compliance Voltage}
\label{sec:compliance_voltage}
The compliance voltage of a current source is the maximum voltage it can output to maintain the requested output current. For an ideal current source, the compliance voltage is infinite, but obviously limited in the physical world.

The precision current source discussed in section \ref{sec:precision_current_source} has several limiting factors of the compliance voltage, which shall be discussed now. The compliance voltage is taxed most at the maximum output current $I_{out,max}$. So for the following discussion, the output is always treated as set to maximum.

Looking at figure \ref{fig:precision_current_source} of the precision current source it is immediately evident, that the output voltage can be calculated by subtracting the voltage accross the source resistor $V_{R_S}$ and the MOSFET $V_{DS}$ from the supply voltage $V_{sup}$
\begin{equation}
    V_{out} = V_{sup} - V_{R_S} - V_{DS} = V_{sup} - V_{ref} - V_{DS} \nonumber\,.
\end{equation}

The voltage $V_{R_S}$ is given by equation \ref{eqn:current_source_transfer_function} and equal to the setpoint voltage and hence given by the system parameters. This leads to the question of the minimum working point voltage $V_{DS}$ at $I_{out,max}$. As a reminder, from equation \ref{eqn:mosfet_id_large_signal} and figure \ref{fig:fet_curret_gate_bias} one can see, that the drain current is almost constant over $V_{DS}$ in the saturation region and in the ohmic region is proportional to $V_{DS}$. The transition point from the ohmic region to the saturation region is at $V_{DS} = V_{GS} - V_{th}$ and putting this into equation \ref{eqn:mosfet_id_large_signal} yields for the drain current
\begin{align}
    I_D &= \frac{1}{2} \kappa V_{DS}^2 \left(1+ \lambda V_{DS}\right) \nonumber\\
    \Rightarrow V_{DS} &\approx \sqrt{\frac{2 I_D}{\kappa}}\\
    &\approx \qty{784}{\mV}
\end{align}

The latter result was calculated using the example parameters from table \ref{tab:current_source_parameters}. At this point it can already be postulated, that the MOSFET will severly change in its function as a current source for $V_{DS} < \qty{0.78}{\V}$. To quantify this, one has to look at the output impedance of the transconductance amplifier once again. In the last section, the output impedance was only treated for the staturation region, but this time, $R_{out}$ must must be considered over a wide range of $V_{DS}$, thus not only in the saturation region, but also in the ohmic region. Instead of using the small-signal model as before, which assumed only small changes of $V_{DS}$, a large-signal model must applied, which also includes the non-linear nature of the piece-wise defined equation \ref{eqn:mosfet_id_large_signal} of the drain current.

For the sake of simplicity, a SPICE simulation of figure \ref{fig:precision_current_source} was carried out in LTSpice \cite{ltspice}. Solving this analytically, bears no educational value over the numerical solution shown below as will be seen. Additionally, the SPICE simulation also offers the opportunity to add additional, parasitic elements to the model to evaluate their effect, for example, the capacitive nature of the MOSFET gate.

The simulation itself is numerically fairly challenging and the typical approaches will lead to the limits of the numerical precision. To make simulation possible, the large-signal model is broken down into several small segements. For each these segments, the small-signal models at its respective working point is evaluated and then the result joined back together to reconstruct the large-signal model sought. How this is done in detail, is shown in appendix \ref{sec:ltspice_current_source} as it is beyond the scope of this section. The final result was calculated for two different frequencies, one frequency was deliberately chosen so low (\qty{1}{\micro\Hz}), that it is well below the dominant pole of the op-amp, meaning, that the full open-loop gain applies and the other frequency chosen was \qty{1}{\MHz}, were the gain had dropped to \qty[per-mode=power]{10}{\V \per V}. This is shown in figure \ref{fig:ltspice_output_impedance_simulation}.

\begin{figure}[ht]
    \centering
    %% Creator: Matplotlib, PGF backend
%%
%% To include the figure in your LaTeX document, write
%%   \input{<filename>.pgf}
%%
%% Make sure the required packages are loaded in your preamble
%%   \usepackage{pgf}
%%
%% Also ensure that all the required font packages are loaded; for instance,
%% the lmodern package is sometimes necessary when using math font.
%%   \usepackage{lmodern}
%%
%% Figures using additional raster images can only be included by \input if
%% they are in the same directory as the main LaTeX file. For loading figures
%% from other directories you can use the `import` package
%%   \usepackage{import}
%%
%% and then include the figures with
%%   \import{<path to file>}{<filename>.pgf}
%%
%% Matplotlib used the following preamble
%%   \def\mathdefault#1{#1}
%%   \everymath=\expandafter{\the\everymath\displaystyle}
%%   \usepackage{siunitx}
%%   \sisetup{per-mode = symbol}%
%%   \ifdefined\pdftexversion\else  % non-pdftex case.
%%     \usepackage{fontspec}
%%   \fi
%%   \makeatletter\@ifpackageloaded{underscore}{}{\usepackage[strings]{underscore}}\makeatother
%%
\begingroup%
\makeatletter%
\begin{pgfpicture}%
\pgfpathrectangle{\pgfpointorigin}{\pgfqpoint{5.431103in}{3.356606in}}%
\pgfusepath{use as bounding box, clip}%
\begin{pgfscope}%
\pgfsetbuttcap%
\pgfsetmiterjoin%
\definecolor{currentfill}{rgb}{1.000000,1.000000,1.000000}%
\pgfsetfillcolor{currentfill}%
\pgfsetlinewidth{0.000000pt}%
\definecolor{currentstroke}{rgb}{1.000000,1.000000,1.000000}%
\pgfsetstrokecolor{currentstroke}%
\pgfsetdash{}{0pt}%
\pgfpathmoveto{\pgfqpoint{0.000000in}{0.000000in}}%
\pgfpathlineto{\pgfqpoint{5.431103in}{0.000000in}}%
\pgfpathlineto{\pgfqpoint{5.431103in}{3.356606in}}%
\pgfpathlineto{\pgfqpoint{0.000000in}{3.356606in}}%
\pgfpathlineto{\pgfqpoint{0.000000in}{0.000000in}}%
\pgfpathclose%
\pgfusepath{fill}%
\end{pgfscope}%
\begin{pgfscope}%
\pgfsetbuttcap%
\pgfsetmiterjoin%
\definecolor{currentfill}{rgb}{1.000000,1.000000,1.000000}%
\pgfsetfillcolor{currentfill}%
\pgfsetlinewidth{0.000000pt}%
\definecolor{currentstroke}{rgb}{0.000000,0.000000,0.000000}%
\pgfsetstrokecolor{currentstroke}%
\pgfsetstrokeopacity{0.000000}%
\pgfsetdash{}{0pt}%
\pgfpathmoveto{\pgfqpoint{0.644859in}{0.524170in}}%
\pgfpathlineto{\pgfqpoint{5.281103in}{0.524170in}}%
\pgfpathlineto{\pgfqpoint{5.281103in}{3.189255in}}%
\pgfpathlineto{\pgfqpoint{0.644859in}{3.189255in}}%
\pgfpathlineto{\pgfqpoint{0.644859in}{0.524170in}}%
\pgfpathclose%
\pgfusepath{fill}%
\end{pgfscope}%
\begin{pgfscope}%
\pgfpathrectangle{\pgfqpoint{0.644859in}{0.524170in}}{\pgfqpoint{4.636243in}{2.665085in}}%
\pgfusepath{clip}%
\pgfsetrectcap%
\pgfsetroundjoin%
\pgfsetlinewidth{0.803000pt}%
\definecolor{currentstroke}{rgb}{0.450000,0.450000,0.450000}%
\pgfsetstrokecolor{currentstroke}%
\pgfsetdash{}{0pt}%
\pgfpathmoveto{\pgfqpoint{0.855598in}{0.524170in}}%
\pgfpathlineto{\pgfqpoint{0.855598in}{3.189255in}}%
\pgfusepath{stroke}%
\end{pgfscope}%
\begin{pgfscope}%
\pgfsetbuttcap%
\pgfsetroundjoin%
\definecolor{currentfill}{rgb}{0.000000,0.000000,0.000000}%
\pgfsetfillcolor{currentfill}%
\pgfsetlinewidth{0.803000pt}%
\definecolor{currentstroke}{rgb}{0.000000,0.000000,0.000000}%
\pgfsetstrokecolor{currentstroke}%
\pgfsetdash{}{0pt}%
\pgfsys@defobject{currentmarker}{\pgfqpoint{0.000000in}{-0.048611in}}{\pgfqpoint{0.000000in}{0.000000in}}{%
\pgfpathmoveto{\pgfqpoint{0.000000in}{0.000000in}}%
\pgfpathlineto{\pgfqpoint{0.000000in}{-0.048611in}}%
\pgfusepath{stroke,fill}%
}%
\begin{pgfscope}%
\pgfsys@transformshift{0.855598in}{0.524170in}%
\pgfsys@useobject{currentmarker}{}%
\end{pgfscope}%
\end{pgfscope}%
\begin{pgfscope}%
\definecolor{textcolor}{rgb}{0.000000,0.000000,0.000000}%
\pgfsetstrokecolor{textcolor}%
\pgfsetfillcolor{textcolor}%
\pgftext[x=0.855598in,y=0.426948in,,top]{\color{textcolor}{\rmfamily\fontsize{8.000000}{9.600000}\selectfont\catcode`\^=\active\def^{\ifmmode\sp\else\^{}\fi}\catcode`\%=\active\def%{\%}$\mathdefault{0.0}$}}%
\end{pgfscope}%
\begin{pgfscope}%
\pgfpathrectangle{\pgfqpoint{0.644859in}{0.524170in}}{\pgfqpoint{4.636243in}{2.665085in}}%
\pgfusepath{clip}%
\pgfsetrectcap%
\pgfsetroundjoin%
\pgfsetlinewidth{0.803000pt}%
\definecolor{currentstroke}{rgb}{0.450000,0.450000,0.450000}%
\pgfsetstrokecolor{currentstroke}%
\pgfsetdash{}{0pt}%
\pgfpathmoveto{\pgfqpoint{1.698551in}{0.524170in}}%
\pgfpathlineto{\pgfqpoint{1.698551in}{3.189255in}}%
\pgfusepath{stroke}%
\end{pgfscope}%
\begin{pgfscope}%
\pgfsetbuttcap%
\pgfsetroundjoin%
\definecolor{currentfill}{rgb}{0.000000,0.000000,0.000000}%
\pgfsetfillcolor{currentfill}%
\pgfsetlinewidth{0.803000pt}%
\definecolor{currentstroke}{rgb}{0.000000,0.000000,0.000000}%
\pgfsetstrokecolor{currentstroke}%
\pgfsetdash{}{0pt}%
\pgfsys@defobject{currentmarker}{\pgfqpoint{0.000000in}{-0.048611in}}{\pgfqpoint{0.000000in}{0.000000in}}{%
\pgfpathmoveto{\pgfqpoint{0.000000in}{0.000000in}}%
\pgfpathlineto{\pgfqpoint{0.000000in}{-0.048611in}}%
\pgfusepath{stroke,fill}%
}%
\begin{pgfscope}%
\pgfsys@transformshift{1.698551in}{0.524170in}%
\pgfsys@useobject{currentmarker}{}%
\end{pgfscope}%
\end{pgfscope}%
\begin{pgfscope}%
\definecolor{textcolor}{rgb}{0.000000,0.000000,0.000000}%
\pgfsetstrokecolor{textcolor}%
\pgfsetfillcolor{textcolor}%
\pgftext[x=1.698551in,y=0.426948in,,top]{\color{textcolor}{\rmfamily\fontsize{8.000000}{9.600000}\selectfont\catcode`\^=\active\def^{\ifmmode\sp\else\^{}\fi}\catcode`\%=\active\def%{\%}$\mathdefault{0.2}$}}%
\end{pgfscope}%
\begin{pgfscope}%
\pgfpathrectangle{\pgfqpoint{0.644859in}{0.524170in}}{\pgfqpoint{4.636243in}{2.665085in}}%
\pgfusepath{clip}%
\pgfsetrectcap%
\pgfsetroundjoin%
\pgfsetlinewidth{0.803000pt}%
\definecolor{currentstroke}{rgb}{0.450000,0.450000,0.450000}%
\pgfsetstrokecolor{currentstroke}%
\pgfsetdash{}{0pt}%
\pgfpathmoveto{\pgfqpoint{2.541504in}{0.524170in}}%
\pgfpathlineto{\pgfqpoint{2.541504in}{3.189255in}}%
\pgfusepath{stroke}%
\end{pgfscope}%
\begin{pgfscope}%
\pgfsetbuttcap%
\pgfsetroundjoin%
\definecolor{currentfill}{rgb}{0.000000,0.000000,0.000000}%
\pgfsetfillcolor{currentfill}%
\pgfsetlinewidth{0.803000pt}%
\definecolor{currentstroke}{rgb}{0.000000,0.000000,0.000000}%
\pgfsetstrokecolor{currentstroke}%
\pgfsetdash{}{0pt}%
\pgfsys@defobject{currentmarker}{\pgfqpoint{0.000000in}{-0.048611in}}{\pgfqpoint{0.000000in}{0.000000in}}{%
\pgfpathmoveto{\pgfqpoint{0.000000in}{0.000000in}}%
\pgfpathlineto{\pgfqpoint{0.000000in}{-0.048611in}}%
\pgfusepath{stroke,fill}%
}%
\begin{pgfscope}%
\pgfsys@transformshift{2.541504in}{0.524170in}%
\pgfsys@useobject{currentmarker}{}%
\end{pgfscope}%
\end{pgfscope}%
\begin{pgfscope}%
\definecolor{textcolor}{rgb}{0.000000,0.000000,0.000000}%
\pgfsetstrokecolor{textcolor}%
\pgfsetfillcolor{textcolor}%
\pgftext[x=2.541504in,y=0.426948in,,top]{\color{textcolor}{\rmfamily\fontsize{8.000000}{9.600000}\selectfont\catcode`\^=\active\def^{\ifmmode\sp\else\^{}\fi}\catcode`\%=\active\def%{\%}$\mathdefault{0.4}$}}%
\end{pgfscope}%
\begin{pgfscope}%
\pgfpathrectangle{\pgfqpoint{0.644859in}{0.524170in}}{\pgfqpoint{4.636243in}{2.665085in}}%
\pgfusepath{clip}%
\pgfsetrectcap%
\pgfsetroundjoin%
\pgfsetlinewidth{0.803000pt}%
\definecolor{currentstroke}{rgb}{0.450000,0.450000,0.450000}%
\pgfsetstrokecolor{currentstroke}%
\pgfsetdash{}{0pt}%
\pgfpathmoveto{\pgfqpoint{3.384458in}{0.524170in}}%
\pgfpathlineto{\pgfqpoint{3.384458in}{3.189255in}}%
\pgfusepath{stroke}%
\end{pgfscope}%
\begin{pgfscope}%
\pgfsetbuttcap%
\pgfsetroundjoin%
\definecolor{currentfill}{rgb}{0.000000,0.000000,0.000000}%
\pgfsetfillcolor{currentfill}%
\pgfsetlinewidth{0.803000pt}%
\definecolor{currentstroke}{rgb}{0.000000,0.000000,0.000000}%
\pgfsetstrokecolor{currentstroke}%
\pgfsetdash{}{0pt}%
\pgfsys@defobject{currentmarker}{\pgfqpoint{0.000000in}{-0.048611in}}{\pgfqpoint{0.000000in}{0.000000in}}{%
\pgfpathmoveto{\pgfqpoint{0.000000in}{0.000000in}}%
\pgfpathlineto{\pgfqpoint{0.000000in}{-0.048611in}}%
\pgfusepath{stroke,fill}%
}%
\begin{pgfscope}%
\pgfsys@transformshift{3.384458in}{0.524170in}%
\pgfsys@useobject{currentmarker}{}%
\end{pgfscope}%
\end{pgfscope}%
\begin{pgfscope}%
\definecolor{textcolor}{rgb}{0.000000,0.000000,0.000000}%
\pgfsetstrokecolor{textcolor}%
\pgfsetfillcolor{textcolor}%
\pgftext[x=3.384458in,y=0.426948in,,top]{\color{textcolor}{\rmfamily\fontsize{8.000000}{9.600000}\selectfont\catcode`\^=\active\def^{\ifmmode\sp\else\^{}\fi}\catcode`\%=\active\def%{\%}$\mathdefault{0.6}$}}%
\end{pgfscope}%
\begin{pgfscope}%
\pgfpathrectangle{\pgfqpoint{0.644859in}{0.524170in}}{\pgfqpoint{4.636243in}{2.665085in}}%
\pgfusepath{clip}%
\pgfsetrectcap%
\pgfsetroundjoin%
\pgfsetlinewidth{0.803000pt}%
\definecolor{currentstroke}{rgb}{0.450000,0.450000,0.450000}%
\pgfsetstrokecolor{currentstroke}%
\pgfsetdash{}{0pt}%
\pgfpathmoveto{\pgfqpoint{4.227411in}{0.524170in}}%
\pgfpathlineto{\pgfqpoint{4.227411in}{3.189255in}}%
\pgfusepath{stroke}%
\end{pgfscope}%
\begin{pgfscope}%
\pgfsetbuttcap%
\pgfsetroundjoin%
\definecolor{currentfill}{rgb}{0.000000,0.000000,0.000000}%
\pgfsetfillcolor{currentfill}%
\pgfsetlinewidth{0.803000pt}%
\definecolor{currentstroke}{rgb}{0.000000,0.000000,0.000000}%
\pgfsetstrokecolor{currentstroke}%
\pgfsetdash{}{0pt}%
\pgfsys@defobject{currentmarker}{\pgfqpoint{0.000000in}{-0.048611in}}{\pgfqpoint{0.000000in}{0.000000in}}{%
\pgfpathmoveto{\pgfqpoint{0.000000in}{0.000000in}}%
\pgfpathlineto{\pgfqpoint{0.000000in}{-0.048611in}}%
\pgfusepath{stroke,fill}%
}%
\begin{pgfscope}%
\pgfsys@transformshift{4.227411in}{0.524170in}%
\pgfsys@useobject{currentmarker}{}%
\end{pgfscope}%
\end{pgfscope}%
\begin{pgfscope}%
\definecolor{textcolor}{rgb}{0.000000,0.000000,0.000000}%
\pgfsetstrokecolor{textcolor}%
\pgfsetfillcolor{textcolor}%
\pgftext[x=4.227411in,y=0.426948in,,top]{\color{textcolor}{\rmfamily\fontsize{8.000000}{9.600000}\selectfont\catcode`\^=\active\def^{\ifmmode\sp\else\^{}\fi}\catcode`\%=\active\def%{\%}$\mathdefault{0.8}$}}%
\end{pgfscope}%
\begin{pgfscope}%
\pgfpathrectangle{\pgfqpoint{0.644859in}{0.524170in}}{\pgfqpoint{4.636243in}{2.665085in}}%
\pgfusepath{clip}%
\pgfsetrectcap%
\pgfsetroundjoin%
\pgfsetlinewidth{0.803000pt}%
\definecolor{currentstroke}{rgb}{0.450000,0.450000,0.450000}%
\pgfsetstrokecolor{currentstroke}%
\pgfsetdash{}{0pt}%
\pgfpathmoveto{\pgfqpoint{5.070364in}{0.524170in}}%
\pgfpathlineto{\pgfqpoint{5.070364in}{3.189255in}}%
\pgfusepath{stroke}%
\end{pgfscope}%
\begin{pgfscope}%
\pgfsetbuttcap%
\pgfsetroundjoin%
\definecolor{currentfill}{rgb}{0.000000,0.000000,0.000000}%
\pgfsetfillcolor{currentfill}%
\pgfsetlinewidth{0.803000pt}%
\definecolor{currentstroke}{rgb}{0.000000,0.000000,0.000000}%
\pgfsetstrokecolor{currentstroke}%
\pgfsetdash{}{0pt}%
\pgfsys@defobject{currentmarker}{\pgfqpoint{0.000000in}{-0.048611in}}{\pgfqpoint{0.000000in}{0.000000in}}{%
\pgfpathmoveto{\pgfqpoint{0.000000in}{0.000000in}}%
\pgfpathlineto{\pgfqpoint{0.000000in}{-0.048611in}}%
\pgfusepath{stroke,fill}%
}%
\begin{pgfscope}%
\pgfsys@transformshift{5.070364in}{0.524170in}%
\pgfsys@useobject{currentmarker}{}%
\end{pgfscope}%
\end{pgfscope}%
\begin{pgfscope}%
\definecolor{textcolor}{rgb}{0.000000,0.000000,0.000000}%
\pgfsetstrokecolor{textcolor}%
\pgfsetfillcolor{textcolor}%
\pgftext[x=5.070364in,y=0.426948in,,top]{\color{textcolor}{\rmfamily\fontsize{8.000000}{9.600000}\selectfont\catcode`\^=\active\def^{\ifmmode\sp\else\^{}\fi}\catcode`\%=\active\def%{\%}$\mathdefault{1.0}$}}%
\end{pgfscope}%
\begin{pgfscope}%
\definecolor{textcolor}{rgb}{0.000000,0.000000,0.000000}%
\pgfsetstrokecolor{textcolor}%
\pgfsetfillcolor{textcolor}%
\pgftext[x=2.962981in,y=0.272725in,,top]{\color{textcolor}{\rmfamily\fontsize{10.000000}{12.000000}\selectfont\catcode`\^=\active\def^{\ifmmode\sp\else\^{}\fi}\catcode`\%=\active\def%{\%}Drain-source voltage $V_{DS}$ in \unit{\V}}}%
\end{pgfscope}%
\begin{pgfscope}%
\pgfpathrectangle{\pgfqpoint{0.644859in}{0.524170in}}{\pgfqpoint{4.636243in}{2.665085in}}%
\pgfusepath{clip}%
\pgfsetrectcap%
\pgfsetroundjoin%
\pgfsetlinewidth{0.803000pt}%
\definecolor{currentstroke}{rgb}{0.450000,0.450000,0.450000}%
\pgfsetstrokecolor{currentstroke}%
\pgfsetdash{}{0pt}%
\pgfpathmoveto{\pgfqpoint{0.644859in}{0.770636in}}%
\pgfpathlineto{\pgfqpoint{5.281103in}{0.770636in}}%
\pgfusepath{stroke}%
\end{pgfscope}%
\begin{pgfscope}%
\pgfsetbuttcap%
\pgfsetroundjoin%
\definecolor{currentfill}{rgb}{0.000000,0.000000,0.000000}%
\pgfsetfillcolor{currentfill}%
\pgfsetlinewidth{0.803000pt}%
\definecolor{currentstroke}{rgb}{0.000000,0.000000,0.000000}%
\pgfsetstrokecolor{currentstroke}%
\pgfsetdash{}{0pt}%
\pgfsys@defobject{currentmarker}{\pgfqpoint{-0.048611in}{0.000000in}}{\pgfqpoint{-0.000000in}{0.000000in}}{%
\pgfpathmoveto{\pgfqpoint{-0.000000in}{0.000000in}}%
\pgfpathlineto{\pgfqpoint{-0.048611in}{0.000000in}}%
\pgfusepath{stroke,fill}%
}%
\begin{pgfscope}%
\pgfsys@transformshift{0.644859in}{0.770636in}%
\pgfsys@useobject{currentmarker}{}%
\end{pgfscope}%
\end{pgfscope}%
\begin{pgfscope}%
\definecolor{textcolor}{rgb}{0.000000,0.000000,0.000000}%
\pgfsetstrokecolor{textcolor}%
\pgfsetfillcolor{textcolor}%
\pgftext[x=0.371710in, y=0.731483in, left, base]{\color{textcolor}{\rmfamily\fontsize{8.000000}{9.600000}\selectfont\catcode`\^=\active\def^{\ifmmode\sp\else\^{}\fi}\catcode`\%=\active\def%{\%}$\mathdefault{10^{2}}$}}%
\end{pgfscope}%
\begin{pgfscope}%
\pgfpathrectangle{\pgfqpoint{0.644859in}{0.524170in}}{\pgfqpoint{4.636243in}{2.665085in}}%
\pgfusepath{clip}%
\pgfsetrectcap%
\pgfsetroundjoin%
\pgfsetlinewidth{0.803000pt}%
\definecolor{currentstroke}{rgb}{0.450000,0.450000,0.450000}%
\pgfsetstrokecolor{currentstroke}%
\pgfsetdash{}{0pt}%
\pgfpathmoveto{\pgfqpoint{0.644859in}{1.250011in}}%
\pgfpathlineto{\pgfqpoint{5.281103in}{1.250011in}}%
\pgfusepath{stroke}%
\end{pgfscope}%
\begin{pgfscope}%
\pgfsetbuttcap%
\pgfsetroundjoin%
\definecolor{currentfill}{rgb}{0.000000,0.000000,0.000000}%
\pgfsetfillcolor{currentfill}%
\pgfsetlinewidth{0.803000pt}%
\definecolor{currentstroke}{rgb}{0.000000,0.000000,0.000000}%
\pgfsetstrokecolor{currentstroke}%
\pgfsetdash{}{0pt}%
\pgfsys@defobject{currentmarker}{\pgfqpoint{-0.048611in}{0.000000in}}{\pgfqpoint{-0.000000in}{0.000000in}}{%
\pgfpathmoveto{\pgfqpoint{-0.000000in}{0.000000in}}%
\pgfpathlineto{\pgfqpoint{-0.048611in}{0.000000in}}%
\pgfusepath{stroke,fill}%
}%
\begin{pgfscope}%
\pgfsys@transformshift{0.644859in}{1.250011in}%
\pgfsys@useobject{currentmarker}{}%
\end{pgfscope}%
\end{pgfscope}%
\begin{pgfscope}%
\definecolor{textcolor}{rgb}{0.000000,0.000000,0.000000}%
\pgfsetstrokecolor{textcolor}%
\pgfsetfillcolor{textcolor}%
\pgftext[x=0.371710in, y=1.210858in, left, base]{\color{textcolor}{\rmfamily\fontsize{8.000000}{9.600000}\selectfont\catcode`\^=\active\def^{\ifmmode\sp\else\^{}\fi}\catcode`\%=\active\def%{\%}$\mathdefault{10^{4}}$}}%
\end{pgfscope}%
\begin{pgfscope}%
\pgfpathrectangle{\pgfqpoint{0.644859in}{0.524170in}}{\pgfqpoint{4.636243in}{2.665085in}}%
\pgfusepath{clip}%
\pgfsetrectcap%
\pgfsetroundjoin%
\pgfsetlinewidth{0.803000pt}%
\definecolor{currentstroke}{rgb}{0.450000,0.450000,0.450000}%
\pgfsetstrokecolor{currentstroke}%
\pgfsetdash{}{0pt}%
\pgfpathmoveto{\pgfqpoint{0.644859in}{1.729387in}}%
\pgfpathlineto{\pgfqpoint{5.281103in}{1.729387in}}%
\pgfusepath{stroke}%
\end{pgfscope}%
\begin{pgfscope}%
\pgfsetbuttcap%
\pgfsetroundjoin%
\definecolor{currentfill}{rgb}{0.000000,0.000000,0.000000}%
\pgfsetfillcolor{currentfill}%
\pgfsetlinewidth{0.803000pt}%
\definecolor{currentstroke}{rgb}{0.000000,0.000000,0.000000}%
\pgfsetstrokecolor{currentstroke}%
\pgfsetdash{}{0pt}%
\pgfsys@defobject{currentmarker}{\pgfqpoint{-0.048611in}{0.000000in}}{\pgfqpoint{-0.000000in}{0.000000in}}{%
\pgfpathmoveto{\pgfqpoint{-0.000000in}{0.000000in}}%
\pgfpathlineto{\pgfqpoint{-0.048611in}{0.000000in}}%
\pgfusepath{stroke,fill}%
}%
\begin{pgfscope}%
\pgfsys@transformshift{0.644859in}{1.729387in}%
\pgfsys@useobject{currentmarker}{}%
\end{pgfscope}%
\end{pgfscope}%
\begin{pgfscope}%
\definecolor{textcolor}{rgb}{0.000000,0.000000,0.000000}%
\pgfsetstrokecolor{textcolor}%
\pgfsetfillcolor{textcolor}%
\pgftext[x=0.371710in, y=1.690234in, left, base]{\color{textcolor}{\rmfamily\fontsize{8.000000}{9.600000}\selectfont\catcode`\^=\active\def^{\ifmmode\sp\else\^{}\fi}\catcode`\%=\active\def%{\%}$\mathdefault{10^{6}}$}}%
\end{pgfscope}%
\begin{pgfscope}%
\pgfpathrectangle{\pgfqpoint{0.644859in}{0.524170in}}{\pgfqpoint{4.636243in}{2.665085in}}%
\pgfusepath{clip}%
\pgfsetrectcap%
\pgfsetroundjoin%
\pgfsetlinewidth{0.803000pt}%
\definecolor{currentstroke}{rgb}{0.450000,0.450000,0.450000}%
\pgfsetstrokecolor{currentstroke}%
\pgfsetdash{}{0pt}%
\pgfpathmoveto{\pgfqpoint{0.644859in}{2.208762in}}%
\pgfpathlineto{\pgfqpoint{5.281103in}{2.208762in}}%
\pgfusepath{stroke}%
\end{pgfscope}%
\begin{pgfscope}%
\pgfsetbuttcap%
\pgfsetroundjoin%
\definecolor{currentfill}{rgb}{0.000000,0.000000,0.000000}%
\pgfsetfillcolor{currentfill}%
\pgfsetlinewidth{0.803000pt}%
\definecolor{currentstroke}{rgb}{0.000000,0.000000,0.000000}%
\pgfsetstrokecolor{currentstroke}%
\pgfsetdash{}{0pt}%
\pgfsys@defobject{currentmarker}{\pgfqpoint{-0.048611in}{0.000000in}}{\pgfqpoint{-0.000000in}{0.000000in}}{%
\pgfpathmoveto{\pgfqpoint{-0.000000in}{0.000000in}}%
\pgfpathlineto{\pgfqpoint{-0.048611in}{0.000000in}}%
\pgfusepath{stroke,fill}%
}%
\begin{pgfscope}%
\pgfsys@transformshift{0.644859in}{2.208762in}%
\pgfsys@useobject{currentmarker}{}%
\end{pgfscope}%
\end{pgfscope}%
\begin{pgfscope}%
\definecolor{textcolor}{rgb}{0.000000,0.000000,0.000000}%
\pgfsetstrokecolor{textcolor}%
\pgfsetfillcolor{textcolor}%
\pgftext[x=0.371710in, y=2.169609in, left, base]{\color{textcolor}{\rmfamily\fontsize{8.000000}{9.600000}\selectfont\catcode`\^=\active\def^{\ifmmode\sp\else\^{}\fi}\catcode`\%=\active\def%{\%}$\mathdefault{10^{8}}$}}%
\end{pgfscope}%
\begin{pgfscope}%
\pgfpathrectangle{\pgfqpoint{0.644859in}{0.524170in}}{\pgfqpoint{4.636243in}{2.665085in}}%
\pgfusepath{clip}%
\pgfsetrectcap%
\pgfsetroundjoin%
\pgfsetlinewidth{0.803000pt}%
\definecolor{currentstroke}{rgb}{0.450000,0.450000,0.450000}%
\pgfsetstrokecolor{currentstroke}%
\pgfsetdash{}{0pt}%
\pgfpathmoveto{\pgfqpoint{0.644859in}{2.688138in}}%
\pgfpathlineto{\pgfqpoint{5.281103in}{2.688138in}}%
\pgfusepath{stroke}%
\end{pgfscope}%
\begin{pgfscope}%
\pgfsetbuttcap%
\pgfsetroundjoin%
\definecolor{currentfill}{rgb}{0.000000,0.000000,0.000000}%
\pgfsetfillcolor{currentfill}%
\pgfsetlinewidth{0.803000pt}%
\definecolor{currentstroke}{rgb}{0.000000,0.000000,0.000000}%
\pgfsetstrokecolor{currentstroke}%
\pgfsetdash{}{0pt}%
\pgfsys@defobject{currentmarker}{\pgfqpoint{-0.048611in}{0.000000in}}{\pgfqpoint{-0.000000in}{0.000000in}}{%
\pgfpathmoveto{\pgfqpoint{-0.000000in}{0.000000in}}%
\pgfpathlineto{\pgfqpoint{-0.048611in}{0.000000in}}%
\pgfusepath{stroke,fill}%
}%
\begin{pgfscope}%
\pgfsys@transformshift{0.644859in}{2.688138in}%
\pgfsys@useobject{currentmarker}{}%
\end{pgfscope}%
\end{pgfscope}%
\begin{pgfscope}%
\definecolor{textcolor}{rgb}{0.000000,0.000000,0.000000}%
\pgfsetstrokecolor{textcolor}%
\pgfsetfillcolor{textcolor}%
\pgftext[x=0.320785in, y=2.648985in, left, base]{\color{textcolor}{\rmfamily\fontsize{8.000000}{9.600000}\selectfont\catcode`\^=\active\def^{\ifmmode\sp\else\^{}\fi}\catcode`\%=\active\def%{\%}$\mathdefault{10^{10}}$}}%
\end{pgfscope}%
\begin{pgfscope}%
\pgfpathrectangle{\pgfqpoint{0.644859in}{0.524170in}}{\pgfqpoint{4.636243in}{2.665085in}}%
\pgfusepath{clip}%
\pgfsetrectcap%
\pgfsetroundjoin%
\pgfsetlinewidth{0.803000pt}%
\definecolor{currentstroke}{rgb}{0.450000,0.450000,0.450000}%
\pgfsetstrokecolor{currentstroke}%
\pgfsetdash{}{0pt}%
\pgfpathmoveto{\pgfqpoint{0.644859in}{3.167513in}}%
\pgfpathlineto{\pgfqpoint{5.281103in}{3.167513in}}%
\pgfusepath{stroke}%
\end{pgfscope}%
\begin{pgfscope}%
\pgfsetbuttcap%
\pgfsetroundjoin%
\definecolor{currentfill}{rgb}{0.000000,0.000000,0.000000}%
\pgfsetfillcolor{currentfill}%
\pgfsetlinewidth{0.803000pt}%
\definecolor{currentstroke}{rgb}{0.000000,0.000000,0.000000}%
\pgfsetstrokecolor{currentstroke}%
\pgfsetdash{}{0pt}%
\pgfsys@defobject{currentmarker}{\pgfqpoint{-0.048611in}{0.000000in}}{\pgfqpoint{-0.000000in}{0.000000in}}{%
\pgfpathmoveto{\pgfqpoint{-0.000000in}{0.000000in}}%
\pgfpathlineto{\pgfqpoint{-0.048611in}{0.000000in}}%
\pgfusepath{stroke,fill}%
}%
\begin{pgfscope}%
\pgfsys@transformshift{0.644859in}{3.167513in}%
\pgfsys@useobject{currentmarker}{}%
\end{pgfscope}%
\end{pgfscope}%
\begin{pgfscope}%
\definecolor{textcolor}{rgb}{0.000000,0.000000,0.000000}%
\pgfsetstrokecolor{textcolor}%
\pgfsetfillcolor{textcolor}%
\pgftext[x=0.320785in, y=3.128360in, left, base]{\color{textcolor}{\rmfamily\fontsize{8.000000}{9.600000}\selectfont\catcode`\^=\active\def^{\ifmmode\sp\else\^{}\fi}\catcode`\%=\active\def%{\%}$\mathdefault{10^{12}}$}}%
\end{pgfscope}%
\begin{pgfscope}%
\definecolor{textcolor}{rgb}{0.000000,0.000000,0.000000}%
\pgfsetstrokecolor{textcolor}%
\pgfsetfillcolor{textcolor}%
\pgftext[x=0.265230in,y=1.856712in,,bottom,rotate=90.000000]{\color{textcolor}{\rmfamily\fontsize{10.000000}{12.000000}\selectfont\catcode`\^=\active\def^{\ifmmode\sp\else\^{}\fi}\catcode`\%=\active\def%{\%}Ouput Impedance $R_{out}$ in \unit{\ohm}}}%
\end{pgfscope}%
\begin{pgfscope}%
\pgfpathrectangle{\pgfqpoint{0.644859in}{0.524170in}}{\pgfqpoint{4.636243in}{2.665085in}}%
\pgfusepath{clip}%
\pgfsetrectcap%
\pgfsetroundjoin%
\pgfsetlinewidth{1.505625pt}%
\definecolor{currentstroke}{rgb}{0.003922,0.450980,0.698039}%
\pgfsetstrokecolor{currentstroke}%
\pgfsetstrokeopacity{0.700000}%
\pgfsetdash{}{0pt}%
\pgfpathmoveto{\pgfqpoint{5.070364in}{3.068115in}}%
\pgfpathlineto{\pgfqpoint{4.155760in}{3.067971in}}%
\pgfpathlineto{\pgfqpoint{4.151545in}{3.005604in}}%
\pgfpathlineto{\pgfqpoint{4.147330in}{2.966691in}}%
\pgfpathlineto{\pgfqpoint{4.138901in}{2.915985in}}%
\pgfpathlineto{\pgfqpoint{4.130471in}{2.881817in}}%
\pgfpathlineto{\pgfqpoint{4.122042in}{2.855955in}}%
\pgfpathlineto{\pgfqpoint{4.109397in}{2.826030in}}%
\pgfpathlineto{\pgfqpoint{4.096753in}{2.802565in}}%
\pgfpathlineto{\pgfqpoint{4.084109in}{2.783222in}}%
\pgfpathlineto{\pgfqpoint{4.067250in}{2.761746in}}%
\pgfpathlineto{\pgfqpoint{4.050391in}{2.743685in}}%
\pgfpathlineto{\pgfqpoint{4.029317in}{2.724474in}}%
\pgfpathlineto{\pgfqpoint{4.004028in}{2.704916in}}%
\pgfpathlineto{\pgfqpoint{3.978740in}{2.688090in}}%
\pgfpathlineto{\pgfqpoint{3.949236in}{2.670964in}}%
\pgfpathlineto{\pgfqpoint{3.915518in}{2.653850in}}%
\pgfpathlineto{\pgfqpoint{3.873370in}{2.635155in}}%
\pgfpathlineto{\pgfqpoint{3.827008in}{2.617130in}}%
\pgfpathlineto{\pgfqpoint{3.768001in}{2.596996in}}%
\pgfpathlineto{\pgfqpoint{3.704780in}{2.577965in}}%
\pgfpathlineto{\pgfqpoint{3.633129in}{2.558694in}}%
\pgfpathlineto{\pgfqpoint{3.553048in}{2.539280in}}%
\pgfpathlineto{\pgfqpoint{3.451894in}{2.517060in}}%
\pgfpathlineto{\pgfqpoint{3.321236in}{2.491042in}}%
\pgfpathlineto{\pgfqpoint{3.173719in}{2.464042in}}%
\pgfpathlineto{\pgfqpoint{2.967196in}{2.428736in}}%
\pgfpathlineto{\pgfqpoint{2.634229in}{2.374003in}}%
\pgfpathlineto{\pgfqpoint{2.356055in}{2.327024in}}%
\pgfpathlineto{\pgfqpoint{2.191679in}{2.297423in}}%
\pgfpathlineto{\pgfqpoint{2.027303in}{2.265496in}}%
\pgfpathlineto{\pgfqpoint{1.909289in}{2.240581in}}%
\pgfpathlineto{\pgfqpoint{1.787061in}{2.212357in}}%
\pgfpathlineto{\pgfqpoint{1.694336in}{2.188844in}}%
\pgfpathlineto{\pgfqpoint{1.610041in}{2.165439in}}%
\pgfpathlineto{\pgfqpoint{1.529960in}{2.140930in}}%
\pgfpathlineto{\pgfqpoint{1.462524in}{2.118136in}}%
\pgfpathlineto{\pgfqpoint{1.399302in}{2.094527in}}%
\pgfpathlineto{\pgfqpoint{1.340296in}{2.070019in}}%
\pgfpathlineto{\pgfqpoint{1.289719in}{2.046625in}}%
\pgfpathlineto{\pgfqpoint{1.243356in}{2.022740in}}%
\pgfpathlineto{\pgfqpoint{1.201208in}{1.998484in}}%
\pgfpathlineto{\pgfqpoint{1.163276in}{1.974048in}}%
\pgfpathlineto{\pgfqpoint{1.129557in}{1.949696in}}%
\pgfpathlineto{\pgfqpoint{1.100054in}{1.925823in}}%
\pgfpathlineto{\pgfqpoint{1.070551in}{1.898918in}}%
\pgfpathlineto{\pgfqpoint{1.045262in}{1.872756in}}%
\pgfpathlineto{\pgfqpoint{1.024188in}{1.848164in}}%
\pgfpathlineto{\pgfqpoint{1.003114in}{1.820300in}}%
\pgfpathlineto{\pgfqpoint{0.982041in}{1.788158in}}%
\pgfpathlineto{\pgfqpoint{0.965182in}{1.758329in}}%
\pgfpathlineto{\pgfqpoint{0.948322in}{1.723526in}}%
\pgfpathlineto{\pgfqpoint{0.935678in}{1.692978in}}%
\pgfpathlineto{\pgfqpoint{0.923034in}{1.657193in}}%
\pgfpathlineto{\pgfqpoint{0.910390in}{1.613965in}}%
\pgfpathlineto{\pgfqpoint{0.901960in}{1.579186in}}%
\pgfpathlineto{\pgfqpoint{0.893530in}{1.537421in}}%
\pgfpathlineto{\pgfqpoint{0.885101in}{1.485130in}}%
\pgfpathlineto{\pgfqpoint{0.876671in}{1.415170in}}%
\pgfpathlineto{\pgfqpoint{0.872457in}{1.368821in}}%
\pgfpathlineto{\pgfqpoint{0.868242in}{1.309160in}}%
\pgfpathlineto{\pgfqpoint{0.864027in}{1.225405in}}%
\pgfpathlineto{\pgfqpoint{0.859812in}{1.084579in}}%
\pgfpathlineto{\pgfqpoint{0.855598in}{0.717466in}}%
\pgfpathlineto{\pgfqpoint{0.855598in}{0.717466in}}%
\pgfusepath{stroke}%
\end{pgfscope}%
\begin{pgfscope}%
\pgfpathrectangle{\pgfqpoint{0.644859in}{0.524170in}}{\pgfqpoint{4.636243in}{2.665085in}}%
\pgfusepath{clip}%
\pgfsetrectcap%
\pgfsetroundjoin%
\pgfsetlinewidth{1.505625pt}%
\definecolor{currentstroke}{rgb}{0.870588,0.560784,0.019608}%
\pgfsetstrokecolor{currentstroke}%
\pgfsetstrokeopacity{0.700000}%
\pgfsetdash{}{0pt}%
\pgfpathmoveto{\pgfqpoint{5.070364in}{1.558405in}}%
\pgfpathlineto{\pgfqpoint{4.155760in}{1.558273in}}%
\pgfpathlineto{\pgfqpoint{4.151545in}{1.495907in}}%
\pgfpathlineto{\pgfqpoint{4.147330in}{1.456984in}}%
\pgfpathlineto{\pgfqpoint{4.138901in}{1.406289in}}%
\pgfpathlineto{\pgfqpoint{4.130471in}{1.372122in}}%
\pgfpathlineto{\pgfqpoint{4.122042in}{1.346267in}}%
\pgfpathlineto{\pgfqpoint{4.109397in}{1.316340in}}%
\pgfpathlineto{\pgfqpoint{4.096753in}{1.292880in}}%
\pgfpathlineto{\pgfqpoint{4.084109in}{1.273545in}}%
\pgfpathlineto{\pgfqpoint{4.067250in}{1.252069in}}%
\pgfpathlineto{\pgfqpoint{4.050391in}{1.234019in}}%
\pgfpathlineto{\pgfqpoint{4.029317in}{1.214819in}}%
\pgfpathlineto{\pgfqpoint{4.004028in}{1.195271in}}%
\pgfpathlineto{\pgfqpoint{3.978740in}{1.178453in}}%
\pgfpathlineto{\pgfqpoint{3.949236in}{1.161343in}}%
\pgfpathlineto{\pgfqpoint{3.915518in}{1.144240in}}%
\pgfpathlineto{\pgfqpoint{3.873370in}{1.125565in}}%
\pgfpathlineto{\pgfqpoint{3.827008in}{1.107569in}}%
\pgfpathlineto{\pgfqpoint{3.772216in}{1.088820in}}%
\pgfpathlineto{\pgfqpoint{3.708995in}{1.069671in}}%
\pgfpathlineto{\pgfqpoint{3.637344in}{1.050321in}}%
\pgfpathlineto{\pgfqpoint{3.553048in}{1.029888in}}%
\pgfpathlineto{\pgfqpoint{3.451894in}{1.007762in}}%
\pgfpathlineto{\pgfqpoint{3.333880in}{0.984272in}}%
\pgfpathlineto{\pgfqpoint{3.186364in}{0.957285in}}%
\pgfpathlineto{\pgfqpoint{2.996699in}{0.924994in}}%
\pgfpathlineto{\pgfqpoint{2.697451in}{0.876580in}}%
\pgfpathlineto{\pgfqpoint{2.288618in}{0.810148in}}%
\pgfpathlineto{\pgfqpoint{2.031517in}{0.766054in}}%
\pgfpathlineto{\pgfqpoint{1.601611in}{0.691613in}}%
\pgfpathlineto{\pgfqpoint{1.500457in}{0.676961in}}%
\pgfpathlineto{\pgfqpoint{1.411947in}{0.666368in}}%
\pgfpathlineto{\pgfqpoint{1.327651in}{0.658578in}}%
\pgfpathlineto{\pgfqpoint{1.243356in}{0.653042in}}%
\pgfpathlineto{\pgfqpoint{1.150631in}{0.649187in}}%
\pgfpathlineto{\pgfqpoint{1.032618in}{0.646652in}}%
\pgfpathlineto{\pgfqpoint{0.855598in}{0.645310in}}%
\pgfpathlineto{\pgfqpoint{0.855598in}{0.645310in}}%
\pgfusepath{stroke}%
\end{pgfscope}%
\begin{pgfscope}%
\pgfsetrectcap%
\pgfsetmiterjoin%
\pgfsetlinewidth{0.803000pt}%
\definecolor{currentstroke}{rgb}{0.000000,0.000000,0.000000}%
\pgfsetstrokecolor{currentstroke}%
\pgfsetdash{}{0pt}%
\pgfpathmoveto{\pgfqpoint{0.644859in}{0.524170in}}%
\pgfpathlineto{\pgfqpoint{0.644859in}{3.189255in}}%
\pgfusepath{stroke}%
\end{pgfscope}%
\begin{pgfscope}%
\pgfsetrectcap%
\pgfsetmiterjoin%
\pgfsetlinewidth{0.803000pt}%
\definecolor{currentstroke}{rgb}{0.000000,0.000000,0.000000}%
\pgfsetstrokecolor{currentstroke}%
\pgfsetdash{}{0pt}%
\pgfpathmoveto{\pgfqpoint{5.281103in}{0.524170in}}%
\pgfpathlineto{\pgfqpoint{5.281103in}{3.189255in}}%
\pgfusepath{stroke}%
\end{pgfscope}%
\begin{pgfscope}%
\pgfsetrectcap%
\pgfsetmiterjoin%
\pgfsetlinewidth{0.803000pt}%
\definecolor{currentstroke}{rgb}{0.000000,0.000000,0.000000}%
\pgfsetstrokecolor{currentstroke}%
\pgfsetdash{}{0pt}%
\pgfpathmoveto{\pgfqpoint{0.644859in}{0.524170in}}%
\pgfpathlineto{\pgfqpoint{5.281103in}{0.524170in}}%
\pgfusepath{stroke}%
\end{pgfscope}%
\begin{pgfscope}%
\pgfsetrectcap%
\pgfsetmiterjoin%
\pgfsetlinewidth{0.803000pt}%
\definecolor{currentstroke}{rgb}{0.000000,0.000000,0.000000}%
\pgfsetstrokecolor{currentstroke}%
\pgfsetdash{}{0pt}%
\pgfpathmoveto{\pgfqpoint{0.644859in}{3.189255in}}%
\pgfpathlineto{\pgfqpoint{5.281103in}{3.189255in}}%
\pgfusepath{stroke}%
\end{pgfscope}%
\begin{pgfscope}%
\pgfsetbuttcap%
\pgfsetmiterjoin%
\definecolor{currentfill}{rgb}{1.000000,1.000000,1.000000}%
\pgfsetfillcolor{currentfill}%
\pgfsetfillopacity{0.800000}%
\pgfsetlinewidth{1.003750pt}%
\definecolor{currentstroke}{rgb}{0.800000,0.800000,0.800000}%
\pgfsetstrokecolor{currentstroke}%
\pgfsetstrokeopacity{0.800000}%
\pgfsetdash{}{0pt}%
\pgfpathmoveto{\pgfqpoint{0.722637in}{2.790588in}}%
\pgfpathlineto{\pgfqpoint{1.404740in}{2.790588in}}%
\pgfpathquadraticcurveto{\pgfqpoint{1.426962in}{2.790588in}}{\pgfqpoint{1.426962in}{2.812811in}}%
\pgfpathlineto{\pgfqpoint{1.426962in}{3.111477in}}%
\pgfpathquadraticcurveto{\pgfqpoint{1.426962in}{3.133699in}}{\pgfqpoint{1.404740in}{3.133699in}}%
\pgfpathlineto{\pgfqpoint{0.722637in}{3.133699in}}%
\pgfpathquadraticcurveto{\pgfqpoint{0.700415in}{3.133699in}}{\pgfqpoint{0.700415in}{3.111477in}}%
\pgfpathlineto{\pgfqpoint{0.700415in}{2.812811in}}%
\pgfpathquadraticcurveto{\pgfqpoint{0.700415in}{2.790588in}}{\pgfqpoint{0.722637in}{2.790588in}}%
\pgfpathlineto{\pgfqpoint{0.722637in}{2.790588in}}%
\pgfpathclose%
\pgfusepath{stroke,fill}%
\end{pgfscope}%
\begin{pgfscope}%
\pgfsetrectcap%
\pgfsetroundjoin%
\pgfsetlinewidth{1.505625pt}%
\definecolor{currentstroke}{rgb}{0.003922,0.450980,0.698039}%
\pgfsetstrokecolor{currentstroke}%
\pgfsetstrokeopacity{0.700000}%
\pgfsetdash{}{0pt}%
\pgfpathmoveto{\pgfqpoint{0.744859in}{3.050366in}}%
\pgfpathlineto{\pgfqpoint{0.855970in}{3.050366in}}%
\pgfpathlineto{\pgfqpoint{0.967081in}{3.050366in}}%
\pgfusepath{stroke}%
\end{pgfscope}%
\begin{pgfscope}%
\definecolor{textcolor}{rgb}{0.000000,0.000000,0.000000}%
\pgfsetstrokecolor{textcolor}%
\pgfsetfillcolor{textcolor}%
\pgftext[x=1.055970in,y=3.011477in,left,base]{\color{textcolor}{\rmfamily\fontsize{8.000000}{9.600000}\selectfont\catcode`\^=\active\def^{\ifmmode\sp\else\^{}\fi}\catcode`\%=\active\def%{\%}DC}}%
\end{pgfscope}%
\begin{pgfscope}%
\pgfsetrectcap%
\pgfsetroundjoin%
\pgfsetlinewidth{1.505625pt}%
\definecolor{currentstroke}{rgb}{0.870588,0.560784,0.019608}%
\pgfsetstrokecolor{currentstroke}%
\pgfsetstrokeopacity{0.700000}%
\pgfsetdash{}{0pt}%
\pgfpathmoveto{\pgfqpoint{0.744859in}{2.895477in}}%
\pgfpathlineto{\pgfqpoint{0.855970in}{2.895477in}}%
\pgfpathlineto{\pgfqpoint{0.967081in}{2.895477in}}%
\pgfusepath{stroke}%
\end{pgfscope}%
\begin{pgfscope}%
\definecolor{textcolor}{rgb}{0.000000,0.000000,0.000000}%
\pgfsetstrokecolor{textcolor}%
\pgfsetfillcolor{textcolor}%
\pgftext[x=1.055970in,y=2.856588in,left,base]{\color{textcolor}{\rmfamily\fontsize{8.000000}{9.600000}\selectfont\catcode`\^=\active\def^{\ifmmode\sp\else\^{}\fi}\catcode`\%=\active\def%{\%}\qty{1}{\MHz}}}%
\end{pgfscope}%
\end{pgfpicture}%
\makeatother%
\endgroup%

    \caption{Simulated output impedance for the precision current source from figure \ref{fig:precision_current_source} at DC and \qty{1}{\MHz} over the drain-source voltage.}
    \label{fig:ltspice_output_impedance_simulation}
\end{figure}

Looking at figure \ref{fig:ltspice_output_impedance_simulation} clearly shows the effect of entering the ohmic region of the MOSFET. Over a range of about \qty{100}{\mV} below the \qty{0.78}{\V} calculated above, the output impedance drops by two orders of magnitude and then keeps dropping at an exponential rate with decreasing $V_{DS}$. The same effect applies to the output impedance at \qty{1}{\MHz}, although the starting value is around \qty{200}{\kilo\ohm} due to the reduced gain from the op-amp at \qty{1}{\MHz}. It can also be seen, that $R_{out}$ levels off at \qty{30}{\ohm}, the value of the sense resistor.

This overall effect of leaving the saturation region is so drastic, that the compliance voltage must be defined in such a way, that the MOSFET remains in saturation and this leads to
\begin{equation}
    V_{comp} = V_{sup} - V_{ref} - \sqrt{\frac{2 I_D}{\kappa}} \,.
\end{equation}

Now turning to the supply voltage, it is limited by the op-amp which must drive the gate of the MOSFET all the way up to the supply to turn off the current source. The reference voltage is, unless one divides it down, which is a delicate matter, dictated by the reference chosen. This, unfortunately, leaves only little room for the MOSFET and it must be carefully chosen not limit the compliance voltage too much.

At this point a fallacy, the author has observed multiple times must addressed. In order to address the limited compliance voltage, one may be tempted to use multiple MOSFETs in parallel to divide the current between the MOSFETs and thereby reduce the voltage that needs to be dropped across the FET proportional to $\frac{1}{\sqrt{N}}$, where $N$ is the number of MOSFETs paralleled.

Imagine the following modified circuit of the precision current source shown in figure \ref{fig:precision_current_source_two_mosfets} with two MOSFETs in parallel. For clarity the gate resistors required are not included.

\begin{figure}[ht]
    \centering
    \import{figures/}{precision_current_source_2fets.tex}
    \caption{Transconductance amplifier with two p-channel MOSFETs in parallel.}
    \label{fig:precision_current_source_two_mosfets}
\end{figure}

While at first, this seems like a solution to the limited $V_{DS}$, it is not that easy and a bad idea for number of reasons given here. The first reason is, MOSFET specifactions are very loose, notably the threshold voltage $V_{th}$, the transconductance $g_m$ and the capacitances, but the latter is of little concern here. Paralleling MOSFETs works well under certain conditions, when using the MOSFETs as a switch, not as a current source. It seems to be a common misunderstanding, that MOSFETs are imune to thermal runaway. This is true, when using them as a switch fully turned on and in the ohmic region. In this case, there a two effects occuring, the first is, that the (absolute) value of $V_{th}$ decreases with temperature, thus increasing $I_D$ and the second effect is, $R_{DS,on}$ is rising with temperature \cite{mosfet_thermal_runaway}. But here, the MOSFET is operating in pinch-off and not the ohmic region, $R_{DS,on}$ has no influence on the current, therefore, the only effect at work is the decreasing $V_{th}$, so depending on how bad the imbalance in $V_{th}$ of the paralled MOSFETs is, one MOSFET will gobble up most of the current and power. Adding source resistors, can compensate for this by pushing down the source voltage as the current goes up. This will then reduce $V_{GS}$. The size of the resistor depends on the transconductance $g_m$ and the temperature coefficent of $V_{GS}$, which is around \qtyrange[range-units = single]{1.5}{2}{\mV \per \K} \cite{mosfet_vgs_tempco}. Unfortunately, \qty{1}{\ohm} or \qty{2}{\ohm}, will already eat up, most of the benefits gained in compliance voltage as will be shown below. A detailed analyis of paralleling MOSFETs can be found in \cite{paralleling_mosfets}.

The second reaon, why paralleling MOSFETs is a bad idea can be found, when
remembering equation \ref{eqn:mosfet_id_large_signal}. We know that the transition from the undesirable ohmic region to the saturation region is
\begin{equation}
    V_{DS} \geq V_{GS} − V_{th}
\end{equation}

Looking at \ref{fig:precision_current_source_two_mosfets}, we see that $V_{GS}$ is set by the op-amp and is the same for both MOSFETs, but $V_{th}$ is device specific and according to the datasheet of our example \device{IRF9610} \cite{datasheet_IRF9610} $V_{th}$ values can show a spread of as much as \qtyrange[range-units = single]{-2}{-4}{\V}, although \cite{appnote_mosfet_parameter_spread} suggests, that MOSFETs from the same reel show a spread of only \qty{\pm 125}{\mV} of $V_{th}$  within the same batch for consecutive devices. The \qty{125}{\mV} was found for the \device{BUK7S1R5-40H} \cite{datasheet_BUK7S1R5}, which was sampled in this report. The number given in the report is for $3\sigma$ and assuming the datasheet values are also referring to $3\sigma$, the value found in the report is about twice as good as the datasheet value of \qtyrange[range-units = single]{2.4}{3.6}{\V}. Assuming similar numbers for \device{IRF9610} MOSFET used in our examples, this leads to \qty{\pm 208}{\mV} for the \device{IRF9610}, again applying $3\sigma$. Using this number, a Monte Carlo simulation (not quite, because the dice were biased to yield a Gaussian distribution) was run using LTSpice, simulating the circuit shown in figure \ref{fig:precision_current_source_two_mosfets} and also the original circuit using only one MOSFET. The current source was set to \qty{250}{\mA} as per table \ref{tab:current_source_parameters}. The load voltage was set to
\begin{equation}
    V_{DS, parallel} = \sqrt{\frac{2 \frac{I_d}{2}}{\kappa}} \approx \qty{555}{\mV}\,,
    V_{DS, single} = \sqrt{\frac{2 I_d}{\kappa}} \approx \qty{784}{\mV} \,. \nonumber
\end{equation}

$\frac{I_D}{2}$ was used for the parallel configuration to show the effect assuming perfect current sharing between the MOSFETs. Additionally, $V_{DS, parallel} + 1\sigma$ was also investigated. \num{4000} samples were drawn and the spread of the output impedance was calculated for each circuit. The results are shown as a histogram in figure \ref{fig:ltpsice_mosfet_mc_output_impedance}. The counts give the number of cases for each bin of the output imdedance.

\begin{figure}[ht]
    \centering
    %% Creator: Matplotlib, PGF backend
%%
%% To include the figure in your LaTeX document, write
%%   \input{<filename>.pgf}
%%
%% Make sure the required packages are loaded in your preamble
%%   \usepackage{pgf}
%%
%% Also ensure that all the required font packages are loaded; for instance,
%% the lmodern package is sometimes necessary when using math font.
%%   \usepackage{lmodern}
%%
%% Figures using additional raster images can only be included by \input if
%% they are in the same directory as the main LaTeX file. For loading figures
%% from other directories you can use the `import` package
%%   \usepackage{import}
%%
%% and then include the figures with
%%   \import{<path to file>}{<filename>.pgf}
%%
%% Matplotlib used the following preamble
%%   \usepackage{siunitx}
%%   \usepackage{fontspec}
%%
\begingroup%
\makeatletter%
\begin{pgfpicture}%
\pgfpathrectangle{\pgfpointorigin}{\pgfqpoint{5.492126in}{3.394321in}}%
\pgfusepath{use as bounding box, clip}%
\begin{pgfscope}%
\pgfsetbuttcap%
\pgfsetmiterjoin%
\definecolor{currentfill}{rgb}{1.000000,1.000000,1.000000}%
\pgfsetfillcolor{currentfill}%
\pgfsetlinewidth{0.000000pt}%
\definecolor{currentstroke}{rgb}{1.000000,1.000000,1.000000}%
\pgfsetstrokecolor{currentstroke}%
\pgfsetdash{}{0pt}%
\pgfpathmoveto{\pgfqpoint{0.000000in}{0.000000in}}%
\pgfpathlineto{\pgfqpoint{5.492126in}{0.000000in}}%
\pgfpathlineto{\pgfqpoint{5.492126in}{3.394321in}}%
\pgfpathlineto{\pgfqpoint{0.000000in}{3.394321in}}%
\pgfpathlineto{\pgfqpoint{0.000000in}{0.000000in}}%
\pgfpathclose%
\pgfusepath{fill}%
\end{pgfscope}%
\begin{pgfscope}%
\pgfsetbuttcap%
\pgfsetmiterjoin%
\definecolor{currentfill}{rgb}{1.000000,1.000000,1.000000}%
\pgfsetfillcolor{currentfill}%
\pgfsetlinewidth{0.000000pt}%
\definecolor{currentstroke}{rgb}{0.000000,0.000000,0.000000}%
\pgfsetstrokecolor{currentstroke}%
\pgfsetstrokeopacity{0.000000}%
\pgfsetdash{}{0pt}%
\pgfpathmoveto{\pgfqpoint{0.661006in}{0.524170in}}%
\pgfpathlineto{\pgfqpoint{5.342126in}{0.524170in}}%
\pgfpathlineto{\pgfqpoint{5.342126in}{3.244321in}}%
\pgfpathlineto{\pgfqpoint{0.661006in}{3.244321in}}%
\pgfpathlineto{\pgfqpoint{0.661006in}{0.524170in}}%
\pgfpathclose%
\pgfusepath{fill}%
\end{pgfscope}%
\begin{pgfscope}%
\pgfpathrectangle{\pgfqpoint{0.661006in}{0.524170in}}{\pgfqpoint{4.681120in}{2.720151in}}%
\pgfusepath{clip}%
\pgfsetbuttcap%
\pgfsetmiterjoin%
\definecolor{currentfill}{rgb}{0.870588,0.560784,0.019608}%
\pgfsetfillcolor{currentfill}%
\pgfsetfillopacity{0.700000}%
\pgfsetlinewidth{0.000000pt}%
\definecolor{currentstroke}{rgb}{0.000000,0.000000,0.000000}%
\pgfsetstrokecolor{currentstroke}%
\pgfsetstrokeopacity{0.700000}%
\pgfsetdash{}{0pt}%
\pgfpathmoveto{\pgfqpoint{0.873784in}{0.524170in}}%
\pgfpathlineto{\pgfqpoint{1.157488in}{0.524170in}}%
\pgfpathlineto{\pgfqpoint{1.157488in}{2.433457in}}%
\pgfpathlineto{\pgfqpoint{0.873784in}{2.433457in}}%
\pgfpathlineto{\pgfqpoint{0.873784in}{0.524170in}}%
\pgfpathclose%
\pgfusepath{fill}%
\end{pgfscope}%
\begin{pgfscope}%
\pgfpathrectangle{\pgfqpoint{0.661006in}{0.524170in}}{\pgfqpoint{4.681120in}{2.720151in}}%
\pgfusepath{clip}%
\pgfsetbuttcap%
\pgfsetmiterjoin%
\definecolor{currentfill}{rgb}{0.870588,0.560784,0.019608}%
\pgfsetfillcolor{currentfill}%
\pgfsetfillopacity{0.700000}%
\pgfsetlinewidth{0.000000pt}%
\definecolor{currentstroke}{rgb}{0.000000,0.000000,0.000000}%
\pgfsetstrokecolor{currentstroke}%
\pgfsetstrokeopacity{0.700000}%
\pgfsetdash{}{0pt}%
\pgfpathmoveto{\pgfqpoint{1.157488in}{0.524170in}}%
\pgfpathlineto{\pgfqpoint{1.441192in}{0.524170in}}%
\pgfpathlineto{\pgfqpoint{1.441192in}{0.822739in}}%
\pgfpathlineto{\pgfqpoint{1.157488in}{0.822739in}}%
\pgfpathlineto{\pgfqpoint{1.157488in}{0.524170in}}%
\pgfpathclose%
\pgfusepath{fill}%
\end{pgfscope}%
\begin{pgfscope}%
\pgfpathrectangle{\pgfqpoint{0.661006in}{0.524170in}}{\pgfqpoint{4.681120in}{2.720151in}}%
\pgfusepath{clip}%
\pgfsetbuttcap%
\pgfsetmiterjoin%
\definecolor{currentfill}{rgb}{0.870588,0.560784,0.019608}%
\pgfsetfillcolor{currentfill}%
\pgfsetfillopacity{0.700000}%
\pgfsetlinewidth{0.000000pt}%
\definecolor{currentstroke}{rgb}{0.000000,0.000000,0.000000}%
\pgfsetstrokecolor{currentstroke}%
\pgfsetstrokeopacity{0.700000}%
\pgfsetdash{}{0pt}%
\pgfpathmoveto{\pgfqpoint{1.441192in}{0.524170in}}%
\pgfpathlineto{\pgfqpoint{1.724897in}{0.524170in}}%
\pgfpathlineto{\pgfqpoint{1.724897in}{0.643986in}}%
\pgfpathlineto{\pgfqpoint{1.441192in}{0.643986in}}%
\pgfpathlineto{\pgfqpoint{1.441192in}{0.524170in}}%
\pgfpathclose%
\pgfusepath{fill}%
\end{pgfscope}%
\begin{pgfscope}%
\pgfpathrectangle{\pgfqpoint{0.661006in}{0.524170in}}{\pgfqpoint{4.681120in}{2.720151in}}%
\pgfusepath{clip}%
\pgfsetbuttcap%
\pgfsetmiterjoin%
\definecolor{currentfill}{rgb}{0.870588,0.560784,0.019608}%
\pgfsetfillcolor{currentfill}%
\pgfsetfillopacity{0.700000}%
\pgfsetlinewidth{0.000000pt}%
\definecolor{currentstroke}{rgb}{0.000000,0.000000,0.000000}%
\pgfsetstrokecolor{currentstroke}%
\pgfsetstrokeopacity{0.700000}%
\pgfsetdash{}{0pt}%
\pgfpathmoveto{\pgfqpoint{1.724897in}{0.524170in}}%
\pgfpathlineto{\pgfqpoint{2.008601in}{0.524170in}}%
\pgfpathlineto{\pgfqpoint{2.008601in}{0.595412in}}%
\pgfpathlineto{\pgfqpoint{1.724897in}{0.595412in}}%
\pgfpathlineto{\pgfqpoint{1.724897in}{0.524170in}}%
\pgfpathclose%
\pgfusepath{fill}%
\end{pgfscope}%
\begin{pgfscope}%
\pgfpathrectangle{\pgfqpoint{0.661006in}{0.524170in}}{\pgfqpoint{4.681120in}{2.720151in}}%
\pgfusepath{clip}%
\pgfsetbuttcap%
\pgfsetmiterjoin%
\definecolor{currentfill}{rgb}{0.870588,0.560784,0.019608}%
\pgfsetfillcolor{currentfill}%
\pgfsetfillopacity{0.700000}%
\pgfsetlinewidth{0.000000pt}%
\definecolor{currentstroke}{rgb}{0.000000,0.000000,0.000000}%
\pgfsetstrokecolor{currentstroke}%
\pgfsetstrokeopacity{0.700000}%
\pgfsetdash{}{0pt}%
\pgfpathmoveto{\pgfqpoint{2.008601in}{0.524170in}}%
\pgfpathlineto{\pgfqpoint{2.292305in}{0.524170in}}%
\pgfpathlineto{\pgfqpoint{2.292305in}{0.567563in}}%
\pgfpathlineto{\pgfqpoint{2.008601in}{0.567563in}}%
\pgfpathlineto{\pgfqpoint{2.008601in}{0.524170in}}%
\pgfpathclose%
\pgfusepath{fill}%
\end{pgfscope}%
\begin{pgfscope}%
\pgfpathrectangle{\pgfqpoint{0.661006in}{0.524170in}}{\pgfqpoint{4.681120in}{2.720151in}}%
\pgfusepath{clip}%
\pgfsetbuttcap%
\pgfsetmiterjoin%
\definecolor{currentfill}{rgb}{0.870588,0.560784,0.019608}%
\pgfsetfillcolor{currentfill}%
\pgfsetfillopacity{0.700000}%
\pgfsetlinewidth{0.000000pt}%
\definecolor{currentstroke}{rgb}{0.000000,0.000000,0.000000}%
\pgfsetstrokecolor{currentstroke}%
\pgfsetstrokeopacity{0.700000}%
\pgfsetdash{}{0pt}%
\pgfpathmoveto{\pgfqpoint{2.292305in}{0.524170in}}%
\pgfpathlineto{\pgfqpoint{2.576010in}{0.524170in}}%
\pgfpathlineto{\pgfqpoint{2.576010in}{0.548781in}}%
\pgfpathlineto{\pgfqpoint{2.292305in}{0.548781in}}%
\pgfpathlineto{\pgfqpoint{2.292305in}{0.524170in}}%
\pgfpathclose%
\pgfusepath{fill}%
\end{pgfscope}%
\begin{pgfscope}%
\pgfpathrectangle{\pgfqpoint{0.661006in}{0.524170in}}{\pgfqpoint{4.681120in}{2.720151in}}%
\pgfusepath{clip}%
\pgfsetbuttcap%
\pgfsetmiterjoin%
\definecolor{currentfill}{rgb}{0.870588,0.560784,0.019608}%
\pgfsetfillcolor{currentfill}%
\pgfsetfillopacity{0.700000}%
\pgfsetlinewidth{0.000000pt}%
\definecolor{currentstroke}{rgb}{0.000000,0.000000,0.000000}%
\pgfsetstrokecolor{currentstroke}%
\pgfsetstrokeopacity{0.700000}%
\pgfsetdash{}{0pt}%
\pgfpathmoveto{\pgfqpoint{2.576010in}{0.524170in}}%
\pgfpathlineto{\pgfqpoint{2.859714in}{0.524170in}}%
\pgfpathlineto{\pgfqpoint{2.859714in}{0.546190in}}%
\pgfpathlineto{\pgfqpoint{2.576010in}{0.546190in}}%
\pgfpathlineto{\pgfqpoint{2.576010in}{0.524170in}}%
\pgfpathclose%
\pgfusepath{fill}%
\end{pgfscope}%
\begin{pgfscope}%
\pgfpathrectangle{\pgfqpoint{0.661006in}{0.524170in}}{\pgfqpoint{4.681120in}{2.720151in}}%
\pgfusepath{clip}%
\pgfsetbuttcap%
\pgfsetmiterjoin%
\definecolor{currentfill}{rgb}{0.870588,0.560784,0.019608}%
\pgfsetfillcolor{currentfill}%
\pgfsetfillopacity{0.700000}%
\pgfsetlinewidth{0.000000pt}%
\definecolor{currentstroke}{rgb}{0.000000,0.000000,0.000000}%
\pgfsetstrokecolor{currentstroke}%
\pgfsetstrokeopacity{0.700000}%
\pgfsetdash{}{0pt}%
\pgfpathmoveto{\pgfqpoint{2.859714in}{0.524170in}}%
\pgfpathlineto{\pgfqpoint{3.143418in}{0.524170in}}%
\pgfpathlineto{\pgfqpoint{3.143418in}{0.537123in}}%
\pgfpathlineto{\pgfqpoint{2.859714in}{0.537123in}}%
\pgfpathlineto{\pgfqpoint{2.859714in}{0.524170in}}%
\pgfpathclose%
\pgfusepath{fill}%
\end{pgfscope}%
\begin{pgfscope}%
\pgfpathrectangle{\pgfqpoint{0.661006in}{0.524170in}}{\pgfqpoint{4.681120in}{2.720151in}}%
\pgfusepath{clip}%
\pgfsetbuttcap%
\pgfsetmiterjoin%
\definecolor{currentfill}{rgb}{0.870588,0.560784,0.019608}%
\pgfsetfillcolor{currentfill}%
\pgfsetfillopacity{0.700000}%
\pgfsetlinewidth{0.000000pt}%
\definecolor{currentstroke}{rgb}{0.000000,0.000000,0.000000}%
\pgfsetstrokecolor{currentstroke}%
\pgfsetstrokeopacity{0.700000}%
\pgfsetdash{}{0pt}%
\pgfpathmoveto{\pgfqpoint{3.143418in}{0.524170in}}%
\pgfpathlineto{\pgfqpoint{3.427122in}{0.524170in}}%
\pgfpathlineto{\pgfqpoint{3.427122in}{0.535828in}}%
\pgfpathlineto{\pgfqpoint{3.143418in}{0.535828in}}%
\pgfpathlineto{\pgfqpoint{3.143418in}{0.524170in}}%
\pgfpathclose%
\pgfusepath{fill}%
\end{pgfscope}%
\begin{pgfscope}%
\pgfpathrectangle{\pgfqpoint{0.661006in}{0.524170in}}{\pgfqpoint{4.681120in}{2.720151in}}%
\pgfusepath{clip}%
\pgfsetbuttcap%
\pgfsetmiterjoin%
\definecolor{currentfill}{rgb}{0.870588,0.560784,0.019608}%
\pgfsetfillcolor{currentfill}%
\pgfsetfillopacity{0.700000}%
\pgfsetlinewidth{0.000000pt}%
\definecolor{currentstroke}{rgb}{0.000000,0.000000,0.000000}%
\pgfsetstrokecolor{currentstroke}%
\pgfsetstrokeopacity{0.700000}%
\pgfsetdash{}{0pt}%
\pgfpathmoveto{\pgfqpoint{3.427122in}{0.524170in}}%
\pgfpathlineto{\pgfqpoint{3.710827in}{0.524170in}}%
\pgfpathlineto{\pgfqpoint{3.710827in}{0.534532in}}%
\pgfpathlineto{\pgfqpoint{3.427122in}{0.534532in}}%
\pgfpathlineto{\pgfqpoint{3.427122in}{0.524170in}}%
\pgfpathclose%
\pgfusepath{fill}%
\end{pgfscope}%
\begin{pgfscope}%
\pgfpathrectangle{\pgfqpoint{0.661006in}{0.524170in}}{\pgfqpoint{4.681120in}{2.720151in}}%
\pgfusepath{clip}%
\pgfsetbuttcap%
\pgfsetmiterjoin%
\definecolor{currentfill}{rgb}{0.870588,0.560784,0.019608}%
\pgfsetfillcolor{currentfill}%
\pgfsetfillopacity{0.700000}%
\pgfsetlinewidth{0.000000pt}%
\definecolor{currentstroke}{rgb}{0.000000,0.000000,0.000000}%
\pgfsetstrokecolor{currentstroke}%
\pgfsetstrokeopacity{0.700000}%
\pgfsetdash{}{0pt}%
\pgfpathmoveto{\pgfqpoint{3.710827in}{0.524170in}}%
\pgfpathlineto{\pgfqpoint{3.994531in}{0.524170in}}%
\pgfpathlineto{\pgfqpoint{3.994531in}{0.527408in}}%
\pgfpathlineto{\pgfqpoint{3.710827in}{0.527408in}}%
\pgfpathlineto{\pgfqpoint{3.710827in}{0.524170in}}%
\pgfpathclose%
\pgfusepath{fill}%
\end{pgfscope}%
\begin{pgfscope}%
\pgfpathrectangle{\pgfqpoint{0.661006in}{0.524170in}}{\pgfqpoint{4.681120in}{2.720151in}}%
\pgfusepath{clip}%
\pgfsetbuttcap%
\pgfsetmiterjoin%
\definecolor{currentfill}{rgb}{0.870588,0.560784,0.019608}%
\pgfsetfillcolor{currentfill}%
\pgfsetfillopacity{0.700000}%
\pgfsetlinewidth{0.000000pt}%
\definecolor{currentstroke}{rgb}{0.000000,0.000000,0.000000}%
\pgfsetstrokecolor{currentstroke}%
\pgfsetstrokeopacity{0.700000}%
\pgfsetdash{}{0pt}%
\pgfpathmoveto{\pgfqpoint{3.994531in}{0.524170in}}%
\pgfpathlineto{\pgfqpoint{4.278235in}{0.524170in}}%
\pgfpathlineto{\pgfqpoint{4.278235in}{0.529999in}}%
\pgfpathlineto{\pgfqpoint{3.994531in}{0.529999in}}%
\pgfpathlineto{\pgfqpoint{3.994531in}{0.524170in}}%
\pgfpathclose%
\pgfusepath{fill}%
\end{pgfscope}%
\begin{pgfscope}%
\pgfpathrectangle{\pgfqpoint{0.661006in}{0.524170in}}{\pgfqpoint{4.681120in}{2.720151in}}%
\pgfusepath{clip}%
\pgfsetbuttcap%
\pgfsetmiterjoin%
\definecolor{currentfill}{rgb}{0.870588,0.560784,0.019608}%
\pgfsetfillcolor{currentfill}%
\pgfsetfillopacity{0.700000}%
\pgfsetlinewidth{0.000000pt}%
\definecolor{currentstroke}{rgb}{0.000000,0.000000,0.000000}%
\pgfsetstrokecolor{currentstroke}%
\pgfsetstrokeopacity{0.700000}%
\pgfsetdash{}{0pt}%
\pgfpathmoveto{\pgfqpoint{4.278235in}{0.524170in}}%
\pgfpathlineto{\pgfqpoint{4.561939in}{0.524170in}}%
\pgfpathlineto{\pgfqpoint{4.561939in}{0.529351in}}%
\pgfpathlineto{\pgfqpoint{4.278235in}{0.529351in}}%
\pgfpathlineto{\pgfqpoint{4.278235in}{0.524170in}}%
\pgfpathclose%
\pgfusepath{fill}%
\end{pgfscope}%
\begin{pgfscope}%
\pgfpathrectangle{\pgfqpoint{0.661006in}{0.524170in}}{\pgfqpoint{4.681120in}{2.720151in}}%
\pgfusepath{clip}%
\pgfsetbuttcap%
\pgfsetmiterjoin%
\definecolor{currentfill}{rgb}{0.870588,0.560784,0.019608}%
\pgfsetfillcolor{currentfill}%
\pgfsetfillopacity{0.700000}%
\pgfsetlinewidth{0.000000pt}%
\definecolor{currentstroke}{rgb}{0.000000,0.000000,0.000000}%
\pgfsetstrokecolor{currentstroke}%
\pgfsetstrokeopacity{0.700000}%
\pgfsetdash{}{0pt}%
\pgfpathmoveto{\pgfqpoint{4.561939in}{0.524170in}}%
\pgfpathlineto{\pgfqpoint{4.845644in}{0.524170in}}%
\pgfpathlineto{\pgfqpoint{4.845644in}{0.526113in}}%
\pgfpathlineto{\pgfqpoint{4.561939in}{0.526113in}}%
\pgfpathlineto{\pgfqpoint{4.561939in}{0.524170in}}%
\pgfpathclose%
\pgfusepath{fill}%
\end{pgfscope}%
\begin{pgfscope}%
\pgfpathrectangle{\pgfqpoint{0.661006in}{0.524170in}}{\pgfqpoint{4.681120in}{2.720151in}}%
\pgfusepath{clip}%
\pgfsetbuttcap%
\pgfsetmiterjoin%
\definecolor{currentfill}{rgb}{0.870588,0.560784,0.019608}%
\pgfsetfillcolor{currentfill}%
\pgfsetfillopacity{0.700000}%
\pgfsetlinewidth{0.000000pt}%
\definecolor{currentstroke}{rgb}{0.000000,0.000000,0.000000}%
\pgfsetstrokecolor{currentstroke}%
\pgfsetstrokeopacity{0.700000}%
\pgfsetdash{}{0pt}%
\pgfpathmoveto{\pgfqpoint{4.845644in}{0.524170in}}%
\pgfpathlineto{\pgfqpoint{5.129348in}{0.524170in}}%
\pgfpathlineto{\pgfqpoint{5.129348in}{0.574687in}}%
\pgfpathlineto{\pgfqpoint{4.845644in}{0.574687in}}%
\pgfpathlineto{\pgfqpoint{4.845644in}{0.524170in}}%
\pgfpathclose%
\pgfusepath{fill}%
\end{pgfscope}%
\begin{pgfscope}%
\pgfpathrectangle{\pgfqpoint{0.661006in}{0.524170in}}{\pgfqpoint{4.681120in}{2.720151in}}%
\pgfusepath{clip}%
\pgfsetbuttcap%
\pgfsetmiterjoin%
\definecolor{currentfill}{rgb}{0.007843,0.619608,0.450980}%
\pgfsetfillcolor{currentfill}%
\pgfsetfillopacity{0.700000}%
\pgfsetlinewidth{0.000000pt}%
\definecolor{currentstroke}{rgb}{0.000000,0.000000,0.000000}%
\pgfsetstrokecolor{currentstroke}%
\pgfsetstrokeopacity{0.700000}%
\pgfsetdash{}{0pt}%
\pgfpathmoveto{\pgfqpoint{0.873784in}{0.524170in}}%
\pgfpathlineto{\pgfqpoint{1.157488in}{0.524170in}}%
\pgfpathlineto{\pgfqpoint{1.157488in}{0.524170in}}%
\pgfpathlineto{\pgfqpoint{0.873784in}{0.524170in}}%
\pgfpathlineto{\pgfqpoint{0.873784in}{0.524170in}}%
\pgfpathclose%
\pgfusepath{fill}%
\end{pgfscope}%
\begin{pgfscope}%
\pgfpathrectangle{\pgfqpoint{0.661006in}{0.524170in}}{\pgfqpoint{4.681120in}{2.720151in}}%
\pgfusepath{clip}%
\pgfsetbuttcap%
\pgfsetmiterjoin%
\definecolor{currentfill}{rgb}{0.007843,0.619608,0.450980}%
\pgfsetfillcolor{currentfill}%
\pgfsetfillopacity{0.700000}%
\pgfsetlinewidth{0.000000pt}%
\definecolor{currentstroke}{rgb}{0.000000,0.000000,0.000000}%
\pgfsetstrokecolor{currentstroke}%
\pgfsetstrokeopacity{0.700000}%
\pgfsetdash{}{0pt}%
\pgfpathmoveto{\pgfqpoint{1.157488in}{0.524170in}}%
\pgfpathlineto{\pgfqpoint{1.441192in}{0.524170in}}%
\pgfpathlineto{\pgfqpoint{1.441192in}{0.524170in}}%
\pgfpathlineto{\pgfqpoint{1.157488in}{0.524170in}}%
\pgfpathlineto{\pgfqpoint{1.157488in}{0.524170in}}%
\pgfpathclose%
\pgfusepath{fill}%
\end{pgfscope}%
\begin{pgfscope}%
\pgfpathrectangle{\pgfqpoint{0.661006in}{0.524170in}}{\pgfqpoint{4.681120in}{2.720151in}}%
\pgfusepath{clip}%
\pgfsetbuttcap%
\pgfsetmiterjoin%
\definecolor{currentfill}{rgb}{0.007843,0.619608,0.450980}%
\pgfsetfillcolor{currentfill}%
\pgfsetfillopacity{0.700000}%
\pgfsetlinewidth{0.000000pt}%
\definecolor{currentstroke}{rgb}{0.000000,0.000000,0.000000}%
\pgfsetstrokecolor{currentstroke}%
\pgfsetstrokeopacity{0.700000}%
\pgfsetdash{}{0pt}%
\pgfpathmoveto{\pgfqpoint{1.441192in}{0.524170in}}%
\pgfpathlineto{\pgfqpoint{1.724897in}{0.524170in}}%
\pgfpathlineto{\pgfqpoint{1.724897in}{0.524170in}}%
\pgfpathlineto{\pgfqpoint{1.441192in}{0.524170in}}%
\pgfpathlineto{\pgfqpoint{1.441192in}{0.524170in}}%
\pgfpathclose%
\pgfusepath{fill}%
\end{pgfscope}%
\begin{pgfscope}%
\pgfpathrectangle{\pgfqpoint{0.661006in}{0.524170in}}{\pgfqpoint{4.681120in}{2.720151in}}%
\pgfusepath{clip}%
\pgfsetbuttcap%
\pgfsetmiterjoin%
\definecolor{currentfill}{rgb}{0.007843,0.619608,0.450980}%
\pgfsetfillcolor{currentfill}%
\pgfsetfillopacity{0.700000}%
\pgfsetlinewidth{0.000000pt}%
\definecolor{currentstroke}{rgb}{0.000000,0.000000,0.000000}%
\pgfsetstrokecolor{currentstroke}%
\pgfsetstrokeopacity{0.700000}%
\pgfsetdash{}{0pt}%
\pgfpathmoveto{\pgfqpoint{1.724897in}{0.524170in}}%
\pgfpathlineto{\pgfqpoint{2.008601in}{0.524170in}}%
\pgfpathlineto{\pgfqpoint{2.008601in}{0.524170in}}%
\pgfpathlineto{\pgfqpoint{1.724897in}{0.524170in}}%
\pgfpathlineto{\pgfqpoint{1.724897in}{0.524170in}}%
\pgfpathclose%
\pgfusepath{fill}%
\end{pgfscope}%
\begin{pgfscope}%
\pgfpathrectangle{\pgfqpoint{0.661006in}{0.524170in}}{\pgfqpoint{4.681120in}{2.720151in}}%
\pgfusepath{clip}%
\pgfsetbuttcap%
\pgfsetmiterjoin%
\definecolor{currentfill}{rgb}{0.007843,0.619608,0.450980}%
\pgfsetfillcolor{currentfill}%
\pgfsetfillopacity{0.700000}%
\pgfsetlinewidth{0.000000pt}%
\definecolor{currentstroke}{rgb}{0.000000,0.000000,0.000000}%
\pgfsetstrokecolor{currentstroke}%
\pgfsetstrokeopacity{0.700000}%
\pgfsetdash{}{0pt}%
\pgfpathmoveto{\pgfqpoint{2.008601in}{0.524170in}}%
\pgfpathlineto{\pgfqpoint{2.292305in}{0.524170in}}%
\pgfpathlineto{\pgfqpoint{2.292305in}{0.524170in}}%
\pgfpathlineto{\pgfqpoint{2.008601in}{0.524170in}}%
\pgfpathlineto{\pgfqpoint{2.008601in}{0.524170in}}%
\pgfpathclose%
\pgfusepath{fill}%
\end{pgfscope}%
\begin{pgfscope}%
\pgfpathrectangle{\pgfqpoint{0.661006in}{0.524170in}}{\pgfqpoint{4.681120in}{2.720151in}}%
\pgfusepath{clip}%
\pgfsetbuttcap%
\pgfsetmiterjoin%
\definecolor{currentfill}{rgb}{0.007843,0.619608,0.450980}%
\pgfsetfillcolor{currentfill}%
\pgfsetfillopacity{0.700000}%
\pgfsetlinewidth{0.000000pt}%
\definecolor{currentstroke}{rgb}{0.000000,0.000000,0.000000}%
\pgfsetstrokecolor{currentstroke}%
\pgfsetstrokeopacity{0.700000}%
\pgfsetdash{}{0pt}%
\pgfpathmoveto{\pgfqpoint{2.292305in}{0.524170in}}%
\pgfpathlineto{\pgfqpoint{2.576010in}{0.524170in}}%
\pgfpathlineto{\pgfqpoint{2.576010in}{0.524170in}}%
\pgfpathlineto{\pgfqpoint{2.292305in}{0.524170in}}%
\pgfpathlineto{\pgfqpoint{2.292305in}{0.524170in}}%
\pgfpathclose%
\pgfusepath{fill}%
\end{pgfscope}%
\begin{pgfscope}%
\pgfpathrectangle{\pgfqpoint{0.661006in}{0.524170in}}{\pgfqpoint{4.681120in}{2.720151in}}%
\pgfusepath{clip}%
\pgfsetbuttcap%
\pgfsetmiterjoin%
\definecolor{currentfill}{rgb}{0.007843,0.619608,0.450980}%
\pgfsetfillcolor{currentfill}%
\pgfsetfillopacity{0.700000}%
\pgfsetlinewidth{0.000000pt}%
\definecolor{currentstroke}{rgb}{0.000000,0.000000,0.000000}%
\pgfsetstrokecolor{currentstroke}%
\pgfsetstrokeopacity{0.700000}%
\pgfsetdash{}{0pt}%
\pgfpathmoveto{\pgfqpoint{2.576010in}{0.524170in}}%
\pgfpathlineto{\pgfqpoint{2.859714in}{0.524170in}}%
\pgfpathlineto{\pgfqpoint{2.859714in}{0.524170in}}%
\pgfpathlineto{\pgfqpoint{2.576010in}{0.524170in}}%
\pgfpathlineto{\pgfqpoint{2.576010in}{0.524170in}}%
\pgfpathclose%
\pgfusepath{fill}%
\end{pgfscope}%
\begin{pgfscope}%
\pgfpathrectangle{\pgfqpoint{0.661006in}{0.524170in}}{\pgfqpoint{4.681120in}{2.720151in}}%
\pgfusepath{clip}%
\pgfsetbuttcap%
\pgfsetmiterjoin%
\definecolor{currentfill}{rgb}{0.007843,0.619608,0.450980}%
\pgfsetfillcolor{currentfill}%
\pgfsetfillopacity{0.700000}%
\pgfsetlinewidth{0.000000pt}%
\definecolor{currentstroke}{rgb}{0.000000,0.000000,0.000000}%
\pgfsetstrokecolor{currentstroke}%
\pgfsetstrokeopacity{0.700000}%
\pgfsetdash{}{0pt}%
\pgfpathmoveto{\pgfqpoint{2.859714in}{0.524170in}}%
\pgfpathlineto{\pgfqpoint{3.143418in}{0.524170in}}%
\pgfpathlineto{\pgfqpoint{3.143418in}{0.524170in}}%
\pgfpathlineto{\pgfqpoint{2.859714in}{0.524170in}}%
\pgfpathlineto{\pgfqpoint{2.859714in}{0.524170in}}%
\pgfpathclose%
\pgfusepath{fill}%
\end{pgfscope}%
\begin{pgfscope}%
\pgfpathrectangle{\pgfqpoint{0.661006in}{0.524170in}}{\pgfqpoint{4.681120in}{2.720151in}}%
\pgfusepath{clip}%
\pgfsetbuttcap%
\pgfsetmiterjoin%
\definecolor{currentfill}{rgb}{0.007843,0.619608,0.450980}%
\pgfsetfillcolor{currentfill}%
\pgfsetfillopacity{0.700000}%
\pgfsetlinewidth{0.000000pt}%
\definecolor{currentstroke}{rgb}{0.000000,0.000000,0.000000}%
\pgfsetstrokecolor{currentstroke}%
\pgfsetstrokeopacity{0.700000}%
\pgfsetdash{}{0pt}%
\pgfpathmoveto{\pgfqpoint{3.143418in}{0.524170in}}%
\pgfpathlineto{\pgfqpoint{3.427122in}{0.524170in}}%
\pgfpathlineto{\pgfqpoint{3.427122in}{0.524170in}}%
\pgfpathlineto{\pgfqpoint{3.143418in}{0.524170in}}%
\pgfpathlineto{\pgfqpoint{3.143418in}{0.524170in}}%
\pgfpathclose%
\pgfusepath{fill}%
\end{pgfscope}%
\begin{pgfscope}%
\pgfpathrectangle{\pgfqpoint{0.661006in}{0.524170in}}{\pgfqpoint{4.681120in}{2.720151in}}%
\pgfusepath{clip}%
\pgfsetbuttcap%
\pgfsetmiterjoin%
\definecolor{currentfill}{rgb}{0.007843,0.619608,0.450980}%
\pgfsetfillcolor{currentfill}%
\pgfsetfillopacity{0.700000}%
\pgfsetlinewidth{0.000000pt}%
\definecolor{currentstroke}{rgb}{0.000000,0.000000,0.000000}%
\pgfsetstrokecolor{currentstroke}%
\pgfsetstrokeopacity{0.700000}%
\pgfsetdash{}{0pt}%
\pgfpathmoveto{\pgfqpoint{3.427122in}{0.524170in}}%
\pgfpathlineto{\pgfqpoint{3.710827in}{0.524170in}}%
\pgfpathlineto{\pgfqpoint{3.710827in}{0.524170in}}%
\pgfpathlineto{\pgfqpoint{3.427122in}{0.524170in}}%
\pgfpathlineto{\pgfqpoint{3.427122in}{0.524170in}}%
\pgfpathclose%
\pgfusepath{fill}%
\end{pgfscope}%
\begin{pgfscope}%
\pgfpathrectangle{\pgfqpoint{0.661006in}{0.524170in}}{\pgfqpoint{4.681120in}{2.720151in}}%
\pgfusepath{clip}%
\pgfsetbuttcap%
\pgfsetmiterjoin%
\definecolor{currentfill}{rgb}{0.007843,0.619608,0.450980}%
\pgfsetfillcolor{currentfill}%
\pgfsetfillopacity{0.700000}%
\pgfsetlinewidth{0.000000pt}%
\definecolor{currentstroke}{rgb}{0.000000,0.000000,0.000000}%
\pgfsetstrokecolor{currentstroke}%
\pgfsetstrokeopacity{0.700000}%
\pgfsetdash{}{0pt}%
\pgfpathmoveto{\pgfqpoint{3.710827in}{0.524170in}}%
\pgfpathlineto{\pgfqpoint{3.994531in}{0.524170in}}%
\pgfpathlineto{\pgfqpoint{3.994531in}{3.114790in}}%
\pgfpathlineto{\pgfqpoint{3.710827in}{3.114790in}}%
\pgfpathlineto{\pgfqpoint{3.710827in}{0.524170in}}%
\pgfpathclose%
\pgfusepath{fill}%
\end{pgfscope}%
\begin{pgfscope}%
\pgfpathrectangle{\pgfqpoint{0.661006in}{0.524170in}}{\pgfqpoint{4.681120in}{2.720151in}}%
\pgfusepath{clip}%
\pgfsetbuttcap%
\pgfsetmiterjoin%
\definecolor{currentfill}{rgb}{0.007843,0.619608,0.450980}%
\pgfsetfillcolor{currentfill}%
\pgfsetfillopacity{0.700000}%
\pgfsetlinewidth{0.000000pt}%
\definecolor{currentstroke}{rgb}{0.000000,0.000000,0.000000}%
\pgfsetstrokecolor{currentstroke}%
\pgfsetstrokeopacity{0.700000}%
\pgfsetdash{}{0pt}%
\pgfpathmoveto{\pgfqpoint{3.994531in}{0.524170in}}%
\pgfpathlineto{\pgfqpoint{4.278235in}{0.524170in}}%
\pgfpathlineto{\pgfqpoint{4.278235in}{0.524170in}}%
\pgfpathlineto{\pgfqpoint{3.994531in}{0.524170in}}%
\pgfpathlineto{\pgfqpoint{3.994531in}{0.524170in}}%
\pgfpathclose%
\pgfusepath{fill}%
\end{pgfscope}%
\begin{pgfscope}%
\pgfpathrectangle{\pgfqpoint{0.661006in}{0.524170in}}{\pgfqpoint{4.681120in}{2.720151in}}%
\pgfusepath{clip}%
\pgfsetbuttcap%
\pgfsetmiterjoin%
\definecolor{currentfill}{rgb}{0.007843,0.619608,0.450980}%
\pgfsetfillcolor{currentfill}%
\pgfsetfillopacity{0.700000}%
\pgfsetlinewidth{0.000000pt}%
\definecolor{currentstroke}{rgb}{0.000000,0.000000,0.000000}%
\pgfsetstrokecolor{currentstroke}%
\pgfsetstrokeopacity{0.700000}%
\pgfsetdash{}{0pt}%
\pgfpathmoveto{\pgfqpoint{4.278235in}{0.524170in}}%
\pgfpathlineto{\pgfqpoint{4.561939in}{0.524170in}}%
\pgfpathlineto{\pgfqpoint{4.561939in}{0.524170in}}%
\pgfpathlineto{\pgfqpoint{4.278235in}{0.524170in}}%
\pgfpathlineto{\pgfqpoint{4.278235in}{0.524170in}}%
\pgfpathclose%
\pgfusepath{fill}%
\end{pgfscope}%
\begin{pgfscope}%
\pgfpathrectangle{\pgfqpoint{0.661006in}{0.524170in}}{\pgfqpoint{4.681120in}{2.720151in}}%
\pgfusepath{clip}%
\pgfsetbuttcap%
\pgfsetmiterjoin%
\definecolor{currentfill}{rgb}{0.007843,0.619608,0.450980}%
\pgfsetfillcolor{currentfill}%
\pgfsetfillopacity{0.700000}%
\pgfsetlinewidth{0.000000pt}%
\definecolor{currentstroke}{rgb}{0.000000,0.000000,0.000000}%
\pgfsetstrokecolor{currentstroke}%
\pgfsetstrokeopacity{0.700000}%
\pgfsetdash{}{0pt}%
\pgfpathmoveto{\pgfqpoint{4.561939in}{0.524170in}}%
\pgfpathlineto{\pgfqpoint{4.845644in}{0.524170in}}%
\pgfpathlineto{\pgfqpoint{4.845644in}{0.524170in}}%
\pgfpathlineto{\pgfqpoint{4.561939in}{0.524170in}}%
\pgfpathlineto{\pgfqpoint{4.561939in}{0.524170in}}%
\pgfpathclose%
\pgfusepath{fill}%
\end{pgfscope}%
\begin{pgfscope}%
\pgfpathrectangle{\pgfqpoint{0.661006in}{0.524170in}}{\pgfqpoint{4.681120in}{2.720151in}}%
\pgfusepath{clip}%
\pgfsetbuttcap%
\pgfsetmiterjoin%
\definecolor{currentfill}{rgb}{0.007843,0.619608,0.450980}%
\pgfsetfillcolor{currentfill}%
\pgfsetfillopacity{0.700000}%
\pgfsetlinewidth{0.000000pt}%
\definecolor{currentstroke}{rgb}{0.000000,0.000000,0.000000}%
\pgfsetstrokecolor{currentstroke}%
\pgfsetstrokeopacity{0.700000}%
\pgfsetdash{}{0pt}%
\pgfpathmoveto{\pgfqpoint{4.845644in}{0.524170in}}%
\pgfpathlineto{\pgfqpoint{5.129348in}{0.524170in}}%
\pgfpathlineto{\pgfqpoint{5.129348in}{0.524170in}}%
\pgfpathlineto{\pgfqpoint{4.845644in}{0.524170in}}%
\pgfpathlineto{\pgfqpoint{4.845644in}{0.524170in}}%
\pgfpathclose%
\pgfusepath{fill}%
\end{pgfscope}%
\begin{pgfscope}%
\pgfpathrectangle{\pgfqpoint{0.661006in}{0.524170in}}{\pgfqpoint{4.681120in}{2.720151in}}%
\pgfusepath{clip}%
\pgfsetbuttcap%
\pgfsetmiterjoin%
\definecolor{currentfill}{rgb}{0.003922,0.450980,0.698039}%
\pgfsetfillcolor{currentfill}%
\pgfsetfillopacity{0.700000}%
\pgfsetlinewidth{0.000000pt}%
\definecolor{currentstroke}{rgb}{0.000000,0.000000,0.000000}%
\pgfsetstrokecolor{currentstroke}%
\pgfsetstrokeopacity{0.700000}%
\pgfsetdash{}{0pt}%
\pgfpathmoveto{\pgfqpoint{0.873784in}{0.524170in}}%
\pgfpathlineto{\pgfqpoint{1.157488in}{0.524170in}}%
\pgfpathlineto{\pgfqpoint{1.157488in}{0.687379in}}%
\pgfpathlineto{\pgfqpoint{0.873784in}{0.687379in}}%
\pgfpathlineto{\pgfqpoint{0.873784in}{0.524170in}}%
\pgfpathclose%
\pgfusepath{fill}%
\end{pgfscope}%
\begin{pgfscope}%
\pgfpathrectangle{\pgfqpoint{0.661006in}{0.524170in}}{\pgfqpoint{4.681120in}{2.720151in}}%
\pgfusepath{clip}%
\pgfsetbuttcap%
\pgfsetmiterjoin%
\definecolor{currentfill}{rgb}{0.003922,0.450980,0.698039}%
\pgfsetfillcolor{currentfill}%
\pgfsetfillopacity{0.700000}%
\pgfsetlinewidth{0.000000pt}%
\definecolor{currentstroke}{rgb}{0.000000,0.000000,0.000000}%
\pgfsetstrokecolor{currentstroke}%
\pgfsetstrokeopacity{0.700000}%
\pgfsetdash{}{0pt}%
\pgfpathmoveto{\pgfqpoint{1.157488in}{0.524170in}}%
\pgfpathlineto{\pgfqpoint{1.441192in}{0.524170in}}%
\pgfpathlineto{\pgfqpoint{1.441192in}{0.592821in}}%
\pgfpathlineto{\pgfqpoint{1.157488in}{0.592821in}}%
\pgfpathlineto{\pgfqpoint{1.157488in}{0.524170in}}%
\pgfpathclose%
\pgfusepath{fill}%
\end{pgfscope}%
\begin{pgfscope}%
\pgfpathrectangle{\pgfqpoint{0.661006in}{0.524170in}}{\pgfqpoint{4.681120in}{2.720151in}}%
\pgfusepath{clip}%
\pgfsetbuttcap%
\pgfsetmiterjoin%
\definecolor{currentfill}{rgb}{0.003922,0.450980,0.698039}%
\pgfsetfillcolor{currentfill}%
\pgfsetfillopacity{0.700000}%
\pgfsetlinewidth{0.000000pt}%
\definecolor{currentstroke}{rgb}{0.000000,0.000000,0.000000}%
\pgfsetstrokecolor{currentstroke}%
\pgfsetstrokeopacity{0.700000}%
\pgfsetdash{}{0pt}%
\pgfpathmoveto{\pgfqpoint{1.441192in}{0.524170in}}%
\pgfpathlineto{\pgfqpoint{1.724897in}{0.524170in}}%
\pgfpathlineto{\pgfqpoint{1.724897in}{0.557200in}}%
\pgfpathlineto{\pgfqpoint{1.441192in}{0.557200in}}%
\pgfpathlineto{\pgfqpoint{1.441192in}{0.524170in}}%
\pgfpathclose%
\pgfusepath{fill}%
\end{pgfscope}%
\begin{pgfscope}%
\pgfpathrectangle{\pgfqpoint{0.661006in}{0.524170in}}{\pgfqpoint{4.681120in}{2.720151in}}%
\pgfusepath{clip}%
\pgfsetbuttcap%
\pgfsetmiterjoin%
\definecolor{currentfill}{rgb}{0.003922,0.450980,0.698039}%
\pgfsetfillcolor{currentfill}%
\pgfsetfillopacity{0.700000}%
\pgfsetlinewidth{0.000000pt}%
\definecolor{currentstroke}{rgb}{0.000000,0.000000,0.000000}%
\pgfsetstrokecolor{currentstroke}%
\pgfsetstrokeopacity{0.700000}%
\pgfsetdash{}{0pt}%
\pgfpathmoveto{\pgfqpoint{1.724897in}{0.524170in}}%
\pgfpathlineto{\pgfqpoint{2.008601in}{0.524170in}}%
\pgfpathlineto{\pgfqpoint{2.008601in}{0.542952in}}%
\pgfpathlineto{\pgfqpoint{1.724897in}{0.542952in}}%
\pgfpathlineto{\pgfqpoint{1.724897in}{0.524170in}}%
\pgfpathclose%
\pgfusepath{fill}%
\end{pgfscope}%
\begin{pgfscope}%
\pgfpathrectangle{\pgfqpoint{0.661006in}{0.524170in}}{\pgfqpoint{4.681120in}{2.720151in}}%
\pgfusepath{clip}%
\pgfsetbuttcap%
\pgfsetmiterjoin%
\definecolor{currentfill}{rgb}{0.003922,0.450980,0.698039}%
\pgfsetfillcolor{currentfill}%
\pgfsetfillopacity{0.700000}%
\pgfsetlinewidth{0.000000pt}%
\definecolor{currentstroke}{rgb}{0.000000,0.000000,0.000000}%
\pgfsetstrokecolor{currentstroke}%
\pgfsetstrokeopacity{0.700000}%
\pgfsetdash{}{0pt}%
\pgfpathmoveto{\pgfqpoint{2.008601in}{0.524170in}}%
\pgfpathlineto{\pgfqpoint{2.292305in}{0.524170in}}%
\pgfpathlineto{\pgfqpoint{2.292305in}{0.534532in}}%
\pgfpathlineto{\pgfqpoint{2.008601in}{0.534532in}}%
\pgfpathlineto{\pgfqpoint{2.008601in}{0.524170in}}%
\pgfpathclose%
\pgfusepath{fill}%
\end{pgfscope}%
\begin{pgfscope}%
\pgfpathrectangle{\pgfqpoint{0.661006in}{0.524170in}}{\pgfqpoint{4.681120in}{2.720151in}}%
\pgfusepath{clip}%
\pgfsetbuttcap%
\pgfsetmiterjoin%
\definecolor{currentfill}{rgb}{0.003922,0.450980,0.698039}%
\pgfsetfillcolor{currentfill}%
\pgfsetfillopacity{0.700000}%
\pgfsetlinewidth{0.000000pt}%
\definecolor{currentstroke}{rgb}{0.000000,0.000000,0.000000}%
\pgfsetstrokecolor{currentstroke}%
\pgfsetstrokeopacity{0.700000}%
\pgfsetdash{}{0pt}%
\pgfpathmoveto{\pgfqpoint{2.292305in}{0.524170in}}%
\pgfpathlineto{\pgfqpoint{2.576010in}{0.524170in}}%
\pgfpathlineto{\pgfqpoint{2.576010in}{0.533237in}}%
\pgfpathlineto{\pgfqpoint{2.292305in}{0.533237in}}%
\pgfpathlineto{\pgfqpoint{2.292305in}{0.524170in}}%
\pgfpathclose%
\pgfusepath{fill}%
\end{pgfscope}%
\begin{pgfscope}%
\pgfpathrectangle{\pgfqpoint{0.661006in}{0.524170in}}{\pgfqpoint{4.681120in}{2.720151in}}%
\pgfusepath{clip}%
\pgfsetbuttcap%
\pgfsetmiterjoin%
\definecolor{currentfill}{rgb}{0.003922,0.450980,0.698039}%
\pgfsetfillcolor{currentfill}%
\pgfsetfillopacity{0.700000}%
\pgfsetlinewidth{0.000000pt}%
\definecolor{currentstroke}{rgb}{0.000000,0.000000,0.000000}%
\pgfsetstrokecolor{currentstroke}%
\pgfsetstrokeopacity{0.700000}%
\pgfsetdash{}{0pt}%
\pgfpathmoveto{\pgfqpoint{2.576010in}{0.524170in}}%
\pgfpathlineto{\pgfqpoint{2.859714in}{0.524170in}}%
\pgfpathlineto{\pgfqpoint{2.859714in}{0.530646in}}%
\pgfpathlineto{\pgfqpoint{2.576010in}{0.530646in}}%
\pgfpathlineto{\pgfqpoint{2.576010in}{0.524170in}}%
\pgfpathclose%
\pgfusepath{fill}%
\end{pgfscope}%
\begin{pgfscope}%
\pgfpathrectangle{\pgfqpoint{0.661006in}{0.524170in}}{\pgfqpoint{4.681120in}{2.720151in}}%
\pgfusepath{clip}%
\pgfsetbuttcap%
\pgfsetmiterjoin%
\definecolor{currentfill}{rgb}{0.003922,0.450980,0.698039}%
\pgfsetfillcolor{currentfill}%
\pgfsetfillopacity{0.700000}%
\pgfsetlinewidth{0.000000pt}%
\definecolor{currentstroke}{rgb}{0.000000,0.000000,0.000000}%
\pgfsetstrokecolor{currentstroke}%
\pgfsetstrokeopacity{0.700000}%
\pgfsetdash{}{0pt}%
\pgfpathmoveto{\pgfqpoint{2.859714in}{0.524170in}}%
\pgfpathlineto{\pgfqpoint{3.143418in}{0.524170in}}%
\pgfpathlineto{\pgfqpoint{3.143418in}{0.530646in}}%
\pgfpathlineto{\pgfqpoint{2.859714in}{0.530646in}}%
\pgfpathlineto{\pgfqpoint{2.859714in}{0.524170in}}%
\pgfpathclose%
\pgfusepath{fill}%
\end{pgfscope}%
\begin{pgfscope}%
\pgfpathrectangle{\pgfqpoint{0.661006in}{0.524170in}}{\pgfqpoint{4.681120in}{2.720151in}}%
\pgfusepath{clip}%
\pgfsetbuttcap%
\pgfsetmiterjoin%
\definecolor{currentfill}{rgb}{0.003922,0.450980,0.698039}%
\pgfsetfillcolor{currentfill}%
\pgfsetfillopacity{0.700000}%
\pgfsetlinewidth{0.000000pt}%
\definecolor{currentstroke}{rgb}{0.000000,0.000000,0.000000}%
\pgfsetstrokecolor{currentstroke}%
\pgfsetstrokeopacity{0.700000}%
\pgfsetdash{}{0pt}%
\pgfpathmoveto{\pgfqpoint{3.143418in}{0.524170in}}%
\pgfpathlineto{\pgfqpoint{3.427122in}{0.524170in}}%
\pgfpathlineto{\pgfqpoint{3.427122in}{0.527408in}}%
\pgfpathlineto{\pgfqpoint{3.143418in}{0.527408in}}%
\pgfpathlineto{\pgfqpoint{3.143418in}{0.524170in}}%
\pgfpathclose%
\pgfusepath{fill}%
\end{pgfscope}%
\begin{pgfscope}%
\pgfpathrectangle{\pgfqpoint{0.661006in}{0.524170in}}{\pgfqpoint{4.681120in}{2.720151in}}%
\pgfusepath{clip}%
\pgfsetbuttcap%
\pgfsetmiterjoin%
\definecolor{currentfill}{rgb}{0.003922,0.450980,0.698039}%
\pgfsetfillcolor{currentfill}%
\pgfsetfillopacity{0.700000}%
\pgfsetlinewidth{0.000000pt}%
\definecolor{currentstroke}{rgb}{0.000000,0.000000,0.000000}%
\pgfsetstrokecolor{currentstroke}%
\pgfsetstrokeopacity{0.700000}%
\pgfsetdash{}{0pt}%
\pgfpathmoveto{\pgfqpoint{3.427122in}{0.524170in}}%
\pgfpathlineto{\pgfqpoint{3.710827in}{0.524170in}}%
\pgfpathlineto{\pgfqpoint{3.710827in}{0.526760in}}%
\pgfpathlineto{\pgfqpoint{3.427122in}{0.526760in}}%
\pgfpathlineto{\pgfqpoint{3.427122in}{0.524170in}}%
\pgfpathclose%
\pgfusepath{fill}%
\end{pgfscope}%
\begin{pgfscope}%
\pgfpathrectangle{\pgfqpoint{0.661006in}{0.524170in}}{\pgfqpoint{4.681120in}{2.720151in}}%
\pgfusepath{clip}%
\pgfsetbuttcap%
\pgfsetmiterjoin%
\definecolor{currentfill}{rgb}{0.003922,0.450980,0.698039}%
\pgfsetfillcolor{currentfill}%
\pgfsetfillopacity{0.700000}%
\pgfsetlinewidth{0.000000pt}%
\definecolor{currentstroke}{rgb}{0.000000,0.000000,0.000000}%
\pgfsetstrokecolor{currentstroke}%
\pgfsetstrokeopacity{0.700000}%
\pgfsetdash{}{0pt}%
\pgfpathmoveto{\pgfqpoint{3.710827in}{0.524170in}}%
\pgfpathlineto{\pgfqpoint{3.994531in}{0.524170in}}%
\pgfpathlineto{\pgfqpoint{3.994531in}{0.529351in}}%
\pgfpathlineto{\pgfqpoint{3.710827in}{0.529351in}}%
\pgfpathlineto{\pgfqpoint{3.710827in}{0.524170in}}%
\pgfpathclose%
\pgfusepath{fill}%
\end{pgfscope}%
\begin{pgfscope}%
\pgfpathrectangle{\pgfqpoint{0.661006in}{0.524170in}}{\pgfqpoint{4.681120in}{2.720151in}}%
\pgfusepath{clip}%
\pgfsetbuttcap%
\pgfsetmiterjoin%
\definecolor{currentfill}{rgb}{0.003922,0.450980,0.698039}%
\pgfsetfillcolor{currentfill}%
\pgfsetfillopacity{0.700000}%
\pgfsetlinewidth{0.000000pt}%
\definecolor{currentstroke}{rgb}{0.000000,0.000000,0.000000}%
\pgfsetstrokecolor{currentstroke}%
\pgfsetstrokeopacity{0.700000}%
\pgfsetdash{}{0pt}%
\pgfpathmoveto{\pgfqpoint{3.994531in}{0.524170in}}%
\pgfpathlineto{\pgfqpoint{4.278235in}{0.524170in}}%
\pgfpathlineto{\pgfqpoint{4.278235in}{0.525465in}}%
\pgfpathlineto{\pgfqpoint{3.994531in}{0.525465in}}%
\pgfpathlineto{\pgfqpoint{3.994531in}{0.524170in}}%
\pgfpathclose%
\pgfusepath{fill}%
\end{pgfscope}%
\begin{pgfscope}%
\pgfpathrectangle{\pgfqpoint{0.661006in}{0.524170in}}{\pgfqpoint{4.681120in}{2.720151in}}%
\pgfusepath{clip}%
\pgfsetbuttcap%
\pgfsetmiterjoin%
\definecolor{currentfill}{rgb}{0.003922,0.450980,0.698039}%
\pgfsetfillcolor{currentfill}%
\pgfsetfillopacity{0.700000}%
\pgfsetlinewidth{0.000000pt}%
\definecolor{currentstroke}{rgb}{0.000000,0.000000,0.000000}%
\pgfsetstrokecolor{currentstroke}%
\pgfsetstrokeopacity{0.700000}%
\pgfsetdash{}{0pt}%
\pgfpathmoveto{\pgfqpoint{4.278235in}{0.524170in}}%
\pgfpathlineto{\pgfqpoint{4.561939in}{0.524170in}}%
\pgfpathlineto{\pgfqpoint{4.561939in}{0.524817in}}%
\pgfpathlineto{\pgfqpoint{4.278235in}{0.524817in}}%
\pgfpathlineto{\pgfqpoint{4.278235in}{0.524170in}}%
\pgfpathclose%
\pgfusepath{fill}%
\end{pgfscope}%
\begin{pgfscope}%
\pgfpathrectangle{\pgfqpoint{0.661006in}{0.524170in}}{\pgfqpoint{4.681120in}{2.720151in}}%
\pgfusepath{clip}%
\pgfsetbuttcap%
\pgfsetmiterjoin%
\definecolor{currentfill}{rgb}{0.003922,0.450980,0.698039}%
\pgfsetfillcolor{currentfill}%
\pgfsetfillopacity{0.700000}%
\pgfsetlinewidth{0.000000pt}%
\definecolor{currentstroke}{rgb}{0.000000,0.000000,0.000000}%
\pgfsetstrokecolor{currentstroke}%
\pgfsetstrokeopacity{0.700000}%
\pgfsetdash{}{0pt}%
\pgfpathmoveto{\pgfqpoint{4.561939in}{0.524170in}}%
\pgfpathlineto{\pgfqpoint{4.845644in}{0.524170in}}%
\pgfpathlineto{\pgfqpoint{4.845644in}{0.524817in}}%
\pgfpathlineto{\pgfqpoint{4.561939in}{0.524817in}}%
\pgfpathlineto{\pgfqpoint{4.561939in}{0.524170in}}%
\pgfpathclose%
\pgfusepath{fill}%
\end{pgfscope}%
\begin{pgfscope}%
\pgfpathrectangle{\pgfqpoint{0.661006in}{0.524170in}}{\pgfqpoint{4.681120in}{2.720151in}}%
\pgfusepath{clip}%
\pgfsetbuttcap%
\pgfsetmiterjoin%
\definecolor{currentfill}{rgb}{0.003922,0.450980,0.698039}%
\pgfsetfillcolor{currentfill}%
\pgfsetfillopacity{0.700000}%
\pgfsetlinewidth{0.000000pt}%
\definecolor{currentstroke}{rgb}{0.000000,0.000000,0.000000}%
\pgfsetstrokecolor{currentstroke}%
\pgfsetstrokeopacity{0.700000}%
\pgfsetdash{}{0pt}%
\pgfpathmoveto{\pgfqpoint{4.845644in}{0.524170in}}%
\pgfpathlineto{\pgfqpoint{5.129348in}{0.524170in}}%
\pgfpathlineto{\pgfqpoint{5.129348in}{2.785133in}}%
\pgfpathlineto{\pgfqpoint{4.845644in}{2.785133in}}%
\pgfpathlineto{\pgfqpoint{4.845644in}{0.524170in}}%
\pgfpathclose%
\pgfusepath{fill}%
\end{pgfscope}%
\begin{pgfscope}%
\pgfpathrectangle{\pgfqpoint{0.661006in}{0.524170in}}{\pgfqpoint{4.681120in}{2.720151in}}%
\pgfusepath{clip}%
\pgfsetrectcap%
\pgfsetroundjoin%
\pgfsetlinewidth{0.803000pt}%
\definecolor{currentstroke}{rgb}{0.450000,0.450000,0.450000}%
\pgfsetstrokecolor{currentstroke}%
\pgfsetdash{}{0pt}%
\pgfpathmoveto{\pgfqpoint{0.847628in}{0.524170in}}%
\pgfpathlineto{\pgfqpoint{0.847628in}{3.244321in}}%
\pgfusepath{stroke}%
\end{pgfscope}%
\begin{pgfscope}%
\pgfsetbuttcap%
\pgfsetroundjoin%
\definecolor{currentfill}{rgb}{0.000000,0.000000,0.000000}%
\pgfsetfillcolor{currentfill}%
\pgfsetlinewidth{0.803000pt}%
\definecolor{currentstroke}{rgb}{0.000000,0.000000,0.000000}%
\pgfsetstrokecolor{currentstroke}%
\pgfsetdash{}{0pt}%
\pgfsys@defobject{currentmarker}{\pgfqpoint{0.000000in}{-0.048611in}}{\pgfqpoint{0.000000in}{0.000000in}}{%
\pgfpathmoveto{\pgfqpoint{0.000000in}{0.000000in}}%
\pgfpathlineto{\pgfqpoint{0.000000in}{-0.048611in}}%
\pgfusepath{stroke,fill}%
}%
\begin{pgfscope}%
\pgfsys@transformshift{0.847628in}{0.524170in}%
\pgfsys@useobject{currentmarker}{}%
\end{pgfscope}%
\end{pgfscope}%
\begin{pgfscope}%
\definecolor{textcolor}{rgb}{0.000000,0.000000,0.000000}%
\pgfsetstrokecolor{textcolor}%
\pgfsetfillcolor{textcolor}%
\pgftext[x=0.847628in,y=0.426948in,,top]{\color{textcolor}\rmfamily\fontsize{8.000000}{9.600000}\selectfont \(\displaystyle {0}\)}%
\end{pgfscope}%
\begin{pgfscope}%
\pgfpathrectangle{\pgfqpoint{0.661006in}{0.524170in}}{\pgfqpoint{4.681120in}{2.720151in}}%
\pgfusepath{clip}%
\pgfsetrectcap%
\pgfsetroundjoin%
\pgfsetlinewidth{0.803000pt}%
\definecolor{currentstroke}{rgb}{0.450000,0.450000,0.450000}%
\pgfsetstrokecolor{currentstroke}%
\pgfsetdash{}{0pt}%
\pgfpathmoveto{\pgfqpoint{1.636158in}{0.524170in}}%
\pgfpathlineto{\pgfqpoint{1.636158in}{3.244321in}}%
\pgfusepath{stroke}%
\end{pgfscope}%
\begin{pgfscope}%
\pgfsetbuttcap%
\pgfsetroundjoin%
\definecolor{currentfill}{rgb}{0.000000,0.000000,0.000000}%
\pgfsetfillcolor{currentfill}%
\pgfsetlinewidth{0.803000pt}%
\definecolor{currentstroke}{rgb}{0.000000,0.000000,0.000000}%
\pgfsetstrokecolor{currentstroke}%
\pgfsetdash{}{0pt}%
\pgfsys@defobject{currentmarker}{\pgfqpoint{0.000000in}{-0.048611in}}{\pgfqpoint{0.000000in}{0.000000in}}{%
\pgfpathmoveto{\pgfqpoint{0.000000in}{0.000000in}}%
\pgfpathlineto{\pgfqpoint{0.000000in}{-0.048611in}}%
\pgfusepath{stroke,fill}%
}%
\begin{pgfscope}%
\pgfsys@transformshift{1.636158in}{0.524170in}%
\pgfsys@useobject{currentmarker}{}%
\end{pgfscope}%
\end{pgfscope}%
\begin{pgfscope}%
\definecolor{textcolor}{rgb}{0.000000,0.000000,0.000000}%
\pgfsetstrokecolor{textcolor}%
\pgfsetfillcolor{textcolor}%
\pgftext[x=1.636158in,y=0.426948in,,top]{\color{textcolor}\rmfamily\fontsize{8.000000}{9.600000}\selectfont \(\displaystyle {1}\)}%
\end{pgfscope}%
\begin{pgfscope}%
\pgfpathrectangle{\pgfqpoint{0.661006in}{0.524170in}}{\pgfqpoint{4.681120in}{2.720151in}}%
\pgfusepath{clip}%
\pgfsetrectcap%
\pgfsetroundjoin%
\pgfsetlinewidth{0.803000pt}%
\definecolor{currentstroke}{rgb}{0.450000,0.450000,0.450000}%
\pgfsetstrokecolor{currentstroke}%
\pgfsetdash{}{0pt}%
\pgfpathmoveto{\pgfqpoint{2.424688in}{0.524170in}}%
\pgfpathlineto{\pgfqpoint{2.424688in}{3.244321in}}%
\pgfusepath{stroke}%
\end{pgfscope}%
\begin{pgfscope}%
\pgfsetbuttcap%
\pgfsetroundjoin%
\definecolor{currentfill}{rgb}{0.000000,0.000000,0.000000}%
\pgfsetfillcolor{currentfill}%
\pgfsetlinewidth{0.803000pt}%
\definecolor{currentstroke}{rgb}{0.000000,0.000000,0.000000}%
\pgfsetstrokecolor{currentstroke}%
\pgfsetdash{}{0pt}%
\pgfsys@defobject{currentmarker}{\pgfqpoint{0.000000in}{-0.048611in}}{\pgfqpoint{0.000000in}{0.000000in}}{%
\pgfpathmoveto{\pgfqpoint{0.000000in}{0.000000in}}%
\pgfpathlineto{\pgfqpoint{0.000000in}{-0.048611in}}%
\pgfusepath{stroke,fill}%
}%
\begin{pgfscope}%
\pgfsys@transformshift{2.424688in}{0.524170in}%
\pgfsys@useobject{currentmarker}{}%
\end{pgfscope}%
\end{pgfscope}%
\begin{pgfscope}%
\definecolor{textcolor}{rgb}{0.000000,0.000000,0.000000}%
\pgfsetstrokecolor{textcolor}%
\pgfsetfillcolor{textcolor}%
\pgftext[x=2.424688in,y=0.426948in,,top]{\color{textcolor}\rmfamily\fontsize{8.000000}{9.600000}\selectfont \(\displaystyle {2}\)}%
\end{pgfscope}%
\begin{pgfscope}%
\pgfpathrectangle{\pgfqpoint{0.661006in}{0.524170in}}{\pgfqpoint{4.681120in}{2.720151in}}%
\pgfusepath{clip}%
\pgfsetrectcap%
\pgfsetroundjoin%
\pgfsetlinewidth{0.803000pt}%
\definecolor{currentstroke}{rgb}{0.450000,0.450000,0.450000}%
\pgfsetstrokecolor{currentstroke}%
\pgfsetdash{}{0pt}%
\pgfpathmoveto{\pgfqpoint{3.213218in}{0.524170in}}%
\pgfpathlineto{\pgfqpoint{3.213218in}{3.244321in}}%
\pgfusepath{stroke}%
\end{pgfscope}%
\begin{pgfscope}%
\pgfsetbuttcap%
\pgfsetroundjoin%
\definecolor{currentfill}{rgb}{0.000000,0.000000,0.000000}%
\pgfsetfillcolor{currentfill}%
\pgfsetlinewidth{0.803000pt}%
\definecolor{currentstroke}{rgb}{0.000000,0.000000,0.000000}%
\pgfsetstrokecolor{currentstroke}%
\pgfsetdash{}{0pt}%
\pgfsys@defobject{currentmarker}{\pgfqpoint{0.000000in}{-0.048611in}}{\pgfqpoint{0.000000in}{0.000000in}}{%
\pgfpathmoveto{\pgfqpoint{0.000000in}{0.000000in}}%
\pgfpathlineto{\pgfqpoint{0.000000in}{-0.048611in}}%
\pgfusepath{stroke,fill}%
}%
\begin{pgfscope}%
\pgfsys@transformshift{3.213218in}{0.524170in}%
\pgfsys@useobject{currentmarker}{}%
\end{pgfscope}%
\end{pgfscope}%
\begin{pgfscope}%
\definecolor{textcolor}{rgb}{0.000000,0.000000,0.000000}%
\pgfsetstrokecolor{textcolor}%
\pgfsetfillcolor{textcolor}%
\pgftext[x=3.213218in,y=0.426948in,,top]{\color{textcolor}\rmfamily\fontsize{8.000000}{9.600000}\selectfont \(\displaystyle {3}\)}%
\end{pgfscope}%
\begin{pgfscope}%
\pgfpathrectangle{\pgfqpoint{0.661006in}{0.524170in}}{\pgfqpoint{4.681120in}{2.720151in}}%
\pgfusepath{clip}%
\pgfsetrectcap%
\pgfsetroundjoin%
\pgfsetlinewidth{0.803000pt}%
\definecolor{currentstroke}{rgb}{0.450000,0.450000,0.450000}%
\pgfsetstrokecolor{currentstroke}%
\pgfsetdash{}{0pt}%
\pgfpathmoveto{\pgfqpoint{4.001748in}{0.524170in}}%
\pgfpathlineto{\pgfqpoint{4.001748in}{3.244321in}}%
\pgfusepath{stroke}%
\end{pgfscope}%
\begin{pgfscope}%
\pgfsetbuttcap%
\pgfsetroundjoin%
\definecolor{currentfill}{rgb}{0.000000,0.000000,0.000000}%
\pgfsetfillcolor{currentfill}%
\pgfsetlinewidth{0.803000pt}%
\definecolor{currentstroke}{rgb}{0.000000,0.000000,0.000000}%
\pgfsetstrokecolor{currentstroke}%
\pgfsetdash{}{0pt}%
\pgfsys@defobject{currentmarker}{\pgfqpoint{0.000000in}{-0.048611in}}{\pgfqpoint{0.000000in}{0.000000in}}{%
\pgfpathmoveto{\pgfqpoint{0.000000in}{0.000000in}}%
\pgfpathlineto{\pgfqpoint{0.000000in}{-0.048611in}}%
\pgfusepath{stroke,fill}%
}%
\begin{pgfscope}%
\pgfsys@transformshift{4.001748in}{0.524170in}%
\pgfsys@useobject{currentmarker}{}%
\end{pgfscope}%
\end{pgfscope}%
\begin{pgfscope}%
\definecolor{textcolor}{rgb}{0.000000,0.000000,0.000000}%
\pgfsetstrokecolor{textcolor}%
\pgfsetfillcolor{textcolor}%
\pgftext[x=4.001748in,y=0.426948in,,top]{\color{textcolor}\rmfamily\fontsize{8.000000}{9.600000}\selectfont \(\displaystyle {4}\)}%
\end{pgfscope}%
\begin{pgfscope}%
\pgfpathrectangle{\pgfqpoint{0.661006in}{0.524170in}}{\pgfqpoint{4.681120in}{2.720151in}}%
\pgfusepath{clip}%
\pgfsetrectcap%
\pgfsetroundjoin%
\pgfsetlinewidth{0.803000pt}%
\definecolor{currentstroke}{rgb}{0.450000,0.450000,0.450000}%
\pgfsetstrokecolor{currentstroke}%
\pgfsetdash{}{0pt}%
\pgfpathmoveto{\pgfqpoint{4.790278in}{0.524170in}}%
\pgfpathlineto{\pgfqpoint{4.790278in}{3.244321in}}%
\pgfusepath{stroke}%
\end{pgfscope}%
\begin{pgfscope}%
\pgfsetbuttcap%
\pgfsetroundjoin%
\definecolor{currentfill}{rgb}{0.000000,0.000000,0.000000}%
\pgfsetfillcolor{currentfill}%
\pgfsetlinewidth{0.803000pt}%
\definecolor{currentstroke}{rgb}{0.000000,0.000000,0.000000}%
\pgfsetstrokecolor{currentstroke}%
\pgfsetdash{}{0pt}%
\pgfsys@defobject{currentmarker}{\pgfqpoint{0.000000in}{-0.048611in}}{\pgfqpoint{0.000000in}{0.000000in}}{%
\pgfpathmoveto{\pgfqpoint{0.000000in}{0.000000in}}%
\pgfpathlineto{\pgfqpoint{0.000000in}{-0.048611in}}%
\pgfusepath{stroke,fill}%
}%
\begin{pgfscope}%
\pgfsys@transformshift{4.790278in}{0.524170in}%
\pgfsys@useobject{currentmarker}{}%
\end{pgfscope}%
\end{pgfscope}%
\begin{pgfscope}%
\definecolor{textcolor}{rgb}{0.000000,0.000000,0.000000}%
\pgfsetstrokecolor{textcolor}%
\pgfsetfillcolor{textcolor}%
\pgftext[x=4.790278in,y=0.426948in,,top]{\color{textcolor}\rmfamily\fontsize{8.000000}{9.600000}\selectfont \(\displaystyle {5}\)}%
\end{pgfscope}%
\begin{pgfscope}%
\definecolor{textcolor}{rgb}{0.000000,0.000000,0.000000}%
\pgfsetstrokecolor{textcolor}%
\pgfsetfillcolor{textcolor}%
\pgftext[x=3.001566in,y=0.272725in,,top]{\color{textcolor}\rmfamily\fontsize{10.000000}{12.000000}\selectfont Output impedance in \unit{\ohm}}%
\end{pgfscope}%
\begin{pgfscope}%
\definecolor{textcolor}{rgb}{0.000000,0.000000,0.000000}%
\pgfsetstrokecolor{textcolor}%
\pgfsetfillcolor{textcolor}%
\pgftext[x=5.342126in,y=0.286614in,right,top]{\color{textcolor}\rmfamily\fontsize{8.000000}{9.600000}\selectfont \(\displaystyle \times{10^{10}}{}\)}%
\end{pgfscope}%
\begin{pgfscope}%
\pgfpathrectangle{\pgfqpoint{0.661006in}{0.524170in}}{\pgfqpoint{4.681120in}{2.720151in}}%
\pgfusepath{clip}%
\pgfsetrectcap%
\pgfsetroundjoin%
\pgfsetlinewidth{0.803000pt}%
\definecolor{currentstroke}{rgb}{0.450000,0.450000,0.450000}%
\pgfsetstrokecolor{currentstroke}%
\pgfsetdash{}{0pt}%
\pgfpathmoveto{\pgfqpoint{0.661006in}{0.524170in}}%
\pgfpathlineto{\pgfqpoint{5.342126in}{0.524170in}}%
\pgfusepath{stroke}%
\end{pgfscope}%
\begin{pgfscope}%
\pgfsetbuttcap%
\pgfsetroundjoin%
\definecolor{currentfill}{rgb}{0.000000,0.000000,0.000000}%
\pgfsetfillcolor{currentfill}%
\pgfsetlinewidth{0.803000pt}%
\definecolor{currentstroke}{rgb}{0.000000,0.000000,0.000000}%
\pgfsetstrokecolor{currentstroke}%
\pgfsetdash{}{0pt}%
\pgfsys@defobject{currentmarker}{\pgfqpoint{-0.048611in}{0.000000in}}{\pgfqpoint{-0.000000in}{0.000000in}}{%
\pgfpathmoveto{\pgfqpoint{-0.000000in}{0.000000in}}%
\pgfpathlineto{\pgfqpoint{-0.048611in}{0.000000in}}%
\pgfusepath{stroke,fill}%
}%
\begin{pgfscope}%
\pgfsys@transformshift{0.661006in}{0.524170in}%
\pgfsys@useobject{currentmarker}{}%
\end{pgfscope}%
\end{pgfscope}%
\begin{pgfscope}%
\definecolor{textcolor}{rgb}{0.000000,0.000000,0.000000}%
\pgfsetstrokecolor{textcolor}%
\pgfsetfillcolor{textcolor}%
\pgftext[x=0.504755in, y=0.485614in, left, base]{\color{textcolor}\rmfamily\fontsize{8.000000}{9.600000}\selectfont \(\displaystyle {0}\)}%
\end{pgfscope}%
\begin{pgfscope}%
\pgfpathrectangle{\pgfqpoint{0.661006in}{0.524170in}}{\pgfqpoint{4.681120in}{2.720151in}}%
\pgfusepath{clip}%
\pgfsetrectcap%
\pgfsetroundjoin%
\pgfsetlinewidth{0.803000pt}%
\definecolor{currentstroke}{rgb}{0.450000,0.450000,0.450000}%
\pgfsetstrokecolor{currentstroke}%
\pgfsetdash{}{0pt}%
\pgfpathmoveto{\pgfqpoint{0.661006in}{0.847997in}}%
\pgfpathlineto{\pgfqpoint{5.342126in}{0.847997in}}%
\pgfusepath{stroke}%
\end{pgfscope}%
\begin{pgfscope}%
\pgfsetbuttcap%
\pgfsetroundjoin%
\definecolor{currentfill}{rgb}{0.000000,0.000000,0.000000}%
\pgfsetfillcolor{currentfill}%
\pgfsetlinewidth{0.803000pt}%
\definecolor{currentstroke}{rgb}{0.000000,0.000000,0.000000}%
\pgfsetstrokecolor{currentstroke}%
\pgfsetdash{}{0pt}%
\pgfsys@defobject{currentmarker}{\pgfqpoint{-0.048611in}{0.000000in}}{\pgfqpoint{-0.000000in}{0.000000in}}{%
\pgfpathmoveto{\pgfqpoint{-0.000000in}{0.000000in}}%
\pgfpathlineto{\pgfqpoint{-0.048611in}{0.000000in}}%
\pgfusepath{stroke,fill}%
}%
\begin{pgfscope}%
\pgfsys@transformshift{0.661006in}{0.847997in}%
\pgfsys@useobject{currentmarker}{}%
\end{pgfscope}%
\end{pgfscope}%
\begin{pgfscope}%
\definecolor{textcolor}{rgb}{0.000000,0.000000,0.000000}%
\pgfsetstrokecolor{textcolor}%
\pgfsetfillcolor{textcolor}%
\pgftext[x=0.386698in, y=0.809442in, left, base]{\color{textcolor}\rmfamily\fontsize{8.000000}{9.600000}\selectfont \(\displaystyle {500}\)}%
\end{pgfscope}%
\begin{pgfscope}%
\pgfpathrectangle{\pgfqpoint{0.661006in}{0.524170in}}{\pgfqpoint{4.681120in}{2.720151in}}%
\pgfusepath{clip}%
\pgfsetrectcap%
\pgfsetroundjoin%
\pgfsetlinewidth{0.803000pt}%
\definecolor{currentstroke}{rgb}{0.450000,0.450000,0.450000}%
\pgfsetstrokecolor{currentstroke}%
\pgfsetdash{}{0pt}%
\pgfpathmoveto{\pgfqpoint{0.661006in}{1.171825in}}%
\pgfpathlineto{\pgfqpoint{5.342126in}{1.171825in}}%
\pgfusepath{stroke}%
\end{pgfscope}%
\begin{pgfscope}%
\pgfsetbuttcap%
\pgfsetroundjoin%
\definecolor{currentfill}{rgb}{0.000000,0.000000,0.000000}%
\pgfsetfillcolor{currentfill}%
\pgfsetlinewidth{0.803000pt}%
\definecolor{currentstroke}{rgb}{0.000000,0.000000,0.000000}%
\pgfsetstrokecolor{currentstroke}%
\pgfsetdash{}{0pt}%
\pgfsys@defobject{currentmarker}{\pgfqpoint{-0.048611in}{0.000000in}}{\pgfqpoint{-0.000000in}{0.000000in}}{%
\pgfpathmoveto{\pgfqpoint{-0.000000in}{0.000000in}}%
\pgfpathlineto{\pgfqpoint{-0.048611in}{0.000000in}}%
\pgfusepath{stroke,fill}%
}%
\begin{pgfscope}%
\pgfsys@transformshift{0.661006in}{1.171825in}%
\pgfsys@useobject{currentmarker}{}%
\end{pgfscope}%
\end{pgfscope}%
\begin{pgfscope}%
\definecolor{textcolor}{rgb}{0.000000,0.000000,0.000000}%
\pgfsetstrokecolor{textcolor}%
\pgfsetfillcolor{textcolor}%
\pgftext[x=0.327669in, y=1.133269in, left, base]{\color{textcolor}\rmfamily\fontsize{8.000000}{9.600000}\selectfont \(\displaystyle {1000}\)}%
\end{pgfscope}%
\begin{pgfscope}%
\pgfpathrectangle{\pgfqpoint{0.661006in}{0.524170in}}{\pgfqpoint{4.681120in}{2.720151in}}%
\pgfusepath{clip}%
\pgfsetrectcap%
\pgfsetroundjoin%
\pgfsetlinewidth{0.803000pt}%
\definecolor{currentstroke}{rgb}{0.450000,0.450000,0.450000}%
\pgfsetstrokecolor{currentstroke}%
\pgfsetdash{}{0pt}%
\pgfpathmoveto{\pgfqpoint{0.661006in}{1.495652in}}%
\pgfpathlineto{\pgfqpoint{5.342126in}{1.495652in}}%
\pgfusepath{stroke}%
\end{pgfscope}%
\begin{pgfscope}%
\pgfsetbuttcap%
\pgfsetroundjoin%
\definecolor{currentfill}{rgb}{0.000000,0.000000,0.000000}%
\pgfsetfillcolor{currentfill}%
\pgfsetlinewidth{0.803000pt}%
\definecolor{currentstroke}{rgb}{0.000000,0.000000,0.000000}%
\pgfsetstrokecolor{currentstroke}%
\pgfsetdash{}{0pt}%
\pgfsys@defobject{currentmarker}{\pgfqpoint{-0.048611in}{0.000000in}}{\pgfqpoint{-0.000000in}{0.000000in}}{%
\pgfpathmoveto{\pgfqpoint{-0.000000in}{0.000000in}}%
\pgfpathlineto{\pgfqpoint{-0.048611in}{0.000000in}}%
\pgfusepath{stroke,fill}%
}%
\begin{pgfscope}%
\pgfsys@transformshift{0.661006in}{1.495652in}%
\pgfsys@useobject{currentmarker}{}%
\end{pgfscope}%
\end{pgfscope}%
\begin{pgfscope}%
\definecolor{textcolor}{rgb}{0.000000,0.000000,0.000000}%
\pgfsetstrokecolor{textcolor}%
\pgfsetfillcolor{textcolor}%
\pgftext[x=0.327669in, y=1.457097in, left, base]{\color{textcolor}\rmfamily\fontsize{8.000000}{9.600000}\selectfont \(\displaystyle {1500}\)}%
\end{pgfscope}%
\begin{pgfscope}%
\pgfpathrectangle{\pgfqpoint{0.661006in}{0.524170in}}{\pgfqpoint{4.681120in}{2.720151in}}%
\pgfusepath{clip}%
\pgfsetrectcap%
\pgfsetroundjoin%
\pgfsetlinewidth{0.803000pt}%
\definecolor{currentstroke}{rgb}{0.450000,0.450000,0.450000}%
\pgfsetstrokecolor{currentstroke}%
\pgfsetdash{}{0pt}%
\pgfpathmoveto{\pgfqpoint{0.661006in}{1.819480in}}%
\pgfpathlineto{\pgfqpoint{5.342126in}{1.819480in}}%
\pgfusepath{stroke}%
\end{pgfscope}%
\begin{pgfscope}%
\pgfsetbuttcap%
\pgfsetroundjoin%
\definecolor{currentfill}{rgb}{0.000000,0.000000,0.000000}%
\pgfsetfillcolor{currentfill}%
\pgfsetlinewidth{0.803000pt}%
\definecolor{currentstroke}{rgb}{0.000000,0.000000,0.000000}%
\pgfsetstrokecolor{currentstroke}%
\pgfsetdash{}{0pt}%
\pgfsys@defobject{currentmarker}{\pgfqpoint{-0.048611in}{0.000000in}}{\pgfqpoint{-0.000000in}{0.000000in}}{%
\pgfpathmoveto{\pgfqpoint{-0.000000in}{0.000000in}}%
\pgfpathlineto{\pgfqpoint{-0.048611in}{0.000000in}}%
\pgfusepath{stroke,fill}%
}%
\begin{pgfscope}%
\pgfsys@transformshift{0.661006in}{1.819480in}%
\pgfsys@useobject{currentmarker}{}%
\end{pgfscope}%
\end{pgfscope}%
\begin{pgfscope}%
\definecolor{textcolor}{rgb}{0.000000,0.000000,0.000000}%
\pgfsetstrokecolor{textcolor}%
\pgfsetfillcolor{textcolor}%
\pgftext[x=0.327669in, y=1.780924in, left, base]{\color{textcolor}\rmfamily\fontsize{8.000000}{9.600000}\selectfont \(\displaystyle {2000}\)}%
\end{pgfscope}%
\begin{pgfscope}%
\pgfpathrectangle{\pgfqpoint{0.661006in}{0.524170in}}{\pgfqpoint{4.681120in}{2.720151in}}%
\pgfusepath{clip}%
\pgfsetrectcap%
\pgfsetroundjoin%
\pgfsetlinewidth{0.803000pt}%
\definecolor{currentstroke}{rgb}{0.450000,0.450000,0.450000}%
\pgfsetstrokecolor{currentstroke}%
\pgfsetdash{}{0pt}%
\pgfpathmoveto{\pgfqpoint{0.661006in}{2.143307in}}%
\pgfpathlineto{\pgfqpoint{5.342126in}{2.143307in}}%
\pgfusepath{stroke}%
\end{pgfscope}%
\begin{pgfscope}%
\pgfsetbuttcap%
\pgfsetroundjoin%
\definecolor{currentfill}{rgb}{0.000000,0.000000,0.000000}%
\pgfsetfillcolor{currentfill}%
\pgfsetlinewidth{0.803000pt}%
\definecolor{currentstroke}{rgb}{0.000000,0.000000,0.000000}%
\pgfsetstrokecolor{currentstroke}%
\pgfsetdash{}{0pt}%
\pgfsys@defobject{currentmarker}{\pgfqpoint{-0.048611in}{0.000000in}}{\pgfqpoint{-0.000000in}{0.000000in}}{%
\pgfpathmoveto{\pgfqpoint{-0.000000in}{0.000000in}}%
\pgfpathlineto{\pgfqpoint{-0.048611in}{0.000000in}}%
\pgfusepath{stroke,fill}%
}%
\begin{pgfscope}%
\pgfsys@transformshift{0.661006in}{2.143307in}%
\pgfsys@useobject{currentmarker}{}%
\end{pgfscope}%
\end{pgfscope}%
\begin{pgfscope}%
\definecolor{textcolor}{rgb}{0.000000,0.000000,0.000000}%
\pgfsetstrokecolor{textcolor}%
\pgfsetfillcolor{textcolor}%
\pgftext[x=0.327669in, y=2.104752in, left, base]{\color{textcolor}\rmfamily\fontsize{8.000000}{9.600000}\selectfont \(\displaystyle {2500}\)}%
\end{pgfscope}%
\begin{pgfscope}%
\pgfpathrectangle{\pgfqpoint{0.661006in}{0.524170in}}{\pgfqpoint{4.681120in}{2.720151in}}%
\pgfusepath{clip}%
\pgfsetrectcap%
\pgfsetroundjoin%
\pgfsetlinewidth{0.803000pt}%
\definecolor{currentstroke}{rgb}{0.450000,0.450000,0.450000}%
\pgfsetstrokecolor{currentstroke}%
\pgfsetdash{}{0pt}%
\pgfpathmoveto{\pgfqpoint{0.661006in}{2.467135in}}%
\pgfpathlineto{\pgfqpoint{5.342126in}{2.467135in}}%
\pgfusepath{stroke}%
\end{pgfscope}%
\begin{pgfscope}%
\pgfsetbuttcap%
\pgfsetroundjoin%
\definecolor{currentfill}{rgb}{0.000000,0.000000,0.000000}%
\pgfsetfillcolor{currentfill}%
\pgfsetlinewidth{0.803000pt}%
\definecolor{currentstroke}{rgb}{0.000000,0.000000,0.000000}%
\pgfsetstrokecolor{currentstroke}%
\pgfsetdash{}{0pt}%
\pgfsys@defobject{currentmarker}{\pgfqpoint{-0.048611in}{0.000000in}}{\pgfqpoint{-0.000000in}{0.000000in}}{%
\pgfpathmoveto{\pgfqpoint{-0.000000in}{0.000000in}}%
\pgfpathlineto{\pgfqpoint{-0.048611in}{0.000000in}}%
\pgfusepath{stroke,fill}%
}%
\begin{pgfscope}%
\pgfsys@transformshift{0.661006in}{2.467135in}%
\pgfsys@useobject{currentmarker}{}%
\end{pgfscope}%
\end{pgfscope}%
\begin{pgfscope}%
\definecolor{textcolor}{rgb}{0.000000,0.000000,0.000000}%
\pgfsetstrokecolor{textcolor}%
\pgfsetfillcolor{textcolor}%
\pgftext[x=0.327669in, y=2.428579in, left, base]{\color{textcolor}\rmfamily\fontsize{8.000000}{9.600000}\selectfont \(\displaystyle {3000}\)}%
\end{pgfscope}%
\begin{pgfscope}%
\pgfpathrectangle{\pgfqpoint{0.661006in}{0.524170in}}{\pgfqpoint{4.681120in}{2.720151in}}%
\pgfusepath{clip}%
\pgfsetrectcap%
\pgfsetroundjoin%
\pgfsetlinewidth{0.803000pt}%
\definecolor{currentstroke}{rgb}{0.450000,0.450000,0.450000}%
\pgfsetstrokecolor{currentstroke}%
\pgfsetdash{}{0pt}%
\pgfpathmoveto{\pgfqpoint{0.661006in}{2.790962in}}%
\pgfpathlineto{\pgfqpoint{5.342126in}{2.790962in}}%
\pgfusepath{stroke}%
\end{pgfscope}%
\begin{pgfscope}%
\pgfsetbuttcap%
\pgfsetroundjoin%
\definecolor{currentfill}{rgb}{0.000000,0.000000,0.000000}%
\pgfsetfillcolor{currentfill}%
\pgfsetlinewidth{0.803000pt}%
\definecolor{currentstroke}{rgb}{0.000000,0.000000,0.000000}%
\pgfsetstrokecolor{currentstroke}%
\pgfsetdash{}{0pt}%
\pgfsys@defobject{currentmarker}{\pgfqpoint{-0.048611in}{0.000000in}}{\pgfqpoint{-0.000000in}{0.000000in}}{%
\pgfpathmoveto{\pgfqpoint{-0.000000in}{0.000000in}}%
\pgfpathlineto{\pgfqpoint{-0.048611in}{0.000000in}}%
\pgfusepath{stroke,fill}%
}%
\begin{pgfscope}%
\pgfsys@transformshift{0.661006in}{2.790962in}%
\pgfsys@useobject{currentmarker}{}%
\end{pgfscope}%
\end{pgfscope}%
\begin{pgfscope}%
\definecolor{textcolor}{rgb}{0.000000,0.000000,0.000000}%
\pgfsetstrokecolor{textcolor}%
\pgfsetfillcolor{textcolor}%
\pgftext[x=0.327669in, y=2.752407in, left, base]{\color{textcolor}\rmfamily\fontsize{8.000000}{9.600000}\selectfont \(\displaystyle {3500}\)}%
\end{pgfscope}%
\begin{pgfscope}%
\pgfpathrectangle{\pgfqpoint{0.661006in}{0.524170in}}{\pgfqpoint{4.681120in}{2.720151in}}%
\pgfusepath{clip}%
\pgfsetrectcap%
\pgfsetroundjoin%
\pgfsetlinewidth{0.803000pt}%
\definecolor{currentstroke}{rgb}{0.450000,0.450000,0.450000}%
\pgfsetstrokecolor{currentstroke}%
\pgfsetdash{}{0pt}%
\pgfpathmoveto{\pgfqpoint{0.661006in}{3.114790in}}%
\pgfpathlineto{\pgfqpoint{5.342126in}{3.114790in}}%
\pgfusepath{stroke}%
\end{pgfscope}%
\begin{pgfscope}%
\pgfsetbuttcap%
\pgfsetroundjoin%
\definecolor{currentfill}{rgb}{0.000000,0.000000,0.000000}%
\pgfsetfillcolor{currentfill}%
\pgfsetlinewidth{0.803000pt}%
\definecolor{currentstroke}{rgb}{0.000000,0.000000,0.000000}%
\pgfsetstrokecolor{currentstroke}%
\pgfsetdash{}{0pt}%
\pgfsys@defobject{currentmarker}{\pgfqpoint{-0.048611in}{0.000000in}}{\pgfqpoint{-0.000000in}{0.000000in}}{%
\pgfpathmoveto{\pgfqpoint{-0.000000in}{0.000000in}}%
\pgfpathlineto{\pgfqpoint{-0.048611in}{0.000000in}}%
\pgfusepath{stroke,fill}%
}%
\begin{pgfscope}%
\pgfsys@transformshift{0.661006in}{3.114790in}%
\pgfsys@useobject{currentmarker}{}%
\end{pgfscope}%
\end{pgfscope}%
\begin{pgfscope}%
\definecolor{textcolor}{rgb}{0.000000,0.000000,0.000000}%
\pgfsetstrokecolor{textcolor}%
\pgfsetfillcolor{textcolor}%
\pgftext[x=0.327669in, y=3.076234in, left, base]{\color{textcolor}\rmfamily\fontsize{8.000000}{9.600000}\selectfont \(\displaystyle {4000}\)}%
\end{pgfscope}%
\begin{pgfscope}%
\definecolor{textcolor}{rgb}{0.000000,0.000000,0.000000}%
\pgfsetstrokecolor{textcolor}%
\pgfsetfillcolor{textcolor}%
\pgftext[x=0.272113in,y=1.884245in,,bottom,rotate=90.000000]{\color{textcolor}\rmfamily\fontsize{10.000000}{12.000000}\selectfont Counts}%
\end{pgfscope}%
\begin{pgfscope}%
\pgfsetrectcap%
\pgfsetmiterjoin%
\pgfsetlinewidth{0.803000pt}%
\definecolor{currentstroke}{rgb}{0.000000,0.000000,0.000000}%
\pgfsetstrokecolor{currentstroke}%
\pgfsetdash{}{0pt}%
\pgfpathmoveto{\pgfqpoint{0.661006in}{0.524170in}}%
\pgfpathlineto{\pgfqpoint{0.661006in}{3.244321in}}%
\pgfusepath{stroke}%
\end{pgfscope}%
\begin{pgfscope}%
\pgfsetrectcap%
\pgfsetmiterjoin%
\pgfsetlinewidth{0.803000pt}%
\definecolor{currentstroke}{rgb}{0.000000,0.000000,0.000000}%
\pgfsetstrokecolor{currentstroke}%
\pgfsetdash{}{0pt}%
\pgfpathmoveto{\pgfqpoint{5.342126in}{0.524170in}}%
\pgfpathlineto{\pgfqpoint{5.342126in}{3.244321in}}%
\pgfusepath{stroke}%
\end{pgfscope}%
\begin{pgfscope}%
\pgfsetrectcap%
\pgfsetmiterjoin%
\pgfsetlinewidth{0.803000pt}%
\definecolor{currentstroke}{rgb}{0.000000,0.000000,0.000000}%
\pgfsetstrokecolor{currentstroke}%
\pgfsetdash{}{0pt}%
\pgfpathmoveto{\pgfqpoint{0.661006in}{0.524170in}}%
\pgfpathlineto{\pgfqpoint{5.342126in}{0.524170in}}%
\pgfusepath{stroke}%
\end{pgfscope}%
\begin{pgfscope}%
\pgfsetrectcap%
\pgfsetmiterjoin%
\pgfsetlinewidth{0.803000pt}%
\definecolor{currentstroke}{rgb}{0.000000,0.000000,0.000000}%
\pgfsetstrokecolor{currentstroke}%
\pgfsetdash{}{0pt}%
\pgfpathmoveto{\pgfqpoint{0.661006in}{3.244321in}}%
\pgfpathlineto{\pgfqpoint{5.342126in}{3.244321in}}%
\pgfusepath{stroke}%
\end{pgfscope}%
\begin{pgfscope}%
\pgfsetbuttcap%
\pgfsetmiterjoin%
\definecolor{currentfill}{rgb}{1.000000,1.000000,1.000000}%
\pgfsetfillcolor{currentfill}%
\pgfsetfillopacity{0.800000}%
\pgfsetlinewidth{1.003750pt}%
\definecolor{currentstroke}{rgb}{0.800000,0.800000,0.800000}%
\pgfsetstrokecolor{currentstroke}%
\pgfsetstrokeopacity{0.800000}%
\pgfsetdash{}{0pt}%
\pgfpathmoveto{\pgfqpoint{0.738783in}{2.689099in}}%
\pgfpathlineto{\pgfqpoint{2.523463in}{2.689099in}}%
\pgfpathquadraticcurveto{\pgfqpoint{2.545685in}{2.689099in}}{\pgfqpoint{2.545685in}{2.711321in}}%
\pgfpathlineto{\pgfqpoint{2.545685in}{3.166543in}}%
\pgfpathquadraticcurveto{\pgfqpoint{2.545685in}{3.188765in}}{\pgfqpoint{2.523463in}{3.188765in}}%
\pgfpathlineto{\pgfqpoint{0.738783in}{3.188765in}}%
\pgfpathquadraticcurveto{\pgfqpoint{0.716561in}{3.188765in}}{\pgfqpoint{0.716561in}{3.166543in}}%
\pgfpathlineto{\pgfqpoint{0.716561in}{2.711321in}}%
\pgfpathquadraticcurveto{\pgfqpoint{0.716561in}{2.689099in}}{\pgfqpoint{0.738783in}{2.689099in}}%
\pgfpathlineto{\pgfqpoint{0.738783in}{2.689099in}}%
\pgfpathclose%
\pgfusepath{stroke,fill}%
\end{pgfscope}%
\begin{pgfscope}%
\pgfsetbuttcap%
\pgfsetmiterjoin%
\definecolor{currentfill}{rgb}{0.870588,0.560784,0.019608}%
\pgfsetfillcolor{currentfill}%
\pgfsetfillopacity{0.700000}%
\pgfsetlinewidth{0.000000pt}%
\definecolor{currentstroke}{rgb}{0.000000,0.000000,0.000000}%
\pgfsetstrokecolor{currentstroke}%
\pgfsetstrokeopacity{0.700000}%
\pgfsetdash{}{0pt}%
\pgfpathmoveto{\pgfqpoint{0.761006in}{3.066543in}}%
\pgfpathlineto{\pgfqpoint{0.983228in}{3.066543in}}%
\pgfpathlineto{\pgfqpoint{0.983228in}{3.144321in}}%
\pgfpathlineto{\pgfqpoint{0.761006in}{3.144321in}}%
\pgfpathlineto{\pgfqpoint{0.761006in}{3.066543in}}%
\pgfpathclose%
\pgfusepath{fill}%
\end{pgfscope}%
\begin{pgfscope}%
\definecolor{textcolor}{rgb}{0.000000,0.000000,0.000000}%
\pgfsetstrokecolor{textcolor}%
\pgfsetfillcolor{textcolor}%
\pgftext[x=1.072117in,y=3.066543in,left,base]{\color{textcolor}\rmfamily\fontsize{8.000000}{9.600000}\selectfont Parallel MOSFETs}%
\end{pgfscope}%
\begin{pgfscope}%
\pgfsetbuttcap%
\pgfsetmiterjoin%
\definecolor{currentfill}{rgb}{0.007843,0.619608,0.450980}%
\pgfsetfillcolor{currentfill}%
\pgfsetfillopacity{0.700000}%
\pgfsetlinewidth{0.000000pt}%
\definecolor{currentstroke}{rgb}{0.000000,0.000000,0.000000}%
\pgfsetstrokecolor{currentstroke}%
\pgfsetstrokeopacity{0.700000}%
\pgfsetdash{}{0pt}%
\pgfpathmoveto{\pgfqpoint{0.761006in}{2.911210in}}%
\pgfpathlineto{\pgfqpoint{0.983228in}{2.911210in}}%
\pgfpathlineto{\pgfqpoint{0.983228in}{2.988987in}}%
\pgfpathlineto{\pgfqpoint{0.761006in}{2.988987in}}%
\pgfpathlineto{\pgfqpoint{0.761006in}{2.911210in}}%
\pgfpathclose%
\pgfusepath{fill}%
\end{pgfscope}%
\begin{pgfscope}%
\definecolor{textcolor}{rgb}{0.000000,0.000000,0.000000}%
\pgfsetstrokecolor{textcolor}%
\pgfsetfillcolor{textcolor}%
\pgftext[x=1.072117in,y=2.911210in,left,base]{\color{textcolor}\rmfamily\fontsize{8.000000}{9.600000}\selectfont Single MOSFET}%
\end{pgfscope}%
\begin{pgfscope}%
\pgfsetbuttcap%
\pgfsetmiterjoin%
\definecolor{currentfill}{rgb}{0.003922,0.450980,0.698039}%
\pgfsetfillcolor{currentfill}%
\pgfsetfillopacity{0.700000}%
\pgfsetlinewidth{0.000000pt}%
\definecolor{currentstroke}{rgb}{0.000000,0.000000,0.000000}%
\pgfsetstrokecolor{currentstroke}%
\pgfsetstrokeopacity{0.700000}%
\pgfsetdash{}{0pt}%
\pgfpathmoveto{\pgfqpoint{0.761006in}{2.755099in}}%
\pgfpathlineto{\pgfqpoint{0.983228in}{2.755099in}}%
\pgfpathlineto{\pgfqpoint{0.983228in}{2.832877in}}%
\pgfpathlineto{\pgfqpoint{0.761006in}{2.832877in}}%
\pgfpathlineto{\pgfqpoint{0.761006in}{2.755099in}}%
\pgfpathclose%
\pgfusepath{fill}%
\end{pgfscope}%
\begin{pgfscope}%
\definecolor{textcolor}{rgb}{0.000000,0.000000,0.000000}%
\pgfsetstrokecolor{textcolor}%
\pgfsetfillcolor{textcolor}%
\pgftext[x=1.072117in,y=2.755099in,left,base]{\color{textcolor}\rmfamily\fontsize{8.000000}{9.600000}\selectfont Parallel MOSFET \(\displaystyle V_{th}-1\sigma\)}%
\end{pgfscope}%
\end{pgfpicture}%
\makeatother%
\endgroup%

    \caption{Results of a Monte Carlo simulation of the output impedance for different configurations of MOSFETs.}
    \label{fig:ltpsice_mosfet_mc_output_impedance}
\end{figure}

Unsurprisingly, there is no variance of the output impedance in the single MOSFET case in accordance to what we have learnt in appendix \ref{sec:transfer_function_transconductance}. The op-amp gain simply supresses all device properties of the MOSFET. The slight variation of $g_m$ for different samples was not simulated, because this variation stems from the variation of $\kappa$ and goes as $\frac{1}{\sqrt{\kappa}}$, so its effect is not as pronounced as the threshold.

In case of two mosfets, the output impedance varies over an order of magnitude from about \qtyrange[range-units = single]{1.8}{52}{\giga \ohm}. Even when increasing the drain-source voltage by $1 \sigma = \qty{70}{\mV}$ to \qty{625}{\mV} on average, the spread is still an order of magnitude. Only when increasing $V_{DS}$ to around \qty{700}{\mV}, the situation stabilizes, but then the net gain from this measure has shrunk to a meager \qty{84}{\mV}. We can see from this simuation, that the system-to-system spread becomes very unstable in tough situations. This instability can also be brought into the system by temperature effects as $V_{th}$ is temperature dependent as discussed above. This is a designers nightmare, because these devices are no longer interchangeable in situations of high load currents and load impedances. Additionally, they may suffer from thermal runaway if each individual MOSFET is not layed to carry the full current. As a final remark: Do not parallel MOSFETs in saturation, ever.

\clearpage
\subsection{Noise Sources}
\label{sec:current_source_noise}
The fundamentals of different types of noise were already introduced in section \ref{sec:allan_deviation}. Here, a subset of these noise types is treated. It is expected, that the dominant noise observed in this circuit is $\frac{1}{f}$-noise at low frequencies and white wideband-noise. All noise components will be converted to the so-called input referred notation to make the noise sources comparable. This can be easily understood, when looking at two amplifiers with different gain. If both of them add a fixed amount of noise to the output signal, the absolute amount of noise may be the same, but the signal to noise ratio shows a different picture. To compare these amplifiers it is useful to divide the noise by the transfer function (gain) of the amplifier. This is called input-referred noise, since it treats the noise in relation to the input signal. Additionally, when calculating noise figures, the noise bandwidth is always considered to be \qty{1}{\Hz}.

\begin{figure}[ht]
    \centering
    \import{figures/}{precision_current_source.tex}
    \caption{Transconductance amplifier with a p-channel MOSFET. Repeated from page \pageref{fig:precision_current_source}.}
    \label{fig:precision_current_source_noise}
\end{figure}

Noise sources are ubiquitous in the circuit in figure \ref{fig:precision_current_source} on page \pageref{fig:precision_current_source}, repeated here as figure \ref{fig:precision_current_source_noise} for clarity. The resistor $R_S$, the MOSFET, the op-amp, the setpoint voltage $V_{ref}$ and the supply voltage $V_{sup}$ can all contribute noise to the output current. Fortunately, some of those noise contributions are either very small or are well suppressed in this design, so each component must be briefly discussed.

Starting with the supply voltage $V_{sup}$, it can be seen, that any change of this voltage affects the string $R_S$-$Q$-$R_{load}$. From equation \ref{eqn:transconductance_amplifier_transfer_function}, we know that if the op-amp gain is high, that is, within the bandwidth of the op-amp, all disturbances of the voltage accross $R_S$ will be suppressed and the output current is only defined by the reference input and $R_S$. Looking closer, the supply noise is present at the inverting and non-inverting input of the op-amp with the same magnitude. If there is no current flowing into the op-amp pins, which is true for low frequecies, the noise is affecting both pins equally and it will be suppressed by the common-mode-rejection ratio (CMRR) of the device. Fortunately, this is a strong quality of precision op-amps and values of more than \qty[per-mode=power]{1}{\uV \per \volt} are not uncommon. The op-amp will therefore take care of the supply noise at low frequencies. At high frequencies the parasitic capacitance of the input pins and the reduced gain and CMRR come into play, reducing the CMRR and the gain also drop at high frequencies. To take care of this, it is therefore prudent to filter the supply for high frequency noise.

The next noise source is the reference voltage. The reference is directly connected to the input and its noise dictates most of the circuit noise. While the high-frequency noise can again be filtered to some extend, the low frequency noise, which is mostly $\frac{1}{f}$-noise can not be filtered as was shown in section \ref{sec:flicker_noise}, so it must be kept low from the start and the reference selected for low flicker noise.

The MOSFET as a noise source is considered in appendix \ref{sec:mosfet_noise} and the interested reader may find the derivation of the MOSFET noise within our circuit there. The two types of noise that need to be considered are the flicker noise of the MOSFET and its wideband thermal noise as calculated in equation \ref{eqn:current_noise_mosfet}
\begin{equation}
    i_{n} = \sqrt{\underbrace{4 k_B T \frac{2}{3} g_m}_{\text{thermal}} + \underbrace{\frac{K_f I_D}{C_{ox} L^2} \frac{1}{f}}_{\text{flicker}}} \,.\nonumber
\end{equation}

To calculate the input referred noise and show that the MOSFET noise will be suppressed by the op-amp, the current noise needs to be divided by the open-loop gain derived as equation \ref{eqn:transconductance_amplifier_open_loop_gain}
\begin{equation}
    e_{n,FET} = \frac{i_n}{A_f} = \frac{\sqrt{4 k_B T \frac{2}{3} g_m + \frac{K_f I_D}{C_{ox} L^2} \frac{1}{f}}}{\frac{A_{op}}{R_S} \frac{g_m \left(R_o || R_S || R_{id}\right)}{g_m \left(R_o || R_S || R_{id}\right) + 1}} \,.
\end{equation}

Looking at the parameters from table \ref{tab:current_source_parameters}, we find $\left(R_o || R_S || R_{id}\right) \approx R_S$ and $e_n$ can be simplified to
\begin{align}
    e_{n,FET} &\approx \frac{\sqrt{4 k_B T \frac{2}{3} g_m + \frac{K_f I_D}{C_{ox} L^2} \frac{1}{f}}}{A_1 \frac{1}{R_S + \frac{1}{g_m}}} \nonumber\\
    &\approx \frac{R_S + \frac{1}{g_m}}{A_1} \sqrt{4 k_B T \frac{2}{3} g_m + \frac{K_f I_D}{C_{ox} L^2} \frac{1}{f}} \nonumber\\
    \overset{A_1 \to \infty}&{=} 0\nonumber
\end{align}

Unless the MOSFET transconductance $g_m$ or the gain of the op-amp $A_1$ become very small, the noise of the MOSFET is very well suppressed. This means, that if the wideband thermal noise contribution is small (it is, see \ref{sec:mosfet_noise}) and the flicker noise corner frequency is within the bandwdith of the op-amp, the noise contribution from the MOSFET can be neglected.

The noise contribution from the sense resistor $R_S$ is the (approximated) Johnson–Nyquist noise, which when transformed to its Norton representation can be written as current noise
\begin{equation}
    i_{n,R} = \sqrt{\frac{4 k_B T}{R_S}} \,.
\end{equation}

Additionally, it was shown, that depending on the material of the resistive element, a flicker noise component can also be present. This is especially prevalent in carbon and tick-film resistors \cite{flicker_noise_carbon_film,1_f_noise_thick_film}. While thin-film resistors are less noisy, their performance varies greatly between different models \cite{resistor_current_noise_ligo}, so their make and model must be carefully selected for the application. Foil and wirewound resistors were shown to perform best and have almost no flicker noise \cite{resistor_current_noise_ligo,flicker_noise_foil_resistor_beev}. Using a high quality resistor the flicker noise can be neglected and only the thermal noise must be taken into account.

The sense resistor is part of the feedback network and therefore it contributes fully to the noise of the transimpedance amplifier. Input referred, the current noise must be divided by the closed-loop gain $A_f$ given by \ref{eqn:transfer_function_closed_loop}.
\begin{equation}
    e_{n,R} = i_{n,R} \cdot \beta \approx i_n \cdot R_S = \sqrt{4 k_B T R_S} \label{eqn:noise_sense_resistor}
\end{equation}

The final component to be discussed is the operational amplifier. Although the op-amp is a rather complex device, its noise can be modeled by a small number of noise sources. This noise model of the op-amp is shown in figure \ref{fig:op-amp_noise_model}.

\begin{figure}[ht]
    \centering
    \scalebox{1}{%
        \import{figures/}{op-amp_noise_model.tex}
    } % scalebox
    \caption{Noise model of the operational amplifier.}
    \label{fig:op-amp_noise_model}
\end{figure}

In figure \ref{fig:op-amp_noise_model} we can see, that there are three noise sources required to treat the op-amp. The input voltage noise $e_{n}$ and two input current noise sources $i_n$. The current noise noise source are assumed to be mostly uncorrelated. This assumption will lead to an upper bound as can seen from figure \ref{fig:op-amp_input_stage}, which shows the the input differential amplifier, that is the first stage of a typical bipolar op-amp.

\begin{figure}[hb]
    \centering
    \scalebox{1}{%
        \import{figures/}{op-amp_input_stage.tex}
    } % scalebox
    \caption{Bipolar op-amp input stage with noise sources.}
    \label{fig:op-amp_input_stage}
\end{figure}

Of the three noise sources $i_{n,p}$ and $i_{n,n}$ are uncorrelated, because it is the input bias current of the individual transistors, and only the effect of $i_{n,EE}$ is correlated, because the current of the emitter bias current souce is equally distributed between the two input transistors. Since effects of equal magnitude and sign cancel out due to the differential nature of the input stage, correlated effects are suppressed. An equal magnitude can be assumed, because the gain of the two transistors is well matched, due to their close proximity on the semiconductor die. Therefore assuming all noise is uncorrelated, presents an upper bound. A more detailed analysis can be found in \cite{op-amp_noise_correlation}, if interested. Due to the matching of the transistors, the magnitude $i_{n,p}$ and $i_{n,n}$ are also closely matched, hence in our model, they are assumed equal.

As we have done before in this section with referring all noise sources to the input, the same was done manufacturer and they are given in the datasheet. The two current noise sources can not be combined into the input voltage noise, because the depend on the external impedances connected to the op-amp. Given the complete circuit as in figure \ref{fig:precision_current_source_noise} it is possible to calculate the full noise contribution of the op-amp
\begin{equation}
    e_{n,op} = \sqrt{e_n^2 + e_{n,+}^2 + e_{n,-}^2}
\end{equation}
given that the noise sources are uncorrelated. The input referred noise $e_{n,-}$ of the inverting input can be calculated in a similar fashion as $e_{n_R}$ in equation \ref{eqn:noise_sense_resistor}. It is likewise part of the feedback network and must therefore be divided by the closed-loop gain $A_f$ as before.
\begin{equation}
    e_{n,-} \approx i_n \cdot R_S
\end{equation}

The current noise of the input can be translated by looking at the input impedance. This will be determined by the output filter of the reference voltage, which is required to remove the high frequency noise as discussed above. Assuming an RC-filter of first order,
the output impedance can be calculated from the transfer function of the low-pass filter, derived in equation \ref{eqn:first_order_model}
\begin{align}
    R_{out,filt} &= R_{filt} \cdot A = \frac{R_{filt}}{1+sR_{filt}C} \nonumber\\
    \lim_{s \to 0} R_{out,filt} &= R_{filt} \nonumber\\
    \lim_{s \to \infty} R_{out,filt} &= 0 \,.\nonumber\\
     e_{n,-} &\approx \frac{i_n R_{filt}}{1+sR_{filt}C}
\end{align}

As it can be seen, for high frequecies, the output impedance goes to \num{0}, while for low frequencies it is $R_{filt}$. If the filter corner frequency $\omega_0 = \frac{1}{RC}$ is close to or at the flicker noise corner frequency of the reference voltage, it means that there is almost no wideband current noise contribution as well. The only the $\frac{1}{f}$ component of the op-amp current noise multiplied with $R_{filt}$ should be lower than the reference noise to have negligible impact.
This leads to the total noise of the op amp
\begin{equation}
    e_{n,op} = \sqrt{e_n^2 + (i_n R_S)^2 + \left|\frac{i_n R_{filt}}{1+sRC}\right|^2} \,.
\end{equation}

To conclude, table \ref{tab:current_source_noise_contributers} is given as a reference for the noise contributions in the low-frequency and also the wideband domain. From this table, it can be seen, that the only wideband-noise contributors are the reference resistor and the op-amp. The low-frequency contributors are the voltage reference and the op-amp, since they have a strong flicker noise component. A low-noise, precision op-amp typically has far less low frequency noise than a voltage reference and the dominant low frequency contributor is the voltage reference.

\begin{table}[ht]
    \centering
    \begin{tabular}{lll}
        Noise component& Low frequency& Wideband \\
        \midrule
        $V_{sup}$ & $\approx 0$ & $\approx 0$\\
        MOSFET & $\approx 0$ & $\approx 0$\\
        $V_{ref}$ & $\sqrt{e_{n,ref}^2 + 4 k_B T R_{filt}} $ & $\approx 0$\\
        $R_S$ & $\sqrt{4 k_B T R_S}$ & $\sqrt{4 k_B T R_S}$\\
        Op-amp & $\sqrt{e_n^2 + i_n^2 (R_S^2 + R_{filt}^2)}$ & $\sqrt{e_n^2 + i_n^2 R_S^2}$
    \end{tabular}
    \caption{Input referred noise components of the transimpedance amplifier. Multiply by $\frac{1}{R_S}$ to get the output referred current noise.}
    \label{tab:current_source_noise_contributers}
\end{table}

\clearpage
\subsection{Component Selection}
\label{sec:component_selection}
%TODO: Fix link to lst:dgDrive_specs_environment and lst:dgDrive_specs_electrical
This section deals with selecting the right components for the precision current source presented in section \ref{sec:precision_current_source}. The focus lies on the requirements defined in section \ref{sec:laser_current_driver}, notably tables \ref{lst:dgDrive_specs_environment} and \ref{lst:dgDrive_specs_electrical}. Most attention will be on the MOSFET, the operational amplifier and the voltage reference. We will start with the voltage reference, because this will define several parameters down the road. Then, the op-amp is discussed, for which several examples from scientific publications and other alternatives are shown and the best solution is presented. Finally, the selection parameters for the MOSFET will be elaborated. The reader must warned though, that the lineup of p-channel MOSFETs in production is decreasing, with more and more products being discontinued in favor of n-channel MOSFETs and the examples may be outdated.

Numerous laser driver designs can be found in literature \cite{libbrecht_hall, laser_driver_mosfet_noise, laser_driver_digital, laser_driver_digital_update, laser_driver_qcl_space, laser_driver_qcl_taubman, laser_driver_qcl_taubman_multiplexer}. While \citeauthor{libbrecht_hall} where not first to present a design and similar design can already be found in \cite{laser_driver_old}, their design stands out for its simplicity. The design in literature can be divided into two groups. High power drivers for quantum cascade lasers (QCL) typically featuring a compliance voltage of more than \qty{10}{\V} and output currents of up to serveral ampere based on the work of \citeauthor{laser_driver_qcl_taubman} and medium power devices for laser diodes having a lower compliance voltage of around \qty{2}{\V} and able to driver a few hundred \unit{\mA} based on the work of \citeauthor{libbrecht_hall}. Our requirements mostly fall into the latter category, except for the compliance voltage, which is targeted to be $\qty{\ge 8}{\V}$. All these drivers share one common aspect, though, the type of voltage reference. Most laser drivers in literature and commercial products are designed around low-noise, low-drift buried Zener diode voltage references, namely the \device{LM399} \cite{datasheet_LM399} or \device{LTZ1000} \cite{datasheet_LTZ1000}.

\begin{table}[ht]
    \centering
    \begin{tabular}{lllll}
        Component& Voltage& Temperature coefficent & Stability& Package \\
        \midrule
        \device{LT1021} & \qty{7}{\V} & \qtyrange[range-units = single]{2}{5}{\uV \per \V \per \K} & \qty[per-mode = symbol, power-half-as-sqrt]{15}{\uV \per \V \per \kilo\hour\tothe{0.5}} & SO-8\\
        \device{LT1027} & \qty{5}{\V} & \qtyrange[range-units = single]{1}{2}{\uV \per \V \per \K} & not specified & SO-8\\
        \device{LM399} & \qty{7}{\V} & \qtyrange[range-units = single]{0.3}{1}{\uV \per \V \per \K} & \qty[power-half-as-sqrt]{8}{\uV \per \V \per \kilo\hour\tothe{0.5}} & TO-46\\
        \device{ADR1399} & \qty{7}{\V} & \qtyrange[range-units = single]{0.2}{1}{\uV \per \V \per \K} & \qty[power-half-as-sqrt]{7}{\uV \per \V \per \kilo\hour\tothe{0.5}} & TO-46\\
        \device{LTZ1000} & \qty{7.2}{\V} & \qty{0.05}{\uV \per \V \per \K} & \qty[power-half-as-sqrt]{0.3}{\uV \per \V \per \kilo\hour\tothe{0.5}} & TO-99\\
        \device{ADR1000} & \qty{6.6}{\V} & \qty{<0.2}{\uV \per \V \per \K} & \qty[power-half-as-sqrt]{0.2}{\uV \per \V \per \kilo\hour\tothe{0.5}} & TO-99
    \end{tabular}
    \caption{List of buried Zener diodes and selected properties.}
    \label{tab:overview_buried_zener_diodes}
\end{table}

The buried types of voltage references are \textit{Zener} diodes, that are created within the bulk silicon using ion implantation. This reduces noise due to surface contamination \cite{zener_diode_stability}. These diodes are not true Zener diodes, but called Zeners nonetheless and use a mix of Zener and avalanche breakdown to compensate the temperature coefficient. The Zener effect is the tunneling of electrons through the barier from the valence band to conduction band. It  has a negative temperature coefficient, because in increase in temperature reduces the size of the bandgap. Avalanche breakdown, on the other hand describes the mechanism, that free electrons (due to temperature) are accelerated to such energies, that they knock out other electrons, causing a avalanche of electrons. This effect has a positive temperature coefficient, because a higher temperature results in more free carriers, that cause leakage but no avalanche. While the zero temperature coefficient point is around \qty{5}{\V}, this operating point implies a high succeptibility to changes in the reverse current. So typically the Zener voltage is shifted slightly upwards to result in a net positive coefficent, which is then compensated by the negative temperature coefficent of a forward biased diode \cite{zener_diode_stability}. This results in the typical Zener diode voltage of around $\qty{6.2}{\V} + \qty{0.7}{\V} = \qty{6.9}{\V}$. In comparison to other types of diodes, buried Zeners have the best stability and lowest noise. In order to achieve high stability and low noise, $V_{ref} \approx \qty{7}{\V}$ is therefore pretty much set in stone. Table \ref{tab:overview_buried_zener_diodes}, lists some commercially available buried Zener diodes. All diodes are manufactured by Analog Devices as they are the sole manufacturer left to produce these kind of diodes.

Choosing a voltage reference can be done according to table \ref{lst:dgDrive_specs_environment}. A temperature coefficent of \qty{<= 1}{\uA \per \A \per \K} rules out any non-hermetic unheated voltage reference. Using a hermetic package improves the stability against humidity as the epoxy used for an SO-8 package is hydrophilic and swells when exposed to water vapour causing pressure on the die, resulting in a change of the output voltage. The hermetic voltage references can be divided into two groups, the \device{LM399} and the newer \device{ADR1399} in one group and the \device{LTZ1000} and its newer counterpart \device{ADR1000} in aother. While the \device{LM399} requires very few external components, the external circuit for the \device{LTZ1000} is far more elaborate requiring more parts and space. Additionally, the \device{LTZ1000} is more than four times the price of the \device{LM399} in quantities of \num{10} at the time of writing. Last but not least, the stability and temperature coefficent of the \device{LTZ1000} cannot be matched by the performance of the sense resistor, so the sense resistor gives a lower bound of about \qty{0.5}{\uA \per \A \per \K}. Unless the better low frequency noise performance is absolutely required, the \device{LM399} and \device{ADR1399} are the more economical parts. The performance of those two references will be discussed in section \ref{sec:zener_diode_selection}.

With the maximum reference voltage of \qty{7}{\V} known, a sense resitor between \qty{14}{\ohm} (\qty{500}{\mA}) and \qty{28}{\ohm} (\qty{250}{\mA} is required. Combined with the requirement for a low noise output, this limits the choice of op-amps to bipolar low-noise devices or discrete implementations. Table \ref{tab:overview_bipolar_op-amps} lists some choices compiled from the literature sources, which will now be discussed.

\begin{table}[ht]
    \centering
    \begin{tabular}{llll}
        Component& Wideband-noise& Low frequency noise & Temperature coefficent \\
        \midrule
        \device{LT1028} & \qty[power-half-as-sqrt]{0.85}{\nV \per \Hz\tothe{0.5}} & \qty{35}{\nV_{p-p}} & \qty{0.2}{\uV \per \K}\\
        \device{AD797} & \qty[power-half-as-sqrt]{0.9}{\nV \per \Hz\tothe{0.5}} & \qty{50}{\nV_{p-p}} & \qty{0.2}{\uV \per \K}\\
        \device{ADA4898} & \qty[power-half-as-sqrt]{0.9}{\nV \per \Hz\tothe{0.5}} & not specified & \qty{1}{\uV \per \K}\\
        \device{ADA4004} & \qty[power-half-as-sqrt]{1.8}{\nV \per \Hz\tothe{0.5}} & \qty{150}{\nV_{p-p}} & \qty{0.7}{\uV \per \K}\\
        \device{AD8671} & \qty[power-half-as-sqrt]{2.8}{\nV \per \Hz\tothe{0.5}} & \qty{77}{\nV_{p-p}} & \qty{0.3}{\uV \per \K}
    \end{tabular}
    \caption{List of low-noise precision bipolar operational amplifiers with typical performance properties.}
    \label{tab:overview_bipolar_op-amps}
\end{table}

The low value of the sense resistor makes a bipolar op-amp the preferred choice, because they have a very low voltage noise and their current noise and input bias current do not interfere with such a low value resistor. While a discrete solution using matched jfets or bipolar transistors may push the input noise even lower, the temperature stability, circuit complexity and again the size speaks against this option, so the disussion will be limited to integrated solutions only. To find a reference point for the choice of op-amp, the thermal noise of the sense resistor must looked at. The \qty{28}{\ohm} sense resistor has a thermal noise of
\begin{equation}
    e_n\left(\qty{23}{\celsius}\right) = \qty[power-half-as-sqrt]{0.67}{\nV \per \Hz\tothe{0.5}} \,. \nonumber
\end{equation}

This means, that even the lowest noise op-amp from table \ref{tab:overview_bipolar_op-amps} dominates the wideband-noise. The \device{AD8671} chosen by \cite{laser_driver_digital} only makes sense, because they have chosen a very large filter resistor $R_{filt}$ of $2 \times \qty{3}{\kilo\ohm}$. The \device{ADA4004} was used by Moglabs in the \device{DLC-202}, again likely due to the high values of $R_{filt}$ used. The \device{ADA4898} might seem like a good choice at first sight but the very limited (in terms of precision op-amps) open-loop gain of \qty{0.14}{\V \per \uV} makes this op-amp a cheap, but poor choice. The final choice is between the \device{AD797} and the \device{LT1028}, both op-amps have very similar specifications, but there is a peculiarity in the datasheet of the \device{LT1028} \cite{datasheet_LT1028}. While there is a current noise spectrum, there is no voltage noise spectrum to be found in the datasheet. The author assumes a good deal of specsmanship at this point. The publication by \citeauthor{libbrecht_hall} already blames the \device{LT1028} for a noise peak around \qty{400}{\kHz}. This peak is also included in the noise models for the op-amp and was additionally confirmed by the author with a measurement. This peak is the reason why \citeauthor{laser_driver_mosfet_noise} found the AD797 to be higher noise than the \device{AD797}. Additionally to the superior noise performance, the \device{AD797B} has excellent specifications overall. The open-loop gain is between \qtyrange[range-units = single]{2}{20}{\V \per \uV}, the supply rejection is greater than \qty{1}{\uV \per \V}, the bias current is almost constant between \qtyrange[range-units = single]{20}{100}{\celsius} and the unity gain bandwidth is around \qty{10}{\MHz}. Finally it does have a very high output drive capability of \qty{50}{\mA}, which allows to drive fairly large MOSFETs. These features make the \device{AD797} the ideal op-amp for those low-value sense resistors, although it puts limits on the maximum filter resistor to limit the low frequency current noise contribution.

Finally, the choice of MOSFETs can be discussed. As it was shown in section \ref{sec:mosfet_current_source} in equation \ref{eqn:mosfet_saturation}, the channel length modulation playes an important role in increasing the channel conductance $g_{DS}$ and limiting the output impedance. To reduce the channel length modulation a longer channel is preferred. Manufacturers do not give these numbers, nor the manufacturing process. Older technologies like the planar (lateral) FET is better suited for operating in the saturation region than the modern trench (vertical) FET. Trench MOSFETs are geared towards a low on-state resistance $R_{DS,on}$, which is important for MOSFETs in switching applications, but their lower resistance comes from a shorter channel. One of the few planar MOSFETs still available on the market is the HEXFET, which was designed for switching applications, but prooves useful nonetheless as we will see. High voltage MOSFETs also have longer channels than low voltage MOSFETs, so browsing for MOSFETs, that are rated for \qtyrange[range-units = single]{60}{100}{\V} or more can narrow down the candidates. While the output impedance is a factor worth keeping in mind, the most important aspect is, whether the MOSFET can drive the load regarding the compliance voltage. To outline the problem, we can again refer to the example parameters from table \ref{tab:current_source_parameters}.

Assuming a supply voltage of \qty{15}{\V} and the \device{AD797} op-amp, the current source supply voltage $V_{sup}$ is then limited to about \qtyrange[range-units = single]{11}{12}{\V}, because the \device{AD797} is no rail-to-rail op-amp and its output only swings to within \qty{3}{\V} of the rail (minimum) and the input is limited to within \qty{2}{\V} of the rail (minimum). Considering the maximum $V_{ref}$ at full output of \qty{7}{\V} and a load voltage of \qty{3}{\V} in case of the \device{L785H1} \cite{datasheet_thorlabs_780nm} used as an example in this section leaves only
\begin{equation}
    V_{DS,min} = V_{sup} - V_{ref} - V_{load} = \qtyrange[range-units = bracket]{11}{12}{\V} - \qty{7}{\V} - \qty{3}{\V} = \qtyrange[range-units = bracket]{1}{2}{\V} \label{eqn:minimum_mosfet_vds}
\end{equation}
for the MOSFET -- a serious challenge.

To find a suitable MOSFET, one has to consult the \textit{Typical Output Characteristics} graph in the datasheet. Using the maxium output current specifaction it is possible to estimate the minimum drain-source voltage $V_{DS}$ to keep the MOSFET in saturation at the given maxium output current. This again narrows down the list of candidates.

The final aspect is the capacitive nature of the MOSFET gate. This property was brushed in appendix \ref{sec:mosfet_noise} and the parasitic capacitances can be found in figure \ref{fig:mosfet_parasitic_capacitors}. The \device{AD797} can drive fairly large capacitive loads and several hundred \unit{\pF} are possible. It is best to keep the input capacitance $C_{iss}$ below \qty{500}{\pF}. Do remember the output impedance of the \device{AD797}, is about \qty{10}{\ohm} at \qty{1}{\MHz} and rising by an order of magnitude at \qty{10}{\MHz}. The \qty{500}{\pF} results in an impedance of around \qty{300}{\ohm} dropping by an order of magnitude at \qty{10}{\MHz}, so keeping capacitance low, allows a higher bandwidth of the current source.

Using these guidelines, searching a MOSFET across a lot of manufactures can still be tedious, but, for example, the distributor Digikey allows filtering and sorting by voltage and input capacitance. The follwing MOSFETs in table are given as an example and can be chosen for their respective current ranges.

\begin{table}[ht]
    \centering
    \begin{tabular}{llll}
        MOSFET& Maximum $V_{DS}$& Input capacitance $C_{iss}$ & Current range \\
        \midrule
        \device{IRF9610} & \qty{200}{\V}& \qty{170}{\V} & \qtyrange[range-units = single]{100}{250}{\mA}\\
        \device{IRF9Z10} & \qty{50}{\V}& \qty{270}{\V} & \qtyrange[range-units = single]{250}{500}{\mA}\\
        \device{IRF9Z14} & \qty{60}{\V}& \qty{270}{\V} & \qtyrange[range-units = single]{250}{500}{\mA}
    \end{tabular}
    \caption{Example MOSFETs for a current source and recommended current ranges.}
    \label{tab:example_mosfet_selection}
\end{table}

The current range of the MOSFETs in table \ref{tab:example_mosfet_selection} is given based on the datasheet, making sure, that the MOSFET can be biased into saturation for the estimated minimum $V_{DS}$ according to \ref{eqn:minimum_mosfet_vds}. The \device{IRF9Z10} is a lower voltage version of the \device{IRF9Z14} and the \device{IRF9Z14} should be preferred if available. Those MOSFETs starting with \textit{IRF} are all HEXFETs formerly made by International Recitfier, whose MOSFET business was bought by Vishay in 2007.

\clearpage
\subsection{Current Source Example Parameters}
\label{sec:current_source_summary}
Throughout this section, example calculations are performed to give the reader an idea of real-life parameters derived from the theoretical models. These parameters are summarized in table \ref{tab:current_source_parameters}, including their origin.

\begin{table}[ht]
    \centering
    \begin{tabular}{lll}
        Parameter& Value& Source \\
        \midrule
        MOSFET drain current $I_D$ & \qty{250}{\mA} & \device{L785H1} \cite{datasheet_thorlabs_780nm}\\
        MOSFET $\kappa$ & \qty[per-mode=power]{0.813}{\ampere \per \square\volt} & \device{IRF9610} SPICE model \cite{irf9610_spice}\\
        MOSFET channel length modulation $\lambda$ & \qty[per-mode=power]{4}{\per \milli \volt} & \device{IRF9610} SPICE model \cite{irf9610_spice}\\
        MOSFET source voltage & \qtyrange{3.5}{4}{\V} & section \ref{}\\
        Source/Sense Resistor $R_S$ & \qty{30}{\ohm} or \qty{50}{\ohm} & section \ref{}\\
        Op-amp differential input impedance $R_{id}$ & \qty{7.5}{\kilo\ohm} & \device{AD797} \cite{datasheet_AD797}\\
        Op-amp open-loop gain $A_{ol}$ & \qty[per-mode=power]{2}{\volt \per \uV} & \device{AD797} \cite{datasheet_AD797}\\
        Op-amp gain bandwidth product $GBP$ & \qty{10}{\MHz} & \device{AD797} \cite{datasheet_AD797}
    \end{tabular}
    \caption{Parameters used throughout this section and their sources.}
    \label{tab:current_source_parameters}
\end{table}

\clearpage
\section{Temperature Controller}
\label{sec:temperature_controller}

% include Investigation of Long-Term Drift of NTC Temperature Sensors with less than 1 mK Uncertainty

\subsection{Tuning of a PID controller}
The number of empirical algorithms to determine a set of PID parameters ($\mathrm{k_p, k_i, k_d}$) are numerous. In this work only the most common algorithms and a few notable exceptions will be presented.
\subsection{Design}

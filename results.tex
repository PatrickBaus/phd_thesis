\chapter{Results}
\section{Laser Current Driver}
\subsection{Zener Diode Selection}
Early tests of the LM399 Zener diode as a reference have confirmed, what the data sheet \cite{datasheet_LM399} already suggest in the 'Low Frequency Noise Voltage' plot. There are random bi-stable voltage step changes. This phenomenon is called burst noise or popcorn noise.

\begin{figure}[ht]
    \centering
    %% Creator: Matplotlib, PGF backend
%%
%% To include the figure in your LaTeX document, write
%%   \input{<filename>.pgf}
%%
%% Make sure the required packages are loaded in your preamble
%%   \usepackage{pgf}
%%
%% Also ensure that all the required font packages are loaded; for instance,
%% the lmodern package is sometimes necessary when using math font.
%%   \usepackage{lmodern}
%%
%% Figures using additional raster images can only be included by \input if
%% they are in the same directory as the main LaTeX file. For loading figures
%% from other directories you can use the `import` package
%%   \usepackage{import}
%%
%% and then include the figures with
%%   \import{<path to file>}{<filename>.pgf}
%%
%% Matplotlib used the following preamble
%%   \usepackage{fontspec}
%%
\begingroup%
\makeatletter%
\begin{pgfpicture}%
\pgfpathrectangle{\pgfpointorigin}{\pgfqpoint{5.200000in}{3.210000in}}%
\pgfusepath{use as bounding box, clip}%
\begin{pgfscope}%
\pgfsetbuttcap%
\pgfsetmiterjoin%
\definecolor{currentfill}{rgb}{1.000000,1.000000,1.000000}%
\pgfsetfillcolor{currentfill}%
\pgfsetlinewidth{0.000000pt}%
\definecolor{currentstroke}{rgb}{1.000000,1.000000,1.000000}%
\pgfsetstrokecolor{currentstroke}%
\pgfsetdash{}{0pt}%
\pgfpathmoveto{\pgfqpoint{0.000000in}{0.000000in}}%
\pgfpathlineto{\pgfqpoint{5.200000in}{0.000000in}}%
\pgfpathlineto{\pgfqpoint{5.200000in}{3.210000in}}%
\pgfpathlineto{\pgfqpoint{0.000000in}{3.210000in}}%
\pgfpathlineto{\pgfqpoint{0.000000in}{0.000000in}}%
\pgfpathclose%
\pgfusepath{fill}%
\end{pgfscope}%
\begin{pgfscope}%
\pgfsetbuttcap%
\pgfsetmiterjoin%
\definecolor{currentfill}{rgb}{1.000000,1.000000,1.000000}%
\pgfsetfillcolor{currentfill}%
\pgfsetlinewidth{0.000000pt}%
\definecolor{currentstroke}{rgb}{0.000000,0.000000,0.000000}%
\pgfsetstrokecolor{currentstroke}%
\pgfsetstrokeopacity{0.000000}%
\pgfsetdash{}{0pt}%
\pgfpathmoveto{\pgfqpoint{0.483776in}{2.351653in}}%
\pgfpathlineto{\pgfqpoint{5.050249in}{2.351653in}}%
\pgfpathlineto{\pgfqpoint{5.050249in}{2.936535in}}%
\pgfpathlineto{\pgfqpoint{0.483776in}{2.936535in}}%
\pgfpathlineto{\pgfqpoint{0.483776in}{2.351653in}}%
\pgfpathclose%
\pgfusepath{fill}%
\end{pgfscope}%
\begin{pgfscope}%
\pgfsetbuttcap%
\pgfsetroundjoin%
\definecolor{currentfill}{rgb}{0.000000,0.000000,0.000000}%
\pgfsetfillcolor{currentfill}%
\pgfsetlinewidth{0.803000pt}%
\definecolor{currentstroke}{rgb}{0.000000,0.000000,0.000000}%
\pgfsetstrokecolor{currentstroke}%
\pgfsetdash{}{0pt}%
\pgfsys@defobject{currentmarker}{\pgfqpoint{0.000000in}{-0.048611in}}{\pgfqpoint{0.000000in}{0.000000in}}{%
\pgfpathmoveto{\pgfqpoint{0.000000in}{0.000000in}}%
\pgfpathlineto{\pgfqpoint{0.000000in}{-0.048611in}}%
\pgfusepath{stroke,fill}%
}%
\begin{pgfscope}%
\pgfsys@transformshift{0.691021in}{2.351653in}%
\pgfsys@useobject{currentmarker}{}%
\end{pgfscope}%
\end{pgfscope}%
\begin{pgfscope}%
\pgfsetbuttcap%
\pgfsetroundjoin%
\definecolor{currentfill}{rgb}{0.000000,0.000000,0.000000}%
\pgfsetfillcolor{currentfill}%
\pgfsetlinewidth{0.803000pt}%
\definecolor{currentstroke}{rgb}{0.000000,0.000000,0.000000}%
\pgfsetstrokecolor{currentstroke}%
\pgfsetdash{}{0pt}%
\pgfsys@defobject{currentmarker}{\pgfqpoint{0.000000in}{-0.048611in}}{\pgfqpoint{0.000000in}{0.000000in}}{%
\pgfpathmoveto{\pgfqpoint{0.000000in}{0.000000in}}%
\pgfpathlineto{\pgfqpoint{0.000000in}{-0.048611in}}%
\pgfusepath{stroke,fill}%
}%
\begin{pgfscope}%
\pgfsys@transformshift{1.210067in}{2.351653in}%
\pgfsys@useobject{currentmarker}{}%
\end{pgfscope}%
\end{pgfscope}%
\begin{pgfscope}%
\pgfsetbuttcap%
\pgfsetroundjoin%
\definecolor{currentfill}{rgb}{0.000000,0.000000,0.000000}%
\pgfsetfillcolor{currentfill}%
\pgfsetlinewidth{0.803000pt}%
\definecolor{currentstroke}{rgb}{0.000000,0.000000,0.000000}%
\pgfsetstrokecolor{currentstroke}%
\pgfsetdash{}{0pt}%
\pgfsys@defobject{currentmarker}{\pgfqpoint{0.000000in}{-0.048611in}}{\pgfqpoint{0.000000in}{0.000000in}}{%
\pgfpathmoveto{\pgfqpoint{0.000000in}{0.000000in}}%
\pgfpathlineto{\pgfqpoint{0.000000in}{-0.048611in}}%
\pgfusepath{stroke,fill}%
}%
\begin{pgfscope}%
\pgfsys@transformshift{1.729114in}{2.351653in}%
\pgfsys@useobject{currentmarker}{}%
\end{pgfscope}%
\end{pgfscope}%
\begin{pgfscope}%
\pgfsetbuttcap%
\pgfsetroundjoin%
\definecolor{currentfill}{rgb}{0.000000,0.000000,0.000000}%
\pgfsetfillcolor{currentfill}%
\pgfsetlinewidth{0.803000pt}%
\definecolor{currentstroke}{rgb}{0.000000,0.000000,0.000000}%
\pgfsetstrokecolor{currentstroke}%
\pgfsetdash{}{0pt}%
\pgfsys@defobject{currentmarker}{\pgfqpoint{0.000000in}{-0.048611in}}{\pgfqpoint{0.000000in}{0.000000in}}{%
\pgfpathmoveto{\pgfqpoint{0.000000in}{0.000000in}}%
\pgfpathlineto{\pgfqpoint{0.000000in}{-0.048611in}}%
\pgfusepath{stroke,fill}%
}%
\begin{pgfscope}%
\pgfsys@transformshift{2.248160in}{2.351653in}%
\pgfsys@useobject{currentmarker}{}%
\end{pgfscope}%
\end{pgfscope}%
\begin{pgfscope}%
\pgfsetbuttcap%
\pgfsetroundjoin%
\definecolor{currentfill}{rgb}{0.000000,0.000000,0.000000}%
\pgfsetfillcolor{currentfill}%
\pgfsetlinewidth{0.803000pt}%
\definecolor{currentstroke}{rgb}{0.000000,0.000000,0.000000}%
\pgfsetstrokecolor{currentstroke}%
\pgfsetdash{}{0pt}%
\pgfsys@defobject{currentmarker}{\pgfqpoint{0.000000in}{-0.048611in}}{\pgfqpoint{0.000000in}{0.000000in}}{%
\pgfpathmoveto{\pgfqpoint{0.000000in}{0.000000in}}%
\pgfpathlineto{\pgfqpoint{0.000000in}{-0.048611in}}%
\pgfusepath{stroke,fill}%
}%
\begin{pgfscope}%
\pgfsys@transformshift{2.767206in}{2.351653in}%
\pgfsys@useobject{currentmarker}{}%
\end{pgfscope}%
\end{pgfscope}%
\begin{pgfscope}%
\pgfsetbuttcap%
\pgfsetroundjoin%
\definecolor{currentfill}{rgb}{0.000000,0.000000,0.000000}%
\pgfsetfillcolor{currentfill}%
\pgfsetlinewidth{0.803000pt}%
\definecolor{currentstroke}{rgb}{0.000000,0.000000,0.000000}%
\pgfsetstrokecolor{currentstroke}%
\pgfsetdash{}{0pt}%
\pgfsys@defobject{currentmarker}{\pgfqpoint{0.000000in}{-0.048611in}}{\pgfqpoint{0.000000in}{0.000000in}}{%
\pgfpathmoveto{\pgfqpoint{0.000000in}{0.000000in}}%
\pgfpathlineto{\pgfqpoint{0.000000in}{-0.048611in}}%
\pgfusepath{stroke,fill}%
}%
\begin{pgfscope}%
\pgfsys@transformshift{3.286252in}{2.351653in}%
\pgfsys@useobject{currentmarker}{}%
\end{pgfscope}%
\end{pgfscope}%
\begin{pgfscope}%
\pgfsetbuttcap%
\pgfsetroundjoin%
\definecolor{currentfill}{rgb}{0.000000,0.000000,0.000000}%
\pgfsetfillcolor{currentfill}%
\pgfsetlinewidth{0.803000pt}%
\definecolor{currentstroke}{rgb}{0.000000,0.000000,0.000000}%
\pgfsetstrokecolor{currentstroke}%
\pgfsetdash{}{0pt}%
\pgfsys@defobject{currentmarker}{\pgfqpoint{0.000000in}{-0.048611in}}{\pgfqpoint{0.000000in}{0.000000in}}{%
\pgfpathmoveto{\pgfqpoint{0.000000in}{0.000000in}}%
\pgfpathlineto{\pgfqpoint{0.000000in}{-0.048611in}}%
\pgfusepath{stroke,fill}%
}%
\begin{pgfscope}%
\pgfsys@transformshift{3.805298in}{2.351653in}%
\pgfsys@useobject{currentmarker}{}%
\end{pgfscope}%
\end{pgfscope}%
\begin{pgfscope}%
\pgfsetbuttcap%
\pgfsetroundjoin%
\definecolor{currentfill}{rgb}{0.000000,0.000000,0.000000}%
\pgfsetfillcolor{currentfill}%
\pgfsetlinewidth{0.803000pt}%
\definecolor{currentstroke}{rgb}{0.000000,0.000000,0.000000}%
\pgfsetstrokecolor{currentstroke}%
\pgfsetdash{}{0pt}%
\pgfsys@defobject{currentmarker}{\pgfqpoint{0.000000in}{-0.048611in}}{\pgfqpoint{0.000000in}{0.000000in}}{%
\pgfpathmoveto{\pgfqpoint{0.000000in}{0.000000in}}%
\pgfpathlineto{\pgfqpoint{0.000000in}{-0.048611in}}%
\pgfusepath{stroke,fill}%
}%
\begin{pgfscope}%
\pgfsys@transformshift{4.324344in}{2.351653in}%
\pgfsys@useobject{currentmarker}{}%
\end{pgfscope}%
\end{pgfscope}%
\begin{pgfscope}%
\pgfsetbuttcap%
\pgfsetroundjoin%
\definecolor{currentfill}{rgb}{0.000000,0.000000,0.000000}%
\pgfsetfillcolor{currentfill}%
\pgfsetlinewidth{0.803000pt}%
\definecolor{currentstroke}{rgb}{0.000000,0.000000,0.000000}%
\pgfsetstrokecolor{currentstroke}%
\pgfsetdash{}{0pt}%
\pgfsys@defobject{currentmarker}{\pgfqpoint{0.000000in}{-0.048611in}}{\pgfqpoint{0.000000in}{0.000000in}}{%
\pgfpathmoveto{\pgfqpoint{0.000000in}{0.000000in}}%
\pgfpathlineto{\pgfqpoint{0.000000in}{-0.048611in}}%
\pgfusepath{stroke,fill}%
}%
\begin{pgfscope}%
\pgfsys@transformshift{4.843390in}{2.351653in}%
\pgfsys@useobject{currentmarker}{}%
\end{pgfscope}%
\end{pgfscope}%
\begin{pgfscope}%
\pgfsetbuttcap%
\pgfsetroundjoin%
\definecolor{currentfill}{rgb}{0.000000,0.000000,0.000000}%
\pgfsetfillcolor{currentfill}%
\pgfsetlinewidth{0.803000pt}%
\definecolor{currentstroke}{rgb}{0.000000,0.000000,0.000000}%
\pgfsetstrokecolor{currentstroke}%
\pgfsetdash{}{0pt}%
\pgfsys@defobject{currentmarker}{\pgfqpoint{-0.048611in}{0.000000in}}{\pgfqpoint{-0.000000in}{0.000000in}}{%
\pgfpathmoveto{\pgfqpoint{-0.000000in}{0.000000in}}%
\pgfpathlineto{\pgfqpoint{-0.048611in}{0.000000in}}%
\pgfusepath{stroke,fill}%
}%
\begin{pgfscope}%
\pgfsys@transformshift{0.483776in}{2.532831in}%
\pgfsys@useobject{currentmarker}{}%
\end{pgfscope}%
\end{pgfscope}%
\begin{pgfscope}%
\definecolor{textcolor}{rgb}{0.000000,0.000000,0.000000}%
\pgfsetstrokecolor{textcolor}%
\pgfsetfillcolor{textcolor}%
\pgftext[x=0.327525in, y=2.494275in, left, base]{\color{textcolor}\rmfamily\fontsize{8.000000}{9.600000}\selectfont \(\displaystyle {0}\)}%
\end{pgfscope}%
\begin{pgfscope}%
\pgfsetbuttcap%
\pgfsetroundjoin%
\definecolor{currentfill}{rgb}{0.000000,0.000000,0.000000}%
\pgfsetfillcolor{currentfill}%
\pgfsetlinewidth{0.803000pt}%
\definecolor{currentstroke}{rgb}{0.000000,0.000000,0.000000}%
\pgfsetstrokecolor{currentstroke}%
\pgfsetdash{}{0pt}%
\pgfsys@defobject{currentmarker}{\pgfqpoint{-0.048611in}{0.000000in}}{\pgfqpoint{-0.000000in}{0.000000in}}{%
\pgfpathmoveto{\pgfqpoint{-0.000000in}{0.000000in}}%
\pgfpathlineto{\pgfqpoint{-0.048611in}{0.000000in}}%
\pgfusepath{stroke,fill}%
}%
\begin{pgfscope}%
\pgfsys@transformshift{0.483776in}{2.740006in}%
\pgfsys@useobject{currentmarker}{}%
\end{pgfscope}%
\end{pgfscope}%
\begin{pgfscope}%
\definecolor{textcolor}{rgb}{0.000000,0.000000,0.000000}%
\pgfsetstrokecolor{textcolor}%
\pgfsetfillcolor{textcolor}%
\pgftext[x=0.327525in, y=2.701450in, left, base]{\color{textcolor}\rmfamily\fontsize{8.000000}{9.600000}\selectfont \(\displaystyle {5}\)}%
\end{pgfscope}%
\begin{pgfscope}%
\definecolor{textcolor}{rgb}{0.000000,0.000000,0.000000}%
\pgfsetstrokecolor{textcolor}%
\pgfsetfillcolor{textcolor}%
\pgftext[x=0.483776in,y=2.978201in,left,base]{\color{textcolor}\rmfamily\fontsize{8.000000}{9.600000}\selectfont \(\displaystyle \times{10^{\ensuremath{-}6}}{}\)}%
\end{pgfscope}%
\begin{pgfscope}%
\pgfpathrectangle{\pgfqpoint{0.483776in}{2.351653in}}{\pgfqpoint{4.566474in}{0.584881in}}%
\pgfusepath{clip}%
\pgfsetrectcap%
\pgfsetroundjoin%
\pgfsetlinewidth{0.501875pt}%
\definecolor{currentstroke}{rgb}{0.121569,0.466667,0.705882}%
\pgfsetstrokecolor{currentstroke}%
\pgfsetstrokeopacity{0.700000}%
\pgfsetdash{}{0pt}%
\pgfpathmoveto{\pgfqpoint{0.691343in}{2.530219in}}%
\pgfpathlineto{\pgfqpoint{0.692205in}{2.507777in}}%
\pgfpathlineto{\pgfqpoint{0.693071in}{2.590474in}}%
\pgfpathlineto{\pgfqpoint{0.694800in}{2.506363in}}%
\pgfpathlineto{\pgfqpoint{0.695666in}{2.545836in}}%
\pgfpathlineto{\pgfqpoint{0.696532in}{2.498186in}}%
\pgfpathlineto{\pgfqpoint{0.697397in}{2.501383in}}%
\pgfpathlineto{\pgfqpoint{0.699128in}{2.538888in}}%
\pgfpathlineto{\pgfqpoint{0.699993in}{2.536184in}}%
\pgfpathlineto{\pgfqpoint{0.700859in}{2.601850in}}%
\pgfpathlineto{\pgfqpoint{0.701725in}{2.538336in}}%
\pgfpathlineto{\pgfqpoint{0.702589in}{2.553523in}}%
\pgfpathlineto{\pgfqpoint{0.703453in}{2.510576in}}%
\pgfpathlineto{\pgfqpoint{0.706051in}{2.564068in}}%
\pgfpathlineto{\pgfqpoint{0.707780in}{2.536184in}}%
\pgfpathlineto{\pgfqpoint{0.708646in}{2.529913in}}%
\pgfpathlineto{\pgfqpoint{0.709512in}{2.574367in}}%
\pgfpathlineto{\pgfqpoint{0.710377in}{2.521305in}}%
\pgfpathlineto{\pgfqpoint{0.712105in}{2.585309in}}%
\pgfpathlineto{\pgfqpoint{0.712971in}{2.517924in}}%
\pgfpathlineto{\pgfqpoint{0.713837in}{2.551432in}}%
\pgfpathlineto{\pgfqpoint{0.714702in}{2.479803in}}%
\pgfpathlineto{\pgfqpoint{0.716430in}{2.549464in}}%
\pgfpathlineto{\pgfqpoint{0.719030in}{2.475436in}}%
\pgfpathlineto{\pgfqpoint{0.719895in}{2.469535in}}%
\pgfpathlineto{\pgfqpoint{0.720762in}{2.400426in}}%
\pgfpathlineto{\pgfqpoint{0.721627in}{2.506486in}}%
\pgfpathlineto{\pgfqpoint{0.722492in}{2.468950in}}%
\pgfpathlineto{\pgfqpoint{0.723356in}{2.490807in}}%
\pgfpathlineto{\pgfqpoint{0.724219in}{2.547435in}}%
\pgfpathlineto{\pgfqpoint{0.725084in}{2.460004in}}%
\pgfpathlineto{\pgfqpoint{0.725947in}{2.531019in}}%
\pgfpathlineto{\pgfqpoint{0.726812in}{2.496280in}}%
\pgfpathlineto{\pgfqpoint{0.728541in}{2.519030in}}%
\pgfpathlineto{\pgfqpoint{0.729407in}{2.497203in}}%
\pgfpathlineto{\pgfqpoint{0.730271in}{2.584758in}}%
\pgfpathlineto{\pgfqpoint{0.732865in}{2.466092in}}%
\pgfpathlineto{\pgfqpoint{0.733731in}{2.535384in}}%
\pgfpathlineto{\pgfqpoint{0.734597in}{2.484044in}}%
\pgfpathlineto{\pgfqpoint{0.735460in}{2.484475in}}%
\pgfpathlineto{\pgfqpoint{0.736325in}{2.528682in}}%
\pgfpathlineto{\pgfqpoint{0.737190in}{2.491331in}}%
\pgfpathlineto{\pgfqpoint{0.738056in}{2.494989in}}%
\pgfpathlineto{\pgfqpoint{0.738920in}{2.504888in}}%
\pgfpathlineto{\pgfqpoint{0.740649in}{2.546144in}}%
\pgfpathlineto{\pgfqpoint{0.742381in}{2.482078in}}%
\pgfpathlineto{\pgfqpoint{0.743246in}{2.512636in}}%
\pgfpathlineto{\pgfqpoint{0.744112in}{2.503659in}}%
\pgfpathlineto{\pgfqpoint{0.744977in}{2.529852in}}%
\pgfpathlineto{\pgfqpoint{0.745840in}{2.522658in}}%
\pgfpathlineto{\pgfqpoint{0.746705in}{2.526285in}}%
\pgfpathlineto{\pgfqpoint{0.747569in}{2.494989in}}%
\pgfpathlineto{\pgfqpoint{0.748435in}{2.558934in}}%
\pgfpathlineto{\pgfqpoint{0.749299in}{2.542886in}}%
\pgfpathlineto{\pgfqpoint{0.750163in}{2.552294in}}%
\pgfpathlineto{\pgfqpoint{0.751028in}{2.540704in}}%
\pgfpathlineto{\pgfqpoint{0.752757in}{2.494006in}}%
\pgfpathlineto{\pgfqpoint{0.753622in}{2.550357in}}%
\pgfpathlineto{\pgfqpoint{0.754488in}{2.440761in}}%
\pgfpathlineto{\pgfqpoint{0.755354in}{2.493268in}}%
\pgfpathlineto{\pgfqpoint{0.756219in}{2.471287in}}%
\pgfpathlineto{\pgfqpoint{0.757950in}{2.535078in}}%
\pgfpathlineto{\pgfqpoint{0.758816in}{2.534524in}}%
\pgfpathlineto{\pgfqpoint{0.759679in}{2.531450in}}%
\pgfpathlineto{\pgfqpoint{0.760542in}{2.495420in}}%
\pgfpathlineto{\pgfqpoint{0.761407in}{2.534339in}}%
\pgfpathlineto{\pgfqpoint{0.762269in}{2.509685in}}%
\pgfpathlineto{\pgfqpoint{0.763135in}{2.514727in}}%
\pgfpathlineto{\pgfqpoint{0.764000in}{2.521429in}}%
\pgfpathlineto{\pgfqpoint{0.764865in}{2.551617in}}%
\pgfpathlineto{\pgfqpoint{0.765730in}{2.507256in}}%
\pgfpathlineto{\pgfqpoint{0.766596in}{2.566988in}}%
\pgfpathlineto{\pgfqpoint{0.767462in}{2.500277in}}%
\pgfpathlineto{\pgfqpoint{0.769191in}{2.567172in}}%
\pgfpathlineto{\pgfqpoint{0.770922in}{2.489763in}}%
\pgfpathlineto{\pgfqpoint{0.771788in}{2.603141in}}%
\pgfpathlineto{\pgfqpoint{0.772652in}{2.517216in}}%
\pgfpathlineto{\pgfqpoint{0.773518in}{2.550939in}}%
\pgfpathlineto{\pgfqpoint{0.774383in}{2.502920in}}%
\pgfpathlineto{\pgfqpoint{0.775249in}{2.504857in}}%
\pgfpathlineto{\pgfqpoint{0.776979in}{2.541778in}}%
\pgfpathlineto{\pgfqpoint{0.779575in}{2.501568in}}%
\pgfpathlineto{\pgfqpoint{0.780439in}{2.532310in}}%
\pgfpathlineto{\pgfqpoint{0.781305in}{2.524870in}}%
\pgfpathlineto{\pgfqpoint{0.782170in}{2.511128in}}%
\pgfpathlineto{\pgfqpoint{0.783036in}{2.593610in}}%
\pgfpathlineto{\pgfqpoint{0.784766in}{2.501321in}}%
\pgfpathlineto{\pgfqpoint{0.785630in}{2.528990in}}%
\pgfpathlineto{\pgfqpoint{0.787360in}{2.487180in}}%
\pgfpathlineto{\pgfqpoint{0.788225in}{2.466951in}}%
\pgfpathlineto{\pgfqpoint{0.789089in}{2.410386in}}%
\pgfpathlineto{\pgfqpoint{0.789954in}{2.538704in}}%
\pgfpathlineto{\pgfqpoint{0.790816in}{2.520444in}}%
\pgfpathlineto{\pgfqpoint{0.791680in}{2.480601in}}%
\pgfpathlineto{\pgfqpoint{0.792546in}{2.576702in}}%
\pgfpathlineto{\pgfqpoint{0.793412in}{2.571906in}}%
\pgfpathlineto{\pgfqpoint{0.794274in}{2.527576in}}%
\pgfpathlineto{\pgfqpoint{0.795138in}{2.567480in}}%
\pgfpathlineto{\pgfqpoint{0.796004in}{2.509254in}}%
\pgfpathlineto{\pgfqpoint{0.796868in}{2.534216in}}%
\pgfpathlineto{\pgfqpoint{0.798599in}{2.477897in}}%
\pgfpathlineto{\pgfqpoint{0.799464in}{2.543440in}}%
\pgfpathlineto{\pgfqpoint{0.800327in}{2.537967in}}%
\pgfpathlineto{\pgfqpoint{0.801192in}{2.472855in}}%
\pgfpathlineto{\pgfqpoint{0.802058in}{2.568894in}}%
\pgfpathlineto{\pgfqpoint{0.803787in}{2.451429in}}%
\pgfpathlineto{\pgfqpoint{0.804652in}{2.464616in}}%
\pgfpathlineto{\pgfqpoint{0.805515in}{2.514727in}}%
\pgfpathlineto{\pgfqpoint{0.806378in}{2.440576in}}%
\pgfpathlineto{\pgfqpoint{0.808107in}{2.534401in}}%
\pgfpathlineto{\pgfqpoint{0.808973in}{2.535938in}}%
\pgfpathlineto{\pgfqpoint{0.809837in}{2.465722in}}%
\pgfpathlineto{\pgfqpoint{0.812431in}{2.522350in}}%
\pgfpathlineto{\pgfqpoint{0.813297in}{2.496218in}}%
\pgfpathlineto{\pgfqpoint{0.814161in}{2.511713in}}%
\pgfpathlineto{\pgfqpoint{0.815026in}{2.557949in}}%
\pgfpathlineto{\pgfqpoint{0.817620in}{2.473284in}}%
\pgfpathlineto{\pgfqpoint{0.818484in}{2.579837in}}%
\pgfpathlineto{\pgfqpoint{0.819349in}{2.551309in}}%
\pgfpathlineto{\pgfqpoint{0.820213in}{2.504272in}}%
\pgfpathlineto{\pgfqpoint{0.821078in}{2.576763in}}%
\pgfpathlineto{\pgfqpoint{0.821942in}{2.542455in}}%
\pgfpathlineto{\pgfqpoint{0.822807in}{2.555642in}}%
\pgfpathlineto{\pgfqpoint{0.823671in}{2.672556in}}%
\pgfpathlineto{\pgfqpoint{0.824536in}{2.516693in}}%
\pgfpathlineto{\pgfqpoint{0.825401in}{2.582297in}}%
\pgfpathlineto{\pgfqpoint{0.826268in}{2.535630in}}%
\pgfpathlineto{\pgfqpoint{0.827133in}{2.539596in}}%
\pgfpathlineto{\pgfqpoint{0.827999in}{2.555244in}}%
\pgfpathlineto{\pgfqpoint{0.828864in}{2.551924in}}%
\pgfpathlineto{\pgfqpoint{0.829730in}{2.505565in}}%
\pgfpathlineto{\pgfqpoint{0.831462in}{2.639784in}}%
\pgfpathlineto{\pgfqpoint{0.832328in}{2.563391in}}%
\pgfpathlineto{\pgfqpoint{0.833193in}{2.597730in}}%
\pgfpathlineto{\pgfqpoint{0.835786in}{2.510053in}}%
\pgfpathlineto{\pgfqpoint{0.836650in}{2.530650in}}%
\pgfpathlineto{\pgfqpoint{0.837516in}{2.501814in}}%
\pgfpathlineto{\pgfqpoint{0.838382in}{2.518968in}}%
\pgfpathlineto{\pgfqpoint{0.840114in}{2.487303in}}%
\pgfpathlineto{\pgfqpoint{0.840980in}{2.522042in}}%
\pgfpathlineto{\pgfqpoint{0.841845in}{2.510145in}}%
\pgfpathlineto{\pgfqpoint{0.842710in}{2.472239in}}%
\pgfpathlineto{\pgfqpoint{0.843575in}{2.501014in}}%
\pgfpathlineto{\pgfqpoint{0.844440in}{2.458220in}}%
\pgfpathlineto{\pgfqpoint{0.845305in}{2.550509in}}%
\pgfpathlineto{\pgfqpoint{0.847033in}{2.473961in}}%
\pgfpathlineto{\pgfqpoint{0.847897in}{2.532125in}}%
\pgfpathlineto{\pgfqpoint{0.849627in}{2.491485in}}%
\pgfpathlineto{\pgfqpoint{0.850491in}{2.489763in}}%
\pgfpathlineto{\pgfqpoint{0.851354in}{2.469073in}}%
\pgfpathlineto{\pgfqpoint{0.853081in}{2.535568in}}%
\pgfpathlineto{\pgfqpoint{0.853945in}{2.515033in}}%
\pgfpathlineto{\pgfqpoint{0.854810in}{2.520690in}}%
\pgfpathlineto{\pgfqpoint{0.855672in}{2.572858in}}%
\pgfpathlineto{\pgfqpoint{0.856537in}{2.558072in}}%
\pgfpathlineto{\pgfqpoint{0.857402in}{2.494250in}}%
\pgfpathlineto{\pgfqpoint{0.858267in}{2.561330in}}%
\pgfpathlineto{\pgfqpoint{0.859131in}{2.549218in}}%
\pgfpathlineto{\pgfqpoint{0.859997in}{2.479187in}}%
\pgfpathlineto{\pgfqpoint{0.860864in}{2.482630in}}%
\pgfpathlineto{\pgfqpoint{0.862597in}{2.580022in}}%
\pgfpathlineto{\pgfqpoint{0.863463in}{2.521796in}}%
\pgfpathlineto{\pgfqpoint{0.864329in}{2.544176in}}%
\pgfpathlineto{\pgfqpoint{0.866062in}{2.483798in}}%
\pgfpathlineto{\pgfqpoint{0.866929in}{2.519643in}}%
\pgfpathlineto{\pgfqpoint{0.867794in}{2.512511in}}%
\pgfpathlineto{\pgfqpoint{0.868659in}{2.511035in}}%
\pgfpathlineto{\pgfqpoint{0.869523in}{2.503841in}}%
\pgfpathlineto{\pgfqpoint{0.870388in}{2.487210in}}%
\pgfpathlineto{\pgfqpoint{0.872119in}{2.555920in}}%
\pgfpathlineto{\pgfqpoint{0.873848in}{2.522902in}}%
\pgfpathlineto{\pgfqpoint{0.874712in}{2.512141in}}%
\pgfpathlineto{\pgfqpoint{0.875577in}{2.538581in}}%
\pgfpathlineto{\pgfqpoint{0.876442in}{2.496955in}}%
\pgfpathlineto{\pgfqpoint{0.877309in}{2.512388in}}%
\pgfpathlineto{\pgfqpoint{0.878174in}{2.500275in}}%
\pgfpathlineto{\pgfqpoint{0.879035in}{2.516323in}}%
\pgfpathlineto{\pgfqpoint{0.879903in}{2.487333in}}%
\pgfpathlineto{\pgfqpoint{0.880769in}{2.493696in}}%
\pgfpathlineto{\pgfqpoint{0.881633in}{2.531694in}}%
\pgfpathlineto{\pgfqpoint{0.882498in}{2.495787in}}%
\pgfpathlineto{\pgfqpoint{0.884229in}{2.560499in}}%
\pgfpathlineto{\pgfqpoint{0.885096in}{2.504395in}}%
\pgfpathlineto{\pgfqpoint{0.886826in}{2.583126in}}%
\pgfpathlineto{\pgfqpoint{0.887691in}{2.422498in}}%
\pgfpathlineto{\pgfqpoint{0.889418in}{2.543961in}}%
\pgfpathlineto{\pgfqpoint{0.890283in}{2.565880in}}%
\pgfpathlineto{\pgfqpoint{0.891148in}{2.557395in}}%
\pgfpathlineto{\pgfqpoint{0.892011in}{2.564035in}}%
\pgfpathlineto{\pgfqpoint{0.893741in}{2.529726in}}%
\pgfpathlineto{\pgfqpoint{0.894605in}{2.551553in}}%
\pgfpathlineto{\pgfqpoint{0.896337in}{2.531725in}}%
\pgfpathlineto{\pgfqpoint{0.897202in}{2.561146in}}%
\pgfpathlineto{\pgfqpoint{0.898931in}{2.473561in}}%
\pgfpathlineto{\pgfqpoint{0.899795in}{2.509252in}}%
\pgfpathlineto{\pgfqpoint{0.900660in}{2.595085in}}%
\pgfpathlineto{\pgfqpoint{0.901525in}{2.502735in}}%
\pgfpathlineto{\pgfqpoint{0.902390in}{2.559547in}}%
\pgfpathlineto{\pgfqpoint{0.903254in}{2.503872in}}%
\pgfpathlineto{\pgfqpoint{0.904119in}{2.575288in}}%
\pgfpathlineto{\pgfqpoint{0.905849in}{2.545405in}}%
\pgfpathlineto{\pgfqpoint{0.907579in}{2.600250in}}%
\pgfpathlineto{\pgfqpoint{0.908445in}{2.501075in}}%
\pgfpathlineto{\pgfqpoint{0.910175in}{2.834938in}}%
\pgfpathlineto{\pgfqpoint{0.911039in}{2.755746in}}%
\pgfpathlineto{\pgfqpoint{0.911905in}{2.798847in}}%
\pgfpathlineto{\pgfqpoint{0.912770in}{2.696475in}}%
\pgfpathlineto{\pgfqpoint{0.913635in}{2.736931in}}%
\pgfpathlineto{\pgfqpoint{0.914501in}{2.692170in}}%
\pgfpathlineto{\pgfqpoint{0.916229in}{2.784951in}}%
\pgfpathlineto{\pgfqpoint{0.917094in}{2.781508in}}%
\pgfpathlineto{\pgfqpoint{0.917959in}{2.753809in}}%
\pgfpathlineto{\pgfqpoint{0.918824in}{2.549341in}}%
\pgfpathlineto{\pgfqpoint{0.919687in}{2.820183in}}%
\pgfpathlineto{\pgfqpoint{0.920552in}{2.780525in}}%
\pgfpathlineto{\pgfqpoint{0.921417in}{2.732813in}}%
\pgfpathlineto{\pgfqpoint{0.922280in}{2.833032in}}%
\pgfpathlineto{\pgfqpoint{0.924010in}{2.523517in}}%
\pgfpathlineto{\pgfqpoint{0.925740in}{2.485827in}}%
\pgfpathlineto{\pgfqpoint{0.927464in}{2.528467in}}%
\pgfpathlineto{\pgfqpoint{0.929193in}{2.484229in}}%
\pgfpathlineto{\pgfqpoint{0.930058in}{2.556166in}}%
\pgfpathlineto{\pgfqpoint{0.931787in}{2.469717in}}%
\pgfpathlineto{\pgfqpoint{0.932652in}{2.481645in}}%
\pgfpathlineto{\pgfqpoint{0.934381in}{2.541685in}}%
\pgfpathlineto{\pgfqpoint{0.935246in}{2.506055in}}%
\pgfpathlineto{\pgfqpoint{0.936109in}{2.512634in}}%
\pgfpathlineto{\pgfqpoint{0.936974in}{2.520934in}}%
\pgfpathlineto{\pgfqpoint{0.937840in}{2.507223in}}%
\pgfpathlineto{\pgfqpoint{0.938706in}{2.528251in}}%
\pgfpathlineto{\pgfqpoint{0.939570in}{2.454285in}}%
\pgfpathlineto{\pgfqpoint{0.941301in}{2.497938in}}%
\pgfpathlineto{\pgfqpoint{0.943032in}{2.479862in}}%
\pgfpathlineto{\pgfqpoint{0.943897in}{2.565664in}}%
\pgfpathlineto{\pgfqpoint{0.944763in}{2.555612in}}%
\pgfpathlineto{\pgfqpoint{0.945628in}{2.568953in}}%
\pgfpathlineto{\pgfqpoint{0.946490in}{2.480109in}}%
\pgfpathlineto{\pgfqpoint{0.947355in}{2.483921in}}%
\pgfpathlineto{\pgfqpoint{0.948218in}{2.525177in}}%
\pgfpathlineto{\pgfqpoint{0.951674in}{2.469410in}}%
\pgfpathlineto{\pgfqpoint{0.954269in}{2.527820in}}%
\pgfpathlineto{\pgfqpoint{0.955133in}{2.489270in}}%
\pgfpathlineto{\pgfqpoint{0.958594in}{2.554198in}}%
\pgfpathlineto{\pgfqpoint{0.959459in}{2.458344in}}%
\pgfpathlineto{\pgfqpoint{0.960325in}{2.558103in}}%
\pgfpathlineto{\pgfqpoint{0.961191in}{2.541041in}}%
\pgfpathlineto{\pgfqpoint{0.962055in}{2.594040in}}%
\pgfpathlineto{\pgfqpoint{0.963783in}{2.480847in}}%
\pgfpathlineto{\pgfqpoint{0.964647in}{2.514786in}}%
\pgfpathlineto{\pgfqpoint{0.965512in}{2.468581in}}%
\pgfpathlineto{\pgfqpoint{0.966375in}{2.564712in}}%
\pgfpathlineto{\pgfqpoint{0.967241in}{2.549279in}}%
\pgfpathlineto{\pgfqpoint{0.968105in}{2.561577in}}%
\pgfpathlineto{\pgfqpoint{0.968970in}{2.459267in}}%
\pgfpathlineto{\pgfqpoint{0.969836in}{2.547373in}}%
\pgfpathlineto{\pgfqpoint{0.970701in}{2.468858in}}%
\pgfpathlineto{\pgfqpoint{0.971567in}{2.494743in}}%
\pgfpathlineto{\pgfqpoint{0.972432in}{2.494681in}}%
\pgfpathlineto{\pgfqpoint{0.973297in}{2.508392in}}%
\pgfpathlineto{\pgfqpoint{0.974162in}{2.473899in}}%
\pgfpathlineto{\pgfqpoint{0.975027in}{2.506517in}}%
\pgfpathlineto{\pgfqpoint{0.975889in}{2.595454in}}%
\pgfpathlineto{\pgfqpoint{0.977616in}{2.480724in}}%
\pgfpathlineto{\pgfqpoint{0.978481in}{2.523517in}}%
\pgfpathlineto{\pgfqpoint{0.979346in}{2.518476in}}%
\pgfpathlineto{\pgfqpoint{0.980211in}{2.480539in}}%
\pgfpathlineto{\pgfqpoint{0.981076in}{2.562929in}}%
\pgfpathlineto{\pgfqpoint{0.981941in}{2.547404in}}%
\pgfpathlineto{\pgfqpoint{0.982806in}{2.574980in}}%
\pgfpathlineto{\pgfqpoint{0.984535in}{2.477404in}}%
\pgfpathlineto{\pgfqpoint{0.985400in}{2.521980in}}%
\pgfpathlineto{\pgfqpoint{0.986264in}{2.462156in}}%
\pgfpathlineto{\pgfqpoint{0.987127in}{2.527022in}}%
\pgfpathlineto{\pgfqpoint{0.987991in}{2.429569in}}%
\pgfpathlineto{\pgfqpoint{0.988856in}{2.514879in}}%
\pgfpathlineto{\pgfqpoint{0.989722in}{2.511158in}}%
\pgfpathlineto{\pgfqpoint{0.990585in}{2.512942in}}%
\pgfpathlineto{\pgfqpoint{0.991449in}{2.500860in}}%
\pgfpathlineto{\pgfqpoint{0.992314in}{2.526714in}}%
\pgfpathlineto{\pgfqpoint{0.993178in}{2.514602in}}%
\pgfpathlineto{\pgfqpoint{0.994043in}{2.534309in}}%
\pgfpathlineto{\pgfqpoint{0.994907in}{2.517062in}}%
\pgfpathlineto{\pgfqpoint{0.995771in}{2.468919in}}%
\pgfpathlineto{\pgfqpoint{0.997501in}{2.512880in}}%
\pgfpathlineto{\pgfqpoint{0.998367in}{2.528467in}}%
\pgfpathlineto{\pgfqpoint{1.000096in}{2.502920in}}%
\pgfpathlineto{\pgfqpoint{1.000961in}{2.560530in}}%
\pgfpathlineto{\pgfqpoint{1.001826in}{2.543684in}}%
\pgfpathlineto{\pgfqpoint{1.002691in}{2.549464in}}%
\pgfpathlineto{\pgfqpoint{1.004419in}{2.516323in}}%
\pgfpathlineto{\pgfqpoint{1.005284in}{2.516508in}}%
\pgfpathlineto{\pgfqpoint{1.006149in}{2.506548in}}%
\pgfpathlineto{\pgfqpoint{1.007014in}{2.552538in}}%
\pgfpathlineto{\pgfqpoint{1.007877in}{2.514355in}}%
\pgfpathlineto{\pgfqpoint{1.009607in}{2.556841in}}%
\pgfpathlineto{\pgfqpoint{1.012203in}{2.436700in}}%
\pgfpathlineto{\pgfqpoint{1.013933in}{2.590597in}}%
\pgfpathlineto{\pgfqpoint{1.014798in}{2.534278in}}%
\pgfpathlineto{\pgfqpoint{1.015662in}{2.474453in}}%
\pgfpathlineto{\pgfqpoint{1.017391in}{2.553277in}}%
\pgfpathlineto{\pgfqpoint{1.018255in}{2.585556in}}%
\pgfpathlineto{\pgfqpoint{1.019120in}{2.565328in}}%
\pgfpathlineto{\pgfqpoint{1.019986in}{2.489671in}}%
\pgfpathlineto{\pgfqpoint{1.020851in}{2.530711in}}%
\pgfpathlineto{\pgfqpoint{1.021716in}{2.492344in}}%
\pgfpathlineto{\pgfqpoint{1.022581in}{2.496311in}}%
\pgfpathlineto{\pgfqpoint{1.023445in}{2.487487in}}%
\pgfpathlineto{\pgfqpoint{1.025173in}{2.552353in}}%
\pgfpathlineto{\pgfqpoint{1.026903in}{2.506917in}}%
\pgfpathlineto{\pgfqpoint{1.027767in}{2.494620in}}%
\pgfpathlineto{\pgfqpoint{1.028629in}{2.617773in}}%
\pgfpathlineto{\pgfqpoint{1.030357in}{2.505072in}}%
\pgfpathlineto{\pgfqpoint{1.031221in}{2.541410in}}%
\pgfpathlineto{\pgfqpoint{1.032087in}{2.529636in}}%
\pgfpathlineto{\pgfqpoint{1.033816in}{2.555060in}}%
\pgfpathlineto{\pgfqpoint{1.035545in}{2.463631in}}%
\pgfpathlineto{\pgfqpoint{1.037275in}{2.479372in}}%
\pgfpathlineto{\pgfqpoint{1.038139in}{2.472239in}}%
\pgfpathlineto{\pgfqpoint{1.040734in}{2.553767in}}%
\pgfpathlineto{\pgfqpoint{1.041600in}{2.555858in}}%
\pgfpathlineto{\pgfqpoint{1.042466in}{2.565143in}}%
\pgfpathlineto{\pgfqpoint{1.043331in}{2.563360in}}%
\pgfpathlineto{\pgfqpoint{1.044195in}{2.552661in}}%
\pgfpathlineto{\pgfqpoint{1.045060in}{2.594777in}}%
\pgfpathlineto{\pgfqpoint{1.045927in}{2.473561in}}%
\pgfpathlineto{\pgfqpoint{1.046792in}{2.524993in}}%
\pgfpathlineto{\pgfqpoint{1.047657in}{2.520259in}}%
\pgfpathlineto{\pgfqpoint{1.048523in}{2.540395in}}%
\pgfpathlineto{\pgfqpoint{1.050256in}{2.485827in}}%
\pgfpathlineto{\pgfqpoint{1.051122in}{2.507777in}}%
\pgfpathlineto{\pgfqpoint{1.051987in}{2.445187in}}%
\pgfpathlineto{\pgfqpoint{1.054584in}{2.557025in}}%
\pgfpathlineto{\pgfqpoint{1.055449in}{2.498707in}}%
\pgfpathlineto{\pgfqpoint{1.057179in}{2.536921in}}%
\pgfpathlineto{\pgfqpoint{1.059773in}{2.497386in}}%
\pgfpathlineto{\pgfqpoint{1.060637in}{2.446323in}}%
\pgfpathlineto{\pgfqpoint{1.061503in}{2.553952in}}%
\pgfpathlineto{\pgfqpoint{1.062369in}{2.518414in}}%
\pgfpathlineto{\pgfqpoint{1.063234in}{2.519643in}}%
\pgfpathlineto{\pgfqpoint{1.064099in}{2.507900in}}%
\pgfpathlineto{\pgfqpoint{1.064962in}{2.556595in}}%
\pgfpathlineto{\pgfqpoint{1.068422in}{2.469410in}}%
\pgfpathlineto{\pgfqpoint{1.069287in}{2.471808in}}%
\pgfpathlineto{\pgfqpoint{1.071017in}{2.570183in}}%
\pgfpathlineto{\pgfqpoint{1.073608in}{2.452563in}}%
\pgfpathlineto{\pgfqpoint{1.074470in}{2.531263in}}%
\pgfpathlineto{\pgfqpoint{1.075335in}{2.509252in}}%
\pgfpathlineto{\pgfqpoint{1.076199in}{2.456499in}}%
\pgfpathlineto{\pgfqpoint{1.077064in}{2.521734in}}%
\pgfpathlineto{\pgfqpoint{1.077927in}{2.519766in}}%
\pgfpathlineto{\pgfqpoint{1.078792in}{2.559301in}}%
\pgfpathlineto{\pgfqpoint{1.080521in}{2.451149in}}%
\pgfpathlineto{\pgfqpoint{1.081384in}{2.500706in}}%
\pgfpathlineto{\pgfqpoint{1.082248in}{2.474759in}}%
\pgfpathlineto{\pgfqpoint{1.083981in}{2.563789in}}%
\pgfpathlineto{\pgfqpoint{1.084846in}{2.555612in}}%
\pgfpathlineto{\pgfqpoint{1.085711in}{2.565880in}}%
\pgfpathlineto{\pgfqpoint{1.086576in}{2.458957in}}%
\pgfpathlineto{\pgfqpoint{1.088306in}{2.596252in}}%
\pgfpathlineto{\pgfqpoint{1.090037in}{2.520626in}}%
\pgfpathlineto{\pgfqpoint{1.090903in}{2.523885in}}%
\pgfpathlineto{\pgfqpoint{1.092632in}{2.492160in}}%
\pgfpathlineto{\pgfqpoint{1.093498in}{2.452625in}}%
\pgfpathlineto{\pgfqpoint{1.094362in}{2.517121in}}%
\pgfpathlineto{\pgfqpoint{1.095225in}{2.508760in}}%
\pgfpathlineto{\pgfqpoint{1.096090in}{2.539133in}}%
\pgfpathlineto{\pgfqpoint{1.097822in}{2.469040in}}%
\pgfpathlineto{\pgfqpoint{1.099551in}{2.585125in}}%
\pgfpathlineto{\pgfqpoint{1.101281in}{2.538273in}}%
\pgfpathlineto{\pgfqpoint{1.102145in}{2.544728in}}%
\pgfpathlineto{\pgfqpoint{1.103871in}{2.487118in}}%
\pgfpathlineto{\pgfqpoint{1.104736in}{2.521119in}}%
\pgfpathlineto{\pgfqpoint{1.106466in}{2.483182in}}%
\pgfpathlineto{\pgfqpoint{1.107328in}{2.523271in}}%
\pgfpathlineto{\pgfqpoint{1.108192in}{2.516446in}}%
\pgfpathlineto{\pgfqpoint{1.109055in}{2.521611in}}%
\pgfpathlineto{\pgfqpoint{1.109918in}{2.573933in}}%
\pgfpathlineto{\pgfqpoint{1.110782in}{2.491913in}}%
\pgfpathlineto{\pgfqpoint{1.111647in}{2.517306in}}%
\pgfpathlineto{\pgfqpoint{1.113375in}{2.447583in}}%
\pgfpathlineto{\pgfqpoint{1.115967in}{2.557518in}}%
\pgfpathlineto{\pgfqpoint{1.117696in}{2.504734in}}%
\pgfpathlineto{\pgfqpoint{1.118561in}{2.575780in}}%
\pgfpathlineto{\pgfqpoint{1.120289in}{2.467321in}}%
\pgfpathlineto{\pgfqpoint{1.122016in}{2.537629in}}%
\pgfpathlineto{\pgfqpoint{1.122881in}{2.487610in}}%
\pgfpathlineto{\pgfqpoint{1.123746in}{2.492898in}}%
\pgfpathlineto{\pgfqpoint{1.124612in}{2.501229in}}%
\pgfpathlineto{\pgfqpoint{1.126342in}{2.547681in}}%
\pgfpathlineto{\pgfqpoint{1.127208in}{2.494066in}}%
\pgfpathlineto{\pgfqpoint{1.128072in}{2.512757in}}%
\pgfpathlineto{\pgfqpoint{1.128936in}{2.510912in}}%
\pgfpathlineto{\pgfqpoint{1.129800in}{2.459880in}}%
\pgfpathlineto{\pgfqpoint{1.130665in}{2.506055in}}%
\pgfpathlineto{\pgfqpoint{1.131530in}{2.488501in}}%
\pgfpathlineto{\pgfqpoint{1.133261in}{2.538396in}}%
\pgfpathlineto{\pgfqpoint{1.134126in}{2.495264in}}%
\pgfpathlineto{\pgfqpoint{1.134991in}{2.518045in}}%
\pgfpathlineto{\pgfqpoint{1.135856in}{2.459388in}}%
\pgfpathlineto{\pgfqpoint{1.136722in}{2.518014in}}%
\pgfpathlineto{\pgfqpoint{1.137586in}{2.437869in}}%
\pgfpathlineto{\pgfqpoint{1.138451in}{2.535137in}}%
\pgfpathlineto{\pgfqpoint{1.139316in}{2.480478in}}%
\pgfpathlineto{\pgfqpoint{1.140181in}{2.499415in}}%
\pgfpathlineto{\pgfqpoint{1.141044in}{2.486841in}}%
\pgfpathlineto{\pgfqpoint{1.142776in}{2.569507in}}%
\pgfpathlineto{\pgfqpoint{1.144506in}{2.463385in}}%
\pgfpathlineto{\pgfqpoint{1.145371in}{2.495295in}}%
\pgfpathlineto{\pgfqpoint{1.146235in}{2.469625in}}%
\pgfpathlineto{\pgfqpoint{1.147966in}{2.566434in}}%
\pgfpathlineto{\pgfqpoint{1.148829in}{2.487364in}}%
\pgfpathlineto{\pgfqpoint{1.149692in}{2.508085in}}%
\pgfpathlineto{\pgfqpoint{1.150558in}{2.502489in}}%
\pgfpathlineto{\pgfqpoint{1.152290in}{2.558932in}}%
\pgfpathlineto{\pgfqpoint{1.154885in}{2.468427in}}%
\pgfpathlineto{\pgfqpoint{1.155749in}{2.533508in}}%
\pgfpathlineto{\pgfqpoint{1.156614in}{2.523517in}}%
\pgfpathlineto{\pgfqpoint{1.158343in}{2.492590in}}%
\pgfpathlineto{\pgfqpoint{1.159208in}{2.547312in}}%
\pgfpathlineto{\pgfqpoint{1.160073in}{2.530065in}}%
\pgfpathlineto{\pgfqpoint{1.160937in}{2.482384in}}%
\pgfpathlineto{\pgfqpoint{1.161802in}{2.611931in}}%
\pgfpathlineto{\pgfqpoint{1.162667in}{2.610702in}}%
\pgfpathlineto{\pgfqpoint{1.164394in}{2.600065in}}%
\pgfpathlineto{\pgfqpoint{1.165259in}{2.497694in}}%
\pgfpathlineto{\pgfqpoint{1.166124in}{2.528190in}}%
\pgfpathlineto{\pgfqpoint{1.166990in}{2.603877in}}%
\pgfpathlineto{\pgfqpoint{1.167854in}{2.583219in}}%
\pgfpathlineto{\pgfqpoint{1.171315in}{2.496095in}}%
\pgfpathlineto{\pgfqpoint{1.172180in}{2.515431in}}%
\pgfpathlineto{\pgfqpoint{1.173045in}{2.574488in}}%
\pgfpathlineto{\pgfqpoint{1.173912in}{2.559178in}}%
\pgfpathlineto{\pgfqpoint{1.174778in}{2.559178in}}%
\pgfpathlineto{\pgfqpoint{1.177374in}{2.461664in}}%
\pgfpathlineto{\pgfqpoint{1.178239in}{2.509806in}}%
\pgfpathlineto{\pgfqpoint{1.179103in}{2.441097in}}%
\pgfpathlineto{\pgfqpoint{1.180834in}{2.542886in}}%
\pgfpathlineto{\pgfqpoint{1.181699in}{2.588876in}}%
\pgfpathlineto{\pgfqpoint{1.183430in}{2.523271in}}%
\pgfpathlineto{\pgfqpoint{1.184295in}{2.576702in}}%
\pgfpathlineto{\pgfqpoint{1.186026in}{2.486135in}}%
\pgfpathlineto{\pgfqpoint{1.186891in}{2.590782in}}%
\pgfpathlineto{\pgfqpoint{1.187758in}{2.588937in}}%
\pgfpathlineto{\pgfqpoint{1.189487in}{2.513496in}}%
\pgfpathlineto{\pgfqpoint{1.190352in}{2.593117in}}%
\pgfpathlineto{\pgfqpoint{1.191218in}{2.492744in}}%
\pgfpathlineto{\pgfqpoint{1.193813in}{2.581743in}}%
\pgfpathlineto{\pgfqpoint{1.194678in}{2.539442in}}%
\pgfpathlineto{\pgfqpoint{1.195541in}{2.600681in}}%
\pgfpathlineto{\pgfqpoint{1.196406in}{2.472362in}}%
\pgfpathlineto{\pgfqpoint{1.197272in}{2.600927in}}%
\pgfpathlineto{\pgfqpoint{1.199001in}{2.544299in}}%
\pgfpathlineto{\pgfqpoint{1.199865in}{2.545898in}}%
\pgfpathlineto{\pgfqpoint{1.200729in}{2.532772in}}%
\pgfpathlineto{\pgfqpoint{1.201595in}{2.496280in}}%
\pgfpathlineto{\pgfqpoint{1.203324in}{2.535661in}}%
\pgfpathlineto{\pgfqpoint{1.204189in}{2.471441in}}%
\pgfpathlineto{\pgfqpoint{1.205054in}{2.528069in}}%
\pgfpathlineto{\pgfqpoint{1.205919in}{2.463018in}}%
\pgfpathlineto{\pgfqpoint{1.206785in}{2.517370in}}%
\pgfpathlineto{\pgfqpoint{1.207651in}{2.503505in}}%
\pgfpathlineto{\pgfqpoint{1.208516in}{2.554752in}}%
\pgfpathlineto{\pgfqpoint{1.210246in}{2.513219in}}%
\pgfpathlineto{\pgfqpoint{1.211111in}{2.493206in}}%
\pgfpathlineto{\pgfqpoint{1.212842in}{2.554044in}}%
\pgfpathlineto{\pgfqpoint{1.214572in}{2.510114in}}%
\pgfpathlineto{\pgfqpoint{1.216303in}{2.592011in}}%
\pgfpathlineto{\pgfqpoint{1.217169in}{2.535445in}}%
\pgfpathlineto{\pgfqpoint{1.218035in}{2.588322in}}%
\pgfpathlineto{\pgfqpoint{1.218902in}{2.547558in}}%
\pgfpathlineto{\pgfqpoint{1.219768in}{2.608888in}}%
\pgfpathlineto{\pgfqpoint{1.220634in}{2.502674in}}%
\pgfpathlineto{\pgfqpoint{1.221500in}{2.524010in}}%
\pgfpathlineto{\pgfqpoint{1.222366in}{2.494281in}}%
\pgfpathlineto{\pgfqpoint{1.224098in}{2.561330in}}%
\pgfpathlineto{\pgfqpoint{1.224965in}{2.537413in}}%
\pgfpathlineto{\pgfqpoint{1.225831in}{2.549464in}}%
\pgfpathlineto{\pgfqpoint{1.226697in}{2.474515in}}%
\pgfpathlineto{\pgfqpoint{1.227562in}{2.481401in}}%
\pgfpathlineto{\pgfqpoint{1.228427in}{2.565205in}}%
\pgfpathlineto{\pgfqpoint{1.229292in}{2.541102in}}%
\pgfpathlineto{\pgfqpoint{1.230156in}{2.479864in}}%
\pgfpathlineto{\pgfqpoint{1.231020in}{2.523825in}}%
\pgfpathlineto{\pgfqpoint{1.231885in}{2.497817in}}%
\pgfpathlineto{\pgfqpoint{1.233617in}{2.549772in}}%
\pgfpathlineto{\pgfqpoint{1.235347in}{2.460004in}}%
\pgfpathlineto{\pgfqpoint{1.237076in}{2.572951in}}%
\pgfpathlineto{\pgfqpoint{1.237942in}{2.579098in}}%
\pgfpathlineto{\pgfqpoint{1.238807in}{2.480355in}}%
\pgfpathlineto{\pgfqpoint{1.240537in}{2.557580in}}%
\pgfpathlineto{\pgfqpoint{1.241402in}{2.549064in}}%
\pgfpathlineto{\pgfqpoint{1.243133in}{2.489086in}}%
\pgfpathlineto{\pgfqpoint{1.243999in}{2.544115in}}%
\pgfpathlineto{\pgfqpoint{1.244864in}{2.501137in}}%
\pgfpathlineto{\pgfqpoint{1.245728in}{2.508239in}}%
\pgfpathlineto{\pgfqpoint{1.246593in}{2.553092in}}%
\pgfpathlineto{\pgfqpoint{1.247458in}{2.468981in}}%
\pgfpathlineto{\pgfqpoint{1.248323in}{2.473376in}}%
\pgfpathlineto{\pgfqpoint{1.249187in}{2.476358in}}%
\pgfpathlineto{\pgfqpoint{1.250053in}{2.558686in}}%
\pgfpathlineto{\pgfqpoint{1.250916in}{2.495326in}}%
\pgfpathlineto{\pgfqpoint{1.251781in}{2.508760in}}%
\pgfpathlineto{\pgfqpoint{1.252647in}{2.540179in}}%
\pgfpathlineto{\pgfqpoint{1.253512in}{2.526776in}}%
\pgfpathlineto{\pgfqpoint{1.254377in}{2.550878in}}%
\pgfpathlineto{\pgfqpoint{1.255243in}{2.489517in}}%
\pgfpathlineto{\pgfqpoint{1.256109in}{2.575103in}}%
\pgfpathlineto{\pgfqpoint{1.257840in}{2.519797in}}%
\pgfpathlineto{\pgfqpoint{1.258706in}{2.577746in}}%
\pgfpathlineto{\pgfqpoint{1.259571in}{2.474944in}}%
\pgfpathlineto{\pgfqpoint{1.260436in}{2.496095in}}%
\pgfpathlineto{\pgfqpoint{1.262165in}{2.559178in}}%
\pgfpathlineto{\pgfqpoint{1.263030in}{2.549033in}}%
\pgfpathlineto{\pgfqpoint{1.265625in}{2.428707in}}%
\pgfpathlineto{\pgfqpoint{1.267353in}{2.603416in}}%
\pgfpathlineto{\pgfqpoint{1.268216in}{2.590351in}}%
\pgfpathlineto{\pgfqpoint{1.269081in}{2.517983in}}%
\pgfpathlineto{\pgfqpoint{1.269946in}{2.536767in}}%
\pgfpathlineto{\pgfqpoint{1.270811in}{2.558809in}}%
\pgfpathlineto{\pgfqpoint{1.274272in}{2.496280in}}%
\pgfpathlineto{\pgfqpoint{1.275137in}{2.497694in}}%
\pgfpathlineto{\pgfqpoint{1.276003in}{2.529113in}}%
\pgfpathlineto{\pgfqpoint{1.276869in}{2.496341in}}%
\pgfpathlineto{\pgfqpoint{1.278600in}{2.620108in}}%
\pgfpathlineto{\pgfqpoint{1.280331in}{2.511035in}}%
\pgfpathlineto{\pgfqpoint{1.281196in}{2.514786in}}%
\pgfpathlineto{\pgfqpoint{1.282062in}{2.492098in}}%
\pgfpathlineto{\pgfqpoint{1.283793in}{2.574734in}}%
\pgfpathlineto{\pgfqpoint{1.284658in}{2.573320in}}%
\pgfpathlineto{\pgfqpoint{1.285521in}{2.592871in}}%
\pgfpathlineto{\pgfqpoint{1.287250in}{2.515033in}}%
\pgfpathlineto{\pgfqpoint{1.288980in}{2.538611in}}%
\pgfpathlineto{\pgfqpoint{1.290711in}{2.564404in}}%
\pgfpathlineto{\pgfqpoint{1.292440in}{2.523333in}}%
\pgfpathlineto{\pgfqpoint{1.293306in}{2.579775in}}%
\pgfpathlineto{\pgfqpoint{1.295037in}{2.525362in}}%
\pgfpathlineto{\pgfqpoint{1.295904in}{2.533354in}}%
\pgfpathlineto{\pgfqpoint{1.296768in}{2.555612in}}%
\pgfpathlineto{\pgfqpoint{1.297633in}{2.509868in}}%
\pgfpathlineto{\pgfqpoint{1.298498in}{2.390250in}}%
\pgfpathlineto{\pgfqpoint{1.300229in}{2.534830in}}%
\pgfpathlineto{\pgfqpoint{1.301092in}{2.457851in}}%
\pgfpathlineto{\pgfqpoint{1.301958in}{2.461356in}}%
\pgfpathlineto{\pgfqpoint{1.302823in}{2.454562in}}%
\pgfpathlineto{\pgfqpoint{1.303687in}{2.514355in}}%
\pgfpathlineto{\pgfqpoint{1.304552in}{2.512880in}}%
\pgfpathlineto{\pgfqpoint{1.305417in}{2.529788in}}%
\pgfpathlineto{\pgfqpoint{1.306282in}{2.518168in}}%
\pgfpathlineto{\pgfqpoint{1.307147in}{2.530096in}}%
\pgfpathlineto{\pgfqpoint{1.308013in}{2.465261in}}%
\pgfpathlineto{\pgfqpoint{1.308879in}{2.566064in}}%
\pgfpathlineto{\pgfqpoint{1.309745in}{2.500583in}}%
\pgfpathlineto{\pgfqpoint{1.310610in}{2.543930in}}%
\pgfpathlineto{\pgfqpoint{1.311476in}{2.524192in}}%
\pgfpathlineto{\pgfqpoint{1.312342in}{2.533016in}}%
\pgfpathlineto{\pgfqpoint{1.313208in}{2.486872in}}%
\pgfpathlineto{\pgfqpoint{1.314074in}{2.542945in}}%
\pgfpathlineto{\pgfqpoint{1.314939in}{2.530188in}}%
\pgfpathlineto{\pgfqpoint{1.315803in}{2.527266in}}%
\pgfpathlineto{\pgfqpoint{1.316665in}{2.527943in}}%
\pgfpathlineto{\pgfqpoint{1.317529in}{2.505655in}}%
\pgfpathlineto{\pgfqpoint{1.318395in}{2.424220in}}%
\pgfpathlineto{\pgfqpoint{1.319260in}{2.523086in}}%
\pgfpathlineto{\pgfqpoint{1.320124in}{2.516015in}}%
\pgfpathlineto{\pgfqpoint{1.320989in}{2.513863in}}%
\pgfpathlineto{\pgfqpoint{1.321854in}{2.544851in}}%
\pgfpathlineto{\pgfqpoint{1.322720in}{2.466213in}}%
\pgfpathlineto{\pgfqpoint{1.323583in}{2.516139in}}%
\pgfpathlineto{\pgfqpoint{1.324448in}{2.476450in}}%
\pgfpathlineto{\pgfqpoint{1.325313in}{2.547004in}}%
\pgfpathlineto{\pgfqpoint{1.327043in}{2.459234in}}%
\pgfpathlineto{\pgfqpoint{1.329636in}{2.539563in}}%
\pgfpathlineto{\pgfqpoint{1.330500in}{2.495541in}}%
\pgfpathlineto{\pgfqpoint{1.331365in}{2.500213in}}%
\pgfpathlineto{\pgfqpoint{1.332230in}{2.566986in}}%
\pgfpathlineto{\pgfqpoint{1.333954in}{2.479862in}}%
\pgfpathlineto{\pgfqpoint{1.336550in}{2.549924in}}%
\pgfpathlineto{\pgfqpoint{1.337413in}{2.541778in}}%
\pgfpathlineto{\pgfqpoint{1.338278in}{2.584448in}}%
\pgfpathlineto{\pgfqpoint{1.339144in}{2.495541in}}%
\pgfpathlineto{\pgfqpoint{1.340007in}{2.551984in}}%
\pgfpathlineto{\pgfqpoint{1.341734in}{2.497324in}}%
\pgfpathlineto{\pgfqpoint{1.342599in}{2.526283in}}%
\pgfpathlineto{\pgfqpoint{1.343462in}{2.516385in}}%
\pgfpathlineto{\pgfqpoint{1.344329in}{2.526960in}}%
\pgfpathlineto{\pgfqpoint{1.345195in}{2.458097in}}%
\pgfpathlineto{\pgfqpoint{1.346925in}{2.544238in}}%
\pgfpathlineto{\pgfqpoint{1.347786in}{2.547927in}}%
\pgfpathlineto{\pgfqpoint{1.348652in}{2.473838in}}%
\pgfpathlineto{\pgfqpoint{1.349515in}{2.560900in}}%
\pgfpathlineto{\pgfqpoint{1.350380in}{2.489024in}}%
\pgfpathlineto{\pgfqpoint{1.351245in}{2.504395in}}%
\pgfpathlineto{\pgfqpoint{1.352107in}{2.505994in}}%
\pgfpathlineto{\pgfqpoint{1.353837in}{2.451088in}}%
\pgfpathlineto{\pgfqpoint{1.354702in}{2.470518in}}%
\pgfpathlineto{\pgfqpoint{1.355566in}{2.519643in}}%
\pgfpathlineto{\pgfqpoint{1.356431in}{2.515279in}}%
\pgfpathlineto{\pgfqpoint{1.357296in}{2.496218in}}%
\pgfpathlineto{\pgfqpoint{1.358162in}{2.511282in}}%
\pgfpathlineto{\pgfqpoint{1.359028in}{2.508331in}}%
\pgfpathlineto{\pgfqpoint{1.359893in}{2.448322in}}%
\pgfpathlineto{\pgfqpoint{1.360758in}{2.508146in}}%
\pgfpathlineto{\pgfqpoint{1.361622in}{2.501445in}}%
\pgfpathlineto{\pgfqpoint{1.362488in}{2.516169in}}%
\pgfpathlineto{\pgfqpoint{1.363353in}{2.571167in}}%
\pgfpathlineto{\pgfqpoint{1.364217in}{2.531817in}}%
\pgfpathlineto{\pgfqpoint{1.365947in}{2.565695in}}%
\pgfpathlineto{\pgfqpoint{1.367678in}{2.532800in}}%
\pgfpathlineto{\pgfqpoint{1.368543in}{2.521796in}}%
\pgfpathlineto{\pgfqpoint{1.369408in}{2.476912in}}%
\pgfpathlineto{\pgfqpoint{1.370273in}{2.503841in}}%
\pgfpathlineto{\pgfqpoint{1.371138in}{2.437562in}}%
\pgfpathlineto{\pgfqpoint{1.372866in}{2.504826in}}%
\pgfpathlineto{\pgfqpoint{1.373729in}{2.458282in}}%
\pgfpathlineto{\pgfqpoint{1.374594in}{2.502181in}}%
\pgfpathlineto{\pgfqpoint{1.375459in}{2.433564in}}%
\pgfpathlineto{\pgfqpoint{1.376322in}{2.510974in}}%
\pgfpathlineto{\pgfqpoint{1.377187in}{2.485766in}}%
\pgfpathlineto{\pgfqpoint{1.378052in}{2.514355in}}%
\pgfpathlineto{\pgfqpoint{1.379781in}{2.456684in}}%
\pgfpathlineto{\pgfqpoint{1.380646in}{2.523456in}}%
\pgfpathlineto{\pgfqpoint{1.381509in}{2.473838in}}%
\pgfpathlineto{\pgfqpoint{1.382373in}{2.523517in}}%
\pgfpathlineto{\pgfqpoint{1.383238in}{2.497263in}}%
\pgfpathlineto{\pgfqpoint{1.384104in}{2.500644in}}%
\pgfpathlineto{\pgfqpoint{1.384968in}{2.542270in}}%
\pgfpathlineto{\pgfqpoint{1.386696in}{2.466951in}}%
\pgfpathlineto{\pgfqpoint{1.387562in}{2.531941in}}%
\pgfpathlineto{\pgfqpoint{1.388427in}{2.443986in}}%
\pgfpathlineto{\pgfqpoint{1.389291in}{2.557395in}}%
\pgfpathlineto{\pgfqpoint{1.391021in}{2.413398in}}%
\pgfpathlineto{\pgfqpoint{1.391886in}{2.555365in}}%
\pgfpathlineto{\pgfqpoint{1.393613in}{2.480447in}}%
\pgfpathlineto{\pgfqpoint{1.394476in}{2.490623in}}%
\pgfpathlineto{\pgfqpoint{1.395342in}{2.549095in}}%
\pgfpathlineto{\pgfqpoint{1.396208in}{2.440820in}}%
\pgfpathlineto{\pgfqpoint{1.397938in}{2.504549in}}%
\pgfpathlineto{\pgfqpoint{1.398803in}{2.506178in}}%
\pgfpathlineto{\pgfqpoint{1.399668in}{2.575840in}}%
\pgfpathlineto{\pgfqpoint{1.400534in}{2.484996in}}%
\pgfpathlineto{\pgfqpoint{1.401400in}{2.519520in}}%
\pgfpathlineto{\pgfqpoint{1.402265in}{2.492529in}}%
\pgfpathlineto{\pgfqpoint{1.403997in}{2.552292in}}%
\pgfpathlineto{\pgfqpoint{1.404863in}{2.549524in}}%
\pgfpathlineto{\pgfqpoint{1.405729in}{2.520934in}}%
\pgfpathlineto{\pgfqpoint{1.406594in}{2.584386in}}%
\pgfpathlineto{\pgfqpoint{1.408324in}{2.495911in}}%
\pgfpathlineto{\pgfqpoint{1.409190in}{2.571229in}}%
\pgfpathlineto{\pgfqpoint{1.410055in}{2.483798in}}%
\pgfpathlineto{\pgfqpoint{1.410921in}{2.584755in}}%
\pgfpathlineto{\pgfqpoint{1.411785in}{2.509437in}}%
\pgfpathlineto{\pgfqpoint{1.412651in}{2.555673in}}%
\pgfpathlineto{\pgfqpoint{1.413515in}{2.524808in}}%
\pgfpathlineto{\pgfqpoint{1.414379in}{2.569569in}}%
\pgfpathlineto{\pgfqpoint{1.416108in}{2.494127in}}%
\pgfpathlineto{\pgfqpoint{1.417838in}{2.592565in}}%
\pgfpathlineto{\pgfqpoint{1.420430in}{2.485519in}}%
\pgfpathlineto{\pgfqpoint{1.422159in}{2.513894in}}%
\pgfpathlineto{\pgfqpoint{1.423024in}{2.522840in}}%
\pgfpathlineto{\pgfqpoint{1.423884in}{2.509622in}}%
\pgfpathlineto{\pgfqpoint{1.424749in}{2.541408in}}%
\pgfpathlineto{\pgfqpoint{1.425614in}{2.494250in}}%
\pgfpathlineto{\pgfqpoint{1.428206in}{2.572828in}}%
\pgfpathlineto{\pgfqpoint{1.429937in}{2.484290in}}%
\pgfpathlineto{\pgfqpoint{1.430801in}{2.498984in}}%
\pgfpathlineto{\pgfqpoint{1.431667in}{2.473068in}}%
\pgfpathlineto{\pgfqpoint{1.433398in}{2.559363in}}%
\pgfpathlineto{\pgfqpoint{1.435130in}{2.445738in}}%
\pgfpathlineto{\pgfqpoint{1.436861in}{2.532923in}}%
\pgfpathlineto{\pgfqpoint{1.437726in}{2.485827in}}%
\pgfpathlineto{\pgfqpoint{1.438592in}{2.524531in}}%
\pgfpathlineto{\pgfqpoint{1.439455in}{2.508944in}}%
\pgfpathlineto{\pgfqpoint{1.441185in}{2.562190in}}%
\pgfpathlineto{\pgfqpoint{1.442915in}{2.521919in}}%
\pgfpathlineto{\pgfqpoint{1.443778in}{2.534031in}}%
\pgfpathlineto{\pgfqpoint{1.444643in}{2.471993in}}%
\pgfpathlineto{\pgfqpoint{1.445508in}{2.539258in}}%
\pgfpathlineto{\pgfqpoint{1.446371in}{2.518291in}}%
\pgfpathlineto{\pgfqpoint{1.447237in}{2.535815in}}%
\pgfpathlineto{\pgfqpoint{1.448967in}{2.472362in}}%
\pgfpathlineto{\pgfqpoint{1.449831in}{2.456991in}}%
\pgfpathlineto{\pgfqpoint{1.452425in}{2.482754in}}%
\pgfpathlineto{\pgfqpoint{1.454154in}{2.525054in}}%
\pgfpathlineto{\pgfqpoint{1.455884in}{2.479126in}}%
\pgfpathlineto{\pgfqpoint{1.457614in}{2.545898in}}%
\pgfpathlineto{\pgfqpoint{1.460209in}{2.482938in}}%
\pgfpathlineto{\pgfqpoint{1.461074in}{2.455208in}}%
\pgfpathlineto{\pgfqpoint{1.461941in}{2.465599in}}%
\pgfpathlineto{\pgfqpoint{1.462806in}{2.448076in}}%
\pgfpathlineto{\pgfqpoint{1.463672in}{2.556658in}}%
\pgfpathlineto{\pgfqpoint{1.465402in}{2.512205in}}%
\pgfpathlineto{\pgfqpoint{1.466268in}{2.491300in}}%
\pgfpathlineto{\pgfqpoint{1.467134in}{2.506517in}}%
\pgfpathlineto{\pgfqpoint{1.467999in}{2.468796in}}%
\pgfpathlineto{\pgfqpoint{1.468864in}{2.540548in}}%
\pgfpathlineto{\pgfqpoint{1.469729in}{2.523671in}}%
\pgfpathlineto{\pgfqpoint{1.471459in}{2.480049in}}%
\pgfpathlineto{\pgfqpoint{1.472326in}{2.510853in}}%
\pgfpathlineto{\pgfqpoint{1.473192in}{2.456193in}}%
\pgfpathlineto{\pgfqpoint{1.474056in}{2.485889in}}%
\pgfpathlineto{\pgfqpoint{1.474922in}{2.474269in}}%
\pgfpathlineto{\pgfqpoint{1.475788in}{2.514111in}}%
\pgfpathlineto{\pgfqpoint{1.476654in}{2.474084in}}%
\pgfpathlineto{\pgfqpoint{1.479253in}{2.593948in}}%
\pgfpathlineto{\pgfqpoint{1.480117in}{2.539442in}}%
\pgfpathlineto{\pgfqpoint{1.480982in}{2.580453in}}%
\pgfpathlineto{\pgfqpoint{1.482711in}{2.496465in}}%
\pgfpathlineto{\pgfqpoint{1.483575in}{2.508177in}}%
\pgfpathlineto{\pgfqpoint{1.484441in}{2.472362in}}%
\pgfpathlineto{\pgfqpoint{1.485307in}{2.481217in}}%
\pgfpathlineto{\pgfqpoint{1.486174in}{2.477835in}}%
\pgfpathlineto{\pgfqpoint{1.487040in}{2.458898in}}%
\pgfpathlineto{\pgfqpoint{1.487905in}{2.484414in}}%
\pgfpathlineto{\pgfqpoint{1.488770in}{2.463539in}}%
\pgfpathlineto{\pgfqpoint{1.490500in}{2.534953in}}%
\pgfpathlineto{\pgfqpoint{1.493097in}{2.472239in}}%
\pgfpathlineto{\pgfqpoint{1.494827in}{2.556474in}}%
\pgfpathlineto{\pgfqpoint{1.495693in}{2.460650in}}%
\pgfpathlineto{\pgfqpoint{1.496558in}{2.578916in}}%
\pgfpathlineto{\pgfqpoint{1.497424in}{2.561392in}}%
\pgfpathlineto{\pgfqpoint{1.498288in}{2.559486in}}%
\pgfpathlineto{\pgfqpoint{1.499153in}{2.591213in}}%
\pgfpathlineto{\pgfqpoint{1.500017in}{2.470764in}}%
\pgfpathlineto{\pgfqpoint{1.500882in}{2.548604in}}%
\pgfpathlineto{\pgfqpoint{1.501745in}{2.520751in}}%
\pgfpathlineto{\pgfqpoint{1.502609in}{2.524410in}}%
\pgfpathlineto{\pgfqpoint{1.503473in}{2.562991in}}%
\pgfpathlineto{\pgfqpoint{1.504339in}{2.553092in}}%
\pgfpathlineto{\pgfqpoint{1.505204in}{2.498309in}}%
\pgfpathlineto{\pgfqpoint{1.506068in}{2.556843in}}%
\pgfpathlineto{\pgfqpoint{1.506933in}{2.487980in}}%
\pgfpathlineto{\pgfqpoint{1.507798in}{2.543070in}}%
\pgfpathlineto{\pgfqpoint{1.508662in}{2.502368in}}%
\pgfpathlineto{\pgfqpoint{1.509524in}{2.596316in}}%
\pgfpathlineto{\pgfqpoint{1.510389in}{2.557274in}}%
\pgfpathlineto{\pgfqpoint{1.511254in}{2.595149in}}%
\pgfpathlineto{\pgfqpoint{1.512984in}{2.526470in}}%
\pgfpathlineto{\pgfqpoint{1.513848in}{2.562685in}}%
\pgfpathlineto{\pgfqpoint{1.515579in}{2.478758in}}%
\pgfpathlineto{\pgfqpoint{1.516443in}{2.510730in}}%
\pgfpathlineto{\pgfqpoint{1.517308in}{2.489978in}}%
\pgfpathlineto{\pgfqpoint{1.519038in}{2.544484in}}%
\pgfpathlineto{\pgfqpoint{1.520767in}{2.504888in}}%
\pgfpathlineto{\pgfqpoint{1.521632in}{2.560684in}}%
\pgfpathlineto{\pgfqpoint{1.523360in}{2.465415in}}%
\pgfpathlineto{\pgfqpoint{1.525091in}{2.574857in}}%
\pgfpathlineto{\pgfqpoint{1.525956in}{2.500277in}}%
\pgfpathlineto{\pgfqpoint{1.526821in}{2.501506in}}%
\pgfpathlineto{\pgfqpoint{1.527685in}{2.506794in}}%
\pgfpathlineto{\pgfqpoint{1.528549in}{2.544546in}}%
\pgfpathlineto{\pgfqpoint{1.529411in}{2.449613in}}%
\pgfpathlineto{\pgfqpoint{1.530275in}{2.581435in}}%
\pgfpathlineto{\pgfqpoint{1.531141in}{2.550909in}}%
\pgfpathlineto{\pgfqpoint{1.532005in}{2.530157in}}%
\pgfpathlineto{\pgfqpoint{1.533734in}{2.571044in}}%
\pgfpathlineto{\pgfqpoint{1.535463in}{2.535630in}}%
\pgfpathlineto{\pgfqpoint{1.536326in}{2.576148in}}%
\pgfpathlineto{\pgfqpoint{1.538921in}{2.482446in}}%
\pgfpathlineto{\pgfqpoint{1.539786in}{2.525547in}}%
\pgfpathlineto{\pgfqpoint{1.541516in}{2.492221in}}%
\pgfpathlineto{\pgfqpoint{1.542382in}{2.558993in}}%
\pgfpathlineto{\pgfqpoint{1.544111in}{2.482138in}}%
\pgfpathlineto{\pgfqpoint{1.544975in}{2.546634in}}%
\pgfpathlineto{\pgfqpoint{1.545841in}{2.526376in}}%
\pgfpathlineto{\pgfqpoint{1.546705in}{2.464737in}}%
\pgfpathlineto{\pgfqpoint{1.548435in}{2.522963in}}%
\pgfpathlineto{\pgfqpoint{1.549300in}{2.480293in}}%
\pgfpathlineto{\pgfqpoint{1.550163in}{2.524993in}}%
\pgfpathlineto{\pgfqpoint{1.551028in}{2.502612in}}%
\pgfpathlineto{\pgfqpoint{1.551893in}{2.580022in}}%
\pgfpathlineto{\pgfqpoint{1.552758in}{2.480940in}}%
\pgfpathlineto{\pgfqpoint{1.553624in}{2.537536in}}%
\pgfpathlineto{\pgfqpoint{1.554488in}{2.523948in}}%
\pgfpathlineto{\pgfqpoint{1.555353in}{2.496311in}}%
\pgfpathlineto{\pgfqpoint{1.556218in}{2.564651in}}%
\pgfpathlineto{\pgfqpoint{1.557945in}{2.447768in}}%
\pgfpathlineto{\pgfqpoint{1.558810in}{2.471808in}}%
\pgfpathlineto{\pgfqpoint{1.559676in}{2.526899in}}%
\pgfpathlineto{\pgfqpoint{1.560542in}{2.510912in}}%
\pgfpathlineto{\pgfqpoint{1.562273in}{2.451488in}}%
\pgfpathlineto{\pgfqpoint{1.564872in}{2.531325in}}%
\pgfpathlineto{\pgfqpoint{1.565738in}{2.475313in}}%
\pgfpathlineto{\pgfqpoint{1.566603in}{2.499169in}}%
\pgfpathlineto{\pgfqpoint{1.567467in}{2.453056in}}%
\pgfpathlineto{\pgfqpoint{1.568332in}{2.461541in}}%
\pgfpathlineto{\pgfqpoint{1.570061in}{2.556595in}}%
\pgfpathlineto{\pgfqpoint{1.572656in}{2.504211in}}%
\pgfpathlineto{\pgfqpoint{1.574383in}{2.516139in}}%
\pgfpathlineto{\pgfqpoint{1.575249in}{2.557826in}}%
\pgfpathlineto{\pgfqpoint{1.576978in}{2.518845in}}%
\pgfpathlineto{\pgfqpoint{1.577842in}{2.571291in}}%
\pgfpathlineto{\pgfqpoint{1.578706in}{2.508023in}}%
\pgfpathlineto{\pgfqpoint{1.579570in}{2.521919in}}%
\pgfpathlineto{\pgfqpoint{1.580437in}{2.571167in}}%
\pgfpathlineto{\pgfqpoint{1.581303in}{2.518014in}}%
\pgfpathlineto{\pgfqpoint{1.583036in}{2.545713in}}%
\pgfpathlineto{\pgfqpoint{1.583903in}{2.472116in}}%
\pgfpathlineto{\pgfqpoint{1.585634in}{2.504703in}}%
\pgfpathlineto{\pgfqpoint{1.586500in}{2.456745in}}%
\pgfpathlineto{\pgfqpoint{1.589098in}{2.578115in}}%
\pgfpathlineto{\pgfqpoint{1.592558in}{2.513373in}}%
\pgfpathlineto{\pgfqpoint{1.593423in}{2.437746in}}%
\pgfpathlineto{\pgfqpoint{1.595155in}{2.552476in}}%
\pgfpathlineto{\pgfqpoint{1.596021in}{2.514540in}}%
\pgfpathlineto{\pgfqpoint{1.596886in}{2.545959in}}%
\pgfpathlineto{\pgfqpoint{1.597751in}{2.462033in}}%
\pgfpathlineto{\pgfqpoint{1.598615in}{2.550201in}}%
\pgfpathlineto{\pgfqpoint{1.599480in}{2.533477in}}%
\pgfpathlineto{\pgfqpoint{1.600345in}{2.569015in}}%
\pgfpathlineto{\pgfqpoint{1.601207in}{2.548849in}}%
\pgfpathlineto{\pgfqpoint{1.602937in}{2.490407in}}%
\pgfpathlineto{\pgfqpoint{1.603803in}{2.548541in}}%
\pgfpathlineto{\pgfqpoint{1.604669in}{2.500275in}}%
\pgfpathlineto{\pgfqpoint{1.606401in}{2.543684in}}%
\pgfpathlineto{\pgfqpoint{1.607265in}{2.521242in}}%
\pgfpathlineto{\pgfqpoint{1.608995in}{2.567909in}}%
\pgfpathlineto{\pgfqpoint{1.610725in}{2.447583in}}%
\pgfpathlineto{\pgfqpoint{1.611590in}{2.499107in}}%
\pgfpathlineto{\pgfqpoint{1.612456in}{2.452779in}}%
\pgfpathlineto{\pgfqpoint{1.615051in}{2.592625in}}%
\pgfpathlineto{\pgfqpoint{1.615917in}{2.516015in}}%
\pgfpathlineto{\pgfqpoint{1.618510in}{2.560407in}}%
\pgfpathlineto{\pgfqpoint{1.619375in}{2.481368in}}%
\pgfpathlineto{\pgfqpoint{1.621104in}{2.557210in}}%
\pgfpathlineto{\pgfqpoint{1.621969in}{2.490161in}}%
\pgfpathlineto{\pgfqpoint{1.622835in}{2.536490in}}%
\pgfpathlineto{\pgfqpoint{1.623700in}{2.536305in}}%
\pgfpathlineto{\pgfqpoint{1.624564in}{2.527759in}}%
\pgfpathlineto{\pgfqpoint{1.625428in}{2.554136in}}%
\pgfpathlineto{\pgfqpoint{1.626293in}{2.545375in}}%
\pgfpathlineto{\pgfqpoint{1.627159in}{2.509868in}}%
\pgfpathlineto{\pgfqpoint{1.628022in}{2.532002in}}%
\pgfpathlineto{\pgfqpoint{1.628886in}{2.490376in}}%
\pgfpathlineto{\pgfqpoint{1.630617in}{2.552661in}}%
\pgfpathlineto{\pgfqpoint{1.631481in}{2.547342in}}%
\pgfpathlineto{\pgfqpoint{1.632344in}{2.523210in}}%
\pgfpathlineto{\pgfqpoint{1.633210in}{2.562991in}}%
\pgfpathlineto{\pgfqpoint{1.634073in}{2.518168in}}%
\pgfpathlineto{\pgfqpoint{1.634936in}{2.527022in}}%
\pgfpathlineto{\pgfqpoint{1.635803in}{2.512880in}}%
\pgfpathlineto{\pgfqpoint{1.637533in}{2.601235in}}%
\pgfpathlineto{\pgfqpoint{1.638397in}{2.523733in}}%
\pgfpathlineto{\pgfqpoint{1.639263in}{2.545528in}}%
\pgfpathlineto{\pgfqpoint{1.640129in}{2.581620in}}%
\pgfpathlineto{\pgfqpoint{1.643590in}{2.477466in}}%
\pgfpathlineto{\pgfqpoint{1.644456in}{2.554567in}}%
\pgfpathlineto{\pgfqpoint{1.645321in}{2.502612in}}%
\pgfpathlineto{\pgfqpoint{1.646187in}{2.505994in}}%
\pgfpathlineto{\pgfqpoint{1.647053in}{2.579591in}}%
\pgfpathlineto{\pgfqpoint{1.647918in}{2.570860in}}%
\pgfpathlineto{\pgfqpoint{1.648784in}{2.467382in}}%
\pgfpathlineto{\pgfqpoint{1.649645in}{2.526345in}}%
\pgfpathlineto{\pgfqpoint{1.650509in}{2.522411in}}%
\pgfpathlineto{\pgfqpoint{1.651372in}{2.532125in}}%
\pgfpathlineto{\pgfqpoint{1.652236in}{2.555920in}}%
\pgfpathlineto{\pgfqpoint{1.653101in}{2.516508in}}%
\pgfpathlineto{\pgfqpoint{1.653965in}{2.589612in}}%
\pgfpathlineto{\pgfqpoint{1.654830in}{2.512757in}}%
\pgfpathlineto{\pgfqpoint{1.655694in}{2.524931in}}%
\pgfpathlineto{\pgfqpoint{1.656558in}{2.545159in}}%
\pgfpathlineto{\pgfqpoint{1.657422in}{2.538858in}}%
\pgfpathlineto{\pgfqpoint{1.658286in}{2.523456in}}%
\pgfpathlineto{\pgfqpoint{1.659151in}{2.566372in}}%
\pgfpathlineto{\pgfqpoint{1.660879in}{2.469902in}}%
\pgfpathlineto{\pgfqpoint{1.661745in}{2.543253in}}%
\pgfpathlineto{\pgfqpoint{1.662610in}{2.497571in}}%
\pgfpathlineto{\pgfqpoint{1.664342in}{2.546881in}}%
\pgfpathlineto{\pgfqpoint{1.665207in}{2.473530in}}%
\pgfpathlineto{\pgfqpoint{1.666935in}{2.508791in}}%
\pgfpathlineto{\pgfqpoint{1.667800in}{2.519397in}}%
\pgfpathlineto{\pgfqpoint{1.668666in}{2.543745in}}%
\pgfpathlineto{\pgfqpoint{1.669531in}{2.467873in}}%
\pgfpathlineto{\pgfqpoint{1.670393in}{2.500891in}}%
\pgfpathlineto{\pgfqpoint{1.671259in}{2.443157in}}%
\pgfpathlineto{\pgfqpoint{1.673852in}{2.542270in}}%
\pgfpathlineto{\pgfqpoint{1.674716in}{2.449551in}}%
\pgfpathlineto{\pgfqpoint{1.676445in}{2.591703in}}%
\pgfpathlineto{\pgfqpoint{1.678175in}{2.498677in}}%
\pgfpathlineto{\pgfqpoint{1.679039in}{2.475067in}}%
\pgfpathlineto{\pgfqpoint{1.679905in}{2.483859in}}%
\pgfpathlineto{\pgfqpoint{1.680770in}{2.524931in}}%
\pgfpathlineto{\pgfqpoint{1.681635in}{2.524685in}}%
\pgfpathlineto{\pgfqpoint{1.683364in}{2.482046in}}%
\pgfpathlineto{\pgfqpoint{1.684229in}{2.514909in}}%
\pgfpathlineto{\pgfqpoint{1.685094in}{2.607382in}}%
\pgfpathlineto{\pgfqpoint{1.686825in}{2.507715in}}%
\pgfpathlineto{\pgfqpoint{1.687689in}{2.520013in}}%
\pgfpathlineto{\pgfqpoint{1.688554in}{2.416441in}}%
\pgfpathlineto{\pgfqpoint{1.689420in}{2.540056in}}%
\pgfpathlineto{\pgfqpoint{1.690285in}{2.524562in}}%
\pgfpathlineto{\pgfqpoint{1.692015in}{2.456376in}}%
\pgfpathlineto{\pgfqpoint{1.692880in}{2.473653in}}%
\pgfpathlineto{\pgfqpoint{1.694607in}{2.549893in}}%
\pgfpathlineto{\pgfqpoint{1.695472in}{2.466490in}}%
\pgfpathlineto{\pgfqpoint{1.697201in}{2.527820in}}%
\pgfpathlineto{\pgfqpoint{1.698064in}{2.509406in}}%
\pgfpathlineto{\pgfqpoint{1.698929in}{2.563850in}}%
\pgfpathlineto{\pgfqpoint{1.700660in}{2.499354in}}%
\pgfpathlineto{\pgfqpoint{1.702389in}{2.446139in}}%
\pgfpathlineto{\pgfqpoint{1.703252in}{2.463631in}}%
\pgfpathlineto{\pgfqpoint{1.704117in}{2.459573in}}%
\pgfpathlineto{\pgfqpoint{1.704981in}{2.473530in}}%
\pgfpathlineto{\pgfqpoint{1.705846in}{2.449182in}}%
\pgfpathlineto{\pgfqpoint{1.708439in}{2.545036in}}%
\pgfpathlineto{\pgfqpoint{1.710169in}{2.497324in}}%
\pgfpathlineto{\pgfqpoint{1.711035in}{2.526283in}}%
\pgfpathlineto{\pgfqpoint{1.711900in}{2.520965in}}%
\pgfpathlineto{\pgfqpoint{1.712763in}{2.464553in}}%
\pgfpathlineto{\pgfqpoint{1.713627in}{2.533293in}}%
\pgfpathlineto{\pgfqpoint{1.714491in}{2.483983in}}%
\pgfpathlineto{\pgfqpoint{1.715355in}{2.519766in}}%
\pgfpathlineto{\pgfqpoint{1.716220in}{2.509622in}}%
\pgfpathlineto{\pgfqpoint{1.717084in}{2.447614in}}%
\pgfpathlineto{\pgfqpoint{1.717949in}{2.518106in}}%
\pgfpathlineto{\pgfqpoint{1.718815in}{2.506363in}}%
\pgfpathlineto{\pgfqpoint{1.721409in}{2.477158in}}%
\pgfpathlineto{\pgfqpoint{1.722274in}{2.529175in}}%
\pgfpathlineto{\pgfqpoint{1.723138in}{2.509806in}}%
\pgfpathlineto{\pgfqpoint{1.725732in}{2.559178in}}%
\pgfpathlineto{\pgfqpoint{1.727461in}{2.496649in}}%
\pgfpathlineto{\pgfqpoint{1.728326in}{2.579775in}}%
\pgfpathlineto{\pgfqpoint{1.729191in}{2.561084in}}%
\pgfpathlineto{\pgfqpoint{1.730056in}{2.524685in}}%
\pgfpathlineto{\pgfqpoint{1.730922in}{2.548572in}}%
\pgfpathlineto{\pgfqpoint{1.731786in}{2.529788in}}%
\pgfpathlineto{\pgfqpoint{1.732650in}{2.572704in}}%
\pgfpathlineto{\pgfqpoint{1.733516in}{2.560653in}}%
\pgfpathlineto{\pgfqpoint{1.736112in}{2.616390in}}%
\pgfpathlineto{\pgfqpoint{1.738706in}{2.513865in}}%
\pgfpathlineto{\pgfqpoint{1.739570in}{2.553154in}}%
\pgfpathlineto{\pgfqpoint{1.742164in}{2.498617in}}%
\pgfpathlineto{\pgfqpoint{1.743026in}{2.518229in}}%
\pgfpathlineto{\pgfqpoint{1.743891in}{2.454716in}}%
\pgfpathlineto{\pgfqpoint{1.744756in}{2.525793in}}%
\pgfpathlineto{\pgfqpoint{1.745620in}{2.463385in}}%
\pgfpathlineto{\pgfqpoint{1.746486in}{2.465599in}}%
\pgfpathlineto{\pgfqpoint{1.747352in}{2.541901in}}%
\pgfpathlineto{\pgfqpoint{1.748217in}{2.493021in}}%
\pgfpathlineto{\pgfqpoint{1.750814in}{2.554013in}}%
\pgfpathlineto{\pgfqpoint{1.751679in}{2.497878in}}%
\pgfpathlineto{\pgfqpoint{1.752545in}{2.511682in}}%
\pgfpathlineto{\pgfqpoint{1.753409in}{2.501137in}}%
\pgfpathlineto{\pgfqpoint{1.755138in}{2.568248in}}%
\pgfpathlineto{\pgfqpoint{1.756004in}{2.524870in}}%
\pgfpathlineto{\pgfqpoint{1.756870in}{2.595331in}}%
\pgfpathlineto{\pgfqpoint{1.757735in}{2.543376in}}%
\pgfpathlineto{\pgfqpoint{1.758602in}{2.561146in}}%
\pgfpathlineto{\pgfqpoint{1.759468in}{2.461263in}}%
\pgfpathlineto{\pgfqpoint{1.760334in}{2.542270in}}%
\pgfpathlineto{\pgfqpoint{1.761200in}{2.535014in}}%
\pgfpathlineto{\pgfqpoint{1.762931in}{2.571537in}}%
\pgfpathlineto{\pgfqpoint{1.763797in}{2.567971in}}%
\pgfpathlineto{\pgfqpoint{1.764663in}{2.472209in}}%
\pgfpathlineto{\pgfqpoint{1.765529in}{2.490500in}}%
\pgfpathlineto{\pgfqpoint{1.766394in}{2.449305in}}%
\pgfpathlineto{\pgfqpoint{1.768123in}{2.529665in}}%
\pgfpathlineto{\pgfqpoint{1.768988in}{2.514817in}}%
\pgfpathlineto{\pgfqpoint{1.771582in}{2.563483in}}%
\pgfpathlineto{\pgfqpoint{1.772446in}{2.537782in}}%
\pgfpathlineto{\pgfqpoint{1.773309in}{2.573751in}}%
\pgfpathlineto{\pgfqpoint{1.775039in}{2.521488in}}%
\pgfpathlineto{\pgfqpoint{1.775903in}{2.547127in}}%
\pgfpathlineto{\pgfqpoint{1.776768in}{2.535630in}}%
\pgfpathlineto{\pgfqpoint{1.777633in}{2.552292in}}%
\pgfpathlineto{\pgfqpoint{1.778496in}{2.549187in}}%
\pgfpathlineto{\pgfqpoint{1.781086in}{2.499631in}}%
\pgfpathlineto{\pgfqpoint{1.781951in}{2.544053in}}%
\pgfpathlineto{\pgfqpoint{1.782816in}{2.533231in}}%
\pgfpathlineto{\pgfqpoint{1.783681in}{2.476450in}}%
\pgfpathlineto{\pgfqpoint{1.784545in}{2.485827in}}%
\pgfpathlineto{\pgfqpoint{1.785410in}{2.498984in}}%
\pgfpathlineto{\pgfqpoint{1.786275in}{2.448137in}}%
\pgfpathlineto{\pgfqpoint{1.787138in}{2.509437in}}%
\pgfpathlineto{\pgfqpoint{1.788003in}{2.475313in}}%
\pgfpathlineto{\pgfqpoint{1.788868in}{2.550509in}}%
\pgfpathlineto{\pgfqpoint{1.789732in}{2.486443in}}%
\pgfpathlineto{\pgfqpoint{1.792326in}{2.530281in}}%
\pgfpathlineto{\pgfqpoint{1.793189in}{2.525054in}}%
\pgfpathlineto{\pgfqpoint{1.794054in}{2.558010in}}%
\pgfpathlineto{\pgfqpoint{1.794918in}{2.438116in}}%
\pgfpathlineto{\pgfqpoint{1.798377in}{2.578423in}}%
\pgfpathlineto{\pgfqpoint{1.800107in}{2.517922in}}%
\pgfpathlineto{\pgfqpoint{1.800973in}{2.558932in}}%
\pgfpathlineto{\pgfqpoint{1.802704in}{2.529726in}}%
\pgfpathlineto{\pgfqpoint{1.803570in}{2.550201in}}%
\pgfpathlineto{\pgfqpoint{1.804436in}{2.509314in}}%
\pgfpathlineto{\pgfqpoint{1.805299in}{2.516446in}}%
\pgfpathlineto{\pgfqpoint{1.807031in}{2.538765in}}%
\pgfpathlineto{\pgfqpoint{1.807897in}{2.593486in}}%
\pgfpathlineto{\pgfqpoint{1.809628in}{2.546881in}}%
\pgfpathlineto{\pgfqpoint{1.810493in}{2.561269in}}%
\pgfpathlineto{\pgfqpoint{1.811358in}{2.497447in}}%
\pgfpathlineto{\pgfqpoint{1.813089in}{2.555981in}}%
\pgfpathlineto{\pgfqpoint{1.813954in}{2.530465in}}%
\pgfpathlineto{\pgfqpoint{1.814820in}{2.539073in}}%
\pgfpathlineto{\pgfqpoint{1.815683in}{2.481093in}}%
\pgfpathlineto{\pgfqpoint{1.816549in}{2.505196in}}%
\pgfpathlineto{\pgfqpoint{1.817413in}{2.448630in}}%
\pgfpathlineto{\pgfqpoint{1.818276in}{2.530835in}}%
\pgfpathlineto{\pgfqpoint{1.819140in}{2.515433in}}%
\pgfpathlineto{\pgfqpoint{1.820870in}{2.492221in}}%
\pgfpathlineto{\pgfqpoint{1.821736in}{2.506209in}}%
\pgfpathlineto{\pgfqpoint{1.822603in}{2.570000in}}%
\pgfpathlineto{\pgfqpoint{1.824334in}{2.476727in}}%
\pgfpathlineto{\pgfqpoint{1.825198in}{2.502735in}}%
\pgfpathlineto{\pgfqpoint{1.827794in}{2.567786in}}%
\pgfpathlineto{\pgfqpoint{1.829522in}{2.543622in}}%
\pgfpathlineto{\pgfqpoint{1.830388in}{2.547250in}}%
\pgfpathlineto{\pgfqpoint{1.831253in}{2.598774in}}%
\pgfpathlineto{\pgfqpoint{1.832979in}{2.550632in}}%
\pgfpathlineto{\pgfqpoint{1.833845in}{2.550878in}}%
\pgfpathlineto{\pgfqpoint{1.835577in}{2.484721in}}%
\pgfpathlineto{\pgfqpoint{1.836442in}{2.543193in}}%
\pgfpathlineto{\pgfqpoint{1.837307in}{2.486012in}}%
\pgfpathlineto{\pgfqpoint{1.838171in}{2.510514in}}%
\pgfpathlineto{\pgfqpoint{1.839035in}{2.570800in}}%
\pgfpathlineto{\pgfqpoint{1.840763in}{2.530681in}}%
\pgfpathlineto{\pgfqpoint{1.841628in}{2.560101in}}%
\pgfpathlineto{\pgfqpoint{1.843359in}{2.469289in}}%
\pgfpathlineto{\pgfqpoint{1.844224in}{2.589060in}}%
\pgfpathlineto{\pgfqpoint{1.845955in}{2.507777in}}%
\pgfpathlineto{\pgfqpoint{1.847684in}{2.547866in}}%
\pgfpathlineto{\pgfqpoint{1.848546in}{2.508454in}}%
\pgfpathlineto{\pgfqpoint{1.849412in}{2.554814in}}%
\pgfpathlineto{\pgfqpoint{1.851141in}{2.485889in}}%
\pgfpathlineto{\pgfqpoint{1.852007in}{2.499354in}}%
\pgfpathlineto{\pgfqpoint{1.852870in}{2.477343in}}%
\pgfpathlineto{\pgfqpoint{1.853735in}{2.533785in}}%
\pgfpathlineto{\pgfqpoint{1.854601in}{2.483675in}}%
\pgfpathlineto{\pgfqpoint{1.856331in}{2.568709in}}%
\pgfpathlineto{\pgfqpoint{1.858062in}{2.471870in}}%
\pgfpathlineto{\pgfqpoint{1.858925in}{2.485581in}}%
\pgfpathlineto{\pgfqpoint{1.859790in}{2.547250in}}%
\pgfpathlineto{\pgfqpoint{1.861521in}{2.488901in}}%
\pgfpathlineto{\pgfqpoint{1.864120in}{2.573074in}}%
\pgfpathlineto{\pgfqpoint{1.865845in}{2.553952in}}%
\pgfpathlineto{\pgfqpoint{1.866708in}{2.563789in}}%
\pgfpathlineto{\pgfqpoint{1.867573in}{2.524746in}}%
\pgfpathlineto{\pgfqpoint{1.868440in}{2.526222in}}%
\pgfpathlineto{\pgfqpoint{1.869304in}{2.567201in}}%
\pgfpathlineto{\pgfqpoint{1.870168in}{2.533108in}}%
\pgfpathlineto{\pgfqpoint{1.871032in}{2.546942in}}%
\pgfpathlineto{\pgfqpoint{1.871897in}{2.469964in}}%
\pgfpathlineto{\pgfqpoint{1.872762in}{2.577500in}}%
\pgfpathlineto{\pgfqpoint{1.873626in}{2.457851in}}%
\pgfpathlineto{\pgfqpoint{1.874493in}{2.517552in}}%
\pgfpathlineto{\pgfqpoint{1.876224in}{2.411522in}}%
\pgfpathlineto{\pgfqpoint{1.877953in}{2.467873in}}%
\pgfpathlineto{\pgfqpoint{1.878816in}{2.454839in}}%
\pgfpathlineto{\pgfqpoint{1.879681in}{2.437500in}}%
\pgfpathlineto{\pgfqpoint{1.880547in}{2.555612in}}%
\pgfpathlineto{\pgfqpoint{1.881412in}{2.481892in}}%
\pgfpathlineto{\pgfqpoint{1.882278in}{2.492283in}}%
\pgfpathlineto{\pgfqpoint{1.883143in}{2.490469in}}%
\pgfpathlineto{\pgfqpoint{1.884008in}{2.479985in}}%
\pgfpathlineto{\pgfqpoint{1.886600in}{2.607382in}}%
\pgfpathlineto{\pgfqpoint{1.888327in}{2.556073in}}%
\pgfpathlineto{\pgfqpoint{1.889190in}{2.573997in}}%
\pgfpathlineto{\pgfqpoint{1.890054in}{2.630686in}}%
\pgfpathlineto{\pgfqpoint{1.890918in}{2.542886in}}%
\pgfpathlineto{\pgfqpoint{1.891784in}{2.558380in}}%
\pgfpathlineto{\pgfqpoint{1.892650in}{2.552846in}}%
\pgfpathlineto{\pgfqpoint{1.893515in}{2.509745in}}%
\pgfpathlineto{\pgfqpoint{1.894381in}{2.537044in}}%
\pgfpathlineto{\pgfqpoint{1.895246in}{2.471316in}}%
\pgfpathlineto{\pgfqpoint{1.896112in}{2.488655in}}%
\pgfpathlineto{\pgfqpoint{1.896975in}{2.454593in}}%
\pgfpathlineto{\pgfqpoint{1.897841in}{2.558932in}}%
\pgfpathlineto{\pgfqpoint{1.898707in}{2.497940in}}%
\pgfpathlineto{\pgfqpoint{1.899572in}{2.535507in}}%
\pgfpathlineto{\pgfqpoint{1.900437in}{2.507838in}}%
\pgfpathlineto{\pgfqpoint{1.901303in}{2.569815in}}%
\pgfpathlineto{\pgfqpoint{1.903034in}{2.469779in}}%
\pgfpathlineto{\pgfqpoint{1.903899in}{2.507592in}}%
\pgfpathlineto{\pgfqpoint{1.904764in}{2.496034in}}%
\pgfpathlineto{\pgfqpoint{1.906494in}{2.579652in}}%
\pgfpathlineto{\pgfqpoint{1.907360in}{2.453179in}}%
\pgfpathlineto{\pgfqpoint{1.908225in}{2.587706in}}%
\pgfpathlineto{\pgfqpoint{1.909953in}{2.515708in}}%
\pgfpathlineto{\pgfqpoint{1.910818in}{2.562498in}}%
\pgfpathlineto{\pgfqpoint{1.911684in}{2.495541in}}%
\pgfpathlineto{\pgfqpoint{1.912550in}{2.571966in}}%
\pgfpathlineto{\pgfqpoint{1.915146in}{2.508944in}}%
\pgfpathlineto{\pgfqpoint{1.916011in}{2.524993in}}%
\pgfpathlineto{\pgfqpoint{1.916877in}{2.478541in}}%
\pgfpathlineto{\pgfqpoint{1.918607in}{2.558993in}}%
\pgfpathlineto{\pgfqpoint{1.920335in}{2.533354in}}%
\pgfpathlineto{\pgfqpoint{1.921200in}{2.478695in}}%
\pgfpathlineto{\pgfqpoint{1.922066in}{2.543684in}}%
\pgfpathlineto{\pgfqpoint{1.923796in}{2.479924in}}%
\pgfpathlineto{\pgfqpoint{1.924661in}{2.536120in}}%
\pgfpathlineto{\pgfqpoint{1.925527in}{2.483367in}}%
\pgfpathlineto{\pgfqpoint{1.927259in}{2.521488in}}%
\pgfpathlineto{\pgfqpoint{1.928122in}{2.519705in}}%
\pgfpathlineto{\pgfqpoint{1.928985in}{2.510604in}}%
\pgfpathlineto{\pgfqpoint{1.930715in}{2.524469in}}%
\pgfpathlineto{\pgfqpoint{1.931580in}{2.596499in}}%
\pgfpathlineto{\pgfqpoint{1.933312in}{2.499107in}}%
\pgfpathlineto{\pgfqpoint{1.934179in}{2.502858in}}%
\pgfpathlineto{\pgfqpoint{1.936775in}{2.465107in}}%
\pgfpathlineto{\pgfqpoint{1.937642in}{2.576148in}}%
\pgfpathlineto{\pgfqpoint{1.938507in}{2.495849in}}%
\pgfpathlineto{\pgfqpoint{1.939373in}{2.539073in}}%
\pgfpathlineto{\pgfqpoint{1.941967in}{2.491546in}}%
\pgfpathlineto{\pgfqpoint{1.942833in}{2.475990in}}%
\pgfpathlineto{\pgfqpoint{1.943696in}{2.561946in}}%
\pgfpathlineto{\pgfqpoint{1.944561in}{2.502060in}}%
\pgfpathlineto{\pgfqpoint{1.945425in}{2.508823in}}%
\pgfpathlineto{\pgfqpoint{1.946290in}{2.498617in}}%
\pgfpathlineto{\pgfqpoint{1.947154in}{2.509070in}}%
\pgfpathlineto{\pgfqpoint{1.948018in}{2.489824in}}%
\pgfpathlineto{\pgfqpoint{1.948883in}{2.515833in}}%
\pgfpathlineto{\pgfqpoint{1.949748in}{2.515310in}}%
\pgfpathlineto{\pgfqpoint{1.953207in}{2.469104in}}%
\pgfpathlineto{\pgfqpoint{1.954072in}{2.548112in}}%
\pgfpathlineto{\pgfqpoint{1.954936in}{2.475713in}}%
\pgfpathlineto{\pgfqpoint{1.956666in}{2.599267in}}%
\pgfpathlineto{\pgfqpoint{1.957531in}{2.548112in}}%
\pgfpathlineto{\pgfqpoint{1.958396in}{2.548420in}}%
\pgfpathlineto{\pgfqpoint{1.959262in}{2.527668in}}%
\pgfpathlineto{\pgfqpoint{1.960126in}{2.541349in}}%
\pgfpathlineto{\pgfqpoint{1.962723in}{2.484229in}}%
\pgfpathlineto{\pgfqpoint{1.963588in}{2.546021in}}%
\pgfpathlineto{\pgfqpoint{1.964454in}{2.490469in}}%
\pgfpathlineto{\pgfqpoint{1.965319in}{2.564158in}}%
\pgfpathlineto{\pgfqpoint{1.967048in}{2.534768in}}%
\pgfpathlineto{\pgfqpoint{1.967912in}{2.528621in}}%
\pgfpathlineto{\pgfqpoint{1.968778in}{2.541931in}}%
\pgfpathlineto{\pgfqpoint{1.970509in}{2.516631in}}%
\pgfpathlineto{\pgfqpoint{1.971375in}{2.543807in}}%
\pgfpathlineto{\pgfqpoint{1.972240in}{2.483921in}}%
\pgfpathlineto{\pgfqpoint{1.973106in}{2.546329in}}%
\pgfpathlineto{\pgfqpoint{1.973972in}{2.426526in}}%
\pgfpathlineto{\pgfqpoint{1.974835in}{2.477527in}}%
\pgfpathlineto{\pgfqpoint{1.975702in}{2.605661in}}%
\pgfpathlineto{\pgfqpoint{1.977433in}{2.473899in}}%
\pgfpathlineto{\pgfqpoint{1.978298in}{2.499661in}}%
\pgfpathlineto{\pgfqpoint{1.979162in}{2.470087in}}%
\pgfpathlineto{\pgfqpoint{1.980028in}{2.499415in}}%
\pgfpathlineto{\pgfqpoint{1.980894in}{2.580391in}}%
\pgfpathlineto{\pgfqpoint{1.981760in}{2.505011in}}%
\pgfpathlineto{\pgfqpoint{1.982625in}{2.518907in}}%
\pgfpathlineto{\pgfqpoint{1.983491in}{2.492960in}}%
\pgfpathlineto{\pgfqpoint{1.985220in}{2.532556in}}%
\pgfpathlineto{\pgfqpoint{1.986085in}{2.506794in}}%
\pgfpathlineto{\pgfqpoint{1.986949in}{2.525793in}}%
\pgfpathlineto{\pgfqpoint{1.987814in}{2.479372in}}%
\pgfpathlineto{\pgfqpoint{1.988674in}{2.544176in}}%
\pgfpathlineto{\pgfqpoint{1.989540in}{2.514909in}}%
\pgfpathlineto{\pgfqpoint{1.990405in}{2.580299in}}%
\pgfpathlineto{\pgfqpoint{1.991267in}{2.527453in}}%
\pgfpathlineto{\pgfqpoint{1.992997in}{2.564651in}}%
\pgfpathlineto{\pgfqpoint{1.996456in}{2.479864in}}%
\pgfpathlineto{\pgfqpoint{1.998186in}{2.563175in}}%
\pgfpathlineto{\pgfqpoint{1.999051in}{2.520813in}}%
\pgfpathlineto{\pgfqpoint{1.999915in}{2.555583in}}%
\pgfpathlineto{\pgfqpoint{2.000777in}{2.501937in}}%
\pgfpathlineto{\pgfqpoint{2.001642in}{2.536369in}}%
\pgfpathlineto{\pgfqpoint{2.002508in}{2.488195in}}%
\pgfpathlineto{\pgfqpoint{2.003373in}{2.507410in}}%
\pgfpathlineto{\pgfqpoint{2.004238in}{2.491854in}}%
\pgfpathlineto{\pgfqpoint{2.005968in}{2.508947in}}%
\pgfpathlineto{\pgfqpoint{2.006833in}{2.539196in}}%
\pgfpathlineto{\pgfqpoint{2.008563in}{2.502920in}}%
\pgfpathlineto{\pgfqpoint{2.009429in}{2.576548in}}%
\pgfpathlineto{\pgfqpoint{2.010294in}{2.567111in}}%
\pgfpathlineto{\pgfqpoint{2.011158in}{2.580206in}}%
\pgfpathlineto{\pgfqpoint{2.012887in}{2.541410in}}%
\pgfpathlineto{\pgfqpoint{2.013752in}{2.549957in}}%
\pgfpathlineto{\pgfqpoint{2.015482in}{2.479926in}}%
\pgfpathlineto{\pgfqpoint{2.016347in}{2.494989in}}%
\pgfpathlineto{\pgfqpoint{2.017211in}{2.480786in}}%
\pgfpathlineto{\pgfqpoint{2.018077in}{2.505503in}}%
\pgfpathlineto{\pgfqpoint{2.018939in}{2.488103in}}%
\pgfpathlineto{\pgfqpoint{2.019804in}{2.544853in}}%
\pgfpathlineto{\pgfqpoint{2.020669in}{2.524256in}}%
\pgfpathlineto{\pgfqpoint{2.021535in}{2.540733in}}%
\pgfpathlineto{\pgfqpoint{2.022400in}{2.480663in}}%
\pgfpathlineto{\pgfqpoint{2.024995in}{2.586415in}}%
\pgfpathlineto{\pgfqpoint{2.025860in}{2.463631in}}%
\pgfpathlineto{\pgfqpoint{2.026727in}{2.570616in}}%
\pgfpathlineto{\pgfqpoint{2.029320in}{2.493268in}}%
\pgfpathlineto{\pgfqpoint{2.030186in}{2.566988in}}%
\pgfpathlineto{\pgfqpoint{2.031052in}{2.475621in}}%
\pgfpathlineto{\pgfqpoint{2.032782in}{2.544299in}}%
\pgfpathlineto{\pgfqpoint{2.033646in}{2.518599in}}%
\pgfpathlineto{\pgfqpoint{2.034510in}{2.559732in}}%
\pgfpathlineto{\pgfqpoint{2.035374in}{2.538950in}}%
\pgfpathlineto{\pgfqpoint{2.036239in}{2.614084in}}%
\pgfpathlineto{\pgfqpoint{2.037104in}{2.594040in}}%
\pgfpathlineto{\pgfqpoint{2.040565in}{2.480109in}}%
\pgfpathlineto{\pgfqpoint{2.041430in}{2.485519in}}%
\pgfpathlineto{\pgfqpoint{2.042296in}{2.549095in}}%
\pgfpathlineto{\pgfqpoint{2.043162in}{2.508085in}}%
\pgfpathlineto{\pgfqpoint{2.044026in}{2.538273in}}%
\pgfpathlineto{\pgfqpoint{2.044891in}{2.505871in}}%
\pgfpathlineto{\pgfqpoint{2.046620in}{2.622507in}}%
\pgfpathlineto{\pgfqpoint{2.047484in}{2.500860in}}%
\pgfpathlineto{\pgfqpoint{2.048349in}{2.578054in}}%
\pgfpathlineto{\pgfqpoint{2.049214in}{2.533293in}}%
\pgfpathlineto{\pgfqpoint{2.050080in}{2.597851in}}%
\pgfpathlineto{\pgfqpoint{2.050945in}{2.550324in}}%
\pgfpathlineto{\pgfqpoint{2.053538in}{2.604922in}}%
\pgfpathlineto{\pgfqpoint{2.054402in}{2.534799in}}%
\pgfpathlineto{\pgfqpoint{2.055268in}{2.577561in}}%
\pgfpathlineto{\pgfqpoint{2.056131in}{2.502735in}}%
\pgfpathlineto{\pgfqpoint{2.057861in}{2.557456in}}%
\pgfpathlineto{\pgfqpoint{2.058726in}{2.568956in}}%
\pgfpathlineto{\pgfqpoint{2.059591in}{2.494651in}}%
\pgfpathlineto{\pgfqpoint{2.060456in}{2.590228in}}%
\pgfpathlineto{\pgfqpoint{2.061322in}{2.582297in}}%
\pgfpathlineto{\pgfqpoint{2.062187in}{2.525547in}}%
\pgfpathlineto{\pgfqpoint{2.063052in}{2.529236in}}%
\pgfpathlineto{\pgfqpoint{2.064780in}{2.496341in}}%
\pgfpathlineto{\pgfqpoint{2.065646in}{2.588445in}}%
\pgfpathlineto{\pgfqpoint{2.066512in}{2.574395in}}%
\pgfpathlineto{\pgfqpoint{2.067377in}{2.604247in}}%
\pgfpathlineto{\pgfqpoint{2.068243in}{2.514786in}}%
\pgfpathlineto{\pgfqpoint{2.069109in}{2.519736in}}%
\pgfpathlineto{\pgfqpoint{2.069974in}{2.497755in}}%
\pgfpathlineto{\pgfqpoint{2.070838in}{2.513188in}}%
\pgfpathlineto{\pgfqpoint{2.072565in}{2.485643in}}%
\pgfpathlineto{\pgfqpoint{2.073429in}{2.579898in}}%
\pgfpathlineto{\pgfqpoint{2.074294in}{2.557887in}}%
\pgfpathlineto{\pgfqpoint{2.075160in}{2.546390in}}%
\pgfpathlineto{\pgfqpoint{2.076025in}{2.496495in}}%
\pgfpathlineto{\pgfqpoint{2.076890in}{2.592319in}}%
\pgfpathlineto{\pgfqpoint{2.077754in}{2.491361in}}%
\pgfpathlineto{\pgfqpoint{2.078618in}{2.520105in}}%
\pgfpathlineto{\pgfqpoint{2.079483in}{2.503228in}}%
\pgfpathlineto{\pgfqpoint{2.080348in}{2.521488in}}%
\pgfpathlineto{\pgfqpoint{2.081213in}{2.495911in}}%
\pgfpathlineto{\pgfqpoint{2.082078in}{2.556043in}}%
\pgfpathlineto{\pgfqpoint{2.082942in}{2.492313in}}%
\pgfpathlineto{\pgfqpoint{2.083807in}{2.505932in}}%
\pgfpathlineto{\pgfqpoint{2.084671in}{2.550693in}}%
\pgfpathlineto{\pgfqpoint{2.085536in}{2.436425in}}%
\pgfpathlineto{\pgfqpoint{2.086401in}{2.518907in}}%
\pgfpathlineto{\pgfqpoint{2.087265in}{2.517860in}}%
\pgfpathlineto{\pgfqpoint{2.088130in}{2.516631in}}%
\pgfpathlineto{\pgfqpoint{2.089859in}{2.547496in}}%
\pgfpathlineto{\pgfqpoint{2.090724in}{2.525424in}}%
\pgfpathlineto{\pgfqpoint{2.091589in}{2.564527in}}%
\pgfpathlineto{\pgfqpoint{2.092454in}{2.452258in}}%
\pgfpathlineto{\pgfqpoint{2.095049in}{2.583896in}}%
\pgfpathlineto{\pgfqpoint{2.095914in}{2.512636in}}%
\pgfpathlineto{\pgfqpoint{2.096780in}{2.520320in}}%
\pgfpathlineto{\pgfqpoint{2.097644in}{2.588014in}}%
\pgfpathlineto{\pgfqpoint{2.098507in}{2.578238in}}%
\pgfpathlineto{\pgfqpoint{2.099373in}{2.566618in}}%
\pgfpathlineto{\pgfqpoint{2.101101in}{2.530034in}}%
\pgfpathlineto{\pgfqpoint{2.101963in}{2.547004in}}%
\pgfpathlineto{\pgfqpoint{2.102828in}{2.533293in}}%
\pgfpathlineto{\pgfqpoint{2.103692in}{2.557456in}}%
\pgfpathlineto{\pgfqpoint{2.104558in}{2.520872in}}%
\pgfpathlineto{\pgfqpoint{2.105422in}{2.583095in}}%
\pgfpathlineto{\pgfqpoint{2.108016in}{2.477404in}}%
\pgfpathlineto{\pgfqpoint{2.109746in}{2.566003in}}%
\pgfpathlineto{\pgfqpoint{2.110611in}{2.594102in}}%
\pgfpathlineto{\pgfqpoint{2.112342in}{2.503228in}}%
\pgfpathlineto{\pgfqpoint{2.113207in}{2.499538in}}%
\pgfpathlineto{\pgfqpoint{2.114071in}{2.508146in}}%
\pgfpathlineto{\pgfqpoint{2.114938in}{2.562436in}}%
\pgfpathlineto{\pgfqpoint{2.115804in}{2.503351in}}%
\pgfpathlineto{\pgfqpoint{2.116670in}{2.512541in}}%
\pgfpathlineto{\pgfqpoint{2.117537in}{2.466890in}}%
\pgfpathlineto{\pgfqpoint{2.118403in}{2.593548in}}%
\pgfpathlineto{\pgfqpoint{2.119267in}{2.564343in}}%
\pgfpathlineto{\pgfqpoint{2.120134in}{2.567232in}}%
\pgfpathlineto{\pgfqpoint{2.120999in}{2.556964in}}%
\pgfpathlineto{\pgfqpoint{2.122731in}{2.587952in}}%
\pgfpathlineto{\pgfqpoint{2.124463in}{2.500521in}}%
\pgfpathlineto{\pgfqpoint{2.125328in}{2.582849in}}%
\pgfpathlineto{\pgfqpoint{2.126193in}{2.524069in}}%
\pgfpathlineto{\pgfqpoint{2.127057in}{2.591457in}}%
\pgfpathlineto{\pgfqpoint{2.127924in}{2.503780in}}%
\pgfpathlineto{\pgfqpoint{2.128790in}{2.525791in}}%
\pgfpathlineto{\pgfqpoint{2.129654in}{2.563912in}}%
\pgfpathlineto{\pgfqpoint{2.130520in}{2.516169in}}%
\pgfpathlineto{\pgfqpoint{2.131385in}{2.528436in}}%
\pgfpathlineto{\pgfqpoint{2.132251in}{2.492590in}}%
\pgfpathlineto{\pgfqpoint{2.133982in}{2.619002in}}%
\pgfpathlineto{\pgfqpoint{2.135713in}{2.510697in}}%
\pgfpathlineto{\pgfqpoint{2.136575in}{2.567355in}}%
\pgfpathlineto{\pgfqpoint{2.137440in}{2.538950in}}%
\pgfpathlineto{\pgfqpoint{2.138303in}{2.577746in}}%
\pgfpathlineto{\pgfqpoint{2.140031in}{2.509191in}}%
\pgfpathlineto{\pgfqpoint{2.140897in}{2.578238in}}%
\pgfpathlineto{\pgfqpoint{2.142629in}{2.520228in}}%
\pgfpathlineto{\pgfqpoint{2.143495in}{2.524870in}}%
\pgfpathlineto{\pgfqpoint{2.144360in}{2.522840in}}%
\pgfpathlineto{\pgfqpoint{2.145225in}{2.495418in}}%
\pgfpathlineto{\pgfqpoint{2.146091in}{2.539196in}}%
\pgfpathlineto{\pgfqpoint{2.147822in}{2.459819in}}%
\pgfpathlineto{\pgfqpoint{2.148687in}{2.466767in}}%
\pgfpathlineto{\pgfqpoint{2.149554in}{2.464368in}}%
\pgfpathlineto{\pgfqpoint{2.150418in}{2.473407in}}%
\pgfpathlineto{\pgfqpoint{2.151284in}{2.535507in}}%
\pgfpathlineto{\pgfqpoint{2.152149in}{2.476850in}}%
\pgfpathlineto{\pgfqpoint{2.153015in}{2.479249in}}%
\pgfpathlineto{\pgfqpoint{2.155607in}{2.547866in}}%
\pgfpathlineto{\pgfqpoint{2.156472in}{2.545652in}}%
\pgfpathlineto{\pgfqpoint{2.157337in}{2.501260in}}%
\pgfpathlineto{\pgfqpoint{2.158202in}{2.605291in}}%
\pgfpathlineto{\pgfqpoint{2.159928in}{2.486564in}}%
\pgfpathlineto{\pgfqpoint{2.160794in}{2.496955in}}%
\pgfpathlineto{\pgfqpoint{2.162520in}{2.528618in}}%
\pgfpathlineto{\pgfqpoint{2.163384in}{2.517306in}}%
\pgfpathlineto{\pgfqpoint{2.164251in}{2.454254in}}%
\pgfpathlineto{\pgfqpoint{2.165115in}{2.611562in}}%
\pgfpathlineto{\pgfqpoint{2.166844in}{2.521365in}}%
\pgfpathlineto{\pgfqpoint{2.167707in}{2.583526in}}%
\pgfpathlineto{\pgfqpoint{2.168572in}{2.516015in}}%
\pgfpathlineto{\pgfqpoint{2.169437in}{2.589797in}}%
\pgfpathlineto{\pgfqpoint{2.170301in}{2.580512in}}%
\pgfpathlineto{\pgfqpoint{2.171166in}{2.583280in}}%
\pgfpathlineto{\pgfqpoint{2.172893in}{2.520136in}}%
\pgfpathlineto{\pgfqpoint{2.173758in}{2.514448in}}%
\pgfpathlineto{\pgfqpoint{2.174620in}{2.444756in}}%
\pgfpathlineto{\pgfqpoint{2.175484in}{2.538765in}}%
\pgfpathlineto{\pgfqpoint{2.176349in}{2.510789in}}%
\pgfpathlineto{\pgfqpoint{2.177215in}{2.628839in}}%
\pgfpathlineto{\pgfqpoint{2.178945in}{2.488409in}}%
\pgfpathlineto{\pgfqpoint{2.179810in}{2.528128in}}%
\pgfpathlineto{\pgfqpoint{2.181542in}{2.491790in}}%
\pgfpathlineto{\pgfqpoint{2.182405in}{2.541962in}}%
\pgfpathlineto{\pgfqpoint{2.183270in}{2.475128in}}%
\pgfpathlineto{\pgfqpoint{2.184136in}{2.541716in}}%
\pgfpathlineto{\pgfqpoint{2.185000in}{2.531941in}}%
\pgfpathlineto{\pgfqpoint{2.185866in}{2.524500in}}%
\pgfpathlineto{\pgfqpoint{2.186731in}{2.570860in}}%
\pgfpathlineto{\pgfqpoint{2.187597in}{2.488993in}}%
\pgfpathlineto{\pgfqpoint{2.190192in}{2.630776in}}%
\pgfpathlineto{\pgfqpoint{2.191056in}{2.542024in}}%
\pgfpathlineto{\pgfqpoint{2.191922in}{2.565633in}}%
\pgfpathlineto{\pgfqpoint{2.192788in}{2.507500in}}%
\pgfpathlineto{\pgfqpoint{2.193652in}{2.550201in}}%
\pgfpathlineto{\pgfqpoint{2.194519in}{2.544420in}}%
\pgfpathlineto{\pgfqpoint{2.195384in}{2.517183in}}%
\pgfpathlineto{\pgfqpoint{2.196249in}{2.549831in}}%
\pgfpathlineto{\pgfqpoint{2.199710in}{2.483244in}}%
\pgfpathlineto{\pgfqpoint{2.202305in}{2.549831in}}%
\pgfpathlineto{\pgfqpoint{2.203171in}{2.539625in}}%
\pgfpathlineto{\pgfqpoint{2.204900in}{2.500706in}}%
\pgfpathlineto{\pgfqpoint{2.206632in}{2.562375in}}%
\pgfpathlineto{\pgfqpoint{2.207495in}{2.526222in}}%
\pgfpathlineto{\pgfqpoint{2.208361in}{2.547127in}}%
\pgfpathlineto{\pgfqpoint{2.209225in}{2.466090in}}%
\pgfpathlineto{\pgfqpoint{2.210954in}{2.504026in}}%
\pgfpathlineto{\pgfqpoint{2.211819in}{2.493789in}}%
\pgfpathlineto{\pgfqpoint{2.213551in}{2.553705in}}%
\pgfpathlineto{\pgfqpoint{2.215281in}{2.523148in}}%
\pgfpathlineto{\pgfqpoint{2.217011in}{2.547312in}}%
\pgfpathlineto{\pgfqpoint{2.217875in}{2.509683in}}%
\pgfpathlineto{\pgfqpoint{2.218741in}{2.533324in}}%
\pgfpathlineto{\pgfqpoint{2.220471in}{2.512634in}}%
\pgfpathlineto{\pgfqpoint{2.221337in}{2.537934in}}%
\pgfpathlineto{\pgfqpoint{2.222202in}{2.503718in}}%
\pgfpathlineto{\pgfqpoint{2.223065in}{2.522779in}}%
\pgfpathlineto{\pgfqpoint{2.223931in}{2.485089in}}%
\pgfpathlineto{\pgfqpoint{2.224796in}{2.549770in}}%
\pgfpathlineto{\pgfqpoint{2.225661in}{2.527851in}}%
\pgfpathlineto{\pgfqpoint{2.226526in}{2.532862in}}%
\pgfpathlineto{\pgfqpoint{2.227389in}{2.550447in}}%
\pgfpathlineto{\pgfqpoint{2.228254in}{2.518751in}}%
\pgfpathlineto{\pgfqpoint{2.229120in}{2.558809in}}%
\pgfpathlineto{\pgfqpoint{2.229986in}{2.536428in}}%
\pgfpathlineto{\pgfqpoint{2.231717in}{2.580820in}}%
\pgfpathlineto{\pgfqpoint{2.233444in}{2.519028in}}%
\pgfpathlineto{\pgfqpoint{2.234309in}{2.536367in}}%
\pgfpathlineto{\pgfqpoint{2.235174in}{2.508791in}}%
\pgfpathlineto{\pgfqpoint{2.236039in}{2.557333in}}%
\pgfpathlineto{\pgfqpoint{2.238633in}{2.474882in}}%
\pgfpathlineto{\pgfqpoint{2.239497in}{2.549587in}}%
\pgfpathlineto{\pgfqpoint{2.240361in}{2.486012in}}%
\pgfpathlineto{\pgfqpoint{2.241226in}{2.486566in}}%
\pgfpathlineto{\pgfqpoint{2.242955in}{2.599021in}}%
\pgfpathlineto{\pgfqpoint{2.243820in}{2.573258in}}%
\pgfpathlineto{\pgfqpoint{2.244685in}{2.513773in}}%
\pgfpathlineto{\pgfqpoint{2.245550in}{2.556166in}}%
\pgfpathlineto{\pgfqpoint{2.246416in}{2.553890in}}%
\pgfpathlineto{\pgfqpoint{2.247281in}{2.576886in}}%
\pgfpathlineto{\pgfqpoint{2.248147in}{2.565143in}}%
\pgfpathlineto{\pgfqpoint{2.249009in}{2.571352in}}%
\pgfpathlineto{\pgfqpoint{2.249874in}{2.516416in}}%
\pgfpathlineto{\pgfqpoint{2.250740in}{2.572951in}}%
\pgfpathlineto{\pgfqpoint{2.252471in}{2.500891in}}%
\pgfpathlineto{\pgfqpoint{2.254201in}{2.545005in}}%
\pgfpathlineto{\pgfqpoint{2.255066in}{2.503780in}}%
\pgfpathlineto{\pgfqpoint{2.255930in}{2.567232in}}%
\pgfpathlineto{\pgfqpoint{2.256794in}{2.565972in}}%
\pgfpathlineto{\pgfqpoint{2.258523in}{2.555612in}}%
\pgfpathlineto{\pgfqpoint{2.259389in}{2.602217in}}%
\pgfpathlineto{\pgfqpoint{2.260254in}{2.557580in}}%
\pgfpathlineto{\pgfqpoint{2.261118in}{2.592134in}}%
\pgfpathlineto{\pgfqpoint{2.262848in}{2.536120in}}%
\pgfpathlineto{\pgfqpoint{2.263712in}{2.553305in}}%
\pgfpathlineto{\pgfqpoint{2.264577in}{2.522286in}}%
\pgfpathlineto{\pgfqpoint{2.265441in}{2.537780in}}%
\pgfpathlineto{\pgfqpoint{2.266306in}{2.583493in}}%
\pgfpathlineto{\pgfqpoint{2.267171in}{2.510050in}}%
\pgfpathlineto{\pgfqpoint{2.268032in}{2.620906in}}%
\pgfpathlineto{\pgfqpoint{2.268898in}{2.530863in}}%
\pgfpathlineto{\pgfqpoint{2.269763in}{2.561821in}}%
\pgfpathlineto{\pgfqpoint{2.271490in}{2.505624in}}%
\pgfpathlineto{\pgfqpoint{2.273219in}{2.618879in}}%
\pgfpathlineto{\pgfqpoint{2.274084in}{2.540118in}}%
\pgfpathlineto{\pgfqpoint{2.274949in}{2.619924in}}%
\pgfpathlineto{\pgfqpoint{2.277541in}{2.553151in}}%
\pgfpathlineto{\pgfqpoint{2.278407in}{2.569752in}}%
\pgfpathlineto{\pgfqpoint{2.280137in}{2.481522in}}%
\pgfpathlineto{\pgfqpoint{2.281868in}{2.584325in}}%
\pgfpathlineto{\pgfqpoint{2.282732in}{2.480232in}}%
\pgfpathlineto{\pgfqpoint{2.283598in}{2.485150in}}%
\pgfpathlineto{\pgfqpoint{2.284462in}{2.476788in}}%
\pgfpathlineto{\pgfqpoint{2.286192in}{2.504642in}}%
\pgfpathlineto{\pgfqpoint{2.287056in}{2.497263in}}%
\pgfpathlineto{\pgfqpoint{2.288787in}{2.461356in}}%
\pgfpathlineto{\pgfqpoint{2.289652in}{2.584940in}}%
\pgfpathlineto{\pgfqpoint{2.290518in}{2.524192in}}%
\pgfpathlineto{\pgfqpoint{2.291384in}{2.565818in}}%
\pgfpathlineto{\pgfqpoint{2.293114in}{2.444756in}}%
\pgfpathlineto{\pgfqpoint{2.293978in}{2.589735in}}%
\pgfpathlineto{\pgfqpoint{2.294842in}{2.506117in}}%
\pgfpathlineto{\pgfqpoint{2.295707in}{2.520872in}}%
\pgfpathlineto{\pgfqpoint{2.297435in}{2.497324in}}%
\pgfpathlineto{\pgfqpoint{2.298302in}{2.609902in}}%
\pgfpathlineto{\pgfqpoint{2.299167in}{2.485150in}}%
\pgfpathlineto{\pgfqpoint{2.300033in}{2.608180in}}%
\pgfpathlineto{\pgfqpoint{2.300900in}{2.545036in}}%
\pgfpathlineto{\pgfqpoint{2.301766in}{2.598220in}}%
\pgfpathlineto{\pgfqpoint{2.304361in}{2.519674in}}%
\pgfpathlineto{\pgfqpoint{2.306957in}{2.547004in}}%
\pgfpathlineto{\pgfqpoint{2.307822in}{2.506486in}}%
\pgfpathlineto{\pgfqpoint{2.308688in}{2.516446in}}%
\pgfpathlineto{\pgfqpoint{2.309554in}{2.516262in}}%
\pgfpathlineto{\pgfqpoint{2.310420in}{2.518660in}}%
\pgfpathlineto{\pgfqpoint{2.311285in}{2.488993in}}%
\pgfpathlineto{\pgfqpoint{2.313017in}{2.581066in}}%
\pgfpathlineto{\pgfqpoint{2.313883in}{2.493635in}}%
\pgfpathlineto{\pgfqpoint{2.314749in}{2.515831in}}%
\pgfpathlineto{\pgfqpoint{2.315613in}{2.582972in}}%
\pgfpathlineto{\pgfqpoint{2.317342in}{2.498677in}}%
\pgfpathlineto{\pgfqpoint{2.318206in}{2.502304in}}%
\pgfpathlineto{\pgfqpoint{2.319937in}{2.566372in}}%
\pgfpathlineto{\pgfqpoint{2.320802in}{2.546144in}}%
\pgfpathlineto{\pgfqpoint{2.321668in}{2.557580in}}%
\pgfpathlineto{\pgfqpoint{2.322533in}{2.550139in}}%
\pgfpathlineto{\pgfqpoint{2.323399in}{2.572643in}}%
\pgfpathlineto{\pgfqpoint{2.325993in}{2.469717in}}%
\pgfpathlineto{\pgfqpoint{2.327723in}{2.553890in}}%
\pgfpathlineto{\pgfqpoint{2.328589in}{2.506117in}}%
\pgfpathlineto{\pgfqpoint{2.329452in}{2.625889in}}%
\pgfpathlineto{\pgfqpoint{2.331179in}{2.508146in}}%
\pgfpathlineto{\pgfqpoint{2.332043in}{2.520567in}}%
\pgfpathlineto{\pgfqpoint{2.332908in}{2.501814in}}%
\pgfpathlineto{\pgfqpoint{2.333775in}{2.519091in}}%
\pgfpathlineto{\pgfqpoint{2.334640in}{2.513188in}}%
\pgfpathlineto{\pgfqpoint{2.335506in}{2.516416in}}%
\pgfpathlineto{\pgfqpoint{2.337235in}{2.586600in}}%
\pgfpathlineto{\pgfqpoint{2.338965in}{2.486749in}}%
\pgfpathlineto{\pgfqpoint{2.339831in}{2.495295in}}%
\pgfpathlineto{\pgfqpoint{2.340692in}{2.537105in}}%
\pgfpathlineto{\pgfqpoint{2.341556in}{2.522717in}}%
\pgfpathlineto{\pgfqpoint{2.342422in}{2.454500in}}%
\pgfpathlineto{\pgfqpoint{2.343286in}{2.531879in}}%
\pgfpathlineto{\pgfqpoint{2.345881in}{2.471685in}}%
\pgfpathlineto{\pgfqpoint{2.346747in}{2.497817in}}%
\pgfpathlineto{\pgfqpoint{2.347612in}{2.494558in}}%
\pgfpathlineto{\pgfqpoint{2.348476in}{2.481645in}}%
\pgfpathlineto{\pgfqpoint{2.349341in}{2.491606in}}%
\pgfpathlineto{\pgfqpoint{2.350207in}{2.544420in}}%
\pgfpathlineto{\pgfqpoint{2.351073in}{2.519335in}}%
\pgfpathlineto{\pgfqpoint{2.351939in}{2.455668in}}%
\pgfpathlineto{\pgfqpoint{2.353668in}{2.531202in}}%
\pgfpathlineto{\pgfqpoint{2.354534in}{2.504241in}}%
\pgfpathlineto{\pgfqpoint{2.356262in}{2.552292in}}%
\pgfpathlineto{\pgfqpoint{2.357127in}{2.497386in}}%
\pgfpathlineto{\pgfqpoint{2.358859in}{2.565818in}}%
\pgfpathlineto{\pgfqpoint{2.359724in}{2.492221in}}%
\pgfpathlineto{\pgfqpoint{2.361451in}{2.571198in}}%
\pgfpathlineto{\pgfqpoint{2.362316in}{2.547127in}}%
\pgfpathlineto{\pgfqpoint{2.363181in}{2.566372in}}%
\pgfpathlineto{\pgfqpoint{2.364045in}{2.489055in}}%
\pgfpathlineto{\pgfqpoint{2.364909in}{2.576825in}}%
\pgfpathlineto{\pgfqpoint{2.365771in}{2.559424in}}%
\pgfpathlineto{\pgfqpoint{2.366638in}{2.538519in}}%
\pgfpathlineto{\pgfqpoint{2.367503in}{2.490069in}}%
\pgfpathlineto{\pgfqpoint{2.368366in}{2.495141in}}%
\pgfpathlineto{\pgfqpoint{2.370096in}{2.565880in}}%
\pgfpathlineto{\pgfqpoint{2.371826in}{2.518045in}}%
\pgfpathlineto{\pgfqpoint{2.372691in}{2.543499in}}%
\pgfpathlineto{\pgfqpoint{2.373557in}{2.492221in}}%
\pgfpathlineto{\pgfqpoint{2.375287in}{2.569077in}}%
\pgfpathlineto{\pgfqpoint{2.376151in}{2.577808in}}%
\pgfpathlineto{\pgfqpoint{2.377017in}{2.603508in}}%
\pgfpathlineto{\pgfqpoint{2.379616in}{2.537536in}}%
\pgfpathlineto{\pgfqpoint{2.380481in}{2.599421in}}%
\pgfpathlineto{\pgfqpoint{2.383076in}{2.525577in}}%
\pgfpathlineto{\pgfqpoint{2.383941in}{2.566803in}}%
\pgfpathlineto{\pgfqpoint{2.385670in}{2.453856in}}%
\pgfpathlineto{\pgfqpoint{2.386535in}{2.561515in}}%
\pgfpathlineto{\pgfqpoint{2.387399in}{2.501168in}}%
\pgfpathlineto{\pgfqpoint{2.388265in}{2.516139in}}%
\pgfpathlineto{\pgfqpoint{2.389131in}{2.567724in}}%
\pgfpathlineto{\pgfqpoint{2.389995in}{2.557857in}}%
\pgfpathlineto{\pgfqpoint{2.390860in}{2.546452in}}%
\pgfpathlineto{\pgfqpoint{2.391725in}{2.555306in}}%
\pgfpathlineto{\pgfqpoint{2.392590in}{2.544669in}}%
\pgfpathlineto{\pgfqpoint{2.393455in}{2.571229in}}%
\pgfpathlineto{\pgfqpoint{2.394320in}{2.565143in}}%
\pgfpathlineto{\pgfqpoint{2.396915in}{2.494189in}}%
\pgfpathlineto{\pgfqpoint{2.397781in}{2.548787in}}%
\pgfpathlineto{\pgfqpoint{2.398647in}{2.543807in}}%
\pgfpathlineto{\pgfqpoint{2.399511in}{2.464737in}}%
\pgfpathlineto{\pgfqpoint{2.400376in}{2.605476in}}%
\pgfpathlineto{\pgfqpoint{2.402105in}{2.529911in}}%
\pgfpathlineto{\pgfqpoint{2.403834in}{2.584940in}}%
\pgfpathlineto{\pgfqpoint{2.404699in}{2.550324in}}%
\pgfpathlineto{\pgfqpoint{2.405564in}{2.595393in}}%
\pgfpathlineto{\pgfqpoint{2.406429in}{2.484167in}}%
\pgfpathlineto{\pgfqpoint{2.408157in}{2.535199in}}%
\pgfpathlineto{\pgfqpoint{2.409022in}{2.542547in}}%
\pgfpathlineto{\pgfqpoint{2.409887in}{2.518168in}}%
\pgfpathlineto{\pgfqpoint{2.410753in}{2.545528in}}%
\pgfpathlineto{\pgfqpoint{2.411618in}{2.540025in}}%
\pgfpathlineto{\pgfqpoint{2.412483in}{2.545467in}}%
\pgfpathlineto{\pgfqpoint{2.414213in}{2.507838in}}%
\pgfpathlineto{\pgfqpoint{2.415079in}{2.557518in}}%
\pgfpathlineto{\pgfqpoint{2.415943in}{2.509868in}}%
\pgfpathlineto{\pgfqpoint{2.416808in}{2.573012in}}%
\pgfpathlineto{\pgfqpoint{2.417672in}{2.545159in}}%
\pgfpathlineto{\pgfqpoint{2.419402in}{2.596622in}}%
\pgfpathlineto{\pgfqpoint{2.421131in}{2.528097in}}%
\pgfpathlineto{\pgfqpoint{2.421995in}{2.555920in}}%
\pgfpathlineto{\pgfqpoint{2.423725in}{2.528621in}}%
\pgfpathlineto{\pgfqpoint{2.424590in}{2.575288in}}%
\pgfpathlineto{\pgfqpoint{2.425455in}{2.515463in}}%
\pgfpathlineto{\pgfqpoint{2.427184in}{2.568278in}}%
\pgfpathlineto{\pgfqpoint{2.428912in}{2.473838in}}%
\pgfpathlineto{\pgfqpoint{2.429777in}{2.533724in}}%
\pgfpathlineto{\pgfqpoint{2.430641in}{2.503105in}}%
\pgfpathlineto{\pgfqpoint{2.432370in}{2.563421in}}%
\pgfpathlineto{\pgfqpoint{2.433233in}{2.470272in}}%
\pgfpathlineto{\pgfqpoint{2.435826in}{2.575165in}}%
\pgfpathlineto{\pgfqpoint{2.436690in}{2.580637in}}%
\pgfpathlineto{\pgfqpoint{2.437555in}{2.524041in}}%
\pgfpathlineto{\pgfqpoint{2.438420in}{2.527022in}}%
\pgfpathlineto{\pgfqpoint{2.439284in}{2.535568in}}%
\pgfpathlineto{\pgfqpoint{2.440150in}{2.566341in}}%
\pgfpathlineto{\pgfqpoint{2.441015in}{2.514848in}}%
\pgfpathlineto{\pgfqpoint{2.441880in}{2.528066in}}%
\pgfpathlineto{\pgfqpoint{2.442744in}{2.481707in}}%
\pgfpathlineto{\pgfqpoint{2.443608in}{2.507962in}}%
\pgfpathlineto{\pgfqpoint{2.444474in}{2.482323in}}%
\pgfpathlineto{\pgfqpoint{2.445339in}{2.550816in}}%
\pgfpathlineto{\pgfqpoint{2.446201in}{2.545098in}}%
\pgfpathlineto{\pgfqpoint{2.447066in}{2.503595in}}%
\pgfpathlineto{\pgfqpoint{2.448795in}{2.553644in}}%
\pgfpathlineto{\pgfqpoint{2.449658in}{2.531202in}}%
\pgfpathlineto{\pgfqpoint{2.450524in}{2.577623in}}%
\pgfpathlineto{\pgfqpoint{2.452253in}{2.523210in}}%
\pgfpathlineto{\pgfqpoint{2.453117in}{2.556533in}}%
\pgfpathlineto{\pgfqpoint{2.453982in}{2.507654in}}%
\pgfpathlineto{\pgfqpoint{2.454848in}{2.562683in}}%
\pgfpathlineto{\pgfqpoint{2.455713in}{2.528066in}}%
\pgfpathlineto{\pgfqpoint{2.457441in}{2.555612in}}%
\pgfpathlineto{\pgfqpoint{2.458306in}{2.524993in}}%
\pgfpathlineto{\pgfqpoint{2.460035in}{2.576578in}}%
\pgfpathlineto{\pgfqpoint{2.461764in}{2.545652in}}%
\pgfpathlineto{\pgfqpoint{2.463494in}{2.564866in}}%
\pgfpathlineto{\pgfqpoint{2.464360in}{2.547927in}}%
\pgfpathlineto{\pgfqpoint{2.465225in}{2.568032in}}%
\pgfpathlineto{\pgfqpoint{2.466090in}{2.493850in}}%
\pgfpathlineto{\pgfqpoint{2.467821in}{2.581743in}}%
\pgfpathlineto{\pgfqpoint{2.469550in}{2.552969in}}%
\pgfpathlineto{\pgfqpoint{2.470414in}{2.611256in}}%
\pgfpathlineto{\pgfqpoint{2.473005in}{2.482507in}}%
\pgfpathlineto{\pgfqpoint{2.473871in}{2.538519in}}%
\pgfpathlineto{\pgfqpoint{2.474737in}{2.533662in}}%
\pgfpathlineto{\pgfqpoint{2.475602in}{2.531787in}}%
\pgfpathlineto{\pgfqpoint{2.476468in}{2.481707in}}%
\pgfpathlineto{\pgfqpoint{2.478200in}{2.558593in}}%
\pgfpathlineto{\pgfqpoint{2.479066in}{2.529480in}}%
\pgfpathlineto{\pgfqpoint{2.479931in}{2.575103in}}%
\pgfpathlineto{\pgfqpoint{2.480797in}{2.548910in}}%
\pgfpathlineto{\pgfqpoint{2.481662in}{2.600742in}}%
\pgfpathlineto{\pgfqpoint{2.482527in}{2.552322in}}%
\pgfpathlineto{\pgfqpoint{2.483391in}{2.593856in}}%
\pgfpathlineto{\pgfqpoint{2.485985in}{2.539196in}}%
\pgfpathlineto{\pgfqpoint{2.486849in}{2.617219in}}%
\pgfpathlineto{\pgfqpoint{2.487715in}{2.605599in}}%
\pgfpathlineto{\pgfqpoint{2.488581in}{2.581805in}}%
\pgfpathlineto{\pgfqpoint{2.489447in}{2.623184in}}%
\pgfpathlineto{\pgfqpoint{2.490310in}{2.549587in}}%
\pgfpathlineto{\pgfqpoint{2.491175in}{2.589859in}}%
\pgfpathlineto{\pgfqpoint{2.492905in}{2.512574in}}%
\pgfpathlineto{\pgfqpoint{2.493769in}{2.578669in}}%
\pgfpathlineto{\pgfqpoint{2.494635in}{2.528713in}}%
\pgfpathlineto{\pgfqpoint{2.495501in}{2.585740in}}%
\pgfpathlineto{\pgfqpoint{2.496365in}{2.576825in}}%
\pgfpathlineto{\pgfqpoint{2.497231in}{2.479064in}}%
\pgfpathlineto{\pgfqpoint{2.498962in}{2.592627in}}%
\pgfpathlineto{\pgfqpoint{2.499826in}{2.551063in}}%
\pgfpathlineto{\pgfqpoint{2.500692in}{2.565882in}}%
\pgfpathlineto{\pgfqpoint{2.501555in}{2.485612in}}%
\pgfpathlineto{\pgfqpoint{2.503285in}{2.583649in}}%
\pgfpathlineto{\pgfqpoint{2.504149in}{2.511589in}}%
\pgfpathlineto{\pgfqpoint{2.505880in}{2.553215in}}%
\pgfpathlineto{\pgfqpoint{2.507611in}{2.511651in}}%
\pgfpathlineto{\pgfqpoint{2.508476in}{2.546513in}}%
\pgfpathlineto{\pgfqpoint{2.509340in}{2.522411in}}%
\pgfpathlineto{\pgfqpoint{2.510205in}{2.584327in}}%
\pgfpathlineto{\pgfqpoint{2.511069in}{2.470949in}}%
\pgfpathlineto{\pgfqpoint{2.511934in}{2.471010in}}%
\pgfpathlineto{\pgfqpoint{2.512798in}{2.536061in}}%
\pgfpathlineto{\pgfqpoint{2.513664in}{2.464186in}}%
\pgfpathlineto{\pgfqpoint{2.514530in}{2.591398in}}%
\pgfpathlineto{\pgfqpoint{2.515396in}{2.546144in}}%
\pgfpathlineto{\pgfqpoint{2.516260in}{2.549587in}}%
\pgfpathlineto{\pgfqpoint{2.517125in}{2.527515in}}%
\pgfpathlineto{\pgfqpoint{2.517990in}{2.561330in}}%
\pgfpathlineto{\pgfqpoint{2.518855in}{2.459082in}}%
\pgfpathlineto{\pgfqpoint{2.520585in}{2.544638in}}%
\pgfpathlineto{\pgfqpoint{2.521451in}{2.482323in}}%
\pgfpathlineto{\pgfqpoint{2.522317in}{2.559917in}}%
\pgfpathlineto{\pgfqpoint{2.523180in}{2.501937in}}%
\pgfpathlineto{\pgfqpoint{2.524912in}{2.579652in}}%
\pgfpathlineto{\pgfqpoint{2.526643in}{2.510176in}}%
\pgfpathlineto{\pgfqpoint{2.527509in}{2.581989in}}%
\pgfpathlineto{\pgfqpoint{2.528374in}{2.506794in}}%
\pgfpathlineto{\pgfqpoint{2.529236in}{2.541226in}}%
\pgfpathlineto{\pgfqpoint{2.530966in}{2.484537in}}%
\pgfpathlineto{\pgfqpoint{2.531831in}{2.520505in}}%
\pgfpathlineto{\pgfqpoint{2.532696in}{2.486566in}}%
\pgfpathlineto{\pgfqpoint{2.534427in}{2.556597in}}%
\pgfpathlineto{\pgfqpoint{2.535290in}{2.495051in}}%
\pgfpathlineto{\pgfqpoint{2.536155in}{2.529421in}}%
\pgfpathlineto{\pgfqpoint{2.537021in}{2.459452in}}%
\pgfpathlineto{\pgfqpoint{2.538751in}{2.531879in}}%
\pgfpathlineto{\pgfqpoint{2.539618in}{2.507685in}}%
\pgfpathlineto{\pgfqpoint{2.540482in}{2.516754in}}%
\pgfpathlineto{\pgfqpoint{2.541346in}{2.458775in}}%
\pgfpathlineto{\pgfqpoint{2.542211in}{2.467475in}}%
\pgfpathlineto{\pgfqpoint{2.543942in}{2.544792in}}%
\pgfpathlineto{\pgfqpoint{2.544808in}{2.554783in}}%
\pgfpathlineto{\pgfqpoint{2.545672in}{2.475744in}}%
\pgfpathlineto{\pgfqpoint{2.546537in}{2.566741in}}%
\pgfpathlineto{\pgfqpoint{2.547403in}{2.501937in}}%
\pgfpathlineto{\pgfqpoint{2.548266in}{2.592442in}}%
\pgfpathlineto{\pgfqpoint{2.549131in}{2.521980in}}%
\pgfpathlineto{\pgfqpoint{2.549993in}{2.534647in}}%
\pgfpathlineto{\pgfqpoint{2.550858in}{2.541780in}}%
\pgfpathlineto{\pgfqpoint{2.551724in}{2.512359in}}%
\pgfpathlineto{\pgfqpoint{2.552590in}{2.516202in}}%
\pgfpathlineto{\pgfqpoint{2.554322in}{2.495112in}}%
\pgfpathlineto{\pgfqpoint{2.555188in}{2.501445in}}%
\pgfpathlineto{\pgfqpoint{2.556051in}{2.541287in}}%
\pgfpathlineto{\pgfqpoint{2.556916in}{2.539258in}}%
\pgfpathlineto{\pgfqpoint{2.557782in}{2.475867in}}%
\pgfpathlineto{\pgfqpoint{2.560377in}{2.577133in}}%
\pgfpathlineto{\pgfqpoint{2.562106in}{2.487980in}}%
\pgfpathlineto{\pgfqpoint{2.564700in}{2.577071in}}%
\pgfpathlineto{\pgfqpoint{2.566428in}{2.516508in}}%
\pgfpathlineto{\pgfqpoint{2.567292in}{2.612424in}}%
\pgfpathlineto{\pgfqpoint{2.568157in}{2.563021in}}%
\pgfpathlineto{\pgfqpoint{2.569022in}{2.585802in}}%
\pgfpathlineto{\pgfqpoint{2.569888in}{2.580514in}}%
\pgfpathlineto{\pgfqpoint{2.570753in}{2.568525in}}%
\pgfpathlineto{\pgfqpoint{2.571618in}{2.582359in}}%
\pgfpathlineto{\pgfqpoint{2.572482in}{2.557028in}}%
\pgfpathlineto{\pgfqpoint{2.573346in}{2.491823in}}%
\pgfpathlineto{\pgfqpoint{2.574211in}{2.592627in}}%
\pgfpathlineto{\pgfqpoint{2.575076in}{2.538950in}}%
\pgfpathlineto{\pgfqpoint{2.575939in}{2.543684in}}%
\pgfpathlineto{\pgfqpoint{2.576802in}{2.568217in}}%
\pgfpathlineto{\pgfqpoint{2.577665in}{2.555827in}}%
\pgfpathlineto{\pgfqpoint{2.578530in}{2.526960in}}%
\pgfpathlineto{\pgfqpoint{2.579394in}{2.547681in}}%
\pgfpathlineto{\pgfqpoint{2.580257in}{2.487857in}}%
\pgfpathlineto{\pgfqpoint{2.581123in}{2.552907in}}%
\pgfpathlineto{\pgfqpoint{2.582854in}{2.505011in}}%
\pgfpathlineto{\pgfqpoint{2.583719in}{2.515710in}}%
\pgfpathlineto{\pgfqpoint{2.584583in}{2.582113in}}%
\pgfpathlineto{\pgfqpoint{2.585448in}{2.473776in}}%
\pgfpathlineto{\pgfqpoint{2.586313in}{2.559486in}}%
\pgfpathlineto{\pgfqpoint{2.587177in}{2.523671in}}%
\pgfpathlineto{\pgfqpoint{2.588042in}{2.533908in}}%
\pgfpathlineto{\pgfqpoint{2.588907in}{2.533785in}}%
\pgfpathlineto{\pgfqpoint{2.590637in}{2.503043in}}%
\pgfpathlineto{\pgfqpoint{2.591500in}{2.496588in}}%
\pgfpathlineto{\pgfqpoint{2.592365in}{2.462279in}}%
\pgfpathlineto{\pgfqpoint{2.593230in}{2.537598in}}%
\pgfpathlineto{\pgfqpoint{2.594095in}{2.531941in}}%
\pgfpathlineto{\pgfqpoint{2.594961in}{2.506425in}}%
\pgfpathlineto{\pgfqpoint{2.595827in}{2.547619in}}%
\pgfpathlineto{\pgfqpoint{2.596692in}{2.528190in}}%
\pgfpathlineto{\pgfqpoint{2.598421in}{2.606584in}}%
\pgfpathlineto{\pgfqpoint{2.599286in}{2.506579in}}%
\pgfpathlineto{\pgfqpoint{2.601015in}{2.592688in}}%
\pgfpathlineto{\pgfqpoint{2.601880in}{2.512236in}}%
\pgfpathlineto{\pgfqpoint{2.602742in}{2.565205in}}%
\pgfpathlineto{\pgfqpoint{2.604469in}{2.455270in}}%
\pgfpathlineto{\pgfqpoint{2.605334in}{2.539627in}}%
\pgfpathlineto{\pgfqpoint{2.606197in}{2.474423in}}%
\pgfpathlineto{\pgfqpoint{2.607063in}{2.536615in}}%
\pgfpathlineto{\pgfqpoint{2.607928in}{2.532679in}}%
\pgfpathlineto{\pgfqpoint{2.608793in}{2.542947in}}%
\pgfpathlineto{\pgfqpoint{2.609657in}{2.541841in}}%
\pgfpathlineto{\pgfqpoint{2.610521in}{2.616852in}}%
\pgfpathlineto{\pgfqpoint{2.611387in}{2.495266in}}%
\pgfpathlineto{\pgfqpoint{2.612252in}{2.539135in}}%
\pgfpathlineto{\pgfqpoint{2.613117in}{2.470949in}}%
\pgfpathlineto{\pgfqpoint{2.613981in}{2.539135in}}%
\pgfpathlineto{\pgfqpoint{2.614846in}{2.469042in}}%
\pgfpathlineto{\pgfqpoint{2.618301in}{2.559794in}}%
\pgfpathlineto{\pgfqpoint{2.619166in}{2.492590in}}%
\pgfpathlineto{\pgfqpoint{2.620031in}{2.506609in}}%
\pgfpathlineto{\pgfqpoint{2.620894in}{2.511620in}}%
\pgfpathlineto{\pgfqpoint{2.622625in}{2.578300in}}%
\pgfpathlineto{\pgfqpoint{2.625219in}{2.455978in}}%
\pgfpathlineto{\pgfqpoint{2.626084in}{2.442113in}}%
\pgfpathlineto{\pgfqpoint{2.628682in}{2.556412in}}%
\pgfpathlineto{\pgfqpoint{2.629548in}{2.549895in}}%
\pgfpathlineto{\pgfqpoint{2.630413in}{2.499754in}}%
\pgfpathlineto{\pgfqpoint{2.631279in}{2.513927in}}%
\pgfpathlineto{\pgfqpoint{2.632145in}{2.549895in}}%
\pgfpathlineto{\pgfqpoint{2.634739in}{2.515094in}}%
\pgfpathlineto{\pgfqpoint{2.635603in}{2.561330in}}%
\pgfpathlineto{\pgfqpoint{2.636470in}{2.546575in}}%
\pgfpathlineto{\pgfqpoint{2.637336in}{2.554506in}}%
\pgfpathlineto{\pgfqpoint{2.638200in}{2.586662in}}%
\pgfpathlineto{\pgfqpoint{2.639929in}{2.501260in}}%
\pgfpathlineto{\pgfqpoint{2.641660in}{2.572889in}}%
\pgfpathlineto{\pgfqpoint{2.642524in}{2.565572in}}%
\pgfpathlineto{\pgfqpoint{2.645115in}{2.483613in}}%
\pgfpathlineto{\pgfqpoint{2.647710in}{2.574549in}}%
\pgfpathlineto{\pgfqpoint{2.648575in}{2.540795in}}%
\pgfpathlineto{\pgfqpoint{2.649441in}{2.546850in}}%
\pgfpathlineto{\pgfqpoint{2.652037in}{2.508146in}}%
\pgfpathlineto{\pgfqpoint{2.652902in}{2.538581in}}%
\pgfpathlineto{\pgfqpoint{2.653767in}{2.526714in}}%
\pgfpathlineto{\pgfqpoint{2.654633in}{2.465107in}}%
\pgfpathlineto{\pgfqpoint{2.655499in}{2.480970in}}%
\pgfpathlineto{\pgfqpoint{2.656365in}{2.552599in}}%
\pgfpathlineto{\pgfqpoint{2.657231in}{2.492590in}}%
\pgfpathlineto{\pgfqpoint{2.659827in}{2.588260in}}%
\pgfpathlineto{\pgfqpoint{2.660689in}{2.559609in}}%
\pgfpathlineto{\pgfqpoint{2.661555in}{2.497571in}}%
\pgfpathlineto{\pgfqpoint{2.662417in}{2.566126in}}%
\pgfpathlineto{\pgfqpoint{2.663283in}{2.500521in}}%
\pgfpathlineto{\pgfqpoint{2.665014in}{2.556104in}}%
\pgfpathlineto{\pgfqpoint{2.666743in}{2.458344in}}%
\pgfpathlineto{\pgfqpoint{2.667607in}{2.542085in}}%
\pgfpathlineto{\pgfqpoint{2.668473in}{2.504919in}}%
\pgfpathlineto{\pgfqpoint{2.669338in}{2.531571in}}%
\pgfpathlineto{\pgfqpoint{2.670204in}{2.487303in}}%
\pgfpathlineto{\pgfqpoint{2.671069in}{2.550816in}}%
\pgfpathlineto{\pgfqpoint{2.671934in}{2.540241in}}%
\pgfpathlineto{\pgfqpoint{2.672800in}{2.552815in}}%
\pgfpathlineto{\pgfqpoint{2.673666in}{2.515217in}}%
\pgfpathlineto{\pgfqpoint{2.674531in}{2.574734in}}%
\pgfpathlineto{\pgfqpoint{2.675397in}{2.506917in}}%
\pgfpathlineto{\pgfqpoint{2.676261in}{2.551370in}}%
\pgfpathlineto{\pgfqpoint{2.677991in}{2.535137in}}%
\pgfpathlineto{\pgfqpoint{2.678857in}{2.576948in}}%
\pgfpathlineto{\pgfqpoint{2.679722in}{2.563298in}}%
\pgfpathlineto{\pgfqpoint{2.680587in}{2.504518in}}%
\pgfpathlineto{\pgfqpoint{2.682316in}{2.591950in}}%
\pgfpathlineto{\pgfqpoint{2.683182in}{2.506486in}}%
\pgfpathlineto{\pgfqpoint{2.684047in}{2.578300in}}%
\pgfpathlineto{\pgfqpoint{2.684912in}{2.505288in}}%
\pgfpathlineto{\pgfqpoint{2.685777in}{2.522042in}}%
\pgfpathlineto{\pgfqpoint{2.686642in}{2.586969in}}%
\pgfpathlineto{\pgfqpoint{2.687507in}{2.554444in}}%
\pgfpathlineto{\pgfqpoint{2.688372in}{2.574672in}}%
\pgfpathlineto{\pgfqpoint{2.690102in}{2.513434in}}%
\pgfpathlineto{\pgfqpoint{2.690968in}{2.523456in}}%
\pgfpathlineto{\pgfqpoint{2.691834in}{2.574028in}}%
\pgfpathlineto{\pgfqpoint{2.692698in}{2.572828in}}%
\pgfpathlineto{\pgfqpoint{2.694429in}{2.584203in}}%
\pgfpathlineto{\pgfqpoint{2.697023in}{2.490900in}}%
\pgfpathlineto{\pgfqpoint{2.697889in}{2.501321in}}%
\pgfpathlineto{\pgfqpoint{2.698752in}{2.515402in}}%
\pgfpathlineto{\pgfqpoint{2.699618in}{2.497878in}}%
\pgfpathlineto{\pgfqpoint{2.700483in}{2.527391in}}%
\pgfpathlineto{\pgfqpoint{2.701346in}{2.517000in}}%
\pgfpathlineto{\pgfqpoint{2.702209in}{2.481278in}}%
\pgfpathlineto{\pgfqpoint{2.703074in}{2.487918in}}%
\pgfpathlineto{\pgfqpoint{2.703940in}{2.499908in}}%
\pgfpathlineto{\pgfqpoint{2.704806in}{2.493021in}}%
\pgfpathlineto{\pgfqpoint{2.705670in}{2.562436in}}%
\pgfpathlineto{\pgfqpoint{2.707401in}{2.526899in}}%
\pgfpathlineto{\pgfqpoint{2.708268in}{2.540672in}}%
\pgfpathlineto{\pgfqpoint{2.709134in}{2.524195in}}%
\pgfpathlineto{\pgfqpoint{2.709995in}{2.600742in}}%
\pgfpathlineto{\pgfqpoint{2.712592in}{2.506486in}}%
\pgfpathlineto{\pgfqpoint{2.713458in}{2.498032in}}%
\pgfpathlineto{\pgfqpoint{2.715189in}{2.577440in}}%
\pgfpathlineto{\pgfqpoint{2.716054in}{2.534493in}}%
\pgfpathlineto{\pgfqpoint{2.716920in}{2.581251in}}%
\pgfpathlineto{\pgfqpoint{2.718650in}{2.546144in}}%
\pgfpathlineto{\pgfqpoint{2.719515in}{2.559363in}}%
\pgfpathlineto{\pgfqpoint{2.720379in}{2.479433in}}%
\pgfpathlineto{\pgfqpoint{2.721245in}{2.563175in}}%
\pgfpathlineto{\pgfqpoint{2.722110in}{2.513249in}}%
\pgfpathlineto{\pgfqpoint{2.723839in}{2.568709in}}%
\pgfpathlineto{\pgfqpoint{2.724705in}{2.552230in}}%
\pgfpathlineto{\pgfqpoint{2.725571in}{2.488011in}}%
\pgfpathlineto{\pgfqpoint{2.726436in}{2.498371in}}%
\pgfpathlineto{\pgfqpoint{2.727301in}{2.558626in}}%
\pgfpathlineto{\pgfqpoint{2.729031in}{2.477527in}}%
\pgfpathlineto{\pgfqpoint{2.730761in}{2.530465in}}%
\pgfpathlineto{\pgfqpoint{2.731626in}{2.512880in}}%
\pgfpathlineto{\pgfqpoint{2.732492in}{2.453025in}}%
\pgfpathlineto{\pgfqpoint{2.734222in}{2.583834in}}%
\pgfpathlineto{\pgfqpoint{2.735950in}{2.530588in}}%
\pgfpathlineto{\pgfqpoint{2.736813in}{2.575411in}}%
\pgfpathlineto{\pgfqpoint{2.738539in}{2.485704in}}%
\pgfpathlineto{\pgfqpoint{2.739404in}{2.511282in}}%
\pgfpathlineto{\pgfqpoint{2.741132in}{2.598467in}}%
\pgfpathlineto{\pgfqpoint{2.741997in}{2.485458in}}%
\pgfpathlineto{\pgfqpoint{2.743725in}{2.547127in}}%
\pgfpathlineto{\pgfqpoint{2.744591in}{2.538919in}}%
\pgfpathlineto{\pgfqpoint{2.745455in}{2.519889in}}%
\pgfpathlineto{\pgfqpoint{2.746322in}{2.549218in}}%
\pgfpathlineto{\pgfqpoint{2.747187in}{2.537598in}}%
\pgfpathlineto{\pgfqpoint{2.748053in}{2.480909in}}%
\pgfpathlineto{\pgfqpoint{2.749784in}{2.559609in}}%
\pgfpathlineto{\pgfqpoint{2.750649in}{2.547127in}}%
\pgfpathlineto{\pgfqpoint{2.751515in}{2.511559in}}%
\pgfpathlineto{\pgfqpoint{2.752380in}{2.583588in}}%
\pgfpathlineto{\pgfqpoint{2.754111in}{2.494435in}}%
\pgfpathlineto{\pgfqpoint{2.754976in}{2.540672in}}%
\pgfpathlineto{\pgfqpoint{2.755840in}{2.492960in}}%
\pgfpathlineto{\pgfqpoint{2.756706in}{2.501198in}}%
\pgfpathlineto{\pgfqpoint{2.757572in}{2.503166in}}%
\pgfpathlineto{\pgfqpoint{2.758437in}{2.497755in}}%
\pgfpathlineto{\pgfqpoint{2.760166in}{2.509068in}}%
\pgfpathlineto{\pgfqpoint{2.761031in}{2.542024in}}%
\pgfpathlineto{\pgfqpoint{2.761898in}{2.472362in}}%
\pgfpathlineto{\pgfqpoint{2.762764in}{2.486012in}}%
\pgfpathlineto{\pgfqpoint{2.763629in}{2.485150in}}%
\pgfpathlineto{\pgfqpoint{2.764491in}{2.541962in}}%
\pgfpathlineto{\pgfqpoint{2.765356in}{2.479801in}}%
\pgfpathlineto{\pgfqpoint{2.766220in}{2.490192in}}%
\pgfpathlineto{\pgfqpoint{2.767951in}{2.574426in}}%
\pgfpathlineto{\pgfqpoint{2.768815in}{2.565757in}}%
\pgfpathlineto{\pgfqpoint{2.770546in}{2.473191in}}%
\pgfpathlineto{\pgfqpoint{2.772275in}{2.533047in}}%
\pgfpathlineto{\pgfqpoint{2.773139in}{2.507284in}}%
\pgfpathlineto{\pgfqpoint{2.774004in}{2.552846in}}%
\pgfpathlineto{\pgfqpoint{2.775728in}{2.449551in}}%
\pgfpathlineto{\pgfqpoint{2.776593in}{2.534155in}}%
\pgfpathlineto{\pgfqpoint{2.777458in}{2.500092in}}%
\pgfpathlineto{\pgfqpoint{2.778324in}{2.504765in}}%
\pgfpathlineto{\pgfqpoint{2.779188in}{2.552907in}}%
\pgfpathlineto{\pgfqpoint{2.780052in}{2.521980in}}%
\pgfpathlineto{\pgfqpoint{2.781780in}{2.548112in}}%
\pgfpathlineto{\pgfqpoint{2.782645in}{2.517339in}}%
\pgfpathlineto{\pgfqpoint{2.783507in}{2.558134in}}%
\pgfpathlineto{\pgfqpoint{2.785235in}{2.528928in}}%
\pgfpathlineto{\pgfqpoint{2.786965in}{2.595270in}}%
\pgfpathlineto{\pgfqpoint{2.787828in}{2.462095in}}%
\pgfpathlineto{\pgfqpoint{2.788692in}{2.587277in}}%
\pgfpathlineto{\pgfqpoint{2.789556in}{2.531081in}}%
\pgfpathlineto{\pgfqpoint{2.790421in}{2.565205in}}%
\pgfpathlineto{\pgfqpoint{2.792151in}{2.529975in}}%
\pgfpathlineto{\pgfqpoint{2.793017in}{2.505134in}}%
\pgfpathlineto{\pgfqpoint{2.793883in}{2.544238in}}%
\pgfpathlineto{\pgfqpoint{2.794748in}{2.524071in}}%
\pgfpathlineto{\pgfqpoint{2.795613in}{2.544792in}}%
\pgfpathlineto{\pgfqpoint{2.797343in}{2.498248in}}%
\pgfpathlineto{\pgfqpoint{2.798208in}{2.534401in}}%
\pgfpathlineto{\pgfqpoint{2.799072in}{2.534339in}}%
\pgfpathlineto{\pgfqpoint{2.799938in}{2.554506in}}%
\pgfpathlineto{\pgfqpoint{2.801665in}{2.504888in}}%
\pgfpathlineto{\pgfqpoint{2.802530in}{2.502368in}}%
\pgfpathlineto{\pgfqpoint{2.804259in}{2.548420in}}%
\pgfpathlineto{\pgfqpoint{2.805124in}{2.513496in}}%
\pgfpathlineto{\pgfqpoint{2.805989in}{2.583834in}}%
\pgfpathlineto{\pgfqpoint{2.806856in}{2.501445in}}%
\pgfpathlineto{\pgfqpoint{2.807721in}{2.554383in}}%
\pgfpathlineto{\pgfqpoint{2.808586in}{2.541133in}}%
\pgfpathlineto{\pgfqpoint{2.811181in}{2.499846in}}%
\pgfpathlineto{\pgfqpoint{2.812912in}{2.553584in}}%
\pgfpathlineto{\pgfqpoint{2.813775in}{2.541164in}}%
\pgfpathlineto{\pgfqpoint{2.814641in}{2.555244in}}%
\pgfpathlineto{\pgfqpoint{2.815503in}{2.465661in}}%
\pgfpathlineto{\pgfqpoint{2.816368in}{2.499231in}}%
\pgfpathlineto{\pgfqpoint{2.817232in}{2.490869in}}%
\pgfpathlineto{\pgfqpoint{2.818096in}{2.468858in}}%
\pgfpathlineto{\pgfqpoint{2.818961in}{2.540548in}}%
\pgfpathlineto{\pgfqpoint{2.819827in}{2.531633in}}%
\pgfpathlineto{\pgfqpoint{2.820692in}{2.481645in}}%
\pgfpathlineto{\pgfqpoint{2.822421in}{2.531510in}}%
\pgfpathlineto{\pgfqpoint{2.823283in}{2.528066in}}%
\pgfpathlineto{\pgfqpoint{2.824146in}{2.470764in}}%
\pgfpathlineto{\pgfqpoint{2.825873in}{2.556104in}}%
\pgfpathlineto{\pgfqpoint{2.826739in}{2.553705in}}%
\pgfpathlineto{\pgfqpoint{2.827603in}{2.494066in}}%
\pgfpathlineto{\pgfqpoint{2.828468in}{2.510974in}}%
\pgfpathlineto{\pgfqpoint{2.829334in}{2.498553in}}%
\pgfpathlineto{\pgfqpoint{2.831064in}{2.588999in}}%
\pgfpathlineto{\pgfqpoint{2.831927in}{2.531019in}}%
\pgfpathlineto{\pgfqpoint{2.832789in}{2.564835in}}%
\pgfpathlineto{\pgfqpoint{2.833654in}{2.533110in}}%
\pgfpathlineto{\pgfqpoint{2.835386in}{2.631731in}}%
\pgfpathlineto{\pgfqpoint{2.837117in}{2.525793in}}%
\pgfpathlineto{\pgfqpoint{2.837982in}{2.537967in}}%
\pgfpathlineto{\pgfqpoint{2.838847in}{2.593119in}}%
\pgfpathlineto{\pgfqpoint{2.839712in}{2.521367in}}%
\pgfpathlineto{\pgfqpoint{2.840577in}{2.576763in}}%
\pgfpathlineto{\pgfqpoint{2.841442in}{2.563483in}}%
\pgfpathlineto{\pgfqpoint{2.843172in}{2.593794in}}%
\pgfpathlineto{\pgfqpoint{2.844901in}{2.487180in}}%
\pgfpathlineto{\pgfqpoint{2.845764in}{2.548725in}}%
\pgfpathlineto{\pgfqpoint{2.846628in}{2.516385in}}%
\pgfpathlineto{\pgfqpoint{2.847492in}{2.557641in}}%
\pgfpathlineto{\pgfqpoint{2.848356in}{2.551001in}}%
\pgfpathlineto{\pgfqpoint{2.849221in}{2.528559in}}%
\pgfpathlineto{\pgfqpoint{2.850086in}{2.571722in}}%
\pgfpathlineto{\pgfqpoint{2.850950in}{2.534401in}}%
\pgfpathlineto{\pgfqpoint{2.851815in}{2.547804in}}%
\pgfpathlineto{\pgfqpoint{2.852680in}{2.599021in}}%
\pgfpathlineto{\pgfqpoint{2.854407in}{2.507962in}}%
\pgfpathlineto{\pgfqpoint{2.856137in}{2.528590in}}%
\pgfpathlineto{\pgfqpoint{2.857003in}{2.461910in}}%
\pgfpathlineto{\pgfqpoint{2.858734in}{2.535476in}}%
\pgfpathlineto{\pgfqpoint{2.859599in}{2.483000in}}%
\pgfpathlineto{\pgfqpoint{2.860465in}{2.565757in}}%
\pgfpathlineto{\pgfqpoint{2.861329in}{2.516693in}}%
\pgfpathlineto{\pgfqpoint{2.862193in}{2.612547in}}%
\pgfpathlineto{\pgfqpoint{2.863056in}{2.499785in}}%
\pgfpathlineto{\pgfqpoint{2.863921in}{2.585494in}}%
\pgfpathlineto{\pgfqpoint{2.864787in}{2.492529in}}%
\pgfpathlineto{\pgfqpoint{2.866517in}{2.573689in}}%
\pgfpathlineto{\pgfqpoint{2.869111in}{2.489270in}}%
\pgfpathlineto{\pgfqpoint{2.869976in}{2.582236in}}%
\pgfpathlineto{\pgfqpoint{2.872568in}{2.498063in}}%
\pgfpathlineto{\pgfqpoint{2.873433in}{2.549464in}}%
\pgfpathlineto{\pgfqpoint{2.874296in}{2.523579in}}%
\pgfpathlineto{\pgfqpoint{2.875161in}{2.543653in}}%
\pgfpathlineto{\pgfqpoint{2.876025in}{2.593548in}}%
\pgfpathlineto{\pgfqpoint{2.877753in}{2.487641in}}%
\pgfpathlineto{\pgfqpoint{2.878619in}{2.542455in}}%
\pgfpathlineto{\pgfqpoint{2.879484in}{2.426003in}}%
\pgfpathlineto{\pgfqpoint{2.881212in}{2.531817in}}%
\pgfpathlineto{\pgfqpoint{2.882076in}{2.567817in}}%
\pgfpathlineto{\pgfqpoint{2.883803in}{2.529175in}}%
\pgfpathlineto{\pgfqpoint{2.884668in}{2.557610in}}%
\pgfpathlineto{\pgfqpoint{2.885532in}{2.556412in}}%
\pgfpathlineto{\pgfqpoint{2.886397in}{2.483983in}}%
\pgfpathlineto{\pgfqpoint{2.887263in}{2.565603in}}%
\pgfpathlineto{\pgfqpoint{2.888127in}{2.503351in}}%
\pgfpathlineto{\pgfqpoint{2.888993in}{2.562131in}}%
\pgfpathlineto{\pgfqpoint{2.889857in}{2.507471in}}%
\pgfpathlineto{\pgfqpoint{2.890721in}{2.559673in}}%
\pgfpathlineto{\pgfqpoint{2.891587in}{2.516079in}}%
\pgfpathlineto{\pgfqpoint{2.892452in}{2.613778in}}%
\pgfpathlineto{\pgfqpoint{2.894183in}{2.510545in}}%
\pgfpathlineto{\pgfqpoint{2.895047in}{2.552907in}}%
\pgfpathlineto{\pgfqpoint{2.896779in}{2.506117in}}%
\pgfpathlineto{\pgfqpoint{2.897643in}{2.568771in}}%
\pgfpathlineto{\pgfqpoint{2.898505in}{2.470579in}}%
\pgfpathlineto{\pgfqpoint{2.899368in}{2.471993in}}%
\pgfpathlineto{\pgfqpoint{2.901099in}{2.521919in}}%
\pgfpathlineto{\pgfqpoint{2.901964in}{2.553829in}}%
\pgfpathlineto{\pgfqpoint{2.903693in}{2.490315in}}%
\pgfpathlineto{\pgfqpoint{2.904557in}{2.524500in}}%
\pgfpathlineto{\pgfqpoint{2.905420in}{2.493450in}}%
\pgfpathlineto{\pgfqpoint{2.906286in}{2.549156in}}%
\pgfpathlineto{\pgfqpoint{2.907152in}{2.501506in}}%
\pgfpathlineto{\pgfqpoint{2.908015in}{2.565633in}}%
\pgfpathlineto{\pgfqpoint{2.909746in}{2.526099in}}%
\pgfpathlineto{\pgfqpoint{2.910611in}{2.530157in}}%
\pgfpathlineto{\pgfqpoint{2.911476in}{2.540302in}}%
\pgfpathlineto{\pgfqpoint{2.912341in}{2.586969in}}%
\pgfpathlineto{\pgfqpoint{2.913205in}{2.499415in}}%
\pgfpathlineto{\pgfqpoint{2.914071in}{2.555673in}}%
\pgfpathlineto{\pgfqpoint{2.914936in}{2.535692in}}%
\pgfpathlineto{\pgfqpoint{2.915801in}{2.449151in}}%
\pgfpathlineto{\pgfqpoint{2.918398in}{2.622138in}}%
\pgfpathlineto{\pgfqpoint{2.920993in}{2.498984in}}%
\pgfpathlineto{\pgfqpoint{2.921858in}{2.516446in}}%
\pgfpathlineto{\pgfqpoint{2.922722in}{2.601971in}}%
\pgfpathlineto{\pgfqpoint{2.924453in}{2.495787in}}%
\pgfpathlineto{\pgfqpoint{2.925318in}{2.502797in}}%
\pgfpathlineto{\pgfqpoint{2.926183in}{2.553952in}}%
\pgfpathlineto{\pgfqpoint{2.927911in}{2.480293in}}%
\pgfpathlineto{\pgfqpoint{2.929642in}{2.564096in}}%
\pgfpathlineto{\pgfqpoint{2.930505in}{2.526960in}}%
\pgfpathlineto{\pgfqpoint{2.931370in}{2.585371in}}%
\pgfpathlineto{\pgfqpoint{2.932235in}{2.500183in}}%
\pgfpathlineto{\pgfqpoint{2.933966in}{2.585433in}}%
\pgfpathlineto{\pgfqpoint{2.935694in}{2.512942in}}%
\pgfpathlineto{\pgfqpoint{2.936558in}{2.511589in}}%
\pgfpathlineto{\pgfqpoint{2.938287in}{2.545467in}}%
\pgfpathlineto{\pgfqpoint{2.939153in}{2.497509in}}%
\pgfpathlineto{\pgfqpoint{2.940017in}{2.556043in}}%
\pgfpathlineto{\pgfqpoint{2.940882in}{2.518045in}}%
\pgfpathlineto{\pgfqpoint{2.941744in}{2.545652in}}%
\pgfpathlineto{\pgfqpoint{2.943475in}{2.447891in}}%
\pgfpathlineto{\pgfqpoint{2.944340in}{2.451273in}}%
\pgfpathlineto{\pgfqpoint{2.946068in}{2.566803in}}%
\pgfpathlineto{\pgfqpoint{2.946935in}{2.558811in}}%
\pgfpathlineto{\pgfqpoint{2.948660in}{2.514358in}}%
\pgfpathlineto{\pgfqpoint{2.949527in}{2.433936in}}%
\pgfpathlineto{\pgfqpoint{2.950389in}{2.572276in}}%
\pgfpathlineto{\pgfqpoint{2.951255in}{2.532987in}}%
\pgfpathlineto{\pgfqpoint{2.952121in}{2.497265in}}%
\pgfpathlineto{\pgfqpoint{2.952984in}{2.537967in}}%
\pgfpathlineto{\pgfqpoint{2.953848in}{2.475898in}}%
\pgfpathlineto{\pgfqpoint{2.955575in}{2.541410in}}%
\pgfpathlineto{\pgfqpoint{2.956440in}{2.515894in}}%
\pgfpathlineto{\pgfqpoint{2.957306in}{2.551863in}}%
\pgfpathlineto{\pgfqpoint{2.959036in}{2.507348in}}%
\pgfpathlineto{\pgfqpoint{2.961632in}{2.607138in}}%
\pgfpathlineto{\pgfqpoint{2.962497in}{2.644582in}}%
\pgfpathlineto{\pgfqpoint{2.963362in}{2.569479in}}%
\pgfpathlineto{\pgfqpoint{2.964228in}{2.596993in}}%
\pgfpathlineto{\pgfqpoint{2.965093in}{2.669605in}}%
\pgfpathlineto{\pgfqpoint{2.965957in}{2.573997in}}%
\pgfpathlineto{\pgfqpoint{2.966821in}{2.625521in}}%
\pgfpathlineto{\pgfqpoint{2.968551in}{2.572276in}}%
\pgfpathlineto{\pgfqpoint{2.970278in}{2.598253in}}%
\pgfpathlineto{\pgfqpoint{2.972009in}{2.682272in}}%
\pgfpathlineto{\pgfqpoint{2.973738in}{2.620849in}}%
\pgfpathlineto{\pgfqpoint{2.974603in}{2.660138in}}%
\pgfpathlineto{\pgfqpoint{2.975468in}{2.654911in}}%
\pgfpathlineto{\pgfqpoint{2.976333in}{2.591521in}}%
\pgfpathlineto{\pgfqpoint{2.978927in}{2.673418in}}%
\pgfpathlineto{\pgfqpoint{2.980655in}{2.613163in}}%
\pgfpathlineto{\pgfqpoint{2.983249in}{2.679198in}}%
\pgfpathlineto{\pgfqpoint{2.984114in}{2.620480in}}%
\pgfpathlineto{\pgfqpoint{2.984979in}{2.673233in}}%
\pgfpathlineto{\pgfqpoint{2.987576in}{2.559732in}}%
\pgfpathlineto{\pgfqpoint{2.990170in}{2.652513in}}%
\pgfpathlineto{\pgfqpoint{2.992763in}{2.610150in}}%
\pgfpathlineto{\pgfqpoint{2.993628in}{2.678521in}}%
\pgfpathlineto{\pgfqpoint{2.994494in}{2.626750in}}%
\pgfpathlineto{\pgfqpoint{2.995360in}{2.644641in}}%
\pgfpathlineto{\pgfqpoint{2.998820in}{2.509529in}}%
\pgfpathlineto{\pgfqpoint{2.999686in}{2.540610in}}%
\pgfpathlineto{\pgfqpoint{3.000551in}{2.535815in}}%
\pgfpathlineto{\pgfqpoint{3.001416in}{2.506363in}}%
\pgfpathlineto{\pgfqpoint{3.002282in}{2.566495in}}%
\pgfpathlineto{\pgfqpoint{3.004010in}{2.519951in}}%
\pgfpathlineto{\pgfqpoint{3.004875in}{2.507346in}}%
\pgfpathlineto{\pgfqpoint{3.005740in}{2.540087in}}%
\pgfpathlineto{\pgfqpoint{3.007470in}{2.497694in}}%
\pgfpathlineto{\pgfqpoint{3.008335in}{2.542393in}}%
\pgfpathlineto{\pgfqpoint{3.010930in}{2.479095in}}%
\pgfpathlineto{\pgfqpoint{3.012659in}{2.533908in}}%
\pgfpathlineto{\pgfqpoint{3.013525in}{2.537413in}}%
\pgfpathlineto{\pgfqpoint{3.014390in}{2.502551in}}%
\pgfpathlineto{\pgfqpoint{3.015255in}{2.542609in}}%
\pgfpathlineto{\pgfqpoint{3.017850in}{2.484383in}}%
\pgfpathlineto{\pgfqpoint{3.019578in}{2.583095in}}%
\pgfpathlineto{\pgfqpoint{3.020443in}{2.542301in}}%
\pgfpathlineto{\pgfqpoint{3.021309in}{2.558072in}}%
\pgfpathlineto{\pgfqpoint{3.023036in}{2.527145in}}%
\pgfpathlineto{\pgfqpoint{3.023900in}{2.528313in}}%
\pgfpathlineto{\pgfqpoint{3.024765in}{2.523302in}}%
\pgfpathlineto{\pgfqpoint{3.025630in}{2.476850in}}%
\pgfpathlineto{\pgfqpoint{3.027361in}{2.522625in}}%
\pgfpathlineto{\pgfqpoint{3.028227in}{2.596745in}}%
\pgfpathlineto{\pgfqpoint{3.029959in}{2.499292in}}%
\pgfpathlineto{\pgfqpoint{3.031687in}{2.573197in}}%
\pgfpathlineto{\pgfqpoint{3.032552in}{2.509622in}}%
\pgfpathlineto{\pgfqpoint{3.033415in}{2.518353in}}%
\pgfpathlineto{\pgfqpoint{3.034281in}{2.494127in}}%
\pgfpathlineto{\pgfqpoint{3.035146in}{2.603323in}}%
\pgfpathlineto{\pgfqpoint{3.036011in}{2.497078in}}%
\pgfpathlineto{\pgfqpoint{3.036876in}{2.579591in}}%
\pgfpathlineto{\pgfqpoint{3.037739in}{2.576332in}}%
\pgfpathlineto{\pgfqpoint{3.038605in}{2.569261in}}%
\pgfpathlineto{\pgfqpoint{3.039470in}{2.442726in}}%
\pgfpathlineto{\pgfqpoint{3.041199in}{2.514048in}}%
\pgfpathlineto{\pgfqpoint{3.042064in}{2.519212in}}%
\pgfpathlineto{\pgfqpoint{3.042930in}{2.500521in}}%
\pgfpathlineto{\pgfqpoint{3.043794in}{2.522779in}}%
\pgfpathlineto{\pgfqpoint{3.045523in}{2.485335in}}%
\pgfpathlineto{\pgfqpoint{3.047252in}{2.519335in}}%
\pgfpathlineto{\pgfqpoint{3.048117in}{2.516508in}}%
\pgfpathlineto{\pgfqpoint{3.048983in}{2.497417in}}%
\pgfpathlineto{\pgfqpoint{3.049848in}{2.507223in}}%
\pgfpathlineto{\pgfqpoint{3.050713in}{2.490007in}}%
\pgfpathlineto{\pgfqpoint{3.052443in}{2.551922in}}%
\pgfpathlineto{\pgfqpoint{3.053308in}{2.502243in}}%
\pgfpathlineto{\pgfqpoint{3.054173in}{2.512018in}}%
\pgfpathlineto{\pgfqpoint{3.055037in}{2.515708in}}%
\pgfpathlineto{\pgfqpoint{3.055901in}{2.502120in}}%
\pgfpathlineto{\pgfqpoint{3.057631in}{2.420959in}}%
\pgfpathlineto{\pgfqpoint{3.059359in}{2.589489in}}%
\pgfpathlineto{\pgfqpoint{3.060224in}{2.552415in}}%
\pgfpathlineto{\pgfqpoint{3.061088in}{2.559178in}}%
\pgfpathlineto{\pgfqpoint{3.061953in}{2.554937in}}%
\pgfpathlineto{\pgfqpoint{3.062818in}{2.537013in}}%
\pgfpathlineto{\pgfqpoint{3.063682in}{2.562252in}}%
\pgfpathlineto{\pgfqpoint{3.064545in}{2.542208in}}%
\pgfpathlineto{\pgfqpoint{3.065410in}{2.548849in}}%
\pgfpathlineto{\pgfqpoint{3.067139in}{2.505871in}}%
\pgfpathlineto{\pgfqpoint{3.068004in}{2.584109in}}%
\pgfpathlineto{\pgfqpoint{3.070600in}{2.488347in}}%
\pgfpathlineto{\pgfqpoint{3.073195in}{2.590166in}}%
\pgfpathlineto{\pgfqpoint{3.074926in}{2.521765in}}%
\pgfpathlineto{\pgfqpoint{3.075790in}{2.595516in}}%
\pgfpathlineto{\pgfqpoint{3.076655in}{2.523702in}}%
\pgfpathlineto{\pgfqpoint{3.077520in}{2.528867in}}%
\pgfpathlineto{\pgfqpoint{3.080116in}{2.482015in}}%
\pgfpathlineto{\pgfqpoint{3.080982in}{2.572520in}}%
\pgfpathlineto{\pgfqpoint{3.081847in}{2.514540in}}%
\pgfpathlineto{\pgfqpoint{3.082712in}{2.605476in}}%
\pgfpathlineto{\pgfqpoint{3.083578in}{2.595393in}}%
\pgfpathlineto{\pgfqpoint{3.084443in}{2.577009in}}%
\pgfpathlineto{\pgfqpoint{3.086170in}{2.511713in}}%
\pgfpathlineto{\pgfqpoint{3.087036in}{2.501321in}}%
\pgfpathlineto{\pgfqpoint{3.088767in}{2.554198in}}%
\pgfpathlineto{\pgfqpoint{3.092227in}{2.495172in}}%
\pgfpathlineto{\pgfqpoint{3.093092in}{2.560407in}}%
\pgfpathlineto{\pgfqpoint{3.093957in}{2.505409in}}%
\pgfpathlineto{\pgfqpoint{3.094822in}{2.512695in}}%
\pgfpathlineto{\pgfqpoint{3.097417in}{2.552230in}}%
\pgfpathlineto{\pgfqpoint{3.098282in}{2.523086in}}%
\pgfpathlineto{\pgfqpoint{3.099147in}{2.550078in}}%
\pgfpathlineto{\pgfqpoint{3.100012in}{2.510176in}}%
\pgfpathlineto{\pgfqpoint{3.100878in}{2.520228in}}%
\pgfpathlineto{\pgfqpoint{3.102605in}{2.490561in}}%
\pgfpathlineto{\pgfqpoint{3.103470in}{2.497170in}}%
\pgfpathlineto{\pgfqpoint{3.104334in}{2.522840in}}%
\pgfpathlineto{\pgfqpoint{3.105199in}{2.497201in}}%
\pgfpathlineto{\pgfqpoint{3.106930in}{2.563114in}}%
\pgfpathlineto{\pgfqpoint{3.107795in}{2.528867in}}%
\pgfpathlineto{\pgfqpoint{3.108661in}{2.535507in}}%
\pgfpathlineto{\pgfqpoint{3.111255in}{2.478264in}}%
\pgfpathlineto{\pgfqpoint{3.112119in}{2.530157in}}%
\pgfpathlineto{\pgfqpoint{3.113850in}{2.394399in}}%
\pgfpathlineto{\pgfqpoint{3.114714in}{2.590780in}}%
\pgfpathlineto{\pgfqpoint{3.115579in}{2.495510in}}%
\pgfpathlineto{\pgfqpoint{3.116443in}{2.585309in}}%
\pgfpathlineto{\pgfqpoint{3.117306in}{2.480109in}}%
\pgfpathlineto{\pgfqpoint{3.118172in}{2.531694in}}%
\pgfpathlineto{\pgfqpoint{3.119902in}{2.485150in}}%
\pgfpathlineto{\pgfqpoint{3.120768in}{2.543992in}}%
\pgfpathlineto{\pgfqpoint{3.121631in}{2.511589in}}%
\pgfpathlineto{\pgfqpoint{3.123362in}{2.560715in}}%
\pgfpathlineto{\pgfqpoint{3.125092in}{2.439745in}}%
\pgfpathlineto{\pgfqpoint{3.125956in}{2.455270in}}%
\pgfpathlineto{\pgfqpoint{3.127686in}{2.537598in}}%
\pgfpathlineto{\pgfqpoint{3.128550in}{2.489211in}}%
\pgfpathlineto{\pgfqpoint{3.130279in}{2.544484in}}%
\pgfpathlineto{\pgfqpoint{3.131143in}{2.416166in}}%
\pgfpathlineto{\pgfqpoint{3.133740in}{2.587216in}}%
\pgfpathlineto{\pgfqpoint{3.134605in}{2.534031in}}%
\pgfpathlineto{\pgfqpoint{3.135471in}{2.552846in}}%
\pgfpathlineto{\pgfqpoint{3.137201in}{2.494928in}}%
\pgfpathlineto{\pgfqpoint{3.138066in}{2.544607in}}%
\pgfpathlineto{\pgfqpoint{3.138930in}{2.513588in}}%
\pgfpathlineto{\pgfqpoint{3.139795in}{2.551863in}}%
\pgfpathlineto{\pgfqpoint{3.141527in}{2.489855in}}%
\pgfpathlineto{\pgfqpoint{3.142392in}{2.501814in}}%
\pgfpathlineto{\pgfqpoint{3.143259in}{2.496465in}}%
\pgfpathlineto{\pgfqpoint{3.144124in}{2.571475in}}%
\pgfpathlineto{\pgfqpoint{3.146723in}{2.463570in}}%
\pgfpathlineto{\pgfqpoint{3.147589in}{2.608673in}}%
\pgfpathlineto{\pgfqpoint{3.148455in}{2.530835in}}%
\pgfpathlineto{\pgfqpoint{3.149320in}{2.551432in}}%
\pgfpathlineto{\pgfqpoint{3.150186in}{2.554506in}}%
\pgfpathlineto{\pgfqpoint{3.151050in}{2.580237in}}%
\pgfpathlineto{\pgfqpoint{3.154510in}{2.494374in}}%
\pgfpathlineto{\pgfqpoint{3.155373in}{2.525547in}}%
\pgfpathlineto{\pgfqpoint{3.156236in}{2.492775in}}%
\pgfpathlineto{\pgfqpoint{3.157101in}{2.559609in}}%
\pgfpathlineto{\pgfqpoint{3.157967in}{2.532556in}}%
\pgfpathlineto{\pgfqpoint{3.158829in}{2.470395in}}%
\pgfpathlineto{\pgfqpoint{3.159695in}{2.510789in}}%
\pgfpathlineto{\pgfqpoint{3.161425in}{2.485766in}}%
\pgfpathlineto{\pgfqpoint{3.162292in}{2.535322in}}%
\pgfpathlineto{\pgfqpoint{3.163156in}{2.485550in}}%
\pgfpathlineto{\pgfqpoint{3.164886in}{2.631053in}}%
\pgfpathlineto{\pgfqpoint{3.165751in}{2.501260in}}%
\pgfpathlineto{\pgfqpoint{3.167480in}{2.564404in}}%
\pgfpathlineto{\pgfqpoint{3.169210in}{2.530650in}}%
\pgfpathlineto{\pgfqpoint{3.170075in}{2.526683in}}%
\pgfpathlineto{\pgfqpoint{3.170938in}{2.465291in}}%
\pgfpathlineto{\pgfqpoint{3.173530in}{2.541962in}}%
\pgfpathlineto{\pgfqpoint{3.174393in}{2.472178in}}%
\pgfpathlineto{\pgfqpoint{3.176124in}{2.519028in}}%
\pgfpathlineto{\pgfqpoint{3.176989in}{2.502150in}}%
\pgfpathlineto{\pgfqpoint{3.177853in}{2.502304in}}%
\pgfpathlineto{\pgfqpoint{3.178719in}{2.507715in}}%
\pgfpathlineto{\pgfqpoint{3.179583in}{2.523394in}}%
\pgfpathlineto{\pgfqpoint{3.180447in}{2.487241in}}%
\pgfpathlineto{\pgfqpoint{3.181311in}{2.523333in}}%
\pgfpathlineto{\pgfqpoint{3.182175in}{2.444725in}}%
\pgfpathlineto{\pgfqpoint{3.183039in}{2.562929in}}%
\pgfpathlineto{\pgfqpoint{3.183904in}{2.533724in}}%
\pgfpathlineto{\pgfqpoint{3.184769in}{2.520444in}}%
\pgfpathlineto{\pgfqpoint{3.185635in}{2.479618in}}%
\pgfpathlineto{\pgfqpoint{3.189095in}{2.562498in}}%
\pgfpathlineto{\pgfqpoint{3.190825in}{2.485273in}}%
\pgfpathlineto{\pgfqpoint{3.192553in}{2.541408in}}%
\pgfpathlineto{\pgfqpoint{3.193417in}{2.552415in}}%
\pgfpathlineto{\pgfqpoint{3.194278in}{2.499169in}}%
\pgfpathlineto{\pgfqpoint{3.195143in}{2.567786in}}%
\pgfpathlineto{\pgfqpoint{3.196006in}{2.547127in}}%
\pgfpathlineto{\pgfqpoint{3.197734in}{2.581866in}}%
\pgfpathlineto{\pgfqpoint{3.199463in}{2.542085in}}%
\pgfpathlineto{\pgfqpoint{3.200327in}{2.522042in}}%
\pgfpathlineto{\pgfqpoint{3.201193in}{2.437315in}}%
\pgfpathlineto{\pgfqpoint{3.202922in}{2.547065in}}%
\pgfpathlineto{\pgfqpoint{3.203786in}{2.495849in}}%
\pgfpathlineto{\pgfqpoint{3.204651in}{2.590043in}}%
\pgfpathlineto{\pgfqpoint{3.205515in}{2.554075in}}%
\pgfpathlineto{\pgfqpoint{3.207245in}{2.576763in}}%
\pgfpathlineto{\pgfqpoint{3.208109in}{2.535753in}}%
\pgfpathlineto{\pgfqpoint{3.208975in}{2.580022in}}%
\pgfpathlineto{\pgfqpoint{3.210707in}{2.519582in}}%
\pgfpathlineto{\pgfqpoint{3.212439in}{2.590720in}}%
\pgfpathlineto{\pgfqpoint{3.213304in}{2.612670in}}%
\pgfpathlineto{\pgfqpoint{3.215032in}{2.528097in}}%
\pgfpathlineto{\pgfqpoint{3.215896in}{2.497817in}}%
\pgfpathlineto{\pgfqpoint{3.216762in}{2.559547in}}%
\pgfpathlineto{\pgfqpoint{3.217627in}{2.556135in}}%
\pgfpathlineto{\pgfqpoint{3.218492in}{2.571044in}}%
\pgfpathlineto{\pgfqpoint{3.219357in}{2.554260in}}%
\pgfpathlineto{\pgfqpoint{3.220222in}{2.569908in}}%
\pgfpathlineto{\pgfqpoint{3.221951in}{2.530342in}}%
\pgfpathlineto{\pgfqpoint{3.222816in}{2.560284in}}%
\pgfpathlineto{\pgfqpoint{3.223681in}{2.540241in}}%
\pgfpathlineto{\pgfqpoint{3.224547in}{2.548695in}}%
\pgfpathlineto{\pgfqpoint{3.225412in}{2.577623in}}%
\pgfpathlineto{\pgfqpoint{3.226276in}{2.573505in}}%
\pgfpathlineto{\pgfqpoint{3.227141in}{2.453364in}}%
\pgfpathlineto{\pgfqpoint{3.228006in}{2.557333in}}%
\pgfpathlineto{\pgfqpoint{3.228872in}{2.540179in}}%
\pgfpathlineto{\pgfqpoint{3.229739in}{2.504488in}}%
\pgfpathlineto{\pgfqpoint{3.230605in}{2.556902in}}%
\pgfpathlineto{\pgfqpoint{3.231470in}{2.523148in}}%
\pgfpathlineto{\pgfqpoint{3.232335in}{2.590474in}}%
\pgfpathlineto{\pgfqpoint{3.233200in}{2.513619in}}%
\pgfpathlineto{\pgfqpoint{3.234064in}{2.532218in}}%
\pgfpathlineto{\pgfqpoint{3.234929in}{2.473961in}}%
\pgfpathlineto{\pgfqpoint{3.237525in}{2.566311in}}%
\pgfpathlineto{\pgfqpoint{3.240120in}{2.478387in}}%
\pgfpathlineto{\pgfqpoint{3.240984in}{2.523025in}}%
\pgfpathlineto{\pgfqpoint{3.241849in}{2.502427in}}%
\pgfpathlineto{\pgfqpoint{3.242713in}{2.527207in}}%
\pgfpathlineto{\pgfqpoint{3.243578in}{2.507746in}}%
\pgfpathlineto{\pgfqpoint{3.245308in}{2.533970in}}%
\pgfpathlineto{\pgfqpoint{3.246173in}{2.504580in}}%
\pgfpathlineto{\pgfqpoint{3.247039in}{2.543376in}}%
\pgfpathlineto{\pgfqpoint{3.247901in}{2.518414in}}%
\pgfpathlineto{\pgfqpoint{3.248767in}{2.542976in}}%
\pgfpathlineto{\pgfqpoint{3.249633in}{2.505994in}}%
\pgfpathlineto{\pgfqpoint{3.250497in}{2.569877in}}%
\pgfpathlineto{\pgfqpoint{3.251361in}{2.471562in}}%
\pgfpathlineto{\pgfqpoint{3.252226in}{2.528374in}}%
\pgfpathlineto{\pgfqpoint{3.253089in}{2.451396in}}%
\pgfpathlineto{\pgfqpoint{3.253954in}{2.548725in}}%
\pgfpathlineto{\pgfqpoint{3.254819in}{2.509006in}}%
\pgfpathlineto{\pgfqpoint{3.255685in}{2.541931in}}%
\pgfpathlineto{\pgfqpoint{3.256550in}{2.456253in}}%
\pgfpathlineto{\pgfqpoint{3.259145in}{2.531325in}}%
\pgfpathlineto{\pgfqpoint{3.260010in}{2.507223in}}%
\pgfpathlineto{\pgfqpoint{3.260876in}{2.539687in}}%
\pgfpathlineto{\pgfqpoint{3.261740in}{2.475498in}}%
\pgfpathlineto{\pgfqpoint{3.262605in}{2.541901in}}%
\pgfpathlineto{\pgfqpoint{3.263469in}{2.470025in}}%
\pgfpathlineto{\pgfqpoint{3.265199in}{2.579529in}}%
\pgfpathlineto{\pgfqpoint{3.266064in}{2.561023in}}%
\pgfpathlineto{\pgfqpoint{3.266929in}{2.545775in}}%
\pgfpathlineto{\pgfqpoint{3.267794in}{2.494466in}}%
\pgfpathlineto{\pgfqpoint{3.269522in}{2.541408in}}%
\pgfpathlineto{\pgfqpoint{3.270385in}{2.512757in}}%
\pgfpathlineto{\pgfqpoint{3.272980in}{2.562252in}}%
\pgfpathlineto{\pgfqpoint{3.273846in}{2.538211in}}%
\pgfpathlineto{\pgfqpoint{3.274709in}{2.541408in}}%
\pgfpathlineto{\pgfqpoint{3.275573in}{2.526160in}}%
\pgfpathlineto{\pgfqpoint{3.276438in}{2.475005in}}%
\pgfpathlineto{\pgfqpoint{3.278164in}{2.570983in}}%
\pgfpathlineto{\pgfqpoint{3.279893in}{2.501198in}}%
\pgfpathlineto{\pgfqpoint{3.280759in}{2.540487in}}%
\pgfpathlineto{\pgfqpoint{3.281623in}{2.493083in}}%
\pgfpathlineto{\pgfqpoint{3.282489in}{2.536736in}}%
\pgfpathlineto{\pgfqpoint{3.283353in}{2.525177in}}%
\pgfpathlineto{\pgfqpoint{3.284218in}{2.529880in}}%
\pgfpathlineto{\pgfqpoint{3.285084in}{2.500213in}}%
\pgfpathlineto{\pgfqpoint{3.285950in}{2.579652in}}%
\pgfpathlineto{\pgfqpoint{3.286815in}{2.466521in}}%
\pgfpathlineto{\pgfqpoint{3.287681in}{2.516015in}}%
\pgfpathlineto{\pgfqpoint{3.288546in}{2.500398in}}%
\pgfpathlineto{\pgfqpoint{3.289412in}{2.557025in}}%
\pgfpathlineto{\pgfqpoint{3.293738in}{2.466767in}}%
\pgfpathlineto{\pgfqpoint{3.294599in}{2.580083in}}%
\pgfpathlineto{\pgfqpoint{3.295464in}{2.571660in}}%
\pgfpathlineto{\pgfqpoint{3.296325in}{2.540518in}}%
\pgfpathlineto{\pgfqpoint{3.297192in}{2.543745in}}%
\pgfpathlineto{\pgfqpoint{3.298922in}{2.499538in}}%
\pgfpathlineto{\pgfqpoint{3.299786in}{2.526776in}}%
\pgfpathlineto{\pgfqpoint{3.300649in}{2.526160in}}%
\pgfpathlineto{\pgfqpoint{3.301516in}{2.505317in}}%
\pgfpathlineto{\pgfqpoint{3.302379in}{2.514109in}}%
\pgfpathlineto{\pgfqpoint{3.303244in}{2.551953in}}%
\pgfpathlineto{\pgfqpoint{3.304108in}{2.514602in}}%
\pgfpathlineto{\pgfqpoint{3.304975in}{2.524931in}}%
\pgfpathlineto{\pgfqpoint{3.306706in}{2.565818in}}%
\pgfpathlineto{\pgfqpoint{3.308436in}{2.489699in}}%
\pgfpathlineto{\pgfqpoint{3.310168in}{2.520595in}}%
\pgfpathlineto{\pgfqpoint{3.311034in}{2.537965in}}%
\pgfpathlineto{\pgfqpoint{3.311899in}{2.439712in}}%
\pgfpathlineto{\pgfqpoint{3.313628in}{2.585615in}}%
\pgfpathlineto{\pgfqpoint{3.314493in}{2.583586in}}%
\pgfpathlineto{\pgfqpoint{3.315359in}{2.490284in}}%
\pgfpathlineto{\pgfqpoint{3.317088in}{2.601417in}}%
\pgfpathlineto{\pgfqpoint{3.317953in}{2.515584in}}%
\pgfpathlineto{\pgfqpoint{3.318817in}{2.528374in}}%
\pgfpathlineto{\pgfqpoint{3.319683in}{2.534860in}}%
\pgfpathlineto{\pgfqpoint{3.320547in}{2.557025in}}%
\pgfpathlineto{\pgfqpoint{3.321413in}{2.513249in}}%
\pgfpathlineto{\pgfqpoint{3.323139in}{2.579529in}}%
\pgfpathlineto{\pgfqpoint{3.324004in}{2.502427in}}%
\pgfpathlineto{\pgfqpoint{3.325734in}{2.636095in}}%
\pgfpathlineto{\pgfqpoint{3.328325in}{2.502735in}}%
\pgfpathlineto{\pgfqpoint{3.329190in}{2.483983in}}%
\pgfpathlineto{\pgfqpoint{3.330054in}{2.570798in}}%
\pgfpathlineto{\pgfqpoint{3.330917in}{2.558378in}}%
\pgfpathlineto{\pgfqpoint{3.331782in}{2.548479in}}%
\pgfpathlineto{\pgfqpoint{3.334376in}{2.480601in}}%
\pgfpathlineto{\pgfqpoint{3.336106in}{2.545528in}}%
\pgfpathlineto{\pgfqpoint{3.336971in}{2.534830in}}%
\pgfpathlineto{\pgfqpoint{3.337833in}{2.515584in}}%
\pgfpathlineto{\pgfqpoint{3.338698in}{2.520872in}}%
\pgfpathlineto{\pgfqpoint{3.339562in}{2.516139in}}%
\pgfpathlineto{\pgfqpoint{3.340427in}{2.474328in}}%
\pgfpathlineto{\pgfqpoint{3.343019in}{2.573626in}}%
\pgfpathlineto{\pgfqpoint{3.343883in}{2.498399in}}%
\pgfpathlineto{\pgfqpoint{3.344746in}{2.557210in}}%
\pgfpathlineto{\pgfqpoint{3.346476in}{2.456745in}}%
\pgfpathlineto{\pgfqpoint{3.347340in}{2.581620in}}%
\pgfpathlineto{\pgfqpoint{3.349071in}{2.453056in}}%
\pgfpathlineto{\pgfqpoint{3.350801in}{2.545867in}}%
\pgfpathlineto{\pgfqpoint{3.351667in}{2.506117in}}%
\pgfpathlineto{\pgfqpoint{3.352533in}{2.537105in}}%
\pgfpathlineto{\pgfqpoint{3.353399in}{2.535168in}}%
\pgfpathlineto{\pgfqpoint{3.354265in}{2.552723in}}%
\pgfpathlineto{\pgfqpoint{3.355130in}{2.596314in}}%
\pgfpathlineto{\pgfqpoint{3.355995in}{2.491300in}}%
\pgfpathlineto{\pgfqpoint{3.356859in}{2.497694in}}%
\pgfpathlineto{\pgfqpoint{3.357724in}{2.505532in}}%
\pgfpathlineto{\pgfqpoint{3.358588in}{2.490561in}}%
\pgfpathlineto{\pgfqpoint{3.359454in}{2.518291in}}%
\pgfpathlineto{\pgfqpoint{3.360319in}{2.463570in}}%
\pgfpathlineto{\pgfqpoint{3.361184in}{2.544053in}}%
\pgfpathlineto{\pgfqpoint{3.362049in}{2.517983in}}%
\pgfpathlineto{\pgfqpoint{3.362913in}{2.521857in}}%
\pgfpathlineto{\pgfqpoint{3.364643in}{2.584817in}}%
\pgfpathlineto{\pgfqpoint{3.365508in}{2.536613in}}%
\pgfpathlineto{\pgfqpoint{3.366374in}{2.547558in}}%
\pgfpathlineto{\pgfqpoint{3.367238in}{2.564497in}}%
\pgfpathlineto{\pgfqpoint{3.368968in}{2.500213in}}%
\pgfpathlineto{\pgfqpoint{3.369832in}{2.524500in}}%
\pgfpathlineto{\pgfqpoint{3.370697in}{2.517799in}}%
\pgfpathlineto{\pgfqpoint{3.372429in}{2.488409in}}%
\pgfpathlineto{\pgfqpoint{3.373294in}{2.581374in}}%
\pgfpathlineto{\pgfqpoint{3.374159in}{2.533785in}}%
\pgfpathlineto{\pgfqpoint{3.375889in}{2.569015in}}%
\pgfpathlineto{\pgfqpoint{3.376753in}{2.480109in}}%
\pgfpathlineto{\pgfqpoint{3.377619in}{2.610456in}}%
\pgfpathlineto{\pgfqpoint{3.380214in}{2.463201in}}%
\pgfpathlineto{\pgfqpoint{3.381944in}{2.536090in}}%
\pgfpathlineto{\pgfqpoint{3.382810in}{2.553767in}}%
\pgfpathlineto{\pgfqpoint{3.383674in}{2.515584in}}%
\pgfpathlineto{\pgfqpoint{3.384537in}{2.521242in}}%
\pgfpathlineto{\pgfqpoint{3.386267in}{2.507100in}}%
\pgfpathlineto{\pgfqpoint{3.387132in}{2.510235in}}%
\pgfpathlineto{\pgfqpoint{3.388863in}{2.536982in}}%
\pgfpathlineto{\pgfqpoint{3.389728in}{2.531571in}}%
\pgfpathlineto{\pgfqpoint{3.390594in}{2.520503in}}%
\pgfpathlineto{\pgfqpoint{3.391460in}{2.491236in}}%
\pgfpathlineto{\pgfqpoint{3.392325in}{2.587891in}}%
\pgfpathlineto{\pgfqpoint{3.393191in}{2.583586in}}%
\pgfpathlineto{\pgfqpoint{3.394056in}{2.568153in}}%
\pgfpathlineto{\pgfqpoint{3.394921in}{2.518781in}}%
\pgfpathlineto{\pgfqpoint{3.395786in}{2.533200in}}%
\pgfpathlineto{\pgfqpoint{3.396649in}{2.476727in}}%
\pgfpathlineto{\pgfqpoint{3.397513in}{2.548356in}}%
\pgfpathlineto{\pgfqpoint{3.398379in}{2.513555in}}%
\pgfpathlineto{\pgfqpoint{3.400975in}{2.558347in}}%
\pgfpathlineto{\pgfqpoint{3.401841in}{2.538273in}}%
\pgfpathlineto{\pgfqpoint{3.402704in}{2.588075in}}%
\pgfpathlineto{\pgfqpoint{3.403569in}{2.537044in}}%
\pgfpathlineto{\pgfqpoint{3.404434in}{2.560592in}}%
\pgfpathlineto{\pgfqpoint{3.405299in}{2.548664in}}%
\pgfpathlineto{\pgfqpoint{3.407893in}{2.448751in}}%
\pgfpathlineto{\pgfqpoint{3.410487in}{2.493543in}}%
\pgfpathlineto{\pgfqpoint{3.411351in}{2.393722in}}%
\pgfpathlineto{\pgfqpoint{3.412214in}{2.559607in}}%
\pgfpathlineto{\pgfqpoint{3.413944in}{2.491482in}}%
\pgfpathlineto{\pgfqpoint{3.414810in}{2.492650in}}%
\pgfpathlineto{\pgfqpoint{3.415676in}{2.547864in}}%
\pgfpathlineto{\pgfqpoint{3.416540in}{2.489945in}}%
\pgfpathlineto{\pgfqpoint{3.419134in}{2.611193in}}%
\pgfpathlineto{\pgfqpoint{3.419999in}{2.501689in}}%
\pgfpathlineto{\pgfqpoint{3.420865in}{2.611254in}}%
\pgfpathlineto{\pgfqpoint{3.422597in}{2.526099in}}%
\pgfpathlineto{\pgfqpoint{3.423462in}{2.608303in}}%
\pgfpathlineto{\pgfqpoint{3.425191in}{2.518843in}}%
\pgfpathlineto{\pgfqpoint{3.426054in}{2.489638in}}%
\pgfpathlineto{\pgfqpoint{3.427782in}{2.531263in}}%
\pgfpathlineto{\pgfqpoint{3.428646in}{2.462523in}}%
\pgfpathlineto{\pgfqpoint{3.430376in}{2.595146in}}%
\pgfpathlineto{\pgfqpoint{3.431241in}{2.519397in}}%
\pgfpathlineto{\pgfqpoint{3.432107in}{2.584078in}}%
\pgfpathlineto{\pgfqpoint{3.432971in}{2.570613in}}%
\pgfpathlineto{\pgfqpoint{3.433836in}{2.603139in}}%
\pgfpathlineto{\pgfqpoint{3.438159in}{2.500090in}}%
\pgfpathlineto{\pgfqpoint{3.439025in}{2.553182in}}%
\pgfpathlineto{\pgfqpoint{3.439889in}{2.535874in}}%
\pgfpathlineto{\pgfqpoint{3.440753in}{2.537903in}}%
\pgfpathlineto{\pgfqpoint{3.441619in}{2.585800in}}%
\pgfpathlineto{\pgfqpoint{3.443347in}{2.491544in}}%
\pgfpathlineto{\pgfqpoint{3.444212in}{2.574180in}}%
\pgfpathlineto{\pgfqpoint{3.445941in}{2.506853in}}%
\pgfpathlineto{\pgfqpoint{3.448533in}{2.572273in}}%
\pgfpathlineto{\pgfqpoint{3.449396in}{2.513924in}}%
\pgfpathlineto{\pgfqpoint{3.451124in}{2.558070in}}%
\pgfpathlineto{\pgfqpoint{3.451988in}{2.494987in}}%
\pgfpathlineto{\pgfqpoint{3.452853in}{2.561421in}}%
\pgfpathlineto{\pgfqpoint{3.454582in}{2.496401in}}%
\pgfpathlineto{\pgfqpoint{3.455444in}{2.496586in}}%
\pgfpathlineto{\pgfqpoint{3.457175in}{2.523515in}}%
\pgfpathlineto{\pgfqpoint{3.458040in}{2.516660in}}%
\pgfpathlineto{\pgfqpoint{3.458905in}{2.534399in}}%
\pgfpathlineto{\pgfqpoint{3.459770in}{2.590472in}}%
\pgfpathlineto{\pgfqpoint{3.460636in}{2.516937in}}%
\pgfpathlineto{\pgfqpoint{3.461498in}{2.544605in}}%
\pgfpathlineto{\pgfqpoint{3.462364in}{2.511987in}}%
\pgfpathlineto{\pgfqpoint{3.463229in}{2.585307in}}%
\pgfpathlineto{\pgfqpoint{3.466686in}{2.457235in}}%
\pgfpathlineto{\pgfqpoint{3.467552in}{2.475003in}}%
\pgfpathlineto{\pgfqpoint{3.469282in}{2.549770in}}%
\pgfpathlineto{\pgfqpoint{3.470145in}{2.542637in}}%
\pgfpathlineto{\pgfqpoint{3.471011in}{2.522286in}}%
\pgfpathlineto{\pgfqpoint{3.471877in}{2.554904in}}%
\pgfpathlineto{\pgfqpoint{3.472743in}{2.525421in}}%
\pgfpathlineto{\pgfqpoint{3.473609in}{2.537349in}}%
\pgfpathlineto{\pgfqpoint{3.474471in}{2.496432in}}%
\pgfpathlineto{\pgfqpoint{3.475336in}{2.549524in}}%
\pgfpathlineto{\pgfqpoint{3.476201in}{2.528865in}}%
\pgfpathlineto{\pgfqpoint{3.477933in}{2.592071in}}%
\pgfpathlineto{\pgfqpoint{3.478798in}{2.528680in}}%
\pgfpathlineto{\pgfqpoint{3.479662in}{2.559545in}}%
\pgfpathlineto{\pgfqpoint{3.480525in}{2.556471in}}%
\pgfpathlineto{\pgfqpoint{3.481389in}{2.478662in}}%
\pgfpathlineto{\pgfqpoint{3.482252in}{2.588196in}}%
\pgfpathlineto{\pgfqpoint{3.483118in}{2.487362in}}%
\pgfpathlineto{\pgfqpoint{3.484847in}{2.631975in}}%
\pgfpathlineto{\pgfqpoint{3.485710in}{2.567047in}}%
\pgfpathlineto{\pgfqpoint{3.486575in}{2.567140in}}%
\pgfpathlineto{\pgfqpoint{3.487439in}{2.570860in}}%
\pgfpathlineto{\pgfqpoint{3.489170in}{2.524931in}}%
\pgfpathlineto{\pgfqpoint{3.490901in}{2.602954in}}%
\pgfpathlineto{\pgfqpoint{3.491766in}{2.591396in}}%
\pgfpathlineto{\pgfqpoint{3.496091in}{2.416349in}}%
\pgfpathlineto{\pgfqpoint{3.497819in}{2.532248in}}%
\pgfpathlineto{\pgfqpoint{3.498684in}{2.488716in}}%
\pgfpathlineto{\pgfqpoint{3.499548in}{2.621155in}}%
\pgfpathlineto{\pgfqpoint{3.502142in}{2.465045in}}%
\pgfpathlineto{\pgfqpoint{3.503870in}{2.573749in}}%
\pgfpathlineto{\pgfqpoint{3.504734in}{2.476358in}}%
\pgfpathlineto{\pgfqpoint{3.505598in}{2.549739in}}%
\pgfpathlineto{\pgfqpoint{3.506463in}{2.482199in}}%
\pgfpathlineto{\pgfqpoint{3.508192in}{2.526283in}}%
\pgfpathlineto{\pgfqpoint{3.509057in}{2.524993in}}%
\pgfpathlineto{\pgfqpoint{3.509923in}{2.514448in}}%
\pgfpathlineto{\pgfqpoint{3.510788in}{2.522840in}}%
\pgfpathlineto{\pgfqpoint{3.511653in}{2.505317in}}%
\pgfpathlineto{\pgfqpoint{3.512519in}{2.530342in}}%
\pgfpathlineto{\pgfqpoint{3.513384in}{2.530034in}}%
\pgfpathlineto{\pgfqpoint{3.514249in}{2.536859in}}%
\pgfpathlineto{\pgfqpoint{3.515113in}{2.458159in}}%
\pgfpathlineto{\pgfqpoint{3.517709in}{2.529973in}}%
\pgfpathlineto{\pgfqpoint{3.518574in}{2.504026in}}%
\pgfpathlineto{\pgfqpoint{3.519437in}{2.522071in}}%
\pgfpathlineto{\pgfqpoint{3.520301in}{2.477710in}}%
\pgfpathlineto{\pgfqpoint{3.521166in}{2.478202in}}%
\pgfpathlineto{\pgfqpoint{3.522032in}{2.479493in}}%
\pgfpathlineto{\pgfqpoint{3.522897in}{2.568399in}}%
\pgfpathlineto{\pgfqpoint{3.523762in}{2.463691in}}%
\pgfpathlineto{\pgfqpoint{3.525491in}{2.503595in}}%
\pgfpathlineto{\pgfqpoint{3.526355in}{2.515584in}}%
\pgfpathlineto{\pgfqpoint{3.527220in}{2.503287in}}%
\pgfpathlineto{\pgfqpoint{3.528086in}{2.580143in}}%
\pgfpathlineto{\pgfqpoint{3.528952in}{2.435040in}}%
\pgfpathlineto{\pgfqpoint{3.529817in}{2.505255in}}%
\pgfpathlineto{\pgfqpoint{3.530682in}{2.462831in}}%
\pgfpathlineto{\pgfqpoint{3.532412in}{2.531448in}}%
\pgfpathlineto{\pgfqpoint{3.533278in}{2.521550in}}%
\pgfpathlineto{\pgfqpoint{3.534143in}{2.545528in}}%
\pgfpathlineto{\pgfqpoint{3.535870in}{2.494127in}}%
\pgfpathlineto{\pgfqpoint{3.536735in}{2.502920in}}%
\pgfpathlineto{\pgfqpoint{3.537601in}{2.440882in}}%
\pgfpathlineto{\pgfqpoint{3.539331in}{2.535076in}}%
\pgfpathlineto{\pgfqpoint{3.541923in}{2.494805in}}%
\pgfpathlineto{\pgfqpoint{3.542787in}{2.554506in}}%
\pgfpathlineto{\pgfqpoint{3.543652in}{2.506640in}}%
\pgfpathlineto{\pgfqpoint{3.544515in}{2.527022in}}%
\pgfpathlineto{\pgfqpoint{3.545379in}{2.601663in}}%
\pgfpathlineto{\pgfqpoint{3.546245in}{2.499107in}}%
\pgfpathlineto{\pgfqpoint{3.547975in}{2.591611in}}%
\pgfpathlineto{\pgfqpoint{3.548839in}{2.587093in}}%
\pgfpathlineto{\pgfqpoint{3.551434in}{2.483798in}}%
\pgfpathlineto{\pgfqpoint{3.552299in}{2.552476in}}%
\pgfpathlineto{\pgfqpoint{3.554030in}{2.487364in}}%
\pgfpathlineto{\pgfqpoint{3.554896in}{2.571598in}}%
\pgfpathlineto{\pgfqpoint{3.555761in}{2.469227in}}%
\pgfpathlineto{\pgfqpoint{3.557492in}{2.529911in}}%
\pgfpathlineto{\pgfqpoint{3.559222in}{2.464922in}}%
\pgfpathlineto{\pgfqpoint{3.560087in}{2.506086in}}%
\pgfpathlineto{\pgfqpoint{3.560951in}{2.499169in}}%
\pgfpathlineto{\pgfqpoint{3.562682in}{2.545251in}}%
\pgfpathlineto{\pgfqpoint{3.565276in}{2.434549in}}%
\pgfpathlineto{\pgfqpoint{3.566138in}{2.435778in}}%
\pgfpathlineto{\pgfqpoint{3.567868in}{2.528559in}}%
\pgfpathlineto{\pgfqpoint{3.568733in}{2.504888in}}%
\pgfpathlineto{\pgfqpoint{3.570464in}{2.454408in}}%
\pgfpathlineto{\pgfqpoint{3.571329in}{2.516569in}}%
\pgfpathlineto{\pgfqpoint{3.572193in}{2.435748in}}%
\pgfpathlineto{\pgfqpoint{3.573923in}{2.538150in}}%
\pgfpathlineto{\pgfqpoint{3.574787in}{2.465722in}}%
\pgfpathlineto{\pgfqpoint{3.576516in}{2.542668in}}%
\pgfpathlineto{\pgfqpoint{3.577381in}{2.433749in}}%
\pgfpathlineto{\pgfqpoint{3.578247in}{2.513617in}}%
\pgfpathlineto{\pgfqpoint{3.579112in}{2.500183in}}%
\pgfpathlineto{\pgfqpoint{3.579978in}{2.531510in}}%
\pgfpathlineto{\pgfqpoint{3.580842in}{2.452625in}}%
\pgfpathlineto{\pgfqpoint{3.582573in}{2.496647in}}%
\pgfpathlineto{\pgfqpoint{3.584300in}{2.466090in}}%
\pgfpathlineto{\pgfqpoint{3.585164in}{2.513309in}}%
\pgfpathlineto{\pgfqpoint{3.586028in}{2.452961in}}%
\pgfpathlineto{\pgfqpoint{3.587756in}{2.530155in}}%
\pgfpathlineto{\pgfqpoint{3.588620in}{2.472053in}}%
\pgfpathlineto{\pgfqpoint{3.589485in}{2.546879in}}%
\pgfpathlineto{\pgfqpoint{3.590351in}{2.517981in}}%
\pgfpathlineto{\pgfqpoint{3.592078in}{2.569690in}}%
\pgfpathlineto{\pgfqpoint{3.592942in}{2.559361in}}%
\pgfpathlineto{\pgfqpoint{3.594668in}{2.502548in}}%
\pgfpathlineto{\pgfqpoint{3.595532in}{2.550599in}}%
\pgfpathlineto{\pgfqpoint{3.597262in}{2.457603in}}%
\pgfpathlineto{\pgfqpoint{3.598128in}{2.466672in}}%
\pgfpathlineto{\pgfqpoint{3.598992in}{2.426186in}}%
\pgfpathlineto{\pgfqpoint{3.599857in}{2.507959in}}%
\pgfpathlineto{\pgfqpoint{3.600722in}{2.486041in}}%
\pgfpathlineto{\pgfqpoint{3.601586in}{2.456558in}}%
\pgfpathlineto{\pgfqpoint{3.603316in}{2.559668in}}%
\pgfpathlineto{\pgfqpoint{3.604181in}{2.558501in}}%
\pgfpathlineto{\pgfqpoint{3.605045in}{2.460833in}}%
\pgfpathlineto{\pgfqpoint{3.605910in}{2.488162in}}%
\pgfpathlineto{\pgfqpoint{3.606775in}{2.490192in}}%
\pgfpathlineto{\pgfqpoint{3.607639in}{2.473438in}}%
\pgfpathlineto{\pgfqpoint{3.608504in}{2.509129in}}%
\pgfpathlineto{\pgfqpoint{3.609370in}{2.500398in}}%
\pgfpathlineto{\pgfqpoint{3.610234in}{2.503564in}}%
\pgfpathlineto{\pgfqpoint{3.611099in}{2.519151in}}%
\pgfpathlineto{\pgfqpoint{3.612827in}{2.505809in}}%
\pgfpathlineto{\pgfqpoint{3.614557in}{2.575624in}}%
\pgfpathlineto{\pgfqpoint{3.617155in}{2.486502in}}%
\pgfpathlineto{\pgfqpoint{3.618881in}{2.567478in}}%
\pgfpathlineto{\pgfqpoint{3.620609in}{2.502304in}}%
\pgfpathlineto{\pgfqpoint{3.621474in}{2.517368in}}%
\pgfpathlineto{\pgfqpoint{3.622340in}{2.482874in}}%
\pgfpathlineto{\pgfqpoint{3.623205in}{2.513678in}}%
\pgfpathlineto{\pgfqpoint{3.624932in}{2.472853in}}%
\pgfpathlineto{\pgfqpoint{3.626663in}{2.484319in}}%
\pgfpathlineto{\pgfqpoint{3.628394in}{2.513124in}}%
\pgfpathlineto{\pgfqpoint{3.630123in}{2.593854in}}%
\pgfpathlineto{\pgfqpoint{3.634446in}{2.500029in}}%
\pgfpathlineto{\pgfqpoint{3.636176in}{2.559976in}}%
\pgfpathlineto{\pgfqpoint{3.637043in}{2.546265in}}%
\pgfpathlineto{\pgfqpoint{3.637907in}{2.560161in}}%
\pgfpathlineto{\pgfqpoint{3.639640in}{2.472422in}}%
\pgfpathlineto{\pgfqpoint{3.641372in}{2.527081in}}%
\pgfpathlineto{\pgfqpoint{3.642235in}{2.475927in}}%
\pgfpathlineto{\pgfqpoint{3.645693in}{2.590657in}}%
\pgfpathlineto{\pgfqpoint{3.647418in}{2.530525in}}%
\pgfpathlineto{\pgfqpoint{3.649149in}{2.560530in}}%
\pgfpathlineto{\pgfqpoint{3.650016in}{2.552415in}}%
\pgfpathlineto{\pgfqpoint{3.650882in}{2.573995in}}%
\pgfpathlineto{\pgfqpoint{3.651747in}{2.538273in}}%
\pgfpathlineto{\pgfqpoint{3.652611in}{2.541501in}}%
\pgfpathlineto{\pgfqpoint{3.653476in}{2.570921in}}%
\pgfpathlineto{\pgfqpoint{3.654343in}{2.519459in}}%
\pgfpathlineto{\pgfqpoint{3.655208in}{2.579344in}}%
\pgfpathlineto{\pgfqpoint{3.656940in}{2.458282in}}%
\pgfpathlineto{\pgfqpoint{3.659533in}{2.585923in}}%
\pgfpathlineto{\pgfqpoint{3.660399in}{2.498615in}}%
\pgfpathlineto{\pgfqpoint{3.661264in}{2.528313in}}%
\pgfpathlineto{\pgfqpoint{3.662127in}{2.497355in}}%
\pgfpathlineto{\pgfqpoint{3.662989in}{2.569077in}}%
\pgfpathlineto{\pgfqpoint{3.663855in}{2.525300in}}%
\pgfpathlineto{\pgfqpoint{3.664721in}{2.531787in}}%
\pgfpathlineto{\pgfqpoint{3.665587in}{2.515831in}}%
\pgfpathlineto{\pgfqpoint{3.666450in}{2.534953in}}%
\pgfpathlineto{\pgfqpoint{3.667313in}{2.502551in}}%
\pgfpathlineto{\pgfqpoint{3.669910in}{2.588137in}}%
\pgfpathlineto{\pgfqpoint{3.671641in}{2.511251in}}%
\pgfpathlineto{\pgfqpoint{3.672504in}{2.519151in}}%
\pgfpathlineto{\pgfqpoint{3.673368in}{2.515708in}}%
\pgfpathlineto{\pgfqpoint{3.674235in}{2.500891in}}%
\pgfpathlineto{\pgfqpoint{3.675101in}{2.514232in}}%
\pgfpathlineto{\pgfqpoint{3.675965in}{2.485581in}}%
\pgfpathlineto{\pgfqpoint{3.677697in}{2.573441in}}%
\pgfpathlineto{\pgfqpoint{3.678563in}{2.543253in}}%
\pgfpathlineto{\pgfqpoint{3.679428in}{2.484842in}}%
\pgfpathlineto{\pgfqpoint{3.680294in}{2.560715in}}%
\pgfpathlineto{\pgfqpoint{3.681160in}{2.454747in}}%
\pgfpathlineto{\pgfqpoint{3.682026in}{2.483798in}}%
\pgfpathlineto{\pgfqpoint{3.682891in}{2.562129in}}%
\pgfpathlineto{\pgfqpoint{3.684621in}{2.471501in}}%
\pgfpathlineto{\pgfqpoint{3.685485in}{2.517799in}}%
\pgfpathlineto{\pgfqpoint{3.686350in}{2.512018in}}%
\pgfpathlineto{\pgfqpoint{3.687215in}{2.522840in}}%
\pgfpathlineto{\pgfqpoint{3.688079in}{2.507408in}}%
\pgfpathlineto{\pgfqpoint{3.688943in}{2.535936in}}%
\pgfpathlineto{\pgfqpoint{3.689808in}{2.467503in}}%
\pgfpathlineto{\pgfqpoint{3.691536in}{2.520749in}}%
\pgfpathlineto{\pgfqpoint{3.692401in}{2.477217in}}%
\pgfpathlineto{\pgfqpoint{3.693266in}{2.539779in}}%
\pgfpathlineto{\pgfqpoint{3.694131in}{2.457728in}}%
\pgfpathlineto{\pgfqpoint{3.694996in}{2.531263in}}%
\pgfpathlineto{\pgfqpoint{3.695860in}{2.524192in}}%
\pgfpathlineto{\pgfqpoint{3.696724in}{2.503533in}}%
\pgfpathlineto{\pgfqpoint{3.698452in}{2.584571in}}%
\pgfpathlineto{\pgfqpoint{3.700181in}{2.481922in}}%
\pgfpathlineto{\pgfqpoint{3.701046in}{2.502304in}}%
\pgfpathlineto{\pgfqpoint{3.701911in}{2.584263in}}%
\pgfpathlineto{\pgfqpoint{3.702776in}{2.548079in}}%
\pgfpathlineto{\pgfqpoint{3.703642in}{2.550078in}}%
\pgfpathlineto{\pgfqpoint{3.705373in}{2.524439in}}%
\pgfpathlineto{\pgfqpoint{3.706238in}{2.550139in}}%
\pgfpathlineto{\pgfqpoint{3.707103in}{2.543745in}}%
\pgfpathlineto{\pgfqpoint{3.709698in}{2.481522in}}%
\pgfpathlineto{\pgfqpoint{3.710563in}{2.479985in}}%
\pgfpathlineto{\pgfqpoint{3.711428in}{2.459142in}}%
\pgfpathlineto{\pgfqpoint{3.712293in}{2.566832in}}%
\pgfpathlineto{\pgfqpoint{3.713157in}{2.522840in}}%
\pgfpathlineto{\pgfqpoint{3.714022in}{2.539194in}}%
\pgfpathlineto{\pgfqpoint{3.715748in}{2.449182in}}%
\pgfpathlineto{\pgfqpoint{3.716611in}{2.502612in}}%
\pgfpathlineto{\pgfqpoint{3.717477in}{2.465045in}}%
\pgfpathlineto{\pgfqpoint{3.719209in}{2.556350in}}%
\pgfpathlineto{\pgfqpoint{3.720074in}{2.551186in}}%
\pgfpathlineto{\pgfqpoint{3.720938in}{2.526837in}}%
\pgfpathlineto{\pgfqpoint{3.722666in}{2.586785in}}%
\pgfpathlineto{\pgfqpoint{3.724396in}{2.529172in}}%
\pgfpathlineto{\pgfqpoint{3.726126in}{2.585433in}}%
\pgfpathlineto{\pgfqpoint{3.726990in}{2.456437in}}%
\pgfpathlineto{\pgfqpoint{3.727855in}{2.555981in}}%
\pgfpathlineto{\pgfqpoint{3.728721in}{2.555766in}}%
\pgfpathlineto{\pgfqpoint{3.729587in}{2.528374in}}%
\pgfpathlineto{\pgfqpoint{3.730452in}{2.563114in}}%
\pgfpathlineto{\pgfqpoint{3.732179in}{2.512818in}}%
\pgfpathlineto{\pgfqpoint{3.733044in}{2.522717in}}%
\pgfpathlineto{\pgfqpoint{3.733910in}{2.490592in}}%
\pgfpathlineto{\pgfqpoint{3.735640in}{2.582972in}}%
\pgfpathlineto{\pgfqpoint{3.736506in}{2.499415in}}%
\pgfpathlineto{\pgfqpoint{3.737372in}{2.516631in}}%
\pgfpathlineto{\pgfqpoint{3.738238in}{2.485366in}}%
\pgfpathlineto{\pgfqpoint{3.739103in}{2.505747in}}%
\pgfpathlineto{\pgfqpoint{3.739967in}{2.484781in}}%
\pgfpathlineto{\pgfqpoint{3.741695in}{2.527574in}}%
\pgfpathlineto{\pgfqpoint{3.742560in}{2.459265in}}%
\pgfpathlineto{\pgfqpoint{3.743425in}{2.507592in}}%
\pgfpathlineto{\pgfqpoint{3.744291in}{2.467873in}}%
\pgfpathlineto{\pgfqpoint{3.745156in}{2.556043in}}%
\pgfpathlineto{\pgfqpoint{3.746022in}{2.540856in}}%
\pgfpathlineto{\pgfqpoint{3.747753in}{2.487210in}}%
\pgfpathlineto{\pgfqpoint{3.749484in}{2.549770in}}%
\pgfpathlineto{\pgfqpoint{3.750349in}{2.526468in}}%
\pgfpathlineto{\pgfqpoint{3.751213in}{2.517245in}}%
\pgfpathlineto{\pgfqpoint{3.752078in}{2.527266in}}%
\pgfpathlineto{\pgfqpoint{3.753810in}{2.481276in}}%
\pgfpathlineto{\pgfqpoint{3.754676in}{2.563419in}}%
\pgfpathlineto{\pgfqpoint{3.755541in}{2.538027in}}%
\pgfpathlineto{\pgfqpoint{3.756406in}{2.464122in}}%
\pgfpathlineto{\pgfqpoint{3.758138in}{2.531325in}}%
\pgfpathlineto{\pgfqpoint{3.759004in}{2.531202in}}%
\pgfpathlineto{\pgfqpoint{3.759869in}{2.484935in}}%
\pgfpathlineto{\pgfqpoint{3.761597in}{2.560715in}}%
\pgfpathlineto{\pgfqpoint{3.762463in}{2.474574in}}%
\pgfpathlineto{\pgfqpoint{3.763326in}{2.478818in}}%
\pgfpathlineto{\pgfqpoint{3.765921in}{2.555119in}}%
\pgfpathlineto{\pgfqpoint{3.767650in}{2.489699in}}%
\pgfpathlineto{\pgfqpoint{3.768513in}{2.561267in}}%
\pgfpathlineto{\pgfqpoint{3.769378in}{2.482721in}}%
\pgfpathlineto{\pgfqpoint{3.771106in}{2.575039in}}%
\pgfpathlineto{\pgfqpoint{3.772835in}{2.542085in}}%
\pgfpathlineto{\pgfqpoint{3.774564in}{2.572397in}}%
\pgfpathlineto{\pgfqpoint{3.775427in}{2.521488in}}%
\pgfpathlineto{\pgfqpoint{3.776290in}{2.528128in}}%
\pgfpathlineto{\pgfqpoint{3.777155in}{2.623736in}}%
\pgfpathlineto{\pgfqpoint{3.778884in}{2.479924in}}%
\pgfpathlineto{\pgfqpoint{3.779750in}{2.562375in}}%
\pgfpathlineto{\pgfqpoint{3.780615in}{2.543253in}}%
\pgfpathlineto{\pgfqpoint{3.781479in}{2.454223in}}%
\pgfpathlineto{\pgfqpoint{3.783208in}{2.560961in}}%
\pgfpathlineto{\pgfqpoint{3.784074in}{2.488593in}}%
\pgfpathlineto{\pgfqpoint{3.784940in}{2.541223in}}%
\pgfpathlineto{\pgfqpoint{3.787530in}{2.500337in}}%
\pgfpathlineto{\pgfqpoint{3.789261in}{2.556841in}}%
\pgfpathlineto{\pgfqpoint{3.790125in}{2.567109in}}%
\pgfpathlineto{\pgfqpoint{3.791853in}{2.457420in}}%
\pgfpathlineto{\pgfqpoint{3.793584in}{2.527389in}}%
\pgfpathlineto{\pgfqpoint{3.795314in}{2.600434in}}%
\pgfpathlineto{\pgfqpoint{3.797911in}{2.458988in}}%
\pgfpathlineto{\pgfqpoint{3.798777in}{2.560530in}}%
\pgfpathlineto{\pgfqpoint{3.799640in}{2.529665in}}%
\pgfpathlineto{\pgfqpoint{3.800503in}{2.541193in}}%
\pgfpathlineto{\pgfqpoint{3.802234in}{2.507531in}}%
\pgfpathlineto{\pgfqpoint{3.803099in}{2.505809in}}%
\pgfpathlineto{\pgfqpoint{3.803964in}{2.542208in}}%
\pgfpathlineto{\pgfqpoint{3.804830in}{2.541624in}}%
\pgfpathlineto{\pgfqpoint{3.807428in}{2.482721in}}%
\pgfpathlineto{\pgfqpoint{3.809157in}{2.548479in}}%
\pgfpathlineto{\pgfqpoint{3.810020in}{2.481892in}}%
\pgfpathlineto{\pgfqpoint{3.810885in}{2.576578in}}%
\pgfpathlineto{\pgfqpoint{3.811751in}{2.495726in}}%
\pgfpathlineto{\pgfqpoint{3.812616in}{2.495849in}}%
\pgfpathlineto{\pgfqpoint{3.813481in}{2.501014in}}%
\pgfpathlineto{\pgfqpoint{3.814344in}{2.452902in}}%
\pgfpathlineto{\pgfqpoint{3.816076in}{2.521180in}}%
\pgfpathlineto{\pgfqpoint{3.817803in}{2.456684in}}%
\pgfpathlineto{\pgfqpoint{3.818667in}{2.514171in}}%
\pgfpathlineto{\pgfqpoint{3.819532in}{2.433687in}}%
\pgfpathlineto{\pgfqpoint{3.821264in}{2.605353in}}%
\pgfpathlineto{\pgfqpoint{3.822992in}{2.548171in}}%
\pgfpathlineto{\pgfqpoint{3.823858in}{2.575132in}}%
\pgfpathlineto{\pgfqpoint{3.824723in}{2.567109in}}%
\pgfpathlineto{\pgfqpoint{3.825589in}{2.580512in}}%
\pgfpathlineto{\pgfqpoint{3.826455in}{2.568615in}}%
\pgfpathlineto{\pgfqpoint{3.827320in}{2.594592in}}%
\pgfpathlineto{\pgfqpoint{3.828185in}{2.556348in}}%
\pgfpathlineto{\pgfqpoint{3.829052in}{2.567999in}}%
\pgfpathlineto{\pgfqpoint{3.830781in}{2.530771in}}%
\pgfpathlineto{\pgfqpoint{3.831644in}{2.582357in}}%
\pgfpathlineto{\pgfqpoint{3.832509in}{2.536120in}}%
\pgfpathlineto{\pgfqpoint{3.835105in}{2.650175in}}%
\pgfpathlineto{\pgfqpoint{3.835971in}{2.562190in}}%
\pgfpathlineto{\pgfqpoint{3.836836in}{2.630376in}}%
\pgfpathlineto{\pgfqpoint{3.837698in}{2.533170in}}%
\pgfpathlineto{\pgfqpoint{3.838562in}{2.567355in}}%
\pgfpathlineto{\pgfqpoint{3.841155in}{2.502120in}}%
\pgfpathlineto{\pgfqpoint{3.842018in}{2.586108in}}%
\pgfpathlineto{\pgfqpoint{3.842884in}{2.528097in}}%
\pgfpathlineto{\pgfqpoint{3.843749in}{2.548171in}}%
\pgfpathlineto{\pgfqpoint{3.844612in}{2.534583in}}%
\pgfpathlineto{\pgfqpoint{3.845478in}{2.541193in}}%
\pgfpathlineto{\pgfqpoint{3.846342in}{2.493142in}}%
\pgfpathlineto{\pgfqpoint{3.847206in}{2.560161in}}%
\pgfpathlineto{\pgfqpoint{3.848933in}{2.517491in}}%
\pgfpathlineto{\pgfqpoint{3.849796in}{2.551799in}}%
\pgfpathlineto{\pgfqpoint{3.850662in}{2.479862in}}%
\pgfpathlineto{\pgfqpoint{3.851528in}{2.584325in}}%
\pgfpathlineto{\pgfqpoint{3.852393in}{2.508729in}}%
\pgfpathlineto{\pgfqpoint{3.853258in}{2.633327in}}%
\pgfpathlineto{\pgfqpoint{3.854987in}{2.544697in}}%
\pgfpathlineto{\pgfqpoint{3.855851in}{2.525421in}}%
\pgfpathlineto{\pgfqpoint{3.856715in}{2.566801in}}%
\pgfpathlineto{\pgfqpoint{3.857580in}{2.489607in}}%
\pgfpathlineto{\pgfqpoint{3.858443in}{2.558439in}}%
\pgfpathlineto{\pgfqpoint{3.859307in}{2.485948in}}%
\pgfpathlineto{\pgfqpoint{3.860171in}{2.524069in}}%
\pgfpathlineto{\pgfqpoint{3.861036in}{2.512141in}}%
\pgfpathlineto{\pgfqpoint{3.861901in}{2.556471in}}%
\pgfpathlineto{\pgfqpoint{3.862767in}{2.548048in}}%
\pgfpathlineto{\pgfqpoint{3.863631in}{2.568215in}}%
\pgfpathlineto{\pgfqpoint{3.864495in}{2.549831in}}%
\pgfpathlineto{\pgfqpoint{3.865358in}{2.586169in}}%
\pgfpathlineto{\pgfqpoint{3.867087in}{2.483367in}}%
\pgfpathlineto{\pgfqpoint{3.867951in}{2.524192in}}%
\pgfpathlineto{\pgfqpoint{3.868815in}{2.517675in}}%
\pgfpathlineto{\pgfqpoint{3.869677in}{2.536367in}}%
\pgfpathlineto{\pgfqpoint{3.870542in}{2.506915in}}%
\pgfpathlineto{\pgfqpoint{3.871407in}{2.577377in}}%
\pgfpathlineto{\pgfqpoint{3.874002in}{2.505409in}}%
\pgfpathlineto{\pgfqpoint{3.874867in}{2.548972in}}%
\pgfpathlineto{\pgfqpoint{3.875730in}{2.506792in}}%
\pgfpathlineto{\pgfqpoint{3.879189in}{2.609779in}}%
\pgfpathlineto{\pgfqpoint{3.880052in}{2.520565in}}%
\pgfpathlineto{\pgfqpoint{3.880914in}{2.544851in}}%
\pgfpathlineto{\pgfqpoint{3.881778in}{2.500583in}}%
\pgfpathlineto{\pgfqpoint{3.882643in}{2.566924in}}%
\pgfpathlineto{\pgfqpoint{3.883507in}{2.508513in}}%
\pgfpathlineto{\pgfqpoint{3.884372in}{2.534645in}}%
\pgfpathlineto{\pgfqpoint{3.885238in}{2.481461in}}%
\pgfpathlineto{\pgfqpoint{3.886103in}{2.536459in}}%
\pgfpathlineto{\pgfqpoint{3.886970in}{2.495172in}}%
\pgfpathlineto{\pgfqpoint{3.887836in}{2.582049in}}%
\pgfpathlineto{\pgfqpoint{3.888702in}{2.479000in}}%
\pgfpathlineto{\pgfqpoint{3.889568in}{2.521917in}}%
\pgfpathlineto{\pgfqpoint{3.890434in}{2.508883in}}%
\pgfpathlineto{\pgfqpoint{3.892166in}{2.547312in}}%
\pgfpathlineto{\pgfqpoint{3.893031in}{2.531417in}}%
\pgfpathlineto{\pgfqpoint{3.893896in}{2.579344in}}%
\pgfpathlineto{\pgfqpoint{3.895627in}{2.527728in}}%
\pgfpathlineto{\pgfqpoint{3.896493in}{2.591272in}}%
\pgfpathlineto{\pgfqpoint{3.897360in}{2.536982in}}%
\pgfpathlineto{\pgfqpoint{3.898225in}{2.540118in}}%
\pgfpathlineto{\pgfqpoint{3.899091in}{2.515954in}}%
\pgfpathlineto{\pgfqpoint{3.900822in}{2.578914in}}%
\pgfpathlineto{\pgfqpoint{3.901689in}{2.484965in}}%
\pgfpathlineto{\pgfqpoint{3.903419in}{2.544913in}}%
\pgfpathlineto{\pgfqpoint{3.904283in}{2.532800in}}%
\pgfpathlineto{\pgfqpoint{3.905147in}{2.548356in}}%
\pgfpathlineto{\pgfqpoint{3.906011in}{2.478387in}}%
\pgfpathlineto{\pgfqpoint{3.906875in}{2.563727in}}%
\pgfpathlineto{\pgfqpoint{3.907739in}{2.484842in}}%
\pgfpathlineto{\pgfqpoint{3.908603in}{2.533722in}}%
\pgfpathlineto{\pgfqpoint{3.909466in}{2.506176in}}%
\pgfpathlineto{\pgfqpoint{3.910332in}{2.527020in}}%
\pgfpathlineto{\pgfqpoint{3.912926in}{2.487670in}}%
\pgfpathlineto{\pgfqpoint{3.913792in}{2.501381in}}%
\pgfpathlineto{\pgfqpoint{3.915522in}{2.566678in}}%
\pgfpathlineto{\pgfqpoint{3.916387in}{2.490990in}}%
\pgfpathlineto{\pgfqpoint{3.917253in}{2.510543in}}%
\pgfpathlineto{\pgfqpoint{3.918118in}{2.582357in}}%
\pgfpathlineto{\pgfqpoint{3.918984in}{2.575255in}}%
\pgfpathlineto{\pgfqpoint{3.919849in}{2.495233in}}%
\pgfpathlineto{\pgfqpoint{3.920712in}{2.568830in}}%
\pgfpathlineto{\pgfqpoint{3.921578in}{2.493481in}}%
\pgfpathlineto{\pgfqpoint{3.922443in}{2.530956in}}%
\pgfpathlineto{\pgfqpoint{3.923309in}{2.527820in}}%
\pgfpathlineto{\pgfqpoint{3.924173in}{2.474451in}}%
\pgfpathlineto{\pgfqpoint{3.925036in}{2.510543in}}%
\pgfpathlineto{\pgfqpoint{3.925900in}{2.482813in}}%
\pgfpathlineto{\pgfqpoint{3.927629in}{2.547494in}}%
\pgfpathlineto{\pgfqpoint{3.928493in}{2.501258in}}%
\pgfpathlineto{\pgfqpoint{3.930224in}{2.559853in}}%
\pgfpathlineto{\pgfqpoint{3.931088in}{2.517245in}}%
\pgfpathlineto{\pgfqpoint{3.931951in}{2.527512in}}%
\pgfpathlineto{\pgfqpoint{3.932817in}{2.506730in}}%
\pgfpathlineto{\pgfqpoint{3.933682in}{2.540762in}}%
\pgfpathlineto{\pgfqpoint{3.934547in}{2.520195in}}%
\pgfpathlineto{\pgfqpoint{3.937139in}{2.571596in}}%
\pgfpathlineto{\pgfqpoint{3.938004in}{2.506884in}}%
\pgfpathlineto{\pgfqpoint{3.939731in}{2.564648in}}%
\pgfpathlineto{\pgfqpoint{3.940595in}{2.511341in}}%
\pgfpathlineto{\pgfqpoint{3.941461in}{2.516814in}}%
\pgfpathlineto{\pgfqpoint{3.942327in}{2.583401in}}%
\pgfpathlineto{\pgfqpoint{3.943191in}{2.536857in}}%
\pgfpathlineto{\pgfqpoint{3.944057in}{2.543620in}}%
\pgfpathlineto{\pgfqpoint{3.945788in}{2.590164in}}%
\pgfpathlineto{\pgfqpoint{3.946653in}{2.496216in}}%
\pgfpathlineto{\pgfqpoint{3.947519in}{2.499413in}}%
\pgfpathlineto{\pgfqpoint{3.948385in}{2.460186in}}%
\pgfpathlineto{\pgfqpoint{3.949249in}{2.516260in}}%
\pgfpathlineto{\pgfqpoint{3.950114in}{2.507775in}}%
\pgfpathlineto{\pgfqpoint{3.950979in}{2.538394in}}%
\pgfpathlineto{\pgfqpoint{3.951843in}{2.530586in}}%
\pgfpathlineto{\pgfqpoint{3.952708in}{2.487362in}}%
\pgfpathlineto{\pgfqpoint{3.953573in}{2.585061in}}%
\pgfpathlineto{\pgfqpoint{3.954436in}{2.578298in}}%
\pgfpathlineto{\pgfqpoint{3.955301in}{2.615926in}}%
\pgfpathlineto{\pgfqpoint{3.957031in}{2.535197in}}%
\pgfpathlineto{\pgfqpoint{3.957894in}{2.555486in}}%
\pgfpathlineto{\pgfqpoint{3.959623in}{2.518289in}}%
\pgfpathlineto{\pgfqpoint{3.960486in}{2.542268in}}%
\pgfpathlineto{\pgfqpoint{3.961351in}{2.515030in}}%
\pgfpathlineto{\pgfqpoint{3.963081in}{2.586967in}}%
\pgfpathlineto{\pgfqpoint{3.963945in}{2.535382in}}%
\pgfpathlineto{\pgfqpoint{3.964811in}{2.588504in}}%
\pgfpathlineto{\pgfqpoint{3.965676in}{2.542514in}}%
\pgfpathlineto{\pgfqpoint{3.966542in}{2.566124in}}%
\pgfpathlineto{\pgfqpoint{3.967407in}{2.545711in}}%
\pgfpathlineto{\pgfqpoint{3.968272in}{2.552905in}}%
\pgfpathlineto{\pgfqpoint{3.969137in}{2.552351in}}%
\pgfpathlineto{\pgfqpoint{3.970002in}{2.580450in}}%
\pgfpathlineto{\pgfqpoint{3.971732in}{2.504147in}}%
\pgfpathlineto{\pgfqpoint{3.972596in}{2.507590in}}%
\pgfpathlineto{\pgfqpoint{3.973460in}{2.534520in}}%
\pgfpathlineto{\pgfqpoint{3.974325in}{2.624350in}}%
\pgfpathlineto{\pgfqpoint{3.976055in}{2.512447in}}%
\pgfpathlineto{\pgfqpoint{3.976921in}{2.574362in}}%
\pgfpathlineto{\pgfqpoint{3.978650in}{2.495631in}}%
\pgfpathlineto{\pgfqpoint{3.979516in}{2.542453in}}%
\pgfpathlineto{\pgfqpoint{3.981249in}{2.457880in}}%
\pgfpathlineto{\pgfqpoint{3.982981in}{2.577682in}}%
\pgfpathlineto{\pgfqpoint{3.983846in}{2.482505in}}%
\pgfpathlineto{\pgfqpoint{3.984712in}{2.513432in}}%
\pgfpathlineto{\pgfqpoint{3.985575in}{2.496247in}}%
\pgfpathlineto{\pgfqpoint{3.986438in}{2.441310in}}%
\pgfpathlineto{\pgfqpoint{3.987304in}{2.451825in}}%
\pgfpathlineto{\pgfqpoint{3.988169in}{2.450472in}}%
\pgfpathlineto{\pgfqpoint{3.989902in}{2.541285in}}%
\pgfpathlineto{\pgfqpoint{3.990765in}{2.445369in}}%
\pgfpathlineto{\pgfqpoint{3.991629in}{2.488039in}}%
\pgfpathlineto{\pgfqpoint{3.992494in}{2.458896in}}%
\pgfpathlineto{\pgfqpoint{3.993360in}{2.554319in}}%
\pgfpathlineto{\pgfqpoint{3.994223in}{2.530956in}}%
\pgfpathlineto{\pgfqpoint{3.995089in}{2.535566in}}%
\pgfpathlineto{\pgfqpoint{3.995954in}{2.587398in}}%
\pgfpathlineto{\pgfqpoint{3.996820in}{2.541593in}}%
\pgfpathlineto{\pgfqpoint{3.997684in}{2.549893in}}%
\pgfpathlineto{\pgfqpoint{3.998549in}{2.554811in}}%
\pgfpathlineto{\pgfqpoint{3.999414in}{2.619554in}}%
\pgfpathlineto{\pgfqpoint{4.000279in}{2.513432in}}%
\pgfpathlineto{\pgfqpoint{4.001143in}{2.524192in}}%
\pgfpathlineto{\pgfqpoint{4.002007in}{2.500337in}}%
\pgfpathlineto{\pgfqpoint{4.002874in}{2.568338in}}%
\pgfpathlineto{\pgfqpoint{4.004603in}{2.513647in}}%
\pgfpathlineto{\pgfqpoint{4.005467in}{2.522779in}}%
\pgfpathlineto{\pgfqpoint{4.006327in}{2.564343in}}%
\pgfpathlineto{\pgfqpoint{4.008057in}{2.470700in}}%
\pgfpathlineto{\pgfqpoint{4.008923in}{2.597420in}}%
\pgfpathlineto{\pgfqpoint{4.009788in}{2.578544in}}%
\pgfpathlineto{\pgfqpoint{4.012381in}{2.492160in}}%
\pgfpathlineto{\pgfqpoint{4.013246in}{2.509006in}}%
\pgfpathlineto{\pgfqpoint{4.014974in}{2.478387in}}%
\pgfpathlineto{\pgfqpoint{4.015839in}{2.570675in}}%
\pgfpathlineto{\pgfqpoint{4.016704in}{2.531140in}}%
\pgfpathlineto{\pgfqpoint{4.018434in}{2.556533in}}%
\pgfpathlineto{\pgfqpoint{4.019301in}{2.481584in}}%
\pgfpathlineto{\pgfqpoint{4.020166in}{2.485150in}}%
\pgfpathlineto{\pgfqpoint{4.021898in}{2.532800in}}%
\pgfpathlineto{\pgfqpoint{4.022764in}{2.508267in}}%
\pgfpathlineto{\pgfqpoint{4.023629in}{2.519028in}}%
\pgfpathlineto{\pgfqpoint{4.025359in}{2.558316in}}%
\pgfpathlineto{\pgfqpoint{4.026224in}{2.555058in}}%
\pgfpathlineto{\pgfqpoint{4.027087in}{2.489945in}}%
\pgfpathlineto{\pgfqpoint{4.028818in}{2.547617in}}%
\pgfpathlineto{\pgfqpoint{4.029684in}{2.484288in}}%
\pgfpathlineto{\pgfqpoint{4.030549in}{2.570365in}}%
\pgfpathlineto{\pgfqpoint{4.031412in}{2.566062in}}%
\pgfpathlineto{\pgfqpoint{4.032277in}{2.561698in}}%
\pgfpathlineto{\pgfqpoint{4.033143in}{2.522040in}}%
\pgfpathlineto{\pgfqpoint{4.034007in}{2.535197in}}%
\pgfpathlineto{\pgfqpoint{4.034872in}{2.580143in}}%
\pgfpathlineto{\pgfqpoint{4.035739in}{2.484165in}}%
\pgfpathlineto{\pgfqpoint{4.036604in}{2.546327in}}%
\pgfpathlineto{\pgfqpoint{4.037469in}{2.491298in}}%
\pgfpathlineto{\pgfqpoint{4.039199in}{2.564710in}}%
\pgfpathlineto{\pgfqpoint{4.040065in}{2.556471in}}%
\pgfpathlineto{\pgfqpoint{4.040931in}{2.503657in}}%
\pgfpathlineto{\pgfqpoint{4.041797in}{2.569998in}}%
\pgfpathlineto{\pgfqpoint{4.044392in}{2.497507in}}%
\pgfpathlineto{\pgfqpoint{4.045257in}{2.513771in}}%
\pgfpathlineto{\pgfqpoint{4.046121in}{2.563789in}}%
\pgfpathlineto{\pgfqpoint{4.046986in}{2.562190in}}%
\pgfpathlineto{\pgfqpoint{4.047850in}{2.508760in}}%
\pgfpathlineto{\pgfqpoint{4.049579in}{2.558562in}}%
\pgfpathlineto{\pgfqpoint{4.051310in}{2.501689in}}%
\pgfpathlineto{\pgfqpoint{4.053040in}{2.575778in}}%
\pgfpathlineto{\pgfqpoint{4.053906in}{2.459080in}}%
\pgfpathlineto{\pgfqpoint{4.054771in}{2.545344in}}%
\pgfpathlineto{\pgfqpoint{4.055636in}{2.535507in}}%
\pgfpathlineto{\pgfqpoint{4.056502in}{2.497140in}}%
\pgfpathlineto{\pgfqpoint{4.057366in}{2.606397in}}%
\pgfpathlineto{\pgfqpoint{4.059092in}{2.529234in}}%
\pgfpathlineto{\pgfqpoint{4.059957in}{2.535751in}}%
\pgfpathlineto{\pgfqpoint{4.060822in}{2.500583in}}%
\pgfpathlineto{\pgfqpoint{4.062552in}{2.552353in}}%
\pgfpathlineto{\pgfqpoint{4.064282in}{2.519766in}}%
\pgfpathlineto{\pgfqpoint{4.065147in}{2.533908in}}%
\pgfpathlineto{\pgfqpoint{4.066012in}{2.528128in}}%
\pgfpathlineto{\pgfqpoint{4.066875in}{2.551953in}}%
\pgfpathlineto{\pgfqpoint{4.067741in}{2.476296in}}%
\pgfpathlineto{\pgfqpoint{4.068604in}{2.512818in}}%
\pgfpathlineto{\pgfqpoint{4.069469in}{2.491975in}}%
\pgfpathlineto{\pgfqpoint{4.071198in}{2.570552in}}%
\pgfpathlineto{\pgfqpoint{4.072928in}{2.485150in}}%
\pgfpathlineto{\pgfqpoint{4.075522in}{2.542945in}}%
\pgfpathlineto{\pgfqpoint{4.076388in}{2.527574in}}%
\pgfpathlineto{\pgfqpoint{4.077253in}{2.533170in}}%
\pgfpathlineto{\pgfqpoint{4.078118in}{2.573810in}}%
\pgfpathlineto{\pgfqpoint{4.078984in}{2.566986in}}%
\pgfpathlineto{\pgfqpoint{4.079849in}{2.536243in}}%
\pgfpathlineto{\pgfqpoint{4.080713in}{2.564127in}}%
\pgfpathlineto{\pgfqpoint{4.081580in}{2.527943in}}%
\pgfpathlineto{\pgfqpoint{4.084175in}{2.581251in}}%
\pgfpathlineto{\pgfqpoint{4.085039in}{2.517429in}}%
\pgfpathlineto{\pgfqpoint{4.087632in}{2.601356in}}%
\pgfpathlineto{\pgfqpoint{4.088498in}{2.505686in}}%
\pgfpathlineto{\pgfqpoint{4.089364in}{2.523271in}}%
\pgfpathlineto{\pgfqpoint{4.090229in}{2.551645in}}%
\pgfpathlineto{\pgfqpoint{4.091095in}{2.620231in}}%
\pgfpathlineto{\pgfqpoint{4.091961in}{2.607136in}}%
\pgfpathlineto{\pgfqpoint{4.092826in}{2.559332in}}%
\pgfpathlineto{\pgfqpoint{4.093691in}{2.653126in}}%
\pgfpathlineto{\pgfqpoint{4.095420in}{2.539042in}}%
\pgfpathlineto{\pgfqpoint{4.096286in}{2.586723in}}%
\pgfpathlineto{\pgfqpoint{4.098016in}{2.545344in}}%
\pgfpathlineto{\pgfqpoint{4.098881in}{2.606890in}}%
\pgfpathlineto{\pgfqpoint{4.099746in}{2.547096in}}%
\pgfpathlineto{\pgfqpoint{4.100611in}{2.559301in}}%
\pgfpathlineto{\pgfqpoint{4.101476in}{2.548541in}}%
\pgfpathlineto{\pgfqpoint{4.102341in}{2.585954in}}%
\pgfpathlineto{\pgfqpoint{4.103206in}{2.524069in}}%
\pgfpathlineto{\pgfqpoint{4.104936in}{2.579129in}}%
\pgfpathlineto{\pgfqpoint{4.105802in}{2.563604in}}%
\pgfpathlineto{\pgfqpoint{4.107530in}{2.410814in}}%
\pgfpathlineto{\pgfqpoint{4.109259in}{2.568092in}}%
\pgfpathlineto{\pgfqpoint{4.111854in}{2.436515in}}%
\pgfpathlineto{\pgfqpoint{4.113584in}{2.551553in}}%
\pgfpathlineto{\pgfqpoint{4.115313in}{2.486318in}}%
\pgfpathlineto{\pgfqpoint{4.116177in}{2.488162in}}%
\pgfpathlineto{\pgfqpoint{4.117040in}{2.511772in}}%
\pgfpathlineto{\pgfqpoint{4.118770in}{2.444877in}}%
\pgfpathlineto{\pgfqpoint{4.120501in}{2.574978in}}%
\pgfpathlineto{\pgfqpoint{4.121367in}{2.511187in}}%
\pgfpathlineto{\pgfqpoint{4.122231in}{2.564587in}}%
\pgfpathlineto{\pgfqpoint{4.124826in}{2.494433in}}%
\pgfpathlineto{\pgfqpoint{4.125692in}{2.556164in}}%
\pgfpathlineto{\pgfqpoint{4.126557in}{2.516383in}}%
\pgfpathlineto{\pgfqpoint{4.128288in}{2.550907in}}%
\pgfpathlineto{\pgfqpoint{4.129154in}{2.530402in}}%
\pgfpathlineto{\pgfqpoint{4.130019in}{2.539440in}}%
\pgfpathlineto{\pgfqpoint{4.130884in}{2.515215in}}%
\pgfpathlineto{\pgfqpoint{4.131746in}{2.541347in}}%
\pgfpathlineto{\pgfqpoint{4.132610in}{2.455021in}}%
\pgfpathlineto{\pgfqpoint{4.134339in}{2.546573in}}%
\pgfpathlineto{\pgfqpoint{4.135203in}{2.509314in}}%
\pgfpathlineto{\pgfqpoint{4.136068in}{2.515831in}}%
\pgfpathlineto{\pgfqpoint{4.136933in}{2.591088in}}%
\pgfpathlineto{\pgfqpoint{4.137798in}{2.539071in}}%
\pgfpathlineto{\pgfqpoint{4.138664in}{2.618140in}}%
\pgfpathlineto{\pgfqpoint{4.139529in}{2.525791in}}%
\pgfpathlineto{\pgfqpoint{4.140395in}{2.562436in}}%
\pgfpathlineto{\pgfqpoint{4.141259in}{2.550385in}}%
\pgfpathlineto{\pgfqpoint{4.142124in}{2.603693in}}%
\pgfpathlineto{\pgfqpoint{4.142987in}{2.461417in}}%
\pgfpathlineto{\pgfqpoint{4.143853in}{2.552415in}}%
\pgfpathlineto{\pgfqpoint{4.144718in}{2.535137in}}%
\pgfpathlineto{\pgfqpoint{4.145583in}{2.555796in}}%
\pgfpathlineto{\pgfqpoint{4.146448in}{2.543376in}}%
\pgfpathlineto{\pgfqpoint{4.148177in}{2.593179in}}%
\pgfpathlineto{\pgfqpoint{4.149043in}{2.490438in}}%
\pgfpathlineto{\pgfqpoint{4.149908in}{2.507377in}}%
\pgfpathlineto{\pgfqpoint{4.150774in}{2.477833in}}%
\pgfpathlineto{\pgfqpoint{4.152507in}{2.532616in}}%
\pgfpathlineto{\pgfqpoint{4.153374in}{2.458649in}}%
\pgfpathlineto{\pgfqpoint{4.154234in}{2.474267in}}%
\pgfpathlineto{\pgfqpoint{4.155100in}{2.479370in}}%
\pgfpathlineto{\pgfqpoint{4.157692in}{2.563727in}}%
\pgfpathlineto{\pgfqpoint{4.158557in}{2.521303in}}%
\pgfpathlineto{\pgfqpoint{4.159420in}{2.599511in}}%
\pgfpathlineto{\pgfqpoint{4.160285in}{2.595822in}}%
\pgfpathlineto{\pgfqpoint{4.162013in}{2.539194in}}%
\pgfpathlineto{\pgfqpoint{4.162876in}{2.547679in}}%
\pgfpathlineto{\pgfqpoint{4.163741in}{2.532369in}}%
\pgfpathlineto{\pgfqpoint{4.164607in}{2.468794in}}%
\pgfpathlineto{\pgfqpoint{4.165472in}{2.562927in}}%
\pgfpathlineto{\pgfqpoint{4.166337in}{2.494772in}}%
\pgfpathlineto{\pgfqpoint{4.167203in}{2.525421in}}%
\pgfpathlineto{\pgfqpoint{4.168067in}{2.442724in}}%
\pgfpathlineto{\pgfqpoint{4.169798in}{2.495785in}}%
\pgfpathlineto{\pgfqpoint{4.170663in}{2.456251in}}%
\pgfpathlineto{\pgfqpoint{4.173258in}{2.540362in}}%
\pgfpathlineto{\pgfqpoint{4.174125in}{2.538948in}}%
\pgfpathlineto{\pgfqpoint{4.176724in}{2.462092in}}%
\pgfpathlineto{\pgfqpoint{4.177588in}{2.537534in}}%
\pgfpathlineto{\pgfqpoint{4.179317in}{2.458772in}}%
\pgfpathlineto{\pgfqpoint{4.181050in}{2.558131in}}%
\pgfpathlineto{\pgfqpoint{4.182780in}{2.477279in}}%
\pgfpathlineto{\pgfqpoint{4.183646in}{2.544482in}}%
\pgfpathlineto{\pgfqpoint{4.184511in}{2.480476in}}%
\pgfpathlineto{\pgfqpoint{4.185377in}{2.496309in}}%
\pgfpathlineto{\pgfqpoint{4.186240in}{2.508637in}}%
\pgfpathlineto{\pgfqpoint{4.187102in}{2.491606in}}%
\pgfpathlineto{\pgfqpoint{4.187968in}{2.501658in}}%
\pgfpathlineto{\pgfqpoint{4.189700in}{2.547186in}}%
\pgfpathlineto{\pgfqpoint{4.190565in}{2.479093in}}%
\pgfpathlineto{\pgfqpoint{4.191432in}{2.529234in}}%
\pgfpathlineto{\pgfqpoint{4.192297in}{2.482382in}}%
\pgfpathlineto{\pgfqpoint{4.193163in}{2.556041in}}%
\pgfpathlineto{\pgfqpoint{4.195760in}{2.489084in}}%
\pgfpathlineto{\pgfqpoint{4.196623in}{2.516321in}}%
\pgfpathlineto{\pgfqpoint{4.198352in}{2.485394in}}%
\pgfpathlineto{\pgfqpoint{4.199218in}{2.591270in}}%
\pgfpathlineto{\pgfqpoint{4.200949in}{2.526958in}}%
\pgfpathlineto{\pgfqpoint{4.201815in}{2.584876in}}%
\pgfpathlineto{\pgfqpoint{4.202680in}{2.499598in}}%
\pgfpathlineto{\pgfqpoint{4.203545in}{2.555979in}}%
\pgfpathlineto{\pgfqpoint{4.204411in}{2.502579in}}%
\pgfpathlineto{\pgfqpoint{4.206142in}{2.542760in}}%
\pgfpathlineto{\pgfqpoint{4.207008in}{2.496863in}}%
\pgfpathlineto{\pgfqpoint{4.208736in}{2.559668in}}%
\pgfpathlineto{\pgfqpoint{4.209602in}{2.549985in}}%
\pgfpathlineto{\pgfqpoint{4.210467in}{2.592871in}}%
\pgfpathlineto{\pgfqpoint{4.211332in}{2.539071in}}%
\pgfpathlineto{\pgfqpoint{4.212198in}{2.546388in}}%
\pgfpathlineto{\pgfqpoint{4.213060in}{2.515030in}}%
\pgfpathlineto{\pgfqpoint{4.213925in}{2.520411in}}%
\pgfpathlineto{\pgfqpoint{4.214789in}{2.549339in}}%
\pgfpathlineto{\pgfqpoint{4.215654in}{2.539810in}}%
\pgfpathlineto{\pgfqpoint{4.216520in}{2.512295in}}%
\pgfpathlineto{\pgfqpoint{4.218250in}{2.592625in}}%
\pgfpathlineto{\pgfqpoint{4.219112in}{2.515677in}}%
\pgfpathlineto{\pgfqpoint{4.219976in}{2.557210in}}%
\pgfpathlineto{\pgfqpoint{4.220842in}{2.548602in}}%
\pgfpathlineto{\pgfqpoint{4.221705in}{2.510420in}}%
\pgfpathlineto{\pgfqpoint{4.222571in}{2.577684in}}%
\pgfpathlineto{\pgfqpoint{4.223436in}{2.536274in}}%
\pgfpathlineto{\pgfqpoint{4.225166in}{2.585985in}}%
\pgfpathlineto{\pgfqpoint{4.226894in}{2.491298in}}%
\pgfpathlineto{\pgfqpoint{4.227760in}{2.568523in}}%
\pgfpathlineto{\pgfqpoint{4.228626in}{2.526099in}}%
\pgfpathlineto{\pgfqpoint{4.229490in}{2.532677in}}%
\pgfpathlineto{\pgfqpoint{4.230356in}{2.531325in}}%
\pgfpathlineto{\pgfqpoint{4.231219in}{2.542945in}}%
\pgfpathlineto{\pgfqpoint{4.232084in}{2.508760in}}%
\pgfpathlineto{\pgfqpoint{4.233814in}{2.539502in}}%
\pgfpathlineto{\pgfqpoint{4.234678in}{2.584509in}}%
\pgfpathlineto{\pgfqpoint{4.235543in}{2.571966in}}%
\pgfpathlineto{\pgfqpoint{4.236409in}{2.589181in}}%
\pgfpathlineto{\pgfqpoint{4.237273in}{2.520749in}}%
\pgfpathlineto{\pgfqpoint{4.238138in}{2.525514in}}%
\pgfpathlineto{\pgfqpoint{4.239867in}{2.576638in}}%
\pgfpathlineto{\pgfqpoint{4.240732in}{2.554134in}}%
\pgfpathlineto{\pgfqpoint{4.241594in}{2.606151in}}%
\pgfpathlineto{\pgfqpoint{4.242461in}{2.525360in}}%
\pgfpathlineto{\pgfqpoint{4.244190in}{2.582172in}}%
\pgfpathlineto{\pgfqpoint{4.245054in}{2.499875in}}%
\pgfpathlineto{\pgfqpoint{4.246784in}{2.591272in}}%
\pgfpathlineto{\pgfqpoint{4.247649in}{2.531017in}}%
\pgfpathlineto{\pgfqpoint{4.248514in}{2.599080in}}%
\pgfpathlineto{\pgfqpoint{4.249380in}{2.541162in}}%
\pgfpathlineto{\pgfqpoint{4.250245in}{2.559915in}}%
\pgfpathlineto{\pgfqpoint{4.251107in}{2.509191in}}%
\pgfpathlineto{\pgfqpoint{4.251972in}{2.513278in}}%
\pgfpathlineto{\pgfqpoint{4.253702in}{2.543191in}}%
\pgfpathlineto{\pgfqpoint{4.255431in}{2.494802in}}%
\pgfpathlineto{\pgfqpoint{4.256297in}{2.507592in}}%
\pgfpathlineto{\pgfqpoint{4.257163in}{2.440143in}}%
\pgfpathlineto{\pgfqpoint{4.258891in}{2.533047in}}%
\pgfpathlineto{\pgfqpoint{4.259758in}{2.458957in}}%
\pgfpathlineto{\pgfqpoint{4.260621in}{2.469841in}}%
\pgfpathlineto{\pgfqpoint{4.263215in}{2.547987in}}%
\pgfpathlineto{\pgfqpoint{4.264079in}{2.503841in}}%
\pgfpathlineto{\pgfqpoint{4.265810in}{2.579837in}}%
\pgfpathlineto{\pgfqpoint{4.266675in}{2.548602in}}%
\pgfpathlineto{\pgfqpoint{4.267541in}{2.572458in}}%
\pgfpathlineto{\pgfqpoint{4.268406in}{2.517922in}}%
\pgfpathlineto{\pgfqpoint{4.270136in}{2.551491in}}%
\pgfpathlineto{\pgfqpoint{4.271000in}{2.495141in}}%
\pgfpathlineto{\pgfqpoint{4.271865in}{2.539194in}}%
\pgfpathlineto{\pgfqpoint{4.272729in}{2.536797in}}%
\pgfpathlineto{\pgfqpoint{4.273593in}{2.509006in}}%
\pgfpathlineto{\pgfqpoint{4.274458in}{2.533416in}}%
\pgfpathlineto{\pgfqpoint{4.276188in}{2.509468in}}%
\pgfpathlineto{\pgfqpoint{4.277054in}{2.511343in}}%
\pgfpathlineto{\pgfqpoint{4.277919in}{2.507715in}}%
\pgfpathlineto{\pgfqpoint{4.279650in}{2.472301in}}%
\pgfpathlineto{\pgfqpoint{4.280515in}{2.504057in}}%
\pgfpathlineto{\pgfqpoint{4.281379in}{2.500583in}}%
\pgfpathlineto{\pgfqpoint{4.282243in}{2.461602in}}%
\pgfpathlineto{\pgfqpoint{4.283109in}{2.548695in}}%
\pgfpathlineto{\pgfqpoint{4.283974in}{2.543007in}}%
\pgfpathlineto{\pgfqpoint{4.284839in}{2.529850in}}%
\pgfpathlineto{\pgfqpoint{4.285704in}{2.491698in}}%
\pgfpathlineto{\pgfqpoint{4.286569in}{2.510912in}}%
\pgfpathlineto{\pgfqpoint{4.288300in}{2.450595in}}%
\pgfpathlineto{\pgfqpoint{4.289164in}{2.524315in}}%
\pgfpathlineto{\pgfqpoint{4.290894in}{2.462646in}}%
\pgfpathlineto{\pgfqpoint{4.292625in}{2.498399in}}%
\pgfpathlineto{\pgfqpoint{4.293491in}{2.488409in}}%
\pgfpathlineto{\pgfqpoint{4.294356in}{2.531140in}}%
\pgfpathlineto{\pgfqpoint{4.295221in}{2.460771in}}%
\pgfpathlineto{\pgfqpoint{4.296948in}{2.528005in}}%
\pgfpathlineto{\pgfqpoint{4.297812in}{2.502243in}}%
\pgfpathlineto{\pgfqpoint{4.298678in}{2.536859in}}%
\pgfpathlineto{\pgfqpoint{4.300410in}{2.463999in}}%
\pgfpathlineto{\pgfqpoint{4.302140in}{2.516783in}}%
\pgfpathlineto{\pgfqpoint{4.303005in}{2.505809in}}%
\pgfpathlineto{\pgfqpoint{4.304734in}{2.484596in}}%
\pgfpathlineto{\pgfqpoint{4.306465in}{2.575409in}}%
\pgfpathlineto{\pgfqpoint{4.307331in}{2.518104in}}%
\pgfpathlineto{\pgfqpoint{4.308196in}{2.607688in}}%
\pgfpathlineto{\pgfqpoint{4.309061in}{2.509650in}}%
\pgfpathlineto{\pgfqpoint{4.309926in}{2.533722in}}%
\pgfpathlineto{\pgfqpoint{4.311656in}{2.480047in}}%
\pgfpathlineto{\pgfqpoint{4.312522in}{2.493512in}}%
\pgfpathlineto{\pgfqpoint{4.313387in}{2.540423in}}%
\pgfpathlineto{\pgfqpoint{4.314253in}{2.533937in}}%
\pgfpathlineto{\pgfqpoint{4.315118in}{2.570059in}}%
\pgfpathlineto{\pgfqpoint{4.315983in}{2.532677in}}%
\pgfpathlineto{\pgfqpoint{4.317712in}{2.580758in}}%
\pgfpathlineto{\pgfqpoint{4.318576in}{2.495079in}}%
\pgfpathlineto{\pgfqpoint{4.320307in}{2.565326in}}%
\pgfpathlineto{\pgfqpoint{4.321170in}{2.491606in}}%
\pgfpathlineto{\pgfqpoint{4.322899in}{2.552782in}}%
\pgfpathlineto{\pgfqpoint{4.323763in}{2.556595in}}%
\pgfpathlineto{\pgfqpoint{4.324629in}{2.528865in}}%
\pgfpathlineto{\pgfqpoint{4.325492in}{2.549524in}}%
\pgfpathlineto{\pgfqpoint{4.326357in}{2.534399in}}%
\pgfpathlineto{\pgfqpoint{4.328083in}{2.558532in}}%
\pgfpathlineto{\pgfqpoint{4.329813in}{2.481584in}}%
\pgfpathlineto{\pgfqpoint{4.331544in}{2.572273in}}%
\pgfpathlineto{\pgfqpoint{4.332409in}{2.567478in}}%
\pgfpathlineto{\pgfqpoint{4.333276in}{2.580943in}}%
\pgfpathlineto{\pgfqpoint{4.334141in}{2.627364in}}%
\pgfpathlineto{\pgfqpoint{4.335868in}{2.486933in}}%
\pgfpathlineto{\pgfqpoint{4.336733in}{2.585430in}}%
\pgfpathlineto{\pgfqpoint{4.337598in}{2.474205in}}%
\pgfpathlineto{\pgfqpoint{4.338461in}{2.526037in}}%
\pgfpathlineto{\pgfqpoint{4.339326in}{2.505501in}}%
\pgfpathlineto{\pgfqpoint{4.340193in}{2.577407in}}%
\pgfpathlineto{\pgfqpoint{4.341058in}{2.537965in}}%
\pgfpathlineto{\pgfqpoint{4.341924in}{2.555796in}}%
\pgfpathlineto{\pgfqpoint{4.342786in}{2.498430in}}%
\pgfpathlineto{\pgfqpoint{4.344511in}{2.580450in}}%
\pgfpathlineto{\pgfqpoint{4.345375in}{2.520257in}}%
\pgfpathlineto{\pgfqpoint{4.346241in}{2.598467in}}%
\pgfpathlineto{\pgfqpoint{4.347106in}{2.527574in}}%
\pgfpathlineto{\pgfqpoint{4.347971in}{2.529850in}}%
\pgfpathlineto{\pgfqpoint{4.348837in}{2.516998in}}%
\pgfpathlineto{\pgfqpoint{4.349701in}{2.573349in}}%
\pgfpathlineto{\pgfqpoint{4.351432in}{2.517429in}}%
\pgfpathlineto{\pgfqpoint{4.352298in}{2.537934in}}%
\pgfpathlineto{\pgfqpoint{4.353162in}{2.613961in}}%
\pgfpathlineto{\pgfqpoint{4.354892in}{2.536120in}}%
\pgfpathlineto{\pgfqpoint{4.355757in}{2.576823in}}%
\pgfpathlineto{\pgfqpoint{4.356622in}{2.564433in}}%
\pgfpathlineto{\pgfqpoint{4.357487in}{2.522163in}}%
\pgfpathlineto{\pgfqpoint{4.358349in}{2.603875in}}%
\pgfpathlineto{\pgfqpoint{4.359212in}{2.586967in}}%
\pgfpathlineto{\pgfqpoint{4.360077in}{2.615742in}}%
\pgfpathlineto{\pgfqpoint{4.360942in}{2.614143in}}%
\pgfpathlineto{\pgfqpoint{4.362672in}{2.525483in}}%
\pgfpathlineto{\pgfqpoint{4.363537in}{2.576515in}}%
\pgfpathlineto{\pgfqpoint{4.364401in}{2.512878in}}%
\pgfpathlineto{\pgfqpoint{4.365264in}{2.518227in}}%
\pgfpathlineto{\pgfqpoint{4.366126in}{2.580204in}}%
\pgfpathlineto{\pgfqpoint{4.366989in}{2.513494in}}%
\pgfpathlineto{\pgfqpoint{4.367853in}{2.525237in}}%
\pgfpathlineto{\pgfqpoint{4.368718in}{2.497969in}}%
\pgfpathlineto{\pgfqpoint{4.369583in}{2.614697in}}%
\pgfpathlineto{\pgfqpoint{4.371313in}{2.501904in}}%
\pgfpathlineto{\pgfqpoint{4.372179in}{2.545159in}}%
\pgfpathlineto{\pgfqpoint{4.373044in}{2.537719in}}%
\pgfpathlineto{\pgfqpoint{4.373909in}{2.535813in}}%
\pgfpathlineto{\pgfqpoint{4.374774in}{2.492896in}}%
\pgfpathlineto{\pgfqpoint{4.377370in}{2.538948in}}%
\pgfpathlineto{\pgfqpoint{4.378235in}{2.506853in}}%
\pgfpathlineto{\pgfqpoint{4.379100in}{2.518658in}}%
\pgfpathlineto{\pgfqpoint{4.379961in}{2.557331in}}%
\pgfpathlineto{\pgfqpoint{4.380827in}{2.498613in}}%
\pgfpathlineto{\pgfqpoint{4.382557in}{2.585492in}}%
\pgfpathlineto{\pgfqpoint{4.383424in}{2.542391in}}%
\pgfpathlineto{\pgfqpoint{4.384290in}{2.573195in}}%
\pgfpathlineto{\pgfqpoint{4.386020in}{2.542268in}}%
\pgfpathlineto{\pgfqpoint{4.386886in}{2.564094in}}%
\pgfpathlineto{\pgfqpoint{4.387751in}{2.557701in}}%
\pgfpathlineto{\pgfqpoint{4.388616in}{2.476111in}}%
\pgfpathlineto{\pgfqpoint{4.389481in}{2.547125in}}%
\pgfpathlineto{\pgfqpoint{4.390346in}{2.489268in}}%
\pgfpathlineto{\pgfqpoint{4.391212in}{2.516444in}}%
\pgfpathlineto{\pgfqpoint{4.392077in}{2.516383in}}%
\pgfpathlineto{\pgfqpoint{4.392939in}{2.476848in}}%
\pgfpathlineto{\pgfqpoint{4.393805in}{2.498430in}}%
\pgfpathlineto{\pgfqpoint{4.394670in}{2.496401in}}%
\pgfpathlineto{\pgfqpoint{4.395533in}{2.462092in}}%
\pgfpathlineto{\pgfqpoint{4.397262in}{2.510143in}}%
\pgfpathlineto{\pgfqpoint{4.398127in}{2.509496in}}%
\pgfpathlineto{\pgfqpoint{4.398990in}{2.593115in}}%
\pgfpathlineto{\pgfqpoint{4.399855in}{2.464581in}}%
\pgfpathlineto{\pgfqpoint{4.401586in}{2.583463in}}%
\pgfpathlineto{\pgfqpoint{4.404184in}{2.477710in}}%
\pgfpathlineto{\pgfqpoint{4.405047in}{2.508206in}}%
\pgfpathlineto{\pgfqpoint{4.405913in}{2.484596in}}%
\pgfpathlineto{\pgfqpoint{4.406779in}{2.575378in}}%
\pgfpathlineto{\pgfqpoint{4.408510in}{2.484165in}}%
\pgfpathlineto{\pgfqpoint{4.410240in}{2.570244in}}%
\pgfpathlineto{\pgfqpoint{4.411104in}{2.566678in}}%
\pgfpathlineto{\pgfqpoint{4.412836in}{2.527451in}}%
\pgfpathlineto{\pgfqpoint{4.413701in}{2.482444in}}%
\pgfpathlineto{\pgfqpoint{4.414565in}{2.600925in}}%
\pgfpathlineto{\pgfqpoint{4.415430in}{2.579221in}}%
\pgfpathlineto{\pgfqpoint{4.417161in}{2.488409in}}%
\pgfpathlineto{\pgfqpoint{4.418026in}{2.557639in}}%
\pgfpathlineto{\pgfqpoint{4.418891in}{2.472053in}}%
\pgfpathlineto{\pgfqpoint{4.419756in}{2.574916in}}%
\pgfpathlineto{\pgfqpoint{4.421485in}{2.482998in}}%
\pgfpathlineto{\pgfqpoint{4.423211in}{2.532523in}}%
\pgfpathlineto{\pgfqpoint{4.424076in}{2.577315in}}%
\pgfpathlineto{\pgfqpoint{4.424939in}{2.548418in}}%
\pgfpathlineto{\pgfqpoint{4.425802in}{2.469656in}}%
\pgfpathlineto{\pgfqpoint{4.426667in}{2.496647in}}%
\pgfpathlineto{\pgfqpoint{4.427532in}{2.581987in}}%
\pgfpathlineto{\pgfqpoint{4.428398in}{2.429259in}}%
\pgfpathlineto{\pgfqpoint{4.429263in}{2.585430in}}%
\pgfpathlineto{\pgfqpoint{4.430129in}{2.557947in}}%
\pgfpathlineto{\pgfqpoint{4.430994in}{2.548110in}}%
\pgfpathlineto{\pgfqpoint{4.431859in}{2.624473in}}%
\pgfpathlineto{\pgfqpoint{4.432725in}{2.530586in}}%
\pgfpathlineto{\pgfqpoint{4.434454in}{2.624288in}}%
\pgfpathlineto{\pgfqpoint{4.435319in}{2.482967in}}%
\pgfpathlineto{\pgfqpoint{4.436184in}{2.615067in}}%
\pgfpathlineto{\pgfqpoint{4.437050in}{2.587337in}}%
\pgfpathlineto{\pgfqpoint{4.438780in}{2.554196in}}%
\pgfpathlineto{\pgfqpoint{4.439645in}{2.556471in}}%
\pgfpathlineto{\pgfqpoint{4.440508in}{2.506607in}}%
\pgfpathlineto{\pgfqpoint{4.441371in}{2.523331in}}%
\pgfpathlineto{\pgfqpoint{4.442235in}{2.490713in}}%
\pgfpathlineto{\pgfqpoint{4.443101in}{2.533475in}}%
\pgfpathlineto{\pgfqpoint{4.443964in}{2.490805in}}%
\pgfpathlineto{\pgfqpoint{4.445695in}{2.544482in}}%
\pgfpathlineto{\pgfqpoint{4.446561in}{2.501381in}}%
\pgfpathlineto{\pgfqpoint{4.448292in}{2.539687in}}%
\pgfpathlineto{\pgfqpoint{4.449156in}{2.534399in}}%
\pgfpathlineto{\pgfqpoint{4.450020in}{2.463629in}}%
\pgfpathlineto{\pgfqpoint{4.451751in}{2.532400in}}%
\pgfpathlineto{\pgfqpoint{4.452617in}{2.453669in}}%
\pgfpathlineto{\pgfqpoint{4.453484in}{2.537657in}}%
\pgfpathlineto{\pgfqpoint{4.455213in}{2.465597in}}%
\pgfpathlineto{\pgfqpoint{4.456077in}{2.547864in}}%
\pgfpathlineto{\pgfqpoint{4.456943in}{2.431196in}}%
\pgfpathlineto{\pgfqpoint{4.457807in}{2.544297in}}%
\pgfpathlineto{\pgfqpoint{4.458672in}{2.515584in}}%
\pgfpathlineto{\pgfqpoint{4.459537in}{2.486256in}}%
\pgfpathlineto{\pgfqpoint{4.461266in}{2.542914in}}%
\pgfpathlineto{\pgfqpoint{4.462131in}{2.531325in}}%
\pgfpathlineto{\pgfqpoint{4.462995in}{2.476419in}}%
\pgfpathlineto{\pgfqpoint{4.463858in}{2.542668in}}%
\pgfpathlineto{\pgfqpoint{4.464723in}{2.508452in}}%
\pgfpathlineto{\pgfqpoint{4.465588in}{2.539194in}}%
\pgfpathlineto{\pgfqpoint{4.466453in}{2.501843in}}%
\pgfpathlineto{\pgfqpoint{4.467318in}{2.594654in}}%
\pgfpathlineto{\pgfqpoint{4.468183in}{2.490007in}}%
\pgfpathlineto{\pgfqpoint{4.469047in}{2.503472in}}%
\pgfpathlineto{\pgfqpoint{4.469913in}{2.532431in}}%
\pgfpathlineto{\pgfqpoint{4.470777in}{2.497784in}}%
\pgfpathlineto{\pgfqpoint{4.471642in}{2.533722in}}%
\pgfpathlineto{\pgfqpoint{4.472507in}{2.501381in}}%
\pgfpathlineto{\pgfqpoint{4.473373in}{2.535505in}}%
\pgfpathlineto{\pgfqpoint{4.475100in}{2.510604in}}%
\pgfpathlineto{\pgfqpoint{4.476831in}{2.541223in}}%
\pgfpathlineto{\pgfqpoint{4.477695in}{2.538027in}}%
\pgfpathlineto{\pgfqpoint{4.478560in}{2.541593in}}%
\pgfpathlineto{\pgfqpoint{4.481157in}{2.490007in}}%
\pgfpathlineto{\pgfqpoint{4.482023in}{2.510297in}}%
\pgfpathlineto{\pgfqpoint{4.482888in}{2.578483in}}%
\pgfpathlineto{\pgfqpoint{4.483754in}{2.508760in}}%
\pgfpathlineto{\pgfqpoint{4.484620in}{2.560284in}}%
\pgfpathlineto{\pgfqpoint{4.485486in}{2.506884in}}%
\pgfpathlineto{\pgfqpoint{4.486350in}{2.606397in}}%
\pgfpathlineto{\pgfqpoint{4.487216in}{2.604306in}}%
\pgfpathlineto{\pgfqpoint{4.490672in}{2.490192in}}%
\pgfpathlineto{\pgfqpoint{4.491537in}{2.498307in}}%
\pgfpathlineto{\pgfqpoint{4.492402in}{2.550016in}}%
\pgfpathlineto{\pgfqpoint{4.494133in}{2.487424in}}%
\pgfpathlineto{\pgfqpoint{4.494998in}{2.552351in}}%
\pgfpathlineto{\pgfqpoint{4.495863in}{2.518289in}}%
\pgfpathlineto{\pgfqpoint{4.497591in}{2.545588in}}%
\pgfpathlineto{\pgfqpoint{4.498455in}{2.475557in}}%
\pgfpathlineto{\pgfqpoint{4.500183in}{2.540608in}}%
\pgfpathlineto{\pgfqpoint{4.501049in}{2.547494in}}%
\pgfpathlineto{\pgfqpoint{4.501914in}{2.508390in}}%
\pgfpathlineto{\pgfqpoint{4.503645in}{2.575224in}}%
\pgfpathlineto{\pgfqpoint{4.504511in}{2.526866in}}%
\pgfpathlineto{\pgfqpoint{4.505375in}{2.580143in}}%
\pgfpathlineto{\pgfqpoint{4.507970in}{2.522900in}}%
\pgfpathlineto{\pgfqpoint{4.508835in}{2.564187in}}%
\pgfpathlineto{\pgfqpoint{4.509700in}{2.522348in}}%
\pgfpathlineto{\pgfqpoint{4.510564in}{2.549277in}}%
\pgfpathlineto{\pgfqpoint{4.512294in}{2.509927in}}%
\pgfpathlineto{\pgfqpoint{4.513159in}{2.582049in}}%
\pgfpathlineto{\pgfqpoint{4.514025in}{2.496647in}}%
\pgfpathlineto{\pgfqpoint{4.514890in}{2.502058in}}%
\pgfpathlineto{\pgfqpoint{4.515753in}{2.521917in}}%
\pgfpathlineto{\pgfqpoint{4.517482in}{2.456743in}}%
\pgfpathlineto{\pgfqpoint{4.518346in}{2.509866in}}%
\pgfpathlineto{\pgfqpoint{4.519212in}{2.495172in}}%
\pgfpathlineto{\pgfqpoint{4.521806in}{2.568461in}}%
\pgfpathlineto{\pgfqpoint{4.522670in}{2.521917in}}%
\pgfpathlineto{\pgfqpoint{4.523535in}{2.532646in}}%
\pgfpathlineto{\pgfqpoint{4.524400in}{2.560284in}}%
\pgfpathlineto{\pgfqpoint{4.525263in}{2.549093in}}%
\pgfpathlineto{\pgfqpoint{4.526993in}{2.515708in}}%
\pgfpathlineto{\pgfqpoint{4.527857in}{2.546327in}}%
\pgfpathlineto{\pgfqpoint{4.528722in}{2.542576in}}%
\pgfpathlineto{\pgfqpoint{4.529586in}{2.481338in}}%
\pgfpathlineto{\pgfqpoint{4.530452in}{2.551399in}}%
\pgfpathlineto{\pgfqpoint{4.532179in}{2.480907in}}%
\pgfpathlineto{\pgfqpoint{4.533043in}{2.541991in}}%
\pgfpathlineto{\pgfqpoint{4.533909in}{2.468732in}}%
\pgfpathlineto{\pgfqpoint{4.534774in}{2.577254in}}%
\pgfpathlineto{\pgfqpoint{4.536505in}{2.502918in}}%
\pgfpathlineto{\pgfqpoint{4.537370in}{2.533845in}}%
\pgfpathlineto{\pgfqpoint{4.539100in}{2.485887in}}%
\pgfpathlineto{\pgfqpoint{4.539965in}{2.504916in}}%
\pgfpathlineto{\pgfqpoint{4.540828in}{2.586906in}}%
\pgfpathlineto{\pgfqpoint{4.541691in}{2.530586in}}%
\pgfpathlineto{\pgfqpoint{4.542555in}{2.574947in}}%
\pgfpathlineto{\pgfqpoint{4.544282in}{2.540362in}}%
\pgfpathlineto{\pgfqpoint{4.545145in}{2.499228in}}%
\pgfpathlineto{\pgfqpoint{4.546010in}{2.573993in}}%
\pgfpathlineto{\pgfqpoint{4.546874in}{2.570365in}}%
\pgfpathlineto{\pgfqpoint{4.547738in}{2.577682in}}%
\pgfpathlineto{\pgfqpoint{4.548603in}{2.564587in}}%
\pgfpathlineto{\pgfqpoint{4.549468in}{2.524991in}}%
\pgfpathlineto{\pgfqpoint{4.550331in}{2.553090in}}%
\pgfpathlineto{\pgfqpoint{4.552060in}{2.516352in}}%
\pgfpathlineto{\pgfqpoint{4.552925in}{2.589918in}}%
\pgfpathlineto{\pgfqpoint{4.553786in}{2.574485in}}%
\pgfpathlineto{\pgfqpoint{4.554650in}{2.516937in}}%
\pgfpathlineto{\pgfqpoint{4.555516in}{2.578298in}}%
\pgfpathlineto{\pgfqpoint{4.556379in}{2.576761in}}%
\pgfpathlineto{\pgfqpoint{4.557244in}{2.532308in}}%
\pgfpathlineto{\pgfqpoint{4.558109in}{2.547494in}}%
\pgfpathlineto{\pgfqpoint{4.558975in}{2.474544in}}%
\pgfpathlineto{\pgfqpoint{4.560705in}{2.547740in}}%
\pgfpathlineto{\pgfqpoint{4.562436in}{2.510174in}}%
\pgfpathlineto{\pgfqpoint{4.564162in}{2.544974in}}%
\pgfpathlineto{\pgfqpoint{4.565028in}{2.501997in}}%
\pgfpathlineto{\pgfqpoint{4.565894in}{2.508821in}}%
\pgfpathlineto{\pgfqpoint{4.566758in}{2.497140in}}%
\pgfpathlineto{\pgfqpoint{4.567622in}{2.561944in}}%
\pgfpathlineto{\pgfqpoint{4.568487in}{2.481153in}}%
\pgfpathlineto{\pgfqpoint{4.569351in}{2.575655in}}%
\pgfpathlineto{\pgfqpoint{4.570217in}{2.442665in}}%
\pgfpathlineto{\pgfqpoint{4.571948in}{2.555119in}}%
\pgfpathlineto{\pgfqpoint{4.573680in}{2.583955in}}%
\pgfpathlineto{\pgfqpoint{4.576275in}{2.467319in}}%
\pgfpathlineto{\pgfqpoint{4.578005in}{2.567907in}}%
\pgfpathlineto{\pgfqpoint{4.578869in}{2.564833in}}%
\pgfpathlineto{\pgfqpoint{4.579732in}{2.462216in}}%
\pgfpathlineto{\pgfqpoint{4.580597in}{2.611991in}}%
\pgfpathlineto{\pgfqpoint{4.582325in}{2.544913in}}%
\pgfpathlineto{\pgfqpoint{4.583189in}{2.761955in}}%
\pgfpathlineto{\pgfqpoint{4.584053in}{2.749965in}}%
\pgfpathlineto{\pgfqpoint{4.586645in}{2.866971in}}%
\pgfpathlineto{\pgfqpoint{4.587511in}{2.772346in}}%
\pgfpathlineto{\pgfqpoint{4.589242in}{2.853260in}}%
\pgfpathlineto{\pgfqpoint{4.590109in}{2.801336in}}%
\pgfpathlineto{\pgfqpoint{4.590975in}{2.809544in}}%
\pgfpathlineto{\pgfqpoint{4.592706in}{2.788823in}}%
\pgfpathlineto{\pgfqpoint{4.593571in}{2.823378in}}%
\pgfpathlineto{\pgfqpoint{4.594436in}{2.794172in}}%
\pgfpathlineto{\pgfqpoint{4.595300in}{2.816738in}}%
\pgfpathlineto{\pgfqpoint{4.597029in}{2.759802in}}%
\pgfpathlineto{\pgfqpoint{4.599625in}{2.814678in}}%
\pgfpathlineto{\pgfqpoint{4.600489in}{2.821779in}}%
\pgfpathlineto{\pgfqpoint{4.601354in}{2.771117in}}%
\pgfpathlineto{\pgfqpoint{4.602218in}{2.843115in}}%
\pgfpathlineto{\pgfqpoint{4.603082in}{2.795404in}}%
\pgfpathlineto{\pgfqpoint{4.603947in}{2.852983in}}%
\pgfpathlineto{\pgfqpoint{4.604812in}{2.751197in}}%
\pgfpathlineto{\pgfqpoint{4.605675in}{2.788394in}}%
\pgfpathlineto{\pgfqpoint{4.606541in}{2.785043in}}%
\pgfpathlineto{\pgfqpoint{4.607407in}{2.741973in}}%
\pgfpathlineto{\pgfqpoint{4.609137in}{2.801828in}}%
\pgfpathlineto{\pgfqpoint{4.610868in}{2.716088in}}%
\pgfpathlineto{\pgfqpoint{4.613459in}{2.818305in}}%
\pgfpathlineto{\pgfqpoint{4.615186in}{2.779294in}}%
\pgfpathlineto{\pgfqpoint{4.616052in}{2.796109in}}%
\pgfpathlineto{\pgfqpoint{4.616918in}{2.777326in}}%
\pgfpathlineto{\pgfqpoint{4.617784in}{2.789254in}}%
\pgfpathlineto{\pgfqpoint{4.618650in}{2.738068in}}%
\pgfpathlineto{\pgfqpoint{4.620380in}{2.774437in}}%
\pgfpathlineto{\pgfqpoint{4.621246in}{2.781323in}}%
\pgfpathlineto{\pgfqpoint{4.622110in}{2.820735in}}%
\pgfpathlineto{\pgfqpoint{4.622974in}{2.727923in}}%
\pgfpathlineto{\pgfqpoint{4.624705in}{2.782614in}}%
\pgfpathlineto{\pgfqpoint{4.626436in}{2.689833in}}%
\pgfpathlineto{\pgfqpoint{4.628165in}{2.856057in}}%
\pgfpathlineto{\pgfqpoint{4.629028in}{2.796263in}}%
\pgfpathlineto{\pgfqpoint{4.630756in}{2.815755in}}%
\pgfpathlineto{\pgfqpoint{4.631622in}{2.871151in}}%
\pgfpathlineto{\pgfqpoint{4.632489in}{2.810067in}}%
\pgfpathlineto{\pgfqpoint{4.633355in}{2.862114in}}%
\pgfpathlineto{\pgfqpoint{4.635081in}{2.778372in}}%
\pgfpathlineto{\pgfqpoint{4.636810in}{2.839734in}}%
\pgfpathlineto{\pgfqpoint{4.637675in}{2.778065in}}%
\pgfpathlineto{\pgfqpoint{4.638541in}{2.909949in}}%
\pgfpathlineto{\pgfqpoint{4.640272in}{2.786734in}}%
\pgfpathlineto{\pgfqpoint{4.641137in}{2.794911in}}%
\pgfpathlineto{\pgfqpoint{4.642003in}{2.791591in}}%
\pgfpathlineto{\pgfqpoint{4.642868in}{2.836167in}}%
\pgfpathlineto{\pgfqpoint{4.643733in}{2.825530in}}%
\pgfpathlineto{\pgfqpoint{4.644598in}{2.827252in}}%
\pgfpathlineto{\pgfqpoint{4.645460in}{2.843362in}}%
\pgfpathlineto{\pgfqpoint{4.646326in}{2.769580in}}%
\pgfpathlineto{\pgfqpoint{4.647191in}{2.801736in}}%
\pgfpathlineto{\pgfqpoint{4.648057in}{2.797495in}}%
\pgfpathlineto{\pgfqpoint{4.648923in}{2.814464in}}%
\pgfpathlineto{\pgfqpoint{4.649788in}{2.778619in}}%
\pgfpathlineto{\pgfqpoint{4.650652in}{2.813297in}}%
\pgfpathlineto{\pgfqpoint{4.652381in}{2.561084in}}%
\pgfpathlineto{\pgfqpoint{4.653247in}{2.511651in}}%
\pgfpathlineto{\pgfqpoint{4.654976in}{2.547373in}}%
\pgfpathlineto{\pgfqpoint{4.655843in}{2.539933in}}%
\pgfpathlineto{\pgfqpoint{4.656707in}{2.544544in}}%
\pgfpathlineto{\pgfqpoint{4.657572in}{2.563912in}}%
\pgfpathlineto{\pgfqpoint{4.659302in}{2.520626in}}%
\pgfpathlineto{\pgfqpoint{4.660168in}{2.622507in}}%
\pgfpathlineto{\pgfqpoint{4.661897in}{2.459080in}}%
\pgfpathlineto{\pgfqpoint{4.663625in}{2.520872in}}%
\pgfpathlineto{\pgfqpoint{4.664490in}{2.531448in}}%
\pgfpathlineto{\pgfqpoint{4.665354in}{2.530034in}}%
\pgfpathlineto{\pgfqpoint{4.666218in}{2.499015in}}%
\pgfpathlineto{\pgfqpoint{4.667082in}{2.544974in}}%
\pgfpathlineto{\pgfqpoint{4.667948in}{2.543438in}}%
\pgfpathlineto{\pgfqpoint{4.668812in}{2.521119in}}%
\pgfpathlineto{\pgfqpoint{4.669677in}{2.596129in}}%
\pgfpathlineto{\pgfqpoint{4.671407in}{2.487978in}}%
\pgfpathlineto{\pgfqpoint{4.672272in}{2.521611in}}%
\pgfpathlineto{\pgfqpoint{4.673137in}{2.473468in}}%
\pgfpathlineto{\pgfqpoint{4.674000in}{2.490315in}}%
\pgfpathlineto{\pgfqpoint{4.674866in}{2.602341in}}%
\pgfpathlineto{\pgfqpoint{4.675730in}{2.495757in}}%
\pgfpathlineto{\pgfqpoint{4.677460in}{2.579283in}}%
\pgfpathlineto{\pgfqpoint{4.678325in}{2.524131in}}%
\pgfpathlineto{\pgfqpoint{4.679190in}{2.555981in}}%
\pgfpathlineto{\pgfqpoint{4.680921in}{2.466521in}}%
\pgfpathlineto{\pgfqpoint{4.681784in}{2.513986in}}%
\pgfpathlineto{\pgfqpoint{4.682650in}{2.471439in}}%
\pgfpathlineto{\pgfqpoint{4.683516in}{2.563912in}}%
\pgfpathlineto{\pgfqpoint{4.684380in}{2.506363in}}%
\pgfpathlineto{\pgfqpoint{4.686110in}{2.555673in}}%
\pgfpathlineto{\pgfqpoint{4.687840in}{2.520503in}}%
\pgfpathlineto{\pgfqpoint{4.688705in}{2.553705in}}%
\pgfpathlineto{\pgfqpoint{4.689568in}{2.527389in}}%
\pgfpathlineto{\pgfqpoint{4.691297in}{2.627426in}}%
\pgfpathlineto{\pgfqpoint{4.693027in}{2.550139in}}%
\pgfpathlineto{\pgfqpoint{4.693889in}{2.618387in}}%
\pgfpathlineto{\pgfqpoint{4.695621in}{2.528742in}}%
\pgfpathlineto{\pgfqpoint{4.696486in}{2.549647in}}%
\pgfpathlineto{\pgfqpoint{4.697352in}{2.496524in}}%
\pgfpathlineto{\pgfqpoint{4.698218in}{2.562129in}}%
\pgfpathlineto{\pgfqpoint{4.699949in}{2.539994in}}%
\pgfpathlineto{\pgfqpoint{4.700812in}{2.542884in}}%
\pgfpathlineto{\pgfqpoint{4.701675in}{2.493481in}}%
\pgfpathlineto{\pgfqpoint{4.702540in}{2.530586in}}%
\pgfpathlineto{\pgfqpoint{4.704271in}{2.477525in}}%
\pgfpathlineto{\pgfqpoint{4.705999in}{2.528557in}}%
\pgfpathlineto{\pgfqpoint{4.706864in}{2.548846in}}%
\pgfpathlineto{\pgfqpoint{4.708594in}{2.481982in}}%
\pgfpathlineto{\pgfqpoint{4.709461in}{2.490744in}}%
\pgfpathlineto{\pgfqpoint{4.710325in}{2.555548in}}%
\pgfpathlineto{\pgfqpoint{4.711190in}{2.493941in}}%
\pgfpathlineto{\pgfqpoint{4.712056in}{2.525421in}}%
\pgfpathlineto{\pgfqpoint{4.714648in}{2.473035in}}%
\pgfpathlineto{\pgfqpoint{4.717245in}{2.545588in}}%
\pgfpathlineto{\pgfqpoint{4.718111in}{2.488961in}}%
\pgfpathlineto{\pgfqpoint{4.718976in}{2.524621in}}%
\pgfpathlineto{\pgfqpoint{4.719839in}{2.494556in}}%
\pgfpathlineto{\pgfqpoint{4.721569in}{2.539992in}}%
\pgfpathlineto{\pgfqpoint{4.722434in}{2.487485in}}%
\pgfpathlineto{\pgfqpoint{4.723300in}{2.574516in}}%
\pgfpathlineto{\pgfqpoint{4.724166in}{2.562681in}}%
\pgfpathlineto{\pgfqpoint{4.725894in}{2.515154in}}%
\pgfpathlineto{\pgfqpoint{4.726759in}{2.521240in}}%
\pgfpathlineto{\pgfqpoint{4.728487in}{2.458588in}}%
\pgfpathlineto{\pgfqpoint{4.729353in}{2.520134in}}%
\pgfpathlineto{\pgfqpoint{4.730215in}{2.486133in}}%
\pgfpathlineto{\pgfqpoint{4.731947in}{2.527574in}}%
\pgfpathlineto{\pgfqpoint{4.733678in}{2.563235in}}%
\pgfpathlineto{\pgfqpoint{4.734543in}{2.527389in}}%
\pgfpathlineto{\pgfqpoint{4.735409in}{2.572643in}}%
\pgfpathlineto{\pgfqpoint{4.737997in}{2.522594in}}%
\pgfpathlineto{\pgfqpoint{4.739729in}{2.498399in}}%
\pgfpathlineto{\pgfqpoint{4.740595in}{2.584078in}}%
\pgfpathlineto{\pgfqpoint{4.742327in}{2.517060in}}%
\pgfpathlineto{\pgfqpoint{4.743193in}{2.549154in}}%
\pgfpathlineto{\pgfqpoint{4.744060in}{2.508821in}}%
\pgfpathlineto{\pgfqpoint{4.745789in}{2.566249in}}%
\pgfpathlineto{\pgfqpoint{4.746655in}{2.451303in}}%
\pgfpathlineto{\pgfqpoint{4.748384in}{2.493327in}}%
\pgfpathlineto{\pgfqpoint{4.749246in}{2.482076in}}%
\pgfpathlineto{\pgfqpoint{4.750112in}{2.490130in}}%
\pgfpathlineto{\pgfqpoint{4.750977in}{2.561759in}}%
\pgfpathlineto{\pgfqpoint{4.751842in}{2.556010in}}%
\pgfpathlineto{\pgfqpoint{4.752708in}{2.504947in}}%
\pgfpathlineto{\pgfqpoint{4.753575in}{2.570059in}}%
\pgfpathlineto{\pgfqpoint{4.754440in}{2.558378in}}%
\pgfpathlineto{\pgfqpoint{4.756170in}{2.516814in}}%
\pgfpathlineto{\pgfqpoint{4.757901in}{2.575163in}}%
\pgfpathlineto{\pgfqpoint{4.758765in}{2.530740in}}%
\pgfpathlineto{\pgfqpoint{4.759630in}{2.567968in}}%
\pgfpathlineto{\pgfqpoint{4.760492in}{2.544051in}}%
\pgfpathlineto{\pgfqpoint{4.761358in}{2.575686in}}%
\pgfpathlineto{\pgfqpoint{4.762223in}{2.527328in}}%
\pgfpathlineto{\pgfqpoint{4.763088in}{2.556533in}}%
\pgfpathlineto{\pgfqpoint{4.763954in}{2.510420in}}%
\pgfpathlineto{\pgfqpoint{4.764818in}{2.606151in}}%
\pgfpathlineto{\pgfqpoint{4.765683in}{2.557149in}}%
\pgfpathlineto{\pgfqpoint{4.766549in}{2.560407in}}%
\pgfpathlineto{\pgfqpoint{4.767415in}{2.579283in}}%
\pgfpathlineto{\pgfqpoint{4.768280in}{2.517275in}}%
\pgfpathlineto{\pgfqpoint{4.769146in}{2.531448in}}%
\pgfpathlineto{\pgfqpoint{4.770012in}{2.517245in}}%
\pgfpathlineto{\pgfqpoint{4.770878in}{2.575532in}}%
\pgfpathlineto{\pgfqpoint{4.772610in}{2.525175in}}%
\pgfpathlineto{\pgfqpoint{4.773477in}{2.548171in}}%
\pgfpathlineto{\pgfqpoint{4.774342in}{2.491482in}}%
\pgfpathlineto{\pgfqpoint{4.775208in}{2.559361in}}%
\pgfpathlineto{\pgfqpoint{4.776073in}{2.542884in}}%
\pgfpathlineto{\pgfqpoint{4.776937in}{2.541408in}}%
\pgfpathlineto{\pgfqpoint{4.777802in}{2.525668in}}%
\pgfpathlineto{\pgfqpoint{4.778668in}{2.541162in}}%
\pgfpathlineto{\pgfqpoint{4.779534in}{2.532185in}}%
\pgfpathlineto{\pgfqpoint{4.780401in}{2.489699in}}%
\pgfpathlineto{\pgfqpoint{4.781267in}{2.535813in}}%
\pgfpathlineto{\pgfqpoint{4.782133in}{2.525976in}}%
\pgfpathlineto{\pgfqpoint{4.782998in}{2.489576in}}%
\pgfpathlineto{\pgfqpoint{4.783865in}{2.596006in}}%
\pgfpathlineto{\pgfqpoint{4.784730in}{2.484658in}}%
\pgfpathlineto{\pgfqpoint{4.786461in}{2.561821in}}%
\pgfpathlineto{\pgfqpoint{4.787327in}{2.556625in}}%
\pgfpathlineto{\pgfqpoint{4.789059in}{2.582418in}}%
\pgfpathlineto{\pgfqpoint{4.790791in}{2.508513in}}%
\pgfpathlineto{\pgfqpoint{4.791657in}{2.532677in}}%
\pgfpathlineto{\pgfqpoint{4.792520in}{2.459203in}}%
\pgfpathlineto{\pgfqpoint{4.794249in}{2.542668in}}%
\pgfpathlineto{\pgfqpoint{4.795115in}{2.454100in}}%
\pgfpathlineto{\pgfqpoint{4.796845in}{2.580789in}}%
\pgfpathlineto{\pgfqpoint{4.797709in}{2.486687in}}%
\pgfpathlineto{\pgfqpoint{4.798574in}{2.631728in}}%
\pgfpathlineto{\pgfqpoint{4.801166in}{2.531879in}}%
\pgfpathlineto{\pgfqpoint{4.802030in}{2.502181in}}%
\pgfpathlineto{\pgfqpoint{4.803757in}{2.550201in}}%
\pgfpathlineto{\pgfqpoint{4.804623in}{2.505317in}}%
\pgfpathlineto{\pgfqpoint{4.805486in}{2.517799in}}%
\pgfpathlineto{\pgfqpoint{4.806350in}{2.558963in}}%
\pgfpathlineto{\pgfqpoint{4.807214in}{2.505932in}}%
\pgfpathlineto{\pgfqpoint{4.808943in}{2.590965in}}%
\pgfpathlineto{\pgfqpoint{4.810674in}{2.506240in}}%
\pgfpathlineto{\pgfqpoint{4.813264in}{2.615436in}}%
\pgfpathlineto{\pgfqpoint{4.814993in}{2.558809in}}%
\pgfpathlineto{\pgfqpoint{4.815858in}{2.556718in}}%
\pgfpathlineto{\pgfqpoint{4.816722in}{2.491359in}}%
\pgfpathlineto{\pgfqpoint{4.817588in}{2.544482in}}%
\pgfpathlineto{\pgfqpoint{4.818454in}{2.486318in}}%
\pgfpathlineto{\pgfqpoint{4.820184in}{2.553275in}}%
\pgfpathlineto{\pgfqpoint{4.821048in}{2.519212in}}%
\pgfpathlineto{\pgfqpoint{4.821914in}{2.568830in}}%
\pgfpathlineto{\pgfqpoint{4.822781in}{2.539625in}}%
\pgfpathlineto{\pgfqpoint{4.823647in}{2.565757in}}%
\pgfpathlineto{\pgfqpoint{4.825379in}{2.466490in}}%
\pgfpathlineto{\pgfqpoint{4.826244in}{2.467934in}}%
\pgfpathlineto{\pgfqpoint{4.827110in}{2.531448in}}%
\pgfpathlineto{\pgfqpoint{4.827976in}{2.524008in}}%
\pgfpathlineto{\pgfqpoint{4.828842in}{2.502674in}}%
\pgfpathlineto{\pgfqpoint{4.829709in}{2.522163in}}%
\pgfpathlineto{\pgfqpoint{4.830573in}{2.484350in}}%
\pgfpathlineto{\pgfqpoint{4.831440in}{2.515646in}}%
\pgfpathlineto{\pgfqpoint{4.832305in}{2.510358in}}%
\pgfpathlineto{\pgfqpoint{4.833170in}{2.515769in}}%
\pgfpathlineto{\pgfqpoint{4.834897in}{2.485058in}}%
\pgfpathlineto{\pgfqpoint{4.835762in}{2.493573in}}%
\pgfpathlineto{\pgfqpoint{4.836629in}{2.486749in}}%
\pgfpathlineto{\pgfqpoint{4.838359in}{2.521057in}}%
\pgfpathlineto{\pgfqpoint{4.839223in}{2.521057in}}%
\pgfpathlineto{\pgfqpoint{4.840088in}{2.516200in}}%
\pgfpathlineto{\pgfqpoint{4.840953in}{2.545159in}}%
\pgfpathlineto{\pgfqpoint{4.841818in}{2.502089in}}%
\pgfpathlineto{\pgfqpoint{4.842682in}{2.548602in}}%
\pgfpathlineto{\pgfqpoint{4.842682in}{2.548602in}}%
\pgfusepath{stroke}%
\end{pgfscope}%
\begin{pgfscope}%
\pgfsetrectcap%
\pgfsetmiterjoin%
\pgfsetlinewidth{0.803000pt}%
\definecolor{currentstroke}{rgb}{0.000000,0.000000,0.000000}%
\pgfsetstrokecolor{currentstroke}%
\pgfsetdash{}{0pt}%
\pgfpathmoveto{\pgfqpoint{0.483776in}{2.351653in}}%
\pgfpathlineto{\pgfqpoint{0.483776in}{2.936535in}}%
\pgfusepath{stroke}%
\end{pgfscope}%
\begin{pgfscope}%
\pgfsetrectcap%
\pgfsetmiterjoin%
\pgfsetlinewidth{0.803000pt}%
\definecolor{currentstroke}{rgb}{0.000000,0.000000,0.000000}%
\pgfsetstrokecolor{currentstroke}%
\pgfsetdash{}{0pt}%
\pgfpathmoveto{\pgfqpoint{5.050249in}{2.351653in}}%
\pgfpathlineto{\pgfqpoint{5.050249in}{2.936535in}}%
\pgfusepath{stroke}%
\end{pgfscope}%
\begin{pgfscope}%
\pgfsetrectcap%
\pgfsetmiterjoin%
\pgfsetlinewidth{0.803000pt}%
\definecolor{currentstroke}{rgb}{0.000000,0.000000,0.000000}%
\pgfsetstrokecolor{currentstroke}%
\pgfsetdash{}{0pt}%
\pgfpathmoveto{\pgfqpoint{0.483776in}{2.351653in}}%
\pgfpathlineto{\pgfqpoint{5.050249in}{2.351653in}}%
\pgfusepath{stroke}%
\end{pgfscope}%
\begin{pgfscope}%
\pgfsetrectcap%
\pgfsetmiterjoin%
\pgfsetlinewidth{0.803000pt}%
\definecolor{currentstroke}{rgb}{0.000000,0.000000,0.000000}%
\pgfsetstrokecolor{currentstroke}%
\pgfsetdash{}{0pt}%
\pgfpathmoveto{\pgfqpoint{0.483776in}{2.936535in}}%
\pgfpathlineto{\pgfqpoint{5.050249in}{2.936535in}}%
\pgfusepath{stroke}%
\end{pgfscope}%
\begin{pgfscope}%
\pgfsetbuttcap%
\pgfsetmiterjoin%
\definecolor{currentfill}{rgb}{1.000000,1.000000,1.000000}%
\pgfsetfillcolor{currentfill}%
\pgfsetlinewidth{0.000000pt}%
\definecolor{currentstroke}{rgb}{0.000000,0.000000,0.000000}%
\pgfsetstrokecolor{currentstroke}%
\pgfsetstrokeopacity{0.000000}%
\pgfsetdash{}{0pt}%
\pgfpathmoveto{\pgfqpoint{0.483776in}{1.444834in}}%
\pgfpathlineto{\pgfqpoint{5.050249in}{1.444834in}}%
\pgfpathlineto{\pgfqpoint{5.050249in}{2.029715in}}%
\pgfpathlineto{\pgfqpoint{0.483776in}{2.029715in}}%
\pgfpathlineto{\pgfqpoint{0.483776in}{1.444834in}}%
\pgfpathclose%
\pgfusepath{fill}%
\end{pgfscope}%
\begin{pgfscope}%
\pgfsetbuttcap%
\pgfsetroundjoin%
\definecolor{currentfill}{rgb}{0.000000,0.000000,0.000000}%
\pgfsetfillcolor{currentfill}%
\pgfsetlinewidth{0.803000pt}%
\definecolor{currentstroke}{rgb}{0.000000,0.000000,0.000000}%
\pgfsetstrokecolor{currentstroke}%
\pgfsetdash{}{0pt}%
\pgfsys@defobject{currentmarker}{\pgfqpoint{0.000000in}{-0.048611in}}{\pgfqpoint{0.000000in}{0.000000in}}{%
\pgfpathmoveto{\pgfqpoint{0.000000in}{0.000000in}}%
\pgfpathlineto{\pgfqpoint{0.000000in}{-0.048611in}}%
\pgfusepath{stroke,fill}%
}%
\begin{pgfscope}%
\pgfsys@transformshift{0.691021in}{1.444834in}%
\pgfsys@useobject{currentmarker}{}%
\end{pgfscope}%
\end{pgfscope}%
\begin{pgfscope}%
\pgfsetbuttcap%
\pgfsetroundjoin%
\definecolor{currentfill}{rgb}{0.000000,0.000000,0.000000}%
\pgfsetfillcolor{currentfill}%
\pgfsetlinewidth{0.803000pt}%
\definecolor{currentstroke}{rgb}{0.000000,0.000000,0.000000}%
\pgfsetstrokecolor{currentstroke}%
\pgfsetdash{}{0pt}%
\pgfsys@defobject{currentmarker}{\pgfqpoint{0.000000in}{-0.048611in}}{\pgfqpoint{0.000000in}{0.000000in}}{%
\pgfpathmoveto{\pgfqpoint{0.000000in}{0.000000in}}%
\pgfpathlineto{\pgfqpoint{0.000000in}{-0.048611in}}%
\pgfusepath{stroke,fill}%
}%
\begin{pgfscope}%
\pgfsys@transformshift{1.210067in}{1.444834in}%
\pgfsys@useobject{currentmarker}{}%
\end{pgfscope}%
\end{pgfscope}%
\begin{pgfscope}%
\pgfsetbuttcap%
\pgfsetroundjoin%
\definecolor{currentfill}{rgb}{0.000000,0.000000,0.000000}%
\pgfsetfillcolor{currentfill}%
\pgfsetlinewidth{0.803000pt}%
\definecolor{currentstroke}{rgb}{0.000000,0.000000,0.000000}%
\pgfsetstrokecolor{currentstroke}%
\pgfsetdash{}{0pt}%
\pgfsys@defobject{currentmarker}{\pgfqpoint{0.000000in}{-0.048611in}}{\pgfqpoint{0.000000in}{0.000000in}}{%
\pgfpathmoveto{\pgfqpoint{0.000000in}{0.000000in}}%
\pgfpathlineto{\pgfqpoint{0.000000in}{-0.048611in}}%
\pgfusepath{stroke,fill}%
}%
\begin{pgfscope}%
\pgfsys@transformshift{1.729114in}{1.444834in}%
\pgfsys@useobject{currentmarker}{}%
\end{pgfscope}%
\end{pgfscope}%
\begin{pgfscope}%
\pgfsetbuttcap%
\pgfsetroundjoin%
\definecolor{currentfill}{rgb}{0.000000,0.000000,0.000000}%
\pgfsetfillcolor{currentfill}%
\pgfsetlinewidth{0.803000pt}%
\definecolor{currentstroke}{rgb}{0.000000,0.000000,0.000000}%
\pgfsetstrokecolor{currentstroke}%
\pgfsetdash{}{0pt}%
\pgfsys@defobject{currentmarker}{\pgfqpoint{0.000000in}{-0.048611in}}{\pgfqpoint{0.000000in}{0.000000in}}{%
\pgfpathmoveto{\pgfqpoint{0.000000in}{0.000000in}}%
\pgfpathlineto{\pgfqpoint{0.000000in}{-0.048611in}}%
\pgfusepath{stroke,fill}%
}%
\begin{pgfscope}%
\pgfsys@transformshift{2.248160in}{1.444834in}%
\pgfsys@useobject{currentmarker}{}%
\end{pgfscope}%
\end{pgfscope}%
\begin{pgfscope}%
\pgfsetbuttcap%
\pgfsetroundjoin%
\definecolor{currentfill}{rgb}{0.000000,0.000000,0.000000}%
\pgfsetfillcolor{currentfill}%
\pgfsetlinewidth{0.803000pt}%
\definecolor{currentstroke}{rgb}{0.000000,0.000000,0.000000}%
\pgfsetstrokecolor{currentstroke}%
\pgfsetdash{}{0pt}%
\pgfsys@defobject{currentmarker}{\pgfqpoint{0.000000in}{-0.048611in}}{\pgfqpoint{0.000000in}{0.000000in}}{%
\pgfpathmoveto{\pgfqpoint{0.000000in}{0.000000in}}%
\pgfpathlineto{\pgfqpoint{0.000000in}{-0.048611in}}%
\pgfusepath{stroke,fill}%
}%
\begin{pgfscope}%
\pgfsys@transformshift{2.767206in}{1.444834in}%
\pgfsys@useobject{currentmarker}{}%
\end{pgfscope}%
\end{pgfscope}%
\begin{pgfscope}%
\pgfsetbuttcap%
\pgfsetroundjoin%
\definecolor{currentfill}{rgb}{0.000000,0.000000,0.000000}%
\pgfsetfillcolor{currentfill}%
\pgfsetlinewidth{0.803000pt}%
\definecolor{currentstroke}{rgb}{0.000000,0.000000,0.000000}%
\pgfsetstrokecolor{currentstroke}%
\pgfsetdash{}{0pt}%
\pgfsys@defobject{currentmarker}{\pgfqpoint{0.000000in}{-0.048611in}}{\pgfqpoint{0.000000in}{0.000000in}}{%
\pgfpathmoveto{\pgfqpoint{0.000000in}{0.000000in}}%
\pgfpathlineto{\pgfqpoint{0.000000in}{-0.048611in}}%
\pgfusepath{stroke,fill}%
}%
\begin{pgfscope}%
\pgfsys@transformshift{3.286252in}{1.444834in}%
\pgfsys@useobject{currentmarker}{}%
\end{pgfscope}%
\end{pgfscope}%
\begin{pgfscope}%
\pgfsetbuttcap%
\pgfsetroundjoin%
\definecolor{currentfill}{rgb}{0.000000,0.000000,0.000000}%
\pgfsetfillcolor{currentfill}%
\pgfsetlinewidth{0.803000pt}%
\definecolor{currentstroke}{rgb}{0.000000,0.000000,0.000000}%
\pgfsetstrokecolor{currentstroke}%
\pgfsetdash{}{0pt}%
\pgfsys@defobject{currentmarker}{\pgfqpoint{0.000000in}{-0.048611in}}{\pgfqpoint{0.000000in}{0.000000in}}{%
\pgfpathmoveto{\pgfqpoint{0.000000in}{0.000000in}}%
\pgfpathlineto{\pgfqpoint{0.000000in}{-0.048611in}}%
\pgfusepath{stroke,fill}%
}%
\begin{pgfscope}%
\pgfsys@transformshift{3.805298in}{1.444834in}%
\pgfsys@useobject{currentmarker}{}%
\end{pgfscope}%
\end{pgfscope}%
\begin{pgfscope}%
\pgfsetbuttcap%
\pgfsetroundjoin%
\definecolor{currentfill}{rgb}{0.000000,0.000000,0.000000}%
\pgfsetfillcolor{currentfill}%
\pgfsetlinewidth{0.803000pt}%
\definecolor{currentstroke}{rgb}{0.000000,0.000000,0.000000}%
\pgfsetstrokecolor{currentstroke}%
\pgfsetdash{}{0pt}%
\pgfsys@defobject{currentmarker}{\pgfqpoint{0.000000in}{-0.048611in}}{\pgfqpoint{0.000000in}{0.000000in}}{%
\pgfpathmoveto{\pgfqpoint{0.000000in}{0.000000in}}%
\pgfpathlineto{\pgfqpoint{0.000000in}{-0.048611in}}%
\pgfusepath{stroke,fill}%
}%
\begin{pgfscope}%
\pgfsys@transformshift{4.324344in}{1.444834in}%
\pgfsys@useobject{currentmarker}{}%
\end{pgfscope}%
\end{pgfscope}%
\begin{pgfscope}%
\pgfsetbuttcap%
\pgfsetroundjoin%
\definecolor{currentfill}{rgb}{0.000000,0.000000,0.000000}%
\pgfsetfillcolor{currentfill}%
\pgfsetlinewidth{0.803000pt}%
\definecolor{currentstroke}{rgb}{0.000000,0.000000,0.000000}%
\pgfsetstrokecolor{currentstroke}%
\pgfsetdash{}{0pt}%
\pgfsys@defobject{currentmarker}{\pgfqpoint{0.000000in}{-0.048611in}}{\pgfqpoint{0.000000in}{0.000000in}}{%
\pgfpathmoveto{\pgfqpoint{0.000000in}{0.000000in}}%
\pgfpathlineto{\pgfqpoint{0.000000in}{-0.048611in}}%
\pgfusepath{stroke,fill}%
}%
\begin{pgfscope}%
\pgfsys@transformshift{4.843390in}{1.444834in}%
\pgfsys@useobject{currentmarker}{}%
\end{pgfscope}%
\end{pgfscope}%
\begin{pgfscope}%
\pgfsetbuttcap%
\pgfsetroundjoin%
\definecolor{currentfill}{rgb}{0.000000,0.000000,0.000000}%
\pgfsetfillcolor{currentfill}%
\pgfsetlinewidth{0.803000pt}%
\definecolor{currentstroke}{rgb}{0.000000,0.000000,0.000000}%
\pgfsetstrokecolor{currentstroke}%
\pgfsetdash{}{0pt}%
\pgfsys@defobject{currentmarker}{\pgfqpoint{-0.048611in}{0.000000in}}{\pgfqpoint{-0.000000in}{0.000000in}}{%
\pgfpathmoveto{\pgfqpoint{-0.000000in}{0.000000in}}%
\pgfpathlineto{\pgfqpoint{-0.048611in}{0.000000in}}%
\pgfusepath{stroke,fill}%
}%
\begin{pgfscope}%
\pgfsys@transformshift{0.483776in}{1.626011in}%
\pgfsys@useobject{currentmarker}{}%
\end{pgfscope}%
\end{pgfscope}%
\begin{pgfscope}%
\definecolor{textcolor}{rgb}{0.000000,0.000000,0.000000}%
\pgfsetstrokecolor{textcolor}%
\pgfsetfillcolor{textcolor}%
\pgftext[x=0.327525in, y=1.587455in, left, base]{\color{textcolor}\rmfamily\fontsize{8.000000}{9.600000}\selectfont \(\displaystyle {0}\)}%
\end{pgfscope}%
\begin{pgfscope}%
\pgfsetbuttcap%
\pgfsetroundjoin%
\definecolor{currentfill}{rgb}{0.000000,0.000000,0.000000}%
\pgfsetfillcolor{currentfill}%
\pgfsetlinewidth{0.803000pt}%
\definecolor{currentstroke}{rgb}{0.000000,0.000000,0.000000}%
\pgfsetstrokecolor{currentstroke}%
\pgfsetdash{}{0pt}%
\pgfsys@defobject{currentmarker}{\pgfqpoint{-0.048611in}{0.000000in}}{\pgfqpoint{-0.000000in}{0.000000in}}{%
\pgfpathmoveto{\pgfqpoint{-0.000000in}{0.000000in}}%
\pgfpathlineto{\pgfqpoint{-0.048611in}{0.000000in}}%
\pgfusepath{stroke,fill}%
}%
\begin{pgfscope}%
\pgfsys@transformshift{0.483776in}{1.833186in}%
\pgfsys@useobject{currentmarker}{}%
\end{pgfscope}%
\end{pgfscope}%
\begin{pgfscope}%
\definecolor{textcolor}{rgb}{0.000000,0.000000,0.000000}%
\pgfsetstrokecolor{textcolor}%
\pgfsetfillcolor{textcolor}%
\pgftext[x=0.327525in, y=1.794630in, left, base]{\color{textcolor}\rmfamily\fontsize{8.000000}{9.600000}\selectfont \(\displaystyle {5}\)}%
\end{pgfscope}%
\begin{pgfscope}%
\definecolor{textcolor}{rgb}{0.000000,0.000000,0.000000}%
\pgfsetstrokecolor{textcolor}%
\pgfsetfillcolor{textcolor}%
\pgftext[x=0.271969in,y=1.737274in,,bottom,rotate=90.000000]{\color{textcolor}\rmfamily\fontsize{10.000000}{12.000000}\selectfont Voltage deviation in V}%
\end{pgfscope}%
\begin{pgfscope}%
\definecolor{textcolor}{rgb}{0.000000,0.000000,0.000000}%
\pgfsetstrokecolor{textcolor}%
\pgfsetfillcolor{textcolor}%
\pgftext[x=0.483776in,y=2.071382in,left,base]{\color{textcolor}\rmfamily\fontsize{8.000000}{9.600000}\selectfont \(\displaystyle \times{10^{\ensuremath{-}6}}{}\)}%
\end{pgfscope}%
\begin{pgfscope}%
\pgfpathrectangle{\pgfqpoint{0.483776in}{1.444834in}}{\pgfqpoint{4.566474in}{0.584881in}}%
\pgfusepath{clip}%
\pgfsetrectcap%
\pgfsetroundjoin%
\pgfsetlinewidth{0.501875pt}%
\definecolor{currentstroke}{rgb}{0.000000,0.419608,0.643137}%
\pgfsetstrokecolor{currentstroke}%
\pgfsetstrokeopacity{0.700000}%
\pgfsetdash{}{0pt}%
\pgfpathmoveto{\pgfqpoint{0.691343in}{1.596619in}}%
\pgfpathlineto{\pgfqpoint{0.692205in}{1.591513in}}%
\pgfpathlineto{\pgfqpoint{0.693071in}{1.636512in}}%
\pgfpathlineto{\pgfqpoint{0.693935in}{1.590000in}}%
\pgfpathlineto{\pgfqpoint{0.694800in}{1.622502in}}%
\pgfpathlineto{\pgfqpoint{0.696532in}{1.564503in}}%
\pgfpathlineto{\pgfqpoint{0.697397in}{1.619059in}}%
\pgfpathlineto{\pgfqpoint{0.698263in}{1.610629in}}%
\pgfpathlineto{\pgfqpoint{0.699128in}{1.631764in}}%
\pgfpathlineto{\pgfqpoint{0.699993in}{1.620010in}}%
\pgfpathlineto{\pgfqpoint{0.700859in}{1.589230in}}%
\pgfpathlineto{\pgfqpoint{0.701725in}{1.635267in}}%
\pgfpathlineto{\pgfqpoint{0.702589in}{1.632299in}}%
\pgfpathlineto{\pgfqpoint{0.703453in}{1.560261in}}%
\pgfpathlineto{\pgfqpoint{0.705185in}{1.630637in}}%
\pgfpathlineto{\pgfqpoint{0.706051in}{1.606446in}}%
\pgfpathlineto{\pgfqpoint{0.706916in}{1.633962in}}%
\pgfpathlineto{\pgfqpoint{0.707780in}{1.563734in}}%
\pgfpathlineto{\pgfqpoint{0.709512in}{1.641264in}}%
\pgfpathlineto{\pgfqpoint{0.710377in}{1.594068in}}%
\pgfpathlineto{\pgfqpoint{0.711241in}{1.638055in}}%
\pgfpathlineto{\pgfqpoint{0.713837in}{1.560053in}}%
\pgfpathlineto{\pgfqpoint{0.715567in}{1.651947in}}%
\pgfpathlineto{\pgfqpoint{0.717297in}{1.619653in}}%
\pgfpathlineto{\pgfqpoint{0.718163in}{1.549188in}}%
\pgfpathlineto{\pgfqpoint{0.719030in}{1.575011in}}%
\pgfpathlineto{\pgfqpoint{0.719895in}{1.559696in}}%
\pgfpathlineto{\pgfqpoint{0.721627in}{1.606059in}}%
\pgfpathlineto{\pgfqpoint{0.722492in}{1.557498in}}%
\pgfpathlineto{\pgfqpoint{0.724219in}{1.598697in}}%
\pgfpathlineto{\pgfqpoint{0.725084in}{1.538948in}}%
\pgfpathlineto{\pgfqpoint{0.727676in}{1.610867in}}%
\pgfpathlineto{\pgfqpoint{0.728541in}{1.538208in}}%
\pgfpathlineto{\pgfqpoint{0.731135in}{1.630161in}}%
\pgfpathlineto{\pgfqpoint{0.731999in}{1.530756in}}%
\pgfpathlineto{\pgfqpoint{0.733731in}{1.628023in}}%
\pgfpathlineto{\pgfqpoint{0.734597in}{1.605970in}}%
\pgfpathlineto{\pgfqpoint{0.735460in}{1.549128in}}%
\pgfpathlineto{\pgfqpoint{0.736325in}{1.631645in}}%
\pgfpathlineto{\pgfqpoint{0.737190in}{1.582195in}}%
\pgfpathlineto{\pgfqpoint{0.738920in}{1.627845in}}%
\pgfpathlineto{\pgfqpoint{0.739784in}{1.595611in}}%
\pgfpathlineto{\pgfqpoint{0.740649in}{1.605643in}}%
\pgfpathlineto{\pgfqpoint{0.742381in}{1.518973in}}%
\pgfpathlineto{\pgfqpoint{0.744977in}{1.639186in}}%
\pgfpathlineto{\pgfqpoint{0.746705in}{1.568661in}}%
\pgfpathlineto{\pgfqpoint{0.748435in}{1.669520in}}%
\pgfpathlineto{\pgfqpoint{0.750163in}{1.617579in}}%
\pgfpathlineto{\pgfqpoint{0.751028in}{1.624822in}}%
\pgfpathlineto{\pgfqpoint{0.751891in}{1.524970in}}%
\pgfpathlineto{\pgfqpoint{0.752757in}{1.642391in}}%
\pgfpathlineto{\pgfqpoint{0.753622in}{1.534824in}}%
\pgfpathlineto{\pgfqpoint{0.754488in}{1.630578in}}%
\pgfpathlineto{\pgfqpoint{0.755354in}{1.628916in}}%
\pgfpathlineto{\pgfqpoint{0.756219in}{1.610454in}}%
\pgfpathlineto{\pgfqpoint{0.757084in}{1.687092in}}%
\pgfpathlineto{\pgfqpoint{0.759679in}{1.594722in}}%
\pgfpathlineto{\pgfqpoint{0.761407in}{1.632478in}}%
\pgfpathlineto{\pgfqpoint{0.762269in}{1.608733in}}%
\pgfpathlineto{\pgfqpoint{0.763135in}{1.634378in}}%
\pgfpathlineto{\pgfqpoint{0.764000in}{1.784806in}}%
\pgfpathlineto{\pgfqpoint{0.766596in}{1.673320in}}%
\pgfpathlineto{\pgfqpoint{0.767462in}{1.752453in}}%
\pgfpathlineto{\pgfqpoint{0.768326in}{1.742183in}}%
\pgfpathlineto{\pgfqpoint{0.769191in}{1.752869in}}%
\pgfpathlineto{\pgfqpoint{0.771788in}{1.592346in}}%
\pgfpathlineto{\pgfqpoint{0.772652in}{1.682965in}}%
\pgfpathlineto{\pgfqpoint{0.774383in}{1.598816in}}%
\pgfpathlineto{\pgfqpoint{0.775249in}{1.627607in}}%
\pgfpathlineto{\pgfqpoint{0.776979in}{1.584330in}}%
\pgfpathlineto{\pgfqpoint{0.777845in}{1.590208in}}%
\pgfpathlineto{\pgfqpoint{0.779575in}{1.572963in}}%
\pgfpathlineto{\pgfqpoint{0.780439in}{1.634553in}}%
\pgfpathlineto{\pgfqpoint{0.781305in}{1.591454in}}%
\pgfpathlineto{\pgfqpoint{0.782170in}{1.593116in}}%
\pgfpathlineto{\pgfqpoint{0.783036in}{1.580117in}}%
\pgfpathlineto{\pgfqpoint{0.783901in}{1.581600in}}%
\pgfpathlineto{\pgfqpoint{0.784766in}{1.593592in}}%
\pgfpathlineto{\pgfqpoint{0.785630in}{1.530012in}}%
\pgfpathlineto{\pgfqpoint{0.786494in}{1.608848in}}%
\pgfpathlineto{\pgfqpoint{0.787360in}{1.607245in}}%
\pgfpathlineto{\pgfqpoint{0.788225in}{1.537137in}}%
\pgfpathlineto{\pgfqpoint{0.789089in}{1.538115in}}%
\pgfpathlineto{\pgfqpoint{0.790816in}{1.602259in}}%
\pgfpathlineto{\pgfqpoint{0.791680in}{1.600210in}}%
\pgfpathlineto{\pgfqpoint{0.793412in}{1.593949in}}%
\pgfpathlineto{\pgfqpoint{0.795138in}{1.575725in}}%
\pgfpathlineto{\pgfqpoint{0.796004in}{1.655747in}}%
\pgfpathlineto{\pgfqpoint{0.798599in}{1.566879in}}%
\pgfpathlineto{\pgfqpoint{0.799464in}{1.545923in}}%
\pgfpathlineto{\pgfqpoint{0.801192in}{1.607840in}}%
\pgfpathlineto{\pgfqpoint{0.802924in}{1.552218in}}%
\pgfpathlineto{\pgfqpoint{0.803787in}{1.549902in}}%
\pgfpathlineto{\pgfqpoint{0.804652in}{1.567474in}}%
\pgfpathlineto{\pgfqpoint{0.805515in}{1.508703in}}%
\pgfpathlineto{\pgfqpoint{0.806378in}{1.522178in}}%
\pgfpathlineto{\pgfqpoint{0.807243in}{1.582017in}}%
\pgfpathlineto{\pgfqpoint{0.808107in}{1.561239in}}%
\pgfpathlineto{\pgfqpoint{0.808973in}{1.571687in}}%
\pgfpathlineto{\pgfqpoint{0.809837in}{1.500333in}}%
\pgfpathlineto{\pgfqpoint{0.811567in}{1.570025in}}%
\pgfpathlineto{\pgfqpoint{0.812431in}{1.578038in}}%
\pgfpathlineto{\pgfqpoint{0.813297in}{1.571449in}}%
\pgfpathlineto{\pgfqpoint{0.814161in}{1.632950in}}%
\pgfpathlineto{\pgfqpoint{0.815026in}{1.547496in}}%
\pgfpathlineto{\pgfqpoint{0.815891in}{1.566166in}}%
\pgfpathlineto{\pgfqpoint{0.817620in}{1.605167in}}%
\pgfpathlineto{\pgfqpoint{0.818484in}{1.600419in}}%
\pgfpathlineto{\pgfqpoint{0.819349in}{1.562306in}}%
\pgfpathlineto{\pgfqpoint{0.820213in}{1.627696in}}%
\pgfpathlineto{\pgfqpoint{0.821078in}{1.621137in}}%
\pgfpathlineto{\pgfqpoint{0.822807in}{1.664887in}}%
\pgfpathlineto{\pgfqpoint{0.823671in}{1.600538in}}%
\pgfpathlineto{\pgfqpoint{0.824536in}{1.619267in}}%
\pgfpathlineto{\pgfqpoint{0.826268in}{1.602616in}}%
\pgfpathlineto{\pgfqpoint{0.827999in}{1.618170in}}%
\pgfpathlineto{\pgfqpoint{0.828864in}{1.556729in}}%
\pgfpathlineto{\pgfqpoint{0.830596in}{1.696054in}}%
\pgfpathlineto{\pgfqpoint{0.831462in}{1.617991in}}%
\pgfpathlineto{\pgfqpoint{0.832328in}{1.622859in}}%
\pgfpathlineto{\pgfqpoint{0.833193in}{1.666196in}}%
\pgfpathlineto{\pgfqpoint{0.834058in}{1.655182in}}%
\pgfpathlineto{\pgfqpoint{0.834922in}{1.663879in}}%
\pgfpathlineto{\pgfqpoint{0.836650in}{1.590744in}}%
\pgfpathlineto{\pgfqpoint{0.837516in}{1.554352in}}%
\pgfpathlineto{\pgfqpoint{0.838382in}{1.571568in}}%
\pgfpathlineto{\pgfqpoint{0.839248in}{1.531764in}}%
\pgfpathlineto{\pgfqpoint{0.840980in}{1.572992in}}%
\pgfpathlineto{\pgfqpoint{0.842710in}{1.547228in}}%
\pgfpathlineto{\pgfqpoint{0.844440in}{1.605107in}}%
\pgfpathlineto{\pgfqpoint{0.845305in}{1.600300in}}%
\pgfpathlineto{\pgfqpoint{0.846169in}{1.545566in}}%
\pgfpathlineto{\pgfqpoint{0.847897in}{1.599292in}}%
\pgfpathlineto{\pgfqpoint{0.849627in}{1.580176in}}%
\pgfpathlineto{\pgfqpoint{0.850491in}{1.575368in}}%
\pgfpathlineto{\pgfqpoint{0.851354in}{1.557498in}}%
\pgfpathlineto{\pgfqpoint{0.853081in}{1.600359in}}%
\pgfpathlineto{\pgfqpoint{0.853945in}{1.542361in}}%
\pgfpathlineto{\pgfqpoint{0.854810in}{1.630991in}}%
\pgfpathlineto{\pgfqpoint{0.855672in}{1.627815in}}%
\pgfpathlineto{\pgfqpoint{0.856537in}{1.595254in}}%
\pgfpathlineto{\pgfqpoint{0.857402in}{1.657231in}}%
\pgfpathlineto{\pgfqpoint{0.858267in}{1.596767in}}%
\pgfpathlineto{\pgfqpoint{0.859131in}{1.636453in}}%
\pgfpathlineto{\pgfqpoint{0.860864in}{1.577741in}}%
\pgfpathlineto{\pgfqpoint{0.861730in}{1.630693in}}%
\pgfpathlineto{\pgfqpoint{0.864329in}{1.560644in}}%
\pgfpathlineto{\pgfqpoint{0.866062in}{1.576019in}}%
\pgfpathlineto{\pgfqpoint{0.866929in}{1.597983in}}%
\pgfpathlineto{\pgfqpoint{0.867794in}{1.565333in}}%
\pgfpathlineto{\pgfqpoint{0.869523in}{1.610391in}}%
\pgfpathlineto{\pgfqpoint{0.870388in}{1.588427in}}%
\pgfpathlineto{\pgfqpoint{0.871253in}{1.636393in}}%
\pgfpathlineto{\pgfqpoint{0.872119in}{1.630693in}}%
\pgfpathlineto{\pgfqpoint{0.873848in}{1.621550in}}%
\pgfpathlineto{\pgfqpoint{0.874712in}{1.570438in}}%
\pgfpathlineto{\pgfqpoint{0.876442in}{1.691008in}}%
\pgfpathlineto{\pgfqpoint{0.877309in}{1.664946in}}%
\pgfpathlineto{\pgfqpoint{0.878174in}{1.667144in}}%
\pgfpathlineto{\pgfqpoint{0.879903in}{1.651709in}}%
\pgfpathlineto{\pgfqpoint{0.881633in}{1.694749in}}%
\pgfpathlineto{\pgfqpoint{0.882498in}{1.646187in}}%
\pgfpathlineto{\pgfqpoint{0.883364in}{1.713150in}}%
\pgfpathlineto{\pgfqpoint{0.884229in}{1.703356in}}%
\pgfpathlineto{\pgfqpoint{0.885096in}{1.669282in}}%
\pgfpathlineto{\pgfqpoint{0.885960in}{1.702999in}}%
\pgfpathlineto{\pgfqpoint{0.886826in}{1.628142in}}%
\pgfpathlineto{\pgfqpoint{0.887691in}{1.653490in}}%
\pgfpathlineto{\pgfqpoint{0.888555in}{1.717129in}}%
\pgfpathlineto{\pgfqpoint{0.892011in}{1.653193in}}%
\pgfpathlineto{\pgfqpoint{0.892875in}{1.734047in}}%
\pgfpathlineto{\pgfqpoint{0.893741in}{1.654557in}}%
\pgfpathlineto{\pgfqpoint{0.894605in}{1.695102in}}%
\pgfpathlineto{\pgfqpoint{0.897202in}{1.588427in}}%
\pgfpathlineto{\pgfqpoint{0.898067in}{1.593235in}}%
\pgfpathlineto{\pgfqpoint{0.899795in}{1.651293in}}%
\pgfpathlineto{\pgfqpoint{0.900660in}{1.598310in}}%
\pgfpathlineto{\pgfqpoint{0.901525in}{1.619356in}}%
\pgfpathlineto{\pgfqpoint{0.902390in}{1.677295in}}%
\pgfpathlineto{\pgfqpoint{0.903254in}{1.612380in}}%
\pgfpathlineto{\pgfqpoint{0.904119in}{1.652244in}}%
\pgfpathlineto{\pgfqpoint{0.904984in}{1.593116in}}%
\pgfpathlineto{\pgfqpoint{0.905849in}{1.665333in}}%
\pgfpathlineto{\pgfqpoint{0.906714in}{1.656279in}}%
\pgfpathlineto{\pgfqpoint{0.907579in}{1.650106in}}%
\pgfpathlineto{\pgfqpoint{0.908445in}{1.596619in}}%
\pgfpathlineto{\pgfqpoint{0.910175in}{1.645061in}}%
\pgfpathlineto{\pgfqpoint{0.912770in}{1.569282in}}%
\pgfpathlineto{\pgfqpoint{0.914501in}{1.562127in}}%
\pgfpathlineto{\pgfqpoint{0.915365in}{1.582786in}}%
\pgfpathlineto{\pgfqpoint{0.916229in}{1.546574in}}%
\pgfpathlineto{\pgfqpoint{0.917959in}{1.613715in}}%
\pgfpathlineto{\pgfqpoint{0.918824in}{1.552690in}}%
\pgfpathlineto{\pgfqpoint{0.919687in}{1.556104in}}%
\pgfpathlineto{\pgfqpoint{0.920552in}{1.542539in}}%
\pgfpathlineto{\pgfqpoint{0.922280in}{1.625499in}}%
\pgfpathlineto{\pgfqpoint{0.923146in}{1.597094in}}%
\pgfpathlineto{\pgfqpoint{0.924010in}{1.553877in}}%
\pgfpathlineto{\pgfqpoint{0.925740in}{1.631110in}}%
\pgfpathlineto{\pgfqpoint{0.928329in}{1.557974in}}%
\pgfpathlineto{\pgfqpoint{0.930058in}{1.603802in}}%
\pgfpathlineto{\pgfqpoint{0.930923in}{1.541528in}}%
\pgfpathlineto{\pgfqpoint{0.931787in}{1.610094in}}%
\pgfpathlineto{\pgfqpoint{0.932652in}{1.555598in}}%
\pgfpathlineto{\pgfqpoint{0.933516in}{1.584092in}}%
\pgfpathlineto{\pgfqpoint{0.934381in}{1.531496in}}%
\pgfpathlineto{\pgfqpoint{0.936109in}{1.601307in}}%
\pgfpathlineto{\pgfqpoint{0.936974in}{1.599526in}}%
\pgfpathlineto{\pgfqpoint{0.937840in}{1.629269in}}%
\pgfpathlineto{\pgfqpoint{0.938706in}{1.543190in}}%
\pgfpathlineto{\pgfqpoint{0.939570in}{1.557082in}}%
\pgfpathlineto{\pgfqpoint{0.940435in}{1.640011in}}%
\pgfpathlineto{\pgfqpoint{0.943032in}{1.518553in}}%
\pgfpathlineto{\pgfqpoint{0.943897in}{1.606175in}}%
\pgfpathlineto{\pgfqpoint{0.944763in}{1.599586in}}%
\pgfpathlineto{\pgfqpoint{0.945628in}{1.627309in}}%
\pgfpathlineto{\pgfqpoint{0.946490in}{1.547228in}}%
\pgfpathlineto{\pgfqpoint{0.947355in}{1.605345in}}%
\pgfpathlineto{\pgfqpoint{0.949082in}{1.571390in}}%
\pgfpathlineto{\pgfqpoint{0.949946in}{1.566225in}}%
\pgfpathlineto{\pgfqpoint{0.950811in}{1.544703in}}%
\pgfpathlineto{\pgfqpoint{0.951674in}{1.488219in}}%
\pgfpathlineto{\pgfqpoint{0.953405in}{1.594272in}}%
\pgfpathlineto{\pgfqpoint{0.954269in}{1.610391in}}%
\pgfpathlineto{\pgfqpoint{0.955133in}{1.519564in}}%
\pgfpathlineto{\pgfqpoint{0.956863in}{1.648325in}}%
\pgfpathlineto{\pgfqpoint{0.957728in}{1.610570in}}%
\pgfpathlineto{\pgfqpoint{0.958594in}{1.615734in}}%
\pgfpathlineto{\pgfqpoint{0.961191in}{1.571033in}}%
\pgfpathlineto{\pgfqpoint{0.962055in}{1.590625in}}%
\pgfpathlineto{\pgfqpoint{0.962920in}{1.537018in}}%
\pgfpathlineto{\pgfqpoint{0.963783in}{1.619059in}}%
\pgfpathlineto{\pgfqpoint{0.964647in}{1.581481in}}%
\pgfpathlineto{\pgfqpoint{0.966375in}{1.614132in}}%
\pgfpathlineto{\pgfqpoint{0.967241in}{1.554085in}}%
\pgfpathlineto{\pgfqpoint{0.968105in}{1.590625in}}%
\pgfpathlineto{\pgfqpoint{0.968970in}{1.574654in}}%
\pgfpathlineto{\pgfqpoint{0.969836in}{1.576555in}}%
\pgfpathlineto{\pgfqpoint{0.970701in}{1.635088in}}%
\pgfpathlineto{\pgfqpoint{0.972432in}{1.491011in}}%
\pgfpathlineto{\pgfqpoint{0.975027in}{1.636215in}}%
\pgfpathlineto{\pgfqpoint{0.975889in}{1.550433in}}%
\pgfpathlineto{\pgfqpoint{0.976751in}{1.557409in}}%
\pgfpathlineto{\pgfqpoint{0.977616in}{1.638293in}}%
\pgfpathlineto{\pgfqpoint{0.978481in}{1.622918in}}%
\pgfpathlineto{\pgfqpoint{0.980211in}{1.599054in}}%
\pgfpathlineto{\pgfqpoint{0.981076in}{1.655093in}}%
\pgfpathlineto{\pgfqpoint{0.982806in}{1.611165in}}%
\pgfpathlineto{\pgfqpoint{0.983671in}{1.629745in}}%
\pgfpathlineto{\pgfqpoint{0.985400in}{1.603981in}}%
\pgfpathlineto{\pgfqpoint{0.986264in}{1.621048in}}%
\pgfpathlineto{\pgfqpoint{0.987127in}{1.532031in}}%
\pgfpathlineto{\pgfqpoint{0.988856in}{1.586051in}}%
\pgfpathlineto{\pgfqpoint{0.989722in}{1.576317in}}%
\pgfpathlineto{\pgfqpoint{0.991449in}{1.613805in}}%
\pgfpathlineto{\pgfqpoint{0.992314in}{1.593413in}}%
\pgfpathlineto{\pgfqpoint{0.993178in}{1.616151in}}%
\pgfpathlineto{\pgfqpoint{0.994043in}{1.511611in}}%
\pgfpathlineto{\pgfqpoint{0.995771in}{1.625885in}}%
\pgfpathlineto{\pgfqpoint{0.996636in}{1.581303in}}%
\pgfpathlineto{\pgfqpoint{0.997501in}{1.613953in}}%
\pgfpathlineto{\pgfqpoint{0.999232in}{1.567530in}}%
\pgfpathlineto{\pgfqpoint{1.000096in}{1.630574in}}%
\pgfpathlineto{\pgfqpoint{1.000961in}{1.555003in}}%
\pgfpathlineto{\pgfqpoint{1.001826in}{1.614370in}}%
\pgfpathlineto{\pgfqpoint{1.002691in}{1.603505in}}%
\pgfpathlineto{\pgfqpoint{1.003553in}{1.587892in}}%
\pgfpathlineto{\pgfqpoint{1.004419in}{1.598340in}}%
\pgfpathlineto{\pgfqpoint{1.005284in}{1.487181in}}%
\pgfpathlineto{\pgfqpoint{1.007014in}{1.632534in}}%
\pgfpathlineto{\pgfqpoint{1.007877in}{1.605078in}}%
\pgfpathlineto{\pgfqpoint{1.008742in}{1.613061in}}%
\pgfpathlineto{\pgfqpoint{1.009607in}{1.592759in}}%
\pgfpathlineto{\pgfqpoint{1.010472in}{1.613537in}}%
\pgfpathlineto{\pgfqpoint{1.011338in}{1.536363in}}%
\pgfpathlineto{\pgfqpoint{1.013067in}{1.643161in}}%
\pgfpathlineto{\pgfqpoint{1.013933in}{1.617753in}}%
\pgfpathlineto{\pgfqpoint{1.014798in}{1.556342in}}%
\pgfpathlineto{\pgfqpoint{1.016526in}{1.631883in}}%
\pgfpathlineto{\pgfqpoint{1.019120in}{1.587122in}}%
\pgfpathlineto{\pgfqpoint{1.019986in}{1.548980in}}%
\pgfpathlineto{\pgfqpoint{1.020851in}{1.606889in}}%
\pgfpathlineto{\pgfqpoint{1.021716in}{1.569668in}}%
\pgfpathlineto{\pgfqpoint{1.022581in}{1.569966in}}%
\pgfpathlineto{\pgfqpoint{1.023445in}{1.627666in}}%
\pgfpathlineto{\pgfqpoint{1.025173in}{1.558982in}}%
\pgfpathlineto{\pgfqpoint{1.027767in}{1.720810in}}%
\pgfpathlineto{\pgfqpoint{1.028629in}{1.710778in}}%
\pgfpathlineto{\pgfqpoint{1.029494in}{1.611343in}}%
\pgfpathlineto{\pgfqpoint{1.030357in}{1.716121in}}%
\pgfpathlineto{\pgfqpoint{1.032087in}{1.678723in}}%
\pgfpathlineto{\pgfqpoint{1.033816in}{1.636988in}}%
\pgfpathlineto{\pgfqpoint{1.034679in}{1.638888in}}%
\pgfpathlineto{\pgfqpoint{1.035545in}{1.650701in}}%
\pgfpathlineto{\pgfqpoint{1.036409in}{1.686884in}}%
\pgfpathlineto{\pgfqpoint{1.037275in}{1.645656in}}%
\pgfpathlineto{\pgfqpoint{1.038139in}{1.703535in}}%
\pgfpathlineto{\pgfqpoint{1.039005in}{1.696946in}}%
\pgfpathlineto{\pgfqpoint{1.039870in}{1.722353in}}%
\pgfpathlineto{\pgfqpoint{1.040734in}{1.708224in}}%
\pgfpathlineto{\pgfqpoint{1.041600in}{1.726153in}}%
\pgfpathlineto{\pgfqpoint{1.042466in}{1.705554in}}%
\pgfpathlineto{\pgfqpoint{1.045060in}{1.762069in}}%
\pgfpathlineto{\pgfqpoint{1.045927in}{1.694867in}}%
\pgfpathlineto{\pgfqpoint{1.046792in}{1.720989in}}%
\pgfpathlineto{\pgfqpoint{1.048523in}{1.686468in}}%
\pgfpathlineto{\pgfqpoint{1.049390in}{1.691543in}}%
\pgfpathlineto{\pgfqpoint{1.050256in}{1.751799in}}%
\pgfpathlineto{\pgfqpoint{1.051987in}{1.667739in}}%
\pgfpathlineto{\pgfqpoint{1.053719in}{1.735174in}}%
\pgfpathlineto{\pgfqpoint{1.055449in}{1.663165in}}%
\pgfpathlineto{\pgfqpoint{1.056313in}{1.710064in}}%
\pgfpathlineto{\pgfqpoint{1.058907in}{1.644823in}}%
\pgfpathlineto{\pgfqpoint{1.061503in}{1.726034in}}%
\pgfpathlineto{\pgfqpoint{1.062369in}{1.679254in}}%
\pgfpathlineto{\pgfqpoint{1.063234in}{1.552036in}}%
\pgfpathlineto{\pgfqpoint{1.064099in}{1.678302in}}%
\pgfpathlineto{\pgfqpoint{1.065827in}{1.572814in}}%
\pgfpathlineto{\pgfqpoint{1.066692in}{1.577860in}}%
\pgfpathlineto{\pgfqpoint{1.068422in}{1.552333in}}%
\pgfpathlineto{\pgfqpoint{1.069287in}{1.575722in}}%
\pgfpathlineto{\pgfqpoint{1.070151in}{1.674859in}}%
\pgfpathlineto{\pgfqpoint{1.071880in}{1.591067in}}%
\pgfpathlineto{\pgfqpoint{1.072744in}{1.634374in}}%
\pgfpathlineto{\pgfqpoint{1.074470in}{1.546455in}}%
\pgfpathlineto{\pgfqpoint{1.075335in}{1.572635in}}%
\pgfpathlineto{\pgfqpoint{1.076199in}{1.557320in}}%
\pgfpathlineto{\pgfqpoint{1.077064in}{1.613864in}}%
\pgfpathlineto{\pgfqpoint{1.077927in}{1.562425in}}%
\pgfpathlineto{\pgfqpoint{1.078792in}{1.615615in}}%
\pgfpathlineto{\pgfqpoint{1.079657in}{1.524729in}}%
\pgfpathlineto{\pgfqpoint{1.081384in}{1.599705in}}%
\pgfpathlineto{\pgfqpoint{1.082248in}{1.551322in}}%
\pgfpathlineto{\pgfqpoint{1.083114in}{1.601010in}}%
\pgfpathlineto{\pgfqpoint{1.083981in}{1.599675in}}%
\pgfpathlineto{\pgfqpoint{1.085711in}{1.559930in}}%
\pgfpathlineto{\pgfqpoint{1.086576in}{1.628763in}}%
\pgfpathlineto{\pgfqpoint{1.087440in}{1.601188in}}%
\pgfpathlineto{\pgfqpoint{1.088306in}{1.630039in}}%
\pgfpathlineto{\pgfqpoint{1.089172in}{1.609558in}}%
\pgfpathlineto{\pgfqpoint{1.090037in}{1.618226in}}%
\pgfpathlineto{\pgfqpoint{1.090903in}{1.575722in}}%
\pgfpathlineto{\pgfqpoint{1.091767in}{1.589732in}}%
\pgfpathlineto{\pgfqpoint{1.092632in}{1.551798in}}%
\pgfpathlineto{\pgfqpoint{1.093498in}{1.595845in}}%
\pgfpathlineto{\pgfqpoint{1.094362in}{1.519798in}}%
\pgfpathlineto{\pgfqpoint{1.095225in}{1.597150in}}%
\pgfpathlineto{\pgfqpoint{1.096090in}{1.559841in}}%
\pgfpathlineto{\pgfqpoint{1.097822in}{1.642506in}}%
\pgfpathlineto{\pgfqpoint{1.098687in}{1.634255in}}%
\pgfpathlineto{\pgfqpoint{1.099551in}{1.637877in}}%
\pgfpathlineto{\pgfqpoint{1.100416in}{1.591275in}}%
\pgfpathlineto{\pgfqpoint{1.101281in}{1.597388in}}%
\pgfpathlineto{\pgfqpoint{1.102145in}{1.596619in}}%
\pgfpathlineto{\pgfqpoint{1.103008in}{1.532980in}}%
\pgfpathlineto{\pgfqpoint{1.104736in}{1.582013in}}%
\pgfpathlineto{\pgfqpoint{1.105601in}{1.573584in}}%
\pgfpathlineto{\pgfqpoint{1.106466in}{1.581124in}}%
\pgfpathlineto{\pgfqpoint{1.107328in}{1.545090in}}%
\pgfpathlineto{\pgfqpoint{1.109055in}{1.576967in}}%
\pgfpathlineto{\pgfqpoint{1.109918in}{1.565333in}}%
\pgfpathlineto{\pgfqpoint{1.111647in}{1.584151in}}%
\pgfpathlineto{\pgfqpoint{1.112512in}{1.529715in}}%
\pgfpathlineto{\pgfqpoint{1.113375in}{1.581184in}}%
\pgfpathlineto{\pgfqpoint{1.114239in}{1.565690in}}%
\pgfpathlineto{\pgfqpoint{1.115102in}{1.637609in}}%
\pgfpathlineto{\pgfqpoint{1.115967in}{1.555896in}}%
\pgfpathlineto{\pgfqpoint{1.116831in}{1.653847in}}%
\pgfpathlineto{\pgfqpoint{1.117696in}{1.549307in}}%
\pgfpathlineto{\pgfqpoint{1.118561in}{1.574773in}}%
\pgfpathlineto{\pgfqpoint{1.119426in}{1.569936in}}%
\pgfpathlineto{\pgfqpoint{1.120289in}{1.573230in}}%
\pgfpathlineto{\pgfqpoint{1.122016in}{1.695254in}}%
\pgfpathlineto{\pgfqpoint{1.122881in}{1.659369in}}%
\pgfpathlineto{\pgfqpoint{1.124612in}{1.682668in}}%
\pgfpathlineto{\pgfqpoint{1.125476in}{1.655688in}}%
\pgfpathlineto{\pgfqpoint{1.126342in}{1.720751in}}%
\pgfpathlineto{\pgfqpoint{1.127208in}{1.646723in}}%
\pgfpathlineto{\pgfqpoint{1.128072in}{1.738974in}}%
\pgfpathlineto{\pgfqpoint{1.128936in}{1.629715in}}%
\pgfpathlineto{\pgfqpoint{1.129800in}{1.650106in}}%
\pgfpathlineto{\pgfqpoint{1.130665in}{1.677057in}}%
\pgfpathlineto{\pgfqpoint{1.132396in}{1.617158in}}%
\pgfpathlineto{\pgfqpoint{1.133261in}{1.670051in}}%
\pgfpathlineto{\pgfqpoint{1.134126in}{1.658566in}}%
\pgfpathlineto{\pgfqpoint{1.134991in}{1.665184in}}%
\pgfpathlineto{\pgfqpoint{1.136722in}{1.604159in}}%
\pgfpathlineto{\pgfqpoint{1.137586in}{1.624164in}}%
\pgfpathlineto{\pgfqpoint{1.138451in}{1.660525in}}%
\pgfpathlineto{\pgfqpoint{1.139316in}{1.643518in}}%
\pgfpathlineto{\pgfqpoint{1.140181in}{1.663701in}}%
\pgfpathlineto{\pgfqpoint{1.141044in}{1.645771in}}%
\pgfpathlineto{\pgfqpoint{1.141910in}{1.687505in}}%
\pgfpathlineto{\pgfqpoint{1.142776in}{1.577979in}}%
\pgfpathlineto{\pgfqpoint{1.143641in}{1.596946in}}%
\pgfpathlineto{\pgfqpoint{1.147100in}{1.737550in}}%
\pgfpathlineto{\pgfqpoint{1.148829in}{1.670825in}}%
\pgfpathlineto{\pgfqpoint{1.149692in}{1.703416in}}%
\pgfpathlineto{\pgfqpoint{1.150558in}{1.664087in}}%
\pgfpathlineto{\pgfqpoint{1.151424in}{1.682578in}}%
\pgfpathlineto{\pgfqpoint{1.152290in}{1.680738in}}%
\pgfpathlineto{\pgfqpoint{1.153155in}{1.634166in}}%
\pgfpathlineto{\pgfqpoint{1.155749in}{1.684419in}}%
\pgfpathlineto{\pgfqpoint{1.156614in}{1.665779in}}%
\pgfpathlineto{\pgfqpoint{1.157478in}{1.675960in}}%
\pgfpathlineto{\pgfqpoint{1.158343in}{1.653669in}}%
\pgfpathlineto{\pgfqpoint{1.159208in}{1.699556in}}%
\pgfpathlineto{\pgfqpoint{1.160073in}{1.631496in}}%
\pgfpathlineto{\pgfqpoint{1.160937in}{1.769665in}}%
\pgfpathlineto{\pgfqpoint{1.164394in}{1.589673in}}%
\pgfpathlineto{\pgfqpoint{1.165259in}{1.588070in}}%
\pgfpathlineto{\pgfqpoint{1.166124in}{1.592878in}}%
\pgfpathlineto{\pgfqpoint{1.166990in}{1.642774in}}%
\pgfpathlineto{\pgfqpoint{1.170449in}{1.526569in}}%
\pgfpathlineto{\pgfqpoint{1.172180in}{1.625528in}}%
\pgfpathlineto{\pgfqpoint{1.173045in}{1.610094in}}%
\pgfpathlineto{\pgfqpoint{1.173912in}{1.617277in}}%
\pgfpathlineto{\pgfqpoint{1.174778in}{1.558863in}}%
\pgfpathlineto{\pgfqpoint{1.176508in}{1.593473in}}%
\pgfpathlineto{\pgfqpoint{1.177374in}{1.607305in}}%
\pgfpathlineto{\pgfqpoint{1.178239in}{1.560882in}}%
\pgfpathlineto{\pgfqpoint{1.179969in}{1.589851in}}%
\pgfpathlineto{\pgfqpoint{1.180834in}{1.564325in}}%
\pgfpathlineto{\pgfqpoint{1.181699in}{1.618940in}}%
\pgfpathlineto{\pgfqpoint{1.182565in}{1.587122in}}%
\pgfpathlineto{\pgfqpoint{1.183430in}{1.608670in}}%
\pgfpathlineto{\pgfqpoint{1.186026in}{1.552482in}}%
\pgfpathlineto{\pgfqpoint{1.187758in}{1.605286in}}%
\pgfpathlineto{\pgfqpoint{1.189487in}{1.569192in}}%
\pgfpathlineto{\pgfqpoint{1.190352in}{1.610510in}}%
\pgfpathlineto{\pgfqpoint{1.191218in}{1.553906in}}%
\pgfpathlineto{\pgfqpoint{1.192948in}{1.677473in}}%
\pgfpathlineto{\pgfqpoint{1.193813in}{1.611046in}}%
\pgfpathlineto{\pgfqpoint{1.194678in}{1.669044in}}%
\pgfpathlineto{\pgfqpoint{1.195541in}{1.542272in}}%
\pgfpathlineto{\pgfqpoint{1.196406in}{1.629626in}}%
\pgfpathlineto{\pgfqpoint{1.198137in}{1.572933in}}%
\pgfpathlineto{\pgfqpoint{1.199001in}{1.648385in}}%
\pgfpathlineto{\pgfqpoint{1.199865in}{1.537315in}}%
\pgfpathlineto{\pgfqpoint{1.200729in}{1.593503in}}%
\pgfpathlineto{\pgfqpoint{1.201595in}{1.495819in}}%
\pgfpathlineto{\pgfqpoint{1.202459in}{1.610867in}}%
\pgfpathlineto{\pgfqpoint{1.203324in}{1.501400in}}%
\pgfpathlineto{\pgfqpoint{1.204189in}{1.568839in}}%
\pgfpathlineto{\pgfqpoint{1.205054in}{1.556996in}}%
\pgfpathlineto{\pgfqpoint{1.206785in}{1.599173in}}%
\pgfpathlineto{\pgfqpoint{1.209379in}{1.578038in}}%
\pgfpathlineto{\pgfqpoint{1.210246in}{1.499559in}}%
\pgfpathlineto{\pgfqpoint{1.211111in}{1.597094in}}%
\pgfpathlineto{\pgfqpoint{1.211977in}{1.579284in}}%
\pgfpathlineto{\pgfqpoint{1.212842in}{1.588130in}}%
\pgfpathlineto{\pgfqpoint{1.213706in}{1.620959in}}%
\pgfpathlineto{\pgfqpoint{1.214572in}{1.593205in}}%
\pgfpathlineto{\pgfqpoint{1.215438in}{1.665422in}}%
\pgfpathlineto{\pgfqpoint{1.218035in}{1.584746in}}%
\pgfpathlineto{\pgfqpoint{1.218902in}{1.684062in}}%
\pgfpathlineto{\pgfqpoint{1.219768in}{1.600181in}}%
\pgfpathlineto{\pgfqpoint{1.220634in}{1.612886in}}%
\pgfpathlineto{\pgfqpoint{1.222366in}{1.609146in}}%
\pgfpathlineto{\pgfqpoint{1.224098in}{1.615943in}}%
\pgfpathlineto{\pgfqpoint{1.224965in}{1.571271in}}%
\pgfpathlineto{\pgfqpoint{1.225831in}{1.607959in}}%
\pgfpathlineto{\pgfqpoint{1.227562in}{1.566701in}}%
\pgfpathlineto{\pgfqpoint{1.228427in}{1.595194in}}%
\pgfpathlineto{\pgfqpoint{1.229292in}{1.563288in}}%
\pgfpathlineto{\pgfqpoint{1.230156in}{1.585281in}}%
\pgfpathlineto{\pgfqpoint{1.231020in}{1.574654in}}%
\pgfpathlineto{\pgfqpoint{1.232751in}{1.587003in}}%
\pgfpathlineto{\pgfqpoint{1.233617in}{1.566552in}}%
\pgfpathlineto{\pgfqpoint{1.236211in}{1.618642in}}%
\pgfpathlineto{\pgfqpoint{1.237942in}{1.538144in}}%
\pgfpathlineto{\pgfqpoint{1.238807in}{1.618702in}}%
\pgfpathlineto{\pgfqpoint{1.239671in}{1.608432in}}%
\pgfpathlineto{\pgfqpoint{1.240537in}{1.636334in}}%
\pgfpathlineto{\pgfqpoint{1.241402in}{1.548950in}}%
\pgfpathlineto{\pgfqpoint{1.242268in}{1.557558in}}%
\pgfpathlineto{\pgfqpoint{1.243133in}{1.555539in}}%
\pgfpathlineto{\pgfqpoint{1.244864in}{1.590208in}}%
\pgfpathlineto{\pgfqpoint{1.245728in}{1.550880in}}%
\pgfpathlineto{\pgfqpoint{1.246593in}{1.602021in}}%
\pgfpathlineto{\pgfqpoint{1.247458in}{1.548771in}}%
\pgfpathlineto{\pgfqpoint{1.249187in}{1.609737in}}%
\pgfpathlineto{\pgfqpoint{1.250053in}{1.606532in}}%
\pgfpathlineto{\pgfqpoint{1.250916in}{1.563076in}}%
\pgfpathlineto{\pgfqpoint{1.251781in}{1.611815in}}%
\pgfpathlineto{\pgfqpoint{1.252647in}{1.605286in}}%
\pgfpathlineto{\pgfqpoint{1.253512in}{1.626064in}}%
\pgfpathlineto{\pgfqpoint{1.255243in}{1.576198in}}%
\pgfpathlineto{\pgfqpoint{1.256974in}{1.600002in}}%
\pgfpathlineto{\pgfqpoint{1.257840in}{1.595934in}}%
\pgfpathlineto{\pgfqpoint{1.258706in}{1.567114in}}%
\pgfpathlineto{\pgfqpoint{1.259571in}{1.619416in}}%
\pgfpathlineto{\pgfqpoint{1.261299in}{1.575841in}}%
\pgfpathlineto{\pgfqpoint{1.262165in}{1.584538in}}%
\pgfpathlineto{\pgfqpoint{1.263030in}{1.581124in}}%
\pgfpathlineto{\pgfqpoint{1.263893in}{1.550612in}}%
\pgfpathlineto{\pgfqpoint{1.264759in}{1.572784in}}%
\pgfpathlineto{\pgfqpoint{1.265625in}{1.632474in}}%
\pgfpathlineto{\pgfqpoint{1.267353in}{1.551679in}}%
\pgfpathlineto{\pgfqpoint{1.268216in}{1.559160in}}%
\pgfpathlineto{\pgfqpoint{1.269081in}{1.598102in}}%
\pgfpathlineto{\pgfqpoint{1.269946in}{1.596262in}}%
\pgfpathlineto{\pgfqpoint{1.270811in}{1.610034in}}%
\pgfpathlineto{\pgfqpoint{1.272543in}{1.531496in}}%
\pgfpathlineto{\pgfqpoint{1.273408in}{1.585162in}}%
\pgfpathlineto{\pgfqpoint{1.274272in}{1.572903in}}%
\pgfpathlineto{\pgfqpoint{1.275137in}{1.591870in}}%
\pgfpathlineto{\pgfqpoint{1.276003in}{1.576614in}}%
\pgfpathlineto{\pgfqpoint{1.276869in}{1.533247in}}%
\pgfpathlineto{\pgfqpoint{1.277734in}{1.743841in}}%
\pgfpathlineto{\pgfqpoint{1.278600in}{1.564028in}}%
\pgfpathlineto{\pgfqpoint{1.279465in}{1.601843in}}%
\pgfpathlineto{\pgfqpoint{1.280331in}{1.626123in}}%
\pgfpathlineto{\pgfqpoint{1.282062in}{1.598043in}}%
\pgfpathlineto{\pgfqpoint{1.282927in}{1.678897in}}%
\pgfpathlineto{\pgfqpoint{1.283793in}{1.586587in}}%
\pgfpathlineto{\pgfqpoint{1.284658in}{1.650757in}}%
\pgfpathlineto{\pgfqpoint{1.285521in}{1.558684in}}%
\pgfpathlineto{\pgfqpoint{1.287250in}{1.614072in}}%
\pgfpathlineto{\pgfqpoint{1.288114in}{1.622383in}}%
\pgfpathlineto{\pgfqpoint{1.290711in}{1.713686in}}%
\pgfpathlineto{\pgfqpoint{1.291576in}{1.660793in}}%
\pgfpathlineto{\pgfqpoint{1.292440in}{1.752985in}}%
\pgfpathlineto{\pgfqpoint{1.293306in}{1.742923in}}%
\pgfpathlineto{\pgfqpoint{1.294172in}{1.621316in}}%
\pgfpathlineto{\pgfqpoint{1.295904in}{1.697835in}}%
\pgfpathlineto{\pgfqpoint{1.296768in}{1.682995in}}%
\pgfpathlineto{\pgfqpoint{1.297633in}{1.572104in}}%
\pgfpathlineto{\pgfqpoint{1.298498in}{1.591573in}}%
\pgfpathlineto{\pgfqpoint{1.299363in}{1.642863in}}%
\pgfpathlineto{\pgfqpoint{1.301092in}{1.619118in}}%
\pgfpathlineto{\pgfqpoint{1.301958in}{1.625885in}}%
\pgfpathlineto{\pgfqpoint{1.305417in}{1.706472in}}%
\pgfpathlineto{\pgfqpoint{1.306282in}{1.703475in}}%
\pgfpathlineto{\pgfqpoint{1.307147in}{1.629983in}}%
\pgfpathlineto{\pgfqpoint{1.308879in}{1.746098in}}%
\pgfpathlineto{\pgfqpoint{1.309745in}{1.713597in}}%
\pgfpathlineto{\pgfqpoint{1.310610in}{1.739034in}}%
\pgfpathlineto{\pgfqpoint{1.312342in}{1.708343in}}%
\pgfpathlineto{\pgfqpoint{1.313208in}{1.716296in}}%
\pgfpathlineto{\pgfqpoint{1.314074in}{1.662039in}}%
\pgfpathlineto{\pgfqpoint{1.314939in}{1.666698in}}%
\pgfpathlineto{\pgfqpoint{1.315803in}{1.731433in}}%
\pgfpathlineto{\pgfqpoint{1.317529in}{1.640312in}}%
\pgfpathlineto{\pgfqpoint{1.318395in}{1.672368in}}%
\pgfpathlineto{\pgfqpoint{1.320124in}{1.621966in}}%
\pgfpathlineto{\pgfqpoint{1.320989in}{1.730188in}}%
\pgfpathlineto{\pgfqpoint{1.321854in}{1.709975in}}%
\pgfpathlineto{\pgfqpoint{1.322720in}{1.596678in}}%
\pgfpathlineto{\pgfqpoint{1.325313in}{1.718434in}}%
\pgfpathlineto{\pgfqpoint{1.326179in}{1.629329in}}%
\pgfpathlineto{\pgfqpoint{1.327908in}{1.685843in}}%
\pgfpathlineto{\pgfqpoint{1.328771in}{1.683556in}}%
\pgfpathlineto{\pgfqpoint{1.329636in}{1.685843in}}%
\pgfpathlineto{\pgfqpoint{1.330500in}{1.728466in}}%
\pgfpathlineto{\pgfqpoint{1.331365in}{1.699051in}}%
\pgfpathlineto{\pgfqpoint{1.332230in}{1.726507in}}%
\pgfpathlineto{\pgfqpoint{1.333954in}{1.694659in}}%
\pgfpathlineto{\pgfqpoint{1.335684in}{1.699199in}}%
\pgfpathlineto{\pgfqpoint{1.336550in}{1.646187in}}%
\pgfpathlineto{\pgfqpoint{1.337413in}{1.728347in}}%
\pgfpathlineto{\pgfqpoint{1.338278in}{1.660049in}}%
\pgfpathlineto{\pgfqpoint{1.340007in}{1.704781in}}%
\pgfpathlineto{\pgfqpoint{1.342599in}{1.546812in}}%
\pgfpathlineto{\pgfqpoint{1.343462in}{1.584538in}}%
\pgfpathlineto{\pgfqpoint{1.344329in}{1.555658in}}%
\pgfpathlineto{\pgfqpoint{1.345195in}{1.573171in}}%
\pgfpathlineto{\pgfqpoint{1.346061in}{1.543904in}}%
\pgfpathlineto{\pgfqpoint{1.346925in}{1.567828in}}%
\pgfpathlineto{\pgfqpoint{1.347786in}{1.548920in}}%
\pgfpathlineto{\pgfqpoint{1.348652in}{1.630872in}}%
\pgfpathlineto{\pgfqpoint{1.349515in}{1.562544in}}%
\pgfpathlineto{\pgfqpoint{1.351245in}{1.600121in}}%
\pgfpathlineto{\pgfqpoint{1.352107in}{1.545150in}}%
\pgfpathlineto{\pgfqpoint{1.352972in}{1.548474in}}%
\pgfpathlineto{\pgfqpoint{1.353837in}{1.554352in}}%
\pgfpathlineto{\pgfqpoint{1.354702in}{1.620483in}}%
\pgfpathlineto{\pgfqpoint{1.355566in}{1.578098in}}%
\pgfpathlineto{\pgfqpoint{1.356431in}{1.624878in}}%
\pgfpathlineto{\pgfqpoint{1.359028in}{1.559398in}}%
\pgfpathlineto{\pgfqpoint{1.359893in}{1.610778in}}%
\pgfpathlineto{\pgfqpoint{1.360758in}{1.581779in}}%
\pgfpathlineto{\pgfqpoint{1.361622in}{1.627488in}}%
\pgfpathlineto{\pgfqpoint{1.362488in}{1.572308in}}%
\pgfpathlineto{\pgfqpoint{1.365081in}{1.640904in}}%
\pgfpathlineto{\pgfqpoint{1.365947in}{1.616623in}}%
\pgfpathlineto{\pgfqpoint{1.366813in}{1.644049in}}%
\pgfpathlineto{\pgfqpoint{1.367678in}{1.567292in}}%
\pgfpathlineto{\pgfqpoint{1.368543in}{1.571211in}}%
\pgfpathlineto{\pgfqpoint{1.369408in}{1.577384in}}%
\pgfpathlineto{\pgfqpoint{1.370273in}{1.562603in}}%
\pgfpathlineto{\pgfqpoint{1.371138in}{1.595075in}}%
\pgfpathlineto{\pgfqpoint{1.372002in}{1.586914in}}%
\pgfpathlineto{\pgfqpoint{1.372866in}{1.555836in}}%
\pgfpathlineto{\pgfqpoint{1.374594in}{1.644287in}}%
\pgfpathlineto{\pgfqpoint{1.375459in}{1.558744in}}%
\pgfpathlineto{\pgfqpoint{1.377187in}{1.615794in}}%
\pgfpathlineto{\pgfqpoint{1.378052in}{1.586468in}}%
\pgfpathlineto{\pgfqpoint{1.378916in}{1.511075in}}%
\pgfpathlineto{\pgfqpoint{1.380646in}{1.578395in}}%
\pgfpathlineto{\pgfqpoint{1.381509in}{1.562782in}}%
\pgfpathlineto{\pgfqpoint{1.383238in}{1.580946in}}%
\pgfpathlineto{\pgfqpoint{1.384104in}{1.522650in}}%
\pgfpathlineto{\pgfqpoint{1.384968in}{1.536601in}}%
\pgfpathlineto{\pgfqpoint{1.387562in}{1.629031in}}%
\pgfpathlineto{\pgfqpoint{1.388427in}{1.628912in}}%
\pgfpathlineto{\pgfqpoint{1.390156in}{1.572992in}}%
\pgfpathlineto{\pgfqpoint{1.391021in}{1.565184in}}%
\pgfpathlineto{\pgfqpoint{1.391886in}{1.519564in}}%
\pgfpathlineto{\pgfqpoint{1.392749in}{1.591751in}}%
\pgfpathlineto{\pgfqpoint{1.393613in}{1.589732in}}%
\pgfpathlineto{\pgfqpoint{1.395342in}{1.589821in}}%
\pgfpathlineto{\pgfqpoint{1.396208in}{1.603743in}}%
\pgfpathlineto{\pgfqpoint{1.397073in}{1.701456in}}%
\pgfpathlineto{\pgfqpoint{1.397938in}{1.679373in}}%
\pgfpathlineto{\pgfqpoint{1.398803in}{1.709588in}}%
\pgfpathlineto{\pgfqpoint{1.400534in}{1.636036in}}%
\pgfpathlineto{\pgfqpoint{1.401400in}{1.703118in}}%
\pgfpathlineto{\pgfqpoint{1.402265in}{1.662039in}}%
\pgfpathlineto{\pgfqpoint{1.403131in}{1.725499in}}%
\pgfpathlineto{\pgfqpoint{1.403997in}{1.699021in}}%
\pgfpathlineto{\pgfqpoint{1.405729in}{1.722885in}}%
\pgfpathlineto{\pgfqpoint{1.407460in}{1.677741in}}%
\pgfpathlineto{\pgfqpoint{1.408324in}{1.732564in}}%
\pgfpathlineto{\pgfqpoint{1.410055in}{1.677295in}}%
\pgfpathlineto{\pgfqpoint{1.410921in}{1.725677in}}%
\pgfpathlineto{\pgfqpoint{1.411785in}{1.691067in}}%
\pgfpathlineto{\pgfqpoint{1.413515in}{1.736007in}}%
\pgfpathlineto{\pgfqpoint{1.414379in}{1.735144in}}%
\pgfpathlineto{\pgfqpoint{1.416108in}{1.693205in}}%
\pgfpathlineto{\pgfqpoint{1.416972in}{1.708819in}}%
\pgfpathlineto{\pgfqpoint{1.417838in}{1.653788in}}%
\pgfpathlineto{\pgfqpoint{1.418701in}{1.725856in}}%
\pgfpathlineto{\pgfqpoint{1.419566in}{1.643339in}}%
\pgfpathlineto{\pgfqpoint{1.420430in}{1.684062in}}%
\pgfpathlineto{\pgfqpoint{1.421294in}{1.639123in}}%
\pgfpathlineto{\pgfqpoint{1.422159in}{1.661384in}}%
\pgfpathlineto{\pgfqpoint{1.423024in}{1.657290in}}%
\pgfpathlineto{\pgfqpoint{1.424749in}{1.677652in}}%
\pgfpathlineto{\pgfqpoint{1.425614in}{1.627607in}}%
\pgfpathlineto{\pgfqpoint{1.427345in}{1.761295in}}%
\pgfpathlineto{\pgfqpoint{1.429937in}{1.596619in}}%
\pgfpathlineto{\pgfqpoint{1.431667in}{1.680738in}}%
\pgfpathlineto{\pgfqpoint{1.432532in}{1.670408in}}%
\pgfpathlineto{\pgfqpoint{1.433398in}{1.694659in}}%
\pgfpathlineto{\pgfqpoint{1.434265in}{1.644763in}}%
\pgfpathlineto{\pgfqpoint{1.435130in}{1.656695in}}%
\pgfpathlineto{\pgfqpoint{1.435996in}{1.673584in}}%
\pgfpathlineto{\pgfqpoint{1.436861in}{1.668746in}}%
\pgfpathlineto{\pgfqpoint{1.438592in}{1.680678in}}%
\pgfpathlineto{\pgfqpoint{1.439455in}{1.660079in}}%
\pgfpathlineto{\pgfqpoint{1.441185in}{1.691008in}}%
\pgfpathlineto{\pgfqpoint{1.442050in}{1.681035in}}%
\pgfpathlineto{\pgfqpoint{1.442915in}{1.610718in}}%
\pgfpathlineto{\pgfqpoint{1.443778in}{1.717308in}}%
\pgfpathlineto{\pgfqpoint{1.444643in}{1.674803in}}%
\pgfpathlineto{\pgfqpoint{1.445508in}{1.698578in}}%
\pgfpathlineto{\pgfqpoint{1.446371in}{1.696768in}}%
\pgfpathlineto{\pgfqpoint{1.448967in}{1.628499in}}%
\pgfpathlineto{\pgfqpoint{1.449831in}{1.705911in}}%
\pgfpathlineto{\pgfqpoint{1.450696in}{1.642629in}}%
\pgfpathlineto{\pgfqpoint{1.451559in}{1.697065in}}%
\pgfpathlineto{\pgfqpoint{1.452425in}{1.659696in}}%
\pgfpathlineto{\pgfqpoint{1.453289in}{1.662990in}}%
\pgfpathlineto{\pgfqpoint{1.455884in}{1.717724in}}%
\pgfpathlineto{\pgfqpoint{1.456749in}{1.695105in}}%
\pgfpathlineto{\pgfqpoint{1.457614in}{1.635237in}}%
\pgfpathlineto{\pgfqpoint{1.458480in}{1.651653in}}%
\pgfpathlineto{\pgfqpoint{1.459345in}{1.688933in}}%
\pgfpathlineto{\pgfqpoint{1.461941in}{1.621405in}}%
\pgfpathlineto{\pgfqpoint{1.462806in}{1.621435in}}%
\pgfpathlineto{\pgfqpoint{1.464538in}{1.688665in}}%
\pgfpathlineto{\pgfqpoint{1.467134in}{1.621583in}}%
\pgfpathlineto{\pgfqpoint{1.467999in}{1.679611in}}%
\pgfpathlineto{\pgfqpoint{1.468864in}{1.666017in}}%
\pgfpathlineto{\pgfqpoint{1.469729in}{1.655866in}}%
\pgfpathlineto{\pgfqpoint{1.470594in}{1.676942in}}%
\pgfpathlineto{\pgfqpoint{1.472326in}{1.637940in}}%
\pgfpathlineto{\pgfqpoint{1.474056in}{1.664801in}}%
\pgfpathlineto{\pgfqpoint{1.474922in}{1.633843in}}%
\pgfpathlineto{\pgfqpoint{1.475788in}{1.705851in}}%
\pgfpathlineto{\pgfqpoint{1.476654in}{1.692346in}}%
\pgfpathlineto{\pgfqpoint{1.477520in}{1.665009in}}%
\pgfpathlineto{\pgfqpoint{1.478386in}{1.711905in}}%
\pgfpathlineto{\pgfqpoint{1.479253in}{1.685073in}}%
\pgfpathlineto{\pgfqpoint{1.480117in}{1.743310in}}%
\pgfpathlineto{\pgfqpoint{1.481847in}{1.638769in}}%
\pgfpathlineto{\pgfqpoint{1.482711in}{1.696887in}}%
\pgfpathlineto{\pgfqpoint{1.483575in}{1.629983in}}%
\pgfpathlineto{\pgfqpoint{1.484441in}{1.638948in}}%
\pgfpathlineto{\pgfqpoint{1.485307in}{1.652780in}}%
\pgfpathlineto{\pgfqpoint{1.487905in}{1.591930in}}%
\pgfpathlineto{\pgfqpoint{1.488770in}{1.530131in}}%
\pgfpathlineto{\pgfqpoint{1.489635in}{1.625885in}}%
\pgfpathlineto{\pgfqpoint{1.490500in}{1.584835in}}%
\pgfpathlineto{\pgfqpoint{1.491365in}{1.585400in}}%
\pgfpathlineto{\pgfqpoint{1.492231in}{1.588725in}}%
\pgfpathlineto{\pgfqpoint{1.493097in}{1.558655in}}%
\pgfpathlineto{\pgfqpoint{1.493962in}{1.569668in}}%
\pgfpathlineto{\pgfqpoint{1.495693in}{1.537285in}}%
\pgfpathlineto{\pgfqpoint{1.497424in}{1.626064in}}%
\pgfpathlineto{\pgfqpoint{1.499153in}{1.588963in}}%
\pgfpathlineto{\pgfqpoint{1.500017in}{1.640818in}}%
\pgfpathlineto{\pgfqpoint{1.500882in}{1.518675in}}%
\pgfpathlineto{\pgfqpoint{1.502609in}{1.582995in}}%
\pgfpathlineto{\pgfqpoint{1.503473in}{1.576495in}}%
\pgfpathlineto{\pgfqpoint{1.504339in}{1.578514in}}%
\pgfpathlineto{\pgfqpoint{1.505204in}{1.566552in}}%
\pgfpathlineto{\pgfqpoint{1.506068in}{1.599292in}}%
\pgfpathlineto{\pgfqpoint{1.506933in}{1.547645in}}%
\pgfpathlineto{\pgfqpoint{1.507798in}{1.560350in}}%
\pgfpathlineto{\pgfqpoint{1.508662in}{1.660793in}}%
\pgfpathlineto{\pgfqpoint{1.509524in}{1.541026in}}%
\pgfpathlineto{\pgfqpoint{1.511254in}{1.632240in}}%
\pgfpathlineto{\pgfqpoint{1.512119in}{1.625592in}}%
\pgfpathlineto{\pgfqpoint{1.512984in}{1.610335in}}%
\pgfpathlineto{\pgfqpoint{1.513848in}{1.675279in}}%
\pgfpathlineto{\pgfqpoint{1.514713in}{1.557145in}}%
\pgfpathlineto{\pgfqpoint{1.515579in}{1.565158in}}%
\pgfpathlineto{\pgfqpoint{1.516443in}{1.587360in}}%
\pgfpathlineto{\pgfqpoint{1.519904in}{1.533753in}}%
\pgfpathlineto{\pgfqpoint{1.520767in}{1.586051in}}%
\pgfpathlineto{\pgfqpoint{1.521632in}{1.545298in}}%
\pgfpathlineto{\pgfqpoint{1.522496in}{1.549009in}}%
\pgfpathlineto{\pgfqpoint{1.523360in}{1.543190in}}%
\pgfpathlineto{\pgfqpoint{1.525956in}{1.640312in}}%
\pgfpathlineto{\pgfqpoint{1.526821in}{1.559458in}}%
\pgfpathlineto{\pgfqpoint{1.527685in}{1.579879in}}%
\pgfpathlineto{\pgfqpoint{1.528549in}{1.567828in}}%
\pgfpathlineto{\pgfqpoint{1.529411in}{1.577562in}}%
\pgfpathlineto{\pgfqpoint{1.532005in}{1.691246in}}%
\pgfpathlineto{\pgfqpoint{1.532870in}{1.615318in}}%
\pgfpathlineto{\pgfqpoint{1.533734in}{1.664798in}}%
\pgfpathlineto{\pgfqpoint{1.534599in}{1.581481in}}%
\pgfpathlineto{\pgfqpoint{1.536326in}{1.641320in}}%
\pgfpathlineto{\pgfqpoint{1.537190in}{1.610689in}}%
\pgfpathlineto{\pgfqpoint{1.538055in}{1.509294in}}%
\pgfpathlineto{\pgfqpoint{1.539786in}{1.627726in}}%
\pgfpathlineto{\pgfqpoint{1.540652in}{1.522710in}}%
\pgfpathlineto{\pgfqpoint{1.542382in}{1.608670in}}%
\pgfpathlineto{\pgfqpoint{1.543246in}{1.514429in}}%
\pgfpathlineto{\pgfqpoint{1.544111in}{1.623688in}}%
\pgfpathlineto{\pgfqpoint{1.544975in}{1.603029in}}%
\pgfpathlineto{\pgfqpoint{1.545841in}{1.584389in}}%
\pgfpathlineto{\pgfqpoint{1.546705in}{1.612053in}}%
\pgfpathlineto{\pgfqpoint{1.548435in}{1.560525in}}%
\pgfpathlineto{\pgfqpoint{1.549300in}{1.577860in}}%
\pgfpathlineto{\pgfqpoint{1.550163in}{1.543190in}}%
\pgfpathlineto{\pgfqpoint{1.551028in}{1.586527in}}%
\pgfpathlineto{\pgfqpoint{1.552758in}{1.567976in}}%
\pgfpathlineto{\pgfqpoint{1.553624in}{1.557855in}}%
\pgfpathlineto{\pgfqpoint{1.554488in}{1.583500in}}%
\pgfpathlineto{\pgfqpoint{1.555353in}{1.576019in}}%
\pgfpathlineto{\pgfqpoint{1.556218in}{1.646247in}}%
\pgfpathlineto{\pgfqpoint{1.557083in}{1.555509in}}%
\pgfpathlineto{\pgfqpoint{1.557945in}{1.591751in}}%
\pgfpathlineto{\pgfqpoint{1.559676in}{1.532890in}}%
\pgfpathlineto{\pgfqpoint{1.560542in}{1.608253in}}%
\pgfpathlineto{\pgfqpoint{1.561408in}{1.554055in}}%
\pgfpathlineto{\pgfqpoint{1.562273in}{1.575900in}}%
\pgfpathlineto{\pgfqpoint{1.563139in}{1.567292in}}%
\pgfpathlineto{\pgfqpoint{1.564005in}{1.639123in}}%
\pgfpathlineto{\pgfqpoint{1.565738in}{1.545150in}}%
\pgfpathlineto{\pgfqpoint{1.566603in}{1.562544in}}%
\pgfpathlineto{\pgfqpoint{1.567467in}{1.541707in}}%
\pgfpathlineto{\pgfqpoint{1.568332in}{1.547641in}}%
\pgfpathlineto{\pgfqpoint{1.569196in}{1.612763in}}%
\pgfpathlineto{\pgfqpoint{1.570061in}{1.571981in}}%
\pgfpathlineto{\pgfqpoint{1.572656in}{1.644525in}}%
\pgfpathlineto{\pgfqpoint{1.574383in}{1.536601in}}%
\pgfpathlineto{\pgfqpoint{1.575249in}{1.632772in}}%
\pgfpathlineto{\pgfqpoint{1.576114in}{1.569728in}}%
\pgfpathlineto{\pgfqpoint{1.576978in}{1.598459in}}%
\pgfpathlineto{\pgfqpoint{1.578706in}{1.552304in}}%
\pgfpathlineto{\pgfqpoint{1.579570in}{1.677354in}}%
\pgfpathlineto{\pgfqpoint{1.580437in}{1.575781in}}%
\pgfpathlineto{\pgfqpoint{1.581303in}{1.658060in}}%
\pgfpathlineto{\pgfqpoint{1.582170in}{1.655271in}}%
\pgfpathlineto{\pgfqpoint{1.583903in}{1.571271in}}%
\pgfpathlineto{\pgfqpoint{1.585634in}{1.555955in}}%
\pgfpathlineto{\pgfqpoint{1.586500in}{1.592227in}}%
\pgfpathlineto{\pgfqpoint{1.587366in}{1.545507in}}%
\pgfpathlineto{\pgfqpoint{1.589098in}{1.632772in}}%
\pgfpathlineto{\pgfqpoint{1.590828in}{1.576079in}}%
\pgfpathlineto{\pgfqpoint{1.591692in}{1.610927in}}%
\pgfpathlineto{\pgfqpoint{1.593423in}{1.541528in}}%
\pgfpathlineto{\pgfqpoint{1.594289in}{1.543309in}}%
\pgfpathlineto{\pgfqpoint{1.595155in}{1.572397in}}%
\pgfpathlineto{\pgfqpoint{1.596886in}{1.526867in}}%
\pgfpathlineto{\pgfqpoint{1.598615in}{1.637639in}}%
\pgfpathlineto{\pgfqpoint{1.599480in}{1.527160in}}%
\pgfpathlineto{\pgfqpoint{1.601207in}{1.591275in}}%
\pgfpathlineto{\pgfqpoint{1.602071in}{1.587892in}}%
\pgfpathlineto{\pgfqpoint{1.603803in}{1.525677in}}%
\pgfpathlineto{\pgfqpoint{1.604669in}{1.598072in}}%
\pgfpathlineto{\pgfqpoint{1.605535in}{1.575424in}}%
\pgfpathlineto{\pgfqpoint{1.608130in}{1.627369in}}%
\pgfpathlineto{\pgfqpoint{1.609860in}{1.517396in}}%
\pgfpathlineto{\pgfqpoint{1.610725in}{1.581243in}}%
\pgfpathlineto{\pgfqpoint{1.611590in}{1.564266in}}%
\pgfpathlineto{\pgfqpoint{1.614186in}{1.622944in}}%
\pgfpathlineto{\pgfqpoint{1.615051in}{1.556487in}}%
\pgfpathlineto{\pgfqpoint{1.615917in}{1.644049in}}%
\pgfpathlineto{\pgfqpoint{1.616782in}{1.597596in}}%
\pgfpathlineto{\pgfqpoint{1.617645in}{1.662987in}}%
\pgfpathlineto{\pgfqpoint{1.618510in}{1.611637in}}%
\pgfpathlineto{\pgfqpoint{1.619375in}{1.628704in}}%
\pgfpathlineto{\pgfqpoint{1.620240in}{1.622026in}}%
\pgfpathlineto{\pgfqpoint{1.621104in}{1.632415in}}%
\pgfpathlineto{\pgfqpoint{1.621969in}{1.581184in}}%
\pgfpathlineto{\pgfqpoint{1.622835in}{1.638587in}}%
\pgfpathlineto{\pgfqpoint{1.623700in}{1.544376in}}%
\pgfpathlineto{\pgfqpoint{1.625428in}{1.634196in}}%
\pgfpathlineto{\pgfqpoint{1.626293in}{1.584924in}}%
\pgfpathlineto{\pgfqpoint{1.628886in}{1.657971in}}%
\pgfpathlineto{\pgfqpoint{1.629752in}{1.593711in}}%
\pgfpathlineto{\pgfqpoint{1.630617in}{1.641320in}}%
\pgfpathlineto{\pgfqpoint{1.631481in}{1.631110in}}%
\pgfpathlineto{\pgfqpoint{1.632344in}{1.589970in}}%
\pgfpathlineto{\pgfqpoint{1.633210in}{1.608759in}}%
\pgfpathlineto{\pgfqpoint{1.634073in}{1.607126in}}%
\pgfpathlineto{\pgfqpoint{1.634936in}{1.615556in}}%
\pgfpathlineto{\pgfqpoint{1.635803in}{1.614281in}}%
\pgfpathlineto{\pgfqpoint{1.636668in}{1.590327in}}%
\pgfpathlineto{\pgfqpoint{1.639263in}{1.668627in}}%
\pgfpathlineto{\pgfqpoint{1.642725in}{1.562931in}}%
\pgfpathlineto{\pgfqpoint{1.644456in}{1.626123in}}%
\pgfpathlineto{\pgfqpoint{1.645321in}{1.586319in}}%
\pgfpathlineto{\pgfqpoint{1.647053in}{1.657822in}}%
\pgfpathlineto{\pgfqpoint{1.647918in}{1.571330in}}%
\pgfpathlineto{\pgfqpoint{1.648784in}{1.588487in}}%
\pgfpathlineto{\pgfqpoint{1.650509in}{1.609205in}}%
\pgfpathlineto{\pgfqpoint{1.651372in}{1.570973in}}%
\pgfpathlineto{\pgfqpoint{1.652236in}{1.579641in}}%
\pgfpathlineto{\pgfqpoint{1.653101in}{1.655981in}}%
\pgfpathlineto{\pgfqpoint{1.653965in}{1.608551in}}%
\pgfpathlineto{\pgfqpoint{1.654830in}{1.617456in}}%
\pgfpathlineto{\pgfqpoint{1.655694in}{1.664827in}}%
\pgfpathlineto{\pgfqpoint{1.657422in}{1.577830in}}%
\pgfpathlineto{\pgfqpoint{1.658286in}{1.591751in}}%
\pgfpathlineto{\pgfqpoint{1.659151in}{1.596975in}}%
\pgfpathlineto{\pgfqpoint{1.660879in}{1.554527in}}%
\pgfpathlineto{\pgfqpoint{1.661745in}{1.606978in}}%
\pgfpathlineto{\pgfqpoint{1.662610in}{1.560346in}}%
\pgfpathlineto{\pgfqpoint{1.663476in}{1.575781in}}%
\pgfpathlineto{\pgfqpoint{1.664342in}{1.562276in}}%
\pgfpathlineto{\pgfqpoint{1.666070in}{1.618404in}}%
\pgfpathlineto{\pgfqpoint{1.667800in}{1.530961in}}%
\pgfpathlineto{\pgfqpoint{1.668666in}{1.579224in}}%
\pgfpathlineto{\pgfqpoint{1.670393in}{1.526331in}}%
\pgfpathlineto{\pgfqpoint{1.671259in}{1.538144in}}%
\pgfpathlineto{\pgfqpoint{1.672989in}{1.620959in}}%
\pgfpathlineto{\pgfqpoint{1.674716in}{1.602200in}}%
\pgfpathlineto{\pgfqpoint{1.675581in}{1.570438in}}%
\pgfpathlineto{\pgfqpoint{1.676445in}{1.640487in}}%
\pgfpathlineto{\pgfqpoint{1.678175in}{1.584627in}}%
\pgfpathlineto{\pgfqpoint{1.679039in}{1.623866in}}%
\pgfpathlineto{\pgfqpoint{1.679905in}{1.553877in}}%
\pgfpathlineto{\pgfqpoint{1.680770in}{1.560436in}}%
\pgfpathlineto{\pgfqpoint{1.681635in}{1.651174in}}%
\pgfpathlineto{\pgfqpoint{1.683364in}{1.593562in}}%
\pgfpathlineto{\pgfqpoint{1.684229in}{1.696113in}}%
\pgfpathlineto{\pgfqpoint{1.685094in}{1.560644in}}%
\pgfpathlineto{\pgfqpoint{1.685959in}{1.649839in}}%
\pgfpathlineto{\pgfqpoint{1.686825in}{1.543666in}}%
\pgfpathlineto{\pgfqpoint{1.688554in}{1.591037in}}%
\pgfpathlineto{\pgfqpoint{1.689420in}{1.575662in}}%
\pgfpathlineto{\pgfqpoint{1.690285in}{1.516121in}}%
\pgfpathlineto{\pgfqpoint{1.692015in}{1.600300in}}%
\pgfpathlineto{\pgfqpoint{1.692880in}{1.493235in}}%
\pgfpathlineto{\pgfqpoint{1.693742in}{1.603858in}}%
\pgfpathlineto{\pgfqpoint{1.695472in}{1.550047in}}%
\pgfpathlineto{\pgfqpoint{1.696337in}{1.633660in}}%
\pgfpathlineto{\pgfqpoint{1.698064in}{1.570378in}}%
\pgfpathlineto{\pgfqpoint{1.698929in}{1.568066in}}%
\pgfpathlineto{\pgfqpoint{1.700660in}{1.622978in}}%
\pgfpathlineto{\pgfqpoint{1.702389in}{1.528380in}}%
\pgfpathlineto{\pgfqpoint{1.703252in}{1.574476in}}%
\pgfpathlineto{\pgfqpoint{1.704981in}{1.498548in}}%
\pgfpathlineto{\pgfqpoint{1.706709in}{1.615794in}}%
\pgfpathlineto{\pgfqpoint{1.707574in}{1.566995in}}%
\pgfpathlineto{\pgfqpoint{1.708439in}{1.622204in}}%
\pgfpathlineto{\pgfqpoint{1.709304in}{1.599972in}}%
\pgfpathlineto{\pgfqpoint{1.710169in}{1.613299in}}%
\pgfpathlineto{\pgfqpoint{1.711035in}{1.649452in}}%
\pgfpathlineto{\pgfqpoint{1.711900in}{1.632474in}}%
\pgfpathlineto{\pgfqpoint{1.712763in}{1.737372in}}%
\pgfpathlineto{\pgfqpoint{1.713627in}{1.640074in}}%
\pgfpathlineto{\pgfqpoint{1.714491in}{1.785933in}}%
\pgfpathlineto{\pgfqpoint{1.715355in}{1.712202in}}%
\pgfpathlineto{\pgfqpoint{1.716220in}{1.715467in}}%
\pgfpathlineto{\pgfqpoint{1.717949in}{1.652007in}}%
\pgfpathlineto{\pgfqpoint{1.718815in}{1.681154in}}%
\pgfpathlineto{\pgfqpoint{1.720545in}{1.656695in}}%
\pgfpathlineto{\pgfqpoint{1.723138in}{1.679314in}}%
\pgfpathlineto{\pgfqpoint{1.724003in}{1.672368in}}%
\pgfpathlineto{\pgfqpoint{1.725732in}{1.752568in}}%
\pgfpathlineto{\pgfqpoint{1.726596in}{1.652661in}}%
\pgfpathlineto{\pgfqpoint{1.727461in}{1.718613in}}%
\pgfpathlineto{\pgfqpoint{1.729191in}{1.622442in}}%
\pgfpathlineto{\pgfqpoint{1.730056in}{1.687327in}}%
\pgfpathlineto{\pgfqpoint{1.730922in}{1.654825in}}%
\pgfpathlineto{\pgfqpoint{1.732650in}{1.704543in}}%
\pgfpathlineto{\pgfqpoint{1.733516in}{1.641528in}}%
\pgfpathlineto{\pgfqpoint{1.735246in}{1.727875in}}%
\pgfpathlineto{\pgfqpoint{1.736112in}{1.644232in}}%
\pgfpathlineto{\pgfqpoint{1.736974in}{1.698668in}}%
\pgfpathlineto{\pgfqpoint{1.738706in}{1.611284in}}%
\pgfpathlineto{\pgfqpoint{1.739570in}{1.625116in}}%
\pgfpathlineto{\pgfqpoint{1.740436in}{1.637077in}}%
\pgfpathlineto{\pgfqpoint{1.741299in}{1.620248in}}%
\pgfpathlineto{\pgfqpoint{1.742164in}{1.641737in}}%
\pgfpathlineto{\pgfqpoint{1.743891in}{1.528410in}}%
\pgfpathlineto{\pgfqpoint{1.744756in}{1.564444in}}%
\pgfpathlineto{\pgfqpoint{1.745620in}{1.557855in}}%
\pgfpathlineto{\pgfqpoint{1.746486in}{1.574357in}}%
\pgfpathlineto{\pgfqpoint{1.747352in}{1.542212in}}%
\pgfpathlineto{\pgfqpoint{1.748217in}{1.554709in}}%
\pgfpathlineto{\pgfqpoint{1.749082in}{1.634791in}}%
\pgfpathlineto{\pgfqpoint{1.750814in}{1.572992in}}%
\pgfpathlineto{\pgfqpoint{1.751679in}{1.578217in}}%
\pgfpathlineto{\pgfqpoint{1.752545in}{1.605702in}}%
\pgfpathlineto{\pgfqpoint{1.754272in}{1.571092in}}%
\pgfpathlineto{\pgfqpoint{1.756004in}{1.635977in}}%
\pgfpathlineto{\pgfqpoint{1.756870in}{1.627220in}}%
\pgfpathlineto{\pgfqpoint{1.757735in}{1.580946in}}%
\pgfpathlineto{\pgfqpoint{1.758602in}{1.588011in}}%
\pgfpathlineto{\pgfqpoint{1.759468in}{1.598400in}}%
\pgfpathlineto{\pgfqpoint{1.760334in}{1.568954in}}%
\pgfpathlineto{\pgfqpoint{1.761200in}{1.603445in}}%
\pgfpathlineto{\pgfqpoint{1.762066in}{1.594778in}}%
\pgfpathlineto{\pgfqpoint{1.762931in}{1.574535in}}%
\pgfpathlineto{\pgfqpoint{1.763797in}{1.609146in}}%
\pgfpathlineto{\pgfqpoint{1.764663in}{1.528112in}}%
\pgfpathlineto{\pgfqpoint{1.765529in}{1.576019in}}%
\pgfpathlineto{\pgfqpoint{1.766394in}{1.553014in}}%
\pgfpathlineto{\pgfqpoint{1.767259in}{1.611815in}}%
\pgfpathlineto{\pgfqpoint{1.768988in}{1.540372in}}%
\pgfpathlineto{\pgfqpoint{1.769853in}{1.621197in}}%
\pgfpathlineto{\pgfqpoint{1.770717in}{1.571687in}}%
\pgfpathlineto{\pgfqpoint{1.772446in}{1.595313in}}%
\pgfpathlineto{\pgfqpoint{1.773309in}{1.548890in}}%
\pgfpathlineto{\pgfqpoint{1.774173in}{1.585932in}}%
\pgfpathlineto{\pgfqpoint{1.775903in}{1.560882in}}%
\pgfpathlineto{\pgfqpoint{1.776768in}{1.632772in}}%
\pgfpathlineto{\pgfqpoint{1.778496in}{1.539687in}}%
\pgfpathlineto{\pgfqpoint{1.779358in}{1.625647in}}%
\pgfpathlineto{\pgfqpoint{1.780222in}{1.554115in}}%
\pgfpathlineto{\pgfqpoint{1.782816in}{1.665125in}}%
\pgfpathlineto{\pgfqpoint{1.783681in}{1.557498in}}%
\pgfpathlineto{\pgfqpoint{1.784545in}{1.584389in}}%
\pgfpathlineto{\pgfqpoint{1.785410in}{1.542063in}}%
\pgfpathlineto{\pgfqpoint{1.786275in}{1.610510in}}%
\pgfpathlineto{\pgfqpoint{1.787138in}{1.589970in}}%
\pgfpathlineto{\pgfqpoint{1.788003in}{1.620483in}}%
\pgfpathlineto{\pgfqpoint{1.788868in}{1.504546in}}%
\pgfpathlineto{\pgfqpoint{1.789732in}{1.631764in}}%
\pgfpathlineto{\pgfqpoint{1.791463in}{1.536423in}}%
\pgfpathlineto{\pgfqpoint{1.792326in}{1.598519in}}%
\pgfpathlineto{\pgfqpoint{1.793189in}{1.526510in}}%
\pgfpathlineto{\pgfqpoint{1.794054in}{1.576852in}}%
\pgfpathlineto{\pgfqpoint{1.794918in}{1.569103in}}%
\pgfpathlineto{\pgfqpoint{1.796649in}{1.601843in}}%
\pgfpathlineto{\pgfqpoint{1.798377in}{1.549188in}}%
\pgfpathlineto{\pgfqpoint{1.799241in}{1.594778in}}%
\pgfpathlineto{\pgfqpoint{1.800107in}{1.534344in}}%
\pgfpathlineto{\pgfqpoint{1.801837in}{1.622264in}}%
\pgfpathlineto{\pgfqpoint{1.802704in}{1.582489in}}%
\pgfpathlineto{\pgfqpoint{1.803570in}{1.644406in}}%
\pgfpathlineto{\pgfqpoint{1.805299in}{1.564087in}}%
\pgfpathlineto{\pgfqpoint{1.806165in}{1.631169in}}%
\pgfpathlineto{\pgfqpoint{1.807031in}{1.606561in}}%
\pgfpathlineto{\pgfqpoint{1.807897in}{1.548474in}}%
\pgfpathlineto{\pgfqpoint{1.808763in}{1.603029in}}%
\pgfpathlineto{\pgfqpoint{1.810493in}{1.540104in}}%
\pgfpathlineto{\pgfqpoint{1.811358in}{1.552690in}}%
\pgfpathlineto{\pgfqpoint{1.812222in}{1.540580in}}%
\pgfpathlineto{\pgfqpoint{1.813089in}{1.589732in}}%
\pgfpathlineto{\pgfqpoint{1.814820in}{1.528291in}}%
\pgfpathlineto{\pgfqpoint{1.815683in}{1.566582in}}%
\pgfpathlineto{\pgfqpoint{1.816549in}{1.540967in}}%
\pgfpathlineto{\pgfqpoint{1.817413in}{1.603981in}}%
\pgfpathlineto{\pgfqpoint{1.819140in}{1.532121in}}%
\pgfpathlineto{\pgfqpoint{1.820005in}{1.625588in}}%
\pgfpathlineto{\pgfqpoint{1.821736in}{1.522948in}}%
\pgfpathlineto{\pgfqpoint{1.822603in}{1.616742in}}%
\pgfpathlineto{\pgfqpoint{1.823469in}{1.560436in}}%
\pgfpathlineto{\pgfqpoint{1.826063in}{1.659752in}}%
\pgfpathlineto{\pgfqpoint{1.826928in}{1.656695in}}%
\pgfpathlineto{\pgfqpoint{1.827794in}{1.687267in}}%
\pgfpathlineto{\pgfqpoint{1.828657in}{1.620364in}}%
\pgfpathlineto{\pgfqpoint{1.829522in}{1.620780in}}%
\pgfpathlineto{\pgfqpoint{1.830388in}{1.670289in}}%
\pgfpathlineto{\pgfqpoint{1.831253in}{1.572754in}}%
\pgfpathlineto{\pgfqpoint{1.832114in}{1.665660in}}%
\pgfpathlineto{\pgfqpoint{1.832979in}{1.639480in}}%
\pgfpathlineto{\pgfqpoint{1.833845in}{1.628202in}}%
\pgfpathlineto{\pgfqpoint{1.834712in}{1.577030in}}%
\pgfpathlineto{\pgfqpoint{1.835577in}{1.585490in}}%
\pgfpathlineto{\pgfqpoint{1.836442in}{1.698906in}}%
\pgfpathlineto{\pgfqpoint{1.838171in}{1.623662in}}%
\pgfpathlineto{\pgfqpoint{1.839899in}{1.683411in}}%
\pgfpathlineto{\pgfqpoint{1.842494in}{1.553166in}}%
\pgfpathlineto{\pgfqpoint{1.844224in}{1.654026in}}%
\pgfpathlineto{\pgfqpoint{1.845089in}{1.605435in}}%
\pgfpathlineto{\pgfqpoint{1.845955in}{1.660376in}}%
\pgfpathlineto{\pgfqpoint{1.846820in}{1.636691in}}%
\pgfpathlineto{\pgfqpoint{1.847684in}{1.568155in}}%
\pgfpathlineto{\pgfqpoint{1.848546in}{1.586825in}}%
\pgfpathlineto{\pgfqpoint{1.849412in}{1.614548in}}%
\pgfpathlineto{\pgfqpoint{1.850276in}{1.549069in}}%
\pgfpathlineto{\pgfqpoint{1.851141in}{1.651709in}}%
\pgfpathlineto{\pgfqpoint{1.852870in}{1.533218in}}%
\pgfpathlineto{\pgfqpoint{1.853735in}{1.584865in}}%
\pgfpathlineto{\pgfqpoint{1.854601in}{1.564117in}}%
\pgfpathlineto{\pgfqpoint{1.856331in}{1.622502in}}%
\pgfpathlineto{\pgfqpoint{1.857196in}{1.555479in}}%
\pgfpathlineto{\pgfqpoint{1.858925in}{1.584389in}}%
\pgfpathlineto{\pgfqpoint{1.859790in}{1.565928in}}%
\pgfpathlineto{\pgfqpoint{1.860656in}{1.588070in}}%
\pgfpathlineto{\pgfqpoint{1.862388in}{1.691365in}}%
\pgfpathlineto{\pgfqpoint{1.863254in}{1.697597in}}%
\pgfpathlineto{\pgfqpoint{1.864120in}{1.577711in}}%
\pgfpathlineto{\pgfqpoint{1.865845in}{1.625171in}}%
\pgfpathlineto{\pgfqpoint{1.867573in}{1.544674in}}%
\pgfpathlineto{\pgfqpoint{1.869304in}{1.630515in}}%
\pgfpathlineto{\pgfqpoint{1.870168in}{1.511904in}}%
\pgfpathlineto{\pgfqpoint{1.871897in}{1.582429in}}%
\pgfpathlineto{\pgfqpoint{1.872762in}{1.555241in}}%
\pgfpathlineto{\pgfqpoint{1.873626in}{1.615020in}}%
\pgfpathlineto{\pgfqpoint{1.874493in}{1.538620in}}%
\pgfpathlineto{\pgfqpoint{1.875359in}{1.553639in}}%
\pgfpathlineto{\pgfqpoint{1.877089in}{1.541647in}}%
\pgfpathlineto{\pgfqpoint{1.877953in}{1.566106in}}%
\pgfpathlineto{\pgfqpoint{1.878816in}{1.499202in}}%
\pgfpathlineto{\pgfqpoint{1.879681in}{1.584389in}}%
\pgfpathlineto{\pgfqpoint{1.880547in}{1.553490in}}%
\pgfpathlineto{\pgfqpoint{1.883143in}{1.639420in}}%
\pgfpathlineto{\pgfqpoint{1.884008in}{1.617158in}}%
\pgfpathlineto{\pgfqpoint{1.884871in}{1.640963in}}%
\pgfpathlineto{\pgfqpoint{1.885735in}{1.610183in}}%
\pgfpathlineto{\pgfqpoint{1.887463in}{1.651828in}}%
\pgfpathlineto{\pgfqpoint{1.890054in}{1.566493in}}%
\pgfpathlineto{\pgfqpoint{1.890918in}{1.605583in}}%
\pgfpathlineto{\pgfqpoint{1.891784in}{1.549128in}}%
\pgfpathlineto{\pgfqpoint{1.892650in}{1.555568in}}%
\pgfpathlineto{\pgfqpoint{1.893515in}{1.550433in}}%
\pgfpathlineto{\pgfqpoint{1.894381in}{1.604988in}}%
\pgfpathlineto{\pgfqpoint{1.895246in}{1.593146in}}%
\pgfpathlineto{\pgfqpoint{1.896112in}{1.545031in}}%
\pgfpathlineto{\pgfqpoint{1.897841in}{1.590089in}}%
\pgfpathlineto{\pgfqpoint{1.898707in}{1.553996in}}%
\pgfpathlineto{\pgfqpoint{1.900437in}{1.603743in}}%
\pgfpathlineto{\pgfqpoint{1.901303in}{1.571033in}}%
\pgfpathlineto{\pgfqpoint{1.902169in}{1.617099in}}%
\pgfpathlineto{\pgfqpoint{1.903034in}{1.605345in}}%
\pgfpathlineto{\pgfqpoint{1.904764in}{1.574298in}}%
\pgfpathlineto{\pgfqpoint{1.907360in}{1.594243in}}%
\pgfpathlineto{\pgfqpoint{1.908225in}{1.563909in}}%
\pgfpathlineto{\pgfqpoint{1.909953in}{1.634196in}}%
\pgfpathlineto{\pgfqpoint{1.910818in}{1.519266in}}%
\pgfpathlineto{\pgfqpoint{1.911684in}{1.600891in}}%
\pgfpathlineto{\pgfqpoint{1.912550in}{1.566757in}}%
\pgfpathlineto{\pgfqpoint{1.913415in}{1.606413in}}%
\pgfpathlineto{\pgfqpoint{1.915146in}{1.586349in}}%
\pgfpathlineto{\pgfqpoint{1.916877in}{1.616742in}}%
\pgfpathlineto{\pgfqpoint{1.917742in}{1.601426in}}%
\pgfpathlineto{\pgfqpoint{1.919471in}{1.551857in}}%
\pgfpathlineto{\pgfqpoint{1.921200in}{1.598221in}}%
\pgfpathlineto{\pgfqpoint{1.922931in}{1.539747in}}%
\pgfpathlineto{\pgfqpoint{1.923796in}{1.582013in}}%
\pgfpathlineto{\pgfqpoint{1.924661in}{1.549957in}}%
\pgfpathlineto{\pgfqpoint{1.925527in}{1.585575in}}%
\pgfpathlineto{\pgfqpoint{1.927259in}{1.517664in}}%
\pgfpathlineto{\pgfqpoint{1.928985in}{1.618464in}}%
\pgfpathlineto{\pgfqpoint{1.929850in}{1.610153in}}%
\pgfpathlineto{\pgfqpoint{1.930715in}{1.630366in}}%
\pgfpathlineto{\pgfqpoint{1.931580in}{1.613537in}}%
\pgfpathlineto{\pgfqpoint{1.932446in}{1.643280in}}%
\pgfpathlineto{\pgfqpoint{1.933312in}{1.537996in}}%
\pgfpathlineto{\pgfqpoint{1.935043in}{1.586230in}}%
\pgfpathlineto{\pgfqpoint{1.936775in}{1.622561in}}%
\pgfpathlineto{\pgfqpoint{1.937642in}{1.648147in}}%
\pgfpathlineto{\pgfqpoint{1.938507in}{1.611165in}}%
\pgfpathlineto{\pgfqpoint{1.939373in}{1.638174in}}%
\pgfpathlineto{\pgfqpoint{1.941967in}{1.535475in}}%
\pgfpathlineto{\pgfqpoint{1.944561in}{1.642510in}}%
\pgfpathlineto{\pgfqpoint{1.946290in}{1.580593in}}%
\pgfpathlineto{\pgfqpoint{1.947154in}{1.592465in}}%
\pgfpathlineto{\pgfqpoint{1.948018in}{1.509948in}}%
\pgfpathlineto{\pgfqpoint{1.949748in}{1.632151in}}%
\pgfpathlineto{\pgfqpoint{1.951475in}{1.623394in}}%
\pgfpathlineto{\pgfqpoint{1.952341in}{1.626748in}}%
\pgfpathlineto{\pgfqpoint{1.954072in}{1.695343in}}%
\pgfpathlineto{\pgfqpoint{1.954936in}{1.669698in}}%
\pgfpathlineto{\pgfqpoint{1.955801in}{1.729418in}}%
\pgfpathlineto{\pgfqpoint{1.956666in}{1.679909in}}%
\pgfpathlineto{\pgfqpoint{1.957531in}{1.687271in}}%
\pgfpathlineto{\pgfqpoint{1.959262in}{1.709443in}}%
\pgfpathlineto{\pgfqpoint{1.962723in}{1.617872in}}%
\pgfpathlineto{\pgfqpoint{1.963588in}{1.708402in}}%
\pgfpathlineto{\pgfqpoint{1.964454in}{1.675454in}}%
\pgfpathlineto{\pgfqpoint{1.966183in}{1.732088in}}%
\pgfpathlineto{\pgfqpoint{1.968778in}{1.655241in}}%
\pgfpathlineto{\pgfqpoint{1.969644in}{1.691900in}}%
\pgfpathlineto{\pgfqpoint{1.970509in}{1.670587in}}%
\pgfpathlineto{\pgfqpoint{1.972240in}{1.572576in}}%
\pgfpathlineto{\pgfqpoint{1.973106in}{1.555126in}}%
\pgfpathlineto{\pgfqpoint{1.974835in}{1.612291in}}%
\pgfpathlineto{\pgfqpoint{1.975702in}{1.556134in}}%
\pgfpathlineto{\pgfqpoint{1.977433in}{1.631407in}}%
\pgfpathlineto{\pgfqpoint{1.978298in}{1.564117in}}%
\pgfpathlineto{\pgfqpoint{1.980028in}{1.646544in}}%
\pgfpathlineto{\pgfqpoint{1.980894in}{1.621226in}}%
\pgfpathlineto{\pgfqpoint{1.981760in}{1.583500in}}%
\pgfpathlineto{\pgfqpoint{1.983491in}{1.675514in}}%
\pgfpathlineto{\pgfqpoint{1.986085in}{1.535594in}}%
\pgfpathlineto{\pgfqpoint{1.989540in}{1.629329in}}%
\pgfpathlineto{\pgfqpoint{1.990405in}{1.634701in}}%
\pgfpathlineto{\pgfqpoint{1.991267in}{1.618110in}}%
\pgfpathlineto{\pgfqpoint{1.992132in}{1.656755in}}%
\pgfpathlineto{\pgfqpoint{1.995590in}{1.534407in}}%
\pgfpathlineto{\pgfqpoint{1.997321in}{1.616181in}}%
\pgfpathlineto{\pgfqpoint{1.999051in}{1.575487in}}%
\pgfpathlineto{\pgfqpoint{1.999915in}{1.627135in}}%
\pgfpathlineto{\pgfqpoint{2.000777in}{1.604516in}}%
\pgfpathlineto{\pgfqpoint{2.001642in}{1.640907in}}%
\pgfpathlineto{\pgfqpoint{2.002508in}{1.563674in}}%
\pgfpathlineto{\pgfqpoint{2.003373in}{1.614965in}}%
\pgfpathlineto{\pgfqpoint{2.004238in}{1.553375in}}%
\pgfpathlineto{\pgfqpoint{2.005103in}{1.606416in}}%
\pgfpathlineto{\pgfqpoint{2.006833in}{1.553996in}}%
\pgfpathlineto{\pgfqpoint{2.007697in}{1.636810in}}%
\pgfpathlineto{\pgfqpoint{2.009429in}{1.547317in}}%
\pgfpathlineto{\pgfqpoint{2.011158in}{1.659071in}}%
\pgfpathlineto{\pgfqpoint{2.012023in}{1.604219in}}%
\pgfpathlineto{\pgfqpoint{2.012887in}{1.647496in}}%
\pgfpathlineto{\pgfqpoint{2.013752in}{1.635148in}}%
\pgfpathlineto{\pgfqpoint{2.014617in}{1.546280in}}%
\pgfpathlineto{\pgfqpoint{2.015482in}{1.565098in}}%
\pgfpathlineto{\pgfqpoint{2.016347in}{1.571539in}}%
\pgfpathlineto{\pgfqpoint{2.017211in}{1.527343in}}%
\pgfpathlineto{\pgfqpoint{2.018077in}{1.590744in}}%
\pgfpathlineto{\pgfqpoint{2.018939in}{1.532388in}}%
\pgfpathlineto{\pgfqpoint{2.020669in}{1.615913in}}%
\pgfpathlineto{\pgfqpoint{2.022400in}{1.564860in}}%
\pgfpathlineto{\pgfqpoint{2.024131in}{1.613180in}}%
\pgfpathlineto{\pgfqpoint{2.024995in}{1.612589in}}%
\pgfpathlineto{\pgfqpoint{2.025860in}{1.606000in}}%
\pgfpathlineto{\pgfqpoint{2.026727in}{1.576852in}}%
\pgfpathlineto{\pgfqpoint{2.027592in}{1.589201in}}%
\pgfpathlineto{\pgfqpoint{2.028455in}{1.633515in}}%
\pgfpathlineto{\pgfqpoint{2.029320in}{1.559815in}}%
\pgfpathlineto{\pgfqpoint{2.031052in}{1.662336in}}%
\pgfpathlineto{\pgfqpoint{2.032782in}{1.576614in}}%
\pgfpathlineto{\pgfqpoint{2.033646in}{1.581838in}}%
\pgfpathlineto{\pgfqpoint{2.035374in}{1.642566in}}%
\pgfpathlineto{\pgfqpoint{2.036239in}{1.641677in}}%
\pgfpathlineto{\pgfqpoint{2.037104in}{1.593532in}}%
\pgfpathlineto{\pgfqpoint{2.037969in}{1.662068in}}%
\pgfpathlineto{\pgfqpoint{2.039700in}{1.617396in}}%
\pgfpathlineto{\pgfqpoint{2.040565in}{1.627428in}}%
\pgfpathlineto{\pgfqpoint{2.041430in}{1.605464in}}%
\pgfpathlineto{\pgfqpoint{2.042296in}{1.665601in}}%
\pgfpathlineto{\pgfqpoint{2.043162in}{1.614013in}}%
\pgfpathlineto{\pgfqpoint{2.044026in}{1.683884in}}%
\pgfpathlineto{\pgfqpoint{2.044891in}{1.652836in}}%
\pgfpathlineto{\pgfqpoint{2.045756in}{1.676224in}}%
\pgfpathlineto{\pgfqpoint{2.047484in}{1.553252in}}%
\pgfpathlineto{\pgfqpoint{2.049214in}{1.666132in}}%
\pgfpathlineto{\pgfqpoint{2.050080in}{1.635263in}}%
\pgfpathlineto{\pgfqpoint{2.050945in}{1.585754in}}%
\pgfpathlineto{\pgfqpoint{2.051810in}{1.666549in}}%
\pgfpathlineto{\pgfqpoint{2.052674in}{1.613358in}}%
\pgfpathlineto{\pgfqpoint{2.053538in}{1.621907in}}%
\pgfpathlineto{\pgfqpoint{2.054402in}{1.640844in}}%
\pgfpathlineto{\pgfqpoint{2.056131in}{1.565511in}}%
\pgfpathlineto{\pgfqpoint{2.056996in}{1.606710in}}%
\pgfpathlineto{\pgfqpoint{2.057861in}{1.591989in}}%
\pgfpathlineto{\pgfqpoint{2.058726in}{1.600776in}}%
\pgfpathlineto{\pgfqpoint{2.060456in}{1.647020in}}%
\pgfpathlineto{\pgfqpoint{2.061322in}{1.612618in}}%
\pgfpathlineto{\pgfqpoint{2.062187in}{1.525264in}}%
\pgfpathlineto{\pgfqpoint{2.063052in}{1.558034in}}%
\pgfpathlineto{\pgfqpoint{2.063916in}{1.540550in}}%
\pgfpathlineto{\pgfqpoint{2.065646in}{1.622085in}}%
\pgfpathlineto{\pgfqpoint{2.068243in}{1.570676in}}%
\pgfpathlineto{\pgfqpoint{2.069109in}{1.557379in}}%
\pgfpathlineto{\pgfqpoint{2.069974in}{1.572576in}}%
\pgfpathlineto{\pgfqpoint{2.070838in}{1.535058in}}%
\pgfpathlineto{\pgfqpoint{2.071701in}{1.645061in}}%
\pgfpathlineto{\pgfqpoint{2.072565in}{1.546752in}}%
\pgfpathlineto{\pgfqpoint{2.074294in}{1.607662in}}%
\pgfpathlineto{\pgfqpoint{2.075160in}{1.575190in}}%
\pgfpathlineto{\pgfqpoint{2.076025in}{1.638085in}}%
\pgfpathlineto{\pgfqpoint{2.076890in}{1.516537in}}%
\pgfpathlineto{\pgfqpoint{2.077754in}{1.545090in}}%
\pgfpathlineto{\pgfqpoint{2.079483in}{1.643399in}}%
\pgfpathlineto{\pgfqpoint{2.080348in}{1.600300in}}%
\pgfpathlineto{\pgfqpoint{2.081213in}{1.610629in}}%
\pgfpathlineto{\pgfqpoint{2.082078in}{1.632950in}}%
\pgfpathlineto{\pgfqpoint{2.084671in}{1.588844in}}%
\pgfpathlineto{\pgfqpoint{2.086401in}{1.645001in}}%
\pgfpathlineto{\pgfqpoint{2.088130in}{1.598935in}}%
\pgfpathlineto{\pgfqpoint{2.088995in}{1.643339in}}%
\pgfpathlineto{\pgfqpoint{2.089859in}{1.585073in}}%
\pgfpathlineto{\pgfqpoint{2.090724in}{1.639599in}}%
\pgfpathlineto{\pgfqpoint{2.091589in}{1.620308in}}%
\pgfpathlineto{\pgfqpoint{2.092454in}{1.519032in}}%
\pgfpathlineto{\pgfqpoint{2.094184in}{1.676109in}}%
\pgfpathlineto{\pgfqpoint{2.095914in}{1.573766in}}%
\pgfpathlineto{\pgfqpoint{2.096780in}{1.661622in}}%
\pgfpathlineto{\pgfqpoint{2.098507in}{1.588427in}}%
\pgfpathlineto{\pgfqpoint{2.099373in}{1.620483in}}%
\pgfpathlineto{\pgfqpoint{2.100239in}{1.600240in}}%
\pgfpathlineto{\pgfqpoint{2.101101in}{1.659900in}}%
\pgfpathlineto{\pgfqpoint{2.102828in}{1.592997in}}%
\pgfpathlineto{\pgfqpoint{2.103692in}{1.597745in}}%
\pgfpathlineto{\pgfqpoint{2.106286in}{1.577800in}}%
\pgfpathlineto{\pgfqpoint{2.107151in}{1.533396in}}%
\pgfpathlineto{\pgfqpoint{2.109746in}{1.621018in}}%
\pgfpathlineto{\pgfqpoint{2.110611in}{1.609384in}}%
\pgfpathlineto{\pgfqpoint{2.112342in}{1.552452in}}%
\pgfpathlineto{\pgfqpoint{2.113207in}{1.598638in}}%
\pgfpathlineto{\pgfqpoint{2.114938in}{1.528410in}}%
\pgfpathlineto{\pgfqpoint{2.115804in}{1.606591in}}%
\pgfpathlineto{\pgfqpoint{2.118403in}{1.541439in}}%
\pgfpathlineto{\pgfqpoint{2.120134in}{1.635560in}}%
\pgfpathlineto{\pgfqpoint{2.120999in}{1.611250in}}%
\pgfpathlineto{\pgfqpoint{2.122731in}{1.634196in}}%
\pgfpathlineto{\pgfqpoint{2.123596in}{1.551798in}}%
\pgfpathlineto{\pgfqpoint{2.125328in}{1.587178in}}%
\pgfpathlineto{\pgfqpoint{2.126193in}{1.558208in}}%
\pgfpathlineto{\pgfqpoint{2.128790in}{1.592699in}}%
\pgfpathlineto{\pgfqpoint{2.131385in}{1.521405in}}%
\pgfpathlineto{\pgfqpoint{2.133116in}{1.635382in}}%
\pgfpathlineto{\pgfqpoint{2.133982in}{1.639182in}}%
\pgfpathlineto{\pgfqpoint{2.135713in}{1.606413in}}%
\pgfpathlineto{\pgfqpoint{2.136575in}{1.624164in}}%
\pgfpathlineto{\pgfqpoint{2.137440in}{1.573911in}}%
\pgfpathlineto{\pgfqpoint{2.138303in}{1.673852in}}%
\pgfpathlineto{\pgfqpoint{2.139169in}{1.661444in}}%
\pgfpathlineto{\pgfqpoint{2.140031in}{1.675811in}}%
\pgfpathlineto{\pgfqpoint{2.141763in}{1.598757in}}%
\pgfpathlineto{\pgfqpoint{2.142629in}{1.601902in}}%
\pgfpathlineto{\pgfqpoint{2.143495in}{1.625647in}}%
\pgfpathlineto{\pgfqpoint{2.145225in}{1.598578in}}%
\pgfpathlineto{\pgfqpoint{2.146091in}{1.610748in}}%
\pgfpathlineto{\pgfqpoint{2.147822in}{1.567709in}}%
\pgfpathlineto{\pgfqpoint{2.149554in}{1.639985in}}%
\pgfpathlineto{\pgfqpoint{2.150418in}{1.659960in}}%
\pgfpathlineto{\pgfqpoint{2.152149in}{1.572933in}}%
\pgfpathlineto{\pgfqpoint{2.153015in}{1.587479in}}%
\pgfpathlineto{\pgfqpoint{2.153879in}{1.650225in}}%
\pgfpathlineto{\pgfqpoint{2.154743in}{1.648682in}}%
\pgfpathlineto{\pgfqpoint{2.156472in}{1.524937in}}%
\pgfpathlineto{\pgfqpoint{2.158202in}{1.609737in}}%
\pgfpathlineto{\pgfqpoint{2.159066in}{1.578838in}}%
\pgfpathlineto{\pgfqpoint{2.159928in}{1.615080in}}%
\pgfpathlineto{\pgfqpoint{2.160794in}{1.543012in}}%
\pgfpathlineto{\pgfqpoint{2.162520in}{1.600712in}}%
\pgfpathlineto{\pgfqpoint{2.163384in}{1.562127in}}%
\pgfpathlineto{\pgfqpoint{2.165115in}{1.621907in}}%
\pgfpathlineto{\pgfqpoint{2.165980in}{1.622383in}}%
\pgfpathlineto{\pgfqpoint{2.166844in}{1.684062in}}%
\pgfpathlineto{\pgfqpoint{2.167707in}{1.675990in}}%
\pgfpathlineto{\pgfqpoint{2.168572in}{1.635650in}}%
\pgfpathlineto{\pgfqpoint{2.169437in}{1.776075in}}%
\pgfpathlineto{\pgfqpoint{2.171166in}{1.645150in}}%
\pgfpathlineto{\pgfqpoint{2.172029in}{1.701754in}}%
\pgfpathlineto{\pgfqpoint{2.172893in}{1.655093in}}%
\pgfpathlineto{\pgfqpoint{2.173758in}{1.659250in}}%
\pgfpathlineto{\pgfqpoint{2.174620in}{1.675752in}}%
\pgfpathlineto{\pgfqpoint{2.175484in}{1.639480in}}%
\pgfpathlineto{\pgfqpoint{2.176349in}{1.738677in}}%
\pgfpathlineto{\pgfqpoint{2.177215in}{1.738558in}}%
\pgfpathlineto{\pgfqpoint{2.178080in}{1.610064in}}%
\pgfpathlineto{\pgfqpoint{2.179810in}{1.673673in}}%
\pgfpathlineto{\pgfqpoint{2.180676in}{1.665035in}}%
\pgfpathlineto{\pgfqpoint{2.181542in}{1.682400in}}%
\pgfpathlineto{\pgfqpoint{2.182405in}{1.653550in}}%
\pgfpathlineto{\pgfqpoint{2.183270in}{1.703178in}}%
\pgfpathlineto{\pgfqpoint{2.184136in}{1.698013in}}%
\pgfpathlineto{\pgfqpoint{2.185000in}{1.679046in}}%
\pgfpathlineto{\pgfqpoint{2.185866in}{1.689286in}}%
\pgfpathlineto{\pgfqpoint{2.186731in}{1.718851in}}%
\pgfpathlineto{\pgfqpoint{2.188460in}{1.644406in}}%
\pgfpathlineto{\pgfqpoint{2.189326in}{1.767825in}}%
\pgfpathlineto{\pgfqpoint{2.190192in}{1.745771in}}%
\pgfpathlineto{\pgfqpoint{2.191056in}{1.766757in}}%
\pgfpathlineto{\pgfqpoint{2.192788in}{1.635977in}}%
\pgfpathlineto{\pgfqpoint{2.193652in}{1.664530in}}%
\pgfpathlineto{\pgfqpoint{2.194519in}{1.571624in}}%
\pgfpathlineto{\pgfqpoint{2.195384in}{1.702821in}}%
\pgfpathlineto{\pgfqpoint{2.196249in}{1.688334in}}%
\pgfpathlineto{\pgfqpoint{2.197115in}{1.633690in}}%
\pgfpathlineto{\pgfqpoint{2.197982in}{1.664470in}}%
\pgfpathlineto{\pgfqpoint{2.198846in}{1.625469in}}%
\pgfpathlineto{\pgfqpoint{2.199710in}{1.695756in}}%
\pgfpathlineto{\pgfqpoint{2.200575in}{1.678659in}}%
\pgfpathlineto{\pgfqpoint{2.201440in}{1.712853in}}%
\pgfpathlineto{\pgfqpoint{2.202305in}{1.628853in}}%
\pgfpathlineto{\pgfqpoint{2.203171in}{1.666430in}}%
\pgfpathlineto{\pgfqpoint{2.204035in}{1.661711in}}%
\pgfpathlineto{\pgfqpoint{2.204900in}{1.651888in}}%
\pgfpathlineto{\pgfqpoint{2.207495in}{1.703178in}}%
\pgfpathlineto{\pgfqpoint{2.208361in}{1.647849in}}%
\pgfpathlineto{\pgfqpoint{2.209225in}{1.655003in}}%
\pgfpathlineto{\pgfqpoint{2.210090in}{1.677354in}}%
\pgfpathlineto{\pgfqpoint{2.210954in}{1.675633in}}%
\pgfpathlineto{\pgfqpoint{2.211819in}{1.646723in}}%
\pgfpathlineto{\pgfqpoint{2.212685in}{1.712675in}}%
\pgfpathlineto{\pgfqpoint{2.214415in}{1.655152in}}%
\pgfpathlineto{\pgfqpoint{2.215281in}{1.697894in}}%
\pgfpathlineto{\pgfqpoint{2.216146in}{1.694749in}}%
\pgfpathlineto{\pgfqpoint{2.218741in}{1.639152in}}%
\pgfpathlineto{\pgfqpoint{2.220471in}{1.682043in}}%
\pgfpathlineto{\pgfqpoint{2.221337in}{1.592937in}}%
\pgfpathlineto{\pgfqpoint{2.222202in}{1.687148in}}%
\pgfpathlineto{\pgfqpoint{2.223065in}{1.615407in}}%
\pgfpathlineto{\pgfqpoint{2.223931in}{1.693856in}}%
\pgfpathlineto{\pgfqpoint{2.226526in}{1.634136in}}%
\pgfpathlineto{\pgfqpoint{2.227389in}{1.703948in}}%
\pgfpathlineto{\pgfqpoint{2.228254in}{1.659841in}}%
\pgfpathlineto{\pgfqpoint{2.229986in}{1.711429in}}%
\pgfpathlineto{\pgfqpoint{2.230853in}{1.709648in}}%
\pgfpathlineto{\pgfqpoint{2.231717in}{1.702583in}}%
\pgfpathlineto{\pgfqpoint{2.232580in}{1.704513in}}%
\pgfpathlineto{\pgfqpoint{2.234309in}{1.592878in}}%
\pgfpathlineto{\pgfqpoint{2.235174in}{1.671981in}}%
\pgfpathlineto{\pgfqpoint{2.236039in}{1.641915in}}%
\pgfpathlineto{\pgfqpoint{2.236904in}{1.713567in}}%
\pgfpathlineto{\pgfqpoint{2.239497in}{1.648920in}}%
\pgfpathlineto{\pgfqpoint{2.240361in}{1.638829in}}%
\pgfpathlineto{\pgfqpoint{2.241226in}{1.609681in}}%
\pgfpathlineto{\pgfqpoint{2.242955in}{1.710362in}}%
\pgfpathlineto{\pgfqpoint{2.243820in}{1.633367in}}%
\pgfpathlineto{\pgfqpoint{2.245550in}{1.721818in}}%
\pgfpathlineto{\pgfqpoint{2.246416in}{1.648563in}}%
\pgfpathlineto{\pgfqpoint{2.247281in}{1.655539in}}%
\pgfpathlineto{\pgfqpoint{2.248147in}{1.700151in}}%
\pgfpathlineto{\pgfqpoint{2.250740in}{1.647080in}}%
\pgfpathlineto{\pgfqpoint{2.253336in}{1.713329in}}%
\pgfpathlineto{\pgfqpoint{2.254201in}{1.628763in}}%
\pgfpathlineto{\pgfqpoint{2.255930in}{1.682281in}}%
\pgfpathlineto{\pgfqpoint{2.256794in}{1.631199in}}%
\pgfpathlineto{\pgfqpoint{2.258523in}{1.720513in}}%
\pgfpathlineto{\pgfqpoint{2.259389in}{1.710064in}}%
\pgfpathlineto{\pgfqpoint{2.260254in}{1.629329in}}%
\pgfpathlineto{\pgfqpoint{2.261984in}{1.718018in}}%
\pgfpathlineto{\pgfqpoint{2.263712in}{1.692194in}}%
\pgfpathlineto{\pgfqpoint{2.265441in}{1.736301in}}%
\pgfpathlineto{\pgfqpoint{2.266306in}{1.686583in}}%
\pgfpathlineto{\pgfqpoint{2.267171in}{1.753041in}}%
\pgfpathlineto{\pgfqpoint{2.268898in}{1.661146in}}%
\pgfpathlineto{\pgfqpoint{2.269763in}{1.709469in}}%
\pgfpathlineto{\pgfqpoint{2.270626in}{1.675514in}}%
\pgfpathlineto{\pgfqpoint{2.271490in}{1.682459in}}%
\pgfpathlineto{\pgfqpoint{2.273219in}{1.728942in}}%
\pgfpathlineto{\pgfqpoint{2.274949in}{1.758979in}}%
\pgfpathlineto{\pgfqpoint{2.275812in}{1.690502in}}%
\pgfpathlineto{\pgfqpoint{2.276676in}{1.705967in}}%
\pgfpathlineto{\pgfqpoint{2.277541in}{1.762005in}}%
\pgfpathlineto{\pgfqpoint{2.280137in}{1.646158in}}%
\pgfpathlineto{\pgfqpoint{2.281003in}{1.669754in}}%
\pgfpathlineto{\pgfqpoint{2.281868in}{1.631050in}}%
\pgfpathlineto{\pgfqpoint{2.283598in}{1.680619in}}%
\pgfpathlineto{\pgfqpoint{2.285328in}{1.639807in}}%
\pgfpathlineto{\pgfqpoint{2.286192in}{1.646425in}}%
\pgfpathlineto{\pgfqpoint{2.287056in}{1.641439in}}%
\pgfpathlineto{\pgfqpoint{2.287921in}{1.519683in}}%
\pgfpathlineto{\pgfqpoint{2.289652in}{1.626982in}}%
\pgfpathlineto{\pgfqpoint{2.292249in}{1.540282in}}%
\pgfpathlineto{\pgfqpoint{2.293114in}{1.568954in}}%
\pgfpathlineto{\pgfqpoint{2.293978in}{1.555360in}}%
\pgfpathlineto{\pgfqpoint{2.295707in}{1.577324in}}%
\pgfpathlineto{\pgfqpoint{2.296571in}{1.563909in}}%
\pgfpathlineto{\pgfqpoint{2.297435in}{1.624636in}}%
\pgfpathlineto{\pgfqpoint{2.298302in}{1.577800in}}%
\pgfpathlineto{\pgfqpoint{2.300033in}{1.635620in}}%
\pgfpathlineto{\pgfqpoint{2.300900in}{1.645830in}}%
\pgfpathlineto{\pgfqpoint{2.301766in}{1.549957in}}%
\pgfpathlineto{\pgfqpoint{2.304361in}{1.617158in}}%
\pgfpathlineto{\pgfqpoint{2.305227in}{1.598578in}}%
\pgfpathlineto{\pgfqpoint{2.306092in}{1.611756in}}%
\pgfpathlineto{\pgfqpoint{2.306957in}{1.540223in}}%
\pgfpathlineto{\pgfqpoint{2.307822in}{1.559160in}}%
\pgfpathlineto{\pgfqpoint{2.308688in}{1.610242in}}%
\pgfpathlineto{\pgfqpoint{2.309554in}{1.543785in}}%
\pgfpathlineto{\pgfqpoint{2.310420in}{1.559279in}}%
\pgfpathlineto{\pgfqpoint{2.311285in}{1.601010in}}%
\pgfpathlineto{\pgfqpoint{2.312152in}{1.585694in}}%
\pgfpathlineto{\pgfqpoint{2.313883in}{1.634166in}}%
\pgfpathlineto{\pgfqpoint{2.314749in}{1.650936in}}%
\pgfpathlineto{\pgfqpoint{2.316478in}{1.568776in}}%
\pgfpathlineto{\pgfqpoint{2.317342in}{1.639063in}}%
\pgfpathlineto{\pgfqpoint{2.318206in}{1.563819in}}%
\pgfpathlineto{\pgfqpoint{2.319072in}{1.569490in}}%
\pgfpathlineto{\pgfqpoint{2.319937in}{1.569430in}}%
\pgfpathlineto{\pgfqpoint{2.321668in}{1.649809in}}%
\pgfpathlineto{\pgfqpoint{2.323399in}{1.577651in}}%
\pgfpathlineto{\pgfqpoint{2.324264in}{1.632950in}}%
\pgfpathlineto{\pgfqpoint{2.325127in}{1.617040in}}%
\pgfpathlineto{\pgfqpoint{2.325993in}{1.637639in}}%
\pgfpathlineto{\pgfqpoint{2.326858in}{1.570676in}}%
\pgfpathlineto{\pgfqpoint{2.327723in}{1.596797in}}%
\pgfpathlineto{\pgfqpoint{2.328589in}{1.670349in}}%
\pgfpathlineto{\pgfqpoint{2.330317in}{1.605554in}}%
\pgfpathlineto{\pgfqpoint{2.332043in}{1.597570in}}%
\pgfpathlineto{\pgfqpoint{2.332908in}{1.609562in}}%
\pgfpathlineto{\pgfqpoint{2.333775in}{1.581838in}}%
\pgfpathlineto{\pgfqpoint{2.334640in}{1.607008in}}%
\pgfpathlineto{\pgfqpoint{2.335506in}{1.509175in}}%
\pgfpathlineto{\pgfqpoint{2.336370in}{1.599288in}}%
\pgfpathlineto{\pgfqpoint{2.337235in}{1.562693in}}%
\pgfpathlineto{\pgfqpoint{2.338100in}{1.639182in}}%
\pgfpathlineto{\pgfqpoint{2.341556in}{1.502170in}}%
\pgfpathlineto{\pgfqpoint{2.342422in}{1.507126in}}%
\pgfpathlineto{\pgfqpoint{2.343286in}{1.533158in}}%
\pgfpathlineto{\pgfqpoint{2.344150in}{1.529774in}}%
\pgfpathlineto{\pgfqpoint{2.345881in}{1.588427in}}%
\pgfpathlineto{\pgfqpoint{2.346747in}{1.533575in}}%
\pgfpathlineto{\pgfqpoint{2.350207in}{1.597567in}}%
\pgfpathlineto{\pgfqpoint{2.353668in}{1.526034in}}%
\pgfpathlineto{\pgfqpoint{2.356262in}{1.624253in}}%
\pgfpathlineto{\pgfqpoint{2.357127in}{1.584746in}}%
\pgfpathlineto{\pgfqpoint{2.358859in}{1.608521in}}%
\pgfpathlineto{\pgfqpoint{2.359724in}{1.511373in}}%
\pgfpathlineto{\pgfqpoint{2.361451in}{1.633928in}}%
\pgfpathlineto{\pgfqpoint{2.362316in}{1.638293in}}%
\pgfpathlineto{\pgfqpoint{2.363181in}{1.613953in}}%
\pgfpathlineto{\pgfqpoint{2.364045in}{1.553936in}}%
\pgfpathlineto{\pgfqpoint{2.364909in}{1.655152in}}%
\pgfpathlineto{\pgfqpoint{2.365771in}{1.649720in}}%
\pgfpathlineto{\pgfqpoint{2.368366in}{1.568359in}}%
\pgfpathlineto{\pgfqpoint{2.369231in}{1.587713in}}%
\pgfpathlineto{\pgfqpoint{2.370096in}{1.650698in}}%
\pgfpathlineto{\pgfqpoint{2.370961in}{1.596410in}}%
\pgfpathlineto{\pgfqpoint{2.371826in}{1.603386in}}%
\pgfpathlineto{\pgfqpoint{2.372691in}{1.634553in}}%
\pgfpathlineto{\pgfqpoint{2.373557in}{1.555182in}}%
\pgfpathlineto{\pgfqpoint{2.374422in}{1.618404in}}%
\pgfpathlineto{\pgfqpoint{2.375287in}{1.613329in}}%
\pgfpathlineto{\pgfqpoint{2.376151in}{1.599943in}}%
\pgfpathlineto{\pgfqpoint{2.377017in}{1.616266in}}%
\pgfpathlineto{\pgfqpoint{2.377884in}{1.607008in}}%
\pgfpathlineto{\pgfqpoint{2.378749in}{1.567649in}}%
\pgfpathlineto{\pgfqpoint{2.380481in}{1.639926in}}%
\pgfpathlineto{\pgfqpoint{2.381346in}{1.568006in}}%
\pgfpathlineto{\pgfqpoint{2.383076in}{1.620721in}}%
\pgfpathlineto{\pgfqpoint{2.383941in}{1.551266in}}%
\pgfpathlineto{\pgfqpoint{2.384807in}{1.656372in}}%
\pgfpathlineto{\pgfqpoint{2.385670in}{1.638710in}}%
\pgfpathlineto{\pgfqpoint{2.389131in}{1.694808in}}%
\pgfpathlineto{\pgfqpoint{2.389995in}{1.636929in}}%
\pgfpathlineto{\pgfqpoint{2.390860in}{1.736840in}}%
\pgfpathlineto{\pgfqpoint{2.391725in}{1.659547in}}%
\pgfpathlineto{\pgfqpoint{2.392590in}{1.660614in}}%
\pgfpathlineto{\pgfqpoint{2.393455in}{1.670884in}}%
\pgfpathlineto{\pgfqpoint{2.394320in}{1.605970in}}%
\pgfpathlineto{\pgfqpoint{2.395187in}{1.676406in}}%
\pgfpathlineto{\pgfqpoint{2.396051in}{1.675990in}}%
\pgfpathlineto{\pgfqpoint{2.397781in}{1.649809in}}%
\pgfpathlineto{\pgfqpoint{2.398647in}{1.698311in}}%
\pgfpathlineto{\pgfqpoint{2.399511in}{1.665452in}}%
\pgfpathlineto{\pgfqpoint{2.400376in}{1.685129in}}%
\pgfpathlineto{\pgfqpoint{2.401240in}{1.675514in}}%
\pgfpathlineto{\pgfqpoint{2.402105in}{1.652895in}}%
\pgfpathlineto{\pgfqpoint{2.403834in}{1.692521in}}%
\pgfpathlineto{\pgfqpoint{2.404699in}{1.658655in}}%
\pgfpathlineto{\pgfqpoint{2.405564in}{1.684835in}}%
\pgfpathlineto{\pgfqpoint{2.407293in}{1.632831in}}%
\pgfpathlineto{\pgfqpoint{2.408157in}{1.667501in}}%
\pgfpathlineto{\pgfqpoint{2.409022in}{1.636245in}}%
\pgfpathlineto{\pgfqpoint{2.409887in}{1.662217in}}%
\pgfpathlineto{\pgfqpoint{2.410753in}{1.658238in}}%
\pgfpathlineto{\pgfqpoint{2.411618in}{1.620780in}}%
\pgfpathlineto{\pgfqpoint{2.412483in}{1.648623in}}%
\pgfpathlineto{\pgfqpoint{2.413348in}{1.644971in}}%
\pgfpathlineto{\pgfqpoint{2.415079in}{1.579760in}}%
\pgfpathlineto{\pgfqpoint{2.416808in}{1.522353in}}%
\pgfpathlineto{\pgfqpoint{2.417672in}{1.645711in}}%
\pgfpathlineto{\pgfqpoint{2.419402in}{1.555003in}}%
\pgfpathlineto{\pgfqpoint{2.421131in}{1.637579in}}%
\pgfpathlineto{\pgfqpoint{2.421995in}{1.605940in}}%
\pgfpathlineto{\pgfqpoint{2.422861in}{1.658774in}}%
\pgfpathlineto{\pgfqpoint{2.424590in}{1.574833in}}%
\pgfpathlineto{\pgfqpoint{2.425455in}{1.586735in}}%
\pgfpathlineto{\pgfqpoint{2.426320in}{1.567590in}}%
\pgfpathlineto{\pgfqpoint{2.427184in}{1.595551in}}%
\pgfpathlineto{\pgfqpoint{2.428048in}{1.543874in}}%
\pgfpathlineto{\pgfqpoint{2.428912in}{1.624521in}}%
\pgfpathlineto{\pgfqpoint{2.429777in}{1.559160in}}%
\pgfpathlineto{\pgfqpoint{2.430641in}{1.564087in}}%
\pgfpathlineto{\pgfqpoint{2.432370in}{1.531585in}}%
\pgfpathlineto{\pgfqpoint{2.433233in}{1.564384in}}%
\pgfpathlineto{\pgfqpoint{2.434097in}{1.538918in}}%
\pgfpathlineto{\pgfqpoint{2.434962in}{1.552304in}}%
\pgfpathlineto{\pgfqpoint{2.435826in}{1.589970in}}%
\pgfpathlineto{\pgfqpoint{2.436690in}{1.559755in}}%
\pgfpathlineto{\pgfqpoint{2.437555in}{1.590119in}}%
\pgfpathlineto{\pgfqpoint{2.438420in}{1.517902in}}%
\pgfpathlineto{\pgfqpoint{2.440150in}{1.622323in}}%
\pgfpathlineto{\pgfqpoint{2.441015in}{1.546812in}}%
\pgfpathlineto{\pgfqpoint{2.441880in}{1.589524in}}%
\pgfpathlineto{\pgfqpoint{2.442744in}{1.586527in}}%
\pgfpathlineto{\pgfqpoint{2.444474in}{1.534761in}}%
\pgfpathlineto{\pgfqpoint{2.445339in}{1.653133in}}%
\pgfpathlineto{\pgfqpoint{2.446201in}{1.580232in}}%
\pgfpathlineto{\pgfqpoint{2.447066in}{1.594570in}}%
\pgfpathlineto{\pgfqpoint{2.448795in}{1.611577in}}%
\pgfpathlineto{\pgfqpoint{2.449658in}{1.575305in}}%
\pgfpathlineto{\pgfqpoint{2.450524in}{1.607008in}}%
\pgfpathlineto{\pgfqpoint{2.451388in}{1.575751in}}%
\pgfpathlineto{\pgfqpoint{2.452253in}{1.638885in}}%
\pgfpathlineto{\pgfqpoint{2.453982in}{1.592937in}}%
\pgfpathlineto{\pgfqpoint{2.454848in}{1.617040in}}%
\pgfpathlineto{\pgfqpoint{2.455713in}{1.606651in}}%
\pgfpathlineto{\pgfqpoint{2.457441in}{1.655212in}}%
\pgfpathlineto{\pgfqpoint{2.458306in}{1.601545in}}%
\pgfpathlineto{\pgfqpoint{2.460035in}{1.631288in}}%
\pgfpathlineto{\pgfqpoint{2.460899in}{1.626004in}}%
\pgfpathlineto{\pgfqpoint{2.462629in}{1.696708in}}%
\pgfpathlineto{\pgfqpoint{2.464360in}{1.559160in}}%
\pgfpathlineto{\pgfqpoint{2.465225in}{1.629150in}}%
\pgfpathlineto{\pgfqpoint{2.466090in}{1.564295in}}%
\pgfpathlineto{\pgfqpoint{2.467821in}{1.608908in}}%
\pgfpathlineto{\pgfqpoint{2.468685in}{1.579938in}}%
\pgfpathlineto{\pgfqpoint{2.469550in}{1.615794in}}%
\pgfpathlineto{\pgfqpoint{2.470414in}{1.552780in}}%
\pgfpathlineto{\pgfqpoint{2.471277in}{1.565098in}}%
\pgfpathlineto{\pgfqpoint{2.472140in}{1.555777in}}%
\pgfpathlineto{\pgfqpoint{2.473871in}{1.629150in}}%
\pgfpathlineto{\pgfqpoint{2.474737in}{1.576019in}}%
\pgfpathlineto{\pgfqpoint{2.475602in}{1.617129in}}%
\pgfpathlineto{\pgfqpoint{2.476468in}{1.565630in}}%
\pgfpathlineto{\pgfqpoint{2.477334in}{1.663106in}}%
\pgfpathlineto{\pgfqpoint{2.478200in}{1.628912in}}%
\pgfpathlineto{\pgfqpoint{2.479066in}{1.656636in}}%
\pgfpathlineto{\pgfqpoint{2.479931in}{1.627488in}}%
\pgfpathlineto{\pgfqpoint{2.480797in}{1.632891in}}%
\pgfpathlineto{\pgfqpoint{2.481662in}{1.615139in}}%
\pgfpathlineto{\pgfqpoint{2.483391in}{1.644287in}}%
\pgfpathlineto{\pgfqpoint{2.485119in}{1.610867in}}%
\pgfpathlineto{\pgfqpoint{2.486849in}{1.680619in}}%
\pgfpathlineto{\pgfqpoint{2.487715in}{1.666787in}}%
\pgfpathlineto{\pgfqpoint{2.489447in}{1.545566in}}%
\pgfpathlineto{\pgfqpoint{2.490310in}{1.633898in}}%
\pgfpathlineto{\pgfqpoint{2.491175in}{1.633069in}}%
\pgfpathlineto{\pgfqpoint{2.492039in}{1.570293in}}%
\pgfpathlineto{\pgfqpoint{2.493769in}{1.641677in}}%
\pgfpathlineto{\pgfqpoint{2.494635in}{1.580176in}}%
\pgfpathlineto{\pgfqpoint{2.495501in}{1.655212in}}%
\pgfpathlineto{\pgfqpoint{2.496365in}{1.564325in}}%
\pgfpathlineto{\pgfqpoint{2.497231in}{1.625707in}}%
\pgfpathlineto{\pgfqpoint{2.498096in}{1.562960in}}%
\pgfpathlineto{\pgfqpoint{2.498962in}{1.629418in}}%
\pgfpathlineto{\pgfqpoint{2.500692in}{1.578157in}}%
\pgfpathlineto{\pgfqpoint{2.501555in}{1.604189in}}%
\pgfpathlineto{\pgfqpoint{2.502421in}{1.576198in}}%
\pgfpathlineto{\pgfqpoint{2.503285in}{1.600478in}}%
\pgfpathlineto{\pgfqpoint{2.504149in}{1.600121in}}%
\pgfpathlineto{\pgfqpoint{2.505014in}{1.609027in}}%
\pgfpathlineto{\pgfqpoint{2.505880in}{1.569192in}}%
\pgfpathlineto{\pgfqpoint{2.507611in}{1.609324in}}%
\pgfpathlineto{\pgfqpoint{2.508476in}{1.571628in}}%
\pgfpathlineto{\pgfqpoint{2.509340in}{1.606000in}}%
\pgfpathlineto{\pgfqpoint{2.510205in}{1.594127in}}%
\pgfpathlineto{\pgfqpoint{2.511069in}{1.516805in}}%
\pgfpathlineto{\pgfqpoint{2.511934in}{1.587181in}}%
\pgfpathlineto{\pgfqpoint{2.512798in}{1.532329in}}%
\pgfpathlineto{\pgfqpoint{2.514530in}{1.607662in}}%
\pgfpathlineto{\pgfqpoint{2.515396in}{1.597213in}}%
\pgfpathlineto{\pgfqpoint{2.517125in}{1.646901in}}%
\pgfpathlineto{\pgfqpoint{2.517990in}{1.513540in}}%
\pgfpathlineto{\pgfqpoint{2.519720in}{1.617694in}}%
\pgfpathlineto{\pgfqpoint{2.522317in}{1.580355in}}%
\pgfpathlineto{\pgfqpoint{2.524046in}{1.665244in}}%
\pgfpathlineto{\pgfqpoint{2.524912in}{1.651947in}}%
\pgfpathlineto{\pgfqpoint{2.525777in}{1.621316in}}%
\pgfpathlineto{\pgfqpoint{2.526643in}{1.643280in}}%
\pgfpathlineto{\pgfqpoint{2.527509in}{1.597303in}}%
\pgfpathlineto{\pgfqpoint{2.528374in}{1.659547in}}%
\pgfpathlineto{\pgfqpoint{2.530102in}{1.556253in}}%
\pgfpathlineto{\pgfqpoint{2.531831in}{1.621970in}}%
\pgfpathlineto{\pgfqpoint{2.533562in}{1.543845in}}%
\pgfpathlineto{\pgfqpoint{2.534427in}{1.613065in}}%
\pgfpathlineto{\pgfqpoint{2.537021in}{1.540729in}}%
\pgfpathlineto{\pgfqpoint{2.537886in}{1.634910in}}%
\pgfpathlineto{\pgfqpoint{2.541346in}{1.540996in}}%
\pgfpathlineto{\pgfqpoint{2.542211in}{1.587241in}}%
\pgfpathlineto{\pgfqpoint{2.543077in}{1.549485in}}%
\pgfpathlineto{\pgfqpoint{2.543942in}{1.614429in}}%
\pgfpathlineto{\pgfqpoint{2.544808in}{1.569906in}}%
\pgfpathlineto{\pgfqpoint{2.545672in}{1.583322in}}%
\pgfpathlineto{\pgfqpoint{2.547403in}{1.556134in}}%
\pgfpathlineto{\pgfqpoint{2.548266in}{1.600597in}}%
\pgfpathlineto{\pgfqpoint{2.549131in}{1.588992in}}%
\pgfpathlineto{\pgfqpoint{2.549993in}{1.523781in}}%
\pgfpathlineto{\pgfqpoint{2.551724in}{1.580860in}}%
\pgfpathlineto{\pgfqpoint{2.552590in}{1.594960in}}%
\pgfpathlineto{\pgfqpoint{2.553455in}{1.594306in}}%
\pgfpathlineto{\pgfqpoint{2.554322in}{1.585400in}}%
\pgfpathlineto{\pgfqpoint{2.555188in}{1.625056in}}%
\pgfpathlineto{\pgfqpoint{2.556916in}{1.595551in}}%
\pgfpathlineto{\pgfqpoint{2.557782in}{1.650225in}}%
\pgfpathlineto{\pgfqpoint{2.558645in}{1.594276in}}%
\pgfpathlineto{\pgfqpoint{2.559511in}{1.650106in}}%
\pgfpathlineto{\pgfqpoint{2.560377in}{1.541234in}}%
\pgfpathlineto{\pgfqpoint{2.561241in}{1.568333in}}%
\pgfpathlineto{\pgfqpoint{2.562106in}{1.592465in}}%
\pgfpathlineto{\pgfqpoint{2.562970in}{1.670944in}}%
\pgfpathlineto{\pgfqpoint{2.564700in}{1.623335in}}%
\pgfpathlineto{\pgfqpoint{2.565564in}{1.665482in}}%
\pgfpathlineto{\pgfqpoint{2.566428in}{1.629567in}}%
\pgfpathlineto{\pgfqpoint{2.567292in}{1.637877in}}%
\pgfpathlineto{\pgfqpoint{2.568157in}{1.609205in}}%
\pgfpathlineto{\pgfqpoint{2.569022in}{1.627904in}}%
\pgfpathlineto{\pgfqpoint{2.569888in}{1.569728in}}%
\pgfpathlineto{\pgfqpoint{2.570753in}{1.598786in}}%
\pgfpathlineto{\pgfqpoint{2.571618in}{1.535415in}}%
\pgfpathlineto{\pgfqpoint{2.572482in}{1.611522in}}%
\pgfpathlineto{\pgfqpoint{2.573346in}{1.556134in}}%
\pgfpathlineto{\pgfqpoint{2.575939in}{1.639242in}}%
\pgfpathlineto{\pgfqpoint{2.576802in}{1.560763in}}%
\pgfpathlineto{\pgfqpoint{2.577665in}{1.598459in}}%
\pgfpathlineto{\pgfqpoint{2.578530in}{1.557498in}}%
\pgfpathlineto{\pgfqpoint{2.579394in}{1.561179in}}%
\pgfpathlineto{\pgfqpoint{2.581123in}{1.638472in}}%
\pgfpathlineto{\pgfqpoint{2.582854in}{1.558093in}}%
\pgfpathlineto{\pgfqpoint{2.585448in}{1.637996in}}%
\pgfpathlineto{\pgfqpoint{2.588042in}{1.526272in}}%
\pgfpathlineto{\pgfqpoint{2.588907in}{1.622323in}}%
\pgfpathlineto{\pgfqpoint{2.589773in}{1.558893in}}%
\pgfpathlineto{\pgfqpoint{2.590637in}{1.588844in}}%
\pgfpathlineto{\pgfqpoint{2.591500in}{1.568304in}}%
\pgfpathlineto{\pgfqpoint{2.593230in}{1.696708in}}%
\pgfpathlineto{\pgfqpoint{2.594961in}{1.626599in}}%
\pgfpathlineto{\pgfqpoint{2.597557in}{1.724610in}}%
\pgfpathlineto{\pgfqpoint{2.598421in}{1.643934in}}%
\pgfpathlineto{\pgfqpoint{2.600150in}{1.705613in}}%
\pgfpathlineto{\pgfqpoint{2.601015in}{1.706149in}}%
\pgfpathlineto{\pgfqpoint{2.601880in}{1.716240in}}%
\pgfpathlineto{\pgfqpoint{2.602742in}{1.652720in}}%
\pgfpathlineto{\pgfqpoint{2.603605in}{1.678306in}}%
\pgfpathlineto{\pgfqpoint{2.604469in}{1.618408in}}%
\pgfpathlineto{\pgfqpoint{2.605334in}{1.666077in}}%
\pgfpathlineto{\pgfqpoint{2.606197in}{1.659696in}}%
\pgfpathlineto{\pgfqpoint{2.607063in}{1.623811in}}%
\pgfpathlineto{\pgfqpoint{2.607928in}{1.679909in}}%
\pgfpathlineto{\pgfqpoint{2.608793in}{1.656431in}}%
\pgfpathlineto{\pgfqpoint{2.609657in}{1.700449in}}%
\pgfpathlineto{\pgfqpoint{2.611387in}{1.669579in}}%
\pgfpathlineto{\pgfqpoint{2.613117in}{1.712202in}}%
\pgfpathlineto{\pgfqpoint{2.613981in}{1.659547in}}%
\pgfpathlineto{\pgfqpoint{2.614846in}{1.685133in}}%
\pgfpathlineto{\pgfqpoint{2.615710in}{1.650315in}}%
\pgfpathlineto{\pgfqpoint{2.617439in}{1.707216in}}%
\pgfpathlineto{\pgfqpoint{2.618301in}{1.715645in}}%
\pgfpathlineto{\pgfqpoint{2.620031in}{1.657350in}}%
\pgfpathlineto{\pgfqpoint{2.621760in}{1.674149in}}%
\pgfpathlineto{\pgfqpoint{2.622625in}{1.658952in}}%
\pgfpathlineto{\pgfqpoint{2.623489in}{1.681154in}}%
\pgfpathlineto{\pgfqpoint{2.625219in}{1.591190in}}%
\pgfpathlineto{\pgfqpoint{2.626084in}{1.600835in}}%
\pgfpathlineto{\pgfqpoint{2.626950in}{1.686974in}}%
\pgfpathlineto{\pgfqpoint{2.627816in}{1.647318in}}%
\pgfpathlineto{\pgfqpoint{2.628682in}{1.652423in}}%
\pgfpathlineto{\pgfqpoint{2.630413in}{1.635564in}}%
\pgfpathlineto{\pgfqpoint{2.631279in}{1.681571in}}%
\pgfpathlineto{\pgfqpoint{2.632145in}{1.628767in}}%
\pgfpathlineto{\pgfqpoint{2.633874in}{1.669877in}}%
\pgfpathlineto{\pgfqpoint{2.634739in}{1.691573in}}%
\pgfpathlineto{\pgfqpoint{2.636470in}{1.646961in}}%
\pgfpathlineto{\pgfqpoint{2.637336in}{1.723658in}}%
\pgfpathlineto{\pgfqpoint{2.638200in}{1.681214in}}%
\pgfpathlineto{\pgfqpoint{2.639064in}{1.733099in}}%
\pgfpathlineto{\pgfqpoint{2.639929in}{1.633724in}}%
\pgfpathlineto{\pgfqpoint{2.642524in}{1.725677in}}%
\pgfpathlineto{\pgfqpoint{2.643388in}{1.694689in}}%
\pgfpathlineto{\pgfqpoint{2.644252in}{1.706740in}}%
\pgfpathlineto{\pgfqpoint{2.645115in}{1.640015in}}%
\pgfpathlineto{\pgfqpoint{2.645978in}{1.734047in}}%
\pgfpathlineto{\pgfqpoint{2.647710in}{1.694749in}}%
\pgfpathlineto{\pgfqpoint{2.648575in}{1.718434in}}%
\pgfpathlineto{\pgfqpoint{2.649441in}{1.693265in}}%
\pgfpathlineto{\pgfqpoint{2.650307in}{1.715051in}}%
\pgfpathlineto{\pgfqpoint{2.651172in}{1.669222in}}%
\pgfpathlineto{\pgfqpoint{2.652037in}{1.766936in}}%
\pgfpathlineto{\pgfqpoint{2.652902in}{1.682697in}}%
\pgfpathlineto{\pgfqpoint{2.653767in}{1.682935in}}%
\pgfpathlineto{\pgfqpoint{2.654633in}{1.686974in}}%
\pgfpathlineto{\pgfqpoint{2.655499in}{1.679314in}}%
\pgfpathlineto{\pgfqpoint{2.657231in}{1.768836in}}%
\pgfpathlineto{\pgfqpoint{2.658097in}{1.681452in}}%
\pgfpathlineto{\pgfqpoint{2.658961in}{1.689108in}}%
\pgfpathlineto{\pgfqpoint{2.659827in}{1.682638in}}%
\pgfpathlineto{\pgfqpoint{2.660689in}{1.633129in}}%
\pgfpathlineto{\pgfqpoint{2.662417in}{1.706324in}}%
\pgfpathlineto{\pgfqpoint{2.663283in}{1.672636in}}%
\pgfpathlineto{\pgfqpoint{2.664149in}{1.682876in}}%
\pgfpathlineto{\pgfqpoint{2.665014in}{1.629329in}}%
\pgfpathlineto{\pgfqpoint{2.665878in}{1.651977in}}%
\pgfpathlineto{\pgfqpoint{2.666743in}{1.648087in}}%
\pgfpathlineto{\pgfqpoint{2.669338in}{1.553758in}}%
\pgfpathlineto{\pgfqpoint{2.670204in}{1.637788in}}%
\pgfpathlineto{\pgfqpoint{2.672800in}{1.542182in}}%
\pgfpathlineto{\pgfqpoint{2.673666in}{1.592937in}}%
\pgfpathlineto{\pgfqpoint{2.674531in}{1.562960in}}%
\pgfpathlineto{\pgfqpoint{2.675397in}{1.640312in}}%
\pgfpathlineto{\pgfqpoint{2.676261in}{1.611224in}}%
\pgfpathlineto{\pgfqpoint{2.677125in}{1.670587in}}%
\pgfpathlineto{\pgfqpoint{2.677991in}{1.614191in}}%
\pgfpathlineto{\pgfqpoint{2.678857in}{1.619594in}}%
\pgfpathlineto{\pgfqpoint{2.679722in}{1.634850in}}%
\pgfpathlineto{\pgfqpoint{2.681452in}{1.721937in}}%
\pgfpathlineto{\pgfqpoint{2.683182in}{1.595075in}}%
\pgfpathlineto{\pgfqpoint{2.684047in}{1.592406in}}%
\pgfpathlineto{\pgfqpoint{2.684912in}{1.621107in}}%
\pgfpathlineto{\pgfqpoint{2.685777in}{1.614013in}}%
\pgfpathlineto{\pgfqpoint{2.686642in}{1.605524in}}%
\pgfpathlineto{\pgfqpoint{2.687507in}{1.620959in}}%
\pgfpathlineto{\pgfqpoint{2.688372in}{1.662276in}}%
\pgfpathlineto{\pgfqpoint{2.690102in}{1.595194in}}%
\pgfpathlineto{\pgfqpoint{2.690968in}{1.588725in}}%
\pgfpathlineto{\pgfqpoint{2.692698in}{1.633307in}}%
\pgfpathlineto{\pgfqpoint{2.693563in}{1.573111in}}%
\pgfpathlineto{\pgfqpoint{2.694429in}{1.623245in}}%
\pgfpathlineto{\pgfqpoint{2.695292in}{1.597451in}}%
\pgfpathlineto{\pgfqpoint{2.696157in}{1.643399in}}%
\pgfpathlineto{\pgfqpoint{2.697023in}{1.570914in}}%
\pgfpathlineto{\pgfqpoint{2.697889in}{1.578633in}}%
\pgfpathlineto{\pgfqpoint{2.698752in}{1.561001in}}%
\pgfpathlineto{\pgfqpoint{2.699618in}{1.573052in}}%
\pgfpathlineto{\pgfqpoint{2.701346in}{1.532388in}}%
\pgfpathlineto{\pgfqpoint{2.702209in}{1.587181in}}%
\pgfpathlineto{\pgfqpoint{2.703074in}{1.566404in}}%
\pgfpathlineto{\pgfqpoint{2.705670in}{1.648801in}}%
\pgfpathlineto{\pgfqpoint{2.707401in}{1.629567in}}%
\pgfpathlineto{\pgfqpoint{2.708268in}{1.537404in}}%
\pgfpathlineto{\pgfqpoint{2.709995in}{1.621018in}}%
\pgfpathlineto{\pgfqpoint{2.711726in}{1.597213in}}%
\pgfpathlineto{\pgfqpoint{2.713458in}{1.656904in}}%
\pgfpathlineto{\pgfqpoint{2.714324in}{1.656104in}}%
\pgfpathlineto{\pgfqpoint{2.715189in}{1.623275in}}%
\pgfpathlineto{\pgfqpoint{2.716054in}{1.637817in}}%
\pgfpathlineto{\pgfqpoint{2.716920in}{1.727102in}}%
\pgfpathlineto{\pgfqpoint{2.718650in}{1.574416in}}%
\pgfpathlineto{\pgfqpoint{2.719515in}{1.592465in}}%
\pgfpathlineto{\pgfqpoint{2.720379in}{1.571449in}}%
\pgfpathlineto{\pgfqpoint{2.721245in}{1.648444in}}%
\pgfpathlineto{\pgfqpoint{2.722110in}{1.627904in}}%
\pgfpathlineto{\pgfqpoint{2.722975in}{1.600805in}}%
\pgfpathlineto{\pgfqpoint{2.723839in}{1.680857in}}%
\pgfpathlineto{\pgfqpoint{2.725571in}{1.587241in}}%
\pgfpathlineto{\pgfqpoint{2.726436in}{1.611343in}}%
\pgfpathlineto{\pgfqpoint{2.727301in}{1.530816in}}%
\pgfpathlineto{\pgfqpoint{2.728166in}{1.533932in}}%
\pgfpathlineto{\pgfqpoint{2.729031in}{1.543845in}}%
\pgfpathlineto{\pgfqpoint{2.729896in}{1.588368in}}%
\pgfpathlineto{\pgfqpoint{2.730761in}{1.561417in}}%
\pgfpathlineto{\pgfqpoint{2.731626in}{1.610391in}}%
\pgfpathlineto{\pgfqpoint{2.732492in}{1.540401in}}%
\pgfpathlineto{\pgfqpoint{2.733357in}{1.632831in}}%
\pgfpathlineto{\pgfqpoint{2.734222in}{1.621554in}}%
\pgfpathlineto{\pgfqpoint{2.735086in}{1.636691in}}%
\pgfpathlineto{\pgfqpoint{2.735950in}{1.555896in}}%
\pgfpathlineto{\pgfqpoint{2.736813in}{1.636126in}}%
\pgfpathlineto{\pgfqpoint{2.737677in}{1.632891in}}%
\pgfpathlineto{\pgfqpoint{2.738539in}{1.545388in}}%
\pgfpathlineto{\pgfqpoint{2.739404in}{1.594391in}}%
\pgfpathlineto{\pgfqpoint{2.740269in}{1.584032in}}%
\pgfpathlineto{\pgfqpoint{2.741132in}{1.581303in}}%
\pgfpathlineto{\pgfqpoint{2.741997in}{1.601635in}}%
\pgfpathlineto{\pgfqpoint{2.742860in}{1.593175in}}%
\pgfpathlineto{\pgfqpoint{2.744591in}{1.651888in}}%
\pgfpathlineto{\pgfqpoint{2.745455in}{1.665422in}}%
\pgfpathlineto{\pgfqpoint{2.746322in}{1.583381in}}%
\pgfpathlineto{\pgfqpoint{2.747187in}{1.597094in}}%
\pgfpathlineto{\pgfqpoint{2.748053in}{1.559458in}}%
\pgfpathlineto{\pgfqpoint{2.750649in}{1.628023in}}%
\pgfpathlineto{\pgfqpoint{2.751515in}{1.545269in}}%
\pgfpathlineto{\pgfqpoint{2.753245in}{1.638353in}}%
\pgfpathlineto{\pgfqpoint{2.754111in}{1.564028in}}%
\pgfpathlineto{\pgfqpoint{2.755840in}{1.633277in}}%
\pgfpathlineto{\pgfqpoint{2.757572in}{1.581422in}}%
\pgfpathlineto{\pgfqpoint{2.758437in}{1.572814in}}%
\pgfpathlineto{\pgfqpoint{2.759302in}{1.651055in}}%
\pgfpathlineto{\pgfqpoint{2.760166in}{1.646306in}}%
\pgfpathlineto{\pgfqpoint{2.761898in}{1.504784in}}%
\pgfpathlineto{\pgfqpoint{2.764491in}{1.601307in}}%
\pgfpathlineto{\pgfqpoint{2.765356in}{1.560227in}}%
\pgfpathlineto{\pgfqpoint{2.766220in}{1.568657in}}%
\pgfpathlineto{\pgfqpoint{2.767085in}{1.562365in}}%
\pgfpathlineto{\pgfqpoint{2.767951in}{1.640636in}}%
\pgfpathlineto{\pgfqpoint{2.769681in}{1.534404in}}%
\pgfpathlineto{\pgfqpoint{2.770546in}{1.609707in}}%
\pgfpathlineto{\pgfqpoint{2.771411in}{1.546633in}}%
\pgfpathlineto{\pgfqpoint{2.772275in}{1.628615in}}%
\pgfpathlineto{\pgfqpoint{2.773139in}{1.548890in}}%
\pgfpathlineto{\pgfqpoint{2.774004in}{1.602438in}}%
\pgfpathlineto{\pgfqpoint{2.774866in}{1.536809in}}%
\pgfpathlineto{\pgfqpoint{2.775728in}{1.607008in}}%
\pgfpathlineto{\pgfqpoint{2.776593in}{1.577030in}}%
\pgfpathlineto{\pgfqpoint{2.779188in}{1.648682in}}%
\pgfpathlineto{\pgfqpoint{2.780052in}{1.557528in}}%
\pgfpathlineto{\pgfqpoint{2.780917in}{1.626837in}}%
\pgfpathlineto{\pgfqpoint{2.781780in}{1.595849in}}%
\pgfpathlineto{\pgfqpoint{2.782645in}{1.635029in}}%
\pgfpathlineto{\pgfqpoint{2.784371in}{1.566463in}}%
\pgfpathlineto{\pgfqpoint{2.786101in}{1.666965in}}%
\pgfpathlineto{\pgfqpoint{2.786965in}{1.542272in}}%
\pgfpathlineto{\pgfqpoint{2.787828in}{1.643339in}}%
\pgfpathlineto{\pgfqpoint{2.788692in}{1.628737in}}%
\pgfpathlineto{\pgfqpoint{2.789556in}{1.621375in}}%
\pgfpathlineto{\pgfqpoint{2.791286in}{1.555185in}}%
\pgfpathlineto{\pgfqpoint{2.793017in}{1.617396in}}%
\pgfpathlineto{\pgfqpoint{2.793883in}{1.576911in}}%
\pgfpathlineto{\pgfqpoint{2.794748in}{1.612767in}}%
\pgfpathlineto{\pgfqpoint{2.795613in}{1.532448in}}%
\pgfpathlineto{\pgfqpoint{2.796478in}{1.542420in}}%
\pgfpathlineto{\pgfqpoint{2.797343in}{1.623097in}}%
\pgfpathlineto{\pgfqpoint{2.799072in}{1.541974in}}%
\pgfpathlineto{\pgfqpoint{2.800803in}{1.623037in}}%
\pgfpathlineto{\pgfqpoint{2.801665in}{1.549723in}}%
\pgfpathlineto{\pgfqpoint{2.803395in}{1.601162in}}%
\pgfpathlineto{\pgfqpoint{2.804259in}{1.582433in}}%
\pgfpathlineto{\pgfqpoint{2.805989in}{1.631496in}}%
\pgfpathlineto{\pgfqpoint{2.806856in}{1.626242in}}%
\pgfpathlineto{\pgfqpoint{2.807721in}{1.641975in}}%
\pgfpathlineto{\pgfqpoint{2.809451in}{1.607959in}}%
\pgfpathlineto{\pgfqpoint{2.810316in}{1.642034in}}%
\pgfpathlineto{\pgfqpoint{2.811181in}{1.609681in}}%
\pgfpathlineto{\pgfqpoint{2.812047in}{1.647556in}}%
\pgfpathlineto{\pgfqpoint{2.814641in}{1.568125in}}%
\pgfpathlineto{\pgfqpoint{2.816368in}{1.610748in}}%
\pgfpathlineto{\pgfqpoint{2.818096in}{1.571449in}}%
\pgfpathlineto{\pgfqpoint{2.818961in}{1.589673in}}%
\pgfpathlineto{\pgfqpoint{2.819827in}{1.546276in}}%
\pgfpathlineto{\pgfqpoint{2.821555in}{1.630812in}}%
\pgfpathlineto{\pgfqpoint{2.822421in}{1.618672in}}%
\pgfpathlineto{\pgfqpoint{2.823283in}{1.622978in}}%
\pgfpathlineto{\pgfqpoint{2.824146in}{1.646425in}}%
\pgfpathlineto{\pgfqpoint{2.825008in}{1.576108in}}%
\pgfpathlineto{\pgfqpoint{2.825873in}{1.610272in}}%
\pgfpathlineto{\pgfqpoint{2.826739in}{1.605345in}}%
\pgfpathlineto{\pgfqpoint{2.827603in}{1.598191in}}%
\pgfpathlineto{\pgfqpoint{2.828468in}{1.574179in}}%
\pgfpathlineto{\pgfqpoint{2.829334in}{1.625469in}}%
\pgfpathlineto{\pgfqpoint{2.830199in}{1.600419in}}%
\pgfpathlineto{\pgfqpoint{2.831927in}{1.648563in}}%
\pgfpathlineto{\pgfqpoint{2.833654in}{1.580176in}}%
\pgfpathlineto{\pgfqpoint{2.834521in}{1.626064in}}%
\pgfpathlineto{\pgfqpoint{2.837117in}{1.564890in}}%
\pgfpathlineto{\pgfqpoint{2.837982in}{1.681809in}}%
\pgfpathlineto{\pgfqpoint{2.839712in}{1.585043in}}%
\pgfpathlineto{\pgfqpoint{2.841442in}{1.614132in}}%
\pgfpathlineto{\pgfqpoint{2.842307in}{1.517485in}}%
\pgfpathlineto{\pgfqpoint{2.843172in}{1.586170in}}%
\pgfpathlineto{\pgfqpoint{2.844037in}{1.562336in}}%
\pgfpathlineto{\pgfqpoint{2.845764in}{1.636096in}}%
\pgfpathlineto{\pgfqpoint{2.846628in}{1.606799in}}%
\pgfpathlineto{\pgfqpoint{2.847492in}{1.632891in}}%
\pgfpathlineto{\pgfqpoint{2.848356in}{1.626183in}}%
\pgfpathlineto{\pgfqpoint{2.849221in}{1.610927in}}%
\pgfpathlineto{\pgfqpoint{2.851815in}{1.696351in}}%
\pgfpathlineto{\pgfqpoint{2.852680in}{1.621375in}}%
\pgfpathlineto{\pgfqpoint{2.853545in}{1.631377in}}%
\pgfpathlineto{\pgfqpoint{2.855272in}{1.698430in}}%
\pgfpathlineto{\pgfqpoint{2.856137in}{1.659101in}}%
\pgfpathlineto{\pgfqpoint{2.857003in}{1.662871in}}%
\pgfpathlineto{\pgfqpoint{2.857868in}{1.700032in}}%
\pgfpathlineto{\pgfqpoint{2.858734in}{1.628559in}}%
\pgfpathlineto{\pgfqpoint{2.859599in}{1.702940in}}%
\pgfpathlineto{\pgfqpoint{2.860465in}{1.633218in}}%
\pgfpathlineto{\pgfqpoint{2.862193in}{1.710897in}}%
\pgfpathlineto{\pgfqpoint{2.863921in}{1.657350in}}%
\pgfpathlineto{\pgfqpoint{2.864787in}{1.722770in}}%
\pgfpathlineto{\pgfqpoint{2.865652in}{1.693622in}}%
\pgfpathlineto{\pgfqpoint{2.866517in}{1.726153in}}%
\pgfpathlineto{\pgfqpoint{2.868246in}{1.693503in}}%
\pgfpathlineto{\pgfqpoint{2.869111in}{1.732921in}}%
\pgfpathlineto{\pgfqpoint{2.869976in}{1.659458in}}%
\pgfpathlineto{\pgfqpoint{2.870840in}{1.669936in}}%
\pgfpathlineto{\pgfqpoint{2.871704in}{1.664652in}}%
\pgfpathlineto{\pgfqpoint{2.873433in}{1.698608in}}%
\pgfpathlineto{\pgfqpoint{2.874296in}{1.677533in}}%
\pgfpathlineto{\pgfqpoint{2.875161in}{1.720275in}}%
\pgfpathlineto{\pgfqpoint{2.876025in}{1.545864in}}%
\pgfpathlineto{\pgfqpoint{2.876888in}{1.578276in}}%
\pgfpathlineto{\pgfqpoint{2.877753in}{1.571152in}}%
\pgfpathlineto{\pgfqpoint{2.878619in}{1.590327in}}%
\pgfpathlineto{\pgfqpoint{2.879484in}{1.547079in}}%
\pgfpathlineto{\pgfqpoint{2.880348in}{1.557022in}}%
\pgfpathlineto{\pgfqpoint{2.881212in}{1.648444in}}%
\pgfpathlineto{\pgfqpoint{2.882076in}{1.637639in}}%
\pgfpathlineto{\pgfqpoint{2.882941in}{1.609146in}}%
\pgfpathlineto{\pgfqpoint{2.883803in}{1.648325in}}%
\pgfpathlineto{\pgfqpoint{2.885532in}{1.471419in}}%
\pgfpathlineto{\pgfqpoint{2.887263in}{1.631050in}}%
\pgfpathlineto{\pgfqpoint{2.888127in}{1.630340in}}%
\pgfpathlineto{\pgfqpoint{2.889857in}{1.587836in}}%
\pgfpathlineto{\pgfqpoint{2.890721in}{1.603271in}}%
\pgfpathlineto{\pgfqpoint{2.891587in}{1.652810in}}%
\pgfpathlineto{\pgfqpoint{2.892452in}{1.650761in}}%
\pgfpathlineto{\pgfqpoint{2.893318in}{1.530667in}}%
\pgfpathlineto{\pgfqpoint{2.895047in}{1.597094in}}%
\pgfpathlineto{\pgfqpoint{2.895913in}{1.586289in}}%
\pgfpathlineto{\pgfqpoint{2.896779in}{1.600776in}}%
\pgfpathlineto{\pgfqpoint{2.898505in}{1.578217in}}%
\pgfpathlineto{\pgfqpoint{2.899368in}{1.584805in}}%
\pgfpathlineto{\pgfqpoint{2.900234in}{1.519564in}}%
\pgfpathlineto{\pgfqpoint{2.901964in}{1.700980in}}%
\pgfpathlineto{\pgfqpoint{2.902827in}{1.577562in}}%
\pgfpathlineto{\pgfqpoint{2.903693in}{1.603118in}}%
\pgfpathlineto{\pgfqpoint{2.904557in}{1.632058in}}%
\pgfpathlineto{\pgfqpoint{2.906286in}{1.540044in}}%
\pgfpathlineto{\pgfqpoint{2.909746in}{1.660376in}}%
\pgfpathlineto{\pgfqpoint{2.910611in}{1.618940in}}%
\pgfpathlineto{\pgfqpoint{2.911476in}{1.684776in}}%
\pgfpathlineto{\pgfqpoint{2.912341in}{1.610570in}}%
\pgfpathlineto{\pgfqpoint{2.913205in}{1.634493in}}%
\pgfpathlineto{\pgfqpoint{2.914071in}{1.579522in}}%
\pgfpathlineto{\pgfqpoint{2.914936in}{1.615377in}}%
\pgfpathlineto{\pgfqpoint{2.915801in}{1.539747in}}%
\pgfpathlineto{\pgfqpoint{2.916667in}{1.584330in}}%
\pgfpathlineto{\pgfqpoint{2.917532in}{1.688602in}}%
\pgfpathlineto{\pgfqpoint{2.920128in}{1.576049in}}%
\pgfpathlineto{\pgfqpoint{2.921858in}{1.635798in}}%
\pgfpathlineto{\pgfqpoint{2.922722in}{1.567471in}}%
\pgfpathlineto{\pgfqpoint{2.923588in}{1.648325in}}%
\pgfpathlineto{\pgfqpoint{2.926183in}{1.574892in}}%
\pgfpathlineto{\pgfqpoint{2.927911in}{1.622502in}}%
\pgfpathlineto{\pgfqpoint{2.928776in}{1.632355in}}%
\pgfpathlineto{\pgfqpoint{2.931370in}{1.737015in}}%
\pgfpathlineto{\pgfqpoint{2.932235in}{1.743901in}}%
\pgfpathlineto{\pgfqpoint{2.933100in}{1.724015in}}%
\pgfpathlineto{\pgfqpoint{2.933966in}{1.677473in}}%
\pgfpathlineto{\pgfqpoint{2.934831in}{1.708640in}}%
\pgfpathlineto{\pgfqpoint{2.936558in}{1.674179in}}%
\pgfpathlineto{\pgfqpoint{2.937423in}{1.727161in}}%
\pgfpathlineto{\pgfqpoint{2.938287in}{1.718970in}}%
\pgfpathlineto{\pgfqpoint{2.941744in}{1.682935in}}%
\pgfpathlineto{\pgfqpoint{2.942610in}{1.700389in}}%
\pgfpathlineto{\pgfqpoint{2.944340in}{1.648325in}}%
\pgfpathlineto{\pgfqpoint{2.945203in}{1.710778in}}%
\pgfpathlineto{\pgfqpoint{2.946068in}{1.680236in}}%
\pgfpathlineto{\pgfqpoint{2.947799in}{1.702765in}}%
\pgfpathlineto{\pgfqpoint{2.948660in}{1.646697in}}%
\pgfpathlineto{\pgfqpoint{2.949527in}{1.711314in}}%
\pgfpathlineto{\pgfqpoint{2.951255in}{1.546221in}}%
\pgfpathlineto{\pgfqpoint{2.952121in}{1.618467in}}%
\pgfpathlineto{\pgfqpoint{2.953848in}{1.528826in}}%
\pgfpathlineto{\pgfqpoint{2.956440in}{1.656283in}}%
\pgfpathlineto{\pgfqpoint{2.957306in}{1.539513in}}%
\pgfpathlineto{\pgfqpoint{2.958170in}{1.570442in}}%
\pgfpathlineto{\pgfqpoint{2.959036in}{1.537732in}}%
\pgfpathlineto{\pgfqpoint{2.959901in}{1.575130in}}%
\pgfpathlineto{\pgfqpoint{2.960766in}{1.532478in}}%
\pgfpathlineto{\pgfqpoint{2.962497in}{1.632240in}}%
\pgfpathlineto{\pgfqpoint{2.964228in}{1.541766in}}%
\pgfpathlineto{\pgfqpoint{2.965957in}{1.593770in}}%
\pgfpathlineto{\pgfqpoint{2.966821in}{1.613303in}}%
\pgfpathlineto{\pgfqpoint{2.967687in}{1.544112in}}%
\pgfpathlineto{\pgfqpoint{2.968551in}{1.601014in}}%
\pgfpathlineto{\pgfqpoint{2.970278in}{1.549783in}}%
\pgfpathlineto{\pgfqpoint{2.971143in}{1.642212in}}%
\pgfpathlineto{\pgfqpoint{2.972009in}{1.635267in}}%
\pgfpathlineto{\pgfqpoint{2.973738in}{1.590803in}}%
\pgfpathlineto{\pgfqpoint{2.974603in}{1.607900in}}%
\pgfpathlineto{\pgfqpoint{2.975468in}{1.573349in}}%
\pgfpathlineto{\pgfqpoint{2.976333in}{1.619832in}}%
\pgfpathlineto{\pgfqpoint{2.977198in}{1.580444in}}%
\pgfpathlineto{\pgfqpoint{2.978062in}{1.583857in}}%
\pgfpathlineto{\pgfqpoint{2.978927in}{1.607483in}}%
\pgfpathlineto{\pgfqpoint{2.979791in}{1.535356in}}%
\pgfpathlineto{\pgfqpoint{2.980655in}{1.614429in}}%
\pgfpathlineto{\pgfqpoint{2.981520in}{1.535534in}}%
\pgfpathlineto{\pgfqpoint{2.982384in}{1.582195in}}%
\pgfpathlineto{\pgfqpoint{2.983249in}{1.568185in}}%
\pgfpathlineto{\pgfqpoint{2.984114in}{1.579730in}}%
\pgfpathlineto{\pgfqpoint{2.984979in}{1.561239in}}%
\pgfpathlineto{\pgfqpoint{2.985844in}{1.586943in}}%
\pgfpathlineto{\pgfqpoint{2.986710in}{1.560703in}}%
\pgfpathlineto{\pgfqpoint{2.988440in}{1.602319in}}%
\pgfpathlineto{\pgfqpoint{2.989305in}{1.560882in}}%
\pgfpathlineto{\pgfqpoint{2.990170in}{1.679314in}}%
\pgfpathlineto{\pgfqpoint{2.991898in}{1.545626in}}%
\pgfpathlineto{\pgfqpoint{2.993628in}{1.629656in}}%
\pgfpathlineto{\pgfqpoint{2.994494in}{1.629150in}}%
\pgfpathlineto{\pgfqpoint{2.995360in}{1.637579in}}%
\pgfpathlineto{\pgfqpoint{2.996226in}{1.611756in}}%
\pgfpathlineto{\pgfqpoint{2.997090in}{1.627726in}}%
\pgfpathlineto{\pgfqpoint{2.997954in}{1.617396in}}%
\pgfpathlineto{\pgfqpoint{2.998820in}{1.624134in}}%
\pgfpathlineto{\pgfqpoint{2.999686in}{1.595432in}}%
\pgfpathlineto{\pgfqpoint{3.000551in}{1.619475in}}%
\pgfpathlineto{\pgfqpoint{3.002282in}{1.589851in}}%
\pgfpathlineto{\pgfqpoint{3.004010in}{1.620721in}}%
\pgfpathlineto{\pgfqpoint{3.004875in}{1.649512in}}%
\pgfpathlineto{\pgfqpoint{3.006606in}{1.583084in}}%
\pgfpathlineto{\pgfqpoint{3.007470in}{1.676109in}}%
\pgfpathlineto{\pgfqpoint{3.008335in}{1.643547in}}%
\pgfpathlineto{\pgfqpoint{3.009201in}{1.699556in}}%
\pgfpathlineto{\pgfqpoint{3.010065in}{1.650761in}}%
\pgfpathlineto{\pgfqpoint{3.010930in}{1.693503in}}%
\pgfpathlineto{\pgfqpoint{3.011793in}{1.669579in}}%
\pgfpathlineto{\pgfqpoint{3.014390in}{1.712381in}}%
\pgfpathlineto{\pgfqpoint{3.015255in}{1.715467in}}%
\pgfpathlineto{\pgfqpoint{3.016120in}{1.647318in}}%
\pgfpathlineto{\pgfqpoint{3.016985in}{1.657588in}}%
\pgfpathlineto{\pgfqpoint{3.018714in}{1.706502in}}%
\pgfpathlineto{\pgfqpoint{3.019578in}{1.708700in}}%
\pgfpathlineto{\pgfqpoint{3.021309in}{1.660495in}}%
\pgfpathlineto{\pgfqpoint{3.022174in}{1.719386in}}%
\pgfpathlineto{\pgfqpoint{3.024765in}{1.650136in}}%
\pgfpathlineto{\pgfqpoint{3.026495in}{1.731969in}}%
\pgfpathlineto{\pgfqpoint{3.027361in}{1.728436in}}%
\pgfpathlineto{\pgfqpoint{3.029093in}{1.675990in}}%
\pgfpathlineto{\pgfqpoint{3.029959in}{1.725499in}}%
\pgfpathlineto{\pgfqpoint{3.031687in}{1.615913in}}%
\pgfpathlineto{\pgfqpoint{3.032552in}{1.694749in}}%
\pgfpathlineto{\pgfqpoint{3.033415in}{1.655688in}}%
\pgfpathlineto{\pgfqpoint{3.034281in}{1.731880in}}%
\pgfpathlineto{\pgfqpoint{3.035146in}{1.684181in}}%
\pgfpathlineto{\pgfqpoint{3.036011in}{1.773465in}}%
\pgfpathlineto{\pgfqpoint{3.038605in}{1.540401in}}%
\pgfpathlineto{\pgfqpoint{3.040334in}{1.616802in}}%
\pgfpathlineto{\pgfqpoint{3.041199in}{1.709826in}}%
\pgfpathlineto{\pgfqpoint{3.042064in}{1.678778in}}%
\pgfpathlineto{\pgfqpoint{3.042930in}{1.709469in}}%
\pgfpathlineto{\pgfqpoint{3.043794in}{1.654264in}}%
\pgfpathlineto{\pgfqpoint{3.045523in}{1.715407in}}%
\pgfpathlineto{\pgfqpoint{3.046388in}{1.682697in}}%
\pgfpathlineto{\pgfqpoint{3.047252in}{1.713269in}}%
\pgfpathlineto{\pgfqpoint{3.048117in}{1.620661in}}%
\pgfpathlineto{\pgfqpoint{3.048983in}{1.708164in}}%
\pgfpathlineto{\pgfqpoint{3.049848in}{1.687327in}}%
\pgfpathlineto{\pgfqpoint{3.050713in}{1.707629in}}%
\pgfpathlineto{\pgfqpoint{3.051578in}{1.644466in}}%
\pgfpathlineto{\pgfqpoint{3.052443in}{1.702821in}}%
\pgfpathlineto{\pgfqpoint{3.053308in}{1.699140in}}%
\pgfpathlineto{\pgfqpoint{3.054173in}{1.726685in}}%
\pgfpathlineto{\pgfqpoint{3.055037in}{1.642744in}}%
\pgfpathlineto{\pgfqpoint{3.056766in}{1.695697in}}%
\pgfpathlineto{\pgfqpoint{3.057631in}{1.695756in}}%
\pgfpathlineto{\pgfqpoint{3.058495in}{1.767289in}}%
\pgfpathlineto{\pgfqpoint{3.061953in}{1.646663in}}%
\pgfpathlineto{\pgfqpoint{3.062818in}{1.721699in}}%
\pgfpathlineto{\pgfqpoint{3.063682in}{1.682757in}}%
\pgfpathlineto{\pgfqpoint{3.064545in}{1.689346in}}%
\pgfpathlineto{\pgfqpoint{3.065410in}{1.663939in}}%
\pgfpathlineto{\pgfqpoint{3.066275in}{1.670527in}}%
\pgfpathlineto{\pgfqpoint{3.067139in}{1.719680in}}%
\pgfpathlineto{\pgfqpoint{3.068004in}{1.701337in}}%
\pgfpathlineto{\pgfqpoint{3.068870in}{1.632950in}}%
\pgfpathlineto{\pgfqpoint{3.071465in}{1.714872in}}%
\pgfpathlineto{\pgfqpoint{3.072330in}{1.704900in}}%
\pgfpathlineto{\pgfqpoint{3.074059in}{1.765571in}}%
\pgfpathlineto{\pgfqpoint{3.074926in}{1.748028in}}%
\pgfpathlineto{\pgfqpoint{3.075790in}{1.677771in}}%
\pgfpathlineto{\pgfqpoint{3.076655in}{1.679730in}}%
\pgfpathlineto{\pgfqpoint{3.078385in}{1.694927in}}%
\pgfpathlineto{\pgfqpoint{3.080116in}{1.666846in}}%
\pgfpathlineto{\pgfqpoint{3.081847in}{1.728942in}}%
\pgfpathlineto{\pgfqpoint{3.082712in}{1.725261in}}%
\pgfpathlineto{\pgfqpoint{3.087036in}{1.586408in}}%
\pgfpathlineto{\pgfqpoint{3.088767in}{1.666936in}}%
\pgfpathlineto{\pgfqpoint{3.090498in}{1.548295in}}%
\pgfpathlineto{\pgfqpoint{3.092227in}{1.673376in}}%
\pgfpathlineto{\pgfqpoint{3.093092in}{1.656279in}}%
\pgfpathlineto{\pgfqpoint{3.093957in}{1.580827in}}%
\pgfpathlineto{\pgfqpoint{3.094822in}{1.613537in}}%
\pgfpathlineto{\pgfqpoint{3.095687in}{1.599824in}}%
\pgfpathlineto{\pgfqpoint{3.096552in}{1.618464in}}%
\pgfpathlineto{\pgfqpoint{3.097417in}{1.579581in}}%
\pgfpathlineto{\pgfqpoint{3.098282in}{1.585545in}}%
\pgfpathlineto{\pgfqpoint{3.099147in}{1.581243in}}%
\pgfpathlineto{\pgfqpoint{3.100878in}{1.612351in}}%
\pgfpathlineto{\pgfqpoint{3.101742in}{1.608729in}}%
\pgfpathlineto{\pgfqpoint{3.102605in}{1.613596in}}%
\pgfpathlineto{\pgfqpoint{3.104334in}{1.675514in}}%
\pgfpathlineto{\pgfqpoint{3.106065in}{1.583203in}}%
\pgfpathlineto{\pgfqpoint{3.106930in}{1.616032in}}%
\pgfpathlineto{\pgfqpoint{3.107795in}{1.576108in}}%
\pgfpathlineto{\pgfqpoint{3.108661in}{1.637758in}}%
\pgfpathlineto{\pgfqpoint{3.110391in}{1.586051in}}%
\pgfpathlineto{\pgfqpoint{3.111255in}{1.598400in}}%
\pgfpathlineto{\pgfqpoint{3.113850in}{1.547641in}}%
\pgfpathlineto{\pgfqpoint{3.114714in}{1.563790in}}%
\pgfpathlineto{\pgfqpoint{3.115579in}{1.610391in}}%
\pgfpathlineto{\pgfqpoint{3.117306in}{1.519058in}}%
\pgfpathlineto{\pgfqpoint{3.118172in}{1.519207in}}%
\pgfpathlineto{\pgfqpoint{3.121631in}{1.631407in}}%
\pgfpathlineto{\pgfqpoint{3.124228in}{1.554412in}}%
\pgfpathlineto{\pgfqpoint{3.125092in}{1.564801in}}%
\pgfpathlineto{\pgfqpoint{3.125956in}{1.628321in}}%
\pgfpathlineto{\pgfqpoint{3.128550in}{1.524435in}}%
\pgfpathlineto{\pgfqpoint{3.129414in}{1.619743in}}%
\pgfpathlineto{\pgfqpoint{3.132007in}{1.530697in}}%
\pgfpathlineto{\pgfqpoint{3.132874in}{1.622918in}}%
\pgfpathlineto{\pgfqpoint{3.133740in}{1.614310in}}%
\pgfpathlineto{\pgfqpoint{3.134605in}{1.557677in}}%
\pgfpathlineto{\pgfqpoint{3.135471in}{1.585519in}}%
\pgfpathlineto{\pgfqpoint{3.136337in}{1.549366in}}%
\pgfpathlineto{\pgfqpoint{3.138066in}{1.646961in}}%
\pgfpathlineto{\pgfqpoint{3.139795in}{1.587538in}}%
\pgfpathlineto{\pgfqpoint{3.141527in}{1.551118in}}%
\pgfpathlineto{\pgfqpoint{3.142392in}{1.604397in}}%
\pgfpathlineto{\pgfqpoint{3.143259in}{1.591632in}}%
\pgfpathlineto{\pgfqpoint{3.144124in}{1.583143in}}%
\pgfpathlineto{\pgfqpoint{3.144990in}{1.593651in}}%
\pgfpathlineto{\pgfqpoint{3.145857in}{1.569757in}}%
\pgfpathlineto{\pgfqpoint{3.146723in}{1.768241in}}%
\pgfpathlineto{\pgfqpoint{3.148455in}{1.674714in}}%
\pgfpathlineto{\pgfqpoint{3.149320in}{1.662395in}}%
\pgfpathlineto{\pgfqpoint{3.150186in}{1.708878in}}%
\pgfpathlineto{\pgfqpoint{3.151050in}{1.636542in}}%
\pgfpathlineto{\pgfqpoint{3.151916in}{1.731378in}}%
\pgfpathlineto{\pgfqpoint{3.152782in}{1.710600in}}%
\pgfpathlineto{\pgfqpoint{3.154510in}{1.685133in}}%
\pgfpathlineto{\pgfqpoint{3.155373in}{1.695403in}}%
\pgfpathlineto{\pgfqpoint{3.156236in}{1.731259in}}%
\pgfpathlineto{\pgfqpoint{3.157967in}{1.662544in}}%
\pgfpathlineto{\pgfqpoint{3.158829in}{1.730664in}}%
\pgfpathlineto{\pgfqpoint{3.160561in}{1.696768in}}%
\pgfpathlineto{\pgfqpoint{3.161425in}{1.698370in}}%
\pgfpathlineto{\pgfqpoint{3.162292in}{1.682995in}}%
\pgfpathlineto{\pgfqpoint{3.163156in}{1.689941in}}%
\pgfpathlineto{\pgfqpoint{3.164021in}{1.805699in}}%
\pgfpathlineto{\pgfqpoint{3.164886in}{1.689881in}}%
\pgfpathlineto{\pgfqpoint{3.166615in}{1.797686in}}%
\pgfpathlineto{\pgfqpoint{3.167480in}{1.676376in}}%
\pgfpathlineto{\pgfqpoint{3.168345in}{1.727637in}}%
\pgfpathlineto{\pgfqpoint{3.169210in}{1.715645in}}%
\pgfpathlineto{\pgfqpoint{3.170075in}{1.681482in}}%
\pgfpathlineto{\pgfqpoint{3.172667in}{1.721937in}}%
\pgfpathlineto{\pgfqpoint{3.173530in}{1.569787in}}%
\pgfpathlineto{\pgfqpoint{3.174393in}{1.598162in}}%
\pgfpathlineto{\pgfqpoint{3.175258in}{1.628853in}}%
\pgfpathlineto{\pgfqpoint{3.176124in}{1.620066in}}%
\pgfpathlineto{\pgfqpoint{3.176989in}{1.551709in}}%
\pgfpathlineto{\pgfqpoint{3.177853in}{1.635858in}}%
\pgfpathlineto{\pgfqpoint{3.178719in}{1.576138in}}%
\pgfpathlineto{\pgfqpoint{3.179583in}{1.583114in}}%
\pgfpathlineto{\pgfqpoint{3.180447in}{1.584865in}}%
\pgfpathlineto{\pgfqpoint{3.181311in}{1.581779in}}%
\pgfpathlineto{\pgfqpoint{3.182175in}{1.583084in}}%
\pgfpathlineto{\pgfqpoint{3.183039in}{1.607364in}}%
\pgfpathlineto{\pgfqpoint{3.184769in}{1.551207in}}%
\pgfpathlineto{\pgfqpoint{3.185635in}{1.564266in}}%
\pgfpathlineto{\pgfqpoint{3.186500in}{1.600210in}}%
\pgfpathlineto{\pgfqpoint{3.187365in}{1.583203in}}%
\pgfpathlineto{\pgfqpoint{3.188230in}{1.658179in}}%
\pgfpathlineto{\pgfqpoint{3.189960in}{1.554055in}}%
\pgfpathlineto{\pgfqpoint{3.190825in}{1.638349in}}%
\pgfpathlineto{\pgfqpoint{3.191688in}{1.623331in}}%
\pgfpathlineto{\pgfqpoint{3.192553in}{1.622145in}}%
\pgfpathlineto{\pgfqpoint{3.193417in}{1.543428in}}%
\pgfpathlineto{\pgfqpoint{3.196006in}{1.620661in}}%
\pgfpathlineto{\pgfqpoint{3.197734in}{1.655331in}}%
\pgfpathlineto{\pgfqpoint{3.200327in}{1.574416in}}%
\pgfpathlineto{\pgfqpoint{3.201193in}{1.575662in}}%
\pgfpathlineto{\pgfqpoint{3.202058in}{1.563730in}}%
\pgfpathlineto{\pgfqpoint{3.204651in}{1.627488in}}%
\pgfpathlineto{\pgfqpoint{3.205515in}{1.625380in}}%
\pgfpathlineto{\pgfqpoint{3.206381in}{1.622264in}}%
\pgfpathlineto{\pgfqpoint{3.207245in}{1.585222in}}%
\pgfpathlineto{\pgfqpoint{3.208109in}{1.687892in}}%
\pgfpathlineto{\pgfqpoint{3.208975in}{1.560941in}}%
\pgfpathlineto{\pgfqpoint{3.210707in}{1.614072in}}%
\pgfpathlineto{\pgfqpoint{3.211573in}{1.596857in}}%
\pgfpathlineto{\pgfqpoint{3.212439in}{1.677533in}}%
\pgfpathlineto{\pgfqpoint{3.213304in}{1.569490in}}%
\pgfpathlineto{\pgfqpoint{3.214169in}{1.627131in}}%
\pgfpathlineto{\pgfqpoint{3.215032in}{1.603743in}}%
\pgfpathlineto{\pgfqpoint{3.215896in}{1.541707in}}%
\pgfpathlineto{\pgfqpoint{3.216762in}{1.613953in}}%
\pgfpathlineto{\pgfqpoint{3.217627in}{1.596529in}}%
\pgfpathlineto{\pgfqpoint{3.218492in}{1.630872in}}%
\pgfpathlineto{\pgfqpoint{3.219357in}{1.617515in}}%
\pgfpathlineto{\pgfqpoint{3.220222in}{1.569192in}}%
\pgfpathlineto{\pgfqpoint{3.221087in}{1.610510in}}%
\pgfpathlineto{\pgfqpoint{3.221951in}{1.606383in}}%
\pgfpathlineto{\pgfqpoint{3.222816in}{1.558327in}}%
\pgfpathlineto{\pgfqpoint{3.224547in}{1.640844in}}%
\pgfpathlineto{\pgfqpoint{3.225412in}{1.508699in}}%
\pgfpathlineto{\pgfqpoint{3.226276in}{1.523305in}}%
\pgfpathlineto{\pgfqpoint{3.227141in}{1.574119in}}%
\pgfpathlineto{\pgfqpoint{3.228006in}{1.572516in}}%
\pgfpathlineto{\pgfqpoint{3.228872in}{1.560584in}}%
\pgfpathlineto{\pgfqpoint{3.230605in}{1.679373in}}%
\pgfpathlineto{\pgfqpoint{3.234064in}{1.528321in}}%
\pgfpathlineto{\pgfqpoint{3.235794in}{1.666549in}}%
\pgfpathlineto{\pgfqpoint{3.236660in}{1.635947in}}%
\pgfpathlineto{\pgfqpoint{3.237525in}{1.565809in}}%
\pgfpathlineto{\pgfqpoint{3.238390in}{1.589970in}}%
\pgfpathlineto{\pgfqpoint{3.239254in}{1.531258in}}%
\pgfpathlineto{\pgfqpoint{3.240120in}{1.613418in}}%
\pgfpathlineto{\pgfqpoint{3.240984in}{1.601545in}}%
\pgfpathlineto{\pgfqpoint{3.241849in}{1.586587in}}%
\pgfpathlineto{\pgfqpoint{3.242713in}{1.524550in}}%
\pgfpathlineto{\pgfqpoint{3.244443in}{1.652007in}}%
\pgfpathlineto{\pgfqpoint{3.245308in}{1.586825in}}%
\pgfpathlineto{\pgfqpoint{3.246173in}{1.644317in}}%
\pgfpathlineto{\pgfqpoint{3.247039in}{1.602140in}}%
\pgfpathlineto{\pgfqpoint{3.247901in}{1.664649in}}%
\pgfpathlineto{\pgfqpoint{3.248767in}{1.659900in}}%
\pgfpathlineto{\pgfqpoint{3.249633in}{1.653431in}}%
\pgfpathlineto{\pgfqpoint{3.250497in}{1.620840in}}%
\pgfpathlineto{\pgfqpoint{3.251361in}{1.644644in}}%
\pgfpathlineto{\pgfqpoint{3.252226in}{1.612529in}}%
\pgfpathlineto{\pgfqpoint{3.253089in}{1.648117in}}%
\pgfpathlineto{\pgfqpoint{3.253954in}{1.639420in}}%
\pgfpathlineto{\pgfqpoint{3.254819in}{1.656338in}}%
\pgfpathlineto{\pgfqpoint{3.255685in}{1.541736in}}%
\pgfpathlineto{\pgfqpoint{3.256550in}{1.628972in}}%
\pgfpathlineto{\pgfqpoint{3.257415in}{1.628377in}}%
\pgfpathlineto{\pgfqpoint{3.258280in}{1.633422in}}%
\pgfpathlineto{\pgfqpoint{3.259145in}{1.603326in}}%
\pgfpathlineto{\pgfqpoint{3.260010in}{1.702583in}}%
\pgfpathlineto{\pgfqpoint{3.260876in}{1.617515in}}%
\pgfpathlineto{\pgfqpoint{3.261740in}{1.711726in}}%
\pgfpathlineto{\pgfqpoint{3.262605in}{1.594986in}}%
\pgfpathlineto{\pgfqpoint{3.264334in}{1.739034in}}%
\pgfpathlineto{\pgfqpoint{3.266929in}{1.672903in}}%
\pgfpathlineto{\pgfqpoint{3.267794in}{1.714694in}}%
\pgfpathlineto{\pgfqpoint{3.268657in}{1.714634in}}%
\pgfpathlineto{\pgfqpoint{3.269522in}{1.666995in}}%
\pgfpathlineto{\pgfqpoint{3.272114in}{1.721788in}}%
\pgfpathlineto{\pgfqpoint{3.273846in}{1.680202in}}%
\pgfpathlineto{\pgfqpoint{3.274709in}{1.704305in}}%
\pgfpathlineto{\pgfqpoint{3.275573in}{1.674030in}}%
\pgfpathlineto{\pgfqpoint{3.277300in}{1.732088in}}%
\pgfpathlineto{\pgfqpoint{3.279893in}{1.700032in}}%
\pgfpathlineto{\pgfqpoint{3.280759in}{1.705732in}}%
\pgfpathlineto{\pgfqpoint{3.282489in}{1.652007in}}%
\pgfpathlineto{\pgfqpoint{3.283353in}{1.655033in}}%
\pgfpathlineto{\pgfqpoint{3.285084in}{1.704245in}}%
\pgfpathlineto{\pgfqpoint{3.285950in}{1.686260in}}%
\pgfpathlineto{\pgfqpoint{3.287681in}{1.704781in}}%
\pgfpathlineto{\pgfqpoint{3.289412in}{1.714456in}}%
\pgfpathlineto{\pgfqpoint{3.290277in}{1.708402in}}%
\pgfpathlineto{\pgfqpoint{3.291141in}{1.693116in}}%
\pgfpathlineto{\pgfqpoint{3.292006in}{1.645120in}}%
\pgfpathlineto{\pgfqpoint{3.292872in}{1.708997in}}%
\pgfpathlineto{\pgfqpoint{3.293738in}{1.658893in}}%
\pgfpathlineto{\pgfqpoint{3.294599in}{1.748355in}}%
\pgfpathlineto{\pgfqpoint{3.295464in}{1.686260in}}%
\pgfpathlineto{\pgfqpoint{3.296325in}{1.761771in}}%
\pgfpathlineto{\pgfqpoint{3.297192in}{1.670944in}}%
\pgfpathlineto{\pgfqpoint{3.298058in}{1.707246in}}%
\pgfpathlineto{\pgfqpoint{3.298922in}{1.552036in}}%
\pgfpathlineto{\pgfqpoint{3.299786in}{1.637460in}}%
\pgfpathlineto{\pgfqpoint{3.300649in}{1.632266in}}%
\pgfpathlineto{\pgfqpoint{3.301516in}{1.640785in}}%
\pgfpathlineto{\pgfqpoint{3.302379in}{1.585932in}}%
\pgfpathlineto{\pgfqpoint{3.303244in}{1.620155in}}%
\pgfpathlineto{\pgfqpoint{3.304108in}{1.559398in}}%
\pgfpathlineto{\pgfqpoint{3.304975in}{1.672071in}}%
\pgfpathlineto{\pgfqpoint{3.305841in}{1.653133in}}%
\pgfpathlineto{\pgfqpoint{3.307572in}{1.596053in}}%
\pgfpathlineto{\pgfqpoint{3.308436in}{1.611518in}}%
\pgfpathlineto{\pgfqpoint{3.310168in}{1.580440in}}%
\pgfpathlineto{\pgfqpoint{3.311899in}{1.655684in}}%
\pgfpathlineto{\pgfqpoint{3.312762in}{1.607450in}}%
\pgfpathlineto{\pgfqpoint{3.313628in}{1.656930in}}%
\pgfpathlineto{\pgfqpoint{3.315359in}{1.587594in}}%
\pgfpathlineto{\pgfqpoint{3.317088in}{1.686970in}}%
\pgfpathlineto{\pgfqpoint{3.318817in}{1.627309in}}%
\pgfpathlineto{\pgfqpoint{3.319683in}{1.647611in}}%
\pgfpathlineto{\pgfqpoint{3.322275in}{1.549779in}}%
\pgfpathlineto{\pgfqpoint{3.324004in}{1.632058in}}%
\pgfpathlineto{\pgfqpoint{3.324870in}{1.643871in}}%
\pgfpathlineto{\pgfqpoint{3.325734in}{1.622026in}}%
\pgfpathlineto{\pgfqpoint{3.326597in}{1.625796in}}%
\pgfpathlineto{\pgfqpoint{3.327461in}{1.644763in}}%
\pgfpathlineto{\pgfqpoint{3.328325in}{1.560941in}}%
\pgfpathlineto{\pgfqpoint{3.329190in}{1.568627in}}%
\pgfpathlineto{\pgfqpoint{3.330917in}{1.608015in}}%
\pgfpathlineto{\pgfqpoint{3.331782in}{1.587773in}}%
\pgfpathlineto{\pgfqpoint{3.332647in}{1.618761in}}%
\pgfpathlineto{\pgfqpoint{3.335241in}{1.569073in}}%
\pgfpathlineto{\pgfqpoint{3.336971in}{1.644168in}}%
\pgfpathlineto{\pgfqpoint{3.337833in}{1.599407in}}%
\pgfpathlineto{\pgfqpoint{3.338698in}{1.608283in}}%
\pgfpathlineto{\pgfqpoint{3.340427in}{1.531139in}}%
\pgfpathlineto{\pgfqpoint{3.343019in}{1.652598in}}%
\pgfpathlineto{\pgfqpoint{3.343883in}{1.582132in}}%
\pgfpathlineto{\pgfqpoint{3.344746in}{1.648266in}}%
\pgfpathlineto{\pgfqpoint{3.345611in}{1.549128in}}%
\pgfpathlineto{\pgfqpoint{3.346476in}{1.659782in}}%
\pgfpathlineto{\pgfqpoint{3.347340in}{1.632653in}}%
\pgfpathlineto{\pgfqpoint{3.349071in}{1.615496in}}%
\pgfpathlineto{\pgfqpoint{3.351667in}{1.565987in}}%
\pgfpathlineto{\pgfqpoint{3.353399in}{1.660555in}}%
\pgfpathlineto{\pgfqpoint{3.354265in}{1.700623in}}%
\pgfpathlineto{\pgfqpoint{3.355130in}{1.542034in}}%
\pgfpathlineto{\pgfqpoint{3.356859in}{1.639063in}}%
\pgfpathlineto{\pgfqpoint{3.358588in}{1.576436in}}%
\pgfpathlineto{\pgfqpoint{3.359454in}{1.583441in}}%
\pgfpathlineto{\pgfqpoint{3.360319in}{1.552274in}}%
\pgfpathlineto{\pgfqpoint{3.361184in}{1.661860in}}%
\pgfpathlineto{\pgfqpoint{3.363780in}{1.570676in}}%
\pgfpathlineto{\pgfqpoint{3.364643in}{1.627191in}}%
\pgfpathlineto{\pgfqpoint{3.365508in}{1.620721in}}%
\pgfpathlineto{\pgfqpoint{3.366374in}{1.595016in}}%
\pgfpathlineto{\pgfqpoint{3.367238in}{1.608045in}}%
\pgfpathlineto{\pgfqpoint{3.368103in}{1.656279in}}%
\pgfpathlineto{\pgfqpoint{3.369832in}{1.626923in}}%
\pgfpathlineto{\pgfqpoint{3.370697in}{1.524193in}}%
\pgfpathlineto{\pgfqpoint{3.373294in}{1.619356in}}%
\pgfpathlineto{\pgfqpoint{3.374159in}{1.600329in}}%
\pgfpathlineto{\pgfqpoint{3.375889in}{1.636155in}}%
\pgfpathlineto{\pgfqpoint{3.376753in}{1.633928in}}%
\pgfpathlineto{\pgfqpoint{3.377619in}{1.628023in}}%
\pgfpathlineto{\pgfqpoint{3.378486in}{1.608670in}}%
\pgfpathlineto{\pgfqpoint{3.379350in}{1.548831in}}%
\pgfpathlineto{\pgfqpoint{3.380214in}{1.567649in}}%
\pgfpathlineto{\pgfqpoint{3.381080in}{1.631760in}}%
\pgfpathlineto{\pgfqpoint{3.381944in}{1.602731in}}%
\pgfpathlineto{\pgfqpoint{3.382810in}{1.626893in}}%
\pgfpathlineto{\pgfqpoint{3.385402in}{1.592461in}}%
\pgfpathlineto{\pgfqpoint{3.387132in}{1.692075in}}%
\pgfpathlineto{\pgfqpoint{3.388863in}{1.557201in}}%
\pgfpathlineto{\pgfqpoint{3.390594in}{1.654141in}}%
\pgfpathlineto{\pgfqpoint{3.391460in}{1.637223in}}%
\pgfpathlineto{\pgfqpoint{3.392325in}{1.640368in}}%
\pgfpathlineto{\pgfqpoint{3.393191in}{1.657108in}}%
\pgfpathlineto{\pgfqpoint{3.394056in}{1.605758in}}%
\pgfpathlineto{\pgfqpoint{3.394921in}{1.633601in}}%
\pgfpathlineto{\pgfqpoint{3.395786in}{1.586259in}}%
\pgfpathlineto{\pgfqpoint{3.400110in}{1.681746in}}%
\pgfpathlineto{\pgfqpoint{3.401841in}{1.565154in}}%
\pgfpathlineto{\pgfqpoint{3.402704in}{1.538055in}}%
\pgfpathlineto{\pgfqpoint{3.404434in}{1.630277in}}%
\pgfpathlineto{\pgfqpoint{3.406164in}{1.587654in}}%
\pgfpathlineto{\pgfqpoint{3.407029in}{1.606472in}}%
\pgfpathlineto{\pgfqpoint{3.407893in}{1.589405in}}%
\pgfpathlineto{\pgfqpoint{3.408757in}{1.548355in}}%
\pgfpathlineto{\pgfqpoint{3.409621in}{1.596024in}}%
\pgfpathlineto{\pgfqpoint{3.410487in}{1.519088in}}%
\pgfpathlineto{\pgfqpoint{3.412214in}{1.628377in}}%
\pgfpathlineto{\pgfqpoint{3.413080in}{1.580767in}}%
\pgfpathlineto{\pgfqpoint{3.413944in}{1.588364in}}%
\pgfpathlineto{\pgfqpoint{3.414810in}{1.593767in}}%
\pgfpathlineto{\pgfqpoint{3.415676in}{1.583794in}}%
\pgfpathlineto{\pgfqpoint{3.416540in}{1.624933in}}%
\pgfpathlineto{\pgfqpoint{3.417405in}{1.581626in}}%
\pgfpathlineto{\pgfqpoint{3.418269in}{1.653784in}}%
\pgfpathlineto{\pgfqpoint{3.419134in}{1.648976in}}%
\pgfpathlineto{\pgfqpoint{3.419999in}{1.626358in}}%
\pgfpathlineto{\pgfqpoint{3.420865in}{1.652598in}}%
\pgfpathlineto{\pgfqpoint{3.421731in}{1.611518in}}%
\pgfpathlineto{\pgfqpoint{3.422597in}{1.666311in}}%
\pgfpathlineto{\pgfqpoint{3.425191in}{1.599348in}}%
\pgfpathlineto{\pgfqpoint{3.426054in}{1.664411in}}%
\pgfpathlineto{\pgfqpoint{3.427782in}{1.566935in}}%
\pgfpathlineto{\pgfqpoint{3.428646in}{1.636568in}}%
\pgfpathlineto{\pgfqpoint{3.429510in}{1.624815in}}%
\pgfpathlineto{\pgfqpoint{3.430376in}{1.601724in}}%
\pgfpathlineto{\pgfqpoint{3.432107in}{1.649155in}}%
\pgfpathlineto{\pgfqpoint{3.433836in}{1.626417in}}%
\pgfpathlineto{\pgfqpoint{3.434701in}{1.645652in}}%
\pgfpathlineto{\pgfqpoint{3.435566in}{1.639714in}}%
\pgfpathlineto{\pgfqpoint{3.436430in}{1.645325in}}%
\pgfpathlineto{\pgfqpoint{3.437295in}{1.602791in}}%
\pgfpathlineto{\pgfqpoint{3.438159in}{1.655862in}}%
\pgfpathlineto{\pgfqpoint{3.439025in}{1.590323in}}%
\pgfpathlineto{\pgfqpoint{3.439889in}{1.630396in}}%
\pgfpathlineto{\pgfqpoint{3.440753in}{1.626536in}}%
\pgfpathlineto{\pgfqpoint{3.441619in}{1.625290in}}%
\pgfpathlineto{\pgfqpoint{3.442483in}{1.632058in}}%
\pgfpathlineto{\pgfqpoint{3.444212in}{1.596024in}}%
\pgfpathlineto{\pgfqpoint{3.445076in}{1.625052in}}%
\pgfpathlineto{\pgfqpoint{3.445941in}{1.572397in}}%
\pgfpathlineto{\pgfqpoint{3.446805in}{1.627369in}}%
\pgfpathlineto{\pgfqpoint{3.447669in}{1.626655in}}%
\pgfpathlineto{\pgfqpoint{3.448533in}{1.601069in}}%
\pgfpathlineto{\pgfqpoint{3.450259in}{1.667586in}}%
\pgfpathlineto{\pgfqpoint{3.451124in}{1.605996in}}%
\pgfpathlineto{\pgfqpoint{3.451988in}{1.635084in}}%
\pgfpathlineto{\pgfqpoint{3.452853in}{1.592372in}}%
\pgfpathlineto{\pgfqpoint{3.454582in}{1.631998in}}%
\pgfpathlineto{\pgfqpoint{3.456310in}{1.581597in}}%
\pgfpathlineto{\pgfqpoint{3.457175in}{1.674383in}}%
\pgfpathlineto{\pgfqpoint{3.458040in}{1.619769in}}%
\pgfpathlineto{\pgfqpoint{3.458905in}{1.645116in}}%
\pgfpathlineto{\pgfqpoint{3.459770in}{1.578451in}}%
\pgfpathlineto{\pgfqpoint{3.460636in}{1.650460in}}%
\pgfpathlineto{\pgfqpoint{3.462364in}{1.627811in}}%
\pgfpathlineto{\pgfqpoint{3.463229in}{1.650222in}}%
\pgfpathlineto{\pgfqpoint{3.464094in}{1.647017in}}%
\pgfpathlineto{\pgfqpoint{3.464958in}{1.513031in}}%
\pgfpathlineto{\pgfqpoint{3.465821in}{1.551441in}}%
\pgfpathlineto{\pgfqpoint{3.467552in}{1.615671in}}%
\pgfpathlineto{\pgfqpoint{3.468416in}{1.605996in}}%
\pgfpathlineto{\pgfqpoint{3.469282in}{1.648768in}}%
\pgfpathlineto{\pgfqpoint{3.470145in}{1.611161in}}%
\pgfpathlineto{\pgfqpoint{3.471011in}{1.624815in}}%
\pgfpathlineto{\pgfqpoint{3.472743in}{1.605996in}}%
\pgfpathlineto{\pgfqpoint{3.473609in}{1.616207in}}%
\pgfpathlineto{\pgfqpoint{3.474471in}{1.561295in}}%
\pgfpathlineto{\pgfqpoint{3.475336in}{1.602434in}}%
\pgfpathlineto{\pgfqpoint{3.476201in}{1.598039in}}%
\pgfpathlineto{\pgfqpoint{3.477067in}{1.603561in}}%
\pgfpathlineto{\pgfqpoint{3.477933in}{1.624339in}}%
\pgfpathlineto{\pgfqpoint{3.478798in}{1.620925in}}%
\pgfpathlineto{\pgfqpoint{3.480525in}{1.574770in}}%
\pgfpathlineto{\pgfqpoint{3.481389in}{1.580793in}}%
\pgfpathlineto{\pgfqpoint{3.482252in}{1.605520in}}%
\pgfpathlineto{\pgfqpoint{3.483118in}{1.696645in}}%
\pgfpathlineto{\pgfqpoint{3.484847in}{1.621609in}}%
\pgfpathlineto{\pgfqpoint{3.485710in}{1.647433in}}%
\pgfpathlineto{\pgfqpoint{3.486575in}{1.594481in}}%
\pgfpathlineto{\pgfqpoint{3.487439in}{1.623271in}}%
\pgfpathlineto{\pgfqpoint{3.488304in}{1.569192in}}%
\pgfpathlineto{\pgfqpoint{3.489170in}{1.572397in}}%
\pgfpathlineto{\pgfqpoint{3.490901in}{1.650698in}}%
\pgfpathlineto{\pgfqpoint{3.491766in}{1.611756in}}%
\pgfpathlineto{\pgfqpoint{3.492631in}{1.629150in}}%
\pgfpathlineto{\pgfqpoint{3.494360in}{1.542301in}}%
\pgfpathlineto{\pgfqpoint{3.495226in}{1.569073in}}%
\pgfpathlineto{\pgfqpoint{3.496091in}{1.531258in}}%
\pgfpathlineto{\pgfqpoint{3.498684in}{1.660138in}}%
\pgfpathlineto{\pgfqpoint{3.499548in}{1.655212in}}%
\pgfpathlineto{\pgfqpoint{3.502142in}{1.593056in}}%
\pgfpathlineto{\pgfqpoint{3.503006in}{1.613091in}}%
\pgfpathlineto{\pgfqpoint{3.504734in}{1.674800in}}%
\pgfpathlineto{\pgfqpoint{3.506463in}{1.622204in}}%
\pgfpathlineto{\pgfqpoint{3.507329in}{1.646158in}}%
\pgfpathlineto{\pgfqpoint{3.508192in}{1.570854in}}%
\pgfpathlineto{\pgfqpoint{3.509923in}{1.639301in}}%
\pgfpathlineto{\pgfqpoint{3.510788in}{1.631998in}}%
\pgfpathlineto{\pgfqpoint{3.512519in}{1.550314in}}%
\pgfpathlineto{\pgfqpoint{3.513384in}{1.610034in}}%
\pgfpathlineto{\pgfqpoint{3.515113in}{1.565511in}}%
\pgfpathlineto{\pgfqpoint{3.515979in}{1.568954in}}%
\pgfpathlineto{\pgfqpoint{3.516844in}{1.620572in}}%
\pgfpathlineto{\pgfqpoint{3.517709in}{1.575484in}}%
\pgfpathlineto{\pgfqpoint{3.518574in}{1.614723in}}%
\pgfpathlineto{\pgfqpoint{3.519437in}{1.599259in}}%
\pgfpathlineto{\pgfqpoint{3.520301in}{1.553163in}}%
\pgfpathlineto{\pgfqpoint{3.523762in}{1.688691in}}%
\pgfpathlineto{\pgfqpoint{3.525491in}{1.580470in}}%
\pgfpathlineto{\pgfqpoint{3.526355in}{1.590472in}}%
\pgfpathlineto{\pgfqpoint{3.527220in}{1.629146in}}%
\pgfpathlineto{\pgfqpoint{3.529817in}{1.544852in}}%
\pgfpathlineto{\pgfqpoint{3.530682in}{1.618523in}}%
\pgfpathlineto{\pgfqpoint{3.531546in}{1.579314in}}%
\pgfpathlineto{\pgfqpoint{3.533278in}{1.636453in}}%
\pgfpathlineto{\pgfqpoint{3.534143in}{1.578038in}}%
\pgfpathlineto{\pgfqpoint{3.535006in}{1.596916in}}%
\pgfpathlineto{\pgfqpoint{3.536735in}{1.558803in}}%
\pgfpathlineto{\pgfqpoint{3.537601in}{1.584984in}}%
\pgfpathlineto{\pgfqpoint{3.538466in}{1.584508in}}%
\pgfpathlineto{\pgfqpoint{3.539331in}{1.584805in}}%
\pgfpathlineto{\pgfqpoint{3.540196in}{1.617396in}}%
\pgfpathlineto{\pgfqpoint{3.541059in}{1.578455in}}%
\pgfpathlineto{\pgfqpoint{3.541923in}{1.588427in}}%
\pgfpathlineto{\pgfqpoint{3.543652in}{1.562603in}}%
\pgfpathlineto{\pgfqpoint{3.544515in}{1.621847in}}%
\pgfpathlineto{\pgfqpoint{3.545379in}{1.555271in}}%
\pgfpathlineto{\pgfqpoint{3.547110in}{1.654855in}}%
\pgfpathlineto{\pgfqpoint{3.547975in}{1.619297in}}%
\pgfpathlineto{\pgfqpoint{3.548839in}{1.727221in}}%
\pgfpathlineto{\pgfqpoint{3.550568in}{1.595105in}}%
\pgfpathlineto{\pgfqpoint{3.551434in}{1.662039in}}%
\pgfpathlineto{\pgfqpoint{3.552299in}{1.614310in}}%
\pgfpathlineto{\pgfqpoint{3.553165in}{1.624402in}}%
\pgfpathlineto{\pgfqpoint{3.554896in}{1.585728in}}%
\pgfpathlineto{\pgfqpoint{3.557492in}{1.686141in}}%
\pgfpathlineto{\pgfqpoint{3.558357in}{1.639896in}}%
\pgfpathlineto{\pgfqpoint{3.559222in}{1.663879in}}%
\pgfpathlineto{\pgfqpoint{3.560087in}{1.533158in}}%
\pgfpathlineto{\pgfqpoint{3.561816in}{1.641201in}}%
\pgfpathlineto{\pgfqpoint{3.562682in}{1.627488in}}%
\pgfpathlineto{\pgfqpoint{3.563548in}{1.579938in}}%
\pgfpathlineto{\pgfqpoint{3.565276in}{1.663109in}}%
\pgfpathlineto{\pgfqpoint{3.566138in}{1.712083in}}%
\pgfpathlineto{\pgfqpoint{3.567868in}{1.648801in}}%
\pgfpathlineto{\pgfqpoint{3.569599in}{1.703475in}}%
\pgfpathlineto{\pgfqpoint{3.570464in}{1.708878in}}%
\pgfpathlineto{\pgfqpoint{3.571329in}{1.677711in}}%
\pgfpathlineto{\pgfqpoint{3.572193in}{1.686200in}}%
\pgfpathlineto{\pgfqpoint{3.573058in}{1.773168in}}%
\pgfpathlineto{\pgfqpoint{3.573923in}{1.633724in}}%
\pgfpathlineto{\pgfqpoint{3.575651in}{1.738260in}}%
\pgfpathlineto{\pgfqpoint{3.576516in}{1.670587in}}%
\pgfpathlineto{\pgfqpoint{3.577381in}{1.704483in}}%
\pgfpathlineto{\pgfqpoint{3.579112in}{1.666757in}}%
\pgfpathlineto{\pgfqpoint{3.579978in}{1.689524in}}%
\pgfpathlineto{\pgfqpoint{3.580842in}{1.742001in}}%
\pgfpathlineto{\pgfqpoint{3.582573in}{1.686022in}}%
\pgfpathlineto{\pgfqpoint{3.583437in}{1.661830in}}%
\pgfpathlineto{\pgfqpoint{3.585164in}{1.683170in}}%
\pgfpathlineto{\pgfqpoint{3.586028in}{1.675246in}}%
\pgfpathlineto{\pgfqpoint{3.586892in}{1.734460in}}%
\pgfpathlineto{\pgfqpoint{3.587756in}{1.622974in}}%
\pgfpathlineto{\pgfqpoint{3.590351in}{1.807893in}}%
\pgfpathlineto{\pgfqpoint{3.591214in}{1.710596in}}%
\pgfpathlineto{\pgfqpoint{3.592078in}{1.746035in}}%
\pgfpathlineto{\pgfqpoint{3.592942in}{1.667170in}}%
\pgfpathlineto{\pgfqpoint{3.593804in}{1.689461in}}%
\pgfpathlineto{\pgfqpoint{3.594668in}{1.758384in}}%
\pgfpathlineto{\pgfqpoint{3.595532in}{1.703680in}}%
\pgfpathlineto{\pgfqpoint{3.596397in}{1.734044in}}%
\pgfpathlineto{\pgfqpoint{3.598128in}{1.669873in}}%
\pgfpathlineto{\pgfqpoint{3.598992in}{1.740160in}}%
\pgfpathlineto{\pgfqpoint{3.599857in}{1.707510in}}%
\pgfpathlineto{\pgfqpoint{3.600722in}{1.712734in}}%
\pgfpathlineto{\pgfqpoint{3.601586in}{1.690413in}}%
\pgfpathlineto{\pgfqpoint{3.603316in}{1.759157in}}%
\pgfpathlineto{\pgfqpoint{3.604181in}{1.749363in}}%
\pgfpathlineto{\pgfqpoint{3.605045in}{1.675157in}}%
\pgfpathlineto{\pgfqpoint{3.605910in}{1.713626in}}%
\pgfpathlineto{\pgfqpoint{3.606775in}{1.700092in}}%
\pgfpathlineto{\pgfqpoint{3.607639in}{1.650017in}}%
\pgfpathlineto{\pgfqpoint{3.609370in}{1.735471in}}%
\pgfpathlineto{\pgfqpoint{3.610234in}{1.721818in}}%
\pgfpathlineto{\pgfqpoint{3.613692in}{1.745801in}}%
\pgfpathlineto{\pgfqpoint{3.614557in}{1.747582in}}%
\pgfpathlineto{\pgfqpoint{3.615422in}{1.701575in}}%
\pgfpathlineto{\pgfqpoint{3.616289in}{1.754528in}}%
\pgfpathlineto{\pgfqpoint{3.617155in}{1.690800in}}%
\pgfpathlineto{\pgfqpoint{3.618017in}{1.690948in}}%
\pgfpathlineto{\pgfqpoint{3.618881in}{1.732266in}}%
\pgfpathlineto{\pgfqpoint{3.621474in}{1.670081in}}%
\pgfpathlineto{\pgfqpoint{3.623205in}{1.759098in}}%
\pgfpathlineto{\pgfqpoint{3.624067in}{1.694273in}}%
\pgfpathlineto{\pgfqpoint{3.624932in}{1.731909in}}%
\pgfpathlineto{\pgfqpoint{3.625798in}{1.697240in}}%
\pgfpathlineto{\pgfqpoint{3.626663in}{1.699051in}}%
\pgfpathlineto{\pgfqpoint{3.628394in}{1.772216in}}%
\pgfpathlineto{\pgfqpoint{3.629258in}{1.707331in}}%
\pgfpathlineto{\pgfqpoint{3.630123in}{1.801066in}}%
\pgfpathlineto{\pgfqpoint{3.631852in}{1.723361in}}%
\pgfpathlineto{\pgfqpoint{3.632716in}{1.728823in}}%
\pgfpathlineto{\pgfqpoint{3.633580in}{1.626834in}}%
\pgfpathlineto{\pgfqpoint{3.635312in}{1.803799in}}%
\pgfpathlineto{\pgfqpoint{3.636176in}{1.698991in}}%
\pgfpathlineto{\pgfqpoint{3.637043in}{1.717661in}}%
\pgfpathlineto{\pgfqpoint{3.638773in}{1.752330in}}%
\pgfpathlineto{\pgfqpoint{3.640506in}{1.683170in}}%
\pgfpathlineto{\pgfqpoint{3.641372in}{1.743603in}}%
\pgfpathlineto{\pgfqpoint{3.643098in}{1.719888in}}%
\pgfpathlineto{\pgfqpoint{3.643962in}{1.776075in}}%
\pgfpathlineto{\pgfqpoint{3.645693in}{1.681954in}}%
\pgfpathlineto{\pgfqpoint{3.647418in}{1.784743in}}%
\pgfpathlineto{\pgfqpoint{3.649149in}{1.667263in}}%
\pgfpathlineto{\pgfqpoint{3.650016in}{1.743931in}}%
\pgfpathlineto{\pgfqpoint{3.650882in}{1.728288in}}%
\pgfpathlineto{\pgfqpoint{3.652611in}{1.668360in}}%
\pgfpathlineto{\pgfqpoint{3.654343in}{1.780233in}}%
\pgfpathlineto{\pgfqpoint{3.656940in}{1.689286in}}%
\pgfpathlineto{\pgfqpoint{3.657803in}{1.785159in}}%
\pgfpathlineto{\pgfqpoint{3.658669in}{1.668389in}}%
\pgfpathlineto{\pgfqpoint{3.659533in}{1.697567in}}%
\pgfpathlineto{\pgfqpoint{3.660399in}{1.722175in}}%
\pgfpathlineto{\pgfqpoint{3.661264in}{1.667263in}}%
\pgfpathlineto{\pgfqpoint{3.662127in}{1.752628in}}%
\pgfpathlineto{\pgfqpoint{3.664721in}{1.643071in}}%
\pgfpathlineto{\pgfqpoint{3.665587in}{1.757911in}}%
\pgfpathlineto{\pgfqpoint{3.667313in}{1.675811in}}%
\pgfpathlineto{\pgfqpoint{3.668179in}{1.687803in}}%
\pgfpathlineto{\pgfqpoint{3.669045in}{1.728674in}}%
\pgfpathlineto{\pgfqpoint{3.669910in}{1.672606in}}%
\pgfpathlineto{\pgfqpoint{3.670775in}{1.682816in}}%
\pgfpathlineto{\pgfqpoint{3.672504in}{1.721758in}}%
\pgfpathlineto{\pgfqpoint{3.673368in}{1.648385in}}%
\pgfpathlineto{\pgfqpoint{3.674235in}{1.663463in}}%
\pgfpathlineto{\pgfqpoint{3.675965in}{1.711131in}}%
\pgfpathlineto{\pgfqpoint{3.676831in}{1.714277in}}%
\pgfpathlineto{\pgfqpoint{3.677697in}{1.728347in}}%
\pgfpathlineto{\pgfqpoint{3.678563in}{1.683943in}}%
\pgfpathlineto{\pgfqpoint{3.679428in}{1.744079in}}%
\pgfpathlineto{\pgfqpoint{3.680294in}{1.677711in}}%
\pgfpathlineto{\pgfqpoint{3.681160in}{1.691335in}}%
\pgfpathlineto{\pgfqpoint{3.682026in}{1.649750in}}%
\pgfpathlineto{\pgfqpoint{3.682891in}{1.692729in}}%
\pgfpathlineto{\pgfqpoint{3.683756in}{1.679225in}}%
\pgfpathlineto{\pgfqpoint{3.684621in}{1.711667in}}%
\pgfpathlineto{\pgfqpoint{3.685485in}{1.656041in}}%
\pgfpathlineto{\pgfqpoint{3.686350in}{1.728585in}}%
\pgfpathlineto{\pgfqpoint{3.688079in}{1.654438in}}%
\pgfpathlineto{\pgfqpoint{3.688943in}{1.735234in}}%
\pgfpathlineto{\pgfqpoint{3.689808in}{1.668508in}}%
\pgfpathlineto{\pgfqpoint{3.690672in}{1.669843in}}%
\pgfpathlineto{\pgfqpoint{3.691536in}{1.719501in}}%
\pgfpathlineto{\pgfqpoint{3.692401in}{1.709529in}}%
\pgfpathlineto{\pgfqpoint{3.693266in}{1.620572in}}%
\pgfpathlineto{\pgfqpoint{3.694996in}{1.709469in}}%
\pgfpathlineto{\pgfqpoint{3.695860in}{1.680202in}}%
\pgfpathlineto{\pgfqpoint{3.696724in}{1.803680in}}%
\pgfpathlineto{\pgfqpoint{3.698452in}{1.707157in}}%
\pgfpathlineto{\pgfqpoint{3.699316in}{1.732683in}}%
\pgfpathlineto{\pgfqpoint{3.700181in}{1.727221in}}%
\pgfpathlineto{\pgfqpoint{3.701046in}{1.666727in}}%
\pgfpathlineto{\pgfqpoint{3.701911in}{1.743544in}}%
\pgfpathlineto{\pgfqpoint{3.704508in}{1.682281in}}%
\pgfpathlineto{\pgfqpoint{3.705373in}{1.675514in}}%
\pgfpathlineto{\pgfqpoint{3.706238in}{1.721818in}}%
\pgfpathlineto{\pgfqpoint{3.707103in}{1.670646in}}%
\pgfpathlineto{\pgfqpoint{3.707967in}{1.675930in}}%
\pgfpathlineto{\pgfqpoint{3.708833in}{1.689524in}}%
\pgfpathlineto{\pgfqpoint{3.709698in}{1.644704in}}%
\pgfpathlineto{\pgfqpoint{3.711428in}{1.764143in}}%
\pgfpathlineto{\pgfqpoint{3.712293in}{1.544822in}}%
\pgfpathlineto{\pgfqpoint{3.713157in}{1.676343in}}%
\pgfpathlineto{\pgfqpoint{3.714022in}{1.652955in}}%
\pgfpathlineto{\pgfqpoint{3.714886in}{1.569073in}}%
\pgfpathlineto{\pgfqpoint{3.715748in}{1.632831in}}%
\pgfpathlineto{\pgfqpoint{3.716611in}{1.596827in}}%
\pgfpathlineto{\pgfqpoint{3.718343in}{1.684657in}}%
\pgfpathlineto{\pgfqpoint{3.719209in}{1.671331in}}%
\pgfpathlineto{\pgfqpoint{3.720074in}{1.713567in}}%
\pgfpathlineto{\pgfqpoint{3.722666in}{1.634493in}}%
\pgfpathlineto{\pgfqpoint{3.723531in}{1.692194in}}%
\pgfpathlineto{\pgfqpoint{3.725261in}{1.642923in}}%
\pgfpathlineto{\pgfqpoint{3.726126in}{1.636631in}}%
\pgfpathlineto{\pgfqpoint{3.726990in}{1.675692in}}%
\pgfpathlineto{\pgfqpoint{3.727855in}{1.644823in}}%
\pgfpathlineto{\pgfqpoint{3.729587in}{1.686260in}}%
\pgfpathlineto{\pgfqpoint{3.730452in}{1.619475in}}%
\pgfpathlineto{\pgfqpoint{3.731315in}{1.667025in}}%
\pgfpathlineto{\pgfqpoint{3.732179in}{1.656338in}}%
\pgfpathlineto{\pgfqpoint{3.733910in}{1.598281in}}%
\pgfpathlineto{\pgfqpoint{3.734774in}{1.687505in}}%
\pgfpathlineto{\pgfqpoint{3.736506in}{1.562901in}}%
\pgfpathlineto{\pgfqpoint{3.737372in}{1.653431in}}%
\pgfpathlineto{\pgfqpoint{3.738238in}{1.592997in}}%
\pgfpathlineto{\pgfqpoint{3.739103in}{1.629031in}}%
\pgfpathlineto{\pgfqpoint{3.739967in}{1.592818in}}%
\pgfpathlineto{\pgfqpoint{3.740831in}{1.642120in}}%
\pgfpathlineto{\pgfqpoint{3.741695in}{1.596500in}}%
\pgfpathlineto{\pgfqpoint{3.743425in}{1.663225in}}%
\pgfpathlineto{\pgfqpoint{3.745156in}{1.560793in}}%
\pgfpathlineto{\pgfqpoint{3.746887in}{1.628972in}}%
\pgfpathlineto{\pgfqpoint{3.747753in}{1.626064in}}%
\pgfpathlineto{\pgfqpoint{3.748618in}{1.688572in}}%
\pgfpathlineto{\pgfqpoint{3.750349in}{1.602880in}}%
\pgfpathlineto{\pgfqpoint{3.752078in}{1.638706in}}%
\pgfpathlineto{\pgfqpoint{3.752944in}{1.585575in}}%
\pgfpathlineto{\pgfqpoint{3.753810in}{1.628317in}}%
\pgfpathlineto{\pgfqpoint{3.754676in}{1.613180in}}%
\pgfpathlineto{\pgfqpoint{3.756406in}{1.679727in}}%
\pgfpathlineto{\pgfqpoint{3.757272in}{1.620155in}}%
\pgfpathlineto{\pgfqpoint{3.758138in}{1.666311in}}%
\pgfpathlineto{\pgfqpoint{3.759004in}{1.561949in}}%
\pgfpathlineto{\pgfqpoint{3.759869in}{1.652627in}}%
\pgfpathlineto{\pgfqpoint{3.760734in}{1.642625in}}%
\pgfpathlineto{\pgfqpoint{3.761597in}{1.573822in}}%
\pgfpathlineto{\pgfqpoint{3.762463in}{1.581303in}}%
\pgfpathlineto{\pgfqpoint{3.763326in}{1.634136in}}%
\pgfpathlineto{\pgfqpoint{3.764191in}{1.625320in}}%
\pgfpathlineto{\pgfqpoint{3.765921in}{1.648798in}}%
\pgfpathlineto{\pgfqpoint{3.766786in}{1.575097in}}%
\pgfpathlineto{\pgfqpoint{3.768513in}{1.666311in}}%
\pgfpathlineto{\pgfqpoint{3.769378in}{1.593116in}}%
\pgfpathlineto{\pgfqpoint{3.770241in}{1.678362in}}%
\pgfpathlineto{\pgfqpoint{3.771106in}{1.670940in}}%
\pgfpathlineto{\pgfqpoint{3.772835in}{1.563611in}}%
\pgfpathlineto{\pgfqpoint{3.773700in}{1.652717in}}%
\pgfpathlineto{\pgfqpoint{3.774564in}{1.620423in}}%
\pgfpathlineto{\pgfqpoint{3.776290in}{1.697270in}}%
\pgfpathlineto{\pgfqpoint{3.778019in}{1.584508in}}%
\pgfpathlineto{\pgfqpoint{3.778884in}{1.626953in}}%
\pgfpathlineto{\pgfqpoint{3.779750in}{1.625588in}}%
\pgfpathlineto{\pgfqpoint{3.780615in}{1.637817in}}%
\pgfpathlineto{\pgfqpoint{3.781479in}{1.632177in}}%
\pgfpathlineto{\pgfqpoint{3.782344in}{1.587773in}}%
\pgfpathlineto{\pgfqpoint{3.783208in}{1.599943in}}%
\pgfpathlineto{\pgfqpoint{3.784074in}{1.618107in}}%
\pgfpathlineto{\pgfqpoint{3.784940in}{1.589018in}}%
\pgfpathlineto{\pgfqpoint{3.785802in}{1.590561in}}%
\pgfpathlineto{\pgfqpoint{3.786665in}{1.606234in}}%
\pgfpathlineto{\pgfqpoint{3.787530in}{1.689818in}}%
\pgfpathlineto{\pgfqpoint{3.788396in}{1.602880in}}%
\pgfpathlineto{\pgfqpoint{3.789261in}{1.611101in}}%
\pgfpathlineto{\pgfqpoint{3.790125in}{1.592461in}}%
\pgfpathlineto{\pgfqpoint{3.790989in}{1.597805in}}%
\pgfpathlineto{\pgfqpoint{3.792718in}{1.561295in}}%
\pgfpathlineto{\pgfqpoint{3.793584in}{1.520334in}}%
\pgfpathlineto{\pgfqpoint{3.794449in}{1.629091in}}%
\pgfpathlineto{\pgfqpoint{3.795314in}{1.584449in}}%
\pgfpathlineto{\pgfqpoint{3.796179in}{1.609796in}}%
\pgfpathlineto{\pgfqpoint{3.797911in}{1.585724in}}%
\pgfpathlineto{\pgfqpoint{3.798777in}{1.601486in}}%
\pgfpathlineto{\pgfqpoint{3.800503in}{1.576908in}}%
\pgfpathlineto{\pgfqpoint{3.802234in}{1.697835in}}%
\pgfpathlineto{\pgfqpoint{3.805695in}{1.636334in}}%
\pgfpathlineto{\pgfqpoint{3.806562in}{1.724904in}}%
\pgfpathlineto{\pgfqpoint{3.807428in}{1.714307in}}%
\pgfpathlineto{\pgfqpoint{3.808293in}{1.705848in}}%
\pgfpathlineto{\pgfqpoint{3.809157in}{1.717070in}}%
\pgfpathlineto{\pgfqpoint{3.810020in}{1.674328in}}%
\pgfpathlineto{\pgfqpoint{3.810885in}{1.746217in}}%
\pgfpathlineto{\pgfqpoint{3.812616in}{1.636334in}}%
\pgfpathlineto{\pgfqpoint{3.813481in}{1.705316in}}%
\pgfpathlineto{\pgfqpoint{3.814344in}{1.639866in}}%
\pgfpathlineto{\pgfqpoint{3.815209in}{1.682459in}}%
\pgfpathlineto{\pgfqpoint{3.816076in}{1.670706in}}%
\pgfpathlineto{\pgfqpoint{3.816939in}{1.636780in}}%
\pgfpathlineto{\pgfqpoint{3.818667in}{1.697954in}}%
\pgfpathlineto{\pgfqpoint{3.819532in}{1.685903in}}%
\pgfpathlineto{\pgfqpoint{3.820398in}{1.689286in}}%
\pgfpathlineto{\pgfqpoint{3.821264in}{1.702970in}}%
\pgfpathlineto{\pgfqpoint{3.822129in}{1.695994in}}%
\pgfpathlineto{\pgfqpoint{3.822992in}{1.753695in}}%
\pgfpathlineto{\pgfqpoint{3.823858in}{1.723182in}}%
\pgfpathlineto{\pgfqpoint{3.824723in}{1.726507in}}%
\pgfpathlineto{\pgfqpoint{3.825589in}{1.705729in}}%
\pgfpathlineto{\pgfqpoint{3.826455in}{1.740220in}}%
\pgfpathlineto{\pgfqpoint{3.827320in}{1.614426in}}%
\pgfpathlineto{\pgfqpoint{3.828185in}{1.762184in}}%
\pgfpathlineto{\pgfqpoint{3.829052in}{1.693916in}}%
\pgfpathlineto{\pgfqpoint{3.829917in}{1.702464in}}%
\pgfpathlineto{\pgfqpoint{3.830781in}{1.693767in}}%
\pgfpathlineto{\pgfqpoint{3.831644in}{1.693916in}}%
\pgfpathlineto{\pgfqpoint{3.832509in}{1.733452in}}%
\pgfpathlineto{\pgfqpoint{3.833375in}{1.729980in}}%
\pgfpathlineto{\pgfqpoint{3.834240in}{1.723718in}}%
\pgfpathlineto{\pgfqpoint{3.835105in}{1.754409in}}%
\pgfpathlineto{\pgfqpoint{3.835971in}{1.709202in}}%
\pgfpathlineto{\pgfqpoint{3.836836in}{1.712972in}}%
\pgfpathlineto{\pgfqpoint{3.837698in}{1.720096in}}%
\pgfpathlineto{\pgfqpoint{3.838562in}{1.651531in}}%
\pgfpathlineto{\pgfqpoint{3.840290in}{1.701367in}}%
\pgfpathlineto{\pgfqpoint{3.841155in}{1.710953in}}%
\pgfpathlineto{\pgfqpoint{3.842018in}{1.668508in}}%
\pgfpathlineto{\pgfqpoint{3.843749in}{1.728347in}}%
\pgfpathlineto{\pgfqpoint{3.844612in}{1.716237in}}%
\pgfpathlineto{\pgfqpoint{3.845478in}{1.661354in}}%
\pgfpathlineto{\pgfqpoint{3.847206in}{1.701099in}}%
\pgfpathlineto{\pgfqpoint{3.849796in}{1.629061in}}%
\pgfpathlineto{\pgfqpoint{3.850662in}{1.697478in}}%
\pgfpathlineto{\pgfqpoint{3.851528in}{1.688929in}}%
\pgfpathlineto{\pgfqpoint{3.853258in}{1.727990in}}%
\pgfpathlineto{\pgfqpoint{3.856715in}{1.568121in}}%
\pgfpathlineto{\pgfqpoint{3.857580in}{1.585813in}}%
\pgfpathlineto{\pgfqpoint{3.858443in}{1.580172in}}%
\pgfpathlineto{\pgfqpoint{3.859307in}{1.614634in}}%
\pgfpathlineto{\pgfqpoint{3.860171in}{1.568657in}}%
\pgfpathlineto{\pgfqpoint{3.861901in}{1.628525in}}%
\pgfpathlineto{\pgfqpoint{3.862767in}{1.576432in}}%
\pgfpathlineto{\pgfqpoint{3.863631in}{1.654022in}}%
\pgfpathlineto{\pgfqpoint{3.864495in}{1.577503in}}%
\pgfpathlineto{\pgfqpoint{3.865358in}{1.702524in}}%
\pgfpathlineto{\pgfqpoint{3.866223in}{1.667025in}}%
\pgfpathlineto{\pgfqpoint{3.867087in}{1.663463in}}%
\pgfpathlineto{\pgfqpoint{3.867951in}{1.666965in}}%
\pgfpathlineto{\pgfqpoint{3.868815in}{1.748917in}}%
\pgfpathlineto{\pgfqpoint{3.869677in}{1.677295in}}%
\pgfpathlineto{\pgfqpoint{3.870542in}{1.731612in}}%
\pgfpathlineto{\pgfqpoint{3.872272in}{1.676878in}}%
\pgfpathlineto{\pgfqpoint{3.873137in}{1.706205in}}%
\pgfpathlineto{\pgfqpoint{3.874002in}{1.623896in}}%
\pgfpathlineto{\pgfqpoint{3.874867in}{1.697775in}}%
\pgfpathlineto{\pgfqpoint{3.875730in}{1.686970in}}%
\pgfpathlineto{\pgfqpoint{3.878324in}{1.638736in}}%
\pgfpathlineto{\pgfqpoint{3.879189in}{1.673138in}}%
\pgfpathlineto{\pgfqpoint{3.880052in}{1.664351in}}%
\pgfpathlineto{\pgfqpoint{3.880914in}{1.568865in}}%
\pgfpathlineto{\pgfqpoint{3.881778in}{1.630039in}}%
\pgfpathlineto{\pgfqpoint{3.882643in}{1.618166in}}%
\pgfpathlineto{\pgfqpoint{3.883507in}{1.618166in}}%
\pgfpathlineto{\pgfqpoint{3.884372in}{1.630753in}}%
\pgfpathlineto{\pgfqpoint{3.886103in}{1.570259in}}%
\pgfpathlineto{\pgfqpoint{3.887836in}{1.664143in}}%
\pgfpathlineto{\pgfqpoint{3.890434in}{1.572308in}}%
\pgfpathlineto{\pgfqpoint{3.891300in}{1.610332in}}%
\pgfpathlineto{\pgfqpoint{3.892166in}{1.584211in}}%
\pgfpathlineto{\pgfqpoint{3.893896in}{1.652479in}}%
\pgfpathlineto{\pgfqpoint{3.895627in}{1.598043in}}%
\pgfpathlineto{\pgfqpoint{3.896493in}{1.632653in}}%
\pgfpathlineto{\pgfqpoint{3.897360in}{1.584151in}}%
\pgfpathlineto{\pgfqpoint{3.898225in}{1.637520in}}%
\pgfpathlineto{\pgfqpoint{3.899958in}{1.600802in}}%
\pgfpathlineto{\pgfqpoint{3.900822in}{1.644109in}}%
\pgfpathlineto{\pgfqpoint{3.901689in}{1.599586in}}%
\pgfpathlineto{\pgfqpoint{3.904283in}{1.673316in}}%
\pgfpathlineto{\pgfqpoint{3.905147in}{1.590030in}}%
\pgfpathlineto{\pgfqpoint{3.906011in}{1.661146in}}%
\pgfpathlineto{\pgfqpoint{3.906875in}{1.562008in}}%
\pgfpathlineto{\pgfqpoint{3.907739in}{1.652241in}}%
\pgfpathlineto{\pgfqpoint{3.908603in}{1.632827in}}%
\pgfpathlineto{\pgfqpoint{3.912926in}{1.583556in}}%
\pgfpathlineto{\pgfqpoint{3.913792in}{1.588542in}}%
\pgfpathlineto{\pgfqpoint{3.914657in}{1.674562in}}%
\pgfpathlineto{\pgfqpoint{3.917253in}{1.597329in}}%
\pgfpathlineto{\pgfqpoint{3.918118in}{1.648976in}}%
\pgfpathlineto{\pgfqpoint{3.918984in}{1.580708in}}%
\pgfpathlineto{\pgfqpoint{3.919849in}{1.666549in}}%
\pgfpathlineto{\pgfqpoint{3.920712in}{1.621907in}}%
\pgfpathlineto{\pgfqpoint{3.921578in}{1.638944in}}%
\pgfpathlineto{\pgfqpoint{3.924173in}{1.583854in}}%
\pgfpathlineto{\pgfqpoint{3.925036in}{1.568597in}}%
\pgfpathlineto{\pgfqpoint{3.928493in}{1.631552in}}%
\pgfpathlineto{\pgfqpoint{3.929358in}{1.643514in}}%
\pgfpathlineto{\pgfqpoint{3.930224in}{1.585397in}}%
\pgfpathlineto{\pgfqpoint{3.932817in}{1.652122in}}%
\pgfpathlineto{\pgfqpoint{3.933682in}{1.595191in}}%
\pgfpathlineto{\pgfqpoint{3.934547in}{1.662273in}}%
\pgfpathlineto{\pgfqpoint{3.935411in}{1.655773in}}%
\pgfpathlineto{\pgfqpoint{3.936274in}{1.691540in}}%
\pgfpathlineto{\pgfqpoint{3.937139in}{1.688334in}}%
\pgfpathlineto{\pgfqpoint{3.939731in}{1.593410in}}%
\pgfpathlineto{\pgfqpoint{3.943191in}{1.668624in}}%
\pgfpathlineto{\pgfqpoint{3.944922in}{1.639476in}}%
\pgfpathlineto{\pgfqpoint{3.945788in}{1.578094in}}%
\pgfpathlineto{\pgfqpoint{3.946653in}{1.631403in}}%
\pgfpathlineto{\pgfqpoint{3.947519in}{1.526714in}}%
\pgfpathlineto{\pgfqpoint{3.949249in}{1.626120in}}%
\pgfpathlineto{\pgfqpoint{3.950114in}{1.595131in}}%
\pgfpathlineto{\pgfqpoint{3.950979in}{1.627901in}}%
\pgfpathlineto{\pgfqpoint{3.951843in}{1.591688in}}%
\pgfpathlineto{\pgfqpoint{3.953573in}{1.663935in}}%
\pgfpathlineto{\pgfqpoint{3.954436in}{1.636062in}}%
\pgfpathlineto{\pgfqpoint{3.957031in}{1.702907in}}%
\pgfpathlineto{\pgfqpoint{3.959623in}{1.639982in}}%
\pgfpathlineto{\pgfqpoint{3.960486in}{1.647136in}}%
\pgfpathlineto{\pgfqpoint{3.961351in}{1.680794in}}%
\pgfpathlineto{\pgfqpoint{3.962216in}{1.670702in}}%
\pgfpathlineto{\pgfqpoint{3.963081in}{1.698426in}}%
\pgfpathlineto{\pgfqpoint{3.963945in}{1.655297in}}%
\pgfpathlineto{\pgfqpoint{3.964811in}{1.701691in}}%
\pgfpathlineto{\pgfqpoint{3.965676in}{1.672721in}}%
\pgfpathlineto{\pgfqpoint{3.966542in}{1.719144in}}%
\pgfpathlineto{\pgfqpoint{3.967407in}{1.653784in}}%
\pgfpathlineto{\pgfqpoint{3.968272in}{1.730779in}}%
\pgfpathlineto{\pgfqpoint{3.969137in}{1.634757in}}%
\pgfpathlineto{\pgfqpoint{3.970867in}{1.708636in}}%
\pgfpathlineto{\pgfqpoint{3.971732in}{1.615314in}}%
\pgfpathlineto{\pgfqpoint{3.973460in}{1.716411in}}%
\pgfpathlineto{\pgfqpoint{3.975191in}{1.653427in}}%
\pgfpathlineto{\pgfqpoint{3.976055in}{1.680407in}}%
\pgfpathlineto{\pgfqpoint{3.978650in}{1.581478in}}%
\pgfpathlineto{\pgfqpoint{3.979516in}{1.591748in}}%
\pgfpathlineto{\pgfqpoint{3.980382in}{1.624339in}}%
\pgfpathlineto{\pgfqpoint{3.981249in}{1.586226in}}%
\pgfpathlineto{\pgfqpoint{3.982115in}{1.628730in}}%
\pgfpathlineto{\pgfqpoint{3.982981in}{1.557792in}}%
\pgfpathlineto{\pgfqpoint{3.983846in}{1.558030in}}%
\pgfpathlineto{\pgfqpoint{3.984712in}{1.585218in}}%
\pgfpathlineto{\pgfqpoint{3.985575in}{1.550906in}}%
\pgfpathlineto{\pgfqpoint{3.986438in}{1.552330in}}%
\pgfpathlineto{\pgfqpoint{3.987304in}{1.539568in}}%
\pgfpathlineto{\pgfqpoint{3.988169in}{1.578302in}}%
\pgfpathlineto{\pgfqpoint{3.989902in}{1.565273in}}%
\pgfpathlineto{\pgfqpoint{3.990765in}{1.625112in}}%
\pgfpathlineto{\pgfqpoint{3.991629in}{1.596321in}}%
\pgfpathlineto{\pgfqpoint{3.992494in}{1.614158in}}%
\pgfpathlineto{\pgfqpoint{3.993360in}{1.528466in}}%
\pgfpathlineto{\pgfqpoint{3.994223in}{1.631522in}}%
\pgfpathlineto{\pgfqpoint{3.995954in}{1.575424in}}%
\pgfpathlineto{\pgfqpoint{3.996820in}{1.676700in}}%
\pgfpathlineto{\pgfqpoint{3.997684in}{1.590621in}}%
\pgfpathlineto{\pgfqpoint{3.998549in}{1.643395in}}%
\pgfpathlineto{\pgfqpoint{4.000279in}{1.584032in}}%
\pgfpathlineto{\pgfqpoint{4.001143in}{1.611280in}}%
\pgfpathlineto{\pgfqpoint{4.002007in}{1.593707in}}%
\pgfpathlineto{\pgfqpoint{4.002874in}{1.614307in}}%
\pgfpathlineto{\pgfqpoint{4.003737in}{1.607896in}}%
\pgfpathlineto{\pgfqpoint{4.004603in}{1.565303in}}%
\pgfpathlineto{\pgfqpoint{4.006327in}{1.606651in}}%
\pgfpathlineto{\pgfqpoint{4.008057in}{1.564500in}}%
\pgfpathlineto{\pgfqpoint{4.008923in}{1.628079in}}%
\pgfpathlineto{\pgfqpoint{4.009788in}{1.527220in}}%
\pgfpathlineto{\pgfqpoint{4.010652in}{1.626536in}}%
\pgfpathlineto{\pgfqpoint{4.011517in}{1.572487in}}%
\pgfpathlineto{\pgfqpoint{4.013246in}{1.615675in}}%
\pgfpathlineto{\pgfqpoint{4.014108in}{1.552066in}}%
\pgfpathlineto{\pgfqpoint{4.014974in}{1.660908in}}%
\pgfpathlineto{\pgfqpoint{4.016704in}{1.578421in}}%
\pgfpathlineto{\pgfqpoint{4.017569in}{1.670940in}}%
\pgfpathlineto{\pgfqpoint{4.018434in}{1.661444in}}%
\pgfpathlineto{\pgfqpoint{4.019301in}{1.558803in}}%
\pgfpathlineto{\pgfqpoint{4.020166in}{1.628615in}}%
\pgfpathlineto{\pgfqpoint{4.021032in}{1.575365in}}%
\pgfpathlineto{\pgfqpoint{4.022764in}{1.614664in}}%
\pgfpathlineto{\pgfqpoint{4.023629in}{1.597180in}}%
\pgfpathlineto{\pgfqpoint{4.024494in}{1.672364in}}%
\pgfpathlineto{\pgfqpoint{4.025359in}{1.588899in}}%
\pgfpathlineto{\pgfqpoint{4.026224in}{1.592670in}}%
\pgfpathlineto{\pgfqpoint{4.027087in}{1.579934in}}%
\pgfpathlineto{\pgfqpoint{4.027953in}{1.660492in}}%
\pgfpathlineto{\pgfqpoint{4.028818in}{1.578332in}}%
\pgfpathlineto{\pgfqpoint{4.030549in}{1.632500in}}%
\pgfpathlineto{\pgfqpoint{4.032277in}{1.595369in}}%
\pgfpathlineto{\pgfqpoint{4.033143in}{1.613741in}}%
\pgfpathlineto{\pgfqpoint{4.034872in}{1.584802in}}%
\pgfpathlineto{\pgfqpoint{4.035739in}{1.557465in}}%
\pgfpathlineto{\pgfqpoint{4.037469in}{1.631582in}}%
\pgfpathlineto{\pgfqpoint{4.038334in}{1.615909in}}%
\pgfpathlineto{\pgfqpoint{4.039199in}{1.679667in}}%
\pgfpathlineto{\pgfqpoint{4.040065in}{1.620185in}}%
\pgfpathlineto{\pgfqpoint{4.040931in}{1.624933in}}%
\pgfpathlineto{\pgfqpoint{4.041797in}{1.674324in}}%
\pgfpathlineto{\pgfqpoint{4.043526in}{1.603680in}}%
\pgfpathlineto{\pgfqpoint{4.044392in}{1.611280in}}%
\pgfpathlineto{\pgfqpoint{4.045257in}{1.635114in}}%
\pgfpathlineto{\pgfqpoint{4.046121in}{1.627250in}}%
\pgfpathlineto{\pgfqpoint{4.046986in}{1.661384in}}%
\pgfpathlineto{\pgfqpoint{4.048715in}{1.632177in}}%
\pgfpathlineto{\pgfqpoint{4.049579in}{1.652271in}}%
\pgfpathlineto{\pgfqpoint{4.050444in}{1.596024in}}%
\pgfpathlineto{\pgfqpoint{4.051310in}{1.647195in}}%
\pgfpathlineto{\pgfqpoint{4.052176in}{1.633660in}}%
\pgfpathlineto{\pgfqpoint{4.053906in}{1.577384in}}%
\pgfpathlineto{\pgfqpoint{4.054771in}{1.613715in}}%
\pgfpathlineto{\pgfqpoint{4.055636in}{1.569192in}}%
\pgfpathlineto{\pgfqpoint{4.057366in}{1.702880in}}%
\pgfpathlineto{\pgfqpoint{4.058228in}{1.656695in}}%
\pgfpathlineto{\pgfqpoint{4.059092in}{1.726269in}}%
\pgfpathlineto{\pgfqpoint{4.060822in}{1.601664in}}%
\pgfpathlineto{\pgfqpoint{4.062552in}{1.581541in}}%
\pgfpathlineto{\pgfqpoint{4.063417in}{1.632712in}}%
\pgfpathlineto{\pgfqpoint{4.064282in}{1.573022in}}%
\pgfpathlineto{\pgfqpoint{4.066012in}{1.665363in}}%
\pgfpathlineto{\pgfqpoint{4.066875in}{1.599824in}}%
\pgfpathlineto{\pgfqpoint{4.067741in}{1.611994in}}%
\pgfpathlineto{\pgfqpoint{4.068604in}{1.620334in}}%
\pgfpathlineto{\pgfqpoint{4.069469in}{1.606829in}}%
\pgfpathlineto{\pgfqpoint{4.070334in}{1.607896in}}%
\pgfpathlineto{\pgfqpoint{4.071198in}{1.639242in}}%
\pgfpathlineto{\pgfqpoint{4.072063in}{1.597983in}}%
\pgfpathlineto{\pgfqpoint{4.072928in}{1.606591in}}%
\pgfpathlineto{\pgfqpoint{4.073792in}{1.572219in}}%
\pgfpathlineto{\pgfqpoint{4.074657in}{1.664589in}}%
\pgfpathlineto{\pgfqpoint{4.075522in}{1.587297in}}%
\pgfpathlineto{\pgfqpoint{4.076388in}{1.604869in}}%
\pgfpathlineto{\pgfqpoint{4.078118in}{1.664500in}}%
\pgfpathlineto{\pgfqpoint{4.078984in}{1.600831in}}%
\pgfpathlineto{\pgfqpoint{4.079849in}{1.670289in}}%
\pgfpathlineto{\pgfqpoint{4.080713in}{1.623807in}}%
\pgfpathlineto{\pgfqpoint{4.082444in}{1.759336in}}%
\pgfpathlineto{\pgfqpoint{4.083309in}{1.695994in}}%
\pgfpathlineto{\pgfqpoint{4.084175in}{1.709707in}}%
\pgfpathlineto{\pgfqpoint{4.085039in}{1.681984in}}%
\pgfpathlineto{\pgfqpoint{4.085903in}{1.696113in}}%
\pgfpathlineto{\pgfqpoint{4.086768in}{1.759038in}}%
\pgfpathlineto{\pgfqpoint{4.088498in}{1.660495in}}%
\pgfpathlineto{\pgfqpoint{4.090229in}{1.741733in}}%
\pgfpathlineto{\pgfqpoint{4.091095in}{1.737312in}}%
\pgfpathlineto{\pgfqpoint{4.091961in}{1.720632in}}%
\pgfpathlineto{\pgfqpoint{4.092826in}{1.757465in}}%
\pgfpathlineto{\pgfqpoint{4.094556in}{1.713686in}}%
\pgfpathlineto{\pgfqpoint{4.095420in}{1.734285in}}%
\pgfpathlineto{\pgfqpoint{4.096286in}{1.705256in}}%
\pgfpathlineto{\pgfqpoint{4.097151in}{1.613656in}}%
\pgfpathlineto{\pgfqpoint{4.098881in}{1.656398in}}%
\pgfpathlineto{\pgfqpoint{4.099746in}{1.620661in}}%
\pgfpathlineto{\pgfqpoint{4.101476in}{1.683408in}}%
\pgfpathlineto{\pgfqpoint{4.103206in}{1.607361in}}%
\pgfpathlineto{\pgfqpoint{4.104071in}{1.666668in}}%
\pgfpathlineto{\pgfqpoint{4.104936in}{1.666311in}}%
\pgfpathlineto{\pgfqpoint{4.106665in}{1.535947in}}%
\pgfpathlineto{\pgfqpoint{4.107530in}{1.642090in}}%
\pgfpathlineto{\pgfqpoint{4.109259in}{1.595875in}}%
\pgfpathlineto{\pgfqpoint{4.110125in}{1.585278in}}%
\pgfpathlineto{\pgfqpoint{4.110990in}{1.611458in}}%
\pgfpathlineto{\pgfqpoint{4.112719in}{1.588840in}}%
\pgfpathlineto{\pgfqpoint{4.113584in}{1.619709in}}%
\pgfpathlineto{\pgfqpoint{4.114449in}{1.605937in}}%
\pgfpathlineto{\pgfqpoint{4.115313in}{1.544614in}}%
\pgfpathlineto{\pgfqpoint{4.116177in}{1.621193in}}%
\pgfpathlineto{\pgfqpoint{4.117040in}{1.586999in}}%
\pgfpathlineto{\pgfqpoint{4.117905in}{1.599348in}}%
\pgfpathlineto{\pgfqpoint{4.118770in}{1.579905in}}%
\pgfpathlineto{\pgfqpoint{4.119636in}{1.628020in}}%
\pgfpathlineto{\pgfqpoint{4.120501in}{1.620657in}}%
\pgfpathlineto{\pgfqpoint{4.121367in}{1.608815in}}%
\pgfpathlineto{\pgfqpoint{4.122231in}{1.610982in}}%
\pgfpathlineto{\pgfqpoint{4.123096in}{1.643692in}}%
\pgfpathlineto{\pgfqpoint{4.124826in}{1.562600in}}%
\pgfpathlineto{\pgfqpoint{4.125692in}{1.591450in}}%
\pgfpathlineto{\pgfqpoint{4.126557in}{1.741287in}}%
\pgfpathlineto{\pgfqpoint{4.128288in}{1.664054in}}%
\pgfpathlineto{\pgfqpoint{4.129154in}{1.671119in}}%
\pgfpathlineto{\pgfqpoint{4.130019in}{1.719501in}}%
\pgfpathlineto{\pgfqpoint{4.130884in}{1.657911in}}%
\pgfpathlineto{\pgfqpoint{4.131746in}{1.665954in}}%
\pgfpathlineto{\pgfqpoint{4.132610in}{1.692135in}}%
\pgfpathlineto{\pgfqpoint{4.133474in}{1.659603in}}%
\pgfpathlineto{\pgfqpoint{4.134339in}{1.692789in}}%
\pgfpathlineto{\pgfqpoint{4.135203in}{1.673614in}}%
\pgfpathlineto{\pgfqpoint{4.136933in}{1.790086in}}%
\pgfpathlineto{\pgfqpoint{4.138664in}{1.686970in}}%
\pgfpathlineto{\pgfqpoint{4.139529in}{1.744258in}}%
\pgfpathlineto{\pgfqpoint{4.140395in}{1.696143in}}%
\pgfpathlineto{\pgfqpoint{4.142124in}{1.727577in}}%
\pgfpathlineto{\pgfqpoint{4.142987in}{1.679135in}}%
\pgfpathlineto{\pgfqpoint{4.143853in}{1.731021in}}%
\pgfpathlineto{\pgfqpoint{4.145583in}{1.695935in}}%
\pgfpathlineto{\pgfqpoint{4.147312in}{1.721550in}}%
\pgfpathlineto{\pgfqpoint{4.149043in}{1.658298in}}%
\pgfpathlineto{\pgfqpoint{4.149908in}{1.651293in}}%
\pgfpathlineto{\pgfqpoint{4.152507in}{1.693916in}}%
\pgfpathlineto{\pgfqpoint{4.153374in}{1.652895in}}%
\pgfpathlineto{\pgfqpoint{4.154234in}{1.709678in}}%
\pgfpathlineto{\pgfqpoint{4.155100in}{1.653074in}}%
\pgfpathlineto{\pgfqpoint{4.157692in}{1.704721in}}%
\pgfpathlineto{\pgfqpoint{4.159420in}{1.669427in}}%
\pgfpathlineto{\pgfqpoint{4.160285in}{1.743128in}}%
\pgfpathlineto{\pgfqpoint{4.161148in}{1.693737in}}%
\pgfpathlineto{\pgfqpoint{4.162013in}{1.726090in}}%
\pgfpathlineto{\pgfqpoint{4.163741in}{1.650727in}}%
\pgfpathlineto{\pgfqpoint{4.164607in}{1.696704in}}%
\pgfpathlineto{\pgfqpoint{4.167203in}{1.652419in}}%
\pgfpathlineto{\pgfqpoint{4.168932in}{1.684534in}}%
\pgfpathlineto{\pgfqpoint{4.169798in}{1.670881in}}%
\pgfpathlineto{\pgfqpoint{4.170663in}{1.636330in}}%
\pgfpathlineto{\pgfqpoint{4.171528in}{1.650400in}}%
\pgfpathlineto{\pgfqpoint{4.172393in}{1.612406in}}%
\pgfpathlineto{\pgfqpoint{4.173258in}{1.631165in}}%
\pgfpathlineto{\pgfqpoint{4.174125in}{1.618464in}}%
\pgfpathlineto{\pgfqpoint{4.174991in}{1.695280in}}%
\pgfpathlineto{\pgfqpoint{4.175858in}{1.616474in}}%
\pgfpathlineto{\pgfqpoint{4.176724in}{1.688275in}}%
\pgfpathlineto{\pgfqpoint{4.178452in}{1.621758in}}%
\pgfpathlineto{\pgfqpoint{4.180184in}{1.681508in}}%
\pgfpathlineto{\pgfqpoint{4.181050in}{1.630217in}}%
\pgfpathlineto{\pgfqpoint{4.181915in}{1.702286in}}%
\pgfpathlineto{\pgfqpoint{4.182780in}{1.664292in}}%
\pgfpathlineto{\pgfqpoint{4.183646in}{1.714039in}}%
\pgfpathlineto{\pgfqpoint{4.184511in}{1.700326in}}%
\pgfpathlineto{\pgfqpoint{4.186240in}{1.688989in}}%
\pgfpathlineto{\pgfqpoint{4.187102in}{1.654319in}}%
\pgfpathlineto{\pgfqpoint{4.187968in}{1.662957in}}%
\pgfpathlineto{\pgfqpoint{4.189700in}{1.704126in}}%
\pgfpathlineto{\pgfqpoint{4.190565in}{1.652598in}}%
\pgfpathlineto{\pgfqpoint{4.191432in}{1.693856in}}%
\pgfpathlineto{\pgfqpoint{4.193163in}{1.643454in}}%
\pgfpathlineto{\pgfqpoint{4.194030in}{1.659187in}}%
\pgfpathlineto{\pgfqpoint{4.194895in}{1.645087in}}%
\pgfpathlineto{\pgfqpoint{4.196623in}{1.662511in}}%
\pgfpathlineto{\pgfqpoint{4.197486in}{1.650460in}}%
\pgfpathlineto{\pgfqpoint{4.199218in}{1.742235in}}%
\pgfpathlineto{\pgfqpoint{4.200083in}{1.643395in}}%
\pgfpathlineto{\pgfqpoint{4.201815in}{1.717601in}}%
\pgfpathlineto{\pgfqpoint{4.202680in}{1.724607in}}%
\pgfpathlineto{\pgfqpoint{4.203545in}{1.705015in}}%
\pgfpathlineto{\pgfqpoint{4.204411in}{1.719055in}}%
\pgfpathlineto{\pgfqpoint{4.205277in}{1.751795in}}%
\pgfpathlineto{\pgfqpoint{4.207008in}{1.655624in}}%
\pgfpathlineto{\pgfqpoint{4.207872in}{1.702048in}}%
\pgfpathlineto{\pgfqpoint{4.208736in}{1.693916in}}%
\pgfpathlineto{\pgfqpoint{4.209602in}{1.657941in}}%
\pgfpathlineto{\pgfqpoint{4.210467in}{1.738022in}}%
\pgfpathlineto{\pgfqpoint{4.212198in}{1.631582in}}%
\pgfpathlineto{\pgfqpoint{4.213060in}{1.697002in}}%
\pgfpathlineto{\pgfqpoint{4.213925in}{1.696913in}}%
\pgfpathlineto{\pgfqpoint{4.214789in}{1.697359in}}%
\pgfpathlineto{\pgfqpoint{4.216520in}{1.714396in}}%
\pgfpathlineto{\pgfqpoint{4.218250in}{1.668151in}}%
\pgfpathlineto{\pgfqpoint{4.219112in}{1.690770in}}%
\pgfpathlineto{\pgfqpoint{4.219976in}{1.675335in}}%
\pgfpathlineto{\pgfqpoint{4.220842in}{1.759157in}}%
\pgfpathlineto{\pgfqpoint{4.221705in}{1.711429in}}%
\pgfpathlineto{\pgfqpoint{4.222571in}{1.740339in}}%
\pgfpathlineto{\pgfqpoint{4.223436in}{1.702702in}}%
\pgfpathlineto{\pgfqpoint{4.224302in}{1.735293in}}%
\pgfpathlineto{\pgfqpoint{4.225166in}{1.662987in}}%
\pgfpathlineto{\pgfqpoint{4.226894in}{1.701869in}}%
\pgfpathlineto{\pgfqpoint{4.227760in}{1.684003in}}%
\pgfpathlineto{\pgfqpoint{4.228626in}{1.704543in}}%
\pgfpathlineto{\pgfqpoint{4.229490in}{1.683705in}}%
\pgfpathlineto{\pgfqpoint{4.230356in}{1.702315in}}%
\pgfpathlineto{\pgfqpoint{4.231219in}{1.644287in}}%
\pgfpathlineto{\pgfqpoint{4.232949in}{1.718137in}}%
\pgfpathlineto{\pgfqpoint{4.233814in}{1.701159in}}%
\pgfpathlineto{\pgfqpoint{4.234678in}{1.734996in}}%
\pgfpathlineto{\pgfqpoint{4.235543in}{1.681865in}}%
\pgfpathlineto{\pgfqpoint{4.236409in}{1.709053in}}%
\pgfpathlineto{\pgfqpoint{4.237273in}{1.702345in}}%
\pgfpathlineto{\pgfqpoint{4.238138in}{1.690056in}}%
\pgfpathlineto{\pgfqpoint{4.239867in}{1.729117in}}%
\pgfpathlineto{\pgfqpoint{4.240732in}{1.700088in}}%
\pgfpathlineto{\pgfqpoint{4.241594in}{1.714396in}}%
\pgfpathlineto{\pgfqpoint{4.242461in}{1.702167in}}%
\pgfpathlineto{\pgfqpoint{4.243325in}{1.717601in}}%
\pgfpathlineto{\pgfqpoint{4.244190in}{1.714515in}}%
\pgfpathlineto{\pgfqpoint{4.245054in}{1.715909in}}%
\pgfpathlineto{\pgfqpoint{4.245919in}{1.752509in}}%
\pgfpathlineto{\pgfqpoint{4.246784in}{1.707748in}}%
\pgfpathlineto{\pgfqpoint{4.248514in}{1.738439in}}%
\pgfpathlineto{\pgfqpoint{4.249380in}{1.750698in}}%
\pgfpathlineto{\pgfqpoint{4.250245in}{1.713210in}}%
\pgfpathlineto{\pgfqpoint{4.251107in}{1.741882in}}%
\pgfpathlineto{\pgfqpoint{4.252837in}{1.699199in}}%
\pgfpathlineto{\pgfqpoint{4.253702in}{1.633422in}}%
\pgfpathlineto{\pgfqpoint{4.254566in}{1.640279in}}%
\pgfpathlineto{\pgfqpoint{4.255431in}{1.651174in}}%
\pgfpathlineto{\pgfqpoint{4.256297in}{1.639242in}}%
\pgfpathlineto{\pgfqpoint{4.257163in}{1.642090in}}%
\pgfpathlineto{\pgfqpoint{4.258027in}{1.704662in}}%
\pgfpathlineto{\pgfqpoint{4.258891in}{1.625528in}}%
\pgfpathlineto{\pgfqpoint{4.259758in}{1.636274in}}%
\pgfpathlineto{\pgfqpoint{4.260621in}{1.672071in}}%
\pgfpathlineto{\pgfqpoint{4.261484in}{1.654587in}}%
\pgfpathlineto{\pgfqpoint{4.263215in}{1.733809in}}%
\pgfpathlineto{\pgfqpoint{4.264079in}{1.666965in}}%
\pgfpathlineto{\pgfqpoint{4.264945in}{1.756547in}}%
\pgfpathlineto{\pgfqpoint{4.265810in}{1.737253in}}%
\pgfpathlineto{\pgfqpoint{4.266675in}{1.667973in}}%
\pgfpathlineto{\pgfqpoint{4.268406in}{1.723301in}}%
\pgfpathlineto{\pgfqpoint{4.270136in}{1.676997in}}%
\pgfpathlineto{\pgfqpoint{4.271865in}{1.714872in}}%
\pgfpathlineto{\pgfqpoint{4.273593in}{1.646277in}}%
\pgfpathlineto{\pgfqpoint{4.275324in}{1.746336in}}%
\pgfpathlineto{\pgfqpoint{4.276188in}{1.727518in}}%
\pgfpathlineto{\pgfqpoint{4.277054in}{1.667382in}}%
\pgfpathlineto{\pgfqpoint{4.277919in}{1.737520in}}%
\pgfpathlineto{\pgfqpoint{4.278785in}{1.654383in}}%
\pgfpathlineto{\pgfqpoint{4.279650in}{1.744674in}}%
\pgfpathlineto{\pgfqpoint{4.281379in}{1.639955in}}%
\pgfpathlineto{\pgfqpoint{4.282243in}{1.641082in}}%
\pgfpathlineto{\pgfqpoint{4.285704in}{1.527339in}}%
\pgfpathlineto{\pgfqpoint{4.289164in}{1.687386in}}%
\pgfpathlineto{\pgfqpoint{4.290894in}{1.535709in}}%
\pgfpathlineto{\pgfqpoint{4.291759in}{1.626953in}}%
\pgfpathlineto{\pgfqpoint{4.292625in}{1.623569in}}%
\pgfpathlineto{\pgfqpoint{4.293491in}{1.587535in}}%
\pgfpathlineto{\pgfqpoint{4.295221in}{1.611399in}}%
\pgfpathlineto{\pgfqpoint{4.296085in}{1.688810in}}%
\pgfpathlineto{\pgfqpoint{4.297812in}{1.619118in}}%
\pgfpathlineto{\pgfqpoint{4.298678in}{1.632593in}}%
\pgfpathlineto{\pgfqpoint{4.299544in}{1.615020in}}%
\pgfpathlineto{\pgfqpoint{4.301276in}{1.684713in}}%
\pgfpathlineto{\pgfqpoint{4.302140in}{1.599824in}}%
\pgfpathlineto{\pgfqpoint{4.303005in}{1.654319in}}%
\pgfpathlineto{\pgfqpoint{4.303869in}{1.611518in}}%
\pgfpathlineto{\pgfqpoint{4.304734in}{1.657762in}}%
\pgfpathlineto{\pgfqpoint{4.306465in}{1.614307in}}%
\pgfpathlineto{\pgfqpoint{4.307331in}{1.661797in}}%
\pgfpathlineto{\pgfqpoint{4.308196in}{1.655565in}}%
\pgfpathlineto{\pgfqpoint{4.311656in}{1.603326in}}%
\pgfpathlineto{\pgfqpoint{4.313387in}{1.646898in}}%
\pgfpathlineto{\pgfqpoint{4.314253in}{1.600177in}}%
\pgfpathlineto{\pgfqpoint{4.316847in}{1.676581in}}%
\pgfpathlineto{\pgfqpoint{4.317712in}{1.664887in}}%
\pgfpathlineto{\pgfqpoint{4.318576in}{1.607866in}}%
\pgfpathlineto{\pgfqpoint{4.319442in}{1.631701in}}%
\pgfpathlineto{\pgfqpoint{4.320307in}{1.580470in}}%
\pgfpathlineto{\pgfqpoint{4.321170in}{1.589346in}}%
\pgfpathlineto{\pgfqpoint{4.323763in}{1.683289in}}%
\pgfpathlineto{\pgfqpoint{4.324629in}{1.679667in}}%
\pgfpathlineto{\pgfqpoint{4.328083in}{1.604988in}}%
\pgfpathlineto{\pgfqpoint{4.329813in}{1.619947in}}%
\pgfpathlineto{\pgfqpoint{4.330678in}{1.602702in}}%
\pgfpathlineto{\pgfqpoint{4.331544in}{1.611458in}}%
\pgfpathlineto{\pgfqpoint{4.332409in}{1.654557in}}%
\pgfpathlineto{\pgfqpoint{4.333276in}{1.583735in}}%
\pgfpathlineto{\pgfqpoint{4.334141in}{1.618285in}}%
\pgfpathlineto{\pgfqpoint{4.335005in}{1.574119in}}%
\pgfpathlineto{\pgfqpoint{4.336733in}{1.645533in}}%
\pgfpathlineto{\pgfqpoint{4.338461in}{1.594540in}}%
\pgfpathlineto{\pgfqpoint{4.339326in}{1.645652in}}%
\pgfpathlineto{\pgfqpoint{4.340193in}{1.591037in}}%
\pgfpathlineto{\pgfqpoint{4.341058in}{1.646128in}}%
\pgfpathlineto{\pgfqpoint{4.341924in}{1.582132in}}%
\pgfpathlineto{\pgfqpoint{4.342786in}{1.616385in}}%
\pgfpathlineto{\pgfqpoint{4.343648in}{1.575008in}}%
\pgfpathlineto{\pgfqpoint{4.345375in}{1.652717in}}%
\pgfpathlineto{\pgfqpoint{4.347106in}{1.587713in}}%
\pgfpathlineto{\pgfqpoint{4.347971in}{1.623450in}}%
\pgfpathlineto{\pgfqpoint{4.348837in}{1.739625in}}%
\pgfpathlineto{\pgfqpoint{4.349701in}{1.621907in}}%
\pgfpathlineto{\pgfqpoint{4.350566in}{1.623628in}}%
\pgfpathlineto{\pgfqpoint{4.351432in}{1.614901in}}%
\pgfpathlineto{\pgfqpoint{4.352298in}{1.621609in}}%
\pgfpathlineto{\pgfqpoint{4.353162in}{1.585635in}}%
\pgfpathlineto{\pgfqpoint{4.354027in}{1.623539in}}%
\pgfpathlineto{\pgfqpoint{4.354892in}{1.597507in}}%
\pgfpathlineto{\pgfqpoint{4.355757in}{1.684118in}}%
\pgfpathlineto{\pgfqpoint{4.356622in}{1.605788in}}%
\pgfpathlineto{\pgfqpoint{4.357487in}{1.613890in}}%
\pgfpathlineto{\pgfqpoint{4.359212in}{1.686137in}}%
\pgfpathlineto{\pgfqpoint{4.360077in}{1.668267in}}%
\pgfpathlineto{\pgfqpoint{4.360942in}{1.686375in}}%
\pgfpathlineto{\pgfqpoint{4.361807in}{1.604096in}}%
\pgfpathlineto{\pgfqpoint{4.362672in}{1.653665in}}%
\pgfpathlineto{\pgfqpoint{4.363537in}{1.593410in}}%
\pgfpathlineto{\pgfqpoint{4.365264in}{1.638290in}}%
\pgfpathlineto{\pgfqpoint{4.366126in}{1.585159in}}%
\pgfpathlineto{\pgfqpoint{4.366989in}{1.593529in}}%
\pgfpathlineto{\pgfqpoint{4.367853in}{1.581775in}}%
\pgfpathlineto{\pgfqpoint{4.368718in}{1.632917in}}%
\pgfpathlineto{\pgfqpoint{4.369583in}{1.582548in}}%
\pgfpathlineto{\pgfqpoint{4.371313in}{1.632058in}}%
\pgfpathlineto{\pgfqpoint{4.372179in}{1.587237in}}%
\pgfpathlineto{\pgfqpoint{4.373044in}{1.617720in}}%
\pgfpathlineto{\pgfqpoint{4.373909in}{1.540576in}}%
\pgfpathlineto{\pgfqpoint{4.374774in}{1.663935in}}%
\pgfpathlineto{\pgfqpoint{4.375640in}{1.614723in}}%
\pgfpathlineto{\pgfqpoint{4.376505in}{1.648381in}}%
\pgfpathlineto{\pgfqpoint{4.377370in}{1.630217in}}%
\pgfpathlineto{\pgfqpoint{4.378235in}{1.533987in}}%
\pgfpathlineto{\pgfqpoint{4.379100in}{1.655446in}}%
\pgfpathlineto{\pgfqpoint{4.379961in}{1.654554in}}%
\pgfpathlineto{\pgfqpoint{4.380827in}{1.636092in}}%
\pgfpathlineto{\pgfqpoint{4.381692in}{1.655386in}}%
\pgfpathlineto{\pgfqpoint{4.382557in}{1.586880in}}%
\pgfpathlineto{\pgfqpoint{4.384290in}{1.657168in}}%
\pgfpathlineto{\pgfqpoint{4.386020in}{1.634252in}}%
\pgfpathlineto{\pgfqpoint{4.386886in}{1.653486in}}%
\pgfpathlineto{\pgfqpoint{4.388616in}{1.572870in}}%
\pgfpathlineto{\pgfqpoint{4.389481in}{1.621371in}}%
\pgfpathlineto{\pgfqpoint{4.390346in}{1.600474in}}%
\pgfpathlineto{\pgfqpoint{4.391212in}{1.667200in}}%
\pgfpathlineto{\pgfqpoint{4.392939in}{1.592342in}}%
\pgfpathlineto{\pgfqpoint{4.394670in}{1.609588in}}%
\pgfpathlineto{\pgfqpoint{4.395533in}{1.672781in}}%
\pgfpathlineto{\pgfqpoint{4.396398in}{1.557554in}}%
\pgfpathlineto{\pgfqpoint{4.397262in}{1.576402in}}%
\pgfpathlineto{\pgfqpoint{4.398127in}{1.609971in}}%
\pgfpathlineto{\pgfqpoint{4.399855in}{1.567526in}}%
\pgfpathlineto{\pgfqpoint{4.400721in}{1.625584in}}%
\pgfpathlineto{\pgfqpoint{4.401586in}{1.619412in}}%
\pgfpathlineto{\pgfqpoint{4.402453in}{1.578391in}}%
\pgfpathlineto{\pgfqpoint{4.403318in}{1.619828in}}%
\pgfpathlineto{\pgfqpoint{4.405913in}{1.588721in}}%
\pgfpathlineto{\pgfqpoint{4.406779in}{1.586999in}}%
\pgfpathlineto{\pgfqpoint{4.407644in}{1.606056in}}%
\pgfpathlineto{\pgfqpoint{4.408510in}{1.679370in}}%
\pgfpathlineto{\pgfqpoint{4.409376in}{1.593588in}}%
\pgfpathlineto{\pgfqpoint{4.410240in}{1.623093in}}%
\pgfpathlineto{\pgfqpoint{4.412836in}{1.587178in}}%
\pgfpathlineto{\pgfqpoint{4.413701in}{1.681299in}}%
\pgfpathlineto{\pgfqpoint{4.414565in}{1.680143in}}%
\pgfpathlineto{\pgfqpoint{4.416295in}{1.590889in}}%
\pgfpathlineto{\pgfqpoint{4.418026in}{1.640249in}}%
\pgfpathlineto{\pgfqpoint{4.420621in}{1.561473in}}%
\pgfpathlineto{\pgfqpoint{4.422347in}{1.634965in}}%
\pgfpathlineto{\pgfqpoint{4.423211in}{1.612942in}}%
\pgfpathlineto{\pgfqpoint{4.424076in}{1.631760in}}%
\pgfpathlineto{\pgfqpoint{4.425802in}{1.513656in}}%
\pgfpathlineto{\pgfqpoint{4.427532in}{1.626417in}}%
\pgfpathlineto{\pgfqpoint{4.428398in}{1.605639in}}%
\pgfpathlineto{\pgfqpoint{4.429263in}{1.667557in}}%
\pgfpathlineto{\pgfqpoint{4.430129in}{1.631731in}}%
\pgfpathlineto{\pgfqpoint{4.430994in}{1.670583in}}%
\pgfpathlineto{\pgfqpoint{4.431859in}{1.648679in}}%
\pgfpathlineto{\pgfqpoint{4.432725in}{1.650043in}}%
\pgfpathlineto{\pgfqpoint{4.433588in}{1.676934in}}%
\pgfpathlineto{\pgfqpoint{4.435319in}{1.561384in}}%
\pgfpathlineto{\pgfqpoint{4.437050in}{1.617690in}}%
\pgfpathlineto{\pgfqpoint{4.437915in}{1.643038in}}%
\pgfpathlineto{\pgfqpoint{4.438780in}{1.626358in}}%
\pgfpathlineto{\pgfqpoint{4.439645in}{1.533690in}}%
\pgfpathlineto{\pgfqpoint{4.441371in}{1.603918in}}%
\pgfpathlineto{\pgfqpoint{4.442235in}{1.586345in}}%
\pgfpathlineto{\pgfqpoint{4.443101in}{1.597091in}}%
\pgfpathlineto{\pgfqpoint{4.443964in}{1.635322in}}%
\pgfpathlineto{\pgfqpoint{4.445695in}{1.585278in}}%
\pgfpathlineto{\pgfqpoint{4.446561in}{1.590026in}}%
\pgfpathlineto{\pgfqpoint{4.447427in}{1.619590in}}%
\pgfpathlineto{\pgfqpoint{4.448292in}{1.616564in}}%
\pgfpathlineto{\pgfqpoint{4.449156in}{1.561473in}}%
\pgfpathlineto{\pgfqpoint{4.450020in}{1.600355in}}%
\pgfpathlineto{\pgfqpoint{4.450885in}{1.538498in}}%
\pgfpathlineto{\pgfqpoint{4.452617in}{1.619531in}}%
\pgfpathlineto{\pgfqpoint{4.454349in}{1.530455in}}%
\pgfpathlineto{\pgfqpoint{4.455213in}{1.619471in}}%
\pgfpathlineto{\pgfqpoint{4.456077in}{1.587237in}}%
\pgfpathlineto{\pgfqpoint{4.456943in}{1.612525in}}%
\pgfpathlineto{\pgfqpoint{4.459537in}{1.556249in}}%
\pgfpathlineto{\pgfqpoint{4.460402in}{1.619828in}}%
\pgfpathlineto{\pgfqpoint{4.461266in}{1.599615in}}%
\pgfpathlineto{\pgfqpoint{4.462131in}{1.616861in}}%
\pgfpathlineto{\pgfqpoint{4.462995in}{1.554170in}}%
\pgfpathlineto{\pgfqpoint{4.464723in}{1.600415in}}%
\pgfpathlineto{\pgfqpoint{4.465588in}{1.599348in}}%
\pgfpathlineto{\pgfqpoint{4.466453in}{1.616207in}}%
\pgfpathlineto{\pgfqpoint{4.468183in}{1.573584in}}%
\pgfpathlineto{\pgfqpoint{4.469913in}{1.617690in}}%
\pgfpathlineto{\pgfqpoint{4.471642in}{1.555713in}}%
\pgfpathlineto{\pgfqpoint{4.472507in}{1.641673in}}%
\pgfpathlineto{\pgfqpoint{4.473373in}{1.611220in}}%
\pgfpathlineto{\pgfqpoint{4.474235in}{1.656160in}}%
\pgfpathlineto{\pgfqpoint{4.475100in}{1.636096in}}%
\pgfpathlineto{\pgfqpoint{4.475966in}{1.690651in}}%
\pgfpathlineto{\pgfqpoint{4.479425in}{1.563730in}}%
\pgfpathlineto{\pgfqpoint{4.480290in}{1.563671in}}%
\pgfpathlineto{\pgfqpoint{4.482023in}{1.662332in}}%
\pgfpathlineto{\pgfqpoint{4.482888in}{1.576967in}}%
\pgfpathlineto{\pgfqpoint{4.483754in}{1.645771in}}%
\pgfpathlineto{\pgfqpoint{4.484620in}{1.631998in}}%
\pgfpathlineto{\pgfqpoint{4.485486in}{1.640190in}}%
\pgfpathlineto{\pgfqpoint{4.487216in}{1.691480in}}%
\pgfpathlineto{\pgfqpoint{4.488081in}{1.582608in}}%
\pgfpathlineto{\pgfqpoint{4.488945in}{1.613299in}}%
\pgfpathlineto{\pgfqpoint{4.489808in}{1.613477in}}%
\pgfpathlineto{\pgfqpoint{4.491537in}{1.702464in}}%
\pgfpathlineto{\pgfqpoint{4.492402in}{1.632088in}}%
\pgfpathlineto{\pgfqpoint{4.493266in}{1.636925in}}%
\pgfpathlineto{\pgfqpoint{4.494133in}{1.725555in}}%
\pgfpathlineto{\pgfqpoint{4.494998in}{1.601958in}}%
\pgfpathlineto{\pgfqpoint{4.495863in}{1.705193in}}%
\pgfpathlineto{\pgfqpoint{4.498455in}{1.627841in}}%
\pgfpathlineto{\pgfqpoint{4.500183in}{1.672662in}}%
\pgfpathlineto{\pgfqpoint{4.501914in}{1.591896in}}%
\pgfpathlineto{\pgfqpoint{4.504511in}{1.646660in}}%
\pgfpathlineto{\pgfqpoint{4.506240in}{1.596972in}}%
\pgfpathlineto{\pgfqpoint{4.507970in}{1.679013in}}%
\pgfpathlineto{\pgfqpoint{4.508835in}{1.586851in}}%
\pgfpathlineto{\pgfqpoint{4.509700in}{1.623152in}}%
\pgfpathlineto{\pgfqpoint{4.510564in}{1.587118in}}%
\pgfpathlineto{\pgfqpoint{4.511429in}{1.604126in}}%
\pgfpathlineto{\pgfqpoint{4.512294in}{1.651824in}}%
\pgfpathlineto{\pgfqpoint{4.514025in}{1.583913in}}%
\pgfpathlineto{\pgfqpoint{4.514890in}{1.578570in}}%
\pgfpathlineto{\pgfqpoint{4.516619in}{1.494153in}}%
\pgfpathlineto{\pgfqpoint{4.517482in}{1.600474in}}%
\pgfpathlineto{\pgfqpoint{4.518346in}{1.595964in}}%
\pgfpathlineto{\pgfqpoint{4.519212in}{1.544317in}}%
\pgfpathlineto{\pgfqpoint{4.520942in}{1.602196in}}%
\pgfpathlineto{\pgfqpoint{4.522670in}{1.640309in}}%
\pgfpathlineto{\pgfqpoint{4.523535in}{1.516712in}}%
\pgfpathlineto{\pgfqpoint{4.524400in}{1.658889in}}%
\pgfpathlineto{\pgfqpoint{4.525263in}{1.569545in}}%
\pgfpathlineto{\pgfqpoint{4.526128in}{1.576848in}}%
\pgfpathlineto{\pgfqpoint{4.526993in}{1.576908in}}%
\pgfpathlineto{\pgfqpoint{4.530452in}{1.612972in}}%
\pgfpathlineto{\pgfqpoint{4.531316in}{1.615909in}}%
\pgfpathlineto{\pgfqpoint{4.532179in}{1.653486in}}%
\pgfpathlineto{\pgfqpoint{4.533043in}{1.538855in}}%
\pgfpathlineto{\pgfqpoint{4.533909in}{1.671059in}}%
\pgfpathlineto{\pgfqpoint{4.534774in}{1.578867in}}%
\pgfpathlineto{\pgfqpoint{4.535638in}{1.600534in}}%
\pgfpathlineto{\pgfqpoint{4.536505in}{1.631284in}}%
\pgfpathlineto{\pgfqpoint{4.538236in}{1.588364in}}%
\pgfpathlineto{\pgfqpoint{4.539965in}{1.620836in}}%
\pgfpathlineto{\pgfqpoint{4.540828in}{1.603858in}}%
\pgfpathlineto{\pgfqpoint{4.541691in}{1.605580in}}%
\pgfpathlineto{\pgfqpoint{4.542555in}{1.668386in}}%
\pgfpathlineto{\pgfqpoint{4.543419in}{1.656989in}}%
\pgfpathlineto{\pgfqpoint{4.544282in}{1.603590in}}%
\pgfpathlineto{\pgfqpoint{4.545145in}{1.673134in}}%
\pgfpathlineto{\pgfqpoint{4.546010in}{1.623446in}}%
\pgfpathlineto{\pgfqpoint{4.546874in}{1.689015in}}%
\pgfpathlineto{\pgfqpoint{4.548603in}{1.624279in}}%
\pgfpathlineto{\pgfqpoint{4.549468in}{1.653962in}}%
\pgfpathlineto{\pgfqpoint{4.550331in}{1.650400in}}%
\pgfpathlineto{\pgfqpoint{4.552060in}{1.598277in}}%
\pgfpathlineto{\pgfqpoint{4.552925in}{1.677113in}}%
\pgfpathlineto{\pgfqpoint{4.554650in}{1.615909in}}%
\pgfpathlineto{\pgfqpoint{4.555516in}{1.608963in}}%
\pgfpathlineto{\pgfqpoint{4.556379in}{1.629741in}}%
\pgfpathlineto{\pgfqpoint{4.558975in}{1.542268in}}%
\pgfpathlineto{\pgfqpoint{4.559839in}{1.631820in}}%
\pgfpathlineto{\pgfqpoint{4.560705in}{1.553873in}}%
\pgfpathlineto{\pgfqpoint{4.561571in}{1.574889in}}%
\pgfpathlineto{\pgfqpoint{4.562436in}{1.586940in}}%
\pgfpathlineto{\pgfqpoint{4.563300in}{1.585605in}}%
\pgfpathlineto{\pgfqpoint{4.565028in}{1.599288in}}%
\pgfpathlineto{\pgfqpoint{4.565894in}{1.579046in}}%
\pgfpathlineto{\pgfqpoint{4.568487in}{1.701929in}}%
\pgfpathlineto{\pgfqpoint{4.569351in}{1.615734in}}%
\pgfpathlineto{\pgfqpoint{4.570217in}{1.620304in}}%
\pgfpathlineto{\pgfqpoint{4.571084in}{1.668211in}}%
\pgfpathlineto{\pgfqpoint{4.572814in}{1.612793in}}%
\pgfpathlineto{\pgfqpoint{4.574545in}{1.659365in}}%
\pgfpathlineto{\pgfqpoint{4.575410in}{1.571713in}}%
\pgfpathlineto{\pgfqpoint{4.576275in}{1.668386in}}%
\pgfpathlineto{\pgfqpoint{4.578869in}{1.566697in}}%
\pgfpathlineto{\pgfqpoint{4.579732in}{1.740752in}}%
\pgfpathlineto{\pgfqpoint{4.580597in}{1.610566in}}%
\pgfpathlineto{\pgfqpoint{4.581462in}{1.640368in}}%
\pgfpathlineto{\pgfqpoint{4.583189in}{1.565154in}}%
\pgfpathlineto{\pgfqpoint{4.584053in}{1.624993in}}%
\pgfpathlineto{\pgfqpoint{4.584916in}{1.540993in}}%
\pgfpathlineto{\pgfqpoint{4.586645in}{1.604691in}}%
\pgfpathlineto{\pgfqpoint{4.587511in}{1.604750in}}%
\pgfpathlineto{\pgfqpoint{4.588377in}{1.568478in}}%
\pgfpathlineto{\pgfqpoint{4.589242in}{1.643930in}}%
\pgfpathlineto{\pgfqpoint{4.590109in}{1.624815in}}%
\pgfpathlineto{\pgfqpoint{4.590975in}{1.634965in}}%
\pgfpathlineto{\pgfqpoint{4.592706in}{1.583675in}}%
\pgfpathlineto{\pgfqpoint{4.593571in}{1.643633in}}%
\pgfpathlineto{\pgfqpoint{4.597029in}{1.521579in}}%
\pgfpathlineto{\pgfqpoint{4.597894in}{1.623863in}}%
\pgfpathlineto{\pgfqpoint{4.598760in}{1.595667in}}%
\pgfpathlineto{\pgfqpoint{4.599625in}{1.525141in}}%
\pgfpathlineto{\pgfqpoint{4.601354in}{1.578927in}}%
\pgfpathlineto{\pgfqpoint{4.602218in}{1.581422in}}%
\pgfpathlineto{\pgfqpoint{4.603082in}{1.654438in}}%
\pgfpathlineto{\pgfqpoint{4.603947in}{1.555955in}}%
\pgfpathlineto{\pgfqpoint{4.604812in}{1.566106in}}%
\pgfpathlineto{\pgfqpoint{4.605675in}{1.615675in}}%
\pgfpathlineto{\pgfqpoint{4.606541in}{1.543398in}}%
\pgfpathlineto{\pgfqpoint{4.607407in}{1.545566in}}%
\pgfpathlineto{\pgfqpoint{4.608272in}{1.525855in}}%
\pgfpathlineto{\pgfqpoint{4.609137in}{1.532920in}}%
\pgfpathlineto{\pgfqpoint{4.610002in}{1.610213in}}%
\pgfpathlineto{\pgfqpoint{4.610868in}{1.520482in}}%
\pgfpathlineto{\pgfqpoint{4.612594in}{1.649036in}}%
\pgfpathlineto{\pgfqpoint{4.614321in}{1.542833in}}%
\pgfpathlineto{\pgfqpoint{4.616052in}{1.607837in}}%
\pgfpathlineto{\pgfqpoint{4.616918in}{1.558268in}}%
\pgfpathlineto{\pgfqpoint{4.617784in}{1.558446in}}%
\pgfpathlineto{\pgfqpoint{4.618650in}{1.532682in}}%
\pgfpathlineto{\pgfqpoint{4.619515in}{1.610213in}}%
\pgfpathlineto{\pgfqpoint{4.620380in}{1.605226in}}%
\pgfpathlineto{\pgfqpoint{4.621246in}{1.644406in}}%
\pgfpathlineto{\pgfqpoint{4.622110in}{1.629269in}}%
\pgfpathlineto{\pgfqpoint{4.622974in}{1.587088in}}%
\pgfpathlineto{\pgfqpoint{4.624705in}{1.604334in}}%
\pgfpathlineto{\pgfqpoint{4.625570in}{1.596053in}}%
\pgfpathlineto{\pgfqpoint{4.626436in}{1.676997in}}%
\pgfpathlineto{\pgfqpoint{4.627301in}{1.630098in}}%
\pgfpathlineto{\pgfqpoint{4.628165in}{1.649512in}}%
\pgfpathlineto{\pgfqpoint{4.629028in}{1.638766in}}%
\pgfpathlineto{\pgfqpoint{4.629894in}{1.594778in}}%
\pgfpathlineto{\pgfqpoint{4.630756in}{1.650519in}}%
\pgfpathlineto{\pgfqpoint{4.631622in}{1.645890in}}%
\pgfpathlineto{\pgfqpoint{4.632489in}{1.659930in}}%
\pgfpathlineto{\pgfqpoint{4.633355in}{1.641082in}}%
\pgfpathlineto{\pgfqpoint{4.634220in}{1.574357in}}%
\pgfpathlineto{\pgfqpoint{4.635081in}{1.574387in}}%
\pgfpathlineto{\pgfqpoint{4.635945in}{1.651293in}}%
\pgfpathlineto{\pgfqpoint{4.637675in}{1.621847in}}%
\pgfpathlineto{\pgfqpoint{4.638541in}{1.663225in}}%
\pgfpathlineto{\pgfqpoint{4.640272in}{1.518969in}}%
\pgfpathlineto{\pgfqpoint{4.641137in}{1.622442in}}%
\pgfpathlineto{\pgfqpoint{4.642003in}{1.608313in}}%
\pgfpathlineto{\pgfqpoint{4.642868in}{1.618464in}}%
\pgfpathlineto{\pgfqpoint{4.643733in}{1.669992in}}%
\pgfpathlineto{\pgfqpoint{4.647191in}{1.526331in}}%
\pgfpathlineto{\pgfqpoint{4.648057in}{1.622799in}}%
\pgfpathlineto{\pgfqpoint{4.648923in}{1.557587in}}%
\pgfpathlineto{\pgfqpoint{4.649788in}{1.576852in}}%
\pgfpathlineto{\pgfqpoint{4.650652in}{1.569430in}}%
\pgfpathlineto{\pgfqpoint{4.652381in}{1.617337in}}%
\pgfpathlineto{\pgfqpoint{4.654112in}{1.569044in}}%
\pgfpathlineto{\pgfqpoint{4.654976in}{1.563909in}}%
\pgfpathlineto{\pgfqpoint{4.655843in}{1.602493in}}%
\pgfpathlineto{\pgfqpoint{4.656707in}{1.542536in}}%
\pgfpathlineto{\pgfqpoint{4.658437in}{1.609618in}}%
\pgfpathlineto{\pgfqpoint{4.660168in}{1.562008in}}%
\pgfpathlineto{\pgfqpoint{4.662761in}{1.647552in}}%
\pgfpathlineto{\pgfqpoint{4.663625in}{1.573970in}}%
\pgfpathlineto{\pgfqpoint{4.664490in}{1.575841in}}%
\pgfpathlineto{\pgfqpoint{4.665354in}{1.560525in}}%
\pgfpathlineto{\pgfqpoint{4.666218in}{1.589316in}}%
\pgfpathlineto{\pgfqpoint{4.667082in}{1.580589in}}%
\pgfpathlineto{\pgfqpoint{4.667948in}{1.532682in}}%
\pgfpathlineto{\pgfqpoint{4.668812in}{1.640190in}}%
\pgfpathlineto{\pgfqpoint{4.669677in}{1.561592in}}%
\pgfpathlineto{\pgfqpoint{4.670542in}{1.599318in}}%
\pgfpathlineto{\pgfqpoint{4.672272in}{1.567590in}}%
\pgfpathlineto{\pgfqpoint{4.674000in}{1.589613in}}%
\pgfpathlineto{\pgfqpoint{4.674866in}{1.629269in}}%
\pgfpathlineto{\pgfqpoint{4.675730in}{1.566816in}}%
\pgfpathlineto{\pgfqpoint{4.677460in}{1.611369in}}%
\pgfpathlineto{\pgfqpoint{4.678325in}{1.575722in}}%
\pgfpathlineto{\pgfqpoint{4.679190in}{1.618583in}}%
\pgfpathlineto{\pgfqpoint{4.680055in}{1.569222in}}%
\pgfpathlineto{\pgfqpoint{4.682650in}{1.632474in}}%
\pgfpathlineto{\pgfqpoint{4.683516in}{1.626123in}}%
\pgfpathlineto{\pgfqpoint{4.684380in}{1.570616in}}%
\pgfpathlineto{\pgfqpoint{4.685244in}{1.606532in}}%
\pgfpathlineto{\pgfqpoint{4.686110in}{1.592878in}}%
\pgfpathlineto{\pgfqpoint{4.687840in}{1.641023in}}%
\pgfpathlineto{\pgfqpoint{4.688705in}{1.609439in}}%
\pgfpathlineto{\pgfqpoint{4.690432in}{1.680084in}}%
\pgfpathlineto{\pgfqpoint{4.693027in}{1.540398in}}%
\pgfpathlineto{\pgfqpoint{4.693889in}{1.631582in}}%
\pgfpathlineto{\pgfqpoint{4.694755in}{1.576729in}}%
\pgfpathlineto{\pgfqpoint{4.695621in}{1.627901in}}%
\pgfpathlineto{\pgfqpoint{4.696486in}{1.590651in}}%
\pgfpathlineto{\pgfqpoint{4.697352in}{1.612823in}}%
\pgfpathlineto{\pgfqpoint{4.698218in}{1.591275in}}%
\pgfpathlineto{\pgfqpoint{4.699949in}{1.619888in}}%
\pgfpathlineto{\pgfqpoint{4.700812in}{1.620245in}}%
\pgfpathlineto{\pgfqpoint{4.701675in}{1.545919in}}%
\pgfpathlineto{\pgfqpoint{4.702540in}{1.595905in}}%
\pgfpathlineto{\pgfqpoint{4.704271in}{1.538736in}}%
\pgfpathlineto{\pgfqpoint{4.705999in}{1.698128in}}%
\pgfpathlineto{\pgfqpoint{4.708594in}{1.665835in}}%
\pgfpathlineto{\pgfqpoint{4.709461in}{1.728700in}}%
\pgfpathlineto{\pgfqpoint{4.710325in}{1.727395in}}%
\pgfpathlineto{\pgfqpoint{4.711190in}{1.626179in}}%
\pgfpathlineto{\pgfqpoint{4.712921in}{1.698485in}}%
\pgfpathlineto{\pgfqpoint{4.713785in}{1.645533in}}%
\pgfpathlineto{\pgfqpoint{4.714648in}{1.686137in}}%
\pgfpathlineto{\pgfqpoint{4.715513in}{1.637371in}}%
\pgfpathlineto{\pgfqpoint{4.718111in}{1.704926in}}%
\pgfpathlineto{\pgfqpoint{4.718976in}{1.633303in}}%
\pgfpathlineto{\pgfqpoint{4.719839in}{1.690472in}}%
\pgfpathlineto{\pgfqpoint{4.720705in}{1.653516in}}%
\pgfpathlineto{\pgfqpoint{4.721569in}{1.667557in}}%
\pgfpathlineto{\pgfqpoint{4.723300in}{1.744373in}}%
\pgfpathlineto{\pgfqpoint{4.724166in}{1.724666in}}%
\pgfpathlineto{\pgfqpoint{4.725030in}{1.648887in}}%
\pgfpathlineto{\pgfqpoint{4.725894in}{1.706260in}}%
\pgfpathlineto{\pgfqpoint{4.726759in}{1.659008in}}%
\pgfpathlineto{\pgfqpoint{4.728487in}{1.700326in}}%
\pgfpathlineto{\pgfqpoint{4.729353in}{1.613001in}}%
\pgfpathlineto{\pgfqpoint{4.731080in}{1.673078in}}%
\pgfpathlineto{\pgfqpoint{4.731947in}{1.624993in}}%
\pgfpathlineto{\pgfqpoint{4.732812in}{1.688453in}}%
\pgfpathlineto{\pgfqpoint{4.733678in}{1.683289in}}%
\pgfpathlineto{\pgfqpoint{4.734543in}{1.639093in}}%
\pgfpathlineto{\pgfqpoint{4.736271in}{1.720453in}}%
\pgfpathlineto{\pgfqpoint{4.737133in}{1.678868in}}%
\pgfpathlineto{\pgfqpoint{4.737997in}{1.724904in}}%
\pgfpathlineto{\pgfqpoint{4.739729in}{1.650846in}}%
\pgfpathlineto{\pgfqpoint{4.740595in}{1.762719in}}%
\pgfpathlineto{\pgfqpoint{4.741460in}{1.633660in}}%
\pgfpathlineto{\pgfqpoint{4.742327in}{1.707153in}}%
\pgfpathlineto{\pgfqpoint{4.743193in}{1.637460in}}%
\pgfpathlineto{\pgfqpoint{4.744060in}{1.705967in}}%
\pgfpathlineto{\pgfqpoint{4.744925in}{1.665363in}}%
\pgfpathlineto{\pgfqpoint{4.745789in}{1.686200in}}%
\pgfpathlineto{\pgfqpoint{4.746655in}{1.600567in}}%
\pgfpathlineto{\pgfqpoint{4.747518in}{1.692492in}}%
\pgfpathlineto{\pgfqpoint{4.748384in}{1.674268in}}%
\pgfpathlineto{\pgfqpoint{4.749246in}{1.644644in}}%
\pgfpathlineto{\pgfqpoint{4.750977in}{1.746630in}}%
\pgfpathlineto{\pgfqpoint{4.751842in}{1.683527in}}%
\pgfpathlineto{\pgfqpoint{4.752708in}{1.762184in}}%
\pgfpathlineto{\pgfqpoint{4.754440in}{1.688751in}}%
\pgfpathlineto{\pgfqpoint{4.755305in}{1.705550in}}%
\pgfpathlineto{\pgfqpoint{4.756170in}{1.664173in}}%
\pgfpathlineto{\pgfqpoint{4.757035in}{1.667438in}}%
\pgfpathlineto{\pgfqpoint{4.757901in}{1.693797in}}%
\pgfpathlineto{\pgfqpoint{4.758765in}{1.688453in}}%
\pgfpathlineto{\pgfqpoint{4.759630in}{1.818401in}}%
\pgfpathlineto{\pgfqpoint{4.761358in}{1.609558in}}%
\pgfpathlineto{\pgfqpoint{4.762223in}{1.631582in}}%
\pgfpathlineto{\pgfqpoint{4.763088in}{1.602702in}}%
\pgfpathlineto{\pgfqpoint{4.763954in}{1.705907in}}%
\pgfpathlineto{\pgfqpoint{4.766549in}{1.592640in}}%
\pgfpathlineto{\pgfqpoint{4.767415in}{1.659484in}}%
\pgfpathlineto{\pgfqpoint{4.770012in}{1.557911in}}%
\pgfpathlineto{\pgfqpoint{4.771744in}{1.640309in}}%
\pgfpathlineto{\pgfqpoint{4.773477in}{1.607896in}}%
\pgfpathlineto{\pgfqpoint{4.774342in}{1.566043in}}%
\pgfpathlineto{\pgfqpoint{4.775208in}{1.612853in}}%
\pgfpathlineto{\pgfqpoint{4.776073in}{1.585694in}}%
\pgfpathlineto{\pgfqpoint{4.776937in}{1.643811in}}%
\pgfpathlineto{\pgfqpoint{4.777802in}{1.559603in}}%
\pgfpathlineto{\pgfqpoint{4.778668in}{1.640249in}}%
\pgfpathlineto{\pgfqpoint{4.781267in}{1.552806in}}%
\pgfpathlineto{\pgfqpoint{4.782133in}{1.636895in}}%
\pgfpathlineto{\pgfqpoint{4.783865in}{1.561176in}}%
\pgfpathlineto{\pgfqpoint{4.784730in}{1.649036in}}%
\pgfpathlineto{\pgfqpoint{4.785596in}{1.624874in}}%
\pgfpathlineto{\pgfqpoint{4.786461in}{1.628198in}}%
\pgfpathlineto{\pgfqpoint{4.787327in}{1.585813in}}%
\pgfpathlineto{\pgfqpoint{4.788193in}{1.670940in}}%
\pgfpathlineto{\pgfqpoint{4.789925in}{1.574948in}}%
\pgfpathlineto{\pgfqpoint{4.790791in}{1.606115in}}%
\pgfpathlineto{\pgfqpoint{4.792520in}{1.527815in}}%
\pgfpathlineto{\pgfqpoint{4.793384in}{1.622855in}}%
\pgfpathlineto{\pgfqpoint{4.794249in}{1.544703in}}%
\pgfpathlineto{\pgfqpoint{4.795115in}{1.583794in}}%
\pgfpathlineto{\pgfqpoint{4.795980in}{1.704126in}}%
\pgfpathlineto{\pgfqpoint{4.796845in}{1.663225in}}%
\pgfpathlineto{\pgfqpoint{4.798574in}{1.786464in}}%
\pgfpathlineto{\pgfqpoint{4.801166in}{1.644585in}}%
\pgfpathlineto{\pgfqpoint{4.802893in}{1.697894in}}%
\pgfpathlineto{\pgfqpoint{4.803757in}{1.675365in}}%
\pgfpathlineto{\pgfqpoint{4.804623in}{1.705967in}}%
\pgfpathlineto{\pgfqpoint{4.806350in}{1.532742in}}%
\pgfpathlineto{\pgfqpoint{4.808078in}{1.650698in}}%
\pgfpathlineto{\pgfqpoint{4.808943in}{1.592997in}}%
\pgfpathlineto{\pgfqpoint{4.809809in}{1.595075in}}%
\pgfpathlineto{\pgfqpoint{4.812399in}{1.733452in}}%
\pgfpathlineto{\pgfqpoint{4.813264in}{1.684032in}}%
\pgfpathlineto{\pgfqpoint{4.814129in}{1.690651in}}%
\pgfpathlineto{\pgfqpoint{4.814993in}{1.743841in}}%
\pgfpathlineto{\pgfqpoint{4.815858in}{1.669814in}}%
\pgfpathlineto{\pgfqpoint{4.816722in}{1.745147in}}%
\pgfpathlineto{\pgfqpoint{4.818454in}{1.616207in}}%
\pgfpathlineto{\pgfqpoint{4.820184in}{1.693945in}}%
\pgfpathlineto{\pgfqpoint{4.821048in}{1.661146in}}%
\pgfpathlineto{\pgfqpoint{4.821914in}{1.690710in}}%
\pgfpathlineto{\pgfqpoint{4.822781in}{1.690562in}}%
\pgfpathlineto{\pgfqpoint{4.824513in}{1.620840in}}%
\pgfpathlineto{\pgfqpoint{4.825379in}{1.630782in}}%
\pgfpathlineto{\pgfqpoint{4.826244in}{1.629269in}}%
\pgfpathlineto{\pgfqpoint{4.827110in}{1.714575in}}%
\pgfpathlineto{\pgfqpoint{4.827976in}{1.582846in}}%
\pgfpathlineto{\pgfqpoint{4.830573in}{1.694154in}}%
\pgfpathlineto{\pgfqpoint{4.831440in}{1.735115in}}%
\pgfpathlineto{\pgfqpoint{4.832305in}{1.532385in}}%
\pgfpathlineto{\pgfqpoint{4.833170in}{1.534523in}}%
\pgfpathlineto{\pgfqpoint{4.834033in}{1.589375in}}%
\pgfpathlineto{\pgfqpoint{4.834897in}{1.574000in}}%
\pgfpathlineto{\pgfqpoint{4.835762in}{1.569014in}}%
\pgfpathlineto{\pgfqpoint{4.836629in}{1.612882in}}%
\pgfpathlineto{\pgfqpoint{4.838359in}{1.565273in}}%
\pgfpathlineto{\pgfqpoint{4.839223in}{1.508937in}}%
\pgfpathlineto{\pgfqpoint{4.840953in}{1.639004in}}%
\pgfpathlineto{\pgfqpoint{4.842682in}{1.551917in}}%
\pgfpathlineto{\pgfqpoint{4.842682in}{1.551917in}}%
\pgfusepath{stroke}%
\end{pgfscope}%
\begin{pgfscope}%
\pgfsetrectcap%
\pgfsetmiterjoin%
\pgfsetlinewidth{0.803000pt}%
\definecolor{currentstroke}{rgb}{0.000000,0.000000,0.000000}%
\pgfsetstrokecolor{currentstroke}%
\pgfsetdash{}{0pt}%
\pgfpathmoveto{\pgfqpoint{0.483776in}{1.444834in}}%
\pgfpathlineto{\pgfqpoint{0.483776in}{2.029715in}}%
\pgfusepath{stroke}%
\end{pgfscope}%
\begin{pgfscope}%
\pgfsetrectcap%
\pgfsetmiterjoin%
\pgfsetlinewidth{0.803000pt}%
\definecolor{currentstroke}{rgb}{0.000000,0.000000,0.000000}%
\pgfsetstrokecolor{currentstroke}%
\pgfsetdash{}{0pt}%
\pgfpathmoveto{\pgfqpoint{5.050249in}{1.444834in}}%
\pgfpathlineto{\pgfqpoint{5.050249in}{2.029715in}}%
\pgfusepath{stroke}%
\end{pgfscope}%
\begin{pgfscope}%
\pgfsetrectcap%
\pgfsetmiterjoin%
\pgfsetlinewidth{0.803000pt}%
\definecolor{currentstroke}{rgb}{0.000000,0.000000,0.000000}%
\pgfsetstrokecolor{currentstroke}%
\pgfsetdash{}{0pt}%
\pgfpathmoveto{\pgfqpoint{0.483776in}{1.444834in}}%
\pgfpathlineto{\pgfqpoint{5.050249in}{1.444834in}}%
\pgfusepath{stroke}%
\end{pgfscope}%
\begin{pgfscope}%
\pgfsetrectcap%
\pgfsetmiterjoin%
\pgfsetlinewidth{0.803000pt}%
\definecolor{currentstroke}{rgb}{0.000000,0.000000,0.000000}%
\pgfsetstrokecolor{currentstroke}%
\pgfsetdash{}{0pt}%
\pgfpathmoveto{\pgfqpoint{0.483776in}{2.029715in}}%
\pgfpathlineto{\pgfqpoint{5.050249in}{2.029715in}}%
\pgfusepath{stroke}%
\end{pgfscope}%
\begin{pgfscope}%
\pgfsetbuttcap%
\pgfsetmiterjoin%
\definecolor{currentfill}{rgb}{1.000000,1.000000,1.000000}%
\pgfsetfillcolor{currentfill}%
\pgfsetlinewidth{0.000000pt}%
\definecolor{currentstroke}{rgb}{0.000000,0.000000,0.000000}%
\pgfsetstrokecolor{currentstroke}%
\pgfsetstrokeopacity{0.000000}%
\pgfsetdash{}{0pt}%
\pgfpathmoveto{\pgfqpoint{0.483776in}{0.538014in}}%
\pgfpathlineto{\pgfqpoint{5.050249in}{0.538014in}}%
\pgfpathlineto{\pgfqpoint{5.050249in}{1.122895in}}%
\pgfpathlineto{\pgfqpoint{0.483776in}{1.122895in}}%
\pgfpathlineto{\pgfqpoint{0.483776in}{0.538014in}}%
\pgfpathclose%
\pgfusepath{fill}%
\end{pgfscope}%
\begin{pgfscope}%
\pgfsetbuttcap%
\pgfsetroundjoin%
\definecolor{currentfill}{rgb}{0.000000,0.000000,0.000000}%
\pgfsetfillcolor{currentfill}%
\pgfsetlinewidth{0.803000pt}%
\definecolor{currentstroke}{rgb}{0.000000,0.000000,0.000000}%
\pgfsetstrokecolor{currentstroke}%
\pgfsetdash{}{0pt}%
\pgfsys@defobject{currentmarker}{\pgfqpoint{0.000000in}{-0.048611in}}{\pgfqpoint{0.000000in}{0.000000in}}{%
\pgfpathmoveto{\pgfqpoint{0.000000in}{0.000000in}}%
\pgfpathlineto{\pgfqpoint{0.000000in}{-0.048611in}}%
\pgfusepath{stroke,fill}%
}%
\begin{pgfscope}%
\pgfsys@transformshift{0.691021in}{0.538014in}%
\pgfsys@useobject{currentmarker}{}%
\end{pgfscope}%
\end{pgfscope}%
\begin{pgfscope}%
\definecolor{textcolor}{rgb}{0.000000,0.000000,0.000000}%
\pgfsetstrokecolor{textcolor}%
\pgfsetfillcolor{textcolor}%
\pgftext[x=0.691021in,y=0.440792in,,top]{\color{textcolor}\rmfamily\fontsize{8.000000}{9.600000}\selectfont \(\displaystyle {06{:}00}\)}%
\end{pgfscope}%
\begin{pgfscope}%
\pgfsetbuttcap%
\pgfsetroundjoin%
\definecolor{currentfill}{rgb}{0.000000,0.000000,0.000000}%
\pgfsetfillcolor{currentfill}%
\pgfsetlinewidth{0.803000pt}%
\definecolor{currentstroke}{rgb}{0.000000,0.000000,0.000000}%
\pgfsetstrokecolor{currentstroke}%
\pgfsetdash{}{0pt}%
\pgfsys@defobject{currentmarker}{\pgfqpoint{0.000000in}{-0.048611in}}{\pgfqpoint{0.000000in}{0.000000in}}{%
\pgfpathmoveto{\pgfqpoint{0.000000in}{0.000000in}}%
\pgfpathlineto{\pgfqpoint{0.000000in}{-0.048611in}}%
\pgfusepath{stroke,fill}%
}%
\begin{pgfscope}%
\pgfsys@transformshift{1.210067in}{0.538014in}%
\pgfsys@useobject{currentmarker}{}%
\end{pgfscope}%
\end{pgfscope}%
\begin{pgfscope}%
\definecolor{textcolor}{rgb}{0.000000,0.000000,0.000000}%
\pgfsetstrokecolor{textcolor}%
\pgfsetfillcolor{textcolor}%
\pgftext[x=1.210067in,y=0.440792in,,top]{\color{textcolor}\rmfamily\fontsize{8.000000}{9.600000}\selectfont \(\displaystyle {09{:}00}\)}%
\end{pgfscope}%
\begin{pgfscope}%
\pgfsetbuttcap%
\pgfsetroundjoin%
\definecolor{currentfill}{rgb}{0.000000,0.000000,0.000000}%
\pgfsetfillcolor{currentfill}%
\pgfsetlinewidth{0.803000pt}%
\definecolor{currentstroke}{rgb}{0.000000,0.000000,0.000000}%
\pgfsetstrokecolor{currentstroke}%
\pgfsetdash{}{0pt}%
\pgfsys@defobject{currentmarker}{\pgfqpoint{0.000000in}{-0.048611in}}{\pgfqpoint{0.000000in}{0.000000in}}{%
\pgfpathmoveto{\pgfqpoint{0.000000in}{0.000000in}}%
\pgfpathlineto{\pgfqpoint{0.000000in}{-0.048611in}}%
\pgfusepath{stroke,fill}%
}%
\begin{pgfscope}%
\pgfsys@transformshift{1.729114in}{0.538014in}%
\pgfsys@useobject{currentmarker}{}%
\end{pgfscope}%
\end{pgfscope}%
\begin{pgfscope}%
\definecolor{textcolor}{rgb}{0.000000,0.000000,0.000000}%
\pgfsetstrokecolor{textcolor}%
\pgfsetfillcolor{textcolor}%
\pgftext[x=1.729114in,y=0.440792in,,top]{\color{textcolor}\rmfamily\fontsize{8.000000}{9.600000}\selectfont \(\displaystyle {12{:}00}\)}%
\end{pgfscope}%
\begin{pgfscope}%
\pgfsetbuttcap%
\pgfsetroundjoin%
\definecolor{currentfill}{rgb}{0.000000,0.000000,0.000000}%
\pgfsetfillcolor{currentfill}%
\pgfsetlinewidth{0.803000pt}%
\definecolor{currentstroke}{rgb}{0.000000,0.000000,0.000000}%
\pgfsetstrokecolor{currentstroke}%
\pgfsetdash{}{0pt}%
\pgfsys@defobject{currentmarker}{\pgfqpoint{0.000000in}{-0.048611in}}{\pgfqpoint{0.000000in}{0.000000in}}{%
\pgfpathmoveto{\pgfqpoint{0.000000in}{0.000000in}}%
\pgfpathlineto{\pgfqpoint{0.000000in}{-0.048611in}}%
\pgfusepath{stroke,fill}%
}%
\begin{pgfscope}%
\pgfsys@transformshift{2.248160in}{0.538014in}%
\pgfsys@useobject{currentmarker}{}%
\end{pgfscope}%
\end{pgfscope}%
\begin{pgfscope}%
\definecolor{textcolor}{rgb}{0.000000,0.000000,0.000000}%
\pgfsetstrokecolor{textcolor}%
\pgfsetfillcolor{textcolor}%
\pgftext[x=2.248160in,y=0.440792in,,top]{\color{textcolor}\rmfamily\fontsize{8.000000}{9.600000}\selectfont \(\displaystyle {15{:}00}\)}%
\end{pgfscope}%
\begin{pgfscope}%
\pgfsetbuttcap%
\pgfsetroundjoin%
\definecolor{currentfill}{rgb}{0.000000,0.000000,0.000000}%
\pgfsetfillcolor{currentfill}%
\pgfsetlinewidth{0.803000pt}%
\definecolor{currentstroke}{rgb}{0.000000,0.000000,0.000000}%
\pgfsetstrokecolor{currentstroke}%
\pgfsetdash{}{0pt}%
\pgfsys@defobject{currentmarker}{\pgfqpoint{0.000000in}{-0.048611in}}{\pgfqpoint{0.000000in}{0.000000in}}{%
\pgfpathmoveto{\pgfqpoint{0.000000in}{0.000000in}}%
\pgfpathlineto{\pgfqpoint{0.000000in}{-0.048611in}}%
\pgfusepath{stroke,fill}%
}%
\begin{pgfscope}%
\pgfsys@transformshift{2.767206in}{0.538014in}%
\pgfsys@useobject{currentmarker}{}%
\end{pgfscope}%
\end{pgfscope}%
\begin{pgfscope}%
\definecolor{textcolor}{rgb}{0.000000,0.000000,0.000000}%
\pgfsetstrokecolor{textcolor}%
\pgfsetfillcolor{textcolor}%
\pgftext[x=2.767206in,y=0.440792in,,top]{\color{textcolor}\rmfamily\fontsize{8.000000}{9.600000}\selectfont \(\displaystyle {18{:}00}\)}%
\end{pgfscope}%
\begin{pgfscope}%
\pgfsetbuttcap%
\pgfsetroundjoin%
\definecolor{currentfill}{rgb}{0.000000,0.000000,0.000000}%
\pgfsetfillcolor{currentfill}%
\pgfsetlinewidth{0.803000pt}%
\definecolor{currentstroke}{rgb}{0.000000,0.000000,0.000000}%
\pgfsetstrokecolor{currentstroke}%
\pgfsetdash{}{0pt}%
\pgfsys@defobject{currentmarker}{\pgfqpoint{0.000000in}{-0.048611in}}{\pgfqpoint{0.000000in}{0.000000in}}{%
\pgfpathmoveto{\pgfqpoint{0.000000in}{0.000000in}}%
\pgfpathlineto{\pgfqpoint{0.000000in}{-0.048611in}}%
\pgfusepath{stroke,fill}%
}%
\begin{pgfscope}%
\pgfsys@transformshift{3.286252in}{0.538014in}%
\pgfsys@useobject{currentmarker}{}%
\end{pgfscope}%
\end{pgfscope}%
\begin{pgfscope}%
\definecolor{textcolor}{rgb}{0.000000,0.000000,0.000000}%
\pgfsetstrokecolor{textcolor}%
\pgfsetfillcolor{textcolor}%
\pgftext[x=3.286252in,y=0.440792in,,top]{\color{textcolor}\rmfamily\fontsize{8.000000}{9.600000}\selectfont \(\displaystyle {21{:}00}\)}%
\end{pgfscope}%
\begin{pgfscope}%
\pgfsetbuttcap%
\pgfsetroundjoin%
\definecolor{currentfill}{rgb}{0.000000,0.000000,0.000000}%
\pgfsetfillcolor{currentfill}%
\pgfsetlinewidth{0.803000pt}%
\definecolor{currentstroke}{rgb}{0.000000,0.000000,0.000000}%
\pgfsetstrokecolor{currentstroke}%
\pgfsetdash{}{0pt}%
\pgfsys@defobject{currentmarker}{\pgfqpoint{0.000000in}{-0.048611in}}{\pgfqpoint{0.000000in}{0.000000in}}{%
\pgfpathmoveto{\pgfqpoint{0.000000in}{0.000000in}}%
\pgfpathlineto{\pgfqpoint{0.000000in}{-0.048611in}}%
\pgfusepath{stroke,fill}%
}%
\begin{pgfscope}%
\pgfsys@transformshift{3.805298in}{0.538014in}%
\pgfsys@useobject{currentmarker}{}%
\end{pgfscope}%
\end{pgfscope}%
\begin{pgfscope}%
\definecolor{textcolor}{rgb}{0.000000,0.000000,0.000000}%
\pgfsetstrokecolor{textcolor}%
\pgfsetfillcolor{textcolor}%
\pgftext[x=3.805298in,y=0.440792in,,top]{\color{textcolor}\rmfamily\fontsize{8.000000}{9.600000}\selectfont \(\displaystyle {00{:}00}\)}%
\end{pgfscope}%
\begin{pgfscope}%
\pgfsetbuttcap%
\pgfsetroundjoin%
\definecolor{currentfill}{rgb}{0.000000,0.000000,0.000000}%
\pgfsetfillcolor{currentfill}%
\pgfsetlinewidth{0.803000pt}%
\definecolor{currentstroke}{rgb}{0.000000,0.000000,0.000000}%
\pgfsetstrokecolor{currentstroke}%
\pgfsetdash{}{0pt}%
\pgfsys@defobject{currentmarker}{\pgfqpoint{0.000000in}{-0.048611in}}{\pgfqpoint{0.000000in}{0.000000in}}{%
\pgfpathmoveto{\pgfqpoint{0.000000in}{0.000000in}}%
\pgfpathlineto{\pgfqpoint{0.000000in}{-0.048611in}}%
\pgfusepath{stroke,fill}%
}%
\begin{pgfscope}%
\pgfsys@transformshift{4.324344in}{0.538014in}%
\pgfsys@useobject{currentmarker}{}%
\end{pgfscope}%
\end{pgfscope}%
\begin{pgfscope}%
\definecolor{textcolor}{rgb}{0.000000,0.000000,0.000000}%
\pgfsetstrokecolor{textcolor}%
\pgfsetfillcolor{textcolor}%
\pgftext[x=4.324344in,y=0.440792in,,top]{\color{textcolor}\rmfamily\fontsize{8.000000}{9.600000}\selectfont \(\displaystyle {03{:}00}\)}%
\end{pgfscope}%
\begin{pgfscope}%
\pgfsetbuttcap%
\pgfsetroundjoin%
\definecolor{currentfill}{rgb}{0.000000,0.000000,0.000000}%
\pgfsetfillcolor{currentfill}%
\pgfsetlinewidth{0.803000pt}%
\definecolor{currentstroke}{rgb}{0.000000,0.000000,0.000000}%
\pgfsetstrokecolor{currentstroke}%
\pgfsetdash{}{0pt}%
\pgfsys@defobject{currentmarker}{\pgfqpoint{0.000000in}{-0.048611in}}{\pgfqpoint{0.000000in}{0.000000in}}{%
\pgfpathmoveto{\pgfqpoint{0.000000in}{0.000000in}}%
\pgfpathlineto{\pgfqpoint{0.000000in}{-0.048611in}}%
\pgfusepath{stroke,fill}%
}%
\begin{pgfscope}%
\pgfsys@transformshift{4.843390in}{0.538014in}%
\pgfsys@useobject{currentmarker}{}%
\end{pgfscope}%
\end{pgfscope}%
\begin{pgfscope}%
\definecolor{textcolor}{rgb}{0.000000,0.000000,0.000000}%
\pgfsetstrokecolor{textcolor}%
\pgfsetfillcolor{textcolor}%
\pgftext[x=4.843390in,y=0.440792in,,top]{\color{textcolor}\rmfamily\fontsize{8.000000}{9.600000}\selectfont \(\displaystyle {06{:}00}\)}%
\end{pgfscope}%
\begin{pgfscope}%
\definecolor{textcolor}{rgb}{0.000000,0.000000,0.000000}%
\pgfsetstrokecolor{textcolor}%
\pgfsetfillcolor{textcolor}%
\pgftext[x=2.767012in,y=0.286570in,,top]{\color{textcolor}\rmfamily\fontsize{10.000000}{12.000000}\selectfont Time (UTC)}%
\end{pgfscope}%
\begin{pgfscope}%
\pgfsetbuttcap%
\pgfsetroundjoin%
\definecolor{currentfill}{rgb}{0.000000,0.000000,0.000000}%
\pgfsetfillcolor{currentfill}%
\pgfsetlinewidth{0.803000pt}%
\definecolor{currentstroke}{rgb}{0.000000,0.000000,0.000000}%
\pgfsetstrokecolor{currentstroke}%
\pgfsetdash{}{0pt}%
\pgfsys@defobject{currentmarker}{\pgfqpoint{-0.048611in}{0.000000in}}{\pgfqpoint{-0.000000in}{0.000000in}}{%
\pgfpathmoveto{\pgfqpoint{-0.000000in}{0.000000in}}%
\pgfpathlineto{\pgfqpoint{-0.048611in}{0.000000in}}%
\pgfusepath{stroke,fill}%
}%
\begin{pgfscope}%
\pgfsys@transformshift{0.483776in}{0.719191in}%
\pgfsys@useobject{currentmarker}{}%
\end{pgfscope}%
\end{pgfscope}%
\begin{pgfscope}%
\definecolor{textcolor}{rgb}{0.000000,0.000000,0.000000}%
\pgfsetstrokecolor{textcolor}%
\pgfsetfillcolor{textcolor}%
\pgftext[x=0.327525in, y=0.680636in, left, base]{\color{textcolor}\rmfamily\fontsize{8.000000}{9.600000}\selectfont \(\displaystyle {0}\)}%
\end{pgfscope}%
\begin{pgfscope}%
\pgfsetbuttcap%
\pgfsetroundjoin%
\definecolor{currentfill}{rgb}{0.000000,0.000000,0.000000}%
\pgfsetfillcolor{currentfill}%
\pgfsetlinewidth{0.803000pt}%
\definecolor{currentstroke}{rgb}{0.000000,0.000000,0.000000}%
\pgfsetstrokecolor{currentstroke}%
\pgfsetdash{}{0pt}%
\pgfsys@defobject{currentmarker}{\pgfqpoint{-0.048611in}{0.000000in}}{\pgfqpoint{-0.000000in}{0.000000in}}{%
\pgfpathmoveto{\pgfqpoint{-0.000000in}{0.000000in}}%
\pgfpathlineto{\pgfqpoint{-0.048611in}{0.000000in}}%
\pgfusepath{stroke,fill}%
}%
\begin{pgfscope}%
\pgfsys@transformshift{0.483776in}{0.926366in}%
\pgfsys@useobject{currentmarker}{}%
\end{pgfscope}%
\end{pgfscope}%
\begin{pgfscope}%
\definecolor{textcolor}{rgb}{0.000000,0.000000,0.000000}%
\pgfsetstrokecolor{textcolor}%
\pgfsetfillcolor{textcolor}%
\pgftext[x=0.327525in, y=0.887811in, left, base]{\color{textcolor}\rmfamily\fontsize{8.000000}{9.600000}\selectfont \(\displaystyle {5}\)}%
\end{pgfscope}%
\begin{pgfscope}%
\definecolor{textcolor}{rgb}{0.000000,0.000000,0.000000}%
\pgfsetstrokecolor{textcolor}%
\pgfsetfillcolor{textcolor}%
\pgftext[x=0.483776in,y=1.164562in,left,base]{\color{textcolor}\rmfamily\fontsize{8.000000}{9.600000}\selectfont \(\displaystyle \times{10^{\ensuremath{-}6}}{}\)}%
\end{pgfscope}%
\begin{pgfscope}%
\pgfpathrectangle{\pgfqpoint{0.483776in}{0.538014in}}{\pgfqpoint{4.566474in}{0.584881in}}%
\pgfusepath{clip}%
\pgfsetrectcap%
\pgfsetroundjoin%
\pgfsetlinewidth{0.501875pt}%
\definecolor{currentstroke}{rgb}{0.000000,0.419608,0.643137}%
\pgfsetstrokecolor{currentstroke}%
\pgfsetstrokeopacity{0.700000}%
\pgfsetdash{}{0pt}%
\pgfpathmoveto{\pgfqpoint{0.691343in}{0.723309in}}%
\pgfpathlineto{\pgfqpoint{0.692205in}{0.730232in}}%
\pgfpathlineto{\pgfqpoint{0.693935in}{0.701294in}}%
\pgfpathlineto{\pgfqpoint{0.694800in}{0.716533in}}%
\pgfpathlineto{\pgfqpoint{0.695666in}{0.686239in}}%
\pgfpathlineto{\pgfqpoint{0.696532in}{0.687741in}}%
\pgfpathlineto{\pgfqpoint{0.698263in}{0.732136in}}%
\pgfpathlineto{\pgfqpoint{0.699128in}{0.727776in}}%
\pgfpathlineto{\pgfqpoint{0.699993in}{0.774404in}}%
\pgfpathlineto{\pgfqpoint{0.701725in}{0.645250in}}%
\pgfpathlineto{\pgfqpoint{0.703453in}{0.750814in}}%
\pgfpathlineto{\pgfqpoint{0.704319in}{0.749056in}}%
\pgfpathlineto{\pgfqpoint{0.705185in}{0.769605in}}%
\pgfpathlineto{\pgfqpoint{0.709512in}{0.701108in}}%
\pgfpathlineto{\pgfqpoint{0.711241in}{0.778473in}}%
\pgfpathlineto{\pgfqpoint{0.712105in}{0.705834in}}%
\pgfpathlineto{\pgfqpoint{0.712971in}{0.751585in}}%
\pgfpathlineto{\pgfqpoint{0.713837in}{0.688691in}}%
\pgfpathlineto{\pgfqpoint{0.714702in}{0.689609in}}%
\pgfpathlineto{\pgfqpoint{0.715567in}{0.766787in}}%
\pgfpathlineto{\pgfqpoint{0.717297in}{0.689609in}}%
\pgfpathlineto{\pgfqpoint{0.718163in}{0.747850in}}%
\pgfpathlineto{\pgfqpoint{0.719030in}{0.739681in}}%
\pgfpathlineto{\pgfqpoint{0.719895in}{0.652319in}}%
\pgfpathlineto{\pgfqpoint{0.720762in}{0.737703in}}%
\pgfpathlineto{\pgfqpoint{0.721627in}{0.672612in}}%
\pgfpathlineto{\pgfqpoint{0.723356in}{0.735287in}}%
\pgfpathlineto{\pgfqpoint{0.725084in}{0.736384in}}%
\pgfpathlineto{\pgfqpoint{0.725947in}{0.694920in}}%
\pgfpathlineto{\pgfqpoint{0.727676in}{0.751951in}}%
\pgfpathlineto{\pgfqpoint{0.728541in}{0.646309in}}%
\pgfpathlineto{\pgfqpoint{0.729407in}{0.755760in}}%
\pgfpathlineto{\pgfqpoint{0.730271in}{0.732867in}}%
\pgfpathlineto{\pgfqpoint{0.731135in}{0.762134in}}%
\pgfpathlineto{\pgfqpoint{0.732865in}{0.715030in}}%
\pgfpathlineto{\pgfqpoint{0.733731in}{0.722026in}}%
\pgfpathlineto{\pgfqpoint{0.735460in}{0.682540in}}%
\pgfpathlineto{\pgfqpoint{0.736325in}{0.782501in}}%
\pgfpathlineto{\pgfqpoint{0.737190in}{0.721036in}}%
\pgfpathlineto{\pgfqpoint{0.738920in}{0.778546in}}%
\pgfpathlineto{\pgfqpoint{0.740649in}{0.685873in}}%
\pgfpathlineto{\pgfqpoint{0.741516in}{0.685906in}}%
\pgfpathlineto{\pgfqpoint{0.742381in}{0.661365in}}%
\pgfpathlineto{\pgfqpoint{0.744112in}{0.728434in}}%
\pgfpathlineto{\pgfqpoint{0.744977in}{0.676640in}}%
\pgfpathlineto{\pgfqpoint{0.745840in}{0.733452in}}%
\pgfpathlineto{\pgfqpoint{0.746705in}{0.708985in}}%
\pgfpathlineto{\pgfqpoint{0.748435in}{0.780925in}}%
\pgfpathlineto{\pgfqpoint{0.750163in}{0.711949in}}%
\pgfpathlineto{\pgfqpoint{0.751028in}{0.710739in}}%
\pgfpathlineto{\pgfqpoint{0.751891in}{0.701656in}}%
\pgfpathlineto{\pgfqpoint{0.752757in}{0.767518in}}%
\pgfpathlineto{\pgfqpoint{0.753622in}{0.661804in}}%
\pgfpathlineto{\pgfqpoint{0.754488in}{0.702391in}}%
\pgfpathlineto{\pgfqpoint{0.755354in}{0.695577in}}%
\pgfpathlineto{\pgfqpoint{0.756219in}{0.680376in}}%
\pgfpathlineto{\pgfqpoint{0.757950in}{0.793269in}}%
\pgfpathlineto{\pgfqpoint{0.758816in}{0.693965in}}%
\pgfpathlineto{\pgfqpoint{0.759679in}{0.695870in}}%
\pgfpathlineto{\pgfqpoint{0.760542in}{0.741402in}}%
\pgfpathlineto{\pgfqpoint{0.761407in}{0.661584in}}%
\pgfpathlineto{\pgfqpoint{0.763135in}{0.730374in}}%
\pgfpathlineto{\pgfqpoint{0.764000in}{0.670744in}}%
\pgfpathlineto{\pgfqpoint{0.764865in}{0.676494in}}%
\pgfpathlineto{\pgfqpoint{0.765730in}{0.703454in}}%
\pgfpathlineto{\pgfqpoint{0.766596in}{0.664150in}}%
\pgfpathlineto{\pgfqpoint{0.768326in}{0.763892in}}%
\pgfpathlineto{\pgfqpoint{0.770057in}{0.681366in}}%
\pgfpathlineto{\pgfqpoint{0.770922in}{0.749900in}}%
\pgfpathlineto{\pgfqpoint{0.771788in}{0.703674in}}%
\pgfpathlineto{\pgfqpoint{0.772652in}{0.790597in}}%
\pgfpathlineto{\pgfqpoint{0.774383in}{0.694554in}}%
\pgfpathlineto{\pgfqpoint{0.776115in}{0.752576in}}%
\pgfpathlineto{\pgfqpoint{0.776979in}{0.711185in}}%
\pgfpathlineto{\pgfqpoint{0.777845in}{0.739279in}}%
\pgfpathlineto{\pgfqpoint{0.778710in}{0.680708in}}%
\pgfpathlineto{\pgfqpoint{0.779575in}{0.777190in}}%
\pgfpathlineto{\pgfqpoint{0.780439in}{0.774884in}}%
\pgfpathlineto{\pgfqpoint{0.781305in}{0.700856in}}%
\pgfpathlineto{\pgfqpoint{0.782170in}{0.760892in}}%
\pgfpathlineto{\pgfqpoint{0.783036in}{0.713601in}}%
\pgfpathlineto{\pgfqpoint{0.783901in}{0.722099in}}%
\pgfpathlineto{\pgfqpoint{0.785630in}{0.682576in}}%
\pgfpathlineto{\pgfqpoint{0.786494in}{0.747484in}}%
\pgfpathlineto{\pgfqpoint{0.787360in}{0.605032in}}%
\pgfpathlineto{\pgfqpoint{0.788225in}{0.615727in}}%
\pgfpathlineto{\pgfqpoint{0.789089in}{0.747338in}}%
\pgfpathlineto{\pgfqpoint{0.789954in}{0.655104in}}%
\pgfpathlineto{\pgfqpoint{0.791680in}{0.787081in}}%
\pgfpathlineto{\pgfqpoint{0.793412in}{0.719022in}}%
\pgfpathlineto{\pgfqpoint{0.794274in}{0.721255in}}%
\pgfpathlineto{\pgfqpoint{0.795138in}{0.700961in}}%
\pgfpathlineto{\pgfqpoint{0.796004in}{0.820449in}}%
\pgfpathlineto{\pgfqpoint{0.796868in}{0.703637in}}%
\pgfpathlineto{\pgfqpoint{0.797734in}{0.729497in}}%
\pgfpathlineto{\pgfqpoint{0.798599in}{0.747152in}}%
\pgfpathlineto{\pgfqpoint{0.799464in}{0.743343in}}%
\pgfpathlineto{\pgfqpoint{0.800327in}{0.693856in}}%
\pgfpathlineto{\pgfqpoint{0.801192in}{0.755321in}}%
\pgfpathlineto{\pgfqpoint{0.802924in}{0.712022in}}%
\pgfpathlineto{\pgfqpoint{0.803787in}{0.714880in}}%
\pgfpathlineto{\pgfqpoint{0.804652in}{0.653196in}}%
\pgfpathlineto{\pgfqpoint{0.805515in}{0.665028in}}%
\pgfpathlineto{\pgfqpoint{0.808107in}{0.727483in}}%
\pgfpathlineto{\pgfqpoint{0.809837in}{0.608435in}}%
\pgfpathlineto{\pgfqpoint{0.810702in}{0.681439in}}%
\pgfpathlineto{\pgfqpoint{0.811567in}{0.658803in}}%
\pgfpathlineto{\pgfqpoint{0.812431in}{0.718510in}}%
\pgfpathlineto{\pgfqpoint{0.813297in}{0.682576in}}%
\pgfpathlineto{\pgfqpoint{0.814161in}{0.740671in}}%
\pgfpathlineto{\pgfqpoint{0.815026in}{0.689682in}}%
\pgfpathlineto{\pgfqpoint{0.817620in}{0.783272in}}%
\pgfpathlineto{\pgfqpoint{0.819349in}{0.683786in}}%
\pgfpathlineto{\pgfqpoint{0.820213in}{0.773381in}}%
\pgfpathlineto{\pgfqpoint{0.821078in}{0.738913in}}%
\pgfpathlineto{\pgfqpoint{0.821942in}{0.767559in}}%
\pgfpathlineto{\pgfqpoint{0.822807in}{0.760599in}}%
\pgfpathlineto{\pgfqpoint{0.823671in}{0.723821in}}%
\pgfpathlineto{\pgfqpoint{0.824536in}{0.777263in}}%
\pgfpathlineto{\pgfqpoint{0.826268in}{0.719095in}}%
\pgfpathlineto{\pgfqpoint{0.827133in}{0.746859in}}%
\pgfpathlineto{\pgfqpoint{0.828864in}{0.724990in}}%
\pgfpathlineto{\pgfqpoint{0.830596in}{0.855140in}}%
\pgfpathlineto{\pgfqpoint{0.831462in}{0.760266in}}%
\pgfpathlineto{\pgfqpoint{0.832328in}{0.803014in}}%
\pgfpathlineto{\pgfqpoint{0.833193in}{0.746494in}}%
\pgfpathlineto{\pgfqpoint{0.834058in}{0.759316in}}%
\pgfpathlineto{\pgfqpoint{0.834922in}{0.777336in}}%
\pgfpathlineto{\pgfqpoint{0.836650in}{0.714076in}}%
\pgfpathlineto{\pgfqpoint{0.838382in}{0.684590in}}%
\pgfpathlineto{\pgfqpoint{0.840114in}{0.723529in}}%
\pgfpathlineto{\pgfqpoint{0.840980in}{0.720378in}}%
\pgfpathlineto{\pgfqpoint{0.841845in}{0.730927in}}%
\pgfpathlineto{\pgfqpoint{0.843575in}{0.677265in}}%
\pgfpathlineto{\pgfqpoint{0.845305in}{0.697705in}}%
\pgfpathlineto{\pgfqpoint{0.846169in}{0.675836in}}%
\pgfpathlineto{\pgfqpoint{0.847897in}{0.724406in}}%
\pgfpathlineto{\pgfqpoint{0.848762in}{0.724406in}}%
\pgfpathlineto{\pgfqpoint{0.849627in}{0.695139in}}%
\pgfpathlineto{\pgfqpoint{0.851354in}{0.746421in}}%
\pgfpathlineto{\pgfqpoint{0.852218in}{0.710194in}}%
\pgfpathlineto{\pgfqpoint{0.853081in}{0.734077in}}%
\pgfpathlineto{\pgfqpoint{0.853945in}{0.705359in}}%
\pgfpathlineto{\pgfqpoint{0.854810in}{0.759316in}}%
\pgfpathlineto{\pgfqpoint{0.855672in}{0.754736in}}%
\pgfpathlineto{\pgfqpoint{0.856537in}{0.740854in}}%
\pgfpathlineto{\pgfqpoint{0.857402in}{0.761549in}}%
\pgfpathlineto{\pgfqpoint{0.858267in}{0.749279in}}%
\pgfpathlineto{\pgfqpoint{0.859131in}{0.670379in}}%
\pgfpathlineto{\pgfqpoint{0.861730in}{0.738292in}}%
\pgfpathlineto{\pgfqpoint{0.862597in}{0.751147in}}%
\pgfpathlineto{\pgfqpoint{0.864329in}{0.720012in}}%
\pgfpathlineto{\pgfqpoint{0.865195in}{0.678694in}}%
\pgfpathlineto{\pgfqpoint{0.866062in}{0.750745in}}%
\pgfpathlineto{\pgfqpoint{0.866929in}{0.734849in}}%
\pgfpathlineto{\pgfqpoint{0.867794in}{0.774737in}}%
\pgfpathlineto{\pgfqpoint{0.868659in}{0.702175in}}%
\pgfpathlineto{\pgfqpoint{0.870388in}{0.747850in}}%
\pgfpathlineto{\pgfqpoint{0.871253in}{0.663200in}}%
\pgfpathlineto{\pgfqpoint{0.872119in}{0.745178in}}%
\pgfpathlineto{\pgfqpoint{0.872983in}{0.674740in}}%
\pgfpathlineto{\pgfqpoint{0.873848in}{0.766901in}}%
\pgfpathlineto{\pgfqpoint{0.874712in}{0.721368in}}%
\pgfpathlineto{\pgfqpoint{0.875577in}{0.750270in}}%
\pgfpathlineto{\pgfqpoint{0.877309in}{0.707084in}}%
\pgfpathlineto{\pgfqpoint{0.878174in}{0.699024in}}%
\pgfpathlineto{\pgfqpoint{0.879035in}{0.658255in}}%
\pgfpathlineto{\pgfqpoint{0.879903in}{0.667853in}}%
\pgfpathlineto{\pgfqpoint{0.881633in}{0.731990in}}%
\pgfpathlineto{\pgfqpoint{0.882498in}{0.705911in}}%
\pgfpathlineto{\pgfqpoint{0.883364in}{0.746680in}}%
\pgfpathlineto{\pgfqpoint{0.884229in}{0.721661in}}%
\pgfpathlineto{\pgfqpoint{0.885096in}{0.727154in}}%
\pgfpathlineto{\pgfqpoint{0.885960in}{0.755508in}}%
\pgfpathlineto{\pgfqpoint{0.886826in}{0.653675in}}%
\pgfpathlineto{\pgfqpoint{0.888555in}{0.733566in}}%
\pgfpathlineto{\pgfqpoint{0.891148in}{0.786350in}}%
\pgfpathlineto{\pgfqpoint{0.892875in}{0.743530in}}%
\pgfpathlineto{\pgfqpoint{0.893741in}{0.754850in}}%
\pgfpathlineto{\pgfqpoint{0.894605in}{0.728588in}}%
\pgfpathlineto{\pgfqpoint{0.896337in}{0.756019in}}%
\pgfpathlineto{\pgfqpoint{0.898067in}{0.704372in}}%
\pgfpathlineto{\pgfqpoint{0.901525in}{0.744041in}}%
\pgfpathlineto{\pgfqpoint{0.903254in}{0.711697in}}%
\pgfpathlineto{\pgfqpoint{0.904119in}{0.713528in}}%
\pgfpathlineto{\pgfqpoint{0.904984in}{0.687010in}}%
\pgfpathlineto{\pgfqpoint{0.906714in}{0.753238in}}%
\pgfpathlineto{\pgfqpoint{0.907579in}{0.773308in}}%
\pgfpathlineto{\pgfqpoint{0.909309in}{0.708107in}}%
\pgfpathlineto{\pgfqpoint{0.910175in}{0.764700in}}%
\pgfpathlineto{\pgfqpoint{0.911039in}{0.744114in}}%
\pgfpathlineto{\pgfqpoint{0.911905in}{0.653382in}}%
\pgfpathlineto{\pgfqpoint{0.912770in}{0.718364in}}%
\pgfpathlineto{\pgfqpoint{0.913635in}{0.653127in}}%
\pgfpathlineto{\pgfqpoint{0.914501in}{0.730049in}}%
\pgfpathlineto{\pgfqpoint{0.915365in}{0.654446in}}%
\pgfpathlineto{\pgfqpoint{0.917094in}{0.728620in}}%
\pgfpathlineto{\pgfqpoint{0.917959in}{0.706386in}}%
\pgfpathlineto{\pgfqpoint{0.918824in}{0.650232in}}%
\pgfpathlineto{\pgfqpoint{0.919687in}{0.701806in}}%
\pgfpathlineto{\pgfqpoint{0.920552in}{0.664406in}}%
\pgfpathlineto{\pgfqpoint{0.922280in}{0.757229in}}%
\pgfpathlineto{\pgfqpoint{0.924010in}{0.651295in}}%
\pgfpathlineto{\pgfqpoint{0.924875in}{0.746201in}}%
\pgfpathlineto{\pgfqpoint{0.925740in}{0.730415in}}%
\pgfpathlineto{\pgfqpoint{0.926603in}{0.781916in}}%
\pgfpathlineto{\pgfqpoint{0.927464in}{0.683161in}}%
\pgfpathlineto{\pgfqpoint{0.928329in}{0.696860in}}%
\pgfpathlineto{\pgfqpoint{0.929193in}{0.704591in}}%
\pgfpathlineto{\pgfqpoint{0.930058in}{0.660086in}}%
\pgfpathlineto{\pgfqpoint{0.931787in}{0.724665in}}%
\pgfpathlineto{\pgfqpoint{0.933516in}{0.679612in}}%
\pgfpathlineto{\pgfqpoint{0.934381in}{0.716423in}}%
\pgfpathlineto{\pgfqpoint{0.935246in}{0.702066in}}%
\pgfpathlineto{\pgfqpoint{0.936974in}{0.782907in}}%
\pgfpathlineto{\pgfqpoint{0.939570in}{0.666607in}}%
\pgfpathlineto{\pgfqpoint{0.940435in}{0.745657in}}%
\pgfpathlineto{\pgfqpoint{0.942166in}{0.707194in}}%
\pgfpathlineto{\pgfqpoint{0.943032in}{0.768988in}}%
\pgfpathlineto{\pgfqpoint{0.943897in}{0.750416in}}%
\pgfpathlineto{\pgfqpoint{0.944763in}{0.775911in}}%
\pgfpathlineto{\pgfqpoint{0.945628in}{0.719428in}}%
\pgfpathlineto{\pgfqpoint{0.946490in}{0.730488in}}%
\pgfpathlineto{\pgfqpoint{0.947355in}{0.733712in}}%
\pgfpathlineto{\pgfqpoint{0.948218in}{0.769865in}}%
\pgfpathlineto{\pgfqpoint{0.949946in}{0.658547in}}%
\pgfpathlineto{\pgfqpoint{0.950811in}{0.714117in}}%
\pgfpathlineto{\pgfqpoint{0.951674in}{0.637194in}}%
\pgfpathlineto{\pgfqpoint{0.952539in}{0.719135in}}%
\pgfpathlineto{\pgfqpoint{0.953405in}{0.686425in}}%
\pgfpathlineto{\pgfqpoint{0.954269in}{0.714738in}}%
\pgfpathlineto{\pgfqpoint{0.955133in}{0.672100in}}%
\pgfpathlineto{\pgfqpoint{0.955998in}{0.761403in}}%
\pgfpathlineto{\pgfqpoint{0.956863in}{0.704372in}}%
\pgfpathlineto{\pgfqpoint{0.957728in}{0.761111in}}%
\pgfpathlineto{\pgfqpoint{0.958594in}{0.675507in}}%
\pgfpathlineto{\pgfqpoint{0.959459in}{0.765212in}}%
\pgfpathlineto{\pgfqpoint{0.960325in}{0.701587in}}%
\pgfpathlineto{\pgfqpoint{0.961191in}{0.744187in}}%
\pgfpathlineto{\pgfqpoint{0.962055in}{0.679572in}}%
\pgfpathlineto{\pgfqpoint{0.962920in}{0.699938in}}%
\pgfpathlineto{\pgfqpoint{0.963783in}{0.766568in}}%
\pgfpathlineto{\pgfqpoint{0.964647in}{0.713345in}}%
\pgfpathlineto{\pgfqpoint{0.965512in}{0.720012in}}%
\pgfpathlineto{\pgfqpoint{0.968105in}{0.654186in}}%
\pgfpathlineto{\pgfqpoint{0.970701in}{0.725798in}}%
\pgfpathlineto{\pgfqpoint{0.972432in}{0.677558in}}%
\pgfpathlineto{\pgfqpoint{0.973297in}{0.720268in}}%
\pgfpathlineto{\pgfqpoint{0.974162in}{0.684517in}}%
\pgfpathlineto{\pgfqpoint{0.975027in}{0.759316in}}%
\pgfpathlineto{\pgfqpoint{0.975889in}{0.702943in}}%
\pgfpathlineto{\pgfqpoint{0.976751in}{0.739388in}}%
\pgfpathlineto{\pgfqpoint{0.978481in}{0.704884in}}%
\pgfpathlineto{\pgfqpoint{0.979346in}{0.715067in}}%
\pgfpathlineto{\pgfqpoint{0.981076in}{0.809242in}}%
\pgfpathlineto{\pgfqpoint{0.981941in}{0.768143in}}%
\pgfpathlineto{\pgfqpoint{0.984535in}{0.715213in}}%
\pgfpathlineto{\pgfqpoint{0.985400in}{0.741037in}}%
\pgfpathlineto{\pgfqpoint{0.986264in}{0.740891in}}%
\pgfpathlineto{\pgfqpoint{0.987991in}{0.701367in}}%
\pgfpathlineto{\pgfqpoint{0.988856in}{0.761111in}}%
\pgfpathlineto{\pgfqpoint{0.989722in}{0.733821in}}%
\pgfpathlineto{\pgfqpoint{0.990585in}{0.663127in}}%
\pgfpathlineto{\pgfqpoint{0.991449in}{0.756019in}}%
\pgfpathlineto{\pgfqpoint{0.992314in}{0.699280in}}%
\pgfpathlineto{\pgfqpoint{0.993178in}{0.737740in}}%
\pgfpathlineto{\pgfqpoint{0.994043in}{0.669023in}}%
\pgfpathlineto{\pgfqpoint{0.996636in}{0.735251in}}%
\pgfpathlineto{\pgfqpoint{0.997501in}{0.713053in}}%
\pgfpathlineto{\pgfqpoint{1.000961in}{0.750562in}}%
\pgfpathlineto{\pgfqpoint{1.001826in}{0.732210in}}%
\pgfpathlineto{\pgfqpoint{1.002691in}{0.752503in}}%
\pgfpathlineto{\pgfqpoint{1.003553in}{0.714628in}}%
\pgfpathlineto{\pgfqpoint{1.004419in}{0.783491in}}%
\pgfpathlineto{\pgfqpoint{1.005284in}{0.710121in}}%
\pgfpathlineto{\pgfqpoint{1.006149in}{0.733529in}}%
\pgfpathlineto{\pgfqpoint{1.007014in}{0.695399in}}%
\pgfpathlineto{\pgfqpoint{1.007877in}{0.762686in}}%
\pgfpathlineto{\pgfqpoint{1.008742in}{0.689576in}}%
\pgfpathlineto{\pgfqpoint{1.009607in}{0.748548in}}%
\pgfpathlineto{\pgfqpoint{1.011338in}{0.687083in}}%
\pgfpathlineto{\pgfqpoint{1.012203in}{0.694042in}}%
\pgfpathlineto{\pgfqpoint{1.013067in}{0.735580in}}%
\pgfpathlineto{\pgfqpoint{1.013933in}{0.722099in}}%
\pgfpathlineto{\pgfqpoint{1.014798in}{0.685029in}}%
\pgfpathlineto{\pgfqpoint{1.015662in}{0.746421in}}%
\pgfpathlineto{\pgfqpoint{1.016526in}{0.731146in}}%
\pgfpathlineto{\pgfqpoint{1.017391in}{0.740891in}}%
\pgfpathlineto{\pgfqpoint{1.020851in}{0.668182in}}%
\pgfpathlineto{\pgfqpoint{1.022581in}{0.687595in}}%
\pgfpathlineto{\pgfqpoint{1.023445in}{0.749133in}}%
\pgfpathlineto{\pgfqpoint{1.026037in}{0.674407in}}%
\pgfpathlineto{\pgfqpoint{1.027767in}{0.768070in}}%
\pgfpathlineto{\pgfqpoint{1.029494in}{0.734516in}}%
\pgfpathlineto{\pgfqpoint{1.030357in}{0.725798in}}%
\pgfpathlineto{\pgfqpoint{1.031221in}{0.743928in}}%
\pgfpathlineto{\pgfqpoint{1.034679in}{0.669059in}}%
\pgfpathlineto{\pgfqpoint{1.035545in}{0.706897in}}%
\pgfpathlineto{\pgfqpoint{1.036409in}{0.705395in}}%
\pgfpathlineto{\pgfqpoint{1.037275in}{0.706934in}}%
\pgfpathlineto{\pgfqpoint{1.038139in}{0.687302in}}%
\pgfpathlineto{\pgfqpoint{1.040734in}{0.746973in}}%
\pgfpathlineto{\pgfqpoint{1.041600in}{0.711916in}}%
\pgfpathlineto{\pgfqpoint{1.042466in}{0.772025in}}%
\pgfpathlineto{\pgfqpoint{1.043331in}{0.735507in}}%
\pgfpathlineto{\pgfqpoint{1.044195in}{0.803200in}}%
\pgfpathlineto{\pgfqpoint{1.045927in}{0.680270in}}%
\pgfpathlineto{\pgfqpoint{1.046792in}{0.684188in}}%
\pgfpathlineto{\pgfqpoint{1.047657in}{0.731771in}}%
\pgfpathlineto{\pgfqpoint{1.049390in}{0.692759in}}%
\pgfpathlineto{\pgfqpoint{1.050256in}{0.744187in}}%
\pgfpathlineto{\pgfqpoint{1.051987in}{0.642980in}}%
\pgfpathlineto{\pgfqpoint{1.052852in}{0.681260in}}%
\pgfpathlineto{\pgfqpoint{1.053719in}{0.660565in}}%
\pgfpathlineto{\pgfqpoint{1.056313in}{0.771294in}}%
\pgfpathlineto{\pgfqpoint{1.057179in}{0.766422in}}%
\pgfpathlineto{\pgfqpoint{1.058042in}{0.690161in}}%
\pgfpathlineto{\pgfqpoint{1.058907in}{0.750562in}}%
\pgfpathlineto{\pgfqpoint{1.059773in}{0.680781in}}%
\pgfpathlineto{\pgfqpoint{1.060637in}{0.712614in}}%
\pgfpathlineto{\pgfqpoint{1.062369in}{0.669721in}}%
\pgfpathlineto{\pgfqpoint{1.063234in}{0.676936in}}%
\pgfpathlineto{\pgfqpoint{1.064962in}{0.771587in}}%
\pgfpathlineto{\pgfqpoint{1.065827in}{0.659830in}}%
\pgfpathlineto{\pgfqpoint{1.066692in}{0.682138in}}%
\pgfpathlineto{\pgfqpoint{1.067557in}{0.745105in}}%
\pgfpathlineto{\pgfqpoint{1.068422in}{0.636240in}}%
\pgfpathlineto{\pgfqpoint{1.070151in}{0.752763in}}%
\pgfpathlineto{\pgfqpoint{1.073608in}{0.646975in}}%
\pgfpathlineto{\pgfqpoint{1.074470in}{0.784994in}}%
\pgfpathlineto{\pgfqpoint{1.075335in}{0.659867in}}%
\pgfpathlineto{\pgfqpoint{1.076199in}{0.704664in}}%
\pgfpathlineto{\pgfqpoint{1.077064in}{0.680197in}}%
\pgfpathlineto{\pgfqpoint{1.079657in}{0.732136in}}%
\pgfpathlineto{\pgfqpoint{1.081384in}{0.673018in}}%
\pgfpathlineto{\pgfqpoint{1.083981in}{0.753859in}}%
\pgfpathlineto{\pgfqpoint{1.085711in}{0.664045in}}%
\pgfpathlineto{\pgfqpoint{1.087440in}{0.773970in}}%
\pgfpathlineto{\pgfqpoint{1.088306in}{0.790857in}}%
\pgfpathlineto{\pgfqpoint{1.089172in}{0.718770in}}%
\pgfpathlineto{\pgfqpoint{1.090037in}{0.751553in}}%
\pgfpathlineto{\pgfqpoint{1.091767in}{0.683457in}}%
\pgfpathlineto{\pgfqpoint{1.095225in}{0.738990in}}%
\pgfpathlineto{\pgfqpoint{1.096956in}{0.696426in}}%
\pgfpathlineto{\pgfqpoint{1.097822in}{0.734556in}}%
\pgfpathlineto{\pgfqpoint{1.098687in}{0.723163in}}%
\pgfpathlineto{\pgfqpoint{1.100416in}{0.743384in}}%
\pgfpathlineto{\pgfqpoint{1.101281in}{0.768659in}}%
\pgfpathlineto{\pgfqpoint{1.102145in}{0.675032in}}%
\pgfpathlineto{\pgfqpoint{1.103008in}{0.687156in}}%
\pgfpathlineto{\pgfqpoint{1.104736in}{0.725656in}}%
\pgfpathlineto{\pgfqpoint{1.105601in}{0.702066in}}%
\pgfpathlineto{\pgfqpoint{1.106466in}{0.712834in}}%
\pgfpathlineto{\pgfqpoint{1.108192in}{0.692394in}}%
\pgfpathlineto{\pgfqpoint{1.109055in}{0.796387in}}%
\pgfpathlineto{\pgfqpoint{1.110782in}{0.715363in}}%
\pgfpathlineto{\pgfqpoint{1.112512in}{0.694116in}}%
\pgfpathlineto{\pgfqpoint{1.114239in}{0.690599in}}%
\pgfpathlineto{\pgfqpoint{1.115102in}{0.721368in}}%
\pgfpathlineto{\pgfqpoint{1.115967in}{0.717008in}}%
\pgfpathlineto{\pgfqpoint{1.116831in}{0.740964in}}%
\pgfpathlineto{\pgfqpoint{1.117696in}{0.729497in}}%
\pgfpathlineto{\pgfqpoint{1.118561in}{0.642980in}}%
\pgfpathlineto{\pgfqpoint{1.119426in}{0.672685in}}%
\pgfpathlineto{\pgfqpoint{1.120289in}{0.645067in}}%
\pgfpathlineto{\pgfqpoint{1.122016in}{0.727337in}}%
\pgfpathlineto{\pgfqpoint{1.122881in}{0.703454in}}%
\pgfpathlineto{\pgfqpoint{1.124612in}{0.771806in}}%
\pgfpathlineto{\pgfqpoint{1.125476in}{0.692979in}}%
\pgfpathlineto{\pgfqpoint{1.126342in}{0.761330in}}%
\pgfpathlineto{\pgfqpoint{1.128072in}{0.700198in}}%
\pgfpathlineto{\pgfqpoint{1.128936in}{0.725396in}}%
\pgfpathlineto{\pgfqpoint{1.130665in}{0.657231in}}%
\pgfpathlineto{\pgfqpoint{1.132396in}{0.717816in}}%
\pgfpathlineto{\pgfqpoint{1.133261in}{0.681187in}}%
\pgfpathlineto{\pgfqpoint{1.134991in}{0.716058in}}%
\pgfpathlineto{\pgfqpoint{1.135856in}{0.729976in}}%
\pgfpathlineto{\pgfqpoint{1.136722in}{0.672466in}}%
\pgfpathlineto{\pgfqpoint{1.137586in}{0.683640in}}%
\pgfpathlineto{\pgfqpoint{1.140181in}{0.745544in}}%
\pgfpathlineto{\pgfqpoint{1.141044in}{0.741296in}}%
\pgfpathlineto{\pgfqpoint{1.141910in}{0.729794in}}%
\pgfpathlineto{\pgfqpoint{1.143641in}{0.656606in}}%
\pgfpathlineto{\pgfqpoint{1.144506in}{0.711112in}}%
\pgfpathlineto{\pgfqpoint{1.145371in}{0.676607in}}%
\pgfpathlineto{\pgfqpoint{1.146235in}{0.767120in}}%
\pgfpathlineto{\pgfqpoint{1.147100in}{0.765983in}}%
\pgfpathlineto{\pgfqpoint{1.148829in}{0.697778in}}%
\pgfpathlineto{\pgfqpoint{1.149692in}{0.749133in}}%
\pgfpathlineto{\pgfqpoint{1.151424in}{0.709135in}}%
\pgfpathlineto{\pgfqpoint{1.153155in}{0.748694in}}%
\pgfpathlineto{\pgfqpoint{1.154020in}{0.690892in}}%
\pgfpathlineto{\pgfqpoint{1.154885in}{0.743895in}}%
\pgfpathlineto{\pgfqpoint{1.155749in}{0.740817in}}%
\pgfpathlineto{\pgfqpoint{1.156614in}{0.771806in}}%
\pgfpathlineto{\pgfqpoint{1.159208in}{0.574742in}}%
\pgfpathlineto{\pgfqpoint{1.160937in}{0.763567in}}%
\pgfpathlineto{\pgfqpoint{1.162667in}{0.719135in}}%
\pgfpathlineto{\pgfqpoint{1.163531in}{0.779098in}}%
\pgfpathlineto{\pgfqpoint{1.165259in}{0.705801in}}%
\pgfpathlineto{\pgfqpoint{1.166990in}{0.769426in}}%
\pgfpathlineto{\pgfqpoint{1.167854in}{0.744849in}}%
\pgfpathlineto{\pgfqpoint{1.168720in}{0.741808in}}%
\pgfpathlineto{\pgfqpoint{1.169584in}{0.682430in}}%
\pgfpathlineto{\pgfqpoint{1.171315in}{0.720784in}}%
\pgfpathlineto{\pgfqpoint{1.172180in}{0.700417in}}%
\pgfpathlineto{\pgfqpoint{1.173912in}{0.728401in}}%
\pgfpathlineto{\pgfqpoint{1.174778in}{0.726972in}}%
\pgfpathlineto{\pgfqpoint{1.176508in}{0.668657in}}%
\pgfpathlineto{\pgfqpoint{1.178239in}{0.722026in}}%
\pgfpathlineto{\pgfqpoint{1.179103in}{0.707742in}}%
\pgfpathlineto{\pgfqpoint{1.181699in}{0.761330in}}%
\pgfpathlineto{\pgfqpoint{1.182565in}{0.723894in}}%
\pgfpathlineto{\pgfqpoint{1.183430in}{0.790378in}}%
\pgfpathlineto{\pgfqpoint{1.185160in}{0.677704in}}%
\pgfpathlineto{\pgfqpoint{1.186891in}{0.733968in}}%
\pgfpathlineto{\pgfqpoint{1.188623in}{0.779058in}}%
\pgfpathlineto{\pgfqpoint{1.189487in}{0.690197in}}%
\pgfpathlineto{\pgfqpoint{1.190352in}{0.744041in}}%
\pgfpathlineto{\pgfqpoint{1.191218in}{0.659465in}}%
\pgfpathlineto{\pgfqpoint{1.192083in}{0.837668in}}%
\pgfpathlineto{\pgfqpoint{1.192948in}{0.716277in}}%
\pgfpathlineto{\pgfqpoint{1.193813in}{0.736164in}}%
\pgfpathlineto{\pgfqpoint{1.194678in}{0.747484in}}%
\pgfpathlineto{\pgfqpoint{1.195541in}{0.708692in}}%
\pgfpathlineto{\pgfqpoint{1.196406in}{0.730378in}}%
\pgfpathlineto{\pgfqpoint{1.198137in}{0.703089in}}%
\pgfpathlineto{\pgfqpoint{1.199001in}{0.763088in}}%
\pgfpathlineto{\pgfqpoint{1.199865in}{0.700304in}}%
\pgfpathlineto{\pgfqpoint{1.200729in}{0.746348in}}%
\pgfpathlineto{\pgfqpoint{1.201595in}{0.691586in}}%
\pgfpathlineto{\pgfqpoint{1.202459in}{0.747923in}}%
\pgfpathlineto{\pgfqpoint{1.203324in}{0.673489in}}%
\pgfpathlineto{\pgfqpoint{1.204189in}{0.702683in}}%
\pgfpathlineto{\pgfqpoint{1.205054in}{0.696089in}}%
\pgfpathlineto{\pgfqpoint{1.205919in}{0.693710in}}%
\pgfpathlineto{\pgfqpoint{1.207651in}{0.749718in}}%
\pgfpathlineto{\pgfqpoint{1.208516in}{0.715140in}}%
\pgfpathlineto{\pgfqpoint{1.209379in}{0.720232in}}%
\pgfpathlineto{\pgfqpoint{1.211111in}{0.680891in}}%
\pgfpathlineto{\pgfqpoint{1.211977in}{0.735141in}}%
\pgfpathlineto{\pgfqpoint{1.212842in}{0.730049in}}%
\pgfpathlineto{\pgfqpoint{1.213706in}{0.732136in}}%
\pgfpathlineto{\pgfqpoint{1.214572in}{0.690599in}}%
\pgfpathlineto{\pgfqpoint{1.217169in}{0.817558in}}%
\pgfpathlineto{\pgfqpoint{1.218035in}{0.682174in}}%
\pgfpathlineto{\pgfqpoint{1.218902in}{0.796826in}}%
\pgfpathlineto{\pgfqpoint{1.220634in}{0.682942in}}%
\pgfpathlineto{\pgfqpoint{1.222366in}{0.742247in}}%
\pgfpathlineto{\pgfqpoint{1.223232in}{0.735507in}}%
\pgfpathlineto{\pgfqpoint{1.224098in}{0.737520in}}%
\pgfpathlineto{\pgfqpoint{1.224965in}{0.768399in}}%
\pgfpathlineto{\pgfqpoint{1.225831in}{0.674553in}}%
\pgfpathlineto{\pgfqpoint{1.228427in}{0.727045in}}%
\pgfpathlineto{\pgfqpoint{1.230156in}{0.686165in}}%
\pgfpathlineto{\pgfqpoint{1.231020in}{0.728913in}}%
\pgfpathlineto{\pgfqpoint{1.232751in}{0.696568in}}%
\pgfpathlineto{\pgfqpoint{1.233617in}{0.701952in}}%
\pgfpathlineto{\pgfqpoint{1.234482in}{0.668913in}}%
\pgfpathlineto{\pgfqpoint{1.236211in}{0.728182in}}%
\pgfpathlineto{\pgfqpoint{1.237076in}{0.737817in}}%
\pgfpathlineto{\pgfqpoint{1.237942in}{0.678548in}}%
\pgfpathlineto{\pgfqpoint{1.238807in}{0.765910in}}%
\pgfpathlineto{\pgfqpoint{1.239671in}{0.757339in}}%
\pgfpathlineto{\pgfqpoint{1.240537in}{0.761257in}}%
\pgfpathlineto{\pgfqpoint{1.241402in}{0.677558in}}%
\pgfpathlineto{\pgfqpoint{1.242268in}{0.686019in}}%
\pgfpathlineto{\pgfqpoint{1.243999in}{0.733895in}}%
\pgfpathlineto{\pgfqpoint{1.244864in}{0.687668in}}%
\pgfpathlineto{\pgfqpoint{1.245728in}{0.748069in}}%
\pgfpathlineto{\pgfqpoint{1.246593in}{0.706678in}}%
\pgfpathlineto{\pgfqpoint{1.247458in}{0.716788in}}%
\pgfpathlineto{\pgfqpoint{1.249187in}{0.708660in}}%
\pgfpathlineto{\pgfqpoint{1.250053in}{0.695179in}}%
\pgfpathlineto{\pgfqpoint{1.250916in}{0.756864in}}%
\pgfpathlineto{\pgfqpoint{1.251781in}{0.736643in}}%
\pgfpathlineto{\pgfqpoint{1.252647in}{0.747484in}}%
\pgfpathlineto{\pgfqpoint{1.253512in}{0.704445in}}%
\pgfpathlineto{\pgfqpoint{1.254377in}{0.776971in}}%
\pgfpathlineto{\pgfqpoint{1.255243in}{0.721807in}}%
\pgfpathlineto{\pgfqpoint{1.256109in}{0.762906in}}%
\pgfpathlineto{\pgfqpoint{1.256974in}{0.742100in}}%
\pgfpathlineto{\pgfqpoint{1.257840in}{0.781112in}}%
\pgfpathlineto{\pgfqpoint{1.258706in}{0.671370in}}%
\pgfpathlineto{\pgfqpoint{1.259571in}{0.697924in}}%
\pgfpathlineto{\pgfqpoint{1.260436in}{0.721734in}}%
\pgfpathlineto{\pgfqpoint{1.261299in}{0.698988in}}%
\pgfpathlineto{\pgfqpoint{1.262165in}{0.755946in}}%
\pgfpathlineto{\pgfqpoint{1.263030in}{0.675032in}}%
\pgfpathlineto{\pgfqpoint{1.264759in}{0.711404in}}%
\pgfpathlineto{\pgfqpoint{1.265625in}{0.719464in}}%
\pgfpathlineto{\pgfqpoint{1.266490in}{0.798328in}}%
\pgfpathlineto{\pgfqpoint{1.269081in}{0.714336in}}%
\pgfpathlineto{\pgfqpoint{1.269946in}{0.713970in}}%
\pgfpathlineto{\pgfqpoint{1.270811in}{0.757229in}}%
\pgfpathlineto{\pgfqpoint{1.271677in}{0.747558in}}%
\pgfpathlineto{\pgfqpoint{1.272543in}{0.735507in}}%
\pgfpathlineto{\pgfqpoint{1.274272in}{0.647852in}}%
\pgfpathlineto{\pgfqpoint{1.276003in}{0.772244in}}%
\pgfpathlineto{\pgfqpoint{1.276869in}{0.718510in}}%
\pgfpathlineto{\pgfqpoint{1.277734in}{0.799757in}}%
\pgfpathlineto{\pgfqpoint{1.279465in}{0.745471in}}%
\pgfpathlineto{\pgfqpoint{1.281196in}{0.705874in}}%
\pgfpathlineto{\pgfqpoint{1.282062in}{0.709244in}}%
\pgfpathlineto{\pgfqpoint{1.282927in}{0.805433in}}%
\pgfpathlineto{\pgfqpoint{1.283793in}{0.702797in}}%
\pgfpathlineto{\pgfqpoint{1.284658in}{0.785802in}}%
\pgfpathlineto{\pgfqpoint{1.285521in}{0.688732in}}%
\pgfpathlineto{\pgfqpoint{1.286386in}{0.704810in}}%
\pgfpathlineto{\pgfqpoint{1.287250in}{0.708254in}}%
\pgfpathlineto{\pgfqpoint{1.288114in}{0.726606in}}%
\pgfpathlineto{\pgfqpoint{1.288980in}{0.771002in}}%
\pgfpathlineto{\pgfqpoint{1.289846in}{0.705874in}}%
\pgfpathlineto{\pgfqpoint{1.290711in}{0.804223in}}%
\pgfpathlineto{\pgfqpoint{1.291576in}{0.756202in}}%
\pgfpathlineto{\pgfqpoint{1.292440in}{0.800342in}}%
\pgfpathlineto{\pgfqpoint{1.293306in}{0.799059in}}%
\pgfpathlineto{\pgfqpoint{1.295037in}{0.714628in}}%
\pgfpathlineto{\pgfqpoint{1.295904in}{0.815836in}}%
\pgfpathlineto{\pgfqpoint{1.297633in}{0.645652in}}%
\pgfpathlineto{\pgfqpoint{1.298498in}{0.659611in}}%
\pgfpathlineto{\pgfqpoint{1.299363in}{0.623092in}}%
\pgfpathlineto{\pgfqpoint{1.300229in}{0.707815in}}%
\pgfpathlineto{\pgfqpoint{1.301092in}{0.675324in}}%
\pgfpathlineto{\pgfqpoint{1.301958in}{0.697413in}}%
\pgfpathlineto{\pgfqpoint{1.302823in}{0.697120in}}%
\pgfpathlineto{\pgfqpoint{1.304552in}{0.714263in}}%
\pgfpathlineto{\pgfqpoint{1.305417in}{0.762979in}}%
\pgfpathlineto{\pgfqpoint{1.307147in}{0.655177in}}%
\pgfpathlineto{\pgfqpoint{1.308013in}{0.760892in}}%
\pgfpathlineto{\pgfqpoint{1.308879in}{0.757302in}}%
\pgfpathlineto{\pgfqpoint{1.309745in}{0.756019in}}%
\pgfpathlineto{\pgfqpoint{1.310610in}{0.707121in}}%
\pgfpathlineto{\pgfqpoint{1.311476in}{0.765764in}}%
\pgfpathlineto{\pgfqpoint{1.312342in}{0.714409in}}%
\pgfpathlineto{\pgfqpoint{1.313208in}{0.784153in}}%
\pgfpathlineto{\pgfqpoint{1.314074in}{0.733566in}}%
\pgfpathlineto{\pgfqpoint{1.315803in}{0.792834in}}%
\pgfpathlineto{\pgfqpoint{1.318395in}{0.699134in}}%
\pgfpathlineto{\pgfqpoint{1.320124in}{0.739429in}}%
\pgfpathlineto{\pgfqpoint{1.321854in}{0.700636in}}%
\pgfpathlineto{\pgfqpoint{1.322720in}{0.723163in}}%
\pgfpathlineto{\pgfqpoint{1.323583in}{0.709025in}}%
\pgfpathlineto{\pgfqpoint{1.324448in}{0.724592in}}%
\pgfpathlineto{\pgfqpoint{1.326179in}{0.681918in}}%
\pgfpathlineto{\pgfqpoint{1.329636in}{0.757704in}}%
\pgfpathlineto{\pgfqpoint{1.330500in}{0.768842in}}%
\pgfpathlineto{\pgfqpoint{1.332230in}{0.725583in}}%
\pgfpathlineto{\pgfqpoint{1.333092in}{0.738146in}}%
\pgfpathlineto{\pgfqpoint{1.333954in}{0.735214in}}%
\pgfpathlineto{\pgfqpoint{1.334819in}{0.674995in}}%
\pgfpathlineto{\pgfqpoint{1.336550in}{0.743457in}}%
\pgfpathlineto{\pgfqpoint{1.337413in}{0.754923in}}%
\pgfpathlineto{\pgfqpoint{1.338278in}{0.709317in}}%
\pgfpathlineto{\pgfqpoint{1.340007in}{0.745251in}}%
\pgfpathlineto{\pgfqpoint{1.340869in}{0.709390in}}%
\pgfpathlineto{\pgfqpoint{1.341734in}{0.729684in}}%
\pgfpathlineto{\pgfqpoint{1.342599in}{0.692394in}}%
\pgfpathlineto{\pgfqpoint{1.343462in}{0.752868in}}%
\pgfpathlineto{\pgfqpoint{1.344329in}{0.670415in}}%
\pgfpathlineto{\pgfqpoint{1.346061in}{0.761549in}}%
\pgfpathlineto{\pgfqpoint{1.347786in}{0.699792in}}%
\pgfpathlineto{\pgfqpoint{1.348652in}{0.755654in}}%
\pgfpathlineto{\pgfqpoint{1.349515in}{0.709902in}}%
\pgfpathlineto{\pgfqpoint{1.350380in}{0.747484in}}%
\pgfpathlineto{\pgfqpoint{1.352107in}{0.668365in}}%
\pgfpathlineto{\pgfqpoint{1.352972in}{0.706240in}}%
\pgfpathlineto{\pgfqpoint{1.353837in}{0.646752in}}%
\pgfpathlineto{\pgfqpoint{1.354702in}{0.781478in}}%
\pgfpathlineto{\pgfqpoint{1.356431in}{0.717227in}}%
\pgfpathlineto{\pgfqpoint{1.357296in}{0.790524in}}%
\pgfpathlineto{\pgfqpoint{1.359028in}{0.696495in}}%
\pgfpathlineto{\pgfqpoint{1.359893in}{0.784003in}}%
\pgfpathlineto{\pgfqpoint{1.360758in}{0.655141in}}%
\pgfpathlineto{\pgfqpoint{1.362488in}{0.752211in}}%
\pgfpathlineto{\pgfqpoint{1.363353in}{0.750672in}}%
\pgfpathlineto{\pgfqpoint{1.364217in}{0.719354in}}%
\pgfpathlineto{\pgfqpoint{1.365081in}{0.794885in}}%
\pgfpathlineto{\pgfqpoint{1.365947in}{0.732762in}}%
\pgfpathlineto{\pgfqpoint{1.366813in}{0.761955in}}%
\pgfpathlineto{\pgfqpoint{1.368543in}{0.666936in}}%
\pgfpathlineto{\pgfqpoint{1.369408in}{0.677964in}}%
\pgfpathlineto{\pgfqpoint{1.370273in}{0.704810in}}%
\pgfpathlineto{\pgfqpoint{1.371138in}{0.683859in}}%
\pgfpathlineto{\pgfqpoint{1.372002in}{0.747996in}}%
\pgfpathlineto{\pgfqpoint{1.372866in}{0.696458in}}%
\pgfpathlineto{\pgfqpoint{1.373729in}{0.736424in}}%
\pgfpathlineto{\pgfqpoint{1.374594in}{0.731698in}}%
\pgfpathlineto{\pgfqpoint{1.375459in}{0.692248in}}%
\pgfpathlineto{\pgfqpoint{1.376322in}{0.697924in}}%
\pgfpathlineto{\pgfqpoint{1.377187in}{0.726570in}}%
\pgfpathlineto{\pgfqpoint{1.378916in}{0.625837in}}%
\pgfpathlineto{\pgfqpoint{1.379781in}{0.713089in}}%
\pgfpathlineto{\pgfqpoint{1.380646in}{0.681074in}}%
\pgfpathlineto{\pgfqpoint{1.381509in}{0.740817in}}%
\pgfpathlineto{\pgfqpoint{1.382373in}{0.684152in}}%
\pgfpathlineto{\pgfqpoint{1.383238in}{0.710966in}}%
\pgfpathlineto{\pgfqpoint{1.384968in}{0.682211in}}%
\pgfpathlineto{\pgfqpoint{1.385831in}{0.717008in}}%
\pgfpathlineto{\pgfqpoint{1.386696in}{0.699244in}}%
\pgfpathlineto{\pgfqpoint{1.387562in}{0.786569in}}%
\pgfpathlineto{\pgfqpoint{1.389291in}{0.703089in}}%
\pgfpathlineto{\pgfqpoint{1.390156in}{0.704884in}}%
\pgfpathlineto{\pgfqpoint{1.391021in}{0.743237in}}%
\pgfpathlineto{\pgfqpoint{1.391886in}{0.715213in}}%
\pgfpathlineto{\pgfqpoint{1.393613in}{0.731990in}}%
\pgfpathlineto{\pgfqpoint{1.395342in}{0.699426in}}%
\pgfpathlineto{\pgfqpoint{1.396208in}{0.669794in}}%
\pgfpathlineto{\pgfqpoint{1.397073in}{0.716131in}}%
\pgfpathlineto{\pgfqpoint{1.397938in}{0.664922in}}%
\pgfpathlineto{\pgfqpoint{1.399668in}{0.738438in}}%
\pgfpathlineto{\pgfqpoint{1.400534in}{0.716058in}}%
\pgfpathlineto{\pgfqpoint{1.401400in}{0.721478in}}%
\pgfpathlineto{\pgfqpoint{1.402265in}{0.776605in}}%
\pgfpathlineto{\pgfqpoint{1.403131in}{0.747119in}}%
\pgfpathlineto{\pgfqpoint{1.403997in}{0.760015in}}%
\pgfpathlineto{\pgfqpoint{1.404863in}{0.751845in}}%
\pgfpathlineto{\pgfqpoint{1.406594in}{0.716788in}}%
\pgfpathlineto{\pgfqpoint{1.408324in}{0.750635in}}%
\pgfpathlineto{\pgfqpoint{1.409190in}{0.735580in}}%
\pgfpathlineto{\pgfqpoint{1.410055in}{0.800196in}}%
\pgfpathlineto{\pgfqpoint{1.411785in}{0.757375in}}%
\pgfpathlineto{\pgfqpoint{1.412651in}{0.760965in}}%
\pgfpathlineto{\pgfqpoint{1.413515in}{0.709390in}}%
\pgfpathlineto{\pgfqpoint{1.414379in}{0.782322in}}%
\pgfpathlineto{\pgfqpoint{1.415244in}{0.766312in}}%
\pgfpathlineto{\pgfqpoint{1.416108in}{0.701075in}}%
\pgfpathlineto{\pgfqpoint{1.416972in}{0.785798in}}%
\pgfpathlineto{\pgfqpoint{1.418701in}{0.702723in}}%
\pgfpathlineto{\pgfqpoint{1.419566in}{0.747558in}}%
\pgfpathlineto{\pgfqpoint{1.420430in}{0.717596in}}%
\pgfpathlineto{\pgfqpoint{1.421294in}{0.727743in}}%
\pgfpathlineto{\pgfqpoint{1.422159in}{0.709025in}}%
\pgfpathlineto{\pgfqpoint{1.423024in}{0.618732in}}%
\pgfpathlineto{\pgfqpoint{1.423884in}{0.720638in}}%
\pgfpathlineto{\pgfqpoint{1.424749in}{0.649610in}}%
\pgfpathlineto{\pgfqpoint{1.426480in}{0.770417in}}%
\pgfpathlineto{\pgfqpoint{1.427345in}{0.746790in}}%
\pgfpathlineto{\pgfqpoint{1.428206in}{0.736936in}}%
\pgfpathlineto{\pgfqpoint{1.429937in}{0.632102in}}%
\pgfpathlineto{\pgfqpoint{1.431667in}{0.753932in}}%
\pgfpathlineto{\pgfqpoint{1.434265in}{0.685288in}}%
\pgfpathlineto{\pgfqpoint{1.435996in}{0.743164in}}%
\pgfpathlineto{\pgfqpoint{1.436861in}{0.699317in}}%
\pgfpathlineto{\pgfqpoint{1.437726in}{0.711039in}}%
\pgfpathlineto{\pgfqpoint{1.438592in}{0.734703in}}%
\pgfpathlineto{\pgfqpoint{1.439455in}{0.680124in}}%
\pgfpathlineto{\pgfqpoint{1.440320in}{0.717706in}}%
\pgfpathlineto{\pgfqpoint{1.441185in}{0.698988in}}%
\pgfpathlineto{\pgfqpoint{1.442050in}{0.707815in}}%
\pgfpathlineto{\pgfqpoint{1.443778in}{0.689353in}}%
\pgfpathlineto{\pgfqpoint{1.447237in}{0.758472in}}%
\pgfpathlineto{\pgfqpoint{1.448103in}{0.729351in}}%
\pgfpathlineto{\pgfqpoint{1.448967in}{0.658949in}}%
\pgfpathlineto{\pgfqpoint{1.449831in}{0.719095in}}%
\pgfpathlineto{\pgfqpoint{1.451559in}{0.667155in}}%
\pgfpathlineto{\pgfqpoint{1.452425in}{0.738398in}}%
\pgfpathlineto{\pgfqpoint{1.453289in}{0.696129in}}%
\pgfpathlineto{\pgfqpoint{1.454154in}{0.802356in}}%
\pgfpathlineto{\pgfqpoint{1.455019in}{0.748142in}}%
\pgfpathlineto{\pgfqpoint{1.455884in}{0.775797in}}%
\pgfpathlineto{\pgfqpoint{1.457614in}{0.724698in}}%
\pgfpathlineto{\pgfqpoint{1.458480in}{0.639021in}}%
\pgfpathlineto{\pgfqpoint{1.459345in}{0.733160in}}%
\pgfpathlineto{\pgfqpoint{1.460209in}{0.719095in}}%
\pgfpathlineto{\pgfqpoint{1.461941in}{0.665433in}}%
\pgfpathlineto{\pgfqpoint{1.462806in}{0.686092in}}%
\pgfpathlineto{\pgfqpoint{1.463672in}{0.749279in}}%
\pgfpathlineto{\pgfqpoint{1.464538in}{0.745138in}}%
\pgfpathlineto{\pgfqpoint{1.466268in}{0.663639in}}%
\pgfpathlineto{\pgfqpoint{1.467999in}{0.720049in}}%
\pgfpathlineto{\pgfqpoint{1.468864in}{0.744553in}}%
\pgfpathlineto{\pgfqpoint{1.470594in}{0.690193in}}%
\pgfpathlineto{\pgfqpoint{1.471459in}{0.759349in}}%
\pgfpathlineto{\pgfqpoint{1.472326in}{0.608544in}}%
\pgfpathlineto{\pgfqpoint{1.473192in}{0.706971in}}%
\pgfpathlineto{\pgfqpoint{1.474056in}{0.696860in}}%
\pgfpathlineto{\pgfqpoint{1.474922in}{0.674699in}}%
\pgfpathlineto{\pgfqpoint{1.475788in}{0.686312in}}%
\pgfpathlineto{\pgfqpoint{1.478386in}{0.755873in}}%
\pgfpathlineto{\pgfqpoint{1.480117in}{0.723675in}}%
\pgfpathlineto{\pgfqpoint{1.481847in}{0.744955in}}%
\pgfpathlineto{\pgfqpoint{1.483575in}{0.674261in}}%
\pgfpathlineto{\pgfqpoint{1.484441in}{0.652172in}}%
\pgfpathlineto{\pgfqpoint{1.485307in}{0.731438in}}%
\pgfpathlineto{\pgfqpoint{1.486174in}{0.719387in}}%
\pgfpathlineto{\pgfqpoint{1.487040in}{0.634847in}}%
\pgfpathlineto{\pgfqpoint{1.489635in}{0.749060in}}%
\pgfpathlineto{\pgfqpoint{1.492231in}{0.682795in}}%
\pgfpathlineto{\pgfqpoint{1.493097in}{0.779902in}}%
\pgfpathlineto{\pgfqpoint{1.494827in}{0.691769in}}%
\pgfpathlineto{\pgfqpoint{1.497424in}{0.778546in}}%
\pgfpathlineto{\pgfqpoint{1.499153in}{0.691696in}}%
\pgfpathlineto{\pgfqpoint{1.500017in}{0.726493in}}%
\pgfpathlineto{\pgfqpoint{1.500882in}{0.724808in}}%
\pgfpathlineto{\pgfqpoint{1.501745in}{0.636678in}}%
\pgfpathlineto{\pgfqpoint{1.504339in}{0.758691in}}%
\pgfpathlineto{\pgfqpoint{1.506068in}{0.727958in}}%
\pgfpathlineto{\pgfqpoint{1.506933in}{0.764002in}}%
\pgfpathlineto{\pgfqpoint{1.507798in}{0.699240in}}%
\pgfpathlineto{\pgfqpoint{1.508662in}{0.782355in}}%
\pgfpathlineto{\pgfqpoint{1.509524in}{0.764440in}}%
\pgfpathlineto{\pgfqpoint{1.510389in}{0.697482in}}%
\pgfpathlineto{\pgfqpoint{1.511254in}{0.786675in}}%
\pgfpathlineto{\pgfqpoint{1.512984in}{0.732096in}}%
\pgfpathlineto{\pgfqpoint{1.513848in}{0.779789in}}%
\pgfpathlineto{\pgfqpoint{1.514713in}{0.653342in}}%
\pgfpathlineto{\pgfqpoint{1.515579in}{0.688508in}}%
\pgfpathlineto{\pgfqpoint{1.516443in}{0.733160in}}%
\pgfpathlineto{\pgfqpoint{1.517308in}{0.701513in}}%
\pgfpathlineto{\pgfqpoint{1.518172in}{0.706532in}}%
\pgfpathlineto{\pgfqpoint{1.519904in}{0.720853in}}%
\pgfpathlineto{\pgfqpoint{1.520767in}{0.726387in}}%
\pgfpathlineto{\pgfqpoint{1.522496in}{0.689609in}}%
\pgfpathlineto{\pgfqpoint{1.524226in}{0.764554in}}%
\pgfpathlineto{\pgfqpoint{1.525091in}{0.718839in}}%
\pgfpathlineto{\pgfqpoint{1.525956in}{0.758472in}}%
\pgfpathlineto{\pgfqpoint{1.527685in}{0.715213in}}%
\pgfpathlineto{\pgfqpoint{1.529411in}{0.771038in}}%
\pgfpathlineto{\pgfqpoint{1.531141in}{0.738803in}}%
\pgfpathlineto{\pgfqpoint{1.532005in}{0.744005in}}%
\pgfpathlineto{\pgfqpoint{1.532870in}{0.778034in}}%
\pgfpathlineto{\pgfqpoint{1.533734in}{0.770783in}}%
\pgfpathlineto{\pgfqpoint{1.534599in}{0.715176in}}%
\pgfpathlineto{\pgfqpoint{1.535463in}{0.724592in}}%
\pgfpathlineto{\pgfqpoint{1.537190in}{0.696056in}}%
\pgfpathlineto{\pgfqpoint{1.538055in}{0.649866in}}%
\pgfpathlineto{\pgfqpoint{1.539786in}{0.731990in}}%
\pgfpathlineto{\pgfqpoint{1.540652in}{0.695691in}}%
\pgfpathlineto{\pgfqpoint{1.541516in}{0.748621in}}%
\pgfpathlineto{\pgfqpoint{1.543246in}{0.706313in}}%
\pgfpathlineto{\pgfqpoint{1.544975in}{0.729538in}}%
\pgfpathlineto{\pgfqpoint{1.545841in}{0.719866in}}%
\pgfpathlineto{\pgfqpoint{1.546705in}{0.685288in}}%
\pgfpathlineto{\pgfqpoint{1.547570in}{0.710893in}}%
\pgfpathlineto{\pgfqpoint{1.548435in}{0.671735in}}%
\pgfpathlineto{\pgfqpoint{1.549300in}{0.751001in}}%
\pgfpathlineto{\pgfqpoint{1.550163in}{0.738803in}}%
\pgfpathlineto{\pgfqpoint{1.551028in}{0.725104in}}%
\pgfpathlineto{\pgfqpoint{1.551893in}{0.729205in}}%
\pgfpathlineto{\pgfqpoint{1.552758in}{0.682576in}}%
\pgfpathlineto{\pgfqpoint{1.553624in}{0.693088in}}%
\pgfpathlineto{\pgfqpoint{1.554488in}{0.768801in}}%
\pgfpathlineto{\pgfqpoint{1.555353in}{0.730269in}}%
\pgfpathlineto{\pgfqpoint{1.556218in}{0.745251in}}%
\pgfpathlineto{\pgfqpoint{1.557083in}{0.677119in}}%
\pgfpathlineto{\pgfqpoint{1.557945in}{0.706313in}}%
\pgfpathlineto{\pgfqpoint{1.558810in}{0.692394in}}%
\pgfpathlineto{\pgfqpoint{1.559676in}{0.745324in}}%
\pgfpathlineto{\pgfqpoint{1.561408in}{0.660123in}}%
\pgfpathlineto{\pgfqpoint{1.562273in}{0.776459in}}%
\pgfpathlineto{\pgfqpoint{1.563139in}{0.704262in}}%
\pgfpathlineto{\pgfqpoint{1.564005in}{0.744959in}}%
\pgfpathlineto{\pgfqpoint{1.564872in}{0.654739in}}%
\pgfpathlineto{\pgfqpoint{1.565738in}{0.687302in}}%
\pgfpathlineto{\pgfqpoint{1.567467in}{0.605069in}}%
\pgfpathlineto{\pgfqpoint{1.569196in}{0.751260in}}%
\pgfpathlineto{\pgfqpoint{1.570061in}{0.715582in}}%
\pgfpathlineto{\pgfqpoint{1.570924in}{0.717121in}}%
\pgfpathlineto{\pgfqpoint{1.571789in}{0.723163in}}%
\pgfpathlineto{\pgfqpoint{1.573519in}{0.692175in}}%
\pgfpathlineto{\pgfqpoint{1.574383in}{0.751001in}}%
\pgfpathlineto{\pgfqpoint{1.576114in}{0.698363in}}%
\pgfpathlineto{\pgfqpoint{1.576978in}{0.755873in}}%
\pgfpathlineto{\pgfqpoint{1.577842in}{0.649720in}}%
\pgfpathlineto{\pgfqpoint{1.578706in}{0.665214in}}%
\pgfpathlineto{\pgfqpoint{1.580437in}{0.688585in}}%
\pgfpathlineto{\pgfqpoint{1.581303in}{0.742100in}}%
\pgfpathlineto{\pgfqpoint{1.583036in}{0.642980in}}%
\pgfpathlineto{\pgfqpoint{1.583903in}{0.704591in}}%
\pgfpathlineto{\pgfqpoint{1.584769in}{0.624042in}}%
\pgfpathlineto{\pgfqpoint{1.585634in}{0.732648in}}%
\pgfpathlineto{\pgfqpoint{1.587366in}{0.672247in}}%
\pgfpathlineto{\pgfqpoint{1.588232in}{0.772943in}}%
\pgfpathlineto{\pgfqpoint{1.589962in}{0.721222in}}%
\pgfpathlineto{\pgfqpoint{1.590828in}{0.732356in}}%
\pgfpathlineto{\pgfqpoint{1.591692in}{0.720378in}}%
\pgfpathlineto{\pgfqpoint{1.592558in}{0.663346in}}%
\pgfpathlineto{\pgfqpoint{1.595155in}{0.723163in}}%
\pgfpathlineto{\pgfqpoint{1.596021in}{0.694335in}}%
\pgfpathlineto{\pgfqpoint{1.596886in}{0.612357in}}%
\pgfpathlineto{\pgfqpoint{1.597751in}{0.751845in}}%
\pgfpathlineto{\pgfqpoint{1.598615in}{0.740233in}}%
\pgfpathlineto{\pgfqpoint{1.601207in}{0.700417in}}%
\pgfpathlineto{\pgfqpoint{1.602071in}{0.719647in}}%
\pgfpathlineto{\pgfqpoint{1.602937in}{0.703714in}}%
\pgfpathlineto{\pgfqpoint{1.603803in}{0.710747in}}%
\pgfpathlineto{\pgfqpoint{1.604669in}{0.728035in}}%
\pgfpathlineto{\pgfqpoint{1.606401in}{0.687010in}}%
\pgfpathlineto{\pgfqpoint{1.608130in}{0.716167in}}%
\pgfpathlineto{\pgfqpoint{1.608995in}{0.690745in}}%
\pgfpathlineto{\pgfqpoint{1.609860in}{0.740598in}}%
\pgfpathlineto{\pgfqpoint{1.610725in}{0.651332in}}%
\pgfpathlineto{\pgfqpoint{1.614186in}{0.758001in}}%
\pgfpathlineto{\pgfqpoint{1.615051in}{0.702175in}}%
\pgfpathlineto{\pgfqpoint{1.615917in}{0.756718in}}%
\pgfpathlineto{\pgfqpoint{1.616782in}{0.744886in}}%
\pgfpathlineto{\pgfqpoint{1.617645in}{0.751626in}}%
\pgfpathlineto{\pgfqpoint{1.619375in}{0.653935in}}%
\pgfpathlineto{\pgfqpoint{1.621104in}{0.709025in}}%
\pgfpathlineto{\pgfqpoint{1.621969in}{0.681334in}}%
\pgfpathlineto{\pgfqpoint{1.622835in}{0.792505in}}%
\pgfpathlineto{\pgfqpoint{1.624564in}{0.702431in}}%
\pgfpathlineto{\pgfqpoint{1.625428in}{0.734922in}}%
\pgfpathlineto{\pgfqpoint{1.627159in}{0.688951in}}%
\pgfpathlineto{\pgfqpoint{1.628022in}{0.718218in}}%
\pgfpathlineto{\pgfqpoint{1.628886in}{0.810598in}}%
\pgfpathlineto{\pgfqpoint{1.629752in}{0.703199in}}%
\pgfpathlineto{\pgfqpoint{1.630617in}{0.730049in}}%
\pgfpathlineto{\pgfqpoint{1.631481in}{0.778327in}}%
\pgfpathlineto{\pgfqpoint{1.632344in}{0.706313in}}%
\pgfpathlineto{\pgfqpoint{1.633210in}{0.769865in}}%
\pgfpathlineto{\pgfqpoint{1.634073in}{0.699573in}}%
\pgfpathlineto{\pgfqpoint{1.634936in}{0.700856in}}%
\pgfpathlineto{\pgfqpoint{1.636668in}{0.751476in}}%
\pgfpathlineto{\pgfqpoint{1.637533in}{0.732429in}}%
\pgfpathlineto{\pgfqpoint{1.638397in}{0.779975in}}%
\pgfpathlineto{\pgfqpoint{1.639263in}{0.726350in}}%
\pgfpathlineto{\pgfqpoint{1.640129in}{0.786277in}}%
\pgfpathlineto{\pgfqpoint{1.641860in}{0.644409in}}%
\pgfpathlineto{\pgfqpoint{1.642725in}{0.689682in}}%
\pgfpathlineto{\pgfqpoint{1.643590in}{0.751037in}}%
\pgfpathlineto{\pgfqpoint{1.644456in}{0.699426in}}%
\pgfpathlineto{\pgfqpoint{1.645321in}{0.719793in}}%
\pgfpathlineto{\pgfqpoint{1.646187in}{0.779683in}}%
\pgfpathlineto{\pgfqpoint{1.647053in}{0.772212in}}%
\pgfpathlineto{\pgfqpoint{1.647918in}{0.705468in}}%
\pgfpathlineto{\pgfqpoint{1.648784in}{0.710012in}}%
\pgfpathlineto{\pgfqpoint{1.649645in}{0.681220in}}%
\pgfpathlineto{\pgfqpoint{1.650509in}{0.730488in}}%
\pgfpathlineto{\pgfqpoint{1.651372in}{0.706240in}}%
\pgfpathlineto{\pgfqpoint{1.654830in}{0.790378in}}%
\pgfpathlineto{\pgfqpoint{1.655694in}{0.722578in}}%
\pgfpathlineto{\pgfqpoint{1.657422in}{0.783784in}}%
\pgfpathlineto{\pgfqpoint{1.660016in}{0.683713in}}%
\pgfpathlineto{\pgfqpoint{1.660879in}{0.763604in}}%
\pgfpathlineto{\pgfqpoint{1.662610in}{0.699280in}}%
\pgfpathlineto{\pgfqpoint{1.663476in}{0.724738in}}%
\pgfpathlineto{\pgfqpoint{1.664342in}{0.654665in}}%
\pgfpathlineto{\pgfqpoint{1.666070in}{0.722505in}}%
\pgfpathlineto{\pgfqpoint{1.667800in}{0.731442in}}%
\pgfpathlineto{\pgfqpoint{1.668666in}{0.685142in}}%
\pgfpathlineto{\pgfqpoint{1.669531in}{0.704372in}}%
\pgfpathlineto{\pgfqpoint{1.670393in}{0.692613in}}%
\pgfpathlineto{\pgfqpoint{1.671259in}{0.707669in}}%
\pgfpathlineto{\pgfqpoint{1.672123in}{0.695764in}}%
\pgfpathlineto{\pgfqpoint{1.672989in}{0.732136in}}%
\pgfpathlineto{\pgfqpoint{1.673852in}{0.661698in}}%
\pgfpathlineto{\pgfqpoint{1.675581in}{0.724519in}}%
\pgfpathlineto{\pgfqpoint{1.676445in}{0.739502in}}%
\pgfpathlineto{\pgfqpoint{1.677310in}{0.725985in}}%
\pgfpathlineto{\pgfqpoint{1.678175in}{0.694042in}}%
\pgfpathlineto{\pgfqpoint{1.679039in}{0.708034in}}%
\pgfpathlineto{\pgfqpoint{1.679905in}{0.689170in}}%
\pgfpathlineto{\pgfqpoint{1.681635in}{0.700490in}}%
\pgfpathlineto{\pgfqpoint{1.682499in}{0.749352in}}%
\pgfpathlineto{\pgfqpoint{1.683364in}{0.702943in}}%
\pgfpathlineto{\pgfqpoint{1.684229in}{0.812429in}}%
\pgfpathlineto{\pgfqpoint{1.685959in}{0.707742in}}%
\pgfpathlineto{\pgfqpoint{1.686825in}{0.678987in}}%
\pgfpathlineto{\pgfqpoint{1.687689in}{0.681626in}}%
\pgfpathlineto{\pgfqpoint{1.688554in}{0.678037in}}%
\pgfpathlineto{\pgfqpoint{1.689420in}{0.659172in}}%
\pgfpathlineto{\pgfqpoint{1.690285in}{0.696349in}}%
\pgfpathlineto{\pgfqpoint{1.691149in}{0.682174in}}%
\pgfpathlineto{\pgfqpoint{1.692015in}{0.637450in}}%
\pgfpathlineto{\pgfqpoint{1.693742in}{0.756316in}}%
\pgfpathlineto{\pgfqpoint{1.695472in}{0.683348in}}%
\pgfpathlineto{\pgfqpoint{1.696337in}{0.754740in}}%
\pgfpathlineto{\pgfqpoint{1.697201in}{0.730123in}}%
\pgfpathlineto{\pgfqpoint{1.698064in}{0.756864in}}%
\pgfpathlineto{\pgfqpoint{1.698929in}{0.662136in}}%
\pgfpathlineto{\pgfqpoint{1.699794in}{0.726972in}}%
\pgfpathlineto{\pgfqpoint{1.700660in}{0.653455in}}%
\pgfpathlineto{\pgfqpoint{1.701524in}{0.715286in}}%
\pgfpathlineto{\pgfqpoint{1.702389in}{0.686165in}}%
\pgfpathlineto{\pgfqpoint{1.703252in}{0.697595in}}%
\pgfpathlineto{\pgfqpoint{1.704117in}{0.630563in}}%
\pgfpathlineto{\pgfqpoint{1.705846in}{0.700563in}}%
\pgfpathlineto{\pgfqpoint{1.706709in}{0.651222in}}%
\pgfpathlineto{\pgfqpoint{1.707574in}{0.750124in}}%
\pgfpathlineto{\pgfqpoint{1.709304in}{0.664629in}}%
\pgfpathlineto{\pgfqpoint{1.712763in}{0.783089in}}%
\pgfpathlineto{\pgfqpoint{1.713627in}{0.703089in}}%
\pgfpathlineto{\pgfqpoint{1.714491in}{0.709464in}}%
\pgfpathlineto{\pgfqpoint{1.715355in}{0.695837in}}%
\pgfpathlineto{\pgfqpoint{1.716220in}{0.733639in}}%
\pgfpathlineto{\pgfqpoint{1.717949in}{0.639244in}}%
\pgfpathlineto{\pgfqpoint{1.719680in}{0.706020in}}%
\pgfpathlineto{\pgfqpoint{1.720545in}{0.714409in}}%
\pgfpathlineto{\pgfqpoint{1.721409in}{0.706971in}}%
\pgfpathlineto{\pgfqpoint{1.723138in}{0.630344in}}%
\pgfpathlineto{\pgfqpoint{1.724003in}{0.754005in}}%
\pgfpathlineto{\pgfqpoint{1.724868in}{0.683676in}}%
\pgfpathlineto{\pgfqpoint{1.725732in}{0.787779in}}%
\pgfpathlineto{\pgfqpoint{1.726596in}{0.667594in}}%
\pgfpathlineto{\pgfqpoint{1.728326in}{0.740452in}}%
\pgfpathlineto{\pgfqpoint{1.729191in}{0.678329in}}%
\pgfpathlineto{\pgfqpoint{1.730922in}{0.769353in}}%
\pgfpathlineto{\pgfqpoint{1.731786in}{0.722834in}}%
\pgfpathlineto{\pgfqpoint{1.732650in}{0.794885in}}%
\pgfpathlineto{\pgfqpoint{1.733516in}{0.735653in}}%
\pgfpathlineto{\pgfqpoint{1.734382in}{0.737520in}}%
\pgfpathlineto{\pgfqpoint{1.735246in}{0.771952in}}%
\pgfpathlineto{\pgfqpoint{1.736974in}{0.735872in}}%
\pgfpathlineto{\pgfqpoint{1.737840in}{0.764002in}}%
\pgfpathlineto{\pgfqpoint{1.739570in}{0.683088in}}%
\pgfpathlineto{\pgfqpoint{1.741299in}{0.716127in}}%
\pgfpathlineto{\pgfqpoint{1.743026in}{0.685800in}}%
\pgfpathlineto{\pgfqpoint{1.743891in}{0.725835in}}%
\pgfpathlineto{\pgfqpoint{1.744756in}{0.686019in}}%
\pgfpathlineto{\pgfqpoint{1.745620in}{0.722611in}}%
\pgfpathlineto{\pgfqpoint{1.746486in}{0.683786in}}%
\pgfpathlineto{\pgfqpoint{1.747352in}{0.722465in}}%
\pgfpathlineto{\pgfqpoint{1.748217in}{0.717519in}}%
\pgfpathlineto{\pgfqpoint{1.749082in}{0.732210in}}%
\pgfpathlineto{\pgfqpoint{1.749948in}{0.779829in}}%
\pgfpathlineto{\pgfqpoint{1.750814in}{0.704884in}}%
\pgfpathlineto{\pgfqpoint{1.751679in}{0.711843in}}%
\pgfpathlineto{\pgfqpoint{1.752545in}{0.727045in}}%
\pgfpathlineto{\pgfqpoint{1.753409in}{0.702577in}}%
\pgfpathlineto{\pgfqpoint{1.754272in}{0.737374in}}%
\pgfpathlineto{\pgfqpoint{1.756004in}{0.693677in}}%
\pgfpathlineto{\pgfqpoint{1.756870in}{0.774299in}}%
\pgfpathlineto{\pgfqpoint{1.758602in}{0.690526in}}%
\pgfpathlineto{\pgfqpoint{1.759468in}{0.692394in}}%
\pgfpathlineto{\pgfqpoint{1.761200in}{0.736936in}}%
\pgfpathlineto{\pgfqpoint{1.762066in}{0.726826in}}%
\pgfpathlineto{\pgfqpoint{1.762931in}{0.730305in}}%
\pgfpathlineto{\pgfqpoint{1.763797in}{0.713491in}}%
\pgfpathlineto{\pgfqpoint{1.764663in}{0.716350in}}%
\pgfpathlineto{\pgfqpoint{1.765529in}{0.709537in}}%
\pgfpathlineto{\pgfqpoint{1.766394in}{0.666059in}}%
\pgfpathlineto{\pgfqpoint{1.768988in}{0.766933in}}%
\pgfpathlineto{\pgfqpoint{1.770717in}{0.695066in}}%
\pgfpathlineto{\pgfqpoint{1.771582in}{0.734516in}}%
\pgfpathlineto{\pgfqpoint{1.773309in}{0.658401in}}%
\pgfpathlineto{\pgfqpoint{1.774173in}{0.747338in}}%
\pgfpathlineto{\pgfqpoint{1.775039in}{0.719939in}}%
\pgfpathlineto{\pgfqpoint{1.775903in}{0.728839in}}%
\pgfpathlineto{\pgfqpoint{1.776768in}{0.723017in}}%
\pgfpathlineto{\pgfqpoint{1.777633in}{0.707888in}}%
\pgfpathlineto{\pgfqpoint{1.778496in}{0.726460in}}%
\pgfpathlineto{\pgfqpoint{1.779358in}{0.709317in}}%
\pgfpathlineto{\pgfqpoint{1.780222in}{0.660196in}}%
\pgfpathlineto{\pgfqpoint{1.781951in}{0.721405in}}%
\pgfpathlineto{\pgfqpoint{1.782816in}{0.733127in}}%
\pgfpathlineto{\pgfqpoint{1.783681in}{0.694554in}}%
\pgfpathlineto{\pgfqpoint{1.784545in}{0.707523in}}%
\pgfpathlineto{\pgfqpoint{1.785410in}{0.660342in}}%
\pgfpathlineto{\pgfqpoint{1.788003in}{0.764262in}}%
\pgfpathlineto{\pgfqpoint{1.788868in}{0.626312in}}%
\pgfpathlineto{\pgfqpoint{1.789732in}{0.705176in}}%
\pgfpathlineto{\pgfqpoint{1.790596in}{0.703527in}}%
\pgfpathlineto{\pgfqpoint{1.792326in}{0.780268in}}%
\pgfpathlineto{\pgfqpoint{1.794054in}{0.641697in}}%
\pgfpathlineto{\pgfqpoint{1.794918in}{0.714701in}}%
\pgfpathlineto{\pgfqpoint{1.795782in}{0.685471in}}%
\pgfpathlineto{\pgfqpoint{1.796649in}{0.737082in}}%
\pgfpathlineto{\pgfqpoint{1.797512in}{0.726826in}}%
\pgfpathlineto{\pgfqpoint{1.798377in}{0.672356in}}%
\pgfpathlineto{\pgfqpoint{1.800107in}{0.738219in}}%
\pgfpathlineto{\pgfqpoint{1.800973in}{0.700746in}}%
\pgfpathlineto{\pgfqpoint{1.801837in}{0.724738in}}%
\pgfpathlineto{\pgfqpoint{1.802704in}{0.713970in}}%
\pgfpathlineto{\pgfqpoint{1.803570in}{0.679685in}}%
\pgfpathlineto{\pgfqpoint{1.806165in}{0.762979in}}%
\pgfpathlineto{\pgfqpoint{1.807031in}{0.784848in}}%
\pgfpathlineto{\pgfqpoint{1.807897in}{0.683567in}}%
\pgfpathlineto{\pgfqpoint{1.808763in}{0.776020in}}%
\pgfpathlineto{\pgfqpoint{1.810493in}{0.695874in}}%
\pgfpathlineto{\pgfqpoint{1.811358in}{0.700157in}}%
\pgfpathlineto{\pgfqpoint{1.813954in}{0.746055in}}%
\pgfpathlineto{\pgfqpoint{1.815683in}{0.700304in}}%
\pgfpathlineto{\pgfqpoint{1.816549in}{0.700523in}}%
\pgfpathlineto{\pgfqpoint{1.817413in}{0.749791in}}%
\pgfpathlineto{\pgfqpoint{1.819140in}{0.687814in}}%
\pgfpathlineto{\pgfqpoint{1.820870in}{0.744187in}}%
\pgfpathlineto{\pgfqpoint{1.821736in}{0.680562in}}%
\pgfpathlineto{\pgfqpoint{1.822603in}{0.817265in}}%
\pgfpathlineto{\pgfqpoint{1.823469in}{0.697413in}}%
\pgfpathlineto{\pgfqpoint{1.825198in}{0.781258in}}%
\pgfpathlineto{\pgfqpoint{1.826928in}{0.759170in}}%
\pgfpathlineto{\pgfqpoint{1.827794in}{0.763823in}}%
\pgfpathlineto{\pgfqpoint{1.828657in}{0.697120in}}%
\pgfpathlineto{\pgfqpoint{1.830388in}{0.765837in}}%
\pgfpathlineto{\pgfqpoint{1.831253in}{0.703089in}}%
\pgfpathlineto{\pgfqpoint{1.832979in}{0.790524in}}%
\pgfpathlineto{\pgfqpoint{1.834712in}{0.686677in}}%
\pgfpathlineto{\pgfqpoint{1.835577in}{0.741695in}}%
\pgfpathlineto{\pgfqpoint{1.837307in}{0.676494in}}%
\pgfpathlineto{\pgfqpoint{1.838171in}{0.736603in}}%
\pgfpathlineto{\pgfqpoint{1.839035in}{0.731182in}}%
\pgfpathlineto{\pgfqpoint{1.839899in}{0.755102in}}%
\pgfpathlineto{\pgfqpoint{1.840763in}{0.749937in}}%
\pgfpathlineto{\pgfqpoint{1.842494in}{0.693637in}}%
\pgfpathlineto{\pgfqpoint{1.843359in}{0.756275in}}%
\pgfpathlineto{\pgfqpoint{1.844224in}{0.753234in}}%
\pgfpathlineto{\pgfqpoint{1.845089in}{0.787008in}}%
\pgfpathlineto{\pgfqpoint{1.845955in}{0.751037in}}%
\pgfpathlineto{\pgfqpoint{1.846820in}{0.763417in}}%
\pgfpathlineto{\pgfqpoint{1.849412in}{0.708619in}}%
\pgfpathlineto{\pgfqpoint{1.850276in}{0.715286in}}%
\pgfpathlineto{\pgfqpoint{1.851141in}{0.676754in}}%
\pgfpathlineto{\pgfqpoint{1.852007in}{0.684371in}}%
\pgfpathlineto{\pgfqpoint{1.852870in}{0.725652in}}%
\pgfpathlineto{\pgfqpoint{1.853735in}{0.689170in}}%
\pgfpathlineto{\pgfqpoint{1.854601in}{0.770011in}}%
\pgfpathlineto{\pgfqpoint{1.855465in}{0.744736in}}%
\pgfpathlineto{\pgfqpoint{1.856331in}{0.780121in}}%
\pgfpathlineto{\pgfqpoint{1.857196in}{0.642687in}}%
\pgfpathlineto{\pgfqpoint{1.858062in}{0.670342in}}%
\pgfpathlineto{\pgfqpoint{1.858925in}{0.719574in}}%
\pgfpathlineto{\pgfqpoint{1.859790in}{0.691403in}}%
\pgfpathlineto{\pgfqpoint{1.861521in}{0.761184in}}%
\pgfpathlineto{\pgfqpoint{1.862388in}{0.707779in}}%
\pgfpathlineto{\pgfqpoint{1.864982in}{0.755544in}}%
\pgfpathlineto{\pgfqpoint{1.865845in}{0.765399in}}%
\pgfpathlineto{\pgfqpoint{1.867573in}{0.688549in}}%
\pgfpathlineto{\pgfqpoint{1.868440in}{0.757010in}}%
\pgfpathlineto{\pgfqpoint{1.871032in}{0.642468in}}%
\pgfpathlineto{\pgfqpoint{1.871897in}{0.742908in}}%
\pgfpathlineto{\pgfqpoint{1.872762in}{0.681845in}}%
\pgfpathlineto{\pgfqpoint{1.873626in}{0.710600in}}%
\pgfpathlineto{\pgfqpoint{1.874493in}{0.632431in}}%
\pgfpathlineto{\pgfqpoint{1.876224in}{0.680781in}}%
\pgfpathlineto{\pgfqpoint{1.877089in}{0.678512in}}%
\pgfpathlineto{\pgfqpoint{1.877953in}{0.658401in}}%
\pgfpathlineto{\pgfqpoint{1.878816in}{0.703016in}}%
\pgfpathlineto{\pgfqpoint{1.879681in}{0.676534in}}%
\pgfpathlineto{\pgfqpoint{1.882278in}{0.776459in}}%
\pgfpathlineto{\pgfqpoint{1.884008in}{0.741954in}}%
\pgfpathlineto{\pgfqpoint{1.884871in}{0.777815in}}%
\pgfpathlineto{\pgfqpoint{1.885735in}{0.774664in}}%
\pgfpathlineto{\pgfqpoint{1.886600in}{0.774518in}}%
\pgfpathlineto{\pgfqpoint{1.887463in}{0.838728in}}%
\pgfpathlineto{\pgfqpoint{1.888327in}{0.769020in}}%
\pgfpathlineto{\pgfqpoint{1.889190in}{0.777336in}}%
\pgfpathlineto{\pgfqpoint{1.892650in}{0.694335in}}%
\pgfpathlineto{\pgfqpoint{1.893515in}{0.682138in}}%
\pgfpathlineto{\pgfqpoint{1.894381in}{0.762540in}}%
\pgfpathlineto{\pgfqpoint{1.895246in}{0.659392in}}%
\pgfpathlineto{\pgfqpoint{1.896975in}{0.721368in}}%
\pgfpathlineto{\pgfqpoint{1.897841in}{0.720451in}}%
\pgfpathlineto{\pgfqpoint{1.898707in}{0.711697in}}%
\pgfpathlineto{\pgfqpoint{1.899572in}{0.790524in}}%
\pgfpathlineto{\pgfqpoint{1.901303in}{0.668438in}}%
\pgfpathlineto{\pgfqpoint{1.902169in}{0.747265in}}%
\pgfpathlineto{\pgfqpoint{1.903034in}{0.740452in}}%
\pgfpathlineto{\pgfqpoint{1.903899in}{0.632285in}}%
\pgfpathlineto{\pgfqpoint{1.905629in}{0.783199in}}%
\pgfpathlineto{\pgfqpoint{1.906494in}{0.716642in}}%
\pgfpathlineto{\pgfqpoint{1.907360in}{0.791003in}}%
\pgfpathlineto{\pgfqpoint{1.908225in}{0.690526in}}%
\pgfpathlineto{\pgfqpoint{1.909089in}{0.739721in}}%
\pgfpathlineto{\pgfqpoint{1.909953in}{0.715948in}}%
\pgfpathlineto{\pgfqpoint{1.910818in}{0.649354in}}%
\pgfpathlineto{\pgfqpoint{1.911684in}{0.783824in}}%
\pgfpathlineto{\pgfqpoint{1.912550in}{0.700344in}}%
\pgfpathlineto{\pgfqpoint{1.913415in}{0.774226in}}%
\pgfpathlineto{\pgfqpoint{1.916011in}{0.673384in}}%
\pgfpathlineto{\pgfqpoint{1.916877in}{0.737561in}}%
\pgfpathlineto{\pgfqpoint{1.917742in}{0.733127in}}%
\pgfpathlineto{\pgfqpoint{1.918607in}{0.726095in}}%
\pgfpathlineto{\pgfqpoint{1.919471in}{0.729867in}}%
\pgfpathlineto{\pgfqpoint{1.920335in}{0.676534in}}%
\pgfpathlineto{\pgfqpoint{1.921200in}{0.719574in}}%
\pgfpathlineto{\pgfqpoint{1.922066in}{0.699573in}}%
\pgfpathlineto{\pgfqpoint{1.922931in}{0.728328in}}%
\pgfpathlineto{\pgfqpoint{1.923796in}{0.701554in}}%
\pgfpathlineto{\pgfqpoint{1.924661in}{0.710783in}}%
\pgfpathlineto{\pgfqpoint{1.926393in}{0.662981in}}%
\pgfpathlineto{\pgfqpoint{1.927259in}{0.671881in}}%
\pgfpathlineto{\pgfqpoint{1.928122in}{0.732908in}}%
\pgfpathlineto{\pgfqpoint{1.928985in}{0.684594in}}%
\pgfpathlineto{\pgfqpoint{1.929850in}{0.704006in}}%
\pgfpathlineto{\pgfqpoint{1.930715in}{0.774006in}}%
\pgfpathlineto{\pgfqpoint{1.932446in}{0.701075in}}%
\pgfpathlineto{\pgfqpoint{1.933312in}{0.733931in}}%
\pgfpathlineto{\pgfqpoint{1.934179in}{0.697778in}}%
\pgfpathlineto{\pgfqpoint{1.935043in}{0.733493in}}%
\pgfpathlineto{\pgfqpoint{1.936775in}{0.697559in}}%
\pgfpathlineto{\pgfqpoint{1.938507in}{0.754992in}}%
\pgfpathlineto{\pgfqpoint{1.939373in}{0.754517in}}%
\pgfpathlineto{\pgfqpoint{1.941103in}{0.712318in}}%
\pgfpathlineto{\pgfqpoint{1.941967in}{0.727703in}}%
\pgfpathlineto{\pgfqpoint{1.943696in}{0.712428in}}%
\pgfpathlineto{\pgfqpoint{1.945425in}{0.745357in}}%
\pgfpathlineto{\pgfqpoint{1.946290in}{0.697153in}}%
\pgfpathlineto{\pgfqpoint{1.947154in}{0.734808in}}%
\pgfpathlineto{\pgfqpoint{1.948018in}{0.685211in}}%
\pgfpathlineto{\pgfqpoint{1.948883in}{0.698363in}}%
\pgfpathlineto{\pgfqpoint{1.949748in}{0.736603in}}%
\pgfpathlineto{\pgfqpoint{1.950610in}{0.728068in}}%
\pgfpathlineto{\pgfqpoint{1.951475in}{0.722684in}}%
\pgfpathlineto{\pgfqpoint{1.952341in}{0.707190in}}%
\pgfpathlineto{\pgfqpoint{1.953207in}{0.658620in}}%
\pgfpathlineto{\pgfqpoint{1.954936in}{0.720816in}}%
\pgfpathlineto{\pgfqpoint{1.955801in}{0.723529in}}%
\pgfpathlineto{\pgfqpoint{1.956666in}{0.755175in}}%
\pgfpathlineto{\pgfqpoint{1.957531in}{0.707555in}}%
\pgfpathlineto{\pgfqpoint{1.958396in}{0.784807in}}%
\pgfpathlineto{\pgfqpoint{1.961857in}{0.699865in}}%
\pgfpathlineto{\pgfqpoint{1.963588in}{0.742027in}}%
\pgfpathlineto{\pgfqpoint{1.964454in}{0.805433in}}%
\pgfpathlineto{\pgfqpoint{1.965319in}{0.735141in}}%
\pgfpathlineto{\pgfqpoint{1.966183in}{0.738146in}}%
\pgfpathlineto{\pgfqpoint{1.967048in}{0.697705in}}%
\pgfpathlineto{\pgfqpoint{1.967912in}{0.748694in}}%
\pgfpathlineto{\pgfqpoint{1.968778in}{0.694335in}}%
\pgfpathlineto{\pgfqpoint{1.969644in}{0.744443in}}%
\pgfpathlineto{\pgfqpoint{1.970509in}{0.738657in}}%
\pgfpathlineto{\pgfqpoint{1.971375in}{0.725323in}}%
\pgfpathlineto{\pgfqpoint{1.972240in}{0.766531in}}%
\pgfpathlineto{\pgfqpoint{1.973972in}{0.689755in}}%
\pgfpathlineto{\pgfqpoint{1.975702in}{0.737886in}}%
\pgfpathlineto{\pgfqpoint{1.976567in}{0.727081in}}%
\pgfpathlineto{\pgfqpoint{1.978298in}{0.699426in}}%
\pgfpathlineto{\pgfqpoint{1.980028in}{0.783491in}}%
\pgfpathlineto{\pgfqpoint{1.980894in}{0.677631in}}%
\pgfpathlineto{\pgfqpoint{1.981760in}{0.720597in}}%
\pgfpathlineto{\pgfqpoint{1.982625in}{0.706751in}}%
\pgfpathlineto{\pgfqpoint{1.983491in}{0.765066in}}%
\pgfpathlineto{\pgfqpoint{1.986949in}{0.629500in}}%
\pgfpathlineto{\pgfqpoint{1.987814in}{0.730488in}}%
\pgfpathlineto{\pgfqpoint{1.988674in}{0.716898in}}%
\pgfpathlineto{\pgfqpoint{1.989540in}{0.793821in}}%
\pgfpathlineto{\pgfqpoint{1.990405in}{0.717300in}}%
\pgfpathlineto{\pgfqpoint{1.992132in}{0.785067in}}%
\pgfpathlineto{\pgfqpoint{1.992997in}{0.756385in}}%
\pgfpathlineto{\pgfqpoint{1.993859in}{0.760193in}}%
\pgfpathlineto{\pgfqpoint{1.996456in}{0.693271in}}%
\pgfpathlineto{\pgfqpoint{1.997321in}{0.727118in}}%
\pgfpathlineto{\pgfqpoint{1.998186in}{0.686239in}}%
\pgfpathlineto{\pgfqpoint{1.999051in}{0.700669in}}%
\pgfpathlineto{\pgfqpoint{1.999915in}{0.760705in}}%
\pgfpathlineto{\pgfqpoint{2.001642in}{0.717885in}}%
\pgfpathlineto{\pgfqpoint{2.002508in}{0.766235in}}%
\pgfpathlineto{\pgfqpoint{2.003373in}{0.714295in}}%
\pgfpathlineto{\pgfqpoint{2.004238in}{0.730155in}}%
\pgfpathlineto{\pgfqpoint{2.005103in}{0.678325in}}%
\pgfpathlineto{\pgfqpoint{2.006833in}{0.728986in}}%
\pgfpathlineto{\pgfqpoint{2.007697in}{0.699719in}}%
\pgfpathlineto{\pgfqpoint{2.008563in}{0.757668in}}%
\pgfpathlineto{\pgfqpoint{2.009429in}{0.738032in}}%
\pgfpathlineto{\pgfqpoint{2.011158in}{0.758252in}}%
\pgfpathlineto{\pgfqpoint{2.012023in}{0.720962in}}%
\pgfpathlineto{\pgfqpoint{2.013752in}{0.785944in}}%
\pgfpathlineto{\pgfqpoint{2.014617in}{0.675138in}}%
\pgfpathlineto{\pgfqpoint{2.015482in}{0.704039in}}%
\pgfpathlineto{\pgfqpoint{2.016347in}{0.737740in}}%
\pgfpathlineto{\pgfqpoint{2.017211in}{0.700669in}}%
\pgfpathlineto{\pgfqpoint{2.018077in}{0.722757in}}%
\pgfpathlineto{\pgfqpoint{2.019804in}{0.633235in}}%
\pgfpathlineto{\pgfqpoint{2.020669in}{0.785652in}}%
\pgfpathlineto{\pgfqpoint{2.022400in}{0.676201in}}%
\pgfpathlineto{\pgfqpoint{2.024131in}{0.789574in}}%
\pgfpathlineto{\pgfqpoint{2.024995in}{0.756531in}}%
\pgfpathlineto{\pgfqpoint{2.026727in}{0.700267in}}%
\pgfpathlineto{\pgfqpoint{2.028455in}{0.734808in}}%
\pgfpathlineto{\pgfqpoint{2.029320in}{0.695248in}}%
\pgfpathlineto{\pgfqpoint{2.031052in}{0.766568in}}%
\pgfpathlineto{\pgfqpoint{2.031917in}{0.696203in}}%
\pgfpathlineto{\pgfqpoint{2.032782in}{0.711843in}}%
\pgfpathlineto{\pgfqpoint{2.033646in}{0.715067in}}%
\pgfpathlineto{\pgfqpoint{2.034510in}{0.730415in}}%
\pgfpathlineto{\pgfqpoint{2.035374in}{0.832979in}}%
\pgfpathlineto{\pgfqpoint{2.037104in}{0.754663in}}%
\pgfpathlineto{\pgfqpoint{2.037969in}{0.749279in}}%
\pgfpathlineto{\pgfqpoint{2.038834in}{0.658657in}}%
\pgfpathlineto{\pgfqpoint{2.039700in}{0.775030in}}%
\pgfpathlineto{\pgfqpoint{2.040565in}{0.694189in}}%
\pgfpathlineto{\pgfqpoint{2.041430in}{0.702870in}}%
\pgfpathlineto{\pgfqpoint{2.042296in}{0.704737in}}%
\pgfpathlineto{\pgfqpoint{2.043162in}{0.712468in}}%
\pgfpathlineto{\pgfqpoint{2.044026in}{0.787998in}}%
\pgfpathlineto{\pgfqpoint{2.044891in}{0.787487in}}%
\pgfpathlineto{\pgfqpoint{2.045756in}{0.806461in}}%
\pgfpathlineto{\pgfqpoint{2.048349in}{0.703933in}}%
\pgfpathlineto{\pgfqpoint{2.049214in}{0.818329in}}%
\pgfpathlineto{\pgfqpoint{2.050080in}{0.762321in}}%
\pgfpathlineto{\pgfqpoint{2.051810in}{0.803273in}}%
\pgfpathlineto{\pgfqpoint{2.052674in}{0.736022in}}%
\pgfpathlineto{\pgfqpoint{2.054402in}{0.801552in}}%
\pgfpathlineto{\pgfqpoint{2.055268in}{0.716460in}}%
\pgfpathlineto{\pgfqpoint{2.056131in}{0.729976in}}%
\pgfpathlineto{\pgfqpoint{2.056996in}{0.800488in}}%
\pgfpathlineto{\pgfqpoint{2.057861in}{0.770669in}}%
\pgfpathlineto{\pgfqpoint{2.058726in}{0.686239in}}%
\pgfpathlineto{\pgfqpoint{2.059591in}{0.760965in}}%
\pgfpathlineto{\pgfqpoint{2.060456in}{0.744480in}}%
\pgfpathlineto{\pgfqpoint{2.061322in}{0.739461in}}%
\pgfpathlineto{\pgfqpoint{2.063052in}{0.667667in}}%
\pgfpathlineto{\pgfqpoint{2.063916in}{0.684444in}}%
\pgfpathlineto{\pgfqpoint{2.064780in}{0.767595in}}%
\pgfpathlineto{\pgfqpoint{2.065646in}{0.755508in}}%
\pgfpathlineto{\pgfqpoint{2.067377in}{0.711441in}}%
\pgfpathlineto{\pgfqpoint{2.068243in}{0.762686in}}%
\pgfpathlineto{\pgfqpoint{2.070838in}{0.644482in}}%
\pgfpathlineto{\pgfqpoint{2.073429in}{0.735580in}}%
\pgfpathlineto{\pgfqpoint{2.074294in}{0.744809in}}%
\pgfpathlineto{\pgfqpoint{2.075160in}{0.724406in}}%
\pgfpathlineto{\pgfqpoint{2.076025in}{0.739023in}}%
\pgfpathlineto{\pgfqpoint{2.076890in}{0.681183in}}%
\pgfpathlineto{\pgfqpoint{2.080348in}{0.739023in}}%
\pgfpathlineto{\pgfqpoint{2.081213in}{0.754298in}}%
\pgfpathlineto{\pgfqpoint{2.082078in}{0.690745in}}%
\pgfpathlineto{\pgfqpoint{2.083807in}{0.747229in}}%
\pgfpathlineto{\pgfqpoint{2.085536in}{0.691549in}}%
\pgfpathlineto{\pgfqpoint{2.086401in}{0.752686in}}%
\pgfpathlineto{\pgfqpoint{2.087265in}{0.721880in}}%
\pgfpathlineto{\pgfqpoint{2.089859in}{0.768582in}}%
\pgfpathlineto{\pgfqpoint{2.090724in}{0.757485in}}%
\pgfpathlineto{\pgfqpoint{2.091589in}{0.697299in}}%
\pgfpathlineto{\pgfqpoint{2.094184in}{0.754590in}}%
\pgfpathlineto{\pgfqpoint{2.095914in}{0.704185in}}%
\pgfpathlineto{\pgfqpoint{2.096780in}{0.803639in}}%
\pgfpathlineto{\pgfqpoint{2.099373in}{0.724519in}}%
\pgfpathlineto{\pgfqpoint{2.100239in}{0.764335in}}%
\pgfpathlineto{\pgfqpoint{2.101101in}{0.762321in}}%
\pgfpathlineto{\pgfqpoint{2.101963in}{0.760819in}}%
\pgfpathlineto{\pgfqpoint{2.102828in}{0.716131in}}%
\pgfpathlineto{\pgfqpoint{2.103692in}{0.782761in}}%
\pgfpathlineto{\pgfqpoint{2.104558in}{0.765837in}}%
\pgfpathlineto{\pgfqpoint{2.107151in}{0.699938in}}%
\pgfpathlineto{\pgfqpoint{2.108016in}{0.718510in}}%
\pgfpathlineto{\pgfqpoint{2.108880in}{0.795250in}}%
\pgfpathlineto{\pgfqpoint{2.109746in}{0.790670in}}%
\pgfpathlineto{\pgfqpoint{2.110611in}{0.729278in}}%
\pgfpathlineto{\pgfqpoint{2.111477in}{0.749864in}}%
\pgfpathlineto{\pgfqpoint{2.112342in}{0.689426in}}%
\pgfpathlineto{\pgfqpoint{2.114071in}{0.753786in}}%
\pgfpathlineto{\pgfqpoint{2.114938in}{0.677521in}}%
\pgfpathlineto{\pgfqpoint{2.115804in}{0.755508in}}%
\pgfpathlineto{\pgfqpoint{2.116670in}{0.691111in}}%
\pgfpathlineto{\pgfqpoint{2.117537in}{0.702577in}}%
\pgfpathlineto{\pgfqpoint{2.120134in}{0.820562in}}%
\pgfpathlineto{\pgfqpoint{2.121865in}{0.775947in}}%
\pgfpathlineto{\pgfqpoint{2.122731in}{0.764189in}}%
\pgfpathlineto{\pgfqpoint{2.124463in}{0.681260in}}%
\pgfpathlineto{\pgfqpoint{2.125328in}{0.710162in}}%
\pgfpathlineto{\pgfqpoint{2.126193in}{0.743164in}}%
\pgfpathlineto{\pgfqpoint{2.127924in}{0.686060in}}%
\pgfpathlineto{\pgfqpoint{2.128790in}{0.744740in}}%
\pgfpathlineto{\pgfqpoint{2.129654in}{0.710820in}}%
\pgfpathlineto{\pgfqpoint{2.130520in}{0.721588in}}%
\pgfpathlineto{\pgfqpoint{2.131385in}{0.663529in}}%
\pgfpathlineto{\pgfqpoint{2.132251in}{0.820197in}}%
\pgfpathlineto{\pgfqpoint{2.133116in}{0.812027in}}%
\pgfpathlineto{\pgfqpoint{2.134847in}{0.769061in}}%
\pgfpathlineto{\pgfqpoint{2.135713in}{0.691517in}}%
\pgfpathlineto{\pgfqpoint{2.137440in}{0.770783in}}%
\pgfpathlineto{\pgfqpoint{2.138303in}{0.731442in}}%
\pgfpathlineto{\pgfqpoint{2.139169in}{0.798255in}}%
\pgfpathlineto{\pgfqpoint{2.140031in}{0.724885in}}%
\pgfpathlineto{\pgfqpoint{2.140897in}{0.749096in}}%
\pgfpathlineto{\pgfqpoint{2.141763in}{0.694481in}}%
\pgfpathlineto{\pgfqpoint{2.142629in}{0.742100in}}%
\pgfpathlineto{\pgfqpoint{2.143495in}{0.711002in}}%
\pgfpathlineto{\pgfqpoint{2.144360in}{0.733346in}}%
\pgfpathlineto{\pgfqpoint{2.145225in}{0.711770in}}%
\pgfpathlineto{\pgfqpoint{2.146091in}{0.742612in}}%
\pgfpathlineto{\pgfqpoint{2.146956in}{0.709829in}}%
\pgfpathlineto{\pgfqpoint{2.147822in}{0.723748in}}%
\pgfpathlineto{\pgfqpoint{2.148687in}{0.696974in}}%
\pgfpathlineto{\pgfqpoint{2.149554in}{0.747484in}}%
\pgfpathlineto{\pgfqpoint{2.150418in}{0.743822in}}%
\pgfpathlineto{\pgfqpoint{2.151284in}{0.766495in}}%
\pgfpathlineto{\pgfqpoint{2.152149in}{0.736237in}}%
\pgfpathlineto{\pgfqpoint{2.153015in}{0.650414in}}%
\pgfpathlineto{\pgfqpoint{2.153879in}{0.776897in}}%
\pgfpathlineto{\pgfqpoint{2.154743in}{0.746275in}}%
\pgfpathlineto{\pgfqpoint{2.155607in}{0.699426in}}%
\pgfpathlineto{\pgfqpoint{2.156472in}{0.709975in}}%
\pgfpathlineto{\pgfqpoint{2.157337in}{0.840965in}}%
\pgfpathlineto{\pgfqpoint{2.159066in}{0.705436in}}%
\pgfpathlineto{\pgfqpoint{2.159928in}{0.724410in}}%
\pgfpathlineto{\pgfqpoint{2.160794in}{0.698842in}}%
\pgfpathlineto{\pgfqpoint{2.161658in}{0.745584in}}%
\pgfpathlineto{\pgfqpoint{2.162520in}{0.742580in}}%
\pgfpathlineto{\pgfqpoint{2.163384in}{0.673237in}}%
\pgfpathlineto{\pgfqpoint{2.164251in}{0.830307in}}%
\pgfpathlineto{\pgfqpoint{2.165115in}{0.738073in}}%
\pgfpathlineto{\pgfqpoint{2.165980in}{0.754078in}}%
\pgfpathlineto{\pgfqpoint{2.166844in}{0.735360in}}%
\pgfpathlineto{\pgfqpoint{2.167707in}{0.673530in}}%
\pgfpathlineto{\pgfqpoint{2.169437in}{0.806059in}}%
\pgfpathlineto{\pgfqpoint{2.171166in}{0.651076in}}%
\pgfpathlineto{\pgfqpoint{2.172029in}{0.709866in}}%
\pgfpathlineto{\pgfqpoint{2.172893in}{0.731771in}}%
\pgfpathlineto{\pgfqpoint{2.174620in}{0.668073in}}%
\pgfpathlineto{\pgfqpoint{2.175484in}{0.683201in}}%
\pgfpathlineto{\pgfqpoint{2.176349in}{0.739356in}}%
\pgfpathlineto{\pgfqpoint{2.178080in}{0.691809in}}%
\pgfpathlineto{\pgfqpoint{2.178945in}{0.712614in}}%
\pgfpathlineto{\pgfqpoint{2.179810in}{0.633202in}}%
\pgfpathlineto{\pgfqpoint{2.180676in}{0.652140in}}%
\pgfpathlineto{\pgfqpoint{2.181542in}{0.731588in}}%
\pgfpathlineto{\pgfqpoint{2.182405in}{0.721661in}}%
\pgfpathlineto{\pgfqpoint{2.183270in}{0.739940in}}%
\pgfpathlineto{\pgfqpoint{2.184136in}{0.709244in}}%
\pgfpathlineto{\pgfqpoint{2.185000in}{0.739940in}}%
\pgfpathlineto{\pgfqpoint{2.185866in}{0.737963in}}%
\pgfpathlineto{\pgfqpoint{2.186731in}{0.752503in}}%
\pgfpathlineto{\pgfqpoint{2.187597in}{0.668073in}}%
\pgfpathlineto{\pgfqpoint{2.189326in}{0.859793in}}%
\pgfpathlineto{\pgfqpoint{2.191056in}{0.733237in}}%
\pgfpathlineto{\pgfqpoint{2.192788in}{0.767339in}}%
\pgfpathlineto{\pgfqpoint{2.195384in}{0.717048in}}%
\pgfpathlineto{\pgfqpoint{2.196249in}{0.744082in}}%
\pgfpathlineto{\pgfqpoint{2.197115in}{0.712103in}}%
\pgfpathlineto{\pgfqpoint{2.197982in}{0.752909in}}%
\pgfpathlineto{\pgfqpoint{2.198846in}{0.686571in}}%
\pgfpathlineto{\pgfqpoint{2.199710in}{0.755069in}}%
\pgfpathlineto{\pgfqpoint{2.200575in}{0.665876in}}%
\pgfpathlineto{\pgfqpoint{2.202305in}{0.760161in}}%
\pgfpathlineto{\pgfqpoint{2.204035in}{0.674886in}}%
\pgfpathlineto{\pgfqpoint{2.204900in}{0.740598in}}%
\pgfpathlineto{\pgfqpoint{2.205766in}{0.727743in}}%
\pgfpathlineto{\pgfqpoint{2.206632in}{0.767486in}}%
\pgfpathlineto{\pgfqpoint{2.207495in}{0.764335in}}%
\pgfpathlineto{\pgfqpoint{2.208361in}{0.753713in}}%
\pgfpathlineto{\pgfqpoint{2.209225in}{0.690599in}}%
\pgfpathlineto{\pgfqpoint{2.210090in}{0.693750in}}%
\pgfpathlineto{\pgfqpoint{2.210954in}{0.788437in}}%
\pgfpathlineto{\pgfqpoint{2.211819in}{0.748329in}}%
\pgfpathlineto{\pgfqpoint{2.212685in}{0.771919in}}%
\pgfpathlineto{\pgfqpoint{2.213551in}{0.763823in}}%
\pgfpathlineto{\pgfqpoint{2.215281in}{0.733200in}}%
\pgfpathlineto{\pgfqpoint{2.216146in}{0.743895in}}%
\pgfpathlineto{\pgfqpoint{2.217875in}{0.688732in}}%
\pgfpathlineto{\pgfqpoint{2.218741in}{0.729319in}}%
\pgfpathlineto{\pgfqpoint{2.219607in}{0.681882in}}%
\pgfpathlineto{\pgfqpoint{2.220471in}{0.755142in}}%
\pgfpathlineto{\pgfqpoint{2.221337in}{0.696755in}}%
\pgfpathlineto{\pgfqpoint{2.222202in}{0.734410in}}%
\pgfpathlineto{\pgfqpoint{2.223931in}{0.679100in}}%
\pgfpathlineto{\pgfqpoint{2.224796in}{0.703714in}}%
\pgfpathlineto{\pgfqpoint{2.225661in}{0.763750in}}%
\pgfpathlineto{\pgfqpoint{2.226526in}{0.712505in}}%
\pgfpathlineto{\pgfqpoint{2.227389in}{0.764887in}}%
\pgfpathlineto{\pgfqpoint{2.228254in}{0.731186in}}%
\pgfpathlineto{\pgfqpoint{2.229120in}{0.732835in}}%
\pgfpathlineto{\pgfqpoint{2.229986in}{0.789720in}}%
\pgfpathlineto{\pgfqpoint{2.232580in}{0.727524in}}%
\pgfpathlineto{\pgfqpoint{2.234309in}{0.774518in}}%
\pgfpathlineto{\pgfqpoint{2.235174in}{0.783126in}}%
\pgfpathlineto{\pgfqpoint{2.236039in}{0.672137in}}%
\pgfpathlineto{\pgfqpoint{2.236904in}{0.728255in}}%
\pgfpathlineto{\pgfqpoint{2.237768in}{0.717925in}}%
\pgfpathlineto{\pgfqpoint{2.238633in}{0.670379in}}%
\pgfpathlineto{\pgfqpoint{2.239497in}{0.691330in}}%
\pgfpathlineto{\pgfqpoint{2.240361in}{0.667886in}}%
\pgfpathlineto{\pgfqpoint{2.242090in}{0.752576in}}%
\pgfpathlineto{\pgfqpoint{2.242955in}{0.747192in}}%
\pgfpathlineto{\pgfqpoint{2.244685in}{0.714921in}}%
\pgfpathlineto{\pgfqpoint{2.245550in}{0.757193in}}%
\pgfpathlineto{\pgfqpoint{2.246416in}{0.718437in}}%
\pgfpathlineto{\pgfqpoint{2.247281in}{0.734297in}}%
\pgfpathlineto{\pgfqpoint{2.248147in}{0.658072in}}%
\pgfpathlineto{\pgfqpoint{2.249874in}{0.769500in}}%
\pgfpathlineto{\pgfqpoint{2.251606in}{0.657889in}}%
\pgfpathlineto{\pgfqpoint{2.253336in}{0.727962in}}%
\pgfpathlineto{\pgfqpoint{2.254201in}{0.710600in}}%
\pgfpathlineto{\pgfqpoint{2.255930in}{0.733054in}}%
\pgfpathlineto{\pgfqpoint{2.257658in}{0.667963in}}%
\pgfpathlineto{\pgfqpoint{2.258523in}{0.766203in}}%
\pgfpathlineto{\pgfqpoint{2.259389in}{0.683567in}}%
\pgfpathlineto{\pgfqpoint{2.260254in}{0.693969in}}%
\pgfpathlineto{\pgfqpoint{2.261118in}{0.803493in}}%
\pgfpathlineto{\pgfqpoint{2.262848in}{0.730821in}}%
\pgfpathlineto{\pgfqpoint{2.263712in}{0.773787in}}%
\pgfpathlineto{\pgfqpoint{2.264577in}{0.730711in}}%
\pgfpathlineto{\pgfqpoint{2.265441in}{0.750676in}}%
\pgfpathlineto{\pgfqpoint{2.266306in}{0.711591in}}%
\pgfpathlineto{\pgfqpoint{2.267171in}{0.712399in}}%
\pgfpathlineto{\pgfqpoint{2.268898in}{0.720345in}}%
\pgfpathlineto{\pgfqpoint{2.269763in}{0.677525in}}%
\pgfpathlineto{\pgfqpoint{2.271490in}{0.769792in}}%
\pgfpathlineto{\pgfqpoint{2.272353in}{0.762321in}}%
\pgfpathlineto{\pgfqpoint{2.273219in}{0.779683in}}%
\pgfpathlineto{\pgfqpoint{2.274084in}{0.770015in}}%
\pgfpathlineto{\pgfqpoint{2.274949in}{0.802689in}}%
\pgfpathlineto{\pgfqpoint{2.275812in}{0.715619in}}%
\pgfpathlineto{\pgfqpoint{2.276676in}{0.735949in}}%
\pgfpathlineto{\pgfqpoint{2.277541in}{0.743643in}}%
\pgfpathlineto{\pgfqpoint{2.280137in}{0.664191in}}%
\pgfpathlineto{\pgfqpoint{2.281003in}{0.773568in}}%
\pgfpathlineto{\pgfqpoint{2.281868in}{0.652651in}}%
\pgfpathlineto{\pgfqpoint{2.283598in}{0.714446in}}%
\pgfpathlineto{\pgfqpoint{2.284462in}{0.717998in}}%
\pgfpathlineto{\pgfqpoint{2.286192in}{0.664629in}}%
\pgfpathlineto{\pgfqpoint{2.287056in}{0.671662in}}%
\pgfpathlineto{\pgfqpoint{2.287921in}{0.670013in}}%
\pgfpathlineto{\pgfqpoint{2.290518in}{0.750233in}}%
\pgfpathlineto{\pgfqpoint{2.292249in}{0.625983in}}%
\pgfpathlineto{\pgfqpoint{2.293114in}{0.734849in}}%
\pgfpathlineto{\pgfqpoint{2.293978in}{0.702650in}}%
\pgfpathlineto{\pgfqpoint{2.294842in}{0.746607in}}%
\pgfpathlineto{\pgfqpoint{2.296571in}{0.656095in}}%
\pgfpathlineto{\pgfqpoint{2.297435in}{0.777011in}}%
\pgfpathlineto{\pgfqpoint{2.298302in}{0.676900in}}%
\pgfpathlineto{\pgfqpoint{2.299167in}{0.794227in}}%
\pgfpathlineto{\pgfqpoint{2.300033in}{0.722542in}}%
\pgfpathlineto{\pgfqpoint{2.300900in}{0.779171in}}%
\pgfpathlineto{\pgfqpoint{2.301766in}{0.766129in}}%
\pgfpathlineto{\pgfqpoint{2.302632in}{0.712834in}}%
\pgfpathlineto{\pgfqpoint{2.303497in}{0.716277in}}%
\pgfpathlineto{\pgfqpoint{2.304361in}{0.735580in}}%
\pgfpathlineto{\pgfqpoint{2.305227in}{0.697522in}}%
\pgfpathlineto{\pgfqpoint{2.306092in}{0.729684in}}%
\pgfpathlineto{\pgfqpoint{2.306957in}{0.673895in}}%
\pgfpathlineto{\pgfqpoint{2.308688in}{0.733493in}}%
\pgfpathlineto{\pgfqpoint{2.310420in}{0.632650in}}%
\pgfpathlineto{\pgfqpoint{2.311285in}{0.741370in}}%
\pgfpathlineto{\pgfqpoint{2.312152in}{0.739685in}}%
\pgfpathlineto{\pgfqpoint{2.313017in}{0.690088in}}%
\pgfpathlineto{\pgfqpoint{2.313883in}{0.798401in}}%
\pgfpathlineto{\pgfqpoint{2.315613in}{0.681553in}}%
\pgfpathlineto{\pgfqpoint{2.316478in}{0.707230in}}%
\pgfpathlineto{\pgfqpoint{2.317342in}{0.736278in}}%
\pgfpathlineto{\pgfqpoint{2.318206in}{0.707669in}}%
\pgfpathlineto{\pgfqpoint{2.319072in}{0.714738in}}%
\pgfpathlineto{\pgfqpoint{2.319937in}{0.696129in}}%
\pgfpathlineto{\pgfqpoint{2.321668in}{0.776130in}}%
\pgfpathlineto{\pgfqpoint{2.323399in}{0.712907in}}%
\pgfpathlineto{\pgfqpoint{2.324264in}{0.706276in}}%
\pgfpathlineto{\pgfqpoint{2.325993in}{0.728328in}}%
\pgfpathlineto{\pgfqpoint{2.326858in}{0.719647in}}%
\pgfpathlineto{\pgfqpoint{2.327723in}{0.697559in}}%
\pgfpathlineto{\pgfqpoint{2.328589in}{0.844482in}}%
\pgfpathlineto{\pgfqpoint{2.329452in}{0.694481in}}%
\pgfpathlineto{\pgfqpoint{2.330317in}{0.737520in}}%
\pgfpathlineto{\pgfqpoint{2.332043in}{0.662429in}}%
\pgfpathlineto{\pgfqpoint{2.332908in}{0.681439in}}%
\pgfpathlineto{\pgfqpoint{2.333775in}{0.656826in}}%
\pgfpathlineto{\pgfqpoint{2.334640in}{0.790451in}}%
\pgfpathlineto{\pgfqpoint{2.335506in}{0.719720in}}%
\pgfpathlineto{\pgfqpoint{2.336370in}{0.768769in}}%
\pgfpathlineto{\pgfqpoint{2.337235in}{0.629280in}}%
\pgfpathlineto{\pgfqpoint{2.338965in}{0.738146in}}%
\pgfpathlineto{\pgfqpoint{2.339831in}{0.675178in}}%
\pgfpathlineto{\pgfqpoint{2.340692in}{0.712176in}}%
\pgfpathlineto{\pgfqpoint{2.341556in}{0.651222in}}%
\pgfpathlineto{\pgfqpoint{2.342422in}{0.697486in}}%
\pgfpathlineto{\pgfqpoint{2.343286in}{0.632285in}}%
\pgfpathlineto{\pgfqpoint{2.345015in}{0.692394in}}%
\pgfpathlineto{\pgfqpoint{2.345881in}{0.654885in}}%
\pgfpathlineto{\pgfqpoint{2.346747in}{0.718802in}}%
\pgfpathlineto{\pgfqpoint{2.347612in}{0.666241in}}%
\pgfpathlineto{\pgfqpoint{2.348476in}{0.691225in}}%
\pgfpathlineto{\pgfqpoint{2.349341in}{0.756279in}}%
\pgfpathlineto{\pgfqpoint{2.350207in}{0.651921in}}%
\pgfpathlineto{\pgfqpoint{2.351073in}{0.746827in}}%
\pgfpathlineto{\pgfqpoint{2.353668in}{0.631993in}}%
\pgfpathlineto{\pgfqpoint{2.355397in}{0.712834in}}%
\pgfpathlineto{\pgfqpoint{2.356262in}{0.712907in}}%
\pgfpathlineto{\pgfqpoint{2.357127in}{0.649720in}}%
\pgfpathlineto{\pgfqpoint{2.357993in}{0.742100in}}%
\pgfpathlineto{\pgfqpoint{2.358859in}{0.723236in}}%
\pgfpathlineto{\pgfqpoint{2.359724in}{0.686275in}}%
\pgfpathlineto{\pgfqpoint{2.361451in}{0.805653in}}%
\pgfpathlineto{\pgfqpoint{2.362316in}{0.692723in}}%
\pgfpathlineto{\pgfqpoint{2.363181in}{0.694116in}}%
\pgfpathlineto{\pgfqpoint{2.364045in}{0.783784in}}%
\pgfpathlineto{\pgfqpoint{2.364909in}{0.697413in}}%
\pgfpathlineto{\pgfqpoint{2.365771in}{0.700636in}}%
\pgfpathlineto{\pgfqpoint{2.366638in}{0.652323in}}%
\pgfpathlineto{\pgfqpoint{2.368366in}{0.777523in}}%
\pgfpathlineto{\pgfqpoint{2.369231in}{0.691846in}}%
\pgfpathlineto{\pgfqpoint{2.370096in}{0.734045in}}%
\pgfpathlineto{\pgfqpoint{2.370961in}{0.692248in}}%
\pgfpathlineto{\pgfqpoint{2.371826in}{0.761074in}}%
\pgfpathlineto{\pgfqpoint{2.373557in}{0.726533in}}%
\pgfpathlineto{\pgfqpoint{2.375287in}{0.746754in}}%
\pgfpathlineto{\pgfqpoint{2.376151in}{0.726022in}}%
\pgfpathlineto{\pgfqpoint{2.377017in}{0.751260in}}%
\pgfpathlineto{\pgfqpoint{2.377884in}{0.688951in}}%
\pgfpathlineto{\pgfqpoint{2.378749in}{0.711697in}}%
\pgfpathlineto{\pgfqpoint{2.379616in}{0.772025in}}%
\pgfpathlineto{\pgfqpoint{2.380481in}{0.728913in}}%
\pgfpathlineto{\pgfqpoint{2.381346in}{0.747960in}}%
\pgfpathlineto{\pgfqpoint{2.382212in}{0.734297in}}%
\pgfpathlineto{\pgfqpoint{2.383076in}{0.765212in}}%
\pgfpathlineto{\pgfqpoint{2.384807in}{0.669608in}}%
\pgfpathlineto{\pgfqpoint{2.385670in}{0.773783in}}%
\pgfpathlineto{\pgfqpoint{2.386535in}{0.749425in}}%
\pgfpathlineto{\pgfqpoint{2.388265in}{0.809059in}}%
\pgfpathlineto{\pgfqpoint{2.389995in}{0.712208in}}%
\pgfpathlineto{\pgfqpoint{2.390860in}{0.724369in}}%
\pgfpathlineto{\pgfqpoint{2.391725in}{0.714076in}}%
\pgfpathlineto{\pgfqpoint{2.392590in}{0.790158in}}%
\pgfpathlineto{\pgfqpoint{2.393455in}{0.706313in}}%
\pgfpathlineto{\pgfqpoint{2.394320in}{0.725762in}}%
\pgfpathlineto{\pgfqpoint{2.395187in}{0.689389in}}%
\pgfpathlineto{\pgfqpoint{2.396051in}{0.692467in}}%
\pgfpathlineto{\pgfqpoint{2.396915in}{0.772870in}}%
\pgfpathlineto{\pgfqpoint{2.397781in}{0.709317in}}%
\pgfpathlineto{\pgfqpoint{2.398647in}{0.716496in}}%
\pgfpathlineto{\pgfqpoint{2.399511in}{0.707303in}}%
\pgfpathlineto{\pgfqpoint{2.401240in}{0.748694in}}%
\pgfpathlineto{\pgfqpoint{2.402968in}{0.735766in}}%
\pgfpathlineto{\pgfqpoint{2.403834in}{0.712395in}}%
\pgfpathlineto{\pgfqpoint{2.404699in}{0.733456in}}%
\pgfpathlineto{\pgfqpoint{2.405564in}{0.651442in}}%
\pgfpathlineto{\pgfqpoint{2.406429in}{0.752357in}}%
\pgfpathlineto{\pgfqpoint{2.408157in}{0.658182in}}%
\pgfpathlineto{\pgfqpoint{2.411618in}{0.774737in}}%
\pgfpathlineto{\pgfqpoint{2.413348in}{0.695837in}}%
\pgfpathlineto{\pgfqpoint{2.414213in}{0.719501in}}%
\pgfpathlineto{\pgfqpoint{2.415079in}{0.668584in}}%
\pgfpathlineto{\pgfqpoint{2.415943in}{0.709317in}}%
\pgfpathlineto{\pgfqpoint{2.416808in}{0.669465in}}%
\pgfpathlineto{\pgfqpoint{2.417672in}{0.749758in}}%
\pgfpathlineto{\pgfqpoint{2.418538in}{0.732835in}}%
\pgfpathlineto{\pgfqpoint{2.419402in}{0.702321in}}%
\pgfpathlineto{\pgfqpoint{2.421131in}{0.754225in}}%
\pgfpathlineto{\pgfqpoint{2.421995in}{0.701733in}}%
\pgfpathlineto{\pgfqpoint{2.422861in}{0.723529in}}%
\pgfpathlineto{\pgfqpoint{2.423725in}{0.691111in}}%
\pgfpathlineto{\pgfqpoint{2.424590in}{0.692979in}}%
\pgfpathlineto{\pgfqpoint{2.425455in}{0.737301in}}%
\pgfpathlineto{\pgfqpoint{2.426320in}{0.671918in}}%
\pgfpathlineto{\pgfqpoint{2.427184in}{0.672393in}}%
\pgfpathlineto{\pgfqpoint{2.428912in}{0.754480in}}%
\pgfpathlineto{\pgfqpoint{2.430641in}{0.692394in}}%
\pgfpathlineto{\pgfqpoint{2.431505in}{0.680562in}}%
\pgfpathlineto{\pgfqpoint{2.432370in}{0.699646in}}%
\pgfpathlineto{\pgfqpoint{2.434097in}{0.681220in}}%
\pgfpathlineto{\pgfqpoint{2.435826in}{0.777117in}}%
\pgfpathlineto{\pgfqpoint{2.436690in}{0.648729in}}%
\pgfpathlineto{\pgfqpoint{2.437555in}{0.739169in}}%
\pgfpathlineto{\pgfqpoint{2.438420in}{0.685946in}}%
\pgfpathlineto{\pgfqpoint{2.439284in}{0.731844in}}%
\pgfpathlineto{\pgfqpoint{2.441015in}{0.690672in}}%
\pgfpathlineto{\pgfqpoint{2.442744in}{0.747704in}}%
\pgfpathlineto{\pgfqpoint{2.443608in}{0.671881in}}%
\pgfpathlineto{\pgfqpoint{2.446201in}{0.787560in}}%
\pgfpathlineto{\pgfqpoint{2.447066in}{0.666059in}}%
\pgfpathlineto{\pgfqpoint{2.448795in}{0.732908in}}%
\pgfpathlineto{\pgfqpoint{2.449658in}{0.737926in}}%
\pgfpathlineto{\pgfqpoint{2.450524in}{0.618293in}}%
\pgfpathlineto{\pgfqpoint{2.451388in}{0.749060in}}%
\pgfpathlineto{\pgfqpoint{2.452253in}{0.682397in}}%
\pgfpathlineto{\pgfqpoint{2.453117in}{0.733419in}}%
\pgfpathlineto{\pgfqpoint{2.453982in}{0.724812in}}%
\pgfpathlineto{\pgfqpoint{2.454848in}{0.734958in}}%
\pgfpathlineto{\pgfqpoint{2.455713in}{0.726314in}}%
\pgfpathlineto{\pgfqpoint{2.456577in}{0.793455in}}%
\pgfpathlineto{\pgfqpoint{2.457441in}{0.724994in}}%
\pgfpathlineto{\pgfqpoint{2.458306in}{0.733273in}}%
\pgfpathlineto{\pgfqpoint{2.459170in}{0.752503in}}%
\pgfpathlineto{\pgfqpoint{2.460035in}{0.716642in}}%
\pgfpathlineto{\pgfqpoint{2.462629in}{0.783784in}}%
\pgfpathlineto{\pgfqpoint{2.463494in}{0.785944in}}%
\pgfpathlineto{\pgfqpoint{2.464360in}{0.713199in}}%
\pgfpathlineto{\pgfqpoint{2.465225in}{0.733346in}}%
\pgfpathlineto{\pgfqpoint{2.466090in}{0.737594in}}%
\pgfpathlineto{\pgfqpoint{2.466956in}{0.689097in}}%
\pgfpathlineto{\pgfqpoint{2.467821in}{0.735580in}}%
\pgfpathlineto{\pgfqpoint{2.468685in}{0.728986in}}%
\pgfpathlineto{\pgfqpoint{2.469550in}{0.747631in}}%
\pgfpathlineto{\pgfqpoint{2.471277in}{0.729278in}}%
\pgfpathlineto{\pgfqpoint{2.472140in}{0.642907in}}%
\pgfpathlineto{\pgfqpoint{2.473871in}{0.723200in}}%
\pgfpathlineto{\pgfqpoint{2.474737in}{0.695764in}}%
\pgfpathlineto{\pgfqpoint{2.475602in}{0.705801in}}%
\pgfpathlineto{\pgfqpoint{2.476468in}{0.656350in}}%
\pgfpathlineto{\pgfqpoint{2.478200in}{0.775874in}}%
\pgfpathlineto{\pgfqpoint{2.479066in}{0.734995in}}%
\pgfpathlineto{\pgfqpoint{2.479931in}{0.760965in}}%
\pgfpathlineto{\pgfqpoint{2.480797in}{0.731990in}}%
\pgfpathlineto{\pgfqpoint{2.481662in}{0.746827in}}%
\pgfpathlineto{\pgfqpoint{2.482527in}{0.707815in}}%
\pgfpathlineto{\pgfqpoint{2.483391in}{0.750306in}}%
\pgfpathlineto{\pgfqpoint{2.485119in}{0.703162in}}%
\pgfpathlineto{\pgfqpoint{2.486849in}{0.783053in}}%
\pgfpathlineto{\pgfqpoint{2.487715in}{0.774957in}}%
\pgfpathlineto{\pgfqpoint{2.489447in}{0.708400in}}%
\pgfpathlineto{\pgfqpoint{2.490310in}{0.743310in}}%
\pgfpathlineto{\pgfqpoint{2.492039in}{0.702537in}}%
\pgfpathlineto{\pgfqpoint{2.492905in}{0.727374in}}%
\pgfpathlineto{\pgfqpoint{2.493769in}{0.677119in}}%
\pgfpathlineto{\pgfqpoint{2.495501in}{0.749096in}}%
\pgfpathlineto{\pgfqpoint{2.496365in}{0.682430in}}%
\pgfpathlineto{\pgfqpoint{2.497231in}{0.769719in}}%
\pgfpathlineto{\pgfqpoint{2.498096in}{0.708692in}}%
\pgfpathlineto{\pgfqpoint{2.498962in}{0.752868in}}%
\pgfpathlineto{\pgfqpoint{2.499826in}{0.671512in}}%
\pgfpathlineto{\pgfqpoint{2.500692in}{0.715213in}}%
\pgfpathlineto{\pgfqpoint{2.501555in}{0.705176in}}%
\pgfpathlineto{\pgfqpoint{2.502421in}{0.707998in}}%
\pgfpathlineto{\pgfqpoint{2.503285in}{0.639390in}}%
\pgfpathlineto{\pgfqpoint{2.504149in}{0.747704in}}%
\pgfpathlineto{\pgfqpoint{2.505014in}{0.716825in}}%
\pgfpathlineto{\pgfqpoint{2.505880in}{0.707888in}}%
\pgfpathlineto{\pgfqpoint{2.506746in}{0.684225in}}%
\pgfpathlineto{\pgfqpoint{2.508476in}{0.751878in}}%
\pgfpathlineto{\pgfqpoint{2.509340in}{0.737886in}}%
\pgfpathlineto{\pgfqpoint{2.510205in}{0.741621in}}%
\pgfpathlineto{\pgfqpoint{2.511069in}{0.635103in}}%
\pgfpathlineto{\pgfqpoint{2.511934in}{0.710487in}}%
\pgfpathlineto{\pgfqpoint{2.512798in}{0.695358in}}%
\pgfpathlineto{\pgfqpoint{2.513664in}{0.745138in}}%
\pgfpathlineto{\pgfqpoint{2.514530in}{0.725652in}}%
\pgfpathlineto{\pgfqpoint{2.515396in}{0.746348in}}%
\pgfpathlineto{\pgfqpoint{2.516260in}{0.725981in}}%
\pgfpathlineto{\pgfqpoint{2.517125in}{0.756385in}}%
\pgfpathlineto{\pgfqpoint{2.518855in}{0.660927in}}%
\pgfpathlineto{\pgfqpoint{2.519720in}{0.737520in}}%
\pgfpathlineto{\pgfqpoint{2.520585in}{0.715140in}}%
\pgfpathlineto{\pgfqpoint{2.521451in}{0.722246in}}%
\pgfpathlineto{\pgfqpoint{2.522317in}{0.714076in}}%
\pgfpathlineto{\pgfqpoint{2.524046in}{0.762540in}}%
\pgfpathlineto{\pgfqpoint{2.524912in}{0.761403in}}%
\pgfpathlineto{\pgfqpoint{2.525777in}{0.691403in}}%
\pgfpathlineto{\pgfqpoint{2.527509in}{0.753453in}}%
\pgfpathlineto{\pgfqpoint{2.528374in}{0.696605in}}%
\pgfpathlineto{\pgfqpoint{2.529236in}{0.712208in}}%
\pgfpathlineto{\pgfqpoint{2.530102in}{0.696787in}}%
\pgfpathlineto{\pgfqpoint{2.530966in}{0.734333in}}%
\pgfpathlineto{\pgfqpoint{2.531831in}{0.680010in}}%
\pgfpathlineto{\pgfqpoint{2.533562in}{0.804772in}}%
\pgfpathlineto{\pgfqpoint{2.535290in}{0.705541in}}%
\pgfpathlineto{\pgfqpoint{2.536155in}{0.667959in}}%
\pgfpathlineto{\pgfqpoint{2.537886in}{0.759426in}}%
\pgfpathlineto{\pgfqpoint{2.540482in}{0.650378in}}%
\pgfpathlineto{\pgfqpoint{2.542211in}{0.750156in}}%
\pgfpathlineto{\pgfqpoint{2.543077in}{0.679572in}}%
\pgfpathlineto{\pgfqpoint{2.543942in}{0.773893in}}%
\pgfpathlineto{\pgfqpoint{2.544808in}{0.736164in}}%
\pgfpathlineto{\pgfqpoint{2.545672in}{0.759974in}}%
\pgfpathlineto{\pgfqpoint{2.546537in}{0.684956in}}%
\pgfpathlineto{\pgfqpoint{2.547403in}{0.725579in}}%
\pgfpathlineto{\pgfqpoint{2.548266in}{0.713565in}}%
\pgfpathlineto{\pgfqpoint{2.549131in}{0.731438in}}%
\pgfpathlineto{\pgfqpoint{2.549993in}{0.666237in}}%
\pgfpathlineto{\pgfqpoint{2.551724in}{0.733744in}}%
\pgfpathlineto{\pgfqpoint{2.552590in}{0.675576in}}%
\pgfpathlineto{\pgfqpoint{2.554322in}{0.749791in}}%
\pgfpathlineto{\pgfqpoint{2.555188in}{0.710560in}}%
\pgfpathlineto{\pgfqpoint{2.556051in}{0.749718in}}%
\pgfpathlineto{\pgfqpoint{2.556916in}{0.698217in}}%
\pgfpathlineto{\pgfqpoint{2.557782in}{0.726972in}}%
\pgfpathlineto{\pgfqpoint{2.558645in}{0.688326in}}%
\pgfpathlineto{\pgfqpoint{2.559511in}{0.742576in}}%
\pgfpathlineto{\pgfqpoint{2.560377in}{0.676786in}}%
\pgfpathlineto{\pgfqpoint{2.561241in}{0.677850in}}%
\pgfpathlineto{\pgfqpoint{2.562106in}{0.698728in}}%
\pgfpathlineto{\pgfqpoint{2.562970in}{0.764920in}}%
\pgfpathlineto{\pgfqpoint{2.563836in}{0.763417in}}%
\pgfpathlineto{\pgfqpoint{2.564700in}{0.655469in}}%
\pgfpathlineto{\pgfqpoint{2.566428in}{0.736972in}}%
\pgfpathlineto{\pgfqpoint{2.568157in}{0.782501in}}%
\pgfpathlineto{\pgfqpoint{2.569022in}{0.768472in}}%
\pgfpathlineto{\pgfqpoint{2.569888in}{0.701660in}}%
\pgfpathlineto{\pgfqpoint{2.570753in}{0.761842in}}%
\pgfpathlineto{\pgfqpoint{2.571618in}{0.701733in}}%
\pgfpathlineto{\pgfqpoint{2.573346in}{0.730780in}}%
\pgfpathlineto{\pgfqpoint{2.574211in}{0.685727in}}%
\pgfpathlineto{\pgfqpoint{2.575076in}{0.748841in}}%
\pgfpathlineto{\pgfqpoint{2.575939in}{0.721734in}}%
\pgfpathlineto{\pgfqpoint{2.578530in}{0.771952in}}%
\pgfpathlineto{\pgfqpoint{2.580257in}{0.715140in}}%
\pgfpathlineto{\pgfqpoint{2.581123in}{0.763527in}}%
\pgfpathlineto{\pgfqpoint{2.582854in}{0.641916in}}%
\pgfpathlineto{\pgfqpoint{2.583719in}{0.758033in}}%
\pgfpathlineto{\pgfqpoint{2.586313in}{0.684152in}}%
\pgfpathlineto{\pgfqpoint{2.587177in}{0.670306in}}%
\pgfpathlineto{\pgfqpoint{2.589773in}{0.703162in}}%
\pgfpathlineto{\pgfqpoint{2.590637in}{0.698399in}}%
\pgfpathlineto{\pgfqpoint{2.591500in}{0.769167in}}%
\pgfpathlineto{\pgfqpoint{2.592365in}{0.748069in}}%
\pgfpathlineto{\pgfqpoint{2.593230in}{0.673603in}}%
\pgfpathlineto{\pgfqpoint{2.594095in}{0.706532in}}%
\pgfpathlineto{\pgfqpoint{2.594961in}{0.696129in}}%
\pgfpathlineto{\pgfqpoint{2.595827in}{0.711039in}}%
\pgfpathlineto{\pgfqpoint{2.596692in}{0.747923in}}%
\pgfpathlineto{\pgfqpoint{2.598421in}{0.715652in}}%
\pgfpathlineto{\pgfqpoint{2.599286in}{0.771660in}}%
\pgfpathlineto{\pgfqpoint{2.600150in}{0.758435in}}%
\pgfpathlineto{\pgfqpoint{2.601015in}{0.755540in}}%
\pgfpathlineto{\pgfqpoint{2.601880in}{0.742978in}}%
\pgfpathlineto{\pgfqpoint{2.602742in}{0.698655in}}%
\pgfpathlineto{\pgfqpoint{2.604469in}{0.733050in}}%
\pgfpathlineto{\pgfqpoint{2.605334in}{0.713638in}}%
\pgfpathlineto{\pgfqpoint{2.606197in}{0.726566in}}%
\pgfpathlineto{\pgfqpoint{2.607063in}{0.664333in}}%
\pgfpathlineto{\pgfqpoint{2.607928in}{0.767080in}}%
\pgfpathlineto{\pgfqpoint{2.608793in}{0.745576in}}%
\pgfpathlineto{\pgfqpoint{2.609657in}{0.672502in}}%
\pgfpathlineto{\pgfqpoint{2.611387in}{0.771952in}}%
\pgfpathlineto{\pgfqpoint{2.614846in}{0.717227in}}%
\pgfpathlineto{\pgfqpoint{2.615710in}{0.714994in}}%
\pgfpathlineto{\pgfqpoint{2.616574in}{0.680306in}}%
\pgfpathlineto{\pgfqpoint{2.617439in}{0.754736in}}%
\pgfpathlineto{\pgfqpoint{2.618301in}{0.737447in}}%
\pgfpathlineto{\pgfqpoint{2.619166in}{0.680855in}}%
\pgfpathlineto{\pgfqpoint{2.620894in}{0.754663in}}%
\pgfpathlineto{\pgfqpoint{2.621760in}{0.756019in}}%
\pgfpathlineto{\pgfqpoint{2.624354in}{0.678873in}}%
\pgfpathlineto{\pgfqpoint{2.625219in}{0.688910in}}%
\pgfpathlineto{\pgfqpoint{2.626950in}{0.725981in}}%
\pgfpathlineto{\pgfqpoint{2.627816in}{0.753088in}}%
\pgfpathlineto{\pgfqpoint{2.628682in}{0.720816in}}%
\pgfpathlineto{\pgfqpoint{2.629548in}{0.734370in}}%
\pgfpathlineto{\pgfqpoint{2.630413in}{0.729643in}}%
\pgfpathlineto{\pgfqpoint{2.632145in}{0.653967in}}%
\pgfpathlineto{\pgfqpoint{2.634739in}{0.764846in}}%
\pgfpathlineto{\pgfqpoint{2.635603in}{0.695139in}}%
\pgfpathlineto{\pgfqpoint{2.636470in}{0.709975in}}%
\pgfpathlineto{\pgfqpoint{2.637336in}{0.827960in}}%
\pgfpathlineto{\pgfqpoint{2.639064in}{0.727264in}}%
\pgfpathlineto{\pgfqpoint{2.639929in}{0.722319in}}%
\pgfpathlineto{\pgfqpoint{2.641660in}{0.789574in}}%
\pgfpathlineto{\pgfqpoint{2.642524in}{0.721551in}}%
\pgfpathlineto{\pgfqpoint{2.643388in}{0.757595in}}%
\pgfpathlineto{\pgfqpoint{2.644252in}{0.673310in}}%
\pgfpathlineto{\pgfqpoint{2.645115in}{0.675397in}}%
\pgfpathlineto{\pgfqpoint{2.646844in}{0.731479in}}%
\pgfpathlineto{\pgfqpoint{2.647710in}{0.699244in}}%
\pgfpathlineto{\pgfqpoint{2.648575in}{0.748548in}}%
\pgfpathlineto{\pgfqpoint{2.649441in}{0.701294in}}%
\pgfpathlineto{\pgfqpoint{2.650307in}{0.772723in}}%
\pgfpathlineto{\pgfqpoint{2.651172in}{0.682430in}}%
\pgfpathlineto{\pgfqpoint{2.652902in}{0.749060in}}%
\pgfpathlineto{\pgfqpoint{2.653767in}{0.723748in}}%
\pgfpathlineto{\pgfqpoint{2.654633in}{0.663968in}}%
\pgfpathlineto{\pgfqpoint{2.655499in}{0.762906in}}%
\pgfpathlineto{\pgfqpoint{2.656365in}{0.732283in}}%
\pgfpathlineto{\pgfqpoint{2.657231in}{0.780085in}}%
\pgfpathlineto{\pgfqpoint{2.658097in}{0.710048in}}%
\pgfpathlineto{\pgfqpoint{2.658961in}{0.824371in}}%
\pgfpathlineto{\pgfqpoint{2.660689in}{0.687302in}}%
\pgfpathlineto{\pgfqpoint{2.661555in}{0.704116in}}%
\pgfpathlineto{\pgfqpoint{2.662417in}{0.765691in}}%
\pgfpathlineto{\pgfqpoint{2.664149in}{0.706240in}}%
\pgfpathlineto{\pgfqpoint{2.665878in}{0.746275in}}%
\pgfpathlineto{\pgfqpoint{2.667607in}{0.699938in}}%
\pgfpathlineto{\pgfqpoint{2.668473in}{0.736716in}}%
\pgfpathlineto{\pgfqpoint{2.669338in}{0.670087in}}%
\pgfpathlineto{\pgfqpoint{2.670204in}{0.762906in}}%
\pgfpathlineto{\pgfqpoint{2.671069in}{0.685398in}}%
\pgfpathlineto{\pgfqpoint{2.671934in}{0.763271in}}%
\pgfpathlineto{\pgfqpoint{2.672800in}{0.713638in}}%
\pgfpathlineto{\pgfqpoint{2.674531in}{0.775834in}}%
\pgfpathlineto{\pgfqpoint{2.676261in}{0.712537in}}%
\pgfpathlineto{\pgfqpoint{2.677125in}{0.812685in}}%
\pgfpathlineto{\pgfqpoint{2.678857in}{0.706751in}}%
\pgfpathlineto{\pgfqpoint{2.681452in}{0.808877in}}%
\pgfpathlineto{\pgfqpoint{2.682316in}{0.720378in}}%
\pgfpathlineto{\pgfqpoint{2.683182in}{0.731954in}}%
\pgfpathlineto{\pgfqpoint{2.684047in}{0.753672in}}%
\pgfpathlineto{\pgfqpoint{2.684912in}{0.732502in}}%
\pgfpathlineto{\pgfqpoint{2.685777in}{0.739132in}}%
\pgfpathlineto{\pgfqpoint{2.686642in}{0.711551in}}%
\pgfpathlineto{\pgfqpoint{2.687507in}{0.779829in}}%
\pgfpathlineto{\pgfqpoint{2.688372in}{0.721807in}}%
\pgfpathlineto{\pgfqpoint{2.689237in}{0.730780in}}%
\pgfpathlineto{\pgfqpoint{2.690102in}{0.735616in}}%
\pgfpathlineto{\pgfqpoint{2.690968in}{0.717227in}}%
\pgfpathlineto{\pgfqpoint{2.691834in}{0.810671in}}%
\pgfpathlineto{\pgfqpoint{2.694429in}{0.727483in}}%
\pgfpathlineto{\pgfqpoint{2.695292in}{0.684773in}}%
\pgfpathlineto{\pgfqpoint{2.696157in}{0.690892in}}%
\pgfpathlineto{\pgfqpoint{2.697023in}{0.739315in}}%
\pgfpathlineto{\pgfqpoint{2.697889in}{0.671037in}}%
\pgfpathlineto{\pgfqpoint{2.698752in}{0.735872in}}%
\pgfpathlineto{\pgfqpoint{2.699618in}{0.718766in}}%
\pgfpathlineto{\pgfqpoint{2.700483in}{0.686385in}}%
\pgfpathlineto{\pgfqpoint{2.701346in}{0.691842in}}%
\pgfpathlineto{\pgfqpoint{2.702209in}{0.705359in}}%
\pgfpathlineto{\pgfqpoint{2.703074in}{0.689755in}}%
\pgfpathlineto{\pgfqpoint{2.703940in}{0.739681in}}%
\pgfpathlineto{\pgfqpoint{2.704806in}{0.697413in}}%
\pgfpathlineto{\pgfqpoint{2.705670in}{0.768655in}}%
\pgfpathlineto{\pgfqpoint{2.706536in}{0.754517in}}%
\pgfpathlineto{\pgfqpoint{2.707401in}{0.726972in}}%
\pgfpathlineto{\pgfqpoint{2.709134in}{0.764518in}}%
\pgfpathlineto{\pgfqpoint{2.709995in}{0.737374in}}%
\pgfpathlineto{\pgfqpoint{2.710861in}{0.746128in}}%
\pgfpathlineto{\pgfqpoint{2.712592in}{0.687814in}}%
\pgfpathlineto{\pgfqpoint{2.713458in}{0.782355in}}%
\pgfpathlineto{\pgfqpoint{2.714324in}{0.713857in}}%
\pgfpathlineto{\pgfqpoint{2.715189in}{0.756969in}}%
\pgfpathlineto{\pgfqpoint{2.716054in}{0.738146in}}%
\pgfpathlineto{\pgfqpoint{2.716920in}{0.796826in}}%
\pgfpathlineto{\pgfqpoint{2.717785in}{0.729059in}}%
\pgfpathlineto{\pgfqpoint{2.718650in}{0.752357in}}%
\pgfpathlineto{\pgfqpoint{2.719515in}{0.662502in}}%
\pgfpathlineto{\pgfqpoint{2.722110in}{0.768363in}}%
\pgfpathlineto{\pgfqpoint{2.722975in}{0.736164in}}%
\pgfpathlineto{\pgfqpoint{2.723839in}{0.753896in}}%
\pgfpathlineto{\pgfqpoint{2.724705in}{0.713857in}}%
\pgfpathlineto{\pgfqpoint{2.726436in}{0.753672in}}%
\pgfpathlineto{\pgfqpoint{2.728166in}{0.700121in}}%
\pgfpathlineto{\pgfqpoint{2.729031in}{0.701587in}}%
\pgfpathlineto{\pgfqpoint{2.729896in}{0.778765in}}%
\pgfpathlineto{\pgfqpoint{2.730761in}{0.679060in}}%
\pgfpathlineto{\pgfqpoint{2.732492in}{0.727191in}}%
\pgfpathlineto{\pgfqpoint{2.733357in}{0.733858in}}%
\pgfpathlineto{\pgfqpoint{2.735086in}{0.794406in}}%
\pgfpathlineto{\pgfqpoint{2.735950in}{0.715359in}}%
\pgfpathlineto{\pgfqpoint{2.736813in}{0.748768in}}%
\pgfpathlineto{\pgfqpoint{2.737677in}{0.691038in}}%
\pgfpathlineto{\pgfqpoint{2.739404in}{0.782687in}}%
\pgfpathlineto{\pgfqpoint{2.740269in}{0.688183in}}%
\pgfpathlineto{\pgfqpoint{2.741132in}{0.697632in}}%
\pgfpathlineto{\pgfqpoint{2.741997in}{0.692248in}}%
\pgfpathlineto{\pgfqpoint{2.742860in}{0.707669in}}%
\pgfpathlineto{\pgfqpoint{2.743725in}{0.818215in}}%
\pgfpathlineto{\pgfqpoint{2.745455in}{0.693896in}}%
\pgfpathlineto{\pgfqpoint{2.746322in}{0.723967in}}%
\pgfpathlineto{\pgfqpoint{2.747187in}{0.717154in}}%
\pgfpathlineto{\pgfqpoint{2.748918in}{0.766568in}}%
\pgfpathlineto{\pgfqpoint{2.749784in}{0.746936in}}%
\pgfpathlineto{\pgfqpoint{2.750649in}{0.734077in}}%
\pgfpathlineto{\pgfqpoint{2.752380in}{0.766495in}}%
\pgfpathlineto{\pgfqpoint{2.753245in}{0.758179in}}%
\pgfpathlineto{\pgfqpoint{2.754976in}{0.682576in}}%
\pgfpathlineto{\pgfqpoint{2.755840in}{0.732283in}}%
\pgfpathlineto{\pgfqpoint{2.756706in}{0.696422in}}%
\pgfpathlineto{\pgfqpoint{2.758437in}{0.800853in}}%
\pgfpathlineto{\pgfqpoint{2.761898in}{0.648729in}}%
\pgfpathlineto{\pgfqpoint{2.763629in}{0.736790in}}%
\pgfpathlineto{\pgfqpoint{2.764491in}{0.693385in}}%
\pgfpathlineto{\pgfqpoint{2.765356in}{0.731479in}}%
\pgfpathlineto{\pgfqpoint{2.766220in}{0.680051in}}%
\pgfpathlineto{\pgfqpoint{2.767951in}{0.769134in}}%
\pgfpathlineto{\pgfqpoint{2.768815in}{0.769426in}}%
\pgfpathlineto{\pgfqpoint{2.769681in}{0.728109in}}%
\pgfpathlineto{\pgfqpoint{2.770546in}{0.774006in}}%
\pgfpathlineto{\pgfqpoint{2.772275in}{0.705143in}}%
\pgfpathlineto{\pgfqpoint{2.773139in}{0.684078in}}%
\pgfpathlineto{\pgfqpoint{2.774004in}{0.708546in}}%
\pgfpathlineto{\pgfqpoint{2.774866in}{0.678914in}}%
\pgfpathlineto{\pgfqpoint{2.775728in}{0.709866in}}%
\pgfpathlineto{\pgfqpoint{2.776593in}{0.700450in}}%
\pgfpathlineto{\pgfqpoint{2.777458in}{0.749645in}}%
\pgfpathlineto{\pgfqpoint{2.778324in}{0.727191in}}%
\pgfpathlineto{\pgfqpoint{2.779188in}{0.763344in}}%
\pgfpathlineto{\pgfqpoint{2.781780in}{0.717519in}}%
\pgfpathlineto{\pgfqpoint{2.782645in}{0.790962in}}%
\pgfpathlineto{\pgfqpoint{2.783507in}{0.720597in}}%
\pgfpathlineto{\pgfqpoint{2.784371in}{0.735945in}}%
\pgfpathlineto{\pgfqpoint{2.785235in}{0.701111in}}%
\pgfpathlineto{\pgfqpoint{2.786101in}{0.753932in}}%
\pgfpathlineto{\pgfqpoint{2.786965in}{0.732867in}}%
\pgfpathlineto{\pgfqpoint{2.787828in}{0.756312in}}%
\pgfpathlineto{\pgfqpoint{2.788692in}{0.741841in}}%
\pgfpathlineto{\pgfqpoint{2.789556in}{0.786017in}}%
\pgfpathlineto{\pgfqpoint{2.790421in}{0.648949in}}%
\pgfpathlineto{\pgfqpoint{2.793017in}{0.756092in}}%
\pgfpathlineto{\pgfqpoint{2.793883in}{0.701879in}}%
\pgfpathlineto{\pgfqpoint{2.794748in}{0.736311in}}%
\pgfpathlineto{\pgfqpoint{2.795613in}{0.702391in}}%
\pgfpathlineto{\pgfqpoint{2.796478in}{0.710487in}}%
\pgfpathlineto{\pgfqpoint{2.797343in}{0.779240in}}%
\pgfpathlineto{\pgfqpoint{2.798208in}{0.726200in}}%
\pgfpathlineto{\pgfqpoint{2.799072in}{0.728766in}}%
\pgfpathlineto{\pgfqpoint{2.799938in}{0.705286in}}%
\pgfpathlineto{\pgfqpoint{2.800803in}{0.718802in}}%
\pgfpathlineto{\pgfqpoint{2.802530in}{0.681585in}}%
\pgfpathlineto{\pgfqpoint{2.803395in}{0.789972in}}%
\pgfpathlineto{\pgfqpoint{2.804259in}{0.694920in}}%
\pgfpathlineto{\pgfqpoint{2.805124in}{0.743895in}}%
\pgfpathlineto{\pgfqpoint{2.805989in}{0.737740in}}%
\pgfpathlineto{\pgfqpoint{2.806856in}{0.701185in}}%
\pgfpathlineto{\pgfqpoint{2.808586in}{0.747079in}}%
\pgfpathlineto{\pgfqpoint{2.809451in}{0.683344in}}%
\pgfpathlineto{\pgfqpoint{2.810316in}{0.734370in}}%
\pgfpathlineto{\pgfqpoint{2.811181in}{0.731877in}}%
\pgfpathlineto{\pgfqpoint{2.812047in}{0.708985in}}%
\pgfpathlineto{\pgfqpoint{2.812912in}{0.710341in}}%
\pgfpathlineto{\pgfqpoint{2.813775in}{0.738142in}}%
\pgfpathlineto{\pgfqpoint{2.814641in}{0.624335in}}%
\pgfpathlineto{\pgfqpoint{2.815503in}{0.754371in}}%
\pgfpathlineto{\pgfqpoint{2.816368in}{0.751220in}}%
\pgfpathlineto{\pgfqpoint{2.818096in}{0.706167in}}%
\pgfpathlineto{\pgfqpoint{2.818961in}{0.714336in}}%
\pgfpathlineto{\pgfqpoint{2.819827in}{0.703787in}}%
\pgfpathlineto{\pgfqpoint{2.822421in}{0.765910in}}%
\pgfpathlineto{\pgfqpoint{2.823283in}{0.712208in}}%
\pgfpathlineto{\pgfqpoint{2.824146in}{0.798839in}}%
\pgfpathlineto{\pgfqpoint{2.825873in}{0.702212in}}%
\pgfpathlineto{\pgfqpoint{2.827603in}{0.724738in}}%
\pgfpathlineto{\pgfqpoint{2.828468in}{0.691773in}}%
\pgfpathlineto{\pgfqpoint{2.829334in}{0.739940in}}%
\pgfpathlineto{\pgfqpoint{2.830199in}{0.739169in}}%
\pgfpathlineto{\pgfqpoint{2.831064in}{0.672539in}}%
\pgfpathlineto{\pgfqpoint{2.831927in}{0.783857in}}%
\pgfpathlineto{\pgfqpoint{2.833654in}{0.657045in}}%
\pgfpathlineto{\pgfqpoint{2.834521in}{0.778400in}}%
\pgfpathlineto{\pgfqpoint{2.836252in}{0.708692in}}%
\pgfpathlineto{\pgfqpoint{2.837117in}{0.764002in}}%
\pgfpathlineto{\pgfqpoint{2.838847in}{0.692207in}}%
\pgfpathlineto{\pgfqpoint{2.839712in}{0.747558in}}%
\pgfpathlineto{\pgfqpoint{2.841442in}{0.702577in}}%
\pgfpathlineto{\pgfqpoint{2.842307in}{0.675215in}}%
\pgfpathlineto{\pgfqpoint{2.843172in}{0.697266in}}%
\pgfpathlineto{\pgfqpoint{2.844037in}{0.663127in}}%
\pgfpathlineto{\pgfqpoint{2.845764in}{0.738584in}}%
\pgfpathlineto{\pgfqpoint{2.847492in}{0.682247in}}%
\pgfpathlineto{\pgfqpoint{2.848356in}{0.735653in}}%
\pgfpathlineto{\pgfqpoint{2.849221in}{0.733931in}}%
\pgfpathlineto{\pgfqpoint{2.850086in}{0.717154in}}%
\pgfpathlineto{\pgfqpoint{2.850950in}{0.665287in}}%
\pgfpathlineto{\pgfqpoint{2.851815in}{0.688841in}}%
\pgfpathlineto{\pgfqpoint{2.852680in}{0.653455in}}%
\pgfpathlineto{\pgfqpoint{2.855272in}{0.716642in}}%
\pgfpathlineto{\pgfqpoint{2.856137in}{0.698217in}}%
\pgfpathlineto{\pgfqpoint{2.857003in}{0.655506in}}%
\pgfpathlineto{\pgfqpoint{2.857868in}{0.747484in}}%
\pgfpathlineto{\pgfqpoint{2.858734in}{0.683234in}}%
\pgfpathlineto{\pgfqpoint{2.859599in}{0.757156in}}%
\pgfpathlineto{\pgfqpoint{2.860465in}{0.711624in}}%
\pgfpathlineto{\pgfqpoint{2.861329in}{0.765764in}}%
\pgfpathlineto{\pgfqpoint{2.862193in}{0.705103in}}%
\pgfpathlineto{\pgfqpoint{2.863056in}{0.730488in}}%
\pgfpathlineto{\pgfqpoint{2.863921in}{0.680672in}}%
\pgfpathlineto{\pgfqpoint{2.864787in}{0.763417in}}%
\pgfpathlineto{\pgfqpoint{2.865652in}{0.744553in}}%
\pgfpathlineto{\pgfqpoint{2.866517in}{0.744261in}}%
\pgfpathlineto{\pgfqpoint{2.867382in}{0.706605in}}%
\pgfpathlineto{\pgfqpoint{2.868246in}{0.739315in}}%
\pgfpathlineto{\pgfqpoint{2.869111in}{0.732210in}}%
\pgfpathlineto{\pgfqpoint{2.870840in}{0.679827in}}%
\pgfpathlineto{\pgfqpoint{2.872568in}{0.753015in}}%
\pgfpathlineto{\pgfqpoint{2.873433in}{0.691001in}}%
\pgfpathlineto{\pgfqpoint{2.875161in}{0.760819in}}%
\pgfpathlineto{\pgfqpoint{2.876025in}{0.683051in}}%
\pgfpathlineto{\pgfqpoint{2.876888in}{0.706751in}}%
\pgfpathlineto{\pgfqpoint{2.877753in}{0.763198in}}%
\pgfpathlineto{\pgfqpoint{2.879484in}{0.687375in}}%
\pgfpathlineto{\pgfqpoint{2.881212in}{0.775103in}}%
\pgfpathlineto{\pgfqpoint{2.882941in}{0.711953in}}%
\pgfpathlineto{\pgfqpoint{2.883803in}{0.740525in}}%
\pgfpathlineto{\pgfqpoint{2.884668in}{0.735872in}}%
\pgfpathlineto{\pgfqpoint{2.885532in}{0.649903in}}%
\pgfpathlineto{\pgfqpoint{2.886397in}{0.786277in}}%
\pgfpathlineto{\pgfqpoint{2.888127in}{0.703820in}}%
\pgfpathlineto{\pgfqpoint{2.888993in}{0.687116in}}%
\pgfpathlineto{\pgfqpoint{2.889857in}{0.699605in}}%
\pgfpathlineto{\pgfqpoint{2.891587in}{0.779569in}}%
\pgfpathlineto{\pgfqpoint{2.893318in}{0.731584in}}%
\pgfpathlineto{\pgfqpoint{2.894183in}{0.735799in}}%
\pgfpathlineto{\pgfqpoint{2.895047in}{0.732356in}}%
\pgfpathlineto{\pgfqpoint{2.895913in}{0.740452in}}%
\pgfpathlineto{\pgfqpoint{2.898505in}{0.658328in}}%
\pgfpathlineto{\pgfqpoint{2.899368in}{0.756019in}}%
\pgfpathlineto{\pgfqpoint{2.900234in}{0.630783in}}%
\pgfpathlineto{\pgfqpoint{2.901099in}{0.740013in}}%
\pgfpathlineto{\pgfqpoint{2.901964in}{0.727780in}}%
\pgfpathlineto{\pgfqpoint{2.902827in}{0.685215in}}%
\pgfpathlineto{\pgfqpoint{2.905420in}{0.759828in}}%
\pgfpathlineto{\pgfqpoint{2.906286in}{0.675836in}}%
\pgfpathlineto{\pgfqpoint{2.907152in}{0.694042in}}%
\pgfpathlineto{\pgfqpoint{2.908881in}{0.756791in}}%
\pgfpathlineto{\pgfqpoint{2.909746in}{0.702650in}}%
\pgfpathlineto{\pgfqpoint{2.911476in}{0.776861in}}%
\pgfpathlineto{\pgfqpoint{2.914071in}{0.695874in}}%
\pgfpathlineto{\pgfqpoint{2.914936in}{0.712687in}}%
\pgfpathlineto{\pgfqpoint{2.915801in}{0.741881in}}%
\pgfpathlineto{\pgfqpoint{2.916667in}{0.714336in}}%
\pgfpathlineto{\pgfqpoint{2.917532in}{0.791003in}}%
\pgfpathlineto{\pgfqpoint{2.918398in}{0.698951in}}%
\pgfpathlineto{\pgfqpoint{2.919262in}{0.726022in}}%
\pgfpathlineto{\pgfqpoint{2.920993in}{0.709281in}}%
\pgfpathlineto{\pgfqpoint{2.921858in}{0.743384in}}%
\pgfpathlineto{\pgfqpoint{2.922722in}{0.696129in}}%
\pgfpathlineto{\pgfqpoint{2.923588in}{0.798766in}}%
\pgfpathlineto{\pgfqpoint{2.924453in}{0.775322in}}%
\pgfpathlineto{\pgfqpoint{2.926183in}{0.675836in}}%
\pgfpathlineto{\pgfqpoint{2.927046in}{0.792246in}}%
\pgfpathlineto{\pgfqpoint{2.927911in}{0.750562in}}%
\pgfpathlineto{\pgfqpoint{2.928776in}{0.769207in}}%
\pgfpathlineto{\pgfqpoint{2.929642in}{0.721880in}}%
\pgfpathlineto{\pgfqpoint{2.931370in}{0.798035in}}%
\pgfpathlineto{\pgfqpoint{2.932235in}{0.770709in}}%
\pgfpathlineto{\pgfqpoint{2.933966in}{0.707669in}}%
\pgfpathlineto{\pgfqpoint{2.935694in}{0.745836in}}%
\pgfpathlineto{\pgfqpoint{2.936558in}{0.754225in}}%
\pgfpathlineto{\pgfqpoint{2.937423in}{0.733383in}}%
\pgfpathlineto{\pgfqpoint{2.939153in}{0.754371in}}%
\pgfpathlineto{\pgfqpoint{2.940017in}{0.721697in}}%
\pgfpathlineto{\pgfqpoint{2.941744in}{0.775322in}}%
\pgfpathlineto{\pgfqpoint{2.942610in}{0.718766in}}%
\pgfpathlineto{\pgfqpoint{2.943475in}{0.728839in}}%
\pgfpathlineto{\pgfqpoint{2.944340in}{0.707596in}}%
\pgfpathlineto{\pgfqpoint{2.945203in}{0.778985in}}%
\pgfpathlineto{\pgfqpoint{2.946068in}{0.774551in}}%
\pgfpathlineto{\pgfqpoint{2.946935in}{0.742868in}}%
\pgfpathlineto{\pgfqpoint{2.947799in}{0.774551in}}%
\pgfpathlineto{\pgfqpoint{2.948660in}{0.692865in}}%
\pgfpathlineto{\pgfqpoint{2.950389in}{0.740192in}}%
\pgfpathlineto{\pgfqpoint{2.951255in}{0.702756in}}%
\pgfpathlineto{\pgfqpoint{2.952121in}{0.717921in}}%
\pgfpathlineto{\pgfqpoint{2.952984in}{0.708838in}}%
\pgfpathlineto{\pgfqpoint{2.953848in}{0.665945in}}%
\pgfpathlineto{\pgfqpoint{2.955575in}{0.740265in}}%
\pgfpathlineto{\pgfqpoint{2.956440in}{0.738178in}}%
\pgfpathlineto{\pgfqpoint{2.957306in}{0.687116in}}%
\pgfpathlineto{\pgfqpoint{2.958170in}{0.762719in}}%
\pgfpathlineto{\pgfqpoint{2.959036in}{0.659534in}}%
\pgfpathlineto{\pgfqpoint{2.961632in}{0.744809in}}%
\pgfpathlineto{\pgfqpoint{2.963362in}{0.695724in}}%
\pgfpathlineto{\pgfqpoint{2.965093in}{0.763856in}}%
\pgfpathlineto{\pgfqpoint{2.966821in}{0.705761in}}%
\pgfpathlineto{\pgfqpoint{2.967687in}{0.725981in}}%
\pgfpathlineto{\pgfqpoint{2.968551in}{0.719351in}}%
\pgfpathlineto{\pgfqpoint{2.969413in}{0.686604in}}%
\pgfpathlineto{\pgfqpoint{2.970278in}{0.714588in}}%
\pgfpathlineto{\pgfqpoint{2.971143in}{0.781108in}}%
\pgfpathlineto{\pgfqpoint{2.972009in}{0.760340in}}%
\pgfpathlineto{\pgfqpoint{2.972874in}{0.760851in}}%
\pgfpathlineto{\pgfqpoint{2.973738in}{0.789972in}}%
\pgfpathlineto{\pgfqpoint{2.975468in}{0.698765in}}%
\pgfpathlineto{\pgfqpoint{2.976333in}{0.751805in}}%
\pgfpathlineto{\pgfqpoint{2.977198in}{0.731730in}}%
\pgfpathlineto{\pgfqpoint{2.978062in}{0.761988in}}%
\pgfpathlineto{\pgfqpoint{2.979791in}{0.692759in}}%
\pgfpathlineto{\pgfqpoint{2.980655in}{0.696495in}}%
\pgfpathlineto{\pgfqpoint{2.982384in}{0.762353in}}%
\pgfpathlineto{\pgfqpoint{2.983249in}{0.658839in}}%
\pgfpathlineto{\pgfqpoint{2.984114in}{0.780341in}}%
\pgfpathlineto{\pgfqpoint{2.984979in}{0.689938in}}%
\pgfpathlineto{\pgfqpoint{2.985844in}{0.742978in}}%
\pgfpathlineto{\pgfqpoint{2.986710in}{0.606607in}}%
\pgfpathlineto{\pgfqpoint{2.989305in}{0.744114in}}%
\pgfpathlineto{\pgfqpoint{2.990170in}{0.744224in}}%
\pgfpathlineto{\pgfqpoint{2.991034in}{0.794479in}}%
\pgfpathlineto{\pgfqpoint{2.991898in}{0.698436in}}%
\pgfpathlineto{\pgfqpoint{2.992763in}{0.753234in}}%
\pgfpathlineto{\pgfqpoint{2.993628in}{0.747631in}}%
\pgfpathlineto{\pgfqpoint{2.994494in}{0.741296in}}%
\pgfpathlineto{\pgfqpoint{2.995360in}{0.846751in}}%
\pgfpathlineto{\pgfqpoint{2.997090in}{0.709646in}}%
\pgfpathlineto{\pgfqpoint{2.997954in}{0.775030in}}%
\pgfpathlineto{\pgfqpoint{2.998820in}{0.763052in}}%
\pgfpathlineto{\pgfqpoint{2.999686in}{0.708436in}}%
\pgfpathlineto{\pgfqpoint{3.000551in}{0.734150in}}%
\pgfpathlineto{\pgfqpoint{3.001416in}{0.716642in}}%
\pgfpathlineto{\pgfqpoint{3.002282in}{0.719939in}}%
\pgfpathlineto{\pgfqpoint{3.003146in}{0.726899in}}%
\pgfpathlineto{\pgfqpoint{3.004010in}{0.702431in}}%
\pgfpathlineto{\pgfqpoint{3.005740in}{0.766568in}}%
\pgfpathlineto{\pgfqpoint{3.008335in}{0.709975in}}%
\pgfpathlineto{\pgfqpoint{3.009201in}{0.738475in}}%
\pgfpathlineto{\pgfqpoint{3.010065in}{0.698363in}}%
\pgfpathlineto{\pgfqpoint{3.010930in}{0.732429in}}%
\pgfpathlineto{\pgfqpoint{3.011793in}{0.691330in}}%
\pgfpathlineto{\pgfqpoint{3.013525in}{0.718181in}}%
\pgfpathlineto{\pgfqpoint{3.014390in}{0.708254in}}%
\pgfpathlineto{\pgfqpoint{3.015255in}{0.752868in}}%
\pgfpathlineto{\pgfqpoint{3.016120in}{0.699938in}}%
\pgfpathlineto{\pgfqpoint{3.016985in}{0.705176in}}%
\pgfpathlineto{\pgfqpoint{3.019578in}{0.795762in}}%
\pgfpathlineto{\pgfqpoint{3.021309in}{0.701075in}}%
\pgfpathlineto{\pgfqpoint{3.023036in}{0.764664in}}%
\pgfpathlineto{\pgfqpoint{3.023900in}{0.721515in}}%
\pgfpathlineto{\pgfqpoint{3.024765in}{0.726826in}}%
\pgfpathlineto{\pgfqpoint{3.025630in}{0.729574in}}%
\pgfpathlineto{\pgfqpoint{3.026495in}{0.787852in}}%
\pgfpathlineto{\pgfqpoint{3.027361in}{0.786350in}}%
\pgfpathlineto{\pgfqpoint{3.028227in}{0.753859in}}%
\pgfpathlineto{\pgfqpoint{3.029093in}{0.776824in}}%
\pgfpathlineto{\pgfqpoint{3.029959in}{0.702504in}}%
\pgfpathlineto{\pgfqpoint{3.030824in}{0.740525in}}%
\pgfpathlineto{\pgfqpoint{3.031687in}{0.733785in}}%
\pgfpathlineto{\pgfqpoint{3.032552in}{0.726899in}}%
\pgfpathlineto{\pgfqpoint{3.033415in}{0.696349in}}%
\pgfpathlineto{\pgfqpoint{3.034281in}{0.818694in}}%
\pgfpathlineto{\pgfqpoint{3.035146in}{0.714299in}}%
\pgfpathlineto{\pgfqpoint{3.036011in}{0.853491in}}%
\pgfpathlineto{\pgfqpoint{3.038605in}{0.687814in}}%
\pgfpathlineto{\pgfqpoint{3.039470in}{0.753640in}}%
\pgfpathlineto{\pgfqpoint{3.040334in}{0.700271in}}%
\pgfpathlineto{\pgfqpoint{3.041199in}{0.777669in}}%
\pgfpathlineto{\pgfqpoint{3.042064in}{0.729538in}}%
\pgfpathlineto{\pgfqpoint{3.042930in}{0.782541in}}%
\pgfpathlineto{\pgfqpoint{3.044659in}{0.740415in}}%
\pgfpathlineto{\pgfqpoint{3.046388in}{0.727889in}}%
\pgfpathlineto{\pgfqpoint{3.047252in}{0.795214in}}%
\pgfpathlineto{\pgfqpoint{3.048983in}{0.719208in}}%
\pgfpathlineto{\pgfqpoint{3.049848in}{0.713970in}}%
\pgfpathlineto{\pgfqpoint{3.050713in}{0.796168in}}%
\pgfpathlineto{\pgfqpoint{3.051578in}{0.781295in}}%
\pgfpathlineto{\pgfqpoint{3.052443in}{0.726022in}}%
\pgfpathlineto{\pgfqpoint{3.053308in}{0.777303in}}%
\pgfpathlineto{\pgfqpoint{3.055037in}{0.686644in}}%
\pgfpathlineto{\pgfqpoint{3.055901in}{0.746827in}}%
\pgfpathlineto{\pgfqpoint{3.056766in}{0.612470in}}%
\pgfpathlineto{\pgfqpoint{3.058495in}{0.827229in}}%
\pgfpathlineto{\pgfqpoint{3.059359in}{0.681845in}}%
\pgfpathlineto{\pgfqpoint{3.060224in}{0.810817in}}%
\pgfpathlineto{\pgfqpoint{3.061088in}{0.732502in}}%
\pgfpathlineto{\pgfqpoint{3.061953in}{0.744041in}}%
\pgfpathlineto{\pgfqpoint{3.062818in}{0.760453in}}%
\pgfpathlineto{\pgfqpoint{3.063682in}{0.740525in}}%
\pgfpathlineto{\pgfqpoint{3.064545in}{0.743603in}}%
\pgfpathlineto{\pgfqpoint{3.065410in}{0.728255in}}%
\pgfpathlineto{\pgfqpoint{3.066275in}{0.678366in}}%
\pgfpathlineto{\pgfqpoint{3.067139in}{0.809940in}}%
\pgfpathlineto{\pgfqpoint{3.068870in}{0.702797in}}%
\pgfpathlineto{\pgfqpoint{3.069735in}{0.733054in}}%
\pgfpathlineto{\pgfqpoint{3.070600in}{0.717816in}}%
\pgfpathlineto{\pgfqpoint{3.072330in}{0.745397in}}%
\pgfpathlineto{\pgfqpoint{3.073195in}{0.712103in}}%
\pgfpathlineto{\pgfqpoint{3.074926in}{0.733566in}}%
\pgfpathlineto{\pgfqpoint{3.075790in}{0.709866in}}%
\pgfpathlineto{\pgfqpoint{3.076655in}{0.723821in}}%
\pgfpathlineto{\pgfqpoint{3.077520in}{0.708473in}}%
\pgfpathlineto{\pgfqpoint{3.078385in}{0.663200in}}%
\pgfpathlineto{\pgfqpoint{3.079251in}{0.664849in}}%
\pgfpathlineto{\pgfqpoint{3.080982in}{0.740452in}}%
\pgfpathlineto{\pgfqpoint{3.081847in}{0.815763in}}%
\pgfpathlineto{\pgfqpoint{3.082712in}{0.750416in}}%
\pgfpathlineto{\pgfqpoint{3.083578in}{0.754590in}}%
\pgfpathlineto{\pgfqpoint{3.084443in}{0.780487in}}%
\pgfpathlineto{\pgfqpoint{3.086170in}{0.735872in}}%
\pgfpathlineto{\pgfqpoint{3.087036in}{0.812174in}}%
\pgfpathlineto{\pgfqpoint{3.090498in}{0.680051in}}%
\pgfpathlineto{\pgfqpoint{3.093092in}{0.750051in}}%
\pgfpathlineto{\pgfqpoint{3.094822in}{0.685142in}}%
\pgfpathlineto{\pgfqpoint{3.095687in}{0.693677in}}%
\pgfpathlineto{\pgfqpoint{3.096552in}{0.734995in}}%
\pgfpathlineto{\pgfqpoint{3.097417in}{0.725177in}}%
\pgfpathlineto{\pgfqpoint{3.099147in}{0.756641in}}%
\pgfpathlineto{\pgfqpoint{3.100012in}{0.682722in}}%
\pgfpathlineto{\pgfqpoint{3.100878in}{0.725177in}}%
\pgfpathlineto{\pgfqpoint{3.101742in}{0.685435in}}%
\pgfpathlineto{\pgfqpoint{3.103470in}{0.738292in}}%
\pgfpathlineto{\pgfqpoint{3.104334in}{0.731588in}}%
\pgfpathlineto{\pgfqpoint{3.105199in}{0.695764in}}%
\pgfpathlineto{\pgfqpoint{3.106065in}{0.713272in}}%
\pgfpathlineto{\pgfqpoint{3.106930in}{0.694554in}}%
\pgfpathlineto{\pgfqpoint{3.107795in}{0.759682in}}%
\pgfpathlineto{\pgfqpoint{3.108661in}{0.742100in}}%
\pgfpathlineto{\pgfqpoint{3.109526in}{0.693823in}}%
\pgfpathlineto{\pgfqpoint{3.110391in}{0.705801in}}%
\pgfpathlineto{\pgfqpoint{3.111255in}{0.688878in}}%
\pgfpathlineto{\pgfqpoint{3.112119in}{0.751512in}}%
\pgfpathlineto{\pgfqpoint{3.112985in}{0.665068in}}%
\pgfpathlineto{\pgfqpoint{3.113850in}{0.777121in}}%
\pgfpathlineto{\pgfqpoint{3.114714in}{0.748914in}}%
\pgfpathlineto{\pgfqpoint{3.115579in}{0.749206in}}%
\pgfpathlineto{\pgfqpoint{3.117306in}{0.717925in}}%
\pgfpathlineto{\pgfqpoint{3.118172in}{0.723711in}}%
\pgfpathlineto{\pgfqpoint{3.119038in}{0.707961in}}%
\pgfpathlineto{\pgfqpoint{3.119902in}{0.749206in}}%
\pgfpathlineto{\pgfqpoint{3.120768in}{0.720122in}}%
\pgfpathlineto{\pgfqpoint{3.122496in}{0.767559in}}%
\pgfpathlineto{\pgfqpoint{3.123362in}{0.732173in}}%
\pgfpathlineto{\pgfqpoint{3.124228in}{0.744918in}}%
\pgfpathlineto{\pgfqpoint{3.125956in}{0.698290in}}%
\pgfpathlineto{\pgfqpoint{3.126821in}{0.730853in}}%
\pgfpathlineto{\pgfqpoint{3.127686in}{0.658507in}}%
\pgfpathlineto{\pgfqpoint{3.128550in}{0.684184in}}%
\pgfpathlineto{\pgfqpoint{3.129414in}{0.751512in}}%
\pgfpathlineto{\pgfqpoint{3.130279in}{0.678179in}}%
\pgfpathlineto{\pgfqpoint{3.132007in}{0.750229in}}%
\pgfpathlineto{\pgfqpoint{3.133740in}{0.713491in}}%
\pgfpathlineto{\pgfqpoint{3.134605in}{0.763271in}}%
\pgfpathlineto{\pgfqpoint{3.135471in}{0.698582in}}%
\pgfpathlineto{\pgfqpoint{3.136337in}{0.731292in}}%
\pgfpathlineto{\pgfqpoint{3.137201in}{0.717081in}}%
\pgfpathlineto{\pgfqpoint{3.138930in}{0.758983in}}%
\pgfpathlineto{\pgfqpoint{3.140660in}{0.678288in}}%
\pgfpathlineto{\pgfqpoint{3.141527in}{0.714076in}}%
\pgfpathlineto{\pgfqpoint{3.142392in}{0.676348in}}%
\pgfpathlineto{\pgfqpoint{3.143259in}{0.751001in}}%
\pgfpathlineto{\pgfqpoint{3.144124in}{0.746348in}}%
\pgfpathlineto{\pgfqpoint{3.144990in}{0.696276in}}%
\pgfpathlineto{\pgfqpoint{3.145857in}{0.770450in}}%
\pgfpathlineto{\pgfqpoint{3.147589in}{0.718949in}}%
\pgfpathlineto{\pgfqpoint{3.148455in}{0.724040in}}%
\pgfpathlineto{\pgfqpoint{3.149320in}{0.720195in}}%
\pgfpathlineto{\pgfqpoint{3.151050in}{0.758033in}}%
\pgfpathlineto{\pgfqpoint{3.151916in}{0.716935in}}%
\pgfpathlineto{\pgfqpoint{3.152782in}{0.761549in}}%
\pgfpathlineto{\pgfqpoint{3.155373in}{0.632650in}}%
\pgfpathlineto{\pgfqpoint{3.156236in}{0.735507in}}%
\pgfpathlineto{\pgfqpoint{3.157101in}{0.723821in}}%
\pgfpathlineto{\pgfqpoint{3.158829in}{0.714263in}}%
\pgfpathlineto{\pgfqpoint{3.159695in}{0.737155in}}%
\pgfpathlineto{\pgfqpoint{3.160561in}{0.685873in}}%
\pgfpathlineto{\pgfqpoint{3.161425in}{0.692540in}}%
\pgfpathlineto{\pgfqpoint{3.162292in}{0.696056in}}%
\pgfpathlineto{\pgfqpoint{3.163156in}{0.712761in}}%
\pgfpathlineto{\pgfqpoint{3.164021in}{0.808146in}}%
\pgfpathlineto{\pgfqpoint{3.165751in}{0.718437in}}%
\pgfpathlineto{\pgfqpoint{3.166615in}{0.767559in}}%
\pgfpathlineto{\pgfqpoint{3.167480in}{0.672320in}}%
\pgfpathlineto{\pgfqpoint{3.168345in}{0.734004in}}%
\pgfpathlineto{\pgfqpoint{3.169210in}{0.732136in}}%
\pgfpathlineto{\pgfqpoint{3.170075in}{0.722246in}}%
\pgfpathlineto{\pgfqpoint{3.170938in}{0.686202in}}%
\pgfpathlineto{\pgfqpoint{3.171802in}{0.743968in}}%
\pgfpathlineto{\pgfqpoint{3.173530in}{0.650085in}}%
\pgfpathlineto{\pgfqpoint{3.174393in}{0.762467in}}%
\pgfpathlineto{\pgfqpoint{3.175258in}{0.686352in}}%
\pgfpathlineto{\pgfqpoint{3.176124in}{0.712103in}}%
\pgfpathlineto{\pgfqpoint{3.176989in}{0.695399in}}%
\pgfpathlineto{\pgfqpoint{3.177853in}{0.707523in}}%
\pgfpathlineto{\pgfqpoint{3.178719in}{0.664922in}}%
\pgfpathlineto{\pgfqpoint{3.180447in}{0.750891in}}%
\pgfpathlineto{\pgfqpoint{3.181311in}{0.711989in}}%
\pgfpathlineto{\pgfqpoint{3.183039in}{0.768509in}}%
\pgfpathlineto{\pgfqpoint{3.183904in}{0.646569in}}%
\pgfpathlineto{\pgfqpoint{3.184769in}{0.655835in}}%
\pgfpathlineto{\pgfqpoint{3.185635in}{0.720232in}}%
\pgfpathlineto{\pgfqpoint{3.186500in}{0.711331in}}%
\pgfpathlineto{\pgfqpoint{3.188230in}{0.715838in}}%
\pgfpathlineto{\pgfqpoint{3.189095in}{0.682138in}}%
\pgfpathlineto{\pgfqpoint{3.189960in}{0.697815in}}%
\pgfpathlineto{\pgfqpoint{3.190825in}{0.769832in}}%
\pgfpathlineto{\pgfqpoint{3.192553in}{0.722871in}}%
\pgfpathlineto{\pgfqpoint{3.193417in}{0.743968in}}%
\pgfpathlineto{\pgfqpoint{3.194278in}{0.743603in}}%
\pgfpathlineto{\pgfqpoint{3.195143in}{0.733712in}}%
\pgfpathlineto{\pgfqpoint{3.196006in}{0.778254in}}%
\pgfpathlineto{\pgfqpoint{3.196869in}{0.728511in}}%
\pgfpathlineto{\pgfqpoint{3.197734in}{0.735507in}}%
\pgfpathlineto{\pgfqpoint{3.198598in}{0.755873in}}%
\pgfpathlineto{\pgfqpoint{3.199463in}{0.644701in}}%
\pgfpathlineto{\pgfqpoint{3.200327in}{0.658035in}}%
\pgfpathlineto{\pgfqpoint{3.201193in}{0.714628in}}%
\pgfpathlineto{\pgfqpoint{3.202922in}{0.685581in}}%
\pgfpathlineto{\pgfqpoint{3.205515in}{0.784555in}}%
\pgfpathlineto{\pgfqpoint{3.206381in}{0.685361in}}%
\pgfpathlineto{\pgfqpoint{3.207245in}{0.758472in}}%
\pgfpathlineto{\pgfqpoint{3.208975in}{0.718802in}}%
\pgfpathlineto{\pgfqpoint{3.209840in}{0.762906in}}%
\pgfpathlineto{\pgfqpoint{3.210707in}{0.706093in}}%
\pgfpathlineto{\pgfqpoint{3.211573in}{0.757960in}}%
\pgfpathlineto{\pgfqpoint{3.212439in}{0.757595in}}%
\pgfpathlineto{\pgfqpoint{3.213304in}{0.758106in}}%
\pgfpathlineto{\pgfqpoint{3.214169in}{0.788071in}}%
\pgfpathlineto{\pgfqpoint{3.215896in}{0.721734in}}%
\pgfpathlineto{\pgfqpoint{3.216762in}{0.742320in}}%
\pgfpathlineto{\pgfqpoint{3.217627in}{0.734849in}}%
\pgfpathlineto{\pgfqpoint{3.218492in}{0.694079in}}%
\pgfpathlineto{\pgfqpoint{3.220222in}{0.742174in}}%
\pgfpathlineto{\pgfqpoint{3.221087in}{0.750928in}}%
\pgfpathlineto{\pgfqpoint{3.222816in}{0.716131in}}%
\pgfpathlineto{\pgfqpoint{3.224547in}{0.774080in}}%
\pgfpathlineto{\pgfqpoint{3.225412in}{0.685508in}}%
\pgfpathlineto{\pgfqpoint{3.226276in}{0.686239in}}%
\pgfpathlineto{\pgfqpoint{3.227141in}{0.760745in}}%
\pgfpathlineto{\pgfqpoint{3.228006in}{0.745361in}}%
\pgfpathlineto{\pgfqpoint{3.228872in}{0.702723in}}%
\pgfpathlineto{\pgfqpoint{3.229739in}{0.794665in}}%
\pgfpathlineto{\pgfqpoint{3.230605in}{0.742320in}}%
\pgfpathlineto{\pgfqpoint{3.231470in}{0.787081in}}%
\pgfpathlineto{\pgfqpoint{3.232335in}{0.677594in}}%
\pgfpathlineto{\pgfqpoint{3.233200in}{0.690892in}}%
\pgfpathlineto{\pgfqpoint{3.234064in}{0.687887in}}%
\pgfpathlineto{\pgfqpoint{3.235794in}{0.796460in}}%
\pgfpathlineto{\pgfqpoint{3.237525in}{0.702833in}}%
\pgfpathlineto{\pgfqpoint{3.239254in}{0.740379in}}%
\pgfpathlineto{\pgfqpoint{3.242713in}{0.594410in}}%
\pgfpathlineto{\pgfqpoint{3.243578in}{0.725396in}}%
\pgfpathlineto{\pgfqpoint{3.244443in}{0.713345in}}%
\pgfpathlineto{\pgfqpoint{3.245308in}{0.691184in}}%
\pgfpathlineto{\pgfqpoint{3.246173in}{0.750197in}}%
\pgfpathlineto{\pgfqpoint{3.247039in}{0.653200in}}%
\pgfpathlineto{\pgfqpoint{3.247901in}{0.758366in}}%
\pgfpathlineto{\pgfqpoint{3.248767in}{0.711185in}}%
\pgfpathlineto{\pgfqpoint{3.250497in}{0.743895in}}%
\pgfpathlineto{\pgfqpoint{3.251361in}{0.714628in}}%
\pgfpathlineto{\pgfqpoint{3.252226in}{0.646277in}}%
\pgfpathlineto{\pgfqpoint{3.253089in}{0.803785in}}%
\pgfpathlineto{\pgfqpoint{3.253954in}{0.686681in}}%
\pgfpathlineto{\pgfqpoint{3.254819in}{0.703641in}}%
\pgfpathlineto{\pgfqpoint{3.255685in}{0.635655in}}%
\pgfpathlineto{\pgfqpoint{3.256550in}{0.726277in}}%
\pgfpathlineto{\pgfqpoint{3.257415in}{0.691078in}}%
\pgfpathlineto{\pgfqpoint{3.258280in}{0.732981in}}%
\pgfpathlineto{\pgfqpoint{3.259145in}{0.717852in}}%
\pgfpathlineto{\pgfqpoint{3.260010in}{0.746461in}}%
\pgfpathlineto{\pgfqpoint{3.260876in}{0.689207in}}%
\pgfpathlineto{\pgfqpoint{3.262605in}{0.721734in}}%
\pgfpathlineto{\pgfqpoint{3.263469in}{0.692613in}}%
\pgfpathlineto{\pgfqpoint{3.264334in}{0.772577in}}%
\pgfpathlineto{\pgfqpoint{3.266929in}{0.679133in}}%
\pgfpathlineto{\pgfqpoint{3.267794in}{0.784263in}}%
\pgfpathlineto{\pgfqpoint{3.268657in}{0.713678in}}%
\pgfpathlineto{\pgfqpoint{3.270385in}{0.745361in}}%
\pgfpathlineto{\pgfqpoint{3.272980in}{0.695070in}}%
\pgfpathlineto{\pgfqpoint{3.273846in}{0.794592in}}%
\pgfpathlineto{\pgfqpoint{3.275573in}{0.709317in}}%
\pgfpathlineto{\pgfqpoint{3.276438in}{0.769280in}}%
\pgfpathlineto{\pgfqpoint{3.277300in}{0.764042in}}%
\pgfpathlineto{\pgfqpoint{3.278164in}{0.707815in}}%
\pgfpathlineto{\pgfqpoint{3.279029in}{0.778765in}}%
\pgfpathlineto{\pgfqpoint{3.280759in}{0.668950in}}%
\pgfpathlineto{\pgfqpoint{3.281623in}{0.760819in}}%
\pgfpathlineto{\pgfqpoint{3.282489in}{0.637486in}}%
\pgfpathlineto{\pgfqpoint{3.284218in}{0.734556in}}%
\pgfpathlineto{\pgfqpoint{3.285084in}{0.722249in}}%
\pgfpathlineto{\pgfqpoint{3.286815in}{0.765910in}}%
\pgfpathlineto{\pgfqpoint{3.287681in}{0.700490in}}%
\pgfpathlineto{\pgfqpoint{3.288546in}{0.768988in}}%
\pgfpathlineto{\pgfqpoint{3.289412in}{0.754777in}}%
\pgfpathlineto{\pgfqpoint{3.290277in}{0.711039in}}%
\pgfpathlineto{\pgfqpoint{3.291141in}{0.750855in}}%
\pgfpathlineto{\pgfqpoint{3.292872in}{0.701440in}}%
\pgfpathlineto{\pgfqpoint{3.293738in}{0.676827in}}%
\pgfpathlineto{\pgfqpoint{3.294599in}{0.730013in}}%
\pgfpathlineto{\pgfqpoint{3.295464in}{0.692394in}}%
\pgfpathlineto{\pgfqpoint{3.296325in}{0.819498in}}%
\pgfpathlineto{\pgfqpoint{3.298058in}{0.687595in}}%
\pgfpathlineto{\pgfqpoint{3.298922in}{0.666643in}}%
\pgfpathlineto{\pgfqpoint{3.300649in}{0.746534in}}%
\pgfpathlineto{\pgfqpoint{3.302379in}{0.691809in}}%
\pgfpathlineto{\pgfqpoint{3.303244in}{0.711185in}}%
\pgfpathlineto{\pgfqpoint{3.304108in}{0.665287in}}%
\pgfpathlineto{\pgfqpoint{3.304975in}{0.786569in}}%
\pgfpathlineto{\pgfqpoint{3.305841in}{0.779537in}}%
\pgfpathlineto{\pgfqpoint{3.306706in}{0.656241in}}%
\pgfpathlineto{\pgfqpoint{3.307572in}{0.701627in}}%
\pgfpathlineto{\pgfqpoint{3.308436in}{0.676096in}}%
\pgfpathlineto{\pgfqpoint{3.309302in}{0.717267in}}%
\pgfpathlineto{\pgfqpoint{3.310168in}{0.672433in}}%
\pgfpathlineto{\pgfqpoint{3.313628in}{0.786354in}}%
\pgfpathlineto{\pgfqpoint{3.314493in}{0.671150in}}%
\pgfpathlineto{\pgfqpoint{3.317088in}{0.832540in}}%
\pgfpathlineto{\pgfqpoint{3.317953in}{0.673128in}}%
\pgfpathlineto{\pgfqpoint{3.319683in}{0.762102in}}%
\pgfpathlineto{\pgfqpoint{3.320547in}{0.730598in}}%
\pgfpathlineto{\pgfqpoint{3.321413in}{0.743237in}}%
\pgfpathlineto{\pgfqpoint{3.322275in}{0.786350in}}%
\pgfpathlineto{\pgfqpoint{3.323139in}{0.657926in}}%
\pgfpathlineto{\pgfqpoint{3.324870in}{0.774006in}}%
\pgfpathlineto{\pgfqpoint{3.325734in}{0.750270in}}%
\pgfpathlineto{\pgfqpoint{3.326597in}{0.768070in}}%
\pgfpathlineto{\pgfqpoint{3.327461in}{0.671772in}}%
\pgfpathlineto{\pgfqpoint{3.328325in}{0.675617in}}%
\pgfpathlineto{\pgfqpoint{3.329190in}{0.688074in}}%
\pgfpathlineto{\pgfqpoint{3.330054in}{0.675548in}}%
\pgfpathlineto{\pgfqpoint{3.330917in}{0.686279in}}%
\pgfpathlineto{\pgfqpoint{3.331782in}{0.717779in}}%
\pgfpathlineto{\pgfqpoint{3.332647in}{0.676534in}}%
\pgfpathlineto{\pgfqpoint{3.333513in}{0.706532in}}%
\pgfpathlineto{\pgfqpoint{3.334376in}{0.682284in}}%
\pgfpathlineto{\pgfqpoint{3.336971in}{0.763714in}}%
\pgfpathlineto{\pgfqpoint{3.341293in}{0.672653in}}%
\pgfpathlineto{\pgfqpoint{3.342154in}{0.785254in}}%
\pgfpathlineto{\pgfqpoint{3.343883in}{0.636646in}}%
\pgfpathlineto{\pgfqpoint{3.344746in}{0.704299in}}%
\pgfpathlineto{\pgfqpoint{3.345611in}{0.648291in}}%
\pgfpathlineto{\pgfqpoint{3.346476in}{0.765691in}}%
\pgfpathlineto{\pgfqpoint{3.347340in}{0.738584in}}%
\pgfpathlineto{\pgfqpoint{3.348206in}{0.727191in}}%
\pgfpathlineto{\pgfqpoint{3.349071in}{0.728364in}}%
\pgfpathlineto{\pgfqpoint{3.349936in}{0.685581in}}%
\pgfpathlineto{\pgfqpoint{3.350801in}{0.721661in}}%
\pgfpathlineto{\pgfqpoint{3.351667in}{0.666716in}}%
\pgfpathlineto{\pgfqpoint{3.352533in}{0.743676in}}%
\pgfpathlineto{\pgfqpoint{3.353399in}{0.714921in}}%
\pgfpathlineto{\pgfqpoint{3.354265in}{0.789135in}}%
\pgfpathlineto{\pgfqpoint{3.355130in}{0.662210in}}%
\pgfpathlineto{\pgfqpoint{3.355995in}{0.687522in}}%
\pgfpathlineto{\pgfqpoint{3.356859in}{0.684631in}}%
\pgfpathlineto{\pgfqpoint{3.357724in}{0.675105in}}%
\pgfpathlineto{\pgfqpoint{3.359454in}{0.629280in}}%
\pgfpathlineto{\pgfqpoint{3.360319in}{0.711404in}}%
\pgfpathlineto{\pgfqpoint{3.361184in}{0.678402in}}%
\pgfpathlineto{\pgfqpoint{3.362913in}{0.773235in}}%
\pgfpathlineto{\pgfqpoint{3.365508in}{0.709646in}}%
\pgfpathlineto{\pgfqpoint{3.366374in}{0.702723in}}%
\pgfpathlineto{\pgfqpoint{3.367238in}{0.705216in}}%
\pgfpathlineto{\pgfqpoint{3.368103in}{0.713897in}}%
\pgfpathlineto{\pgfqpoint{3.368968in}{0.698696in}}%
\pgfpathlineto{\pgfqpoint{3.369832in}{0.790085in}}%
\pgfpathlineto{\pgfqpoint{3.370697in}{0.635290in}}%
\pgfpathlineto{\pgfqpoint{3.372429in}{0.748621in}}%
\pgfpathlineto{\pgfqpoint{3.373294in}{0.658255in}}%
\pgfpathlineto{\pgfqpoint{3.375025in}{0.741589in}}%
\pgfpathlineto{\pgfqpoint{3.375889in}{0.705947in}}%
\pgfpathlineto{\pgfqpoint{3.376753in}{0.775947in}}%
\pgfpathlineto{\pgfqpoint{3.377619in}{0.696203in}}%
\pgfpathlineto{\pgfqpoint{3.378486in}{0.697413in}}%
\pgfpathlineto{\pgfqpoint{3.379350in}{0.690819in}}%
\pgfpathlineto{\pgfqpoint{3.380214in}{0.666680in}}%
\pgfpathlineto{\pgfqpoint{3.381080in}{0.732323in}}%
\pgfpathlineto{\pgfqpoint{3.382810in}{0.656939in}}%
\pgfpathlineto{\pgfqpoint{3.383674in}{0.704006in}}%
\pgfpathlineto{\pgfqpoint{3.384537in}{0.680124in}}%
\pgfpathlineto{\pgfqpoint{3.387132in}{0.784044in}}%
\pgfpathlineto{\pgfqpoint{3.388863in}{0.724665in}}%
\pgfpathlineto{\pgfqpoint{3.389728in}{0.756498in}}%
\pgfpathlineto{\pgfqpoint{3.390594in}{0.691005in}}%
\pgfpathlineto{\pgfqpoint{3.391460in}{0.718916in}}%
\pgfpathlineto{\pgfqpoint{3.392325in}{0.713386in}}%
\pgfpathlineto{\pgfqpoint{3.393191in}{0.728295in}}%
\pgfpathlineto{\pgfqpoint{3.394056in}{0.682397in}}%
\pgfpathlineto{\pgfqpoint{3.394921in}{0.737634in}}%
\pgfpathlineto{\pgfqpoint{3.395786in}{0.706020in}}%
\pgfpathlineto{\pgfqpoint{3.396649in}{0.717304in}}%
\pgfpathlineto{\pgfqpoint{3.397513in}{0.751114in}}%
\pgfpathlineto{\pgfqpoint{3.398379in}{0.684484in}}%
\pgfpathlineto{\pgfqpoint{3.399244in}{0.821334in}}%
\pgfpathlineto{\pgfqpoint{3.402704in}{0.681991in}}%
\pgfpathlineto{\pgfqpoint{3.404434in}{0.715400in}}%
\pgfpathlineto{\pgfqpoint{3.406164in}{0.666022in}}%
\pgfpathlineto{\pgfqpoint{3.407893in}{0.698696in}}%
\pgfpathlineto{\pgfqpoint{3.409621in}{0.727670in}}%
\pgfpathlineto{\pgfqpoint{3.410487in}{0.631554in}}%
\pgfpathlineto{\pgfqpoint{3.411351in}{0.747306in}}%
\pgfpathlineto{\pgfqpoint{3.412214in}{0.729172in}}%
\pgfpathlineto{\pgfqpoint{3.413080in}{0.665730in}}%
\pgfpathlineto{\pgfqpoint{3.413944in}{0.752544in}}%
\pgfpathlineto{\pgfqpoint{3.415676in}{0.686279in}}%
\pgfpathlineto{\pgfqpoint{3.418269in}{0.778075in}}%
\pgfpathlineto{\pgfqpoint{3.419134in}{0.752982in}}%
\pgfpathlineto{\pgfqpoint{3.419999in}{0.778294in}}%
\pgfpathlineto{\pgfqpoint{3.420865in}{0.658515in}}%
\pgfpathlineto{\pgfqpoint{3.421731in}{0.664045in}}%
\pgfpathlineto{\pgfqpoint{3.422597in}{0.763421in}}%
\pgfpathlineto{\pgfqpoint{3.423462in}{0.671296in}}%
\pgfpathlineto{\pgfqpoint{3.424327in}{0.692873in}}%
\pgfpathlineto{\pgfqpoint{3.425191in}{0.667195in}}%
\pgfpathlineto{\pgfqpoint{3.426054in}{0.728807in}}%
\pgfpathlineto{\pgfqpoint{3.427782in}{0.671004in}}%
\pgfpathlineto{\pgfqpoint{3.428646in}{0.804191in}}%
\pgfpathlineto{\pgfqpoint{3.430376in}{0.671516in}}%
\pgfpathlineto{\pgfqpoint{3.432107in}{0.768915in}}%
\pgfpathlineto{\pgfqpoint{3.433836in}{0.742360in}}%
\pgfpathlineto{\pgfqpoint{3.434701in}{0.673972in}}%
\pgfpathlineto{\pgfqpoint{3.435566in}{0.822763in}}%
\pgfpathlineto{\pgfqpoint{3.437295in}{0.672653in}}%
\pgfpathlineto{\pgfqpoint{3.438159in}{0.756425in}}%
\pgfpathlineto{\pgfqpoint{3.439025in}{0.677160in}}%
\pgfpathlineto{\pgfqpoint{3.439889in}{0.691444in}}%
\pgfpathlineto{\pgfqpoint{3.440753in}{0.739136in}}%
\pgfpathlineto{\pgfqpoint{3.441619in}{0.718952in}}%
\pgfpathlineto{\pgfqpoint{3.442483in}{0.744740in}}%
\pgfpathlineto{\pgfqpoint{3.443347in}{0.720711in}}%
\pgfpathlineto{\pgfqpoint{3.444212in}{0.736351in}}%
\pgfpathlineto{\pgfqpoint{3.445076in}{0.706719in}}%
\pgfpathlineto{\pgfqpoint{3.445941in}{0.724958in}}%
\pgfpathlineto{\pgfqpoint{3.446805in}{0.655766in}}%
\pgfpathlineto{\pgfqpoint{3.447669in}{0.742726in}}%
\pgfpathlineto{\pgfqpoint{3.448533in}{0.669063in}}%
\pgfpathlineto{\pgfqpoint{3.450259in}{0.761298in}}%
\pgfpathlineto{\pgfqpoint{3.451988in}{0.702910in}}%
\pgfpathlineto{\pgfqpoint{3.452853in}{0.699979in}}%
\pgfpathlineto{\pgfqpoint{3.453718in}{0.682580in}}%
\pgfpathlineto{\pgfqpoint{3.454582in}{0.696462in}}%
\pgfpathlineto{\pgfqpoint{3.455444in}{0.691078in}}%
\pgfpathlineto{\pgfqpoint{3.456310in}{0.718331in}}%
\pgfpathlineto{\pgfqpoint{3.457175in}{0.718039in}}%
\pgfpathlineto{\pgfqpoint{3.458905in}{0.745584in}}%
\pgfpathlineto{\pgfqpoint{3.460636in}{0.675292in}}%
\pgfpathlineto{\pgfqpoint{3.462364in}{0.800528in}}%
\pgfpathlineto{\pgfqpoint{3.463229in}{0.731629in}}%
\pgfpathlineto{\pgfqpoint{3.464094in}{0.740639in}}%
\pgfpathlineto{\pgfqpoint{3.464958in}{0.737634in}}%
\pgfpathlineto{\pgfqpoint{3.465821in}{0.583679in}}%
\pgfpathlineto{\pgfqpoint{3.467552in}{0.726574in}}%
\pgfpathlineto{\pgfqpoint{3.468416in}{0.718624in}}%
\pgfpathlineto{\pgfqpoint{3.469282in}{0.756206in}}%
\pgfpathlineto{\pgfqpoint{3.471877in}{0.711445in}}%
\pgfpathlineto{\pgfqpoint{3.472743in}{0.718404in}}%
\pgfpathlineto{\pgfqpoint{3.473609in}{0.696097in}}%
\pgfpathlineto{\pgfqpoint{3.474471in}{0.711445in}}%
\pgfpathlineto{\pgfqpoint{3.475336in}{0.685804in}}%
\pgfpathlineto{\pgfqpoint{3.476201in}{0.758626in}}%
\pgfpathlineto{\pgfqpoint{3.477067in}{0.743570in}}%
\pgfpathlineto{\pgfqpoint{3.477933in}{0.741995in}}%
\pgfpathlineto{\pgfqpoint{3.478798in}{0.750676in}}%
\pgfpathlineto{\pgfqpoint{3.480525in}{0.711445in}}%
\pgfpathlineto{\pgfqpoint{3.481389in}{0.767161in}}%
\pgfpathlineto{\pgfqpoint{3.482252in}{0.697416in}}%
\pgfpathlineto{\pgfqpoint{3.483118in}{0.809136in}}%
\pgfpathlineto{\pgfqpoint{3.484847in}{0.722323in}}%
\pgfpathlineto{\pgfqpoint{3.486575in}{0.760453in}}%
\pgfpathlineto{\pgfqpoint{3.488304in}{0.711185in}}%
\pgfpathlineto{\pgfqpoint{3.489170in}{0.717706in}}%
\pgfpathlineto{\pgfqpoint{3.490036in}{0.782176in}}%
\pgfpathlineto{\pgfqpoint{3.490901in}{0.762029in}}%
\pgfpathlineto{\pgfqpoint{3.491766in}{0.707340in}}%
\pgfpathlineto{\pgfqpoint{3.492631in}{0.712687in}}%
\pgfpathlineto{\pgfqpoint{3.493495in}{0.747411in}}%
\pgfpathlineto{\pgfqpoint{3.494360in}{0.621001in}}%
\pgfpathlineto{\pgfqpoint{3.496091in}{0.683348in}}%
\pgfpathlineto{\pgfqpoint{3.497819in}{0.676680in}}%
\pgfpathlineto{\pgfqpoint{3.498684in}{0.750599in}}%
\pgfpathlineto{\pgfqpoint{3.499548in}{0.718145in}}%
\pgfpathlineto{\pgfqpoint{3.500413in}{0.742174in}}%
\pgfpathlineto{\pgfqpoint{3.502142in}{0.680051in}}%
\pgfpathlineto{\pgfqpoint{3.503006in}{0.777011in}}%
\pgfpathlineto{\pgfqpoint{3.503870in}{0.715692in}}%
\pgfpathlineto{\pgfqpoint{3.504734in}{0.765033in}}%
\pgfpathlineto{\pgfqpoint{3.505598in}{0.711551in}}%
\pgfpathlineto{\pgfqpoint{3.507329in}{0.743457in}}%
\pgfpathlineto{\pgfqpoint{3.508192in}{0.682320in}}%
\pgfpathlineto{\pgfqpoint{3.509057in}{0.774737in}}%
\pgfpathlineto{\pgfqpoint{3.509923in}{0.769280in}}%
\pgfpathlineto{\pgfqpoint{3.511653in}{0.706240in}}%
\pgfpathlineto{\pgfqpoint{3.513384in}{0.741589in}}%
\pgfpathlineto{\pgfqpoint{3.515979in}{0.659392in}}%
\pgfpathlineto{\pgfqpoint{3.516844in}{0.662835in}}%
\pgfpathlineto{\pgfqpoint{3.517709in}{0.657633in}}%
\pgfpathlineto{\pgfqpoint{3.519437in}{0.724081in}}%
\pgfpathlineto{\pgfqpoint{3.521166in}{0.678037in}}%
\pgfpathlineto{\pgfqpoint{3.522897in}{0.765216in}}%
\pgfpathlineto{\pgfqpoint{3.523762in}{0.713532in}}%
\pgfpathlineto{\pgfqpoint{3.524627in}{0.743091in}}%
\pgfpathlineto{\pgfqpoint{3.526355in}{0.672141in}}%
\pgfpathlineto{\pgfqpoint{3.527220in}{0.743607in}}%
\pgfpathlineto{\pgfqpoint{3.528086in}{0.684411in}}%
\pgfpathlineto{\pgfqpoint{3.528952in}{0.706865in}}%
\pgfpathlineto{\pgfqpoint{3.529817in}{0.702431in}}%
\pgfpathlineto{\pgfqpoint{3.530682in}{0.688220in}}%
\pgfpathlineto{\pgfqpoint{3.531546in}{0.740013in}}%
\pgfpathlineto{\pgfqpoint{3.532412in}{0.679722in}}%
\pgfpathlineto{\pgfqpoint{3.534143in}{0.740598in}}%
\pgfpathlineto{\pgfqpoint{3.535870in}{0.696129in}}%
\pgfpathlineto{\pgfqpoint{3.536735in}{0.596278in}}%
\pgfpathlineto{\pgfqpoint{3.540196in}{0.778473in}}%
\pgfpathlineto{\pgfqpoint{3.541923in}{0.655141in}}%
\pgfpathlineto{\pgfqpoint{3.542787in}{0.754956in}}%
\pgfpathlineto{\pgfqpoint{3.543652in}{0.694408in}}%
\pgfpathlineto{\pgfqpoint{3.545379in}{0.762832in}}%
\pgfpathlineto{\pgfqpoint{3.546245in}{0.714336in}}%
\pgfpathlineto{\pgfqpoint{3.547110in}{0.796899in}}%
\pgfpathlineto{\pgfqpoint{3.547975in}{0.792319in}}%
\pgfpathlineto{\pgfqpoint{3.550568in}{0.670087in}}%
\pgfpathlineto{\pgfqpoint{3.552299in}{0.739315in}}%
\pgfpathlineto{\pgfqpoint{3.553165in}{0.692613in}}%
\pgfpathlineto{\pgfqpoint{3.554030in}{0.737447in}}%
\pgfpathlineto{\pgfqpoint{3.554896in}{0.691842in}}%
\pgfpathlineto{\pgfqpoint{3.555761in}{0.738621in}}%
\pgfpathlineto{\pgfqpoint{3.556626in}{0.702212in}}%
\pgfpathlineto{\pgfqpoint{3.557492in}{0.718364in}}%
\pgfpathlineto{\pgfqpoint{3.558357in}{0.714921in}}%
\pgfpathlineto{\pgfqpoint{3.559222in}{0.714774in}}%
\pgfpathlineto{\pgfqpoint{3.560951in}{0.657597in}}%
\pgfpathlineto{\pgfqpoint{3.561816in}{0.724738in}}%
\pgfpathlineto{\pgfqpoint{3.562682in}{0.707669in}}%
\pgfpathlineto{\pgfqpoint{3.563548in}{0.630929in}}%
\pgfpathlineto{\pgfqpoint{3.564412in}{0.682942in}}%
\pgfpathlineto{\pgfqpoint{3.565276in}{0.665178in}}%
\pgfpathlineto{\pgfqpoint{3.567003in}{0.745836in}}%
\pgfpathlineto{\pgfqpoint{3.567868in}{0.694664in}}%
\pgfpathlineto{\pgfqpoint{3.568733in}{0.715067in}}%
\pgfpathlineto{\pgfqpoint{3.569599in}{0.671662in}}%
\pgfpathlineto{\pgfqpoint{3.571329in}{0.721734in}}%
\pgfpathlineto{\pgfqpoint{3.572193in}{0.633860in}}%
\pgfpathlineto{\pgfqpoint{3.573058in}{0.780820in}}%
\pgfpathlineto{\pgfqpoint{3.573923in}{0.636678in}}%
\pgfpathlineto{\pgfqpoint{3.575651in}{0.744082in}}%
\pgfpathlineto{\pgfqpoint{3.576516in}{0.640674in}}%
\pgfpathlineto{\pgfqpoint{3.577381in}{0.732615in}}%
\pgfpathlineto{\pgfqpoint{3.579112in}{0.691882in}}%
\pgfpathlineto{\pgfqpoint{3.579978in}{0.683567in}}%
\pgfpathlineto{\pgfqpoint{3.580842in}{0.687635in}}%
\pgfpathlineto{\pgfqpoint{3.581708in}{0.746461in}}%
\pgfpathlineto{\pgfqpoint{3.582573in}{0.659392in}}%
\pgfpathlineto{\pgfqpoint{3.583437in}{0.740087in}}%
\pgfpathlineto{\pgfqpoint{3.584300in}{0.665145in}}%
\pgfpathlineto{\pgfqpoint{3.585164in}{0.781957in}}%
\pgfpathlineto{\pgfqpoint{3.586028in}{0.679393in}}%
\pgfpathlineto{\pgfqpoint{3.586892in}{0.753973in}}%
\pgfpathlineto{\pgfqpoint{3.587756in}{0.687343in}}%
\pgfpathlineto{\pgfqpoint{3.588620in}{0.774047in}}%
\pgfpathlineto{\pgfqpoint{3.589485in}{0.762252in}}%
\pgfpathlineto{\pgfqpoint{3.591214in}{0.760786in}}%
\pgfpathlineto{\pgfqpoint{3.592942in}{0.710349in}}%
\pgfpathlineto{\pgfqpoint{3.593804in}{0.750237in}}%
\pgfpathlineto{\pgfqpoint{3.594668in}{0.703828in}}%
\pgfpathlineto{\pgfqpoint{3.595532in}{0.750822in}}%
\pgfpathlineto{\pgfqpoint{3.596397in}{0.736355in}}%
\pgfpathlineto{\pgfqpoint{3.597262in}{0.693864in}}%
\pgfpathlineto{\pgfqpoint{3.598992in}{0.714852in}}%
\pgfpathlineto{\pgfqpoint{3.599857in}{0.711664in}}%
\pgfpathlineto{\pgfqpoint{3.600722in}{0.683900in}}%
\pgfpathlineto{\pgfqpoint{3.603316in}{0.811187in}}%
\pgfpathlineto{\pgfqpoint{3.605045in}{0.732835in}}%
\pgfpathlineto{\pgfqpoint{3.607639in}{0.697047in}}%
\pgfpathlineto{\pgfqpoint{3.610234in}{0.760380in}}%
\pgfpathlineto{\pgfqpoint{3.611099in}{0.741808in}}%
\pgfpathlineto{\pgfqpoint{3.612827in}{0.707779in}}%
\pgfpathlineto{\pgfqpoint{3.613692in}{0.808584in}}%
\pgfpathlineto{\pgfqpoint{3.614557in}{0.801698in}}%
\pgfpathlineto{\pgfqpoint{3.615422in}{0.682028in}}%
\pgfpathlineto{\pgfqpoint{3.616289in}{0.743310in}}%
\pgfpathlineto{\pgfqpoint{3.618017in}{0.677890in}}%
\pgfpathlineto{\pgfqpoint{3.618881in}{0.786496in}}%
\pgfpathlineto{\pgfqpoint{3.619745in}{0.755654in}}%
\pgfpathlineto{\pgfqpoint{3.621474in}{0.686206in}}%
\pgfpathlineto{\pgfqpoint{3.622340in}{0.692617in}}%
\pgfpathlineto{\pgfqpoint{3.623205in}{0.754631in}}%
\pgfpathlineto{\pgfqpoint{3.624067in}{0.702066in}}%
\pgfpathlineto{\pgfqpoint{3.624932in}{0.713751in}}%
\pgfpathlineto{\pgfqpoint{3.625798in}{0.673863in}}%
\pgfpathlineto{\pgfqpoint{3.626663in}{0.680603in}}%
\pgfpathlineto{\pgfqpoint{3.628394in}{0.729724in}}%
\pgfpathlineto{\pgfqpoint{3.629258in}{0.814228in}}%
\pgfpathlineto{\pgfqpoint{3.630123in}{0.763644in}}%
\pgfpathlineto{\pgfqpoint{3.630987in}{0.792432in}}%
\pgfpathlineto{\pgfqpoint{3.632716in}{0.698696in}}%
\pgfpathlineto{\pgfqpoint{3.633580in}{0.701627in}}%
\pgfpathlineto{\pgfqpoint{3.634446in}{0.722359in}}%
\pgfpathlineto{\pgfqpoint{3.635312in}{0.714961in}}%
\pgfpathlineto{\pgfqpoint{3.637043in}{0.756243in}}%
\pgfpathlineto{\pgfqpoint{3.638773in}{0.704997in}}%
\pgfpathlineto{\pgfqpoint{3.639640in}{0.610091in}}%
\pgfpathlineto{\pgfqpoint{3.641372in}{0.751334in}}%
\pgfpathlineto{\pgfqpoint{3.642235in}{0.720345in}}%
\pgfpathlineto{\pgfqpoint{3.643962in}{0.768001in}}%
\pgfpathlineto{\pgfqpoint{3.644829in}{0.767892in}}%
\pgfpathlineto{\pgfqpoint{3.645693in}{0.691078in}}%
\pgfpathlineto{\pgfqpoint{3.647418in}{0.752690in}}%
\pgfpathlineto{\pgfqpoint{3.648283in}{0.739794in}}%
\pgfpathlineto{\pgfqpoint{3.649149in}{0.706093in}}%
\pgfpathlineto{\pgfqpoint{3.650016in}{0.735986in}}%
\pgfpathlineto{\pgfqpoint{3.651747in}{0.719354in}}%
\pgfpathlineto{\pgfqpoint{3.652611in}{0.739867in}}%
\pgfpathlineto{\pgfqpoint{3.653476in}{0.726204in}}%
\pgfpathlineto{\pgfqpoint{3.654343in}{0.755215in}}%
\pgfpathlineto{\pgfqpoint{3.655208in}{0.742100in}}%
\pgfpathlineto{\pgfqpoint{3.656074in}{0.680599in}}%
\pgfpathlineto{\pgfqpoint{3.656940in}{0.681553in}}%
\pgfpathlineto{\pgfqpoint{3.657803in}{0.746754in}}%
\pgfpathlineto{\pgfqpoint{3.658669in}{0.741516in}}%
\pgfpathlineto{\pgfqpoint{3.659533in}{0.738146in}}%
\pgfpathlineto{\pgfqpoint{3.661264in}{0.702650in}}%
\pgfpathlineto{\pgfqpoint{3.662127in}{0.777596in}}%
\pgfpathlineto{\pgfqpoint{3.662989in}{0.696020in}}%
\pgfpathlineto{\pgfqpoint{3.663855in}{0.716496in}}%
\pgfpathlineto{\pgfqpoint{3.664721in}{0.722651in}}%
\pgfpathlineto{\pgfqpoint{3.665587in}{0.745032in}}%
\pgfpathlineto{\pgfqpoint{3.666450in}{0.682284in}}%
\pgfpathlineto{\pgfqpoint{3.667313in}{0.714555in}}%
\pgfpathlineto{\pgfqpoint{3.668179in}{0.649135in}}%
\pgfpathlineto{\pgfqpoint{3.669045in}{0.734703in}}%
\pgfpathlineto{\pgfqpoint{3.670775in}{0.637303in}}%
\pgfpathlineto{\pgfqpoint{3.671641in}{0.681187in}}%
\pgfpathlineto{\pgfqpoint{3.672504in}{0.665730in}}%
\pgfpathlineto{\pgfqpoint{3.674235in}{0.727962in}}%
\pgfpathlineto{\pgfqpoint{3.675101in}{0.611845in}}%
\pgfpathlineto{\pgfqpoint{3.676831in}{0.750822in}}%
\pgfpathlineto{\pgfqpoint{3.678563in}{0.745032in}}%
\pgfpathlineto{\pgfqpoint{3.679428in}{0.729538in}}%
\pgfpathlineto{\pgfqpoint{3.680294in}{0.732063in}}%
\pgfpathlineto{\pgfqpoint{3.681160in}{0.661479in}}%
\pgfpathlineto{\pgfqpoint{3.682891in}{0.746973in}}%
\pgfpathlineto{\pgfqpoint{3.683756in}{0.688732in}}%
\pgfpathlineto{\pgfqpoint{3.685485in}{0.742872in}}%
\pgfpathlineto{\pgfqpoint{3.687215in}{0.628622in}}%
\pgfpathlineto{\pgfqpoint{3.688079in}{0.735766in}}%
\pgfpathlineto{\pgfqpoint{3.688943in}{0.689649in}}%
\pgfpathlineto{\pgfqpoint{3.689808in}{0.695179in}}%
\pgfpathlineto{\pgfqpoint{3.692401in}{0.763677in}}%
\pgfpathlineto{\pgfqpoint{3.693266in}{0.671516in}}%
\pgfpathlineto{\pgfqpoint{3.694131in}{0.710710in}}%
\pgfpathlineto{\pgfqpoint{3.694996in}{0.703349in}}%
\pgfpathlineto{\pgfqpoint{3.695860in}{0.710308in}}%
\pgfpathlineto{\pgfqpoint{3.697589in}{0.782761in}}%
\pgfpathlineto{\pgfqpoint{3.698452in}{0.692650in}}%
\pgfpathlineto{\pgfqpoint{3.700181in}{0.762540in}}%
\pgfpathlineto{\pgfqpoint{3.701046in}{0.725839in}}%
\pgfpathlineto{\pgfqpoint{3.701911in}{0.785546in}}%
\pgfpathlineto{\pgfqpoint{3.704508in}{0.676242in}}%
\pgfpathlineto{\pgfqpoint{3.705373in}{0.763823in}}%
\pgfpathlineto{\pgfqpoint{3.706238in}{0.757083in}}%
\pgfpathlineto{\pgfqpoint{3.707967in}{0.666790in}}%
\pgfpathlineto{\pgfqpoint{3.708833in}{0.669794in}}%
\pgfpathlineto{\pgfqpoint{3.709698in}{0.704080in}}%
\pgfpathlineto{\pgfqpoint{3.710563in}{0.693677in}}%
\pgfpathlineto{\pgfqpoint{3.711428in}{0.737269in}}%
\pgfpathlineto{\pgfqpoint{3.712293in}{0.715911in}}%
\pgfpathlineto{\pgfqpoint{3.713157in}{0.788733in}}%
\pgfpathlineto{\pgfqpoint{3.714886in}{0.671589in}}%
\pgfpathlineto{\pgfqpoint{3.716611in}{0.706459in}}%
\pgfpathlineto{\pgfqpoint{3.717477in}{0.684261in}}%
\pgfpathlineto{\pgfqpoint{3.718343in}{0.749352in}}%
\pgfpathlineto{\pgfqpoint{3.719209in}{0.732356in}}%
\pgfpathlineto{\pgfqpoint{3.720938in}{0.760672in}}%
\pgfpathlineto{\pgfqpoint{3.721802in}{0.690380in}}%
\pgfpathlineto{\pgfqpoint{3.724396in}{0.774591in}}%
\pgfpathlineto{\pgfqpoint{3.726126in}{0.730561in}}%
\pgfpathlineto{\pgfqpoint{3.727855in}{0.733566in}}%
\pgfpathlineto{\pgfqpoint{3.728721in}{0.716350in}}%
\pgfpathlineto{\pgfqpoint{3.729587in}{0.766531in}}%
\pgfpathlineto{\pgfqpoint{3.730452in}{0.703162in}}%
\pgfpathlineto{\pgfqpoint{3.731315in}{0.721442in}}%
\pgfpathlineto{\pgfqpoint{3.732179in}{0.719391in}}%
\pgfpathlineto{\pgfqpoint{3.733044in}{0.678841in}}%
\pgfpathlineto{\pgfqpoint{3.734774in}{0.741735in}}%
\pgfpathlineto{\pgfqpoint{3.735640in}{0.721880in}}%
\pgfpathlineto{\pgfqpoint{3.736506in}{0.646167in}}%
\pgfpathlineto{\pgfqpoint{3.738238in}{0.721222in}}%
\pgfpathlineto{\pgfqpoint{3.739103in}{0.657305in}}%
\pgfpathlineto{\pgfqpoint{3.739967in}{0.762394in}}%
\pgfpathlineto{\pgfqpoint{3.742560in}{0.678402in}}%
\pgfpathlineto{\pgfqpoint{3.744291in}{0.723675in}}%
\pgfpathlineto{\pgfqpoint{3.745156in}{0.654885in}}%
\pgfpathlineto{\pgfqpoint{3.746022in}{0.757266in}}%
\pgfpathlineto{\pgfqpoint{3.746887in}{0.731406in}}%
\pgfpathlineto{\pgfqpoint{3.747753in}{0.687229in}}%
\pgfpathlineto{\pgfqpoint{3.748618in}{0.785729in}}%
\pgfpathlineto{\pgfqpoint{3.750349in}{0.722286in}}%
\pgfpathlineto{\pgfqpoint{3.751213in}{0.730455in}}%
\pgfpathlineto{\pgfqpoint{3.752078in}{0.723788in}}%
\pgfpathlineto{\pgfqpoint{3.752944in}{0.682909in}}%
\pgfpathlineto{\pgfqpoint{3.754676in}{0.739648in}}%
\pgfpathlineto{\pgfqpoint{3.755541in}{0.674411in}}%
\pgfpathlineto{\pgfqpoint{3.756406in}{0.759576in}}%
\pgfpathlineto{\pgfqpoint{3.757272in}{0.695106in}}%
\pgfpathlineto{\pgfqpoint{3.759004in}{0.728109in}}%
\pgfpathlineto{\pgfqpoint{3.760734in}{0.722542in}}%
\pgfpathlineto{\pgfqpoint{3.762463in}{0.678914in}}%
\pgfpathlineto{\pgfqpoint{3.764191in}{0.779317in}}%
\pgfpathlineto{\pgfqpoint{3.765055in}{0.682836in}}%
\pgfpathlineto{\pgfqpoint{3.765921in}{0.795802in}}%
\pgfpathlineto{\pgfqpoint{3.766786in}{0.677525in}}%
\pgfpathlineto{\pgfqpoint{3.768513in}{0.749685in}}%
\pgfpathlineto{\pgfqpoint{3.769378in}{0.710674in}}%
\pgfpathlineto{\pgfqpoint{3.771970in}{0.766203in}}%
\pgfpathlineto{\pgfqpoint{3.772835in}{0.712541in}}%
\pgfpathlineto{\pgfqpoint{3.773700in}{0.799830in}}%
\pgfpathlineto{\pgfqpoint{3.775427in}{0.719866in}}%
\pgfpathlineto{\pgfqpoint{3.776290in}{0.778806in}}%
\pgfpathlineto{\pgfqpoint{3.778019in}{0.728328in}}%
\pgfpathlineto{\pgfqpoint{3.778884in}{0.757156in}}%
\pgfpathlineto{\pgfqpoint{3.780615in}{0.685215in}}%
\pgfpathlineto{\pgfqpoint{3.781479in}{0.736132in}}%
\pgfpathlineto{\pgfqpoint{3.782344in}{0.695399in}}%
\pgfpathlineto{\pgfqpoint{3.783208in}{0.743676in}}%
\pgfpathlineto{\pgfqpoint{3.784074in}{0.682690in}}%
\pgfpathlineto{\pgfqpoint{3.785802in}{0.768549in}}%
\pgfpathlineto{\pgfqpoint{3.786665in}{0.686352in}}%
\pgfpathlineto{\pgfqpoint{3.787530in}{0.783386in}}%
\pgfpathlineto{\pgfqpoint{3.788396in}{0.711957in}}%
\pgfpathlineto{\pgfqpoint{3.789261in}{0.727012in}}%
\pgfpathlineto{\pgfqpoint{3.791853in}{0.658588in}}%
\pgfpathlineto{\pgfqpoint{3.792718in}{0.758220in}}%
\pgfpathlineto{\pgfqpoint{3.793584in}{0.694668in}}%
\pgfpathlineto{\pgfqpoint{3.794449in}{0.740160in}}%
\pgfpathlineto{\pgfqpoint{3.795314in}{0.654154in}}%
\pgfpathlineto{\pgfqpoint{3.797911in}{0.751845in}}%
\pgfpathlineto{\pgfqpoint{3.798777in}{0.648952in}}%
\pgfpathlineto{\pgfqpoint{3.799640in}{0.698330in}}%
\pgfpathlineto{\pgfqpoint{3.800503in}{0.668844in}}%
\pgfpathlineto{\pgfqpoint{3.801369in}{0.674740in}}%
\pgfpathlineto{\pgfqpoint{3.802234in}{0.695472in}}%
\pgfpathlineto{\pgfqpoint{3.803099in}{0.751037in}}%
\pgfpathlineto{\pgfqpoint{3.803964in}{0.696755in}}%
\pgfpathlineto{\pgfqpoint{3.804830in}{0.760234in}}%
\pgfpathlineto{\pgfqpoint{3.806562in}{0.672287in}}%
\pgfpathlineto{\pgfqpoint{3.808293in}{0.744411in}}%
\pgfpathlineto{\pgfqpoint{3.809157in}{0.695618in}}%
\pgfpathlineto{\pgfqpoint{3.810020in}{0.737009in}}%
\pgfpathlineto{\pgfqpoint{3.810885in}{0.642103in}}%
\pgfpathlineto{\pgfqpoint{3.811751in}{0.706240in}}%
\pgfpathlineto{\pgfqpoint{3.812616in}{0.694444in}}%
\pgfpathlineto{\pgfqpoint{3.813481in}{0.611626in}}%
\pgfpathlineto{\pgfqpoint{3.815209in}{0.726350in}}%
\pgfpathlineto{\pgfqpoint{3.816076in}{0.716423in}}%
\pgfpathlineto{\pgfqpoint{3.816939in}{0.634445in}}%
\pgfpathlineto{\pgfqpoint{3.817803in}{0.722725in}}%
\pgfpathlineto{\pgfqpoint{3.818667in}{0.705728in}}%
\pgfpathlineto{\pgfqpoint{3.819532in}{0.726314in}}%
\pgfpathlineto{\pgfqpoint{3.820398in}{0.705655in}}%
\pgfpathlineto{\pgfqpoint{3.821264in}{0.757229in}}%
\pgfpathlineto{\pgfqpoint{3.822129in}{0.708660in}}%
\pgfpathlineto{\pgfqpoint{3.822992in}{0.719907in}}%
\pgfpathlineto{\pgfqpoint{3.823858in}{0.717048in}}%
\pgfpathlineto{\pgfqpoint{3.824723in}{0.751187in}}%
\pgfpathlineto{\pgfqpoint{3.825589in}{0.740858in}}%
\pgfpathlineto{\pgfqpoint{3.826455in}{0.692873in}}%
\pgfpathlineto{\pgfqpoint{3.828185in}{0.771408in}}%
\pgfpathlineto{\pgfqpoint{3.829917in}{0.672214in}}%
\pgfpathlineto{\pgfqpoint{3.830781in}{0.742580in}}%
\pgfpathlineto{\pgfqpoint{3.831644in}{0.698001in}}%
\pgfpathlineto{\pgfqpoint{3.832509in}{0.723861in}}%
\pgfpathlineto{\pgfqpoint{3.833375in}{0.790930in}}%
\pgfpathlineto{\pgfqpoint{3.834240in}{0.740013in}}%
\pgfpathlineto{\pgfqpoint{3.835971in}{0.769905in}}%
\pgfpathlineto{\pgfqpoint{3.836836in}{0.739575in}}%
\pgfpathlineto{\pgfqpoint{3.837698in}{0.789793in}}%
\pgfpathlineto{\pgfqpoint{3.838562in}{0.652286in}}%
\pgfpathlineto{\pgfqpoint{3.839427in}{0.740379in}}%
\pgfpathlineto{\pgfqpoint{3.840290in}{0.715838in}}%
\pgfpathlineto{\pgfqpoint{3.841155in}{0.744520in}}%
\pgfpathlineto{\pgfqpoint{3.843749in}{0.710272in}}%
\pgfpathlineto{\pgfqpoint{3.844612in}{0.730748in}}%
\pgfpathlineto{\pgfqpoint{3.845478in}{0.672360in}}%
\pgfpathlineto{\pgfqpoint{3.846342in}{0.726976in}}%
\pgfpathlineto{\pgfqpoint{3.848071in}{0.701919in}}%
\pgfpathlineto{\pgfqpoint{3.848933in}{0.746607in}}%
\pgfpathlineto{\pgfqpoint{3.849796in}{0.714263in}}%
\pgfpathlineto{\pgfqpoint{3.850662in}{0.715327in}}%
\pgfpathlineto{\pgfqpoint{3.851528in}{0.691444in}}%
\pgfpathlineto{\pgfqpoint{3.852393in}{0.767599in}}%
\pgfpathlineto{\pgfqpoint{3.853258in}{0.725218in}}%
\pgfpathlineto{\pgfqpoint{3.854122in}{0.758147in}}%
\pgfpathlineto{\pgfqpoint{3.855851in}{0.661519in}}%
\pgfpathlineto{\pgfqpoint{3.856715in}{0.666976in}}%
\pgfpathlineto{\pgfqpoint{3.857580in}{0.747964in}}%
\pgfpathlineto{\pgfqpoint{3.858443in}{0.660090in}}%
\pgfpathlineto{\pgfqpoint{3.859307in}{0.677452in}}%
\pgfpathlineto{\pgfqpoint{3.860171in}{0.734410in}}%
\pgfpathlineto{\pgfqpoint{3.861901in}{0.673863in}}%
\pgfpathlineto{\pgfqpoint{3.862767in}{0.690603in}}%
\pgfpathlineto{\pgfqpoint{3.863631in}{0.756279in}}%
\pgfpathlineto{\pgfqpoint{3.866223in}{0.659319in}}%
\pgfpathlineto{\pgfqpoint{3.867087in}{0.693092in}}%
\pgfpathlineto{\pgfqpoint{3.867951in}{0.676461in}}%
\pgfpathlineto{\pgfqpoint{3.870542in}{0.756425in}}%
\pgfpathlineto{\pgfqpoint{3.872272in}{0.675876in}}%
\pgfpathlineto{\pgfqpoint{3.873137in}{0.732835in}}%
\pgfpathlineto{\pgfqpoint{3.874002in}{0.654300in}}%
\pgfpathlineto{\pgfqpoint{3.874867in}{0.719172in}}%
\pgfpathlineto{\pgfqpoint{3.875730in}{0.694595in}}%
\pgfpathlineto{\pgfqpoint{3.877459in}{0.715400in}}%
\pgfpathlineto{\pgfqpoint{3.878324in}{0.783751in}}%
\pgfpathlineto{\pgfqpoint{3.879189in}{0.689174in}}%
\pgfpathlineto{\pgfqpoint{3.880052in}{0.764887in}}%
\pgfpathlineto{\pgfqpoint{3.880914in}{0.716058in}}%
\pgfpathlineto{\pgfqpoint{3.881778in}{0.720272in}}%
\pgfpathlineto{\pgfqpoint{3.882643in}{0.720491in}}%
\pgfpathlineto{\pgfqpoint{3.883507in}{0.748329in}}%
\pgfpathlineto{\pgfqpoint{3.884372in}{0.647158in}}%
\pgfpathlineto{\pgfqpoint{3.885238in}{0.715619in}}%
\pgfpathlineto{\pgfqpoint{3.886103in}{0.696828in}}%
\pgfpathlineto{\pgfqpoint{3.886970in}{0.709870in}}%
\pgfpathlineto{\pgfqpoint{3.887836in}{0.751114in}}%
\pgfpathlineto{\pgfqpoint{3.888702in}{0.701042in}}%
\pgfpathlineto{\pgfqpoint{3.890434in}{0.728109in}}%
\pgfpathlineto{\pgfqpoint{3.891300in}{0.685325in}}%
\pgfpathlineto{\pgfqpoint{3.892166in}{0.690088in}}%
\pgfpathlineto{\pgfqpoint{3.893031in}{0.736205in}}%
\pgfpathlineto{\pgfqpoint{3.893896in}{0.653057in}}%
\pgfpathlineto{\pgfqpoint{3.895627in}{0.760161in}}%
\pgfpathlineto{\pgfqpoint{3.896493in}{0.736863in}}%
\pgfpathlineto{\pgfqpoint{3.897360in}{0.619576in}}%
\pgfpathlineto{\pgfqpoint{3.899091in}{0.711737in}}%
\pgfpathlineto{\pgfqpoint{3.899958in}{0.711591in}}%
\pgfpathlineto{\pgfqpoint{3.900822in}{0.707084in}}%
\pgfpathlineto{\pgfqpoint{3.901689in}{0.684777in}}%
\pgfpathlineto{\pgfqpoint{3.902555in}{0.739648in}}%
\pgfpathlineto{\pgfqpoint{3.903419in}{0.676315in}}%
\pgfpathlineto{\pgfqpoint{3.904283in}{0.739502in}}%
\pgfpathlineto{\pgfqpoint{3.905147in}{0.688732in}}%
\pgfpathlineto{\pgfqpoint{3.906011in}{0.692873in}}%
\pgfpathlineto{\pgfqpoint{3.906875in}{0.684777in}}%
\pgfpathlineto{\pgfqpoint{3.907739in}{0.750456in}}%
\pgfpathlineto{\pgfqpoint{3.910332in}{0.654925in}}%
\pgfpathlineto{\pgfqpoint{3.911197in}{0.658515in}}%
\pgfpathlineto{\pgfqpoint{3.912061in}{0.709870in}}%
\pgfpathlineto{\pgfqpoint{3.912926in}{0.682324in}}%
\pgfpathlineto{\pgfqpoint{3.913792in}{0.701115in}}%
\pgfpathlineto{\pgfqpoint{3.914657in}{0.697599in}}%
\pgfpathlineto{\pgfqpoint{3.916387in}{0.701846in}}%
\pgfpathlineto{\pgfqpoint{3.917253in}{0.667232in}}%
\pgfpathlineto{\pgfqpoint{3.918984in}{0.733054in}}%
\pgfpathlineto{\pgfqpoint{3.919849in}{0.727487in}}%
\pgfpathlineto{\pgfqpoint{3.920712in}{0.705363in}}%
\pgfpathlineto{\pgfqpoint{3.921578in}{0.753201in}}%
\pgfpathlineto{\pgfqpoint{3.922443in}{0.659172in}}%
\pgfpathlineto{\pgfqpoint{3.923309in}{0.706865in}}%
\pgfpathlineto{\pgfqpoint{3.925036in}{0.660675in}}%
\pgfpathlineto{\pgfqpoint{3.926765in}{0.747086in}}%
\pgfpathlineto{\pgfqpoint{3.927629in}{0.702983in}}%
\pgfpathlineto{\pgfqpoint{3.928493in}{0.766389in}}%
\pgfpathlineto{\pgfqpoint{3.929358in}{0.647673in}}%
\pgfpathlineto{\pgfqpoint{3.930224in}{0.720418in}}%
\pgfpathlineto{\pgfqpoint{3.931951in}{0.663716in}}%
\pgfpathlineto{\pgfqpoint{3.932817in}{0.759503in}}%
\pgfpathlineto{\pgfqpoint{3.933682in}{0.737195in}}%
\pgfpathlineto{\pgfqpoint{3.934547in}{0.738625in}}%
\pgfpathlineto{\pgfqpoint{3.935411in}{0.737342in}}%
\pgfpathlineto{\pgfqpoint{3.936274in}{0.752434in}}%
\pgfpathlineto{\pgfqpoint{3.937139in}{0.744082in}}%
\pgfpathlineto{\pgfqpoint{3.938004in}{0.699613in}}%
\pgfpathlineto{\pgfqpoint{3.938870in}{0.740237in}}%
\pgfpathlineto{\pgfqpoint{3.939731in}{0.705915in}}%
\pgfpathlineto{\pgfqpoint{3.940595in}{0.738990in}}%
\pgfpathlineto{\pgfqpoint{3.941461in}{0.717820in}}%
\pgfpathlineto{\pgfqpoint{3.943191in}{0.788039in}}%
\pgfpathlineto{\pgfqpoint{3.944057in}{0.787747in}}%
\pgfpathlineto{\pgfqpoint{3.944922in}{0.764010in}}%
\pgfpathlineto{\pgfqpoint{3.945788in}{0.659944in}}%
\pgfpathlineto{\pgfqpoint{3.946653in}{0.673351in}}%
\pgfpathlineto{\pgfqpoint{3.947519in}{0.701335in}}%
\pgfpathlineto{\pgfqpoint{3.948385in}{0.700421in}}%
\pgfpathlineto{\pgfqpoint{3.949249in}{0.711079in}}%
\pgfpathlineto{\pgfqpoint{3.950114in}{0.744926in}}%
\pgfpathlineto{\pgfqpoint{3.950979in}{0.709102in}}%
\pgfpathlineto{\pgfqpoint{3.951843in}{0.714523in}}%
\pgfpathlineto{\pgfqpoint{3.952708in}{0.760640in}}%
\pgfpathlineto{\pgfqpoint{3.953573in}{0.731665in}}%
\pgfpathlineto{\pgfqpoint{3.954436in}{0.777709in}}%
\pgfpathlineto{\pgfqpoint{3.955301in}{0.736538in}}%
\pgfpathlineto{\pgfqpoint{3.956166in}{0.764156in}}%
\pgfpathlineto{\pgfqpoint{3.957031in}{0.751666in}}%
\pgfpathlineto{\pgfqpoint{3.957894in}{0.719614in}}%
\pgfpathlineto{\pgfqpoint{3.958759in}{0.723131in}}%
\pgfpathlineto{\pgfqpoint{3.959623in}{0.741995in}}%
\pgfpathlineto{\pgfqpoint{3.960486in}{0.694119in}}%
\pgfpathlineto{\pgfqpoint{3.962216in}{0.772910in}}%
\pgfpathlineto{\pgfqpoint{3.963081in}{0.748954in}}%
\pgfpathlineto{\pgfqpoint{3.963945in}{0.764083in}}%
\pgfpathlineto{\pgfqpoint{3.964811in}{0.726464in}}%
\pgfpathlineto{\pgfqpoint{3.965676in}{0.790418in}}%
\pgfpathlineto{\pgfqpoint{3.966542in}{0.776426in}}%
\pgfpathlineto{\pgfqpoint{3.967407in}{0.649322in}}%
\pgfpathlineto{\pgfqpoint{3.968272in}{0.698590in}}%
\pgfpathlineto{\pgfqpoint{3.969137in}{0.697599in}}%
\pgfpathlineto{\pgfqpoint{3.970867in}{0.745365in}}%
\pgfpathlineto{\pgfqpoint{3.971732in}{0.770969in}}%
\pgfpathlineto{\pgfqpoint{3.972596in}{0.739615in}}%
\pgfpathlineto{\pgfqpoint{3.973460in}{0.799871in}}%
\pgfpathlineto{\pgfqpoint{3.974325in}{0.745036in}}%
\pgfpathlineto{\pgfqpoint{3.975191in}{0.762434in}}%
\pgfpathlineto{\pgfqpoint{3.976055in}{0.755037in}}%
\pgfpathlineto{\pgfqpoint{3.976921in}{0.709468in}}%
\pgfpathlineto{\pgfqpoint{3.977787in}{0.735035in}}%
\pgfpathlineto{\pgfqpoint{3.978650in}{0.733606in}}%
\pgfpathlineto{\pgfqpoint{3.979516in}{0.649212in}}%
\pgfpathlineto{\pgfqpoint{3.981249in}{0.748589in}}%
\pgfpathlineto{\pgfqpoint{3.982981in}{0.690932in}}%
\pgfpathlineto{\pgfqpoint{3.983846in}{0.699394in}}%
\pgfpathlineto{\pgfqpoint{3.984712in}{0.709870in}}%
\pgfpathlineto{\pgfqpoint{3.985575in}{0.630823in}}%
\pgfpathlineto{\pgfqpoint{3.986438in}{0.682397in}}%
\pgfpathlineto{\pgfqpoint{3.987304in}{0.640308in}}%
\pgfpathlineto{\pgfqpoint{3.988169in}{0.702837in}}%
\pgfpathlineto{\pgfqpoint{3.989902in}{0.672872in}}%
\pgfpathlineto{\pgfqpoint{3.990765in}{0.729245in}}%
\pgfpathlineto{\pgfqpoint{3.991629in}{0.640235in}}%
\pgfpathlineto{\pgfqpoint{3.992494in}{0.726501in}}%
\pgfpathlineto{\pgfqpoint{3.994223in}{0.664889in}}%
\pgfpathlineto{\pgfqpoint{3.995089in}{0.769979in}}%
\pgfpathlineto{\pgfqpoint{3.995954in}{0.684704in}}%
\pgfpathlineto{\pgfqpoint{3.996820in}{0.773860in}}%
\pgfpathlineto{\pgfqpoint{3.997684in}{0.730528in}}%
\pgfpathlineto{\pgfqpoint{3.998549in}{0.755694in}}%
\pgfpathlineto{\pgfqpoint{3.999414in}{0.750822in}}%
\pgfpathlineto{\pgfqpoint{4.000279in}{0.665547in}}%
\pgfpathlineto{\pgfqpoint{4.002007in}{0.725071in}}%
\pgfpathlineto{\pgfqpoint{4.002874in}{0.698294in}}%
\pgfpathlineto{\pgfqpoint{4.004603in}{0.719281in}}%
\pgfpathlineto{\pgfqpoint{4.005467in}{0.714153in}}%
\pgfpathlineto{\pgfqpoint{4.007191in}{0.667122in}}%
\pgfpathlineto{\pgfqpoint{4.008923in}{0.738917in}}%
\pgfpathlineto{\pgfqpoint{4.009788in}{0.703495in}}%
\pgfpathlineto{\pgfqpoint{4.010652in}{0.706426in}}%
\pgfpathlineto{\pgfqpoint{4.011517in}{0.742100in}}%
\pgfpathlineto{\pgfqpoint{4.013246in}{0.697559in}}%
\pgfpathlineto{\pgfqpoint{4.014108in}{0.692175in}}%
\pgfpathlineto{\pgfqpoint{4.014974in}{0.734154in}}%
\pgfpathlineto{\pgfqpoint{4.015839in}{0.708879in}}%
\pgfpathlineto{\pgfqpoint{4.016704in}{0.773276in}}%
\pgfpathlineto{\pgfqpoint{4.017569in}{0.725181in}}%
\pgfpathlineto{\pgfqpoint{4.018434in}{0.751772in}}%
\pgfpathlineto{\pgfqpoint{4.019301in}{0.700636in}}%
\pgfpathlineto{\pgfqpoint{4.020166in}{0.736132in}}%
\pgfpathlineto{\pgfqpoint{4.021032in}{0.732689in}}%
\pgfpathlineto{\pgfqpoint{4.021898in}{0.737122in}}%
\pgfpathlineto{\pgfqpoint{4.023629in}{0.705915in}}%
\pgfpathlineto{\pgfqpoint{4.024494in}{0.740492in}}%
\pgfpathlineto{\pgfqpoint{4.025359in}{0.730821in}}%
\pgfpathlineto{\pgfqpoint{4.026224in}{0.710527in}}%
\pgfpathlineto{\pgfqpoint{4.027953in}{0.750603in}}%
\pgfpathlineto{\pgfqpoint{4.028818in}{0.692727in}}%
\pgfpathlineto{\pgfqpoint{4.029684in}{0.777271in}}%
\pgfpathlineto{\pgfqpoint{4.031412in}{0.680566in}}%
\pgfpathlineto{\pgfqpoint{4.032277in}{0.682105in}}%
\pgfpathlineto{\pgfqpoint{4.034007in}{0.771517in}}%
\pgfpathlineto{\pgfqpoint{4.036604in}{0.708477in}}%
\pgfpathlineto{\pgfqpoint{4.037469in}{0.749320in}}%
\pgfpathlineto{\pgfqpoint{4.038334in}{0.705915in}}%
\pgfpathlineto{\pgfqpoint{4.039199in}{0.709577in}}%
\pgfpathlineto{\pgfqpoint{4.040065in}{0.696682in}}%
\pgfpathlineto{\pgfqpoint{4.040931in}{0.738734in}}%
\pgfpathlineto{\pgfqpoint{4.042662in}{0.685110in}}%
\pgfpathlineto{\pgfqpoint{4.045257in}{0.753348in}}%
\pgfpathlineto{\pgfqpoint{4.046121in}{0.736497in}}%
\pgfpathlineto{\pgfqpoint{4.046986in}{0.737926in}}%
\pgfpathlineto{\pgfqpoint{4.047850in}{0.756718in}}%
\pgfpathlineto{\pgfqpoint{4.048715in}{0.691517in}}%
\pgfpathlineto{\pgfqpoint{4.049579in}{0.785546in}}%
\pgfpathlineto{\pgfqpoint{4.050444in}{0.706792in}}%
\pgfpathlineto{\pgfqpoint{4.052176in}{0.787706in}}%
\pgfpathlineto{\pgfqpoint{4.053040in}{0.681443in}}%
\pgfpathlineto{\pgfqpoint{4.053906in}{0.740087in}}%
\pgfpathlineto{\pgfqpoint{4.054771in}{0.711478in}}%
\pgfpathlineto{\pgfqpoint{4.055636in}{0.645948in}}%
\pgfpathlineto{\pgfqpoint{4.056502in}{0.775363in}}%
\pgfpathlineto{\pgfqpoint{4.058228in}{0.714409in}}%
\pgfpathlineto{\pgfqpoint{4.059092in}{0.730382in}}%
\pgfpathlineto{\pgfqpoint{4.060822in}{0.690380in}}%
\pgfpathlineto{\pgfqpoint{4.061687in}{0.740306in}}%
\pgfpathlineto{\pgfqpoint{4.062552in}{0.685910in}}%
\pgfpathlineto{\pgfqpoint{4.063417in}{0.733493in}}%
\pgfpathlineto{\pgfqpoint{4.064282in}{0.721953in}}%
\pgfpathlineto{\pgfqpoint{4.065147in}{0.712432in}}%
\pgfpathlineto{\pgfqpoint{4.066012in}{0.764408in}}%
\pgfpathlineto{\pgfqpoint{4.067741in}{0.705874in}}%
\pgfpathlineto{\pgfqpoint{4.068604in}{0.738438in}}%
\pgfpathlineto{\pgfqpoint{4.069469in}{0.672762in}}%
\pgfpathlineto{\pgfqpoint{4.071198in}{0.700344in}}%
\pgfpathlineto{\pgfqpoint{4.072928in}{0.676315in}}%
\pgfpathlineto{\pgfqpoint{4.073792in}{0.688585in}}%
\pgfpathlineto{\pgfqpoint{4.074657in}{0.729136in}}%
\pgfpathlineto{\pgfqpoint{4.075522in}{0.720638in}}%
\pgfpathlineto{\pgfqpoint{4.076388in}{0.705582in}}%
\pgfpathlineto{\pgfqpoint{4.078118in}{0.761663in}}%
\pgfpathlineto{\pgfqpoint{4.079849in}{0.695252in}}%
\pgfpathlineto{\pgfqpoint{4.080713in}{0.698622in}}%
\pgfpathlineto{\pgfqpoint{4.081580in}{0.700381in}}%
\pgfpathlineto{\pgfqpoint{4.082444in}{0.782468in}}%
\pgfpathlineto{\pgfqpoint{4.084175in}{0.686608in}}%
\pgfpathlineto{\pgfqpoint{4.085039in}{0.725802in}}%
\pgfpathlineto{\pgfqpoint{4.085903in}{0.695106in}}%
\pgfpathlineto{\pgfqpoint{4.086768in}{0.739063in}}%
\pgfpathlineto{\pgfqpoint{4.087632in}{0.728035in}}%
\pgfpathlineto{\pgfqpoint{4.088498in}{0.652798in}}%
\pgfpathlineto{\pgfqpoint{4.090229in}{0.740858in}}%
\pgfpathlineto{\pgfqpoint{4.091095in}{0.708915in}}%
\pgfpathlineto{\pgfqpoint{4.093691in}{0.825029in}}%
\pgfpathlineto{\pgfqpoint{4.094556in}{0.743895in}}%
\pgfpathlineto{\pgfqpoint{4.095420in}{0.805653in}}%
\pgfpathlineto{\pgfqpoint{4.097151in}{0.707450in}}%
\pgfpathlineto{\pgfqpoint{4.098016in}{0.758439in}}%
\pgfpathlineto{\pgfqpoint{4.098881in}{0.741589in}}%
\pgfpathlineto{\pgfqpoint{4.099746in}{0.688732in}}%
\pgfpathlineto{\pgfqpoint{4.100611in}{0.695216in}}%
\pgfpathlineto{\pgfqpoint{4.101476in}{0.725437in}}%
\pgfpathlineto{\pgfqpoint{4.102341in}{0.687781in}}%
\pgfpathlineto{\pgfqpoint{4.103206in}{0.764923in}}%
\pgfpathlineto{\pgfqpoint{4.104071in}{0.738073in}}%
\pgfpathlineto{\pgfqpoint{4.104936in}{0.753128in}}%
\pgfpathlineto{\pgfqpoint{4.107530in}{0.608771in}}%
\pgfpathlineto{\pgfqpoint{4.108395in}{0.726939in}}%
\pgfpathlineto{\pgfqpoint{4.110125in}{0.670310in}}%
\pgfpathlineto{\pgfqpoint{4.110990in}{0.689722in}}%
\pgfpathlineto{\pgfqpoint{4.111854in}{0.645473in}}%
\pgfpathlineto{\pgfqpoint{4.113584in}{0.713386in}}%
\pgfpathlineto{\pgfqpoint{4.114449in}{0.703422in}}%
\pgfpathlineto{\pgfqpoint{4.115313in}{0.678037in}}%
\pgfpathlineto{\pgfqpoint{4.116177in}{0.733972in}}%
\pgfpathlineto{\pgfqpoint{4.117040in}{0.677562in}}%
\pgfpathlineto{\pgfqpoint{4.117905in}{0.691371in}}%
\pgfpathlineto{\pgfqpoint{4.119636in}{0.766243in}}%
\pgfpathlineto{\pgfqpoint{4.120501in}{0.755475in}}%
\pgfpathlineto{\pgfqpoint{4.122231in}{0.690859in}}%
\pgfpathlineto{\pgfqpoint{4.123096in}{0.727012in}}%
\pgfpathlineto{\pgfqpoint{4.123961in}{0.683900in}}%
\pgfpathlineto{\pgfqpoint{4.125692in}{0.767745in}}%
\pgfpathlineto{\pgfqpoint{4.126557in}{0.697563in}}%
\pgfpathlineto{\pgfqpoint{4.127422in}{0.716244in}}%
\pgfpathlineto{\pgfqpoint{4.129154in}{0.692617in}}%
\pgfpathlineto{\pgfqpoint{4.130019in}{0.702983in}}%
\pgfpathlineto{\pgfqpoint{4.130884in}{0.667195in}}%
\pgfpathlineto{\pgfqpoint{4.132610in}{0.711810in}}%
\pgfpathlineto{\pgfqpoint{4.133474in}{0.753348in}}%
\pgfpathlineto{\pgfqpoint{4.134339in}{0.635143in}}%
\pgfpathlineto{\pgfqpoint{4.135203in}{0.724373in}}%
\pgfpathlineto{\pgfqpoint{4.136068in}{0.722067in}}%
\pgfpathlineto{\pgfqpoint{4.137798in}{0.785180in}}%
\pgfpathlineto{\pgfqpoint{4.139529in}{0.708952in}}%
\pgfpathlineto{\pgfqpoint{4.141259in}{0.722761in}}%
\pgfpathlineto{\pgfqpoint{4.142124in}{0.721661in}}%
\pgfpathlineto{\pgfqpoint{4.143853in}{0.747484in}}%
\pgfpathlineto{\pgfqpoint{4.144718in}{0.714482in}}%
\pgfpathlineto{\pgfqpoint{4.145583in}{0.772577in}}%
\pgfpathlineto{\pgfqpoint{4.148177in}{0.673310in}}%
\pgfpathlineto{\pgfqpoint{4.149043in}{0.710820in}}%
\pgfpathlineto{\pgfqpoint{4.150774in}{0.667195in}}%
\pgfpathlineto{\pgfqpoint{4.151640in}{0.725656in}}%
\pgfpathlineto{\pgfqpoint{4.152507in}{0.711883in}}%
\pgfpathlineto{\pgfqpoint{4.153374in}{0.688147in}}%
\pgfpathlineto{\pgfqpoint{4.154234in}{0.744667in}}%
\pgfpathlineto{\pgfqpoint{4.155100in}{0.669100in}}%
\pgfpathlineto{\pgfqpoint{4.156830in}{0.737780in}}%
\pgfpathlineto{\pgfqpoint{4.157692in}{0.674703in}}%
\pgfpathlineto{\pgfqpoint{4.158557in}{0.725583in}}%
\pgfpathlineto{\pgfqpoint{4.159420in}{0.702618in}}%
\pgfpathlineto{\pgfqpoint{4.160285in}{0.716354in}}%
\pgfpathlineto{\pgfqpoint{4.161148in}{0.709943in}}%
\pgfpathlineto{\pgfqpoint{4.162013in}{0.726793in}}%
\pgfpathlineto{\pgfqpoint{4.162876in}{0.658149in}}%
\pgfpathlineto{\pgfqpoint{4.164607in}{0.693937in}}%
\pgfpathlineto{\pgfqpoint{4.165472in}{0.658441in}}%
\pgfpathlineto{\pgfqpoint{4.166337in}{0.766535in}}%
\pgfpathlineto{\pgfqpoint{4.167203in}{0.637490in}}%
\pgfpathlineto{\pgfqpoint{4.168067in}{0.728515in}}%
\pgfpathlineto{\pgfqpoint{4.169798in}{0.666318in}}%
\pgfpathlineto{\pgfqpoint{4.170663in}{0.693717in}}%
\pgfpathlineto{\pgfqpoint{4.171528in}{0.661811in}}%
\pgfpathlineto{\pgfqpoint{4.173258in}{0.693937in}}%
\pgfpathlineto{\pgfqpoint{4.174125in}{0.646719in}}%
\pgfpathlineto{\pgfqpoint{4.174991in}{0.695951in}}%
\pgfpathlineto{\pgfqpoint{4.175858in}{0.625805in}}%
\pgfpathlineto{\pgfqpoint{4.177588in}{0.662396in}}%
\pgfpathlineto{\pgfqpoint{4.178452in}{0.646610in}}%
\pgfpathlineto{\pgfqpoint{4.180184in}{0.713166in}}%
\pgfpathlineto{\pgfqpoint{4.181050in}{0.677086in}}%
\pgfpathlineto{\pgfqpoint{4.181915in}{0.735693in}}%
\pgfpathlineto{\pgfqpoint{4.183646in}{0.668077in}}%
\pgfpathlineto{\pgfqpoint{4.185377in}{0.736059in}}%
\pgfpathlineto{\pgfqpoint{4.186240in}{0.682909in}}%
\pgfpathlineto{\pgfqpoint{4.187102in}{0.729099in}}%
\pgfpathlineto{\pgfqpoint{4.187968in}{0.684265in}}%
\pgfpathlineto{\pgfqpoint{4.188833in}{0.707271in}}%
\pgfpathlineto{\pgfqpoint{4.189700in}{0.697965in}}%
\pgfpathlineto{\pgfqpoint{4.190565in}{0.717048in}}%
\pgfpathlineto{\pgfqpoint{4.191432in}{0.680822in}}%
\pgfpathlineto{\pgfqpoint{4.192297in}{0.724779in}}%
\pgfpathlineto{\pgfqpoint{4.193163in}{0.678589in}}%
\pgfpathlineto{\pgfqpoint{4.194030in}{0.692581in}}%
\pgfpathlineto{\pgfqpoint{4.194895in}{0.680237in}}%
\pgfpathlineto{\pgfqpoint{4.195760in}{0.702289in}}%
\pgfpathlineto{\pgfqpoint{4.196623in}{0.637344in}}%
\pgfpathlineto{\pgfqpoint{4.197486in}{0.645513in}}%
\pgfpathlineto{\pgfqpoint{4.199218in}{0.775330in}}%
\pgfpathlineto{\pgfqpoint{4.200083in}{0.675072in}}%
\pgfpathlineto{\pgfqpoint{4.201815in}{0.744155in}}%
\pgfpathlineto{\pgfqpoint{4.202680in}{0.661081in}}%
\pgfpathlineto{\pgfqpoint{4.204411in}{0.779870in}}%
\pgfpathlineto{\pgfqpoint{4.206142in}{0.638367in}}%
\pgfpathlineto{\pgfqpoint{4.208736in}{0.766974in}}%
\pgfpathlineto{\pgfqpoint{4.210467in}{0.695585in}}%
\pgfpathlineto{\pgfqpoint{4.212198in}{0.659871in}}%
\pgfpathlineto{\pgfqpoint{4.213925in}{0.786975in}}%
\pgfpathlineto{\pgfqpoint{4.215654in}{0.664191in}}%
\pgfpathlineto{\pgfqpoint{4.217385in}{0.721774in}}%
\pgfpathlineto{\pgfqpoint{4.218250in}{0.698184in}}%
\pgfpathlineto{\pgfqpoint{4.219976in}{0.707157in}}%
\pgfpathlineto{\pgfqpoint{4.220842in}{0.701700in}}%
\pgfpathlineto{\pgfqpoint{4.221705in}{0.686681in}}%
\pgfpathlineto{\pgfqpoint{4.223436in}{0.756937in}}%
\pgfpathlineto{\pgfqpoint{4.224302in}{0.775582in}}%
\pgfpathlineto{\pgfqpoint{4.226031in}{0.682544in}}%
\pgfpathlineto{\pgfqpoint{4.226894in}{0.705951in}}%
\pgfpathlineto{\pgfqpoint{4.227760in}{0.634924in}}%
\pgfpathlineto{\pgfqpoint{4.229490in}{0.710308in}}%
\pgfpathlineto{\pgfqpoint{4.230356in}{0.718697in}}%
\pgfpathlineto{\pgfqpoint{4.232084in}{0.643313in}}%
\pgfpathlineto{\pgfqpoint{4.232949in}{0.732469in}}%
\pgfpathlineto{\pgfqpoint{4.233814in}{0.688110in}}%
\pgfpathlineto{\pgfqpoint{4.234678in}{0.758074in}}%
\pgfpathlineto{\pgfqpoint{4.236409in}{0.658186in}}%
\pgfpathlineto{\pgfqpoint{4.238138in}{0.724998in}}%
\pgfpathlineto{\pgfqpoint{4.239003in}{0.711006in}}%
\pgfpathlineto{\pgfqpoint{4.240732in}{0.740419in}}%
\pgfpathlineto{\pgfqpoint{4.243325in}{0.711372in}}%
\pgfpathlineto{\pgfqpoint{4.244190in}{0.734191in}}%
\pgfpathlineto{\pgfqpoint{4.245054in}{0.733825in}}%
\pgfpathlineto{\pgfqpoint{4.245919in}{0.739502in}}%
\pgfpathlineto{\pgfqpoint{4.246784in}{0.710235in}}%
\pgfpathlineto{\pgfqpoint{4.248514in}{0.754631in}}%
\pgfpathlineto{\pgfqpoint{4.249380in}{0.720126in}}%
\pgfpathlineto{\pgfqpoint{4.250245in}{0.638331in}}%
\pgfpathlineto{\pgfqpoint{4.251107in}{0.766682in}}%
\pgfpathlineto{\pgfqpoint{4.251972in}{0.749320in}}%
\pgfpathlineto{\pgfqpoint{4.252837in}{0.679356in}}%
\pgfpathlineto{\pgfqpoint{4.253702in}{0.727085in}}%
\pgfpathlineto{\pgfqpoint{4.255431in}{0.650418in}}%
\pgfpathlineto{\pgfqpoint{4.257163in}{0.720126in}}%
\pgfpathlineto{\pgfqpoint{4.258027in}{0.719281in}}%
\pgfpathlineto{\pgfqpoint{4.258891in}{0.724300in}}%
\pgfpathlineto{\pgfqpoint{4.259758in}{0.652578in}}%
\pgfpathlineto{\pgfqpoint{4.262349in}{0.770015in}}%
\pgfpathlineto{\pgfqpoint{4.263215in}{0.700198in}}%
\pgfpathlineto{\pgfqpoint{4.265810in}{0.789720in}}%
\pgfpathlineto{\pgfqpoint{4.267541in}{0.704080in}}%
\pgfpathlineto{\pgfqpoint{4.268406in}{0.742872in}}%
\pgfpathlineto{\pgfqpoint{4.269270in}{0.675511in}}%
\pgfpathlineto{\pgfqpoint{4.271000in}{0.745876in}}%
\pgfpathlineto{\pgfqpoint{4.272729in}{0.645327in}}%
\pgfpathlineto{\pgfqpoint{4.273593in}{0.719647in}}%
\pgfpathlineto{\pgfqpoint{4.274458in}{0.694335in}}%
\pgfpathlineto{\pgfqpoint{4.276188in}{0.757887in}}%
\pgfpathlineto{\pgfqpoint{4.277054in}{0.625325in}}%
\pgfpathlineto{\pgfqpoint{4.277919in}{0.687010in}}%
\pgfpathlineto{\pgfqpoint{4.278785in}{0.624335in}}%
\pgfpathlineto{\pgfqpoint{4.280515in}{0.760819in}}%
\pgfpathlineto{\pgfqpoint{4.281379in}{0.653053in}}%
\pgfpathlineto{\pgfqpoint{4.282243in}{0.750416in}}%
\pgfpathlineto{\pgfqpoint{4.284839in}{0.667561in}}%
\pgfpathlineto{\pgfqpoint{4.285704in}{0.681626in}}%
\pgfpathlineto{\pgfqpoint{4.286569in}{0.667707in}}%
\pgfpathlineto{\pgfqpoint{4.287434in}{0.614411in}}%
\pgfpathlineto{\pgfqpoint{4.289164in}{0.701993in}}%
\pgfpathlineto{\pgfqpoint{4.290029in}{0.691882in}}%
\pgfpathlineto{\pgfqpoint{4.290894in}{0.705436in}}%
\pgfpathlineto{\pgfqpoint{4.291759in}{0.663899in}}%
\pgfpathlineto{\pgfqpoint{4.292625in}{0.695252in}}%
\pgfpathlineto{\pgfqpoint{4.293491in}{0.658953in}}%
\pgfpathlineto{\pgfqpoint{4.296948in}{0.707377in}}%
\pgfpathlineto{\pgfqpoint{4.298678in}{0.697193in}}%
\pgfpathlineto{\pgfqpoint{4.300410in}{0.599871in}}%
\pgfpathlineto{\pgfqpoint{4.301276in}{0.681886in}}%
\pgfpathlineto{\pgfqpoint{4.302140in}{0.681699in}}%
\pgfpathlineto{\pgfqpoint{4.303005in}{0.682653in}}%
\pgfpathlineto{\pgfqpoint{4.303869in}{0.638148in}}%
\pgfpathlineto{\pgfqpoint{4.304734in}{0.725802in}}%
\pgfpathlineto{\pgfqpoint{4.305599in}{0.684119in}}%
\pgfpathlineto{\pgfqpoint{4.306465in}{0.762581in}}%
\pgfpathlineto{\pgfqpoint{4.307331in}{0.709139in}}%
\pgfpathlineto{\pgfqpoint{4.308196in}{0.714742in}}%
\pgfpathlineto{\pgfqpoint{4.309061in}{0.709358in}}%
\pgfpathlineto{\pgfqpoint{4.309926in}{0.658039in}}%
\pgfpathlineto{\pgfqpoint{4.311656in}{0.724446in}}%
\pgfpathlineto{\pgfqpoint{4.312522in}{0.695658in}}%
\pgfpathlineto{\pgfqpoint{4.313387in}{0.731885in}}%
\pgfpathlineto{\pgfqpoint{4.314253in}{0.728515in}}%
\pgfpathlineto{\pgfqpoint{4.315118in}{0.677598in}}%
\pgfpathlineto{\pgfqpoint{4.316847in}{0.735730in}}%
\pgfpathlineto{\pgfqpoint{4.318576in}{0.713605in}}%
\pgfpathlineto{\pgfqpoint{4.319442in}{0.723752in}}%
\pgfpathlineto{\pgfqpoint{4.320307in}{0.720711in}}%
\pgfpathlineto{\pgfqpoint{4.321170in}{0.696901in}}%
\pgfpathlineto{\pgfqpoint{4.322035in}{0.744045in}}%
\pgfpathlineto{\pgfqpoint{4.323763in}{0.708002in}}%
\pgfpathlineto{\pgfqpoint{4.324629in}{0.750895in}}%
\pgfpathlineto{\pgfqpoint{4.326357in}{0.678991in}}%
\pgfpathlineto{\pgfqpoint{4.328083in}{0.741954in}}%
\pgfpathlineto{\pgfqpoint{4.328948in}{0.672982in}}%
\pgfpathlineto{\pgfqpoint{4.329813in}{0.756718in}}%
\pgfpathlineto{\pgfqpoint{4.330678in}{0.735986in}}%
\pgfpathlineto{\pgfqpoint{4.331544in}{0.705326in}}%
\pgfpathlineto{\pgfqpoint{4.332409in}{0.762102in}}%
\pgfpathlineto{\pgfqpoint{4.333276in}{0.701554in}}%
\pgfpathlineto{\pgfqpoint{4.334141in}{0.717194in}}%
\pgfpathlineto{\pgfqpoint{4.335005in}{0.662835in}}%
\pgfpathlineto{\pgfqpoint{4.335868in}{0.766353in}}%
\pgfpathlineto{\pgfqpoint{4.336733in}{0.654852in}}%
\pgfpathlineto{\pgfqpoint{4.337598in}{0.677744in}}%
\pgfpathlineto{\pgfqpoint{4.338461in}{0.707230in}}%
\pgfpathlineto{\pgfqpoint{4.339326in}{0.787560in}}%
\pgfpathlineto{\pgfqpoint{4.341058in}{0.672835in}}%
\pgfpathlineto{\pgfqpoint{4.341924in}{0.684484in}}%
\pgfpathlineto{\pgfqpoint{4.342786in}{0.693165in}}%
\pgfpathlineto{\pgfqpoint{4.344511in}{0.732104in}}%
\pgfpathlineto{\pgfqpoint{4.345375in}{0.739794in}}%
\pgfpathlineto{\pgfqpoint{4.347106in}{0.702285in}}%
\pgfpathlineto{\pgfqpoint{4.347971in}{0.723898in}}%
\pgfpathlineto{\pgfqpoint{4.348837in}{0.809063in}}%
\pgfpathlineto{\pgfqpoint{4.350566in}{0.684594in}}%
\pgfpathlineto{\pgfqpoint{4.352298in}{0.736424in}}%
\pgfpathlineto{\pgfqpoint{4.353162in}{0.686279in}}%
\pgfpathlineto{\pgfqpoint{4.355757in}{0.770677in}}%
\pgfpathlineto{\pgfqpoint{4.356622in}{0.718550in}}%
\pgfpathlineto{\pgfqpoint{4.358349in}{0.793642in}}%
\pgfpathlineto{\pgfqpoint{4.359212in}{0.772325in}}%
\pgfpathlineto{\pgfqpoint{4.360077in}{0.803314in}}%
\pgfpathlineto{\pgfqpoint{4.363537in}{0.683315in}}%
\pgfpathlineto{\pgfqpoint{4.365264in}{0.719834in}}%
\pgfpathlineto{\pgfqpoint{4.366126in}{0.656501in}}%
\pgfpathlineto{\pgfqpoint{4.366989in}{0.728441in}}%
\pgfpathlineto{\pgfqpoint{4.367853in}{0.694375in}}%
\pgfpathlineto{\pgfqpoint{4.368718in}{0.755914in}}%
\pgfpathlineto{\pgfqpoint{4.370449in}{0.698330in}}%
\pgfpathlineto{\pgfqpoint{4.371313in}{0.758878in}}%
\pgfpathlineto{\pgfqpoint{4.373044in}{0.663314in}}%
\pgfpathlineto{\pgfqpoint{4.373909in}{0.665510in}}%
\pgfpathlineto{\pgfqpoint{4.374774in}{0.745438in}}%
\pgfpathlineto{\pgfqpoint{4.375640in}{0.682617in}}%
\pgfpathlineto{\pgfqpoint{4.376505in}{0.738588in}}%
\pgfpathlineto{\pgfqpoint{4.378235in}{0.665108in}}%
\pgfpathlineto{\pgfqpoint{4.380827in}{0.736392in}}%
\pgfpathlineto{\pgfqpoint{4.383424in}{0.708846in}}%
\pgfpathlineto{\pgfqpoint{4.384290in}{0.673278in}}%
\pgfpathlineto{\pgfqpoint{4.386020in}{0.743132in}}%
\pgfpathlineto{\pgfqpoint{4.387751in}{0.693238in}}%
\pgfpathlineto{\pgfqpoint{4.388616in}{0.692325in}}%
\pgfpathlineto{\pgfqpoint{4.389481in}{0.706426in}}%
\pgfpathlineto{\pgfqpoint{4.390346in}{0.695512in}}%
\pgfpathlineto{\pgfqpoint{4.391212in}{0.719395in}}%
\pgfpathlineto{\pgfqpoint{4.392077in}{0.709285in}}%
\pgfpathlineto{\pgfqpoint{4.392939in}{0.660346in}}%
\pgfpathlineto{\pgfqpoint{4.394670in}{0.722213in}}%
\pgfpathlineto{\pgfqpoint{4.395533in}{0.719834in}}%
\pgfpathlineto{\pgfqpoint{4.397262in}{0.662729in}}%
\pgfpathlineto{\pgfqpoint{4.398127in}{0.756575in}}%
\pgfpathlineto{\pgfqpoint{4.398990in}{0.703681in}}%
\pgfpathlineto{\pgfqpoint{4.401586in}{0.745730in}}%
\pgfpathlineto{\pgfqpoint{4.403318in}{0.685987in}}%
\pgfpathlineto{\pgfqpoint{4.405047in}{0.725693in}}%
\pgfpathlineto{\pgfqpoint{4.405913in}{0.703056in}}%
\pgfpathlineto{\pgfqpoint{4.406779in}{0.716829in}}%
\pgfpathlineto{\pgfqpoint{4.407644in}{0.632581in}}%
\pgfpathlineto{\pgfqpoint{4.408510in}{0.805620in}}%
\pgfpathlineto{\pgfqpoint{4.409376in}{0.718404in}}%
\pgfpathlineto{\pgfqpoint{4.410240in}{0.723642in}}%
\pgfpathlineto{\pgfqpoint{4.411970in}{0.701152in}}%
\pgfpathlineto{\pgfqpoint{4.412836in}{0.691225in}}%
\pgfpathlineto{\pgfqpoint{4.413701in}{0.771408in}}%
\pgfpathlineto{\pgfqpoint{4.415430in}{0.682836in}}%
\pgfpathlineto{\pgfqpoint{4.418891in}{0.762581in}}%
\pgfpathlineto{\pgfqpoint{4.419756in}{0.660382in}}%
\pgfpathlineto{\pgfqpoint{4.420621in}{0.677744in}}%
\pgfpathlineto{\pgfqpoint{4.422347in}{0.735474in}}%
\pgfpathlineto{\pgfqpoint{4.423211in}{0.718624in}}%
\pgfpathlineto{\pgfqpoint{4.424076in}{0.748073in}}%
\pgfpathlineto{\pgfqpoint{4.424939in}{0.680051in}}%
\pgfpathlineto{\pgfqpoint{4.425802in}{0.693092in}}%
\pgfpathlineto{\pgfqpoint{4.426667in}{0.704339in}}%
\pgfpathlineto{\pgfqpoint{4.428398in}{0.771408in}}%
\pgfpathlineto{\pgfqpoint{4.430129in}{0.725583in}}%
\pgfpathlineto{\pgfqpoint{4.430994in}{0.772764in}}%
\pgfpathlineto{\pgfqpoint{4.431859in}{0.711299in}}%
\pgfpathlineto{\pgfqpoint{4.432725in}{0.781445in}}%
\pgfpathlineto{\pgfqpoint{4.433588in}{0.777307in}}%
\pgfpathlineto{\pgfqpoint{4.436184in}{0.663095in}}%
\pgfpathlineto{\pgfqpoint{4.437050in}{0.757781in}}%
\pgfpathlineto{\pgfqpoint{4.437915in}{0.750456in}}%
\pgfpathlineto{\pgfqpoint{4.438780in}{0.726866in}}%
\pgfpathlineto{\pgfqpoint{4.439645in}{0.665108in}}%
\pgfpathlineto{\pgfqpoint{4.442235in}{0.746940in}}%
\pgfpathlineto{\pgfqpoint{4.443964in}{0.699759in}}%
\pgfpathlineto{\pgfqpoint{4.444829in}{0.723642in}}%
\pgfpathlineto{\pgfqpoint{4.445695in}{0.680676in}}%
\pgfpathlineto{\pgfqpoint{4.446561in}{0.699101in}}%
\pgfpathlineto{\pgfqpoint{4.447427in}{0.698257in}}%
\pgfpathlineto{\pgfqpoint{4.449156in}{0.629028in}}%
\pgfpathlineto{\pgfqpoint{4.450020in}{0.698846in}}%
\pgfpathlineto{\pgfqpoint{4.450885in}{0.683753in}}%
\pgfpathlineto{\pgfqpoint{4.451751in}{0.715546in}}%
\pgfpathlineto{\pgfqpoint{4.452617in}{0.702947in}}%
\pgfpathlineto{\pgfqpoint{4.453484in}{0.717706in}}%
\pgfpathlineto{\pgfqpoint{4.454349in}{0.684411in}}%
\pgfpathlineto{\pgfqpoint{4.455213in}{0.740675in}}%
\pgfpathlineto{\pgfqpoint{4.456077in}{0.674082in}}%
\pgfpathlineto{\pgfqpoint{4.456943in}{0.756133in}}%
\pgfpathlineto{\pgfqpoint{4.457807in}{0.742872in}}%
\pgfpathlineto{\pgfqpoint{4.458672in}{0.730894in}}%
\pgfpathlineto{\pgfqpoint{4.459537in}{0.672762in}}%
\pgfpathlineto{\pgfqpoint{4.461266in}{0.715765in}}%
\pgfpathlineto{\pgfqpoint{4.462131in}{0.700417in}}%
\pgfpathlineto{\pgfqpoint{4.462995in}{0.737342in}}%
\pgfpathlineto{\pgfqpoint{4.464723in}{0.702581in}}%
\pgfpathlineto{\pgfqpoint{4.465588in}{0.691371in}}%
\pgfpathlineto{\pgfqpoint{4.466453in}{0.749758in}}%
\pgfpathlineto{\pgfqpoint{4.467318in}{0.693604in}}%
\pgfpathlineto{\pgfqpoint{4.468183in}{0.706792in}}%
\pgfpathlineto{\pgfqpoint{4.469047in}{0.671995in}}%
\pgfpathlineto{\pgfqpoint{4.470777in}{0.742068in}}%
\pgfpathlineto{\pgfqpoint{4.471642in}{0.702800in}}%
\pgfpathlineto{\pgfqpoint{4.472507in}{0.785180in}}%
\pgfpathlineto{\pgfqpoint{4.473373in}{0.651701in}}%
\pgfpathlineto{\pgfqpoint{4.474235in}{0.675840in}}%
\pgfpathlineto{\pgfqpoint{4.476831in}{0.794665in}}%
\pgfpathlineto{\pgfqpoint{4.478560in}{0.700783in}}%
\pgfpathlineto{\pgfqpoint{4.479425in}{0.709244in}}%
\pgfpathlineto{\pgfqpoint{4.480290in}{0.653788in}}%
\pgfpathlineto{\pgfqpoint{4.482023in}{0.761078in}}%
\pgfpathlineto{\pgfqpoint{4.482888in}{0.653569in}}%
\pgfpathlineto{\pgfqpoint{4.483754in}{0.749685in}}%
\pgfpathlineto{\pgfqpoint{4.484620in}{0.703349in}}%
\pgfpathlineto{\pgfqpoint{4.485486in}{0.770052in}}%
\pgfpathlineto{\pgfqpoint{4.486350in}{0.730382in}}%
\pgfpathlineto{\pgfqpoint{4.487216in}{0.800675in}}%
\pgfpathlineto{\pgfqpoint{4.488081in}{0.680124in}}%
\pgfpathlineto{\pgfqpoint{4.488945in}{0.695033in}}%
\pgfpathlineto{\pgfqpoint{4.489808in}{0.657451in}}%
\pgfpathlineto{\pgfqpoint{4.492402in}{0.737853in}}%
\pgfpathlineto{\pgfqpoint{4.493266in}{0.654889in}}%
\pgfpathlineto{\pgfqpoint{4.496727in}{0.757343in}}%
\pgfpathlineto{\pgfqpoint{4.497591in}{0.682178in}}%
\pgfpathlineto{\pgfqpoint{4.499321in}{0.776499in}}%
\pgfpathlineto{\pgfqpoint{4.501049in}{0.699686in}}%
\pgfpathlineto{\pgfqpoint{4.501914in}{0.670200in}}%
\pgfpathlineto{\pgfqpoint{4.504511in}{0.759211in}}%
\pgfpathlineto{\pgfqpoint{4.505375in}{0.690421in}}%
\pgfpathlineto{\pgfqpoint{4.506240in}{0.716171in}}%
\pgfpathlineto{\pgfqpoint{4.507105in}{0.689799in}}%
\pgfpathlineto{\pgfqpoint{4.507970in}{0.727711in}}%
\pgfpathlineto{\pgfqpoint{4.508835in}{0.682617in}}%
\pgfpathlineto{\pgfqpoint{4.509700in}{0.716573in}}%
\pgfpathlineto{\pgfqpoint{4.510564in}{0.654998in}}%
\pgfpathlineto{\pgfqpoint{4.512294in}{0.728186in}}%
\pgfpathlineto{\pgfqpoint{4.513159in}{0.657085in}}%
\pgfpathlineto{\pgfqpoint{4.514025in}{0.677890in}}%
\pgfpathlineto{\pgfqpoint{4.514890in}{0.639577in}}%
\pgfpathlineto{\pgfqpoint{4.515753in}{0.670346in}}%
\pgfpathlineto{\pgfqpoint{4.516619in}{0.612032in}}%
\pgfpathlineto{\pgfqpoint{4.518346in}{0.712322in}}%
\pgfpathlineto{\pgfqpoint{4.519212in}{0.686060in}}%
\pgfpathlineto{\pgfqpoint{4.520076in}{0.750822in}}%
\pgfpathlineto{\pgfqpoint{4.521806in}{0.682434in}}%
\pgfpathlineto{\pgfqpoint{4.522670in}{0.685767in}}%
\pgfpathlineto{\pgfqpoint{4.523535in}{0.694741in}}%
\pgfpathlineto{\pgfqpoint{4.524400in}{0.728368in}}%
\pgfpathlineto{\pgfqpoint{4.525263in}{0.716463in}}%
\pgfpathlineto{\pgfqpoint{4.526128in}{0.669100in}}%
\pgfpathlineto{\pgfqpoint{4.527857in}{0.701042in}}%
\pgfpathlineto{\pgfqpoint{4.528722in}{0.646719in}}%
\pgfpathlineto{\pgfqpoint{4.529586in}{0.654852in}}%
\pgfpathlineto{\pgfqpoint{4.530452in}{0.663387in}}%
\pgfpathlineto{\pgfqpoint{4.532179in}{0.753900in}}%
\pgfpathlineto{\pgfqpoint{4.533043in}{0.689211in}}%
\pgfpathlineto{\pgfqpoint{4.533909in}{0.730236in}}%
\pgfpathlineto{\pgfqpoint{4.535638in}{0.664962in}}%
\pgfpathlineto{\pgfqpoint{4.537370in}{0.696170in}}%
\pgfpathlineto{\pgfqpoint{4.538236in}{0.695878in}}%
\pgfpathlineto{\pgfqpoint{4.539100in}{0.697892in}}%
\pgfpathlineto{\pgfqpoint{4.539965in}{0.734962in}}%
\pgfpathlineto{\pgfqpoint{4.541691in}{0.688845in}}%
\pgfpathlineto{\pgfqpoint{4.542555in}{0.743205in}}%
\pgfpathlineto{\pgfqpoint{4.544282in}{0.709285in}}%
\pgfpathlineto{\pgfqpoint{4.545145in}{0.723683in}}%
\pgfpathlineto{\pgfqpoint{4.546010in}{0.706467in}}%
\pgfpathlineto{\pgfqpoint{4.546874in}{0.768663in}}%
\pgfpathlineto{\pgfqpoint{4.548603in}{0.714376in}}%
\pgfpathlineto{\pgfqpoint{4.549468in}{0.682324in}}%
\pgfpathlineto{\pgfqpoint{4.551196in}{0.726354in}}%
\pgfpathlineto{\pgfqpoint{4.552060in}{0.722400in}}%
\pgfpathlineto{\pgfqpoint{4.552925in}{0.725770in}}%
\pgfpathlineto{\pgfqpoint{4.553786in}{0.744301in}}%
\pgfpathlineto{\pgfqpoint{4.554650in}{0.721299in}}%
\pgfpathlineto{\pgfqpoint{4.555516in}{0.728368in}}%
\pgfpathlineto{\pgfqpoint{4.556379in}{0.747598in}}%
\pgfpathlineto{\pgfqpoint{4.558975in}{0.665182in}}%
\pgfpathlineto{\pgfqpoint{4.559839in}{0.678881in}}%
\pgfpathlineto{\pgfqpoint{4.560705in}{0.674999in}}%
\pgfpathlineto{\pgfqpoint{4.561571in}{0.682470in}}%
\pgfpathlineto{\pgfqpoint{4.562436in}{0.721921in}}%
\pgfpathlineto{\pgfqpoint{4.563300in}{0.705289in}}%
\pgfpathlineto{\pgfqpoint{4.564162in}{0.657195in}}%
\pgfpathlineto{\pgfqpoint{4.565028in}{0.667195in}}%
\pgfpathlineto{\pgfqpoint{4.566758in}{0.696426in}}%
\pgfpathlineto{\pgfqpoint{4.567622in}{0.674301in}}%
\pgfpathlineto{\pgfqpoint{4.568487in}{0.741077in}}%
\pgfpathlineto{\pgfqpoint{4.569351in}{0.617010in}}%
\pgfpathlineto{\pgfqpoint{4.571948in}{0.751480in}}%
\pgfpathlineto{\pgfqpoint{4.573680in}{0.686023in}}%
\pgfpathlineto{\pgfqpoint{4.574545in}{0.723788in}}%
\pgfpathlineto{\pgfqpoint{4.575410in}{0.644815in}}%
\pgfpathlineto{\pgfqpoint{4.577141in}{0.746867in}}%
\pgfpathlineto{\pgfqpoint{4.578005in}{0.711226in}}%
\pgfpathlineto{\pgfqpoint{4.578869in}{0.583057in}}%
\pgfpathlineto{\pgfqpoint{4.579732in}{0.765293in}}%
\pgfpathlineto{\pgfqpoint{4.583189in}{0.584852in}}%
\pgfpathlineto{\pgfqpoint{4.584916in}{0.663825in}}%
\pgfpathlineto{\pgfqpoint{4.586645in}{0.674374in}}%
\pgfpathlineto{\pgfqpoint{4.587511in}{0.767266in}}%
\pgfpathlineto{\pgfqpoint{4.588377in}{0.662031in}}%
\pgfpathlineto{\pgfqpoint{4.590109in}{0.758001in}}%
\pgfpathlineto{\pgfqpoint{4.591841in}{0.690786in}}%
\pgfpathlineto{\pgfqpoint{4.592706in}{0.751041in}}%
\pgfpathlineto{\pgfqpoint{4.593571in}{0.670127in}}%
\pgfpathlineto{\pgfqpoint{4.594436in}{0.730821in}}%
\pgfpathlineto{\pgfqpoint{4.595300in}{0.667195in}}%
\pgfpathlineto{\pgfqpoint{4.596165in}{0.673716in}}%
\pgfpathlineto{\pgfqpoint{4.597894in}{0.684927in}}%
\pgfpathlineto{\pgfqpoint{4.598760in}{0.634266in}}%
\pgfpathlineto{\pgfqpoint{4.599625in}{0.689357in}}%
\pgfpathlineto{\pgfqpoint{4.600489in}{0.647194in}}%
\pgfpathlineto{\pgfqpoint{4.603082in}{0.732689in}}%
\pgfpathlineto{\pgfqpoint{4.603947in}{0.667301in}}%
\pgfpathlineto{\pgfqpoint{4.604812in}{0.711514in}}%
\pgfpathlineto{\pgfqpoint{4.607407in}{0.643089in}}%
\pgfpathlineto{\pgfqpoint{4.609137in}{0.657816in}}%
\pgfpathlineto{\pgfqpoint{4.610002in}{0.688585in}}%
\pgfpathlineto{\pgfqpoint{4.610868in}{0.675803in}}%
\pgfpathlineto{\pgfqpoint{4.613459in}{0.739429in}}%
\pgfpathlineto{\pgfqpoint{4.615186in}{0.612251in}}%
\pgfpathlineto{\pgfqpoint{4.616918in}{0.717231in}}%
\pgfpathlineto{\pgfqpoint{4.617784in}{0.652286in}}%
\pgfpathlineto{\pgfqpoint{4.619515in}{0.716131in}}%
\pgfpathlineto{\pgfqpoint{4.620380in}{0.676607in}}%
\pgfpathlineto{\pgfqpoint{4.621246in}{0.683238in}}%
\pgfpathlineto{\pgfqpoint{4.622110in}{0.681845in}}%
\pgfpathlineto{\pgfqpoint{4.622974in}{0.716756in}}%
\pgfpathlineto{\pgfqpoint{4.623839in}{0.596684in}}%
\pgfpathlineto{\pgfqpoint{4.626436in}{0.691627in}}%
\pgfpathlineto{\pgfqpoint{4.627301in}{0.696243in}}%
\pgfpathlineto{\pgfqpoint{4.629028in}{0.756571in}}%
\pgfpathlineto{\pgfqpoint{4.629894in}{0.735141in}}%
\pgfpathlineto{\pgfqpoint{4.630756in}{0.748991in}}%
\pgfpathlineto{\pgfqpoint{4.631622in}{0.712103in}}%
\pgfpathlineto{\pgfqpoint{4.632489in}{0.776313in}}%
\pgfpathlineto{\pgfqpoint{4.633355in}{0.633312in}}%
\pgfpathlineto{\pgfqpoint{4.634220in}{0.728547in}}%
\pgfpathlineto{\pgfqpoint{4.635081in}{0.724812in}}%
\pgfpathlineto{\pgfqpoint{4.635945in}{0.712505in}}%
\pgfpathlineto{\pgfqpoint{4.637675in}{0.805141in}}%
\pgfpathlineto{\pgfqpoint{4.640272in}{0.662798in}}%
\pgfpathlineto{\pgfqpoint{4.642003in}{0.771002in}}%
\pgfpathlineto{\pgfqpoint{4.642868in}{0.724227in}}%
\pgfpathlineto{\pgfqpoint{4.643733in}{0.748256in}}%
\pgfpathlineto{\pgfqpoint{4.645460in}{0.692357in}}%
\pgfpathlineto{\pgfqpoint{4.646326in}{0.722651in}}%
\pgfpathlineto{\pgfqpoint{4.648057in}{0.678914in}}%
\pgfpathlineto{\pgfqpoint{4.649788in}{0.705322in}}%
\pgfpathlineto{\pgfqpoint{4.650652in}{0.654812in}}%
\pgfpathlineto{\pgfqpoint{4.651516in}{0.706020in}}%
\pgfpathlineto{\pgfqpoint{4.652381in}{0.671918in}}%
\pgfpathlineto{\pgfqpoint{4.653247in}{0.699207in}}%
\pgfpathlineto{\pgfqpoint{4.654112in}{0.685435in}}%
\pgfpathlineto{\pgfqpoint{4.654976in}{0.698915in}}%
\pgfpathlineto{\pgfqpoint{4.655843in}{0.794300in}}%
\pgfpathlineto{\pgfqpoint{4.657572in}{0.703787in}}%
\pgfpathlineto{\pgfqpoint{4.658437in}{0.734264in}}%
\pgfpathlineto{\pgfqpoint{4.659302in}{0.727707in}}%
\pgfpathlineto{\pgfqpoint{4.660168in}{0.661040in}}%
\pgfpathlineto{\pgfqpoint{4.662761in}{0.792944in}}%
\pgfpathlineto{\pgfqpoint{4.664490in}{0.709281in}}%
\pgfpathlineto{\pgfqpoint{4.665354in}{0.700929in}}%
\pgfpathlineto{\pgfqpoint{4.666218in}{0.665693in}}%
\pgfpathlineto{\pgfqpoint{4.667082in}{0.722578in}}%
\pgfpathlineto{\pgfqpoint{4.667948in}{0.684923in}}%
\pgfpathlineto{\pgfqpoint{4.668812in}{0.759686in}}%
\pgfpathlineto{\pgfqpoint{4.669677in}{0.605438in}}%
\pgfpathlineto{\pgfqpoint{4.670542in}{0.711957in}}%
\pgfpathlineto{\pgfqpoint{4.672272in}{0.620234in}}%
\pgfpathlineto{\pgfqpoint{4.673137in}{0.728182in}}%
\pgfpathlineto{\pgfqpoint{4.674000in}{0.705801in}}%
\pgfpathlineto{\pgfqpoint{4.674866in}{0.702504in}}%
\pgfpathlineto{\pgfqpoint{4.675730in}{0.720930in}}%
\pgfpathlineto{\pgfqpoint{4.676595in}{0.765764in}}%
\pgfpathlineto{\pgfqpoint{4.677460in}{0.705070in}}%
\pgfpathlineto{\pgfqpoint{4.678325in}{0.723017in}}%
\pgfpathlineto{\pgfqpoint{4.680055in}{0.690672in}}%
\pgfpathlineto{\pgfqpoint{4.681784in}{0.726533in}}%
\pgfpathlineto{\pgfqpoint{4.682650in}{0.791368in}}%
\pgfpathlineto{\pgfqpoint{4.683516in}{0.663310in}}%
\pgfpathlineto{\pgfqpoint{4.685244in}{0.745324in}}%
\pgfpathlineto{\pgfqpoint{4.686110in}{0.715984in}}%
\pgfpathlineto{\pgfqpoint{4.686974in}{0.747890in}}%
\pgfpathlineto{\pgfqpoint{4.688705in}{0.684192in}}%
\pgfpathlineto{\pgfqpoint{4.689568in}{0.704153in}}%
\pgfpathlineto{\pgfqpoint{4.690432in}{0.772139in}}%
\pgfpathlineto{\pgfqpoint{4.691297in}{0.755361in}}%
\pgfpathlineto{\pgfqpoint{4.692162in}{0.719062in}}%
\pgfpathlineto{\pgfqpoint{4.693027in}{0.756425in}}%
\pgfpathlineto{\pgfqpoint{4.693889in}{0.706280in}}%
\pgfpathlineto{\pgfqpoint{4.695621in}{0.745584in}}%
\pgfpathlineto{\pgfqpoint{4.697352in}{0.727597in}}%
\pgfpathlineto{\pgfqpoint{4.699083in}{0.671077in}}%
\pgfpathlineto{\pgfqpoint{4.699949in}{0.739356in}}%
\pgfpathlineto{\pgfqpoint{4.701675in}{0.685621in}}%
\pgfpathlineto{\pgfqpoint{4.702540in}{0.710235in}}%
\pgfpathlineto{\pgfqpoint{4.703406in}{0.618845in}}%
\pgfpathlineto{\pgfqpoint{4.704271in}{0.634632in}}%
\pgfpathlineto{\pgfqpoint{4.705135in}{0.649687in}}%
\pgfpathlineto{\pgfqpoint{4.706864in}{0.724633in}}%
\pgfpathlineto{\pgfqpoint{4.707730in}{0.707490in}}%
\pgfpathlineto{\pgfqpoint{4.708594in}{0.714596in}}%
\pgfpathlineto{\pgfqpoint{4.709461in}{0.694818in}}%
\pgfpathlineto{\pgfqpoint{4.710325in}{0.733095in}}%
\pgfpathlineto{\pgfqpoint{4.712056in}{0.652838in}}%
\pgfpathlineto{\pgfqpoint{4.713785in}{0.750164in}}%
\pgfpathlineto{\pgfqpoint{4.715513in}{0.657012in}}%
\pgfpathlineto{\pgfqpoint{4.716380in}{0.707234in}}%
\pgfpathlineto{\pgfqpoint{4.717245in}{0.613096in}}%
\pgfpathlineto{\pgfqpoint{4.718111in}{0.691557in}}%
\pgfpathlineto{\pgfqpoint{4.719839in}{0.655071in}}%
\pgfpathlineto{\pgfqpoint{4.720705in}{0.722400in}}%
\pgfpathlineto{\pgfqpoint{4.721569in}{0.652655in}}%
\pgfpathlineto{\pgfqpoint{4.723300in}{0.752032in}}%
\pgfpathlineto{\pgfqpoint{4.725030in}{0.713240in}}%
\pgfpathlineto{\pgfqpoint{4.725894in}{0.719030in}}%
\pgfpathlineto{\pgfqpoint{4.726759in}{0.718185in}}%
\pgfpathlineto{\pgfqpoint{4.729353in}{0.606867in}}%
\pgfpathlineto{\pgfqpoint{4.730215in}{0.724706in}}%
\pgfpathlineto{\pgfqpoint{4.731080in}{0.682434in}}%
\pgfpathlineto{\pgfqpoint{4.731947in}{0.692873in}}%
\pgfpathlineto{\pgfqpoint{4.732812in}{0.756133in}}%
\pgfpathlineto{\pgfqpoint{4.733678in}{0.739063in}}%
\pgfpathlineto{\pgfqpoint{4.734543in}{0.691663in}}%
\pgfpathlineto{\pgfqpoint{4.735409in}{0.703641in}}%
\pgfpathlineto{\pgfqpoint{4.736271in}{0.729976in}}%
\pgfpathlineto{\pgfqpoint{4.737133in}{0.720784in}}%
\pgfpathlineto{\pgfqpoint{4.737997in}{0.696462in}}%
\pgfpathlineto{\pgfqpoint{4.739729in}{0.740858in}}%
\pgfpathlineto{\pgfqpoint{4.741460in}{0.698111in}}%
\pgfpathlineto{\pgfqpoint{4.742327in}{0.779870in}}%
\pgfpathlineto{\pgfqpoint{4.743193in}{0.719208in}}%
\pgfpathlineto{\pgfqpoint{4.744060in}{0.754923in}}%
\pgfpathlineto{\pgfqpoint{4.746655in}{0.660049in}}%
\pgfpathlineto{\pgfqpoint{4.747518in}{0.667451in}}%
\pgfpathlineto{\pgfqpoint{4.748384in}{0.718291in}}%
\pgfpathlineto{\pgfqpoint{4.749246in}{0.715911in}}%
\pgfpathlineto{\pgfqpoint{4.750112in}{0.716902in}}%
\pgfpathlineto{\pgfqpoint{4.752708in}{0.762800in}}%
\pgfpathlineto{\pgfqpoint{4.754440in}{0.694814in}}%
\pgfpathlineto{\pgfqpoint{4.755305in}{0.745803in}}%
\pgfpathlineto{\pgfqpoint{4.757035in}{0.659505in}}%
\pgfpathlineto{\pgfqpoint{4.758765in}{0.757343in}}%
\pgfpathlineto{\pgfqpoint{4.760492in}{0.780747in}}%
\pgfpathlineto{\pgfqpoint{4.761358in}{0.704778in}}%
\pgfpathlineto{\pgfqpoint{4.763954in}{0.829978in}}%
\pgfpathlineto{\pgfqpoint{4.766549in}{0.688037in}}%
\pgfpathlineto{\pgfqpoint{4.767415in}{0.744593in}}%
\pgfpathlineto{\pgfqpoint{4.770012in}{0.682617in}}%
\pgfpathlineto{\pgfqpoint{4.770878in}{0.769134in}}%
\pgfpathlineto{\pgfqpoint{4.771744in}{0.715034in}}%
\pgfpathlineto{\pgfqpoint{4.772610in}{0.717048in}}%
\pgfpathlineto{\pgfqpoint{4.773477in}{0.729063in}}%
\pgfpathlineto{\pgfqpoint{4.774342in}{0.690640in}}%
\pgfpathlineto{\pgfqpoint{4.775208in}{0.744447in}}%
\pgfpathlineto{\pgfqpoint{4.776073in}{0.691444in}}%
\pgfpathlineto{\pgfqpoint{4.777802in}{0.765106in}}%
\pgfpathlineto{\pgfqpoint{4.779534in}{0.682690in}}%
\pgfpathlineto{\pgfqpoint{4.780401in}{0.721847in}}%
\pgfpathlineto{\pgfqpoint{4.781267in}{0.637929in}}%
\pgfpathlineto{\pgfqpoint{4.782998in}{0.800967in}}%
\pgfpathlineto{\pgfqpoint{4.783865in}{0.735766in}}%
\pgfpathlineto{\pgfqpoint{4.784730in}{0.735986in}}%
\pgfpathlineto{\pgfqpoint{4.785596in}{0.732360in}}%
\pgfpathlineto{\pgfqpoint{4.786461in}{0.795875in}}%
\pgfpathlineto{\pgfqpoint{4.787327in}{0.751187in}}%
\pgfpathlineto{\pgfqpoint{4.788193in}{0.795583in}}%
\pgfpathlineto{\pgfqpoint{4.789059in}{0.704778in}}%
\pgfpathlineto{\pgfqpoint{4.789925in}{0.742799in}}%
\pgfpathlineto{\pgfqpoint{4.792520in}{0.696974in}}%
\pgfpathlineto{\pgfqpoint{4.793384in}{0.776719in}}%
\pgfpathlineto{\pgfqpoint{4.795115in}{0.689503in}}%
\pgfpathlineto{\pgfqpoint{4.795980in}{0.753055in}}%
\pgfpathlineto{\pgfqpoint{4.796845in}{0.721076in}}%
\pgfpathlineto{\pgfqpoint{4.798574in}{0.788843in}}%
\pgfpathlineto{\pgfqpoint{4.801166in}{0.717633in}}%
\pgfpathlineto{\pgfqpoint{4.802893in}{0.681553in}}%
\pgfpathlineto{\pgfqpoint{4.804623in}{0.715692in}}%
\pgfpathlineto{\pgfqpoint{4.805486in}{0.705801in}}%
\pgfpathlineto{\pgfqpoint{4.806350in}{0.681845in}}%
\pgfpathlineto{\pgfqpoint{4.808078in}{0.821845in}}%
\pgfpathlineto{\pgfqpoint{4.809809in}{0.668365in}}%
\pgfpathlineto{\pgfqpoint{4.810674in}{0.799392in}}%
\pgfpathlineto{\pgfqpoint{4.811537in}{0.732944in}}%
\pgfpathlineto{\pgfqpoint{4.812399in}{0.772066in}}%
\pgfpathlineto{\pgfqpoint{4.814129in}{0.665693in}}%
\pgfpathlineto{\pgfqpoint{4.815858in}{0.760088in}}%
\pgfpathlineto{\pgfqpoint{4.816722in}{0.747562in}}%
\pgfpathlineto{\pgfqpoint{4.817588in}{0.734264in}}%
\pgfpathlineto{\pgfqpoint{4.818454in}{0.675219in}}%
\pgfpathlineto{\pgfqpoint{4.821048in}{0.732506in}}%
\pgfpathlineto{\pgfqpoint{4.821914in}{0.672580in}}%
\pgfpathlineto{\pgfqpoint{4.823647in}{0.715984in}}%
\pgfpathlineto{\pgfqpoint{4.824513in}{0.682064in}}%
\pgfpathlineto{\pgfqpoint{4.825379in}{0.698915in}}%
\pgfpathlineto{\pgfqpoint{4.826244in}{0.750672in}}%
\pgfpathlineto{\pgfqpoint{4.827110in}{0.723935in}}%
\pgfpathlineto{\pgfqpoint{4.827976in}{0.640747in}}%
\pgfpathlineto{\pgfqpoint{4.828842in}{0.669063in}}%
\pgfpathlineto{\pgfqpoint{4.829709in}{0.649760in}}%
\pgfpathlineto{\pgfqpoint{4.831440in}{0.691078in}}%
\pgfpathlineto{\pgfqpoint{4.832305in}{0.693312in}}%
\pgfpathlineto{\pgfqpoint{4.833170in}{0.643897in}}%
\pgfpathlineto{\pgfqpoint{4.834033in}{0.695252in}}%
\pgfpathlineto{\pgfqpoint{4.834897in}{0.650491in}}%
\pgfpathlineto{\pgfqpoint{4.835762in}{0.656022in}}%
\pgfpathlineto{\pgfqpoint{4.837494in}{0.745105in}}%
\pgfpathlineto{\pgfqpoint{4.839223in}{0.638733in}}%
\pgfpathlineto{\pgfqpoint{4.840088in}{0.720930in}}%
\pgfpathlineto{\pgfqpoint{4.840953in}{0.710381in}}%
\pgfpathlineto{\pgfqpoint{4.841818in}{0.762321in}}%
\pgfpathlineto{\pgfqpoint{4.842682in}{0.699098in}}%
\pgfpathlineto{\pgfqpoint{4.842682in}{0.699098in}}%
\pgfusepath{stroke}%
\end{pgfscope}%
\begin{pgfscope}%
\pgfsetrectcap%
\pgfsetmiterjoin%
\pgfsetlinewidth{0.803000pt}%
\definecolor{currentstroke}{rgb}{0.000000,0.000000,0.000000}%
\pgfsetstrokecolor{currentstroke}%
\pgfsetdash{}{0pt}%
\pgfpathmoveto{\pgfqpoint{0.483776in}{0.538014in}}%
\pgfpathlineto{\pgfqpoint{0.483776in}{1.122895in}}%
\pgfusepath{stroke}%
\end{pgfscope}%
\begin{pgfscope}%
\pgfsetrectcap%
\pgfsetmiterjoin%
\pgfsetlinewidth{0.803000pt}%
\definecolor{currentstroke}{rgb}{0.000000,0.000000,0.000000}%
\pgfsetstrokecolor{currentstroke}%
\pgfsetdash{}{0pt}%
\pgfpathmoveto{\pgfqpoint{5.050249in}{0.538014in}}%
\pgfpathlineto{\pgfqpoint{5.050249in}{1.122895in}}%
\pgfusepath{stroke}%
\end{pgfscope}%
\begin{pgfscope}%
\pgfsetrectcap%
\pgfsetmiterjoin%
\pgfsetlinewidth{0.803000pt}%
\definecolor{currentstroke}{rgb}{0.000000,0.000000,0.000000}%
\pgfsetstrokecolor{currentstroke}%
\pgfsetdash{}{0pt}%
\pgfpathmoveto{\pgfqpoint{0.483776in}{0.538014in}}%
\pgfpathlineto{\pgfqpoint{5.050249in}{0.538014in}}%
\pgfusepath{stroke}%
\end{pgfscope}%
\begin{pgfscope}%
\pgfsetrectcap%
\pgfsetmiterjoin%
\pgfsetlinewidth{0.803000pt}%
\definecolor{currentstroke}{rgb}{0.000000,0.000000,0.000000}%
\pgfsetstrokecolor{currentstroke}%
\pgfsetdash{}{0pt}%
\pgfpathmoveto{\pgfqpoint{0.483776in}{1.122895in}}%
\pgfpathlineto{\pgfqpoint{5.050249in}{1.122895in}}%
\pgfusepath{stroke}%
\end{pgfscope}%
\end{pgfpicture}%
\makeatother%
\endgroup%

    \caption{Popcorn noise in different samples of the LM399 over a \qty{24}{\hour} period.}
    \label{fig:popcorn_noise_lm399}
\end{figure}

Figure \ref{fig:popcorn_noise_lm399} shows two samples of the LM399, that exhibit popcorn noise, while the last one does not.

The sources of popcorn noise in semiconductor devices are not yet fully understood, but some sources have been identified. Defects in the semiconductor crystal lattice and contamination of the semiconductor material have been linked to popcorn noise \cite{technote_ti_popcorn_noise}. This problem has improved over the years as manufacturing processes and wafer quality has evolved. Unfortunately the LM399 is built around a process from 1991, as can be seen etched into the die \cite{lm399_richi}.

The popcorn noise caused by defects and contamination can be reduced by lowering the strain on the lattice and removing surface contaminants on the die. This can be can be achieved using a high-temperature burn-in process. Manufacturers like Fluke and Keysight use similar techniques in their products. Fluke, for example, uses a period of \qty{60}{\day} burn-in for their references \cite{zener_popcorn_noise}.

Fortunately, the LM399 is a heated reference, which regulates its die to \qty{90}{\celsius} when turned on, so it is only required put the diodes in a simple test circuit and wait. The use of a separate test setup instead of the final circuit has both advantages and disadvantages. The disadvantage is, that the Zener diode will subjected to mechanical stress when soldered, this stress will not be removed by the burn-in process as it happens after the testing, when diode is soldered into the final circuit, but this mainly affects the voltage drift properties of the Zener diode and not the popcorn noise. The drift of the diode is also only of secondary concern in our setup, as the drift is mainly caused by the reference resistors used and are typically at least an order of magnitude worse than the the drift of the diode judging by the data sheet \cite{datasheet_LM399,datasheet_VPR}.

The advantages of testing the Zener diodes separately, on the other hand, are, that more diodes can be tested at the same time, as a special compact test fixture can be used. It is also simpler to remove the diodes from the test fixture, because they be socketed. Therefore for our application a separate test board was used. Building this test setup is detailed in the next sections.

\subsection{Building a Test Setup for Zener Diodes}
There are several ways to measure the popcorn noise of semiconductor devices. The most trivial one is to directly monitor the device in the time-domain. In this case, the Zener voltage can be monitored with a long-scale multimeter. It requires a low noise DMM, that can reliably distinguish between both voltage levels, which are about \qty{4}{\micro \volt} apart.
A related option is to use a second reference, whose voltage is similar to the device under test (DUT). Measuring the the voltage difference between the two references, less resolution is required. Directly comparing the difference of two references using a millivolt meter is commonly done when intercomparing primary voltage references. This method, however, increases the measurement noise by a factor of $\sqrt{2}$, if both references produce the same level of uncorrelated noise. The noise of the LM399 with a \qty{100}{\plc} integration time (\qty{2}{\second}) is about \qty{1.5}{\micro \volt_{pp}} as can be determined from the data in figure \ref{fig:popcorn_noise_lm399}.

\begin{figure}[ht]
    \centering
    %%% Creator: Matplotlib, PGF backend
%%
%% To include the figure in your LaTeX document, write
%%   \input{<filename>.pgf}
%%
%% Make sure the required packages are loaded in your preamble
%%   \usepackage{pgf}
%%
%% Also ensure that all the required font packages are loaded; for instance,
%% the lmodern package is sometimes necessary when using math font.
%%   \usepackage{lmodern}
%%
%% Figures using additional raster images can only be included by \input if
%% they are in the same directory as the main LaTeX file. For loading figures
%% from other directories you can use the `import` package
%%   \usepackage{import}
%%
%% and then include the figures with
%%   \import{<path to file>}{<filename>.pgf}
%%
%% Matplotlib used the following preamble
%%   \usepackage{fontspec}
%%
\begingroup%
\makeatletter%
\begin{pgfpicture}%
\pgfpathrectangle{\pgfpointorigin}{\pgfqpoint{5.200000in}{3.210000in}}%
\pgfusepath{use as bounding box, clip}%
\begin{pgfscope}%
\pgfsetbuttcap%
\pgfsetmiterjoin%
\definecolor{currentfill}{rgb}{1.000000,1.000000,1.000000}%
\pgfsetfillcolor{currentfill}%
\pgfsetlinewidth{0.000000pt}%
\definecolor{currentstroke}{rgb}{1.000000,1.000000,1.000000}%
\pgfsetstrokecolor{currentstroke}%
\pgfsetdash{}{0pt}%
\pgfpathmoveto{\pgfqpoint{0.000000in}{0.000000in}}%
\pgfpathlineto{\pgfqpoint{5.200000in}{0.000000in}}%
\pgfpathlineto{\pgfqpoint{5.200000in}{3.210000in}}%
\pgfpathlineto{\pgfqpoint{0.000000in}{3.210000in}}%
\pgfpathlineto{\pgfqpoint{0.000000in}{0.000000in}}%
\pgfpathclose%
\pgfusepath{fill}%
\end{pgfscope}%
\begin{pgfscope}%
\pgfsetbuttcap%
\pgfsetmiterjoin%
\definecolor{currentfill}{rgb}{1.000000,1.000000,1.000000}%
\pgfsetfillcolor{currentfill}%
\pgfsetlinewidth{0.000000pt}%
\definecolor{currentstroke}{rgb}{0.000000,0.000000,0.000000}%
\pgfsetstrokecolor{currentstroke}%
\pgfsetstrokeopacity{0.000000}%
\pgfsetdash{}{0pt}%
\pgfpathmoveto{\pgfqpoint{0.483776in}{2.351653in}}%
\pgfpathlineto{\pgfqpoint{5.050249in}{2.351653in}}%
\pgfpathlineto{\pgfqpoint{5.050249in}{2.936535in}}%
\pgfpathlineto{\pgfqpoint{0.483776in}{2.936535in}}%
\pgfpathlineto{\pgfqpoint{0.483776in}{2.351653in}}%
\pgfpathclose%
\pgfusepath{fill}%
\end{pgfscope}%
\begin{pgfscope}%
\pgfsetbuttcap%
\pgfsetroundjoin%
\definecolor{currentfill}{rgb}{0.000000,0.000000,0.000000}%
\pgfsetfillcolor{currentfill}%
\pgfsetlinewidth{0.803000pt}%
\definecolor{currentstroke}{rgb}{0.000000,0.000000,0.000000}%
\pgfsetstrokecolor{currentstroke}%
\pgfsetdash{}{0pt}%
\pgfsys@defobject{currentmarker}{\pgfqpoint{0.000000in}{-0.048611in}}{\pgfqpoint{0.000000in}{0.000000in}}{%
\pgfpathmoveto{\pgfqpoint{0.000000in}{0.000000in}}%
\pgfpathlineto{\pgfqpoint{0.000000in}{-0.048611in}}%
\pgfusepath{stroke,fill}%
}%
\begin{pgfscope}%
\pgfsys@transformshift{0.691021in}{2.351653in}%
\pgfsys@useobject{currentmarker}{}%
\end{pgfscope}%
\end{pgfscope}%
\begin{pgfscope}%
\pgfsetbuttcap%
\pgfsetroundjoin%
\definecolor{currentfill}{rgb}{0.000000,0.000000,0.000000}%
\pgfsetfillcolor{currentfill}%
\pgfsetlinewidth{0.803000pt}%
\definecolor{currentstroke}{rgb}{0.000000,0.000000,0.000000}%
\pgfsetstrokecolor{currentstroke}%
\pgfsetdash{}{0pt}%
\pgfsys@defobject{currentmarker}{\pgfqpoint{0.000000in}{-0.048611in}}{\pgfqpoint{0.000000in}{0.000000in}}{%
\pgfpathmoveto{\pgfqpoint{0.000000in}{0.000000in}}%
\pgfpathlineto{\pgfqpoint{0.000000in}{-0.048611in}}%
\pgfusepath{stroke,fill}%
}%
\begin{pgfscope}%
\pgfsys@transformshift{1.210067in}{2.351653in}%
\pgfsys@useobject{currentmarker}{}%
\end{pgfscope}%
\end{pgfscope}%
\begin{pgfscope}%
\pgfsetbuttcap%
\pgfsetroundjoin%
\definecolor{currentfill}{rgb}{0.000000,0.000000,0.000000}%
\pgfsetfillcolor{currentfill}%
\pgfsetlinewidth{0.803000pt}%
\definecolor{currentstroke}{rgb}{0.000000,0.000000,0.000000}%
\pgfsetstrokecolor{currentstroke}%
\pgfsetdash{}{0pt}%
\pgfsys@defobject{currentmarker}{\pgfqpoint{0.000000in}{-0.048611in}}{\pgfqpoint{0.000000in}{0.000000in}}{%
\pgfpathmoveto{\pgfqpoint{0.000000in}{0.000000in}}%
\pgfpathlineto{\pgfqpoint{0.000000in}{-0.048611in}}%
\pgfusepath{stroke,fill}%
}%
\begin{pgfscope}%
\pgfsys@transformshift{1.729114in}{2.351653in}%
\pgfsys@useobject{currentmarker}{}%
\end{pgfscope}%
\end{pgfscope}%
\begin{pgfscope}%
\pgfsetbuttcap%
\pgfsetroundjoin%
\definecolor{currentfill}{rgb}{0.000000,0.000000,0.000000}%
\pgfsetfillcolor{currentfill}%
\pgfsetlinewidth{0.803000pt}%
\definecolor{currentstroke}{rgb}{0.000000,0.000000,0.000000}%
\pgfsetstrokecolor{currentstroke}%
\pgfsetdash{}{0pt}%
\pgfsys@defobject{currentmarker}{\pgfqpoint{0.000000in}{-0.048611in}}{\pgfqpoint{0.000000in}{0.000000in}}{%
\pgfpathmoveto{\pgfqpoint{0.000000in}{0.000000in}}%
\pgfpathlineto{\pgfqpoint{0.000000in}{-0.048611in}}%
\pgfusepath{stroke,fill}%
}%
\begin{pgfscope}%
\pgfsys@transformshift{2.248160in}{2.351653in}%
\pgfsys@useobject{currentmarker}{}%
\end{pgfscope}%
\end{pgfscope}%
\begin{pgfscope}%
\pgfsetbuttcap%
\pgfsetroundjoin%
\definecolor{currentfill}{rgb}{0.000000,0.000000,0.000000}%
\pgfsetfillcolor{currentfill}%
\pgfsetlinewidth{0.803000pt}%
\definecolor{currentstroke}{rgb}{0.000000,0.000000,0.000000}%
\pgfsetstrokecolor{currentstroke}%
\pgfsetdash{}{0pt}%
\pgfsys@defobject{currentmarker}{\pgfqpoint{0.000000in}{-0.048611in}}{\pgfqpoint{0.000000in}{0.000000in}}{%
\pgfpathmoveto{\pgfqpoint{0.000000in}{0.000000in}}%
\pgfpathlineto{\pgfqpoint{0.000000in}{-0.048611in}}%
\pgfusepath{stroke,fill}%
}%
\begin{pgfscope}%
\pgfsys@transformshift{2.767206in}{2.351653in}%
\pgfsys@useobject{currentmarker}{}%
\end{pgfscope}%
\end{pgfscope}%
\begin{pgfscope}%
\pgfsetbuttcap%
\pgfsetroundjoin%
\definecolor{currentfill}{rgb}{0.000000,0.000000,0.000000}%
\pgfsetfillcolor{currentfill}%
\pgfsetlinewidth{0.803000pt}%
\definecolor{currentstroke}{rgb}{0.000000,0.000000,0.000000}%
\pgfsetstrokecolor{currentstroke}%
\pgfsetdash{}{0pt}%
\pgfsys@defobject{currentmarker}{\pgfqpoint{0.000000in}{-0.048611in}}{\pgfqpoint{0.000000in}{0.000000in}}{%
\pgfpathmoveto{\pgfqpoint{0.000000in}{0.000000in}}%
\pgfpathlineto{\pgfqpoint{0.000000in}{-0.048611in}}%
\pgfusepath{stroke,fill}%
}%
\begin{pgfscope}%
\pgfsys@transformshift{3.286252in}{2.351653in}%
\pgfsys@useobject{currentmarker}{}%
\end{pgfscope}%
\end{pgfscope}%
\begin{pgfscope}%
\pgfsetbuttcap%
\pgfsetroundjoin%
\definecolor{currentfill}{rgb}{0.000000,0.000000,0.000000}%
\pgfsetfillcolor{currentfill}%
\pgfsetlinewidth{0.803000pt}%
\definecolor{currentstroke}{rgb}{0.000000,0.000000,0.000000}%
\pgfsetstrokecolor{currentstroke}%
\pgfsetdash{}{0pt}%
\pgfsys@defobject{currentmarker}{\pgfqpoint{0.000000in}{-0.048611in}}{\pgfqpoint{0.000000in}{0.000000in}}{%
\pgfpathmoveto{\pgfqpoint{0.000000in}{0.000000in}}%
\pgfpathlineto{\pgfqpoint{0.000000in}{-0.048611in}}%
\pgfusepath{stroke,fill}%
}%
\begin{pgfscope}%
\pgfsys@transformshift{3.805298in}{2.351653in}%
\pgfsys@useobject{currentmarker}{}%
\end{pgfscope}%
\end{pgfscope}%
\begin{pgfscope}%
\pgfsetbuttcap%
\pgfsetroundjoin%
\definecolor{currentfill}{rgb}{0.000000,0.000000,0.000000}%
\pgfsetfillcolor{currentfill}%
\pgfsetlinewidth{0.803000pt}%
\definecolor{currentstroke}{rgb}{0.000000,0.000000,0.000000}%
\pgfsetstrokecolor{currentstroke}%
\pgfsetdash{}{0pt}%
\pgfsys@defobject{currentmarker}{\pgfqpoint{0.000000in}{-0.048611in}}{\pgfqpoint{0.000000in}{0.000000in}}{%
\pgfpathmoveto{\pgfqpoint{0.000000in}{0.000000in}}%
\pgfpathlineto{\pgfqpoint{0.000000in}{-0.048611in}}%
\pgfusepath{stroke,fill}%
}%
\begin{pgfscope}%
\pgfsys@transformshift{4.324344in}{2.351653in}%
\pgfsys@useobject{currentmarker}{}%
\end{pgfscope}%
\end{pgfscope}%
\begin{pgfscope}%
\pgfsetbuttcap%
\pgfsetroundjoin%
\definecolor{currentfill}{rgb}{0.000000,0.000000,0.000000}%
\pgfsetfillcolor{currentfill}%
\pgfsetlinewidth{0.803000pt}%
\definecolor{currentstroke}{rgb}{0.000000,0.000000,0.000000}%
\pgfsetstrokecolor{currentstroke}%
\pgfsetdash{}{0pt}%
\pgfsys@defobject{currentmarker}{\pgfqpoint{0.000000in}{-0.048611in}}{\pgfqpoint{0.000000in}{0.000000in}}{%
\pgfpathmoveto{\pgfqpoint{0.000000in}{0.000000in}}%
\pgfpathlineto{\pgfqpoint{0.000000in}{-0.048611in}}%
\pgfusepath{stroke,fill}%
}%
\begin{pgfscope}%
\pgfsys@transformshift{4.843390in}{2.351653in}%
\pgfsys@useobject{currentmarker}{}%
\end{pgfscope}%
\end{pgfscope}%
\begin{pgfscope}%
\pgfsetbuttcap%
\pgfsetroundjoin%
\definecolor{currentfill}{rgb}{0.000000,0.000000,0.000000}%
\pgfsetfillcolor{currentfill}%
\pgfsetlinewidth{0.803000pt}%
\definecolor{currentstroke}{rgb}{0.000000,0.000000,0.000000}%
\pgfsetstrokecolor{currentstroke}%
\pgfsetdash{}{0pt}%
\pgfsys@defobject{currentmarker}{\pgfqpoint{-0.048611in}{0.000000in}}{\pgfqpoint{-0.000000in}{0.000000in}}{%
\pgfpathmoveto{\pgfqpoint{-0.000000in}{0.000000in}}%
\pgfpathlineto{\pgfqpoint{-0.048611in}{0.000000in}}%
\pgfusepath{stroke,fill}%
}%
\begin{pgfscope}%
\pgfsys@transformshift{0.483776in}{2.532831in}%
\pgfsys@useobject{currentmarker}{}%
\end{pgfscope}%
\end{pgfscope}%
\begin{pgfscope}%
\definecolor{textcolor}{rgb}{0.000000,0.000000,0.000000}%
\pgfsetstrokecolor{textcolor}%
\pgfsetfillcolor{textcolor}%
\pgftext[x=0.327525in, y=2.494275in, left, base]{\color{textcolor}\rmfamily\fontsize{8.000000}{9.600000}\selectfont \(\displaystyle {0}\)}%
\end{pgfscope}%
\begin{pgfscope}%
\pgfsetbuttcap%
\pgfsetroundjoin%
\definecolor{currentfill}{rgb}{0.000000,0.000000,0.000000}%
\pgfsetfillcolor{currentfill}%
\pgfsetlinewidth{0.803000pt}%
\definecolor{currentstroke}{rgb}{0.000000,0.000000,0.000000}%
\pgfsetstrokecolor{currentstroke}%
\pgfsetdash{}{0pt}%
\pgfsys@defobject{currentmarker}{\pgfqpoint{-0.048611in}{0.000000in}}{\pgfqpoint{-0.000000in}{0.000000in}}{%
\pgfpathmoveto{\pgfqpoint{-0.000000in}{0.000000in}}%
\pgfpathlineto{\pgfqpoint{-0.048611in}{0.000000in}}%
\pgfusepath{stroke,fill}%
}%
\begin{pgfscope}%
\pgfsys@transformshift{0.483776in}{2.740006in}%
\pgfsys@useobject{currentmarker}{}%
\end{pgfscope}%
\end{pgfscope}%
\begin{pgfscope}%
\definecolor{textcolor}{rgb}{0.000000,0.000000,0.000000}%
\pgfsetstrokecolor{textcolor}%
\pgfsetfillcolor{textcolor}%
\pgftext[x=0.327525in, y=2.701450in, left, base]{\color{textcolor}\rmfamily\fontsize{8.000000}{9.600000}\selectfont \(\displaystyle {5}\)}%
\end{pgfscope}%
\begin{pgfscope}%
\definecolor{textcolor}{rgb}{0.000000,0.000000,0.000000}%
\pgfsetstrokecolor{textcolor}%
\pgfsetfillcolor{textcolor}%
\pgftext[x=0.483776in,y=2.978201in,left,base]{\color{textcolor}\rmfamily\fontsize{8.000000}{9.600000}\selectfont \(\displaystyle \times{10^{\ensuremath{-}6}}{}\)}%
\end{pgfscope}%
\begin{pgfscope}%
\pgfpathrectangle{\pgfqpoint{0.483776in}{2.351653in}}{\pgfqpoint{4.566474in}{0.584881in}}%
\pgfusepath{clip}%
\pgfsetrectcap%
\pgfsetroundjoin%
\pgfsetlinewidth{0.501875pt}%
\definecolor{currentstroke}{rgb}{0.121569,0.466667,0.705882}%
\pgfsetstrokecolor{currentstroke}%
\pgfsetstrokeopacity{0.700000}%
\pgfsetdash{}{0pt}%
\pgfpathmoveto{\pgfqpoint{0.691343in}{2.530219in}}%
\pgfpathlineto{\pgfqpoint{0.692205in}{2.507777in}}%
\pgfpathlineto{\pgfqpoint{0.693071in}{2.590474in}}%
\pgfpathlineto{\pgfqpoint{0.694800in}{2.506363in}}%
\pgfpathlineto{\pgfqpoint{0.695666in}{2.545836in}}%
\pgfpathlineto{\pgfqpoint{0.696532in}{2.498186in}}%
\pgfpathlineto{\pgfqpoint{0.697397in}{2.501383in}}%
\pgfpathlineto{\pgfqpoint{0.699128in}{2.538888in}}%
\pgfpathlineto{\pgfqpoint{0.699993in}{2.536184in}}%
\pgfpathlineto{\pgfqpoint{0.700859in}{2.601850in}}%
\pgfpathlineto{\pgfqpoint{0.701725in}{2.538336in}}%
\pgfpathlineto{\pgfqpoint{0.702589in}{2.553523in}}%
\pgfpathlineto{\pgfqpoint{0.703453in}{2.510576in}}%
\pgfpathlineto{\pgfqpoint{0.706051in}{2.564068in}}%
\pgfpathlineto{\pgfqpoint{0.707780in}{2.536184in}}%
\pgfpathlineto{\pgfqpoint{0.708646in}{2.529913in}}%
\pgfpathlineto{\pgfqpoint{0.709512in}{2.574367in}}%
\pgfpathlineto{\pgfqpoint{0.710377in}{2.521305in}}%
\pgfpathlineto{\pgfqpoint{0.712105in}{2.585309in}}%
\pgfpathlineto{\pgfqpoint{0.712971in}{2.517924in}}%
\pgfpathlineto{\pgfqpoint{0.713837in}{2.551432in}}%
\pgfpathlineto{\pgfqpoint{0.714702in}{2.479803in}}%
\pgfpathlineto{\pgfqpoint{0.716430in}{2.549464in}}%
\pgfpathlineto{\pgfqpoint{0.719030in}{2.475436in}}%
\pgfpathlineto{\pgfqpoint{0.719895in}{2.469535in}}%
\pgfpathlineto{\pgfqpoint{0.720762in}{2.400426in}}%
\pgfpathlineto{\pgfqpoint{0.721627in}{2.506486in}}%
\pgfpathlineto{\pgfqpoint{0.722492in}{2.468950in}}%
\pgfpathlineto{\pgfqpoint{0.723356in}{2.490807in}}%
\pgfpathlineto{\pgfqpoint{0.724219in}{2.547435in}}%
\pgfpathlineto{\pgfqpoint{0.725084in}{2.460004in}}%
\pgfpathlineto{\pgfqpoint{0.725947in}{2.531019in}}%
\pgfpathlineto{\pgfqpoint{0.726812in}{2.496280in}}%
\pgfpathlineto{\pgfqpoint{0.728541in}{2.519030in}}%
\pgfpathlineto{\pgfqpoint{0.729407in}{2.497203in}}%
\pgfpathlineto{\pgfqpoint{0.730271in}{2.584758in}}%
\pgfpathlineto{\pgfqpoint{0.732865in}{2.466092in}}%
\pgfpathlineto{\pgfqpoint{0.733731in}{2.535384in}}%
\pgfpathlineto{\pgfqpoint{0.734597in}{2.484044in}}%
\pgfpathlineto{\pgfqpoint{0.735460in}{2.484475in}}%
\pgfpathlineto{\pgfqpoint{0.736325in}{2.528682in}}%
\pgfpathlineto{\pgfqpoint{0.737190in}{2.491331in}}%
\pgfpathlineto{\pgfqpoint{0.738056in}{2.494989in}}%
\pgfpathlineto{\pgfqpoint{0.738920in}{2.504888in}}%
\pgfpathlineto{\pgfqpoint{0.740649in}{2.546144in}}%
\pgfpathlineto{\pgfqpoint{0.742381in}{2.482078in}}%
\pgfpathlineto{\pgfqpoint{0.743246in}{2.512636in}}%
\pgfpathlineto{\pgfqpoint{0.744112in}{2.503659in}}%
\pgfpathlineto{\pgfqpoint{0.744977in}{2.529852in}}%
\pgfpathlineto{\pgfqpoint{0.745840in}{2.522658in}}%
\pgfpathlineto{\pgfqpoint{0.746705in}{2.526285in}}%
\pgfpathlineto{\pgfqpoint{0.747569in}{2.494989in}}%
\pgfpathlineto{\pgfqpoint{0.748435in}{2.558934in}}%
\pgfpathlineto{\pgfqpoint{0.749299in}{2.542886in}}%
\pgfpathlineto{\pgfqpoint{0.750163in}{2.552294in}}%
\pgfpathlineto{\pgfqpoint{0.751028in}{2.540704in}}%
\pgfpathlineto{\pgfqpoint{0.752757in}{2.494006in}}%
\pgfpathlineto{\pgfqpoint{0.753622in}{2.550357in}}%
\pgfpathlineto{\pgfqpoint{0.754488in}{2.440761in}}%
\pgfpathlineto{\pgfqpoint{0.755354in}{2.493268in}}%
\pgfpathlineto{\pgfqpoint{0.756219in}{2.471287in}}%
\pgfpathlineto{\pgfqpoint{0.757950in}{2.535078in}}%
\pgfpathlineto{\pgfqpoint{0.758816in}{2.534524in}}%
\pgfpathlineto{\pgfqpoint{0.759679in}{2.531450in}}%
\pgfpathlineto{\pgfqpoint{0.760542in}{2.495420in}}%
\pgfpathlineto{\pgfqpoint{0.761407in}{2.534339in}}%
\pgfpathlineto{\pgfqpoint{0.762269in}{2.509685in}}%
\pgfpathlineto{\pgfqpoint{0.763135in}{2.514727in}}%
\pgfpathlineto{\pgfqpoint{0.764000in}{2.521429in}}%
\pgfpathlineto{\pgfqpoint{0.764865in}{2.551617in}}%
\pgfpathlineto{\pgfqpoint{0.765730in}{2.507256in}}%
\pgfpathlineto{\pgfqpoint{0.766596in}{2.566988in}}%
\pgfpathlineto{\pgfqpoint{0.767462in}{2.500277in}}%
\pgfpathlineto{\pgfqpoint{0.769191in}{2.567172in}}%
\pgfpathlineto{\pgfqpoint{0.770922in}{2.489763in}}%
\pgfpathlineto{\pgfqpoint{0.771788in}{2.603141in}}%
\pgfpathlineto{\pgfqpoint{0.772652in}{2.517216in}}%
\pgfpathlineto{\pgfqpoint{0.773518in}{2.550939in}}%
\pgfpathlineto{\pgfqpoint{0.774383in}{2.502920in}}%
\pgfpathlineto{\pgfqpoint{0.775249in}{2.504857in}}%
\pgfpathlineto{\pgfqpoint{0.776979in}{2.541778in}}%
\pgfpathlineto{\pgfqpoint{0.779575in}{2.501568in}}%
\pgfpathlineto{\pgfqpoint{0.780439in}{2.532310in}}%
\pgfpathlineto{\pgfqpoint{0.781305in}{2.524870in}}%
\pgfpathlineto{\pgfqpoint{0.782170in}{2.511128in}}%
\pgfpathlineto{\pgfqpoint{0.783036in}{2.593610in}}%
\pgfpathlineto{\pgfqpoint{0.784766in}{2.501321in}}%
\pgfpathlineto{\pgfqpoint{0.785630in}{2.528990in}}%
\pgfpathlineto{\pgfqpoint{0.787360in}{2.487180in}}%
\pgfpathlineto{\pgfqpoint{0.788225in}{2.466951in}}%
\pgfpathlineto{\pgfqpoint{0.789089in}{2.410386in}}%
\pgfpathlineto{\pgfqpoint{0.789954in}{2.538704in}}%
\pgfpathlineto{\pgfqpoint{0.790816in}{2.520444in}}%
\pgfpathlineto{\pgfqpoint{0.791680in}{2.480601in}}%
\pgfpathlineto{\pgfqpoint{0.792546in}{2.576702in}}%
\pgfpathlineto{\pgfqpoint{0.793412in}{2.571906in}}%
\pgfpathlineto{\pgfqpoint{0.794274in}{2.527576in}}%
\pgfpathlineto{\pgfqpoint{0.795138in}{2.567480in}}%
\pgfpathlineto{\pgfqpoint{0.796004in}{2.509254in}}%
\pgfpathlineto{\pgfqpoint{0.796868in}{2.534216in}}%
\pgfpathlineto{\pgfqpoint{0.798599in}{2.477897in}}%
\pgfpathlineto{\pgfqpoint{0.799464in}{2.543440in}}%
\pgfpathlineto{\pgfqpoint{0.800327in}{2.537967in}}%
\pgfpathlineto{\pgfqpoint{0.801192in}{2.472855in}}%
\pgfpathlineto{\pgfqpoint{0.802058in}{2.568894in}}%
\pgfpathlineto{\pgfqpoint{0.803787in}{2.451429in}}%
\pgfpathlineto{\pgfqpoint{0.804652in}{2.464616in}}%
\pgfpathlineto{\pgfqpoint{0.805515in}{2.514727in}}%
\pgfpathlineto{\pgfqpoint{0.806378in}{2.440576in}}%
\pgfpathlineto{\pgfqpoint{0.808107in}{2.534401in}}%
\pgfpathlineto{\pgfqpoint{0.808973in}{2.535938in}}%
\pgfpathlineto{\pgfqpoint{0.809837in}{2.465722in}}%
\pgfpathlineto{\pgfqpoint{0.812431in}{2.522350in}}%
\pgfpathlineto{\pgfqpoint{0.813297in}{2.496218in}}%
\pgfpathlineto{\pgfqpoint{0.814161in}{2.511713in}}%
\pgfpathlineto{\pgfqpoint{0.815026in}{2.557949in}}%
\pgfpathlineto{\pgfqpoint{0.817620in}{2.473284in}}%
\pgfpathlineto{\pgfqpoint{0.818484in}{2.579837in}}%
\pgfpathlineto{\pgfqpoint{0.819349in}{2.551309in}}%
\pgfpathlineto{\pgfqpoint{0.820213in}{2.504272in}}%
\pgfpathlineto{\pgfqpoint{0.821078in}{2.576763in}}%
\pgfpathlineto{\pgfqpoint{0.821942in}{2.542455in}}%
\pgfpathlineto{\pgfqpoint{0.822807in}{2.555642in}}%
\pgfpathlineto{\pgfqpoint{0.823671in}{2.672556in}}%
\pgfpathlineto{\pgfqpoint{0.824536in}{2.516693in}}%
\pgfpathlineto{\pgfqpoint{0.825401in}{2.582297in}}%
\pgfpathlineto{\pgfqpoint{0.826268in}{2.535630in}}%
\pgfpathlineto{\pgfqpoint{0.827133in}{2.539596in}}%
\pgfpathlineto{\pgfqpoint{0.827999in}{2.555244in}}%
\pgfpathlineto{\pgfqpoint{0.828864in}{2.551924in}}%
\pgfpathlineto{\pgfqpoint{0.829730in}{2.505565in}}%
\pgfpathlineto{\pgfqpoint{0.831462in}{2.639784in}}%
\pgfpathlineto{\pgfqpoint{0.832328in}{2.563391in}}%
\pgfpathlineto{\pgfqpoint{0.833193in}{2.597730in}}%
\pgfpathlineto{\pgfqpoint{0.835786in}{2.510053in}}%
\pgfpathlineto{\pgfqpoint{0.836650in}{2.530650in}}%
\pgfpathlineto{\pgfqpoint{0.837516in}{2.501814in}}%
\pgfpathlineto{\pgfqpoint{0.838382in}{2.518968in}}%
\pgfpathlineto{\pgfqpoint{0.840114in}{2.487303in}}%
\pgfpathlineto{\pgfqpoint{0.840980in}{2.522042in}}%
\pgfpathlineto{\pgfqpoint{0.841845in}{2.510145in}}%
\pgfpathlineto{\pgfqpoint{0.842710in}{2.472239in}}%
\pgfpathlineto{\pgfqpoint{0.843575in}{2.501014in}}%
\pgfpathlineto{\pgfqpoint{0.844440in}{2.458220in}}%
\pgfpathlineto{\pgfqpoint{0.845305in}{2.550509in}}%
\pgfpathlineto{\pgfqpoint{0.847033in}{2.473961in}}%
\pgfpathlineto{\pgfqpoint{0.847897in}{2.532125in}}%
\pgfpathlineto{\pgfqpoint{0.849627in}{2.491485in}}%
\pgfpathlineto{\pgfqpoint{0.850491in}{2.489763in}}%
\pgfpathlineto{\pgfqpoint{0.851354in}{2.469073in}}%
\pgfpathlineto{\pgfqpoint{0.853081in}{2.535568in}}%
\pgfpathlineto{\pgfqpoint{0.853945in}{2.515033in}}%
\pgfpathlineto{\pgfqpoint{0.854810in}{2.520690in}}%
\pgfpathlineto{\pgfqpoint{0.855672in}{2.572858in}}%
\pgfpathlineto{\pgfqpoint{0.856537in}{2.558072in}}%
\pgfpathlineto{\pgfqpoint{0.857402in}{2.494250in}}%
\pgfpathlineto{\pgfqpoint{0.858267in}{2.561330in}}%
\pgfpathlineto{\pgfqpoint{0.859131in}{2.549218in}}%
\pgfpathlineto{\pgfqpoint{0.859997in}{2.479187in}}%
\pgfpathlineto{\pgfqpoint{0.860864in}{2.482630in}}%
\pgfpathlineto{\pgfqpoint{0.862597in}{2.580022in}}%
\pgfpathlineto{\pgfqpoint{0.863463in}{2.521796in}}%
\pgfpathlineto{\pgfqpoint{0.864329in}{2.544176in}}%
\pgfpathlineto{\pgfqpoint{0.866062in}{2.483798in}}%
\pgfpathlineto{\pgfqpoint{0.866929in}{2.519643in}}%
\pgfpathlineto{\pgfqpoint{0.867794in}{2.512511in}}%
\pgfpathlineto{\pgfqpoint{0.868659in}{2.511035in}}%
\pgfpathlineto{\pgfqpoint{0.869523in}{2.503841in}}%
\pgfpathlineto{\pgfqpoint{0.870388in}{2.487210in}}%
\pgfpathlineto{\pgfqpoint{0.872119in}{2.555920in}}%
\pgfpathlineto{\pgfqpoint{0.873848in}{2.522902in}}%
\pgfpathlineto{\pgfqpoint{0.874712in}{2.512141in}}%
\pgfpathlineto{\pgfqpoint{0.875577in}{2.538581in}}%
\pgfpathlineto{\pgfqpoint{0.876442in}{2.496955in}}%
\pgfpathlineto{\pgfqpoint{0.877309in}{2.512388in}}%
\pgfpathlineto{\pgfqpoint{0.878174in}{2.500275in}}%
\pgfpathlineto{\pgfqpoint{0.879035in}{2.516323in}}%
\pgfpathlineto{\pgfqpoint{0.879903in}{2.487333in}}%
\pgfpathlineto{\pgfqpoint{0.880769in}{2.493696in}}%
\pgfpathlineto{\pgfqpoint{0.881633in}{2.531694in}}%
\pgfpathlineto{\pgfqpoint{0.882498in}{2.495787in}}%
\pgfpathlineto{\pgfqpoint{0.884229in}{2.560499in}}%
\pgfpathlineto{\pgfqpoint{0.885096in}{2.504395in}}%
\pgfpathlineto{\pgfqpoint{0.886826in}{2.583126in}}%
\pgfpathlineto{\pgfqpoint{0.887691in}{2.422498in}}%
\pgfpathlineto{\pgfqpoint{0.889418in}{2.543961in}}%
\pgfpathlineto{\pgfqpoint{0.890283in}{2.565880in}}%
\pgfpathlineto{\pgfqpoint{0.891148in}{2.557395in}}%
\pgfpathlineto{\pgfqpoint{0.892011in}{2.564035in}}%
\pgfpathlineto{\pgfqpoint{0.893741in}{2.529726in}}%
\pgfpathlineto{\pgfqpoint{0.894605in}{2.551553in}}%
\pgfpathlineto{\pgfqpoint{0.896337in}{2.531725in}}%
\pgfpathlineto{\pgfqpoint{0.897202in}{2.561146in}}%
\pgfpathlineto{\pgfqpoint{0.898931in}{2.473561in}}%
\pgfpathlineto{\pgfqpoint{0.899795in}{2.509252in}}%
\pgfpathlineto{\pgfqpoint{0.900660in}{2.595085in}}%
\pgfpathlineto{\pgfqpoint{0.901525in}{2.502735in}}%
\pgfpathlineto{\pgfqpoint{0.902390in}{2.559547in}}%
\pgfpathlineto{\pgfqpoint{0.903254in}{2.503872in}}%
\pgfpathlineto{\pgfqpoint{0.904119in}{2.575288in}}%
\pgfpathlineto{\pgfqpoint{0.905849in}{2.545405in}}%
\pgfpathlineto{\pgfqpoint{0.907579in}{2.600250in}}%
\pgfpathlineto{\pgfqpoint{0.908445in}{2.501075in}}%
\pgfpathlineto{\pgfqpoint{0.910175in}{2.834938in}}%
\pgfpathlineto{\pgfqpoint{0.911039in}{2.755746in}}%
\pgfpathlineto{\pgfqpoint{0.911905in}{2.798847in}}%
\pgfpathlineto{\pgfqpoint{0.912770in}{2.696475in}}%
\pgfpathlineto{\pgfqpoint{0.913635in}{2.736931in}}%
\pgfpathlineto{\pgfqpoint{0.914501in}{2.692170in}}%
\pgfpathlineto{\pgfqpoint{0.916229in}{2.784951in}}%
\pgfpathlineto{\pgfqpoint{0.917094in}{2.781508in}}%
\pgfpathlineto{\pgfqpoint{0.917959in}{2.753809in}}%
\pgfpathlineto{\pgfqpoint{0.918824in}{2.549341in}}%
\pgfpathlineto{\pgfqpoint{0.919687in}{2.820183in}}%
\pgfpathlineto{\pgfqpoint{0.920552in}{2.780525in}}%
\pgfpathlineto{\pgfqpoint{0.921417in}{2.732813in}}%
\pgfpathlineto{\pgfqpoint{0.922280in}{2.833032in}}%
\pgfpathlineto{\pgfqpoint{0.924010in}{2.523517in}}%
\pgfpathlineto{\pgfqpoint{0.925740in}{2.485827in}}%
\pgfpathlineto{\pgfqpoint{0.927464in}{2.528467in}}%
\pgfpathlineto{\pgfqpoint{0.929193in}{2.484229in}}%
\pgfpathlineto{\pgfqpoint{0.930058in}{2.556166in}}%
\pgfpathlineto{\pgfqpoint{0.931787in}{2.469717in}}%
\pgfpathlineto{\pgfqpoint{0.932652in}{2.481645in}}%
\pgfpathlineto{\pgfqpoint{0.934381in}{2.541685in}}%
\pgfpathlineto{\pgfqpoint{0.935246in}{2.506055in}}%
\pgfpathlineto{\pgfqpoint{0.936109in}{2.512634in}}%
\pgfpathlineto{\pgfqpoint{0.936974in}{2.520934in}}%
\pgfpathlineto{\pgfqpoint{0.937840in}{2.507223in}}%
\pgfpathlineto{\pgfqpoint{0.938706in}{2.528251in}}%
\pgfpathlineto{\pgfqpoint{0.939570in}{2.454285in}}%
\pgfpathlineto{\pgfqpoint{0.941301in}{2.497938in}}%
\pgfpathlineto{\pgfqpoint{0.943032in}{2.479862in}}%
\pgfpathlineto{\pgfqpoint{0.943897in}{2.565664in}}%
\pgfpathlineto{\pgfqpoint{0.944763in}{2.555612in}}%
\pgfpathlineto{\pgfqpoint{0.945628in}{2.568953in}}%
\pgfpathlineto{\pgfqpoint{0.946490in}{2.480109in}}%
\pgfpathlineto{\pgfqpoint{0.947355in}{2.483921in}}%
\pgfpathlineto{\pgfqpoint{0.948218in}{2.525177in}}%
\pgfpathlineto{\pgfqpoint{0.951674in}{2.469410in}}%
\pgfpathlineto{\pgfqpoint{0.954269in}{2.527820in}}%
\pgfpathlineto{\pgfqpoint{0.955133in}{2.489270in}}%
\pgfpathlineto{\pgfqpoint{0.958594in}{2.554198in}}%
\pgfpathlineto{\pgfqpoint{0.959459in}{2.458344in}}%
\pgfpathlineto{\pgfqpoint{0.960325in}{2.558103in}}%
\pgfpathlineto{\pgfqpoint{0.961191in}{2.541041in}}%
\pgfpathlineto{\pgfqpoint{0.962055in}{2.594040in}}%
\pgfpathlineto{\pgfqpoint{0.963783in}{2.480847in}}%
\pgfpathlineto{\pgfqpoint{0.964647in}{2.514786in}}%
\pgfpathlineto{\pgfqpoint{0.965512in}{2.468581in}}%
\pgfpathlineto{\pgfqpoint{0.966375in}{2.564712in}}%
\pgfpathlineto{\pgfqpoint{0.967241in}{2.549279in}}%
\pgfpathlineto{\pgfqpoint{0.968105in}{2.561577in}}%
\pgfpathlineto{\pgfqpoint{0.968970in}{2.459267in}}%
\pgfpathlineto{\pgfqpoint{0.969836in}{2.547373in}}%
\pgfpathlineto{\pgfqpoint{0.970701in}{2.468858in}}%
\pgfpathlineto{\pgfqpoint{0.971567in}{2.494743in}}%
\pgfpathlineto{\pgfqpoint{0.972432in}{2.494681in}}%
\pgfpathlineto{\pgfqpoint{0.973297in}{2.508392in}}%
\pgfpathlineto{\pgfqpoint{0.974162in}{2.473899in}}%
\pgfpathlineto{\pgfqpoint{0.975027in}{2.506517in}}%
\pgfpathlineto{\pgfqpoint{0.975889in}{2.595454in}}%
\pgfpathlineto{\pgfqpoint{0.977616in}{2.480724in}}%
\pgfpathlineto{\pgfqpoint{0.978481in}{2.523517in}}%
\pgfpathlineto{\pgfqpoint{0.979346in}{2.518476in}}%
\pgfpathlineto{\pgfqpoint{0.980211in}{2.480539in}}%
\pgfpathlineto{\pgfqpoint{0.981076in}{2.562929in}}%
\pgfpathlineto{\pgfqpoint{0.981941in}{2.547404in}}%
\pgfpathlineto{\pgfqpoint{0.982806in}{2.574980in}}%
\pgfpathlineto{\pgfqpoint{0.984535in}{2.477404in}}%
\pgfpathlineto{\pgfqpoint{0.985400in}{2.521980in}}%
\pgfpathlineto{\pgfqpoint{0.986264in}{2.462156in}}%
\pgfpathlineto{\pgfqpoint{0.987127in}{2.527022in}}%
\pgfpathlineto{\pgfqpoint{0.987991in}{2.429569in}}%
\pgfpathlineto{\pgfqpoint{0.988856in}{2.514879in}}%
\pgfpathlineto{\pgfqpoint{0.989722in}{2.511158in}}%
\pgfpathlineto{\pgfqpoint{0.990585in}{2.512942in}}%
\pgfpathlineto{\pgfqpoint{0.991449in}{2.500860in}}%
\pgfpathlineto{\pgfqpoint{0.992314in}{2.526714in}}%
\pgfpathlineto{\pgfqpoint{0.993178in}{2.514602in}}%
\pgfpathlineto{\pgfqpoint{0.994043in}{2.534309in}}%
\pgfpathlineto{\pgfqpoint{0.994907in}{2.517062in}}%
\pgfpathlineto{\pgfqpoint{0.995771in}{2.468919in}}%
\pgfpathlineto{\pgfqpoint{0.997501in}{2.512880in}}%
\pgfpathlineto{\pgfqpoint{0.998367in}{2.528467in}}%
\pgfpathlineto{\pgfqpoint{1.000096in}{2.502920in}}%
\pgfpathlineto{\pgfqpoint{1.000961in}{2.560530in}}%
\pgfpathlineto{\pgfqpoint{1.001826in}{2.543684in}}%
\pgfpathlineto{\pgfqpoint{1.002691in}{2.549464in}}%
\pgfpathlineto{\pgfqpoint{1.004419in}{2.516323in}}%
\pgfpathlineto{\pgfqpoint{1.005284in}{2.516508in}}%
\pgfpathlineto{\pgfqpoint{1.006149in}{2.506548in}}%
\pgfpathlineto{\pgfqpoint{1.007014in}{2.552538in}}%
\pgfpathlineto{\pgfqpoint{1.007877in}{2.514355in}}%
\pgfpathlineto{\pgfqpoint{1.009607in}{2.556841in}}%
\pgfpathlineto{\pgfqpoint{1.012203in}{2.436700in}}%
\pgfpathlineto{\pgfqpoint{1.013933in}{2.590597in}}%
\pgfpathlineto{\pgfqpoint{1.014798in}{2.534278in}}%
\pgfpathlineto{\pgfqpoint{1.015662in}{2.474453in}}%
\pgfpathlineto{\pgfqpoint{1.017391in}{2.553277in}}%
\pgfpathlineto{\pgfqpoint{1.018255in}{2.585556in}}%
\pgfpathlineto{\pgfqpoint{1.019120in}{2.565328in}}%
\pgfpathlineto{\pgfqpoint{1.019986in}{2.489671in}}%
\pgfpathlineto{\pgfqpoint{1.020851in}{2.530711in}}%
\pgfpathlineto{\pgfqpoint{1.021716in}{2.492344in}}%
\pgfpathlineto{\pgfqpoint{1.022581in}{2.496311in}}%
\pgfpathlineto{\pgfqpoint{1.023445in}{2.487487in}}%
\pgfpathlineto{\pgfqpoint{1.025173in}{2.552353in}}%
\pgfpathlineto{\pgfqpoint{1.026903in}{2.506917in}}%
\pgfpathlineto{\pgfqpoint{1.027767in}{2.494620in}}%
\pgfpathlineto{\pgfqpoint{1.028629in}{2.617773in}}%
\pgfpathlineto{\pgfqpoint{1.030357in}{2.505072in}}%
\pgfpathlineto{\pgfqpoint{1.031221in}{2.541410in}}%
\pgfpathlineto{\pgfqpoint{1.032087in}{2.529636in}}%
\pgfpathlineto{\pgfqpoint{1.033816in}{2.555060in}}%
\pgfpathlineto{\pgfqpoint{1.035545in}{2.463631in}}%
\pgfpathlineto{\pgfqpoint{1.037275in}{2.479372in}}%
\pgfpathlineto{\pgfqpoint{1.038139in}{2.472239in}}%
\pgfpathlineto{\pgfqpoint{1.040734in}{2.553767in}}%
\pgfpathlineto{\pgfqpoint{1.041600in}{2.555858in}}%
\pgfpathlineto{\pgfqpoint{1.042466in}{2.565143in}}%
\pgfpathlineto{\pgfqpoint{1.043331in}{2.563360in}}%
\pgfpathlineto{\pgfqpoint{1.044195in}{2.552661in}}%
\pgfpathlineto{\pgfqpoint{1.045060in}{2.594777in}}%
\pgfpathlineto{\pgfqpoint{1.045927in}{2.473561in}}%
\pgfpathlineto{\pgfqpoint{1.046792in}{2.524993in}}%
\pgfpathlineto{\pgfqpoint{1.047657in}{2.520259in}}%
\pgfpathlineto{\pgfqpoint{1.048523in}{2.540395in}}%
\pgfpathlineto{\pgfqpoint{1.050256in}{2.485827in}}%
\pgfpathlineto{\pgfqpoint{1.051122in}{2.507777in}}%
\pgfpathlineto{\pgfqpoint{1.051987in}{2.445187in}}%
\pgfpathlineto{\pgfqpoint{1.054584in}{2.557025in}}%
\pgfpathlineto{\pgfqpoint{1.055449in}{2.498707in}}%
\pgfpathlineto{\pgfqpoint{1.057179in}{2.536921in}}%
\pgfpathlineto{\pgfqpoint{1.059773in}{2.497386in}}%
\pgfpathlineto{\pgfqpoint{1.060637in}{2.446323in}}%
\pgfpathlineto{\pgfqpoint{1.061503in}{2.553952in}}%
\pgfpathlineto{\pgfqpoint{1.062369in}{2.518414in}}%
\pgfpathlineto{\pgfqpoint{1.063234in}{2.519643in}}%
\pgfpathlineto{\pgfqpoint{1.064099in}{2.507900in}}%
\pgfpathlineto{\pgfqpoint{1.064962in}{2.556595in}}%
\pgfpathlineto{\pgfqpoint{1.068422in}{2.469410in}}%
\pgfpathlineto{\pgfqpoint{1.069287in}{2.471808in}}%
\pgfpathlineto{\pgfqpoint{1.071017in}{2.570183in}}%
\pgfpathlineto{\pgfqpoint{1.073608in}{2.452563in}}%
\pgfpathlineto{\pgfqpoint{1.074470in}{2.531263in}}%
\pgfpathlineto{\pgfqpoint{1.075335in}{2.509252in}}%
\pgfpathlineto{\pgfqpoint{1.076199in}{2.456499in}}%
\pgfpathlineto{\pgfqpoint{1.077064in}{2.521734in}}%
\pgfpathlineto{\pgfqpoint{1.077927in}{2.519766in}}%
\pgfpathlineto{\pgfqpoint{1.078792in}{2.559301in}}%
\pgfpathlineto{\pgfqpoint{1.080521in}{2.451149in}}%
\pgfpathlineto{\pgfqpoint{1.081384in}{2.500706in}}%
\pgfpathlineto{\pgfqpoint{1.082248in}{2.474759in}}%
\pgfpathlineto{\pgfqpoint{1.083981in}{2.563789in}}%
\pgfpathlineto{\pgfqpoint{1.084846in}{2.555612in}}%
\pgfpathlineto{\pgfqpoint{1.085711in}{2.565880in}}%
\pgfpathlineto{\pgfqpoint{1.086576in}{2.458957in}}%
\pgfpathlineto{\pgfqpoint{1.088306in}{2.596252in}}%
\pgfpathlineto{\pgfqpoint{1.090037in}{2.520626in}}%
\pgfpathlineto{\pgfqpoint{1.090903in}{2.523885in}}%
\pgfpathlineto{\pgfqpoint{1.092632in}{2.492160in}}%
\pgfpathlineto{\pgfqpoint{1.093498in}{2.452625in}}%
\pgfpathlineto{\pgfqpoint{1.094362in}{2.517121in}}%
\pgfpathlineto{\pgfqpoint{1.095225in}{2.508760in}}%
\pgfpathlineto{\pgfqpoint{1.096090in}{2.539133in}}%
\pgfpathlineto{\pgfqpoint{1.097822in}{2.469040in}}%
\pgfpathlineto{\pgfqpoint{1.099551in}{2.585125in}}%
\pgfpathlineto{\pgfqpoint{1.101281in}{2.538273in}}%
\pgfpathlineto{\pgfqpoint{1.102145in}{2.544728in}}%
\pgfpathlineto{\pgfqpoint{1.103871in}{2.487118in}}%
\pgfpathlineto{\pgfqpoint{1.104736in}{2.521119in}}%
\pgfpathlineto{\pgfqpoint{1.106466in}{2.483182in}}%
\pgfpathlineto{\pgfqpoint{1.107328in}{2.523271in}}%
\pgfpathlineto{\pgfqpoint{1.108192in}{2.516446in}}%
\pgfpathlineto{\pgfqpoint{1.109055in}{2.521611in}}%
\pgfpathlineto{\pgfqpoint{1.109918in}{2.573933in}}%
\pgfpathlineto{\pgfqpoint{1.110782in}{2.491913in}}%
\pgfpathlineto{\pgfqpoint{1.111647in}{2.517306in}}%
\pgfpathlineto{\pgfqpoint{1.113375in}{2.447583in}}%
\pgfpathlineto{\pgfqpoint{1.115967in}{2.557518in}}%
\pgfpathlineto{\pgfqpoint{1.117696in}{2.504734in}}%
\pgfpathlineto{\pgfqpoint{1.118561in}{2.575780in}}%
\pgfpathlineto{\pgfqpoint{1.120289in}{2.467321in}}%
\pgfpathlineto{\pgfqpoint{1.122016in}{2.537629in}}%
\pgfpathlineto{\pgfqpoint{1.122881in}{2.487610in}}%
\pgfpathlineto{\pgfqpoint{1.123746in}{2.492898in}}%
\pgfpathlineto{\pgfqpoint{1.124612in}{2.501229in}}%
\pgfpathlineto{\pgfqpoint{1.126342in}{2.547681in}}%
\pgfpathlineto{\pgfqpoint{1.127208in}{2.494066in}}%
\pgfpathlineto{\pgfqpoint{1.128072in}{2.512757in}}%
\pgfpathlineto{\pgfqpoint{1.128936in}{2.510912in}}%
\pgfpathlineto{\pgfqpoint{1.129800in}{2.459880in}}%
\pgfpathlineto{\pgfqpoint{1.130665in}{2.506055in}}%
\pgfpathlineto{\pgfqpoint{1.131530in}{2.488501in}}%
\pgfpathlineto{\pgfqpoint{1.133261in}{2.538396in}}%
\pgfpathlineto{\pgfqpoint{1.134126in}{2.495264in}}%
\pgfpathlineto{\pgfqpoint{1.134991in}{2.518045in}}%
\pgfpathlineto{\pgfqpoint{1.135856in}{2.459388in}}%
\pgfpathlineto{\pgfqpoint{1.136722in}{2.518014in}}%
\pgfpathlineto{\pgfqpoint{1.137586in}{2.437869in}}%
\pgfpathlineto{\pgfqpoint{1.138451in}{2.535137in}}%
\pgfpathlineto{\pgfqpoint{1.139316in}{2.480478in}}%
\pgfpathlineto{\pgfqpoint{1.140181in}{2.499415in}}%
\pgfpathlineto{\pgfqpoint{1.141044in}{2.486841in}}%
\pgfpathlineto{\pgfqpoint{1.142776in}{2.569507in}}%
\pgfpathlineto{\pgfqpoint{1.144506in}{2.463385in}}%
\pgfpathlineto{\pgfqpoint{1.145371in}{2.495295in}}%
\pgfpathlineto{\pgfqpoint{1.146235in}{2.469625in}}%
\pgfpathlineto{\pgfqpoint{1.147966in}{2.566434in}}%
\pgfpathlineto{\pgfqpoint{1.148829in}{2.487364in}}%
\pgfpathlineto{\pgfqpoint{1.149692in}{2.508085in}}%
\pgfpathlineto{\pgfqpoint{1.150558in}{2.502489in}}%
\pgfpathlineto{\pgfqpoint{1.152290in}{2.558932in}}%
\pgfpathlineto{\pgfqpoint{1.154885in}{2.468427in}}%
\pgfpathlineto{\pgfqpoint{1.155749in}{2.533508in}}%
\pgfpathlineto{\pgfqpoint{1.156614in}{2.523517in}}%
\pgfpathlineto{\pgfqpoint{1.158343in}{2.492590in}}%
\pgfpathlineto{\pgfqpoint{1.159208in}{2.547312in}}%
\pgfpathlineto{\pgfqpoint{1.160073in}{2.530065in}}%
\pgfpathlineto{\pgfqpoint{1.160937in}{2.482384in}}%
\pgfpathlineto{\pgfqpoint{1.161802in}{2.611931in}}%
\pgfpathlineto{\pgfqpoint{1.162667in}{2.610702in}}%
\pgfpathlineto{\pgfqpoint{1.164394in}{2.600065in}}%
\pgfpathlineto{\pgfqpoint{1.165259in}{2.497694in}}%
\pgfpathlineto{\pgfqpoint{1.166124in}{2.528190in}}%
\pgfpathlineto{\pgfqpoint{1.166990in}{2.603877in}}%
\pgfpathlineto{\pgfqpoint{1.167854in}{2.583219in}}%
\pgfpathlineto{\pgfqpoint{1.171315in}{2.496095in}}%
\pgfpathlineto{\pgfqpoint{1.172180in}{2.515431in}}%
\pgfpathlineto{\pgfqpoint{1.173045in}{2.574488in}}%
\pgfpathlineto{\pgfqpoint{1.173912in}{2.559178in}}%
\pgfpathlineto{\pgfqpoint{1.174778in}{2.559178in}}%
\pgfpathlineto{\pgfqpoint{1.177374in}{2.461664in}}%
\pgfpathlineto{\pgfqpoint{1.178239in}{2.509806in}}%
\pgfpathlineto{\pgfqpoint{1.179103in}{2.441097in}}%
\pgfpathlineto{\pgfqpoint{1.180834in}{2.542886in}}%
\pgfpathlineto{\pgfqpoint{1.181699in}{2.588876in}}%
\pgfpathlineto{\pgfqpoint{1.183430in}{2.523271in}}%
\pgfpathlineto{\pgfqpoint{1.184295in}{2.576702in}}%
\pgfpathlineto{\pgfqpoint{1.186026in}{2.486135in}}%
\pgfpathlineto{\pgfqpoint{1.186891in}{2.590782in}}%
\pgfpathlineto{\pgfqpoint{1.187758in}{2.588937in}}%
\pgfpathlineto{\pgfqpoint{1.189487in}{2.513496in}}%
\pgfpathlineto{\pgfqpoint{1.190352in}{2.593117in}}%
\pgfpathlineto{\pgfqpoint{1.191218in}{2.492744in}}%
\pgfpathlineto{\pgfqpoint{1.193813in}{2.581743in}}%
\pgfpathlineto{\pgfqpoint{1.194678in}{2.539442in}}%
\pgfpathlineto{\pgfqpoint{1.195541in}{2.600681in}}%
\pgfpathlineto{\pgfqpoint{1.196406in}{2.472362in}}%
\pgfpathlineto{\pgfqpoint{1.197272in}{2.600927in}}%
\pgfpathlineto{\pgfqpoint{1.199001in}{2.544299in}}%
\pgfpathlineto{\pgfqpoint{1.199865in}{2.545898in}}%
\pgfpathlineto{\pgfqpoint{1.200729in}{2.532772in}}%
\pgfpathlineto{\pgfqpoint{1.201595in}{2.496280in}}%
\pgfpathlineto{\pgfqpoint{1.203324in}{2.535661in}}%
\pgfpathlineto{\pgfqpoint{1.204189in}{2.471441in}}%
\pgfpathlineto{\pgfqpoint{1.205054in}{2.528069in}}%
\pgfpathlineto{\pgfqpoint{1.205919in}{2.463018in}}%
\pgfpathlineto{\pgfqpoint{1.206785in}{2.517370in}}%
\pgfpathlineto{\pgfqpoint{1.207651in}{2.503505in}}%
\pgfpathlineto{\pgfqpoint{1.208516in}{2.554752in}}%
\pgfpathlineto{\pgfqpoint{1.210246in}{2.513219in}}%
\pgfpathlineto{\pgfqpoint{1.211111in}{2.493206in}}%
\pgfpathlineto{\pgfqpoint{1.212842in}{2.554044in}}%
\pgfpathlineto{\pgfqpoint{1.214572in}{2.510114in}}%
\pgfpathlineto{\pgfqpoint{1.216303in}{2.592011in}}%
\pgfpathlineto{\pgfqpoint{1.217169in}{2.535445in}}%
\pgfpathlineto{\pgfqpoint{1.218035in}{2.588322in}}%
\pgfpathlineto{\pgfqpoint{1.218902in}{2.547558in}}%
\pgfpathlineto{\pgfqpoint{1.219768in}{2.608888in}}%
\pgfpathlineto{\pgfqpoint{1.220634in}{2.502674in}}%
\pgfpathlineto{\pgfqpoint{1.221500in}{2.524010in}}%
\pgfpathlineto{\pgfqpoint{1.222366in}{2.494281in}}%
\pgfpathlineto{\pgfqpoint{1.224098in}{2.561330in}}%
\pgfpathlineto{\pgfqpoint{1.224965in}{2.537413in}}%
\pgfpathlineto{\pgfqpoint{1.225831in}{2.549464in}}%
\pgfpathlineto{\pgfqpoint{1.226697in}{2.474515in}}%
\pgfpathlineto{\pgfqpoint{1.227562in}{2.481401in}}%
\pgfpathlineto{\pgfqpoint{1.228427in}{2.565205in}}%
\pgfpathlineto{\pgfqpoint{1.229292in}{2.541102in}}%
\pgfpathlineto{\pgfqpoint{1.230156in}{2.479864in}}%
\pgfpathlineto{\pgfqpoint{1.231020in}{2.523825in}}%
\pgfpathlineto{\pgfqpoint{1.231885in}{2.497817in}}%
\pgfpathlineto{\pgfqpoint{1.233617in}{2.549772in}}%
\pgfpathlineto{\pgfqpoint{1.235347in}{2.460004in}}%
\pgfpathlineto{\pgfqpoint{1.237076in}{2.572951in}}%
\pgfpathlineto{\pgfqpoint{1.237942in}{2.579098in}}%
\pgfpathlineto{\pgfqpoint{1.238807in}{2.480355in}}%
\pgfpathlineto{\pgfqpoint{1.240537in}{2.557580in}}%
\pgfpathlineto{\pgfqpoint{1.241402in}{2.549064in}}%
\pgfpathlineto{\pgfqpoint{1.243133in}{2.489086in}}%
\pgfpathlineto{\pgfqpoint{1.243999in}{2.544115in}}%
\pgfpathlineto{\pgfqpoint{1.244864in}{2.501137in}}%
\pgfpathlineto{\pgfqpoint{1.245728in}{2.508239in}}%
\pgfpathlineto{\pgfqpoint{1.246593in}{2.553092in}}%
\pgfpathlineto{\pgfqpoint{1.247458in}{2.468981in}}%
\pgfpathlineto{\pgfqpoint{1.248323in}{2.473376in}}%
\pgfpathlineto{\pgfqpoint{1.249187in}{2.476358in}}%
\pgfpathlineto{\pgfqpoint{1.250053in}{2.558686in}}%
\pgfpathlineto{\pgfqpoint{1.250916in}{2.495326in}}%
\pgfpathlineto{\pgfqpoint{1.251781in}{2.508760in}}%
\pgfpathlineto{\pgfqpoint{1.252647in}{2.540179in}}%
\pgfpathlineto{\pgfqpoint{1.253512in}{2.526776in}}%
\pgfpathlineto{\pgfqpoint{1.254377in}{2.550878in}}%
\pgfpathlineto{\pgfqpoint{1.255243in}{2.489517in}}%
\pgfpathlineto{\pgfqpoint{1.256109in}{2.575103in}}%
\pgfpathlineto{\pgfqpoint{1.257840in}{2.519797in}}%
\pgfpathlineto{\pgfqpoint{1.258706in}{2.577746in}}%
\pgfpathlineto{\pgfqpoint{1.259571in}{2.474944in}}%
\pgfpathlineto{\pgfqpoint{1.260436in}{2.496095in}}%
\pgfpathlineto{\pgfqpoint{1.262165in}{2.559178in}}%
\pgfpathlineto{\pgfqpoint{1.263030in}{2.549033in}}%
\pgfpathlineto{\pgfqpoint{1.265625in}{2.428707in}}%
\pgfpathlineto{\pgfqpoint{1.267353in}{2.603416in}}%
\pgfpathlineto{\pgfqpoint{1.268216in}{2.590351in}}%
\pgfpathlineto{\pgfqpoint{1.269081in}{2.517983in}}%
\pgfpathlineto{\pgfqpoint{1.269946in}{2.536767in}}%
\pgfpathlineto{\pgfqpoint{1.270811in}{2.558809in}}%
\pgfpathlineto{\pgfqpoint{1.274272in}{2.496280in}}%
\pgfpathlineto{\pgfqpoint{1.275137in}{2.497694in}}%
\pgfpathlineto{\pgfqpoint{1.276003in}{2.529113in}}%
\pgfpathlineto{\pgfqpoint{1.276869in}{2.496341in}}%
\pgfpathlineto{\pgfqpoint{1.278600in}{2.620108in}}%
\pgfpathlineto{\pgfqpoint{1.280331in}{2.511035in}}%
\pgfpathlineto{\pgfqpoint{1.281196in}{2.514786in}}%
\pgfpathlineto{\pgfqpoint{1.282062in}{2.492098in}}%
\pgfpathlineto{\pgfqpoint{1.283793in}{2.574734in}}%
\pgfpathlineto{\pgfqpoint{1.284658in}{2.573320in}}%
\pgfpathlineto{\pgfqpoint{1.285521in}{2.592871in}}%
\pgfpathlineto{\pgfqpoint{1.287250in}{2.515033in}}%
\pgfpathlineto{\pgfqpoint{1.288980in}{2.538611in}}%
\pgfpathlineto{\pgfqpoint{1.290711in}{2.564404in}}%
\pgfpathlineto{\pgfqpoint{1.292440in}{2.523333in}}%
\pgfpathlineto{\pgfqpoint{1.293306in}{2.579775in}}%
\pgfpathlineto{\pgfqpoint{1.295037in}{2.525362in}}%
\pgfpathlineto{\pgfqpoint{1.295904in}{2.533354in}}%
\pgfpathlineto{\pgfqpoint{1.296768in}{2.555612in}}%
\pgfpathlineto{\pgfqpoint{1.297633in}{2.509868in}}%
\pgfpathlineto{\pgfqpoint{1.298498in}{2.390250in}}%
\pgfpathlineto{\pgfqpoint{1.300229in}{2.534830in}}%
\pgfpathlineto{\pgfqpoint{1.301092in}{2.457851in}}%
\pgfpathlineto{\pgfqpoint{1.301958in}{2.461356in}}%
\pgfpathlineto{\pgfqpoint{1.302823in}{2.454562in}}%
\pgfpathlineto{\pgfqpoint{1.303687in}{2.514355in}}%
\pgfpathlineto{\pgfqpoint{1.304552in}{2.512880in}}%
\pgfpathlineto{\pgfqpoint{1.305417in}{2.529788in}}%
\pgfpathlineto{\pgfqpoint{1.306282in}{2.518168in}}%
\pgfpathlineto{\pgfqpoint{1.307147in}{2.530096in}}%
\pgfpathlineto{\pgfqpoint{1.308013in}{2.465261in}}%
\pgfpathlineto{\pgfqpoint{1.308879in}{2.566064in}}%
\pgfpathlineto{\pgfqpoint{1.309745in}{2.500583in}}%
\pgfpathlineto{\pgfqpoint{1.310610in}{2.543930in}}%
\pgfpathlineto{\pgfqpoint{1.311476in}{2.524192in}}%
\pgfpathlineto{\pgfqpoint{1.312342in}{2.533016in}}%
\pgfpathlineto{\pgfqpoint{1.313208in}{2.486872in}}%
\pgfpathlineto{\pgfqpoint{1.314074in}{2.542945in}}%
\pgfpathlineto{\pgfqpoint{1.314939in}{2.530188in}}%
\pgfpathlineto{\pgfqpoint{1.315803in}{2.527266in}}%
\pgfpathlineto{\pgfqpoint{1.316665in}{2.527943in}}%
\pgfpathlineto{\pgfqpoint{1.317529in}{2.505655in}}%
\pgfpathlineto{\pgfqpoint{1.318395in}{2.424220in}}%
\pgfpathlineto{\pgfqpoint{1.319260in}{2.523086in}}%
\pgfpathlineto{\pgfqpoint{1.320124in}{2.516015in}}%
\pgfpathlineto{\pgfqpoint{1.320989in}{2.513863in}}%
\pgfpathlineto{\pgfqpoint{1.321854in}{2.544851in}}%
\pgfpathlineto{\pgfqpoint{1.322720in}{2.466213in}}%
\pgfpathlineto{\pgfqpoint{1.323583in}{2.516139in}}%
\pgfpathlineto{\pgfqpoint{1.324448in}{2.476450in}}%
\pgfpathlineto{\pgfqpoint{1.325313in}{2.547004in}}%
\pgfpathlineto{\pgfqpoint{1.327043in}{2.459234in}}%
\pgfpathlineto{\pgfqpoint{1.329636in}{2.539563in}}%
\pgfpathlineto{\pgfqpoint{1.330500in}{2.495541in}}%
\pgfpathlineto{\pgfqpoint{1.331365in}{2.500213in}}%
\pgfpathlineto{\pgfqpoint{1.332230in}{2.566986in}}%
\pgfpathlineto{\pgfqpoint{1.333954in}{2.479862in}}%
\pgfpathlineto{\pgfqpoint{1.336550in}{2.549924in}}%
\pgfpathlineto{\pgfqpoint{1.337413in}{2.541778in}}%
\pgfpathlineto{\pgfqpoint{1.338278in}{2.584448in}}%
\pgfpathlineto{\pgfqpoint{1.339144in}{2.495541in}}%
\pgfpathlineto{\pgfqpoint{1.340007in}{2.551984in}}%
\pgfpathlineto{\pgfqpoint{1.341734in}{2.497324in}}%
\pgfpathlineto{\pgfqpoint{1.342599in}{2.526283in}}%
\pgfpathlineto{\pgfqpoint{1.343462in}{2.516385in}}%
\pgfpathlineto{\pgfqpoint{1.344329in}{2.526960in}}%
\pgfpathlineto{\pgfqpoint{1.345195in}{2.458097in}}%
\pgfpathlineto{\pgfqpoint{1.346925in}{2.544238in}}%
\pgfpathlineto{\pgfqpoint{1.347786in}{2.547927in}}%
\pgfpathlineto{\pgfqpoint{1.348652in}{2.473838in}}%
\pgfpathlineto{\pgfqpoint{1.349515in}{2.560900in}}%
\pgfpathlineto{\pgfqpoint{1.350380in}{2.489024in}}%
\pgfpathlineto{\pgfqpoint{1.351245in}{2.504395in}}%
\pgfpathlineto{\pgfqpoint{1.352107in}{2.505994in}}%
\pgfpathlineto{\pgfqpoint{1.353837in}{2.451088in}}%
\pgfpathlineto{\pgfqpoint{1.354702in}{2.470518in}}%
\pgfpathlineto{\pgfqpoint{1.355566in}{2.519643in}}%
\pgfpathlineto{\pgfqpoint{1.356431in}{2.515279in}}%
\pgfpathlineto{\pgfqpoint{1.357296in}{2.496218in}}%
\pgfpathlineto{\pgfqpoint{1.358162in}{2.511282in}}%
\pgfpathlineto{\pgfqpoint{1.359028in}{2.508331in}}%
\pgfpathlineto{\pgfqpoint{1.359893in}{2.448322in}}%
\pgfpathlineto{\pgfqpoint{1.360758in}{2.508146in}}%
\pgfpathlineto{\pgfqpoint{1.361622in}{2.501445in}}%
\pgfpathlineto{\pgfqpoint{1.362488in}{2.516169in}}%
\pgfpathlineto{\pgfqpoint{1.363353in}{2.571167in}}%
\pgfpathlineto{\pgfqpoint{1.364217in}{2.531817in}}%
\pgfpathlineto{\pgfqpoint{1.365947in}{2.565695in}}%
\pgfpathlineto{\pgfqpoint{1.367678in}{2.532800in}}%
\pgfpathlineto{\pgfqpoint{1.368543in}{2.521796in}}%
\pgfpathlineto{\pgfqpoint{1.369408in}{2.476912in}}%
\pgfpathlineto{\pgfqpoint{1.370273in}{2.503841in}}%
\pgfpathlineto{\pgfqpoint{1.371138in}{2.437562in}}%
\pgfpathlineto{\pgfqpoint{1.372866in}{2.504826in}}%
\pgfpathlineto{\pgfqpoint{1.373729in}{2.458282in}}%
\pgfpathlineto{\pgfqpoint{1.374594in}{2.502181in}}%
\pgfpathlineto{\pgfqpoint{1.375459in}{2.433564in}}%
\pgfpathlineto{\pgfqpoint{1.376322in}{2.510974in}}%
\pgfpathlineto{\pgfqpoint{1.377187in}{2.485766in}}%
\pgfpathlineto{\pgfqpoint{1.378052in}{2.514355in}}%
\pgfpathlineto{\pgfqpoint{1.379781in}{2.456684in}}%
\pgfpathlineto{\pgfqpoint{1.380646in}{2.523456in}}%
\pgfpathlineto{\pgfqpoint{1.381509in}{2.473838in}}%
\pgfpathlineto{\pgfqpoint{1.382373in}{2.523517in}}%
\pgfpathlineto{\pgfqpoint{1.383238in}{2.497263in}}%
\pgfpathlineto{\pgfqpoint{1.384104in}{2.500644in}}%
\pgfpathlineto{\pgfqpoint{1.384968in}{2.542270in}}%
\pgfpathlineto{\pgfqpoint{1.386696in}{2.466951in}}%
\pgfpathlineto{\pgfqpoint{1.387562in}{2.531941in}}%
\pgfpathlineto{\pgfqpoint{1.388427in}{2.443986in}}%
\pgfpathlineto{\pgfqpoint{1.389291in}{2.557395in}}%
\pgfpathlineto{\pgfqpoint{1.391021in}{2.413398in}}%
\pgfpathlineto{\pgfqpoint{1.391886in}{2.555365in}}%
\pgfpathlineto{\pgfqpoint{1.393613in}{2.480447in}}%
\pgfpathlineto{\pgfqpoint{1.394476in}{2.490623in}}%
\pgfpathlineto{\pgfqpoint{1.395342in}{2.549095in}}%
\pgfpathlineto{\pgfqpoint{1.396208in}{2.440820in}}%
\pgfpathlineto{\pgfqpoint{1.397938in}{2.504549in}}%
\pgfpathlineto{\pgfqpoint{1.398803in}{2.506178in}}%
\pgfpathlineto{\pgfqpoint{1.399668in}{2.575840in}}%
\pgfpathlineto{\pgfqpoint{1.400534in}{2.484996in}}%
\pgfpathlineto{\pgfqpoint{1.401400in}{2.519520in}}%
\pgfpathlineto{\pgfqpoint{1.402265in}{2.492529in}}%
\pgfpathlineto{\pgfqpoint{1.403997in}{2.552292in}}%
\pgfpathlineto{\pgfqpoint{1.404863in}{2.549524in}}%
\pgfpathlineto{\pgfqpoint{1.405729in}{2.520934in}}%
\pgfpathlineto{\pgfqpoint{1.406594in}{2.584386in}}%
\pgfpathlineto{\pgfqpoint{1.408324in}{2.495911in}}%
\pgfpathlineto{\pgfqpoint{1.409190in}{2.571229in}}%
\pgfpathlineto{\pgfqpoint{1.410055in}{2.483798in}}%
\pgfpathlineto{\pgfqpoint{1.410921in}{2.584755in}}%
\pgfpathlineto{\pgfqpoint{1.411785in}{2.509437in}}%
\pgfpathlineto{\pgfqpoint{1.412651in}{2.555673in}}%
\pgfpathlineto{\pgfqpoint{1.413515in}{2.524808in}}%
\pgfpathlineto{\pgfqpoint{1.414379in}{2.569569in}}%
\pgfpathlineto{\pgfqpoint{1.416108in}{2.494127in}}%
\pgfpathlineto{\pgfqpoint{1.417838in}{2.592565in}}%
\pgfpathlineto{\pgfqpoint{1.420430in}{2.485519in}}%
\pgfpathlineto{\pgfqpoint{1.422159in}{2.513894in}}%
\pgfpathlineto{\pgfqpoint{1.423024in}{2.522840in}}%
\pgfpathlineto{\pgfqpoint{1.423884in}{2.509622in}}%
\pgfpathlineto{\pgfqpoint{1.424749in}{2.541408in}}%
\pgfpathlineto{\pgfqpoint{1.425614in}{2.494250in}}%
\pgfpathlineto{\pgfqpoint{1.428206in}{2.572828in}}%
\pgfpathlineto{\pgfqpoint{1.429937in}{2.484290in}}%
\pgfpathlineto{\pgfqpoint{1.430801in}{2.498984in}}%
\pgfpathlineto{\pgfqpoint{1.431667in}{2.473068in}}%
\pgfpathlineto{\pgfqpoint{1.433398in}{2.559363in}}%
\pgfpathlineto{\pgfqpoint{1.435130in}{2.445738in}}%
\pgfpathlineto{\pgfqpoint{1.436861in}{2.532923in}}%
\pgfpathlineto{\pgfqpoint{1.437726in}{2.485827in}}%
\pgfpathlineto{\pgfqpoint{1.438592in}{2.524531in}}%
\pgfpathlineto{\pgfqpoint{1.439455in}{2.508944in}}%
\pgfpathlineto{\pgfqpoint{1.441185in}{2.562190in}}%
\pgfpathlineto{\pgfqpoint{1.442915in}{2.521919in}}%
\pgfpathlineto{\pgfqpoint{1.443778in}{2.534031in}}%
\pgfpathlineto{\pgfqpoint{1.444643in}{2.471993in}}%
\pgfpathlineto{\pgfqpoint{1.445508in}{2.539258in}}%
\pgfpathlineto{\pgfqpoint{1.446371in}{2.518291in}}%
\pgfpathlineto{\pgfqpoint{1.447237in}{2.535815in}}%
\pgfpathlineto{\pgfqpoint{1.448967in}{2.472362in}}%
\pgfpathlineto{\pgfqpoint{1.449831in}{2.456991in}}%
\pgfpathlineto{\pgfqpoint{1.452425in}{2.482754in}}%
\pgfpathlineto{\pgfqpoint{1.454154in}{2.525054in}}%
\pgfpathlineto{\pgfqpoint{1.455884in}{2.479126in}}%
\pgfpathlineto{\pgfqpoint{1.457614in}{2.545898in}}%
\pgfpathlineto{\pgfqpoint{1.460209in}{2.482938in}}%
\pgfpathlineto{\pgfqpoint{1.461074in}{2.455208in}}%
\pgfpathlineto{\pgfqpoint{1.461941in}{2.465599in}}%
\pgfpathlineto{\pgfqpoint{1.462806in}{2.448076in}}%
\pgfpathlineto{\pgfqpoint{1.463672in}{2.556658in}}%
\pgfpathlineto{\pgfqpoint{1.465402in}{2.512205in}}%
\pgfpathlineto{\pgfqpoint{1.466268in}{2.491300in}}%
\pgfpathlineto{\pgfqpoint{1.467134in}{2.506517in}}%
\pgfpathlineto{\pgfqpoint{1.467999in}{2.468796in}}%
\pgfpathlineto{\pgfqpoint{1.468864in}{2.540548in}}%
\pgfpathlineto{\pgfqpoint{1.469729in}{2.523671in}}%
\pgfpathlineto{\pgfqpoint{1.471459in}{2.480049in}}%
\pgfpathlineto{\pgfqpoint{1.472326in}{2.510853in}}%
\pgfpathlineto{\pgfqpoint{1.473192in}{2.456193in}}%
\pgfpathlineto{\pgfqpoint{1.474056in}{2.485889in}}%
\pgfpathlineto{\pgfqpoint{1.474922in}{2.474269in}}%
\pgfpathlineto{\pgfqpoint{1.475788in}{2.514111in}}%
\pgfpathlineto{\pgfqpoint{1.476654in}{2.474084in}}%
\pgfpathlineto{\pgfqpoint{1.479253in}{2.593948in}}%
\pgfpathlineto{\pgfqpoint{1.480117in}{2.539442in}}%
\pgfpathlineto{\pgfqpoint{1.480982in}{2.580453in}}%
\pgfpathlineto{\pgfqpoint{1.482711in}{2.496465in}}%
\pgfpathlineto{\pgfqpoint{1.483575in}{2.508177in}}%
\pgfpathlineto{\pgfqpoint{1.484441in}{2.472362in}}%
\pgfpathlineto{\pgfqpoint{1.485307in}{2.481217in}}%
\pgfpathlineto{\pgfqpoint{1.486174in}{2.477835in}}%
\pgfpathlineto{\pgfqpoint{1.487040in}{2.458898in}}%
\pgfpathlineto{\pgfqpoint{1.487905in}{2.484414in}}%
\pgfpathlineto{\pgfqpoint{1.488770in}{2.463539in}}%
\pgfpathlineto{\pgfqpoint{1.490500in}{2.534953in}}%
\pgfpathlineto{\pgfqpoint{1.493097in}{2.472239in}}%
\pgfpathlineto{\pgfqpoint{1.494827in}{2.556474in}}%
\pgfpathlineto{\pgfqpoint{1.495693in}{2.460650in}}%
\pgfpathlineto{\pgfqpoint{1.496558in}{2.578916in}}%
\pgfpathlineto{\pgfqpoint{1.497424in}{2.561392in}}%
\pgfpathlineto{\pgfqpoint{1.498288in}{2.559486in}}%
\pgfpathlineto{\pgfqpoint{1.499153in}{2.591213in}}%
\pgfpathlineto{\pgfqpoint{1.500017in}{2.470764in}}%
\pgfpathlineto{\pgfqpoint{1.500882in}{2.548604in}}%
\pgfpathlineto{\pgfqpoint{1.501745in}{2.520751in}}%
\pgfpathlineto{\pgfqpoint{1.502609in}{2.524410in}}%
\pgfpathlineto{\pgfqpoint{1.503473in}{2.562991in}}%
\pgfpathlineto{\pgfqpoint{1.504339in}{2.553092in}}%
\pgfpathlineto{\pgfqpoint{1.505204in}{2.498309in}}%
\pgfpathlineto{\pgfqpoint{1.506068in}{2.556843in}}%
\pgfpathlineto{\pgfqpoint{1.506933in}{2.487980in}}%
\pgfpathlineto{\pgfqpoint{1.507798in}{2.543070in}}%
\pgfpathlineto{\pgfqpoint{1.508662in}{2.502368in}}%
\pgfpathlineto{\pgfqpoint{1.509524in}{2.596316in}}%
\pgfpathlineto{\pgfqpoint{1.510389in}{2.557274in}}%
\pgfpathlineto{\pgfqpoint{1.511254in}{2.595149in}}%
\pgfpathlineto{\pgfqpoint{1.512984in}{2.526470in}}%
\pgfpathlineto{\pgfqpoint{1.513848in}{2.562685in}}%
\pgfpathlineto{\pgfqpoint{1.515579in}{2.478758in}}%
\pgfpathlineto{\pgfqpoint{1.516443in}{2.510730in}}%
\pgfpathlineto{\pgfqpoint{1.517308in}{2.489978in}}%
\pgfpathlineto{\pgfqpoint{1.519038in}{2.544484in}}%
\pgfpathlineto{\pgfqpoint{1.520767in}{2.504888in}}%
\pgfpathlineto{\pgfqpoint{1.521632in}{2.560684in}}%
\pgfpathlineto{\pgfqpoint{1.523360in}{2.465415in}}%
\pgfpathlineto{\pgfqpoint{1.525091in}{2.574857in}}%
\pgfpathlineto{\pgfqpoint{1.525956in}{2.500277in}}%
\pgfpathlineto{\pgfqpoint{1.526821in}{2.501506in}}%
\pgfpathlineto{\pgfqpoint{1.527685in}{2.506794in}}%
\pgfpathlineto{\pgfqpoint{1.528549in}{2.544546in}}%
\pgfpathlineto{\pgfqpoint{1.529411in}{2.449613in}}%
\pgfpathlineto{\pgfqpoint{1.530275in}{2.581435in}}%
\pgfpathlineto{\pgfqpoint{1.531141in}{2.550909in}}%
\pgfpathlineto{\pgfqpoint{1.532005in}{2.530157in}}%
\pgfpathlineto{\pgfqpoint{1.533734in}{2.571044in}}%
\pgfpathlineto{\pgfqpoint{1.535463in}{2.535630in}}%
\pgfpathlineto{\pgfqpoint{1.536326in}{2.576148in}}%
\pgfpathlineto{\pgfqpoint{1.538921in}{2.482446in}}%
\pgfpathlineto{\pgfqpoint{1.539786in}{2.525547in}}%
\pgfpathlineto{\pgfqpoint{1.541516in}{2.492221in}}%
\pgfpathlineto{\pgfqpoint{1.542382in}{2.558993in}}%
\pgfpathlineto{\pgfqpoint{1.544111in}{2.482138in}}%
\pgfpathlineto{\pgfqpoint{1.544975in}{2.546634in}}%
\pgfpathlineto{\pgfqpoint{1.545841in}{2.526376in}}%
\pgfpathlineto{\pgfqpoint{1.546705in}{2.464737in}}%
\pgfpathlineto{\pgfqpoint{1.548435in}{2.522963in}}%
\pgfpathlineto{\pgfqpoint{1.549300in}{2.480293in}}%
\pgfpathlineto{\pgfqpoint{1.550163in}{2.524993in}}%
\pgfpathlineto{\pgfqpoint{1.551028in}{2.502612in}}%
\pgfpathlineto{\pgfqpoint{1.551893in}{2.580022in}}%
\pgfpathlineto{\pgfqpoint{1.552758in}{2.480940in}}%
\pgfpathlineto{\pgfqpoint{1.553624in}{2.537536in}}%
\pgfpathlineto{\pgfqpoint{1.554488in}{2.523948in}}%
\pgfpathlineto{\pgfqpoint{1.555353in}{2.496311in}}%
\pgfpathlineto{\pgfqpoint{1.556218in}{2.564651in}}%
\pgfpathlineto{\pgfqpoint{1.557945in}{2.447768in}}%
\pgfpathlineto{\pgfqpoint{1.558810in}{2.471808in}}%
\pgfpathlineto{\pgfqpoint{1.559676in}{2.526899in}}%
\pgfpathlineto{\pgfqpoint{1.560542in}{2.510912in}}%
\pgfpathlineto{\pgfqpoint{1.562273in}{2.451488in}}%
\pgfpathlineto{\pgfqpoint{1.564872in}{2.531325in}}%
\pgfpathlineto{\pgfqpoint{1.565738in}{2.475313in}}%
\pgfpathlineto{\pgfqpoint{1.566603in}{2.499169in}}%
\pgfpathlineto{\pgfqpoint{1.567467in}{2.453056in}}%
\pgfpathlineto{\pgfqpoint{1.568332in}{2.461541in}}%
\pgfpathlineto{\pgfqpoint{1.570061in}{2.556595in}}%
\pgfpathlineto{\pgfqpoint{1.572656in}{2.504211in}}%
\pgfpathlineto{\pgfqpoint{1.574383in}{2.516139in}}%
\pgfpathlineto{\pgfqpoint{1.575249in}{2.557826in}}%
\pgfpathlineto{\pgfqpoint{1.576978in}{2.518845in}}%
\pgfpathlineto{\pgfqpoint{1.577842in}{2.571291in}}%
\pgfpathlineto{\pgfqpoint{1.578706in}{2.508023in}}%
\pgfpathlineto{\pgfqpoint{1.579570in}{2.521919in}}%
\pgfpathlineto{\pgfqpoint{1.580437in}{2.571167in}}%
\pgfpathlineto{\pgfqpoint{1.581303in}{2.518014in}}%
\pgfpathlineto{\pgfqpoint{1.583036in}{2.545713in}}%
\pgfpathlineto{\pgfqpoint{1.583903in}{2.472116in}}%
\pgfpathlineto{\pgfqpoint{1.585634in}{2.504703in}}%
\pgfpathlineto{\pgfqpoint{1.586500in}{2.456745in}}%
\pgfpathlineto{\pgfqpoint{1.589098in}{2.578115in}}%
\pgfpathlineto{\pgfqpoint{1.592558in}{2.513373in}}%
\pgfpathlineto{\pgfqpoint{1.593423in}{2.437746in}}%
\pgfpathlineto{\pgfqpoint{1.595155in}{2.552476in}}%
\pgfpathlineto{\pgfqpoint{1.596021in}{2.514540in}}%
\pgfpathlineto{\pgfqpoint{1.596886in}{2.545959in}}%
\pgfpathlineto{\pgfqpoint{1.597751in}{2.462033in}}%
\pgfpathlineto{\pgfqpoint{1.598615in}{2.550201in}}%
\pgfpathlineto{\pgfqpoint{1.599480in}{2.533477in}}%
\pgfpathlineto{\pgfqpoint{1.600345in}{2.569015in}}%
\pgfpathlineto{\pgfqpoint{1.601207in}{2.548849in}}%
\pgfpathlineto{\pgfqpoint{1.602937in}{2.490407in}}%
\pgfpathlineto{\pgfqpoint{1.603803in}{2.548541in}}%
\pgfpathlineto{\pgfqpoint{1.604669in}{2.500275in}}%
\pgfpathlineto{\pgfqpoint{1.606401in}{2.543684in}}%
\pgfpathlineto{\pgfqpoint{1.607265in}{2.521242in}}%
\pgfpathlineto{\pgfqpoint{1.608995in}{2.567909in}}%
\pgfpathlineto{\pgfqpoint{1.610725in}{2.447583in}}%
\pgfpathlineto{\pgfqpoint{1.611590in}{2.499107in}}%
\pgfpathlineto{\pgfqpoint{1.612456in}{2.452779in}}%
\pgfpathlineto{\pgfqpoint{1.615051in}{2.592625in}}%
\pgfpathlineto{\pgfqpoint{1.615917in}{2.516015in}}%
\pgfpathlineto{\pgfqpoint{1.618510in}{2.560407in}}%
\pgfpathlineto{\pgfqpoint{1.619375in}{2.481368in}}%
\pgfpathlineto{\pgfqpoint{1.621104in}{2.557210in}}%
\pgfpathlineto{\pgfqpoint{1.621969in}{2.490161in}}%
\pgfpathlineto{\pgfqpoint{1.622835in}{2.536490in}}%
\pgfpathlineto{\pgfqpoint{1.623700in}{2.536305in}}%
\pgfpathlineto{\pgfqpoint{1.624564in}{2.527759in}}%
\pgfpathlineto{\pgfqpoint{1.625428in}{2.554136in}}%
\pgfpathlineto{\pgfqpoint{1.626293in}{2.545375in}}%
\pgfpathlineto{\pgfqpoint{1.627159in}{2.509868in}}%
\pgfpathlineto{\pgfqpoint{1.628022in}{2.532002in}}%
\pgfpathlineto{\pgfqpoint{1.628886in}{2.490376in}}%
\pgfpathlineto{\pgfqpoint{1.630617in}{2.552661in}}%
\pgfpathlineto{\pgfqpoint{1.631481in}{2.547342in}}%
\pgfpathlineto{\pgfqpoint{1.632344in}{2.523210in}}%
\pgfpathlineto{\pgfqpoint{1.633210in}{2.562991in}}%
\pgfpathlineto{\pgfqpoint{1.634073in}{2.518168in}}%
\pgfpathlineto{\pgfqpoint{1.634936in}{2.527022in}}%
\pgfpathlineto{\pgfqpoint{1.635803in}{2.512880in}}%
\pgfpathlineto{\pgfqpoint{1.637533in}{2.601235in}}%
\pgfpathlineto{\pgfqpoint{1.638397in}{2.523733in}}%
\pgfpathlineto{\pgfqpoint{1.639263in}{2.545528in}}%
\pgfpathlineto{\pgfqpoint{1.640129in}{2.581620in}}%
\pgfpathlineto{\pgfqpoint{1.643590in}{2.477466in}}%
\pgfpathlineto{\pgfqpoint{1.644456in}{2.554567in}}%
\pgfpathlineto{\pgfqpoint{1.645321in}{2.502612in}}%
\pgfpathlineto{\pgfqpoint{1.646187in}{2.505994in}}%
\pgfpathlineto{\pgfqpoint{1.647053in}{2.579591in}}%
\pgfpathlineto{\pgfqpoint{1.647918in}{2.570860in}}%
\pgfpathlineto{\pgfqpoint{1.648784in}{2.467382in}}%
\pgfpathlineto{\pgfqpoint{1.649645in}{2.526345in}}%
\pgfpathlineto{\pgfqpoint{1.650509in}{2.522411in}}%
\pgfpathlineto{\pgfqpoint{1.651372in}{2.532125in}}%
\pgfpathlineto{\pgfqpoint{1.652236in}{2.555920in}}%
\pgfpathlineto{\pgfqpoint{1.653101in}{2.516508in}}%
\pgfpathlineto{\pgfqpoint{1.653965in}{2.589612in}}%
\pgfpathlineto{\pgfqpoint{1.654830in}{2.512757in}}%
\pgfpathlineto{\pgfqpoint{1.655694in}{2.524931in}}%
\pgfpathlineto{\pgfqpoint{1.656558in}{2.545159in}}%
\pgfpathlineto{\pgfqpoint{1.657422in}{2.538858in}}%
\pgfpathlineto{\pgfqpoint{1.658286in}{2.523456in}}%
\pgfpathlineto{\pgfqpoint{1.659151in}{2.566372in}}%
\pgfpathlineto{\pgfqpoint{1.660879in}{2.469902in}}%
\pgfpathlineto{\pgfqpoint{1.661745in}{2.543253in}}%
\pgfpathlineto{\pgfqpoint{1.662610in}{2.497571in}}%
\pgfpathlineto{\pgfqpoint{1.664342in}{2.546881in}}%
\pgfpathlineto{\pgfqpoint{1.665207in}{2.473530in}}%
\pgfpathlineto{\pgfqpoint{1.666935in}{2.508791in}}%
\pgfpathlineto{\pgfqpoint{1.667800in}{2.519397in}}%
\pgfpathlineto{\pgfqpoint{1.668666in}{2.543745in}}%
\pgfpathlineto{\pgfqpoint{1.669531in}{2.467873in}}%
\pgfpathlineto{\pgfqpoint{1.670393in}{2.500891in}}%
\pgfpathlineto{\pgfqpoint{1.671259in}{2.443157in}}%
\pgfpathlineto{\pgfqpoint{1.673852in}{2.542270in}}%
\pgfpathlineto{\pgfqpoint{1.674716in}{2.449551in}}%
\pgfpathlineto{\pgfqpoint{1.676445in}{2.591703in}}%
\pgfpathlineto{\pgfqpoint{1.678175in}{2.498677in}}%
\pgfpathlineto{\pgfqpoint{1.679039in}{2.475067in}}%
\pgfpathlineto{\pgfqpoint{1.679905in}{2.483859in}}%
\pgfpathlineto{\pgfqpoint{1.680770in}{2.524931in}}%
\pgfpathlineto{\pgfqpoint{1.681635in}{2.524685in}}%
\pgfpathlineto{\pgfqpoint{1.683364in}{2.482046in}}%
\pgfpathlineto{\pgfqpoint{1.684229in}{2.514909in}}%
\pgfpathlineto{\pgfqpoint{1.685094in}{2.607382in}}%
\pgfpathlineto{\pgfqpoint{1.686825in}{2.507715in}}%
\pgfpathlineto{\pgfqpoint{1.687689in}{2.520013in}}%
\pgfpathlineto{\pgfqpoint{1.688554in}{2.416441in}}%
\pgfpathlineto{\pgfqpoint{1.689420in}{2.540056in}}%
\pgfpathlineto{\pgfqpoint{1.690285in}{2.524562in}}%
\pgfpathlineto{\pgfqpoint{1.692015in}{2.456376in}}%
\pgfpathlineto{\pgfqpoint{1.692880in}{2.473653in}}%
\pgfpathlineto{\pgfqpoint{1.694607in}{2.549893in}}%
\pgfpathlineto{\pgfqpoint{1.695472in}{2.466490in}}%
\pgfpathlineto{\pgfqpoint{1.697201in}{2.527820in}}%
\pgfpathlineto{\pgfqpoint{1.698064in}{2.509406in}}%
\pgfpathlineto{\pgfqpoint{1.698929in}{2.563850in}}%
\pgfpathlineto{\pgfqpoint{1.700660in}{2.499354in}}%
\pgfpathlineto{\pgfqpoint{1.702389in}{2.446139in}}%
\pgfpathlineto{\pgfqpoint{1.703252in}{2.463631in}}%
\pgfpathlineto{\pgfqpoint{1.704117in}{2.459573in}}%
\pgfpathlineto{\pgfqpoint{1.704981in}{2.473530in}}%
\pgfpathlineto{\pgfqpoint{1.705846in}{2.449182in}}%
\pgfpathlineto{\pgfqpoint{1.708439in}{2.545036in}}%
\pgfpathlineto{\pgfqpoint{1.710169in}{2.497324in}}%
\pgfpathlineto{\pgfqpoint{1.711035in}{2.526283in}}%
\pgfpathlineto{\pgfqpoint{1.711900in}{2.520965in}}%
\pgfpathlineto{\pgfqpoint{1.712763in}{2.464553in}}%
\pgfpathlineto{\pgfqpoint{1.713627in}{2.533293in}}%
\pgfpathlineto{\pgfqpoint{1.714491in}{2.483983in}}%
\pgfpathlineto{\pgfqpoint{1.715355in}{2.519766in}}%
\pgfpathlineto{\pgfqpoint{1.716220in}{2.509622in}}%
\pgfpathlineto{\pgfqpoint{1.717084in}{2.447614in}}%
\pgfpathlineto{\pgfqpoint{1.717949in}{2.518106in}}%
\pgfpathlineto{\pgfqpoint{1.718815in}{2.506363in}}%
\pgfpathlineto{\pgfqpoint{1.721409in}{2.477158in}}%
\pgfpathlineto{\pgfqpoint{1.722274in}{2.529175in}}%
\pgfpathlineto{\pgfqpoint{1.723138in}{2.509806in}}%
\pgfpathlineto{\pgfqpoint{1.725732in}{2.559178in}}%
\pgfpathlineto{\pgfqpoint{1.727461in}{2.496649in}}%
\pgfpathlineto{\pgfqpoint{1.728326in}{2.579775in}}%
\pgfpathlineto{\pgfqpoint{1.729191in}{2.561084in}}%
\pgfpathlineto{\pgfqpoint{1.730056in}{2.524685in}}%
\pgfpathlineto{\pgfqpoint{1.730922in}{2.548572in}}%
\pgfpathlineto{\pgfqpoint{1.731786in}{2.529788in}}%
\pgfpathlineto{\pgfqpoint{1.732650in}{2.572704in}}%
\pgfpathlineto{\pgfqpoint{1.733516in}{2.560653in}}%
\pgfpathlineto{\pgfqpoint{1.736112in}{2.616390in}}%
\pgfpathlineto{\pgfqpoint{1.738706in}{2.513865in}}%
\pgfpathlineto{\pgfqpoint{1.739570in}{2.553154in}}%
\pgfpathlineto{\pgfqpoint{1.742164in}{2.498617in}}%
\pgfpathlineto{\pgfqpoint{1.743026in}{2.518229in}}%
\pgfpathlineto{\pgfqpoint{1.743891in}{2.454716in}}%
\pgfpathlineto{\pgfqpoint{1.744756in}{2.525793in}}%
\pgfpathlineto{\pgfqpoint{1.745620in}{2.463385in}}%
\pgfpathlineto{\pgfqpoint{1.746486in}{2.465599in}}%
\pgfpathlineto{\pgfqpoint{1.747352in}{2.541901in}}%
\pgfpathlineto{\pgfqpoint{1.748217in}{2.493021in}}%
\pgfpathlineto{\pgfqpoint{1.750814in}{2.554013in}}%
\pgfpathlineto{\pgfqpoint{1.751679in}{2.497878in}}%
\pgfpathlineto{\pgfqpoint{1.752545in}{2.511682in}}%
\pgfpathlineto{\pgfqpoint{1.753409in}{2.501137in}}%
\pgfpathlineto{\pgfqpoint{1.755138in}{2.568248in}}%
\pgfpathlineto{\pgfqpoint{1.756004in}{2.524870in}}%
\pgfpathlineto{\pgfqpoint{1.756870in}{2.595331in}}%
\pgfpathlineto{\pgfqpoint{1.757735in}{2.543376in}}%
\pgfpathlineto{\pgfqpoint{1.758602in}{2.561146in}}%
\pgfpathlineto{\pgfqpoint{1.759468in}{2.461263in}}%
\pgfpathlineto{\pgfqpoint{1.760334in}{2.542270in}}%
\pgfpathlineto{\pgfqpoint{1.761200in}{2.535014in}}%
\pgfpathlineto{\pgfqpoint{1.762931in}{2.571537in}}%
\pgfpathlineto{\pgfqpoint{1.763797in}{2.567971in}}%
\pgfpathlineto{\pgfqpoint{1.764663in}{2.472209in}}%
\pgfpathlineto{\pgfqpoint{1.765529in}{2.490500in}}%
\pgfpathlineto{\pgfqpoint{1.766394in}{2.449305in}}%
\pgfpathlineto{\pgfqpoint{1.768123in}{2.529665in}}%
\pgfpathlineto{\pgfqpoint{1.768988in}{2.514817in}}%
\pgfpathlineto{\pgfqpoint{1.771582in}{2.563483in}}%
\pgfpathlineto{\pgfqpoint{1.772446in}{2.537782in}}%
\pgfpathlineto{\pgfqpoint{1.773309in}{2.573751in}}%
\pgfpathlineto{\pgfqpoint{1.775039in}{2.521488in}}%
\pgfpathlineto{\pgfqpoint{1.775903in}{2.547127in}}%
\pgfpathlineto{\pgfqpoint{1.776768in}{2.535630in}}%
\pgfpathlineto{\pgfqpoint{1.777633in}{2.552292in}}%
\pgfpathlineto{\pgfqpoint{1.778496in}{2.549187in}}%
\pgfpathlineto{\pgfqpoint{1.781086in}{2.499631in}}%
\pgfpathlineto{\pgfqpoint{1.781951in}{2.544053in}}%
\pgfpathlineto{\pgfqpoint{1.782816in}{2.533231in}}%
\pgfpathlineto{\pgfqpoint{1.783681in}{2.476450in}}%
\pgfpathlineto{\pgfqpoint{1.784545in}{2.485827in}}%
\pgfpathlineto{\pgfqpoint{1.785410in}{2.498984in}}%
\pgfpathlineto{\pgfqpoint{1.786275in}{2.448137in}}%
\pgfpathlineto{\pgfqpoint{1.787138in}{2.509437in}}%
\pgfpathlineto{\pgfqpoint{1.788003in}{2.475313in}}%
\pgfpathlineto{\pgfqpoint{1.788868in}{2.550509in}}%
\pgfpathlineto{\pgfqpoint{1.789732in}{2.486443in}}%
\pgfpathlineto{\pgfqpoint{1.792326in}{2.530281in}}%
\pgfpathlineto{\pgfqpoint{1.793189in}{2.525054in}}%
\pgfpathlineto{\pgfqpoint{1.794054in}{2.558010in}}%
\pgfpathlineto{\pgfqpoint{1.794918in}{2.438116in}}%
\pgfpathlineto{\pgfqpoint{1.798377in}{2.578423in}}%
\pgfpathlineto{\pgfqpoint{1.800107in}{2.517922in}}%
\pgfpathlineto{\pgfqpoint{1.800973in}{2.558932in}}%
\pgfpathlineto{\pgfqpoint{1.802704in}{2.529726in}}%
\pgfpathlineto{\pgfqpoint{1.803570in}{2.550201in}}%
\pgfpathlineto{\pgfqpoint{1.804436in}{2.509314in}}%
\pgfpathlineto{\pgfqpoint{1.805299in}{2.516446in}}%
\pgfpathlineto{\pgfqpoint{1.807031in}{2.538765in}}%
\pgfpathlineto{\pgfqpoint{1.807897in}{2.593486in}}%
\pgfpathlineto{\pgfqpoint{1.809628in}{2.546881in}}%
\pgfpathlineto{\pgfqpoint{1.810493in}{2.561269in}}%
\pgfpathlineto{\pgfqpoint{1.811358in}{2.497447in}}%
\pgfpathlineto{\pgfqpoint{1.813089in}{2.555981in}}%
\pgfpathlineto{\pgfqpoint{1.813954in}{2.530465in}}%
\pgfpathlineto{\pgfqpoint{1.814820in}{2.539073in}}%
\pgfpathlineto{\pgfqpoint{1.815683in}{2.481093in}}%
\pgfpathlineto{\pgfqpoint{1.816549in}{2.505196in}}%
\pgfpathlineto{\pgfqpoint{1.817413in}{2.448630in}}%
\pgfpathlineto{\pgfqpoint{1.818276in}{2.530835in}}%
\pgfpathlineto{\pgfqpoint{1.819140in}{2.515433in}}%
\pgfpathlineto{\pgfqpoint{1.820870in}{2.492221in}}%
\pgfpathlineto{\pgfqpoint{1.821736in}{2.506209in}}%
\pgfpathlineto{\pgfqpoint{1.822603in}{2.570000in}}%
\pgfpathlineto{\pgfqpoint{1.824334in}{2.476727in}}%
\pgfpathlineto{\pgfqpoint{1.825198in}{2.502735in}}%
\pgfpathlineto{\pgfqpoint{1.827794in}{2.567786in}}%
\pgfpathlineto{\pgfqpoint{1.829522in}{2.543622in}}%
\pgfpathlineto{\pgfqpoint{1.830388in}{2.547250in}}%
\pgfpathlineto{\pgfqpoint{1.831253in}{2.598774in}}%
\pgfpathlineto{\pgfqpoint{1.832979in}{2.550632in}}%
\pgfpathlineto{\pgfqpoint{1.833845in}{2.550878in}}%
\pgfpathlineto{\pgfqpoint{1.835577in}{2.484721in}}%
\pgfpathlineto{\pgfqpoint{1.836442in}{2.543193in}}%
\pgfpathlineto{\pgfqpoint{1.837307in}{2.486012in}}%
\pgfpathlineto{\pgfqpoint{1.838171in}{2.510514in}}%
\pgfpathlineto{\pgfqpoint{1.839035in}{2.570800in}}%
\pgfpathlineto{\pgfqpoint{1.840763in}{2.530681in}}%
\pgfpathlineto{\pgfqpoint{1.841628in}{2.560101in}}%
\pgfpathlineto{\pgfqpoint{1.843359in}{2.469289in}}%
\pgfpathlineto{\pgfqpoint{1.844224in}{2.589060in}}%
\pgfpathlineto{\pgfqpoint{1.845955in}{2.507777in}}%
\pgfpathlineto{\pgfqpoint{1.847684in}{2.547866in}}%
\pgfpathlineto{\pgfqpoint{1.848546in}{2.508454in}}%
\pgfpathlineto{\pgfqpoint{1.849412in}{2.554814in}}%
\pgfpathlineto{\pgfqpoint{1.851141in}{2.485889in}}%
\pgfpathlineto{\pgfqpoint{1.852007in}{2.499354in}}%
\pgfpathlineto{\pgfqpoint{1.852870in}{2.477343in}}%
\pgfpathlineto{\pgfqpoint{1.853735in}{2.533785in}}%
\pgfpathlineto{\pgfqpoint{1.854601in}{2.483675in}}%
\pgfpathlineto{\pgfqpoint{1.856331in}{2.568709in}}%
\pgfpathlineto{\pgfqpoint{1.858062in}{2.471870in}}%
\pgfpathlineto{\pgfqpoint{1.858925in}{2.485581in}}%
\pgfpathlineto{\pgfqpoint{1.859790in}{2.547250in}}%
\pgfpathlineto{\pgfqpoint{1.861521in}{2.488901in}}%
\pgfpathlineto{\pgfqpoint{1.864120in}{2.573074in}}%
\pgfpathlineto{\pgfqpoint{1.865845in}{2.553952in}}%
\pgfpathlineto{\pgfqpoint{1.866708in}{2.563789in}}%
\pgfpathlineto{\pgfqpoint{1.867573in}{2.524746in}}%
\pgfpathlineto{\pgfqpoint{1.868440in}{2.526222in}}%
\pgfpathlineto{\pgfqpoint{1.869304in}{2.567201in}}%
\pgfpathlineto{\pgfqpoint{1.870168in}{2.533108in}}%
\pgfpathlineto{\pgfqpoint{1.871032in}{2.546942in}}%
\pgfpathlineto{\pgfqpoint{1.871897in}{2.469964in}}%
\pgfpathlineto{\pgfqpoint{1.872762in}{2.577500in}}%
\pgfpathlineto{\pgfqpoint{1.873626in}{2.457851in}}%
\pgfpathlineto{\pgfqpoint{1.874493in}{2.517552in}}%
\pgfpathlineto{\pgfqpoint{1.876224in}{2.411522in}}%
\pgfpathlineto{\pgfqpoint{1.877953in}{2.467873in}}%
\pgfpathlineto{\pgfqpoint{1.878816in}{2.454839in}}%
\pgfpathlineto{\pgfqpoint{1.879681in}{2.437500in}}%
\pgfpathlineto{\pgfqpoint{1.880547in}{2.555612in}}%
\pgfpathlineto{\pgfqpoint{1.881412in}{2.481892in}}%
\pgfpathlineto{\pgfqpoint{1.882278in}{2.492283in}}%
\pgfpathlineto{\pgfqpoint{1.883143in}{2.490469in}}%
\pgfpathlineto{\pgfqpoint{1.884008in}{2.479985in}}%
\pgfpathlineto{\pgfqpoint{1.886600in}{2.607382in}}%
\pgfpathlineto{\pgfqpoint{1.888327in}{2.556073in}}%
\pgfpathlineto{\pgfqpoint{1.889190in}{2.573997in}}%
\pgfpathlineto{\pgfqpoint{1.890054in}{2.630686in}}%
\pgfpathlineto{\pgfqpoint{1.890918in}{2.542886in}}%
\pgfpathlineto{\pgfqpoint{1.891784in}{2.558380in}}%
\pgfpathlineto{\pgfqpoint{1.892650in}{2.552846in}}%
\pgfpathlineto{\pgfqpoint{1.893515in}{2.509745in}}%
\pgfpathlineto{\pgfqpoint{1.894381in}{2.537044in}}%
\pgfpathlineto{\pgfqpoint{1.895246in}{2.471316in}}%
\pgfpathlineto{\pgfqpoint{1.896112in}{2.488655in}}%
\pgfpathlineto{\pgfqpoint{1.896975in}{2.454593in}}%
\pgfpathlineto{\pgfqpoint{1.897841in}{2.558932in}}%
\pgfpathlineto{\pgfqpoint{1.898707in}{2.497940in}}%
\pgfpathlineto{\pgfqpoint{1.899572in}{2.535507in}}%
\pgfpathlineto{\pgfqpoint{1.900437in}{2.507838in}}%
\pgfpathlineto{\pgfqpoint{1.901303in}{2.569815in}}%
\pgfpathlineto{\pgfqpoint{1.903034in}{2.469779in}}%
\pgfpathlineto{\pgfqpoint{1.903899in}{2.507592in}}%
\pgfpathlineto{\pgfqpoint{1.904764in}{2.496034in}}%
\pgfpathlineto{\pgfqpoint{1.906494in}{2.579652in}}%
\pgfpathlineto{\pgfqpoint{1.907360in}{2.453179in}}%
\pgfpathlineto{\pgfqpoint{1.908225in}{2.587706in}}%
\pgfpathlineto{\pgfqpoint{1.909953in}{2.515708in}}%
\pgfpathlineto{\pgfqpoint{1.910818in}{2.562498in}}%
\pgfpathlineto{\pgfqpoint{1.911684in}{2.495541in}}%
\pgfpathlineto{\pgfqpoint{1.912550in}{2.571966in}}%
\pgfpathlineto{\pgfqpoint{1.915146in}{2.508944in}}%
\pgfpathlineto{\pgfqpoint{1.916011in}{2.524993in}}%
\pgfpathlineto{\pgfqpoint{1.916877in}{2.478541in}}%
\pgfpathlineto{\pgfqpoint{1.918607in}{2.558993in}}%
\pgfpathlineto{\pgfqpoint{1.920335in}{2.533354in}}%
\pgfpathlineto{\pgfqpoint{1.921200in}{2.478695in}}%
\pgfpathlineto{\pgfqpoint{1.922066in}{2.543684in}}%
\pgfpathlineto{\pgfqpoint{1.923796in}{2.479924in}}%
\pgfpathlineto{\pgfqpoint{1.924661in}{2.536120in}}%
\pgfpathlineto{\pgfqpoint{1.925527in}{2.483367in}}%
\pgfpathlineto{\pgfqpoint{1.927259in}{2.521488in}}%
\pgfpathlineto{\pgfqpoint{1.928122in}{2.519705in}}%
\pgfpathlineto{\pgfqpoint{1.928985in}{2.510604in}}%
\pgfpathlineto{\pgfqpoint{1.930715in}{2.524469in}}%
\pgfpathlineto{\pgfqpoint{1.931580in}{2.596499in}}%
\pgfpathlineto{\pgfqpoint{1.933312in}{2.499107in}}%
\pgfpathlineto{\pgfqpoint{1.934179in}{2.502858in}}%
\pgfpathlineto{\pgfqpoint{1.936775in}{2.465107in}}%
\pgfpathlineto{\pgfqpoint{1.937642in}{2.576148in}}%
\pgfpathlineto{\pgfqpoint{1.938507in}{2.495849in}}%
\pgfpathlineto{\pgfqpoint{1.939373in}{2.539073in}}%
\pgfpathlineto{\pgfqpoint{1.941967in}{2.491546in}}%
\pgfpathlineto{\pgfqpoint{1.942833in}{2.475990in}}%
\pgfpathlineto{\pgfqpoint{1.943696in}{2.561946in}}%
\pgfpathlineto{\pgfqpoint{1.944561in}{2.502060in}}%
\pgfpathlineto{\pgfqpoint{1.945425in}{2.508823in}}%
\pgfpathlineto{\pgfqpoint{1.946290in}{2.498617in}}%
\pgfpathlineto{\pgfqpoint{1.947154in}{2.509070in}}%
\pgfpathlineto{\pgfqpoint{1.948018in}{2.489824in}}%
\pgfpathlineto{\pgfqpoint{1.948883in}{2.515833in}}%
\pgfpathlineto{\pgfqpoint{1.949748in}{2.515310in}}%
\pgfpathlineto{\pgfqpoint{1.953207in}{2.469104in}}%
\pgfpathlineto{\pgfqpoint{1.954072in}{2.548112in}}%
\pgfpathlineto{\pgfqpoint{1.954936in}{2.475713in}}%
\pgfpathlineto{\pgfqpoint{1.956666in}{2.599267in}}%
\pgfpathlineto{\pgfqpoint{1.957531in}{2.548112in}}%
\pgfpathlineto{\pgfqpoint{1.958396in}{2.548420in}}%
\pgfpathlineto{\pgfqpoint{1.959262in}{2.527668in}}%
\pgfpathlineto{\pgfqpoint{1.960126in}{2.541349in}}%
\pgfpathlineto{\pgfqpoint{1.962723in}{2.484229in}}%
\pgfpathlineto{\pgfqpoint{1.963588in}{2.546021in}}%
\pgfpathlineto{\pgfqpoint{1.964454in}{2.490469in}}%
\pgfpathlineto{\pgfqpoint{1.965319in}{2.564158in}}%
\pgfpathlineto{\pgfqpoint{1.967048in}{2.534768in}}%
\pgfpathlineto{\pgfqpoint{1.967912in}{2.528621in}}%
\pgfpathlineto{\pgfqpoint{1.968778in}{2.541931in}}%
\pgfpathlineto{\pgfqpoint{1.970509in}{2.516631in}}%
\pgfpathlineto{\pgfqpoint{1.971375in}{2.543807in}}%
\pgfpathlineto{\pgfqpoint{1.972240in}{2.483921in}}%
\pgfpathlineto{\pgfqpoint{1.973106in}{2.546329in}}%
\pgfpathlineto{\pgfqpoint{1.973972in}{2.426526in}}%
\pgfpathlineto{\pgfqpoint{1.974835in}{2.477527in}}%
\pgfpathlineto{\pgfqpoint{1.975702in}{2.605661in}}%
\pgfpathlineto{\pgfqpoint{1.977433in}{2.473899in}}%
\pgfpathlineto{\pgfqpoint{1.978298in}{2.499661in}}%
\pgfpathlineto{\pgfqpoint{1.979162in}{2.470087in}}%
\pgfpathlineto{\pgfqpoint{1.980028in}{2.499415in}}%
\pgfpathlineto{\pgfqpoint{1.980894in}{2.580391in}}%
\pgfpathlineto{\pgfqpoint{1.981760in}{2.505011in}}%
\pgfpathlineto{\pgfqpoint{1.982625in}{2.518907in}}%
\pgfpathlineto{\pgfqpoint{1.983491in}{2.492960in}}%
\pgfpathlineto{\pgfqpoint{1.985220in}{2.532556in}}%
\pgfpathlineto{\pgfqpoint{1.986085in}{2.506794in}}%
\pgfpathlineto{\pgfqpoint{1.986949in}{2.525793in}}%
\pgfpathlineto{\pgfqpoint{1.987814in}{2.479372in}}%
\pgfpathlineto{\pgfqpoint{1.988674in}{2.544176in}}%
\pgfpathlineto{\pgfqpoint{1.989540in}{2.514909in}}%
\pgfpathlineto{\pgfqpoint{1.990405in}{2.580299in}}%
\pgfpathlineto{\pgfqpoint{1.991267in}{2.527453in}}%
\pgfpathlineto{\pgfqpoint{1.992997in}{2.564651in}}%
\pgfpathlineto{\pgfqpoint{1.996456in}{2.479864in}}%
\pgfpathlineto{\pgfqpoint{1.998186in}{2.563175in}}%
\pgfpathlineto{\pgfqpoint{1.999051in}{2.520813in}}%
\pgfpathlineto{\pgfqpoint{1.999915in}{2.555583in}}%
\pgfpathlineto{\pgfqpoint{2.000777in}{2.501937in}}%
\pgfpathlineto{\pgfqpoint{2.001642in}{2.536369in}}%
\pgfpathlineto{\pgfqpoint{2.002508in}{2.488195in}}%
\pgfpathlineto{\pgfqpoint{2.003373in}{2.507410in}}%
\pgfpathlineto{\pgfqpoint{2.004238in}{2.491854in}}%
\pgfpathlineto{\pgfqpoint{2.005968in}{2.508947in}}%
\pgfpathlineto{\pgfqpoint{2.006833in}{2.539196in}}%
\pgfpathlineto{\pgfqpoint{2.008563in}{2.502920in}}%
\pgfpathlineto{\pgfqpoint{2.009429in}{2.576548in}}%
\pgfpathlineto{\pgfqpoint{2.010294in}{2.567111in}}%
\pgfpathlineto{\pgfqpoint{2.011158in}{2.580206in}}%
\pgfpathlineto{\pgfqpoint{2.012887in}{2.541410in}}%
\pgfpathlineto{\pgfqpoint{2.013752in}{2.549957in}}%
\pgfpathlineto{\pgfqpoint{2.015482in}{2.479926in}}%
\pgfpathlineto{\pgfqpoint{2.016347in}{2.494989in}}%
\pgfpathlineto{\pgfqpoint{2.017211in}{2.480786in}}%
\pgfpathlineto{\pgfqpoint{2.018077in}{2.505503in}}%
\pgfpathlineto{\pgfqpoint{2.018939in}{2.488103in}}%
\pgfpathlineto{\pgfqpoint{2.019804in}{2.544853in}}%
\pgfpathlineto{\pgfqpoint{2.020669in}{2.524256in}}%
\pgfpathlineto{\pgfqpoint{2.021535in}{2.540733in}}%
\pgfpathlineto{\pgfqpoint{2.022400in}{2.480663in}}%
\pgfpathlineto{\pgfqpoint{2.024995in}{2.586415in}}%
\pgfpathlineto{\pgfqpoint{2.025860in}{2.463631in}}%
\pgfpathlineto{\pgfqpoint{2.026727in}{2.570616in}}%
\pgfpathlineto{\pgfqpoint{2.029320in}{2.493268in}}%
\pgfpathlineto{\pgfqpoint{2.030186in}{2.566988in}}%
\pgfpathlineto{\pgfqpoint{2.031052in}{2.475621in}}%
\pgfpathlineto{\pgfqpoint{2.032782in}{2.544299in}}%
\pgfpathlineto{\pgfqpoint{2.033646in}{2.518599in}}%
\pgfpathlineto{\pgfqpoint{2.034510in}{2.559732in}}%
\pgfpathlineto{\pgfqpoint{2.035374in}{2.538950in}}%
\pgfpathlineto{\pgfqpoint{2.036239in}{2.614084in}}%
\pgfpathlineto{\pgfqpoint{2.037104in}{2.594040in}}%
\pgfpathlineto{\pgfqpoint{2.040565in}{2.480109in}}%
\pgfpathlineto{\pgfqpoint{2.041430in}{2.485519in}}%
\pgfpathlineto{\pgfqpoint{2.042296in}{2.549095in}}%
\pgfpathlineto{\pgfqpoint{2.043162in}{2.508085in}}%
\pgfpathlineto{\pgfqpoint{2.044026in}{2.538273in}}%
\pgfpathlineto{\pgfqpoint{2.044891in}{2.505871in}}%
\pgfpathlineto{\pgfqpoint{2.046620in}{2.622507in}}%
\pgfpathlineto{\pgfqpoint{2.047484in}{2.500860in}}%
\pgfpathlineto{\pgfqpoint{2.048349in}{2.578054in}}%
\pgfpathlineto{\pgfqpoint{2.049214in}{2.533293in}}%
\pgfpathlineto{\pgfqpoint{2.050080in}{2.597851in}}%
\pgfpathlineto{\pgfqpoint{2.050945in}{2.550324in}}%
\pgfpathlineto{\pgfqpoint{2.053538in}{2.604922in}}%
\pgfpathlineto{\pgfqpoint{2.054402in}{2.534799in}}%
\pgfpathlineto{\pgfqpoint{2.055268in}{2.577561in}}%
\pgfpathlineto{\pgfqpoint{2.056131in}{2.502735in}}%
\pgfpathlineto{\pgfqpoint{2.057861in}{2.557456in}}%
\pgfpathlineto{\pgfqpoint{2.058726in}{2.568956in}}%
\pgfpathlineto{\pgfqpoint{2.059591in}{2.494651in}}%
\pgfpathlineto{\pgfqpoint{2.060456in}{2.590228in}}%
\pgfpathlineto{\pgfqpoint{2.061322in}{2.582297in}}%
\pgfpathlineto{\pgfqpoint{2.062187in}{2.525547in}}%
\pgfpathlineto{\pgfqpoint{2.063052in}{2.529236in}}%
\pgfpathlineto{\pgfqpoint{2.064780in}{2.496341in}}%
\pgfpathlineto{\pgfqpoint{2.065646in}{2.588445in}}%
\pgfpathlineto{\pgfqpoint{2.066512in}{2.574395in}}%
\pgfpathlineto{\pgfqpoint{2.067377in}{2.604247in}}%
\pgfpathlineto{\pgfqpoint{2.068243in}{2.514786in}}%
\pgfpathlineto{\pgfqpoint{2.069109in}{2.519736in}}%
\pgfpathlineto{\pgfqpoint{2.069974in}{2.497755in}}%
\pgfpathlineto{\pgfqpoint{2.070838in}{2.513188in}}%
\pgfpathlineto{\pgfqpoint{2.072565in}{2.485643in}}%
\pgfpathlineto{\pgfqpoint{2.073429in}{2.579898in}}%
\pgfpathlineto{\pgfqpoint{2.074294in}{2.557887in}}%
\pgfpathlineto{\pgfqpoint{2.075160in}{2.546390in}}%
\pgfpathlineto{\pgfqpoint{2.076025in}{2.496495in}}%
\pgfpathlineto{\pgfqpoint{2.076890in}{2.592319in}}%
\pgfpathlineto{\pgfqpoint{2.077754in}{2.491361in}}%
\pgfpathlineto{\pgfqpoint{2.078618in}{2.520105in}}%
\pgfpathlineto{\pgfqpoint{2.079483in}{2.503228in}}%
\pgfpathlineto{\pgfqpoint{2.080348in}{2.521488in}}%
\pgfpathlineto{\pgfqpoint{2.081213in}{2.495911in}}%
\pgfpathlineto{\pgfqpoint{2.082078in}{2.556043in}}%
\pgfpathlineto{\pgfqpoint{2.082942in}{2.492313in}}%
\pgfpathlineto{\pgfqpoint{2.083807in}{2.505932in}}%
\pgfpathlineto{\pgfqpoint{2.084671in}{2.550693in}}%
\pgfpathlineto{\pgfqpoint{2.085536in}{2.436425in}}%
\pgfpathlineto{\pgfqpoint{2.086401in}{2.518907in}}%
\pgfpathlineto{\pgfqpoint{2.087265in}{2.517860in}}%
\pgfpathlineto{\pgfqpoint{2.088130in}{2.516631in}}%
\pgfpathlineto{\pgfqpoint{2.089859in}{2.547496in}}%
\pgfpathlineto{\pgfqpoint{2.090724in}{2.525424in}}%
\pgfpathlineto{\pgfqpoint{2.091589in}{2.564527in}}%
\pgfpathlineto{\pgfqpoint{2.092454in}{2.452258in}}%
\pgfpathlineto{\pgfqpoint{2.095049in}{2.583896in}}%
\pgfpathlineto{\pgfqpoint{2.095914in}{2.512636in}}%
\pgfpathlineto{\pgfqpoint{2.096780in}{2.520320in}}%
\pgfpathlineto{\pgfqpoint{2.097644in}{2.588014in}}%
\pgfpathlineto{\pgfqpoint{2.098507in}{2.578238in}}%
\pgfpathlineto{\pgfqpoint{2.099373in}{2.566618in}}%
\pgfpathlineto{\pgfqpoint{2.101101in}{2.530034in}}%
\pgfpathlineto{\pgfqpoint{2.101963in}{2.547004in}}%
\pgfpathlineto{\pgfqpoint{2.102828in}{2.533293in}}%
\pgfpathlineto{\pgfqpoint{2.103692in}{2.557456in}}%
\pgfpathlineto{\pgfqpoint{2.104558in}{2.520872in}}%
\pgfpathlineto{\pgfqpoint{2.105422in}{2.583095in}}%
\pgfpathlineto{\pgfqpoint{2.108016in}{2.477404in}}%
\pgfpathlineto{\pgfqpoint{2.109746in}{2.566003in}}%
\pgfpathlineto{\pgfqpoint{2.110611in}{2.594102in}}%
\pgfpathlineto{\pgfqpoint{2.112342in}{2.503228in}}%
\pgfpathlineto{\pgfqpoint{2.113207in}{2.499538in}}%
\pgfpathlineto{\pgfqpoint{2.114071in}{2.508146in}}%
\pgfpathlineto{\pgfqpoint{2.114938in}{2.562436in}}%
\pgfpathlineto{\pgfqpoint{2.115804in}{2.503351in}}%
\pgfpathlineto{\pgfqpoint{2.116670in}{2.512541in}}%
\pgfpathlineto{\pgfqpoint{2.117537in}{2.466890in}}%
\pgfpathlineto{\pgfqpoint{2.118403in}{2.593548in}}%
\pgfpathlineto{\pgfqpoint{2.119267in}{2.564343in}}%
\pgfpathlineto{\pgfqpoint{2.120134in}{2.567232in}}%
\pgfpathlineto{\pgfqpoint{2.120999in}{2.556964in}}%
\pgfpathlineto{\pgfqpoint{2.122731in}{2.587952in}}%
\pgfpathlineto{\pgfqpoint{2.124463in}{2.500521in}}%
\pgfpathlineto{\pgfqpoint{2.125328in}{2.582849in}}%
\pgfpathlineto{\pgfqpoint{2.126193in}{2.524069in}}%
\pgfpathlineto{\pgfqpoint{2.127057in}{2.591457in}}%
\pgfpathlineto{\pgfqpoint{2.127924in}{2.503780in}}%
\pgfpathlineto{\pgfqpoint{2.128790in}{2.525791in}}%
\pgfpathlineto{\pgfqpoint{2.129654in}{2.563912in}}%
\pgfpathlineto{\pgfqpoint{2.130520in}{2.516169in}}%
\pgfpathlineto{\pgfqpoint{2.131385in}{2.528436in}}%
\pgfpathlineto{\pgfqpoint{2.132251in}{2.492590in}}%
\pgfpathlineto{\pgfqpoint{2.133982in}{2.619002in}}%
\pgfpathlineto{\pgfqpoint{2.135713in}{2.510697in}}%
\pgfpathlineto{\pgfqpoint{2.136575in}{2.567355in}}%
\pgfpathlineto{\pgfqpoint{2.137440in}{2.538950in}}%
\pgfpathlineto{\pgfqpoint{2.138303in}{2.577746in}}%
\pgfpathlineto{\pgfqpoint{2.140031in}{2.509191in}}%
\pgfpathlineto{\pgfqpoint{2.140897in}{2.578238in}}%
\pgfpathlineto{\pgfqpoint{2.142629in}{2.520228in}}%
\pgfpathlineto{\pgfqpoint{2.143495in}{2.524870in}}%
\pgfpathlineto{\pgfqpoint{2.144360in}{2.522840in}}%
\pgfpathlineto{\pgfqpoint{2.145225in}{2.495418in}}%
\pgfpathlineto{\pgfqpoint{2.146091in}{2.539196in}}%
\pgfpathlineto{\pgfqpoint{2.147822in}{2.459819in}}%
\pgfpathlineto{\pgfqpoint{2.148687in}{2.466767in}}%
\pgfpathlineto{\pgfqpoint{2.149554in}{2.464368in}}%
\pgfpathlineto{\pgfqpoint{2.150418in}{2.473407in}}%
\pgfpathlineto{\pgfqpoint{2.151284in}{2.535507in}}%
\pgfpathlineto{\pgfqpoint{2.152149in}{2.476850in}}%
\pgfpathlineto{\pgfqpoint{2.153015in}{2.479249in}}%
\pgfpathlineto{\pgfqpoint{2.155607in}{2.547866in}}%
\pgfpathlineto{\pgfqpoint{2.156472in}{2.545652in}}%
\pgfpathlineto{\pgfqpoint{2.157337in}{2.501260in}}%
\pgfpathlineto{\pgfqpoint{2.158202in}{2.605291in}}%
\pgfpathlineto{\pgfqpoint{2.159928in}{2.486564in}}%
\pgfpathlineto{\pgfqpoint{2.160794in}{2.496955in}}%
\pgfpathlineto{\pgfqpoint{2.162520in}{2.528618in}}%
\pgfpathlineto{\pgfqpoint{2.163384in}{2.517306in}}%
\pgfpathlineto{\pgfqpoint{2.164251in}{2.454254in}}%
\pgfpathlineto{\pgfqpoint{2.165115in}{2.611562in}}%
\pgfpathlineto{\pgfqpoint{2.166844in}{2.521365in}}%
\pgfpathlineto{\pgfqpoint{2.167707in}{2.583526in}}%
\pgfpathlineto{\pgfqpoint{2.168572in}{2.516015in}}%
\pgfpathlineto{\pgfqpoint{2.169437in}{2.589797in}}%
\pgfpathlineto{\pgfqpoint{2.170301in}{2.580512in}}%
\pgfpathlineto{\pgfqpoint{2.171166in}{2.583280in}}%
\pgfpathlineto{\pgfqpoint{2.172893in}{2.520136in}}%
\pgfpathlineto{\pgfqpoint{2.173758in}{2.514448in}}%
\pgfpathlineto{\pgfqpoint{2.174620in}{2.444756in}}%
\pgfpathlineto{\pgfqpoint{2.175484in}{2.538765in}}%
\pgfpathlineto{\pgfqpoint{2.176349in}{2.510789in}}%
\pgfpathlineto{\pgfqpoint{2.177215in}{2.628839in}}%
\pgfpathlineto{\pgfqpoint{2.178945in}{2.488409in}}%
\pgfpathlineto{\pgfqpoint{2.179810in}{2.528128in}}%
\pgfpathlineto{\pgfqpoint{2.181542in}{2.491790in}}%
\pgfpathlineto{\pgfqpoint{2.182405in}{2.541962in}}%
\pgfpathlineto{\pgfqpoint{2.183270in}{2.475128in}}%
\pgfpathlineto{\pgfqpoint{2.184136in}{2.541716in}}%
\pgfpathlineto{\pgfqpoint{2.185000in}{2.531941in}}%
\pgfpathlineto{\pgfqpoint{2.185866in}{2.524500in}}%
\pgfpathlineto{\pgfqpoint{2.186731in}{2.570860in}}%
\pgfpathlineto{\pgfqpoint{2.187597in}{2.488993in}}%
\pgfpathlineto{\pgfqpoint{2.190192in}{2.630776in}}%
\pgfpathlineto{\pgfqpoint{2.191056in}{2.542024in}}%
\pgfpathlineto{\pgfqpoint{2.191922in}{2.565633in}}%
\pgfpathlineto{\pgfqpoint{2.192788in}{2.507500in}}%
\pgfpathlineto{\pgfqpoint{2.193652in}{2.550201in}}%
\pgfpathlineto{\pgfqpoint{2.194519in}{2.544420in}}%
\pgfpathlineto{\pgfqpoint{2.195384in}{2.517183in}}%
\pgfpathlineto{\pgfqpoint{2.196249in}{2.549831in}}%
\pgfpathlineto{\pgfqpoint{2.199710in}{2.483244in}}%
\pgfpathlineto{\pgfqpoint{2.202305in}{2.549831in}}%
\pgfpathlineto{\pgfqpoint{2.203171in}{2.539625in}}%
\pgfpathlineto{\pgfqpoint{2.204900in}{2.500706in}}%
\pgfpathlineto{\pgfqpoint{2.206632in}{2.562375in}}%
\pgfpathlineto{\pgfqpoint{2.207495in}{2.526222in}}%
\pgfpathlineto{\pgfqpoint{2.208361in}{2.547127in}}%
\pgfpathlineto{\pgfqpoint{2.209225in}{2.466090in}}%
\pgfpathlineto{\pgfqpoint{2.210954in}{2.504026in}}%
\pgfpathlineto{\pgfqpoint{2.211819in}{2.493789in}}%
\pgfpathlineto{\pgfqpoint{2.213551in}{2.553705in}}%
\pgfpathlineto{\pgfqpoint{2.215281in}{2.523148in}}%
\pgfpathlineto{\pgfqpoint{2.217011in}{2.547312in}}%
\pgfpathlineto{\pgfqpoint{2.217875in}{2.509683in}}%
\pgfpathlineto{\pgfqpoint{2.218741in}{2.533324in}}%
\pgfpathlineto{\pgfqpoint{2.220471in}{2.512634in}}%
\pgfpathlineto{\pgfqpoint{2.221337in}{2.537934in}}%
\pgfpathlineto{\pgfqpoint{2.222202in}{2.503718in}}%
\pgfpathlineto{\pgfqpoint{2.223065in}{2.522779in}}%
\pgfpathlineto{\pgfqpoint{2.223931in}{2.485089in}}%
\pgfpathlineto{\pgfqpoint{2.224796in}{2.549770in}}%
\pgfpathlineto{\pgfqpoint{2.225661in}{2.527851in}}%
\pgfpathlineto{\pgfqpoint{2.226526in}{2.532862in}}%
\pgfpathlineto{\pgfqpoint{2.227389in}{2.550447in}}%
\pgfpathlineto{\pgfqpoint{2.228254in}{2.518751in}}%
\pgfpathlineto{\pgfqpoint{2.229120in}{2.558809in}}%
\pgfpathlineto{\pgfqpoint{2.229986in}{2.536428in}}%
\pgfpathlineto{\pgfqpoint{2.231717in}{2.580820in}}%
\pgfpathlineto{\pgfqpoint{2.233444in}{2.519028in}}%
\pgfpathlineto{\pgfqpoint{2.234309in}{2.536367in}}%
\pgfpathlineto{\pgfqpoint{2.235174in}{2.508791in}}%
\pgfpathlineto{\pgfqpoint{2.236039in}{2.557333in}}%
\pgfpathlineto{\pgfqpoint{2.238633in}{2.474882in}}%
\pgfpathlineto{\pgfqpoint{2.239497in}{2.549587in}}%
\pgfpathlineto{\pgfqpoint{2.240361in}{2.486012in}}%
\pgfpathlineto{\pgfqpoint{2.241226in}{2.486566in}}%
\pgfpathlineto{\pgfqpoint{2.242955in}{2.599021in}}%
\pgfpathlineto{\pgfqpoint{2.243820in}{2.573258in}}%
\pgfpathlineto{\pgfqpoint{2.244685in}{2.513773in}}%
\pgfpathlineto{\pgfqpoint{2.245550in}{2.556166in}}%
\pgfpathlineto{\pgfqpoint{2.246416in}{2.553890in}}%
\pgfpathlineto{\pgfqpoint{2.247281in}{2.576886in}}%
\pgfpathlineto{\pgfqpoint{2.248147in}{2.565143in}}%
\pgfpathlineto{\pgfqpoint{2.249009in}{2.571352in}}%
\pgfpathlineto{\pgfqpoint{2.249874in}{2.516416in}}%
\pgfpathlineto{\pgfqpoint{2.250740in}{2.572951in}}%
\pgfpathlineto{\pgfqpoint{2.252471in}{2.500891in}}%
\pgfpathlineto{\pgfqpoint{2.254201in}{2.545005in}}%
\pgfpathlineto{\pgfqpoint{2.255066in}{2.503780in}}%
\pgfpathlineto{\pgfqpoint{2.255930in}{2.567232in}}%
\pgfpathlineto{\pgfqpoint{2.256794in}{2.565972in}}%
\pgfpathlineto{\pgfqpoint{2.258523in}{2.555612in}}%
\pgfpathlineto{\pgfqpoint{2.259389in}{2.602217in}}%
\pgfpathlineto{\pgfqpoint{2.260254in}{2.557580in}}%
\pgfpathlineto{\pgfqpoint{2.261118in}{2.592134in}}%
\pgfpathlineto{\pgfqpoint{2.262848in}{2.536120in}}%
\pgfpathlineto{\pgfqpoint{2.263712in}{2.553305in}}%
\pgfpathlineto{\pgfqpoint{2.264577in}{2.522286in}}%
\pgfpathlineto{\pgfqpoint{2.265441in}{2.537780in}}%
\pgfpathlineto{\pgfqpoint{2.266306in}{2.583493in}}%
\pgfpathlineto{\pgfqpoint{2.267171in}{2.510050in}}%
\pgfpathlineto{\pgfqpoint{2.268032in}{2.620906in}}%
\pgfpathlineto{\pgfqpoint{2.268898in}{2.530863in}}%
\pgfpathlineto{\pgfqpoint{2.269763in}{2.561821in}}%
\pgfpathlineto{\pgfqpoint{2.271490in}{2.505624in}}%
\pgfpathlineto{\pgfqpoint{2.273219in}{2.618879in}}%
\pgfpathlineto{\pgfqpoint{2.274084in}{2.540118in}}%
\pgfpathlineto{\pgfqpoint{2.274949in}{2.619924in}}%
\pgfpathlineto{\pgfqpoint{2.277541in}{2.553151in}}%
\pgfpathlineto{\pgfqpoint{2.278407in}{2.569752in}}%
\pgfpathlineto{\pgfqpoint{2.280137in}{2.481522in}}%
\pgfpathlineto{\pgfqpoint{2.281868in}{2.584325in}}%
\pgfpathlineto{\pgfqpoint{2.282732in}{2.480232in}}%
\pgfpathlineto{\pgfqpoint{2.283598in}{2.485150in}}%
\pgfpathlineto{\pgfqpoint{2.284462in}{2.476788in}}%
\pgfpathlineto{\pgfqpoint{2.286192in}{2.504642in}}%
\pgfpathlineto{\pgfqpoint{2.287056in}{2.497263in}}%
\pgfpathlineto{\pgfqpoint{2.288787in}{2.461356in}}%
\pgfpathlineto{\pgfqpoint{2.289652in}{2.584940in}}%
\pgfpathlineto{\pgfqpoint{2.290518in}{2.524192in}}%
\pgfpathlineto{\pgfqpoint{2.291384in}{2.565818in}}%
\pgfpathlineto{\pgfqpoint{2.293114in}{2.444756in}}%
\pgfpathlineto{\pgfqpoint{2.293978in}{2.589735in}}%
\pgfpathlineto{\pgfqpoint{2.294842in}{2.506117in}}%
\pgfpathlineto{\pgfqpoint{2.295707in}{2.520872in}}%
\pgfpathlineto{\pgfqpoint{2.297435in}{2.497324in}}%
\pgfpathlineto{\pgfqpoint{2.298302in}{2.609902in}}%
\pgfpathlineto{\pgfqpoint{2.299167in}{2.485150in}}%
\pgfpathlineto{\pgfqpoint{2.300033in}{2.608180in}}%
\pgfpathlineto{\pgfqpoint{2.300900in}{2.545036in}}%
\pgfpathlineto{\pgfqpoint{2.301766in}{2.598220in}}%
\pgfpathlineto{\pgfqpoint{2.304361in}{2.519674in}}%
\pgfpathlineto{\pgfqpoint{2.306957in}{2.547004in}}%
\pgfpathlineto{\pgfqpoint{2.307822in}{2.506486in}}%
\pgfpathlineto{\pgfqpoint{2.308688in}{2.516446in}}%
\pgfpathlineto{\pgfqpoint{2.309554in}{2.516262in}}%
\pgfpathlineto{\pgfqpoint{2.310420in}{2.518660in}}%
\pgfpathlineto{\pgfqpoint{2.311285in}{2.488993in}}%
\pgfpathlineto{\pgfqpoint{2.313017in}{2.581066in}}%
\pgfpathlineto{\pgfqpoint{2.313883in}{2.493635in}}%
\pgfpathlineto{\pgfqpoint{2.314749in}{2.515831in}}%
\pgfpathlineto{\pgfqpoint{2.315613in}{2.582972in}}%
\pgfpathlineto{\pgfqpoint{2.317342in}{2.498677in}}%
\pgfpathlineto{\pgfqpoint{2.318206in}{2.502304in}}%
\pgfpathlineto{\pgfqpoint{2.319937in}{2.566372in}}%
\pgfpathlineto{\pgfqpoint{2.320802in}{2.546144in}}%
\pgfpathlineto{\pgfqpoint{2.321668in}{2.557580in}}%
\pgfpathlineto{\pgfqpoint{2.322533in}{2.550139in}}%
\pgfpathlineto{\pgfqpoint{2.323399in}{2.572643in}}%
\pgfpathlineto{\pgfqpoint{2.325993in}{2.469717in}}%
\pgfpathlineto{\pgfqpoint{2.327723in}{2.553890in}}%
\pgfpathlineto{\pgfqpoint{2.328589in}{2.506117in}}%
\pgfpathlineto{\pgfqpoint{2.329452in}{2.625889in}}%
\pgfpathlineto{\pgfqpoint{2.331179in}{2.508146in}}%
\pgfpathlineto{\pgfqpoint{2.332043in}{2.520567in}}%
\pgfpathlineto{\pgfqpoint{2.332908in}{2.501814in}}%
\pgfpathlineto{\pgfqpoint{2.333775in}{2.519091in}}%
\pgfpathlineto{\pgfqpoint{2.334640in}{2.513188in}}%
\pgfpathlineto{\pgfqpoint{2.335506in}{2.516416in}}%
\pgfpathlineto{\pgfqpoint{2.337235in}{2.586600in}}%
\pgfpathlineto{\pgfqpoint{2.338965in}{2.486749in}}%
\pgfpathlineto{\pgfqpoint{2.339831in}{2.495295in}}%
\pgfpathlineto{\pgfqpoint{2.340692in}{2.537105in}}%
\pgfpathlineto{\pgfqpoint{2.341556in}{2.522717in}}%
\pgfpathlineto{\pgfqpoint{2.342422in}{2.454500in}}%
\pgfpathlineto{\pgfqpoint{2.343286in}{2.531879in}}%
\pgfpathlineto{\pgfqpoint{2.345881in}{2.471685in}}%
\pgfpathlineto{\pgfqpoint{2.346747in}{2.497817in}}%
\pgfpathlineto{\pgfqpoint{2.347612in}{2.494558in}}%
\pgfpathlineto{\pgfqpoint{2.348476in}{2.481645in}}%
\pgfpathlineto{\pgfqpoint{2.349341in}{2.491606in}}%
\pgfpathlineto{\pgfqpoint{2.350207in}{2.544420in}}%
\pgfpathlineto{\pgfqpoint{2.351073in}{2.519335in}}%
\pgfpathlineto{\pgfqpoint{2.351939in}{2.455668in}}%
\pgfpathlineto{\pgfqpoint{2.353668in}{2.531202in}}%
\pgfpathlineto{\pgfqpoint{2.354534in}{2.504241in}}%
\pgfpathlineto{\pgfqpoint{2.356262in}{2.552292in}}%
\pgfpathlineto{\pgfqpoint{2.357127in}{2.497386in}}%
\pgfpathlineto{\pgfqpoint{2.358859in}{2.565818in}}%
\pgfpathlineto{\pgfqpoint{2.359724in}{2.492221in}}%
\pgfpathlineto{\pgfqpoint{2.361451in}{2.571198in}}%
\pgfpathlineto{\pgfqpoint{2.362316in}{2.547127in}}%
\pgfpathlineto{\pgfqpoint{2.363181in}{2.566372in}}%
\pgfpathlineto{\pgfqpoint{2.364045in}{2.489055in}}%
\pgfpathlineto{\pgfqpoint{2.364909in}{2.576825in}}%
\pgfpathlineto{\pgfqpoint{2.365771in}{2.559424in}}%
\pgfpathlineto{\pgfqpoint{2.366638in}{2.538519in}}%
\pgfpathlineto{\pgfqpoint{2.367503in}{2.490069in}}%
\pgfpathlineto{\pgfqpoint{2.368366in}{2.495141in}}%
\pgfpathlineto{\pgfqpoint{2.370096in}{2.565880in}}%
\pgfpathlineto{\pgfqpoint{2.371826in}{2.518045in}}%
\pgfpathlineto{\pgfqpoint{2.372691in}{2.543499in}}%
\pgfpathlineto{\pgfqpoint{2.373557in}{2.492221in}}%
\pgfpathlineto{\pgfqpoint{2.375287in}{2.569077in}}%
\pgfpathlineto{\pgfqpoint{2.376151in}{2.577808in}}%
\pgfpathlineto{\pgfqpoint{2.377017in}{2.603508in}}%
\pgfpathlineto{\pgfqpoint{2.379616in}{2.537536in}}%
\pgfpathlineto{\pgfqpoint{2.380481in}{2.599421in}}%
\pgfpathlineto{\pgfqpoint{2.383076in}{2.525577in}}%
\pgfpathlineto{\pgfqpoint{2.383941in}{2.566803in}}%
\pgfpathlineto{\pgfqpoint{2.385670in}{2.453856in}}%
\pgfpathlineto{\pgfqpoint{2.386535in}{2.561515in}}%
\pgfpathlineto{\pgfqpoint{2.387399in}{2.501168in}}%
\pgfpathlineto{\pgfqpoint{2.388265in}{2.516139in}}%
\pgfpathlineto{\pgfqpoint{2.389131in}{2.567724in}}%
\pgfpathlineto{\pgfqpoint{2.389995in}{2.557857in}}%
\pgfpathlineto{\pgfqpoint{2.390860in}{2.546452in}}%
\pgfpathlineto{\pgfqpoint{2.391725in}{2.555306in}}%
\pgfpathlineto{\pgfqpoint{2.392590in}{2.544669in}}%
\pgfpathlineto{\pgfqpoint{2.393455in}{2.571229in}}%
\pgfpathlineto{\pgfqpoint{2.394320in}{2.565143in}}%
\pgfpathlineto{\pgfqpoint{2.396915in}{2.494189in}}%
\pgfpathlineto{\pgfqpoint{2.397781in}{2.548787in}}%
\pgfpathlineto{\pgfqpoint{2.398647in}{2.543807in}}%
\pgfpathlineto{\pgfqpoint{2.399511in}{2.464737in}}%
\pgfpathlineto{\pgfqpoint{2.400376in}{2.605476in}}%
\pgfpathlineto{\pgfqpoint{2.402105in}{2.529911in}}%
\pgfpathlineto{\pgfqpoint{2.403834in}{2.584940in}}%
\pgfpathlineto{\pgfqpoint{2.404699in}{2.550324in}}%
\pgfpathlineto{\pgfqpoint{2.405564in}{2.595393in}}%
\pgfpathlineto{\pgfqpoint{2.406429in}{2.484167in}}%
\pgfpathlineto{\pgfqpoint{2.408157in}{2.535199in}}%
\pgfpathlineto{\pgfqpoint{2.409022in}{2.542547in}}%
\pgfpathlineto{\pgfqpoint{2.409887in}{2.518168in}}%
\pgfpathlineto{\pgfqpoint{2.410753in}{2.545528in}}%
\pgfpathlineto{\pgfqpoint{2.411618in}{2.540025in}}%
\pgfpathlineto{\pgfqpoint{2.412483in}{2.545467in}}%
\pgfpathlineto{\pgfqpoint{2.414213in}{2.507838in}}%
\pgfpathlineto{\pgfqpoint{2.415079in}{2.557518in}}%
\pgfpathlineto{\pgfqpoint{2.415943in}{2.509868in}}%
\pgfpathlineto{\pgfqpoint{2.416808in}{2.573012in}}%
\pgfpathlineto{\pgfqpoint{2.417672in}{2.545159in}}%
\pgfpathlineto{\pgfqpoint{2.419402in}{2.596622in}}%
\pgfpathlineto{\pgfqpoint{2.421131in}{2.528097in}}%
\pgfpathlineto{\pgfqpoint{2.421995in}{2.555920in}}%
\pgfpathlineto{\pgfqpoint{2.423725in}{2.528621in}}%
\pgfpathlineto{\pgfqpoint{2.424590in}{2.575288in}}%
\pgfpathlineto{\pgfqpoint{2.425455in}{2.515463in}}%
\pgfpathlineto{\pgfqpoint{2.427184in}{2.568278in}}%
\pgfpathlineto{\pgfqpoint{2.428912in}{2.473838in}}%
\pgfpathlineto{\pgfqpoint{2.429777in}{2.533724in}}%
\pgfpathlineto{\pgfqpoint{2.430641in}{2.503105in}}%
\pgfpathlineto{\pgfqpoint{2.432370in}{2.563421in}}%
\pgfpathlineto{\pgfqpoint{2.433233in}{2.470272in}}%
\pgfpathlineto{\pgfqpoint{2.435826in}{2.575165in}}%
\pgfpathlineto{\pgfqpoint{2.436690in}{2.580637in}}%
\pgfpathlineto{\pgfqpoint{2.437555in}{2.524041in}}%
\pgfpathlineto{\pgfqpoint{2.438420in}{2.527022in}}%
\pgfpathlineto{\pgfqpoint{2.439284in}{2.535568in}}%
\pgfpathlineto{\pgfqpoint{2.440150in}{2.566341in}}%
\pgfpathlineto{\pgfqpoint{2.441015in}{2.514848in}}%
\pgfpathlineto{\pgfqpoint{2.441880in}{2.528066in}}%
\pgfpathlineto{\pgfqpoint{2.442744in}{2.481707in}}%
\pgfpathlineto{\pgfqpoint{2.443608in}{2.507962in}}%
\pgfpathlineto{\pgfqpoint{2.444474in}{2.482323in}}%
\pgfpathlineto{\pgfqpoint{2.445339in}{2.550816in}}%
\pgfpathlineto{\pgfqpoint{2.446201in}{2.545098in}}%
\pgfpathlineto{\pgfqpoint{2.447066in}{2.503595in}}%
\pgfpathlineto{\pgfqpoint{2.448795in}{2.553644in}}%
\pgfpathlineto{\pgfqpoint{2.449658in}{2.531202in}}%
\pgfpathlineto{\pgfqpoint{2.450524in}{2.577623in}}%
\pgfpathlineto{\pgfqpoint{2.452253in}{2.523210in}}%
\pgfpathlineto{\pgfqpoint{2.453117in}{2.556533in}}%
\pgfpathlineto{\pgfqpoint{2.453982in}{2.507654in}}%
\pgfpathlineto{\pgfqpoint{2.454848in}{2.562683in}}%
\pgfpathlineto{\pgfqpoint{2.455713in}{2.528066in}}%
\pgfpathlineto{\pgfqpoint{2.457441in}{2.555612in}}%
\pgfpathlineto{\pgfqpoint{2.458306in}{2.524993in}}%
\pgfpathlineto{\pgfqpoint{2.460035in}{2.576578in}}%
\pgfpathlineto{\pgfqpoint{2.461764in}{2.545652in}}%
\pgfpathlineto{\pgfqpoint{2.463494in}{2.564866in}}%
\pgfpathlineto{\pgfqpoint{2.464360in}{2.547927in}}%
\pgfpathlineto{\pgfqpoint{2.465225in}{2.568032in}}%
\pgfpathlineto{\pgfqpoint{2.466090in}{2.493850in}}%
\pgfpathlineto{\pgfqpoint{2.467821in}{2.581743in}}%
\pgfpathlineto{\pgfqpoint{2.469550in}{2.552969in}}%
\pgfpathlineto{\pgfqpoint{2.470414in}{2.611256in}}%
\pgfpathlineto{\pgfqpoint{2.473005in}{2.482507in}}%
\pgfpathlineto{\pgfqpoint{2.473871in}{2.538519in}}%
\pgfpathlineto{\pgfqpoint{2.474737in}{2.533662in}}%
\pgfpathlineto{\pgfqpoint{2.475602in}{2.531787in}}%
\pgfpathlineto{\pgfqpoint{2.476468in}{2.481707in}}%
\pgfpathlineto{\pgfqpoint{2.478200in}{2.558593in}}%
\pgfpathlineto{\pgfqpoint{2.479066in}{2.529480in}}%
\pgfpathlineto{\pgfqpoint{2.479931in}{2.575103in}}%
\pgfpathlineto{\pgfqpoint{2.480797in}{2.548910in}}%
\pgfpathlineto{\pgfqpoint{2.481662in}{2.600742in}}%
\pgfpathlineto{\pgfqpoint{2.482527in}{2.552322in}}%
\pgfpathlineto{\pgfqpoint{2.483391in}{2.593856in}}%
\pgfpathlineto{\pgfqpoint{2.485985in}{2.539196in}}%
\pgfpathlineto{\pgfqpoint{2.486849in}{2.617219in}}%
\pgfpathlineto{\pgfqpoint{2.487715in}{2.605599in}}%
\pgfpathlineto{\pgfqpoint{2.488581in}{2.581805in}}%
\pgfpathlineto{\pgfqpoint{2.489447in}{2.623184in}}%
\pgfpathlineto{\pgfqpoint{2.490310in}{2.549587in}}%
\pgfpathlineto{\pgfqpoint{2.491175in}{2.589859in}}%
\pgfpathlineto{\pgfqpoint{2.492905in}{2.512574in}}%
\pgfpathlineto{\pgfqpoint{2.493769in}{2.578669in}}%
\pgfpathlineto{\pgfqpoint{2.494635in}{2.528713in}}%
\pgfpathlineto{\pgfqpoint{2.495501in}{2.585740in}}%
\pgfpathlineto{\pgfqpoint{2.496365in}{2.576825in}}%
\pgfpathlineto{\pgfqpoint{2.497231in}{2.479064in}}%
\pgfpathlineto{\pgfqpoint{2.498962in}{2.592627in}}%
\pgfpathlineto{\pgfqpoint{2.499826in}{2.551063in}}%
\pgfpathlineto{\pgfqpoint{2.500692in}{2.565882in}}%
\pgfpathlineto{\pgfqpoint{2.501555in}{2.485612in}}%
\pgfpathlineto{\pgfqpoint{2.503285in}{2.583649in}}%
\pgfpathlineto{\pgfqpoint{2.504149in}{2.511589in}}%
\pgfpathlineto{\pgfqpoint{2.505880in}{2.553215in}}%
\pgfpathlineto{\pgfqpoint{2.507611in}{2.511651in}}%
\pgfpathlineto{\pgfqpoint{2.508476in}{2.546513in}}%
\pgfpathlineto{\pgfqpoint{2.509340in}{2.522411in}}%
\pgfpathlineto{\pgfqpoint{2.510205in}{2.584327in}}%
\pgfpathlineto{\pgfqpoint{2.511069in}{2.470949in}}%
\pgfpathlineto{\pgfqpoint{2.511934in}{2.471010in}}%
\pgfpathlineto{\pgfqpoint{2.512798in}{2.536061in}}%
\pgfpathlineto{\pgfqpoint{2.513664in}{2.464186in}}%
\pgfpathlineto{\pgfqpoint{2.514530in}{2.591398in}}%
\pgfpathlineto{\pgfqpoint{2.515396in}{2.546144in}}%
\pgfpathlineto{\pgfqpoint{2.516260in}{2.549587in}}%
\pgfpathlineto{\pgfqpoint{2.517125in}{2.527515in}}%
\pgfpathlineto{\pgfqpoint{2.517990in}{2.561330in}}%
\pgfpathlineto{\pgfqpoint{2.518855in}{2.459082in}}%
\pgfpathlineto{\pgfqpoint{2.520585in}{2.544638in}}%
\pgfpathlineto{\pgfqpoint{2.521451in}{2.482323in}}%
\pgfpathlineto{\pgfqpoint{2.522317in}{2.559917in}}%
\pgfpathlineto{\pgfqpoint{2.523180in}{2.501937in}}%
\pgfpathlineto{\pgfqpoint{2.524912in}{2.579652in}}%
\pgfpathlineto{\pgfqpoint{2.526643in}{2.510176in}}%
\pgfpathlineto{\pgfqpoint{2.527509in}{2.581989in}}%
\pgfpathlineto{\pgfqpoint{2.528374in}{2.506794in}}%
\pgfpathlineto{\pgfqpoint{2.529236in}{2.541226in}}%
\pgfpathlineto{\pgfqpoint{2.530966in}{2.484537in}}%
\pgfpathlineto{\pgfqpoint{2.531831in}{2.520505in}}%
\pgfpathlineto{\pgfqpoint{2.532696in}{2.486566in}}%
\pgfpathlineto{\pgfqpoint{2.534427in}{2.556597in}}%
\pgfpathlineto{\pgfqpoint{2.535290in}{2.495051in}}%
\pgfpathlineto{\pgfqpoint{2.536155in}{2.529421in}}%
\pgfpathlineto{\pgfqpoint{2.537021in}{2.459452in}}%
\pgfpathlineto{\pgfqpoint{2.538751in}{2.531879in}}%
\pgfpathlineto{\pgfqpoint{2.539618in}{2.507685in}}%
\pgfpathlineto{\pgfqpoint{2.540482in}{2.516754in}}%
\pgfpathlineto{\pgfqpoint{2.541346in}{2.458775in}}%
\pgfpathlineto{\pgfqpoint{2.542211in}{2.467475in}}%
\pgfpathlineto{\pgfqpoint{2.543942in}{2.544792in}}%
\pgfpathlineto{\pgfqpoint{2.544808in}{2.554783in}}%
\pgfpathlineto{\pgfqpoint{2.545672in}{2.475744in}}%
\pgfpathlineto{\pgfqpoint{2.546537in}{2.566741in}}%
\pgfpathlineto{\pgfqpoint{2.547403in}{2.501937in}}%
\pgfpathlineto{\pgfqpoint{2.548266in}{2.592442in}}%
\pgfpathlineto{\pgfqpoint{2.549131in}{2.521980in}}%
\pgfpathlineto{\pgfqpoint{2.549993in}{2.534647in}}%
\pgfpathlineto{\pgfqpoint{2.550858in}{2.541780in}}%
\pgfpathlineto{\pgfqpoint{2.551724in}{2.512359in}}%
\pgfpathlineto{\pgfqpoint{2.552590in}{2.516202in}}%
\pgfpathlineto{\pgfqpoint{2.554322in}{2.495112in}}%
\pgfpathlineto{\pgfqpoint{2.555188in}{2.501445in}}%
\pgfpathlineto{\pgfqpoint{2.556051in}{2.541287in}}%
\pgfpathlineto{\pgfqpoint{2.556916in}{2.539258in}}%
\pgfpathlineto{\pgfqpoint{2.557782in}{2.475867in}}%
\pgfpathlineto{\pgfqpoint{2.560377in}{2.577133in}}%
\pgfpathlineto{\pgfqpoint{2.562106in}{2.487980in}}%
\pgfpathlineto{\pgfqpoint{2.564700in}{2.577071in}}%
\pgfpathlineto{\pgfqpoint{2.566428in}{2.516508in}}%
\pgfpathlineto{\pgfqpoint{2.567292in}{2.612424in}}%
\pgfpathlineto{\pgfqpoint{2.568157in}{2.563021in}}%
\pgfpathlineto{\pgfqpoint{2.569022in}{2.585802in}}%
\pgfpathlineto{\pgfqpoint{2.569888in}{2.580514in}}%
\pgfpathlineto{\pgfqpoint{2.570753in}{2.568525in}}%
\pgfpathlineto{\pgfqpoint{2.571618in}{2.582359in}}%
\pgfpathlineto{\pgfqpoint{2.572482in}{2.557028in}}%
\pgfpathlineto{\pgfqpoint{2.573346in}{2.491823in}}%
\pgfpathlineto{\pgfqpoint{2.574211in}{2.592627in}}%
\pgfpathlineto{\pgfqpoint{2.575076in}{2.538950in}}%
\pgfpathlineto{\pgfqpoint{2.575939in}{2.543684in}}%
\pgfpathlineto{\pgfqpoint{2.576802in}{2.568217in}}%
\pgfpathlineto{\pgfqpoint{2.577665in}{2.555827in}}%
\pgfpathlineto{\pgfqpoint{2.578530in}{2.526960in}}%
\pgfpathlineto{\pgfqpoint{2.579394in}{2.547681in}}%
\pgfpathlineto{\pgfqpoint{2.580257in}{2.487857in}}%
\pgfpathlineto{\pgfqpoint{2.581123in}{2.552907in}}%
\pgfpathlineto{\pgfqpoint{2.582854in}{2.505011in}}%
\pgfpathlineto{\pgfqpoint{2.583719in}{2.515710in}}%
\pgfpathlineto{\pgfqpoint{2.584583in}{2.582113in}}%
\pgfpathlineto{\pgfqpoint{2.585448in}{2.473776in}}%
\pgfpathlineto{\pgfqpoint{2.586313in}{2.559486in}}%
\pgfpathlineto{\pgfqpoint{2.587177in}{2.523671in}}%
\pgfpathlineto{\pgfqpoint{2.588042in}{2.533908in}}%
\pgfpathlineto{\pgfqpoint{2.588907in}{2.533785in}}%
\pgfpathlineto{\pgfqpoint{2.590637in}{2.503043in}}%
\pgfpathlineto{\pgfqpoint{2.591500in}{2.496588in}}%
\pgfpathlineto{\pgfqpoint{2.592365in}{2.462279in}}%
\pgfpathlineto{\pgfqpoint{2.593230in}{2.537598in}}%
\pgfpathlineto{\pgfqpoint{2.594095in}{2.531941in}}%
\pgfpathlineto{\pgfqpoint{2.594961in}{2.506425in}}%
\pgfpathlineto{\pgfqpoint{2.595827in}{2.547619in}}%
\pgfpathlineto{\pgfqpoint{2.596692in}{2.528190in}}%
\pgfpathlineto{\pgfqpoint{2.598421in}{2.606584in}}%
\pgfpathlineto{\pgfqpoint{2.599286in}{2.506579in}}%
\pgfpathlineto{\pgfqpoint{2.601015in}{2.592688in}}%
\pgfpathlineto{\pgfqpoint{2.601880in}{2.512236in}}%
\pgfpathlineto{\pgfqpoint{2.602742in}{2.565205in}}%
\pgfpathlineto{\pgfqpoint{2.604469in}{2.455270in}}%
\pgfpathlineto{\pgfqpoint{2.605334in}{2.539627in}}%
\pgfpathlineto{\pgfqpoint{2.606197in}{2.474423in}}%
\pgfpathlineto{\pgfqpoint{2.607063in}{2.536615in}}%
\pgfpathlineto{\pgfqpoint{2.607928in}{2.532679in}}%
\pgfpathlineto{\pgfqpoint{2.608793in}{2.542947in}}%
\pgfpathlineto{\pgfqpoint{2.609657in}{2.541841in}}%
\pgfpathlineto{\pgfqpoint{2.610521in}{2.616852in}}%
\pgfpathlineto{\pgfqpoint{2.611387in}{2.495266in}}%
\pgfpathlineto{\pgfqpoint{2.612252in}{2.539135in}}%
\pgfpathlineto{\pgfqpoint{2.613117in}{2.470949in}}%
\pgfpathlineto{\pgfqpoint{2.613981in}{2.539135in}}%
\pgfpathlineto{\pgfqpoint{2.614846in}{2.469042in}}%
\pgfpathlineto{\pgfqpoint{2.618301in}{2.559794in}}%
\pgfpathlineto{\pgfqpoint{2.619166in}{2.492590in}}%
\pgfpathlineto{\pgfqpoint{2.620031in}{2.506609in}}%
\pgfpathlineto{\pgfqpoint{2.620894in}{2.511620in}}%
\pgfpathlineto{\pgfqpoint{2.622625in}{2.578300in}}%
\pgfpathlineto{\pgfqpoint{2.625219in}{2.455978in}}%
\pgfpathlineto{\pgfqpoint{2.626084in}{2.442113in}}%
\pgfpathlineto{\pgfqpoint{2.628682in}{2.556412in}}%
\pgfpathlineto{\pgfqpoint{2.629548in}{2.549895in}}%
\pgfpathlineto{\pgfqpoint{2.630413in}{2.499754in}}%
\pgfpathlineto{\pgfqpoint{2.631279in}{2.513927in}}%
\pgfpathlineto{\pgfqpoint{2.632145in}{2.549895in}}%
\pgfpathlineto{\pgfqpoint{2.634739in}{2.515094in}}%
\pgfpathlineto{\pgfqpoint{2.635603in}{2.561330in}}%
\pgfpathlineto{\pgfqpoint{2.636470in}{2.546575in}}%
\pgfpathlineto{\pgfqpoint{2.637336in}{2.554506in}}%
\pgfpathlineto{\pgfqpoint{2.638200in}{2.586662in}}%
\pgfpathlineto{\pgfqpoint{2.639929in}{2.501260in}}%
\pgfpathlineto{\pgfqpoint{2.641660in}{2.572889in}}%
\pgfpathlineto{\pgfqpoint{2.642524in}{2.565572in}}%
\pgfpathlineto{\pgfqpoint{2.645115in}{2.483613in}}%
\pgfpathlineto{\pgfqpoint{2.647710in}{2.574549in}}%
\pgfpathlineto{\pgfqpoint{2.648575in}{2.540795in}}%
\pgfpathlineto{\pgfqpoint{2.649441in}{2.546850in}}%
\pgfpathlineto{\pgfqpoint{2.652037in}{2.508146in}}%
\pgfpathlineto{\pgfqpoint{2.652902in}{2.538581in}}%
\pgfpathlineto{\pgfqpoint{2.653767in}{2.526714in}}%
\pgfpathlineto{\pgfqpoint{2.654633in}{2.465107in}}%
\pgfpathlineto{\pgfqpoint{2.655499in}{2.480970in}}%
\pgfpathlineto{\pgfqpoint{2.656365in}{2.552599in}}%
\pgfpathlineto{\pgfqpoint{2.657231in}{2.492590in}}%
\pgfpathlineto{\pgfqpoint{2.659827in}{2.588260in}}%
\pgfpathlineto{\pgfqpoint{2.660689in}{2.559609in}}%
\pgfpathlineto{\pgfqpoint{2.661555in}{2.497571in}}%
\pgfpathlineto{\pgfqpoint{2.662417in}{2.566126in}}%
\pgfpathlineto{\pgfqpoint{2.663283in}{2.500521in}}%
\pgfpathlineto{\pgfqpoint{2.665014in}{2.556104in}}%
\pgfpathlineto{\pgfqpoint{2.666743in}{2.458344in}}%
\pgfpathlineto{\pgfqpoint{2.667607in}{2.542085in}}%
\pgfpathlineto{\pgfqpoint{2.668473in}{2.504919in}}%
\pgfpathlineto{\pgfqpoint{2.669338in}{2.531571in}}%
\pgfpathlineto{\pgfqpoint{2.670204in}{2.487303in}}%
\pgfpathlineto{\pgfqpoint{2.671069in}{2.550816in}}%
\pgfpathlineto{\pgfqpoint{2.671934in}{2.540241in}}%
\pgfpathlineto{\pgfqpoint{2.672800in}{2.552815in}}%
\pgfpathlineto{\pgfqpoint{2.673666in}{2.515217in}}%
\pgfpathlineto{\pgfqpoint{2.674531in}{2.574734in}}%
\pgfpathlineto{\pgfqpoint{2.675397in}{2.506917in}}%
\pgfpathlineto{\pgfqpoint{2.676261in}{2.551370in}}%
\pgfpathlineto{\pgfqpoint{2.677991in}{2.535137in}}%
\pgfpathlineto{\pgfqpoint{2.678857in}{2.576948in}}%
\pgfpathlineto{\pgfqpoint{2.679722in}{2.563298in}}%
\pgfpathlineto{\pgfqpoint{2.680587in}{2.504518in}}%
\pgfpathlineto{\pgfqpoint{2.682316in}{2.591950in}}%
\pgfpathlineto{\pgfqpoint{2.683182in}{2.506486in}}%
\pgfpathlineto{\pgfqpoint{2.684047in}{2.578300in}}%
\pgfpathlineto{\pgfqpoint{2.684912in}{2.505288in}}%
\pgfpathlineto{\pgfqpoint{2.685777in}{2.522042in}}%
\pgfpathlineto{\pgfqpoint{2.686642in}{2.586969in}}%
\pgfpathlineto{\pgfqpoint{2.687507in}{2.554444in}}%
\pgfpathlineto{\pgfqpoint{2.688372in}{2.574672in}}%
\pgfpathlineto{\pgfqpoint{2.690102in}{2.513434in}}%
\pgfpathlineto{\pgfqpoint{2.690968in}{2.523456in}}%
\pgfpathlineto{\pgfqpoint{2.691834in}{2.574028in}}%
\pgfpathlineto{\pgfqpoint{2.692698in}{2.572828in}}%
\pgfpathlineto{\pgfqpoint{2.694429in}{2.584203in}}%
\pgfpathlineto{\pgfqpoint{2.697023in}{2.490900in}}%
\pgfpathlineto{\pgfqpoint{2.697889in}{2.501321in}}%
\pgfpathlineto{\pgfqpoint{2.698752in}{2.515402in}}%
\pgfpathlineto{\pgfqpoint{2.699618in}{2.497878in}}%
\pgfpathlineto{\pgfqpoint{2.700483in}{2.527391in}}%
\pgfpathlineto{\pgfqpoint{2.701346in}{2.517000in}}%
\pgfpathlineto{\pgfqpoint{2.702209in}{2.481278in}}%
\pgfpathlineto{\pgfqpoint{2.703074in}{2.487918in}}%
\pgfpathlineto{\pgfqpoint{2.703940in}{2.499908in}}%
\pgfpathlineto{\pgfqpoint{2.704806in}{2.493021in}}%
\pgfpathlineto{\pgfqpoint{2.705670in}{2.562436in}}%
\pgfpathlineto{\pgfqpoint{2.707401in}{2.526899in}}%
\pgfpathlineto{\pgfqpoint{2.708268in}{2.540672in}}%
\pgfpathlineto{\pgfqpoint{2.709134in}{2.524195in}}%
\pgfpathlineto{\pgfqpoint{2.709995in}{2.600742in}}%
\pgfpathlineto{\pgfqpoint{2.712592in}{2.506486in}}%
\pgfpathlineto{\pgfqpoint{2.713458in}{2.498032in}}%
\pgfpathlineto{\pgfqpoint{2.715189in}{2.577440in}}%
\pgfpathlineto{\pgfqpoint{2.716054in}{2.534493in}}%
\pgfpathlineto{\pgfqpoint{2.716920in}{2.581251in}}%
\pgfpathlineto{\pgfqpoint{2.718650in}{2.546144in}}%
\pgfpathlineto{\pgfqpoint{2.719515in}{2.559363in}}%
\pgfpathlineto{\pgfqpoint{2.720379in}{2.479433in}}%
\pgfpathlineto{\pgfqpoint{2.721245in}{2.563175in}}%
\pgfpathlineto{\pgfqpoint{2.722110in}{2.513249in}}%
\pgfpathlineto{\pgfqpoint{2.723839in}{2.568709in}}%
\pgfpathlineto{\pgfqpoint{2.724705in}{2.552230in}}%
\pgfpathlineto{\pgfqpoint{2.725571in}{2.488011in}}%
\pgfpathlineto{\pgfqpoint{2.726436in}{2.498371in}}%
\pgfpathlineto{\pgfqpoint{2.727301in}{2.558626in}}%
\pgfpathlineto{\pgfqpoint{2.729031in}{2.477527in}}%
\pgfpathlineto{\pgfqpoint{2.730761in}{2.530465in}}%
\pgfpathlineto{\pgfqpoint{2.731626in}{2.512880in}}%
\pgfpathlineto{\pgfqpoint{2.732492in}{2.453025in}}%
\pgfpathlineto{\pgfqpoint{2.734222in}{2.583834in}}%
\pgfpathlineto{\pgfqpoint{2.735950in}{2.530588in}}%
\pgfpathlineto{\pgfqpoint{2.736813in}{2.575411in}}%
\pgfpathlineto{\pgfqpoint{2.738539in}{2.485704in}}%
\pgfpathlineto{\pgfqpoint{2.739404in}{2.511282in}}%
\pgfpathlineto{\pgfqpoint{2.741132in}{2.598467in}}%
\pgfpathlineto{\pgfqpoint{2.741997in}{2.485458in}}%
\pgfpathlineto{\pgfqpoint{2.743725in}{2.547127in}}%
\pgfpathlineto{\pgfqpoint{2.744591in}{2.538919in}}%
\pgfpathlineto{\pgfqpoint{2.745455in}{2.519889in}}%
\pgfpathlineto{\pgfqpoint{2.746322in}{2.549218in}}%
\pgfpathlineto{\pgfqpoint{2.747187in}{2.537598in}}%
\pgfpathlineto{\pgfqpoint{2.748053in}{2.480909in}}%
\pgfpathlineto{\pgfqpoint{2.749784in}{2.559609in}}%
\pgfpathlineto{\pgfqpoint{2.750649in}{2.547127in}}%
\pgfpathlineto{\pgfqpoint{2.751515in}{2.511559in}}%
\pgfpathlineto{\pgfqpoint{2.752380in}{2.583588in}}%
\pgfpathlineto{\pgfqpoint{2.754111in}{2.494435in}}%
\pgfpathlineto{\pgfqpoint{2.754976in}{2.540672in}}%
\pgfpathlineto{\pgfqpoint{2.755840in}{2.492960in}}%
\pgfpathlineto{\pgfqpoint{2.756706in}{2.501198in}}%
\pgfpathlineto{\pgfqpoint{2.757572in}{2.503166in}}%
\pgfpathlineto{\pgfqpoint{2.758437in}{2.497755in}}%
\pgfpathlineto{\pgfqpoint{2.760166in}{2.509068in}}%
\pgfpathlineto{\pgfqpoint{2.761031in}{2.542024in}}%
\pgfpathlineto{\pgfqpoint{2.761898in}{2.472362in}}%
\pgfpathlineto{\pgfqpoint{2.762764in}{2.486012in}}%
\pgfpathlineto{\pgfqpoint{2.763629in}{2.485150in}}%
\pgfpathlineto{\pgfqpoint{2.764491in}{2.541962in}}%
\pgfpathlineto{\pgfqpoint{2.765356in}{2.479801in}}%
\pgfpathlineto{\pgfqpoint{2.766220in}{2.490192in}}%
\pgfpathlineto{\pgfqpoint{2.767951in}{2.574426in}}%
\pgfpathlineto{\pgfqpoint{2.768815in}{2.565757in}}%
\pgfpathlineto{\pgfqpoint{2.770546in}{2.473191in}}%
\pgfpathlineto{\pgfqpoint{2.772275in}{2.533047in}}%
\pgfpathlineto{\pgfqpoint{2.773139in}{2.507284in}}%
\pgfpathlineto{\pgfqpoint{2.774004in}{2.552846in}}%
\pgfpathlineto{\pgfqpoint{2.775728in}{2.449551in}}%
\pgfpathlineto{\pgfqpoint{2.776593in}{2.534155in}}%
\pgfpathlineto{\pgfqpoint{2.777458in}{2.500092in}}%
\pgfpathlineto{\pgfqpoint{2.778324in}{2.504765in}}%
\pgfpathlineto{\pgfqpoint{2.779188in}{2.552907in}}%
\pgfpathlineto{\pgfqpoint{2.780052in}{2.521980in}}%
\pgfpathlineto{\pgfqpoint{2.781780in}{2.548112in}}%
\pgfpathlineto{\pgfqpoint{2.782645in}{2.517339in}}%
\pgfpathlineto{\pgfqpoint{2.783507in}{2.558134in}}%
\pgfpathlineto{\pgfqpoint{2.785235in}{2.528928in}}%
\pgfpathlineto{\pgfqpoint{2.786965in}{2.595270in}}%
\pgfpathlineto{\pgfqpoint{2.787828in}{2.462095in}}%
\pgfpathlineto{\pgfqpoint{2.788692in}{2.587277in}}%
\pgfpathlineto{\pgfqpoint{2.789556in}{2.531081in}}%
\pgfpathlineto{\pgfqpoint{2.790421in}{2.565205in}}%
\pgfpathlineto{\pgfqpoint{2.792151in}{2.529975in}}%
\pgfpathlineto{\pgfqpoint{2.793017in}{2.505134in}}%
\pgfpathlineto{\pgfqpoint{2.793883in}{2.544238in}}%
\pgfpathlineto{\pgfqpoint{2.794748in}{2.524071in}}%
\pgfpathlineto{\pgfqpoint{2.795613in}{2.544792in}}%
\pgfpathlineto{\pgfqpoint{2.797343in}{2.498248in}}%
\pgfpathlineto{\pgfqpoint{2.798208in}{2.534401in}}%
\pgfpathlineto{\pgfqpoint{2.799072in}{2.534339in}}%
\pgfpathlineto{\pgfqpoint{2.799938in}{2.554506in}}%
\pgfpathlineto{\pgfqpoint{2.801665in}{2.504888in}}%
\pgfpathlineto{\pgfqpoint{2.802530in}{2.502368in}}%
\pgfpathlineto{\pgfqpoint{2.804259in}{2.548420in}}%
\pgfpathlineto{\pgfqpoint{2.805124in}{2.513496in}}%
\pgfpathlineto{\pgfqpoint{2.805989in}{2.583834in}}%
\pgfpathlineto{\pgfqpoint{2.806856in}{2.501445in}}%
\pgfpathlineto{\pgfqpoint{2.807721in}{2.554383in}}%
\pgfpathlineto{\pgfqpoint{2.808586in}{2.541133in}}%
\pgfpathlineto{\pgfqpoint{2.811181in}{2.499846in}}%
\pgfpathlineto{\pgfqpoint{2.812912in}{2.553584in}}%
\pgfpathlineto{\pgfqpoint{2.813775in}{2.541164in}}%
\pgfpathlineto{\pgfqpoint{2.814641in}{2.555244in}}%
\pgfpathlineto{\pgfqpoint{2.815503in}{2.465661in}}%
\pgfpathlineto{\pgfqpoint{2.816368in}{2.499231in}}%
\pgfpathlineto{\pgfqpoint{2.817232in}{2.490869in}}%
\pgfpathlineto{\pgfqpoint{2.818096in}{2.468858in}}%
\pgfpathlineto{\pgfqpoint{2.818961in}{2.540548in}}%
\pgfpathlineto{\pgfqpoint{2.819827in}{2.531633in}}%
\pgfpathlineto{\pgfqpoint{2.820692in}{2.481645in}}%
\pgfpathlineto{\pgfqpoint{2.822421in}{2.531510in}}%
\pgfpathlineto{\pgfqpoint{2.823283in}{2.528066in}}%
\pgfpathlineto{\pgfqpoint{2.824146in}{2.470764in}}%
\pgfpathlineto{\pgfqpoint{2.825873in}{2.556104in}}%
\pgfpathlineto{\pgfqpoint{2.826739in}{2.553705in}}%
\pgfpathlineto{\pgfqpoint{2.827603in}{2.494066in}}%
\pgfpathlineto{\pgfqpoint{2.828468in}{2.510974in}}%
\pgfpathlineto{\pgfqpoint{2.829334in}{2.498553in}}%
\pgfpathlineto{\pgfqpoint{2.831064in}{2.588999in}}%
\pgfpathlineto{\pgfqpoint{2.831927in}{2.531019in}}%
\pgfpathlineto{\pgfqpoint{2.832789in}{2.564835in}}%
\pgfpathlineto{\pgfqpoint{2.833654in}{2.533110in}}%
\pgfpathlineto{\pgfqpoint{2.835386in}{2.631731in}}%
\pgfpathlineto{\pgfqpoint{2.837117in}{2.525793in}}%
\pgfpathlineto{\pgfqpoint{2.837982in}{2.537967in}}%
\pgfpathlineto{\pgfqpoint{2.838847in}{2.593119in}}%
\pgfpathlineto{\pgfqpoint{2.839712in}{2.521367in}}%
\pgfpathlineto{\pgfqpoint{2.840577in}{2.576763in}}%
\pgfpathlineto{\pgfqpoint{2.841442in}{2.563483in}}%
\pgfpathlineto{\pgfqpoint{2.843172in}{2.593794in}}%
\pgfpathlineto{\pgfqpoint{2.844901in}{2.487180in}}%
\pgfpathlineto{\pgfqpoint{2.845764in}{2.548725in}}%
\pgfpathlineto{\pgfqpoint{2.846628in}{2.516385in}}%
\pgfpathlineto{\pgfqpoint{2.847492in}{2.557641in}}%
\pgfpathlineto{\pgfqpoint{2.848356in}{2.551001in}}%
\pgfpathlineto{\pgfqpoint{2.849221in}{2.528559in}}%
\pgfpathlineto{\pgfqpoint{2.850086in}{2.571722in}}%
\pgfpathlineto{\pgfqpoint{2.850950in}{2.534401in}}%
\pgfpathlineto{\pgfqpoint{2.851815in}{2.547804in}}%
\pgfpathlineto{\pgfqpoint{2.852680in}{2.599021in}}%
\pgfpathlineto{\pgfqpoint{2.854407in}{2.507962in}}%
\pgfpathlineto{\pgfqpoint{2.856137in}{2.528590in}}%
\pgfpathlineto{\pgfqpoint{2.857003in}{2.461910in}}%
\pgfpathlineto{\pgfqpoint{2.858734in}{2.535476in}}%
\pgfpathlineto{\pgfqpoint{2.859599in}{2.483000in}}%
\pgfpathlineto{\pgfqpoint{2.860465in}{2.565757in}}%
\pgfpathlineto{\pgfqpoint{2.861329in}{2.516693in}}%
\pgfpathlineto{\pgfqpoint{2.862193in}{2.612547in}}%
\pgfpathlineto{\pgfqpoint{2.863056in}{2.499785in}}%
\pgfpathlineto{\pgfqpoint{2.863921in}{2.585494in}}%
\pgfpathlineto{\pgfqpoint{2.864787in}{2.492529in}}%
\pgfpathlineto{\pgfqpoint{2.866517in}{2.573689in}}%
\pgfpathlineto{\pgfqpoint{2.869111in}{2.489270in}}%
\pgfpathlineto{\pgfqpoint{2.869976in}{2.582236in}}%
\pgfpathlineto{\pgfqpoint{2.872568in}{2.498063in}}%
\pgfpathlineto{\pgfqpoint{2.873433in}{2.549464in}}%
\pgfpathlineto{\pgfqpoint{2.874296in}{2.523579in}}%
\pgfpathlineto{\pgfqpoint{2.875161in}{2.543653in}}%
\pgfpathlineto{\pgfqpoint{2.876025in}{2.593548in}}%
\pgfpathlineto{\pgfqpoint{2.877753in}{2.487641in}}%
\pgfpathlineto{\pgfqpoint{2.878619in}{2.542455in}}%
\pgfpathlineto{\pgfqpoint{2.879484in}{2.426003in}}%
\pgfpathlineto{\pgfqpoint{2.881212in}{2.531817in}}%
\pgfpathlineto{\pgfqpoint{2.882076in}{2.567817in}}%
\pgfpathlineto{\pgfqpoint{2.883803in}{2.529175in}}%
\pgfpathlineto{\pgfqpoint{2.884668in}{2.557610in}}%
\pgfpathlineto{\pgfqpoint{2.885532in}{2.556412in}}%
\pgfpathlineto{\pgfqpoint{2.886397in}{2.483983in}}%
\pgfpathlineto{\pgfqpoint{2.887263in}{2.565603in}}%
\pgfpathlineto{\pgfqpoint{2.888127in}{2.503351in}}%
\pgfpathlineto{\pgfqpoint{2.888993in}{2.562131in}}%
\pgfpathlineto{\pgfqpoint{2.889857in}{2.507471in}}%
\pgfpathlineto{\pgfqpoint{2.890721in}{2.559673in}}%
\pgfpathlineto{\pgfqpoint{2.891587in}{2.516079in}}%
\pgfpathlineto{\pgfqpoint{2.892452in}{2.613778in}}%
\pgfpathlineto{\pgfqpoint{2.894183in}{2.510545in}}%
\pgfpathlineto{\pgfqpoint{2.895047in}{2.552907in}}%
\pgfpathlineto{\pgfqpoint{2.896779in}{2.506117in}}%
\pgfpathlineto{\pgfqpoint{2.897643in}{2.568771in}}%
\pgfpathlineto{\pgfqpoint{2.898505in}{2.470579in}}%
\pgfpathlineto{\pgfqpoint{2.899368in}{2.471993in}}%
\pgfpathlineto{\pgfqpoint{2.901099in}{2.521919in}}%
\pgfpathlineto{\pgfqpoint{2.901964in}{2.553829in}}%
\pgfpathlineto{\pgfqpoint{2.903693in}{2.490315in}}%
\pgfpathlineto{\pgfqpoint{2.904557in}{2.524500in}}%
\pgfpathlineto{\pgfqpoint{2.905420in}{2.493450in}}%
\pgfpathlineto{\pgfqpoint{2.906286in}{2.549156in}}%
\pgfpathlineto{\pgfqpoint{2.907152in}{2.501506in}}%
\pgfpathlineto{\pgfqpoint{2.908015in}{2.565633in}}%
\pgfpathlineto{\pgfqpoint{2.909746in}{2.526099in}}%
\pgfpathlineto{\pgfqpoint{2.910611in}{2.530157in}}%
\pgfpathlineto{\pgfqpoint{2.911476in}{2.540302in}}%
\pgfpathlineto{\pgfqpoint{2.912341in}{2.586969in}}%
\pgfpathlineto{\pgfqpoint{2.913205in}{2.499415in}}%
\pgfpathlineto{\pgfqpoint{2.914071in}{2.555673in}}%
\pgfpathlineto{\pgfqpoint{2.914936in}{2.535692in}}%
\pgfpathlineto{\pgfqpoint{2.915801in}{2.449151in}}%
\pgfpathlineto{\pgfqpoint{2.918398in}{2.622138in}}%
\pgfpathlineto{\pgfqpoint{2.920993in}{2.498984in}}%
\pgfpathlineto{\pgfqpoint{2.921858in}{2.516446in}}%
\pgfpathlineto{\pgfqpoint{2.922722in}{2.601971in}}%
\pgfpathlineto{\pgfqpoint{2.924453in}{2.495787in}}%
\pgfpathlineto{\pgfqpoint{2.925318in}{2.502797in}}%
\pgfpathlineto{\pgfqpoint{2.926183in}{2.553952in}}%
\pgfpathlineto{\pgfqpoint{2.927911in}{2.480293in}}%
\pgfpathlineto{\pgfqpoint{2.929642in}{2.564096in}}%
\pgfpathlineto{\pgfqpoint{2.930505in}{2.526960in}}%
\pgfpathlineto{\pgfqpoint{2.931370in}{2.585371in}}%
\pgfpathlineto{\pgfqpoint{2.932235in}{2.500183in}}%
\pgfpathlineto{\pgfqpoint{2.933966in}{2.585433in}}%
\pgfpathlineto{\pgfqpoint{2.935694in}{2.512942in}}%
\pgfpathlineto{\pgfqpoint{2.936558in}{2.511589in}}%
\pgfpathlineto{\pgfqpoint{2.938287in}{2.545467in}}%
\pgfpathlineto{\pgfqpoint{2.939153in}{2.497509in}}%
\pgfpathlineto{\pgfqpoint{2.940017in}{2.556043in}}%
\pgfpathlineto{\pgfqpoint{2.940882in}{2.518045in}}%
\pgfpathlineto{\pgfqpoint{2.941744in}{2.545652in}}%
\pgfpathlineto{\pgfqpoint{2.943475in}{2.447891in}}%
\pgfpathlineto{\pgfqpoint{2.944340in}{2.451273in}}%
\pgfpathlineto{\pgfqpoint{2.946068in}{2.566803in}}%
\pgfpathlineto{\pgfqpoint{2.946935in}{2.558811in}}%
\pgfpathlineto{\pgfqpoint{2.948660in}{2.514358in}}%
\pgfpathlineto{\pgfqpoint{2.949527in}{2.433936in}}%
\pgfpathlineto{\pgfqpoint{2.950389in}{2.572276in}}%
\pgfpathlineto{\pgfqpoint{2.951255in}{2.532987in}}%
\pgfpathlineto{\pgfqpoint{2.952121in}{2.497265in}}%
\pgfpathlineto{\pgfqpoint{2.952984in}{2.537967in}}%
\pgfpathlineto{\pgfqpoint{2.953848in}{2.475898in}}%
\pgfpathlineto{\pgfqpoint{2.955575in}{2.541410in}}%
\pgfpathlineto{\pgfqpoint{2.956440in}{2.515894in}}%
\pgfpathlineto{\pgfqpoint{2.957306in}{2.551863in}}%
\pgfpathlineto{\pgfqpoint{2.959036in}{2.507348in}}%
\pgfpathlineto{\pgfqpoint{2.961632in}{2.607138in}}%
\pgfpathlineto{\pgfqpoint{2.962497in}{2.644582in}}%
\pgfpathlineto{\pgfqpoint{2.963362in}{2.569479in}}%
\pgfpathlineto{\pgfqpoint{2.964228in}{2.596993in}}%
\pgfpathlineto{\pgfqpoint{2.965093in}{2.669605in}}%
\pgfpathlineto{\pgfqpoint{2.965957in}{2.573997in}}%
\pgfpathlineto{\pgfqpoint{2.966821in}{2.625521in}}%
\pgfpathlineto{\pgfqpoint{2.968551in}{2.572276in}}%
\pgfpathlineto{\pgfqpoint{2.970278in}{2.598253in}}%
\pgfpathlineto{\pgfqpoint{2.972009in}{2.682272in}}%
\pgfpathlineto{\pgfqpoint{2.973738in}{2.620849in}}%
\pgfpathlineto{\pgfqpoint{2.974603in}{2.660138in}}%
\pgfpathlineto{\pgfqpoint{2.975468in}{2.654911in}}%
\pgfpathlineto{\pgfqpoint{2.976333in}{2.591521in}}%
\pgfpathlineto{\pgfqpoint{2.978927in}{2.673418in}}%
\pgfpathlineto{\pgfqpoint{2.980655in}{2.613163in}}%
\pgfpathlineto{\pgfqpoint{2.983249in}{2.679198in}}%
\pgfpathlineto{\pgfqpoint{2.984114in}{2.620480in}}%
\pgfpathlineto{\pgfqpoint{2.984979in}{2.673233in}}%
\pgfpathlineto{\pgfqpoint{2.987576in}{2.559732in}}%
\pgfpathlineto{\pgfqpoint{2.990170in}{2.652513in}}%
\pgfpathlineto{\pgfqpoint{2.992763in}{2.610150in}}%
\pgfpathlineto{\pgfqpoint{2.993628in}{2.678521in}}%
\pgfpathlineto{\pgfqpoint{2.994494in}{2.626750in}}%
\pgfpathlineto{\pgfqpoint{2.995360in}{2.644641in}}%
\pgfpathlineto{\pgfqpoint{2.998820in}{2.509529in}}%
\pgfpathlineto{\pgfqpoint{2.999686in}{2.540610in}}%
\pgfpathlineto{\pgfqpoint{3.000551in}{2.535815in}}%
\pgfpathlineto{\pgfqpoint{3.001416in}{2.506363in}}%
\pgfpathlineto{\pgfqpoint{3.002282in}{2.566495in}}%
\pgfpathlineto{\pgfqpoint{3.004010in}{2.519951in}}%
\pgfpathlineto{\pgfqpoint{3.004875in}{2.507346in}}%
\pgfpathlineto{\pgfqpoint{3.005740in}{2.540087in}}%
\pgfpathlineto{\pgfqpoint{3.007470in}{2.497694in}}%
\pgfpathlineto{\pgfqpoint{3.008335in}{2.542393in}}%
\pgfpathlineto{\pgfqpoint{3.010930in}{2.479095in}}%
\pgfpathlineto{\pgfqpoint{3.012659in}{2.533908in}}%
\pgfpathlineto{\pgfqpoint{3.013525in}{2.537413in}}%
\pgfpathlineto{\pgfqpoint{3.014390in}{2.502551in}}%
\pgfpathlineto{\pgfqpoint{3.015255in}{2.542609in}}%
\pgfpathlineto{\pgfqpoint{3.017850in}{2.484383in}}%
\pgfpathlineto{\pgfqpoint{3.019578in}{2.583095in}}%
\pgfpathlineto{\pgfqpoint{3.020443in}{2.542301in}}%
\pgfpathlineto{\pgfqpoint{3.021309in}{2.558072in}}%
\pgfpathlineto{\pgfqpoint{3.023036in}{2.527145in}}%
\pgfpathlineto{\pgfqpoint{3.023900in}{2.528313in}}%
\pgfpathlineto{\pgfqpoint{3.024765in}{2.523302in}}%
\pgfpathlineto{\pgfqpoint{3.025630in}{2.476850in}}%
\pgfpathlineto{\pgfqpoint{3.027361in}{2.522625in}}%
\pgfpathlineto{\pgfqpoint{3.028227in}{2.596745in}}%
\pgfpathlineto{\pgfqpoint{3.029959in}{2.499292in}}%
\pgfpathlineto{\pgfqpoint{3.031687in}{2.573197in}}%
\pgfpathlineto{\pgfqpoint{3.032552in}{2.509622in}}%
\pgfpathlineto{\pgfqpoint{3.033415in}{2.518353in}}%
\pgfpathlineto{\pgfqpoint{3.034281in}{2.494127in}}%
\pgfpathlineto{\pgfqpoint{3.035146in}{2.603323in}}%
\pgfpathlineto{\pgfqpoint{3.036011in}{2.497078in}}%
\pgfpathlineto{\pgfqpoint{3.036876in}{2.579591in}}%
\pgfpathlineto{\pgfqpoint{3.037739in}{2.576332in}}%
\pgfpathlineto{\pgfqpoint{3.038605in}{2.569261in}}%
\pgfpathlineto{\pgfqpoint{3.039470in}{2.442726in}}%
\pgfpathlineto{\pgfqpoint{3.041199in}{2.514048in}}%
\pgfpathlineto{\pgfqpoint{3.042064in}{2.519212in}}%
\pgfpathlineto{\pgfqpoint{3.042930in}{2.500521in}}%
\pgfpathlineto{\pgfqpoint{3.043794in}{2.522779in}}%
\pgfpathlineto{\pgfqpoint{3.045523in}{2.485335in}}%
\pgfpathlineto{\pgfqpoint{3.047252in}{2.519335in}}%
\pgfpathlineto{\pgfqpoint{3.048117in}{2.516508in}}%
\pgfpathlineto{\pgfqpoint{3.048983in}{2.497417in}}%
\pgfpathlineto{\pgfqpoint{3.049848in}{2.507223in}}%
\pgfpathlineto{\pgfqpoint{3.050713in}{2.490007in}}%
\pgfpathlineto{\pgfqpoint{3.052443in}{2.551922in}}%
\pgfpathlineto{\pgfqpoint{3.053308in}{2.502243in}}%
\pgfpathlineto{\pgfqpoint{3.054173in}{2.512018in}}%
\pgfpathlineto{\pgfqpoint{3.055037in}{2.515708in}}%
\pgfpathlineto{\pgfqpoint{3.055901in}{2.502120in}}%
\pgfpathlineto{\pgfqpoint{3.057631in}{2.420959in}}%
\pgfpathlineto{\pgfqpoint{3.059359in}{2.589489in}}%
\pgfpathlineto{\pgfqpoint{3.060224in}{2.552415in}}%
\pgfpathlineto{\pgfqpoint{3.061088in}{2.559178in}}%
\pgfpathlineto{\pgfqpoint{3.061953in}{2.554937in}}%
\pgfpathlineto{\pgfqpoint{3.062818in}{2.537013in}}%
\pgfpathlineto{\pgfqpoint{3.063682in}{2.562252in}}%
\pgfpathlineto{\pgfqpoint{3.064545in}{2.542208in}}%
\pgfpathlineto{\pgfqpoint{3.065410in}{2.548849in}}%
\pgfpathlineto{\pgfqpoint{3.067139in}{2.505871in}}%
\pgfpathlineto{\pgfqpoint{3.068004in}{2.584109in}}%
\pgfpathlineto{\pgfqpoint{3.070600in}{2.488347in}}%
\pgfpathlineto{\pgfqpoint{3.073195in}{2.590166in}}%
\pgfpathlineto{\pgfqpoint{3.074926in}{2.521765in}}%
\pgfpathlineto{\pgfqpoint{3.075790in}{2.595516in}}%
\pgfpathlineto{\pgfqpoint{3.076655in}{2.523702in}}%
\pgfpathlineto{\pgfqpoint{3.077520in}{2.528867in}}%
\pgfpathlineto{\pgfqpoint{3.080116in}{2.482015in}}%
\pgfpathlineto{\pgfqpoint{3.080982in}{2.572520in}}%
\pgfpathlineto{\pgfqpoint{3.081847in}{2.514540in}}%
\pgfpathlineto{\pgfqpoint{3.082712in}{2.605476in}}%
\pgfpathlineto{\pgfqpoint{3.083578in}{2.595393in}}%
\pgfpathlineto{\pgfqpoint{3.084443in}{2.577009in}}%
\pgfpathlineto{\pgfqpoint{3.086170in}{2.511713in}}%
\pgfpathlineto{\pgfqpoint{3.087036in}{2.501321in}}%
\pgfpathlineto{\pgfqpoint{3.088767in}{2.554198in}}%
\pgfpathlineto{\pgfqpoint{3.092227in}{2.495172in}}%
\pgfpathlineto{\pgfqpoint{3.093092in}{2.560407in}}%
\pgfpathlineto{\pgfqpoint{3.093957in}{2.505409in}}%
\pgfpathlineto{\pgfqpoint{3.094822in}{2.512695in}}%
\pgfpathlineto{\pgfqpoint{3.097417in}{2.552230in}}%
\pgfpathlineto{\pgfqpoint{3.098282in}{2.523086in}}%
\pgfpathlineto{\pgfqpoint{3.099147in}{2.550078in}}%
\pgfpathlineto{\pgfqpoint{3.100012in}{2.510176in}}%
\pgfpathlineto{\pgfqpoint{3.100878in}{2.520228in}}%
\pgfpathlineto{\pgfqpoint{3.102605in}{2.490561in}}%
\pgfpathlineto{\pgfqpoint{3.103470in}{2.497170in}}%
\pgfpathlineto{\pgfqpoint{3.104334in}{2.522840in}}%
\pgfpathlineto{\pgfqpoint{3.105199in}{2.497201in}}%
\pgfpathlineto{\pgfqpoint{3.106930in}{2.563114in}}%
\pgfpathlineto{\pgfqpoint{3.107795in}{2.528867in}}%
\pgfpathlineto{\pgfqpoint{3.108661in}{2.535507in}}%
\pgfpathlineto{\pgfqpoint{3.111255in}{2.478264in}}%
\pgfpathlineto{\pgfqpoint{3.112119in}{2.530157in}}%
\pgfpathlineto{\pgfqpoint{3.113850in}{2.394399in}}%
\pgfpathlineto{\pgfqpoint{3.114714in}{2.590780in}}%
\pgfpathlineto{\pgfqpoint{3.115579in}{2.495510in}}%
\pgfpathlineto{\pgfqpoint{3.116443in}{2.585309in}}%
\pgfpathlineto{\pgfqpoint{3.117306in}{2.480109in}}%
\pgfpathlineto{\pgfqpoint{3.118172in}{2.531694in}}%
\pgfpathlineto{\pgfqpoint{3.119902in}{2.485150in}}%
\pgfpathlineto{\pgfqpoint{3.120768in}{2.543992in}}%
\pgfpathlineto{\pgfqpoint{3.121631in}{2.511589in}}%
\pgfpathlineto{\pgfqpoint{3.123362in}{2.560715in}}%
\pgfpathlineto{\pgfqpoint{3.125092in}{2.439745in}}%
\pgfpathlineto{\pgfqpoint{3.125956in}{2.455270in}}%
\pgfpathlineto{\pgfqpoint{3.127686in}{2.537598in}}%
\pgfpathlineto{\pgfqpoint{3.128550in}{2.489211in}}%
\pgfpathlineto{\pgfqpoint{3.130279in}{2.544484in}}%
\pgfpathlineto{\pgfqpoint{3.131143in}{2.416166in}}%
\pgfpathlineto{\pgfqpoint{3.133740in}{2.587216in}}%
\pgfpathlineto{\pgfqpoint{3.134605in}{2.534031in}}%
\pgfpathlineto{\pgfqpoint{3.135471in}{2.552846in}}%
\pgfpathlineto{\pgfqpoint{3.137201in}{2.494928in}}%
\pgfpathlineto{\pgfqpoint{3.138066in}{2.544607in}}%
\pgfpathlineto{\pgfqpoint{3.138930in}{2.513588in}}%
\pgfpathlineto{\pgfqpoint{3.139795in}{2.551863in}}%
\pgfpathlineto{\pgfqpoint{3.141527in}{2.489855in}}%
\pgfpathlineto{\pgfqpoint{3.142392in}{2.501814in}}%
\pgfpathlineto{\pgfqpoint{3.143259in}{2.496465in}}%
\pgfpathlineto{\pgfqpoint{3.144124in}{2.571475in}}%
\pgfpathlineto{\pgfqpoint{3.146723in}{2.463570in}}%
\pgfpathlineto{\pgfqpoint{3.147589in}{2.608673in}}%
\pgfpathlineto{\pgfqpoint{3.148455in}{2.530835in}}%
\pgfpathlineto{\pgfqpoint{3.149320in}{2.551432in}}%
\pgfpathlineto{\pgfqpoint{3.150186in}{2.554506in}}%
\pgfpathlineto{\pgfqpoint{3.151050in}{2.580237in}}%
\pgfpathlineto{\pgfqpoint{3.154510in}{2.494374in}}%
\pgfpathlineto{\pgfqpoint{3.155373in}{2.525547in}}%
\pgfpathlineto{\pgfqpoint{3.156236in}{2.492775in}}%
\pgfpathlineto{\pgfqpoint{3.157101in}{2.559609in}}%
\pgfpathlineto{\pgfqpoint{3.157967in}{2.532556in}}%
\pgfpathlineto{\pgfqpoint{3.158829in}{2.470395in}}%
\pgfpathlineto{\pgfqpoint{3.159695in}{2.510789in}}%
\pgfpathlineto{\pgfqpoint{3.161425in}{2.485766in}}%
\pgfpathlineto{\pgfqpoint{3.162292in}{2.535322in}}%
\pgfpathlineto{\pgfqpoint{3.163156in}{2.485550in}}%
\pgfpathlineto{\pgfqpoint{3.164886in}{2.631053in}}%
\pgfpathlineto{\pgfqpoint{3.165751in}{2.501260in}}%
\pgfpathlineto{\pgfqpoint{3.167480in}{2.564404in}}%
\pgfpathlineto{\pgfqpoint{3.169210in}{2.530650in}}%
\pgfpathlineto{\pgfqpoint{3.170075in}{2.526683in}}%
\pgfpathlineto{\pgfqpoint{3.170938in}{2.465291in}}%
\pgfpathlineto{\pgfqpoint{3.173530in}{2.541962in}}%
\pgfpathlineto{\pgfqpoint{3.174393in}{2.472178in}}%
\pgfpathlineto{\pgfqpoint{3.176124in}{2.519028in}}%
\pgfpathlineto{\pgfqpoint{3.176989in}{2.502150in}}%
\pgfpathlineto{\pgfqpoint{3.177853in}{2.502304in}}%
\pgfpathlineto{\pgfqpoint{3.178719in}{2.507715in}}%
\pgfpathlineto{\pgfqpoint{3.179583in}{2.523394in}}%
\pgfpathlineto{\pgfqpoint{3.180447in}{2.487241in}}%
\pgfpathlineto{\pgfqpoint{3.181311in}{2.523333in}}%
\pgfpathlineto{\pgfqpoint{3.182175in}{2.444725in}}%
\pgfpathlineto{\pgfqpoint{3.183039in}{2.562929in}}%
\pgfpathlineto{\pgfqpoint{3.183904in}{2.533724in}}%
\pgfpathlineto{\pgfqpoint{3.184769in}{2.520444in}}%
\pgfpathlineto{\pgfqpoint{3.185635in}{2.479618in}}%
\pgfpathlineto{\pgfqpoint{3.189095in}{2.562498in}}%
\pgfpathlineto{\pgfqpoint{3.190825in}{2.485273in}}%
\pgfpathlineto{\pgfqpoint{3.192553in}{2.541408in}}%
\pgfpathlineto{\pgfqpoint{3.193417in}{2.552415in}}%
\pgfpathlineto{\pgfqpoint{3.194278in}{2.499169in}}%
\pgfpathlineto{\pgfqpoint{3.195143in}{2.567786in}}%
\pgfpathlineto{\pgfqpoint{3.196006in}{2.547127in}}%
\pgfpathlineto{\pgfqpoint{3.197734in}{2.581866in}}%
\pgfpathlineto{\pgfqpoint{3.199463in}{2.542085in}}%
\pgfpathlineto{\pgfqpoint{3.200327in}{2.522042in}}%
\pgfpathlineto{\pgfqpoint{3.201193in}{2.437315in}}%
\pgfpathlineto{\pgfqpoint{3.202922in}{2.547065in}}%
\pgfpathlineto{\pgfqpoint{3.203786in}{2.495849in}}%
\pgfpathlineto{\pgfqpoint{3.204651in}{2.590043in}}%
\pgfpathlineto{\pgfqpoint{3.205515in}{2.554075in}}%
\pgfpathlineto{\pgfqpoint{3.207245in}{2.576763in}}%
\pgfpathlineto{\pgfqpoint{3.208109in}{2.535753in}}%
\pgfpathlineto{\pgfqpoint{3.208975in}{2.580022in}}%
\pgfpathlineto{\pgfqpoint{3.210707in}{2.519582in}}%
\pgfpathlineto{\pgfqpoint{3.212439in}{2.590720in}}%
\pgfpathlineto{\pgfqpoint{3.213304in}{2.612670in}}%
\pgfpathlineto{\pgfqpoint{3.215032in}{2.528097in}}%
\pgfpathlineto{\pgfqpoint{3.215896in}{2.497817in}}%
\pgfpathlineto{\pgfqpoint{3.216762in}{2.559547in}}%
\pgfpathlineto{\pgfqpoint{3.217627in}{2.556135in}}%
\pgfpathlineto{\pgfqpoint{3.218492in}{2.571044in}}%
\pgfpathlineto{\pgfqpoint{3.219357in}{2.554260in}}%
\pgfpathlineto{\pgfqpoint{3.220222in}{2.569908in}}%
\pgfpathlineto{\pgfqpoint{3.221951in}{2.530342in}}%
\pgfpathlineto{\pgfqpoint{3.222816in}{2.560284in}}%
\pgfpathlineto{\pgfqpoint{3.223681in}{2.540241in}}%
\pgfpathlineto{\pgfqpoint{3.224547in}{2.548695in}}%
\pgfpathlineto{\pgfqpoint{3.225412in}{2.577623in}}%
\pgfpathlineto{\pgfqpoint{3.226276in}{2.573505in}}%
\pgfpathlineto{\pgfqpoint{3.227141in}{2.453364in}}%
\pgfpathlineto{\pgfqpoint{3.228006in}{2.557333in}}%
\pgfpathlineto{\pgfqpoint{3.228872in}{2.540179in}}%
\pgfpathlineto{\pgfqpoint{3.229739in}{2.504488in}}%
\pgfpathlineto{\pgfqpoint{3.230605in}{2.556902in}}%
\pgfpathlineto{\pgfqpoint{3.231470in}{2.523148in}}%
\pgfpathlineto{\pgfqpoint{3.232335in}{2.590474in}}%
\pgfpathlineto{\pgfqpoint{3.233200in}{2.513619in}}%
\pgfpathlineto{\pgfqpoint{3.234064in}{2.532218in}}%
\pgfpathlineto{\pgfqpoint{3.234929in}{2.473961in}}%
\pgfpathlineto{\pgfqpoint{3.237525in}{2.566311in}}%
\pgfpathlineto{\pgfqpoint{3.240120in}{2.478387in}}%
\pgfpathlineto{\pgfqpoint{3.240984in}{2.523025in}}%
\pgfpathlineto{\pgfqpoint{3.241849in}{2.502427in}}%
\pgfpathlineto{\pgfqpoint{3.242713in}{2.527207in}}%
\pgfpathlineto{\pgfqpoint{3.243578in}{2.507746in}}%
\pgfpathlineto{\pgfqpoint{3.245308in}{2.533970in}}%
\pgfpathlineto{\pgfqpoint{3.246173in}{2.504580in}}%
\pgfpathlineto{\pgfqpoint{3.247039in}{2.543376in}}%
\pgfpathlineto{\pgfqpoint{3.247901in}{2.518414in}}%
\pgfpathlineto{\pgfqpoint{3.248767in}{2.542976in}}%
\pgfpathlineto{\pgfqpoint{3.249633in}{2.505994in}}%
\pgfpathlineto{\pgfqpoint{3.250497in}{2.569877in}}%
\pgfpathlineto{\pgfqpoint{3.251361in}{2.471562in}}%
\pgfpathlineto{\pgfqpoint{3.252226in}{2.528374in}}%
\pgfpathlineto{\pgfqpoint{3.253089in}{2.451396in}}%
\pgfpathlineto{\pgfqpoint{3.253954in}{2.548725in}}%
\pgfpathlineto{\pgfqpoint{3.254819in}{2.509006in}}%
\pgfpathlineto{\pgfqpoint{3.255685in}{2.541931in}}%
\pgfpathlineto{\pgfqpoint{3.256550in}{2.456253in}}%
\pgfpathlineto{\pgfqpoint{3.259145in}{2.531325in}}%
\pgfpathlineto{\pgfqpoint{3.260010in}{2.507223in}}%
\pgfpathlineto{\pgfqpoint{3.260876in}{2.539687in}}%
\pgfpathlineto{\pgfqpoint{3.261740in}{2.475498in}}%
\pgfpathlineto{\pgfqpoint{3.262605in}{2.541901in}}%
\pgfpathlineto{\pgfqpoint{3.263469in}{2.470025in}}%
\pgfpathlineto{\pgfqpoint{3.265199in}{2.579529in}}%
\pgfpathlineto{\pgfqpoint{3.266064in}{2.561023in}}%
\pgfpathlineto{\pgfqpoint{3.266929in}{2.545775in}}%
\pgfpathlineto{\pgfqpoint{3.267794in}{2.494466in}}%
\pgfpathlineto{\pgfqpoint{3.269522in}{2.541408in}}%
\pgfpathlineto{\pgfqpoint{3.270385in}{2.512757in}}%
\pgfpathlineto{\pgfqpoint{3.272980in}{2.562252in}}%
\pgfpathlineto{\pgfqpoint{3.273846in}{2.538211in}}%
\pgfpathlineto{\pgfqpoint{3.274709in}{2.541408in}}%
\pgfpathlineto{\pgfqpoint{3.275573in}{2.526160in}}%
\pgfpathlineto{\pgfqpoint{3.276438in}{2.475005in}}%
\pgfpathlineto{\pgfqpoint{3.278164in}{2.570983in}}%
\pgfpathlineto{\pgfqpoint{3.279893in}{2.501198in}}%
\pgfpathlineto{\pgfqpoint{3.280759in}{2.540487in}}%
\pgfpathlineto{\pgfqpoint{3.281623in}{2.493083in}}%
\pgfpathlineto{\pgfqpoint{3.282489in}{2.536736in}}%
\pgfpathlineto{\pgfqpoint{3.283353in}{2.525177in}}%
\pgfpathlineto{\pgfqpoint{3.284218in}{2.529880in}}%
\pgfpathlineto{\pgfqpoint{3.285084in}{2.500213in}}%
\pgfpathlineto{\pgfqpoint{3.285950in}{2.579652in}}%
\pgfpathlineto{\pgfqpoint{3.286815in}{2.466521in}}%
\pgfpathlineto{\pgfqpoint{3.287681in}{2.516015in}}%
\pgfpathlineto{\pgfqpoint{3.288546in}{2.500398in}}%
\pgfpathlineto{\pgfqpoint{3.289412in}{2.557025in}}%
\pgfpathlineto{\pgfqpoint{3.293738in}{2.466767in}}%
\pgfpathlineto{\pgfqpoint{3.294599in}{2.580083in}}%
\pgfpathlineto{\pgfqpoint{3.295464in}{2.571660in}}%
\pgfpathlineto{\pgfqpoint{3.296325in}{2.540518in}}%
\pgfpathlineto{\pgfqpoint{3.297192in}{2.543745in}}%
\pgfpathlineto{\pgfqpoint{3.298922in}{2.499538in}}%
\pgfpathlineto{\pgfqpoint{3.299786in}{2.526776in}}%
\pgfpathlineto{\pgfqpoint{3.300649in}{2.526160in}}%
\pgfpathlineto{\pgfqpoint{3.301516in}{2.505317in}}%
\pgfpathlineto{\pgfqpoint{3.302379in}{2.514109in}}%
\pgfpathlineto{\pgfqpoint{3.303244in}{2.551953in}}%
\pgfpathlineto{\pgfqpoint{3.304108in}{2.514602in}}%
\pgfpathlineto{\pgfqpoint{3.304975in}{2.524931in}}%
\pgfpathlineto{\pgfqpoint{3.306706in}{2.565818in}}%
\pgfpathlineto{\pgfqpoint{3.308436in}{2.489699in}}%
\pgfpathlineto{\pgfqpoint{3.310168in}{2.520595in}}%
\pgfpathlineto{\pgfqpoint{3.311034in}{2.537965in}}%
\pgfpathlineto{\pgfqpoint{3.311899in}{2.439712in}}%
\pgfpathlineto{\pgfqpoint{3.313628in}{2.585615in}}%
\pgfpathlineto{\pgfqpoint{3.314493in}{2.583586in}}%
\pgfpathlineto{\pgfqpoint{3.315359in}{2.490284in}}%
\pgfpathlineto{\pgfqpoint{3.317088in}{2.601417in}}%
\pgfpathlineto{\pgfqpoint{3.317953in}{2.515584in}}%
\pgfpathlineto{\pgfqpoint{3.318817in}{2.528374in}}%
\pgfpathlineto{\pgfqpoint{3.319683in}{2.534860in}}%
\pgfpathlineto{\pgfqpoint{3.320547in}{2.557025in}}%
\pgfpathlineto{\pgfqpoint{3.321413in}{2.513249in}}%
\pgfpathlineto{\pgfqpoint{3.323139in}{2.579529in}}%
\pgfpathlineto{\pgfqpoint{3.324004in}{2.502427in}}%
\pgfpathlineto{\pgfqpoint{3.325734in}{2.636095in}}%
\pgfpathlineto{\pgfqpoint{3.328325in}{2.502735in}}%
\pgfpathlineto{\pgfqpoint{3.329190in}{2.483983in}}%
\pgfpathlineto{\pgfqpoint{3.330054in}{2.570798in}}%
\pgfpathlineto{\pgfqpoint{3.330917in}{2.558378in}}%
\pgfpathlineto{\pgfqpoint{3.331782in}{2.548479in}}%
\pgfpathlineto{\pgfqpoint{3.334376in}{2.480601in}}%
\pgfpathlineto{\pgfqpoint{3.336106in}{2.545528in}}%
\pgfpathlineto{\pgfqpoint{3.336971in}{2.534830in}}%
\pgfpathlineto{\pgfqpoint{3.337833in}{2.515584in}}%
\pgfpathlineto{\pgfqpoint{3.338698in}{2.520872in}}%
\pgfpathlineto{\pgfqpoint{3.339562in}{2.516139in}}%
\pgfpathlineto{\pgfqpoint{3.340427in}{2.474328in}}%
\pgfpathlineto{\pgfqpoint{3.343019in}{2.573626in}}%
\pgfpathlineto{\pgfqpoint{3.343883in}{2.498399in}}%
\pgfpathlineto{\pgfqpoint{3.344746in}{2.557210in}}%
\pgfpathlineto{\pgfqpoint{3.346476in}{2.456745in}}%
\pgfpathlineto{\pgfqpoint{3.347340in}{2.581620in}}%
\pgfpathlineto{\pgfqpoint{3.349071in}{2.453056in}}%
\pgfpathlineto{\pgfqpoint{3.350801in}{2.545867in}}%
\pgfpathlineto{\pgfqpoint{3.351667in}{2.506117in}}%
\pgfpathlineto{\pgfqpoint{3.352533in}{2.537105in}}%
\pgfpathlineto{\pgfqpoint{3.353399in}{2.535168in}}%
\pgfpathlineto{\pgfqpoint{3.354265in}{2.552723in}}%
\pgfpathlineto{\pgfqpoint{3.355130in}{2.596314in}}%
\pgfpathlineto{\pgfqpoint{3.355995in}{2.491300in}}%
\pgfpathlineto{\pgfqpoint{3.356859in}{2.497694in}}%
\pgfpathlineto{\pgfqpoint{3.357724in}{2.505532in}}%
\pgfpathlineto{\pgfqpoint{3.358588in}{2.490561in}}%
\pgfpathlineto{\pgfqpoint{3.359454in}{2.518291in}}%
\pgfpathlineto{\pgfqpoint{3.360319in}{2.463570in}}%
\pgfpathlineto{\pgfqpoint{3.361184in}{2.544053in}}%
\pgfpathlineto{\pgfqpoint{3.362049in}{2.517983in}}%
\pgfpathlineto{\pgfqpoint{3.362913in}{2.521857in}}%
\pgfpathlineto{\pgfqpoint{3.364643in}{2.584817in}}%
\pgfpathlineto{\pgfqpoint{3.365508in}{2.536613in}}%
\pgfpathlineto{\pgfqpoint{3.366374in}{2.547558in}}%
\pgfpathlineto{\pgfqpoint{3.367238in}{2.564497in}}%
\pgfpathlineto{\pgfqpoint{3.368968in}{2.500213in}}%
\pgfpathlineto{\pgfqpoint{3.369832in}{2.524500in}}%
\pgfpathlineto{\pgfqpoint{3.370697in}{2.517799in}}%
\pgfpathlineto{\pgfqpoint{3.372429in}{2.488409in}}%
\pgfpathlineto{\pgfqpoint{3.373294in}{2.581374in}}%
\pgfpathlineto{\pgfqpoint{3.374159in}{2.533785in}}%
\pgfpathlineto{\pgfqpoint{3.375889in}{2.569015in}}%
\pgfpathlineto{\pgfqpoint{3.376753in}{2.480109in}}%
\pgfpathlineto{\pgfqpoint{3.377619in}{2.610456in}}%
\pgfpathlineto{\pgfqpoint{3.380214in}{2.463201in}}%
\pgfpathlineto{\pgfqpoint{3.381944in}{2.536090in}}%
\pgfpathlineto{\pgfqpoint{3.382810in}{2.553767in}}%
\pgfpathlineto{\pgfqpoint{3.383674in}{2.515584in}}%
\pgfpathlineto{\pgfqpoint{3.384537in}{2.521242in}}%
\pgfpathlineto{\pgfqpoint{3.386267in}{2.507100in}}%
\pgfpathlineto{\pgfqpoint{3.387132in}{2.510235in}}%
\pgfpathlineto{\pgfqpoint{3.388863in}{2.536982in}}%
\pgfpathlineto{\pgfqpoint{3.389728in}{2.531571in}}%
\pgfpathlineto{\pgfqpoint{3.390594in}{2.520503in}}%
\pgfpathlineto{\pgfqpoint{3.391460in}{2.491236in}}%
\pgfpathlineto{\pgfqpoint{3.392325in}{2.587891in}}%
\pgfpathlineto{\pgfqpoint{3.393191in}{2.583586in}}%
\pgfpathlineto{\pgfqpoint{3.394056in}{2.568153in}}%
\pgfpathlineto{\pgfqpoint{3.394921in}{2.518781in}}%
\pgfpathlineto{\pgfqpoint{3.395786in}{2.533200in}}%
\pgfpathlineto{\pgfqpoint{3.396649in}{2.476727in}}%
\pgfpathlineto{\pgfqpoint{3.397513in}{2.548356in}}%
\pgfpathlineto{\pgfqpoint{3.398379in}{2.513555in}}%
\pgfpathlineto{\pgfqpoint{3.400975in}{2.558347in}}%
\pgfpathlineto{\pgfqpoint{3.401841in}{2.538273in}}%
\pgfpathlineto{\pgfqpoint{3.402704in}{2.588075in}}%
\pgfpathlineto{\pgfqpoint{3.403569in}{2.537044in}}%
\pgfpathlineto{\pgfqpoint{3.404434in}{2.560592in}}%
\pgfpathlineto{\pgfqpoint{3.405299in}{2.548664in}}%
\pgfpathlineto{\pgfqpoint{3.407893in}{2.448751in}}%
\pgfpathlineto{\pgfqpoint{3.410487in}{2.493543in}}%
\pgfpathlineto{\pgfqpoint{3.411351in}{2.393722in}}%
\pgfpathlineto{\pgfqpoint{3.412214in}{2.559607in}}%
\pgfpathlineto{\pgfqpoint{3.413944in}{2.491482in}}%
\pgfpathlineto{\pgfqpoint{3.414810in}{2.492650in}}%
\pgfpathlineto{\pgfqpoint{3.415676in}{2.547864in}}%
\pgfpathlineto{\pgfqpoint{3.416540in}{2.489945in}}%
\pgfpathlineto{\pgfqpoint{3.419134in}{2.611193in}}%
\pgfpathlineto{\pgfqpoint{3.419999in}{2.501689in}}%
\pgfpathlineto{\pgfqpoint{3.420865in}{2.611254in}}%
\pgfpathlineto{\pgfqpoint{3.422597in}{2.526099in}}%
\pgfpathlineto{\pgfqpoint{3.423462in}{2.608303in}}%
\pgfpathlineto{\pgfqpoint{3.425191in}{2.518843in}}%
\pgfpathlineto{\pgfqpoint{3.426054in}{2.489638in}}%
\pgfpathlineto{\pgfqpoint{3.427782in}{2.531263in}}%
\pgfpathlineto{\pgfqpoint{3.428646in}{2.462523in}}%
\pgfpathlineto{\pgfqpoint{3.430376in}{2.595146in}}%
\pgfpathlineto{\pgfqpoint{3.431241in}{2.519397in}}%
\pgfpathlineto{\pgfqpoint{3.432107in}{2.584078in}}%
\pgfpathlineto{\pgfqpoint{3.432971in}{2.570613in}}%
\pgfpathlineto{\pgfqpoint{3.433836in}{2.603139in}}%
\pgfpathlineto{\pgfqpoint{3.438159in}{2.500090in}}%
\pgfpathlineto{\pgfqpoint{3.439025in}{2.553182in}}%
\pgfpathlineto{\pgfqpoint{3.439889in}{2.535874in}}%
\pgfpathlineto{\pgfqpoint{3.440753in}{2.537903in}}%
\pgfpathlineto{\pgfqpoint{3.441619in}{2.585800in}}%
\pgfpathlineto{\pgfqpoint{3.443347in}{2.491544in}}%
\pgfpathlineto{\pgfqpoint{3.444212in}{2.574180in}}%
\pgfpathlineto{\pgfqpoint{3.445941in}{2.506853in}}%
\pgfpathlineto{\pgfqpoint{3.448533in}{2.572273in}}%
\pgfpathlineto{\pgfqpoint{3.449396in}{2.513924in}}%
\pgfpathlineto{\pgfqpoint{3.451124in}{2.558070in}}%
\pgfpathlineto{\pgfqpoint{3.451988in}{2.494987in}}%
\pgfpathlineto{\pgfqpoint{3.452853in}{2.561421in}}%
\pgfpathlineto{\pgfqpoint{3.454582in}{2.496401in}}%
\pgfpathlineto{\pgfqpoint{3.455444in}{2.496586in}}%
\pgfpathlineto{\pgfqpoint{3.457175in}{2.523515in}}%
\pgfpathlineto{\pgfqpoint{3.458040in}{2.516660in}}%
\pgfpathlineto{\pgfqpoint{3.458905in}{2.534399in}}%
\pgfpathlineto{\pgfqpoint{3.459770in}{2.590472in}}%
\pgfpathlineto{\pgfqpoint{3.460636in}{2.516937in}}%
\pgfpathlineto{\pgfqpoint{3.461498in}{2.544605in}}%
\pgfpathlineto{\pgfqpoint{3.462364in}{2.511987in}}%
\pgfpathlineto{\pgfqpoint{3.463229in}{2.585307in}}%
\pgfpathlineto{\pgfqpoint{3.466686in}{2.457235in}}%
\pgfpathlineto{\pgfqpoint{3.467552in}{2.475003in}}%
\pgfpathlineto{\pgfqpoint{3.469282in}{2.549770in}}%
\pgfpathlineto{\pgfqpoint{3.470145in}{2.542637in}}%
\pgfpathlineto{\pgfqpoint{3.471011in}{2.522286in}}%
\pgfpathlineto{\pgfqpoint{3.471877in}{2.554904in}}%
\pgfpathlineto{\pgfqpoint{3.472743in}{2.525421in}}%
\pgfpathlineto{\pgfqpoint{3.473609in}{2.537349in}}%
\pgfpathlineto{\pgfqpoint{3.474471in}{2.496432in}}%
\pgfpathlineto{\pgfqpoint{3.475336in}{2.549524in}}%
\pgfpathlineto{\pgfqpoint{3.476201in}{2.528865in}}%
\pgfpathlineto{\pgfqpoint{3.477933in}{2.592071in}}%
\pgfpathlineto{\pgfqpoint{3.478798in}{2.528680in}}%
\pgfpathlineto{\pgfqpoint{3.479662in}{2.559545in}}%
\pgfpathlineto{\pgfqpoint{3.480525in}{2.556471in}}%
\pgfpathlineto{\pgfqpoint{3.481389in}{2.478662in}}%
\pgfpathlineto{\pgfqpoint{3.482252in}{2.588196in}}%
\pgfpathlineto{\pgfqpoint{3.483118in}{2.487362in}}%
\pgfpathlineto{\pgfqpoint{3.484847in}{2.631975in}}%
\pgfpathlineto{\pgfqpoint{3.485710in}{2.567047in}}%
\pgfpathlineto{\pgfqpoint{3.486575in}{2.567140in}}%
\pgfpathlineto{\pgfqpoint{3.487439in}{2.570860in}}%
\pgfpathlineto{\pgfqpoint{3.489170in}{2.524931in}}%
\pgfpathlineto{\pgfqpoint{3.490901in}{2.602954in}}%
\pgfpathlineto{\pgfqpoint{3.491766in}{2.591396in}}%
\pgfpathlineto{\pgfqpoint{3.496091in}{2.416349in}}%
\pgfpathlineto{\pgfqpoint{3.497819in}{2.532248in}}%
\pgfpathlineto{\pgfqpoint{3.498684in}{2.488716in}}%
\pgfpathlineto{\pgfqpoint{3.499548in}{2.621155in}}%
\pgfpathlineto{\pgfqpoint{3.502142in}{2.465045in}}%
\pgfpathlineto{\pgfqpoint{3.503870in}{2.573749in}}%
\pgfpathlineto{\pgfqpoint{3.504734in}{2.476358in}}%
\pgfpathlineto{\pgfqpoint{3.505598in}{2.549739in}}%
\pgfpathlineto{\pgfqpoint{3.506463in}{2.482199in}}%
\pgfpathlineto{\pgfqpoint{3.508192in}{2.526283in}}%
\pgfpathlineto{\pgfqpoint{3.509057in}{2.524993in}}%
\pgfpathlineto{\pgfqpoint{3.509923in}{2.514448in}}%
\pgfpathlineto{\pgfqpoint{3.510788in}{2.522840in}}%
\pgfpathlineto{\pgfqpoint{3.511653in}{2.505317in}}%
\pgfpathlineto{\pgfqpoint{3.512519in}{2.530342in}}%
\pgfpathlineto{\pgfqpoint{3.513384in}{2.530034in}}%
\pgfpathlineto{\pgfqpoint{3.514249in}{2.536859in}}%
\pgfpathlineto{\pgfqpoint{3.515113in}{2.458159in}}%
\pgfpathlineto{\pgfqpoint{3.517709in}{2.529973in}}%
\pgfpathlineto{\pgfqpoint{3.518574in}{2.504026in}}%
\pgfpathlineto{\pgfqpoint{3.519437in}{2.522071in}}%
\pgfpathlineto{\pgfqpoint{3.520301in}{2.477710in}}%
\pgfpathlineto{\pgfqpoint{3.521166in}{2.478202in}}%
\pgfpathlineto{\pgfqpoint{3.522032in}{2.479493in}}%
\pgfpathlineto{\pgfqpoint{3.522897in}{2.568399in}}%
\pgfpathlineto{\pgfqpoint{3.523762in}{2.463691in}}%
\pgfpathlineto{\pgfqpoint{3.525491in}{2.503595in}}%
\pgfpathlineto{\pgfqpoint{3.526355in}{2.515584in}}%
\pgfpathlineto{\pgfqpoint{3.527220in}{2.503287in}}%
\pgfpathlineto{\pgfqpoint{3.528086in}{2.580143in}}%
\pgfpathlineto{\pgfqpoint{3.528952in}{2.435040in}}%
\pgfpathlineto{\pgfqpoint{3.529817in}{2.505255in}}%
\pgfpathlineto{\pgfqpoint{3.530682in}{2.462831in}}%
\pgfpathlineto{\pgfqpoint{3.532412in}{2.531448in}}%
\pgfpathlineto{\pgfqpoint{3.533278in}{2.521550in}}%
\pgfpathlineto{\pgfqpoint{3.534143in}{2.545528in}}%
\pgfpathlineto{\pgfqpoint{3.535870in}{2.494127in}}%
\pgfpathlineto{\pgfqpoint{3.536735in}{2.502920in}}%
\pgfpathlineto{\pgfqpoint{3.537601in}{2.440882in}}%
\pgfpathlineto{\pgfqpoint{3.539331in}{2.535076in}}%
\pgfpathlineto{\pgfqpoint{3.541923in}{2.494805in}}%
\pgfpathlineto{\pgfqpoint{3.542787in}{2.554506in}}%
\pgfpathlineto{\pgfqpoint{3.543652in}{2.506640in}}%
\pgfpathlineto{\pgfqpoint{3.544515in}{2.527022in}}%
\pgfpathlineto{\pgfqpoint{3.545379in}{2.601663in}}%
\pgfpathlineto{\pgfqpoint{3.546245in}{2.499107in}}%
\pgfpathlineto{\pgfqpoint{3.547975in}{2.591611in}}%
\pgfpathlineto{\pgfqpoint{3.548839in}{2.587093in}}%
\pgfpathlineto{\pgfqpoint{3.551434in}{2.483798in}}%
\pgfpathlineto{\pgfqpoint{3.552299in}{2.552476in}}%
\pgfpathlineto{\pgfqpoint{3.554030in}{2.487364in}}%
\pgfpathlineto{\pgfqpoint{3.554896in}{2.571598in}}%
\pgfpathlineto{\pgfqpoint{3.555761in}{2.469227in}}%
\pgfpathlineto{\pgfqpoint{3.557492in}{2.529911in}}%
\pgfpathlineto{\pgfqpoint{3.559222in}{2.464922in}}%
\pgfpathlineto{\pgfqpoint{3.560087in}{2.506086in}}%
\pgfpathlineto{\pgfqpoint{3.560951in}{2.499169in}}%
\pgfpathlineto{\pgfqpoint{3.562682in}{2.545251in}}%
\pgfpathlineto{\pgfqpoint{3.565276in}{2.434549in}}%
\pgfpathlineto{\pgfqpoint{3.566138in}{2.435778in}}%
\pgfpathlineto{\pgfqpoint{3.567868in}{2.528559in}}%
\pgfpathlineto{\pgfqpoint{3.568733in}{2.504888in}}%
\pgfpathlineto{\pgfqpoint{3.570464in}{2.454408in}}%
\pgfpathlineto{\pgfqpoint{3.571329in}{2.516569in}}%
\pgfpathlineto{\pgfqpoint{3.572193in}{2.435748in}}%
\pgfpathlineto{\pgfqpoint{3.573923in}{2.538150in}}%
\pgfpathlineto{\pgfqpoint{3.574787in}{2.465722in}}%
\pgfpathlineto{\pgfqpoint{3.576516in}{2.542668in}}%
\pgfpathlineto{\pgfqpoint{3.577381in}{2.433749in}}%
\pgfpathlineto{\pgfqpoint{3.578247in}{2.513617in}}%
\pgfpathlineto{\pgfqpoint{3.579112in}{2.500183in}}%
\pgfpathlineto{\pgfqpoint{3.579978in}{2.531510in}}%
\pgfpathlineto{\pgfqpoint{3.580842in}{2.452625in}}%
\pgfpathlineto{\pgfqpoint{3.582573in}{2.496647in}}%
\pgfpathlineto{\pgfqpoint{3.584300in}{2.466090in}}%
\pgfpathlineto{\pgfqpoint{3.585164in}{2.513309in}}%
\pgfpathlineto{\pgfqpoint{3.586028in}{2.452961in}}%
\pgfpathlineto{\pgfqpoint{3.587756in}{2.530155in}}%
\pgfpathlineto{\pgfqpoint{3.588620in}{2.472053in}}%
\pgfpathlineto{\pgfqpoint{3.589485in}{2.546879in}}%
\pgfpathlineto{\pgfqpoint{3.590351in}{2.517981in}}%
\pgfpathlineto{\pgfqpoint{3.592078in}{2.569690in}}%
\pgfpathlineto{\pgfqpoint{3.592942in}{2.559361in}}%
\pgfpathlineto{\pgfqpoint{3.594668in}{2.502548in}}%
\pgfpathlineto{\pgfqpoint{3.595532in}{2.550599in}}%
\pgfpathlineto{\pgfqpoint{3.597262in}{2.457603in}}%
\pgfpathlineto{\pgfqpoint{3.598128in}{2.466672in}}%
\pgfpathlineto{\pgfqpoint{3.598992in}{2.426186in}}%
\pgfpathlineto{\pgfqpoint{3.599857in}{2.507959in}}%
\pgfpathlineto{\pgfqpoint{3.600722in}{2.486041in}}%
\pgfpathlineto{\pgfqpoint{3.601586in}{2.456558in}}%
\pgfpathlineto{\pgfqpoint{3.603316in}{2.559668in}}%
\pgfpathlineto{\pgfqpoint{3.604181in}{2.558501in}}%
\pgfpathlineto{\pgfqpoint{3.605045in}{2.460833in}}%
\pgfpathlineto{\pgfqpoint{3.605910in}{2.488162in}}%
\pgfpathlineto{\pgfqpoint{3.606775in}{2.490192in}}%
\pgfpathlineto{\pgfqpoint{3.607639in}{2.473438in}}%
\pgfpathlineto{\pgfqpoint{3.608504in}{2.509129in}}%
\pgfpathlineto{\pgfqpoint{3.609370in}{2.500398in}}%
\pgfpathlineto{\pgfqpoint{3.610234in}{2.503564in}}%
\pgfpathlineto{\pgfqpoint{3.611099in}{2.519151in}}%
\pgfpathlineto{\pgfqpoint{3.612827in}{2.505809in}}%
\pgfpathlineto{\pgfqpoint{3.614557in}{2.575624in}}%
\pgfpathlineto{\pgfqpoint{3.617155in}{2.486502in}}%
\pgfpathlineto{\pgfqpoint{3.618881in}{2.567478in}}%
\pgfpathlineto{\pgfqpoint{3.620609in}{2.502304in}}%
\pgfpathlineto{\pgfqpoint{3.621474in}{2.517368in}}%
\pgfpathlineto{\pgfqpoint{3.622340in}{2.482874in}}%
\pgfpathlineto{\pgfqpoint{3.623205in}{2.513678in}}%
\pgfpathlineto{\pgfqpoint{3.624932in}{2.472853in}}%
\pgfpathlineto{\pgfqpoint{3.626663in}{2.484319in}}%
\pgfpathlineto{\pgfqpoint{3.628394in}{2.513124in}}%
\pgfpathlineto{\pgfqpoint{3.630123in}{2.593854in}}%
\pgfpathlineto{\pgfqpoint{3.634446in}{2.500029in}}%
\pgfpathlineto{\pgfqpoint{3.636176in}{2.559976in}}%
\pgfpathlineto{\pgfqpoint{3.637043in}{2.546265in}}%
\pgfpathlineto{\pgfqpoint{3.637907in}{2.560161in}}%
\pgfpathlineto{\pgfqpoint{3.639640in}{2.472422in}}%
\pgfpathlineto{\pgfqpoint{3.641372in}{2.527081in}}%
\pgfpathlineto{\pgfqpoint{3.642235in}{2.475927in}}%
\pgfpathlineto{\pgfqpoint{3.645693in}{2.590657in}}%
\pgfpathlineto{\pgfqpoint{3.647418in}{2.530525in}}%
\pgfpathlineto{\pgfqpoint{3.649149in}{2.560530in}}%
\pgfpathlineto{\pgfqpoint{3.650016in}{2.552415in}}%
\pgfpathlineto{\pgfqpoint{3.650882in}{2.573995in}}%
\pgfpathlineto{\pgfqpoint{3.651747in}{2.538273in}}%
\pgfpathlineto{\pgfqpoint{3.652611in}{2.541501in}}%
\pgfpathlineto{\pgfqpoint{3.653476in}{2.570921in}}%
\pgfpathlineto{\pgfqpoint{3.654343in}{2.519459in}}%
\pgfpathlineto{\pgfqpoint{3.655208in}{2.579344in}}%
\pgfpathlineto{\pgfqpoint{3.656940in}{2.458282in}}%
\pgfpathlineto{\pgfqpoint{3.659533in}{2.585923in}}%
\pgfpathlineto{\pgfqpoint{3.660399in}{2.498615in}}%
\pgfpathlineto{\pgfqpoint{3.661264in}{2.528313in}}%
\pgfpathlineto{\pgfqpoint{3.662127in}{2.497355in}}%
\pgfpathlineto{\pgfqpoint{3.662989in}{2.569077in}}%
\pgfpathlineto{\pgfqpoint{3.663855in}{2.525300in}}%
\pgfpathlineto{\pgfqpoint{3.664721in}{2.531787in}}%
\pgfpathlineto{\pgfqpoint{3.665587in}{2.515831in}}%
\pgfpathlineto{\pgfqpoint{3.666450in}{2.534953in}}%
\pgfpathlineto{\pgfqpoint{3.667313in}{2.502551in}}%
\pgfpathlineto{\pgfqpoint{3.669910in}{2.588137in}}%
\pgfpathlineto{\pgfqpoint{3.671641in}{2.511251in}}%
\pgfpathlineto{\pgfqpoint{3.672504in}{2.519151in}}%
\pgfpathlineto{\pgfqpoint{3.673368in}{2.515708in}}%
\pgfpathlineto{\pgfqpoint{3.674235in}{2.500891in}}%
\pgfpathlineto{\pgfqpoint{3.675101in}{2.514232in}}%
\pgfpathlineto{\pgfqpoint{3.675965in}{2.485581in}}%
\pgfpathlineto{\pgfqpoint{3.677697in}{2.573441in}}%
\pgfpathlineto{\pgfqpoint{3.678563in}{2.543253in}}%
\pgfpathlineto{\pgfqpoint{3.679428in}{2.484842in}}%
\pgfpathlineto{\pgfqpoint{3.680294in}{2.560715in}}%
\pgfpathlineto{\pgfqpoint{3.681160in}{2.454747in}}%
\pgfpathlineto{\pgfqpoint{3.682026in}{2.483798in}}%
\pgfpathlineto{\pgfqpoint{3.682891in}{2.562129in}}%
\pgfpathlineto{\pgfqpoint{3.684621in}{2.471501in}}%
\pgfpathlineto{\pgfqpoint{3.685485in}{2.517799in}}%
\pgfpathlineto{\pgfqpoint{3.686350in}{2.512018in}}%
\pgfpathlineto{\pgfqpoint{3.687215in}{2.522840in}}%
\pgfpathlineto{\pgfqpoint{3.688079in}{2.507408in}}%
\pgfpathlineto{\pgfqpoint{3.688943in}{2.535936in}}%
\pgfpathlineto{\pgfqpoint{3.689808in}{2.467503in}}%
\pgfpathlineto{\pgfqpoint{3.691536in}{2.520749in}}%
\pgfpathlineto{\pgfqpoint{3.692401in}{2.477217in}}%
\pgfpathlineto{\pgfqpoint{3.693266in}{2.539779in}}%
\pgfpathlineto{\pgfqpoint{3.694131in}{2.457728in}}%
\pgfpathlineto{\pgfqpoint{3.694996in}{2.531263in}}%
\pgfpathlineto{\pgfqpoint{3.695860in}{2.524192in}}%
\pgfpathlineto{\pgfqpoint{3.696724in}{2.503533in}}%
\pgfpathlineto{\pgfqpoint{3.698452in}{2.584571in}}%
\pgfpathlineto{\pgfqpoint{3.700181in}{2.481922in}}%
\pgfpathlineto{\pgfqpoint{3.701046in}{2.502304in}}%
\pgfpathlineto{\pgfqpoint{3.701911in}{2.584263in}}%
\pgfpathlineto{\pgfqpoint{3.702776in}{2.548079in}}%
\pgfpathlineto{\pgfqpoint{3.703642in}{2.550078in}}%
\pgfpathlineto{\pgfqpoint{3.705373in}{2.524439in}}%
\pgfpathlineto{\pgfqpoint{3.706238in}{2.550139in}}%
\pgfpathlineto{\pgfqpoint{3.707103in}{2.543745in}}%
\pgfpathlineto{\pgfqpoint{3.709698in}{2.481522in}}%
\pgfpathlineto{\pgfqpoint{3.710563in}{2.479985in}}%
\pgfpathlineto{\pgfqpoint{3.711428in}{2.459142in}}%
\pgfpathlineto{\pgfqpoint{3.712293in}{2.566832in}}%
\pgfpathlineto{\pgfqpoint{3.713157in}{2.522840in}}%
\pgfpathlineto{\pgfqpoint{3.714022in}{2.539194in}}%
\pgfpathlineto{\pgfqpoint{3.715748in}{2.449182in}}%
\pgfpathlineto{\pgfqpoint{3.716611in}{2.502612in}}%
\pgfpathlineto{\pgfqpoint{3.717477in}{2.465045in}}%
\pgfpathlineto{\pgfqpoint{3.719209in}{2.556350in}}%
\pgfpathlineto{\pgfqpoint{3.720074in}{2.551186in}}%
\pgfpathlineto{\pgfqpoint{3.720938in}{2.526837in}}%
\pgfpathlineto{\pgfqpoint{3.722666in}{2.586785in}}%
\pgfpathlineto{\pgfqpoint{3.724396in}{2.529172in}}%
\pgfpathlineto{\pgfqpoint{3.726126in}{2.585433in}}%
\pgfpathlineto{\pgfqpoint{3.726990in}{2.456437in}}%
\pgfpathlineto{\pgfqpoint{3.727855in}{2.555981in}}%
\pgfpathlineto{\pgfqpoint{3.728721in}{2.555766in}}%
\pgfpathlineto{\pgfqpoint{3.729587in}{2.528374in}}%
\pgfpathlineto{\pgfqpoint{3.730452in}{2.563114in}}%
\pgfpathlineto{\pgfqpoint{3.732179in}{2.512818in}}%
\pgfpathlineto{\pgfqpoint{3.733044in}{2.522717in}}%
\pgfpathlineto{\pgfqpoint{3.733910in}{2.490592in}}%
\pgfpathlineto{\pgfqpoint{3.735640in}{2.582972in}}%
\pgfpathlineto{\pgfqpoint{3.736506in}{2.499415in}}%
\pgfpathlineto{\pgfqpoint{3.737372in}{2.516631in}}%
\pgfpathlineto{\pgfqpoint{3.738238in}{2.485366in}}%
\pgfpathlineto{\pgfqpoint{3.739103in}{2.505747in}}%
\pgfpathlineto{\pgfqpoint{3.739967in}{2.484781in}}%
\pgfpathlineto{\pgfqpoint{3.741695in}{2.527574in}}%
\pgfpathlineto{\pgfqpoint{3.742560in}{2.459265in}}%
\pgfpathlineto{\pgfqpoint{3.743425in}{2.507592in}}%
\pgfpathlineto{\pgfqpoint{3.744291in}{2.467873in}}%
\pgfpathlineto{\pgfqpoint{3.745156in}{2.556043in}}%
\pgfpathlineto{\pgfqpoint{3.746022in}{2.540856in}}%
\pgfpathlineto{\pgfqpoint{3.747753in}{2.487210in}}%
\pgfpathlineto{\pgfqpoint{3.749484in}{2.549770in}}%
\pgfpathlineto{\pgfqpoint{3.750349in}{2.526468in}}%
\pgfpathlineto{\pgfqpoint{3.751213in}{2.517245in}}%
\pgfpathlineto{\pgfqpoint{3.752078in}{2.527266in}}%
\pgfpathlineto{\pgfqpoint{3.753810in}{2.481276in}}%
\pgfpathlineto{\pgfqpoint{3.754676in}{2.563419in}}%
\pgfpathlineto{\pgfqpoint{3.755541in}{2.538027in}}%
\pgfpathlineto{\pgfqpoint{3.756406in}{2.464122in}}%
\pgfpathlineto{\pgfqpoint{3.758138in}{2.531325in}}%
\pgfpathlineto{\pgfqpoint{3.759004in}{2.531202in}}%
\pgfpathlineto{\pgfqpoint{3.759869in}{2.484935in}}%
\pgfpathlineto{\pgfqpoint{3.761597in}{2.560715in}}%
\pgfpathlineto{\pgfqpoint{3.762463in}{2.474574in}}%
\pgfpathlineto{\pgfqpoint{3.763326in}{2.478818in}}%
\pgfpathlineto{\pgfqpoint{3.765921in}{2.555119in}}%
\pgfpathlineto{\pgfqpoint{3.767650in}{2.489699in}}%
\pgfpathlineto{\pgfqpoint{3.768513in}{2.561267in}}%
\pgfpathlineto{\pgfqpoint{3.769378in}{2.482721in}}%
\pgfpathlineto{\pgfqpoint{3.771106in}{2.575039in}}%
\pgfpathlineto{\pgfqpoint{3.772835in}{2.542085in}}%
\pgfpathlineto{\pgfqpoint{3.774564in}{2.572397in}}%
\pgfpathlineto{\pgfqpoint{3.775427in}{2.521488in}}%
\pgfpathlineto{\pgfqpoint{3.776290in}{2.528128in}}%
\pgfpathlineto{\pgfqpoint{3.777155in}{2.623736in}}%
\pgfpathlineto{\pgfqpoint{3.778884in}{2.479924in}}%
\pgfpathlineto{\pgfqpoint{3.779750in}{2.562375in}}%
\pgfpathlineto{\pgfqpoint{3.780615in}{2.543253in}}%
\pgfpathlineto{\pgfqpoint{3.781479in}{2.454223in}}%
\pgfpathlineto{\pgfqpoint{3.783208in}{2.560961in}}%
\pgfpathlineto{\pgfqpoint{3.784074in}{2.488593in}}%
\pgfpathlineto{\pgfqpoint{3.784940in}{2.541223in}}%
\pgfpathlineto{\pgfqpoint{3.787530in}{2.500337in}}%
\pgfpathlineto{\pgfqpoint{3.789261in}{2.556841in}}%
\pgfpathlineto{\pgfqpoint{3.790125in}{2.567109in}}%
\pgfpathlineto{\pgfqpoint{3.791853in}{2.457420in}}%
\pgfpathlineto{\pgfqpoint{3.793584in}{2.527389in}}%
\pgfpathlineto{\pgfqpoint{3.795314in}{2.600434in}}%
\pgfpathlineto{\pgfqpoint{3.797911in}{2.458988in}}%
\pgfpathlineto{\pgfqpoint{3.798777in}{2.560530in}}%
\pgfpathlineto{\pgfqpoint{3.799640in}{2.529665in}}%
\pgfpathlineto{\pgfqpoint{3.800503in}{2.541193in}}%
\pgfpathlineto{\pgfqpoint{3.802234in}{2.507531in}}%
\pgfpathlineto{\pgfqpoint{3.803099in}{2.505809in}}%
\pgfpathlineto{\pgfqpoint{3.803964in}{2.542208in}}%
\pgfpathlineto{\pgfqpoint{3.804830in}{2.541624in}}%
\pgfpathlineto{\pgfqpoint{3.807428in}{2.482721in}}%
\pgfpathlineto{\pgfqpoint{3.809157in}{2.548479in}}%
\pgfpathlineto{\pgfqpoint{3.810020in}{2.481892in}}%
\pgfpathlineto{\pgfqpoint{3.810885in}{2.576578in}}%
\pgfpathlineto{\pgfqpoint{3.811751in}{2.495726in}}%
\pgfpathlineto{\pgfqpoint{3.812616in}{2.495849in}}%
\pgfpathlineto{\pgfqpoint{3.813481in}{2.501014in}}%
\pgfpathlineto{\pgfqpoint{3.814344in}{2.452902in}}%
\pgfpathlineto{\pgfqpoint{3.816076in}{2.521180in}}%
\pgfpathlineto{\pgfqpoint{3.817803in}{2.456684in}}%
\pgfpathlineto{\pgfqpoint{3.818667in}{2.514171in}}%
\pgfpathlineto{\pgfqpoint{3.819532in}{2.433687in}}%
\pgfpathlineto{\pgfqpoint{3.821264in}{2.605353in}}%
\pgfpathlineto{\pgfqpoint{3.822992in}{2.548171in}}%
\pgfpathlineto{\pgfqpoint{3.823858in}{2.575132in}}%
\pgfpathlineto{\pgfqpoint{3.824723in}{2.567109in}}%
\pgfpathlineto{\pgfqpoint{3.825589in}{2.580512in}}%
\pgfpathlineto{\pgfqpoint{3.826455in}{2.568615in}}%
\pgfpathlineto{\pgfqpoint{3.827320in}{2.594592in}}%
\pgfpathlineto{\pgfqpoint{3.828185in}{2.556348in}}%
\pgfpathlineto{\pgfqpoint{3.829052in}{2.567999in}}%
\pgfpathlineto{\pgfqpoint{3.830781in}{2.530771in}}%
\pgfpathlineto{\pgfqpoint{3.831644in}{2.582357in}}%
\pgfpathlineto{\pgfqpoint{3.832509in}{2.536120in}}%
\pgfpathlineto{\pgfqpoint{3.835105in}{2.650175in}}%
\pgfpathlineto{\pgfqpoint{3.835971in}{2.562190in}}%
\pgfpathlineto{\pgfqpoint{3.836836in}{2.630376in}}%
\pgfpathlineto{\pgfqpoint{3.837698in}{2.533170in}}%
\pgfpathlineto{\pgfqpoint{3.838562in}{2.567355in}}%
\pgfpathlineto{\pgfqpoint{3.841155in}{2.502120in}}%
\pgfpathlineto{\pgfqpoint{3.842018in}{2.586108in}}%
\pgfpathlineto{\pgfqpoint{3.842884in}{2.528097in}}%
\pgfpathlineto{\pgfqpoint{3.843749in}{2.548171in}}%
\pgfpathlineto{\pgfqpoint{3.844612in}{2.534583in}}%
\pgfpathlineto{\pgfqpoint{3.845478in}{2.541193in}}%
\pgfpathlineto{\pgfqpoint{3.846342in}{2.493142in}}%
\pgfpathlineto{\pgfqpoint{3.847206in}{2.560161in}}%
\pgfpathlineto{\pgfqpoint{3.848933in}{2.517491in}}%
\pgfpathlineto{\pgfqpoint{3.849796in}{2.551799in}}%
\pgfpathlineto{\pgfqpoint{3.850662in}{2.479862in}}%
\pgfpathlineto{\pgfqpoint{3.851528in}{2.584325in}}%
\pgfpathlineto{\pgfqpoint{3.852393in}{2.508729in}}%
\pgfpathlineto{\pgfqpoint{3.853258in}{2.633327in}}%
\pgfpathlineto{\pgfqpoint{3.854987in}{2.544697in}}%
\pgfpathlineto{\pgfqpoint{3.855851in}{2.525421in}}%
\pgfpathlineto{\pgfqpoint{3.856715in}{2.566801in}}%
\pgfpathlineto{\pgfqpoint{3.857580in}{2.489607in}}%
\pgfpathlineto{\pgfqpoint{3.858443in}{2.558439in}}%
\pgfpathlineto{\pgfqpoint{3.859307in}{2.485948in}}%
\pgfpathlineto{\pgfqpoint{3.860171in}{2.524069in}}%
\pgfpathlineto{\pgfqpoint{3.861036in}{2.512141in}}%
\pgfpathlineto{\pgfqpoint{3.861901in}{2.556471in}}%
\pgfpathlineto{\pgfqpoint{3.862767in}{2.548048in}}%
\pgfpathlineto{\pgfqpoint{3.863631in}{2.568215in}}%
\pgfpathlineto{\pgfqpoint{3.864495in}{2.549831in}}%
\pgfpathlineto{\pgfqpoint{3.865358in}{2.586169in}}%
\pgfpathlineto{\pgfqpoint{3.867087in}{2.483367in}}%
\pgfpathlineto{\pgfqpoint{3.867951in}{2.524192in}}%
\pgfpathlineto{\pgfqpoint{3.868815in}{2.517675in}}%
\pgfpathlineto{\pgfqpoint{3.869677in}{2.536367in}}%
\pgfpathlineto{\pgfqpoint{3.870542in}{2.506915in}}%
\pgfpathlineto{\pgfqpoint{3.871407in}{2.577377in}}%
\pgfpathlineto{\pgfqpoint{3.874002in}{2.505409in}}%
\pgfpathlineto{\pgfqpoint{3.874867in}{2.548972in}}%
\pgfpathlineto{\pgfqpoint{3.875730in}{2.506792in}}%
\pgfpathlineto{\pgfqpoint{3.879189in}{2.609779in}}%
\pgfpathlineto{\pgfqpoint{3.880052in}{2.520565in}}%
\pgfpathlineto{\pgfqpoint{3.880914in}{2.544851in}}%
\pgfpathlineto{\pgfqpoint{3.881778in}{2.500583in}}%
\pgfpathlineto{\pgfqpoint{3.882643in}{2.566924in}}%
\pgfpathlineto{\pgfqpoint{3.883507in}{2.508513in}}%
\pgfpathlineto{\pgfqpoint{3.884372in}{2.534645in}}%
\pgfpathlineto{\pgfqpoint{3.885238in}{2.481461in}}%
\pgfpathlineto{\pgfqpoint{3.886103in}{2.536459in}}%
\pgfpathlineto{\pgfqpoint{3.886970in}{2.495172in}}%
\pgfpathlineto{\pgfqpoint{3.887836in}{2.582049in}}%
\pgfpathlineto{\pgfqpoint{3.888702in}{2.479000in}}%
\pgfpathlineto{\pgfqpoint{3.889568in}{2.521917in}}%
\pgfpathlineto{\pgfqpoint{3.890434in}{2.508883in}}%
\pgfpathlineto{\pgfqpoint{3.892166in}{2.547312in}}%
\pgfpathlineto{\pgfqpoint{3.893031in}{2.531417in}}%
\pgfpathlineto{\pgfqpoint{3.893896in}{2.579344in}}%
\pgfpathlineto{\pgfqpoint{3.895627in}{2.527728in}}%
\pgfpathlineto{\pgfqpoint{3.896493in}{2.591272in}}%
\pgfpathlineto{\pgfqpoint{3.897360in}{2.536982in}}%
\pgfpathlineto{\pgfqpoint{3.898225in}{2.540118in}}%
\pgfpathlineto{\pgfqpoint{3.899091in}{2.515954in}}%
\pgfpathlineto{\pgfqpoint{3.900822in}{2.578914in}}%
\pgfpathlineto{\pgfqpoint{3.901689in}{2.484965in}}%
\pgfpathlineto{\pgfqpoint{3.903419in}{2.544913in}}%
\pgfpathlineto{\pgfqpoint{3.904283in}{2.532800in}}%
\pgfpathlineto{\pgfqpoint{3.905147in}{2.548356in}}%
\pgfpathlineto{\pgfqpoint{3.906011in}{2.478387in}}%
\pgfpathlineto{\pgfqpoint{3.906875in}{2.563727in}}%
\pgfpathlineto{\pgfqpoint{3.907739in}{2.484842in}}%
\pgfpathlineto{\pgfqpoint{3.908603in}{2.533722in}}%
\pgfpathlineto{\pgfqpoint{3.909466in}{2.506176in}}%
\pgfpathlineto{\pgfqpoint{3.910332in}{2.527020in}}%
\pgfpathlineto{\pgfqpoint{3.912926in}{2.487670in}}%
\pgfpathlineto{\pgfqpoint{3.913792in}{2.501381in}}%
\pgfpathlineto{\pgfqpoint{3.915522in}{2.566678in}}%
\pgfpathlineto{\pgfqpoint{3.916387in}{2.490990in}}%
\pgfpathlineto{\pgfqpoint{3.917253in}{2.510543in}}%
\pgfpathlineto{\pgfqpoint{3.918118in}{2.582357in}}%
\pgfpathlineto{\pgfqpoint{3.918984in}{2.575255in}}%
\pgfpathlineto{\pgfqpoint{3.919849in}{2.495233in}}%
\pgfpathlineto{\pgfqpoint{3.920712in}{2.568830in}}%
\pgfpathlineto{\pgfqpoint{3.921578in}{2.493481in}}%
\pgfpathlineto{\pgfqpoint{3.922443in}{2.530956in}}%
\pgfpathlineto{\pgfqpoint{3.923309in}{2.527820in}}%
\pgfpathlineto{\pgfqpoint{3.924173in}{2.474451in}}%
\pgfpathlineto{\pgfqpoint{3.925036in}{2.510543in}}%
\pgfpathlineto{\pgfqpoint{3.925900in}{2.482813in}}%
\pgfpathlineto{\pgfqpoint{3.927629in}{2.547494in}}%
\pgfpathlineto{\pgfqpoint{3.928493in}{2.501258in}}%
\pgfpathlineto{\pgfqpoint{3.930224in}{2.559853in}}%
\pgfpathlineto{\pgfqpoint{3.931088in}{2.517245in}}%
\pgfpathlineto{\pgfqpoint{3.931951in}{2.527512in}}%
\pgfpathlineto{\pgfqpoint{3.932817in}{2.506730in}}%
\pgfpathlineto{\pgfqpoint{3.933682in}{2.540762in}}%
\pgfpathlineto{\pgfqpoint{3.934547in}{2.520195in}}%
\pgfpathlineto{\pgfqpoint{3.937139in}{2.571596in}}%
\pgfpathlineto{\pgfqpoint{3.938004in}{2.506884in}}%
\pgfpathlineto{\pgfqpoint{3.939731in}{2.564648in}}%
\pgfpathlineto{\pgfqpoint{3.940595in}{2.511341in}}%
\pgfpathlineto{\pgfqpoint{3.941461in}{2.516814in}}%
\pgfpathlineto{\pgfqpoint{3.942327in}{2.583401in}}%
\pgfpathlineto{\pgfqpoint{3.943191in}{2.536857in}}%
\pgfpathlineto{\pgfqpoint{3.944057in}{2.543620in}}%
\pgfpathlineto{\pgfqpoint{3.945788in}{2.590164in}}%
\pgfpathlineto{\pgfqpoint{3.946653in}{2.496216in}}%
\pgfpathlineto{\pgfqpoint{3.947519in}{2.499413in}}%
\pgfpathlineto{\pgfqpoint{3.948385in}{2.460186in}}%
\pgfpathlineto{\pgfqpoint{3.949249in}{2.516260in}}%
\pgfpathlineto{\pgfqpoint{3.950114in}{2.507775in}}%
\pgfpathlineto{\pgfqpoint{3.950979in}{2.538394in}}%
\pgfpathlineto{\pgfqpoint{3.951843in}{2.530586in}}%
\pgfpathlineto{\pgfqpoint{3.952708in}{2.487362in}}%
\pgfpathlineto{\pgfqpoint{3.953573in}{2.585061in}}%
\pgfpathlineto{\pgfqpoint{3.954436in}{2.578298in}}%
\pgfpathlineto{\pgfqpoint{3.955301in}{2.615926in}}%
\pgfpathlineto{\pgfqpoint{3.957031in}{2.535197in}}%
\pgfpathlineto{\pgfqpoint{3.957894in}{2.555486in}}%
\pgfpathlineto{\pgfqpoint{3.959623in}{2.518289in}}%
\pgfpathlineto{\pgfqpoint{3.960486in}{2.542268in}}%
\pgfpathlineto{\pgfqpoint{3.961351in}{2.515030in}}%
\pgfpathlineto{\pgfqpoint{3.963081in}{2.586967in}}%
\pgfpathlineto{\pgfqpoint{3.963945in}{2.535382in}}%
\pgfpathlineto{\pgfqpoint{3.964811in}{2.588504in}}%
\pgfpathlineto{\pgfqpoint{3.965676in}{2.542514in}}%
\pgfpathlineto{\pgfqpoint{3.966542in}{2.566124in}}%
\pgfpathlineto{\pgfqpoint{3.967407in}{2.545711in}}%
\pgfpathlineto{\pgfqpoint{3.968272in}{2.552905in}}%
\pgfpathlineto{\pgfqpoint{3.969137in}{2.552351in}}%
\pgfpathlineto{\pgfqpoint{3.970002in}{2.580450in}}%
\pgfpathlineto{\pgfqpoint{3.971732in}{2.504147in}}%
\pgfpathlineto{\pgfqpoint{3.972596in}{2.507590in}}%
\pgfpathlineto{\pgfqpoint{3.973460in}{2.534520in}}%
\pgfpathlineto{\pgfqpoint{3.974325in}{2.624350in}}%
\pgfpathlineto{\pgfqpoint{3.976055in}{2.512447in}}%
\pgfpathlineto{\pgfqpoint{3.976921in}{2.574362in}}%
\pgfpathlineto{\pgfqpoint{3.978650in}{2.495631in}}%
\pgfpathlineto{\pgfqpoint{3.979516in}{2.542453in}}%
\pgfpathlineto{\pgfqpoint{3.981249in}{2.457880in}}%
\pgfpathlineto{\pgfqpoint{3.982981in}{2.577682in}}%
\pgfpathlineto{\pgfqpoint{3.983846in}{2.482505in}}%
\pgfpathlineto{\pgfqpoint{3.984712in}{2.513432in}}%
\pgfpathlineto{\pgfqpoint{3.985575in}{2.496247in}}%
\pgfpathlineto{\pgfqpoint{3.986438in}{2.441310in}}%
\pgfpathlineto{\pgfqpoint{3.987304in}{2.451825in}}%
\pgfpathlineto{\pgfqpoint{3.988169in}{2.450472in}}%
\pgfpathlineto{\pgfqpoint{3.989902in}{2.541285in}}%
\pgfpathlineto{\pgfqpoint{3.990765in}{2.445369in}}%
\pgfpathlineto{\pgfqpoint{3.991629in}{2.488039in}}%
\pgfpathlineto{\pgfqpoint{3.992494in}{2.458896in}}%
\pgfpathlineto{\pgfqpoint{3.993360in}{2.554319in}}%
\pgfpathlineto{\pgfqpoint{3.994223in}{2.530956in}}%
\pgfpathlineto{\pgfqpoint{3.995089in}{2.535566in}}%
\pgfpathlineto{\pgfqpoint{3.995954in}{2.587398in}}%
\pgfpathlineto{\pgfqpoint{3.996820in}{2.541593in}}%
\pgfpathlineto{\pgfqpoint{3.997684in}{2.549893in}}%
\pgfpathlineto{\pgfqpoint{3.998549in}{2.554811in}}%
\pgfpathlineto{\pgfqpoint{3.999414in}{2.619554in}}%
\pgfpathlineto{\pgfqpoint{4.000279in}{2.513432in}}%
\pgfpathlineto{\pgfqpoint{4.001143in}{2.524192in}}%
\pgfpathlineto{\pgfqpoint{4.002007in}{2.500337in}}%
\pgfpathlineto{\pgfqpoint{4.002874in}{2.568338in}}%
\pgfpathlineto{\pgfqpoint{4.004603in}{2.513647in}}%
\pgfpathlineto{\pgfqpoint{4.005467in}{2.522779in}}%
\pgfpathlineto{\pgfqpoint{4.006327in}{2.564343in}}%
\pgfpathlineto{\pgfqpoint{4.008057in}{2.470700in}}%
\pgfpathlineto{\pgfqpoint{4.008923in}{2.597420in}}%
\pgfpathlineto{\pgfqpoint{4.009788in}{2.578544in}}%
\pgfpathlineto{\pgfqpoint{4.012381in}{2.492160in}}%
\pgfpathlineto{\pgfqpoint{4.013246in}{2.509006in}}%
\pgfpathlineto{\pgfqpoint{4.014974in}{2.478387in}}%
\pgfpathlineto{\pgfqpoint{4.015839in}{2.570675in}}%
\pgfpathlineto{\pgfqpoint{4.016704in}{2.531140in}}%
\pgfpathlineto{\pgfqpoint{4.018434in}{2.556533in}}%
\pgfpathlineto{\pgfqpoint{4.019301in}{2.481584in}}%
\pgfpathlineto{\pgfqpoint{4.020166in}{2.485150in}}%
\pgfpathlineto{\pgfqpoint{4.021898in}{2.532800in}}%
\pgfpathlineto{\pgfqpoint{4.022764in}{2.508267in}}%
\pgfpathlineto{\pgfqpoint{4.023629in}{2.519028in}}%
\pgfpathlineto{\pgfqpoint{4.025359in}{2.558316in}}%
\pgfpathlineto{\pgfqpoint{4.026224in}{2.555058in}}%
\pgfpathlineto{\pgfqpoint{4.027087in}{2.489945in}}%
\pgfpathlineto{\pgfqpoint{4.028818in}{2.547617in}}%
\pgfpathlineto{\pgfqpoint{4.029684in}{2.484288in}}%
\pgfpathlineto{\pgfqpoint{4.030549in}{2.570365in}}%
\pgfpathlineto{\pgfqpoint{4.031412in}{2.566062in}}%
\pgfpathlineto{\pgfqpoint{4.032277in}{2.561698in}}%
\pgfpathlineto{\pgfqpoint{4.033143in}{2.522040in}}%
\pgfpathlineto{\pgfqpoint{4.034007in}{2.535197in}}%
\pgfpathlineto{\pgfqpoint{4.034872in}{2.580143in}}%
\pgfpathlineto{\pgfqpoint{4.035739in}{2.484165in}}%
\pgfpathlineto{\pgfqpoint{4.036604in}{2.546327in}}%
\pgfpathlineto{\pgfqpoint{4.037469in}{2.491298in}}%
\pgfpathlineto{\pgfqpoint{4.039199in}{2.564710in}}%
\pgfpathlineto{\pgfqpoint{4.040065in}{2.556471in}}%
\pgfpathlineto{\pgfqpoint{4.040931in}{2.503657in}}%
\pgfpathlineto{\pgfqpoint{4.041797in}{2.569998in}}%
\pgfpathlineto{\pgfqpoint{4.044392in}{2.497507in}}%
\pgfpathlineto{\pgfqpoint{4.045257in}{2.513771in}}%
\pgfpathlineto{\pgfqpoint{4.046121in}{2.563789in}}%
\pgfpathlineto{\pgfqpoint{4.046986in}{2.562190in}}%
\pgfpathlineto{\pgfqpoint{4.047850in}{2.508760in}}%
\pgfpathlineto{\pgfqpoint{4.049579in}{2.558562in}}%
\pgfpathlineto{\pgfqpoint{4.051310in}{2.501689in}}%
\pgfpathlineto{\pgfqpoint{4.053040in}{2.575778in}}%
\pgfpathlineto{\pgfqpoint{4.053906in}{2.459080in}}%
\pgfpathlineto{\pgfqpoint{4.054771in}{2.545344in}}%
\pgfpathlineto{\pgfqpoint{4.055636in}{2.535507in}}%
\pgfpathlineto{\pgfqpoint{4.056502in}{2.497140in}}%
\pgfpathlineto{\pgfqpoint{4.057366in}{2.606397in}}%
\pgfpathlineto{\pgfqpoint{4.059092in}{2.529234in}}%
\pgfpathlineto{\pgfqpoint{4.059957in}{2.535751in}}%
\pgfpathlineto{\pgfqpoint{4.060822in}{2.500583in}}%
\pgfpathlineto{\pgfqpoint{4.062552in}{2.552353in}}%
\pgfpathlineto{\pgfqpoint{4.064282in}{2.519766in}}%
\pgfpathlineto{\pgfqpoint{4.065147in}{2.533908in}}%
\pgfpathlineto{\pgfqpoint{4.066012in}{2.528128in}}%
\pgfpathlineto{\pgfqpoint{4.066875in}{2.551953in}}%
\pgfpathlineto{\pgfqpoint{4.067741in}{2.476296in}}%
\pgfpathlineto{\pgfqpoint{4.068604in}{2.512818in}}%
\pgfpathlineto{\pgfqpoint{4.069469in}{2.491975in}}%
\pgfpathlineto{\pgfqpoint{4.071198in}{2.570552in}}%
\pgfpathlineto{\pgfqpoint{4.072928in}{2.485150in}}%
\pgfpathlineto{\pgfqpoint{4.075522in}{2.542945in}}%
\pgfpathlineto{\pgfqpoint{4.076388in}{2.527574in}}%
\pgfpathlineto{\pgfqpoint{4.077253in}{2.533170in}}%
\pgfpathlineto{\pgfqpoint{4.078118in}{2.573810in}}%
\pgfpathlineto{\pgfqpoint{4.078984in}{2.566986in}}%
\pgfpathlineto{\pgfqpoint{4.079849in}{2.536243in}}%
\pgfpathlineto{\pgfqpoint{4.080713in}{2.564127in}}%
\pgfpathlineto{\pgfqpoint{4.081580in}{2.527943in}}%
\pgfpathlineto{\pgfqpoint{4.084175in}{2.581251in}}%
\pgfpathlineto{\pgfqpoint{4.085039in}{2.517429in}}%
\pgfpathlineto{\pgfqpoint{4.087632in}{2.601356in}}%
\pgfpathlineto{\pgfqpoint{4.088498in}{2.505686in}}%
\pgfpathlineto{\pgfqpoint{4.089364in}{2.523271in}}%
\pgfpathlineto{\pgfqpoint{4.090229in}{2.551645in}}%
\pgfpathlineto{\pgfqpoint{4.091095in}{2.620231in}}%
\pgfpathlineto{\pgfqpoint{4.091961in}{2.607136in}}%
\pgfpathlineto{\pgfqpoint{4.092826in}{2.559332in}}%
\pgfpathlineto{\pgfqpoint{4.093691in}{2.653126in}}%
\pgfpathlineto{\pgfqpoint{4.095420in}{2.539042in}}%
\pgfpathlineto{\pgfqpoint{4.096286in}{2.586723in}}%
\pgfpathlineto{\pgfqpoint{4.098016in}{2.545344in}}%
\pgfpathlineto{\pgfqpoint{4.098881in}{2.606890in}}%
\pgfpathlineto{\pgfqpoint{4.099746in}{2.547096in}}%
\pgfpathlineto{\pgfqpoint{4.100611in}{2.559301in}}%
\pgfpathlineto{\pgfqpoint{4.101476in}{2.548541in}}%
\pgfpathlineto{\pgfqpoint{4.102341in}{2.585954in}}%
\pgfpathlineto{\pgfqpoint{4.103206in}{2.524069in}}%
\pgfpathlineto{\pgfqpoint{4.104936in}{2.579129in}}%
\pgfpathlineto{\pgfqpoint{4.105802in}{2.563604in}}%
\pgfpathlineto{\pgfqpoint{4.107530in}{2.410814in}}%
\pgfpathlineto{\pgfqpoint{4.109259in}{2.568092in}}%
\pgfpathlineto{\pgfqpoint{4.111854in}{2.436515in}}%
\pgfpathlineto{\pgfqpoint{4.113584in}{2.551553in}}%
\pgfpathlineto{\pgfqpoint{4.115313in}{2.486318in}}%
\pgfpathlineto{\pgfqpoint{4.116177in}{2.488162in}}%
\pgfpathlineto{\pgfqpoint{4.117040in}{2.511772in}}%
\pgfpathlineto{\pgfqpoint{4.118770in}{2.444877in}}%
\pgfpathlineto{\pgfqpoint{4.120501in}{2.574978in}}%
\pgfpathlineto{\pgfqpoint{4.121367in}{2.511187in}}%
\pgfpathlineto{\pgfqpoint{4.122231in}{2.564587in}}%
\pgfpathlineto{\pgfqpoint{4.124826in}{2.494433in}}%
\pgfpathlineto{\pgfqpoint{4.125692in}{2.556164in}}%
\pgfpathlineto{\pgfqpoint{4.126557in}{2.516383in}}%
\pgfpathlineto{\pgfqpoint{4.128288in}{2.550907in}}%
\pgfpathlineto{\pgfqpoint{4.129154in}{2.530402in}}%
\pgfpathlineto{\pgfqpoint{4.130019in}{2.539440in}}%
\pgfpathlineto{\pgfqpoint{4.130884in}{2.515215in}}%
\pgfpathlineto{\pgfqpoint{4.131746in}{2.541347in}}%
\pgfpathlineto{\pgfqpoint{4.132610in}{2.455021in}}%
\pgfpathlineto{\pgfqpoint{4.134339in}{2.546573in}}%
\pgfpathlineto{\pgfqpoint{4.135203in}{2.509314in}}%
\pgfpathlineto{\pgfqpoint{4.136068in}{2.515831in}}%
\pgfpathlineto{\pgfqpoint{4.136933in}{2.591088in}}%
\pgfpathlineto{\pgfqpoint{4.137798in}{2.539071in}}%
\pgfpathlineto{\pgfqpoint{4.138664in}{2.618140in}}%
\pgfpathlineto{\pgfqpoint{4.139529in}{2.525791in}}%
\pgfpathlineto{\pgfqpoint{4.140395in}{2.562436in}}%
\pgfpathlineto{\pgfqpoint{4.141259in}{2.550385in}}%
\pgfpathlineto{\pgfqpoint{4.142124in}{2.603693in}}%
\pgfpathlineto{\pgfqpoint{4.142987in}{2.461417in}}%
\pgfpathlineto{\pgfqpoint{4.143853in}{2.552415in}}%
\pgfpathlineto{\pgfqpoint{4.144718in}{2.535137in}}%
\pgfpathlineto{\pgfqpoint{4.145583in}{2.555796in}}%
\pgfpathlineto{\pgfqpoint{4.146448in}{2.543376in}}%
\pgfpathlineto{\pgfqpoint{4.148177in}{2.593179in}}%
\pgfpathlineto{\pgfqpoint{4.149043in}{2.490438in}}%
\pgfpathlineto{\pgfqpoint{4.149908in}{2.507377in}}%
\pgfpathlineto{\pgfqpoint{4.150774in}{2.477833in}}%
\pgfpathlineto{\pgfqpoint{4.152507in}{2.532616in}}%
\pgfpathlineto{\pgfqpoint{4.153374in}{2.458649in}}%
\pgfpathlineto{\pgfqpoint{4.154234in}{2.474267in}}%
\pgfpathlineto{\pgfqpoint{4.155100in}{2.479370in}}%
\pgfpathlineto{\pgfqpoint{4.157692in}{2.563727in}}%
\pgfpathlineto{\pgfqpoint{4.158557in}{2.521303in}}%
\pgfpathlineto{\pgfqpoint{4.159420in}{2.599511in}}%
\pgfpathlineto{\pgfqpoint{4.160285in}{2.595822in}}%
\pgfpathlineto{\pgfqpoint{4.162013in}{2.539194in}}%
\pgfpathlineto{\pgfqpoint{4.162876in}{2.547679in}}%
\pgfpathlineto{\pgfqpoint{4.163741in}{2.532369in}}%
\pgfpathlineto{\pgfqpoint{4.164607in}{2.468794in}}%
\pgfpathlineto{\pgfqpoint{4.165472in}{2.562927in}}%
\pgfpathlineto{\pgfqpoint{4.166337in}{2.494772in}}%
\pgfpathlineto{\pgfqpoint{4.167203in}{2.525421in}}%
\pgfpathlineto{\pgfqpoint{4.168067in}{2.442724in}}%
\pgfpathlineto{\pgfqpoint{4.169798in}{2.495785in}}%
\pgfpathlineto{\pgfqpoint{4.170663in}{2.456251in}}%
\pgfpathlineto{\pgfqpoint{4.173258in}{2.540362in}}%
\pgfpathlineto{\pgfqpoint{4.174125in}{2.538948in}}%
\pgfpathlineto{\pgfqpoint{4.176724in}{2.462092in}}%
\pgfpathlineto{\pgfqpoint{4.177588in}{2.537534in}}%
\pgfpathlineto{\pgfqpoint{4.179317in}{2.458772in}}%
\pgfpathlineto{\pgfqpoint{4.181050in}{2.558131in}}%
\pgfpathlineto{\pgfqpoint{4.182780in}{2.477279in}}%
\pgfpathlineto{\pgfqpoint{4.183646in}{2.544482in}}%
\pgfpathlineto{\pgfqpoint{4.184511in}{2.480476in}}%
\pgfpathlineto{\pgfqpoint{4.185377in}{2.496309in}}%
\pgfpathlineto{\pgfqpoint{4.186240in}{2.508637in}}%
\pgfpathlineto{\pgfqpoint{4.187102in}{2.491606in}}%
\pgfpathlineto{\pgfqpoint{4.187968in}{2.501658in}}%
\pgfpathlineto{\pgfqpoint{4.189700in}{2.547186in}}%
\pgfpathlineto{\pgfqpoint{4.190565in}{2.479093in}}%
\pgfpathlineto{\pgfqpoint{4.191432in}{2.529234in}}%
\pgfpathlineto{\pgfqpoint{4.192297in}{2.482382in}}%
\pgfpathlineto{\pgfqpoint{4.193163in}{2.556041in}}%
\pgfpathlineto{\pgfqpoint{4.195760in}{2.489084in}}%
\pgfpathlineto{\pgfqpoint{4.196623in}{2.516321in}}%
\pgfpathlineto{\pgfqpoint{4.198352in}{2.485394in}}%
\pgfpathlineto{\pgfqpoint{4.199218in}{2.591270in}}%
\pgfpathlineto{\pgfqpoint{4.200949in}{2.526958in}}%
\pgfpathlineto{\pgfqpoint{4.201815in}{2.584876in}}%
\pgfpathlineto{\pgfqpoint{4.202680in}{2.499598in}}%
\pgfpathlineto{\pgfqpoint{4.203545in}{2.555979in}}%
\pgfpathlineto{\pgfqpoint{4.204411in}{2.502579in}}%
\pgfpathlineto{\pgfqpoint{4.206142in}{2.542760in}}%
\pgfpathlineto{\pgfqpoint{4.207008in}{2.496863in}}%
\pgfpathlineto{\pgfqpoint{4.208736in}{2.559668in}}%
\pgfpathlineto{\pgfqpoint{4.209602in}{2.549985in}}%
\pgfpathlineto{\pgfqpoint{4.210467in}{2.592871in}}%
\pgfpathlineto{\pgfqpoint{4.211332in}{2.539071in}}%
\pgfpathlineto{\pgfqpoint{4.212198in}{2.546388in}}%
\pgfpathlineto{\pgfqpoint{4.213060in}{2.515030in}}%
\pgfpathlineto{\pgfqpoint{4.213925in}{2.520411in}}%
\pgfpathlineto{\pgfqpoint{4.214789in}{2.549339in}}%
\pgfpathlineto{\pgfqpoint{4.215654in}{2.539810in}}%
\pgfpathlineto{\pgfqpoint{4.216520in}{2.512295in}}%
\pgfpathlineto{\pgfqpoint{4.218250in}{2.592625in}}%
\pgfpathlineto{\pgfqpoint{4.219112in}{2.515677in}}%
\pgfpathlineto{\pgfqpoint{4.219976in}{2.557210in}}%
\pgfpathlineto{\pgfqpoint{4.220842in}{2.548602in}}%
\pgfpathlineto{\pgfqpoint{4.221705in}{2.510420in}}%
\pgfpathlineto{\pgfqpoint{4.222571in}{2.577684in}}%
\pgfpathlineto{\pgfqpoint{4.223436in}{2.536274in}}%
\pgfpathlineto{\pgfqpoint{4.225166in}{2.585985in}}%
\pgfpathlineto{\pgfqpoint{4.226894in}{2.491298in}}%
\pgfpathlineto{\pgfqpoint{4.227760in}{2.568523in}}%
\pgfpathlineto{\pgfqpoint{4.228626in}{2.526099in}}%
\pgfpathlineto{\pgfqpoint{4.229490in}{2.532677in}}%
\pgfpathlineto{\pgfqpoint{4.230356in}{2.531325in}}%
\pgfpathlineto{\pgfqpoint{4.231219in}{2.542945in}}%
\pgfpathlineto{\pgfqpoint{4.232084in}{2.508760in}}%
\pgfpathlineto{\pgfqpoint{4.233814in}{2.539502in}}%
\pgfpathlineto{\pgfqpoint{4.234678in}{2.584509in}}%
\pgfpathlineto{\pgfqpoint{4.235543in}{2.571966in}}%
\pgfpathlineto{\pgfqpoint{4.236409in}{2.589181in}}%
\pgfpathlineto{\pgfqpoint{4.237273in}{2.520749in}}%
\pgfpathlineto{\pgfqpoint{4.238138in}{2.525514in}}%
\pgfpathlineto{\pgfqpoint{4.239867in}{2.576638in}}%
\pgfpathlineto{\pgfqpoint{4.240732in}{2.554134in}}%
\pgfpathlineto{\pgfqpoint{4.241594in}{2.606151in}}%
\pgfpathlineto{\pgfqpoint{4.242461in}{2.525360in}}%
\pgfpathlineto{\pgfqpoint{4.244190in}{2.582172in}}%
\pgfpathlineto{\pgfqpoint{4.245054in}{2.499875in}}%
\pgfpathlineto{\pgfqpoint{4.246784in}{2.591272in}}%
\pgfpathlineto{\pgfqpoint{4.247649in}{2.531017in}}%
\pgfpathlineto{\pgfqpoint{4.248514in}{2.599080in}}%
\pgfpathlineto{\pgfqpoint{4.249380in}{2.541162in}}%
\pgfpathlineto{\pgfqpoint{4.250245in}{2.559915in}}%
\pgfpathlineto{\pgfqpoint{4.251107in}{2.509191in}}%
\pgfpathlineto{\pgfqpoint{4.251972in}{2.513278in}}%
\pgfpathlineto{\pgfqpoint{4.253702in}{2.543191in}}%
\pgfpathlineto{\pgfqpoint{4.255431in}{2.494802in}}%
\pgfpathlineto{\pgfqpoint{4.256297in}{2.507592in}}%
\pgfpathlineto{\pgfqpoint{4.257163in}{2.440143in}}%
\pgfpathlineto{\pgfqpoint{4.258891in}{2.533047in}}%
\pgfpathlineto{\pgfqpoint{4.259758in}{2.458957in}}%
\pgfpathlineto{\pgfqpoint{4.260621in}{2.469841in}}%
\pgfpathlineto{\pgfqpoint{4.263215in}{2.547987in}}%
\pgfpathlineto{\pgfqpoint{4.264079in}{2.503841in}}%
\pgfpathlineto{\pgfqpoint{4.265810in}{2.579837in}}%
\pgfpathlineto{\pgfqpoint{4.266675in}{2.548602in}}%
\pgfpathlineto{\pgfqpoint{4.267541in}{2.572458in}}%
\pgfpathlineto{\pgfqpoint{4.268406in}{2.517922in}}%
\pgfpathlineto{\pgfqpoint{4.270136in}{2.551491in}}%
\pgfpathlineto{\pgfqpoint{4.271000in}{2.495141in}}%
\pgfpathlineto{\pgfqpoint{4.271865in}{2.539194in}}%
\pgfpathlineto{\pgfqpoint{4.272729in}{2.536797in}}%
\pgfpathlineto{\pgfqpoint{4.273593in}{2.509006in}}%
\pgfpathlineto{\pgfqpoint{4.274458in}{2.533416in}}%
\pgfpathlineto{\pgfqpoint{4.276188in}{2.509468in}}%
\pgfpathlineto{\pgfqpoint{4.277054in}{2.511343in}}%
\pgfpathlineto{\pgfqpoint{4.277919in}{2.507715in}}%
\pgfpathlineto{\pgfqpoint{4.279650in}{2.472301in}}%
\pgfpathlineto{\pgfqpoint{4.280515in}{2.504057in}}%
\pgfpathlineto{\pgfqpoint{4.281379in}{2.500583in}}%
\pgfpathlineto{\pgfqpoint{4.282243in}{2.461602in}}%
\pgfpathlineto{\pgfqpoint{4.283109in}{2.548695in}}%
\pgfpathlineto{\pgfqpoint{4.283974in}{2.543007in}}%
\pgfpathlineto{\pgfqpoint{4.284839in}{2.529850in}}%
\pgfpathlineto{\pgfqpoint{4.285704in}{2.491698in}}%
\pgfpathlineto{\pgfqpoint{4.286569in}{2.510912in}}%
\pgfpathlineto{\pgfqpoint{4.288300in}{2.450595in}}%
\pgfpathlineto{\pgfqpoint{4.289164in}{2.524315in}}%
\pgfpathlineto{\pgfqpoint{4.290894in}{2.462646in}}%
\pgfpathlineto{\pgfqpoint{4.292625in}{2.498399in}}%
\pgfpathlineto{\pgfqpoint{4.293491in}{2.488409in}}%
\pgfpathlineto{\pgfqpoint{4.294356in}{2.531140in}}%
\pgfpathlineto{\pgfqpoint{4.295221in}{2.460771in}}%
\pgfpathlineto{\pgfqpoint{4.296948in}{2.528005in}}%
\pgfpathlineto{\pgfqpoint{4.297812in}{2.502243in}}%
\pgfpathlineto{\pgfqpoint{4.298678in}{2.536859in}}%
\pgfpathlineto{\pgfqpoint{4.300410in}{2.463999in}}%
\pgfpathlineto{\pgfqpoint{4.302140in}{2.516783in}}%
\pgfpathlineto{\pgfqpoint{4.303005in}{2.505809in}}%
\pgfpathlineto{\pgfqpoint{4.304734in}{2.484596in}}%
\pgfpathlineto{\pgfqpoint{4.306465in}{2.575409in}}%
\pgfpathlineto{\pgfqpoint{4.307331in}{2.518104in}}%
\pgfpathlineto{\pgfqpoint{4.308196in}{2.607688in}}%
\pgfpathlineto{\pgfqpoint{4.309061in}{2.509650in}}%
\pgfpathlineto{\pgfqpoint{4.309926in}{2.533722in}}%
\pgfpathlineto{\pgfqpoint{4.311656in}{2.480047in}}%
\pgfpathlineto{\pgfqpoint{4.312522in}{2.493512in}}%
\pgfpathlineto{\pgfqpoint{4.313387in}{2.540423in}}%
\pgfpathlineto{\pgfqpoint{4.314253in}{2.533937in}}%
\pgfpathlineto{\pgfqpoint{4.315118in}{2.570059in}}%
\pgfpathlineto{\pgfqpoint{4.315983in}{2.532677in}}%
\pgfpathlineto{\pgfqpoint{4.317712in}{2.580758in}}%
\pgfpathlineto{\pgfqpoint{4.318576in}{2.495079in}}%
\pgfpathlineto{\pgfqpoint{4.320307in}{2.565326in}}%
\pgfpathlineto{\pgfqpoint{4.321170in}{2.491606in}}%
\pgfpathlineto{\pgfqpoint{4.322899in}{2.552782in}}%
\pgfpathlineto{\pgfqpoint{4.323763in}{2.556595in}}%
\pgfpathlineto{\pgfqpoint{4.324629in}{2.528865in}}%
\pgfpathlineto{\pgfqpoint{4.325492in}{2.549524in}}%
\pgfpathlineto{\pgfqpoint{4.326357in}{2.534399in}}%
\pgfpathlineto{\pgfqpoint{4.328083in}{2.558532in}}%
\pgfpathlineto{\pgfqpoint{4.329813in}{2.481584in}}%
\pgfpathlineto{\pgfqpoint{4.331544in}{2.572273in}}%
\pgfpathlineto{\pgfqpoint{4.332409in}{2.567478in}}%
\pgfpathlineto{\pgfqpoint{4.333276in}{2.580943in}}%
\pgfpathlineto{\pgfqpoint{4.334141in}{2.627364in}}%
\pgfpathlineto{\pgfqpoint{4.335868in}{2.486933in}}%
\pgfpathlineto{\pgfqpoint{4.336733in}{2.585430in}}%
\pgfpathlineto{\pgfqpoint{4.337598in}{2.474205in}}%
\pgfpathlineto{\pgfqpoint{4.338461in}{2.526037in}}%
\pgfpathlineto{\pgfqpoint{4.339326in}{2.505501in}}%
\pgfpathlineto{\pgfqpoint{4.340193in}{2.577407in}}%
\pgfpathlineto{\pgfqpoint{4.341058in}{2.537965in}}%
\pgfpathlineto{\pgfqpoint{4.341924in}{2.555796in}}%
\pgfpathlineto{\pgfqpoint{4.342786in}{2.498430in}}%
\pgfpathlineto{\pgfqpoint{4.344511in}{2.580450in}}%
\pgfpathlineto{\pgfqpoint{4.345375in}{2.520257in}}%
\pgfpathlineto{\pgfqpoint{4.346241in}{2.598467in}}%
\pgfpathlineto{\pgfqpoint{4.347106in}{2.527574in}}%
\pgfpathlineto{\pgfqpoint{4.347971in}{2.529850in}}%
\pgfpathlineto{\pgfqpoint{4.348837in}{2.516998in}}%
\pgfpathlineto{\pgfqpoint{4.349701in}{2.573349in}}%
\pgfpathlineto{\pgfqpoint{4.351432in}{2.517429in}}%
\pgfpathlineto{\pgfqpoint{4.352298in}{2.537934in}}%
\pgfpathlineto{\pgfqpoint{4.353162in}{2.613961in}}%
\pgfpathlineto{\pgfqpoint{4.354892in}{2.536120in}}%
\pgfpathlineto{\pgfqpoint{4.355757in}{2.576823in}}%
\pgfpathlineto{\pgfqpoint{4.356622in}{2.564433in}}%
\pgfpathlineto{\pgfqpoint{4.357487in}{2.522163in}}%
\pgfpathlineto{\pgfqpoint{4.358349in}{2.603875in}}%
\pgfpathlineto{\pgfqpoint{4.359212in}{2.586967in}}%
\pgfpathlineto{\pgfqpoint{4.360077in}{2.615742in}}%
\pgfpathlineto{\pgfqpoint{4.360942in}{2.614143in}}%
\pgfpathlineto{\pgfqpoint{4.362672in}{2.525483in}}%
\pgfpathlineto{\pgfqpoint{4.363537in}{2.576515in}}%
\pgfpathlineto{\pgfqpoint{4.364401in}{2.512878in}}%
\pgfpathlineto{\pgfqpoint{4.365264in}{2.518227in}}%
\pgfpathlineto{\pgfqpoint{4.366126in}{2.580204in}}%
\pgfpathlineto{\pgfqpoint{4.366989in}{2.513494in}}%
\pgfpathlineto{\pgfqpoint{4.367853in}{2.525237in}}%
\pgfpathlineto{\pgfqpoint{4.368718in}{2.497969in}}%
\pgfpathlineto{\pgfqpoint{4.369583in}{2.614697in}}%
\pgfpathlineto{\pgfqpoint{4.371313in}{2.501904in}}%
\pgfpathlineto{\pgfqpoint{4.372179in}{2.545159in}}%
\pgfpathlineto{\pgfqpoint{4.373044in}{2.537719in}}%
\pgfpathlineto{\pgfqpoint{4.373909in}{2.535813in}}%
\pgfpathlineto{\pgfqpoint{4.374774in}{2.492896in}}%
\pgfpathlineto{\pgfqpoint{4.377370in}{2.538948in}}%
\pgfpathlineto{\pgfqpoint{4.378235in}{2.506853in}}%
\pgfpathlineto{\pgfqpoint{4.379100in}{2.518658in}}%
\pgfpathlineto{\pgfqpoint{4.379961in}{2.557331in}}%
\pgfpathlineto{\pgfqpoint{4.380827in}{2.498613in}}%
\pgfpathlineto{\pgfqpoint{4.382557in}{2.585492in}}%
\pgfpathlineto{\pgfqpoint{4.383424in}{2.542391in}}%
\pgfpathlineto{\pgfqpoint{4.384290in}{2.573195in}}%
\pgfpathlineto{\pgfqpoint{4.386020in}{2.542268in}}%
\pgfpathlineto{\pgfqpoint{4.386886in}{2.564094in}}%
\pgfpathlineto{\pgfqpoint{4.387751in}{2.557701in}}%
\pgfpathlineto{\pgfqpoint{4.388616in}{2.476111in}}%
\pgfpathlineto{\pgfqpoint{4.389481in}{2.547125in}}%
\pgfpathlineto{\pgfqpoint{4.390346in}{2.489268in}}%
\pgfpathlineto{\pgfqpoint{4.391212in}{2.516444in}}%
\pgfpathlineto{\pgfqpoint{4.392077in}{2.516383in}}%
\pgfpathlineto{\pgfqpoint{4.392939in}{2.476848in}}%
\pgfpathlineto{\pgfqpoint{4.393805in}{2.498430in}}%
\pgfpathlineto{\pgfqpoint{4.394670in}{2.496401in}}%
\pgfpathlineto{\pgfqpoint{4.395533in}{2.462092in}}%
\pgfpathlineto{\pgfqpoint{4.397262in}{2.510143in}}%
\pgfpathlineto{\pgfqpoint{4.398127in}{2.509496in}}%
\pgfpathlineto{\pgfqpoint{4.398990in}{2.593115in}}%
\pgfpathlineto{\pgfqpoint{4.399855in}{2.464581in}}%
\pgfpathlineto{\pgfqpoint{4.401586in}{2.583463in}}%
\pgfpathlineto{\pgfqpoint{4.404184in}{2.477710in}}%
\pgfpathlineto{\pgfqpoint{4.405047in}{2.508206in}}%
\pgfpathlineto{\pgfqpoint{4.405913in}{2.484596in}}%
\pgfpathlineto{\pgfqpoint{4.406779in}{2.575378in}}%
\pgfpathlineto{\pgfqpoint{4.408510in}{2.484165in}}%
\pgfpathlineto{\pgfqpoint{4.410240in}{2.570244in}}%
\pgfpathlineto{\pgfqpoint{4.411104in}{2.566678in}}%
\pgfpathlineto{\pgfqpoint{4.412836in}{2.527451in}}%
\pgfpathlineto{\pgfqpoint{4.413701in}{2.482444in}}%
\pgfpathlineto{\pgfqpoint{4.414565in}{2.600925in}}%
\pgfpathlineto{\pgfqpoint{4.415430in}{2.579221in}}%
\pgfpathlineto{\pgfqpoint{4.417161in}{2.488409in}}%
\pgfpathlineto{\pgfqpoint{4.418026in}{2.557639in}}%
\pgfpathlineto{\pgfqpoint{4.418891in}{2.472053in}}%
\pgfpathlineto{\pgfqpoint{4.419756in}{2.574916in}}%
\pgfpathlineto{\pgfqpoint{4.421485in}{2.482998in}}%
\pgfpathlineto{\pgfqpoint{4.423211in}{2.532523in}}%
\pgfpathlineto{\pgfqpoint{4.424076in}{2.577315in}}%
\pgfpathlineto{\pgfqpoint{4.424939in}{2.548418in}}%
\pgfpathlineto{\pgfqpoint{4.425802in}{2.469656in}}%
\pgfpathlineto{\pgfqpoint{4.426667in}{2.496647in}}%
\pgfpathlineto{\pgfqpoint{4.427532in}{2.581987in}}%
\pgfpathlineto{\pgfqpoint{4.428398in}{2.429259in}}%
\pgfpathlineto{\pgfqpoint{4.429263in}{2.585430in}}%
\pgfpathlineto{\pgfqpoint{4.430129in}{2.557947in}}%
\pgfpathlineto{\pgfqpoint{4.430994in}{2.548110in}}%
\pgfpathlineto{\pgfqpoint{4.431859in}{2.624473in}}%
\pgfpathlineto{\pgfqpoint{4.432725in}{2.530586in}}%
\pgfpathlineto{\pgfqpoint{4.434454in}{2.624288in}}%
\pgfpathlineto{\pgfqpoint{4.435319in}{2.482967in}}%
\pgfpathlineto{\pgfqpoint{4.436184in}{2.615067in}}%
\pgfpathlineto{\pgfqpoint{4.437050in}{2.587337in}}%
\pgfpathlineto{\pgfqpoint{4.438780in}{2.554196in}}%
\pgfpathlineto{\pgfqpoint{4.439645in}{2.556471in}}%
\pgfpathlineto{\pgfqpoint{4.440508in}{2.506607in}}%
\pgfpathlineto{\pgfqpoint{4.441371in}{2.523331in}}%
\pgfpathlineto{\pgfqpoint{4.442235in}{2.490713in}}%
\pgfpathlineto{\pgfqpoint{4.443101in}{2.533475in}}%
\pgfpathlineto{\pgfqpoint{4.443964in}{2.490805in}}%
\pgfpathlineto{\pgfqpoint{4.445695in}{2.544482in}}%
\pgfpathlineto{\pgfqpoint{4.446561in}{2.501381in}}%
\pgfpathlineto{\pgfqpoint{4.448292in}{2.539687in}}%
\pgfpathlineto{\pgfqpoint{4.449156in}{2.534399in}}%
\pgfpathlineto{\pgfqpoint{4.450020in}{2.463629in}}%
\pgfpathlineto{\pgfqpoint{4.451751in}{2.532400in}}%
\pgfpathlineto{\pgfqpoint{4.452617in}{2.453669in}}%
\pgfpathlineto{\pgfqpoint{4.453484in}{2.537657in}}%
\pgfpathlineto{\pgfqpoint{4.455213in}{2.465597in}}%
\pgfpathlineto{\pgfqpoint{4.456077in}{2.547864in}}%
\pgfpathlineto{\pgfqpoint{4.456943in}{2.431196in}}%
\pgfpathlineto{\pgfqpoint{4.457807in}{2.544297in}}%
\pgfpathlineto{\pgfqpoint{4.458672in}{2.515584in}}%
\pgfpathlineto{\pgfqpoint{4.459537in}{2.486256in}}%
\pgfpathlineto{\pgfqpoint{4.461266in}{2.542914in}}%
\pgfpathlineto{\pgfqpoint{4.462131in}{2.531325in}}%
\pgfpathlineto{\pgfqpoint{4.462995in}{2.476419in}}%
\pgfpathlineto{\pgfqpoint{4.463858in}{2.542668in}}%
\pgfpathlineto{\pgfqpoint{4.464723in}{2.508452in}}%
\pgfpathlineto{\pgfqpoint{4.465588in}{2.539194in}}%
\pgfpathlineto{\pgfqpoint{4.466453in}{2.501843in}}%
\pgfpathlineto{\pgfqpoint{4.467318in}{2.594654in}}%
\pgfpathlineto{\pgfqpoint{4.468183in}{2.490007in}}%
\pgfpathlineto{\pgfqpoint{4.469047in}{2.503472in}}%
\pgfpathlineto{\pgfqpoint{4.469913in}{2.532431in}}%
\pgfpathlineto{\pgfqpoint{4.470777in}{2.497784in}}%
\pgfpathlineto{\pgfqpoint{4.471642in}{2.533722in}}%
\pgfpathlineto{\pgfqpoint{4.472507in}{2.501381in}}%
\pgfpathlineto{\pgfqpoint{4.473373in}{2.535505in}}%
\pgfpathlineto{\pgfqpoint{4.475100in}{2.510604in}}%
\pgfpathlineto{\pgfqpoint{4.476831in}{2.541223in}}%
\pgfpathlineto{\pgfqpoint{4.477695in}{2.538027in}}%
\pgfpathlineto{\pgfqpoint{4.478560in}{2.541593in}}%
\pgfpathlineto{\pgfqpoint{4.481157in}{2.490007in}}%
\pgfpathlineto{\pgfqpoint{4.482023in}{2.510297in}}%
\pgfpathlineto{\pgfqpoint{4.482888in}{2.578483in}}%
\pgfpathlineto{\pgfqpoint{4.483754in}{2.508760in}}%
\pgfpathlineto{\pgfqpoint{4.484620in}{2.560284in}}%
\pgfpathlineto{\pgfqpoint{4.485486in}{2.506884in}}%
\pgfpathlineto{\pgfqpoint{4.486350in}{2.606397in}}%
\pgfpathlineto{\pgfqpoint{4.487216in}{2.604306in}}%
\pgfpathlineto{\pgfqpoint{4.490672in}{2.490192in}}%
\pgfpathlineto{\pgfqpoint{4.491537in}{2.498307in}}%
\pgfpathlineto{\pgfqpoint{4.492402in}{2.550016in}}%
\pgfpathlineto{\pgfqpoint{4.494133in}{2.487424in}}%
\pgfpathlineto{\pgfqpoint{4.494998in}{2.552351in}}%
\pgfpathlineto{\pgfqpoint{4.495863in}{2.518289in}}%
\pgfpathlineto{\pgfqpoint{4.497591in}{2.545588in}}%
\pgfpathlineto{\pgfqpoint{4.498455in}{2.475557in}}%
\pgfpathlineto{\pgfqpoint{4.500183in}{2.540608in}}%
\pgfpathlineto{\pgfqpoint{4.501049in}{2.547494in}}%
\pgfpathlineto{\pgfqpoint{4.501914in}{2.508390in}}%
\pgfpathlineto{\pgfqpoint{4.503645in}{2.575224in}}%
\pgfpathlineto{\pgfqpoint{4.504511in}{2.526866in}}%
\pgfpathlineto{\pgfqpoint{4.505375in}{2.580143in}}%
\pgfpathlineto{\pgfqpoint{4.507970in}{2.522900in}}%
\pgfpathlineto{\pgfqpoint{4.508835in}{2.564187in}}%
\pgfpathlineto{\pgfqpoint{4.509700in}{2.522348in}}%
\pgfpathlineto{\pgfqpoint{4.510564in}{2.549277in}}%
\pgfpathlineto{\pgfqpoint{4.512294in}{2.509927in}}%
\pgfpathlineto{\pgfqpoint{4.513159in}{2.582049in}}%
\pgfpathlineto{\pgfqpoint{4.514025in}{2.496647in}}%
\pgfpathlineto{\pgfqpoint{4.514890in}{2.502058in}}%
\pgfpathlineto{\pgfqpoint{4.515753in}{2.521917in}}%
\pgfpathlineto{\pgfqpoint{4.517482in}{2.456743in}}%
\pgfpathlineto{\pgfqpoint{4.518346in}{2.509866in}}%
\pgfpathlineto{\pgfqpoint{4.519212in}{2.495172in}}%
\pgfpathlineto{\pgfqpoint{4.521806in}{2.568461in}}%
\pgfpathlineto{\pgfqpoint{4.522670in}{2.521917in}}%
\pgfpathlineto{\pgfqpoint{4.523535in}{2.532646in}}%
\pgfpathlineto{\pgfqpoint{4.524400in}{2.560284in}}%
\pgfpathlineto{\pgfqpoint{4.525263in}{2.549093in}}%
\pgfpathlineto{\pgfqpoint{4.526993in}{2.515708in}}%
\pgfpathlineto{\pgfqpoint{4.527857in}{2.546327in}}%
\pgfpathlineto{\pgfqpoint{4.528722in}{2.542576in}}%
\pgfpathlineto{\pgfqpoint{4.529586in}{2.481338in}}%
\pgfpathlineto{\pgfqpoint{4.530452in}{2.551399in}}%
\pgfpathlineto{\pgfqpoint{4.532179in}{2.480907in}}%
\pgfpathlineto{\pgfqpoint{4.533043in}{2.541991in}}%
\pgfpathlineto{\pgfqpoint{4.533909in}{2.468732in}}%
\pgfpathlineto{\pgfqpoint{4.534774in}{2.577254in}}%
\pgfpathlineto{\pgfqpoint{4.536505in}{2.502918in}}%
\pgfpathlineto{\pgfqpoint{4.537370in}{2.533845in}}%
\pgfpathlineto{\pgfqpoint{4.539100in}{2.485887in}}%
\pgfpathlineto{\pgfqpoint{4.539965in}{2.504916in}}%
\pgfpathlineto{\pgfqpoint{4.540828in}{2.586906in}}%
\pgfpathlineto{\pgfqpoint{4.541691in}{2.530586in}}%
\pgfpathlineto{\pgfqpoint{4.542555in}{2.574947in}}%
\pgfpathlineto{\pgfqpoint{4.544282in}{2.540362in}}%
\pgfpathlineto{\pgfqpoint{4.545145in}{2.499228in}}%
\pgfpathlineto{\pgfqpoint{4.546010in}{2.573993in}}%
\pgfpathlineto{\pgfqpoint{4.546874in}{2.570365in}}%
\pgfpathlineto{\pgfqpoint{4.547738in}{2.577682in}}%
\pgfpathlineto{\pgfqpoint{4.548603in}{2.564587in}}%
\pgfpathlineto{\pgfqpoint{4.549468in}{2.524991in}}%
\pgfpathlineto{\pgfqpoint{4.550331in}{2.553090in}}%
\pgfpathlineto{\pgfqpoint{4.552060in}{2.516352in}}%
\pgfpathlineto{\pgfqpoint{4.552925in}{2.589918in}}%
\pgfpathlineto{\pgfqpoint{4.553786in}{2.574485in}}%
\pgfpathlineto{\pgfqpoint{4.554650in}{2.516937in}}%
\pgfpathlineto{\pgfqpoint{4.555516in}{2.578298in}}%
\pgfpathlineto{\pgfqpoint{4.556379in}{2.576761in}}%
\pgfpathlineto{\pgfqpoint{4.557244in}{2.532308in}}%
\pgfpathlineto{\pgfqpoint{4.558109in}{2.547494in}}%
\pgfpathlineto{\pgfqpoint{4.558975in}{2.474544in}}%
\pgfpathlineto{\pgfqpoint{4.560705in}{2.547740in}}%
\pgfpathlineto{\pgfqpoint{4.562436in}{2.510174in}}%
\pgfpathlineto{\pgfqpoint{4.564162in}{2.544974in}}%
\pgfpathlineto{\pgfqpoint{4.565028in}{2.501997in}}%
\pgfpathlineto{\pgfqpoint{4.565894in}{2.508821in}}%
\pgfpathlineto{\pgfqpoint{4.566758in}{2.497140in}}%
\pgfpathlineto{\pgfqpoint{4.567622in}{2.561944in}}%
\pgfpathlineto{\pgfqpoint{4.568487in}{2.481153in}}%
\pgfpathlineto{\pgfqpoint{4.569351in}{2.575655in}}%
\pgfpathlineto{\pgfqpoint{4.570217in}{2.442665in}}%
\pgfpathlineto{\pgfqpoint{4.571948in}{2.555119in}}%
\pgfpathlineto{\pgfqpoint{4.573680in}{2.583955in}}%
\pgfpathlineto{\pgfqpoint{4.576275in}{2.467319in}}%
\pgfpathlineto{\pgfqpoint{4.578005in}{2.567907in}}%
\pgfpathlineto{\pgfqpoint{4.578869in}{2.564833in}}%
\pgfpathlineto{\pgfqpoint{4.579732in}{2.462216in}}%
\pgfpathlineto{\pgfqpoint{4.580597in}{2.611991in}}%
\pgfpathlineto{\pgfqpoint{4.582325in}{2.544913in}}%
\pgfpathlineto{\pgfqpoint{4.583189in}{2.761955in}}%
\pgfpathlineto{\pgfqpoint{4.584053in}{2.749965in}}%
\pgfpathlineto{\pgfqpoint{4.586645in}{2.866971in}}%
\pgfpathlineto{\pgfqpoint{4.587511in}{2.772346in}}%
\pgfpathlineto{\pgfqpoint{4.589242in}{2.853260in}}%
\pgfpathlineto{\pgfqpoint{4.590109in}{2.801336in}}%
\pgfpathlineto{\pgfqpoint{4.590975in}{2.809544in}}%
\pgfpathlineto{\pgfqpoint{4.592706in}{2.788823in}}%
\pgfpathlineto{\pgfqpoint{4.593571in}{2.823378in}}%
\pgfpathlineto{\pgfqpoint{4.594436in}{2.794172in}}%
\pgfpathlineto{\pgfqpoint{4.595300in}{2.816738in}}%
\pgfpathlineto{\pgfqpoint{4.597029in}{2.759802in}}%
\pgfpathlineto{\pgfqpoint{4.599625in}{2.814678in}}%
\pgfpathlineto{\pgfqpoint{4.600489in}{2.821779in}}%
\pgfpathlineto{\pgfqpoint{4.601354in}{2.771117in}}%
\pgfpathlineto{\pgfqpoint{4.602218in}{2.843115in}}%
\pgfpathlineto{\pgfqpoint{4.603082in}{2.795404in}}%
\pgfpathlineto{\pgfqpoint{4.603947in}{2.852983in}}%
\pgfpathlineto{\pgfqpoint{4.604812in}{2.751197in}}%
\pgfpathlineto{\pgfqpoint{4.605675in}{2.788394in}}%
\pgfpathlineto{\pgfqpoint{4.606541in}{2.785043in}}%
\pgfpathlineto{\pgfqpoint{4.607407in}{2.741973in}}%
\pgfpathlineto{\pgfqpoint{4.609137in}{2.801828in}}%
\pgfpathlineto{\pgfqpoint{4.610868in}{2.716088in}}%
\pgfpathlineto{\pgfqpoint{4.613459in}{2.818305in}}%
\pgfpathlineto{\pgfqpoint{4.615186in}{2.779294in}}%
\pgfpathlineto{\pgfqpoint{4.616052in}{2.796109in}}%
\pgfpathlineto{\pgfqpoint{4.616918in}{2.777326in}}%
\pgfpathlineto{\pgfqpoint{4.617784in}{2.789254in}}%
\pgfpathlineto{\pgfqpoint{4.618650in}{2.738068in}}%
\pgfpathlineto{\pgfqpoint{4.620380in}{2.774437in}}%
\pgfpathlineto{\pgfqpoint{4.621246in}{2.781323in}}%
\pgfpathlineto{\pgfqpoint{4.622110in}{2.820735in}}%
\pgfpathlineto{\pgfqpoint{4.622974in}{2.727923in}}%
\pgfpathlineto{\pgfqpoint{4.624705in}{2.782614in}}%
\pgfpathlineto{\pgfqpoint{4.626436in}{2.689833in}}%
\pgfpathlineto{\pgfqpoint{4.628165in}{2.856057in}}%
\pgfpathlineto{\pgfqpoint{4.629028in}{2.796263in}}%
\pgfpathlineto{\pgfqpoint{4.630756in}{2.815755in}}%
\pgfpathlineto{\pgfqpoint{4.631622in}{2.871151in}}%
\pgfpathlineto{\pgfqpoint{4.632489in}{2.810067in}}%
\pgfpathlineto{\pgfqpoint{4.633355in}{2.862114in}}%
\pgfpathlineto{\pgfqpoint{4.635081in}{2.778372in}}%
\pgfpathlineto{\pgfqpoint{4.636810in}{2.839734in}}%
\pgfpathlineto{\pgfqpoint{4.637675in}{2.778065in}}%
\pgfpathlineto{\pgfqpoint{4.638541in}{2.909949in}}%
\pgfpathlineto{\pgfqpoint{4.640272in}{2.786734in}}%
\pgfpathlineto{\pgfqpoint{4.641137in}{2.794911in}}%
\pgfpathlineto{\pgfqpoint{4.642003in}{2.791591in}}%
\pgfpathlineto{\pgfqpoint{4.642868in}{2.836167in}}%
\pgfpathlineto{\pgfqpoint{4.643733in}{2.825530in}}%
\pgfpathlineto{\pgfqpoint{4.644598in}{2.827252in}}%
\pgfpathlineto{\pgfqpoint{4.645460in}{2.843362in}}%
\pgfpathlineto{\pgfqpoint{4.646326in}{2.769580in}}%
\pgfpathlineto{\pgfqpoint{4.647191in}{2.801736in}}%
\pgfpathlineto{\pgfqpoint{4.648057in}{2.797495in}}%
\pgfpathlineto{\pgfqpoint{4.648923in}{2.814464in}}%
\pgfpathlineto{\pgfqpoint{4.649788in}{2.778619in}}%
\pgfpathlineto{\pgfqpoint{4.650652in}{2.813297in}}%
\pgfpathlineto{\pgfqpoint{4.652381in}{2.561084in}}%
\pgfpathlineto{\pgfqpoint{4.653247in}{2.511651in}}%
\pgfpathlineto{\pgfqpoint{4.654976in}{2.547373in}}%
\pgfpathlineto{\pgfqpoint{4.655843in}{2.539933in}}%
\pgfpathlineto{\pgfqpoint{4.656707in}{2.544544in}}%
\pgfpathlineto{\pgfqpoint{4.657572in}{2.563912in}}%
\pgfpathlineto{\pgfqpoint{4.659302in}{2.520626in}}%
\pgfpathlineto{\pgfqpoint{4.660168in}{2.622507in}}%
\pgfpathlineto{\pgfqpoint{4.661897in}{2.459080in}}%
\pgfpathlineto{\pgfqpoint{4.663625in}{2.520872in}}%
\pgfpathlineto{\pgfqpoint{4.664490in}{2.531448in}}%
\pgfpathlineto{\pgfqpoint{4.665354in}{2.530034in}}%
\pgfpathlineto{\pgfqpoint{4.666218in}{2.499015in}}%
\pgfpathlineto{\pgfqpoint{4.667082in}{2.544974in}}%
\pgfpathlineto{\pgfqpoint{4.667948in}{2.543438in}}%
\pgfpathlineto{\pgfqpoint{4.668812in}{2.521119in}}%
\pgfpathlineto{\pgfqpoint{4.669677in}{2.596129in}}%
\pgfpathlineto{\pgfqpoint{4.671407in}{2.487978in}}%
\pgfpathlineto{\pgfqpoint{4.672272in}{2.521611in}}%
\pgfpathlineto{\pgfqpoint{4.673137in}{2.473468in}}%
\pgfpathlineto{\pgfqpoint{4.674000in}{2.490315in}}%
\pgfpathlineto{\pgfqpoint{4.674866in}{2.602341in}}%
\pgfpathlineto{\pgfqpoint{4.675730in}{2.495757in}}%
\pgfpathlineto{\pgfqpoint{4.677460in}{2.579283in}}%
\pgfpathlineto{\pgfqpoint{4.678325in}{2.524131in}}%
\pgfpathlineto{\pgfqpoint{4.679190in}{2.555981in}}%
\pgfpathlineto{\pgfqpoint{4.680921in}{2.466521in}}%
\pgfpathlineto{\pgfqpoint{4.681784in}{2.513986in}}%
\pgfpathlineto{\pgfqpoint{4.682650in}{2.471439in}}%
\pgfpathlineto{\pgfqpoint{4.683516in}{2.563912in}}%
\pgfpathlineto{\pgfqpoint{4.684380in}{2.506363in}}%
\pgfpathlineto{\pgfqpoint{4.686110in}{2.555673in}}%
\pgfpathlineto{\pgfqpoint{4.687840in}{2.520503in}}%
\pgfpathlineto{\pgfqpoint{4.688705in}{2.553705in}}%
\pgfpathlineto{\pgfqpoint{4.689568in}{2.527389in}}%
\pgfpathlineto{\pgfqpoint{4.691297in}{2.627426in}}%
\pgfpathlineto{\pgfqpoint{4.693027in}{2.550139in}}%
\pgfpathlineto{\pgfqpoint{4.693889in}{2.618387in}}%
\pgfpathlineto{\pgfqpoint{4.695621in}{2.528742in}}%
\pgfpathlineto{\pgfqpoint{4.696486in}{2.549647in}}%
\pgfpathlineto{\pgfqpoint{4.697352in}{2.496524in}}%
\pgfpathlineto{\pgfqpoint{4.698218in}{2.562129in}}%
\pgfpathlineto{\pgfqpoint{4.699949in}{2.539994in}}%
\pgfpathlineto{\pgfqpoint{4.700812in}{2.542884in}}%
\pgfpathlineto{\pgfqpoint{4.701675in}{2.493481in}}%
\pgfpathlineto{\pgfqpoint{4.702540in}{2.530586in}}%
\pgfpathlineto{\pgfqpoint{4.704271in}{2.477525in}}%
\pgfpathlineto{\pgfqpoint{4.705999in}{2.528557in}}%
\pgfpathlineto{\pgfqpoint{4.706864in}{2.548846in}}%
\pgfpathlineto{\pgfqpoint{4.708594in}{2.481982in}}%
\pgfpathlineto{\pgfqpoint{4.709461in}{2.490744in}}%
\pgfpathlineto{\pgfqpoint{4.710325in}{2.555548in}}%
\pgfpathlineto{\pgfqpoint{4.711190in}{2.493941in}}%
\pgfpathlineto{\pgfqpoint{4.712056in}{2.525421in}}%
\pgfpathlineto{\pgfqpoint{4.714648in}{2.473035in}}%
\pgfpathlineto{\pgfqpoint{4.717245in}{2.545588in}}%
\pgfpathlineto{\pgfqpoint{4.718111in}{2.488961in}}%
\pgfpathlineto{\pgfqpoint{4.718976in}{2.524621in}}%
\pgfpathlineto{\pgfqpoint{4.719839in}{2.494556in}}%
\pgfpathlineto{\pgfqpoint{4.721569in}{2.539992in}}%
\pgfpathlineto{\pgfqpoint{4.722434in}{2.487485in}}%
\pgfpathlineto{\pgfqpoint{4.723300in}{2.574516in}}%
\pgfpathlineto{\pgfqpoint{4.724166in}{2.562681in}}%
\pgfpathlineto{\pgfqpoint{4.725894in}{2.515154in}}%
\pgfpathlineto{\pgfqpoint{4.726759in}{2.521240in}}%
\pgfpathlineto{\pgfqpoint{4.728487in}{2.458588in}}%
\pgfpathlineto{\pgfqpoint{4.729353in}{2.520134in}}%
\pgfpathlineto{\pgfqpoint{4.730215in}{2.486133in}}%
\pgfpathlineto{\pgfqpoint{4.731947in}{2.527574in}}%
\pgfpathlineto{\pgfqpoint{4.733678in}{2.563235in}}%
\pgfpathlineto{\pgfqpoint{4.734543in}{2.527389in}}%
\pgfpathlineto{\pgfqpoint{4.735409in}{2.572643in}}%
\pgfpathlineto{\pgfqpoint{4.737997in}{2.522594in}}%
\pgfpathlineto{\pgfqpoint{4.739729in}{2.498399in}}%
\pgfpathlineto{\pgfqpoint{4.740595in}{2.584078in}}%
\pgfpathlineto{\pgfqpoint{4.742327in}{2.517060in}}%
\pgfpathlineto{\pgfqpoint{4.743193in}{2.549154in}}%
\pgfpathlineto{\pgfqpoint{4.744060in}{2.508821in}}%
\pgfpathlineto{\pgfqpoint{4.745789in}{2.566249in}}%
\pgfpathlineto{\pgfqpoint{4.746655in}{2.451303in}}%
\pgfpathlineto{\pgfqpoint{4.748384in}{2.493327in}}%
\pgfpathlineto{\pgfqpoint{4.749246in}{2.482076in}}%
\pgfpathlineto{\pgfqpoint{4.750112in}{2.490130in}}%
\pgfpathlineto{\pgfqpoint{4.750977in}{2.561759in}}%
\pgfpathlineto{\pgfqpoint{4.751842in}{2.556010in}}%
\pgfpathlineto{\pgfqpoint{4.752708in}{2.504947in}}%
\pgfpathlineto{\pgfqpoint{4.753575in}{2.570059in}}%
\pgfpathlineto{\pgfqpoint{4.754440in}{2.558378in}}%
\pgfpathlineto{\pgfqpoint{4.756170in}{2.516814in}}%
\pgfpathlineto{\pgfqpoint{4.757901in}{2.575163in}}%
\pgfpathlineto{\pgfqpoint{4.758765in}{2.530740in}}%
\pgfpathlineto{\pgfqpoint{4.759630in}{2.567968in}}%
\pgfpathlineto{\pgfqpoint{4.760492in}{2.544051in}}%
\pgfpathlineto{\pgfqpoint{4.761358in}{2.575686in}}%
\pgfpathlineto{\pgfqpoint{4.762223in}{2.527328in}}%
\pgfpathlineto{\pgfqpoint{4.763088in}{2.556533in}}%
\pgfpathlineto{\pgfqpoint{4.763954in}{2.510420in}}%
\pgfpathlineto{\pgfqpoint{4.764818in}{2.606151in}}%
\pgfpathlineto{\pgfqpoint{4.765683in}{2.557149in}}%
\pgfpathlineto{\pgfqpoint{4.766549in}{2.560407in}}%
\pgfpathlineto{\pgfqpoint{4.767415in}{2.579283in}}%
\pgfpathlineto{\pgfqpoint{4.768280in}{2.517275in}}%
\pgfpathlineto{\pgfqpoint{4.769146in}{2.531448in}}%
\pgfpathlineto{\pgfqpoint{4.770012in}{2.517245in}}%
\pgfpathlineto{\pgfqpoint{4.770878in}{2.575532in}}%
\pgfpathlineto{\pgfqpoint{4.772610in}{2.525175in}}%
\pgfpathlineto{\pgfqpoint{4.773477in}{2.548171in}}%
\pgfpathlineto{\pgfqpoint{4.774342in}{2.491482in}}%
\pgfpathlineto{\pgfqpoint{4.775208in}{2.559361in}}%
\pgfpathlineto{\pgfqpoint{4.776073in}{2.542884in}}%
\pgfpathlineto{\pgfqpoint{4.776937in}{2.541408in}}%
\pgfpathlineto{\pgfqpoint{4.777802in}{2.525668in}}%
\pgfpathlineto{\pgfqpoint{4.778668in}{2.541162in}}%
\pgfpathlineto{\pgfqpoint{4.779534in}{2.532185in}}%
\pgfpathlineto{\pgfqpoint{4.780401in}{2.489699in}}%
\pgfpathlineto{\pgfqpoint{4.781267in}{2.535813in}}%
\pgfpathlineto{\pgfqpoint{4.782133in}{2.525976in}}%
\pgfpathlineto{\pgfqpoint{4.782998in}{2.489576in}}%
\pgfpathlineto{\pgfqpoint{4.783865in}{2.596006in}}%
\pgfpathlineto{\pgfqpoint{4.784730in}{2.484658in}}%
\pgfpathlineto{\pgfqpoint{4.786461in}{2.561821in}}%
\pgfpathlineto{\pgfqpoint{4.787327in}{2.556625in}}%
\pgfpathlineto{\pgfqpoint{4.789059in}{2.582418in}}%
\pgfpathlineto{\pgfqpoint{4.790791in}{2.508513in}}%
\pgfpathlineto{\pgfqpoint{4.791657in}{2.532677in}}%
\pgfpathlineto{\pgfqpoint{4.792520in}{2.459203in}}%
\pgfpathlineto{\pgfqpoint{4.794249in}{2.542668in}}%
\pgfpathlineto{\pgfqpoint{4.795115in}{2.454100in}}%
\pgfpathlineto{\pgfqpoint{4.796845in}{2.580789in}}%
\pgfpathlineto{\pgfqpoint{4.797709in}{2.486687in}}%
\pgfpathlineto{\pgfqpoint{4.798574in}{2.631728in}}%
\pgfpathlineto{\pgfqpoint{4.801166in}{2.531879in}}%
\pgfpathlineto{\pgfqpoint{4.802030in}{2.502181in}}%
\pgfpathlineto{\pgfqpoint{4.803757in}{2.550201in}}%
\pgfpathlineto{\pgfqpoint{4.804623in}{2.505317in}}%
\pgfpathlineto{\pgfqpoint{4.805486in}{2.517799in}}%
\pgfpathlineto{\pgfqpoint{4.806350in}{2.558963in}}%
\pgfpathlineto{\pgfqpoint{4.807214in}{2.505932in}}%
\pgfpathlineto{\pgfqpoint{4.808943in}{2.590965in}}%
\pgfpathlineto{\pgfqpoint{4.810674in}{2.506240in}}%
\pgfpathlineto{\pgfqpoint{4.813264in}{2.615436in}}%
\pgfpathlineto{\pgfqpoint{4.814993in}{2.558809in}}%
\pgfpathlineto{\pgfqpoint{4.815858in}{2.556718in}}%
\pgfpathlineto{\pgfqpoint{4.816722in}{2.491359in}}%
\pgfpathlineto{\pgfqpoint{4.817588in}{2.544482in}}%
\pgfpathlineto{\pgfqpoint{4.818454in}{2.486318in}}%
\pgfpathlineto{\pgfqpoint{4.820184in}{2.553275in}}%
\pgfpathlineto{\pgfqpoint{4.821048in}{2.519212in}}%
\pgfpathlineto{\pgfqpoint{4.821914in}{2.568830in}}%
\pgfpathlineto{\pgfqpoint{4.822781in}{2.539625in}}%
\pgfpathlineto{\pgfqpoint{4.823647in}{2.565757in}}%
\pgfpathlineto{\pgfqpoint{4.825379in}{2.466490in}}%
\pgfpathlineto{\pgfqpoint{4.826244in}{2.467934in}}%
\pgfpathlineto{\pgfqpoint{4.827110in}{2.531448in}}%
\pgfpathlineto{\pgfqpoint{4.827976in}{2.524008in}}%
\pgfpathlineto{\pgfqpoint{4.828842in}{2.502674in}}%
\pgfpathlineto{\pgfqpoint{4.829709in}{2.522163in}}%
\pgfpathlineto{\pgfqpoint{4.830573in}{2.484350in}}%
\pgfpathlineto{\pgfqpoint{4.831440in}{2.515646in}}%
\pgfpathlineto{\pgfqpoint{4.832305in}{2.510358in}}%
\pgfpathlineto{\pgfqpoint{4.833170in}{2.515769in}}%
\pgfpathlineto{\pgfqpoint{4.834897in}{2.485058in}}%
\pgfpathlineto{\pgfqpoint{4.835762in}{2.493573in}}%
\pgfpathlineto{\pgfqpoint{4.836629in}{2.486749in}}%
\pgfpathlineto{\pgfqpoint{4.838359in}{2.521057in}}%
\pgfpathlineto{\pgfqpoint{4.839223in}{2.521057in}}%
\pgfpathlineto{\pgfqpoint{4.840088in}{2.516200in}}%
\pgfpathlineto{\pgfqpoint{4.840953in}{2.545159in}}%
\pgfpathlineto{\pgfqpoint{4.841818in}{2.502089in}}%
\pgfpathlineto{\pgfqpoint{4.842682in}{2.548602in}}%
\pgfpathlineto{\pgfqpoint{4.842682in}{2.548602in}}%
\pgfusepath{stroke}%
\end{pgfscope}%
\begin{pgfscope}%
\pgfsetrectcap%
\pgfsetmiterjoin%
\pgfsetlinewidth{0.803000pt}%
\definecolor{currentstroke}{rgb}{0.000000,0.000000,0.000000}%
\pgfsetstrokecolor{currentstroke}%
\pgfsetdash{}{0pt}%
\pgfpathmoveto{\pgfqpoint{0.483776in}{2.351653in}}%
\pgfpathlineto{\pgfqpoint{0.483776in}{2.936535in}}%
\pgfusepath{stroke}%
\end{pgfscope}%
\begin{pgfscope}%
\pgfsetrectcap%
\pgfsetmiterjoin%
\pgfsetlinewidth{0.803000pt}%
\definecolor{currentstroke}{rgb}{0.000000,0.000000,0.000000}%
\pgfsetstrokecolor{currentstroke}%
\pgfsetdash{}{0pt}%
\pgfpathmoveto{\pgfqpoint{5.050249in}{2.351653in}}%
\pgfpathlineto{\pgfqpoint{5.050249in}{2.936535in}}%
\pgfusepath{stroke}%
\end{pgfscope}%
\begin{pgfscope}%
\pgfsetrectcap%
\pgfsetmiterjoin%
\pgfsetlinewidth{0.803000pt}%
\definecolor{currentstroke}{rgb}{0.000000,0.000000,0.000000}%
\pgfsetstrokecolor{currentstroke}%
\pgfsetdash{}{0pt}%
\pgfpathmoveto{\pgfqpoint{0.483776in}{2.351653in}}%
\pgfpathlineto{\pgfqpoint{5.050249in}{2.351653in}}%
\pgfusepath{stroke}%
\end{pgfscope}%
\begin{pgfscope}%
\pgfsetrectcap%
\pgfsetmiterjoin%
\pgfsetlinewidth{0.803000pt}%
\definecolor{currentstroke}{rgb}{0.000000,0.000000,0.000000}%
\pgfsetstrokecolor{currentstroke}%
\pgfsetdash{}{0pt}%
\pgfpathmoveto{\pgfqpoint{0.483776in}{2.936535in}}%
\pgfpathlineto{\pgfqpoint{5.050249in}{2.936535in}}%
\pgfusepath{stroke}%
\end{pgfscope}%
\begin{pgfscope}%
\pgfsetbuttcap%
\pgfsetmiterjoin%
\definecolor{currentfill}{rgb}{1.000000,1.000000,1.000000}%
\pgfsetfillcolor{currentfill}%
\pgfsetlinewidth{0.000000pt}%
\definecolor{currentstroke}{rgb}{0.000000,0.000000,0.000000}%
\pgfsetstrokecolor{currentstroke}%
\pgfsetstrokeopacity{0.000000}%
\pgfsetdash{}{0pt}%
\pgfpathmoveto{\pgfqpoint{0.483776in}{1.444834in}}%
\pgfpathlineto{\pgfqpoint{5.050249in}{1.444834in}}%
\pgfpathlineto{\pgfqpoint{5.050249in}{2.029715in}}%
\pgfpathlineto{\pgfqpoint{0.483776in}{2.029715in}}%
\pgfpathlineto{\pgfqpoint{0.483776in}{1.444834in}}%
\pgfpathclose%
\pgfusepath{fill}%
\end{pgfscope}%
\begin{pgfscope}%
\pgfsetbuttcap%
\pgfsetroundjoin%
\definecolor{currentfill}{rgb}{0.000000,0.000000,0.000000}%
\pgfsetfillcolor{currentfill}%
\pgfsetlinewidth{0.803000pt}%
\definecolor{currentstroke}{rgb}{0.000000,0.000000,0.000000}%
\pgfsetstrokecolor{currentstroke}%
\pgfsetdash{}{0pt}%
\pgfsys@defobject{currentmarker}{\pgfqpoint{0.000000in}{-0.048611in}}{\pgfqpoint{0.000000in}{0.000000in}}{%
\pgfpathmoveto{\pgfqpoint{0.000000in}{0.000000in}}%
\pgfpathlineto{\pgfqpoint{0.000000in}{-0.048611in}}%
\pgfusepath{stroke,fill}%
}%
\begin{pgfscope}%
\pgfsys@transformshift{0.691021in}{1.444834in}%
\pgfsys@useobject{currentmarker}{}%
\end{pgfscope}%
\end{pgfscope}%
\begin{pgfscope}%
\pgfsetbuttcap%
\pgfsetroundjoin%
\definecolor{currentfill}{rgb}{0.000000,0.000000,0.000000}%
\pgfsetfillcolor{currentfill}%
\pgfsetlinewidth{0.803000pt}%
\definecolor{currentstroke}{rgb}{0.000000,0.000000,0.000000}%
\pgfsetstrokecolor{currentstroke}%
\pgfsetdash{}{0pt}%
\pgfsys@defobject{currentmarker}{\pgfqpoint{0.000000in}{-0.048611in}}{\pgfqpoint{0.000000in}{0.000000in}}{%
\pgfpathmoveto{\pgfqpoint{0.000000in}{0.000000in}}%
\pgfpathlineto{\pgfqpoint{0.000000in}{-0.048611in}}%
\pgfusepath{stroke,fill}%
}%
\begin{pgfscope}%
\pgfsys@transformshift{1.210067in}{1.444834in}%
\pgfsys@useobject{currentmarker}{}%
\end{pgfscope}%
\end{pgfscope}%
\begin{pgfscope}%
\pgfsetbuttcap%
\pgfsetroundjoin%
\definecolor{currentfill}{rgb}{0.000000,0.000000,0.000000}%
\pgfsetfillcolor{currentfill}%
\pgfsetlinewidth{0.803000pt}%
\definecolor{currentstroke}{rgb}{0.000000,0.000000,0.000000}%
\pgfsetstrokecolor{currentstroke}%
\pgfsetdash{}{0pt}%
\pgfsys@defobject{currentmarker}{\pgfqpoint{0.000000in}{-0.048611in}}{\pgfqpoint{0.000000in}{0.000000in}}{%
\pgfpathmoveto{\pgfqpoint{0.000000in}{0.000000in}}%
\pgfpathlineto{\pgfqpoint{0.000000in}{-0.048611in}}%
\pgfusepath{stroke,fill}%
}%
\begin{pgfscope}%
\pgfsys@transformshift{1.729114in}{1.444834in}%
\pgfsys@useobject{currentmarker}{}%
\end{pgfscope}%
\end{pgfscope}%
\begin{pgfscope}%
\pgfsetbuttcap%
\pgfsetroundjoin%
\definecolor{currentfill}{rgb}{0.000000,0.000000,0.000000}%
\pgfsetfillcolor{currentfill}%
\pgfsetlinewidth{0.803000pt}%
\definecolor{currentstroke}{rgb}{0.000000,0.000000,0.000000}%
\pgfsetstrokecolor{currentstroke}%
\pgfsetdash{}{0pt}%
\pgfsys@defobject{currentmarker}{\pgfqpoint{0.000000in}{-0.048611in}}{\pgfqpoint{0.000000in}{0.000000in}}{%
\pgfpathmoveto{\pgfqpoint{0.000000in}{0.000000in}}%
\pgfpathlineto{\pgfqpoint{0.000000in}{-0.048611in}}%
\pgfusepath{stroke,fill}%
}%
\begin{pgfscope}%
\pgfsys@transformshift{2.248160in}{1.444834in}%
\pgfsys@useobject{currentmarker}{}%
\end{pgfscope}%
\end{pgfscope}%
\begin{pgfscope}%
\pgfsetbuttcap%
\pgfsetroundjoin%
\definecolor{currentfill}{rgb}{0.000000,0.000000,0.000000}%
\pgfsetfillcolor{currentfill}%
\pgfsetlinewidth{0.803000pt}%
\definecolor{currentstroke}{rgb}{0.000000,0.000000,0.000000}%
\pgfsetstrokecolor{currentstroke}%
\pgfsetdash{}{0pt}%
\pgfsys@defobject{currentmarker}{\pgfqpoint{0.000000in}{-0.048611in}}{\pgfqpoint{0.000000in}{0.000000in}}{%
\pgfpathmoveto{\pgfqpoint{0.000000in}{0.000000in}}%
\pgfpathlineto{\pgfqpoint{0.000000in}{-0.048611in}}%
\pgfusepath{stroke,fill}%
}%
\begin{pgfscope}%
\pgfsys@transformshift{2.767206in}{1.444834in}%
\pgfsys@useobject{currentmarker}{}%
\end{pgfscope}%
\end{pgfscope}%
\begin{pgfscope}%
\pgfsetbuttcap%
\pgfsetroundjoin%
\definecolor{currentfill}{rgb}{0.000000,0.000000,0.000000}%
\pgfsetfillcolor{currentfill}%
\pgfsetlinewidth{0.803000pt}%
\definecolor{currentstroke}{rgb}{0.000000,0.000000,0.000000}%
\pgfsetstrokecolor{currentstroke}%
\pgfsetdash{}{0pt}%
\pgfsys@defobject{currentmarker}{\pgfqpoint{0.000000in}{-0.048611in}}{\pgfqpoint{0.000000in}{0.000000in}}{%
\pgfpathmoveto{\pgfqpoint{0.000000in}{0.000000in}}%
\pgfpathlineto{\pgfqpoint{0.000000in}{-0.048611in}}%
\pgfusepath{stroke,fill}%
}%
\begin{pgfscope}%
\pgfsys@transformshift{3.286252in}{1.444834in}%
\pgfsys@useobject{currentmarker}{}%
\end{pgfscope}%
\end{pgfscope}%
\begin{pgfscope}%
\pgfsetbuttcap%
\pgfsetroundjoin%
\definecolor{currentfill}{rgb}{0.000000,0.000000,0.000000}%
\pgfsetfillcolor{currentfill}%
\pgfsetlinewidth{0.803000pt}%
\definecolor{currentstroke}{rgb}{0.000000,0.000000,0.000000}%
\pgfsetstrokecolor{currentstroke}%
\pgfsetdash{}{0pt}%
\pgfsys@defobject{currentmarker}{\pgfqpoint{0.000000in}{-0.048611in}}{\pgfqpoint{0.000000in}{0.000000in}}{%
\pgfpathmoveto{\pgfqpoint{0.000000in}{0.000000in}}%
\pgfpathlineto{\pgfqpoint{0.000000in}{-0.048611in}}%
\pgfusepath{stroke,fill}%
}%
\begin{pgfscope}%
\pgfsys@transformshift{3.805298in}{1.444834in}%
\pgfsys@useobject{currentmarker}{}%
\end{pgfscope}%
\end{pgfscope}%
\begin{pgfscope}%
\pgfsetbuttcap%
\pgfsetroundjoin%
\definecolor{currentfill}{rgb}{0.000000,0.000000,0.000000}%
\pgfsetfillcolor{currentfill}%
\pgfsetlinewidth{0.803000pt}%
\definecolor{currentstroke}{rgb}{0.000000,0.000000,0.000000}%
\pgfsetstrokecolor{currentstroke}%
\pgfsetdash{}{0pt}%
\pgfsys@defobject{currentmarker}{\pgfqpoint{0.000000in}{-0.048611in}}{\pgfqpoint{0.000000in}{0.000000in}}{%
\pgfpathmoveto{\pgfqpoint{0.000000in}{0.000000in}}%
\pgfpathlineto{\pgfqpoint{0.000000in}{-0.048611in}}%
\pgfusepath{stroke,fill}%
}%
\begin{pgfscope}%
\pgfsys@transformshift{4.324344in}{1.444834in}%
\pgfsys@useobject{currentmarker}{}%
\end{pgfscope}%
\end{pgfscope}%
\begin{pgfscope}%
\pgfsetbuttcap%
\pgfsetroundjoin%
\definecolor{currentfill}{rgb}{0.000000,0.000000,0.000000}%
\pgfsetfillcolor{currentfill}%
\pgfsetlinewidth{0.803000pt}%
\definecolor{currentstroke}{rgb}{0.000000,0.000000,0.000000}%
\pgfsetstrokecolor{currentstroke}%
\pgfsetdash{}{0pt}%
\pgfsys@defobject{currentmarker}{\pgfqpoint{0.000000in}{-0.048611in}}{\pgfqpoint{0.000000in}{0.000000in}}{%
\pgfpathmoveto{\pgfqpoint{0.000000in}{0.000000in}}%
\pgfpathlineto{\pgfqpoint{0.000000in}{-0.048611in}}%
\pgfusepath{stroke,fill}%
}%
\begin{pgfscope}%
\pgfsys@transformshift{4.843390in}{1.444834in}%
\pgfsys@useobject{currentmarker}{}%
\end{pgfscope}%
\end{pgfscope}%
\begin{pgfscope}%
\pgfsetbuttcap%
\pgfsetroundjoin%
\definecolor{currentfill}{rgb}{0.000000,0.000000,0.000000}%
\pgfsetfillcolor{currentfill}%
\pgfsetlinewidth{0.803000pt}%
\definecolor{currentstroke}{rgb}{0.000000,0.000000,0.000000}%
\pgfsetstrokecolor{currentstroke}%
\pgfsetdash{}{0pt}%
\pgfsys@defobject{currentmarker}{\pgfqpoint{-0.048611in}{0.000000in}}{\pgfqpoint{-0.000000in}{0.000000in}}{%
\pgfpathmoveto{\pgfqpoint{-0.000000in}{0.000000in}}%
\pgfpathlineto{\pgfqpoint{-0.048611in}{0.000000in}}%
\pgfusepath{stroke,fill}%
}%
\begin{pgfscope}%
\pgfsys@transformshift{0.483776in}{1.626011in}%
\pgfsys@useobject{currentmarker}{}%
\end{pgfscope}%
\end{pgfscope}%
\begin{pgfscope}%
\definecolor{textcolor}{rgb}{0.000000,0.000000,0.000000}%
\pgfsetstrokecolor{textcolor}%
\pgfsetfillcolor{textcolor}%
\pgftext[x=0.327525in, y=1.587455in, left, base]{\color{textcolor}\rmfamily\fontsize{8.000000}{9.600000}\selectfont \(\displaystyle {0}\)}%
\end{pgfscope}%
\begin{pgfscope}%
\pgfsetbuttcap%
\pgfsetroundjoin%
\definecolor{currentfill}{rgb}{0.000000,0.000000,0.000000}%
\pgfsetfillcolor{currentfill}%
\pgfsetlinewidth{0.803000pt}%
\definecolor{currentstroke}{rgb}{0.000000,0.000000,0.000000}%
\pgfsetstrokecolor{currentstroke}%
\pgfsetdash{}{0pt}%
\pgfsys@defobject{currentmarker}{\pgfqpoint{-0.048611in}{0.000000in}}{\pgfqpoint{-0.000000in}{0.000000in}}{%
\pgfpathmoveto{\pgfqpoint{-0.000000in}{0.000000in}}%
\pgfpathlineto{\pgfqpoint{-0.048611in}{0.000000in}}%
\pgfusepath{stroke,fill}%
}%
\begin{pgfscope}%
\pgfsys@transformshift{0.483776in}{1.833186in}%
\pgfsys@useobject{currentmarker}{}%
\end{pgfscope}%
\end{pgfscope}%
\begin{pgfscope}%
\definecolor{textcolor}{rgb}{0.000000,0.000000,0.000000}%
\pgfsetstrokecolor{textcolor}%
\pgfsetfillcolor{textcolor}%
\pgftext[x=0.327525in, y=1.794630in, left, base]{\color{textcolor}\rmfamily\fontsize{8.000000}{9.600000}\selectfont \(\displaystyle {5}\)}%
\end{pgfscope}%
\begin{pgfscope}%
\definecolor{textcolor}{rgb}{0.000000,0.000000,0.000000}%
\pgfsetstrokecolor{textcolor}%
\pgfsetfillcolor{textcolor}%
\pgftext[x=0.271969in,y=1.737274in,,bottom,rotate=90.000000]{\color{textcolor}\rmfamily\fontsize{10.000000}{12.000000}\selectfont Voltage deviation in V}%
\end{pgfscope}%
\begin{pgfscope}%
\definecolor{textcolor}{rgb}{0.000000,0.000000,0.000000}%
\pgfsetstrokecolor{textcolor}%
\pgfsetfillcolor{textcolor}%
\pgftext[x=0.483776in,y=2.071382in,left,base]{\color{textcolor}\rmfamily\fontsize{8.000000}{9.600000}\selectfont \(\displaystyle \times{10^{\ensuremath{-}6}}{}\)}%
\end{pgfscope}%
\begin{pgfscope}%
\pgfpathrectangle{\pgfqpoint{0.483776in}{1.444834in}}{\pgfqpoint{4.566474in}{0.584881in}}%
\pgfusepath{clip}%
\pgfsetrectcap%
\pgfsetroundjoin%
\pgfsetlinewidth{0.501875pt}%
\definecolor{currentstroke}{rgb}{0.000000,0.419608,0.643137}%
\pgfsetstrokecolor{currentstroke}%
\pgfsetstrokeopacity{0.700000}%
\pgfsetdash{}{0pt}%
\pgfpathmoveto{\pgfqpoint{0.691343in}{1.596619in}}%
\pgfpathlineto{\pgfqpoint{0.692205in}{1.591513in}}%
\pgfpathlineto{\pgfqpoint{0.693071in}{1.636512in}}%
\pgfpathlineto{\pgfqpoint{0.693935in}{1.590000in}}%
\pgfpathlineto{\pgfqpoint{0.694800in}{1.622502in}}%
\pgfpathlineto{\pgfqpoint{0.696532in}{1.564503in}}%
\pgfpathlineto{\pgfqpoint{0.697397in}{1.619059in}}%
\pgfpathlineto{\pgfqpoint{0.698263in}{1.610629in}}%
\pgfpathlineto{\pgfqpoint{0.699128in}{1.631764in}}%
\pgfpathlineto{\pgfqpoint{0.699993in}{1.620010in}}%
\pgfpathlineto{\pgfqpoint{0.700859in}{1.589230in}}%
\pgfpathlineto{\pgfqpoint{0.701725in}{1.635267in}}%
\pgfpathlineto{\pgfqpoint{0.702589in}{1.632299in}}%
\pgfpathlineto{\pgfqpoint{0.703453in}{1.560261in}}%
\pgfpathlineto{\pgfqpoint{0.705185in}{1.630637in}}%
\pgfpathlineto{\pgfqpoint{0.706051in}{1.606446in}}%
\pgfpathlineto{\pgfqpoint{0.706916in}{1.633962in}}%
\pgfpathlineto{\pgfqpoint{0.707780in}{1.563734in}}%
\pgfpathlineto{\pgfqpoint{0.709512in}{1.641264in}}%
\pgfpathlineto{\pgfqpoint{0.710377in}{1.594068in}}%
\pgfpathlineto{\pgfqpoint{0.711241in}{1.638055in}}%
\pgfpathlineto{\pgfqpoint{0.713837in}{1.560053in}}%
\pgfpathlineto{\pgfqpoint{0.715567in}{1.651947in}}%
\pgfpathlineto{\pgfqpoint{0.717297in}{1.619653in}}%
\pgfpathlineto{\pgfqpoint{0.718163in}{1.549188in}}%
\pgfpathlineto{\pgfqpoint{0.719030in}{1.575011in}}%
\pgfpathlineto{\pgfqpoint{0.719895in}{1.559696in}}%
\pgfpathlineto{\pgfqpoint{0.721627in}{1.606059in}}%
\pgfpathlineto{\pgfqpoint{0.722492in}{1.557498in}}%
\pgfpathlineto{\pgfqpoint{0.724219in}{1.598697in}}%
\pgfpathlineto{\pgfqpoint{0.725084in}{1.538948in}}%
\pgfpathlineto{\pgfqpoint{0.727676in}{1.610867in}}%
\pgfpathlineto{\pgfqpoint{0.728541in}{1.538208in}}%
\pgfpathlineto{\pgfqpoint{0.731135in}{1.630161in}}%
\pgfpathlineto{\pgfqpoint{0.731999in}{1.530756in}}%
\pgfpathlineto{\pgfqpoint{0.733731in}{1.628023in}}%
\pgfpathlineto{\pgfqpoint{0.734597in}{1.605970in}}%
\pgfpathlineto{\pgfqpoint{0.735460in}{1.549128in}}%
\pgfpathlineto{\pgfqpoint{0.736325in}{1.631645in}}%
\pgfpathlineto{\pgfqpoint{0.737190in}{1.582195in}}%
\pgfpathlineto{\pgfqpoint{0.738920in}{1.627845in}}%
\pgfpathlineto{\pgfqpoint{0.739784in}{1.595611in}}%
\pgfpathlineto{\pgfqpoint{0.740649in}{1.605643in}}%
\pgfpathlineto{\pgfqpoint{0.742381in}{1.518973in}}%
\pgfpathlineto{\pgfqpoint{0.744977in}{1.639186in}}%
\pgfpathlineto{\pgfqpoint{0.746705in}{1.568661in}}%
\pgfpathlineto{\pgfqpoint{0.748435in}{1.669520in}}%
\pgfpathlineto{\pgfqpoint{0.750163in}{1.617579in}}%
\pgfpathlineto{\pgfqpoint{0.751028in}{1.624822in}}%
\pgfpathlineto{\pgfqpoint{0.751891in}{1.524970in}}%
\pgfpathlineto{\pgfqpoint{0.752757in}{1.642391in}}%
\pgfpathlineto{\pgfqpoint{0.753622in}{1.534824in}}%
\pgfpathlineto{\pgfqpoint{0.754488in}{1.630578in}}%
\pgfpathlineto{\pgfqpoint{0.755354in}{1.628916in}}%
\pgfpathlineto{\pgfqpoint{0.756219in}{1.610454in}}%
\pgfpathlineto{\pgfqpoint{0.757084in}{1.687092in}}%
\pgfpathlineto{\pgfqpoint{0.759679in}{1.594722in}}%
\pgfpathlineto{\pgfqpoint{0.761407in}{1.632478in}}%
\pgfpathlineto{\pgfqpoint{0.762269in}{1.608733in}}%
\pgfpathlineto{\pgfqpoint{0.763135in}{1.634378in}}%
\pgfpathlineto{\pgfqpoint{0.764000in}{1.784806in}}%
\pgfpathlineto{\pgfqpoint{0.766596in}{1.673320in}}%
\pgfpathlineto{\pgfqpoint{0.767462in}{1.752453in}}%
\pgfpathlineto{\pgfqpoint{0.768326in}{1.742183in}}%
\pgfpathlineto{\pgfqpoint{0.769191in}{1.752869in}}%
\pgfpathlineto{\pgfqpoint{0.771788in}{1.592346in}}%
\pgfpathlineto{\pgfqpoint{0.772652in}{1.682965in}}%
\pgfpathlineto{\pgfqpoint{0.774383in}{1.598816in}}%
\pgfpathlineto{\pgfqpoint{0.775249in}{1.627607in}}%
\pgfpathlineto{\pgfqpoint{0.776979in}{1.584330in}}%
\pgfpathlineto{\pgfqpoint{0.777845in}{1.590208in}}%
\pgfpathlineto{\pgfqpoint{0.779575in}{1.572963in}}%
\pgfpathlineto{\pgfqpoint{0.780439in}{1.634553in}}%
\pgfpathlineto{\pgfqpoint{0.781305in}{1.591454in}}%
\pgfpathlineto{\pgfqpoint{0.782170in}{1.593116in}}%
\pgfpathlineto{\pgfqpoint{0.783036in}{1.580117in}}%
\pgfpathlineto{\pgfqpoint{0.783901in}{1.581600in}}%
\pgfpathlineto{\pgfqpoint{0.784766in}{1.593592in}}%
\pgfpathlineto{\pgfqpoint{0.785630in}{1.530012in}}%
\pgfpathlineto{\pgfqpoint{0.786494in}{1.608848in}}%
\pgfpathlineto{\pgfqpoint{0.787360in}{1.607245in}}%
\pgfpathlineto{\pgfqpoint{0.788225in}{1.537137in}}%
\pgfpathlineto{\pgfqpoint{0.789089in}{1.538115in}}%
\pgfpathlineto{\pgfqpoint{0.790816in}{1.602259in}}%
\pgfpathlineto{\pgfqpoint{0.791680in}{1.600210in}}%
\pgfpathlineto{\pgfqpoint{0.793412in}{1.593949in}}%
\pgfpathlineto{\pgfqpoint{0.795138in}{1.575725in}}%
\pgfpathlineto{\pgfqpoint{0.796004in}{1.655747in}}%
\pgfpathlineto{\pgfqpoint{0.798599in}{1.566879in}}%
\pgfpathlineto{\pgfqpoint{0.799464in}{1.545923in}}%
\pgfpathlineto{\pgfqpoint{0.801192in}{1.607840in}}%
\pgfpathlineto{\pgfqpoint{0.802924in}{1.552218in}}%
\pgfpathlineto{\pgfqpoint{0.803787in}{1.549902in}}%
\pgfpathlineto{\pgfqpoint{0.804652in}{1.567474in}}%
\pgfpathlineto{\pgfqpoint{0.805515in}{1.508703in}}%
\pgfpathlineto{\pgfqpoint{0.806378in}{1.522178in}}%
\pgfpathlineto{\pgfqpoint{0.807243in}{1.582017in}}%
\pgfpathlineto{\pgfqpoint{0.808107in}{1.561239in}}%
\pgfpathlineto{\pgfqpoint{0.808973in}{1.571687in}}%
\pgfpathlineto{\pgfqpoint{0.809837in}{1.500333in}}%
\pgfpathlineto{\pgfqpoint{0.811567in}{1.570025in}}%
\pgfpathlineto{\pgfqpoint{0.812431in}{1.578038in}}%
\pgfpathlineto{\pgfqpoint{0.813297in}{1.571449in}}%
\pgfpathlineto{\pgfqpoint{0.814161in}{1.632950in}}%
\pgfpathlineto{\pgfqpoint{0.815026in}{1.547496in}}%
\pgfpathlineto{\pgfqpoint{0.815891in}{1.566166in}}%
\pgfpathlineto{\pgfqpoint{0.817620in}{1.605167in}}%
\pgfpathlineto{\pgfqpoint{0.818484in}{1.600419in}}%
\pgfpathlineto{\pgfqpoint{0.819349in}{1.562306in}}%
\pgfpathlineto{\pgfqpoint{0.820213in}{1.627696in}}%
\pgfpathlineto{\pgfqpoint{0.821078in}{1.621137in}}%
\pgfpathlineto{\pgfqpoint{0.822807in}{1.664887in}}%
\pgfpathlineto{\pgfqpoint{0.823671in}{1.600538in}}%
\pgfpathlineto{\pgfqpoint{0.824536in}{1.619267in}}%
\pgfpathlineto{\pgfqpoint{0.826268in}{1.602616in}}%
\pgfpathlineto{\pgfqpoint{0.827999in}{1.618170in}}%
\pgfpathlineto{\pgfqpoint{0.828864in}{1.556729in}}%
\pgfpathlineto{\pgfqpoint{0.830596in}{1.696054in}}%
\pgfpathlineto{\pgfqpoint{0.831462in}{1.617991in}}%
\pgfpathlineto{\pgfqpoint{0.832328in}{1.622859in}}%
\pgfpathlineto{\pgfqpoint{0.833193in}{1.666196in}}%
\pgfpathlineto{\pgfqpoint{0.834058in}{1.655182in}}%
\pgfpathlineto{\pgfqpoint{0.834922in}{1.663879in}}%
\pgfpathlineto{\pgfqpoint{0.836650in}{1.590744in}}%
\pgfpathlineto{\pgfqpoint{0.837516in}{1.554352in}}%
\pgfpathlineto{\pgfqpoint{0.838382in}{1.571568in}}%
\pgfpathlineto{\pgfqpoint{0.839248in}{1.531764in}}%
\pgfpathlineto{\pgfqpoint{0.840980in}{1.572992in}}%
\pgfpathlineto{\pgfqpoint{0.842710in}{1.547228in}}%
\pgfpathlineto{\pgfqpoint{0.844440in}{1.605107in}}%
\pgfpathlineto{\pgfqpoint{0.845305in}{1.600300in}}%
\pgfpathlineto{\pgfqpoint{0.846169in}{1.545566in}}%
\pgfpathlineto{\pgfqpoint{0.847897in}{1.599292in}}%
\pgfpathlineto{\pgfqpoint{0.849627in}{1.580176in}}%
\pgfpathlineto{\pgfqpoint{0.850491in}{1.575368in}}%
\pgfpathlineto{\pgfqpoint{0.851354in}{1.557498in}}%
\pgfpathlineto{\pgfqpoint{0.853081in}{1.600359in}}%
\pgfpathlineto{\pgfqpoint{0.853945in}{1.542361in}}%
\pgfpathlineto{\pgfqpoint{0.854810in}{1.630991in}}%
\pgfpathlineto{\pgfqpoint{0.855672in}{1.627815in}}%
\pgfpathlineto{\pgfqpoint{0.856537in}{1.595254in}}%
\pgfpathlineto{\pgfqpoint{0.857402in}{1.657231in}}%
\pgfpathlineto{\pgfqpoint{0.858267in}{1.596767in}}%
\pgfpathlineto{\pgfqpoint{0.859131in}{1.636453in}}%
\pgfpathlineto{\pgfqpoint{0.860864in}{1.577741in}}%
\pgfpathlineto{\pgfqpoint{0.861730in}{1.630693in}}%
\pgfpathlineto{\pgfqpoint{0.864329in}{1.560644in}}%
\pgfpathlineto{\pgfqpoint{0.866062in}{1.576019in}}%
\pgfpathlineto{\pgfqpoint{0.866929in}{1.597983in}}%
\pgfpathlineto{\pgfqpoint{0.867794in}{1.565333in}}%
\pgfpathlineto{\pgfqpoint{0.869523in}{1.610391in}}%
\pgfpathlineto{\pgfqpoint{0.870388in}{1.588427in}}%
\pgfpathlineto{\pgfqpoint{0.871253in}{1.636393in}}%
\pgfpathlineto{\pgfqpoint{0.872119in}{1.630693in}}%
\pgfpathlineto{\pgfqpoint{0.873848in}{1.621550in}}%
\pgfpathlineto{\pgfqpoint{0.874712in}{1.570438in}}%
\pgfpathlineto{\pgfqpoint{0.876442in}{1.691008in}}%
\pgfpathlineto{\pgfqpoint{0.877309in}{1.664946in}}%
\pgfpathlineto{\pgfqpoint{0.878174in}{1.667144in}}%
\pgfpathlineto{\pgfqpoint{0.879903in}{1.651709in}}%
\pgfpathlineto{\pgfqpoint{0.881633in}{1.694749in}}%
\pgfpathlineto{\pgfqpoint{0.882498in}{1.646187in}}%
\pgfpathlineto{\pgfqpoint{0.883364in}{1.713150in}}%
\pgfpathlineto{\pgfqpoint{0.884229in}{1.703356in}}%
\pgfpathlineto{\pgfqpoint{0.885096in}{1.669282in}}%
\pgfpathlineto{\pgfqpoint{0.885960in}{1.702999in}}%
\pgfpathlineto{\pgfqpoint{0.886826in}{1.628142in}}%
\pgfpathlineto{\pgfqpoint{0.887691in}{1.653490in}}%
\pgfpathlineto{\pgfqpoint{0.888555in}{1.717129in}}%
\pgfpathlineto{\pgfqpoint{0.892011in}{1.653193in}}%
\pgfpathlineto{\pgfqpoint{0.892875in}{1.734047in}}%
\pgfpathlineto{\pgfqpoint{0.893741in}{1.654557in}}%
\pgfpathlineto{\pgfqpoint{0.894605in}{1.695102in}}%
\pgfpathlineto{\pgfqpoint{0.897202in}{1.588427in}}%
\pgfpathlineto{\pgfqpoint{0.898067in}{1.593235in}}%
\pgfpathlineto{\pgfqpoint{0.899795in}{1.651293in}}%
\pgfpathlineto{\pgfqpoint{0.900660in}{1.598310in}}%
\pgfpathlineto{\pgfqpoint{0.901525in}{1.619356in}}%
\pgfpathlineto{\pgfqpoint{0.902390in}{1.677295in}}%
\pgfpathlineto{\pgfqpoint{0.903254in}{1.612380in}}%
\pgfpathlineto{\pgfqpoint{0.904119in}{1.652244in}}%
\pgfpathlineto{\pgfqpoint{0.904984in}{1.593116in}}%
\pgfpathlineto{\pgfqpoint{0.905849in}{1.665333in}}%
\pgfpathlineto{\pgfqpoint{0.906714in}{1.656279in}}%
\pgfpathlineto{\pgfqpoint{0.907579in}{1.650106in}}%
\pgfpathlineto{\pgfqpoint{0.908445in}{1.596619in}}%
\pgfpathlineto{\pgfqpoint{0.910175in}{1.645061in}}%
\pgfpathlineto{\pgfqpoint{0.912770in}{1.569282in}}%
\pgfpathlineto{\pgfqpoint{0.914501in}{1.562127in}}%
\pgfpathlineto{\pgfqpoint{0.915365in}{1.582786in}}%
\pgfpathlineto{\pgfqpoint{0.916229in}{1.546574in}}%
\pgfpathlineto{\pgfqpoint{0.917959in}{1.613715in}}%
\pgfpathlineto{\pgfqpoint{0.918824in}{1.552690in}}%
\pgfpathlineto{\pgfqpoint{0.919687in}{1.556104in}}%
\pgfpathlineto{\pgfqpoint{0.920552in}{1.542539in}}%
\pgfpathlineto{\pgfqpoint{0.922280in}{1.625499in}}%
\pgfpathlineto{\pgfqpoint{0.923146in}{1.597094in}}%
\pgfpathlineto{\pgfqpoint{0.924010in}{1.553877in}}%
\pgfpathlineto{\pgfqpoint{0.925740in}{1.631110in}}%
\pgfpathlineto{\pgfqpoint{0.928329in}{1.557974in}}%
\pgfpathlineto{\pgfqpoint{0.930058in}{1.603802in}}%
\pgfpathlineto{\pgfqpoint{0.930923in}{1.541528in}}%
\pgfpathlineto{\pgfqpoint{0.931787in}{1.610094in}}%
\pgfpathlineto{\pgfqpoint{0.932652in}{1.555598in}}%
\pgfpathlineto{\pgfqpoint{0.933516in}{1.584092in}}%
\pgfpathlineto{\pgfqpoint{0.934381in}{1.531496in}}%
\pgfpathlineto{\pgfqpoint{0.936109in}{1.601307in}}%
\pgfpathlineto{\pgfqpoint{0.936974in}{1.599526in}}%
\pgfpathlineto{\pgfqpoint{0.937840in}{1.629269in}}%
\pgfpathlineto{\pgfqpoint{0.938706in}{1.543190in}}%
\pgfpathlineto{\pgfqpoint{0.939570in}{1.557082in}}%
\pgfpathlineto{\pgfqpoint{0.940435in}{1.640011in}}%
\pgfpathlineto{\pgfqpoint{0.943032in}{1.518553in}}%
\pgfpathlineto{\pgfqpoint{0.943897in}{1.606175in}}%
\pgfpathlineto{\pgfqpoint{0.944763in}{1.599586in}}%
\pgfpathlineto{\pgfqpoint{0.945628in}{1.627309in}}%
\pgfpathlineto{\pgfqpoint{0.946490in}{1.547228in}}%
\pgfpathlineto{\pgfqpoint{0.947355in}{1.605345in}}%
\pgfpathlineto{\pgfqpoint{0.949082in}{1.571390in}}%
\pgfpathlineto{\pgfqpoint{0.949946in}{1.566225in}}%
\pgfpathlineto{\pgfqpoint{0.950811in}{1.544703in}}%
\pgfpathlineto{\pgfqpoint{0.951674in}{1.488219in}}%
\pgfpathlineto{\pgfqpoint{0.953405in}{1.594272in}}%
\pgfpathlineto{\pgfqpoint{0.954269in}{1.610391in}}%
\pgfpathlineto{\pgfqpoint{0.955133in}{1.519564in}}%
\pgfpathlineto{\pgfqpoint{0.956863in}{1.648325in}}%
\pgfpathlineto{\pgfqpoint{0.957728in}{1.610570in}}%
\pgfpathlineto{\pgfqpoint{0.958594in}{1.615734in}}%
\pgfpathlineto{\pgfqpoint{0.961191in}{1.571033in}}%
\pgfpathlineto{\pgfqpoint{0.962055in}{1.590625in}}%
\pgfpathlineto{\pgfqpoint{0.962920in}{1.537018in}}%
\pgfpathlineto{\pgfqpoint{0.963783in}{1.619059in}}%
\pgfpathlineto{\pgfqpoint{0.964647in}{1.581481in}}%
\pgfpathlineto{\pgfqpoint{0.966375in}{1.614132in}}%
\pgfpathlineto{\pgfqpoint{0.967241in}{1.554085in}}%
\pgfpathlineto{\pgfqpoint{0.968105in}{1.590625in}}%
\pgfpathlineto{\pgfqpoint{0.968970in}{1.574654in}}%
\pgfpathlineto{\pgfqpoint{0.969836in}{1.576555in}}%
\pgfpathlineto{\pgfqpoint{0.970701in}{1.635088in}}%
\pgfpathlineto{\pgfqpoint{0.972432in}{1.491011in}}%
\pgfpathlineto{\pgfqpoint{0.975027in}{1.636215in}}%
\pgfpathlineto{\pgfqpoint{0.975889in}{1.550433in}}%
\pgfpathlineto{\pgfqpoint{0.976751in}{1.557409in}}%
\pgfpathlineto{\pgfqpoint{0.977616in}{1.638293in}}%
\pgfpathlineto{\pgfqpoint{0.978481in}{1.622918in}}%
\pgfpathlineto{\pgfqpoint{0.980211in}{1.599054in}}%
\pgfpathlineto{\pgfqpoint{0.981076in}{1.655093in}}%
\pgfpathlineto{\pgfqpoint{0.982806in}{1.611165in}}%
\pgfpathlineto{\pgfqpoint{0.983671in}{1.629745in}}%
\pgfpathlineto{\pgfqpoint{0.985400in}{1.603981in}}%
\pgfpathlineto{\pgfqpoint{0.986264in}{1.621048in}}%
\pgfpathlineto{\pgfqpoint{0.987127in}{1.532031in}}%
\pgfpathlineto{\pgfqpoint{0.988856in}{1.586051in}}%
\pgfpathlineto{\pgfqpoint{0.989722in}{1.576317in}}%
\pgfpathlineto{\pgfqpoint{0.991449in}{1.613805in}}%
\pgfpathlineto{\pgfqpoint{0.992314in}{1.593413in}}%
\pgfpathlineto{\pgfqpoint{0.993178in}{1.616151in}}%
\pgfpathlineto{\pgfqpoint{0.994043in}{1.511611in}}%
\pgfpathlineto{\pgfqpoint{0.995771in}{1.625885in}}%
\pgfpathlineto{\pgfqpoint{0.996636in}{1.581303in}}%
\pgfpathlineto{\pgfqpoint{0.997501in}{1.613953in}}%
\pgfpathlineto{\pgfqpoint{0.999232in}{1.567530in}}%
\pgfpathlineto{\pgfqpoint{1.000096in}{1.630574in}}%
\pgfpathlineto{\pgfqpoint{1.000961in}{1.555003in}}%
\pgfpathlineto{\pgfqpoint{1.001826in}{1.614370in}}%
\pgfpathlineto{\pgfqpoint{1.002691in}{1.603505in}}%
\pgfpathlineto{\pgfqpoint{1.003553in}{1.587892in}}%
\pgfpathlineto{\pgfqpoint{1.004419in}{1.598340in}}%
\pgfpathlineto{\pgfqpoint{1.005284in}{1.487181in}}%
\pgfpathlineto{\pgfqpoint{1.007014in}{1.632534in}}%
\pgfpathlineto{\pgfqpoint{1.007877in}{1.605078in}}%
\pgfpathlineto{\pgfqpoint{1.008742in}{1.613061in}}%
\pgfpathlineto{\pgfqpoint{1.009607in}{1.592759in}}%
\pgfpathlineto{\pgfqpoint{1.010472in}{1.613537in}}%
\pgfpathlineto{\pgfqpoint{1.011338in}{1.536363in}}%
\pgfpathlineto{\pgfqpoint{1.013067in}{1.643161in}}%
\pgfpathlineto{\pgfqpoint{1.013933in}{1.617753in}}%
\pgfpathlineto{\pgfqpoint{1.014798in}{1.556342in}}%
\pgfpathlineto{\pgfqpoint{1.016526in}{1.631883in}}%
\pgfpathlineto{\pgfqpoint{1.019120in}{1.587122in}}%
\pgfpathlineto{\pgfqpoint{1.019986in}{1.548980in}}%
\pgfpathlineto{\pgfqpoint{1.020851in}{1.606889in}}%
\pgfpathlineto{\pgfqpoint{1.021716in}{1.569668in}}%
\pgfpathlineto{\pgfqpoint{1.022581in}{1.569966in}}%
\pgfpathlineto{\pgfqpoint{1.023445in}{1.627666in}}%
\pgfpathlineto{\pgfqpoint{1.025173in}{1.558982in}}%
\pgfpathlineto{\pgfqpoint{1.027767in}{1.720810in}}%
\pgfpathlineto{\pgfqpoint{1.028629in}{1.710778in}}%
\pgfpathlineto{\pgfqpoint{1.029494in}{1.611343in}}%
\pgfpathlineto{\pgfqpoint{1.030357in}{1.716121in}}%
\pgfpathlineto{\pgfqpoint{1.032087in}{1.678723in}}%
\pgfpathlineto{\pgfqpoint{1.033816in}{1.636988in}}%
\pgfpathlineto{\pgfqpoint{1.034679in}{1.638888in}}%
\pgfpathlineto{\pgfqpoint{1.035545in}{1.650701in}}%
\pgfpathlineto{\pgfqpoint{1.036409in}{1.686884in}}%
\pgfpathlineto{\pgfqpoint{1.037275in}{1.645656in}}%
\pgfpathlineto{\pgfqpoint{1.038139in}{1.703535in}}%
\pgfpathlineto{\pgfqpoint{1.039005in}{1.696946in}}%
\pgfpathlineto{\pgfqpoint{1.039870in}{1.722353in}}%
\pgfpathlineto{\pgfqpoint{1.040734in}{1.708224in}}%
\pgfpathlineto{\pgfqpoint{1.041600in}{1.726153in}}%
\pgfpathlineto{\pgfqpoint{1.042466in}{1.705554in}}%
\pgfpathlineto{\pgfqpoint{1.045060in}{1.762069in}}%
\pgfpathlineto{\pgfqpoint{1.045927in}{1.694867in}}%
\pgfpathlineto{\pgfqpoint{1.046792in}{1.720989in}}%
\pgfpathlineto{\pgfqpoint{1.048523in}{1.686468in}}%
\pgfpathlineto{\pgfqpoint{1.049390in}{1.691543in}}%
\pgfpathlineto{\pgfqpoint{1.050256in}{1.751799in}}%
\pgfpathlineto{\pgfqpoint{1.051987in}{1.667739in}}%
\pgfpathlineto{\pgfqpoint{1.053719in}{1.735174in}}%
\pgfpathlineto{\pgfqpoint{1.055449in}{1.663165in}}%
\pgfpathlineto{\pgfqpoint{1.056313in}{1.710064in}}%
\pgfpathlineto{\pgfqpoint{1.058907in}{1.644823in}}%
\pgfpathlineto{\pgfqpoint{1.061503in}{1.726034in}}%
\pgfpathlineto{\pgfqpoint{1.062369in}{1.679254in}}%
\pgfpathlineto{\pgfqpoint{1.063234in}{1.552036in}}%
\pgfpathlineto{\pgfqpoint{1.064099in}{1.678302in}}%
\pgfpathlineto{\pgfqpoint{1.065827in}{1.572814in}}%
\pgfpathlineto{\pgfqpoint{1.066692in}{1.577860in}}%
\pgfpathlineto{\pgfqpoint{1.068422in}{1.552333in}}%
\pgfpathlineto{\pgfqpoint{1.069287in}{1.575722in}}%
\pgfpathlineto{\pgfqpoint{1.070151in}{1.674859in}}%
\pgfpathlineto{\pgfqpoint{1.071880in}{1.591067in}}%
\pgfpathlineto{\pgfqpoint{1.072744in}{1.634374in}}%
\pgfpathlineto{\pgfqpoint{1.074470in}{1.546455in}}%
\pgfpathlineto{\pgfqpoint{1.075335in}{1.572635in}}%
\pgfpathlineto{\pgfqpoint{1.076199in}{1.557320in}}%
\pgfpathlineto{\pgfqpoint{1.077064in}{1.613864in}}%
\pgfpathlineto{\pgfqpoint{1.077927in}{1.562425in}}%
\pgfpathlineto{\pgfqpoint{1.078792in}{1.615615in}}%
\pgfpathlineto{\pgfqpoint{1.079657in}{1.524729in}}%
\pgfpathlineto{\pgfqpoint{1.081384in}{1.599705in}}%
\pgfpathlineto{\pgfqpoint{1.082248in}{1.551322in}}%
\pgfpathlineto{\pgfqpoint{1.083114in}{1.601010in}}%
\pgfpathlineto{\pgfqpoint{1.083981in}{1.599675in}}%
\pgfpathlineto{\pgfqpoint{1.085711in}{1.559930in}}%
\pgfpathlineto{\pgfqpoint{1.086576in}{1.628763in}}%
\pgfpathlineto{\pgfqpoint{1.087440in}{1.601188in}}%
\pgfpathlineto{\pgfqpoint{1.088306in}{1.630039in}}%
\pgfpathlineto{\pgfqpoint{1.089172in}{1.609558in}}%
\pgfpathlineto{\pgfqpoint{1.090037in}{1.618226in}}%
\pgfpathlineto{\pgfqpoint{1.090903in}{1.575722in}}%
\pgfpathlineto{\pgfqpoint{1.091767in}{1.589732in}}%
\pgfpathlineto{\pgfqpoint{1.092632in}{1.551798in}}%
\pgfpathlineto{\pgfqpoint{1.093498in}{1.595845in}}%
\pgfpathlineto{\pgfqpoint{1.094362in}{1.519798in}}%
\pgfpathlineto{\pgfqpoint{1.095225in}{1.597150in}}%
\pgfpathlineto{\pgfqpoint{1.096090in}{1.559841in}}%
\pgfpathlineto{\pgfqpoint{1.097822in}{1.642506in}}%
\pgfpathlineto{\pgfqpoint{1.098687in}{1.634255in}}%
\pgfpathlineto{\pgfqpoint{1.099551in}{1.637877in}}%
\pgfpathlineto{\pgfqpoint{1.100416in}{1.591275in}}%
\pgfpathlineto{\pgfqpoint{1.101281in}{1.597388in}}%
\pgfpathlineto{\pgfqpoint{1.102145in}{1.596619in}}%
\pgfpathlineto{\pgfqpoint{1.103008in}{1.532980in}}%
\pgfpathlineto{\pgfqpoint{1.104736in}{1.582013in}}%
\pgfpathlineto{\pgfqpoint{1.105601in}{1.573584in}}%
\pgfpathlineto{\pgfqpoint{1.106466in}{1.581124in}}%
\pgfpathlineto{\pgfqpoint{1.107328in}{1.545090in}}%
\pgfpathlineto{\pgfqpoint{1.109055in}{1.576967in}}%
\pgfpathlineto{\pgfqpoint{1.109918in}{1.565333in}}%
\pgfpathlineto{\pgfqpoint{1.111647in}{1.584151in}}%
\pgfpathlineto{\pgfqpoint{1.112512in}{1.529715in}}%
\pgfpathlineto{\pgfqpoint{1.113375in}{1.581184in}}%
\pgfpathlineto{\pgfqpoint{1.114239in}{1.565690in}}%
\pgfpathlineto{\pgfqpoint{1.115102in}{1.637609in}}%
\pgfpathlineto{\pgfqpoint{1.115967in}{1.555896in}}%
\pgfpathlineto{\pgfqpoint{1.116831in}{1.653847in}}%
\pgfpathlineto{\pgfqpoint{1.117696in}{1.549307in}}%
\pgfpathlineto{\pgfqpoint{1.118561in}{1.574773in}}%
\pgfpathlineto{\pgfqpoint{1.119426in}{1.569936in}}%
\pgfpathlineto{\pgfqpoint{1.120289in}{1.573230in}}%
\pgfpathlineto{\pgfqpoint{1.122016in}{1.695254in}}%
\pgfpathlineto{\pgfqpoint{1.122881in}{1.659369in}}%
\pgfpathlineto{\pgfqpoint{1.124612in}{1.682668in}}%
\pgfpathlineto{\pgfqpoint{1.125476in}{1.655688in}}%
\pgfpathlineto{\pgfqpoint{1.126342in}{1.720751in}}%
\pgfpathlineto{\pgfqpoint{1.127208in}{1.646723in}}%
\pgfpathlineto{\pgfqpoint{1.128072in}{1.738974in}}%
\pgfpathlineto{\pgfqpoint{1.128936in}{1.629715in}}%
\pgfpathlineto{\pgfqpoint{1.129800in}{1.650106in}}%
\pgfpathlineto{\pgfqpoint{1.130665in}{1.677057in}}%
\pgfpathlineto{\pgfqpoint{1.132396in}{1.617158in}}%
\pgfpathlineto{\pgfqpoint{1.133261in}{1.670051in}}%
\pgfpathlineto{\pgfqpoint{1.134126in}{1.658566in}}%
\pgfpathlineto{\pgfqpoint{1.134991in}{1.665184in}}%
\pgfpathlineto{\pgfqpoint{1.136722in}{1.604159in}}%
\pgfpathlineto{\pgfqpoint{1.137586in}{1.624164in}}%
\pgfpathlineto{\pgfqpoint{1.138451in}{1.660525in}}%
\pgfpathlineto{\pgfqpoint{1.139316in}{1.643518in}}%
\pgfpathlineto{\pgfqpoint{1.140181in}{1.663701in}}%
\pgfpathlineto{\pgfqpoint{1.141044in}{1.645771in}}%
\pgfpathlineto{\pgfqpoint{1.141910in}{1.687505in}}%
\pgfpathlineto{\pgfqpoint{1.142776in}{1.577979in}}%
\pgfpathlineto{\pgfqpoint{1.143641in}{1.596946in}}%
\pgfpathlineto{\pgfqpoint{1.147100in}{1.737550in}}%
\pgfpathlineto{\pgfqpoint{1.148829in}{1.670825in}}%
\pgfpathlineto{\pgfqpoint{1.149692in}{1.703416in}}%
\pgfpathlineto{\pgfqpoint{1.150558in}{1.664087in}}%
\pgfpathlineto{\pgfqpoint{1.151424in}{1.682578in}}%
\pgfpathlineto{\pgfqpoint{1.152290in}{1.680738in}}%
\pgfpathlineto{\pgfqpoint{1.153155in}{1.634166in}}%
\pgfpathlineto{\pgfqpoint{1.155749in}{1.684419in}}%
\pgfpathlineto{\pgfqpoint{1.156614in}{1.665779in}}%
\pgfpathlineto{\pgfqpoint{1.157478in}{1.675960in}}%
\pgfpathlineto{\pgfqpoint{1.158343in}{1.653669in}}%
\pgfpathlineto{\pgfqpoint{1.159208in}{1.699556in}}%
\pgfpathlineto{\pgfqpoint{1.160073in}{1.631496in}}%
\pgfpathlineto{\pgfqpoint{1.160937in}{1.769665in}}%
\pgfpathlineto{\pgfqpoint{1.164394in}{1.589673in}}%
\pgfpathlineto{\pgfqpoint{1.165259in}{1.588070in}}%
\pgfpathlineto{\pgfqpoint{1.166124in}{1.592878in}}%
\pgfpathlineto{\pgfqpoint{1.166990in}{1.642774in}}%
\pgfpathlineto{\pgfqpoint{1.170449in}{1.526569in}}%
\pgfpathlineto{\pgfqpoint{1.172180in}{1.625528in}}%
\pgfpathlineto{\pgfqpoint{1.173045in}{1.610094in}}%
\pgfpathlineto{\pgfqpoint{1.173912in}{1.617277in}}%
\pgfpathlineto{\pgfqpoint{1.174778in}{1.558863in}}%
\pgfpathlineto{\pgfqpoint{1.176508in}{1.593473in}}%
\pgfpathlineto{\pgfqpoint{1.177374in}{1.607305in}}%
\pgfpathlineto{\pgfqpoint{1.178239in}{1.560882in}}%
\pgfpathlineto{\pgfqpoint{1.179969in}{1.589851in}}%
\pgfpathlineto{\pgfqpoint{1.180834in}{1.564325in}}%
\pgfpathlineto{\pgfqpoint{1.181699in}{1.618940in}}%
\pgfpathlineto{\pgfqpoint{1.182565in}{1.587122in}}%
\pgfpathlineto{\pgfqpoint{1.183430in}{1.608670in}}%
\pgfpathlineto{\pgfqpoint{1.186026in}{1.552482in}}%
\pgfpathlineto{\pgfqpoint{1.187758in}{1.605286in}}%
\pgfpathlineto{\pgfqpoint{1.189487in}{1.569192in}}%
\pgfpathlineto{\pgfqpoint{1.190352in}{1.610510in}}%
\pgfpathlineto{\pgfqpoint{1.191218in}{1.553906in}}%
\pgfpathlineto{\pgfqpoint{1.192948in}{1.677473in}}%
\pgfpathlineto{\pgfqpoint{1.193813in}{1.611046in}}%
\pgfpathlineto{\pgfqpoint{1.194678in}{1.669044in}}%
\pgfpathlineto{\pgfqpoint{1.195541in}{1.542272in}}%
\pgfpathlineto{\pgfqpoint{1.196406in}{1.629626in}}%
\pgfpathlineto{\pgfqpoint{1.198137in}{1.572933in}}%
\pgfpathlineto{\pgfqpoint{1.199001in}{1.648385in}}%
\pgfpathlineto{\pgfqpoint{1.199865in}{1.537315in}}%
\pgfpathlineto{\pgfqpoint{1.200729in}{1.593503in}}%
\pgfpathlineto{\pgfqpoint{1.201595in}{1.495819in}}%
\pgfpathlineto{\pgfqpoint{1.202459in}{1.610867in}}%
\pgfpathlineto{\pgfqpoint{1.203324in}{1.501400in}}%
\pgfpathlineto{\pgfqpoint{1.204189in}{1.568839in}}%
\pgfpathlineto{\pgfqpoint{1.205054in}{1.556996in}}%
\pgfpathlineto{\pgfqpoint{1.206785in}{1.599173in}}%
\pgfpathlineto{\pgfqpoint{1.209379in}{1.578038in}}%
\pgfpathlineto{\pgfqpoint{1.210246in}{1.499559in}}%
\pgfpathlineto{\pgfqpoint{1.211111in}{1.597094in}}%
\pgfpathlineto{\pgfqpoint{1.211977in}{1.579284in}}%
\pgfpathlineto{\pgfqpoint{1.212842in}{1.588130in}}%
\pgfpathlineto{\pgfqpoint{1.213706in}{1.620959in}}%
\pgfpathlineto{\pgfqpoint{1.214572in}{1.593205in}}%
\pgfpathlineto{\pgfqpoint{1.215438in}{1.665422in}}%
\pgfpathlineto{\pgfqpoint{1.218035in}{1.584746in}}%
\pgfpathlineto{\pgfqpoint{1.218902in}{1.684062in}}%
\pgfpathlineto{\pgfqpoint{1.219768in}{1.600181in}}%
\pgfpathlineto{\pgfqpoint{1.220634in}{1.612886in}}%
\pgfpathlineto{\pgfqpoint{1.222366in}{1.609146in}}%
\pgfpathlineto{\pgfqpoint{1.224098in}{1.615943in}}%
\pgfpathlineto{\pgfqpoint{1.224965in}{1.571271in}}%
\pgfpathlineto{\pgfqpoint{1.225831in}{1.607959in}}%
\pgfpathlineto{\pgfqpoint{1.227562in}{1.566701in}}%
\pgfpathlineto{\pgfqpoint{1.228427in}{1.595194in}}%
\pgfpathlineto{\pgfqpoint{1.229292in}{1.563288in}}%
\pgfpathlineto{\pgfqpoint{1.230156in}{1.585281in}}%
\pgfpathlineto{\pgfqpoint{1.231020in}{1.574654in}}%
\pgfpathlineto{\pgfqpoint{1.232751in}{1.587003in}}%
\pgfpathlineto{\pgfqpoint{1.233617in}{1.566552in}}%
\pgfpathlineto{\pgfqpoint{1.236211in}{1.618642in}}%
\pgfpathlineto{\pgfqpoint{1.237942in}{1.538144in}}%
\pgfpathlineto{\pgfqpoint{1.238807in}{1.618702in}}%
\pgfpathlineto{\pgfqpoint{1.239671in}{1.608432in}}%
\pgfpathlineto{\pgfqpoint{1.240537in}{1.636334in}}%
\pgfpathlineto{\pgfqpoint{1.241402in}{1.548950in}}%
\pgfpathlineto{\pgfqpoint{1.242268in}{1.557558in}}%
\pgfpathlineto{\pgfqpoint{1.243133in}{1.555539in}}%
\pgfpathlineto{\pgfqpoint{1.244864in}{1.590208in}}%
\pgfpathlineto{\pgfqpoint{1.245728in}{1.550880in}}%
\pgfpathlineto{\pgfqpoint{1.246593in}{1.602021in}}%
\pgfpathlineto{\pgfqpoint{1.247458in}{1.548771in}}%
\pgfpathlineto{\pgfqpoint{1.249187in}{1.609737in}}%
\pgfpathlineto{\pgfqpoint{1.250053in}{1.606532in}}%
\pgfpathlineto{\pgfqpoint{1.250916in}{1.563076in}}%
\pgfpathlineto{\pgfqpoint{1.251781in}{1.611815in}}%
\pgfpathlineto{\pgfqpoint{1.252647in}{1.605286in}}%
\pgfpathlineto{\pgfqpoint{1.253512in}{1.626064in}}%
\pgfpathlineto{\pgfqpoint{1.255243in}{1.576198in}}%
\pgfpathlineto{\pgfqpoint{1.256974in}{1.600002in}}%
\pgfpathlineto{\pgfqpoint{1.257840in}{1.595934in}}%
\pgfpathlineto{\pgfqpoint{1.258706in}{1.567114in}}%
\pgfpathlineto{\pgfqpoint{1.259571in}{1.619416in}}%
\pgfpathlineto{\pgfqpoint{1.261299in}{1.575841in}}%
\pgfpathlineto{\pgfqpoint{1.262165in}{1.584538in}}%
\pgfpathlineto{\pgfqpoint{1.263030in}{1.581124in}}%
\pgfpathlineto{\pgfqpoint{1.263893in}{1.550612in}}%
\pgfpathlineto{\pgfqpoint{1.264759in}{1.572784in}}%
\pgfpathlineto{\pgfqpoint{1.265625in}{1.632474in}}%
\pgfpathlineto{\pgfqpoint{1.267353in}{1.551679in}}%
\pgfpathlineto{\pgfqpoint{1.268216in}{1.559160in}}%
\pgfpathlineto{\pgfqpoint{1.269081in}{1.598102in}}%
\pgfpathlineto{\pgfqpoint{1.269946in}{1.596262in}}%
\pgfpathlineto{\pgfqpoint{1.270811in}{1.610034in}}%
\pgfpathlineto{\pgfqpoint{1.272543in}{1.531496in}}%
\pgfpathlineto{\pgfqpoint{1.273408in}{1.585162in}}%
\pgfpathlineto{\pgfqpoint{1.274272in}{1.572903in}}%
\pgfpathlineto{\pgfqpoint{1.275137in}{1.591870in}}%
\pgfpathlineto{\pgfqpoint{1.276003in}{1.576614in}}%
\pgfpathlineto{\pgfqpoint{1.276869in}{1.533247in}}%
\pgfpathlineto{\pgfqpoint{1.277734in}{1.743841in}}%
\pgfpathlineto{\pgfqpoint{1.278600in}{1.564028in}}%
\pgfpathlineto{\pgfqpoint{1.279465in}{1.601843in}}%
\pgfpathlineto{\pgfqpoint{1.280331in}{1.626123in}}%
\pgfpathlineto{\pgfqpoint{1.282062in}{1.598043in}}%
\pgfpathlineto{\pgfqpoint{1.282927in}{1.678897in}}%
\pgfpathlineto{\pgfqpoint{1.283793in}{1.586587in}}%
\pgfpathlineto{\pgfqpoint{1.284658in}{1.650757in}}%
\pgfpathlineto{\pgfqpoint{1.285521in}{1.558684in}}%
\pgfpathlineto{\pgfqpoint{1.287250in}{1.614072in}}%
\pgfpathlineto{\pgfqpoint{1.288114in}{1.622383in}}%
\pgfpathlineto{\pgfqpoint{1.290711in}{1.713686in}}%
\pgfpathlineto{\pgfqpoint{1.291576in}{1.660793in}}%
\pgfpathlineto{\pgfqpoint{1.292440in}{1.752985in}}%
\pgfpathlineto{\pgfqpoint{1.293306in}{1.742923in}}%
\pgfpathlineto{\pgfqpoint{1.294172in}{1.621316in}}%
\pgfpathlineto{\pgfqpoint{1.295904in}{1.697835in}}%
\pgfpathlineto{\pgfqpoint{1.296768in}{1.682995in}}%
\pgfpathlineto{\pgfqpoint{1.297633in}{1.572104in}}%
\pgfpathlineto{\pgfqpoint{1.298498in}{1.591573in}}%
\pgfpathlineto{\pgfqpoint{1.299363in}{1.642863in}}%
\pgfpathlineto{\pgfqpoint{1.301092in}{1.619118in}}%
\pgfpathlineto{\pgfqpoint{1.301958in}{1.625885in}}%
\pgfpathlineto{\pgfqpoint{1.305417in}{1.706472in}}%
\pgfpathlineto{\pgfqpoint{1.306282in}{1.703475in}}%
\pgfpathlineto{\pgfqpoint{1.307147in}{1.629983in}}%
\pgfpathlineto{\pgfqpoint{1.308879in}{1.746098in}}%
\pgfpathlineto{\pgfqpoint{1.309745in}{1.713597in}}%
\pgfpathlineto{\pgfqpoint{1.310610in}{1.739034in}}%
\pgfpathlineto{\pgfqpoint{1.312342in}{1.708343in}}%
\pgfpathlineto{\pgfqpoint{1.313208in}{1.716296in}}%
\pgfpathlineto{\pgfqpoint{1.314074in}{1.662039in}}%
\pgfpathlineto{\pgfqpoint{1.314939in}{1.666698in}}%
\pgfpathlineto{\pgfqpoint{1.315803in}{1.731433in}}%
\pgfpathlineto{\pgfqpoint{1.317529in}{1.640312in}}%
\pgfpathlineto{\pgfqpoint{1.318395in}{1.672368in}}%
\pgfpathlineto{\pgfqpoint{1.320124in}{1.621966in}}%
\pgfpathlineto{\pgfqpoint{1.320989in}{1.730188in}}%
\pgfpathlineto{\pgfqpoint{1.321854in}{1.709975in}}%
\pgfpathlineto{\pgfqpoint{1.322720in}{1.596678in}}%
\pgfpathlineto{\pgfqpoint{1.325313in}{1.718434in}}%
\pgfpathlineto{\pgfqpoint{1.326179in}{1.629329in}}%
\pgfpathlineto{\pgfqpoint{1.327908in}{1.685843in}}%
\pgfpathlineto{\pgfqpoint{1.328771in}{1.683556in}}%
\pgfpathlineto{\pgfqpoint{1.329636in}{1.685843in}}%
\pgfpathlineto{\pgfqpoint{1.330500in}{1.728466in}}%
\pgfpathlineto{\pgfqpoint{1.331365in}{1.699051in}}%
\pgfpathlineto{\pgfqpoint{1.332230in}{1.726507in}}%
\pgfpathlineto{\pgfqpoint{1.333954in}{1.694659in}}%
\pgfpathlineto{\pgfqpoint{1.335684in}{1.699199in}}%
\pgfpathlineto{\pgfqpoint{1.336550in}{1.646187in}}%
\pgfpathlineto{\pgfqpoint{1.337413in}{1.728347in}}%
\pgfpathlineto{\pgfqpoint{1.338278in}{1.660049in}}%
\pgfpathlineto{\pgfqpoint{1.340007in}{1.704781in}}%
\pgfpathlineto{\pgfqpoint{1.342599in}{1.546812in}}%
\pgfpathlineto{\pgfqpoint{1.343462in}{1.584538in}}%
\pgfpathlineto{\pgfqpoint{1.344329in}{1.555658in}}%
\pgfpathlineto{\pgfqpoint{1.345195in}{1.573171in}}%
\pgfpathlineto{\pgfqpoint{1.346061in}{1.543904in}}%
\pgfpathlineto{\pgfqpoint{1.346925in}{1.567828in}}%
\pgfpathlineto{\pgfqpoint{1.347786in}{1.548920in}}%
\pgfpathlineto{\pgfqpoint{1.348652in}{1.630872in}}%
\pgfpathlineto{\pgfqpoint{1.349515in}{1.562544in}}%
\pgfpathlineto{\pgfqpoint{1.351245in}{1.600121in}}%
\pgfpathlineto{\pgfqpoint{1.352107in}{1.545150in}}%
\pgfpathlineto{\pgfqpoint{1.352972in}{1.548474in}}%
\pgfpathlineto{\pgfqpoint{1.353837in}{1.554352in}}%
\pgfpathlineto{\pgfqpoint{1.354702in}{1.620483in}}%
\pgfpathlineto{\pgfqpoint{1.355566in}{1.578098in}}%
\pgfpathlineto{\pgfqpoint{1.356431in}{1.624878in}}%
\pgfpathlineto{\pgfqpoint{1.359028in}{1.559398in}}%
\pgfpathlineto{\pgfqpoint{1.359893in}{1.610778in}}%
\pgfpathlineto{\pgfqpoint{1.360758in}{1.581779in}}%
\pgfpathlineto{\pgfqpoint{1.361622in}{1.627488in}}%
\pgfpathlineto{\pgfqpoint{1.362488in}{1.572308in}}%
\pgfpathlineto{\pgfqpoint{1.365081in}{1.640904in}}%
\pgfpathlineto{\pgfqpoint{1.365947in}{1.616623in}}%
\pgfpathlineto{\pgfqpoint{1.366813in}{1.644049in}}%
\pgfpathlineto{\pgfqpoint{1.367678in}{1.567292in}}%
\pgfpathlineto{\pgfqpoint{1.368543in}{1.571211in}}%
\pgfpathlineto{\pgfqpoint{1.369408in}{1.577384in}}%
\pgfpathlineto{\pgfqpoint{1.370273in}{1.562603in}}%
\pgfpathlineto{\pgfqpoint{1.371138in}{1.595075in}}%
\pgfpathlineto{\pgfqpoint{1.372002in}{1.586914in}}%
\pgfpathlineto{\pgfqpoint{1.372866in}{1.555836in}}%
\pgfpathlineto{\pgfqpoint{1.374594in}{1.644287in}}%
\pgfpathlineto{\pgfqpoint{1.375459in}{1.558744in}}%
\pgfpathlineto{\pgfqpoint{1.377187in}{1.615794in}}%
\pgfpathlineto{\pgfqpoint{1.378052in}{1.586468in}}%
\pgfpathlineto{\pgfqpoint{1.378916in}{1.511075in}}%
\pgfpathlineto{\pgfqpoint{1.380646in}{1.578395in}}%
\pgfpathlineto{\pgfqpoint{1.381509in}{1.562782in}}%
\pgfpathlineto{\pgfqpoint{1.383238in}{1.580946in}}%
\pgfpathlineto{\pgfqpoint{1.384104in}{1.522650in}}%
\pgfpathlineto{\pgfqpoint{1.384968in}{1.536601in}}%
\pgfpathlineto{\pgfqpoint{1.387562in}{1.629031in}}%
\pgfpathlineto{\pgfqpoint{1.388427in}{1.628912in}}%
\pgfpathlineto{\pgfqpoint{1.390156in}{1.572992in}}%
\pgfpathlineto{\pgfqpoint{1.391021in}{1.565184in}}%
\pgfpathlineto{\pgfqpoint{1.391886in}{1.519564in}}%
\pgfpathlineto{\pgfqpoint{1.392749in}{1.591751in}}%
\pgfpathlineto{\pgfqpoint{1.393613in}{1.589732in}}%
\pgfpathlineto{\pgfqpoint{1.395342in}{1.589821in}}%
\pgfpathlineto{\pgfqpoint{1.396208in}{1.603743in}}%
\pgfpathlineto{\pgfqpoint{1.397073in}{1.701456in}}%
\pgfpathlineto{\pgfqpoint{1.397938in}{1.679373in}}%
\pgfpathlineto{\pgfqpoint{1.398803in}{1.709588in}}%
\pgfpathlineto{\pgfqpoint{1.400534in}{1.636036in}}%
\pgfpathlineto{\pgfqpoint{1.401400in}{1.703118in}}%
\pgfpathlineto{\pgfqpoint{1.402265in}{1.662039in}}%
\pgfpathlineto{\pgfqpoint{1.403131in}{1.725499in}}%
\pgfpathlineto{\pgfqpoint{1.403997in}{1.699021in}}%
\pgfpathlineto{\pgfqpoint{1.405729in}{1.722885in}}%
\pgfpathlineto{\pgfqpoint{1.407460in}{1.677741in}}%
\pgfpathlineto{\pgfqpoint{1.408324in}{1.732564in}}%
\pgfpathlineto{\pgfqpoint{1.410055in}{1.677295in}}%
\pgfpathlineto{\pgfqpoint{1.410921in}{1.725677in}}%
\pgfpathlineto{\pgfqpoint{1.411785in}{1.691067in}}%
\pgfpathlineto{\pgfqpoint{1.413515in}{1.736007in}}%
\pgfpathlineto{\pgfqpoint{1.414379in}{1.735144in}}%
\pgfpathlineto{\pgfqpoint{1.416108in}{1.693205in}}%
\pgfpathlineto{\pgfqpoint{1.416972in}{1.708819in}}%
\pgfpathlineto{\pgfqpoint{1.417838in}{1.653788in}}%
\pgfpathlineto{\pgfqpoint{1.418701in}{1.725856in}}%
\pgfpathlineto{\pgfqpoint{1.419566in}{1.643339in}}%
\pgfpathlineto{\pgfqpoint{1.420430in}{1.684062in}}%
\pgfpathlineto{\pgfqpoint{1.421294in}{1.639123in}}%
\pgfpathlineto{\pgfqpoint{1.422159in}{1.661384in}}%
\pgfpathlineto{\pgfqpoint{1.423024in}{1.657290in}}%
\pgfpathlineto{\pgfqpoint{1.424749in}{1.677652in}}%
\pgfpathlineto{\pgfqpoint{1.425614in}{1.627607in}}%
\pgfpathlineto{\pgfqpoint{1.427345in}{1.761295in}}%
\pgfpathlineto{\pgfqpoint{1.429937in}{1.596619in}}%
\pgfpathlineto{\pgfqpoint{1.431667in}{1.680738in}}%
\pgfpathlineto{\pgfqpoint{1.432532in}{1.670408in}}%
\pgfpathlineto{\pgfqpoint{1.433398in}{1.694659in}}%
\pgfpathlineto{\pgfqpoint{1.434265in}{1.644763in}}%
\pgfpathlineto{\pgfqpoint{1.435130in}{1.656695in}}%
\pgfpathlineto{\pgfqpoint{1.435996in}{1.673584in}}%
\pgfpathlineto{\pgfqpoint{1.436861in}{1.668746in}}%
\pgfpathlineto{\pgfqpoint{1.438592in}{1.680678in}}%
\pgfpathlineto{\pgfqpoint{1.439455in}{1.660079in}}%
\pgfpathlineto{\pgfqpoint{1.441185in}{1.691008in}}%
\pgfpathlineto{\pgfqpoint{1.442050in}{1.681035in}}%
\pgfpathlineto{\pgfqpoint{1.442915in}{1.610718in}}%
\pgfpathlineto{\pgfqpoint{1.443778in}{1.717308in}}%
\pgfpathlineto{\pgfqpoint{1.444643in}{1.674803in}}%
\pgfpathlineto{\pgfqpoint{1.445508in}{1.698578in}}%
\pgfpathlineto{\pgfqpoint{1.446371in}{1.696768in}}%
\pgfpathlineto{\pgfqpoint{1.448967in}{1.628499in}}%
\pgfpathlineto{\pgfqpoint{1.449831in}{1.705911in}}%
\pgfpathlineto{\pgfqpoint{1.450696in}{1.642629in}}%
\pgfpathlineto{\pgfqpoint{1.451559in}{1.697065in}}%
\pgfpathlineto{\pgfqpoint{1.452425in}{1.659696in}}%
\pgfpathlineto{\pgfqpoint{1.453289in}{1.662990in}}%
\pgfpathlineto{\pgfqpoint{1.455884in}{1.717724in}}%
\pgfpathlineto{\pgfqpoint{1.456749in}{1.695105in}}%
\pgfpathlineto{\pgfqpoint{1.457614in}{1.635237in}}%
\pgfpathlineto{\pgfqpoint{1.458480in}{1.651653in}}%
\pgfpathlineto{\pgfqpoint{1.459345in}{1.688933in}}%
\pgfpathlineto{\pgfqpoint{1.461941in}{1.621405in}}%
\pgfpathlineto{\pgfqpoint{1.462806in}{1.621435in}}%
\pgfpathlineto{\pgfqpoint{1.464538in}{1.688665in}}%
\pgfpathlineto{\pgfqpoint{1.467134in}{1.621583in}}%
\pgfpathlineto{\pgfqpoint{1.467999in}{1.679611in}}%
\pgfpathlineto{\pgfqpoint{1.468864in}{1.666017in}}%
\pgfpathlineto{\pgfqpoint{1.469729in}{1.655866in}}%
\pgfpathlineto{\pgfqpoint{1.470594in}{1.676942in}}%
\pgfpathlineto{\pgfqpoint{1.472326in}{1.637940in}}%
\pgfpathlineto{\pgfqpoint{1.474056in}{1.664801in}}%
\pgfpathlineto{\pgfqpoint{1.474922in}{1.633843in}}%
\pgfpathlineto{\pgfqpoint{1.475788in}{1.705851in}}%
\pgfpathlineto{\pgfqpoint{1.476654in}{1.692346in}}%
\pgfpathlineto{\pgfqpoint{1.477520in}{1.665009in}}%
\pgfpathlineto{\pgfqpoint{1.478386in}{1.711905in}}%
\pgfpathlineto{\pgfqpoint{1.479253in}{1.685073in}}%
\pgfpathlineto{\pgfqpoint{1.480117in}{1.743310in}}%
\pgfpathlineto{\pgfqpoint{1.481847in}{1.638769in}}%
\pgfpathlineto{\pgfqpoint{1.482711in}{1.696887in}}%
\pgfpathlineto{\pgfqpoint{1.483575in}{1.629983in}}%
\pgfpathlineto{\pgfqpoint{1.484441in}{1.638948in}}%
\pgfpathlineto{\pgfqpoint{1.485307in}{1.652780in}}%
\pgfpathlineto{\pgfqpoint{1.487905in}{1.591930in}}%
\pgfpathlineto{\pgfqpoint{1.488770in}{1.530131in}}%
\pgfpathlineto{\pgfqpoint{1.489635in}{1.625885in}}%
\pgfpathlineto{\pgfqpoint{1.490500in}{1.584835in}}%
\pgfpathlineto{\pgfqpoint{1.491365in}{1.585400in}}%
\pgfpathlineto{\pgfqpoint{1.492231in}{1.588725in}}%
\pgfpathlineto{\pgfqpoint{1.493097in}{1.558655in}}%
\pgfpathlineto{\pgfqpoint{1.493962in}{1.569668in}}%
\pgfpathlineto{\pgfqpoint{1.495693in}{1.537285in}}%
\pgfpathlineto{\pgfqpoint{1.497424in}{1.626064in}}%
\pgfpathlineto{\pgfqpoint{1.499153in}{1.588963in}}%
\pgfpathlineto{\pgfqpoint{1.500017in}{1.640818in}}%
\pgfpathlineto{\pgfqpoint{1.500882in}{1.518675in}}%
\pgfpathlineto{\pgfqpoint{1.502609in}{1.582995in}}%
\pgfpathlineto{\pgfqpoint{1.503473in}{1.576495in}}%
\pgfpathlineto{\pgfqpoint{1.504339in}{1.578514in}}%
\pgfpathlineto{\pgfqpoint{1.505204in}{1.566552in}}%
\pgfpathlineto{\pgfqpoint{1.506068in}{1.599292in}}%
\pgfpathlineto{\pgfqpoint{1.506933in}{1.547645in}}%
\pgfpathlineto{\pgfqpoint{1.507798in}{1.560350in}}%
\pgfpathlineto{\pgfqpoint{1.508662in}{1.660793in}}%
\pgfpathlineto{\pgfqpoint{1.509524in}{1.541026in}}%
\pgfpathlineto{\pgfqpoint{1.511254in}{1.632240in}}%
\pgfpathlineto{\pgfqpoint{1.512119in}{1.625592in}}%
\pgfpathlineto{\pgfqpoint{1.512984in}{1.610335in}}%
\pgfpathlineto{\pgfqpoint{1.513848in}{1.675279in}}%
\pgfpathlineto{\pgfqpoint{1.514713in}{1.557145in}}%
\pgfpathlineto{\pgfqpoint{1.515579in}{1.565158in}}%
\pgfpathlineto{\pgfqpoint{1.516443in}{1.587360in}}%
\pgfpathlineto{\pgfqpoint{1.519904in}{1.533753in}}%
\pgfpathlineto{\pgfqpoint{1.520767in}{1.586051in}}%
\pgfpathlineto{\pgfqpoint{1.521632in}{1.545298in}}%
\pgfpathlineto{\pgfqpoint{1.522496in}{1.549009in}}%
\pgfpathlineto{\pgfqpoint{1.523360in}{1.543190in}}%
\pgfpathlineto{\pgfqpoint{1.525956in}{1.640312in}}%
\pgfpathlineto{\pgfqpoint{1.526821in}{1.559458in}}%
\pgfpathlineto{\pgfqpoint{1.527685in}{1.579879in}}%
\pgfpathlineto{\pgfqpoint{1.528549in}{1.567828in}}%
\pgfpathlineto{\pgfqpoint{1.529411in}{1.577562in}}%
\pgfpathlineto{\pgfqpoint{1.532005in}{1.691246in}}%
\pgfpathlineto{\pgfqpoint{1.532870in}{1.615318in}}%
\pgfpathlineto{\pgfqpoint{1.533734in}{1.664798in}}%
\pgfpathlineto{\pgfqpoint{1.534599in}{1.581481in}}%
\pgfpathlineto{\pgfqpoint{1.536326in}{1.641320in}}%
\pgfpathlineto{\pgfqpoint{1.537190in}{1.610689in}}%
\pgfpathlineto{\pgfqpoint{1.538055in}{1.509294in}}%
\pgfpathlineto{\pgfqpoint{1.539786in}{1.627726in}}%
\pgfpathlineto{\pgfqpoint{1.540652in}{1.522710in}}%
\pgfpathlineto{\pgfqpoint{1.542382in}{1.608670in}}%
\pgfpathlineto{\pgfqpoint{1.543246in}{1.514429in}}%
\pgfpathlineto{\pgfqpoint{1.544111in}{1.623688in}}%
\pgfpathlineto{\pgfqpoint{1.544975in}{1.603029in}}%
\pgfpathlineto{\pgfqpoint{1.545841in}{1.584389in}}%
\pgfpathlineto{\pgfqpoint{1.546705in}{1.612053in}}%
\pgfpathlineto{\pgfqpoint{1.548435in}{1.560525in}}%
\pgfpathlineto{\pgfqpoint{1.549300in}{1.577860in}}%
\pgfpathlineto{\pgfqpoint{1.550163in}{1.543190in}}%
\pgfpathlineto{\pgfqpoint{1.551028in}{1.586527in}}%
\pgfpathlineto{\pgfqpoint{1.552758in}{1.567976in}}%
\pgfpathlineto{\pgfqpoint{1.553624in}{1.557855in}}%
\pgfpathlineto{\pgfqpoint{1.554488in}{1.583500in}}%
\pgfpathlineto{\pgfqpoint{1.555353in}{1.576019in}}%
\pgfpathlineto{\pgfqpoint{1.556218in}{1.646247in}}%
\pgfpathlineto{\pgfqpoint{1.557083in}{1.555509in}}%
\pgfpathlineto{\pgfqpoint{1.557945in}{1.591751in}}%
\pgfpathlineto{\pgfqpoint{1.559676in}{1.532890in}}%
\pgfpathlineto{\pgfqpoint{1.560542in}{1.608253in}}%
\pgfpathlineto{\pgfqpoint{1.561408in}{1.554055in}}%
\pgfpathlineto{\pgfqpoint{1.562273in}{1.575900in}}%
\pgfpathlineto{\pgfqpoint{1.563139in}{1.567292in}}%
\pgfpathlineto{\pgfqpoint{1.564005in}{1.639123in}}%
\pgfpathlineto{\pgfqpoint{1.565738in}{1.545150in}}%
\pgfpathlineto{\pgfqpoint{1.566603in}{1.562544in}}%
\pgfpathlineto{\pgfqpoint{1.567467in}{1.541707in}}%
\pgfpathlineto{\pgfqpoint{1.568332in}{1.547641in}}%
\pgfpathlineto{\pgfqpoint{1.569196in}{1.612763in}}%
\pgfpathlineto{\pgfqpoint{1.570061in}{1.571981in}}%
\pgfpathlineto{\pgfqpoint{1.572656in}{1.644525in}}%
\pgfpathlineto{\pgfqpoint{1.574383in}{1.536601in}}%
\pgfpathlineto{\pgfqpoint{1.575249in}{1.632772in}}%
\pgfpathlineto{\pgfqpoint{1.576114in}{1.569728in}}%
\pgfpathlineto{\pgfqpoint{1.576978in}{1.598459in}}%
\pgfpathlineto{\pgfqpoint{1.578706in}{1.552304in}}%
\pgfpathlineto{\pgfqpoint{1.579570in}{1.677354in}}%
\pgfpathlineto{\pgfqpoint{1.580437in}{1.575781in}}%
\pgfpathlineto{\pgfqpoint{1.581303in}{1.658060in}}%
\pgfpathlineto{\pgfqpoint{1.582170in}{1.655271in}}%
\pgfpathlineto{\pgfqpoint{1.583903in}{1.571271in}}%
\pgfpathlineto{\pgfqpoint{1.585634in}{1.555955in}}%
\pgfpathlineto{\pgfqpoint{1.586500in}{1.592227in}}%
\pgfpathlineto{\pgfqpoint{1.587366in}{1.545507in}}%
\pgfpathlineto{\pgfqpoint{1.589098in}{1.632772in}}%
\pgfpathlineto{\pgfqpoint{1.590828in}{1.576079in}}%
\pgfpathlineto{\pgfqpoint{1.591692in}{1.610927in}}%
\pgfpathlineto{\pgfqpoint{1.593423in}{1.541528in}}%
\pgfpathlineto{\pgfqpoint{1.594289in}{1.543309in}}%
\pgfpathlineto{\pgfqpoint{1.595155in}{1.572397in}}%
\pgfpathlineto{\pgfqpoint{1.596886in}{1.526867in}}%
\pgfpathlineto{\pgfqpoint{1.598615in}{1.637639in}}%
\pgfpathlineto{\pgfqpoint{1.599480in}{1.527160in}}%
\pgfpathlineto{\pgfqpoint{1.601207in}{1.591275in}}%
\pgfpathlineto{\pgfqpoint{1.602071in}{1.587892in}}%
\pgfpathlineto{\pgfqpoint{1.603803in}{1.525677in}}%
\pgfpathlineto{\pgfqpoint{1.604669in}{1.598072in}}%
\pgfpathlineto{\pgfqpoint{1.605535in}{1.575424in}}%
\pgfpathlineto{\pgfqpoint{1.608130in}{1.627369in}}%
\pgfpathlineto{\pgfqpoint{1.609860in}{1.517396in}}%
\pgfpathlineto{\pgfqpoint{1.610725in}{1.581243in}}%
\pgfpathlineto{\pgfqpoint{1.611590in}{1.564266in}}%
\pgfpathlineto{\pgfqpoint{1.614186in}{1.622944in}}%
\pgfpathlineto{\pgfqpoint{1.615051in}{1.556487in}}%
\pgfpathlineto{\pgfqpoint{1.615917in}{1.644049in}}%
\pgfpathlineto{\pgfqpoint{1.616782in}{1.597596in}}%
\pgfpathlineto{\pgfqpoint{1.617645in}{1.662987in}}%
\pgfpathlineto{\pgfqpoint{1.618510in}{1.611637in}}%
\pgfpathlineto{\pgfqpoint{1.619375in}{1.628704in}}%
\pgfpathlineto{\pgfqpoint{1.620240in}{1.622026in}}%
\pgfpathlineto{\pgfqpoint{1.621104in}{1.632415in}}%
\pgfpathlineto{\pgfqpoint{1.621969in}{1.581184in}}%
\pgfpathlineto{\pgfqpoint{1.622835in}{1.638587in}}%
\pgfpathlineto{\pgfqpoint{1.623700in}{1.544376in}}%
\pgfpathlineto{\pgfqpoint{1.625428in}{1.634196in}}%
\pgfpathlineto{\pgfqpoint{1.626293in}{1.584924in}}%
\pgfpathlineto{\pgfqpoint{1.628886in}{1.657971in}}%
\pgfpathlineto{\pgfqpoint{1.629752in}{1.593711in}}%
\pgfpathlineto{\pgfqpoint{1.630617in}{1.641320in}}%
\pgfpathlineto{\pgfqpoint{1.631481in}{1.631110in}}%
\pgfpathlineto{\pgfqpoint{1.632344in}{1.589970in}}%
\pgfpathlineto{\pgfqpoint{1.633210in}{1.608759in}}%
\pgfpathlineto{\pgfqpoint{1.634073in}{1.607126in}}%
\pgfpathlineto{\pgfqpoint{1.634936in}{1.615556in}}%
\pgfpathlineto{\pgfqpoint{1.635803in}{1.614281in}}%
\pgfpathlineto{\pgfqpoint{1.636668in}{1.590327in}}%
\pgfpathlineto{\pgfqpoint{1.639263in}{1.668627in}}%
\pgfpathlineto{\pgfqpoint{1.642725in}{1.562931in}}%
\pgfpathlineto{\pgfqpoint{1.644456in}{1.626123in}}%
\pgfpathlineto{\pgfqpoint{1.645321in}{1.586319in}}%
\pgfpathlineto{\pgfqpoint{1.647053in}{1.657822in}}%
\pgfpathlineto{\pgfqpoint{1.647918in}{1.571330in}}%
\pgfpathlineto{\pgfqpoint{1.648784in}{1.588487in}}%
\pgfpathlineto{\pgfqpoint{1.650509in}{1.609205in}}%
\pgfpathlineto{\pgfqpoint{1.651372in}{1.570973in}}%
\pgfpathlineto{\pgfqpoint{1.652236in}{1.579641in}}%
\pgfpathlineto{\pgfqpoint{1.653101in}{1.655981in}}%
\pgfpathlineto{\pgfqpoint{1.653965in}{1.608551in}}%
\pgfpathlineto{\pgfqpoint{1.654830in}{1.617456in}}%
\pgfpathlineto{\pgfqpoint{1.655694in}{1.664827in}}%
\pgfpathlineto{\pgfqpoint{1.657422in}{1.577830in}}%
\pgfpathlineto{\pgfqpoint{1.658286in}{1.591751in}}%
\pgfpathlineto{\pgfqpoint{1.659151in}{1.596975in}}%
\pgfpathlineto{\pgfqpoint{1.660879in}{1.554527in}}%
\pgfpathlineto{\pgfqpoint{1.661745in}{1.606978in}}%
\pgfpathlineto{\pgfqpoint{1.662610in}{1.560346in}}%
\pgfpathlineto{\pgfqpoint{1.663476in}{1.575781in}}%
\pgfpathlineto{\pgfqpoint{1.664342in}{1.562276in}}%
\pgfpathlineto{\pgfqpoint{1.666070in}{1.618404in}}%
\pgfpathlineto{\pgfqpoint{1.667800in}{1.530961in}}%
\pgfpathlineto{\pgfqpoint{1.668666in}{1.579224in}}%
\pgfpathlineto{\pgfqpoint{1.670393in}{1.526331in}}%
\pgfpathlineto{\pgfqpoint{1.671259in}{1.538144in}}%
\pgfpathlineto{\pgfqpoint{1.672989in}{1.620959in}}%
\pgfpathlineto{\pgfqpoint{1.674716in}{1.602200in}}%
\pgfpathlineto{\pgfqpoint{1.675581in}{1.570438in}}%
\pgfpathlineto{\pgfqpoint{1.676445in}{1.640487in}}%
\pgfpathlineto{\pgfqpoint{1.678175in}{1.584627in}}%
\pgfpathlineto{\pgfqpoint{1.679039in}{1.623866in}}%
\pgfpathlineto{\pgfqpoint{1.679905in}{1.553877in}}%
\pgfpathlineto{\pgfqpoint{1.680770in}{1.560436in}}%
\pgfpathlineto{\pgfqpoint{1.681635in}{1.651174in}}%
\pgfpathlineto{\pgfqpoint{1.683364in}{1.593562in}}%
\pgfpathlineto{\pgfqpoint{1.684229in}{1.696113in}}%
\pgfpathlineto{\pgfqpoint{1.685094in}{1.560644in}}%
\pgfpathlineto{\pgfqpoint{1.685959in}{1.649839in}}%
\pgfpathlineto{\pgfqpoint{1.686825in}{1.543666in}}%
\pgfpathlineto{\pgfqpoint{1.688554in}{1.591037in}}%
\pgfpathlineto{\pgfqpoint{1.689420in}{1.575662in}}%
\pgfpathlineto{\pgfqpoint{1.690285in}{1.516121in}}%
\pgfpathlineto{\pgfqpoint{1.692015in}{1.600300in}}%
\pgfpathlineto{\pgfqpoint{1.692880in}{1.493235in}}%
\pgfpathlineto{\pgfqpoint{1.693742in}{1.603858in}}%
\pgfpathlineto{\pgfqpoint{1.695472in}{1.550047in}}%
\pgfpathlineto{\pgfqpoint{1.696337in}{1.633660in}}%
\pgfpathlineto{\pgfqpoint{1.698064in}{1.570378in}}%
\pgfpathlineto{\pgfqpoint{1.698929in}{1.568066in}}%
\pgfpathlineto{\pgfqpoint{1.700660in}{1.622978in}}%
\pgfpathlineto{\pgfqpoint{1.702389in}{1.528380in}}%
\pgfpathlineto{\pgfqpoint{1.703252in}{1.574476in}}%
\pgfpathlineto{\pgfqpoint{1.704981in}{1.498548in}}%
\pgfpathlineto{\pgfqpoint{1.706709in}{1.615794in}}%
\pgfpathlineto{\pgfqpoint{1.707574in}{1.566995in}}%
\pgfpathlineto{\pgfqpoint{1.708439in}{1.622204in}}%
\pgfpathlineto{\pgfqpoint{1.709304in}{1.599972in}}%
\pgfpathlineto{\pgfqpoint{1.710169in}{1.613299in}}%
\pgfpathlineto{\pgfqpoint{1.711035in}{1.649452in}}%
\pgfpathlineto{\pgfqpoint{1.711900in}{1.632474in}}%
\pgfpathlineto{\pgfqpoint{1.712763in}{1.737372in}}%
\pgfpathlineto{\pgfqpoint{1.713627in}{1.640074in}}%
\pgfpathlineto{\pgfqpoint{1.714491in}{1.785933in}}%
\pgfpathlineto{\pgfqpoint{1.715355in}{1.712202in}}%
\pgfpathlineto{\pgfqpoint{1.716220in}{1.715467in}}%
\pgfpathlineto{\pgfqpoint{1.717949in}{1.652007in}}%
\pgfpathlineto{\pgfqpoint{1.718815in}{1.681154in}}%
\pgfpathlineto{\pgfqpoint{1.720545in}{1.656695in}}%
\pgfpathlineto{\pgfqpoint{1.723138in}{1.679314in}}%
\pgfpathlineto{\pgfqpoint{1.724003in}{1.672368in}}%
\pgfpathlineto{\pgfqpoint{1.725732in}{1.752568in}}%
\pgfpathlineto{\pgfqpoint{1.726596in}{1.652661in}}%
\pgfpathlineto{\pgfqpoint{1.727461in}{1.718613in}}%
\pgfpathlineto{\pgfqpoint{1.729191in}{1.622442in}}%
\pgfpathlineto{\pgfqpoint{1.730056in}{1.687327in}}%
\pgfpathlineto{\pgfqpoint{1.730922in}{1.654825in}}%
\pgfpathlineto{\pgfqpoint{1.732650in}{1.704543in}}%
\pgfpathlineto{\pgfqpoint{1.733516in}{1.641528in}}%
\pgfpathlineto{\pgfqpoint{1.735246in}{1.727875in}}%
\pgfpathlineto{\pgfqpoint{1.736112in}{1.644232in}}%
\pgfpathlineto{\pgfqpoint{1.736974in}{1.698668in}}%
\pgfpathlineto{\pgfqpoint{1.738706in}{1.611284in}}%
\pgfpathlineto{\pgfqpoint{1.739570in}{1.625116in}}%
\pgfpathlineto{\pgfqpoint{1.740436in}{1.637077in}}%
\pgfpathlineto{\pgfqpoint{1.741299in}{1.620248in}}%
\pgfpathlineto{\pgfqpoint{1.742164in}{1.641737in}}%
\pgfpathlineto{\pgfqpoint{1.743891in}{1.528410in}}%
\pgfpathlineto{\pgfqpoint{1.744756in}{1.564444in}}%
\pgfpathlineto{\pgfqpoint{1.745620in}{1.557855in}}%
\pgfpathlineto{\pgfqpoint{1.746486in}{1.574357in}}%
\pgfpathlineto{\pgfqpoint{1.747352in}{1.542212in}}%
\pgfpathlineto{\pgfqpoint{1.748217in}{1.554709in}}%
\pgfpathlineto{\pgfqpoint{1.749082in}{1.634791in}}%
\pgfpathlineto{\pgfqpoint{1.750814in}{1.572992in}}%
\pgfpathlineto{\pgfqpoint{1.751679in}{1.578217in}}%
\pgfpathlineto{\pgfqpoint{1.752545in}{1.605702in}}%
\pgfpathlineto{\pgfqpoint{1.754272in}{1.571092in}}%
\pgfpathlineto{\pgfqpoint{1.756004in}{1.635977in}}%
\pgfpathlineto{\pgfqpoint{1.756870in}{1.627220in}}%
\pgfpathlineto{\pgfqpoint{1.757735in}{1.580946in}}%
\pgfpathlineto{\pgfqpoint{1.758602in}{1.588011in}}%
\pgfpathlineto{\pgfqpoint{1.759468in}{1.598400in}}%
\pgfpathlineto{\pgfqpoint{1.760334in}{1.568954in}}%
\pgfpathlineto{\pgfqpoint{1.761200in}{1.603445in}}%
\pgfpathlineto{\pgfqpoint{1.762066in}{1.594778in}}%
\pgfpathlineto{\pgfqpoint{1.762931in}{1.574535in}}%
\pgfpathlineto{\pgfqpoint{1.763797in}{1.609146in}}%
\pgfpathlineto{\pgfqpoint{1.764663in}{1.528112in}}%
\pgfpathlineto{\pgfqpoint{1.765529in}{1.576019in}}%
\pgfpathlineto{\pgfqpoint{1.766394in}{1.553014in}}%
\pgfpathlineto{\pgfqpoint{1.767259in}{1.611815in}}%
\pgfpathlineto{\pgfqpoint{1.768988in}{1.540372in}}%
\pgfpathlineto{\pgfqpoint{1.769853in}{1.621197in}}%
\pgfpathlineto{\pgfqpoint{1.770717in}{1.571687in}}%
\pgfpathlineto{\pgfqpoint{1.772446in}{1.595313in}}%
\pgfpathlineto{\pgfqpoint{1.773309in}{1.548890in}}%
\pgfpathlineto{\pgfqpoint{1.774173in}{1.585932in}}%
\pgfpathlineto{\pgfqpoint{1.775903in}{1.560882in}}%
\pgfpathlineto{\pgfqpoint{1.776768in}{1.632772in}}%
\pgfpathlineto{\pgfqpoint{1.778496in}{1.539687in}}%
\pgfpathlineto{\pgfqpoint{1.779358in}{1.625647in}}%
\pgfpathlineto{\pgfqpoint{1.780222in}{1.554115in}}%
\pgfpathlineto{\pgfqpoint{1.782816in}{1.665125in}}%
\pgfpathlineto{\pgfqpoint{1.783681in}{1.557498in}}%
\pgfpathlineto{\pgfqpoint{1.784545in}{1.584389in}}%
\pgfpathlineto{\pgfqpoint{1.785410in}{1.542063in}}%
\pgfpathlineto{\pgfqpoint{1.786275in}{1.610510in}}%
\pgfpathlineto{\pgfqpoint{1.787138in}{1.589970in}}%
\pgfpathlineto{\pgfqpoint{1.788003in}{1.620483in}}%
\pgfpathlineto{\pgfqpoint{1.788868in}{1.504546in}}%
\pgfpathlineto{\pgfqpoint{1.789732in}{1.631764in}}%
\pgfpathlineto{\pgfqpoint{1.791463in}{1.536423in}}%
\pgfpathlineto{\pgfqpoint{1.792326in}{1.598519in}}%
\pgfpathlineto{\pgfqpoint{1.793189in}{1.526510in}}%
\pgfpathlineto{\pgfqpoint{1.794054in}{1.576852in}}%
\pgfpathlineto{\pgfqpoint{1.794918in}{1.569103in}}%
\pgfpathlineto{\pgfqpoint{1.796649in}{1.601843in}}%
\pgfpathlineto{\pgfqpoint{1.798377in}{1.549188in}}%
\pgfpathlineto{\pgfqpoint{1.799241in}{1.594778in}}%
\pgfpathlineto{\pgfqpoint{1.800107in}{1.534344in}}%
\pgfpathlineto{\pgfqpoint{1.801837in}{1.622264in}}%
\pgfpathlineto{\pgfqpoint{1.802704in}{1.582489in}}%
\pgfpathlineto{\pgfqpoint{1.803570in}{1.644406in}}%
\pgfpathlineto{\pgfqpoint{1.805299in}{1.564087in}}%
\pgfpathlineto{\pgfqpoint{1.806165in}{1.631169in}}%
\pgfpathlineto{\pgfqpoint{1.807031in}{1.606561in}}%
\pgfpathlineto{\pgfqpoint{1.807897in}{1.548474in}}%
\pgfpathlineto{\pgfqpoint{1.808763in}{1.603029in}}%
\pgfpathlineto{\pgfqpoint{1.810493in}{1.540104in}}%
\pgfpathlineto{\pgfqpoint{1.811358in}{1.552690in}}%
\pgfpathlineto{\pgfqpoint{1.812222in}{1.540580in}}%
\pgfpathlineto{\pgfqpoint{1.813089in}{1.589732in}}%
\pgfpathlineto{\pgfqpoint{1.814820in}{1.528291in}}%
\pgfpathlineto{\pgfqpoint{1.815683in}{1.566582in}}%
\pgfpathlineto{\pgfqpoint{1.816549in}{1.540967in}}%
\pgfpathlineto{\pgfqpoint{1.817413in}{1.603981in}}%
\pgfpathlineto{\pgfqpoint{1.819140in}{1.532121in}}%
\pgfpathlineto{\pgfqpoint{1.820005in}{1.625588in}}%
\pgfpathlineto{\pgfqpoint{1.821736in}{1.522948in}}%
\pgfpathlineto{\pgfqpoint{1.822603in}{1.616742in}}%
\pgfpathlineto{\pgfqpoint{1.823469in}{1.560436in}}%
\pgfpathlineto{\pgfqpoint{1.826063in}{1.659752in}}%
\pgfpathlineto{\pgfqpoint{1.826928in}{1.656695in}}%
\pgfpathlineto{\pgfqpoint{1.827794in}{1.687267in}}%
\pgfpathlineto{\pgfqpoint{1.828657in}{1.620364in}}%
\pgfpathlineto{\pgfqpoint{1.829522in}{1.620780in}}%
\pgfpathlineto{\pgfqpoint{1.830388in}{1.670289in}}%
\pgfpathlineto{\pgfqpoint{1.831253in}{1.572754in}}%
\pgfpathlineto{\pgfqpoint{1.832114in}{1.665660in}}%
\pgfpathlineto{\pgfqpoint{1.832979in}{1.639480in}}%
\pgfpathlineto{\pgfqpoint{1.833845in}{1.628202in}}%
\pgfpathlineto{\pgfqpoint{1.834712in}{1.577030in}}%
\pgfpathlineto{\pgfqpoint{1.835577in}{1.585490in}}%
\pgfpathlineto{\pgfqpoint{1.836442in}{1.698906in}}%
\pgfpathlineto{\pgfqpoint{1.838171in}{1.623662in}}%
\pgfpathlineto{\pgfqpoint{1.839899in}{1.683411in}}%
\pgfpathlineto{\pgfqpoint{1.842494in}{1.553166in}}%
\pgfpathlineto{\pgfqpoint{1.844224in}{1.654026in}}%
\pgfpathlineto{\pgfqpoint{1.845089in}{1.605435in}}%
\pgfpathlineto{\pgfqpoint{1.845955in}{1.660376in}}%
\pgfpathlineto{\pgfqpoint{1.846820in}{1.636691in}}%
\pgfpathlineto{\pgfqpoint{1.847684in}{1.568155in}}%
\pgfpathlineto{\pgfqpoint{1.848546in}{1.586825in}}%
\pgfpathlineto{\pgfqpoint{1.849412in}{1.614548in}}%
\pgfpathlineto{\pgfqpoint{1.850276in}{1.549069in}}%
\pgfpathlineto{\pgfqpoint{1.851141in}{1.651709in}}%
\pgfpathlineto{\pgfqpoint{1.852870in}{1.533218in}}%
\pgfpathlineto{\pgfqpoint{1.853735in}{1.584865in}}%
\pgfpathlineto{\pgfqpoint{1.854601in}{1.564117in}}%
\pgfpathlineto{\pgfqpoint{1.856331in}{1.622502in}}%
\pgfpathlineto{\pgfqpoint{1.857196in}{1.555479in}}%
\pgfpathlineto{\pgfqpoint{1.858925in}{1.584389in}}%
\pgfpathlineto{\pgfqpoint{1.859790in}{1.565928in}}%
\pgfpathlineto{\pgfqpoint{1.860656in}{1.588070in}}%
\pgfpathlineto{\pgfqpoint{1.862388in}{1.691365in}}%
\pgfpathlineto{\pgfqpoint{1.863254in}{1.697597in}}%
\pgfpathlineto{\pgfqpoint{1.864120in}{1.577711in}}%
\pgfpathlineto{\pgfqpoint{1.865845in}{1.625171in}}%
\pgfpathlineto{\pgfqpoint{1.867573in}{1.544674in}}%
\pgfpathlineto{\pgfqpoint{1.869304in}{1.630515in}}%
\pgfpathlineto{\pgfqpoint{1.870168in}{1.511904in}}%
\pgfpathlineto{\pgfqpoint{1.871897in}{1.582429in}}%
\pgfpathlineto{\pgfqpoint{1.872762in}{1.555241in}}%
\pgfpathlineto{\pgfqpoint{1.873626in}{1.615020in}}%
\pgfpathlineto{\pgfqpoint{1.874493in}{1.538620in}}%
\pgfpathlineto{\pgfqpoint{1.875359in}{1.553639in}}%
\pgfpathlineto{\pgfqpoint{1.877089in}{1.541647in}}%
\pgfpathlineto{\pgfqpoint{1.877953in}{1.566106in}}%
\pgfpathlineto{\pgfqpoint{1.878816in}{1.499202in}}%
\pgfpathlineto{\pgfqpoint{1.879681in}{1.584389in}}%
\pgfpathlineto{\pgfqpoint{1.880547in}{1.553490in}}%
\pgfpathlineto{\pgfqpoint{1.883143in}{1.639420in}}%
\pgfpathlineto{\pgfqpoint{1.884008in}{1.617158in}}%
\pgfpathlineto{\pgfqpoint{1.884871in}{1.640963in}}%
\pgfpathlineto{\pgfqpoint{1.885735in}{1.610183in}}%
\pgfpathlineto{\pgfqpoint{1.887463in}{1.651828in}}%
\pgfpathlineto{\pgfqpoint{1.890054in}{1.566493in}}%
\pgfpathlineto{\pgfqpoint{1.890918in}{1.605583in}}%
\pgfpathlineto{\pgfqpoint{1.891784in}{1.549128in}}%
\pgfpathlineto{\pgfqpoint{1.892650in}{1.555568in}}%
\pgfpathlineto{\pgfqpoint{1.893515in}{1.550433in}}%
\pgfpathlineto{\pgfqpoint{1.894381in}{1.604988in}}%
\pgfpathlineto{\pgfqpoint{1.895246in}{1.593146in}}%
\pgfpathlineto{\pgfqpoint{1.896112in}{1.545031in}}%
\pgfpathlineto{\pgfqpoint{1.897841in}{1.590089in}}%
\pgfpathlineto{\pgfqpoint{1.898707in}{1.553996in}}%
\pgfpathlineto{\pgfqpoint{1.900437in}{1.603743in}}%
\pgfpathlineto{\pgfqpoint{1.901303in}{1.571033in}}%
\pgfpathlineto{\pgfqpoint{1.902169in}{1.617099in}}%
\pgfpathlineto{\pgfqpoint{1.903034in}{1.605345in}}%
\pgfpathlineto{\pgfqpoint{1.904764in}{1.574298in}}%
\pgfpathlineto{\pgfqpoint{1.907360in}{1.594243in}}%
\pgfpathlineto{\pgfqpoint{1.908225in}{1.563909in}}%
\pgfpathlineto{\pgfqpoint{1.909953in}{1.634196in}}%
\pgfpathlineto{\pgfqpoint{1.910818in}{1.519266in}}%
\pgfpathlineto{\pgfqpoint{1.911684in}{1.600891in}}%
\pgfpathlineto{\pgfqpoint{1.912550in}{1.566757in}}%
\pgfpathlineto{\pgfqpoint{1.913415in}{1.606413in}}%
\pgfpathlineto{\pgfqpoint{1.915146in}{1.586349in}}%
\pgfpathlineto{\pgfqpoint{1.916877in}{1.616742in}}%
\pgfpathlineto{\pgfqpoint{1.917742in}{1.601426in}}%
\pgfpathlineto{\pgfqpoint{1.919471in}{1.551857in}}%
\pgfpathlineto{\pgfqpoint{1.921200in}{1.598221in}}%
\pgfpathlineto{\pgfqpoint{1.922931in}{1.539747in}}%
\pgfpathlineto{\pgfqpoint{1.923796in}{1.582013in}}%
\pgfpathlineto{\pgfqpoint{1.924661in}{1.549957in}}%
\pgfpathlineto{\pgfqpoint{1.925527in}{1.585575in}}%
\pgfpathlineto{\pgfqpoint{1.927259in}{1.517664in}}%
\pgfpathlineto{\pgfqpoint{1.928985in}{1.618464in}}%
\pgfpathlineto{\pgfqpoint{1.929850in}{1.610153in}}%
\pgfpathlineto{\pgfqpoint{1.930715in}{1.630366in}}%
\pgfpathlineto{\pgfqpoint{1.931580in}{1.613537in}}%
\pgfpathlineto{\pgfqpoint{1.932446in}{1.643280in}}%
\pgfpathlineto{\pgfqpoint{1.933312in}{1.537996in}}%
\pgfpathlineto{\pgfqpoint{1.935043in}{1.586230in}}%
\pgfpathlineto{\pgfqpoint{1.936775in}{1.622561in}}%
\pgfpathlineto{\pgfqpoint{1.937642in}{1.648147in}}%
\pgfpathlineto{\pgfqpoint{1.938507in}{1.611165in}}%
\pgfpathlineto{\pgfqpoint{1.939373in}{1.638174in}}%
\pgfpathlineto{\pgfqpoint{1.941967in}{1.535475in}}%
\pgfpathlineto{\pgfqpoint{1.944561in}{1.642510in}}%
\pgfpathlineto{\pgfqpoint{1.946290in}{1.580593in}}%
\pgfpathlineto{\pgfqpoint{1.947154in}{1.592465in}}%
\pgfpathlineto{\pgfqpoint{1.948018in}{1.509948in}}%
\pgfpathlineto{\pgfqpoint{1.949748in}{1.632151in}}%
\pgfpathlineto{\pgfqpoint{1.951475in}{1.623394in}}%
\pgfpathlineto{\pgfqpoint{1.952341in}{1.626748in}}%
\pgfpathlineto{\pgfqpoint{1.954072in}{1.695343in}}%
\pgfpathlineto{\pgfqpoint{1.954936in}{1.669698in}}%
\pgfpathlineto{\pgfqpoint{1.955801in}{1.729418in}}%
\pgfpathlineto{\pgfqpoint{1.956666in}{1.679909in}}%
\pgfpathlineto{\pgfqpoint{1.957531in}{1.687271in}}%
\pgfpathlineto{\pgfqpoint{1.959262in}{1.709443in}}%
\pgfpathlineto{\pgfqpoint{1.962723in}{1.617872in}}%
\pgfpathlineto{\pgfqpoint{1.963588in}{1.708402in}}%
\pgfpathlineto{\pgfqpoint{1.964454in}{1.675454in}}%
\pgfpathlineto{\pgfqpoint{1.966183in}{1.732088in}}%
\pgfpathlineto{\pgfqpoint{1.968778in}{1.655241in}}%
\pgfpathlineto{\pgfqpoint{1.969644in}{1.691900in}}%
\pgfpathlineto{\pgfqpoint{1.970509in}{1.670587in}}%
\pgfpathlineto{\pgfqpoint{1.972240in}{1.572576in}}%
\pgfpathlineto{\pgfqpoint{1.973106in}{1.555126in}}%
\pgfpathlineto{\pgfqpoint{1.974835in}{1.612291in}}%
\pgfpathlineto{\pgfqpoint{1.975702in}{1.556134in}}%
\pgfpathlineto{\pgfqpoint{1.977433in}{1.631407in}}%
\pgfpathlineto{\pgfqpoint{1.978298in}{1.564117in}}%
\pgfpathlineto{\pgfqpoint{1.980028in}{1.646544in}}%
\pgfpathlineto{\pgfqpoint{1.980894in}{1.621226in}}%
\pgfpathlineto{\pgfqpoint{1.981760in}{1.583500in}}%
\pgfpathlineto{\pgfqpoint{1.983491in}{1.675514in}}%
\pgfpathlineto{\pgfqpoint{1.986085in}{1.535594in}}%
\pgfpathlineto{\pgfqpoint{1.989540in}{1.629329in}}%
\pgfpathlineto{\pgfqpoint{1.990405in}{1.634701in}}%
\pgfpathlineto{\pgfqpoint{1.991267in}{1.618110in}}%
\pgfpathlineto{\pgfqpoint{1.992132in}{1.656755in}}%
\pgfpathlineto{\pgfqpoint{1.995590in}{1.534407in}}%
\pgfpathlineto{\pgfqpoint{1.997321in}{1.616181in}}%
\pgfpathlineto{\pgfqpoint{1.999051in}{1.575487in}}%
\pgfpathlineto{\pgfqpoint{1.999915in}{1.627135in}}%
\pgfpathlineto{\pgfqpoint{2.000777in}{1.604516in}}%
\pgfpathlineto{\pgfqpoint{2.001642in}{1.640907in}}%
\pgfpathlineto{\pgfqpoint{2.002508in}{1.563674in}}%
\pgfpathlineto{\pgfqpoint{2.003373in}{1.614965in}}%
\pgfpathlineto{\pgfqpoint{2.004238in}{1.553375in}}%
\pgfpathlineto{\pgfqpoint{2.005103in}{1.606416in}}%
\pgfpathlineto{\pgfqpoint{2.006833in}{1.553996in}}%
\pgfpathlineto{\pgfqpoint{2.007697in}{1.636810in}}%
\pgfpathlineto{\pgfqpoint{2.009429in}{1.547317in}}%
\pgfpathlineto{\pgfqpoint{2.011158in}{1.659071in}}%
\pgfpathlineto{\pgfqpoint{2.012023in}{1.604219in}}%
\pgfpathlineto{\pgfqpoint{2.012887in}{1.647496in}}%
\pgfpathlineto{\pgfqpoint{2.013752in}{1.635148in}}%
\pgfpathlineto{\pgfqpoint{2.014617in}{1.546280in}}%
\pgfpathlineto{\pgfqpoint{2.015482in}{1.565098in}}%
\pgfpathlineto{\pgfqpoint{2.016347in}{1.571539in}}%
\pgfpathlineto{\pgfqpoint{2.017211in}{1.527343in}}%
\pgfpathlineto{\pgfqpoint{2.018077in}{1.590744in}}%
\pgfpathlineto{\pgfqpoint{2.018939in}{1.532388in}}%
\pgfpathlineto{\pgfqpoint{2.020669in}{1.615913in}}%
\pgfpathlineto{\pgfqpoint{2.022400in}{1.564860in}}%
\pgfpathlineto{\pgfqpoint{2.024131in}{1.613180in}}%
\pgfpathlineto{\pgfqpoint{2.024995in}{1.612589in}}%
\pgfpathlineto{\pgfqpoint{2.025860in}{1.606000in}}%
\pgfpathlineto{\pgfqpoint{2.026727in}{1.576852in}}%
\pgfpathlineto{\pgfqpoint{2.027592in}{1.589201in}}%
\pgfpathlineto{\pgfqpoint{2.028455in}{1.633515in}}%
\pgfpathlineto{\pgfqpoint{2.029320in}{1.559815in}}%
\pgfpathlineto{\pgfqpoint{2.031052in}{1.662336in}}%
\pgfpathlineto{\pgfqpoint{2.032782in}{1.576614in}}%
\pgfpathlineto{\pgfqpoint{2.033646in}{1.581838in}}%
\pgfpathlineto{\pgfqpoint{2.035374in}{1.642566in}}%
\pgfpathlineto{\pgfqpoint{2.036239in}{1.641677in}}%
\pgfpathlineto{\pgfqpoint{2.037104in}{1.593532in}}%
\pgfpathlineto{\pgfqpoint{2.037969in}{1.662068in}}%
\pgfpathlineto{\pgfqpoint{2.039700in}{1.617396in}}%
\pgfpathlineto{\pgfqpoint{2.040565in}{1.627428in}}%
\pgfpathlineto{\pgfqpoint{2.041430in}{1.605464in}}%
\pgfpathlineto{\pgfqpoint{2.042296in}{1.665601in}}%
\pgfpathlineto{\pgfqpoint{2.043162in}{1.614013in}}%
\pgfpathlineto{\pgfqpoint{2.044026in}{1.683884in}}%
\pgfpathlineto{\pgfqpoint{2.044891in}{1.652836in}}%
\pgfpathlineto{\pgfqpoint{2.045756in}{1.676224in}}%
\pgfpathlineto{\pgfqpoint{2.047484in}{1.553252in}}%
\pgfpathlineto{\pgfqpoint{2.049214in}{1.666132in}}%
\pgfpathlineto{\pgfqpoint{2.050080in}{1.635263in}}%
\pgfpathlineto{\pgfqpoint{2.050945in}{1.585754in}}%
\pgfpathlineto{\pgfqpoint{2.051810in}{1.666549in}}%
\pgfpathlineto{\pgfqpoint{2.052674in}{1.613358in}}%
\pgfpathlineto{\pgfqpoint{2.053538in}{1.621907in}}%
\pgfpathlineto{\pgfqpoint{2.054402in}{1.640844in}}%
\pgfpathlineto{\pgfqpoint{2.056131in}{1.565511in}}%
\pgfpathlineto{\pgfqpoint{2.056996in}{1.606710in}}%
\pgfpathlineto{\pgfqpoint{2.057861in}{1.591989in}}%
\pgfpathlineto{\pgfqpoint{2.058726in}{1.600776in}}%
\pgfpathlineto{\pgfqpoint{2.060456in}{1.647020in}}%
\pgfpathlineto{\pgfqpoint{2.061322in}{1.612618in}}%
\pgfpathlineto{\pgfqpoint{2.062187in}{1.525264in}}%
\pgfpathlineto{\pgfqpoint{2.063052in}{1.558034in}}%
\pgfpathlineto{\pgfqpoint{2.063916in}{1.540550in}}%
\pgfpathlineto{\pgfqpoint{2.065646in}{1.622085in}}%
\pgfpathlineto{\pgfqpoint{2.068243in}{1.570676in}}%
\pgfpathlineto{\pgfqpoint{2.069109in}{1.557379in}}%
\pgfpathlineto{\pgfqpoint{2.069974in}{1.572576in}}%
\pgfpathlineto{\pgfqpoint{2.070838in}{1.535058in}}%
\pgfpathlineto{\pgfqpoint{2.071701in}{1.645061in}}%
\pgfpathlineto{\pgfqpoint{2.072565in}{1.546752in}}%
\pgfpathlineto{\pgfqpoint{2.074294in}{1.607662in}}%
\pgfpathlineto{\pgfqpoint{2.075160in}{1.575190in}}%
\pgfpathlineto{\pgfqpoint{2.076025in}{1.638085in}}%
\pgfpathlineto{\pgfqpoint{2.076890in}{1.516537in}}%
\pgfpathlineto{\pgfqpoint{2.077754in}{1.545090in}}%
\pgfpathlineto{\pgfqpoint{2.079483in}{1.643399in}}%
\pgfpathlineto{\pgfqpoint{2.080348in}{1.600300in}}%
\pgfpathlineto{\pgfqpoint{2.081213in}{1.610629in}}%
\pgfpathlineto{\pgfqpoint{2.082078in}{1.632950in}}%
\pgfpathlineto{\pgfqpoint{2.084671in}{1.588844in}}%
\pgfpathlineto{\pgfqpoint{2.086401in}{1.645001in}}%
\pgfpathlineto{\pgfqpoint{2.088130in}{1.598935in}}%
\pgfpathlineto{\pgfqpoint{2.088995in}{1.643339in}}%
\pgfpathlineto{\pgfqpoint{2.089859in}{1.585073in}}%
\pgfpathlineto{\pgfqpoint{2.090724in}{1.639599in}}%
\pgfpathlineto{\pgfqpoint{2.091589in}{1.620308in}}%
\pgfpathlineto{\pgfqpoint{2.092454in}{1.519032in}}%
\pgfpathlineto{\pgfqpoint{2.094184in}{1.676109in}}%
\pgfpathlineto{\pgfqpoint{2.095914in}{1.573766in}}%
\pgfpathlineto{\pgfqpoint{2.096780in}{1.661622in}}%
\pgfpathlineto{\pgfqpoint{2.098507in}{1.588427in}}%
\pgfpathlineto{\pgfqpoint{2.099373in}{1.620483in}}%
\pgfpathlineto{\pgfqpoint{2.100239in}{1.600240in}}%
\pgfpathlineto{\pgfqpoint{2.101101in}{1.659900in}}%
\pgfpathlineto{\pgfqpoint{2.102828in}{1.592997in}}%
\pgfpathlineto{\pgfqpoint{2.103692in}{1.597745in}}%
\pgfpathlineto{\pgfqpoint{2.106286in}{1.577800in}}%
\pgfpathlineto{\pgfqpoint{2.107151in}{1.533396in}}%
\pgfpathlineto{\pgfqpoint{2.109746in}{1.621018in}}%
\pgfpathlineto{\pgfqpoint{2.110611in}{1.609384in}}%
\pgfpathlineto{\pgfqpoint{2.112342in}{1.552452in}}%
\pgfpathlineto{\pgfqpoint{2.113207in}{1.598638in}}%
\pgfpathlineto{\pgfqpoint{2.114938in}{1.528410in}}%
\pgfpathlineto{\pgfqpoint{2.115804in}{1.606591in}}%
\pgfpathlineto{\pgfqpoint{2.118403in}{1.541439in}}%
\pgfpathlineto{\pgfqpoint{2.120134in}{1.635560in}}%
\pgfpathlineto{\pgfqpoint{2.120999in}{1.611250in}}%
\pgfpathlineto{\pgfqpoint{2.122731in}{1.634196in}}%
\pgfpathlineto{\pgfqpoint{2.123596in}{1.551798in}}%
\pgfpathlineto{\pgfqpoint{2.125328in}{1.587178in}}%
\pgfpathlineto{\pgfqpoint{2.126193in}{1.558208in}}%
\pgfpathlineto{\pgfqpoint{2.128790in}{1.592699in}}%
\pgfpathlineto{\pgfqpoint{2.131385in}{1.521405in}}%
\pgfpathlineto{\pgfqpoint{2.133116in}{1.635382in}}%
\pgfpathlineto{\pgfqpoint{2.133982in}{1.639182in}}%
\pgfpathlineto{\pgfqpoint{2.135713in}{1.606413in}}%
\pgfpathlineto{\pgfqpoint{2.136575in}{1.624164in}}%
\pgfpathlineto{\pgfqpoint{2.137440in}{1.573911in}}%
\pgfpathlineto{\pgfqpoint{2.138303in}{1.673852in}}%
\pgfpathlineto{\pgfqpoint{2.139169in}{1.661444in}}%
\pgfpathlineto{\pgfqpoint{2.140031in}{1.675811in}}%
\pgfpathlineto{\pgfqpoint{2.141763in}{1.598757in}}%
\pgfpathlineto{\pgfqpoint{2.142629in}{1.601902in}}%
\pgfpathlineto{\pgfqpoint{2.143495in}{1.625647in}}%
\pgfpathlineto{\pgfqpoint{2.145225in}{1.598578in}}%
\pgfpathlineto{\pgfqpoint{2.146091in}{1.610748in}}%
\pgfpathlineto{\pgfqpoint{2.147822in}{1.567709in}}%
\pgfpathlineto{\pgfqpoint{2.149554in}{1.639985in}}%
\pgfpathlineto{\pgfqpoint{2.150418in}{1.659960in}}%
\pgfpathlineto{\pgfqpoint{2.152149in}{1.572933in}}%
\pgfpathlineto{\pgfqpoint{2.153015in}{1.587479in}}%
\pgfpathlineto{\pgfqpoint{2.153879in}{1.650225in}}%
\pgfpathlineto{\pgfqpoint{2.154743in}{1.648682in}}%
\pgfpathlineto{\pgfqpoint{2.156472in}{1.524937in}}%
\pgfpathlineto{\pgfqpoint{2.158202in}{1.609737in}}%
\pgfpathlineto{\pgfqpoint{2.159066in}{1.578838in}}%
\pgfpathlineto{\pgfqpoint{2.159928in}{1.615080in}}%
\pgfpathlineto{\pgfqpoint{2.160794in}{1.543012in}}%
\pgfpathlineto{\pgfqpoint{2.162520in}{1.600712in}}%
\pgfpathlineto{\pgfqpoint{2.163384in}{1.562127in}}%
\pgfpathlineto{\pgfqpoint{2.165115in}{1.621907in}}%
\pgfpathlineto{\pgfqpoint{2.165980in}{1.622383in}}%
\pgfpathlineto{\pgfqpoint{2.166844in}{1.684062in}}%
\pgfpathlineto{\pgfqpoint{2.167707in}{1.675990in}}%
\pgfpathlineto{\pgfqpoint{2.168572in}{1.635650in}}%
\pgfpathlineto{\pgfqpoint{2.169437in}{1.776075in}}%
\pgfpathlineto{\pgfqpoint{2.171166in}{1.645150in}}%
\pgfpathlineto{\pgfqpoint{2.172029in}{1.701754in}}%
\pgfpathlineto{\pgfqpoint{2.172893in}{1.655093in}}%
\pgfpathlineto{\pgfqpoint{2.173758in}{1.659250in}}%
\pgfpathlineto{\pgfqpoint{2.174620in}{1.675752in}}%
\pgfpathlineto{\pgfqpoint{2.175484in}{1.639480in}}%
\pgfpathlineto{\pgfqpoint{2.176349in}{1.738677in}}%
\pgfpathlineto{\pgfqpoint{2.177215in}{1.738558in}}%
\pgfpathlineto{\pgfqpoint{2.178080in}{1.610064in}}%
\pgfpathlineto{\pgfqpoint{2.179810in}{1.673673in}}%
\pgfpathlineto{\pgfqpoint{2.180676in}{1.665035in}}%
\pgfpathlineto{\pgfqpoint{2.181542in}{1.682400in}}%
\pgfpathlineto{\pgfqpoint{2.182405in}{1.653550in}}%
\pgfpathlineto{\pgfqpoint{2.183270in}{1.703178in}}%
\pgfpathlineto{\pgfqpoint{2.184136in}{1.698013in}}%
\pgfpathlineto{\pgfqpoint{2.185000in}{1.679046in}}%
\pgfpathlineto{\pgfqpoint{2.185866in}{1.689286in}}%
\pgfpathlineto{\pgfqpoint{2.186731in}{1.718851in}}%
\pgfpathlineto{\pgfqpoint{2.188460in}{1.644406in}}%
\pgfpathlineto{\pgfqpoint{2.189326in}{1.767825in}}%
\pgfpathlineto{\pgfqpoint{2.190192in}{1.745771in}}%
\pgfpathlineto{\pgfqpoint{2.191056in}{1.766757in}}%
\pgfpathlineto{\pgfqpoint{2.192788in}{1.635977in}}%
\pgfpathlineto{\pgfqpoint{2.193652in}{1.664530in}}%
\pgfpathlineto{\pgfqpoint{2.194519in}{1.571624in}}%
\pgfpathlineto{\pgfqpoint{2.195384in}{1.702821in}}%
\pgfpathlineto{\pgfqpoint{2.196249in}{1.688334in}}%
\pgfpathlineto{\pgfqpoint{2.197115in}{1.633690in}}%
\pgfpathlineto{\pgfqpoint{2.197982in}{1.664470in}}%
\pgfpathlineto{\pgfqpoint{2.198846in}{1.625469in}}%
\pgfpathlineto{\pgfqpoint{2.199710in}{1.695756in}}%
\pgfpathlineto{\pgfqpoint{2.200575in}{1.678659in}}%
\pgfpathlineto{\pgfqpoint{2.201440in}{1.712853in}}%
\pgfpathlineto{\pgfqpoint{2.202305in}{1.628853in}}%
\pgfpathlineto{\pgfqpoint{2.203171in}{1.666430in}}%
\pgfpathlineto{\pgfqpoint{2.204035in}{1.661711in}}%
\pgfpathlineto{\pgfqpoint{2.204900in}{1.651888in}}%
\pgfpathlineto{\pgfqpoint{2.207495in}{1.703178in}}%
\pgfpathlineto{\pgfqpoint{2.208361in}{1.647849in}}%
\pgfpathlineto{\pgfqpoint{2.209225in}{1.655003in}}%
\pgfpathlineto{\pgfqpoint{2.210090in}{1.677354in}}%
\pgfpathlineto{\pgfqpoint{2.210954in}{1.675633in}}%
\pgfpathlineto{\pgfqpoint{2.211819in}{1.646723in}}%
\pgfpathlineto{\pgfqpoint{2.212685in}{1.712675in}}%
\pgfpathlineto{\pgfqpoint{2.214415in}{1.655152in}}%
\pgfpathlineto{\pgfqpoint{2.215281in}{1.697894in}}%
\pgfpathlineto{\pgfqpoint{2.216146in}{1.694749in}}%
\pgfpathlineto{\pgfqpoint{2.218741in}{1.639152in}}%
\pgfpathlineto{\pgfqpoint{2.220471in}{1.682043in}}%
\pgfpathlineto{\pgfqpoint{2.221337in}{1.592937in}}%
\pgfpathlineto{\pgfqpoint{2.222202in}{1.687148in}}%
\pgfpathlineto{\pgfqpoint{2.223065in}{1.615407in}}%
\pgfpathlineto{\pgfqpoint{2.223931in}{1.693856in}}%
\pgfpathlineto{\pgfqpoint{2.226526in}{1.634136in}}%
\pgfpathlineto{\pgfqpoint{2.227389in}{1.703948in}}%
\pgfpathlineto{\pgfqpoint{2.228254in}{1.659841in}}%
\pgfpathlineto{\pgfqpoint{2.229986in}{1.711429in}}%
\pgfpathlineto{\pgfqpoint{2.230853in}{1.709648in}}%
\pgfpathlineto{\pgfqpoint{2.231717in}{1.702583in}}%
\pgfpathlineto{\pgfqpoint{2.232580in}{1.704513in}}%
\pgfpathlineto{\pgfqpoint{2.234309in}{1.592878in}}%
\pgfpathlineto{\pgfqpoint{2.235174in}{1.671981in}}%
\pgfpathlineto{\pgfqpoint{2.236039in}{1.641915in}}%
\pgfpathlineto{\pgfqpoint{2.236904in}{1.713567in}}%
\pgfpathlineto{\pgfqpoint{2.239497in}{1.648920in}}%
\pgfpathlineto{\pgfqpoint{2.240361in}{1.638829in}}%
\pgfpathlineto{\pgfqpoint{2.241226in}{1.609681in}}%
\pgfpathlineto{\pgfqpoint{2.242955in}{1.710362in}}%
\pgfpathlineto{\pgfqpoint{2.243820in}{1.633367in}}%
\pgfpathlineto{\pgfqpoint{2.245550in}{1.721818in}}%
\pgfpathlineto{\pgfqpoint{2.246416in}{1.648563in}}%
\pgfpathlineto{\pgfqpoint{2.247281in}{1.655539in}}%
\pgfpathlineto{\pgfqpoint{2.248147in}{1.700151in}}%
\pgfpathlineto{\pgfqpoint{2.250740in}{1.647080in}}%
\pgfpathlineto{\pgfqpoint{2.253336in}{1.713329in}}%
\pgfpathlineto{\pgfqpoint{2.254201in}{1.628763in}}%
\pgfpathlineto{\pgfqpoint{2.255930in}{1.682281in}}%
\pgfpathlineto{\pgfqpoint{2.256794in}{1.631199in}}%
\pgfpathlineto{\pgfqpoint{2.258523in}{1.720513in}}%
\pgfpathlineto{\pgfqpoint{2.259389in}{1.710064in}}%
\pgfpathlineto{\pgfqpoint{2.260254in}{1.629329in}}%
\pgfpathlineto{\pgfqpoint{2.261984in}{1.718018in}}%
\pgfpathlineto{\pgfqpoint{2.263712in}{1.692194in}}%
\pgfpathlineto{\pgfqpoint{2.265441in}{1.736301in}}%
\pgfpathlineto{\pgfqpoint{2.266306in}{1.686583in}}%
\pgfpathlineto{\pgfqpoint{2.267171in}{1.753041in}}%
\pgfpathlineto{\pgfqpoint{2.268898in}{1.661146in}}%
\pgfpathlineto{\pgfqpoint{2.269763in}{1.709469in}}%
\pgfpathlineto{\pgfqpoint{2.270626in}{1.675514in}}%
\pgfpathlineto{\pgfqpoint{2.271490in}{1.682459in}}%
\pgfpathlineto{\pgfqpoint{2.273219in}{1.728942in}}%
\pgfpathlineto{\pgfqpoint{2.274949in}{1.758979in}}%
\pgfpathlineto{\pgfqpoint{2.275812in}{1.690502in}}%
\pgfpathlineto{\pgfqpoint{2.276676in}{1.705967in}}%
\pgfpathlineto{\pgfqpoint{2.277541in}{1.762005in}}%
\pgfpathlineto{\pgfqpoint{2.280137in}{1.646158in}}%
\pgfpathlineto{\pgfqpoint{2.281003in}{1.669754in}}%
\pgfpathlineto{\pgfqpoint{2.281868in}{1.631050in}}%
\pgfpathlineto{\pgfqpoint{2.283598in}{1.680619in}}%
\pgfpathlineto{\pgfqpoint{2.285328in}{1.639807in}}%
\pgfpathlineto{\pgfqpoint{2.286192in}{1.646425in}}%
\pgfpathlineto{\pgfqpoint{2.287056in}{1.641439in}}%
\pgfpathlineto{\pgfqpoint{2.287921in}{1.519683in}}%
\pgfpathlineto{\pgfqpoint{2.289652in}{1.626982in}}%
\pgfpathlineto{\pgfqpoint{2.292249in}{1.540282in}}%
\pgfpathlineto{\pgfqpoint{2.293114in}{1.568954in}}%
\pgfpathlineto{\pgfqpoint{2.293978in}{1.555360in}}%
\pgfpathlineto{\pgfqpoint{2.295707in}{1.577324in}}%
\pgfpathlineto{\pgfqpoint{2.296571in}{1.563909in}}%
\pgfpathlineto{\pgfqpoint{2.297435in}{1.624636in}}%
\pgfpathlineto{\pgfqpoint{2.298302in}{1.577800in}}%
\pgfpathlineto{\pgfqpoint{2.300033in}{1.635620in}}%
\pgfpathlineto{\pgfqpoint{2.300900in}{1.645830in}}%
\pgfpathlineto{\pgfqpoint{2.301766in}{1.549957in}}%
\pgfpathlineto{\pgfqpoint{2.304361in}{1.617158in}}%
\pgfpathlineto{\pgfqpoint{2.305227in}{1.598578in}}%
\pgfpathlineto{\pgfqpoint{2.306092in}{1.611756in}}%
\pgfpathlineto{\pgfqpoint{2.306957in}{1.540223in}}%
\pgfpathlineto{\pgfqpoint{2.307822in}{1.559160in}}%
\pgfpathlineto{\pgfqpoint{2.308688in}{1.610242in}}%
\pgfpathlineto{\pgfqpoint{2.309554in}{1.543785in}}%
\pgfpathlineto{\pgfqpoint{2.310420in}{1.559279in}}%
\pgfpathlineto{\pgfqpoint{2.311285in}{1.601010in}}%
\pgfpathlineto{\pgfqpoint{2.312152in}{1.585694in}}%
\pgfpathlineto{\pgfqpoint{2.313883in}{1.634166in}}%
\pgfpathlineto{\pgfqpoint{2.314749in}{1.650936in}}%
\pgfpathlineto{\pgfqpoint{2.316478in}{1.568776in}}%
\pgfpathlineto{\pgfqpoint{2.317342in}{1.639063in}}%
\pgfpathlineto{\pgfqpoint{2.318206in}{1.563819in}}%
\pgfpathlineto{\pgfqpoint{2.319072in}{1.569490in}}%
\pgfpathlineto{\pgfqpoint{2.319937in}{1.569430in}}%
\pgfpathlineto{\pgfqpoint{2.321668in}{1.649809in}}%
\pgfpathlineto{\pgfqpoint{2.323399in}{1.577651in}}%
\pgfpathlineto{\pgfqpoint{2.324264in}{1.632950in}}%
\pgfpathlineto{\pgfqpoint{2.325127in}{1.617040in}}%
\pgfpathlineto{\pgfqpoint{2.325993in}{1.637639in}}%
\pgfpathlineto{\pgfqpoint{2.326858in}{1.570676in}}%
\pgfpathlineto{\pgfqpoint{2.327723in}{1.596797in}}%
\pgfpathlineto{\pgfqpoint{2.328589in}{1.670349in}}%
\pgfpathlineto{\pgfqpoint{2.330317in}{1.605554in}}%
\pgfpathlineto{\pgfqpoint{2.332043in}{1.597570in}}%
\pgfpathlineto{\pgfqpoint{2.332908in}{1.609562in}}%
\pgfpathlineto{\pgfqpoint{2.333775in}{1.581838in}}%
\pgfpathlineto{\pgfqpoint{2.334640in}{1.607008in}}%
\pgfpathlineto{\pgfqpoint{2.335506in}{1.509175in}}%
\pgfpathlineto{\pgfqpoint{2.336370in}{1.599288in}}%
\pgfpathlineto{\pgfqpoint{2.337235in}{1.562693in}}%
\pgfpathlineto{\pgfqpoint{2.338100in}{1.639182in}}%
\pgfpathlineto{\pgfqpoint{2.341556in}{1.502170in}}%
\pgfpathlineto{\pgfqpoint{2.342422in}{1.507126in}}%
\pgfpathlineto{\pgfqpoint{2.343286in}{1.533158in}}%
\pgfpathlineto{\pgfqpoint{2.344150in}{1.529774in}}%
\pgfpathlineto{\pgfqpoint{2.345881in}{1.588427in}}%
\pgfpathlineto{\pgfqpoint{2.346747in}{1.533575in}}%
\pgfpathlineto{\pgfqpoint{2.350207in}{1.597567in}}%
\pgfpathlineto{\pgfqpoint{2.353668in}{1.526034in}}%
\pgfpathlineto{\pgfqpoint{2.356262in}{1.624253in}}%
\pgfpathlineto{\pgfqpoint{2.357127in}{1.584746in}}%
\pgfpathlineto{\pgfqpoint{2.358859in}{1.608521in}}%
\pgfpathlineto{\pgfqpoint{2.359724in}{1.511373in}}%
\pgfpathlineto{\pgfqpoint{2.361451in}{1.633928in}}%
\pgfpathlineto{\pgfqpoint{2.362316in}{1.638293in}}%
\pgfpathlineto{\pgfqpoint{2.363181in}{1.613953in}}%
\pgfpathlineto{\pgfqpoint{2.364045in}{1.553936in}}%
\pgfpathlineto{\pgfqpoint{2.364909in}{1.655152in}}%
\pgfpathlineto{\pgfqpoint{2.365771in}{1.649720in}}%
\pgfpathlineto{\pgfqpoint{2.368366in}{1.568359in}}%
\pgfpathlineto{\pgfqpoint{2.369231in}{1.587713in}}%
\pgfpathlineto{\pgfqpoint{2.370096in}{1.650698in}}%
\pgfpathlineto{\pgfqpoint{2.370961in}{1.596410in}}%
\pgfpathlineto{\pgfqpoint{2.371826in}{1.603386in}}%
\pgfpathlineto{\pgfqpoint{2.372691in}{1.634553in}}%
\pgfpathlineto{\pgfqpoint{2.373557in}{1.555182in}}%
\pgfpathlineto{\pgfqpoint{2.374422in}{1.618404in}}%
\pgfpathlineto{\pgfqpoint{2.375287in}{1.613329in}}%
\pgfpathlineto{\pgfqpoint{2.376151in}{1.599943in}}%
\pgfpathlineto{\pgfqpoint{2.377017in}{1.616266in}}%
\pgfpathlineto{\pgfqpoint{2.377884in}{1.607008in}}%
\pgfpathlineto{\pgfqpoint{2.378749in}{1.567649in}}%
\pgfpathlineto{\pgfqpoint{2.380481in}{1.639926in}}%
\pgfpathlineto{\pgfqpoint{2.381346in}{1.568006in}}%
\pgfpathlineto{\pgfqpoint{2.383076in}{1.620721in}}%
\pgfpathlineto{\pgfqpoint{2.383941in}{1.551266in}}%
\pgfpathlineto{\pgfqpoint{2.384807in}{1.656372in}}%
\pgfpathlineto{\pgfqpoint{2.385670in}{1.638710in}}%
\pgfpathlineto{\pgfqpoint{2.389131in}{1.694808in}}%
\pgfpathlineto{\pgfqpoint{2.389995in}{1.636929in}}%
\pgfpathlineto{\pgfqpoint{2.390860in}{1.736840in}}%
\pgfpathlineto{\pgfqpoint{2.391725in}{1.659547in}}%
\pgfpathlineto{\pgfqpoint{2.392590in}{1.660614in}}%
\pgfpathlineto{\pgfqpoint{2.393455in}{1.670884in}}%
\pgfpathlineto{\pgfqpoint{2.394320in}{1.605970in}}%
\pgfpathlineto{\pgfqpoint{2.395187in}{1.676406in}}%
\pgfpathlineto{\pgfqpoint{2.396051in}{1.675990in}}%
\pgfpathlineto{\pgfqpoint{2.397781in}{1.649809in}}%
\pgfpathlineto{\pgfqpoint{2.398647in}{1.698311in}}%
\pgfpathlineto{\pgfqpoint{2.399511in}{1.665452in}}%
\pgfpathlineto{\pgfqpoint{2.400376in}{1.685129in}}%
\pgfpathlineto{\pgfqpoint{2.401240in}{1.675514in}}%
\pgfpathlineto{\pgfqpoint{2.402105in}{1.652895in}}%
\pgfpathlineto{\pgfqpoint{2.403834in}{1.692521in}}%
\pgfpathlineto{\pgfqpoint{2.404699in}{1.658655in}}%
\pgfpathlineto{\pgfqpoint{2.405564in}{1.684835in}}%
\pgfpathlineto{\pgfqpoint{2.407293in}{1.632831in}}%
\pgfpathlineto{\pgfqpoint{2.408157in}{1.667501in}}%
\pgfpathlineto{\pgfqpoint{2.409022in}{1.636245in}}%
\pgfpathlineto{\pgfqpoint{2.409887in}{1.662217in}}%
\pgfpathlineto{\pgfqpoint{2.410753in}{1.658238in}}%
\pgfpathlineto{\pgfqpoint{2.411618in}{1.620780in}}%
\pgfpathlineto{\pgfqpoint{2.412483in}{1.648623in}}%
\pgfpathlineto{\pgfqpoint{2.413348in}{1.644971in}}%
\pgfpathlineto{\pgfqpoint{2.415079in}{1.579760in}}%
\pgfpathlineto{\pgfqpoint{2.416808in}{1.522353in}}%
\pgfpathlineto{\pgfqpoint{2.417672in}{1.645711in}}%
\pgfpathlineto{\pgfqpoint{2.419402in}{1.555003in}}%
\pgfpathlineto{\pgfqpoint{2.421131in}{1.637579in}}%
\pgfpathlineto{\pgfqpoint{2.421995in}{1.605940in}}%
\pgfpathlineto{\pgfqpoint{2.422861in}{1.658774in}}%
\pgfpathlineto{\pgfqpoint{2.424590in}{1.574833in}}%
\pgfpathlineto{\pgfqpoint{2.425455in}{1.586735in}}%
\pgfpathlineto{\pgfqpoint{2.426320in}{1.567590in}}%
\pgfpathlineto{\pgfqpoint{2.427184in}{1.595551in}}%
\pgfpathlineto{\pgfqpoint{2.428048in}{1.543874in}}%
\pgfpathlineto{\pgfqpoint{2.428912in}{1.624521in}}%
\pgfpathlineto{\pgfqpoint{2.429777in}{1.559160in}}%
\pgfpathlineto{\pgfqpoint{2.430641in}{1.564087in}}%
\pgfpathlineto{\pgfqpoint{2.432370in}{1.531585in}}%
\pgfpathlineto{\pgfqpoint{2.433233in}{1.564384in}}%
\pgfpathlineto{\pgfqpoint{2.434097in}{1.538918in}}%
\pgfpathlineto{\pgfqpoint{2.434962in}{1.552304in}}%
\pgfpathlineto{\pgfqpoint{2.435826in}{1.589970in}}%
\pgfpathlineto{\pgfqpoint{2.436690in}{1.559755in}}%
\pgfpathlineto{\pgfqpoint{2.437555in}{1.590119in}}%
\pgfpathlineto{\pgfqpoint{2.438420in}{1.517902in}}%
\pgfpathlineto{\pgfqpoint{2.440150in}{1.622323in}}%
\pgfpathlineto{\pgfqpoint{2.441015in}{1.546812in}}%
\pgfpathlineto{\pgfqpoint{2.441880in}{1.589524in}}%
\pgfpathlineto{\pgfqpoint{2.442744in}{1.586527in}}%
\pgfpathlineto{\pgfqpoint{2.444474in}{1.534761in}}%
\pgfpathlineto{\pgfqpoint{2.445339in}{1.653133in}}%
\pgfpathlineto{\pgfqpoint{2.446201in}{1.580232in}}%
\pgfpathlineto{\pgfqpoint{2.447066in}{1.594570in}}%
\pgfpathlineto{\pgfqpoint{2.448795in}{1.611577in}}%
\pgfpathlineto{\pgfqpoint{2.449658in}{1.575305in}}%
\pgfpathlineto{\pgfqpoint{2.450524in}{1.607008in}}%
\pgfpathlineto{\pgfqpoint{2.451388in}{1.575751in}}%
\pgfpathlineto{\pgfqpoint{2.452253in}{1.638885in}}%
\pgfpathlineto{\pgfqpoint{2.453982in}{1.592937in}}%
\pgfpathlineto{\pgfqpoint{2.454848in}{1.617040in}}%
\pgfpathlineto{\pgfqpoint{2.455713in}{1.606651in}}%
\pgfpathlineto{\pgfqpoint{2.457441in}{1.655212in}}%
\pgfpathlineto{\pgfqpoint{2.458306in}{1.601545in}}%
\pgfpathlineto{\pgfqpoint{2.460035in}{1.631288in}}%
\pgfpathlineto{\pgfqpoint{2.460899in}{1.626004in}}%
\pgfpathlineto{\pgfqpoint{2.462629in}{1.696708in}}%
\pgfpathlineto{\pgfqpoint{2.464360in}{1.559160in}}%
\pgfpathlineto{\pgfqpoint{2.465225in}{1.629150in}}%
\pgfpathlineto{\pgfqpoint{2.466090in}{1.564295in}}%
\pgfpathlineto{\pgfqpoint{2.467821in}{1.608908in}}%
\pgfpathlineto{\pgfqpoint{2.468685in}{1.579938in}}%
\pgfpathlineto{\pgfqpoint{2.469550in}{1.615794in}}%
\pgfpathlineto{\pgfqpoint{2.470414in}{1.552780in}}%
\pgfpathlineto{\pgfqpoint{2.471277in}{1.565098in}}%
\pgfpathlineto{\pgfqpoint{2.472140in}{1.555777in}}%
\pgfpathlineto{\pgfqpoint{2.473871in}{1.629150in}}%
\pgfpathlineto{\pgfqpoint{2.474737in}{1.576019in}}%
\pgfpathlineto{\pgfqpoint{2.475602in}{1.617129in}}%
\pgfpathlineto{\pgfqpoint{2.476468in}{1.565630in}}%
\pgfpathlineto{\pgfqpoint{2.477334in}{1.663106in}}%
\pgfpathlineto{\pgfqpoint{2.478200in}{1.628912in}}%
\pgfpathlineto{\pgfqpoint{2.479066in}{1.656636in}}%
\pgfpathlineto{\pgfqpoint{2.479931in}{1.627488in}}%
\pgfpathlineto{\pgfqpoint{2.480797in}{1.632891in}}%
\pgfpathlineto{\pgfqpoint{2.481662in}{1.615139in}}%
\pgfpathlineto{\pgfqpoint{2.483391in}{1.644287in}}%
\pgfpathlineto{\pgfqpoint{2.485119in}{1.610867in}}%
\pgfpathlineto{\pgfqpoint{2.486849in}{1.680619in}}%
\pgfpathlineto{\pgfqpoint{2.487715in}{1.666787in}}%
\pgfpathlineto{\pgfqpoint{2.489447in}{1.545566in}}%
\pgfpathlineto{\pgfqpoint{2.490310in}{1.633898in}}%
\pgfpathlineto{\pgfqpoint{2.491175in}{1.633069in}}%
\pgfpathlineto{\pgfqpoint{2.492039in}{1.570293in}}%
\pgfpathlineto{\pgfqpoint{2.493769in}{1.641677in}}%
\pgfpathlineto{\pgfqpoint{2.494635in}{1.580176in}}%
\pgfpathlineto{\pgfqpoint{2.495501in}{1.655212in}}%
\pgfpathlineto{\pgfqpoint{2.496365in}{1.564325in}}%
\pgfpathlineto{\pgfqpoint{2.497231in}{1.625707in}}%
\pgfpathlineto{\pgfqpoint{2.498096in}{1.562960in}}%
\pgfpathlineto{\pgfqpoint{2.498962in}{1.629418in}}%
\pgfpathlineto{\pgfqpoint{2.500692in}{1.578157in}}%
\pgfpathlineto{\pgfqpoint{2.501555in}{1.604189in}}%
\pgfpathlineto{\pgfqpoint{2.502421in}{1.576198in}}%
\pgfpathlineto{\pgfqpoint{2.503285in}{1.600478in}}%
\pgfpathlineto{\pgfqpoint{2.504149in}{1.600121in}}%
\pgfpathlineto{\pgfqpoint{2.505014in}{1.609027in}}%
\pgfpathlineto{\pgfqpoint{2.505880in}{1.569192in}}%
\pgfpathlineto{\pgfqpoint{2.507611in}{1.609324in}}%
\pgfpathlineto{\pgfqpoint{2.508476in}{1.571628in}}%
\pgfpathlineto{\pgfqpoint{2.509340in}{1.606000in}}%
\pgfpathlineto{\pgfqpoint{2.510205in}{1.594127in}}%
\pgfpathlineto{\pgfqpoint{2.511069in}{1.516805in}}%
\pgfpathlineto{\pgfqpoint{2.511934in}{1.587181in}}%
\pgfpathlineto{\pgfqpoint{2.512798in}{1.532329in}}%
\pgfpathlineto{\pgfqpoint{2.514530in}{1.607662in}}%
\pgfpathlineto{\pgfqpoint{2.515396in}{1.597213in}}%
\pgfpathlineto{\pgfqpoint{2.517125in}{1.646901in}}%
\pgfpathlineto{\pgfqpoint{2.517990in}{1.513540in}}%
\pgfpathlineto{\pgfqpoint{2.519720in}{1.617694in}}%
\pgfpathlineto{\pgfqpoint{2.522317in}{1.580355in}}%
\pgfpathlineto{\pgfqpoint{2.524046in}{1.665244in}}%
\pgfpathlineto{\pgfqpoint{2.524912in}{1.651947in}}%
\pgfpathlineto{\pgfqpoint{2.525777in}{1.621316in}}%
\pgfpathlineto{\pgfqpoint{2.526643in}{1.643280in}}%
\pgfpathlineto{\pgfqpoint{2.527509in}{1.597303in}}%
\pgfpathlineto{\pgfqpoint{2.528374in}{1.659547in}}%
\pgfpathlineto{\pgfqpoint{2.530102in}{1.556253in}}%
\pgfpathlineto{\pgfqpoint{2.531831in}{1.621970in}}%
\pgfpathlineto{\pgfqpoint{2.533562in}{1.543845in}}%
\pgfpathlineto{\pgfqpoint{2.534427in}{1.613065in}}%
\pgfpathlineto{\pgfqpoint{2.537021in}{1.540729in}}%
\pgfpathlineto{\pgfqpoint{2.537886in}{1.634910in}}%
\pgfpathlineto{\pgfqpoint{2.541346in}{1.540996in}}%
\pgfpathlineto{\pgfqpoint{2.542211in}{1.587241in}}%
\pgfpathlineto{\pgfqpoint{2.543077in}{1.549485in}}%
\pgfpathlineto{\pgfqpoint{2.543942in}{1.614429in}}%
\pgfpathlineto{\pgfqpoint{2.544808in}{1.569906in}}%
\pgfpathlineto{\pgfqpoint{2.545672in}{1.583322in}}%
\pgfpathlineto{\pgfqpoint{2.547403in}{1.556134in}}%
\pgfpathlineto{\pgfqpoint{2.548266in}{1.600597in}}%
\pgfpathlineto{\pgfqpoint{2.549131in}{1.588992in}}%
\pgfpathlineto{\pgfqpoint{2.549993in}{1.523781in}}%
\pgfpathlineto{\pgfqpoint{2.551724in}{1.580860in}}%
\pgfpathlineto{\pgfqpoint{2.552590in}{1.594960in}}%
\pgfpathlineto{\pgfqpoint{2.553455in}{1.594306in}}%
\pgfpathlineto{\pgfqpoint{2.554322in}{1.585400in}}%
\pgfpathlineto{\pgfqpoint{2.555188in}{1.625056in}}%
\pgfpathlineto{\pgfqpoint{2.556916in}{1.595551in}}%
\pgfpathlineto{\pgfqpoint{2.557782in}{1.650225in}}%
\pgfpathlineto{\pgfqpoint{2.558645in}{1.594276in}}%
\pgfpathlineto{\pgfqpoint{2.559511in}{1.650106in}}%
\pgfpathlineto{\pgfqpoint{2.560377in}{1.541234in}}%
\pgfpathlineto{\pgfqpoint{2.561241in}{1.568333in}}%
\pgfpathlineto{\pgfqpoint{2.562106in}{1.592465in}}%
\pgfpathlineto{\pgfqpoint{2.562970in}{1.670944in}}%
\pgfpathlineto{\pgfqpoint{2.564700in}{1.623335in}}%
\pgfpathlineto{\pgfqpoint{2.565564in}{1.665482in}}%
\pgfpathlineto{\pgfqpoint{2.566428in}{1.629567in}}%
\pgfpathlineto{\pgfqpoint{2.567292in}{1.637877in}}%
\pgfpathlineto{\pgfqpoint{2.568157in}{1.609205in}}%
\pgfpathlineto{\pgfqpoint{2.569022in}{1.627904in}}%
\pgfpathlineto{\pgfqpoint{2.569888in}{1.569728in}}%
\pgfpathlineto{\pgfqpoint{2.570753in}{1.598786in}}%
\pgfpathlineto{\pgfqpoint{2.571618in}{1.535415in}}%
\pgfpathlineto{\pgfqpoint{2.572482in}{1.611522in}}%
\pgfpathlineto{\pgfqpoint{2.573346in}{1.556134in}}%
\pgfpathlineto{\pgfqpoint{2.575939in}{1.639242in}}%
\pgfpathlineto{\pgfqpoint{2.576802in}{1.560763in}}%
\pgfpathlineto{\pgfqpoint{2.577665in}{1.598459in}}%
\pgfpathlineto{\pgfqpoint{2.578530in}{1.557498in}}%
\pgfpathlineto{\pgfqpoint{2.579394in}{1.561179in}}%
\pgfpathlineto{\pgfqpoint{2.581123in}{1.638472in}}%
\pgfpathlineto{\pgfqpoint{2.582854in}{1.558093in}}%
\pgfpathlineto{\pgfqpoint{2.585448in}{1.637996in}}%
\pgfpathlineto{\pgfqpoint{2.588042in}{1.526272in}}%
\pgfpathlineto{\pgfqpoint{2.588907in}{1.622323in}}%
\pgfpathlineto{\pgfqpoint{2.589773in}{1.558893in}}%
\pgfpathlineto{\pgfqpoint{2.590637in}{1.588844in}}%
\pgfpathlineto{\pgfqpoint{2.591500in}{1.568304in}}%
\pgfpathlineto{\pgfqpoint{2.593230in}{1.696708in}}%
\pgfpathlineto{\pgfqpoint{2.594961in}{1.626599in}}%
\pgfpathlineto{\pgfqpoint{2.597557in}{1.724610in}}%
\pgfpathlineto{\pgfqpoint{2.598421in}{1.643934in}}%
\pgfpathlineto{\pgfqpoint{2.600150in}{1.705613in}}%
\pgfpathlineto{\pgfqpoint{2.601015in}{1.706149in}}%
\pgfpathlineto{\pgfqpoint{2.601880in}{1.716240in}}%
\pgfpathlineto{\pgfqpoint{2.602742in}{1.652720in}}%
\pgfpathlineto{\pgfqpoint{2.603605in}{1.678306in}}%
\pgfpathlineto{\pgfqpoint{2.604469in}{1.618408in}}%
\pgfpathlineto{\pgfqpoint{2.605334in}{1.666077in}}%
\pgfpathlineto{\pgfqpoint{2.606197in}{1.659696in}}%
\pgfpathlineto{\pgfqpoint{2.607063in}{1.623811in}}%
\pgfpathlineto{\pgfqpoint{2.607928in}{1.679909in}}%
\pgfpathlineto{\pgfqpoint{2.608793in}{1.656431in}}%
\pgfpathlineto{\pgfqpoint{2.609657in}{1.700449in}}%
\pgfpathlineto{\pgfqpoint{2.611387in}{1.669579in}}%
\pgfpathlineto{\pgfqpoint{2.613117in}{1.712202in}}%
\pgfpathlineto{\pgfqpoint{2.613981in}{1.659547in}}%
\pgfpathlineto{\pgfqpoint{2.614846in}{1.685133in}}%
\pgfpathlineto{\pgfqpoint{2.615710in}{1.650315in}}%
\pgfpathlineto{\pgfqpoint{2.617439in}{1.707216in}}%
\pgfpathlineto{\pgfqpoint{2.618301in}{1.715645in}}%
\pgfpathlineto{\pgfqpoint{2.620031in}{1.657350in}}%
\pgfpathlineto{\pgfqpoint{2.621760in}{1.674149in}}%
\pgfpathlineto{\pgfqpoint{2.622625in}{1.658952in}}%
\pgfpathlineto{\pgfqpoint{2.623489in}{1.681154in}}%
\pgfpathlineto{\pgfqpoint{2.625219in}{1.591190in}}%
\pgfpathlineto{\pgfqpoint{2.626084in}{1.600835in}}%
\pgfpathlineto{\pgfqpoint{2.626950in}{1.686974in}}%
\pgfpathlineto{\pgfqpoint{2.627816in}{1.647318in}}%
\pgfpathlineto{\pgfqpoint{2.628682in}{1.652423in}}%
\pgfpathlineto{\pgfqpoint{2.630413in}{1.635564in}}%
\pgfpathlineto{\pgfqpoint{2.631279in}{1.681571in}}%
\pgfpathlineto{\pgfqpoint{2.632145in}{1.628767in}}%
\pgfpathlineto{\pgfqpoint{2.633874in}{1.669877in}}%
\pgfpathlineto{\pgfqpoint{2.634739in}{1.691573in}}%
\pgfpathlineto{\pgfqpoint{2.636470in}{1.646961in}}%
\pgfpathlineto{\pgfqpoint{2.637336in}{1.723658in}}%
\pgfpathlineto{\pgfqpoint{2.638200in}{1.681214in}}%
\pgfpathlineto{\pgfqpoint{2.639064in}{1.733099in}}%
\pgfpathlineto{\pgfqpoint{2.639929in}{1.633724in}}%
\pgfpathlineto{\pgfqpoint{2.642524in}{1.725677in}}%
\pgfpathlineto{\pgfqpoint{2.643388in}{1.694689in}}%
\pgfpathlineto{\pgfqpoint{2.644252in}{1.706740in}}%
\pgfpathlineto{\pgfqpoint{2.645115in}{1.640015in}}%
\pgfpathlineto{\pgfqpoint{2.645978in}{1.734047in}}%
\pgfpathlineto{\pgfqpoint{2.647710in}{1.694749in}}%
\pgfpathlineto{\pgfqpoint{2.648575in}{1.718434in}}%
\pgfpathlineto{\pgfqpoint{2.649441in}{1.693265in}}%
\pgfpathlineto{\pgfqpoint{2.650307in}{1.715051in}}%
\pgfpathlineto{\pgfqpoint{2.651172in}{1.669222in}}%
\pgfpathlineto{\pgfqpoint{2.652037in}{1.766936in}}%
\pgfpathlineto{\pgfqpoint{2.652902in}{1.682697in}}%
\pgfpathlineto{\pgfqpoint{2.653767in}{1.682935in}}%
\pgfpathlineto{\pgfqpoint{2.654633in}{1.686974in}}%
\pgfpathlineto{\pgfqpoint{2.655499in}{1.679314in}}%
\pgfpathlineto{\pgfqpoint{2.657231in}{1.768836in}}%
\pgfpathlineto{\pgfqpoint{2.658097in}{1.681452in}}%
\pgfpathlineto{\pgfqpoint{2.658961in}{1.689108in}}%
\pgfpathlineto{\pgfqpoint{2.659827in}{1.682638in}}%
\pgfpathlineto{\pgfqpoint{2.660689in}{1.633129in}}%
\pgfpathlineto{\pgfqpoint{2.662417in}{1.706324in}}%
\pgfpathlineto{\pgfqpoint{2.663283in}{1.672636in}}%
\pgfpathlineto{\pgfqpoint{2.664149in}{1.682876in}}%
\pgfpathlineto{\pgfqpoint{2.665014in}{1.629329in}}%
\pgfpathlineto{\pgfqpoint{2.665878in}{1.651977in}}%
\pgfpathlineto{\pgfqpoint{2.666743in}{1.648087in}}%
\pgfpathlineto{\pgfqpoint{2.669338in}{1.553758in}}%
\pgfpathlineto{\pgfqpoint{2.670204in}{1.637788in}}%
\pgfpathlineto{\pgfqpoint{2.672800in}{1.542182in}}%
\pgfpathlineto{\pgfqpoint{2.673666in}{1.592937in}}%
\pgfpathlineto{\pgfqpoint{2.674531in}{1.562960in}}%
\pgfpathlineto{\pgfqpoint{2.675397in}{1.640312in}}%
\pgfpathlineto{\pgfqpoint{2.676261in}{1.611224in}}%
\pgfpathlineto{\pgfqpoint{2.677125in}{1.670587in}}%
\pgfpathlineto{\pgfqpoint{2.677991in}{1.614191in}}%
\pgfpathlineto{\pgfqpoint{2.678857in}{1.619594in}}%
\pgfpathlineto{\pgfqpoint{2.679722in}{1.634850in}}%
\pgfpathlineto{\pgfqpoint{2.681452in}{1.721937in}}%
\pgfpathlineto{\pgfqpoint{2.683182in}{1.595075in}}%
\pgfpathlineto{\pgfqpoint{2.684047in}{1.592406in}}%
\pgfpathlineto{\pgfqpoint{2.684912in}{1.621107in}}%
\pgfpathlineto{\pgfqpoint{2.685777in}{1.614013in}}%
\pgfpathlineto{\pgfqpoint{2.686642in}{1.605524in}}%
\pgfpathlineto{\pgfqpoint{2.687507in}{1.620959in}}%
\pgfpathlineto{\pgfqpoint{2.688372in}{1.662276in}}%
\pgfpathlineto{\pgfqpoint{2.690102in}{1.595194in}}%
\pgfpathlineto{\pgfqpoint{2.690968in}{1.588725in}}%
\pgfpathlineto{\pgfqpoint{2.692698in}{1.633307in}}%
\pgfpathlineto{\pgfqpoint{2.693563in}{1.573111in}}%
\pgfpathlineto{\pgfqpoint{2.694429in}{1.623245in}}%
\pgfpathlineto{\pgfqpoint{2.695292in}{1.597451in}}%
\pgfpathlineto{\pgfqpoint{2.696157in}{1.643399in}}%
\pgfpathlineto{\pgfqpoint{2.697023in}{1.570914in}}%
\pgfpathlineto{\pgfqpoint{2.697889in}{1.578633in}}%
\pgfpathlineto{\pgfqpoint{2.698752in}{1.561001in}}%
\pgfpathlineto{\pgfqpoint{2.699618in}{1.573052in}}%
\pgfpathlineto{\pgfqpoint{2.701346in}{1.532388in}}%
\pgfpathlineto{\pgfqpoint{2.702209in}{1.587181in}}%
\pgfpathlineto{\pgfqpoint{2.703074in}{1.566404in}}%
\pgfpathlineto{\pgfqpoint{2.705670in}{1.648801in}}%
\pgfpathlineto{\pgfqpoint{2.707401in}{1.629567in}}%
\pgfpathlineto{\pgfqpoint{2.708268in}{1.537404in}}%
\pgfpathlineto{\pgfqpoint{2.709995in}{1.621018in}}%
\pgfpathlineto{\pgfqpoint{2.711726in}{1.597213in}}%
\pgfpathlineto{\pgfqpoint{2.713458in}{1.656904in}}%
\pgfpathlineto{\pgfqpoint{2.714324in}{1.656104in}}%
\pgfpathlineto{\pgfqpoint{2.715189in}{1.623275in}}%
\pgfpathlineto{\pgfqpoint{2.716054in}{1.637817in}}%
\pgfpathlineto{\pgfqpoint{2.716920in}{1.727102in}}%
\pgfpathlineto{\pgfqpoint{2.718650in}{1.574416in}}%
\pgfpathlineto{\pgfqpoint{2.719515in}{1.592465in}}%
\pgfpathlineto{\pgfqpoint{2.720379in}{1.571449in}}%
\pgfpathlineto{\pgfqpoint{2.721245in}{1.648444in}}%
\pgfpathlineto{\pgfqpoint{2.722110in}{1.627904in}}%
\pgfpathlineto{\pgfqpoint{2.722975in}{1.600805in}}%
\pgfpathlineto{\pgfqpoint{2.723839in}{1.680857in}}%
\pgfpathlineto{\pgfqpoint{2.725571in}{1.587241in}}%
\pgfpathlineto{\pgfqpoint{2.726436in}{1.611343in}}%
\pgfpathlineto{\pgfqpoint{2.727301in}{1.530816in}}%
\pgfpathlineto{\pgfqpoint{2.728166in}{1.533932in}}%
\pgfpathlineto{\pgfqpoint{2.729031in}{1.543845in}}%
\pgfpathlineto{\pgfqpoint{2.729896in}{1.588368in}}%
\pgfpathlineto{\pgfqpoint{2.730761in}{1.561417in}}%
\pgfpathlineto{\pgfqpoint{2.731626in}{1.610391in}}%
\pgfpathlineto{\pgfqpoint{2.732492in}{1.540401in}}%
\pgfpathlineto{\pgfqpoint{2.733357in}{1.632831in}}%
\pgfpathlineto{\pgfqpoint{2.734222in}{1.621554in}}%
\pgfpathlineto{\pgfqpoint{2.735086in}{1.636691in}}%
\pgfpathlineto{\pgfqpoint{2.735950in}{1.555896in}}%
\pgfpathlineto{\pgfqpoint{2.736813in}{1.636126in}}%
\pgfpathlineto{\pgfqpoint{2.737677in}{1.632891in}}%
\pgfpathlineto{\pgfqpoint{2.738539in}{1.545388in}}%
\pgfpathlineto{\pgfqpoint{2.739404in}{1.594391in}}%
\pgfpathlineto{\pgfqpoint{2.740269in}{1.584032in}}%
\pgfpathlineto{\pgfqpoint{2.741132in}{1.581303in}}%
\pgfpathlineto{\pgfqpoint{2.741997in}{1.601635in}}%
\pgfpathlineto{\pgfqpoint{2.742860in}{1.593175in}}%
\pgfpathlineto{\pgfqpoint{2.744591in}{1.651888in}}%
\pgfpathlineto{\pgfqpoint{2.745455in}{1.665422in}}%
\pgfpathlineto{\pgfqpoint{2.746322in}{1.583381in}}%
\pgfpathlineto{\pgfqpoint{2.747187in}{1.597094in}}%
\pgfpathlineto{\pgfqpoint{2.748053in}{1.559458in}}%
\pgfpathlineto{\pgfqpoint{2.750649in}{1.628023in}}%
\pgfpathlineto{\pgfqpoint{2.751515in}{1.545269in}}%
\pgfpathlineto{\pgfqpoint{2.753245in}{1.638353in}}%
\pgfpathlineto{\pgfqpoint{2.754111in}{1.564028in}}%
\pgfpathlineto{\pgfqpoint{2.755840in}{1.633277in}}%
\pgfpathlineto{\pgfqpoint{2.757572in}{1.581422in}}%
\pgfpathlineto{\pgfqpoint{2.758437in}{1.572814in}}%
\pgfpathlineto{\pgfqpoint{2.759302in}{1.651055in}}%
\pgfpathlineto{\pgfqpoint{2.760166in}{1.646306in}}%
\pgfpathlineto{\pgfqpoint{2.761898in}{1.504784in}}%
\pgfpathlineto{\pgfqpoint{2.764491in}{1.601307in}}%
\pgfpathlineto{\pgfqpoint{2.765356in}{1.560227in}}%
\pgfpathlineto{\pgfqpoint{2.766220in}{1.568657in}}%
\pgfpathlineto{\pgfqpoint{2.767085in}{1.562365in}}%
\pgfpathlineto{\pgfqpoint{2.767951in}{1.640636in}}%
\pgfpathlineto{\pgfqpoint{2.769681in}{1.534404in}}%
\pgfpathlineto{\pgfqpoint{2.770546in}{1.609707in}}%
\pgfpathlineto{\pgfqpoint{2.771411in}{1.546633in}}%
\pgfpathlineto{\pgfqpoint{2.772275in}{1.628615in}}%
\pgfpathlineto{\pgfqpoint{2.773139in}{1.548890in}}%
\pgfpathlineto{\pgfqpoint{2.774004in}{1.602438in}}%
\pgfpathlineto{\pgfqpoint{2.774866in}{1.536809in}}%
\pgfpathlineto{\pgfqpoint{2.775728in}{1.607008in}}%
\pgfpathlineto{\pgfqpoint{2.776593in}{1.577030in}}%
\pgfpathlineto{\pgfqpoint{2.779188in}{1.648682in}}%
\pgfpathlineto{\pgfqpoint{2.780052in}{1.557528in}}%
\pgfpathlineto{\pgfqpoint{2.780917in}{1.626837in}}%
\pgfpathlineto{\pgfqpoint{2.781780in}{1.595849in}}%
\pgfpathlineto{\pgfqpoint{2.782645in}{1.635029in}}%
\pgfpathlineto{\pgfqpoint{2.784371in}{1.566463in}}%
\pgfpathlineto{\pgfqpoint{2.786101in}{1.666965in}}%
\pgfpathlineto{\pgfqpoint{2.786965in}{1.542272in}}%
\pgfpathlineto{\pgfqpoint{2.787828in}{1.643339in}}%
\pgfpathlineto{\pgfqpoint{2.788692in}{1.628737in}}%
\pgfpathlineto{\pgfqpoint{2.789556in}{1.621375in}}%
\pgfpathlineto{\pgfqpoint{2.791286in}{1.555185in}}%
\pgfpathlineto{\pgfqpoint{2.793017in}{1.617396in}}%
\pgfpathlineto{\pgfqpoint{2.793883in}{1.576911in}}%
\pgfpathlineto{\pgfqpoint{2.794748in}{1.612767in}}%
\pgfpathlineto{\pgfqpoint{2.795613in}{1.532448in}}%
\pgfpathlineto{\pgfqpoint{2.796478in}{1.542420in}}%
\pgfpathlineto{\pgfqpoint{2.797343in}{1.623097in}}%
\pgfpathlineto{\pgfqpoint{2.799072in}{1.541974in}}%
\pgfpathlineto{\pgfqpoint{2.800803in}{1.623037in}}%
\pgfpathlineto{\pgfqpoint{2.801665in}{1.549723in}}%
\pgfpathlineto{\pgfqpoint{2.803395in}{1.601162in}}%
\pgfpathlineto{\pgfqpoint{2.804259in}{1.582433in}}%
\pgfpathlineto{\pgfqpoint{2.805989in}{1.631496in}}%
\pgfpathlineto{\pgfqpoint{2.806856in}{1.626242in}}%
\pgfpathlineto{\pgfqpoint{2.807721in}{1.641975in}}%
\pgfpathlineto{\pgfqpoint{2.809451in}{1.607959in}}%
\pgfpathlineto{\pgfqpoint{2.810316in}{1.642034in}}%
\pgfpathlineto{\pgfqpoint{2.811181in}{1.609681in}}%
\pgfpathlineto{\pgfqpoint{2.812047in}{1.647556in}}%
\pgfpathlineto{\pgfqpoint{2.814641in}{1.568125in}}%
\pgfpathlineto{\pgfqpoint{2.816368in}{1.610748in}}%
\pgfpathlineto{\pgfqpoint{2.818096in}{1.571449in}}%
\pgfpathlineto{\pgfqpoint{2.818961in}{1.589673in}}%
\pgfpathlineto{\pgfqpoint{2.819827in}{1.546276in}}%
\pgfpathlineto{\pgfqpoint{2.821555in}{1.630812in}}%
\pgfpathlineto{\pgfqpoint{2.822421in}{1.618672in}}%
\pgfpathlineto{\pgfqpoint{2.823283in}{1.622978in}}%
\pgfpathlineto{\pgfqpoint{2.824146in}{1.646425in}}%
\pgfpathlineto{\pgfqpoint{2.825008in}{1.576108in}}%
\pgfpathlineto{\pgfqpoint{2.825873in}{1.610272in}}%
\pgfpathlineto{\pgfqpoint{2.826739in}{1.605345in}}%
\pgfpathlineto{\pgfqpoint{2.827603in}{1.598191in}}%
\pgfpathlineto{\pgfqpoint{2.828468in}{1.574179in}}%
\pgfpathlineto{\pgfqpoint{2.829334in}{1.625469in}}%
\pgfpathlineto{\pgfqpoint{2.830199in}{1.600419in}}%
\pgfpathlineto{\pgfqpoint{2.831927in}{1.648563in}}%
\pgfpathlineto{\pgfqpoint{2.833654in}{1.580176in}}%
\pgfpathlineto{\pgfqpoint{2.834521in}{1.626064in}}%
\pgfpathlineto{\pgfqpoint{2.837117in}{1.564890in}}%
\pgfpathlineto{\pgfqpoint{2.837982in}{1.681809in}}%
\pgfpathlineto{\pgfqpoint{2.839712in}{1.585043in}}%
\pgfpathlineto{\pgfqpoint{2.841442in}{1.614132in}}%
\pgfpathlineto{\pgfqpoint{2.842307in}{1.517485in}}%
\pgfpathlineto{\pgfqpoint{2.843172in}{1.586170in}}%
\pgfpathlineto{\pgfqpoint{2.844037in}{1.562336in}}%
\pgfpathlineto{\pgfqpoint{2.845764in}{1.636096in}}%
\pgfpathlineto{\pgfqpoint{2.846628in}{1.606799in}}%
\pgfpathlineto{\pgfqpoint{2.847492in}{1.632891in}}%
\pgfpathlineto{\pgfqpoint{2.848356in}{1.626183in}}%
\pgfpathlineto{\pgfqpoint{2.849221in}{1.610927in}}%
\pgfpathlineto{\pgfqpoint{2.851815in}{1.696351in}}%
\pgfpathlineto{\pgfqpoint{2.852680in}{1.621375in}}%
\pgfpathlineto{\pgfqpoint{2.853545in}{1.631377in}}%
\pgfpathlineto{\pgfqpoint{2.855272in}{1.698430in}}%
\pgfpathlineto{\pgfqpoint{2.856137in}{1.659101in}}%
\pgfpathlineto{\pgfqpoint{2.857003in}{1.662871in}}%
\pgfpathlineto{\pgfqpoint{2.857868in}{1.700032in}}%
\pgfpathlineto{\pgfqpoint{2.858734in}{1.628559in}}%
\pgfpathlineto{\pgfqpoint{2.859599in}{1.702940in}}%
\pgfpathlineto{\pgfqpoint{2.860465in}{1.633218in}}%
\pgfpathlineto{\pgfqpoint{2.862193in}{1.710897in}}%
\pgfpathlineto{\pgfqpoint{2.863921in}{1.657350in}}%
\pgfpathlineto{\pgfqpoint{2.864787in}{1.722770in}}%
\pgfpathlineto{\pgfqpoint{2.865652in}{1.693622in}}%
\pgfpathlineto{\pgfqpoint{2.866517in}{1.726153in}}%
\pgfpathlineto{\pgfqpoint{2.868246in}{1.693503in}}%
\pgfpathlineto{\pgfqpoint{2.869111in}{1.732921in}}%
\pgfpathlineto{\pgfqpoint{2.869976in}{1.659458in}}%
\pgfpathlineto{\pgfqpoint{2.870840in}{1.669936in}}%
\pgfpathlineto{\pgfqpoint{2.871704in}{1.664652in}}%
\pgfpathlineto{\pgfqpoint{2.873433in}{1.698608in}}%
\pgfpathlineto{\pgfqpoint{2.874296in}{1.677533in}}%
\pgfpathlineto{\pgfqpoint{2.875161in}{1.720275in}}%
\pgfpathlineto{\pgfqpoint{2.876025in}{1.545864in}}%
\pgfpathlineto{\pgfqpoint{2.876888in}{1.578276in}}%
\pgfpathlineto{\pgfqpoint{2.877753in}{1.571152in}}%
\pgfpathlineto{\pgfqpoint{2.878619in}{1.590327in}}%
\pgfpathlineto{\pgfqpoint{2.879484in}{1.547079in}}%
\pgfpathlineto{\pgfqpoint{2.880348in}{1.557022in}}%
\pgfpathlineto{\pgfqpoint{2.881212in}{1.648444in}}%
\pgfpathlineto{\pgfqpoint{2.882076in}{1.637639in}}%
\pgfpathlineto{\pgfqpoint{2.882941in}{1.609146in}}%
\pgfpathlineto{\pgfqpoint{2.883803in}{1.648325in}}%
\pgfpathlineto{\pgfqpoint{2.885532in}{1.471419in}}%
\pgfpathlineto{\pgfqpoint{2.887263in}{1.631050in}}%
\pgfpathlineto{\pgfqpoint{2.888127in}{1.630340in}}%
\pgfpathlineto{\pgfqpoint{2.889857in}{1.587836in}}%
\pgfpathlineto{\pgfqpoint{2.890721in}{1.603271in}}%
\pgfpathlineto{\pgfqpoint{2.891587in}{1.652810in}}%
\pgfpathlineto{\pgfqpoint{2.892452in}{1.650761in}}%
\pgfpathlineto{\pgfqpoint{2.893318in}{1.530667in}}%
\pgfpathlineto{\pgfqpoint{2.895047in}{1.597094in}}%
\pgfpathlineto{\pgfqpoint{2.895913in}{1.586289in}}%
\pgfpathlineto{\pgfqpoint{2.896779in}{1.600776in}}%
\pgfpathlineto{\pgfqpoint{2.898505in}{1.578217in}}%
\pgfpathlineto{\pgfqpoint{2.899368in}{1.584805in}}%
\pgfpathlineto{\pgfqpoint{2.900234in}{1.519564in}}%
\pgfpathlineto{\pgfqpoint{2.901964in}{1.700980in}}%
\pgfpathlineto{\pgfqpoint{2.902827in}{1.577562in}}%
\pgfpathlineto{\pgfqpoint{2.903693in}{1.603118in}}%
\pgfpathlineto{\pgfqpoint{2.904557in}{1.632058in}}%
\pgfpathlineto{\pgfqpoint{2.906286in}{1.540044in}}%
\pgfpathlineto{\pgfqpoint{2.909746in}{1.660376in}}%
\pgfpathlineto{\pgfqpoint{2.910611in}{1.618940in}}%
\pgfpathlineto{\pgfqpoint{2.911476in}{1.684776in}}%
\pgfpathlineto{\pgfqpoint{2.912341in}{1.610570in}}%
\pgfpathlineto{\pgfqpoint{2.913205in}{1.634493in}}%
\pgfpathlineto{\pgfqpoint{2.914071in}{1.579522in}}%
\pgfpathlineto{\pgfqpoint{2.914936in}{1.615377in}}%
\pgfpathlineto{\pgfqpoint{2.915801in}{1.539747in}}%
\pgfpathlineto{\pgfqpoint{2.916667in}{1.584330in}}%
\pgfpathlineto{\pgfqpoint{2.917532in}{1.688602in}}%
\pgfpathlineto{\pgfqpoint{2.920128in}{1.576049in}}%
\pgfpathlineto{\pgfqpoint{2.921858in}{1.635798in}}%
\pgfpathlineto{\pgfqpoint{2.922722in}{1.567471in}}%
\pgfpathlineto{\pgfqpoint{2.923588in}{1.648325in}}%
\pgfpathlineto{\pgfqpoint{2.926183in}{1.574892in}}%
\pgfpathlineto{\pgfqpoint{2.927911in}{1.622502in}}%
\pgfpathlineto{\pgfqpoint{2.928776in}{1.632355in}}%
\pgfpathlineto{\pgfqpoint{2.931370in}{1.737015in}}%
\pgfpathlineto{\pgfqpoint{2.932235in}{1.743901in}}%
\pgfpathlineto{\pgfqpoint{2.933100in}{1.724015in}}%
\pgfpathlineto{\pgfqpoint{2.933966in}{1.677473in}}%
\pgfpathlineto{\pgfqpoint{2.934831in}{1.708640in}}%
\pgfpathlineto{\pgfqpoint{2.936558in}{1.674179in}}%
\pgfpathlineto{\pgfqpoint{2.937423in}{1.727161in}}%
\pgfpathlineto{\pgfqpoint{2.938287in}{1.718970in}}%
\pgfpathlineto{\pgfqpoint{2.941744in}{1.682935in}}%
\pgfpathlineto{\pgfqpoint{2.942610in}{1.700389in}}%
\pgfpathlineto{\pgfqpoint{2.944340in}{1.648325in}}%
\pgfpathlineto{\pgfqpoint{2.945203in}{1.710778in}}%
\pgfpathlineto{\pgfqpoint{2.946068in}{1.680236in}}%
\pgfpathlineto{\pgfqpoint{2.947799in}{1.702765in}}%
\pgfpathlineto{\pgfqpoint{2.948660in}{1.646697in}}%
\pgfpathlineto{\pgfqpoint{2.949527in}{1.711314in}}%
\pgfpathlineto{\pgfqpoint{2.951255in}{1.546221in}}%
\pgfpathlineto{\pgfqpoint{2.952121in}{1.618467in}}%
\pgfpathlineto{\pgfqpoint{2.953848in}{1.528826in}}%
\pgfpathlineto{\pgfqpoint{2.956440in}{1.656283in}}%
\pgfpathlineto{\pgfqpoint{2.957306in}{1.539513in}}%
\pgfpathlineto{\pgfqpoint{2.958170in}{1.570442in}}%
\pgfpathlineto{\pgfqpoint{2.959036in}{1.537732in}}%
\pgfpathlineto{\pgfqpoint{2.959901in}{1.575130in}}%
\pgfpathlineto{\pgfqpoint{2.960766in}{1.532478in}}%
\pgfpathlineto{\pgfqpoint{2.962497in}{1.632240in}}%
\pgfpathlineto{\pgfqpoint{2.964228in}{1.541766in}}%
\pgfpathlineto{\pgfqpoint{2.965957in}{1.593770in}}%
\pgfpathlineto{\pgfqpoint{2.966821in}{1.613303in}}%
\pgfpathlineto{\pgfqpoint{2.967687in}{1.544112in}}%
\pgfpathlineto{\pgfqpoint{2.968551in}{1.601014in}}%
\pgfpathlineto{\pgfqpoint{2.970278in}{1.549783in}}%
\pgfpathlineto{\pgfqpoint{2.971143in}{1.642212in}}%
\pgfpathlineto{\pgfqpoint{2.972009in}{1.635267in}}%
\pgfpathlineto{\pgfqpoint{2.973738in}{1.590803in}}%
\pgfpathlineto{\pgfqpoint{2.974603in}{1.607900in}}%
\pgfpathlineto{\pgfqpoint{2.975468in}{1.573349in}}%
\pgfpathlineto{\pgfqpoint{2.976333in}{1.619832in}}%
\pgfpathlineto{\pgfqpoint{2.977198in}{1.580444in}}%
\pgfpathlineto{\pgfqpoint{2.978062in}{1.583857in}}%
\pgfpathlineto{\pgfqpoint{2.978927in}{1.607483in}}%
\pgfpathlineto{\pgfqpoint{2.979791in}{1.535356in}}%
\pgfpathlineto{\pgfqpoint{2.980655in}{1.614429in}}%
\pgfpathlineto{\pgfqpoint{2.981520in}{1.535534in}}%
\pgfpathlineto{\pgfqpoint{2.982384in}{1.582195in}}%
\pgfpathlineto{\pgfqpoint{2.983249in}{1.568185in}}%
\pgfpathlineto{\pgfqpoint{2.984114in}{1.579730in}}%
\pgfpathlineto{\pgfqpoint{2.984979in}{1.561239in}}%
\pgfpathlineto{\pgfqpoint{2.985844in}{1.586943in}}%
\pgfpathlineto{\pgfqpoint{2.986710in}{1.560703in}}%
\pgfpathlineto{\pgfqpoint{2.988440in}{1.602319in}}%
\pgfpathlineto{\pgfqpoint{2.989305in}{1.560882in}}%
\pgfpathlineto{\pgfqpoint{2.990170in}{1.679314in}}%
\pgfpathlineto{\pgfqpoint{2.991898in}{1.545626in}}%
\pgfpathlineto{\pgfqpoint{2.993628in}{1.629656in}}%
\pgfpathlineto{\pgfqpoint{2.994494in}{1.629150in}}%
\pgfpathlineto{\pgfqpoint{2.995360in}{1.637579in}}%
\pgfpathlineto{\pgfqpoint{2.996226in}{1.611756in}}%
\pgfpathlineto{\pgfqpoint{2.997090in}{1.627726in}}%
\pgfpathlineto{\pgfqpoint{2.997954in}{1.617396in}}%
\pgfpathlineto{\pgfqpoint{2.998820in}{1.624134in}}%
\pgfpathlineto{\pgfqpoint{2.999686in}{1.595432in}}%
\pgfpathlineto{\pgfqpoint{3.000551in}{1.619475in}}%
\pgfpathlineto{\pgfqpoint{3.002282in}{1.589851in}}%
\pgfpathlineto{\pgfqpoint{3.004010in}{1.620721in}}%
\pgfpathlineto{\pgfqpoint{3.004875in}{1.649512in}}%
\pgfpathlineto{\pgfqpoint{3.006606in}{1.583084in}}%
\pgfpathlineto{\pgfqpoint{3.007470in}{1.676109in}}%
\pgfpathlineto{\pgfqpoint{3.008335in}{1.643547in}}%
\pgfpathlineto{\pgfqpoint{3.009201in}{1.699556in}}%
\pgfpathlineto{\pgfqpoint{3.010065in}{1.650761in}}%
\pgfpathlineto{\pgfqpoint{3.010930in}{1.693503in}}%
\pgfpathlineto{\pgfqpoint{3.011793in}{1.669579in}}%
\pgfpathlineto{\pgfqpoint{3.014390in}{1.712381in}}%
\pgfpathlineto{\pgfqpoint{3.015255in}{1.715467in}}%
\pgfpathlineto{\pgfqpoint{3.016120in}{1.647318in}}%
\pgfpathlineto{\pgfqpoint{3.016985in}{1.657588in}}%
\pgfpathlineto{\pgfqpoint{3.018714in}{1.706502in}}%
\pgfpathlineto{\pgfqpoint{3.019578in}{1.708700in}}%
\pgfpathlineto{\pgfqpoint{3.021309in}{1.660495in}}%
\pgfpathlineto{\pgfqpoint{3.022174in}{1.719386in}}%
\pgfpathlineto{\pgfqpoint{3.024765in}{1.650136in}}%
\pgfpathlineto{\pgfqpoint{3.026495in}{1.731969in}}%
\pgfpathlineto{\pgfqpoint{3.027361in}{1.728436in}}%
\pgfpathlineto{\pgfqpoint{3.029093in}{1.675990in}}%
\pgfpathlineto{\pgfqpoint{3.029959in}{1.725499in}}%
\pgfpathlineto{\pgfqpoint{3.031687in}{1.615913in}}%
\pgfpathlineto{\pgfqpoint{3.032552in}{1.694749in}}%
\pgfpathlineto{\pgfqpoint{3.033415in}{1.655688in}}%
\pgfpathlineto{\pgfqpoint{3.034281in}{1.731880in}}%
\pgfpathlineto{\pgfqpoint{3.035146in}{1.684181in}}%
\pgfpathlineto{\pgfqpoint{3.036011in}{1.773465in}}%
\pgfpathlineto{\pgfqpoint{3.038605in}{1.540401in}}%
\pgfpathlineto{\pgfqpoint{3.040334in}{1.616802in}}%
\pgfpathlineto{\pgfqpoint{3.041199in}{1.709826in}}%
\pgfpathlineto{\pgfqpoint{3.042064in}{1.678778in}}%
\pgfpathlineto{\pgfqpoint{3.042930in}{1.709469in}}%
\pgfpathlineto{\pgfqpoint{3.043794in}{1.654264in}}%
\pgfpathlineto{\pgfqpoint{3.045523in}{1.715407in}}%
\pgfpathlineto{\pgfqpoint{3.046388in}{1.682697in}}%
\pgfpathlineto{\pgfqpoint{3.047252in}{1.713269in}}%
\pgfpathlineto{\pgfqpoint{3.048117in}{1.620661in}}%
\pgfpathlineto{\pgfqpoint{3.048983in}{1.708164in}}%
\pgfpathlineto{\pgfqpoint{3.049848in}{1.687327in}}%
\pgfpathlineto{\pgfqpoint{3.050713in}{1.707629in}}%
\pgfpathlineto{\pgfqpoint{3.051578in}{1.644466in}}%
\pgfpathlineto{\pgfqpoint{3.052443in}{1.702821in}}%
\pgfpathlineto{\pgfqpoint{3.053308in}{1.699140in}}%
\pgfpathlineto{\pgfqpoint{3.054173in}{1.726685in}}%
\pgfpathlineto{\pgfqpoint{3.055037in}{1.642744in}}%
\pgfpathlineto{\pgfqpoint{3.056766in}{1.695697in}}%
\pgfpathlineto{\pgfqpoint{3.057631in}{1.695756in}}%
\pgfpathlineto{\pgfqpoint{3.058495in}{1.767289in}}%
\pgfpathlineto{\pgfqpoint{3.061953in}{1.646663in}}%
\pgfpathlineto{\pgfqpoint{3.062818in}{1.721699in}}%
\pgfpathlineto{\pgfqpoint{3.063682in}{1.682757in}}%
\pgfpathlineto{\pgfqpoint{3.064545in}{1.689346in}}%
\pgfpathlineto{\pgfqpoint{3.065410in}{1.663939in}}%
\pgfpathlineto{\pgfqpoint{3.066275in}{1.670527in}}%
\pgfpathlineto{\pgfqpoint{3.067139in}{1.719680in}}%
\pgfpathlineto{\pgfqpoint{3.068004in}{1.701337in}}%
\pgfpathlineto{\pgfqpoint{3.068870in}{1.632950in}}%
\pgfpathlineto{\pgfqpoint{3.071465in}{1.714872in}}%
\pgfpathlineto{\pgfqpoint{3.072330in}{1.704900in}}%
\pgfpathlineto{\pgfqpoint{3.074059in}{1.765571in}}%
\pgfpathlineto{\pgfqpoint{3.074926in}{1.748028in}}%
\pgfpathlineto{\pgfqpoint{3.075790in}{1.677771in}}%
\pgfpathlineto{\pgfqpoint{3.076655in}{1.679730in}}%
\pgfpathlineto{\pgfqpoint{3.078385in}{1.694927in}}%
\pgfpathlineto{\pgfqpoint{3.080116in}{1.666846in}}%
\pgfpathlineto{\pgfqpoint{3.081847in}{1.728942in}}%
\pgfpathlineto{\pgfqpoint{3.082712in}{1.725261in}}%
\pgfpathlineto{\pgfqpoint{3.087036in}{1.586408in}}%
\pgfpathlineto{\pgfqpoint{3.088767in}{1.666936in}}%
\pgfpathlineto{\pgfqpoint{3.090498in}{1.548295in}}%
\pgfpathlineto{\pgfqpoint{3.092227in}{1.673376in}}%
\pgfpathlineto{\pgfqpoint{3.093092in}{1.656279in}}%
\pgfpathlineto{\pgfqpoint{3.093957in}{1.580827in}}%
\pgfpathlineto{\pgfqpoint{3.094822in}{1.613537in}}%
\pgfpathlineto{\pgfqpoint{3.095687in}{1.599824in}}%
\pgfpathlineto{\pgfqpoint{3.096552in}{1.618464in}}%
\pgfpathlineto{\pgfqpoint{3.097417in}{1.579581in}}%
\pgfpathlineto{\pgfqpoint{3.098282in}{1.585545in}}%
\pgfpathlineto{\pgfqpoint{3.099147in}{1.581243in}}%
\pgfpathlineto{\pgfqpoint{3.100878in}{1.612351in}}%
\pgfpathlineto{\pgfqpoint{3.101742in}{1.608729in}}%
\pgfpathlineto{\pgfqpoint{3.102605in}{1.613596in}}%
\pgfpathlineto{\pgfqpoint{3.104334in}{1.675514in}}%
\pgfpathlineto{\pgfqpoint{3.106065in}{1.583203in}}%
\pgfpathlineto{\pgfqpoint{3.106930in}{1.616032in}}%
\pgfpathlineto{\pgfqpoint{3.107795in}{1.576108in}}%
\pgfpathlineto{\pgfqpoint{3.108661in}{1.637758in}}%
\pgfpathlineto{\pgfqpoint{3.110391in}{1.586051in}}%
\pgfpathlineto{\pgfqpoint{3.111255in}{1.598400in}}%
\pgfpathlineto{\pgfqpoint{3.113850in}{1.547641in}}%
\pgfpathlineto{\pgfqpoint{3.114714in}{1.563790in}}%
\pgfpathlineto{\pgfqpoint{3.115579in}{1.610391in}}%
\pgfpathlineto{\pgfqpoint{3.117306in}{1.519058in}}%
\pgfpathlineto{\pgfqpoint{3.118172in}{1.519207in}}%
\pgfpathlineto{\pgfqpoint{3.121631in}{1.631407in}}%
\pgfpathlineto{\pgfqpoint{3.124228in}{1.554412in}}%
\pgfpathlineto{\pgfqpoint{3.125092in}{1.564801in}}%
\pgfpathlineto{\pgfqpoint{3.125956in}{1.628321in}}%
\pgfpathlineto{\pgfqpoint{3.128550in}{1.524435in}}%
\pgfpathlineto{\pgfqpoint{3.129414in}{1.619743in}}%
\pgfpathlineto{\pgfqpoint{3.132007in}{1.530697in}}%
\pgfpathlineto{\pgfqpoint{3.132874in}{1.622918in}}%
\pgfpathlineto{\pgfqpoint{3.133740in}{1.614310in}}%
\pgfpathlineto{\pgfqpoint{3.134605in}{1.557677in}}%
\pgfpathlineto{\pgfqpoint{3.135471in}{1.585519in}}%
\pgfpathlineto{\pgfqpoint{3.136337in}{1.549366in}}%
\pgfpathlineto{\pgfqpoint{3.138066in}{1.646961in}}%
\pgfpathlineto{\pgfqpoint{3.139795in}{1.587538in}}%
\pgfpathlineto{\pgfqpoint{3.141527in}{1.551118in}}%
\pgfpathlineto{\pgfqpoint{3.142392in}{1.604397in}}%
\pgfpathlineto{\pgfqpoint{3.143259in}{1.591632in}}%
\pgfpathlineto{\pgfqpoint{3.144124in}{1.583143in}}%
\pgfpathlineto{\pgfqpoint{3.144990in}{1.593651in}}%
\pgfpathlineto{\pgfqpoint{3.145857in}{1.569757in}}%
\pgfpathlineto{\pgfqpoint{3.146723in}{1.768241in}}%
\pgfpathlineto{\pgfqpoint{3.148455in}{1.674714in}}%
\pgfpathlineto{\pgfqpoint{3.149320in}{1.662395in}}%
\pgfpathlineto{\pgfqpoint{3.150186in}{1.708878in}}%
\pgfpathlineto{\pgfqpoint{3.151050in}{1.636542in}}%
\pgfpathlineto{\pgfqpoint{3.151916in}{1.731378in}}%
\pgfpathlineto{\pgfqpoint{3.152782in}{1.710600in}}%
\pgfpathlineto{\pgfqpoint{3.154510in}{1.685133in}}%
\pgfpathlineto{\pgfqpoint{3.155373in}{1.695403in}}%
\pgfpathlineto{\pgfqpoint{3.156236in}{1.731259in}}%
\pgfpathlineto{\pgfqpoint{3.157967in}{1.662544in}}%
\pgfpathlineto{\pgfqpoint{3.158829in}{1.730664in}}%
\pgfpathlineto{\pgfqpoint{3.160561in}{1.696768in}}%
\pgfpathlineto{\pgfqpoint{3.161425in}{1.698370in}}%
\pgfpathlineto{\pgfqpoint{3.162292in}{1.682995in}}%
\pgfpathlineto{\pgfqpoint{3.163156in}{1.689941in}}%
\pgfpathlineto{\pgfqpoint{3.164021in}{1.805699in}}%
\pgfpathlineto{\pgfqpoint{3.164886in}{1.689881in}}%
\pgfpathlineto{\pgfqpoint{3.166615in}{1.797686in}}%
\pgfpathlineto{\pgfqpoint{3.167480in}{1.676376in}}%
\pgfpathlineto{\pgfqpoint{3.168345in}{1.727637in}}%
\pgfpathlineto{\pgfqpoint{3.169210in}{1.715645in}}%
\pgfpathlineto{\pgfqpoint{3.170075in}{1.681482in}}%
\pgfpathlineto{\pgfqpoint{3.172667in}{1.721937in}}%
\pgfpathlineto{\pgfqpoint{3.173530in}{1.569787in}}%
\pgfpathlineto{\pgfqpoint{3.174393in}{1.598162in}}%
\pgfpathlineto{\pgfqpoint{3.175258in}{1.628853in}}%
\pgfpathlineto{\pgfqpoint{3.176124in}{1.620066in}}%
\pgfpathlineto{\pgfqpoint{3.176989in}{1.551709in}}%
\pgfpathlineto{\pgfqpoint{3.177853in}{1.635858in}}%
\pgfpathlineto{\pgfqpoint{3.178719in}{1.576138in}}%
\pgfpathlineto{\pgfqpoint{3.179583in}{1.583114in}}%
\pgfpathlineto{\pgfqpoint{3.180447in}{1.584865in}}%
\pgfpathlineto{\pgfqpoint{3.181311in}{1.581779in}}%
\pgfpathlineto{\pgfqpoint{3.182175in}{1.583084in}}%
\pgfpathlineto{\pgfqpoint{3.183039in}{1.607364in}}%
\pgfpathlineto{\pgfqpoint{3.184769in}{1.551207in}}%
\pgfpathlineto{\pgfqpoint{3.185635in}{1.564266in}}%
\pgfpathlineto{\pgfqpoint{3.186500in}{1.600210in}}%
\pgfpathlineto{\pgfqpoint{3.187365in}{1.583203in}}%
\pgfpathlineto{\pgfqpoint{3.188230in}{1.658179in}}%
\pgfpathlineto{\pgfqpoint{3.189960in}{1.554055in}}%
\pgfpathlineto{\pgfqpoint{3.190825in}{1.638349in}}%
\pgfpathlineto{\pgfqpoint{3.191688in}{1.623331in}}%
\pgfpathlineto{\pgfqpoint{3.192553in}{1.622145in}}%
\pgfpathlineto{\pgfqpoint{3.193417in}{1.543428in}}%
\pgfpathlineto{\pgfqpoint{3.196006in}{1.620661in}}%
\pgfpathlineto{\pgfqpoint{3.197734in}{1.655331in}}%
\pgfpathlineto{\pgfqpoint{3.200327in}{1.574416in}}%
\pgfpathlineto{\pgfqpoint{3.201193in}{1.575662in}}%
\pgfpathlineto{\pgfqpoint{3.202058in}{1.563730in}}%
\pgfpathlineto{\pgfqpoint{3.204651in}{1.627488in}}%
\pgfpathlineto{\pgfqpoint{3.205515in}{1.625380in}}%
\pgfpathlineto{\pgfqpoint{3.206381in}{1.622264in}}%
\pgfpathlineto{\pgfqpoint{3.207245in}{1.585222in}}%
\pgfpathlineto{\pgfqpoint{3.208109in}{1.687892in}}%
\pgfpathlineto{\pgfqpoint{3.208975in}{1.560941in}}%
\pgfpathlineto{\pgfqpoint{3.210707in}{1.614072in}}%
\pgfpathlineto{\pgfqpoint{3.211573in}{1.596857in}}%
\pgfpathlineto{\pgfqpoint{3.212439in}{1.677533in}}%
\pgfpathlineto{\pgfqpoint{3.213304in}{1.569490in}}%
\pgfpathlineto{\pgfqpoint{3.214169in}{1.627131in}}%
\pgfpathlineto{\pgfqpoint{3.215032in}{1.603743in}}%
\pgfpathlineto{\pgfqpoint{3.215896in}{1.541707in}}%
\pgfpathlineto{\pgfqpoint{3.216762in}{1.613953in}}%
\pgfpathlineto{\pgfqpoint{3.217627in}{1.596529in}}%
\pgfpathlineto{\pgfqpoint{3.218492in}{1.630872in}}%
\pgfpathlineto{\pgfqpoint{3.219357in}{1.617515in}}%
\pgfpathlineto{\pgfqpoint{3.220222in}{1.569192in}}%
\pgfpathlineto{\pgfqpoint{3.221087in}{1.610510in}}%
\pgfpathlineto{\pgfqpoint{3.221951in}{1.606383in}}%
\pgfpathlineto{\pgfqpoint{3.222816in}{1.558327in}}%
\pgfpathlineto{\pgfqpoint{3.224547in}{1.640844in}}%
\pgfpathlineto{\pgfqpoint{3.225412in}{1.508699in}}%
\pgfpathlineto{\pgfqpoint{3.226276in}{1.523305in}}%
\pgfpathlineto{\pgfqpoint{3.227141in}{1.574119in}}%
\pgfpathlineto{\pgfqpoint{3.228006in}{1.572516in}}%
\pgfpathlineto{\pgfqpoint{3.228872in}{1.560584in}}%
\pgfpathlineto{\pgfqpoint{3.230605in}{1.679373in}}%
\pgfpathlineto{\pgfqpoint{3.234064in}{1.528321in}}%
\pgfpathlineto{\pgfqpoint{3.235794in}{1.666549in}}%
\pgfpathlineto{\pgfqpoint{3.236660in}{1.635947in}}%
\pgfpathlineto{\pgfqpoint{3.237525in}{1.565809in}}%
\pgfpathlineto{\pgfqpoint{3.238390in}{1.589970in}}%
\pgfpathlineto{\pgfqpoint{3.239254in}{1.531258in}}%
\pgfpathlineto{\pgfqpoint{3.240120in}{1.613418in}}%
\pgfpathlineto{\pgfqpoint{3.240984in}{1.601545in}}%
\pgfpathlineto{\pgfqpoint{3.241849in}{1.586587in}}%
\pgfpathlineto{\pgfqpoint{3.242713in}{1.524550in}}%
\pgfpathlineto{\pgfqpoint{3.244443in}{1.652007in}}%
\pgfpathlineto{\pgfqpoint{3.245308in}{1.586825in}}%
\pgfpathlineto{\pgfqpoint{3.246173in}{1.644317in}}%
\pgfpathlineto{\pgfqpoint{3.247039in}{1.602140in}}%
\pgfpathlineto{\pgfqpoint{3.247901in}{1.664649in}}%
\pgfpathlineto{\pgfqpoint{3.248767in}{1.659900in}}%
\pgfpathlineto{\pgfqpoint{3.249633in}{1.653431in}}%
\pgfpathlineto{\pgfqpoint{3.250497in}{1.620840in}}%
\pgfpathlineto{\pgfqpoint{3.251361in}{1.644644in}}%
\pgfpathlineto{\pgfqpoint{3.252226in}{1.612529in}}%
\pgfpathlineto{\pgfqpoint{3.253089in}{1.648117in}}%
\pgfpathlineto{\pgfqpoint{3.253954in}{1.639420in}}%
\pgfpathlineto{\pgfqpoint{3.254819in}{1.656338in}}%
\pgfpathlineto{\pgfqpoint{3.255685in}{1.541736in}}%
\pgfpathlineto{\pgfqpoint{3.256550in}{1.628972in}}%
\pgfpathlineto{\pgfqpoint{3.257415in}{1.628377in}}%
\pgfpathlineto{\pgfqpoint{3.258280in}{1.633422in}}%
\pgfpathlineto{\pgfqpoint{3.259145in}{1.603326in}}%
\pgfpathlineto{\pgfqpoint{3.260010in}{1.702583in}}%
\pgfpathlineto{\pgfqpoint{3.260876in}{1.617515in}}%
\pgfpathlineto{\pgfqpoint{3.261740in}{1.711726in}}%
\pgfpathlineto{\pgfqpoint{3.262605in}{1.594986in}}%
\pgfpathlineto{\pgfqpoint{3.264334in}{1.739034in}}%
\pgfpathlineto{\pgfqpoint{3.266929in}{1.672903in}}%
\pgfpathlineto{\pgfqpoint{3.267794in}{1.714694in}}%
\pgfpathlineto{\pgfqpoint{3.268657in}{1.714634in}}%
\pgfpathlineto{\pgfqpoint{3.269522in}{1.666995in}}%
\pgfpathlineto{\pgfqpoint{3.272114in}{1.721788in}}%
\pgfpathlineto{\pgfqpoint{3.273846in}{1.680202in}}%
\pgfpathlineto{\pgfqpoint{3.274709in}{1.704305in}}%
\pgfpathlineto{\pgfqpoint{3.275573in}{1.674030in}}%
\pgfpathlineto{\pgfqpoint{3.277300in}{1.732088in}}%
\pgfpathlineto{\pgfqpoint{3.279893in}{1.700032in}}%
\pgfpathlineto{\pgfqpoint{3.280759in}{1.705732in}}%
\pgfpathlineto{\pgfqpoint{3.282489in}{1.652007in}}%
\pgfpathlineto{\pgfqpoint{3.283353in}{1.655033in}}%
\pgfpathlineto{\pgfqpoint{3.285084in}{1.704245in}}%
\pgfpathlineto{\pgfqpoint{3.285950in}{1.686260in}}%
\pgfpathlineto{\pgfqpoint{3.287681in}{1.704781in}}%
\pgfpathlineto{\pgfqpoint{3.289412in}{1.714456in}}%
\pgfpathlineto{\pgfqpoint{3.290277in}{1.708402in}}%
\pgfpathlineto{\pgfqpoint{3.291141in}{1.693116in}}%
\pgfpathlineto{\pgfqpoint{3.292006in}{1.645120in}}%
\pgfpathlineto{\pgfqpoint{3.292872in}{1.708997in}}%
\pgfpathlineto{\pgfqpoint{3.293738in}{1.658893in}}%
\pgfpathlineto{\pgfqpoint{3.294599in}{1.748355in}}%
\pgfpathlineto{\pgfqpoint{3.295464in}{1.686260in}}%
\pgfpathlineto{\pgfqpoint{3.296325in}{1.761771in}}%
\pgfpathlineto{\pgfqpoint{3.297192in}{1.670944in}}%
\pgfpathlineto{\pgfqpoint{3.298058in}{1.707246in}}%
\pgfpathlineto{\pgfqpoint{3.298922in}{1.552036in}}%
\pgfpathlineto{\pgfqpoint{3.299786in}{1.637460in}}%
\pgfpathlineto{\pgfqpoint{3.300649in}{1.632266in}}%
\pgfpathlineto{\pgfqpoint{3.301516in}{1.640785in}}%
\pgfpathlineto{\pgfqpoint{3.302379in}{1.585932in}}%
\pgfpathlineto{\pgfqpoint{3.303244in}{1.620155in}}%
\pgfpathlineto{\pgfqpoint{3.304108in}{1.559398in}}%
\pgfpathlineto{\pgfqpoint{3.304975in}{1.672071in}}%
\pgfpathlineto{\pgfqpoint{3.305841in}{1.653133in}}%
\pgfpathlineto{\pgfqpoint{3.307572in}{1.596053in}}%
\pgfpathlineto{\pgfqpoint{3.308436in}{1.611518in}}%
\pgfpathlineto{\pgfqpoint{3.310168in}{1.580440in}}%
\pgfpathlineto{\pgfqpoint{3.311899in}{1.655684in}}%
\pgfpathlineto{\pgfqpoint{3.312762in}{1.607450in}}%
\pgfpathlineto{\pgfqpoint{3.313628in}{1.656930in}}%
\pgfpathlineto{\pgfqpoint{3.315359in}{1.587594in}}%
\pgfpathlineto{\pgfqpoint{3.317088in}{1.686970in}}%
\pgfpathlineto{\pgfqpoint{3.318817in}{1.627309in}}%
\pgfpathlineto{\pgfqpoint{3.319683in}{1.647611in}}%
\pgfpathlineto{\pgfqpoint{3.322275in}{1.549779in}}%
\pgfpathlineto{\pgfqpoint{3.324004in}{1.632058in}}%
\pgfpathlineto{\pgfqpoint{3.324870in}{1.643871in}}%
\pgfpathlineto{\pgfqpoint{3.325734in}{1.622026in}}%
\pgfpathlineto{\pgfqpoint{3.326597in}{1.625796in}}%
\pgfpathlineto{\pgfqpoint{3.327461in}{1.644763in}}%
\pgfpathlineto{\pgfqpoint{3.328325in}{1.560941in}}%
\pgfpathlineto{\pgfqpoint{3.329190in}{1.568627in}}%
\pgfpathlineto{\pgfqpoint{3.330917in}{1.608015in}}%
\pgfpathlineto{\pgfqpoint{3.331782in}{1.587773in}}%
\pgfpathlineto{\pgfqpoint{3.332647in}{1.618761in}}%
\pgfpathlineto{\pgfqpoint{3.335241in}{1.569073in}}%
\pgfpathlineto{\pgfqpoint{3.336971in}{1.644168in}}%
\pgfpathlineto{\pgfqpoint{3.337833in}{1.599407in}}%
\pgfpathlineto{\pgfqpoint{3.338698in}{1.608283in}}%
\pgfpathlineto{\pgfqpoint{3.340427in}{1.531139in}}%
\pgfpathlineto{\pgfqpoint{3.343019in}{1.652598in}}%
\pgfpathlineto{\pgfqpoint{3.343883in}{1.582132in}}%
\pgfpathlineto{\pgfqpoint{3.344746in}{1.648266in}}%
\pgfpathlineto{\pgfqpoint{3.345611in}{1.549128in}}%
\pgfpathlineto{\pgfqpoint{3.346476in}{1.659782in}}%
\pgfpathlineto{\pgfqpoint{3.347340in}{1.632653in}}%
\pgfpathlineto{\pgfqpoint{3.349071in}{1.615496in}}%
\pgfpathlineto{\pgfqpoint{3.351667in}{1.565987in}}%
\pgfpathlineto{\pgfqpoint{3.353399in}{1.660555in}}%
\pgfpathlineto{\pgfqpoint{3.354265in}{1.700623in}}%
\pgfpathlineto{\pgfqpoint{3.355130in}{1.542034in}}%
\pgfpathlineto{\pgfqpoint{3.356859in}{1.639063in}}%
\pgfpathlineto{\pgfqpoint{3.358588in}{1.576436in}}%
\pgfpathlineto{\pgfqpoint{3.359454in}{1.583441in}}%
\pgfpathlineto{\pgfqpoint{3.360319in}{1.552274in}}%
\pgfpathlineto{\pgfqpoint{3.361184in}{1.661860in}}%
\pgfpathlineto{\pgfqpoint{3.363780in}{1.570676in}}%
\pgfpathlineto{\pgfqpoint{3.364643in}{1.627191in}}%
\pgfpathlineto{\pgfqpoint{3.365508in}{1.620721in}}%
\pgfpathlineto{\pgfqpoint{3.366374in}{1.595016in}}%
\pgfpathlineto{\pgfqpoint{3.367238in}{1.608045in}}%
\pgfpathlineto{\pgfqpoint{3.368103in}{1.656279in}}%
\pgfpathlineto{\pgfqpoint{3.369832in}{1.626923in}}%
\pgfpathlineto{\pgfqpoint{3.370697in}{1.524193in}}%
\pgfpathlineto{\pgfqpoint{3.373294in}{1.619356in}}%
\pgfpathlineto{\pgfqpoint{3.374159in}{1.600329in}}%
\pgfpathlineto{\pgfqpoint{3.375889in}{1.636155in}}%
\pgfpathlineto{\pgfqpoint{3.376753in}{1.633928in}}%
\pgfpathlineto{\pgfqpoint{3.377619in}{1.628023in}}%
\pgfpathlineto{\pgfqpoint{3.378486in}{1.608670in}}%
\pgfpathlineto{\pgfqpoint{3.379350in}{1.548831in}}%
\pgfpathlineto{\pgfqpoint{3.380214in}{1.567649in}}%
\pgfpathlineto{\pgfqpoint{3.381080in}{1.631760in}}%
\pgfpathlineto{\pgfqpoint{3.381944in}{1.602731in}}%
\pgfpathlineto{\pgfqpoint{3.382810in}{1.626893in}}%
\pgfpathlineto{\pgfqpoint{3.385402in}{1.592461in}}%
\pgfpathlineto{\pgfqpoint{3.387132in}{1.692075in}}%
\pgfpathlineto{\pgfqpoint{3.388863in}{1.557201in}}%
\pgfpathlineto{\pgfqpoint{3.390594in}{1.654141in}}%
\pgfpathlineto{\pgfqpoint{3.391460in}{1.637223in}}%
\pgfpathlineto{\pgfqpoint{3.392325in}{1.640368in}}%
\pgfpathlineto{\pgfqpoint{3.393191in}{1.657108in}}%
\pgfpathlineto{\pgfqpoint{3.394056in}{1.605758in}}%
\pgfpathlineto{\pgfqpoint{3.394921in}{1.633601in}}%
\pgfpathlineto{\pgfqpoint{3.395786in}{1.586259in}}%
\pgfpathlineto{\pgfqpoint{3.400110in}{1.681746in}}%
\pgfpathlineto{\pgfqpoint{3.401841in}{1.565154in}}%
\pgfpathlineto{\pgfqpoint{3.402704in}{1.538055in}}%
\pgfpathlineto{\pgfqpoint{3.404434in}{1.630277in}}%
\pgfpathlineto{\pgfqpoint{3.406164in}{1.587654in}}%
\pgfpathlineto{\pgfqpoint{3.407029in}{1.606472in}}%
\pgfpathlineto{\pgfqpoint{3.407893in}{1.589405in}}%
\pgfpathlineto{\pgfqpoint{3.408757in}{1.548355in}}%
\pgfpathlineto{\pgfqpoint{3.409621in}{1.596024in}}%
\pgfpathlineto{\pgfqpoint{3.410487in}{1.519088in}}%
\pgfpathlineto{\pgfqpoint{3.412214in}{1.628377in}}%
\pgfpathlineto{\pgfqpoint{3.413080in}{1.580767in}}%
\pgfpathlineto{\pgfqpoint{3.413944in}{1.588364in}}%
\pgfpathlineto{\pgfqpoint{3.414810in}{1.593767in}}%
\pgfpathlineto{\pgfqpoint{3.415676in}{1.583794in}}%
\pgfpathlineto{\pgfqpoint{3.416540in}{1.624933in}}%
\pgfpathlineto{\pgfqpoint{3.417405in}{1.581626in}}%
\pgfpathlineto{\pgfqpoint{3.418269in}{1.653784in}}%
\pgfpathlineto{\pgfqpoint{3.419134in}{1.648976in}}%
\pgfpathlineto{\pgfqpoint{3.419999in}{1.626358in}}%
\pgfpathlineto{\pgfqpoint{3.420865in}{1.652598in}}%
\pgfpathlineto{\pgfqpoint{3.421731in}{1.611518in}}%
\pgfpathlineto{\pgfqpoint{3.422597in}{1.666311in}}%
\pgfpathlineto{\pgfqpoint{3.425191in}{1.599348in}}%
\pgfpathlineto{\pgfqpoint{3.426054in}{1.664411in}}%
\pgfpathlineto{\pgfqpoint{3.427782in}{1.566935in}}%
\pgfpathlineto{\pgfqpoint{3.428646in}{1.636568in}}%
\pgfpathlineto{\pgfqpoint{3.429510in}{1.624815in}}%
\pgfpathlineto{\pgfqpoint{3.430376in}{1.601724in}}%
\pgfpathlineto{\pgfqpoint{3.432107in}{1.649155in}}%
\pgfpathlineto{\pgfqpoint{3.433836in}{1.626417in}}%
\pgfpathlineto{\pgfqpoint{3.434701in}{1.645652in}}%
\pgfpathlineto{\pgfqpoint{3.435566in}{1.639714in}}%
\pgfpathlineto{\pgfqpoint{3.436430in}{1.645325in}}%
\pgfpathlineto{\pgfqpoint{3.437295in}{1.602791in}}%
\pgfpathlineto{\pgfqpoint{3.438159in}{1.655862in}}%
\pgfpathlineto{\pgfqpoint{3.439025in}{1.590323in}}%
\pgfpathlineto{\pgfqpoint{3.439889in}{1.630396in}}%
\pgfpathlineto{\pgfqpoint{3.440753in}{1.626536in}}%
\pgfpathlineto{\pgfqpoint{3.441619in}{1.625290in}}%
\pgfpathlineto{\pgfqpoint{3.442483in}{1.632058in}}%
\pgfpathlineto{\pgfqpoint{3.444212in}{1.596024in}}%
\pgfpathlineto{\pgfqpoint{3.445076in}{1.625052in}}%
\pgfpathlineto{\pgfqpoint{3.445941in}{1.572397in}}%
\pgfpathlineto{\pgfqpoint{3.446805in}{1.627369in}}%
\pgfpathlineto{\pgfqpoint{3.447669in}{1.626655in}}%
\pgfpathlineto{\pgfqpoint{3.448533in}{1.601069in}}%
\pgfpathlineto{\pgfqpoint{3.450259in}{1.667586in}}%
\pgfpathlineto{\pgfqpoint{3.451124in}{1.605996in}}%
\pgfpathlineto{\pgfqpoint{3.451988in}{1.635084in}}%
\pgfpathlineto{\pgfqpoint{3.452853in}{1.592372in}}%
\pgfpathlineto{\pgfqpoint{3.454582in}{1.631998in}}%
\pgfpathlineto{\pgfqpoint{3.456310in}{1.581597in}}%
\pgfpathlineto{\pgfqpoint{3.457175in}{1.674383in}}%
\pgfpathlineto{\pgfqpoint{3.458040in}{1.619769in}}%
\pgfpathlineto{\pgfqpoint{3.458905in}{1.645116in}}%
\pgfpathlineto{\pgfqpoint{3.459770in}{1.578451in}}%
\pgfpathlineto{\pgfqpoint{3.460636in}{1.650460in}}%
\pgfpathlineto{\pgfqpoint{3.462364in}{1.627811in}}%
\pgfpathlineto{\pgfqpoint{3.463229in}{1.650222in}}%
\pgfpathlineto{\pgfqpoint{3.464094in}{1.647017in}}%
\pgfpathlineto{\pgfqpoint{3.464958in}{1.513031in}}%
\pgfpathlineto{\pgfqpoint{3.465821in}{1.551441in}}%
\pgfpathlineto{\pgfqpoint{3.467552in}{1.615671in}}%
\pgfpathlineto{\pgfqpoint{3.468416in}{1.605996in}}%
\pgfpathlineto{\pgfqpoint{3.469282in}{1.648768in}}%
\pgfpathlineto{\pgfqpoint{3.470145in}{1.611161in}}%
\pgfpathlineto{\pgfqpoint{3.471011in}{1.624815in}}%
\pgfpathlineto{\pgfqpoint{3.472743in}{1.605996in}}%
\pgfpathlineto{\pgfqpoint{3.473609in}{1.616207in}}%
\pgfpathlineto{\pgfqpoint{3.474471in}{1.561295in}}%
\pgfpathlineto{\pgfqpoint{3.475336in}{1.602434in}}%
\pgfpathlineto{\pgfqpoint{3.476201in}{1.598039in}}%
\pgfpathlineto{\pgfqpoint{3.477067in}{1.603561in}}%
\pgfpathlineto{\pgfqpoint{3.477933in}{1.624339in}}%
\pgfpathlineto{\pgfqpoint{3.478798in}{1.620925in}}%
\pgfpathlineto{\pgfqpoint{3.480525in}{1.574770in}}%
\pgfpathlineto{\pgfqpoint{3.481389in}{1.580793in}}%
\pgfpathlineto{\pgfqpoint{3.482252in}{1.605520in}}%
\pgfpathlineto{\pgfqpoint{3.483118in}{1.696645in}}%
\pgfpathlineto{\pgfqpoint{3.484847in}{1.621609in}}%
\pgfpathlineto{\pgfqpoint{3.485710in}{1.647433in}}%
\pgfpathlineto{\pgfqpoint{3.486575in}{1.594481in}}%
\pgfpathlineto{\pgfqpoint{3.487439in}{1.623271in}}%
\pgfpathlineto{\pgfqpoint{3.488304in}{1.569192in}}%
\pgfpathlineto{\pgfqpoint{3.489170in}{1.572397in}}%
\pgfpathlineto{\pgfqpoint{3.490901in}{1.650698in}}%
\pgfpathlineto{\pgfqpoint{3.491766in}{1.611756in}}%
\pgfpathlineto{\pgfqpoint{3.492631in}{1.629150in}}%
\pgfpathlineto{\pgfqpoint{3.494360in}{1.542301in}}%
\pgfpathlineto{\pgfqpoint{3.495226in}{1.569073in}}%
\pgfpathlineto{\pgfqpoint{3.496091in}{1.531258in}}%
\pgfpathlineto{\pgfqpoint{3.498684in}{1.660138in}}%
\pgfpathlineto{\pgfqpoint{3.499548in}{1.655212in}}%
\pgfpathlineto{\pgfqpoint{3.502142in}{1.593056in}}%
\pgfpathlineto{\pgfqpoint{3.503006in}{1.613091in}}%
\pgfpathlineto{\pgfqpoint{3.504734in}{1.674800in}}%
\pgfpathlineto{\pgfqpoint{3.506463in}{1.622204in}}%
\pgfpathlineto{\pgfqpoint{3.507329in}{1.646158in}}%
\pgfpathlineto{\pgfqpoint{3.508192in}{1.570854in}}%
\pgfpathlineto{\pgfqpoint{3.509923in}{1.639301in}}%
\pgfpathlineto{\pgfqpoint{3.510788in}{1.631998in}}%
\pgfpathlineto{\pgfqpoint{3.512519in}{1.550314in}}%
\pgfpathlineto{\pgfqpoint{3.513384in}{1.610034in}}%
\pgfpathlineto{\pgfqpoint{3.515113in}{1.565511in}}%
\pgfpathlineto{\pgfqpoint{3.515979in}{1.568954in}}%
\pgfpathlineto{\pgfqpoint{3.516844in}{1.620572in}}%
\pgfpathlineto{\pgfqpoint{3.517709in}{1.575484in}}%
\pgfpathlineto{\pgfqpoint{3.518574in}{1.614723in}}%
\pgfpathlineto{\pgfqpoint{3.519437in}{1.599259in}}%
\pgfpathlineto{\pgfqpoint{3.520301in}{1.553163in}}%
\pgfpathlineto{\pgfqpoint{3.523762in}{1.688691in}}%
\pgfpathlineto{\pgfqpoint{3.525491in}{1.580470in}}%
\pgfpathlineto{\pgfqpoint{3.526355in}{1.590472in}}%
\pgfpathlineto{\pgfqpoint{3.527220in}{1.629146in}}%
\pgfpathlineto{\pgfqpoint{3.529817in}{1.544852in}}%
\pgfpathlineto{\pgfqpoint{3.530682in}{1.618523in}}%
\pgfpathlineto{\pgfqpoint{3.531546in}{1.579314in}}%
\pgfpathlineto{\pgfqpoint{3.533278in}{1.636453in}}%
\pgfpathlineto{\pgfqpoint{3.534143in}{1.578038in}}%
\pgfpathlineto{\pgfqpoint{3.535006in}{1.596916in}}%
\pgfpathlineto{\pgfqpoint{3.536735in}{1.558803in}}%
\pgfpathlineto{\pgfqpoint{3.537601in}{1.584984in}}%
\pgfpathlineto{\pgfqpoint{3.538466in}{1.584508in}}%
\pgfpathlineto{\pgfqpoint{3.539331in}{1.584805in}}%
\pgfpathlineto{\pgfqpoint{3.540196in}{1.617396in}}%
\pgfpathlineto{\pgfqpoint{3.541059in}{1.578455in}}%
\pgfpathlineto{\pgfqpoint{3.541923in}{1.588427in}}%
\pgfpathlineto{\pgfqpoint{3.543652in}{1.562603in}}%
\pgfpathlineto{\pgfqpoint{3.544515in}{1.621847in}}%
\pgfpathlineto{\pgfqpoint{3.545379in}{1.555271in}}%
\pgfpathlineto{\pgfqpoint{3.547110in}{1.654855in}}%
\pgfpathlineto{\pgfqpoint{3.547975in}{1.619297in}}%
\pgfpathlineto{\pgfqpoint{3.548839in}{1.727221in}}%
\pgfpathlineto{\pgfqpoint{3.550568in}{1.595105in}}%
\pgfpathlineto{\pgfqpoint{3.551434in}{1.662039in}}%
\pgfpathlineto{\pgfqpoint{3.552299in}{1.614310in}}%
\pgfpathlineto{\pgfqpoint{3.553165in}{1.624402in}}%
\pgfpathlineto{\pgfqpoint{3.554896in}{1.585728in}}%
\pgfpathlineto{\pgfqpoint{3.557492in}{1.686141in}}%
\pgfpathlineto{\pgfqpoint{3.558357in}{1.639896in}}%
\pgfpathlineto{\pgfqpoint{3.559222in}{1.663879in}}%
\pgfpathlineto{\pgfqpoint{3.560087in}{1.533158in}}%
\pgfpathlineto{\pgfqpoint{3.561816in}{1.641201in}}%
\pgfpathlineto{\pgfqpoint{3.562682in}{1.627488in}}%
\pgfpathlineto{\pgfqpoint{3.563548in}{1.579938in}}%
\pgfpathlineto{\pgfqpoint{3.565276in}{1.663109in}}%
\pgfpathlineto{\pgfqpoint{3.566138in}{1.712083in}}%
\pgfpathlineto{\pgfqpoint{3.567868in}{1.648801in}}%
\pgfpathlineto{\pgfqpoint{3.569599in}{1.703475in}}%
\pgfpathlineto{\pgfqpoint{3.570464in}{1.708878in}}%
\pgfpathlineto{\pgfqpoint{3.571329in}{1.677711in}}%
\pgfpathlineto{\pgfqpoint{3.572193in}{1.686200in}}%
\pgfpathlineto{\pgfqpoint{3.573058in}{1.773168in}}%
\pgfpathlineto{\pgfqpoint{3.573923in}{1.633724in}}%
\pgfpathlineto{\pgfqpoint{3.575651in}{1.738260in}}%
\pgfpathlineto{\pgfqpoint{3.576516in}{1.670587in}}%
\pgfpathlineto{\pgfqpoint{3.577381in}{1.704483in}}%
\pgfpathlineto{\pgfqpoint{3.579112in}{1.666757in}}%
\pgfpathlineto{\pgfqpoint{3.579978in}{1.689524in}}%
\pgfpathlineto{\pgfqpoint{3.580842in}{1.742001in}}%
\pgfpathlineto{\pgfqpoint{3.582573in}{1.686022in}}%
\pgfpathlineto{\pgfqpoint{3.583437in}{1.661830in}}%
\pgfpathlineto{\pgfqpoint{3.585164in}{1.683170in}}%
\pgfpathlineto{\pgfqpoint{3.586028in}{1.675246in}}%
\pgfpathlineto{\pgfqpoint{3.586892in}{1.734460in}}%
\pgfpathlineto{\pgfqpoint{3.587756in}{1.622974in}}%
\pgfpathlineto{\pgfqpoint{3.590351in}{1.807893in}}%
\pgfpathlineto{\pgfqpoint{3.591214in}{1.710596in}}%
\pgfpathlineto{\pgfqpoint{3.592078in}{1.746035in}}%
\pgfpathlineto{\pgfqpoint{3.592942in}{1.667170in}}%
\pgfpathlineto{\pgfqpoint{3.593804in}{1.689461in}}%
\pgfpathlineto{\pgfqpoint{3.594668in}{1.758384in}}%
\pgfpathlineto{\pgfqpoint{3.595532in}{1.703680in}}%
\pgfpathlineto{\pgfqpoint{3.596397in}{1.734044in}}%
\pgfpathlineto{\pgfqpoint{3.598128in}{1.669873in}}%
\pgfpathlineto{\pgfqpoint{3.598992in}{1.740160in}}%
\pgfpathlineto{\pgfqpoint{3.599857in}{1.707510in}}%
\pgfpathlineto{\pgfqpoint{3.600722in}{1.712734in}}%
\pgfpathlineto{\pgfqpoint{3.601586in}{1.690413in}}%
\pgfpathlineto{\pgfqpoint{3.603316in}{1.759157in}}%
\pgfpathlineto{\pgfqpoint{3.604181in}{1.749363in}}%
\pgfpathlineto{\pgfqpoint{3.605045in}{1.675157in}}%
\pgfpathlineto{\pgfqpoint{3.605910in}{1.713626in}}%
\pgfpathlineto{\pgfqpoint{3.606775in}{1.700092in}}%
\pgfpathlineto{\pgfqpoint{3.607639in}{1.650017in}}%
\pgfpathlineto{\pgfqpoint{3.609370in}{1.735471in}}%
\pgfpathlineto{\pgfqpoint{3.610234in}{1.721818in}}%
\pgfpathlineto{\pgfqpoint{3.613692in}{1.745801in}}%
\pgfpathlineto{\pgfqpoint{3.614557in}{1.747582in}}%
\pgfpathlineto{\pgfqpoint{3.615422in}{1.701575in}}%
\pgfpathlineto{\pgfqpoint{3.616289in}{1.754528in}}%
\pgfpathlineto{\pgfqpoint{3.617155in}{1.690800in}}%
\pgfpathlineto{\pgfqpoint{3.618017in}{1.690948in}}%
\pgfpathlineto{\pgfqpoint{3.618881in}{1.732266in}}%
\pgfpathlineto{\pgfqpoint{3.621474in}{1.670081in}}%
\pgfpathlineto{\pgfqpoint{3.623205in}{1.759098in}}%
\pgfpathlineto{\pgfqpoint{3.624067in}{1.694273in}}%
\pgfpathlineto{\pgfqpoint{3.624932in}{1.731909in}}%
\pgfpathlineto{\pgfqpoint{3.625798in}{1.697240in}}%
\pgfpathlineto{\pgfqpoint{3.626663in}{1.699051in}}%
\pgfpathlineto{\pgfqpoint{3.628394in}{1.772216in}}%
\pgfpathlineto{\pgfqpoint{3.629258in}{1.707331in}}%
\pgfpathlineto{\pgfqpoint{3.630123in}{1.801066in}}%
\pgfpathlineto{\pgfqpoint{3.631852in}{1.723361in}}%
\pgfpathlineto{\pgfqpoint{3.632716in}{1.728823in}}%
\pgfpathlineto{\pgfqpoint{3.633580in}{1.626834in}}%
\pgfpathlineto{\pgfqpoint{3.635312in}{1.803799in}}%
\pgfpathlineto{\pgfqpoint{3.636176in}{1.698991in}}%
\pgfpathlineto{\pgfqpoint{3.637043in}{1.717661in}}%
\pgfpathlineto{\pgfqpoint{3.638773in}{1.752330in}}%
\pgfpathlineto{\pgfqpoint{3.640506in}{1.683170in}}%
\pgfpathlineto{\pgfqpoint{3.641372in}{1.743603in}}%
\pgfpathlineto{\pgfqpoint{3.643098in}{1.719888in}}%
\pgfpathlineto{\pgfqpoint{3.643962in}{1.776075in}}%
\pgfpathlineto{\pgfqpoint{3.645693in}{1.681954in}}%
\pgfpathlineto{\pgfqpoint{3.647418in}{1.784743in}}%
\pgfpathlineto{\pgfqpoint{3.649149in}{1.667263in}}%
\pgfpathlineto{\pgfqpoint{3.650016in}{1.743931in}}%
\pgfpathlineto{\pgfqpoint{3.650882in}{1.728288in}}%
\pgfpathlineto{\pgfqpoint{3.652611in}{1.668360in}}%
\pgfpathlineto{\pgfqpoint{3.654343in}{1.780233in}}%
\pgfpathlineto{\pgfqpoint{3.656940in}{1.689286in}}%
\pgfpathlineto{\pgfqpoint{3.657803in}{1.785159in}}%
\pgfpathlineto{\pgfqpoint{3.658669in}{1.668389in}}%
\pgfpathlineto{\pgfqpoint{3.659533in}{1.697567in}}%
\pgfpathlineto{\pgfqpoint{3.660399in}{1.722175in}}%
\pgfpathlineto{\pgfqpoint{3.661264in}{1.667263in}}%
\pgfpathlineto{\pgfqpoint{3.662127in}{1.752628in}}%
\pgfpathlineto{\pgfqpoint{3.664721in}{1.643071in}}%
\pgfpathlineto{\pgfqpoint{3.665587in}{1.757911in}}%
\pgfpathlineto{\pgfqpoint{3.667313in}{1.675811in}}%
\pgfpathlineto{\pgfqpoint{3.668179in}{1.687803in}}%
\pgfpathlineto{\pgfqpoint{3.669045in}{1.728674in}}%
\pgfpathlineto{\pgfqpoint{3.669910in}{1.672606in}}%
\pgfpathlineto{\pgfqpoint{3.670775in}{1.682816in}}%
\pgfpathlineto{\pgfqpoint{3.672504in}{1.721758in}}%
\pgfpathlineto{\pgfqpoint{3.673368in}{1.648385in}}%
\pgfpathlineto{\pgfqpoint{3.674235in}{1.663463in}}%
\pgfpathlineto{\pgfqpoint{3.675965in}{1.711131in}}%
\pgfpathlineto{\pgfqpoint{3.676831in}{1.714277in}}%
\pgfpathlineto{\pgfqpoint{3.677697in}{1.728347in}}%
\pgfpathlineto{\pgfqpoint{3.678563in}{1.683943in}}%
\pgfpathlineto{\pgfqpoint{3.679428in}{1.744079in}}%
\pgfpathlineto{\pgfqpoint{3.680294in}{1.677711in}}%
\pgfpathlineto{\pgfqpoint{3.681160in}{1.691335in}}%
\pgfpathlineto{\pgfqpoint{3.682026in}{1.649750in}}%
\pgfpathlineto{\pgfqpoint{3.682891in}{1.692729in}}%
\pgfpathlineto{\pgfqpoint{3.683756in}{1.679225in}}%
\pgfpathlineto{\pgfqpoint{3.684621in}{1.711667in}}%
\pgfpathlineto{\pgfqpoint{3.685485in}{1.656041in}}%
\pgfpathlineto{\pgfqpoint{3.686350in}{1.728585in}}%
\pgfpathlineto{\pgfqpoint{3.688079in}{1.654438in}}%
\pgfpathlineto{\pgfqpoint{3.688943in}{1.735234in}}%
\pgfpathlineto{\pgfqpoint{3.689808in}{1.668508in}}%
\pgfpathlineto{\pgfqpoint{3.690672in}{1.669843in}}%
\pgfpathlineto{\pgfqpoint{3.691536in}{1.719501in}}%
\pgfpathlineto{\pgfqpoint{3.692401in}{1.709529in}}%
\pgfpathlineto{\pgfqpoint{3.693266in}{1.620572in}}%
\pgfpathlineto{\pgfqpoint{3.694996in}{1.709469in}}%
\pgfpathlineto{\pgfqpoint{3.695860in}{1.680202in}}%
\pgfpathlineto{\pgfqpoint{3.696724in}{1.803680in}}%
\pgfpathlineto{\pgfqpoint{3.698452in}{1.707157in}}%
\pgfpathlineto{\pgfqpoint{3.699316in}{1.732683in}}%
\pgfpathlineto{\pgfqpoint{3.700181in}{1.727221in}}%
\pgfpathlineto{\pgfqpoint{3.701046in}{1.666727in}}%
\pgfpathlineto{\pgfqpoint{3.701911in}{1.743544in}}%
\pgfpathlineto{\pgfqpoint{3.704508in}{1.682281in}}%
\pgfpathlineto{\pgfqpoint{3.705373in}{1.675514in}}%
\pgfpathlineto{\pgfqpoint{3.706238in}{1.721818in}}%
\pgfpathlineto{\pgfqpoint{3.707103in}{1.670646in}}%
\pgfpathlineto{\pgfqpoint{3.707967in}{1.675930in}}%
\pgfpathlineto{\pgfqpoint{3.708833in}{1.689524in}}%
\pgfpathlineto{\pgfqpoint{3.709698in}{1.644704in}}%
\pgfpathlineto{\pgfqpoint{3.711428in}{1.764143in}}%
\pgfpathlineto{\pgfqpoint{3.712293in}{1.544822in}}%
\pgfpathlineto{\pgfqpoint{3.713157in}{1.676343in}}%
\pgfpathlineto{\pgfqpoint{3.714022in}{1.652955in}}%
\pgfpathlineto{\pgfqpoint{3.714886in}{1.569073in}}%
\pgfpathlineto{\pgfqpoint{3.715748in}{1.632831in}}%
\pgfpathlineto{\pgfqpoint{3.716611in}{1.596827in}}%
\pgfpathlineto{\pgfqpoint{3.718343in}{1.684657in}}%
\pgfpathlineto{\pgfqpoint{3.719209in}{1.671331in}}%
\pgfpathlineto{\pgfqpoint{3.720074in}{1.713567in}}%
\pgfpathlineto{\pgfqpoint{3.722666in}{1.634493in}}%
\pgfpathlineto{\pgfqpoint{3.723531in}{1.692194in}}%
\pgfpathlineto{\pgfqpoint{3.725261in}{1.642923in}}%
\pgfpathlineto{\pgfqpoint{3.726126in}{1.636631in}}%
\pgfpathlineto{\pgfqpoint{3.726990in}{1.675692in}}%
\pgfpathlineto{\pgfqpoint{3.727855in}{1.644823in}}%
\pgfpathlineto{\pgfqpoint{3.729587in}{1.686260in}}%
\pgfpathlineto{\pgfqpoint{3.730452in}{1.619475in}}%
\pgfpathlineto{\pgfqpoint{3.731315in}{1.667025in}}%
\pgfpathlineto{\pgfqpoint{3.732179in}{1.656338in}}%
\pgfpathlineto{\pgfqpoint{3.733910in}{1.598281in}}%
\pgfpathlineto{\pgfqpoint{3.734774in}{1.687505in}}%
\pgfpathlineto{\pgfqpoint{3.736506in}{1.562901in}}%
\pgfpathlineto{\pgfqpoint{3.737372in}{1.653431in}}%
\pgfpathlineto{\pgfqpoint{3.738238in}{1.592997in}}%
\pgfpathlineto{\pgfqpoint{3.739103in}{1.629031in}}%
\pgfpathlineto{\pgfqpoint{3.739967in}{1.592818in}}%
\pgfpathlineto{\pgfqpoint{3.740831in}{1.642120in}}%
\pgfpathlineto{\pgfqpoint{3.741695in}{1.596500in}}%
\pgfpathlineto{\pgfqpoint{3.743425in}{1.663225in}}%
\pgfpathlineto{\pgfqpoint{3.745156in}{1.560793in}}%
\pgfpathlineto{\pgfqpoint{3.746887in}{1.628972in}}%
\pgfpathlineto{\pgfqpoint{3.747753in}{1.626064in}}%
\pgfpathlineto{\pgfqpoint{3.748618in}{1.688572in}}%
\pgfpathlineto{\pgfqpoint{3.750349in}{1.602880in}}%
\pgfpathlineto{\pgfqpoint{3.752078in}{1.638706in}}%
\pgfpathlineto{\pgfqpoint{3.752944in}{1.585575in}}%
\pgfpathlineto{\pgfqpoint{3.753810in}{1.628317in}}%
\pgfpathlineto{\pgfqpoint{3.754676in}{1.613180in}}%
\pgfpathlineto{\pgfqpoint{3.756406in}{1.679727in}}%
\pgfpathlineto{\pgfqpoint{3.757272in}{1.620155in}}%
\pgfpathlineto{\pgfqpoint{3.758138in}{1.666311in}}%
\pgfpathlineto{\pgfqpoint{3.759004in}{1.561949in}}%
\pgfpathlineto{\pgfqpoint{3.759869in}{1.652627in}}%
\pgfpathlineto{\pgfqpoint{3.760734in}{1.642625in}}%
\pgfpathlineto{\pgfqpoint{3.761597in}{1.573822in}}%
\pgfpathlineto{\pgfqpoint{3.762463in}{1.581303in}}%
\pgfpathlineto{\pgfqpoint{3.763326in}{1.634136in}}%
\pgfpathlineto{\pgfqpoint{3.764191in}{1.625320in}}%
\pgfpathlineto{\pgfqpoint{3.765921in}{1.648798in}}%
\pgfpathlineto{\pgfqpoint{3.766786in}{1.575097in}}%
\pgfpathlineto{\pgfqpoint{3.768513in}{1.666311in}}%
\pgfpathlineto{\pgfqpoint{3.769378in}{1.593116in}}%
\pgfpathlineto{\pgfqpoint{3.770241in}{1.678362in}}%
\pgfpathlineto{\pgfqpoint{3.771106in}{1.670940in}}%
\pgfpathlineto{\pgfqpoint{3.772835in}{1.563611in}}%
\pgfpathlineto{\pgfqpoint{3.773700in}{1.652717in}}%
\pgfpathlineto{\pgfqpoint{3.774564in}{1.620423in}}%
\pgfpathlineto{\pgfqpoint{3.776290in}{1.697270in}}%
\pgfpathlineto{\pgfqpoint{3.778019in}{1.584508in}}%
\pgfpathlineto{\pgfqpoint{3.778884in}{1.626953in}}%
\pgfpathlineto{\pgfqpoint{3.779750in}{1.625588in}}%
\pgfpathlineto{\pgfqpoint{3.780615in}{1.637817in}}%
\pgfpathlineto{\pgfqpoint{3.781479in}{1.632177in}}%
\pgfpathlineto{\pgfqpoint{3.782344in}{1.587773in}}%
\pgfpathlineto{\pgfqpoint{3.783208in}{1.599943in}}%
\pgfpathlineto{\pgfqpoint{3.784074in}{1.618107in}}%
\pgfpathlineto{\pgfqpoint{3.784940in}{1.589018in}}%
\pgfpathlineto{\pgfqpoint{3.785802in}{1.590561in}}%
\pgfpathlineto{\pgfqpoint{3.786665in}{1.606234in}}%
\pgfpathlineto{\pgfqpoint{3.787530in}{1.689818in}}%
\pgfpathlineto{\pgfqpoint{3.788396in}{1.602880in}}%
\pgfpathlineto{\pgfqpoint{3.789261in}{1.611101in}}%
\pgfpathlineto{\pgfqpoint{3.790125in}{1.592461in}}%
\pgfpathlineto{\pgfqpoint{3.790989in}{1.597805in}}%
\pgfpathlineto{\pgfqpoint{3.792718in}{1.561295in}}%
\pgfpathlineto{\pgfqpoint{3.793584in}{1.520334in}}%
\pgfpathlineto{\pgfqpoint{3.794449in}{1.629091in}}%
\pgfpathlineto{\pgfqpoint{3.795314in}{1.584449in}}%
\pgfpathlineto{\pgfqpoint{3.796179in}{1.609796in}}%
\pgfpathlineto{\pgfqpoint{3.797911in}{1.585724in}}%
\pgfpathlineto{\pgfqpoint{3.798777in}{1.601486in}}%
\pgfpathlineto{\pgfqpoint{3.800503in}{1.576908in}}%
\pgfpathlineto{\pgfqpoint{3.802234in}{1.697835in}}%
\pgfpathlineto{\pgfqpoint{3.805695in}{1.636334in}}%
\pgfpathlineto{\pgfqpoint{3.806562in}{1.724904in}}%
\pgfpathlineto{\pgfqpoint{3.807428in}{1.714307in}}%
\pgfpathlineto{\pgfqpoint{3.808293in}{1.705848in}}%
\pgfpathlineto{\pgfqpoint{3.809157in}{1.717070in}}%
\pgfpathlineto{\pgfqpoint{3.810020in}{1.674328in}}%
\pgfpathlineto{\pgfqpoint{3.810885in}{1.746217in}}%
\pgfpathlineto{\pgfqpoint{3.812616in}{1.636334in}}%
\pgfpathlineto{\pgfqpoint{3.813481in}{1.705316in}}%
\pgfpathlineto{\pgfqpoint{3.814344in}{1.639866in}}%
\pgfpathlineto{\pgfqpoint{3.815209in}{1.682459in}}%
\pgfpathlineto{\pgfqpoint{3.816076in}{1.670706in}}%
\pgfpathlineto{\pgfqpoint{3.816939in}{1.636780in}}%
\pgfpathlineto{\pgfqpoint{3.818667in}{1.697954in}}%
\pgfpathlineto{\pgfqpoint{3.819532in}{1.685903in}}%
\pgfpathlineto{\pgfqpoint{3.820398in}{1.689286in}}%
\pgfpathlineto{\pgfqpoint{3.821264in}{1.702970in}}%
\pgfpathlineto{\pgfqpoint{3.822129in}{1.695994in}}%
\pgfpathlineto{\pgfqpoint{3.822992in}{1.753695in}}%
\pgfpathlineto{\pgfqpoint{3.823858in}{1.723182in}}%
\pgfpathlineto{\pgfqpoint{3.824723in}{1.726507in}}%
\pgfpathlineto{\pgfqpoint{3.825589in}{1.705729in}}%
\pgfpathlineto{\pgfqpoint{3.826455in}{1.740220in}}%
\pgfpathlineto{\pgfqpoint{3.827320in}{1.614426in}}%
\pgfpathlineto{\pgfqpoint{3.828185in}{1.762184in}}%
\pgfpathlineto{\pgfqpoint{3.829052in}{1.693916in}}%
\pgfpathlineto{\pgfqpoint{3.829917in}{1.702464in}}%
\pgfpathlineto{\pgfqpoint{3.830781in}{1.693767in}}%
\pgfpathlineto{\pgfqpoint{3.831644in}{1.693916in}}%
\pgfpathlineto{\pgfqpoint{3.832509in}{1.733452in}}%
\pgfpathlineto{\pgfqpoint{3.833375in}{1.729980in}}%
\pgfpathlineto{\pgfqpoint{3.834240in}{1.723718in}}%
\pgfpathlineto{\pgfqpoint{3.835105in}{1.754409in}}%
\pgfpathlineto{\pgfqpoint{3.835971in}{1.709202in}}%
\pgfpathlineto{\pgfqpoint{3.836836in}{1.712972in}}%
\pgfpathlineto{\pgfqpoint{3.837698in}{1.720096in}}%
\pgfpathlineto{\pgfqpoint{3.838562in}{1.651531in}}%
\pgfpathlineto{\pgfqpoint{3.840290in}{1.701367in}}%
\pgfpathlineto{\pgfqpoint{3.841155in}{1.710953in}}%
\pgfpathlineto{\pgfqpoint{3.842018in}{1.668508in}}%
\pgfpathlineto{\pgfqpoint{3.843749in}{1.728347in}}%
\pgfpathlineto{\pgfqpoint{3.844612in}{1.716237in}}%
\pgfpathlineto{\pgfqpoint{3.845478in}{1.661354in}}%
\pgfpathlineto{\pgfqpoint{3.847206in}{1.701099in}}%
\pgfpathlineto{\pgfqpoint{3.849796in}{1.629061in}}%
\pgfpathlineto{\pgfqpoint{3.850662in}{1.697478in}}%
\pgfpathlineto{\pgfqpoint{3.851528in}{1.688929in}}%
\pgfpathlineto{\pgfqpoint{3.853258in}{1.727990in}}%
\pgfpathlineto{\pgfqpoint{3.856715in}{1.568121in}}%
\pgfpathlineto{\pgfqpoint{3.857580in}{1.585813in}}%
\pgfpathlineto{\pgfqpoint{3.858443in}{1.580172in}}%
\pgfpathlineto{\pgfqpoint{3.859307in}{1.614634in}}%
\pgfpathlineto{\pgfqpoint{3.860171in}{1.568657in}}%
\pgfpathlineto{\pgfqpoint{3.861901in}{1.628525in}}%
\pgfpathlineto{\pgfqpoint{3.862767in}{1.576432in}}%
\pgfpathlineto{\pgfqpoint{3.863631in}{1.654022in}}%
\pgfpathlineto{\pgfqpoint{3.864495in}{1.577503in}}%
\pgfpathlineto{\pgfqpoint{3.865358in}{1.702524in}}%
\pgfpathlineto{\pgfqpoint{3.866223in}{1.667025in}}%
\pgfpathlineto{\pgfqpoint{3.867087in}{1.663463in}}%
\pgfpathlineto{\pgfqpoint{3.867951in}{1.666965in}}%
\pgfpathlineto{\pgfqpoint{3.868815in}{1.748917in}}%
\pgfpathlineto{\pgfqpoint{3.869677in}{1.677295in}}%
\pgfpathlineto{\pgfqpoint{3.870542in}{1.731612in}}%
\pgfpathlineto{\pgfqpoint{3.872272in}{1.676878in}}%
\pgfpathlineto{\pgfqpoint{3.873137in}{1.706205in}}%
\pgfpathlineto{\pgfqpoint{3.874002in}{1.623896in}}%
\pgfpathlineto{\pgfqpoint{3.874867in}{1.697775in}}%
\pgfpathlineto{\pgfqpoint{3.875730in}{1.686970in}}%
\pgfpathlineto{\pgfqpoint{3.878324in}{1.638736in}}%
\pgfpathlineto{\pgfqpoint{3.879189in}{1.673138in}}%
\pgfpathlineto{\pgfqpoint{3.880052in}{1.664351in}}%
\pgfpathlineto{\pgfqpoint{3.880914in}{1.568865in}}%
\pgfpathlineto{\pgfqpoint{3.881778in}{1.630039in}}%
\pgfpathlineto{\pgfqpoint{3.882643in}{1.618166in}}%
\pgfpathlineto{\pgfqpoint{3.883507in}{1.618166in}}%
\pgfpathlineto{\pgfqpoint{3.884372in}{1.630753in}}%
\pgfpathlineto{\pgfqpoint{3.886103in}{1.570259in}}%
\pgfpathlineto{\pgfqpoint{3.887836in}{1.664143in}}%
\pgfpathlineto{\pgfqpoint{3.890434in}{1.572308in}}%
\pgfpathlineto{\pgfqpoint{3.891300in}{1.610332in}}%
\pgfpathlineto{\pgfqpoint{3.892166in}{1.584211in}}%
\pgfpathlineto{\pgfqpoint{3.893896in}{1.652479in}}%
\pgfpathlineto{\pgfqpoint{3.895627in}{1.598043in}}%
\pgfpathlineto{\pgfqpoint{3.896493in}{1.632653in}}%
\pgfpathlineto{\pgfqpoint{3.897360in}{1.584151in}}%
\pgfpathlineto{\pgfqpoint{3.898225in}{1.637520in}}%
\pgfpathlineto{\pgfqpoint{3.899958in}{1.600802in}}%
\pgfpathlineto{\pgfqpoint{3.900822in}{1.644109in}}%
\pgfpathlineto{\pgfqpoint{3.901689in}{1.599586in}}%
\pgfpathlineto{\pgfqpoint{3.904283in}{1.673316in}}%
\pgfpathlineto{\pgfqpoint{3.905147in}{1.590030in}}%
\pgfpathlineto{\pgfqpoint{3.906011in}{1.661146in}}%
\pgfpathlineto{\pgfqpoint{3.906875in}{1.562008in}}%
\pgfpathlineto{\pgfqpoint{3.907739in}{1.652241in}}%
\pgfpathlineto{\pgfqpoint{3.908603in}{1.632827in}}%
\pgfpathlineto{\pgfqpoint{3.912926in}{1.583556in}}%
\pgfpathlineto{\pgfqpoint{3.913792in}{1.588542in}}%
\pgfpathlineto{\pgfqpoint{3.914657in}{1.674562in}}%
\pgfpathlineto{\pgfqpoint{3.917253in}{1.597329in}}%
\pgfpathlineto{\pgfqpoint{3.918118in}{1.648976in}}%
\pgfpathlineto{\pgfqpoint{3.918984in}{1.580708in}}%
\pgfpathlineto{\pgfqpoint{3.919849in}{1.666549in}}%
\pgfpathlineto{\pgfqpoint{3.920712in}{1.621907in}}%
\pgfpathlineto{\pgfqpoint{3.921578in}{1.638944in}}%
\pgfpathlineto{\pgfqpoint{3.924173in}{1.583854in}}%
\pgfpathlineto{\pgfqpoint{3.925036in}{1.568597in}}%
\pgfpathlineto{\pgfqpoint{3.928493in}{1.631552in}}%
\pgfpathlineto{\pgfqpoint{3.929358in}{1.643514in}}%
\pgfpathlineto{\pgfqpoint{3.930224in}{1.585397in}}%
\pgfpathlineto{\pgfqpoint{3.932817in}{1.652122in}}%
\pgfpathlineto{\pgfqpoint{3.933682in}{1.595191in}}%
\pgfpathlineto{\pgfqpoint{3.934547in}{1.662273in}}%
\pgfpathlineto{\pgfqpoint{3.935411in}{1.655773in}}%
\pgfpathlineto{\pgfqpoint{3.936274in}{1.691540in}}%
\pgfpathlineto{\pgfqpoint{3.937139in}{1.688334in}}%
\pgfpathlineto{\pgfqpoint{3.939731in}{1.593410in}}%
\pgfpathlineto{\pgfqpoint{3.943191in}{1.668624in}}%
\pgfpathlineto{\pgfqpoint{3.944922in}{1.639476in}}%
\pgfpathlineto{\pgfqpoint{3.945788in}{1.578094in}}%
\pgfpathlineto{\pgfqpoint{3.946653in}{1.631403in}}%
\pgfpathlineto{\pgfqpoint{3.947519in}{1.526714in}}%
\pgfpathlineto{\pgfqpoint{3.949249in}{1.626120in}}%
\pgfpathlineto{\pgfqpoint{3.950114in}{1.595131in}}%
\pgfpathlineto{\pgfqpoint{3.950979in}{1.627901in}}%
\pgfpathlineto{\pgfqpoint{3.951843in}{1.591688in}}%
\pgfpathlineto{\pgfqpoint{3.953573in}{1.663935in}}%
\pgfpathlineto{\pgfqpoint{3.954436in}{1.636062in}}%
\pgfpathlineto{\pgfqpoint{3.957031in}{1.702907in}}%
\pgfpathlineto{\pgfqpoint{3.959623in}{1.639982in}}%
\pgfpathlineto{\pgfqpoint{3.960486in}{1.647136in}}%
\pgfpathlineto{\pgfqpoint{3.961351in}{1.680794in}}%
\pgfpathlineto{\pgfqpoint{3.962216in}{1.670702in}}%
\pgfpathlineto{\pgfqpoint{3.963081in}{1.698426in}}%
\pgfpathlineto{\pgfqpoint{3.963945in}{1.655297in}}%
\pgfpathlineto{\pgfqpoint{3.964811in}{1.701691in}}%
\pgfpathlineto{\pgfqpoint{3.965676in}{1.672721in}}%
\pgfpathlineto{\pgfqpoint{3.966542in}{1.719144in}}%
\pgfpathlineto{\pgfqpoint{3.967407in}{1.653784in}}%
\pgfpathlineto{\pgfqpoint{3.968272in}{1.730779in}}%
\pgfpathlineto{\pgfqpoint{3.969137in}{1.634757in}}%
\pgfpathlineto{\pgfqpoint{3.970867in}{1.708636in}}%
\pgfpathlineto{\pgfqpoint{3.971732in}{1.615314in}}%
\pgfpathlineto{\pgfqpoint{3.973460in}{1.716411in}}%
\pgfpathlineto{\pgfqpoint{3.975191in}{1.653427in}}%
\pgfpathlineto{\pgfqpoint{3.976055in}{1.680407in}}%
\pgfpathlineto{\pgfqpoint{3.978650in}{1.581478in}}%
\pgfpathlineto{\pgfqpoint{3.979516in}{1.591748in}}%
\pgfpathlineto{\pgfqpoint{3.980382in}{1.624339in}}%
\pgfpathlineto{\pgfqpoint{3.981249in}{1.586226in}}%
\pgfpathlineto{\pgfqpoint{3.982115in}{1.628730in}}%
\pgfpathlineto{\pgfqpoint{3.982981in}{1.557792in}}%
\pgfpathlineto{\pgfqpoint{3.983846in}{1.558030in}}%
\pgfpathlineto{\pgfqpoint{3.984712in}{1.585218in}}%
\pgfpathlineto{\pgfqpoint{3.985575in}{1.550906in}}%
\pgfpathlineto{\pgfqpoint{3.986438in}{1.552330in}}%
\pgfpathlineto{\pgfqpoint{3.987304in}{1.539568in}}%
\pgfpathlineto{\pgfqpoint{3.988169in}{1.578302in}}%
\pgfpathlineto{\pgfqpoint{3.989902in}{1.565273in}}%
\pgfpathlineto{\pgfqpoint{3.990765in}{1.625112in}}%
\pgfpathlineto{\pgfqpoint{3.991629in}{1.596321in}}%
\pgfpathlineto{\pgfqpoint{3.992494in}{1.614158in}}%
\pgfpathlineto{\pgfqpoint{3.993360in}{1.528466in}}%
\pgfpathlineto{\pgfqpoint{3.994223in}{1.631522in}}%
\pgfpathlineto{\pgfqpoint{3.995954in}{1.575424in}}%
\pgfpathlineto{\pgfqpoint{3.996820in}{1.676700in}}%
\pgfpathlineto{\pgfqpoint{3.997684in}{1.590621in}}%
\pgfpathlineto{\pgfqpoint{3.998549in}{1.643395in}}%
\pgfpathlineto{\pgfqpoint{4.000279in}{1.584032in}}%
\pgfpathlineto{\pgfqpoint{4.001143in}{1.611280in}}%
\pgfpathlineto{\pgfqpoint{4.002007in}{1.593707in}}%
\pgfpathlineto{\pgfqpoint{4.002874in}{1.614307in}}%
\pgfpathlineto{\pgfqpoint{4.003737in}{1.607896in}}%
\pgfpathlineto{\pgfqpoint{4.004603in}{1.565303in}}%
\pgfpathlineto{\pgfqpoint{4.006327in}{1.606651in}}%
\pgfpathlineto{\pgfqpoint{4.008057in}{1.564500in}}%
\pgfpathlineto{\pgfqpoint{4.008923in}{1.628079in}}%
\pgfpathlineto{\pgfqpoint{4.009788in}{1.527220in}}%
\pgfpathlineto{\pgfqpoint{4.010652in}{1.626536in}}%
\pgfpathlineto{\pgfqpoint{4.011517in}{1.572487in}}%
\pgfpathlineto{\pgfqpoint{4.013246in}{1.615675in}}%
\pgfpathlineto{\pgfqpoint{4.014108in}{1.552066in}}%
\pgfpathlineto{\pgfqpoint{4.014974in}{1.660908in}}%
\pgfpathlineto{\pgfqpoint{4.016704in}{1.578421in}}%
\pgfpathlineto{\pgfqpoint{4.017569in}{1.670940in}}%
\pgfpathlineto{\pgfqpoint{4.018434in}{1.661444in}}%
\pgfpathlineto{\pgfqpoint{4.019301in}{1.558803in}}%
\pgfpathlineto{\pgfqpoint{4.020166in}{1.628615in}}%
\pgfpathlineto{\pgfqpoint{4.021032in}{1.575365in}}%
\pgfpathlineto{\pgfqpoint{4.022764in}{1.614664in}}%
\pgfpathlineto{\pgfqpoint{4.023629in}{1.597180in}}%
\pgfpathlineto{\pgfqpoint{4.024494in}{1.672364in}}%
\pgfpathlineto{\pgfqpoint{4.025359in}{1.588899in}}%
\pgfpathlineto{\pgfqpoint{4.026224in}{1.592670in}}%
\pgfpathlineto{\pgfqpoint{4.027087in}{1.579934in}}%
\pgfpathlineto{\pgfqpoint{4.027953in}{1.660492in}}%
\pgfpathlineto{\pgfqpoint{4.028818in}{1.578332in}}%
\pgfpathlineto{\pgfqpoint{4.030549in}{1.632500in}}%
\pgfpathlineto{\pgfqpoint{4.032277in}{1.595369in}}%
\pgfpathlineto{\pgfqpoint{4.033143in}{1.613741in}}%
\pgfpathlineto{\pgfqpoint{4.034872in}{1.584802in}}%
\pgfpathlineto{\pgfqpoint{4.035739in}{1.557465in}}%
\pgfpathlineto{\pgfqpoint{4.037469in}{1.631582in}}%
\pgfpathlineto{\pgfqpoint{4.038334in}{1.615909in}}%
\pgfpathlineto{\pgfqpoint{4.039199in}{1.679667in}}%
\pgfpathlineto{\pgfqpoint{4.040065in}{1.620185in}}%
\pgfpathlineto{\pgfqpoint{4.040931in}{1.624933in}}%
\pgfpathlineto{\pgfqpoint{4.041797in}{1.674324in}}%
\pgfpathlineto{\pgfqpoint{4.043526in}{1.603680in}}%
\pgfpathlineto{\pgfqpoint{4.044392in}{1.611280in}}%
\pgfpathlineto{\pgfqpoint{4.045257in}{1.635114in}}%
\pgfpathlineto{\pgfqpoint{4.046121in}{1.627250in}}%
\pgfpathlineto{\pgfqpoint{4.046986in}{1.661384in}}%
\pgfpathlineto{\pgfqpoint{4.048715in}{1.632177in}}%
\pgfpathlineto{\pgfqpoint{4.049579in}{1.652271in}}%
\pgfpathlineto{\pgfqpoint{4.050444in}{1.596024in}}%
\pgfpathlineto{\pgfqpoint{4.051310in}{1.647195in}}%
\pgfpathlineto{\pgfqpoint{4.052176in}{1.633660in}}%
\pgfpathlineto{\pgfqpoint{4.053906in}{1.577384in}}%
\pgfpathlineto{\pgfqpoint{4.054771in}{1.613715in}}%
\pgfpathlineto{\pgfqpoint{4.055636in}{1.569192in}}%
\pgfpathlineto{\pgfqpoint{4.057366in}{1.702880in}}%
\pgfpathlineto{\pgfqpoint{4.058228in}{1.656695in}}%
\pgfpathlineto{\pgfqpoint{4.059092in}{1.726269in}}%
\pgfpathlineto{\pgfqpoint{4.060822in}{1.601664in}}%
\pgfpathlineto{\pgfqpoint{4.062552in}{1.581541in}}%
\pgfpathlineto{\pgfqpoint{4.063417in}{1.632712in}}%
\pgfpathlineto{\pgfqpoint{4.064282in}{1.573022in}}%
\pgfpathlineto{\pgfqpoint{4.066012in}{1.665363in}}%
\pgfpathlineto{\pgfqpoint{4.066875in}{1.599824in}}%
\pgfpathlineto{\pgfqpoint{4.067741in}{1.611994in}}%
\pgfpathlineto{\pgfqpoint{4.068604in}{1.620334in}}%
\pgfpathlineto{\pgfqpoint{4.069469in}{1.606829in}}%
\pgfpathlineto{\pgfqpoint{4.070334in}{1.607896in}}%
\pgfpathlineto{\pgfqpoint{4.071198in}{1.639242in}}%
\pgfpathlineto{\pgfqpoint{4.072063in}{1.597983in}}%
\pgfpathlineto{\pgfqpoint{4.072928in}{1.606591in}}%
\pgfpathlineto{\pgfqpoint{4.073792in}{1.572219in}}%
\pgfpathlineto{\pgfqpoint{4.074657in}{1.664589in}}%
\pgfpathlineto{\pgfqpoint{4.075522in}{1.587297in}}%
\pgfpathlineto{\pgfqpoint{4.076388in}{1.604869in}}%
\pgfpathlineto{\pgfqpoint{4.078118in}{1.664500in}}%
\pgfpathlineto{\pgfqpoint{4.078984in}{1.600831in}}%
\pgfpathlineto{\pgfqpoint{4.079849in}{1.670289in}}%
\pgfpathlineto{\pgfqpoint{4.080713in}{1.623807in}}%
\pgfpathlineto{\pgfqpoint{4.082444in}{1.759336in}}%
\pgfpathlineto{\pgfqpoint{4.083309in}{1.695994in}}%
\pgfpathlineto{\pgfqpoint{4.084175in}{1.709707in}}%
\pgfpathlineto{\pgfqpoint{4.085039in}{1.681984in}}%
\pgfpathlineto{\pgfqpoint{4.085903in}{1.696113in}}%
\pgfpathlineto{\pgfqpoint{4.086768in}{1.759038in}}%
\pgfpathlineto{\pgfqpoint{4.088498in}{1.660495in}}%
\pgfpathlineto{\pgfqpoint{4.090229in}{1.741733in}}%
\pgfpathlineto{\pgfqpoint{4.091095in}{1.737312in}}%
\pgfpathlineto{\pgfqpoint{4.091961in}{1.720632in}}%
\pgfpathlineto{\pgfqpoint{4.092826in}{1.757465in}}%
\pgfpathlineto{\pgfqpoint{4.094556in}{1.713686in}}%
\pgfpathlineto{\pgfqpoint{4.095420in}{1.734285in}}%
\pgfpathlineto{\pgfqpoint{4.096286in}{1.705256in}}%
\pgfpathlineto{\pgfqpoint{4.097151in}{1.613656in}}%
\pgfpathlineto{\pgfqpoint{4.098881in}{1.656398in}}%
\pgfpathlineto{\pgfqpoint{4.099746in}{1.620661in}}%
\pgfpathlineto{\pgfqpoint{4.101476in}{1.683408in}}%
\pgfpathlineto{\pgfqpoint{4.103206in}{1.607361in}}%
\pgfpathlineto{\pgfqpoint{4.104071in}{1.666668in}}%
\pgfpathlineto{\pgfqpoint{4.104936in}{1.666311in}}%
\pgfpathlineto{\pgfqpoint{4.106665in}{1.535947in}}%
\pgfpathlineto{\pgfqpoint{4.107530in}{1.642090in}}%
\pgfpathlineto{\pgfqpoint{4.109259in}{1.595875in}}%
\pgfpathlineto{\pgfqpoint{4.110125in}{1.585278in}}%
\pgfpathlineto{\pgfqpoint{4.110990in}{1.611458in}}%
\pgfpathlineto{\pgfqpoint{4.112719in}{1.588840in}}%
\pgfpathlineto{\pgfqpoint{4.113584in}{1.619709in}}%
\pgfpathlineto{\pgfqpoint{4.114449in}{1.605937in}}%
\pgfpathlineto{\pgfqpoint{4.115313in}{1.544614in}}%
\pgfpathlineto{\pgfqpoint{4.116177in}{1.621193in}}%
\pgfpathlineto{\pgfqpoint{4.117040in}{1.586999in}}%
\pgfpathlineto{\pgfqpoint{4.117905in}{1.599348in}}%
\pgfpathlineto{\pgfqpoint{4.118770in}{1.579905in}}%
\pgfpathlineto{\pgfqpoint{4.119636in}{1.628020in}}%
\pgfpathlineto{\pgfqpoint{4.120501in}{1.620657in}}%
\pgfpathlineto{\pgfqpoint{4.121367in}{1.608815in}}%
\pgfpathlineto{\pgfqpoint{4.122231in}{1.610982in}}%
\pgfpathlineto{\pgfqpoint{4.123096in}{1.643692in}}%
\pgfpathlineto{\pgfqpoint{4.124826in}{1.562600in}}%
\pgfpathlineto{\pgfqpoint{4.125692in}{1.591450in}}%
\pgfpathlineto{\pgfqpoint{4.126557in}{1.741287in}}%
\pgfpathlineto{\pgfqpoint{4.128288in}{1.664054in}}%
\pgfpathlineto{\pgfqpoint{4.129154in}{1.671119in}}%
\pgfpathlineto{\pgfqpoint{4.130019in}{1.719501in}}%
\pgfpathlineto{\pgfqpoint{4.130884in}{1.657911in}}%
\pgfpathlineto{\pgfqpoint{4.131746in}{1.665954in}}%
\pgfpathlineto{\pgfqpoint{4.132610in}{1.692135in}}%
\pgfpathlineto{\pgfqpoint{4.133474in}{1.659603in}}%
\pgfpathlineto{\pgfqpoint{4.134339in}{1.692789in}}%
\pgfpathlineto{\pgfqpoint{4.135203in}{1.673614in}}%
\pgfpathlineto{\pgfqpoint{4.136933in}{1.790086in}}%
\pgfpathlineto{\pgfqpoint{4.138664in}{1.686970in}}%
\pgfpathlineto{\pgfqpoint{4.139529in}{1.744258in}}%
\pgfpathlineto{\pgfqpoint{4.140395in}{1.696143in}}%
\pgfpathlineto{\pgfqpoint{4.142124in}{1.727577in}}%
\pgfpathlineto{\pgfqpoint{4.142987in}{1.679135in}}%
\pgfpathlineto{\pgfqpoint{4.143853in}{1.731021in}}%
\pgfpathlineto{\pgfqpoint{4.145583in}{1.695935in}}%
\pgfpathlineto{\pgfqpoint{4.147312in}{1.721550in}}%
\pgfpathlineto{\pgfqpoint{4.149043in}{1.658298in}}%
\pgfpathlineto{\pgfqpoint{4.149908in}{1.651293in}}%
\pgfpathlineto{\pgfqpoint{4.152507in}{1.693916in}}%
\pgfpathlineto{\pgfqpoint{4.153374in}{1.652895in}}%
\pgfpathlineto{\pgfqpoint{4.154234in}{1.709678in}}%
\pgfpathlineto{\pgfqpoint{4.155100in}{1.653074in}}%
\pgfpathlineto{\pgfqpoint{4.157692in}{1.704721in}}%
\pgfpathlineto{\pgfqpoint{4.159420in}{1.669427in}}%
\pgfpathlineto{\pgfqpoint{4.160285in}{1.743128in}}%
\pgfpathlineto{\pgfqpoint{4.161148in}{1.693737in}}%
\pgfpathlineto{\pgfqpoint{4.162013in}{1.726090in}}%
\pgfpathlineto{\pgfqpoint{4.163741in}{1.650727in}}%
\pgfpathlineto{\pgfqpoint{4.164607in}{1.696704in}}%
\pgfpathlineto{\pgfqpoint{4.167203in}{1.652419in}}%
\pgfpathlineto{\pgfqpoint{4.168932in}{1.684534in}}%
\pgfpathlineto{\pgfqpoint{4.169798in}{1.670881in}}%
\pgfpathlineto{\pgfqpoint{4.170663in}{1.636330in}}%
\pgfpathlineto{\pgfqpoint{4.171528in}{1.650400in}}%
\pgfpathlineto{\pgfqpoint{4.172393in}{1.612406in}}%
\pgfpathlineto{\pgfqpoint{4.173258in}{1.631165in}}%
\pgfpathlineto{\pgfqpoint{4.174125in}{1.618464in}}%
\pgfpathlineto{\pgfqpoint{4.174991in}{1.695280in}}%
\pgfpathlineto{\pgfqpoint{4.175858in}{1.616474in}}%
\pgfpathlineto{\pgfqpoint{4.176724in}{1.688275in}}%
\pgfpathlineto{\pgfqpoint{4.178452in}{1.621758in}}%
\pgfpathlineto{\pgfqpoint{4.180184in}{1.681508in}}%
\pgfpathlineto{\pgfqpoint{4.181050in}{1.630217in}}%
\pgfpathlineto{\pgfqpoint{4.181915in}{1.702286in}}%
\pgfpathlineto{\pgfqpoint{4.182780in}{1.664292in}}%
\pgfpathlineto{\pgfqpoint{4.183646in}{1.714039in}}%
\pgfpathlineto{\pgfqpoint{4.184511in}{1.700326in}}%
\pgfpathlineto{\pgfqpoint{4.186240in}{1.688989in}}%
\pgfpathlineto{\pgfqpoint{4.187102in}{1.654319in}}%
\pgfpathlineto{\pgfqpoint{4.187968in}{1.662957in}}%
\pgfpathlineto{\pgfqpoint{4.189700in}{1.704126in}}%
\pgfpathlineto{\pgfqpoint{4.190565in}{1.652598in}}%
\pgfpathlineto{\pgfqpoint{4.191432in}{1.693856in}}%
\pgfpathlineto{\pgfqpoint{4.193163in}{1.643454in}}%
\pgfpathlineto{\pgfqpoint{4.194030in}{1.659187in}}%
\pgfpathlineto{\pgfqpoint{4.194895in}{1.645087in}}%
\pgfpathlineto{\pgfqpoint{4.196623in}{1.662511in}}%
\pgfpathlineto{\pgfqpoint{4.197486in}{1.650460in}}%
\pgfpathlineto{\pgfqpoint{4.199218in}{1.742235in}}%
\pgfpathlineto{\pgfqpoint{4.200083in}{1.643395in}}%
\pgfpathlineto{\pgfqpoint{4.201815in}{1.717601in}}%
\pgfpathlineto{\pgfqpoint{4.202680in}{1.724607in}}%
\pgfpathlineto{\pgfqpoint{4.203545in}{1.705015in}}%
\pgfpathlineto{\pgfqpoint{4.204411in}{1.719055in}}%
\pgfpathlineto{\pgfqpoint{4.205277in}{1.751795in}}%
\pgfpathlineto{\pgfqpoint{4.207008in}{1.655624in}}%
\pgfpathlineto{\pgfqpoint{4.207872in}{1.702048in}}%
\pgfpathlineto{\pgfqpoint{4.208736in}{1.693916in}}%
\pgfpathlineto{\pgfqpoint{4.209602in}{1.657941in}}%
\pgfpathlineto{\pgfqpoint{4.210467in}{1.738022in}}%
\pgfpathlineto{\pgfqpoint{4.212198in}{1.631582in}}%
\pgfpathlineto{\pgfqpoint{4.213060in}{1.697002in}}%
\pgfpathlineto{\pgfqpoint{4.213925in}{1.696913in}}%
\pgfpathlineto{\pgfqpoint{4.214789in}{1.697359in}}%
\pgfpathlineto{\pgfqpoint{4.216520in}{1.714396in}}%
\pgfpathlineto{\pgfqpoint{4.218250in}{1.668151in}}%
\pgfpathlineto{\pgfqpoint{4.219112in}{1.690770in}}%
\pgfpathlineto{\pgfqpoint{4.219976in}{1.675335in}}%
\pgfpathlineto{\pgfqpoint{4.220842in}{1.759157in}}%
\pgfpathlineto{\pgfqpoint{4.221705in}{1.711429in}}%
\pgfpathlineto{\pgfqpoint{4.222571in}{1.740339in}}%
\pgfpathlineto{\pgfqpoint{4.223436in}{1.702702in}}%
\pgfpathlineto{\pgfqpoint{4.224302in}{1.735293in}}%
\pgfpathlineto{\pgfqpoint{4.225166in}{1.662987in}}%
\pgfpathlineto{\pgfqpoint{4.226894in}{1.701869in}}%
\pgfpathlineto{\pgfqpoint{4.227760in}{1.684003in}}%
\pgfpathlineto{\pgfqpoint{4.228626in}{1.704543in}}%
\pgfpathlineto{\pgfqpoint{4.229490in}{1.683705in}}%
\pgfpathlineto{\pgfqpoint{4.230356in}{1.702315in}}%
\pgfpathlineto{\pgfqpoint{4.231219in}{1.644287in}}%
\pgfpathlineto{\pgfqpoint{4.232949in}{1.718137in}}%
\pgfpathlineto{\pgfqpoint{4.233814in}{1.701159in}}%
\pgfpathlineto{\pgfqpoint{4.234678in}{1.734996in}}%
\pgfpathlineto{\pgfqpoint{4.235543in}{1.681865in}}%
\pgfpathlineto{\pgfqpoint{4.236409in}{1.709053in}}%
\pgfpathlineto{\pgfqpoint{4.237273in}{1.702345in}}%
\pgfpathlineto{\pgfqpoint{4.238138in}{1.690056in}}%
\pgfpathlineto{\pgfqpoint{4.239867in}{1.729117in}}%
\pgfpathlineto{\pgfqpoint{4.240732in}{1.700088in}}%
\pgfpathlineto{\pgfqpoint{4.241594in}{1.714396in}}%
\pgfpathlineto{\pgfqpoint{4.242461in}{1.702167in}}%
\pgfpathlineto{\pgfqpoint{4.243325in}{1.717601in}}%
\pgfpathlineto{\pgfqpoint{4.244190in}{1.714515in}}%
\pgfpathlineto{\pgfqpoint{4.245054in}{1.715909in}}%
\pgfpathlineto{\pgfqpoint{4.245919in}{1.752509in}}%
\pgfpathlineto{\pgfqpoint{4.246784in}{1.707748in}}%
\pgfpathlineto{\pgfqpoint{4.248514in}{1.738439in}}%
\pgfpathlineto{\pgfqpoint{4.249380in}{1.750698in}}%
\pgfpathlineto{\pgfqpoint{4.250245in}{1.713210in}}%
\pgfpathlineto{\pgfqpoint{4.251107in}{1.741882in}}%
\pgfpathlineto{\pgfqpoint{4.252837in}{1.699199in}}%
\pgfpathlineto{\pgfqpoint{4.253702in}{1.633422in}}%
\pgfpathlineto{\pgfqpoint{4.254566in}{1.640279in}}%
\pgfpathlineto{\pgfqpoint{4.255431in}{1.651174in}}%
\pgfpathlineto{\pgfqpoint{4.256297in}{1.639242in}}%
\pgfpathlineto{\pgfqpoint{4.257163in}{1.642090in}}%
\pgfpathlineto{\pgfqpoint{4.258027in}{1.704662in}}%
\pgfpathlineto{\pgfqpoint{4.258891in}{1.625528in}}%
\pgfpathlineto{\pgfqpoint{4.259758in}{1.636274in}}%
\pgfpathlineto{\pgfqpoint{4.260621in}{1.672071in}}%
\pgfpathlineto{\pgfqpoint{4.261484in}{1.654587in}}%
\pgfpathlineto{\pgfqpoint{4.263215in}{1.733809in}}%
\pgfpathlineto{\pgfqpoint{4.264079in}{1.666965in}}%
\pgfpathlineto{\pgfqpoint{4.264945in}{1.756547in}}%
\pgfpathlineto{\pgfqpoint{4.265810in}{1.737253in}}%
\pgfpathlineto{\pgfqpoint{4.266675in}{1.667973in}}%
\pgfpathlineto{\pgfqpoint{4.268406in}{1.723301in}}%
\pgfpathlineto{\pgfqpoint{4.270136in}{1.676997in}}%
\pgfpathlineto{\pgfqpoint{4.271865in}{1.714872in}}%
\pgfpathlineto{\pgfqpoint{4.273593in}{1.646277in}}%
\pgfpathlineto{\pgfqpoint{4.275324in}{1.746336in}}%
\pgfpathlineto{\pgfqpoint{4.276188in}{1.727518in}}%
\pgfpathlineto{\pgfqpoint{4.277054in}{1.667382in}}%
\pgfpathlineto{\pgfqpoint{4.277919in}{1.737520in}}%
\pgfpathlineto{\pgfqpoint{4.278785in}{1.654383in}}%
\pgfpathlineto{\pgfqpoint{4.279650in}{1.744674in}}%
\pgfpathlineto{\pgfqpoint{4.281379in}{1.639955in}}%
\pgfpathlineto{\pgfqpoint{4.282243in}{1.641082in}}%
\pgfpathlineto{\pgfqpoint{4.285704in}{1.527339in}}%
\pgfpathlineto{\pgfqpoint{4.289164in}{1.687386in}}%
\pgfpathlineto{\pgfqpoint{4.290894in}{1.535709in}}%
\pgfpathlineto{\pgfqpoint{4.291759in}{1.626953in}}%
\pgfpathlineto{\pgfqpoint{4.292625in}{1.623569in}}%
\pgfpathlineto{\pgfqpoint{4.293491in}{1.587535in}}%
\pgfpathlineto{\pgfqpoint{4.295221in}{1.611399in}}%
\pgfpathlineto{\pgfqpoint{4.296085in}{1.688810in}}%
\pgfpathlineto{\pgfqpoint{4.297812in}{1.619118in}}%
\pgfpathlineto{\pgfqpoint{4.298678in}{1.632593in}}%
\pgfpathlineto{\pgfqpoint{4.299544in}{1.615020in}}%
\pgfpathlineto{\pgfqpoint{4.301276in}{1.684713in}}%
\pgfpathlineto{\pgfqpoint{4.302140in}{1.599824in}}%
\pgfpathlineto{\pgfqpoint{4.303005in}{1.654319in}}%
\pgfpathlineto{\pgfqpoint{4.303869in}{1.611518in}}%
\pgfpathlineto{\pgfqpoint{4.304734in}{1.657762in}}%
\pgfpathlineto{\pgfqpoint{4.306465in}{1.614307in}}%
\pgfpathlineto{\pgfqpoint{4.307331in}{1.661797in}}%
\pgfpathlineto{\pgfqpoint{4.308196in}{1.655565in}}%
\pgfpathlineto{\pgfqpoint{4.311656in}{1.603326in}}%
\pgfpathlineto{\pgfqpoint{4.313387in}{1.646898in}}%
\pgfpathlineto{\pgfqpoint{4.314253in}{1.600177in}}%
\pgfpathlineto{\pgfqpoint{4.316847in}{1.676581in}}%
\pgfpathlineto{\pgfqpoint{4.317712in}{1.664887in}}%
\pgfpathlineto{\pgfqpoint{4.318576in}{1.607866in}}%
\pgfpathlineto{\pgfqpoint{4.319442in}{1.631701in}}%
\pgfpathlineto{\pgfqpoint{4.320307in}{1.580470in}}%
\pgfpathlineto{\pgfqpoint{4.321170in}{1.589346in}}%
\pgfpathlineto{\pgfqpoint{4.323763in}{1.683289in}}%
\pgfpathlineto{\pgfqpoint{4.324629in}{1.679667in}}%
\pgfpathlineto{\pgfqpoint{4.328083in}{1.604988in}}%
\pgfpathlineto{\pgfqpoint{4.329813in}{1.619947in}}%
\pgfpathlineto{\pgfqpoint{4.330678in}{1.602702in}}%
\pgfpathlineto{\pgfqpoint{4.331544in}{1.611458in}}%
\pgfpathlineto{\pgfqpoint{4.332409in}{1.654557in}}%
\pgfpathlineto{\pgfqpoint{4.333276in}{1.583735in}}%
\pgfpathlineto{\pgfqpoint{4.334141in}{1.618285in}}%
\pgfpathlineto{\pgfqpoint{4.335005in}{1.574119in}}%
\pgfpathlineto{\pgfqpoint{4.336733in}{1.645533in}}%
\pgfpathlineto{\pgfqpoint{4.338461in}{1.594540in}}%
\pgfpathlineto{\pgfqpoint{4.339326in}{1.645652in}}%
\pgfpathlineto{\pgfqpoint{4.340193in}{1.591037in}}%
\pgfpathlineto{\pgfqpoint{4.341058in}{1.646128in}}%
\pgfpathlineto{\pgfqpoint{4.341924in}{1.582132in}}%
\pgfpathlineto{\pgfqpoint{4.342786in}{1.616385in}}%
\pgfpathlineto{\pgfqpoint{4.343648in}{1.575008in}}%
\pgfpathlineto{\pgfqpoint{4.345375in}{1.652717in}}%
\pgfpathlineto{\pgfqpoint{4.347106in}{1.587713in}}%
\pgfpathlineto{\pgfqpoint{4.347971in}{1.623450in}}%
\pgfpathlineto{\pgfqpoint{4.348837in}{1.739625in}}%
\pgfpathlineto{\pgfqpoint{4.349701in}{1.621907in}}%
\pgfpathlineto{\pgfqpoint{4.350566in}{1.623628in}}%
\pgfpathlineto{\pgfqpoint{4.351432in}{1.614901in}}%
\pgfpathlineto{\pgfqpoint{4.352298in}{1.621609in}}%
\pgfpathlineto{\pgfqpoint{4.353162in}{1.585635in}}%
\pgfpathlineto{\pgfqpoint{4.354027in}{1.623539in}}%
\pgfpathlineto{\pgfqpoint{4.354892in}{1.597507in}}%
\pgfpathlineto{\pgfqpoint{4.355757in}{1.684118in}}%
\pgfpathlineto{\pgfqpoint{4.356622in}{1.605788in}}%
\pgfpathlineto{\pgfqpoint{4.357487in}{1.613890in}}%
\pgfpathlineto{\pgfqpoint{4.359212in}{1.686137in}}%
\pgfpathlineto{\pgfqpoint{4.360077in}{1.668267in}}%
\pgfpathlineto{\pgfqpoint{4.360942in}{1.686375in}}%
\pgfpathlineto{\pgfqpoint{4.361807in}{1.604096in}}%
\pgfpathlineto{\pgfqpoint{4.362672in}{1.653665in}}%
\pgfpathlineto{\pgfqpoint{4.363537in}{1.593410in}}%
\pgfpathlineto{\pgfqpoint{4.365264in}{1.638290in}}%
\pgfpathlineto{\pgfqpoint{4.366126in}{1.585159in}}%
\pgfpathlineto{\pgfqpoint{4.366989in}{1.593529in}}%
\pgfpathlineto{\pgfqpoint{4.367853in}{1.581775in}}%
\pgfpathlineto{\pgfqpoint{4.368718in}{1.632917in}}%
\pgfpathlineto{\pgfqpoint{4.369583in}{1.582548in}}%
\pgfpathlineto{\pgfqpoint{4.371313in}{1.632058in}}%
\pgfpathlineto{\pgfqpoint{4.372179in}{1.587237in}}%
\pgfpathlineto{\pgfqpoint{4.373044in}{1.617720in}}%
\pgfpathlineto{\pgfqpoint{4.373909in}{1.540576in}}%
\pgfpathlineto{\pgfqpoint{4.374774in}{1.663935in}}%
\pgfpathlineto{\pgfqpoint{4.375640in}{1.614723in}}%
\pgfpathlineto{\pgfqpoint{4.376505in}{1.648381in}}%
\pgfpathlineto{\pgfqpoint{4.377370in}{1.630217in}}%
\pgfpathlineto{\pgfqpoint{4.378235in}{1.533987in}}%
\pgfpathlineto{\pgfqpoint{4.379100in}{1.655446in}}%
\pgfpathlineto{\pgfqpoint{4.379961in}{1.654554in}}%
\pgfpathlineto{\pgfqpoint{4.380827in}{1.636092in}}%
\pgfpathlineto{\pgfqpoint{4.381692in}{1.655386in}}%
\pgfpathlineto{\pgfqpoint{4.382557in}{1.586880in}}%
\pgfpathlineto{\pgfqpoint{4.384290in}{1.657168in}}%
\pgfpathlineto{\pgfqpoint{4.386020in}{1.634252in}}%
\pgfpathlineto{\pgfqpoint{4.386886in}{1.653486in}}%
\pgfpathlineto{\pgfqpoint{4.388616in}{1.572870in}}%
\pgfpathlineto{\pgfqpoint{4.389481in}{1.621371in}}%
\pgfpathlineto{\pgfqpoint{4.390346in}{1.600474in}}%
\pgfpathlineto{\pgfqpoint{4.391212in}{1.667200in}}%
\pgfpathlineto{\pgfqpoint{4.392939in}{1.592342in}}%
\pgfpathlineto{\pgfqpoint{4.394670in}{1.609588in}}%
\pgfpathlineto{\pgfqpoint{4.395533in}{1.672781in}}%
\pgfpathlineto{\pgfqpoint{4.396398in}{1.557554in}}%
\pgfpathlineto{\pgfqpoint{4.397262in}{1.576402in}}%
\pgfpathlineto{\pgfqpoint{4.398127in}{1.609971in}}%
\pgfpathlineto{\pgfqpoint{4.399855in}{1.567526in}}%
\pgfpathlineto{\pgfqpoint{4.400721in}{1.625584in}}%
\pgfpathlineto{\pgfqpoint{4.401586in}{1.619412in}}%
\pgfpathlineto{\pgfqpoint{4.402453in}{1.578391in}}%
\pgfpathlineto{\pgfqpoint{4.403318in}{1.619828in}}%
\pgfpathlineto{\pgfqpoint{4.405913in}{1.588721in}}%
\pgfpathlineto{\pgfqpoint{4.406779in}{1.586999in}}%
\pgfpathlineto{\pgfqpoint{4.407644in}{1.606056in}}%
\pgfpathlineto{\pgfqpoint{4.408510in}{1.679370in}}%
\pgfpathlineto{\pgfqpoint{4.409376in}{1.593588in}}%
\pgfpathlineto{\pgfqpoint{4.410240in}{1.623093in}}%
\pgfpathlineto{\pgfqpoint{4.412836in}{1.587178in}}%
\pgfpathlineto{\pgfqpoint{4.413701in}{1.681299in}}%
\pgfpathlineto{\pgfqpoint{4.414565in}{1.680143in}}%
\pgfpathlineto{\pgfqpoint{4.416295in}{1.590889in}}%
\pgfpathlineto{\pgfqpoint{4.418026in}{1.640249in}}%
\pgfpathlineto{\pgfqpoint{4.420621in}{1.561473in}}%
\pgfpathlineto{\pgfqpoint{4.422347in}{1.634965in}}%
\pgfpathlineto{\pgfqpoint{4.423211in}{1.612942in}}%
\pgfpathlineto{\pgfqpoint{4.424076in}{1.631760in}}%
\pgfpathlineto{\pgfqpoint{4.425802in}{1.513656in}}%
\pgfpathlineto{\pgfqpoint{4.427532in}{1.626417in}}%
\pgfpathlineto{\pgfqpoint{4.428398in}{1.605639in}}%
\pgfpathlineto{\pgfqpoint{4.429263in}{1.667557in}}%
\pgfpathlineto{\pgfqpoint{4.430129in}{1.631731in}}%
\pgfpathlineto{\pgfqpoint{4.430994in}{1.670583in}}%
\pgfpathlineto{\pgfqpoint{4.431859in}{1.648679in}}%
\pgfpathlineto{\pgfqpoint{4.432725in}{1.650043in}}%
\pgfpathlineto{\pgfqpoint{4.433588in}{1.676934in}}%
\pgfpathlineto{\pgfqpoint{4.435319in}{1.561384in}}%
\pgfpathlineto{\pgfqpoint{4.437050in}{1.617690in}}%
\pgfpathlineto{\pgfqpoint{4.437915in}{1.643038in}}%
\pgfpathlineto{\pgfqpoint{4.438780in}{1.626358in}}%
\pgfpathlineto{\pgfqpoint{4.439645in}{1.533690in}}%
\pgfpathlineto{\pgfqpoint{4.441371in}{1.603918in}}%
\pgfpathlineto{\pgfqpoint{4.442235in}{1.586345in}}%
\pgfpathlineto{\pgfqpoint{4.443101in}{1.597091in}}%
\pgfpathlineto{\pgfqpoint{4.443964in}{1.635322in}}%
\pgfpathlineto{\pgfqpoint{4.445695in}{1.585278in}}%
\pgfpathlineto{\pgfqpoint{4.446561in}{1.590026in}}%
\pgfpathlineto{\pgfqpoint{4.447427in}{1.619590in}}%
\pgfpathlineto{\pgfqpoint{4.448292in}{1.616564in}}%
\pgfpathlineto{\pgfqpoint{4.449156in}{1.561473in}}%
\pgfpathlineto{\pgfqpoint{4.450020in}{1.600355in}}%
\pgfpathlineto{\pgfqpoint{4.450885in}{1.538498in}}%
\pgfpathlineto{\pgfqpoint{4.452617in}{1.619531in}}%
\pgfpathlineto{\pgfqpoint{4.454349in}{1.530455in}}%
\pgfpathlineto{\pgfqpoint{4.455213in}{1.619471in}}%
\pgfpathlineto{\pgfqpoint{4.456077in}{1.587237in}}%
\pgfpathlineto{\pgfqpoint{4.456943in}{1.612525in}}%
\pgfpathlineto{\pgfqpoint{4.459537in}{1.556249in}}%
\pgfpathlineto{\pgfqpoint{4.460402in}{1.619828in}}%
\pgfpathlineto{\pgfqpoint{4.461266in}{1.599615in}}%
\pgfpathlineto{\pgfqpoint{4.462131in}{1.616861in}}%
\pgfpathlineto{\pgfqpoint{4.462995in}{1.554170in}}%
\pgfpathlineto{\pgfqpoint{4.464723in}{1.600415in}}%
\pgfpathlineto{\pgfqpoint{4.465588in}{1.599348in}}%
\pgfpathlineto{\pgfqpoint{4.466453in}{1.616207in}}%
\pgfpathlineto{\pgfqpoint{4.468183in}{1.573584in}}%
\pgfpathlineto{\pgfqpoint{4.469913in}{1.617690in}}%
\pgfpathlineto{\pgfqpoint{4.471642in}{1.555713in}}%
\pgfpathlineto{\pgfqpoint{4.472507in}{1.641673in}}%
\pgfpathlineto{\pgfqpoint{4.473373in}{1.611220in}}%
\pgfpathlineto{\pgfqpoint{4.474235in}{1.656160in}}%
\pgfpathlineto{\pgfqpoint{4.475100in}{1.636096in}}%
\pgfpathlineto{\pgfqpoint{4.475966in}{1.690651in}}%
\pgfpathlineto{\pgfqpoint{4.479425in}{1.563730in}}%
\pgfpathlineto{\pgfqpoint{4.480290in}{1.563671in}}%
\pgfpathlineto{\pgfqpoint{4.482023in}{1.662332in}}%
\pgfpathlineto{\pgfqpoint{4.482888in}{1.576967in}}%
\pgfpathlineto{\pgfqpoint{4.483754in}{1.645771in}}%
\pgfpathlineto{\pgfqpoint{4.484620in}{1.631998in}}%
\pgfpathlineto{\pgfqpoint{4.485486in}{1.640190in}}%
\pgfpathlineto{\pgfqpoint{4.487216in}{1.691480in}}%
\pgfpathlineto{\pgfqpoint{4.488081in}{1.582608in}}%
\pgfpathlineto{\pgfqpoint{4.488945in}{1.613299in}}%
\pgfpathlineto{\pgfqpoint{4.489808in}{1.613477in}}%
\pgfpathlineto{\pgfqpoint{4.491537in}{1.702464in}}%
\pgfpathlineto{\pgfqpoint{4.492402in}{1.632088in}}%
\pgfpathlineto{\pgfqpoint{4.493266in}{1.636925in}}%
\pgfpathlineto{\pgfqpoint{4.494133in}{1.725555in}}%
\pgfpathlineto{\pgfqpoint{4.494998in}{1.601958in}}%
\pgfpathlineto{\pgfqpoint{4.495863in}{1.705193in}}%
\pgfpathlineto{\pgfqpoint{4.498455in}{1.627841in}}%
\pgfpathlineto{\pgfqpoint{4.500183in}{1.672662in}}%
\pgfpathlineto{\pgfqpoint{4.501914in}{1.591896in}}%
\pgfpathlineto{\pgfqpoint{4.504511in}{1.646660in}}%
\pgfpathlineto{\pgfqpoint{4.506240in}{1.596972in}}%
\pgfpathlineto{\pgfqpoint{4.507970in}{1.679013in}}%
\pgfpathlineto{\pgfqpoint{4.508835in}{1.586851in}}%
\pgfpathlineto{\pgfqpoint{4.509700in}{1.623152in}}%
\pgfpathlineto{\pgfqpoint{4.510564in}{1.587118in}}%
\pgfpathlineto{\pgfqpoint{4.511429in}{1.604126in}}%
\pgfpathlineto{\pgfqpoint{4.512294in}{1.651824in}}%
\pgfpathlineto{\pgfqpoint{4.514025in}{1.583913in}}%
\pgfpathlineto{\pgfqpoint{4.514890in}{1.578570in}}%
\pgfpathlineto{\pgfqpoint{4.516619in}{1.494153in}}%
\pgfpathlineto{\pgfqpoint{4.517482in}{1.600474in}}%
\pgfpathlineto{\pgfqpoint{4.518346in}{1.595964in}}%
\pgfpathlineto{\pgfqpoint{4.519212in}{1.544317in}}%
\pgfpathlineto{\pgfqpoint{4.520942in}{1.602196in}}%
\pgfpathlineto{\pgfqpoint{4.522670in}{1.640309in}}%
\pgfpathlineto{\pgfqpoint{4.523535in}{1.516712in}}%
\pgfpathlineto{\pgfqpoint{4.524400in}{1.658889in}}%
\pgfpathlineto{\pgfqpoint{4.525263in}{1.569545in}}%
\pgfpathlineto{\pgfqpoint{4.526128in}{1.576848in}}%
\pgfpathlineto{\pgfqpoint{4.526993in}{1.576908in}}%
\pgfpathlineto{\pgfqpoint{4.530452in}{1.612972in}}%
\pgfpathlineto{\pgfqpoint{4.531316in}{1.615909in}}%
\pgfpathlineto{\pgfqpoint{4.532179in}{1.653486in}}%
\pgfpathlineto{\pgfqpoint{4.533043in}{1.538855in}}%
\pgfpathlineto{\pgfqpoint{4.533909in}{1.671059in}}%
\pgfpathlineto{\pgfqpoint{4.534774in}{1.578867in}}%
\pgfpathlineto{\pgfqpoint{4.535638in}{1.600534in}}%
\pgfpathlineto{\pgfqpoint{4.536505in}{1.631284in}}%
\pgfpathlineto{\pgfqpoint{4.538236in}{1.588364in}}%
\pgfpathlineto{\pgfqpoint{4.539965in}{1.620836in}}%
\pgfpathlineto{\pgfqpoint{4.540828in}{1.603858in}}%
\pgfpathlineto{\pgfqpoint{4.541691in}{1.605580in}}%
\pgfpathlineto{\pgfqpoint{4.542555in}{1.668386in}}%
\pgfpathlineto{\pgfqpoint{4.543419in}{1.656989in}}%
\pgfpathlineto{\pgfqpoint{4.544282in}{1.603590in}}%
\pgfpathlineto{\pgfqpoint{4.545145in}{1.673134in}}%
\pgfpathlineto{\pgfqpoint{4.546010in}{1.623446in}}%
\pgfpathlineto{\pgfqpoint{4.546874in}{1.689015in}}%
\pgfpathlineto{\pgfqpoint{4.548603in}{1.624279in}}%
\pgfpathlineto{\pgfqpoint{4.549468in}{1.653962in}}%
\pgfpathlineto{\pgfqpoint{4.550331in}{1.650400in}}%
\pgfpathlineto{\pgfqpoint{4.552060in}{1.598277in}}%
\pgfpathlineto{\pgfqpoint{4.552925in}{1.677113in}}%
\pgfpathlineto{\pgfqpoint{4.554650in}{1.615909in}}%
\pgfpathlineto{\pgfqpoint{4.555516in}{1.608963in}}%
\pgfpathlineto{\pgfqpoint{4.556379in}{1.629741in}}%
\pgfpathlineto{\pgfqpoint{4.558975in}{1.542268in}}%
\pgfpathlineto{\pgfqpoint{4.559839in}{1.631820in}}%
\pgfpathlineto{\pgfqpoint{4.560705in}{1.553873in}}%
\pgfpathlineto{\pgfqpoint{4.561571in}{1.574889in}}%
\pgfpathlineto{\pgfqpoint{4.562436in}{1.586940in}}%
\pgfpathlineto{\pgfqpoint{4.563300in}{1.585605in}}%
\pgfpathlineto{\pgfqpoint{4.565028in}{1.599288in}}%
\pgfpathlineto{\pgfqpoint{4.565894in}{1.579046in}}%
\pgfpathlineto{\pgfqpoint{4.568487in}{1.701929in}}%
\pgfpathlineto{\pgfqpoint{4.569351in}{1.615734in}}%
\pgfpathlineto{\pgfqpoint{4.570217in}{1.620304in}}%
\pgfpathlineto{\pgfqpoint{4.571084in}{1.668211in}}%
\pgfpathlineto{\pgfqpoint{4.572814in}{1.612793in}}%
\pgfpathlineto{\pgfqpoint{4.574545in}{1.659365in}}%
\pgfpathlineto{\pgfqpoint{4.575410in}{1.571713in}}%
\pgfpathlineto{\pgfqpoint{4.576275in}{1.668386in}}%
\pgfpathlineto{\pgfqpoint{4.578869in}{1.566697in}}%
\pgfpathlineto{\pgfqpoint{4.579732in}{1.740752in}}%
\pgfpathlineto{\pgfqpoint{4.580597in}{1.610566in}}%
\pgfpathlineto{\pgfqpoint{4.581462in}{1.640368in}}%
\pgfpathlineto{\pgfqpoint{4.583189in}{1.565154in}}%
\pgfpathlineto{\pgfqpoint{4.584053in}{1.624993in}}%
\pgfpathlineto{\pgfqpoint{4.584916in}{1.540993in}}%
\pgfpathlineto{\pgfqpoint{4.586645in}{1.604691in}}%
\pgfpathlineto{\pgfqpoint{4.587511in}{1.604750in}}%
\pgfpathlineto{\pgfqpoint{4.588377in}{1.568478in}}%
\pgfpathlineto{\pgfqpoint{4.589242in}{1.643930in}}%
\pgfpathlineto{\pgfqpoint{4.590109in}{1.624815in}}%
\pgfpathlineto{\pgfqpoint{4.590975in}{1.634965in}}%
\pgfpathlineto{\pgfqpoint{4.592706in}{1.583675in}}%
\pgfpathlineto{\pgfqpoint{4.593571in}{1.643633in}}%
\pgfpathlineto{\pgfqpoint{4.597029in}{1.521579in}}%
\pgfpathlineto{\pgfqpoint{4.597894in}{1.623863in}}%
\pgfpathlineto{\pgfqpoint{4.598760in}{1.595667in}}%
\pgfpathlineto{\pgfqpoint{4.599625in}{1.525141in}}%
\pgfpathlineto{\pgfqpoint{4.601354in}{1.578927in}}%
\pgfpathlineto{\pgfqpoint{4.602218in}{1.581422in}}%
\pgfpathlineto{\pgfqpoint{4.603082in}{1.654438in}}%
\pgfpathlineto{\pgfqpoint{4.603947in}{1.555955in}}%
\pgfpathlineto{\pgfqpoint{4.604812in}{1.566106in}}%
\pgfpathlineto{\pgfqpoint{4.605675in}{1.615675in}}%
\pgfpathlineto{\pgfqpoint{4.606541in}{1.543398in}}%
\pgfpathlineto{\pgfqpoint{4.607407in}{1.545566in}}%
\pgfpathlineto{\pgfqpoint{4.608272in}{1.525855in}}%
\pgfpathlineto{\pgfqpoint{4.609137in}{1.532920in}}%
\pgfpathlineto{\pgfqpoint{4.610002in}{1.610213in}}%
\pgfpathlineto{\pgfqpoint{4.610868in}{1.520482in}}%
\pgfpathlineto{\pgfqpoint{4.612594in}{1.649036in}}%
\pgfpathlineto{\pgfqpoint{4.614321in}{1.542833in}}%
\pgfpathlineto{\pgfqpoint{4.616052in}{1.607837in}}%
\pgfpathlineto{\pgfqpoint{4.616918in}{1.558268in}}%
\pgfpathlineto{\pgfqpoint{4.617784in}{1.558446in}}%
\pgfpathlineto{\pgfqpoint{4.618650in}{1.532682in}}%
\pgfpathlineto{\pgfqpoint{4.619515in}{1.610213in}}%
\pgfpathlineto{\pgfqpoint{4.620380in}{1.605226in}}%
\pgfpathlineto{\pgfqpoint{4.621246in}{1.644406in}}%
\pgfpathlineto{\pgfqpoint{4.622110in}{1.629269in}}%
\pgfpathlineto{\pgfqpoint{4.622974in}{1.587088in}}%
\pgfpathlineto{\pgfqpoint{4.624705in}{1.604334in}}%
\pgfpathlineto{\pgfqpoint{4.625570in}{1.596053in}}%
\pgfpathlineto{\pgfqpoint{4.626436in}{1.676997in}}%
\pgfpathlineto{\pgfqpoint{4.627301in}{1.630098in}}%
\pgfpathlineto{\pgfqpoint{4.628165in}{1.649512in}}%
\pgfpathlineto{\pgfqpoint{4.629028in}{1.638766in}}%
\pgfpathlineto{\pgfqpoint{4.629894in}{1.594778in}}%
\pgfpathlineto{\pgfqpoint{4.630756in}{1.650519in}}%
\pgfpathlineto{\pgfqpoint{4.631622in}{1.645890in}}%
\pgfpathlineto{\pgfqpoint{4.632489in}{1.659930in}}%
\pgfpathlineto{\pgfqpoint{4.633355in}{1.641082in}}%
\pgfpathlineto{\pgfqpoint{4.634220in}{1.574357in}}%
\pgfpathlineto{\pgfqpoint{4.635081in}{1.574387in}}%
\pgfpathlineto{\pgfqpoint{4.635945in}{1.651293in}}%
\pgfpathlineto{\pgfqpoint{4.637675in}{1.621847in}}%
\pgfpathlineto{\pgfqpoint{4.638541in}{1.663225in}}%
\pgfpathlineto{\pgfqpoint{4.640272in}{1.518969in}}%
\pgfpathlineto{\pgfqpoint{4.641137in}{1.622442in}}%
\pgfpathlineto{\pgfqpoint{4.642003in}{1.608313in}}%
\pgfpathlineto{\pgfqpoint{4.642868in}{1.618464in}}%
\pgfpathlineto{\pgfqpoint{4.643733in}{1.669992in}}%
\pgfpathlineto{\pgfqpoint{4.647191in}{1.526331in}}%
\pgfpathlineto{\pgfqpoint{4.648057in}{1.622799in}}%
\pgfpathlineto{\pgfqpoint{4.648923in}{1.557587in}}%
\pgfpathlineto{\pgfqpoint{4.649788in}{1.576852in}}%
\pgfpathlineto{\pgfqpoint{4.650652in}{1.569430in}}%
\pgfpathlineto{\pgfqpoint{4.652381in}{1.617337in}}%
\pgfpathlineto{\pgfqpoint{4.654112in}{1.569044in}}%
\pgfpathlineto{\pgfqpoint{4.654976in}{1.563909in}}%
\pgfpathlineto{\pgfqpoint{4.655843in}{1.602493in}}%
\pgfpathlineto{\pgfqpoint{4.656707in}{1.542536in}}%
\pgfpathlineto{\pgfqpoint{4.658437in}{1.609618in}}%
\pgfpathlineto{\pgfqpoint{4.660168in}{1.562008in}}%
\pgfpathlineto{\pgfqpoint{4.662761in}{1.647552in}}%
\pgfpathlineto{\pgfqpoint{4.663625in}{1.573970in}}%
\pgfpathlineto{\pgfqpoint{4.664490in}{1.575841in}}%
\pgfpathlineto{\pgfqpoint{4.665354in}{1.560525in}}%
\pgfpathlineto{\pgfqpoint{4.666218in}{1.589316in}}%
\pgfpathlineto{\pgfqpoint{4.667082in}{1.580589in}}%
\pgfpathlineto{\pgfqpoint{4.667948in}{1.532682in}}%
\pgfpathlineto{\pgfqpoint{4.668812in}{1.640190in}}%
\pgfpathlineto{\pgfqpoint{4.669677in}{1.561592in}}%
\pgfpathlineto{\pgfqpoint{4.670542in}{1.599318in}}%
\pgfpathlineto{\pgfqpoint{4.672272in}{1.567590in}}%
\pgfpathlineto{\pgfqpoint{4.674000in}{1.589613in}}%
\pgfpathlineto{\pgfqpoint{4.674866in}{1.629269in}}%
\pgfpathlineto{\pgfqpoint{4.675730in}{1.566816in}}%
\pgfpathlineto{\pgfqpoint{4.677460in}{1.611369in}}%
\pgfpathlineto{\pgfqpoint{4.678325in}{1.575722in}}%
\pgfpathlineto{\pgfqpoint{4.679190in}{1.618583in}}%
\pgfpathlineto{\pgfqpoint{4.680055in}{1.569222in}}%
\pgfpathlineto{\pgfqpoint{4.682650in}{1.632474in}}%
\pgfpathlineto{\pgfqpoint{4.683516in}{1.626123in}}%
\pgfpathlineto{\pgfqpoint{4.684380in}{1.570616in}}%
\pgfpathlineto{\pgfqpoint{4.685244in}{1.606532in}}%
\pgfpathlineto{\pgfqpoint{4.686110in}{1.592878in}}%
\pgfpathlineto{\pgfqpoint{4.687840in}{1.641023in}}%
\pgfpathlineto{\pgfqpoint{4.688705in}{1.609439in}}%
\pgfpathlineto{\pgfqpoint{4.690432in}{1.680084in}}%
\pgfpathlineto{\pgfqpoint{4.693027in}{1.540398in}}%
\pgfpathlineto{\pgfqpoint{4.693889in}{1.631582in}}%
\pgfpathlineto{\pgfqpoint{4.694755in}{1.576729in}}%
\pgfpathlineto{\pgfqpoint{4.695621in}{1.627901in}}%
\pgfpathlineto{\pgfqpoint{4.696486in}{1.590651in}}%
\pgfpathlineto{\pgfqpoint{4.697352in}{1.612823in}}%
\pgfpathlineto{\pgfqpoint{4.698218in}{1.591275in}}%
\pgfpathlineto{\pgfqpoint{4.699949in}{1.619888in}}%
\pgfpathlineto{\pgfqpoint{4.700812in}{1.620245in}}%
\pgfpathlineto{\pgfqpoint{4.701675in}{1.545919in}}%
\pgfpathlineto{\pgfqpoint{4.702540in}{1.595905in}}%
\pgfpathlineto{\pgfqpoint{4.704271in}{1.538736in}}%
\pgfpathlineto{\pgfqpoint{4.705999in}{1.698128in}}%
\pgfpathlineto{\pgfqpoint{4.708594in}{1.665835in}}%
\pgfpathlineto{\pgfqpoint{4.709461in}{1.728700in}}%
\pgfpathlineto{\pgfqpoint{4.710325in}{1.727395in}}%
\pgfpathlineto{\pgfqpoint{4.711190in}{1.626179in}}%
\pgfpathlineto{\pgfqpoint{4.712921in}{1.698485in}}%
\pgfpathlineto{\pgfqpoint{4.713785in}{1.645533in}}%
\pgfpathlineto{\pgfqpoint{4.714648in}{1.686137in}}%
\pgfpathlineto{\pgfqpoint{4.715513in}{1.637371in}}%
\pgfpathlineto{\pgfqpoint{4.718111in}{1.704926in}}%
\pgfpathlineto{\pgfqpoint{4.718976in}{1.633303in}}%
\pgfpathlineto{\pgfqpoint{4.719839in}{1.690472in}}%
\pgfpathlineto{\pgfqpoint{4.720705in}{1.653516in}}%
\pgfpathlineto{\pgfqpoint{4.721569in}{1.667557in}}%
\pgfpathlineto{\pgfqpoint{4.723300in}{1.744373in}}%
\pgfpathlineto{\pgfqpoint{4.724166in}{1.724666in}}%
\pgfpathlineto{\pgfqpoint{4.725030in}{1.648887in}}%
\pgfpathlineto{\pgfqpoint{4.725894in}{1.706260in}}%
\pgfpathlineto{\pgfqpoint{4.726759in}{1.659008in}}%
\pgfpathlineto{\pgfqpoint{4.728487in}{1.700326in}}%
\pgfpathlineto{\pgfqpoint{4.729353in}{1.613001in}}%
\pgfpathlineto{\pgfqpoint{4.731080in}{1.673078in}}%
\pgfpathlineto{\pgfqpoint{4.731947in}{1.624993in}}%
\pgfpathlineto{\pgfqpoint{4.732812in}{1.688453in}}%
\pgfpathlineto{\pgfqpoint{4.733678in}{1.683289in}}%
\pgfpathlineto{\pgfqpoint{4.734543in}{1.639093in}}%
\pgfpathlineto{\pgfqpoint{4.736271in}{1.720453in}}%
\pgfpathlineto{\pgfqpoint{4.737133in}{1.678868in}}%
\pgfpathlineto{\pgfqpoint{4.737997in}{1.724904in}}%
\pgfpathlineto{\pgfqpoint{4.739729in}{1.650846in}}%
\pgfpathlineto{\pgfqpoint{4.740595in}{1.762719in}}%
\pgfpathlineto{\pgfqpoint{4.741460in}{1.633660in}}%
\pgfpathlineto{\pgfqpoint{4.742327in}{1.707153in}}%
\pgfpathlineto{\pgfqpoint{4.743193in}{1.637460in}}%
\pgfpathlineto{\pgfqpoint{4.744060in}{1.705967in}}%
\pgfpathlineto{\pgfqpoint{4.744925in}{1.665363in}}%
\pgfpathlineto{\pgfqpoint{4.745789in}{1.686200in}}%
\pgfpathlineto{\pgfqpoint{4.746655in}{1.600567in}}%
\pgfpathlineto{\pgfqpoint{4.747518in}{1.692492in}}%
\pgfpathlineto{\pgfqpoint{4.748384in}{1.674268in}}%
\pgfpathlineto{\pgfqpoint{4.749246in}{1.644644in}}%
\pgfpathlineto{\pgfqpoint{4.750977in}{1.746630in}}%
\pgfpathlineto{\pgfqpoint{4.751842in}{1.683527in}}%
\pgfpathlineto{\pgfqpoint{4.752708in}{1.762184in}}%
\pgfpathlineto{\pgfqpoint{4.754440in}{1.688751in}}%
\pgfpathlineto{\pgfqpoint{4.755305in}{1.705550in}}%
\pgfpathlineto{\pgfqpoint{4.756170in}{1.664173in}}%
\pgfpathlineto{\pgfqpoint{4.757035in}{1.667438in}}%
\pgfpathlineto{\pgfqpoint{4.757901in}{1.693797in}}%
\pgfpathlineto{\pgfqpoint{4.758765in}{1.688453in}}%
\pgfpathlineto{\pgfqpoint{4.759630in}{1.818401in}}%
\pgfpathlineto{\pgfqpoint{4.761358in}{1.609558in}}%
\pgfpathlineto{\pgfqpoint{4.762223in}{1.631582in}}%
\pgfpathlineto{\pgfqpoint{4.763088in}{1.602702in}}%
\pgfpathlineto{\pgfqpoint{4.763954in}{1.705907in}}%
\pgfpathlineto{\pgfqpoint{4.766549in}{1.592640in}}%
\pgfpathlineto{\pgfqpoint{4.767415in}{1.659484in}}%
\pgfpathlineto{\pgfqpoint{4.770012in}{1.557911in}}%
\pgfpathlineto{\pgfqpoint{4.771744in}{1.640309in}}%
\pgfpathlineto{\pgfqpoint{4.773477in}{1.607896in}}%
\pgfpathlineto{\pgfqpoint{4.774342in}{1.566043in}}%
\pgfpathlineto{\pgfqpoint{4.775208in}{1.612853in}}%
\pgfpathlineto{\pgfqpoint{4.776073in}{1.585694in}}%
\pgfpathlineto{\pgfqpoint{4.776937in}{1.643811in}}%
\pgfpathlineto{\pgfqpoint{4.777802in}{1.559603in}}%
\pgfpathlineto{\pgfqpoint{4.778668in}{1.640249in}}%
\pgfpathlineto{\pgfqpoint{4.781267in}{1.552806in}}%
\pgfpathlineto{\pgfqpoint{4.782133in}{1.636895in}}%
\pgfpathlineto{\pgfqpoint{4.783865in}{1.561176in}}%
\pgfpathlineto{\pgfqpoint{4.784730in}{1.649036in}}%
\pgfpathlineto{\pgfqpoint{4.785596in}{1.624874in}}%
\pgfpathlineto{\pgfqpoint{4.786461in}{1.628198in}}%
\pgfpathlineto{\pgfqpoint{4.787327in}{1.585813in}}%
\pgfpathlineto{\pgfqpoint{4.788193in}{1.670940in}}%
\pgfpathlineto{\pgfqpoint{4.789925in}{1.574948in}}%
\pgfpathlineto{\pgfqpoint{4.790791in}{1.606115in}}%
\pgfpathlineto{\pgfqpoint{4.792520in}{1.527815in}}%
\pgfpathlineto{\pgfqpoint{4.793384in}{1.622855in}}%
\pgfpathlineto{\pgfqpoint{4.794249in}{1.544703in}}%
\pgfpathlineto{\pgfqpoint{4.795115in}{1.583794in}}%
\pgfpathlineto{\pgfqpoint{4.795980in}{1.704126in}}%
\pgfpathlineto{\pgfqpoint{4.796845in}{1.663225in}}%
\pgfpathlineto{\pgfqpoint{4.798574in}{1.786464in}}%
\pgfpathlineto{\pgfqpoint{4.801166in}{1.644585in}}%
\pgfpathlineto{\pgfqpoint{4.802893in}{1.697894in}}%
\pgfpathlineto{\pgfqpoint{4.803757in}{1.675365in}}%
\pgfpathlineto{\pgfqpoint{4.804623in}{1.705967in}}%
\pgfpathlineto{\pgfqpoint{4.806350in}{1.532742in}}%
\pgfpathlineto{\pgfqpoint{4.808078in}{1.650698in}}%
\pgfpathlineto{\pgfqpoint{4.808943in}{1.592997in}}%
\pgfpathlineto{\pgfqpoint{4.809809in}{1.595075in}}%
\pgfpathlineto{\pgfqpoint{4.812399in}{1.733452in}}%
\pgfpathlineto{\pgfqpoint{4.813264in}{1.684032in}}%
\pgfpathlineto{\pgfqpoint{4.814129in}{1.690651in}}%
\pgfpathlineto{\pgfqpoint{4.814993in}{1.743841in}}%
\pgfpathlineto{\pgfqpoint{4.815858in}{1.669814in}}%
\pgfpathlineto{\pgfqpoint{4.816722in}{1.745147in}}%
\pgfpathlineto{\pgfqpoint{4.818454in}{1.616207in}}%
\pgfpathlineto{\pgfqpoint{4.820184in}{1.693945in}}%
\pgfpathlineto{\pgfqpoint{4.821048in}{1.661146in}}%
\pgfpathlineto{\pgfqpoint{4.821914in}{1.690710in}}%
\pgfpathlineto{\pgfqpoint{4.822781in}{1.690562in}}%
\pgfpathlineto{\pgfqpoint{4.824513in}{1.620840in}}%
\pgfpathlineto{\pgfqpoint{4.825379in}{1.630782in}}%
\pgfpathlineto{\pgfqpoint{4.826244in}{1.629269in}}%
\pgfpathlineto{\pgfqpoint{4.827110in}{1.714575in}}%
\pgfpathlineto{\pgfqpoint{4.827976in}{1.582846in}}%
\pgfpathlineto{\pgfqpoint{4.830573in}{1.694154in}}%
\pgfpathlineto{\pgfqpoint{4.831440in}{1.735115in}}%
\pgfpathlineto{\pgfqpoint{4.832305in}{1.532385in}}%
\pgfpathlineto{\pgfqpoint{4.833170in}{1.534523in}}%
\pgfpathlineto{\pgfqpoint{4.834033in}{1.589375in}}%
\pgfpathlineto{\pgfqpoint{4.834897in}{1.574000in}}%
\pgfpathlineto{\pgfqpoint{4.835762in}{1.569014in}}%
\pgfpathlineto{\pgfqpoint{4.836629in}{1.612882in}}%
\pgfpathlineto{\pgfqpoint{4.838359in}{1.565273in}}%
\pgfpathlineto{\pgfqpoint{4.839223in}{1.508937in}}%
\pgfpathlineto{\pgfqpoint{4.840953in}{1.639004in}}%
\pgfpathlineto{\pgfqpoint{4.842682in}{1.551917in}}%
\pgfpathlineto{\pgfqpoint{4.842682in}{1.551917in}}%
\pgfusepath{stroke}%
\end{pgfscope}%
\begin{pgfscope}%
\pgfsetrectcap%
\pgfsetmiterjoin%
\pgfsetlinewidth{0.803000pt}%
\definecolor{currentstroke}{rgb}{0.000000,0.000000,0.000000}%
\pgfsetstrokecolor{currentstroke}%
\pgfsetdash{}{0pt}%
\pgfpathmoveto{\pgfqpoint{0.483776in}{1.444834in}}%
\pgfpathlineto{\pgfqpoint{0.483776in}{2.029715in}}%
\pgfusepath{stroke}%
\end{pgfscope}%
\begin{pgfscope}%
\pgfsetrectcap%
\pgfsetmiterjoin%
\pgfsetlinewidth{0.803000pt}%
\definecolor{currentstroke}{rgb}{0.000000,0.000000,0.000000}%
\pgfsetstrokecolor{currentstroke}%
\pgfsetdash{}{0pt}%
\pgfpathmoveto{\pgfqpoint{5.050249in}{1.444834in}}%
\pgfpathlineto{\pgfqpoint{5.050249in}{2.029715in}}%
\pgfusepath{stroke}%
\end{pgfscope}%
\begin{pgfscope}%
\pgfsetrectcap%
\pgfsetmiterjoin%
\pgfsetlinewidth{0.803000pt}%
\definecolor{currentstroke}{rgb}{0.000000,0.000000,0.000000}%
\pgfsetstrokecolor{currentstroke}%
\pgfsetdash{}{0pt}%
\pgfpathmoveto{\pgfqpoint{0.483776in}{1.444834in}}%
\pgfpathlineto{\pgfqpoint{5.050249in}{1.444834in}}%
\pgfusepath{stroke}%
\end{pgfscope}%
\begin{pgfscope}%
\pgfsetrectcap%
\pgfsetmiterjoin%
\pgfsetlinewidth{0.803000pt}%
\definecolor{currentstroke}{rgb}{0.000000,0.000000,0.000000}%
\pgfsetstrokecolor{currentstroke}%
\pgfsetdash{}{0pt}%
\pgfpathmoveto{\pgfqpoint{0.483776in}{2.029715in}}%
\pgfpathlineto{\pgfqpoint{5.050249in}{2.029715in}}%
\pgfusepath{stroke}%
\end{pgfscope}%
\begin{pgfscope}%
\pgfsetbuttcap%
\pgfsetmiterjoin%
\definecolor{currentfill}{rgb}{1.000000,1.000000,1.000000}%
\pgfsetfillcolor{currentfill}%
\pgfsetlinewidth{0.000000pt}%
\definecolor{currentstroke}{rgb}{0.000000,0.000000,0.000000}%
\pgfsetstrokecolor{currentstroke}%
\pgfsetstrokeopacity{0.000000}%
\pgfsetdash{}{0pt}%
\pgfpathmoveto{\pgfqpoint{0.483776in}{0.538014in}}%
\pgfpathlineto{\pgfqpoint{5.050249in}{0.538014in}}%
\pgfpathlineto{\pgfqpoint{5.050249in}{1.122895in}}%
\pgfpathlineto{\pgfqpoint{0.483776in}{1.122895in}}%
\pgfpathlineto{\pgfqpoint{0.483776in}{0.538014in}}%
\pgfpathclose%
\pgfusepath{fill}%
\end{pgfscope}%
\begin{pgfscope}%
\pgfsetbuttcap%
\pgfsetroundjoin%
\definecolor{currentfill}{rgb}{0.000000,0.000000,0.000000}%
\pgfsetfillcolor{currentfill}%
\pgfsetlinewidth{0.803000pt}%
\definecolor{currentstroke}{rgb}{0.000000,0.000000,0.000000}%
\pgfsetstrokecolor{currentstroke}%
\pgfsetdash{}{0pt}%
\pgfsys@defobject{currentmarker}{\pgfqpoint{0.000000in}{-0.048611in}}{\pgfqpoint{0.000000in}{0.000000in}}{%
\pgfpathmoveto{\pgfqpoint{0.000000in}{0.000000in}}%
\pgfpathlineto{\pgfqpoint{0.000000in}{-0.048611in}}%
\pgfusepath{stroke,fill}%
}%
\begin{pgfscope}%
\pgfsys@transformshift{0.691021in}{0.538014in}%
\pgfsys@useobject{currentmarker}{}%
\end{pgfscope}%
\end{pgfscope}%
\begin{pgfscope}%
\definecolor{textcolor}{rgb}{0.000000,0.000000,0.000000}%
\pgfsetstrokecolor{textcolor}%
\pgfsetfillcolor{textcolor}%
\pgftext[x=0.691021in,y=0.440792in,,top]{\color{textcolor}\rmfamily\fontsize{8.000000}{9.600000}\selectfont \(\displaystyle {06{:}00}\)}%
\end{pgfscope}%
\begin{pgfscope}%
\pgfsetbuttcap%
\pgfsetroundjoin%
\definecolor{currentfill}{rgb}{0.000000,0.000000,0.000000}%
\pgfsetfillcolor{currentfill}%
\pgfsetlinewidth{0.803000pt}%
\definecolor{currentstroke}{rgb}{0.000000,0.000000,0.000000}%
\pgfsetstrokecolor{currentstroke}%
\pgfsetdash{}{0pt}%
\pgfsys@defobject{currentmarker}{\pgfqpoint{0.000000in}{-0.048611in}}{\pgfqpoint{0.000000in}{0.000000in}}{%
\pgfpathmoveto{\pgfqpoint{0.000000in}{0.000000in}}%
\pgfpathlineto{\pgfqpoint{0.000000in}{-0.048611in}}%
\pgfusepath{stroke,fill}%
}%
\begin{pgfscope}%
\pgfsys@transformshift{1.210067in}{0.538014in}%
\pgfsys@useobject{currentmarker}{}%
\end{pgfscope}%
\end{pgfscope}%
\begin{pgfscope}%
\definecolor{textcolor}{rgb}{0.000000,0.000000,0.000000}%
\pgfsetstrokecolor{textcolor}%
\pgfsetfillcolor{textcolor}%
\pgftext[x=1.210067in,y=0.440792in,,top]{\color{textcolor}\rmfamily\fontsize{8.000000}{9.600000}\selectfont \(\displaystyle {09{:}00}\)}%
\end{pgfscope}%
\begin{pgfscope}%
\pgfsetbuttcap%
\pgfsetroundjoin%
\definecolor{currentfill}{rgb}{0.000000,0.000000,0.000000}%
\pgfsetfillcolor{currentfill}%
\pgfsetlinewidth{0.803000pt}%
\definecolor{currentstroke}{rgb}{0.000000,0.000000,0.000000}%
\pgfsetstrokecolor{currentstroke}%
\pgfsetdash{}{0pt}%
\pgfsys@defobject{currentmarker}{\pgfqpoint{0.000000in}{-0.048611in}}{\pgfqpoint{0.000000in}{0.000000in}}{%
\pgfpathmoveto{\pgfqpoint{0.000000in}{0.000000in}}%
\pgfpathlineto{\pgfqpoint{0.000000in}{-0.048611in}}%
\pgfusepath{stroke,fill}%
}%
\begin{pgfscope}%
\pgfsys@transformshift{1.729114in}{0.538014in}%
\pgfsys@useobject{currentmarker}{}%
\end{pgfscope}%
\end{pgfscope}%
\begin{pgfscope}%
\definecolor{textcolor}{rgb}{0.000000,0.000000,0.000000}%
\pgfsetstrokecolor{textcolor}%
\pgfsetfillcolor{textcolor}%
\pgftext[x=1.729114in,y=0.440792in,,top]{\color{textcolor}\rmfamily\fontsize{8.000000}{9.600000}\selectfont \(\displaystyle {12{:}00}\)}%
\end{pgfscope}%
\begin{pgfscope}%
\pgfsetbuttcap%
\pgfsetroundjoin%
\definecolor{currentfill}{rgb}{0.000000,0.000000,0.000000}%
\pgfsetfillcolor{currentfill}%
\pgfsetlinewidth{0.803000pt}%
\definecolor{currentstroke}{rgb}{0.000000,0.000000,0.000000}%
\pgfsetstrokecolor{currentstroke}%
\pgfsetdash{}{0pt}%
\pgfsys@defobject{currentmarker}{\pgfqpoint{0.000000in}{-0.048611in}}{\pgfqpoint{0.000000in}{0.000000in}}{%
\pgfpathmoveto{\pgfqpoint{0.000000in}{0.000000in}}%
\pgfpathlineto{\pgfqpoint{0.000000in}{-0.048611in}}%
\pgfusepath{stroke,fill}%
}%
\begin{pgfscope}%
\pgfsys@transformshift{2.248160in}{0.538014in}%
\pgfsys@useobject{currentmarker}{}%
\end{pgfscope}%
\end{pgfscope}%
\begin{pgfscope}%
\definecolor{textcolor}{rgb}{0.000000,0.000000,0.000000}%
\pgfsetstrokecolor{textcolor}%
\pgfsetfillcolor{textcolor}%
\pgftext[x=2.248160in,y=0.440792in,,top]{\color{textcolor}\rmfamily\fontsize{8.000000}{9.600000}\selectfont \(\displaystyle {15{:}00}\)}%
\end{pgfscope}%
\begin{pgfscope}%
\pgfsetbuttcap%
\pgfsetroundjoin%
\definecolor{currentfill}{rgb}{0.000000,0.000000,0.000000}%
\pgfsetfillcolor{currentfill}%
\pgfsetlinewidth{0.803000pt}%
\definecolor{currentstroke}{rgb}{0.000000,0.000000,0.000000}%
\pgfsetstrokecolor{currentstroke}%
\pgfsetdash{}{0pt}%
\pgfsys@defobject{currentmarker}{\pgfqpoint{0.000000in}{-0.048611in}}{\pgfqpoint{0.000000in}{0.000000in}}{%
\pgfpathmoveto{\pgfqpoint{0.000000in}{0.000000in}}%
\pgfpathlineto{\pgfqpoint{0.000000in}{-0.048611in}}%
\pgfusepath{stroke,fill}%
}%
\begin{pgfscope}%
\pgfsys@transformshift{2.767206in}{0.538014in}%
\pgfsys@useobject{currentmarker}{}%
\end{pgfscope}%
\end{pgfscope}%
\begin{pgfscope}%
\definecolor{textcolor}{rgb}{0.000000,0.000000,0.000000}%
\pgfsetstrokecolor{textcolor}%
\pgfsetfillcolor{textcolor}%
\pgftext[x=2.767206in,y=0.440792in,,top]{\color{textcolor}\rmfamily\fontsize{8.000000}{9.600000}\selectfont \(\displaystyle {18{:}00}\)}%
\end{pgfscope}%
\begin{pgfscope}%
\pgfsetbuttcap%
\pgfsetroundjoin%
\definecolor{currentfill}{rgb}{0.000000,0.000000,0.000000}%
\pgfsetfillcolor{currentfill}%
\pgfsetlinewidth{0.803000pt}%
\definecolor{currentstroke}{rgb}{0.000000,0.000000,0.000000}%
\pgfsetstrokecolor{currentstroke}%
\pgfsetdash{}{0pt}%
\pgfsys@defobject{currentmarker}{\pgfqpoint{0.000000in}{-0.048611in}}{\pgfqpoint{0.000000in}{0.000000in}}{%
\pgfpathmoveto{\pgfqpoint{0.000000in}{0.000000in}}%
\pgfpathlineto{\pgfqpoint{0.000000in}{-0.048611in}}%
\pgfusepath{stroke,fill}%
}%
\begin{pgfscope}%
\pgfsys@transformshift{3.286252in}{0.538014in}%
\pgfsys@useobject{currentmarker}{}%
\end{pgfscope}%
\end{pgfscope}%
\begin{pgfscope}%
\definecolor{textcolor}{rgb}{0.000000,0.000000,0.000000}%
\pgfsetstrokecolor{textcolor}%
\pgfsetfillcolor{textcolor}%
\pgftext[x=3.286252in,y=0.440792in,,top]{\color{textcolor}\rmfamily\fontsize{8.000000}{9.600000}\selectfont \(\displaystyle {21{:}00}\)}%
\end{pgfscope}%
\begin{pgfscope}%
\pgfsetbuttcap%
\pgfsetroundjoin%
\definecolor{currentfill}{rgb}{0.000000,0.000000,0.000000}%
\pgfsetfillcolor{currentfill}%
\pgfsetlinewidth{0.803000pt}%
\definecolor{currentstroke}{rgb}{0.000000,0.000000,0.000000}%
\pgfsetstrokecolor{currentstroke}%
\pgfsetdash{}{0pt}%
\pgfsys@defobject{currentmarker}{\pgfqpoint{0.000000in}{-0.048611in}}{\pgfqpoint{0.000000in}{0.000000in}}{%
\pgfpathmoveto{\pgfqpoint{0.000000in}{0.000000in}}%
\pgfpathlineto{\pgfqpoint{0.000000in}{-0.048611in}}%
\pgfusepath{stroke,fill}%
}%
\begin{pgfscope}%
\pgfsys@transformshift{3.805298in}{0.538014in}%
\pgfsys@useobject{currentmarker}{}%
\end{pgfscope}%
\end{pgfscope}%
\begin{pgfscope}%
\definecolor{textcolor}{rgb}{0.000000,0.000000,0.000000}%
\pgfsetstrokecolor{textcolor}%
\pgfsetfillcolor{textcolor}%
\pgftext[x=3.805298in,y=0.440792in,,top]{\color{textcolor}\rmfamily\fontsize{8.000000}{9.600000}\selectfont \(\displaystyle {00{:}00}\)}%
\end{pgfscope}%
\begin{pgfscope}%
\pgfsetbuttcap%
\pgfsetroundjoin%
\definecolor{currentfill}{rgb}{0.000000,0.000000,0.000000}%
\pgfsetfillcolor{currentfill}%
\pgfsetlinewidth{0.803000pt}%
\definecolor{currentstroke}{rgb}{0.000000,0.000000,0.000000}%
\pgfsetstrokecolor{currentstroke}%
\pgfsetdash{}{0pt}%
\pgfsys@defobject{currentmarker}{\pgfqpoint{0.000000in}{-0.048611in}}{\pgfqpoint{0.000000in}{0.000000in}}{%
\pgfpathmoveto{\pgfqpoint{0.000000in}{0.000000in}}%
\pgfpathlineto{\pgfqpoint{0.000000in}{-0.048611in}}%
\pgfusepath{stroke,fill}%
}%
\begin{pgfscope}%
\pgfsys@transformshift{4.324344in}{0.538014in}%
\pgfsys@useobject{currentmarker}{}%
\end{pgfscope}%
\end{pgfscope}%
\begin{pgfscope}%
\definecolor{textcolor}{rgb}{0.000000,0.000000,0.000000}%
\pgfsetstrokecolor{textcolor}%
\pgfsetfillcolor{textcolor}%
\pgftext[x=4.324344in,y=0.440792in,,top]{\color{textcolor}\rmfamily\fontsize{8.000000}{9.600000}\selectfont \(\displaystyle {03{:}00}\)}%
\end{pgfscope}%
\begin{pgfscope}%
\pgfsetbuttcap%
\pgfsetroundjoin%
\definecolor{currentfill}{rgb}{0.000000,0.000000,0.000000}%
\pgfsetfillcolor{currentfill}%
\pgfsetlinewidth{0.803000pt}%
\definecolor{currentstroke}{rgb}{0.000000,0.000000,0.000000}%
\pgfsetstrokecolor{currentstroke}%
\pgfsetdash{}{0pt}%
\pgfsys@defobject{currentmarker}{\pgfqpoint{0.000000in}{-0.048611in}}{\pgfqpoint{0.000000in}{0.000000in}}{%
\pgfpathmoveto{\pgfqpoint{0.000000in}{0.000000in}}%
\pgfpathlineto{\pgfqpoint{0.000000in}{-0.048611in}}%
\pgfusepath{stroke,fill}%
}%
\begin{pgfscope}%
\pgfsys@transformshift{4.843390in}{0.538014in}%
\pgfsys@useobject{currentmarker}{}%
\end{pgfscope}%
\end{pgfscope}%
\begin{pgfscope}%
\definecolor{textcolor}{rgb}{0.000000,0.000000,0.000000}%
\pgfsetstrokecolor{textcolor}%
\pgfsetfillcolor{textcolor}%
\pgftext[x=4.843390in,y=0.440792in,,top]{\color{textcolor}\rmfamily\fontsize{8.000000}{9.600000}\selectfont \(\displaystyle {06{:}00}\)}%
\end{pgfscope}%
\begin{pgfscope}%
\definecolor{textcolor}{rgb}{0.000000,0.000000,0.000000}%
\pgfsetstrokecolor{textcolor}%
\pgfsetfillcolor{textcolor}%
\pgftext[x=2.767012in,y=0.286570in,,top]{\color{textcolor}\rmfamily\fontsize{10.000000}{12.000000}\selectfont Time (UTC)}%
\end{pgfscope}%
\begin{pgfscope}%
\pgfsetbuttcap%
\pgfsetroundjoin%
\definecolor{currentfill}{rgb}{0.000000,0.000000,0.000000}%
\pgfsetfillcolor{currentfill}%
\pgfsetlinewidth{0.803000pt}%
\definecolor{currentstroke}{rgb}{0.000000,0.000000,0.000000}%
\pgfsetstrokecolor{currentstroke}%
\pgfsetdash{}{0pt}%
\pgfsys@defobject{currentmarker}{\pgfqpoint{-0.048611in}{0.000000in}}{\pgfqpoint{-0.000000in}{0.000000in}}{%
\pgfpathmoveto{\pgfqpoint{-0.000000in}{0.000000in}}%
\pgfpathlineto{\pgfqpoint{-0.048611in}{0.000000in}}%
\pgfusepath{stroke,fill}%
}%
\begin{pgfscope}%
\pgfsys@transformshift{0.483776in}{0.719191in}%
\pgfsys@useobject{currentmarker}{}%
\end{pgfscope}%
\end{pgfscope}%
\begin{pgfscope}%
\definecolor{textcolor}{rgb}{0.000000,0.000000,0.000000}%
\pgfsetstrokecolor{textcolor}%
\pgfsetfillcolor{textcolor}%
\pgftext[x=0.327525in, y=0.680636in, left, base]{\color{textcolor}\rmfamily\fontsize{8.000000}{9.600000}\selectfont \(\displaystyle {0}\)}%
\end{pgfscope}%
\begin{pgfscope}%
\pgfsetbuttcap%
\pgfsetroundjoin%
\definecolor{currentfill}{rgb}{0.000000,0.000000,0.000000}%
\pgfsetfillcolor{currentfill}%
\pgfsetlinewidth{0.803000pt}%
\definecolor{currentstroke}{rgb}{0.000000,0.000000,0.000000}%
\pgfsetstrokecolor{currentstroke}%
\pgfsetdash{}{0pt}%
\pgfsys@defobject{currentmarker}{\pgfqpoint{-0.048611in}{0.000000in}}{\pgfqpoint{-0.000000in}{0.000000in}}{%
\pgfpathmoveto{\pgfqpoint{-0.000000in}{0.000000in}}%
\pgfpathlineto{\pgfqpoint{-0.048611in}{0.000000in}}%
\pgfusepath{stroke,fill}%
}%
\begin{pgfscope}%
\pgfsys@transformshift{0.483776in}{0.926366in}%
\pgfsys@useobject{currentmarker}{}%
\end{pgfscope}%
\end{pgfscope}%
\begin{pgfscope}%
\definecolor{textcolor}{rgb}{0.000000,0.000000,0.000000}%
\pgfsetstrokecolor{textcolor}%
\pgfsetfillcolor{textcolor}%
\pgftext[x=0.327525in, y=0.887811in, left, base]{\color{textcolor}\rmfamily\fontsize{8.000000}{9.600000}\selectfont \(\displaystyle {5}\)}%
\end{pgfscope}%
\begin{pgfscope}%
\definecolor{textcolor}{rgb}{0.000000,0.000000,0.000000}%
\pgfsetstrokecolor{textcolor}%
\pgfsetfillcolor{textcolor}%
\pgftext[x=0.483776in,y=1.164562in,left,base]{\color{textcolor}\rmfamily\fontsize{8.000000}{9.600000}\selectfont \(\displaystyle \times{10^{\ensuremath{-}6}}{}\)}%
\end{pgfscope}%
\begin{pgfscope}%
\pgfpathrectangle{\pgfqpoint{0.483776in}{0.538014in}}{\pgfqpoint{4.566474in}{0.584881in}}%
\pgfusepath{clip}%
\pgfsetrectcap%
\pgfsetroundjoin%
\pgfsetlinewidth{0.501875pt}%
\definecolor{currentstroke}{rgb}{0.000000,0.419608,0.643137}%
\pgfsetstrokecolor{currentstroke}%
\pgfsetstrokeopacity{0.700000}%
\pgfsetdash{}{0pt}%
\pgfpathmoveto{\pgfqpoint{0.691343in}{0.723309in}}%
\pgfpathlineto{\pgfqpoint{0.692205in}{0.730232in}}%
\pgfpathlineto{\pgfqpoint{0.693935in}{0.701294in}}%
\pgfpathlineto{\pgfqpoint{0.694800in}{0.716533in}}%
\pgfpathlineto{\pgfqpoint{0.695666in}{0.686239in}}%
\pgfpathlineto{\pgfqpoint{0.696532in}{0.687741in}}%
\pgfpathlineto{\pgfqpoint{0.698263in}{0.732136in}}%
\pgfpathlineto{\pgfqpoint{0.699128in}{0.727776in}}%
\pgfpathlineto{\pgfqpoint{0.699993in}{0.774404in}}%
\pgfpathlineto{\pgfqpoint{0.701725in}{0.645250in}}%
\pgfpathlineto{\pgfqpoint{0.703453in}{0.750814in}}%
\pgfpathlineto{\pgfqpoint{0.704319in}{0.749056in}}%
\pgfpathlineto{\pgfqpoint{0.705185in}{0.769605in}}%
\pgfpathlineto{\pgfqpoint{0.709512in}{0.701108in}}%
\pgfpathlineto{\pgfqpoint{0.711241in}{0.778473in}}%
\pgfpathlineto{\pgfqpoint{0.712105in}{0.705834in}}%
\pgfpathlineto{\pgfqpoint{0.712971in}{0.751585in}}%
\pgfpathlineto{\pgfqpoint{0.713837in}{0.688691in}}%
\pgfpathlineto{\pgfqpoint{0.714702in}{0.689609in}}%
\pgfpathlineto{\pgfqpoint{0.715567in}{0.766787in}}%
\pgfpathlineto{\pgfqpoint{0.717297in}{0.689609in}}%
\pgfpathlineto{\pgfqpoint{0.718163in}{0.747850in}}%
\pgfpathlineto{\pgfqpoint{0.719030in}{0.739681in}}%
\pgfpathlineto{\pgfqpoint{0.719895in}{0.652319in}}%
\pgfpathlineto{\pgfqpoint{0.720762in}{0.737703in}}%
\pgfpathlineto{\pgfqpoint{0.721627in}{0.672612in}}%
\pgfpathlineto{\pgfqpoint{0.723356in}{0.735287in}}%
\pgfpathlineto{\pgfqpoint{0.725084in}{0.736384in}}%
\pgfpathlineto{\pgfqpoint{0.725947in}{0.694920in}}%
\pgfpathlineto{\pgfqpoint{0.727676in}{0.751951in}}%
\pgfpathlineto{\pgfqpoint{0.728541in}{0.646309in}}%
\pgfpathlineto{\pgfqpoint{0.729407in}{0.755760in}}%
\pgfpathlineto{\pgfqpoint{0.730271in}{0.732867in}}%
\pgfpathlineto{\pgfqpoint{0.731135in}{0.762134in}}%
\pgfpathlineto{\pgfqpoint{0.732865in}{0.715030in}}%
\pgfpathlineto{\pgfqpoint{0.733731in}{0.722026in}}%
\pgfpathlineto{\pgfqpoint{0.735460in}{0.682540in}}%
\pgfpathlineto{\pgfqpoint{0.736325in}{0.782501in}}%
\pgfpathlineto{\pgfqpoint{0.737190in}{0.721036in}}%
\pgfpathlineto{\pgfqpoint{0.738920in}{0.778546in}}%
\pgfpathlineto{\pgfqpoint{0.740649in}{0.685873in}}%
\pgfpathlineto{\pgfqpoint{0.741516in}{0.685906in}}%
\pgfpathlineto{\pgfqpoint{0.742381in}{0.661365in}}%
\pgfpathlineto{\pgfqpoint{0.744112in}{0.728434in}}%
\pgfpathlineto{\pgfqpoint{0.744977in}{0.676640in}}%
\pgfpathlineto{\pgfqpoint{0.745840in}{0.733452in}}%
\pgfpathlineto{\pgfqpoint{0.746705in}{0.708985in}}%
\pgfpathlineto{\pgfqpoint{0.748435in}{0.780925in}}%
\pgfpathlineto{\pgfqpoint{0.750163in}{0.711949in}}%
\pgfpathlineto{\pgfqpoint{0.751028in}{0.710739in}}%
\pgfpathlineto{\pgfqpoint{0.751891in}{0.701656in}}%
\pgfpathlineto{\pgfqpoint{0.752757in}{0.767518in}}%
\pgfpathlineto{\pgfqpoint{0.753622in}{0.661804in}}%
\pgfpathlineto{\pgfqpoint{0.754488in}{0.702391in}}%
\pgfpathlineto{\pgfqpoint{0.755354in}{0.695577in}}%
\pgfpathlineto{\pgfqpoint{0.756219in}{0.680376in}}%
\pgfpathlineto{\pgfqpoint{0.757950in}{0.793269in}}%
\pgfpathlineto{\pgfqpoint{0.758816in}{0.693965in}}%
\pgfpathlineto{\pgfqpoint{0.759679in}{0.695870in}}%
\pgfpathlineto{\pgfqpoint{0.760542in}{0.741402in}}%
\pgfpathlineto{\pgfqpoint{0.761407in}{0.661584in}}%
\pgfpathlineto{\pgfqpoint{0.763135in}{0.730374in}}%
\pgfpathlineto{\pgfqpoint{0.764000in}{0.670744in}}%
\pgfpathlineto{\pgfqpoint{0.764865in}{0.676494in}}%
\pgfpathlineto{\pgfqpoint{0.765730in}{0.703454in}}%
\pgfpathlineto{\pgfqpoint{0.766596in}{0.664150in}}%
\pgfpathlineto{\pgfqpoint{0.768326in}{0.763892in}}%
\pgfpathlineto{\pgfqpoint{0.770057in}{0.681366in}}%
\pgfpathlineto{\pgfqpoint{0.770922in}{0.749900in}}%
\pgfpathlineto{\pgfqpoint{0.771788in}{0.703674in}}%
\pgfpathlineto{\pgfqpoint{0.772652in}{0.790597in}}%
\pgfpathlineto{\pgfqpoint{0.774383in}{0.694554in}}%
\pgfpathlineto{\pgfqpoint{0.776115in}{0.752576in}}%
\pgfpathlineto{\pgfqpoint{0.776979in}{0.711185in}}%
\pgfpathlineto{\pgfqpoint{0.777845in}{0.739279in}}%
\pgfpathlineto{\pgfqpoint{0.778710in}{0.680708in}}%
\pgfpathlineto{\pgfqpoint{0.779575in}{0.777190in}}%
\pgfpathlineto{\pgfqpoint{0.780439in}{0.774884in}}%
\pgfpathlineto{\pgfqpoint{0.781305in}{0.700856in}}%
\pgfpathlineto{\pgfqpoint{0.782170in}{0.760892in}}%
\pgfpathlineto{\pgfqpoint{0.783036in}{0.713601in}}%
\pgfpathlineto{\pgfqpoint{0.783901in}{0.722099in}}%
\pgfpathlineto{\pgfqpoint{0.785630in}{0.682576in}}%
\pgfpathlineto{\pgfqpoint{0.786494in}{0.747484in}}%
\pgfpathlineto{\pgfqpoint{0.787360in}{0.605032in}}%
\pgfpathlineto{\pgfqpoint{0.788225in}{0.615727in}}%
\pgfpathlineto{\pgfqpoint{0.789089in}{0.747338in}}%
\pgfpathlineto{\pgfqpoint{0.789954in}{0.655104in}}%
\pgfpathlineto{\pgfqpoint{0.791680in}{0.787081in}}%
\pgfpathlineto{\pgfqpoint{0.793412in}{0.719022in}}%
\pgfpathlineto{\pgfqpoint{0.794274in}{0.721255in}}%
\pgfpathlineto{\pgfqpoint{0.795138in}{0.700961in}}%
\pgfpathlineto{\pgfqpoint{0.796004in}{0.820449in}}%
\pgfpathlineto{\pgfqpoint{0.796868in}{0.703637in}}%
\pgfpathlineto{\pgfqpoint{0.797734in}{0.729497in}}%
\pgfpathlineto{\pgfqpoint{0.798599in}{0.747152in}}%
\pgfpathlineto{\pgfqpoint{0.799464in}{0.743343in}}%
\pgfpathlineto{\pgfqpoint{0.800327in}{0.693856in}}%
\pgfpathlineto{\pgfqpoint{0.801192in}{0.755321in}}%
\pgfpathlineto{\pgfqpoint{0.802924in}{0.712022in}}%
\pgfpathlineto{\pgfqpoint{0.803787in}{0.714880in}}%
\pgfpathlineto{\pgfqpoint{0.804652in}{0.653196in}}%
\pgfpathlineto{\pgfqpoint{0.805515in}{0.665028in}}%
\pgfpathlineto{\pgfqpoint{0.808107in}{0.727483in}}%
\pgfpathlineto{\pgfqpoint{0.809837in}{0.608435in}}%
\pgfpathlineto{\pgfqpoint{0.810702in}{0.681439in}}%
\pgfpathlineto{\pgfqpoint{0.811567in}{0.658803in}}%
\pgfpathlineto{\pgfqpoint{0.812431in}{0.718510in}}%
\pgfpathlineto{\pgfqpoint{0.813297in}{0.682576in}}%
\pgfpathlineto{\pgfqpoint{0.814161in}{0.740671in}}%
\pgfpathlineto{\pgfqpoint{0.815026in}{0.689682in}}%
\pgfpathlineto{\pgfqpoint{0.817620in}{0.783272in}}%
\pgfpathlineto{\pgfqpoint{0.819349in}{0.683786in}}%
\pgfpathlineto{\pgfqpoint{0.820213in}{0.773381in}}%
\pgfpathlineto{\pgfqpoint{0.821078in}{0.738913in}}%
\pgfpathlineto{\pgfqpoint{0.821942in}{0.767559in}}%
\pgfpathlineto{\pgfqpoint{0.822807in}{0.760599in}}%
\pgfpathlineto{\pgfqpoint{0.823671in}{0.723821in}}%
\pgfpathlineto{\pgfqpoint{0.824536in}{0.777263in}}%
\pgfpathlineto{\pgfqpoint{0.826268in}{0.719095in}}%
\pgfpathlineto{\pgfqpoint{0.827133in}{0.746859in}}%
\pgfpathlineto{\pgfqpoint{0.828864in}{0.724990in}}%
\pgfpathlineto{\pgfqpoint{0.830596in}{0.855140in}}%
\pgfpathlineto{\pgfqpoint{0.831462in}{0.760266in}}%
\pgfpathlineto{\pgfqpoint{0.832328in}{0.803014in}}%
\pgfpathlineto{\pgfqpoint{0.833193in}{0.746494in}}%
\pgfpathlineto{\pgfqpoint{0.834058in}{0.759316in}}%
\pgfpathlineto{\pgfqpoint{0.834922in}{0.777336in}}%
\pgfpathlineto{\pgfqpoint{0.836650in}{0.714076in}}%
\pgfpathlineto{\pgfqpoint{0.838382in}{0.684590in}}%
\pgfpathlineto{\pgfqpoint{0.840114in}{0.723529in}}%
\pgfpathlineto{\pgfqpoint{0.840980in}{0.720378in}}%
\pgfpathlineto{\pgfqpoint{0.841845in}{0.730927in}}%
\pgfpathlineto{\pgfqpoint{0.843575in}{0.677265in}}%
\pgfpathlineto{\pgfqpoint{0.845305in}{0.697705in}}%
\pgfpathlineto{\pgfqpoint{0.846169in}{0.675836in}}%
\pgfpathlineto{\pgfqpoint{0.847897in}{0.724406in}}%
\pgfpathlineto{\pgfqpoint{0.848762in}{0.724406in}}%
\pgfpathlineto{\pgfqpoint{0.849627in}{0.695139in}}%
\pgfpathlineto{\pgfqpoint{0.851354in}{0.746421in}}%
\pgfpathlineto{\pgfqpoint{0.852218in}{0.710194in}}%
\pgfpathlineto{\pgfqpoint{0.853081in}{0.734077in}}%
\pgfpathlineto{\pgfqpoint{0.853945in}{0.705359in}}%
\pgfpathlineto{\pgfqpoint{0.854810in}{0.759316in}}%
\pgfpathlineto{\pgfqpoint{0.855672in}{0.754736in}}%
\pgfpathlineto{\pgfqpoint{0.856537in}{0.740854in}}%
\pgfpathlineto{\pgfqpoint{0.857402in}{0.761549in}}%
\pgfpathlineto{\pgfqpoint{0.858267in}{0.749279in}}%
\pgfpathlineto{\pgfqpoint{0.859131in}{0.670379in}}%
\pgfpathlineto{\pgfqpoint{0.861730in}{0.738292in}}%
\pgfpathlineto{\pgfqpoint{0.862597in}{0.751147in}}%
\pgfpathlineto{\pgfqpoint{0.864329in}{0.720012in}}%
\pgfpathlineto{\pgfqpoint{0.865195in}{0.678694in}}%
\pgfpathlineto{\pgfqpoint{0.866062in}{0.750745in}}%
\pgfpathlineto{\pgfqpoint{0.866929in}{0.734849in}}%
\pgfpathlineto{\pgfqpoint{0.867794in}{0.774737in}}%
\pgfpathlineto{\pgfqpoint{0.868659in}{0.702175in}}%
\pgfpathlineto{\pgfqpoint{0.870388in}{0.747850in}}%
\pgfpathlineto{\pgfqpoint{0.871253in}{0.663200in}}%
\pgfpathlineto{\pgfqpoint{0.872119in}{0.745178in}}%
\pgfpathlineto{\pgfqpoint{0.872983in}{0.674740in}}%
\pgfpathlineto{\pgfqpoint{0.873848in}{0.766901in}}%
\pgfpathlineto{\pgfqpoint{0.874712in}{0.721368in}}%
\pgfpathlineto{\pgfqpoint{0.875577in}{0.750270in}}%
\pgfpathlineto{\pgfqpoint{0.877309in}{0.707084in}}%
\pgfpathlineto{\pgfqpoint{0.878174in}{0.699024in}}%
\pgfpathlineto{\pgfqpoint{0.879035in}{0.658255in}}%
\pgfpathlineto{\pgfqpoint{0.879903in}{0.667853in}}%
\pgfpathlineto{\pgfqpoint{0.881633in}{0.731990in}}%
\pgfpathlineto{\pgfqpoint{0.882498in}{0.705911in}}%
\pgfpathlineto{\pgfqpoint{0.883364in}{0.746680in}}%
\pgfpathlineto{\pgfqpoint{0.884229in}{0.721661in}}%
\pgfpathlineto{\pgfqpoint{0.885096in}{0.727154in}}%
\pgfpathlineto{\pgfqpoint{0.885960in}{0.755508in}}%
\pgfpathlineto{\pgfqpoint{0.886826in}{0.653675in}}%
\pgfpathlineto{\pgfqpoint{0.888555in}{0.733566in}}%
\pgfpathlineto{\pgfqpoint{0.891148in}{0.786350in}}%
\pgfpathlineto{\pgfqpoint{0.892875in}{0.743530in}}%
\pgfpathlineto{\pgfqpoint{0.893741in}{0.754850in}}%
\pgfpathlineto{\pgfqpoint{0.894605in}{0.728588in}}%
\pgfpathlineto{\pgfqpoint{0.896337in}{0.756019in}}%
\pgfpathlineto{\pgfqpoint{0.898067in}{0.704372in}}%
\pgfpathlineto{\pgfqpoint{0.901525in}{0.744041in}}%
\pgfpathlineto{\pgfqpoint{0.903254in}{0.711697in}}%
\pgfpathlineto{\pgfqpoint{0.904119in}{0.713528in}}%
\pgfpathlineto{\pgfqpoint{0.904984in}{0.687010in}}%
\pgfpathlineto{\pgfqpoint{0.906714in}{0.753238in}}%
\pgfpathlineto{\pgfqpoint{0.907579in}{0.773308in}}%
\pgfpathlineto{\pgfqpoint{0.909309in}{0.708107in}}%
\pgfpathlineto{\pgfqpoint{0.910175in}{0.764700in}}%
\pgfpathlineto{\pgfqpoint{0.911039in}{0.744114in}}%
\pgfpathlineto{\pgfqpoint{0.911905in}{0.653382in}}%
\pgfpathlineto{\pgfqpoint{0.912770in}{0.718364in}}%
\pgfpathlineto{\pgfqpoint{0.913635in}{0.653127in}}%
\pgfpathlineto{\pgfqpoint{0.914501in}{0.730049in}}%
\pgfpathlineto{\pgfqpoint{0.915365in}{0.654446in}}%
\pgfpathlineto{\pgfqpoint{0.917094in}{0.728620in}}%
\pgfpathlineto{\pgfqpoint{0.917959in}{0.706386in}}%
\pgfpathlineto{\pgfqpoint{0.918824in}{0.650232in}}%
\pgfpathlineto{\pgfqpoint{0.919687in}{0.701806in}}%
\pgfpathlineto{\pgfqpoint{0.920552in}{0.664406in}}%
\pgfpathlineto{\pgfqpoint{0.922280in}{0.757229in}}%
\pgfpathlineto{\pgfqpoint{0.924010in}{0.651295in}}%
\pgfpathlineto{\pgfqpoint{0.924875in}{0.746201in}}%
\pgfpathlineto{\pgfqpoint{0.925740in}{0.730415in}}%
\pgfpathlineto{\pgfqpoint{0.926603in}{0.781916in}}%
\pgfpathlineto{\pgfqpoint{0.927464in}{0.683161in}}%
\pgfpathlineto{\pgfqpoint{0.928329in}{0.696860in}}%
\pgfpathlineto{\pgfqpoint{0.929193in}{0.704591in}}%
\pgfpathlineto{\pgfqpoint{0.930058in}{0.660086in}}%
\pgfpathlineto{\pgfqpoint{0.931787in}{0.724665in}}%
\pgfpathlineto{\pgfqpoint{0.933516in}{0.679612in}}%
\pgfpathlineto{\pgfqpoint{0.934381in}{0.716423in}}%
\pgfpathlineto{\pgfqpoint{0.935246in}{0.702066in}}%
\pgfpathlineto{\pgfqpoint{0.936974in}{0.782907in}}%
\pgfpathlineto{\pgfqpoint{0.939570in}{0.666607in}}%
\pgfpathlineto{\pgfqpoint{0.940435in}{0.745657in}}%
\pgfpathlineto{\pgfqpoint{0.942166in}{0.707194in}}%
\pgfpathlineto{\pgfqpoint{0.943032in}{0.768988in}}%
\pgfpathlineto{\pgfqpoint{0.943897in}{0.750416in}}%
\pgfpathlineto{\pgfqpoint{0.944763in}{0.775911in}}%
\pgfpathlineto{\pgfqpoint{0.945628in}{0.719428in}}%
\pgfpathlineto{\pgfqpoint{0.946490in}{0.730488in}}%
\pgfpathlineto{\pgfqpoint{0.947355in}{0.733712in}}%
\pgfpathlineto{\pgfqpoint{0.948218in}{0.769865in}}%
\pgfpathlineto{\pgfqpoint{0.949946in}{0.658547in}}%
\pgfpathlineto{\pgfqpoint{0.950811in}{0.714117in}}%
\pgfpathlineto{\pgfqpoint{0.951674in}{0.637194in}}%
\pgfpathlineto{\pgfqpoint{0.952539in}{0.719135in}}%
\pgfpathlineto{\pgfqpoint{0.953405in}{0.686425in}}%
\pgfpathlineto{\pgfqpoint{0.954269in}{0.714738in}}%
\pgfpathlineto{\pgfqpoint{0.955133in}{0.672100in}}%
\pgfpathlineto{\pgfqpoint{0.955998in}{0.761403in}}%
\pgfpathlineto{\pgfqpoint{0.956863in}{0.704372in}}%
\pgfpathlineto{\pgfqpoint{0.957728in}{0.761111in}}%
\pgfpathlineto{\pgfqpoint{0.958594in}{0.675507in}}%
\pgfpathlineto{\pgfqpoint{0.959459in}{0.765212in}}%
\pgfpathlineto{\pgfqpoint{0.960325in}{0.701587in}}%
\pgfpathlineto{\pgfqpoint{0.961191in}{0.744187in}}%
\pgfpathlineto{\pgfqpoint{0.962055in}{0.679572in}}%
\pgfpathlineto{\pgfqpoint{0.962920in}{0.699938in}}%
\pgfpathlineto{\pgfqpoint{0.963783in}{0.766568in}}%
\pgfpathlineto{\pgfqpoint{0.964647in}{0.713345in}}%
\pgfpathlineto{\pgfqpoint{0.965512in}{0.720012in}}%
\pgfpathlineto{\pgfqpoint{0.968105in}{0.654186in}}%
\pgfpathlineto{\pgfqpoint{0.970701in}{0.725798in}}%
\pgfpathlineto{\pgfqpoint{0.972432in}{0.677558in}}%
\pgfpathlineto{\pgfqpoint{0.973297in}{0.720268in}}%
\pgfpathlineto{\pgfqpoint{0.974162in}{0.684517in}}%
\pgfpathlineto{\pgfqpoint{0.975027in}{0.759316in}}%
\pgfpathlineto{\pgfqpoint{0.975889in}{0.702943in}}%
\pgfpathlineto{\pgfqpoint{0.976751in}{0.739388in}}%
\pgfpathlineto{\pgfqpoint{0.978481in}{0.704884in}}%
\pgfpathlineto{\pgfqpoint{0.979346in}{0.715067in}}%
\pgfpathlineto{\pgfqpoint{0.981076in}{0.809242in}}%
\pgfpathlineto{\pgfqpoint{0.981941in}{0.768143in}}%
\pgfpathlineto{\pgfqpoint{0.984535in}{0.715213in}}%
\pgfpathlineto{\pgfqpoint{0.985400in}{0.741037in}}%
\pgfpathlineto{\pgfqpoint{0.986264in}{0.740891in}}%
\pgfpathlineto{\pgfqpoint{0.987991in}{0.701367in}}%
\pgfpathlineto{\pgfqpoint{0.988856in}{0.761111in}}%
\pgfpathlineto{\pgfqpoint{0.989722in}{0.733821in}}%
\pgfpathlineto{\pgfqpoint{0.990585in}{0.663127in}}%
\pgfpathlineto{\pgfqpoint{0.991449in}{0.756019in}}%
\pgfpathlineto{\pgfqpoint{0.992314in}{0.699280in}}%
\pgfpathlineto{\pgfqpoint{0.993178in}{0.737740in}}%
\pgfpathlineto{\pgfqpoint{0.994043in}{0.669023in}}%
\pgfpathlineto{\pgfqpoint{0.996636in}{0.735251in}}%
\pgfpathlineto{\pgfqpoint{0.997501in}{0.713053in}}%
\pgfpathlineto{\pgfqpoint{1.000961in}{0.750562in}}%
\pgfpathlineto{\pgfqpoint{1.001826in}{0.732210in}}%
\pgfpathlineto{\pgfqpoint{1.002691in}{0.752503in}}%
\pgfpathlineto{\pgfqpoint{1.003553in}{0.714628in}}%
\pgfpathlineto{\pgfqpoint{1.004419in}{0.783491in}}%
\pgfpathlineto{\pgfqpoint{1.005284in}{0.710121in}}%
\pgfpathlineto{\pgfqpoint{1.006149in}{0.733529in}}%
\pgfpathlineto{\pgfqpoint{1.007014in}{0.695399in}}%
\pgfpathlineto{\pgfqpoint{1.007877in}{0.762686in}}%
\pgfpathlineto{\pgfqpoint{1.008742in}{0.689576in}}%
\pgfpathlineto{\pgfqpoint{1.009607in}{0.748548in}}%
\pgfpathlineto{\pgfqpoint{1.011338in}{0.687083in}}%
\pgfpathlineto{\pgfqpoint{1.012203in}{0.694042in}}%
\pgfpathlineto{\pgfqpoint{1.013067in}{0.735580in}}%
\pgfpathlineto{\pgfqpoint{1.013933in}{0.722099in}}%
\pgfpathlineto{\pgfqpoint{1.014798in}{0.685029in}}%
\pgfpathlineto{\pgfqpoint{1.015662in}{0.746421in}}%
\pgfpathlineto{\pgfqpoint{1.016526in}{0.731146in}}%
\pgfpathlineto{\pgfqpoint{1.017391in}{0.740891in}}%
\pgfpathlineto{\pgfqpoint{1.020851in}{0.668182in}}%
\pgfpathlineto{\pgfqpoint{1.022581in}{0.687595in}}%
\pgfpathlineto{\pgfqpoint{1.023445in}{0.749133in}}%
\pgfpathlineto{\pgfqpoint{1.026037in}{0.674407in}}%
\pgfpathlineto{\pgfqpoint{1.027767in}{0.768070in}}%
\pgfpathlineto{\pgfqpoint{1.029494in}{0.734516in}}%
\pgfpathlineto{\pgfqpoint{1.030357in}{0.725798in}}%
\pgfpathlineto{\pgfqpoint{1.031221in}{0.743928in}}%
\pgfpathlineto{\pgfqpoint{1.034679in}{0.669059in}}%
\pgfpathlineto{\pgfqpoint{1.035545in}{0.706897in}}%
\pgfpathlineto{\pgfqpoint{1.036409in}{0.705395in}}%
\pgfpathlineto{\pgfqpoint{1.037275in}{0.706934in}}%
\pgfpathlineto{\pgfqpoint{1.038139in}{0.687302in}}%
\pgfpathlineto{\pgfqpoint{1.040734in}{0.746973in}}%
\pgfpathlineto{\pgfqpoint{1.041600in}{0.711916in}}%
\pgfpathlineto{\pgfqpoint{1.042466in}{0.772025in}}%
\pgfpathlineto{\pgfqpoint{1.043331in}{0.735507in}}%
\pgfpathlineto{\pgfqpoint{1.044195in}{0.803200in}}%
\pgfpathlineto{\pgfqpoint{1.045927in}{0.680270in}}%
\pgfpathlineto{\pgfqpoint{1.046792in}{0.684188in}}%
\pgfpathlineto{\pgfqpoint{1.047657in}{0.731771in}}%
\pgfpathlineto{\pgfqpoint{1.049390in}{0.692759in}}%
\pgfpathlineto{\pgfqpoint{1.050256in}{0.744187in}}%
\pgfpathlineto{\pgfqpoint{1.051987in}{0.642980in}}%
\pgfpathlineto{\pgfqpoint{1.052852in}{0.681260in}}%
\pgfpathlineto{\pgfqpoint{1.053719in}{0.660565in}}%
\pgfpathlineto{\pgfqpoint{1.056313in}{0.771294in}}%
\pgfpathlineto{\pgfqpoint{1.057179in}{0.766422in}}%
\pgfpathlineto{\pgfqpoint{1.058042in}{0.690161in}}%
\pgfpathlineto{\pgfqpoint{1.058907in}{0.750562in}}%
\pgfpathlineto{\pgfqpoint{1.059773in}{0.680781in}}%
\pgfpathlineto{\pgfqpoint{1.060637in}{0.712614in}}%
\pgfpathlineto{\pgfqpoint{1.062369in}{0.669721in}}%
\pgfpathlineto{\pgfqpoint{1.063234in}{0.676936in}}%
\pgfpathlineto{\pgfqpoint{1.064962in}{0.771587in}}%
\pgfpathlineto{\pgfqpoint{1.065827in}{0.659830in}}%
\pgfpathlineto{\pgfqpoint{1.066692in}{0.682138in}}%
\pgfpathlineto{\pgfqpoint{1.067557in}{0.745105in}}%
\pgfpathlineto{\pgfqpoint{1.068422in}{0.636240in}}%
\pgfpathlineto{\pgfqpoint{1.070151in}{0.752763in}}%
\pgfpathlineto{\pgfqpoint{1.073608in}{0.646975in}}%
\pgfpathlineto{\pgfqpoint{1.074470in}{0.784994in}}%
\pgfpathlineto{\pgfqpoint{1.075335in}{0.659867in}}%
\pgfpathlineto{\pgfqpoint{1.076199in}{0.704664in}}%
\pgfpathlineto{\pgfqpoint{1.077064in}{0.680197in}}%
\pgfpathlineto{\pgfqpoint{1.079657in}{0.732136in}}%
\pgfpathlineto{\pgfqpoint{1.081384in}{0.673018in}}%
\pgfpathlineto{\pgfqpoint{1.083981in}{0.753859in}}%
\pgfpathlineto{\pgfqpoint{1.085711in}{0.664045in}}%
\pgfpathlineto{\pgfqpoint{1.087440in}{0.773970in}}%
\pgfpathlineto{\pgfqpoint{1.088306in}{0.790857in}}%
\pgfpathlineto{\pgfqpoint{1.089172in}{0.718770in}}%
\pgfpathlineto{\pgfqpoint{1.090037in}{0.751553in}}%
\pgfpathlineto{\pgfqpoint{1.091767in}{0.683457in}}%
\pgfpathlineto{\pgfqpoint{1.095225in}{0.738990in}}%
\pgfpathlineto{\pgfqpoint{1.096956in}{0.696426in}}%
\pgfpathlineto{\pgfqpoint{1.097822in}{0.734556in}}%
\pgfpathlineto{\pgfqpoint{1.098687in}{0.723163in}}%
\pgfpathlineto{\pgfqpoint{1.100416in}{0.743384in}}%
\pgfpathlineto{\pgfqpoint{1.101281in}{0.768659in}}%
\pgfpathlineto{\pgfqpoint{1.102145in}{0.675032in}}%
\pgfpathlineto{\pgfqpoint{1.103008in}{0.687156in}}%
\pgfpathlineto{\pgfqpoint{1.104736in}{0.725656in}}%
\pgfpathlineto{\pgfqpoint{1.105601in}{0.702066in}}%
\pgfpathlineto{\pgfqpoint{1.106466in}{0.712834in}}%
\pgfpathlineto{\pgfqpoint{1.108192in}{0.692394in}}%
\pgfpathlineto{\pgfqpoint{1.109055in}{0.796387in}}%
\pgfpathlineto{\pgfqpoint{1.110782in}{0.715363in}}%
\pgfpathlineto{\pgfqpoint{1.112512in}{0.694116in}}%
\pgfpathlineto{\pgfqpoint{1.114239in}{0.690599in}}%
\pgfpathlineto{\pgfqpoint{1.115102in}{0.721368in}}%
\pgfpathlineto{\pgfqpoint{1.115967in}{0.717008in}}%
\pgfpathlineto{\pgfqpoint{1.116831in}{0.740964in}}%
\pgfpathlineto{\pgfqpoint{1.117696in}{0.729497in}}%
\pgfpathlineto{\pgfqpoint{1.118561in}{0.642980in}}%
\pgfpathlineto{\pgfqpoint{1.119426in}{0.672685in}}%
\pgfpathlineto{\pgfqpoint{1.120289in}{0.645067in}}%
\pgfpathlineto{\pgfqpoint{1.122016in}{0.727337in}}%
\pgfpathlineto{\pgfqpoint{1.122881in}{0.703454in}}%
\pgfpathlineto{\pgfqpoint{1.124612in}{0.771806in}}%
\pgfpathlineto{\pgfqpoint{1.125476in}{0.692979in}}%
\pgfpathlineto{\pgfqpoint{1.126342in}{0.761330in}}%
\pgfpathlineto{\pgfqpoint{1.128072in}{0.700198in}}%
\pgfpathlineto{\pgfqpoint{1.128936in}{0.725396in}}%
\pgfpathlineto{\pgfqpoint{1.130665in}{0.657231in}}%
\pgfpathlineto{\pgfqpoint{1.132396in}{0.717816in}}%
\pgfpathlineto{\pgfqpoint{1.133261in}{0.681187in}}%
\pgfpathlineto{\pgfqpoint{1.134991in}{0.716058in}}%
\pgfpathlineto{\pgfqpoint{1.135856in}{0.729976in}}%
\pgfpathlineto{\pgfqpoint{1.136722in}{0.672466in}}%
\pgfpathlineto{\pgfqpoint{1.137586in}{0.683640in}}%
\pgfpathlineto{\pgfqpoint{1.140181in}{0.745544in}}%
\pgfpathlineto{\pgfqpoint{1.141044in}{0.741296in}}%
\pgfpathlineto{\pgfqpoint{1.141910in}{0.729794in}}%
\pgfpathlineto{\pgfqpoint{1.143641in}{0.656606in}}%
\pgfpathlineto{\pgfqpoint{1.144506in}{0.711112in}}%
\pgfpathlineto{\pgfqpoint{1.145371in}{0.676607in}}%
\pgfpathlineto{\pgfqpoint{1.146235in}{0.767120in}}%
\pgfpathlineto{\pgfqpoint{1.147100in}{0.765983in}}%
\pgfpathlineto{\pgfqpoint{1.148829in}{0.697778in}}%
\pgfpathlineto{\pgfqpoint{1.149692in}{0.749133in}}%
\pgfpathlineto{\pgfqpoint{1.151424in}{0.709135in}}%
\pgfpathlineto{\pgfqpoint{1.153155in}{0.748694in}}%
\pgfpathlineto{\pgfqpoint{1.154020in}{0.690892in}}%
\pgfpathlineto{\pgfqpoint{1.154885in}{0.743895in}}%
\pgfpathlineto{\pgfqpoint{1.155749in}{0.740817in}}%
\pgfpathlineto{\pgfqpoint{1.156614in}{0.771806in}}%
\pgfpathlineto{\pgfqpoint{1.159208in}{0.574742in}}%
\pgfpathlineto{\pgfqpoint{1.160937in}{0.763567in}}%
\pgfpathlineto{\pgfqpoint{1.162667in}{0.719135in}}%
\pgfpathlineto{\pgfqpoint{1.163531in}{0.779098in}}%
\pgfpathlineto{\pgfqpoint{1.165259in}{0.705801in}}%
\pgfpathlineto{\pgfqpoint{1.166990in}{0.769426in}}%
\pgfpathlineto{\pgfqpoint{1.167854in}{0.744849in}}%
\pgfpathlineto{\pgfqpoint{1.168720in}{0.741808in}}%
\pgfpathlineto{\pgfqpoint{1.169584in}{0.682430in}}%
\pgfpathlineto{\pgfqpoint{1.171315in}{0.720784in}}%
\pgfpathlineto{\pgfqpoint{1.172180in}{0.700417in}}%
\pgfpathlineto{\pgfqpoint{1.173912in}{0.728401in}}%
\pgfpathlineto{\pgfqpoint{1.174778in}{0.726972in}}%
\pgfpathlineto{\pgfqpoint{1.176508in}{0.668657in}}%
\pgfpathlineto{\pgfqpoint{1.178239in}{0.722026in}}%
\pgfpathlineto{\pgfqpoint{1.179103in}{0.707742in}}%
\pgfpathlineto{\pgfqpoint{1.181699in}{0.761330in}}%
\pgfpathlineto{\pgfqpoint{1.182565in}{0.723894in}}%
\pgfpathlineto{\pgfqpoint{1.183430in}{0.790378in}}%
\pgfpathlineto{\pgfqpoint{1.185160in}{0.677704in}}%
\pgfpathlineto{\pgfqpoint{1.186891in}{0.733968in}}%
\pgfpathlineto{\pgfqpoint{1.188623in}{0.779058in}}%
\pgfpathlineto{\pgfqpoint{1.189487in}{0.690197in}}%
\pgfpathlineto{\pgfqpoint{1.190352in}{0.744041in}}%
\pgfpathlineto{\pgfqpoint{1.191218in}{0.659465in}}%
\pgfpathlineto{\pgfqpoint{1.192083in}{0.837668in}}%
\pgfpathlineto{\pgfqpoint{1.192948in}{0.716277in}}%
\pgfpathlineto{\pgfqpoint{1.193813in}{0.736164in}}%
\pgfpathlineto{\pgfqpoint{1.194678in}{0.747484in}}%
\pgfpathlineto{\pgfqpoint{1.195541in}{0.708692in}}%
\pgfpathlineto{\pgfqpoint{1.196406in}{0.730378in}}%
\pgfpathlineto{\pgfqpoint{1.198137in}{0.703089in}}%
\pgfpathlineto{\pgfqpoint{1.199001in}{0.763088in}}%
\pgfpathlineto{\pgfqpoint{1.199865in}{0.700304in}}%
\pgfpathlineto{\pgfqpoint{1.200729in}{0.746348in}}%
\pgfpathlineto{\pgfqpoint{1.201595in}{0.691586in}}%
\pgfpathlineto{\pgfqpoint{1.202459in}{0.747923in}}%
\pgfpathlineto{\pgfqpoint{1.203324in}{0.673489in}}%
\pgfpathlineto{\pgfqpoint{1.204189in}{0.702683in}}%
\pgfpathlineto{\pgfqpoint{1.205054in}{0.696089in}}%
\pgfpathlineto{\pgfqpoint{1.205919in}{0.693710in}}%
\pgfpathlineto{\pgfqpoint{1.207651in}{0.749718in}}%
\pgfpathlineto{\pgfqpoint{1.208516in}{0.715140in}}%
\pgfpathlineto{\pgfqpoint{1.209379in}{0.720232in}}%
\pgfpathlineto{\pgfqpoint{1.211111in}{0.680891in}}%
\pgfpathlineto{\pgfqpoint{1.211977in}{0.735141in}}%
\pgfpathlineto{\pgfqpoint{1.212842in}{0.730049in}}%
\pgfpathlineto{\pgfqpoint{1.213706in}{0.732136in}}%
\pgfpathlineto{\pgfqpoint{1.214572in}{0.690599in}}%
\pgfpathlineto{\pgfqpoint{1.217169in}{0.817558in}}%
\pgfpathlineto{\pgfqpoint{1.218035in}{0.682174in}}%
\pgfpathlineto{\pgfqpoint{1.218902in}{0.796826in}}%
\pgfpathlineto{\pgfqpoint{1.220634in}{0.682942in}}%
\pgfpathlineto{\pgfqpoint{1.222366in}{0.742247in}}%
\pgfpathlineto{\pgfqpoint{1.223232in}{0.735507in}}%
\pgfpathlineto{\pgfqpoint{1.224098in}{0.737520in}}%
\pgfpathlineto{\pgfqpoint{1.224965in}{0.768399in}}%
\pgfpathlineto{\pgfqpoint{1.225831in}{0.674553in}}%
\pgfpathlineto{\pgfqpoint{1.228427in}{0.727045in}}%
\pgfpathlineto{\pgfqpoint{1.230156in}{0.686165in}}%
\pgfpathlineto{\pgfqpoint{1.231020in}{0.728913in}}%
\pgfpathlineto{\pgfqpoint{1.232751in}{0.696568in}}%
\pgfpathlineto{\pgfqpoint{1.233617in}{0.701952in}}%
\pgfpathlineto{\pgfqpoint{1.234482in}{0.668913in}}%
\pgfpathlineto{\pgfqpoint{1.236211in}{0.728182in}}%
\pgfpathlineto{\pgfqpoint{1.237076in}{0.737817in}}%
\pgfpathlineto{\pgfqpoint{1.237942in}{0.678548in}}%
\pgfpathlineto{\pgfqpoint{1.238807in}{0.765910in}}%
\pgfpathlineto{\pgfqpoint{1.239671in}{0.757339in}}%
\pgfpathlineto{\pgfqpoint{1.240537in}{0.761257in}}%
\pgfpathlineto{\pgfqpoint{1.241402in}{0.677558in}}%
\pgfpathlineto{\pgfqpoint{1.242268in}{0.686019in}}%
\pgfpathlineto{\pgfqpoint{1.243999in}{0.733895in}}%
\pgfpathlineto{\pgfqpoint{1.244864in}{0.687668in}}%
\pgfpathlineto{\pgfqpoint{1.245728in}{0.748069in}}%
\pgfpathlineto{\pgfqpoint{1.246593in}{0.706678in}}%
\pgfpathlineto{\pgfqpoint{1.247458in}{0.716788in}}%
\pgfpathlineto{\pgfqpoint{1.249187in}{0.708660in}}%
\pgfpathlineto{\pgfqpoint{1.250053in}{0.695179in}}%
\pgfpathlineto{\pgfqpoint{1.250916in}{0.756864in}}%
\pgfpathlineto{\pgfqpoint{1.251781in}{0.736643in}}%
\pgfpathlineto{\pgfqpoint{1.252647in}{0.747484in}}%
\pgfpathlineto{\pgfqpoint{1.253512in}{0.704445in}}%
\pgfpathlineto{\pgfqpoint{1.254377in}{0.776971in}}%
\pgfpathlineto{\pgfqpoint{1.255243in}{0.721807in}}%
\pgfpathlineto{\pgfqpoint{1.256109in}{0.762906in}}%
\pgfpathlineto{\pgfqpoint{1.256974in}{0.742100in}}%
\pgfpathlineto{\pgfqpoint{1.257840in}{0.781112in}}%
\pgfpathlineto{\pgfqpoint{1.258706in}{0.671370in}}%
\pgfpathlineto{\pgfqpoint{1.259571in}{0.697924in}}%
\pgfpathlineto{\pgfqpoint{1.260436in}{0.721734in}}%
\pgfpathlineto{\pgfqpoint{1.261299in}{0.698988in}}%
\pgfpathlineto{\pgfqpoint{1.262165in}{0.755946in}}%
\pgfpathlineto{\pgfqpoint{1.263030in}{0.675032in}}%
\pgfpathlineto{\pgfqpoint{1.264759in}{0.711404in}}%
\pgfpathlineto{\pgfqpoint{1.265625in}{0.719464in}}%
\pgfpathlineto{\pgfqpoint{1.266490in}{0.798328in}}%
\pgfpathlineto{\pgfqpoint{1.269081in}{0.714336in}}%
\pgfpathlineto{\pgfqpoint{1.269946in}{0.713970in}}%
\pgfpathlineto{\pgfqpoint{1.270811in}{0.757229in}}%
\pgfpathlineto{\pgfqpoint{1.271677in}{0.747558in}}%
\pgfpathlineto{\pgfqpoint{1.272543in}{0.735507in}}%
\pgfpathlineto{\pgfqpoint{1.274272in}{0.647852in}}%
\pgfpathlineto{\pgfqpoint{1.276003in}{0.772244in}}%
\pgfpathlineto{\pgfqpoint{1.276869in}{0.718510in}}%
\pgfpathlineto{\pgfqpoint{1.277734in}{0.799757in}}%
\pgfpathlineto{\pgfqpoint{1.279465in}{0.745471in}}%
\pgfpathlineto{\pgfqpoint{1.281196in}{0.705874in}}%
\pgfpathlineto{\pgfqpoint{1.282062in}{0.709244in}}%
\pgfpathlineto{\pgfqpoint{1.282927in}{0.805433in}}%
\pgfpathlineto{\pgfqpoint{1.283793in}{0.702797in}}%
\pgfpathlineto{\pgfqpoint{1.284658in}{0.785802in}}%
\pgfpathlineto{\pgfqpoint{1.285521in}{0.688732in}}%
\pgfpathlineto{\pgfqpoint{1.286386in}{0.704810in}}%
\pgfpathlineto{\pgfqpoint{1.287250in}{0.708254in}}%
\pgfpathlineto{\pgfqpoint{1.288114in}{0.726606in}}%
\pgfpathlineto{\pgfqpoint{1.288980in}{0.771002in}}%
\pgfpathlineto{\pgfqpoint{1.289846in}{0.705874in}}%
\pgfpathlineto{\pgfqpoint{1.290711in}{0.804223in}}%
\pgfpathlineto{\pgfqpoint{1.291576in}{0.756202in}}%
\pgfpathlineto{\pgfqpoint{1.292440in}{0.800342in}}%
\pgfpathlineto{\pgfqpoint{1.293306in}{0.799059in}}%
\pgfpathlineto{\pgfqpoint{1.295037in}{0.714628in}}%
\pgfpathlineto{\pgfqpoint{1.295904in}{0.815836in}}%
\pgfpathlineto{\pgfqpoint{1.297633in}{0.645652in}}%
\pgfpathlineto{\pgfqpoint{1.298498in}{0.659611in}}%
\pgfpathlineto{\pgfqpoint{1.299363in}{0.623092in}}%
\pgfpathlineto{\pgfqpoint{1.300229in}{0.707815in}}%
\pgfpathlineto{\pgfqpoint{1.301092in}{0.675324in}}%
\pgfpathlineto{\pgfqpoint{1.301958in}{0.697413in}}%
\pgfpathlineto{\pgfqpoint{1.302823in}{0.697120in}}%
\pgfpathlineto{\pgfqpoint{1.304552in}{0.714263in}}%
\pgfpathlineto{\pgfqpoint{1.305417in}{0.762979in}}%
\pgfpathlineto{\pgfqpoint{1.307147in}{0.655177in}}%
\pgfpathlineto{\pgfqpoint{1.308013in}{0.760892in}}%
\pgfpathlineto{\pgfqpoint{1.308879in}{0.757302in}}%
\pgfpathlineto{\pgfqpoint{1.309745in}{0.756019in}}%
\pgfpathlineto{\pgfqpoint{1.310610in}{0.707121in}}%
\pgfpathlineto{\pgfqpoint{1.311476in}{0.765764in}}%
\pgfpathlineto{\pgfqpoint{1.312342in}{0.714409in}}%
\pgfpathlineto{\pgfqpoint{1.313208in}{0.784153in}}%
\pgfpathlineto{\pgfqpoint{1.314074in}{0.733566in}}%
\pgfpathlineto{\pgfqpoint{1.315803in}{0.792834in}}%
\pgfpathlineto{\pgfqpoint{1.318395in}{0.699134in}}%
\pgfpathlineto{\pgfqpoint{1.320124in}{0.739429in}}%
\pgfpathlineto{\pgfqpoint{1.321854in}{0.700636in}}%
\pgfpathlineto{\pgfqpoint{1.322720in}{0.723163in}}%
\pgfpathlineto{\pgfqpoint{1.323583in}{0.709025in}}%
\pgfpathlineto{\pgfqpoint{1.324448in}{0.724592in}}%
\pgfpathlineto{\pgfqpoint{1.326179in}{0.681918in}}%
\pgfpathlineto{\pgfqpoint{1.329636in}{0.757704in}}%
\pgfpathlineto{\pgfqpoint{1.330500in}{0.768842in}}%
\pgfpathlineto{\pgfqpoint{1.332230in}{0.725583in}}%
\pgfpathlineto{\pgfqpoint{1.333092in}{0.738146in}}%
\pgfpathlineto{\pgfqpoint{1.333954in}{0.735214in}}%
\pgfpathlineto{\pgfqpoint{1.334819in}{0.674995in}}%
\pgfpathlineto{\pgfqpoint{1.336550in}{0.743457in}}%
\pgfpathlineto{\pgfqpoint{1.337413in}{0.754923in}}%
\pgfpathlineto{\pgfqpoint{1.338278in}{0.709317in}}%
\pgfpathlineto{\pgfqpoint{1.340007in}{0.745251in}}%
\pgfpathlineto{\pgfqpoint{1.340869in}{0.709390in}}%
\pgfpathlineto{\pgfqpoint{1.341734in}{0.729684in}}%
\pgfpathlineto{\pgfqpoint{1.342599in}{0.692394in}}%
\pgfpathlineto{\pgfqpoint{1.343462in}{0.752868in}}%
\pgfpathlineto{\pgfqpoint{1.344329in}{0.670415in}}%
\pgfpathlineto{\pgfqpoint{1.346061in}{0.761549in}}%
\pgfpathlineto{\pgfqpoint{1.347786in}{0.699792in}}%
\pgfpathlineto{\pgfqpoint{1.348652in}{0.755654in}}%
\pgfpathlineto{\pgfqpoint{1.349515in}{0.709902in}}%
\pgfpathlineto{\pgfqpoint{1.350380in}{0.747484in}}%
\pgfpathlineto{\pgfqpoint{1.352107in}{0.668365in}}%
\pgfpathlineto{\pgfqpoint{1.352972in}{0.706240in}}%
\pgfpathlineto{\pgfqpoint{1.353837in}{0.646752in}}%
\pgfpathlineto{\pgfqpoint{1.354702in}{0.781478in}}%
\pgfpathlineto{\pgfqpoint{1.356431in}{0.717227in}}%
\pgfpathlineto{\pgfqpoint{1.357296in}{0.790524in}}%
\pgfpathlineto{\pgfqpoint{1.359028in}{0.696495in}}%
\pgfpathlineto{\pgfqpoint{1.359893in}{0.784003in}}%
\pgfpathlineto{\pgfqpoint{1.360758in}{0.655141in}}%
\pgfpathlineto{\pgfqpoint{1.362488in}{0.752211in}}%
\pgfpathlineto{\pgfqpoint{1.363353in}{0.750672in}}%
\pgfpathlineto{\pgfqpoint{1.364217in}{0.719354in}}%
\pgfpathlineto{\pgfqpoint{1.365081in}{0.794885in}}%
\pgfpathlineto{\pgfqpoint{1.365947in}{0.732762in}}%
\pgfpathlineto{\pgfqpoint{1.366813in}{0.761955in}}%
\pgfpathlineto{\pgfqpoint{1.368543in}{0.666936in}}%
\pgfpathlineto{\pgfqpoint{1.369408in}{0.677964in}}%
\pgfpathlineto{\pgfqpoint{1.370273in}{0.704810in}}%
\pgfpathlineto{\pgfqpoint{1.371138in}{0.683859in}}%
\pgfpathlineto{\pgfqpoint{1.372002in}{0.747996in}}%
\pgfpathlineto{\pgfqpoint{1.372866in}{0.696458in}}%
\pgfpathlineto{\pgfqpoint{1.373729in}{0.736424in}}%
\pgfpathlineto{\pgfqpoint{1.374594in}{0.731698in}}%
\pgfpathlineto{\pgfqpoint{1.375459in}{0.692248in}}%
\pgfpathlineto{\pgfqpoint{1.376322in}{0.697924in}}%
\pgfpathlineto{\pgfqpoint{1.377187in}{0.726570in}}%
\pgfpathlineto{\pgfqpoint{1.378916in}{0.625837in}}%
\pgfpathlineto{\pgfqpoint{1.379781in}{0.713089in}}%
\pgfpathlineto{\pgfqpoint{1.380646in}{0.681074in}}%
\pgfpathlineto{\pgfqpoint{1.381509in}{0.740817in}}%
\pgfpathlineto{\pgfqpoint{1.382373in}{0.684152in}}%
\pgfpathlineto{\pgfqpoint{1.383238in}{0.710966in}}%
\pgfpathlineto{\pgfqpoint{1.384968in}{0.682211in}}%
\pgfpathlineto{\pgfqpoint{1.385831in}{0.717008in}}%
\pgfpathlineto{\pgfqpoint{1.386696in}{0.699244in}}%
\pgfpathlineto{\pgfqpoint{1.387562in}{0.786569in}}%
\pgfpathlineto{\pgfqpoint{1.389291in}{0.703089in}}%
\pgfpathlineto{\pgfqpoint{1.390156in}{0.704884in}}%
\pgfpathlineto{\pgfqpoint{1.391021in}{0.743237in}}%
\pgfpathlineto{\pgfqpoint{1.391886in}{0.715213in}}%
\pgfpathlineto{\pgfqpoint{1.393613in}{0.731990in}}%
\pgfpathlineto{\pgfqpoint{1.395342in}{0.699426in}}%
\pgfpathlineto{\pgfqpoint{1.396208in}{0.669794in}}%
\pgfpathlineto{\pgfqpoint{1.397073in}{0.716131in}}%
\pgfpathlineto{\pgfqpoint{1.397938in}{0.664922in}}%
\pgfpathlineto{\pgfqpoint{1.399668in}{0.738438in}}%
\pgfpathlineto{\pgfqpoint{1.400534in}{0.716058in}}%
\pgfpathlineto{\pgfqpoint{1.401400in}{0.721478in}}%
\pgfpathlineto{\pgfqpoint{1.402265in}{0.776605in}}%
\pgfpathlineto{\pgfqpoint{1.403131in}{0.747119in}}%
\pgfpathlineto{\pgfqpoint{1.403997in}{0.760015in}}%
\pgfpathlineto{\pgfqpoint{1.404863in}{0.751845in}}%
\pgfpathlineto{\pgfqpoint{1.406594in}{0.716788in}}%
\pgfpathlineto{\pgfqpoint{1.408324in}{0.750635in}}%
\pgfpathlineto{\pgfqpoint{1.409190in}{0.735580in}}%
\pgfpathlineto{\pgfqpoint{1.410055in}{0.800196in}}%
\pgfpathlineto{\pgfqpoint{1.411785in}{0.757375in}}%
\pgfpathlineto{\pgfqpoint{1.412651in}{0.760965in}}%
\pgfpathlineto{\pgfqpoint{1.413515in}{0.709390in}}%
\pgfpathlineto{\pgfqpoint{1.414379in}{0.782322in}}%
\pgfpathlineto{\pgfqpoint{1.415244in}{0.766312in}}%
\pgfpathlineto{\pgfqpoint{1.416108in}{0.701075in}}%
\pgfpathlineto{\pgfqpoint{1.416972in}{0.785798in}}%
\pgfpathlineto{\pgfqpoint{1.418701in}{0.702723in}}%
\pgfpathlineto{\pgfqpoint{1.419566in}{0.747558in}}%
\pgfpathlineto{\pgfqpoint{1.420430in}{0.717596in}}%
\pgfpathlineto{\pgfqpoint{1.421294in}{0.727743in}}%
\pgfpathlineto{\pgfqpoint{1.422159in}{0.709025in}}%
\pgfpathlineto{\pgfqpoint{1.423024in}{0.618732in}}%
\pgfpathlineto{\pgfqpoint{1.423884in}{0.720638in}}%
\pgfpathlineto{\pgfqpoint{1.424749in}{0.649610in}}%
\pgfpathlineto{\pgfqpoint{1.426480in}{0.770417in}}%
\pgfpathlineto{\pgfqpoint{1.427345in}{0.746790in}}%
\pgfpathlineto{\pgfqpoint{1.428206in}{0.736936in}}%
\pgfpathlineto{\pgfqpoint{1.429937in}{0.632102in}}%
\pgfpathlineto{\pgfqpoint{1.431667in}{0.753932in}}%
\pgfpathlineto{\pgfqpoint{1.434265in}{0.685288in}}%
\pgfpathlineto{\pgfqpoint{1.435996in}{0.743164in}}%
\pgfpathlineto{\pgfqpoint{1.436861in}{0.699317in}}%
\pgfpathlineto{\pgfqpoint{1.437726in}{0.711039in}}%
\pgfpathlineto{\pgfqpoint{1.438592in}{0.734703in}}%
\pgfpathlineto{\pgfqpoint{1.439455in}{0.680124in}}%
\pgfpathlineto{\pgfqpoint{1.440320in}{0.717706in}}%
\pgfpathlineto{\pgfqpoint{1.441185in}{0.698988in}}%
\pgfpathlineto{\pgfqpoint{1.442050in}{0.707815in}}%
\pgfpathlineto{\pgfqpoint{1.443778in}{0.689353in}}%
\pgfpathlineto{\pgfqpoint{1.447237in}{0.758472in}}%
\pgfpathlineto{\pgfqpoint{1.448103in}{0.729351in}}%
\pgfpathlineto{\pgfqpoint{1.448967in}{0.658949in}}%
\pgfpathlineto{\pgfqpoint{1.449831in}{0.719095in}}%
\pgfpathlineto{\pgfqpoint{1.451559in}{0.667155in}}%
\pgfpathlineto{\pgfqpoint{1.452425in}{0.738398in}}%
\pgfpathlineto{\pgfqpoint{1.453289in}{0.696129in}}%
\pgfpathlineto{\pgfqpoint{1.454154in}{0.802356in}}%
\pgfpathlineto{\pgfqpoint{1.455019in}{0.748142in}}%
\pgfpathlineto{\pgfqpoint{1.455884in}{0.775797in}}%
\pgfpathlineto{\pgfqpoint{1.457614in}{0.724698in}}%
\pgfpathlineto{\pgfqpoint{1.458480in}{0.639021in}}%
\pgfpathlineto{\pgfqpoint{1.459345in}{0.733160in}}%
\pgfpathlineto{\pgfqpoint{1.460209in}{0.719095in}}%
\pgfpathlineto{\pgfqpoint{1.461941in}{0.665433in}}%
\pgfpathlineto{\pgfqpoint{1.462806in}{0.686092in}}%
\pgfpathlineto{\pgfqpoint{1.463672in}{0.749279in}}%
\pgfpathlineto{\pgfqpoint{1.464538in}{0.745138in}}%
\pgfpathlineto{\pgfqpoint{1.466268in}{0.663639in}}%
\pgfpathlineto{\pgfqpoint{1.467999in}{0.720049in}}%
\pgfpathlineto{\pgfqpoint{1.468864in}{0.744553in}}%
\pgfpathlineto{\pgfqpoint{1.470594in}{0.690193in}}%
\pgfpathlineto{\pgfqpoint{1.471459in}{0.759349in}}%
\pgfpathlineto{\pgfqpoint{1.472326in}{0.608544in}}%
\pgfpathlineto{\pgfqpoint{1.473192in}{0.706971in}}%
\pgfpathlineto{\pgfqpoint{1.474056in}{0.696860in}}%
\pgfpathlineto{\pgfqpoint{1.474922in}{0.674699in}}%
\pgfpathlineto{\pgfqpoint{1.475788in}{0.686312in}}%
\pgfpathlineto{\pgfqpoint{1.478386in}{0.755873in}}%
\pgfpathlineto{\pgfqpoint{1.480117in}{0.723675in}}%
\pgfpathlineto{\pgfqpoint{1.481847in}{0.744955in}}%
\pgfpathlineto{\pgfqpoint{1.483575in}{0.674261in}}%
\pgfpathlineto{\pgfqpoint{1.484441in}{0.652172in}}%
\pgfpathlineto{\pgfqpoint{1.485307in}{0.731438in}}%
\pgfpathlineto{\pgfqpoint{1.486174in}{0.719387in}}%
\pgfpathlineto{\pgfqpoint{1.487040in}{0.634847in}}%
\pgfpathlineto{\pgfqpoint{1.489635in}{0.749060in}}%
\pgfpathlineto{\pgfqpoint{1.492231in}{0.682795in}}%
\pgfpathlineto{\pgfqpoint{1.493097in}{0.779902in}}%
\pgfpathlineto{\pgfqpoint{1.494827in}{0.691769in}}%
\pgfpathlineto{\pgfqpoint{1.497424in}{0.778546in}}%
\pgfpathlineto{\pgfqpoint{1.499153in}{0.691696in}}%
\pgfpathlineto{\pgfqpoint{1.500017in}{0.726493in}}%
\pgfpathlineto{\pgfqpoint{1.500882in}{0.724808in}}%
\pgfpathlineto{\pgfqpoint{1.501745in}{0.636678in}}%
\pgfpathlineto{\pgfqpoint{1.504339in}{0.758691in}}%
\pgfpathlineto{\pgfqpoint{1.506068in}{0.727958in}}%
\pgfpathlineto{\pgfqpoint{1.506933in}{0.764002in}}%
\pgfpathlineto{\pgfqpoint{1.507798in}{0.699240in}}%
\pgfpathlineto{\pgfqpoint{1.508662in}{0.782355in}}%
\pgfpathlineto{\pgfqpoint{1.509524in}{0.764440in}}%
\pgfpathlineto{\pgfqpoint{1.510389in}{0.697482in}}%
\pgfpathlineto{\pgfqpoint{1.511254in}{0.786675in}}%
\pgfpathlineto{\pgfqpoint{1.512984in}{0.732096in}}%
\pgfpathlineto{\pgfqpoint{1.513848in}{0.779789in}}%
\pgfpathlineto{\pgfqpoint{1.514713in}{0.653342in}}%
\pgfpathlineto{\pgfqpoint{1.515579in}{0.688508in}}%
\pgfpathlineto{\pgfqpoint{1.516443in}{0.733160in}}%
\pgfpathlineto{\pgfqpoint{1.517308in}{0.701513in}}%
\pgfpathlineto{\pgfqpoint{1.518172in}{0.706532in}}%
\pgfpathlineto{\pgfqpoint{1.519904in}{0.720853in}}%
\pgfpathlineto{\pgfqpoint{1.520767in}{0.726387in}}%
\pgfpathlineto{\pgfqpoint{1.522496in}{0.689609in}}%
\pgfpathlineto{\pgfqpoint{1.524226in}{0.764554in}}%
\pgfpathlineto{\pgfqpoint{1.525091in}{0.718839in}}%
\pgfpathlineto{\pgfqpoint{1.525956in}{0.758472in}}%
\pgfpathlineto{\pgfqpoint{1.527685in}{0.715213in}}%
\pgfpathlineto{\pgfqpoint{1.529411in}{0.771038in}}%
\pgfpathlineto{\pgfqpoint{1.531141in}{0.738803in}}%
\pgfpathlineto{\pgfqpoint{1.532005in}{0.744005in}}%
\pgfpathlineto{\pgfqpoint{1.532870in}{0.778034in}}%
\pgfpathlineto{\pgfqpoint{1.533734in}{0.770783in}}%
\pgfpathlineto{\pgfqpoint{1.534599in}{0.715176in}}%
\pgfpathlineto{\pgfqpoint{1.535463in}{0.724592in}}%
\pgfpathlineto{\pgfqpoint{1.537190in}{0.696056in}}%
\pgfpathlineto{\pgfqpoint{1.538055in}{0.649866in}}%
\pgfpathlineto{\pgfqpoint{1.539786in}{0.731990in}}%
\pgfpathlineto{\pgfqpoint{1.540652in}{0.695691in}}%
\pgfpathlineto{\pgfqpoint{1.541516in}{0.748621in}}%
\pgfpathlineto{\pgfqpoint{1.543246in}{0.706313in}}%
\pgfpathlineto{\pgfqpoint{1.544975in}{0.729538in}}%
\pgfpathlineto{\pgfqpoint{1.545841in}{0.719866in}}%
\pgfpathlineto{\pgfqpoint{1.546705in}{0.685288in}}%
\pgfpathlineto{\pgfqpoint{1.547570in}{0.710893in}}%
\pgfpathlineto{\pgfqpoint{1.548435in}{0.671735in}}%
\pgfpathlineto{\pgfqpoint{1.549300in}{0.751001in}}%
\pgfpathlineto{\pgfqpoint{1.550163in}{0.738803in}}%
\pgfpathlineto{\pgfqpoint{1.551028in}{0.725104in}}%
\pgfpathlineto{\pgfqpoint{1.551893in}{0.729205in}}%
\pgfpathlineto{\pgfqpoint{1.552758in}{0.682576in}}%
\pgfpathlineto{\pgfqpoint{1.553624in}{0.693088in}}%
\pgfpathlineto{\pgfqpoint{1.554488in}{0.768801in}}%
\pgfpathlineto{\pgfqpoint{1.555353in}{0.730269in}}%
\pgfpathlineto{\pgfqpoint{1.556218in}{0.745251in}}%
\pgfpathlineto{\pgfqpoint{1.557083in}{0.677119in}}%
\pgfpathlineto{\pgfqpoint{1.557945in}{0.706313in}}%
\pgfpathlineto{\pgfqpoint{1.558810in}{0.692394in}}%
\pgfpathlineto{\pgfqpoint{1.559676in}{0.745324in}}%
\pgfpathlineto{\pgfqpoint{1.561408in}{0.660123in}}%
\pgfpathlineto{\pgfqpoint{1.562273in}{0.776459in}}%
\pgfpathlineto{\pgfqpoint{1.563139in}{0.704262in}}%
\pgfpathlineto{\pgfqpoint{1.564005in}{0.744959in}}%
\pgfpathlineto{\pgfqpoint{1.564872in}{0.654739in}}%
\pgfpathlineto{\pgfqpoint{1.565738in}{0.687302in}}%
\pgfpathlineto{\pgfqpoint{1.567467in}{0.605069in}}%
\pgfpathlineto{\pgfqpoint{1.569196in}{0.751260in}}%
\pgfpathlineto{\pgfqpoint{1.570061in}{0.715582in}}%
\pgfpathlineto{\pgfqpoint{1.570924in}{0.717121in}}%
\pgfpathlineto{\pgfqpoint{1.571789in}{0.723163in}}%
\pgfpathlineto{\pgfqpoint{1.573519in}{0.692175in}}%
\pgfpathlineto{\pgfqpoint{1.574383in}{0.751001in}}%
\pgfpathlineto{\pgfqpoint{1.576114in}{0.698363in}}%
\pgfpathlineto{\pgfqpoint{1.576978in}{0.755873in}}%
\pgfpathlineto{\pgfqpoint{1.577842in}{0.649720in}}%
\pgfpathlineto{\pgfqpoint{1.578706in}{0.665214in}}%
\pgfpathlineto{\pgfqpoint{1.580437in}{0.688585in}}%
\pgfpathlineto{\pgfqpoint{1.581303in}{0.742100in}}%
\pgfpathlineto{\pgfqpoint{1.583036in}{0.642980in}}%
\pgfpathlineto{\pgfqpoint{1.583903in}{0.704591in}}%
\pgfpathlineto{\pgfqpoint{1.584769in}{0.624042in}}%
\pgfpathlineto{\pgfqpoint{1.585634in}{0.732648in}}%
\pgfpathlineto{\pgfqpoint{1.587366in}{0.672247in}}%
\pgfpathlineto{\pgfqpoint{1.588232in}{0.772943in}}%
\pgfpathlineto{\pgfqpoint{1.589962in}{0.721222in}}%
\pgfpathlineto{\pgfqpoint{1.590828in}{0.732356in}}%
\pgfpathlineto{\pgfqpoint{1.591692in}{0.720378in}}%
\pgfpathlineto{\pgfqpoint{1.592558in}{0.663346in}}%
\pgfpathlineto{\pgfqpoint{1.595155in}{0.723163in}}%
\pgfpathlineto{\pgfqpoint{1.596021in}{0.694335in}}%
\pgfpathlineto{\pgfqpoint{1.596886in}{0.612357in}}%
\pgfpathlineto{\pgfqpoint{1.597751in}{0.751845in}}%
\pgfpathlineto{\pgfqpoint{1.598615in}{0.740233in}}%
\pgfpathlineto{\pgfqpoint{1.601207in}{0.700417in}}%
\pgfpathlineto{\pgfqpoint{1.602071in}{0.719647in}}%
\pgfpathlineto{\pgfqpoint{1.602937in}{0.703714in}}%
\pgfpathlineto{\pgfqpoint{1.603803in}{0.710747in}}%
\pgfpathlineto{\pgfqpoint{1.604669in}{0.728035in}}%
\pgfpathlineto{\pgfqpoint{1.606401in}{0.687010in}}%
\pgfpathlineto{\pgfqpoint{1.608130in}{0.716167in}}%
\pgfpathlineto{\pgfqpoint{1.608995in}{0.690745in}}%
\pgfpathlineto{\pgfqpoint{1.609860in}{0.740598in}}%
\pgfpathlineto{\pgfqpoint{1.610725in}{0.651332in}}%
\pgfpathlineto{\pgfqpoint{1.614186in}{0.758001in}}%
\pgfpathlineto{\pgfqpoint{1.615051in}{0.702175in}}%
\pgfpathlineto{\pgfqpoint{1.615917in}{0.756718in}}%
\pgfpathlineto{\pgfqpoint{1.616782in}{0.744886in}}%
\pgfpathlineto{\pgfqpoint{1.617645in}{0.751626in}}%
\pgfpathlineto{\pgfqpoint{1.619375in}{0.653935in}}%
\pgfpathlineto{\pgfqpoint{1.621104in}{0.709025in}}%
\pgfpathlineto{\pgfqpoint{1.621969in}{0.681334in}}%
\pgfpathlineto{\pgfqpoint{1.622835in}{0.792505in}}%
\pgfpathlineto{\pgfqpoint{1.624564in}{0.702431in}}%
\pgfpathlineto{\pgfqpoint{1.625428in}{0.734922in}}%
\pgfpathlineto{\pgfqpoint{1.627159in}{0.688951in}}%
\pgfpathlineto{\pgfqpoint{1.628022in}{0.718218in}}%
\pgfpathlineto{\pgfqpoint{1.628886in}{0.810598in}}%
\pgfpathlineto{\pgfqpoint{1.629752in}{0.703199in}}%
\pgfpathlineto{\pgfqpoint{1.630617in}{0.730049in}}%
\pgfpathlineto{\pgfqpoint{1.631481in}{0.778327in}}%
\pgfpathlineto{\pgfqpoint{1.632344in}{0.706313in}}%
\pgfpathlineto{\pgfqpoint{1.633210in}{0.769865in}}%
\pgfpathlineto{\pgfqpoint{1.634073in}{0.699573in}}%
\pgfpathlineto{\pgfqpoint{1.634936in}{0.700856in}}%
\pgfpathlineto{\pgfqpoint{1.636668in}{0.751476in}}%
\pgfpathlineto{\pgfqpoint{1.637533in}{0.732429in}}%
\pgfpathlineto{\pgfqpoint{1.638397in}{0.779975in}}%
\pgfpathlineto{\pgfqpoint{1.639263in}{0.726350in}}%
\pgfpathlineto{\pgfqpoint{1.640129in}{0.786277in}}%
\pgfpathlineto{\pgfqpoint{1.641860in}{0.644409in}}%
\pgfpathlineto{\pgfqpoint{1.642725in}{0.689682in}}%
\pgfpathlineto{\pgfqpoint{1.643590in}{0.751037in}}%
\pgfpathlineto{\pgfqpoint{1.644456in}{0.699426in}}%
\pgfpathlineto{\pgfqpoint{1.645321in}{0.719793in}}%
\pgfpathlineto{\pgfqpoint{1.646187in}{0.779683in}}%
\pgfpathlineto{\pgfqpoint{1.647053in}{0.772212in}}%
\pgfpathlineto{\pgfqpoint{1.647918in}{0.705468in}}%
\pgfpathlineto{\pgfqpoint{1.648784in}{0.710012in}}%
\pgfpathlineto{\pgfqpoint{1.649645in}{0.681220in}}%
\pgfpathlineto{\pgfqpoint{1.650509in}{0.730488in}}%
\pgfpathlineto{\pgfqpoint{1.651372in}{0.706240in}}%
\pgfpathlineto{\pgfqpoint{1.654830in}{0.790378in}}%
\pgfpathlineto{\pgfqpoint{1.655694in}{0.722578in}}%
\pgfpathlineto{\pgfqpoint{1.657422in}{0.783784in}}%
\pgfpathlineto{\pgfqpoint{1.660016in}{0.683713in}}%
\pgfpathlineto{\pgfqpoint{1.660879in}{0.763604in}}%
\pgfpathlineto{\pgfqpoint{1.662610in}{0.699280in}}%
\pgfpathlineto{\pgfqpoint{1.663476in}{0.724738in}}%
\pgfpathlineto{\pgfqpoint{1.664342in}{0.654665in}}%
\pgfpathlineto{\pgfqpoint{1.666070in}{0.722505in}}%
\pgfpathlineto{\pgfqpoint{1.667800in}{0.731442in}}%
\pgfpathlineto{\pgfqpoint{1.668666in}{0.685142in}}%
\pgfpathlineto{\pgfqpoint{1.669531in}{0.704372in}}%
\pgfpathlineto{\pgfqpoint{1.670393in}{0.692613in}}%
\pgfpathlineto{\pgfqpoint{1.671259in}{0.707669in}}%
\pgfpathlineto{\pgfqpoint{1.672123in}{0.695764in}}%
\pgfpathlineto{\pgfqpoint{1.672989in}{0.732136in}}%
\pgfpathlineto{\pgfqpoint{1.673852in}{0.661698in}}%
\pgfpathlineto{\pgfqpoint{1.675581in}{0.724519in}}%
\pgfpathlineto{\pgfqpoint{1.676445in}{0.739502in}}%
\pgfpathlineto{\pgfqpoint{1.677310in}{0.725985in}}%
\pgfpathlineto{\pgfqpoint{1.678175in}{0.694042in}}%
\pgfpathlineto{\pgfqpoint{1.679039in}{0.708034in}}%
\pgfpathlineto{\pgfqpoint{1.679905in}{0.689170in}}%
\pgfpathlineto{\pgfqpoint{1.681635in}{0.700490in}}%
\pgfpathlineto{\pgfqpoint{1.682499in}{0.749352in}}%
\pgfpathlineto{\pgfqpoint{1.683364in}{0.702943in}}%
\pgfpathlineto{\pgfqpoint{1.684229in}{0.812429in}}%
\pgfpathlineto{\pgfqpoint{1.685959in}{0.707742in}}%
\pgfpathlineto{\pgfqpoint{1.686825in}{0.678987in}}%
\pgfpathlineto{\pgfqpoint{1.687689in}{0.681626in}}%
\pgfpathlineto{\pgfqpoint{1.688554in}{0.678037in}}%
\pgfpathlineto{\pgfqpoint{1.689420in}{0.659172in}}%
\pgfpathlineto{\pgfqpoint{1.690285in}{0.696349in}}%
\pgfpathlineto{\pgfqpoint{1.691149in}{0.682174in}}%
\pgfpathlineto{\pgfqpoint{1.692015in}{0.637450in}}%
\pgfpathlineto{\pgfqpoint{1.693742in}{0.756316in}}%
\pgfpathlineto{\pgfqpoint{1.695472in}{0.683348in}}%
\pgfpathlineto{\pgfqpoint{1.696337in}{0.754740in}}%
\pgfpathlineto{\pgfqpoint{1.697201in}{0.730123in}}%
\pgfpathlineto{\pgfqpoint{1.698064in}{0.756864in}}%
\pgfpathlineto{\pgfqpoint{1.698929in}{0.662136in}}%
\pgfpathlineto{\pgfqpoint{1.699794in}{0.726972in}}%
\pgfpathlineto{\pgfqpoint{1.700660in}{0.653455in}}%
\pgfpathlineto{\pgfqpoint{1.701524in}{0.715286in}}%
\pgfpathlineto{\pgfqpoint{1.702389in}{0.686165in}}%
\pgfpathlineto{\pgfqpoint{1.703252in}{0.697595in}}%
\pgfpathlineto{\pgfqpoint{1.704117in}{0.630563in}}%
\pgfpathlineto{\pgfqpoint{1.705846in}{0.700563in}}%
\pgfpathlineto{\pgfqpoint{1.706709in}{0.651222in}}%
\pgfpathlineto{\pgfqpoint{1.707574in}{0.750124in}}%
\pgfpathlineto{\pgfqpoint{1.709304in}{0.664629in}}%
\pgfpathlineto{\pgfqpoint{1.712763in}{0.783089in}}%
\pgfpathlineto{\pgfqpoint{1.713627in}{0.703089in}}%
\pgfpathlineto{\pgfqpoint{1.714491in}{0.709464in}}%
\pgfpathlineto{\pgfqpoint{1.715355in}{0.695837in}}%
\pgfpathlineto{\pgfqpoint{1.716220in}{0.733639in}}%
\pgfpathlineto{\pgfqpoint{1.717949in}{0.639244in}}%
\pgfpathlineto{\pgfqpoint{1.719680in}{0.706020in}}%
\pgfpathlineto{\pgfqpoint{1.720545in}{0.714409in}}%
\pgfpathlineto{\pgfqpoint{1.721409in}{0.706971in}}%
\pgfpathlineto{\pgfqpoint{1.723138in}{0.630344in}}%
\pgfpathlineto{\pgfqpoint{1.724003in}{0.754005in}}%
\pgfpathlineto{\pgfqpoint{1.724868in}{0.683676in}}%
\pgfpathlineto{\pgfqpoint{1.725732in}{0.787779in}}%
\pgfpathlineto{\pgfqpoint{1.726596in}{0.667594in}}%
\pgfpathlineto{\pgfqpoint{1.728326in}{0.740452in}}%
\pgfpathlineto{\pgfqpoint{1.729191in}{0.678329in}}%
\pgfpathlineto{\pgfqpoint{1.730922in}{0.769353in}}%
\pgfpathlineto{\pgfqpoint{1.731786in}{0.722834in}}%
\pgfpathlineto{\pgfqpoint{1.732650in}{0.794885in}}%
\pgfpathlineto{\pgfqpoint{1.733516in}{0.735653in}}%
\pgfpathlineto{\pgfqpoint{1.734382in}{0.737520in}}%
\pgfpathlineto{\pgfqpoint{1.735246in}{0.771952in}}%
\pgfpathlineto{\pgfqpoint{1.736974in}{0.735872in}}%
\pgfpathlineto{\pgfqpoint{1.737840in}{0.764002in}}%
\pgfpathlineto{\pgfqpoint{1.739570in}{0.683088in}}%
\pgfpathlineto{\pgfqpoint{1.741299in}{0.716127in}}%
\pgfpathlineto{\pgfqpoint{1.743026in}{0.685800in}}%
\pgfpathlineto{\pgfqpoint{1.743891in}{0.725835in}}%
\pgfpathlineto{\pgfqpoint{1.744756in}{0.686019in}}%
\pgfpathlineto{\pgfqpoint{1.745620in}{0.722611in}}%
\pgfpathlineto{\pgfqpoint{1.746486in}{0.683786in}}%
\pgfpathlineto{\pgfqpoint{1.747352in}{0.722465in}}%
\pgfpathlineto{\pgfqpoint{1.748217in}{0.717519in}}%
\pgfpathlineto{\pgfqpoint{1.749082in}{0.732210in}}%
\pgfpathlineto{\pgfqpoint{1.749948in}{0.779829in}}%
\pgfpathlineto{\pgfqpoint{1.750814in}{0.704884in}}%
\pgfpathlineto{\pgfqpoint{1.751679in}{0.711843in}}%
\pgfpathlineto{\pgfqpoint{1.752545in}{0.727045in}}%
\pgfpathlineto{\pgfqpoint{1.753409in}{0.702577in}}%
\pgfpathlineto{\pgfqpoint{1.754272in}{0.737374in}}%
\pgfpathlineto{\pgfqpoint{1.756004in}{0.693677in}}%
\pgfpathlineto{\pgfqpoint{1.756870in}{0.774299in}}%
\pgfpathlineto{\pgfqpoint{1.758602in}{0.690526in}}%
\pgfpathlineto{\pgfqpoint{1.759468in}{0.692394in}}%
\pgfpathlineto{\pgfqpoint{1.761200in}{0.736936in}}%
\pgfpathlineto{\pgfqpoint{1.762066in}{0.726826in}}%
\pgfpathlineto{\pgfqpoint{1.762931in}{0.730305in}}%
\pgfpathlineto{\pgfqpoint{1.763797in}{0.713491in}}%
\pgfpathlineto{\pgfqpoint{1.764663in}{0.716350in}}%
\pgfpathlineto{\pgfqpoint{1.765529in}{0.709537in}}%
\pgfpathlineto{\pgfqpoint{1.766394in}{0.666059in}}%
\pgfpathlineto{\pgfqpoint{1.768988in}{0.766933in}}%
\pgfpathlineto{\pgfqpoint{1.770717in}{0.695066in}}%
\pgfpathlineto{\pgfqpoint{1.771582in}{0.734516in}}%
\pgfpathlineto{\pgfqpoint{1.773309in}{0.658401in}}%
\pgfpathlineto{\pgfqpoint{1.774173in}{0.747338in}}%
\pgfpathlineto{\pgfqpoint{1.775039in}{0.719939in}}%
\pgfpathlineto{\pgfqpoint{1.775903in}{0.728839in}}%
\pgfpathlineto{\pgfqpoint{1.776768in}{0.723017in}}%
\pgfpathlineto{\pgfqpoint{1.777633in}{0.707888in}}%
\pgfpathlineto{\pgfqpoint{1.778496in}{0.726460in}}%
\pgfpathlineto{\pgfqpoint{1.779358in}{0.709317in}}%
\pgfpathlineto{\pgfqpoint{1.780222in}{0.660196in}}%
\pgfpathlineto{\pgfqpoint{1.781951in}{0.721405in}}%
\pgfpathlineto{\pgfqpoint{1.782816in}{0.733127in}}%
\pgfpathlineto{\pgfqpoint{1.783681in}{0.694554in}}%
\pgfpathlineto{\pgfqpoint{1.784545in}{0.707523in}}%
\pgfpathlineto{\pgfqpoint{1.785410in}{0.660342in}}%
\pgfpathlineto{\pgfqpoint{1.788003in}{0.764262in}}%
\pgfpathlineto{\pgfqpoint{1.788868in}{0.626312in}}%
\pgfpathlineto{\pgfqpoint{1.789732in}{0.705176in}}%
\pgfpathlineto{\pgfqpoint{1.790596in}{0.703527in}}%
\pgfpathlineto{\pgfqpoint{1.792326in}{0.780268in}}%
\pgfpathlineto{\pgfqpoint{1.794054in}{0.641697in}}%
\pgfpathlineto{\pgfqpoint{1.794918in}{0.714701in}}%
\pgfpathlineto{\pgfqpoint{1.795782in}{0.685471in}}%
\pgfpathlineto{\pgfqpoint{1.796649in}{0.737082in}}%
\pgfpathlineto{\pgfqpoint{1.797512in}{0.726826in}}%
\pgfpathlineto{\pgfqpoint{1.798377in}{0.672356in}}%
\pgfpathlineto{\pgfqpoint{1.800107in}{0.738219in}}%
\pgfpathlineto{\pgfqpoint{1.800973in}{0.700746in}}%
\pgfpathlineto{\pgfqpoint{1.801837in}{0.724738in}}%
\pgfpathlineto{\pgfqpoint{1.802704in}{0.713970in}}%
\pgfpathlineto{\pgfqpoint{1.803570in}{0.679685in}}%
\pgfpathlineto{\pgfqpoint{1.806165in}{0.762979in}}%
\pgfpathlineto{\pgfqpoint{1.807031in}{0.784848in}}%
\pgfpathlineto{\pgfqpoint{1.807897in}{0.683567in}}%
\pgfpathlineto{\pgfqpoint{1.808763in}{0.776020in}}%
\pgfpathlineto{\pgfqpoint{1.810493in}{0.695874in}}%
\pgfpathlineto{\pgfqpoint{1.811358in}{0.700157in}}%
\pgfpathlineto{\pgfqpoint{1.813954in}{0.746055in}}%
\pgfpathlineto{\pgfqpoint{1.815683in}{0.700304in}}%
\pgfpathlineto{\pgfqpoint{1.816549in}{0.700523in}}%
\pgfpathlineto{\pgfqpoint{1.817413in}{0.749791in}}%
\pgfpathlineto{\pgfqpoint{1.819140in}{0.687814in}}%
\pgfpathlineto{\pgfqpoint{1.820870in}{0.744187in}}%
\pgfpathlineto{\pgfqpoint{1.821736in}{0.680562in}}%
\pgfpathlineto{\pgfqpoint{1.822603in}{0.817265in}}%
\pgfpathlineto{\pgfqpoint{1.823469in}{0.697413in}}%
\pgfpathlineto{\pgfqpoint{1.825198in}{0.781258in}}%
\pgfpathlineto{\pgfqpoint{1.826928in}{0.759170in}}%
\pgfpathlineto{\pgfqpoint{1.827794in}{0.763823in}}%
\pgfpathlineto{\pgfqpoint{1.828657in}{0.697120in}}%
\pgfpathlineto{\pgfqpoint{1.830388in}{0.765837in}}%
\pgfpathlineto{\pgfqpoint{1.831253in}{0.703089in}}%
\pgfpathlineto{\pgfqpoint{1.832979in}{0.790524in}}%
\pgfpathlineto{\pgfqpoint{1.834712in}{0.686677in}}%
\pgfpathlineto{\pgfqpoint{1.835577in}{0.741695in}}%
\pgfpathlineto{\pgfqpoint{1.837307in}{0.676494in}}%
\pgfpathlineto{\pgfqpoint{1.838171in}{0.736603in}}%
\pgfpathlineto{\pgfqpoint{1.839035in}{0.731182in}}%
\pgfpathlineto{\pgfqpoint{1.839899in}{0.755102in}}%
\pgfpathlineto{\pgfqpoint{1.840763in}{0.749937in}}%
\pgfpathlineto{\pgfqpoint{1.842494in}{0.693637in}}%
\pgfpathlineto{\pgfqpoint{1.843359in}{0.756275in}}%
\pgfpathlineto{\pgfqpoint{1.844224in}{0.753234in}}%
\pgfpathlineto{\pgfqpoint{1.845089in}{0.787008in}}%
\pgfpathlineto{\pgfqpoint{1.845955in}{0.751037in}}%
\pgfpathlineto{\pgfqpoint{1.846820in}{0.763417in}}%
\pgfpathlineto{\pgfqpoint{1.849412in}{0.708619in}}%
\pgfpathlineto{\pgfqpoint{1.850276in}{0.715286in}}%
\pgfpathlineto{\pgfqpoint{1.851141in}{0.676754in}}%
\pgfpathlineto{\pgfqpoint{1.852007in}{0.684371in}}%
\pgfpathlineto{\pgfqpoint{1.852870in}{0.725652in}}%
\pgfpathlineto{\pgfqpoint{1.853735in}{0.689170in}}%
\pgfpathlineto{\pgfqpoint{1.854601in}{0.770011in}}%
\pgfpathlineto{\pgfqpoint{1.855465in}{0.744736in}}%
\pgfpathlineto{\pgfqpoint{1.856331in}{0.780121in}}%
\pgfpathlineto{\pgfqpoint{1.857196in}{0.642687in}}%
\pgfpathlineto{\pgfqpoint{1.858062in}{0.670342in}}%
\pgfpathlineto{\pgfqpoint{1.858925in}{0.719574in}}%
\pgfpathlineto{\pgfqpoint{1.859790in}{0.691403in}}%
\pgfpathlineto{\pgfqpoint{1.861521in}{0.761184in}}%
\pgfpathlineto{\pgfqpoint{1.862388in}{0.707779in}}%
\pgfpathlineto{\pgfqpoint{1.864982in}{0.755544in}}%
\pgfpathlineto{\pgfqpoint{1.865845in}{0.765399in}}%
\pgfpathlineto{\pgfqpoint{1.867573in}{0.688549in}}%
\pgfpathlineto{\pgfqpoint{1.868440in}{0.757010in}}%
\pgfpathlineto{\pgfqpoint{1.871032in}{0.642468in}}%
\pgfpathlineto{\pgfqpoint{1.871897in}{0.742908in}}%
\pgfpathlineto{\pgfqpoint{1.872762in}{0.681845in}}%
\pgfpathlineto{\pgfqpoint{1.873626in}{0.710600in}}%
\pgfpathlineto{\pgfqpoint{1.874493in}{0.632431in}}%
\pgfpathlineto{\pgfqpoint{1.876224in}{0.680781in}}%
\pgfpathlineto{\pgfqpoint{1.877089in}{0.678512in}}%
\pgfpathlineto{\pgfqpoint{1.877953in}{0.658401in}}%
\pgfpathlineto{\pgfqpoint{1.878816in}{0.703016in}}%
\pgfpathlineto{\pgfqpoint{1.879681in}{0.676534in}}%
\pgfpathlineto{\pgfqpoint{1.882278in}{0.776459in}}%
\pgfpathlineto{\pgfqpoint{1.884008in}{0.741954in}}%
\pgfpathlineto{\pgfqpoint{1.884871in}{0.777815in}}%
\pgfpathlineto{\pgfqpoint{1.885735in}{0.774664in}}%
\pgfpathlineto{\pgfqpoint{1.886600in}{0.774518in}}%
\pgfpathlineto{\pgfqpoint{1.887463in}{0.838728in}}%
\pgfpathlineto{\pgfqpoint{1.888327in}{0.769020in}}%
\pgfpathlineto{\pgfqpoint{1.889190in}{0.777336in}}%
\pgfpathlineto{\pgfqpoint{1.892650in}{0.694335in}}%
\pgfpathlineto{\pgfqpoint{1.893515in}{0.682138in}}%
\pgfpathlineto{\pgfqpoint{1.894381in}{0.762540in}}%
\pgfpathlineto{\pgfqpoint{1.895246in}{0.659392in}}%
\pgfpathlineto{\pgfqpoint{1.896975in}{0.721368in}}%
\pgfpathlineto{\pgfqpoint{1.897841in}{0.720451in}}%
\pgfpathlineto{\pgfqpoint{1.898707in}{0.711697in}}%
\pgfpathlineto{\pgfqpoint{1.899572in}{0.790524in}}%
\pgfpathlineto{\pgfqpoint{1.901303in}{0.668438in}}%
\pgfpathlineto{\pgfqpoint{1.902169in}{0.747265in}}%
\pgfpathlineto{\pgfqpoint{1.903034in}{0.740452in}}%
\pgfpathlineto{\pgfqpoint{1.903899in}{0.632285in}}%
\pgfpathlineto{\pgfqpoint{1.905629in}{0.783199in}}%
\pgfpathlineto{\pgfqpoint{1.906494in}{0.716642in}}%
\pgfpathlineto{\pgfqpoint{1.907360in}{0.791003in}}%
\pgfpathlineto{\pgfqpoint{1.908225in}{0.690526in}}%
\pgfpathlineto{\pgfqpoint{1.909089in}{0.739721in}}%
\pgfpathlineto{\pgfqpoint{1.909953in}{0.715948in}}%
\pgfpathlineto{\pgfqpoint{1.910818in}{0.649354in}}%
\pgfpathlineto{\pgfqpoint{1.911684in}{0.783824in}}%
\pgfpathlineto{\pgfqpoint{1.912550in}{0.700344in}}%
\pgfpathlineto{\pgfqpoint{1.913415in}{0.774226in}}%
\pgfpathlineto{\pgfqpoint{1.916011in}{0.673384in}}%
\pgfpathlineto{\pgfqpoint{1.916877in}{0.737561in}}%
\pgfpathlineto{\pgfqpoint{1.917742in}{0.733127in}}%
\pgfpathlineto{\pgfqpoint{1.918607in}{0.726095in}}%
\pgfpathlineto{\pgfqpoint{1.919471in}{0.729867in}}%
\pgfpathlineto{\pgfqpoint{1.920335in}{0.676534in}}%
\pgfpathlineto{\pgfqpoint{1.921200in}{0.719574in}}%
\pgfpathlineto{\pgfqpoint{1.922066in}{0.699573in}}%
\pgfpathlineto{\pgfqpoint{1.922931in}{0.728328in}}%
\pgfpathlineto{\pgfqpoint{1.923796in}{0.701554in}}%
\pgfpathlineto{\pgfqpoint{1.924661in}{0.710783in}}%
\pgfpathlineto{\pgfqpoint{1.926393in}{0.662981in}}%
\pgfpathlineto{\pgfqpoint{1.927259in}{0.671881in}}%
\pgfpathlineto{\pgfqpoint{1.928122in}{0.732908in}}%
\pgfpathlineto{\pgfqpoint{1.928985in}{0.684594in}}%
\pgfpathlineto{\pgfqpoint{1.929850in}{0.704006in}}%
\pgfpathlineto{\pgfqpoint{1.930715in}{0.774006in}}%
\pgfpathlineto{\pgfqpoint{1.932446in}{0.701075in}}%
\pgfpathlineto{\pgfqpoint{1.933312in}{0.733931in}}%
\pgfpathlineto{\pgfqpoint{1.934179in}{0.697778in}}%
\pgfpathlineto{\pgfqpoint{1.935043in}{0.733493in}}%
\pgfpathlineto{\pgfqpoint{1.936775in}{0.697559in}}%
\pgfpathlineto{\pgfqpoint{1.938507in}{0.754992in}}%
\pgfpathlineto{\pgfqpoint{1.939373in}{0.754517in}}%
\pgfpathlineto{\pgfqpoint{1.941103in}{0.712318in}}%
\pgfpathlineto{\pgfqpoint{1.941967in}{0.727703in}}%
\pgfpathlineto{\pgfqpoint{1.943696in}{0.712428in}}%
\pgfpathlineto{\pgfqpoint{1.945425in}{0.745357in}}%
\pgfpathlineto{\pgfqpoint{1.946290in}{0.697153in}}%
\pgfpathlineto{\pgfqpoint{1.947154in}{0.734808in}}%
\pgfpathlineto{\pgfqpoint{1.948018in}{0.685211in}}%
\pgfpathlineto{\pgfqpoint{1.948883in}{0.698363in}}%
\pgfpathlineto{\pgfqpoint{1.949748in}{0.736603in}}%
\pgfpathlineto{\pgfqpoint{1.950610in}{0.728068in}}%
\pgfpathlineto{\pgfqpoint{1.951475in}{0.722684in}}%
\pgfpathlineto{\pgfqpoint{1.952341in}{0.707190in}}%
\pgfpathlineto{\pgfqpoint{1.953207in}{0.658620in}}%
\pgfpathlineto{\pgfqpoint{1.954936in}{0.720816in}}%
\pgfpathlineto{\pgfqpoint{1.955801in}{0.723529in}}%
\pgfpathlineto{\pgfqpoint{1.956666in}{0.755175in}}%
\pgfpathlineto{\pgfqpoint{1.957531in}{0.707555in}}%
\pgfpathlineto{\pgfqpoint{1.958396in}{0.784807in}}%
\pgfpathlineto{\pgfqpoint{1.961857in}{0.699865in}}%
\pgfpathlineto{\pgfqpoint{1.963588in}{0.742027in}}%
\pgfpathlineto{\pgfqpoint{1.964454in}{0.805433in}}%
\pgfpathlineto{\pgfqpoint{1.965319in}{0.735141in}}%
\pgfpathlineto{\pgfqpoint{1.966183in}{0.738146in}}%
\pgfpathlineto{\pgfqpoint{1.967048in}{0.697705in}}%
\pgfpathlineto{\pgfqpoint{1.967912in}{0.748694in}}%
\pgfpathlineto{\pgfqpoint{1.968778in}{0.694335in}}%
\pgfpathlineto{\pgfqpoint{1.969644in}{0.744443in}}%
\pgfpathlineto{\pgfqpoint{1.970509in}{0.738657in}}%
\pgfpathlineto{\pgfqpoint{1.971375in}{0.725323in}}%
\pgfpathlineto{\pgfqpoint{1.972240in}{0.766531in}}%
\pgfpathlineto{\pgfqpoint{1.973972in}{0.689755in}}%
\pgfpathlineto{\pgfqpoint{1.975702in}{0.737886in}}%
\pgfpathlineto{\pgfqpoint{1.976567in}{0.727081in}}%
\pgfpathlineto{\pgfqpoint{1.978298in}{0.699426in}}%
\pgfpathlineto{\pgfqpoint{1.980028in}{0.783491in}}%
\pgfpathlineto{\pgfqpoint{1.980894in}{0.677631in}}%
\pgfpathlineto{\pgfqpoint{1.981760in}{0.720597in}}%
\pgfpathlineto{\pgfqpoint{1.982625in}{0.706751in}}%
\pgfpathlineto{\pgfqpoint{1.983491in}{0.765066in}}%
\pgfpathlineto{\pgfqpoint{1.986949in}{0.629500in}}%
\pgfpathlineto{\pgfqpoint{1.987814in}{0.730488in}}%
\pgfpathlineto{\pgfqpoint{1.988674in}{0.716898in}}%
\pgfpathlineto{\pgfqpoint{1.989540in}{0.793821in}}%
\pgfpathlineto{\pgfqpoint{1.990405in}{0.717300in}}%
\pgfpathlineto{\pgfqpoint{1.992132in}{0.785067in}}%
\pgfpathlineto{\pgfqpoint{1.992997in}{0.756385in}}%
\pgfpathlineto{\pgfqpoint{1.993859in}{0.760193in}}%
\pgfpathlineto{\pgfqpoint{1.996456in}{0.693271in}}%
\pgfpathlineto{\pgfqpoint{1.997321in}{0.727118in}}%
\pgfpathlineto{\pgfqpoint{1.998186in}{0.686239in}}%
\pgfpathlineto{\pgfqpoint{1.999051in}{0.700669in}}%
\pgfpathlineto{\pgfqpoint{1.999915in}{0.760705in}}%
\pgfpathlineto{\pgfqpoint{2.001642in}{0.717885in}}%
\pgfpathlineto{\pgfqpoint{2.002508in}{0.766235in}}%
\pgfpathlineto{\pgfqpoint{2.003373in}{0.714295in}}%
\pgfpathlineto{\pgfqpoint{2.004238in}{0.730155in}}%
\pgfpathlineto{\pgfqpoint{2.005103in}{0.678325in}}%
\pgfpathlineto{\pgfqpoint{2.006833in}{0.728986in}}%
\pgfpathlineto{\pgfqpoint{2.007697in}{0.699719in}}%
\pgfpathlineto{\pgfqpoint{2.008563in}{0.757668in}}%
\pgfpathlineto{\pgfqpoint{2.009429in}{0.738032in}}%
\pgfpathlineto{\pgfqpoint{2.011158in}{0.758252in}}%
\pgfpathlineto{\pgfqpoint{2.012023in}{0.720962in}}%
\pgfpathlineto{\pgfqpoint{2.013752in}{0.785944in}}%
\pgfpathlineto{\pgfqpoint{2.014617in}{0.675138in}}%
\pgfpathlineto{\pgfqpoint{2.015482in}{0.704039in}}%
\pgfpathlineto{\pgfqpoint{2.016347in}{0.737740in}}%
\pgfpathlineto{\pgfqpoint{2.017211in}{0.700669in}}%
\pgfpathlineto{\pgfqpoint{2.018077in}{0.722757in}}%
\pgfpathlineto{\pgfqpoint{2.019804in}{0.633235in}}%
\pgfpathlineto{\pgfqpoint{2.020669in}{0.785652in}}%
\pgfpathlineto{\pgfqpoint{2.022400in}{0.676201in}}%
\pgfpathlineto{\pgfqpoint{2.024131in}{0.789574in}}%
\pgfpathlineto{\pgfqpoint{2.024995in}{0.756531in}}%
\pgfpathlineto{\pgfqpoint{2.026727in}{0.700267in}}%
\pgfpathlineto{\pgfqpoint{2.028455in}{0.734808in}}%
\pgfpathlineto{\pgfqpoint{2.029320in}{0.695248in}}%
\pgfpathlineto{\pgfqpoint{2.031052in}{0.766568in}}%
\pgfpathlineto{\pgfqpoint{2.031917in}{0.696203in}}%
\pgfpathlineto{\pgfqpoint{2.032782in}{0.711843in}}%
\pgfpathlineto{\pgfqpoint{2.033646in}{0.715067in}}%
\pgfpathlineto{\pgfqpoint{2.034510in}{0.730415in}}%
\pgfpathlineto{\pgfqpoint{2.035374in}{0.832979in}}%
\pgfpathlineto{\pgfqpoint{2.037104in}{0.754663in}}%
\pgfpathlineto{\pgfqpoint{2.037969in}{0.749279in}}%
\pgfpathlineto{\pgfqpoint{2.038834in}{0.658657in}}%
\pgfpathlineto{\pgfqpoint{2.039700in}{0.775030in}}%
\pgfpathlineto{\pgfqpoint{2.040565in}{0.694189in}}%
\pgfpathlineto{\pgfqpoint{2.041430in}{0.702870in}}%
\pgfpathlineto{\pgfqpoint{2.042296in}{0.704737in}}%
\pgfpathlineto{\pgfqpoint{2.043162in}{0.712468in}}%
\pgfpathlineto{\pgfqpoint{2.044026in}{0.787998in}}%
\pgfpathlineto{\pgfqpoint{2.044891in}{0.787487in}}%
\pgfpathlineto{\pgfqpoint{2.045756in}{0.806461in}}%
\pgfpathlineto{\pgfqpoint{2.048349in}{0.703933in}}%
\pgfpathlineto{\pgfqpoint{2.049214in}{0.818329in}}%
\pgfpathlineto{\pgfqpoint{2.050080in}{0.762321in}}%
\pgfpathlineto{\pgfqpoint{2.051810in}{0.803273in}}%
\pgfpathlineto{\pgfqpoint{2.052674in}{0.736022in}}%
\pgfpathlineto{\pgfqpoint{2.054402in}{0.801552in}}%
\pgfpathlineto{\pgfqpoint{2.055268in}{0.716460in}}%
\pgfpathlineto{\pgfqpoint{2.056131in}{0.729976in}}%
\pgfpathlineto{\pgfqpoint{2.056996in}{0.800488in}}%
\pgfpathlineto{\pgfqpoint{2.057861in}{0.770669in}}%
\pgfpathlineto{\pgfqpoint{2.058726in}{0.686239in}}%
\pgfpathlineto{\pgfqpoint{2.059591in}{0.760965in}}%
\pgfpathlineto{\pgfqpoint{2.060456in}{0.744480in}}%
\pgfpathlineto{\pgfqpoint{2.061322in}{0.739461in}}%
\pgfpathlineto{\pgfqpoint{2.063052in}{0.667667in}}%
\pgfpathlineto{\pgfqpoint{2.063916in}{0.684444in}}%
\pgfpathlineto{\pgfqpoint{2.064780in}{0.767595in}}%
\pgfpathlineto{\pgfqpoint{2.065646in}{0.755508in}}%
\pgfpathlineto{\pgfqpoint{2.067377in}{0.711441in}}%
\pgfpathlineto{\pgfqpoint{2.068243in}{0.762686in}}%
\pgfpathlineto{\pgfqpoint{2.070838in}{0.644482in}}%
\pgfpathlineto{\pgfqpoint{2.073429in}{0.735580in}}%
\pgfpathlineto{\pgfqpoint{2.074294in}{0.744809in}}%
\pgfpathlineto{\pgfqpoint{2.075160in}{0.724406in}}%
\pgfpathlineto{\pgfqpoint{2.076025in}{0.739023in}}%
\pgfpathlineto{\pgfqpoint{2.076890in}{0.681183in}}%
\pgfpathlineto{\pgfqpoint{2.080348in}{0.739023in}}%
\pgfpathlineto{\pgfqpoint{2.081213in}{0.754298in}}%
\pgfpathlineto{\pgfqpoint{2.082078in}{0.690745in}}%
\pgfpathlineto{\pgfqpoint{2.083807in}{0.747229in}}%
\pgfpathlineto{\pgfqpoint{2.085536in}{0.691549in}}%
\pgfpathlineto{\pgfqpoint{2.086401in}{0.752686in}}%
\pgfpathlineto{\pgfqpoint{2.087265in}{0.721880in}}%
\pgfpathlineto{\pgfqpoint{2.089859in}{0.768582in}}%
\pgfpathlineto{\pgfqpoint{2.090724in}{0.757485in}}%
\pgfpathlineto{\pgfqpoint{2.091589in}{0.697299in}}%
\pgfpathlineto{\pgfqpoint{2.094184in}{0.754590in}}%
\pgfpathlineto{\pgfqpoint{2.095914in}{0.704185in}}%
\pgfpathlineto{\pgfqpoint{2.096780in}{0.803639in}}%
\pgfpathlineto{\pgfqpoint{2.099373in}{0.724519in}}%
\pgfpathlineto{\pgfqpoint{2.100239in}{0.764335in}}%
\pgfpathlineto{\pgfqpoint{2.101101in}{0.762321in}}%
\pgfpathlineto{\pgfqpoint{2.101963in}{0.760819in}}%
\pgfpathlineto{\pgfqpoint{2.102828in}{0.716131in}}%
\pgfpathlineto{\pgfqpoint{2.103692in}{0.782761in}}%
\pgfpathlineto{\pgfqpoint{2.104558in}{0.765837in}}%
\pgfpathlineto{\pgfqpoint{2.107151in}{0.699938in}}%
\pgfpathlineto{\pgfqpoint{2.108016in}{0.718510in}}%
\pgfpathlineto{\pgfqpoint{2.108880in}{0.795250in}}%
\pgfpathlineto{\pgfqpoint{2.109746in}{0.790670in}}%
\pgfpathlineto{\pgfqpoint{2.110611in}{0.729278in}}%
\pgfpathlineto{\pgfqpoint{2.111477in}{0.749864in}}%
\pgfpathlineto{\pgfqpoint{2.112342in}{0.689426in}}%
\pgfpathlineto{\pgfqpoint{2.114071in}{0.753786in}}%
\pgfpathlineto{\pgfqpoint{2.114938in}{0.677521in}}%
\pgfpathlineto{\pgfqpoint{2.115804in}{0.755508in}}%
\pgfpathlineto{\pgfqpoint{2.116670in}{0.691111in}}%
\pgfpathlineto{\pgfqpoint{2.117537in}{0.702577in}}%
\pgfpathlineto{\pgfqpoint{2.120134in}{0.820562in}}%
\pgfpathlineto{\pgfqpoint{2.121865in}{0.775947in}}%
\pgfpathlineto{\pgfqpoint{2.122731in}{0.764189in}}%
\pgfpathlineto{\pgfqpoint{2.124463in}{0.681260in}}%
\pgfpathlineto{\pgfqpoint{2.125328in}{0.710162in}}%
\pgfpathlineto{\pgfqpoint{2.126193in}{0.743164in}}%
\pgfpathlineto{\pgfqpoint{2.127924in}{0.686060in}}%
\pgfpathlineto{\pgfqpoint{2.128790in}{0.744740in}}%
\pgfpathlineto{\pgfqpoint{2.129654in}{0.710820in}}%
\pgfpathlineto{\pgfqpoint{2.130520in}{0.721588in}}%
\pgfpathlineto{\pgfqpoint{2.131385in}{0.663529in}}%
\pgfpathlineto{\pgfqpoint{2.132251in}{0.820197in}}%
\pgfpathlineto{\pgfqpoint{2.133116in}{0.812027in}}%
\pgfpathlineto{\pgfqpoint{2.134847in}{0.769061in}}%
\pgfpathlineto{\pgfqpoint{2.135713in}{0.691517in}}%
\pgfpathlineto{\pgfqpoint{2.137440in}{0.770783in}}%
\pgfpathlineto{\pgfqpoint{2.138303in}{0.731442in}}%
\pgfpathlineto{\pgfqpoint{2.139169in}{0.798255in}}%
\pgfpathlineto{\pgfqpoint{2.140031in}{0.724885in}}%
\pgfpathlineto{\pgfqpoint{2.140897in}{0.749096in}}%
\pgfpathlineto{\pgfqpoint{2.141763in}{0.694481in}}%
\pgfpathlineto{\pgfqpoint{2.142629in}{0.742100in}}%
\pgfpathlineto{\pgfqpoint{2.143495in}{0.711002in}}%
\pgfpathlineto{\pgfqpoint{2.144360in}{0.733346in}}%
\pgfpathlineto{\pgfqpoint{2.145225in}{0.711770in}}%
\pgfpathlineto{\pgfqpoint{2.146091in}{0.742612in}}%
\pgfpathlineto{\pgfqpoint{2.146956in}{0.709829in}}%
\pgfpathlineto{\pgfqpoint{2.147822in}{0.723748in}}%
\pgfpathlineto{\pgfqpoint{2.148687in}{0.696974in}}%
\pgfpathlineto{\pgfqpoint{2.149554in}{0.747484in}}%
\pgfpathlineto{\pgfqpoint{2.150418in}{0.743822in}}%
\pgfpathlineto{\pgfqpoint{2.151284in}{0.766495in}}%
\pgfpathlineto{\pgfqpoint{2.152149in}{0.736237in}}%
\pgfpathlineto{\pgfqpoint{2.153015in}{0.650414in}}%
\pgfpathlineto{\pgfqpoint{2.153879in}{0.776897in}}%
\pgfpathlineto{\pgfqpoint{2.154743in}{0.746275in}}%
\pgfpathlineto{\pgfqpoint{2.155607in}{0.699426in}}%
\pgfpathlineto{\pgfqpoint{2.156472in}{0.709975in}}%
\pgfpathlineto{\pgfqpoint{2.157337in}{0.840965in}}%
\pgfpathlineto{\pgfqpoint{2.159066in}{0.705436in}}%
\pgfpathlineto{\pgfqpoint{2.159928in}{0.724410in}}%
\pgfpathlineto{\pgfqpoint{2.160794in}{0.698842in}}%
\pgfpathlineto{\pgfqpoint{2.161658in}{0.745584in}}%
\pgfpathlineto{\pgfqpoint{2.162520in}{0.742580in}}%
\pgfpathlineto{\pgfqpoint{2.163384in}{0.673237in}}%
\pgfpathlineto{\pgfqpoint{2.164251in}{0.830307in}}%
\pgfpathlineto{\pgfqpoint{2.165115in}{0.738073in}}%
\pgfpathlineto{\pgfqpoint{2.165980in}{0.754078in}}%
\pgfpathlineto{\pgfqpoint{2.166844in}{0.735360in}}%
\pgfpathlineto{\pgfqpoint{2.167707in}{0.673530in}}%
\pgfpathlineto{\pgfqpoint{2.169437in}{0.806059in}}%
\pgfpathlineto{\pgfqpoint{2.171166in}{0.651076in}}%
\pgfpathlineto{\pgfqpoint{2.172029in}{0.709866in}}%
\pgfpathlineto{\pgfqpoint{2.172893in}{0.731771in}}%
\pgfpathlineto{\pgfqpoint{2.174620in}{0.668073in}}%
\pgfpathlineto{\pgfqpoint{2.175484in}{0.683201in}}%
\pgfpathlineto{\pgfqpoint{2.176349in}{0.739356in}}%
\pgfpathlineto{\pgfqpoint{2.178080in}{0.691809in}}%
\pgfpathlineto{\pgfqpoint{2.178945in}{0.712614in}}%
\pgfpathlineto{\pgfqpoint{2.179810in}{0.633202in}}%
\pgfpathlineto{\pgfqpoint{2.180676in}{0.652140in}}%
\pgfpathlineto{\pgfqpoint{2.181542in}{0.731588in}}%
\pgfpathlineto{\pgfqpoint{2.182405in}{0.721661in}}%
\pgfpathlineto{\pgfqpoint{2.183270in}{0.739940in}}%
\pgfpathlineto{\pgfqpoint{2.184136in}{0.709244in}}%
\pgfpathlineto{\pgfqpoint{2.185000in}{0.739940in}}%
\pgfpathlineto{\pgfqpoint{2.185866in}{0.737963in}}%
\pgfpathlineto{\pgfqpoint{2.186731in}{0.752503in}}%
\pgfpathlineto{\pgfqpoint{2.187597in}{0.668073in}}%
\pgfpathlineto{\pgfqpoint{2.189326in}{0.859793in}}%
\pgfpathlineto{\pgfqpoint{2.191056in}{0.733237in}}%
\pgfpathlineto{\pgfqpoint{2.192788in}{0.767339in}}%
\pgfpathlineto{\pgfqpoint{2.195384in}{0.717048in}}%
\pgfpathlineto{\pgfqpoint{2.196249in}{0.744082in}}%
\pgfpathlineto{\pgfqpoint{2.197115in}{0.712103in}}%
\pgfpathlineto{\pgfqpoint{2.197982in}{0.752909in}}%
\pgfpathlineto{\pgfqpoint{2.198846in}{0.686571in}}%
\pgfpathlineto{\pgfqpoint{2.199710in}{0.755069in}}%
\pgfpathlineto{\pgfqpoint{2.200575in}{0.665876in}}%
\pgfpathlineto{\pgfqpoint{2.202305in}{0.760161in}}%
\pgfpathlineto{\pgfqpoint{2.204035in}{0.674886in}}%
\pgfpathlineto{\pgfqpoint{2.204900in}{0.740598in}}%
\pgfpathlineto{\pgfqpoint{2.205766in}{0.727743in}}%
\pgfpathlineto{\pgfqpoint{2.206632in}{0.767486in}}%
\pgfpathlineto{\pgfqpoint{2.207495in}{0.764335in}}%
\pgfpathlineto{\pgfqpoint{2.208361in}{0.753713in}}%
\pgfpathlineto{\pgfqpoint{2.209225in}{0.690599in}}%
\pgfpathlineto{\pgfqpoint{2.210090in}{0.693750in}}%
\pgfpathlineto{\pgfqpoint{2.210954in}{0.788437in}}%
\pgfpathlineto{\pgfqpoint{2.211819in}{0.748329in}}%
\pgfpathlineto{\pgfqpoint{2.212685in}{0.771919in}}%
\pgfpathlineto{\pgfqpoint{2.213551in}{0.763823in}}%
\pgfpathlineto{\pgfqpoint{2.215281in}{0.733200in}}%
\pgfpathlineto{\pgfqpoint{2.216146in}{0.743895in}}%
\pgfpathlineto{\pgfqpoint{2.217875in}{0.688732in}}%
\pgfpathlineto{\pgfqpoint{2.218741in}{0.729319in}}%
\pgfpathlineto{\pgfqpoint{2.219607in}{0.681882in}}%
\pgfpathlineto{\pgfqpoint{2.220471in}{0.755142in}}%
\pgfpathlineto{\pgfqpoint{2.221337in}{0.696755in}}%
\pgfpathlineto{\pgfqpoint{2.222202in}{0.734410in}}%
\pgfpathlineto{\pgfqpoint{2.223931in}{0.679100in}}%
\pgfpathlineto{\pgfqpoint{2.224796in}{0.703714in}}%
\pgfpathlineto{\pgfqpoint{2.225661in}{0.763750in}}%
\pgfpathlineto{\pgfqpoint{2.226526in}{0.712505in}}%
\pgfpathlineto{\pgfqpoint{2.227389in}{0.764887in}}%
\pgfpathlineto{\pgfqpoint{2.228254in}{0.731186in}}%
\pgfpathlineto{\pgfqpoint{2.229120in}{0.732835in}}%
\pgfpathlineto{\pgfqpoint{2.229986in}{0.789720in}}%
\pgfpathlineto{\pgfqpoint{2.232580in}{0.727524in}}%
\pgfpathlineto{\pgfqpoint{2.234309in}{0.774518in}}%
\pgfpathlineto{\pgfqpoint{2.235174in}{0.783126in}}%
\pgfpathlineto{\pgfqpoint{2.236039in}{0.672137in}}%
\pgfpathlineto{\pgfqpoint{2.236904in}{0.728255in}}%
\pgfpathlineto{\pgfqpoint{2.237768in}{0.717925in}}%
\pgfpathlineto{\pgfqpoint{2.238633in}{0.670379in}}%
\pgfpathlineto{\pgfqpoint{2.239497in}{0.691330in}}%
\pgfpathlineto{\pgfqpoint{2.240361in}{0.667886in}}%
\pgfpathlineto{\pgfqpoint{2.242090in}{0.752576in}}%
\pgfpathlineto{\pgfqpoint{2.242955in}{0.747192in}}%
\pgfpathlineto{\pgfqpoint{2.244685in}{0.714921in}}%
\pgfpathlineto{\pgfqpoint{2.245550in}{0.757193in}}%
\pgfpathlineto{\pgfqpoint{2.246416in}{0.718437in}}%
\pgfpathlineto{\pgfqpoint{2.247281in}{0.734297in}}%
\pgfpathlineto{\pgfqpoint{2.248147in}{0.658072in}}%
\pgfpathlineto{\pgfqpoint{2.249874in}{0.769500in}}%
\pgfpathlineto{\pgfqpoint{2.251606in}{0.657889in}}%
\pgfpathlineto{\pgfqpoint{2.253336in}{0.727962in}}%
\pgfpathlineto{\pgfqpoint{2.254201in}{0.710600in}}%
\pgfpathlineto{\pgfqpoint{2.255930in}{0.733054in}}%
\pgfpathlineto{\pgfqpoint{2.257658in}{0.667963in}}%
\pgfpathlineto{\pgfqpoint{2.258523in}{0.766203in}}%
\pgfpathlineto{\pgfqpoint{2.259389in}{0.683567in}}%
\pgfpathlineto{\pgfqpoint{2.260254in}{0.693969in}}%
\pgfpathlineto{\pgfqpoint{2.261118in}{0.803493in}}%
\pgfpathlineto{\pgfqpoint{2.262848in}{0.730821in}}%
\pgfpathlineto{\pgfqpoint{2.263712in}{0.773787in}}%
\pgfpathlineto{\pgfqpoint{2.264577in}{0.730711in}}%
\pgfpathlineto{\pgfqpoint{2.265441in}{0.750676in}}%
\pgfpathlineto{\pgfqpoint{2.266306in}{0.711591in}}%
\pgfpathlineto{\pgfqpoint{2.267171in}{0.712399in}}%
\pgfpathlineto{\pgfqpoint{2.268898in}{0.720345in}}%
\pgfpathlineto{\pgfqpoint{2.269763in}{0.677525in}}%
\pgfpathlineto{\pgfqpoint{2.271490in}{0.769792in}}%
\pgfpathlineto{\pgfqpoint{2.272353in}{0.762321in}}%
\pgfpathlineto{\pgfqpoint{2.273219in}{0.779683in}}%
\pgfpathlineto{\pgfqpoint{2.274084in}{0.770015in}}%
\pgfpathlineto{\pgfqpoint{2.274949in}{0.802689in}}%
\pgfpathlineto{\pgfqpoint{2.275812in}{0.715619in}}%
\pgfpathlineto{\pgfqpoint{2.276676in}{0.735949in}}%
\pgfpathlineto{\pgfqpoint{2.277541in}{0.743643in}}%
\pgfpathlineto{\pgfqpoint{2.280137in}{0.664191in}}%
\pgfpathlineto{\pgfqpoint{2.281003in}{0.773568in}}%
\pgfpathlineto{\pgfqpoint{2.281868in}{0.652651in}}%
\pgfpathlineto{\pgfqpoint{2.283598in}{0.714446in}}%
\pgfpathlineto{\pgfqpoint{2.284462in}{0.717998in}}%
\pgfpathlineto{\pgfqpoint{2.286192in}{0.664629in}}%
\pgfpathlineto{\pgfqpoint{2.287056in}{0.671662in}}%
\pgfpathlineto{\pgfqpoint{2.287921in}{0.670013in}}%
\pgfpathlineto{\pgfqpoint{2.290518in}{0.750233in}}%
\pgfpathlineto{\pgfqpoint{2.292249in}{0.625983in}}%
\pgfpathlineto{\pgfqpoint{2.293114in}{0.734849in}}%
\pgfpathlineto{\pgfqpoint{2.293978in}{0.702650in}}%
\pgfpathlineto{\pgfqpoint{2.294842in}{0.746607in}}%
\pgfpathlineto{\pgfqpoint{2.296571in}{0.656095in}}%
\pgfpathlineto{\pgfqpoint{2.297435in}{0.777011in}}%
\pgfpathlineto{\pgfqpoint{2.298302in}{0.676900in}}%
\pgfpathlineto{\pgfqpoint{2.299167in}{0.794227in}}%
\pgfpathlineto{\pgfqpoint{2.300033in}{0.722542in}}%
\pgfpathlineto{\pgfqpoint{2.300900in}{0.779171in}}%
\pgfpathlineto{\pgfqpoint{2.301766in}{0.766129in}}%
\pgfpathlineto{\pgfqpoint{2.302632in}{0.712834in}}%
\pgfpathlineto{\pgfqpoint{2.303497in}{0.716277in}}%
\pgfpathlineto{\pgfqpoint{2.304361in}{0.735580in}}%
\pgfpathlineto{\pgfqpoint{2.305227in}{0.697522in}}%
\pgfpathlineto{\pgfqpoint{2.306092in}{0.729684in}}%
\pgfpathlineto{\pgfqpoint{2.306957in}{0.673895in}}%
\pgfpathlineto{\pgfqpoint{2.308688in}{0.733493in}}%
\pgfpathlineto{\pgfqpoint{2.310420in}{0.632650in}}%
\pgfpathlineto{\pgfqpoint{2.311285in}{0.741370in}}%
\pgfpathlineto{\pgfqpoint{2.312152in}{0.739685in}}%
\pgfpathlineto{\pgfqpoint{2.313017in}{0.690088in}}%
\pgfpathlineto{\pgfqpoint{2.313883in}{0.798401in}}%
\pgfpathlineto{\pgfqpoint{2.315613in}{0.681553in}}%
\pgfpathlineto{\pgfqpoint{2.316478in}{0.707230in}}%
\pgfpathlineto{\pgfqpoint{2.317342in}{0.736278in}}%
\pgfpathlineto{\pgfqpoint{2.318206in}{0.707669in}}%
\pgfpathlineto{\pgfqpoint{2.319072in}{0.714738in}}%
\pgfpathlineto{\pgfqpoint{2.319937in}{0.696129in}}%
\pgfpathlineto{\pgfqpoint{2.321668in}{0.776130in}}%
\pgfpathlineto{\pgfqpoint{2.323399in}{0.712907in}}%
\pgfpathlineto{\pgfqpoint{2.324264in}{0.706276in}}%
\pgfpathlineto{\pgfqpoint{2.325993in}{0.728328in}}%
\pgfpathlineto{\pgfqpoint{2.326858in}{0.719647in}}%
\pgfpathlineto{\pgfqpoint{2.327723in}{0.697559in}}%
\pgfpathlineto{\pgfqpoint{2.328589in}{0.844482in}}%
\pgfpathlineto{\pgfqpoint{2.329452in}{0.694481in}}%
\pgfpathlineto{\pgfqpoint{2.330317in}{0.737520in}}%
\pgfpathlineto{\pgfqpoint{2.332043in}{0.662429in}}%
\pgfpathlineto{\pgfqpoint{2.332908in}{0.681439in}}%
\pgfpathlineto{\pgfqpoint{2.333775in}{0.656826in}}%
\pgfpathlineto{\pgfqpoint{2.334640in}{0.790451in}}%
\pgfpathlineto{\pgfqpoint{2.335506in}{0.719720in}}%
\pgfpathlineto{\pgfqpoint{2.336370in}{0.768769in}}%
\pgfpathlineto{\pgfqpoint{2.337235in}{0.629280in}}%
\pgfpathlineto{\pgfqpoint{2.338965in}{0.738146in}}%
\pgfpathlineto{\pgfqpoint{2.339831in}{0.675178in}}%
\pgfpathlineto{\pgfqpoint{2.340692in}{0.712176in}}%
\pgfpathlineto{\pgfqpoint{2.341556in}{0.651222in}}%
\pgfpathlineto{\pgfqpoint{2.342422in}{0.697486in}}%
\pgfpathlineto{\pgfqpoint{2.343286in}{0.632285in}}%
\pgfpathlineto{\pgfqpoint{2.345015in}{0.692394in}}%
\pgfpathlineto{\pgfqpoint{2.345881in}{0.654885in}}%
\pgfpathlineto{\pgfqpoint{2.346747in}{0.718802in}}%
\pgfpathlineto{\pgfqpoint{2.347612in}{0.666241in}}%
\pgfpathlineto{\pgfqpoint{2.348476in}{0.691225in}}%
\pgfpathlineto{\pgfqpoint{2.349341in}{0.756279in}}%
\pgfpathlineto{\pgfqpoint{2.350207in}{0.651921in}}%
\pgfpathlineto{\pgfqpoint{2.351073in}{0.746827in}}%
\pgfpathlineto{\pgfqpoint{2.353668in}{0.631993in}}%
\pgfpathlineto{\pgfqpoint{2.355397in}{0.712834in}}%
\pgfpathlineto{\pgfqpoint{2.356262in}{0.712907in}}%
\pgfpathlineto{\pgfqpoint{2.357127in}{0.649720in}}%
\pgfpathlineto{\pgfqpoint{2.357993in}{0.742100in}}%
\pgfpathlineto{\pgfqpoint{2.358859in}{0.723236in}}%
\pgfpathlineto{\pgfqpoint{2.359724in}{0.686275in}}%
\pgfpathlineto{\pgfqpoint{2.361451in}{0.805653in}}%
\pgfpathlineto{\pgfqpoint{2.362316in}{0.692723in}}%
\pgfpathlineto{\pgfqpoint{2.363181in}{0.694116in}}%
\pgfpathlineto{\pgfqpoint{2.364045in}{0.783784in}}%
\pgfpathlineto{\pgfqpoint{2.364909in}{0.697413in}}%
\pgfpathlineto{\pgfqpoint{2.365771in}{0.700636in}}%
\pgfpathlineto{\pgfqpoint{2.366638in}{0.652323in}}%
\pgfpathlineto{\pgfqpoint{2.368366in}{0.777523in}}%
\pgfpathlineto{\pgfqpoint{2.369231in}{0.691846in}}%
\pgfpathlineto{\pgfqpoint{2.370096in}{0.734045in}}%
\pgfpathlineto{\pgfqpoint{2.370961in}{0.692248in}}%
\pgfpathlineto{\pgfqpoint{2.371826in}{0.761074in}}%
\pgfpathlineto{\pgfqpoint{2.373557in}{0.726533in}}%
\pgfpathlineto{\pgfqpoint{2.375287in}{0.746754in}}%
\pgfpathlineto{\pgfqpoint{2.376151in}{0.726022in}}%
\pgfpathlineto{\pgfqpoint{2.377017in}{0.751260in}}%
\pgfpathlineto{\pgfqpoint{2.377884in}{0.688951in}}%
\pgfpathlineto{\pgfqpoint{2.378749in}{0.711697in}}%
\pgfpathlineto{\pgfqpoint{2.379616in}{0.772025in}}%
\pgfpathlineto{\pgfqpoint{2.380481in}{0.728913in}}%
\pgfpathlineto{\pgfqpoint{2.381346in}{0.747960in}}%
\pgfpathlineto{\pgfqpoint{2.382212in}{0.734297in}}%
\pgfpathlineto{\pgfqpoint{2.383076in}{0.765212in}}%
\pgfpathlineto{\pgfqpoint{2.384807in}{0.669608in}}%
\pgfpathlineto{\pgfqpoint{2.385670in}{0.773783in}}%
\pgfpathlineto{\pgfqpoint{2.386535in}{0.749425in}}%
\pgfpathlineto{\pgfqpoint{2.388265in}{0.809059in}}%
\pgfpathlineto{\pgfqpoint{2.389995in}{0.712208in}}%
\pgfpathlineto{\pgfqpoint{2.390860in}{0.724369in}}%
\pgfpathlineto{\pgfqpoint{2.391725in}{0.714076in}}%
\pgfpathlineto{\pgfqpoint{2.392590in}{0.790158in}}%
\pgfpathlineto{\pgfqpoint{2.393455in}{0.706313in}}%
\pgfpathlineto{\pgfqpoint{2.394320in}{0.725762in}}%
\pgfpathlineto{\pgfqpoint{2.395187in}{0.689389in}}%
\pgfpathlineto{\pgfqpoint{2.396051in}{0.692467in}}%
\pgfpathlineto{\pgfqpoint{2.396915in}{0.772870in}}%
\pgfpathlineto{\pgfqpoint{2.397781in}{0.709317in}}%
\pgfpathlineto{\pgfqpoint{2.398647in}{0.716496in}}%
\pgfpathlineto{\pgfqpoint{2.399511in}{0.707303in}}%
\pgfpathlineto{\pgfqpoint{2.401240in}{0.748694in}}%
\pgfpathlineto{\pgfqpoint{2.402968in}{0.735766in}}%
\pgfpathlineto{\pgfqpoint{2.403834in}{0.712395in}}%
\pgfpathlineto{\pgfqpoint{2.404699in}{0.733456in}}%
\pgfpathlineto{\pgfqpoint{2.405564in}{0.651442in}}%
\pgfpathlineto{\pgfqpoint{2.406429in}{0.752357in}}%
\pgfpathlineto{\pgfqpoint{2.408157in}{0.658182in}}%
\pgfpathlineto{\pgfqpoint{2.411618in}{0.774737in}}%
\pgfpathlineto{\pgfqpoint{2.413348in}{0.695837in}}%
\pgfpathlineto{\pgfqpoint{2.414213in}{0.719501in}}%
\pgfpathlineto{\pgfqpoint{2.415079in}{0.668584in}}%
\pgfpathlineto{\pgfqpoint{2.415943in}{0.709317in}}%
\pgfpathlineto{\pgfqpoint{2.416808in}{0.669465in}}%
\pgfpathlineto{\pgfqpoint{2.417672in}{0.749758in}}%
\pgfpathlineto{\pgfqpoint{2.418538in}{0.732835in}}%
\pgfpathlineto{\pgfqpoint{2.419402in}{0.702321in}}%
\pgfpathlineto{\pgfqpoint{2.421131in}{0.754225in}}%
\pgfpathlineto{\pgfqpoint{2.421995in}{0.701733in}}%
\pgfpathlineto{\pgfqpoint{2.422861in}{0.723529in}}%
\pgfpathlineto{\pgfqpoint{2.423725in}{0.691111in}}%
\pgfpathlineto{\pgfqpoint{2.424590in}{0.692979in}}%
\pgfpathlineto{\pgfqpoint{2.425455in}{0.737301in}}%
\pgfpathlineto{\pgfqpoint{2.426320in}{0.671918in}}%
\pgfpathlineto{\pgfqpoint{2.427184in}{0.672393in}}%
\pgfpathlineto{\pgfqpoint{2.428912in}{0.754480in}}%
\pgfpathlineto{\pgfqpoint{2.430641in}{0.692394in}}%
\pgfpathlineto{\pgfqpoint{2.431505in}{0.680562in}}%
\pgfpathlineto{\pgfqpoint{2.432370in}{0.699646in}}%
\pgfpathlineto{\pgfqpoint{2.434097in}{0.681220in}}%
\pgfpathlineto{\pgfqpoint{2.435826in}{0.777117in}}%
\pgfpathlineto{\pgfqpoint{2.436690in}{0.648729in}}%
\pgfpathlineto{\pgfqpoint{2.437555in}{0.739169in}}%
\pgfpathlineto{\pgfqpoint{2.438420in}{0.685946in}}%
\pgfpathlineto{\pgfqpoint{2.439284in}{0.731844in}}%
\pgfpathlineto{\pgfqpoint{2.441015in}{0.690672in}}%
\pgfpathlineto{\pgfqpoint{2.442744in}{0.747704in}}%
\pgfpathlineto{\pgfqpoint{2.443608in}{0.671881in}}%
\pgfpathlineto{\pgfqpoint{2.446201in}{0.787560in}}%
\pgfpathlineto{\pgfqpoint{2.447066in}{0.666059in}}%
\pgfpathlineto{\pgfqpoint{2.448795in}{0.732908in}}%
\pgfpathlineto{\pgfqpoint{2.449658in}{0.737926in}}%
\pgfpathlineto{\pgfqpoint{2.450524in}{0.618293in}}%
\pgfpathlineto{\pgfqpoint{2.451388in}{0.749060in}}%
\pgfpathlineto{\pgfqpoint{2.452253in}{0.682397in}}%
\pgfpathlineto{\pgfqpoint{2.453117in}{0.733419in}}%
\pgfpathlineto{\pgfqpoint{2.453982in}{0.724812in}}%
\pgfpathlineto{\pgfqpoint{2.454848in}{0.734958in}}%
\pgfpathlineto{\pgfqpoint{2.455713in}{0.726314in}}%
\pgfpathlineto{\pgfqpoint{2.456577in}{0.793455in}}%
\pgfpathlineto{\pgfqpoint{2.457441in}{0.724994in}}%
\pgfpathlineto{\pgfqpoint{2.458306in}{0.733273in}}%
\pgfpathlineto{\pgfqpoint{2.459170in}{0.752503in}}%
\pgfpathlineto{\pgfqpoint{2.460035in}{0.716642in}}%
\pgfpathlineto{\pgfqpoint{2.462629in}{0.783784in}}%
\pgfpathlineto{\pgfqpoint{2.463494in}{0.785944in}}%
\pgfpathlineto{\pgfqpoint{2.464360in}{0.713199in}}%
\pgfpathlineto{\pgfqpoint{2.465225in}{0.733346in}}%
\pgfpathlineto{\pgfqpoint{2.466090in}{0.737594in}}%
\pgfpathlineto{\pgfqpoint{2.466956in}{0.689097in}}%
\pgfpathlineto{\pgfqpoint{2.467821in}{0.735580in}}%
\pgfpathlineto{\pgfqpoint{2.468685in}{0.728986in}}%
\pgfpathlineto{\pgfqpoint{2.469550in}{0.747631in}}%
\pgfpathlineto{\pgfqpoint{2.471277in}{0.729278in}}%
\pgfpathlineto{\pgfqpoint{2.472140in}{0.642907in}}%
\pgfpathlineto{\pgfqpoint{2.473871in}{0.723200in}}%
\pgfpathlineto{\pgfqpoint{2.474737in}{0.695764in}}%
\pgfpathlineto{\pgfqpoint{2.475602in}{0.705801in}}%
\pgfpathlineto{\pgfqpoint{2.476468in}{0.656350in}}%
\pgfpathlineto{\pgfqpoint{2.478200in}{0.775874in}}%
\pgfpathlineto{\pgfqpoint{2.479066in}{0.734995in}}%
\pgfpathlineto{\pgfqpoint{2.479931in}{0.760965in}}%
\pgfpathlineto{\pgfqpoint{2.480797in}{0.731990in}}%
\pgfpathlineto{\pgfqpoint{2.481662in}{0.746827in}}%
\pgfpathlineto{\pgfqpoint{2.482527in}{0.707815in}}%
\pgfpathlineto{\pgfqpoint{2.483391in}{0.750306in}}%
\pgfpathlineto{\pgfqpoint{2.485119in}{0.703162in}}%
\pgfpathlineto{\pgfqpoint{2.486849in}{0.783053in}}%
\pgfpathlineto{\pgfqpoint{2.487715in}{0.774957in}}%
\pgfpathlineto{\pgfqpoint{2.489447in}{0.708400in}}%
\pgfpathlineto{\pgfqpoint{2.490310in}{0.743310in}}%
\pgfpathlineto{\pgfqpoint{2.492039in}{0.702537in}}%
\pgfpathlineto{\pgfqpoint{2.492905in}{0.727374in}}%
\pgfpathlineto{\pgfqpoint{2.493769in}{0.677119in}}%
\pgfpathlineto{\pgfqpoint{2.495501in}{0.749096in}}%
\pgfpathlineto{\pgfqpoint{2.496365in}{0.682430in}}%
\pgfpathlineto{\pgfqpoint{2.497231in}{0.769719in}}%
\pgfpathlineto{\pgfqpoint{2.498096in}{0.708692in}}%
\pgfpathlineto{\pgfqpoint{2.498962in}{0.752868in}}%
\pgfpathlineto{\pgfqpoint{2.499826in}{0.671512in}}%
\pgfpathlineto{\pgfqpoint{2.500692in}{0.715213in}}%
\pgfpathlineto{\pgfqpoint{2.501555in}{0.705176in}}%
\pgfpathlineto{\pgfqpoint{2.502421in}{0.707998in}}%
\pgfpathlineto{\pgfqpoint{2.503285in}{0.639390in}}%
\pgfpathlineto{\pgfqpoint{2.504149in}{0.747704in}}%
\pgfpathlineto{\pgfqpoint{2.505014in}{0.716825in}}%
\pgfpathlineto{\pgfqpoint{2.505880in}{0.707888in}}%
\pgfpathlineto{\pgfqpoint{2.506746in}{0.684225in}}%
\pgfpathlineto{\pgfqpoint{2.508476in}{0.751878in}}%
\pgfpathlineto{\pgfqpoint{2.509340in}{0.737886in}}%
\pgfpathlineto{\pgfqpoint{2.510205in}{0.741621in}}%
\pgfpathlineto{\pgfqpoint{2.511069in}{0.635103in}}%
\pgfpathlineto{\pgfqpoint{2.511934in}{0.710487in}}%
\pgfpathlineto{\pgfqpoint{2.512798in}{0.695358in}}%
\pgfpathlineto{\pgfqpoint{2.513664in}{0.745138in}}%
\pgfpathlineto{\pgfqpoint{2.514530in}{0.725652in}}%
\pgfpathlineto{\pgfqpoint{2.515396in}{0.746348in}}%
\pgfpathlineto{\pgfqpoint{2.516260in}{0.725981in}}%
\pgfpathlineto{\pgfqpoint{2.517125in}{0.756385in}}%
\pgfpathlineto{\pgfqpoint{2.518855in}{0.660927in}}%
\pgfpathlineto{\pgfqpoint{2.519720in}{0.737520in}}%
\pgfpathlineto{\pgfqpoint{2.520585in}{0.715140in}}%
\pgfpathlineto{\pgfqpoint{2.521451in}{0.722246in}}%
\pgfpathlineto{\pgfqpoint{2.522317in}{0.714076in}}%
\pgfpathlineto{\pgfqpoint{2.524046in}{0.762540in}}%
\pgfpathlineto{\pgfqpoint{2.524912in}{0.761403in}}%
\pgfpathlineto{\pgfqpoint{2.525777in}{0.691403in}}%
\pgfpathlineto{\pgfqpoint{2.527509in}{0.753453in}}%
\pgfpathlineto{\pgfqpoint{2.528374in}{0.696605in}}%
\pgfpathlineto{\pgfqpoint{2.529236in}{0.712208in}}%
\pgfpathlineto{\pgfqpoint{2.530102in}{0.696787in}}%
\pgfpathlineto{\pgfqpoint{2.530966in}{0.734333in}}%
\pgfpathlineto{\pgfqpoint{2.531831in}{0.680010in}}%
\pgfpathlineto{\pgfqpoint{2.533562in}{0.804772in}}%
\pgfpathlineto{\pgfqpoint{2.535290in}{0.705541in}}%
\pgfpathlineto{\pgfqpoint{2.536155in}{0.667959in}}%
\pgfpathlineto{\pgfqpoint{2.537886in}{0.759426in}}%
\pgfpathlineto{\pgfqpoint{2.540482in}{0.650378in}}%
\pgfpathlineto{\pgfqpoint{2.542211in}{0.750156in}}%
\pgfpathlineto{\pgfqpoint{2.543077in}{0.679572in}}%
\pgfpathlineto{\pgfqpoint{2.543942in}{0.773893in}}%
\pgfpathlineto{\pgfqpoint{2.544808in}{0.736164in}}%
\pgfpathlineto{\pgfqpoint{2.545672in}{0.759974in}}%
\pgfpathlineto{\pgfqpoint{2.546537in}{0.684956in}}%
\pgfpathlineto{\pgfqpoint{2.547403in}{0.725579in}}%
\pgfpathlineto{\pgfqpoint{2.548266in}{0.713565in}}%
\pgfpathlineto{\pgfqpoint{2.549131in}{0.731438in}}%
\pgfpathlineto{\pgfqpoint{2.549993in}{0.666237in}}%
\pgfpathlineto{\pgfqpoint{2.551724in}{0.733744in}}%
\pgfpathlineto{\pgfqpoint{2.552590in}{0.675576in}}%
\pgfpathlineto{\pgfqpoint{2.554322in}{0.749791in}}%
\pgfpathlineto{\pgfqpoint{2.555188in}{0.710560in}}%
\pgfpathlineto{\pgfqpoint{2.556051in}{0.749718in}}%
\pgfpathlineto{\pgfqpoint{2.556916in}{0.698217in}}%
\pgfpathlineto{\pgfqpoint{2.557782in}{0.726972in}}%
\pgfpathlineto{\pgfqpoint{2.558645in}{0.688326in}}%
\pgfpathlineto{\pgfqpoint{2.559511in}{0.742576in}}%
\pgfpathlineto{\pgfqpoint{2.560377in}{0.676786in}}%
\pgfpathlineto{\pgfqpoint{2.561241in}{0.677850in}}%
\pgfpathlineto{\pgfqpoint{2.562106in}{0.698728in}}%
\pgfpathlineto{\pgfqpoint{2.562970in}{0.764920in}}%
\pgfpathlineto{\pgfqpoint{2.563836in}{0.763417in}}%
\pgfpathlineto{\pgfqpoint{2.564700in}{0.655469in}}%
\pgfpathlineto{\pgfqpoint{2.566428in}{0.736972in}}%
\pgfpathlineto{\pgfqpoint{2.568157in}{0.782501in}}%
\pgfpathlineto{\pgfqpoint{2.569022in}{0.768472in}}%
\pgfpathlineto{\pgfqpoint{2.569888in}{0.701660in}}%
\pgfpathlineto{\pgfqpoint{2.570753in}{0.761842in}}%
\pgfpathlineto{\pgfqpoint{2.571618in}{0.701733in}}%
\pgfpathlineto{\pgfqpoint{2.573346in}{0.730780in}}%
\pgfpathlineto{\pgfqpoint{2.574211in}{0.685727in}}%
\pgfpathlineto{\pgfqpoint{2.575076in}{0.748841in}}%
\pgfpathlineto{\pgfqpoint{2.575939in}{0.721734in}}%
\pgfpathlineto{\pgfqpoint{2.578530in}{0.771952in}}%
\pgfpathlineto{\pgfqpoint{2.580257in}{0.715140in}}%
\pgfpathlineto{\pgfqpoint{2.581123in}{0.763527in}}%
\pgfpathlineto{\pgfqpoint{2.582854in}{0.641916in}}%
\pgfpathlineto{\pgfqpoint{2.583719in}{0.758033in}}%
\pgfpathlineto{\pgfqpoint{2.586313in}{0.684152in}}%
\pgfpathlineto{\pgfqpoint{2.587177in}{0.670306in}}%
\pgfpathlineto{\pgfqpoint{2.589773in}{0.703162in}}%
\pgfpathlineto{\pgfqpoint{2.590637in}{0.698399in}}%
\pgfpathlineto{\pgfqpoint{2.591500in}{0.769167in}}%
\pgfpathlineto{\pgfqpoint{2.592365in}{0.748069in}}%
\pgfpathlineto{\pgfqpoint{2.593230in}{0.673603in}}%
\pgfpathlineto{\pgfqpoint{2.594095in}{0.706532in}}%
\pgfpathlineto{\pgfqpoint{2.594961in}{0.696129in}}%
\pgfpathlineto{\pgfqpoint{2.595827in}{0.711039in}}%
\pgfpathlineto{\pgfqpoint{2.596692in}{0.747923in}}%
\pgfpathlineto{\pgfqpoint{2.598421in}{0.715652in}}%
\pgfpathlineto{\pgfqpoint{2.599286in}{0.771660in}}%
\pgfpathlineto{\pgfqpoint{2.600150in}{0.758435in}}%
\pgfpathlineto{\pgfqpoint{2.601015in}{0.755540in}}%
\pgfpathlineto{\pgfqpoint{2.601880in}{0.742978in}}%
\pgfpathlineto{\pgfqpoint{2.602742in}{0.698655in}}%
\pgfpathlineto{\pgfqpoint{2.604469in}{0.733050in}}%
\pgfpathlineto{\pgfqpoint{2.605334in}{0.713638in}}%
\pgfpathlineto{\pgfqpoint{2.606197in}{0.726566in}}%
\pgfpathlineto{\pgfqpoint{2.607063in}{0.664333in}}%
\pgfpathlineto{\pgfqpoint{2.607928in}{0.767080in}}%
\pgfpathlineto{\pgfqpoint{2.608793in}{0.745576in}}%
\pgfpathlineto{\pgfqpoint{2.609657in}{0.672502in}}%
\pgfpathlineto{\pgfqpoint{2.611387in}{0.771952in}}%
\pgfpathlineto{\pgfqpoint{2.614846in}{0.717227in}}%
\pgfpathlineto{\pgfqpoint{2.615710in}{0.714994in}}%
\pgfpathlineto{\pgfqpoint{2.616574in}{0.680306in}}%
\pgfpathlineto{\pgfqpoint{2.617439in}{0.754736in}}%
\pgfpathlineto{\pgfqpoint{2.618301in}{0.737447in}}%
\pgfpathlineto{\pgfqpoint{2.619166in}{0.680855in}}%
\pgfpathlineto{\pgfqpoint{2.620894in}{0.754663in}}%
\pgfpathlineto{\pgfqpoint{2.621760in}{0.756019in}}%
\pgfpathlineto{\pgfqpoint{2.624354in}{0.678873in}}%
\pgfpathlineto{\pgfqpoint{2.625219in}{0.688910in}}%
\pgfpathlineto{\pgfqpoint{2.626950in}{0.725981in}}%
\pgfpathlineto{\pgfqpoint{2.627816in}{0.753088in}}%
\pgfpathlineto{\pgfqpoint{2.628682in}{0.720816in}}%
\pgfpathlineto{\pgfqpoint{2.629548in}{0.734370in}}%
\pgfpathlineto{\pgfqpoint{2.630413in}{0.729643in}}%
\pgfpathlineto{\pgfqpoint{2.632145in}{0.653967in}}%
\pgfpathlineto{\pgfqpoint{2.634739in}{0.764846in}}%
\pgfpathlineto{\pgfqpoint{2.635603in}{0.695139in}}%
\pgfpathlineto{\pgfqpoint{2.636470in}{0.709975in}}%
\pgfpathlineto{\pgfqpoint{2.637336in}{0.827960in}}%
\pgfpathlineto{\pgfqpoint{2.639064in}{0.727264in}}%
\pgfpathlineto{\pgfqpoint{2.639929in}{0.722319in}}%
\pgfpathlineto{\pgfqpoint{2.641660in}{0.789574in}}%
\pgfpathlineto{\pgfqpoint{2.642524in}{0.721551in}}%
\pgfpathlineto{\pgfqpoint{2.643388in}{0.757595in}}%
\pgfpathlineto{\pgfqpoint{2.644252in}{0.673310in}}%
\pgfpathlineto{\pgfqpoint{2.645115in}{0.675397in}}%
\pgfpathlineto{\pgfqpoint{2.646844in}{0.731479in}}%
\pgfpathlineto{\pgfqpoint{2.647710in}{0.699244in}}%
\pgfpathlineto{\pgfqpoint{2.648575in}{0.748548in}}%
\pgfpathlineto{\pgfqpoint{2.649441in}{0.701294in}}%
\pgfpathlineto{\pgfqpoint{2.650307in}{0.772723in}}%
\pgfpathlineto{\pgfqpoint{2.651172in}{0.682430in}}%
\pgfpathlineto{\pgfqpoint{2.652902in}{0.749060in}}%
\pgfpathlineto{\pgfqpoint{2.653767in}{0.723748in}}%
\pgfpathlineto{\pgfqpoint{2.654633in}{0.663968in}}%
\pgfpathlineto{\pgfqpoint{2.655499in}{0.762906in}}%
\pgfpathlineto{\pgfqpoint{2.656365in}{0.732283in}}%
\pgfpathlineto{\pgfqpoint{2.657231in}{0.780085in}}%
\pgfpathlineto{\pgfqpoint{2.658097in}{0.710048in}}%
\pgfpathlineto{\pgfqpoint{2.658961in}{0.824371in}}%
\pgfpathlineto{\pgfqpoint{2.660689in}{0.687302in}}%
\pgfpathlineto{\pgfqpoint{2.661555in}{0.704116in}}%
\pgfpathlineto{\pgfqpoint{2.662417in}{0.765691in}}%
\pgfpathlineto{\pgfqpoint{2.664149in}{0.706240in}}%
\pgfpathlineto{\pgfqpoint{2.665878in}{0.746275in}}%
\pgfpathlineto{\pgfqpoint{2.667607in}{0.699938in}}%
\pgfpathlineto{\pgfqpoint{2.668473in}{0.736716in}}%
\pgfpathlineto{\pgfqpoint{2.669338in}{0.670087in}}%
\pgfpathlineto{\pgfqpoint{2.670204in}{0.762906in}}%
\pgfpathlineto{\pgfqpoint{2.671069in}{0.685398in}}%
\pgfpathlineto{\pgfqpoint{2.671934in}{0.763271in}}%
\pgfpathlineto{\pgfqpoint{2.672800in}{0.713638in}}%
\pgfpathlineto{\pgfqpoint{2.674531in}{0.775834in}}%
\pgfpathlineto{\pgfqpoint{2.676261in}{0.712537in}}%
\pgfpathlineto{\pgfqpoint{2.677125in}{0.812685in}}%
\pgfpathlineto{\pgfqpoint{2.678857in}{0.706751in}}%
\pgfpathlineto{\pgfqpoint{2.681452in}{0.808877in}}%
\pgfpathlineto{\pgfqpoint{2.682316in}{0.720378in}}%
\pgfpathlineto{\pgfqpoint{2.683182in}{0.731954in}}%
\pgfpathlineto{\pgfqpoint{2.684047in}{0.753672in}}%
\pgfpathlineto{\pgfqpoint{2.684912in}{0.732502in}}%
\pgfpathlineto{\pgfqpoint{2.685777in}{0.739132in}}%
\pgfpathlineto{\pgfqpoint{2.686642in}{0.711551in}}%
\pgfpathlineto{\pgfqpoint{2.687507in}{0.779829in}}%
\pgfpathlineto{\pgfqpoint{2.688372in}{0.721807in}}%
\pgfpathlineto{\pgfqpoint{2.689237in}{0.730780in}}%
\pgfpathlineto{\pgfqpoint{2.690102in}{0.735616in}}%
\pgfpathlineto{\pgfqpoint{2.690968in}{0.717227in}}%
\pgfpathlineto{\pgfqpoint{2.691834in}{0.810671in}}%
\pgfpathlineto{\pgfqpoint{2.694429in}{0.727483in}}%
\pgfpathlineto{\pgfqpoint{2.695292in}{0.684773in}}%
\pgfpathlineto{\pgfqpoint{2.696157in}{0.690892in}}%
\pgfpathlineto{\pgfqpoint{2.697023in}{0.739315in}}%
\pgfpathlineto{\pgfqpoint{2.697889in}{0.671037in}}%
\pgfpathlineto{\pgfqpoint{2.698752in}{0.735872in}}%
\pgfpathlineto{\pgfqpoint{2.699618in}{0.718766in}}%
\pgfpathlineto{\pgfqpoint{2.700483in}{0.686385in}}%
\pgfpathlineto{\pgfqpoint{2.701346in}{0.691842in}}%
\pgfpathlineto{\pgfqpoint{2.702209in}{0.705359in}}%
\pgfpathlineto{\pgfqpoint{2.703074in}{0.689755in}}%
\pgfpathlineto{\pgfqpoint{2.703940in}{0.739681in}}%
\pgfpathlineto{\pgfqpoint{2.704806in}{0.697413in}}%
\pgfpathlineto{\pgfqpoint{2.705670in}{0.768655in}}%
\pgfpathlineto{\pgfqpoint{2.706536in}{0.754517in}}%
\pgfpathlineto{\pgfqpoint{2.707401in}{0.726972in}}%
\pgfpathlineto{\pgfqpoint{2.709134in}{0.764518in}}%
\pgfpathlineto{\pgfqpoint{2.709995in}{0.737374in}}%
\pgfpathlineto{\pgfqpoint{2.710861in}{0.746128in}}%
\pgfpathlineto{\pgfqpoint{2.712592in}{0.687814in}}%
\pgfpathlineto{\pgfqpoint{2.713458in}{0.782355in}}%
\pgfpathlineto{\pgfqpoint{2.714324in}{0.713857in}}%
\pgfpathlineto{\pgfqpoint{2.715189in}{0.756969in}}%
\pgfpathlineto{\pgfqpoint{2.716054in}{0.738146in}}%
\pgfpathlineto{\pgfqpoint{2.716920in}{0.796826in}}%
\pgfpathlineto{\pgfqpoint{2.717785in}{0.729059in}}%
\pgfpathlineto{\pgfqpoint{2.718650in}{0.752357in}}%
\pgfpathlineto{\pgfqpoint{2.719515in}{0.662502in}}%
\pgfpathlineto{\pgfqpoint{2.722110in}{0.768363in}}%
\pgfpathlineto{\pgfqpoint{2.722975in}{0.736164in}}%
\pgfpathlineto{\pgfqpoint{2.723839in}{0.753896in}}%
\pgfpathlineto{\pgfqpoint{2.724705in}{0.713857in}}%
\pgfpathlineto{\pgfqpoint{2.726436in}{0.753672in}}%
\pgfpathlineto{\pgfqpoint{2.728166in}{0.700121in}}%
\pgfpathlineto{\pgfqpoint{2.729031in}{0.701587in}}%
\pgfpathlineto{\pgfqpoint{2.729896in}{0.778765in}}%
\pgfpathlineto{\pgfqpoint{2.730761in}{0.679060in}}%
\pgfpathlineto{\pgfqpoint{2.732492in}{0.727191in}}%
\pgfpathlineto{\pgfqpoint{2.733357in}{0.733858in}}%
\pgfpathlineto{\pgfqpoint{2.735086in}{0.794406in}}%
\pgfpathlineto{\pgfqpoint{2.735950in}{0.715359in}}%
\pgfpathlineto{\pgfqpoint{2.736813in}{0.748768in}}%
\pgfpathlineto{\pgfqpoint{2.737677in}{0.691038in}}%
\pgfpathlineto{\pgfqpoint{2.739404in}{0.782687in}}%
\pgfpathlineto{\pgfqpoint{2.740269in}{0.688183in}}%
\pgfpathlineto{\pgfqpoint{2.741132in}{0.697632in}}%
\pgfpathlineto{\pgfqpoint{2.741997in}{0.692248in}}%
\pgfpathlineto{\pgfqpoint{2.742860in}{0.707669in}}%
\pgfpathlineto{\pgfqpoint{2.743725in}{0.818215in}}%
\pgfpathlineto{\pgfqpoint{2.745455in}{0.693896in}}%
\pgfpathlineto{\pgfqpoint{2.746322in}{0.723967in}}%
\pgfpathlineto{\pgfqpoint{2.747187in}{0.717154in}}%
\pgfpathlineto{\pgfqpoint{2.748918in}{0.766568in}}%
\pgfpathlineto{\pgfqpoint{2.749784in}{0.746936in}}%
\pgfpathlineto{\pgfqpoint{2.750649in}{0.734077in}}%
\pgfpathlineto{\pgfqpoint{2.752380in}{0.766495in}}%
\pgfpathlineto{\pgfqpoint{2.753245in}{0.758179in}}%
\pgfpathlineto{\pgfqpoint{2.754976in}{0.682576in}}%
\pgfpathlineto{\pgfqpoint{2.755840in}{0.732283in}}%
\pgfpathlineto{\pgfqpoint{2.756706in}{0.696422in}}%
\pgfpathlineto{\pgfqpoint{2.758437in}{0.800853in}}%
\pgfpathlineto{\pgfqpoint{2.761898in}{0.648729in}}%
\pgfpathlineto{\pgfqpoint{2.763629in}{0.736790in}}%
\pgfpathlineto{\pgfqpoint{2.764491in}{0.693385in}}%
\pgfpathlineto{\pgfqpoint{2.765356in}{0.731479in}}%
\pgfpathlineto{\pgfqpoint{2.766220in}{0.680051in}}%
\pgfpathlineto{\pgfqpoint{2.767951in}{0.769134in}}%
\pgfpathlineto{\pgfqpoint{2.768815in}{0.769426in}}%
\pgfpathlineto{\pgfqpoint{2.769681in}{0.728109in}}%
\pgfpathlineto{\pgfqpoint{2.770546in}{0.774006in}}%
\pgfpathlineto{\pgfqpoint{2.772275in}{0.705143in}}%
\pgfpathlineto{\pgfqpoint{2.773139in}{0.684078in}}%
\pgfpathlineto{\pgfqpoint{2.774004in}{0.708546in}}%
\pgfpathlineto{\pgfqpoint{2.774866in}{0.678914in}}%
\pgfpathlineto{\pgfqpoint{2.775728in}{0.709866in}}%
\pgfpathlineto{\pgfqpoint{2.776593in}{0.700450in}}%
\pgfpathlineto{\pgfqpoint{2.777458in}{0.749645in}}%
\pgfpathlineto{\pgfqpoint{2.778324in}{0.727191in}}%
\pgfpathlineto{\pgfqpoint{2.779188in}{0.763344in}}%
\pgfpathlineto{\pgfqpoint{2.781780in}{0.717519in}}%
\pgfpathlineto{\pgfqpoint{2.782645in}{0.790962in}}%
\pgfpathlineto{\pgfqpoint{2.783507in}{0.720597in}}%
\pgfpathlineto{\pgfqpoint{2.784371in}{0.735945in}}%
\pgfpathlineto{\pgfqpoint{2.785235in}{0.701111in}}%
\pgfpathlineto{\pgfqpoint{2.786101in}{0.753932in}}%
\pgfpathlineto{\pgfqpoint{2.786965in}{0.732867in}}%
\pgfpathlineto{\pgfqpoint{2.787828in}{0.756312in}}%
\pgfpathlineto{\pgfqpoint{2.788692in}{0.741841in}}%
\pgfpathlineto{\pgfqpoint{2.789556in}{0.786017in}}%
\pgfpathlineto{\pgfqpoint{2.790421in}{0.648949in}}%
\pgfpathlineto{\pgfqpoint{2.793017in}{0.756092in}}%
\pgfpathlineto{\pgfqpoint{2.793883in}{0.701879in}}%
\pgfpathlineto{\pgfqpoint{2.794748in}{0.736311in}}%
\pgfpathlineto{\pgfqpoint{2.795613in}{0.702391in}}%
\pgfpathlineto{\pgfqpoint{2.796478in}{0.710487in}}%
\pgfpathlineto{\pgfqpoint{2.797343in}{0.779240in}}%
\pgfpathlineto{\pgfqpoint{2.798208in}{0.726200in}}%
\pgfpathlineto{\pgfqpoint{2.799072in}{0.728766in}}%
\pgfpathlineto{\pgfqpoint{2.799938in}{0.705286in}}%
\pgfpathlineto{\pgfqpoint{2.800803in}{0.718802in}}%
\pgfpathlineto{\pgfqpoint{2.802530in}{0.681585in}}%
\pgfpathlineto{\pgfqpoint{2.803395in}{0.789972in}}%
\pgfpathlineto{\pgfqpoint{2.804259in}{0.694920in}}%
\pgfpathlineto{\pgfqpoint{2.805124in}{0.743895in}}%
\pgfpathlineto{\pgfqpoint{2.805989in}{0.737740in}}%
\pgfpathlineto{\pgfqpoint{2.806856in}{0.701185in}}%
\pgfpathlineto{\pgfqpoint{2.808586in}{0.747079in}}%
\pgfpathlineto{\pgfqpoint{2.809451in}{0.683344in}}%
\pgfpathlineto{\pgfqpoint{2.810316in}{0.734370in}}%
\pgfpathlineto{\pgfqpoint{2.811181in}{0.731877in}}%
\pgfpathlineto{\pgfqpoint{2.812047in}{0.708985in}}%
\pgfpathlineto{\pgfqpoint{2.812912in}{0.710341in}}%
\pgfpathlineto{\pgfqpoint{2.813775in}{0.738142in}}%
\pgfpathlineto{\pgfqpoint{2.814641in}{0.624335in}}%
\pgfpathlineto{\pgfqpoint{2.815503in}{0.754371in}}%
\pgfpathlineto{\pgfqpoint{2.816368in}{0.751220in}}%
\pgfpathlineto{\pgfqpoint{2.818096in}{0.706167in}}%
\pgfpathlineto{\pgfqpoint{2.818961in}{0.714336in}}%
\pgfpathlineto{\pgfqpoint{2.819827in}{0.703787in}}%
\pgfpathlineto{\pgfqpoint{2.822421in}{0.765910in}}%
\pgfpathlineto{\pgfqpoint{2.823283in}{0.712208in}}%
\pgfpathlineto{\pgfqpoint{2.824146in}{0.798839in}}%
\pgfpathlineto{\pgfqpoint{2.825873in}{0.702212in}}%
\pgfpathlineto{\pgfqpoint{2.827603in}{0.724738in}}%
\pgfpathlineto{\pgfqpoint{2.828468in}{0.691773in}}%
\pgfpathlineto{\pgfqpoint{2.829334in}{0.739940in}}%
\pgfpathlineto{\pgfqpoint{2.830199in}{0.739169in}}%
\pgfpathlineto{\pgfqpoint{2.831064in}{0.672539in}}%
\pgfpathlineto{\pgfqpoint{2.831927in}{0.783857in}}%
\pgfpathlineto{\pgfqpoint{2.833654in}{0.657045in}}%
\pgfpathlineto{\pgfqpoint{2.834521in}{0.778400in}}%
\pgfpathlineto{\pgfqpoint{2.836252in}{0.708692in}}%
\pgfpathlineto{\pgfqpoint{2.837117in}{0.764002in}}%
\pgfpathlineto{\pgfqpoint{2.838847in}{0.692207in}}%
\pgfpathlineto{\pgfqpoint{2.839712in}{0.747558in}}%
\pgfpathlineto{\pgfqpoint{2.841442in}{0.702577in}}%
\pgfpathlineto{\pgfqpoint{2.842307in}{0.675215in}}%
\pgfpathlineto{\pgfqpoint{2.843172in}{0.697266in}}%
\pgfpathlineto{\pgfqpoint{2.844037in}{0.663127in}}%
\pgfpathlineto{\pgfqpoint{2.845764in}{0.738584in}}%
\pgfpathlineto{\pgfqpoint{2.847492in}{0.682247in}}%
\pgfpathlineto{\pgfqpoint{2.848356in}{0.735653in}}%
\pgfpathlineto{\pgfqpoint{2.849221in}{0.733931in}}%
\pgfpathlineto{\pgfqpoint{2.850086in}{0.717154in}}%
\pgfpathlineto{\pgfqpoint{2.850950in}{0.665287in}}%
\pgfpathlineto{\pgfqpoint{2.851815in}{0.688841in}}%
\pgfpathlineto{\pgfqpoint{2.852680in}{0.653455in}}%
\pgfpathlineto{\pgfqpoint{2.855272in}{0.716642in}}%
\pgfpathlineto{\pgfqpoint{2.856137in}{0.698217in}}%
\pgfpathlineto{\pgfqpoint{2.857003in}{0.655506in}}%
\pgfpathlineto{\pgfqpoint{2.857868in}{0.747484in}}%
\pgfpathlineto{\pgfqpoint{2.858734in}{0.683234in}}%
\pgfpathlineto{\pgfqpoint{2.859599in}{0.757156in}}%
\pgfpathlineto{\pgfqpoint{2.860465in}{0.711624in}}%
\pgfpathlineto{\pgfqpoint{2.861329in}{0.765764in}}%
\pgfpathlineto{\pgfqpoint{2.862193in}{0.705103in}}%
\pgfpathlineto{\pgfqpoint{2.863056in}{0.730488in}}%
\pgfpathlineto{\pgfqpoint{2.863921in}{0.680672in}}%
\pgfpathlineto{\pgfqpoint{2.864787in}{0.763417in}}%
\pgfpathlineto{\pgfqpoint{2.865652in}{0.744553in}}%
\pgfpathlineto{\pgfqpoint{2.866517in}{0.744261in}}%
\pgfpathlineto{\pgfqpoint{2.867382in}{0.706605in}}%
\pgfpathlineto{\pgfqpoint{2.868246in}{0.739315in}}%
\pgfpathlineto{\pgfqpoint{2.869111in}{0.732210in}}%
\pgfpathlineto{\pgfqpoint{2.870840in}{0.679827in}}%
\pgfpathlineto{\pgfqpoint{2.872568in}{0.753015in}}%
\pgfpathlineto{\pgfqpoint{2.873433in}{0.691001in}}%
\pgfpathlineto{\pgfqpoint{2.875161in}{0.760819in}}%
\pgfpathlineto{\pgfqpoint{2.876025in}{0.683051in}}%
\pgfpathlineto{\pgfqpoint{2.876888in}{0.706751in}}%
\pgfpathlineto{\pgfqpoint{2.877753in}{0.763198in}}%
\pgfpathlineto{\pgfqpoint{2.879484in}{0.687375in}}%
\pgfpathlineto{\pgfqpoint{2.881212in}{0.775103in}}%
\pgfpathlineto{\pgfqpoint{2.882941in}{0.711953in}}%
\pgfpathlineto{\pgfqpoint{2.883803in}{0.740525in}}%
\pgfpathlineto{\pgfqpoint{2.884668in}{0.735872in}}%
\pgfpathlineto{\pgfqpoint{2.885532in}{0.649903in}}%
\pgfpathlineto{\pgfqpoint{2.886397in}{0.786277in}}%
\pgfpathlineto{\pgfqpoint{2.888127in}{0.703820in}}%
\pgfpathlineto{\pgfqpoint{2.888993in}{0.687116in}}%
\pgfpathlineto{\pgfqpoint{2.889857in}{0.699605in}}%
\pgfpathlineto{\pgfqpoint{2.891587in}{0.779569in}}%
\pgfpathlineto{\pgfqpoint{2.893318in}{0.731584in}}%
\pgfpathlineto{\pgfqpoint{2.894183in}{0.735799in}}%
\pgfpathlineto{\pgfqpoint{2.895047in}{0.732356in}}%
\pgfpathlineto{\pgfqpoint{2.895913in}{0.740452in}}%
\pgfpathlineto{\pgfqpoint{2.898505in}{0.658328in}}%
\pgfpathlineto{\pgfqpoint{2.899368in}{0.756019in}}%
\pgfpathlineto{\pgfqpoint{2.900234in}{0.630783in}}%
\pgfpathlineto{\pgfqpoint{2.901099in}{0.740013in}}%
\pgfpathlineto{\pgfqpoint{2.901964in}{0.727780in}}%
\pgfpathlineto{\pgfqpoint{2.902827in}{0.685215in}}%
\pgfpathlineto{\pgfqpoint{2.905420in}{0.759828in}}%
\pgfpathlineto{\pgfqpoint{2.906286in}{0.675836in}}%
\pgfpathlineto{\pgfqpoint{2.907152in}{0.694042in}}%
\pgfpathlineto{\pgfqpoint{2.908881in}{0.756791in}}%
\pgfpathlineto{\pgfqpoint{2.909746in}{0.702650in}}%
\pgfpathlineto{\pgfqpoint{2.911476in}{0.776861in}}%
\pgfpathlineto{\pgfqpoint{2.914071in}{0.695874in}}%
\pgfpathlineto{\pgfqpoint{2.914936in}{0.712687in}}%
\pgfpathlineto{\pgfqpoint{2.915801in}{0.741881in}}%
\pgfpathlineto{\pgfqpoint{2.916667in}{0.714336in}}%
\pgfpathlineto{\pgfqpoint{2.917532in}{0.791003in}}%
\pgfpathlineto{\pgfqpoint{2.918398in}{0.698951in}}%
\pgfpathlineto{\pgfqpoint{2.919262in}{0.726022in}}%
\pgfpathlineto{\pgfqpoint{2.920993in}{0.709281in}}%
\pgfpathlineto{\pgfqpoint{2.921858in}{0.743384in}}%
\pgfpathlineto{\pgfqpoint{2.922722in}{0.696129in}}%
\pgfpathlineto{\pgfqpoint{2.923588in}{0.798766in}}%
\pgfpathlineto{\pgfqpoint{2.924453in}{0.775322in}}%
\pgfpathlineto{\pgfqpoint{2.926183in}{0.675836in}}%
\pgfpathlineto{\pgfqpoint{2.927046in}{0.792246in}}%
\pgfpathlineto{\pgfqpoint{2.927911in}{0.750562in}}%
\pgfpathlineto{\pgfqpoint{2.928776in}{0.769207in}}%
\pgfpathlineto{\pgfqpoint{2.929642in}{0.721880in}}%
\pgfpathlineto{\pgfqpoint{2.931370in}{0.798035in}}%
\pgfpathlineto{\pgfqpoint{2.932235in}{0.770709in}}%
\pgfpathlineto{\pgfqpoint{2.933966in}{0.707669in}}%
\pgfpathlineto{\pgfqpoint{2.935694in}{0.745836in}}%
\pgfpathlineto{\pgfqpoint{2.936558in}{0.754225in}}%
\pgfpathlineto{\pgfqpoint{2.937423in}{0.733383in}}%
\pgfpathlineto{\pgfqpoint{2.939153in}{0.754371in}}%
\pgfpathlineto{\pgfqpoint{2.940017in}{0.721697in}}%
\pgfpathlineto{\pgfqpoint{2.941744in}{0.775322in}}%
\pgfpathlineto{\pgfqpoint{2.942610in}{0.718766in}}%
\pgfpathlineto{\pgfqpoint{2.943475in}{0.728839in}}%
\pgfpathlineto{\pgfqpoint{2.944340in}{0.707596in}}%
\pgfpathlineto{\pgfqpoint{2.945203in}{0.778985in}}%
\pgfpathlineto{\pgfqpoint{2.946068in}{0.774551in}}%
\pgfpathlineto{\pgfqpoint{2.946935in}{0.742868in}}%
\pgfpathlineto{\pgfqpoint{2.947799in}{0.774551in}}%
\pgfpathlineto{\pgfqpoint{2.948660in}{0.692865in}}%
\pgfpathlineto{\pgfqpoint{2.950389in}{0.740192in}}%
\pgfpathlineto{\pgfqpoint{2.951255in}{0.702756in}}%
\pgfpathlineto{\pgfqpoint{2.952121in}{0.717921in}}%
\pgfpathlineto{\pgfqpoint{2.952984in}{0.708838in}}%
\pgfpathlineto{\pgfqpoint{2.953848in}{0.665945in}}%
\pgfpathlineto{\pgfqpoint{2.955575in}{0.740265in}}%
\pgfpathlineto{\pgfqpoint{2.956440in}{0.738178in}}%
\pgfpathlineto{\pgfqpoint{2.957306in}{0.687116in}}%
\pgfpathlineto{\pgfqpoint{2.958170in}{0.762719in}}%
\pgfpathlineto{\pgfqpoint{2.959036in}{0.659534in}}%
\pgfpathlineto{\pgfqpoint{2.961632in}{0.744809in}}%
\pgfpathlineto{\pgfqpoint{2.963362in}{0.695724in}}%
\pgfpathlineto{\pgfqpoint{2.965093in}{0.763856in}}%
\pgfpathlineto{\pgfqpoint{2.966821in}{0.705761in}}%
\pgfpathlineto{\pgfqpoint{2.967687in}{0.725981in}}%
\pgfpathlineto{\pgfqpoint{2.968551in}{0.719351in}}%
\pgfpathlineto{\pgfqpoint{2.969413in}{0.686604in}}%
\pgfpathlineto{\pgfqpoint{2.970278in}{0.714588in}}%
\pgfpathlineto{\pgfqpoint{2.971143in}{0.781108in}}%
\pgfpathlineto{\pgfqpoint{2.972009in}{0.760340in}}%
\pgfpathlineto{\pgfqpoint{2.972874in}{0.760851in}}%
\pgfpathlineto{\pgfqpoint{2.973738in}{0.789972in}}%
\pgfpathlineto{\pgfqpoint{2.975468in}{0.698765in}}%
\pgfpathlineto{\pgfqpoint{2.976333in}{0.751805in}}%
\pgfpathlineto{\pgfqpoint{2.977198in}{0.731730in}}%
\pgfpathlineto{\pgfqpoint{2.978062in}{0.761988in}}%
\pgfpathlineto{\pgfqpoint{2.979791in}{0.692759in}}%
\pgfpathlineto{\pgfqpoint{2.980655in}{0.696495in}}%
\pgfpathlineto{\pgfqpoint{2.982384in}{0.762353in}}%
\pgfpathlineto{\pgfqpoint{2.983249in}{0.658839in}}%
\pgfpathlineto{\pgfqpoint{2.984114in}{0.780341in}}%
\pgfpathlineto{\pgfqpoint{2.984979in}{0.689938in}}%
\pgfpathlineto{\pgfqpoint{2.985844in}{0.742978in}}%
\pgfpathlineto{\pgfqpoint{2.986710in}{0.606607in}}%
\pgfpathlineto{\pgfqpoint{2.989305in}{0.744114in}}%
\pgfpathlineto{\pgfqpoint{2.990170in}{0.744224in}}%
\pgfpathlineto{\pgfqpoint{2.991034in}{0.794479in}}%
\pgfpathlineto{\pgfqpoint{2.991898in}{0.698436in}}%
\pgfpathlineto{\pgfqpoint{2.992763in}{0.753234in}}%
\pgfpathlineto{\pgfqpoint{2.993628in}{0.747631in}}%
\pgfpathlineto{\pgfqpoint{2.994494in}{0.741296in}}%
\pgfpathlineto{\pgfqpoint{2.995360in}{0.846751in}}%
\pgfpathlineto{\pgfqpoint{2.997090in}{0.709646in}}%
\pgfpathlineto{\pgfqpoint{2.997954in}{0.775030in}}%
\pgfpathlineto{\pgfqpoint{2.998820in}{0.763052in}}%
\pgfpathlineto{\pgfqpoint{2.999686in}{0.708436in}}%
\pgfpathlineto{\pgfqpoint{3.000551in}{0.734150in}}%
\pgfpathlineto{\pgfqpoint{3.001416in}{0.716642in}}%
\pgfpathlineto{\pgfqpoint{3.002282in}{0.719939in}}%
\pgfpathlineto{\pgfqpoint{3.003146in}{0.726899in}}%
\pgfpathlineto{\pgfqpoint{3.004010in}{0.702431in}}%
\pgfpathlineto{\pgfqpoint{3.005740in}{0.766568in}}%
\pgfpathlineto{\pgfqpoint{3.008335in}{0.709975in}}%
\pgfpathlineto{\pgfqpoint{3.009201in}{0.738475in}}%
\pgfpathlineto{\pgfqpoint{3.010065in}{0.698363in}}%
\pgfpathlineto{\pgfqpoint{3.010930in}{0.732429in}}%
\pgfpathlineto{\pgfqpoint{3.011793in}{0.691330in}}%
\pgfpathlineto{\pgfqpoint{3.013525in}{0.718181in}}%
\pgfpathlineto{\pgfqpoint{3.014390in}{0.708254in}}%
\pgfpathlineto{\pgfqpoint{3.015255in}{0.752868in}}%
\pgfpathlineto{\pgfqpoint{3.016120in}{0.699938in}}%
\pgfpathlineto{\pgfqpoint{3.016985in}{0.705176in}}%
\pgfpathlineto{\pgfqpoint{3.019578in}{0.795762in}}%
\pgfpathlineto{\pgfqpoint{3.021309in}{0.701075in}}%
\pgfpathlineto{\pgfqpoint{3.023036in}{0.764664in}}%
\pgfpathlineto{\pgfqpoint{3.023900in}{0.721515in}}%
\pgfpathlineto{\pgfqpoint{3.024765in}{0.726826in}}%
\pgfpathlineto{\pgfqpoint{3.025630in}{0.729574in}}%
\pgfpathlineto{\pgfqpoint{3.026495in}{0.787852in}}%
\pgfpathlineto{\pgfqpoint{3.027361in}{0.786350in}}%
\pgfpathlineto{\pgfqpoint{3.028227in}{0.753859in}}%
\pgfpathlineto{\pgfqpoint{3.029093in}{0.776824in}}%
\pgfpathlineto{\pgfqpoint{3.029959in}{0.702504in}}%
\pgfpathlineto{\pgfqpoint{3.030824in}{0.740525in}}%
\pgfpathlineto{\pgfqpoint{3.031687in}{0.733785in}}%
\pgfpathlineto{\pgfqpoint{3.032552in}{0.726899in}}%
\pgfpathlineto{\pgfqpoint{3.033415in}{0.696349in}}%
\pgfpathlineto{\pgfqpoint{3.034281in}{0.818694in}}%
\pgfpathlineto{\pgfqpoint{3.035146in}{0.714299in}}%
\pgfpathlineto{\pgfqpoint{3.036011in}{0.853491in}}%
\pgfpathlineto{\pgfqpoint{3.038605in}{0.687814in}}%
\pgfpathlineto{\pgfqpoint{3.039470in}{0.753640in}}%
\pgfpathlineto{\pgfqpoint{3.040334in}{0.700271in}}%
\pgfpathlineto{\pgfqpoint{3.041199in}{0.777669in}}%
\pgfpathlineto{\pgfqpoint{3.042064in}{0.729538in}}%
\pgfpathlineto{\pgfqpoint{3.042930in}{0.782541in}}%
\pgfpathlineto{\pgfqpoint{3.044659in}{0.740415in}}%
\pgfpathlineto{\pgfqpoint{3.046388in}{0.727889in}}%
\pgfpathlineto{\pgfqpoint{3.047252in}{0.795214in}}%
\pgfpathlineto{\pgfqpoint{3.048983in}{0.719208in}}%
\pgfpathlineto{\pgfqpoint{3.049848in}{0.713970in}}%
\pgfpathlineto{\pgfqpoint{3.050713in}{0.796168in}}%
\pgfpathlineto{\pgfqpoint{3.051578in}{0.781295in}}%
\pgfpathlineto{\pgfqpoint{3.052443in}{0.726022in}}%
\pgfpathlineto{\pgfqpoint{3.053308in}{0.777303in}}%
\pgfpathlineto{\pgfqpoint{3.055037in}{0.686644in}}%
\pgfpathlineto{\pgfqpoint{3.055901in}{0.746827in}}%
\pgfpathlineto{\pgfqpoint{3.056766in}{0.612470in}}%
\pgfpathlineto{\pgfqpoint{3.058495in}{0.827229in}}%
\pgfpathlineto{\pgfqpoint{3.059359in}{0.681845in}}%
\pgfpathlineto{\pgfqpoint{3.060224in}{0.810817in}}%
\pgfpathlineto{\pgfqpoint{3.061088in}{0.732502in}}%
\pgfpathlineto{\pgfqpoint{3.061953in}{0.744041in}}%
\pgfpathlineto{\pgfqpoint{3.062818in}{0.760453in}}%
\pgfpathlineto{\pgfqpoint{3.063682in}{0.740525in}}%
\pgfpathlineto{\pgfqpoint{3.064545in}{0.743603in}}%
\pgfpathlineto{\pgfqpoint{3.065410in}{0.728255in}}%
\pgfpathlineto{\pgfqpoint{3.066275in}{0.678366in}}%
\pgfpathlineto{\pgfqpoint{3.067139in}{0.809940in}}%
\pgfpathlineto{\pgfqpoint{3.068870in}{0.702797in}}%
\pgfpathlineto{\pgfqpoint{3.069735in}{0.733054in}}%
\pgfpathlineto{\pgfqpoint{3.070600in}{0.717816in}}%
\pgfpathlineto{\pgfqpoint{3.072330in}{0.745397in}}%
\pgfpathlineto{\pgfqpoint{3.073195in}{0.712103in}}%
\pgfpathlineto{\pgfqpoint{3.074926in}{0.733566in}}%
\pgfpathlineto{\pgfqpoint{3.075790in}{0.709866in}}%
\pgfpathlineto{\pgfqpoint{3.076655in}{0.723821in}}%
\pgfpathlineto{\pgfqpoint{3.077520in}{0.708473in}}%
\pgfpathlineto{\pgfqpoint{3.078385in}{0.663200in}}%
\pgfpathlineto{\pgfqpoint{3.079251in}{0.664849in}}%
\pgfpathlineto{\pgfqpoint{3.080982in}{0.740452in}}%
\pgfpathlineto{\pgfqpoint{3.081847in}{0.815763in}}%
\pgfpathlineto{\pgfqpoint{3.082712in}{0.750416in}}%
\pgfpathlineto{\pgfqpoint{3.083578in}{0.754590in}}%
\pgfpathlineto{\pgfqpoint{3.084443in}{0.780487in}}%
\pgfpathlineto{\pgfqpoint{3.086170in}{0.735872in}}%
\pgfpathlineto{\pgfqpoint{3.087036in}{0.812174in}}%
\pgfpathlineto{\pgfqpoint{3.090498in}{0.680051in}}%
\pgfpathlineto{\pgfqpoint{3.093092in}{0.750051in}}%
\pgfpathlineto{\pgfqpoint{3.094822in}{0.685142in}}%
\pgfpathlineto{\pgfqpoint{3.095687in}{0.693677in}}%
\pgfpathlineto{\pgfqpoint{3.096552in}{0.734995in}}%
\pgfpathlineto{\pgfqpoint{3.097417in}{0.725177in}}%
\pgfpathlineto{\pgfqpoint{3.099147in}{0.756641in}}%
\pgfpathlineto{\pgfqpoint{3.100012in}{0.682722in}}%
\pgfpathlineto{\pgfqpoint{3.100878in}{0.725177in}}%
\pgfpathlineto{\pgfqpoint{3.101742in}{0.685435in}}%
\pgfpathlineto{\pgfqpoint{3.103470in}{0.738292in}}%
\pgfpathlineto{\pgfqpoint{3.104334in}{0.731588in}}%
\pgfpathlineto{\pgfqpoint{3.105199in}{0.695764in}}%
\pgfpathlineto{\pgfqpoint{3.106065in}{0.713272in}}%
\pgfpathlineto{\pgfqpoint{3.106930in}{0.694554in}}%
\pgfpathlineto{\pgfqpoint{3.107795in}{0.759682in}}%
\pgfpathlineto{\pgfqpoint{3.108661in}{0.742100in}}%
\pgfpathlineto{\pgfqpoint{3.109526in}{0.693823in}}%
\pgfpathlineto{\pgfqpoint{3.110391in}{0.705801in}}%
\pgfpathlineto{\pgfqpoint{3.111255in}{0.688878in}}%
\pgfpathlineto{\pgfqpoint{3.112119in}{0.751512in}}%
\pgfpathlineto{\pgfqpoint{3.112985in}{0.665068in}}%
\pgfpathlineto{\pgfqpoint{3.113850in}{0.777121in}}%
\pgfpathlineto{\pgfqpoint{3.114714in}{0.748914in}}%
\pgfpathlineto{\pgfqpoint{3.115579in}{0.749206in}}%
\pgfpathlineto{\pgfqpoint{3.117306in}{0.717925in}}%
\pgfpathlineto{\pgfqpoint{3.118172in}{0.723711in}}%
\pgfpathlineto{\pgfqpoint{3.119038in}{0.707961in}}%
\pgfpathlineto{\pgfqpoint{3.119902in}{0.749206in}}%
\pgfpathlineto{\pgfqpoint{3.120768in}{0.720122in}}%
\pgfpathlineto{\pgfqpoint{3.122496in}{0.767559in}}%
\pgfpathlineto{\pgfqpoint{3.123362in}{0.732173in}}%
\pgfpathlineto{\pgfqpoint{3.124228in}{0.744918in}}%
\pgfpathlineto{\pgfqpoint{3.125956in}{0.698290in}}%
\pgfpathlineto{\pgfqpoint{3.126821in}{0.730853in}}%
\pgfpathlineto{\pgfqpoint{3.127686in}{0.658507in}}%
\pgfpathlineto{\pgfqpoint{3.128550in}{0.684184in}}%
\pgfpathlineto{\pgfqpoint{3.129414in}{0.751512in}}%
\pgfpathlineto{\pgfqpoint{3.130279in}{0.678179in}}%
\pgfpathlineto{\pgfqpoint{3.132007in}{0.750229in}}%
\pgfpathlineto{\pgfqpoint{3.133740in}{0.713491in}}%
\pgfpathlineto{\pgfqpoint{3.134605in}{0.763271in}}%
\pgfpathlineto{\pgfqpoint{3.135471in}{0.698582in}}%
\pgfpathlineto{\pgfqpoint{3.136337in}{0.731292in}}%
\pgfpathlineto{\pgfqpoint{3.137201in}{0.717081in}}%
\pgfpathlineto{\pgfqpoint{3.138930in}{0.758983in}}%
\pgfpathlineto{\pgfqpoint{3.140660in}{0.678288in}}%
\pgfpathlineto{\pgfqpoint{3.141527in}{0.714076in}}%
\pgfpathlineto{\pgfqpoint{3.142392in}{0.676348in}}%
\pgfpathlineto{\pgfqpoint{3.143259in}{0.751001in}}%
\pgfpathlineto{\pgfqpoint{3.144124in}{0.746348in}}%
\pgfpathlineto{\pgfqpoint{3.144990in}{0.696276in}}%
\pgfpathlineto{\pgfqpoint{3.145857in}{0.770450in}}%
\pgfpathlineto{\pgfqpoint{3.147589in}{0.718949in}}%
\pgfpathlineto{\pgfqpoint{3.148455in}{0.724040in}}%
\pgfpathlineto{\pgfqpoint{3.149320in}{0.720195in}}%
\pgfpathlineto{\pgfqpoint{3.151050in}{0.758033in}}%
\pgfpathlineto{\pgfqpoint{3.151916in}{0.716935in}}%
\pgfpathlineto{\pgfqpoint{3.152782in}{0.761549in}}%
\pgfpathlineto{\pgfqpoint{3.155373in}{0.632650in}}%
\pgfpathlineto{\pgfqpoint{3.156236in}{0.735507in}}%
\pgfpathlineto{\pgfqpoint{3.157101in}{0.723821in}}%
\pgfpathlineto{\pgfqpoint{3.158829in}{0.714263in}}%
\pgfpathlineto{\pgfqpoint{3.159695in}{0.737155in}}%
\pgfpathlineto{\pgfqpoint{3.160561in}{0.685873in}}%
\pgfpathlineto{\pgfqpoint{3.161425in}{0.692540in}}%
\pgfpathlineto{\pgfqpoint{3.162292in}{0.696056in}}%
\pgfpathlineto{\pgfqpoint{3.163156in}{0.712761in}}%
\pgfpathlineto{\pgfqpoint{3.164021in}{0.808146in}}%
\pgfpathlineto{\pgfqpoint{3.165751in}{0.718437in}}%
\pgfpathlineto{\pgfqpoint{3.166615in}{0.767559in}}%
\pgfpathlineto{\pgfqpoint{3.167480in}{0.672320in}}%
\pgfpathlineto{\pgfqpoint{3.168345in}{0.734004in}}%
\pgfpathlineto{\pgfqpoint{3.169210in}{0.732136in}}%
\pgfpathlineto{\pgfqpoint{3.170075in}{0.722246in}}%
\pgfpathlineto{\pgfqpoint{3.170938in}{0.686202in}}%
\pgfpathlineto{\pgfqpoint{3.171802in}{0.743968in}}%
\pgfpathlineto{\pgfqpoint{3.173530in}{0.650085in}}%
\pgfpathlineto{\pgfqpoint{3.174393in}{0.762467in}}%
\pgfpathlineto{\pgfqpoint{3.175258in}{0.686352in}}%
\pgfpathlineto{\pgfqpoint{3.176124in}{0.712103in}}%
\pgfpathlineto{\pgfqpoint{3.176989in}{0.695399in}}%
\pgfpathlineto{\pgfqpoint{3.177853in}{0.707523in}}%
\pgfpathlineto{\pgfqpoint{3.178719in}{0.664922in}}%
\pgfpathlineto{\pgfqpoint{3.180447in}{0.750891in}}%
\pgfpathlineto{\pgfqpoint{3.181311in}{0.711989in}}%
\pgfpathlineto{\pgfqpoint{3.183039in}{0.768509in}}%
\pgfpathlineto{\pgfqpoint{3.183904in}{0.646569in}}%
\pgfpathlineto{\pgfqpoint{3.184769in}{0.655835in}}%
\pgfpathlineto{\pgfqpoint{3.185635in}{0.720232in}}%
\pgfpathlineto{\pgfqpoint{3.186500in}{0.711331in}}%
\pgfpathlineto{\pgfqpoint{3.188230in}{0.715838in}}%
\pgfpathlineto{\pgfqpoint{3.189095in}{0.682138in}}%
\pgfpathlineto{\pgfqpoint{3.189960in}{0.697815in}}%
\pgfpathlineto{\pgfqpoint{3.190825in}{0.769832in}}%
\pgfpathlineto{\pgfqpoint{3.192553in}{0.722871in}}%
\pgfpathlineto{\pgfqpoint{3.193417in}{0.743968in}}%
\pgfpathlineto{\pgfqpoint{3.194278in}{0.743603in}}%
\pgfpathlineto{\pgfqpoint{3.195143in}{0.733712in}}%
\pgfpathlineto{\pgfqpoint{3.196006in}{0.778254in}}%
\pgfpathlineto{\pgfqpoint{3.196869in}{0.728511in}}%
\pgfpathlineto{\pgfqpoint{3.197734in}{0.735507in}}%
\pgfpathlineto{\pgfqpoint{3.198598in}{0.755873in}}%
\pgfpathlineto{\pgfqpoint{3.199463in}{0.644701in}}%
\pgfpathlineto{\pgfqpoint{3.200327in}{0.658035in}}%
\pgfpathlineto{\pgfqpoint{3.201193in}{0.714628in}}%
\pgfpathlineto{\pgfqpoint{3.202922in}{0.685581in}}%
\pgfpathlineto{\pgfqpoint{3.205515in}{0.784555in}}%
\pgfpathlineto{\pgfqpoint{3.206381in}{0.685361in}}%
\pgfpathlineto{\pgfqpoint{3.207245in}{0.758472in}}%
\pgfpathlineto{\pgfqpoint{3.208975in}{0.718802in}}%
\pgfpathlineto{\pgfqpoint{3.209840in}{0.762906in}}%
\pgfpathlineto{\pgfqpoint{3.210707in}{0.706093in}}%
\pgfpathlineto{\pgfqpoint{3.211573in}{0.757960in}}%
\pgfpathlineto{\pgfqpoint{3.212439in}{0.757595in}}%
\pgfpathlineto{\pgfqpoint{3.213304in}{0.758106in}}%
\pgfpathlineto{\pgfqpoint{3.214169in}{0.788071in}}%
\pgfpathlineto{\pgfqpoint{3.215896in}{0.721734in}}%
\pgfpathlineto{\pgfqpoint{3.216762in}{0.742320in}}%
\pgfpathlineto{\pgfqpoint{3.217627in}{0.734849in}}%
\pgfpathlineto{\pgfqpoint{3.218492in}{0.694079in}}%
\pgfpathlineto{\pgfqpoint{3.220222in}{0.742174in}}%
\pgfpathlineto{\pgfqpoint{3.221087in}{0.750928in}}%
\pgfpathlineto{\pgfqpoint{3.222816in}{0.716131in}}%
\pgfpathlineto{\pgfqpoint{3.224547in}{0.774080in}}%
\pgfpathlineto{\pgfqpoint{3.225412in}{0.685508in}}%
\pgfpathlineto{\pgfqpoint{3.226276in}{0.686239in}}%
\pgfpathlineto{\pgfqpoint{3.227141in}{0.760745in}}%
\pgfpathlineto{\pgfqpoint{3.228006in}{0.745361in}}%
\pgfpathlineto{\pgfqpoint{3.228872in}{0.702723in}}%
\pgfpathlineto{\pgfqpoint{3.229739in}{0.794665in}}%
\pgfpathlineto{\pgfqpoint{3.230605in}{0.742320in}}%
\pgfpathlineto{\pgfqpoint{3.231470in}{0.787081in}}%
\pgfpathlineto{\pgfqpoint{3.232335in}{0.677594in}}%
\pgfpathlineto{\pgfqpoint{3.233200in}{0.690892in}}%
\pgfpathlineto{\pgfqpoint{3.234064in}{0.687887in}}%
\pgfpathlineto{\pgfqpoint{3.235794in}{0.796460in}}%
\pgfpathlineto{\pgfqpoint{3.237525in}{0.702833in}}%
\pgfpathlineto{\pgfqpoint{3.239254in}{0.740379in}}%
\pgfpathlineto{\pgfqpoint{3.242713in}{0.594410in}}%
\pgfpathlineto{\pgfqpoint{3.243578in}{0.725396in}}%
\pgfpathlineto{\pgfqpoint{3.244443in}{0.713345in}}%
\pgfpathlineto{\pgfqpoint{3.245308in}{0.691184in}}%
\pgfpathlineto{\pgfqpoint{3.246173in}{0.750197in}}%
\pgfpathlineto{\pgfqpoint{3.247039in}{0.653200in}}%
\pgfpathlineto{\pgfqpoint{3.247901in}{0.758366in}}%
\pgfpathlineto{\pgfqpoint{3.248767in}{0.711185in}}%
\pgfpathlineto{\pgfqpoint{3.250497in}{0.743895in}}%
\pgfpathlineto{\pgfqpoint{3.251361in}{0.714628in}}%
\pgfpathlineto{\pgfqpoint{3.252226in}{0.646277in}}%
\pgfpathlineto{\pgfqpoint{3.253089in}{0.803785in}}%
\pgfpathlineto{\pgfqpoint{3.253954in}{0.686681in}}%
\pgfpathlineto{\pgfqpoint{3.254819in}{0.703641in}}%
\pgfpathlineto{\pgfqpoint{3.255685in}{0.635655in}}%
\pgfpathlineto{\pgfqpoint{3.256550in}{0.726277in}}%
\pgfpathlineto{\pgfqpoint{3.257415in}{0.691078in}}%
\pgfpathlineto{\pgfqpoint{3.258280in}{0.732981in}}%
\pgfpathlineto{\pgfqpoint{3.259145in}{0.717852in}}%
\pgfpathlineto{\pgfqpoint{3.260010in}{0.746461in}}%
\pgfpathlineto{\pgfqpoint{3.260876in}{0.689207in}}%
\pgfpathlineto{\pgfqpoint{3.262605in}{0.721734in}}%
\pgfpathlineto{\pgfqpoint{3.263469in}{0.692613in}}%
\pgfpathlineto{\pgfqpoint{3.264334in}{0.772577in}}%
\pgfpathlineto{\pgfqpoint{3.266929in}{0.679133in}}%
\pgfpathlineto{\pgfqpoint{3.267794in}{0.784263in}}%
\pgfpathlineto{\pgfqpoint{3.268657in}{0.713678in}}%
\pgfpathlineto{\pgfqpoint{3.270385in}{0.745361in}}%
\pgfpathlineto{\pgfqpoint{3.272980in}{0.695070in}}%
\pgfpathlineto{\pgfqpoint{3.273846in}{0.794592in}}%
\pgfpathlineto{\pgfqpoint{3.275573in}{0.709317in}}%
\pgfpathlineto{\pgfqpoint{3.276438in}{0.769280in}}%
\pgfpathlineto{\pgfqpoint{3.277300in}{0.764042in}}%
\pgfpathlineto{\pgfqpoint{3.278164in}{0.707815in}}%
\pgfpathlineto{\pgfqpoint{3.279029in}{0.778765in}}%
\pgfpathlineto{\pgfqpoint{3.280759in}{0.668950in}}%
\pgfpathlineto{\pgfqpoint{3.281623in}{0.760819in}}%
\pgfpathlineto{\pgfqpoint{3.282489in}{0.637486in}}%
\pgfpathlineto{\pgfqpoint{3.284218in}{0.734556in}}%
\pgfpathlineto{\pgfqpoint{3.285084in}{0.722249in}}%
\pgfpathlineto{\pgfqpoint{3.286815in}{0.765910in}}%
\pgfpathlineto{\pgfqpoint{3.287681in}{0.700490in}}%
\pgfpathlineto{\pgfqpoint{3.288546in}{0.768988in}}%
\pgfpathlineto{\pgfqpoint{3.289412in}{0.754777in}}%
\pgfpathlineto{\pgfqpoint{3.290277in}{0.711039in}}%
\pgfpathlineto{\pgfqpoint{3.291141in}{0.750855in}}%
\pgfpathlineto{\pgfqpoint{3.292872in}{0.701440in}}%
\pgfpathlineto{\pgfqpoint{3.293738in}{0.676827in}}%
\pgfpathlineto{\pgfqpoint{3.294599in}{0.730013in}}%
\pgfpathlineto{\pgfqpoint{3.295464in}{0.692394in}}%
\pgfpathlineto{\pgfqpoint{3.296325in}{0.819498in}}%
\pgfpathlineto{\pgfqpoint{3.298058in}{0.687595in}}%
\pgfpathlineto{\pgfqpoint{3.298922in}{0.666643in}}%
\pgfpathlineto{\pgfqpoint{3.300649in}{0.746534in}}%
\pgfpathlineto{\pgfqpoint{3.302379in}{0.691809in}}%
\pgfpathlineto{\pgfqpoint{3.303244in}{0.711185in}}%
\pgfpathlineto{\pgfqpoint{3.304108in}{0.665287in}}%
\pgfpathlineto{\pgfqpoint{3.304975in}{0.786569in}}%
\pgfpathlineto{\pgfqpoint{3.305841in}{0.779537in}}%
\pgfpathlineto{\pgfqpoint{3.306706in}{0.656241in}}%
\pgfpathlineto{\pgfqpoint{3.307572in}{0.701627in}}%
\pgfpathlineto{\pgfqpoint{3.308436in}{0.676096in}}%
\pgfpathlineto{\pgfqpoint{3.309302in}{0.717267in}}%
\pgfpathlineto{\pgfqpoint{3.310168in}{0.672433in}}%
\pgfpathlineto{\pgfqpoint{3.313628in}{0.786354in}}%
\pgfpathlineto{\pgfqpoint{3.314493in}{0.671150in}}%
\pgfpathlineto{\pgfqpoint{3.317088in}{0.832540in}}%
\pgfpathlineto{\pgfqpoint{3.317953in}{0.673128in}}%
\pgfpathlineto{\pgfqpoint{3.319683in}{0.762102in}}%
\pgfpathlineto{\pgfqpoint{3.320547in}{0.730598in}}%
\pgfpathlineto{\pgfqpoint{3.321413in}{0.743237in}}%
\pgfpathlineto{\pgfqpoint{3.322275in}{0.786350in}}%
\pgfpathlineto{\pgfqpoint{3.323139in}{0.657926in}}%
\pgfpathlineto{\pgfqpoint{3.324870in}{0.774006in}}%
\pgfpathlineto{\pgfqpoint{3.325734in}{0.750270in}}%
\pgfpathlineto{\pgfqpoint{3.326597in}{0.768070in}}%
\pgfpathlineto{\pgfqpoint{3.327461in}{0.671772in}}%
\pgfpathlineto{\pgfqpoint{3.328325in}{0.675617in}}%
\pgfpathlineto{\pgfqpoint{3.329190in}{0.688074in}}%
\pgfpathlineto{\pgfqpoint{3.330054in}{0.675548in}}%
\pgfpathlineto{\pgfqpoint{3.330917in}{0.686279in}}%
\pgfpathlineto{\pgfqpoint{3.331782in}{0.717779in}}%
\pgfpathlineto{\pgfqpoint{3.332647in}{0.676534in}}%
\pgfpathlineto{\pgfqpoint{3.333513in}{0.706532in}}%
\pgfpathlineto{\pgfqpoint{3.334376in}{0.682284in}}%
\pgfpathlineto{\pgfqpoint{3.336971in}{0.763714in}}%
\pgfpathlineto{\pgfqpoint{3.341293in}{0.672653in}}%
\pgfpathlineto{\pgfqpoint{3.342154in}{0.785254in}}%
\pgfpathlineto{\pgfqpoint{3.343883in}{0.636646in}}%
\pgfpathlineto{\pgfqpoint{3.344746in}{0.704299in}}%
\pgfpathlineto{\pgfqpoint{3.345611in}{0.648291in}}%
\pgfpathlineto{\pgfqpoint{3.346476in}{0.765691in}}%
\pgfpathlineto{\pgfqpoint{3.347340in}{0.738584in}}%
\pgfpathlineto{\pgfqpoint{3.348206in}{0.727191in}}%
\pgfpathlineto{\pgfqpoint{3.349071in}{0.728364in}}%
\pgfpathlineto{\pgfqpoint{3.349936in}{0.685581in}}%
\pgfpathlineto{\pgfqpoint{3.350801in}{0.721661in}}%
\pgfpathlineto{\pgfqpoint{3.351667in}{0.666716in}}%
\pgfpathlineto{\pgfqpoint{3.352533in}{0.743676in}}%
\pgfpathlineto{\pgfqpoint{3.353399in}{0.714921in}}%
\pgfpathlineto{\pgfqpoint{3.354265in}{0.789135in}}%
\pgfpathlineto{\pgfqpoint{3.355130in}{0.662210in}}%
\pgfpathlineto{\pgfqpoint{3.355995in}{0.687522in}}%
\pgfpathlineto{\pgfqpoint{3.356859in}{0.684631in}}%
\pgfpathlineto{\pgfqpoint{3.357724in}{0.675105in}}%
\pgfpathlineto{\pgfqpoint{3.359454in}{0.629280in}}%
\pgfpathlineto{\pgfqpoint{3.360319in}{0.711404in}}%
\pgfpathlineto{\pgfqpoint{3.361184in}{0.678402in}}%
\pgfpathlineto{\pgfqpoint{3.362913in}{0.773235in}}%
\pgfpathlineto{\pgfqpoint{3.365508in}{0.709646in}}%
\pgfpathlineto{\pgfqpoint{3.366374in}{0.702723in}}%
\pgfpathlineto{\pgfqpoint{3.367238in}{0.705216in}}%
\pgfpathlineto{\pgfqpoint{3.368103in}{0.713897in}}%
\pgfpathlineto{\pgfqpoint{3.368968in}{0.698696in}}%
\pgfpathlineto{\pgfqpoint{3.369832in}{0.790085in}}%
\pgfpathlineto{\pgfqpoint{3.370697in}{0.635290in}}%
\pgfpathlineto{\pgfqpoint{3.372429in}{0.748621in}}%
\pgfpathlineto{\pgfqpoint{3.373294in}{0.658255in}}%
\pgfpathlineto{\pgfqpoint{3.375025in}{0.741589in}}%
\pgfpathlineto{\pgfqpoint{3.375889in}{0.705947in}}%
\pgfpathlineto{\pgfqpoint{3.376753in}{0.775947in}}%
\pgfpathlineto{\pgfqpoint{3.377619in}{0.696203in}}%
\pgfpathlineto{\pgfqpoint{3.378486in}{0.697413in}}%
\pgfpathlineto{\pgfqpoint{3.379350in}{0.690819in}}%
\pgfpathlineto{\pgfqpoint{3.380214in}{0.666680in}}%
\pgfpathlineto{\pgfqpoint{3.381080in}{0.732323in}}%
\pgfpathlineto{\pgfqpoint{3.382810in}{0.656939in}}%
\pgfpathlineto{\pgfqpoint{3.383674in}{0.704006in}}%
\pgfpathlineto{\pgfqpoint{3.384537in}{0.680124in}}%
\pgfpathlineto{\pgfqpoint{3.387132in}{0.784044in}}%
\pgfpathlineto{\pgfqpoint{3.388863in}{0.724665in}}%
\pgfpathlineto{\pgfqpoint{3.389728in}{0.756498in}}%
\pgfpathlineto{\pgfqpoint{3.390594in}{0.691005in}}%
\pgfpathlineto{\pgfqpoint{3.391460in}{0.718916in}}%
\pgfpathlineto{\pgfqpoint{3.392325in}{0.713386in}}%
\pgfpathlineto{\pgfqpoint{3.393191in}{0.728295in}}%
\pgfpathlineto{\pgfqpoint{3.394056in}{0.682397in}}%
\pgfpathlineto{\pgfqpoint{3.394921in}{0.737634in}}%
\pgfpathlineto{\pgfqpoint{3.395786in}{0.706020in}}%
\pgfpathlineto{\pgfqpoint{3.396649in}{0.717304in}}%
\pgfpathlineto{\pgfqpoint{3.397513in}{0.751114in}}%
\pgfpathlineto{\pgfqpoint{3.398379in}{0.684484in}}%
\pgfpathlineto{\pgfqpoint{3.399244in}{0.821334in}}%
\pgfpathlineto{\pgfqpoint{3.402704in}{0.681991in}}%
\pgfpathlineto{\pgfqpoint{3.404434in}{0.715400in}}%
\pgfpathlineto{\pgfqpoint{3.406164in}{0.666022in}}%
\pgfpathlineto{\pgfqpoint{3.407893in}{0.698696in}}%
\pgfpathlineto{\pgfqpoint{3.409621in}{0.727670in}}%
\pgfpathlineto{\pgfqpoint{3.410487in}{0.631554in}}%
\pgfpathlineto{\pgfqpoint{3.411351in}{0.747306in}}%
\pgfpathlineto{\pgfqpoint{3.412214in}{0.729172in}}%
\pgfpathlineto{\pgfqpoint{3.413080in}{0.665730in}}%
\pgfpathlineto{\pgfqpoint{3.413944in}{0.752544in}}%
\pgfpathlineto{\pgfqpoint{3.415676in}{0.686279in}}%
\pgfpathlineto{\pgfqpoint{3.418269in}{0.778075in}}%
\pgfpathlineto{\pgfqpoint{3.419134in}{0.752982in}}%
\pgfpathlineto{\pgfqpoint{3.419999in}{0.778294in}}%
\pgfpathlineto{\pgfqpoint{3.420865in}{0.658515in}}%
\pgfpathlineto{\pgfqpoint{3.421731in}{0.664045in}}%
\pgfpathlineto{\pgfqpoint{3.422597in}{0.763421in}}%
\pgfpathlineto{\pgfqpoint{3.423462in}{0.671296in}}%
\pgfpathlineto{\pgfqpoint{3.424327in}{0.692873in}}%
\pgfpathlineto{\pgfqpoint{3.425191in}{0.667195in}}%
\pgfpathlineto{\pgfqpoint{3.426054in}{0.728807in}}%
\pgfpathlineto{\pgfqpoint{3.427782in}{0.671004in}}%
\pgfpathlineto{\pgfqpoint{3.428646in}{0.804191in}}%
\pgfpathlineto{\pgfqpoint{3.430376in}{0.671516in}}%
\pgfpathlineto{\pgfqpoint{3.432107in}{0.768915in}}%
\pgfpathlineto{\pgfqpoint{3.433836in}{0.742360in}}%
\pgfpathlineto{\pgfqpoint{3.434701in}{0.673972in}}%
\pgfpathlineto{\pgfqpoint{3.435566in}{0.822763in}}%
\pgfpathlineto{\pgfqpoint{3.437295in}{0.672653in}}%
\pgfpathlineto{\pgfqpoint{3.438159in}{0.756425in}}%
\pgfpathlineto{\pgfqpoint{3.439025in}{0.677160in}}%
\pgfpathlineto{\pgfqpoint{3.439889in}{0.691444in}}%
\pgfpathlineto{\pgfqpoint{3.440753in}{0.739136in}}%
\pgfpathlineto{\pgfqpoint{3.441619in}{0.718952in}}%
\pgfpathlineto{\pgfqpoint{3.442483in}{0.744740in}}%
\pgfpathlineto{\pgfqpoint{3.443347in}{0.720711in}}%
\pgfpathlineto{\pgfqpoint{3.444212in}{0.736351in}}%
\pgfpathlineto{\pgfqpoint{3.445076in}{0.706719in}}%
\pgfpathlineto{\pgfqpoint{3.445941in}{0.724958in}}%
\pgfpathlineto{\pgfqpoint{3.446805in}{0.655766in}}%
\pgfpathlineto{\pgfqpoint{3.447669in}{0.742726in}}%
\pgfpathlineto{\pgfqpoint{3.448533in}{0.669063in}}%
\pgfpathlineto{\pgfqpoint{3.450259in}{0.761298in}}%
\pgfpathlineto{\pgfqpoint{3.451988in}{0.702910in}}%
\pgfpathlineto{\pgfqpoint{3.452853in}{0.699979in}}%
\pgfpathlineto{\pgfqpoint{3.453718in}{0.682580in}}%
\pgfpathlineto{\pgfqpoint{3.454582in}{0.696462in}}%
\pgfpathlineto{\pgfqpoint{3.455444in}{0.691078in}}%
\pgfpathlineto{\pgfqpoint{3.456310in}{0.718331in}}%
\pgfpathlineto{\pgfqpoint{3.457175in}{0.718039in}}%
\pgfpathlineto{\pgfqpoint{3.458905in}{0.745584in}}%
\pgfpathlineto{\pgfqpoint{3.460636in}{0.675292in}}%
\pgfpathlineto{\pgfqpoint{3.462364in}{0.800528in}}%
\pgfpathlineto{\pgfqpoint{3.463229in}{0.731629in}}%
\pgfpathlineto{\pgfqpoint{3.464094in}{0.740639in}}%
\pgfpathlineto{\pgfqpoint{3.464958in}{0.737634in}}%
\pgfpathlineto{\pgfqpoint{3.465821in}{0.583679in}}%
\pgfpathlineto{\pgfqpoint{3.467552in}{0.726574in}}%
\pgfpathlineto{\pgfqpoint{3.468416in}{0.718624in}}%
\pgfpathlineto{\pgfqpoint{3.469282in}{0.756206in}}%
\pgfpathlineto{\pgfqpoint{3.471877in}{0.711445in}}%
\pgfpathlineto{\pgfqpoint{3.472743in}{0.718404in}}%
\pgfpathlineto{\pgfqpoint{3.473609in}{0.696097in}}%
\pgfpathlineto{\pgfqpoint{3.474471in}{0.711445in}}%
\pgfpathlineto{\pgfqpoint{3.475336in}{0.685804in}}%
\pgfpathlineto{\pgfqpoint{3.476201in}{0.758626in}}%
\pgfpathlineto{\pgfqpoint{3.477067in}{0.743570in}}%
\pgfpathlineto{\pgfqpoint{3.477933in}{0.741995in}}%
\pgfpathlineto{\pgfqpoint{3.478798in}{0.750676in}}%
\pgfpathlineto{\pgfqpoint{3.480525in}{0.711445in}}%
\pgfpathlineto{\pgfqpoint{3.481389in}{0.767161in}}%
\pgfpathlineto{\pgfqpoint{3.482252in}{0.697416in}}%
\pgfpathlineto{\pgfqpoint{3.483118in}{0.809136in}}%
\pgfpathlineto{\pgfqpoint{3.484847in}{0.722323in}}%
\pgfpathlineto{\pgfqpoint{3.486575in}{0.760453in}}%
\pgfpathlineto{\pgfqpoint{3.488304in}{0.711185in}}%
\pgfpathlineto{\pgfqpoint{3.489170in}{0.717706in}}%
\pgfpathlineto{\pgfqpoint{3.490036in}{0.782176in}}%
\pgfpathlineto{\pgfqpoint{3.490901in}{0.762029in}}%
\pgfpathlineto{\pgfqpoint{3.491766in}{0.707340in}}%
\pgfpathlineto{\pgfqpoint{3.492631in}{0.712687in}}%
\pgfpathlineto{\pgfqpoint{3.493495in}{0.747411in}}%
\pgfpathlineto{\pgfqpoint{3.494360in}{0.621001in}}%
\pgfpathlineto{\pgfqpoint{3.496091in}{0.683348in}}%
\pgfpathlineto{\pgfqpoint{3.497819in}{0.676680in}}%
\pgfpathlineto{\pgfqpoint{3.498684in}{0.750599in}}%
\pgfpathlineto{\pgfqpoint{3.499548in}{0.718145in}}%
\pgfpathlineto{\pgfqpoint{3.500413in}{0.742174in}}%
\pgfpathlineto{\pgfqpoint{3.502142in}{0.680051in}}%
\pgfpathlineto{\pgfqpoint{3.503006in}{0.777011in}}%
\pgfpathlineto{\pgfqpoint{3.503870in}{0.715692in}}%
\pgfpathlineto{\pgfqpoint{3.504734in}{0.765033in}}%
\pgfpathlineto{\pgfqpoint{3.505598in}{0.711551in}}%
\pgfpathlineto{\pgfqpoint{3.507329in}{0.743457in}}%
\pgfpathlineto{\pgfqpoint{3.508192in}{0.682320in}}%
\pgfpathlineto{\pgfqpoint{3.509057in}{0.774737in}}%
\pgfpathlineto{\pgfqpoint{3.509923in}{0.769280in}}%
\pgfpathlineto{\pgfqpoint{3.511653in}{0.706240in}}%
\pgfpathlineto{\pgfqpoint{3.513384in}{0.741589in}}%
\pgfpathlineto{\pgfqpoint{3.515979in}{0.659392in}}%
\pgfpathlineto{\pgfqpoint{3.516844in}{0.662835in}}%
\pgfpathlineto{\pgfqpoint{3.517709in}{0.657633in}}%
\pgfpathlineto{\pgfqpoint{3.519437in}{0.724081in}}%
\pgfpathlineto{\pgfqpoint{3.521166in}{0.678037in}}%
\pgfpathlineto{\pgfqpoint{3.522897in}{0.765216in}}%
\pgfpathlineto{\pgfqpoint{3.523762in}{0.713532in}}%
\pgfpathlineto{\pgfqpoint{3.524627in}{0.743091in}}%
\pgfpathlineto{\pgfqpoint{3.526355in}{0.672141in}}%
\pgfpathlineto{\pgfqpoint{3.527220in}{0.743607in}}%
\pgfpathlineto{\pgfqpoint{3.528086in}{0.684411in}}%
\pgfpathlineto{\pgfqpoint{3.528952in}{0.706865in}}%
\pgfpathlineto{\pgfqpoint{3.529817in}{0.702431in}}%
\pgfpathlineto{\pgfqpoint{3.530682in}{0.688220in}}%
\pgfpathlineto{\pgfqpoint{3.531546in}{0.740013in}}%
\pgfpathlineto{\pgfqpoint{3.532412in}{0.679722in}}%
\pgfpathlineto{\pgfqpoint{3.534143in}{0.740598in}}%
\pgfpathlineto{\pgfqpoint{3.535870in}{0.696129in}}%
\pgfpathlineto{\pgfqpoint{3.536735in}{0.596278in}}%
\pgfpathlineto{\pgfqpoint{3.540196in}{0.778473in}}%
\pgfpathlineto{\pgfqpoint{3.541923in}{0.655141in}}%
\pgfpathlineto{\pgfqpoint{3.542787in}{0.754956in}}%
\pgfpathlineto{\pgfqpoint{3.543652in}{0.694408in}}%
\pgfpathlineto{\pgfqpoint{3.545379in}{0.762832in}}%
\pgfpathlineto{\pgfqpoint{3.546245in}{0.714336in}}%
\pgfpathlineto{\pgfqpoint{3.547110in}{0.796899in}}%
\pgfpathlineto{\pgfqpoint{3.547975in}{0.792319in}}%
\pgfpathlineto{\pgfqpoint{3.550568in}{0.670087in}}%
\pgfpathlineto{\pgfqpoint{3.552299in}{0.739315in}}%
\pgfpathlineto{\pgfqpoint{3.553165in}{0.692613in}}%
\pgfpathlineto{\pgfqpoint{3.554030in}{0.737447in}}%
\pgfpathlineto{\pgfqpoint{3.554896in}{0.691842in}}%
\pgfpathlineto{\pgfqpoint{3.555761in}{0.738621in}}%
\pgfpathlineto{\pgfqpoint{3.556626in}{0.702212in}}%
\pgfpathlineto{\pgfqpoint{3.557492in}{0.718364in}}%
\pgfpathlineto{\pgfqpoint{3.558357in}{0.714921in}}%
\pgfpathlineto{\pgfqpoint{3.559222in}{0.714774in}}%
\pgfpathlineto{\pgfqpoint{3.560951in}{0.657597in}}%
\pgfpathlineto{\pgfqpoint{3.561816in}{0.724738in}}%
\pgfpathlineto{\pgfqpoint{3.562682in}{0.707669in}}%
\pgfpathlineto{\pgfqpoint{3.563548in}{0.630929in}}%
\pgfpathlineto{\pgfqpoint{3.564412in}{0.682942in}}%
\pgfpathlineto{\pgfqpoint{3.565276in}{0.665178in}}%
\pgfpathlineto{\pgfqpoint{3.567003in}{0.745836in}}%
\pgfpathlineto{\pgfqpoint{3.567868in}{0.694664in}}%
\pgfpathlineto{\pgfqpoint{3.568733in}{0.715067in}}%
\pgfpathlineto{\pgfqpoint{3.569599in}{0.671662in}}%
\pgfpathlineto{\pgfqpoint{3.571329in}{0.721734in}}%
\pgfpathlineto{\pgfqpoint{3.572193in}{0.633860in}}%
\pgfpathlineto{\pgfqpoint{3.573058in}{0.780820in}}%
\pgfpathlineto{\pgfqpoint{3.573923in}{0.636678in}}%
\pgfpathlineto{\pgfqpoint{3.575651in}{0.744082in}}%
\pgfpathlineto{\pgfqpoint{3.576516in}{0.640674in}}%
\pgfpathlineto{\pgfqpoint{3.577381in}{0.732615in}}%
\pgfpathlineto{\pgfqpoint{3.579112in}{0.691882in}}%
\pgfpathlineto{\pgfqpoint{3.579978in}{0.683567in}}%
\pgfpathlineto{\pgfqpoint{3.580842in}{0.687635in}}%
\pgfpathlineto{\pgfqpoint{3.581708in}{0.746461in}}%
\pgfpathlineto{\pgfqpoint{3.582573in}{0.659392in}}%
\pgfpathlineto{\pgfqpoint{3.583437in}{0.740087in}}%
\pgfpathlineto{\pgfqpoint{3.584300in}{0.665145in}}%
\pgfpathlineto{\pgfqpoint{3.585164in}{0.781957in}}%
\pgfpathlineto{\pgfqpoint{3.586028in}{0.679393in}}%
\pgfpathlineto{\pgfqpoint{3.586892in}{0.753973in}}%
\pgfpathlineto{\pgfqpoint{3.587756in}{0.687343in}}%
\pgfpathlineto{\pgfqpoint{3.588620in}{0.774047in}}%
\pgfpathlineto{\pgfqpoint{3.589485in}{0.762252in}}%
\pgfpathlineto{\pgfqpoint{3.591214in}{0.760786in}}%
\pgfpathlineto{\pgfqpoint{3.592942in}{0.710349in}}%
\pgfpathlineto{\pgfqpoint{3.593804in}{0.750237in}}%
\pgfpathlineto{\pgfqpoint{3.594668in}{0.703828in}}%
\pgfpathlineto{\pgfqpoint{3.595532in}{0.750822in}}%
\pgfpathlineto{\pgfqpoint{3.596397in}{0.736355in}}%
\pgfpathlineto{\pgfqpoint{3.597262in}{0.693864in}}%
\pgfpathlineto{\pgfqpoint{3.598992in}{0.714852in}}%
\pgfpathlineto{\pgfqpoint{3.599857in}{0.711664in}}%
\pgfpathlineto{\pgfqpoint{3.600722in}{0.683900in}}%
\pgfpathlineto{\pgfqpoint{3.603316in}{0.811187in}}%
\pgfpathlineto{\pgfqpoint{3.605045in}{0.732835in}}%
\pgfpathlineto{\pgfqpoint{3.607639in}{0.697047in}}%
\pgfpathlineto{\pgfqpoint{3.610234in}{0.760380in}}%
\pgfpathlineto{\pgfqpoint{3.611099in}{0.741808in}}%
\pgfpathlineto{\pgfqpoint{3.612827in}{0.707779in}}%
\pgfpathlineto{\pgfqpoint{3.613692in}{0.808584in}}%
\pgfpathlineto{\pgfqpoint{3.614557in}{0.801698in}}%
\pgfpathlineto{\pgfqpoint{3.615422in}{0.682028in}}%
\pgfpathlineto{\pgfqpoint{3.616289in}{0.743310in}}%
\pgfpathlineto{\pgfqpoint{3.618017in}{0.677890in}}%
\pgfpathlineto{\pgfqpoint{3.618881in}{0.786496in}}%
\pgfpathlineto{\pgfqpoint{3.619745in}{0.755654in}}%
\pgfpathlineto{\pgfqpoint{3.621474in}{0.686206in}}%
\pgfpathlineto{\pgfqpoint{3.622340in}{0.692617in}}%
\pgfpathlineto{\pgfqpoint{3.623205in}{0.754631in}}%
\pgfpathlineto{\pgfqpoint{3.624067in}{0.702066in}}%
\pgfpathlineto{\pgfqpoint{3.624932in}{0.713751in}}%
\pgfpathlineto{\pgfqpoint{3.625798in}{0.673863in}}%
\pgfpathlineto{\pgfqpoint{3.626663in}{0.680603in}}%
\pgfpathlineto{\pgfqpoint{3.628394in}{0.729724in}}%
\pgfpathlineto{\pgfqpoint{3.629258in}{0.814228in}}%
\pgfpathlineto{\pgfqpoint{3.630123in}{0.763644in}}%
\pgfpathlineto{\pgfqpoint{3.630987in}{0.792432in}}%
\pgfpathlineto{\pgfqpoint{3.632716in}{0.698696in}}%
\pgfpathlineto{\pgfqpoint{3.633580in}{0.701627in}}%
\pgfpathlineto{\pgfqpoint{3.634446in}{0.722359in}}%
\pgfpathlineto{\pgfqpoint{3.635312in}{0.714961in}}%
\pgfpathlineto{\pgfqpoint{3.637043in}{0.756243in}}%
\pgfpathlineto{\pgfqpoint{3.638773in}{0.704997in}}%
\pgfpathlineto{\pgfqpoint{3.639640in}{0.610091in}}%
\pgfpathlineto{\pgfqpoint{3.641372in}{0.751334in}}%
\pgfpathlineto{\pgfqpoint{3.642235in}{0.720345in}}%
\pgfpathlineto{\pgfqpoint{3.643962in}{0.768001in}}%
\pgfpathlineto{\pgfqpoint{3.644829in}{0.767892in}}%
\pgfpathlineto{\pgfqpoint{3.645693in}{0.691078in}}%
\pgfpathlineto{\pgfqpoint{3.647418in}{0.752690in}}%
\pgfpathlineto{\pgfqpoint{3.648283in}{0.739794in}}%
\pgfpathlineto{\pgfqpoint{3.649149in}{0.706093in}}%
\pgfpathlineto{\pgfqpoint{3.650016in}{0.735986in}}%
\pgfpathlineto{\pgfqpoint{3.651747in}{0.719354in}}%
\pgfpathlineto{\pgfqpoint{3.652611in}{0.739867in}}%
\pgfpathlineto{\pgfqpoint{3.653476in}{0.726204in}}%
\pgfpathlineto{\pgfqpoint{3.654343in}{0.755215in}}%
\pgfpathlineto{\pgfqpoint{3.655208in}{0.742100in}}%
\pgfpathlineto{\pgfqpoint{3.656074in}{0.680599in}}%
\pgfpathlineto{\pgfqpoint{3.656940in}{0.681553in}}%
\pgfpathlineto{\pgfqpoint{3.657803in}{0.746754in}}%
\pgfpathlineto{\pgfqpoint{3.658669in}{0.741516in}}%
\pgfpathlineto{\pgfqpoint{3.659533in}{0.738146in}}%
\pgfpathlineto{\pgfqpoint{3.661264in}{0.702650in}}%
\pgfpathlineto{\pgfqpoint{3.662127in}{0.777596in}}%
\pgfpathlineto{\pgfqpoint{3.662989in}{0.696020in}}%
\pgfpathlineto{\pgfqpoint{3.663855in}{0.716496in}}%
\pgfpathlineto{\pgfqpoint{3.664721in}{0.722651in}}%
\pgfpathlineto{\pgfqpoint{3.665587in}{0.745032in}}%
\pgfpathlineto{\pgfqpoint{3.666450in}{0.682284in}}%
\pgfpathlineto{\pgfqpoint{3.667313in}{0.714555in}}%
\pgfpathlineto{\pgfqpoint{3.668179in}{0.649135in}}%
\pgfpathlineto{\pgfqpoint{3.669045in}{0.734703in}}%
\pgfpathlineto{\pgfqpoint{3.670775in}{0.637303in}}%
\pgfpathlineto{\pgfqpoint{3.671641in}{0.681187in}}%
\pgfpathlineto{\pgfqpoint{3.672504in}{0.665730in}}%
\pgfpathlineto{\pgfqpoint{3.674235in}{0.727962in}}%
\pgfpathlineto{\pgfqpoint{3.675101in}{0.611845in}}%
\pgfpathlineto{\pgfqpoint{3.676831in}{0.750822in}}%
\pgfpathlineto{\pgfqpoint{3.678563in}{0.745032in}}%
\pgfpathlineto{\pgfqpoint{3.679428in}{0.729538in}}%
\pgfpathlineto{\pgfqpoint{3.680294in}{0.732063in}}%
\pgfpathlineto{\pgfqpoint{3.681160in}{0.661479in}}%
\pgfpathlineto{\pgfqpoint{3.682891in}{0.746973in}}%
\pgfpathlineto{\pgfqpoint{3.683756in}{0.688732in}}%
\pgfpathlineto{\pgfqpoint{3.685485in}{0.742872in}}%
\pgfpathlineto{\pgfqpoint{3.687215in}{0.628622in}}%
\pgfpathlineto{\pgfqpoint{3.688079in}{0.735766in}}%
\pgfpathlineto{\pgfqpoint{3.688943in}{0.689649in}}%
\pgfpathlineto{\pgfqpoint{3.689808in}{0.695179in}}%
\pgfpathlineto{\pgfqpoint{3.692401in}{0.763677in}}%
\pgfpathlineto{\pgfqpoint{3.693266in}{0.671516in}}%
\pgfpathlineto{\pgfqpoint{3.694131in}{0.710710in}}%
\pgfpathlineto{\pgfqpoint{3.694996in}{0.703349in}}%
\pgfpathlineto{\pgfqpoint{3.695860in}{0.710308in}}%
\pgfpathlineto{\pgfqpoint{3.697589in}{0.782761in}}%
\pgfpathlineto{\pgfqpoint{3.698452in}{0.692650in}}%
\pgfpathlineto{\pgfqpoint{3.700181in}{0.762540in}}%
\pgfpathlineto{\pgfqpoint{3.701046in}{0.725839in}}%
\pgfpathlineto{\pgfqpoint{3.701911in}{0.785546in}}%
\pgfpathlineto{\pgfqpoint{3.704508in}{0.676242in}}%
\pgfpathlineto{\pgfqpoint{3.705373in}{0.763823in}}%
\pgfpathlineto{\pgfqpoint{3.706238in}{0.757083in}}%
\pgfpathlineto{\pgfqpoint{3.707967in}{0.666790in}}%
\pgfpathlineto{\pgfqpoint{3.708833in}{0.669794in}}%
\pgfpathlineto{\pgfqpoint{3.709698in}{0.704080in}}%
\pgfpathlineto{\pgfqpoint{3.710563in}{0.693677in}}%
\pgfpathlineto{\pgfqpoint{3.711428in}{0.737269in}}%
\pgfpathlineto{\pgfqpoint{3.712293in}{0.715911in}}%
\pgfpathlineto{\pgfqpoint{3.713157in}{0.788733in}}%
\pgfpathlineto{\pgfqpoint{3.714886in}{0.671589in}}%
\pgfpathlineto{\pgfqpoint{3.716611in}{0.706459in}}%
\pgfpathlineto{\pgfqpoint{3.717477in}{0.684261in}}%
\pgfpathlineto{\pgfqpoint{3.718343in}{0.749352in}}%
\pgfpathlineto{\pgfqpoint{3.719209in}{0.732356in}}%
\pgfpathlineto{\pgfqpoint{3.720938in}{0.760672in}}%
\pgfpathlineto{\pgfqpoint{3.721802in}{0.690380in}}%
\pgfpathlineto{\pgfqpoint{3.724396in}{0.774591in}}%
\pgfpathlineto{\pgfqpoint{3.726126in}{0.730561in}}%
\pgfpathlineto{\pgfqpoint{3.727855in}{0.733566in}}%
\pgfpathlineto{\pgfqpoint{3.728721in}{0.716350in}}%
\pgfpathlineto{\pgfqpoint{3.729587in}{0.766531in}}%
\pgfpathlineto{\pgfqpoint{3.730452in}{0.703162in}}%
\pgfpathlineto{\pgfqpoint{3.731315in}{0.721442in}}%
\pgfpathlineto{\pgfqpoint{3.732179in}{0.719391in}}%
\pgfpathlineto{\pgfqpoint{3.733044in}{0.678841in}}%
\pgfpathlineto{\pgfqpoint{3.734774in}{0.741735in}}%
\pgfpathlineto{\pgfqpoint{3.735640in}{0.721880in}}%
\pgfpathlineto{\pgfqpoint{3.736506in}{0.646167in}}%
\pgfpathlineto{\pgfqpoint{3.738238in}{0.721222in}}%
\pgfpathlineto{\pgfqpoint{3.739103in}{0.657305in}}%
\pgfpathlineto{\pgfqpoint{3.739967in}{0.762394in}}%
\pgfpathlineto{\pgfqpoint{3.742560in}{0.678402in}}%
\pgfpathlineto{\pgfqpoint{3.744291in}{0.723675in}}%
\pgfpathlineto{\pgfqpoint{3.745156in}{0.654885in}}%
\pgfpathlineto{\pgfqpoint{3.746022in}{0.757266in}}%
\pgfpathlineto{\pgfqpoint{3.746887in}{0.731406in}}%
\pgfpathlineto{\pgfqpoint{3.747753in}{0.687229in}}%
\pgfpathlineto{\pgfqpoint{3.748618in}{0.785729in}}%
\pgfpathlineto{\pgfqpoint{3.750349in}{0.722286in}}%
\pgfpathlineto{\pgfqpoint{3.751213in}{0.730455in}}%
\pgfpathlineto{\pgfqpoint{3.752078in}{0.723788in}}%
\pgfpathlineto{\pgfqpoint{3.752944in}{0.682909in}}%
\pgfpathlineto{\pgfqpoint{3.754676in}{0.739648in}}%
\pgfpathlineto{\pgfqpoint{3.755541in}{0.674411in}}%
\pgfpathlineto{\pgfqpoint{3.756406in}{0.759576in}}%
\pgfpathlineto{\pgfqpoint{3.757272in}{0.695106in}}%
\pgfpathlineto{\pgfqpoint{3.759004in}{0.728109in}}%
\pgfpathlineto{\pgfqpoint{3.760734in}{0.722542in}}%
\pgfpathlineto{\pgfqpoint{3.762463in}{0.678914in}}%
\pgfpathlineto{\pgfqpoint{3.764191in}{0.779317in}}%
\pgfpathlineto{\pgfqpoint{3.765055in}{0.682836in}}%
\pgfpathlineto{\pgfqpoint{3.765921in}{0.795802in}}%
\pgfpathlineto{\pgfqpoint{3.766786in}{0.677525in}}%
\pgfpathlineto{\pgfqpoint{3.768513in}{0.749685in}}%
\pgfpathlineto{\pgfqpoint{3.769378in}{0.710674in}}%
\pgfpathlineto{\pgfqpoint{3.771970in}{0.766203in}}%
\pgfpathlineto{\pgfqpoint{3.772835in}{0.712541in}}%
\pgfpathlineto{\pgfqpoint{3.773700in}{0.799830in}}%
\pgfpathlineto{\pgfqpoint{3.775427in}{0.719866in}}%
\pgfpathlineto{\pgfqpoint{3.776290in}{0.778806in}}%
\pgfpathlineto{\pgfqpoint{3.778019in}{0.728328in}}%
\pgfpathlineto{\pgfqpoint{3.778884in}{0.757156in}}%
\pgfpathlineto{\pgfqpoint{3.780615in}{0.685215in}}%
\pgfpathlineto{\pgfqpoint{3.781479in}{0.736132in}}%
\pgfpathlineto{\pgfqpoint{3.782344in}{0.695399in}}%
\pgfpathlineto{\pgfqpoint{3.783208in}{0.743676in}}%
\pgfpathlineto{\pgfqpoint{3.784074in}{0.682690in}}%
\pgfpathlineto{\pgfqpoint{3.785802in}{0.768549in}}%
\pgfpathlineto{\pgfqpoint{3.786665in}{0.686352in}}%
\pgfpathlineto{\pgfqpoint{3.787530in}{0.783386in}}%
\pgfpathlineto{\pgfqpoint{3.788396in}{0.711957in}}%
\pgfpathlineto{\pgfqpoint{3.789261in}{0.727012in}}%
\pgfpathlineto{\pgfqpoint{3.791853in}{0.658588in}}%
\pgfpathlineto{\pgfqpoint{3.792718in}{0.758220in}}%
\pgfpathlineto{\pgfqpoint{3.793584in}{0.694668in}}%
\pgfpathlineto{\pgfqpoint{3.794449in}{0.740160in}}%
\pgfpathlineto{\pgfqpoint{3.795314in}{0.654154in}}%
\pgfpathlineto{\pgfqpoint{3.797911in}{0.751845in}}%
\pgfpathlineto{\pgfqpoint{3.798777in}{0.648952in}}%
\pgfpathlineto{\pgfqpoint{3.799640in}{0.698330in}}%
\pgfpathlineto{\pgfqpoint{3.800503in}{0.668844in}}%
\pgfpathlineto{\pgfqpoint{3.801369in}{0.674740in}}%
\pgfpathlineto{\pgfqpoint{3.802234in}{0.695472in}}%
\pgfpathlineto{\pgfqpoint{3.803099in}{0.751037in}}%
\pgfpathlineto{\pgfqpoint{3.803964in}{0.696755in}}%
\pgfpathlineto{\pgfqpoint{3.804830in}{0.760234in}}%
\pgfpathlineto{\pgfqpoint{3.806562in}{0.672287in}}%
\pgfpathlineto{\pgfqpoint{3.808293in}{0.744411in}}%
\pgfpathlineto{\pgfqpoint{3.809157in}{0.695618in}}%
\pgfpathlineto{\pgfqpoint{3.810020in}{0.737009in}}%
\pgfpathlineto{\pgfqpoint{3.810885in}{0.642103in}}%
\pgfpathlineto{\pgfqpoint{3.811751in}{0.706240in}}%
\pgfpathlineto{\pgfqpoint{3.812616in}{0.694444in}}%
\pgfpathlineto{\pgfqpoint{3.813481in}{0.611626in}}%
\pgfpathlineto{\pgfqpoint{3.815209in}{0.726350in}}%
\pgfpathlineto{\pgfqpoint{3.816076in}{0.716423in}}%
\pgfpathlineto{\pgfqpoint{3.816939in}{0.634445in}}%
\pgfpathlineto{\pgfqpoint{3.817803in}{0.722725in}}%
\pgfpathlineto{\pgfqpoint{3.818667in}{0.705728in}}%
\pgfpathlineto{\pgfqpoint{3.819532in}{0.726314in}}%
\pgfpathlineto{\pgfqpoint{3.820398in}{0.705655in}}%
\pgfpathlineto{\pgfqpoint{3.821264in}{0.757229in}}%
\pgfpathlineto{\pgfqpoint{3.822129in}{0.708660in}}%
\pgfpathlineto{\pgfqpoint{3.822992in}{0.719907in}}%
\pgfpathlineto{\pgfqpoint{3.823858in}{0.717048in}}%
\pgfpathlineto{\pgfqpoint{3.824723in}{0.751187in}}%
\pgfpathlineto{\pgfqpoint{3.825589in}{0.740858in}}%
\pgfpathlineto{\pgfqpoint{3.826455in}{0.692873in}}%
\pgfpathlineto{\pgfqpoint{3.828185in}{0.771408in}}%
\pgfpathlineto{\pgfqpoint{3.829917in}{0.672214in}}%
\pgfpathlineto{\pgfqpoint{3.830781in}{0.742580in}}%
\pgfpathlineto{\pgfqpoint{3.831644in}{0.698001in}}%
\pgfpathlineto{\pgfqpoint{3.832509in}{0.723861in}}%
\pgfpathlineto{\pgfqpoint{3.833375in}{0.790930in}}%
\pgfpathlineto{\pgfqpoint{3.834240in}{0.740013in}}%
\pgfpathlineto{\pgfqpoint{3.835971in}{0.769905in}}%
\pgfpathlineto{\pgfqpoint{3.836836in}{0.739575in}}%
\pgfpathlineto{\pgfqpoint{3.837698in}{0.789793in}}%
\pgfpathlineto{\pgfqpoint{3.838562in}{0.652286in}}%
\pgfpathlineto{\pgfqpoint{3.839427in}{0.740379in}}%
\pgfpathlineto{\pgfqpoint{3.840290in}{0.715838in}}%
\pgfpathlineto{\pgfqpoint{3.841155in}{0.744520in}}%
\pgfpathlineto{\pgfqpoint{3.843749in}{0.710272in}}%
\pgfpathlineto{\pgfqpoint{3.844612in}{0.730748in}}%
\pgfpathlineto{\pgfqpoint{3.845478in}{0.672360in}}%
\pgfpathlineto{\pgfqpoint{3.846342in}{0.726976in}}%
\pgfpathlineto{\pgfqpoint{3.848071in}{0.701919in}}%
\pgfpathlineto{\pgfqpoint{3.848933in}{0.746607in}}%
\pgfpathlineto{\pgfqpoint{3.849796in}{0.714263in}}%
\pgfpathlineto{\pgfqpoint{3.850662in}{0.715327in}}%
\pgfpathlineto{\pgfqpoint{3.851528in}{0.691444in}}%
\pgfpathlineto{\pgfqpoint{3.852393in}{0.767599in}}%
\pgfpathlineto{\pgfqpoint{3.853258in}{0.725218in}}%
\pgfpathlineto{\pgfqpoint{3.854122in}{0.758147in}}%
\pgfpathlineto{\pgfqpoint{3.855851in}{0.661519in}}%
\pgfpathlineto{\pgfqpoint{3.856715in}{0.666976in}}%
\pgfpathlineto{\pgfqpoint{3.857580in}{0.747964in}}%
\pgfpathlineto{\pgfqpoint{3.858443in}{0.660090in}}%
\pgfpathlineto{\pgfqpoint{3.859307in}{0.677452in}}%
\pgfpathlineto{\pgfqpoint{3.860171in}{0.734410in}}%
\pgfpathlineto{\pgfqpoint{3.861901in}{0.673863in}}%
\pgfpathlineto{\pgfqpoint{3.862767in}{0.690603in}}%
\pgfpathlineto{\pgfqpoint{3.863631in}{0.756279in}}%
\pgfpathlineto{\pgfqpoint{3.866223in}{0.659319in}}%
\pgfpathlineto{\pgfqpoint{3.867087in}{0.693092in}}%
\pgfpathlineto{\pgfqpoint{3.867951in}{0.676461in}}%
\pgfpathlineto{\pgfqpoint{3.870542in}{0.756425in}}%
\pgfpathlineto{\pgfqpoint{3.872272in}{0.675876in}}%
\pgfpathlineto{\pgfqpoint{3.873137in}{0.732835in}}%
\pgfpathlineto{\pgfqpoint{3.874002in}{0.654300in}}%
\pgfpathlineto{\pgfqpoint{3.874867in}{0.719172in}}%
\pgfpathlineto{\pgfqpoint{3.875730in}{0.694595in}}%
\pgfpathlineto{\pgfqpoint{3.877459in}{0.715400in}}%
\pgfpathlineto{\pgfqpoint{3.878324in}{0.783751in}}%
\pgfpathlineto{\pgfqpoint{3.879189in}{0.689174in}}%
\pgfpathlineto{\pgfqpoint{3.880052in}{0.764887in}}%
\pgfpathlineto{\pgfqpoint{3.880914in}{0.716058in}}%
\pgfpathlineto{\pgfqpoint{3.881778in}{0.720272in}}%
\pgfpathlineto{\pgfqpoint{3.882643in}{0.720491in}}%
\pgfpathlineto{\pgfqpoint{3.883507in}{0.748329in}}%
\pgfpathlineto{\pgfqpoint{3.884372in}{0.647158in}}%
\pgfpathlineto{\pgfqpoint{3.885238in}{0.715619in}}%
\pgfpathlineto{\pgfqpoint{3.886103in}{0.696828in}}%
\pgfpathlineto{\pgfqpoint{3.886970in}{0.709870in}}%
\pgfpathlineto{\pgfqpoint{3.887836in}{0.751114in}}%
\pgfpathlineto{\pgfqpoint{3.888702in}{0.701042in}}%
\pgfpathlineto{\pgfqpoint{3.890434in}{0.728109in}}%
\pgfpathlineto{\pgfqpoint{3.891300in}{0.685325in}}%
\pgfpathlineto{\pgfqpoint{3.892166in}{0.690088in}}%
\pgfpathlineto{\pgfqpoint{3.893031in}{0.736205in}}%
\pgfpathlineto{\pgfqpoint{3.893896in}{0.653057in}}%
\pgfpathlineto{\pgfqpoint{3.895627in}{0.760161in}}%
\pgfpathlineto{\pgfqpoint{3.896493in}{0.736863in}}%
\pgfpathlineto{\pgfqpoint{3.897360in}{0.619576in}}%
\pgfpathlineto{\pgfqpoint{3.899091in}{0.711737in}}%
\pgfpathlineto{\pgfqpoint{3.899958in}{0.711591in}}%
\pgfpathlineto{\pgfqpoint{3.900822in}{0.707084in}}%
\pgfpathlineto{\pgfqpoint{3.901689in}{0.684777in}}%
\pgfpathlineto{\pgfqpoint{3.902555in}{0.739648in}}%
\pgfpathlineto{\pgfqpoint{3.903419in}{0.676315in}}%
\pgfpathlineto{\pgfqpoint{3.904283in}{0.739502in}}%
\pgfpathlineto{\pgfqpoint{3.905147in}{0.688732in}}%
\pgfpathlineto{\pgfqpoint{3.906011in}{0.692873in}}%
\pgfpathlineto{\pgfqpoint{3.906875in}{0.684777in}}%
\pgfpathlineto{\pgfqpoint{3.907739in}{0.750456in}}%
\pgfpathlineto{\pgfqpoint{3.910332in}{0.654925in}}%
\pgfpathlineto{\pgfqpoint{3.911197in}{0.658515in}}%
\pgfpathlineto{\pgfqpoint{3.912061in}{0.709870in}}%
\pgfpathlineto{\pgfqpoint{3.912926in}{0.682324in}}%
\pgfpathlineto{\pgfqpoint{3.913792in}{0.701115in}}%
\pgfpathlineto{\pgfqpoint{3.914657in}{0.697599in}}%
\pgfpathlineto{\pgfqpoint{3.916387in}{0.701846in}}%
\pgfpathlineto{\pgfqpoint{3.917253in}{0.667232in}}%
\pgfpathlineto{\pgfqpoint{3.918984in}{0.733054in}}%
\pgfpathlineto{\pgfqpoint{3.919849in}{0.727487in}}%
\pgfpathlineto{\pgfqpoint{3.920712in}{0.705363in}}%
\pgfpathlineto{\pgfqpoint{3.921578in}{0.753201in}}%
\pgfpathlineto{\pgfqpoint{3.922443in}{0.659172in}}%
\pgfpathlineto{\pgfqpoint{3.923309in}{0.706865in}}%
\pgfpathlineto{\pgfqpoint{3.925036in}{0.660675in}}%
\pgfpathlineto{\pgfqpoint{3.926765in}{0.747086in}}%
\pgfpathlineto{\pgfqpoint{3.927629in}{0.702983in}}%
\pgfpathlineto{\pgfqpoint{3.928493in}{0.766389in}}%
\pgfpathlineto{\pgfqpoint{3.929358in}{0.647673in}}%
\pgfpathlineto{\pgfqpoint{3.930224in}{0.720418in}}%
\pgfpathlineto{\pgfqpoint{3.931951in}{0.663716in}}%
\pgfpathlineto{\pgfqpoint{3.932817in}{0.759503in}}%
\pgfpathlineto{\pgfqpoint{3.933682in}{0.737195in}}%
\pgfpathlineto{\pgfqpoint{3.934547in}{0.738625in}}%
\pgfpathlineto{\pgfqpoint{3.935411in}{0.737342in}}%
\pgfpathlineto{\pgfqpoint{3.936274in}{0.752434in}}%
\pgfpathlineto{\pgfqpoint{3.937139in}{0.744082in}}%
\pgfpathlineto{\pgfqpoint{3.938004in}{0.699613in}}%
\pgfpathlineto{\pgfqpoint{3.938870in}{0.740237in}}%
\pgfpathlineto{\pgfqpoint{3.939731in}{0.705915in}}%
\pgfpathlineto{\pgfqpoint{3.940595in}{0.738990in}}%
\pgfpathlineto{\pgfqpoint{3.941461in}{0.717820in}}%
\pgfpathlineto{\pgfqpoint{3.943191in}{0.788039in}}%
\pgfpathlineto{\pgfqpoint{3.944057in}{0.787747in}}%
\pgfpathlineto{\pgfqpoint{3.944922in}{0.764010in}}%
\pgfpathlineto{\pgfqpoint{3.945788in}{0.659944in}}%
\pgfpathlineto{\pgfqpoint{3.946653in}{0.673351in}}%
\pgfpathlineto{\pgfqpoint{3.947519in}{0.701335in}}%
\pgfpathlineto{\pgfqpoint{3.948385in}{0.700421in}}%
\pgfpathlineto{\pgfqpoint{3.949249in}{0.711079in}}%
\pgfpathlineto{\pgfqpoint{3.950114in}{0.744926in}}%
\pgfpathlineto{\pgfqpoint{3.950979in}{0.709102in}}%
\pgfpathlineto{\pgfqpoint{3.951843in}{0.714523in}}%
\pgfpathlineto{\pgfqpoint{3.952708in}{0.760640in}}%
\pgfpathlineto{\pgfqpoint{3.953573in}{0.731665in}}%
\pgfpathlineto{\pgfqpoint{3.954436in}{0.777709in}}%
\pgfpathlineto{\pgfqpoint{3.955301in}{0.736538in}}%
\pgfpathlineto{\pgfqpoint{3.956166in}{0.764156in}}%
\pgfpathlineto{\pgfqpoint{3.957031in}{0.751666in}}%
\pgfpathlineto{\pgfqpoint{3.957894in}{0.719614in}}%
\pgfpathlineto{\pgfqpoint{3.958759in}{0.723131in}}%
\pgfpathlineto{\pgfqpoint{3.959623in}{0.741995in}}%
\pgfpathlineto{\pgfqpoint{3.960486in}{0.694119in}}%
\pgfpathlineto{\pgfqpoint{3.962216in}{0.772910in}}%
\pgfpathlineto{\pgfqpoint{3.963081in}{0.748954in}}%
\pgfpathlineto{\pgfqpoint{3.963945in}{0.764083in}}%
\pgfpathlineto{\pgfqpoint{3.964811in}{0.726464in}}%
\pgfpathlineto{\pgfqpoint{3.965676in}{0.790418in}}%
\pgfpathlineto{\pgfqpoint{3.966542in}{0.776426in}}%
\pgfpathlineto{\pgfqpoint{3.967407in}{0.649322in}}%
\pgfpathlineto{\pgfqpoint{3.968272in}{0.698590in}}%
\pgfpathlineto{\pgfqpoint{3.969137in}{0.697599in}}%
\pgfpathlineto{\pgfqpoint{3.970867in}{0.745365in}}%
\pgfpathlineto{\pgfqpoint{3.971732in}{0.770969in}}%
\pgfpathlineto{\pgfqpoint{3.972596in}{0.739615in}}%
\pgfpathlineto{\pgfqpoint{3.973460in}{0.799871in}}%
\pgfpathlineto{\pgfqpoint{3.974325in}{0.745036in}}%
\pgfpathlineto{\pgfqpoint{3.975191in}{0.762434in}}%
\pgfpathlineto{\pgfqpoint{3.976055in}{0.755037in}}%
\pgfpathlineto{\pgfqpoint{3.976921in}{0.709468in}}%
\pgfpathlineto{\pgfqpoint{3.977787in}{0.735035in}}%
\pgfpathlineto{\pgfqpoint{3.978650in}{0.733606in}}%
\pgfpathlineto{\pgfqpoint{3.979516in}{0.649212in}}%
\pgfpathlineto{\pgfqpoint{3.981249in}{0.748589in}}%
\pgfpathlineto{\pgfqpoint{3.982981in}{0.690932in}}%
\pgfpathlineto{\pgfqpoint{3.983846in}{0.699394in}}%
\pgfpathlineto{\pgfqpoint{3.984712in}{0.709870in}}%
\pgfpathlineto{\pgfqpoint{3.985575in}{0.630823in}}%
\pgfpathlineto{\pgfqpoint{3.986438in}{0.682397in}}%
\pgfpathlineto{\pgfqpoint{3.987304in}{0.640308in}}%
\pgfpathlineto{\pgfqpoint{3.988169in}{0.702837in}}%
\pgfpathlineto{\pgfqpoint{3.989902in}{0.672872in}}%
\pgfpathlineto{\pgfqpoint{3.990765in}{0.729245in}}%
\pgfpathlineto{\pgfqpoint{3.991629in}{0.640235in}}%
\pgfpathlineto{\pgfqpoint{3.992494in}{0.726501in}}%
\pgfpathlineto{\pgfqpoint{3.994223in}{0.664889in}}%
\pgfpathlineto{\pgfqpoint{3.995089in}{0.769979in}}%
\pgfpathlineto{\pgfqpoint{3.995954in}{0.684704in}}%
\pgfpathlineto{\pgfqpoint{3.996820in}{0.773860in}}%
\pgfpathlineto{\pgfqpoint{3.997684in}{0.730528in}}%
\pgfpathlineto{\pgfqpoint{3.998549in}{0.755694in}}%
\pgfpathlineto{\pgfqpoint{3.999414in}{0.750822in}}%
\pgfpathlineto{\pgfqpoint{4.000279in}{0.665547in}}%
\pgfpathlineto{\pgfqpoint{4.002007in}{0.725071in}}%
\pgfpathlineto{\pgfqpoint{4.002874in}{0.698294in}}%
\pgfpathlineto{\pgfqpoint{4.004603in}{0.719281in}}%
\pgfpathlineto{\pgfqpoint{4.005467in}{0.714153in}}%
\pgfpathlineto{\pgfqpoint{4.007191in}{0.667122in}}%
\pgfpathlineto{\pgfqpoint{4.008923in}{0.738917in}}%
\pgfpathlineto{\pgfqpoint{4.009788in}{0.703495in}}%
\pgfpathlineto{\pgfqpoint{4.010652in}{0.706426in}}%
\pgfpathlineto{\pgfqpoint{4.011517in}{0.742100in}}%
\pgfpathlineto{\pgfqpoint{4.013246in}{0.697559in}}%
\pgfpathlineto{\pgfqpoint{4.014108in}{0.692175in}}%
\pgfpathlineto{\pgfqpoint{4.014974in}{0.734154in}}%
\pgfpathlineto{\pgfqpoint{4.015839in}{0.708879in}}%
\pgfpathlineto{\pgfqpoint{4.016704in}{0.773276in}}%
\pgfpathlineto{\pgfqpoint{4.017569in}{0.725181in}}%
\pgfpathlineto{\pgfqpoint{4.018434in}{0.751772in}}%
\pgfpathlineto{\pgfqpoint{4.019301in}{0.700636in}}%
\pgfpathlineto{\pgfqpoint{4.020166in}{0.736132in}}%
\pgfpathlineto{\pgfqpoint{4.021032in}{0.732689in}}%
\pgfpathlineto{\pgfqpoint{4.021898in}{0.737122in}}%
\pgfpathlineto{\pgfqpoint{4.023629in}{0.705915in}}%
\pgfpathlineto{\pgfqpoint{4.024494in}{0.740492in}}%
\pgfpathlineto{\pgfqpoint{4.025359in}{0.730821in}}%
\pgfpathlineto{\pgfqpoint{4.026224in}{0.710527in}}%
\pgfpathlineto{\pgfqpoint{4.027953in}{0.750603in}}%
\pgfpathlineto{\pgfqpoint{4.028818in}{0.692727in}}%
\pgfpathlineto{\pgfqpoint{4.029684in}{0.777271in}}%
\pgfpathlineto{\pgfqpoint{4.031412in}{0.680566in}}%
\pgfpathlineto{\pgfqpoint{4.032277in}{0.682105in}}%
\pgfpathlineto{\pgfqpoint{4.034007in}{0.771517in}}%
\pgfpathlineto{\pgfqpoint{4.036604in}{0.708477in}}%
\pgfpathlineto{\pgfqpoint{4.037469in}{0.749320in}}%
\pgfpathlineto{\pgfqpoint{4.038334in}{0.705915in}}%
\pgfpathlineto{\pgfqpoint{4.039199in}{0.709577in}}%
\pgfpathlineto{\pgfqpoint{4.040065in}{0.696682in}}%
\pgfpathlineto{\pgfqpoint{4.040931in}{0.738734in}}%
\pgfpathlineto{\pgfqpoint{4.042662in}{0.685110in}}%
\pgfpathlineto{\pgfqpoint{4.045257in}{0.753348in}}%
\pgfpathlineto{\pgfqpoint{4.046121in}{0.736497in}}%
\pgfpathlineto{\pgfqpoint{4.046986in}{0.737926in}}%
\pgfpathlineto{\pgfqpoint{4.047850in}{0.756718in}}%
\pgfpathlineto{\pgfqpoint{4.048715in}{0.691517in}}%
\pgfpathlineto{\pgfqpoint{4.049579in}{0.785546in}}%
\pgfpathlineto{\pgfqpoint{4.050444in}{0.706792in}}%
\pgfpathlineto{\pgfqpoint{4.052176in}{0.787706in}}%
\pgfpathlineto{\pgfqpoint{4.053040in}{0.681443in}}%
\pgfpathlineto{\pgfqpoint{4.053906in}{0.740087in}}%
\pgfpathlineto{\pgfqpoint{4.054771in}{0.711478in}}%
\pgfpathlineto{\pgfqpoint{4.055636in}{0.645948in}}%
\pgfpathlineto{\pgfqpoint{4.056502in}{0.775363in}}%
\pgfpathlineto{\pgfqpoint{4.058228in}{0.714409in}}%
\pgfpathlineto{\pgfqpoint{4.059092in}{0.730382in}}%
\pgfpathlineto{\pgfqpoint{4.060822in}{0.690380in}}%
\pgfpathlineto{\pgfqpoint{4.061687in}{0.740306in}}%
\pgfpathlineto{\pgfqpoint{4.062552in}{0.685910in}}%
\pgfpathlineto{\pgfqpoint{4.063417in}{0.733493in}}%
\pgfpathlineto{\pgfqpoint{4.064282in}{0.721953in}}%
\pgfpathlineto{\pgfqpoint{4.065147in}{0.712432in}}%
\pgfpathlineto{\pgfqpoint{4.066012in}{0.764408in}}%
\pgfpathlineto{\pgfqpoint{4.067741in}{0.705874in}}%
\pgfpathlineto{\pgfqpoint{4.068604in}{0.738438in}}%
\pgfpathlineto{\pgfqpoint{4.069469in}{0.672762in}}%
\pgfpathlineto{\pgfqpoint{4.071198in}{0.700344in}}%
\pgfpathlineto{\pgfqpoint{4.072928in}{0.676315in}}%
\pgfpathlineto{\pgfqpoint{4.073792in}{0.688585in}}%
\pgfpathlineto{\pgfqpoint{4.074657in}{0.729136in}}%
\pgfpathlineto{\pgfqpoint{4.075522in}{0.720638in}}%
\pgfpathlineto{\pgfqpoint{4.076388in}{0.705582in}}%
\pgfpathlineto{\pgfqpoint{4.078118in}{0.761663in}}%
\pgfpathlineto{\pgfqpoint{4.079849in}{0.695252in}}%
\pgfpathlineto{\pgfqpoint{4.080713in}{0.698622in}}%
\pgfpathlineto{\pgfqpoint{4.081580in}{0.700381in}}%
\pgfpathlineto{\pgfqpoint{4.082444in}{0.782468in}}%
\pgfpathlineto{\pgfqpoint{4.084175in}{0.686608in}}%
\pgfpathlineto{\pgfqpoint{4.085039in}{0.725802in}}%
\pgfpathlineto{\pgfqpoint{4.085903in}{0.695106in}}%
\pgfpathlineto{\pgfqpoint{4.086768in}{0.739063in}}%
\pgfpathlineto{\pgfqpoint{4.087632in}{0.728035in}}%
\pgfpathlineto{\pgfqpoint{4.088498in}{0.652798in}}%
\pgfpathlineto{\pgfqpoint{4.090229in}{0.740858in}}%
\pgfpathlineto{\pgfqpoint{4.091095in}{0.708915in}}%
\pgfpathlineto{\pgfqpoint{4.093691in}{0.825029in}}%
\pgfpathlineto{\pgfqpoint{4.094556in}{0.743895in}}%
\pgfpathlineto{\pgfqpoint{4.095420in}{0.805653in}}%
\pgfpathlineto{\pgfqpoint{4.097151in}{0.707450in}}%
\pgfpathlineto{\pgfqpoint{4.098016in}{0.758439in}}%
\pgfpathlineto{\pgfqpoint{4.098881in}{0.741589in}}%
\pgfpathlineto{\pgfqpoint{4.099746in}{0.688732in}}%
\pgfpathlineto{\pgfqpoint{4.100611in}{0.695216in}}%
\pgfpathlineto{\pgfqpoint{4.101476in}{0.725437in}}%
\pgfpathlineto{\pgfqpoint{4.102341in}{0.687781in}}%
\pgfpathlineto{\pgfqpoint{4.103206in}{0.764923in}}%
\pgfpathlineto{\pgfqpoint{4.104071in}{0.738073in}}%
\pgfpathlineto{\pgfqpoint{4.104936in}{0.753128in}}%
\pgfpathlineto{\pgfqpoint{4.107530in}{0.608771in}}%
\pgfpathlineto{\pgfqpoint{4.108395in}{0.726939in}}%
\pgfpathlineto{\pgfqpoint{4.110125in}{0.670310in}}%
\pgfpathlineto{\pgfqpoint{4.110990in}{0.689722in}}%
\pgfpathlineto{\pgfqpoint{4.111854in}{0.645473in}}%
\pgfpathlineto{\pgfqpoint{4.113584in}{0.713386in}}%
\pgfpathlineto{\pgfqpoint{4.114449in}{0.703422in}}%
\pgfpathlineto{\pgfqpoint{4.115313in}{0.678037in}}%
\pgfpathlineto{\pgfqpoint{4.116177in}{0.733972in}}%
\pgfpathlineto{\pgfqpoint{4.117040in}{0.677562in}}%
\pgfpathlineto{\pgfqpoint{4.117905in}{0.691371in}}%
\pgfpathlineto{\pgfqpoint{4.119636in}{0.766243in}}%
\pgfpathlineto{\pgfqpoint{4.120501in}{0.755475in}}%
\pgfpathlineto{\pgfqpoint{4.122231in}{0.690859in}}%
\pgfpathlineto{\pgfqpoint{4.123096in}{0.727012in}}%
\pgfpathlineto{\pgfqpoint{4.123961in}{0.683900in}}%
\pgfpathlineto{\pgfqpoint{4.125692in}{0.767745in}}%
\pgfpathlineto{\pgfqpoint{4.126557in}{0.697563in}}%
\pgfpathlineto{\pgfqpoint{4.127422in}{0.716244in}}%
\pgfpathlineto{\pgfqpoint{4.129154in}{0.692617in}}%
\pgfpathlineto{\pgfqpoint{4.130019in}{0.702983in}}%
\pgfpathlineto{\pgfqpoint{4.130884in}{0.667195in}}%
\pgfpathlineto{\pgfqpoint{4.132610in}{0.711810in}}%
\pgfpathlineto{\pgfqpoint{4.133474in}{0.753348in}}%
\pgfpathlineto{\pgfqpoint{4.134339in}{0.635143in}}%
\pgfpathlineto{\pgfqpoint{4.135203in}{0.724373in}}%
\pgfpathlineto{\pgfqpoint{4.136068in}{0.722067in}}%
\pgfpathlineto{\pgfqpoint{4.137798in}{0.785180in}}%
\pgfpathlineto{\pgfqpoint{4.139529in}{0.708952in}}%
\pgfpathlineto{\pgfqpoint{4.141259in}{0.722761in}}%
\pgfpathlineto{\pgfqpoint{4.142124in}{0.721661in}}%
\pgfpathlineto{\pgfqpoint{4.143853in}{0.747484in}}%
\pgfpathlineto{\pgfqpoint{4.144718in}{0.714482in}}%
\pgfpathlineto{\pgfqpoint{4.145583in}{0.772577in}}%
\pgfpathlineto{\pgfqpoint{4.148177in}{0.673310in}}%
\pgfpathlineto{\pgfqpoint{4.149043in}{0.710820in}}%
\pgfpathlineto{\pgfqpoint{4.150774in}{0.667195in}}%
\pgfpathlineto{\pgfqpoint{4.151640in}{0.725656in}}%
\pgfpathlineto{\pgfqpoint{4.152507in}{0.711883in}}%
\pgfpathlineto{\pgfqpoint{4.153374in}{0.688147in}}%
\pgfpathlineto{\pgfqpoint{4.154234in}{0.744667in}}%
\pgfpathlineto{\pgfqpoint{4.155100in}{0.669100in}}%
\pgfpathlineto{\pgfqpoint{4.156830in}{0.737780in}}%
\pgfpathlineto{\pgfqpoint{4.157692in}{0.674703in}}%
\pgfpathlineto{\pgfqpoint{4.158557in}{0.725583in}}%
\pgfpathlineto{\pgfqpoint{4.159420in}{0.702618in}}%
\pgfpathlineto{\pgfqpoint{4.160285in}{0.716354in}}%
\pgfpathlineto{\pgfqpoint{4.161148in}{0.709943in}}%
\pgfpathlineto{\pgfqpoint{4.162013in}{0.726793in}}%
\pgfpathlineto{\pgfqpoint{4.162876in}{0.658149in}}%
\pgfpathlineto{\pgfqpoint{4.164607in}{0.693937in}}%
\pgfpathlineto{\pgfqpoint{4.165472in}{0.658441in}}%
\pgfpathlineto{\pgfqpoint{4.166337in}{0.766535in}}%
\pgfpathlineto{\pgfqpoint{4.167203in}{0.637490in}}%
\pgfpathlineto{\pgfqpoint{4.168067in}{0.728515in}}%
\pgfpathlineto{\pgfqpoint{4.169798in}{0.666318in}}%
\pgfpathlineto{\pgfqpoint{4.170663in}{0.693717in}}%
\pgfpathlineto{\pgfqpoint{4.171528in}{0.661811in}}%
\pgfpathlineto{\pgfqpoint{4.173258in}{0.693937in}}%
\pgfpathlineto{\pgfqpoint{4.174125in}{0.646719in}}%
\pgfpathlineto{\pgfqpoint{4.174991in}{0.695951in}}%
\pgfpathlineto{\pgfqpoint{4.175858in}{0.625805in}}%
\pgfpathlineto{\pgfqpoint{4.177588in}{0.662396in}}%
\pgfpathlineto{\pgfqpoint{4.178452in}{0.646610in}}%
\pgfpathlineto{\pgfqpoint{4.180184in}{0.713166in}}%
\pgfpathlineto{\pgfqpoint{4.181050in}{0.677086in}}%
\pgfpathlineto{\pgfqpoint{4.181915in}{0.735693in}}%
\pgfpathlineto{\pgfqpoint{4.183646in}{0.668077in}}%
\pgfpathlineto{\pgfqpoint{4.185377in}{0.736059in}}%
\pgfpathlineto{\pgfqpoint{4.186240in}{0.682909in}}%
\pgfpathlineto{\pgfqpoint{4.187102in}{0.729099in}}%
\pgfpathlineto{\pgfqpoint{4.187968in}{0.684265in}}%
\pgfpathlineto{\pgfqpoint{4.188833in}{0.707271in}}%
\pgfpathlineto{\pgfqpoint{4.189700in}{0.697965in}}%
\pgfpathlineto{\pgfqpoint{4.190565in}{0.717048in}}%
\pgfpathlineto{\pgfqpoint{4.191432in}{0.680822in}}%
\pgfpathlineto{\pgfqpoint{4.192297in}{0.724779in}}%
\pgfpathlineto{\pgfqpoint{4.193163in}{0.678589in}}%
\pgfpathlineto{\pgfqpoint{4.194030in}{0.692581in}}%
\pgfpathlineto{\pgfqpoint{4.194895in}{0.680237in}}%
\pgfpathlineto{\pgfqpoint{4.195760in}{0.702289in}}%
\pgfpathlineto{\pgfqpoint{4.196623in}{0.637344in}}%
\pgfpathlineto{\pgfqpoint{4.197486in}{0.645513in}}%
\pgfpathlineto{\pgfqpoint{4.199218in}{0.775330in}}%
\pgfpathlineto{\pgfqpoint{4.200083in}{0.675072in}}%
\pgfpathlineto{\pgfqpoint{4.201815in}{0.744155in}}%
\pgfpathlineto{\pgfqpoint{4.202680in}{0.661081in}}%
\pgfpathlineto{\pgfqpoint{4.204411in}{0.779870in}}%
\pgfpathlineto{\pgfqpoint{4.206142in}{0.638367in}}%
\pgfpathlineto{\pgfqpoint{4.208736in}{0.766974in}}%
\pgfpathlineto{\pgfqpoint{4.210467in}{0.695585in}}%
\pgfpathlineto{\pgfqpoint{4.212198in}{0.659871in}}%
\pgfpathlineto{\pgfqpoint{4.213925in}{0.786975in}}%
\pgfpathlineto{\pgfqpoint{4.215654in}{0.664191in}}%
\pgfpathlineto{\pgfqpoint{4.217385in}{0.721774in}}%
\pgfpathlineto{\pgfqpoint{4.218250in}{0.698184in}}%
\pgfpathlineto{\pgfqpoint{4.219976in}{0.707157in}}%
\pgfpathlineto{\pgfqpoint{4.220842in}{0.701700in}}%
\pgfpathlineto{\pgfqpoint{4.221705in}{0.686681in}}%
\pgfpathlineto{\pgfqpoint{4.223436in}{0.756937in}}%
\pgfpathlineto{\pgfqpoint{4.224302in}{0.775582in}}%
\pgfpathlineto{\pgfqpoint{4.226031in}{0.682544in}}%
\pgfpathlineto{\pgfqpoint{4.226894in}{0.705951in}}%
\pgfpathlineto{\pgfqpoint{4.227760in}{0.634924in}}%
\pgfpathlineto{\pgfqpoint{4.229490in}{0.710308in}}%
\pgfpathlineto{\pgfqpoint{4.230356in}{0.718697in}}%
\pgfpathlineto{\pgfqpoint{4.232084in}{0.643313in}}%
\pgfpathlineto{\pgfqpoint{4.232949in}{0.732469in}}%
\pgfpathlineto{\pgfqpoint{4.233814in}{0.688110in}}%
\pgfpathlineto{\pgfqpoint{4.234678in}{0.758074in}}%
\pgfpathlineto{\pgfqpoint{4.236409in}{0.658186in}}%
\pgfpathlineto{\pgfqpoint{4.238138in}{0.724998in}}%
\pgfpathlineto{\pgfqpoint{4.239003in}{0.711006in}}%
\pgfpathlineto{\pgfqpoint{4.240732in}{0.740419in}}%
\pgfpathlineto{\pgfqpoint{4.243325in}{0.711372in}}%
\pgfpathlineto{\pgfqpoint{4.244190in}{0.734191in}}%
\pgfpathlineto{\pgfqpoint{4.245054in}{0.733825in}}%
\pgfpathlineto{\pgfqpoint{4.245919in}{0.739502in}}%
\pgfpathlineto{\pgfqpoint{4.246784in}{0.710235in}}%
\pgfpathlineto{\pgfqpoint{4.248514in}{0.754631in}}%
\pgfpathlineto{\pgfqpoint{4.249380in}{0.720126in}}%
\pgfpathlineto{\pgfqpoint{4.250245in}{0.638331in}}%
\pgfpathlineto{\pgfqpoint{4.251107in}{0.766682in}}%
\pgfpathlineto{\pgfqpoint{4.251972in}{0.749320in}}%
\pgfpathlineto{\pgfqpoint{4.252837in}{0.679356in}}%
\pgfpathlineto{\pgfqpoint{4.253702in}{0.727085in}}%
\pgfpathlineto{\pgfqpoint{4.255431in}{0.650418in}}%
\pgfpathlineto{\pgfqpoint{4.257163in}{0.720126in}}%
\pgfpathlineto{\pgfqpoint{4.258027in}{0.719281in}}%
\pgfpathlineto{\pgfqpoint{4.258891in}{0.724300in}}%
\pgfpathlineto{\pgfqpoint{4.259758in}{0.652578in}}%
\pgfpathlineto{\pgfqpoint{4.262349in}{0.770015in}}%
\pgfpathlineto{\pgfqpoint{4.263215in}{0.700198in}}%
\pgfpathlineto{\pgfqpoint{4.265810in}{0.789720in}}%
\pgfpathlineto{\pgfqpoint{4.267541in}{0.704080in}}%
\pgfpathlineto{\pgfqpoint{4.268406in}{0.742872in}}%
\pgfpathlineto{\pgfqpoint{4.269270in}{0.675511in}}%
\pgfpathlineto{\pgfqpoint{4.271000in}{0.745876in}}%
\pgfpathlineto{\pgfqpoint{4.272729in}{0.645327in}}%
\pgfpathlineto{\pgfqpoint{4.273593in}{0.719647in}}%
\pgfpathlineto{\pgfqpoint{4.274458in}{0.694335in}}%
\pgfpathlineto{\pgfqpoint{4.276188in}{0.757887in}}%
\pgfpathlineto{\pgfqpoint{4.277054in}{0.625325in}}%
\pgfpathlineto{\pgfqpoint{4.277919in}{0.687010in}}%
\pgfpathlineto{\pgfqpoint{4.278785in}{0.624335in}}%
\pgfpathlineto{\pgfqpoint{4.280515in}{0.760819in}}%
\pgfpathlineto{\pgfqpoint{4.281379in}{0.653053in}}%
\pgfpathlineto{\pgfqpoint{4.282243in}{0.750416in}}%
\pgfpathlineto{\pgfqpoint{4.284839in}{0.667561in}}%
\pgfpathlineto{\pgfqpoint{4.285704in}{0.681626in}}%
\pgfpathlineto{\pgfqpoint{4.286569in}{0.667707in}}%
\pgfpathlineto{\pgfqpoint{4.287434in}{0.614411in}}%
\pgfpathlineto{\pgfqpoint{4.289164in}{0.701993in}}%
\pgfpathlineto{\pgfqpoint{4.290029in}{0.691882in}}%
\pgfpathlineto{\pgfqpoint{4.290894in}{0.705436in}}%
\pgfpathlineto{\pgfqpoint{4.291759in}{0.663899in}}%
\pgfpathlineto{\pgfqpoint{4.292625in}{0.695252in}}%
\pgfpathlineto{\pgfqpoint{4.293491in}{0.658953in}}%
\pgfpathlineto{\pgfqpoint{4.296948in}{0.707377in}}%
\pgfpathlineto{\pgfqpoint{4.298678in}{0.697193in}}%
\pgfpathlineto{\pgfqpoint{4.300410in}{0.599871in}}%
\pgfpathlineto{\pgfqpoint{4.301276in}{0.681886in}}%
\pgfpathlineto{\pgfqpoint{4.302140in}{0.681699in}}%
\pgfpathlineto{\pgfqpoint{4.303005in}{0.682653in}}%
\pgfpathlineto{\pgfqpoint{4.303869in}{0.638148in}}%
\pgfpathlineto{\pgfqpoint{4.304734in}{0.725802in}}%
\pgfpathlineto{\pgfqpoint{4.305599in}{0.684119in}}%
\pgfpathlineto{\pgfqpoint{4.306465in}{0.762581in}}%
\pgfpathlineto{\pgfqpoint{4.307331in}{0.709139in}}%
\pgfpathlineto{\pgfqpoint{4.308196in}{0.714742in}}%
\pgfpathlineto{\pgfqpoint{4.309061in}{0.709358in}}%
\pgfpathlineto{\pgfqpoint{4.309926in}{0.658039in}}%
\pgfpathlineto{\pgfqpoint{4.311656in}{0.724446in}}%
\pgfpathlineto{\pgfqpoint{4.312522in}{0.695658in}}%
\pgfpathlineto{\pgfqpoint{4.313387in}{0.731885in}}%
\pgfpathlineto{\pgfqpoint{4.314253in}{0.728515in}}%
\pgfpathlineto{\pgfqpoint{4.315118in}{0.677598in}}%
\pgfpathlineto{\pgfqpoint{4.316847in}{0.735730in}}%
\pgfpathlineto{\pgfqpoint{4.318576in}{0.713605in}}%
\pgfpathlineto{\pgfqpoint{4.319442in}{0.723752in}}%
\pgfpathlineto{\pgfqpoint{4.320307in}{0.720711in}}%
\pgfpathlineto{\pgfqpoint{4.321170in}{0.696901in}}%
\pgfpathlineto{\pgfqpoint{4.322035in}{0.744045in}}%
\pgfpathlineto{\pgfqpoint{4.323763in}{0.708002in}}%
\pgfpathlineto{\pgfqpoint{4.324629in}{0.750895in}}%
\pgfpathlineto{\pgfqpoint{4.326357in}{0.678991in}}%
\pgfpathlineto{\pgfqpoint{4.328083in}{0.741954in}}%
\pgfpathlineto{\pgfqpoint{4.328948in}{0.672982in}}%
\pgfpathlineto{\pgfqpoint{4.329813in}{0.756718in}}%
\pgfpathlineto{\pgfqpoint{4.330678in}{0.735986in}}%
\pgfpathlineto{\pgfqpoint{4.331544in}{0.705326in}}%
\pgfpathlineto{\pgfqpoint{4.332409in}{0.762102in}}%
\pgfpathlineto{\pgfqpoint{4.333276in}{0.701554in}}%
\pgfpathlineto{\pgfqpoint{4.334141in}{0.717194in}}%
\pgfpathlineto{\pgfqpoint{4.335005in}{0.662835in}}%
\pgfpathlineto{\pgfqpoint{4.335868in}{0.766353in}}%
\pgfpathlineto{\pgfqpoint{4.336733in}{0.654852in}}%
\pgfpathlineto{\pgfqpoint{4.337598in}{0.677744in}}%
\pgfpathlineto{\pgfqpoint{4.338461in}{0.707230in}}%
\pgfpathlineto{\pgfqpoint{4.339326in}{0.787560in}}%
\pgfpathlineto{\pgfqpoint{4.341058in}{0.672835in}}%
\pgfpathlineto{\pgfqpoint{4.341924in}{0.684484in}}%
\pgfpathlineto{\pgfqpoint{4.342786in}{0.693165in}}%
\pgfpathlineto{\pgfqpoint{4.344511in}{0.732104in}}%
\pgfpathlineto{\pgfqpoint{4.345375in}{0.739794in}}%
\pgfpathlineto{\pgfqpoint{4.347106in}{0.702285in}}%
\pgfpathlineto{\pgfqpoint{4.347971in}{0.723898in}}%
\pgfpathlineto{\pgfqpoint{4.348837in}{0.809063in}}%
\pgfpathlineto{\pgfqpoint{4.350566in}{0.684594in}}%
\pgfpathlineto{\pgfqpoint{4.352298in}{0.736424in}}%
\pgfpathlineto{\pgfqpoint{4.353162in}{0.686279in}}%
\pgfpathlineto{\pgfqpoint{4.355757in}{0.770677in}}%
\pgfpathlineto{\pgfqpoint{4.356622in}{0.718550in}}%
\pgfpathlineto{\pgfqpoint{4.358349in}{0.793642in}}%
\pgfpathlineto{\pgfqpoint{4.359212in}{0.772325in}}%
\pgfpathlineto{\pgfqpoint{4.360077in}{0.803314in}}%
\pgfpathlineto{\pgfqpoint{4.363537in}{0.683315in}}%
\pgfpathlineto{\pgfqpoint{4.365264in}{0.719834in}}%
\pgfpathlineto{\pgfqpoint{4.366126in}{0.656501in}}%
\pgfpathlineto{\pgfqpoint{4.366989in}{0.728441in}}%
\pgfpathlineto{\pgfqpoint{4.367853in}{0.694375in}}%
\pgfpathlineto{\pgfqpoint{4.368718in}{0.755914in}}%
\pgfpathlineto{\pgfqpoint{4.370449in}{0.698330in}}%
\pgfpathlineto{\pgfqpoint{4.371313in}{0.758878in}}%
\pgfpathlineto{\pgfqpoint{4.373044in}{0.663314in}}%
\pgfpathlineto{\pgfqpoint{4.373909in}{0.665510in}}%
\pgfpathlineto{\pgfqpoint{4.374774in}{0.745438in}}%
\pgfpathlineto{\pgfqpoint{4.375640in}{0.682617in}}%
\pgfpathlineto{\pgfqpoint{4.376505in}{0.738588in}}%
\pgfpathlineto{\pgfqpoint{4.378235in}{0.665108in}}%
\pgfpathlineto{\pgfqpoint{4.380827in}{0.736392in}}%
\pgfpathlineto{\pgfqpoint{4.383424in}{0.708846in}}%
\pgfpathlineto{\pgfqpoint{4.384290in}{0.673278in}}%
\pgfpathlineto{\pgfqpoint{4.386020in}{0.743132in}}%
\pgfpathlineto{\pgfqpoint{4.387751in}{0.693238in}}%
\pgfpathlineto{\pgfqpoint{4.388616in}{0.692325in}}%
\pgfpathlineto{\pgfqpoint{4.389481in}{0.706426in}}%
\pgfpathlineto{\pgfqpoint{4.390346in}{0.695512in}}%
\pgfpathlineto{\pgfqpoint{4.391212in}{0.719395in}}%
\pgfpathlineto{\pgfqpoint{4.392077in}{0.709285in}}%
\pgfpathlineto{\pgfqpoint{4.392939in}{0.660346in}}%
\pgfpathlineto{\pgfqpoint{4.394670in}{0.722213in}}%
\pgfpathlineto{\pgfqpoint{4.395533in}{0.719834in}}%
\pgfpathlineto{\pgfqpoint{4.397262in}{0.662729in}}%
\pgfpathlineto{\pgfqpoint{4.398127in}{0.756575in}}%
\pgfpathlineto{\pgfqpoint{4.398990in}{0.703681in}}%
\pgfpathlineto{\pgfqpoint{4.401586in}{0.745730in}}%
\pgfpathlineto{\pgfqpoint{4.403318in}{0.685987in}}%
\pgfpathlineto{\pgfqpoint{4.405047in}{0.725693in}}%
\pgfpathlineto{\pgfqpoint{4.405913in}{0.703056in}}%
\pgfpathlineto{\pgfqpoint{4.406779in}{0.716829in}}%
\pgfpathlineto{\pgfqpoint{4.407644in}{0.632581in}}%
\pgfpathlineto{\pgfqpoint{4.408510in}{0.805620in}}%
\pgfpathlineto{\pgfqpoint{4.409376in}{0.718404in}}%
\pgfpathlineto{\pgfqpoint{4.410240in}{0.723642in}}%
\pgfpathlineto{\pgfqpoint{4.411970in}{0.701152in}}%
\pgfpathlineto{\pgfqpoint{4.412836in}{0.691225in}}%
\pgfpathlineto{\pgfqpoint{4.413701in}{0.771408in}}%
\pgfpathlineto{\pgfqpoint{4.415430in}{0.682836in}}%
\pgfpathlineto{\pgfqpoint{4.418891in}{0.762581in}}%
\pgfpathlineto{\pgfqpoint{4.419756in}{0.660382in}}%
\pgfpathlineto{\pgfqpoint{4.420621in}{0.677744in}}%
\pgfpathlineto{\pgfqpoint{4.422347in}{0.735474in}}%
\pgfpathlineto{\pgfqpoint{4.423211in}{0.718624in}}%
\pgfpathlineto{\pgfqpoint{4.424076in}{0.748073in}}%
\pgfpathlineto{\pgfqpoint{4.424939in}{0.680051in}}%
\pgfpathlineto{\pgfqpoint{4.425802in}{0.693092in}}%
\pgfpathlineto{\pgfqpoint{4.426667in}{0.704339in}}%
\pgfpathlineto{\pgfqpoint{4.428398in}{0.771408in}}%
\pgfpathlineto{\pgfqpoint{4.430129in}{0.725583in}}%
\pgfpathlineto{\pgfqpoint{4.430994in}{0.772764in}}%
\pgfpathlineto{\pgfqpoint{4.431859in}{0.711299in}}%
\pgfpathlineto{\pgfqpoint{4.432725in}{0.781445in}}%
\pgfpathlineto{\pgfqpoint{4.433588in}{0.777307in}}%
\pgfpathlineto{\pgfqpoint{4.436184in}{0.663095in}}%
\pgfpathlineto{\pgfqpoint{4.437050in}{0.757781in}}%
\pgfpathlineto{\pgfqpoint{4.437915in}{0.750456in}}%
\pgfpathlineto{\pgfqpoint{4.438780in}{0.726866in}}%
\pgfpathlineto{\pgfqpoint{4.439645in}{0.665108in}}%
\pgfpathlineto{\pgfqpoint{4.442235in}{0.746940in}}%
\pgfpathlineto{\pgfqpoint{4.443964in}{0.699759in}}%
\pgfpathlineto{\pgfqpoint{4.444829in}{0.723642in}}%
\pgfpathlineto{\pgfqpoint{4.445695in}{0.680676in}}%
\pgfpathlineto{\pgfqpoint{4.446561in}{0.699101in}}%
\pgfpathlineto{\pgfqpoint{4.447427in}{0.698257in}}%
\pgfpathlineto{\pgfqpoint{4.449156in}{0.629028in}}%
\pgfpathlineto{\pgfqpoint{4.450020in}{0.698846in}}%
\pgfpathlineto{\pgfqpoint{4.450885in}{0.683753in}}%
\pgfpathlineto{\pgfqpoint{4.451751in}{0.715546in}}%
\pgfpathlineto{\pgfqpoint{4.452617in}{0.702947in}}%
\pgfpathlineto{\pgfqpoint{4.453484in}{0.717706in}}%
\pgfpathlineto{\pgfqpoint{4.454349in}{0.684411in}}%
\pgfpathlineto{\pgfqpoint{4.455213in}{0.740675in}}%
\pgfpathlineto{\pgfqpoint{4.456077in}{0.674082in}}%
\pgfpathlineto{\pgfqpoint{4.456943in}{0.756133in}}%
\pgfpathlineto{\pgfqpoint{4.457807in}{0.742872in}}%
\pgfpathlineto{\pgfqpoint{4.458672in}{0.730894in}}%
\pgfpathlineto{\pgfqpoint{4.459537in}{0.672762in}}%
\pgfpathlineto{\pgfqpoint{4.461266in}{0.715765in}}%
\pgfpathlineto{\pgfqpoint{4.462131in}{0.700417in}}%
\pgfpathlineto{\pgfqpoint{4.462995in}{0.737342in}}%
\pgfpathlineto{\pgfqpoint{4.464723in}{0.702581in}}%
\pgfpathlineto{\pgfqpoint{4.465588in}{0.691371in}}%
\pgfpathlineto{\pgfqpoint{4.466453in}{0.749758in}}%
\pgfpathlineto{\pgfqpoint{4.467318in}{0.693604in}}%
\pgfpathlineto{\pgfqpoint{4.468183in}{0.706792in}}%
\pgfpathlineto{\pgfqpoint{4.469047in}{0.671995in}}%
\pgfpathlineto{\pgfqpoint{4.470777in}{0.742068in}}%
\pgfpathlineto{\pgfqpoint{4.471642in}{0.702800in}}%
\pgfpathlineto{\pgfqpoint{4.472507in}{0.785180in}}%
\pgfpathlineto{\pgfqpoint{4.473373in}{0.651701in}}%
\pgfpathlineto{\pgfqpoint{4.474235in}{0.675840in}}%
\pgfpathlineto{\pgfqpoint{4.476831in}{0.794665in}}%
\pgfpathlineto{\pgfqpoint{4.478560in}{0.700783in}}%
\pgfpathlineto{\pgfqpoint{4.479425in}{0.709244in}}%
\pgfpathlineto{\pgfqpoint{4.480290in}{0.653788in}}%
\pgfpathlineto{\pgfqpoint{4.482023in}{0.761078in}}%
\pgfpathlineto{\pgfqpoint{4.482888in}{0.653569in}}%
\pgfpathlineto{\pgfqpoint{4.483754in}{0.749685in}}%
\pgfpathlineto{\pgfqpoint{4.484620in}{0.703349in}}%
\pgfpathlineto{\pgfqpoint{4.485486in}{0.770052in}}%
\pgfpathlineto{\pgfqpoint{4.486350in}{0.730382in}}%
\pgfpathlineto{\pgfqpoint{4.487216in}{0.800675in}}%
\pgfpathlineto{\pgfqpoint{4.488081in}{0.680124in}}%
\pgfpathlineto{\pgfqpoint{4.488945in}{0.695033in}}%
\pgfpathlineto{\pgfqpoint{4.489808in}{0.657451in}}%
\pgfpathlineto{\pgfqpoint{4.492402in}{0.737853in}}%
\pgfpathlineto{\pgfqpoint{4.493266in}{0.654889in}}%
\pgfpathlineto{\pgfqpoint{4.496727in}{0.757343in}}%
\pgfpathlineto{\pgfqpoint{4.497591in}{0.682178in}}%
\pgfpathlineto{\pgfqpoint{4.499321in}{0.776499in}}%
\pgfpathlineto{\pgfqpoint{4.501049in}{0.699686in}}%
\pgfpathlineto{\pgfqpoint{4.501914in}{0.670200in}}%
\pgfpathlineto{\pgfqpoint{4.504511in}{0.759211in}}%
\pgfpathlineto{\pgfqpoint{4.505375in}{0.690421in}}%
\pgfpathlineto{\pgfqpoint{4.506240in}{0.716171in}}%
\pgfpathlineto{\pgfqpoint{4.507105in}{0.689799in}}%
\pgfpathlineto{\pgfqpoint{4.507970in}{0.727711in}}%
\pgfpathlineto{\pgfqpoint{4.508835in}{0.682617in}}%
\pgfpathlineto{\pgfqpoint{4.509700in}{0.716573in}}%
\pgfpathlineto{\pgfqpoint{4.510564in}{0.654998in}}%
\pgfpathlineto{\pgfqpoint{4.512294in}{0.728186in}}%
\pgfpathlineto{\pgfqpoint{4.513159in}{0.657085in}}%
\pgfpathlineto{\pgfqpoint{4.514025in}{0.677890in}}%
\pgfpathlineto{\pgfqpoint{4.514890in}{0.639577in}}%
\pgfpathlineto{\pgfqpoint{4.515753in}{0.670346in}}%
\pgfpathlineto{\pgfqpoint{4.516619in}{0.612032in}}%
\pgfpathlineto{\pgfqpoint{4.518346in}{0.712322in}}%
\pgfpathlineto{\pgfqpoint{4.519212in}{0.686060in}}%
\pgfpathlineto{\pgfqpoint{4.520076in}{0.750822in}}%
\pgfpathlineto{\pgfqpoint{4.521806in}{0.682434in}}%
\pgfpathlineto{\pgfqpoint{4.522670in}{0.685767in}}%
\pgfpathlineto{\pgfqpoint{4.523535in}{0.694741in}}%
\pgfpathlineto{\pgfqpoint{4.524400in}{0.728368in}}%
\pgfpathlineto{\pgfqpoint{4.525263in}{0.716463in}}%
\pgfpathlineto{\pgfqpoint{4.526128in}{0.669100in}}%
\pgfpathlineto{\pgfqpoint{4.527857in}{0.701042in}}%
\pgfpathlineto{\pgfqpoint{4.528722in}{0.646719in}}%
\pgfpathlineto{\pgfqpoint{4.529586in}{0.654852in}}%
\pgfpathlineto{\pgfqpoint{4.530452in}{0.663387in}}%
\pgfpathlineto{\pgfqpoint{4.532179in}{0.753900in}}%
\pgfpathlineto{\pgfqpoint{4.533043in}{0.689211in}}%
\pgfpathlineto{\pgfqpoint{4.533909in}{0.730236in}}%
\pgfpathlineto{\pgfqpoint{4.535638in}{0.664962in}}%
\pgfpathlineto{\pgfqpoint{4.537370in}{0.696170in}}%
\pgfpathlineto{\pgfqpoint{4.538236in}{0.695878in}}%
\pgfpathlineto{\pgfqpoint{4.539100in}{0.697892in}}%
\pgfpathlineto{\pgfqpoint{4.539965in}{0.734962in}}%
\pgfpathlineto{\pgfqpoint{4.541691in}{0.688845in}}%
\pgfpathlineto{\pgfqpoint{4.542555in}{0.743205in}}%
\pgfpathlineto{\pgfqpoint{4.544282in}{0.709285in}}%
\pgfpathlineto{\pgfqpoint{4.545145in}{0.723683in}}%
\pgfpathlineto{\pgfqpoint{4.546010in}{0.706467in}}%
\pgfpathlineto{\pgfqpoint{4.546874in}{0.768663in}}%
\pgfpathlineto{\pgfqpoint{4.548603in}{0.714376in}}%
\pgfpathlineto{\pgfqpoint{4.549468in}{0.682324in}}%
\pgfpathlineto{\pgfqpoint{4.551196in}{0.726354in}}%
\pgfpathlineto{\pgfqpoint{4.552060in}{0.722400in}}%
\pgfpathlineto{\pgfqpoint{4.552925in}{0.725770in}}%
\pgfpathlineto{\pgfqpoint{4.553786in}{0.744301in}}%
\pgfpathlineto{\pgfqpoint{4.554650in}{0.721299in}}%
\pgfpathlineto{\pgfqpoint{4.555516in}{0.728368in}}%
\pgfpathlineto{\pgfqpoint{4.556379in}{0.747598in}}%
\pgfpathlineto{\pgfqpoint{4.558975in}{0.665182in}}%
\pgfpathlineto{\pgfqpoint{4.559839in}{0.678881in}}%
\pgfpathlineto{\pgfqpoint{4.560705in}{0.674999in}}%
\pgfpathlineto{\pgfqpoint{4.561571in}{0.682470in}}%
\pgfpathlineto{\pgfqpoint{4.562436in}{0.721921in}}%
\pgfpathlineto{\pgfqpoint{4.563300in}{0.705289in}}%
\pgfpathlineto{\pgfqpoint{4.564162in}{0.657195in}}%
\pgfpathlineto{\pgfqpoint{4.565028in}{0.667195in}}%
\pgfpathlineto{\pgfqpoint{4.566758in}{0.696426in}}%
\pgfpathlineto{\pgfqpoint{4.567622in}{0.674301in}}%
\pgfpathlineto{\pgfqpoint{4.568487in}{0.741077in}}%
\pgfpathlineto{\pgfqpoint{4.569351in}{0.617010in}}%
\pgfpathlineto{\pgfqpoint{4.571948in}{0.751480in}}%
\pgfpathlineto{\pgfqpoint{4.573680in}{0.686023in}}%
\pgfpathlineto{\pgfqpoint{4.574545in}{0.723788in}}%
\pgfpathlineto{\pgfqpoint{4.575410in}{0.644815in}}%
\pgfpathlineto{\pgfqpoint{4.577141in}{0.746867in}}%
\pgfpathlineto{\pgfqpoint{4.578005in}{0.711226in}}%
\pgfpathlineto{\pgfqpoint{4.578869in}{0.583057in}}%
\pgfpathlineto{\pgfqpoint{4.579732in}{0.765293in}}%
\pgfpathlineto{\pgfqpoint{4.583189in}{0.584852in}}%
\pgfpathlineto{\pgfqpoint{4.584916in}{0.663825in}}%
\pgfpathlineto{\pgfqpoint{4.586645in}{0.674374in}}%
\pgfpathlineto{\pgfqpoint{4.587511in}{0.767266in}}%
\pgfpathlineto{\pgfqpoint{4.588377in}{0.662031in}}%
\pgfpathlineto{\pgfqpoint{4.590109in}{0.758001in}}%
\pgfpathlineto{\pgfqpoint{4.591841in}{0.690786in}}%
\pgfpathlineto{\pgfqpoint{4.592706in}{0.751041in}}%
\pgfpathlineto{\pgfqpoint{4.593571in}{0.670127in}}%
\pgfpathlineto{\pgfqpoint{4.594436in}{0.730821in}}%
\pgfpathlineto{\pgfqpoint{4.595300in}{0.667195in}}%
\pgfpathlineto{\pgfqpoint{4.596165in}{0.673716in}}%
\pgfpathlineto{\pgfqpoint{4.597894in}{0.684927in}}%
\pgfpathlineto{\pgfqpoint{4.598760in}{0.634266in}}%
\pgfpathlineto{\pgfqpoint{4.599625in}{0.689357in}}%
\pgfpathlineto{\pgfqpoint{4.600489in}{0.647194in}}%
\pgfpathlineto{\pgfqpoint{4.603082in}{0.732689in}}%
\pgfpathlineto{\pgfqpoint{4.603947in}{0.667301in}}%
\pgfpathlineto{\pgfqpoint{4.604812in}{0.711514in}}%
\pgfpathlineto{\pgfqpoint{4.607407in}{0.643089in}}%
\pgfpathlineto{\pgfqpoint{4.609137in}{0.657816in}}%
\pgfpathlineto{\pgfqpoint{4.610002in}{0.688585in}}%
\pgfpathlineto{\pgfqpoint{4.610868in}{0.675803in}}%
\pgfpathlineto{\pgfqpoint{4.613459in}{0.739429in}}%
\pgfpathlineto{\pgfqpoint{4.615186in}{0.612251in}}%
\pgfpathlineto{\pgfqpoint{4.616918in}{0.717231in}}%
\pgfpathlineto{\pgfqpoint{4.617784in}{0.652286in}}%
\pgfpathlineto{\pgfqpoint{4.619515in}{0.716131in}}%
\pgfpathlineto{\pgfqpoint{4.620380in}{0.676607in}}%
\pgfpathlineto{\pgfqpoint{4.621246in}{0.683238in}}%
\pgfpathlineto{\pgfqpoint{4.622110in}{0.681845in}}%
\pgfpathlineto{\pgfqpoint{4.622974in}{0.716756in}}%
\pgfpathlineto{\pgfqpoint{4.623839in}{0.596684in}}%
\pgfpathlineto{\pgfqpoint{4.626436in}{0.691627in}}%
\pgfpathlineto{\pgfqpoint{4.627301in}{0.696243in}}%
\pgfpathlineto{\pgfqpoint{4.629028in}{0.756571in}}%
\pgfpathlineto{\pgfqpoint{4.629894in}{0.735141in}}%
\pgfpathlineto{\pgfqpoint{4.630756in}{0.748991in}}%
\pgfpathlineto{\pgfqpoint{4.631622in}{0.712103in}}%
\pgfpathlineto{\pgfqpoint{4.632489in}{0.776313in}}%
\pgfpathlineto{\pgfqpoint{4.633355in}{0.633312in}}%
\pgfpathlineto{\pgfqpoint{4.634220in}{0.728547in}}%
\pgfpathlineto{\pgfqpoint{4.635081in}{0.724812in}}%
\pgfpathlineto{\pgfqpoint{4.635945in}{0.712505in}}%
\pgfpathlineto{\pgfqpoint{4.637675in}{0.805141in}}%
\pgfpathlineto{\pgfqpoint{4.640272in}{0.662798in}}%
\pgfpathlineto{\pgfqpoint{4.642003in}{0.771002in}}%
\pgfpathlineto{\pgfqpoint{4.642868in}{0.724227in}}%
\pgfpathlineto{\pgfqpoint{4.643733in}{0.748256in}}%
\pgfpathlineto{\pgfqpoint{4.645460in}{0.692357in}}%
\pgfpathlineto{\pgfqpoint{4.646326in}{0.722651in}}%
\pgfpathlineto{\pgfqpoint{4.648057in}{0.678914in}}%
\pgfpathlineto{\pgfqpoint{4.649788in}{0.705322in}}%
\pgfpathlineto{\pgfqpoint{4.650652in}{0.654812in}}%
\pgfpathlineto{\pgfqpoint{4.651516in}{0.706020in}}%
\pgfpathlineto{\pgfqpoint{4.652381in}{0.671918in}}%
\pgfpathlineto{\pgfqpoint{4.653247in}{0.699207in}}%
\pgfpathlineto{\pgfqpoint{4.654112in}{0.685435in}}%
\pgfpathlineto{\pgfqpoint{4.654976in}{0.698915in}}%
\pgfpathlineto{\pgfqpoint{4.655843in}{0.794300in}}%
\pgfpathlineto{\pgfqpoint{4.657572in}{0.703787in}}%
\pgfpathlineto{\pgfqpoint{4.658437in}{0.734264in}}%
\pgfpathlineto{\pgfqpoint{4.659302in}{0.727707in}}%
\pgfpathlineto{\pgfqpoint{4.660168in}{0.661040in}}%
\pgfpathlineto{\pgfqpoint{4.662761in}{0.792944in}}%
\pgfpathlineto{\pgfqpoint{4.664490in}{0.709281in}}%
\pgfpathlineto{\pgfqpoint{4.665354in}{0.700929in}}%
\pgfpathlineto{\pgfqpoint{4.666218in}{0.665693in}}%
\pgfpathlineto{\pgfqpoint{4.667082in}{0.722578in}}%
\pgfpathlineto{\pgfqpoint{4.667948in}{0.684923in}}%
\pgfpathlineto{\pgfqpoint{4.668812in}{0.759686in}}%
\pgfpathlineto{\pgfqpoint{4.669677in}{0.605438in}}%
\pgfpathlineto{\pgfqpoint{4.670542in}{0.711957in}}%
\pgfpathlineto{\pgfqpoint{4.672272in}{0.620234in}}%
\pgfpathlineto{\pgfqpoint{4.673137in}{0.728182in}}%
\pgfpathlineto{\pgfqpoint{4.674000in}{0.705801in}}%
\pgfpathlineto{\pgfqpoint{4.674866in}{0.702504in}}%
\pgfpathlineto{\pgfqpoint{4.675730in}{0.720930in}}%
\pgfpathlineto{\pgfqpoint{4.676595in}{0.765764in}}%
\pgfpathlineto{\pgfqpoint{4.677460in}{0.705070in}}%
\pgfpathlineto{\pgfqpoint{4.678325in}{0.723017in}}%
\pgfpathlineto{\pgfqpoint{4.680055in}{0.690672in}}%
\pgfpathlineto{\pgfqpoint{4.681784in}{0.726533in}}%
\pgfpathlineto{\pgfqpoint{4.682650in}{0.791368in}}%
\pgfpathlineto{\pgfqpoint{4.683516in}{0.663310in}}%
\pgfpathlineto{\pgfqpoint{4.685244in}{0.745324in}}%
\pgfpathlineto{\pgfqpoint{4.686110in}{0.715984in}}%
\pgfpathlineto{\pgfqpoint{4.686974in}{0.747890in}}%
\pgfpathlineto{\pgfqpoint{4.688705in}{0.684192in}}%
\pgfpathlineto{\pgfqpoint{4.689568in}{0.704153in}}%
\pgfpathlineto{\pgfqpoint{4.690432in}{0.772139in}}%
\pgfpathlineto{\pgfqpoint{4.691297in}{0.755361in}}%
\pgfpathlineto{\pgfqpoint{4.692162in}{0.719062in}}%
\pgfpathlineto{\pgfqpoint{4.693027in}{0.756425in}}%
\pgfpathlineto{\pgfqpoint{4.693889in}{0.706280in}}%
\pgfpathlineto{\pgfqpoint{4.695621in}{0.745584in}}%
\pgfpathlineto{\pgfqpoint{4.697352in}{0.727597in}}%
\pgfpathlineto{\pgfqpoint{4.699083in}{0.671077in}}%
\pgfpathlineto{\pgfqpoint{4.699949in}{0.739356in}}%
\pgfpathlineto{\pgfqpoint{4.701675in}{0.685621in}}%
\pgfpathlineto{\pgfqpoint{4.702540in}{0.710235in}}%
\pgfpathlineto{\pgfqpoint{4.703406in}{0.618845in}}%
\pgfpathlineto{\pgfqpoint{4.704271in}{0.634632in}}%
\pgfpathlineto{\pgfqpoint{4.705135in}{0.649687in}}%
\pgfpathlineto{\pgfqpoint{4.706864in}{0.724633in}}%
\pgfpathlineto{\pgfqpoint{4.707730in}{0.707490in}}%
\pgfpathlineto{\pgfqpoint{4.708594in}{0.714596in}}%
\pgfpathlineto{\pgfqpoint{4.709461in}{0.694818in}}%
\pgfpathlineto{\pgfqpoint{4.710325in}{0.733095in}}%
\pgfpathlineto{\pgfqpoint{4.712056in}{0.652838in}}%
\pgfpathlineto{\pgfqpoint{4.713785in}{0.750164in}}%
\pgfpathlineto{\pgfqpoint{4.715513in}{0.657012in}}%
\pgfpathlineto{\pgfqpoint{4.716380in}{0.707234in}}%
\pgfpathlineto{\pgfqpoint{4.717245in}{0.613096in}}%
\pgfpathlineto{\pgfqpoint{4.718111in}{0.691557in}}%
\pgfpathlineto{\pgfqpoint{4.719839in}{0.655071in}}%
\pgfpathlineto{\pgfqpoint{4.720705in}{0.722400in}}%
\pgfpathlineto{\pgfqpoint{4.721569in}{0.652655in}}%
\pgfpathlineto{\pgfqpoint{4.723300in}{0.752032in}}%
\pgfpathlineto{\pgfqpoint{4.725030in}{0.713240in}}%
\pgfpathlineto{\pgfqpoint{4.725894in}{0.719030in}}%
\pgfpathlineto{\pgfqpoint{4.726759in}{0.718185in}}%
\pgfpathlineto{\pgfqpoint{4.729353in}{0.606867in}}%
\pgfpathlineto{\pgfqpoint{4.730215in}{0.724706in}}%
\pgfpathlineto{\pgfqpoint{4.731080in}{0.682434in}}%
\pgfpathlineto{\pgfqpoint{4.731947in}{0.692873in}}%
\pgfpathlineto{\pgfqpoint{4.732812in}{0.756133in}}%
\pgfpathlineto{\pgfqpoint{4.733678in}{0.739063in}}%
\pgfpathlineto{\pgfqpoint{4.734543in}{0.691663in}}%
\pgfpathlineto{\pgfqpoint{4.735409in}{0.703641in}}%
\pgfpathlineto{\pgfqpoint{4.736271in}{0.729976in}}%
\pgfpathlineto{\pgfqpoint{4.737133in}{0.720784in}}%
\pgfpathlineto{\pgfqpoint{4.737997in}{0.696462in}}%
\pgfpathlineto{\pgfqpoint{4.739729in}{0.740858in}}%
\pgfpathlineto{\pgfqpoint{4.741460in}{0.698111in}}%
\pgfpathlineto{\pgfqpoint{4.742327in}{0.779870in}}%
\pgfpathlineto{\pgfqpoint{4.743193in}{0.719208in}}%
\pgfpathlineto{\pgfqpoint{4.744060in}{0.754923in}}%
\pgfpathlineto{\pgfqpoint{4.746655in}{0.660049in}}%
\pgfpathlineto{\pgfqpoint{4.747518in}{0.667451in}}%
\pgfpathlineto{\pgfqpoint{4.748384in}{0.718291in}}%
\pgfpathlineto{\pgfqpoint{4.749246in}{0.715911in}}%
\pgfpathlineto{\pgfqpoint{4.750112in}{0.716902in}}%
\pgfpathlineto{\pgfqpoint{4.752708in}{0.762800in}}%
\pgfpathlineto{\pgfqpoint{4.754440in}{0.694814in}}%
\pgfpathlineto{\pgfqpoint{4.755305in}{0.745803in}}%
\pgfpathlineto{\pgfqpoint{4.757035in}{0.659505in}}%
\pgfpathlineto{\pgfqpoint{4.758765in}{0.757343in}}%
\pgfpathlineto{\pgfqpoint{4.760492in}{0.780747in}}%
\pgfpathlineto{\pgfqpoint{4.761358in}{0.704778in}}%
\pgfpathlineto{\pgfqpoint{4.763954in}{0.829978in}}%
\pgfpathlineto{\pgfqpoint{4.766549in}{0.688037in}}%
\pgfpathlineto{\pgfqpoint{4.767415in}{0.744593in}}%
\pgfpathlineto{\pgfqpoint{4.770012in}{0.682617in}}%
\pgfpathlineto{\pgfqpoint{4.770878in}{0.769134in}}%
\pgfpathlineto{\pgfqpoint{4.771744in}{0.715034in}}%
\pgfpathlineto{\pgfqpoint{4.772610in}{0.717048in}}%
\pgfpathlineto{\pgfqpoint{4.773477in}{0.729063in}}%
\pgfpathlineto{\pgfqpoint{4.774342in}{0.690640in}}%
\pgfpathlineto{\pgfqpoint{4.775208in}{0.744447in}}%
\pgfpathlineto{\pgfqpoint{4.776073in}{0.691444in}}%
\pgfpathlineto{\pgfqpoint{4.777802in}{0.765106in}}%
\pgfpathlineto{\pgfqpoint{4.779534in}{0.682690in}}%
\pgfpathlineto{\pgfqpoint{4.780401in}{0.721847in}}%
\pgfpathlineto{\pgfqpoint{4.781267in}{0.637929in}}%
\pgfpathlineto{\pgfqpoint{4.782998in}{0.800967in}}%
\pgfpathlineto{\pgfqpoint{4.783865in}{0.735766in}}%
\pgfpathlineto{\pgfqpoint{4.784730in}{0.735986in}}%
\pgfpathlineto{\pgfqpoint{4.785596in}{0.732360in}}%
\pgfpathlineto{\pgfqpoint{4.786461in}{0.795875in}}%
\pgfpathlineto{\pgfqpoint{4.787327in}{0.751187in}}%
\pgfpathlineto{\pgfqpoint{4.788193in}{0.795583in}}%
\pgfpathlineto{\pgfqpoint{4.789059in}{0.704778in}}%
\pgfpathlineto{\pgfqpoint{4.789925in}{0.742799in}}%
\pgfpathlineto{\pgfqpoint{4.792520in}{0.696974in}}%
\pgfpathlineto{\pgfqpoint{4.793384in}{0.776719in}}%
\pgfpathlineto{\pgfqpoint{4.795115in}{0.689503in}}%
\pgfpathlineto{\pgfqpoint{4.795980in}{0.753055in}}%
\pgfpathlineto{\pgfqpoint{4.796845in}{0.721076in}}%
\pgfpathlineto{\pgfqpoint{4.798574in}{0.788843in}}%
\pgfpathlineto{\pgfqpoint{4.801166in}{0.717633in}}%
\pgfpathlineto{\pgfqpoint{4.802893in}{0.681553in}}%
\pgfpathlineto{\pgfqpoint{4.804623in}{0.715692in}}%
\pgfpathlineto{\pgfqpoint{4.805486in}{0.705801in}}%
\pgfpathlineto{\pgfqpoint{4.806350in}{0.681845in}}%
\pgfpathlineto{\pgfqpoint{4.808078in}{0.821845in}}%
\pgfpathlineto{\pgfqpoint{4.809809in}{0.668365in}}%
\pgfpathlineto{\pgfqpoint{4.810674in}{0.799392in}}%
\pgfpathlineto{\pgfqpoint{4.811537in}{0.732944in}}%
\pgfpathlineto{\pgfqpoint{4.812399in}{0.772066in}}%
\pgfpathlineto{\pgfqpoint{4.814129in}{0.665693in}}%
\pgfpathlineto{\pgfqpoint{4.815858in}{0.760088in}}%
\pgfpathlineto{\pgfqpoint{4.816722in}{0.747562in}}%
\pgfpathlineto{\pgfqpoint{4.817588in}{0.734264in}}%
\pgfpathlineto{\pgfqpoint{4.818454in}{0.675219in}}%
\pgfpathlineto{\pgfqpoint{4.821048in}{0.732506in}}%
\pgfpathlineto{\pgfqpoint{4.821914in}{0.672580in}}%
\pgfpathlineto{\pgfqpoint{4.823647in}{0.715984in}}%
\pgfpathlineto{\pgfqpoint{4.824513in}{0.682064in}}%
\pgfpathlineto{\pgfqpoint{4.825379in}{0.698915in}}%
\pgfpathlineto{\pgfqpoint{4.826244in}{0.750672in}}%
\pgfpathlineto{\pgfqpoint{4.827110in}{0.723935in}}%
\pgfpathlineto{\pgfqpoint{4.827976in}{0.640747in}}%
\pgfpathlineto{\pgfqpoint{4.828842in}{0.669063in}}%
\pgfpathlineto{\pgfqpoint{4.829709in}{0.649760in}}%
\pgfpathlineto{\pgfqpoint{4.831440in}{0.691078in}}%
\pgfpathlineto{\pgfqpoint{4.832305in}{0.693312in}}%
\pgfpathlineto{\pgfqpoint{4.833170in}{0.643897in}}%
\pgfpathlineto{\pgfqpoint{4.834033in}{0.695252in}}%
\pgfpathlineto{\pgfqpoint{4.834897in}{0.650491in}}%
\pgfpathlineto{\pgfqpoint{4.835762in}{0.656022in}}%
\pgfpathlineto{\pgfqpoint{4.837494in}{0.745105in}}%
\pgfpathlineto{\pgfqpoint{4.839223in}{0.638733in}}%
\pgfpathlineto{\pgfqpoint{4.840088in}{0.720930in}}%
\pgfpathlineto{\pgfqpoint{4.840953in}{0.710381in}}%
\pgfpathlineto{\pgfqpoint{4.841818in}{0.762321in}}%
\pgfpathlineto{\pgfqpoint{4.842682in}{0.699098in}}%
\pgfpathlineto{\pgfqpoint{4.842682in}{0.699098in}}%
\pgfusepath{stroke}%
\end{pgfscope}%
\begin{pgfscope}%
\pgfsetrectcap%
\pgfsetmiterjoin%
\pgfsetlinewidth{0.803000pt}%
\definecolor{currentstroke}{rgb}{0.000000,0.000000,0.000000}%
\pgfsetstrokecolor{currentstroke}%
\pgfsetdash{}{0pt}%
\pgfpathmoveto{\pgfqpoint{0.483776in}{0.538014in}}%
\pgfpathlineto{\pgfqpoint{0.483776in}{1.122895in}}%
\pgfusepath{stroke}%
\end{pgfscope}%
\begin{pgfscope}%
\pgfsetrectcap%
\pgfsetmiterjoin%
\pgfsetlinewidth{0.803000pt}%
\definecolor{currentstroke}{rgb}{0.000000,0.000000,0.000000}%
\pgfsetstrokecolor{currentstroke}%
\pgfsetdash{}{0pt}%
\pgfpathmoveto{\pgfqpoint{5.050249in}{0.538014in}}%
\pgfpathlineto{\pgfqpoint{5.050249in}{1.122895in}}%
\pgfusepath{stroke}%
\end{pgfscope}%
\begin{pgfscope}%
\pgfsetrectcap%
\pgfsetmiterjoin%
\pgfsetlinewidth{0.803000pt}%
\definecolor{currentstroke}{rgb}{0.000000,0.000000,0.000000}%
\pgfsetstrokecolor{currentstroke}%
\pgfsetdash{}{0pt}%
\pgfpathmoveto{\pgfqpoint{0.483776in}{0.538014in}}%
\pgfpathlineto{\pgfqpoint{5.050249in}{0.538014in}}%
\pgfusepath{stroke}%
\end{pgfscope}%
\begin{pgfscope}%
\pgfsetrectcap%
\pgfsetmiterjoin%
\pgfsetlinewidth{0.803000pt}%
\definecolor{currentstroke}{rgb}{0.000000,0.000000,0.000000}%
\pgfsetstrokecolor{currentstroke}%
\pgfsetdash{}{0pt}%
\pgfpathmoveto{\pgfqpoint{0.483776in}{1.122895in}}%
\pgfpathlineto{\pgfqpoint{5.050249in}{1.122895in}}%
\pgfusepath{stroke}%
\end{pgfscope}%
\end{pgfpicture}%
\makeatother%
\endgroup%

    \includegraphics[width=0.75\textwidth]{example-image-golden}
    \caption{Voltage noise of an LM399, measured with a bandwidth of \qtyrange{0.1}{10}{\Hz}.}
    \label{fig:noise_lm399}
\end{figure}

Measuring two references against each other would then result in around \qty{2.1}{\micro \volt} of noise. This make distinguishing the jumps possible, but challenging.

A third option is to use a high-pass filter and an amplifier. Additionally, the signal can be low-pass filtered to remove any excess high frequency noise. This approach also requires less resolution than directly measuring the voltage, because the signal-to-noise-ratio is improved du the amplifier. It is therefore possible to use an off-the-shelf analog-to-digital converter (ADC). One such circuit, along with some examples, is demonstrated in \cite{technote_ti_popcorn_noise,kay2012operational}. It must be noted, that due to the high-pass filtering, it not possible to measure slow voltage drifts using this method.

The forth and final option presented here, is approaching the problem in the frequency domain and requires a low-noise amplifier with a low frequency cutoff. As it was already discussed in section \ref{}, popcorn noise is found to have a frequency dependence of $1/f^2$. This can be used to distinguish it from other random noise processes that show a frequency dependence of $1/f$. An good example is given in Horowitz and Hill in \textit{The Art of Electronics} on page 478 \cite{horowitz1989}. Going to frequencies below \qty{10}{\Hz}, one can sort the references by their noise spectrum.

In this work only options one and two were were tested, as it was said above, with options three and four there is a chicken and egg problem. One needs a number on known good devices to compare other DUTs to. At the start of the evaluation, most of the data available about the LM399 was from the data sheet. Compiling a dataset of the performance of dozens of LM399 is expensive and time consuming and companies typically treat such data as a closely guarded secret.

The next section deals with the choice of multimeter to satisfy the requirements test the Zener diodes according to options one and two, so either directly measuring the output voltage or difference of a known good sample against the DUT.

\subsection{Choosing a Multimeter for Testing Zener Diodes}
The DMM used plays an important role for the test setup. In this section, some of the challenges, that can be encountered will be discussed. The expected amplitude of the popcorn noise is around \qty[per-mode=symbol]{0.5}{\micro\volt \per \volt} or \qty{3.5}{\micro\volt} of the output voltage, when considering the \qty{7}{\volt} Zener voltage of the LM399 diode.

The \qty{7}{\volt} will typically be measured on the \qty{10}{\volt} range. It is not a trivial task, because a signal-to-noise-ratio of \qty[per-mode=symbol]{0.35}{\micro\volt \per \volt} or more than \qty{130}{\decibel} is required. This calls for a device, that not only has the required resolution, but also the stability over time and temperature to ensure the measurement will not be distorted by the DMM.

Therefore, a voltmeter with lower noise and a more stable reference, than the DUT is mandatory. This only leaves the class of very low noise \num{7.5} or \num{8.5} digit multimeters. These multimeters feature a different type of voltage reference, because the LM399 is not suitable due to its noise. The only Zener diodes that meet those requirements are the Analog Devices LTZ1000 \cite{datasheet_LTZ1000}, the Motorola SZA263 (out of production) and the Linear Technology (LT) LTFLU-1, a proprietary design by Fluke and LT. The LTZ1000, for example, is specified for a typical noise of \qty{1.2}{\micro\volt_{pp}} in a frequency range of \qtyrange{0.1}{10}{\Hz} \cite{datasheet_LTZ1000}. Additionally, in comparison to the LM399, those Zener diodes do not suffer from the popcorn noise issue.

The equipment manufacturers typically have a preference for one of those diodes. Keysight utilizes the LTZ1000, Fluke uses the SZA263 (in older devices) or the LTFLU-1 in newer model, while Keithley employs the LTZ1000 in their \device{Model 2002} and the LTFLU-1 in the newer \device{DMM7510}, because they were bought by Fortive, the same company that owns Fluke. To sum it it up, Keysight uses the LTZ1000 and Fluke/Keithley the LTFLU-1 in their top end meters.

Comparing only \num{7.5} and \num{8.5} digit voltmeters, narrows down the choice of multimeters considerably. The market for high-end \num{8.5} digit DMMs is limited and therefore every device on the market caters for a certain niche. It is therefore prudent to look at their specifications to choose the correct device for this purpose. In table \ref{tab:list_of_dmms} a list of popular \num{8.5} DMMs can be found. Several models included in the table, are already discontinued, but these DMMs can still be acquired on the second-hand market.

\begin{table}[h]
    \centering
    \begin{tabular}{ |l|l|l| }
        \hline
        Manufacturer & Model & Remarks \\
        \hline
        Advantest & \device{R6581} & Discontinued. Scanner cards available. \\
        Datron/Wavetek & \device{1812} & Discontinued. Wavetek was bought by Fluke. \\
        Fluke & \device{8508A} & Discontinued. \qty{20}{\volt} range. \\
        Fluke & \device{8588A} & In production. \\
        Keithley/Tektronix & \device{2002} & In production. Scanner card available. \qty{20}{\volt} range. \\
        Keysight & \device{3458A} & In Production. \\
        Solartron & \device{7081} & Discontinued. Slow. \\
        Transmille & \device{8104} & In Production. External scanner available. Slow. \\
        \hline
    \end{tabular}
    \caption{Overview of \num{8.5} digit multimeters.}
    \label{tab:list_of_dmms}
\end{table}

While the author has not tested every multimeter in table \ref{tab:list_of_dmms}, it is possible to judge some of them apriori by their specifications. The \device{Solartron 7081} (also sold as \device{Guildline 9578}) is a less optimal choice, because a conversion takes \qty{52}{\s} for \num{8.5} digits. The discontinued \device{Fluke 8508A} and the \device{Wavetek 1812} multimeter are very similar devices, because Fluke bought Wavetek in 2000 and as a result, the \device{Fluke 8508A} is more of an update to the \device{Wavetek 1812} than a new device. They are both in included in the list, because it is very rare to see one of the Fluke devices on the second hand market, while the \device{Wavetek 1812} can be found with a bit of patience. Again they are fairly slow, taking \qty{25}{\second} for a conversion at \num{8.5} digits.

The other multimeters are still in production and similar in price, but their field of use is slightly is different. The \device{Fluke 8588A} excels at stability and features a modern user interface, whereas the \device{Keysight 3458A} is unbeaten in linearity and noise. A detailed comparison of those two meters can be found in the work of \citeauthor*{article_fluke_8588A_noise} \cite{article_fluke_8588A_noise}. The \device{Keithley Model 2002} focuses on its scanning capability and the \device{Transmille 8104} does have electrometer functions. Unfortunately, the \device{8104} is also fairly slow at \num{8.5} digit with conversions taking \qty{4}{\s} at its fastest setting \cite{datasheet_transmille8104}, so it will not be considered.

To narrow it down even further, several \num{7.5} and \num{8.5} digit multimeters were tested. The results of those tests will be discussed here to give an impression of the performance of these devices. The tested multimeters are the \device{Keysight 3458A}, the \device{Keithley Model 2002}, the \device{Keysight 34470A} and a \device{Keithley DMM6500}. The \device{3458A} was chosen, because it is very fast and already used in section \ref{} of this work. The \device{Model 2002} was chosen for its internal scanning unit. The \device{34470A} was chosen as a lower-end and cheaper alternative and because it is a fairly low noise device. Finally the \device{DMM6500} is on the list to compare a DMM with an LM399 reference. A \device{Fluke 8588A} was not tested, because it was not released at the time of testing and the older model \device{8508A} is considered too slow as mentioned above.

\minisec{The tests}

Two test were run on this selection of devices. The first one was done using a \device{Fluke 5440B} calibrator supplying \qty{10}{\volt} to all mulimeters and taking readings over the course of a week. This data was used to estimate the noise and the stability of the multimeters, including burst noise. The noise of the DMM at \qty{10}{\volt} is typically not found in the datasheet, because the noise performance is usually quoted for shorted inputs, which does not include the internal reference noise. This test allows to check for popcorn noise of the internal reference. The calibrator has a specified output noise of \qty{< 1.5}{\micro \volt} within a bandwidth of \qtyrange{0.1}{10}{\Hz} at \qty{1}{\volt} and is stable to within \qty{5}{\micro \volt_{RMS}} over \qty{30}{\day}, a specification far superior to the LM399.

The second test was done using a known bad LM399 voltage reference instead of the calibrator. This test was done to see how well a DMM can make out the popcorn noise.

Based on these two tests, a multimeter was chosen for an automated test setup to bin the LM399s.

\minisec{Test Setup}

The tests were done in a stable and monitored lab environment, with a temperature deviation of at most $\Delta T = \qty{\pm 0.2}{\kelvin}$. All multimeters were connected to the same DUT. Although this might potentially cause interference between the multimeters due to the pump out current spikes caused by the switching interals, no ill effects, like voltage offsets or increased noise, were observed during the setup of the tests. A more detailed discussion of the pump out current of the \device{3458A} can be found in \cite{article_3458A_input_mpedance}.

The three \num{8.5} and \num{7.5} digit multimeters were connected using shielded cables, either Pomona 1167-60 or self-made cables. See section \ref{} for details on the self-made cables. The GUARD terminal of the calibrator was connected to chassis GROUND at the calibrator and then connected to the cable shield. On the \device{3458A}, the shield was connected to the GUARD terminal and the GUARD switch was set to open according to the manual \cite{manual_keysight3458a}. For the other multimeters, that do not have a GUARD terminal, the shield was left floating at the DMM side. Additionally the \device{Fluke 5440B}, the \device{HP 3458A} and the \device{Keysight 34470A} have an autocalibration routine, which was run once prior to the measurement. The detailed settings used for the DMMs can be found in the appendix \ref{appendix:dmm_test} on page \pageref{appendix:dmm_test}, a summary ist given in table \ref{tab:dmm_settings_concise} to show the important differences.

\begin{table}[ht]
    \centering
    \begin{tabular}{lcc}
        \toprule
        DMM& Integration time in \unit{NPLC}& Conversion time in \unit{\s}\\
        \midrule
        \device{HP 3458A}& 100 & \qty{0}{\s}\\
        \device{Keithley Model 2002} & 40& \qty{0}{\s}\\
        \device{Keysight 34470A}& 100    & \qty{0}{\s}\\
        \device{Keithley DMM6500}& 90& \qty{0}{\s}\\
        \bottomrule
    \end{tabular}
    \caption{Concise list of differences in the settings used for comparing the DMMs.}
    \label{tab:dmm_settings_concise}
\end{table}

All DMMs were configured to have a similar conversion time. This lead to different integration times, which are given in power line cycles at \qty{50}{\Hz}. The \device{Model 2002} takes considerable longer for a measurement than the Keysight multimeters. The reason is the auto-zero function, which is shown in figure \ref{dmm_autozero_comparison}. The \device{Model 2002} does three steps when doing auto-zeroing, it measures the signal, the zero point for an offset compensation and also the reference voltage for a gain correction. In comparison, the \device{3458A} only corrects for the offset drift. The gain is adjusted when using the ACAL function. The former auto-zero routine, therefore takes longer by one half, but results in more stable measurements.

\begin{figure}[ht]
    \centering
    %\resizebox {0.8\textwidth} {!} {
        \import{figures/}{dmm_autozero.tex}
    %} % resizebox
    \label{fig:dmm_autozero_comparison}
    \caption{Auto-zero phases of the \device{HP 3458A} and \device{Keithley Model 2002}.}
\end{figure}

These measurements were done by measuring the output voltage of a pre-production version of the reference PCB for the digital current driver. The reference board was kept at \qty{23}{\celsius} in a custom thermal chamber. The chamber is detailed in section \ref{}. Additionally, a \qty{500}{\g}  bag of Bentonite desiccant was added to keep the references at a low humidity of around \qty{20}{\percent} relative humidity. The reference board inserted into a motherboard holding up to 4 reference modules. The motherboard, also called LM399 breakout board, provides the voltage regulators and the operational amplifier for the kelvin sensed pins of the reference. The multimeter was directly connected to the reference via a DB9 connector, without an other components in between the reference and the DMM like buffers, multiplexers or filters. The DMM itself was exposed to the ambient temperature of the lab. The setup is shown in figure \ref{fig:lm399_vs_34470a_setup}.

\begin{figure}[ht]
    \centering
    \resizebox {0.8\textwidth} {!} {
        \import{figures/}{34470A_vs_LM399.tex}
    } % resizebox
    \caption{Measurement setup for tesing an LM399 reference board with the \device{Keysight 34470A}}
    \label{fig:lm399_vs_34470a_setup}
\end{figure}

The reference boards amplify the Zener voltage to \qty{10}{\volt}, which improves the signal to noise ratio, because it makes use of the full DMM range. The \qty{10}{\volt} range is typically the lowest (relative) noise and lowest drift range those multimeter because no internal pre-amplifiers of attenuators are required. It is important to keep the temperature drift of the DMM low or at least predictable, because the device is exposed to the ambient laboratory and not in a temperature controlled environment like the references.

\newpage
The reference is a negative \qty{10}{\volt} reference that uses a self-biasing technique to derive its \qty{1}{\mA} Zener current from its own \qty{-10}{\volt} output. The details of this circuit are discussed in section \ref{}.

% \begin{figure}[h]
%     \centering
%     \scalebox{0.7}{%
%         \import{figures/}{lm399_reference_circuit.tex}
%     } % scalebox
%     \caption{Self-biased LM399 negative voltage reference.}
%     \label{fig:lm399_negative_10V}
% \end{figure}

With the amplified output the expected burst noise step size of about \qty[per-mode=symbol]{0.5}{\micro\volt \per \volt}, becomes \qty{5}{\micro\volt}. The resolution of the \qty{10}{\volt} range of the \device{34470A} is \qty{100}{\nano \volt}, but the measurement is not limited by quantization. See section \ref{} of this work for a detailed characterization.

\begin{figure}[ht]
    \centering
    %% Creator: Matplotlib, PGF backend
%%
%% To include the figure in your LaTeX document, write
%%   \input{<filename>.pgf}
%%
%% Make sure the required packages are loaded in your preamble
%%   \usepackage{pgf}
%%
%% Also ensure that all the required font packages are loaded; for instance,
%% the lmodern package is sometimes necessary when using math font.
%%   \usepackage{lmodern}
%%
%% Figures using additional raster images can only be included by \input if
%% they are in the same directory as the main LaTeX file. For loading figures
%% from other directories you can use the `import` package
%%   \usepackage{import}
%%
%% and then include the figures with
%%   \import{<path to file>}{<filename>.pgf}
%%
%% Matplotlib used the following preamble
%%   \usepackage{siunitx}
%%   \usepackage{fontspec}
%%
\begingroup%
\makeatletter%
\begin{pgfpicture}%
\pgfpathrectangle{\pgfpointorigin}{\pgfqpoint{5.208662in}{3.219130in}}%
\pgfusepath{use as bounding box, clip}%
\begin{pgfscope}%
\pgfsetbuttcap%
\pgfsetmiterjoin%
\definecolor{currentfill}{rgb}{1.000000,1.000000,1.000000}%
\pgfsetfillcolor{currentfill}%
\pgfsetlinewidth{0.000000pt}%
\definecolor{currentstroke}{rgb}{1.000000,1.000000,1.000000}%
\pgfsetstrokecolor{currentstroke}%
\pgfsetdash{}{0pt}%
\pgfpathmoveto{\pgfqpoint{0.000000in}{0.000000in}}%
\pgfpathlineto{\pgfqpoint{5.208662in}{0.000000in}}%
\pgfpathlineto{\pgfqpoint{5.208662in}{3.219130in}}%
\pgfpathlineto{\pgfqpoint{0.000000in}{3.219130in}}%
\pgfpathlineto{\pgfqpoint{0.000000in}{0.000000in}}%
\pgfpathclose%
\pgfusepath{fill}%
\end{pgfscope}%
\begin{pgfscope}%
\pgfsetbuttcap%
\pgfsetmiterjoin%
\definecolor{currentfill}{rgb}{1.000000,1.000000,1.000000}%
\pgfsetfillcolor{currentfill}%
\pgfsetlinewidth{0.000000pt}%
\definecolor{currentstroke}{rgb}{0.000000,0.000000,0.000000}%
\pgfsetstrokecolor{currentstroke}%
\pgfsetstrokeopacity{0.000000}%
\pgfsetdash{}{0pt}%
\pgfpathmoveto{\pgfqpoint{0.634869in}{0.539544in}}%
\pgfpathlineto{\pgfqpoint{4.514985in}{0.539544in}}%
\pgfpathlineto{\pgfqpoint{4.514985in}{2.944887in}}%
\pgfpathlineto{\pgfqpoint{0.634869in}{2.944887in}}%
\pgfpathlineto{\pgfqpoint{0.634869in}{0.539544in}}%
\pgfpathclose%
\pgfusepath{fill}%
\end{pgfscope}%
\begin{pgfscope}%
\pgfsetbuttcap%
\pgfsetroundjoin%
\definecolor{currentfill}{rgb}{0.000000,0.000000,0.000000}%
\pgfsetfillcolor{currentfill}%
\pgfsetlinewidth{0.803000pt}%
\definecolor{currentstroke}{rgb}{0.000000,0.000000,0.000000}%
\pgfsetstrokecolor{currentstroke}%
\pgfsetdash{}{0pt}%
\pgfsys@defobject{currentmarker}{\pgfqpoint{0.000000in}{-0.048611in}}{\pgfqpoint{0.000000in}{0.000000in}}{%
\pgfpathmoveto{\pgfqpoint{0.000000in}{0.000000in}}%
\pgfpathlineto{\pgfqpoint{0.000000in}{-0.048611in}}%
\pgfusepath{stroke,fill}%
}%
\begin{pgfscope}%
\pgfsys@transformshift{0.811157in}{0.539544in}%
\pgfsys@useobject{currentmarker}{}%
\end{pgfscope}%
\end{pgfscope}%
\begin{pgfscope}%
\definecolor{textcolor}{rgb}{0.000000,0.000000,0.000000}%
\pgfsetstrokecolor{textcolor}%
\pgfsetfillcolor{textcolor}%
\pgftext[x=0.811157in,y=0.442322in,,top]{\color{textcolor}\rmfamily\fontsize{8.000000}{9.600000}\selectfont \(\displaystyle {00{:}00}\)}%
\end{pgfscope}%
\begin{pgfscope}%
\pgfsetbuttcap%
\pgfsetroundjoin%
\definecolor{currentfill}{rgb}{0.000000,0.000000,0.000000}%
\pgfsetfillcolor{currentfill}%
\pgfsetlinewidth{0.803000pt}%
\definecolor{currentstroke}{rgb}{0.000000,0.000000,0.000000}%
\pgfsetstrokecolor{currentstroke}%
\pgfsetdash{}{0pt}%
\pgfsys@defobject{currentmarker}{\pgfqpoint{0.000000in}{-0.048611in}}{\pgfqpoint{0.000000in}{0.000000in}}{%
\pgfpathmoveto{\pgfqpoint{0.000000in}{0.000000in}}%
\pgfpathlineto{\pgfqpoint{0.000000in}{-0.048611in}}%
\pgfusepath{stroke,fill}%
}%
\begin{pgfscope}%
\pgfsys@transformshift{1.252105in}{0.539544in}%
\pgfsys@useobject{currentmarker}{}%
\end{pgfscope}%
\end{pgfscope}%
\begin{pgfscope}%
\definecolor{textcolor}{rgb}{0.000000,0.000000,0.000000}%
\pgfsetstrokecolor{textcolor}%
\pgfsetfillcolor{textcolor}%
\pgftext[x=1.252105in,y=0.442322in,,top]{\color{textcolor}\rmfamily\fontsize{8.000000}{9.600000}\selectfont \(\displaystyle {03{:}00}\)}%
\end{pgfscope}%
\begin{pgfscope}%
\pgfsetbuttcap%
\pgfsetroundjoin%
\definecolor{currentfill}{rgb}{0.000000,0.000000,0.000000}%
\pgfsetfillcolor{currentfill}%
\pgfsetlinewidth{0.803000pt}%
\definecolor{currentstroke}{rgb}{0.000000,0.000000,0.000000}%
\pgfsetstrokecolor{currentstroke}%
\pgfsetdash{}{0pt}%
\pgfsys@defobject{currentmarker}{\pgfqpoint{0.000000in}{-0.048611in}}{\pgfqpoint{0.000000in}{0.000000in}}{%
\pgfpathmoveto{\pgfqpoint{0.000000in}{0.000000in}}%
\pgfpathlineto{\pgfqpoint{0.000000in}{-0.048611in}}%
\pgfusepath{stroke,fill}%
}%
\begin{pgfscope}%
\pgfsys@transformshift{1.693053in}{0.539544in}%
\pgfsys@useobject{currentmarker}{}%
\end{pgfscope}%
\end{pgfscope}%
\begin{pgfscope}%
\definecolor{textcolor}{rgb}{0.000000,0.000000,0.000000}%
\pgfsetstrokecolor{textcolor}%
\pgfsetfillcolor{textcolor}%
\pgftext[x=1.693053in,y=0.442322in,,top]{\color{textcolor}\rmfamily\fontsize{8.000000}{9.600000}\selectfont \(\displaystyle {06{:}00}\)}%
\end{pgfscope}%
\begin{pgfscope}%
\pgfsetbuttcap%
\pgfsetroundjoin%
\definecolor{currentfill}{rgb}{0.000000,0.000000,0.000000}%
\pgfsetfillcolor{currentfill}%
\pgfsetlinewidth{0.803000pt}%
\definecolor{currentstroke}{rgb}{0.000000,0.000000,0.000000}%
\pgfsetstrokecolor{currentstroke}%
\pgfsetdash{}{0pt}%
\pgfsys@defobject{currentmarker}{\pgfqpoint{0.000000in}{-0.048611in}}{\pgfqpoint{0.000000in}{0.000000in}}{%
\pgfpathmoveto{\pgfqpoint{0.000000in}{0.000000in}}%
\pgfpathlineto{\pgfqpoint{0.000000in}{-0.048611in}}%
\pgfusepath{stroke,fill}%
}%
\begin{pgfscope}%
\pgfsys@transformshift{2.134001in}{0.539544in}%
\pgfsys@useobject{currentmarker}{}%
\end{pgfscope}%
\end{pgfscope}%
\begin{pgfscope}%
\definecolor{textcolor}{rgb}{0.000000,0.000000,0.000000}%
\pgfsetstrokecolor{textcolor}%
\pgfsetfillcolor{textcolor}%
\pgftext[x=2.134001in,y=0.442322in,,top]{\color{textcolor}\rmfamily\fontsize{8.000000}{9.600000}\selectfont \(\displaystyle {09{:}00}\)}%
\end{pgfscope}%
\begin{pgfscope}%
\pgfsetbuttcap%
\pgfsetroundjoin%
\definecolor{currentfill}{rgb}{0.000000,0.000000,0.000000}%
\pgfsetfillcolor{currentfill}%
\pgfsetlinewidth{0.803000pt}%
\definecolor{currentstroke}{rgb}{0.000000,0.000000,0.000000}%
\pgfsetstrokecolor{currentstroke}%
\pgfsetdash{}{0pt}%
\pgfsys@defobject{currentmarker}{\pgfqpoint{0.000000in}{-0.048611in}}{\pgfqpoint{0.000000in}{0.000000in}}{%
\pgfpathmoveto{\pgfqpoint{0.000000in}{0.000000in}}%
\pgfpathlineto{\pgfqpoint{0.000000in}{-0.048611in}}%
\pgfusepath{stroke,fill}%
}%
\begin{pgfscope}%
\pgfsys@transformshift{2.574949in}{0.539544in}%
\pgfsys@useobject{currentmarker}{}%
\end{pgfscope}%
\end{pgfscope}%
\begin{pgfscope}%
\definecolor{textcolor}{rgb}{0.000000,0.000000,0.000000}%
\pgfsetstrokecolor{textcolor}%
\pgfsetfillcolor{textcolor}%
\pgftext[x=2.574949in,y=0.442322in,,top]{\color{textcolor}\rmfamily\fontsize{8.000000}{9.600000}\selectfont \(\displaystyle {12{:}00}\)}%
\end{pgfscope}%
\begin{pgfscope}%
\pgfsetbuttcap%
\pgfsetroundjoin%
\definecolor{currentfill}{rgb}{0.000000,0.000000,0.000000}%
\pgfsetfillcolor{currentfill}%
\pgfsetlinewidth{0.803000pt}%
\definecolor{currentstroke}{rgb}{0.000000,0.000000,0.000000}%
\pgfsetstrokecolor{currentstroke}%
\pgfsetdash{}{0pt}%
\pgfsys@defobject{currentmarker}{\pgfqpoint{0.000000in}{-0.048611in}}{\pgfqpoint{0.000000in}{0.000000in}}{%
\pgfpathmoveto{\pgfqpoint{0.000000in}{0.000000in}}%
\pgfpathlineto{\pgfqpoint{0.000000in}{-0.048611in}}%
\pgfusepath{stroke,fill}%
}%
\begin{pgfscope}%
\pgfsys@transformshift{3.015896in}{0.539544in}%
\pgfsys@useobject{currentmarker}{}%
\end{pgfscope}%
\end{pgfscope}%
\begin{pgfscope}%
\definecolor{textcolor}{rgb}{0.000000,0.000000,0.000000}%
\pgfsetstrokecolor{textcolor}%
\pgfsetfillcolor{textcolor}%
\pgftext[x=3.015896in,y=0.442322in,,top]{\color{textcolor}\rmfamily\fontsize{8.000000}{9.600000}\selectfont \(\displaystyle {15{:}00}\)}%
\end{pgfscope}%
\begin{pgfscope}%
\pgfsetbuttcap%
\pgfsetroundjoin%
\definecolor{currentfill}{rgb}{0.000000,0.000000,0.000000}%
\pgfsetfillcolor{currentfill}%
\pgfsetlinewidth{0.803000pt}%
\definecolor{currentstroke}{rgb}{0.000000,0.000000,0.000000}%
\pgfsetstrokecolor{currentstroke}%
\pgfsetdash{}{0pt}%
\pgfsys@defobject{currentmarker}{\pgfqpoint{0.000000in}{-0.048611in}}{\pgfqpoint{0.000000in}{0.000000in}}{%
\pgfpathmoveto{\pgfqpoint{0.000000in}{0.000000in}}%
\pgfpathlineto{\pgfqpoint{0.000000in}{-0.048611in}}%
\pgfusepath{stroke,fill}%
}%
\begin{pgfscope}%
\pgfsys@transformshift{3.456844in}{0.539544in}%
\pgfsys@useobject{currentmarker}{}%
\end{pgfscope}%
\end{pgfscope}%
\begin{pgfscope}%
\definecolor{textcolor}{rgb}{0.000000,0.000000,0.000000}%
\pgfsetstrokecolor{textcolor}%
\pgfsetfillcolor{textcolor}%
\pgftext[x=3.456844in,y=0.442322in,,top]{\color{textcolor}\rmfamily\fontsize{8.000000}{9.600000}\selectfont \(\displaystyle {18{:}00}\)}%
\end{pgfscope}%
\begin{pgfscope}%
\pgfsetbuttcap%
\pgfsetroundjoin%
\definecolor{currentfill}{rgb}{0.000000,0.000000,0.000000}%
\pgfsetfillcolor{currentfill}%
\pgfsetlinewidth{0.803000pt}%
\definecolor{currentstroke}{rgb}{0.000000,0.000000,0.000000}%
\pgfsetstrokecolor{currentstroke}%
\pgfsetdash{}{0pt}%
\pgfsys@defobject{currentmarker}{\pgfqpoint{0.000000in}{-0.048611in}}{\pgfqpoint{0.000000in}{0.000000in}}{%
\pgfpathmoveto{\pgfqpoint{0.000000in}{0.000000in}}%
\pgfpathlineto{\pgfqpoint{0.000000in}{-0.048611in}}%
\pgfusepath{stroke,fill}%
}%
\begin{pgfscope}%
\pgfsys@transformshift{3.897792in}{0.539544in}%
\pgfsys@useobject{currentmarker}{}%
\end{pgfscope}%
\end{pgfscope}%
\begin{pgfscope}%
\definecolor{textcolor}{rgb}{0.000000,0.000000,0.000000}%
\pgfsetstrokecolor{textcolor}%
\pgfsetfillcolor{textcolor}%
\pgftext[x=3.897792in,y=0.442322in,,top]{\color{textcolor}\rmfamily\fontsize{8.000000}{9.600000}\selectfont \(\displaystyle {21{:}00}\)}%
\end{pgfscope}%
\begin{pgfscope}%
\pgfsetbuttcap%
\pgfsetroundjoin%
\definecolor{currentfill}{rgb}{0.000000,0.000000,0.000000}%
\pgfsetfillcolor{currentfill}%
\pgfsetlinewidth{0.803000pt}%
\definecolor{currentstroke}{rgb}{0.000000,0.000000,0.000000}%
\pgfsetstrokecolor{currentstroke}%
\pgfsetdash{}{0pt}%
\pgfsys@defobject{currentmarker}{\pgfqpoint{0.000000in}{-0.048611in}}{\pgfqpoint{0.000000in}{0.000000in}}{%
\pgfpathmoveto{\pgfqpoint{0.000000in}{0.000000in}}%
\pgfpathlineto{\pgfqpoint{0.000000in}{-0.048611in}}%
\pgfusepath{stroke,fill}%
}%
\begin{pgfscope}%
\pgfsys@transformshift{4.338740in}{0.539544in}%
\pgfsys@useobject{currentmarker}{}%
\end{pgfscope}%
\end{pgfscope}%
\begin{pgfscope}%
\definecolor{textcolor}{rgb}{0.000000,0.000000,0.000000}%
\pgfsetstrokecolor{textcolor}%
\pgfsetfillcolor{textcolor}%
\pgftext[x=4.338740in,y=0.442322in,,top]{\color{textcolor}\rmfamily\fontsize{8.000000}{9.600000}\selectfont \(\displaystyle {00{:}00}\)}%
\end{pgfscope}%
\begin{pgfscope}%
\definecolor{textcolor}{rgb}{0.000000,0.000000,0.000000}%
\pgfsetstrokecolor{textcolor}%
\pgfsetfillcolor{textcolor}%
\pgftext[x=2.574927in,y=0.288100in,,top]{\color{textcolor}\rmfamily\fontsize{10.000000}{12.000000}\selectfont Time (UTC)}%
\end{pgfscope}%
\begin{pgfscope}%
\pgfsetbuttcap%
\pgfsetroundjoin%
\definecolor{currentfill}{rgb}{0.000000,0.000000,0.000000}%
\pgfsetfillcolor{currentfill}%
\pgfsetlinewidth{0.803000pt}%
\definecolor{currentstroke}{rgb}{0.000000,0.000000,0.000000}%
\pgfsetstrokecolor{currentstroke}%
\pgfsetdash{}{0pt}%
\pgfsys@defobject{currentmarker}{\pgfqpoint{-0.048611in}{0.000000in}}{\pgfqpoint{-0.000000in}{0.000000in}}{%
\pgfpathmoveto{\pgfqpoint{-0.000000in}{0.000000in}}%
\pgfpathlineto{\pgfqpoint{-0.048611in}{0.000000in}}%
\pgfusepath{stroke,fill}%
}%
\begin{pgfscope}%
\pgfsys@transformshift{0.634869in}{0.746268in}%
\pgfsys@useobject{currentmarker}{}%
\end{pgfscope}%
\end{pgfscope}%
\begin{pgfscope}%
\definecolor{textcolor}{rgb}{0.000000,0.000000,0.000000}%
\pgfsetstrokecolor{textcolor}%
\pgfsetfillcolor{textcolor}%
\pgftext[x=0.327767in, y=0.707713in, left, base]{\color{textcolor}\rmfamily\fontsize{8.000000}{9.600000}\selectfont \(\displaystyle {\ensuremath{-}15}\)}%
\end{pgfscope}%
\begin{pgfscope}%
\pgfsetbuttcap%
\pgfsetroundjoin%
\definecolor{currentfill}{rgb}{0.000000,0.000000,0.000000}%
\pgfsetfillcolor{currentfill}%
\pgfsetlinewidth{0.803000pt}%
\definecolor{currentstroke}{rgb}{0.000000,0.000000,0.000000}%
\pgfsetstrokecolor{currentstroke}%
\pgfsetdash{}{0pt}%
\pgfsys@defobject{currentmarker}{\pgfqpoint{-0.048611in}{0.000000in}}{\pgfqpoint{-0.000000in}{0.000000in}}{%
\pgfpathmoveto{\pgfqpoint{-0.000000in}{0.000000in}}%
\pgfpathlineto{\pgfqpoint{-0.048611in}{0.000000in}}%
\pgfusepath{stroke,fill}%
}%
\begin{pgfscope}%
\pgfsys@transformshift{0.634869in}{1.158849in}%
\pgfsys@useobject{currentmarker}{}%
\end{pgfscope}%
\end{pgfscope}%
\begin{pgfscope}%
\definecolor{textcolor}{rgb}{0.000000,0.000000,0.000000}%
\pgfsetstrokecolor{textcolor}%
\pgfsetfillcolor{textcolor}%
\pgftext[x=0.327767in, y=1.120293in, left, base]{\color{textcolor}\rmfamily\fontsize{8.000000}{9.600000}\selectfont \(\displaystyle {\ensuremath{-}10}\)}%
\end{pgfscope}%
\begin{pgfscope}%
\pgfsetbuttcap%
\pgfsetroundjoin%
\definecolor{currentfill}{rgb}{0.000000,0.000000,0.000000}%
\pgfsetfillcolor{currentfill}%
\pgfsetlinewidth{0.803000pt}%
\definecolor{currentstroke}{rgb}{0.000000,0.000000,0.000000}%
\pgfsetstrokecolor{currentstroke}%
\pgfsetdash{}{0pt}%
\pgfsys@defobject{currentmarker}{\pgfqpoint{-0.048611in}{0.000000in}}{\pgfqpoint{-0.000000in}{0.000000in}}{%
\pgfpathmoveto{\pgfqpoint{-0.000000in}{0.000000in}}%
\pgfpathlineto{\pgfqpoint{-0.048611in}{0.000000in}}%
\pgfusepath{stroke,fill}%
}%
\begin{pgfscope}%
\pgfsys@transformshift{0.634869in}{1.571429in}%
\pgfsys@useobject{currentmarker}{}%
\end{pgfscope}%
\end{pgfscope}%
\begin{pgfscope}%
\definecolor{textcolor}{rgb}{0.000000,0.000000,0.000000}%
\pgfsetstrokecolor{textcolor}%
\pgfsetfillcolor{textcolor}%
\pgftext[x=0.386796in, y=1.532873in, left, base]{\color{textcolor}\rmfamily\fontsize{8.000000}{9.600000}\selectfont \(\displaystyle {\ensuremath{-}5}\)}%
\end{pgfscope}%
\begin{pgfscope}%
\pgfsetbuttcap%
\pgfsetroundjoin%
\definecolor{currentfill}{rgb}{0.000000,0.000000,0.000000}%
\pgfsetfillcolor{currentfill}%
\pgfsetlinewidth{0.803000pt}%
\definecolor{currentstroke}{rgb}{0.000000,0.000000,0.000000}%
\pgfsetstrokecolor{currentstroke}%
\pgfsetdash{}{0pt}%
\pgfsys@defobject{currentmarker}{\pgfqpoint{-0.048611in}{0.000000in}}{\pgfqpoint{-0.000000in}{0.000000in}}{%
\pgfpathmoveto{\pgfqpoint{-0.000000in}{0.000000in}}%
\pgfpathlineto{\pgfqpoint{-0.048611in}{0.000000in}}%
\pgfusepath{stroke,fill}%
}%
\begin{pgfscope}%
\pgfsys@transformshift{0.634869in}{1.984009in}%
\pgfsys@useobject{currentmarker}{}%
\end{pgfscope}%
\end{pgfscope}%
\begin{pgfscope}%
\definecolor{textcolor}{rgb}{0.000000,0.000000,0.000000}%
\pgfsetstrokecolor{textcolor}%
\pgfsetfillcolor{textcolor}%
\pgftext[x=0.478618in, y=1.945454in, left, base]{\color{textcolor}\rmfamily\fontsize{8.000000}{9.600000}\selectfont \(\displaystyle {0}\)}%
\end{pgfscope}%
\begin{pgfscope}%
\pgfsetbuttcap%
\pgfsetroundjoin%
\definecolor{currentfill}{rgb}{0.000000,0.000000,0.000000}%
\pgfsetfillcolor{currentfill}%
\pgfsetlinewidth{0.803000pt}%
\definecolor{currentstroke}{rgb}{0.000000,0.000000,0.000000}%
\pgfsetstrokecolor{currentstroke}%
\pgfsetdash{}{0pt}%
\pgfsys@defobject{currentmarker}{\pgfqpoint{-0.048611in}{0.000000in}}{\pgfqpoint{-0.000000in}{0.000000in}}{%
\pgfpathmoveto{\pgfqpoint{-0.000000in}{0.000000in}}%
\pgfpathlineto{\pgfqpoint{-0.048611in}{0.000000in}}%
\pgfusepath{stroke,fill}%
}%
\begin{pgfscope}%
\pgfsys@transformshift{0.634869in}{2.396589in}%
\pgfsys@useobject{currentmarker}{}%
\end{pgfscope}%
\end{pgfscope}%
\begin{pgfscope}%
\definecolor{textcolor}{rgb}{0.000000,0.000000,0.000000}%
\pgfsetstrokecolor{textcolor}%
\pgfsetfillcolor{textcolor}%
\pgftext[x=0.478618in, y=2.358034in, left, base]{\color{textcolor}\rmfamily\fontsize{8.000000}{9.600000}\selectfont \(\displaystyle {5}\)}%
\end{pgfscope}%
\begin{pgfscope}%
\pgfsetbuttcap%
\pgfsetroundjoin%
\definecolor{currentfill}{rgb}{0.000000,0.000000,0.000000}%
\pgfsetfillcolor{currentfill}%
\pgfsetlinewidth{0.803000pt}%
\definecolor{currentstroke}{rgb}{0.000000,0.000000,0.000000}%
\pgfsetstrokecolor{currentstroke}%
\pgfsetdash{}{0pt}%
\pgfsys@defobject{currentmarker}{\pgfqpoint{-0.048611in}{0.000000in}}{\pgfqpoint{-0.000000in}{0.000000in}}{%
\pgfpathmoveto{\pgfqpoint{-0.000000in}{0.000000in}}%
\pgfpathlineto{\pgfqpoint{-0.048611in}{0.000000in}}%
\pgfusepath{stroke,fill}%
}%
\begin{pgfscope}%
\pgfsys@transformshift{0.634869in}{2.809170in}%
\pgfsys@useobject{currentmarker}{}%
\end{pgfscope}%
\end{pgfscope}%
\begin{pgfscope}%
\definecolor{textcolor}{rgb}{0.000000,0.000000,0.000000}%
\pgfsetstrokecolor{textcolor}%
\pgfsetfillcolor{textcolor}%
\pgftext[x=0.419589in, y=2.770614in, left, base]{\color{textcolor}\rmfamily\fontsize{8.000000}{9.600000}\selectfont \(\displaystyle {10}\)}%
\end{pgfscope}%
\begin{pgfscope}%
\definecolor{textcolor}{rgb}{0.000000,0.000000,0.000000}%
\pgfsetstrokecolor{textcolor}%
\pgfsetfillcolor{textcolor}%
\pgftext[x=0.272211in,y=1.742216in,,bottom,rotate=90.000000]{\color{textcolor}\rmfamily\fontsize{10.000000}{12.000000}\selectfont Voltage deviation in V}%
\end{pgfscope}%
\begin{pgfscope}%
\definecolor{textcolor}{rgb}{0.000000,0.000000,0.000000}%
\pgfsetstrokecolor{textcolor}%
\pgfsetfillcolor{textcolor}%
\pgftext[x=0.634869in,y=2.986554in,left,base]{\color{textcolor}\rmfamily\fontsize{8.000000}{9.600000}\selectfont \(\displaystyle \times{10^{\ensuremath{-}6}}{}\)}%
\end{pgfscope}%
\begin{pgfscope}%
\pgfpathrectangle{\pgfqpoint{0.634869in}{0.539544in}}{\pgfqpoint{3.880116in}{2.405343in}}%
\pgfusepath{clip}%
\pgfsetrectcap%
\pgfsetroundjoin%
\pgfsetlinewidth{0.501875pt}%
\definecolor{currentstroke}{rgb}{0.121569,0.466667,0.705882}%
\pgfsetstrokecolor{currentstroke}%
\pgfsetstrokeopacity{0.700000}%
\pgfsetdash{}{0pt}%
\pgfpathmoveto{\pgfqpoint{0.811238in}{1.927877in}}%
\pgfpathlineto{\pgfqpoint{0.811442in}{1.919625in}}%
\pgfpathlineto{\pgfqpoint{0.812667in}{2.233186in}}%
\pgfpathlineto{\pgfqpoint{0.814096in}{2.142418in}}%
\pgfpathlineto{\pgfqpoint{0.814912in}{2.092909in}}%
\pgfpathlineto{\pgfqpoint{0.815321in}{2.200180in}}%
\pgfpathlineto{\pgfqpoint{0.816341in}{1.969135in}}%
\pgfpathlineto{\pgfqpoint{0.815729in}{2.241438in}}%
\pgfpathlineto{\pgfqpoint{0.816750in}{2.117664in}}%
\pgfpathlineto{\pgfqpoint{0.817362in}{2.208431in}}%
\pgfpathlineto{\pgfqpoint{0.817770in}{2.167173in}}%
\pgfpathlineto{\pgfqpoint{0.818179in}{2.084657in}}%
\pgfpathlineto{\pgfqpoint{0.818587in}{2.200180in}}%
\pgfpathlineto{\pgfqpoint{0.819403in}{2.241438in}}%
\pgfpathlineto{\pgfqpoint{0.819812in}{2.233186in}}%
\pgfpathlineto{\pgfqpoint{0.820016in}{2.224934in}}%
\pgfpathlineto{\pgfqpoint{0.820220in}{2.249689in}}%
\pgfpathlineto{\pgfqpoint{0.820628in}{2.290947in}}%
\pgfpathlineto{\pgfqpoint{0.821241in}{2.257941in}}%
\pgfpathlineto{\pgfqpoint{0.821853in}{2.224934in}}%
\pgfpathlineto{\pgfqpoint{0.822057in}{2.257941in}}%
\pgfpathlineto{\pgfqpoint{0.822261in}{2.274444in}}%
\pgfpathlineto{\pgfqpoint{0.822465in}{2.224934in}}%
\pgfpathlineto{\pgfqpoint{0.822874in}{2.257941in}}%
\pgfpathlineto{\pgfqpoint{0.823282in}{2.224934in}}%
\pgfpathlineto{\pgfqpoint{0.823486in}{2.249689in}}%
\pgfpathlineto{\pgfqpoint{0.824507in}{2.307450in}}%
\pgfpathlineto{\pgfqpoint{0.824711in}{2.266192in}}%
\pgfpathlineto{\pgfqpoint{0.825119in}{2.299199in}}%
\pgfpathlineto{\pgfqpoint{0.825732in}{2.348709in}}%
\pgfpathlineto{\pgfqpoint{0.826140in}{2.290947in}}%
\pgfpathlineto{\pgfqpoint{0.826548in}{2.340457in}}%
\pgfpathlineto{\pgfqpoint{0.827161in}{2.175425in}}%
\pgfpathlineto{\pgfqpoint{0.827569in}{2.282696in}}%
\pgfpathlineto{\pgfqpoint{0.827773in}{2.282696in}}%
\pgfpathlineto{\pgfqpoint{0.828181in}{2.365212in}}%
\pgfpathlineto{\pgfqpoint{0.828794in}{2.274444in}}%
\pgfpathlineto{\pgfqpoint{0.830019in}{2.365212in}}%
\pgfpathlineto{\pgfqpoint{0.831039in}{2.290947in}}%
\pgfpathlineto{\pgfqpoint{0.831244in}{2.348709in}}%
\pgfpathlineto{\pgfqpoint{0.831652in}{2.274444in}}%
\pgfpathlineto{\pgfqpoint{0.831856in}{2.134167in}}%
\pgfpathlineto{\pgfqpoint{0.832468in}{2.340457in}}%
\pgfpathlineto{\pgfqpoint{0.832673in}{2.389967in}}%
\pgfpathlineto{\pgfqpoint{0.832877in}{2.340457in}}%
\pgfpathlineto{\pgfqpoint{0.833081in}{2.191928in}}%
\pgfpathlineto{\pgfqpoint{0.833897in}{2.282696in}}%
\pgfpathlineto{\pgfqpoint{0.834102in}{2.290947in}}%
\pgfpathlineto{\pgfqpoint{0.834306in}{2.109412in}}%
\pgfpathlineto{\pgfqpoint{0.834918in}{2.307450in}}%
\pgfpathlineto{\pgfqpoint{0.835122in}{2.274444in}}%
\pgfpathlineto{\pgfqpoint{0.835735in}{2.348709in}}%
\pgfpathlineto{\pgfqpoint{0.835939in}{2.266192in}}%
\pgfpathlineto{\pgfqpoint{0.836143in}{2.241438in}}%
\pgfpathlineto{\pgfqpoint{0.836755in}{2.282696in}}%
\pgfpathlineto{\pgfqpoint{0.836960in}{2.249689in}}%
\pgfpathlineto{\pgfqpoint{0.837572in}{2.241438in}}%
\pgfpathlineto{\pgfqpoint{0.837980in}{2.307450in}}%
\pgfpathlineto{\pgfqpoint{0.838184in}{2.274444in}}%
\pgfpathlineto{\pgfqpoint{0.838797in}{2.290947in}}%
\pgfpathlineto{\pgfqpoint{0.839409in}{2.406470in}}%
\pgfpathlineto{\pgfqpoint{0.839818in}{2.274444in}}%
\pgfpathlineto{\pgfqpoint{0.840634in}{2.340457in}}%
\pgfpathlineto{\pgfqpoint{0.841042in}{2.332205in}}%
\pgfpathlineto{\pgfqpoint{0.841451in}{2.249689in}}%
\pgfpathlineto{\pgfqpoint{0.841655in}{2.299199in}}%
\pgfpathlineto{\pgfqpoint{0.841859in}{2.406470in}}%
\pgfpathlineto{\pgfqpoint{0.842880in}{2.381715in}}%
\pgfpathlineto{\pgfqpoint{0.843900in}{2.464231in}}%
\pgfpathlineto{\pgfqpoint{0.844105in}{2.422973in}}%
\pgfpathlineto{\pgfqpoint{0.844309in}{2.365212in}}%
\pgfpathlineto{\pgfqpoint{0.844717in}{2.447728in}}%
\pgfpathlineto{\pgfqpoint{0.844921in}{2.439476in}}%
\pgfpathlineto{\pgfqpoint{0.845534in}{2.505489in}}%
\pgfpathlineto{\pgfqpoint{0.845738in}{2.464231in}}%
\pgfpathlineto{\pgfqpoint{0.845942in}{2.373463in}}%
\pgfpathlineto{\pgfqpoint{0.846758in}{2.447728in}}%
\pgfpathlineto{\pgfqpoint{0.847371in}{2.356960in}}%
\pgfpathlineto{\pgfqpoint{0.847779in}{2.455979in}}%
\pgfpathlineto{\pgfqpoint{0.847983in}{2.455979in}}%
\pgfpathlineto{\pgfqpoint{0.848187in}{2.480734in}}%
\pgfpathlineto{\pgfqpoint{0.848596in}{2.455979in}}%
\pgfpathlineto{\pgfqpoint{0.849821in}{2.315702in}}%
\pgfpathlineto{\pgfqpoint{0.850025in}{2.365212in}}%
\pgfpathlineto{\pgfqpoint{0.850637in}{2.323954in}}%
\pgfpathlineto{\pgfqpoint{0.850841in}{2.356960in}}%
\pgfpathlineto{\pgfqpoint{0.853291in}{2.497237in}}%
\pgfpathlineto{\pgfqpoint{0.853495in}{2.488986in}}%
\pgfpathlineto{\pgfqpoint{0.854720in}{2.406470in}}%
\pgfpathlineto{\pgfqpoint{0.855128in}{2.422973in}}%
\pgfpathlineto{\pgfqpoint{0.855537in}{2.439476in}}%
\pgfpathlineto{\pgfqpoint{0.856149in}{2.398218in}}%
\pgfpathlineto{\pgfqpoint{0.856353in}{2.464231in}}%
\pgfpathlineto{\pgfqpoint{0.857170in}{2.398218in}}%
\pgfpathlineto{\pgfqpoint{0.857782in}{2.348709in}}%
\pgfpathlineto{\pgfqpoint{0.857578in}{2.431225in}}%
\pgfpathlineto{\pgfqpoint{0.857986in}{2.422973in}}%
\pgfpathlineto{\pgfqpoint{0.858395in}{2.422973in}}%
\pgfpathlineto{\pgfqpoint{0.859007in}{2.373463in}}%
\pgfpathlineto{\pgfqpoint{0.859415in}{2.398218in}}%
\pgfpathlineto{\pgfqpoint{0.860436in}{2.422973in}}%
\pgfpathlineto{\pgfqpoint{0.860640in}{2.381715in}}%
\pgfpathlineto{\pgfqpoint{0.860844in}{2.455979in}}%
\pgfpathlineto{\pgfqpoint{0.861661in}{2.398218in}}%
\pgfpathlineto{\pgfqpoint{0.861865in}{2.398218in}}%
\pgfpathlineto{\pgfqpoint{0.862273in}{2.365212in}}%
\pgfpathlineto{\pgfqpoint{0.863090in}{2.439476in}}%
\pgfpathlineto{\pgfqpoint{0.863702in}{2.398218in}}%
\pgfpathlineto{\pgfqpoint{0.863498in}{2.447728in}}%
\pgfpathlineto{\pgfqpoint{0.863906in}{2.422973in}}%
\pgfpathlineto{\pgfqpoint{0.864315in}{2.406470in}}%
\pgfpathlineto{\pgfqpoint{0.864723in}{2.455979in}}%
\pgfpathlineto{\pgfqpoint{0.864927in}{2.315702in}}%
\pgfpathlineto{\pgfqpoint{0.865131in}{2.521992in}}%
\pgfpathlineto{\pgfqpoint{0.865744in}{2.480734in}}%
\pgfpathlineto{\pgfqpoint{0.865948in}{2.455979in}}%
\pgfpathlineto{\pgfqpoint{0.866152in}{2.249689in}}%
\pgfpathlineto{\pgfqpoint{0.866356in}{2.480734in}}%
\pgfpathlineto{\pgfqpoint{0.866969in}{2.455979in}}%
\pgfpathlineto{\pgfqpoint{0.868602in}{2.554999in}}%
\pgfpathlineto{\pgfqpoint{0.869214in}{2.439476in}}%
\pgfpathlineto{\pgfqpoint{0.869622in}{2.488986in}}%
\pgfpathlineto{\pgfqpoint{0.870847in}{2.521992in}}%
\pgfpathlineto{\pgfqpoint{0.871051in}{2.513741in}}%
\pgfpathlineto{\pgfqpoint{0.871256in}{2.596257in}}%
\pgfpathlineto{\pgfqpoint{0.871664in}{2.488986in}}%
\pgfpathlineto{\pgfqpoint{0.872072in}{2.579753in}}%
\pgfpathlineto{\pgfqpoint{0.873093in}{2.480734in}}%
\pgfpathlineto{\pgfqpoint{0.873297in}{2.554999in}}%
\pgfpathlineto{\pgfqpoint{0.874114in}{2.513741in}}%
\pgfpathlineto{\pgfqpoint{0.874930in}{2.472483in}}%
\pgfpathlineto{\pgfqpoint{0.875543in}{2.315702in}}%
\pgfpathlineto{\pgfqpoint{0.876155in}{2.579753in}}%
\pgfpathlineto{\pgfqpoint{0.876563in}{2.398218in}}%
\pgfpathlineto{\pgfqpoint{0.877380in}{2.538495in}}%
\pgfpathlineto{\pgfqpoint{0.877788in}{2.497237in}}%
\pgfpathlineto{\pgfqpoint{0.878196in}{2.191928in}}%
\pgfpathlineto{\pgfqpoint{0.878809in}{2.439476in}}%
\pgfpathlineto{\pgfqpoint{0.879013in}{2.439476in}}%
\pgfpathlineto{\pgfqpoint{0.879217in}{2.455979in}}%
\pgfpathlineto{\pgfqpoint{0.879830in}{2.472483in}}%
\pgfpathlineto{\pgfqpoint{0.880442in}{2.200180in}}%
\pgfpathlineto{\pgfqpoint{0.880646in}{2.497237in}}%
\pgfpathlineto{\pgfqpoint{0.881667in}{2.431225in}}%
\pgfpathlineto{\pgfqpoint{0.881871in}{2.431225in}}%
\pgfpathlineto{\pgfqpoint{0.883096in}{2.596257in}}%
\pgfpathlineto{\pgfqpoint{0.883300in}{2.538495in}}%
\pgfpathlineto{\pgfqpoint{0.884117in}{2.505489in}}%
\pgfpathlineto{\pgfqpoint{0.883912in}{2.563250in}}%
\pgfpathlineto{\pgfqpoint{0.884321in}{2.513741in}}%
\pgfpathlineto{\pgfqpoint{0.885137in}{2.596257in}}%
\pgfpathlineto{\pgfqpoint{0.885546in}{2.554999in}}%
\pgfpathlineto{\pgfqpoint{0.885750in}{2.554999in}}%
\pgfpathlineto{\pgfqpoint{0.886362in}{2.612760in}}%
\pgfpathlineto{\pgfqpoint{0.886975in}{2.596257in}}%
\pgfpathlineto{\pgfqpoint{0.887179in}{2.488986in}}%
\pgfpathlineto{\pgfqpoint{0.887995in}{2.579753in}}%
\pgfpathlineto{\pgfqpoint{0.888404in}{2.588005in}}%
\pgfpathlineto{\pgfqpoint{0.888812in}{2.530244in}}%
\pgfpathlineto{\pgfqpoint{0.889833in}{2.637515in}}%
\pgfpathlineto{\pgfqpoint{0.890037in}{2.621011in}}%
\pgfpathlineto{\pgfqpoint{0.890649in}{2.579753in}}%
\pgfpathlineto{\pgfqpoint{0.891057in}{2.612760in}}%
\pgfpathlineto{\pgfqpoint{0.891262in}{2.629263in}}%
\pgfpathlineto{\pgfqpoint{0.891466in}{2.579753in}}%
\pgfpathlineto{\pgfqpoint{0.893099in}{2.472483in}}%
\pgfpathlineto{\pgfqpoint{0.893303in}{2.464231in}}%
\pgfpathlineto{\pgfqpoint{0.893507in}{2.480734in}}%
\pgfpathlineto{\pgfqpoint{0.894936in}{2.654018in}}%
\pgfpathlineto{\pgfqpoint{0.895140in}{2.662269in}}%
\pgfpathlineto{\pgfqpoint{0.896569in}{2.455979in}}%
\pgfpathlineto{\pgfqpoint{0.896773in}{2.505489in}}%
\pgfpathlineto{\pgfqpoint{0.897386in}{2.546747in}}%
\pgfpathlineto{\pgfqpoint{0.897590in}{2.521992in}}%
\pgfpathlineto{\pgfqpoint{0.898611in}{2.480734in}}%
\pgfpathlineto{\pgfqpoint{0.899019in}{2.546747in}}%
\pgfpathlineto{\pgfqpoint{0.899836in}{2.538495in}}%
\pgfpathlineto{\pgfqpoint{0.900040in}{2.530244in}}%
\pgfpathlineto{\pgfqpoint{0.901265in}{2.579753in}}%
\pgfpathlineto{\pgfqpoint{0.901469in}{2.554999in}}%
\pgfpathlineto{\pgfqpoint{0.901673in}{2.612760in}}%
\pgfpathlineto{\pgfqpoint{0.902285in}{2.571502in}}%
\pgfpathlineto{\pgfqpoint{0.902489in}{2.654018in}}%
\pgfpathlineto{\pgfqpoint{0.903306in}{2.637515in}}%
\pgfpathlineto{\pgfqpoint{0.903510in}{2.604508in}}%
\pgfpathlineto{\pgfqpoint{0.904327in}{2.629263in}}%
\pgfpathlineto{\pgfqpoint{0.904531in}{2.670521in}}%
\pgfpathlineto{\pgfqpoint{0.905143in}{2.596257in}}%
\pgfpathlineto{\pgfqpoint{0.905347in}{2.637515in}}%
\pgfpathlineto{\pgfqpoint{0.905960in}{2.645766in}}%
\pgfpathlineto{\pgfqpoint{0.906572in}{2.563250in}}%
\pgfpathlineto{\pgfqpoint{0.908001in}{2.447728in}}%
\pgfpathlineto{\pgfqpoint{0.908205in}{2.505489in}}%
\pgfpathlineto{\pgfqpoint{0.909022in}{2.464231in}}%
\pgfpathlineto{\pgfqpoint{0.909226in}{2.464231in}}%
\pgfpathlineto{\pgfqpoint{0.909838in}{2.422973in}}%
\pgfpathlineto{\pgfqpoint{0.910043in}{2.472483in}}%
\pgfpathlineto{\pgfqpoint{0.910247in}{2.455979in}}%
\pgfpathlineto{\pgfqpoint{0.911880in}{2.579753in}}%
\pgfpathlineto{\pgfqpoint{0.910655in}{2.447728in}}%
\pgfpathlineto{\pgfqpoint{0.912288in}{2.554999in}}%
\pgfpathlineto{\pgfqpoint{0.912492in}{2.521992in}}%
\pgfpathlineto{\pgfqpoint{0.913105in}{2.596257in}}%
\pgfpathlineto{\pgfqpoint{0.913717in}{2.563250in}}%
\pgfpathlineto{\pgfqpoint{0.913921in}{2.629263in}}%
\pgfpathlineto{\pgfqpoint{0.916371in}{2.497237in}}%
\pgfpathlineto{\pgfqpoint{0.917188in}{2.554999in}}%
\pgfpathlineto{\pgfqpoint{0.917392in}{2.530244in}}%
\pgfpathlineto{\pgfqpoint{0.917596in}{2.480734in}}%
\pgfpathlineto{\pgfqpoint{0.918004in}{2.571502in}}%
\pgfpathlineto{\pgfqpoint{0.918412in}{2.538495in}}%
\pgfpathlineto{\pgfqpoint{0.919025in}{2.563250in}}%
\pgfpathlineto{\pgfqpoint{0.919229in}{2.546747in}}%
\pgfpathlineto{\pgfqpoint{0.919433in}{2.513741in}}%
\pgfpathlineto{\pgfqpoint{0.919841in}{2.612760in}}%
\pgfpathlineto{\pgfqpoint{0.920046in}{2.579753in}}%
\pgfpathlineto{\pgfqpoint{0.920862in}{2.554999in}}%
\pgfpathlineto{\pgfqpoint{0.920454in}{2.596257in}}%
\pgfpathlineto{\pgfqpoint{0.921066in}{2.579753in}}%
\pgfpathlineto{\pgfqpoint{0.921475in}{2.670521in}}%
\pgfpathlineto{\pgfqpoint{0.922291in}{2.662269in}}%
\pgfpathlineto{\pgfqpoint{0.923108in}{2.588005in}}%
\pgfpathlineto{\pgfqpoint{0.923720in}{2.604508in}}%
\pgfpathlineto{\pgfqpoint{0.923924in}{2.612760in}}%
\pgfpathlineto{\pgfqpoint{0.924537in}{2.497237in}}%
\pgfpathlineto{\pgfqpoint{0.925149in}{2.538495in}}%
\pgfpathlineto{\pgfqpoint{0.925966in}{2.497237in}}%
\pgfpathlineto{\pgfqpoint{0.926782in}{2.505489in}}%
\pgfpathlineto{\pgfqpoint{0.928415in}{2.596257in}}%
\pgfpathlineto{\pgfqpoint{0.929028in}{2.612760in}}%
\pgfpathlineto{\pgfqpoint{0.929640in}{2.571502in}}%
\pgfpathlineto{\pgfqpoint{0.930457in}{2.654018in}}%
\pgfpathlineto{\pgfqpoint{0.930865in}{2.596257in}}%
\pgfpathlineto{\pgfqpoint{0.931069in}{2.596257in}}%
\pgfpathlineto{\pgfqpoint{0.932090in}{2.505489in}}%
\pgfpathlineto{\pgfqpoint{0.931682in}{2.637515in}}%
\pgfpathlineto{\pgfqpoint{0.932498in}{2.521992in}}%
\pgfpathlineto{\pgfqpoint{0.932702in}{2.563250in}}%
\pgfpathlineto{\pgfqpoint{0.933315in}{2.505489in}}%
\pgfpathlineto{\pgfqpoint{0.933519in}{2.505489in}}%
\pgfpathlineto{\pgfqpoint{0.934540in}{2.621011in}}%
\pgfpathlineto{\pgfqpoint{0.933927in}{2.497237in}}%
\pgfpathlineto{\pgfqpoint{0.934948in}{2.563250in}}%
\pgfpathlineto{\pgfqpoint{0.935152in}{2.546747in}}%
\pgfpathlineto{\pgfqpoint{0.935765in}{2.579753in}}%
\pgfpathlineto{\pgfqpoint{0.935969in}{2.579753in}}%
\pgfpathlineto{\pgfqpoint{0.936173in}{2.307450in}}%
\pgfpathlineto{\pgfqpoint{0.936989in}{2.389967in}}%
\pgfpathlineto{\pgfqpoint{0.937806in}{2.612760in}}%
\pgfpathlineto{\pgfqpoint{0.938214in}{2.571502in}}%
\pgfpathlineto{\pgfqpoint{0.939847in}{2.505489in}}%
\pgfpathlineto{\pgfqpoint{0.940052in}{2.546747in}}%
\pgfpathlineto{\pgfqpoint{0.940664in}{2.505489in}}%
\pgfpathlineto{\pgfqpoint{0.940868in}{2.389967in}}%
\pgfpathlineto{\pgfqpoint{0.941481in}{2.604508in}}%
\pgfpathlineto{\pgfqpoint{0.941889in}{2.422973in}}%
\pgfpathlineto{\pgfqpoint{0.942297in}{2.554999in}}%
\pgfpathlineto{\pgfqpoint{0.943114in}{2.505489in}}%
\pgfpathlineto{\pgfqpoint{0.943318in}{2.513741in}}%
\pgfpathlineto{\pgfqpoint{0.943726in}{2.488986in}}%
\pgfpathlineto{\pgfqpoint{0.943930in}{2.439476in}}%
\pgfpathlineto{\pgfqpoint{0.944339in}{2.554999in}}%
\pgfpathlineto{\pgfqpoint{0.944543in}{2.521992in}}%
\pgfpathlineto{\pgfqpoint{0.945768in}{2.662269in}}%
\pgfpathlineto{\pgfqpoint{0.946176in}{2.637515in}}%
\pgfpathlineto{\pgfqpoint{0.946584in}{2.414721in}}%
\pgfpathlineto{\pgfqpoint{0.947197in}{2.588005in}}%
\pgfpathlineto{\pgfqpoint{0.947605in}{2.687024in}}%
\pgfpathlineto{\pgfqpoint{0.948421in}{2.645766in}}%
\pgfpathlineto{\pgfqpoint{0.949646in}{2.563250in}}%
\pgfpathlineto{\pgfqpoint{0.950463in}{2.637515in}}%
\pgfpathlineto{\pgfqpoint{0.950667in}{2.629263in}}%
\pgfpathlineto{\pgfqpoint{0.950871in}{2.546747in}}%
\pgfpathlineto{\pgfqpoint{0.951279in}{2.637515in}}%
\pgfpathlineto{\pgfqpoint{0.951688in}{2.596257in}}%
\pgfpathlineto{\pgfqpoint{0.951892in}{2.637515in}}%
\pgfpathlineto{\pgfqpoint{0.952096in}{2.588005in}}%
\pgfpathlineto{\pgfqpoint{0.952300in}{2.621011in}}%
\pgfpathlineto{\pgfqpoint{0.952504in}{2.447728in}}%
\pgfpathlineto{\pgfqpoint{0.952913in}{2.728282in}}%
\pgfpathlineto{\pgfqpoint{0.953321in}{2.678773in}}%
\pgfpathlineto{\pgfqpoint{0.954546in}{2.769540in}}%
\pgfpathlineto{\pgfqpoint{0.954750in}{2.720031in}}%
\pgfpathlineto{\pgfqpoint{0.955771in}{2.480734in}}%
\pgfpathlineto{\pgfqpoint{0.955975in}{2.645766in}}%
\pgfpathlineto{\pgfqpoint{0.956383in}{2.835553in}}%
\pgfpathlineto{\pgfqpoint{0.956995in}{2.753037in}}%
\pgfpathlineto{\pgfqpoint{0.957200in}{2.711779in}}%
\pgfpathlineto{\pgfqpoint{0.958016in}{2.744786in}}%
\pgfpathlineto{\pgfqpoint{0.958424in}{2.769540in}}%
\pgfpathlineto{\pgfqpoint{0.958833in}{2.720031in}}%
\pgfpathlineto{\pgfqpoint{0.959853in}{2.670521in}}%
\pgfpathlineto{\pgfqpoint{0.960058in}{2.678773in}}%
\pgfpathlineto{\pgfqpoint{0.960262in}{2.695276in}}%
\pgfpathlineto{\pgfqpoint{0.960670in}{2.670521in}}%
\pgfpathlineto{\pgfqpoint{0.962099in}{2.571502in}}%
\pgfpathlineto{\pgfqpoint{0.962711in}{2.703527in}}%
\pgfpathlineto{\pgfqpoint{0.962507in}{2.431225in}}%
\pgfpathlineto{\pgfqpoint{0.963120in}{2.678773in}}%
\pgfpathlineto{\pgfqpoint{0.963528in}{2.604508in}}%
\pgfpathlineto{\pgfqpoint{0.964345in}{2.621011in}}%
\pgfpathlineto{\pgfqpoint{0.964549in}{2.629263in}}%
\pgfpathlineto{\pgfqpoint{0.965365in}{2.406470in}}%
\pgfpathlineto{\pgfqpoint{0.965569in}{2.579753in}}%
\pgfpathlineto{\pgfqpoint{0.966998in}{2.678773in}}%
\pgfpathlineto{\pgfqpoint{0.968223in}{2.406470in}}%
\pgfpathlineto{\pgfqpoint{0.968427in}{2.521992in}}%
\pgfpathlineto{\pgfqpoint{0.969448in}{2.621011in}}%
\pgfpathlineto{\pgfqpoint{0.969652in}{2.612760in}}%
\pgfpathlineto{\pgfqpoint{0.969856in}{2.596257in}}%
\pgfpathlineto{\pgfqpoint{0.970061in}{2.621011in}}%
\pgfpathlineto{\pgfqpoint{0.970265in}{2.604508in}}%
\pgfpathlineto{\pgfqpoint{0.970673in}{2.662269in}}%
\pgfpathlineto{\pgfqpoint{0.971081in}{2.554999in}}%
\pgfpathlineto{\pgfqpoint{0.971285in}{2.621011in}}%
\pgfpathlineto{\pgfqpoint{0.972510in}{2.282696in}}%
\pgfpathlineto{\pgfqpoint{0.973531in}{2.662269in}}%
\pgfpathlineto{\pgfqpoint{0.973735in}{2.645766in}}%
\pgfpathlineto{\pgfqpoint{0.973939in}{2.654018in}}%
\pgfpathlineto{\pgfqpoint{0.974143in}{2.637515in}}%
\pgfpathlineto{\pgfqpoint{0.975368in}{2.480734in}}%
\pgfpathlineto{\pgfqpoint{0.974756in}{2.678773in}}%
\pgfpathlineto{\pgfqpoint{0.975572in}{2.588005in}}%
\pgfpathlineto{\pgfqpoint{0.975777in}{2.621011in}}%
\pgfpathlineto{\pgfqpoint{0.976389in}{2.604508in}}%
\pgfpathlineto{\pgfqpoint{0.976593in}{2.554999in}}%
\pgfpathlineto{\pgfqpoint{0.977206in}{2.621011in}}%
\pgfpathlineto{\pgfqpoint{0.977614in}{2.563250in}}%
\pgfpathlineto{\pgfqpoint{0.979043in}{2.662269in}}%
\pgfpathlineto{\pgfqpoint{0.979247in}{2.488986in}}%
\pgfpathlineto{\pgfqpoint{0.980064in}{2.662269in}}%
\pgfpathlineto{\pgfqpoint{0.981493in}{2.480734in}}%
\pgfpathlineto{\pgfqpoint{0.980472in}{2.687024in}}%
\pgfpathlineto{\pgfqpoint{0.981697in}{2.513741in}}%
\pgfpathlineto{\pgfqpoint{0.981901in}{2.554999in}}%
\pgfpathlineto{\pgfqpoint{0.982309in}{2.414721in}}%
\pgfpathlineto{\pgfqpoint{0.982922in}{2.546747in}}%
\pgfpathlineto{\pgfqpoint{0.983126in}{2.538495in}}%
\pgfpathlineto{\pgfqpoint{0.983534in}{2.488986in}}%
\pgfpathlineto{\pgfqpoint{0.984555in}{2.654018in}}%
\pgfpathlineto{\pgfqpoint{0.985984in}{2.554999in}}%
\pgfpathlineto{\pgfqpoint{0.987821in}{2.720031in}}%
\pgfpathlineto{\pgfqpoint{0.988842in}{2.365212in}}%
\pgfpathlineto{\pgfqpoint{0.989250in}{2.554999in}}%
\pgfpathlineto{\pgfqpoint{0.989658in}{2.571502in}}%
\pgfpathlineto{\pgfqpoint{0.990679in}{2.488986in}}%
\pgfpathlineto{\pgfqpoint{0.990883in}{2.505489in}}%
\pgfpathlineto{\pgfqpoint{0.991700in}{2.654018in}}%
\pgfpathlineto{\pgfqpoint{0.991904in}{2.406470in}}%
\pgfpathlineto{\pgfqpoint{0.992720in}{2.645766in}}%
\pgfpathlineto{\pgfqpoint{0.993333in}{2.687024in}}%
\pgfpathlineto{\pgfqpoint{0.993537in}{2.621011in}}%
\pgfpathlineto{\pgfqpoint{0.994149in}{2.554999in}}%
\pgfpathlineto{\pgfqpoint{0.994558in}{2.662269in}}%
\pgfpathlineto{\pgfqpoint{0.994966in}{2.645766in}}%
\pgfpathlineto{\pgfqpoint{0.995170in}{2.365212in}}%
\pgfpathlineto{\pgfqpoint{0.995987in}{2.711779in}}%
\pgfpathlineto{\pgfqpoint{0.997007in}{2.612760in}}%
\pgfpathlineto{\pgfqpoint{0.997416in}{2.621011in}}%
\pgfpathlineto{\pgfqpoint{0.998028in}{2.596257in}}%
\pgfpathlineto{\pgfqpoint{0.998640in}{2.645766in}}%
\pgfpathlineto{\pgfqpoint{0.999253in}{2.521992in}}%
\pgfpathlineto{\pgfqpoint{0.999865in}{2.554999in}}%
\pgfpathlineto{\pgfqpoint{1.000069in}{2.554999in}}%
\pgfpathlineto{\pgfqpoint{1.000274in}{2.546747in}}%
\pgfpathlineto{\pgfqpoint{1.000478in}{2.398218in}}%
\pgfpathlineto{\pgfqpoint{1.001294in}{2.579753in}}%
\pgfpathlineto{\pgfqpoint{1.001498in}{2.538495in}}%
\pgfpathlineto{\pgfqpoint{1.002111in}{2.612760in}}%
\pgfpathlineto{\pgfqpoint{1.003336in}{2.703527in}}%
\pgfpathlineto{\pgfqpoint{1.004561in}{2.554999in}}%
\pgfpathlineto{\pgfqpoint{1.005990in}{2.645766in}}%
\pgfpathlineto{\pgfqpoint{1.006602in}{2.546747in}}%
\pgfpathlineto{\pgfqpoint{1.007010in}{2.604508in}}%
\pgfpathlineto{\pgfqpoint{1.007827in}{2.662269in}}%
\pgfpathlineto{\pgfqpoint{1.008031in}{2.629263in}}%
\pgfpathlineto{\pgfqpoint{1.008439in}{2.604508in}}%
\pgfpathlineto{\pgfqpoint{1.008848in}{2.662269in}}%
\pgfpathlineto{\pgfqpoint{1.009052in}{2.654018in}}%
\pgfpathlineto{\pgfqpoint{1.009256in}{2.703527in}}%
\pgfpathlineto{\pgfqpoint{1.010277in}{2.695276in}}%
\pgfpathlineto{\pgfqpoint{1.010889in}{2.720031in}}%
\pgfpathlineto{\pgfqpoint{1.011501in}{2.645766in}}%
\pgfpathlineto{\pgfqpoint{1.011706in}{2.720031in}}%
\pgfpathlineto{\pgfqpoint{1.012522in}{2.678773in}}%
\pgfpathlineto{\pgfqpoint{1.013543in}{2.554999in}}%
\pgfpathlineto{\pgfqpoint{1.013951in}{2.645766in}}%
\pgfpathlineto{\pgfqpoint{1.014564in}{2.662269in}}%
\pgfpathlineto{\pgfqpoint{1.014359in}{2.637515in}}%
\pgfpathlineto{\pgfqpoint{1.014972in}{2.645766in}}%
\pgfpathlineto{\pgfqpoint{1.015993in}{2.596257in}}%
\pgfpathlineto{\pgfqpoint{1.015584in}{2.687024in}}%
\pgfpathlineto{\pgfqpoint{1.016197in}{2.604508in}}%
\pgfpathlineto{\pgfqpoint{1.016401in}{2.645766in}}%
\pgfpathlineto{\pgfqpoint{1.017013in}{2.571502in}}%
\pgfpathlineto{\pgfqpoint{1.017217in}{2.604508in}}%
\pgfpathlineto{\pgfqpoint{1.017422in}{2.612760in}}%
\pgfpathlineto{\pgfqpoint{1.018851in}{2.422973in}}%
\pgfpathlineto{\pgfqpoint{1.019259in}{2.439476in}}%
\pgfpathlineto{\pgfqpoint{1.020484in}{2.307450in}}%
\pgfpathlineto{\pgfqpoint{1.021300in}{2.373463in}}%
\pgfpathlineto{\pgfqpoint{1.021096in}{2.290947in}}%
\pgfpathlineto{\pgfqpoint{1.021504in}{2.299199in}}%
\pgfpathlineto{\pgfqpoint{1.021709in}{2.315702in}}%
\pgfpathlineto{\pgfqpoint{1.022117in}{2.266192in}}%
\pgfpathlineto{\pgfqpoint{1.022525in}{2.299199in}}%
\pgfpathlineto{\pgfqpoint{1.023546in}{2.257941in}}%
\pgfpathlineto{\pgfqpoint{1.022933in}{2.340457in}}%
\pgfpathlineto{\pgfqpoint{1.023750in}{2.274444in}}%
\pgfpathlineto{\pgfqpoint{1.024975in}{2.373463in}}%
\pgfpathlineto{\pgfqpoint{1.026812in}{2.150670in}}%
\pgfpathlineto{\pgfqpoint{1.027629in}{2.315702in}}%
\pgfpathlineto{\pgfqpoint{1.028241in}{2.282696in}}%
\pgfpathlineto{\pgfqpoint{1.028445in}{2.290947in}}%
\pgfpathlineto{\pgfqpoint{1.028649in}{2.282696in}}%
\pgfpathlineto{\pgfqpoint{1.029466in}{2.109412in}}%
\pgfpathlineto{\pgfqpoint{1.030078in}{2.183676in}}%
\pgfpathlineto{\pgfqpoint{1.030283in}{2.183676in}}%
\pgfpathlineto{\pgfqpoint{1.030487in}{2.059902in}}%
\pgfpathlineto{\pgfqpoint{1.031507in}{2.101160in}}%
\pgfpathlineto{\pgfqpoint{1.032120in}{2.092909in}}%
\pgfpathlineto{\pgfqpoint{1.032324in}{2.134167in}}%
\pgfpathlineto{\pgfqpoint{1.032528in}{2.241438in}}%
\pgfpathlineto{\pgfqpoint{1.033141in}{2.076406in}}%
\pgfpathlineto{\pgfqpoint{1.034161in}{2.010393in}}%
\pgfpathlineto{\pgfqpoint{1.034570in}{2.026896in}}%
\pgfpathlineto{\pgfqpoint{1.035794in}{2.200180in}}%
\pgfpathlineto{\pgfqpoint{1.034978in}{1.969135in}}%
\pgfpathlineto{\pgfqpoint{1.035999in}{2.158922in}}%
\pgfpathlineto{\pgfqpoint{1.036203in}{1.993890in}}%
\pgfpathlineto{\pgfqpoint{1.036611in}{2.200180in}}%
\pgfpathlineto{\pgfqpoint{1.037019in}{2.084657in}}%
\pgfpathlineto{\pgfqpoint{1.038244in}{2.224934in}}%
\pgfpathlineto{\pgfqpoint{1.038448in}{2.191928in}}%
\pgfpathlineto{\pgfqpoint{1.038652in}{1.919625in}}%
\pgfpathlineto{\pgfqpoint{1.039469in}{2.092909in}}%
\pgfpathlineto{\pgfqpoint{1.039877in}{2.134167in}}%
\pgfpathlineto{\pgfqpoint{1.040286in}{2.084657in}}%
\pgfpathlineto{\pgfqpoint{1.040490in}{2.059902in}}%
\pgfpathlineto{\pgfqpoint{1.040694in}{2.109412in}}%
\pgfpathlineto{\pgfqpoint{1.041306in}{2.084657in}}%
\pgfpathlineto{\pgfqpoint{1.041715in}{2.018644in}}%
\pgfpathlineto{\pgfqpoint{1.042531in}{2.167173in}}%
\pgfpathlineto{\pgfqpoint{1.043348in}{2.026896in}}%
\pgfpathlineto{\pgfqpoint{1.043960in}{2.084657in}}%
\pgfpathlineto{\pgfqpoint{1.044368in}{2.109412in}}%
\pgfpathlineto{\pgfqpoint{1.044777in}{1.853612in}}%
\pgfpathlineto{\pgfqpoint{1.045593in}{1.993890in}}%
\pgfpathlineto{\pgfqpoint{1.046002in}{2.035148in}}%
\pgfpathlineto{\pgfqpoint{1.046206in}{1.977386in}}%
\pgfpathlineto{\pgfqpoint{1.046614in}{1.993890in}}%
\pgfpathlineto{\pgfqpoint{1.047431in}{1.903122in}}%
\pgfpathlineto{\pgfqpoint{1.047635in}{1.969135in}}%
\pgfpathlineto{\pgfqpoint{1.048451in}{1.894870in}}%
\pgfpathlineto{\pgfqpoint{1.048043in}{1.985638in}}%
\pgfpathlineto{\pgfqpoint{1.049064in}{1.936128in}}%
\pgfpathlineto{\pgfqpoint{1.049880in}{1.977386in}}%
\pgfpathlineto{\pgfqpoint{1.050084in}{1.969135in}}%
\pgfpathlineto{\pgfqpoint{1.050493in}{1.762845in}}%
\pgfpathlineto{\pgfqpoint{1.051105in}{1.927877in}}%
\pgfpathlineto{\pgfqpoint{1.051718in}{1.985638in}}%
\pgfpathlineto{\pgfqpoint{1.051513in}{1.870115in}}%
\pgfpathlineto{\pgfqpoint{1.051922in}{1.936128in}}%
\pgfpathlineto{\pgfqpoint{1.052126in}{1.828857in}}%
\pgfpathlineto{\pgfqpoint{1.052942in}{1.944380in}}%
\pgfpathlineto{\pgfqpoint{1.053351in}{2.002141in}}%
\pgfpathlineto{\pgfqpoint{1.053963in}{1.977386in}}%
\pgfpathlineto{\pgfqpoint{1.054780in}{1.878367in}}%
\pgfpathlineto{\pgfqpoint{1.054984in}{1.919625in}}%
\pgfpathlineto{\pgfqpoint{1.055392in}{2.010393in}}%
\pgfpathlineto{\pgfqpoint{1.056209in}{1.960883in}}%
\pgfpathlineto{\pgfqpoint{1.057842in}{2.101160in}}%
\pgfpathlineto{\pgfqpoint{1.058250in}{2.018644in}}%
\pgfpathlineto{\pgfqpoint{1.058863in}{2.059902in}}%
\pgfpathlineto{\pgfqpoint{1.059067in}{2.076406in}}%
\pgfpathlineto{\pgfqpoint{1.059271in}{2.010393in}}%
\pgfpathlineto{\pgfqpoint{1.059475in}{2.010393in}}%
\pgfpathlineto{\pgfqpoint{1.060700in}{2.134167in}}%
\pgfpathlineto{\pgfqpoint{1.061516in}{1.886619in}}%
\pgfpathlineto{\pgfqpoint{1.061721in}{2.035148in}}%
\pgfpathlineto{\pgfqpoint{1.061925in}{2.142418in}}%
\pgfpathlineto{\pgfqpoint{1.062129in}{1.936128in}}%
\pgfpathlineto{\pgfqpoint{1.062741in}{2.117664in}}%
\pgfpathlineto{\pgfqpoint{1.063966in}{1.993890in}}%
\pgfpathlineto{\pgfqpoint{1.064783in}{2.158922in}}%
\pgfpathlineto{\pgfqpoint{1.065191in}{2.117664in}}%
\pgfpathlineto{\pgfqpoint{1.065599in}{2.216683in}}%
\pgfpathlineto{\pgfqpoint{1.066212in}{2.109412in}}%
\pgfpathlineto{\pgfqpoint{1.067641in}{1.977386in}}%
\pgfpathlineto{\pgfqpoint{1.066620in}{2.117664in}}%
\pgfpathlineto{\pgfqpoint{1.068049in}{1.985638in}}%
\pgfpathlineto{\pgfqpoint{1.068661in}{1.969135in}}%
\pgfpathlineto{\pgfqpoint{1.069070in}{2.010393in}}%
\pgfpathlineto{\pgfqpoint{1.069478in}{2.051651in}}%
\pgfpathlineto{\pgfqpoint{1.070295in}{1.944380in}}%
\pgfpathlineto{\pgfqpoint{1.070907in}{2.035148in}}%
\pgfpathlineto{\pgfqpoint{1.071315in}{1.911373in}}%
\pgfpathlineto{\pgfqpoint{1.073153in}{1.771096in}}%
\pgfpathlineto{\pgfqpoint{1.074377in}{1.878367in}}%
\pgfpathlineto{\pgfqpoint{1.074582in}{1.878367in}}%
\pgfpathlineto{\pgfqpoint{1.074786in}{1.861864in}}%
\pgfpathlineto{\pgfqpoint{1.075194in}{1.960883in}}%
\pgfpathlineto{\pgfqpoint{1.075806in}{1.944380in}}%
\pgfpathlineto{\pgfqpoint{1.077031in}{1.853612in}}%
\pgfpathlineto{\pgfqpoint{1.078256in}{1.977386in}}%
\pgfpathlineto{\pgfqpoint{1.078460in}{1.993890in}}%
\pgfpathlineto{\pgfqpoint{1.078664in}{1.960883in}}%
\pgfpathlineto{\pgfqpoint{1.078868in}{1.894870in}}%
\pgfpathlineto{\pgfqpoint{1.079073in}{2.002141in}}%
\pgfpathlineto{\pgfqpoint{1.079685in}{1.927877in}}%
\pgfpathlineto{\pgfqpoint{1.080093in}{1.870115in}}%
\pgfpathlineto{\pgfqpoint{1.080706in}{1.969135in}}%
\pgfpathlineto{\pgfqpoint{1.080910in}{1.746341in}}%
\pgfpathlineto{\pgfqpoint{1.081114in}{1.985638in}}%
\pgfpathlineto{\pgfqpoint{1.081726in}{1.894870in}}%
\pgfpathlineto{\pgfqpoint{1.082339in}{1.960883in}}%
\pgfpathlineto{\pgfqpoint{1.082135in}{1.886619in}}%
\pgfpathlineto{\pgfqpoint{1.082747in}{1.936128in}}%
\pgfpathlineto{\pgfqpoint{1.083155in}{1.878367in}}%
\pgfpathlineto{\pgfqpoint{1.083360in}{2.010393in}}%
\pgfpathlineto{\pgfqpoint{1.083564in}{2.059902in}}%
\pgfpathlineto{\pgfqpoint{1.084176in}{2.010393in}}%
\pgfpathlineto{\pgfqpoint{1.085605in}{1.878367in}}%
\pgfpathlineto{\pgfqpoint{1.085809in}{1.878367in}}%
\pgfpathlineto{\pgfqpoint{1.086830in}{1.977386in}}%
\pgfpathlineto{\pgfqpoint{1.087238in}{1.936128in}}%
\pgfpathlineto{\pgfqpoint{1.088463in}{1.820606in}}%
\pgfpathlineto{\pgfqpoint{1.089892in}{2.026896in}}%
\pgfpathlineto{\pgfqpoint{1.091117in}{1.762845in}}%
\pgfpathlineto{\pgfqpoint{1.091934in}{2.043399in}}%
\pgfpathlineto{\pgfqpoint{1.092342in}{1.985638in}}%
\pgfpathlineto{\pgfqpoint{1.092750in}{1.729838in}}%
\pgfpathlineto{\pgfqpoint{1.092954in}{2.002141in}}%
\pgfpathlineto{\pgfqpoint{1.093363in}{2.002141in}}%
\pgfpathlineto{\pgfqpoint{1.093975in}{2.051651in}}%
\pgfpathlineto{\pgfqpoint{1.094383in}{2.043399in}}%
\pgfpathlineto{\pgfqpoint{1.096016in}{1.870115in}}%
\pgfpathlineto{\pgfqpoint{1.096629in}{1.936128in}}%
\pgfpathlineto{\pgfqpoint{1.097037in}{1.878367in}}%
\pgfpathlineto{\pgfqpoint{1.097445in}{1.878367in}}%
\pgfpathlineto{\pgfqpoint{1.097650in}{1.845361in}}%
\pgfpathlineto{\pgfqpoint{1.097854in}{1.911373in}}%
\pgfpathlineto{\pgfqpoint{1.098058in}{1.903122in}}%
\pgfpathlineto{\pgfqpoint{1.098874in}{2.092909in}}%
\pgfpathlineto{\pgfqpoint{1.099283in}{1.969135in}}%
\pgfpathlineto{\pgfqpoint{1.099691in}{1.911373in}}%
\pgfpathlineto{\pgfqpoint{1.100099in}{1.927877in}}%
\pgfpathlineto{\pgfqpoint{1.100712in}{2.084657in}}%
\pgfpathlineto{\pgfqpoint{1.101120in}{1.977386in}}%
\pgfpathlineto{\pgfqpoint{1.101937in}{1.911373in}}%
\pgfpathlineto{\pgfqpoint{1.102345in}{1.919625in}}%
\pgfpathlineto{\pgfqpoint{1.103161in}{1.993890in}}%
\pgfpathlineto{\pgfqpoint{1.103570in}{1.944380in}}%
\pgfpathlineto{\pgfqpoint{1.104795in}{2.018644in}}%
\pgfpathlineto{\pgfqpoint{1.105203in}{1.936128in}}%
\pgfpathlineto{\pgfqpoint{1.105815in}{2.010393in}}%
\pgfpathlineto{\pgfqpoint{1.106224in}{2.002141in}}%
\pgfpathlineto{\pgfqpoint{1.106428in}{2.026896in}}%
\pgfpathlineto{\pgfqpoint{1.106632in}{2.018644in}}%
\pgfpathlineto{\pgfqpoint{1.107244in}{2.092909in}}%
\pgfpathlineto{\pgfqpoint{1.107448in}{2.010393in}}%
\pgfpathlineto{\pgfqpoint{1.108877in}{1.853612in}}%
\pgfpathlineto{\pgfqpoint{1.108061in}{2.035148in}}%
\pgfpathlineto{\pgfqpoint{1.109490in}{1.894870in}}%
\pgfpathlineto{\pgfqpoint{1.109694in}{1.952632in}}%
\pgfpathlineto{\pgfqpoint{1.109898in}{1.886619in}}%
\pgfpathlineto{\pgfqpoint{1.110306in}{1.919625in}}%
\pgfpathlineto{\pgfqpoint{1.111327in}{1.787599in}}%
\pgfpathlineto{\pgfqpoint{1.111531in}{1.853612in}}%
\pgfpathlineto{\pgfqpoint{1.112960in}{1.680329in}}%
\pgfpathlineto{\pgfqpoint{1.113573in}{1.771096in}}%
\pgfpathlineto{\pgfqpoint{1.113981in}{1.696832in}}%
\pgfpathlineto{\pgfqpoint{1.114185in}{1.705083in}}%
\pgfpathlineto{\pgfqpoint{1.115002in}{1.696832in}}%
\pgfpathlineto{\pgfqpoint{1.115410in}{1.787599in}}%
\pgfpathlineto{\pgfqpoint{1.116022in}{1.837109in}}%
\pgfpathlineto{\pgfqpoint{1.116227in}{1.787599in}}%
\pgfpathlineto{\pgfqpoint{1.116431in}{1.771096in}}%
\pgfpathlineto{\pgfqpoint{1.116839in}{1.820606in}}%
\pgfpathlineto{\pgfqpoint{1.117043in}{1.870115in}}%
\pgfpathlineto{\pgfqpoint{1.117656in}{1.754593in}}%
\pgfpathlineto{\pgfqpoint{1.118472in}{1.556554in}}%
\pgfpathlineto{\pgfqpoint{1.118676in}{1.705083in}}%
\pgfpathlineto{\pgfqpoint{1.119697in}{1.581309in}}%
\pgfpathlineto{\pgfqpoint{1.120105in}{1.614316in}}%
\pgfpathlineto{\pgfqpoint{1.121534in}{1.771096in}}%
\pgfpathlineto{\pgfqpoint{1.122351in}{1.507045in}}%
\pgfpathlineto{\pgfqpoint{1.122963in}{1.639071in}}%
\pgfpathlineto{\pgfqpoint{1.123167in}{1.639071in}}%
\pgfpathlineto{\pgfqpoint{1.125005in}{1.779348in}}%
\pgfpathlineto{\pgfqpoint{1.125413in}{1.820606in}}%
\pgfpathlineto{\pgfqpoint{1.125617in}{1.787599in}}%
\pgfpathlineto{\pgfqpoint{1.126434in}{1.705083in}}%
\pgfpathlineto{\pgfqpoint{1.127046in}{1.738090in}}%
\pgfpathlineto{\pgfqpoint{1.127659in}{1.696832in}}%
\pgfpathlineto{\pgfqpoint{1.128067in}{1.647322in}}%
\pgfpathlineto{\pgfqpoint{1.128679in}{1.705083in}}%
\pgfpathlineto{\pgfqpoint{1.129496in}{1.680329in}}%
\pgfpathlineto{\pgfqpoint{1.130108in}{1.779348in}}%
\pgfpathlineto{\pgfqpoint{1.131946in}{1.647322in}}%
\pgfpathlineto{\pgfqpoint{1.132558in}{1.729838in}}%
\pgfpathlineto{\pgfqpoint{1.132966in}{1.639071in}}%
\pgfpathlineto{\pgfqpoint{1.133170in}{1.705083in}}%
\pgfpathlineto{\pgfqpoint{1.133579in}{1.540051in}}%
\pgfpathlineto{\pgfqpoint{1.134395in}{1.672077in}}%
\pgfpathlineto{\pgfqpoint{1.134599in}{1.663825in}}%
\pgfpathlineto{\pgfqpoint{1.134804in}{1.738090in}}%
\pgfpathlineto{\pgfqpoint{1.135008in}{1.622567in}}%
\pgfpathlineto{\pgfqpoint{1.135620in}{1.663825in}}%
\pgfpathlineto{\pgfqpoint{1.135824in}{1.647322in}}%
\pgfpathlineto{\pgfqpoint{1.136028in}{1.721587in}}%
\pgfpathlineto{\pgfqpoint{1.136641in}{1.639071in}}%
\pgfpathlineto{\pgfqpoint{1.137866in}{1.556554in}}%
\pgfpathlineto{\pgfqpoint{1.138070in}{1.556554in}}%
\pgfpathlineto{\pgfqpoint{1.138478in}{1.465787in}}%
\pgfpathlineto{\pgfqpoint{1.139091in}{1.490542in}}%
\pgfpathlineto{\pgfqpoint{1.139295in}{1.597813in}}%
\pgfpathlineto{\pgfqpoint{1.140111in}{1.531800in}}%
\pgfpathlineto{\pgfqpoint{1.140315in}{1.498793in}}%
\pgfpathlineto{\pgfqpoint{1.140520in}{1.589561in}}%
\pgfpathlineto{\pgfqpoint{1.140928in}{1.523548in}}%
\pgfpathlineto{\pgfqpoint{1.142153in}{1.672077in}}%
\pgfpathlineto{\pgfqpoint{1.143582in}{1.540051in}}%
\pgfpathlineto{\pgfqpoint{1.143990in}{1.531800in}}%
\pgfpathlineto{\pgfqpoint{1.144807in}{1.581309in}}%
\pgfpathlineto{\pgfqpoint{1.145623in}{1.383271in}}%
\pgfpathlineto{\pgfqpoint{1.146031in}{1.498793in}}%
\pgfpathlineto{\pgfqpoint{1.146644in}{1.399774in}}%
\pgfpathlineto{\pgfqpoint{1.147052in}{1.465787in}}%
\pgfpathlineto{\pgfqpoint{1.147460in}{1.449284in}}%
\pgfpathlineto{\pgfqpoint{1.147665in}{1.490542in}}%
\pgfpathlineto{\pgfqpoint{1.148073in}{1.375019in}}%
\pgfpathlineto{\pgfqpoint{1.148481in}{1.457535in}}%
\pgfpathlineto{\pgfqpoint{1.149706in}{1.383271in}}%
\pgfpathlineto{\pgfqpoint{1.150318in}{1.424529in}}%
\pgfpathlineto{\pgfqpoint{1.150931in}{1.399774in}}%
\pgfpathlineto{\pgfqpoint{1.151135in}{1.408026in}}%
\pgfpathlineto{\pgfqpoint{1.151952in}{1.540051in}}%
\pgfpathlineto{\pgfqpoint{1.152564in}{1.507045in}}%
\pgfpathlineto{\pgfqpoint{1.153176in}{1.531800in}}%
\pgfpathlineto{\pgfqpoint{1.153789in}{1.432780in}}%
\pgfpathlineto{\pgfqpoint{1.153993in}{1.432780in}}%
\pgfpathlineto{\pgfqpoint{1.154401in}{1.523548in}}%
\pgfpathlineto{\pgfqpoint{1.154605in}{1.267748in}}%
\pgfpathlineto{\pgfqpoint{1.155014in}{1.606064in}}%
\pgfpathlineto{\pgfqpoint{1.155422in}{1.498793in}}%
\pgfpathlineto{\pgfqpoint{1.156239in}{1.589561in}}%
\pgfpathlineto{\pgfqpoint{1.156647in}{1.540051in}}%
\pgfpathlineto{\pgfqpoint{1.156851in}{1.490542in}}%
\pgfpathlineto{\pgfqpoint{1.157055in}{1.564806in}}%
\pgfpathlineto{\pgfqpoint{1.157259in}{1.523548in}}%
\pgfpathlineto{\pgfqpoint{1.157463in}{1.630819in}}%
\pgfpathlineto{\pgfqpoint{1.158280in}{1.540051in}}%
\pgfpathlineto{\pgfqpoint{1.159097in}{1.556554in}}%
\pgfpathlineto{\pgfqpoint{1.160321in}{1.408026in}}%
\pgfpathlineto{\pgfqpoint{1.160525in}{1.449284in}}%
\pgfpathlineto{\pgfqpoint{1.160934in}{1.457535in}}%
\pgfpathlineto{\pgfqpoint{1.162159in}{1.581309in}}%
\pgfpathlineto{\pgfqpoint{1.162363in}{1.573058in}}%
\pgfpathlineto{\pgfqpoint{1.162771in}{1.523548in}}%
\pgfpathlineto{\pgfqpoint{1.163179in}{1.597813in}}%
\pgfpathlineto{\pgfqpoint{1.163588in}{1.581309in}}%
\pgfpathlineto{\pgfqpoint{1.165629in}{1.804103in}}%
\pgfpathlineto{\pgfqpoint{1.166446in}{1.630819in}}%
\pgfpathlineto{\pgfqpoint{1.166854in}{1.738090in}}%
\pgfpathlineto{\pgfqpoint{1.167058in}{1.696832in}}%
\pgfpathlineto{\pgfqpoint{1.167670in}{1.416277in}}%
\pgfpathlineto{\pgfqpoint{1.168079in}{1.663825in}}%
\pgfpathlineto{\pgfqpoint{1.168487in}{1.688580in}}%
\pgfpathlineto{\pgfqpoint{1.169304in}{1.589561in}}%
\pgfpathlineto{\pgfqpoint{1.170120in}{1.630819in}}%
\pgfpathlineto{\pgfqpoint{1.170528in}{1.606064in}}%
\pgfpathlineto{\pgfqpoint{1.170733in}{1.366768in}}%
\pgfpathlineto{\pgfqpoint{1.171345in}{1.771096in}}%
\pgfpathlineto{\pgfqpoint{1.172774in}{1.647322in}}%
\pgfpathlineto{\pgfqpoint{1.172978in}{1.696832in}}%
\pgfpathlineto{\pgfqpoint{1.173182in}{1.672077in}}%
\pgfpathlineto{\pgfqpoint{1.173386in}{1.432780in}}%
\pgfpathlineto{\pgfqpoint{1.174203in}{1.630819in}}%
\pgfpathlineto{\pgfqpoint{1.174407in}{1.639071in}}%
\pgfpathlineto{\pgfqpoint{1.175020in}{1.531800in}}%
\pgfpathlineto{\pgfqpoint{1.175428in}{1.589561in}}%
\pgfpathlineto{\pgfqpoint{1.175836in}{1.655574in}}%
\pgfpathlineto{\pgfqpoint{1.176040in}{1.606064in}}%
\pgfpathlineto{\pgfqpoint{1.177878in}{1.457535in}}%
\pgfpathlineto{\pgfqpoint{1.179307in}{1.589561in}}%
\pgfpathlineto{\pgfqpoint{1.179919in}{1.581309in}}%
\pgfpathlineto{\pgfqpoint{1.180940in}{1.333761in}}%
\pgfpathlineto{\pgfqpoint{1.181144in}{1.366768in}}%
\pgfpathlineto{\pgfqpoint{1.181348in}{1.606064in}}%
\pgfpathlineto{\pgfqpoint{1.182369in}{1.548303in}}%
\pgfpathlineto{\pgfqpoint{1.183389in}{1.639071in}}%
\pgfpathlineto{\pgfqpoint{1.182981in}{1.540051in}}%
\pgfpathlineto{\pgfqpoint{1.183594in}{1.573058in}}%
\pgfpathlineto{\pgfqpoint{1.184410in}{1.548303in}}%
\pgfpathlineto{\pgfqpoint{1.185431in}{1.622567in}}%
\pgfpathlineto{\pgfqpoint{1.185635in}{1.589561in}}%
\pgfpathlineto{\pgfqpoint{1.186452in}{1.515296in}}%
\pgfpathlineto{\pgfqpoint{1.186656in}{1.581309in}}%
\pgfpathlineto{\pgfqpoint{1.187064in}{1.523548in}}%
\pgfpathlineto{\pgfqpoint{1.187268in}{1.432780in}}%
\pgfpathlineto{\pgfqpoint{1.188085in}{1.523548in}}%
\pgfpathlineto{\pgfqpoint{1.188289in}{1.531800in}}%
\pgfpathlineto{\pgfqpoint{1.188493in}{1.366768in}}%
\pgfpathlineto{\pgfqpoint{1.188697in}{1.564806in}}%
\pgfpathlineto{\pgfqpoint{1.189310in}{1.556554in}}%
\pgfpathlineto{\pgfqpoint{1.189514in}{1.474038in}}%
\pgfpathlineto{\pgfqpoint{1.189718in}{1.606064in}}%
\pgfpathlineto{\pgfqpoint{1.190330in}{1.581309in}}%
\pgfpathlineto{\pgfqpoint{1.191759in}{1.498793in}}%
\pgfpathlineto{\pgfqpoint{1.192576in}{1.597813in}}%
\pgfpathlineto{\pgfqpoint{1.192168in}{1.482290in}}%
\pgfpathlineto{\pgfqpoint{1.192780in}{1.564806in}}%
\pgfpathlineto{\pgfqpoint{1.192984in}{1.474038in}}%
\pgfpathlineto{\pgfqpoint{1.194005in}{1.498793in}}%
\pgfpathlineto{\pgfqpoint{1.194209in}{1.498793in}}%
\pgfpathlineto{\pgfqpoint{1.194617in}{1.531800in}}%
\pgfpathlineto{\pgfqpoint{1.194821in}{1.474038in}}%
\pgfpathlineto{\pgfqpoint{1.195026in}{1.474038in}}%
\pgfpathlineto{\pgfqpoint{1.195230in}{1.474038in}}%
\pgfpathlineto{\pgfqpoint{1.195434in}{1.457535in}}%
\pgfpathlineto{\pgfqpoint{1.195638in}{1.284252in}}%
\pgfpathlineto{\pgfqpoint{1.196046in}{1.606064in}}%
\pgfpathlineto{\pgfqpoint{1.196455in}{1.490542in}}%
\pgfpathlineto{\pgfqpoint{1.197475in}{1.573058in}}%
\pgfpathlineto{\pgfqpoint{1.198496in}{1.556554in}}%
\pgfpathlineto{\pgfqpoint{1.198700in}{1.647322in}}%
\pgfpathlineto{\pgfqpoint{1.199721in}{1.630819in}}%
\pgfpathlineto{\pgfqpoint{1.199925in}{1.622567in}}%
\pgfpathlineto{\pgfqpoint{1.200129in}{1.515296in}}%
\pgfpathlineto{\pgfqpoint{1.200946in}{1.573058in}}%
\pgfpathlineto{\pgfqpoint{1.201150in}{1.573058in}}%
\pgfpathlineto{\pgfqpoint{1.201762in}{1.498793in}}%
\pgfpathlineto{\pgfqpoint{1.202171in}{1.540051in}}%
\pgfpathlineto{\pgfqpoint{1.203804in}{1.688580in}}%
\pgfpathlineto{\pgfqpoint{1.204008in}{1.647322in}}%
\pgfpathlineto{\pgfqpoint{1.205029in}{1.474038in}}%
\pgfpathlineto{\pgfqpoint{1.205233in}{1.556554in}}%
\pgfpathlineto{\pgfqpoint{1.205641in}{1.581309in}}%
\pgfpathlineto{\pgfqpoint{1.206049in}{1.540051in}}%
\pgfpathlineto{\pgfqpoint{1.206866in}{1.606064in}}%
\pgfpathlineto{\pgfqpoint{1.207478in}{1.366768in}}%
\pgfpathlineto{\pgfqpoint{1.207682in}{1.300755in}}%
\pgfpathlineto{\pgfqpoint{1.207887in}{1.564806in}}%
\pgfpathlineto{\pgfqpoint{1.208295in}{1.432780in}}%
\pgfpathlineto{\pgfqpoint{1.208499in}{1.416277in}}%
\pgfpathlineto{\pgfqpoint{1.208907in}{1.176981in}}%
\pgfpathlineto{\pgfqpoint{1.209520in}{1.416277in}}%
\pgfpathlineto{\pgfqpoint{1.209724in}{1.441032in}}%
\pgfpathlineto{\pgfqpoint{1.211153in}{1.218239in}}%
\pgfpathlineto{\pgfqpoint{1.211561in}{1.102716in}}%
\pgfpathlineto{\pgfqpoint{1.212174in}{1.143974in}}%
\pgfpathlineto{\pgfqpoint{1.212990in}{1.086213in}}%
\pgfpathlineto{\pgfqpoint{1.213194in}{1.259497in}}%
\pgfpathlineto{\pgfqpoint{1.213603in}{1.086213in}}%
\pgfpathlineto{\pgfqpoint{1.214419in}{1.127471in}}%
\pgfpathlineto{\pgfqpoint{1.216052in}{1.284252in}}%
\pgfpathlineto{\pgfqpoint{1.217481in}{1.135723in}}%
\pgfpathlineto{\pgfqpoint{1.217685in}{1.193484in}}%
\pgfpathlineto{\pgfqpoint{1.218298in}{1.077961in}}%
\pgfpathlineto{\pgfqpoint{1.219319in}{1.127471in}}%
\pgfpathlineto{\pgfqpoint{1.218706in}{1.053207in}}%
\pgfpathlineto{\pgfqpoint{1.219727in}{1.119219in}}%
\pgfpathlineto{\pgfqpoint{1.219931in}{1.110968in}}%
\pgfpathlineto{\pgfqpoint{1.221360in}{0.822162in}}%
\pgfpathlineto{\pgfqpoint{1.222177in}{1.053207in}}%
\pgfpathlineto{\pgfqpoint{1.222585in}{1.036703in}}%
\pgfpathlineto{\pgfqpoint{1.222789in}{1.036703in}}%
\pgfpathlineto{\pgfqpoint{1.222993in}{1.011949in}}%
\pgfpathlineto{\pgfqpoint{1.223401in}{1.168729in}}%
\pgfpathlineto{\pgfqpoint{1.224014in}{1.061458in}}%
\pgfpathlineto{\pgfqpoint{1.225239in}{0.978942in}}%
\pgfpathlineto{\pgfqpoint{1.226055in}{1.276000in}}%
\pgfpathlineto{\pgfqpoint{1.226464in}{1.168729in}}%
\pgfpathlineto{\pgfqpoint{1.227688in}{1.011949in}}%
\pgfpathlineto{\pgfqpoint{1.227893in}{1.036703in}}%
\pgfpathlineto{\pgfqpoint{1.228709in}{0.838665in}}%
\pgfpathlineto{\pgfqpoint{1.229117in}{0.912929in}}%
\pgfpathlineto{\pgfqpoint{1.229526in}{0.863420in}}%
\pgfpathlineto{\pgfqpoint{1.230342in}{0.879923in}}%
\pgfpathlineto{\pgfqpoint{1.230546in}{0.945936in}}%
\pgfpathlineto{\pgfqpoint{1.231567in}{0.937684in}}%
\pgfpathlineto{\pgfqpoint{1.231771in}{0.896426in}}%
\pgfpathlineto{\pgfqpoint{1.232384in}{0.987194in}}%
\pgfpathlineto{\pgfqpoint{1.232588in}{0.954187in}}%
\pgfpathlineto{\pgfqpoint{1.233609in}{1.069710in}}%
\pgfpathlineto{\pgfqpoint{1.234221in}{1.044955in}}%
\pgfpathlineto{\pgfqpoint{1.234629in}{0.937684in}}%
\pgfpathlineto{\pgfqpoint{1.235242in}{1.053207in}}%
\pgfpathlineto{\pgfqpoint{1.235854in}{1.020200in}}%
\pgfpathlineto{\pgfqpoint{1.236058in}{1.036703in}}%
\pgfpathlineto{\pgfqpoint{1.236262in}{1.102716in}}%
\pgfpathlineto{\pgfqpoint{1.236875in}{1.020200in}}%
\pgfpathlineto{\pgfqpoint{1.237079in}{1.020200in}}%
\pgfpathlineto{\pgfqpoint{1.237487in}{1.061458in}}%
\pgfpathlineto{\pgfqpoint{1.237691in}{1.011949in}}%
\pgfpathlineto{\pgfqpoint{1.239120in}{0.888175in}}%
\pgfpathlineto{\pgfqpoint{1.239529in}{1.028452in}}%
\pgfpathlineto{\pgfqpoint{1.240754in}{1.011949in}}%
\pgfpathlineto{\pgfqpoint{1.241570in}{0.912929in}}%
\pgfpathlineto{\pgfqpoint{1.241162in}{1.044955in}}%
\pgfpathlineto{\pgfqpoint{1.242183in}{0.945936in}}%
\pgfpathlineto{\pgfqpoint{1.243816in}{1.069710in}}%
\pgfpathlineto{\pgfqpoint{1.242591in}{0.937684in}}%
\pgfpathlineto{\pgfqpoint{1.244020in}{1.011949in}}%
\pgfpathlineto{\pgfqpoint{1.244224in}{1.020200in}}%
\pgfpathlineto{\pgfqpoint{1.244428in}{0.995445in}}%
\pgfpathlineto{\pgfqpoint{1.245653in}{1.176981in}}%
\pgfpathlineto{\pgfqpoint{1.245857in}{1.102716in}}%
\pgfpathlineto{\pgfqpoint{1.246265in}{1.284252in}}%
\pgfpathlineto{\pgfqpoint{1.246469in}{1.259497in}}%
\pgfpathlineto{\pgfqpoint{1.248715in}{1.515296in}}%
\pgfpathlineto{\pgfqpoint{1.250348in}{1.391522in}}%
\pgfpathlineto{\pgfqpoint{1.250552in}{1.449284in}}%
\pgfpathlineto{\pgfqpoint{1.250756in}{1.218239in}}%
\pgfpathlineto{\pgfqpoint{1.251573in}{1.507045in}}%
\pgfpathlineto{\pgfqpoint{1.252185in}{1.540051in}}%
\pgfpathlineto{\pgfqpoint{1.252594in}{1.515296in}}%
\pgfpathlineto{\pgfqpoint{1.253206in}{1.490542in}}%
\pgfpathlineto{\pgfqpoint{1.253410in}{1.523548in}}%
\pgfpathlineto{\pgfqpoint{1.253614in}{1.507045in}}%
\pgfpathlineto{\pgfqpoint{1.253819in}{1.548303in}}%
\pgfpathlineto{\pgfqpoint{1.254431in}{1.498793in}}%
\pgfpathlineto{\pgfqpoint{1.254839in}{1.531800in}}%
\pgfpathlineto{\pgfqpoint{1.255043in}{1.507045in}}%
\pgfpathlineto{\pgfqpoint{1.255248in}{1.606064in}}%
\pgfpathlineto{\pgfqpoint{1.255452in}{1.564806in}}%
\pgfpathlineto{\pgfqpoint{1.256268in}{1.738090in}}%
\pgfpathlineto{\pgfqpoint{1.257085in}{1.713335in}}%
\pgfpathlineto{\pgfqpoint{1.257289in}{1.688580in}}%
\pgfpathlineto{\pgfqpoint{1.257697in}{1.746341in}}%
\pgfpathlineto{\pgfqpoint{1.258310in}{1.853612in}}%
\pgfpathlineto{\pgfqpoint{1.258718in}{1.820606in}}%
\pgfpathlineto{\pgfqpoint{1.259330in}{1.738090in}}%
\pgfpathlineto{\pgfqpoint{1.259739in}{1.762845in}}%
\pgfpathlineto{\pgfqpoint{1.261168in}{1.894870in}}%
\pgfpathlineto{\pgfqpoint{1.262188in}{1.804103in}}%
\pgfpathlineto{\pgfqpoint{1.262393in}{1.812354in}}%
\pgfpathlineto{\pgfqpoint{1.263209in}{1.919625in}}%
\pgfpathlineto{\pgfqpoint{1.262801in}{1.804103in}}%
\pgfpathlineto{\pgfqpoint{1.263413in}{1.903122in}}%
\pgfpathlineto{\pgfqpoint{1.264638in}{1.820606in}}%
\pgfpathlineto{\pgfqpoint{1.265659in}{1.927877in}}%
\pgfpathlineto{\pgfqpoint{1.265251in}{1.804103in}}%
\pgfpathlineto{\pgfqpoint{1.265863in}{1.886619in}}%
\pgfpathlineto{\pgfqpoint{1.266067in}{1.861864in}}%
\pgfpathlineto{\pgfqpoint{1.266271in}{1.911373in}}%
\pgfpathlineto{\pgfqpoint{1.266680in}{1.911373in}}%
\pgfpathlineto{\pgfqpoint{1.266884in}{1.969135in}}%
\pgfpathlineto{\pgfqpoint{1.267292in}{1.886619in}}%
\pgfpathlineto{\pgfqpoint{1.267700in}{1.903122in}}%
\pgfpathlineto{\pgfqpoint{1.267904in}{1.911373in}}%
\pgfpathlineto{\pgfqpoint{1.268109in}{1.886619in}}%
\pgfpathlineto{\pgfqpoint{1.268313in}{1.894870in}}%
\pgfpathlineto{\pgfqpoint{1.268721in}{1.845361in}}%
\pgfpathlineto{\pgfqpoint{1.268925in}{1.903122in}}%
\pgfpathlineto{\pgfqpoint{1.269129in}{1.903122in}}%
\pgfpathlineto{\pgfqpoint{1.269742in}{2.051651in}}%
\pgfpathlineto{\pgfqpoint{1.270558in}{2.002141in}}%
\pgfpathlineto{\pgfqpoint{1.270762in}{2.018644in}}%
\pgfpathlineto{\pgfqpoint{1.270967in}{1.993890in}}%
\pgfpathlineto{\pgfqpoint{1.271171in}{1.927877in}}%
\pgfpathlineto{\pgfqpoint{1.271987in}{2.018644in}}%
\pgfpathlineto{\pgfqpoint{1.272396in}{1.977386in}}%
\pgfpathlineto{\pgfqpoint{1.273008in}{2.035148in}}%
\pgfpathlineto{\pgfqpoint{1.273212in}{2.051651in}}%
\pgfpathlineto{\pgfqpoint{1.274029in}{1.944380in}}%
\pgfpathlineto{\pgfqpoint{1.274233in}{2.002141in}}%
\pgfpathlineto{\pgfqpoint{1.275049in}{2.101160in}}%
\pgfpathlineto{\pgfqpoint{1.275458in}{2.084657in}}%
\pgfpathlineto{\pgfqpoint{1.276070in}{2.109412in}}%
\pgfpathlineto{\pgfqpoint{1.276683in}{2.018644in}}%
\pgfpathlineto{\pgfqpoint{1.276887in}{2.084657in}}%
\pgfpathlineto{\pgfqpoint{1.277499in}{1.977386in}}%
\pgfpathlineto{\pgfqpoint{1.277703in}{2.035148in}}%
\pgfpathlineto{\pgfqpoint{1.277907in}{2.018644in}}%
\pgfpathlineto{\pgfqpoint{1.278112in}{2.043399in}}%
\pgfpathlineto{\pgfqpoint{1.279336in}{2.109412in}}%
\pgfpathlineto{\pgfqpoint{1.279541in}{2.109412in}}%
\pgfpathlineto{\pgfqpoint{1.279745in}{2.183676in}}%
\pgfpathlineto{\pgfqpoint{1.280561in}{2.150670in}}%
\pgfpathlineto{\pgfqpoint{1.280765in}{2.150670in}}%
\pgfpathlineto{\pgfqpoint{1.280970in}{2.191928in}}%
\pgfpathlineto{\pgfqpoint{1.281378in}{2.142418in}}%
\pgfpathlineto{\pgfqpoint{1.282603in}{1.886619in}}%
\pgfpathlineto{\pgfqpoint{1.283828in}{2.158922in}}%
\pgfpathlineto{\pgfqpoint{1.284644in}{1.927877in}}%
\pgfpathlineto{\pgfqpoint{1.285052in}{2.117664in}}%
\pgfpathlineto{\pgfqpoint{1.286073in}{2.191928in}}%
\pgfpathlineto{\pgfqpoint{1.286277in}{2.142418in}}%
\pgfpathlineto{\pgfqpoint{1.286890in}{2.241438in}}%
\pgfpathlineto{\pgfqpoint{1.287094in}{2.274444in}}%
\pgfpathlineto{\pgfqpoint{1.287298in}{2.208431in}}%
\pgfpathlineto{\pgfqpoint{1.287910in}{2.241438in}}%
\pgfpathlineto{\pgfqpoint{1.288115in}{2.241438in}}%
\pgfpathlineto{\pgfqpoint{1.288931in}{2.101160in}}%
\pgfpathlineto{\pgfqpoint{1.289544in}{2.175425in}}%
\pgfpathlineto{\pgfqpoint{1.289748in}{2.183676in}}%
\pgfpathlineto{\pgfqpoint{1.289952in}{2.158922in}}%
\pgfpathlineto{\pgfqpoint{1.290360in}{2.043399in}}%
\pgfpathlineto{\pgfqpoint{1.291177in}{2.076406in}}%
\pgfpathlineto{\pgfqpoint{1.291381in}{2.076406in}}%
\pgfpathlineto{\pgfqpoint{1.291585in}{2.068154in}}%
\pgfpathlineto{\pgfqpoint{1.291789in}{2.092909in}}%
\pgfpathlineto{\pgfqpoint{1.292197in}{2.158922in}}%
\pgfpathlineto{\pgfqpoint{1.292402in}{2.101160in}}%
\pgfpathlineto{\pgfqpoint{1.292810in}{2.018644in}}%
\pgfpathlineto{\pgfqpoint{1.293422in}{2.125915in}}%
\pgfpathlineto{\pgfqpoint{1.294035in}{2.068154in}}%
\pgfpathlineto{\pgfqpoint{1.294239in}{2.142418in}}%
\pgfpathlineto{\pgfqpoint{1.294443in}{2.109412in}}%
\pgfpathlineto{\pgfqpoint{1.295055in}{2.142418in}}%
\pgfpathlineto{\pgfqpoint{1.295260in}{2.092909in}}%
\pgfpathlineto{\pgfqpoint{1.295464in}{2.101160in}}%
\pgfpathlineto{\pgfqpoint{1.295668in}{2.068154in}}%
\pgfpathlineto{\pgfqpoint{1.296484in}{2.043399in}}%
\pgfpathlineto{\pgfqpoint{1.297301in}{2.109412in}}%
\pgfpathlineto{\pgfqpoint{1.297709in}{1.985638in}}%
\pgfpathlineto{\pgfqpoint{1.298526in}{2.051651in}}%
\pgfpathlineto{\pgfqpoint{1.299138in}{2.002141in}}%
\pgfpathlineto{\pgfqpoint{1.299342in}{2.084657in}}%
\pgfpathlineto{\pgfqpoint{1.299751in}{2.117664in}}%
\pgfpathlineto{\pgfqpoint{1.300159in}{2.084657in}}%
\pgfpathlineto{\pgfqpoint{1.300976in}{2.101160in}}%
\pgfpathlineto{\pgfqpoint{1.301384in}{1.993890in}}%
\pgfpathlineto{\pgfqpoint{1.302405in}{2.142418in}}%
\pgfpathlineto{\pgfqpoint{1.302609in}{2.117664in}}%
\pgfpathlineto{\pgfqpoint{1.302813in}{2.043399in}}%
\pgfpathlineto{\pgfqpoint{1.303425in}{2.150670in}}%
\pgfpathlineto{\pgfqpoint{1.303834in}{2.068154in}}%
\pgfpathlineto{\pgfqpoint{1.304242in}{2.167173in}}%
\pgfpathlineto{\pgfqpoint{1.304854in}{2.125915in}}%
\pgfpathlineto{\pgfqpoint{1.306079in}{2.051651in}}%
\pgfpathlineto{\pgfqpoint{1.306487in}{2.068154in}}%
\pgfpathlineto{\pgfqpoint{1.306692in}{2.035148in}}%
\pgfpathlineto{\pgfqpoint{1.306896in}{2.002141in}}%
\pgfpathlineto{\pgfqpoint{1.307100in}{2.117664in}}%
\pgfpathlineto{\pgfqpoint{1.307304in}{2.092909in}}%
\pgfpathlineto{\pgfqpoint{1.307916in}{2.150670in}}%
\pgfpathlineto{\pgfqpoint{1.308121in}{2.125915in}}%
\pgfpathlineto{\pgfqpoint{1.309141in}{2.043399in}}%
\pgfpathlineto{\pgfqpoint{1.309345in}{2.051651in}}%
\pgfpathlineto{\pgfqpoint{1.309550in}{2.043399in}}%
\pgfpathlineto{\pgfqpoint{1.309754in}{2.059902in}}%
\pgfpathlineto{\pgfqpoint{1.309958in}{2.117664in}}%
\pgfpathlineto{\pgfqpoint{1.310570in}{2.026896in}}%
\pgfpathlineto{\pgfqpoint{1.310979in}{1.993890in}}%
\pgfpathlineto{\pgfqpoint{1.311387in}{2.035148in}}%
\pgfpathlineto{\pgfqpoint{1.311591in}{2.043399in}}%
\pgfpathlineto{\pgfqpoint{1.312612in}{2.142418in}}%
\pgfpathlineto{\pgfqpoint{1.312816in}{2.117664in}}%
\pgfpathlineto{\pgfqpoint{1.313020in}{2.117664in}}%
\pgfpathlineto{\pgfqpoint{1.314857in}{1.985638in}}%
\pgfpathlineto{\pgfqpoint{1.316286in}{2.109412in}}%
\pgfpathlineto{\pgfqpoint{1.316490in}{2.092909in}}%
\pgfpathlineto{\pgfqpoint{1.316695in}{2.125915in}}%
\pgfpathlineto{\pgfqpoint{1.317307in}{2.109412in}}%
\pgfpathlineto{\pgfqpoint{1.317715in}{2.200180in}}%
\pgfpathlineto{\pgfqpoint{1.318124in}{2.257941in}}%
\pgfpathlineto{\pgfqpoint{1.318940in}{2.101160in}}%
\pgfpathlineto{\pgfqpoint{1.319144in}{2.158922in}}%
\pgfpathlineto{\pgfqpoint{1.320165in}{2.134167in}}%
\pgfpathlineto{\pgfqpoint{1.320982in}{2.051651in}}%
\pgfpathlineto{\pgfqpoint{1.320777in}{2.142418in}}%
\pgfpathlineto{\pgfqpoint{1.321390in}{2.084657in}}%
\pgfpathlineto{\pgfqpoint{1.321594in}{2.084657in}}%
\pgfpathlineto{\pgfqpoint{1.323023in}{2.241438in}}%
\pgfpathlineto{\pgfqpoint{1.323635in}{2.282696in}}%
\pgfpathlineto{\pgfqpoint{1.324452in}{2.101160in}}%
\pgfpathlineto{\pgfqpoint{1.324656in}{2.076406in}}%
\pgfpathlineto{\pgfqpoint{1.324860in}{2.142418in}}%
\pgfpathlineto{\pgfqpoint{1.325269in}{2.125915in}}%
\pgfpathlineto{\pgfqpoint{1.327718in}{2.332205in}}%
\pgfpathlineto{\pgfqpoint{1.327922in}{2.282696in}}%
\pgfpathlineto{\pgfqpoint{1.329556in}{2.158922in}}%
\pgfpathlineto{\pgfqpoint{1.330168in}{2.233186in}}%
\pgfpathlineto{\pgfqpoint{1.330576in}{2.150670in}}%
\pgfpathlineto{\pgfqpoint{1.330780in}{2.200180in}}%
\pgfpathlineto{\pgfqpoint{1.332413in}{1.977386in}}%
\pgfpathlineto{\pgfqpoint{1.333026in}{2.233186in}}%
\pgfpathlineto{\pgfqpoint{1.333638in}{2.150670in}}%
\pgfpathlineto{\pgfqpoint{1.334251in}{2.076406in}}%
\pgfpathlineto{\pgfqpoint{1.334455in}{2.167173in}}%
\pgfpathlineto{\pgfqpoint{1.334659in}{2.167173in}}%
\pgfpathlineto{\pgfqpoint{1.335884in}{2.266192in}}%
\pgfpathlineto{\pgfqpoint{1.337109in}{2.167173in}}%
\pgfpathlineto{\pgfqpoint{1.337313in}{2.175425in}}%
\pgfpathlineto{\pgfqpoint{1.337517in}{2.200180in}}%
\pgfpathlineto{\pgfqpoint{1.337721in}{1.927877in}}%
\pgfpathlineto{\pgfqpoint{1.338538in}{2.150670in}}%
\pgfpathlineto{\pgfqpoint{1.339150in}{2.092909in}}%
\pgfpathlineto{\pgfqpoint{1.339763in}{2.101160in}}%
\pgfpathlineto{\pgfqpoint{1.340783in}{2.158922in}}%
\pgfpathlineto{\pgfqpoint{1.340987in}{2.142418in}}%
\pgfpathlineto{\pgfqpoint{1.343029in}{2.274444in}}%
\pgfpathlineto{\pgfqpoint{1.344050in}{2.158922in}}%
\pgfpathlineto{\pgfqpoint{1.344254in}{2.233186in}}%
\pgfpathlineto{\pgfqpoint{1.344866in}{2.340457in}}%
\pgfpathlineto{\pgfqpoint{1.345070in}{2.290947in}}%
\pgfpathlineto{\pgfqpoint{1.346499in}{2.101160in}}%
\pgfpathlineto{\pgfqpoint{1.346703in}{2.092909in}}%
\pgfpathlineto{\pgfqpoint{1.346908in}{2.109412in}}%
\pgfpathlineto{\pgfqpoint{1.347112in}{2.101160in}}%
\pgfpathlineto{\pgfqpoint{1.348337in}{2.200180in}}%
\pgfpathlineto{\pgfqpoint{1.347928in}{2.076406in}}%
\pgfpathlineto{\pgfqpoint{1.348745in}{2.191928in}}%
\pgfpathlineto{\pgfqpoint{1.349153in}{2.134167in}}%
\pgfpathlineto{\pgfqpoint{1.349766in}{2.018644in}}%
\pgfpathlineto{\pgfqpoint{1.349970in}{2.084657in}}%
\pgfpathlineto{\pgfqpoint{1.351195in}{2.183676in}}%
\pgfpathlineto{\pgfqpoint{1.352419in}{2.117664in}}%
\pgfpathlineto{\pgfqpoint{1.353032in}{2.158922in}}%
\pgfpathlineto{\pgfqpoint{1.353440in}{2.150670in}}%
\pgfpathlineto{\pgfqpoint{1.353644in}{2.117664in}}%
\pgfpathlineto{\pgfqpoint{1.353848in}{2.200180in}}%
\pgfpathlineto{\pgfqpoint{1.354053in}{2.183676in}}%
\pgfpathlineto{\pgfqpoint{1.354257in}{2.200180in}}%
\pgfpathlineto{\pgfqpoint{1.354461in}{2.183676in}}%
\pgfpathlineto{\pgfqpoint{1.355277in}{2.101160in}}%
\pgfpathlineto{\pgfqpoint{1.355686in}{2.125915in}}%
\pgfpathlineto{\pgfqpoint{1.355890in}{2.134167in}}%
\pgfpathlineto{\pgfqpoint{1.356094in}{2.117664in}}%
\pgfpathlineto{\pgfqpoint{1.356298in}{2.092909in}}%
\pgfpathlineto{\pgfqpoint{1.356502in}{2.134167in}}%
\pgfpathlineto{\pgfqpoint{1.357115in}{2.109412in}}%
\pgfpathlineto{\pgfqpoint{1.357319in}{2.150670in}}%
\pgfpathlineto{\pgfqpoint{1.357931in}{2.092909in}}%
\pgfpathlineto{\pgfqpoint{1.358135in}{2.125915in}}%
\pgfpathlineto{\pgfqpoint{1.358544in}{2.043399in}}%
\pgfpathlineto{\pgfqpoint{1.358748in}{2.092909in}}%
\pgfpathlineto{\pgfqpoint{1.359769in}{2.175425in}}%
\pgfpathlineto{\pgfqpoint{1.360381in}{2.109412in}}%
\pgfpathlineto{\pgfqpoint{1.360789in}{2.150670in}}%
\pgfpathlineto{\pgfqpoint{1.361606in}{2.183676in}}%
\pgfpathlineto{\pgfqpoint{1.362831in}{2.084657in}}%
\pgfpathlineto{\pgfqpoint{1.363443in}{2.142418in}}%
\pgfpathlineto{\pgfqpoint{1.363239in}{2.068154in}}%
\pgfpathlineto{\pgfqpoint{1.363647in}{2.068154in}}%
\pgfpathlineto{\pgfqpoint{1.363851in}{1.960883in}}%
\pgfpathlineto{\pgfqpoint{1.364056in}{2.142418in}}%
\pgfpathlineto{\pgfqpoint{1.364464in}{2.101160in}}%
\pgfpathlineto{\pgfqpoint{1.365485in}{2.266192in}}%
\pgfpathlineto{\pgfqpoint{1.366301in}{2.191928in}}%
\pgfpathlineto{\pgfqpoint{1.366709in}{2.224934in}}%
\pgfpathlineto{\pgfqpoint{1.367730in}{1.952632in}}%
\pgfpathlineto{\pgfqpoint{1.368751in}{2.241438in}}%
\pgfpathlineto{\pgfqpoint{1.368955in}{2.191928in}}%
\pgfpathlineto{\pgfqpoint{1.369159in}{2.158922in}}%
\pgfpathlineto{\pgfqpoint{1.369567in}{2.208431in}}%
\pgfpathlineto{\pgfqpoint{1.369772in}{2.191928in}}%
\pgfpathlineto{\pgfqpoint{1.369976in}{2.249689in}}%
\pgfpathlineto{\pgfqpoint{1.370792in}{2.224934in}}%
\pgfpathlineto{\pgfqpoint{1.371405in}{2.142418in}}%
\pgfpathlineto{\pgfqpoint{1.371201in}{2.241438in}}%
\pgfpathlineto{\pgfqpoint{1.371813in}{2.167173in}}%
\pgfpathlineto{\pgfqpoint{1.372425in}{2.282696in}}%
\pgfpathlineto{\pgfqpoint{1.372834in}{2.158922in}}%
\pgfpathlineto{\pgfqpoint{1.373446in}{2.059902in}}%
\pgfpathlineto{\pgfqpoint{1.374263in}{2.068154in}}%
\pgfpathlineto{\pgfqpoint{1.374671in}{2.035148in}}%
\pgfpathlineto{\pgfqpoint{1.375692in}{2.134167in}}%
\pgfpathlineto{\pgfqpoint{1.377121in}{2.059902in}}%
\pgfpathlineto{\pgfqpoint{1.377325in}{1.870115in}}%
\pgfpathlineto{\pgfqpoint{1.377937in}{2.117664in}}%
\pgfpathlineto{\pgfqpoint{1.378141in}{2.068154in}}%
\pgfpathlineto{\pgfqpoint{1.378754in}{2.175425in}}%
\pgfpathlineto{\pgfqpoint{1.379162in}{2.117664in}}%
\pgfpathlineto{\pgfqpoint{1.379775in}{2.092909in}}%
\pgfpathlineto{\pgfqpoint{1.379570in}{2.125915in}}%
\pgfpathlineto{\pgfqpoint{1.380183in}{2.109412in}}%
\pgfpathlineto{\pgfqpoint{1.380795in}{2.274444in}}%
\pgfpathlineto{\pgfqpoint{1.381204in}{2.175425in}}%
\pgfpathlineto{\pgfqpoint{1.382020in}{2.092909in}}%
\pgfpathlineto{\pgfqpoint{1.382224in}{2.158922in}}%
\pgfpathlineto{\pgfqpoint{1.382428in}{2.183676in}}%
\pgfpathlineto{\pgfqpoint{1.382633in}{2.084657in}}%
\pgfpathlineto{\pgfqpoint{1.382837in}{2.076406in}}%
\pgfpathlineto{\pgfqpoint{1.383653in}{2.010393in}}%
\pgfpathlineto{\pgfqpoint{1.383857in}{2.076406in}}%
\pgfpathlineto{\pgfqpoint{1.384062in}{2.101160in}}%
\pgfpathlineto{\pgfqpoint{1.384470in}{2.092909in}}%
\pgfpathlineto{\pgfqpoint{1.384674in}{2.010393in}}%
\pgfpathlineto{\pgfqpoint{1.384878in}{2.101160in}}%
\pgfpathlineto{\pgfqpoint{1.385491in}{2.101160in}}%
\pgfpathlineto{\pgfqpoint{1.386511in}{2.026896in}}%
\pgfpathlineto{\pgfqpoint{1.386920in}{2.035148in}}%
\pgfpathlineto{\pgfqpoint{1.387736in}{1.969135in}}%
\pgfpathlineto{\pgfqpoint{1.387940in}{2.002141in}}%
\pgfpathlineto{\pgfqpoint{1.388144in}{2.051651in}}%
\pgfpathlineto{\pgfqpoint{1.388553in}{1.894870in}}%
\pgfpathlineto{\pgfqpoint{1.391002in}{2.142418in}}%
\pgfpathlineto{\pgfqpoint{1.391615in}{2.059902in}}%
\pgfpathlineto{\pgfqpoint{1.392023in}{2.068154in}}%
\pgfpathlineto{\pgfqpoint{1.392636in}{2.142418in}}%
\pgfpathlineto{\pgfqpoint{1.393248in}{2.109412in}}%
\pgfpathlineto{\pgfqpoint{1.393656in}{2.150670in}}%
\pgfpathlineto{\pgfqpoint{1.393860in}{2.117664in}}%
\pgfpathlineto{\pgfqpoint{1.395085in}{2.043399in}}%
\pgfpathlineto{\pgfqpoint{1.396106in}{2.241438in}}%
\pgfpathlineto{\pgfqpoint{1.396718in}{2.191928in}}%
\pgfpathlineto{\pgfqpoint{1.396923in}{2.150670in}}%
\pgfpathlineto{\pgfqpoint{1.397535in}{2.158922in}}%
\pgfpathlineto{\pgfqpoint{1.398352in}{2.266192in}}%
\pgfpathlineto{\pgfqpoint{1.398760in}{2.257941in}}%
\pgfpathlineto{\pgfqpoint{1.399985in}{2.076406in}}%
\pgfpathlineto{\pgfqpoint{1.399168in}{2.266192in}}%
\pgfpathlineto{\pgfqpoint{1.400393in}{2.125915in}}%
\pgfpathlineto{\pgfqpoint{1.401414in}{2.257941in}}%
\pgfpathlineto{\pgfqpoint{1.401618in}{2.233186in}}%
\pgfpathlineto{\pgfqpoint{1.401822in}{2.249689in}}%
\pgfpathlineto{\pgfqpoint{1.402026in}{2.191928in}}%
\pgfpathlineto{\pgfqpoint{1.402230in}{2.191928in}}%
\pgfpathlineto{\pgfqpoint{1.403455in}{2.125915in}}%
\pgfpathlineto{\pgfqpoint{1.404272in}{2.241438in}}%
\pgfpathlineto{\pgfqpoint{1.404884in}{2.216683in}}%
\pgfpathlineto{\pgfqpoint{1.405292in}{2.158922in}}%
\pgfpathlineto{\pgfqpoint{1.405905in}{2.175425in}}%
\pgfpathlineto{\pgfqpoint{1.406926in}{2.241438in}}%
\pgfpathlineto{\pgfqpoint{1.407334in}{2.233186in}}%
\pgfpathlineto{\pgfqpoint{1.408150in}{2.158922in}}%
\pgfpathlineto{\pgfqpoint{1.408355in}{2.175425in}}%
\pgfpathlineto{\pgfqpoint{1.408559in}{2.241438in}}%
\pgfpathlineto{\pgfqpoint{1.408967in}{2.167173in}}%
\pgfpathlineto{\pgfqpoint{1.409171in}{2.167173in}}%
\pgfpathlineto{\pgfqpoint{1.409375in}{2.142418in}}%
\pgfpathlineto{\pgfqpoint{1.409579in}{2.183676in}}%
\pgfpathlineto{\pgfqpoint{1.409784in}{2.167173in}}%
\pgfpathlineto{\pgfqpoint{1.410600in}{2.224934in}}%
\pgfpathlineto{\pgfqpoint{1.411008in}{2.134167in}}%
\pgfpathlineto{\pgfqpoint{1.411417in}{2.158922in}}%
\pgfpathlineto{\pgfqpoint{1.411825in}{2.274444in}}%
\pgfpathlineto{\pgfqpoint{1.412437in}{2.257941in}}%
\pgfpathlineto{\pgfqpoint{1.412642in}{2.208431in}}%
\pgfpathlineto{\pgfqpoint{1.413050in}{2.274444in}}%
\pgfpathlineto{\pgfqpoint{1.413458in}{2.249689in}}%
\pgfpathlineto{\pgfqpoint{1.414887in}{2.150670in}}%
\pgfpathlineto{\pgfqpoint{1.415091in}{2.224934in}}%
\pgfpathlineto{\pgfqpoint{1.415908in}{2.200180in}}%
\pgfpathlineto{\pgfqpoint{1.416724in}{2.150670in}}%
\pgfpathlineto{\pgfqpoint{1.417337in}{2.224934in}}%
\pgfpathlineto{\pgfqpoint{1.417541in}{2.142418in}}%
\pgfpathlineto{\pgfqpoint{1.417745in}{2.183676in}}%
\pgfpathlineto{\pgfqpoint{1.417949in}{2.175425in}}%
\pgfpathlineto{\pgfqpoint{1.419582in}{2.299199in}}%
\pgfpathlineto{\pgfqpoint{1.419991in}{2.191928in}}%
\pgfpathlineto{\pgfqpoint{1.420603in}{2.224934in}}%
\pgfpathlineto{\pgfqpoint{1.421420in}{2.249689in}}%
\pgfpathlineto{\pgfqpoint{1.422440in}{2.175425in}}%
\pgfpathlineto{\pgfqpoint{1.422644in}{2.233186in}}%
\pgfpathlineto{\pgfqpoint{1.423257in}{2.158922in}}%
\pgfpathlineto{\pgfqpoint{1.423461in}{2.167173in}}%
\pgfpathlineto{\pgfqpoint{1.423869in}{2.200180in}}%
\pgfpathlineto{\pgfqpoint{1.424686in}{2.183676in}}%
\pgfpathlineto{\pgfqpoint{1.425094in}{2.125915in}}%
\pgfpathlineto{\pgfqpoint{1.425298in}{2.208431in}}%
\pgfpathlineto{\pgfqpoint{1.425707in}{2.200180in}}%
\pgfpathlineto{\pgfqpoint{1.426931in}{2.257941in}}%
\pgfpathlineto{\pgfqpoint{1.426523in}{2.183676in}}%
\pgfpathlineto{\pgfqpoint{1.427136in}{2.233186in}}%
\pgfpathlineto{\pgfqpoint{1.428769in}{2.142418in}}%
\pgfpathlineto{\pgfqpoint{1.429381in}{2.101160in}}%
\pgfpathlineto{\pgfqpoint{1.429994in}{2.216683in}}%
\pgfpathlineto{\pgfqpoint{1.431218in}{2.134167in}}%
\pgfpathlineto{\pgfqpoint{1.432035in}{2.233186in}}%
\pgfpathlineto{\pgfqpoint{1.432647in}{2.224934in}}%
\pgfpathlineto{\pgfqpoint{1.432852in}{2.191928in}}%
\pgfpathlineto{\pgfqpoint{1.433260in}{2.249689in}}%
\pgfpathlineto{\pgfqpoint{1.433668in}{2.282696in}}%
\pgfpathlineto{\pgfqpoint{1.434076in}{2.224934in}}%
\pgfpathlineto{\pgfqpoint{1.434281in}{2.241438in}}%
\pgfpathlineto{\pgfqpoint{1.434485in}{2.249689in}}%
\pgfpathlineto{\pgfqpoint{1.434689in}{2.233186in}}%
\pgfpathlineto{\pgfqpoint{1.434893in}{2.200180in}}%
\pgfpathlineto{\pgfqpoint{1.435505in}{2.257941in}}%
\pgfpathlineto{\pgfqpoint{1.435710in}{2.241438in}}%
\pgfpathlineto{\pgfqpoint{1.436118in}{2.290947in}}%
\pgfpathlineto{\pgfqpoint{1.436322in}{2.274444in}}%
\pgfpathlineto{\pgfqpoint{1.436526in}{2.216683in}}%
\pgfpathlineto{\pgfqpoint{1.437343in}{2.299199in}}%
\pgfpathlineto{\pgfqpoint{1.437547in}{2.307450in}}%
\pgfpathlineto{\pgfqpoint{1.437751in}{2.365212in}}%
\pgfpathlineto{\pgfqpoint{1.438363in}{2.257941in}}%
\pgfpathlineto{\pgfqpoint{1.438568in}{2.315702in}}%
\pgfpathlineto{\pgfqpoint{1.438772in}{2.307450in}}%
\pgfpathlineto{\pgfqpoint{1.438976in}{2.315702in}}%
\pgfpathlineto{\pgfqpoint{1.439384in}{2.348709in}}%
\pgfpathlineto{\pgfqpoint{1.439997in}{2.315702in}}%
\pgfpathlineto{\pgfqpoint{1.440201in}{2.266192in}}%
\pgfpathlineto{\pgfqpoint{1.440609in}{2.381715in}}%
\pgfpathlineto{\pgfqpoint{1.440813in}{2.389967in}}%
\pgfpathlineto{\pgfqpoint{1.441017in}{2.282696in}}%
\pgfpathlineto{\pgfqpoint{1.442038in}{2.307450in}}%
\pgfpathlineto{\pgfqpoint{1.443263in}{2.381715in}}%
\pgfpathlineto{\pgfqpoint{1.444692in}{2.282696in}}%
\pgfpathlineto{\pgfqpoint{1.445304in}{2.249689in}}%
\pgfpathlineto{\pgfqpoint{1.445713in}{2.299199in}}%
\pgfpathlineto{\pgfqpoint{1.446121in}{2.241438in}}%
\pgfpathlineto{\pgfqpoint{1.446937in}{2.257941in}}%
\pgfpathlineto{\pgfqpoint{1.447142in}{2.249689in}}%
\pgfpathlineto{\pgfqpoint{1.447958in}{2.299199in}}%
\pgfpathlineto{\pgfqpoint{1.448162in}{2.266192in}}%
\pgfpathlineto{\pgfqpoint{1.448366in}{2.249689in}}%
\pgfpathlineto{\pgfqpoint{1.448571in}{2.282696in}}%
\pgfpathlineto{\pgfqpoint{1.448775in}{2.274444in}}%
\pgfpathlineto{\pgfqpoint{1.448979in}{2.307450in}}%
\pgfpathlineto{\pgfqpoint{1.449387in}{2.249689in}}%
\pgfpathlineto{\pgfqpoint{1.449795in}{2.274444in}}%
\pgfpathlineto{\pgfqpoint{1.450408in}{2.216683in}}%
\pgfpathlineto{\pgfqpoint{1.451020in}{2.249689in}}%
\pgfpathlineto{\pgfqpoint{1.451633in}{2.101160in}}%
\pgfpathlineto{\pgfqpoint{1.452449in}{2.191928in}}%
\pgfpathlineto{\pgfqpoint{1.452653in}{2.241438in}}%
\pgfpathlineto{\pgfqpoint{1.452858in}{2.183676in}}%
\pgfpathlineto{\pgfqpoint{1.453470in}{2.191928in}}%
\pgfpathlineto{\pgfqpoint{1.454695in}{2.299199in}}%
\pgfpathlineto{\pgfqpoint{1.455103in}{2.266192in}}%
\pgfpathlineto{\pgfqpoint{1.455511in}{2.274444in}}%
\pgfpathlineto{\pgfqpoint{1.456124in}{2.142418in}}%
\pgfpathlineto{\pgfqpoint{1.456736in}{2.340457in}}%
\pgfpathlineto{\pgfqpoint{1.457349in}{2.266192in}}%
\pgfpathlineto{\pgfqpoint{1.457553in}{2.266192in}}%
\pgfpathlineto{\pgfqpoint{1.457757in}{2.183676in}}%
\pgfpathlineto{\pgfqpoint{1.457961in}{2.307450in}}%
\pgfpathlineto{\pgfqpoint{1.458574in}{2.266192in}}%
\pgfpathlineto{\pgfqpoint{1.459594in}{2.356960in}}%
\pgfpathlineto{\pgfqpoint{1.460207in}{2.233186in}}%
\pgfpathlineto{\pgfqpoint{1.460819in}{2.290947in}}%
\pgfpathlineto{\pgfqpoint{1.461227in}{2.224934in}}%
\pgfpathlineto{\pgfqpoint{1.461636in}{2.299199in}}%
\pgfpathlineto{\pgfqpoint{1.462044in}{2.241438in}}%
\pgfpathlineto{\pgfqpoint{1.462656in}{2.183676in}}%
\pgfpathlineto{\pgfqpoint{1.462452in}{2.274444in}}%
\pgfpathlineto{\pgfqpoint{1.463065in}{2.191928in}}%
\pgfpathlineto{\pgfqpoint{1.464494in}{2.348709in}}%
\pgfpathlineto{\pgfqpoint{1.464698in}{2.348709in}}%
\pgfpathlineto{\pgfqpoint{1.465923in}{2.224934in}}%
\pgfpathlineto{\pgfqpoint{1.466127in}{2.241438in}}%
\pgfpathlineto{\pgfqpoint{1.466331in}{2.266192in}}%
\pgfpathlineto{\pgfqpoint{1.466943in}{2.208431in}}%
\pgfpathlineto{\pgfqpoint{1.467148in}{2.249689in}}%
\pgfpathlineto{\pgfqpoint{1.467352in}{1.985638in}}%
\pgfpathlineto{\pgfqpoint{1.467964in}{2.299199in}}%
\pgfpathlineto{\pgfqpoint{1.468168in}{2.208431in}}%
\pgfpathlineto{\pgfqpoint{1.468372in}{2.266192in}}%
\pgfpathlineto{\pgfqpoint{1.469189in}{2.191928in}}%
\pgfpathlineto{\pgfqpoint{1.469801in}{2.290947in}}%
\pgfpathlineto{\pgfqpoint{1.470414in}{2.241438in}}%
\pgfpathlineto{\pgfqpoint{1.470618in}{2.208431in}}%
\pgfpathlineto{\pgfqpoint{1.470822in}{2.249689in}}%
\pgfpathlineto{\pgfqpoint{1.471230in}{2.249689in}}%
\pgfpathlineto{\pgfqpoint{1.471435in}{2.323954in}}%
\pgfpathlineto{\pgfqpoint{1.472251in}{2.249689in}}%
\pgfpathlineto{\pgfqpoint{1.472864in}{2.200180in}}%
\pgfpathlineto{\pgfqpoint{1.473068in}{2.257941in}}%
\pgfpathlineto{\pgfqpoint{1.473476in}{2.224934in}}%
\pgfpathlineto{\pgfqpoint{1.473680in}{2.216683in}}%
\pgfpathlineto{\pgfqpoint{1.473884in}{2.241438in}}%
\pgfpathlineto{\pgfqpoint{1.474088in}{2.233186in}}%
\pgfpathlineto{\pgfqpoint{1.474497in}{2.315702in}}%
\pgfpathlineto{\pgfqpoint{1.474905in}{2.299199in}}%
\pgfpathlineto{\pgfqpoint{1.476130in}{2.142418in}}%
\pgfpathlineto{\pgfqpoint{1.477355in}{2.356960in}}%
\pgfpathlineto{\pgfqpoint{1.477559in}{2.315702in}}%
\pgfpathlineto{\pgfqpoint{1.477763in}{2.332205in}}%
\pgfpathlineto{\pgfqpoint{1.478171in}{2.117664in}}%
\pgfpathlineto{\pgfqpoint{1.478784in}{2.282696in}}%
\pgfpathlineto{\pgfqpoint{1.480009in}{2.224934in}}%
\pgfpathlineto{\pgfqpoint{1.480417in}{2.266192in}}%
\pgfpathlineto{\pgfqpoint{1.481029in}{2.216683in}}%
\pgfpathlineto{\pgfqpoint{1.481233in}{2.216683in}}%
\pgfpathlineto{\pgfqpoint{1.481642in}{2.323954in}}%
\pgfpathlineto{\pgfqpoint{1.482458in}{2.257941in}}%
\pgfpathlineto{\pgfqpoint{1.483275in}{2.299199in}}%
\pgfpathlineto{\pgfqpoint{1.483479in}{2.249689in}}%
\pgfpathlineto{\pgfqpoint{1.483887in}{2.307450in}}%
\pgfpathlineto{\pgfqpoint{1.484296in}{2.257941in}}%
\pgfpathlineto{\pgfqpoint{1.484500in}{2.274444in}}%
\pgfpathlineto{\pgfqpoint{1.484908in}{2.241438in}}%
\pgfpathlineto{\pgfqpoint{1.485112in}{2.233186in}}%
\pgfpathlineto{\pgfqpoint{1.485929in}{2.299199in}}%
\pgfpathlineto{\pgfqpoint{1.486133in}{2.109412in}}%
\pgfpathlineto{\pgfqpoint{1.486541in}{2.356960in}}%
\pgfpathlineto{\pgfqpoint{1.486949in}{2.340457in}}%
\pgfpathlineto{\pgfqpoint{1.487154in}{2.398218in}}%
\pgfpathlineto{\pgfqpoint{1.488174in}{2.381715in}}%
\pgfpathlineto{\pgfqpoint{1.490216in}{2.233186in}}%
\pgfpathlineto{\pgfqpoint{1.490828in}{2.241438in}}%
\pgfpathlineto{\pgfqpoint{1.491441in}{2.274444in}}%
\pgfpathlineto{\pgfqpoint{1.491849in}{2.002141in}}%
\pgfpathlineto{\pgfqpoint{1.492461in}{2.249689in}}%
\pgfpathlineto{\pgfqpoint{1.493890in}{2.340457in}}%
\pgfpathlineto{\pgfqpoint{1.494094in}{2.356960in}}%
\pgfpathlineto{\pgfqpoint{1.494299in}{2.315702in}}%
\pgfpathlineto{\pgfqpoint{1.494707in}{2.340457in}}%
\pgfpathlineto{\pgfqpoint{1.497157in}{1.812354in}}%
\pgfpathlineto{\pgfqpoint{1.497565in}{1.878367in}}%
\pgfpathlineto{\pgfqpoint{1.497973in}{1.919625in}}%
\pgfpathlineto{\pgfqpoint{1.498177in}{1.828857in}}%
\pgfpathlineto{\pgfqpoint{1.498586in}{1.688580in}}%
\pgfpathlineto{\pgfqpoint{1.499402in}{1.993890in}}%
\pgfpathlineto{\pgfqpoint{1.499810in}{1.828857in}}%
\pgfpathlineto{\pgfqpoint{1.500014in}{1.903122in}}%
\pgfpathlineto{\pgfqpoint{1.500627in}{1.812354in}}%
\pgfpathlineto{\pgfqpoint{1.500831in}{1.820606in}}%
\pgfpathlineto{\pgfqpoint{1.501035in}{1.837109in}}%
\pgfpathlineto{\pgfqpoint{1.501443in}{1.804103in}}%
\pgfpathlineto{\pgfqpoint{1.501648in}{1.804103in}}%
\pgfpathlineto{\pgfqpoint{1.502260in}{1.680329in}}%
\pgfpathlineto{\pgfqpoint{1.502668in}{1.787599in}}%
\pgfpathlineto{\pgfqpoint{1.502872in}{1.903122in}}%
\pgfpathlineto{\pgfqpoint{1.503893in}{1.870115in}}%
\pgfpathlineto{\pgfqpoint{1.504914in}{1.878367in}}%
\pgfpathlineto{\pgfqpoint{1.505322in}{1.754593in}}%
\pgfpathlineto{\pgfqpoint{1.506955in}{1.919625in}}%
\pgfpathlineto{\pgfqpoint{1.507568in}{1.870115in}}%
\pgfpathlineto{\pgfqpoint{1.507364in}{1.936128in}}%
\pgfpathlineto{\pgfqpoint{1.507772in}{1.927877in}}%
\pgfpathlineto{\pgfqpoint{1.508384in}{1.911373in}}%
\pgfpathlineto{\pgfqpoint{1.508997in}{1.977386in}}%
\pgfpathlineto{\pgfqpoint{1.510426in}{1.861864in}}%
\pgfpathlineto{\pgfqpoint{1.510630in}{1.927877in}}%
\pgfpathlineto{\pgfqpoint{1.511242in}{1.837109in}}%
\pgfpathlineto{\pgfqpoint{1.511446in}{1.845361in}}%
\pgfpathlineto{\pgfqpoint{1.512059in}{1.952632in}}%
\pgfpathlineto{\pgfqpoint{1.512467in}{1.936128in}}%
\pgfpathlineto{\pgfqpoint{1.513692in}{1.738090in}}%
\pgfpathlineto{\pgfqpoint{1.514509in}{1.870115in}}%
\pgfpathlineto{\pgfqpoint{1.514917in}{1.828857in}}%
\pgfpathlineto{\pgfqpoint{1.515733in}{1.944380in}}%
\pgfpathlineto{\pgfqpoint{1.516142in}{1.903122in}}%
\pgfpathlineto{\pgfqpoint{1.516754in}{1.754593in}}%
\pgfpathlineto{\pgfqpoint{1.516958in}{1.886619in}}%
\pgfpathlineto{\pgfqpoint{1.517162in}{1.927877in}}%
\pgfpathlineto{\pgfqpoint{1.517367in}{1.680329in}}%
\pgfpathlineto{\pgfqpoint{1.518183in}{1.944380in}}%
\pgfpathlineto{\pgfqpoint{1.518796in}{1.828857in}}%
\pgfpathlineto{\pgfqpoint{1.519816in}{1.861864in}}%
\pgfpathlineto{\pgfqpoint{1.520633in}{1.960883in}}%
\pgfpathlineto{\pgfqpoint{1.521041in}{1.894870in}}%
\pgfpathlineto{\pgfqpoint{1.521858in}{1.795851in}}%
\pgfpathlineto{\pgfqpoint{1.522674in}{1.845361in}}%
\pgfpathlineto{\pgfqpoint{1.523083in}{1.969135in}}%
\pgfpathlineto{\pgfqpoint{1.524103in}{1.936128in}}%
\pgfpathlineto{\pgfqpoint{1.524512in}{1.853612in}}%
\pgfpathlineto{\pgfqpoint{1.524920in}{1.919625in}}%
\pgfpathlineto{\pgfqpoint{1.525941in}{2.051651in}}%
\pgfpathlineto{\pgfqpoint{1.526349in}{1.969135in}}%
\pgfpathlineto{\pgfqpoint{1.526757in}{1.985638in}}%
\pgfpathlineto{\pgfqpoint{1.527574in}{1.894870in}}%
\pgfpathlineto{\pgfqpoint{1.528186in}{1.886619in}}%
\pgfpathlineto{\pgfqpoint{1.528799in}{1.985638in}}%
\pgfpathlineto{\pgfqpoint{1.530023in}{1.787599in}}%
\pgfpathlineto{\pgfqpoint{1.530636in}{2.076406in}}%
\pgfpathlineto{\pgfqpoint{1.531452in}{2.035148in}}%
\pgfpathlineto{\pgfqpoint{1.532269in}{2.043399in}}%
\pgfpathlineto{\pgfqpoint{1.532677in}{1.952632in}}%
\pgfpathlineto{\pgfqpoint{1.533494in}{2.068154in}}%
\pgfpathlineto{\pgfqpoint{1.533902in}{2.051651in}}%
\pgfpathlineto{\pgfqpoint{1.534106in}{2.051651in}}%
\pgfpathlineto{\pgfqpoint{1.534515in}{2.035148in}}%
\pgfpathlineto{\pgfqpoint{1.534923in}{1.952632in}}%
\pgfpathlineto{\pgfqpoint{1.535535in}{2.018644in}}%
\pgfpathlineto{\pgfqpoint{1.535739in}{2.026896in}}%
\pgfpathlineto{\pgfqpoint{1.536352in}{1.936128in}}%
\pgfpathlineto{\pgfqpoint{1.536556in}{2.035148in}}%
\pgfpathlineto{\pgfqpoint{1.536760in}{2.018644in}}%
\pgfpathlineto{\pgfqpoint{1.538802in}{1.837109in}}%
\pgfpathlineto{\pgfqpoint{1.539006in}{1.903122in}}%
\pgfpathlineto{\pgfqpoint{1.539618in}{1.795851in}}%
\pgfpathlineto{\pgfqpoint{1.539822in}{1.820606in}}%
\pgfpathlineto{\pgfqpoint{1.541251in}{1.672077in}}%
\pgfpathlineto{\pgfqpoint{1.541455in}{1.672077in}}%
\pgfpathlineto{\pgfqpoint{1.542476in}{1.738090in}}%
\pgfpathlineto{\pgfqpoint{1.542680in}{1.705083in}}%
\pgfpathlineto{\pgfqpoint{1.543497in}{1.655574in}}%
\pgfpathlineto{\pgfqpoint{1.543701in}{1.705083in}}%
\pgfpathlineto{\pgfqpoint{1.543905in}{1.738090in}}%
\pgfpathlineto{\pgfqpoint{1.544313in}{1.639071in}}%
\pgfpathlineto{\pgfqpoint{1.544926in}{1.597813in}}%
\pgfpathlineto{\pgfqpoint{1.545130in}{1.738090in}}%
\pgfpathlineto{\pgfqpoint{1.546151in}{1.705083in}}%
\pgfpathlineto{\pgfqpoint{1.546355in}{1.705083in}}%
\pgfpathlineto{\pgfqpoint{1.546559in}{1.688580in}}%
\pgfpathlineto{\pgfqpoint{1.546967in}{1.738090in}}%
\pgfpathlineto{\pgfqpoint{1.547171in}{1.845361in}}%
\pgfpathlineto{\pgfqpoint{1.547988in}{1.721587in}}%
\pgfpathlineto{\pgfqpoint{1.548396in}{1.795851in}}%
\pgfpathlineto{\pgfqpoint{1.549213in}{1.754593in}}%
\pgfpathlineto{\pgfqpoint{1.549825in}{1.705083in}}%
\pgfpathlineto{\pgfqpoint{1.550642in}{1.688580in}}%
\pgfpathlineto{\pgfqpoint{1.550846in}{1.812354in}}%
\pgfpathlineto{\pgfqpoint{1.551458in}{1.672077in}}%
\pgfpathlineto{\pgfqpoint{1.552071in}{1.729838in}}%
\pgfpathlineto{\pgfqpoint{1.553092in}{1.779348in}}%
\pgfpathlineto{\pgfqpoint{1.553500in}{1.771096in}}%
\pgfpathlineto{\pgfqpoint{1.554725in}{1.729838in}}%
\pgfpathlineto{\pgfqpoint{1.554929in}{1.787599in}}%
\pgfpathlineto{\pgfqpoint{1.555950in}{1.779348in}}%
\pgfpathlineto{\pgfqpoint{1.556358in}{1.779348in}}%
\pgfpathlineto{\pgfqpoint{1.556766in}{1.705083in}}%
\pgfpathlineto{\pgfqpoint{1.556970in}{1.779348in}}%
\pgfpathlineto{\pgfqpoint{1.557174in}{1.837109in}}%
\pgfpathlineto{\pgfqpoint{1.557583in}{1.713335in}}%
\pgfpathlineto{\pgfqpoint{1.557991in}{1.779348in}}%
\pgfpathlineto{\pgfqpoint{1.558603in}{1.713335in}}%
\pgfpathlineto{\pgfqpoint{1.559012in}{1.721587in}}%
\pgfpathlineto{\pgfqpoint{1.559216in}{1.754593in}}%
\pgfpathlineto{\pgfqpoint{1.559420in}{1.581309in}}%
\pgfpathlineto{\pgfqpoint{1.560032in}{1.812354in}}%
\pgfpathlineto{\pgfqpoint{1.560237in}{1.804103in}}%
\pgfpathlineto{\pgfqpoint{1.560441in}{1.828857in}}%
\pgfpathlineto{\pgfqpoint{1.560849in}{1.812354in}}%
\pgfpathlineto{\pgfqpoint{1.561053in}{1.721587in}}%
\pgfpathlineto{\pgfqpoint{1.561666in}{1.861864in}}%
\pgfpathlineto{\pgfqpoint{1.561870in}{1.828857in}}%
\pgfpathlineto{\pgfqpoint{1.562074in}{1.853612in}}%
\pgfpathlineto{\pgfqpoint{1.562278in}{1.779348in}}%
\pgfpathlineto{\pgfqpoint{1.562890in}{1.828857in}}%
\pgfpathlineto{\pgfqpoint{1.563095in}{1.663825in}}%
\pgfpathlineto{\pgfqpoint{1.563911in}{1.721587in}}%
\pgfpathlineto{\pgfqpoint{1.564319in}{1.754593in}}%
\pgfpathlineto{\pgfqpoint{1.564728in}{1.696832in}}%
\pgfpathlineto{\pgfqpoint{1.564932in}{1.688580in}}%
\pgfpathlineto{\pgfqpoint{1.565136in}{1.713335in}}%
\pgfpathlineto{\pgfqpoint{1.565340in}{1.771096in}}%
\pgfpathlineto{\pgfqpoint{1.566157in}{1.696832in}}%
\pgfpathlineto{\pgfqpoint{1.566565in}{1.705083in}}%
\pgfpathlineto{\pgfqpoint{1.567382in}{1.639071in}}%
\pgfpathlineto{\pgfqpoint{1.567790in}{1.663825in}}%
\pgfpathlineto{\pgfqpoint{1.567994in}{1.738090in}}%
\pgfpathlineto{\pgfqpoint{1.568606in}{1.655574in}}%
\pgfpathlineto{\pgfqpoint{1.568811in}{1.696832in}}%
\pgfpathlineto{\pgfqpoint{1.570035in}{1.622567in}}%
\pgfpathlineto{\pgfqpoint{1.570240in}{1.622567in}}%
\pgfpathlineto{\pgfqpoint{1.570444in}{1.630819in}}%
\pgfpathlineto{\pgfqpoint{1.571056in}{1.540051in}}%
\pgfpathlineto{\pgfqpoint{1.571873in}{1.548303in}}%
\pgfpathlineto{\pgfqpoint{1.573506in}{1.655574in}}%
\pgfpathlineto{\pgfqpoint{1.573914in}{1.647322in}}%
\pgfpathlineto{\pgfqpoint{1.574118in}{1.622567in}}%
\pgfpathlineto{\pgfqpoint{1.574322in}{1.729838in}}%
\pgfpathlineto{\pgfqpoint{1.574527in}{1.680329in}}%
\pgfpathlineto{\pgfqpoint{1.575343in}{1.639071in}}%
\pgfpathlineto{\pgfqpoint{1.575956in}{1.787599in}}%
\pgfpathlineto{\pgfqpoint{1.577589in}{1.581309in}}%
\pgfpathlineto{\pgfqpoint{1.578201in}{1.655574in}}%
\pgfpathlineto{\pgfqpoint{1.578609in}{1.614316in}}%
\pgfpathlineto{\pgfqpoint{1.578814in}{2.142418in}}%
\pgfpathlineto{\pgfqpoint{1.579630in}{1.614316in}}%
\pgfpathlineto{\pgfqpoint{1.579834in}{1.531800in}}%
\pgfpathlineto{\pgfqpoint{1.580243in}{1.680329in}}%
\pgfpathlineto{\pgfqpoint{1.580447in}{1.655574in}}%
\pgfpathlineto{\pgfqpoint{1.580651in}{1.663825in}}%
\pgfpathlineto{\pgfqpoint{1.580855in}{1.630819in}}%
\pgfpathlineto{\pgfqpoint{1.581059in}{1.606064in}}%
\pgfpathlineto{\pgfqpoint{1.581263in}{1.696832in}}%
\pgfpathlineto{\pgfqpoint{1.581467in}{1.672077in}}%
\pgfpathlineto{\pgfqpoint{1.581672in}{1.688580in}}%
\pgfpathlineto{\pgfqpoint{1.582080in}{1.449284in}}%
\pgfpathlineto{\pgfqpoint{1.582692in}{1.655574in}}%
\pgfpathlineto{\pgfqpoint{1.582896in}{1.647322in}}%
\pgfpathlineto{\pgfqpoint{1.583101in}{1.663825in}}%
\pgfpathlineto{\pgfqpoint{1.583509in}{1.754593in}}%
\pgfpathlineto{\pgfqpoint{1.583917in}{1.556554in}}%
\pgfpathlineto{\pgfqpoint{1.584121in}{1.672077in}}%
\pgfpathlineto{\pgfqpoint{1.584325in}{1.680329in}}%
\pgfpathlineto{\pgfqpoint{1.584529in}{1.672077in}}%
\pgfpathlineto{\pgfqpoint{1.584734in}{1.771096in}}%
\pgfpathlineto{\pgfqpoint{1.585550in}{1.672077in}}%
\pgfpathlineto{\pgfqpoint{1.585958in}{1.622567in}}%
\pgfpathlineto{\pgfqpoint{1.586571in}{1.672077in}}%
\pgfpathlineto{\pgfqpoint{1.587183in}{1.779348in}}%
\pgfpathlineto{\pgfqpoint{1.587387in}{1.746341in}}%
\pgfpathlineto{\pgfqpoint{1.588000in}{1.432780in}}%
\pgfpathlineto{\pgfqpoint{1.588612in}{1.663825in}}%
\pgfpathlineto{\pgfqpoint{1.588816in}{1.647322in}}%
\pgfpathlineto{\pgfqpoint{1.589021in}{1.713335in}}%
\pgfpathlineto{\pgfqpoint{1.589633in}{1.663825in}}%
\pgfpathlineto{\pgfqpoint{1.589837in}{1.688580in}}%
\pgfpathlineto{\pgfqpoint{1.590654in}{1.754593in}}%
\pgfpathlineto{\pgfqpoint{1.590858in}{1.688580in}}%
\pgfpathlineto{\pgfqpoint{1.591062in}{1.630819in}}%
\pgfpathlineto{\pgfqpoint{1.591674in}{1.713335in}}%
\pgfpathlineto{\pgfqpoint{1.591879in}{1.738090in}}%
\pgfpathlineto{\pgfqpoint{1.592287in}{1.540051in}}%
\pgfpathlineto{\pgfqpoint{1.592695in}{1.779348in}}%
\pgfpathlineto{\pgfqpoint{1.592899in}{1.795851in}}%
\pgfpathlineto{\pgfqpoint{1.593512in}{1.762845in}}%
\pgfpathlineto{\pgfqpoint{1.594532in}{1.647322in}}%
\pgfpathlineto{\pgfqpoint{1.594941in}{1.655574in}}%
\pgfpathlineto{\pgfqpoint{1.595145in}{1.688580in}}%
\pgfpathlineto{\pgfqpoint{1.595553in}{1.729838in}}%
\pgfpathlineto{\pgfqpoint{1.596166in}{1.449284in}}%
\pgfpathlineto{\pgfqpoint{1.597186in}{1.713335in}}%
\pgfpathlineto{\pgfqpoint{1.597390in}{1.696832in}}%
\pgfpathlineto{\pgfqpoint{1.597595in}{1.639071in}}%
\pgfpathlineto{\pgfqpoint{1.598615in}{1.647322in}}%
\pgfpathlineto{\pgfqpoint{1.598819in}{1.647322in}}%
\pgfpathlineto{\pgfqpoint{1.599228in}{1.622567in}}%
\pgfpathlineto{\pgfqpoint{1.599432in}{1.663825in}}%
\pgfpathlineto{\pgfqpoint{1.599636in}{1.639071in}}%
\pgfpathlineto{\pgfqpoint{1.600657in}{1.680329in}}%
\pgfpathlineto{\pgfqpoint{1.601065in}{1.713335in}}%
\pgfpathlineto{\pgfqpoint{1.601882in}{1.564806in}}%
\pgfpathlineto{\pgfqpoint{1.602290in}{1.630819in}}%
\pgfpathlineto{\pgfqpoint{1.602494in}{1.366768in}}%
\pgfpathlineto{\pgfqpoint{1.603106in}{1.663825in}}%
\pgfpathlineto{\pgfqpoint{1.603311in}{1.639071in}}%
\pgfpathlineto{\pgfqpoint{1.603719in}{1.614316in}}%
\pgfpathlineto{\pgfqpoint{1.604535in}{1.721587in}}%
\pgfpathlineto{\pgfqpoint{1.605760in}{1.342013in}}%
\pgfpathlineto{\pgfqpoint{1.605964in}{1.350264in}}%
\pgfpathlineto{\pgfqpoint{1.606169in}{1.647322in}}%
\pgfpathlineto{\pgfqpoint{1.607189in}{1.606064in}}%
\pgfpathlineto{\pgfqpoint{1.608822in}{1.795851in}}%
\pgfpathlineto{\pgfqpoint{1.609027in}{1.787599in}}%
\pgfpathlineto{\pgfqpoint{1.609435in}{1.787599in}}%
\pgfpathlineto{\pgfqpoint{1.609843in}{1.886619in}}%
\pgfpathlineto{\pgfqpoint{1.610251in}{1.812354in}}%
\pgfpathlineto{\pgfqpoint{1.610456in}{1.713335in}}%
\pgfpathlineto{\pgfqpoint{1.611476in}{1.721587in}}%
\pgfpathlineto{\pgfqpoint{1.612293in}{1.795851in}}%
\pgfpathlineto{\pgfqpoint{1.612701in}{1.771096in}}%
\pgfpathlineto{\pgfqpoint{1.613109in}{1.779348in}}%
\pgfpathlineto{\pgfqpoint{1.613722in}{1.597813in}}%
\pgfpathlineto{\pgfqpoint{1.613926in}{1.828857in}}%
\pgfpathlineto{\pgfqpoint{1.614743in}{1.820606in}}%
\pgfpathlineto{\pgfqpoint{1.615967in}{1.399774in}}%
\pgfpathlineto{\pgfqpoint{1.616172in}{1.432780in}}%
\pgfpathlineto{\pgfqpoint{1.616784in}{1.639071in}}%
\pgfpathlineto{\pgfqpoint{1.617396in}{1.581309in}}%
\pgfpathlineto{\pgfqpoint{1.617601in}{1.589561in}}%
\pgfpathlineto{\pgfqpoint{1.618417in}{1.597813in}}%
\pgfpathlineto{\pgfqpoint{1.618621in}{1.523548in}}%
\pgfpathlineto{\pgfqpoint{1.619234in}{1.614316in}}%
\pgfpathlineto{\pgfqpoint{1.619642in}{1.523548in}}%
\pgfpathlineto{\pgfqpoint{1.621479in}{1.630819in}}%
\pgfpathlineto{\pgfqpoint{1.621683in}{1.630819in}}%
\pgfpathlineto{\pgfqpoint{1.622704in}{1.548303in}}%
\pgfpathlineto{\pgfqpoint{1.623112in}{1.564806in}}%
\pgfpathlineto{\pgfqpoint{1.623929in}{1.622567in}}%
\pgfpathlineto{\pgfqpoint{1.624133in}{1.614316in}}%
\pgfpathlineto{\pgfqpoint{1.624746in}{1.556554in}}%
\pgfpathlineto{\pgfqpoint{1.624541in}{1.622567in}}%
\pgfpathlineto{\pgfqpoint{1.624950in}{1.589561in}}%
\pgfpathlineto{\pgfqpoint{1.625154in}{1.655574in}}%
\pgfpathlineto{\pgfqpoint{1.625766in}{1.498793in}}%
\pgfpathlineto{\pgfqpoint{1.625970in}{1.498793in}}%
\pgfpathlineto{\pgfqpoint{1.626379in}{1.490542in}}%
\pgfpathlineto{\pgfqpoint{1.627195in}{1.540051in}}%
\pgfpathlineto{\pgfqpoint{1.628216in}{1.449284in}}%
\pgfpathlineto{\pgfqpoint{1.628420in}{1.482290in}}%
\pgfpathlineto{\pgfqpoint{1.628624in}{1.531800in}}%
\pgfpathlineto{\pgfqpoint{1.628828in}{1.457535in}}%
\pgfpathlineto{\pgfqpoint{1.629237in}{1.474038in}}%
\pgfpathlineto{\pgfqpoint{1.630257in}{1.432780in}}%
\pgfpathlineto{\pgfqpoint{1.630462in}{1.457535in}}%
\pgfpathlineto{\pgfqpoint{1.630870in}{1.548303in}}%
\pgfpathlineto{\pgfqpoint{1.631482in}{1.498793in}}%
\pgfpathlineto{\pgfqpoint{1.632503in}{1.276000in}}%
\pgfpathlineto{\pgfqpoint{1.632911in}{1.317258in}}%
\pgfpathlineto{\pgfqpoint{1.633524in}{1.523548in}}%
\pgfpathlineto{\pgfqpoint{1.634136in}{1.474038in}}%
\pgfpathlineto{\pgfqpoint{1.635157in}{1.441032in}}%
\pgfpathlineto{\pgfqpoint{1.636382in}{1.531800in}}%
\pgfpathlineto{\pgfqpoint{1.635769in}{1.416277in}}%
\pgfpathlineto{\pgfqpoint{1.636586in}{1.507045in}}%
\pgfpathlineto{\pgfqpoint{1.637198in}{1.474038in}}%
\pgfpathlineto{\pgfqpoint{1.637402in}{1.490542in}}%
\pgfpathlineto{\pgfqpoint{1.637811in}{1.482290in}}%
\pgfpathlineto{\pgfqpoint{1.638831in}{1.589561in}}%
\pgfpathlineto{\pgfqpoint{1.639444in}{1.474038in}}%
\pgfpathlineto{\pgfqpoint{1.640056in}{1.507045in}}%
\pgfpathlineto{\pgfqpoint{1.640260in}{1.457535in}}%
\pgfpathlineto{\pgfqpoint{1.640465in}{1.556554in}}%
\pgfpathlineto{\pgfqpoint{1.641077in}{1.498793in}}%
\pgfpathlineto{\pgfqpoint{1.641281in}{1.523548in}}%
\pgfpathlineto{\pgfqpoint{1.641485in}{1.465787in}}%
\pgfpathlineto{\pgfqpoint{1.641689in}{1.391522in}}%
\pgfpathlineto{\pgfqpoint{1.642506in}{1.432780in}}%
\pgfpathlineto{\pgfqpoint{1.642710in}{1.449284in}}%
\pgfpathlineto{\pgfqpoint{1.642914in}{1.441032in}}%
\pgfpathlineto{\pgfqpoint{1.643527in}{1.201736in}}%
\pgfpathlineto{\pgfqpoint{1.643731in}{1.309006in}}%
\pgfpathlineto{\pgfqpoint{1.644343in}{1.515296in}}%
\pgfpathlineto{\pgfqpoint{1.644956in}{1.441032in}}%
\pgfpathlineto{\pgfqpoint{1.645160in}{1.482290in}}%
\pgfpathlineto{\pgfqpoint{1.645364in}{1.399774in}}%
\pgfpathlineto{\pgfqpoint{1.645772in}{1.416277in}}%
\pgfpathlineto{\pgfqpoint{1.645976in}{1.399774in}}%
\pgfpathlineto{\pgfqpoint{1.646385in}{1.449284in}}%
\pgfpathlineto{\pgfqpoint{1.646589in}{1.465787in}}%
\pgfpathlineto{\pgfqpoint{1.646997in}{1.432780in}}%
\pgfpathlineto{\pgfqpoint{1.647814in}{1.325510in}}%
\pgfpathlineto{\pgfqpoint{1.648222in}{1.366768in}}%
\pgfpathlineto{\pgfqpoint{1.648630in}{1.399774in}}%
\pgfpathlineto{\pgfqpoint{1.648834in}{1.242994in}}%
\pgfpathlineto{\pgfqpoint{1.649651in}{1.465787in}}%
\pgfpathlineto{\pgfqpoint{1.650059in}{1.531800in}}%
\pgfpathlineto{\pgfqpoint{1.650468in}{1.457535in}}%
\pgfpathlineto{\pgfqpoint{1.650672in}{1.457535in}}%
\pgfpathlineto{\pgfqpoint{1.650876in}{1.350264in}}%
\pgfpathlineto{\pgfqpoint{1.651692in}{1.408026in}}%
\pgfpathlineto{\pgfqpoint{1.652101in}{1.515296in}}%
\pgfpathlineto{\pgfqpoint{1.652917in}{1.482290in}}%
\pgfpathlineto{\pgfqpoint{1.654142in}{1.416277in}}%
\pgfpathlineto{\pgfqpoint{1.654346in}{1.474038in}}%
\pgfpathlineto{\pgfqpoint{1.655163in}{1.457535in}}%
\pgfpathlineto{\pgfqpoint{1.655979in}{1.424529in}}%
\pgfpathlineto{\pgfqpoint{1.655571in}{1.498793in}}%
\pgfpathlineto{\pgfqpoint{1.656184in}{1.457535in}}%
\pgfpathlineto{\pgfqpoint{1.656388in}{1.457535in}}%
\pgfpathlineto{\pgfqpoint{1.656592in}{1.548303in}}%
\pgfpathlineto{\pgfqpoint{1.657000in}{1.432780in}}%
\pgfpathlineto{\pgfqpoint{1.657204in}{1.449284in}}%
\pgfpathlineto{\pgfqpoint{1.657613in}{1.185232in}}%
\pgfpathlineto{\pgfqpoint{1.658429in}{1.391522in}}%
\pgfpathlineto{\pgfqpoint{1.658633in}{1.482290in}}%
\pgfpathlineto{\pgfqpoint{1.659246in}{1.383271in}}%
\pgfpathlineto{\pgfqpoint{1.659450in}{1.408026in}}%
\pgfpathlineto{\pgfqpoint{1.660266in}{1.300755in}}%
\pgfpathlineto{\pgfqpoint{1.660471in}{1.375019in}}%
\pgfpathlineto{\pgfqpoint{1.661083in}{1.474038in}}%
\pgfpathlineto{\pgfqpoint{1.661491in}{1.375019in}}%
\pgfpathlineto{\pgfqpoint{1.662104in}{1.209987in}}%
\pgfpathlineto{\pgfqpoint{1.662308in}{1.061458in}}%
\pgfpathlineto{\pgfqpoint{1.662716in}{1.498793in}}%
\pgfpathlineto{\pgfqpoint{1.662920in}{1.391522in}}%
\pgfpathlineto{\pgfqpoint{1.664553in}{1.201736in}}%
\pgfpathlineto{\pgfqpoint{1.665982in}{1.399774in}}%
\pgfpathlineto{\pgfqpoint{1.667003in}{1.276000in}}%
\pgfpathlineto{\pgfqpoint{1.666799in}{1.416277in}}%
\pgfpathlineto{\pgfqpoint{1.667207in}{1.284252in}}%
\pgfpathlineto{\pgfqpoint{1.667411in}{1.342013in}}%
\pgfpathlineto{\pgfqpoint{1.668024in}{1.251245in}}%
\pgfpathlineto{\pgfqpoint{1.668228in}{1.077961in}}%
\pgfpathlineto{\pgfqpoint{1.668636in}{1.350264in}}%
\pgfpathlineto{\pgfqpoint{1.669044in}{1.267748in}}%
\pgfpathlineto{\pgfqpoint{1.669453in}{1.251245in}}%
\pgfpathlineto{\pgfqpoint{1.669657in}{1.276000in}}%
\pgfpathlineto{\pgfqpoint{1.670065in}{1.325510in}}%
\pgfpathlineto{\pgfqpoint{1.670473in}{1.242994in}}%
\pgfpathlineto{\pgfqpoint{1.671698in}{1.383271in}}%
\pgfpathlineto{\pgfqpoint{1.671902in}{1.350264in}}%
\pgfpathlineto{\pgfqpoint{1.672719in}{1.234742in}}%
\pgfpathlineto{\pgfqpoint{1.673127in}{1.242994in}}%
\pgfpathlineto{\pgfqpoint{1.674148in}{1.317258in}}%
\pgfpathlineto{\pgfqpoint{1.673740in}{1.226490in}}%
\pgfpathlineto{\pgfqpoint{1.674352in}{1.284252in}}%
\pgfpathlineto{\pgfqpoint{1.674965in}{1.391522in}}%
\pgfpathlineto{\pgfqpoint{1.674760in}{1.276000in}}%
\pgfpathlineto{\pgfqpoint{1.675169in}{1.300755in}}%
\pgfpathlineto{\pgfqpoint{1.675577in}{1.325510in}}%
\pgfpathlineto{\pgfqpoint{1.675985in}{1.135723in}}%
\pgfpathlineto{\pgfqpoint{1.677210in}{1.399774in}}%
\pgfpathlineto{\pgfqpoint{1.677823in}{1.342013in}}%
\pgfpathlineto{\pgfqpoint{1.678027in}{1.391522in}}%
\pgfpathlineto{\pgfqpoint{1.678231in}{1.408026in}}%
\pgfpathlineto{\pgfqpoint{1.678435in}{1.366768in}}%
\pgfpathlineto{\pgfqpoint{1.678843in}{1.391522in}}%
\pgfpathlineto{\pgfqpoint{1.679047in}{1.317258in}}%
\pgfpathlineto{\pgfqpoint{1.679252in}{1.408026in}}%
\pgfpathlineto{\pgfqpoint{1.679864in}{1.399774in}}%
\pgfpathlineto{\pgfqpoint{1.680068in}{1.408026in}}%
\pgfpathlineto{\pgfqpoint{1.680272in}{1.399774in}}%
\pgfpathlineto{\pgfqpoint{1.681497in}{1.119219in}}%
\pgfpathlineto{\pgfqpoint{1.681701in}{1.292503in}}%
\pgfpathlineto{\pgfqpoint{1.681905in}{1.300755in}}%
\pgfpathlineto{\pgfqpoint{1.682110in}{1.556554in}}%
\pgfpathlineto{\pgfqpoint{1.682926in}{1.399774in}}%
\pgfpathlineto{\pgfqpoint{1.683130in}{1.399774in}}%
\pgfpathlineto{\pgfqpoint{1.683334in}{1.416277in}}%
\pgfpathlineto{\pgfqpoint{1.683947in}{1.168729in}}%
\pgfpathlineto{\pgfqpoint{1.684355in}{1.449284in}}%
\pgfpathlineto{\pgfqpoint{1.684968in}{1.540051in}}%
\pgfpathlineto{\pgfqpoint{1.685376in}{1.482290in}}%
\pgfpathlineto{\pgfqpoint{1.686805in}{1.383271in}}%
\pgfpathlineto{\pgfqpoint{1.686192in}{1.507045in}}%
\pgfpathlineto{\pgfqpoint{1.687009in}{1.399774in}}%
\pgfpathlineto{\pgfqpoint{1.687213in}{1.399774in}}%
\pgfpathlineto{\pgfqpoint{1.687621in}{1.176981in}}%
\pgfpathlineto{\pgfqpoint{1.688234in}{1.408026in}}%
\pgfpathlineto{\pgfqpoint{1.688438in}{1.333761in}}%
\pgfpathlineto{\pgfqpoint{1.688642in}{1.441032in}}%
\pgfpathlineto{\pgfqpoint{1.689459in}{1.383271in}}%
\pgfpathlineto{\pgfqpoint{1.689867in}{1.284252in}}%
\pgfpathlineto{\pgfqpoint{1.690275in}{1.457535in}}%
\pgfpathlineto{\pgfqpoint{1.691092in}{1.317258in}}%
\pgfpathlineto{\pgfqpoint{1.693950in}{0.970691in}}%
\pgfpathlineto{\pgfqpoint{1.691704in}{1.342013in}}%
\pgfpathlineto{\pgfqpoint{1.694154in}{0.978942in}}%
\pgfpathlineto{\pgfqpoint{1.695175in}{0.970691in}}%
\pgfpathlineto{\pgfqpoint{1.695787in}{1.366768in}}%
\pgfpathlineto{\pgfqpoint{1.696808in}{1.061458in}}%
\pgfpathlineto{\pgfqpoint{1.697012in}{1.242994in}}%
\pgfpathlineto{\pgfqpoint{1.697420in}{1.193484in}}%
\pgfpathlineto{\pgfqpoint{1.697624in}{1.251245in}}%
\pgfpathlineto{\pgfqpoint{1.697829in}{1.209987in}}%
\pgfpathlineto{\pgfqpoint{1.698033in}{1.284252in}}%
\pgfpathlineto{\pgfqpoint{1.698441in}{1.193484in}}%
\pgfpathlineto{\pgfqpoint{1.698645in}{1.193484in}}%
\pgfpathlineto{\pgfqpoint{1.699258in}{1.218239in}}%
\pgfpathlineto{\pgfqpoint{1.699666in}{1.110968in}}%
\pgfpathlineto{\pgfqpoint{1.699870in}{1.267748in}}%
\pgfpathlineto{\pgfqpoint{1.700687in}{1.209987in}}%
\pgfpathlineto{\pgfqpoint{1.701095in}{1.168729in}}%
\pgfpathlineto{\pgfqpoint{1.701299in}{1.185232in}}%
\pgfpathlineto{\pgfqpoint{1.701911in}{1.317258in}}%
\pgfpathlineto{\pgfqpoint{1.702116in}{1.276000in}}%
\pgfpathlineto{\pgfqpoint{1.703136in}{0.970691in}}%
\pgfpathlineto{\pgfqpoint{1.703340in}{1.110968in}}%
\pgfpathlineto{\pgfqpoint{1.704157in}{1.135723in}}%
\pgfpathlineto{\pgfqpoint{1.704769in}{1.036703in}}%
\pgfpathlineto{\pgfqpoint{1.705994in}{1.218239in}}%
\pgfpathlineto{\pgfqpoint{1.706198in}{1.185232in}}%
\pgfpathlineto{\pgfqpoint{1.707832in}{0.978942in}}%
\pgfpathlineto{\pgfqpoint{1.708036in}{1.011949in}}%
\pgfpathlineto{\pgfqpoint{1.708240in}{0.838665in}}%
\pgfpathlineto{\pgfqpoint{1.708444in}{1.119219in}}%
\pgfpathlineto{\pgfqpoint{1.709056in}{1.011949in}}%
\pgfpathlineto{\pgfqpoint{1.709669in}{1.094465in}}%
\pgfpathlineto{\pgfqpoint{1.710281in}{1.069710in}}%
\pgfpathlineto{\pgfqpoint{1.710485in}{1.003697in}}%
\pgfpathlineto{\pgfqpoint{1.711098in}{1.135723in}}%
\pgfpathlineto{\pgfqpoint{1.711302in}{1.036703in}}%
\pgfpathlineto{\pgfqpoint{1.711914in}{1.234742in}}%
\pgfpathlineto{\pgfqpoint{1.712527in}{1.102716in}}%
\pgfpathlineto{\pgfqpoint{1.712731in}{1.069710in}}%
\pgfpathlineto{\pgfqpoint{1.713139in}{1.102716in}}%
\pgfpathlineto{\pgfqpoint{1.713343in}{1.160477in}}%
\pgfpathlineto{\pgfqpoint{1.713548in}{1.053207in}}%
\pgfpathlineto{\pgfqpoint{1.713956in}{1.119219in}}%
\pgfpathlineto{\pgfqpoint{1.714364in}{0.838665in}}%
\pgfpathlineto{\pgfqpoint{1.714977in}{0.962439in}}%
\pgfpathlineto{\pgfqpoint{1.715793in}{1.069710in}}%
\pgfpathlineto{\pgfqpoint{1.715589in}{0.921181in}}%
\pgfpathlineto{\pgfqpoint{1.715997in}{1.011949in}}%
\pgfpathlineto{\pgfqpoint{1.717426in}{0.904678in}}%
\pgfpathlineto{\pgfqpoint{1.718447in}{1.119219in}}%
\pgfpathlineto{\pgfqpoint{1.718651in}{1.086213in}}%
\pgfpathlineto{\pgfqpoint{1.719672in}{1.020200in}}%
\pgfpathlineto{\pgfqpoint{1.719876in}{1.028452in}}%
\pgfpathlineto{\pgfqpoint{1.721509in}{1.259497in}}%
\pgfpathlineto{\pgfqpoint{1.722938in}{1.424529in}}%
\pgfpathlineto{\pgfqpoint{1.723755in}{1.408026in}}%
\pgfpathlineto{\pgfqpoint{1.723959in}{1.391522in}}%
\pgfpathlineto{\pgfqpoint{1.724163in}{1.432780in}}%
\pgfpathlineto{\pgfqpoint{1.724571in}{1.424529in}}%
\pgfpathlineto{\pgfqpoint{1.724775in}{1.432780in}}%
\pgfpathlineto{\pgfqpoint{1.724980in}{1.391522in}}%
\pgfpathlineto{\pgfqpoint{1.725592in}{1.490542in}}%
\pgfpathlineto{\pgfqpoint{1.725796in}{1.449284in}}%
\pgfpathlineto{\pgfqpoint{1.726000in}{1.449284in}}%
\pgfpathlineto{\pgfqpoint{1.726409in}{1.441032in}}%
\pgfpathlineto{\pgfqpoint{1.726817in}{1.523548in}}%
\pgfpathlineto{\pgfqpoint{1.727633in}{1.647322in}}%
\pgfpathlineto{\pgfqpoint{1.728042in}{1.614316in}}%
\pgfpathlineto{\pgfqpoint{1.728246in}{1.490542in}}%
\pgfpathlineto{\pgfqpoint{1.729062in}{1.655574in}}%
\pgfpathlineto{\pgfqpoint{1.729879in}{1.573058in}}%
\pgfpathlineto{\pgfqpoint{1.730287in}{1.630819in}}%
\pgfpathlineto{\pgfqpoint{1.731716in}{1.746341in}}%
\pgfpathlineto{\pgfqpoint{1.731104in}{1.622567in}}%
\pgfpathlineto{\pgfqpoint{1.732125in}{1.721587in}}%
\pgfpathlineto{\pgfqpoint{1.733145in}{1.342013in}}%
\pgfpathlineto{\pgfqpoint{1.733554in}{1.531800in}}%
\pgfpathlineto{\pgfqpoint{1.733758in}{1.540051in}}%
\pgfpathlineto{\pgfqpoint{1.734370in}{1.457535in}}%
\pgfpathlineto{\pgfqpoint{1.734778in}{1.482290in}}%
\pgfpathlineto{\pgfqpoint{1.736207in}{1.696832in}}%
\pgfpathlineto{\pgfqpoint{1.736616in}{1.614316in}}%
\pgfpathlineto{\pgfqpoint{1.737024in}{1.630819in}}%
\pgfpathlineto{\pgfqpoint{1.738249in}{1.861864in}}%
\pgfpathlineto{\pgfqpoint{1.738453in}{1.837109in}}%
\pgfpathlineto{\pgfqpoint{1.739065in}{1.771096in}}%
\pgfpathlineto{\pgfqpoint{1.739678in}{1.787599in}}%
\pgfpathlineto{\pgfqpoint{1.739882in}{1.795851in}}%
\pgfpathlineto{\pgfqpoint{1.740903in}{1.952632in}}%
\pgfpathlineto{\pgfqpoint{1.741107in}{1.936128in}}%
\pgfpathlineto{\pgfqpoint{1.741311in}{1.738090in}}%
\pgfpathlineto{\pgfqpoint{1.741515in}{2.018644in}}%
\pgfpathlineto{\pgfqpoint{1.741923in}{2.010393in}}%
\pgfpathlineto{\pgfqpoint{1.742128in}{2.051651in}}%
\pgfpathlineto{\pgfqpoint{1.743148in}{2.043399in}}%
\pgfpathlineto{\pgfqpoint{1.743352in}{2.035148in}}%
\pgfpathlineto{\pgfqpoint{1.743965in}{1.861864in}}%
\pgfpathlineto{\pgfqpoint{1.744373in}{1.944380in}}%
\pgfpathlineto{\pgfqpoint{1.745802in}{2.051651in}}%
\pgfpathlineto{\pgfqpoint{1.746006in}{2.051651in}}%
\pgfpathlineto{\pgfqpoint{1.746619in}{2.084657in}}%
\pgfpathlineto{\pgfqpoint{1.746823in}{1.944380in}}%
\pgfpathlineto{\pgfqpoint{1.747639in}{2.117664in}}%
\pgfpathlineto{\pgfqpoint{1.747844in}{2.117664in}}%
\pgfpathlineto{\pgfqpoint{1.748048in}{2.084657in}}%
\pgfpathlineto{\pgfqpoint{1.748456in}{2.150670in}}%
\pgfpathlineto{\pgfqpoint{1.748660in}{2.224934in}}%
\pgfpathlineto{\pgfqpoint{1.749477in}{2.142418in}}%
\pgfpathlineto{\pgfqpoint{1.750293in}{2.084657in}}%
\pgfpathlineto{\pgfqpoint{1.750702in}{2.109412in}}%
\pgfpathlineto{\pgfqpoint{1.751314in}{2.175425in}}%
\pgfpathlineto{\pgfqpoint{1.751722in}{2.117664in}}%
\pgfpathlineto{\pgfqpoint{1.752743in}{2.043399in}}%
\pgfpathlineto{\pgfqpoint{1.752335in}{2.134167in}}%
\pgfpathlineto{\pgfqpoint{1.753355in}{2.092909in}}%
\pgfpathlineto{\pgfqpoint{1.754784in}{2.208431in}}%
\pgfpathlineto{\pgfqpoint{1.754988in}{2.026896in}}%
\pgfpathlineto{\pgfqpoint{1.755805in}{2.282696in}}%
\pgfpathlineto{\pgfqpoint{1.756213in}{2.183676in}}%
\pgfpathlineto{\pgfqpoint{1.756622in}{2.224934in}}%
\pgfpathlineto{\pgfqpoint{1.756826in}{2.051651in}}%
\pgfpathlineto{\pgfqpoint{1.757642in}{2.200180in}}%
\pgfpathlineto{\pgfqpoint{1.757846in}{2.191928in}}%
\pgfpathlineto{\pgfqpoint{1.758459in}{2.299199in}}%
\pgfpathlineto{\pgfqpoint{1.759071in}{2.266192in}}%
\pgfpathlineto{\pgfqpoint{1.759275in}{2.233186in}}%
\pgfpathlineto{\pgfqpoint{1.759888in}{2.290947in}}%
\pgfpathlineto{\pgfqpoint{1.760500in}{2.332205in}}%
\pgfpathlineto{\pgfqpoint{1.760704in}{2.282696in}}%
\pgfpathlineto{\pgfqpoint{1.760909in}{2.299199in}}%
\pgfpathlineto{\pgfqpoint{1.761929in}{2.134167in}}%
\pgfpathlineto{\pgfqpoint{1.762542in}{2.167173in}}%
\pgfpathlineto{\pgfqpoint{1.762746in}{2.224934in}}%
\pgfpathlineto{\pgfqpoint{1.763562in}{2.200180in}}%
\pgfpathlineto{\pgfqpoint{1.763767in}{2.142418in}}%
\pgfpathlineto{\pgfqpoint{1.764175in}{2.233186in}}%
\pgfpathlineto{\pgfqpoint{1.764379in}{2.183676in}}%
\pgfpathlineto{\pgfqpoint{1.765808in}{2.299199in}}%
\pgfpathlineto{\pgfqpoint{1.766625in}{2.233186in}}%
\pgfpathlineto{\pgfqpoint{1.767033in}{2.282696in}}%
\pgfpathlineto{\pgfqpoint{1.767645in}{2.332205in}}%
\pgfpathlineto{\pgfqpoint{1.768054in}{2.290947in}}%
\pgfpathlineto{\pgfqpoint{1.768258in}{2.257941in}}%
\pgfpathlineto{\pgfqpoint{1.768666in}{2.356960in}}%
\pgfpathlineto{\pgfqpoint{1.769074in}{2.299199in}}%
\pgfpathlineto{\pgfqpoint{1.769483in}{2.348709in}}%
\pgfpathlineto{\pgfqpoint{1.769687in}{2.233186in}}%
\pgfpathlineto{\pgfqpoint{1.770503in}{2.332205in}}%
\pgfpathlineto{\pgfqpoint{1.770707in}{2.323954in}}%
\pgfpathlineto{\pgfqpoint{1.770912in}{2.365212in}}%
\pgfpathlineto{\pgfqpoint{1.771320in}{2.315702in}}%
\pgfpathlineto{\pgfqpoint{1.771728in}{2.340457in}}%
\pgfpathlineto{\pgfqpoint{1.772341in}{2.282696in}}%
\pgfpathlineto{\pgfqpoint{1.772749in}{2.348709in}}%
\pgfpathlineto{\pgfqpoint{1.772953in}{2.356960in}}%
\pgfpathlineto{\pgfqpoint{1.774178in}{2.266192in}}%
\pgfpathlineto{\pgfqpoint{1.775199in}{2.373463in}}%
\pgfpathlineto{\pgfqpoint{1.775607in}{2.365212in}}%
\pgfpathlineto{\pgfqpoint{1.776423in}{2.282696in}}%
\pgfpathlineto{\pgfqpoint{1.776628in}{2.315702in}}%
\pgfpathlineto{\pgfqpoint{1.776832in}{2.373463in}}%
\pgfpathlineto{\pgfqpoint{1.777240in}{2.266192in}}%
\pgfpathlineto{\pgfqpoint{1.777648in}{2.315702in}}%
\pgfpathlineto{\pgfqpoint{1.777852in}{2.290947in}}%
\pgfpathlineto{\pgfqpoint{1.778465in}{2.340457in}}%
\pgfpathlineto{\pgfqpoint{1.778669in}{2.348709in}}%
\pgfpathlineto{\pgfqpoint{1.779486in}{2.233186in}}%
\pgfpathlineto{\pgfqpoint{1.779894in}{2.307450in}}%
\pgfpathlineto{\pgfqpoint{1.780098in}{2.340457in}}%
\pgfpathlineto{\pgfqpoint{1.780506in}{2.249689in}}%
\pgfpathlineto{\pgfqpoint{1.780710in}{2.299199in}}%
\pgfpathlineto{\pgfqpoint{1.780915in}{2.282696in}}%
\pgfpathlineto{\pgfqpoint{1.781323in}{2.315702in}}%
\pgfpathlineto{\pgfqpoint{1.781527in}{2.340457in}}%
\pgfpathlineto{\pgfqpoint{1.781935in}{2.274444in}}%
\pgfpathlineto{\pgfqpoint{1.782139in}{2.282696in}}%
\pgfpathlineto{\pgfqpoint{1.783160in}{2.224934in}}%
\pgfpathlineto{\pgfqpoint{1.783364in}{2.257941in}}%
\pgfpathlineto{\pgfqpoint{1.784793in}{2.431225in}}%
\pgfpathlineto{\pgfqpoint{1.784997in}{2.431225in}}%
\pgfpathlineto{\pgfqpoint{1.785202in}{2.497237in}}%
\pgfpathlineto{\pgfqpoint{1.786222in}{2.472483in}}%
\pgfpathlineto{\pgfqpoint{1.786631in}{2.521992in}}%
\pgfpathlineto{\pgfqpoint{1.787855in}{2.389967in}}%
\pgfpathlineto{\pgfqpoint{1.788672in}{2.538495in}}%
\pgfpathlineto{\pgfqpoint{1.788876in}{2.530244in}}%
\pgfpathlineto{\pgfqpoint{1.789080in}{2.290947in}}%
\pgfpathlineto{\pgfqpoint{1.789897in}{2.604508in}}%
\pgfpathlineto{\pgfqpoint{1.790101in}{2.629263in}}%
\pgfpathlineto{\pgfqpoint{1.790305in}{2.521992in}}%
\pgfpathlineto{\pgfqpoint{1.790509in}{2.538495in}}%
\pgfpathlineto{\pgfqpoint{1.791122in}{2.422973in}}%
\pgfpathlineto{\pgfqpoint{1.791734in}{2.464231in}}%
\pgfpathlineto{\pgfqpoint{1.792142in}{2.464231in}}%
\pgfpathlineto{\pgfqpoint{1.792347in}{2.472483in}}%
\pgfpathlineto{\pgfqpoint{1.792551in}{2.414721in}}%
\pgfpathlineto{\pgfqpoint{1.792755in}{2.538495in}}%
\pgfpathlineto{\pgfqpoint{1.793571in}{2.431225in}}%
\pgfpathlineto{\pgfqpoint{1.793776in}{2.431225in}}%
\pgfpathlineto{\pgfqpoint{1.794388in}{2.422973in}}%
\pgfpathlineto{\pgfqpoint{1.794796in}{2.480734in}}%
\pgfpathlineto{\pgfqpoint{1.795409in}{2.439476in}}%
\pgfpathlineto{\pgfqpoint{1.795613in}{2.431225in}}%
\pgfpathlineto{\pgfqpoint{1.795817in}{2.480734in}}%
\pgfpathlineto{\pgfqpoint{1.796634in}{2.472483in}}%
\pgfpathlineto{\pgfqpoint{1.797042in}{2.389967in}}%
\pgfpathlineto{\pgfqpoint{1.797654in}{2.472483in}}%
\pgfpathlineto{\pgfqpoint{1.799492in}{2.563250in}}%
\pgfpathlineto{\pgfqpoint{1.800104in}{2.513741in}}%
\pgfpathlineto{\pgfqpoint{1.800308in}{2.546747in}}%
\pgfpathlineto{\pgfqpoint{1.801125in}{2.629263in}}%
\pgfpathlineto{\pgfqpoint{1.801329in}{2.554999in}}%
\pgfpathlineto{\pgfqpoint{1.801533in}{2.546747in}}%
\pgfpathlineto{\pgfqpoint{1.802554in}{2.406470in}}%
\pgfpathlineto{\pgfqpoint{1.803574in}{2.521992in}}%
\pgfpathlineto{\pgfqpoint{1.803779in}{2.513741in}}%
\pgfpathlineto{\pgfqpoint{1.804187in}{2.472483in}}%
\pgfpathlineto{\pgfqpoint{1.804595in}{2.554999in}}%
\pgfpathlineto{\pgfqpoint{1.804799in}{2.505489in}}%
\pgfpathlineto{\pgfqpoint{1.806024in}{2.571502in}}%
\pgfpathlineto{\pgfqpoint{1.807045in}{2.513741in}}%
\pgfpathlineto{\pgfqpoint{1.807249in}{2.571502in}}%
\pgfpathlineto{\pgfqpoint{1.807861in}{2.472483in}}%
\pgfpathlineto{\pgfqpoint{1.808066in}{2.505489in}}%
\pgfpathlineto{\pgfqpoint{1.809086in}{2.249689in}}%
\pgfpathlineto{\pgfqpoint{1.809699in}{2.340457in}}%
\pgfpathlineto{\pgfqpoint{1.811740in}{2.604508in}}%
\pgfpathlineto{\pgfqpoint{1.812148in}{2.629263in}}%
\pgfpathlineto{\pgfqpoint{1.812353in}{2.596257in}}%
\pgfpathlineto{\pgfqpoint{1.812965in}{2.299199in}}%
\pgfpathlineto{\pgfqpoint{1.813373in}{2.414721in}}%
\pgfpathlineto{\pgfqpoint{1.813577in}{2.654018in}}%
\pgfpathlineto{\pgfqpoint{1.814598in}{2.579753in}}%
\pgfpathlineto{\pgfqpoint{1.815619in}{2.365212in}}%
\pgfpathlineto{\pgfqpoint{1.815823in}{2.596257in}}%
\pgfpathlineto{\pgfqpoint{1.816435in}{2.521992in}}%
\pgfpathlineto{\pgfqpoint{1.816640in}{2.257941in}}%
\pgfpathlineto{\pgfqpoint{1.817456in}{2.621011in}}%
\pgfpathlineto{\pgfqpoint{1.818273in}{2.373463in}}%
\pgfpathlineto{\pgfqpoint{1.819089in}{2.464231in}}%
\pgfpathlineto{\pgfqpoint{1.819498in}{2.480734in}}%
\pgfpathlineto{\pgfqpoint{1.819702in}{2.447728in}}%
\pgfpathlineto{\pgfqpoint{1.820314in}{2.480734in}}%
\pgfpathlineto{\pgfqpoint{1.820927in}{2.340457in}}%
\pgfpathlineto{\pgfqpoint{1.821539in}{2.497237in}}%
\pgfpathlineto{\pgfqpoint{1.822151in}{2.480734in}}%
\pgfpathlineto{\pgfqpoint{1.822356in}{2.538495in}}%
\pgfpathlineto{\pgfqpoint{1.822968in}{2.414721in}}%
\pgfpathlineto{\pgfqpoint{1.824601in}{2.538495in}}%
\pgfpathlineto{\pgfqpoint{1.825009in}{2.216683in}}%
\pgfpathlineto{\pgfqpoint{1.825622in}{2.414721in}}%
\pgfpathlineto{\pgfqpoint{1.826030in}{2.538495in}}%
\pgfpathlineto{\pgfqpoint{1.826438in}{2.406470in}}%
\pgfpathlineto{\pgfqpoint{1.826643in}{2.414721in}}%
\pgfpathlineto{\pgfqpoint{1.827459in}{2.315702in}}%
\pgfpathlineto{\pgfqpoint{1.827051in}{2.472483in}}%
\pgfpathlineto{\pgfqpoint{1.827867in}{2.373463in}}%
\pgfpathlineto{\pgfqpoint{1.828072in}{2.373463in}}%
\pgfpathlineto{\pgfqpoint{1.828480in}{2.398218in}}%
\pgfpathlineto{\pgfqpoint{1.829705in}{2.257941in}}%
\pgfpathlineto{\pgfqpoint{1.829909in}{2.266192in}}%
\pgfpathlineto{\pgfqpoint{1.830113in}{2.307450in}}%
\pgfpathlineto{\pgfqpoint{1.830317in}{2.233186in}}%
\pgfpathlineto{\pgfqpoint{1.830930in}{2.257941in}}%
\pgfpathlineto{\pgfqpoint{1.831950in}{2.208431in}}%
\pgfpathlineto{\pgfqpoint{1.833175in}{2.274444in}}%
\pgfpathlineto{\pgfqpoint{1.833583in}{2.158922in}}%
\pgfpathlineto{\pgfqpoint{1.834196in}{2.233186in}}%
\pgfpathlineto{\pgfqpoint{1.834400in}{2.257941in}}%
\pgfpathlineto{\pgfqpoint{1.834604in}{2.216683in}}%
\pgfpathlineto{\pgfqpoint{1.835217in}{2.224934in}}%
\pgfpathlineto{\pgfqpoint{1.835421in}{2.224934in}}%
\pgfpathlineto{\pgfqpoint{1.835625in}{2.274444in}}%
\pgfpathlineto{\pgfqpoint{1.836033in}{2.208431in}}%
\pgfpathlineto{\pgfqpoint{1.836441in}{2.208431in}}%
\pgfpathlineto{\pgfqpoint{1.836850in}{2.134167in}}%
\pgfpathlineto{\pgfqpoint{1.837054in}{2.208431in}}%
\pgfpathlineto{\pgfqpoint{1.837258in}{2.290947in}}%
\pgfpathlineto{\pgfqpoint{1.837870in}{2.183676in}}%
\pgfpathlineto{\pgfqpoint{1.838074in}{2.183676in}}%
\pgfpathlineto{\pgfqpoint{1.838483in}{2.200180in}}%
\pgfpathlineto{\pgfqpoint{1.839095in}{2.158922in}}%
\pgfpathlineto{\pgfqpoint{1.839299in}{1.985638in}}%
\pgfpathlineto{\pgfqpoint{1.839912in}{2.241438in}}%
\pgfpathlineto{\pgfqpoint{1.840116in}{2.249689in}}%
\pgfpathlineto{\pgfqpoint{1.840320in}{2.241438in}}%
\pgfpathlineto{\pgfqpoint{1.840932in}{1.944380in}}%
\pgfpathlineto{\pgfqpoint{1.841341in}{2.208431in}}%
\pgfpathlineto{\pgfqpoint{1.842566in}{2.266192in}}%
\pgfpathlineto{\pgfqpoint{1.842974in}{2.233186in}}%
\pgfpathlineto{\pgfqpoint{1.843178in}{2.224934in}}%
\pgfpathlineto{\pgfqpoint{1.843382in}{2.257941in}}%
\pgfpathlineto{\pgfqpoint{1.843586in}{2.257941in}}%
\pgfpathlineto{\pgfqpoint{1.843790in}{2.282696in}}%
\pgfpathlineto{\pgfqpoint{1.844199in}{2.241438in}}%
\pgfpathlineto{\pgfqpoint{1.844403in}{2.257941in}}%
\pgfpathlineto{\pgfqpoint{1.844607in}{2.200180in}}%
\pgfpathlineto{\pgfqpoint{1.845015in}{2.274444in}}%
\pgfpathlineto{\pgfqpoint{1.845832in}{2.389967in}}%
\pgfpathlineto{\pgfqpoint{1.846036in}{2.323954in}}%
\pgfpathlineto{\pgfqpoint{1.846240in}{2.315702in}}%
\pgfpathlineto{\pgfqpoint{1.846444in}{2.323954in}}%
\pgfpathlineto{\pgfqpoint{1.847057in}{2.365212in}}%
\pgfpathlineto{\pgfqpoint{1.846853in}{2.299199in}}%
\pgfpathlineto{\pgfqpoint{1.847465in}{2.323954in}}%
\pgfpathlineto{\pgfqpoint{1.847873in}{2.299199in}}%
\pgfpathlineto{\pgfqpoint{1.848486in}{2.340457in}}%
\pgfpathlineto{\pgfqpoint{1.849302in}{2.323954in}}%
\pgfpathlineto{\pgfqpoint{1.849506in}{2.365212in}}%
\pgfpathlineto{\pgfqpoint{1.849915in}{2.340457in}}%
\pgfpathlineto{\pgfqpoint{1.850119in}{2.241438in}}%
\pgfpathlineto{\pgfqpoint{1.850935in}{2.365212in}}%
\pgfpathlineto{\pgfqpoint{1.851140in}{2.389967in}}%
\pgfpathlineto{\pgfqpoint{1.851344in}{2.348709in}}%
\pgfpathlineto{\pgfqpoint{1.851548in}{2.266192in}}%
\pgfpathlineto{\pgfqpoint{1.852364in}{2.315702in}}%
\pgfpathlineto{\pgfqpoint{1.853589in}{2.224934in}}%
\pgfpathlineto{\pgfqpoint{1.853793in}{2.208431in}}%
\pgfpathlineto{\pgfqpoint{1.853998in}{2.233186in}}%
\pgfpathlineto{\pgfqpoint{1.854202in}{2.266192in}}%
\pgfpathlineto{\pgfqpoint{1.854406in}{2.208431in}}%
\pgfpathlineto{\pgfqpoint{1.855018in}{2.224934in}}%
\pgfpathlineto{\pgfqpoint{1.855222in}{2.233186in}}%
\pgfpathlineto{\pgfqpoint{1.855427in}{2.158922in}}%
\pgfpathlineto{\pgfqpoint{1.856243in}{2.183676in}}%
\pgfpathlineto{\pgfqpoint{1.857264in}{2.249689in}}%
\pgfpathlineto{\pgfqpoint{1.857468in}{2.183676in}}%
\pgfpathlineto{\pgfqpoint{1.858080in}{2.323954in}}%
\pgfpathlineto{\pgfqpoint{1.858285in}{2.323954in}}%
\pgfpathlineto{\pgfqpoint{1.858693in}{2.290947in}}%
\pgfpathlineto{\pgfqpoint{1.859305in}{2.307450in}}%
\pgfpathlineto{\pgfqpoint{1.859509in}{2.315702in}}%
\pgfpathlineto{\pgfqpoint{1.860122in}{2.373463in}}%
\pgfpathlineto{\pgfqpoint{1.860734in}{2.233186in}}%
\pgfpathlineto{\pgfqpoint{1.861959in}{2.348709in}}%
\pgfpathlineto{\pgfqpoint{1.862163in}{2.323954in}}%
\pgfpathlineto{\pgfqpoint{1.862980in}{2.059902in}}%
\pgfpathlineto{\pgfqpoint{1.863388in}{2.274444in}}%
\pgfpathlineto{\pgfqpoint{1.864205in}{2.257941in}}%
\pgfpathlineto{\pgfqpoint{1.864409in}{2.315702in}}%
\pgfpathlineto{\pgfqpoint{1.865021in}{2.084657in}}%
\pgfpathlineto{\pgfqpoint{1.865838in}{2.233186in}}%
\pgfpathlineto{\pgfqpoint{1.866042in}{2.274444in}}%
\pgfpathlineto{\pgfqpoint{1.866654in}{2.175425in}}%
\pgfpathlineto{\pgfqpoint{1.868492in}{2.348709in}}%
\pgfpathlineto{\pgfqpoint{1.869512in}{2.307450in}}%
\pgfpathlineto{\pgfqpoint{1.869717in}{2.365212in}}%
\pgfpathlineto{\pgfqpoint{1.870533in}{2.307450in}}%
\pgfpathlineto{\pgfqpoint{1.870737in}{2.274444in}}%
\pgfpathlineto{\pgfqpoint{1.871350in}{2.348709in}}%
\pgfpathlineto{\pgfqpoint{1.871554in}{2.307450in}}%
\pgfpathlineto{\pgfqpoint{1.871962in}{2.257941in}}%
\pgfpathlineto{\pgfqpoint{1.872575in}{2.389967in}}%
\pgfpathlineto{\pgfqpoint{1.873799in}{2.290947in}}%
\pgfpathlineto{\pgfqpoint{1.874208in}{2.282696in}}%
\pgfpathlineto{\pgfqpoint{1.875024in}{2.373463in}}%
\pgfpathlineto{\pgfqpoint{1.875433in}{2.315702in}}%
\pgfpathlineto{\pgfqpoint{1.875841in}{2.323954in}}%
\pgfpathlineto{\pgfqpoint{1.876249in}{2.373463in}}%
\pgfpathlineto{\pgfqpoint{1.877066in}{2.257941in}}%
\pgfpathlineto{\pgfqpoint{1.877678in}{2.315702in}}%
\pgfpathlineto{\pgfqpoint{1.878291in}{2.307450in}}%
\pgfpathlineto{\pgfqpoint{1.878903in}{2.299199in}}%
\pgfpathlineto{\pgfqpoint{1.879107in}{2.340457in}}%
\pgfpathlineto{\pgfqpoint{1.879515in}{2.274444in}}%
\pgfpathlineto{\pgfqpoint{1.879924in}{2.356960in}}%
\pgfpathlineto{\pgfqpoint{1.880128in}{2.373463in}}%
\pgfpathlineto{\pgfqpoint{1.880332in}{2.365212in}}%
\pgfpathlineto{\pgfqpoint{1.881353in}{2.059902in}}%
\pgfpathlineto{\pgfqpoint{1.881557in}{2.101160in}}%
\pgfpathlineto{\pgfqpoint{1.881965in}{2.431225in}}%
\pgfpathlineto{\pgfqpoint{1.882782in}{2.389967in}}%
\pgfpathlineto{\pgfqpoint{1.884007in}{2.084657in}}%
\pgfpathlineto{\pgfqpoint{1.885027in}{2.348709in}}%
\pgfpathlineto{\pgfqpoint{1.885231in}{2.332205in}}%
\pgfpathlineto{\pgfqpoint{1.886048in}{2.414721in}}%
\pgfpathlineto{\pgfqpoint{1.886252in}{2.373463in}}%
\pgfpathlineto{\pgfqpoint{1.887273in}{2.117664in}}%
\pgfpathlineto{\pgfqpoint{1.886865in}{2.422973in}}%
\pgfpathlineto{\pgfqpoint{1.887477in}{2.224934in}}%
\pgfpathlineto{\pgfqpoint{1.888702in}{2.414721in}}%
\pgfpathlineto{\pgfqpoint{1.888906in}{2.406470in}}%
\pgfpathlineto{\pgfqpoint{1.889723in}{2.455979in}}%
\pgfpathlineto{\pgfqpoint{1.890947in}{2.290947in}}%
\pgfpathlineto{\pgfqpoint{1.891968in}{2.464231in}}%
\pgfpathlineto{\pgfqpoint{1.891560in}{2.183676in}}%
\pgfpathlineto{\pgfqpoint{1.892172in}{2.414721in}}%
\pgfpathlineto{\pgfqpoint{1.892376in}{2.150670in}}%
\pgfpathlineto{\pgfqpoint{1.893193in}{2.299199in}}%
\pgfpathlineto{\pgfqpoint{1.893397in}{2.373463in}}%
\pgfpathlineto{\pgfqpoint{1.893805in}{2.134167in}}%
\pgfpathlineto{\pgfqpoint{1.894010in}{2.068154in}}%
\pgfpathlineto{\pgfqpoint{1.894214in}{2.323954in}}%
\pgfpathlineto{\pgfqpoint{1.894418in}{2.323954in}}%
\pgfpathlineto{\pgfqpoint{1.894826in}{2.340457in}}%
\pgfpathlineto{\pgfqpoint{1.895030in}{2.414721in}}%
\pgfpathlineto{\pgfqpoint{1.895439in}{2.365212in}}%
\pgfpathlineto{\pgfqpoint{1.895643in}{2.233186in}}%
\pgfpathlineto{\pgfqpoint{1.896663in}{2.282696in}}%
\pgfpathlineto{\pgfqpoint{1.897480in}{2.464231in}}%
\pgfpathlineto{\pgfqpoint{1.897888in}{2.414721in}}%
\pgfpathlineto{\pgfqpoint{1.898705in}{2.422973in}}%
\pgfpathlineto{\pgfqpoint{1.899317in}{2.348709in}}%
\pgfpathlineto{\pgfqpoint{1.900134in}{2.340457in}}%
\pgfpathlineto{\pgfqpoint{1.900746in}{2.431225in}}%
\pgfpathlineto{\pgfqpoint{1.901563in}{2.299199in}}%
\pgfpathlineto{\pgfqpoint{1.902175in}{2.365212in}}%
\pgfpathlineto{\pgfqpoint{1.902379in}{2.365212in}}%
\pgfpathlineto{\pgfqpoint{1.903604in}{2.084657in}}%
\pgfpathlineto{\pgfqpoint{1.902992in}{2.381715in}}%
\pgfpathlineto{\pgfqpoint{1.903808in}{2.299199in}}%
\pgfpathlineto{\pgfqpoint{1.905033in}{2.381715in}}%
\pgfpathlineto{\pgfqpoint{1.905442in}{2.365212in}}%
\pgfpathlineto{\pgfqpoint{1.906666in}{2.274444in}}%
\pgfpathlineto{\pgfqpoint{1.905850in}{2.389967in}}%
\pgfpathlineto{\pgfqpoint{1.907075in}{2.299199in}}%
\pgfpathlineto{\pgfqpoint{1.908095in}{2.414721in}}%
\pgfpathlineto{\pgfqpoint{1.908300in}{2.348709in}}%
\pgfpathlineto{\pgfqpoint{1.908504in}{2.356960in}}%
\pgfpathlineto{\pgfqpoint{1.908708in}{2.332205in}}%
\pgfpathlineto{\pgfqpoint{1.908912in}{2.332205in}}%
\pgfpathlineto{\pgfqpoint{1.909116in}{2.323954in}}%
\pgfpathlineto{\pgfqpoint{1.909524in}{2.307450in}}%
\pgfpathlineto{\pgfqpoint{1.910749in}{2.439476in}}%
\pgfpathlineto{\pgfqpoint{1.912791in}{2.340457in}}%
\pgfpathlineto{\pgfqpoint{1.912995in}{2.406470in}}%
\pgfpathlineto{\pgfqpoint{1.913811in}{2.323954in}}%
\pgfpathlineto{\pgfqpoint{1.914016in}{2.381715in}}%
\pgfpathlineto{\pgfqpoint{1.914220in}{2.282696in}}%
\pgfpathlineto{\pgfqpoint{1.915036in}{2.406470in}}%
\pgfpathlineto{\pgfqpoint{1.915853in}{2.307450in}}%
\pgfpathlineto{\pgfqpoint{1.915445in}{2.422973in}}%
\pgfpathlineto{\pgfqpoint{1.916057in}{2.356960in}}%
\pgfpathlineto{\pgfqpoint{1.916261in}{2.414721in}}%
\pgfpathlineto{\pgfqpoint{1.917078in}{2.340457in}}%
\pgfpathlineto{\pgfqpoint{1.917282in}{2.332205in}}%
\pgfpathlineto{\pgfqpoint{1.917486in}{2.365212in}}%
\pgfpathlineto{\pgfqpoint{1.917690in}{2.373463in}}%
\pgfpathlineto{\pgfqpoint{1.918711in}{2.241438in}}%
\pgfpathlineto{\pgfqpoint{1.918915in}{2.340457in}}%
\pgfpathlineto{\pgfqpoint{1.919323in}{2.332205in}}%
\pgfpathlineto{\pgfqpoint{1.919527in}{2.348709in}}%
\pgfpathlineto{\pgfqpoint{1.919936in}{2.315702in}}%
\pgfpathlineto{\pgfqpoint{1.920956in}{2.422973in}}%
\pgfpathlineto{\pgfqpoint{1.922589in}{2.356960in}}%
\pgfpathlineto{\pgfqpoint{1.922794in}{2.439476in}}%
\pgfpathlineto{\pgfqpoint{1.923406in}{2.323954in}}%
\pgfpathlineto{\pgfqpoint{1.924223in}{2.142418in}}%
\pgfpathlineto{\pgfqpoint{1.923814in}{2.381715in}}%
\pgfpathlineto{\pgfqpoint{1.924427in}{2.167173in}}%
\pgfpathlineto{\pgfqpoint{1.925447in}{2.389967in}}%
\pgfpathlineto{\pgfqpoint{1.925652in}{2.365212in}}%
\pgfpathlineto{\pgfqpoint{1.926264in}{2.084657in}}%
\pgfpathlineto{\pgfqpoint{1.926672in}{2.381715in}}%
\pgfpathlineto{\pgfqpoint{1.926876in}{2.365212in}}%
\pgfpathlineto{\pgfqpoint{1.927081in}{2.406470in}}%
\pgfpathlineto{\pgfqpoint{1.927489in}{2.464231in}}%
\pgfpathlineto{\pgfqpoint{1.927693in}{2.414721in}}%
\pgfpathlineto{\pgfqpoint{1.928510in}{2.340457in}}%
\pgfpathlineto{\pgfqpoint{1.928714in}{2.389967in}}%
\pgfpathlineto{\pgfqpoint{1.929326in}{2.414721in}}%
\pgfpathlineto{\pgfqpoint{1.930755in}{2.299199in}}%
\pgfpathlineto{\pgfqpoint{1.930959in}{2.274444in}}%
\pgfpathlineto{\pgfqpoint{1.931163in}{2.356960in}}%
\pgfpathlineto{\pgfqpoint{1.931776in}{2.348709in}}%
\pgfpathlineto{\pgfqpoint{1.932388in}{2.373463in}}%
\pgfpathlineto{\pgfqpoint{1.932592in}{2.340457in}}%
\pgfpathlineto{\pgfqpoint{1.932797in}{2.365212in}}%
\pgfpathlineto{\pgfqpoint{1.933001in}{2.092909in}}%
\pgfpathlineto{\pgfqpoint{1.933817in}{2.233186in}}%
\pgfpathlineto{\pgfqpoint{1.935450in}{2.299199in}}%
\pgfpathlineto{\pgfqpoint{1.935655in}{2.290947in}}%
\pgfpathlineto{\pgfqpoint{1.936471in}{2.257941in}}%
\pgfpathlineto{\pgfqpoint{1.936879in}{2.340457in}}%
\pgfpathlineto{\pgfqpoint{1.938104in}{2.274444in}}%
\pgfpathlineto{\pgfqpoint{1.937288in}{2.373463in}}%
\pgfpathlineto{\pgfqpoint{1.938308in}{2.299199in}}%
\pgfpathlineto{\pgfqpoint{1.939125in}{2.340457in}}%
\pgfpathlineto{\pgfqpoint{1.940554in}{2.216683in}}%
\pgfpathlineto{\pgfqpoint{1.941779in}{2.348709in}}%
\pgfpathlineto{\pgfqpoint{1.942187in}{2.356960in}}%
\pgfpathlineto{\pgfqpoint{1.942391in}{2.323954in}}%
\pgfpathlineto{\pgfqpoint{1.942595in}{2.381715in}}%
\pgfpathlineto{\pgfqpoint{1.943004in}{2.373463in}}%
\pgfpathlineto{\pgfqpoint{1.943208in}{2.381715in}}%
\pgfpathlineto{\pgfqpoint{1.943412in}{2.373463in}}%
\pgfpathlineto{\pgfqpoint{1.944024in}{2.282696in}}%
\pgfpathlineto{\pgfqpoint{1.944637in}{2.340457in}}%
\pgfpathlineto{\pgfqpoint{1.944841in}{2.332205in}}%
\pgfpathlineto{\pgfqpoint{1.945045in}{2.340457in}}%
\pgfpathlineto{\pgfqpoint{1.945658in}{2.505489in}}%
\pgfpathlineto{\pgfqpoint{1.946066in}{2.398218in}}%
\pgfpathlineto{\pgfqpoint{1.947087in}{2.315702in}}%
\pgfpathlineto{\pgfqpoint{1.947291in}{2.340457in}}%
\pgfpathlineto{\pgfqpoint{1.947495in}{2.315702in}}%
\pgfpathlineto{\pgfqpoint{1.947699in}{2.365212in}}%
\pgfpathlineto{\pgfqpoint{1.948924in}{2.431225in}}%
\pgfpathlineto{\pgfqpoint{1.949536in}{2.381715in}}%
\pgfpathlineto{\pgfqpoint{1.949740in}{2.332205in}}%
\pgfpathlineto{\pgfqpoint{1.950353in}{2.422973in}}%
\pgfpathlineto{\pgfqpoint{1.950761in}{2.398218in}}%
\pgfpathlineto{\pgfqpoint{1.950965in}{2.406470in}}%
\pgfpathlineto{\pgfqpoint{1.951169in}{2.455979in}}%
\pgfpathlineto{\pgfqpoint{1.951782in}{2.414721in}}%
\pgfpathlineto{\pgfqpoint{1.952394in}{2.447728in}}%
\pgfpathlineto{\pgfqpoint{1.953415in}{2.175425in}}%
\pgfpathlineto{\pgfqpoint{1.954640in}{2.431225in}}%
\pgfpathlineto{\pgfqpoint{1.954844in}{2.414721in}}%
\pgfpathlineto{\pgfqpoint{1.955048in}{2.431225in}}%
\pgfpathlineto{\pgfqpoint{1.955252in}{2.365212in}}%
\pgfpathlineto{\pgfqpoint{1.955661in}{2.373463in}}%
\pgfpathlineto{\pgfqpoint{1.955865in}{2.472483in}}%
\pgfpathlineto{\pgfqpoint{1.956681in}{2.389967in}}%
\pgfpathlineto{\pgfqpoint{1.957498in}{2.472483in}}%
\pgfpathlineto{\pgfqpoint{1.957702in}{2.414721in}}%
\pgfpathlineto{\pgfqpoint{1.958110in}{2.315702in}}%
\pgfpathlineto{\pgfqpoint{1.959131in}{2.373463in}}%
\pgfpathlineto{\pgfqpoint{1.959335in}{2.480734in}}%
\pgfpathlineto{\pgfqpoint{1.960152in}{2.365212in}}%
\pgfpathlineto{\pgfqpoint{1.960356in}{2.224934in}}%
\pgfpathlineto{\pgfqpoint{1.960764in}{2.472483in}}%
\pgfpathlineto{\pgfqpoint{1.961172in}{2.356960in}}%
\pgfpathlineto{\pgfqpoint{1.962397in}{2.546747in}}%
\pgfpathlineto{\pgfqpoint{1.962806in}{2.332205in}}%
\pgfpathlineto{\pgfqpoint{1.963010in}{2.588005in}}%
\pgfpathlineto{\pgfqpoint{1.963418in}{2.546747in}}%
\pgfpathlineto{\pgfqpoint{1.964030in}{2.497237in}}%
\pgfpathlineto{\pgfqpoint{1.964235in}{2.365212in}}%
\pgfpathlineto{\pgfqpoint{1.965051in}{2.505489in}}%
\pgfpathlineto{\pgfqpoint{1.965664in}{2.257941in}}%
\pgfpathlineto{\pgfqpoint{1.966072in}{2.513741in}}%
\pgfpathlineto{\pgfqpoint{1.966480in}{2.464231in}}%
\pgfpathlineto{\pgfqpoint{1.966684in}{2.216683in}}%
\pgfpathlineto{\pgfqpoint{1.967093in}{2.497237in}}%
\pgfpathlineto{\pgfqpoint{1.967501in}{2.455979in}}%
\pgfpathlineto{\pgfqpoint{1.968317in}{2.497237in}}%
\pgfpathlineto{\pgfqpoint{1.968113in}{2.439476in}}%
\pgfpathlineto{\pgfqpoint{1.968726in}{2.488986in}}%
\pgfpathlineto{\pgfqpoint{1.968930in}{2.505489in}}%
\pgfpathlineto{\pgfqpoint{1.969338in}{2.497237in}}%
\pgfpathlineto{\pgfqpoint{1.970155in}{2.356960in}}%
\pgfpathlineto{\pgfqpoint{1.969746in}{2.505489in}}%
\pgfpathlineto{\pgfqpoint{1.970359in}{2.431225in}}%
\pgfpathlineto{\pgfqpoint{1.971380in}{2.530244in}}%
\pgfpathlineto{\pgfqpoint{1.970971in}{2.389967in}}%
\pgfpathlineto{\pgfqpoint{1.971584in}{2.521992in}}%
\pgfpathlineto{\pgfqpoint{1.971992in}{2.431225in}}%
\pgfpathlineto{\pgfqpoint{1.972604in}{2.538495in}}%
\pgfpathlineto{\pgfqpoint{1.973013in}{2.488986in}}%
\pgfpathlineto{\pgfqpoint{1.973217in}{2.530244in}}%
\pgfpathlineto{\pgfqpoint{1.973421in}{2.389967in}}%
\pgfpathlineto{\pgfqpoint{1.974238in}{2.505489in}}%
\pgfpathlineto{\pgfqpoint{1.974850in}{2.472483in}}%
\pgfpathlineto{\pgfqpoint{1.975667in}{2.480734in}}%
\pgfpathlineto{\pgfqpoint{1.975871in}{2.497237in}}%
\pgfpathlineto{\pgfqpoint{1.976279in}{2.480734in}}%
\pgfpathlineto{\pgfqpoint{1.976483in}{2.439476in}}%
\pgfpathlineto{\pgfqpoint{1.976687in}{2.488986in}}%
\pgfpathlineto{\pgfqpoint{1.977300in}{2.472483in}}%
\pgfpathlineto{\pgfqpoint{1.978320in}{2.406470in}}%
\pgfpathlineto{\pgfqpoint{1.978933in}{2.332205in}}%
\pgfpathlineto{\pgfqpoint{1.979341in}{2.389967in}}%
\pgfpathlineto{\pgfqpoint{1.979545in}{2.414721in}}%
\pgfpathlineto{\pgfqpoint{1.979954in}{2.348709in}}%
\pgfpathlineto{\pgfqpoint{1.980158in}{2.365212in}}%
\pgfpathlineto{\pgfqpoint{1.981178in}{2.422973in}}%
\pgfpathlineto{\pgfqpoint{1.981383in}{2.406470in}}%
\pgfpathlineto{\pgfqpoint{1.982812in}{2.117664in}}%
\pgfpathlineto{\pgfqpoint{1.984036in}{2.233186in}}%
\pgfpathlineto{\pgfqpoint{1.985465in}{1.952632in}}%
\pgfpathlineto{\pgfqpoint{1.985874in}{2.059902in}}%
\pgfpathlineto{\pgfqpoint{1.986486in}{1.960883in}}%
\pgfpathlineto{\pgfqpoint{1.987507in}{2.018644in}}%
\pgfpathlineto{\pgfqpoint{1.988732in}{1.870115in}}%
\pgfpathlineto{\pgfqpoint{1.988936in}{1.886619in}}%
\pgfpathlineto{\pgfqpoint{1.989140in}{1.886619in}}%
\pgfpathlineto{\pgfqpoint{1.989548in}{1.680329in}}%
\pgfpathlineto{\pgfqpoint{1.989957in}{1.985638in}}%
\pgfpathlineto{\pgfqpoint{1.990365in}{2.035148in}}%
\pgfpathlineto{\pgfqpoint{1.990977in}{1.969135in}}%
\pgfpathlineto{\pgfqpoint{1.992202in}{2.150670in}}%
\pgfpathlineto{\pgfqpoint{1.992610in}{2.076406in}}%
\pgfpathlineto{\pgfqpoint{1.992815in}{2.002141in}}%
\pgfpathlineto{\pgfqpoint{1.993427in}{2.142418in}}%
\pgfpathlineto{\pgfqpoint{1.994652in}{2.076406in}}%
\pgfpathlineto{\pgfqpoint{1.995060in}{2.101160in}}%
\pgfpathlineto{\pgfqpoint{1.996081in}{2.142418in}}%
\pgfpathlineto{\pgfqpoint{1.997510in}{2.002141in}}%
\pgfpathlineto{\pgfqpoint{1.998122in}{2.051651in}}%
\pgfpathlineto{\pgfqpoint{1.998326in}{1.969135in}}%
\pgfpathlineto{\pgfqpoint{1.998531in}{1.886619in}}%
\pgfpathlineto{\pgfqpoint{1.999143in}{2.076406in}}%
\pgfpathlineto{\pgfqpoint{1.999551in}{2.010393in}}%
\pgfpathlineto{\pgfqpoint{1.999960in}{2.150670in}}%
\pgfpathlineto{\pgfqpoint{2.000980in}{1.985638in}}%
\pgfpathlineto{\pgfqpoint{2.001389in}{1.993890in}}%
\pgfpathlineto{\pgfqpoint{2.002818in}{1.845361in}}%
\pgfpathlineto{\pgfqpoint{2.003838in}{1.927877in}}%
\pgfpathlineto{\pgfqpoint{2.004042in}{1.870115in}}%
\pgfpathlineto{\pgfqpoint{2.004247in}{1.639071in}}%
\pgfpathlineto{\pgfqpoint{2.005063in}{1.960883in}}%
\pgfpathlineto{\pgfqpoint{2.005267in}{1.919625in}}%
\pgfpathlineto{\pgfqpoint{2.005880in}{1.993890in}}%
\pgfpathlineto{\pgfqpoint{2.006084in}{1.960883in}}%
\pgfpathlineto{\pgfqpoint{2.006288in}{1.952632in}}%
\pgfpathlineto{\pgfqpoint{2.006696in}{1.977386in}}%
\pgfpathlineto{\pgfqpoint{2.007309in}{2.051651in}}%
\pgfpathlineto{\pgfqpoint{2.007513in}{2.002141in}}%
\pgfpathlineto{\pgfqpoint{2.008738in}{1.762845in}}%
\pgfpathlineto{\pgfqpoint{2.009758in}{1.993890in}}%
\pgfpathlineto{\pgfqpoint{2.009962in}{1.944380in}}%
\pgfpathlineto{\pgfqpoint{2.010371in}{2.002141in}}%
\pgfpathlineto{\pgfqpoint{2.010779in}{1.911373in}}%
\pgfpathlineto{\pgfqpoint{2.010983in}{1.944380in}}%
\pgfpathlineto{\pgfqpoint{2.011391in}{1.886619in}}%
\pgfpathlineto{\pgfqpoint{2.012004in}{1.927877in}}%
\pgfpathlineto{\pgfqpoint{2.012616in}{1.911373in}}%
\pgfpathlineto{\pgfqpoint{2.013637in}{2.018644in}}%
\pgfpathlineto{\pgfqpoint{2.013841in}{1.911373in}}%
\pgfpathlineto{\pgfqpoint{2.014658in}{1.985638in}}%
\pgfpathlineto{\pgfqpoint{2.015883in}{1.919625in}}%
\pgfpathlineto{\pgfqpoint{2.015270in}{2.002141in}}%
\pgfpathlineto{\pgfqpoint{2.016087in}{1.944380in}}%
\pgfpathlineto{\pgfqpoint{2.016291in}{1.952632in}}%
\pgfpathlineto{\pgfqpoint{2.016495in}{1.911373in}}%
\pgfpathlineto{\pgfqpoint{2.016903in}{1.985638in}}%
\pgfpathlineto{\pgfqpoint{2.017312in}{1.952632in}}%
\pgfpathlineto{\pgfqpoint{2.018332in}{1.903122in}}%
\pgfpathlineto{\pgfqpoint{2.019353in}{1.977386in}}%
\pgfpathlineto{\pgfqpoint{2.019557in}{1.952632in}}%
\pgfpathlineto{\pgfqpoint{2.020374in}{1.894870in}}%
\pgfpathlineto{\pgfqpoint{2.019965in}{2.002141in}}%
\pgfpathlineto{\pgfqpoint{2.020578in}{1.936128in}}%
\pgfpathlineto{\pgfqpoint{2.021394in}{2.051651in}}%
\pgfpathlineto{\pgfqpoint{2.021599in}{1.969135in}}%
\pgfpathlineto{\pgfqpoint{2.022415in}{1.870115in}}%
\pgfpathlineto{\pgfqpoint{2.022823in}{1.919625in}}%
\pgfpathlineto{\pgfqpoint{2.023640in}{2.043399in}}%
\pgfpathlineto{\pgfqpoint{2.024252in}{2.002141in}}%
\pgfpathlineto{\pgfqpoint{2.025681in}{1.804103in}}%
\pgfpathlineto{\pgfqpoint{2.026702in}{1.886619in}}%
\pgfpathlineto{\pgfqpoint{2.026906in}{1.837109in}}%
\pgfpathlineto{\pgfqpoint{2.027519in}{1.746341in}}%
\pgfpathlineto{\pgfqpoint{2.028335in}{1.762845in}}%
\pgfpathlineto{\pgfqpoint{2.030581in}{1.993890in}}%
\pgfpathlineto{\pgfqpoint{2.030785in}{1.985638in}}%
\pgfpathlineto{\pgfqpoint{2.031602in}{2.002141in}}%
\pgfpathlineto{\pgfqpoint{2.031193in}{1.977386in}}%
\pgfpathlineto{\pgfqpoint{2.031806in}{1.985638in}}%
\pgfpathlineto{\pgfqpoint{2.032010in}{1.985638in}}%
\pgfpathlineto{\pgfqpoint{2.032214in}{1.977386in}}%
\pgfpathlineto{\pgfqpoint{2.032418in}{2.010393in}}%
\pgfpathlineto{\pgfqpoint{2.032622in}{2.010393in}}%
\pgfpathlineto{\pgfqpoint{2.033643in}{1.911373in}}%
\pgfpathlineto{\pgfqpoint{2.033847in}{1.919625in}}%
\pgfpathlineto{\pgfqpoint{2.034664in}{1.993890in}}%
\pgfpathlineto{\pgfqpoint{2.035072in}{1.977386in}}%
\pgfpathlineto{\pgfqpoint{2.036297in}{1.894870in}}%
\pgfpathlineto{\pgfqpoint{2.036501in}{1.919625in}}%
\pgfpathlineto{\pgfqpoint{2.036705in}{1.993890in}}%
\pgfpathlineto{\pgfqpoint{2.037318in}{1.837109in}}%
\pgfpathlineto{\pgfqpoint{2.037522in}{1.960883in}}%
\pgfpathlineto{\pgfqpoint{2.037930in}{1.639071in}}%
\pgfpathlineto{\pgfqpoint{2.038747in}{1.845361in}}%
\pgfpathlineto{\pgfqpoint{2.039767in}{1.688580in}}%
\pgfpathlineto{\pgfqpoint{2.040176in}{1.696832in}}%
\pgfpathlineto{\pgfqpoint{2.040584in}{1.762845in}}%
\pgfpathlineto{\pgfqpoint{2.041400in}{1.754593in}}%
\pgfpathlineto{\pgfqpoint{2.041605in}{1.713335in}}%
\pgfpathlineto{\pgfqpoint{2.042013in}{1.820606in}}%
\pgfpathlineto{\pgfqpoint{2.042421in}{1.762845in}}%
\pgfpathlineto{\pgfqpoint{2.042829in}{1.738090in}}%
\pgfpathlineto{\pgfqpoint{2.043238in}{1.820606in}}%
\pgfpathlineto{\pgfqpoint{2.044054in}{1.696832in}}%
\pgfpathlineto{\pgfqpoint{2.044463in}{1.754593in}}%
\pgfpathlineto{\pgfqpoint{2.045279in}{1.729838in}}%
\pgfpathlineto{\pgfqpoint{2.045483in}{1.738090in}}%
\pgfpathlineto{\pgfqpoint{2.046504in}{1.795851in}}%
\pgfpathlineto{\pgfqpoint{2.046708in}{1.614316in}}%
\pgfpathlineto{\pgfqpoint{2.047525in}{1.861864in}}%
\pgfpathlineto{\pgfqpoint{2.047729in}{1.853612in}}%
\pgfpathlineto{\pgfqpoint{2.047933in}{1.870115in}}%
\pgfpathlineto{\pgfqpoint{2.048341in}{1.944380in}}%
\pgfpathlineto{\pgfqpoint{2.048750in}{1.828857in}}%
\pgfpathlineto{\pgfqpoint{2.048954in}{1.911373in}}%
\pgfpathlineto{\pgfqpoint{2.050383in}{1.820606in}}%
\pgfpathlineto{\pgfqpoint{2.050791in}{1.919625in}}%
\pgfpathlineto{\pgfqpoint{2.050995in}{1.795851in}}%
\pgfpathlineto{\pgfqpoint{2.051199in}{1.861864in}}%
\pgfpathlineto{\pgfqpoint{2.052832in}{1.581309in}}%
\pgfpathlineto{\pgfqpoint{2.053241in}{1.787599in}}%
\pgfpathlineto{\pgfqpoint{2.054057in}{1.746341in}}%
\pgfpathlineto{\pgfqpoint{2.054874in}{1.853612in}}%
\pgfpathlineto{\pgfqpoint{2.055486in}{1.845361in}}%
\pgfpathlineto{\pgfqpoint{2.056507in}{1.738090in}}%
\pgfpathlineto{\pgfqpoint{2.056711in}{1.754593in}}%
\pgfpathlineto{\pgfqpoint{2.056915in}{1.746341in}}%
\pgfpathlineto{\pgfqpoint{2.057324in}{1.828857in}}%
\pgfpathlineto{\pgfqpoint{2.057936in}{1.729838in}}%
\pgfpathlineto{\pgfqpoint{2.058344in}{1.837109in}}%
\pgfpathlineto{\pgfqpoint{2.058753in}{1.828857in}}%
\pgfpathlineto{\pgfqpoint{2.058957in}{1.647322in}}%
\pgfpathlineto{\pgfqpoint{2.059161in}{1.870115in}}%
\pgfpathlineto{\pgfqpoint{2.059773in}{1.771096in}}%
\pgfpathlineto{\pgfqpoint{2.060182in}{1.894870in}}%
\pgfpathlineto{\pgfqpoint{2.060794in}{1.787599in}}%
\pgfpathlineto{\pgfqpoint{2.060998in}{1.771096in}}%
\pgfpathlineto{\pgfqpoint{2.061202in}{1.779348in}}%
\pgfpathlineto{\pgfqpoint{2.061406in}{1.837109in}}%
\pgfpathlineto{\pgfqpoint{2.061611in}{1.771096in}}%
\pgfpathlineto{\pgfqpoint{2.062223in}{1.795851in}}%
\pgfpathlineto{\pgfqpoint{2.062427in}{1.771096in}}%
\pgfpathlineto{\pgfqpoint{2.062631in}{1.861864in}}%
\pgfpathlineto{\pgfqpoint{2.062835in}{1.903122in}}%
\pgfpathlineto{\pgfqpoint{2.063244in}{1.845361in}}%
\pgfpathlineto{\pgfqpoint{2.063448in}{1.853612in}}%
\pgfpathlineto{\pgfqpoint{2.063652in}{1.622567in}}%
\pgfpathlineto{\pgfqpoint{2.064264in}{1.886619in}}%
\pgfpathlineto{\pgfqpoint{2.064469in}{1.878367in}}%
\pgfpathlineto{\pgfqpoint{2.064673in}{1.903122in}}%
\pgfpathlineto{\pgfqpoint{2.064877in}{1.828857in}}%
\pgfpathlineto{\pgfqpoint{2.065285in}{1.861864in}}%
\pgfpathlineto{\pgfqpoint{2.066510in}{1.696832in}}%
\pgfpathlineto{\pgfqpoint{2.066918in}{1.787599in}}%
\pgfpathlineto{\pgfqpoint{2.067122in}{1.886619in}}%
\pgfpathlineto{\pgfqpoint{2.067327in}{1.672077in}}%
\pgfpathlineto{\pgfqpoint{2.067939in}{1.746341in}}%
\pgfpathlineto{\pgfqpoint{2.068551in}{1.713335in}}%
\pgfpathlineto{\pgfqpoint{2.068347in}{1.762845in}}%
\pgfpathlineto{\pgfqpoint{2.068960in}{1.746341in}}%
\pgfpathlineto{\pgfqpoint{2.069368in}{1.779348in}}%
\pgfpathlineto{\pgfqpoint{2.069980in}{1.762845in}}%
\pgfpathlineto{\pgfqpoint{2.070185in}{1.705083in}}%
\pgfpathlineto{\pgfqpoint{2.070389in}{1.787599in}}%
\pgfpathlineto{\pgfqpoint{2.071205in}{1.721587in}}%
\pgfpathlineto{\pgfqpoint{2.072022in}{1.837109in}}%
\pgfpathlineto{\pgfqpoint{2.073247in}{1.606064in}}%
\pgfpathlineto{\pgfqpoint{2.074676in}{1.762845in}}%
\pgfpathlineto{\pgfqpoint{2.075696in}{1.663825in}}%
\pgfpathlineto{\pgfqpoint{2.075901in}{1.696832in}}%
\pgfpathlineto{\pgfqpoint{2.076309in}{1.729838in}}%
\pgfpathlineto{\pgfqpoint{2.076717in}{1.672077in}}%
\pgfpathlineto{\pgfqpoint{2.076921in}{1.647322in}}%
\pgfpathlineto{\pgfqpoint{2.077738in}{1.672077in}}%
\pgfpathlineto{\pgfqpoint{2.079167in}{1.795851in}}%
\pgfpathlineto{\pgfqpoint{2.079371in}{1.779348in}}%
\pgfpathlineto{\pgfqpoint{2.079575in}{1.746341in}}%
\pgfpathlineto{\pgfqpoint{2.079779in}{1.804103in}}%
\pgfpathlineto{\pgfqpoint{2.080188in}{1.762845in}}%
\pgfpathlineto{\pgfqpoint{2.081004in}{1.911373in}}%
\pgfpathlineto{\pgfqpoint{2.081208in}{1.870115in}}%
\pgfpathlineto{\pgfqpoint{2.082433in}{1.729838in}}%
\pgfpathlineto{\pgfqpoint{2.083454in}{1.812354in}}%
\pgfpathlineto{\pgfqpoint{2.083658in}{1.771096in}}%
\pgfpathlineto{\pgfqpoint{2.084270in}{1.828857in}}%
\pgfpathlineto{\pgfqpoint{2.084679in}{1.771096in}}%
\pgfpathlineto{\pgfqpoint{2.085087in}{1.672077in}}%
\pgfpathlineto{\pgfqpoint{2.085495in}{1.779348in}}%
\pgfpathlineto{\pgfqpoint{2.085699in}{1.795851in}}%
\pgfpathlineto{\pgfqpoint{2.085904in}{1.771096in}}%
\pgfpathlineto{\pgfqpoint{2.086516in}{1.474038in}}%
\pgfpathlineto{\pgfqpoint{2.086924in}{1.746341in}}%
\pgfpathlineto{\pgfqpoint{2.087128in}{1.762845in}}%
\pgfpathlineto{\pgfqpoint{2.088353in}{1.597813in}}%
\pgfpathlineto{\pgfqpoint{2.089374in}{1.672077in}}%
\pgfpathlineto{\pgfqpoint{2.089986in}{1.721587in}}%
\pgfpathlineto{\pgfqpoint{2.090191in}{1.663825in}}%
\pgfpathlineto{\pgfqpoint{2.091007in}{1.531800in}}%
\pgfpathlineto{\pgfqpoint{2.091211in}{1.663825in}}%
\pgfpathlineto{\pgfqpoint{2.091620in}{1.639071in}}%
\pgfpathlineto{\pgfqpoint{2.091824in}{1.738090in}}%
\pgfpathlineto{\pgfqpoint{2.092640in}{1.630819in}}%
\pgfpathlineto{\pgfqpoint{2.093457in}{1.705083in}}%
\pgfpathlineto{\pgfqpoint{2.093661in}{1.655574in}}%
\pgfpathlineto{\pgfqpoint{2.094273in}{1.540051in}}%
\pgfpathlineto{\pgfqpoint{2.095090in}{1.573058in}}%
\pgfpathlineto{\pgfqpoint{2.095906in}{1.655574in}}%
\pgfpathlineto{\pgfqpoint{2.096315in}{1.606064in}}%
\pgfpathlineto{\pgfqpoint{2.096723in}{1.614316in}}%
\pgfpathlineto{\pgfqpoint{2.097335in}{1.564806in}}%
\pgfpathlineto{\pgfqpoint{2.097540in}{1.589561in}}%
\pgfpathlineto{\pgfqpoint{2.097744in}{1.515296in}}%
\pgfpathlineto{\pgfqpoint{2.098356in}{1.573058in}}%
\pgfpathlineto{\pgfqpoint{2.098560in}{1.573058in}}%
\pgfpathlineto{\pgfqpoint{2.098764in}{1.564806in}}%
\pgfpathlineto{\pgfqpoint{2.098969in}{1.597813in}}%
\pgfpathlineto{\pgfqpoint{2.099173in}{1.581309in}}%
\pgfpathlineto{\pgfqpoint{2.100806in}{1.705083in}}%
\pgfpathlineto{\pgfqpoint{2.101010in}{1.622567in}}%
\pgfpathlineto{\pgfqpoint{2.101827in}{1.738090in}}%
\pgfpathlineto{\pgfqpoint{2.103256in}{1.449284in}}%
\pgfpathlineto{\pgfqpoint{2.103460in}{1.564806in}}%
\pgfpathlineto{\pgfqpoint{2.104685in}{1.680329in}}%
\pgfpathlineto{\pgfqpoint{2.105909in}{1.597813in}}%
\pgfpathlineto{\pgfqpoint{2.106114in}{1.647322in}}%
\pgfpathlineto{\pgfqpoint{2.106318in}{1.498793in}}%
\pgfpathlineto{\pgfqpoint{2.107134in}{1.630819in}}%
\pgfpathlineto{\pgfqpoint{2.107747in}{1.375019in}}%
\pgfpathlineto{\pgfqpoint{2.107543in}{1.655574in}}%
\pgfpathlineto{\pgfqpoint{2.108155in}{1.589561in}}%
\pgfpathlineto{\pgfqpoint{2.108359in}{1.614316in}}%
\pgfpathlineto{\pgfqpoint{2.108767in}{1.556554in}}%
\pgfpathlineto{\pgfqpoint{2.109176in}{1.597813in}}%
\pgfpathlineto{\pgfqpoint{2.109788in}{1.556554in}}%
\pgfpathlineto{\pgfqpoint{2.109992in}{1.639071in}}%
\pgfpathlineto{\pgfqpoint{2.110196in}{1.606064in}}%
\pgfpathlineto{\pgfqpoint{2.111013in}{1.556554in}}%
\pgfpathlineto{\pgfqpoint{2.110605in}{1.622567in}}%
\pgfpathlineto{\pgfqpoint{2.111217in}{1.581309in}}%
\pgfpathlineto{\pgfqpoint{2.111421in}{1.639071in}}%
\pgfpathlineto{\pgfqpoint{2.111830in}{1.556554in}}%
\pgfpathlineto{\pgfqpoint{2.112238in}{1.606064in}}%
\pgfpathlineto{\pgfqpoint{2.112442in}{1.597813in}}%
\pgfpathlineto{\pgfqpoint{2.112850in}{1.614316in}}%
\pgfpathlineto{\pgfqpoint{2.113463in}{1.639071in}}%
\pgfpathlineto{\pgfqpoint{2.113667in}{1.589561in}}%
\pgfpathlineto{\pgfqpoint{2.114075in}{1.630819in}}%
\pgfpathlineto{\pgfqpoint{2.114279in}{1.663825in}}%
\pgfpathlineto{\pgfqpoint{2.114688in}{1.581309in}}%
\pgfpathlineto{\pgfqpoint{2.114892in}{1.589561in}}%
\pgfpathlineto{\pgfqpoint{2.115912in}{1.515296in}}%
\pgfpathlineto{\pgfqpoint{2.115504in}{1.614316in}}%
\pgfpathlineto{\pgfqpoint{2.116117in}{1.548303in}}%
\pgfpathlineto{\pgfqpoint{2.116525in}{1.515296in}}%
\pgfpathlineto{\pgfqpoint{2.116729in}{1.523548in}}%
\pgfpathlineto{\pgfqpoint{2.117750in}{1.515296in}}%
\pgfpathlineto{\pgfqpoint{2.117954in}{1.589561in}}%
\pgfpathlineto{\pgfqpoint{2.118566in}{1.490542in}}%
\pgfpathlineto{\pgfqpoint{2.118975in}{1.581309in}}%
\pgfpathlineto{\pgfqpoint{2.119383in}{1.573058in}}%
\pgfpathlineto{\pgfqpoint{2.119587in}{1.622567in}}%
\pgfpathlineto{\pgfqpoint{2.119995in}{1.515296in}}%
\pgfpathlineto{\pgfqpoint{2.120608in}{1.432780in}}%
\pgfpathlineto{\pgfqpoint{2.121016in}{1.490542in}}%
\pgfpathlineto{\pgfqpoint{2.121833in}{1.548303in}}%
\pgfpathlineto{\pgfqpoint{2.121424in}{1.457535in}}%
\pgfpathlineto{\pgfqpoint{2.122241in}{1.507045in}}%
\pgfpathlineto{\pgfqpoint{2.122445in}{1.498793in}}%
\pgfpathlineto{\pgfqpoint{2.122853in}{1.556554in}}%
\pgfpathlineto{\pgfqpoint{2.123262in}{1.449284in}}%
\pgfpathlineto{\pgfqpoint{2.123466in}{1.490542in}}%
\pgfpathlineto{\pgfqpoint{2.123670in}{1.474038in}}%
\pgfpathlineto{\pgfqpoint{2.123874in}{1.507045in}}%
\pgfpathlineto{\pgfqpoint{2.125303in}{1.655574in}}%
\pgfpathlineto{\pgfqpoint{2.126732in}{1.498793in}}%
\pgfpathlineto{\pgfqpoint{2.128161in}{1.663825in}}%
\pgfpathlineto{\pgfqpoint{2.128773in}{1.655574in}}%
\pgfpathlineto{\pgfqpoint{2.129590in}{1.672077in}}%
\pgfpathlineto{\pgfqpoint{2.129794in}{1.630819in}}%
\pgfpathlineto{\pgfqpoint{2.130407in}{1.738090in}}%
\pgfpathlineto{\pgfqpoint{2.130815in}{1.655574in}}%
\pgfpathlineto{\pgfqpoint{2.131019in}{1.647322in}}%
\pgfpathlineto{\pgfqpoint{2.131427in}{1.721587in}}%
\pgfpathlineto{\pgfqpoint{2.131631in}{1.639071in}}%
\pgfpathlineto{\pgfqpoint{2.132040in}{1.696832in}}%
\pgfpathlineto{\pgfqpoint{2.132448in}{1.531800in}}%
\pgfpathlineto{\pgfqpoint{2.133469in}{1.556554in}}%
\pgfpathlineto{\pgfqpoint{2.133673in}{1.630819in}}%
\pgfpathlineto{\pgfqpoint{2.134489in}{1.556554in}}%
\pgfpathlineto{\pgfqpoint{2.135102in}{1.581309in}}%
\pgfpathlineto{\pgfqpoint{2.134898in}{1.548303in}}%
\pgfpathlineto{\pgfqpoint{2.135306in}{1.556554in}}%
\pgfpathlineto{\pgfqpoint{2.135510in}{1.515296in}}%
\pgfpathlineto{\pgfqpoint{2.135918in}{1.589561in}}%
\pgfpathlineto{\pgfqpoint{2.136123in}{1.622567in}}%
\pgfpathlineto{\pgfqpoint{2.136327in}{1.507045in}}%
\pgfpathlineto{\pgfqpoint{2.136531in}{1.523548in}}%
\pgfpathlineto{\pgfqpoint{2.136735in}{1.523548in}}%
\pgfpathlineto{\pgfqpoint{2.136939in}{1.515296in}}%
\pgfpathlineto{\pgfqpoint{2.137143in}{1.531800in}}%
\pgfpathlineto{\pgfqpoint{2.137347in}{1.531800in}}%
\pgfpathlineto{\pgfqpoint{2.137756in}{1.564806in}}%
\pgfpathlineto{\pgfqpoint{2.137960in}{1.531800in}}%
\pgfpathlineto{\pgfqpoint{2.138572in}{1.474038in}}%
\pgfpathlineto{\pgfqpoint{2.138776in}{1.548303in}}%
\pgfpathlineto{\pgfqpoint{2.138981in}{1.622567in}}%
\pgfpathlineto{\pgfqpoint{2.139797in}{1.531800in}}%
\pgfpathlineto{\pgfqpoint{2.140614in}{1.548303in}}%
\pgfpathlineto{\pgfqpoint{2.141226in}{1.424529in}}%
\pgfpathlineto{\pgfqpoint{2.142451in}{1.507045in}}%
\pgfpathlineto{\pgfqpoint{2.142655in}{1.441032in}}%
\pgfpathlineto{\pgfqpoint{2.143063in}{1.548303in}}%
\pgfpathlineto{\pgfqpoint{2.143472in}{1.498793in}}%
\pgfpathlineto{\pgfqpoint{2.143676in}{1.548303in}}%
\pgfpathlineto{\pgfqpoint{2.143880in}{1.325510in}}%
\pgfpathlineto{\pgfqpoint{2.144084in}{1.573058in}}%
\pgfpathlineto{\pgfqpoint{2.144697in}{1.490542in}}%
\pgfpathlineto{\pgfqpoint{2.144901in}{1.482290in}}%
\pgfpathlineto{\pgfqpoint{2.145105in}{1.531800in}}%
\pgfpathlineto{\pgfqpoint{2.145513in}{1.465787in}}%
\pgfpathlineto{\pgfqpoint{2.145717in}{1.490542in}}%
\pgfpathlineto{\pgfqpoint{2.146126in}{1.358516in}}%
\pgfpathlineto{\pgfqpoint{2.146738in}{1.391522in}}%
\pgfpathlineto{\pgfqpoint{2.147759in}{1.523548in}}%
\pgfpathlineto{\pgfqpoint{2.147146in}{1.366768in}}%
\pgfpathlineto{\pgfqpoint{2.147963in}{1.515296in}}%
\pgfpathlineto{\pgfqpoint{2.148371in}{1.168729in}}%
\pgfpathlineto{\pgfqpoint{2.148984in}{1.218239in}}%
\pgfpathlineto{\pgfqpoint{2.150004in}{1.482290in}}%
\pgfpathlineto{\pgfqpoint{2.150208in}{1.416277in}}%
\pgfpathlineto{\pgfqpoint{2.151229in}{1.515296in}}%
\pgfpathlineto{\pgfqpoint{2.151637in}{1.490542in}}%
\pgfpathlineto{\pgfqpoint{2.152454in}{1.457535in}}%
\pgfpathlineto{\pgfqpoint{2.152658in}{1.515296in}}%
\pgfpathlineto{\pgfqpoint{2.153475in}{1.498793in}}%
\pgfpathlineto{\pgfqpoint{2.154291in}{1.408026in}}%
\pgfpathlineto{\pgfqpoint{2.154495in}{1.474038in}}%
\pgfpathlineto{\pgfqpoint{2.154904in}{1.573058in}}%
\pgfpathlineto{\pgfqpoint{2.155516in}{1.474038in}}%
\pgfpathlineto{\pgfqpoint{2.155720in}{1.242994in}}%
\pgfpathlineto{\pgfqpoint{2.156333in}{1.573058in}}%
\pgfpathlineto{\pgfqpoint{2.156537in}{1.531800in}}%
\pgfpathlineto{\pgfqpoint{2.156741in}{1.531800in}}%
\pgfpathlineto{\pgfqpoint{2.156945in}{1.515296in}}%
\pgfpathlineto{\pgfqpoint{2.157149in}{1.564806in}}%
\pgfpathlineto{\pgfqpoint{2.157353in}{1.540051in}}%
\pgfpathlineto{\pgfqpoint{2.158374in}{1.655574in}}%
\pgfpathlineto{\pgfqpoint{2.158782in}{1.614316in}}%
\pgfpathlineto{\pgfqpoint{2.159803in}{1.531800in}}%
\pgfpathlineto{\pgfqpoint{2.160007in}{1.556554in}}%
\pgfpathlineto{\pgfqpoint{2.160211in}{1.564806in}}%
\pgfpathlineto{\pgfqpoint{2.161232in}{1.779348in}}%
\pgfpathlineto{\pgfqpoint{2.161436in}{1.705083in}}%
\pgfpathlineto{\pgfqpoint{2.161640in}{1.449284in}}%
\pgfpathlineto{\pgfqpoint{2.162457in}{1.597813in}}%
\pgfpathlineto{\pgfqpoint{2.163886in}{1.482290in}}%
\pgfpathlineto{\pgfqpoint{2.164090in}{1.490542in}}%
\pgfpathlineto{\pgfqpoint{2.164294in}{1.556554in}}%
\pgfpathlineto{\pgfqpoint{2.164498in}{1.474038in}}%
\pgfpathlineto{\pgfqpoint{2.165111in}{1.531800in}}%
\pgfpathlineto{\pgfqpoint{2.165927in}{1.432780in}}%
\pgfpathlineto{\pgfqpoint{2.166336in}{1.474038in}}%
\pgfpathlineto{\pgfqpoint{2.166744in}{1.515296in}}%
\pgfpathlineto{\pgfqpoint{2.166948in}{1.465787in}}%
\pgfpathlineto{\pgfqpoint{2.167356in}{1.573058in}}%
\pgfpathlineto{\pgfqpoint{2.167969in}{1.482290in}}%
\pgfpathlineto{\pgfqpoint{2.168785in}{1.383271in}}%
\pgfpathlineto{\pgfqpoint{2.169194in}{1.416277in}}%
\pgfpathlineto{\pgfqpoint{2.169806in}{1.465787in}}%
\pgfpathlineto{\pgfqpoint{2.171031in}{1.325510in}}%
\pgfpathlineto{\pgfqpoint{2.172052in}{1.399774in}}%
\pgfpathlineto{\pgfqpoint{2.171439in}{1.292503in}}%
\pgfpathlineto{\pgfqpoint{2.172460in}{1.391522in}}%
\pgfpathlineto{\pgfqpoint{2.173072in}{1.176981in}}%
\pgfpathlineto{\pgfqpoint{2.173277in}{1.399774in}}%
\pgfpathlineto{\pgfqpoint{2.174297in}{1.548303in}}%
\pgfpathlineto{\pgfqpoint{2.174706in}{1.507045in}}%
\pgfpathlineto{\pgfqpoint{2.174910in}{1.292503in}}%
\pgfpathlineto{\pgfqpoint{2.175522in}{1.573058in}}%
\pgfpathlineto{\pgfqpoint{2.175726in}{1.548303in}}%
\pgfpathlineto{\pgfqpoint{2.176747in}{1.482290in}}%
\pgfpathlineto{\pgfqpoint{2.176951in}{1.490542in}}%
\pgfpathlineto{\pgfqpoint{2.177155in}{1.523548in}}%
\pgfpathlineto{\pgfqpoint{2.177768in}{1.597813in}}%
\pgfpathlineto{\pgfqpoint{2.178176in}{1.333761in}}%
\pgfpathlineto{\pgfqpoint{2.178584in}{1.564806in}}%
\pgfpathlineto{\pgfqpoint{2.179401in}{1.482290in}}%
\pgfpathlineto{\pgfqpoint{2.180217in}{1.424529in}}%
\pgfpathlineto{\pgfqpoint{2.180421in}{1.482290in}}%
\pgfpathlineto{\pgfqpoint{2.181238in}{1.408026in}}%
\pgfpathlineto{\pgfqpoint{2.181850in}{1.135723in}}%
\pgfpathlineto{\pgfqpoint{2.182055in}{1.432780in}}%
\pgfpathlineto{\pgfqpoint{2.182871in}{1.498793in}}%
\pgfpathlineto{\pgfqpoint{2.183075in}{1.482290in}}%
\pgfpathlineto{\pgfqpoint{2.185321in}{1.143974in}}%
\pgfpathlineto{\pgfqpoint{2.185525in}{0.888175in}}%
\pgfpathlineto{\pgfqpoint{2.185933in}{1.160477in}}%
\pgfpathlineto{\pgfqpoint{2.186342in}{1.102716in}}%
\pgfpathlineto{\pgfqpoint{2.187158in}{1.176981in}}%
\pgfpathlineto{\pgfqpoint{2.186750in}{1.094465in}}%
\pgfpathlineto{\pgfqpoint{2.187362in}{1.160477in}}%
\pgfpathlineto{\pgfqpoint{2.188179in}{0.871671in}}%
\pgfpathlineto{\pgfqpoint{2.188383in}{1.044955in}}%
\pgfpathlineto{\pgfqpoint{2.188587in}{1.168729in}}%
\pgfpathlineto{\pgfqpoint{2.189404in}{1.044955in}}%
\pgfpathlineto{\pgfqpoint{2.189812in}{1.102716in}}%
\pgfpathlineto{\pgfqpoint{2.190220in}{1.020200in}}%
\pgfpathlineto{\pgfqpoint{2.190424in}{1.061458in}}%
\pgfpathlineto{\pgfqpoint{2.190629in}{0.756149in}}%
\pgfpathlineto{\pgfqpoint{2.191037in}{1.069710in}}%
\pgfpathlineto{\pgfqpoint{2.191445in}{0.838665in}}%
\pgfpathlineto{\pgfqpoint{2.192058in}{1.028452in}}%
\pgfpathlineto{\pgfqpoint{2.192670in}{0.945936in}}%
\pgfpathlineto{\pgfqpoint{2.192874in}{0.690136in}}%
\pgfpathlineto{\pgfqpoint{2.193282in}{1.011949in}}%
\pgfpathlineto{\pgfqpoint{2.193691in}{0.921181in}}%
\pgfpathlineto{\pgfqpoint{2.193895in}{0.855168in}}%
\pgfpathlineto{\pgfqpoint{2.194507in}{0.970691in}}%
\pgfpathlineto{\pgfqpoint{2.194711in}{1.061458in}}%
\pgfpathlineto{\pgfqpoint{2.194916in}{0.937684in}}%
\pgfpathlineto{\pgfqpoint{2.195120in}{0.723142in}}%
\pgfpathlineto{\pgfqpoint{2.195936in}{0.888175in}}%
\pgfpathlineto{\pgfqpoint{2.197365in}{0.995445in}}%
\pgfpathlineto{\pgfqpoint{2.196345in}{0.855168in}}%
\pgfpathlineto{\pgfqpoint{2.197569in}{0.954187in}}%
\pgfpathlineto{\pgfqpoint{2.197978in}{0.690136in}}%
\pgfpathlineto{\pgfqpoint{2.198794in}{0.863420in}}%
\pgfpathlineto{\pgfqpoint{2.199815in}{1.086213in}}%
\pgfpathlineto{\pgfqpoint{2.200223in}{1.011949in}}%
\pgfpathlineto{\pgfqpoint{2.200427in}{0.912929in}}%
\pgfpathlineto{\pgfqpoint{2.201244in}{1.003697in}}%
\pgfpathlineto{\pgfqpoint{2.202673in}{1.152226in}}%
\pgfpathlineto{\pgfqpoint{2.203081in}{1.185232in}}%
\pgfpathlineto{\pgfqpoint{2.203490in}{1.110968in}}%
\pgfpathlineto{\pgfqpoint{2.203694in}{1.168729in}}%
\pgfpathlineto{\pgfqpoint{2.204306in}{1.069710in}}%
\pgfpathlineto{\pgfqpoint{2.204714in}{1.003697in}}%
\pgfpathlineto{\pgfqpoint{2.205123in}{1.069710in}}%
\pgfpathlineto{\pgfqpoint{2.205735in}{1.152226in}}%
\pgfpathlineto{\pgfqpoint{2.205939in}{1.044955in}}%
\pgfpathlineto{\pgfqpoint{2.206348in}{1.061458in}}%
\pgfpathlineto{\pgfqpoint{2.206756in}{0.912929in}}%
\pgfpathlineto{\pgfqpoint{2.206960in}{0.698388in}}%
\pgfpathlineto{\pgfqpoint{2.207368in}{1.028452in}}%
\pgfpathlineto{\pgfqpoint{2.207777in}{0.929433in}}%
\pgfpathlineto{\pgfqpoint{2.208389in}{0.970691in}}%
\pgfpathlineto{\pgfqpoint{2.208593in}{0.954187in}}%
\pgfpathlineto{\pgfqpoint{2.209001in}{0.805659in}}%
\pgfpathlineto{\pgfqpoint{2.209614in}{0.896426in}}%
\pgfpathlineto{\pgfqpoint{2.211043in}{1.086213in}}%
\pgfpathlineto{\pgfqpoint{2.211247in}{1.053207in}}%
\pgfpathlineto{\pgfqpoint{2.211655in}{1.119219in}}%
\pgfpathlineto{\pgfqpoint{2.211859in}{1.077961in}}%
\pgfpathlineto{\pgfqpoint{2.212268in}{1.135723in}}%
\pgfpathlineto{\pgfqpoint{2.212676in}{1.011949in}}%
\pgfpathlineto{\pgfqpoint{2.213084in}{1.044955in}}%
\pgfpathlineto{\pgfqpoint{2.213493in}{1.020200in}}%
\pgfpathlineto{\pgfqpoint{2.213697in}{0.987194in}}%
\pgfpathlineto{\pgfqpoint{2.214309in}{1.061458in}}%
\pgfpathlineto{\pgfqpoint{2.214513in}{1.028452in}}%
\pgfpathlineto{\pgfqpoint{2.215942in}{1.201736in}}%
\pgfpathlineto{\pgfqpoint{2.216146in}{1.284252in}}%
\pgfpathlineto{\pgfqpoint{2.216759in}{1.160477in}}%
\pgfpathlineto{\pgfqpoint{2.216963in}{1.234742in}}%
\pgfpathlineto{\pgfqpoint{2.217167in}{1.185232in}}%
\pgfpathlineto{\pgfqpoint{2.217984in}{1.226490in}}%
\pgfpathlineto{\pgfqpoint{2.218188in}{1.242994in}}%
\pgfpathlineto{\pgfqpoint{2.218596in}{1.193484in}}%
\pgfpathlineto{\pgfqpoint{2.218800in}{1.193484in}}%
\pgfpathlineto{\pgfqpoint{2.219004in}{1.185232in}}%
\pgfpathlineto{\pgfqpoint{2.219413in}{1.309006in}}%
\pgfpathlineto{\pgfqpoint{2.219821in}{1.251245in}}%
\pgfpathlineto{\pgfqpoint{2.220025in}{1.152226in}}%
\pgfpathlineto{\pgfqpoint{2.220229in}{1.342013in}}%
\pgfpathlineto{\pgfqpoint{2.220842in}{1.292503in}}%
\pgfpathlineto{\pgfqpoint{2.221454in}{1.375019in}}%
\pgfpathlineto{\pgfqpoint{2.221862in}{1.209987in}}%
\pgfpathlineto{\pgfqpoint{2.222067in}{1.531800in}}%
\pgfpathlineto{\pgfqpoint{2.222475in}{1.226490in}}%
\pgfpathlineto{\pgfqpoint{2.222679in}{1.317258in}}%
\pgfpathlineto{\pgfqpoint{2.223291in}{1.176981in}}%
\pgfpathlineto{\pgfqpoint{2.223496in}{1.193484in}}%
\pgfpathlineto{\pgfqpoint{2.224108in}{1.077961in}}%
\pgfpathlineto{\pgfqpoint{2.224720in}{1.160477in}}%
\pgfpathlineto{\pgfqpoint{2.225537in}{1.069710in}}%
\pgfpathlineto{\pgfqpoint{2.225741in}{1.119219in}}%
\pgfpathlineto{\pgfqpoint{2.227170in}{1.226490in}}%
\pgfpathlineto{\pgfqpoint{2.227374in}{1.209987in}}%
\pgfpathlineto{\pgfqpoint{2.228191in}{1.267748in}}%
\pgfpathlineto{\pgfqpoint{2.227987in}{1.201736in}}%
\pgfpathlineto{\pgfqpoint{2.228803in}{1.251245in}}%
\pgfpathlineto{\pgfqpoint{2.229007in}{1.226490in}}%
\pgfpathlineto{\pgfqpoint{2.229416in}{1.284252in}}%
\pgfpathlineto{\pgfqpoint{2.229620in}{1.259497in}}%
\pgfpathlineto{\pgfqpoint{2.229824in}{1.350264in}}%
\pgfpathlineto{\pgfqpoint{2.230232in}{1.226490in}}%
\pgfpathlineto{\pgfqpoint{2.230641in}{1.276000in}}%
\pgfpathlineto{\pgfqpoint{2.232274in}{1.383271in}}%
\pgfpathlineto{\pgfqpoint{2.232682in}{1.333761in}}%
\pgfpathlineto{\pgfqpoint{2.232886in}{1.292503in}}%
\pgfpathlineto{\pgfqpoint{2.233294in}{1.416277in}}%
\pgfpathlineto{\pgfqpoint{2.234111in}{1.482290in}}%
\pgfpathlineto{\pgfqpoint{2.234928in}{1.556554in}}%
\pgfpathlineto{\pgfqpoint{2.234723in}{1.465787in}}%
\pgfpathlineto{\pgfqpoint{2.235132in}{1.548303in}}%
\pgfpathlineto{\pgfqpoint{2.235744in}{1.474038in}}%
\pgfpathlineto{\pgfqpoint{2.236152in}{1.523548in}}%
\pgfpathlineto{\pgfqpoint{2.237377in}{1.606064in}}%
\pgfpathlineto{\pgfqpoint{2.237581in}{1.564806in}}%
\pgfpathlineto{\pgfqpoint{2.237990in}{1.655574in}}%
\pgfpathlineto{\pgfqpoint{2.238194in}{1.630819in}}%
\pgfpathlineto{\pgfqpoint{2.238602in}{1.663825in}}%
\pgfpathlineto{\pgfqpoint{2.238806in}{1.482290in}}%
\pgfpathlineto{\pgfqpoint{2.239623in}{1.738090in}}%
\pgfpathlineto{\pgfqpoint{2.239827in}{1.713335in}}%
\pgfpathlineto{\pgfqpoint{2.240031in}{1.771096in}}%
\pgfpathlineto{\pgfqpoint{2.241052in}{1.870115in}}%
\pgfpathlineto{\pgfqpoint{2.240848in}{1.705083in}}%
\pgfpathlineto{\pgfqpoint{2.241256in}{1.853612in}}%
\pgfpathlineto{\pgfqpoint{2.241868in}{1.861864in}}%
\pgfpathlineto{\pgfqpoint{2.242277in}{1.804103in}}%
\pgfpathlineto{\pgfqpoint{2.243502in}{1.861864in}}%
\pgfpathlineto{\pgfqpoint{2.243706in}{1.853612in}}%
\pgfpathlineto{\pgfqpoint{2.244114in}{1.845361in}}%
\pgfpathlineto{\pgfqpoint{2.244726in}{1.894870in}}%
\pgfpathlineto{\pgfqpoint{2.244931in}{1.853612in}}%
\pgfpathlineto{\pgfqpoint{2.245339in}{1.919625in}}%
\pgfpathlineto{\pgfqpoint{2.245747in}{1.894870in}}%
\pgfpathlineto{\pgfqpoint{2.247584in}{2.043399in}}%
\pgfpathlineto{\pgfqpoint{2.247789in}{2.059902in}}%
\pgfpathlineto{\pgfqpoint{2.248605in}{2.068154in}}%
\pgfpathlineto{\pgfqpoint{2.249013in}{1.878367in}}%
\pgfpathlineto{\pgfqpoint{2.249830in}{2.068154in}}%
\pgfpathlineto{\pgfqpoint{2.250034in}{1.812354in}}%
\pgfpathlineto{\pgfqpoint{2.250442in}{2.117664in}}%
\pgfpathlineto{\pgfqpoint{2.250851in}{2.101160in}}%
\pgfpathlineto{\pgfqpoint{2.251259in}{2.051651in}}%
\pgfpathlineto{\pgfqpoint{2.251463in}{2.018644in}}%
\pgfpathlineto{\pgfqpoint{2.251871in}{2.059902in}}%
\pgfpathlineto{\pgfqpoint{2.252688in}{2.158922in}}%
\pgfpathlineto{\pgfqpoint{2.253096in}{2.092909in}}%
\pgfpathlineto{\pgfqpoint{2.254117in}{2.175425in}}%
\pgfpathlineto{\pgfqpoint{2.254729in}{2.158922in}}%
\pgfpathlineto{\pgfqpoint{2.254934in}{2.216683in}}%
\pgfpathlineto{\pgfqpoint{2.255750in}{2.134167in}}%
\pgfpathlineto{\pgfqpoint{2.256158in}{2.092909in}}%
\pgfpathlineto{\pgfqpoint{2.256567in}{2.167173in}}%
\pgfpathlineto{\pgfqpoint{2.256771in}{2.142418in}}%
\pgfpathlineto{\pgfqpoint{2.257587in}{2.208431in}}%
\pgfpathlineto{\pgfqpoint{2.257383in}{2.125915in}}%
\pgfpathlineto{\pgfqpoint{2.258200in}{2.183676in}}%
\pgfpathlineto{\pgfqpoint{2.259016in}{2.224934in}}%
\pgfpathlineto{\pgfqpoint{2.259425in}{2.109412in}}%
\pgfpathlineto{\pgfqpoint{2.259629in}{2.109412in}}%
\pgfpathlineto{\pgfqpoint{2.259833in}{1.927877in}}%
\pgfpathlineto{\pgfqpoint{2.260241in}{2.142418in}}%
\pgfpathlineto{\pgfqpoint{2.260650in}{2.092909in}}%
\pgfpathlineto{\pgfqpoint{2.261262in}{2.150670in}}%
\pgfpathlineto{\pgfqpoint{2.261466in}{2.059902in}}%
\pgfpathlineto{\pgfqpoint{2.261670in}{2.101160in}}%
\pgfpathlineto{\pgfqpoint{2.262487in}{2.158922in}}%
\pgfpathlineto{\pgfqpoint{2.262078in}{2.092909in}}%
\pgfpathlineto{\pgfqpoint{2.262691in}{2.150670in}}%
\pgfpathlineto{\pgfqpoint{2.263303in}{1.894870in}}%
\pgfpathlineto{\pgfqpoint{2.263712in}{2.101160in}}%
\pgfpathlineto{\pgfqpoint{2.263916in}{2.084657in}}%
\pgfpathlineto{\pgfqpoint{2.264120in}{2.142418in}}%
\pgfpathlineto{\pgfqpoint{2.264528in}{2.109412in}}%
\pgfpathlineto{\pgfqpoint{2.264936in}{2.134167in}}%
\pgfpathlineto{\pgfqpoint{2.265141in}{2.092909in}}%
\pgfpathlineto{\pgfqpoint{2.266161in}{2.010393in}}%
\pgfpathlineto{\pgfqpoint{2.266365in}{2.018644in}}%
\pgfpathlineto{\pgfqpoint{2.268203in}{2.109412in}}%
\pgfpathlineto{\pgfqpoint{2.269019in}{2.043399in}}%
\pgfpathlineto{\pgfqpoint{2.269428in}{2.051651in}}%
\pgfpathlineto{\pgfqpoint{2.269632in}{2.059902in}}%
\pgfpathlineto{\pgfqpoint{2.269836in}{2.035148in}}%
\pgfpathlineto{\pgfqpoint{2.270244in}{1.969135in}}%
\pgfpathlineto{\pgfqpoint{2.270857in}{2.035148in}}%
\pgfpathlineto{\pgfqpoint{2.271265in}{2.035148in}}%
\pgfpathlineto{\pgfqpoint{2.271673in}{2.117664in}}%
\pgfpathlineto{\pgfqpoint{2.272286in}{2.059902in}}%
\pgfpathlineto{\pgfqpoint{2.272898in}{2.026896in}}%
\pgfpathlineto{\pgfqpoint{2.273306in}{2.035148in}}%
\pgfpathlineto{\pgfqpoint{2.274531in}{2.101160in}}%
\pgfpathlineto{\pgfqpoint{2.274735in}{2.101160in}}%
\pgfpathlineto{\pgfqpoint{2.274939in}{2.117664in}}%
\pgfpathlineto{\pgfqpoint{2.275348in}{2.068154in}}%
\pgfpathlineto{\pgfqpoint{2.275552in}{2.059902in}}%
\pgfpathlineto{\pgfqpoint{2.275756in}{2.125915in}}%
\pgfpathlineto{\pgfqpoint{2.276368in}{2.051651in}}%
\pgfpathlineto{\pgfqpoint{2.276573in}{2.059902in}}%
\pgfpathlineto{\pgfqpoint{2.276777in}{2.068154in}}%
\pgfpathlineto{\pgfqpoint{2.276981in}{2.043399in}}%
\pgfpathlineto{\pgfqpoint{2.277185in}{2.035148in}}%
\pgfpathlineto{\pgfqpoint{2.277389in}{2.043399in}}%
\pgfpathlineto{\pgfqpoint{2.277593in}{2.092909in}}%
\pgfpathlineto{\pgfqpoint{2.278002in}{2.026896in}}%
\pgfpathlineto{\pgfqpoint{2.278206in}{2.026896in}}%
\pgfpathlineto{\pgfqpoint{2.278410in}{1.969135in}}%
\pgfpathlineto{\pgfqpoint{2.279431in}{1.993890in}}%
\pgfpathlineto{\pgfqpoint{2.279635in}{2.002141in}}%
\pgfpathlineto{\pgfqpoint{2.280043in}{1.894870in}}%
\pgfpathlineto{\pgfqpoint{2.280451in}{2.026896in}}%
\pgfpathlineto{\pgfqpoint{2.280655in}{1.985638in}}%
\pgfpathlineto{\pgfqpoint{2.281472in}{2.117664in}}%
\pgfpathlineto{\pgfqpoint{2.281880in}{2.092909in}}%
\pgfpathlineto{\pgfqpoint{2.282289in}{2.068154in}}%
\pgfpathlineto{\pgfqpoint{2.282697in}{2.101160in}}%
\pgfpathlineto{\pgfqpoint{2.282901in}{2.092909in}}%
\pgfpathlineto{\pgfqpoint{2.283105in}{2.101160in}}%
\pgfpathlineto{\pgfqpoint{2.283309in}{2.059902in}}%
\pgfpathlineto{\pgfqpoint{2.283922in}{2.150670in}}%
\pgfpathlineto{\pgfqpoint{2.284330in}{2.076406in}}%
\pgfpathlineto{\pgfqpoint{2.284942in}{2.043399in}}%
\pgfpathlineto{\pgfqpoint{2.285759in}{2.167173in}}%
\pgfpathlineto{\pgfqpoint{2.286167in}{2.125915in}}%
\pgfpathlineto{\pgfqpoint{2.286780in}{2.175425in}}%
\pgfpathlineto{\pgfqpoint{2.287188in}{2.068154in}}%
\pgfpathlineto{\pgfqpoint{2.288005in}{2.167173in}}%
\pgfpathlineto{\pgfqpoint{2.287800in}{2.018644in}}%
\pgfpathlineto{\pgfqpoint{2.288413in}{2.134167in}}%
\pgfpathlineto{\pgfqpoint{2.288617in}{2.142418in}}%
\pgfpathlineto{\pgfqpoint{2.290046in}{2.059902in}}%
\pgfpathlineto{\pgfqpoint{2.290250in}{2.084657in}}%
\pgfpathlineto{\pgfqpoint{2.290658in}{2.010393in}}%
\pgfpathlineto{\pgfqpoint{2.291067in}{2.076406in}}%
\pgfpathlineto{\pgfqpoint{2.291475in}{2.035148in}}%
\pgfpathlineto{\pgfqpoint{2.291883in}{2.084657in}}%
\pgfpathlineto{\pgfqpoint{2.292087in}{2.084657in}}%
\pgfpathlineto{\pgfqpoint{2.293721in}{2.241438in}}%
\pgfpathlineto{\pgfqpoint{2.293925in}{2.241438in}}%
\pgfpathlineto{\pgfqpoint{2.294537in}{2.010393in}}%
\pgfpathlineto{\pgfqpoint{2.295150in}{2.117664in}}%
\pgfpathlineto{\pgfqpoint{2.295354in}{2.002141in}}%
\pgfpathlineto{\pgfqpoint{2.295558in}{2.167173in}}%
\pgfpathlineto{\pgfqpoint{2.296170in}{2.142418in}}%
\pgfpathlineto{\pgfqpoint{2.296987in}{2.125915in}}%
\pgfpathlineto{\pgfqpoint{2.297395in}{2.183676in}}%
\pgfpathlineto{\pgfqpoint{2.297803in}{2.175425in}}%
\pgfpathlineto{\pgfqpoint{2.298212in}{1.878367in}}%
\pgfpathlineto{\pgfqpoint{2.298620in}{2.208431in}}%
\pgfpathlineto{\pgfqpoint{2.298824in}{2.183676in}}%
\pgfpathlineto{\pgfqpoint{2.299437in}{2.183676in}}%
\pgfpathlineto{\pgfqpoint{2.299845in}{2.142418in}}%
\pgfpathlineto{\pgfqpoint{2.300661in}{2.167173in}}%
\pgfpathlineto{\pgfqpoint{2.301070in}{2.142418in}}%
\pgfpathlineto{\pgfqpoint{2.302090in}{2.249689in}}%
\pgfpathlineto{\pgfqpoint{2.302703in}{2.043399in}}%
\pgfpathlineto{\pgfqpoint{2.302499in}{2.257941in}}%
\pgfpathlineto{\pgfqpoint{2.303315in}{2.150670in}}%
\pgfpathlineto{\pgfqpoint{2.304540in}{2.282696in}}%
\pgfpathlineto{\pgfqpoint{2.305153in}{2.208431in}}%
\pgfpathlineto{\pgfqpoint{2.305561in}{2.266192in}}%
\pgfpathlineto{\pgfqpoint{2.305765in}{2.266192in}}%
\pgfpathlineto{\pgfqpoint{2.305969in}{2.216683in}}%
\pgfpathlineto{\pgfqpoint{2.306582in}{2.299199in}}%
\pgfpathlineto{\pgfqpoint{2.306786in}{2.299199in}}%
\pgfpathlineto{\pgfqpoint{2.307194in}{2.356960in}}%
\pgfpathlineto{\pgfqpoint{2.307806in}{2.307450in}}%
\pgfpathlineto{\pgfqpoint{2.309440in}{2.018644in}}%
\pgfpathlineto{\pgfqpoint{2.309644in}{2.068154in}}%
\pgfpathlineto{\pgfqpoint{2.310052in}{2.274444in}}%
\pgfpathlineto{\pgfqpoint{2.310869in}{2.249689in}}%
\pgfpathlineto{\pgfqpoint{2.311073in}{2.200180in}}%
\pgfpathlineto{\pgfqpoint{2.311685in}{2.290947in}}%
\pgfpathlineto{\pgfqpoint{2.312502in}{2.257941in}}%
\pgfpathlineto{\pgfqpoint{2.312706in}{2.332205in}}%
\pgfpathlineto{\pgfqpoint{2.313318in}{2.200180in}}%
\pgfpathlineto{\pgfqpoint{2.313522in}{2.241438in}}%
\pgfpathlineto{\pgfqpoint{2.314339in}{2.175425in}}%
\pgfpathlineto{\pgfqpoint{2.314543in}{2.216683in}}%
\pgfpathlineto{\pgfqpoint{2.315156in}{2.200180in}}%
\pgfpathlineto{\pgfqpoint{2.315972in}{2.307450in}}%
\pgfpathlineto{\pgfqpoint{2.316585in}{1.993890in}}%
\pgfpathlineto{\pgfqpoint{2.316789in}{2.332205in}}%
\pgfpathlineto{\pgfqpoint{2.316993in}{2.290947in}}%
\pgfpathlineto{\pgfqpoint{2.317401in}{2.323954in}}%
\pgfpathlineto{\pgfqpoint{2.317605in}{2.282696in}}%
\pgfpathlineto{\pgfqpoint{2.317809in}{2.216683in}}%
\pgfpathlineto{\pgfqpoint{2.318422in}{2.290947in}}%
\pgfpathlineto{\pgfqpoint{2.318626in}{2.274444in}}%
\pgfpathlineto{\pgfqpoint{2.319238in}{2.315702in}}%
\pgfpathlineto{\pgfqpoint{2.320055in}{2.125915in}}%
\pgfpathlineto{\pgfqpoint{2.320463in}{2.191928in}}%
\pgfpathlineto{\pgfqpoint{2.320667in}{2.191928in}}%
\pgfpathlineto{\pgfqpoint{2.320872in}{2.224934in}}%
\pgfpathlineto{\pgfqpoint{2.321280in}{2.150670in}}%
\pgfpathlineto{\pgfqpoint{2.321484in}{2.183676in}}%
\pgfpathlineto{\pgfqpoint{2.321688in}{2.167173in}}%
\pgfpathlineto{\pgfqpoint{2.321892in}{2.216683in}}%
\pgfpathlineto{\pgfqpoint{2.322301in}{2.191928in}}%
\pgfpathlineto{\pgfqpoint{2.322505in}{2.216683in}}%
\pgfpathlineto{\pgfqpoint{2.322913in}{2.208431in}}%
\pgfpathlineto{\pgfqpoint{2.323117in}{2.002141in}}%
\pgfpathlineto{\pgfqpoint{2.323934in}{2.249689in}}%
\pgfpathlineto{\pgfqpoint{2.324342in}{2.018644in}}%
\pgfpathlineto{\pgfqpoint{2.324954in}{2.233186in}}%
\pgfpathlineto{\pgfqpoint{2.325975in}{2.158922in}}%
\pgfpathlineto{\pgfqpoint{2.326179in}{2.167173in}}%
\pgfpathlineto{\pgfqpoint{2.326383in}{2.125915in}}%
\pgfpathlineto{\pgfqpoint{2.326792in}{2.216683in}}%
\pgfpathlineto{\pgfqpoint{2.326996in}{2.216683in}}%
\pgfpathlineto{\pgfqpoint{2.328221in}{2.315702in}}%
\pgfpathlineto{\pgfqpoint{2.328629in}{2.257941in}}%
\pgfpathlineto{\pgfqpoint{2.328833in}{2.348709in}}%
\pgfpathlineto{\pgfqpoint{2.329241in}{2.282696in}}%
\pgfpathlineto{\pgfqpoint{2.329854in}{2.422973in}}%
\pgfpathlineto{\pgfqpoint{2.330670in}{2.381715in}}%
\pgfpathlineto{\pgfqpoint{2.332916in}{2.241438in}}%
\pgfpathlineto{\pgfqpoint{2.333120in}{2.224934in}}%
\pgfpathlineto{\pgfqpoint{2.333733in}{2.249689in}}%
\pgfpathlineto{\pgfqpoint{2.333937in}{2.241438in}}%
\pgfpathlineto{\pgfqpoint{2.335366in}{2.389967in}}%
\pgfpathlineto{\pgfqpoint{2.335774in}{2.299199in}}%
\pgfpathlineto{\pgfqpoint{2.336591in}{2.356960in}}%
\pgfpathlineto{\pgfqpoint{2.337407in}{2.373463in}}%
\pgfpathlineto{\pgfqpoint{2.337611in}{2.356960in}}%
\pgfpathlineto{\pgfqpoint{2.338224in}{2.299199in}}%
\pgfpathlineto{\pgfqpoint{2.339449in}{2.307450in}}%
\pgfpathlineto{\pgfqpoint{2.339653in}{2.340457in}}%
\pgfpathlineto{\pgfqpoint{2.340061in}{2.249689in}}%
\pgfpathlineto{\pgfqpoint{2.340265in}{2.323954in}}%
\pgfpathlineto{\pgfqpoint{2.340469in}{2.266192in}}%
\pgfpathlineto{\pgfqpoint{2.341286in}{2.307450in}}%
\pgfpathlineto{\pgfqpoint{2.342511in}{2.224934in}}%
\pgfpathlineto{\pgfqpoint{2.342715in}{2.274444in}}%
\pgfpathlineto{\pgfqpoint{2.343940in}{2.356960in}}%
\pgfpathlineto{\pgfqpoint{2.343123in}{2.216683in}}%
\pgfpathlineto{\pgfqpoint{2.344144in}{2.332205in}}%
\pgfpathlineto{\pgfqpoint{2.344348in}{2.332205in}}%
\pgfpathlineto{\pgfqpoint{2.344756in}{2.257941in}}%
\pgfpathlineto{\pgfqpoint{2.345165in}{2.340457in}}%
\pgfpathlineto{\pgfqpoint{2.345369in}{2.323954in}}%
\pgfpathlineto{\pgfqpoint{2.345777in}{2.398218in}}%
\pgfpathlineto{\pgfqpoint{2.346389in}{2.356960in}}%
\pgfpathlineto{\pgfqpoint{2.347002in}{2.389967in}}%
\pgfpathlineto{\pgfqpoint{2.347206in}{2.340457in}}%
\pgfpathlineto{\pgfqpoint{2.348431in}{2.480734in}}%
\pgfpathlineto{\pgfqpoint{2.350472in}{2.315702in}}%
\pgfpathlineto{\pgfqpoint{2.351901in}{2.546747in}}%
\pgfpathlineto{\pgfqpoint{2.352105in}{2.530244in}}%
\pgfpathlineto{\pgfqpoint{2.352309in}{2.530244in}}%
\pgfpathlineto{\pgfqpoint{2.352514in}{2.571502in}}%
\pgfpathlineto{\pgfqpoint{2.353126in}{2.513741in}}%
\pgfpathlineto{\pgfqpoint{2.353330in}{2.554999in}}%
\pgfpathlineto{\pgfqpoint{2.354555in}{2.472483in}}%
\pgfpathlineto{\pgfqpoint{2.355576in}{2.538495in}}%
\pgfpathlineto{\pgfqpoint{2.356392in}{2.447728in}}%
\pgfpathlineto{\pgfqpoint{2.356801in}{2.455979in}}%
\pgfpathlineto{\pgfqpoint{2.357005in}{2.439476in}}%
\pgfpathlineto{\pgfqpoint{2.357209in}{2.455979in}}%
\pgfpathlineto{\pgfqpoint{2.357413in}{2.513741in}}%
\pgfpathlineto{\pgfqpoint{2.358025in}{2.505489in}}%
\pgfpathlineto{\pgfqpoint{2.358230in}{2.406470in}}%
\pgfpathlineto{\pgfqpoint{2.359046in}{2.505489in}}%
\pgfpathlineto{\pgfqpoint{2.359250in}{2.455979in}}%
\pgfpathlineto{\pgfqpoint{2.359863in}{2.554999in}}%
\pgfpathlineto{\pgfqpoint{2.360271in}{2.464231in}}%
\pgfpathlineto{\pgfqpoint{2.360475in}{2.455979in}}%
\pgfpathlineto{\pgfqpoint{2.360679in}{2.464231in}}%
\pgfpathlineto{\pgfqpoint{2.361700in}{2.571502in}}%
\pgfpathlineto{\pgfqpoint{2.362108in}{2.538495in}}%
\pgfpathlineto{\pgfqpoint{2.362312in}{2.546747in}}%
\pgfpathlineto{\pgfqpoint{2.362517in}{2.521992in}}%
\pgfpathlineto{\pgfqpoint{2.363129in}{2.455979in}}%
\pgfpathlineto{\pgfqpoint{2.363537in}{2.530244in}}%
\pgfpathlineto{\pgfqpoint{2.363741in}{2.530244in}}%
\pgfpathlineto{\pgfqpoint{2.364966in}{2.447728in}}%
\pgfpathlineto{\pgfqpoint{2.365375in}{2.455979in}}%
\pgfpathlineto{\pgfqpoint{2.365783in}{2.538495in}}%
\pgfpathlineto{\pgfqpoint{2.366395in}{2.480734in}}%
\pgfpathlineto{\pgfqpoint{2.366599in}{2.480734in}}%
\pgfpathlineto{\pgfqpoint{2.367620in}{2.398218in}}%
\pgfpathlineto{\pgfqpoint{2.367824in}{2.472483in}}%
\pgfpathlineto{\pgfqpoint{2.368845in}{2.455979in}}%
\pgfpathlineto{\pgfqpoint{2.369866in}{2.414721in}}%
\pgfpathlineto{\pgfqpoint{2.370070in}{2.422973in}}%
\pgfpathlineto{\pgfqpoint{2.370886in}{2.398218in}}%
\pgfpathlineto{\pgfqpoint{2.371091in}{2.406470in}}%
\pgfpathlineto{\pgfqpoint{2.371907in}{2.398218in}}%
\pgfpathlineto{\pgfqpoint{2.372724in}{2.579753in}}%
\pgfpathlineto{\pgfqpoint{2.373540in}{2.472483in}}%
\pgfpathlineto{\pgfqpoint{2.373949in}{2.480734in}}%
\pgfpathlineto{\pgfqpoint{2.374153in}{2.546747in}}%
\pgfpathlineto{\pgfqpoint{2.374969in}{2.505489in}}%
\pgfpathlineto{\pgfqpoint{2.375173in}{2.488986in}}%
\pgfpathlineto{\pgfqpoint{2.375378in}{2.505489in}}%
\pgfpathlineto{\pgfqpoint{2.375582in}{2.332205in}}%
\pgfpathlineto{\pgfqpoint{2.375786in}{2.546747in}}%
\pgfpathlineto{\pgfqpoint{2.376398in}{2.521992in}}%
\pgfpathlineto{\pgfqpoint{2.376602in}{2.538495in}}%
\pgfpathlineto{\pgfqpoint{2.377215in}{2.521992in}}%
\pgfpathlineto{\pgfqpoint{2.378440in}{2.422973in}}%
\pgfpathlineto{\pgfqpoint{2.378644in}{2.431225in}}%
\pgfpathlineto{\pgfqpoint{2.379665in}{2.472483in}}%
\pgfpathlineto{\pgfqpoint{2.379869in}{2.406470in}}%
\pgfpathlineto{\pgfqpoint{2.380073in}{2.480734in}}%
\pgfpathlineto{\pgfqpoint{2.380685in}{2.464231in}}%
\pgfpathlineto{\pgfqpoint{2.380889in}{2.464231in}}%
\pgfpathlineto{\pgfqpoint{2.381502in}{2.431225in}}%
\pgfpathlineto{\pgfqpoint{2.382727in}{2.224934in}}%
\pgfpathlineto{\pgfqpoint{2.382114in}{2.464231in}}%
\pgfpathlineto{\pgfqpoint{2.382931in}{2.299199in}}%
\pgfpathlineto{\pgfqpoint{2.384564in}{2.464231in}}%
\pgfpathlineto{\pgfqpoint{2.385381in}{2.398218in}}%
\pgfpathlineto{\pgfqpoint{2.385585in}{2.431225in}}%
\pgfpathlineto{\pgfqpoint{2.385789in}{2.464231in}}%
\pgfpathlineto{\pgfqpoint{2.385993in}{2.422973in}}%
\pgfpathlineto{\pgfqpoint{2.386605in}{2.191928in}}%
\pgfpathlineto{\pgfqpoint{2.386810in}{2.488986in}}%
\pgfpathlineto{\pgfqpoint{2.387014in}{2.464231in}}%
\pgfpathlineto{\pgfqpoint{2.387218in}{2.579753in}}%
\pgfpathlineto{\pgfqpoint{2.387422in}{2.422973in}}%
\pgfpathlineto{\pgfqpoint{2.388034in}{2.513741in}}%
\pgfpathlineto{\pgfqpoint{2.389259in}{2.431225in}}%
\pgfpathlineto{\pgfqpoint{2.389463in}{2.439476in}}%
\pgfpathlineto{\pgfqpoint{2.389668in}{2.513741in}}%
\pgfpathlineto{\pgfqpoint{2.390076in}{2.389967in}}%
\pgfpathlineto{\pgfqpoint{2.390484in}{2.447728in}}%
\pgfpathlineto{\pgfqpoint{2.391301in}{2.340457in}}%
\pgfpathlineto{\pgfqpoint{2.391913in}{2.348709in}}%
\pgfpathlineto{\pgfqpoint{2.392730in}{2.389967in}}%
\pgfpathlineto{\pgfqpoint{2.392934in}{2.381715in}}%
\pgfpathlineto{\pgfqpoint{2.393750in}{2.183676in}}%
\pgfpathlineto{\pgfqpoint{2.393955in}{2.332205in}}%
\pgfpathlineto{\pgfqpoint{2.394771in}{2.422973in}}%
\pgfpathlineto{\pgfqpoint{2.394567in}{2.315702in}}%
\pgfpathlineto{\pgfqpoint{2.394975in}{2.365212in}}%
\pgfpathlineto{\pgfqpoint{2.395179in}{2.340457in}}%
\pgfpathlineto{\pgfqpoint{2.396608in}{2.554999in}}%
\pgfpathlineto{\pgfqpoint{2.396813in}{2.554999in}}%
\pgfpathlineto{\pgfqpoint{2.397017in}{2.563250in}}%
\pgfpathlineto{\pgfqpoint{2.397221in}{2.621011in}}%
\pgfpathlineto{\pgfqpoint{2.397629in}{2.464231in}}%
\pgfpathlineto{\pgfqpoint{2.398037in}{2.431225in}}%
\pgfpathlineto{\pgfqpoint{2.398242in}{2.389967in}}%
\pgfpathlineto{\pgfqpoint{2.398854in}{2.472483in}}%
\pgfpathlineto{\pgfqpoint{2.399058in}{2.480734in}}%
\pgfpathlineto{\pgfqpoint{2.399262in}{2.455979in}}%
\pgfpathlineto{\pgfqpoint{2.399671in}{2.398218in}}%
\pgfpathlineto{\pgfqpoint{2.399875in}{2.530244in}}%
\pgfpathlineto{\pgfqpoint{2.400895in}{2.488986in}}%
\pgfpathlineto{\pgfqpoint{2.401304in}{2.488986in}}%
\pgfpathlineto{\pgfqpoint{2.401712in}{2.521992in}}%
\pgfpathlineto{\pgfqpoint{2.402120in}{2.497237in}}%
\pgfpathlineto{\pgfqpoint{2.402529in}{2.398218in}}%
\pgfpathlineto{\pgfqpoint{2.403549in}{2.439476in}}%
\pgfpathlineto{\pgfqpoint{2.405182in}{2.546747in}}%
\pgfpathlineto{\pgfqpoint{2.405387in}{2.480734in}}%
\pgfpathlineto{\pgfqpoint{2.406203in}{2.571502in}}%
\pgfpathlineto{\pgfqpoint{2.407836in}{2.480734in}}%
\pgfpathlineto{\pgfqpoint{2.408245in}{2.455979in}}%
\pgfpathlineto{\pgfqpoint{2.409061in}{2.521992in}}%
\pgfpathlineto{\pgfqpoint{2.409265in}{2.521992in}}%
\pgfpathlineto{\pgfqpoint{2.410082in}{2.439476in}}%
\pgfpathlineto{\pgfqpoint{2.410286in}{2.464231in}}%
\pgfpathlineto{\pgfqpoint{2.410490in}{2.505489in}}%
\pgfpathlineto{\pgfqpoint{2.410898in}{2.439476in}}%
\pgfpathlineto{\pgfqpoint{2.411511in}{2.497237in}}%
\pgfpathlineto{\pgfqpoint{2.412736in}{2.439476in}}%
\pgfpathlineto{\pgfqpoint{2.413756in}{2.521992in}}%
\pgfpathlineto{\pgfqpoint{2.413348in}{2.422973in}}%
\pgfpathlineto{\pgfqpoint{2.413961in}{2.497237in}}%
\pgfpathlineto{\pgfqpoint{2.414369in}{2.521992in}}%
\pgfpathlineto{\pgfqpoint{2.415390in}{2.422973in}}%
\pgfpathlineto{\pgfqpoint{2.416002in}{2.488986in}}%
\pgfpathlineto{\pgfqpoint{2.416410in}{2.447728in}}%
\pgfpathlineto{\pgfqpoint{2.416614in}{2.299199in}}%
\pgfpathlineto{\pgfqpoint{2.416819in}{2.488986in}}%
\pgfpathlineto{\pgfqpoint{2.417431in}{2.480734in}}%
\pgfpathlineto{\pgfqpoint{2.417839in}{2.488986in}}%
\pgfpathlineto{\pgfqpoint{2.418248in}{2.604508in}}%
\pgfpathlineto{\pgfqpoint{2.418860in}{2.588005in}}%
\pgfpathlineto{\pgfqpoint{2.419268in}{2.464231in}}%
\pgfpathlineto{\pgfqpoint{2.419881in}{2.521992in}}%
\pgfpathlineto{\pgfqpoint{2.420493in}{2.546747in}}%
\pgfpathlineto{\pgfqpoint{2.421106in}{2.422973in}}%
\pgfpathlineto{\pgfqpoint{2.421514in}{2.455979in}}%
\pgfpathlineto{\pgfqpoint{2.421718in}{2.521992in}}%
\pgfpathlineto{\pgfqpoint{2.421922in}{2.431225in}}%
\pgfpathlineto{\pgfqpoint{2.422535in}{2.447728in}}%
\pgfpathlineto{\pgfqpoint{2.423351in}{2.505489in}}%
\pgfpathlineto{\pgfqpoint{2.423964in}{2.480734in}}%
\pgfpathlineto{\pgfqpoint{2.425188in}{2.422973in}}%
\pgfpathlineto{\pgfqpoint{2.426209in}{2.488986in}}%
\pgfpathlineto{\pgfqpoint{2.426413in}{2.307450in}}%
\pgfpathlineto{\pgfqpoint{2.427230in}{2.530244in}}%
\pgfpathlineto{\pgfqpoint{2.427434in}{2.554999in}}%
\pgfpathlineto{\pgfqpoint{2.427638in}{2.497237in}}%
\pgfpathlineto{\pgfqpoint{2.427842in}{2.513741in}}%
\pgfpathlineto{\pgfqpoint{2.429475in}{2.422973in}}%
\pgfpathlineto{\pgfqpoint{2.430496in}{2.554999in}}%
\pgfpathlineto{\pgfqpoint{2.430088in}{2.389967in}}%
\pgfpathlineto{\pgfqpoint{2.430904in}{2.488986in}}%
\pgfpathlineto{\pgfqpoint{2.431313in}{2.158922in}}%
\pgfpathlineto{\pgfqpoint{2.432129in}{2.323954in}}%
\pgfpathlineto{\pgfqpoint{2.433354in}{2.455979in}}%
\pgfpathlineto{\pgfqpoint{2.433558in}{2.323954in}}%
\pgfpathlineto{\pgfqpoint{2.434375in}{2.472483in}}%
\pgfpathlineto{\pgfqpoint{2.436008in}{2.175425in}}%
\pgfpathlineto{\pgfqpoint{2.437233in}{2.579753in}}%
\pgfpathlineto{\pgfqpoint{2.439274in}{2.422973in}}%
\pgfpathlineto{\pgfqpoint{2.437641in}{2.596257in}}%
\pgfpathlineto{\pgfqpoint{2.439478in}{2.455979in}}%
\pgfpathlineto{\pgfqpoint{2.440907in}{2.563250in}}%
\pgfpathlineto{\pgfqpoint{2.441316in}{2.563250in}}%
\pgfpathlineto{\pgfqpoint{2.441520in}{2.488986in}}%
\pgfpathlineto{\pgfqpoint{2.442132in}{2.579753in}}%
\pgfpathlineto{\pgfqpoint{2.442336in}{2.513741in}}%
\pgfpathlineto{\pgfqpoint{2.442949in}{2.497237in}}%
\pgfpathlineto{\pgfqpoint{2.443357in}{2.563250in}}%
\pgfpathlineto{\pgfqpoint{2.444174in}{2.480734in}}%
\pgfpathlineto{\pgfqpoint{2.444378in}{2.563250in}}%
\pgfpathlineto{\pgfqpoint{2.445194in}{2.538495in}}%
\pgfpathlineto{\pgfqpoint{2.445603in}{2.612760in}}%
\pgfpathlineto{\pgfqpoint{2.446011in}{2.505489in}}%
\pgfpathlineto{\pgfqpoint{2.446827in}{2.554999in}}%
\pgfpathlineto{\pgfqpoint{2.447032in}{2.579753in}}%
\pgfpathlineto{\pgfqpoint{2.447440in}{2.554999in}}%
\pgfpathlineto{\pgfqpoint{2.447848in}{2.257941in}}%
\pgfpathlineto{\pgfqpoint{2.448461in}{2.464231in}}%
\pgfpathlineto{\pgfqpoint{2.449481in}{2.579753in}}%
\pgfpathlineto{\pgfqpoint{2.449685in}{2.538495in}}%
\pgfpathlineto{\pgfqpoint{2.450094in}{2.513741in}}%
\pgfpathlineto{\pgfqpoint{2.450706in}{2.563250in}}%
\pgfpathlineto{\pgfqpoint{2.450910in}{2.299199in}}%
\pgfpathlineto{\pgfqpoint{2.451931in}{2.406470in}}%
\pgfpathlineto{\pgfqpoint{2.452339in}{2.389967in}}%
\pgfpathlineto{\pgfqpoint{2.453156in}{2.447728in}}%
\pgfpathlineto{\pgfqpoint{2.453360in}{2.398218in}}%
\pgfpathlineto{\pgfqpoint{2.453564in}{2.464231in}}%
\pgfpathlineto{\pgfqpoint{2.453972in}{2.439476in}}%
\pgfpathlineto{\pgfqpoint{2.454177in}{2.480734in}}%
\pgfpathlineto{\pgfqpoint{2.454789in}{2.389967in}}%
\pgfpathlineto{\pgfqpoint{2.454993in}{2.381715in}}%
\pgfpathlineto{\pgfqpoint{2.455197in}{2.158922in}}%
\pgfpathlineto{\pgfqpoint{2.456014in}{2.439476in}}%
\pgfpathlineto{\pgfqpoint{2.456218in}{2.406470in}}%
\pgfpathlineto{\pgfqpoint{2.456422in}{2.464231in}}%
\pgfpathlineto{\pgfqpoint{2.457239in}{2.604508in}}%
\pgfpathlineto{\pgfqpoint{2.457647in}{2.579753in}}%
\pgfpathlineto{\pgfqpoint{2.458668in}{2.596257in}}%
\pgfpathlineto{\pgfqpoint{2.458872in}{2.497237in}}%
\pgfpathlineto{\pgfqpoint{2.459688in}{2.621011in}}%
\pgfpathlineto{\pgfqpoint{2.459893in}{2.546747in}}%
\pgfpathlineto{\pgfqpoint{2.460097in}{2.340457in}}%
\pgfpathlineto{\pgfqpoint{2.461117in}{2.381715in}}%
\pgfpathlineto{\pgfqpoint{2.462955in}{2.142418in}}%
\pgfpathlineto{\pgfqpoint{2.463975in}{2.290947in}}%
\pgfpathlineto{\pgfqpoint{2.464180in}{2.241438in}}%
\pgfpathlineto{\pgfqpoint{2.464384in}{2.290947in}}%
\pgfpathlineto{\pgfqpoint{2.464588in}{2.158922in}}%
\pgfpathlineto{\pgfqpoint{2.464792in}{2.084657in}}%
\pgfpathlineto{\pgfqpoint{2.465200in}{2.274444in}}%
\pgfpathlineto{\pgfqpoint{2.465404in}{2.241438in}}%
\pgfpathlineto{\pgfqpoint{2.466425in}{2.018644in}}%
\pgfpathlineto{\pgfqpoint{2.466221in}{2.249689in}}%
\pgfpathlineto{\pgfqpoint{2.466629in}{2.134167in}}%
\pgfpathlineto{\pgfqpoint{2.467650in}{2.356960in}}%
\pgfpathlineto{\pgfqpoint{2.467854in}{2.257941in}}%
\pgfpathlineto{\pgfqpoint{2.468058in}{2.233186in}}%
\pgfpathlineto{\pgfqpoint{2.468467in}{2.315702in}}%
\pgfpathlineto{\pgfqpoint{2.468671in}{2.389967in}}%
\pgfpathlineto{\pgfqpoint{2.469079in}{2.266192in}}%
\pgfpathlineto{\pgfqpoint{2.469283in}{2.323954in}}%
\pgfpathlineto{\pgfqpoint{2.469487in}{2.257941in}}%
\pgfpathlineto{\pgfqpoint{2.469691in}{2.332205in}}%
\pgfpathlineto{\pgfqpoint{2.470304in}{2.282696in}}%
\pgfpathlineto{\pgfqpoint{2.471529in}{2.398218in}}%
\pgfpathlineto{\pgfqpoint{2.472141in}{2.389967in}}%
\pgfpathlineto{\pgfqpoint{2.473366in}{2.142418in}}%
\pgfpathlineto{\pgfqpoint{2.473570in}{2.356960in}}%
\pgfpathlineto{\pgfqpoint{2.474591in}{2.299199in}}%
\pgfpathlineto{\pgfqpoint{2.474795in}{2.332205in}}%
\pgfpathlineto{\pgfqpoint{2.474999in}{2.257941in}}%
\pgfpathlineto{\pgfqpoint{2.475203in}{2.200180in}}%
\pgfpathlineto{\pgfqpoint{2.475816in}{2.340457in}}%
\pgfpathlineto{\pgfqpoint{2.476020in}{2.348709in}}%
\pgfpathlineto{\pgfqpoint{2.476428in}{2.224934in}}%
\pgfpathlineto{\pgfqpoint{2.477041in}{2.299199in}}%
\pgfpathlineto{\pgfqpoint{2.477449in}{2.332205in}}%
\pgfpathlineto{\pgfqpoint{2.477653in}{2.323954in}}%
\pgfpathlineto{\pgfqpoint{2.477857in}{2.233186in}}%
\pgfpathlineto{\pgfqpoint{2.478061in}{2.365212in}}%
\pgfpathlineto{\pgfqpoint{2.478878in}{2.249689in}}%
\pgfpathlineto{\pgfqpoint{2.479082in}{2.299199in}}%
\pgfpathlineto{\pgfqpoint{2.479694in}{2.233186in}}%
\pgfpathlineto{\pgfqpoint{2.479899in}{2.266192in}}%
\pgfpathlineto{\pgfqpoint{2.480307in}{1.993890in}}%
\pgfpathlineto{\pgfqpoint{2.481123in}{2.200180in}}%
\pgfpathlineto{\pgfqpoint{2.482348in}{2.348709in}}%
\pgfpathlineto{\pgfqpoint{2.482552in}{2.266192in}}%
\pgfpathlineto{\pgfqpoint{2.483369in}{2.348709in}}%
\pgfpathlineto{\pgfqpoint{2.483777in}{2.439476in}}%
\pgfpathlineto{\pgfqpoint{2.483981in}{2.340457in}}%
\pgfpathlineto{\pgfqpoint{2.484390in}{2.422973in}}%
\pgfpathlineto{\pgfqpoint{2.485615in}{2.200180in}}%
\pgfpathlineto{\pgfqpoint{2.486023in}{2.249689in}}%
\pgfpathlineto{\pgfqpoint{2.486227in}{2.307450in}}%
\pgfpathlineto{\pgfqpoint{2.486839in}{2.241438in}}%
\pgfpathlineto{\pgfqpoint{2.487044in}{2.266192in}}%
\pgfpathlineto{\pgfqpoint{2.488064in}{2.216683in}}%
\pgfpathlineto{\pgfqpoint{2.488268in}{2.125915in}}%
\pgfpathlineto{\pgfqpoint{2.488677in}{2.249689in}}%
\pgfpathlineto{\pgfqpoint{2.489085in}{2.233186in}}%
\pgfpathlineto{\pgfqpoint{2.490106in}{2.340457in}}%
\pgfpathlineto{\pgfqpoint{2.489697in}{2.216683in}}%
\pgfpathlineto{\pgfqpoint{2.490514in}{2.266192in}}%
\pgfpathlineto{\pgfqpoint{2.491943in}{1.952632in}}%
\pgfpathlineto{\pgfqpoint{2.492147in}{2.002141in}}%
\pgfpathlineto{\pgfqpoint{2.492555in}{2.191928in}}%
\pgfpathlineto{\pgfqpoint{2.493372in}{2.183676in}}%
\pgfpathlineto{\pgfqpoint{2.494597in}{2.092909in}}%
\pgfpathlineto{\pgfqpoint{2.495413in}{2.150670in}}%
\pgfpathlineto{\pgfqpoint{2.496638in}{1.977386in}}%
\pgfpathlineto{\pgfqpoint{2.498067in}{2.191928in}}%
\pgfpathlineto{\pgfqpoint{2.499088in}{2.134167in}}%
\pgfpathlineto{\pgfqpoint{2.498884in}{2.224934in}}%
\pgfpathlineto{\pgfqpoint{2.499292in}{2.167173in}}%
\pgfpathlineto{\pgfqpoint{2.499496in}{2.233186in}}%
\pgfpathlineto{\pgfqpoint{2.499700in}{2.134167in}}%
\pgfpathlineto{\pgfqpoint{2.499905in}{2.191928in}}%
\pgfpathlineto{\pgfqpoint{2.501129in}{2.018644in}}%
\pgfpathlineto{\pgfqpoint{2.502150in}{2.158922in}}%
\pgfpathlineto{\pgfqpoint{2.502354in}{2.125915in}}%
\pgfpathlineto{\pgfqpoint{2.502763in}{2.150670in}}%
\pgfpathlineto{\pgfqpoint{2.502967in}{2.200180in}}%
\pgfpathlineto{\pgfqpoint{2.503783in}{2.134167in}}%
\pgfpathlineto{\pgfqpoint{2.505416in}{2.241438in}}%
\pgfpathlineto{\pgfqpoint{2.505825in}{2.150670in}}%
\pgfpathlineto{\pgfqpoint{2.506437in}{2.233186in}}%
\pgfpathlineto{\pgfqpoint{2.506641in}{2.249689in}}%
\pgfpathlineto{\pgfqpoint{2.506845in}{2.216683in}}%
\pgfpathlineto{\pgfqpoint{2.507662in}{2.167173in}}%
\pgfpathlineto{\pgfqpoint{2.507866in}{2.175425in}}%
\pgfpathlineto{\pgfqpoint{2.508070in}{2.208431in}}%
\pgfpathlineto{\pgfqpoint{2.508683in}{2.158922in}}%
\pgfpathlineto{\pgfqpoint{2.509908in}{2.026896in}}%
\pgfpathlineto{\pgfqpoint{2.510112in}{2.035148in}}%
\pgfpathlineto{\pgfqpoint{2.510520in}{2.150670in}}%
\pgfpathlineto{\pgfqpoint{2.510928in}{2.026896in}}%
\pgfpathlineto{\pgfqpoint{2.511132in}{2.051651in}}%
\pgfpathlineto{\pgfqpoint{2.511337in}{2.002141in}}%
\pgfpathlineto{\pgfqpoint{2.512153in}{2.068154in}}%
\pgfpathlineto{\pgfqpoint{2.512357in}{2.068154in}}%
\pgfpathlineto{\pgfqpoint{2.512561in}{2.026896in}}%
\pgfpathlineto{\pgfqpoint{2.512970in}{2.084657in}}%
\pgfpathlineto{\pgfqpoint{2.513174in}{2.175425in}}%
\pgfpathlineto{\pgfqpoint{2.513786in}{1.977386in}}%
\pgfpathlineto{\pgfqpoint{2.514399in}{2.142418in}}%
\pgfpathlineto{\pgfqpoint{2.514807in}{1.977386in}}%
\pgfpathlineto{\pgfqpoint{2.515011in}{2.002141in}}%
\pgfpathlineto{\pgfqpoint{2.515215in}{1.944380in}}%
\pgfpathlineto{\pgfqpoint{2.515828in}{1.977386in}}%
\pgfpathlineto{\pgfqpoint{2.517257in}{1.828857in}}%
\pgfpathlineto{\pgfqpoint{2.516236in}{1.985638in}}%
\pgfpathlineto{\pgfqpoint{2.517461in}{1.853612in}}%
\pgfpathlineto{\pgfqpoint{2.517665in}{1.903122in}}%
\pgfpathlineto{\pgfqpoint{2.518073in}{1.845361in}}%
\pgfpathlineto{\pgfqpoint{2.518277in}{1.639071in}}%
\pgfpathlineto{\pgfqpoint{2.519094in}{1.903122in}}%
\pgfpathlineto{\pgfqpoint{2.519706in}{1.944380in}}%
\pgfpathlineto{\pgfqpoint{2.519502in}{1.878367in}}%
\pgfpathlineto{\pgfqpoint{2.519910in}{1.894870in}}%
\pgfpathlineto{\pgfqpoint{2.521135in}{1.746341in}}%
\pgfpathlineto{\pgfqpoint{2.522156in}{1.936128in}}%
\pgfpathlineto{\pgfqpoint{2.522360in}{1.894870in}}%
\pgfpathlineto{\pgfqpoint{2.523177in}{1.746341in}}%
\pgfpathlineto{\pgfqpoint{2.523381in}{1.771096in}}%
\pgfpathlineto{\pgfqpoint{2.523993in}{1.927877in}}%
\pgfpathlineto{\pgfqpoint{2.524606in}{1.886619in}}%
\pgfpathlineto{\pgfqpoint{2.526035in}{1.993890in}}%
\pgfpathlineto{\pgfqpoint{2.526239in}{2.018644in}}%
\pgfpathlineto{\pgfqpoint{2.526647in}{1.977386in}}%
\pgfpathlineto{\pgfqpoint{2.527055in}{2.010393in}}%
\pgfpathlineto{\pgfqpoint{2.527668in}{1.919625in}}%
\pgfpathlineto{\pgfqpoint{2.528893in}{2.051651in}}%
\pgfpathlineto{\pgfqpoint{2.529097in}{2.043399in}}%
\pgfpathlineto{\pgfqpoint{2.529505in}{2.092909in}}%
\pgfpathlineto{\pgfqpoint{2.529913in}{2.035148in}}%
\pgfpathlineto{\pgfqpoint{2.530118in}{2.084657in}}%
\pgfpathlineto{\pgfqpoint{2.530526in}{2.035148in}}%
\pgfpathlineto{\pgfqpoint{2.530934in}{2.134167in}}%
\pgfpathlineto{\pgfqpoint{2.531955in}{2.076406in}}%
\pgfpathlineto{\pgfqpoint{2.532363in}{2.101160in}}%
\pgfpathlineto{\pgfqpoint{2.532567in}{2.125915in}}%
\pgfpathlineto{\pgfqpoint{2.532976in}{2.059902in}}%
\pgfpathlineto{\pgfqpoint{2.533180in}{2.068154in}}%
\pgfpathlineto{\pgfqpoint{2.533384in}{2.068154in}}%
\pgfpathlineto{\pgfqpoint{2.533792in}{2.117664in}}%
\pgfpathlineto{\pgfqpoint{2.534405in}{2.084657in}}%
\pgfpathlineto{\pgfqpoint{2.535017in}{2.002141in}}%
\pgfpathlineto{\pgfqpoint{2.535629in}{2.010393in}}%
\pgfpathlineto{\pgfqpoint{2.535834in}{2.068154in}}%
\pgfpathlineto{\pgfqpoint{2.536650in}{1.993890in}}%
\pgfpathlineto{\pgfqpoint{2.536854in}{2.043399in}}%
\pgfpathlineto{\pgfqpoint{2.537058in}{2.018644in}}%
\pgfpathlineto{\pgfqpoint{2.537467in}{2.084657in}}%
\pgfpathlineto{\pgfqpoint{2.537671in}{2.068154in}}%
\pgfpathlineto{\pgfqpoint{2.537875in}{2.076406in}}%
\pgfpathlineto{\pgfqpoint{2.538487in}{2.084657in}}%
\pgfpathlineto{\pgfqpoint{2.539304in}{1.977386in}}%
\pgfpathlineto{\pgfqpoint{2.539508in}{1.977386in}}%
\pgfpathlineto{\pgfqpoint{2.540121in}{1.936128in}}%
\pgfpathlineto{\pgfqpoint{2.540325in}{1.969135in}}%
\pgfpathlineto{\pgfqpoint{2.540529in}{2.010393in}}%
\pgfpathlineto{\pgfqpoint{2.540937in}{1.919625in}}%
\pgfpathlineto{\pgfqpoint{2.541345in}{1.960883in}}%
\pgfpathlineto{\pgfqpoint{2.542366in}{1.894870in}}%
\pgfpathlineto{\pgfqpoint{2.542162in}{1.969135in}}%
\pgfpathlineto{\pgfqpoint{2.542774in}{1.919625in}}%
\pgfpathlineto{\pgfqpoint{2.542979in}{1.919625in}}%
\pgfpathlineto{\pgfqpoint{2.544408in}{1.787599in}}%
\pgfpathlineto{\pgfqpoint{2.543387in}{1.936128in}}%
\pgfpathlineto{\pgfqpoint{2.544612in}{1.837109in}}%
\pgfpathlineto{\pgfqpoint{2.545020in}{1.870115in}}%
\pgfpathlineto{\pgfqpoint{2.545224in}{1.845361in}}%
\pgfpathlineto{\pgfqpoint{2.546245in}{1.729838in}}%
\pgfpathlineto{\pgfqpoint{2.546449in}{1.787599in}}%
\pgfpathlineto{\pgfqpoint{2.546653in}{1.787599in}}%
\pgfpathlineto{\pgfqpoint{2.548082in}{1.903122in}}%
\pgfpathlineto{\pgfqpoint{2.548286in}{1.812354in}}%
\pgfpathlineto{\pgfqpoint{2.548899in}{1.927877in}}%
\pgfpathlineto{\pgfqpoint{2.549103in}{1.870115in}}%
\pgfpathlineto{\pgfqpoint{2.549511in}{1.828857in}}%
\pgfpathlineto{\pgfqpoint{2.549919in}{1.853612in}}%
\pgfpathlineto{\pgfqpoint{2.550736in}{1.721587in}}%
\pgfpathlineto{\pgfqpoint{2.550940in}{1.828857in}}%
\pgfpathlineto{\pgfqpoint{2.551144in}{1.903122in}}%
\pgfpathlineto{\pgfqpoint{2.551757in}{1.746341in}}%
\pgfpathlineto{\pgfqpoint{2.552165in}{1.540051in}}%
\pgfpathlineto{\pgfqpoint{2.552573in}{1.606064in}}%
\pgfpathlineto{\pgfqpoint{2.552982in}{1.845361in}}%
\pgfpathlineto{\pgfqpoint{2.553798in}{1.804103in}}%
\pgfpathlineto{\pgfqpoint{2.554819in}{1.746341in}}%
\pgfpathlineto{\pgfqpoint{2.555431in}{1.853612in}}%
\pgfpathlineto{\pgfqpoint{2.556044in}{1.845361in}}%
\pgfpathlineto{\pgfqpoint{2.557881in}{1.688580in}}%
\pgfpathlineto{\pgfqpoint{2.556656in}{1.861864in}}%
\pgfpathlineto{\pgfqpoint{2.558085in}{1.729838in}}%
\pgfpathlineto{\pgfqpoint{2.558493in}{1.696832in}}%
\pgfpathlineto{\pgfqpoint{2.558902in}{1.746341in}}%
\pgfpathlineto{\pgfqpoint{2.559310in}{1.705083in}}%
\pgfpathlineto{\pgfqpoint{2.559514in}{1.729838in}}%
\pgfpathlineto{\pgfqpoint{2.560127in}{1.713335in}}%
\pgfpathlineto{\pgfqpoint{2.560331in}{1.655574in}}%
\pgfpathlineto{\pgfqpoint{2.560943in}{1.779348in}}%
\pgfpathlineto{\pgfqpoint{2.561147in}{1.738090in}}%
\pgfpathlineto{\pgfqpoint{2.561351in}{1.738090in}}%
\pgfpathlineto{\pgfqpoint{2.561556in}{1.771096in}}%
\pgfpathlineto{\pgfqpoint{2.562168in}{1.713335in}}%
\pgfpathlineto{\pgfqpoint{2.562372in}{1.738090in}}%
\pgfpathlineto{\pgfqpoint{2.563393in}{1.870115in}}%
\pgfpathlineto{\pgfqpoint{2.563597in}{1.795851in}}%
\pgfpathlineto{\pgfqpoint{2.563801in}{1.754593in}}%
\pgfpathlineto{\pgfqpoint{2.564414in}{1.812354in}}%
\pgfpathlineto{\pgfqpoint{2.564618in}{1.804103in}}%
\pgfpathlineto{\pgfqpoint{2.566455in}{1.993890in}}%
\pgfpathlineto{\pgfqpoint{2.566863in}{1.746341in}}%
\pgfpathlineto{\pgfqpoint{2.567476in}{2.002141in}}%
\pgfpathlineto{\pgfqpoint{2.567680in}{1.853612in}}%
\pgfpathlineto{\pgfqpoint{2.568088in}{1.985638in}}%
\pgfpathlineto{\pgfqpoint{2.568496in}{1.969135in}}%
\pgfpathlineto{\pgfqpoint{2.568701in}{1.746341in}}%
\pgfpathlineto{\pgfqpoint{2.569313in}{1.977386in}}%
\pgfpathlineto{\pgfqpoint{2.569721in}{1.795851in}}%
\pgfpathlineto{\pgfqpoint{2.570334in}{2.018644in}}%
\pgfpathlineto{\pgfqpoint{2.570946in}{1.936128in}}%
\pgfpathlineto{\pgfqpoint{2.571150in}{1.927877in}}%
\pgfpathlineto{\pgfqpoint{2.571354in}{1.944380in}}%
\pgfpathlineto{\pgfqpoint{2.571559in}{1.936128in}}%
\pgfpathlineto{\pgfqpoint{2.572171in}{2.018644in}}%
\pgfpathlineto{\pgfqpoint{2.572375in}{2.010393in}}%
\pgfpathlineto{\pgfqpoint{2.572988in}{1.853612in}}%
\pgfpathlineto{\pgfqpoint{2.573600in}{1.870115in}}%
\pgfpathlineto{\pgfqpoint{2.574212in}{1.886619in}}%
\pgfpathlineto{\pgfqpoint{2.574621in}{1.804103in}}%
\pgfpathlineto{\pgfqpoint{2.576050in}{1.944380in}}%
\pgfpathlineto{\pgfqpoint{2.577479in}{1.853612in}}%
\pgfpathlineto{\pgfqpoint{2.578091in}{1.903122in}}%
\pgfpathlineto{\pgfqpoint{2.578499in}{1.845361in}}%
\pgfpathlineto{\pgfqpoint{2.578704in}{1.630819in}}%
\pgfpathlineto{\pgfqpoint{2.579316in}{2.002141in}}%
\pgfpathlineto{\pgfqpoint{2.579724in}{1.705083in}}%
\pgfpathlineto{\pgfqpoint{2.580337in}{2.068154in}}%
\pgfpathlineto{\pgfqpoint{2.581153in}{1.977386in}}%
\pgfpathlineto{\pgfqpoint{2.581970in}{1.787599in}}%
\pgfpathlineto{\pgfqpoint{2.581766in}{2.035148in}}%
\pgfpathlineto{\pgfqpoint{2.582174in}{1.936128in}}%
\pgfpathlineto{\pgfqpoint{2.582378in}{1.960883in}}%
\pgfpathlineto{\pgfqpoint{2.582786in}{1.944380in}}%
\pgfpathlineto{\pgfqpoint{2.584215in}{1.771096in}}%
\pgfpathlineto{\pgfqpoint{2.585440in}{1.837109in}}%
\pgfpathlineto{\pgfqpoint{2.585644in}{1.853612in}}%
\pgfpathlineto{\pgfqpoint{2.585849in}{1.812354in}}%
\pgfpathlineto{\pgfqpoint{2.587073in}{1.606064in}}%
\pgfpathlineto{\pgfqpoint{2.587278in}{1.614316in}}%
\pgfpathlineto{\pgfqpoint{2.587686in}{1.878367in}}%
\pgfpathlineto{\pgfqpoint{2.588502in}{1.812354in}}%
\pgfpathlineto{\pgfqpoint{2.588911in}{1.837109in}}%
\pgfpathlineto{\pgfqpoint{2.589115in}{1.795851in}}%
\pgfpathlineto{\pgfqpoint{2.589523in}{1.812354in}}%
\pgfpathlineto{\pgfqpoint{2.589931in}{1.771096in}}%
\pgfpathlineto{\pgfqpoint{2.590136in}{1.820606in}}%
\pgfpathlineto{\pgfqpoint{2.590340in}{1.779348in}}%
\pgfpathlineto{\pgfqpoint{2.591360in}{1.886619in}}%
\pgfpathlineto{\pgfqpoint{2.591565in}{1.804103in}}%
\pgfpathlineto{\pgfqpoint{2.592585in}{1.820606in}}%
\pgfpathlineto{\pgfqpoint{2.592789in}{1.861864in}}%
\pgfpathlineto{\pgfqpoint{2.592994in}{1.804103in}}%
\pgfpathlineto{\pgfqpoint{2.593198in}{1.804103in}}%
\pgfpathlineto{\pgfqpoint{2.593606in}{1.696832in}}%
\pgfpathlineto{\pgfqpoint{2.594423in}{1.705083in}}%
\pgfpathlineto{\pgfqpoint{2.595239in}{1.911373in}}%
\pgfpathlineto{\pgfqpoint{2.595647in}{1.820606in}}%
\pgfpathlineto{\pgfqpoint{2.595852in}{1.804103in}}%
\pgfpathlineto{\pgfqpoint{2.596260in}{1.853612in}}%
\pgfpathlineto{\pgfqpoint{2.596464in}{1.845361in}}%
\pgfpathlineto{\pgfqpoint{2.596668in}{1.870115in}}%
\pgfpathlineto{\pgfqpoint{2.596872in}{1.886619in}}%
\pgfpathlineto{\pgfqpoint{2.597076in}{1.837109in}}%
\pgfpathlineto{\pgfqpoint{2.597689in}{1.614316in}}%
\pgfpathlineto{\pgfqpoint{2.597893in}{1.853612in}}%
\pgfpathlineto{\pgfqpoint{2.598097in}{1.837109in}}%
\pgfpathlineto{\pgfqpoint{2.598301in}{1.820606in}}%
\pgfpathlineto{\pgfqpoint{2.598505in}{1.861864in}}%
\pgfpathlineto{\pgfqpoint{2.599934in}{1.985638in}}%
\pgfpathlineto{\pgfqpoint{2.600751in}{1.919625in}}%
\pgfpathlineto{\pgfqpoint{2.601159in}{1.927877in}}%
\pgfpathlineto{\pgfqpoint{2.603201in}{1.754593in}}%
\pgfpathlineto{\pgfqpoint{2.603405in}{1.828857in}}%
\pgfpathlineto{\pgfqpoint{2.603609in}{1.639071in}}%
\pgfpathlineto{\pgfqpoint{2.604221in}{1.936128in}}%
\pgfpathlineto{\pgfqpoint{2.604425in}{1.911373in}}%
\pgfpathlineto{\pgfqpoint{2.604834in}{1.927877in}}%
\pgfpathlineto{\pgfqpoint{2.605242in}{1.861864in}}%
\pgfpathlineto{\pgfqpoint{2.605446in}{1.919625in}}%
\pgfpathlineto{\pgfqpoint{2.605854in}{1.820606in}}%
\pgfpathlineto{\pgfqpoint{2.606263in}{1.870115in}}%
\pgfpathlineto{\pgfqpoint{2.607488in}{1.779348in}}%
\pgfpathlineto{\pgfqpoint{2.607692in}{1.886619in}}%
\pgfpathlineto{\pgfqpoint{2.608712in}{1.853612in}}%
\pgfpathlineto{\pgfqpoint{2.608917in}{1.696832in}}%
\pgfpathlineto{\pgfqpoint{2.609529in}{1.820606in}}%
\pgfpathlineto{\pgfqpoint{2.609733in}{1.985638in}}%
\pgfpathlineto{\pgfqpoint{2.610550in}{1.870115in}}%
\pgfpathlineto{\pgfqpoint{2.611775in}{2.018644in}}%
\pgfpathlineto{\pgfqpoint{2.612183in}{1.960883in}}%
\pgfpathlineto{\pgfqpoint{2.613408in}{1.870115in}}%
\pgfpathlineto{\pgfqpoint{2.613816in}{1.861864in}}%
\pgfpathlineto{\pgfqpoint{2.614224in}{1.911373in}}%
\pgfpathlineto{\pgfqpoint{2.614428in}{1.795851in}}%
\pgfpathlineto{\pgfqpoint{2.615449in}{1.837109in}}%
\pgfpathlineto{\pgfqpoint{2.615653in}{1.936128in}}%
\pgfpathlineto{\pgfqpoint{2.616266in}{1.746341in}}%
\pgfpathlineto{\pgfqpoint{2.616470in}{1.696832in}}%
\pgfpathlineto{\pgfqpoint{2.617286in}{1.721587in}}%
\pgfpathlineto{\pgfqpoint{2.618103in}{1.762845in}}%
\pgfpathlineto{\pgfqpoint{2.618715in}{1.408026in}}%
\pgfpathlineto{\pgfqpoint{2.619328in}{1.606064in}}%
\pgfpathlineto{\pgfqpoint{2.620757in}{1.812354in}}%
\pgfpathlineto{\pgfqpoint{2.621165in}{1.870115in}}%
\pgfpathlineto{\pgfqpoint{2.621573in}{1.771096in}}%
\pgfpathlineto{\pgfqpoint{2.622594in}{1.713335in}}%
\pgfpathlineto{\pgfqpoint{2.622798in}{1.721587in}}%
\pgfpathlineto{\pgfqpoint{2.623002in}{1.564806in}}%
\pgfpathlineto{\pgfqpoint{2.623207in}{1.762845in}}%
\pgfpathlineto{\pgfqpoint{2.623819in}{1.696832in}}%
\pgfpathlineto{\pgfqpoint{2.624227in}{1.581309in}}%
\pgfpathlineto{\pgfqpoint{2.624636in}{1.630819in}}%
\pgfpathlineto{\pgfqpoint{2.624840in}{1.350264in}}%
\pgfpathlineto{\pgfqpoint{2.625452in}{1.713335in}}%
\pgfpathlineto{\pgfqpoint{2.625656in}{1.762845in}}%
\pgfpathlineto{\pgfqpoint{2.625860in}{1.663825in}}%
\pgfpathlineto{\pgfqpoint{2.626269in}{1.672077in}}%
\pgfpathlineto{\pgfqpoint{2.626473in}{1.680329in}}%
\pgfpathlineto{\pgfqpoint{2.626881in}{1.663825in}}%
\pgfpathlineto{\pgfqpoint{2.627085in}{1.655574in}}%
\pgfpathlineto{\pgfqpoint{2.627289in}{1.672077in}}%
\pgfpathlineto{\pgfqpoint{2.627494in}{1.663825in}}%
\pgfpathlineto{\pgfqpoint{2.628310in}{1.391522in}}%
\pgfpathlineto{\pgfqpoint{2.628514in}{1.457535in}}%
\pgfpathlineto{\pgfqpoint{2.629739in}{1.738090in}}%
\pgfpathlineto{\pgfqpoint{2.629943in}{1.729838in}}%
\pgfpathlineto{\pgfqpoint{2.630147in}{1.762845in}}%
\pgfpathlineto{\pgfqpoint{2.630352in}{1.762845in}}%
\pgfpathlineto{\pgfqpoint{2.630964in}{1.738090in}}%
\pgfpathlineto{\pgfqpoint{2.631372in}{1.762845in}}%
\pgfpathlineto{\pgfqpoint{2.632189in}{1.845361in}}%
\pgfpathlineto{\pgfqpoint{2.632393in}{1.721587in}}%
\pgfpathlineto{\pgfqpoint{2.633210in}{1.870115in}}%
\pgfpathlineto{\pgfqpoint{2.634230in}{1.878367in}}%
\pgfpathlineto{\pgfqpoint{2.634434in}{1.746341in}}%
\pgfpathlineto{\pgfqpoint{2.634843in}{1.936128in}}%
\pgfpathlineto{\pgfqpoint{2.635047in}{1.721587in}}%
\pgfpathlineto{\pgfqpoint{2.635455in}{1.828857in}}%
\pgfpathlineto{\pgfqpoint{2.636884in}{1.647322in}}%
\pgfpathlineto{\pgfqpoint{2.637088in}{1.696832in}}%
\pgfpathlineto{\pgfqpoint{2.637497in}{1.762845in}}%
\pgfpathlineto{\pgfqpoint{2.637905in}{1.680329in}}%
\pgfpathlineto{\pgfqpoint{2.638109in}{1.713335in}}%
\pgfpathlineto{\pgfqpoint{2.639130in}{1.597813in}}%
\pgfpathlineto{\pgfqpoint{2.639742in}{1.622567in}}%
\pgfpathlineto{\pgfqpoint{2.640763in}{1.589561in}}%
\pgfpathlineto{\pgfqpoint{2.640967in}{1.597813in}}%
\pgfpathlineto{\pgfqpoint{2.641784in}{1.556554in}}%
\pgfpathlineto{\pgfqpoint{2.642192in}{1.680329in}}%
\pgfpathlineto{\pgfqpoint{2.642396in}{1.597813in}}%
\pgfpathlineto{\pgfqpoint{2.643417in}{1.622567in}}%
\pgfpathlineto{\pgfqpoint{2.644029in}{1.606064in}}%
\pgfpathlineto{\pgfqpoint{2.644642in}{1.696832in}}%
\pgfpathlineto{\pgfqpoint{2.644846in}{1.663825in}}%
\pgfpathlineto{\pgfqpoint{2.645050in}{1.696832in}}%
\pgfpathlineto{\pgfqpoint{2.645254in}{1.787599in}}%
\pgfpathlineto{\pgfqpoint{2.645866in}{1.738090in}}%
\pgfpathlineto{\pgfqpoint{2.646275in}{1.639071in}}%
\pgfpathlineto{\pgfqpoint{2.647091in}{1.647322in}}%
\pgfpathlineto{\pgfqpoint{2.648520in}{1.540051in}}%
\pgfpathlineto{\pgfqpoint{2.649745in}{1.606064in}}%
\pgfpathlineto{\pgfqpoint{2.649949in}{1.606064in}}%
\pgfpathlineto{\pgfqpoint{2.650562in}{1.680329in}}%
\pgfpathlineto{\pgfqpoint{2.651174in}{1.639071in}}%
\pgfpathlineto{\pgfqpoint{2.651991in}{1.573058in}}%
\pgfpathlineto{\pgfqpoint{2.652195in}{1.647322in}}%
\pgfpathlineto{\pgfqpoint{2.652603in}{1.655574in}}%
\pgfpathlineto{\pgfqpoint{2.653011in}{1.573058in}}%
\pgfpathlineto{\pgfqpoint{2.654236in}{1.762845in}}%
\pgfpathlineto{\pgfqpoint{2.655869in}{1.209987in}}%
\pgfpathlineto{\pgfqpoint{2.656278in}{1.234742in}}%
\pgfpathlineto{\pgfqpoint{2.656890in}{1.457535in}}%
\pgfpathlineto{\pgfqpoint{2.657298in}{1.342013in}}%
\pgfpathlineto{\pgfqpoint{2.657707in}{1.284252in}}%
\pgfpathlineto{\pgfqpoint{2.658115in}{1.375019in}}%
\pgfpathlineto{\pgfqpoint{2.658932in}{1.507045in}}%
\pgfpathlineto{\pgfqpoint{2.659340in}{1.474038in}}%
\pgfpathlineto{\pgfqpoint{2.659748in}{1.309006in}}%
\pgfpathlineto{\pgfqpoint{2.660361in}{1.358516in}}%
\pgfpathlineto{\pgfqpoint{2.660565in}{1.465787in}}%
\pgfpathlineto{\pgfqpoint{2.661177in}{1.226490in}}%
\pgfpathlineto{\pgfqpoint{2.661994in}{1.259497in}}%
\pgfpathlineto{\pgfqpoint{2.662402in}{1.094465in}}%
\pgfpathlineto{\pgfqpoint{2.662810in}{1.251245in}}%
\pgfpathlineto{\pgfqpoint{2.663423in}{1.077961in}}%
\pgfpathlineto{\pgfqpoint{2.663627in}{1.176981in}}%
\pgfpathlineto{\pgfqpoint{2.664648in}{1.094465in}}%
\pgfpathlineto{\pgfqpoint{2.664852in}{1.135723in}}%
\pgfpathlineto{\pgfqpoint{2.665056in}{1.226490in}}%
\pgfpathlineto{\pgfqpoint{2.665872in}{1.168729in}}%
\pgfpathlineto{\pgfqpoint{2.666281in}{1.160477in}}%
\pgfpathlineto{\pgfqpoint{2.666485in}{1.193484in}}%
\pgfpathlineto{\pgfqpoint{2.667097in}{1.300755in}}%
\pgfpathlineto{\pgfqpoint{2.667301in}{1.242994in}}%
\pgfpathlineto{\pgfqpoint{2.668526in}{1.086213in}}%
\pgfpathlineto{\pgfqpoint{2.668730in}{1.094465in}}%
\pgfpathlineto{\pgfqpoint{2.669139in}{0.995445in}}%
\pgfpathlineto{\pgfqpoint{2.669343in}{1.069710in}}%
\pgfpathlineto{\pgfqpoint{2.669751in}{1.193484in}}%
\pgfpathlineto{\pgfqpoint{2.670159in}{1.036703in}}%
\pgfpathlineto{\pgfqpoint{2.670364in}{0.987194in}}%
\pgfpathlineto{\pgfqpoint{2.670772in}{1.077961in}}%
\pgfpathlineto{\pgfqpoint{2.670976in}{1.160477in}}%
\pgfpathlineto{\pgfqpoint{2.671384in}{1.061458in}}%
\pgfpathlineto{\pgfqpoint{2.671793in}{1.094465in}}%
\pgfpathlineto{\pgfqpoint{2.672201in}{1.044955in}}%
\pgfpathlineto{\pgfqpoint{2.672405in}{1.086213in}}%
\pgfpathlineto{\pgfqpoint{2.672609in}{1.011949in}}%
\pgfpathlineto{\pgfqpoint{2.673017in}{1.135723in}}%
\pgfpathlineto{\pgfqpoint{2.673222in}{1.110968in}}%
\pgfpathlineto{\pgfqpoint{2.673834in}{1.218239in}}%
\pgfpathlineto{\pgfqpoint{2.674446in}{1.160477in}}%
\pgfpathlineto{\pgfqpoint{2.674855in}{1.176981in}}%
\pgfpathlineto{\pgfqpoint{2.675263in}{1.135723in}}%
\pgfpathlineto{\pgfqpoint{2.675875in}{1.152226in}}%
\pgfpathlineto{\pgfqpoint{2.676080in}{1.160477in}}%
\pgfpathlineto{\pgfqpoint{2.676284in}{1.102716in}}%
\pgfpathlineto{\pgfqpoint{2.676896in}{1.168729in}}%
\pgfpathlineto{\pgfqpoint{2.677100in}{1.152226in}}%
\pgfpathlineto{\pgfqpoint{2.679958in}{1.441032in}}%
\pgfpathlineto{\pgfqpoint{2.680367in}{1.416277in}}%
\pgfpathlineto{\pgfqpoint{2.680775in}{1.284252in}}%
\pgfpathlineto{\pgfqpoint{2.681591in}{1.309006in}}%
\pgfpathlineto{\pgfqpoint{2.681796in}{1.375019in}}%
\pgfpathlineto{\pgfqpoint{2.682204in}{1.242994in}}%
\pgfpathlineto{\pgfqpoint{2.682408in}{1.242994in}}%
\pgfpathlineto{\pgfqpoint{2.684245in}{1.523548in}}%
\pgfpathlineto{\pgfqpoint{2.682816in}{1.226490in}}%
\pgfpathlineto{\pgfqpoint{2.684653in}{1.383271in}}%
\pgfpathlineto{\pgfqpoint{2.685878in}{1.251245in}}%
\pgfpathlineto{\pgfqpoint{2.687511in}{1.424529in}}%
\pgfpathlineto{\pgfqpoint{2.688940in}{1.135723in}}%
\pgfpathlineto{\pgfqpoint{2.689349in}{1.143974in}}%
\pgfpathlineto{\pgfqpoint{2.689757in}{1.110968in}}%
\pgfpathlineto{\pgfqpoint{2.690574in}{1.242994in}}%
\pgfpathlineto{\pgfqpoint{2.690778in}{1.185232in}}%
\pgfpathlineto{\pgfqpoint{2.690982in}{1.135723in}}%
\pgfpathlineto{\pgfqpoint{2.691186in}{1.300755in}}%
\pgfpathlineto{\pgfqpoint{2.691798in}{1.474038in}}%
\pgfpathlineto{\pgfqpoint{2.692207in}{1.284252in}}%
\pgfpathlineto{\pgfqpoint{2.692411in}{1.284252in}}%
\pgfpathlineto{\pgfqpoint{2.693023in}{1.358516in}}%
\pgfpathlineto{\pgfqpoint{2.693432in}{1.292503in}}%
\pgfpathlineto{\pgfqpoint{2.694248in}{1.325510in}}%
\pgfpathlineto{\pgfqpoint{2.694656in}{1.201736in}}%
\pgfpathlineto{\pgfqpoint{2.694861in}{1.201736in}}%
\pgfpathlineto{\pgfqpoint{2.695065in}{1.003697in}}%
\pgfpathlineto{\pgfqpoint{2.695269in}{1.209987in}}%
\pgfpathlineto{\pgfqpoint{2.696085in}{1.077961in}}%
\pgfpathlineto{\pgfqpoint{2.696698in}{1.317258in}}%
\pgfpathlineto{\pgfqpoint{2.697310in}{1.185232in}}%
\pgfpathlineto{\pgfqpoint{2.698331in}{1.110968in}}%
\pgfpathlineto{\pgfqpoint{2.699556in}{1.267748in}}%
\pgfpathlineto{\pgfqpoint{2.699760in}{1.251245in}}%
\pgfpathlineto{\pgfqpoint{2.700372in}{1.259497in}}%
\pgfpathlineto{\pgfqpoint{2.702414in}{1.589561in}}%
\pgfpathlineto{\pgfqpoint{2.702618in}{1.597813in}}%
\pgfpathlineto{\pgfqpoint{2.702822in}{1.564806in}}%
\pgfpathlineto{\pgfqpoint{2.704251in}{1.424529in}}%
\pgfpathlineto{\pgfqpoint{2.704455in}{1.432780in}}%
\pgfpathlineto{\pgfqpoint{2.706497in}{1.589561in}}%
\pgfpathlineto{\pgfqpoint{2.704864in}{1.424529in}}%
\pgfpathlineto{\pgfqpoint{2.706701in}{1.564806in}}%
\pgfpathlineto{\pgfqpoint{2.706905in}{1.540051in}}%
\pgfpathlineto{\pgfqpoint{2.707313in}{1.597813in}}%
\pgfpathlineto{\pgfqpoint{2.707517in}{1.581309in}}%
\pgfpathlineto{\pgfqpoint{2.707722in}{1.630819in}}%
\pgfpathlineto{\pgfqpoint{2.708130in}{1.540051in}}%
\pgfpathlineto{\pgfqpoint{2.708334in}{1.556554in}}%
\pgfpathlineto{\pgfqpoint{2.708538in}{1.531800in}}%
\pgfpathlineto{\pgfqpoint{2.708742in}{1.556554in}}%
\pgfpathlineto{\pgfqpoint{2.709355in}{1.663825in}}%
\pgfpathlineto{\pgfqpoint{2.709967in}{1.647322in}}%
\pgfpathlineto{\pgfqpoint{2.710171in}{1.655574in}}%
\pgfpathlineto{\pgfqpoint{2.710375in}{1.647322in}}%
\pgfpathlineto{\pgfqpoint{2.710784in}{1.606064in}}%
\pgfpathlineto{\pgfqpoint{2.710988in}{1.696832in}}%
\pgfpathlineto{\pgfqpoint{2.711192in}{1.746341in}}%
\pgfpathlineto{\pgfqpoint{2.711396in}{1.647322in}}%
\pgfpathlineto{\pgfqpoint{2.711600in}{1.647322in}}%
\pgfpathlineto{\pgfqpoint{2.712009in}{1.606064in}}%
\pgfpathlineto{\pgfqpoint{2.712417in}{1.647322in}}%
\pgfpathlineto{\pgfqpoint{2.713642in}{1.721587in}}%
\pgfpathlineto{\pgfqpoint{2.713846in}{1.680329in}}%
\pgfpathlineto{\pgfqpoint{2.714050in}{1.655574in}}%
\pgfpathlineto{\pgfqpoint{2.714662in}{1.713335in}}%
\pgfpathlineto{\pgfqpoint{2.715683in}{1.762845in}}%
\pgfpathlineto{\pgfqpoint{2.715887in}{1.754593in}}%
\pgfpathlineto{\pgfqpoint{2.716296in}{1.812354in}}%
\pgfpathlineto{\pgfqpoint{2.716704in}{1.787599in}}%
\pgfpathlineto{\pgfqpoint{2.716908in}{1.713335in}}%
\pgfpathlineto{\pgfqpoint{2.717520in}{1.886619in}}%
\pgfpathlineto{\pgfqpoint{2.717929in}{1.894870in}}%
\pgfpathlineto{\pgfqpoint{2.718133in}{1.861864in}}%
\pgfpathlineto{\pgfqpoint{2.718949in}{1.944380in}}%
\pgfpathlineto{\pgfqpoint{2.719154in}{1.894870in}}%
\pgfpathlineto{\pgfqpoint{2.719766in}{1.936128in}}%
\pgfpathlineto{\pgfqpoint{2.720174in}{1.870115in}}%
\pgfpathlineto{\pgfqpoint{2.721399in}{2.035148in}}%
\pgfpathlineto{\pgfqpoint{2.722216in}{1.985638in}}%
\pgfpathlineto{\pgfqpoint{2.721807in}{2.043399in}}%
\pgfpathlineto{\pgfqpoint{2.722420in}{2.010393in}}%
\pgfpathlineto{\pgfqpoint{2.724257in}{2.142418in}}%
\pgfpathlineto{\pgfqpoint{2.724461in}{2.134167in}}%
\pgfpathlineto{\pgfqpoint{2.725074in}{2.109412in}}%
\pgfpathlineto{\pgfqpoint{2.724870in}{2.191928in}}%
\pgfpathlineto{\pgfqpoint{2.725278in}{2.167173in}}%
\pgfpathlineto{\pgfqpoint{2.726094in}{2.117664in}}%
\pgfpathlineto{\pgfqpoint{2.727319in}{2.249689in}}%
\pgfpathlineto{\pgfqpoint{2.727728in}{2.216683in}}%
\pgfpathlineto{\pgfqpoint{2.728748in}{2.092909in}}%
\pgfpathlineto{\pgfqpoint{2.729769in}{2.191928in}}%
\pgfpathlineto{\pgfqpoint{2.729973in}{2.175425in}}%
\pgfpathlineto{\pgfqpoint{2.730381in}{2.142418in}}%
\pgfpathlineto{\pgfqpoint{2.730586in}{2.191928in}}%
\pgfpathlineto{\pgfqpoint{2.730994in}{2.175425in}}%
\pgfpathlineto{\pgfqpoint{2.731198in}{2.249689in}}%
\pgfpathlineto{\pgfqpoint{2.731810in}{2.208431in}}%
\pgfpathlineto{\pgfqpoint{2.732015in}{2.125915in}}%
\pgfpathlineto{\pgfqpoint{2.732219in}{2.274444in}}%
\pgfpathlineto{\pgfqpoint{2.733035in}{2.142418in}}%
\pgfpathlineto{\pgfqpoint{2.733239in}{2.150670in}}%
\pgfpathlineto{\pgfqpoint{2.733444in}{2.142418in}}%
\pgfpathlineto{\pgfqpoint{2.733648in}{2.117664in}}%
\pgfpathlineto{\pgfqpoint{2.734056in}{2.158922in}}%
\pgfpathlineto{\pgfqpoint{2.734464in}{2.150670in}}%
\pgfpathlineto{\pgfqpoint{2.734668in}{2.216683in}}%
\pgfpathlineto{\pgfqpoint{2.735485in}{2.142418in}}%
\pgfpathlineto{\pgfqpoint{2.735689in}{2.134167in}}%
\pgfpathlineto{\pgfqpoint{2.736710in}{2.216683in}}%
\pgfpathlineto{\pgfqpoint{2.736914in}{1.993890in}}%
\pgfpathlineto{\pgfqpoint{2.737731in}{2.158922in}}%
\pgfpathlineto{\pgfqpoint{2.737935in}{2.158922in}}%
\pgfpathlineto{\pgfqpoint{2.738139in}{2.125915in}}%
\pgfpathlineto{\pgfqpoint{2.738751in}{2.158922in}}%
\pgfpathlineto{\pgfqpoint{2.740180in}{2.323954in}}%
\pgfpathlineto{\pgfqpoint{2.740384in}{2.290947in}}%
\pgfpathlineto{\pgfqpoint{2.740997in}{2.257941in}}%
\pgfpathlineto{\pgfqpoint{2.741201in}{2.290947in}}%
\pgfpathlineto{\pgfqpoint{2.741609in}{2.323954in}}%
\pgfpathlineto{\pgfqpoint{2.741813in}{2.233186in}}%
\pgfpathlineto{\pgfqpoint{2.742630in}{2.332205in}}%
\pgfpathlineto{\pgfqpoint{2.742834in}{2.241438in}}%
\pgfpathlineto{\pgfqpoint{2.744467in}{2.365212in}}%
\pgfpathlineto{\pgfqpoint{2.744876in}{2.381715in}}%
\pgfpathlineto{\pgfqpoint{2.745080in}{2.373463in}}%
\pgfpathlineto{\pgfqpoint{2.745488in}{2.150670in}}%
\pgfpathlineto{\pgfqpoint{2.746305in}{2.282696in}}%
\pgfpathlineto{\pgfqpoint{2.746509in}{2.315702in}}%
\pgfpathlineto{\pgfqpoint{2.746917in}{2.274444in}}%
\pgfpathlineto{\pgfqpoint{2.747325in}{1.985638in}}%
\pgfpathlineto{\pgfqpoint{2.747938in}{2.216683in}}%
\pgfpathlineto{\pgfqpoint{2.748754in}{2.348709in}}%
\pgfpathlineto{\pgfqpoint{2.748958in}{2.274444in}}%
\pgfpathlineto{\pgfqpoint{2.749367in}{2.092909in}}%
\pgfpathlineto{\pgfqpoint{2.750183in}{2.224934in}}%
\pgfpathlineto{\pgfqpoint{2.751408in}{2.365212in}}%
\pgfpathlineto{\pgfqpoint{2.752633in}{2.233186in}}%
\pgfpathlineto{\pgfqpoint{2.753245in}{2.332205in}}%
\pgfpathlineto{\pgfqpoint{2.753858in}{2.315702in}}%
\pgfpathlineto{\pgfqpoint{2.754470in}{2.233186in}}%
\pgfpathlineto{\pgfqpoint{2.755083in}{2.249689in}}%
\pgfpathlineto{\pgfqpoint{2.755695in}{2.348709in}}%
\pgfpathlineto{\pgfqpoint{2.756103in}{2.233186in}}%
\pgfpathlineto{\pgfqpoint{2.756716in}{2.307450in}}%
\pgfpathlineto{\pgfqpoint{2.756920in}{2.175425in}}%
\pgfpathlineto{\pgfqpoint{2.757532in}{2.323954in}}%
\pgfpathlineto{\pgfqpoint{2.757737in}{2.299199in}}%
\pgfpathlineto{\pgfqpoint{2.758145in}{2.299199in}}%
\pgfpathlineto{\pgfqpoint{2.758961in}{2.365212in}}%
\pgfpathlineto{\pgfqpoint{2.759574in}{2.340457in}}%
\pgfpathlineto{\pgfqpoint{2.760186in}{2.282696in}}%
\pgfpathlineto{\pgfqpoint{2.760390in}{2.323954in}}%
\pgfpathlineto{\pgfqpoint{2.761003in}{2.307450in}}%
\pgfpathlineto{\pgfqpoint{2.761615in}{2.406470in}}%
\pgfpathlineto{\pgfqpoint{2.763248in}{2.266192in}}%
\pgfpathlineto{\pgfqpoint{2.764065in}{2.365212in}}%
\pgfpathlineto{\pgfqpoint{2.764269in}{2.266192in}}%
\pgfpathlineto{\pgfqpoint{2.764473in}{2.274444in}}%
\pgfpathlineto{\pgfqpoint{2.764677in}{2.266192in}}%
\pgfpathlineto{\pgfqpoint{2.764882in}{2.241438in}}%
\pgfpathlineto{\pgfqpoint{2.765290in}{2.315702in}}%
\pgfpathlineto{\pgfqpoint{2.765902in}{2.356960in}}%
\pgfpathlineto{\pgfqpoint{2.766106in}{2.299199in}}%
\pgfpathlineto{\pgfqpoint{2.766311in}{2.323954in}}%
\pgfpathlineto{\pgfqpoint{2.766515in}{2.282696in}}%
\pgfpathlineto{\pgfqpoint{2.766923in}{2.373463in}}%
\pgfpathlineto{\pgfqpoint{2.767331in}{2.340457in}}%
\pgfpathlineto{\pgfqpoint{2.767535in}{2.389967in}}%
\pgfpathlineto{\pgfqpoint{2.768148in}{2.332205in}}%
\pgfpathlineto{\pgfqpoint{2.768760in}{2.282696in}}%
\pgfpathlineto{\pgfqpoint{2.769168in}{2.323954in}}%
\pgfpathlineto{\pgfqpoint{2.769373in}{2.348709in}}%
\pgfpathlineto{\pgfqpoint{2.769577in}{2.299199in}}%
\pgfpathlineto{\pgfqpoint{2.769781in}{2.307450in}}%
\pgfpathlineto{\pgfqpoint{2.770802in}{2.208431in}}%
\pgfpathlineto{\pgfqpoint{2.771414in}{2.224934in}}%
\pgfpathlineto{\pgfqpoint{2.772026in}{2.373463in}}%
\pgfpathlineto{\pgfqpoint{2.772843in}{2.332205in}}%
\pgfpathlineto{\pgfqpoint{2.773864in}{2.299199in}}%
\pgfpathlineto{\pgfqpoint{2.774068in}{2.307450in}}%
\pgfpathlineto{\pgfqpoint{2.774680in}{2.381715in}}%
\pgfpathlineto{\pgfqpoint{2.774476in}{2.299199in}}%
\pgfpathlineto{\pgfqpoint{2.775497in}{2.365212in}}%
\pgfpathlineto{\pgfqpoint{2.775701in}{2.365212in}}%
\pgfpathlineto{\pgfqpoint{2.776109in}{2.373463in}}%
\pgfpathlineto{\pgfqpoint{2.777130in}{2.282696in}}%
\pgfpathlineto{\pgfqpoint{2.777538in}{2.373463in}}%
\pgfpathlineto{\pgfqpoint{2.778151in}{2.356960in}}%
\pgfpathlineto{\pgfqpoint{2.779580in}{2.200180in}}%
\pgfpathlineto{\pgfqpoint{2.779784in}{2.233186in}}%
\pgfpathlineto{\pgfqpoint{2.780396in}{2.282696in}}%
\pgfpathlineto{\pgfqpoint{2.780192in}{2.216683in}}%
\pgfpathlineto{\pgfqpoint{2.780600in}{2.257941in}}%
\pgfpathlineto{\pgfqpoint{2.781417in}{2.208431in}}%
\pgfpathlineto{\pgfqpoint{2.781621in}{2.266192in}}%
\pgfpathlineto{\pgfqpoint{2.782234in}{2.299199in}}%
\pgfpathlineto{\pgfqpoint{2.782029in}{2.257941in}}%
\pgfpathlineto{\pgfqpoint{2.782642in}{2.266192in}}%
\pgfpathlineto{\pgfqpoint{2.782846in}{2.266192in}}%
\pgfpathlineto{\pgfqpoint{2.783050in}{2.216683in}}%
\pgfpathlineto{\pgfqpoint{2.783867in}{2.282696in}}%
\pgfpathlineto{\pgfqpoint{2.784071in}{2.274444in}}%
\pgfpathlineto{\pgfqpoint{2.784275in}{2.200180in}}%
\pgfpathlineto{\pgfqpoint{2.784887in}{2.249689in}}%
\pgfpathlineto{\pgfqpoint{2.785092in}{2.323954in}}%
\pgfpathlineto{\pgfqpoint{2.785908in}{2.233186in}}%
\pgfpathlineto{\pgfqpoint{2.786112in}{2.290947in}}%
\pgfpathlineto{\pgfqpoint{2.786929in}{2.282696in}}%
\pgfpathlineto{\pgfqpoint{2.787133in}{2.233186in}}%
\pgfpathlineto{\pgfqpoint{2.787745in}{2.348709in}}%
\pgfpathlineto{\pgfqpoint{2.788562in}{2.406470in}}%
\pgfpathlineto{\pgfqpoint{2.788154in}{2.340457in}}%
\pgfpathlineto{\pgfqpoint{2.789174in}{2.389967in}}%
\pgfpathlineto{\pgfqpoint{2.789991in}{2.266192in}}%
\pgfpathlineto{\pgfqpoint{2.790195in}{2.307450in}}%
\pgfpathlineto{\pgfqpoint{2.790808in}{2.373463in}}%
\pgfpathlineto{\pgfqpoint{2.791012in}{2.299199in}}%
\pgfpathlineto{\pgfqpoint{2.791420in}{2.282696in}}%
\pgfpathlineto{\pgfqpoint{2.791624in}{2.323954in}}%
\pgfpathlineto{\pgfqpoint{2.791828in}{2.315702in}}%
\pgfpathlineto{\pgfqpoint{2.792645in}{2.406470in}}%
\pgfpathlineto{\pgfqpoint{2.792849in}{2.356960in}}%
\pgfpathlineto{\pgfqpoint{2.793461in}{2.282696in}}%
\pgfpathlineto{\pgfqpoint{2.793870in}{2.332205in}}%
\pgfpathlineto{\pgfqpoint{2.794890in}{2.373463in}}%
\pgfpathlineto{\pgfqpoint{2.796524in}{2.109412in}}%
\pgfpathlineto{\pgfqpoint{2.797748in}{2.414721in}}%
\pgfpathlineto{\pgfqpoint{2.798973in}{2.299199in}}%
\pgfpathlineto{\pgfqpoint{2.799177in}{2.389967in}}%
\pgfpathlineto{\pgfqpoint{2.799994in}{2.381715in}}%
\pgfpathlineto{\pgfqpoint{2.800198in}{2.332205in}}%
\pgfpathlineto{\pgfqpoint{2.801015in}{2.356960in}}%
\pgfpathlineto{\pgfqpoint{2.801627in}{2.406470in}}%
\pgfpathlineto{\pgfqpoint{2.802035in}{2.398218in}}%
\pgfpathlineto{\pgfqpoint{2.802852in}{2.282696in}}%
\pgfpathlineto{\pgfqpoint{2.802648in}{2.406470in}}%
\pgfpathlineto{\pgfqpoint{2.803056in}{2.398218in}}%
\pgfpathlineto{\pgfqpoint{2.803260in}{2.389967in}}%
\pgfpathlineto{\pgfqpoint{2.803464in}{2.455979in}}%
\pgfpathlineto{\pgfqpoint{2.803873in}{2.381715in}}%
\pgfpathlineto{\pgfqpoint{2.804077in}{2.381715in}}%
\pgfpathlineto{\pgfqpoint{2.804689in}{2.431225in}}%
\pgfpathlineto{\pgfqpoint{2.805302in}{2.332205in}}%
\pgfpathlineto{\pgfqpoint{2.806322in}{2.389967in}}%
\pgfpathlineto{\pgfqpoint{2.807547in}{2.125915in}}%
\pgfpathlineto{\pgfqpoint{2.806935in}{2.447728in}}%
\pgfpathlineto{\pgfqpoint{2.807751in}{2.323954in}}%
\pgfpathlineto{\pgfqpoint{2.808160in}{2.332205in}}%
\pgfpathlineto{\pgfqpoint{2.808364in}{2.307450in}}%
\pgfpathlineto{\pgfqpoint{2.808568in}{2.142418in}}%
\pgfpathlineto{\pgfqpoint{2.809385in}{2.332205in}}%
\pgfpathlineto{\pgfqpoint{2.809793in}{2.398218in}}%
\pgfpathlineto{\pgfqpoint{2.810201in}{2.315702in}}%
\pgfpathlineto{\pgfqpoint{2.810405in}{2.356960in}}%
\pgfpathlineto{\pgfqpoint{2.811834in}{2.257941in}}%
\pgfpathlineto{\pgfqpoint{2.812038in}{2.282696in}}%
\pgfpathlineto{\pgfqpoint{2.813263in}{2.447728in}}%
\pgfpathlineto{\pgfqpoint{2.814080in}{2.332205in}}%
\pgfpathlineto{\pgfqpoint{2.814488in}{2.356960in}}%
\pgfpathlineto{\pgfqpoint{2.814692in}{2.373463in}}%
\pgfpathlineto{\pgfqpoint{2.814896in}{2.332205in}}%
\pgfpathlineto{\pgfqpoint{2.815101in}{2.332205in}}%
\pgfpathlineto{\pgfqpoint{2.816325in}{2.101160in}}%
\pgfpathlineto{\pgfqpoint{2.815509in}{2.373463in}}%
\pgfpathlineto{\pgfqpoint{2.816530in}{2.233186in}}%
\pgfpathlineto{\pgfqpoint{2.816938in}{2.406470in}}%
\pgfpathlineto{\pgfqpoint{2.817550in}{2.257941in}}%
\pgfpathlineto{\pgfqpoint{2.818163in}{2.422973in}}%
\pgfpathlineto{\pgfqpoint{2.818775in}{2.332205in}}%
\pgfpathlineto{\pgfqpoint{2.818979in}{2.315702in}}%
\pgfpathlineto{\pgfqpoint{2.819183in}{2.365212in}}%
\pgfpathlineto{\pgfqpoint{2.819388in}{2.348709in}}%
\pgfpathlineto{\pgfqpoint{2.820000in}{2.323954in}}%
\pgfpathlineto{\pgfqpoint{2.820204in}{2.365212in}}%
\pgfpathlineto{\pgfqpoint{2.821633in}{2.208431in}}%
\pgfpathlineto{\pgfqpoint{2.822246in}{2.497237in}}%
\pgfpathlineto{\pgfqpoint{2.823062in}{2.356960in}}%
\pgfpathlineto{\pgfqpoint{2.823266in}{2.365212in}}%
\pgfpathlineto{\pgfqpoint{2.823470in}{2.134167in}}%
\pgfpathlineto{\pgfqpoint{2.823879in}{2.439476in}}%
\pgfpathlineto{\pgfqpoint{2.824287in}{2.406470in}}%
\pgfpathlineto{\pgfqpoint{2.824899in}{2.488986in}}%
\pgfpathlineto{\pgfqpoint{2.825308in}{2.455979in}}%
\pgfpathlineto{\pgfqpoint{2.825920in}{2.142418in}}%
\pgfpathlineto{\pgfqpoint{2.826328in}{2.340457in}}%
\pgfpathlineto{\pgfqpoint{2.826737in}{2.274444in}}%
\pgfpathlineto{\pgfqpoint{2.827145in}{2.315702in}}%
\pgfpathlineto{\pgfqpoint{2.827962in}{2.389967in}}%
\pgfpathlineto{\pgfqpoint{2.828370in}{2.381715in}}%
\pgfpathlineto{\pgfqpoint{2.828982in}{2.414721in}}%
\pgfpathlineto{\pgfqpoint{2.829186in}{2.266192in}}%
\pgfpathlineto{\pgfqpoint{2.830207in}{2.546747in}}%
\pgfpathlineto{\pgfqpoint{2.830411in}{2.455979in}}%
\pgfpathlineto{\pgfqpoint{2.831228in}{2.621011in}}%
\pgfpathlineto{\pgfqpoint{2.832044in}{2.579753in}}%
\pgfpathlineto{\pgfqpoint{2.832249in}{2.530244in}}%
\pgfpathlineto{\pgfqpoint{2.833269in}{2.538495in}}%
\pgfpathlineto{\pgfqpoint{2.833473in}{2.554999in}}%
\pgfpathlineto{\pgfqpoint{2.833678in}{2.521992in}}%
\pgfpathlineto{\pgfqpoint{2.833882in}{2.521992in}}%
\pgfpathlineto{\pgfqpoint{2.834086in}{2.447728in}}%
\pgfpathlineto{\pgfqpoint{2.834698in}{2.554999in}}%
\pgfpathlineto{\pgfqpoint{2.835107in}{2.455979in}}%
\pgfpathlineto{\pgfqpoint{2.835311in}{2.472483in}}%
\pgfpathlineto{\pgfqpoint{2.835515in}{2.398218in}}%
\pgfpathlineto{\pgfqpoint{2.835719in}{2.513741in}}%
\pgfpathlineto{\pgfqpoint{2.836331in}{2.455979in}}%
\pgfpathlineto{\pgfqpoint{2.836536in}{2.513741in}}%
\pgfpathlineto{\pgfqpoint{2.836740in}{2.282696in}}%
\pgfpathlineto{\pgfqpoint{2.837148in}{2.472483in}}%
\pgfpathlineto{\pgfqpoint{2.837352in}{2.406470in}}%
\pgfpathlineto{\pgfqpoint{2.837760in}{2.480734in}}%
\pgfpathlineto{\pgfqpoint{2.838373in}{2.414721in}}%
\pgfpathlineto{\pgfqpoint{2.839802in}{2.538495in}}%
\pgfpathlineto{\pgfqpoint{2.841027in}{2.414721in}}%
\pgfpathlineto{\pgfqpoint{2.841435in}{2.398218in}}%
\pgfpathlineto{\pgfqpoint{2.842252in}{2.464231in}}%
\pgfpathlineto{\pgfqpoint{2.842456in}{2.422973in}}%
\pgfpathlineto{\pgfqpoint{2.842864in}{2.497237in}}%
\pgfpathlineto{\pgfqpoint{2.843068in}{2.488986in}}%
\pgfpathlineto{\pgfqpoint{2.843272in}{2.497237in}}%
\pgfpathlineto{\pgfqpoint{2.843476in}{2.563250in}}%
\pgfpathlineto{\pgfqpoint{2.844089in}{2.455979in}}%
\pgfpathlineto{\pgfqpoint{2.844293in}{2.497237in}}%
\pgfpathlineto{\pgfqpoint{2.844497in}{2.431225in}}%
\pgfpathlineto{\pgfqpoint{2.845314in}{2.521992in}}%
\pgfpathlineto{\pgfqpoint{2.845518in}{2.579753in}}%
\pgfpathlineto{\pgfqpoint{2.846334in}{2.538495in}}%
\pgfpathlineto{\pgfqpoint{2.846743in}{2.588005in}}%
\pgfpathlineto{\pgfqpoint{2.846947in}{2.513741in}}%
\pgfpathlineto{\pgfqpoint{2.847763in}{2.563250in}}%
\pgfpathlineto{\pgfqpoint{2.847968in}{2.579753in}}%
\pgfpathlineto{\pgfqpoint{2.848172in}{2.530244in}}%
\pgfpathlineto{\pgfqpoint{2.848580in}{2.571502in}}%
\pgfpathlineto{\pgfqpoint{2.848784in}{2.530244in}}%
\pgfpathlineto{\pgfqpoint{2.849192in}{2.588005in}}%
\pgfpathlineto{\pgfqpoint{2.849601in}{2.563250in}}%
\pgfpathlineto{\pgfqpoint{2.849805in}{2.579753in}}%
\pgfpathlineto{\pgfqpoint{2.850009in}{2.546747in}}%
\pgfpathlineto{\pgfqpoint{2.850621in}{2.563250in}}%
\pgfpathlineto{\pgfqpoint{2.851234in}{2.488986in}}%
\pgfpathlineto{\pgfqpoint{2.851846in}{2.530244in}}%
\pgfpathlineto{\pgfqpoint{2.852867in}{2.480734in}}%
\pgfpathlineto{\pgfqpoint{2.853683in}{2.406470in}}%
\pgfpathlineto{\pgfqpoint{2.853888in}{2.455979in}}%
\pgfpathlineto{\pgfqpoint{2.854500in}{2.563250in}}%
\pgfpathlineto{\pgfqpoint{2.854704in}{2.447728in}}%
\pgfpathlineto{\pgfqpoint{2.855521in}{2.521992in}}%
\pgfpathlineto{\pgfqpoint{2.856337in}{2.447728in}}%
\pgfpathlineto{\pgfqpoint{2.855929in}{2.546747in}}%
\pgfpathlineto{\pgfqpoint{2.856746in}{2.464231in}}%
\pgfpathlineto{\pgfqpoint{2.857154in}{2.538495in}}%
\pgfpathlineto{\pgfqpoint{2.857358in}{2.431225in}}%
\pgfpathlineto{\pgfqpoint{2.857562in}{2.447728in}}%
\pgfpathlineto{\pgfqpoint{2.858175in}{2.480734in}}%
\pgfpathlineto{\pgfqpoint{2.858787in}{2.381715in}}%
\pgfpathlineto{\pgfqpoint{2.859604in}{2.414721in}}%
\pgfpathlineto{\pgfqpoint{2.859808in}{2.340457in}}%
\pgfpathlineto{\pgfqpoint{2.860420in}{2.505489in}}%
\pgfpathlineto{\pgfqpoint{2.860624in}{2.472483in}}%
\pgfpathlineto{\pgfqpoint{2.861033in}{2.538495in}}%
\pgfpathlineto{\pgfqpoint{2.861237in}{2.554999in}}%
\pgfpathlineto{\pgfqpoint{2.861645in}{2.282696in}}%
\pgfpathlineto{\pgfqpoint{2.862257in}{2.299199in}}%
\pgfpathlineto{\pgfqpoint{2.862462in}{2.546747in}}%
\pgfpathlineto{\pgfqpoint{2.863074in}{2.290947in}}%
\pgfpathlineto{\pgfqpoint{2.863278in}{2.290947in}}%
\pgfpathlineto{\pgfqpoint{2.864911in}{2.546747in}}%
\pgfpathlineto{\pgfqpoint{2.863686in}{2.257941in}}%
\pgfpathlineto{\pgfqpoint{2.865115in}{2.530244in}}%
\pgfpathlineto{\pgfqpoint{2.866136in}{2.422973in}}%
\pgfpathlineto{\pgfqpoint{2.866340in}{2.464231in}}%
\pgfpathlineto{\pgfqpoint{2.866953in}{2.480734in}}%
\pgfpathlineto{\pgfqpoint{2.867565in}{2.348709in}}%
\pgfpathlineto{\pgfqpoint{2.868790in}{2.521992in}}%
\pgfpathlineto{\pgfqpoint{2.870015in}{2.422973in}}%
\pgfpathlineto{\pgfqpoint{2.870831in}{2.513741in}}%
\pgfpathlineto{\pgfqpoint{2.871240in}{2.497237in}}%
\pgfpathlineto{\pgfqpoint{2.871852in}{2.480734in}}%
\pgfpathlineto{\pgfqpoint{2.872056in}{2.513741in}}%
\pgfpathlineto{\pgfqpoint{2.872260in}{2.414721in}}%
\pgfpathlineto{\pgfqpoint{2.873077in}{2.497237in}}%
\pgfpathlineto{\pgfqpoint{2.873894in}{2.530244in}}%
\pgfpathlineto{\pgfqpoint{2.874098in}{2.480734in}}%
\pgfpathlineto{\pgfqpoint{2.874302in}{2.571502in}}%
\pgfpathlineto{\pgfqpoint{2.874914in}{2.546747in}}%
\pgfpathlineto{\pgfqpoint{2.875118in}{2.571502in}}%
\pgfpathlineto{\pgfqpoint{2.875323in}{2.530244in}}%
\pgfpathlineto{\pgfqpoint{2.875935in}{2.282696in}}%
\pgfpathlineto{\pgfqpoint{2.876343in}{2.521992in}}%
\pgfpathlineto{\pgfqpoint{2.876752in}{2.480734in}}%
\pgfpathlineto{\pgfqpoint{2.876956in}{2.546747in}}%
\pgfpathlineto{\pgfqpoint{2.877160in}{2.521992in}}%
\pgfpathlineto{\pgfqpoint{2.877976in}{2.662269in}}%
\pgfpathlineto{\pgfqpoint{2.878181in}{2.629263in}}%
\pgfpathlineto{\pgfqpoint{2.878997in}{2.464231in}}%
\pgfpathlineto{\pgfqpoint{2.879201in}{2.505489in}}%
\pgfpathlineto{\pgfqpoint{2.879610in}{2.637515in}}%
\pgfpathlineto{\pgfqpoint{2.880426in}{2.596257in}}%
\pgfpathlineto{\pgfqpoint{2.881447in}{2.488986in}}%
\pgfpathlineto{\pgfqpoint{2.881651in}{2.521992in}}%
\pgfpathlineto{\pgfqpoint{2.881855in}{2.530244in}}%
\pgfpathlineto{\pgfqpoint{2.882059in}{2.521992in}}%
\pgfpathlineto{\pgfqpoint{2.882263in}{2.497237in}}%
\pgfpathlineto{\pgfqpoint{2.882672in}{2.554999in}}%
\pgfpathlineto{\pgfqpoint{2.882876in}{2.563250in}}%
\pgfpathlineto{\pgfqpoint{2.883897in}{2.455979in}}%
\pgfpathlineto{\pgfqpoint{2.884713in}{2.588005in}}%
\pgfpathlineto{\pgfqpoint{2.885121in}{2.530244in}}%
\pgfpathlineto{\pgfqpoint{2.885938in}{2.398218in}}%
\pgfpathlineto{\pgfqpoint{2.886755in}{2.439476in}}%
\pgfpathlineto{\pgfqpoint{2.886959in}{2.431225in}}%
\pgfpathlineto{\pgfqpoint{2.887163in}{2.455979in}}%
\pgfpathlineto{\pgfqpoint{2.887775in}{2.497237in}}%
\pgfpathlineto{\pgfqpoint{2.887571in}{2.439476in}}%
\pgfpathlineto{\pgfqpoint{2.888388in}{2.472483in}}%
\pgfpathlineto{\pgfqpoint{2.888592in}{2.431225in}}%
\pgfpathlineto{\pgfqpoint{2.888796in}{2.488986in}}%
\pgfpathlineto{\pgfqpoint{2.889204in}{2.472483in}}%
\pgfpathlineto{\pgfqpoint{2.889817in}{2.497237in}}%
\pgfpathlineto{\pgfqpoint{2.890021in}{2.472483in}}%
\pgfpathlineto{\pgfqpoint{2.890429in}{2.398218in}}%
\pgfpathlineto{\pgfqpoint{2.891042in}{2.447728in}}%
\pgfpathlineto{\pgfqpoint{2.891450in}{2.472483in}}%
\pgfpathlineto{\pgfqpoint{2.891654in}{2.455979in}}%
\pgfpathlineto{\pgfqpoint{2.892266in}{2.406470in}}%
\pgfpathlineto{\pgfqpoint{2.892675in}{2.431225in}}%
\pgfpathlineto{\pgfqpoint{2.892879in}{2.447728in}}%
\pgfpathlineto{\pgfqpoint{2.893083in}{2.398218in}}%
\pgfpathlineto{\pgfqpoint{2.893287in}{2.389967in}}%
\pgfpathlineto{\pgfqpoint{2.893491in}{2.422973in}}%
\pgfpathlineto{\pgfqpoint{2.893695in}{2.398218in}}%
\pgfpathlineto{\pgfqpoint{2.894716in}{2.497237in}}%
\pgfpathlineto{\pgfqpoint{2.894920in}{2.439476in}}%
\pgfpathlineto{\pgfqpoint{2.896553in}{2.546747in}}%
\pgfpathlineto{\pgfqpoint{2.897778in}{2.464231in}}%
\pgfpathlineto{\pgfqpoint{2.897574in}{2.563250in}}%
\pgfpathlineto{\pgfqpoint{2.897982in}{2.472483in}}%
\pgfpathlineto{\pgfqpoint{2.898187in}{2.472483in}}%
\pgfpathlineto{\pgfqpoint{2.899411in}{2.563250in}}%
\pgfpathlineto{\pgfqpoint{2.900636in}{2.365212in}}%
\pgfpathlineto{\pgfqpoint{2.901861in}{2.629263in}}%
\pgfpathlineto{\pgfqpoint{2.902065in}{2.546747in}}%
\pgfpathlineto{\pgfqpoint{2.902474in}{2.571502in}}%
\pgfpathlineto{\pgfqpoint{2.902678in}{2.538495in}}%
\pgfpathlineto{\pgfqpoint{2.903903in}{2.323954in}}%
\pgfpathlineto{\pgfqpoint{2.904107in}{2.332205in}}%
\pgfpathlineto{\pgfqpoint{2.904515in}{2.381715in}}%
\pgfpathlineto{\pgfqpoint{2.905127in}{2.348709in}}%
\pgfpathlineto{\pgfqpoint{2.906556in}{2.282696in}}%
\pgfpathlineto{\pgfqpoint{2.907169in}{2.406470in}}%
\pgfpathlineto{\pgfqpoint{2.907985in}{2.373463in}}%
\pgfpathlineto{\pgfqpoint{2.908598in}{2.332205in}}%
\pgfpathlineto{\pgfqpoint{2.909414in}{2.274444in}}%
\pgfpathlineto{\pgfqpoint{2.909619in}{2.299199in}}%
\pgfpathlineto{\pgfqpoint{2.910843in}{2.389967in}}%
\pgfpathlineto{\pgfqpoint{2.910027in}{2.282696in}}%
\pgfpathlineto{\pgfqpoint{2.911252in}{2.365212in}}%
\pgfpathlineto{\pgfqpoint{2.911456in}{2.290947in}}%
\pgfpathlineto{\pgfqpoint{2.912068in}{2.373463in}}%
\pgfpathlineto{\pgfqpoint{2.912272in}{2.373463in}}%
\pgfpathlineto{\pgfqpoint{2.912681in}{2.406470in}}%
\pgfpathlineto{\pgfqpoint{2.912885in}{2.340457in}}%
\pgfpathlineto{\pgfqpoint{2.913089in}{2.356960in}}%
\pgfpathlineto{\pgfqpoint{2.913701in}{2.373463in}}%
\pgfpathlineto{\pgfqpoint{2.914722in}{2.142418in}}%
\pgfpathlineto{\pgfqpoint{2.915130in}{2.389967in}}%
\pgfpathlineto{\pgfqpoint{2.915539in}{2.076406in}}%
\pgfpathlineto{\pgfqpoint{2.915947in}{2.290947in}}%
\pgfpathlineto{\pgfqpoint{2.916559in}{2.414721in}}%
\pgfpathlineto{\pgfqpoint{2.917172in}{2.332205in}}%
\pgfpathlineto{\pgfqpoint{2.917376in}{2.340457in}}%
\pgfpathlineto{\pgfqpoint{2.917580in}{2.323954in}}%
\pgfpathlineto{\pgfqpoint{2.917784in}{2.290947in}}%
\pgfpathlineto{\pgfqpoint{2.918193in}{2.348709in}}%
\pgfpathlineto{\pgfqpoint{2.918397in}{2.340457in}}%
\pgfpathlineto{\pgfqpoint{2.918601in}{2.398218in}}%
\pgfpathlineto{\pgfqpoint{2.919417in}{2.348709in}}%
\pgfpathlineto{\pgfqpoint{2.919622in}{2.348709in}}%
\pgfpathlineto{\pgfqpoint{2.920030in}{2.266192in}}%
\pgfpathlineto{\pgfqpoint{2.920846in}{2.290947in}}%
\pgfpathlineto{\pgfqpoint{2.921663in}{2.365212in}}%
\pgfpathlineto{\pgfqpoint{2.921255in}{2.282696in}}%
\pgfpathlineto{\pgfqpoint{2.921867in}{2.299199in}}%
\pgfpathlineto{\pgfqpoint{2.922275in}{2.323954in}}%
\pgfpathlineto{\pgfqpoint{2.923704in}{2.241438in}}%
\pgfpathlineto{\pgfqpoint{2.924521in}{2.323954in}}%
\pgfpathlineto{\pgfqpoint{2.924929in}{2.307450in}}%
\pgfpathlineto{\pgfqpoint{2.925338in}{2.373463in}}%
\pgfpathlineto{\pgfqpoint{2.925746in}{2.332205in}}%
\pgfpathlineto{\pgfqpoint{2.926358in}{2.150670in}}%
\pgfpathlineto{\pgfqpoint{2.926767in}{2.191928in}}%
\pgfpathlineto{\pgfqpoint{2.927379in}{2.414721in}}%
\pgfpathlineto{\pgfqpoint{2.927991in}{2.340457in}}%
\pgfpathlineto{\pgfqpoint{2.929012in}{2.414721in}}%
\pgfpathlineto{\pgfqpoint{2.928604in}{2.315702in}}%
\pgfpathlineto{\pgfqpoint{2.929216in}{2.381715in}}%
\pgfpathlineto{\pgfqpoint{2.929625in}{2.241438in}}%
\pgfpathlineto{\pgfqpoint{2.930237in}{2.356960in}}%
\pgfpathlineto{\pgfqpoint{2.930441in}{2.414721in}}%
\pgfpathlineto{\pgfqpoint{2.930849in}{2.348709in}}%
\pgfpathlineto{\pgfqpoint{2.931666in}{2.092909in}}%
\pgfpathlineto{\pgfqpoint{2.932278in}{2.150670in}}%
\pgfpathlineto{\pgfqpoint{2.933707in}{2.398218in}}%
\pgfpathlineto{\pgfqpoint{2.934320in}{2.389967in}}%
\pgfpathlineto{\pgfqpoint{2.935545in}{2.323954in}}%
\pgfpathlineto{\pgfqpoint{2.935136in}{2.398218in}}%
\pgfpathlineto{\pgfqpoint{2.935953in}{2.340457in}}%
\pgfpathlineto{\pgfqpoint{2.937178in}{2.414721in}}%
\pgfpathlineto{\pgfqpoint{2.937382in}{2.398218in}}%
\pgfpathlineto{\pgfqpoint{2.937994in}{2.447728in}}%
\pgfpathlineto{\pgfqpoint{2.938811in}{2.332205in}}%
\pgfpathlineto{\pgfqpoint{2.939423in}{2.356960in}}%
\pgfpathlineto{\pgfqpoint{2.939627in}{2.348709in}}%
\pgfpathlineto{\pgfqpoint{2.940444in}{2.257941in}}%
\pgfpathlineto{\pgfqpoint{2.940648in}{2.282696in}}%
\pgfpathlineto{\pgfqpoint{2.941261in}{2.356960in}}%
\pgfpathlineto{\pgfqpoint{2.941465in}{2.150670in}}%
\pgfpathlineto{\pgfqpoint{2.942077in}{2.389967in}}%
\pgfpathlineto{\pgfqpoint{2.942281in}{2.373463in}}%
\pgfpathlineto{\pgfqpoint{2.943506in}{2.101160in}}%
\pgfpathlineto{\pgfqpoint{2.943710in}{2.200180in}}%
\pgfpathlineto{\pgfqpoint{2.943914in}{2.348709in}}%
\pgfpathlineto{\pgfqpoint{2.944935in}{2.299199in}}%
\pgfpathlineto{\pgfqpoint{2.946364in}{2.398218in}}%
\pgfpathlineto{\pgfqpoint{2.948814in}{2.142418in}}%
\pgfpathlineto{\pgfqpoint{2.949222in}{2.200180in}}%
\pgfpathlineto{\pgfqpoint{2.950447in}{1.936128in}}%
\pgfpathlineto{\pgfqpoint{2.951059in}{2.175425in}}%
\pgfpathlineto{\pgfqpoint{2.951468in}{2.010393in}}%
\pgfpathlineto{\pgfqpoint{2.953305in}{1.853612in}}%
\pgfpathlineto{\pgfqpoint{2.954326in}{2.010393in}}%
\pgfpathlineto{\pgfqpoint{2.954530in}{2.002141in}}%
\pgfpathlineto{\pgfqpoint{2.954734in}{1.927877in}}%
\pgfpathlineto{\pgfqpoint{2.954938in}{2.035148in}}%
\pgfpathlineto{\pgfqpoint{2.955551in}{2.026896in}}%
\pgfpathlineto{\pgfqpoint{2.955755in}{2.035148in}}%
\pgfpathlineto{\pgfqpoint{2.956163in}{1.936128in}}%
\pgfpathlineto{\pgfqpoint{2.956980in}{1.944380in}}%
\pgfpathlineto{\pgfqpoint{2.957184in}{1.919625in}}%
\pgfpathlineto{\pgfqpoint{2.957592in}{1.985638in}}%
\pgfpathlineto{\pgfqpoint{2.957796in}{1.969135in}}%
\pgfpathlineto{\pgfqpoint{2.958000in}{2.043399in}}%
\pgfpathlineto{\pgfqpoint{2.958613in}{1.944380in}}%
\pgfpathlineto{\pgfqpoint{2.959021in}{1.861864in}}%
\pgfpathlineto{\pgfqpoint{2.959429in}{2.010393in}}%
\pgfpathlineto{\pgfqpoint{2.959838in}{1.969135in}}%
\pgfpathlineto{\pgfqpoint{2.960450in}{1.985638in}}%
\pgfpathlineto{\pgfqpoint{2.960654in}{2.043399in}}%
\pgfpathlineto{\pgfqpoint{2.961267in}{1.952632in}}%
\pgfpathlineto{\pgfqpoint{2.961471in}{2.002141in}}%
\pgfpathlineto{\pgfqpoint{2.962287in}{2.018644in}}%
\pgfpathlineto{\pgfqpoint{2.962900in}{1.927877in}}%
\pgfpathlineto{\pgfqpoint{2.964533in}{2.043399in}}%
\pgfpathlineto{\pgfqpoint{2.965962in}{1.911373in}}%
\pgfpathlineto{\pgfqpoint{2.966778in}{1.977386in}}%
\pgfpathlineto{\pgfqpoint{2.967187in}{1.952632in}}%
\pgfpathlineto{\pgfqpoint{2.967391in}{1.985638in}}%
\pgfpathlineto{\pgfqpoint{2.967595in}{1.936128in}}%
\pgfpathlineto{\pgfqpoint{2.968207in}{1.969135in}}%
\pgfpathlineto{\pgfqpoint{2.968412in}{1.894870in}}%
\pgfpathlineto{\pgfqpoint{2.968616in}{1.977386in}}%
\pgfpathlineto{\pgfqpoint{2.969228in}{1.936128in}}%
\pgfpathlineto{\pgfqpoint{2.970453in}{2.076406in}}%
\pgfpathlineto{\pgfqpoint{2.970861in}{2.035148in}}%
\pgfpathlineto{\pgfqpoint{2.971065in}{2.035148in}}%
\pgfpathlineto{\pgfqpoint{2.972699in}{2.101160in}}%
\pgfpathlineto{\pgfqpoint{2.972903in}{2.035148in}}%
\pgfpathlineto{\pgfqpoint{2.973515in}{2.175425in}}%
\pgfpathlineto{\pgfqpoint{2.973719in}{2.125915in}}%
\pgfpathlineto{\pgfqpoint{2.974536in}{2.175425in}}%
\pgfpathlineto{\pgfqpoint{2.975557in}{2.026896in}}%
\pgfpathlineto{\pgfqpoint{2.975761in}{2.051651in}}%
\pgfpathlineto{\pgfqpoint{2.975965in}{2.142418in}}%
\pgfpathlineto{\pgfqpoint{2.976373in}{2.026896in}}%
\pgfpathlineto{\pgfqpoint{2.976577in}{2.043399in}}%
\pgfpathlineto{\pgfqpoint{2.977394in}{1.936128in}}%
\pgfpathlineto{\pgfqpoint{2.977802in}{1.952632in}}%
\pgfpathlineto{\pgfqpoint{2.978823in}{2.059902in}}%
\pgfpathlineto{\pgfqpoint{2.979027in}{2.043399in}}%
\pgfpathlineto{\pgfqpoint{2.979435in}{1.812354in}}%
\pgfpathlineto{\pgfqpoint{2.980252in}{1.960883in}}%
\pgfpathlineto{\pgfqpoint{2.980456in}{1.985638in}}%
\pgfpathlineto{\pgfqpoint{2.980864in}{1.911373in}}%
\pgfpathlineto{\pgfqpoint{2.981068in}{1.927877in}}%
\pgfpathlineto{\pgfqpoint{2.981273in}{1.853612in}}%
\pgfpathlineto{\pgfqpoint{2.982089in}{1.894870in}}%
\pgfpathlineto{\pgfqpoint{2.982293in}{1.894870in}}%
\pgfpathlineto{\pgfqpoint{2.982702in}{1.886619in}}%
\pgfpathlineto{\pgfqpoint{2.982906in}{1.969135in}}%
\pgfpathlineto{\pgfqpoint{2.983926in}{1.936128in}}%
\pgfpathlineto{\pgfqpoint{2.985355in}{1.837109in}}%
\pgfpathlineto{\pgfqpoint{2.985560in}{1.936128in}}%
\pgfpathlineto{\pgfqpoint{2.986580in}{1.919625in}}%
\pgfpathlineto{\pgfqpoint{2.986989in}{1.903122in}}%
\pgfpathlineto{\pgfqpoint{2.987397in}{1.969135in}}%
\pgfpathlineto{\pgfqpoint{2.988213in}{1.837109in}}%
\pgfpathlineto{\pgfqpoint{2.988622in}{1.845361in}}%
\pgfpathlineto{\pgfqpoint{2.989847in}{1.927877in}}%
\pgfpathlineto{\pgfqpoint{2.990051in}{1.919625in}}%
\pgfpathlineto{\pgfqpoint{2.990255in}{1.894870in}}%
\pgfpathlineto{\pgfqpoint{2.990867in}{1.911373in}}%
\pgfpathlineto{\pgfqpoint{2.991071in}{1.960883in}}%
\pgfpathlineto{\pgfqpoint{2.991684in}{1.853612in}}%
\pgfpathlineto{\pgfqpoint{2.991888in}{1.721587in}}%
\pgfpathlineto{\pgfqpoint{2.992296in}{1.944380in}}%
\pgfpathlineto{\pgfqpoint{2.992705in}{1.878367in}}%
\pgfpathlineto{\pgfqpoint{2.993725in}{2.043399in}}%
\pgfpathlineto{\pgfqpoint{2.994134in}{1.960883in}}%
\pgfpathlineto{\pgfqpoint{2.995358in}{2.068154in}}%
\pgfpathlineto{\pgfqpoint{2.995563in}{2.026896in}}%
\pgfpathlineto{\pgfqpoint{2.996992in}{1.870115in}}%
\pgfpathlineto{\pgfqpoint{2.995971in}{2.043399in}}%
\pgfpathlineto{\pgfqpoint{2.997400in}{1.886619in}}%
\pgfpathlineto{\pgfqpoint{2.998829in}{2.084657in}}%
\pgfpathlineto{\pgfqpoint{2.998421in}{1.870115in}}%
\pgfpathlineto{\pgfqpoint{2.999033in}{2.035148in}}%
\pgfpathlineto{\pgfqpoint{3.000054in}{1.861864in}}%
\pgfpathlineto{\pgfqpoint{3.000258in}{1.878367in}}%
\pgfpathlineto{\pgfqpoint{3.001074in}{1.919625in}}%
\pgfpathlineto{\pgfqpoint{3.001483in}{1.894870in}}%
\pgfpathlineto{\pgfqpoint{3.002095in}{1.779348in}}%
\pgfpathlineto{\pgfqpoint{3.002299in}{1.622567in}}%
\pgfpathlineto{\pgfqpoint{3.002912in}{1.985638in}}%
\pgfpathlineto{\pgfqpoint{3.004749in}{1.589561in}}%
\pgfpathlineto{\pgfqpoint{3.005157in}{1.754593in}}%
\pgfpathlineto{\pgfqpoint{3.006586in}{1.944380in}}%
\pgfpathlineto{\pgfqpoint{3.006995in}{1.696832in}}%
\pgfpathlineto{\pgfqpoint{3.007403in}{1.960883in}}%
\pgfpathlineto{\pgfqpoint{3.007811in}{1.993890in}}%
\pgfpathlineto{\pgfqpoint{3.008015in}{1.960883in}}%
\pgfpathlineto{\pgfqpoint{3.008424in}{1.795851in}}%
\pgfpathlineto{\pgfqpoint{3.009036in}{2.010393in}}%
\pgfpathlineto{\pgfqpoint{3.009240in}{2.010393in}}%
\pgfpathlineto{\pgfqpoint{3.010669in}{1.853612in}}%
\pgfpathlineto{\pgfqpoint{3.009648in}{2.018644in}}%
\pgfpathlineto{\pgfqpoint{3.010873in}{1.919625in}}%
\pgfpathlineto{\pgfqpoint{3.011077in}{1.919625in}}%
\pgfpathlineto{\pgfqpoint{3.011282in}{1.837109in}}%
\pgfpathlineto{\pgfqpoint{3.012098in}{1.936128in}}%
\pgfpathlineto{\pgfqpoint{3.012302in}{1.936128in}}%
\pgfpathlineto{\pgfqpoint{3.013119in}{1.845361in}}%
\pgfpathlineto{\pgfqpoint{3.013527in}{1.886619in}}%
\pgfpathlineto{\pgfqpoint{3.014140in}{1.870115in}}%
\pgfpathlineto{\pgfqpoint{3.014548in}{1.952632in}}%
\pgfpathlineto{\pgfqpoint{3.014752in}{1.911373in}}%
\pgfpathlineto{\pgfqpoint{3.015569in}{1.960883in}}%
\pgfpathlineto{\pgfqpoint{3.016589in}{1.894870in}}%
\pgfpathlineto{\pgfqpoint{3.016793in}{1.911373in}}%
\pgfpathlineto{\pgfqpoint{3.017202in}{1.977386in}}%
\pgfpathlineto{\pgfqpoint{3.017406in}{1.886619in}}%
\pgfpathlineto{\pgfqpoint{3.017610in}{1.630819in}}%
\pgfpathlineto{\pgfqpoint{3.018222in}{2.035148in}}%
\pgfpathlineto{\pgfqpoint{3.018427in}{1.969135in}}%
\pgfpathlineto{\pgfqpoint{3.019039in}{1.771096in}}%
\pgfpathlineto{\pgfqpoint{3.020264in}{1.795851in}}%
\pgfpathlineto{\pgfqpoint{3.020672in}{1.787599in}}%
\pgfpathlineto{\pgfqpoint{3.021693in}{1.853612in}}%
\pgfpathlineto{\pgfqpoint{3.022714in}{1.812354in}}%
\pgfpathlineto{\pgfqpoint{3.022101in}{1.886619in}}%
\pgfpathlineto{\pgfqpoint{3.022918in}{1.820606in}}%
\pgfpathlineto{\pgfqpoint{3.023734in}{1.886619in}}%
\pgfpathlineto{\pgfqpoint{3.023938in}{1.828857in}}%
\pgfpathlineto{\pgfqpoint{3.024551in}{1.771096in}}%
\pgfpathlineto{\pgfqpoint{3.024755in}{1.804103in}}%
\pgfpathlineto{\pgfqpoint{3.026184in}{2.026896in}}%
\pgfpathlineto{\pgfqpoint{3.026592in}{1.861864in}}%
\pgfpathlineto{\pgfqpoint{3.027409in}{1.870115in}}%
\pgfpathlineto{\pgfqpoint{3.028838in}{1.523548in}}%
\pgfpathlineto{\pgfqpoint{3.029858in}{1.870115in}}%
\pgfpathlineto{\pgfqpoint{3.030063in}{1.804103in}}%
\pgfpathlineto{\pgfqpoint{3.030267in}{1.804103in}}%
\pgfpathlineto{\pgfqpoint{3.030879in}{1.754593in}}%
\pgfpathlineto{\pgfqpoint{3.031492in}{1.779348in}}%
\pgfpathlineto{\pgfqpoint{3.032104in}{1.721587in}}%
\pgfpathlineto{\pgfqpoint{3.032308in}{1.540051in}}%
\pgfpathlineto{\pgfqpoint{3.032716in}{1.812354in}}%
\pgfpathlineto{\pgfqpoint{3.032921in}{1.771096in}}%
\pgfpathlineto{\pgfqpoint{3.033125in}{1.903122in}}%
\pgfpathlineto{\pgfqpoint{3.033329in}{1.614316in}}%
\pgfpathlineto{\pgfqpoint{3.034145in}{1.878367in}}%
\pgfpathlineto{\pgfqpoint{3.034350in}{1.853612in}}%
\pgfpathlineto{\pgfqpoint{3.034758in}{1.936128in}}%
\pgfpathlineto{\pgfqpoint{3.037003in}{1.787599in}}%
\pgfpathlineto{\pgfqpoint{3.037820in}{1.853612in}}%
\pgfpathlineto{\pgfqpoint{3.038024in}{1.779348in}}%
\pgfpathlineto{\pgfqpoint{3.039657in}{1.919625in}}%
\pgfpathlineto{\pgfqpoint{3.040066in}{1.878367in}}%
\pgfpathlineto{\pgfqpoint{3.041086in}{1.705083in}}%
\pgfpathlineto{\pgfqpoint{3.041290in}{1.713335in}}%
\pgfpathlineto{\pgfqpoint{3.041495in}{1.738090in}}%
\pgfpathlineto{\pgfqpoint{3.041903in}{1.680329in}}%
\pgfpathlineto{\pgfqpoint{3.042107in}{1.647322in}}%
\pgfpathlineto{\pgfqpoint{3.042515in}{1.721587in}}%
\pgfpathlineto{\pgfqpoint{3.043536in}{1.795851in}}%
\pgfpathlineto{\pgfqpoint{3.043128in}{1.688580in}}%
\pgfpathlineto{\pgfqpoint{3.043740in}{1.762845in}}%
\pgfpathlineto{\pgfqpoint{3.044557in}{1.663825in}}%
\pgfpathlineto{\pgfqpoint{3.044965in}{1.713335in}}%
\pgfpathlineto{\pgfqpoint{3.045373in}{1.828857in}}%
\pgfpathlineto{\pgfqpoint{3.045577in}{1.705083in}}%
\pgfpathlineto{\pgfqpoint{3.045986in}{1.787599in}}%
\pgfpathlineto{\pgfqpoint{3.047006in}{1.729838in}}%
\pgfpathlineto{\pgfqpoint{3.046802in}{1.820606in}}%
\pgfpathlineto{\pgfqpoint{3.047211in}{1.738090in}}%
\pgfpathlineto{\pgfqpoint{3.048027in}{1.672077in}}%
\pgfpathlineto{\pgfqpoint{3.049456in}{1.837109in}}%
\pgfpathlineto{\pgfqpoint{3.049864in}{1.754593in}}%
\pgfpathlineto{\pgfqpoint{3.050273in}{1.804103in}}%
\pgfpathlineto{\pgfqpoint{3.050681in}{1.861864in}}%
\pgfpathlineto{\pgfqpoint{3.050885in}{1.787599in}}%
\pgfpathlineto{\pgfqpoint{3.051089in}{1.828857in}}%
\pgfpathlineto{\pgfqpoint{3.051498in}{1.713335in}}%
\pgfpathlineto{\pgfqpoint{3.052110in}{1.754593in}}%
\pgfpathlineto{\pgfqpoint{3.052518in}{1.861864in}}%
\pgfpathlineto{\pgfqpoint{3.052722in}{1.746341in}}%
\pgfpathlineto{\pgfqpoint{3.053131in}{1.837109in}}%
\pgfpathlineto{\pgfqpoint{3.054151in}{1.622567in}}%
\pgfpathlineto{\pgfqpoint{3.054560in}{1.663825in}}%
\pgfpathlineto{\pgfqpoint{3.055376in}{1.771096in}}%
\pgfpathlineto{\pgfqpoint{3.055785in}{1.754593in}}%
\pgfpathlineto{\pgfqpoint{3.056601in}{1.672077in}}%
\pgfpathlineto{\pgfqpoint{3.056805in}{1.680329in}}%
\pgfpathlineto{\pgfqpoint{3.057418in}{1.787599in}}%
\pgfpathlineto{\pgfqpoint{3.057826in}{1.721587in}}%
\pgfpathlineto{\pgfqpoint{3.058234in}{1.639071in}}%
\pgfpathlineto{\pgfqpoint{3.058847in}{1.696832in}}%
\pgfpathlineto{\pgfqpoint{3.059051in}{1.721587in}}%
\pgfpathlineto{\pgfqpoint{3.059255in}{1.647322in}}%
\pgfpathlineto{\pgfqpoint{3.059663in}{1.614316in}}%
\pgfpathlineto{\pgfqpoint{3.060072in}{1.688580in}}%
\pgfpathlineto{\pgfqpoint{3.060276in}{1.746341in}}%
\pgfpathlineto{\pgfqpoint{3.060684in}{1.680329in}}%
\pgfpathlineto{\pgfqpoint{3.061092in}{1.713335in}}%
\pgfpathlineto{\pgfqpoint{3.061296in}{1.713335in}}%
\pgfpathlineto{\pgfqpoint{3.062521in}{1.630819in}}%
\pgfpathlineto{\pgfqpoint{3.062725in}{1.639071in}}%
\pgfpathlineto{\pgfqpoint{3.062930in}{1.614316in}}%
\pgfpathlineto{\pgfqpoint{3.063338in}{1.564806in}}%
\pgfpathlineto{\pgfqpoint{3.063746in}{1.647322in}}%
\pgfpathlineto{\pgfqpoint{3.063950in}{1.581309in}}%
\pgfpathlineto{\pgfqpoint{3.064154in}{1.647322in}}%
\pgfpathlineto{\pgfqpoint{3.064767in}{1.531800in}}%
\pgfpathlineto{\pgfqpoint{3.064971in}{1.564806in}}%
\pgfpathlineto{\pgfqpoint{3.065379in}{1.597813in}}%
\pgfpathlineto{\pgfqpoint{3.065583in}{1.589561in}}%
\pgfpathlineto{\pgfqpoint{3.066400in}{1.383271in}}%
\pgfpathlineto{\pgfqpoint{3.066604in}{1.614316in}}%
\pgfpathlineto{\pgfqpoint{3.067421in}{1.721587in}}%
\pgfpathlineto{\pgfqpoint{3.067625in}{1.474038in}}%
\pgfpathlineto{\pgfqpoint{3.068441in}{1.738090in}}%
\pgfpathlineto{\pgfqpoint{3.068646in}{1.721587in}}%
\pgfpathlineto{\pgfqpoint{3.068850in}{1.754593in}}%
\pgfpathlineto{\pgfqpoint{3.069054in}{1.754593in}}%
\pgfpathlineto{\pgfqpoint{3.069462in}{1.630819in}}%
\pgfpathlineto{\pgfqpoint{3.070075in}{1.812354in}}%
\pgfpathlineto{\pgfqpoint{3.071095in}{1.771096in}}%
\pgfpathlineto{\pgfqpoint{3.071504in}{1.911373in}}%
\pgfpathlineto{\pgfqpoint{3.071912in}{1.804103in}}%
\pgfpathlineto{\pgfqpoint{3.073137in}{1.721587in}}%
\pgfpathlineto{\pgfqpoint{3.073749in}{1.754593in}}%
\pgfpathlineto{\pgfqpoint{3.074974in}{1.614316in}}%
\pgfpathlineto{\pgfqpoint{3.075178in}{1.622567in}}%
\pgfpathlineto{\pgfqpoint{3.077015in}{1.812354in}}%
\pgfpathlineto{\pgfqpoint{3.077220in}{1.746341in}}%
\pgfpathlineto{\pgfqpoint{3.077424in}{1.449284in}}%
\pgfpathlineto{\pgfqpoint{3.078240in}{1.828857in}}%
\pgfpathlineto{\pgfqpoint{3.079261in}{1.581309in}}%
\pgfpathlineto{\pgfqpoint{3.079669in}{1.630819in}}%
\pgfpathlineto{\pgfqpoint{3.080078in}{1.705083in}}%
\pgfpathlineto{\pgfqpoint{3.080690in}{1.672077in}}%
\pgfpathlineto{\pgfqpoint{3.080894in}{1.622567in}}%
\pgfpathlineto{\pgfqpoint{3.081507in}{1.680329in}}%
\pgfpathlineto{\pgfqpoint{3.082527in}{1.779348in}}%
\pgfpathlineto{\pgfqpoint{3.082936in}{1.729838in}}%
\pgfpathlineto{\pgfqpoint{3.083956in}{1.647322in}}%
\pgfpathlineto{\pgfqpoint{3.084160in}{1.688580in}}%
\pgfpathlineto{\pgfqpoint{3.084977in}{1.713335in}}%
\pgfpathlineto{\pgfqpoint{3.085181in}{1.688580in}}%
\pgfpathlineto{\pgfqpoint{3.085794in}{1.647322in}}%
\pgfpathlineto{\pgfqpoint{3.086202in}{1.696832in}}%
\pgfpathlineto{\pgfqpoint{3.086814in}{1.663825in}}%
\pgfpathlineto{\pgfqpoint{3.087427in}{1.746341in}}%
\pgfpathlineto{\pgfqpoint{3.089060in}{1.622567in}}%
\pgfpathlineto{\pgfqpoint{3.089468in}{1.630819in}}%
\pgfpathlineto{\pgfqpoint{3.091101in}{1.762845in}}%
\pgfpathlineto{\pgfqpoint{3.091510in}{1.771096in}}%
\pgfpathlineto{\pgfqpoint{3.092530in}{1.564806in}}%
\pgfpathlineto{\pgfqpoint{3.093143in}{1.787599in}}%
\pgfpathlineto{\pgfqpoint{3.093551in}{1.688580in}}%
\pgfpathlineto{\pgfqpoint{3.093755in}{1.556554in}}%
\pgfpathlineto{\pgfqpoint{3.093959in}{1.779348in}}%
\pgfpathlineto{\pgfqpoint{3.094572in}{1.688580in}}%
\pgfpathlineto{\pgfqpoint{3.095184in}{1.647322in}}%
\pgfpathlineto{\pgfqpoint{3.095797in}{1.754593in}}%
\pgfpathlineto{\pgfqpoint{3.097226in}{1.663825in}}%
\pgfpathlineto{\pgfqpoint{3.097430in}{1.696832in}}%
\pgfpathlineto{\pgfqpoint{3.097838in}{1.606064in}}%
\pgfpathlineto{\pgfqpoint{3.098042in}{1.647322in}}%
\pgfpathlineto{\pgfqpoint{3.098859in}{1.606064in}}%
\pgfpathlineto{\pgfqpoint{3.099675in}{1.729838in}}%
\pgfpathlineto{\pgfqpoint{3.100084in}{1.680329in}}%
\pgfpathlineto{\pgfqpoint{3.100696in}{1.540051in}}%
\pgfpathlineto{\pgfqpoint{3.101513in}{1.614316in}}%
\pgfpathlineto{\pgfqpoint{3.101921in}{1.581309in}}%
\pgfpathlineto{\pgfqpoint{3.102329in}{1.309006in}}%
\pgfpathlineto{\pgfqpoint{3.103350in}{1.358516in}}%
\pgfpathlineto{\pgfqpoint{3.103554in}{1.350264in}}%
\pgfpathlineto{\pgfqpoint{3.104575in}{1.721587in}}%
\pgfpathlineto{\pgfqpoint{3.104983in}{1.705083in}}%
\pgfpathlineto{\pgfqpoint{3.105391in}{1.721587in}}%
\pgfpathlineto{\pgfqpoint{3.106208in}{1.663825in}}%
\pgfpathlineto{\pgfqpoint{3.106616in}{1.705083in}}%
\pgfpathlineto{\pgfqpoint{3.106820in}{1.606064in}}%
\pgfpathlineto{\pgfqpoint{3.107637in}{1.713335in}}%
\pgfpathlineto{\pgfqpoint{3.108045in}{1.630819in}}%
\pgfpathlineto{\pgfqpoint{3.108453in}{1.573058in}}%
\pgfpathlineto{\pgfqpoint{3.108862in}{1.606064in}}%
\pgfpathlineto{\pgfqpoint{3.109678in}{1.738090in}}%
\pgfpathlineto{\pgfqpoint{3.109474in}{1.581309in}}%
\pgfpathlineto{\pgfqpoint{3.109882in}{1.597813in}}%
\pgfpathlineto{\pgfqpoint{3.110903in}{1.465787in}}%
\pgfpathlineto{\pgfqpoint{3.111107in}{1.482290in}}%
\pgfpathlineto{\pgfqpoint{3.111924in}{1.523548in}}%
\pgfpathlineto{\pgfqpoint{3.111720in}{1.432780in}}%
\pgfpathlineto{\pgfqpoint{3.112128in}{1.482290in}}%
\pgfpathlineto{\pgfqpoint{3.112332in}{1.465787in}}%
\pgfpathlineto{\pgfqpoint{3.112536in}{1.564806in}}%
\pgfpathlineto{\pgfqpoint{3.113353in}{1.498793in}}%
\pgfpathlineto{\pgfqpoint{3.113965in}{1.581309in}}%
\pgfpathlineto{\pgfqpoint{3.114169in}{1.556554in}}%
\pgfpathlineto{\pgfqpoint{3.114782in}{1.465787in}}%
\pgfpathlineto{\pgfqpoint{3.114986in}{1.515296in}}%
\pgfpathlineto{\pgfqpoint{3.115190in}{1.589561in}}%
\pgfpathlineto{\pgfqpoint{3.116007in}{1.490542in}}%
\pgfpathlineto{\pgfqpoint{3.116211in}{1.482290in}}%
\pgfpathlineto{\pgfqpoint{3.117436in}{1.639071in}}%
\pgfpathlineto{\pgfqpoint{3.117640in}{1.597813in}}%
\pgfpathlineto{\pgfqpoint{3.118252in}{1.647322in}}%
\pgfpathlineto{\pgfqpoint{3.118456in}{1.474038in}}%
\pgfpathlineto{\pgfqpoint{3.119273in}{1.688580in}}%
\pgfpathlineto{\pgfqpoint{3.120294in}{1.705083in}}%
\pgfpathlineto{\pgfqpoint{3.120702in}{1.564806in}}%
\pgfpathlineto{\pgfqpoint{3.120906in}{1.663825in}}%
\pgfpathlineto{\pgfqpoint{3.121518in}{1.465787in}}%
\pgfpathlineto{\pgfqpoint{3.121723in}{1.391522in}}%
\pgfpathlineto{\pgfqpoint{3.122539in}{1.465787in}}%
\pgfpathlineto{\pgfqpoint{3.122947in}{1.449284in}}%
\pgfpathlineto{\pgfqpoint{3.123764in}{1.507045in}}%
\pgfpathlineto{\pgfqpoint{3.124581in}{1.432780in}}%
\pgfpathlineto{\pgfqpoint{3.124785in}{1.457535in}}%
\pgfpathlineto{\pgfqpoint{3.125397in}{1.449284in}}%
\pgfpathlineto{\pgfqpoint{3.126010in}{1.515296in}}%
\pgfpathlineto{\pgfqpoint{3.127439in}{1.424529in}}%
\pgfpathlineto{\pgfqpoint{3.128663in}{1.606064in}}%
\pgfpathlineto{\pgfqpoint{3.129072in}{1.333761in}}%
\pgfpathlineto{\pgfqpoint{3.129684in}{1.523548in}}%
\pgfpathlineto{\pgfqpoint{3.132134in}{1.713335in}}%
\pgfpathlineto{\pgfqpoint{3.132338in}{1.688580in}}%
\pgfpathlineto{\pgfqpoint{3.132950in}{1.639071in}}%
\pgfpathlineto{\pgfqpoint{3.133563in}{1.663825in}}%
\pgfpathlineto{\pgfqpoint{3.134379in}{1.762845in}}%
\pgfpathlineto{\pgfqpoint{3.134584in}{1.647322in}}%
\pgfpathlineto{\pgfqpoint{3.135604in}{1.474038in}}%
\pgfpathlineto{\pgfqpoint{3.135808in}{1.498793in}}%
\pgfpathlineto{\pgfqpoint{3.136625in}{1.622567in}}%
\pgfpathlineto{\pgfqpoint{3.137033in}{1.548303in}}%
\pgfpathlineto{\pgfqpoint{3.137442in}{1.507045in}}%
\pgfpathlineto{\pgfqpoint{3.137646in}{1.523548in}}%
\pgfpathlineto{\pgfqpoint{3.137850in}{1.581309in}}%
\pgfpathlineto{\pgfqpoint{3.138258in}{1.498793in}}%
\pgfpathlineto{\pgfqpoint{3.138666in}{1.573058in}}%
\pgfpathlineto{\pgfqpoint{3.140300in}{1.457535in}}%
\pgfpathlineto{\pgfqpoint{3.140504in}{1.432780in}}%
\pgfpathlineto{\pgfqpoint{3.140708in}{1.267748in}}%
\pgfpathlineto{\pgfqpoint{3.141116in}{1.474038in}}%
\pgfpathlineto{\pgfqpoint{3.141524in}{1.449284in}}%
\pgfpathlineto{\pgfqpoint{3.141729in}{1.531800in}}%
\pgfpathlineto{\pgfqpoint{3.142749in}{1.507045in}}%
\pgfpathlineto{\pgfqpoint{3.143362in}{1.209987in}}%
\pgfpathlineto{\pgfqpoint{3.143770in}{1.449284in}}%
\pgfpathlineto{\pgfqpoint{3.144995in}{1.564806in}}%
\pgfpathlineto{\pgfqpoint{3.145199in}{1.531800in}}%
\pgfpathlineto{\pgfqpoint{3.146220in}{1.424529in}}%
\pgfpathlineto{\pgfqpoint{3.146424in}{1.432780in}}%
\pgfpathlineto{\pgfqpoint{3.147853in}{1.573058in}}%
\pgfpathlineto{\pgfqpoint{3.148057in}{1.507045in}}%
\pgfpathlineto{\pgfqpoint{3.148874in}{1.540051in}}%
\pgfpathlineto{\pgfqpoint{3.149486in}{1.581309in}}%
\pgfpathlineto{\pgfqpoint{3.149894in}{1.540051in}}%
\pgfpathlineto{\pgfqpoint{3.150098in}{1.540051in}}%
\pgfpathlineto{\pgfqpoint{3.150711in}{1.515296in}}%
\pgfpathlineto{\pgfqpoint{3.151323in}{1.589561in}}%
\pgfpathlineto{\pgfqpoint{3.152956in}{1.416277in}}%
\pgfpathlineto{\pgfqpoint{3.153773in}{1.474038in}}%
\pgfpathlineto{\pgfqpoint{3.153365in}{1.408026in}}%
\pgfpathlineto{\pgfqpoint{3.153977in}{1.441032in}}%
\pgfpathlineto{\pgfqpoint{3.154181in}{1.432780in}}%
\pgfpathlineto{\pgfqpoint{3.154385in}{1.465787in}}%
\pgfpathlineto{\pgfqpoint{3.154590in}{1.474038in}}%
\pgfpathlineto{\pgfqpoint{3.154794in}{1.449284in}}%
\pgfpathlineto{\pgfqpoint{3.155406in}{1.399774in}}%
\pgfpathlineto{\pgfqpoint{3.155202in}{1.457535in}}%
\pgfpathlineto{\pgfqpoint{3.155814in}{1.432780in}}%
\pgfpathlineto{\pgfqpoint{3.157039in}{1.531800in}}%
\pgfpathlineto{\pgfqpoint{3.156631in}{1.391522in}}%
\pgfpathlineto{\pgfqpoint{3.157243in}{1.523548in}}%
\pgfpathlineto{\pgfqpoint{3.157448in}{1.523548in}}%
\pgfpathlineto{\pgfqpoint{3.157652in}{1.465787in}}%
\pgfpathlineto{\pgfqpoint{3.158468in}{1.548303in}}%
\pgfpathlineto{\pgfqpoint{3.159081in}{1.515296in}}%
\pgfpathlineto{\pgfqpoint{3.159285in}{1.581309in}}%
\pgfpathlineto{\pgfqpoint{3.159489in}{1.548303in}}%
\pgfpathlineto{\pgfqpoint{3.159693in}{2.315702in}}%
\pgfpathlineto{\pgfqpoint{3.160101in}{1.540051in}}%
\pgfpathlineto{\pgfqpoint{3.160510in}{1.639071in}}%
\pgfpathlineto{\pgfqpoint{3.161326in}{1.465787in}}%
\pgfpathlineto{\pgfqpoint{3.161939in}{1.482290in}}%
\pgfpathlineto{\pgfqpoint{3.162551in}{1.548303in}}%
\pgfpathlineto{\pgfqpoint{3.163164in}{1.531800in}}%
\pgfpathlineto{\pgfqpoint{3.163572in}{1.482290in}}%
\pgfpathlineto{\pgfqpoint{3.164184in}{1.531800in}}%
\pgfpathlineto{\pgfqpoint{3.164797in}{1.573058in}}%
\pgfpathlineto{\pgfqpoint{3.165409in}{1.548303in}}%
\pgfpathlineto{\pgfqpoint{3.165613in}{1.548303in}}%
\pgfpathlineto{\pgfqpoint{3.165817in}{1.581309in}}%
\pgfpathlineto{\pgfqpoint{3.166226in}{1.474038in}}%
\pgfpathlineto{\pgfqpoint{3.166430in}{1.507045in}}%
\pgfpathlineto{\pgfqpoint{3.166838in}{1.383271in}}%
\pgfpathlineto{\pgfqpoint{3.167042in}{1.135723in}}%
\pgfpathlineto{\pgfqpoint{3.167655in}{1.416277in}}%
\pgfpathlineto{\pgfqpoint{3.167859in}{1.383271in}}%
\pgfpathlineto{\pgfqpoint{3.168267in}{1.309006in}}%
\pgfpathlineto{\pgfqpoint{3.168880in}{1.366768in}}%
\pgfpathlineto{\pgfqpoint{3.169084in}{1.383271in}}%
\pgfpathlineto{\pgfqpoint{3.169288in}{1.342013in}}%
\pgfpathlineto{\pgfqpoint{3.169492in}{1.309006in}}%
\pgfpathlineto{\pgfqpoint{3.169696in}{1.383271in}}%
\pgfpathlineto{\pgfqpoint{3.170309in}{1.358516in}}%
\pgfpathlineto{\pgfqpoint{3.170717in}{1.490542in}}%
\pgfpathlineto{\pgfqpoint{3.172350in}{1.086213in}}%
\pgfpathlineto{\pgfqpoint{3.173983in}{1.358516in}}%
\pgfpathlineto{\pgfqpoint{3.174596in}{0.879923in}}%
\pgfpathlineto{\pgfqpoint{3.175208in}{1.061458in}}%
\pgfpathlineto{\pgfqpoint{3.175412in}{1.020200in}}%
\pgfpathlineto{\pgfqpoint{3.175820in}{1.110968in}}%
\pgfpathlineto{\pgfqpoint{3.176025in}{1.077961in}}%
\pgfpathlineto{\pgfqpoint{3.177249in}{1.226490in}}%
\pgfpathlineto{\pgfqpoint{3.177454in}{1.143974in}}%
\pgfpathlineto{\pgfqpoint{3.178270in}{1.218239in}}%
\pgfpathlineto{\pgfqpoint{3.177862in}{1.053207in}}%
\pgfpathlineto{\pgfqpoint{3.178678in}{1.193484in}}%
\pgfpathlineto{\pgfqpoint{3.178883in}{1.185232in}}%
\pgfpathlineto{\pgfqpoint{3.179087in}{1.342013in}}%
\pgfpathlineto{\pgfqpoint{3.179903in}{1.209987in}}%
\pgfpathlineto{\pgfqpoint{3.180720in}{1.325510in}}%
\pgfpathlineto{\pgfqpoint{3.181536in}{1.317258in}}%
\pgfpathlineto{\pgfqpoint{3.182761in}{1.086213in}}%
\pgfpathlineto{\pgfqpoint{3.182965in}{1.251245in}}%
\pgfpathlineto{\pgfqpoint{3.183374in}{1.309006in}}%
\pgfpathlineto{\pgfqpoint{3.184190in}{1.201736in}}%
\pgfpathlineto{\pgfqpoint{3.184394in}{1.276000in}}%
\pgfpathlineto{\pgfqpoint{3.184803in}{1.193484in}}%
\pgfpathlineto{\pgfqpoint{3.185007in}{1.209987in}}%
\pgfpathlineto{\pgfqpoint{3.185211in}{1.152226in}}%
\pgfpathlineto{\pgfqpoint{3.185619in}{1.333761in}}%
\pgfpathlineto{\pgfqpoint{3.186028in}{1.218239in}}%
\pgfpathlineto{\pgfqpoint{3.186640in}{1.267748in}}%
\pgfpathlineto{\pgfqpoint{3.186844in}{1.242994in}}%
\pgfpathlineto{\pgfqpoint{3.187252in}{1.028452in}}%
\pgfpathlineto{\pgfqpoint{3.187865in}{1.209987in}}%
\pgfpathlineto{\pgfqpoint{3.188069in}{1.251245in}}%
\pgfpathlineto{\pgfqpoint{3.188273in}{1.242994in}}%
\pgfpathlineto{\pgfqpoint{3.188477in}{1.003697in}}%
\pgfpathlineto{\pgfqpoint{3.188886in}{1.300755in}}%
\pgfpathlineto{\pgfqpoint{3.189294in}{1.300755in}}%
\pgfpathlineto{\pgfqpoint{3.189906in}{1.325510in}}%
\pgfpathlineto{\pgfqpoint{3.190519in}{1.251245in}}%
\pgfpathlineto{\pgfqpoint{3.190927in}{1.317258in}}%
\pgfpathlineto{\pgfqpoint{3.191335in}{1.226490in}}%
\pgfpathlineto{\pgfqpoint{3.191539in}{1.201736in}}%
\pgfpathlineto{\pgfqpoint{3.191744in}{1.226490in}}%
\pgfpathlineto{\pgfqpoint{3.192968in}{1.399774in}}%
\pgfpathlineto{\pgfqpoint{3.193172in}{1.391522in}}%
\pgfpathlineto{\pgfqpoint{3.193377in}{1.391522in}}%
\pgfpathlineto{\pgfqpoint{3.193581in}{1.358516in}}%
\pgfpathlineto{\pgfqpoint{3.193989in}{1.441032in}}%
\pgfpathlineto{\pgfqpoint{3.194193in}{1.449284in}}%
\pgfpathlineto{\pgfqpoint{3.194397in}{1.416277in}}%
\pgfpathlineto{\pgfqpoint{3.194601in}{1.416277in}}%
\pgfpathlineto{\pgfqpoint{3.195010in}{1.391522in}}%
\pgfpathlineto{\pgfqpoint{3.196030in}{1.284252in}}%
\pgfpathlineto{\pgfqpoint{3.196235in}{1.309006in}}%
\pgfpathlineto{\pgfqpoint{3.197051in}{1.234742in}}%
\pgfpathlineto{\pgfqpoint{3.196643in}{1.317258in}}%
\pgfpathlineto{\pgfqpoint{3.197255in}{1.300755in}}%
\pgfpathlineto{\pgfqpoint{3.197459in}{1.366768in}}%
\pgfpathlineto{\pgfqpoint{3.197868in}{1.317258in}}%
\pgfpathlineto{\pgfqpoint{3.198072in}{1.185232in}}%
\pgfpathlineto{\pgfqpoint{3.199093in}{1.218239in}}%
\pgfpathlineto{\pgfqpoint{3.199297in}{1.209987in}}%
\pgfpathlineto{\pgfqpoint{3.199501in}{1.242994in}}%
\pgfpathlineto{\pgfqpoint{3.199705in}{1.226490in}}%
\pgfpathlineto{\pgfqpoint{3.200522in}{1.350264in}}%
\pgfpathlineto{\pgfqpoint{3.200317in}{1.218239in}}%
\pgfpathlineto{\pgfqpoint{3.200930in}{1.333761in}}%
\pgfpathlineto{\pgfqpoint{3.201746in}{1.259497in}}%
\pgfpathlineto{\pgfqpoint{3.202563in}{1.432780in}}%
\pgfpathlineto{\pgfqpoint{3.202971in}{1.416277in}}%
\pgfpathlineto{\pgfqpoint{3.203175in}{1.375019in}}%
\pgfpathlineto{\pgfqpoint{3.203788in}{1.465787in}}%
\pgfpathlineto{\pgfqpoint{3.203992in}{1.416277in}}%
\pgfpathlineto{\pgfqpoint{3.204196in}{1.424529in}}%
\pgfpathlineto{\pgfqpoint{3.204400in}{1.399774in}}%
\pgfpathlineto{\pgfqpoint{3.205013in}{1.226490in}}%
\pgfpathlineto{\pgfqpoint{3.205625in}{1.292503in}}%
\pgfpathlineto{\pgfqpoint{3.206033in}{1.350264in}}%
\pgfpathlineto{\pgfqpoint{3.206238in}{1.325510in}}%
\pgfpathlineto{\pgfqpoint{3.207258in}{1.474038in}}%
\pgfpathlineto{\pgfqpoint{3.207462in}{1.465787in}}%
\pgfpathlineto{\pgfqpoint{3.208891in}{1.209987in}}%
\pgfpathlineto{\pgfqpoint{3.210525in}{1.408026in}}%
\pgfpathlineto{\pgfqpoint{3.211137in}{1.300755in}}%
\pgfpathlineto{\pgfqpoint{3.211545in}{1.399774in}}%
\pgfpathlineto{\pgfqpoint{3.211749in}{1.432780in}}%
\pgfpathlineto{\pgfqpoint{3.212158in}{1.375019in}}%
\pgfpathlineto{\pgfqpoint{3.212566in}{1.391522in}}%
\pgfpathlineto{\pgfqpoint{3.213587in}{1.457535in}}%
\pgfpathlineto{\pgfqpoint{3.213995in}{1.408026in}}%
\pgfpathlineto{\pgfqpoint{3.214199in}{1.408026in}}%
\pgfpathlineto{\pgfqpoint{3.215016in}{1.300755in}}%
\pgfpathlineto{\pgfqpoint{3.215220in}{1.366768in}}%
\pgfpathlineto{\pgfqpoint{3.215424in}{1.383271in}}%
\pgfpathlineto{\pgfqpoint{3.215832in}{1.375019in}}%
\pgfpathlineto{\pgfqpoint{3.216036in}{1.292503in}}%
\pgfpathlineto{\pgfqpoint{3.216649in}{1.441032in}}%
\pgfpathlineto{\pgfqpoint{3.217057in}{1.465787in}}%
\pgfpathlineto{\pgfqpoint{3.217261in}{1.424529in}}%
\pgfpathlineto{\pgfqpoint{3.218486in}{1.606064in}}%
\pgfpathlineto{\pgfqpoint{3.218690in}{1.531800in}}%
\pgfpathlineto{\pgfqpoint{3.219507in}{1.589561in}}%
\pgfpathlineto{\pgfqpoint{3.219915in}{1.622567in}}%
\pgfpathlineto{\pgfqpoint{3.220119in}{1.696832in}}%
\pgfpathlineto{\pgfqpoint{3.220323in}{1.474038in}}%
\pgfpathlineto{\pgfqpoint{3.221140in}{1.680329in}}%
\pgfpathlineto{\pgfqpoint{3.222365in}{1.589561in}}%
\pgfpathlineto{\pgfqpoint{3.222569in}{1.597813in}}%
\pgfpathlineto{\pgfqpoint{3.223590in}{1.721587in}}%
\pgfpathlineto{\pgfqpoint{3.223794in}{1.573058in}}%
\pgfpathlineto{\pgfqpoint{3.224610in}{1.696832in}}%
\pgfpathlineto{\pgfqpoint{3.224815in}{1.672077in}}%
\pgfpathlineto{\pgfqpoint{3.225019in}{1.721587in}}%
\pgfpathlineto{\pgfqpoint{3.225223in}{1.713335in}}%
\pgfpathlineto{\pgfqpoint{3.226039in}{1.804103in}}%
\pgfpathlineto{\pgfqpoint{3.226244in}{1.523548in}}%
\pgfpathlineto{\pgfqpoint{3.227060in}{1.878367in}}%
\pgfpathlineto{\pgfqpoint{3.228285in}{2.035148in}}%
\pgfpathlineto{\pgfqpoint{3.228489in}{1.828857in}}%
\pgfpathlineto{\pgfqpoint{3.229306in}{2.092909in}}%
\pgfpathlineto{\pgfqpoint{3.229714in}{2.117664in}}%
\pgfpathlineto{\pgfqpoint{3.230735in}{1.787599in}}%
\pgfpathlineto{\pgfqpoint{3.230939in}{1.870115in}}%
\pgfpathlineto{\pgfqpoint{3.231755in}{1.713335in}}%
\pgfpathlineto{\pgfqpoint{3.232164in}{2.018644in}}%
\pgfpathlineto{\pgfqpoint{3.232980in}{1.845361in}}%
\pgfpathlineto{\pgfqpoint{3.232776in}{2.043399in}}%
\pgfpathlineto{\pgfqpoint{3.233184in}{2.026896in}}%
\pgfpathlineto{\pgfqpoint{3.233389in}{1.993890in}}%
\pgfpathlineto{\pgfqpoint{3.233593in}{2.076406in}}%
\pgfpathlineto{\pgfqpoint{3.234205in}{2.018644in}}%
\pgfpathlineto{\pgfqpoint{3.234613in}{2.010393in}}%
\pgfpathlineto{\pgfqpoint{3.235430in}{1.837109in}}%
\pgfpathlineto{\pgfqpoint{3.235634in}{2.026896in}}%
\pgfpathlineto{\pgfqpoint{3.236859in}{1.927877in}}%
\pgfpathlineto{\pgfqpoint{3.238696in}{2.150670in}}%
\pgfpathlineto{\pgfqpoint{3.238900in}{2.117664in}}%
\pgfpathlineto{\pgfqpoint{3.239105in}{2.092909in}}%
\pgfpathlineto{\pgfqpoint{3.239513in}{2.142418in}}%
\pgfpathlineto{\pgfqpoint{3.239717in}{2.142418in}}%
\pgfpathlineto{\pgfqpoint{3.240738in}{2.092909in}}%
\pgfpathlineto{\pgfqpoint{3.240942in}{2.109412in}}%
\pgfpathlineto{\pgfqpoint{3.241554in}{2.208431in}}%
\pgfpathlineto{\pgfqpoint{3.242371in}{2.167173in}}%
\pgfpathlineto{\pgfqpoint{3.242575in}{2.167173in}}%
\pgfpathlineto{\pgfqpoint{3.242983in}{2.216683in}}%
\pgfpathlineto{\pgfqpoint{3.243187in}{2.142418in}}%
\pgfpathlineto{\pgfqpoint{3.243392in}{2.142418in}}%
\pgfpathlineto{\pgfqpoint{3.245025in}{2.200180in}}%
\pgfpathlineto{\pgfqpoint{3.245229in}{2.142418in}}%
\pgfpathlineto{\pgfqpoint{3.245841in}{2.257941in}}%
\pgfpathlineto{\pgfqpoint{3.246250in}{2.241438in}}%
\pgfpathlineto{\pgfqpoint{3.246454in}{2.249689in}}%
\pgfpathlineto{\pgfqpoint{3.246658in}{2.167173in}}%
\pgfpathlineto{\pgfqpoint{3.247270in}{2.299199in}}%
\pgfpathlineto{\pgfqpoint{3.247474in}{2.257941in}}%
\pgfpathlineto{\pgfqpoint{3.247679in}{2.249689in}}%
\pgfpathlineto{\pgfqpoint{3.247883in}{2.290947in}}%
\pgfpathlineto{\pgfqpoint{3.248087in}{2.142418in}}%
\pgfpathlineto{\pgfqpoint{3.248903in}{2.026896in}}%
\pgfpathlineto{\pgfqpoint{3.249108in}{2.134167in}}%
\pgfpathlineto{\pgfqpoint{3.250128in}{2.274444in}}%
\pgfpathlineto{\pgfqpoint{3.250332in}{2.266192in}}%
\pgfpathlineto{\pgfqpoint{3.250537in}{2.299199in}}%
\pgfpathlineto{\pgfqpoint{3.250741in}{2.249689in}}%
\pgfpathlineto{\pgfqpoint{3.250945in}{2.257941in}}%
\pgfpathlineto{\pgfqpoint{3.252170in}{2.051651in}}%
\pgfpathlineto{\pgfqpoint{3.253395in}{2.183676in}}%
\pgfpathlineto{\pgfqpoint{3.253599in}{1.944380in}}%
\pgfpathlineto{\pgfqpoint{3.254415in}{2.125915in}}%
\pgfpathlineto{\pgfqpoint{3.254619in}{2.134167in}}%
\pgfpathlineto{\pgfqpoint{3.255640in}{1.977386in}}%
\pgfpathlineto{\pgfqpoint{3.256048in}{2.051651in}}%
\pgfpathlineto{\pgfqpoint{3.256253in}{2.010393in}}%
\pgfpathlineto{\pgfqpoint{3.256457in}{2.059902in}}%
\pgfpathlineto{\pgfqpoint{3.256865in}{2.051651in}}%
\pgfpathlineto{\pgfqpoint{3.258090in}{2.216683in}}%
\pgfpathlineto{\pgfqpoint{3.258294in}{1.985638in}}%
\pgfpathlineto{\pgfqpoint{3.258906in}{2.299199in}}%
\pgfpathlineto{\pgfqpoint{3.259111in}{2.299199in}}%
\pgfpathlineto{\pgfqpoint{3.260335in}{2.084657in}}%
\pgfpathlineto{\pgfqpoint{3.260744in}{2.134167in}}%
\pgfpathlineto{\pgfqpoint{3.260948in}{2.134167in}}%
\pgfpathlineto{\pgfqpoint{3.261152in}{2.125915in}}%
\pgfpathlineto{\pgfqpoint{3.262581in}{2.249689in}}%
\pgfpathlineto{\pgfqpoint{3.263602in}{2.307450in}}%
\pgfpathlineto{\pgfqpoint{3.265031in}{2.084657in}}%
\pgfpathlineto{\pgfqpoint{3.265643in}{2.323954in}}%
\pgfpathlineto{\pgfqpoint{3.266256in}{2.307450in}}%
\pgfpathlineto{\pgfqpoint{3.266460in}{2.249689in}}%
\pgfpathlineto{\pgfqpoint{3.267072in}{2.323954in}}%
\pgfpathlineto{\pgfqpoint{3.267276in}{2.290947in}}%
\pgfpathlineto{\pgfqpoint{3.267480in}{2.299199in}}%
\pgfpathlineto{\pgfqpoint{3.267685in}{2.282696in}}%
\pgfpathlineto{\pgfqpoint{3.268093in}{2.282696in}}%
\pgfpathlineto{\pgfqpoint{3.268297in}{2.241438in}}%
\pgfpathlineto{\pgfqpoint{3.268705in}{2.315702in}}%
\pgfpathlineto{\pgfqpoint{3.269114in}{2.299199in}}%
\pgfpathlineto{\pgfqpoint{3.270134in}{2.398218in}}%
\pgfpathlineto{\pgfqpoint{3.270747in}{2.249689in}}%
\pgfpathlineto{\pgfqpoint{3.271563in}{2.282696in}}%
\pgfpathlineto{\pgfqpoint{3.272380in}{2.406470in}}%
\pgfpathlineto{\pgfqpoint{3.272788in}{2.315702in}}%
\pgfpathlineto{\pgfqpoint{3.272992in}{2.332205in}}%
\pgfpathlineto{\pgfqpoint{3.273196in}{2.274444in}}%
\pgfpathlineto{\pgfqpoint{3.273605in}{2.307450in}}%
\pgfpathlineto{\pgfqpoint{3.274013in}{2.266192in}}%
\pgfpathlineto{\pgfqpoint{3.274625in}{2.191928in}}%
\pgfpathlineto{\pgfqpoint{3.275034in}{2.266192in}}%
\pgfpathlineto{\pgfqpoint{3.276054in}{2.323954in}}%
\pgfpathlineto{\pgfqpoint{3.275850in}{2.249689in}}%
\pgfpathlineto{\pgfqpoint{3.276259in}{2.307450in}}%
\pgfpathlineto{\pgfqpoint{3.276463in}{2.299199in}}%
\pgfpathlineto{\pgfqpoint{3.276667in}{2.332205in}}%
\pgfpathlineto{\pgfqpoint{3.276871in}{2.018644in}}%
\pgfpathlineto{\pgfqpoint{3.277687in}{2.373463in}}%
\pgfpathlineto{\pgfqpoint{3.278708in}{2.398218in}}%
\pgfpathlineto{\pgfqpoint{3.278912in}{2.389967in}}%
\pgfpathlineto{\pgfqpoint{3.279729in}{2.274444in}}%
\pgfpathlineto{\pgfqpoint{3.280137in}{2.332205in}}%
\pgfpathlineto{\pgfqpoint{3.280341in}{2.356960in}}%
\pgfpathlineto{\pgfqpoint{3.280750in}{2.142418in}}%
\pgfpathlineto{\pgfqpoint{3.280954in}{2.406470in}}%
\pgfpathlineto{\pgfqpoint{3.281362in}{2.340457in}}%
\pgfpathlineto{\pgfqpoint{3.281566in}{2.340457in}}%
\pgfpathlineto{\pgfqpoint{3.282791in}{2.488986in}}%
\pgfpathlineto{\pgfqpoint{3.282179in}{2.282696in}}%
\pgfpathlineto{\pgfqpoint{3.283199in}{2.439476in}}%
\pgfpathlineto{\pgfqpoint{3.283608in}{2.472483in}}%
\pgfpathlineto{\pgfqpoint{3.283812in}{2.398218in}}%
\pgfpathlineto{\pgfqpoint{3.284016in}{2.266192in}}%
\pgfpathlineto{\pgfqpoint{3.284628in}{2.439476in}}%
\pgfpathlineto{\pgfqpoint{3.285037in}{2.480734in}}%
\pgfpathlineto{\pgfqpoint{3.285241in}{2.431225in}}%
\pgfpathlineto{\pgfqpoint{3.285649in}{2.439476in}}%
\pgfpathlineto{\pgfqpoint{3.285853in}{2.439476in}}%
\pgfpathlineto{\pgfqpoint{3.287078in}{2.340457in}}%
\pgfpathlineto{\pgfqpoint{3.287282in}{2.340457in}}%
\pgfpathlineto{\pgfqpoint{3.287895in}{2.332205in}}%
\pgfpathlineto{\pgfqpoint{3.288507in}{2.422973in}}%
\pgfpathlineto{\pgfqpoint{3.288711in}{2.472483in}}%
\pgfpathlineto{\pgfqpoint{3.289324in}{2.414721in}}%
\pgfpathlineto{\pgfqpoint{3.290344in}{2.282696in}}%
\pgfpathlineto{\pgfqpoint{3.290753in}{2.323954in}}%
\pgfpathlineto{\pgfqpoint{3.290957in}{2.323954in}}%
\pgfpathlineto{\pgfqpoint{3.291365in}{2.290947in}}%
\pgfpathlineto{\pgfqpoint{3.291773in}{2.348709in}}%
\pgfpathlineto{\pgfqpoint{3.291977in}{2.373463in}}%
\pgfpathlineto{\pgfqpoint{3.292386in}{2.307450in}}%
\pgfpathlineto{\pgfqpoint{3.292590in}{2.315702in}}%
\pgfpathlineto{\pgfqpoint{3.294019in}{2.439476in}}%
\pgfpathlineto{\pgfqpoint{3.294835in}{2.414721in}}%
\pgfpathlineto{\pgfqpoint{3.295448in}{2.323954in}}%
\pgfpathlineto{\pgfqpoint{3.296060in}{2.348709in}}%
\pgfpathlineto{\pgfqpoint{3.296264in}{2.348709in}}%
\pgfpathlineto{\pgfqpoint{3.296673in}{2.282696in}}%
\pgfpathlineto{\pgfqpoint{3.297285in}{2.365212in}}%
\pgfpathlineto{\pgfqpoint{3.297693in}{2.414721in}}%
\pgfpathlineto{\pgfqpoint{3.298306in}{2.365212in}}%
\pgfpathlineto{\pgfqpoint{3.298510in}{2.365212in}}%
\pgfpathlineto{\pgfqpoint{3.299531in}{2.290947in}}%
\pgfpathlineto{\pgfqpoint{3.299939in}{2.299199in}}%
\pgfpathlineto{\pgfqpoint{3.300551in}{2.373463in}}%
\pgfpathlineto{\pgfqpoint{3.300756in}{2.315702in}}%
\pgfpathlineto{\pgfqpoint{3.301572in}{2.167173in}}%
\pgfpathlineto{\pgfqpoint{3.301776in}{2.224934in}}%
\pgfpathlineto{\pgfqpoint{3.302593in}{2.290947in}}%
\pgfpathlineto{\pgfqpoint{3.302797in}{2.233186in}}%
\pgfpathlineto{\pgfqpoint{3.303001in}{2.233186in}}%
\pgfpathlineto{\pgfqpoint{3.303409in}{2.282696in}}%
\pgfpathlineto{\pgfqpoint{3.304022in}{2.348709in}}%
\pgfpathlineto{\pgfqpoint{3.304430in}{2.299199in}}%
\pgfpathlineto{\pgfqpoint{3.305451in}{2.373463in}}%
\pgfpathlineto{\pgfqpoint{3.305859in}{2.340457in}}%
\pgfpathlineto{\pgfqpoint{3.306880in}{2.266192in}}%
\pgfpathlineto{\pgfqpoint{3.307084in}{2.299199in}}%
\pgfpathlineto{\pgfqpoint{3.307288in}{2.356960in}}%
\pgfpathlineto{\pgfqpoint{3.307696in}{2.249689in}}%
\pgfpathlineto{\pgfqpoint{3.308309in}{2.332205in}}%
\pgfpathlineto{\pgfqpoint{3.308921in}{2.422973in}}%
\pgfpathlineto{\pgfqpoint{3.309330in}{2.332205in}}%
\pgfpathlineto{\pgfqpoint{3.310146in}{2.323954in}}%
\pgfpathlineto{\pgfqpoint{3.310554in}{2.348709in}}%
\pgfpathlineto{\pgfqpoint{3.310963in}{2.282696in}}%
\pgfpathlineto{\pgfqpoint{3.311779in}{2.315702in}}%
\pgfpathlineto{\pgfqpoint{3.312188in}{2.340457in}}%
\pgfpathlineto{\pgfqpoint{3.312596in}{2.224934in}}%
\pgfpathlineto{\pgfqpoint{3.312800in}{2.010393in}}%
\pgfpathlineto{\pgfqpoint{3.313617in}{2.224934in}}%
\pgfpathlineto{\pgfqpoint{3.313821in}{2.249689in}}%
\pgfpathlineto{\pgfqpoint{3.314229in}{2.233186in}}%
\pgfpathlineto{\pgfqpoint{3.314433in}{2.183676in}}%
\pgfpathlineto{\pgfqpoint{3.314637in}{2.249689in}}%
\pgfpathlineto{\pgfqpoint{3.314841in}{2.233186in}}%
\pgfpathlineto{\pgfqpoint{3.315454in}{2.356960in}}%
\pgfpathlineto{\pgfqpoint{3.316066in}{2.307450in}}%
\pgfpathlineto{\pgfqpoint{3.316270in}{2.315702in}}%
\pgfpathlineto{\pgfqpoint{3.316475in}{2.307450in}}%
\pgfpathlineto{\pgfqpoint{3.316883in}{2.068154in}}%
\pgfpathlineto{\pgfqpoint{3.317495in}{2.373463in}}%
\pgfpathlineto{\pgfqpoint{3.317699in}{2.200180in}}%
\pgfpathlineto{\pgfqpoint{3.318720in}{2.389967in}}%
\pgfpathlineto{\pgfqpoint{3.319128in}{2.348709in}}%
\pgfpathlineto{\pgfqpoint{3.319333in}{2.282696in}}%
\pgfpathlineto{\pgfqpoint{3.320149in}{2.356960in}}%
\pgfpathlineto{\pgfqpoint{3.320762in}{2.274444in}}%
\pgfpathlineto{\pgfqpoint{3.320966in}{2.290947in}}%
\pgfpathlineto{\pgfqpoint{3.321170in}{2.068154in}}%
\pgfpathlineto{\pgfqpoint{3.321578in}{2.307450in}}%
\pgfpathlineto{\pgfqpoint{3.321986in}{2.257941in}}%
\pgfpathlineto{\pgfqpoint{3.322191in}{2.299199in}}%
\pgfpathlineto{\pgfqpoint{3.322395in}{2.092909in}}%
\pgfpathlineto{\pgfqpoint{3.323007in}{2.422973in}}%
\pgfpathlineto{\pgfqpoint{3.323211in}{2.381715in}}%
\pgfpathlineto{\pgfqpoint{3.324436in}{2.323954in}}%
\pgfpathlineto{\pgfqpoint{3.325253in}{2.406470in}}%
\pgfpathlineto{\pgfqpoint{3.325457in}{2.389967in}}%
\pgfpathlineto{\pgfqpoint{3.325865in}{2.422973in}}%
\pgfpathlineto{\pgfqpoint{3.326478in}{2.332205in}}%
\pgfpathlineto{\pgfqpoint{3.327090in}{2.315702in}}%
\pgfpathlineto{\pgfqpoint{3.327702in}{2.389967in}}%
\pgfpathlineto{\pgfqpoint{3.328723in}{2.340457in}}%
\pgfpathlineto{\pgfqpoint{3.329131in}{2.348709in}}%
\pgfpathlineto{\pgfqpoint{3.329336in}{2.266192in}}%
\pgfpathlineto{\pgfqpoint{3.330356in}{2.282696in}}%
\pgfpathlineto{\pgfqpoint{3.331173in}{2.348709in}}%
\pgfpathlineto{\pgfqpoint{3.331581in}{2.332205in}}%
\pgfpathlineto{\pgfqpoint{3.331785in}{2.315702in}}%
\pgfpathlineto{\pgfqpoint{3.331989in}{2.340457in}}%
\pgfpathlineto{\pgfqpoint{3.333010in}{2.414721in}}%
\pgfpathlineto{\pgfqpoint{3.333214in}{2.373463in}}%
\pgfpathlineto{\pgfqpoint{3.334031in}{2.381715in}}%
\pgfpathlineto{\pgfqpoint{3.334847in}{2.340457in}}%
\pgfpathlineto{\pgfqpoint{3.335664in}{2.406470in}}%
\pgfpathlineto{\pgfqpoint{3.335868in}{2.348709in}}%
\pgfpathlineto{\pgfqpoint{3.336276in}{2.340457in}}%
\pgfpathlineto{\pgfqpoint{3.336481in}{2.348709in}}%
\pgfpathlineto{\pgfqpoint{3.337297in}{2.431225in}}%
\pgfpathlineto{\pgfqpoint{3.337501in}{2.406470in}}%
\pgfpathlineto{\pgfqpoint{3.337910in}{2.208431in}}%
\pgfpathlineto{\pgfqpoint{3.338726in}{2.348709in}}%
\pgfpathlineto{\pgfqpoint{3.339339in}{2.307450in}}%
\pgfpathlineto{\pgfqpoint{3.339134in}{2.365212in}}%
\pgfpathlineto{\pgfqpoint{3.339543in}{2.356960in}}%
\pgfpathlineto{\pgfqpoint{3.340155in}{2.348709in}}%
\pgfpathlineto{\pgfqpoint{3.340972in}{2.422973in}}%
\pgfpathlineto{\pgfqpoint{3.341176in}{2.422973in}}%
\pgfpathlineto{\pgfqpoint{3.341584in}{2.356960in}}%
\pgfpathlineto{\pgfqpoint{3.342197in}{2.422973in}}%
\pgfpathlineto{\pgfqpoint{3.343217in}{2.348709in}}%
\pgfpathlineto{\pgfqpoint{3.342605in}{2.447728in}}%
\pgfpathlineto{\pgfqpoint{3.343421in}{2.389967in}}%
\pgfpathlineto{\pgfqpoint{3.343626in}{2.389967in}}%
\pgfpathlineto{\pgfqpoint{3.344442in}{2.431225in}}%
\pgfpathlineto{\pgfqpoint{3.344238in}{2.365212in}}%
\pgfpathlineto{\pgfqpoint{3.344646in}{2.414721in}}%
\pgfpathlineto{\pgfqpoint{3.345259in}{2.340457in}}%
\pgfpathlineto{\pgfqpoint{3.345871in}{2.348709in}}%
\pgfpathlineto{\pgfqpoint{3.346688in}{2.422973in}}%
\pgfpathlineto{\pgfqpoint{3.346892in}{2.406470in}}%
\pgfpathlineto{\pgfqpoint{3.347913in}{2.332205in}}%
\pgfpathlineto{\pgfqpoint{3.348117in}{2.365212in}}%
\pgfpathlineto{\pgfqpoint{3.348321in}{2.373463in}}%
\pgfpathlineto{\pgfqpoint{3.349137in}{2.266192in}}%
\pgfpathlineto{\pgfqpoint{3.349546in}{2.290947in}}%
\pgfpathlineto{\pgfqpoint{3.349750in}{2.315702in}}%
\pgfpathlineto{\pgfqpoint{3.350362in}{2.266192in}}%
\pgfpathlineto{\pgfqpoint{3.350566in}{2.249689in}}%
\pgfpathlineto{\pgfqpoint{3.350771in}{2.315702in}}%
\pgfpathlineto{\pgfqpoint{3.350975in}{2.299199in}}%
\pgfpathlineto{\pgfqpoint{3.351791in}{2.373463in}}%
\pgfpathlineto{\pgfqpoint{3.352200in}{2.348709in}}%
\pgfpathlineto{\pgfqpoint{3.352812in}{2.389967in}}%
\pgfpathlineto{\pgfqpoint{3.353220in}{2.356960in}}%
\pgfpathlineto{\pgfqpoint{3.353424in}{2.348709in}}%
\pgfpathlineto{\pgfqpoint{3.353629in}{2.381715in}}%
\pgfpathlineto{\pgfqpoint{3.353833in}{2.455979in}}%
\pgfpathlineto{\pgfqpoint{3.354445in}{2.389967in}}%
\pgfpathlineto{\pgfqpoint{3.354853in}{2.315702in}}%
\pgfpathlineto{\pgfqpoint{3.355058in}{2.398218in}}%
\pgfpathlineto{\pgfqpoint{3.355466in}{2.398218in}}%
\pgfpathlineto{\pgfqpoint{3.355670in}{2.422973in}}%
\pgfpathlineto{\pgfqpoint{3.356078in}{2.373463in}}%
\pgfpathlineto{\pgfqpoint{3.357507in}{2.290947in}}%
\pgfpathlineto{\pgfqpoint{3.357711in}{2.307450in}}%
\pgfpathlineto{\pgfqpoint{3.357916in}{2.307450in}}%
\pgfpathlineto{\pgfqpoint{3.358120in}{2.299199in}}%
\pgfpathlineto{\pgfqpoint{3.358324in}{2.332205in}}%
\pgfpathlineto{\pgfqpoint{3.358528in}{2.332205in}}%
\pgfpathlineto{\pgfqpoint{3.358732in}{2.290947in}}%
\pgfpathlineto{\pgfqpoint{3.359140in}{2.365212in}}%
\pgfpathlineto{\pgfqpoint{3.359345in}{2.431225in}}%
\pgfpathlineto{\pgfqpoint{3.360365in}{2.414721in}}%
\pgfpathlineto{\pgfqpoint{3.360569in}{2.447728in}}%
\pgfpathlineto{\pgfqpoint{3.360978in}{2.389967in}}%
\pgfpathlineto{\pgfqpoint{3.361386in}{2.414721in}}%
\pgfpathlineto{\pgfqpoint{3.361998in}{2.422973in}}%
\pgfpathlineto{\pgfqpoint{3.362202in}{2.381715in}}%
\pgfpathlineto{\pgfqpoint{3.362815in}{2.472483in}}%
\pgfpathlineto{\pgfqpoint{3.363019in}{2.439476in}}%
\pgfpathlineto{\pgfqpoint{3.363836in}{2.480734in}}%
\pgfpathlineto{\pgfqpoint{3.364040in}{2.414721in}}%
\pgfpathlineto{\pgfqpoint{3.364856in}{2.505489in}}%
\pgfpathlineto{\pgfqpoint{3.365060in}{2.464231in}}%
\pgfpathlineto{\pgfqpoint{3.365265in}{2.530244in}}%
\pgfpathlineto{\pgfqpoint{3.365877in}{2.488986in}}%
\pgfpathlineto{\pgfqpoint{3.366081in}{2.505489in}}%
\pgfpathlineto{\pgfqpoint{3.366489in}{2.472483in}}%
\pgfpathlineto{\pgfqpoint{3.367102in}{2.398218in}}%
\pgfpathlineto{\pgfqpoint{3.367510in}{2.439476in}}%
\pgfpathlineto{\pgfqpoint{3.368735in}{2.513741in}}%
\pgfpathlineto{\pgfqpoint{3.369552in}{2.406470in}}%
\pgfpathlineto{\pgfqpoint{3.370164in}{2.472483in}}%
\pgfpathlineto{\pgfqpoint{3.370368in}{2.513741in}}%
\pgfpathlineto{\pgfqpoint{3.371185in}{2.472483in}}%
\pgfpathlineto{\pgfqpoint{3.372614in}{2.406470in}}%
\pgfpathlineto{\pgfqpoint{3.372818in}{2.389967in}}%
\pgfpathlineto{\pgfqpoint{3.373022in}{2.414721in}}%
\pgfpathlineto{\pgfqpoint{3.373430in}{2.505489in}}%
\pgfpathlineto{\pgfqpoint{3.374043in}{2.398218in}}%
\pgfpathlineto{\pgfqpoint{3.374247in}{2.398218in}}%
\pgfpathlineto{\pgfqpoint{3.374451in}{2.389967in}}%
\pgfpathlineto{\pgfqpoint{3.375268in}{2.513741in}}%
\pgfpathlineto{\pgfqpoint{3.375880in}{2.480734in}}%
\pgfpathlineto{\pgfqpoint{3.376084in}{2.521992in}}%
\pgfpathlineto{\pgfqpoint{3.376492in}{2.406470in}}%
\pgfpathlineto{\pgfqpoint{3.376697in}{2.389967in}}%
\pgfpathlineto{\pgfqpoint{3.376901in}{2.431225in}}%
\pgfpathlineto{\pgfqpoint{3.377105in}{2.422973in}}%
\pgfpathlineto{\pgfqpoint{3.378330in}{2.497237in}}%
\pgfpathlineto{\pgfqpoint{3.380167in}{2.398218in}}%
\pgfpathlineto{\pgfqpoint{3.380371in}{2.406470in}}%
\pgfpathlineto{\pgfqpoint{3.381392in}{2.464231in}}%
\pgfpathlineto{\pgfqpoint{3.381596in}{2.439476in}}%
\pgfpathlineto{\pgfqpoint{3.382617in}{2.323954in}}%
\pgfpathlineto{\pgfqpoint{3.382004in}{2.472483in}}%
\pgfpathlineto{\pgfqpoint{3.383025in}{2.332205in}}%
\pgfpathlineto{\pgfqpoint{3.383229in}{2.356960in}}%
\pgfpathlineto{\pgfqpoint{3.383637in}{2.299199in}}%
\pgfpathlineto{\pgfqpoint{3.383842in}{2.323954in}}%
\pgfpathlineto{\pgfqpoint{3.385066in}{2.290947in}}%
\pgfpathlineto{\pgfqpoint{3.386087in}{2.480734in}}%
\pgfpathlineto{\pgfqpoint{3.386700in}{2.274444in}}%
\pgfpathlineto{\pgfqpoint{3.387312in}{2.323954in}}%
\pgfpathlineto{\pgfqpoint{3.388741in}{2.398218in}}%
\pgfpathlineto{\pgfqpoint{3.388945in}{2.381715in}}%
\pgfpathlineto{\pgfqpoint{3.390578in}{2.117664in}}%
\pgfpathlineto{\pgfqpoint{3.391803in}{2.455979in}}%
\pgfpathlineto{\pgfqpoint{3.392007in}{2.447728in}}%
\pgfpathlineto{\pgfqpoint{3.392211in}{2.455979in}}%
\pgfpathlineto{\pgfqpoint{3.393436in}{2.538495in}}%
\pgfpathlineto{\pgfqpoint{3.393640in}{2.315702in}}%
\pgfpathlineto{\pgfqpoint{3.394457in}{2.439476in}}%
\pgfpathlineto{\pgfqpoint{3.395069in}{2.381715in}}%
\pgfpathlineto{\pgfqpoint{3.395274in}{2.200180in}}%
\pgfpathlineto{\pgfqpoint{3.395886in}{2.406470in}}%
\pgfpathlineto{\pgfqpoint{3.396090in}{2.373463in}}%
\pgfpathlineto{\pgfqpoint{3.396294in}{2.365212in}}%
\pgfpathlineto{\pgfqpoint{3.396498in}{2.373463in}}%
\pgfpathlineto{\pgfqpoint{3.397519in}{2.488986in}}%
\pgfpathlineto{\pgfqpoint{3.396907in}{2.200180in}}%
\pgfpathlineto{\pgfqpoint{3.397723in}{2.455979in}}%
\pgfpathlineto{\pgfqpoint{3.398132in}{2.158922in}}%
\pgfpathlineto{\pgfqpoint{3.398540in}{2.389967in}}%
\pgfpathlineto{\pgfqpoint{3.399561in}{2.546747in}}%
\pgfpathlineto{\pgfqpoint{3.399969in}{2.266192in}}%
\pgfpathlineto{\pgfqpoint{3.400785in}{2.455979in}}%
\pgfpathlineto{\pgfqpoint{3.400990in}{2.530244in}}%
\pgfpathlineto{\pgfqpoint{3.401806in}{2.447728in}}%
\pgfpathlineto{\pgfqpoint{3.402827in}{2.150670in}}%
\pgfpathlineto{\pgfqpoint{3.403031in}{2.290947in}}%
\pgfpathlineto{\pgfqpoint{3.403235in}{2.472483in}}%
\pgfpathlineto{\pgfqpoint{3.404256in}{2.439476in}}%
\pgfpathlineto{\pgfqpoint{3.404868in}{2.480734in}}%
\pgfpathlineto{\pgfqpoint{3.405481in}{2.356960in}}%
\pgfpathlineto{\pgfqpoint{3.405685in}{2.447728in}}%
\pgfpathlineto{\pgfqpoint{3.406093in}{2.200180in}}%
\pgfpathlineto{\pgfqpoint{3.406706in}{2.422973in}}%
\pgfpathlineto{\pgfqpoint{3.407726in}{2.365212in}}%
\pgfpathlineto{\pgfqpoint{3.407930in}{2.406470in}}%
\pgfpathlineto{\pgfqpoint{3.408543in}{2.455979in}}%
\pgfpathlineto{\pgfqpoint{3.408747in}{2.365212in}}%
\pgfpathlineto{\pgfqpoint{3.408951in}{2.332205in}}%
\pgfpathlineto{\pgfqpoint{3.409564in}{2.406470in}}%
\pgfpathlineto{\pgfqpoint{3.410788in}{2.340457in}}%
\pgfpathlineto{\pgfqpoint{3.410993in}{2.356960in}}%
\pgfpathlineto{\pgfqpoint{3.411401in}{2.398218in}}%
\pgfpathlineto{\pgfqpoint{3.411605in}{2.348709in}}%
\pgfpathlineto{\pgfqpoint{3.411809in}{2.257941in}}%
\pgfpathlineto{\pgfqpoint{3.412422in}{2.431225in}}%
\pgfpathlineto{\pgfqpoint{3.412626in}{2.447728in}}%
\pgfpathlineto{\pgfqpoint{3.412830in}{2.406470in}}%
\pgfpathlineto{\pgfqpoint{3.413238in}{2.422973in}}%
\pgfpathlineto{\pgfqpoint{3.414259in}{2.323954in}}%
\pgfpathlineto{\pgfqpoint{3.414463in}{2.373463in}}%
\pgfpathlineto{\pgfqpoint{3.415075in}{2.323954in}}%
\pgfpathlineto{\pgfqpoint{3.415280in}{2.373463in}}%
\pgfpathlineto{\pgfqpoint{3.416096in}{2.422973in}}%
\pgfpathlineto{\pgfqpoint{3.416913in}{2.340457in}}%
\pgfpathlineto{\pgfqpoint{3.417117in}{2.365212in}}%
\pgfpathlineto{\pgfqpoint{3.417525in}{2.398218in}}%
\pgfpathlineto{\pgfqpoint{3.417729in}{2.348709in}}%
\pgfpathlineto{\pgfqpoint{3.417933in}{2.208431in}}%
\pgfpathlineto{\pgfqpoint{3.418546in}{2.365212in}}%
\pgfpathlineto{\pgfqpoint{3.418750in}{2.323954in}}%
\pgfpathlineto{\pgfqpoint{3.418954in}{2.348709in}}%
\pgfpathlineto{\pgfqpoint{3.419158in}{2.167173in}}%
\pgfpathlineto{\pgfqpoint{3.419975in}{2.282696in}}%
\pgfpathlineto{\pgfqpoint{3.421200in}{2.381715in}}%
\pgfpathlineto{\pgfqpoint{3.421404in}{2.373463in}}%
\pgfpathlineto{\pgfqpoint{3.421812in}{2.356960in}}%
\pgfpathlineto{\pgfqpoint{3.422220in}{2.373463in}}%
\pgfpathlineto{\pgfqpoint{3.423241in}{2.414721in}}%
\pgfpathlineto{\pgfqpoint{3.423445in}{2.381715in}}%
\pgfpathlineto{\pgfqpoint{3.423649in}{2.439476in}}%
\pgfpathlineto{\pgfqpoint{3.423854in}{2.406470in}}%
\pgfpathlineto{\pgfqpoint{3.424670in}{2.472483in}}%
\pgfpathlineto{\pgfqpoint{3.424874in}{2.406470in}}%
\pgfpathlineto{\pgfqpoint{3.425283in}{2.381715in}}%
\pgfpathlineto{\pgfqpoint{3.425487in}{2.480734in}}%
\pgfpathlineto{\pgfqpoint{3.426303in}{2.389967in}}%
\pgfpathlineto{\pgfqpoint{3.427120in}{2.398218in}}%
\pgfpathlineto{\pgfqpoint{3.427732in}{2.340457in}}%
\pgfpathlineto{\pgfqpoint{3.428753in}{2.431225in}}%
\pgfpathlineto{\pgfqpoint{3.428141in}{2.282696in}}%
\pgfpathlineto{\pgfqpoint{3.428957in}{2.398218in}}%
\pgfpathlineto{\pgfqpoint{3.429161in}{2.315702in}}%
\pgfpathlineto{\pgfqpoint{3.429978in}{2.406470in}}%
\pgfpathlineto{\pgfqpoint{3.430590in}{2.381715in}}%
\pgfpathlineto{\pgfqpoint{3.430999in}{2.447728in}}%
\pgfpathlineto{\pgfqpoint{3.431203in}{2.406470in}}%
\pgfpathlineto{\pgfqpoint{3.431611in}{2.422973in}}%
\pgfpathlineto{\pgfqpoint{3.432223in}{2.315702in}}%
\pgfpathlineto{\pgfqpoint{3.432836in}{2.546747in}}%
\pgfpathlineto{\pgfqpoint{3.434673in}{2.315702in}}%
\pgfpathlineto{\pgfqpoint{3.434877in}{2.348709in}}%
\pgfpathlineto{\pgfqpoint{3.435286in}{2.274444in}}%
\pgfpathlineto{\pgfqpoint{3.436102in}{2.183676in}}%
\pgfpathlineto{\pgfqpoint{3.436510in}{2.224934in}}%
\pgfpathlineto{\pgfqpoint{3.436715in}{2.290947in}}%
\pgfpathlineto{\pgfqpoint{3.436919in}{2.216683in}}%
\pgfpathlineto{\pgfqpoint{3.437327in}{2.249689in}}%
\pgfpathlineto{\pgfqpoint{3.437531in}{2.167173in}}%
\pgfpathlineto{\pgfqpoint{3.438144in}{2.266192in}}%
\pgfpathlineto{\pgfqpoint{3.438348in}{2.241438in}}%
\pgfpathlineto{\pgfqpoint{3.438960in}{1.969135in}}%
\pgfpathlineto{\pgfqpoint{3.439573in}{2.191928in}}%
\pgfpathlineto{\pgfqpoint{3.439981in}{2.208431in}}%
\pgfpathlineto{\pgfqpoint{3.440185in}{2.134167in}}%
\pgfpathlineto{\pgfqpoint{3.440797in}{2.216683in}}%
\pgfpathlineto{\pgfqpoint{3.441002in}{2.191928in}}%
\pgfpathlineto{\pgfqpoint{3.442226in}{2.315702in}}%
\pgfpathlineto{\pgfqpoint{3.442431in}{2.257941in}}%
\pgfpathlineto{\pgfqpoint{3.443655in}{2.340457in}}%
\pgfpathlineto{\pgfqpoint{3.444676in}{2.224934in}}%
\pgfpathlineto{\pgfqpoint{3.444880in}{2.282696in}}%
\pgfpathlineto{\pgfqpoint{3.446717in}{2.084657in}}%
\pgfpathlineto{\pgfqpoint{3.447330in}{2.035148in}}%
\pgfpathlineto{\pgfqpoint{3.447534in}{2.059902in}}%
\pgfpathlineto{\pgfqpoint{3.448759in}{2.158922in}}%
\pgfpathlineto{\pgfqpoint{3.449780in}{2.101160in}}%
\pgfpathlineto{\pgfqpoint{3.449984in}{2.142418in}}%
\pgfpathlineto{\pgfqpoint{3.450392in}{2.068154in}}%
\pgfpathlineto{\pgfqpoint{3.450800in}{2.092909in}}%
\pgfpathlineto{\pgfqpoint{3.451413in}{2.167173in}}%
\pgfpathlineto{\pgfqpoint{3.451821in}{2.109412in}}%
\pgfpathlineto{\pgfqpoint{3.452025in}{2.084657in}}%
\pgfpathlineto{\pgfqpoint{3.452229in}{2.117664in}}%
\pgfpathlineto{\pgfqpoint{3.452433in}{2.175425in}}%
\pgfpathlineto{\pgfqpoint{3.453046in}{2.092909in}}%
\pgfpathlineto{\pgfqpoint{3.453454in}{2.076406in}}%
\pgfpathlineto{\pgfqpoint{3.453658in}{2.092909in}}%
\pgfpathlineto{\pgfqpoint{3.454679in}{2.257941in}}%
\pgfpathlineto{\pgfqpoint{3.455087in}{2.216683in}}%
\pgfpathlineto{\pgfqpoint{3.455496in}{1.993890in}}%
\pgfpathlineto{\pgfqpoint{3.455904in}{2.315702in}}%
\pgfpathlineto{\pgfqpoint{3.456516in}{2.274444in}}%
\pgfpathlineto{\pgfqpoint{3.456720in}{2.340457in}}%
\pgfpathlineto{\pgfqpoint{3.457129in}{2.249689in}}%
\pgfpathlineto{\pgfqpoint{3.457537in}{2.290947in}}%
\pgfpathlineto{\pgfqpoint{3.457741in}{2.299199in}}%
\pgfpathlineto{\pgfqpoint{3.457945in}{2.266192in}}%
\pgfpathlineto{\pgfqpoint{3.458149in}{2.282696in}}%
\pgfpathlineto{\pgfqpoint{3.458354in}{2.241438in}}%
\pgfpathlineto{\pgfqpoint{3.459170in}{2.266192in}}%
\pgfpathlineto{\pgfqpoint{3.459783in}{2.332205in}}%
\pgfpathlineto{\pgfqpoint{3.459987in}{2.266192in}}%
\pgfpathlineto{\pgfqpoint{3.461416in}{2.200180in}}%
\pgfpathlineto{\pgfqpoint{3.462028in}{2.068154in}}%
\pgfpathlineto{\pgfqpoint{3.462232in}{1.969135in}}%
\pgfpathlineto{\pgfqpoint{3.462436in}{2.233186in}}%
\pgfpathlineto{\pgfqpoint{3.462641in}{2.183676in}}%
\pgfpathlineto{\pgfqpoint{3.463049in}{2.241438in}}%
\pgfpathlineto{\pgfqpoint{3.463457in}{2.158922in}}%
\pgfpathlineto{\pgfqpoint{3.463865in}{2.233186in}}%
\pgfpathlineto{\pgfqpoint{3.464070in}{2.233186in}}%
\pgfpathlineto{\pgfqpoint{3.464274in}{2.208431in}}%
\pgfpathlineto{\pgfqpoint{3.464886in}{2.249689in}}%
\pgfpathlineto{\pgfqpoint{3.465294in}{2.290947in}}%
\pgfpathlineto{\pgfqpoint{3.465499in}{2.216683in}}%
\pgfpathlineto{\pgfqpoint{3.466315in}{2.266192in}}%
\pgfpathlineto{\pgfqpoint{3.466723in}{2.249689in}}%
\pgfpathlineto{\pgfqpoint{3.466928in}{2.266192in}}%
\pgfpathlineto{\pgfqpoint{3.468152in}{2.307450in}}%
\pgfpathlineto{\pgfqpoint{3.468765in}{2.208431in}}%
\pgfpathlineto{\pgfqpoint{3.468969in}{2.249689in}}%
\pgfpathlineto{\pgfqpoint{3.469786in}{2.365212in}}%
\pgfpathlineto{\pgfqpoint{3.469377in}{2.208431in}}%
\pgfpathlineto{\pgfqpoint{3.469990in}{2.323954in}}%
\pgfpathlineto{\pgfqpoint{3.471419in}{2.092909in}}%
\pgfpathlineto{\pgfqpoint{3.472031in}{2.282696in}}%
\pgfpathlineto{\pgfqpoint{3.472644in}{2.200180in}}%
\pgfpathlineto{\pgfqpoint{3.472848in}{2.233186in}}%
\pgfpathlineto{\pgfqpoint{3.473664in}{1.903122in}}%
\pgfpathlineto{\pgfqpoint{3.473868in}{2.018644in}}%
\pgfpathlineto{\pgfqpoint{3.474889in}{2.167173in}}%
\pgfpathlineto{\pgfqpoint{3.475502in}{2.018644in}}%
\pgfpathlineto{\pgfqpoint{3.476114in}{2.068154in}}%
\pgfpathlineto{\pgfqpoint{3.476726in}{2.018644in}}%
\pgfpathlineto{\pgfqpoint{3.476522in}{2.084657in}}%
\pgfpathlineto{\pgfqpoint{3.476931in}{2.043399in}}%
\pgfpathlineto{\pgfqpoint{3.477135in}{2.076406in}}%
\pgfpathlineto{\pgfqpoint{3.477543in}{2.010393in}}%
\pgfpathlineto{\pgfqpoint{3.478155in}{2.059902in}}%
\pgfpathlineto{\pgfqpoint{3.478768in}{1.845361in}}%
\pgfpathlineto{\pgfqpoint{3.479380in}{2.076406in}}%
\pgfpathlineto{\pgfqpoint{3.479584in}{1.903122in}}%
\pgfpathlineto{\pgfqpoint{3.479789in}{1.746341in}}%
\pgfpathlineto{\pgfqpoint{3.480401in}{2.076406in}}%
\pgfpathlineto{\pgfqpoint{3.480605in}{2.010393in}}%
\pgfpathlineto{\pgfqpoint{3.481218in}{2.092909in}}%
\pgfpathlineto{\pgfqpoint{3.481422in}{2.068154in}}%
\pgfpathlineto{\pgfqpoint{3.482034in}{2.010393in}}%
\pgfpathlineto{\pgfqpoint{3.482238in}{2.018644in}}%
\pgfpathlineto{\pgfqpoint{3.482442in}{1.729838in}}%
\pgfpathlineto{\pgfqpoint{3.483259in}{2.035148in}}%
\pgfpathlineto{\pgfqpoint{3.483667in}{2.018644in}}%
\pgfpathlineto{\pgfqpoint{3.483871in}{2.051651in}}%
\pgfpathlineto{\pgfqpoint{3.484076in}{1.993890in}}%
\pgfpathlineto{\pgfqpoint{3.484688in}{2.125915in}}%
\pgfpathlineto{\pgfqpoint{3.484892in}{2.191928in}}%
\pgfpathlineto{\pgfqpoint{3.485300in}{2.043399in}}%
\pgfpathlineto{\pgfqpoint{3.485709in}{2.101160in}}%
\pgfpathlineto{\pgfqpoint{3.485913in}{2.109412in}}%
\pgfpathlineto{\pgfqpoint{3.486117in}{2.068154in}}%
\pgfpathlineto{\pgfqpoint{3.486525in}{2.134167in}}%
\pgfpathlineto{\pgfqpoint{3.486934in}{2.101160in}}%
\pgfpathlineto{\pgfqpoint{3.487138in}{2.101160in}}%
\pgfpathlineto{\pgfqpoint{3.487342in}{2.150670in}}%
\pgfpathlineto{\pgfqpoint{3.487750in}{2.051651in}}%
\pgfpathlineto{\pgfqpoint{3.488158in}{2.109412in}}%
\pgfpathlineto{\pgfqpoint{3.488363in}{2.117664in}}%
\pgfpathlineto{\pgfqpoint{3.489383in}{1.919625in}}%
\pgfpathlineto{\pgfqpoint{3.489996in}{1.960883in}}%
\pgfpathlineto{\pgfqpoint{3.490200in}{1.944380in}}%
\pgfpathlineto{\pgfqpoint{3.490404in}{1.985638in}}%
\pgfpathlineto{\pgfqpoint{3.490608in}{2.043399in}}%
\pgfpathlineto{\pgfqpoint{3.491425in}{1.960883in}}%
\pgfpathlineto{\pgfqpoint{3.492037in}{1.919625in}}%
\pgfpathlineto{\pgfqpoint{3.492854in}{2.059902in}}%
\pgfpathlineto{\pgfqpoint{3.493262in}{2.035148in}}%
\pgfpathlineto{\pgfqpoint{3.494487in}{1.960883in}}%
\pgfpathlineto{\pgfqpoint{3.494895in}{1.985638in}}%
\pgfpathlineto{\pgfqpoint{3.495099in}{1.985638in}}%
\pgfpathlineto{\pgfqpoint{3.495303in}{2.002141in}}%
\pgfpathlineto{\pgfqpoint{3.495508in}{1.977386in}}%
\pgfpathlineto{\pgfqpoint{3.496120in}{1.911373in}}%
\pgfpathlineto{\pgfqpoint{3.496324in}{1.985638in}}%
\pgfpathlineto{\pgfqpoint{3.496528in}{1.969135in}}%
\pgfpathlineto{\pgfqpoint{3.496937in}{2.010393in}}%
\pgfpathlineto{\pgfqpoint{3.497141in}{1.903122in}}%
\pgfpathlineto{\pgfqpoint{3.497345in}{1.919625in}}%
\pgfpathlineto{\pgfqpoint{3.497549in}{1.886619in}}%
\pgfpathlineto{\pgfqpoint{3.498161in}{1.911373in}}%
\pgfpathlineto{\pgfqpoint{3.498774in}{1.804103in}}%
\pgfpathlineto{\pgfqpoint{3.499182in}{1.853612in}}%
\pgfpathlineto{\pgfqpoint{3.499795in}{1.804103in}}%
\pgfpathlineto{\pgfqpoint{3.499999in}{1.804103in}}%
\pgfpathlineto{\pgfqpoint{3.501019in}{1.870115in}}%
\pgfpathlineto{\pgfqpoint{3.500407in}{1.779348in}}%
\pgfpathlineto{\pgfqpoint{3.501428in}{1.820606in}}%
\pgfpathlineto{\pgfqpoint{3.502448in}{1.779348in}}%
\pgfpathlineto{\pgfqpoint{3.502040in}{1.861864in}}%
\pgfpathlineto{\pgfqpoint{3.502653in}{1.804103in}}%
\pgfpathlineto{\pgfqpoint{3.504286in}{1.977386in}}%
\pgfpathlineto{\pgfqpoint{3.505511in}{1.663825in}}%
\pgfpathlineto{\pgfqpoint{3.505715in}{1.820606in}}%
\pgfpathlineto{\pgfqpoint{3.506940in}{2.035148in}}%
\pgfpathlineto{\pgfqpoint{3.507144in}{2.010393in}}%
\pgfpathlineto{\pgfqpoint{3.508164in}{1.886619in}}%
\pgfpathlineto{\pgfqpoint{3.508369in}{1.903122in}}%
\pgfpathlineto{\pgfqpoint{3.509389in}{1.861864in}}%
\pgfpathlineto{\pgfqpoint{3.508981in}{1.911373in}}%
\pgfpathlineto{\pgfqpoint{3.509798in}{1.878367in}}%
\pgfpathlineto{\pgfqpoint{3.510002in}{1.886619in}}%
\pgfpathlineto{\pgfqpoint{3.510206in}{1.870115in}}%
\pgfpathlineto{\pgfqpoint{3.511022in}{1.804103in}}%
\pgfpathlineto{\pgfqpoint{3.511227in}{1.878367in}}%
\pgfpathlineto{\pgfqpoint{3.512043in}{1.779348in}}%
\pgfpathlineto{\pgfqpoint{3.512451in}{1.795851in}}%
\pgfpathlineto{\pgfqpoint{3.512860in}{1.754593in}}%
\pgfpathlineto{\pgfqpoint{3.513268in}{1.886619in}}%
\pgfpathlineto{\pgfqpoint{3.513676in}{1.696832in}}%
\pgfpathlineto{\pgfqpoint{3.513880in}{1.762845in}}%
\pgfpathlineto{\pgfqpoint{3.515309in}{1.556554in}}%
\pgfpathlineto{\pgfqpoint{3.515514in}{1.630819in}}%
\pgfpathlineto{\pgfqpoint{3.516126in}{1.606064in}}%
\pgfpathlineto{\pgfqpoint{3.516330in}{1.729838in}}%
\pgfpathlineto{\pgfqpoint{3.517147in}{1.614316in}}%
\pgfpathlineto{\pgfqpoint{3.517351in}{1.581309in}}%
\pgfpathlineto{\pgfqpoint{3.518167in}{1.597813in}}%
\pgfpathlineto{\pgfqpoint{3.519188in}{1.639071in}}%
\pgfpathlineto{\pgfqpoint{3.519801in}{1.647322in}}%
\pgfpathlineto{\pgfqpoint{3.520005in}{1.581309in}}%
\pgfpathlineto{\pgfqpoint{3.521230in}{1.705083in}}%
\pgfpathlineto{\pgfqpoint{3.521842in}{1.630819in}}%
\pgfpathlineto{\pgfqpoint{3.522250in}{1.705083in}}%
\pgfpathlineto{\pgfqpoint{3.522454in}{1.705083in}}%
\pgfpathlineto{\pgfqpoint{3.523271in}{1.589561in}}%
\pgfpathlineto{\pgfqpoint{3.523883in}{1.630819in}}%
\pgfpathlineto{\pgfqpoint{3.524088in}{1.680329in}}%
\pgfpathlineto{\pgfqpoint{3.524292in}{1.630819in}}%
\pgfpathlineto{\pgfqpoint{3.524496in}{1.383271in}}%
\pgfpathlineto{\pgfqpoint{3.525312in}{1.564806in}}%
\pgfpathlineto{\pgfqpoint{3.525517in}{1.606064in}}%
\pgfpathlineto{\pgfqpoint{3.525721in}{1.490542in}}%
\pgfpathlineto{\pgfqpoint{3.526129in}{1.507045in}}%
\pgfpathlineto{\pgfqpoint{3.526741in}{1.540051in}}%
\pgfpathlineto{\pgfqpoint{3.526946in}{1.490542in}}%
\pgfpathlineto{\pgfqpoint{3.527150in}{1.515296in}}%
\pgfpathlineto{\pgfqpoint{3.528375in}{1.350264in}}%
\pgfpathlineto{\pgfqpoint{3.529599in}{1.515296in}}%
\pgfpathlineto{\pgfqpoint{3.530212in}{1.498793in}}%
\pgfpathlineto{\pgfqpoint{3.530416in}{1.531800in}}%
\pgfpathlineto{\pgfqpoint{3.530620in}{1.523548in}}%
\pgfpathlineto{\pgfqpoint{3.532049in}{1.647322in}}%
\pgfpathlineto{\pgfqpoint{3.532866in}{1.556554in}}%
\pgfpathlineto{\pgfqpoint{3.533886in}{1.705083in}}%
\pgfpathlineto{\pgfqpoint{3.534090in}{1.663825in}}%
\pgfpathlineto{\pgfqpoint{3.534295in}{1.729838in}}%
\pgfpathlineto{\pgfqpoint{3.534499in}{1.630819in}}%
\pgfpathlineto{\pgfqpoint{3.535111in}{1.647322in}}%
\pgfpathlineto{\pgfqpoint{3.536540in}{1.738090in}}%
\pgfpathlineto{\pgfqpoint{3.536744in}{1.705083in}}%
\pgfpathlineto{\pgfqpoint{3.537969in}{1.564806in}}%
\pgfpathlineto{\pgfqpoint{3.537153in}{1.713335in}}%
\pgfpathlineto{\pgfqpoint{3.538582in}{1.597813in}}%
\pgfpathlineto{\pgfqpoint{3.539398in}{1.696832in}}%
\pgfpathlineto{\pgfqpoint{3.539806in}{1.655574in}}%
\pgfpathlineto{\pgfqpoint{3.540215in}{1.655574in}}%
\pgfpathlineto{\pgfqpoint{3.541031in}{1.705083in}}%
\pgfpathlineto{\pgfqpoint{3.540623in}{1.639071in}}%
\pgfpathlineto{\pgfqpoint{3.541440in}{1.663825in}}%
\pgfpathlineto{\pgfqpoint{3.541644in}{1.663825in}}%
\pgfpathlineto{\pgfqpoint{3.541848in}{1.672077in}}%
\pgfpathlineto{\pgfqpoint{3.542052in}{1.663825in}}%
\pgfpathlineto{\pgfqpoint{3.542256in}{1.762845in}}%
\pgfpathlineto{\pgfqpoint{3.543073in}{1.672077in}}%
\pgfpathlineto{\pgfqpoint{3.543481in}{1.647322in}}%
\pgfpathlineto{\pgfqpoint{3.543685in}{1.672077in}}%
\pgfpathlineto{\pgfqpoint{3.544502in}{1.713335in}}%
\pgfpathlineto{\pgfqpoint{3.544706in}{1.696832in}}%
\pgfpathlineto{\pgfqpoint{3.545522in}{1.647322in}}%
\pgfpathlineto{\pgfqpoint{3.545727in}{1.672077in}}%
\pgfpathlineto{\pgfqpoint{3.546543in}{1.738090in}}%
\pgfpathlineto{\pgfqpoint{3.546135in}{1.622567in}}%
\pgfpathlineto{\pgfqpoint{3.546747in}{1.705083in}}%
\pgfpathlineto{\pgfqpoint{3.547564in}{1.663825in}}%
\pgfpathlineto{\pgfqpoint{3.547360in}{1.713335in}}%
\pgfpathlineto{\pgfqpoint{3.547768in}{1.680329in}}%
\pgfpathlineto{\pgfqpoint{3.549401in}{1.853612in}}%
\pgfpathlineto{\pgfqpoint{3.550626in}{1.754593in}}%
\pgfpathlineto{\pgfqpoint{3.553076in}{1.936128in}}%
\pgfpathlineto{\pgfqpoint{3.553688in}{1.820606in}}%
\pgfpathlineto{\pgfqpoint{3.554505in}{1.886619in}}%
\pgfpathlineto{\pgfqpoint{3.555117in}{1.853612in}}%
\pgfpathlineto{\pgfqpoint{3.554913in}{1.894870in}}%
\pgfpathlineto{\pgfqpoint{3.555321in}{1.878367in}}%
\pgfpathlineto{\pgfqpoint{3.555525in}{1.894870in}}%
\pgfpathlineto{\pgfqpoint{3.555730in}{1.853612in}}%
\pgfpathlineto{\pgfqpoint{3.556954in}{1.721587in}}%
\pgfpathlineto{\pgfqpoint{3.556138in}{1.919625in}}%
\pgfpathlineto{\pgfqpoint{3.557159in}{1.779348in}}%
\pgfpathlineto{\pgfqpoint{3.557567in}{1.820606in}}%
\pgfpathlineto{\pgfqpoint{3.557771in}{1.779348in}}%
\pgfpathlineto{\pgfqpoint{3.558996in}{1.647322in}}%
\pgfpathlineto{\pgfqpoint{3.559200in}{1.705083in}}%
\pgfpathlineto{\pgfqpoint{3.560017in}{1.655574in}}%
\pgfpathlineto{\pgfqpoint{3.560629in}{1.597813in}}%
\pgfpathlineto{\pgfqpoint{3.560833in}{1.688580in}}%
\pgfpathlineto{\pgfqpoint{3.561037in}{1.622567in}}%
\pgfpathlineto{\pgfqpoint{3.561446in}{1.713335in}}%
\pgfpathlineto{\pgfqpoint{3.561854in}{1.573058in}}%
\pgfpathlineto{\pgfqpoint{3.562670in}{1.597813in}}%
\pgfpathlineto{\pgfqpoint{3.563691in}{1.531800in}}%
\pgfpathlineto{\pgfqpoint{3.563283in}{1.622567in}}%
\pgfpathlineto{\pgfqpoint{3.563895in}{1.573058in}}%
\pgfpathlineto{\pgfqpoint{3.564304in}{1.663825in}}%
\pgfpathlineto{\pgfqpoint{3.565120in}{1.630819in}}%
\pgfpathlineto{\pgfqpoint{3.565324in}{1.630819in}}%
\pgfpathlineto{\pgfqpoint{3.566549in}{1.729838in}}%
\pgfpathlineto{\pgfqpoint{3.566753in}{1.721587in}}%
\pgfpathlineto{\pgfqpoint{3.566957in}{1.738090in}}%
\pgfpathlineto{\pgfqpoint{3.567162in}{1.696832in}}%
\pgfpathlineto{\pgfqpoint{3.567570in}{1.713335in}}%
\pgfpathlineto{\pgfqpoint{3.568386in}{1.647322in}}%
\pgfpathlineto{\pgfqpoint{3.568591in}{1.622567in}}%
\pgfpathlineto{\pgfqpoint{3.568795in}{1.672077in}}%
\pgfpathlineto{\pgfqpoint{3.568999in}{1.655574in}}%
\pgfpathlineto{\pgfqpoint{3.570428in}{1.812354in}}%
\pgfpathlineto{\pgfqpoint{3.571857in}{1.655574in}}%
\pgfpathlineto{\pgfqpoint{3.572061in}{1.663825in}}%
\pgfpathlineto{\pgfqpoint{3.572265in}{1.630819in}}%
\pgfpathlineto{\pgfqpoint{3.572469in}{1.622567in}}%
\pgfpathlineto{\pgfqpoint{3.573694in}{1.729838in}}%
\pgfpathlineto{\pgfqpoint{3.573898in}{1.688580in}}%
\pgfpathlineto{\pgfqpoint{3.574511in}{1.738090in}}%
\pgfpathlineto{\pgfqpoint{3.575327in}{1.812354in}}%
\pgfpathlineto{\pgfqpoint{3.575531in}{1.738090in}}%
\pgfpathlineto{\pgfqpoint{3.576960in}{1.655574in}}%
\pgfpathlineto{\pgfqpoint{3.577369in}{1.771096in}}%
\pgfpathlineto{\pgfqpoint{3.578594in}{1.729838in}}%
\pgfpathlineto{\pgfqpoint{3.580023in}{1.589561in}}%
\pgfpathlineto{\pgfqpoint{3.579002in}{1.738090in}}%
\pgfpathlineto{\pgfqpoint{3.580431in}{1.630819in}}%
\pgfpathlineto{\pgfqpoint{3.582268in}{1.820606in}}%
\pgfpathlineto{\pgfqpoint{3.583697in}{1.647322in}}%
\pgfpathlineto{\pgfqpoint{3.583901in}{1.680329in}}%
\pgfpathlineto{\pgfqpoint{3.584105in}{1.680329in}}%
\pgfpathlineto{\pgfqpoint{3.584514in}{1.663825in}}%
\pgfpathlineto{\pgfqpoint{3.585330in}{1.754593in}}%
\pgfpathlineto{\pgfqpoint{3.586351in}{1.614316in}}%
\pgfpathlineto{\pgfqpoint{3.586759in}{1.647322in}}%
\pgfpathlineto{\pgfqpoint{3.588188in}{1.738090in}}%
\pgfpathlineto{\pgfqpoint{3.589209in}{1.622567in}}%
\pgfpathlineto{\pgfqpoint{3.589413in}{1.655574in}}%
\pgfpathlineto{\pgfqpoint{3.590434in}{1.804103in}}%
\pgfpathlineto{\pgfqpoint{3.590842in}{1.787599in}}%
\pgfpathlineto{\pgfqpoint{3.591659in}{1.696832in}}%
\pgfpathlineto{\pgfqpoint{3.591863in}{1.738090in}}%
\pgfpathlineto{\pgfqpoint{3.592067in}{1.787599in}}%
\pgfpathlineto{\pgfqpoint{3.592271in}{1.663825in}}%
\pgfpathlineto{\pgfqpoint{3.592475in}{1.688580in}}%
\pgfpathlineto{\pgfqpoint{3.593088in}{1.581309in}}%
\pgfpathlineto{\pgfqpoint{3.593700in}{1.639071in}}%
\pgfpathlineto{\pgfqpoint{3.594108in}{1.680329in}}%
\pgfpathlineto{\pgfqpoint{3.594517in}{1.630819in}}%
\pgfpathlineto{\pgfqpoint{3.595537in}{1.597813in}}%
\pgfpathlineto{\pgfqpoint{3.595742in}{1.639071in}}%
\pgfpathlineto{\pgfqpoint{3.596150in}{1.556554in}}%
\pgfpathlineto{\pgfqpoint{3.596354in}{1.606064in}}%
\pgfpathlineto{\pgfqpoint{3.596558in}{1.573058in}}%
\pgfpathlineto{\pgfqpoint{3.596762in}{1.630819in}}%
\pgfpathlineto{\pgfqpoint{3.597375in}{1.581309in}}%
\pgfpathlineto{\pgfqpoint{3.597579in}{1.672077in}}%
\pgfpathlineto{\pgfqpoint{3.598600in}{1.647322in}}%
\pgfpathlineto{\pgfqpoint{3.598804in}{1.622567in}}%
\pgfpathlineto{\pgfqpoint{3.599008in}{1.696832in}}%
\pgfpathlineto{\pgfqpoint{3.599212in}{1.696832in}}%
\pgfpathlineto{\pgfqpoint{3.599416in}{1.771096in}}%
\pgfpathlineto{\pgfqpoint{3.600029in}{1.655574in}}%
\pgfpathlineto{\pgfqpoint{3.600233in}{1.721587in}}%
\pgfpathlineto{\pgfqpoint{3.601253in}{1.589561in}}%
\pgfpathlineto{\pgfqpoint{3.601458in}{1.639071in}}%
\pgfpathlineto{\pgfqpoint{3.601662in}{1.647322in}}%
\pgfpathlineto{\pgfqpoint{3.601866in}{1.482290in}}%
\pgfpathlineto{\pgfqpoint{3.602478in}{1.721587in}}%
\pgfpathlineto{\pgfqpoint{3.602682in}{1.729838in}}%
\pgfpathlineto{\pgfqpoint{3.602887in}{1.713335in}}%
\pgfpathlineto{\pgfqpoint{3.603499in}{1.630819in}}%
\pgfpathlineto{\pgfqpoint{3.604316in}{1.639071in}}%
\pgfpathlineto{\pgfqpoint{3.604724in}{1.696832in}}%
\pgfpathlineto{\pgfqpoint{3.604928in}{1.366768in}}%
\pgfpathlineto{\pgfqpoint{3.605745in}{1.639071in}}%
\pgfpathlineto{\pgfqpoint{3.606153in}{1.762845in}}%
\pgfpathlineto{\pgfqpoint{3.606357in}{1.556554in}}%
\pgfpathlineto{\pgfqpoint{3.606969in}{1.713335in}}%
\pgfpathlineto{\pgfqpoint{3.607582in}{1.771096in}}%
\pgfpathlineto{\pgfqpoint{3.607786in}{1.705083in}}%
\pgfpathlineto{\pgfqpoint{3.607990in}{1.729838in}}%
\pgfpathlineto{\pgfqpoint{3.608398in}{1.705083in}}%
\pgfpathlineto{\pgfqpoint{3.608807in}{1.754593in}}%
\pgfpathlineto{\pgfqpoint{3.609011in}{1.771096in}}%
\pgfpathlineto{\pgfqpoint{3.609215in}{1.713335in}}%
\pgfpathlineto{\pgfqpoint{3.609623in}{1.738090in}}%
\pgfpathlineto{\pgfqpoint{3.609827in}{1.564806in}}%
\pgfpathlineto{\pgfqpoint{3.610236in}{1.754593in}}%
\pgfpathlineto{\pgfqpoint{3.610644in}{1.672077in}}%
\pgfpathlineto{\pgfqpoint{3.611665in}{1.779348in}}%
\pgfpathlineto{\pgfqpoint{3.611869in}{1.729838in}}%
\pgfpathlineto{\pgfqpoint{3.612481in}{1.441032in}}%
\pgfpathlineto{\pgfqpoint{3.613094in}{1.672077in}}%
\pgfpathlineto{\pgfqpoint{3.613298in}{1.754593in}}%
\pgfpathlineto{\pgfqpoint{3.613910in}{1.663825in}}%
\pgfpathlineto{\pgfqpoint{3.614114in}{1.663825in}}%
\pgfpathlineto{\pgfqpoint{3.614319in}{1.432780in}}%
\pgfpathlineto{\pgfqpoint{3.614931in}{1.754593in}}%
\pgfpathlineto{\pgfqpoint{3.615135in}{1.705083in}}%
\pgfpathlineto{\pgfqpoint{3.615339in}{1.729838in}}%
\pgfpathlineto{\pgfqpoint{3.615952in}{1.688580in}}%
\pgfpathlineto{\pgfqpoint{3.616156in}{1.680329in}}%
\pgfpathlineto{\pgfqpoint{3.616564in}{1.696832in}}%
\pgfpathlineto{\pgfqpoint{3.617585in}{1.820606in}}%
\pgfpathlineto{\pgfqpoint{3.617993in}{1.779348in}}%
\pgfpathlineto{\pgfqpoint{3.618197in}{1.564806in}}%
\pgfpathlineto{\pgfqpoint{3.619014in}{1.721587in}}%
\pgfpathlineto{\pgfqpoint{3.619218in}{1.762845in}}%
\pgfpathlineto{\pgfqpoint{3.619422in}{1.705083in}}%
\pgfpathlineto{\pgfqpoint{3.619830in}{1.713335in}}%
\pgfpathlineto{\pgfqpoint{3.620443in}{1.729838in}}%
\pgfpathlineto{\pgfqpoint{3.621055in}{1.663825in}}%
\pgfpathlineto{\pgfqpoint{3.622892in}{1.845361in}}%
\pgfpathlineto{\pgfqpoint{3.621463in}{1.622567in}}%
\pgfpathlineto{\pgfqpoint{3.623097in}{1.837109in}}%
\pgfpathlineto{\pgfqpoint{3.624526in}{1.515296in}}%
\pgfpathlineto{\pgfqpoint{3.624730in}{1.498793in}}%
\pgfpathlineto{\pgfqpoint{3.624934in}{1.556554in}}%
\pgfpathlineto{\pgfqpoint{3.625138in}{1.564806in}}%
\pgfpathlineto{\pgfqpoint{3.625546in}{1.490542in}}%
\pgfpathlineto{\pgfqpoint{3.626159in}{1.540051in}}%
\pgfpathlineto{\pgfqpoint{3.626567in}{1.573058in}}%
\pgfpathlineto{\pgfqpoint{3.627588in}{1.408026in}}%
\pgfpathlineto{\pgfqpoint{3.627792in}{1.457535in}}%
\pgfpathlineto{\pgfqpoint{3.629425in}{1.309006in}}%
\pgfpathlineto{\pgfqpoint{3.628608in}{1.498793in}}%
\pgfpathlineto{\pgfqpoint{3.629629in}{1.333761in}}%
\pgfpathlineto{\pgfqpoint{3.629833in}{1.309006in}}%
\pgfpathlineto{\pgfqpoint{3.630037in}{1.366768in}}%
\pgfpathlineto{\pgfqpoint{3.630446in}{1.350264in}}%
\pgfpathlineto{\pgfqpoint{3.630650in}{1.375019in}}%
\pgfpathlineto{\pgfqpoint{3.630854in}{1.309006in}}%
\pgfpathlineto{\pgfqpoint{3.631875in}{1.102716in}}%
\pgfpathlineto{\pgfqpoint{3.632487in}{1.152226in}}%
\pgfpathlineto{\pgfqpoint{3.632691in}{1.383271in}}%
\pgfpathlineto{\pgfqpoint{3.633712in}{1.284252in}}%
\pgfpathlineto{\pgfqpoint{3.633916in}{1.408026in}}%
\pgfpathlineto{\pgfqpoint{3.634733in}{1.325510in}}%
\pgfpathlineto{\pgfqpoint{3.635753in}{1.110968in}}%
\pgfpathlineto{\pgfqpoint{3.636366in}{1.226490in}}%
\pgfpathlineto{\pgfqpoint{3.636570in}{1.209987in}}%
\pgfpathlineto{\pgfqpoint{3.636774in}{1.276000in}}%
\pgfpathlineto{\pgfqpoint{3.636978in}{1.317258in}}%
\pgfpathlineto{\pgfqpoint{3.637182in}{1.242994in}}%
\pgfpathlineto{\pgfqpoint{3.637591in}{1.044955in}}%
\pgfpathlineto{\pgfqpoint{3.638407in}{1.135723in}}%
\pgfpathlineto{\pgfqpoint{3.638816in}{1.176981in}}%
\pgfpathlineto{\pgfqpoint{3.640245in}{0.896426in}}%
\pgfpathlineto{\pgfqpoint{3.640449in}{0.921181in}}%
\pgfpathlineto{\pgfqpoint{3.640857in}{0.912929in}}%
\pgfpathlineto{\pgfqpoint{3.641674in}{0.731394in}}%
\pgfpathlineto{\pgfqpoint{3.641878in}{0.747897in}}%
\pgfpathlineto{\pgfqpoint{3.643103in}{1.003697in}}%
\pgfpathlineto{\pgfqpoint{3.643511in}{1.077961in}}%
\pgfpathlineto{\pgfqpoint{3.644327in}{0.896426in}}%
\pgfpathlineto{\pgfqpoint{3.645144in}{1.152226in}}%
\pgfpathlineto{\pgfqpoint{3.645552in}{0.995445in}}%
\pgfpathlineto{\pgfqpoint{3.645756in}{1.011949in}}%
\pgfpathlineto{\pgfqpoint{3.646165in}{0.648878in}}%
\pgfpathlineto{\pgfqpoint{3.646573in}{1.069710in}}%
\pgfpathlineto{\pgfqpoint{3.646777in}{1.028452in}}%
\pgfpathlineto{\pgfqpoint{3.647185in}{0.921181in}}%
\pgfpathlineto{\pgfqpoint{3.647594in}{1.094465in}}%
\pgfpathlineto{\pgfqpoint{3.647798in}{1.036703in}}%
\pgfpathlineto{\pgfqpoint{3.648206in}{1.127471in}}%
\pgfpathlineto{\pgfqpoint{3.648819in}{1.044955in}}%
\pgfpathlineto{\pgfqpoint{3.649431in}{0.987194in}}%
\pgfpathlineto{\pgfqpoint{3.649839in}{0.995445in}}%
\pgfpathlineto{\pgfqpoint{3.650043in}{1.028452in}}%
\pgfpathlineto{\pgfqpoint{3.650248in}{1.011949in}}%
\pgfpathlineto{\pgfqpoint{3.651472in}{0.813910in}}%
\pgfpathlineto{\pgfqpoint{3.651677in}{0.855168in}}%
\pgfpathlineto{\pgfqpoint{3.651881in}{0.797407in}}%
\pgfpathlineto{\pgfqpoint{3.652493in}{0.879923in}}%
\pgfpathlineto{\pgfqpoint{3.652697in}{0.871671in}}%
\pgfpathlineto{\pgfqpoint{3.653718in}{1.135723in}}%
\pgfpathlineto{\pgfqpoint{3.653922in}{1.094465in}}%
\pgfpathlineto{\pgfqpoint{3.655147in}{0.846917in}}%
\pgfpathlineto{\pgfqpoint{3.656372in}{1.218239in}}%
\pgfpathlineto{\pgfqpoint{3.656780in}{1.069710in}}%
\pgfpathlineto{\pgfqpoint{3.657393in}{1.143974in}}%
\pgfpathlineto{\pgfqpoint{3.658209in}{1.259497in}}%
\pgfpathlineto{\pgfqpoint{3.658413in}{1.069710in}}%
\pgfpathlineto{\pgfqpoint{3.659230in}{1.218239in}}%
\pgfpathlineto{\pgfqpoint{3.659842in}{1.317258in}}%
\pgfpathlineto{\pgfqpoint{3.660046in}{1.185232in}}%
\pgfpathlineto{\pgfqpoint{3.660251in}{1.185232in}}%
\pgfpathlineto{\pgfqpoint{3.660455in}{1.226490in}}%
\pgfpathlineto{\pgfqpoint{3.660863in}{1.127471in}}%
\pgfpathlineto{\pgfqpoint{3.661067in}{1.044955in}}%
\pgfpathlineto{\pgfqpoint{3.661884in}{1.160477in}}%
\pgfpathlineto{\pgfqpoint{3.662088in}{1.135723in}}%
\pgfpathlineto{\pgfqpoint{3.662496in}{1.201736in}}%
\pgfpathlineto{\pgfqpoint{3.662700in}{1.242994in}}%
\pgfpathlineto{\pgfqpoint{3.663109in}{1.135723in}}%
\pgfpathlineto{\pgfqpoint{3.663313in}{1.201736in}}%
\pgfpathlineto{\pgfqpoint{3.663517in}{1.044955in}}%
\pgfpathlineto{\pgfqpoint{3.664333in}{1.152226in}}%
\pgfpathlineto{\pgfqpoint{3.664538in}{1.259497in}}%
\pgfpathlineto{\pgfqpoint{3.664946in}{1.094465in}}%
\pgfpathlineto{\pgfqpoint{3.665354in}{1.185232in}}%
\pgfpathlineto{\pgfqpoint{3.666171in}{1.143974in}}%
\pgfpathlineto{\pgfqpoint{3.666579in}{1.168729in}}%
\pgfpathlineto{\pgfqpoint{3.667191in}{1.135723in}}%
\pgfpathlineto{\pgfqpoint{3.667396in}{1.176981in}}%
\pgfpathlineto{\pgfqpoint{3.668212in}{1.259497in}}%
\pgfpathlineto{\pgfqpoint{3.669029in}{1.234742in}}%
\pgfpathlineto{\pgfqpoint{3.669233in}{1.160477in}}%
\pgfpathlineto{\pgfqpoint{3.669845in}{1.276000in}}%
\pgfpathlineto{\pgfqpoint{3.670049in}{1.259497in}}%
\pgfpathlineto{\pgfqpoint{3.671478in}{1.358516in}}%
\pgfpathlineto{\pgfqpoint{3.672907in}{1.523548in}}%
\pgfpathlineto{\pgfqpoint{3.673316in}{1.482290in}}%
\pgfpathlineto{\pgfqpoint{3.673724in}{1.457535in}}%
\pgfpathlineto{\pgfqpoint{3.673928in}{1.490542in}}%
\pgfpathlineto{\pgfqpoint{3.675153in}{1.531800in}}%
\pgfpathlineto{\pgfqpoint{3.675357in}{1.490542in}}%
\pgfpathlineto{\pgfqpoint{3.675561in}{1.564806in}}%
\pgfpathlineto{\pgfqpoint{3.675765in}{1.556554in}}%
\pgfpathlineto{\pgfqpoint{3.676378in}{1.746341in}}%
\pgfpathlineto{\pgfqpoint{3.676786in}{1.556554in}}%
\pgfpathlineto{\pgfqpoint{3.677194in}{1.573058in}}%
\pgfpathlineto{\pgfqpoint{3.677399in}{1.548303in}}%
\pgfpathlineto{\pgfqpoint{3.677603in}{1.639071in}}%
\pgfpathlineto{\pgfqpoint{3.678419in}{1.589561in}}%
\pgfpathlineto{\pgfqpoint{3.679644in}{1.490542in}}%
\pgfpathlineto{\pgfqpoint{3.679848in}{1.498793in}}%
\pgfpathlineto{\pgfqpoint{3.680052in}{1.531800in}}%
\pgfpathlineto{\pgfqpoint{3.680461in}{1.482290in}}%
\pgfpathlineto{\pgfqpoint{3.680869in}{1.507045in}}%
\pgfpathlineto{\pgfqpoint{3.681277in}{1.515296in}}%
\pgfpathlineto{\pgfqpoint{3.681481in}{1.630819in}}%
\pgfpathlineto{\pgfqpoint{3.682094in}{1.457535in}}%
\pgfpathlineto{\pgfqpoint{3.682298in}{1.482290in}}%
\pgfpathlineto{\pgfqpoint{3.682502in}{1.457535in}}%
\pgfpathlineto{\pgfqpoint{3.682910in}{1.490542in}}%
\pgfpathlineto{\pgfqpoint{3.684339in}{1.647322in}}%
\pgfpathlineto{\pgfqpoint{3.684748in}{1.622567in}}%
\pgfpathlineto{\pgfqpoint{3.684952in}{1.581309in}}%
\pgfpathlineto{\pgfqpoint{3.685360in}{1.680329in}}%
\pgfpathlineto{\pgfqpoint{3.685564in}{1.639071in}}%
\pgfpathlineto{\pgfqpoint{3.686177in}{1.804103in}}%
\pgfpathlineto{\pgfqpoint{3.686789in}{1.762845in}}%
\pgfpathlineto{\pgfqpoint{3.690055in}{2.026896in}}%
\pgfpathlineto{\pgfqpoint{3.690260in}{2.035148in}}%
\pgfpathlineto{\pgfqpoint{3.690464in}{1.927877in}}%
\pgfpathlineto{\pgfqpoint{3.691280in}{2.026896in}}%
\pgfpathlineto{\pgfqpoint{3.691484in}{2.035148in}}%
\pgfpathlineto{\pgfqpoint{3.692301in}{2.167173in}}%
\pgfpathlineto{\pgfqpoint{3.692505in}{2.109412in}}%
\pgfpathlineto{\pgfqpoint{3.693526in}{2.117664in}}%
\pgfpathlineto{\pgfqpoint{3.693934in}{1.795851in}}%
\pgfpathlineto{\pgfqpoint{3.694547in}{2.092909in}}%
\pgfpathlineto{\pgfqpoint{3.695159in}{2.076406in}}%
\pgfpathlineto{\pgfqpoint{3.695771in}{2.010393in}}%
\pgfpathlineto{\pgfqpoint{3.696180in}{2.084657in}}%
\pgfpathlineto{\pgfqpoint{3.696384in}{2.076406in}}%
\pgfpathlineto{\pgfqpoint{3.696588in}{2.142418in}}%
\pgfpathlineto{\pgfqpoint{3.696996in}{2.018644in}}%
\pgfpathlineto{\pgfqpoint{3.697200in}{2.059902in}}%
\pgfpathlineto{\pgfqpoint{3.697405in}{1.795851in}}%
\pgfpathlineto{\pgfqpoint{3.698221in}{1.969135in}}%
\pgfpathlineto{\pgfqpoint{3.698425in}{1.969135in}}%
\pgfpathlineto{\pgfqpoint{3.700262in}{2.183676in}}%
\pgfpathlineto{\pgfqpoint{3.700875in}{2.059902in}}%
\pgfpathlineto{\pgfqpoint{3.701691in}{2.092909in}}%
\pgfpathlineto{\pgfqpoint{3.701896in}{2.092909in}}%
\pgfpathlineto{\pgfqpoint{3.702304in}{2.134167in}}%
\pgfpathlineto{\pgfqpoint{3.702916in}{2.092909in}}%
\pgfpathlineto{\pgfqpoint{3.703529in}{2.101160in}}%
\pgfpathlineto{\pgfqpoint{3.703937in}{2.059902in}}%
\pgfpathlineto{\pgfqpoint{3.705162in}{2.175425in}}%
\pgfpathlineto{\pgfqpoint{3.705570in}{2.142418in}}%
\pgfpathlineto{\pgfqpoint{3.705978in}{2.158922in}}%
\pgfpathlineto{\pgfqpoint{3.706183in}{2.208431in}}%
\pgfpathlineto{\pgfqpoint{3.706999in}{2.167173in}}%
\pgfpathlineto{\pgfqpoint{3.707203in}{2.167173in}}%
\pgfpathlineto{\pgfqpoint{3.707612in}{2.233186in}}%
\pgfpathlineto{\pgfqpoint{3.707816in}{2.158922in}}%
\pgfpathlineto{\pgfqpoint{3.708020in}{2.125915in}}%
\pgfpathlineto{\pgfqpoint{3.708224in}{2.200180in}}%
\pgfpathlineto{\pgfqpoint{3.708632in}{2.191928in}}%
\pgfpathlineto{\pgfqpoint{3.708836in}{2.191928in}}%
\pgfpathlineto{\pgfqpoint{3.709245in}{2.266192in}}%
\pgfpathlineto{\pgfqpoint{3.710061in}{2.249689in}}%
\pgfpathlineto{\pgfqpoint{3.710470in}{2.191928in}}%
\pgfpathlineto{\pgfqpoint{3.710674in}{2.175425in}}%
\pgfpathlineto{\pgfqpoint{3.710878in}{2.216683in}}%
\pgfpathlineto{\pgfqpoint{3.711082in}{2.200180in}}%
\pgfpathlineto{\pgfqpoint{3.711286in}{2.224934in}}%
\pgfpathlineto{\pgfqpoint{3.712103in}{2.216683in}}%
\pgfpathlineto{\pgfqpoint{3.712511in}{2.183676in}}%
\pgfpathlineto{\pgfqpoint{3.712919in}{2.200180in}}%
\pgfpathlineto{\pgfqpoint{3.713123in}{2.257941in}}%
\pgfpathlineto{\pgfqpoint{3.714144in}{2.249689in}}%
\pgfpathlineto{\pgfqpoint{3.715981in}{2.200180in}}%
\pgfpathlineto{\pgfqpoint{3.716186in}{2.208431in}}%
\pgfpathlineto{\pgfqpoint{3.718227in}{2.373463in}}%
\pgfpathlineto{\pgfqpoint{3.718635in}{2.323954in}}%
\pgfpathlineto{\pgfqpoint{3.719044in}{2.348709in}}%
\pgfpathlineto{\pgfqpoint{3.720064in}{2.191928in}}%
\pgfpathlineto{\pgfqpoint{3.720473in}{2.224934in}}%
\pgfpathlineto{\pgfqpoint{3.720881in}{2.167173in}}%
\pgfpathlineto{\pgfqpoint{3.721085in}{2.191928in}}%
\pgfpathlineto{\pgfqpoint{3.721697in}{2.175425in}}%
\pgfpathlineto{\pgfqpoint{3.721902in}{2.200180in}}%
\pgfpathlineto{\pgfqpoint{3.722514in}{2.282696in}}%
\pgfpathlineto{\pgfqpoint{3.723126in}{2.274444in}}%
\pgfpathlineto{\pgfqpoint{3.724555in}{2.208431in}}%
\pgfpathlineto{\pgfqpoint{3.725576in}{2.158922in}}%
\pgfpathlineto{\pgfqpoint{3.724964in}{2.216683in}}%
\pgfpathlineto{\pgfqpoint{3.725780in}{2.191928in}}%
\pgfpathlineto{\pgfqpoint{3.726189in}{2.249689in}}%
\pgfpathlineto{\pgfqpoint{3.726801in}{2.233186in}}%
\pgfpathlineto{\pgfqpoint{3.727005in}{2.191928in}}%
\pgfpathlineto{\pgfqpoint{3.727413in}{2.274444in}}%
\pgfpathlineto{\pgfqpoint{3.727618in}{2.249689in}}%
\pgfpathlineto{\pgfqpoint{3.727822in}{2.274444in}}%
\pgfpathlineto{\pgfqpoint{3.728230in}{2.224934in}}%
\pgfpathlineto{\pgfqpoint{3.728434in}{2.249689in}}%
\pgfpathlineto{\pgfqpoint{3.729047in}{1.903122in}}%
\pgfpathlineto{\pgfqpoint{3.729659in}{2.134167in}}%
\pgfpathlineto{\pgfqpoint{3.730067in}{2.208431in}}%
\pgfpathlineto{\pgfqpoint{3.730271in}{2.125915in}}%
\pgfpathlineto{\pgfqpoint{3.730476in}{1.886619in}}%
\pgfpathlineto{\pgfqpoint{3.731088in}{2.241438in}}%
\pgfpathlineto{\pgfqpoint{3.731292in}{2.158922in}}%
\pgfpathlineto{\pgfqpoint{3.731700in}{2.315702in}}%
\pgfpathlineto{\pgfqpoint{3.732517in}{2.290947in}}%
\pgfpathlineto{\pgfqpoint{3.735171in}{2.059902in}}%
\pgfpathlineto{\pgfqpoint{3.736192in}{2.307450in}}%
\pgfpathlineto{\pgfqpoint{3.736600in}{2.282696in}}%
\pgfpathlineto{\pgfqpoint{3.736804in}{2.282696in}}%
\pgfpathlineto{\pgfqpoint{3.737212in}{2.274444in}}%
\pgfpathlineto{\pgfqpoint{3.738029in}{2.332205in}}%
\pgfpathlineto{\pgfqpoint{3.738233in}{2.323954in}}%
\pgfpathlineto{\pgfqpoint{3.738437in}{2.381715in}}%
\pgfpathlineto{\pgfqpoint{3.738845in}{2.274444in}}%
\pgfpathlineto{\pgfqpoint{3.739050in}{2.274444in}}%
\pgfpathlineto{\pgfqpoint{3.739254in}{2.035148in}}%
\pgfpathlineto{\pgfqpoint{3.739662in}{2.307450in}}%
\pgfpathlineto{\pgfqpoint{3.740070in}{2.290947in}}%
\pgfpathlineto{\pgfqpoint{3.741295in}{2.249689in}}%
\pgfpathlineto{\pgfqpoint{3.741908in}{2.323954in}}%
\pgfpathlineto{\pgfqpoint{3.742316in}{2.266192in}}%
\pgfpathlineto{\pgfqpoint{3.742724in}{2.323954in}}%
\pgfpathlineto{\pgfqpoint{3.743541in}{2.307450in}}%
\pgfpathlineto{\pgfqpoint{3.743745in}{2.274444in}}%
\pgfpathlineto{\pgfqpoint{3.744153in}{2.365212in}}%
\pgfpathlineto{\pgfqpoint{3.744357in}{2.365212in}}%
\pgfpathlineto{\pgfqpoint{3.744561in}{2.431225in}}%
\pgfpathlineto{\pgfqpoint{3.745174in}{2.315702in}}%
\pgfpathlineto{\pgfqpoint{3.745378in}{2.356960in}}%
\pgfpathlineto{\pgfqpoint{3.746195in}{2.299199in}}%
\pgfpathlineto{\pgfqpoint{3.746807in}{2.373463in}}%
\pgfpathlineto{\pgfqpoint{3.747215in}{2.332205in}}%
\pgfpathlineto{\pgfqpoint{3.747419in}{2.315702in}}%
\pgfpathlineto{\pgfqpoint{3.747828in}{2.431225in}}%
\pgfpathlineto{\pgfqpoint{3.748440in}{2.389967in}}%
\pgfpathlineto{\pgfqpoint{3.749665in}{2.299199in}}%
\pgfpathlineto{\pgfqpoint{3.748848in}{2.406470in}}%
\pgfpathlineto{\pgfqpoint{3.749869in}{2.323954in}}%
\pgfpathlineto{\pgfqpoint{3.750073in}{2.340457in}}%
\pgfpathlineto{\pgfqpoint{3.750277in}{2.315702in}}%
\pgfpathlineto{\pgfqpoint{3.750482in}{2.257941in}}%
\pgfpathlineto{\pgfqpoint{3.750686in}{2.332205in}}%
\pgfpathlineto{\pgfqpoint{3.751094in}{2.332205in}}%
\pgfpathlineto{\pgfqpoint{3.751706in}{2.381715in}}%
\pgfpathlineto{\pgfqpoint{3.752115in}{2.348709in}}%
\pgfpathlineto{\pgfqpoint{3.752523in}{2.307450in}}%
\pgfpathlineto{\pgfqpoint{3.752727in}{2.398218in}}%
\pgfpathlineto{\pgfqpoint{3.753544in}{2.299199in}}%
\pgfpathlineto{\pgfqpoint{3.754769in}{2.406470in}}%
\pgfpathlineto{\pgfqpoint{3.755177in}{2.365212in}}%
\pgfpathlineto{\pgfqpoint{3.755789in}{2.332205in}}%
\pgfpathlineto{\pgfqpoint{3.756198in}{2.381715in}}%
\pgfpathlineto{\pgfqpoint{3.756810in}{2.356960in}}%
\pgfpathlineto{\pgfqpoint{3.758035in}{2.274444in}}%
\pgfpathlineto{\pgfqpoint{3.758443in}{2.282696in}}%
\pgfpathlineto{\pgfqpoint{3.758647in}{2.332205in}}%
\pgfpathlineto{\pgfqpoint{3.759464in}{2.290947in}}%
\pgfpathlineto{\pgfqpoint{3.759872in}{2.307450in}}%
\pgfpathlineto{\pgfqpoint{3.760076in}{2.257941in}}%
\pgfpathlineto{\pgfqpoint{3.760280in}{2.274444in}}%
\pgfpathlineto{\pgfqpoint{3.760485in}{2.233186in}}%
\pgfpathlineto{\pgfqpoint{3.760893in}{2.323954in}}%
\pgfpathlineto{\pgfqpoint{3.761301in}{2.282696in}}%
\pgfpathlineto{\pgfqpoint{3.762526in}{2.373463in}}%
\pgfpathlineto{\pgfqpoint{3.763343in}{2.455979in}}%
\pgfpathlineto{\pgfqpoint{3.763751in}{2.406470in}}%
\pgfpathlineto{\pgfqpoint{3.765180in}{2.340457in}}%
\pgfpathlineto{\pgfqpoint{3.765384in}{2.398218in}}%
\pgfpathlineto{\pgfqpoint{3.765588in}{2.315702in}}%
\pgfpathlineto{\pgfqpoint{3.765792in}{2.191928in}}%
\pgfpathlineto{\pgfqpoint{3.766405in}{2.381715in}}%
\pgfpathlineto{\pgfqpoint{3.766609in}{2.447728in}}%
\pgfpathlineto{\pgfqpoint{3.767425in}{2.365212in}}%
\pgfpathlineto{\pgfqpoint{3.767630in}{2.422973in}}%
\pgfpathlineto{\pgfqpoint{3.768038in}{2.406470in}}%
\pgfpathlineto{\pgfqpoint{3.768242in}{2.431225in}}%
\pgfpathlineto{\pgfqpoint{3.768854in}{2.117664in}}%
\pgfpathlineto{\pgfqpoint{3.769263in}{2.282696in}}%
\pgfpathlineto{\pgfqpoint{3.770079in}{2.455979in}}%
\pgfpathlineto{\pgfqpoint{3.770283in}{2.208431in}}%
\pgfpathlineto{\pgfqpoint{3.771100in}{2.414721in}}%
\pgfpathlineto{\pgfqpoint{3.771304in}{2.414721in}}%
\pgfpathlineto{\pgfqpoint{3.771712in}{2.381715in}}%
\pgfpathlineto{\pgfqpoint{3.772325in}{2.406470in}}%
\pgfpathlineto{\pgfqpoint{3.772733in}{2.422973in}}%
\pgfpathlineto{\pgfqpoint{3.772937in}{2.398218in}}%
\pgfpathlineto{\pgfqpoint{3.773141in}{2.406470in}}%
\pgfpathlineto{\pgfqpoint{3.773346in}{2.381715in}}%
\pgfpathlineto{\pgfqpoint{3.773550in}{2.109412in}}%
\pgfpathlineto{\pgfqpoint{3.774366in}{2.323954in}}%
\pgfpathlineto{\pgfqpoint{3.775387in}{2.389967in}}%
\pgfpathlineto{\pgfqpoint{3.775591in}{2.365212in}}%
\pgfpathlineto{\pgfqpoint{3.776408in}{2.290947in}}%
\pgfpathlineto{\pgfqpoint{3.776612in}{2.315702in}}%
\pgfpathlineto{\pgfqpoint{3.776816in}{2.389967in}}%
\pgfpathlineto{\pgfqpoint{3.777633in}{2.332205in}}%
\pgfpathlineto{\pgfqpoint{3.778857in}{2.257941in}}%
\pgfpathlineto{\pgfqpoint{3.779062in}{2.299199in}}%
\pgfpathlineto{\pgfqpoint{3.779878in}{2.266192in}}%
\pgfpathlineto{\pgfqpoint{3.780082in}{2.266192in}}%
\pgfpathlineto{\pgfqpoint{3.780286in}{2.307450in}}%
\pgfpathlineto{\pgfqpoint{3.780695in}{2.233186in}}%
\pgfpathlineto{\pgfqpoint{3.781103in}{2.249689in}}%
\pgfpathlineto{\pgfqpoint{3.782940in}{2.167173in}}%
\pgfpathlineto{\pgfqpoint{3.784165in}{2.249689in}}%
\pgfpathlineto{\pgfqpoint{3.784369in}{2.216683in}}%
\pgfpathlineto{\pgfqpoint{3.784573in}{2.175425in}}%
\pgfpathlineto{\pgfqpoint{3.785186in}{2.274444in}}%
\pgfpathlineto{\pgfqpoint{3.785798in}{2.282696in}}%
\pgfpathlineto{\pgfqpoint{3.786411in}{2.208431in}}%
\pgfpathlineto{\pgfqpoint{3.787840in}{2.315702in}}%
\pgfpathlineto{\pgfqpoint{3.788248in}{2.233186in}}%
\pgfpathlineto{\pgfqpoint{3.788452in}{2.125915in}}%
\pgfpathlineto{\pgfqpoint{3.789064in}{2.290947in}}%
\pgfpathlineto{\pgfqpoint{3.789269in}{2.241438in}}%
\pgfpathlineto{\pgfqpoint{3.789881in}{2.158922in}}%
\pgfpathlineto{\pgfqpoint{3.790085in}{2.175425in}}%
\pgfpathlineto{\pgfqpoint{3.790289in}{1.977386in}}%
\pgfpathlineto{\pgfqpoint{3.790698in}{2.266192in}}%
\pgfpathlineto{\pgfqpoint{3.791106in}{1.993890in}}%
\pgfpathlineto{\pgfqpoint{3.791310in}{2.249689in}}%
\pgfpathlineto{\pgfqpoint{3.791718in}{1.870115in}}%
\pgfpathlineto{\pgfqpoint{3.792331in}{2.167173in}}%
\pgfpathlineto{\pgfqpoint{3.792739in}{2.200180in}}%
\pgfpathlineto{\pgfqpoint{3.793147in}{2.142418in}}%
\pgfpathlineto{\pgfqpoint{3.793351in}{2.076406in}}%
\pgfpathlineto{\pgfqpoint{3.793556in}{2.241438in}}%
\pgfpathlineto{\pgfqpoint{3.793964in}{2.167173in}}%
\pgfpathlineto{\pgfqpoint{3.795801in}{2.356960in}}%
\pgfpathlineto{\pgfqpoint{3.796005in}{2.315702in}}%
\pgfpathlineto{\pgfqpoint{3.796618in}{2.389967in}}%
\pgfpathlineto{\pgfqpoint{3.796822in}{2.406470in}}%
\pgfpathlineto{\pgfqpoint{3.797843in}{2.241438in}}%
\pgfpathlineto{\pgfqpoint{3.798251in}{2.257941in}}%
\pgfpathlineto{\pgfqpoint{3.799272in}{2.249689in}}%
\pgfpathlineto{\pgfqpoint{3.800088in}{2.348709in}}%
\pgfpathlineto{\pgfqpoint{3.800292in}{2.315702in}}%
\pgfpathlineto{\pgfqpoint{3.800496in}{2.356960in}}%
\pgfpathlineto{\pgfqpoint{3.800905in}{2.332205in}}%
\pgfpathlineto{\pgfqpoint{3.801109in}{2.406470in}}%
\pgfpathlineto{\pgfqpoint{3.801925in}{2.340457in}}%
\pgfpathlineto{\pgfqpoint{3.803354in}{2.406470in}}%
\pgfpathlineto{\pgfqpoint{3.804171in}{2.274444in}}%
\pgfpathlineto{\pgfqpoint{3.804579in}{2.290947in}}%
\pgfpathlineto{\pgfqpoint{3.804988in}{2.274444in}}%
\pgfpathlineto{\pgfqpoint{3.805396in}{2.117664in}}%
\pgfpathlineto{\pgfqpoint{3.805600in}{2.356960in}}%
\pgfpathlineto{\pgfqpoint{3.805804in}{2.332205in}}%
\pgfpathlineto{\pgfqpoint{3.806621in}{2.389967in}}%
\pgfpathlineto{\pgfqpoint{3.807029in}{2.348709in}}%
\pgfpathlineto{\pgfqpoint{3.808458in}{2.249689in}}%
\pgfpathlineto{\pgfqpoint{3.807437in}{2.356960in}}%
\pgfpathlineto{\pgfqpoint{3.808662in}{2.274444in}}%
\pgfpathlineto{\pgfqpoint{3.809479in}{2.348709in}}%
\pgfpathlineto{\pgfqpoint{3.809887in}{2.068154in}}%
\pgfpathlineto{\pgfqpoint{3.810499in}{2.373463in}}%
\pgfpathlineto{\pgfqpoint{3.810908in}{2.092909in}}%
\pgfpathlineto{\pgfqpoint{3.811724in}{2.332205in}}%
\pgfpathlineto{\pgfqpoint{3.812745in}{2.208431in}}%
\pgfpathlineto{\pgfqpoint{3.813153in}{2.249689in}}%
\pgfpathlineto{\pgfqpoint{3.813357in}{2.323954in}}%
\pgfpathlineto{\pgfqpoint{3.814378in}{2.299199in}}%
\pgfpathlineto{\pgfqpoint{3.814582in}{2.299199in}}%
\pgfpathlineto{\pgfqpoint{3.814786in}{2.274444in}}%
\pgfpathlineto{\pgfqpoint{3.814991in}{2.340457in}}%
\pgfpathlineto{\pgfqpoint{3.815195in}{2.323954in}}%
\pgfpathlineto{\pgfqpoint{3.815807in}{2.307450in}}%
\pgfpathlineto{\pgfqpoint{3.816215in}{2.348709in}}%
\pgfpathlineto{\pgfqpoint{3.817032in}{2.216683in}}%
\pgfpathlineto{\pgfqpoint{3.817236in}{2.282696in}}%
\pgfpathlineto{\pgfqpoint{3.818665in}{2.373463in}}%
\pgfpathlineto{\pgfqpoint{3.820298in}{2.290947in}}%
\pgfpathlineto{\pgfqpoint{3.820502in}{2.332205in}}%
\pgfpathlineto{\pgfqpoint{3.820911in}{2.249689in}}%
\pgfpathlineto{\pgfqpoint{3.821319in}{2.290947in}}%
\pgfpathlineto{\pgfqpoint{3.821523in}{2.266192in}}%
\pgfpathlineto{\pgfqpoint{3.821727in}{2.323954in}}%
\pgfpathlineto{\pgfqpoint{3.821931in}{2.323954in}}%
\pgfpathlineto{\pgfqpoint{3.822340in}{2.373463in}}%
\pgfpathlineto{\pgfqpoint{3.822748in}{2.340457in}}%
\pgfpathlineto{\pgfqpoint{3.823769in}{2.233186in}}%
\pgfpathlineto{\pgfqpoint{3.823973in}{2.266192in}}%
\pgfpathlineto{\pgfqpoint{3.825402in}{2.373463in}}%
\pgfpathlineto{\pgfqpoint{3.825606in}{2.422973in}}%
\pgfpathlineto{\pgfqpoint{3.826218in}{2.332205in}}%
\pgfpathlineto{\pgfqpoint{3.826423in}{2.340457in}}%
\pgfpathlineto{\pgfqpoint{3.826627in}{2.332205in}}%
\pgfpathlineto{\pgfqpoint{3.826831in}{2.257941in}}%
\pgfpathlineto{\pgfqpoint{3.827647in}{2.299199in}}%
\pgfpathlineto{\pgfqpoint{3.828668in}{2.348709in}}%
\pgfpathlineto{\pgfqpoint{3.829281in}{2.224934in}}%
\pgfpathlineto{\pgfqpoint{3.829893in}{2.323954in}}%
\pgfpathlineto{\pgfqpoint{3.831934in}{2.480734in}}%
\pgfpathlineto{\pgfqpoint{3.832547in}{2.348709in}}%
\pgfpathlineto{\pgfqpoint{3.833159in}{2.422973in}}%
\pgfpathlineto{\pgfqpoint{3.833363in}{2.431225in}}%
\pgfpathlineto{\pgfqpoint{3.833568in}{2.398218in}}%
\pgfpathlineto{\pgfqpoint{3.834180in}{2.373463in}}%
\pgfpathlineto{\pgfqpoint{3.834792in}{2.439476in}}%
\pgfpathlineto{\pgfqpoint{3.835201in}{2.398218in}}%
\pgfpathlineto{\pgfqpoint{3.835405in}{2.365212in}}%
\pgfpathlineto{\pgfqpoint{3.835813in}{2.439476in}}%
\pgfpathlineto{\pgfqpoint{3.836017in}{2.447728in}}%
\pgfpathlineto{\pgfqpoint{3.836221in}{2.398218in}}%
\pgfpathlineto{\pgfqpoint{3.836834in}{2.480734in}}%
\pgfpathlineto{\pgfqpoint{3.837038in}{2.439476in}}%
\pgfpathlineto{\pgfqpoint{3.837242in}{2.472483in}}%
\pgfpathlineto{\pgfqpoint{3.837446in}{2.422973in}}%
\pgfpathlineto{\pgfqpoint{3.837855in}{2.422973in}}%
\pgfpathlineto{\pgfqpoint{3.838059in}{2.422973in}}%
\pgfpathlineto{\pgfqpoint{3.838671in}{2.398218in}}%
\pgfpathlineto{\pgfqpoint{3.838875in}{2.464231in}}%
\pgfpathlineto{\pgfqpoint{3.839284in}{2.389967in}}%
\pgfpathlineto{\pgfqpoint{3.839896in}{2.439476in}}%
\pgfpathlineto{\pgfqpoint{3.840304in}{2.389967in}}%
\pgfpathlineto{\pgfqpoint{3.840508in}{2.431225in}}%
\pgfpathlineto{\pgfqpoint{3.841529in}{2.480734in}}%
\pgfpathlineto{\pgfqpoint{3.841937in}{2.472483in}}%
\pgfpathlineto{\pgfqpoint{3.843162in}{2.389967in}}%
\pgfpathlineto{\pgfqpoint{3.843571in}{2.414721in}}%
\pgfpathlineto{\pgfqpoint{3.843775in}{2.472483in}}%
\pgfpathlineto{\pgfqpoint{3.844183in}{2.381715in}}%
\pgfpathlineto{\pgfqpoint{3.844591in}{2.422973in}}%
\pgfpathlineto{\pgfqpoint{3.846224in}{2.340457in}}%
\pgfpathlineto{\pgfqpoint{3.846633in}{2.257941in}}%
\pgfpathlineto{\pgfqpoint{3.847245in}{2.323954in}}%
\pgfpathlineto{\pgfqpoint{3.848470in}{2.398218in}}%
\pgfpathlineto{\pgfqpoint{3.848674in}{2.389967in}}%
\pgfpathlineto{\pgfqpoint{3.849287in}{2.373463in}}%
\pgfpathlineto{\pgfqpoint{3.849491in}{2.406470in}}%
\pgfpathlineto{\pgfqpoint{3.849899in}{2.381715in}}%
\pgfpathlineto{\pgfqpoint{3.850103in}{2.406470in}}%
\pgfpathlineto{\pgfqpoint{3.850307in}{2.439476in}}%
\pgfpathlineto{\pgfqpoint{3.850716in}{2.365212in}}%
\pgfpathlineto{\pgfqpoint{3.851736in}{2.266192in}}%
\pgfpathlineto{\pgfqpoint{3.852145in}{2.299199in}}%
\pgfpathlineto{\pgfqpoint{3.852553in}{2.257941in}}%
\pgfpathlineto{\pgfqpoint{3.852757in}{2.356960in}}%
\pgfpathlineto{\pgfqpoint{3.853574in}{2.340457in}}%
\pgfpathlineto{\pgfqpoint{3.853982in}{2.224934in}}%
\pgfpathlineto{\pgfqpoint{3.854798in}{2.241438in}}%
\pgfpathlineto{\pgfqpoint{3.856227in}{2.323954in}}%
\pgfpathlineto{\pgfqpoint{3.856432in}{2.323954in}}%
\pgfpathlineto{\pgfqpoint{3.857248in}{2.348709in}}%
\pgfpathlineto{\pgfqpoint{3.857656in}{2.249689in}}%
\pgfpathlineto{\pgfqpoint{3.858473in}{2.290947in}}%
\pgfpathlineto{\pgfqpoint{3.859290in}{2.249689in}}%
\pgfpathlineto{\pgfqpoint{3.859698in}{2.266192in}}%
\pgfpathlineto{\pgfqpoint{3.860719in}{2.431225in}}%
\pgfpathlineto{\pgfqpoint{3.860923in}{2.117664in}}%
\pgfpathlineto{\pgfqpoint{3.861739in}{2.356960in}}%
\pgfpathlineto{\pgfqpoint{3.861943in}{2.381715in}}%
\pgfpathlineto{\pgfqpoint{3.862352in}{2.233186in}}%
\pgfpathlineto{\pgfqpoint{3.862964in}{2.307450in}}%
\pgfpathlineto{\pgfqpoint{3.863372in}{2.340457in}}%
\pgfpathlineto{\pgfqpoint{3.863577in}{2.158922in}}%
\pgfpathlineto{\pgfqpoint{3.864393in}{2.389967in}}%
\pgfpathlineto{\pgfqpoint{3.864801in}{2.299199in}}%
\pgfpathlineto{\pgfqpoint{3.865414in}{2.381715in}}%
\pgfpathlineto{\pgfqpoint{3.865618in}{2.365212in}}%
\pgfpathlineto{\pgfqpoint{3.866026in}{2.150670in}}%
\pgfpathlineto{\pgfqpoint{3.866639in}{2.356960in}}%
\pgfpathlineto{\pgfqpoint{3.867251in}{2.381715in}}%
\pgfpathlineto{\pgfqpoint{3.868068in}{2.290947in}}%
\pgfpathlineto{\pgfqpoint{3.868272in}{2.290947in}}%
\pgfpathlineto{\pgfqpoint{3.868476in}{2.084657in}}%
\pgfpathlineto{\pgfqpoint{3.869088in}{2.414721in}}%
\pgfpathlineto{\pgfqpoint{3.869293in}{2.340457in}}%
\pgfpathlineto{\pgfqpoint{3.869701in}{2.365212in}}%
\pgfpathlineto{\pgfqpoint{3.869905in}{2.323954in}}%
\pgfpathlineto{\pgfqpoint{3.870517in}{2.431225in}}%
\pgfpathlineto{\pgfqpoint{3.870721in}{2.307450in}}%
\pgfpathlineto{\pgfqpoint{3.870926in}{2.307450in}}%
\pgfpathlineto{\pgfqpoint{3.871130in}{2.332205in}}%
\pgfpathlineto{\pgfqpoint{3.871742in}{2.315702in}}%
\pgfpathlineto{\pgfqpoint{3.872763in}{2.175425in}}%
\pgfpathlineto{\pgfqpoint{3.873579in}{2.216683in}}%
\pgfpathlineto{\pgfqpoint{3.875008in}{2.340457in}}%
\pgfpathlineto{\pgfqpoint{3.874396in}{2.208431in}}%
\pgfpathlineto{\pgfqpoint{3.875213in}{2.332205in}}%
\pgfpathlineto{\pgfqpoint{3.875417in}{2.348709in}}%
\pgfpathlineto{\pgfqpoint{3.875621in}{2.307450in}}%
\pgfpathlineto{\pgfqpoint{3.875825in}{2.266192in}}%
\pgfpathlineto{\pgfqpoint{3.876437in}{2.340457in}}%
\pgfpathlineto{\pgfqpoint{3.876642in}{2.356960in}}%
\pgfpathlineto{\pgfqpoint{3.877050in}{2.315702in}}%
\pgfpathlineto{\pgfqpoint{3.877866in}{2.249689in}}%
\pgfpathlineto{\pgfqpoint{3.877662in}{2.356960in}}%
\pgfpathlineto{\pgfqpoint{3.878071in}{2.257941in}}%
\pgfpathlineto{\pgfqpoint{3.878275in}{2.307450in}}%
\pgfpathlineto{\pgfqpoint{3.878683in}{2.241438in}}%
\pgfpathlineto{\pgfqpoint{3.878887in}{2.249689in}}%
\pgfpathlineto{\pgfqpoint{3.879500in}{2.167173in}}%
\pgfpathlineto{\pgfqpoint{3.879908in}{2.216683in}}%
\pgfpathlineto{\pgfqpoint{3.880724in}{2.282696in}}%
\pgfpathlineto{\pgfqpoint{3.880929in}{2.233186in}}%
\pgfpathlineto{\pgfqpoint{3.881541in}{2.208431in}}%
\pgfpathlineto{\pgfqpoint{3.881337in}{2.241438in}}%
\pgfpathlineto{\pgfqpoint{3.881745in}{2.233186in}}%
\pgfpathlineto{\pgfqpoint{3.882358in}{2.274444in}}%
\pgfpathlineto{\pgfqpoint{3.883174in}{2.266192in}}%
\pgfpathlineto{\pgfqpoint{3.884399in}{2.125915in}}%
\pgfpathlineto{\pgfqpoint{3.884807in}{2.183676in}}%
\pgfpathlineto{\pgfqpoint{3.885216in}{2.068154in}}%
\pgfpathlineto{\pgfqpoint{3.885828in}{2.101160in}}%
\pgfpathlineto{\pgfqpoint{3.886645in}{1.795851in}}%
\pgfpathlineto{\pgfqpoint{3.887665in}{2.026896in}}%
\pgfpathlineto{\pgfqpoint{3.887869in}{1.993890in}}%
\pgfpathlineto{\pgfqpoint{3.888074in}{2.018644in}}%
\pgfpathlineto{\pgfqpoint{3.888278in}{1.952632in}}%
\pgfpathlineto{\pgfqpoint{3.888482in}{1.985638in}}%
\pgfpathlineto{\pgfqpoint{3.889503in}{1.853612in}}%
\pgfpathlineto{\pgfqpoint{3.890115in}{1.861864in}}%
\pgfpathlineto{\pgfqpoint{3.890932in}{1.936128in}}%
\pgfpathlineto{\pgfqpoint{3.891340in}{2.076406in}}%
\pgfpathlineto{\pgfqpoint{3.891952in}{1.927877in}}%
\pgfpathlineto{\pgfqpoint{3.892156in}{1.936128in}}%
\pgfpathlineto{\pgfqpoint{3.892361in}{2.059902in}}%
\pgfpathlineto{\pgfqpoint{3.893177in}{1.985638in}}%
\pgfpathlineto{\pgfqpoint{3.893381in}{1.993890in}}%
\pgfpathlineto{\pgfqpoint{3.893585in}{1.977386in}}%
\pgfpathlineto{\pgfqpoint{3.893994in}{1.952632in}}%
\pgfpathlineto{\pgfqpoint{3.894402in}{1.985638in}}%
\pgfpathlineto{\pgfqpoint{3.895423in}{2.158922in}}%
\pgfpathlineto{\pgfqpoint{3.895831in}{2.109412in}}%
\pgfpathlineto{\pgfqpoint{3.896852in}{2.051651in}}%
\pgfpathlineto{\pgfqpoint{3.897668in}{2.101160in}}%
\pgfpathlineto{\pgfqpoint{3.897872in}{2.076406in}}%
\pgfpathlineto{\pgfqpoint{3.898485in}{2.101160in}}%
\pgfpathlineto{\pgfqpoint{3.898893in}{2.051651in}}%
\pgfpathlineto{\pgfqpoint{3.899710in}{2.101160in}}%
\pgfpathlineto{\pgfqpoint{3.899914in}{2.043399in}}%
\pgfpathlineto{\pgfqpoint{3.900118in}{2.035148in}}%
\pgfpathlineto{\pgfqpoint{3.900526in}{2.109412in}}%
\pgfpathlineto{\pgfqpoint{3.901139in}{2.018644in}}%
\pgfpathlineto{\pgfqpoint{3.901751in}{2.076406in}}%
\pgfpathlineto{\pgfqpoint{3.902364in}{2.051651in}}%
\pgfpathlineto{\pgfqpoint{3.902772in}{1.944380in}}%
\pgfpathlineto{\pgfqpoint{3.903384in}{2.051651in}}%
\pgfpathlineto{\pgfqpoint{3.903588in}{2.043399in}}%
\pgfpathlineto{\pgfqpoint{3.903793in}{2.092909in}}%
\pgfpathlineto{\pgfqpoint{3.903997in}{1.960883in}}%
\pgfpathlineto{\pgfqpoint{3.904405in}{1.977386in}}%
\pgfpathlineto{\pgfqpoint{3.904813in}{1.960883in}}%
\pgfpathlineto{\pgfqpoint{3.905017in}{2.051651in}}%
\pgfpathlineto{\pgfqpoint{3.905834in}{1.977386in}}%
\pgfpathlineto{\pgfqpoint{3.906446in}{1.985638in}}%
\pgfpathlineto{\pgfqpoint{3.907059in}{1.919625in}}%
\pgfpathlineto{\pgfqpoint{3.907263in}{1.911373in}}%
\pgfpathlineto{\pgfqpoint{3.907467in}{1.977386in}}%
\pgfpathlineto{\pgfqpoint{3.907875in}{1.960883in}}%
\pgfpathlineto{\pgfqpoint{3.908488in}{2.026896in}}%
\pgfpathlineto{\pgfqpoint{3.909100in}{1.787599in}}%
\pgfpathlineto{\pgfqpoint{3.909917in}{2.092909in}}%
\pgfpathlineto{\pgfqpoint{3.910325in}{2.018644in}}%
\pgfpathlineto{\pgfqpoint{3.910529in}{1.985638in}}%
\pgfpathlineto{\pgfqpoint{3.910733in}{2.051651in}}%
\pgfpathlineto{\pgfqpoint{3.910938in}{2.026896in}}%
\pgfpathlineto{\pgfqpoint{3.911754in}{2.092909in}}%
\pgfpathlineto{\pgfqpoint{3.912162in}{1.787599in}}%
\pgfpathlineto{\pgfqpoint{3.912775in}{2.059902in}}%
\pgfpathlineto{\pgfqpoint{3.913183in}{2.035148in}}%
\pgfpathlineto{\pgfqpoint{3.913591in}{1.936128in}}%
\pgfpathlineto{\pgfqpoint{3.914204in}{2.010393in}}%
\pgfpathlineto{\pgfqpoint{3.914612in}{2.051651in}}%
\pgfpathlineto{\pgfqpoint{3.914816in}{2.026896in}}%
\pgfpathlineto{\pgfqpoint{3.915020in}{1.960883in}}%
\pgfpathlineto{\pgfqpoint{3.915633in}{2.059902in}}%
\pgfpathlineto{\pgfqpoint{3.915837in}{2.059902in}}%
\pgfpathlineto{\pgfqpoint{3.916041in}{2.092909in}}%
\pgfpathlineto{\pgfqpoint{3.916449in}{1.985638in}}%
\pgfpathlineto{\pgfqpoint{3.916858in}{2.051651in}}%
\pgfpathlineto{\pgfqpoint{3.917062in}{2.068154in}}%
\pgfpathlineto{\pgfqpoint{3.917470in}{1.771096in}}%
\pgfpathlineto{\pgfqpoint{3.918287in}{1.960883in}}%
\pgfpathlineto{\pgfqpoint{3.919307in}{2.010393in}}%
\pgfpathlineto{\pgfqpoint{3.919512in}{1.985638in}}%
\pgfpathlineto{\pgfqpoint{3.919920in}{2.035148in}}%
\pgfpathlineto{\pgfqpoint{3.920124in}{2.051651in}}%
\pgfpathlineto{\pgfqpoint{3.920941in}{2.035148in}}%
\pgfpathlineto{\pgfqpoint{3.921145in}{2.026896in}}%
\pgfpathlineto{\pgfqpoint{3.921553in}{1.762845in}}%
\pgfpathlineto{\pgfqpoint{3.922370in}{1.779348in}}%
\pgfpathlineto{\pgfqpoint{3.923390in}{1.969135in}}%
\pgfpathlineto{\pgfqpoint{3.923594in}{1.944380in}}%
\pgfpathlineto{\pgfqpoint{3.924615in}{1.894870in}}%
\pgfpathlineto{\pgfqpoint{3.925432in}{2.051651in}}%
\pgfpathlineto{\pgfqpoint{3.926044in}{1.993890in}}%
\pgfpathlineto{\pgfqpoint{3.926657in}{1.861864in}}%
\pgfpathlineto{\pgfqpoint{3.927065in}{1.936128in}}%
\pgfpathlineto{\pgfqpoint{3.927677in}{2.002141in}}%
\pgfpathlineto{\pgfqpoint{3.928290in}{1.960883in}}%
\pgfpathlineto{\pgfqpoint{3.929719in}{1.903122in}}%
\pgfpathlineto{\pgfqpoint{3.930127in}{1.927877in}}%
\pgfpathlineto{\pgfqpoint{3.930331in}{1.952632in}}%
\pgfpathlineto{\pgfqpoint{3.930739in}{1.911373in}}%
\pgfpathlineto{\pgfqpoint{3.930944in}{1.870115in}}%
\pgfpathlineto{\pgfqpoint{3.931352in}{1.944380in}}%
\pgfpathlineto{\pgfqpoint{3.931556in}{1.936128in}}%
\pgfpathlineto{\pgfqpoint{3.931760in}{1.944380in}}%
\pgfpathlineto{\pgfqpoint{3.932168in}{1.705083in}}%
\pgfpathlineto{\pgfqpoint{3.932985in}{1.878367in}}%
\pgfpathlineto{\pgfqpoint{3.933802in}{1.927877in}}%
\pgfpathlineto{\pgfqpoint{3.934210in}{1.795851in}}%
\pgfpathlineto{\pgfqpoint{3.935026in}{1.837109in}}%
\pgfpathlineto{\pgfqpoint{3.935231in}{1.837109in}}%
\pgfpathlineto{\pgfqpoint{3.936455in}{1.960883in}}%
\pgfpathlineto{\pgfqpoint{3.936660in}{1.919625in}}%
\pgfpathlineto{\pgfqpoint{3.936864in}{1.919625in}}%
\pgfpathlineto{\pgfqpoint{3.937680in}{1.878367in}}%
\pgfpathlineto{\pgfqpoint{3.937476in}{1.927877in}}%
\pgfpathlineto{\pgfqpoint{3.937884in}{1.894870in}}%
\pgfpathlineto{\pgfqpoint{3.938293in}{1.688580in}}%
\pgfpathlineto{\pgfqpoint{3.938701in}{1.911373in}}%
\pgfpathlineto{\pgfqpoint{3.938905in}{1.886619in}}%
\pgfpathlineto{\pgfqpoint{3.939313in}{1.960883in}}%
\pgfpathlineto{\pgfqpoint{3.939926in}{1.944380in}}%
\pgfpathlineto{\pgfqpoint{3.940130in}{1.903122in}}%
\pgfpathlineto{\pgfqpoint{3.940538in}{1.952632in}}%
\pgfpathlineto{\pgfqpoint{3.940947in}{2.026896in}}%
\pgfpathlineto{\pgfqpoint{3.941355in}{1.936128in}}%
\pgfpathlineto{\pgfqpoint{3.941559in}{1.977386in}}%
\pgfpathlineto{\pgfqpoint{3.942580in}{1.894870in}}%
\pgfpathlineto{\pgfqpoint{3.942784in}{1.927877in}}%
\pgfpathlineto{\pgfqpoint{3.942988in}{1.969135in}}%
\pgfpathlineto{\pgfqpoint{3.943805in}{1.919625in}}%
\pgfpathlineto{\pgfqpoint{3.944213in}{2.018644in}}%
\pgfpathlineto{\pgfqpoint{3.944621in}{1.936128in}}%
\pgfpathlineto{\pgfqpoint{3.944825in}{1.911373in}}%
\pgfpathlineto{\pgfqpoint{3.945029in}{1.985638in}}%
\pgfpathlineto{\pgfqpoint{3.945234in}{1.952632in}}%
\pgfpathlineto{\pgfqpoint{3.945846in}{2.018644in}}%
\pgfpathlineto{\pgfqpoint{3.945642in}{1.936128in}}%
\pgfpathlineto{\pgfqpoint{3.946254in}{2.010393in}}%
\pgfpathlineto{\pgfqpoint{3.947479in}{1.886619in}}%
\pgfpathlineto{\pgfqpoint{3.947683in}{1.886619in}}%
\pgfpathlineto{\pgfqpoint{3.947887in}{1.647322in}}%
\pgfpathlineto{\pgfqpoint{3.948704in}{1.960883in}}%
\pgfpathlineto{\pgfqpoint{3.949112in}{1.894870in}}%
\pgfpathlineto{\pgfqpoint{3.949521in}{1.804103in}}%
\pgfpathlineto{\pgfqpoint{3.950133in}{1.903122in}}%
\pgfpathlineto{\pgfqpoint{3.950337in}{1.903122in}}%
\pgfpathlineto{\pgfqpoint{3.950745in}{1.853612in}}%
\pgfpathlineto{\pgfqpoint{3.951970in}{1.762845in}}%
\pgfpathlineto{\pgfqpoint{3.952174in}{1.812354in}}%
\pgfpathlineto{\pgfqpoint{3.952379in}{1.729838in}}%
\pgfpathlineto{\pgfqpoint{3.952991in}{1.787599in}}%
\pgfpathlineto{\pgfqpoint{3.954012in}{1.713335in}}%
\pgfpathlineto{\pgfqpoint{3.954216in}{1.738090in}}%
\pgfpathlineto{\pgfqpoint{3.955032in}{1.795851in}}%
\pgfpathlineto{\pgfqpoint{3.955236in}{1.787599in}}%
\pgfpathlineto{\pgfqpoint{3.955441in}{1.845361in}}%
\pgfpathlineto{\pgfqpoint{3.955849in}{1.771096in}}%
\pgfpathlineto{\pgfqpoint{3.956053in}{1.597813in}}%
\pgfpathlineto{\pgfqpoint{3.956257in}{1.853612in}}%
\pgfpathlineto{\pgfqpoint{3.956870in}{1.820606in}}%
\pgfpathlineto{\pgfqpoint{3.957890in}{1.960883in}}%
\pgfpathlineto{\pgfqpoint{3.958299in}{1.911373in}}%
\pgfpathlineto{\pgfqpoint{3.960544in}{1.688580in}}%
\pgfpathlineto{\pgfqpoint{3.960952in}{1.762845in}}%
\pgfpathlineto{\pgfqpoint{3.961157in}{1.507045in}}%
\pgfpathlineto{\pgfqpoint{3.961769in}{1.837109in}}%
\pgfpathlineto{\pgfqpoint{3.961973in}{1.812354in}}%
\pgfpathlineto{\pgfqpoint{3.962177in}{1.886619in}}%
\pgfpathlineto{\pgfqpoint{3.962586in}{1.746341in}}%
\pgfpathlineto{\pgfqpoint{3.962790in}{1.746341in}}%
\pgfpathlineto{\pgfqpoint{3.962994in}{1.729838in}}%
\pgfpathlineto{\pgfqpoint{3.963198in}{1.828857in}}%
\pgfpathlineto{\pgfqpoint{3.964219in}{1.795851in}}%
\pgfpathlineto{\pgfqpoint{3.964423in}{1.738090in}}%
\pgfpathlineto{\pgfqpoint{3.964831in}{1.820606in}}%
\pgfpathlineto{\pgfqpoint{3.965239in}{1.762845in}}%
\pgfpathlineto{\pgfqpoint{3.965444in}{1.787599in}}%
\pgfpathlineto{\pgfqpoint{3.965648in}{1.688580in}}%
\pgfpathlineto{\pgfqpoint{3.965852in}{1.721587in}}%
\pgfpathlineto{\pgfqpoint{3.966056in}{1.680329in}}%
\pgfpathlineto{\pgfqpoint{3.966260in}{1.837109in}}%
\pgfpathlineto{\pgfqpoint{3.966668in}{1.779348in}}%
\pgfpathlineto{\pgfqpoint{3.967893in}{1.845361in}}%
\pgfpathlineto{\pgfqpoint{3.967077in}{1.754593in}}%
\pgfpathlineto{\pgfqpoint{3.968097in}{1.828857in}}%
\pgfpathlineto{\pgfqpoint{3.968914in}{1.771096in}}%
\pgfpathlineto{\pgfqpoint{3.968506in}{1.837109in}}%
\pgfpathlineto{\pgfqpoint{3.969118in}{1.804103in}}%
\pgfpathlineto{\pgfqpoint{3.969526in}{1.894870in}}%
\pgfpathlineto{\pgfqpoint{3.969731in}{1.812354in}}%
\pgfpathlineto{\pgfqpoint{3.969935in}{1.762845in}}%
\pgfpathlineto{\pgfqpoint{3.970547in}{1.870115in}}%
\pgfpathlineto{\pgfqpoint{3.970751in}{1.870115in}}%
\pgfpathlineto{\pgfqpoint{3.971772in}{1.837109in}}%
\pgfpathlineto{\pgfqpoint{3.971976in}{1.878367in}}%
\pgfpathlineto{\pgfqpoint{3.972384in}{1.812354in}}%
\pgfpathlineto{\pgfqpoint{3.973813in}{1.573058in}}%
\pgfpathlineto{\pgfqpoint{3.975242in}{1.754593in}}%
\pgfpathlineto{\pgfqpoint{3.975447in}{1.713335in}}%
\pgfpathlineto{\pgfqpoint{3.977080in}{1.540051in}}%
\pgfpathlineto{\pgfqpoint{3.976059in}{1.738090in}}%
\pgfpathlineto{\pgfqpoint{3.977488in}{1.581309in}}%
\pgfpathlineto{\pgfqpoint{3.978305in}{1.738090in}}%
\pgfpathlineto{\pgfqpoint{3.979121in}{1.680329in}}%
\pgfpathlineto{\pgfqpoint{3.979325in}{1.630819in}}%
\pgfpathlineto{\pgfqpoint{3.979734in}{1.729838in}}%
\pgfpathlineto{\pgfqpoint{3.979938in}{1.705083in}}%
\pgfpathlineto{\pgfqpoint{3.981979in}{1.861864in}}%
\pgfpathlineto{\pgfqpoint{3.983000in}{1.721587in}}%
\pgfpathlineto{\pgfqpoint{3.983204in}{1.738090in}}%
\pgfpathlineto{\pgfqpoint{3.983408in}{1.754593in}}%
\pgfpathlineto{\pgfqpoint{3.983816in}{1.713335in}}%
\pgfpathlineto{\pgfqpoint{3.984021in}{1.721587in}}%
\pgfpathlineto{\pgfqpoint{3.984429in}{1.688580in}}%
\pgfpathlineto{\pgfqpoint{3.984633in}{1.779348in}}%
\pgfpathlineto{\pgfqpoint{3.985245in}{1.705083in}}%
\pgfpathlineto{\pgfqpoint{3.986674in}{1.597813in}}%
\pgfpathlineto{\pgfqpoint{3.987695in}{1.705083in}}%
\pgfpathlineto{\pgfqpoint{3.987899in}{1.655574in}}%
\pgfpathlineto{\pgfqpoint{3.988103in}{1.630819in}}%
\pgfpathlineto{\pgfqpoint{3.988512in}{1.713335in}}%
\pgfpathlineto{\pgfqpoint{3.989532in}{1.647322in}}%
\pgfpathlineto{\pgfqpoint{3.989737in}{1.721587in}}%
\pgfpathlineto{\pgfqpoint{3.990145in}{1.606064in}}%
\pgfpathlineto{\pgfqpoint{3.990553in}{1.680329in}}%
\pgfpathlineto{\pgfqpoint{3.990757in}{1.680329in}}%
\pgfpathlineto{\pgfqpoint{3.990961in}{1.663825in}}%
\pgfpathlineto{\pgfqpoint{3.991166in}{1.688580in}}%
\pgfpathlineto{\pgfqpoint{3.991370in}{1.688580in}}%
\pgfpathlineto{\pgfqpoint{3.992186in}{1.746341in}}%
\pgfpathlineto{\pgfqpoint{3.992595in}{1.721587in}}%
\pgfpathlineto{\pgfqpoint{3.993003in}{1.663825in}}%
\pgfpathlineto{\pgfqpoint{3.993207in}{1.746341in}}%
\pgfpathlineto{\pgfqpoint{3.993615in}{1.688580in}}%
\pgfpathlineto{\pgfqpoint{3.994024in}{1.680329in}}%
\pgfpathlineto{\pgfqpoint{3.994432in}{1.738090in}}%
\pgfpathlineto{\pgfqpoint{3.995657in}{1.630819in}}%
\pgfpathlineto{\pgfqpoint{3.996065in}{1.746341in}}%
\pgfpathlineto{\pgfqpoint{3.997086in}{1.688580in}}%
\pgfpathlineto{\pgfqpoint{3.997494in}{1.738090in}}%
\pgfpathlineto{\pgfqpoint{3.997902in}{1.680329in}}%
\pgfpathlineto{\pgfqpoint{3.998106in}{1.721587in}}%
\pgfpathlineto{\pgfqpoint{3.999331in}{1.606064in}}%
\pgfpathlineto{\pgfqpoint{3.999535in}{1.663825in}}%
\pgfpathlineto{\pgfqpoint{4.000352in}{1.622567in}}%
\pgfpathlineto{\pgfqpoint{4.000760in}{1.680329in}}%
\pgfpathlineto{\pgfqpoint{4.002189in}{1.795851in}}%
\pgfpathlineto{\pgfqpoint{4.002393in}{1.746341in}}%
\pgfpathlineto{\pgfqpoint{4.003006in}{1.779348in}}%
\pgfpathlineto{\pgfqpoint{4.003414in}{1.696832in}}%
\pgfpathlineto{\pgfqpoint{4.003618in}{1.738090in}}%
\pgfpathlineto{\pgfqpoint{4.004231in}{1.672077in}}%
\pgfpathlineto{\pgfqpoint{4.004639in}{1.630819in}}%
\pgfpathlineto{\pgfqpoint{4.005251in}{1.680329in}}%
\pgfpathlineto{\pgfqpoint{4.005456in}{1.688580in}}%
\pgfpathlineto{\pgfqpoint{4.006068in}{1.713335in}}%
\pgfpathlineto{\pgfqpoint{4.006476in}{1.622567in}}%
\pgfpathlineto{\pgfqpoint{4.007701in}{1.696832in}}%
\pgfpathlineto{\pgfqpoint{4.009334in}{1.540051in}}%
\pgfpathlineto{\pgfqpoint{4.009947in}{1.713335in}}%
\pgfpathlineto{\pgfqpoint{4.010559in}{1.647322in}}%
\pgfpathlineto{\pgfqpoint{4.010763in}{1.639071in}}%
\pgfpathlineto{\pgfqpoint{4.010967in}{1.655574in}}%
\pgfpathlineto{\pgfqpoint{4.011172in}{1.696832in}}%
\pgfpathlineto{\pgfqpoint{4.011580in}{1.614316in}}%
\pgfpathlineto{\pgfqpoint{4.011784in}{1.375019in}}%
\pgfpathlineto{\pgfqpoint{4.012601in}{1.581309in}}%
\pgfpathlineto{\pgfqpoint{4.012805in}{1.573058in}}%
\pgfpathlineto{\pgfqpoint{4.013009in}{1.606064in}}%
\pgfpathlineto{\pgfqpoint{4.013213in}{1.606064in}}%
\pgfpathlineto{\pgfqpoint{4.014234in}{1.713335in}}%
\pgfpathlineto{\pgfqpoint{4.014438in}{1.639071in}}%
\pgfpathlineto{\pgfqpoint{4.014642in}{1.424529in}}%
\pgfpathlineto{\pgfqpoint{4.014846in}{1.663825in}}%
\pgfpathlineto{\pgfqpoint{4.015459in}{1.581309in}}%
\pgfpathlineto{\pgfqpoint{4.015663in}{1.556554in}}%
\pgfpathlineto{\pgfqpoint{4.016071in}{1.614316in}}%
\pgfpathlineto{\pgfqpoint{4.016275in}{1.614316in}}%
\pgfpathlineto{\pgfqpoint{4.017092in}{1.688580in}}%
\pgfpathlineto{\pgfqpoint{4.017296in}{1.647322in}}%
\pgfpathlineto{\pgfqpoint{4.018317in}{1.548303in}}%
\pgfpathlineto{\pgfqpoint{4.017704in}{1.688580in}}%
\pgfpathlineto{\pgfqpoint{4.018521in}{1.597813in}}%
\pgfpathlineto{\pgfqpoint{4.019133in}{1.680329in}}%
\pgfpathlineto{\pgfqpoint{4.019337in}{1.564806in}}%
\pgfpathlineto{\pgfqpoint{4.019541in}{1.523548in}}%
\pgfpathlineto{\pgfqpoint{4.020154in}{1.614316in}}%
\pgfpathlineto{\pgfqpoint{4.020358in}{1.614316in}}%
\pgfpathlineto{\pgfqpoint{4.020970in}{1.606064in}}%
\pgfpathlineto{\pgfqpoint{4.021991in}{1.696832in}}%
\pgfpathlineto{\pgfqpoint{4.023012in}{1.531800in}}%
\pgfpathlineto{\pgfqpoint{4.023420in}{1.630819in}}%
\pgfpathlineto{\pgfqpoint{4.025053in}{1.713335in}}%
\pgfpathlineto{\pgfqpoint{4.024237in}{1.622567in}}%
\pgfpathlineto{\pgfqpoint{4.025257in}{1.680329in}}%
\pgfpathlineto{\pgfqpoint{4.025666in}{1.614316in}}%
\pgfpathlineto{\pgfqpoint{4.026074in}{1.713335in}}%
\pgfpathlineto{\pgfqpoint{4.026278in}{1.688580in}}%
\pgfpathlineto{\pgfqpoint{4.026891in}{1.655574in}}%
\pgfpathlineto{\pgfqpoint{4.027503in}{1.779348in}}%
\pgfpathlineto{\pgfqpoint{4.027911in}{1.672077in}}%
\pgfpathlineto{\pgfqpoint{4.028115in}{1.663825in}}%
\pgfpathlineto{\pgfqpoint{4.028320in}{1.672077in}}%
\pgfpathlineto{\pgfqpoint{4.028524in}{1.696832in}}%
\pgfpathlineto{\pgfqpoint{4.028932in}{1.647322in}}%
\pgfpathlineto{\pgfqpoint{4.029340in}{1.680329in}}%
\pgfpathlineto{\pgfqpoint{4.029749in}{1.614316in}}%
\pgfpathlineto{\pgfqpoint{4.030157in}{1.630819in}}%
\pgfpathlineto{\pgfqpoint{4.030361in}{1.416277in}}%
\pgfpathlineto{\pgfqpoint{4.031178in}{1.672077in}}%
\pgfpathlineto{\pgfqpoint{4.031994in}{1.655574in}}%
\pgfpathlineto{\pgfqpoint{4.032198in}{1.729838in}}%
\pgfpathlineto{\pgfqpoint{4.032402in}{1.482290in}}%
\pgfpathlineto{\pgfqpoint{4.033219in}{1.672077in}}%
\pgfpathlineto{\pgfqpoint{4.033423in}{1.655574in}}%
\pgfpathlineto{\pgfqpoint{4.033831in}{1.705083in}}%
\pgfpathlineto{\pgfqpoint{4.034036in}{1.713335in}}%
\pgfpathlineto{\pgfqpoint{4.034240in}{1.705083in}}%
\pgfpathlineto{\pgfqpoint{4.034444in}{1.465787in}}%
\pgfpathlineto{\pgfqpoint{4.034852in}{1.721587in}}%
\pgfpathlineto{\pgfqpoint{4.035260in}{1.672077in}}%
\pgfpathlineto{\pgfqpoint{4.035465in}{1.688580in}}%
\pgfpathlineto{\pgfqpoint{4.035669in}{1.622567in}}%
\pgfpathlineto{\pgfqpoint{4.036281in}{1.672077in}}%
\pgfpathlineto{\pgfqpoint{4.036485in}{1.597813in}}%
\pgfpathlineto{\pgfqpoint{4.037506in}{1.622567in}}%
\pgfpathlineto{\pgfqpoint{4.037710in}{1.639071in}}%
\pgfpathlineto{\pgfqpoint{4.037914in}{1.622567in}}%
\pgfpathlineto{\pgfqpoint{4.038118in}{1.564806in}}%
\pgfpathlineto{\pgfqpoint{4.038935in}{1.573058in}}%
\pgfpathlineto{\pgfqpoint{4.040364in}{1.655574in}}%
\pgfpathlineto{\pgfqpoint{4.041180in}{1.606064in}}%
\pgfpathlineto{\pgfqpoint{4.041589in}{1.713335in}}%
\pgfpathlineto{\pgfqpoint{4.042201in}{1.606064in}}%
\pgfpathlineto{\pgfqpoint{4.042405in}{1.606064in}}%
\pgfpathlineto{\pgfqpoint{4.043630in}{1.482290in}}%
\pgfpathlineto{\pgfqpoint{4.043834in}{1.523548in}}%
\pgfpathlineto{\pgfqpoint{4.044038in}{1.515296in}}%
\pgfpathlineto{\pgfqpoint{4.044651in}{1.622567in}}%
\pgfpathlineto{\pgfqpoint{4.045263in}{1.581309in}}%
\pgfpathlineto{\pgfqpoint{4.046284in}{1.474038in}}%
\pgfpathlineto{\pgfqpoint{4.046488in}{1.540051in}}%
\pgfpathlineto{\pgfqpoint{4.047101in}{1.457535in}}%
\pgfpathlineto{\pgfqpoint{4.047305in}{1.490542in}}%
\pgfpathlineto{\pgfqpoint{4.048121in}{1.375019in}}%
\pgfpathlineto{\pgfqpoint{4.048530in}{1.424529in}}%
\pgfpathlineto{\pgfqpoint{4.049754in}{1.523548in}}%
\pgfpathlineto{\pgfqpoint{4.049959in}{1.515296in}}%
\pgfpathlineto{\pgfqpoint{4.050163in}{1.465787in}}%
\pgfpathlineto{\pgfqpoint{4.050367in}{1.531800in}}%
\pgfpathlineto{\pgfqpoint{4.050979in}{1.531800in}}%
\pgfpathlineto{\pgfqpoint{4.051592in}{1.573058in}}%
\pgfpathlineto{\pgfqpoint{4.052817in}{1.424529in}}%
\pgfpathlineto{\pgfqpoint{4.053021in}{1.424529in}}%
\pgfpathlineto{\pgfqpoint{4.054450in}{1.556554in}}%
\pgfpathlineto{\pgfqpoint{4.055266in}{1.474038in}}%
\pgfpathlineto{\pgfqpoint{4.055675in}{1.482290in}}%
\pgfpathlineto{\pgfqpoint{4.056287in}{1.548303in}}%
\pgfpathlineto{\pgfqpoint{4.056695in}{1.490542in}}%
\pgfpathlineto{\pgfqpoint{4.056899in}{1.465787in}}%
\pgfpathlineto{\pgfqpoint{4.057308in}{1.540051in}}%
\pgfpathlineto{\pgfqpoint{4.058941in}{1.589561in}}%
\pgfpathlineto{\pgfqpoint{4.059349in}{1.564806in}}%
\pgfpathlineto{\pgfqpoint{4.059962in}{1.614316in}}%
\pgfpathlineto{\pgfqpoint{4.060982in}{1.531800in}}%
\pgfpathlineto{\pgfqpoint{4.060370in}{1.647322in}}%
\pgfpathlineto{\pgfqpoint{4.061391in}{1.548303in}}%
\pgfpathlineto{\pgfqpoint{4.062820in}{1.672077in}}%
\pgfpathlineto{\pgfqpoint{4.063432in}{1.663825in}}%
\pgfpathlineto{\pgfqpoint{4.063636in}{1.573058in}}%
\pgfpathlineto{\pgfqpoint{4.064453in}{1.630819in}}%
\pgfpathlineto{\pgfqpoint{4.064657in}{1.614316in}}%
\pgfpathlineto{\pgfqpoint{4.064861in}{1.655574in}}%
\pgfpathlineto{\pgfqpoint{4.065065in}{1.639071in}}%
\pgfpathlineto{\pgfqpoint{4.066086in}{1.754593in}}%
\pgfpathlineto{\pgfqpoint{4.066290in}{1.688580in}}%
\pgfpathlineto{\pgfqpoint{4.067719in}{1.556554in}}%
\pgfpathlineto{\pgfqpoint{4.067923in}{1.573058in}}%
\pgfpathlineto{\pgfqpoint{4.068127in}{1.507045in}}%
\pgfpathlineto{\pgfqpoint{4.068331in}{1.507045in}}%
\pgfpathlineto{\pgfqpoint{4.070577in}{1.622567in}}%
\pgfpathlineto{\pgfqpoint{4.070781in}{1.606064in}}%
\pgfpathlineto{\pgfqpoint{4.070985in}{1.663825in}}%
\pgfpathlineto{\pgfqpoint{4.071189in}{1.861864in}}%
\pgfpathlineto{\pgfqpoint{4.071394in}{1.589561in}}%
\pgfpathlineto{\pgfqpoint{4.072006in}{1.655574in}}%
\pgfpathlineto{\pgfqpoint{4.072210in}{1.457535in}}%
\pgfpathlineto{\pgfqpoint{4.072414in}{1.754593in}}%
\pgfpathlineto{\pgfqpoint{4.073027in}{1.688580in}}%
\pgfpathlineto{\pgfqpoint{4.074456in}{1.597813in}}%
\pgfpathlineto{\pgfqpoint{4.075681in}{1.713335in}}%
\pgfpathlineto{\pgfqpoint{4.076089in}{1.639071in}}%
\pgfpathlineto{\pgfqpoint{4.076293in}{1.672077in}}%
\pgfpathlineto{\pgfqpoint{4.076701in}{1.358516in}}%
\pgfpathlineto{\pgfqpoint{4.077314in}{1.573058in}}%
\pgfpathlineto{\pgfqpoint{4.078947in}{1.729838in}}%
\pgfpathlineto{\pgfqpoint{4.079968in}{1.531800in}}%
\pgfpathlineto{\pgfqpoint{4.080172in}{1.573058in}}%
\pgfpathlineto{\pgfqpoint{4.081601in}{1.441032in}}%
\pgfpathlineto{\pgfqpoint{4.082417in}{1.589561in}}%
\pgfpathlineto{\pgfqpoint{4.082621in}{1.457535in}}%
\pgfpathlineto{\pgfqpoint{4.083234in}{1.218239in}}%
\pgfpathlineto{\pgfqpoint{4.084050in}{1.284252in}}%
\pgfpathlineto{\pgfqpoint{4.084255in}{1.292503in}}%
\pgfpathlineto{\pgfqpoint{4.085071in}{1.020200in}}%
\pgfpathlineto{\pgfqpoint{4.085275in}{1.350264in}}%
\pgfpathlineto{\pgfqpoint{4.086296in}{1.251245in}}%
\pgfpathlineto{\pgfqpoint{4.087521in}{0.954187in}}%
\pgfpathlineto{\pgfqpoint{4.087725in}{1.069710in}}%
\pgfpathlineto{\pgfqpoint{4.089154in}{1.218239in}}%
\pgfpathlineto{\pgfqpoint{4.090583in}{0.822162in}}%
\pgfpathlineto{\pgfqpoint{4.090787in}{1.020200in}}%
\pgfpathlineto{\pgfqpoint{4.091400in}{0.937684in}}%
\pgfpathlineto{\pgfqpoint{4.092012in}{0.970691in}}%
\pgfpathlineto{\pgfqpoint{4.093237in}{1.201736in}}%
\pgfpathlineto{\pgfqpoint{4.093441in}{1.143974in}}%
\pgfpathlineto{\pgfqpoint{4.094462in}{0.764400in}}%
\pgfpathlineto{\pgfqpoint{4.094666in}{1.028452in}}%
\pgfpathlineto{\pgfqpoint{4.094870in}{1.028452in}}%
\pgfpathlineto{\pgfqpoint{4.095074in}{1.061458in}}%
\pgfpathlineto{\pgfqpoint{4.095278in}{1.003697in}}%
\pgfpathlineto{\pgfqpoint{4.095482in}{1.028452in}}%
\pgfpathlineto{\pgfqpoint{4.096503in}{0.789155in}}%
\pgfpathlineto{\pgfqpoint{4.097116in}{0.830413in}}%
\pgfpathlineto{\pgfqpoint{4.098545in}{1.160477in}}%
\pgfpathlineto{\pgfqpoint{4.099157in}{1.003697in}}%
\pgfpathlineto{\pgfqpoint{4.099361in}{0.970691in}}%
\pgfpathlineto{\pgfqpoint{4.099974in}{1.044955in}}%
\pgfpathlineto{\pgfqpoint{4.100790in}{1.185232in}}%
\pgfpathlineto{\pgfqpoint{4.101198in}{1.143974in}}%
\pgfpathlineto{\pgfqpoint{4.102015in}{0.805659in}}%
\pgfpathlineto{\pgfqpoint{4.102423in}{0.888175in}}%
\pgfpathlineto{\pgfqpoint{4.103852in}{1.218239in}}%
\pgfpathlineto{\pgfqpoint{4.104669in}{1.094465in}}%
\pgfpathlineto{\pgfqpoint{4.105077in}{1.160477in}}%
\pgfpathlineto{\pgfqpoint{4.105281in}{1.234742in}}%
\pgfpathlineto{\pgfqpoint{4.105894in}{1.127471in}}%
\pgfpathlineto{\pgfqpoint{4.106098in}{1.160477in}}%
\pgfpathlineto{\pgfqpoint{4.106710in}{1.086213in}}%
\pgfpathlineto{\pgfqpoint{4.107323in}{1.135723in}}%
\pgfpathlineto{\pgfqpoint{4.107527in}{1.168729in}}%
\pgfpathlineto{\pgfqpoint{4.107731in}{1.127471in}}%
\pgfpathlineto{\pgfqpoint{4.108139in}{1.044955in}}%
\pgfpathlineto{\pgfqpoint{4.108956in}{1.069710in}}%
\pgfpathlineto{\pgfqpoint{4.109568in}{1.119219in}}%
\pgfpathlineto{\pgfqpoint{4.109772in}{1.061458in}}%
\pgfpathlineto{\pgfqpoint{4.110181in}{1.028452in}}%
\pgfpathlineto{\pgfqpoint{4.110385in}{1.086213in}}%
\pgfpathlineto{\pgfqpoint{4.110589in}{1.086213in}}%
\pgfpathlineto{\pgfqpoint{4.110997in}{1.127471in}}%
\pgfpathlineto{\pgfqpoint{4.111406in}{1.119219in}}%
\pgfpathlineto{\pgfqpoint{4.111610in}{1.077961in}}%
\pgfpathlineto{\pgfqpoint{4.111814in}{1.176981in}}%
\pgfpathlineto{\pgfqpoint{4.112222in}{1.300755in}}%
\pgfpathlineto{\pgfqpoint{4.112426in}{1.201736in}}%
\pgfpathlineto{\pgfqpoint{4.112630in}{1.102716in}}%
\pgfpathlineto{\pgfqpoint{4.113243in}{1.317258in}}%
\pgfpathlineto{\pgfqpoint{4.114468in}{1.209987in}}%
\pgfpathlineto{\pgfqpoint{4.115080in}{1.366768in}}%
\pgfpathlineto{\pgfqpoint{4.115693in}{1.267748in}}%
\pgfpathlineto{\pgfqpoint{4.116509in}{1.358516in}}%
\pgfpathlineto{\pgfqpoint{4.116713in}{1.300755in}}%
\pgfpathlineto{\pgfqpoint{4.116917in}{1.209987in}}%
\pgfpathlineto{\pgfqpoint{4.117326in}{1.408026in}}%
\pgfpathlineto{\pgfqpoint{4.117530in}{1.432780in}}%
\pgfpathlineto{\pgfqpoint{4.117734in}{1.366768in}}%
\pgfpathlineto{\pgfqpoint{4.118346in}{1.309006in}}%
\pgfpathlineto{\pgfqpoint{4.118142in}{1.383271in}}%
\pgfpathlineto{\pgfqpoint{4.118551in}{1.350264in}}%
\pgfpathlineto{\pgfqpoint{4.118755in}{1.391522in}}%
\pgfpathlineto{\pgfqpoint{4.118959in}{1.300755in}}%
\pgfpathlineto{\pgfqpoint{4.119775in}{1.383271in}}%
\pgfpathlineto{\pgfqpoint{4.119980in}{1.383271in}}%
\pgfpathlineto{\pgfqpoint{4.121204in}{1.515296in}}%
\pgfpathlineto{\pgfqpoint{4.120388in}{1.366768in}}%
\pgfpathlineto{\pgfqpoint{4.121409in}{1.490542in}}%
\pgfpathlineto{\pgfqpoint{4.122225in}{1.507045in}}%
\pgfpathlineto{\pgfqpoint{4.122633in}{1.449284in}}%
\pgfpathlineto{\pgfqpoint{4.122838in}{1.515296in}}%
\pgfpathlineto{\pgfqpoint{4.123246in}{1.391522in}}%
\pgfpathlineto{\pgfqpoint{4.123858in}{1.490542in}}%
\pgfpathlineto{\pgfqpoint{4.125287in}{1.201736in}}%
\pgfpathlineto{\pgfqpoint{4.126512in}{1.523548in}}%
\pgfpathlineto{\pgfqpoint{4.127533in}{1.342013in}}%
\pgfpathlineto{\pgfqpoint{4.127737in}{1.441032in}}%
\pgfpathlineto{\pgfqpoint{4.128349in}{1.498793in}}%
\pgfpathlineto{\pgfqpoint{4.128962in}{1.482290in}}%
\pgfpathlineto{\pgfqpoint{4.129370in}{1.507045in}}%
\pgfpathlineto{\pgfqpoint{4.130187in}{1.408026in}}%
\pgfpathlineto{\pgfqpoint{4.131616in}{1.564806in}}%
\pgfpathlineto{\pgfqpoint{4.131820in}{1.556554in}}%
\pgfpathlineto{\pgfqpoint{4.132432in}{1.614316in}}%
\pgfpathlineto{\pgfqpoint{4.132636in}{1.573058in}}%
\pgfpathlineto{\pgfqpoint{4.134474in}{1.399774in}}%
\pgfpathlineto{\pgfqpoint{4.135494in}{1.540051in}}%
\pgfpathlineto{\pgfqpoint{4.135903in}{1.531800in}}%
\pgfpathlineto{\pgfqpoint{4.136719in}{1.498793in}}%
\pgfpathlineto{\pgfqpoint{4.138148in}{1.639071in}}%
\pgfpathlineto{\pgfqpoint{4.139169in}{1.573058in}}%
\pgfpathlineto{\pgfqpoint{4.139373in}{1.606064in}}%
\pgfpathlineto{\pgfqpoint{4.139781in}{1.606064in}}%
\pgfpathlineto{\pgfqpoint{4.140394in}{1.688580in}}%
\pgfpathlineto{\pgfqpoint{4.141210in}{1.663825in}}%
\pgfpathlineto{\pgfqpoint{4.141414in}{1.647322in}}%
\pgfpathlineto{\pgfqpoint{4.141619in}{1.663825in}}%
\pgfpathlineto{\pgfqpoint{4.142027in}{1.721587in}}%
\pgfpathlineto{\pgfqpoint{4.142639in}{1.705083in}}%
\pgfpathlineto{\pgfqpoint{4.142843in}{1.655574in}}%
\pgfpathlineto{\pgfqpoint{4.143456in}{1.771096in}}%
\pgfpathlineto{\pgfqpoint{4.143660in}{1.729838in}}%
\pgfpathlineto{\pgfqpoint{4.143864in}{1.779348in}}%
\pgfpathlineto{\pgfqpoint{4.144681in}{1.746341in}}%
\pgfpathlineto{\pgfqpoint{4.145701in}{1.828857in}}%
\pgfpathlineto{\pgfqpoint{4.145906in}{1.812354in}}%
\pgfpathlineto{\pgfqpoint{4.146110in}{1.771096in}}%
\pgfpathlineto{\pgfqpoint{4.146518in}{1.886619in}}%
\pgfpathlineto{\pgfqpoint{4.146722in}{1.837109in}}%
\pgfpathlineto{\pgfqpoint{4.146926in}{1.837109in}}%
\pgfpathlineto{\pgfqpoint{4.148355in}{1.952632in}}%
\pgfpathlineto{\pgfqpoint{4.148764in}{1.919625in}}%
\pgfpathlineto{\pgfqpoint{4.148968in}{1.894870in}}%
\pgfpathlineto{\pgfqpoint{4.149376in}{1.977386in}}%
\pgfpathlineto{\pgfqpoint{4.149580in}{1.969135in}}%
\pgfpathlineto{\pgfqpoint{4.150805in}{2.026896in}}%
\pgfpathlineto{\pgfqpoint{4.151417in}{2.018644in}}%
\pgfpathlineto{\pgfqpoint{4.151622in}{2.043399in}}%
\pgfpathlineto{\pgfqpoint{4.151826in}{1.853612in}}%
\pgfpathlineto{\pgfqpoint{4.152642in}{2.101160in}}%
\pgfpathlineto{\pgfqpoint{4.152846in}{2.076406in}}%
\pgfpathlineto{\pgfqpoint{4.153051in}{2.134167in}}%
\pgfpathlineto{\pgfqpoint{4.153663in}{2.109412in}}%
\pgfpathlineto{\pgfqpoint{4.153867in}{2.092909in}}%
\pgfpathlineto{\pgfqpoint{4.154071in}{1.977386in}}%
\pgfpathlineto{\pgfqpoint{4.154888in}{2.068154in}}%
\pgfpathlineto{\pgfqpoint{4.155500in}{1.894870in}}%
\pgfpathlineto{\pgfqpoint{4.156113in}{1.911373in}}%
\pgfpathlineto{\pgfqpoint{4.156929in}{2.134167in}}%
\pgfpathlineto{\pgfqpoint{4.157338in}{2.117664in}}%
\pgfpathlineto{\pgfqpoint{4.157746in}{2.158922in}}%
\pgfpathlineto{\pgfqpoint{4.157950in}{1.845361in}}%
\pgfpathlineto{\pgfqpoint{4.158767in}{2.175425in}}%
\pgfpathlineto{\pgfqpoint{4.158971in}{2.208431in}}%
\pgfpathlineto{\pgfqpoint{4.159583in}{2.142418in}}%
\pgfpathlineto{\pgfqpoint{4.159787in}{2.125915in}}%
\pgfpathlineto{\pgfqpoint{4.159991in}{2.158922in}}%
\pgfpathlineto{\pgfqpoint{4.160196in}{2.142418in}}%
\pgfpathlineto{\pgfqpoint{4.160400in}{2.200180in}}%
\pgfpathlineto{\pgfqpoint{4.161216in}{2.175425in}}%
\pgfpathlineto{\pgfqpoint{4.162645in}{2.051651in}}%
\pgfpathlineto{\pgfqpoint{4.163666in}{2.043399in}}%
\pgfpathlineto{\pgfqpoint{4.163870in}{2.150670in}}%
\pgfpathlineto{\pgfqpoint{4.164278in}{2.109412in}}%
\pgfpathlineto{\pgfqpoint{4.164687in}{2.183676in}}%
\pgfpathlineto{\pgfqpoint{4.164891in}{2.191928in}}%
\pgfpathlineto{\pgfqpoint{4.165707in}{2.216683in}}%
\pgfpathlineto{\pgfqpoint{4.166116in}{2.010393in}}%
\pgfpathlineto{\pgfqpoint{4.166320in}{2.026896in}}%
\pgfpathlineto{\pgfqpoint{4.166524in}{2.010393in}}%
\pgfpathlineto{\pgfqpoint{4.166728in}{1.969135in}}%
\pgfpathlineto{\pgfqpoint{4.167136in}{2.233186in}}%
\pgfpathlineto{\pgfqpoint{4.167953in}{2.142418in}}%
\pgfpathlineto{\pgfqpoint{4.168974in}{2.233186in}}%
\pgfpathlineto{\pgfqpoint{4.169382in}{2.208431in}}%
\pgfpathlineto{\pgfqpoint{4.170199in}{2.068154in}}%
\pgfpathlineto{\pgfqpoint{4.170607in}{2.125915in}}%
\pgfpathlineto{\pgfqpoint{4.171219in}{2.200180in}}%
\pgfpathlineto{\pgfqpoint{4.171628in}{2.109412in}}%
\pgfpathlineto{\pgfqpoint{4.171832in}{2.084657in}}%
\pgfpathlineto{\pgfqpoint{4.172036in}{2.142418in}}%
\pgfpathlineto{\pgfqpoint{4.172240in}{2.142418in}}%
\pgfpathlineto{\pgfqpoint{4.172444in}{2.241438in}}%
\pgfpathlineto{\pgfqpoint{4.173057in}{2.101160in}}%
\pgfpathlineto{\pgfqpoint{4.173465in}{2.208431in}}%
\pgfpathlineto{\pgfqpoint{4.173873in}{2.167173in}}%
\pgfpathlineto{\pgfqpoint{4.174690in}{2.233186in}}%
\pgfpathlineto{\pgfqpoint{4.175098in}{2.175425in}}%
\pgfpathlineto{\pgfqpoint{4.175506in}{2.233186in}}%
\pgfpathlineto{\pgfqpoint{4.175710in}{2.299199in}}%
\pgfpathlineto{\pgfqpoint{4.176527in}{2.208431in}}%
\pgfpathlineto{\pgfqpoint{4.176731in}{2.290947in}}%
\pgfpathlineto{\pgfqpoint{4.178160in}{2.208431in}}%
\pgfpathlineto{\pgfqpoint{4.179181in}{2.315702in}}%
\pgfpathlineto{\pgfqpoint{4.180202in}{2.241438in}}%
\pgfpathlineto{\pgfqpoint{4.181018in}{2.323954in}}%
\pgfpathlineto{\pgfqpoint{4.181426in}{2.307450in}}%
\pgfpathlineto{\pgfqpoint{4.181631in}{2.315702in}}%
\pgfpathlineto{\pgfqpoint{4.182039in}{2.340457in}}%
\pgfpathlineto{\pgfqpoint{4.183264in}{2.150670in}}%
\pgfpathlineto{\pgfqpoint{4.185305in}{2.266192in}}%
\pgfpathlineto{\pgfqpoint{4.185713in}{2.035148in}}%
\pgfpathlineto{\pgfqpoint{4.186122in}{2.356960in}}%
\pgfpathlineto{\pgfqpoint{4.186326in}{2.266192in}}%
\pgfpathlineto{\pgfqpoint{4.186530in}{2.257941in}}%
\pgfpathlineto{\pgfqpoint{4.187347in}{2.323954in}}%
\pgfpathlineto{\pgfqpoint{4.187551in}{2.299199in}}%
\pgfpathlineto{\pgfqpoint{4.188163in}{2.257941in}}%
\pgfpathlineto{\pgfqpoint{4.188776in}{2.274444in}}%
\pgfpathlineto{\pgfqpoint{4.188980in}{2.109412in}}%
\pgfpathlineto{\pgfqpoint{4.189796in}{2.323954in}}%
\pgfpathlineto{\pgfqpoint{4.190613in}{2.356960in}}%
\pgfpathlineto{\pgfqpoint{4.190409in}{2.299199in}}%
\pgfpathlineto{\pgfqpoint{4.190817in}{2.332205in}}%
\pgfpathlineto{\pgfqpoint{4.191838in}{2.076406in}}%
\pgfpathlineto{\pgfqpoint{4.191429in}{2.356960in}}%
\pgfpathlineto{\pgfqpoint{4.192246in}{2.290947in}}%
\pgfpathlineto{\pgfqpoint{4.192450in}{2.299199in}}%
\pgfpathlineto{\pgfqpoint{4.193063in}{2.233186in}}%
\pgfpathlineto{\pgfqpoint{4.193267in}{2.257941in}}%
\pgfpathlineto{\pgfqpoint{4.193471in}{2.323954in}}%
\pgfpathlineto{\pgfqpoint{4.194287in}{2.249689in}}%
\pgfpathlineto{\pgfqpoint{4.195104in}{2.233186in}}%
\pgfpathlineto{\pgfqpoint{4.196533in}{2.356960in}}%
\pgfpathlineto{\pgfqpoint{4.196737in}{2.365212in}}%
\pgfpathlineto{\pgfqpoint{4.197554in}{2.233186in}}%
\pgfpathlineto{\pgfqpoint{4.197962in}{2.299199in}}%
\pgfpathlineto{\pgfqpoint{4.198983in}{2.348709in}}%
\pgfpathlineto{\pgfqpoint{4.199187in}{2.282696in}}%
\pgfpathlineto{\pgfqpoint{4.199799in}{2.373463in}}%
\pgfpathlineto{\pgfqpoint{4.200003in}{2.348709in}}%
\pgfpathlineto{\pgfqpoint{4.200208in}{2.381715in}}%
\pgfpathlineto{\pgfqpoint{4.200616in}{2.323954in}}%
\pgfpathlineto{\pgfqpoint{4.200820in}{2.356960in}}%
\pgfpathlineto{\pgfqpoint{4.201637in}{2.117664in}}%
\pgfpathlineto{\pgfqpoint{4.201841in}{2.307450in}}%
\pgfpathlineto{\pgfqpoint{4.202861in}{2.439476in}}%
\pgfpathlineto{\pgfqpoint{4.203066in}{2.389967in}}%
\pgfpathlineto{\pgfqpoint{4.203678in}{2.447728in}}%
\pgfpathlineto{\pgfqpoint{4.204086in}{2.422973in}}%
\pgfpathlineto{\pgfqpoint{4.205719in}{2.076406in}}%
\pgfpathlineto{\pgfqpoint{4.206944in}{2.340457in}}%
\pgfpathlineto{\pgfqpoint{4.207965in}{2.373463in}}%
\pgfpathlineto{\pgfqpoint{4.208169in}{2.365212in}}%
\pgfpathlineto{\pgfqpoint{4.208986in}{2.274444in}}%
\pgfpathlineto{\pgfqpoint{4.209190in}{2.373463in}}%
\pgfpathlineto{\pgfqpoint{4.209598in}{2.315702in}}%
\pgfpathlineto{\pgfqpoint{4.210006in}{2.373463in}}%
\pgfpathlineto{\pgfqpoint{4.210210in}{2.398218in}}%
\pgfpathlineto{\pgfqpoint{4.210415in}{2.307450in}}%
\pgfpathlineto{\pgfqpoint{4.210619in}{2.307450in}}%
\pgfpathlineto{\pgfqpoint{4.211027in}{2.257941in}}%
\pgfpathlineto{\pgfqpoint{4.211231in}{2.323954in}}%
\pgfpathlineto{\pgfqpoint{4.211639in}{2.406470in}}%
\pgfpathlineto{\pgfqpoint{4.212252in}{2.381715in}}%
\pgfpathlineto{\pgfqpoint{4.213885in}{2.158922in}}%
\pgfpathlineto{\pgfqpoint{4.214293in}{2.340457in}}%
\pgfpathlineto{\pgfqpoint{4.215110in}{2.282696in}}%
\pgfpathlineto{\pgfqpoint{4.215314in}{2.274444in}}%
\pgfpathlineto{\pgfqpoint{4.216335in}{2.389967in}}%
\pgfpathlineto{\pgfqpoint{4.215722in}{2.266192in}}%
\pgfpathlineto{\pgfqpoint{4.216539in}{2.365212in}}%
\pgfpathlineto{\pgfqpoint{4.217764in}{2.290947in}}%
\pgfpathlineto{\pgfqpoint{4.218989in}{2.422973in}}%
\pgfpathlineto{\pgfqpoint{4.220009in}{2.323954in}}%
\pgfpathlineto{\pgfqpoint{4.220213in}{2.365212in}}%
\pgfpathlineto{\pgfqpoint{4.221030in}{2.332205in}}%
\pgfpathlineto{\pgfqpoint{4.220622in}{2.389967in}}%
\pgfpathlineto{\pgfqpoint{4.221234in}{2.365212in}}%
\pgfpathlineto{\pgfqpoint{4.221642in}{2.381715in}}%
\pgfpathlineto{\pgfqpoint{4.221847in}{2.348709in}}%
\pgfpathlineto{\pgfqpoint{4.222051in}{2.365212in}}%
\pgfpathlineto{\pgfqpoint{4.223071in}{2.315702in}}%
\pgfpathlineto{\pgfqpoint{4.223276in}{2.406470in}}%
\pgfpathlineto{\pgfqpoint{4.224092in}{2.381715in}}%
\pgfpathlineto{\pgfqpoint{4.224296in}{2.340457in}}%
\pgfpathlineto{\pgfqpoint{4.224705in}{2.398218in}}%
\pgfpathlineto{\pgfqpoint{4.225113in}{2.398218in}}%
\pgfpathlineto{\pgfqpoint{4.225317in}{2.414721in}}%
\pgfpathlineto{\pgfqpoint{4.225521in}{2.389967in}}%
\pgfpathlineto{\pgfqpoint{4.225929in}{2.389967in}}%
\pgfpathlineto{\pgfqpoint{4.227563in}{2.274444in}}%
\pgfpathlineto{\pgfqpoint{4.229400in}{2.373463in}}%
\pgfpathlineto{\pgfqpoint{4.228175in}{2.257941in}}%
\pgfpathlineto{\pgfqpoint{4.229604in}{2.356960in}}%
\pgfpathlineto{\pgfqpoint{4.229808in}{2.216683in}}%
\pgfpathlineto{\pgfqpoint{4.230421in}{2.389967in}}%
\pgfpathlineto{\pgfqpoint{4.230625in}{2.356960in}}%
\pgfpathlineto{\pgfqpoint{4.231237in}{2.323954in}}%
\pgfpathlineto{\pgfqpoint{4.232054in}{2.422973in}}%
\pgfpathlineto{\pgfqpoint{4.232666in}{2.340457in}}%
\pgfpathlineto{\pgfqpoint{4.232870in}{2.431225in}}%
\pgfpathlineto{\pgfqpoint{4.233074in}{2.505489in}}%
\pgfpathlineto{\pgfqpoint{4.233687in}{2.356960in}}%
\pgfpathlineto{\pgfqpoint{4.233891in}{2.422973in}}%
\pgfpathlineto{\pgfqpoint{4.234503in}{2.431225in}}%
\pgfpathlineto{\pgfqpoint{4.235320in}{2.249689in}}%
\pgfpathlineto{\pgfqpoint{4.236137in}{2.340457in}}%
\pgfpathlineto{\pgfqpoint{4.236749in}{2.365212in}}%
\pgfpathlineto{\pgfqpoint{4.237361in}{2.299199in}}%
\pgfpathlineto{\pgfqpoint{4.237566in}{2.348709in}}%
\pgfpathlineto{\pgfqpoint{4.238382in}{2.340457in}}%
\pgfpathlineto{\pgfqpoint{4.239607in}{2.290947in}}%
\pgfpathlineto{\pgfqpoint{4.240015in}{2.356960in}}%
\pgfpathlineto{\pgfqpoint{4.240424in}{2.307450in}}%
\pgfpathlineto{\pgfqpoint{4.240628in}{2.266192in}}%
\pgfpathlineto{\pgfqpoint{4.241240in}{2.315702in}}%
\pgfpathlineto{\pgfqpoint{4.241444in}{2.315702in}}%
\pgfpathlineto{\pgfqpoint{4.243282in}{2.142418in}}%
\pgfpathlineto{\pgfqpoint{4.243486in}{2.191928in}}%
\pgfpathlineto{\pgfqpoint{4.244302in}{2.307450in}}%
\pgfpathlineto{\pgfqpoint{4.244915in}{2.274444in}}%
\pgfpathlineto{\pgfqpoint{4.246548in}{2.183676in}}%
\pgfpathlineto{\pgfqpoint{4.246956in}{2.134167in}}%
\pgfpathlineto{\pgfqpoint{4.247364in}{2.233186in}}%
\pgfpathlineto{\pgfqpoint{4.247977in}{2.290947in}}%
\pgfpathlineto{\pgfqpoint{4.248385in}{2.076406in}}%
\pgfpathlineto{\pgfqpoint{4.248793in}{2.332205in}}%
\pgfpathlineto{\pgfqpoint{4.248998in}{2.315702in}}%
\pgfpathlineto{\pgfqpoint{4.250222in}{2.431225in}}%
\pgfpathlineto{\pgfqpoint{4.250427in}{2.398218in}}%
\pgfpathlineto{\pgfqpoint{4.251039in}{2.323954in}}%
\pgfpathlineto{\pgfqpoint{4.250835in}{2.431225in}}%
\pgfpathlineto{\pgfqpoint{4.251651in}{2.348709in}}%
\pgfpathlineto{\pgfqpoint{4.252264in}{2.381715in}}%
\pgfpathlineto{\pgfqpoint{4.252876in}{2.076406in}}%
\pgfpathlineto{\pgfqpoint{4.253285in}{2.274444in}}%
\pgfpathlineto{\pgfqpoint{4.254509in}{2.406470in}}%
\pgfpathlineto{\pgfqpoint{4.253693in}{2.257941in}}%
\pgfpathlineto{\pgfqpoint{4.254714in}{2.348709in}}%
\pgfpathlineto{\pgfqpoint{4.256551in}{2.282696in}}%
\pgfpathlineto{\pgfqpoint{4.257163in}{2.332205in}}%
\pgfpathlineto{\pgfqpoint{4.257776in}{2.323954in}}%
\pgfpathlineto{\pgfqpoint{4.258592in}{2.307450in}}%
\pgfpathlineto{\pgfqpoint{4.258796in}{2.381715in}}%
\pgfpathlineto{\pgfqpoint{4.260021in}{2.101160in}}%
\pgfpathlineto{\pgfqpoint{4.260225in}{2.290947in}}%
\pgfpathlineto{\pgfqpoint{4.260430in}{2.356960in}}%
\pgfpathlineto{\pgfqpoint{4.261246in}{2.266192in}}%
\pgfpathlineto{\pgfqpoint{4.261450in}{2.290947in}}%
\pgfpathlineto{\pgfqpoint{4.261654in}{2.282696in}}%
\pgfpathlineto{\pgfqpoint{4.261859in}{2.084657in}}%
\pgfpathlineto{\pgfqpoint{4.262267in}{2.381715in}}%
\pgfpathlineto{\pgfqpoint{4.262675in}{2.150670in}}%
\pgfpathlineto{\pgfqpoint{4.263696in}{2.381715in}}%
\pgfpathlineto{\pgfqpoint{4.263900in}{2.348709in}}%
\pgfpathlineto{\pgfqpoint{4.264104in}{2.340457in}}%
\pgfpathlineto{\pgfqpoint{4.264308in}{2.348709in}}%
\pgfpathlineto{\pgfqpoint{4.264921in}{2.389967in}}%
\pgfpathlineto{\pgfqpoint{4.266146in}{2.290947in}}%
\pgfpathlineto{\pgfqpoint{4.266350in}{2.299199in}}%
\pgfpathlineto{\pgfqpoint{4.266554in}{2.282696in}}%
\pgfpathlineto{\pgfqpoint{4.266962in}{2.158922in}}%
\pgfpathlineto{\pgfqpoint{4.267779in}{2.257941in}}%
\pgfpathlineto{\pgfqpoint{4.268187in}{2.241438in}}%
\pgfpathlineto{\pgfqpoint{4.268391in}{2.257941in}}%
\pgfpathlineto{\pgfqpoint{4.268595in}{2.299199in}}%
\pgfpathlineto{\pgfqpoint{4.268799in}{2.101160in}}%
\pgfpathlineto{\pgfqpoint{4.269004in}{2.332205in}}%
\pgfpathlineto{\pgfqpoint{4.269616in}{2.274444in}}%
\pgfpathlineto{\pgfqpoint{4.270228in}{2.365212in}}%
\pgfpathlineto{\pgfqpoint{4.270637in}{2.315702in}}%
\pgfpathlineto{\pgfqpoint{4.270841in}{2.282696in}}%
\pgfpathlineto{\pgfqpoint{4.271045in}{2.389967in}}%
\pgfpathlineto{\pgfqpoint{4.271249in}{2.389967in}}%
\pgfpathlineto{\pgfqpoint{4.271657in}{2.365212in}}%
\pgfpathlineto{\pgfqpoint{4.271862in}{2.422973in}}%
\pgfpathlineto{\pgfqpoint{4.273086in}{2.117664in}}%
\pgfpathlineto{\pgfqpoint{4.274107in}{2.381715in}}%
\pgfpathlineto{\pgfqpoint{4.274311in}{2.323954in}}%
\pgfpathlineto{\pgfqpoint{4.274515in}{2.249689in}}%
\pgfpathlineto{\pgfqpoint{4.274720in}{2.431225in}}%
\pgfpathlineto{\pgfqpoint{4.275128in}{2.365212in}}%
\pgfpathlineto{\pgfqpoint{4.275536in}{2.414721in}}%
\pgfpathlineto{\pgfqpoint{4.276149in}{2.373463in}}%
\pgfpathlineto{\pgfqpoint{4.276761in}{2.389967in}}%
\pgfpathlineto{\pgfqpoint{4.277169in}{2.340457in}}%
\pgfpathlineto{\pgfqpoint{4.277578in}{2.332205in}}%
\pgfpathlineto{\pgfqpoint{4.278394in}{2.381715in}}%
\pgfpathlineto{\pgfqpoint{4.278598in}{2.373463in}}%
\pgfpathlineto{\pgfqpoint{4.279007in}{2.422973in}}%
\pgfpathlineto{\pgfqpoint{4.279415in}{2.356960in}}%
\pgfpathlineto{\pgfqpoint{4.279619in}{2.356960in}}%
\pgfpathlineto{\pgfqpoint{4.279823in}{2.315702in}}%
\pgfpathlineto{\pgfqpoint{4.280231in}{2.373463in}}%
\pgfpathlineto{\pgfqpoint{4.280844in}{2.480734in}}%
\pgfpathlineto{\pgfqpoint{4.281456in}{2.464231in}}%
\pgfpathlineto{\pgfqpoint{4.282477in}{2.406470in}}%
\pgfpathlineto{\pgfqpoint{4.282681in}{2.414721in}}%
\pgfpathlineto{\pgfqpoint{4.282885in}{2.447728in}}%
\pgfpathlineto{\pgfqpoint{4.283294in}{2.381715in}}%
\pgfpathlineto{\pgfqpoint{4.284314in}{2.282696in}}%
\pgfpathlineto{\pgfqpoint{4.284518in}{2.340457in}}%
\pgfpathlineto{\pgfqpoint{4.284927in}{2.109412in}}%
\pgfpathlineto{\pgfqpoint{4.285539in}{2.299199in}}%
\pgfpathlineto{\pgfqpoint{4.286356in}{2.422973in}}%
\pgfpathlineto{\pgfqpoint{4.286764in}{2.381715in}}%
\pgfpathlineto{\pgfqpoint{4.286968in}{2.373463in}}%
\pgfpathlineto{\pgfqpoint{4.287376in}{2.497237in}}%
\pgfpathlineto{\pgfqpoint{4.287581in}{2.175425in}}%
\pgfpathlineto{\pgfqpoint{4.288397in}{2.422973in}}%
\pgfpathlineto{\pgfqpoint{4.288601in}{2.406470in}}%
\pgfpathlineto{\pgfqpoint{4.288805in}{2.472483in}}%
\pgfpathlineto{\pgfqpoint{4.289010in}{2.447728in}}%
\pgfpathlineto{\pgfqpoint{4.289418in}{2.505489in}}%
\pgfpathlineto{\pgfqpoint{4.289826in}{2.356960in}}%
\pgfpathlineto{\pgfqpoint{4.290643in}{2.389967in}}%
\pgfpathlineto{\pgfqpoint{4.290847in}{2.398218in}}%
\pgfpathlineto{\pgfqpoint{4.291051in}{2.373463in}}%
\pgfpathlineto{\pgfqpoint{4.291459in}{2.389967in}}%
\pgfpathlineto{\pgfqpoint{4.291663in}{2.340457in}}%
\pgfpathlineto{\pgfqpoint{4.292480in}{2.365212in}}%
\pgfpathlineto{\pgfqpoint{4.292684in}{2.431225in}}%
\pgfpathlineto{\pgfqpoint{4.293296in}{2.290947in}}%
\pgfpathlineto{\pgfqpoint{4.293501in}{2.282696in}}%
\pgfpathlineto{\pgfqpoint{4.293705in}{2.290947in}}%
\pgfpathlineto{\pgfqpoint{4.295338in}{2.365212in}}%
\pgfpathlineto{\pgfqpoint{4.296154in}{2.282696in}}%
\pgfpathlineto{\pgfqpoint{4.296563in}{2.332205in}}%
\pgfpathlineto{\pgfqpoint{4.297583in}{2.257941in}}%
\pgfpathlineto{\pgfqpoint{4.298808in}{2.389967in}}%
\pgfpathlineto{\pgfqpoint{4.299012in}{2.381715in}}%
\pgfpathlineto{\pgfqpoint{4.300850in}{2.480734in}}%
\pgfpathlineto{\pgfqpoint{4.302075in}{2.398218in}}%
\pgfpathlineto{\pgfqpoint{4.301258in}{2.488986in}}%
\pgfpathlineto{\pgfqpoint{4.302279in}{2.414721in}}%
\pgfpathlineto{\pgfqpoint{4.302687in}{2.480734in}}%
\pgfpathlineto{\pgfqpoint{4.303095in}{2.389967in}}%
\pgfpathlineto{\pgfqpoint{4.303504in}{2.365212in}}%
\pgfpathlineto{\pgfqpoint{4.303708in}{2.381715in}}%
\pgfpathlineto{\pgfqpoint{4.303912in}{2.431225in}}%
\pgfpathlineto{\pgfqpoint{4.304728in}{2.381715in}}%
\pgfpathlineto{\pgfqpoint{4.305137in}{2.365212in}}%
\pgfpathlineto{\pgfqpoint{4.305341in}{2.414721in}}%
\pgfpathlineto{\pgfqpoint{4.305545in}{2.381715in}}%
\pgfpathlineto{\pgfqpoint{4.306362in}{2.373463in}}%
\pgfpathlineto{\pgfqpoint{4.306770in}{2.472483in}}%
\pgfpathlineto{\pgfqpoint{4.307382in}{2.348709in}}%
\pgfpathlineto{\pgfqpoint{4.307791in}{2.414721in}}%
\pgfpathlineto{\pgfqpoint{4.307995in}{2.406470in}}%
\pgfpathlineto{\pgfqpoint{4.309220in}{2.455979in}}%
\pgfpathlineto{\pgfqpoint{4.309628in}{2.472483in}}%
\pgfpathlineto{\pgfqpoint{4.309832in}{2.439476in}}%
\pgfpathlineto{\pgfqpoint{4.311057in}{2.521992in}}%
\pgfpathlineto{\pgfqpoint{4.311873in}{2.414721in}}%
\pgfpathlineto{\pgfqpoint{4.312282in}{2.431225in}}%
\pgfpathlineto{\pgfqpoint{4.312486in}{2.488986in}}%
\pgfpathlineto{\pgfqpoint{4.313302in}{2.406470in}}%
\pgfpathlineto{\pgfqpoint{4.313915in}{2.447728in}}%
\pgfpathlineto{\pgfqpoint{4.314119in}{2.381715in}}%
\pgfpathlineto{\pgfqpoint{4.314323in}{2.422973in}}%
\pgfpathlineto{\pgfqpoint{4.314527in}{2.365212in}}%
\pgfpathlineto{\pgfqpoint{4.315344in}{2.414721in}}%
\pgfpathlineto{\pgfqpoint{4.315548in}{2.431225in}}%
\pgfpathlineto{\pgfqpoint{4.316160in}{2.340457in}}%
\pgfpathlineto{\pgfqpoint{4.316773in}{2.356960in}}%
\pgfpathlineto{\pgfqpoint{4.317589in}{2.505489in}}%
\pgfpathlineto{\pgfqpoint{4.318202in}{2.455979in}}%
\pgfpathlineto{\pgfqpoint{4.318406in}{2.315702in}}%
\pgfpathlineto{\pgfqpoint{4.319223in}{2.398218in}}%
\pgfpathlineto{\pgfqpoint{4.319427in}{2.439476in}}%
\pgfpathlineto{\pgfqpoint{4.320039in}{2.356960in}}%
\pgfpathlineto{\pgfqpoint{4.320447in}{2.406470in}}%
\pgfpathlineto{\pgfqpoint{4.320856in}{2.373463in}}%
\pgfpathlineto{\pgfqpoint{4.321672in}{2.464231in}}%
\pgfpathlineto{\pgfqpoint{4.321876in}{2.282696in}}%
\pgfpathlineto{\pgfqpoint{4.322693in}{2.373463in}}%
\pgfpathlineto{\pgfqpoint{4.323714in}{2.315702in}}%
\pgfpathlineto{\pgfqpoint{4.323918in}{2.084657in}}%
\pgfpathlineto{\pgfqpoint{4.324734in}{2.389967in}}%
\pgfpathlineto{\pgfqpoint{4.324939in}{2.447728in}}%
\pgfpathlineto{\pgfqpoint{4.325551in}{2.348709in}}%
\pgfpathlineto{\pgfqpoint{4.325959in}{2.422973in}}%
\pgfpathlineto{\pgfqpoint{4.326163in}{2.373463in}}%
\pgfpathlineto{\pgfqpoint{4.326368in}{2.431225in}}%
\pgfpathlineto{\pgfqpoint{4.326776in}{2.414721in}}%
\pgfpathlineto{\pgfqpoint{4.327592in}{2.505489in}}%
\pgfpathlineto{\pgfqpoint{4.327184in}{2.389967in}}%
\pgfpathlineto{\pgfqpoint{4.327797in}{2.480734in}}%
\pgfpathlineto{\pgfqpoint{4.328409in}{2.431225in}}%
\pgfpathlineto{\pgfqpoint{4.328817in}{2.439476in}}%
\pgfpathlineto{\pgfqpoint{4.329021in}{2.505489in}}%
\pgfpathlineto{\pgfqpoint{4.329226in}{2.414721in}}%
\pgfpathlineto{\pgfqpoint{4.329838in}{2.439476in}}%
\pgfpathlineto{\pgfqpoint{4.330042in}{2.455979in}}%
\pgfpathlineto{\pgfqpoint{4.330450in}{2.183676in}}%
\pgfpathlineto{\pgfqpoint{4.331063in}{2.389967in}}%
\pgfpathlineto{\pgfqpoint{4.331471in}{2.431225in}}%
\pgfpathlineto{\pgfqpoint{4.331879in}{2.356960in}}%
\pgfpathlineto{\pgfqpoint{4.332084in}{2.414721in}}%
\pgfpathlineto{\pgfqpoint{4.332492in}{2.365212in}}%
\pgfpathlineto{\pgfqpoint{4.332696in}{2.422973in}}%
\pgfpathlineto{\pgfqpoint{4.332900in}{2.422973in}}%
\pgfpathlineto{\pgfqpoint{4.335758in}{2.579753in}}%
\pgfpathlineto{\pgfqpoint{4.335962in}{2.612760in}}%
\pgfpathlineto{\pgfqpoint{4.336371in}{2.521992in}}%
\pgfpathlineto{\pgfqpoint{4.336575in}{2.588005in}}%
\pgfpathlineto{\pgfqpoint{4.337800in}{2.538495in}}%
\pgfpathlineto{\pgfqpoint{4.338004in}{2.604508in}}%
\pgfpathlineto{\pgfqpoint{4.338208in}{2.389967in}}%
\pgfpathlineto{\pgfqpoint{4.338616in}{2.323954in}}%
\pgfpathlineto{\pgfqpoint{4.338616in}{2.323954in}}%
\pgfusepath{stroke}%
\end{pgfscope}%
\begin{pgfscope}%
\pgfsetrectcap%
\pgfsetmiterjoin%
\pgfsetlinewidth{0.803000pt}%
\definecolor{currentstroke}{rgb}{0.000000,0.000000,0.000000}%
\pgfsetstrokecolor{currentstroke}%
\pgfsetdash{}{0pt}%
\pgfpathmoveto{\pgfqpoint{0.634869in}{0.539544in}}%
\pgfpathlineto{\pgfqpoint{0.634869in}{2.944887in}}%
\pgfusepath{stroke}%
\end{pgfscope}%
\begin{pgfscope}%
\pgfsetrectcap%
\pgfsetmiterjoin%
\pgfsetlinewidth{0.803000pt}%
\definecolor{currentstroke}{rgb}{0.000000,0.000000,0.000000}%
\pgfsetstrokecolor{currentstroke}%
\pgfsetdash{}{0pt}%
\pgfpathmoveto{\pgfqpoint{4.514985in}{0.539544in}}%
\pgfpathlineto{\pgfqpoint{4.514985in}{2.944887in}}%
\pgfusepath{stroke}%
\end{pgfscope}%
\begin{pgfscope}%
\pgfsetrectcap%
\pgfsetmiterjoin%
\pgfsetlinewidth{0.803000pt}%
\definecolor{currentstroke}{rgb}{0.000000,0.000000,0.000000}%
\pgfsetstrokecolor{currentstroke}%
\pgfsetdash{}{0pt}%
\pgfpathmoveto{\pgfqpoint{0.634869in}{0.539544in}}%
\pgfpathlineto{\pgfqpoint{4.514985in}{0.539544in}}%
\pgfusepath{stroke}%
\end{pgfscope}%
\begin{pgfscope}%
\pgfsetrectcap%
\pgfsetmiterjoin%
\pgfsetlinewidth{0.803000pt}%
\definecolor{currentstroke}{rgb}{0.000000,0.000000,0.000000}%
\pgfsetstrokecolor{currentstroke}%
\pgfsetdash{}{0pt}%
\pgfpathmoveto{\pgfqpoint{0.634869in}{2.944887in}}%
\pgfpathlineto{\pgfqpoint{4.514985in}{2.944887in}}%
\pgfusepath{stroke}%
\end{pgfscope}%
\begin{pgfscope}%
\pgfsetbuttcap%
\pgfsetroundjoin%
\definecolor{currentfill}{rgb}{0.000000,0.000000,0.000000}%
\pgfsetfillcolor{currentfill}%
\pgfsetlinewidth{0.803000pt}%
\definecolor{currentstroke}{rgb}{0.000000,0.000000,0.000000}%
\pgfsetstrokecolor{currentstroke}%
\pgfsetdash{}{0pt}%
\pgfsys@defobject{currentmarker}{\pgfqpoint{0.000000in}{0.000000in}}{\pgfqpoint{0.048611in}{0.000000in}}{%
\pgfpathmoveto{\pgfqpoint{0.000000in}{0.000000in}}%
\pgfpathlineto{\pgfqpoint{0.048611in}{0.000000in}}%
\pgfusepath{stroke,fill}%
}%
\begin{pgfscope}%
\pgfsys@transformshift{4.514985in}{0.725412in}%
\pgfsys@useobject{currentmarker}{}%
\end{pgfscope}%
\end{pgfscope}%
\begin{pgfscope}%
\definecolor{textcolor}{rgb}{0.000000,0.000000,0.000000}%
\pgfsetstrokecolor{textcolor}%
\pgfsetfillcolor{textcolor}%
\pgftext[x=4.612207in, y=0.686856in, left, base]{\color{textcolor}\rmfamily\fontsize{8.000000}{9.600000}\selectfont \(\displaystyle {20.00}\)}%
\end{pgfscope}%
\begin{pgfscope}%
\pgfsetbuttcap%
\pgfsetroundjoin%
\definecolor{currentfill}{rgb}{0.000000,0.000000,0.000000}%
\pgfsetfillcolor{currentfill}%
\pgfsetlinewidth{0.803000pt}%
\definecolor{currentstroke}{rgb}{0.000000,0.000000,0.000000}%
\pgfsetstrokecolor{currentstroke}%
\pgfsetdash{}{0pt}%
\pgfsys@defobject{currentmarker}{\pgfqpoint{0.000000in}{0.000000in}}{\pgfqpoint{0.048611in}{0.000000in}}{%
\pgfpathmoveto{\pgfqpoint{0.000000in}{0.000000in}}%
\pgfpathlineto{\pgfqpoint{0.048611in}{0.000000in}}%
\pgfusepath{stroke,fill}%
}%
\begin{pgfscope}%
\pgfsys@transformshift{4.514985in}{0.998746in}%
\pgfsys@useobject{currentmarker}{}%
\end{pgfscope}%
\end{pgfscope}%
\begin{pgfscope}%
\definecolor{textcolor}{rgb}{0.000000,0.000000,0.000000}%
\pgfsetstrokecolor{textcolor}%
\pgfsetfillcolor{textcolor}%
\pgftext[x=4.612207in, y=0.960190in, left, base]{\color{textcolor}\rmfamily\fontsize{8.000000}{9.600000}\selectfont \(\displaystyle {20.25}\)}%
\end{pgfscope}%
\begin{pgfscope}%
\pgfsetbuttcap%
\pgfsetroundjoin%
\definecolor{currentfill}{rgb}{0.000000,0.000000,0.000000}%
\pgfsetfillcolor{currentfill}%
\pgfsetlinewidth{0.803000pt}%
\definecolor{currentstroke}{rgb}{0.000000,0.000000,0.000000}%
\pgfsetstrokecolor{currentstroke}%
\pgfsetdash{}{0pt}%
\pgfsys@defobject{currentmarker}{\pgfqpoint{0.000000in}{0.000000in}}{\pgfqpoint{0.048611in}{0.000000in}}{%
\pgfpathmoveto{\pgfqpoint{0.000000in}{0.000000in}}%
\pgfpathlineto{\pgfqpoint{0.048611in}{0.000000in}}%
\pgfusepath{stroke,fill}%
}%
\begin{pgfscope}%
\pgfsys@transformshift{4.514985in}{1.272080in}%
\pgfsys@useobject{currentmarker}{}%
\end{pgfscope}%
\end{pgfscope}%
\begin{pgfscope}%
\definecolor{textcolor}{rgb}{0.000000,0.000000,0.000000}%
\pgfsetstrokecolor{textcolor}%
\pgfsetfillcolor{textcolor}%
\pgftext[x=4.612207in, y=1.233525in, left, base]{\color{textcolor}\rmfamily\fontsize{8.000000}{9.600000}\selectfont \(\displaystyle {20.50}\)}%
\end{pgfscope}%
\begin{pgfscope}%
\pgfsetbuttcap%
\pgfsetroundjoin%
\definecolor{currentfill}{rgb}{0.000000,0.000000,0.000000}%
\pgfsetfillcolor{currentfill}%
\pgfsetlinewidth{0.803000pt}%
\definecolor{currentstroke}{rgb}{0.000000,0.000000,0.000000}%
\pgfsetstrokecolor{currentstroke}%
\pgfsetdash{}{0pt}%
\pgfsys@defobject{currentmarker}{\pgfqpoint{0.000000in}{0.000000in}}{\pgfqpoint{0.048611in}{0.000000in}}{%
\pgfpathmoveto{\pgfqpoint{0.000000in}{0.000000in}}%
\pgfpathlineto{\pgfqpoint{0.048611in}{0.000000in}}%
\pgfusepath{stroke,fill}%
}%
\begin{pgfscope}%
\pgfsys@transformshift{4.514985in}{1.545415in}%
\pgfsys@useobject{currentmarker}{}%
\end{pgfscope}%
\end{pgfscope}%
\begin{pgfscope}%
\definecolor{textcolor}{rgb}{0.000000,0.000000,0.000000}%
\pgfsetstrokecolor{textcolor}%
\pgfsetfillcolor{textcolor}%
\pgftext[x=4.612207in, y=1.506859in, left, base]{\color{textcolor}\rmfamily\fontsize{8.000000}{9.600000}\selectfont \(\displaystyle {20.75}\)}%
\end{pgfscope}%
\begin{pgfscope}%
\pgfsetbuttcap%
\pgfsetroundjoin%
\definecolor{currentfill}{rgb}{0.000000,0.000000,0.000000}%
\pgfsetfillcolor{currentfill}%
\pgfsetlinewidth{0.803000pt}%
\definecolor{currentstroke}{rgb}{0.000000,0.000000,0.000000}%
\pgfsetstrokecolor{currentstroke}%
\pgfsetdash{}{0pt}%
\pgfsys@defobject{currentmarker}{\pgfqpoint{0.000000in}{0.000000in}}{\pgfqpoint{0.048611in}{0.000000in}}{%
\pgfpathmoveto{\pgfqpoint{0.000000in}{0.000000in}}%
\pgfpathlineto{\pgfqpoint{0.048611in}{0.000000in}}%
\pgfusepath{stroke,fill}%
}%
\begin{pgfscope}%
\pgfsys@transformshift{4.514985in}{1.818749in}%
\pgfsys@useobject{currentmarker}{}%
\end{pgfscope}%
\end{pgfscope}%
\begin{pgfscope}%
\definecolor{textcolor}{rgb}{0.000000,0.000000,0.000000}%
\pgfsetstrokecolor{textcolor}%
\pgfsetfillcolor{textcolor}%
\pgftext[x=4.612207in, y=1.780194in, left, base]{\color{textcolor}\rmfamily\fontsize{8.000000}{9.600000}\selectfont \(\displaystyle {21.00}\)}%
\end{pgfscope}%
\begin{pgfscope}%
\pgfsetbuttcap%
\pgfsetroundjoin%
\definecolor{currentfill}{rgb}{0.000000,0.000000,0.000000}%
\pgfsetfillcolor{currentfill}%
\pgfsetlinewidth{0.803000pt}%
\definecolor{currentstroke}{rgb}{0.000000,0.000000,0.000000}%
\pgfsetstrokecolor{currentstroke}%
\pgfsetdash{}{0pt}%
\pgfsys@defobject{currentmarker}{\pgfqpoint{0.000000in}{0.000000in}}{\pgfqpoint{0.048611in}{0.000000in}}{%
\pgfpathmoveto{\pgfqpoint{0.000000in}{0.000000in}}%
\pgfpathlineto{\pgfqpoint{0.048611in}{0.000000in}}%
\pgfusepath{stroke,fill}%
}%
\begin{pgfscope}%
\pgfsys@transformshift{4.514985in}{2.092084in}%
\pgfsys@useobject{currentmarker}{}%
\end{pgfscope}%
\end{pgfscope}%
\begin{pgfscope}%
\definecolor{textcolor}{rgb}{0.000000,0.000000,0.000000}%
\pgfsetstrokecolor{textcolor}%
\pgfsetfillcolor{textcolor}%
\pgftext[x=4.612207in, y=2.053528in, left, base]{\color{textcolor}\rmfamily\fontsize{8.000000}{9.600000}\selectfont \(\displaystyle {21.25}\)}%
\end{pgfscope}%
\begin{pgfscope}%
\pgfsetbuttcap%
\pgfsetroundjoin%
\definecolor{currentfill}{rgb}{0.000000,0.000000,0.000000}%
\pgfsetfillcolor{currentfill}%
\pgfsetlinewidth{0.803000pt}%
\definecolor{currentstroke}{rgb}{0.000000,0.000000,0.000000}%
\pgfsetstrokecolor{currentstroke}%
\pgfsetdash{}{0pt}%
\pgfsys@defobject{currentmarker}{\pgfqpoint{0.000000in}{0.000000in}}{\pgfqpoint{0.048611in}{0.000000in}}{%
\pgfpathmoveto{\pgfqpoint{0.000000in}{0.000000in}}%
\pgfpathlineto{\pgfqpoint{0.048611in}{0.000000in}}%
\pgfusepath{stroke,fill}%
}%
\begin{pgfscope}%
\pgfsys@transformshift{4.514985in}{2.365418in}%
\pgfsys@useobject{currentmarker}{}%
\end{pgfscope}%
\end{pgfscope}%
\begin{pgfscope}%
\definecolor{textcolor}{rgb}{0.000000,0.000000,0.000000}%
\pgfsetstrokecolor{textcolor}%
\pgfsetfillcolor{textcolor}%
\pgftext[x=4.612207in, y=2.326862in, left, base]{\color{textcolor}\rmfamily\fontsize{8.000000}{9.600000}\selectfont \(\displaystyle {21.50}\)}%
\end{pgfscope}%
\begin{pgfscope}%
\pgfsetbuttcap%
\pgfsetroundjoin%
\definecolor{currentfill}{rgb}{0.000000,0.000000,0.000000}%
\pgfsetfillcolor{currentfill}%
\pgfsetlinewidth{0.803000pt}%
\definecolor{currentstroke}{rgb}{0.000000,0.000000,0.000000}%
\pgfsetstrokecolor{currentstroke}%
\pgfsetdash{}{0pt}%
\pgfsys@defobject{currentmarker}{\pgfqpoint{0.000000in}{0.000000in}}{\pgfqpoint{0.048611in}{0.000000in}}{%
\pgfpathmoveto{\pgfqpoint{0.000000in}{0.000000in}}%
\pgfpathlineto{\pgfqpoint{0.048611in}{0.000000in}}%
\pgfusepath{stroke,fill}%
}%
\begin{pgfscope}%
\pgfsys@transformshift{4.514985in}{2.638752in}%
\pgfsys@useobject{currentmarker}{}%
\end{pgfscope}%
\end{pgfscope}%
\begin{pgfscope}%
\definecolor{textcolor}{rgb}{0.000000,0.000000,0.000000}%
\pgfsetstrokecolor{textcolor}%
\pgfsetfillcolor{textcolor}%
\pgftext[x=4.612207in, y=2.600197in, left, base]{\color{textcolor}\rmfamily\fontsize{8.000000}{9.600000}\selectfont \(\displaystyle {21.75}\)}%
\end{pgfscope}%
\begin{pgfscope}%
\pgfsetbuttcap%
\pgfsetroundjoin%
\definecolor{currentfill}{rgb}{0.000000,0.000000,0.000000}%
\pgfsetfillcolor{currentfill}%
\pgfsetlinewidth{0.803000pt}%
\definecolor{currentstroke}{rgb}{0.000000,0.000000,0.000000}%
\pgfsetstrokecolor{currentstroke}%
\pgfsetdash{}{0pt}%
\pgfsys@defobject{currentmarker}{\pgfqpoint{0.000000in}{0.000000in}}{\pgfqpoint{0.048611in}{0.000000in}}{%
\pgfpathmoveto{\pgfqpoint{0.000000in}{0.000000in}}%
\pgfpathlineto{\pgfqpoint{0.048611in}{0.000000in}}%
\pgfusepath{stroke,fill}%
}%
\begin{pgfscope}%
\pgfsys@transformshift{4.514985in}{2.912087in}%
\pgfsys@useobject{currentmarker}{}%
\end{pgfscope}%
\end{pgfscope}%
\begin{pgfscope}%
\definecolor{textcolor}{rgb}{0.000000,0.000000,0.000000}%
\pgfsetstrokecolor{textcolor}%
\pgfsetfillcolor{textcolor}%
\pgftext[x=4.612207in, y=2.873531in, left, base]{\color{textcolor}\rmfamily\fontsize{8.000000}{9.600000}\selectfont \(\displaystyle {22.00}\)}%
\end{pgfscope}%
\begin{pgfscope}%
\definecolor{textcolor}{rgb}{0.000000,0.000000,0.000000}%
\pgfsetstrokecolor{textcolor}%
\pgfsetfillcolor{textcolor}%
\pgftext[x=4.936671in,y=1.742216in,,top,rotate=90.000000]{\color{textcolor}\rmfamily\fontsize{10.000000}{12.000000}\selectfont Temperature in °C}%
\end{pgfscope}%
\begin{pgfscope}%
\pgfpathrectangle{\pgfqpoint{0.634869in}{0.539544in}}{\pgfqpoint{3.880116in}{2.405343in}}%
\pgfusepath{clip}%
\pgfsetrectcap%
\pgfsetroundjoin%
\pgfsetlinewidth{0.501875pt}%
\definecolor{currentstroke}{rgb}{0.698039,0.133333,0.133333}%
\pgfsetstrokecolor{currentstroke}%
\pgfsetstrokeopacity{0.700000}%
\pgfsetdash{}{0pt}%
\pgfpathmoveto{\pgfqpoint{0.811770in}{1.949950in}}%
\pgfpathlineto{\pgfqpoint{0.817935in}{2.092084in}}%
\pgfpathlineto{\pgfqpoint{0.820507in}{2.015550in}}%
\pgfpathlineto{\pgfqpoint{0.825774in}{2.157684in}}%
\pgfpathlineto{\pgfqpoint{0.828224in}{2.092084in}}%
\pgfpathlineto{\pgfqpoint{0.834185in}{2.223284in}}%
\pgfpathlineto{\pgfqpoint{0.836634in}{2.157684in}}%
\pgfpathlineto{\pgfqpoint{0.839084in}{2.223284in}}%
\pgfpathlineto{\pgfqpoint{0.844759in}{2.288884in}}%
\pgfpathlineto{\pgfqpoint{0.847209in}{2.223284in}}%
\pgfpathlineto{\pgfqpoint{0.849659in}{2.288884in}}%
\pgfpathlineto{\pgfqpoint{0.858192in}{2.365418in}}%
\pgfpathlineto{\pgfqpoint{0.860641in}{2.288884in}}%
\pgfpathlineto{\pgfqpoint{0.863091in}{2.365418in}}%
\pgfpathlineto{\pgfqpoint{0.869542in}{2.431018in}}%
\pgfpathlineto{\pgfqpoint{0.871992in}{2.365418in}}%
\pgfpathlineto{\pgfqpoint{0.874441in}{2.431018in}}%
\pgfpathlineto{\pgfqpoint{0.877299in}{2.365418in}}%
\pgfpathlineto{\pgfqpoint{0.879749in}{2.431018in}}%
\pgfpathlineto{\pgfqpoint{0.886282in}{2.496619in}}%
\pgfpathlineto{\pgfqpoint{0.888731in}{2.431018in}}%
\pgfpathlineto{\pgfqpoint{0.891181in}{2.496619in}}%
\pgfpathlineto{\pgfqpoint{0.903144in}{2.562219in}}%
\pgfpathlineto{\pgfqpoint{0.905594in}{2.496619in}}%
\pgfpathlineto{\pgfqpoint{0.908084in}{2.562219in}}%
\pgfpathlineto{\pgfqpoint{0.910738in}{2.496619in}}%
\pgfpathlineto{\pgfqpoint{0.913188in}{2.562219in}}%
\pgfpathlineto{\pgfqpoint{0.923966in}{2.638752in}}%
\pgfpathlineto{\pgfqpoint{0.926416in}{2.562219in}}%
\pgfpathlineto{\pgfqpoint{0.928866in}{2.638752in}}%
\pgfpathlineto{\pgfqpoint{0.931560in}{2.562219in}}%
\pgfpathlineto{\pgfqpoint{0.934010in}{2.638752in}}%
\pgfpathlineto{\pgfqpoint{0.946830in}{2.704353in}}%
\pgfpathlineto{\pgfqpoint{0.949280in}{2.638752in}}%
\pgfpathlineto{\pgfqpoint{0.951730in}{2.704353in}}%
\pgfpathlineto{\pgfqpoint{0.954179in}{2.638752in}}%
\pgfpathlineto{\pgfqpoint{0.956629in}{2.704353in}}%
\pgfpathlineto{\pgfqpoint{0.971205in}{2.769953in}}%
\pgfpathlineto{\pgfqpoint{0.973655in}{2.704353in}}%
\pgfpathlineto{\pgfqpoint{0.976145in}{2.769953in}}%
\pgfpathlineto{\pgfqpoint{0.978595in}{2.704353in}}%
\pgfpathlineto{\pgfqpoint{0.981045in}{2.769953in}}%
\pgfpathlineto{\pgfqpoint{0.983780in}{2.704353in}}%
\pgfpathlineto{\pgfqpoint{0.986230in}{2.769953in}}%
\pgfpathlineto{\pgfqpoint{0.998805in}{2.835553in}}%
\pgfpathlineto{\pgfqpoint{1.001255in}{2.769953in}}%
\pgfpathlineto{\pgfqpoint{1.003704in}{2.835553in}}%
\pgfpathlineto{\pgfqpoint{1.006154in}{2.769953in}}%
\pgfpathlineto{\pgfqpoint{1.008604in}{2.835553in}}%
\pgfpathlineto{\pgfqpoint{1.011176in}{2.769953in}}%
\pgfpathlineto{\pgfqpoint{1.013626in}{2.835553in}}%
\pgfpathlineto{\pgfqpoint{1.019342in}{2.704353in}}%
\pgfpathlineto{\pgfqpoint{1.021791in}{2.638752in}}%
\pgfpathlineto{\pgfqpoint{1.026691in}{2.365418in}}%
\pgfpathlineto{\pgfqpoint{1.034040in}{2.157684in}}%
\pgfpathlineto{\pgfqpoint{1.036490in}{2.015550in}}%
\pgfpathlineto{\pgfqpoint{1.052617in}{1.611015in}}%
\pgfpathlineto{\pgfqpoint{1.059150in}{1.468881in}}%
\pgfpathlineto{\pgfqpoint{1.068540in}{1.337681in}}%
\pgfpathlineto{\pgfqpoint{1.071031in}{1.403281in}}%
\pgfpathlineto{\pgfqpoint{1.076093in}{1.272080in}}%
\pgfpathlineto{\pgfqpoint{1.081483in}{1.195547in}}%
\pgfpathlineto{\pgfqpoint{1.089322in}{1.129947in}}%
\pgfpathlineto{\pgfqpoint{1.091853in}{1.195547in}}%
\pgfpathlineto{\pgfqpoint{1.097161in}{1.064346in}}%
\pgfpathlineto{\pgfqpoint{1.099611in}{1.129947in}}%
\pgfpathlineto{\pgfqpoint{1.102060in}{1.064346in}}%
\pgfpathlineto{\pgfqpoint{1.104592in}{1.129947in}}%
\pgfpathlineto{\pgfqpoint{1.107041in}{1.064346in}}%
\pgfpathlineto{\pgfqpoint{1.113206in}{0.998746in}}%
\pgfpathlineto{\pgfqpoint{1.115656in}{1.064346in}}%
\pgfpathlineto{\pgfqpoint{1.118147in}{0.998746in}}%
\pgfpathlineto{\pgfqpoint{1.120719in}{1.064346in}}%
\pgfpathlineto{\pgfqpoint{1.123169in}{0.998746in}}%
\pgfpathlineto{\pgfqpoint{1.127864in}{0.922212in}}%
\pgfpathlineto{\pgfqpoint{1.130354in}{0.998746in}}%
\pgfpathlineto{\pgfqpoint{1.132804in}{0.922212in}}%
\pgfpathlineto{\pgfqpoint{1.140112in}{0.856612in}}%
\pgfpathlineto{\pgfqpoint{1.142562in}{0.922212in}}%
\pgfpathlineto{\pgfqpoint{1.145012in}{0.856612in}}%
\pgfpathlineto{\pgfqpoint{1.148197in}{0.922212in}}%
\pgfpathlineto{\pgfqpoint{1.153259in}{0.791012in}}%
\pgfpathlineto{\pgfqpoint{1.155709in}{0.856612in}}%
\pgfpathlineto{\pgfqpoint{1.158200in}{0.791012in}}%
\pgfpathlineto{\pgfqpoint{1.160649in}{0.856612in}}%
\pgfpathlineto{\pgfqpoint{1.163099in}{0.791012in}}%
\pgfpathlineto{\pgfqpoint{1.173674in}{0.856612in}}%
\pgfpathlineto{\pgfqpoint{1.176123in}{0.791012in}}%
\pgfpathlineto{\pgfqpoint{1.178614in}{0.856612in}}%
\pgfpathlineto{\pgfqpoint{1.181063in}{0.791012in}}%
\pgfpathlineto{\pgfqpoint{1.183799in}{0.856612in}}%
\pgfpathlineto{\pgfqpoint{1.186249in}{0.791012in}}%
\pgfpathlineto{\pgfqpoint{1.204581in}{0.856612in}}%
\pgfpathlineto{\pgfqpoint{1.209194in}{0.922212in}}%
\pgfpathlineto{\pgfqpoint{1.226342in}{1.272080in}}%
\pgfpathlineto{\pgfqpoint{1.252473in}{1.676615in}}%
\pgfpathlineto{\pgfqpoint{1.276031in}{1.949950in}}%
\pgfpathlineto{\pgfqpoint{1.278603in}{1.884349in}}%
\pgfpathlineto{\pgfqpoint{1.283584in}{2.015550in}}%
\pgfpathlineto{\pgfqpoint{1.286850in}{1.949950in}}%
\pgfpathlineto{\pgfqpoint{1.292117in}{2.092084in}}%
\pgfpathlineto{\pgfqpoint{1.294812in}{2.015550in}}%
\pgfpathlineto{\pgfqpoint{1.297261in}{2.092084in}}%
\pgfpathlineto{\pgfqpoint{1.300650in}{2.157684in}}%
\pgfpathlineto{\pgfqpoint{1.303100in}{2.092084in}}%
\pgfpathlineto{\pgfqpoint{1.305550in}{2.157684in}}%
\pgfpathlineto{\pgfqpoint{1.310531in}{2.223284in}}%
\pgfpathlineto{\pgfqpoint{1.313021in}{2.157684in}}%
\pgfpathlineto{\pgfqpoint{1.315471in}{2.223284in}}%
\pgfpathlineto{\pgfqpoint{1.320901in}{2.288884in}}%
\pgfpathlineto{\pgfqpoint{1.323351in}{2.223284in}}%
\pgfpathlineto{\pgfqpoint{1.325801in}{2.288884in}}%
\pgfpathlineto{\pgfqpoint{1.334456in}{2.365418in}}%
\pgfpathlineto{\pgfqpoint{1.336947in}{2.288884in}}%
\pgfpathlineto{\pgfqpoint{1.339396in}{2.365418in}}%
\pgfpathlineto{\pgfqpoint{1.346419in}{2.431018in}}%
\pgfpathlineto{\pgfqpoint{1.348869in}{2.365418in}}%
\pgfpathlineto{\pgfqpoint{1.351318in}{2.431018in}}%
\pgfpathlineto{\pgfqpoint{1.354176in}{2.365418in}}%
\pgfpathlineto{\pgfqpoint{1.356626in}{2.431018in}}%
\pgfpathlineto{\pgfqpoint{1.361403in}{2.496619in}}%
\pgfpathlineto{\pgfqpoint{1.363853in}{2.431018in}}%
\pgfpathlineto{\pgfqpoint{1.366302in}{2.496619in}}%
\pgfpathlineto{\pgfqpoint{1.370957in}{2.431018in}}%
\pgfpathlineto{\pgfqpoint{1.373407in}{2.496619in}}%
\pgfpathlineto{\pgfqpoint{1.380511in}{2.562219in}}%
\pgfpathlineto{\pgfqpoint{1.382960in}{2.496619in}}%
\pgfpathlineto{\pgfqpoint{1.385410in}{2.562219in}}%
\pgfpathlineto{\pgfqpoint{1.388186in}{2.496619in}}%
\pgfpathlineto{\pgfqpoint{1.390636in}{2.562219in}}%
\pgfpathlineto{\pgfqpoint{1.399455in}{2.638752in}}%
\pgfpathlineto{\pgfqpoint{1.401905in}{2.562219in}}%
\pgfpathlineto{\pgfqpoint{1.404355in}{2.638752in}}%
\pgfpathlineto{\pgfqpoint{1.406845in}{2.562219in}}%
\pgfpathlineto{\pgfqpoint{1.409295in}{2.638752in}}%
\pgfpathlineto{\pgfqpoint{1.421258in}{2.704353in}}%
\pgfpathlineto{\pgfqpoint{1.423707in}{2.638752in}}%
\pgfpathlineto{\pgfqpoint{1.426198in}{2.704353in}}%
\pgfpathlineto{\pgfqpoint{1.428648in}{2.638752in}}%
\pgfpathlineto{\pgfqpoint{1.431097in}{2.704353in}}%
\pgfpathlineto{\pgfqpoint{1.434078in}{2.638752in}}%
\pgfpathlineto{\pgfqpoint{1.436527in}{2.704353in}}%
\pgfpathlineto{\pgfqpoint{1.444203in}{2.769953in}}%
\pgfpathlineto{\pgfqpoint{1.446653in}{2.704353in}}%
\pgfpathlineto{\pgfqpoint{1.449143in}{2.769953in}}%
\pgfpathlineto{\pgfqpoint{1.451593in}{2.704353in}}%
\pgfpathlineto{\pgfqpoint{1.454043in}{2.769953in}}%
\pgfpathlineto{\pgfqpoint{1.456493in}{2.704353in}}%
\pgfpathlineto{\pgfqpoint{1.458942in}{2.769953in}}%
\pgfpathlineto{\pgfqpoint{1.475886in}{2.835553in}}%
\pgfpathlineto{\pgfqpoint{1.478336in}{2.769953in}}%
\pgfpathlineto{\pgfqpoint{1.480908in}{2.835553in}}%
\pgfpathlineto{\pgfqpoint{1.483358in}{2.769953in}}%
\pgfpathlineto{\pgfqpoint{1.485889in}{2.835553in}}%
\pgfpathlineto{\pgfqpoint{1.488339in}{2.769953in}}%
\pgfpathlineto{\pgfqpoint{1.490788in}{2.835553in}}%
\pgfpathlineto{\pgfqpoint{1.496504in}{2.704353in}}%
\pgfpathlineto{\pgfqpoint{1.498954in}{2.638752in}}%
\pgfpathlineto{\pgfqpoint{1.501404in}{2.496619in}}%
\pgfpathlineto{\pgfqpoint{1.503854in}{2.431018in}}%
\pgfpathlineto{\pgfqpoint{1.506303in}{2.288884in}}%
\pgfpathlineto{\pgfqpoint{1.511203in}{2.157684in}}%
\pgfpathlineto{\pgfqpoint{1.513652in}{2.015550in}}%
\pgfpathlineto{\pgfqpoint{1.529821in}{1.611015in}}%
\pgfpathlineto{\pgfqpoint{1.538844in}{1.468881in}}%
\pgfpathlineto{\pgfqpoint{1.544355in}{1.403281in}}%
\pgfpathlineto{\pgfqpoint{1.546846in}{1.468881in}}%
\pgfpathlineto{\pgfqpoint{1.551745in}{1.337681in}}%
\pgfpathlineto{\pgfqpoint{1.558605in}{1.272080in}}%
\pgfpathlineto{\pgfqpoint{1.561054in}{1.337681in}}%
\pgfpathlineto{\pgfqpoint{1.566321in}{1.195547in}}%
\pgfpathlineto{\pgfqpoint{1.569098in}{1.272080in}}%
\pgfpathlineto{\pgfqpoint{1.571547in}{1.195547in}}%
\pgfpathlineto{\pgfqpoint{1.577672in}{1.129947in}}%
\pgfpathlineto{\pgfqpoint{1.580162in}{1.195547in}}%
\pgfpathlineto{\pgfqpoint{1.582612in}{1.129947in}}%
\pgfpathlineto{\pgfqpoint{1.586246in}{1.064346in}}%
\pgfpathlineto{\pgfqpoint{1.591962in}{0.998746in}}%
\pgfpathlineto{\pgfqpoint{1.594411in}{1.064346in}}%
\pgfpathlineto{\pgfqpoint{1.596861in}{0.998746in}}%
\pgfpathlineto{\pgfqpoint{1.599351in}{1.064346in}}%
\pgfpathlineto{\pgfqpoint{1.601801in}{0.998746in}}%
\pgfpathlineto{\pgfqpoint{1.605802in}{0.922212in}}%
\pgfpathlineto{\pgfqpoint{1.608252in}{0.998746in}}%
\pgfpathlineto{\pgfqpoint{1.610702in}{0.922212in}}%
\pgfpathlineto{\pgfqpoint{1.620419in}{0.856612in}}%
\pgfpathlineto{\pgfqpoint{1.622869in}{0.922212in}}%
\pgfpathlineto{\pgfqpoint{1.625318in}{0.856612in}}%
\pgfpathlineto{\pgfqpoint{1.627809in}{0.922212in}}%
\pgfpathlineto{\pgfqpoint{1.630259in}{0.856612in}}%
\pgfpathlineto{\pgfqpoint{1.635607in}{0.791012in}}%
\pgfpathlineto{\pgfqpoint{1.638057in}{0.856612in}}%
\pgfpathlineto{\pgfqpoint{1.640507in}{0.791012in}}%
\pgfpathlineto{\pgfqpoint{1.643120in}{0.856612in}}%
\pgfpathlineto{\pgfqpoint{1.645569in}{0.791012in}}%
\pgfpathlineto{\pgfqpoint{1.648060in}{0.856612in}}%
\pgfpathlineto{\pgfqpoint{1.650510in}{0.791012in}}%
\pgfpathlineto{\pgfqpoint{1.653408in}{0.856612in}}%
\pgfpathlineto{\pgfqpoint{1.655858in}{0.791012in}}%
\pgfpathlineto{\pgfqpoint{1.674353in}{0.725412in}}%
\pgfpathlineto{\pgfqpoint{1.676803in}{0.791012in}}%
\pgfpathlineto{\pgfqpoint{1.679253in}{0.725412in}}%
\pgfpathlineto{\pgfqpoint{1.681703in}{0.791012in}}%
\pgfpathlineto{\pgfqpoint{1.684152in}{0.725412in}}%
\pgfpathlineto{\pgfqpoint{1.692277in}{0.922212in}}%
\pgfpathlineto{\pgfqpoint{1.713916in}{1.337681in}}%
\pgfpathlineto{\pgfqpoint{1.732861in}{1.611015in}}%
\pgfpathlineto{\pgfqpoint{1.748498in}{1.818749in}}%
\pgfpathlineto{\pgfqpoint{1.760012in}{1.949950in}}%
\pgfpathlineto{\pgfqpoint{1.762461in}{1.884349in}}%
\pgfpathlineto{\pgfqpoint{1.768177in}{2.015550in}}%
\pgfpathlineto{\pgfqpoint{1.771117in}{1.949950in}}%
\pgfpathlineto{\pgfqpoint{1.776261in}{2.092084in}}%
\pgfpathlineto{\pgfqpoint{1.778997in}{2.015550in}}%
\pgfpathlineto{\pgfqpoint{1.781447in}{2.092084in}}%
\pgfpathlineto{\pgfqpoint{1.785734in}{2.157684in}}%
\pgfpathlineto{\pgfqpoint{1.788224in}{2.092084in}}%
\pgfpathlineto{\pgfqpoint{1.793858in}{2.223284in}}%
\pgfpathlineto{\pgfqpoint{1.796349in}{2.157684in}}%
\pgfpathlineto{\pgfqpoint{1.798799in}{2.223284in}}%
\pgfpathlineto{\pgfqpoint{1.804229in}{2.288884in}}%
\pgfpathlineto{\pgfqpoint{1.806679in}{2.223284in}}%
\pgfpathlineto{\pgfqpoint{1.809128in}{2.288884in}}%
\pgfpathlineto{\pgfqpoint{1.818192in}{2.365418in}}%
\pgfpathlineto{\pgfqpoint{1.820642in}{2.288884in}}%
\pgfpathlineto{\pgfqpoint{1.823092in}{2.365418in}}%
\pgfpathlineto{\pgfqpoint{1.829706in}{2.431018in}}%
\pgfpathlineto{\pgfqpoint{1.832156in}{2.365418in}}%
\pgfpathlineto{\pgfqpoint{1.834646in}{2.431018in}}%
\pgfpathlineto{\pgfqpoint{1.837953in}{2.365418in}}%
\pgfpathlineto{\pgfqpoint{1.840403in}{2.431018in}}%
\pgfpathlineto{\pgfqpoint{1.847303in}{2.496619in}}%
\pgfpathlineto{\pgfqpoint{1.849753in}{2.431018in}}%
\pgfpathlineto{\pgfqpoint{1.852202in}{2.496619in}}%
\pgfpathlineto{\pgfqpoint{1.865594in}{2.562219in}}%
\pgfpathlineto{\pgfqpoint{1.868044in}{2.496619in}}%
\pgfpathlineto{\pgfqpoint{1.870494in}{2.562219in}}%
\pgfpathlineto{\pgfqpoint{1.872943in}{2.496619in}}%
\pgfpathlineto{\pgfqpoint{1.875393in}{2.562219in}}%
\pgfpathlineto{\pgfqpoint{1.884988in}{2.638752in}}%
\pgfpathlineto{\pgfqpoint{1.887437in}{2.562219in}}%
\pgfpathlineto{\pgfqpoint{1.889887in}{2.638752in}}%
\pgfpathlineto{\pgfqpoint{1.892418in}{2.562219in}}%
\pgfpathlineto{\pgfqpoint{1.894868in}{2.638752in}}%
\pgfpathlineto{\pgfqpoint{1.908137in}{2.704353in}}%
\pgfpathlineto{\pgfqpoint{1.910587in}{2.638752in}}%
\pgfpathlineto{\pgfqpoint{1.913037in}{2.704353in}}%
\pgfpathlineto{\pgfqpoint{1.915527in}{2.638752in}}%
\pgfpathlineto{\pgfqpoint{1.917977in}{2.704353in}}%
\pgfpathlineto{\pgfqpoint{1.932635in}{2.769953in}}%
\pgfpathlineto{\pgfqpoint{1.935084in}{2.704353in}}%
\pgfpathlineto{\pgfqpoint{1.937983in}{2.769953in}}%
\pgfpathlineto{\pgfqpoint{1.940433in}{2.704353in}}%
\pgfpathlineto{\pgfqpoint{1.942923in}{2.769953in}}%
\pgfpathlineto{\pgfqpoint{1.945414in}{2.704353in}}%
\pgfpathlineto{\pgfqpoint{1.947864in}{2.769953in}}%
\pgfpathlineto{\pgfqpoint{1.962807in}{2.835553in}}%
\pgfpathlineto{\pgfqpoint{1.965257in}{2.769953in}}%
\pgfpathlineto{\pgfqpoint{1.968115in}{2.835553in}}%
\pgfpathlineto{\pgfqpoint{1.970564in}{2.769953in}}%
\pgfpathlineto{\pgfqpoint{1.973014in}{2.835553in}}%
\pgfpathlineto{\pgfqpoint{1.975504in}{2.769953in}}%
\pgfpathlineto{\pgfqpoint{1.977954in}{2.835553in}}%
\pgfpathlineto{\pgfqpoint{1.983221in}{2.704353in}}%
\pgfpathlineto{\pgfqpoint{1.985671in}{2.562219in}}%
\pgfpathlineto{\pgfqpoint{1.988120in}{2.496619in}}%
\pgfpathlineto{\pgfqpoint{1.993020in}{2.288884in}}%
\pgfpathlineto{\pgfqpoint{1.995470in}{2.223284in}}%
\pgfpathlineto{\pgfqpoint{2.000369in}{2.015550in}}%
\pgfpathlineto{\pgfqpoint{2.015884in}{1.611015in}}%
\pgfpathlineto{\pgfqpoint{2.024499in}{1.468881in}}%
\pgfpathlineto{\pgfqpoint{2.026989in}{1.545415in}}%
\pgfpathlineto{\pgfqpoint{2.031889in}{1.403281in}}%
\pgfpathlineto{\pgfqpoint{2.035604in}{1.337681in}}%
\pgfpathlineto{\pgfqpoint{2.038095in}{1.403281in}}%
\pgfpathlineto{\pgfqpoint{2.042994in}{1.272080in}}%
\pgfpathlineto{\pgfqpoint{2.047934in}{1.195547in}}%
\pgfpathlineto{\pgfqpoint{2.050425in}{1.272080in}}%
\pgfpathlineto{\pgfqpoint{2.055692in}{1.129947in}}%
\pgfpathlineto{\pgfqpoint{2.067491in}{0.998746in}}%
\pgfpathlineto{\pgfqpoint{2.069941in}{1.064346in}}%
\pgfpathlineto{\pgfqpoint{2.072391in}{0.998746in}}%
\pgfpathlineto{\pgfqpoint{2.081128in}{0.922212in}}%
\pgfpathlineto{\pgfqpoint{2.083578in}{0.998746in}}%
\pgfpathlineto{\pgfqpoint{2.086027in}{0.922212in}}%
\pgfpathlineto{\pgfqpoint{2.088599in}{0.998746in}}%
\pgfpathlineto{\pgfqpoint{2.091049in}{0.922212in}}%
\pgfpathlineto{\pgfqpoint{2.098807in}{0.856612in}}%
\pgfpathlineto{\pgfqpoint{2.101256in}{0.922212in}}%
\pgfpathlineto{\pgfqpoint{2.103706in}{0.856612in}}%
\pgfpathlineto{\pgfqpoint{2.106156in}{0.922212in}}%
\pgfpathlineto{\pgfqpoint{2.108605in}{0.856612in}}%
\pgfpathlineto{\pgfqpoint{2.111586in}{0.922212in}}%
\pgfpathlineto{\pgfqpoint{2.114036in}{0.856612in}}%
\pgfpathlineto{\pgfqpoint{2.119017in}{0.791012in}}%
\pgfpathlineto{\pgfqpoint{2.121466in}{0.856612in}}%
\pgfpathlineto{\pgfqpoint{2.123916in}{0.791012in}}%
\pgfpathlineto{\pgfqpoint{2.126447in}{0.856612in}}%
\pgfpathlineto{\pgfqpoint{2.128897in}{0.791012in}}%
\pgfpathlineto{\pgfqpoint{2.143065in}{0.725412in}}%
\pgfpathlineto{\pgfqpoint{2.145514in}{0.791012in}}%
\pgfpathlineto{\pgfqpoint{2.148005in}{0.725412in}}%
\pgfpathlineto{\pgfqpoint{2.150495in}{0.791012in}}%
\pgfpathlineto{\pgfqpoint{2.152945in}{0.725412in}}%
\pgfpathlineto{\pgfqpoint{2.158253in}{0.648878in}}%
\pgfpathlineto{\pgfqpoint{2.160703in}{0.725412in}}%
\pgfpathlineto{\pgfqpoint{2.163152in}{0.648878in}}%
\pgfpathlineto{\pgfqpoint{2.165602in}{0.725412in}}%
\pgfpathlineto{\pgfqpoint{2.168052in}{0.648878in}}%
\pgfpathlineto{\pgfqpoint{2.177973in}{0.725412in}}%
\pgfpathlineto{\pgfqpoint{2.201858in}{1.195547in}}%
\pgfpathlineto{\pgfqpoint{2.214310in}{1.403281in}}%
\pgfpathlineto{\pgfqpoint{2.249954in}{1.884349in}}%
\pgfpathlineto{\pgfqpoint{2.263917in}{2.015550in}}%
\pgfpathlineto{\pgfqpoint{2.266448in}{1.949950in}}%
\pgfpathlineto{\pgfqpoint{2.271756in}{2.092084in}}%
\pgfpathlineto{\pgfqpoint{2.274287in}{2.015550in}}%
\pgfpathlineto{\pgfqpoint{2.276737in}{2.092084in}}%
\pgfpathlineto{\pgfqpoint{2.280657in}{2.157684in}}%
\pgfpathlineto{\pgfqpoint{2.283229in}{2.092084in}}%
\pgfpathlineto{\pgfqpoint{2.288577in}{2.223284in}}%
\pgfpathlineto{\pgfqpoint{2.291027in}{2.157684in}}%
\pgfpathlineto{\pgfqpoint{2.293477in}{2.223284in}}%
\pgfpathlineto{\pgfqpoint{2.298948in}{2.288884in}}%
\pgfpathlineto{\pgfqpoint{2.301398in}{2.223284in}}%
\pgfpathlineto{\pgfqpoint{2.303847in}{2.288884in}}%
\pgfpathlineto{\pgfqpoint{2.310870in}{2.365418in}}%
\pgfpathlineto{\pgfqpoint{2.313319in}{2.288884in}}%
\pgfpathlineto{\pgfqpoint{2.315769in}{2.365418in}}%
\pgfpathlineto{\pgfqpoint{2.323567in}{2.431018in}}%
\pgfpathlineto{\pgfqpoint{2.326017in}{2.365418in}}%
\pgfpathlineto{\pgfqpoint{2.328467in}{2.431018in}}%
\pgfpathlineto{\pgfqpoint{2.338674in}{2.496619in}}%
\pgfpathlineto{\pgfqpoint{2.341124in}{2.431018in}}%
\pgfpathlineto{\pgfqpoint{2.343573in}{2.496619in}}%
\pgfpathlineto{\pgfqpoint{2.346023in}{2.431018in}}%
\pgfpathlineto{\pgfqpoint{2.348473in}{2.496619in}}%
\pgfpathlineto{\pgfqpoint{2.357700in}{2.562219in}}%
\pgfpathlineto{\pgfqpoint{2.360150in}{2.496619in}}%
\pgfpathlineto{\pgfqpoint{2.362640in}{2.562219in}}%
\pgfpathlineto{\pgfqpoint{2.365376in}{2.496619in}}%
\pgfpathlineto{\pgfqpoint{2.367826in}{2.562219in}}%
\pgfpathlineto{\pgfqpoint{2.378359in}{2.638752in}}%
\pgfpathlineto{\pgfqpoint{2.380809in}{2.562219in}}%
\pgfpathlineto{\pgfqpoint{2.383259in}{2.638752in}}%
\pgfpathlineto{\pgfqpoint{2.385708in}{2.562219in}}%
\pgfpathlineto{\pgfqpoint{2.388158in}{2.638752in}}%
\pgfpathlineto{\pgfqpoint{2.391465in}{2.562219in}}%
\pgfpathlineto{\pgfqpoint{2.393915in}{2.638752in}}%
\pgfpathlineto{\pgfqpoint{2.404326in}{2.704353in}}%
\pgfpathlineto{\pgfqpoint{2.406776in}{2.638752in}}%
\pgfpathlineto{\pgfqpoint{2.409226in}{2.704353in}}%
\pgfpathlineto{\pgfqpoint{2.411757in}{2.638752in}}%
\pgfpathlineto{\pgfqpoint{2.414207in}{2.704353in}}%
\pgfpathlineto{\pgfqpoint{2.417187in}{2.638752in}}%
\pgfpathlineto{\pgfqpoint{2.419637in}{2.704353in}}%
\pgfpathlineto{\pgfqpoint{2.430293in}{2.769953in}}%
\pgfpathlineto{\pgfqpoint{2.432743in}{2.704353in}}%
\pgfpathlineto{\pgfqpoint{2.435233in}{2.769953in}}%
\pgfpathlineto{\pgfqpoint{2.437683in}{2.704353in}}%
\pgfpathlineto{\pgfqpoint{2.440174in}{2.769953in}}%
\pgfpathlineto{\pgfqpoint{2.442705in}{2.704353in}}%
\pgfpathlineto{\pgfqpoint{2.445155in}{2.769953in}}%
\pgfpathlineto{\pgfqpoint{2.460506in}{2.704353in}}%
\pgfpathlineto{\pgfqpoint{2.462956in}{2.638752in}}%
\pgfpathlineto{\pgfqpoint{2.465406in}{2.496619in}}%
\pgfpathlineto{\pgfqpoint{2.467855in}{2.431018in}}%
\pgfpathlineto{\pgfqpoint{2.470305in}{2.288884in}}%
\pgfpathlineto{\pgfqpoint{2.472755in}{2.223284in}}%
\pgfpathlineto{\pgfqpoint{2.477654in}{2.015550in}}%
\pgfpathlineto{\pgfqpoint{2.492720in}{1.611015in}}%
\pgfpathlineto{\pgfqpoint{2.514971in}{1.272080in}}%
\pgfpathlineto{\pgfqpoint{2.517707in}{1.337681in}}%
\pgfpathlineto{\pgfqpoint{2.522974in}{1.195547in}}%
\pgfpathlineto{\pgfqpoint{2.525832in}{1.272080in}}%
\pgfpathlineto{\pgfqpoint{2.528282in}{1.195547in}}%
\pgfpathlineto{\pgfqpoint{2.533548in}{1.129947in}}%
\pgfpathlineto{\pgfqpoint{2.535998in}{1.195547in}}%
\pgfpathlineto{\pgfqpoint{2.538448in}{1.129947in}}%
\pgfpathlineto{\pgfqpoint{2.542817in}{1.064346in}}%
\pgfpathlineto{\pgfqpoint{2.545307in}{1.129947in}}%
\pgfpathlineto{\pgfqpoint{2.547757in}{1.064346in}}%
\pgfpathlineto{\pgfqpoint{2.552901in}{0.998746in}}%
\pgfpathlineto{\pgfqpoint{2.555392in}{1.064346in}}%
\pgfpathlineto{\pgfqpoint{2.557841in}{0.998746in}}%
\pgfpathlineto{\pgfqpoint{2.564619in}{0.922212in}}%
\pgfpathlineto{\pgfqpoint{2.567069in}{0.998746in}}%
\pgfpathlineto{\pgfqpoint{2.569518in}{0.922212in}}%
\pgfpathlineto{\pgfqpoint{2.572009in}{0.998746in}}%
\pgfpathlineto{\pgfqpoint{2.574459in}{0.922212in}}%
\pgfpathlineto{\pgfqpoint{2.578296in}{0.856612in}}%
\pgfpathlineto{\pgfqpoint{2.580828in}{0.922212in}}%
\pgfpathlineto{\pgfqpoint{2.583278in}{0.856612in}}%
\pgfpathlineto{\pgfqpoint{2.590953in}{0.791012in}}%
\pgfpathlineto{\pgfqpoint{2.593403in}{0.856612in}}%
\pgfpathlineto{\pgfqpoint{2.596506in}{0.791012in}}%
\pgfpathlineto{\pgfqpoint{2.598956in}{0.856612in}}%
\pgfpathlineto{\pgfqpoint{2.601405in}{0.791012in}}%
\pgfpathlineto{\pgfqpoint{2.609163in}{0.725412in}}%
\pgfpathlineto{\pgfqpoint{2.611613in}{0.791012in}}%
\pgfpathlineto{\pgfqpoint{2.614144in}{0.725412in}}%
\pgfpathlineto{\pgfqpoint{2.616594in}{0.791012in}}%
\pgfpathlineto{\pgfqpoint{2.619084in}{0.725412in}}%
\pgfpathlineto{\pgfqpoint{2.621534in}{0.791012in}}%
\pgfpathlineto{\pgfqpoint{2.623984in}{0.725412in}}%
\pgfpathlineto{\pgfqpoint{2.626433in}{0.791012in}}%
\pgfpathlineto{\pgfqpoint{2.628883in}{0.725412in}}%
\pgfpathlineto{\pgfqpoint{2.639988in}{0.648878in}}%
\pgfpathlineto{\pgfqpoint{2.642438in}{0.725412in}}%
\pgfpathlineto{\pgfqpoint{2.644888in}{0.648878in}}%
\pgfpathlineto{\pgfqpoint{2.647419in}{0.725412in}}%
\pgfpathlineto{\pgfqpoint{2.649869in}{0.648878in}}%
\pgfpathlineto{\pgfqpoint{2.652972in}{0.725412in}}%
\pgfpathlineto{\pgfqpoint{2.660893in}{0.856612in}}%
\pgfpathlineto{\pgfqpoint{2.664649in}{0.922212in}}%
\pgfpathlineto{\pgfqpoint{2.671222in}{1.064346in}}%
\pgfpathlineto{\pgfqpoint{2.691146in}{1.403281in}}%
\pgfpathlineto{\pgfqpoint{2.709928in}{1.676615in}}%
\pgfpathlineto{\pgfqpoint{2.712704in}{1.611015in}}%
\pgfpathlineto{\pgfqpoint{2.717603in}{1.742216in}}%
\pgfpathlineto{\pgfqpoint{2.722054in}{1.818749in}}%
\pgfpathlineto{\pgfqpoint{2.727565in}{1.884349in}}%
\pgfpathlineto{\pgfqpoint{2.730097in}{1.818749in}}%
\pgfpathlineto{\pgfqpoint{2.735078in}{1.949950in}}%
\pgfpathlineto{\pgfqpoint{2.742264in}{2.015550in}}%
\pgfpathlineto{\pgfqpoint{2.744795in}{1.949950in}}%
\pgfpathlineto{\pgfqpoint{2.750307in}{2.092084in}}%
\pgfpathlineto{\pgfqpoint{2.752879in}{2.015550in}}%
\pgfpathlineto{\pgfqpoint{2.755329in}{2.092084in}}%
\pgfpathlineto{\pgfqpoint{2.759208in}{2.157684in}}%
\pgfpathlineto{\pgfqpoint{2.761739in}{2.092084in}}%
\pgfpathlineto{\pgfqpoint{2.764189in}{2.157684in}}%
\pgfpathlineto{\pgfqpoint{2.767945in}{2.223284in}}%
\pgfpathlineto{\pgfqpoint{2.770395in}{2.157684in}}%
\pgfpathlineto{\pgfqpoint{2.772844in}{2.223284in}}%
\pgfpathlineto{\pgfqpoint{2.777948in}{2.288884in}}%
\pgfpathlineto{\pgfqpoint{2.780398in}{2.223284in}}%
\pgfpathlineto{\pgfqpoint{2.782847in}{2.288884in}}%
\pgfpathlineto{\pgfqpoint{2.790646in}{2.365418in}}%
\pgfpathlineto{\pgfqpoint{2.793095in}{2.288884in}}%
\pgfpathlineto{\pgfqpoint{2.795545in}{2.365418in}}%
\pgfpathlineto{\pgfqpoint{2.801833in}{2.431018in}}%
\pgfpathlineto{\pgfqpoint{2.804282in}{2.365418in}}%
\pgfpathlineto{\pgfqpoint{2.806732in}{2.431018in}}%
\pgfpathlineto{\pgfqpoint{2.809263in}{2.365418in}}%
\pgfpathlineto{\pgfqpoint{2.811713in}{2.431018in}}%
\pgfpathlineto{\pgfqpoint{2.820369in}{2.496619in}}%
\pgfpathlineto{\pgfqpoint{2.822818in}{2.431018in}}%
\pgfpathlineto{\pgfqpoint{2.825268in}{2.496619in}}%
\pgfpathlineto{\pgfqpoint{2.837639in}{2.562219in}}%
\pgfpathlineto{\pgfqpoint{2.840089in}{2.496619in}}%
\pgfpathlineto{\pgfqpoint{2.842539in}{2.562219in}}%
\pgfpathlineto{\pgfqpoint{2.845111in}{2.496619in}}%
\pgfpathlineto{\pgfqpoint{2.847560in}{2.562219in}}%
\pgfpathlineto{\pgfqpoint{2.857073in}{2.638752in}}%
\pgfpathlineto{\pgfqpoint{2.859523in}{2.562219in}}%
\pgfpathlineto{\pgfqpoint{2.862014in}{2.638752in}}%
\pgfpathlineto{\pgfqpoint{2.864463in}{2.562219in}}%
\pgfpathlineto{\pgfqpoint{2.866913in}{2.638752in}}%
\pgfpathlineto{\pgfqpoint{2.883694in}{2.704353in}}%
\pgfpathlineto{\pgfqpoint{2.886143in}{2.638752in}}%
\pgfpathlineto{\pgfqpoint{2.888675in}{2.704353in}}%
\pgfpathlineto{\pgfqpoint{2.891124in}{2.638752in}}%
\pgfpathlineto{\pgfqpoint{2.893574in}{2.704353in}}%
\pgfpathlineto{\pgfqpoint{2.896187in}{2.638752in}}%
\pgfpathlineto{\pgfqpoint{2.898637in}{2.704353in}}%
\pgfpathlineto{\pgfqpoint{2.909497in}{2.769953in}}%
\pgfpathlineto{\pgfqpoint{2.911947in}{2.704353in}}%
\pgfpathlineto{\pgfqpoint{2.914478in}{2.769953in}}%
\pgfpathlineto{\pgfqpoint{2.916928in}{2.704353in}}%
\pgfpathlineto{\pgfqpoint{2.919378in}{2.769953in}}%
\pgfpathlineto{\pgfqpoint{2.921868in}{2.704353in}}%
\pgfpathlineto{\pgfqpoint{2.924318in}{2.769953in}}%
\pgfpathlineto{\pgfqpoint{2.927870in}{2.704353in}}%
\pgfpathlineto{\pgfqpoint{2.930320in}{2.769953in}}%
\pgfpathlineto{\pgfqpoint{2.941139in}{2.835553in}}%
\pgfpathlineto{\pgfqpoint{2.943589in}{2.769953in}}%
\pgfpathlineto{\pgfqpoint{2.946039in}{2.835553in}}%
\pgfpathlineto{\pgfqpoint{2.958287in}{2.223284in}}%
\pgfpathlineto{\pgfqpoint{2.960737in}{2.157684in}}%
\pgfpathlineto{\pgfqpoint{2.963187in}{2.015550in}}%
\pgfpathlineto{\pgfqpoint{2.980702in}{1.545415in}}%
\pgfpathlineto{\pgfqpoint{2.989521in}{1.403281in}}%
\pgfpathlineto{\pgfqpoint{2.994380in}{1.337681in}}%
\pgfpathlineto{\pgfqpoint{2.996952in}{1.403281in}}%
\pgfpathlineto{\pgfqpoint{3.001851in}{1.272080in}}%
\pgfpathlineto{\pgfqpoint{3.007077in}{1.195547in}}%
\pgfpathlineto{\pgfqpoint{3.009568in}{1.272080in}}%
\pgfpathlineto{\pgfqpoint{3.012018in}{1.195547in}}%
\pgfpathlineto{\pgfqpoint{3.015529in}{1.129947in}}%
\pgfpathlineto{\pgfqpoint{3.017979in}{1.195547in}}%
\pgfpathlineto{\pgfqpoint{3.020428in}{1.129947in}}%
\pgfpathlineto{\pgfqpoint{3.025450in}{1.064346in}}%
\pgfpathlineto{\pgfqpoint{3.027900in}{1.129947in}}%
\pgfpathlineto{\pgfqpoint{3.030350in}{1.064346in}}%
\pgfpathlineto{\pgfqpoint{3.032799in}{1.129947in}}%
\pgfpathlineto{\pgfqpoint{3.035249in}{1.064346in}}%
\pgfpathlineto{\pgfqpoint{3.045905in}{0.998746in}}%
\pgfpathlineto{\pgfqpoint{3.048437in}{1.064346in}}%
\pgfpathlineto{\pgfqpoint{3.050886in}{0.998746in}}%
\pgfpathlineto{\pgfqpoint{3.057011in}{0.922212in}}%
\pgfpathlineto{\pgfqpoint{3.059460in}{0.998746in}}%
\pgfpathlineto{\pgfqpoint{3.061910in}{0.922212in}}%
\pgfpathlineto{\pgfqpoint{3.070362in}{0.856612in}}%
\pgfpathlineto{\pgfqpoint{3.072811in}{0.922212in}}%
\pgfpathlineto{\pgfqpoint{3.075261in}{0.856612in}}%
\pgfpathlineto{\pgfqpoint{3.077752in}{0.922212in}}%
\pgfpathlineto{\pgfqpoint{3.080201in}{0.856612in}}%
\pgfpathlineto{\pgfqpoint{3.095471in}{0.791012in}}%
\pgfpathlineto{\pgfqpoint{3.097921in}{0.856612in}}%
\pgfpathlineto{\pgfqpoint{3.100371in}{0.791012in}}%
\pgfpathlineto{\pgfqpoint{3.102902in}{0.856612in}}%
\pgfpathlineto{\pgfqpoint{3.105352in}{0.791012in}}%
\pgfpathlineto{\pgfqpoint{3.110496in}{0.856612in}}%
\pgfpathlineto{\pgfqpoint{3.112946in}{0.791012in}}%
\pgfpathlineto{\pgfqpoint{3.119519in}{0.725412in}}%
\pgfpathlineto{\pgfqpoint{3.121969in}{0.791012in}}%
\pgfpathlineto{\pgfqpoint{3.124541in}{0.725412in}}%
\pgfpathlineto{\pgfqpoint{3.127032in}{0.791012in}}%
\pgfpathlineto{\pgfqpoint{3.129481in}{0.725412in}}%
\pgfpathlineto{\pgfqpoint{3.131972in}{0.791012in}}%
\pgfpathlineto{\pgfqpoint{3.134422in}{0.725412in}}%
\pgfpathlineto{\pgfqpoint{3.136871in}{0.791012in}}%
\pgfpathlineto{\pgfqpoint{3.139321in}{0.725412in}}%
\pgfpathlineto{\pgfqpoint{3.154713in}{0.648878in}}%
\pgfpathlineto{\pgfqpoint{3.157163in}{0.725412in}}%
\pgfpathlineto{\pgfqpoint{3.159654in}{0.648878in}}%
\pgfpathlineto{\pgfqpoint{3.162103in}{0.725412in}}%
\pgfpathlineto{\pgfqpoint{3.192847in}{1.337681in}}%
\pgfpathlineto{\pgfqpoint{3.201993in}{1.468881in}}%
\pgfpathlineto{\pgfqpoint{3.206321in}{1.545415in}}%
\pgfpathlineto{\pgfqpoint{3.212118in}{1.611015in}}%
\pgfpathlineto{\pgfqpoint{3.215956in}{1.676615in}}%
\pgfpathlineto{\pgfqpoint{3.218487in}{1.611015in}}%
\pgfpathlineto{\pgfqpoint{3.223387in}{1.742216in}}%
\pgfpathlineto{\pgfqpoint{3.227592in}{1.818749in}}%
\pgfpathlineto{\pgfqpoint{3.230042in}{1.742216in}}%
\pgfpathlineto{\pgfqpoint{3.234982in}{1.884349in}}%
\pgfpathlineto{\pgfqpoint{3.248374in}{2.015550in}}%
\pgfpathlineto{\pgfqpoint{3.250905in}{1.949950in}}%
\pgfpathlineto{\pgfqpoint{3.255968in}{2.092084in}}%
\pgfpathlineto{\pgfqpoint{3.258418in}{2.015550in}}%
\pgfpathlineto{\pgfqpoint{3.263970in}{2.157684in}}%
\pgfpathlineto{\pgfqpoint{3.266461in}{2.092084in}}%
\pgfpathlineto{\pgfqpoint{3.268911in}{2.157684in}}%
\pgfpathlineto{\pgfqpoint{3.272667in}{2.223284in}}%
\pgfpathlineto{\pgfqpoint{3.275157in}{2.157684in}}%
\pgfpathlineto{\pgfqpoint{3.277607in}{2.223284in}}%
\pgfpathlineto{\pgfqpoint{3.282343in}{2.288884in}}%
\pgfpathlineto{\pgfqpoint{3.284793in}{2.223284in}}%
\pgfpathlineto{\pgfqpoint{3.287243in}{2.288884in}}%
\pgfpathlineto{\pgfqpoint{3.295490in}{2.365418in}}%
\pgfpathlineto{\pgfqpoint{3.297940in}{2.288884in}}%
\pgfpathlineto{\pgfqpoint{3.300389in}{2.365418in}}%
\pgfpathlineto{\pgfqpoint{3.307494in}{2.431018in}}%
\pgfpathlineto{\pgfqpoint{3.309943in}{2.365418in}}%
\pgfpathlineto{\pgfqpoint{3.312393in}{2.431018in}}%
\pgfpathlineto{\pgfqpoint{3.322723in}{2.496619in}}%
\pgfpathlineto{\pgfqpoint{3.325172in}{2.431018in}}%
\pgfpathlineto{\pgfqpoint{3.327622in}{2.496619in}}%
\pgfpathlineto{\pgfqpoint{3.330194in}{2.431018in}}%
\pgfpathlineto{\pgfqpoint{3.332644in}{2.496619in}}%
\pgfpathlineto{\pgfqpoint{3.340320in}{2.562219in}}%
\pgfpathlineto{\pgfqpoint{3.342769in}{2.496619in}}%
\pgfpathlineto{\pgfqpoint{3.345219in}{2.562219in}}%
\pgfpathlineto{\pgfqpoint{3.347669in}{2.496619in}}%
\pgfpathlineto{\pgfqpoint{3.350118in}{2.562219in}}%
\pgfpathlineto{\pgfqpoint{3.362326in}{2.638752in}}%
\pgfpathlineto{\pgfqpoint{3.364776in}{2.562219in}}%
\pgfpathlineto{\pgfqpoint{3.367226in}{2.638752in}}%
\pgfpathlineto{\pgfqpoint{3.369839in}{2.562219in}}%
\pgfpathlineto{\pgfqpoint{3.372288in}{2.638752in}}%
\pgfpathlineto{\pgfqpoint{3.383516in}{2.704353in}}%
\pgfpathlineto{\pgfqpoint{3.385966in}{2.638752in}}%
\pgfpathlineto{\pgfqpoint{3.388416in}{2.704353in}}%
\pgfpathlineto{\pgfqpoint{3.390906in}{2.638752in}}%
\pgfpathlineto{\pgfqpoint{3.393356in}{2.704353in}}%
\pgfpathlineto{\pgfqpoint{3.396377in}{2.638752in}}%
\pgfpathlineto{\pgfqpoint{3.398827in}{2.704353in}}%
\pgfpathlineto{\pgfqpoint{3.409483in}{2.769953in}}%
\pgfpathlineto{\pgfqpoint{3.411933in}{2.704353in}}%
\pgfpathlineto{\pgfqpoint{3.414464in}{2.769953in}}%
\pgfpathlineto{\pgfqpoint{3.416914in}{2.704353in}}%
\pgfpathlineto{\pgfqpoint{3.419364in}{2.769953in}}%
\pgfpathlineto{\pgfqpoint{3.421895in}{2.704353in}}%
\pgfpathlineto{\pgfqpoint{3.424345in}{2.769953in}}%
\pgfpathlineto{\pgfqpoint{3.434511in}{2.704353in}}%
\pgfpathlineto{\pgfqpoint{3.439410in}{2.562219in}}%
\pgfpathlineto{\pgfqpoint{3.446760in}{2.223284in}}%
\pgfpathlineto{\pgfqpoint{3.449209in}{2.157684in}}%
\pgfpathlineto{\pgfqpoint{3.451659in}{2.015550in}}%
\pgfpathlineto{\pgfqpoint{3.466806in}{1.611015in}}%
\pgfpathlineto{\pgfqpoint{3.486159in}{1.337681in}}%
\pgfpathlineto{\pgfqpoint{3.488609in}{1.403281in}}%
\pgfpathlineto{\pgfqpoint{3.493508in}{1.272080in}}%
\pgfpathlineto{\pgfqpoint{3.497999in}{1.195547in}}%
\pgfpathlineto{\pgfqpoint{3.500490in}{1.272080in}}%
\pgfpathlineto{\pgfqpoint{3.502940in}{1.195547in}}%
\pgfpathlineto{\pgfqpoint{3.506737in}{1.129947in}}%
\pgfpathlineto{\pgfqpoint{3.509227in}{1.195547in}}%
\pgfpathlineto{\pgfqpoint{3.514494in}{1.064346in}}%
\pgfpathlineto{\pgfqpoint{3.516944in}{1.129947in}}%
\pgfpathlineto{\pgfqpoint{3.522374in}{0.998746in}}%
\pgfpathlineto{\pgfqpoint{3.524824in}{1.064346in}}%
\pgfpathlineto{\pgfqpoint{3.527273in}{0.998746in}}%
\pgfpathlineto{\pgfqpoint{3.537031in}{0.922212in}}%
\pgfpathlineto{\pgfqpoint{3.539563in}{0.998746in}}%
\pgfpathlineto{\pgfqpoint{3.542012in}{0.922212in}}%
\pgfpathlineto{\pgfqpoint{3.548137in}{0.856612in}}%
\pgfpathlineto{\pgfqpoint{3.550586in}{0.922212in}}%
\pgfpathlineto{\pgfqpoint{3.553036in}{0.856612in}}%
\pgfpathlineto{\pgfqpoint{3.565611in}{0.791012in}}%
\pgfpathlineto{\pgfqpoint{3.568061in}{0.856612in}}%
\pgfpathlineto{\pgfqpoint{3.570511in}{0.791012in}}%
\pgfpathlineto{\pgfqpoint{3.579493in}{0.725412in}}%
\pgfpathlineto{\pgfqpoint{3.581943in}{0.791012in}}%
\pgfpathlineto{\pgfqpoint{3.584392in}{0.725412in}}%
\pgfpathlineto{\pgfqpoint{3.586924in}{0.791012in}}%
\pgfpathlineto{\pgfqpoint{3.589374in}{0.725412in}}%
\pgfpathlineto{\pgfqpoint{3.596396in}{0.648878in}}%
\pgfpathlineto{\pgfqpoint{3.598927in}{0.725412in}}%
\pgfpathlineto{\pgfqpoint{3.601377in}{0.648878in}}%
\pgfpathlineto{\pgfqpoint{3.603827in}{0.725412in}}%
\pgfpathlineto{\pgfqpoint{3.606277in}{0.648878in}}%
\pgfpathlineto{\pgfqpoint{3.621955in}{0.725412in}}%
\pgfpathlineto{\pgfqpoint{3.636612in}{0.998746in}}%
\pgfpathlineto{\pgfqpoint{3.651351in}{1.272080in}}%
\pgfpathlineto{\pgfqpoint{3.659272in}{1.403281in}}%
\pgfpathlineto{\pgfqpoint{3.664702in}{1.468881in}}%
\pgfpathlineto{\pgfqpoint{3.679523in}{1.676615in}}%
\pgfpathlineto{\pgfqpoint{3.685402in}{1.742216in}}%
\pgfpathlineto{\pgfqpoint{3.690832in}{1.818749in}}%
\pgfpathlineto{\pgfqpoint{3.693568in}{1.742216in}}%
\pgfpathlineto{\pgfqpoint{3.698467in}{1.884349in}}%
\pgfpathlineto{\pgfqpoint{3.703285in}{1.949950in}}%
\pgfpathlineto{\pgfqpoint{3.705816in}{1.884349in}}%
\pgfpathlineto{\pgfqpoint{3.711451in}{2.015550in}}%
\pgfpathlineto{\pgfqpoint{3.713982in}{1.949950in}}%
\pgfpathlineto{\pgfqpoint{3.718882in}{2.092084in}}%
\pgfpathlineto{\pgfqpoint{3.721372in}{2.015550in}}%
\pgfpathlineto{\pgfqpoint{3.726966in}{2.157684in}}%
\pgfpathlineto{\pgfqpoint{3.729415in}{2.092084in}}%
\pgfpathlineto{\pgfqpoint{3.731865in}{2.157684in}}%
\pgfpathlineto{\pgfqpoint{3.736315in}{2.223284in}}%
\pgfpathlineto{\pgfqpoint{3.738765in}{2.157684in}}%
\pgfpathlineto{\pgfqpoint{3.741215in}{2.223284in}}%
\pgfpathlineto{\pgfqpoint{3.746727in}{2.288884in}}%
\pgfpathlineto{\pgfqpoint{3.749176in}{2.223284in}}%
\pgfpathlineto{\pgfqpoint{3.751626in}{2.288884in}}%
\pgfpathlineto{\pgfqpoint{3.758281in}{2.365418in}}%
\pgfpathlineto{\pgfqpoint{3.760731in}{2.288884in}}%
\pgfpathlineto{\pgfqpoint{3.763180in}{2.365418in}}%
\pgfpathlineto{\pgfqpoint{3.770121in}{2.431018in}}%
\pgfpathlineto{\pgfqpoint{3.772571in}{2.365418in}}%
\pgfpathlineto{\pgfqpoint{3.775021in}{2.431018in}}%
\pgfpathlineto{\pgfqpoint{3.785391in}{2.496619in}}%
\pgfpathlineto{\pgfqpoint{3.787841in}{2.431018in}}%
\pgfpathlineto{\pgfqpoint{3.790291in}{2.496619in}}%
\pgfpathlineto{\pgfqpoint{3.792740in}{2.431018in}}%
\pgfpathlineto{\pgfqpoint{3.795190in}{2.496619in}}%
\pgfpathlineto{\pgfqpoint{3.803723in}{2.562219in}}%
\pgfpathlineto{\pgfqpoint{3.806173in}{2.496619in}}%
\pgfpathlineto{\pgfqpoint{3.808663in}{2.562219in}}%
\pgfpathlineto{\pgfqpoint{3.811113in}{2.496619in}}%
\pgfpathlineto{\pgfqpoint{3.813563in}{2.562219in}}%
\pgfpathlineto{\pgfqpoint{3.824913in}{2.638752in}}%
\pgfpathlineto{\pgfqpoint{3.827363in}{2.562219in}}%
\pgfpathlineto{\pgfqpoint{3.829813in}{2.638752in}}%
\pgfpathlineto{\pgfqpoint{3.832303in}{2.562219in}}%
\pgfpathlineto{\pgfqpoint{3.834753in}{2.638752in}}%
\pgfpathlineto{\pgfqpoint{3.847450in}{2.704353in}}%
\pgfpathlineto{\pgfqpoint{3.849900in}{2.638752in}}%
\pgfpathlineto{\pgfqpoint{3.852391in}{2.704353in}}%
\pgfpathlineto{\pgfqpoint{3.854840in}{2.638752in}}%
\pgfpathlineto{\pgfqpoint{3.857331in}{2.704353in}}%
\pgfpathlineto{\pgfqpoint{3.861495in}{2.638752in}}%
\pgfpathlineto{\pgfqpoint{3.863945in}{2.704353in}}%
\pgfpathlineto{\pgfqpoint{3.873417in}{2.769953in}}%
\pgfpathlineto{\pgfqpoint{3.875867in}{2.704353in}}%
\pgfpathlineto{\pgfqpoint{3.878358in}{2.769953in}}%
\pgfpathlineto{\pgfqpoint{3.880807in}{2.704353in}}%
\pgfpathlineto{\pgfqpoint{3.883257in}{2.769953in}}%
\pgfpathlineto{\pgfqpoint{3.890606in}{2.431018in}}%
\pgfpathlineto{\pgfqpoint{3.895506in}{2.288884in}}%
\pgfpathlineto{\pgfqpoint{3.897955in}{2.157684in}}%
\pgfpathlineto{\pgfqpoint{3.907754in}{1.884349in}}%
\pgfpathlineto{\pgfqpoint{3.910204in}{1.742216in}}%
\pgfpathlineto{\pgfqpoint{3.923922in}{1.468881in}}%
\pgfpathlineto{\pgfqpoint{3.934252in}{1.337681in}}%
\pgfpathlineto{\pgfqpoint{3.940580in}{1.272080in}}%
\pgfpathlineto{\pgfqpoint{3.943275in}{1.337681in}}%
\pgfpathlineto{\pgfqpoint{3.948174in}{1.195547in}}%
\pgfpathlineto{\pgfqpoint{3.953482in}{1.129947in}}%
\pgfpathlineto{\pgfqpoint{3.956013in}{1.195547in}}%
\pgfpathlineto{\pgfqpoint{3.958463in}{1.129947in}}%
\pgfpathlineto{\pgfqpoint{3.962546in}{1.064346in}}%
\pgfpathlineto{\pgfqpoint{3.964996in}{1.129947in}}%
\pgfpathlineto{\pgfqpoint{3.970508in}{0.998746in}}%
\pgfpathlineto{\pgfqpoint{3.972957in}{1.064346in}}%
\pgfpathlineto{\pgfqpoint{3.975407in}{0.998746in}}%
\pgfpathlineto{\pgfqpoint{3.982348in}{0.922212in}}%
\pgfpathlineto{\pgfqpoint{3.984798in}{0.998746in}}%
\pgfpathlineto{\pgfqpoint{3.987247in}{0.922212in}}%
\pgfpathlineto{\pgfqpoint{3.989738in}{0.998746in}}%
\pgfpathlineto{\pgfqpoint{3.992188in}{0.922212in}}%
\pgfpathlineto{\pgfqpoint{3.999904in}{0.856612in}}%
\pgfpathlineto{\pgfqpoint{4.002354in}{0.922212in}}%
\pgfpathlineto{\pgfqpoint{4.005661in}{0.856612in}}%
\pgfpathlineto{\pgfqpoint{4.008111in}{0.922212in}}%
\pgfpathlineto{\pgfqpoint{4.010560in}{0.856612in}}%
\pgfpathlineto{\pgfqpoint{4.013786in}{0.922212in}}%
\pgfpathlineto{\pgfqpoint{4.016236in}{0.856612in}}%
\pgfpathlineto{\pgfqpoint{4.021135in}{0.791012in}}%
\pgfpathlineto{\pgfqpoint{4.023585in}{0.856612in}}%
\pgfpathlineto{\pgfqpoint{4.026034in}{0.791012in}}%
\pgfpathlineto{\pgfqpoint{4.028607in}{0.856612in}}%
\pgfpathlineto{\pgfqpoint{4.031056in}{0.791012in}}%
\pgfpathlineto{\pgfqpoint{4.041263in}{0.725412in}}%
\pgfpathlineto{\pgfqpoint{4.043713in}{0.791012in}}%
\pgfpathlineto{\pgfqpoint{4.046163in}{0.725412in}}%
\pgfpathlineto{\pgfqpoint{4.048613in}{0.791012in}}%
\pgfpathlineto{\pgfqpoint{4.051062in}{0.725412in}}%
\pgfpathlineto{\pgfqpoint{4.058003in}{0.648878in}}%
\pgfpathlineto{\pgfqpoint{4.060453in}{0.725412in}}%
\pgfpathlineto{\pgfqpoint{4.062902in}{0.648878in}}%
\pgfpathlineto{\pgfqpoint{4.065393in}{0.725412in}}%
\pgfpathlineto{\pgfqpoint{4.067843in}{0.648878in}}%
\pgfpathlineto{\pgfqpoint{4.070333in}{0.725412in}}%
\pgfpathlineto{\pgfqpoint{4.072783in}{0.648878in}}%
\pgfpathlineto{\pgfqpoint{4.077764in}{0.791012in}}%
\pgfpathlineto{\pgfqpoint{4.092503in}{1.064346in}}%
\pgfpathlineto{\pgfqpoint{4.107895in}{1.337681in}}%
\pgfpathlineto{\pgfqpoint{4.132597in}{1.676615in}}%
\pgfpathlineto{\pgfqpoint{4.135210in}{1.611015in}}%
\pgfpathlineto{\pgfqpoint{4.140109in}{1.742216in}}%
\pgfpathlineto{\pgfqpoint{4.145009in}{1.818749in}}%
\pgfpathlineto{\pgfqpoint{4.150684in}{1.884349in}}%
\pgfpathlineto{\pgfqpoint{4.165423in}{2.015550in}}%
\pgfpathlineto{\pgfqpoint{4.167913in}{1.949950in}}%
\pgfpathlineto{\pgfqpoint{4.174201in}{2.092084in}}%
\pgfpathlineto{\pgfqpoint{4.176651in}{2.015550in}}%
\pgfpathlineto{\pgfqpoint{4.179100in}{2.092084in}}%
\pgfpathlineto{\pgfqpoint{4.183387in}{2.157684in}}%
\pgfpathlineto{\pgfqpoint{4.185837in}{2.092084in}}%
\pgfpathlineto{\pgfqpoint{4.188287in}{2.157684in}}%
\pgfpathlineto{\pgfqpoint{4.193023in}{2.223284in}}%
\pgfpathlineto{\pgfqpoint{4.195473in}{2.157684in}}%
\pgfpathlineto{\pgfqpoint{4.197922in}{2.223284in}}%
\pgfpathlineto{\pgfqpoint{4.204537in}{2.288884in}}%
\pgfpathlineto{\pgfqpoint{4.206986in}{2.223284in}}%
\pgfpathlineto{\pgfqpoint{4.209436in}{2.288884in}}%
\pgfpathlineto{\pgfqpoint{4.217398in}{2.365418in}}%
\pgfpathlineto{\pgfqpoint{4.219888in}{2.288884in}}%
\pgfpathlineto{\pgfqpoint{4.222338in}{2.365418in}}%
\pgfpathlineto{\pgfqpoint{4.231075in}{2.431018in}}%
\pgfpathlineto{\pgfqpoint{4.233525in}{2.365418in}}%
\pgfpathlineto{\pgfqpoint{4.235975in}{2.431018in}}%
\pgfpathlineto{\pgfqpoint{4.248141in}{2.496619in}}%
\pgfpathlineto{\pgfqpoint{4.250591in}{2.431018in}}%
\pgfpathlineto{\pgfqpoint{4.253082in}{2.496619in}}%
\pgfpathlineto{\pgfqpoint{4.255654in}{2.431018in}}%
\pgfpathlineto{\pgfqpoint{4.258104in}{2.496619in}}%
\pgfpathlineto{\pgfqpoint{4.266228in}{2.562219in}}%
\pgfpathlineto{\pgfqpoint{4.268678in}{2.496619in}}%
\pgfpathlineto{\pgfqpoint{4.271291in}{2.562219in}}%
\pgfpathlineto{\pgfqpoint{4.273741in}{2.496619in}}%
\pgfpathlineto{\pgfqpoint{4.276191in}{2.562219in}}%
\pgfpathlineto{\pgfqpoint{4.289460in}{2.638752in}}%
\pgfpathlineto{\pgfqpoint{4.291910in}{2.562219in}}%
\pgfpathlineto{\pgfqpoint{4.294359in}{2.638752in}}%
\pgfpathlineto{\pgfqpoint{4.296850in}{2.562219in}}%
\pgfpathlineto{\pgfqpoint{4.299300in}{2.638752in}}%
\pgfpathlineto{\pgfqpoint{4.302443in}{2.562219in}}%
\pgfpathlineto{\pgfqpoint{4.304893in}{2.638752in}}%
\pgfpathlineto{\pgfqpoint{4.318244in}{2.704353in}}%
\pgfpathlineto{\pgfqpoint{4.320694in}{2.638752in}}%
\pgfpathlineto{\pgfqpoint{4.323184in}{2.704353in}}%
\pgfpathlineto{\pgfqpoint{4.325634in}{2.638752in}}%
\pgfpathlineto{\pgfqpoint{4.328084in}{2.704353in}}%
\pgfpathlineto{\pgfqpoint{4.331146in}{2.638752in}}%
\pgfpathlineto{\pgfqpoint{4.333595in}{2.704353in}}%
\pgfpathlineto{\pgfqpoint{4.333595in}{2.704353in}}%
\pgfusepath{stroke}%
\end{pgfscope}%
\begin{pgfscope}%
\pgfsetrectcap%
\pgfsetmiterjoin%
\pgfsetlinewidth{0.803000pt}%
\definecolor{currentstroke}{rgb}{0.000000,0.000000,0.000000}%
\pgfsetstrokecolor{currentstroke}%
\pgfsetdash{}{0pt}%
\pgfpathmoveto{\pgfqpoint{0.634869in}{0.539544in}}%
\pgfpathlineto{\pgfqpoint{0.634869in}{2.944887in}}%
\pgfusepath{stroke}%
\end{pgfscope}%
\begin{pgfscope}%
\pgfsetrectcap%
\pgfsetmiterjoin%
\pgfsetlinewidth{0.803000pt}%
\definecolor{currentstroke}{rgb}{0.000000,0.000000,0.000000}%
\pgfsetstrokecolor{currentstroke}%
\pgfsetdash{}{0pt}%
\pgfpathmoveto{\pgfqpoint{4.514985in}{0.539544in}}%
\pgfpathlineto{\pgfqpoint{4.514985in}{2.944887in}}%
\pgfusepath{stroke}%
\end{pgfscope}%
\begin{pgfscope}%
\pgfsetrectcap%
\pgfsetmiterjoin%
\pgfsetlinewidth{0.803000pt}%
\definecolor{currentstroke}{rgb}{0.000000,0.000000,0.000000}%
\pgfsetstrokecolor{currentstroke}%
\pgfsetdash{}{0pt}%
\pgfpathmoveto{\pgfqpoint{0.634869in}{0.539544in}}%
\pgfpathlineto{\pgfqpoint{4.514985in}{0.539544in}}%
\pgfusepath{stroke}%
\end{pgfscope}%
\begin{pgfscope}%
\pgfsetrectcap%
\pgfsetmiterjoin%
\pgfsetlinewidth{0.803000pt}%
\definecolor{currentstroke}{rgb}{0.000000,0.000000,0.000000}%
\pgfsetstrokecolor{currentstroke}%
\pgfsetdash{}{0pt}%
\pgfpathmoveto{\pgfqpoint{0.634869in}{2.944887in}}%
\pgfpathlineto{\pgfqpoint{4.514985in}{2.944887in}}%
\pgfusepath{stroke}%
\end{pgfscope}%
\begin{pgfscope}%
\pgfsetbuttcap%
\pgfsetmiterjoin%
\definecolor{currentfill}{rgb}{1.000000,1.000000,1.000000}%
\pgfsetfillcolor{currentfill}%
\pgfsetfillopacity{0.800000}%
\pgfsetlinewidth{1.003750pt}%
\definecolor{currentstroke}{rgb}{0.800000,0.800000,0.800000}%
\pgfsetstrokecolor{currentstroke}%
\pgfsetstrokeopacity{0.800000}%
\pgfsetdash{}{0pt}%
\pgfpathmoveto{\pgfqpoint{0.712647in}{2.544665in}}%
\pgfpathlineto{\pgfqpoint{2.212202in}{2.544665in}}%
\pgfpathquadraticcurveto{\pgfqpoint{2.234424in}{2.544665in}}{\pgfqpoint{2.234424in}{2.566887in}}%
\pgfpathlineto{\pgfqpoint{2.234424in}{2.867109in}}%
\pgfpathquadraticcurveto{\pgfqpoint{2.234424in}{2.889331in}}{\pgfqpoint{2.212202in}{2.889331in}}%
\pgfpathlineto{\pgfqpoint{0.712647in}{2.889331in}}%
\pgfpathquadraticcurveto{\pgfqpoint{0.690424in}{2.889331in}}{\pgfqpoint{0.690424in}{2.867109in}}%
\pgfpathlineto{\pgfqpoint{0.690424in}{2.566887in}}%
\pgfpathquadraticcurveto{\pgfqpoint{0.690424in}{2.544665in}}{\pgfqpoint{0.712647in}{2.544665in}}%
\pgfpathlineto{\pgfqpoint{0.712647in}{2.544665in}}%
\pgfpathclose%
\pgfusepath{stroke,fill}%
\end{pgfscope}%
\begin{pgfscope}%
\pgfsetrectcap%
\pgfsetroundjoin%
\pgfsetlinewidth{0.501875pt}%
\definecolor{currentstroke}{rgb}{0.121569,0.466667,0.705882}%
\pgfsetstrokecolor{currentstroke}%
\pgfsetstrokeopacity{0.700000}%
\pgfsetdash{}{0pt}%
\pgfpathmoveto{\pgfqpoint{0.734869in}{2.805998in}}%
\pgfpathlineto{\pgfqpoint{0.845980in}{2.805998in}}%
\pgfpathlineto{\pgfqpoint{0.957091in}{2.805998in}}%
\pgfusepath{stroke}%
\end{pgfscope}%
\begin{pgfscope}%
\definecolor{textcolor}{rgb}{0.000000,0.000000,0.000000}%
\pgfsetstrokecolor{textcolor}%
\pgfsetfillcolor{textcolor}%
\pgftext[x=1.045980in,y=2.767109in,left,base]{\color{textcolor}\rmfamily\fontsize{8.000000}{9.600000}\selectfont DUT vs KS34470A}%
\end{pgfscope}%
\begin{pgfscope}%
\pgfsetrectcap%
\pgfsetroundjoin%
\pgfsetlinewidth{0.501875pt}%
\definecolor{currentstroke}{rgb}{0.698039,0.133333,0.133333}%
\pgfsetstrokecolor{currentstroke}%
\pgfsetstrokeopacity{0.700000}%
\pgfsetdash{}{0pt}%
\pgfpathmoveto{\pgfqpoint{0.734869in}{2.649554in}}%
\pgfpathlineto{\pgfqpoint{0.845980in}{2.649554in}}%
\pgfpathlineto{\pgfqpoint{0.957091in}{2.649554in}}%
\pgfusepath{stroke}%
\end{pgfscope}%
\begin{pgfscope}%
\definecolor{textcolor}{rgb}{0.000000,0.000000,0.000000}%
\pgfsetstrokecolor{textcolor}%
\pgfsetfillcolor{textcolor}%
\pgftext[x=1.045980in,y=2.610665in,left,base]{\color{textcolor}\rmfamily\fontsize{8.000000}{9.600000}\selectfont Ambient Temperature}%
\end{pgfscope}%
\end{pgfpicture}%
\makeatother%
\endgroup%

    \caption{Voltage deviation from the mean voltage of a LM399 negative \qty{10}{\volt} reference measured with a Keysight 34470A at \qty{100}{\plc}.}
    \label{fig:lm399_vs_34470a}
\end{figure}

Figure \ref{fig:lm399_vs_34470a} shows an example of such measurement. This measurement highlights one the problems encountered during those measurements. From this measurement it is unclear whether the features seen in the graph are only a result of ambient temperature changes due to the cycling of the air conditioning or popcorn noise on top of that. These results hightlight the fact, that sub-\unit{ppm} measurements not only requires high-end gear, but also a very stable environment. From the data it follows, that the temperature coefficient of the DMM in linear approximation is:
\begin{equation}
    \alpha_\device{34470A} \approx \frac{\qty{6.08}{\micro\volt}-(-\qty{9.30}{\micro\volt})}{(\qty{21.85}{\celsius}-\qty{19.96}{\celsius})\qty{10}{\volt}} = \qty[per-mode=symbol]{0.86}{\micro\volt \per \volt \per \kelvin}
\end{equation}

While the temperature coeeficient is vastly better than the specified \qty[per-mode=symbol]{2}{\micro\volt \per \volt \per \kelvin} \cite{datasheet_keysight34470A}, it is not low enough for this type of measurement. The multimeter must therefore be kept in a temperature controlled environment. This issue was resolved by replacing the stock air conditioning controller with a custom PID controller as discussed in section \ref{}. Lastly, the noise floor of the measurement is \qty{1.5}{\micro\volt_{RMS}}, resulting in an estimated signal-to-noise ratio (SNR) of about \qty{10}{\decibel}, which is suffient to detect the popcorn noise.

While the temperature issue was being worked on, testing of the Zener diodes continued. To work around the temperature drift of the DMM, the amplification of the reference voltage was increased to \qty{15}{\volt}, the same voltage required by the digital current driver, and a differential measurement was realized. This measurement was done against a primary \qty{15}{\volt} reference board. To ensure, that any popcorn noise found originates only in the DUT and not the primary reference used, several reference boards where tested against a \device{Fluke 5440B}. The \device{5440B} does not exhibit popcorn noise as it uses a different voltage reference ic, namely two Motorola SZA263 in series \cite{service_manual_fluke_5440b}. Finally, a board that did not show popcorn noise in a period of three days was selected. The serial number of this primary or golden reference is \textit{\#1}.

Using this differential technique, the results greatly improved

In order to test a large amount of Zener diodes, and considering the duration of the burn-in process, which can take anything between \qtyrange{100}{1000}{\hour}, it is necessary to have an automated setup. This consists of a digital multimeter (DMM) a scanner and test board, that holds the Zener diodes and provides the necessary infrastructure for the diodes.



To conclude, we need a high performance DMM, a scanner, and a test fixture. The choices will detailed in the following sections.



\subsection{A Scanner System for Testing Zener Diodes}
As discussed before the diodes need to be tested for \qty{1000}{\hour} and it is not be feasible to test them individually. So a minimum of 10 diodes must be tested at the same time. To keep the system compact, the test setup a scanner to multiplex a single multimeter input. Several commercial options currently available were considered for this project and are shown in table \ref{tab:list_of_daqs}.

\begin{table}[h]
    \centering
    \small
    \begin{tabular}{ |l|l|l|l|l|l|l|l| }
        \hline
        \multirow{2}{*}{} & \multicolumn{2}{l|}{Keysight} & \multicolumn{3}{l|}{Keithley} & Fluke & Rigol \\
        \cline{2-8}
        & DAQ973A & 34980A & DAQ6510 & 2750 & 3706 & 2680 & M300 \\
        \hline
        DMM & \num{6.5} & \num{6.5} & \num{6.5} & \num{6.5} & \num{7.5} & \qty{18}{\bit} & \num{6.5} \\
        \hline
        Channels & 3x20 & 8x40 & 2x10 & 5x20 & 6x60 & 6x20 & 5x32 \\
        \hline
        FET & \textcolor{green!60!black}{\checkmark} & \textcolor{green!60!black}{\checkmark} & \textcolor{green!60!black}{\checkmark} & \textcolor{green!60!black}{\checkmark} & \textcolor{green!60!black}{\checkmark} & \textcolor{red!80!black}{\ding{55}} & \textcolor{red!80!black}{\ding{55}} \\
        \hline
        Voltage & \qty{120}{\volt} & \qty{80}{\volt} & \qty{60}{\volt} & \qty{60}{\volt} & \qty{200}{\volt} & \qty{75}{\volt} & \qty{300}{\volt} \\
        \hline
        Card & DAQM900A & 34925A & 7710 & 7710 & 3724 & 2680A-PAI & MC3132 \\
        \hline
        USB & \textcolor{green!60!black}{\checkmark} & \textcolor{green!60!black}{\checkmark} & \textcolor{green!60!black}{\checkmark} & \textcolor{red!80!black}{\ding{55}} & \textcolor{green!60!black}{\checkmark} & \textcolor{red!80!black}{\ding{55}} & \textcolor{green!60!black}{\checkmark} \\
        \hline
        Ethernet & \textcolor{green!60!black}{\checkmark} & \textcolor{green!60!black}{\checkmark} & \textcolor{green!60!black}{\checkmark} & \textcolor{red!80!black}{\ding{55}} & \textcolor{green!60!black}{\checkmark} & \textcolor{green!60!black}{\checkmark} & \textcolor{green!60!black}{\checkmark} \\
        \hline
        GPIB & \textcolor{green!60!black}{\checkmark} & \textcolor{green!60!black}{\checkmark} & \textcolor{green!60!black}{\checkmark} & \textcolor{green!60!black}{\checkmark} & \textcolor{green!60!black}{\checkmark} & \textcolor{red!80!black}{\ding{55}} & \textcolor{green!60!black}{\checkmark} \\
        %DMM & & & & & & & \\
        \hline
    \end{tabular}
    \caption{Overview of scanner mainframes}
    \label{tab:list_of_daqs}
\end{table}

A recent trend to more compact devices has led major manufacturers to include multimeters in the scanner mainframe creating so called data acquisition units. Legacy devices, that only have switching capabilities are no available. For example Keithley replaced the small desktop switch mainframe \device{Model 7001} with the \device{DAQ6510} and Keysight is offering the \device{DAQ973A}, a scanning \num{6.5} digit DMM, that accepts extension cards. Unfortunately, for this project, as discussed above, the integrated \num{6.5} digit multimeter does not add any value.

The simplest option is to go with an \num{8.5} digit multimeter that already included a scanner option or buy a used \device{Keithley 7001} from a second-hand dealer. The author has tested both options and the simplicity of only having a single device to connect and program makes the integrated scanner card of the \device{Model 2002} very attractive.

\begin{figure}[ht]
    \centering
    \scalebox{0.7}{%
        \import{figures/}{simplified_scanner.tex}
    } % scalebox
    \caption{Simplified schematic of the scanner front-end with parasitic elements}
\end{figure}

The scanner card used to multiplex the DMM does have to meet several specifications. The most important aspects are the number channels and the lifetime of the relays. Other factors, such as channel to channel isolation, the contact potential, resistance and maximum voltage is not the limiting factor.

The reason is, that in this case, the voltage is low, there is no ac component involved and the the typical input impedance of high-end multimeters is far more than \qty{100}{\giga\ohm} \cite{datasheet_fluke8588A,article_3458A_input_mpedance_2,datasheet_keithley2002,article_3458A_input_mpedance}.



In this work the Keithley (now Tektronix) \device{Model 2002} was chosen for three reasons. It is a very compact system requiring only a half-sized 2U rack in comparison to the other DMMs, that are typically full-sized 2U rack devices. The other two advantages are the integrated scanner card slot, that allows to to fit a 10 channel scanner card and finally the \qty{20}{\volt} range. The latter is interesting for testing the final voltage reference boards, as these have a \qty{15}{\volt} output, which is too much for the \qty{10}{\volt} range of most DMMs, so that testing the voltage reference Printed circuit boards
(PCBs) one would have to switch to the \qty{100}{\volt} range and forgo an extra digit of resolution and add more noise.



The test setup consists of a mounting PCB, that holds up to 20 Zener diode. It provides power regulation and a minimal circuit required to support each diode. This circuit is given here:

\begin{figure}[ht]
    \centering
    \scalebox{0.7}{%
        \import{figures/}{zener_burnin.tex}
    } % scalebox
    \caption{Circuit used for burning in the Zener diodes}
\end{figure}

The compensation network is required when using the ADR1399, because of its very low dynamic impedance as recommended in the data sheet \cite{datasheet_ADR1399}. It is not strictly required for the LM399, but fitted nonetheless, because there are no downsides to it. This makes the board compatible with both types of references. Each Zener output is protected using an output buffer, which provides isolation and short circuit protection. Finally there is a common mode filter at the output to suppress high frequency noise via ground loops.

The two key metrics of concern, that need to measured are popcorn noise and drift.

digital multimeter and a scanner card


\begin{figure}[ht]
    \centering
    \import{figures/}{DIN_41612.tex}
    \caption{The extension connector used in several Keithley multimeters}
\end{figure}

\begin{table}[ht]
    \centering
    \begin{tabular}{llllll}
        \toprule
        Pin    & Function    & Cable Colour    & Pin    & Function    &  Cable Colour\\
        \midrule
        a1, b1    & \SIrange[explicit-sign=+]{6}{20}{\volt}    & brown    & \num{6}    & GND    & green/white\\
        a2, b2    & PD cathode    & red    & \num{7}    & LD Cathode    & blue/white\\
        a3, b3    & LD case (GND)    & red/white    & \num{8}    & LD Anode    & blue\\
        a4    & PD anode (GND)    & red/white    & \num{9}    & LD current    & green\\
        a5    & \SIrange{-6}{-20}{\volt}    & brown/white\\
        \bottomrule
    \end{tabular}
\end{table}

As a sidenote, for the pure entertainment of the author, several batches of LM399 Zener diodes were purchased from non-authorized dealers. Some were marked as refurbished, the others were not marked as such, but clearly were. These so-called refurbished diodes are not to be used in production devices. To entertain and warn the reader a small selection of examples are shown here in figure \ref{fig:fake_lm399}. All but one diode, which is shown for comparison, are refurbished.

\begin{figure}[h]
    \centering
    %\includegraphics[width=0.75\textwidth]{images/foo.png}
    \caption{Refurbished LM399 Zener diodes. From left to right: }
    \label{fig:fake_lm399}
\end{figure}

As it can be clearly seen, the sellers have gone to some effort to hide the fact, that these diodes have been used before. When a through-hole is soldered to the PCB, its legs will be trimmed to match the PCB thickness. In order to conceal this, the legs need to be extended to their original length. The legs of the LM399 are Kovar, because the LM399 is hermetically sealed with a glass seal and Kovar has the same coefficient of expansion as borosilicate glass. The forgers typically weld steel legs to the Kovar legs and then either gold-plate or tin them, as can be seen in fig. \ref{fig:fake_lm399_legs}.

\begin{figure}[h]
    \centering
    %\includegraphics[width=0.75\textwidth]{images/foo.png}
    \caption{Fake steel legs of a refurbished LM399.}
    \label{fig:fake_lm399_legs}
\end{figure}

Much to the delight of the author the refurbished diodes prove valuable for educational purposes. As the origin and method of extraction from the original circuit is unknown, but can be imagined to be rather savage, the diodes are typically faulty. They can therefore be used to validate the test setup and demonstrate the popcorn noise found in the LM399. A very good example in shown in fig. \ref{fig:fake_lm399_popcorn_noise}.

\begin{figure}[h]
    \centering
    %% Creator: Matplotlib, PGF backend
%%
%% To include the figure in your LaTeX document, write
%%   \input{<filename>.pgf}
%%
%% Make sure the required packages are loaded in your preamble
%%   \usepackage{pgf}
%%
%% Also ensure that all the required font packages are loaded; for instance,
%% the lmodern package is sometimes necessary when using math font.
%%   \usepackage{lmodern}
%%
%% Figures using additional raster images can only be included by \input if
%% they are in the same directory as the main LaTeX file. For loading figures
%% from other directories you can use the `import` package
%%   \usepackage{import}
%%
%% and then include the figures with
%%   \import{<path to file>}{<filename>.pgf}
%%
%% Matplotlib used the following preamble
%%   \usepackage{siunitx}
%%   \usepackage{fontspec}
%%
\begingroup%
\makeatletter%
\begin{pgfpicture}%
\pgfpathrectangle{\pgfpointorigin}{\pgfqpoint{5.208662in}{3.219130in}}%
\pgfusepath{use as bounding box, clip}%
\begin{pgfscope}%
\pgfsetbuttcap%
\pgfsetmiterjoin%
\definecolor{currentfill}{rgb}{1.000000,1.000000,1.000000}%
\pgfsetfillcolor{currentfill}%
\pgfsetlinewidth{0.000000pt}%
\definecolor{currentstroke}{rgb}{1.000000,1.000000,1.000000}%
\pgfsetstrokecolor{currentstroke}%
\pgfsetdash{}{0pt}%
\pgfpathmoveto{\pgfqpoint{0.000000in}{0.000000in}}%
\pgfpathlineto{\pgfqpoint{5.208662in}{0.000000in}}%
\pgfpathlineto{\pgfqpoint{5.208662in}{3.219130in}}%
\pgfpathlineto{\pgfqpoint{0.000000in}{3.219130in}}%
\pgfpathlineto{\pgfqpoint{0.000000in}{0.000000in}}%
\pgfpathclose%
\pgfusepath{fill}%
\end{pgfscope}%
\begin{pgfscope}%
\pgfsetbuttcap%
\pgfsetmiterjoin%
\definecolor{currentfill}{rgb}{1.000000,1.000000,1.000000}%
\pgfsetfillcolor{currentfill}%
\pgfsetlinewidth{0.000000pt}%
\definecolor{currentstroke}{rgb}{0.000000,0.000000,0.000000}%
\pgfsetstrokecolor{currentstroke}%
\pgfsetstrokeopacity{0.000000}%
\pgfsetdash{}{0pt}%
\pgfpathmoveto{\pgfqpoint{0.667540in}{0.539544in}}%
\pgfpathlineto{\pgfqpoint{5.058662in}{0.539544in}}%
\pgfpathlineto{\pgfqpoint{5.058662in}{2.944887in}}%
\pgfpathlineto{\pgfqpoint{0.667540in}{2.944887in}}%
\pgfpathlineto{\pgfqpoint{0.667540in}{0.539544in}}%
\pgfpathclose%
\pgfusepath{fill}%
\end{pgfscope}%
\begin{pgfscope}%
\pgfsetbuttcap%
\pgfsetroundjoin%
\definecolor{currentfill}{rgb}{0.000000,0.000000,0.000000}%
\pgfsetfillcolor{currentfill}%
\pgfsetlinewidth{0.803000pt}%
\definecolor{currentstroke}{rgb}{0.000000,0.000000,0.000000}%
\pgfsetstrokecolor{currentstroke}%
\pgfsetdash{}{0pt}%
\pgfsys@defobject{currentmarker}{\pgfqpoint{0.000000in}{-0.048611in}}{\pgfqpoint{0.000000in}{0.000000in}}{%
\pgfpathmoveto{\pgfqpoint{0.000000in}{0.000000in}}%
\pgfpathlineto{\pgfqpoint{0.000000in}{-0.048611in}}%
\pgfusepath{stroke,fill}%
}%
\begin{pgfscope}%
\pgfsys@transformshift{0.866046in}{0.539544in}%
\pgfsys@useobject{currentmarker}{}%
\end{pgfscope}%
\end{pgfscope}%
\begin{pgfscope}%
\definecolor{textcolor}{rgb}{0.000000,0.000000,0.000000}%
\pgfsetstrokecolor{textcolor}%
\pgfsetfillcolor{textcolor}%
\pgftext[x=0.866046in,y=0.442322in,,top]{\color{textcolor}\rmfamily\fontsize{8.000000}{9.600000}\selectfont \(\displaystyle {06{:}45}\)}%
\end{pgfscope}%
\begin{pgfscope}%
\pgfsetbuttcap%
\pgfsetroundjoin%
\definecolor{currentfill}{rgb}{0.000000,0.000000,0.000000}%
\pgfsetfillcolor{currentfill}%
\pgfsetlinewidth{0.803000pt}%
\definecolor{currentstroke}{rgb}{0.000000,0.000000,0.000000}%
\pgfsetstrokecolor{currentstroke}%
\pgfsetdash{}{0pt}%
\pgfsys@defobject{currentmarker}{\pgfqpoint{0.000000in}{-0.048611in}}{\pgfqpoint{0.000000in}{0.000000in}}{%
\pgfpathmoveto{\pgfqpoint{0.000000in}{0.000000in}}%
\pgfpathlineto{\pgfqpoint{0.000000in}{-0.048611in}}%
\pgfusepath{stroke,fill}%
}%
\begin{pgfscope}%
\pgfsys@transformshift{2.197480in}{0.539544in}%
\pgfsys@useobject{currentmarker}{}%
\end{pgfscope}%
\end{pgfscope}%
\begin{pgfscope}%
\definecolor{textcolor}{rgb}{0.000000,0.000000,0.000000}%
\pgfsetstrokecolor{textcolor}%
\pgfsetfillcolor{textcolor}%
\pgftext[x=2.197480in,y=0.442322in,,top]{\color{textcolor}\rmfamily\fontsize{8.000000}{9.600000}\selectfont \(\displaystyle {06{:}50}\)}%
\end{pgfscope}%
\begin{pgfscope}%
\pgfsetbuttcap%
\pgfsetroundjoin%
\definecolor{currentfill}{rgb}{0.000000,0.000000,0.000000}%
\pgfsetfillcolor{currentfill}%
\pgfsetlinewidth{0.803000pt}%
\definecolor{currentstroke}{rgb}{0.000000,0.000000,0.000000}%
\pgfsetstrokecolor{currentstroke}%
\pgfsetdash{}{0pt}%
\pgfsys@defobject{currentmarker}{\pgfqpoint{0.000000in}{-0.048611in}}{\pgfqpoint{0.000000in}{0.000000in}}{%
\pgfpathmoveto{\pgfqpoint{0.000000in}{0.000000in}}%
\pgfpathlineto{\pgfqpoint{0.000000in}{-0.048611in}}%
\pgfusepath{stroke,fill}%
}%
\begin{pgfscope}%
\pgfsys@transformshift{3.528915in}{0.539544in}%
\pgfsys@useobject{currentmarker}{}%
\end{pgfscope}%
\end{pgfscope}%
\begin{pgfscope}%
\definecolor{textcolor}{rgb}{0.000000,0.000000,0.000000}%
\pgfsetstrokecolor{textcolor}%
\pgfsetfillcolor{textcolor}%
\pgftext[x=3.528915in,y=0.442322in,,top]{\color{textcolor}\rmfamily\fontsize{8.000000}{9.600000}\selectfont \(\displaystyle {06{:}55}\)}%
\end{pgfscope}%
\begin{pgfscope}%
\pgfsetbuttcap%
\pgfsetroundjoin%
\definecolor{currentfill}{rgb}{0.000000,0.000000,0.000000}%
\pgfsetfillcolor{currentfill}%
\pgfsetlinewidth{0.803000pt}%
\definecolor{currentstroke}{rgb}{0.000000,0.000000,0.000000}%
\pgfsetstrokecolor{currentstroke}%
\pgfsetdash{}{0pt}%
\pgfsys@defobject{currentmarker}{\pgfqpoint{0.000000in}{-0.048611in}}{\pgfqpoint{0.000000in}{0.000000in}}{%
\pgfpathmoveto{\pgfqpoint{0.000000in}{0.000000in}}%
\pgfpathlineto{\pgfqpoint{0.000000in}{-0.048611in}}%
\pgfusepath{stroke,fill}%
}%
\begin{pgfscope}%
\pgfsys@transformshift{4.860349in}{0.539544in}%
\pgfsys@useobject{currentmarker}{}%
\end{pgfscope}%
\end{pgfscope}%
\begin{pgfscope}%
\definecolor{textcolor}{rgb}{0.000000,0.000000,0.000000}%
\pgfsetstrokecolor{textcolor}%
\pgfsetfillcolor{textcolor}%
\pgftext[x=4.860349in,y=0.442322in,,top]{\color{textcolor}\rmfamily\fontsize{8.000000}{9.600000}\selectfont \(\displaystyle {07{:}00}\)}%
\end{pgfscope}%
\begin{pgfscope}%
\definecolor{textcolor}{rgb}{0.000000,0.000000,0.000000}%
\pgfsetstrokecolor{textcolor}%
\pgfsetfillcolor{textcolor}%
\pgftext[x=2.863101in,y=0.288100in,,top]{\color{textcolor}\rmfamily\fontsize{10.000000}{12.000000}\selectfont Time (UTC)}%
\end{pgfscope}%
\begin{pgfscope}%
\pgfsetbuttcap%
\pgfsetroundjoin%
\definecolor{currentfill}{rgb}{0.000000,0.000000,0.000000}%
\pgfsetfillcolor{currentfill}%
\pgfsetlinewidth{0.803000pt}%
\definecolor{currentstroke}{rgb}{0.000000,0.000000,0.000000}%
\pgfsetstrokecolor{currentstroke}%
\pgfsetdash{}{0pt}%
\pgfsys@defobject{currentmarker}{\pgfqpoint{-0.048611in}{0.000000in}}{\pgfqpoint{-0.000000in}{0.000000in}}{%
\pgfpathmoveto{\pgfqpoint{-0.000000in}{0.000000in}}%
\pgfpathlineto{\pgfqpoint{-0.048611in}{0.000000in}}%
\pgfusepath{stroke,fill}%
}%
\begin{pgfscope}%
\pgfsys@transformshift{0.667540in}{0.611455in}%
\pgfsys@useobject{currentmarker}{}%
\end{pgfscope}%
\end{pgfscope}%
\begin{pgfscope}%
\definecolor{textcolor}{rgb}{0.000000,0.000000,0.000000}%
\pgfsetstrokecolor{textcolor}%
\pgfsetfillcolor{textcolor}%
\pgftext[x=0.327644in, y=0.572899in, left, base]{\color{textcolor}\rmfamily\fontsize{8.000000}{9.600000}\selectfont \(\displaystyle {\ensuremath{-}7.5}\)}%
\end{pgfscope}%
\begin{pgfscope}%
\pgfsetbuttcap%
\pgfsetroundjoin%
\definecolor{currentfill}{rgb}{0.000000,0.000000,0.000000}%
\pgfsetfillcolor{currentfill}%
\pgfsetlinewidth{0.803000pt}%
\definecolor{currentstroke}{rgb}{0.000000,0.000000,0.000000}%
\pgfsetstrokecolor{currentstroke}%
\pgfsetdash{}{0pt}%
\pgfsys@defobject{currentmarker}{\pgfqpoint{-0.048611in}{0.000000in}}{\pgfqpoint{-0.000000in}{0.000000in}}{%
\pgfpathmoveto{\pgfqpoint{-0.000000in}{0.000000in}}%
\pgfpathlineto{\pgfqpoint{-0.048611in}{0.000000in}}%
\pgfusepath{stroke,fill}%
}%
\begin{pgfscope}%
\pgfsys@transformshift{0.667540in}{0.925106in}%
\pgfsys@useobject{currentmarker}{}%
\end{pgfscope}%
\end{pgfscope}%
\begin{pgfscope}%
\definecolor{textcolor}{rgb}{0.000000,0.000000,0.000000}%
\pgfsetstrokecolor{textcolor}%
\pgfsetfillcolor{textcolor}%
\pgftext[x=0.327644in, y=0.886551in, left, base]{\color{textcolor}\rmfamily\fontsize{8.000000}{9.600000}\selectfont \(\displaystyle {\ensuremath{-}5.0}\)}%
\end{pgfscope}%
\begin{pgfscope}%
\pgfsetbuttcap%
\pgfsetroundjoin%
\definecolor{currentfill}{rgb}{0.000000,0.000000,0.000000}%
\pgfsetfillcolor{currentfill}%
\pgfsetlinewidth{0.803000pt}%
\definecolor{currentstroke}{rgb}{0.000000,0.000000,0.000000}%
\pgfsetstrokecolor{currentstroke}%
\pgfsetdash{}{0pt}%
\pgfsys@defobject{currentmarker}{\pgfqpoint{-0.048611in}{0.000000in}}{\pgfqpoint{-0.000000in}{0.000000in}}{%
\pgfpathmoveto{\pgfqpoint{-0.000000in}{0.000000in}}%
\pgfpathlineto{\pgfqpoint{-0.048611in}{0.000000in}}%
\pgfusepath{stroke,fill}%
}%
\begin{pgfscope}%
\pgfsys@transformshift{0.667540in}{1.238757in}%
\pgfsys@useobject{currentmarker}{}%
\end{pgfscope}%
\end{pgfscope}%
\begin{pgfscope}%
\definecolor{textcolor}{rgb}{0.000000,0.000000,0.000000}%
\pgfsetstrokecolor{textcolor}%
\pgfsetfillcolor{textcolor}%
\pgftext[x=0.327644in, y=1.200202in, left, base]{\color{textcolor}\rmfamily\fontsize{8.000000}{9.600000}\selectfont \(\displaystyle {\ensuremath{-}2.5}\)}%
\end{pgfscope}%
\begin{pgfscope}%
\pgfsetbuttcap%
\pgfsetroundjoin%
\definecolor{currentfill}{rgb}{0.000000,0.000000,0.000000}%
\pgfsetfillcolor{currentfill}%
\pgfsetlinewidth{0.803000pt}%
\definecolor{currentstroke}{rgb}{0.000000,0.000000,0.000000}%
\pgfsetstrokecolor{currentstroke}%
\pgfsetdash{}{0pt}%
\pgfsys@defobject{currentmarker}{\pgfqpoint{-0.048611in}{0.000000in}}{\pgfqpoint{-0.000000in}{0.000000in}}{%
\pgfpathmoveto{\pgfqpoint{-0.000000in}{0.000000in}}%
\pgfpathlineto{\pgfqpoint{-0.048611in}{0.000000in}}%
\pgfusepath{stroke,fill}%
}%
\begin{pgfscope}%
\pgfsys@transformshift{0.667540in}{1.552408in}%
\pgfsys@useobject{currentmarker}{}%
\end{pgfscope}%
\end{pgfscope}%
\begin{pgfscope}%
\definecolor{textcolor}{rgb}{0.000000,0.000000,0.000000}%
\pgfsetstrokecolor{textcolor}%
\pgfsetfillcolor{textcolor}%
\pgftext[x=0.419467in, y=1.513853in, left, base]{\color{textcolor}\rmfamily\fontsize{8.000000}{9.600000}\selectfont \(\displaystyle {0.0}\)}%
\end{pgfscope}%
\begin{pgfscope}%
\pgfsetbuttcap%
\pgfsetroundjoin%
\definecolor{currentfill}{rgb}{0.000000,0.000000,0.000000}%
\pgfsetfillcolor{currentfill}%
\pgfsetlinewidth{0.803000pt}%
\definecolor{currentstroke}{rgb}{0.000000,0.000000,0.000000}%
\pgfsetstrokecolor{currentstroke}%
\pgfsetdash{}{0pt}%
\pgfsys@defobject{currentmarker}{\pgfqpoint{-0.048611in}{0.000000in}}{\pgfqpoint{-0.000000in}{0.000000in}}{%
\pgfpathmoveto{\pgfqpoint{-0.000000in}{0.000000in}}%
\pgfpathlineto{\pgfqpoint{-0.048611in}{0.000000in}}%
\pgfusepath{stroke,fill}%
}%
\begin{pgfscope}%
\pgfsys@transformshift{0.667540in}{1.866059in}%
\pgfsys@useobject{currentmarker}{}%
\end{pgfscope}%
\end{pgfscope}%
\begin{pgfscope}%
\definecolor{textcolor}{rgb}{0.000000,0.000000,0.000000}%
\pgfsetstrokecolor{textcolor}%
\pgfsetfillcolor{textcolor}%
\pgftext[x=0.419467in, y=1.827504in, left, base]{\color{textcolor}\rmfamily\fontsize{8.000000}{9.600000}\selectfont \(\displaystyle {2.5}\)}%
\end{pgfscope}%
\begin{pgfscope}%
\pgfsetbuttcap%
\pgfsetroundjoin%
\definecolor{currentfill}{rgb}{0.000000,0.000000,0.000000}%
\pgfsetfillcolor{currentfill}%
\pgfsetlinewidth{0.803000pt}%
\definecolor{currentstroke}{rgb}{0.000000,0.000000,0.000000}%
\pgfsetstrokecolor{currentstroke}%
\pgfsetdash{}{0pt}%
\pgfsys@defobject{currentmarker}{\pgfqpoint{-0.048611in}{0.000000in}}{\pgfqpoint{-0.000000in}{0.000000in}}{%
\pgfpathmoveto{\pgfqpoint{-0.000000in}{0.000000in}}%
\pgfpathlineto{\pgfqpoint{-0.048611in}{0.000000in}}%
\pgfusepath{stroke,fill}%
}%
\begin{pgfscope}%
\pgfsys@transformshift{0.667540in}{2.179710in}%
\pgfsys@useobject{currentmarker}{}%
\end{pgfscope}%
\end{pgfscope}%
\begin{pgfscope}%
\definecolor{textcolor}{rgb}{0.000000,0.000000,0.000000}%
\pgfsetstrokecolor{textcolor}%
\pgfsetfillcolor{textcolor}%
\pgftext[x=0.419467in, y=2.141155in, left, base]{\color{textcolor}\rmfamily\fontsize{8.000000}{9.600000}\selectfont \(\displaystyle {5.0}\)}%
\end{pgfscope}%
\begin{pgfscope}%
\pgfsetbuttcap%
\pgfsetroundjoin%
\definecolor{currentfill}{rgb}{0.000000,0.000000,0.000000}%
\pgfsetfillcolor{currentfill}%
\pgfsetlinewidth{0.803000pt}%
\definecolor{currentstroke}{rgb}{0.000000,0.000000,0.000000}%
\pgfsetstrokecolor{currentstroke}%
\pgfsetdash{}{0pt}%
\pgfsys@defobject{currentmarker}{\pgfqpoint{-0.048611in}{0.000000in}}{\pgfqpoint{-0.000000in}{0.000000in}}{%
\pgfpathmoveto{\pgfqpoint{-0.000000in}{0.000000in}}%
\pgfpathlineto{\pgfqpoint{-0.048611in}{0.000000in}}%
\pgfusepath{stroke,fill}%
}%
\begin{pgfscope}%
\pgfsys@transformshift{0.667540in}{2.493362in}%
\pgfsys@useobject{currentmarker}{}%
\end{pgfscope}%
\end{pgfscope}%
\begin{pgfscope}%
\definecolor{textcolor}{rgb}{0.000000,0.000000,0.000000}%
\pgfsetstrokecolor{textcolor}%
\pgfsetfillcolor{textcolor}%
\pgftext[x=0.419467in, y=2.454806in, left, base]{\color{textcolor}\rmfamily\fontsize{8.000000}{9.600000}\selectfont \(\displaystyle {7.5}\)}%
\end{pgfscope}%
\begin{pgfscope}%
\pgfsetbuttcap%
\pgfsetroundjoin%
\definecolor{currentfill}{rgb}{0.000000,0.000000,0.000000}%
\pgfsetfillcolor{currentfill}%
\pgfsetlinewidth{0.803000pt}%
\definecolor{currentstroke}{rgb}{0.000000,0.000000,0.000000}%
\pgfsetstrokecolor{currentstroke}%
\pgfsetdash{}{0pt}%
\pgfsys@defobject{currentmarker}{\pgfqpoint{-0.048611in}{0.000000in}}{\pgfqpoint{-0.000000in}{0.000000in}}{%
\pgfpathmoveto{\pgfqpoint{-0.000000in}{0.000000in}}%
\pgfpathlineto{\pgfqpoint{-0.048611in}{0.000000in}}%
\pgfusepath{stroke,fill}%
}%
\begin{pgfscope}%
\pgfsys@transformshift{0.667540in}{2.807013in}%
\pgfsys@useobject{currentmarker}{}%
\end{pgfscope}%
\end{pgfscope}%
\begin{pgfscope}%
\definecolor{textcolor}{rgb}{0.000000,0.000000,0.000000}%
\pgfsetstrokecolor{textcolor}%
\pgfsetfillcolor{textcolor}%
\pgftext[x=0.360438in, y=2.768457in, left, base]{\color{textcolor}\rmfamily\fontsize{8.000000}{9.600000}\selectfont \(\displaystyle {10.0}\)}%
\end{pgfscope}%
\begin{pgfscope}%
\definecolor{textcolor}{rgb}{0.000000,0.000000,0.000000}%
\pgfsetstrokecolor{textcolor}%
\pgfsetfillcolor{textcolor}%
\pgftext[x=0.272089in,y=1.742216in,,bottom,rotate=90.000000]{\color{textcolor}\rmfamily\fontsize{10.000000}{12.000000}\selectfont Voltage deviation in V}%
\end{pgfscope}%
\begin{pgfscope}%
\definecolor{textcolor}{rgb}{0.000000,0.000000,0.000000}%
\pgfsetstrokecolor{textcolor}%
\pgfsetfillcolor{textcolor}%
\pgftext[x=0.667540in,y=2.986554in,left,base]{\color{textcolor}\rmfamily\fontsize{8.000000}{9.600000}\selectfont \(\displaystyle \times{10^{\ensuremath{-}6}}{}\)}%
\end{pgfscope}%
\begin{pgfscope}%
\pgfpathrectangle{\pgfqpoint{0.667540in}{0.539544in}}{\pgfqpoint{4.391122in}{2.405343in}}%
\pgfusepath{clip}%
\pgfsetrectcap%
\pgfsetroundjoin%
\pgfsetlinewidth{0.501875pt}%
\definecolor{currentstroke}{rgb}{0.121569,0.466667,0.705882}%
\pgfsetstrokecolor{currentstroke}%
\pgfsetstrokeopacity{0.700000}%
\pgfsetdash{}{0pt}%
\pgfpathmoveto{\pgfqpoint{0.867136in}{1.087387in}}%
\pgfpathlineto{\pgfqpoint{0.870808in}{1.118966in}}%
\pgfpathlineto{\pgfqpoint{0.872643in}{1.181144in}}%
\pgfpathlineto{\pgfqpoint{0.874481in}{0.958565in}}%
\pgfpathlineto{\pgfqpoint{0.876316in}{1.101426in}}%
\pgfpathlineto{\pgfqpoint{0.878153in}{0.951576in}}%
\pgfpathlineto{\pgfqpoint{0.881823in}{1.075920in}}%
\pgfpathlineto{\pgfqpoint{0.883659in}{1.252581in}}%
\pgfpathlineto{\pgfqpoint{0.885494in}{1.091590in}}%
\pgfpathlineto{\pgfqpoint{0.887330in}{1.123470in}}%
\pgfpathlineto{\pgfqpoint{0.889165in}{1.181997in}}%
\pgfpathlineto{\pgfqpoint{0.892838in}{1.123909in}}%
\pgfpathlineto{\pgfqpoint{0.894674in}{1.172450in}}%
\pgfpathlineto{\pgfqpoint{0.896510in}{1.289040in}}%
\pgfpathlineto{\pgfqpoint{0.898346in}{1.152940in}}%
\pgfpathlineto{\pgfqpoint{0.902018in}{1.215583in}}%
\pgfpathlineto{\pgfqpoint{0.905690in}{1.384741in}}%
\pgfpathlineto{\pgfqpoint{0.909363in}{1.223286in}}%
\pgfpathlineto{\pgfqpoint{0.911199in}{1.421614in}}%
\pgfpathlineto{\pgfqpoint{0.913035in}{1.319953in}}%
\pgfpathlineto{\pgfqpoint{0.914871in}{1.291248in}}%
\pgfpathlineto{\pgfqpoint{0.918543in}{1.115240in}}%
\pgfpathlineto{\pgfqpoint{0.920378in}{1.088742in}}%
\pgfpathlineto{\pgfqpoint{0.922213in}{1.270585in}}%
\pgfpathlineto{\pgfqpoint{0.924048in}{1.272203in}}%
\pgfpathlineto{\pgfqpoint{0.925883in}{1.289065in}}%
\pgfpathlineto{\pgfqpoint{0.929556in}{1.198307in}}%
\pgfpathlineto{\pgfqpoint{0.931392in}{1.190227in}}%
\pgfpathlineto{\pgfqpoint{0.933228in}{1.146442in}}%
\pgfpathlineto{\pgfqpoint{0.935065in}{1.283369in}}%
\pgfpathlineto{\pgfqpoint{0.938735in}{1.029588in}}%
\pgfpathlineto{\pgfqpoint{0.946079in}{1.626290in}}%
\pgfpathlineto{\pgfqpoint{0.949753in}{1.444096in}}%
\pgfpathlineto{\pgfqpoint{0.951589in}{1.616956in}}%
\pgfpathlineto{\pgfqpoint{0.953425in}{1.532446in}}%
\pgfpathlineto{\pgfqpoint{0.955260in}{1.652950in}}%
\pgfpathlineto{\pgfqpoint{0.957096in}{1.510703in}}%
\pgfpathlineto{\pgfqpoint{0.958931in}{1.617922in}}%
\pgfpathlineto{\pgfqpoint{0.960766in}{1.610194in}}%
\pgfpathlineto{\pgfqpoint{0.962600in}{1.313542in}}%
\pgfpathlineto{\pgfqpoint{0.964437in}{1.329589in}}%
\pgfpathlineto{\pgfqpoint{0.966272in}{1.506526in}}%
\pgfpathlineto{\pgfqpoint{0.968108in}{1.501068in}}%
\pgfpathlineto{\pgfqpoint{0.969944in}{1.597183in}}%
\pgfpathlineto{\pgfqpoint{0.971780in}{1.593921in}}%
\pgfpathlineto{\pgfqpoint{0.973616in}{1.657530in}}%
\pgfpathlineto{\pgfqpoint{0.975451in}{1.470130in}}%
\pgfpathlineto{\pgfqpoint{0.979123in}{1.384151in}}%
\pgfpathlineto{\pgfqpoint{0.980959in}{1.395556in}}%
\pgfpathlineto{\pgfqpoint{0.982794in}{1.581990in}}%
\pgfpathlineto{\pgfqpoint{0.984630in}{1.471886in}}%
\pgfpathlineto{\pgfqpoint{0.986466in}{1.237476in}}%
\pgfpathlineto{\pgfqpoint{0.990138in}{1.537000in}}%
\pgfpathlineto{\pgfqpoint{0.991973in}{1.372346in}}%
\pgfpathlineto{\pgfqpoint{0.995643in}{1.405605in}}%
\pgfpathlineto{\pgfqpoint{0.997478in}{1.710361in}}%
\pgfpathlineto{\pgfqpoint{0.999315in}{1.353740in}}%
\pgfpathlineto{\pgfqpoint{1.001150in}{1.559683in}}%
\pgfpathlineto{\pgfqpoint{1.002986in}{1.584449in}}%
\pgfpathlineto{\pgfqpoint{1.004822in}{1.532120in}}%
\pgfpathlineto{\pgfqpoint{1.006658in}{1.099607in}}%
\pgfpathlineto{\pgfqpoint{1.008497in}{1.116206in}}%
\pgfpathlineto{\pgfqpoint{1.010333in}{1.363701in}}%
\pgfpathlineto{\pgfqpoint{1.012168in}{1.297822in}}%
\pgfpathlineto{\pgfqpoint{1.014005in}{1.069810in}}%
\pgfpathlineto{\pgfqpoint{1.017676in}{1.366574in}}%
\pgfpathlineto{\pgfqpoint{1.019512in}{1.219133in}}%
\pgfpathlineto{\pgfqpoint{1.021348in}{1.249871in}}%
\pgfpathlineto{\pgfqpoint{1.023183in}{1.354882in}}%
\pgfpathlineto{\pgfqpoint{1.025021in}{1.274637in}}%
\pgfpathlineto{\pgfqpoint{1.026856in}{1.591236in}}%
\pgfpathlineto{\pgfqpoint{1.028692in}{1.362409in}}%
\pgfpathlineto{\pgfqpoint{1.030527in}{1.339613in}}%
\pgfpathlineto{\pgfqpoint{1.034199in}{1.115679in}}%
\pgfpathlineto{\pgfqpoint{1.036034in}{1.211944in}}%
\pgfpathlineto{\pgfqpoint{1.037869in}{1.103158in}}%
\pgfpathlineto{\pgfqpoint{1.039706in}{1.258302in}}%
\pgfpathlineto{\pgfqpoint{1.041541in}{1.234803in}}%
\pgfpathlineto{\pgfqpoint{1.043377in}{1.057089in}}%
\pgfpathlineto{\pgfqpoint{1.048886in}{2.234585in}}%
\pgfpathlineto{\pgfqpoint{1.050723in}{2.532253in}}%
\pgfpathlineto{\pgfqpoint{1.052559in}{2.137178in}}%
\pgfpathlineto{\pgfqpoint{1.054395in}{1.468800in}}%
\pgfpathlineto{\pgfqpoint{1.056245in}{1.494895in}}%
\pgfpathlineto{\pgfqpoint{1.058082in}{1.229421in}}%
\pgfpathlineto{\pgfqpoint{1.061752in}{1.352648in}}%
\pgfpathlineto{\pgfqpoint{1.063588in}{1.260711in}}%
\pgfpathlineto{\pgfqpoint{1.065424in}{1.255956in}}%
\pgfpathlineto{\pgfqpoint{1.067259in}{1.191256in}}%
\pgfpathlineto{\pgfqpoint{1.069094in}{1.201155in}}%
\pgfpathlineto{\pgfqpoint{1.070929in}{1.157846in}}%
\pgfpathlineto{\pgfqpoint{1.072765in}{1.334406in}}%
\pgfpathlineto{\pgfqpoint{1.074601in}{1.051305in}}%
\pgfpathlineto{\pgfqpoint{1.076437in}{1.210828in}}%
\pgfpathlineto{\pgfqpoint{1.078273in}{1.206675in}}%
\pgfpathlineto{\pgfqpoint{1.080109in}{1.457370in}}%
\pgfpathlineto{\pgfqpoint{1.081946in}{2.094308in}}%
\pgfpathlineto{\pgfqpoint{1.083781in}{2.144203in}}%
\pgfpathlineto{\pgfqpoint{1.085616in}{2.054763in}}%
\pgfpathlineto{\pgfqpoint{1.087454in}{2.151279in}}%
\pgfpathlineto{\pgfqpoint{1.089290in}{2.347499in}}%
\pgfpathlineto{\pgfqpoint{1.091125in}{2.151543in}}%
\pgfpathlineto{\pgfqpoint{1.092961in}{2.094546in}}%
\pgfpathlineto{\pgfqpoint{1.094797in}{2.186985in}}%
\pgfpathlineto{\pgfqpoint{1.096632in}{2.073795in}}%
\pgfpathlineto{\pgfqpoint{1.098468in}{2.131156in}}%
\pgfpathlineto{\pgfqpoint{1.100303in}{2.013587in}}%
\pgfpathlineto{\pgfqpoint{1.102138in}{2.176083in}}%
\pgfpathlineto{\pgfqpoint{1.105808in}{2.148871in}}%
\pgfpathlineto{\pgfqpoint{1.107646in}{2.056369in}}%
\pgfpathlineto{\pgfqpoint{1.111317in}{2.206846in}}%
\pgfpathlineto{\pgfqpoint{1.113154in}{2.051614in}}%
\pgfpathlineto{\pgfqpoint{1.114989in}{2.200761in}}%
\pgfpathlineto{\pgfqpoint{1.116824in}{2.146725in}}%
\pgfpathlineto{\pgfqpoint{1.118661in}{2.019144in}}%
\pgfpathlineto{\pgfqpoint{1.120497in}{2.141468in}}%
\pgfpathlineto{\pgfqpoint{1.124169in}{1.269732in}}%
\pgfpathlineto{\pgfqpoint{1.126004in}{1.341721in}}%
\pgfpathlineto{\pgfqpoint{1.127845in}{1.286606in}}%
\pgfpathlineto{\pgfqpoint{1.129681in}{1.146241in}}%
\pgfpathlineto{\pgfqpoint{1.133355in}{1.080562in}}%
\pgfpathlineto{\pgfqpoint{1.135191in}{1.274085in}}%
\pgfpathlineto{\pgfqpoint{1.137027in}{1.059849in}}%
\pgfpathlineto{\pgfqpoint{1.138862in}{1.087538in}}%
\pgfpathlineto{\pgfqpoint{1.140696in}{1.174469in}}%
\pgfpathlineto{\pgfqpoint{1.142533in}{1.206085in}}%
\pgfpathlineto{\pgfqpoint{1.144368in}{1.114060in}}%
\pgfpathlineto{\pgfqpoint{1.146205in}{1.310669in}}%
\pgfpathlineto{\pgfqpoint{1.148041in}{1.983175in}}%
\pgfpathlineto{\pgfqpoint{1.149878in}{2.057134in}}%
\pgfpathlineto{\pgfqpoint{1.151713in}{1.299027in}}%
\pgfpathlineto{\pgfqpoint{1.153550in}{2.024163in}}%
\pgfpathlineto{\pgfqpoint{1.155386in}{1.916505in}}%
\pgfpathlineto{\pgfqpoint{1.157223in}{1.936228in}}%
\pgfpathlineto{\pgfqpoint{1.159058in}{2.108121in}}%
\pgfpathlineto{\pgfqpoint{1.160894in}{2.131682in}}%
\pgfpathlineto{\pgfqpoint{1.162731in}{2.048352in}}%
\pgfpathlineto{\pgfqpoint{1.166405in}{2.197474in}}%
\pgfpathlineto{\pgfqpoint{1.168241in}{2.119337in}}%
\pgfpathlineto{\pgfqpoint{1.170076in}{2.177965in}}%
\pgfpathlineto{\pgfqpoint{1.171914in}{2.155921in}}%
\pgfpathlineto{\pgfqpoint{1.173748in}{2.047235in}}%
\pgfpathlineto{\pgfqpoint{1.175582in}{2.072151in}}%
\pgfpathlineto{\pgfqpoint{1.177418in}{2.185630in}}%
\pgfpathlineto{\pgfqpoint{1.181087in}{1.964230in}}%
\pgfpathlineto{\pgfqpoint{1.186596in}{2.183071in}}%
\pgfpathlineto{\pgfqpoint{1.190268in}{1.984066in}}%
\pgfpathlineto{\pgfqpoint{1.192105in}{1.972047in}}%
\pgfpathlineto{\pgfqpoint{1.193940in}{1.421413in}}%
\pgfpathlineto{\pgfqpoint{1.197612in}{1.164069in}}%
\pgfpathlineto{\pgfqpoint{1.199448in}{1.245317in}}%
\pgfpathlineto{\pgfqpoint{1.201285in}{1.047541in}}%
\pgfpathlineto{\pgfqpoint{1.203121in}{1.305262in}}%
\pgfpathlineto{\pgfqpoint{1.204957in}{1.405982in}}%
\pgfpathlineto{\pgfqpoint{1.208632in}{1.280609in}}%
\pgfpathlineto{\pgfqpoint{1.210467in}{1.242118in}}%
\pgfpathlineto{\pgfqpoint{1.212302in}{1.359611in}}%
\pgfpathlineto{\pgfqpoint{1.214137in}{1.275666in}}%
\pgfpathlineto{\pgfqpoint{1.215972in}{1.375043in}}%
\pgfpathlineto{\pgfqpoint{1.217807in}{1.207415in}}%
\pgfpathlineto{\pgfqpoint{1.221479in}{1.316428in}}%
\pgfpathlineto{\pgfqpoint{1.223316in}{1.354831in}}%
\pgfpathlineto{\pgfqpoint{1.225150in}{1.049398in}}%
\pgfpathlineto{\pgfqpoint{1.226988in}{1.149678in}}%
\pgfpathlineto{\pgfqpoint{1.228824in}{1.126845in}}%
\pgfpathlineto{\pgfqpoint{1.230659in}{1.174971in}}%
\pgfpathlineto{\pgfqpoint{1.232495in}{1.334820in}}%
\pgfpathlineto{\pgfqpoint{1.234331in}{1.203852in}}%
\pgfpathlineto{\pgfqpoint{1.236168in}{1.285715in}}%
\pgfpathlineto{\pgfqpoint{1.238004in}{1.260862in}}%
\pgfpathlineto{\pgfqpoint{1.239840in}{1.427523in}}%
\pgfpathlineto{\pgfqpoint{1.241677in}{1.373400in}}%
\pgfpathlineto{\pgfqpoint{1.243513in}{1.231328in}}%
\pgfpathlineto{\pgfqpoint{1.245348in}{1.389910in}}%
\pgfpathlineto{\pgfqpoint{1.249019in}{1.220689in}}%
\pgfpathlineto{\pgfqpoint{1.250854in}{1.366261in}}%
\pgfpathlineto{\pgfqpoint{1.252690in}{1.261564in}}%
\pgfpathlineto{\pgfqpoint{1.256359in}{1.400549in}}%
\pgfpathlineto{\pgfqpoint{1.258196in}{1.241904in}}%
\pgfpathlineto{\pgfqpoint{1.260032in}{1.239847in}}%
\pgfpathlineto{\pgfqpoint{1.261867in}{1.542972in}}%
\pgfpathlineto{\pgfqpoint{1.263704in}{2.158004in}}%
\pgfpathlineto{\pgfqpoint{1.265540in}{2.350322in}}%
\pgfpathlineto{\pgfqpoint{1.269213in}{2.139147in}}%
\pgfpathlineto{\pgfqpoint{1.271049in}{2.259288in}}%
\pgfpathlineto{\pgfqpoint{1.272886in}{2.184689in}}%
\pgfpathlineto{\pgfqpoint{1.274721in}{2.231022in}}%
\pgfpathlineto{\pgfqpoint{1.276557in}{2.170939in}}%
\pgfpathlineto{\pgfqpoint{1.278395in}{2.312145in}}%
\pgfpathlineto{\pgfqpoint{1.280229in}{2.275749in}}%
\pgfpathlineto{\pgfqpoint{1.283924in}{2.348616in}}%
\pgfpathlineto{\pgfqpoint{1.285760in}{2.339119in}}%
\pgfpathlineto{\pgfqpoint{1.287594in}{2.442611in}}%
\pgfpathlineto{\pgfqpoint{1.289430in}{2.232402in}}%
\pgfpathlineto{\pgfqpoint{1.291265in}{2.217159in}}%
\pgfpathlineto{\pgfqpoint{1.294938in}{2.254496in}}%
\pgfpathlineto{\pgfqpoint{1.296774in}{2.273478in}}%
\pgfpathlineto{\pgfqpoint{1.298608in}{2.227710in}}%
\pgfpathlineto{\pgfqpoint{1.300446in}{2.217096in}}%
\pgfpathlineto{\pgfqpoint{1.302282in}{2.156800in}}%
\pgfpathlineto{\pgfqpoint{1.304117in}{1.349863in}}%
\pgfpathlineto{\pgfqpoint{1.305953in}{1.111012in}}%
\pgfpathlineto{\pgfqpoint{1.307790in}{1.240211in}}%
\pgfpathlineto{\pgfqpoint{1.309627in}{1.209059in}}%
\pgfpathlineto{\pgfqpoint{1.311463in}{1.149503in}}%
\pgfpathlineto{\pgfqpoint{1.315135in}{1.233423in}}%
\pgfpathlineto{\pgfqpoint{1.316970in}{1.146203in}}%
\pgfpathlineto{\pgfqpoint{1.318807in}{1.237614in}}%
\pgfpathlineto{\pgfqpoint{1.320643in}{1.170279in}}%
\pgfpathlineto{\pgfqpoint{1.322477in}{1.275264in}}%
\pgfpathlineto{\pgfqpoint{1.326148in}{1.116733in}}%
\pgfpathlineto{\pgfqpoint{1.327983in}{1.249369in}}%
\pgfpathlineto{\pgfqpoint{1.329819in}{1.117410in}}%
\pgfpathlineto{\pgfqpoint{1.331656in}{1.345899in}}%
\pgfpathlineto{\pgfqpoint{1.333492in}{1.277786in}}%
\pgfpathlineto{\pgfqpoint{1.335329in}{1.164044in}}%
\pgfpathlineto{\pgfqpoint{1.337164in}{1.195396in}}%
\pgfpathlineto{\pgfqpoint{1.339000in}{1.120935in}}%
\pgfpathlineto{\pgfqpoint{1.342670in}{1.215344in}}%
\pgfpathlineto{\pgfqpoint{1.344507in}{1.183641in}}%
\pgfpathlineto{\pgfqpoint{1.347225in}{1.035560in}}%
\pgfpathlineto{\pgfqpoint{1.349060in}{1.134811in}}%
\pgfpathlineto{\pgfqpoint{1.350897in}{1.093146in}}%
\pgfpathlineto{\pgfqpoint{1.352733in}{1.092556in}}%
\pgfpathlineto{\pgfqpoint{1.354568in}{1.137546in}}%
\pgfpathlineto{\pgfqpoint{1.358240in}{0.961801in}}%
\pgfpathlineto{\pgfqpoint{1.360072in}{1.107900in}}%
\pgfpathlineto{\pgfqpoint{1.361907in}{1.141072in}}%
\pgfpathlineto{\pgfqpoint{1.363739in}{1.103434in}}%
\pgfpathlineto{\pgfqpoint{1.365572in}{1.188345in}}%
\pgfpathlineto{\pgfqpoint{1.367406in}{1.130960in}}%
\pgfpathlineto{\pgfqpoint{1.369240in}{1.462226in}}%
\pgfpathlineto{\pgfqpoint{1.372906in}{1.214930in}}%
\pgfpathlineto{\pgfqpoint{1.374739in}{1.296116in}}%
\pgfpathlineto{\pgfqpoint{1.376572in}{1.258653in}}%
\pgfpathlineto{\pgfqpoint{1.378404in}{1.542846in}}%
\pgfpathlineto{\pgfqpoint{1.383907in}{1.135514in}}%
\pgfpathlineto{\pgfqpoint{1.385743in}{0.977346in}}%
\pgfpathlineto{\pgfqpoint{1.387576in}{1.130307in}}%
\pgfpathlineto{\pgfqpoint{1.389409in}{1.049160in}}%
\pgfpathlineto{\pgfqpoint{1.391244in}{1.216725in}}%
\pgfpathlineto{\pgfqpoint{1.393077in}{1.149152in}}%
\pgfpathlineto{\pgfqpoint{1.396744in}{1.240123in}}%
\pgfpathlineto{\pgfqpoint{1.398578in}{1.367377in}}%
\pgfpathlineto{\pgfqpoint{1.402245in}{1.101690in}}%
\pgfpathlineto{\pgfqpoint{1.404078in}{1.141423in}}%
\pgfpathlineto{\pgfqpoint{1.405913in}{1.019538in}}%
\pgfpathlineto{\pgfqpoint{1.407746in}{1.120622in}}%
\pgfpathlineto{\pgfqpoint{1.409581in}{1.145764in}}%
\pgfpathlineto{\pgfqpoint{1.413248in}{0.967886in}}%
\pgfpathlineto{\pgfqpoint{1.415082in}{1.012161in}}%
\pgfpathlineto{\pgfqpoint{1.416914in}{0.978927in}}%
\pgfpathlineto{\pgfqpoint{1.420582in}{1.154847in}}%
\pgfpathlineto{\pgfqpoint{1.424249in}{1.255780in}}%
\pgfpathlineto{\pgfqpoint{1.426084in}{1.173892in}}%
\pgfpathlineto{\pgfqpoint{1.427917in}{1.306793in}}%
\pgfpathlineto{\pgfqpoint{1.429749in}{1.031294in}}%
\pgfpathlineto{\pgfqpoint{1.431585in}{1.137459in}}%
\pgfpathlineto{\pgfqpoint{1.433419in}{1.176301in}}%
\pgfpathlineto{\pgfqpoint{1.437087in}{1.023239in}}%
\pgfpathlineto{\pgfqpoint{1.438920in}{1.168046in}}%
\pgfpathlineto{\pgfqpoint{1.440752in}{1.202008in}}%
\pgfpathlineto{\pgfqpoint{1.442587in}{1.384064in}}%
\pgfpathlineto{\pgfqpoint{1.448088in}{1.099670in}}%
\pgfpathlineto{\pgfqpoint{1.449922in}{1.246283in}}%
\pgfpathlineto{\pgfqpoint{1.453590in}{1.226661in}}%
\pgfpathlineto{\pgfqpoint{1.455424in}{1.263183in}}%
\pgfpathlineto{\pgfqpoint{1.457277in}{1.175297in}}%
\pgfpathlineto{\pgfqpoint{1.459481in}{1.229948in}}%
\pgfpathlineto{\pgfqpoint{1.461314in}{1.053388in}}%
\pgfpathlineto{\pgfqpoint{1.463148in}{1.119819in}}%
\pgfpathlineto{\pgfqpoint{1.464981in}{0.999415in}}%
\pgfpathlineto{\pgfqpoint{1.466813in}{1.487932in}}%
\pgfpathlineto{\pgfqpoint{1.472313in}{1.085242in}}%
\pgfpathlineto{\pgfqpoint{1.474146in}{1.262216in}}%
\pgfpathlineto{\pgfqpoint{1.475980in}{1.228681in}}%
\pgfpathlineto{\pgfqpoint{1.477812in}{1.015599in}}%
\pgfpathlineto{\pgfqpoint{1.479648in}{2.006573in}}%
\pgfpathlineto{\pgfqpoint{1.481481in}{1.754410in}}%
\pgfpathlineto{\pgfqpoint{1.483314in}{0.936647in}}%
\pgfpathlineto{\pgfqpoint{1.486984in}{1.164922in}}%
\pgfpathlineto{\pgfqpoint{1.488817in}{1.159377in}}%
\pgfpathlineto{\pgfqpoint{1.490650in}{1.092908in}}%
\pgfpathlineto{\pgfqpoint{1.492484in}{1.070902in}}%
\pgfpathlineto{\pgfqpoint{1.494318in}{0.980257in}}%
\pgfpathlineto{\pgfqpoint{1.496151in}{1.178233in}}%
\pgfpathlineto{\pgfqpoint{1.497985in}{1.212873in}}%
\pgfpathlineto{\pgfqpoint{1.499818in}{1.077953in}}%
\pgfpathlineto{\pgfqpoint{1.503486in}{1.267034in}}%
\pgfpathlineto{\pgfqpoint{1.505321in}{1.082269in}}%
\pgfpathlineto{\pgfqpoint{1.507154in}{1.092494in}}%
\pgfpathlineto{\pgfqpoint{1.508989in}{1.132515in}}%
\pgfpathlineto{\pgfqpoint{1.510822in}{0.911943in}}%
\pgfpathlineto{\pgfqpoint{1.512654in}{1.099319in}}%
\pgfpathlineto{\pgfqpoint{1.514489in}{1.161133in}}%
\pgfpathlineto{\pgfqpoint{1.516322in}{1.102781in}}%
\pgfpathlineto{\pgfqpoint{1.518156in}{0.998825in}}%
\pgfpathlineto{\pgfqpoint{1.523657in}{1.369636in}}%
\pgfpathlineto{\pgfqpoint{1.525491in}{1.289366in}}%
\pgfpathlineto{\pgfqpoint{1.527325in}{1.364379in}}%
\pgfpathlineto{\pgfqpoint{1.529158in}{1.253961in}}%
\pgfpathlineto{\pgfqpoint{1.531015in}{1.482964in}}%
\pgfpathlineto{\pgfqpoint{1.532848in}{1.217289in}}%
\pgfpathlineto{\pgfqpoint{1.534681in}{1.226310in}}%
\pgfpathlineto{\pgfqpoint{1.536516in}{1.525370in}}%
\pgfpathlineto{\pgfqpoint{1.540182in}{1.090762in}}%
\pgfpathlineto{\pgfqpoint{1.542016in}{1.211468in}}%
\pgfpathlineto{\pgfqpoint{1.543849in}{1.068230in}}%
\pgfpathlineto{\pgfqpoint{1.547517in}{1.425528in}}%
\pgfpathlineto{\pgfqpoint{1.549349in}{1.303756in}}%
\pgfpathlineto{\pgfqpoint{1.551181in}{1.381981in}}%
\pgfpathlineto{\pgfqpoint{1.553015in}{1.334469in}}%
\pgfpathlineto{\pgfqpoint{1.554849in}{1.321509in}}%
\pgfpathlineto{\pgfqpoint{1.556681in}{1.274988in}}%
\pgfpathlineto{\pgfqpoint{1.558518in}{1.349713in}}%
\pgfpathlineto{\pgfqpoint{1.560351in}{1.174796in}}%
\pgfpathlineto{\pgfqpoint{1.562183in}{1.204392in}}%
\pgfpathlineto{\pgfqpoint{1.564017in}{1.295238in}}%
\pgfpathlineto{\pgfqpoint{1.565850in}{1.920482in}}%
\pgfpathlineto{\pgfqpoint{1.567684in}{2.088135in}}%
\pgfpathlineto{\pgfqpoint{1.569520in}{2.135358in}}%
\pgfpathlineto{\pgfqpoint{1.571352in}{2.117430in}}%
\pgfpathlineto{\pgfqpoint{1.575017in}{1.991731in}}%
\pgfpathlineto{\pgfqpoint{1.578712in}{2.385790in}}%
\pgfpathlineto{\pgfqpoint{1.580545in}{2.154278in}}%
\pgfpathlineto{\pgfqpoint{1.582379in}{2.229529in}}%
\pgfpathlineto{\pgfqpoint{1.584213in}{2.219003in}}%
\pgfpathlineto{\pgfqpoint{1.586048in}{2.173461in}}%
\pgfpathlineto{\pgfqpoint{1.589716in}{1.343314in}}%
\pgfpathlineto{\pgfqpoint{1.593382in}{1.132867in}}%
\pgfpathlineto{\pgfqpoint{1.595217in}{1.185999in}}%
\pgfpathlineto{\pgfqpoint{1.597050in}{1.140834in}}%
\pgfpathlineto{\pgfqpoint{1.598882in}{1.236234in}}%
\pgfpathlineto{\pgfqpoint{1.600716in}{2.220007in}}%
\pgfpathlineto{\pgfqpoint{1.602565in}{2.084020in}}%
\pgfpathlineto{\pgfqpoint{1.604399in}{2.099125in}}%
\pgfpathlineto{\pgfqpoint{1.606234in}{2.076555in}}%
\pgfpathlineto{\pgfqpoint{1.608066in}{2.084283in}}%
\pgfpathlineto{\pgfqpoint{1.609901in}{2.215302in}}%
\pgfpathlineto{\pgfqpoint{1.611736in}{2.072214in}}%
\pgfpathlineto{\pgfqpoint{1.613569in}{2.225351in}}%
\pgfpathlineto{\pgfqpoint{1.615402in}{2.118221in}}%
\pgfpathlineto{\pgfqpoint{1.617235in}{2.101007in}}%
\pgfpathlineto{\pgfqpoint{1.619070in}{2.177752in}}%
\pgfpathlineto{\pgfqpoint{1.620903in}{2.181954in}}%
\pgfpathlineto{\pgfqpoint{1.626403in}{1.094463in}}%
\pgfpathlineto{\pgfqpoint{1.628237in}{1.137283in}}%
\pgfpathlineto{\pgfqpoint{1.630070in}{1.216135in}}%
\pgfpathlineto{\pgfqpoint{1.631903in}{1.150996in}}%
\pgfpathlineto{\pgfqpoint{1.633737in}{1.274022in}}%
\pgfpathlineto{\pgfqpoint{1.635570in}{1.224716in}}%
\pgfpathlineto{\pgfqpoint{1.637403in}{1.308286in}}%
\pgfpathlineto{\pgfqpoint{1.639237in}{1.318925in}}%
\pgfpathlineto{\pgfqpoint{1.642905in}{0.961099in}}%
\pgfpathlineto{\pgfqpoint{1.644738in}{1.003404in}}%
\pgfpathlineto{\pgfqpoint{1.648404in}{1.246760in}}%
\pgfpathlineto{\pgfqpoint{1.650238in}{1.113270in}}%
\pgfpathlineto{\pgfqpoint{1.652073in}{1.205032in}}%
\pgfpathlineto{\pgfqpoint{1.653906in}{1.088203in}}%
\pgfpathlineto{\pgfqpoint{1.655738in}{1.330680in}}%
\pgfpathlineto{\pgfqpoint{1.659405in}{1.214930in}}%
\pgfpathlineto{\pgfqpoint{1.661238in}{1.358319in}}%
\pgfpathlineto{\pgfqpoint{1.663073in}{1.318448in}}%
\pgfpathlineto{\pgfqpoint{1.664906in}{1.330204in}}%
\pgfpathlineto{\pgfqpoint{1.666740in}{1.293419in}}%
\pgfpathlineto{\pgfqpoint{1.668574in}{1.187555in}}%
\pgfpathlineto{\pgfqpoint{1.670408in}{1.339111in}}%
\pgfpathlineto{\pgfqpoint{1.672240in}{1.212559in}}%
\pgfpathlineto{\pgfqpoint{1.674074in}{1.192486in}}%
\pgfpathlineto{\pgfqpoint{1.675908in}{1.292239in}}%
\pgfpathlineto{\pgfqpoint{1.677742in}{1.175862in}}%
\pgfpathlineto{\pgfqpoint{1.679575in}{1.524278in}}%
\pgfpathlineto{\pgfqpoint{1.681409in}{2.329019in}}%
\pgfpathlineto{\pgfqpoint{1.685075in}{1.986261in}}%
\pgfpathlineto{\pgfqpoint{1.686909in}{1.267536in}}%
\pgfpathlineto{\pgfqpoint{1.688743in}{1.353326in}}%
\pgfpathlineto{\pgfqpoint{1.690577in}{1.222195in}}%
\pgfpathlineto{\pgfqpoint{1.692411in}{1.172625in}}%
\pgfpathlineto{\pgfqpoint{1.694244in}{1.195020in}}%
\pgfpathlineto{\pgfqpoint{1.696076in}{1.164345in}}%
\pgfpathlineto{\pgfqpoint{1.697909in}{1.208507in}}%
\pgfpathlineto{\pgfqpoint{1.699744in}{1.574902in}}%
\pgfpathlineto{\pgfqpoint{1.701578in}{1.477419in}}%
\pgfpathlineto{\pgfqpoint{1.703412in}{1.219083in}}%
\pgfpathlineto{\pgfqpoint{1.705247in}{1.361268in}}%
\pgfpathlineto{\pgfqpoint{1.707080in}{1.268615in}}%
\pgfpathlineto{\pgfqpoint{1.710745in}{1.311786in}}%
\pgfpathlineto{\pgfqpoint{1.712579in}{1.307282in}}%
\pgfpathlineto{\pgfqpoint{1.714412in}{1.367729in}}%
\pgfpathlineto{\pgfqpoint{1.716246in}{1.267298in}}%
\pgfpathlineto{\pgfqpoint{1.718078in}{1.320154in}}%
\pgfpathlineto{\pgfqpoint{1.719912in}{1.241026in}}%
\pgfpathlineto{\pgfqpoint{1.721745in}{1.353326in}}%
\pgfpathlineto{\pgfqpoint{1.723579in}{1.252255in}}%
\pgfpathlineto{\pgfqpoint{1.725413in}{1.259682in}}%
\pgfpathlineto{\pgfqpoint{1.727245in}{1.202246in}}%
\pgfpathlineto{\pgfqpoint{1.729080in}{1.195396in}}%
\pgfpathlineto{\pgfqpoint{1.730913in}{1.052033in}}%
\pgfpathlineto{\pgfqpoint{1.732745in}{1.267624in}}%
\pgfpathlineto{\pgfqpoint{1.734578in}{1.143656in}}%
\pgfpathlineto{\pgfqpoint{1.738246in}{1.307784in}}%
\pgfpathlineto{\pgfqpoint{1.740080in}{1.190779in}}%
\pgfpathlineto{\pgfqpoint{1.741914in}{1.490718in}}%
\pgfpathlineto{\pgfqpoint{1.743747in}{1.342047in}}%
\pgfpathlineto{\pgfqpoint{1.745581in}{1.308900in}}%
\pgfpathlineto{\pgfqpoint{1.747414in}{1.156491in}}%
\pgfpathlineto{\pgfqpoint{1.749249in}{1.336138in}}%
\pgfpathlineto{\pgfqpoint{1.751084in}{1.295087in}}%
\pgfpathlineto{\pgfqpoint{1.752918in}{1.298650in}}%
\pgfpathlineto{\pgfqpoint{1.754750in}{1.439066in}}%
\pgfpathlineto{\pgfqpoint{1.756586in}{1.238856in}}%
\pgfpathlineto{\pgfqpoint{1.758418in}{1.359172in}}%
\pgfpathlineto{\pgfqpoint{1.762086in}{1.262304in}}%
\pgfpathlineto{\pgfqpoint{1.763919in}{1.389032in}}%
\pgfpathlineto{\pgfqpoint{1.765752in}{1.300645in}}%
\pgfpathlineto{\pgfqpoint{1.767584in}{1.365759in}}%
\pgfpathlineto{\pgfqpoint{1.769418in}{1.488547in}}%
\pgfpathlineto{\pgfqpoint{1.771251in}{1.353125in}}%
\pgfpathlineto{\pgfqpoint{1.773085in}{1.071454in}}%
\pgfpathlineto{\pgfqpoint{1.776754in}{1.204003in}}%
\pgfpathlineto{\pgfqpoint{1.778589in}{1.116595in}}%
\pgfpathlineto{\pgfqpoint{1.780423in}{1.329526in}}%
\pgfpathlineto{\pgfqpoint{1.784088in}{1.396785in}}%
\pgfpathlineto{\pgfqpoint{1.785923in}{1.105604in}}%
\pgfpathlineto{\pgfqpoint{1.791425in}{1.381128in}}%
\pgfpathlineto{\pgfqpoint{1.793259in}{1.368318in}}%
\pgfpathlineto{\pgfqpoint{1.795091in}{1.368958in}}%
\pgfpathlineto{\pgfqpoint{1.796923in}{1.396961in}}%
\pgfpathlineto{\pgfqpoint{1.804261in}{1.188759in}}%
\pgfpathlineto{\pgfqpoint{1.806094in}{1.199950in}}%
\pgfpathlineto{\pgfqpoint{1.807926in}{1.299616in}}%
\pgfpathlineto{\pgfqpoint{1.809760in}{1.281048in}}%
\pgfpathlineto{\pgfqpoint{1.811593in}{1.423697in}}%
\pgfpathlineto{\pgfqpoint{1.813426in}{1.365082in}}%
\pgfpathlineto{\pgfqpoint{1.815261in}{1.409156in}}%
\pgfpathlineto{\pgfqpoint{1.818926in}{1.198332in}}%
\pgfpathlineto{\pgfqpoint{1.820760in}{1.233950in}}%
\pgfpathlineto{\pgfqpoint{1.822594in}{1.157520in}}%
\pgfpathlineto{\pgfqpoint{1.824427in}{1.136016in}}%
\pgfpathlineto{\pgfqpoint{1.826262in}{1.382157in}}%
\pgfpathlineto{\pgfqpoint{1.828096in}{1.172475in}}%
\pgfpathlineto{\pgfqpoint{1.829929in}{1.186727in}}%
\pgfpathlineto{\pgfqpoint{1.831763in}{1.150055in}}%
\pgfpathlineto{\pgfqpoint{1.833596in}{1.177857in}}%
\pgfpathlineto{\pgfqpoint{1.835429in}{1.140005in}}%
\pgfpathlineto{\pgfqpoint{1.839098in}{1.297094in}}%
\pgfpathlineto{\pgfqpoint{1.840930in}{1.399458in}}%
\pgfpathlineto{\pgfqpoint{1.842762in}{1.623216in}}%
\pgfpathlineto{\pgfqpoint{1.844596in}{2.350285in}}%
\pgfpathlineto{\pgfqpoint{1.846431in}{2.251058in}}%
\pgfpathlineto{\pgfqpoint{1.848264in}{2.248950in}}%
\pgfpathlineto{\pgfqpoint{1.850099in}{2.485719in}}%
\pgfpathlineto{\pgfqpoint{1.853767in}{2.166297in}}%
\pgfpathlineto{\pgfqpoint{1.855603in}{2.253354in}}%
\pgfpathlineto{\pgfqpoint{1.857436in}{2.148080in}}%
\pgfpathlineto{\pgfqpoint{1.859269in}{2.115347in}}%
\pgfpathlineto{\pgfqpoint{1.861102in}{2.146311in}}%
\pgfpathlineto{\pgfqpoint{1.862937in}{2.134330in}}%
\pgfpathlineto{\pgfqpoint{1.864769in}{2.200284in}}%
\pgfpathlineto{\pgfqpoint{1.868437in}{1.999159in}}%
\pgfpathlineto{\pgfqpoint{1.870270in}{2.312383in}}%
\pgfpathlineto{\pgfqpoint{1.872104in}{1.429631in}}%
\pgfpathlineto{\pgfqpoint{1.873938in}{1.215495in}}%
\pgfpathlineto{\pgfqpoint{1.875771in}{1.158047in}}%
\pgfpathlineto{\pgfqpoint{1.877984in}{1.237212in}}%
\pgfpathlineto{\pgfqpoint{1.881649in}{1.528895in}}%
\pgfpathlineto{\pgfqpoint{1.883483in}{2.110969in}}%
\pgfpathlineto{\pgfqpoint{1.885316in}{2.210547in}}%
\pgfpathlineto{\pgfqpoint{1.887148in}{2.115573in}}%
\pgfpathlineto{\pgfqpoint{1.888982in}{2.242778in}}%
\pgfpathlineto{\pgfqpoint{1.890815in}{2.198967in}}%
\pgfpathlineto{\pgfqpoint{1.894481in}{2.279011in}}%
\pgfpathlineto{\pgfqpoint{1.896315in}{2.248536in}}%
\pgfpathlineto{\pgfqpoint{1.898148in}{2.142999in}}%
\pgfpathlineto{\pgfqpoint{1.899983in}{2.221060in}}%
\pgfpathlineto{\pgfqpoint{1.901816in}{2.358101in}}%
\pgfpathlineto{\pgfqpoint{1.903651in}{2.255035in}}%
\pgfpathlineto{\pgfqpoint{1.907317in}{2.443075in}}%
\pgfpathlineto{\pgfqpoint{1.909150in}{2.615270in}}%
\pgfpathlineto{\pgfqpoint{1.910984in}{2.267606in}}%
\pgfpathlineto{\pgfqpoint{1.912818in}{2.255060in}}%
\pgfpathlineto{\pgfqpoint{1.916485in}{2.465708in}}%
\pgfpathlineto{\pgfqpoint{1.920151in}{2.112261in}}%
\pgfpathlineto{\pgfqpoint{1.921984in}{2.204136in}}%
\pgfpathlineto{\pgfqpoint{1.923819in}{2.207134in}}%
\pgfpathlineto{\pgfqpoint{1.925653in}{2.334652in}}%
\pgfpathlineto{\pgfqpoint{1.927486in}{1.253836in}}%
\pgfpathlineto{\pgfqpoint{1.929321in}{1.344079in}}%
\pgfpathlineto{\pgfqpoint{1.931154in}{1.131248in}}%
\pgfpathlineto{\pgfqpoint{1.932987in}{1.597384in}}%
\pgfpathlineto{\pgfqpoint{1.934824in}{1.776378in}}%
\pgfpathlineto{\pgfqpoint{1.936657in}{1.695105in}}%
\pgfpathlineto{\pgfqpoint{1.938490in}{1.516613in}}%
\pgfpathlineto{\pgfqpoint{1.940325in}{1.617219in}}%
\pgfpathlineto{\pgfqpoint{1.942159in}{1.541855in}}%
\pgfpathlineto{\pgfqpoint{1.945827in}{1.306052in}}%
\pgfpathlineto{\pgfqpoint{1.947660in}{1.349625in}}%
\pgfpathlineto{\pgfqpoint{1.949493in}{1.056035in}}%
\pgfpathlineto{\pgfqpoint{1.951341in}{1.290269in}}%
\pgfpathlineto{\pgfqpoint{1.953174in}{1.252845in}}%
\pgfpathlineto{\pgfqpoint{1.955006in}{1.070965in}}%
\pgfpathlineto{\pgfqpoint{1.956840in}{1.311874in}}%
\pgfpathlineto{\pgfqpoint{1.960506in}{1.188935in}}%
\pgfpathlineto{\pgfqpoint{1.962341in}{1.290771in}}%
\pgfpathlineto{\pgfqpoint{1.964174in}{1.229798in}}%
\pgfpathlineto{\pgfqpoint{1.966007in}{1.209937in}}%
\pgfpathlineto{\pgfqpoint{1.967841in}{1.210916in}}%
\pgfpathlineto{\pgfqpoint{1.971508in}{2.044324in}}%
\pgfpathlineto{\pgfqpoint{1.973343in}{2.051413in}}%
\pgfpathlineto{\pgfqpoint{1.977011in}{2.339031in}}%
\pgfpathlineto{\pgfqpoint{1.978845in}{2.248361in}}%
\pgfpathlineto{\pgfqpoint{1.980679in}{1.942099in}}%
\pgfpathlineto{\pgfqpoint{1.982512in}{2.126012in}}%
\pgfpathlineto{\pgfqpoint{1.984346in}{2.094283in}}%
\pgfpathlineto{\pgfqpoint{1.986181in}{2.142372in}}%
\pgfpathlineto{\pgfqpoint{1.988015in}{2.394158in}}%
\pgfpathlineto{\pgfqpoint{1.989847in}{2.089189in}}%
\pgfpathlineto{\pgfqpoint{1.991680in}{2.035868in}}%
\pgfpathlineto{\pgfqpoint{1.995348in}{2.190862in}}%
\pgfpathlineto{\pgfqpoint{1.997182in}{2.178981in}}%
\pgfpathlineto{\pgfqpoint{1.999016in}{2.002132in}}%
\pgfpathlineto{\pgfqpoint{2.000873in}{2.293903in}}%
\pgfpathlineto{\pgfqpoint{2.002706in}{1.409507in}}%
\pgfpathlineto{\pgfqpoint{2.004539in}{1.377778in}}%
\pgfpathlineto{\pgfqpoint{2.006374in}{1.125051in}}%
\pgfpathlineto{\pgfqpoint{2.008208in}{1.186727in}}%
\pgfpathlineto{\pgfqpoint{2.010042in}{1.146391in}}%
\pgfpathlineto{\pgfqpoint{2.011876in}{1.228330in}}%
\pgfpathlineto{\pgfqpoint{2.013709in}{1.123733in}}%
\pgfpathlineto{\pgfqpoint{2.015542in}{1.282014in}}%
\pgfpathlineto{\pgfqpoint{2.017375in}{0.981963in}}%
\pgfpathlineto{\pgfqpoint{2.019208in}{1.294886in}}%
\pgfpathlineto{\pgfqpoint{2.022876in}{1.117849in}}%
\pgfpathlineto{\pgfqpoint{2.024711in}{1.052033in}}%
\pgfpathlineto{\pgfqpoint{2.026543in}{1.145149in}}%
\pgfpathlineto{\pgfqpoint{2.028377in}{1.134397in}}%
\pgfpathlineto{\pgfqpoint{2.030211in}{1.103923in}}%
\pgfpathlineto{\pgfqpoint{2.032044in}{1.309227in}}%
\pgfpathlineto{\pgfqpoint{2.033877in}{1.243849in}}%
\pgfpathlineto{\pgfqpoint{2.035712in}{1.300093in}}%
\pgfpathlineto{\pgfqpoint{2.039381in}{1.157143in}}%
\pgfpathlineto{\pgfqpoint{2.041214in}{1.375984in}}%
\pgfpathlineto{\pgfqpoint{2.043048in}{2.166347in}}%
\pgfpathlineto{\pgfqpoint{2.044882in}{1.992083in}}%
\pgfpathlineto{\pgfqpoint{2.046716in}{2.385577in}}%
\pgfpathlineto{\pgfqpoint{2.050384in}{2.229491in}}%
\pgfpathlineto{\pgfqpoint{2.054049in}{2.055252in}}%
\pgfpathlineto{\pgfqpoint{2.055883in}{2.069216in}}%
\pgfpathlineto{\pgfqpoint{2.057717in}{2.196671in}}%
\pgfpathlineto{\pgfqpoint{2.059552in}{1.349775in}}%
\pgfpathlineto{\pgfqpoint{2.061385in}{1.258980in}}%
\pgfpathlineto{\pgfqpoint{2.063219in}{1.267586in}}%
\pgfpathlineto{\pgfqpoint{2.065051in}{1.313580in}}%
\pgfpathlineto{\pgfqpoint{2.066885in}{1.268477in}}%
\pgfpathlineto{\pgfqpoint{2.068720in}{1.133569in}}%
\pgfpathlineto{\pgfqpoint{2.070554in}{1.164809in}}%
\pgfpathlineto{\pgfqpoint{2.072387in}{1.166453in}}%
\pgfpathlineto{\pgfqpoint{2.074222in}{1.187818in}}%
\pgfpathlineto{\pgfqpoint{2.076056in}{1.271827in}}%
\pgfpathlineto{\pgfqpoint{2.077889in}{1.118062in}}%
\pgfpathlineto{\pgfqpoint{2.079722in}{1.189725in}}%
\pgfpathlineto{\pgfqpoint{2.081556in}{0.964386in}}%
\pgfpathlineto{\pgfqpoint{2.083390in}{1.215257in}}%
\pgfpathlineto{\pgfqpoint{2.085224in}{1.270472in}}%
\pgfpathlineto{\pgfqpoint{2.087060in}{1.137484in}}%
\pgfpathlineto{\pgfqpoint{2.088893in}{1.109217in}}%
\pgfpathlineto{\pgfqpoint{2.090725in}{1.192511in}}%
\pgfpathlineto{\pgfqpoint{2.092559in}{1.181470in}}%
\pgfpathlineto{\pgfqpoint{2.094392in}{1.283131in}}%
\pgfpathlineto{\pgfqpoint{2.096225in}{1.178534in}}%
\pgfpathlineto{\pgfqpoint{2.098059in}{1.221894in}}%
\pgfpathlineto{\pgfqpoint{2.099907in}{1.421238in}}%
\pgfpathlineto{\pgfqpoint{2.101741in}{1.193903in}}%
\pgfpathlineto{\pgfqpoint{2.103575in}{1.272002in}}%
\pgfpathlineto{\pgfqpoint{2.105407in}{1.269945in}}%
\pgfpathlineto{\pgfqpoint{2.109075in}{1.055157in}}%
\pgfpathlineto{\pgfqpoint{2.110908in}{1.064315in}}%
\pgfpathlineto{\pgfqpoint{2.112743in}{1.028672in}}%
\pgfpathlineto{\pgfqpoint{2.114577in}{1.128701in}}%
\pgfpathlineto{\pgfqpoint{2.116411in}{1.030943in}}%
\pgfpathlineto{\pgfqpoint{2.118245in}{1.024005in}}%
\pgfpathlineto{\pgfqpoint{2.121912in}{1.177029in}}%
\pgfpathlineto{\pgfqpoint{2.125580in}{1.027530in}}%
\pgfpathlineto{\pgfqpoint{2.127413in}{1.272906in}}%
\pgfpathlineto{\pgfqpoint{2.131080in}{1.159577in}}%
\pgfpathlineto{\pgfqpoint{2.132913in}{1.205709in}}%
\pgfpathlineto{\pgfqpoint{2.136582in}{1.199599in}}%
\pgfpathlineto{\pgfqpoint{2.138416in}{1.144447in}}%
\pgfpathlineto{\pgfqpoint{2.140250in}{1.135514in}}%
\pgfpathlineto{\pgfqpoint{2.142083in}{1.154910in}}%
\pgfpathlineto{\pgfqpoint{2.143917in}{1.290219in}}%
\pgfpathlineto{\pgfqpoint{2.145750in}{1.320656in}}%
\pgfpathlineto{\pgfqpoint{2.147586in}{1.234891in}}%
\pgfpathlineto{\pgfqpoint{2.149421in}{1.246634in}}%
\pgfpathlineto{\pgfqpoint{2.153089in}{1.157821in}}%
\pgfpathlineto{\pgfqpoint{2.154922in}{1.071454in}}%
\pgfpathlineto{\pgfqpoint{2.156755in}{1.217753in}}%
\pgfpathlineto{\pgfqpoint{2.158588in}{1.094639in}}%
\pgfpathlineto{\pgfqpoint{2.160423in}{1.142239in}}%
\pgfpathlineto{\pgfqpoint{2.162254in}{1.275703in}}%
\pgfpathlineto{\pgfqpoint{2.164087in}{1.283432in}}%
\pgfpathlineto{\pgfqpoint{2.167753in}{1.059937in}}%
\pgfpathlineto{\pgfqpoint{2.169586in}{1.231240in}}%
\pgfpathlineto{\pgfqpoint{2.171421in}{1.087852in}}%
\pgfpathlineto{\pgfqpoint{2.173255in}{1.199775in}}%
\pgfpathlineto{\pgfqpoint{2.175088in}{1.118000in}}%
\pgfpathlineto{\pgfqpoint{2.178756in}{1.935839in}}%
\pgfpathlineto{\pgfqpoint{2.180589in}{2.126338in}}%
\pgfpathlineto{\pgfqpoint{2.182425in}{2.032305in}}%
\pgfpathlineto{\pgfqpoint{2.184258in}{2.207373in}}%
\pgfpathlineto{\pgfqpoint{2.186092in}{2.025167in}}%
\pgfpathlineto{\pgfqpoint{2.187926in}{1.065030in}}%
\pgfpathlineto{\pgfqpoint{2.189760in}{1.289805in}}%
\pgfpathlineto{\pgfqpoint{2.191595in}{1.217251in}}%
\pgfpathlineto{\pgfqpoint{2.193430in}{1.218544in}}%
\pgfpathlineto{\pgfqpoint{2.195262in}{1.127196in}}%
\pgfpathlineto{\pgfqpoint{2.197096in}{1.322387in}}%
\pgfpathlineto{\pgfqpoint{2.198950in}{1.382684in}}%
\pgfpathlineto{\pgfqpoint{2.200782in}{1.168071in}}%
\pgfpathlineto{\pgfqpoint{2.202617in}{1.245054in}}%
\pgfpathlineto{\pgfqpoint{2.204450in}{1.213086in}}%
\pgfpathlineto{\pgfqpoint{2.206283in}{1.372283in}}%
\pgfpathlineto{\pgfqpoint{2.208117in}{1.284661in}}%
\pgfpathlineto{\pgfqpoint{2.209951in}{1.447936in}}%
\pgfpathlineto{\pgfqpoint{2.211786in}{1.227778in}}%
\pgfpathlineto{\pgfqpoint{2.213619in}{1.462640in}}%
\pgfpathlineto{\pgfqpoint{2.219122in}{1.289391in}}%
\pgfpathlineto{\pgfqpoint{2.220957in}{1.236146in}}%
\pgfpathlineto{\pgfqpoint{2.222791in}{1.074164in}}%
\pgfpathlineto{\pgfqpoint{2.224623in}{1.046537in}}%
\pgfpathlineto{\pgfqpoint{2.226457in}{1.119242in}}%
\pgfpathlineto{\pgfqpoint{2.228291in}{0.930863in}}%
\pgfpathlineto{\pgfqpoint{2.230125in}{1.210201in}}%
\pgfpathlineto{\pgfqpoint{2.231958in}{1.108038in}}%
\pgfpathlineto{\pgfqpoint{2.233792in}{1.210966in}}%
\pgfpathlineto{\pgfqpoint{2.237460in}{1.083335in}}%
\pgfpathlineto{\pgfqpoint{2.239295in}{1.061643in}}%
\pgfpathlineto{\pgfqpoint{2.241128in}{1.072520in}}%
\pgfpathlineto{\pgfqpoint{2.242963in}{1.029174in}}%
\pgfpathlineto{\pgfqpoint{2.246630in}{1.215495in}}%
\pgfpathlineto{\pgfqpoint{2.248478in}{0.970145in}}%
\pgfpathlineto{\pgfqpoint{2.250312in}{1.126255in}}%
\pgfpathlineto{\pgfqpoint{2.252145in}{1.032147in}}%
\pgfpathlineto{\pgfqpoint{2.253979in}{1.223161in}}%
\pgfpathlineto{\pgfqpoint{2.255812in}{1.093911in}}%
\pgfpathlineto{\pgfqpoint{2.257648in}{1.120710in}}%
\pgfpathlineto{\pgfqpoint{2.261314in}{0.861283in}}%
\pgfpathlineto{\pgfqpoint{2.263148in}{0.931867in}}%
\pgfpathlineto{\pgfqpoint{2.264982in}{1.100548in}}%
\pgfpathlineto{\pgfqpoint{2.266816in}{1.111212in}}%
\pgfpathlineto{\pgfqpoint{2.268648in}{0.909271in}}%
\pgfpathlineto{\pgfqpoint{2.270482in}{0.943171in}}%
\pgfpathlineto{\pgfqpoint{2.272317in}{1.219108in}}%
\pgfpathlineto{\pgfqpoint{2.274151in}{1.189462in}}%
\pgfpathlineto{\pgfqpoint{2.275986in}{1.033640in}}%
\pgfpathlineto{\pgfqpoint{2.277818in}{1.310807in}}%
\pgfpathlineto{\pgfqpoint{2.279652in}{0.981549in}}%
\pgfpathlineto{\pgfqpoint{2.281486in}{0.945316in}}%
\pgfpathlineto{\pgfqpoint{2.285153in}{0.801024in}}%
\pgfpathlineto{\pgfqpoint{2.286987in}{0.983983in}}%
\pgfpathlineto{\pgfqpoint{2.288821in}{0.738821in}}%
\pgfpathlineto{\pgfqpoint{2.290654in}{0.790410in}}%
\pgfpathlineto{\pgfqpoint{2.292489in}{0.986342in}}%
\pgfpathlineto{\pgfqpoint{2.294323in}{1.005499in}}%
\pgfpathlineto{\pgfqpoint{2.296157in}{0.935743in}}%
\pgfpathlineto{\pgfqpoint{2.298007in}{0.963395in}}%
\pgfpathlineto{\pgfqpoint{2.299841in}{0.803746in}}%
\pgfpathlineto{\pgfqpoint{2.301675in}{0.839716in}}%
\pgfpathlineto{\pgfqpoint{2.305342in}{0.944613in}}%
\pgfpathlineto{\pgfqpoint{2.307176in}{0.989014in}}%
\pgfpathlineto{\pgfqpoint{2.309010in}{0.999000in}}%
\pgfpathlineto{\pgfqpoint{2.310843in}{1.024005in}}%
\pgfpathlineto{\pgfqpoint{2.312677in}{1.002288in}}%
\pgfpathlineto{\pgfqpoint{2.316343in}{0.874569in}}%
\pgfpathlineto{\pgfqpoint{2.320010in}{1.113445in}}%
\pgfpathlineto{\pgfqpoint{2.323678in}{1.022035in}}%
\pgfpathlineto{\pgfqpoint{2.325512in}{1.086973in}}%
\pgfpathlineto{\pgfqpoint{2.327345in}{1.044480in}}%
\pgfpathlineto{\pgfqpoint{2.329179in}{0.861345in}}%
\pgfpathlineto{\pgfqpoint{2.331014in}{0.815126in}}%
\pgfpathlineto{\pgfqpoint{2.332847in}{0.970822in}}%
\pgfpathlineto{\pgfqpoint{2.334681in}{0.995563in}}%
\pgfpathlineto{\pgfqpoint{2.336516in}{0.912408in}}%
\pgfpathlineto{\pgfqpoint{2.338350in}{1.038232in}}%
\pgfpathlineto{\pgfqpoint{2.340183in}{1.060288in}}%
\pgfpathlineto{\pgfqpoint{2.342017in}{0.985752in}}%
\pgfpathlineto{\pgfqpoint{2.343851in}{1.101602in}}%
\pgfpathlineto{\pgfqpoint{2.345685in}{1.035434in}}%
\pgfpathlineto{\pgfqpoint{2.347518in}{1.112944in}}%
\pgfpathlineto{\pgfqpoint{2.349353in}{1.019890in}}%
\pgfpathlineto{\pgfqpoint{2.351186in}{1.018422in}}%
\pgfpathlineto{\pgfqpoint{2.353019in}{0.900564in}}%
\pgfpathlineto{\pgfqpoint{2.354854in}{1.045132in}}%
\pgfpathlineto{\pgfqpoint{2.356687in}{1.070162in}}%
\pgfpathlineto{\pgfqpoint{2.358520in}{1.025623in}}%
\pgfpathlineto{\pgfqpoint{2.360354in}{1.040202in}}%
\pgfpathlineto{\pgfqpoint{2.362187in}{0.987308in}}%
\pgfpathlineto{\pgfqpoint{2.364021in}{0.991385in}}%
\pgfpathlineto{\pgfqpoint{2.365856in}{0.912056in}}%
\pgfpathlineto{\pgfqpoint{2.369522in}{0.871771in}}%
\pgfpathlineto{\pgfqpoint{2.371356in}{0.873389in}}%
\pgfpathlineto{\pgfqpoint{2.373192in}{1.075280in}}%
\pgfpathlineto{\pgfqpoint{2.375026in}{0.964913in}}%
\pgfpathlineto{\pgfqpoint{2.376861in}{0.979291in}}%
\pgfpathlineto{\pgfqpoint{2.378694in}{1.046011in}}%
\pgfpathlineto{\pgfqpoint{2.380528in}{0.849941in}}%
\pgfpathlineto{\pgfqpoint{2.382365in}{0.875648in}}%
\pgfpathlineto{\pgfqpoint{2.386030in}{0.971964in}}%
\pgfpathlineto{\pgfqpoint{2.389698in}{0.835137in}}%
\pgfpathlineto{\pgfqpoint{2.391531in}{0.648878in}}%
\pgfpathlineto{\pgfqpoint{2.393366in}{0.690744in}}%
\pgfpathlineto{\pgfqpoint{2.397033in}{1.258428in}}%
\pgfpathlineto{\pgfqpoint{2.400702in}{0.871595in}}%
\pgfpathlineto{\pgfqpoint{2.402535in}{1.079358in}}%
\pgfpathlineto{\pgfqpoint{2.404369in}{0.981925in}}%
\pgfpathlineto{\pgfqpoint{2.406204in}{1.019363in}}%
\pgfpathlineto{\pgfqpoint{2.408037in}{0.916962in}}%
\pgfpathlineto{\pgfqpoint{2.409871in}{0.967974in}}%
\pgfpathlineto{\pgfqpoint{2.411705in}{1.051242in}}%
\pgfpathlineto{\pgfqpoint{2.415371in}{0.880240in}}%
\pgfpathlineto{\pgfqpoint{2.417205in}{0.915645in}}%
\pgfpathlineto{\pgfqpoint{2.419039in}{1.045596in}}%
\pgfpathlineto{\pgfqpoint{2.420872in}{0.994622in}}%
\pgfpathlineto{\pgfqpoint{2.424540in}{1.062320in}}%
\pgfpathlineto{\pgfqpoint{2.426374in}{0.912822in}}%
\pgfpathlineto{\pgfqpoint{2.428209in}{1.008586in}}%
\pgfpathlineto{\pgfqpoint{2.430043in}{1.014457in}}%
\pgfpathlineto{\pgfqpoint{2.431877in}{1.037793in}}%
\pgfpathlineto{\pgfqpoint{2.433711in}{0.772983in}}%
\pgfpathlineto{\pgfqpoint{2.435544in}{0.991385in}}%
\pgfpathlineto{\pgfqpoint{2.437378in}{0.962630in}}%
\pgfpathlineto{\pgfqpoint{2.439213in}{1.210765in}}%
\pgfpathlineto{\pgfqpoint{2.441046in}{1.815610in}}%
\pgfpathlineto{\pgfqpoint{2.442879in}{1.656739in}}%
\pgfpathlineto{\pgfqpoint{2.444714in}{1.981004in}}%
\pgfpathlineto{\pgfqpoint{2.446550in}{1.928411in}}%
\pgfpathlineto{\pgfqpoint{2.448385in}{1.950154in}}%
\pgfpathlineto{\pgfqpoint{2.450220in}{1.290621in}}%
\pgfpathlineto{\pgfqpoint{2.452054in}{1.294096in}}%
\pgfpathlineto{\pgfqpoint{2.453888in}{1.511406in}}%
\pgfpathlineto{\pgfqpoint{2.455723in}{1.110773in}}%
\pgfpathlineto{\pgfqpoint{2.459389in}{1.204304in}}%
\pgfpathlineto{\pgfqpoint{2.461223in}{1.071078in}}%
\pgfpathlineto{\pgfqpoint{2.463057in}{1.193953in}}%
\pgfpathlineto{\pgfqpoint{2.464891in}{1.058619in}}%
\pgfpathlineto{\pgfqpoint{2.466724in}{1.107097in}}%
\pgfpathlineto{\pgfqpoint{2.468558in}{1.103522in}}%
\pgfpathlineto{\pgfqpoint{2.470392in}{1.147320in}}%
\pgfpathlineto{\pgfqpoint{2.474059in}{2.006511in}}%
\pgfpathlineto{\pgfqpoint{2.475893in}{2.000049in}}%
\pgfpathlineto{\pgfqpoint{2.477727in}{1.851692in}}%
\pgfpathlineto{\pgfqpoint{2.479561in}{1.029036in}}%
\pgfpathlineto{\pgfqpoint{2.481394in}{1.026715in}}%
\pgfpathlineto{\pgfqpoint{2.483228in}{0.866866in}}%
\pgfpathlineto{\pgfqpoint{2.485062in}{0.929897in}}%
\pgfpathlineto{\pgfqpoint{2.486896in}{1.048156in}}%
\pgfpathlineto{\pgfqpoint{2.488729in}{0.928918in}}%
\pgfpathlineto{\pgfqpoint{2.490563in}{0.961889in}}%
\pgfpathlineto{\pgfqpoint{2.492397in}{0.857995in}}%
\pgfpathlineto{\pgfqpoint{2.494230in}{0.893375in}}%
\pgfpathlineto{\pgfqpoint{2.496065in}{0.872009in}}%
\pgfpathlineto{\pgfqpoint{2.497897in}{1.098641in}}%
\pgfpathlineto{\pgfqpoint{2.499731in}{0.992953in}}%
\pgfpathlineto{\pgfqpoint{2.503403in}{1.168096in}}%
\pgfpathlineto{\pgfqpoint{2.505236in}{1.478849in}}%
\pgfpathlineto{\pgfqpoint{2.507071in}{1.334406in}}%
\pgfpathlineto{\pgfqpoint{2.508905in}{1.393523in}}%
\pgfpathlineto{\pgfqpoint{2.510738in}{1.348333in}}%
\pgfpathlineto{\pgfqpoint{2.512572in}{1.338898in}}%
\pgfpathlineto{\pgfqpoint{2.514406in}{1.346752in}}%
\pgfpathlineto{\pgfqpoint{2.518075in}{1.100398in}}%
\pgfpathlineto{\pgfqpoint{2.519909in}{1.250461in}}%
\pgfpathlineto{\pgfqpoint{2.521742in}{1.113245in}}%
\pgfpathlineto{\pgfqpoint{2.523576in}{1.221718in}}%
\pgfpathlineto{\pgfqpoint{2.527245in}{1.021332in}}%
\pgfpathlineto{\pgfqpoint{2.529079in}{1.287773in}}%
\pgfpathlineto{\pgfqpoint{2.530912in}{1.257775in}}%
\pgfpathlineto{\pgfqpoint{2.532745in}{1.174118in}}%
\pgfpathlineto{\pgfqpoint{2.534578in}{1.298738in}}%
\pgfpathlineto{\pgfqpoint{2.536412in}{1.293795in}}%
\pgfpathlineto{\pgfqpoint{2.538246in}{1.268301in}}%
\pgfpathlineto{\pgfqpoint{2.540080in}{1.260008in}}%
\pgfpathlineto{\pgfqpoint{2.541915in}{1.242444in}}%
\pgfpathlineto{\pgfqpoint{2.547418in}{2.259941in}}%
\pgfpathlineto{\pgfqpoint{2.549252in}{2.315319in}}%
\pgfpathlineto{\pgfqpoint{2.551087in}{2.307653in}}%
\pgfpathlineto{\pgfqpoint{2.552922in}{2.231060in}}%
\pgfpathlineto{\pgfqpoint{2.554755in}{2.348641in}}%
\pgfpathlineto{\pgfqpoint{2.556590in}{2.236868in}}%
\pgfpathlineto{\pgfqpoint{2.558425in}{2.250795in}}%
\pgfpathlineto{\pgfqpoint{2.560258in}{2.104909in}}%
\pgfpathlineto{\pgfqpoint{2.562091in}{2.130089in}}%
\pgfpathlineto{\pgfqpoint{2.563926in}{2.072854in}}%
\pgfpathlineto{\pgfqpoint{2.565761in}{2.329609in}}%
\pgfpathlineto{\pgfqpoint{2.567594in}{2.319472in}}%
\pgfpathlineto{\pgfqpoint{2.569431in}{2.393305in}}%
\pgfpathlineto{\pgfqpoint{2.571263in}{2.301079in}}%
\pgfpathlineto{\pgfqpoint{2.573096in}{2.430354in}}%
\pgfpathlineto{\pgfqpoint{2.574931in}{2.383406in}}%
\pgfpathlineto{\pgfqpoint{2.576766in}{2.444104in}}%
\pgfpathlineto{\pgfqpoint{2.578600in}{2.332043in}}%
\pgfpathlineto{\pgfqpoint{2.580434in}{2.401736in}}%
\pgfpathlineto{\pgfqpoint{2.584101in}{2.219856in}}%
\pgfpathlineto{\pgfqpoint{2.585936in}{2.234610in}}%
\pgfpathlineto{\pgfqpoint{2.589603in}{2.443878in}}%
\pgfpathlineto{\pgfqpoint{2.591437in}{2.428597in}}%
\pgfpathlineto{\pgfqpoint{2.595105in}{2.225941in}}%
\pgfpathlineto{\pgfqpoint{2.596940in}{2.246805in}}%
\pgfpathlineto{\pgfqpoint{2.598773in}{2.450189in}}%
\pgfpathlineto{\pgfqpoint{2.600607in}{2.379843in}}%
\pgfpathlineto{\pgfqpoint{2.602442in}{2.401799in}}%
\pgfpathlineto{\pgfqpoint{2.604276in}{2.479835in}}%
\pgfpathlineto{\pgfqpoint{2.606110in}{2.304981in}}%
\pgfpathlineto{\pgfqpoint{2.607943in}{2.478756in}}%
\pgfpathlineto{\pgfqpoint{2.611611in}{1.396873in}}%
\pgfpathlineto{\pgfqpoint{2.613445in}{1.390788in}}%
\pgfpathlineto{\pgfqpoint{2.615280in}{1.459189in}}%
\pgfpathlineto{\pgfqpoint{2.617113in}{1.454823in}}%
\pgfpathlineto{\pgfqpoint{2.618945in}{1.555894in}}%
\pgfpathlineto{\pgfqpoint{2.620781in}{1.540525in}}%
\pgfpathlineto{\pgfqpoint{2.622614in}{1.455702in}}%
\pgfpathlineto{\pgfqpoint{2.624446in}{1.550838in}}%
\pgfpathlineto{\pgfqpoint{2.626282in}{1.843588in}}%
\pgfpathlineto{\pgfqpoint{2.628117in}{1.812938in}}%
\pgfpathlineto{\pgfqpoint{2.629951in}{1.620895in}}%
\pgfpathlineto{\pgfqpoint{2.633620in}{1.527365in}}%
\pgfpathlineto{\pgfqpoint{2.637287in}{1.633027in}}%
\pgfpathlineto{\pgfqpoint{2.639121in}{1.552544in}}%
\pgfpathlineto{\pgfqpoint{2.640953in}{1.653917in}}%
\pgfpathlineto{\pgfqpoint{2.642788in}{1.496715in}}%
\pgfpathlineto{\pgfqpoint{2.646455in}{1.703511in}}%
\pgfpathlineto{\pgfqpoint{2.648288in}{1.722380in}}%
\pgfpathlineto{\pgfqpoint{2.650122in}{1.792488in}}%
\pgfpathlineto{\pgfqpoint{2.657458in}{1.602553in}}%
\pgfpathlineto{\pgfqpoint{2.659291in}{1.605614in}}%
\pgfpathlineto{\pgfqpoint{2.661125in}{1.775714in}}%
\pgfpathlineto{\pgfqpoint{2.662959in}{1.770858in}}%
\pgfpathlineto{\pgfqpoint{2.664792in}{1.387326in}}%
\pgfpathlineto{\pgfqpoint{2.666626in}{1.444498in}}%
\pgfpathlineto{\pgfqpoint{2.668460in}{1.272994in}}%
\pgfpathlineto{\pgfqpoint{2.670294in}{1.391378in}}%
\pgfpathlineto{\pgfqpoint{2.672129in}{1.301147in}}%
\pgfpathlineto{\pgfqpoint{2.675797in}{1.282077in}}%
\pgfpathlineto{\pgfqpoint{2.677631in}{1.327180in}}%
\pgfpathlineto{\pgfqpoint{2.679465in}{1.298587in}}%
\pgfpathlineto{\pgfqpoint{2.681298in}{1.296530in}}%
\pgfpathlineto{\pgfqpoint{2.683133in}{1.372433in}}%
\pgfpathlineto{\pgfqpoint{2.684967in}{1.294008in}}%
\pgfpathlineto{\pgfqpoint{2.686800in}{1.320305in}}%
\pgfpathlineto{\pgfqpoint{2.688636in}{1.226636in}}%
\pgfpathlineto{\pgfqpoint{2.690470in}{1.308173in}}%
\pgfpathlineto{\pgfqpoint{2.692303in}{1.294385in}}%
\pgfpathlineto{\pgfqpoint{2.694139in}{1.293481in}}%
\pgfpathlineto{\pgfqpoint{2.695974in}{1.375959in}}%
\pgfpathlineto{\pgfqpoint{2.697807in}{1.311610in}}%
\pgfpathlineto{\pgfqpoint{2.699642in}{1.287396in}}%
\pgfpathlineto{\pgfqpoint{2.701476in}{1.193226in}}%
\pgfpathlineto{\pgfqpoint{2.703309in}{1.280396in}}%
\pgfpathlineto{\pgfqpoint{2.705144in}{1.439856in}}%
\pgfpathlineto{\pgfqpoint{2.706978in}{1.375921in}}%
\pgfpathlineto{\pgfqpoint{2.708810in}{1.504656in}}%
\pgfpathlineto{\pgfqpoint{2.710645in}{1.347893in}}%
\pgfpathlineto{\pgfqpoint{2.712479in}{1.353301in}}%
\pgfpathlineto{\pgfqpoint{2.714312in}{1.232119in}}%
\pgfpathlineto{\pgfqpoint{2.716146in}{1.324244in}}%
\pgfpathlineto{\pgfqpoint{2.719814in}{1.198395in}}%
\pgfpathlineto{\pgfqpoint{2.721649in}{1.311221in}}%
\pgfpathlineto{\pgfqpoint{2.723486in}{1.338434in}}%
\pgfpathlineto{\pgfqpoint{2.725319in}{1.394088in}}%
\pgfpathlineto{\pgfqpoint{2.727153in}{1.236999in}}%
\pgfpathlineto{\pgfqpoint{2.728986in}{1.307934in}}%
\pgfpathlineto{\pgfqpoint{2.730821in}{1.280433in}}%
\pgfpathlineto{\pgfqpoint{2.732655in}{1.329940in}}%
\pgfpathlineto{\pgfqpoint{2.734489in}{1.281424in}}%
\pgfpathlineto{\pgfqpoint{2.736323in}{1.154584in}}%
\pgfpathlineto{\pgfqpoint{2.738156in}{1.132666in}}%
\pgfpathlineto{\pgfqpoint{2.739990in}{1.187254in}}%
\pgfpathlineto{\pgfqpoint{2.741824in}{1.146028in}}%
\pgfpathlineto{\pgfqpoint{2.743657in}{1.356763in}}%
\pgfpathlineto{\pgfqpoint{2.747324in}{2.163537in}}%
\pgfpathlineto{\pgfqpoint{2.749158in}{2.217748in}}%
\pgfpathlineto{\pgfqpoint{2.750992in}{2.141029in}}%
\pgfpathlineto{\pgfqpoint{2.754676in}{2.262789in}}%
\pgfpathlineto{\pgfqpoint{2.756510in}{2.219982in}}%
\pgfpathlineto{\pgfqpoint{2.758345in}{2.138558in}}%
\pgfpathlineto{\pgfqpoint{2.762014in}{2.354638in}}%
\pgfpathlineto{\pgfqpoint{2.763847in}{2.189005in}}%
\pgfpathlineto{\pgfqpoint{2.765683in}{2.189306in}}%
\pgfpathlineto{\pgfqpoint{2.767516in}{2.292171in}}%
\pgfpathlineto{\pgfqpoint{2.769350in}{2.295082in}}%
\pgfpathlineto{\pgfqpoint{2.771184in}{2.274657in}}%
\pgfpathlineto{\pgfqpoint{2.773018in}{2.351263in}}%
\pgfpathlineto{\pgfqpoint{2.776687in}{2.170262in}}%
\pgfpathlineto{\pgfqpoint{2.778521in}{2.590466in}}%
\pgfpathlineto{\pgfqpoint{2.780356in}{2.303626in}}%
\pgfpathlineto{\pgfqpoint{2.782191in}{2.256528in}}%
\pgfpathlineto{\pgfqpoint{2.784025in}{2.169383in}}%
\pgfpathlineto{\pgfqpoint{2.785858in}{1.276017in}}%
\pgfpathlineto{\pgfqpoint{2.787692in}{1.340253in}}%
\pgfpathlineto{\pgfqpoint{2.789527in}{1.261125in}}%
\pgfpathlineto{\pgfqpoint{2.791360in}{1.346338in}}%
\pgfpathlineto{\pgfqpoint{2.793195in}{1.252042in}}%
\pgfpathlineto{\pgfqpoint{2.795029in}{1.292829in}}%
\pgfpathlineto{\pgfqpoint{2.796863in}{1.220049in}}%
\pgfpathlineto{\pgfqpoint{2.798697in}{1.217640in}}%
\pgfpathlineto{\pgfqpoint{2.800531in}{1.317218in}}%
\pgfpathlineto{\pgfqpoint{2.802364in}{1.265968in}}%
\pgfpathlineto{\pgfqpoint{2.804199in}{1.392620in}}%
\pgfpathlineto{\pgfqpoint{2.806034in}{1.422881in}}%
\pgfpathlineto{\pgfqpoint{2.811535in}{1.242758in}}%
\pgfpathlineto{\pgfqpoint{2.813369in}{1.462539in}}%
\pgfpathlineto{\pgfqpoint{2.815202in}{1.126782in}}%
\pgfpathlineto{\pgfqpoint{2.817037in}{1.348546in}}%
\pgfpathlineto{\pgfqpoint{2.818870in}{1.245555in}}%
\pgfpathlineto{\pgfqpoint{2.820704in}{1.281286in}}%
\pgfpathlineto{\pgfqpoint{2.822539in}{1.267737in}}%
\pgfpathlineto{\pgfqpoint{2.826205in}{1.419092in}}%
\pgfpathlineto{\pgfqpoint{2.828040in}{1.313580in}}%
\pgfpathlineto{\pgfqpoint{2.831708in}{1.424638in}}%
\pgfpathlineto{\pgfqpoint{2.833542in}{1.194280in}}%
\pgfpathlineto{\pgfqpoint{2.835376in}{1.393197in}}%
\pgfpathlineto{\pgfqpoint{2.837209in}{1.312049in}}%
\pgfpathlineto{\pgfqpoint{2.839043in}{1.344983in}}%
\pgfpathlineto{\pgfqpoint{2.840878in}{1.414061in}}%
\pgfpathlineto{\pgfqpoint{2.842711in}{1.411100in}}%
\pgfpathlineto{\pgfqpoint{2.844547in}{1.444360in}}%
\pgfpathlineto{\pgfqpoint{2.846380in}{1.272755in}}%
\pgfpathlineto{\pgfqpoint{2.848214in}{1.269318in}}%
\pgfpathlineto{\pgfqpoint{2.851883in}{1.443181in}}%
\pgfpathlineto{\pgfqpoint{2.853718in}{1.241026in}}%
\pgfpathlineto{\pgfqpoint{2.857387in}{1.493666in}}%
\pgfpathlineto{\pgfqpoint{2.859250in}{1.399583in}}%
\pgfpathlineto{\pgfqpoint{2.861437in}{1.383323in}}%
\pgfpathlineto{\pgfqpoint{2.863271in}{1.552720in}}%
\pgfpathlineto{\pgfqpoint{2.866939in}{1.375871in}}%
\pgfpathlineto{\pgfqpoint{2.870607in}{1.266796in}}%
\pgfpathlineto{\pgfqpoint{2.874275in}{2.185242in}}%
\pgfpathlineto{\pgfqpoint{2.876108in}{1.586720in}}%
\pgfpathlineto{\pgfqpoint{2.877943in}{1.824217in}}%
\pgfpathlineto{\pgfqpoint{2.879777in}{1.612753in}}%
\pgfpathlineto{\pgfqpoint{2.881615in}{1.246935in}}%
\pgfpathlineto{\pgfqpoint{2.883449in}{1.504531in}}%
\pgfpathlineto{\pgfqpoint{2.885282in}{1.295978in}}%
\pgfpathlineto{\pgfqpoint{2.888951in}{1.228593in}}%
\pgfpathlineto{\pgfqpoint{2.892618in}{1.422116in}}%
\pgfpathlineto{\pgfqpoint{2.894452in}{1.428728in}}%
\pgfpathlineto{\pgfqpoint{2.896285in}{1.293117in}}%
\pgfpathlineto{\pgfqpoint{2.898120in}{1.445351in}}%
\pgfpathlineto{\pgfqpoint{2.899953in}{1.768688in}}%
\pgfpathlineto{\pgfqpoint{2.901787in}{1.678306in}}%
\pgfpathlineto{\pgfqpoint{2.903622in}{1.415153in}}%
\pgfpathlineto{\pgfqpoint{2.905456in}{1.437836in}}%
\pgfpathlineto{\pgfqpoint{2.907289in}{1.567324in}}%
\pgfpathlineto{\pgfqpoint{2.909124in}{1.300319in}}%
\pgfpathlineto{\pgfqpoint{2.910958in}{1.322149in}}%
\pgfpathlineto{\pgfqpoint{2.912791in}{1.422705in}}%
\pgfpathlineto{\pgfqpoint{2.914626in}{1.453581in}}%
\pgfpathlineto{\pgfqpoint{2.916459in}{1.364818in}}%
\pgfpathlineto{\pgfqpoint{2.918292in}{1.373186in}}%
\pgfpathlineto{\pgfqpoint{2.920127in}{1.415918in}}%
\pgfpathlineto{\pgfqpoint{2.921961in}{1.604322in}}%
\pgfpathlineto{\pgfqpoint{2.923795in}{1.547401in}}%
\pgfpathlineto{\pgfqpoint{2.925630in}{1.266821in}}%
\pgfpathlineto{\pgfqpoint{2.927464in}{1.285514in}}%
\pgfpathlineto{\pgfqpoint{2.929298in}{1.337581in}}%
\pgfpathlineto{\pgfqpoint{2.931134in}{1.538995in}}%
\pgfpathlineto{\pgfqpoint{2.932967in}{1.204153in}}%
\pgfpathlineto{\pgfqpoint{2.934799in}{1.228656in}}%
\pgfpathlineto{\pgfqpoint{2.936633in}{1.316215in}}%
\pgfpathlineto{\pgfqpoint{2.938468in}{1.276519in}}%
\pgfpathlineto{\pgfqpoint{2.940301in}{1.208118in}}%
\pgfpathlineto{\pgfqpoint{2.943970in}{1.382859in}}%
\pgfpathlineto{\pgfqpoint{2.945803in}{1.625036in}}%
\pgfpathlineto{\pgfqpoint{2.947636in}{1.333353in}}%
\pgfpathlineto{\pgfqpoint{2.949470in}{1.384980in}}%
\pgfpathlineto{\pgfqpoint{2.951303in}{1.399545in}}%
\pgfpathlineto{\pgfqpoint{2.954971in}{1.128727in}}%
\pgfpathlineto{\pgfqpoint{2.956806in}{1.183352in}}%
\pgfpathlineto{\pgfqpoint{2.958640in}{1.129166in}}%
\pgfpathlineto{\pgfqpoint{2.960474in}{1.364354in}}%
\pgfpathlineto{\pgfqpoint{2.962307in}{1.189048in}}%
\pgfpathlineto{\pgfqpoint{2.964141in}{1.322500in}}%
\pgfpathlineto{\pgfqpoint{2.965976in}{1.118150in}}%
\pgfpathlineto{\pgfqpoint{2.967811in}{1.460457in}}%
\pgfpathlineto{\pgfqpoint{2.969644in}{1.254187in}}%
\pgfpathlineto{\pgfqpoint{2.971478in}{1.235443in}}%
\pgfpathlineto{\pgfqpoint{2.973311in}{1.271764in}}%
\pgfpathlineto{\pgfqpoint{2.975146in}{1.344807in}}%
\pgfpathlineto{\pgfqpoint{2.976981in}{1.275026in}}%
\pgfpathlineto{\pgfqpoint{2.978814in}{1.300319in}}%
\pgfpathlineto{\pgfqpoint{2.980649in}{1.169740in}}%
\pgfpathlineto{\pgfqpoint{2.982484in}{1.399345in}}%
\pgfpathlineto{\pgfqpoint{2.984318in}{1.235506in}}%
\pgfpathlineto{\pgfqpoint{2.986151in}{1.231102in}}%
\pgfpathlineto{\pgfqpoint{2.987986in}{1.174796in}}%
\pgfpathlineto{\pgfqpoint{2.989819in}{1.318624in}}%
\pgfpathlineto{\pgfqpoint{2.991653in}{1.280960in}}%
\pgfpathlineto{\pgfqpoint{2.993487in}{1.282491in}}%
\pgfpathlineto{\pgfqpoint{2.995321in}{1.362346in}}%
\pgfpathlineto{\pgfqpoint{2.997154in}{1.356450in}}%
\pgfpathlineto{\pgfqpoint{2.998989in}{1.258867in}}%
\pgfpathlineto{\pgfqpoint{3.000823in}{1.441976in}}%
\pgfpathlineto{\pgfqpoint{3.002655in}{1.174093in}}%
\pgfpathlineto{\pgfqpoint{3.004488in}{1.242845in}}%
\pgfpathlineto{\pgfqpoint{3.006324in}{1.427021in}}%
\pgfpathlineto{\pgfqpoint{3.008156in}{1.451235in}}%
\pgfpathlineto{\pgfqpoint{3.015492in}{1.220689in}}%
\pgfpathlineto{\pgfqpoint{3.017327in}{1.163228in}}%
\pgfpathlineto{\pgfqpoint{3.019162in}{1.466955in}}%
\pgfpathlineto{\pgfqpoint{3.024663in}{1.341257in}}%
\pgfpathlineto{\pgfqpoint{3.028332in}{1.136455in}}%
\pgfpathlineto{\pgfqpoint{3.033835in}{1.361355in}}%
\pgfpathlineto{\pgfqpoint{3.035669in}{1.186024in}}%
\pgfpathlineto{\pgfqpoint{3.039337in}{2.118459in}}%
\pgfpathlineto{\pgfqpoint{3.041170in}{1.711001in}}%
\pgfpathlineto{\pgfqpoint{3.043003in}{1.868178in}}%
\pgfpathlineto{\pgfqpoint{3.046673in}{2.457654in}}%
\pgfpathlineto{\pgfqpoint{3.050343in}{1.904524in}}%
\pgfpathlineto{\pgfqpoint{3.052175in}{2.014766in}}%
\pgfpathlineto{\pgfqpoint{3.054009in}{2.029520in}}%
\pgfpathlineto{\pgfqpoint{3.055844in}{2.119400in}}%
\pgfpathlineto{\pgfqpoint{3.057678in}{2.118283in}}%
\pgfpathlineto{\pgfqpoint{3.059511in}{2.213131in}}%
\pgfpathlineto{\pgfqpoint{3.061345in}{2.242426in}}%
\pgfpathlineto{\pgfqpoint{3.063179in}{1.405191in}}%
\pgfpathlineto{\pgfqpoint{3.065013in}{1.188433in}}%
\pgfpathlineto{\pgfqpoint{3.066847in}{1.467896in}}%
\pgfpathlineto{\pgfqpoint{3.068682in}{1.312137in}}%
\pgfpathlineto{\pgfqpoint{3.070517in}{1.404363in}}%
\pgfpathlineto{\pgfqpoint{3.072352in}{1.429606in}}%
\pgfpathlineto{\pgfqpoint{3.076019in}{1.215909in}}%
\pgfpathlineto{\pgfqpoint{3.077854in}{1.211794in}}%
\pgfpathlineto{\pgfqpoint{3.079689in}{1.255454in}}%
\pgfpathlineto{\pgfqpoint{3.081521in}{1.115177in}}%
\pgfpathlineto{\pgfqpoint{3.085189in}{1.337693in}}%
\pgfpathlineto{\pgfqpoint{3.087023in}{1.201393in}}%
\pgfpathlineto{\pgfqpoint{3.088855in}{1.352448in}}%
\pgfpathlineto{\pgfqpoint{3.090689in}{1.256069in}}%
\pgfpathlineto{\pgfqpoint{3.092523in}{1.229772in}}%
\pgfpathlineto{\pgfqpoint{3.094357in}{1.420585in}}%
\pgfpathlineto{\pgfqpoint{3.096190in}{1.273671in}}%
\pgfpathlineto{\pgfqpoint{3.098024in}{1.340642in}}%
\pgfpathlineto{\pgfqpoint{3.099858in}{1.320455in}}%
\pgfpathlineto{\pgfqpoint{3.101693in}{1.257888in}}%
\pgfpathlineto{\pgfqpoint{3.103527in}{1.393937in}}%
\pgfpathlineto{\pgfqpoint{3.107195in}{2.219806in}}%
\pgfpathlineto{\pgfqpoint{3.109029in}{2.172871in}}%
\pgfpathlineto{\pgfqpoint{3.110862in}{2.395363in}}%
\pgfpathlineto{\pgfqpoint{3.112696in}{2.438998in}}%
\pgfpathlineto{\pgfqpoint{3.114530in}{2.311856in}}%
\pgfpathlineto{\pgfqpoint{3.116365in}{2.079905in}}%
\pgfpathlineto{\pgfqpoint{3.118199in}{2.026697in}}%
\pgfpathlineto{\pgfqpoint{3.120033in}{2.254408in}}%
\pgfpathlineto{\pgfqpoint{3.121867in}{2.274306in}}%
\pgfpathlineto{\pgfqpoint{3.123701in}{2.375615in}}%
\pgfpathlineto{\pgfqpoint{3.125535in}{2.372704in}}%
\pgfpathlineto{\pgfqpoint{3.127369in}{2.071474in}}%
\pgfpathlineto{\pgfqpoint{3.132873in}{2.440792in}}%
\pgfpathlineto{\pgfqpoint{3.134707in}{1.208708in}}%
\pgfpathlineto{\pgfqpoint{3.136542in}{1.386498in}}%
\pgfpathlineto{\pgfqpoint{3.138375in}{1.210878in}}%
\pgfpathlineto{\pgfqpoint{3.140209in}{1.424136in}}%
\pgfpathlineto{\pgfqpoint{3.142043in}{1.361932in}}%
\pgfpathlineto{\pgfqpoint{3.143877in}{1.405304in}}%
\pgfpathlineto{\pgfqpoint{3.145710in}{1.349926in}}%
\pgfpathlineto{\pgfqpoint{3.147544in}{1.442152in}}%
\pgfpathlineto{\pgfqpoint{3.149377in}{1.307959in}}%
\pgfpathlineto{\pgfqpoint{3.151211in}{1.360678in}}%
\pgfpathlineto{\pgfqpoint{3.153045in}{1.355509in}}%
\pgfpathlineto{\pgfqpoint{3.154878in}{1.408654in}}%
\pgfpathlineto{\pgfqpoint{3.156713in}{1.553373in}}%
\pgfpathlineto{\pgfqpoint{3.158548in}{1.235381in}}%
\pgfpathlineto{\pgfqpoint{3.164050in}{1.501507in}}%
\pgfpathlineto{\pgfqpoint{3.165884in}{1.799451in}}%
\pgfpathlineto{\pgfqpoint{3.167717in}{1.806150in}}%
\pgfpathlineto{\pgfqpoint{3.169551in}{1.397927in}}%
\pgfpathlineto{\pgfqpoint{3.171384in}{1.345485in}}%
\pgfpathlineto{\pgfqpoint{3.173218in}{1.433156in}}%
\pgfpathlineto{\pgfqpoint{3.175055in}{1.248755in}}%
\pgfpathlineto{\pgfqpoint{3.176889in}{1.566885in}}%
\pgfpathlineto{\pgfqpoint{3.178722in}{1.473504in}}%
\pgfpathlineto{\pgfqpoint{3.180569in}{1.252368in}}%
\pgfpathlineto{\pgfqpoint{3.182403in}{1.340604in}}%
\pgfpathlineto{\pgfqpoint{3.184237in}{1.269995in}}%
\pgfpathlineto{\pgfqpoint{3.186071in}{1.435603in}}%
\pgfpathlineto{\pgfqpoint{3.187905in}{2.051701in}}%
\pgfpathlineto{\pgfqpoint{3.189739in}{2.101421in}}%
\pgfpathlineto{\pgfqpoint{3.191573in}{2.276301in}}%
\pgfpathlineto{\pgfqpoint{3.193406in}{2.291431in}}%
\pgfpathlineto{\pgfqpoint{3.195241in}{2.229843in}}%
\pgfpathlineto{\pgfqpoint{3.197075in}{2.045504in}}%
\pgfpathlineto{\pgfqpoint{3.198909in}{2.223883in}}%
\pgfpathlineto{\pgfqpoint{3.202579in}{2.192067in}}%
\pgfpathlineto{\pgfqpoint{3.204428in}{2.160037in}}%
\pgfpathlineto{\pgfqpoint{3.206260in}{2.211990in}}%
\pgfpathlineto{\pgfqpoint{3.208094in}{2.057636in}}%
\pgfpathlineto{\pgfqpoint{3.209927in}{2.248210in}}%
\pgfpathlineto{\pgfqpoint{3.211761in}{1.515847in}}%
\pgfpathlineto{\pgfqpoint{3.213595in}{1.387526in}}%
\pgfpathlineto{\pgfqpoint{3.215429in}{1.343603in}}%
\pgfpathlineto{\pgfqpoint{3.217263in}{1.178710in}}%
\pgfpathlineto{\pgfqpoint{3.222765in}{1.463869in}}%
\pgfpathlineto{\pgfqpoint{3.224599in}{1.310230in}}%
\pgfpathlineto{\pgfqpoint{3.226432in}{1.311610in}}%
\pgfpathlineto{\pgfqpoint{3.228266in}{1.445765in}}%
\pgfpathlineto{\pgfqpoint{3.230099in}{1.192021in}}%
\pgfpathlineto{\pgfqpoint{3.231933in}{1.092619in}}%
\pgfpathlineto{\pgfqpoint{3.235603in}{1.318097in}}%
\pgfpathlineto{\pgfqpoint{3.237436in}{1.130169in}}%
\pgfpathlineto{\pgfqpoint{3.239271in}{1.210702in}}%
\pgfpathlineto{\pgfqpoint{3.241104in}{1.145237in}}%
\pgfpathlineto{\pgfqpoint{3.242938in}{1.394088in}}%
\pgfpathlineto{\pgfqpoint{3.244772in}{1.172562in}}%
\pgfpathlineto{\pgfqpoint{3.246606in}{1.132252in}}%
\pgfpathlineto{\pgfqpoint{3.248439in}{1.143267in}}%
\pgfpathlineto{\pgfqpoint{3.250273in}{1.583220in}}%
\pgfpathlineto{\pgfqpoint{3.252109in}{1.613481in}}%
\pgfpathlineto{\pgfqpoint{3.253942in}{1.321685in}}%
\pgfpathlineto{\pgfqpoint{3.255778in}{1.368870in}}%
\pgfpathlineto{\pgfqpoint{3.257612in}{1.342511in}}%
\pgfpathlineto{\pgfqpoint{3.259444in}{1.284047in}}%
\pgfpathlineto{\pgfqpoint{3.261278in}{1.122993in}}%
\pgfpathlineto{\pgfqpoint{3.263112in}{1.341219in}}%
\pgfpathlineto{\pgfqpoint{3.264945in}{2.062253in}}%
\pgfpathlineto{\pgfqpoint{3.266779in}{1.943743in}}%
\pgfpathlineto{\pgfqpoint{3.268613in}{2.065251in}}%
\pgfpathlineto{\pgfqpoint{3.270446in}{1.999108in}}%
\pgfpathlineto{\pgfqpoint{3.272279in}{2.039971in}}%
\pgfpathlineto{\pgfqpoint{3.274116in}{2.165946in}}%
\pgfpathlineto{\pgfqpoint{3.275950in}{2.197474in}}%
\pgfpathlineto{\pgfqpoint{3.277784in}{1.971633in}}%
\pgfpathlineto{\pgfqpoint{3.279618in}{1.219485in}}%
\pgfpathlineto{\pgfqpoint{3.283286in}{1.114951in}}%
\pgfpathlineto{\pgfqpoint{3.285121in}{1.140156in}}%
\pgfpathlineto{\pgfqpoint{3.286954in}{1.291449in}}%
\pgfpathlineto{\pgfqpoint{3.290622in}{1.146115in}}%
\pgfpathlineto{\pgfqpoint{3.292455in}{1.202359in}}%
\pgfpathlineto{\pgfqpoint{3.294288in}{1.189550in}}%
\pgfpathlineto{\pgfqpoint{3.296121in}{1.245204in}}%
\pgfpathlineto{\pgfqpoint{3.297955in}{1.365169in}}%
\pgfpathlineto{\pgfqpoint{3.299789in}{1.406948in}}%
\pgfpathlineto{\pgfqpoint{3.301622in}{1.333942in}}%
\pgfpathlineto{\pgfqpoint{3.303456in}{1.128074in}}%
\pgfpathlineto{\pgfqpoint{3.307123in}{1.334231in}}%
\pgfpathlineto{\pgfqpoint{3.308957in}{1.215344in}}%
\pgfpathlineto{\pgfqpoint{3.312625in}{1.255040in}}%
\pgfpathlineto{\pgfqpoint{3.314460in}{1.193427in}}%
\pgfpathlineto{\pgfqpoint{3.316293in}{1.323215in}}%
\pgfpathlineto{\pgfqpoint{3.318126in}{1.115302in}}%
\pgfpathlineto{\pgfqpoint{3.323628in}{1.260159in}}%
\pgfpathlineto{\pgfqpoint{3.325463in}{2.005369in}}%
\pgfpathlineto{\pgfqpoint{3.327297in}{2.055603in}}%
\pgfpathlineto{\pgfqpoint{3.329130in}{2.170763in}}%
\pgfpathlineto{\pgfqpoint{3.330965in}{2.090895in}}%
\pgfpathlineto{\pgfqpoint{3.332797in}{2.090657in}}%
\pgfpathlineto{\pgfqpoint{3.334631in}{2.024966in}}%
\pgfpathlineto{\pgfqpoint{3.336466in}{2.023372in}}%
\pgfpathlineto{\pgfqpoint{3.338299in}{2.009471in}}%
\pgfpathlineto{\pgfqpoint{3.340132in}{2.154215in}}%
\pgfpathlineto{\pgfqpoint{3.343803in}{2.003600in}}%
\pgfpathlineto{\pgfqpoint{3.345637in}{2.074146in}}%
\pgfpathlineto{\pgfqpoint{3.349319in}{2.099677in}}%
\pgfpathlineto{\pgfqpoint{3.351153in}{2.242803in}}%
\pgfpathlineto{\pgfqpoint{3.352986in}{2.098297in}}%
\pgfpathlineto{\pgfqpoint{3.354819in}{2.155984in}}%
\pgfpathlineto{\pgfqpoint{3.356653in}{1.976952in}}%
\pgfpathlineto{\pgfqpoint{3.358488in}{1.399784in}}%
\pgfpathlineto{\pgfqpoint{3.360321in}{1.198508in}}%
\pgfpathlineto{\pgfqpoint{3.362156in}{1.255655in}}%
\pgfpathlineto{\pgfqpoint{3.363989in}{1.180943in}}%
\pgfpathlineto{\pgfqpoint{3.365824in}{1.160581in}}%
\pgfpathlineto{\pgfqpoint{3.367658in}{1.330818in}}%
\pgfpathlineto{\pgfqpoint{3.369491in}{1.255780in}}%
\pgfpathlineto{\pgfqpoint{3.371324in}{1.307081in}}%
\pgfpathlineto{\pgfqpoint{3.373159in}{1.420610in}}%
\pgfpathlineto{\pgfqpoint{3.374993in}{1.362698in}}%
\pgfpathlineto{\pgfqpoint{3.376827in}{1.266859in}}%
\pgfpathlineto{\pgfqpoint{3.378660in}{1.417411in}}%
\pgfpathlineto{\pgfqpoint{3.380495in}{1.195748in}}%
\pgfpathlineto{\pgfqpoint{3.382330in}{1.266683in}}%
\pgfpathlineto{\pgfqpoint{3.384164in}{1.226071in}}%
\pgfpathlineto{\pgfqpoint{3.386001in}{1.245932in}}%
\pgfpathlineto{\pgfqpoint{3.387834in}{1.332788in}}%
\pgfpathlineto{\pgfqpoint{3.389668in}{1.263421in}}%
\pgfpathlineto{\pgfqpoint{3.391503in}{1.124825in}}%
\pgfpathlineto{\pgfqpoint{3.393338in}{1.261037in}}%
\pgfpathlineto{\pgfqpoint{3.395171in}{1.265391in}}%
\pgfpathlineto{\pgfqpoint{3.397006in}{1.099670in}}%
\pgfpathlineto{\pgfqpoint{3.398840in}{1.234327in}}%
\pgfpathlineto{\pgfqpoint{3.400674in}{1.287007in}}%
\pgfpathlineto{\pgfqpoint{3.402510in}{1.138952in}}%
\pgfpathlineto{\pgfqpoint{3.404343in}{1.092494in}}%
\pgfpathlineto{\pgfqpoint{3.406177in}{1.233599in}}%
\pgfpathlineto{\pgfqpoint{3.408012in}{1.111513in}}%
\pgfpathlineto{\pgfqpoint{3.409845in}{1.350741in}}%
\pgfpathlineto{\pgfqpoint{3.411678in}{1.161785in}}%
\pgfpathlineto{\pgfqpoint{3.413513in}{1.243197in}}%
\pgfpathlineto{\pgfqpoint{3.415347in}{1.195045in}}%
\pgfpathlineto{\pgfqpoint{3.417181in}{1.086647in}}%
\pgfpathlineto{\pgfqpoint{3.419015in}{1.252493in}}%
\pgfpathlineto{\pgfqpoint{3.420849in}{1.223073in}}%
\pgfpathlineto{\pgfqpoint{3.422686in}{1.018271in}}%
\pgfpathlineto{\pgfqpoint{3.424521in}{1.031031in}}%
\pgfpathlineto{\pgfqpoint{3.426354in}{1.276168in}}%
\pgfpathlineto{\pgfqpoint{3.428187in}{1.104488in}}%
\pgfpathlineto{\pgfqpoint{3.430022in}{1.161020in}}%
\pgfpathlineto{\pgfqpoint{3.431855in}{1.147972in}}%
\pgfpathlineto{\pgfqpoint{3.435523in}{1.417499in}}%
\pgfpathlineto{\pgfqpoint{3.437356in}{1.311723in}}%
\pgfpathlineto{\pgfqpoint{3.439189in}{1.349951in}}%
\pgfpathlineto{\pgfqpoint{3.441023in}{1.173089in}}%
\pgfpathlineto{\pgfqpoint{3.442856in}{1.293331in}}%
\pgfpathlineto{\pgfqpoint{3.444691in}{1.658797in}}%
\pgfpathlineto{\pgfqpoint{3.446524in}{2.249101in}}%
\pgfpathlineto{\pgfqpoint{3.448382in}{2.235049in}}%
\pgfpathlineto{\pgfqpoint{3.450217in}{2.271746in}}%
\pgfpathlineto{\pgfqpoint{3.452050in}{2.094069in}}%
\pgfpathlineto{\pgfqpoint{3.455718in}{2.261961in}}%
\pgfpathlineto{\pgfqpoint{3.457551in}{2.291318in}}%
\pgfpathlineto{\pgfqpoint{3.461218in}{2.217422in}}%
\pgfpathlineto{\pgfqpoint{3.463056in}{2.192267in}}%
\pgfpathlineto{\pgfqpoint{3.464890in}{2.139825in}}%
\pgfpathlineto{\pgfqpoint{3.466727in}{2.141054in}}%
\pgfpathlineto{\pgfqpoint{3.468561in}{2.050522in}}%
\pgfpathlineto{\pgfqpoint{3.470395in}{2.179872in}}%
\pgfpathlineto{\pgfqpoint{3.474066in}{2.010588in}}%
\pgfpathlineto{\pgfqpoint{3.475899in}{2.344062in}}%
\pgfpathlineto{\pgfqpoint{3.477733in}{2.143965in}}%
\pgfpathlineto{\pgfqpoint{3.479567in}{2.309623in}}%
\pgfpathlineto{\pgfqpoint{3.481400in}{2.241398in}}%
\pgfpathlineto{\pgfqpoint{3.483235in}{2.483097in}}%
\pgfpathlineto{\pgfqpoint{3.486903in}{2.232051in}}%
\pgfpathlineto{\pgfqpoint{3.488739in}{2.200937in}}%
\pgfpathlineto{\pgfqpoint{3.490574in}{2.377409in}}%
\pgfpathlineto{\pgfqpoint{3.492407in}{2.192418in}}%
\pgfpathlineto{\pgfqpoint{3.494242in}{2.152484in}}%
\pgfpathlineto{\pgfqpoint{3.496076in}{2.186421in}}%
\pgfpathlineto{\pgfqpoint{3.497910in}{2.158807in}}%
\pgfpathlineto{\pgfqpoint{3.499744in}{2.431533in}}%
\pgfpathlineto{\pgfqpoint{3.501578in}{2.351815in}}%
\pgfpathlineto{\pgfqpoint{3.503411in}{2.183046in}}%
\pgfpathlineto{\pgfqpoint{3.505250in}{2.229730in}}%
\pgfpathlineto{\pgfqpoint{3.507084in}{2.382290in}}%
\pgfpathlineto{\pgfqpoint{3.508918in}{2.074974in}}%
\pgfpathlineto{\pgfqpoint{3.510752in}{2.204751in}}%
\pgfpathlineto{\pgfqpoint{3.512586in}{2.241423in}}%
\pgfpathlineto{\pgfqpoint{3.516251in}{1.332085in}}%
\pgfpathlineto{\pgfqpoint{3.518086in}{1.356174in}}%
\pgfpathlineto{\pgfqpoint{3.519919in}{1.276469in}}%
\pgfpathlineto{\pgfqpoint{3.521752in}{1.257487in}}%
\pgfpathlineto{\pgfqpoint{3.523588in}{1.351394in}}%
\pgfpathlineto{\pgfqpoint{3.525421in}{1.387326in}}%
\pgfpathlineto{\pgfqpoint{3.527256in}{1.190102in}}%
\pgfpathlineto{\pgfqpoint{3.529091in}{1.186614in}}%
\pgfpathlineto{\pgfqpoint{3.530924in}{1.361707in}}%
\pgfpathlineto{\pgfqpoint{3.532759in}{1.217929in}}%
\pgfpathlineto{\pgfqpoint{3.534593in}{1.300557in}}%
\pgfpathlineto{\pgfqpoint{3.536425in}{1.128701in}}%
\pgfpathlineto{\pgfqpoint{3.538258in}{1.259067in}}%
\pgfpathlineto{\pgfqpoint{3.540092in}{1.187379in}}%
\pgfpathlineto{\pgfqpoint{3.541927in}{1.243761in}}%
\pgfpathlineto{\pgfqpoint{3.543760in}{1.079860in}}%
\pgfpathlineto{\pgfqpoint{3.545594in}{1.216173in}}%
\pgfpathlineto{\pgfqpoint{3.547428in}{1.206299in}}%
\pgfpathlineto{\pgfqpoint{3.549263in}{1.213262in}}%
\pgfpathlineto{\pgfqpoint{3.551099in}{1.177380in}}%
\pgfpathlineto{\pgfqpoint{3.554765in}{1.013893in}}%
\pgfpathlineto{\pgfqpoint{3.556600in}{1.177531in}}%
\pgfpathlineto{\pgfqpoint{3.558433in}{0.895195in}}%
\pgfpathlineto{\pgfqpoint{3.562100in}{1.285339in}}%
\pgfpathlineto{\pgfqpoint{3.563934in}{1.035459in}}%
\pgfpathlineto{\pgfqpoint{3.565768in}{1.001083in}}%
\pgfpathlineto{\pgfqpoint{3.567603in}{1.187204in}}%
\pgfpathlineto{\pgfqpoint{3.569436in}{0.962567in}}%
\pgfpathlineto{\pgfqpoint{3.573103in}{1.134335in}}%
\pgfpathlineto{\pgfqpoint{3.576774in}{1.013140in}}%
\pgfpathlineto{\pgfqpoint{3.578608in}{1.080186in}}%
\pgfpathlineto{\pgfqpoint{3.580443in}{1.216173in}}%
\pgfpathlineto{\pgfqpoint{3.584114in}{1.005324in}}%
\pgfpathlineto{\pgfqpoint{3.585947in}{1.240963in}}%
\pgfpathlineto{\pgfqpoint{3.587780in}{1.265880in}}%
\pgfpathlineto{\pgfqpoint{3.589616in}{1.274901in}}%
\pgfpathlineto{\pgfqpoint{3.591449in}{1.469063in}}%
\pgfpathlineto{\pgfqpoint{3.595117in}{1.342486in}}%
\pgfpathlineto{\pgfqpoint{3.602455in}{0.984284in}}%
\pgfpathlineto{\pgfqpoint{3.604289in}{1.122403in}}%
\pgfpathlineto{\pgfqpoint{3.606124in}{1.041456in}}%
\pgfpathlineto{\pgfqpoint{3.607958in}{1.108302in}}%
\pgfpathlineto{\pgfqpoint{3.609790in}{1.069396in}}%
\pgfpathlineto{\pgfqpoint{3.611625in}{1.107750in}}%
\pgfpathlineto{\pgfqpoint{3.613459in}{1.110685in}}%
\pgfpathlineto{\pgfqpoint{3.615293in}{1.064265in}}%
\pgfpathlineto{\pgfqpoint{3.617126in}{1.049486in}}%
\pgfpathlineto{\pgfqpoint{3.618961in}{1.066548in}}%
\pgfpathlineto{\pgfqpoint{3.620794in}{1.072633in}}%
\pgfpathlineto{\pgfqpoint{3.622628in}{1.136191in}}%
\pgfpathlineto{\pgfqpoint{3.624462in}{1.414588in}}%
\pgfpathlineto{\pgfqpoint{3.626295in}{1.142364in}}%
\pgfpathlineto{\pgfqpoint{3.628128in}{1.125314in}}%
\pgfpathlineto{\pgfqpoint{3.631796in}{1.388154in}}%
\pgfpathlineto{\pgfqpoint{3.633629in}{1.422053in}}%
\pgfpathlineto{\pgfqpoint{3.635464in}{1.361757in}}%
\pgfpathlineto{\pgfqpoint{3.637298in}{1.382420in}}%
\pgfpathlineto{\pgfqpoint{3.639130in}{1.091879in}}%
\pgfpathlineto{\pgfqpoint{3.640965in}{1.317507in}}%
\pgfpathlineto{\pgfqpoint{3.644632in}{1.169740in}}%
\pgfpathlineto{\pgfqpoint{3.646465in}{1.215232in}}%
\pgfpathlineto{\pgfqpoint{3.648299in}{1.142151in}}%
\pgfpathlineto{\pgfqpoint{3.650133in}{1.251552in}}%
\pgfpathlineto{\pgfqpoint{3.651968in}{1.235092in}}%
\pgfpathlineto{\pgfqpoint{3.653803in}{1.106784in}}%
\pgfpathlineto{\pgfqpoint{3.655636in}{1.222019in}}%
\pgfpathlineto{\pgfqpoint{3.657471in}{1.034293in}}%
\pgfpathlineto{\pgfqpoint{3.662969in}{1.721025in}}%
\pgfpathlineto{\pgfqpoint{3.664803in}{1.622213in}}%
\pgfpathlineto{\pgfqpoint{3.670303in}{2.439111in}}%
\pgfpathlineto{\pgfqpoint{3.672137in}{2.351728in}}%
\pgfpathlineto{\pgfqpoint{3.673971in}{2.581684in}}%
\pgfpathlineto{\pgfqpoint{3.675806in}{2.414985in}}%
\pgfpathlineto{\pgfqpoint{3.677640in}{2.567018in}}%
\pgfpathlineto{\pgfqpoint{3.679474in}{2.565612in}}%
\pgfpathlineto{\pgfqpoint{3.681309in}{2.517687in}}%
\pgfpathlineto{\pgfqpoint{3.683143in}{2.542565in}}%
\pgfpathlineto{\pgfqpoint{3.684977in}{2.494790in}}%
\pgfpathlineto{\pgfqpoint{3.686813in}{2.498090in}}%
\pgfpathlineto{\pgfqpoint{3.688646in}{2.614793in}}%
\pgfpathlineto{\pgfqpoint{3.690479in}{2.633926in}}%
\pgfpathlineto{\pgfqpoint{3.692315in}{2.509481in}}%
\pgfpathlineto{\pgfqpoint{3.694149in}{2.495969in}}%
\pgfpathlineto{\pgfqpoint{3.695982in}{2.643448in}}%
\pgfpathlineto{\pgfqpoint{3.697816in}{2.486359in}}%
\pgfpathlineto{\pgfqpoint{3.699648in}{2.614994in}}%
\pgfpathlineto{\pgfqpoint{3.701483in}{2.419689in}}%
\pgfpathlineto{\pgfqpoint{3.703318in}{2.683457in}}%
\pgfpathlineto{\pgfqpoint{3.705152in}{2.482545in}}%
\pgfpathlineto{\pgfqpoint{3.706986in}{2.576778in}}%
\pgfpathlineto{\pgfqpoint{3.708820in}{2.414320in}}%
\pgfpathlineto{\pgfqpoint{3.710655in}{2.377786in}}%
\pgfpathlineto{\pgfqpoint{3.712488in}{2.541361in}}%
\pgfpathlineto{\pgfqpoint{3.714321in}{2.452134in}}%
\pgfpathlineto{\pgfqpoint{3.716156in}{2.436740in}}%
\pgfpathlineto{\pgfqpoint{3.717990in}{2.389453in}}%
\pgfpathlineto{\pgfqpoint{3.719822in}{2.598069in}}%
\pgfpathlineto{\pgfqpoint{3.721657in}{2.423541in}}%
\pgfpathlineto{\pgfqpoint{3.723491in}{2.546593in}}%
\pgfpathlineto{\pgfqpoint{3.725324in}{2.422976in}}%
\pgfpathlineto{\pgfqpoint{3.727159in}{2.712878in}}%
\pgfpathlineto{\pgfqpoint{3.728993in}{2.478957in}}%
\pgfpathlineto{\pgfqpoint{3.732661in}{2.370534in}}%
\pgfpathlineto{\pgfqpoint{3.734494in}{2.388011in}}%
\pgfpathlineto{\pgfqpoint{3.736327in}{2.574219in}}%
\pgfpathlineto{\pgfqpoint{3.738163in}{2.532077in}}%
\pgfpathlineto{\pgfqpoint{3.739999in}{2.572337in}}%
\pgfpathlineto{\pgfqpoint{3.741832in}{2.541750in}}%
\pgfpathlineto{\pgfqpoint{3.745501in}{2.604192in}}%
\pgfpathlineto{\pgfqpoint{3.747335in}{2.498315in}}%
\pgfpathlineto{\pgfqpoint{3.749169in}{2.535866in}}%
\pgfpathlineto{\pgfqpoint{3.751003in}{2.519970in}}%
\pgfpathlineto{\pgfqpoint{3.752838in}{2.389717in}}%
\pgfpathlineto{\pgfqpoint{3.754673in}{2.579237in}}%
\pgfpathlineto{\pgfqpoint{3.758339in}{2.384899in}}%
\pgfpathlineto{\pgfqpoint{3.760174in}{2.417218in}}%
\pgfpathlineto{\pgfqpoint{3.762009in}{2.410430in}}%
\pgfpathlineto{\pgfqpoint{3.763842in}{2.305006in}}%
\pgfpathlineto{\pgfqpoint{3.765677in}{2.378902in}}%
\pgfpathlineto{\pgfqpoint{3.767511in}{2.337688in}}%
\pgfpathlineto{\pgfqpoint{3.769345in}{2.527585in}}%
\pgfpathlineto{\pgfqpoint{3.771179in}{2.469961in}}%
\pgfpathlineto{\pgfqpoint{3.773011in}{2.506433in}}%
\pgfpathlineto{\pgfqpoint{3.774845in}{2.394484in}}%
\pgfpathlineto{\pgfqpoint{3.776679in}{2.403267in}}%
\pgfpathlineto{\pgfqpoint{3.778514in}{2.543720in}}%
\pgfpathlineto{\pgfqpoint{3.780349in}{2.486974in}}%
\pgfpathlineto{\pgfqpoint{3.782186in}{2.637828in}}%
\pgfpathlineto{\pgfqpoint{3.784019in}{1.627557in}}%
\pgfpathlineto{\pgfqpoint{3.787687in}{1.566232in}}%
\pgfpathlineto{\pgfqpoint{3.789522in}{1.507027in}}%
\pgfpathlineto{\pgfqpoint{3.791355in}{1.594674in}}%
\pgfpathlineto{\pgfqpoint{3.793189in}{1.462138in}}%
\pgfpathlineto{\pgfqpoint{3.795023in}{1.574588in}}%
\pgfpathlineto{\pgfqpoint{3.796856in}{1.554803in}}%
\pgfpathlineto{\pgfqpoint{3.798692in}{1.590835in}}%
\pgfpathlineto{\pgfqpoint{3.802362in}{1.383035in}}%
\pgfpathlineto{\pgfqpoint{3.804196in}{1.489312in}}%
\pgfpathlineto{\pgfqpoint{3.806032in}{1.387827in}}%
\pgfpathlineto{\pgfqpoint{3.807865in}{1.401816in}}%
\pgfpathlineto{\pgfqpoint{3.809697in}{1.539760in}}%
\pgfpathlineto{\pgfqpoint{3.811531in}{1.429694in}}%
\pgfpathlineto{\pgfqpoint{3.813365in}{1.393611in}}%
\pgfpathlineto{\pgfqpoint{3.815198in}{1.600583in}}%
\pgfpathlineto{\pgfqpoint{3.817032in}{1.399081in}}%
\pgfpathlineto{\pgfqpoint{3.818866in}{1.481057in}}%
\pgfpathlineto{\pgfqpoint{3.820699in}{1.312049in}}%
\pgfpathlineto{\pgfqpoint{3.822534in}{1.432717in}}%
\pgfpathlineto{\pgfqpoint{3.824367in}{1.416621in}}%
\pgfpathlineto{\pgfqpoint{3.826201in}{1.426432in}}%
\pgfpathlineto{\pgfqpoint{3.828036in}{1.360640in}}%
\pgfpathlineto{\pgfqpoint{3.829870in}{1.483842in}}%
\pgfpathlineto{\pgfqpoint{3.831703in}{1.433922in}}%
\pgfpathlineto{\pgfqpoint{3.833538in}{1.441299in}}%
\pgfpathlineto{\pgfqpoint{3.835371in}{1.568792in}}%
\pgfpathlineto{\pgfqpoint{3.837205in}{1.524742in}}%
\pgfpathlineto{\pgfqpoint{3.839039in}{1.433608in}}%
\pgfpathlineto{\pgfqpoint{3.840875in}{1.514643in}}%
\pgfpathlineto{\pgfqpoint{3.842708in}{1.358470in}}%
\pgfpathlineto{\pgfqpoint{3.844542in}{1.567211in}}%
\pgfpathlineto{\pgfqpoint{3.848208in}{1.461636in}}%
\pgfpathlineto{\pgfqpoint{3.850044in}{1.552921in}}%
\pgfpathlineto{\pgfqpoint{3.851891in}{1.584838in}}%
\pgfpathlineto{\pgfqpoint{3.853727in}{1.500328in}}%
\pgfpathlineto{\pgfqpoint{3.855561in}{1.478560in}}%
\pgfpathlineto{\pgfqpoint{3.857394in}{1.340730in}}%
\pgfpathlineto{\pgfqpoint{3.861063in}{1.523450in}}%
\pgfpathlineto{\pgfqpoint{3.862896in}{1.620067in}}%
\pgfpathlineto{\pgfqpoint{3.866563in}{1.322036in}}%
\pgfpathlineto{\pgfqpoint{3.868397in}{1.460983in}}%
\pgfpathlineto{\pgfqpoint{3.870230in}{1.477996in}}%
\pgfpathlineto{\pgfqpoint{3.872064in}{1.483491in}}%
\pgfpathlineto{\pgfqpoint{3.873898in}{1.588125in}}%
\pgfpathlineto{\pgfqpoint{3.877566in}{1.393022in}}%
\pgfpathlineto{\pgfqpoint{3.879399in}{1.462903in}}%
\pgfpathlineto{\pgfqpoint{3.881233in}{1.443921in}}%
\pgfpathlineto{\pgfqpoint{3.883067in}{1.365194in}}%
\pgfpathlineto{\pgfqpoint{3.884901in}{1.440508in}}%
\pgfpathlineto{\pgfqpoint{3.886734in}{1.327330in}}%
\pgfpathlineto{\pgfqpoint{3.888568in}{1.367553in}}%
\pgfpathlineto{\pgfqpoint{3.890402in}{1.552921in}}%
\pgfpathlineto{\pgfqpoint{3.894071in}{1.421639in}}%
\pgfpathlineto{\pgfqpoint{3.895904in}{1.447007in}}%
\pgfpathlineto{\pgfqpoint{3.897738in}{1.542056in}}%
\pgfpathlineto{\pgfqpoint{3.899574in}{1.356676in}}%
\pgfpathlineto{\pgfqpoint{3.901409in}{1.340228in}}%
\pgfpathlineto{\pgfqpoint{3.905078in}{1.476440in}}%
\pgfpathlineto{\pgfqpoint{3.906913in}{1.320919in}}%
\pgfpathlineto{\pgfqpoint{3.908748in}{1.358382in}}%
\pgfpathlineto{\pgfqpoint{3.910582in}{1.442829in}}%
\pgfpathlineto{\pgfqpoint{3.912415in}{1.597146in}}%
\pgfpathlineto{\pgfqpoint{3.914249in}{1.323128in}}%
\pgfpathlineto{\pgfqpoint{3.916085in}{1.429016in}}%
\pgfpathlineto{\pgfqpoint{3.917917in}{1.452264in}}%
\pgfpathlineto{\pgfqpoint{3.919750in}{1.408955in}}%
\pgfpathlineto{\pgfqpoint{3.923418in}{1.530802in}}%
\pgfpathlineto{\pgfqpoint{3.925252in}{1.457696in}}%
\pgfpathlineto{\pgfqpoint{3.927086in}{1.433483in}}%
\pgfpathlineto{\pgfqpoint{3.928919in}{1.599090in}}%
\pgfpathlineto{\pgfqpoint{3.932587in}{1.333528in}}%
\pgfpathlineto{\pgfqpoint{3.934422in}{1.484733in}}%
\pgfpathlineto{\pgfqpoint{3.936255in}{1.262392in}}%
\pgfpathlineto{\pgfqpoint{3.938088in}{1.510051in}}%
\pgfpathlineto{\pgfqpoint{3.939922in}{1.434160in}}%
\pgfpathlineto{\pgfqpoint{3.941755in}{1.457232in}}%
\pgfpathlineto{\pgfqpoint{3.943588in}{1.444448in}}%
\pgfpathlineto{\pgfqpoint{3.945425in}{1.465538in}}%
\pgfpathlineto{\pgfqpoint{3.947259in}{1.359210in}}%
\pgfpathlineto{\pgfqpoint{3.950929in}{1.514467in}}%
\pgfpathlineto{\pgfqpoint{3.952762in}{1.422843in}}%
\pgfpathlineto{\pgfqpoint{3.954594in}{1.427172in}}%
\pgfpathlineto{\pgfqpoint{3.956429in}{1.598262in}}%
\pgfpathlineto{\pgfqpoint{3.958263in}{1.519486in}}%
\pgfpathlineto{\pgfqpoint{3.960096in}{1.546196in}}%
\pgfpathlineto{\pgfqpoint{3.961930in}{1.552281in}}%
\pgfpathlineto{\pgfqpoint{3.963764in}{1.442478in}}%
\pgfpathlineto{\pgfqpoint{3.965599in}{1.600671in}}%
\pgfpathlineto{\pgfqpoint{3.967434in}{1.558504in}}%
\pgfpathlineto{\pgfqpoint{3.969267in}{1.594235in}}%
\pgfpathlineto{\pgfqpoint{3.971101in}{1.607371in}}%
\pgfpathlineto{\pgfqpoint{3.972938in}{1.422994in}}%
\pgfpathlineto{\pgfqpoint{3.974772in}{1.639225in}}%
\pgfpathlineto{\pgfqpoint{3.976604in}{1.614572in}}%
\pgfpathlineto{\pgfqpoint{3.980273in}{1.810905in}}%
\pgfpathlineto{\pgfqpoint{3.982106in}{1.556358in}}%
\pgfpathlineto{\pgfqpoint{3.983941in}{1.637080in}}%
\pgfpathlineto{\pgfqpoint{3.987610in}{1.467721in}}%
\pgfpathlineto{\pgfqpoint{3.989444in}{1.530150in}}%
\pgfpathlineto{\pgfqpoint{3.991276in}{1.634282in}}%
\pgfpathlineto{\pgfqpoint{3.993109in}{1.554125in}}%
\pgfpathlineto{\pgfqpoint{3.994943in}{1.790756in}}%
\pgfpathlineto{\pgfqpoint{3.998610in}{1.493603in}}%
\pgfpathlineto{\pgfqpoint{4.002279in}{1.611988in}}%
\pgfpathlineto{\pgfqpoint{4.004112in}{1.550286in}}%
\pgfpathlineto{\pgfqpoint{4.005945in}{1.533562in}}%
\pgfpathlineto{\pgfqpoint{4.007780in}{1.588564in}}%
\pgfpathlineto{\pgfqpoint{4.009614in}{1.503414in}}%
\pgfpathlineto{\pgfqpoint{4.011448in}{1.614723in}}%
\pgfpathlineto{\pgfqpoint{4.015116in}{1.443206in}}%
\pgfpathlineto{\pgfqpoint{4.016949in}{1.511531in}}%
\pgfpathlineto{\pgfqpoint{4.020618in}{1.328384in}}%
\pgfpathlineto{\pgfqpoint{4.022452in}{1.415065in}}%
\pgfpathlineto{\pgfqpoint{4.024289in}{1.661319in}}%
\pgfpathlineto{\pgfqpoint{4.026122in}{1.615225in}}%
\pgfpathlineto{\pgfqpoint{4.029789in}{1.419820in}}%
\pgfpathlineto{\pgfqpoint{4.031623in}{1.363789in}}%
\pgfpathlineto{\pgfqpoint{4.033457in}{1.433520in}}%
\pgfpathlineto{\pgfqpoint{4.035291in}{1.647016in}}%
\pgfpathlineto{\pgfqpoint{4.037125in}{1.504832in}}%
\pgfpathlineto{\pgfqpoint{4.040794in}{1.569933in}}%
\pgfpathlineto{\pgfqpoint{4.042629in}{1.631033in}}%
\pgfpathlineto{\pgfqpoint{4.044463in}{1.576909in}}%
\pgfpathlineto{\pgfqpoint{4.046298in}{1.622501in}}%
\pgfpathlineto{\pgfqpoint{4.048133in}{1.695845in}}%
\pgfpathlineto{\pgfqpoint{4.049965in}{1.907347in}}%
\pgfpathlineto{\pgfqpoint{4.051799in}{1.966815in}}%
\pgfpathlineto{\pgfqpoint{4.055467in}{1.556835in}}%
\pgfpathlineto{\pgfqpoint{4.057301in}{1.620657in}}%
\pgfpathlineto{\pgfqpoint{4.059136in}{1.768035in}}%
\pgfpathlineto{\pgfqpoint{4.060970in}{1.552369in}}%
\pgfpathlineto{\pgfqpoint{4.062804in}{1.476352in}}%
\pgfpathlineto{\pgfqpoint{4.064637in}{1.482676in}}%
\pgfpathlineto{\pgfqpoint{4.066470in}{1.557713in}}%
\pgfpathlineto{\pgfqpoint{4.068303in}{1.501620in}}%
\pgfpathlineto{\pgfqpoint{4.070138in}{1.586858in}}%
\pgfpathlineto{\pgfqpoint{4.071971in}{1.576545in}}%
\pgfpathlineto{\pgfqpoint{4.073806in}{1.447321in}}%
\pgfpathlineto{\pgfqpoint{4.075640in}{1.622413in}}%
\pgfpathlineto{\pgfqpoint{4.077474in}{1.547488in}}%
\pgfpathlineto{\pgfqpoint{4.081142in}{1.553774in}}%
\pgfpathlineto{\pgfqpoint{4.082976in}{1.225544in}}%
\pgfpathlineto{\pgfqpoint{4.084809in}{1.531003in}}%
\pgfpathlineto{\pgfqpoint{4.086642in}{1.455840in}}%
\pgfpathlineto{\pgfqpoint{4.088476in}{1.546259in}}%
\pgfpathlineto{\pgfqpoint{4.090309in}{1.365470in}}%
\pgfpathlineto{\pgfqpoint{4.092143in}{1.611599in}}%
\pgfpathlineto{\pgfqpoint{4.095810in}{1.299641in}}%
\pgfpathlineto{\pgfqpoint{4.097644in}{0.944463in}}%
\pgfpathlineto{\pgfqpoint{4.099480in}{1.302351in}}%
\pgfpathlineto{\pgfqpoint{4.101313in}{1.390199in}}%
\pgfpathlineto{\pgfqpoint{4.103146in}{1.319213in}}%
\pgfpathlineto{\pgfqpoint{4.104981in}{1.164307in}}%
\pgfpathlineto{\pgfqpoint{4.106815in}{1.356036in}}%
\pgfpathlineto{\pgfqpoint{4.108648in}{1.170279in}}%
\pgfpathlineto{\pgfqpoint{4.110482in}{1.220225in}}%
\pgfpathlineto{\pgfqpoint{4.112316in}{1.342248in}}%
\pgfpathlineto{\pgfqpoint{4.114150in}{1.191984in}}%
\pgfpathlineto{\pgfqpoint{4.115985in}{1.172475in}}%
\pgfpathlineto{\pgfqpoint{4.117818in}{1.168485in}}%
\pgfpathlineto{\pgfqpoint{4.123319in}{1.273408in}}%
\pgfpathlineto{\pgfqpoint{4.125154in}{1.267825in}}%
\pgfpathlineto{\pgfqpoint{4.126989in}{1.296116in}}%
\pgfpathlineto{\pgfqpoint{4.128822in}{1.529861in}}%
\pgfpathlineto{\pgfqpoint{4.130656in}{1.581513in}}%
\pgfpathlineto{\pgfqpoint{4.136157in}{1.188960in}}%
\pgfpathlineto{\pgfqpoint{4.139824in}{1.967367in}}%
\pgfpathlineto{\pgfqpoint{4.141657in}{2.147955in}}%
\pgfpathlineto{\pgfqpoint{4.143492in}{2.087696in}}%
\pgfpathlineto{\pgfqpoint{4.147158in}{2.331779in}}%
\pgfpathlineto{\pgfqpoint{4.148992in}{2.327137in}}%
\pgfpathlineto{\pgfqpoint{4.150827in}{2.360723in}}%
\pgfpathlineto{\pgfqpoint{4.154496in}{2.265875in}}%
\pgfpathlineto{\pgfqpoint{4.158163in}{2.399478in}}%
\pgfpathlineto{\pgfqpoint{4.159995in}{2.169321in}}%
\pgfpathlineto{\pgfqpoint{4.161830in}{2.220534in}}%
\pgfpathlineto{\pgfqpoint{4.163663in}{2.185191in}}%
\pgfpathlineto{\pgfqpoint{4.165497in}{2.347826in}}%
\pgfpathlineto{\pgfqpoint{4.167331in}{2.116314in}}%
\pgfpathlineto{\pgfqpoint{4.169164in}{2.238286in}}%
\pgfpathlineto{\pgfqpoint{4.170998in}{2.243894in}}%
\pgfpathlineto{\pgfqpoint{4.172834in}{2.218100in}}%
\pgfpathlineto{\pgfqpoint{4.174667in}{2.249628in}}%
\pgfpathlineto{\pgfqpoint{4.178336in}{1.245468in}}%
\pgfpathlineto{\pgfqpoint{4.180169in}{1.347517in}}%
\pgfpathlineto{\pgfqpoint{4.182002in}{1.072282in}}%
\pgfpathlineto{\pgfqpoint{4.183837in}{1.124674in}}%
\pgfpathlineto{\pgfqpoint{4.185671in}{1.104751in}}%
\pgfpathlineto{\pgfqpoint{4.187504in}{1.125113in}}%
\pgfpathlineto{\pgfqpoint{4.189339in}{1.255655in}}%
\pgfpathlineto{\pgfqpoint{4.191172in}{0.986981in}}%
\pgfpathlineto{\pgfqpoint{4.193005in}{1.023503in}}%
\pgfpathlineto{\pgfqpoint{4.194838in}{1.104136in}}%
\pgfpathlineto{\pgfqpoint{4.196673in}{1.130583in}}%
\pgfpathlineto{\pgfqpoint{4.198506in}{0.996830in}}%
\pgfpathlineto{\pgfqpoint{4.202200in}{1.083297in}}%
\pgfpathlineto{\pgfqpoint{4.204035in}{0.929069in}}%
\pgfpathlineto{\pgfqpoint{4.205869in}{1.217490in}}%
\pgfpathlineto{\pgfqpoint{4.207702in}{1.169627in}}%
\pgfpathlineto{\pgfqpoint{4.209536in}{1.069585in}}%
\pgfpathlineto{\pgfqpoint{4.211368in}{1.271438in}}%
\pgfpathlineto{\pgfqpoint{4.213201in}{1.271501in}}%
\pgfpathlineto{\pgfqpoint{4.215035in}{1.348596in}}%
\pgfpathlineto{\pgfqpoint{4.216869in}{1.361644in}}%
\pgfpathlineto{\pgfqpoint{4.218702in}{1.327180in}}%
\pgfpathlineto{\pgfqpoint{4.220537in}{1.238241in}}%
\pgfpathlineto{\pgfqpoint{4.222371in}{1.343189in}}%
\pgfpathlineto{\pgfqpoint{4.226039in}{1.094589in}}%
\pgfpathlineto{\pgfqpoint{4.227874in}{1.070024in}}%
\pgfpathlineto{\pgfqpoint{4.229707in}{0.991360in}}%
\pgfpathlineto{\pgfqpoint{4.231541in}{1.040465in}}%
\pgfpathlineto{\pgfqpoint{4.233374in}{1.252782in}}%
\pgfpathlineto{\pgfqpoint{4.235208in}{1.261125in}}%
\pgfpathlineto{\pgfqpoint{4.237041in}{1.161459in}}%
\pgfpathlineto{\pgfqpoint{4.238876in}{1.176565in}}%
\pgfpathlineto{\pgfqpoint{4.240710in}{1.310431in}}%
\pgfpathlineto{\pgfqpoint{4.242543in}{1.298148in}}%
\pgfpathlineto{\pgfqpoint{4.244379in}{1.008962in}}%
\pgfpathlineto{\pgfqpoint{4.248044in}{1.158762in}}%
\pgfpathlineto{\pgfqpoint{4.249878in}{1.154496in}}%
\pgfpathlineto{\pgfqpoint{4.251713in}{1.218017in}}%
\pgfpathlineto{\pgfqpoint{4.253547in}{1.084038in}}%
\pgfpathlineto{\pgfqpoint{4.255380in}{1.117448in}}%
\pgfpathlineto{\pgfqpoint{4.257214in}{1.178534in}}%
\pgfpathlineto{\pgfqpoint{4.259047in}{1.055357in}}%
\pgfpathlineto{\pgfqpoint{4.260882in}{1.310042in}}%
\pgfpathlineto{\pgfqpoint{4.262717in}{1.153053in}}%
\pgfpathlineto{\pgfqpoint{4.266386in}{1.243347in}}%
\pgfpathlineto{\pgfqpoint{4.268220in}{1.212735in}}%
\pgfpathlineto{\pgfqpoint{4.273723in}{1.433043in}}%
\pgfpathlineto{\pgfqpoint{4.275557in}{2.301543in}}%
\pgfpathlineto{\pgfqpoint{4.277392in}{2.229968in}}%
\pgfpathlineto{\pgfqpoint{4.279227in}{2.207573in}}%
\pgfpathlineto{\pgfqpoint{4.281060in}{2.328317in}}%
\pgfpathlineto{\pgfqpoint{4.282894in}{2.276037in}}%
\pgfpathlineto{\pgfqpoint{4.286560in}{2.077647in}}%
\pgfpathlineto{\pgfqpoint{4.288394in}{2.307829in}}%
\pgfpathlineto{\pgfqpoint{4.292060in}{2.333536in}}%
\pgfpathlineto{\pgfqpoint{4.293894in}{2.300452in}}%
\pgfpathlineto{\pgfqpoint{4.295728in}{2.325632in}}%
\pgfpathlineto{\pgfqpoint{4.297561in}{2.213156in}}%
\pgfpathlineto{\pgfqpoint{4.299395in}{2.285208in}}%
\pgfpathlineto{\pgfqpoint{4.303066in}{1.192611in}}%
\pgfpathlineto{\pgfqpoint{4.304899in}{1.243285in}}%
\pgfpathlineto{\pgfqpoint{4.306733in}{1.435603in}}%
\pgfpathlineto{\pgfqpoint{4.308566in}{1.345246in}}%
\pgfpathlineto{\pgfqpoint{4.312234in}{2.107293in}}%
\pgfpathlineto{\pgfqpoint{4.314067in}{2.162910in}}%
\pgfpathlineto{\pgfqpoint{4.315900in}{2.180512in}}%
\pgfpathlineto{\pgfqpoint{4.317734in}{2.163499in}}%
\pgfpathlineto{\pgfqpoint{4.319568in}{2.412526in}}%
\pgfpathlineto{\pgfqpoint{4.321401in}{2.469083in}}%
\pgfpathlineto{\pgfqpoint{4.325069in}{2.297253in}}%
\pgfpathlineto{\pgfqpoint{4.326902in}{2.462911in}}%
\pgfpathlineto{\pgfqpoint{4.328736in}{2.214248in}}%
\pgfpathlineto{\pgfqpoint{4.330571in}{2.145282in}}%
\pgfpathlineto{\pgfqpoint{4.332404in}{2.262312in}}%
\pgfpathlineto{\pgfqpoint{4.337905in}{2.039532in}}%
\pgfpathlineto{\pgfqpoint{4.339740in}{2.080231in}}%
\pgfpathlineto{\pgfqpoint{4.341576in}{2.219179in}}%
\pgfpathlineto{\pgfqpoint{4.343408in}{2.173762in}}%
\pgfpathlineto{\pgfqpoint{4.345242in}{2.096102in}}%
\pgfpathlineto{\pgfqpoint{4.347076in}{2.188541in}}%
\pgfpathlineto{\pgfqpoint{4.348910in}{2.155834in}}%
\pgfpathlineto{\pgfqpoint{4.350743in}{2.199757in}}%
\pgfpathlineto{\pgfqpoint{4.352577in}{2.290641in}}%
\pgfpathlineto{\pgfqpoint{4.354411in}{1.228305in}}%
\pgfpathlineto{\pgfqpoint{4.356245in}{1.269995in}}%
\pgfpathlineto{\pgfqpoint{4.358079in}{1.269004in}}%
\pgfpathlineto{\pgfqpoint{4.359912in}{1.320091in}}%
\pgfpathlineto{\pgfqpoint{4.361745in}{1.236359in}}%
\pgfpathlineto{\pgfqpoint{4.363581in}{1.376448in}}%
\pgfpathlineto{\pgfqpoint{4.365414in}{1.386058in}}%
\pgfpathlineto{\pgfqpoint{4.367247in}{1.491282in}}%
\pgfpathlineto{\pgfqpoint{4.369083in}{1.437899in}}%
\pgfpathlineto{\pgfqpoint{4.370916in}{1.427674in}}%
\pgfpathlineto{\pgfqpoint{4.372750in}{1.120446in}}%
\pgfpathlineto{\pgfqpoint{4.374585in}{1.203212in}}%
\pgfpathlineto{\pgfqpoint{4.376419in}{1.429631in}}%
\pgfpathlineto{\pgfqpoint{4.378253in}{1.358031in}}%
\pgfpathlineto{\pgfqpoint{4.380088in}{1.207880in}}%
\pgfpathlineto{\pgfqpoint{4.381923in}{1.388354in}}%
\pgfpathlineto{\pgfqpoint{4.383756in}{1.398253in}}%
\pgfpathlineto{\pgfqpoint{4.385591in}{1.305965in}}%
\pgfpathlineto{\pgfqpoint{4.387426in}{1.318598in}}%
\pgfpathlineto{\pgfqpoint{4.389259in}{1.313166in}}%
\pgfpathlineto{\pgfqpoint{4.391093in}{1.476239in}}%
\pgfpathlineto{\pgfqpoint{4.392929in}{1.337468in}}%
\pgfpathlineto{\pgfqpoint{4.394762in}{1.340755in}}%
\pgfpathlineto{\pgfqpoint{4.396597in}{1.354442in}}%
\pgfpathlineto{\pgfqpoint{4.398431in}{1.386209in}}%
\pgfpathlineto{\pgfqpoint{4.400264in}{1.308022in}}%
\pgfpathlineto{\pgfqpoint{4.403933in}{1.481647in}}%
\pgfpathlineto{\pgfqpoint{4.405798in}{1.339375in}}%
\pgfpathlineto{\pgfqpoint{4.407631in}{1.347868in}}%
\pgfpathlineto{\pgfqpoint{4.409466in}{1.256420in}}%
\pgfpathlineto{\pgfqpoint{4.411300in}{1.295325in}}%
\pgfpathlineto{\pgfqpoint{4.413134in}{1.297446in}}%
\pgfpathlineto{\pgfqpoint{4.414969in}{1.222809in}}%
\pgfpathlineto{\pgfqpoint{4.416802in}{1.453167in}}%
\pgfpathlineto{\pgfqpoint{4.418637in}{1.297910in}}%
\pgfpathlineto{\pgfqpoint{4.420471in}{1.393937in}}%
\pgfpathlineto{\pgfqpoint{4.422305in}{1.321798in}}%
\pgfpathlineto{\pgfqpoint{4.424138in}{1.379974in}}%
\pgfpathlineto{\pgfqpoint{4.425971in}{1.298889in}}%
\pgfpathlineto{\pgfqpoint{4.427805in}{1.402569in}}%
\pgfpathlineto{\pgfqpoint{4.429639in}{1.381830in}}%
\pgfpathlineto{\pgfqpoint{4.431472in}{1.341570in}}%
\pgfpathlineto{\pgfqpoint{4.435139in}{1.513790in}}%
\pgfpathlineto{\pgfqpoint{4.436973in}{1.574048in}}%
\pgfpathlineto{\pgfqpoint{4.438807in}{1.573521in}}%
\pgfpathlineto{\pgfqpoint{4.442474in}{1.284398in}}%
\pgfpathlineto{\pgfqpoint{4.444309in}{1.479740in}}%
\pgfpathlineto{\pgfqpoint{4.446143in}{1.241503in}}%
\pgfpathlineto{\pgfqpoint{4.447977in}{1.303669in}}%
\pgfpathlineto{\pgfqpoint{4.449812in}{1.283607in}}%
\pgfpathlineto{\pgfqpoint{4.451645in}{1.410097in}}%
\pgfpathlineto{\pgfqpoint{4.453478in}{1.260598in}}%
\pgfpathlineto{\pgfqpoint{4.458998in}{1.494456in}}%
\pgfpathlineto{\pgfqpoint{4.462665in}{1.328999in}}%
\pgfpathlineto{\pgfqpoint{4.464499in}{1.633642in}}%
\pgfpathlineto{\pgfqpoint{4.466333in}{2.176221in}}%
\pgfpathlineto{\pgfqpoint{4.469999in}{2.438973in}}%
\pgfpathlineto{\pgfqpoint{4.473668in}{2.248248in}}%
\pgfpathlineto{\pgfqpoint{4.477335in}{2.515566in}}%
\pgfpathlineto{\pgfqpoint{4.479184in}{2.207573in}}%
\pgfpathlineto{\pgfqpoint{4.481019in}{2.481930in}}%
\pgfpathlineto{\pgfqpoint{4.482853in}{2.321441in}}%
\pgfpathlineto{\pgfqpoint{4.484685in}{2.323813in}}%
\pgfpathlineto{\pgfqpoint{4.486520in}{2.311505in}}%
\pgfpathlineto{\pgfqpoint{4.488354in}{2.491152in}}%
\pgfpathlineto{\pgfqpoint{4.490186in}{2.382616in}}%
\pgfpathlineto{\pgfqpoint{4.492020in}{2.628719in}}%
\pgfpathlineto{\pgfqpoint{4.493854in}{2.442021in}}%
\pgfpathlineto{\pgfqpoint{4.495689in}{2.515504in}}%
\pgfpathlineto{\pgfqpoint{4.499358in}{2.405914in}}%
\pgfpathlineto{\pgfqpoint{4.501192in}{2.585849in}}%
\pgfpathlineto{\pgfqpoint{4.503027in}{2.337914in}}%
\pgfpathlineto{\pgfqpoint{4.506693in}{2.643937in}}%
\pgfpathlineto{\pgfqpoint{4.508528in}{2.462208in}}%
\pgfpathlineto{\pgfqpoint{4.510362in}{2.442498in}}%
\pgfpathlineto{\pgfqpoint{4.512194in}{2.406679in}}%
\pgfpathlineto{\pgfqpoint{4.514031in}{2.275749in}}%
\pgfpathlineto{\pgfqpoint{4.517699in}{2.494025in}}%
\pgfpathlineto{\pgfqpoint{4.521370in}{2.389127in}}%
\pgfpathlineto{\pgfqpoint{4.523203in}{2.303802in}}%
\pgfpathlineto{\pgfqpoint{4.525039in}{2.448784in}}%
\pgfpathlineto{\pgfqpoint{4.528706in}{2.320137in}}%
\pgfpathlineto{\pgfqpoint{4.530541in}{2.235702in}}%
\pgfpathlineto{\pgfqpoint{4.532377in}{2.276426in}}%
\pgfpathlineto{\pgfqpoint{4.534210in}{2.194563in}}%
\pgfpathlineto{\pgfqpoint{4.537878in}{2.315996in}}%
\pgfpathlineto{\pgfqpoint{4.541544in}{2.282360in}}%
\pgfpathlineto{\pgfqpoint{4.543377in}{2.193672in}}%
\pgfpathlineto{\pgfqpoint{4.545211in}{2.222591in}}%
\pgfpathlineto{\pgfqpoint{4.547045in}{2.590554in}}%
\pgfpathlineto{\pgfqpoint{4.548878in}{2.601331in}}%
\pgfpathlineto{\pgfqpoint{4.550711in}{2.835553in}}%
\pgfpathlineto{\pgfqpoint{4.552574in}{2.192380in}}%
\pgfpathlineto{\pgfqpoint{4.554408in}{2.299925in}}%
\pgfpathlineto{\pgfqpoint{4.556243in}{2.255888in}}%
\pgfpathlineto{\pgfqpoint{4.558077in}{2.241460in}}%
\pgfpathlineto{\pgfqpoint{4.559910in}{2.245864in}}%
\pgfpathlineto{\pgfqpoint{4.561745in}{2.306298in}}%
\pgfpathlineto{\pgfqpoint{4.565411in}{2.137027in}}%
\pgfpathlineto{\pgfqpoint{4.567246in}{2.415110in}}%
\pgfpathlineto{\pgfqpoint{4.569079in}{2.262136in}}%
\pgfpathlineto{\pgfqpoint{4.572747in}{2.401498in}}%
\pgfpathlineto{\pgfqpoint{4.574580in}{2.298846in}}%
\pgfpathlineto{\pgfqpoint{4.576415in}{2.314830in}}%
\pgfpathlineto{\pgfqpoint{4.578250in}{2.207260in}}%
\pgfpathlineto{\pgfqpoint{4.580083in}{2.321705in}}%
\pgfpathlineto{\pgfqpoint{4.581917in}{2.319610in}}%
\pgfpathlineto{\pgfqpoint{4.583750in}{2.331629in}}%
\pgfpathlineto{\pgfqpoint{4.585584in}{1.448914in}}%
\pgfpathlineto{\pgfqpoint{4.587417in}{1.422492in}}%
\pgfpathlineto{\pgfqpoint{4.589252in}{1.508997in}}%
\pgfpathlineto{\pgfqpoint{4.591086in}{1.481521in}}%
\pgfpathlineto{\pgfqpoint{4.592920in}{1.490454in}}%
\pgfpathlineto{\pgfqpoint{4.596587in}{1.432805in}}%
\pgfpathlineto{\pgfqpoint{4.598421in}{1.491659in}}%
\pgfpathlineto{\pgfqpoint{4.600254in}{1.447496in}}%
\pgfpathlineto{\pgfqpoint{4.602089in}{1.472099in}}%
\pgfpathlineto{\pgfqpoint{4.605756in}{1.294623in}}%
\pgfpathlineto{\pgfqpoint{4.607590in}{1.417323in}}%
\pgfpathlineto{\pgfqpoint{4.609425in}{2.295760in}}%
\pgfpathlineto{\pgfqpoint{4.611259in}{2.346295in}}%
\pgfpathlineto{\pgfqpoint{4.614929in}{2.163825in}}%
\pgfpathlineto{\pgfqpoint{4.616761in}{1.949301in}}%
\pgfpathlineto{\pgfqpoint{4.618596in}{2.319321in}}%
\pgfpathlineto{\pgfqpoint{4.620430in}{2.292962in}}%
\pgfpathlineto{\pgfqpoint{4.622262in}{2.119011in}}%
\pgfpathlineto{\pgfqpoint{4.624096in}{2.287705in}}%
\pgfpathlineto{\pgfqpoint{4.625930in}{2.334163in}}%
\pgfpathlineto{\pgfqpoint{4.629613in}{2.104821in}}%
\pgfpathlineto{\pgfqpoint{4.631447in}{2.167790in}}%
\pgfpathlineto{\pgfqpoint{4.633281in}{2.148017in}}%
\pgfpathlineto{\pgfqpoint{4.635115in}{1.205069in}}%
\pgfpathlineto{\pgfqpoint{4.636949in}{1.250072in}}%
\pgfpathlineto{\pgfqpoint{4.638784in}{1.253510in}}%
\pgfpathlineto{\pgfqpoint{4.642452in}{1.396672in}}%
\pgfpathlineto{\pgfqpoint{4.644286in}{1.339312in}}%
\pgfpathlineto{\pgfqpoint{4.646122in}{1.344456in}}%
\pgfpathlineto{\pgfqpoint{4.647955in}{1.353765in}}%
\pgfpathlineto{\pgfqpoint{4.651625in}{2.255700in}}%
\pgfpathlineto{\pgfqpoint{4.653458in}{2.160714in}}%
\pgfpathlineto{\pgfqpoint{4.655293in}{1.984091in}}%
\pgfpathlineto{\pgfqpoint{4.657854in}{2.046407in}}%
\pgfpathlineto{\pgfqpoint{4.659687in}{2.042330in}}%
\pgfpathlineto{\pgfqpoint{4.661523in}{2.114407in}}%
\pgfpathlineto{\pgfqpoint{4.663356in}{1.999660in}}%
\pgfpathlineto{\pgfqpoint{4.665190in}{1.964958in}}%
\pgfpathlineto{\pgfqpoint{4.667025in}{1.806941in}}%
\pgfpathlineto{\pgfqpoint{4.668858in}{1.970842in}}%
\pgfpathlineto{\pgfqpoint{4.672527in}{1.864678in}}%
\pgfpathlineto{\pgfqpoint{4.674360in}{2.003901in}}%
\pgfpathlineto{\pgfqpoint{4.676194in}{2.272048in}}%
\pgfpathlineto{\pgfqpoint{4.679861in}{2.104232in}}%
\pgfpathlineto{\pgfqpoint{4.681693in}{2.218150in}}%
\pgfpathlineto{\pgfqpoint{4.683527in}{2.132410in}}%
\pgfpathlineto{\pgfqpoint{4.685359in}{2.311354in}}%
\pgfpathlineto{\pgfqpoint{4.687192in}{2.137002in}}%
\pgfpathlineto{\pgfqpoint{4.689027in}{2.097043in}}%
\pgfpathlineto{\pgfqpoint{4.690860in}{2.243744in}}%
\pgfpathlineto{\pgfqpoint{4.694526in}{2.149071in}}%
\pgfpathlineto{\pgfqpoint{4.696359in}{1.854478in}}%
\pgfpathlineto{\pgfqpoint{4.698192in}{1.233474in}}%
\pgfpathlineto{\pgfqpoint{4.700028in}{1.407625in}}%
\pgfpathlineto{\pgfqpoint{4.701864in}{1.122905in}}%
\pgfpathlineto{\pgfqpoint{4.703698in}{1.129931in}}%
\pgfpathlineto{\pgfqpoint{4.705532in}{1.118351in}}%
\pgfpathlineto{\pgfqpoint{4.707367in}{1.212120in}}%
\pgfpathlineto{\pgfqpoint{4.709200in}{0.858083in}}%
\pgfpathlineto{\pgfqpoint{4.712867in}{1.312601in}}%
\pgfpathlineto{\pgfqpoint{4.714701in}{1.254488in}}%
\pgfpathlineto{\pgfqpoint{4.716533in}{1.062471in}}%
\pgfpathlineto{\pgfqpoint{4.718366in}{1.239671in}}%
\pgfpathlineto{\pgfqpoint{4.720199in}{1.284247in}}%
\pgfpathlineto{\pgfqpoint{4.722032in}{1.356625in}}%
\pgfpathlineto{\pgfqpoint{4.723865in}{1.988996in}}%
\pgfpathlineto{\pgfqpoint{4.725698in}{2.017413in}}%
\pgfpathlineto{\pgfqpoint{4.727532in}{1.936378in}}%
\pgfpathlineto{\pgfqpoint{4.729366in}{1.976362in}}%
\pgfpathlineto{\pgfqpoint{4.731200in}{2.057109in}}%
\pgfpathlineto{\pgfqpoint{4.733032in}{2.232440in}}%
\pgfpathlineto{\pgfqpoint{4.734866in}{2.219505in}}%
\pgfpathlineto{\pgfqpoint{4.736700in}{2.248160in}}%
\pgfpathlineto{\pgfqpoint{4.738533in}{2.256854in}}%
\pgfpathlineto{\pgfqpoint{4.740366in}{2.181954in}}%
\pgfpathlineto{\pgfqpoint{4.742199in}{2.159735in}}%
\pgfpathlineto{\pgfqpoint{4.744033in}{2.093016in}}%
\pgfpathlineto{\pgfqpoint{4.745866in}{2.253203in}}%
\pgfpathlineto{\pgfqpoint{4.747700in}{2.146110in}}%
\pgfpathlineto{\pgfqpoint{4.749534in}{2.152747in}}%
\pgfpathlineto{\pgfqpoint{4.751367in}{2.187161in}}%
\pgfpathlineto{\pgfqpoint{4.753202in}{2.157778in}}%
\pgfpathlineto{\pgfqpoint{4.755035in}{2.196972in}}%
\pgfpathlineto{\pgfqpoint{4.756868in}{1.926241in}}%
\pgfpathlineto{\pgfqpoint{4.758701in}{2.092049in}}%
\pgfpathlineto{\pgfqpoint{4.760535in}{2.138056in}}%
\pgfpathlineto{\pgfqpoint{4.764201in}{1.274587in}}%
\pgfpathlineto{\pgfqpoint{4.766036in}{1.197479in}}%
\pgfpathlineto{\pgfqpoint{4.767868in}{1.317570in}}%
\pgfpathlineto{\pgfqpoint{4.771535in}{1.194543in}}%
\pgfpathlineto{\pgfqpoint{4.773368in}{2.037838in}}%
\pgfpathlineto{\pgfqpoint{4.777037in}{2.132824in}}%
\pgfpathlineto{\pgfqpoint{4.778871in}{2.149699in}}%
\pgfpathlineto{\pgfqpoint{4.780704in}{2.062604in}}%
\pgfpathlineto{\pgfqpoint{4.782543in}{2.207072in}}%
\pgfpathlineto{\pgfqpoint{4.784383in}{2.222528in}}%
\pgfpathlineto{\pgfqpoint{4.788065in}{1.263270in}}%
\pgfpathlineto{\pgfqpoint{4.789906in}{1.241352in}}%
\pgfpathlineto{\pgfqpoint{4.791746in}{1.111802in}}%
\pgfpathlineto{\pgfqpoint{4.795426in}{1.164395in}}%
\pgfpathlineto{\pgfqpoint{4.797636in}{1.171948in}}%
\pgfpathlineto{\pgfqpoint{4.799475in}{2.813949in}}%
\pgfpathlineto{\pgfqpoint{4.801317in}{1.208469in}}%
\pgfpathlineto{\pgfqpoint{4.803157in}{1.281286in}}%
\pgfpathlineto{\pgfqpoint{4.804997in}{2.003186in}}%
\pgfpathlineto{\pgfqpoint{4.806836in}{2.013674in}}%
\pgfpathlineto{\pgfqpoint{4.810517in}{1.872117in}}%
\pgfpathlineto{\pgfqpoint{4.812358in}{1.914247in}}%
\pgfpathlineto{\pgfqpoint{4.814200in}{2.164616in}}%
\pgfpathlineto{\pgfqpoint{4.816040in}{2.025907in}}%
\pgfpathlineto{\pgfqpoint{4.817881in}{2.000689in}}%
\pgfpathlineto{\pgfqpoint{4.819721in}{2.196295in}}%
\pgfpathlineto{\pgfqpoint{4.821560in}{1.897711in}}%
\pgfpathlineto{\pgfqpoint{4.823400in}{1.955122in}}%
\pgfpathlineto{\pgfqpoint{4.825241in}{2.069454in}}%
\pgfpathlineto{\pgfqpoint{4.827080in}{2.062892in}}%
\pgfpathlineto{\pgfqpoint{4.828920in}{2.021930in}}%
\pgfpathlineto{\pgfqpoint{4.830759in}{2.116163in}}%
\pgfpathlineto{\pgfqpoint{4.836650in}{2.209581in}}%
\pgfpathlineto{\pgfqpoint{4.838489in}{2.140352in}}%
\pgfpathlineto{\pgfqpoint{4.840330in}{2.135760in}}%
\pgfpathlineto{\pgfqpoint{4.842169in}{2.034488in}}%
\pgfpathlineto{\pgfqpoint{4.844009in}{2.063683in}}%
\pgfpathlineto{\pgfqpoint{4.846214in}{2.033045in}}%
\pgfpathlineto{\pgfqpoint{4.848056in}{2.254031in}}%
\pgfpathlineto{\pgfqpoint{4.849893in}{2.000338in}}%
\pgfpathlineto{\pgfqpoint{4.851730in}{1.916831in}}%
\pgfpathlineto{\pgfqpoint{4.853564in}{1.924008in}}%
\pgfpathlineto{\pgfqpoint{4.857232in}{2.171077in}}%
\pgfpathlineto{\pgfqpoint{4.859065in}{2.124456in}}%
\pgfpathlineto{\pgfqpoint{4.859065in}{2.124456in}}%
\pgfusepath{stroke}%
\end{pgfscope}%
\begin{pgfscope}%
\pgfsetrectcap%
\pgfsetmiterjoin%
\pgfsetlinewidth{0.803000pt}%
\definecolor{currentstroke}{rgb}{0.000000,0.000000,0.000000}%
\pgfsetstrokecolor{currentstroke}%
\pgfsetdash{}{0pt}%
\pgfpathmoveto{\pgfqpoint{0.667540in}{0.539544in}}%
\pgfpathlineto{\pgfqpoint{0.667540in}{2.944887in}}%
\pgfusepath{stroke}%
\end{pgfscope}%
\begin{pgfscope}%
\pgfsetrectcap%
\pgfsetmiterjoin%
\pgfsetlinewidth{0.803000pt}%
\definecolor{currentstroke}{rgb}{0.000000,0.000000,0.000000}%
\pgfsetstrokecolor{currentstroke}%
\pgfsetdash{}{0pt}%
\pgfpathmoveto{\pgfqpoint{5.058662in}{0.539544in}}%
\pgfpathlineto{\pgfqpoint{5.058662in}{2.944887in}}%
\pgfusepath{stroke}%
\end{pgfscope}%
\begin{pgfscope}%
\pgfsetrectcap%
\pgfsetmiterjoin%
\pgfsetlinewidth{0.803000pt}%
\definecolor{currentstroke}{rgb}{0.000000,0.000000,0.000000}%
\pgfsetstrokecolor{currentstroke}%
\pgfsetdash{}{0pt}%
\pgfpathmoveto{\pgfqpoint{0.667540in}{0.539544in}}%
\pgfpathlineto{\pgfqpoint{5.058662in}{0.539544in}}%
\pgfusepath{stroke}%
\end{pgfscope}%
\begin{pgfscope}%
\pgfsetrectcap%
\pgfsetmiterjoin%
\pgfsetlinewidth{0.803000pt}%
\definecolor{currentstroke}{rgb}{0.000000,0.000000,0.000000}%
\pgfsetstrokecolor{currentstroke}%
\pgfsetdash{}{0pt}%
\pgfpathmoveto{\pgfqpoint{0.667540in}{2.944887in}}%
\pgfpathlineto{\pgfqpoint{5.058662in}{2.944887in}}%
\pgfusepath{stroke}%
\end{pgfscope}%
\begin{pgfscope}%
\pgfsetbuttcap%
\pgfsetmiterjoin%
\definecolor{currentfill}{rgb}{1.000000,1.000000,1.000000}%
\pgfsetfillcolor{currentfill}%
\pgfsetfillopacity{0.800000}%
\pgfsetlinewidth{1.003750pt}%
\definecolor{currentstroke}{rgb}{0.800000,0.800000,0.800000}%
\pgfsetstrokecolor{currentstroke}%
\pgfsetstrokeopacity{0.800000}%
\pgfsetdash{}{0pt}%
\pgfpathmoveto{\pgfqpoint{0.745318in}{2.701109in}}%
\pgfpathlineto{\pgfqpoint{2.092873in}{2.701109in}}%
\pgfpathquadraticcurveto{\pgfqpoint{2.115096in}{2.701109in}}{\pgfqpoint{2.115096in}{2.723331in}}%
\pgfpathlineto{\pgfqpoint{2.115096in}{2.867109in}}%
\pgfpathquadraticcurveto{\pgfqpoint{2.115096in}{2.889331in}}{\pgfqpoint{2.092873in}{2.889331in}}%
\pgfpathlineto{\pgfqpoint{0.745318in}{2.889331in}}%
\pgfpathquadraticcurveto{\pgfqpoint{0.723095in}{2.889331in}}{\pgfqpoint{0.723095in}{2.867109in}}%
\pgfpathlineto{\pgfqpoint{0.723095in}{2.723331in}}%
\pgfpathquadraticcurveto{\pgfqpoint{0.723095in}{2.701109in}}{\pgfqpoint{0.745318in}{2.701109in}}%
\pgfpathlineto{\pgfqpoint{0.745318in}{2.701109in}}%
\pgfpathclose%
\pgfusepath{stroke,fill}%
\end{pgfscope}%
\begin{pgfscope}%
\pgfsetrectcap%
\pgfsetroundjoin%
\pgfsetlinewidth{0.501875pt}%
\definecolor{currentstroke}{rgb}{0.121569,0.466667,0.705882}%
\pgfsetstrokecolor{currentstroke}%
\pgfsetstrokeopacity{0.700000}%
\pgfsetdash{}{0pt}%
\pgfpathmoveto{\pgfqpoint{0.767540in}{2.805998in}}%
\pgfpathlineto{\pgfqpoint{0.878651in}{2.805998in}}%
\pgfpathlineto{\pgfqpoint{0.989762in}{2.805998in}}%
\pgfusepath{stroke}%
\end{pgfscope}%
\begin{pgfscope}%
\definecolor{textcolor}{rgb}{0.000000,0.000000,0.000000}%
\pgfsetstrokecolor{textcolor}%
\pgfsetfillcolor{textcolor}%
\pgftext[x=1.078651in,y=2.767109in,left,base]{\color{textcolor}\rmfamily\fontsize{8.000000}{9.600000}\selectfont DUT vs KS34470A}%
\end{pgfscope}%
\end{pgfpicture}%
\makeatother%
\endgroup%

    \caption{Popcorn noise of a refurbished LM399 (\#15) over a period of \qty{15}{\minute}.}
    \label{fig:fake_lm399_popcorn_noise}
\end{figure}

TODO: Chinese/Ebay Zeners. Welded legs. Photots. Decap one of those.

%\begin{figure}[h]
%    \centering
    %\import{figures/}{dgDrive_protocol.tex}
%\end{figure}

%\subsection{Current Sources}
%Discuss Op amp choice (AD797)

%\subsection{Temperature Coeeficient}
%Discuss each section (Reference, DAC, Buffer/Divider, Filter, CC)
%\subsubsection{Voltage Reference}
%\subsubsection{DAC}
%\subsubsection{Divider}
%\subsubsection{Filter}
%Choice of components. Leakage current, size of resistor (input bias current of AD797), size of capacitor

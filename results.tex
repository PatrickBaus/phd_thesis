\chapter{Results}
\begin{chapquote}{Sir Terry Pratchett, \textit{A Hat Full of Sky}}
``It's still magic even if you know how it's done.''
\end{chapquote}

\section{Laser Current Driver}
For this project several commercial and publicly available laser current drivers were evaluated for their performance. The following devices were all tested for the requirements listed in \ref{lst:dgDrive_specs_environment} and \ref{lst:dgDrive_specs_electrical}.
\begin{itemize}
    \item Moglabs \device{DLC-202}
    \item Newport \device{TLB-6800-LN}
    \item SISYPH \device{SMC11 Puy Mary}
    \item Toptica \device{DCC 110}
    \item Vescent \device{D2-105}
    \item LQO \device{LQprO} \cite{datasheet_LQprO}
    \item A driver based on the work of \citeauthor{laser_driver_digital} \cite{laser_driver_digital}
\end{itemize}

As a disclaimer, Moglabs, Vescent and SISYPH provided demo units, free of charge to the author and without any obligations regarding this work. The opinions and measurements in this work are in no way biased by this service. All of these drivers claim low-noise in various comparative forms, but vary in features. The \device{DLC-202}, the \device{TLB-6800-LN} and the \device{D2-105} additionally include a peltier controller. The \device{DLC-202} and the \device{TLB-6800-LN} have a modulation source. The \device{TLB-6800-LN}, the \device{DCC 110} and the design by \citeauthor{laser_driver_digital} also feature a digital interface.

The drivers were all compared to our requirements. While not all drivers feature a remote accessible interface, their performance was assesed nonetheless to have a broader range of choices. A performance comparison can be found in section \ref{}. Unfortunately, none the drivers was able to properly drive the high compliance voltage required by the blue laser diode \device{PL 450B} of about \qtyrange[range-units = single]{6}{7}{\V} \cite{datasheet_osram_pl450b}. As it was discussed in section \ref{sec:compliance_voltage}, the compliance voltage of all laser drivers based on the design of \citeauthor{libbrecht_hall} \cite{libbrecht_hall} is limited to around \qtyrange[range-units = single]{2}{3}{\V} at full output (compare \ref{eqn:minimum_mosfet_vds} for details). Since the compliance voltage is rougly proportional to the reciprocal of the output current current, limiting the maximum output current to about \qtyrange[range-units = single]{30}{40}{\percent} increases the compliance voltage to the required level. Not only does this limit the choice of drivers, but also requiring a \qty{500}{\mA} driver for a \qty{150}{\mA} laser diode seems excessive and does not help with the noise requirements, because the output noise of those drivers scales roughly with $I_{max}$ as detailed in section \ref{sec:current_source_noise}, since the op-amp noise is the limiting factor. This lead to the decision to design a current source that meets all of our requirements, while surpassing all available alternative and tackling the compliance voltage limit. This design and its individual components are discussed in the following sections. First, the state of the art is presented, then the problems we encountered are outlined and finally our design, that resolves these issues is presented and the caveats and technical challenges are discussed.

\subsection{The State of the Art in Laser Current Drivers}
Prior to this work, all laser drivers for scientific purposes, were more or less strictly following the design proposed by \citeauthor{libbrecht_hall} \cite{libbrecht_hall}. This design was presented in 1993 and back then, blue laser diodes were were not available and only developed in 1996. See \cite{blue_laser_diodes_history} for an interesting historic summary. Finally, the efforts of Isamu Akasaki, Hiroshi Amano and Shuji Nakamura were rewarded with the Nobel Prize in Physics in 2014. The original laser driver design was therefore created for laser diodes requiring a low current and low compliance voltage compared to modern laser diodes. While the design remains useful for many low power near-infrared (NIR) laser diodes, these shortcommings were never addressed or even acknowleged by commercial alternatives. Sadly, the topic of the compliance voltage is usually not even mentioned in the datasheets -- the Moglabs \device{DLC-202} and SISYPH \device{SMC11} are notable exceptions, but it is unclear from the datasheet to which version and/or currents of the devices the numbers relate. The Newport \device{TLB-6800-LN} is a bit different to the rest of the drivers tested, because it comes with Newport laser heads and reads its configuration data from the laser head. Ours came with a \device{Vantage TLB-7100} and the laser head needs to be connected for it to work. Without some reverse engineering, these drivers can only be used with certain Newport laser heads. The \device{TLB-6800-LN} is included in the list of devices anyway to give an idea of its performance in existing systems.

\begin{table}[ht]
    \centering
    \begin{tabularx}{0.95\linewidth}{l>{\raggedright\arraybackslash}Xl>{\raggedright\arraybackslash}X}
        Laser driver& Output current& Compliance voltage & Additional features \\
        \midrule
        Moglabs \device{DLC-202} & \textbf{\qty[text-series-to-math, reset-text-series = false, reset-math-version = false]{100}{\mA}}, \qty{250}{\mA}, \qty{500}{\mA}& \qty{3.1}{\V} & TEC, PID, Piezo\\
        Newport \device{TLB-6800-LN} & & --& TEC, Piezo, Digital\\
        SISYPH \device{SMC11} & \qty{210}{\mA}, \textbf{\qty[text-series-to-math, reset-text-series = false, reset-math-version = false]{470}{\mA}}& \qty{5}{\V}&\\
        Toptica \device{DCC 110} & \textbf{\qty[text-series-to-math, reset-text-series = false, reset-math-version = false]{100}{\mA}}, \qty{500}{\mA}, \qty{3}{\A}, \qty{5}{\A} & --&\\
        Vescent \device{D2-105} & \qty{200}{\mA}, \textbf{\qty[text-series-to-math, reset-text-series = false, reset-math-version = false]{500}{\mA}} &  --& TEC\\
        LQO \device{LQprO} & \textbf{\qty[text-series-to-math, reset-text-series = false, reset-math-version = false]{140}{\mA}}, \qty{400}{\mA} & --&
    \end{tabularx}
    \caption{Overview of laser current drivers tested. Marked in bold is the version tested in this work. A dash denotes that no official information is available.}
    \label{tab:laser_current_drivers_tested}
\end{table}

The drivers shown in table \ref{tab:laser_current_drivers_tested} will now be discussed in a little more detail to familiarize with them.

Starting with the Moglabs \device{DLC-202}, which is fully integrated unit, that leaves little functions to be desired, it includes the current driver, a pizo driver, a temperature controller for a thermoelectric cooler (TEC) and finally a lock-in amplifier with a PID controller. It is the most integrated solution, that was tested and brings all features required to set up an ECDL locked to an atomic transition. Some of its features are accessible via pin headers. The current source can be remotely enabled and the PID controller can be manipulated to allow relocking the laser or deliberately taking it out lock remotely. There is no way directly adjust the output current over the whole range. It also features a broad range of protection mechanisms for the laser diode, for example disconneting the driver in case of a short or open condition. All relavant quantities can be adjusted and read back from the front panel display. The manual is fairly comprehensive and gives a lot of examples to set up a laser system.

The Vescent \device{D2-105} also features more than just the laser driver and includes a TEC controller as well. Unfortunately, adjusting this temperature controller prooved fairly cumbersome, because the driver has to be opened to adjust trimpots inside the driver. When the unit is placed in a rack, this might even be impossible to realize. The laser current is normally adjusted via a 10-turn potentiometer and can also be steered via an external input, but the stability of the current then depends on the external control voltage. It does not feature any protection features like open or short detection, so disconnecting and reconnecting the cable to the laser will most likely damage the laser diode. The display can be switched to show all relavant quantities. Moreover, the manual is not as comprehensive as the manual of the \device{DLC-202}, but covers all relavant settings of the driver.

The SISYPH \device{SMC11} does not include any additional features, but covers the most important protection features like an open detection and shuts down the driver accordingly. It is fully rackmountable, unlike the drivers discussed so far. The setpoint of the driver is adjusted via a recessed trimpot using a screwdriver, which has prooven troublesome to adjust in the lab when not directly in front of the unit. The current can be externally adjusted using an input connector, but again this limits the stability to that of the external source. The driver does not have any display and all setpoints must either be adjusted blindly or a volmeter must be attached to the monitoring connector limiting the usefulness in a lab environment. The user manual covers only the basic settings and gives no details regaring the layout of the pin headers or external connectors making it hard to understand without having the device at hand.

The Toptica \device{DCC 110} is also only a current source and is rack mounted. It comes with a separate display module, which connect via the backplane. The setpoint is adjusted via 10-turn potentiometer and can additionally be adjusted via the backplane with an external signal, again limiting the stability of the driver to the external source. The manual is fairly comprehensice and covers all essentials.

The Newport \device{TLB-6800-LN} is the only driver, that has a digital interface and supports Standard Commands for Programmable Instruments (SCPI) commands. It incorporates a TEC controller and a piezo controller. Unfortunately, it only works with a limited number of lasers, because it reads some parameters, like the maximum output current from the laser head at startup. The user manual covers all device functions, but gives little detail about the hardware, making this a closed system.

The results of the performance tests conducted prior to building our own solution will be presented on the following pages, including problems typically encountered and their solution. These tests include the stability, current noise and output impedance of the drivers. Not all drivers were put through the full test, if it was already clear that they could not perform in our setup for a laser system for the spectroscopy of highly charged ions driving a blue laser diode.

\clearpage
\subsection{Laser Driver: Design Concept}
In order to interpret the results in sections \ref{sec:results_current_noise}, \ref{sec:results_stability}, it it helpful to discuss in more detail the design concept of the current state of the art, which is based on the driver design presented by \citeauthor{libbrecht_hall} \cite{libbrecht_hall}. The design can be split into the four building blocks shown in figure \ref{fig:laser_driver_libbrecht_hall_concept}. A supply voltage input filter, a reference voltage to create the setpoint, a unidrectional current source and some form of bidirectional current source used for modulating the laser current at high frequency.

\begin{figure}[ht]
    \centering
    %\resizebox {0.8\textwidth} {!} {
        \import{figures/}{laser_driver_libbrecht_hall_concept.tex}
    %} % resizebox
    \caption{Building blocks of a laser driver based on \cite{libbrecht_hall}.}
    \label{fig:laser_driver_libbrecht_hall_concept}
\end{figure}

The original design is the most straightforward approach and it is possible to reproduce it even on prototype printed circuit boards (PCBs). \citeauthor{laser_driver_digital} \cite{laser_driver_digital} replaced the potentiometer with a DAC, but left the other parts untouched. \citeauthor{laser_driver_qcl_taubman} \cite{laser_driver_qcl_taubman,laser_driver_qcl_taubman_multiplexer} published some extensive modifications, which not only replaced the reference circuit with a DAC and an \device{LTZ1000} reference, but also added extensive filtering of the supply. The next sections will discuss these different elements separately and give some in insight into the different versions of the elements found in literature. The sections also details problems discoverd and the solution proposed in the design presented here.

\clearpage
\subsection{Supply Filtering}
The supply section of the design by \citeauthor{libbrecht_hall} was shown simplified in figure \ref{fig:laser_driver_libbrecht_hall_concept}. The original filteronly consists of a CLC filter or sometimes called \pi-filter shown in figure \ref{fig:laser_driver_libbrecht_hall_filter}. Do note, that due to the small input capacitance, the filter is basically just an LC-filter.

\begin{figure}[ht]
    \centering
    %\resizebox {0.8\textwidth} {!} {
        \import{figures/}{laser_driver_libbrecht_hall_supply_filter.tex}
    %} % resizebox
    \caption{Power supply filter of a laser driver based on \cite{libbrecht_hall}. The op-amps are supply by the filtered voltage and the current source is supplied by the \device{LM317}.}
    \label{fig:laser_driver_libbrecht_hall_filter}
\end{figure}

The LC-filter is best suited for a low impedance source like a power supply, because it has a high input impedance. From the transfer function
\begin{equation}
    H(s) = \frac{Z_{out}}{Z_{in}} = \frac{\frac{1}{sC}}{sL + \frac{1}{sC}} = \frac{1}{s^2LC +1} = \frac{\frac{1}{LC}}{s^2 + \frac{1}{LC}} = \frac{\frac{1}{LC}}{\left(s+i\frac{1}{\sqrt{LC}}\right)\left(s-i\frac{1}{\sqrt{LC}}\right)}\,, \label{eqn:transfer_function_lc_filter}
\end{equation}
one can deduce, that the passband gain at DC is \num{1} (obviously) and additionally, there are two complex poles in the imaginary plane at $s = \frac{\pm i}{\sqrt{LC}}$, putting the cutoff frequency of the 2\textsuperscript{nd} order filter at $f_c = \qty{2.3}{\kHz}$. Do note, due to the imaginary poles, there is some gain peaking at $f_c$. Normally, this is damped by the parasitic resistance of the inductor and capacitor. A more detailed analysis follows below, because the cutoff frequency is all that is of interest right now.

In this design, the op-amps are directly driven off the filtered supply rail. Using this information it is possible to estimate the effectiveness of the filter. Using the \qty{30}{\nA_{rms}} in a \qty{100}{\kHz} bandwidth current noise requirement from table \ref{lst:dgDrive_specs_electrical}, the voltage noise at the sense resistor (\qty{14}{\ohm}, (see section \ref{sec:component_selection}) must be no more than \qty{420}{\nV_{rms}}. Now, taking for example a low-noise switch-mode power supply like the Rohde \& Schwarz \device{HMP4040} used at CERN \cite{hmp4040_noise}, which does have fairly pronouced noise at the switching frequency of around \qty{170}{\kHz} and harmonics. The author measured these glitches to be about \qty{3}{\mV_{pp}}. The noise will have to go through the filter and the supply rejection ratio (PSRR) of the op-amp. The PSRR of the \device{LT1028} at \qty{170}{\kHz} is about \num{e-2} under ideal conditions \cite{datasheet_LT1028} and the filter adds another \num{e-2} when accounting for a \qty{0.5}{\ohm} series resistance of the output capacitor. The total filtering adds up to about \num{e-4}, which still leaves \qty{21}{\nA_{pp}} of ripple on the drive current in a very small bandwidth.

To have a better rejection of such switch-mode noise, the filter must be improved. The paper presented by \citeauthor{laser_driver_qcl_taubman} \cite{laser_driver_qcl_taubman} shows a brute-force approach. They applied extremely high values for the capacitor $C_{LC}$ of the LC filter of \qty{10}{\milli\farad} and then put a second filter based on a so-called capacitance multiplier behind it. This implementation is shown in a simplified form in figure \ref{fig:laser_driver_taubman_filter} and briefly discussed now. For a more detailed schematic and part names see \cite{laser_driver_qcl_taubman}.

\begin{figure}[ht]
    \centering
    %\resizebox {0.8\textwidth} {!} {
        \import{figures/}{laser_driver_taubman_filter.tex}
    %} % resizebox
    \caption{Power supply filter using a capacitance multiplier for a cutoff frequency of \qty{0.5}{\Hz}. This is a simplified schematic based on \cite{laser_driver_qcl_taubman}. Only the positive rail is shown.}
    \label{fig:laser_driver_taubman_filter}
\end{figure}

\citeauthor{laser_driver_qcl_taubman} built this filter for a driver with a driving capacity of \qty{2}{\A}, which limits the size of the inductor that can be used, in order to make up for that, he is forced to use giant capacitors. The second stage of the filter comprises a capacitance multiplier, which is formed by wrapping a feedback loop around the 2\textsuperscript{nd} order filter created by $R_1 C_1$ and $R_2 C_2$. This feedback loop removes the main current from the filter resistors, to allow larger values for $R_1$ and $R_2$, while maintaining a fairly low output impedance of the filter. The properties of this construction will be discussed now.

As a first note, the circuit presented by \citeauthor{laser_driver_qcl_taubman} misses a detail, which should included, when handling such capacitances. The circuit must include a reverse polarity protection, or rather a reverse current protection. If the input is shorted by accident, the \qty{10}{\milli\farad} of capacitance would immediately discharge via the parasitic body diode of the slow start-up transistor, likely vaporizing everything in its path. This could be implemented by adding another transistor to act as a reverse current protection.

% TODO: Add an appendix about the capacitance multiplier. See capacitance multipler/The Capacitance Multiplier Theory.pdf
The following explanation relies on some basic knowlegde about transistors. A good introduction to transistors is the Transistor Manual \cite{transistor_bible}, some even call the bible of transistors. Armed with some basic knowledge, the capacitance multiplier can be discussed and it must already be said, that the term capacitance multiplier is a bit misleading. It neither multiplies the capacitance, nor does is behave like a real 2\textsuperscript{nd} order filter. The only thing that is multiplied by the gain of the transitor, is the output capacitance seen by load, making the capacitor look more ideal. Unfortunately, it  highly depends on the properties of the transistor(s) and the gain of transistors drops with increasing output current (although it rises with temperature). Make sure to consult the datasheet, which typically gives a plot of the ac current gain $h_{fe}$ vs the collector curren $I_C$ to see at which load current, $h_{fe}$ starts dropping. Another issue is the bandwidth of the circuit, because $h_{fe}$ of any transistor rolls off with frequency. The high frequency resonse of this filter then becomes constant. This limits the suppression at around \qty{1}{\kHz} (depending on the output current of course) to around \numrange{500}{1000}. It is usefuly at low frequencies though, as it is shown in the publication of \citeauthor{laser_driver_qcl_taubman}. Another point is the maximum ripple voltage, that can be filtered. The muliplied capacitance does, of course, not store the same amount of energy as a real capacitor. This means the maximum peak-to-peak input ripple is limited to about one diode drop of \qty{0.68}{\V} from the Collector-Emitter diode. If more ripplle rejection is required an additional resistor from the base of $Q_2$ to ground like shown in figure \ref{fig:laser_driver_taubman_filter} can be applied. This reduces the output voltage further though. In this design, the current through $R_m$ is sufficient.

As a final remark regarding the capacitance multiplier is the output impedance. The transistor has an output impedance like a diode, so it increases with decreasing current. This means it will also drop about \qty{1.2}{\V} at \qty{500}{\mA} and about \qty{0.3}{\V} at \qty{1}{\mA}. This behaviour must be taken care of by the voltage regulator following the capacitance multiplier. The \qty{2}{\V} drop is also not a problem in this use case and even comes in handy. If just the supply rail of the laser diode current is fed through the capacitance multiplier, as it is the most sensitive, and the supply rail for the op-amps is not, then those extra \qty{2}{\V} will do not be a problem. In section \ref{sec:component_selection} it was already mentioned, that for example the \device{AD797} op-amp needs a supply that is \qty{3}{\V} above the diode supply. This means, that less voltage needs to be dropped by the linear regulator that follows the filter. To sum it up, the capacitance multiplier behaves like an ordinary $RC$ filter, but with a lower output impedance and only works at low ripple voltages, is limited in the high frequency domain.

% Note: 'Minimizing Input Filter Requirements In Military Power Supply Designs' has a more elaborate design
The power supply filters applied in this design use a passive LC filter for the negative and positive rail, then a capacitance multiplier on the diode supply. The negative rail is simply mirrored from the positive rail and pnp instead of npn transistors and vice versa are used. The combined filter is shown in figure \ref{fig:laser_driver_dgdrive_filter}. The diode supply and the analog rail, which is taken before the capacitance multiplier, are fed to low noise post-regulators, the \device{LT3045} and its negative counterpart, the \device{LT3094}. Both regulators have excellent power supply ripple rejection (PSRR) out to at least \qty{1}{\MHz} of more than \num{e3}. At low frequency the PSRR is even higher and more than \num{e5} can be expected. This allows a combined PSRR of better than \num{e6} from low to high frequencies, even beyond \qty{1}{\MHz}.

\begin{figure}[ht]
    \centering
    %\resizebox {0.8\textwidth} {!} {
        \import{figures/}{laser_driver_dgdrive_supply_filter.tex}
    %} % resizebox
    \caption{Power supply filter of the digital current driver.}
    \label{fig:laser_driver_dgdrive_filter}
\end{figure}

Regarding the filter circuit shown in figure \ref{fig:laser_driver_dgdrive_filter} a few explaining words on the choice of components are in order before proceeding to the measurement of the PSRR. Going back to equation \ref{eqn:transfer_function_lc_filter}, we saw, that the undamped second filter has excessive ringing at the cutoff frequency, because the filter poles are imaginary. To address this, there are several solutions. The most simple one is adding a damping element, either in parallel to the capacitor or in parallel the inductor. In this case a damping element in parallel to the capacitor was chosen, because using a a damping element parallel to the inductor will degrade the filter performance by making the blocking inductor lossy. Using the arangement shown in figure \ref{fig:laser_driver_dgdrive_filter}, the transfer function can be calculated.
\begin{align}
    H(s) &= \frac{Z_{out}}{Z_{in}} \nonumber\\
    Z_{out} &= \left(R_d + Z_{C_d}\right) || Z_{C_1} = \left(R_d + \frac{1}{s C_d}\right) || \frac{1}{s C_1} = \left(\left(R_d + \frac{1}{sC_d}\right)^{-1} + sC_1\right)^{-1} \nonumber\\
    &= \frac{s C_d R_d +1}{s^2 C_1 C_d R_d + s \left(C_1 C_d\right)}\\
    Z_{in} &= sL_1 + \left(R_d + Z_{C_d}\right) \nonumber\\
    H(s) &= \frac{\frac{s C_d R_d +1}{s^2 C_1 C_d R_d + s \left(C_1 C_d\right)}}{s L_1  + \frac{s C_d R_d +1}{s^2 C_1 C_d R_d + s \left(C_1 + C_d\right)}} = \frac{s C_d R_d +1}{s^3 L_1 C_1 C_d R_d + s^2 L_1 \left(C_1 + C_d\right) + s C_d R_d +1 }
\end{align}

% TODO: Add derivation, because the Middlebrook paper is hard to come by. This is already prepared in 'Input Filter Derivation.pdf' and Input Filter Theory.pdf', but needs to be checked again.
This is the transfer function of a 3\textsuperscript{rd} order filter. This type of filter was discussed by \citeauthor{input_filter_middlebrook} \cite{input_filter_middlebrook} (reprinted in \cite{input_filter_middlebrook_reprint1} and \cite{input_filter_middlebrook_reprint2}). \citeauthor{input_filter_middlebrook} derived, that there is an optimal value for the series resitance $R_d$ given a capacitance $C_d$ and the filter components $L_1$ and $C_1$. This optimal value has minimal gain peaking, hence a minimal quality factor $Q$ at the resonance freuency. The existance of such an optimal value can be easily understood from the fact, that if $R_d = \infty$ the resonance frequency is $\omega_0 = \frac{1}{\sqrt{L_1 C_1}}$ and in case $R_d = 0$ it is $\omega_1 = \frac{1}{\sqrt{L_1 \left(C_1 + C_d\right)}}$. In between $\omega_0$ and $\omega_1$, there is a lossy zone, where $R_d$ due to its lossy nature reduces $Q$, but at both ends $Q = \infty$, so there must be a minimum in between. By calculating, the minimum value of the transfer function at the point of resonance, \citeauthor{input_filter_middlebrook} found the following results:
\begin{align}
    R_0 &\coloneqq \sqrt{\frac{L_1}{C_1}}\\
    n &\coloneqq \frac{C_d}{C_1} \Rightarrow C_d = n C_1 \label{eqn:lc_filter_cd}\\
    Q_{optimal} &= \sqrt{\frac{(4+3 n) (2+n)}{2 n^2 (4+n)}}\\
    R_d &= R_0 \cdot Q_{optimal} \label{eqn:lc_filter_rd}
\end{align}

% TODO: Derive the 2x / 10 K rule. This is prepared in 'Capacitor life doubles theory.pdf'.
From these equations, it can be seen, that the damping capacitor $C_d$ needs to be fairly large, depending on $n$. A critically damped system with $Q = \num{0.5}$ would be preferred, but this would require $n \approx 6$ making the $C_d$ prohibitively large. For this filter $n=4$ was chosen, so making the filter slightly underdamped, so a slight gain peaking at the resonace can be expected. The following componets were chosen. First, a large, low resistance inductor $L_1$ capable of carrying at least \qty{1}{\A} was chosen. In this case a Coilcraft \device{MSS1210-125KEB}. High reliability capacitors were chosen to ensure a long lifetime of the device. Choosing capacitors rated with a lifetime of \qty{5000}{\hour} at \qty{105}{\celsius} gives a expected service life of more than \qty{10}{\year}, when assuming an Arrhenius law with a doubling of the lifetime every \qty{10}{\kelvin}. Apart from the reliability of the capacitors, there are no special requirements for them as there is little ripple current to be expected. The input power supply is supposed to be a filtered low noise supply and not the unfiltered output of a DC/DC regulator. So it is possible to maximize $L_1$ and choose a physically smaller $C_1$ since board space is limited. This results in the following design values, calculated from equations \ref{eqn:lc_filter_cd} and \ref{eqn:lc_filter_cd}, given the components values discussed above.
\begin{align}
    C_1 &= \qty{100}{\uF} \nonumber\\
    n &= 4 \nonumber\\
    Q_{optimal} &\approx \num{0.61} \nonumber\\
    R_d &= \num{0.61} \approx R_0 \approx \qty{2}{\ohm} \nonumber\\
    C_d &= \qty{400}{\uF} \approx \qty{390}{\uF} \nonumber\\
    f_c &\approx \qty{300}{\Hz} \nonumber
\end{align}

Do note, that $R_d$ does include the equivalent series resistance (ESR) of $C_d$, so the ESR of the capacitor must be subtracted from the final value of the damping resistor placed on the board. This may even absolve one from the need for a discrete resistor if the ESR of the capacitor is high enough. The transistors chose were a combination of a Toshiba \device{TTA004B}/\device{TTC004B} and Onsemi \device{BC817-40}/\device{BC807-40} for the positive/negative rail. The \device{TTA004B}/\device{TTC004B} are good up to about \qty{500}{\mA}. At this point the gains start dropping. A higher power transistor like the Onsemi \device{D45H8}/\device{D44H8} used by \citeauthor{laser_driver_qcl_taubman} is recommended for $Q_1$ and $Q_3$.

Finally, one last part of the capacitance multiplier should be explained. Highlighted in green in figure \ref{fig:laser_driver_dgdrive_filter} is a fast startup circuit. At startup, the capacitor $C_m$ is discharged and \qty{15}{\V} will be applied, it will then begin to charge with a current of \qty{1.5}{\mA} through the \qty{10}{\kilo\ohm} resistor. Because the $Q_2$ is an emitter follower, hence the emitter follows the voltage at the base (minus a diode drop for the base-emitter diode). As a sidenote, when using this kind of circuit, since $Q_2$ is an emitter follower, all output capacitors, that follow the capacitance multiplier will charge at the same rate as $C_m$, voltage-wise, this means, that for every \qty{10}{\uF} of output capacitance, a current of \qty{1.5}{\mA} will flow. While this not significant at this moment it become so, when looking at the fast startup circuit. Applying the input voltage of \qty{15}{\V} at startup over the LED, a \qty{625}{\nm} Würth Elektronik \device{150080RS75000}, it will start conducting, resulting in a \qtyrange[range-units = single]{1.8}{2}{\V} drop. The current flowing into $C_m$ is therefore dependent on the diode series resistor, which was chosen to be \qty{510}{\ohm}, a value not particularily important in this case. So at startup about \qty{25}{\mA} will flow into $C_m$, which means \qty{2.5}{\mA \per \uF} will flow through $Q_1$ and $Q_2$. Assuming roughly \qty{100}{\uF} of distributed bypassing capacitance around the board, this is around \qty{500}{\mA}. All these values are still well below the damage threshold of the transistors (\qty{2.5}{\A} and \qty{0.5}{\A}) and the LED (\qty{30}{\mA}), but these values must be kept in mind, when adding larger output capacitors. The fast startup circuit ensures an output voltage of \qty{13}{\V} within \qty{100}{\ms} instead of around \qty{0.5}{\second}, reducing the time to boot and leaving more time for self-checks without impacting the user experience.

After choosing the values above, the filter was simulated using LTSpice to assert the validity of the parameters chosen. The simulation was conducted with a load current of \qty{500}{\mA} running through the capacitance multiplier to simulate the worst case. As discussed above, the gain of the transistor $h_{fe}$ drops at higher currents as the transistor saturates. This particularily affects the high frequency behaviour above \qty{10}{\kHz}. The source file can be found in \external{source/spice/input\_filter\_dgdrive.asc}. The simulation additionally includes the series resistance and parasitic parallel capacitance of $L_1$, the latter will induce some ringing at the self resonance frequency of the inductor at \qty{1}{\MHz} and limit the useful attenuation beyond that to around \num{e3} due to the capacitive coupling of the conductor windings. At the \qty{170}{\kHz} discussed before, the damping is about the same figure of merit, \num{e3}.

The suppression is an order of magnitude better than the filter used by \citeauthor{libbrecht_hall} and it does not even include the high performance regulators that follow. The transfer function for both the damped LC filter and the LC filter with the capacitance multiplier in series is plotted in figure \ref{fig:laser_driver_input_filter}. The self resonance peak at \qty{1}{\MHz} can be clearly seen and is not damped, but from the output impedance shows shows that, there is enough capacitance present to compensate for this. The output impedance above \qty{1}{\MHz}, it is dominated by the local bypass capacitors and not accurately represented by the simulation. It can be expected to by lower than the simulated results, which do not include those capacitors.
\begin{figure}[ht]
    \centering
    %% Creator: Matplotlib, PGF backend
%%
%% To include the figure in your LaTeX document, write
%%   \input{<filename>.pgf}
%%
%% Make sure the required packages are loaded in your preamble
%%   \usepackage{pgf}
%%
%% Also ensure that all the required font packages are loaded; for instance,
%% the lmodern package is sometimes necessary when using math font.
%%   \usepackage{lmodern}
%%
%% Figures using additional raster images can only be included by \input if
%% they are in the same directory as the main LaTeX file. For loading figures
%% from other directories you can use the `import` package
%%   \usepackage{import}
%%
%% and then include the figures with
%%   \import{<path to file>}{<filename>.pgf}
%%
%% Matplotlib used the following preamble
%%   \usepackage{siunitx}
%%   \usepackage{fontspec}
%%
\begingroup%
\makeatletter%
\begin{pgfpicture}%
\pgfpathrectangle{\pgfpointorigin}{\pgfqpoint{5.492126in}{3.394321in}}%
\pgfusepath{use as bounding box, clip}%
\begin{pgfscope}%
\pgfsetbuttcap%
\pgfsetmiterjoin%
\definecolor{currentfill}{rgb}{1.000000,1.000000,1.000000}%
\pgfsetfillcolor{currentfill}%
\pgfsetlinewidth{0.000000pt}%
\definecolor{currentstroke}{rgb}{1.000000,1.000000,1.000000}%
\pgfsetstrokecolor{currentstroke}%
\pgfsetdash{}{0pt}%
\pgfpathmoveto{\pgfqpoint{0.000000in}{0.000000in}}%
\pgfpathlineto{\pgfqpoint{5.492126in}{0.000000in}}%
\pgfpathlineto{\pgfqpoint{5.492126in}{3.394321in}}%
\pgfpathlineto{\pgfqpoint{0.000000in}{3.394321in}}%
\pgfpathlineto{\pgfqpoint{0.000000in}{0.000000in}}%
\pgfpathclose%
\pgfusepath{fill}%
\end{pgfscope}%
\begin{pgfscope}%
\pgfsetbuttcap%
\pgfsetmiterjoin%
\definecolor{currentfill}{rgb}{1.000000,1.000000,1.000000}%
\pgfsetfillcolor{currentfill}%
\pgfsetlinewidth{0.000000pt}%
\definecolor{currentstroke}{rgb}{0.000000,0.000000,0.000000}%
\pgfsetstrokecolor{currentstroke}%
\pgfsetstrokeopacity{0.000000}%
\pgfsetdash{}{0pt}%
\pgfpathmoveto{\pgfqpoint{0.690614in}{0.524170in}}%
\pgfpathlineto{\pgfqpoint{4.857257in}{0.524170in}}%
\pgfpathlineto{\pgfqpoint{4.857257in}{3.244321in}}%
\pgfpathlineto{\pgfqpoint{0.690614in}{3.244321in}}%
\pgfpathlineto{\pgfqpoint{0.690614in}{0.524170in}}%
\pgfpathclose%
\pgfusepath{fill}%
\end{pgfscope}%
\begin{pgfscope}%
\pgfpathrectangle{\pgfqpoint{0.690614in}{0.524170in}}{\pgfqpoint{4.166643in}{2.720151in}}%
\pgfusepath{clip}%
\pgfsetrectcap%
\pgfsetroundjoin%
\pgfsetlinewidth{0.803000pt}%
\definecolor{currentstroke}{rgb}{0.450000,0.450000,0.450000}%
\pgfsetstrokecolor{currentstroke}%
\pgfsetdash{}{0pt}%
\pgfpathmoveto{\pgfqpoint{1.353489in}{0.524170in}}%
\pgfpathlineto{\pgfqpoint{1.353489in}{3.244321in}}%
\pgfusepath{stroke}%
\end{pgfscope}%
\begin{pgfscope}%
\pgfsetbuttcap%
\pgfsetroundjoin%
\definecolor{currentfill}{rgb}{0.000000,0.000000,0.000000}%
\pgfsetfillcolor{currentfill}%
\pgfsetlinewidth{0.803000pt}%
\definecolor{currentstroke}{rgb}{0.000000,0.000000,0.000000}%
\pgfsetstrokecolor{currentstroke}%
\pgfsetdash{}{0pt}%
\pgfsys@defobject{currentmarker}{\pgfqpoint{0.000000in}{-0.048611in}}{\pgfqpoint{0.000000in}{0.000000in}}{%
\pgfpathmoveto{\pgfqpoint{0.000000in}{0.000000in}}%
\pgfpathlineto{\pgfqpoint{0.000000in}{-0.048611in}}%
\pgfusepath{stroke,fill}%
}%
\begin{pgfscope}%
\pgfsys@transformshift{1.353489in}{0.524170in}%
\pgfsys@useobject{currentmarker}{}%
\end{pgfscope}%
\end{pgfscope}%
\begin{pgfscope}%
\definecolor{textcolor}{rgb}{0.000000,0.000000,0.000000}%
\pgfsetstrokecolor{textcolor}%
\pgfsetfillcolor{textcolor}%
\pgftext[x=1.353489in,y=0.426948in,,top]{\color{textcolor}\rmfamily\fontsize{8.000000}{9.600000}\selectfont \(\displaystyle {10^{0}}\)}%
\end{pgfscope}%
\begin{pgfscope}%
\pgfpathrectangle{\pgfqpoint{0.690614in}{0.524170in}}{\pgfqpoint{4.166643in}{2.720151in}}%
\pgfusepath{clip}%
\pgfsetrectcap%
\pgfsetroundjoin%
\pgfsetlinewidth{0.803000pt}%
\definecolor{currentstroke}{rgb}{0.450000,0.450000,0.450000}%
\pgfsetstrokecolor{currentstroke}%
\pgfsetdash{}{0pt}%
\pgfpathmoveto{\pgfqpoint{2.300454in}{0.524170in}}%
\pgfpathlineto{\pgfqpoint{2.300454in}{3.244321in}}%
\pgfusepath{stroke}%
\end{pgfscope}%
\begin{pgfscope}%
\pgfsetbuttcap%
\pgfsetroundjoin%
\definecolor{currentfill}{rgb}{0.000000,0.000000,0.000000}%
\pgfsetfillcolor{currentfill}%
\pgfsetlinewidth{0.803000pt}%
\definecolor{currentstroke}{rgb}{0.000000,0.000000,0.000000}%
\pgfsetstrokecolor{currentstroke}%
\pgfsetdash{}{0pt}%
\pgfsys@defobject{currentmarker}{\pgfqpoint{0.000000in}{-0.048611in}}{\pgfqpoint{0.000000in}{0.000000in}}{%
\pgfpathmoveto{\pgfqpoint{0.000000in}{0.000000in}}%
\pgfpathlineto{\pgfqpoint{0.000000in}{-0.048611in}}%
\pgfusepath{stroke,fill}%
}%
\begin{pgfscope}%
\pgfsys@transformshift{2.300454in}{0.524170in}%
\pgfsys@useobject{currentmarker}{}%
\end{pgfscope}%
\end{pgfscope}%
\begin{pgfscope}%
\definecolor{textcolor}{rgb}{0.000000,0.000000,0.000000}%
\pgfsetstrokecolor{textcolor}%
\pgfsetfillcolor{textcolor}%
\pgftext[x=2.300454in,y=0.426948in,,top]{\color{textcolor}\rmfamily\fontsize{8.000000}{9.600000}\selectfont \(\displaystyle {10^{2}}\)}%
\end{pgfscope}%
\begin{pgfscope}%
\pgfpathrectangle{\pgfqpoint{0.690614in}{0.524170in}}{\pgfqpoint{4.166643in}{2.720151in}}%
\pgfusepath{clip}%
\pgfsetrectcap%
\pgfsetroundjoin%
\pgfsetlinewidth{0.803000pt}%
\definecolor{currentstroke}{rgb}{0.450000,0.450000,0.450000}%
\pgfsetstrokecolor{currentstroke}%
\pgfsetdash{}{0pt}%
\pgfpathmoveto{\pgfqpoint{3.247418in}{0.524170in}}%
\pgfpathlineto{\pgfqpoint{3.247418in}{3.244321in}}%
\pgfusepath{stroke}%
\end{pgfscope}%
\begin{pgfscope}%
\pgfsetbuttcap%
\pgfsetroundjoin%
\definecolor{currentfill}{rgb}{0.000000,0.000000,0.000000}%
\pgfsetfillcolor{currentfill}%
\pgfsetlinewidth{0.803000pt}%
\definecolor{currentstroke}{rgb}{0.000000,0.000000,0.000000}%
\pgfsetstrokecolor{currentstroke}%
\pgfsetdash{}{0pt}%
\pgfsys@defobject{currentmarker}{\pgfqpoint{0.000000in}{-0.048611in}}{\pgfqpoint{0.000000in}{0.000000in}}{%
\pgfpathmoveto{\pgfqpoint{0.000000in}{0.000000in}}%
\pgfpathlineto{\pgfqpoint{0.000000in}{-0.048611in}}%
\pgfusepath{stroke,fill}%
}%
\begin{pgfscope}%
\pgfsys@transformshift{3.247418in}{0.524170in}%
\pgfsys@useobject{currentmarker}{}%
\end{pgfscope}%
\end{pgfscope}%
\begin{pgfscope}%
\definecolor{textcolor}{rgb}{0.000000,0.000000,0.000000}%
\pgfsetstrokecolor{textcolor}%
\pgfsetfillcolor{textcolor}%
\pgftext[x=3.247418in,y=0.426948in,,top]{\color{textcolor}\rmfamily\fontsize{8.000000}{9.600000}\selectfont \(\displaystyle {10^{4}}\)}%
\end{pgfscope}%
\begin{pgfscope}%
\pgfpathrectangle{\pgfqpoint{0.690614in}{0.524170in}}{\pgfqpoint{4.166643in}{2.720151in}}%
\pgfusepath{clip}%
\pgfsetrectcap%
\pgfsetroundjoin%
\pgfsetlinewidth{0.803000pt}%
\definecolor{currentstroke}{rgb}{0.450000,0.450000,0.450000}%
\pgfsetstrokecolor{currentstroke}%
\pgfsetdash{}{0pt}%
\pgfpathmoveto{\pgfqpoint{4.194382in}{0.524170in}}%
\pgfpathlineto{\pgfqpoint{4.194382in}{3.244321in}}%
\pgfusepath{stroke}%
\end{pgfscope}%
\begin{pgfscope}%
\pgfsetbuttcap%
\pgfsetroundjoin%
\definecolor{currentfill}{rgb}{0.000000,0.000000,0.000000}%
\pgfsetfillcolor{currentfill}%
\pgfsetlinewidth{0.803000pt}%
\definecolor{currentstroke}{rgb}{0.000000,0.000000,0.000000}%
\pgfsetstrokecolor{currentstroke}%
\pgfsetdash{}{0pt}%
\pgfsys@defobject{currentmarker}{\pgfqpoint{0.000000in}{-0.048611in}}{\pgfqpoint{0.000000in}{0.000000in}}{%
\pgfpathmoveto{\pgfqpoint{0.000000in}{0.000000in}}%
\pgfpathlineto{\pgfqpoint{0.000000in}{-0.048611in}}%
\pgfusepath{stroke,fill}%
}%
\begin{pgfscope}%
\pgfsys@transformshift{4.194382in}{0.524170in}%
\pgfsys@useobject{currentmarker}{}%
\end{pgfscope}%
\end{pgfscope}%
\begin{pgfscope}%
\definecolor{textcolor}{rgb}{0.000000,0.000000,0.000000}%
\pgfsetstrokecolor{textcolor}%
\pgfsetfillcolor{textcolor}%
\pgftext[x=4.194382in,y=0.426948in,,top]{\color{textcolor}\rmfamily\fontsize{8.000000}{9.600000}\selectfont \(\displaystyle {10^{6}}\)}%
\end{pgfscope}%
\begin{pgfscope}%
\definecolor{textcolor}{rgb}{0.000000,0.000000,0.000000}%
\pgfsetstrokecolor{textcolor}%
\pgfsetfillcolor{textcolor}%
\pgftext[x=2.773936in,y=0.271531in,,top]{\color{textcolor}\rmfamily\fontsize{10.000000}{12.000000}\selectfont Frequency in \unit{\Hz}}%
\end{pgfscope}%
\begin{pgfscope}%
\pgfpathrectangle{\pgfqpoint{0.690614in}{0.524170in}}{\pgfqpoint{4.166643in}{2.720151in}}%
\pgfusepath{clip}%
\pgfsetrectcap%
\pgfsetroundjoin%
\pgfsetlinewidth{0.803000pt}%
\definecolor{currentstroke}{rgb}{0.450000,0.450000,0.450000}%
\pgfsetstrokecolor{currentstroke}%
\pgfsetdash{}{0pt}%
\pgfpathmoveto{\pgfqpoint{0.690614in}{1.044229in}}%
\pgfpathlineto{\pgfqpoint{4.857257in}{1.044229in}}%
\pgfusepath{stroke}%
\end{pgfscope}%
\begin{pgfscope}%
\pgfsetbuttcap%
\pgfsetroundjoin%
\definecolor{currentfill}{rgb}{0.000000,0.000000,0.000000}%
\pgfsetfillcolor{currentfill}%
\pgfsetlinewidth{0.803000pt}%
\definecolor{currentstroke}{rgb}{0.000000,0.000000,0.000000}%
\pgfsetstrokecolor{currentstroke}%
\pgfsetdash{}{0pt}%
\pgfsys@defobject{currentmarker}{\pgfqpoint{-0.048611in}{0.000000in}}{\pgfqpoint{-0.000000in}{0.000000in}}{%
\pgfpathmoveto{\pgfqpoint{-0.000000in}{0.000000in}}%
\pgfpathlineto{\pgfqpoint{-0.048611in}{0.000000in}}%
\pgfusepath{stroke,fill}%
}%
\begin{pgfscope}%
\pgfsys@transformshift{0.690614in}{1.044229in}%
\pgfsys@useobject{currentmarker}{}%
\end{pgfscope}%
\end{pgfscope}%
\begin{pgfscope}%
\definecolor{textcolor}{rgb}{0.000000,0.000000,0.000000}%
\pgfsetstrokecolor{textcolor}%
\pgfsetfillcolor{textcolor}%
\pgftext[x=0.337219in, y=1.005076in, left, base]{\color{textcolor}\rmfamily\fontsize{8.000000}{9.600000}\selectfont \(\displaystyle {10^{-7}}\)}%
\end{pgfscope}%
\begin{pgfscope}%
\pgfpathrectangle{\pgfqpoint{0.690614in}{0.524170in}}{\pgfqpoint{4.166643in}{2.720151in}}%
\pgfusepath{clip}%
\pgfsetrectcap%
\pgfsetroundjoin%
\pgfsetlinewidth{0.803000pt}%
\definecolor{currentstroke}{rgb}{0.450000,0.450000,0.450000}%
\pgfsetstrokecolor{currentstroke}%
\pgfsetdash{}{0pt}%
\pgfpathmoveto{\pgfqpoint{0.690614in}{1.638163in}}%
\pgfpathlineto{\pgfqpoint{4.857257in}{1.638163in}}%
\pgfusepath{stroke}%
\end{pgfscope}%
\begin{pgfscope}%
\pgfsetbuttcap%
\pgfsetroundjoin%
\definecolor{currentfill}{rgb}{0.000000,0.000000,0.000000}%
\pgfsetfillcolor{currentfill}%
\pgfsetlinewidth{0.803000pt}%
\definecolor{currentstroke}{rgb}{0.000000,0.000000,0.000000}%
\pgfsetstrokecolor{currentstroke}%
\pgfsetdash{}{0pt}%
\pgfsys@defobject{currentmarker}{\pgfqpoint{-0.048611in}{0.000000in}}{\pgfqpoint{-0.000000in}{0.000000in}}{%
\pgfpathmoveto{\pgfqpoint{-0.000000in}{0.000000in}}%
\pgfpathlineto{\pgfqpoint{-0.048611in}{0.000000in}}%
\pgfusepath{stroke,fill}%
}%
\begin{pgfscope}%
\pgfsys@transformshift{0.690614in}{1.638163in}%
\pgfsys@useobject{currentmarker}{}%
\end{pgfscope}%
\end{pgfscope}%
\begin{pgfscope}%
\definecolor{textcolor}{rgb}{0.000000,0.000000,0.000000}%
\pgfsetstrokecolor{textcolor}%
\pgfsetfillcolor{textcolor}%
\pgftext[x=0.337219in, y=1.599010in, left, base]{\color{textcolor}\rmfamily\fontsize{8.000000}{9.600000}\selectfont \(\displaystyle {10^{-5}}\)}%
\end{pgfscope}%
\begin{pgfscope}%
\pgfpathrectangle{\pgfqpoint{0.690614in}{0.524170in}}{\pgfqpoint{4.166643in}{2.720151in}}%
\pgfusepath{clip}%
\pgfsetrectcap%
\pgfsetroundjoin%
\pgfsetlinewidth{0.803000pt}%
\definecolor{currentstroke}{rgb}{0.450000,0.450000,0.450000}%
\pgfsetstrokecolor{currentstroke}%
\pgfsetdash{}{0pt}%
\pgfpathmoveto{\pgfqpoint{0.690614in}{2.232097in}}%
\pgfpathlineto{\pgfqpoint{4.857257in}{2.232097in}}%
\pgfusepath{stroke}%
\end{pgfscope}%
\begin{pgfscope}%
\pgfsetbuttcap%
\pgfsetroundjoin%
\definecolor{currentfill}{rgb}{0.000000,0.000000,0.000000}%
\pgfsetfillcolor{currentfill}%
\pgfsetlinewidth{0.803000pt}%
\definecolor{currentstroke}{rgb}{0.000000,0.000000,0.000000}%
\pgfsetstrokecolor{currentstroke}%
\pgfsetdash{}{0pt}%
\pgfsys@defobject{currentmarker}{\pgfqpoint{-0.048611in}{0.000000in}}{\pgfqpoint{-0.000000in}{0.000000in}}{%
\pgfpathmoveto{\pgfqpoint{-0.000000in}{0.000000in}}%
\pgfpathlineto{\pgfqpoint{-0.048611in}{0.000000in}}%
\pgfusepath{stroke,fill}%
}%
\begin{pgfscope}%
\pgfsys@transformshift{0.690614in}{2.232097in}%
\pgfsys@useobject{currentmarker}{}%
\end{pgfscope}%
\end{pgfscope}%
\begin{pgfscope}%
\definecolor{textcolor}{rgb}{0.000000,0.000000,0.000000}%
\pgfsetstrokecolor{textcolor}%
\pgfsetfillcolor{textcolor}%
\pgftext[x=0.337219in, y=2.192944in, left, base]{\color{textcolor}\rmfamily\fontsize{8.000000}{9.600000}\selectfont \(\displaystyle {10^{-3}}\)}%
\end{pgfscope}%
\begin{pgfscope}%
\pgfpathrectangle{\pgfqpoint{0.690614in}{0.524170in}}{\pgfqpoint{4.166643in}{2.720151in}}%
\pgfusepath{clip}%
\pgfsetrectcap%
\pgfsetroundjoin%
\pgfsetlinewidth{0.803000pt}%
\definecolor{currentstroke}{rgb}{0.450000,0.450000,0.450000}%
\pgfsetstrokecolor{currentstroke}%
\pgfsetdash{}{0pt}%
\pgfpathmoveto{\pgfqpoint{0.690614in}{2.826032in}}%
\pgfpathlineto{\pgfqpoint{4.857257in}{2.826032in}}%
\pgfusepath{stroke}%
\end{pgfscope}%
\begin{pgfscope}%
\pgfsetbuttcap%
\pgfsetroundjoin%
\definecolor{currentfill}{rgb}{0.000000,0.000000,0.000000}%
\pgfsetfillcolor{currentfill}%
\pgfsetlinewidth{0.803000pt}%
\definecolor{currentstroke}{rgb}{0.000000,0.000000,0.000000}%
\pgfsetstrokecolor{currentstroke}%
\pgfsetdash{}{0pt}%
\pgfsys@defobject{currentmarker}{\pgfqpoint{-0.048611in}{0.000000in}}{\pgfqpoint{-0.000000in}{0.000000in}}{%
\pgfpathmoveto{\pgfqpoint{-0.000000in}{0.000000in}}%
\pgfpathlineto{\pgfqpoint{-0.048611in}{0.000000in}}%
\pgfusepath{stroke,fill}%
}%
\begin{pgfscope}%
\pgfsys@transformshift{0.690614in}{2.826032in}%
\pgfsys@useobject{currentmarker}{}%
\end{pgfscope}%
\end{pgfscope}%
\begin{pgfscope}%
\definecolor{textcolor}{rgb}{0.000000,0.000000,0.000000}%
\pgfsetstrokecolor{textcolor}%
\pgfsetfillcolor{textcolor}%
\pgftext[x=0.337219in, y=2.786879in, left, base]{\color{textcolor}\rmfamily\fontsize{8.000000}{9.600000}\selectfont \(\displaystyle {10^{-1}}\)}%
\end{pgfscope}%
\begin{pgfscope}%
\definecolor{textcolor}{rgb}{0.000000,0.000000,0.000000}%
\pgfsetstrokecolor{textcolor}%
\pgfsetfillcolor{textcolor}%
\pgftext[x=0.281663in,y=1.884245in,,bottom,rotate=90.000000]{\color{textcolor}\rmfamily\fontsize{10.000000}{12.000000}\selectfont Magnitude in \unit{\V \per \V}}%
\end{pgfscope}%
\begin{pgfscope}%
\pgfpathrectangle{\pgfqpoint{0.690614in}{0.524170in}}{\pgfqpoint{4.166643in}{2.720151in}}%
\pgfusepath{clip}%
\pgfsetrectcap%
\pgfsetroundjoin%
\pgfsetlinewidth{1.003750pt}%
\definecolor{currentstroke}{rgb}{0.003922,0.450980,0.698039}%
\pgfsetstrokecolor{currentstroke}%
\pgfsetstrokeopacity{0.700000}%
\pgfsetdash{}{0pt}%
\pgfpathmoveto{\pgfqpoint{0.880007in}{3.114663in}}%
\pgfpathlineto{\pgfqpoint{1.344020in}{3.115827in}}%
\pgfpathlineto{\pgfqpoint{1.926403in}{3.119999in}}%
\pgfpathlineto{\pgfqpoint{2.333597in}{3.119427in}}%
\pgfpathlineto{\pgfqpoint{2.380946in}{3.116493in}}%
\pgfpathlineto{\pgfqpoint{2.414089in}{3.112418in}}%
\pgfpathlineto{\pgfqpoint{2.442498in}{3.106988in}}%
\pgfpathlineto{\pgfqpoint{2.470907in}{3.099328in}}%
\pgfpathlineto{\pgfqpoint{2.499316in}{3.089142in}}%
\pgfpathlineto{\pgfqpoint{2.527725in}{3.076314in}}%
\pgfpathlineto{\pgfqpoint{2.556134in}{3.060902in}}%
\pgfpathlineto{\pgfqpoint{2.584543in}{3.043083in}}%
\pgfpathlineto{\pgfqpoint{2.617687in}{3.019553in}}%
\pgfpathlineto{\pgfqpoint{2.650830in}{2.993403in}}%
\pgfpathlineto{\pgfqpoint{2.688709in}{2.960706in}}%
\pgfpathlineto{\pgfqpoint{2.731322in}{2.920846in}}%
\pgfpathlineto{\pgfqpoint{2.778671in}{2.873490in}}%
\pgfpathlineto{\pgfqpoint{2.844958in}{2.803637in}}%
\pgfpathlineto{\pgfqpoint{2.977533in}{2.662852in}}%
\pgfpathlineto{\pgfqpoint{3.029616in}{2.611098in}}%
\pgfpathlineto{\pgfqpoint{3.076964in}{2.567042in}}%
\pgfpathlineto{\pgfqpoint{3.124313in}{2.526049in}}%
\pgfpathlineto{\pgfqpoint{3.171661in}{2.487931in}}%
\pgfpathlineto{\pgfqpoint{3.223744in}{2.448781in}}%
\pgfpathlineto{\pgfqpoint{3.285297in}{2.405305in}}%
\pgfpathlineto{\pgfqpoint{3.365788in}{2.351347in}}%
\pgfpathlineto{\pgfqpoint{3.484159in}{2.274928in}}%
\pgfpathlineto{\pgfqpoint{3.844006in}{2.044085in}}%
\pgfpathlineto{\pgfqpoint{3.915028in}{1.995230in}}%
\pgfpathlineto{\pgfqpoint{3.967111in}{1.956769in}}%
\pgfpathlineto{\pgfqpoint{4.004989in}{1.926310in}}%
\pgfpathlineto{\pgfqpoint{4.038133in}{1.896859in}}%
\pgfpathlineto{\pgfqpoint{4.066542in}{1.868373in}}%
\pgfpathlineto{\pgfqpoint{4.090216in}{1.841118in}}%
\pgfpathlineto{\pgfqpoint{4.109156in}{1.815847in}}%
\pgfpathlineto{\pgfqpoint{4.123360in}{1.793943in}}%
\pgfpathlineto{\pgfqpoint{4.137564in}{1.768386in}}%
\pgfpathlineto{\pgfqpoint{4.151769in}{1.737419in}}%
\pgfpathlineto{\pgfqpoint{4.161239in}{1.712259in}}%
\pgfpathlineto{\pgfqpoint{4.170708in}{1.681570in}}%
\pgfpathlineto{\pgfqpoint{4.180178in}{1.642236in}}%
\pgfpathlineto{\pgfqpoint{4.189648in}{1.588193in}}%
\pgfpathlineto{\pgfqpoint{4.203852in}{1.479808in}}%
\pgfpathlineto{\pgfqpoint{4.208587in}{1.490156in}}%
\pgfpathlineto{\pgfqpoint{4.222791in}{1.601152in}}%
\pgfpathlineto{\pgfqpoint{4.232261in}{1.651070in}}%
\pgfpathlineto{\pgfqpoint{4.241731in}{1.687948in}}%
\pgfpathlineto{\pgfqpoint{4.251200in}{1.717028in}}%
\pgfpathlineto{\pgfqpoint{4.265405in}{1.751648in}}%
\pgfpathlineto{\pgfqpoint{4.279609in}{1.779417in}}%
\pgfpathlineto{\pgfqpoint{4.293814in}{1.802735in}}%
\pgfpathlineto{\pgfqpoint{4.312753in}{1.829167in}}%
\pgfpathlineto{\pgfqpoint{4.331692in}{1.851936in}}%
\pgfpathlineto{\pgfqpoint{4.355366in}{1.876896in}}%
\pgfpathlineto{\pgfqpoint{4.383775in}{1.903373in}}%
\pgfpathlineto{\pgfqpoint{4.416919in}{1.931134in}}%
\pgfpathlineto{\pgfqpoint{4.459532in}{1.963780in}}%
\pgfpathlineto{\pgfqpoint{4.516350in}{2.004362in}}%
\pgfpathlineto{\pgfqpoint{4.606312in}{2.065544in}}%
\pgfpathlineto{\pgfqpoint{4.667865in}{2.106364in}}%
\pgfpathlineto{\pgfqpoint{4.667865in}{2.106364in}}%
\pgfusepath{stroke}%
\end{pgfscope}%
\begin{pgfscope}%
\pgfpathrectangle{\pgfqpoint{0.690614in}{0.524170in}}{\pgfqpoint{4.166643in}{2.720151in}}%
\pgfusepath{clip}%
\pgfsetrectcap%
\pgfsetroundjoin%
\pgfsetlinewidth{1.003750pt}%
\definecolor{currentstroke}{rgb}{0.870588,0.560784,0.019608}%
\pgfsetstrokecolor{currentstroke}%
\pgfsetstrokeopacity{0.700000}%
\pgfsetdash{}{0pt}%
\pgfpathmoveto{\pgfqpoint{0.880007in}{3.110064in}}%
\pgfpathlineto{\pgfqpoint{1.107278in}{3.108529in}}%
\pgfpathlineto{\pgfqpoint{1.206710in}{3.105810in}}%
\pgfpathlineto{\pgfqpoint{1.277732in}{3.101671in}}%
\pgfpathlineto{\pgfqpoint{1.329815in}{3.096576in}}%
\pgfpathlineto{\pgfqpoint{1.377163in}{3.089742in}}%
\pgfpathlineto{\pgfqpoint{1.419777in}{3.081297in}}%
\pgfpathlineto{\pgfqpoint{1.457655in}{3.071662in}}%
\pgfpathlineto{\pgfqpoint{1.495534in}{3.059873in}}%
\pgfpathlineto{\pgfqpoint{1.533412in}{3.045923in}}%
\pgfpathlineto{\pgfqpoint{1.576026in}{3.027817in}}%
\pgfpathlineto{\pgfqpoint{1.623374in}{3.005116in}}%
\pgfpathlineto{\pgfqpoint{1.675457in}{2.977680in}}%
\pgfpathlineto{\pgfqpoint{1.741745in}{2.940162in}}%
\pgfpathlineto{\pgfqpoint{1.831706in}{2.886521in}}%
\pgfpathlineto{\pgfqpoint{1.987955in}{2.790327in}}%
\pgfpathlineto{\pgfqpoint{2.276779in}{2.612494in}}%
\pgfpathlineto{\pgfqpoint{2.371476in}{2.553631in}}%
\pgfpathlineto{\pgfqpoint{2.418824in}{2.521741in}}%
\pgfpathlineto{\pgfqpoint{2.456703in}{2.493731in}}%
\pgfpathlineto{\pgfqpoint{2.494581in}{2.462834in}}%
\pgfpathlineto{\pgfqpoint{2.532460in}{2.428931in}}%
\pgfpathlineto{\pgfqpoint{2.575073in}{2.387695in}}%
\pgfpathlineto{\pgfqpoint{2.627156in}{2.334114in}}%
\pgfpathlineto{\pgfqpoint{2.698179in}{2.257547in}}%
\pgfpathlineto{\pgfqpoint{2.783405in}{2.162217in}}%
\pgfpathlineto{\pgfqpoint{2.963329in}{1.959488in}}%
\pgfpathlineto{\pgfqpoint{3.020146in}{1.899750in}}%
\pgfpathlineto{\pgfqpoint{3.067495in}{1.853128in}}%
\pgfpathlineto{\pgfqpoint{3.114843in}{1.809756in}}%
\pgfpathlineto{\pgfqpoint{3.162191in}{1.769578in}}%
\pgfpathlineto{\pgfqpoint{3.214274in}{1.728587in}}%
\pgfpathlineto{\pgfqpoint{3.271092in}{1.686796in}}%
\pgfpathlineto{\pgfqpoint{3.346849in}{1.634066in}}%
\pgfpathlineto{\pgfqpoint{3.540977in}{1.503163in}}%
\pgfpathlineto{\pgfqpoint{3.725635in}{1.376476in}}%
\pgfpathlineto{\pgfqpoint{3.815597in}{1.312620in}}%
\pgfpathlineto{\pgfqpoint{3.872414in}{1.269743in}}%
\pgfpathlineto{\pgfqpoint{3.919763in}{1.231192in}}%
\pgfpathlineto{\pgfqpoint{3.957641in}{1.197623in}}%
\pgfpathlineto{\pgfqpoint{3.990785in}{1.165556in}}%
\pgfpathlineto{\pgfqpoint{4.023929in}{1.130178in}}%
\pgfpathlineto{\pgfqpoint{4.052338in}{1.096366in}}%
\pgfpathlineto{\pgfqpoint{4.076012in}{1.064859in}}%
\pgfpathlineto{\pgfqpoint{4.094951in}{1.036681in}}%
\pgfpathlineto{\pgfqpoint{4.113890in}{1.004824in}}%
\pgfpathlineto{\pgfqpoint{4.128095in}{0.977543in}}%
\pgfpathlineto{\pgfqpoint{4.142299in}{0.946059in}}%
\pgfpathlineto{\pgfqpoint{4.156504in}{0.908259in}}%
\pgfpathlineto{\pgfqpoint{4.165973in}{0.877652in}}%
\pgfpathlineto{\pgfqpoint{4.175443in}{0.840193in}}%
\pgfpathlineto{\pgfqpoint{4.184913in}{0.791563in}}%
\pgfpathlineto{\pgfqpoint{4.194382in}{0.723611in}}%
\pgfpathlineto{\pgfqpoint{4.203852in}{0.647813in}}%
\pgfpathlineto{\pgfqpoint{4.208587in}{0.656677in}}%
\pgfpathlineto{\pgfqpoint{4.222791in}{0.763232in}}%
\pgfpathlineto{\pgfqpoint{4.232261in}{0.810198in}}%
\pgfpathlineto{\pgfqpoint{4.241731in}{0.844130in}}%
\pgfpathlineto{\pgfqpoint{4.251200in}{0.870271in}}%
\pgfpathlineto{\pgfqpoint{4.260670in}{0.891344in}}%
\pgfpathlineto{\pgfqpoint{4.274874in}{0.916650in}}%
\pgfpathlineto{\pgfqpoint{4.289079in}{0.936854in}}%
\pgfpathlineto{\pgfqpoint{4.303283in}{0.953562in}}%
\pgfpathlineto{\pgfqpoint{4.322223in}{0.972009in}}%
\pgfpathlineto{\pgfqpoint{4.341162in}{0.987300in}}%
\pgfpathlineto{\pgfqpoint{4.364836in}{1.003215in}}%
\pgfpathlineto{\pgfqpoint{4.388510in}{1.016488in}}%
\pgfpathlineto{\pgfqpoint{4.416919in}{1.029794in}}%
\pgfpathlineto{\pgfqpoint{4.450063in}{1.042574in}}%
\pgfpathlineto{\pgfqpoint{4.487941in}{1.054464in}}%
\pgfpathlineto{\pgfqpoint{4.530555in}{1.065293in}}%
\pgfpathlineto{\pgfqpoint{4.577903in}{1.075035in}}%
\pgfpathlineto{\pgfqpoint{4.634721in}{1.084434in}}%
\pgfpathlineto{\pgfqpoint{4.667865in}{1.089055in}}%
\pgfpathlineto{\pgfqpoint{4.667865in}{1.089055in}}%
\pgfusepath{stroke}%
\end{pgfscope}%
\begin{pgfscope}%
\pgfsetrectcap%
\pgfsetmiterjoin%
\pgfsetlinewidth{0.803000pt}%
\definecolor{currentstroke}{rgb}{0.000000,0.000000,0.000000}%
\pgfsetstrokecolor{currentstroke}%
\pgfsetdash{}{0pt}%
\pgfpathmoveto{\pgfqpoint{0.690614in}{0.524170in}}%
\pgfpathlineto{\pgfqpoint{0.690614in}{3.244321in}}%
\pgfusepath{stroke}%
\end{pgfscope}%
\begin{pgfscope}%
\pgfsetrectcap%
\pgfsetmiterjoin%
\pgfsetlinewidth{0.803000pt}%
\definecolor{currentstroke}{rgb}{0.000000,0.000000,0.000000}%
\pgfsetstrokecolor{currentstroke}%
\pgfsetdash{}{0pt}%
\pgfpathmoveto{\pgfqpoint{4.857257in}{0.524170in}}%
\pgfpathlineto{\pgfqpoint{4.857257in}{3.244321in}}%
\pgfusepath{stroke}%
\end{pgfscope}%
\begin{pgfscope}%
\pgfsetrectcap%
\pgfsetmiterjoin%
\pgfsetlinewidth{0.803000pt}%
\definecolor{currentstroke}{rgb}{0.000000,0.000000,0.000000}%
\pgfsetstrokecolor{currentstroke}%
\pgfsetdash{}{0pt}%
\pgfpathmoveto{\pgfqpoint{0.690614in}{0.524170in}}%
\pgfpathlineto{\pgfqpoint{4.857257in}{0.524170in}}%
\pgfusepath{stroke}%
\end{pgfscope}%
\begin{pgfscope}%
\pgfsetrectcap%
\pgfsetmiterjoin%
\pgfsetlinewidth{0.803000pt}%
\definecolor{currentstroke}{rgb}{0.000000,0.000000,0.000000}%
\pgfsetstrokecolor{currentstroke}%
\pgfsetdash{}{0pt}%
\pgfpathmoveto{\pgfqpoint{0.690614in}{3.244321in}}%
\pgfpathlineto{\pgfqpoint{4.857257in}{3.244321in}}%
\pgfusepath{stroke}%
\end{pgfscope}%
\begin{pgfscope}%
\pgfsetbuttcap%
\pgfsetroundjoin%
\definecolor{currentfill}{rgb}{0.000000,0.000000,0.000000}%
\pgfsetfillcolor{currentfill}%
\pgfsetlinewidth{0.803000pt}%
\definecolor{currentstroke}{rgb}{0.000000,0.000000,0.000000}%
\pgfsetstrokecolor{currentstroke}%
\pgfsetdash{}{0pt}%
\pgfsys@defobject{currentmarker}{\pgfqpoint{0.000000in}{0.000000in}}{\pgfqpoint{0.048611in}{0.000000in}}{%
\pgfpathmoveto{\pgfqpoint{0.000000in}{0.000000in}}%
\pgfpathlineto{\pgfqpoint{0.048611in}{0.000000in}}%
\pgfusepath{stroke,fill}%
}%
\begin{pgfscope}%
\pgfsys@transformshift{4.857257in}{0.644015in}%
\pgfsys@useobject{currentmarker}{}%
\end{pgfscope}%
\end{pgfscope}%
\begin{pgfscope}%
\definecolor{textcolor}{rgb}{0.000000,0.000000,0.000000}%
\pgfsetstrokecolor{textcolor}%
\pgfsetfillcolor{textcolor}%
\pgftext[x=4.954480in, y=0.605460in, left, base]{\color{textcolor}\rmfamily\fontsize{8.000000}{9.600000}\selectfont \(\displaystyle {0.00}\)}%
\end{pgfscope}%
\begin{pgfscope}%
\pgfsetbuttcap%
\pgfsetroundjoin%
\definecolor{currentfill}{rgb}{0.000000,0.000000,0.000000}%
\pgfsetfillcolor{currentfill}%
\pgfsetlinewidth{0.803000pt}%
\definecolor{currentstroke}{rgb}{0.000000,0.000000,0.000000}%
\pgfsetstrokecolor{currentstroke}%
\pgfsetdash{}{0pt}%
\pgfsys@defobject{currentmarker}{\pgfqpoint{0.000000in}{0.000000in}}{\pgfqpoint{0.048611in}{0.000000in}}{%
\pgfpathmoveto{\pgfqpoint{0.000000in}{0.000000in}}%
\pgfpathlineto{\pgfqpoint{0.048611in}{0.000000in}}%
\pgfusepath{stroke,fill}%
}%
\begin{pgfscope}%
\pgfsys@transformshift{4.857257in}{0.952421in}%
\pgfsys@useobject{currentmarker}{}%
\end{pgfscope}%
\end{pgfscope}%
\begin{pgfscope}%
\definecolor{textcolor}{rgb}{0.000000,0.000000,0.000000}%
\pgfsetstrokecolor{textcolor}%
\pgfsetfillcolor{textcolor}%
\pgftext[x=4.954480in, y=0.913865in, left, base]{\color{textcolor}\rmfamily\fontsize{8.000000}{9.600000}\selectfont \(\displaystyle {0.25}\)}%
\end{pgfscope}%
\begin{pgfscope}%
\pgfsetbuttcap%
\pgfsetroundjoin%
\definecolor{currentfill}{rgb}{0.000000,0.000000,0.000000}%
\pgfsetfillcolor{currentfill}%
\pgfsetlinewidth{0.803000pt}%
\definecolor{currentstroke}{rgb}{0.000000,0.000000,0.000000}%
\pgfsetstrokecolor{currentstroke}%
\pgfsetdash{}{0pt}%
\pgfsys@defobject{currentmarker}{\pgfqpoint{0.000000in}{0.000000in}}{\pgfqpoint{0.048611in}{0.000000in}}{%
\pgfpathmoveto{\pgfqpoint{0.000000in}{0.000000in}}%
\pgfpathlineto{\pgfqpoint{0.048611in}{0.000000in}}%
\pgfusepath{stroke,fill}%
}%
\begin{pgfscope}%
\pgfsys@transformshift{4.857257in}{1.260826in}%
\pgfsys@useobject{currentmarker}{}%
\end{pgfscope}%
\end{pgfscope}%
\begin{pgfscope}%
\definecolor{textcolor}{rgb}{0.000000,0.000000,0.000000}%
\pgfsetstrokecolor{textcolor}%
\pgfsetfillcolor{textcolor}%
\pgftext[x=4.954480in, y=1.222271in, left, base]{\color{textcolor}\rmfamily\fontsize{8.000000}{9.600000}\selectfont \(\displaystyle {0.50}\)}%
\end{pgfscope}%
\begin{pgfscope}%
\pgfsetbuttcap%
\pgfsetroundjoin%
\definecolor{currentfill}{rgb}{0.000000,0.000000,0.000000}%
\pgfsetfillcolor{currentfill}%
\pgfsetlinewidth{0.803000pt}%
\definecolor{currentstroke}{rgb}{0.000000,0.000000,0.000000}%
\pgfsetstrokecolor{currentstroke}%
\pgfsetdash{}{0pt}%
\pgfsys@defobject{currentmarker}{\pgfqpoint{0.000000in}{0.000000in}}{\pgfqpoint{0.048611in}{0.000000in}}{%
\pgfpathmoveto{\pgfqpoint{0.000000in}{0.000000in}}%
\pgfpathlineto{\pgfqpoint{0.048611in}{0.000000in}}%
\pgfusepath{stroke,fill}%
}%
\begin{pgfscope}%
\pgfsys@transformshift{4.857257in}{1.569232in}%
\pgfsys@useobject{currentmarker}{}%
\end{pgfscope}%
\end{pgfscope}%
\begin{pgfscope}%
\definecolor{textcolor}{rgb}{0.000000,0.000000,0.000000}%
\pgfsetstrokecolor{textcolor}%
\pgfsetfillcolor{textcolor}%
\pgftext[x=4.954480in, y=1.530676in, left, base]{\color{textcolor}\rmfamily\fontsize{8.000000}{9.600000}\selectfont \(\displaystyle {0.75}\)}%
\end{pgfscope}%
\begin{pgfscope}%
\pgfsetbuttcap%
\pgfsetroundjoin%
\definecolor{currentfill}{rgb}{0.000000,0.000000,0.000000}%
\pgfsetfillcolor{currentfill}%
\pgfsetlinewidth{0.803000pt}%
\definecolor{currentstroke}{rgb}{0.000000,0.000000,0.000000}%
\pgfsetstrokecolor{currentstroke}%
\pgfsetdash{}{0pt}%
\pgfsys@defobject{currentmarker}{\pgfqpoint{0.000000in}{0.000000in}}{\pgfqpoint{0.048611in}{0.000000in}}{%
\pgfpathmoveto{\pgfqpoint{0.000000in}{0.000000in}}%
\pgfpathlineto{\pgfqpoint{0.048611in}{0.000000in}}%
\pgfusepath{stroke,fill}%
}%
\begin{pgfscope}%
\pgfsys@transformshift{4.857257in}{1.877637in}%
\pgfsys@useobject{currentmarker}{}%
\end{pgfscope}%
\end{pgfscope}%
\begin{pgfscope}%
\definecolor{textcolor}{rgb}{0.000000,0.000000,0.000000}%
\pgfsetstrokecolor{textcolor}%
\pgfsetfillcolor{textcolor}%
\pgftext[x=4.954480in, y=1.839081in, left, base]{\color{textcolor}\rmfamily\fontsize{8.000000}{9.600000}\selectfont \(\displaystyle {1.00}\)}%
\end{pgfscope}%
\begin{pgfscope}%
\pgfsetbuttcap%
\pgfsetroundjoin%
\definecolor{currentfill}{rgb}{0.000000,0.000000,0.000000}%
\pgfsetfillcolor{currentfill}%
\pgfsetlinewidth{0.803000pt}%
\definecolor{currentstroke}{rgb}{0.000000,0.000000,0.000000}%
\pgfsetstrokecolor{currentstroke}%
\pgfsetdash{}{0pt}%
\pgfsys@defobject{currentmarker}{\pgfqpoint{0.000000in}{0.000000in}}{\pgfqpoint{0.048611in}{0.000000in}}{%
\pgfpathmoveto{\pgfqpoint{0.000000in}{0.000000in}}%
\pgfpathlineto{\pgfqpoint{0.048611in}{0.000000in}}%
\pgfusepath{stroke,fill}%
}%
\begin{pgfscope}%
\pgfsys@transformshift{4.857257in}{2.186042in}%
\pgfsys@useobject{currentmarker}{}%
\end{pgfscope}%
\end{pgfscope}%
\begin{pgfscope}%
\definecolor{textcolor}{rgb}{0.000000,0.000000,0.000000}%
\pgfsetstrokecolor{textcolor}%
\pgfsetfillcolor{textcolor}%
\pgftext[x=4.954480in, y=2.147487in, left, base]{\color{textcolor}\rmfamily\fontsize{8.000000}{9.600000}\selectfont \(\displaystyle {1.25}\)}%
\end{pgfscope}%
\begin{pgfscope}%
\pgfsetbuttcap%
\pgfsetroundjoin%
\definecolor{currentfill}{rgb}{0.000000,0.000000,0.000000}%
\pgfsetfillcolor{currentfill}%
\pgfsetlinewidth{0.803000pt}%
\definecolor{currentstroke}{rgb}{0.000000,0.000000,0.000000}%
\pgfsetstrokecolor{currentstroke}%
\pgfsetdash{}{0pt}%
\pgfsys@defobject{currentmarker}{\pgfqpoint{0.000000in}{0.000000in}}{\pgfqpoint{0.048611in}{0.000000in}}{%
\pgfpathmoveto{\pgfqpoint{0.000000in}{0.000000in}}%
\pgfpathlineto{\pgfqpoint{0.048611in}{0.000000in}}%
\pgfusepath{stroke,fill}%
}%
\begin{pgfscope}%
\pgfsys@transformshift{4.857257in}{2.494448in}%
\pgfsys@useobject{currentmarker}{}%
\end{pgfscope}%
\end{pgfscope}%
\begin{pgfscope}%
\definecolor{textcolor}{rgb}{0.000000,0.000000,0.000000}%
\pgfsetstrokecolor{textcolor}%
\pgfsetfillcolor{textcolor}%
\pgftext[x=4.954480in, y=2.455892in, left, base]{\color{textcolor}\rmfamily\fontsize{8.000000}{9.600000}\selectfont \(\displaystyle {1.50}\)}%
\end{pgfscope}%
\begin{pgfscope}%
\pgfsetbuttcap%
\pgfsetroundjoin%
\definecolor{currentfill}{rgb}{0.000000,0.000000,0.000000}%
\pgfsetfillcolor{currentfill}%
\pgfsetlinewidth{0.803000pt}%
\definecolor{currentstroke}{rgb}{0.000000,0.000000,0.000000}%
\pgfsetstrokecolor{currentstroke}%
\pgfsetdash{}{0pt}%
\pgfsys@defobject{currentmarker}{\pgfqpoint{0.000000in}{0.000000in}}{\pgfqpoint{0.048611in}{0.000000in}}{%
\pgfpathmoveto{\pgfqpoint{0.000000in}{0.000000in}}%
\pgfpathlineto{\pgfqpoint{0.048611in}{0.000000in}}%
\pgfusepath{stroke,fill}%
}%
\begin{pgfscope}%
\pgfsys@transformshift{4.857257in}{2.802853in}%
\pgfsys@useobject{currentmarker}{}%
\end{pgfscope}%
\end{pgfscope}%
\begin{pgfscope}%
\definecolor{textcolor}{rgb}{0.000000,0.000000,0.000000}%
\pgfsetstrokecolor{textcolor}%
\pgfsetfillcolor{textcolor}%
\pgftext[x=4.954480in, y=2.764298in, left, base]{\color{textcolor}\rmfamily\fontsize{8.000000}{9.600000}\selectfont \(\displaystyle {1.75}\)}%
\end{pgfscope}%
\begin{pgfscope}%
\pgfsetbuttcap%
\pgfsetroundjoin%
\definecolor{currentfill}{rgb}{0.000000,0.000000,0.000000}%
\pgfsetfillcolor{currentfill}%
\pgfsetlinewidth{0.803000pt}%
\definecolor{currentstroke}{rgb}{0.000000,0.000000,0.000000}%
\pgfsetstrokecolor{currentstroke}%
\pgfsetdash{}{0pt}%
\pgfsys@defobject{currentmarker}{\pgfqpoint{0.000000in}{0.000000in}}{\pgfqpoint{0.048611in}{0.000000in}}{%
\pgfpathmoveto{\pgfqpoint{0.000000in}{0.000000in}}%
\pgfpathlineto{\pgfqpoint{0.048611in}{0.000000in}}%
\pgfusepath{stroke,fill}%
}%
\begin{pgfscope}%
\pgfsys@transformshift{4.857257in}{3.111259in}%
\pgfsys@useobject{currentmarker}{}%
\end{pgfscope}%
\end{pgfscope}%
\begin{pgfscope}%
\definecolor{textcolor}{rgb}{0.000000,0.000000,0.000000}%
\pgfsetstrokecolor{textcolor}%
\pgfsetfillcolor{textcolor}%
\pgftext[x=4.954480in, y=3.072703in, left, base]{\color{textcolor}\rmfamily\fontsize{8.000000}{9.600000}\selectfont \(\displaystyle {2.00}\)}%
\end{pgfscope}%
\begin{pgfscope}%
\definecolor{textcolor}{rgb}{0.000000,0.000000,0.000000}%
\pgfsetstrokecolor{textcolor}%
\pgfsetfillcolor{textcolor}%
\pgftext[x=5.219915in,y=1.884245in,,top,rotate=90.000000]{\color{textcolor}\rmfamily\fontsize{10.000000}{12.000000}\selectfont Impedance in \unit{\ohm}}%
\end{pgfscope}%
\begin{pgfscope}%
\pgfpathrectangle{\pgfqpoint{0.690614in}{0.524170in}}{\pgfqpoint{4.166643in}{2.720151in}}%
\pgfusepath{clip}%
\pgfsetbuttcap%
\pgfsetroundjoin%
\pgfsetlinewidth{1.003750pt}%
\definecolor{currentstroke}{rgb}{0.007843,0.619608,0.450980}%
\pgfsetstrokecolor{currentstroke}%
\pgfsetstrokeopacity{0.700000}%
\pgfsetdash{{3.700000pt}{1.600000pt}}{0.000000pt}%
\pgfpathmoveto{\pgfqpoint{0.880007in}{2.025061in}}%
\pgfpathlineto{\pgfqpoint{1.112013in}{2.026565in}}%
\pgfpathlineto{\pgfqpoint{1.216179in}{2.029303in}}%
\pgfpathlineto{\pgfqpoint{1.291936in}{2.033433in}}%
\pgfpathlineto{\pgfqpoint{1.358224in}{2.039314in}}%
\pgfpathlineto{\pgfqpoint{1.424511in}{2.047467in}}%
\pgfpathlineto{\pgfqpoint{1.613904in}{2.072632in}}%
\pgfpathlineto{\pgfqpoint{1.684927in}{2.078517in}}%
\pgfpathlineto{\pgfqpoint{1.779623in}{2.083750in}}%
\pgfpathlineto{\pgfqpoint{1.902729in}{2.090648in}}%
\pgfpathlineto{\pgfqpoint{1.959546in}{2.096149in}}%
\pgfpathlineto{\pgfqpoint{2.002160in}{2.102478in}}%
\pgfpathlineto{\pgfqpoint{2.035304in}{2.109422in}}%
\pgfpathlineto{\pgfqpoint{2.063712in}{2.117333in}}%
\pgfpathlineto{\pgfqpoint{2.092121in}{2.127629in}}%
\pgfpathlineto{\pgfqpoint{2.115796in}{2.138532in}}%
\pgfpathlineto{\pgfqpoint{2.139470in}{2.152066in}}%
\pgfpathlineto{\pgfqpoint{2.158409in}{2.165178in}}%
\pgfpathlineto{\pgfqpoint{2.177348in}{2.180689in}}%
\pgfpathlineto{\pgfqpoint{2.196287in}{2.198981in}}%
\pgfpathlineto{\pgfqpoint{2.215227in}{2.220465in}}%
\pgfpathlineto{\pgfqpoint{2.234166in}{2.245580in}}%
\pgfpathlineto{\pgfqpoint{2.253105in}{2.274764in}}%
\pgfpathlineto{\pgfqpoint{2.272045in}{2.308432in}}%
\pgfpathlineto{\pgfqpoint{2.290984in}{2.346929in}}%
\pgfpathlineto{\pgfqpoint{2.309923in}{2.390472in}}%
\pgfpathlineto{\pgfqpoint{2.328862in}{2.439065in}}%
\pgfpathlineto{\pgfqpoint{2.352537in}{2.506416in}}%
\pgfpathlineto{\pgfqpoint{2.380946in}{2.594777in}}%
\pgfpathlineto{\pgfqpoint{2.433029in}{2.759929in}}%
\pgfpathlineto{\pgfqpoint{2.451968in}{2.812788in}}%
\pgfpathlineto{\pgfqpoint{2.466172in}{2.847071in}}%
\pgfpathlineto{\pgfqpoint{2.480377in}{2.875489in}}%
\pgfpathlineto{\pgfqpoint{2.489846in}{2.890692in}}%
\pgfpathlineto{\pgfqpoint{2.499316in}{2.902628in}}%
\pgfpathlineto{\pgfqpoint{2.508786in}{2.911118in}}%
\pgfpathlineto{\pgfqpoint{2.518255in}{2.916041in}}%
\pgfpathlineto{\pgfqpoint{2.527725in}{2.917327in}}%
\pgfpathlineto{\pgfqpoint{2.537195in}{2.914957in}}%
\pgfpathlineto{\pgfqpoint{2.546664in}{2.908961in}}%
\pgfpathlineto{\pgfqpoint{2.556134in}{2.899409in}}%
\pgfpathlineto{\pgfqpoint{2.565604in}{2.886405in}}%
\pgfpathlineto{\pgfqpoint{2.575073in}{2.870081in}}%
\pgfpathlineto{\pgfqpoint{2.589278in}{2.839710in}}%
\pgfpathlineto{\pgfqpoint{2.603482in}{2.802798in}}%
\pgfpathlineto{\pgfqpoint{2.617687in}{2.759966in}}%
\pgfpathlineto{\pgfqpoint{2.636626in}{2.694764in}}%
\pgfpathlineto{\pgfqpoint{2.655565in}{2.621722in}}%
\pgfpathlineto{\pgfqpoint{2.679239in}{2.521713in}}%
\pgfpathlineto{\pgfqpoint{2.712383in}{2.370645in}}%
\pgfpathlineto{\pgfqpoint{2.811814in}{1.907478in}}%
\pgfpathlineto{\pgfqpoint{2.840223in}{1.787591in}}%
\pgfpathlineto{\pgfqpoint{2.868632in}{1.677471in}}%
\pgfpathlineto{\pgfqpoint{2.892306in}{1.593942in}}%
\pgfpathlineto{\pgfqpoint{2.915980in}{1.518201in}}%
\pgfpathlineto{\pgfqpoint{2.939655in}{1.450260in}}%
\pgfpathlineto{\pgfqpoint{2.963329in}{1.389934in}}%
\pgfpathlineto{\pgfqpoint{2.987003in}{1.336889in}}%
\pgfpathlineto{\pgfqpoint{3.005942in}{1.299402in}}%
\pgfpathlineto{\pgfqpoint{3.024881in}{1.266033in}}%
\pgfpathlineto{\pgfqpoint{3.043821in}{1.236500in}}%
\pgfpathlineto{\pgfqpoint{3.062760in}{1.210507in}}%
\pgfpathlineto{\pgfqpoint{3.081699in}{1.187754in}}%
\pgfpathlineto{\pgfqpoint{3.100638in}{1.167941in}}%
\pgfpathlineto{\pgfqpoint{3.119578in}{1.150772in}}%
\pgfpathlineto{\pgfqpoint{3.143252in}{1.132601in}}%
\pgfpathlineto{\pgfqpoint{3.166926in}{1.117598in}}%
\pgfpathlineto{\pgfqpoint{3.190600in}{1.105286in}}%
\pgfpathlineto{\pgfqpoint{3.214274in}{1.095236in}}%
\pgfpathlineto{\pgfqpoint{3.242683in}{1.085633in}}%
\pgfpathlineto{\pgfqpoint{3.275827in}{1.077125in}}%
\pgfpathlineto{\pgfqpoint{3.313705in}{1.070091in}}%
\pgfpathlineto{\pgfqpoint{3.356319in}{1.064640in}}%
\pgfpathlineto{\pgfqpoint{3.408402in}{1.060356in}}%
\pgfpathlineto{\pgfqpoint{3.474689in}{1.057214in}}%
\pgfpathlineto{\pgfqpoint{3.574121in}{1.054966in}}%
\pgfpathlineto{\pgfqpoint{3.787188in}{1.053080in}}%
\pgfpathlineto{\pgfqpoint{3.971846in}{1.050397in}}%
\pgfpathlineto{\pgfqpoint{4.170708in}{1.044912in}}%
\pgfpathlineto{\pgfqpoint{4.331692in}{1.038568in}}%
\pgfpathlineto{\pgfqpoint{4.450063in}{1.034427in}}%
\pgfpathlineto{\pgfqpoint{4.516350in}{1.034341in}}%
\pgfpathlineto{\pgfqpoint{4.587373in}{1.036506in}}%
\pgfpathlineto{\pgfqpoint{4.667865in}{1.040535in}}%
\pgfpathlineto{\pgfqpoint{4.667865in}{1.040535in}}%
\pgfusepath{stroke}%
\end{pgfscope}%
\begin{pgfscope}%
\pgfpathrectangle{\pgfqpoint{0.690614in}{0.524170in}}{\pgfqpoint{4.166643in}{2.720151in}}%
\pgfusepath{clip}%
\pgfsetbuttcap%
\pgfsetroundjoin%
\pgfsetlinewidth{1.003750pt}%
\definecolor{currentstroke}{rgb}{0.800000,0.470588,0.737255}%
\pgfsetstrokecolor{currentstroke}%
\pgfsetstrokeopacity{0.700000}%
\pgfsetdash{{3.700000pt}{1.600000pt}}{0.000000pt}%
\pgfpathmoveto{\pgfqpoint{0.880007in}{3.120677in}}%
\pgfpathlineto{\pgfqpoint{0.946294in}{3.116825in}}%
\pgfpathlineto{\pgfqpoint{0.993643in}{3.112108in}}%
\pgfpathlineto{\pgfqpoint{1.036256in}{3.105588in}}%
\pgfpathlineto{\pgfqpoint{1.069400in}{3.098341in}}%
\pgfpathlineto{\pgfqpoint{1.097809in}{3.090049in}}%
\pgfpathlineto{\pgfqpoint{1.126218in}{3.079246in}}%
\pgfpathlineto{\pgfqpoint{1.149892in}{3.067818in}}%
\pgfpathlineto{\pgfqpoint{1.173566in}{3.053657in}}%
\pgfpathlineto{\pgfqpoint{1.192505in}{3.039971in}}%
\pgfpathlineto{\pgfqpoint{1.211445in}{3.023821in}}%
\pgfpathlineto{\pgfqpoint{1.230384in}{3.004830in}}%
\pgfpathlineto{\pgfqpoint{1.249323in}{2.982590in}}%
\pgfpathlineto{\pgfqpoint{1.268262in}{2.956667in}}%
\pgfpathlineto{\pgfqpoint{1.287202in}{2.926614in}}%
\pgfpathlineto{\pgfqpoint{1.306141in}{2.891989in}}%
\pgfpathlineto{\pgfqpoint{1.325080in}{2.852378in}}%
\pgfpathlineto{\pgfqpoint{1.344020in}{2.807421in}}%
\pgfpathlineto{\pgfqpoint{1.362959in}{2.756845in}}%
\pgfpathlineto{\pgfqpoint{1.381898in}{2.700498in}}%
\pgfpathlineto{\pgfqpoint{1.405572in}{2.621966in}}%
\pgfpathlineto{\pgfqpoint{1.429246in}{2.534826in}}%
\pgfpathlineto{\pgfqpoint{1.457655in}{2.420146in}}%
\pgfpathlineto{\pgfqpoint{1.490799in}{2.275454in}}%
\pgfpathlineto{\pgfqpoint{1.547617in}{2.014028in}}%
\pgfpathlineto{\pgfqpoint{1.594965in}{1.799641in}}%
\pgfpathlineto{\pgfqpoint{1.628109in}{1.658834in}}%
\pgfpathlineto{\pgfqpoint{1.656518in}{1.546756in}}%
\pgfpathlineto{\pgfqpoint{1.684927in}{1.443698in}}%
\pgfpathlineto{\pgfqpoint{1.713336in}{1.350071in}}%
\pgfpathlineto{\pgfqpoint{1.737010in}{1.279241in}}%
\pgfpathlineto{\pgfqpoint{1.760684in}{1.214754in}}%
\pgfpathlineto{\pgfqpoint{1.784358in}{1.156317in}}%
\pgfpathlineto{\pgfqpoint{1.808032in}{1.103572in}}%
\pgfpathlineto{\pgfqpoint{1.831706in}{1.056125in}}%
\pgfpathlineto{\pgfqpoint{1.855380in}{1.013570in}}%
\pgfpathlineto{\pgfqpoint{1.879054in}{0.975502in}}%
\pgfpathlineto{\pgfqpoint{1.902729in}{0.941531in}}%
\pgfpathlineto{\pgfqpoint{1.926403in}{0.911287in}}%
\pgfpathlineto{\pgfqpoint{1.950077in}{0.884426in}}%
\pgfpathlineto{\pgfqpoint{1.973751in}{0.860629in}}%
\pgfpathlineto{\pgfqpoint{1.997425in}{0.839602in}}%
\pgfpathlineto{\pgfqpoint{2.025834in}{0.817654in}}%
\pgfpathlineto{\pgfqpoint{2.054243in}{0.798896in}}%
\pgfpathlineto{\pgfqpoint{2.082652in}{0.782955in}}%
\pgfpathlineto{\pgfqpoint{2.111061in}{0.769496in}}%
\pgfpathlineto{\pgfqpoint{2.144204in}{0.756525in}}%
\pgfpathlineto{\pgfqpoint{2.177348in}{0.746080in}}%
\pgfpathlineto{\pgfqpoint{2.215227in}{0.736722in}}%
\pgfpathlineto{\pgfqpoint{2.257840in}{0.728834in}}%
\pgfpathlineto{\pgfqpoint{2.305188in}{0.722530in}}%
\pgfpathlineto{\pgfqpoint{2.366741in}{0.716880in}}%
\pgfpathlineto{\pgfqpoint{2.466172in}{0.710510in}}%
\pgfpathlineto{\pgfqpoint{2.589278in}{0.704578in}}%
\pgfpathlineto{\pgfqpoint{2.707648in}{0.701252in}}%
\pgfpathlineto{\pgfqpoint{2.911246in}{0.698176in}}%
\pgfpathlineto{\pgfqpoint{3.318440in}{0.693843in}}%
\pgfpathlineto{\pgfqpoint{3.791922in}{0.691127in}}%
\pgfpathlineto{\pgfqpoint{3.891354in}{0.688013in}}%
\pgfpathlineto{\pgfqpoint{3.976581in}{0.683125in}}%
\pgfpathlineto{\pgfqpoint{4.071277in}{0.675294in}}%
\pgfpathlineto{\pgfqpoint{4.251200in}{0.659975in}}%
\pgfpathlineto{\pgfqpoint{4.345897in}{0.654524in}}%
\pgfpathlineto{\pgfqpoint{4.450063in}{0.650793in}}%
\pgfpathlineto{\pgfqpoint{4.577903in}{0.648517in}}%
\pgfpathlineto{\pgfqpoint{4.667865in}{0.647813in}}%
\pgfpathlineto{\pgfqpoint{4.667865in}{0.647813in}}%
\pgfusepath{stroke}%
\end{pgfscope}%
\begin{pgfscope}%
\pgfsetrectcap%
\pgfsetmiterjoin%
\pgfsetlinewidth{0.803000pt}%
\definecolor{currentstroke}{rgb}{0.000000,0.000000,0.000000}%
\pgfsetstrokecolor{currentstroke}%
\pgfsetdash{}{0pt}%
\pgfpathmoveto{\pgfqpoint{0.690614in}{0.524170in}}%
\pgfpathlineto{\pgfqpoint{0.690614in}{3.244321in}}%
\pgfusepath{stroke}%
\end{pgfscope}%
\begin{pgfscope}%
\pgfsetrectcap%
\pgfsetmiterjoin%
\pgfsetlinewidth{0.803000pt}%
\definecolor{currentstroke}{rgb}{0.000000,0.000000,0.000000}%
\pgfsetstrokecolor{currentstroke}%
\pgfsetdash{}{0pt}%
\pgfpathmoveto{\pgfqpoint{4.857257in}{0.524170in}}%
\pgfpathlineto{\pgfqpoint{4.857257in}{3.244321in}}%
\pgfusepath{stroke}%
\end{pgfscope}%
\begin{pgfscope}%
\pgfsetrectcap%
\pgfsetmiterjoin%
\pgfsetlinewidth{0.803000pt}%
\definecolor{currentstroke}{rgb}{0.000000,0.000000,0.000000}%
\pgfsetstrokecolor{currentstroke}%
\pgfsetdash{}{0pt}%
\pgfpathmoveto{\pgfqpoint{0.690614in}{0.524170in}}%
\pgfpathlineto{\pgfqpoint{4.857257in}{0.524170in}}%
\pgfusepath{stroke}%
\end{pgfscope}%
\begin{pgfscope}%
\pgfsetrectcap%
\pgfsetmiterjoin%
\pgfsetlinewidth{0.803000pt}%
\definecolor{currentstroke}{rgb}{0.000000,0.000000,0.000000}%
\pgfsetstrokecolor{currentstroke}%
\pgfsetdash{}{0pt}%
\pgfpathmoveto{\pgfqpoint{0.690614in}{3.244321in}}%
\pgfpathlineto{\pgfqpoint{4.857257in}{3.244321in}}%
\pgfusepath{stroke}%
\end{pgfscope}%
\begin{pgfscope}%
\pgfsetbuttcap%
\pgfsetmiterjoin%
\definecolor{currentfill}{rgb}{1.000000,1.000000,1.000000}%
\pgfsetfillcolor{currentfill}%
\pgfsetfillopacity{0.800000}%
\pgfsetlinewidth{1.003750pt}%
\definecolor{currentstroke}{rgb}{0.800000,0.800000,0.800000}%
\pgfsetstrokecolor{currentstroke}%
\pgfsetstrokeopacity{0.800000}%
\pgfsetdash{}{0pt}%
\pgfpathmoveto{\pgfqpoint{3.373480in}{2.532544in}}%
\pgfpathlineto{\pgfqpoint{4.779480in}{2.532544in}}%
\pgfpathquadraticcurveto{\pgfqpoint{4.801702in}{2.532544in}}{\pgfqpoint{4.801702in}{2.554766in}}%
\pgfpathlineto{\pgfqpoint{4.801702in}{3.166543in}}%
\pgfpathquadraticcurveto{\pgfqpoint{4.801702in}{3.188765in}}{\pgfqpoint{4.779480in}{3.188765in}}%
\pgfpathlineto{\pgfqpoint{3.373480in}{3.188765in}}%
\pgfpathquadraticcurveto{\pgfqpoint{3.351257in}{3.188765in}}{\pgfqpoint{3.351257in}{3.166543in}}%
\pgfpathlineto{\pgfqpoint{3.351257in}{2.554766in}}%
\pgfpathquadraticcurveto{\pgfqpoint{3.351257in}{2.532544in}}{\pgfqpoint{3.373480in}{2.532544in}}%
\pgfpathlineto{\pgfqpoint{3.373480in}{2.532544in}}%
\pgfpathclose%
\pgfusepath{stroke,fill}%
\end{pgfscope}%
\begin{pgfscope}%
\pgfsetrectcap%
\pgfsetroundjoin%
\pgfsetlinewidth{1.003750pt}%
\definecolor{currentstroke}{rgb}{0.003922,0.450980,0.698039}%
\pgfsetstrokecolor{currentstroke}%
\pgfsetstrokeopacity{0.700000}%
\pgfsetdash{}{0pt}%
\pgfpathmoveto{\pgfqpoint{3.395702in}{3.104987in}}%
\pgfpathlineto{\pgfqpoint{3.506813in}{3.104987in}}%
\pgfpathlineto{\pgfqpoint{3.617924in}{3.104987in}}%
\pgfusepath{stroke}%
\end{pgfscope}%
\begin{pgfscope}%
\definecolor{textcolor}{rgb}{0.000000,0.000000,0.000000}%
\pgfsetstrokecolor{textcolor}%
\pgfsetfillcolor{textcolor}%
\pgftext[x=3.706813in,y=3.066098in,left,base]{\color{textcolor}\rmfamily\fontsize{8.000000}{9.600000}\selectfont Mag. LC only}%
\end{pgfscope}%
\begin{pgfscope}%
\pgfsetrectcap%
\pgfsetroundjoin%
\pgfsetlinewidth{1.003750pt}%
\definecolor{currentstroke}{rgb}{0.870588,0.560784,0.019608}%
\pgfsetstrokecolor{currentstroke}%
\pgfsetstrokeopacity{0.700000}%
\pgfsetdash{}{0pt}%
\pgfpathmoveto{\pgfqpoint{3.395702in}{2.948432in}}%
\pgfpathlineto{\pgfqpoint{3.506813in}{2.948432in}}%
\pgfpathlineto{\pgfqpoint{3.617924in}{2.948432in}}%
\pgfusepath{stroke}%
\end{pgfscope}%
\begin{pgfscope}%
\definecolor{textcolor}{rgb}{0.000000,0.000000,0.000000}%
\pgfsetstrokecolor{textcolor}%
\pgfsetfillcolor{textcolor}%
\pgftext[x=3.706813in,y=2.909543in,left,base]{\color{textcolor}\rmfamily\fontsize{8.000000}{9.600000}\selectfont Mag. LC + C Mult.}%
\end{pgfscope}%
\begin{pgfscope}%
\pgfsetbuttcap%
\pgfsetroundjoin%
\pgfsetlinewidth{1.003750pt}%
\definecolor{currentstroke}{rgb}{0.007843,0.619608,0.450980}%
\pgfsetstrokecolor{currentstroke}%
\pgfsetstrokeopacity{0.700000}%
\pgfsetdash{{3.700000pt}{1.600000pt}}{0.000000pt}%
\pgfpathmoveto{\pgfqpoint{3.395702in}{2.792321in}}%
\pgfpathlineto{\pgfqpoint{3.506813in}{2.792321in}}%
\pgfpathlineto{\pgfqpoint{3.617924in}{2.792321in}}%
\pgfusepath{stroke}%
\end{pgfscope}%
\begin{pgfscope}%
\definecolor{textcolor}{rgb}{0.000000,0.000000,0.000000}%
\pgfsetstrokecolor{textcolor}%
\pgfsetfillcolor{textcolor}%
\pgftext[x=3.706813in,y=2.753432in,left,base]{\color{textcolor}\rmfamily\fontsize{8.000000}{9.600000}\selectfont \(\displaystyle Z_{out}\) LC filter}%
\end{pgfscope}%
\begin{pgfscope}%
\pgfsetbuttcap%
\pgfsetroundjoin%
\pgfsetlinewidth{1.003750pt}%
\definecolor{currentstroke}{rgb}{0.800000,0.470588,0.737255}%
\pgfsetstrokecolor{currentstroke}%
\pgfsetstrokeopacity{0.700000}%
\pgfsetdash{{3.700000pt}{1.600000pt}}{0.000000pt}%
\pgfpathmoveto{\pgfqpoint{3.395702in}{2.637432in}}%
\pgfpathlineto{\pgfqpoint{3.506813in}{2.637432in}}%
\pgfpathlineto{\pgfqpoint{3.617924in}{2.637432in}}%
\pgfusepath{stroke}%
\end{pgfscope}%
\begin{pgfscope}%
\definecolor{textcolor}{rgb}{0.000000,0.000000,0.000000}%
\pgfsetstrokecolor{textcolor}%
\pgfsetfillcolor{textcolor}%
\pgftext[x=3.706813in,y=2.598544in,left,base]{\color{textcolor}\rmfamily\fontsize{8.000000}{9.600000}\selectfont \(\displaystyle Z_{out}\) C Mult.}%
\end{pgfscope}%
\end{pgfpicture}%
\makeatother%
\endgroup%

    \caption{Simulated response of input filter used in the digital current driver. Both magnitude and output impedance of the stages are shown.}
    \label{fig:laser_driver_input_filter}
\end{figure}

At \qty{300}{\Hz}, the LC filter cutoff frequency, the output imedance of the LC filter shows some gain peaking. This is due to the underdamped response chosen and discussed above. This peaking increases the output impeance of \qty{1.2}{\ohm} in the passband, which is mostly the resistance of the inductor, to a total of \qty{1.8}{\ohm}. This can be easily compensated for by the regulators downstream.

The rejection ratio of the LC filter and the capacitance multiplier is better than \num{e3} at at \qty{100}{\kHz} an above, delivering the performance estimated above. This is expected to keep switch-mode noise away from the laser driver current.

The high rejection ratio of the filter is expected to make the the experimental validation rather challenging, because there are a number of complications that derive from the active nature of the circuit. The capacitance multiplier must be loaded, preferably at the maximum current to show the worst case and additionally, the ripple voltage must be low enough to not saturate the capacitance multiplier.

This requires a highly sensitive VNA, that has a low frequency range. This setup uses an Omicron Lab \device{Bode 100}, which can measure from \qty{1}{\Hz} to \qty{50}{\MHz} with an exceptionally low noise floor of about \qty{180}{\nV \Hz\tothe{-0.5}} \cite{datasheet_bode100}. Additionally a Stanford Research \device{SR560} was used as a pre-amplifier. To apply the ripple voltage to the power supply rails a Picotest \device{J2123A} negative line injector and a self-designed positive line injector was used. The positive line injector design is available open-source and be found in a Github repository at \cite{line_injector_github}. This injector is called \device{PB02}. During the measurement, it was found, that since the expected signal is extremely small, ground currents became an issue. There is an inherent ground loop issue built into the VNA. The outputs and inputs of the \device{Bode 100} are not isolated. The measrement is a 3-port measurement as shown in figure \ref{fig:laser_driver_supply_filter_measurement}. The \device{Bode 100} is driving the line injectors, measuring the signal going into the line injector and finally sampling the signal accross the output capacitor of the filter. The ground current now has two choices of flowing. One is through the low side of the measurement cables and their resistance, or through the ground plane of the VNA. The latter is the dreaded ground loop. This ground loop becomes more pronouced at higher frequencies, because the return path trough the cable is inductive and its impedance increases with frequency. Typically this problem would be addressed using a common-mode choke inserted into CH2. CH2 is the VNA input measuring the filter output. This common-mode choke prevents any current flowing through CH2, that has not flown through the cable.

%TODO: Add an image showing the ground loop, the Bode 100 and the 2-port shunt-thru measurement.
Unfortunately, the author did not have a suitable common mode choke at hand, so the only feasable solution to at least suppress the ground loop for low frequencies was to add transformers at the output and the CH2. This isolates the output and the battery powered \device{SR560} is driving the VNA via the transformer, isolating the input as well. The transformer used for isolating the VNA output, was an injection transformer named \device{PB01}. It is center tapped to create an anti-symmetrical output for the injection transformers. The center tap reduces the output amplitude by one half. The details regarding this device and its construction can be found in annex \ref{sec:injection_transformer}. The transformer used at the output of the \device{SR560} is a Picotest \device{J2123A}. Both transformers are unfortunately injection transformers and not dedicated isolation transformers as discussed in annex \ref{sec:injection_transformer}, yet the only transformers available at the time. The consequences of this subtle detail will become imminent in a moment.

\begin{figure}[ht]
    \centering
    \resizebox {0.9\textwidth} {!} {
        \import{figures/}{laser_driver_dgdrive_supply_filter_measurement_setup.tex}
    } % resizebox
    \caption{Power and grounding scheme for a low noise measurement of the line filter rejection ratio, minimizing the interfernce of circuit return currents.}
    \label{fig:laser_driver_supply_filter_measurement}
\end{figure}

The digital current driver is powered by the Rohde \& Schwarz \device{HMP4040} and there is a single point of ground connected to protective earth at the power supply. The power supply feeds into the line injectors, through which the current driver is powered. The output current is set to \qty{500}{\mA} accross a \qty{10}{\ohm} dummy resistor. The SR560 measures the ripple voltage after the LC-filter and drives the VNA input via the transformer. The measurement cable used is a short  twisted pair to reduce noise pickup.

The ouput of the VNA was set to \qty{-27}{\dB m} (\qty{10}{\mV_{rms}}), which must by multiplied by about $0.5 \cdot 0.975 = 0.485$ to give the ripple voltage on the positive supply, the latter term comes from the line injector \cite{line_injector_github}. Putting this into perspective, given a \qty{-60}{\dB} (\num{e-3}) suppression, results in a ripple voltage of only \qty{4.9}{\uV_{rms}}.

\begin{figure}[ht]
    \centering
    %% Creator: Matplotlib, PGF backend
%%
%% To include the figure in your LaTeX document, write
%%   \input{<filename>.pgf}
%%
%% Make sure the required packages are loaded in your preamble
%%   \usepackage{pgf}
%%
%% Also ensure that all the required font packages are loaded; for instance,
%% the lmodern package is sometimes necessary when using math font.
%%   \usepackage{lmodern}
%%
%% Figures using additional raster images can only be included by \input if
%% they are in the same directory as the main LaTeX file. For loading figures
%% from other directories you can use the `import` package
%%   \usepackage{import}
%%
%% and then include the figures with
%%   \import{<path to file>}{<filename>.pgf}
%%
%% Matplotlib used the following preamble
%%   \def\mathdefault#1{#1}
%%   \everymath=\expandafter{\the\everymath\displaystyle}
%%   \usepackage{siunitx}
%%   \sisetup{per-mode = symbol}%
%%   \ifdefined\pdftexversion\else  % non-pdftex case.
%%     \usepackage{fontspec}
%%   \fi
%%   \makeatletter\@ifpackageloaded{underscore}{}{\usepackage[strings]{underscore}}\makeatother
%%
\begingroup%
\makeatletter%
\begin{pgfpicture}%
\pgfpathrectangle{\pgfpointorigin}{\pgfqpoint{5.431103in}{3.356606in}}%
\pgfusepath{use as bounding box, clip}%
\begin{pgfscope}%
\pgfsetbuttcap%
\pgfsetmiterjoin%
\definecolor{currentfill}{rgb}{1.000000,1.000000,1.000000}%
\pgfsetfillcolor{currentfill}%
\pgfsetlinewidth{0.000000pt}%
\definecolor{currentstroke}{rgb}{1.000000,1.000000,1.000000}%
\pgfsetstrokecolor{currentstroke}%
\pgfsetdash{}{0pt}%
\pgfpathmoveto{\pgfqpoint{0.000000in}{0.000000in}}%
\pgfpathlineto{\pgfqpoint{5.431103in}{0.000000in}}%
\pgfpathlineto{\pgfqpoint{5.431103in}{3.356606in}}%
\pgfpathlineto{\pgfqpoint{0.000000in}{3.356606in}}%
\pgfpathlineto{\pgfqpoint{0.000000in}{0.000000in}}%
\pgfpathclose%
\pgfusepath{fill}%
\end{pgfscope}%
\begin{pgfscope}%
\pgfsetbuttcap%
\pgfsetmiterjoin%
\definecolor{currentfill}{rgb}{1.000000,1.000000,1.000000}%
\pgfsetfillcolor{currentfill}%
\pgfsetlinewidth{0.000000pt}%
\definecolor{currentstroke}{rgb}{0.000000,0.000000,0.000000}%
\pgfsetstrokecolor{currentstroke}%
\pgfsetstrokeopacity{0.000000}%
\pgfsetdash{}{0pt}%
\pgfpathmoveto{\pgfqpoint{0.693677in}{0.524170in}}%
\pgfpathlineto{\pgfqpoint{5.281103in}{0.524170in}}%
\pgfpathlineto{\pgfqpoint{5.281103in}{3.206606in}}%
\pgfpathlineto{\pgfqpoint{0.693677in}{3.206606in}}%
\pgfpathlineto{\pgfqpoint{0.693677in}{0.524170in}}%
\pgfpathclose%
\pgfusepath{fill}%
\end{pgfscope}%
\begin{pgfscope}%
\pgfpathrectangle{\pgfqpoint{0.693677in}{0.524170in}}{\pgfqpoint{4.587426in}{2.682436in}}%
\pgfusepath{clip}%
\pgfsetrectcap%
\pgfsetroundjoin%
\pgfsetlinewidth{0.803000pt}%
\definecolor{currentstroke}{rgb}{0.450000,0.450000,0.450000}%
\pgfsetstrokecolor{currentstroke}%
\pgfsetdash{}{0pt}%
\pgfpathmoveto{\pgfqpoint{0.902196in}{0.524170in}}%
\pgfpathlineto{\pgfqpoint{0.902196in}{3.206606in}}%
\pgfusepath{stroke}%
\end{pgfscope}%
\begin{pgfscope}%
\pgfsetbuttcap%
\pgfsetroundjoin%
\definecolor{currentfill}{rgb}{0.000000,0.000000,0.000000}%
\pgfsetfillcolor{currentfill}%
\pgfsetlinewidth{0.803000pt}%
\definecolor{currentstroke}{rgb}{0.000000,0.000000,0.000000}%
\pgfsetstrokecolor{currentstroke}%
\pgfsetdash{}{0pt}%
\pgfsys@defobject{currentmarker}{\pgfqpoint{0.000000in}{-0.048611in}}{\pgfqpoint{0.000000in}{0.000000in}}{%
\pgfpathmoveto{\pgfqpoint{0.000000in}{0.000000in}}%
\pgfpathlineto{\pgfqpoint{0.000000in}{-0.048611in}}%
\pgfusepath{stroke,fill}%
}%
\begin{pgfscope}%
\pgfsys@transformshift{0.902196in}{0.524170in}%
\pgfsys@useobject{currentmarker}{}%
\end{pgfscope}%
\end{pgfscope}%
\begin{pgfscope}%
\definecolor{textcolor}{rgb}{0.000000,0.000000,0.000000}%
\pgfsetstrokecolor{textcolor}%
\pgfsetfillcolor{textcolor}%
\pgftext[x=0.902196in,y=0.426948in,,top]{\color{textcolor}{\rmfamily\fontsize{8.000000}{9.600000}\selectfont\catcode`\^=\active\def^{\ifmmode\sp\else\^{}\fi}\catcode`\%=\active\def%{\%}$\mathdefault{10^{2}}$}}%
\end{pgfscope}%
\begin{pgfscope}%
\pgfpathrectangle{\pgfqpoint{0.693677in}{0.524170in}}{\pgfqpoint{4.587426in}{2.682436in}}%
\pgfusepath{clip}%
\pgfsetrectcap%
\pgfsetroundjoin%
\pgfsetlinewidth{0.803000pt}%
\definecolor{currentstroke}{rgb}{0.450000,0.450000,0.450000}%
\pgfsetstrokecolor{currentstroke}%
\pgfsetdash{}{0pt}%
\pgfpathmoveto{\pgfqpoint{1.944793in}{0.524170in}}%
\pgfpathlineto{\pgfqpoint{1.944793in}{3.206606in}}%
\pgfusepath{stroke}%
\end{pgfscope}%
\begin{pgfscope}%
\pgfsetbuttcap%
\pgfsetroundjoin%
\definecolor{currentfill}{rgb}{0.000000,0.000000,0.000000}%
\pgfsetfillcolor{currentfill}%
\pgfsetlinewidth{0.803000pt}%
\definecolor{currentstroke}{rgb}{0.000000,0.000000,0.000000}%
\pgfsetstrokecolor{currentstroke}%
\pgfsetdash{}{0pt}%
\pgfsys@defobject{currentmarker}{\pgfqpoint{0.000000in}{-0.048611in}}{\pgfqpoint{0.000000in}{0.000000in}}{%
\pgfpathmoveto{\pgfqpoint{0.000000in}{0.000000in}}%
\pgfpathlineto{\pgfqpoint{0.000000in}{-0.048611in}}%
\pgfusepath{stroke,fill}%
}%
\begin{pgfscope}%
\pgfsys@transformshift{1.944793in}{0.524170in}%
\pgfsys@useobject{currentmarker}{}%
\end{pgfscope}%
\end{pgfscope}%
\begin{pgfscope}%
\definecolor{textcolor}{rgb}{0.000000,0.000000,0.000000}%
\pgfsetstrokecolor{textcolor}%
\pgfsetfillcolor{textcolor}%
\pgftext[x=1.944793in,y=0.426948in,,top]{\color{textcolor}{\rmfamily\fontsize{8.000000}{9.600000}\selectfont\catcode`\^=\active\def^{\ifmmode\sp\else\^{}\fi}\catcode`\%=\active\def%{\%}$\mathdefault{10^{3}}$}}%
\end{pgfscope}%
\begin{pgfscope}%
\pgfpathrectangle{\pgfqpoint{0.693677in}{0.524170in}}{\pgfqpoint{4.587426in}{2.682436in}}%
\pgfusepath{clip}%
\pgfsetrectcap%
\pgfsetroundjoin%
\pgfsetlinewidth{0.803000pt}%
\definecolor{currentstroke}{rgb}{0.450000,0.450000,0.450000}%
\pgfsetstrokecolor{currentstroke}%
\pgfsetdash{}{0pt}%
\pgfpathmoveto{\pgfqpoint{2.987390in}{0.524170in}}%
\pgfpathlineto{\pgfqpoint{2.987390in}{3.206606in}}%
\pgfusepath{stroke}%
\end{pgfscope}%
\begin{pgfscope}%
\pgfsetbuttcap%
\pgfsetroundjoin%
\definecolor{currentfill}{rgb}{0.000000,0.000000,0.000000}%
\pgfsetfillcolor{currentfill}%
\pgfsetlinewidth{0.803000pt}%
\definecolor{currentstroke}{rgb}{0.000000,0.000000,0.000000}%
\pgfsetstrokecolor{currentstroke}%
\pgfsetdash{}{0pt}%
\pgfsys@defobject{currentmarker}{\pgfqpoint{0.000000in}{-0.048611in}}{\pgfqpoint{0.000000in}{0.000000in}}{%
\pgfpathmoveto{\pgfqpoint{0.000000in}{0.000000in}}%
\pgfpathlineto{\pgfqpoint{0.000000in}{-0.048611in}}%
\pgfusepath{stroke,fill}%
}%
\begin{pgfscope}%
\pgfsys@transformshift{2.987390in}{0.524170in}%
\pgfsys@useobject{currentmarker}{}%
\end{pgfscope}%
\end{pgfscope}%
\begin{pgfscope}%
\definecolor{textcolor}{rgb}{0.000000,0.000000,0.000000}%
\pgfsetstrokecolor{textcolor}%
\pgfsetfillcolor{textcolor}%
\pgftext[x=2.987390in,y=0.426948in,,top]{\color{textcolor}{\rmfamily\fontsize{8.000000}{9.600000}\selectfont\catcode`\^=\active\def^{\ifmmode\sp\else\^{}\fi}\catcode`\%=\active\def%{\%}$\mathdefault{10^{4}}$}}%
\end{pgfscope}%
\begin{pgfscope}%
\pgfpathrectangle{\pgfqpoint{0.693677in}{0.524170in}}{\pgfqpoint{4.587426in}{2.682436in}}%
\pgfusepath{clip}%
\pgfsetrectcap%
\pgfsetroundjoin%
\pgfsetlinewidth{0.803000pt}%
\definecolor{currentstroke}{rgb}{0.450000,0.450000,0.450000}%
\pgfsetstrokecolor{currentstroke}%
\pgfsetdash{}{0pt}%
\pgfpathmoveto{\pgfqpoint{4.029986in}{0.524170in}}%
\pgfpathlineto{\pgfqpoint{4.029986in}{3.206606in}}%
\pgfusepath{stroke}%
\end{pgfscope}%
\begin{pgfscope}%
\pgfsetbuttcap%
\pgfsetroundjoin%
\definecolor{currentfill}{rgb}{0.000000,0.000000,0.000000}%
\pgfsetfillcolor{currentfill}%
\pgfsetlinewidth{0.803000pt}%
\definecolor{currentstroke}{rgb}{0.000000,0.000000,0.000000}%
\pgfsetstrokecolor{currentstroke}%
\pgfsetdash{}{0pt}%
\pgfsys@defobject{currentmarker}{\pgfqpoint{0.000000in}{-0.048611in}}{\pgfqpoint{0.000000in}{0.000000in}}{%
\pgfpathmoveto{\pgfqpoint{0.000000in}{0.000000in}}%
\pgfpathlineto{\pgfqpoint{0.000000in}{-0.048611in}}%
\pgfusepath{stroke,fill}%
}%
\begin{pgfscope}%
\pgfsys@transformshift{4.029986in}{0.524170in}%
\pgfsys@useobject{currentmarker}{}%
\end{pgfscope}%
\end{pgfscope}%
\begin{pgfscope}%
\definecolor{textcolor}{rgb}{0.000000,0.000000,0.000000}%
\pgfsetstrokecolor{textcolor}%
\pgfsetfillcolor{textcolor}%
\pgftext[x=4.029986in,y=0.426948in,,top]{\color{textcolor}{\rmfamily\fontsize{8.000000}{9.600000}\selectfont\catcode`\^=\active\def^{\ifmmode\sp\else\^{}\fi}\catcode`\%=\active\def%{\%}$\mathdefault{10^{5}}$}}%
\end{pgfscope}%
\begin{pgfscope}%
\pgfpathrectangle{\pgfqpoint{0.693677in}{0.524170in}}{\pgfqpoint{4.587426in}{2.682436in}}%
\pgfusepath{clip}%
\pgfsetrectcap%
\pgfsetroundjoin%
\pgfsetlinewidth{0.803000pt}%
\definecolor{currentstroke}{rgb}{0.450000,0.450000,0.450000}%
\pgfsetstrokecolor{currentstroke}%
\pgfsetdash{}{0pt}%
\pgfpathmoveto{\pgfqpoint{5.072583in}{0.524170in}}%
\pgfpathlineto{\pgfqpoint{5.072583in}{3.206606in}}%
\pgfusepath{stroke}%
\end{pgfscope}%
\begin{pgfscope}%
\pgfsetbuttcap%
\pgfsetroundjoin%
\definecolor{currentfill}{rgb}{0.000000,0.000000,0.000000}%
\pgfsetfillcolor{currentfill}%
\pgfsetlinewidth{0.803000pt}%
\definecolor{currentstroke}{rgb}{0.000000,0.000000,0.000000}%
\pgfsetstrokecolor{currentstroke}%
\pgfsetdash{}{0pt}%
\pgfsys@defobject{currentmarker}{\pgfqpoint{0.000000in}{-0.048611in}}{\pgfqpoint{0.000000in}{0.000000in}}{%
\pgfpathmoveto{\pgfqpoint{0.000000in}{0.000000in}}%
\pgfpathlineto{\pgfqpoint{0.000000in}{-0.048611in}}%
\pgfusepath{stroke,fill}%
}%
\begin{pgfscope}%
\pgfsys@transformshift{5.072583in}{0.524170in}%
\pgfsys@useobject{currentmarker}{}%
\end{pgfscope}%
\end{pgfscope}%
\begin{pgfscope}%
\definecolor{textcolor}{rgb}{0.000000,0.000000,0.000000}%
\pgfsetstrokecolor{textcolor}%
\pgfsetfillcolor{textcolor}%
\pgftext[x=5.072583in,y=0.426948in,,top]{\color{textcolor}{\rmfamily\fontsize{8.000000}{9.600000}\selectfont\catcode`\^=\active\def^{\ifmmode\sp\else\^{}\fi}\catcode`\%=\active\def%{\%}$\mathdefault{10^{6}}$}}%
\end{pgfscope}%
\begin{pgfscope}%
\pgfpathrectangle{\pgfqpoint{0.693677in}{0.524170in}}{\pgfqpoint{4.587426in}{2.682436in}}%
\pgfusepath{clip}%
\pgfsetrectcap%
\pgfsetroundjoin%
\pgfsetlinewidth{0.803000pt}%
\definecolor{currentstroke}{rgb}{0.850000,0.850000,0.850000}%
\pgfsetstrokecolor{currentstroke}%
\pgfsetdash{}{0pt}%
\pgfpathmoveto{\pgfqpoint{0.740696in}{0.524170in}}%
\pgfpathlineto{\pgfqpoint{0.740696in}{3.206606in}}%
\pgfusepath{stroke}%
\end{pgfscope}%
\begin{pgfscope}%
\pgfsetbuttcap%
\pgfsetroundjoin%
\definecolor{currentfill}{rgb}{0.000000,0.000000,0.000000}%
\pgfsetfillcolor{currentfill}%
\pgfsetlinewidth{0.602250pt}%
\definecolor{currentstroke}{rgb}{0.000000,0.000000,0.000000}%
\pgfsetstrokecolor{currentstroke}%
\pgfsetdash{}{0pt}%
\pgfsys@defobject{currentmarker}{\pgfqpoint{0.000000in}{-0.027778in}}{\pgfqpoint{0.000000in}{0.000000in}}{%
\pgfpathmoveto{\pgfqpoint{0.000000in}{0.000000in}}%
\pgfpathlineto{\pgfqpoint{0.000000in}{-0.027778in}}%
\pgfusepath{stroke,fill}%
}%
\begin{pgfscope}%
\pgfsys@transformshift{0.740696in}{0.524170in}%
\pgfsys@useobject{currentmarker}{}%
\end{pgfscope}%
\end{pgfscope}%
\begin{pgfscope}%
\pgfpathrectangle{\pgfqpoint{0.693677in}{0.524170in}}{\pgfqpoint{4.587426in}{2.682436in}}%
\pgfusepath{clip}%
\pgfsetrectcap%
\pgfsetroundjoin%
\pgfsetlinewidth{0.803000pt}%
\definecolor{currentstroke}{rgb}{0.850000,0.850000,0.850000}%
\pgfsetstrokecolor{currentstroke}%
\pgfsetdash{}{0pt}%
\pgfpathmoveto{\pgfqpoint{0.801158in}{0.524170in}}%
\pgfpathlineto{\pgfqpoint{0.801158in}{3.206606in}}%
\pgfusepath{stroke}%
\end{pgfscope}%
\begin{pgfscope}%
\pgfsetbuttcap%
\pgfsetroundjoin%
\definecolor{currentfill}{rgb}{0.000000,0.000000,0.000000}%
\pgfsetfillcolor{currentfill}%
\pgfsetlinewidth{0.602250pt}%
\definecolor{currentstroke}{rgb}{0.000000,0.000000,0.000000}%
\pgfsetstrokecolor{currentstroke}%
\pgfsetdash{}{0pt}%
\pgfsys@defobject{currentmarker}{\pgfqpoint{0.000000in}{-0.027778in}}{\pgfqpoint{0.000000in}{0.000000in}}{%
\pgfpathmoveto{\pgfqpoint{0.000000in}{0.000000in}}%
\pgfpathlineto{\pgfqpoint{0.000000in}{-0.027778in}}%
\pgfusepath{stroke,fill}%
}%
\begin{pgfscope}%
\pgfsys@transformshift{0.801158in}{0.524170in}%
\pgfsys@useobject{currentmarker}{}%
\end{pgfscope}%
\end{pgfscope}%
\begin{pgfscope}%
\pgfpathrectangle{\pgfqpoint{0.693677in}{0.524170in}}{\pgfqpoint{4.587426in}{2.682436in}}%
\pgfusepath{clip}%
\pgfsetrectcap%
\pgfsetroundjoin%
\pgfsetlinewidth{0.803000pt}%
\definecolor{currentstroke}{rgb}{0.850000,0.850000,0.850000}%
\pgfsetstrokecolor{currentstroke}%
\pgfsetdash{}{0pt}%
\pgfpathmoveto{\pgfqpoint{0.854490in}{0.524170in}}%
\pgfpathlineto{\pgfqpoint{0.854490in}{3.206606in}}%
\pgfusepath{stroke}%
\end{pgfscope}%
\begin{pgfscope}%
\pgfsetbuttcap%
\pgfsetroundjoin%
\definecolor{currentfill}{rgb}{0.000000,0.000000,0.000000}%
\pgfsetfillcolor{currentfill}%
\pgfsetlinewidth{0.602250pt}%
\definecolor{currentstroke}{rgb}{0.000000,0.000000,0.000000}%
\pgfsetstrokecolor{currentstroke}%
\pgfsetdash{}{0pt}%
\pgfsys@defobject{currentmarker}{\pgfqpoint{0.000000in}{-0.027778in}}{\pgfqpoint{0.000000in}{0.000000in}}{%
\pgfpathmoveto{\pgfqpoint{0.000000in}{0.000000in}}%
\pgfpathlineto{\pgfqpoint{0.000000in}{-0.027778in}}%
\pgfusepath{stroke,fill}%
}%
\begin{pgfscope}%
\pgfsys@transformshift{0.854490in}{0.524170in}%
\pgfsys@useobject{currentmarker}{}%
\end{pgfscope}%
\end{pgfscope}%
\begin{pgfscope}%
\pgfpathrectangle{\pgfqpoint{0.693677in}{0.524170in}}{\pgfqpoint{4.587426in}{2.682436in}}%
\pgfusepath{clip}%
\pgfsetrectcap%
\pgfsetroundjoin%
\pgfsetlinewidth{0.803000pt}%
\definecolor{currentstroke}{rgb}{0.850000,0.850000,0.850000}%
\pgfsetstrokecolor{currentstroke}%
\pgfsetdash{}{0pt}%
\pgfpathmoveto{\pgfqpoint{1.216049in}{0.524170in}}%
\pgfpathlineto{\pgfqpoint{1.216049in}{3.206606in}}%
\pgfusepath{stroke}%
\end{pgfscope}%
\begin{pgfscope}%
\pgfsetbuttcap%
\pgfsetroundjoin%
\definecolor{currentfill}{rgb}{0.000000,0.000000,0.000000}%
\pgfsetfillcolor{currentfill}%
\pgfsetlinewidth{0.602250pt}%
\definecolor{currentstroke}{rgb}{0.000000,0.000000,0.000000}%
\pgfsetstrokecolor{currentstroke}%
\pgfsetdash{}{0pt}%
\pgfsys@defobject{currentmarker}{\pgfqpoint{0.000000in}{-0.027778in}}{\pgfqpoint{0.000000in}{0.000000in}}{%
\pgfpathmoveto{\pgfqpoint{0.000000in}{0.000000in}}%
\pgfpathlineto{\pgfqpoint{0.000000in}{-0.027778in}}%
\pgfusepath{stroke,fill}%
}%
\begin{pgfscope}%
\pgfsys@transformshift{1.216049in}{0.524170in}%
\pgfsys@useobject{currentmarker}{}%
\end{pgfscope}%
\end{pgfscope}%
\begin{pgfscope}%
\pgfpathrectangle{\pgfqpoint{0.693677in}{0.524170in}}{\pgfqpoint{4.587426in}{2.682436in}}%
\pgfusepath{clip}%
\pgfsetrectcap%
\pgfsetroundjoin%
\pgfsetlinewidth{0.803000pt}%
\definecolor{currentstroke}{rgb}{0.850000,0.850000,0.850000}%
\pgfsetstrokecolor{currentstroke}%
\pgfsetdash{}{0pt}%
\pgfpathmoveto{\pgfqpoint{1.399641in}{0.524170in}}%
\pgfpathlineto{\pgfqpoint{1.399641in}{3.206606in}}%
\pgfusepath{stroke}%
\end{pgfscope}%
\begin{pgfscope}%
\pgfsetbuttcap%
\pgfsetroundjoin%
\definecolor{currentfill}{rgb}{0.000000,0.000000,0.000000}%
\pgfsetfillcolor{currentfill}%
\pgfsetlinewidth{0.602250pt}%
\definecolor{currentstroke}{rgb}{0.000000,0.000000,0.000000}%
\pgfsetstrokecolor{currentstroke}%
\pgfsetdash{}{0pt}%
\pgfsys@defobject{currentmarker}{\pgfqpoint{0.000000in}{-0.027778in}}{\pgfqpoint{0.000000in}{0.000000in}}{%
\pgfpathmoveto{\pgfqpoint{0.000000in}{0.000000in}}%
\pgfpathlineto{\pgfqpoint{0.000000in}{-0.027778in}}%
\pgfusepath{stroke,fill}%
}%
\begin{pgfscope}%
\pgfsys@transformshift{1.399641in}{0.524170in}%
\pgfsys@useobject{currentmarker}{}%
\end{pgfscope}%
\end{pgfscope}%
\begin{pgfscope}%
\pgfpathrectangle{\pgfqpoint{0.693677in}{0.524170in}}{\pgfqpoint{4.587426in}{2.682436in}}%
\pgfusepath{clip}%
\pgfsetrectcap%
\pgfsetroundjoin%
\pgfsetlinewidth{0.803000pt}%
\definecolor{currentstroke}{rgb}{0.850000,0.850000,0.850000}%
\pgfsetstrokecolor{currentstroke}%
\pgfsetdash{}{0pt}%
\pgfpathmoveto{\pgfqpoint{1.529902in}{0.524170in}}%
\pgfpathlineto{\pgfqpoint{1.529902in}{3.206606in}}%
\pgfusepath{stroke}%
\end{pgfscope}%
\begin{pgfscope}%
\pgfsetbuttcap%
\pgfsetroundjoin%
\definecolor{currentfill}{rgb}{0.000000,0.000000,0.000000}%
\pgfsetfillcolor{currentfill}%
\pgfsetlinewidth{0.602250pt}%
\definecolor{currentstroke}{rgb}{0.000000,0.000000,0.000000}%
\pgfsetstrokecolor{currentstroke}%
\pgfsetdash{}{0pt}%
\pgfsys@defobject{currentmarker}{\pgfqpoint{0.000000in}{-0.027778in}}{\pgfqpoint{0.000000in}{0.000000in}}{%
\pgfpathmoveto{\pgfqpoint{0.000000in}{0.000000in}}%
\pgfpathlineto{\pgfqpoint{0.000000in}{-0.027778in}}%
\pgfusepath{stroke,fill}%
}%
\begin{pgfscope}%
\pgfsys@transformshift{1.529902in}{0.524170in}%
\pgfsys@useobject{currentmarker}{}%
\end{pgfscope}%
\end{pgfscope}%
\begin{pgfscope}%
\pgfpathrectangle{\pgfqpoint{0.693677in}{0.524170in}}{\pgfqpoint{4.587426in}{2.682436in}}%
\pgfusepath{clip}%
\pgfsetrectcap%
\pgfsetroundjoin%
\pgfsetlinewidth{0.803000pt}%
\definecolor{currentstroke}{rgb}{0.850000,0.850000,0.850000}%
\pgfsetstrokecolor{currentstroke}%
\pgfsetdash{}{0pt}%
\pgfpathmoveto{\pgfqpoint{1.630940in}{0.524170in}}%
\pgfpathlineto{\pgfqpoint{1.630940in}{3.206606in}}%
\pgfusepath{stroke}%
\end{pgfscope}%
\begin{pgfscope}%
\pgfsetbuttcap%
\pgfsetroundjoin%
\definecolor{currentfill}{rgb}{0.000000,0.000000,0.000000}%
\pgfsetfillcolor{currentfill}%
\pgfsetlinewidth{0.602250pt}%
\definecolor{currentstroke}{rgb}{0.000000,0.000000,0.000000}%
\pgfsetstrokecolor{currentstroke}%
\pgfsetdash{}{0pt}%
\pgfsys@defobject{currentmarker}{\pgfqpoint{0.000000in}{-0.027778in}}{\pgfqpoint{0.000000in}{0.000000in}}{%
\pgfpathmoveto{\pgfqpoint{0.000000in}{0.000000in}}%
\pgfpathlineto{\pgfqpoint{0.000000in}{-0.027778in}}%
\pgfusepath{stroke,fill}%
}%
\begin{pgfscope}%
\pgfsys@transformshift{1.630940in}{0.524170in}%
\pgfsys@useobject{currentmarker}{}%
\end{pgfscope}%
\end{pgfscope}%
\begin{pgfscope}%
\pgfpathrectangle{\pgfqpoint{0.693677in}{0.524170in}}{\pgfqpoint{4.587426in}{2.682436in}}%
\pgfusepath{clip}%
\pgfsetrectcap%
\pgfsetroundjoin%
\pgfsetlinewidth{0.803000pt}%
\definecolor{currentstroke}{rgb}{0.850000,0.850000,0.850000}%
\pgfsetstrokecolor{currentstroke}%
\pgfsetdash{}{0pt}%
\pgfpathmoveto{\pgfqpoint{1.713494in}{0.524170in}}%
\pgfpathlineto{\pgfqpoint{1.713494in}{3.206606in}}%
\pgfusepath{stroke}%
\end{pgfscope}%
\begin{pgfscope}%
\pgfsetbuttcap%
\pgfsetroundjoin%
\definecolor{currentfill}{rgb}{0.000000,0.000000,0.000000}%
\pgfsetfillcolor{currentfill}%
\pgfsetlinewidth{0.602250pt}%
\definecolor{currentstroke}{rgb}{0.000000,0.000000,0.000000}%
\pgfsetstrokecolor{currentstroke}%
\pgfsetdash{}{0pt}%
\pgfsys@defobject{currentmarker}{\pgfqpoint{0.000000in}{-0.027778in}}{\pgfqpoint{0.000000in}{0.000000in}}{%
\pgfpathmoveto{\pgfqpoint{0.000000in}{0.000000in}}%
\pgfpathlineto{\pgfqpoint{0.000000in}{-0.027778in}}%
\pgfusepath{stroke,fill}%
}%
\begin{pgfscope}%
\pgfsys@transformshift{1.713494in}{0.524170in}%
\pgfsys@useobject{currentmarker}{}%
\end{pgfscope}%
\end{pgfscope}%
\begin{pgfscope}%
\pgfpathrectangle{\pgfqpoint{0.693677in}{0.524170in}}{\pgfqpoint{4.587426in}{2.682436in}}%
\pgfusepath{clip}%
\pgfsetrectcap%
\pgfsetroundjoin%
\pgfsetlinewidth{0.803000pt}%
\definecolor{currentstroke}{rgb}{0.850000,0.850000,0.850000}%
\pgfsetstrokecolor{currentstroke}%
\pgfsetdash{}{0pt}%
\pgfpathmoveto{\pgfqpoint{1.783293in}{0.524170in}}%
\pgfpathlineto{\pgfqpoint{1.783293in}{3.206606in}}%
\pgfusepath{stroke}%
\end{pgfscope}%
\begin{pgfscope}%
\pgfsetbuttcap%
\pgfsetroundjoin%
\definecolor{currentfill}{rgb}{0.000000,0.000000,0.000000}%
\pgfsetfillcolor{currentfill}%
\pgfsetlinewidth{0.602250pt}%
\definecolor{currentstroke}{rgb}{0.000000,0.000000,0.000000}%
\pgfsetstrokecolor{currentstroke}%
\pgfsetdash{}{0pt}%
\pgfsys@defobject{currentmarker}{\pgfqpoint{0.000000in}{-0.027778in}}{\pgfqpoint{0.000000in}{0.000000in}}{%
\pgfpathmoveto{\pgfqpoint{0.000000in}{0.000000in}}%
\pgfpathlineto{\pgfqpoint{0.000000in}{-0.027778in}}%
\pgfusepath{stroke,fill}%
}%
\begin{pgfscope}%
\pgfsys@transformshift{1.783293in}{0.524170in}%
\pgfsys@useobject{currentmarker}{}%
\end{pgfscope}%
\end{pgfscope}%
\begin{pgfscope}%
\pgfpathrectangle{\pgfqpoint{0.693677in}{0.524170in}}{\pgfqpoint{4.587426in}{2.682436in}}%
\pgfusepath{clip}%
\pgfsetrectcap%
\pgfsetroundjoin%
\pgfsetlinewidth{0.803000pt}%
\definecolor{currentstroke}{rgb}{0.850000,0.850000,0.850000}%
\pgfsetstrokecolor{currentstroke}%
\pgfsetdash{}{0pt}%
\pgfpathmoveto{\pgfqpoint{1.843755in}{0.524170in}}%
\pgfpathlineto{\pgfqpoint{1.843755in}{3.206606in}}%
\pgfusepath{stroke}%
\end{pgfscope}%
\begin{pgfscope}%
\pgfsetbuttcap%
\pgfsetroundjoin%
\definecolor{currentfill}{rgb}{0.000000,0.000000,0.000000}%
\pgfsetfillcolor{currentfill}%
\pgfsetlinewidth{0.602250pt}%
\definecolor{currentstroke}{rgb}{0.000000,0.000000,0.000000}%
\pgfsetstrokecolor{currentstroke}%
\pgfsetdash{}{0pt}%
\pgfsys@defobject{currentmarker}{\pgfqpoint{0.000000in}{-0.027778in}}{\pgfqpoint{0.000000in}{0.000000in}}{%
\pgfpathmoveto{\pgfqpoint{0.000000in}{0.000000in}}%
\pgfpathlineto{\pgfqpoint{0.000000in}{-0.027778in}}%
\pgfusepath{stroke,fill}%
}%
\begin{pgfscope}%
\pgfsys@transformshift{1.843755in}{0.524170in}%
\pgfsys@useobject{currentmarker}{}%
\end{pgfscope}%
\end{pgfscope}%
\begin{pgfscope}%
\pgfpathrectangle{\pgfqpoint{0.693677in}{0.524170in}}{\pgfqpoint{4.587426in}{2.682436in}}%
\pgfusepath{clip}%
\pgfsetrectcap%
\pgfsetroundjoin%
\pgfsetlinewidth{0.803000pt}%
\definecolor{currentstroke}{rgb}{0.850000,0.850000,0.850000}%
\pgfsetstrokecolor{currentstroke}%
\pgfsetdash{}{0pt}%
\pgfpathmoveto{\pgfqpoint{1.897086in}{0.524170in}}%
\pgfpathlineto{\pgfqpoint{1.897086in}{3.206606in}}%
\pgfusepath{stroke}%
\end{pgfscope}%
\begin{pgfscope}%
\pgfsetbuttcap%
\pgfsetroundjoin%
\definecolor{currentfill}{rgb}{0.000000,0.000000,0.000000}%
\pgfsetfillcolor{currentfill}%
\pgfsetlinewidth{0.602250pt}%
\definecolor{currentstroke}{rgb}{0.000000,0.000000,0.000000}%
\pgfsetstrokecolor{currentstroke}%
\pgfsetdash{}{0pt}%
\pgfsys@defobject{currentmarker}{\pgfqpoint{0.000000in}{-0.027778in}}{\pgfqpoint{0.000000in}{0.000000in}}{%
\pgfpathmoveto{\pgfqpoint{0.000000in}{0.000000in}}%
\pgfpathlineto{\pgfqpoint{0.000000in}{-0.027778in}}%
\pgfusepath{stroke,fill}%
}%
\begin{pgfscope}%
\pgfsys@transformshift{1.897086in}{0.524170in}%
\pgfsys@useobject{currentmarker}{}%
\end{pgfscope}%
\end{pgfscope}%
\begin{pgfscope}%
\pgfpathrectangle{\pgfqpoint{0.693677in}{0.524170in}}{\pgfqpoint{4.587426in}{2.682436in}}%
\pgfusepath{clip}%
\pgfsetrectcap%
\pgfsetroundjoin%
\pgfsetlinewidth{0.803000pt}%
\definecolor{currentstroke}{rgb}{0.850000,0.850000,0.850000}%
\pgfsetstrokecolor{currentstroke}%
\pgfsetdash{}{0pt}%
\pgfpathmoveto{\pgfqpoint{2.258646in}{0.524170in}}%
\pgfpathlineto{\pgfqpoint{2.258646in}{3.206606in}}%
\pgfusepath{stroke}%
\end{pgfscope}%
\begin{pgfscope}%
\pgfsetbuttcap%
\pgfsetroundjoin%
\definecolor{currentfill}{rgb}{0.000000,0.000000,0.000000}%
\pgfsetfillcolor{currentfill}%
\pgfsetlinewidth{0.602250pt}%
\definecolor{currentstroke}{rgb}{0.000000,0.000000,0.000000}%
\pgfsetstrokecolor{currentstroke}%
\pgfsetdash{}{0pt}%
\pgfsys@defobject{currentmarker}{\pgfqpoint{0.000000in}{-0.027778in}}{\pgfqpoint{0.000000in}{0.000000in}}{%
\pgfpathmoveto{\pgfqpoint{0.000000in}{0.000000in}}%
\pgfpathlineto{\pgfqpoint{0.000000in}{-0.027778in}}%
\pgfusepath{stroke,fill}%
}%
\begin{pgfscope}%
\pgfsys@transformshift{2.258646in}{0.524170in}%
\pgfsys@useobject{currentmarker}{}%
\end{pgfscope}%
\end{pgfscope}%
\begin{pgfscope}%
\pgfpathrectangle{\pgfqpoint{0.693677in}{0.524170in}}{\pgfqpoint{4.587426in}{2.682436in}}%
\pgfusepath{clip}%
\pgfsetrectcap%
\pgfsetroundjoin%
\pgfsetlinewidth{0.803000pt}%
\definecolor{currentstroke}{rgb}{0.850000,0.850000,0.850000}%
\pgfsetstrokecolor{currentstroke}%
\pgfsetdash{}{0pt}%
\pgfpathmoveto{\pgfqpoint{2.442238in}{0.524170in}}%
\pgfpathlineto{\pgfqpoint{2.442238in}{3.206606in}}%
\pgfusepath{stroke}%
\end{pgfscope}%
\begin{pgfscope}%
\pgfsetbuttcap%
\pgfsetroundjoin%
\definecolor{currentfill}{rgb}{0.000000,0.000000,0.000000}%
\pgfsetfillcolor{currentfill}%
\pgfsetlinewidth{0.602250pt}%
\definecolor{currentstroke}{rgb}{0.000000,0.000000,0.000000}%
\pgfsetstrokecolor{currentstroke}%
\pgfsetdash{}{0pt}%
\pgfsys@defobject{currentmarker}{\pgfqpoint{0.000000in}{-0.027778in}}{\pgfqpoint{0.000000in}{0.000000in}}{%
\pgfpathmoveto{\pgfqpoint{0.000000in}{0.000000in}}%
\pgfpathlineto{\pgfqpoint{0.000000in}{-0.027778in}}%
\pgfusepath{stroke,fill}%
}%
\begin{pgfscope}%
\pgfsys@transformshift{2.442238in}{0.524170in}%
\pgfsys@useobject{currentmarker}{}%
\end{pgfscope}%
\end{pgfscope}%
\begin{pgfscope}%
\pgfpathrectangle{\pgfqpoint{0.693677in}{0.524170in}}{\pgfqpoint{4.587426in}{2.682436in}}%
\pgfusepath{clip}%
\pgfsetrectcap%
\pgfsetroundjoin%
\pgfsetlinewidth{0.803000pt}%
\definecolor{currentstroke}{rgb}{0.850000,0.850000,0.850000}%
\pgfsetstrokecolor{currentstroke}%
\pgfsetdash{}{0pt}%
\pgfpathmoveto{\pgfqpoint{2.572499in}{0.524170in}}%
\pgfpathlineto{\pgfqpoint{2.572499in}{3.206606in}}%
\pgfusepath{stroke}%
\end{pgfscope}%
\begin{pgfscope}%
\pgfsetbuttcap%
\pgfsetroundjoin%
\definecolor{currentfill}{rgb}{0.000000,0.000000,0.000000}%
\pgfsetfillcolor{currentfill}%
\pgfsetlinewidth{0.602250pt}%
\definecolor{currentstroke}{rgb}{0.000000,0.000000,0.000000}%
\pgfsetstrokecolor{currentstroke}%
\pgfsetdash{}{0pt}%
\pgfsys@defobject{currentmarker}{\pgfqpoint{0.000000in}{-0.027778in}}{\pgfqpoint{0.000000in}{0.000000in}}{%
\pgfpathmoveto{\pgfqpoint{0.000000in}{0.000000in}}%
\pgfpathlineto{\pgfqpoint{0.000000in}{-0.027778in}}%
\pgfusepath{stroke,fill}%
}%
\begin{pgfscope}%
\pgfsys@transformshift{2.572499in}{0.524170in}%
\pgfsys@useobject{currentmarker}{}%
\end{pgfscope}%
\end{pgfscope}%
\begin{pgfscope}%
\pgfpathrectangle{\pgfqpoint{0.693677in}{0.524170in}}{\pgfqpoint{4.587426in}{2.682436in}}%
\pgfusepath{clip}%
\pgfsetrectcap%
\pgfsetroundjoin%
\pgfsetlinewidth{0.803000pt}%
\definecolor{currentstroke}{rgb}{0.850000,0.850000,0.850000}%
\pgfsetstrokecolor{currentstroke}%
\pgfsetdash{}{0pt}%
\pgfpathmoveto{\pgfqpoint{2.673537in}{0.524170in}}%
\pgfpathlineto{\pgfqpoint{2.673537in}{3.206606in}}%
\pgfusepath{stroke}%
\end{pgfscope}%
\begin{pgfscope}%
\pgfsetbuttcap%
\pgfsetroundjoin%
\definecolor{currentfill}{rgb}{0.000000,0.000000,0.000000}%
\pgfsetfillcolor{currentfill}%
\pgfsetlinewidth{0.602250pt}%
\definecolor{currentstroke}{rgb}{0.000000,0.000000,0.000000}%
\pgfsetstrokecolor{currentstroke}%
\pgfsetdash{}{0pt}%
\pgfsys@defobject{currentmarker}{\pgfqpoint{0.000000in}{-0.027778in}}{\pgfqpoint{0.000000in}{0.000000in}}{%
\pgfpathmoveto{\pgfqpoint{0.000000in}{0.000000in}}%
\pgfpathlineto{\pgfqpoint{0.000000in}{-0.027778in}}%
\pgfusepath{stroke,fill}%
}%
\begin{pgfscope}%
\pgfsys@transformshift{2.673537in}{0.524170in}%
\pgfsys@useobject{currentmarker}{}%
\end{pgfscope}%
\end{pgfscope}%
\begin{pgfscope}%
\pgfpathrectangle{\pgfqpoint{0.693677in}{0.524170in}}{\pgfqpoint{4.587426in}{2.682436in}}%
\pgfusepath{clip}%
\pgfsetrectcap%
\pgfsetroundjoin%
\pgfsetlinewidth{0.803000pt}%
\definecolor{currentstroke}{rgb}{0.850000,0.850000,0.850000}%
\pgfsetstrokecolor{currentstroke}%
\pgfsetdash{}{0pt}%
\pgfpathmoveto{\pgfqpoint{2.756091in}{0.524170in}}%
\pgfpathlineto{\pgfqpoint{2.756091in}{3.206606in}}%
\pgfusepath{stroke}%
\end{pgfscope}%
\begin{pgfscope}%
\pgfsetbuttcap%
\pgfsetroundjoin%
\definecolor{currentfill}{rgb}{0.000000,0.000000,0.000000}%
\pgfsetfillcolor{currentfill}%
\pgfsetlinewidth{0.602250pt}%
\definecolor{currentstroke}{rgb}{0.000000,0.000000,0.000000}%
\pgfsetstrokecolor{currentstroke}%
\pgfsetdash{}{0pt}%
\pgfsys@defobject{currentmarker}{\pgfqpoint{0.000000in}{-0.027778in}}{\pgfqpoint{0.000000in}{0.000000in}}{%
\pgfpathmoveto{\pgfqpoint{0.000000in}{0.000000in}}%
\pgfpathlineto{\pgfqpoint{0.000000in}{-0.027778in}}%
\pgfusepath{stroke,fill}%
}%
\begin{pgfscope}%
\pgfsys@transformshift{2.756091in}{0.524170in}%
\pgfsys@useobject{currentmarker}{}%
\end{pgfscope}%
\end{pgfscope}%
\begin{pgfscope}%
\pgfpathrectangle{\pgfqpoint{0.693677in}{0.524170in}}{\pgfqpoint{4.587426in}{2.682436in}}%
\pgfusepath{clip}%
\pgfsetrectcap%
\pgfsetroundjoin%
\pgfsetlinewidth{0.803000pt}%
\definecolor{currentstroke}{rgb}{0.850000,0.850000,0.850000}%
\pgfsetstrokecolor{currentstroke}%
\pgfsetdash{}{0pt}%
\pgfpathmoveto{\pgfqpoint{2.825889in}{0.524170in}}%
\pgfpathlineto{\pgfqpoint{2.825889in}{3.206606in}}%
\pgfusepath{stroke}%
\end{pgfscope}%
\begin{pgfscope}%
\pgfsetbuttcap%
\pgfsetroundjoin%
\definecolor{currentfill}{rgb}{0.000000,0.000000,0.000000}%
\pgfsetfillcolor{currentfill}%
\pgfsetlinewidth{0.602250pt}%
\definecolor{currentstroke}{rgb}{0.000000,0.000000,0.000000}%
\pgfsetstrokecolor{currentstroke}%
\pgfsetdash{}{0pt}%
\pgfsys@defobject{currentmarker}{\pgfqpoint{0.000000in}{-0.027778in}}{\pgfqpoint{0.000000in}{0.000000in}}{%
\pgfpathmoveto{\pgfqpoint{0.000000in}{0.000000in}}%
\pgfpathlineto{\pgfqpoint{0.000000in}{-0.027778in}}%
\pgfusepath{stroke,fill}%
}%
\begin{pgfscope}%
\pgfsys@transformshift{2.825889in}{0.524170in}%
\pgfsys@useobject{currentmarker}{}%
\end{pgfscope}%
\end{pgfscope}%
\begin{pgfscope}%
\pgfpathrectangle{\pgfqpoint{0.693677in}{0.524170in}}{\pgfqpoint{4.587426in}{2.682436in}}%
\pgfusepath{clip}%
\pgfsetrectcap%
\pgfsetroundjoin%
\pgfsetlinewidth{0.803000pt}%
\definecolor{currentstroke}{rgb}{0.850000,0.850000,0.850000}%
\pgfsetstrokecolor{currentstroke}%
\pgfsetdash{}{0pt}%
\pgfpathmoveto{\pgfqpoint{2.886352in}{0.524170in}}%
\pgfpathlineto{\pgfqpoint{2.886352in}{3.206606in}}%
\pgfusepath{stroke}%
\end{pgfscope}%
\begin{pgfscope}%
\pgfsetbuttcap%
\pgfsetroundjoin%
\definecolor{currentfill}{rgb}{0.000000,0.000000,0.000000}%
\pgfsetfillcolor{currentfill}%
\pgfsetlinewidth{0.602250pt}%
\definecolor{currentstroke}{rgb}{0.000000,0.000000,0.000000}%
\pgfsetstrokecolor{currentstroke}%
\pgfsetdash{}{0pt}%
\pgfsys@defobject{currentmarker}{\pgfqpoint{0.000000in}{-0.027778in}}{\pgfqpoint{0.000000in}{0.000000in}}{%
\pgfpathmoveto{\pgfqpoint{0.000000in}{0.000000in}}%
\pgfpathlineto{\pgfqpoint{0.000000in}{-0.027778in}}%
\pgfusepath{stroke,fill}%
}%
\begin{pgfscope}%
\pgfsys@transformshift{2.886352in}{0.524170in}%
\pgfsys@useobject{currentmarker}{}%
\end{pgfscope}%
\end{pgfscope}%
\begin{pgfscope}%
\pgfpathrectangle{\pgfqpoint{0.693677in}{0.524170in}}{\pgfqpoint{4.587426in}{2.682436in}}%
\pgfusepath{clip}%
\pgfsetrectcap%
\pgfsetroundjoin%
\pgfsetlinewidth{0.803000pt}%
\definecolor{currentstroke}{rgb}{0.850000,0.850000,0.850000}%
\pgfsetstrokecolor{currentstroke}%
\pgfsetdash{}{0pt}%
\pgfpathmoveto{\pgfqpoint{2.939683in}{0.524170in}}%
\pgfpathlineto{\pgfqpoint{2.939683in}{3.206606in}}%
\pgfusepath{stroke}%
\end{pgfscope}%
\begin{pgfscope}%
\pgfsetbuttcap%
\pgfsetroundjoin%
\definecolor{currentfill}{rgb}{0.000000,0.000000,0.000000}%
\pgfsetfillcolor{currentfill}%
\pgfsetlinewidth{0.602250pt}%
\definecolor{currentstroke}{rgb}{0.000000,0.000000,0.000000}%
\pgfsetstrokecolor{currentstroke}%
\pgfsetdash{}{0pt}%
\pgfsys@defobject{currentmarker}{\pgfqpoint{0.000000in}{-0.027778in}}{\pgfqpoint{0.000000in}{0.000000in}}{%
\pgfpathmoveto{\pgfqpoint{0.000000in}{0.000000in}}%
\pgfpathlineto{\pgfqpoint{0.000000in}{-0.027778in}}%
\pgfusepath{stroke,fill}%
}%
\begin{pgfscope}%
\pgfsys@transformshift{2.939683in}{0.524170in}%
\pgfsys@useobject{currentmarker}{}%
\end{pgfscope}%
\end{pgfscope}%
\begin{pgfscope}%
\pgfpathrectangle{\pgfqpoint{0.693677in}{0.524170in}}{\pgfqpoint{4.587426in}{2.682436in}}%
\pgfusepath{clip}%
\pgfsetrectcap%
\pgfsetroundjoin%
\pgfsetlinewidth{0.803000pt}%
\definecolor{currentstroke}{rgb}{0.850000,0.850000,0.850000}%
\pgfsetstrokecolor{currentstroke}%
\pgfsetdash{}{0pt}%
\pgfpathmoveto{\pgfqpoint{3.301243in}{0.524170in}}%
\pgfpathlineto{\pgfqpoint{3.301243in}{3.206606in}}%
\pgfusepath{stroke}%
\end{pgfscope}%
\begin{pgfscope}%
\pgfsetbuttcap%
\pgfsetroundjoin%
\definecolor{currentfill}{rgb}{0.000000,0.000000,0.000000}%
\pgfsetfillcolor{currentfill}%
\pgfsetlinewidth{0.602250pt}%
\definecolor{currentstroke}{rgb}{0.000000,0.000000,0.000000}%
\pgfsetstrokecolor{currentstroke}%
\pgfsetdash{}{0pt}%
\pgfsys@defobject{currentmarker}{\pgfqpoint{0.000000in}{-0.027778in}}{\pgfqpoint{0.000000in}{0.000000in}}{%
\pgfpathmoveto{\pgfqpoint{0.000000in}{0.000000in}}%
\pgfpathlineto{\pgfqpoint{0.000000in}{-0.027778in}}%
\pgfusepath{stroke,fill}%
}%
\begin{pgfscope}%
\pgfsys@transformshift{3.301243in}{0.524170in}%
\pgfsys@useobject{currentmarker}{}%
\end{pgfscope}%
\end{pgfscope}%
\begin{pgfscope}%
\pgfpathrectangle{\pgfqpoint{0.693677in}{0.524170in}}{\pgfqpoint{4.587426in}{2.682436in}}%
\pgfusepath{clip}%
\pgfsetrectcap%
\pgfsetroundjoin%
\pgfsetlinewidth{0.803000pt}%
\definecolor{currentstroke}{rgb}{0.850000,0.850000,0.850000}%
\pgfsetstrokecolor{currentstroke}%
\pgfsetdash{}{0pt}%
\pgfpathmoveto{\pgfqpoint{3.484835in}{0.524170in}}%
\pgfpathlineto{\pgfqpoint{3.484835in}{3.206606in}}%
\pgfusepath{stroke}%
\end{pgfscope}%
\begin{pgfscope}%
\pgfsetbuttcap%
\pgfsetroundjoin%
\definecolor{currentfill}{rgb}{0.000000,0.000000,0.000000}%
\pgfsetfillcolor{currentfill}%
\pgfsetlinewidth{0.602250pt}%
\definecolor{currentstroke}{rgb}{0.000000,0.000000,0.000000}%
\pgfsetstrokecolor{currentstroke}%
\pgfsetdash{}{0pt}%
\pgfsys@defobject{currentmarker}{\pgfqpoint{0.000000in}{-0.027778in}}{\pgfqpoint{0.000000in}{0.000000in}}{%
\pgfpathmoveto{\pgfqpoint{0.000000in}{0.000000in}}%
\pgfpathlineto{\pgfqpoint{0.000000in}{-0.027778in}}%
\pgfusepath{stroke,fill}%
}%
\begin{pgfscope}%
\pgfsys@transformshift{3.484835in}{0.524170in}%
\pgfsys@useobject{currentmarker}{}%
\end{pgfscope}%
\end{pgfscope}%
\begin{pgfscope}%
\pgfpathrectangle{\pgfqpoint{0.693677in}{0.524170in}}{\pgfqpoint{4.587426in}{2.682436in}}%
\pgfusepath{clip}%
\pgfsetrectcap%
\pgfsetroundjoin%
\pgfsetlinewidth{0.803000pt}%
\definecolor{currentstroke}{rgb}{0.850000,0.850000,0.850000}%
\pgfsetstrokecolor{currentstroke}%
\pgfsetdash{}{0pt}%
\pgfpathmoveto{\pgfqpoint{3.615095in}{0.524170in}}%
\pgfpathlineto{\pgfqpoint{3.615095in}{3.206606in}}%
\pgfusepath{stroke}%
\end{pgfscope}%
\begin{pgfscope}%
\pgfsetbuttcap%
\pgfsetroundjoin%
\definecolor{currentfill}{rgb}{0.000000,0.000000,0.000000}%
\pgfsetfillcolor{currentfill}%
\pgfsetlinewidth{0.602250pt}%
\definecolor{currentstroke}{rgb}{0.000000,0.000000,0.000000}%
\pgfsetstrokecolor{currentstroke}%
\pgfsetdash{}{0pt}%
\pgfsys@defobject{currentmarker}{\pgfqpoint{0.000000in}{-0.027778in}}{\pgfqpoint{0.000000in}{0.000000in}}{%
\pgfpathmoveto{\pgfqpoint{0.000000in}{0.000000in}}%
\pgfpathlineto{\pgfqpoint{0.000000in}{-0.027778in}}%
\pgfusepath{stroke,fill}%
}%
\begin{pgfscope}%
\pgfsys@transformshift{3.615095in}{0.524170in}%
\pgfsys@useobject{currentmarker}{}%
\end{pgfscope}%
\end{pgfscope}%
\begin{pgfscope}%
\pgfpathrectangle{\pgfqpoint{0.693677in}{0.524170in}}{\pgfqpoint{4.587426in}{2.682436in}}%
\pgfusepath{clip}%
\pgfsetrectcap%
\pgfsetroundjoin%
\pgfsetlinewidth{0.803000pt}%
\definecolor{currentstroke}{rgb}{0.850000,0.850000,0.850000}%
\pgfsetstrokecolor{currentstroke}%
\pgfsetdash{}{0pt}%
\pgfpathmoveto{\pgfqpoint{3.716134in}{0.524170in}}%
\pgfpathlineto{\pgfqpoint{3.716134in}{3.206606in}}%
\pgfusepath{stroke}%
\end{pgfscope}%
\begin{pgfscope}%
\pgfsetbuttcap%
\pgfsetroundjoin%
\definecolor{currentfill}{rgb}{0.000000,0.000000,0.000000}%
\pgfsetfillcolor{currentfill}%
\pgfsetlinewidth{0.602250pt}%
\definecolor{currentstroke}{rgb}{0.000000,0.000000,0.000000}%
\pgfsetstrokecolor{currentstroke}%
\pgfsetdash{}{0pt}%
\pgfsys@defobject{currentmarker}{\pgfqpoint{0.000000in}{-0.027778in}}{\pgfqpoint{0.000000in}{0.000000in}}{%
\pgfpathmoveto{\pgfqpoint{0.000000in}{0.000000in}}%
\pgfpathlineto{\pgfqpoint{0.000000in}{-0.027778in}}%
\pgfusepath{stroke,fill}%
}%
\begin{pgfscope}%
\pgfsys@transformshift{3.716134in}{0.524170in}%
\pgfsys@useobject{currentmarker}{}%
\end{pgfscope}%
\end{pgfscope}%
\begin{pgfscope}%
\pgfpathrectangle{\pgfqpoint{0.693677in}{0.524170in}}{\pgfqpoint{4.587426in}{2.682436in}}%
\pgfusepath{clip}%
\pgfsetrectcap%
\pgfsetroundjoin%
\pgfsetlinewidth{0.803000pt}%
\definecolor{currentstroke}{rgb}{0.850000,0.850000,0.850000}%
\pgfsetstrokecolor{currentstroke}%
\pgfsetdash{}{0pt}%
\pgfpathmoveto{\pgfqpoint{3.798688in}{0.524170in}}%
\pgfpathlineto{\pgfqpoint{3.798688in}{3.206606in}}%
\pgfusepath{stroke}%
\end{pgfscope}%
\begin{pgfscope}%
\pgfsetbuttcap%
\pgfsetroundjoin%
\definecolor{currentfill}{rgb}{0.000000,0.000000,0.000000}%
\pgfsetfillcolor{currentfill}%
\pgfsetlinewidth{0.602250pt}%
\definecolor{currentstroke}{rgb}{0.000000,0.000000,0.000000}%
\pgfsetstrokecolor{currentstroke}%
\pgfsetdash{}{0pt}%
\pgfsys@defobject{currentmarker}{\pgfqpoint{0.000000in}{-0.027778in}}{\pgfqpoint{0.000000in}{0.000000in}}{%
\pgfpathmoveto{\pgfqpoint{0.000000in}{0.000000in}}%
\pgfpathlineto{\pgfqpoint{0.000000in}{-0.027778in}}%
\pgfusepath{stroke,fill}%
}%
\begin{pgfscope}%
\pgfsys@transformshift{3.798688in}{0.524170in}%
\pgfsys@useobject{currentmarker}{}%
\end{pgfscope}%
\end{pgfscope}%
\begin{pgfscope}%
\pgfpathrectangle{\pgfqpoint{0.693677in}{0.524170in}}{\pgfqpoint{4.587426in}{2.682436in}}%
\pgfusepath{clip}%
\pgfsetrectcap%
\pgfsetroundjoin%
\pgfsetlinewidth{0.803000pt}%
\definecolor{currentstroke}{rgb}{0.850000,0.850000,0.850000}%
\pgfsetstrokecolor{currentstroke}%
\pgfsetdash{}{0pt}%
\pgfpathmoveto{\pgfqpoint{3.868486in}{0.524170in}}%
\pgfpathlineto{\pgfqpoint{3.868486in}{3.206606in}}%
\pgfusepath{stroke}%
\end{pgfscope}%
\begin{pgfscope}%
\pgfsetbuttcap%
\pgfsetroundjoin%
\definecolor{currentfill}{rgb}{0.000000,0.000000,0.000000}%
\pgfsetfillcolor{currentfill}%
\pgfsetlinewidth{0.602250pt}%
\definecolor{currentstroke}{rgb}{0.000000,0.000000,0.000000}%
\pgfsetstrokecolor{currentstroke}%
\pgfsetdash{}{0pt}%
\pgfsys@defobject{currentmarker}{\pgfqpoint{0.000000in}{-0.027778in}}{\pgfqpoint{0.000000in}{0.000000in}}{%
\pgfpathmoveto{\pgfqpoint{0.000000in}{0.000000in}}%
\pgfpathlineto{\pgfqpoint{0.000000in}{-0.027778in}}%
\pgfusepath{stroke,fill}%
}%
\begin{pgfscope}%
\pgfsys@transformshift{3.868486in}{0.524170in}%
\pgfsys@useobject{currentmarker}{}%
\end{pgfscope}%
\end{pgfscope}%
\begin{pgfscope}%
\pgfpathrectangle{\pgfqpoint{0.693677in}{0.524170in}}{\pgfqpoint{4.587426in}{2.682436in}}%
\pgfusepath{clip}%
\pgfsetrectcap%
\pgfsetroundjoin%
\pgfsetlinewidth{0.803000pt}%
\definecolor{currentstroke}{rgb}{0.850000,0.850000,0.850000}%
\pgfsetstrokecolor{currentstroke}%
\pgfsetdash{}{0pt}%
\pgfpathmoveto{\pgfqpoint{3.928948in}{0.524170in}}%
\pgfpathlineto{\pgfqpoint{3.928948in}{3.206606in}}%
\pgfusepath{stroke}%
\end{pgfscope}%
\begin{pgfscope}%
\pgfsetbuttcap%
\pgfsetroundjoin%
\definecolor{currentfill}{rgb}{0.000000,0.000000,0.000000}%
\pgfsetfillcolor{currentfill}%
\pgfsetlinewidth{0.602250pt}%
\definecolor{currentstroke}{rgb}{0.000000,0.000000,0.000000}%
\pgfsetstrokecolor{currentstroke}%
\pgfsetdash{}{0pt}%
\pgfsys@defobject{currentmarker}{\pgfqpoint{0.000000in}{-0.027778in}}{\pgfqpoint{0.000000in}{0.000000in}}{%
\pgfpathmoveto{\pgfqpoint{0.000000in}{0.000000in}}%
\pgfpathlineto{\pgfqpoint{0.000000in}{-0.027778in}}%
\pgfusepath{stroke,fill}%
}%
\begin{pgfscope}%
\pgfsys@transformshift{3.928948in}{0.524170in}%
\pgfsys@useobject{currentmarker}{}%
\end{pgfscope}%
\end{pgfscope}%
\begin{pgfscope}%
\pgfpathrectangle{\pgfqpoint{0.693677in}{0.524170in}}{\pgfqpoint{4.587426in}{2.682436in}}%
\pgfusepath{clip}%
\pgfsetrectcap%
\pgfsetroundjoin%
\pgfsetlinewidth{0.803000pt}%
\definecolor{currentstroke}{rgb}{0.850000,0.850000,0.850000}%
\pgfsetstrokecolor{currentstroke}%
\pgfsetdash{}{0pt}%
\pgfpathmoveto{\pgfqpoint{3.982280in}{0.524170in}}%
\pgfpathlineto{\pgfqpoint{3.982280in}{3.206606in}}%
\pgfusepath{stroke}%
\end{pgfscope}%
\begin{pgfscope}%
\pgfsetbuttcap%
\pgfsetroundjoin%
\definecolor{currentfill}{rgb}{0.000000,0.000000,0.000000}%
\pgfsetfillcolor{currentfill}%
\pgfsetlinewidth{0.602250pt}%
\definecolor{currentstroke}{rgb}{0.000000,0.000000,0.000000}%
\pgfsetstrokecolor{currentstroke}%
\pgfsetdash{}{0pt}%
\pgfsys@defobject{currentmarker}{\pgfqpoint{0.000000in}{-0.027778in}}{\pgfqpoint{0.000000in}{0.000000in}}{%
\pgfpathmoveto{\pgfqpoint{0.000000in}{0.000000in}}%
\pgfpathlineto{\pgfqpoint{0.000000in}{-0.027778in}}%
\pgfusepath{stroke,fill}%
}%
\begin{pgfscope}%
\pgfsys@transformshift{3.982280in}{0.524170in}%
\pgfsys@useobject{currentmarker}{}%
\end{pgfscope}%
\end{pgfscope}%
\begin{pgfscope}%
\pgfpathrectangle{\pgfqpoint{0.693677in}{0.524170in}}{\pgfqpoint{4.587426in}{2.682436in}}%
\pgfusepath{clip}%
\pgfsetrectcap%
\pgfsetroundjoin%
\pgfsetlinewidth{0.803000pt}%
\definecolor{currentstroke}{rgb}{0.850000,0.850000,0.850000}%
\pgfsetstrokecolor{currentstroke}%
\pgfsetdash{}{0pt}%
\pgfpathmoveto{\pgfqpoint{4.343839in}{0.524170in}}%
\pgfpathlineto{\pgfqpoint{4.343839in}{3.206606in}}%
\pgfusepath{stroke}%
\end{pgfscope}%
\begin{pgfscope}%
\pgfsetbuttcap%
\pgfsetroundjoin%
\definecolor{currentfill}{rgb}{0.000000,0.000000,0.000000}%
\pgfsetfillcolor{currentfill}%
\pgfsetlinewidth{0.602250pt}%
\definecolor{currentstroke}{rgb}{0.000000,0.000000,0.000000}%
\pgfsetstrokecolor{currentstroke}%
\pgfsetdash{}{0pt}%
\pgfsys@defobject{currentmarker}{\pgfqpoint{0.000000in}{-0.027778in}}{\pgfqpoint{0.000000in}{0.000000in}}{%
\pgfpathmoveto{\pgfqpoint{0.000000in}{0.000000in}}%
\pgfpathlineto{\pgfqpoint{0.000000in}{-0.027778in}}%
\pgfusepath{stroke,fill}%
}%
\begin{pgfscope}%
\pgfsys@transformshift{4.343839in}{0.524170in}%
\pgfsys@useobject{currentmarker}{}%
\end{pgfscope}%
\end{pgfscope}%
\begin{pgfscope}%
\pgfpathrectangle{\pgfqpoint{0.693677in}{0.524170in}}{\pgfqpoint{4.587426in}{2.682436in}}%
\pgfusepath{clip}%
\pgfsetrectcap%
\pgfsetroundjoin%
\pgfsetlinewidth{0.803000pt}%
\definecolor{currentstroke}{rgb}{0.850000,0.850000,0.850000}%
\pgfsetstrokecolor{currentstroke}%
\pgfsetdash{}{0pt}%
\pgfpathmoveto{\pgfqpoint{4.527431in}{0.524170in}}%
\pgfpathlineto{\pgfqpoint{4.527431in}{3.206606in}}%
\pgfusepath{stroke}%
\end{pgfscope}%
\begin{pgfscope}%
\pgfsetbuttcap%
\pgfsetroundjoin%
\definecolor{currentfill}{rgb}{0.000000,0.000000,0.000000}%
\pgfsetfillcolor{currentfill}%
\pgfsetlinewidth{0.602250pt}%
\definecolor{currentstroke}{rgb}{0.000000,0.000000,0.000000}%
\pgfsetstrokecolor{currentstroke}%
\pgfsetdash{}{0pt}%
\pgfsys@defobject{currentmarker}{\pgfqpoint{0.000000in}{-0.027778in}}{\pgfqpoint{0.000000in}{0.000000in}}{%
\pgfpathmoveto{\pgfqpoint{0.000000in}{0.000000in}}%
\pgfpathlineto{\pgfqpoint{0.000000in}{-0.027778in}}%
\pgfusepath{stroke,fill}%
}%
\begin{pgfscope}%
\pgfsys@transformshift{4.527431in}{0.524170in}%
\pgfsys@useobject{currentmarker}{}%
\end{pgfscope}%
\end{pgfscope}%
\begin{pgfscope}%
\pgfpathrectangle{\pgfqpoint{0.693677in}{0.524170in}}{\pgfqpoint{4.587426in}{2.682436in}}%
\pgfusepath{clip}%
\pgfsetrectcap%
\pgfsetroundjoin%
\pgfsetlinewidth{0.803000pt}%
\definecolor{currentstroke}{rgb}{0.850000,0.850000,0.850000}%
\pgfsetstrokecolor{currentstroke}%
\pgfsetdash{}{0pt}%
\pgfpathmoveto{\pgfqpoint{4.657692in}{0.524170in}}%
\pgfpathlineto{\pgfqpoint{4.657692in}{3.206606in}}%
\pgfusepath{stroke}%
\end{pgfscope}%
\begin{pgfscope}%
\pgfsetbuttcap%
\pgfsetroundjoin%
\definecolor{currentfill}{rgb}{0.000000,0.000000,0.000000}%
\pgfsetfillcolor{currentfill}%
\pgfsetlinewidth{0.602250pt}%
\definecolor{currentstroke}{rgb}{0.000000,0.000000,0.000000}%
\pgfsetstrokecolor{currentstroke}%
\pgfsetdash{}{0pt}%
\pgfsys@defobject{currentmarker}{\pgfqpoint{0.000000in}{-0.027778in}}{\pgfqpoint{0.000000in}{0.000000in}}{%
\pgfpathmoveto{\pgfqpoint{0.000000in}{0.000000in}}%
\pgfpathlineto{\pgfqpoint{0.000000in}{-0.027778in}}%
\pgfusepath{stroke,fill}%
}%
\begin{pgfscope}%
\pgfsys@transformshift{4.657692in}{0.524170in}%
\pgfsys@useobject{currentmarker}{}%
\end{pgfscope}%
\end{pgfscope}%
\begin{pgfscope}%
\pgfpathrectangle{\pgfqpoint{0.693677in}{0.524170in}}{\pgfqpoint{4.587426in}{2.682436in}}%
\pgfusepath{clip}%
\pgfsetrectcap%
\pgfsetroundjoin%
\pgfsetlinewidth{0.803000pt}%
\definecolor{currentstroke}{rgb}{0.850000,0.850000,0.850000}%
\pgfsetstrokecolor{currentstroke}%
\pgfsetdash{}{0pt}%
\pgfpathmoveto{\pgfqpoint{4.758730in}{0.524170in}}%
\pgfpathlineto{\pgfqpoint{4.758730in}{3.206606in}}%
\pgfusepath{stroke}%
\end{pgfscope}%
\begin{pgfscope}%
\pgfsetbuttcap%
\pgfsetroundjoin%
\definecolor{currentfill}{rgb}{0.000000,0.000000,0.000000}%
\pgfsetfillcolor{currentfill}%
\pgfsetlinewidth{0.602250pt}%
\definecolor{currentstroke}{rgb}{0.000000,0.000000,0.000000}%
\pgfsetstrokecolor{currentstroke}%
\pgfsetdash{}{0pt}%
\pgfsys@defobject{currentmarker}{\pgfqpoint{0.000000in}{-0.027778in}}{\pgfqpoint{0.000000in}{0.000000in}}{%
\pgfpathmoveto{\pgfqpoint{0.000000in}{0.000000in}}%
\pgfpathlineto{\pgfqpoint{0.000000in}{-0.027778in}}%
\pgfusepath{stroke,fill}%
}%
\begin{pgfscope}%
\pgfsys@transformshift{4.758730in}{0.524170in}%
\pgfsys@useobject{currentmarker}{}%
\end{pgfscope}%
\end{pgfscope}%
\begin{pgfscope}%
\pgfpathrectangle{\pgfqpoint{0.693677in}{0.524170in}}{\pgfqpoint{4.587426in}{2.682436in}}%
\pgfusepath{clip}%
\pgfsetrectcap%
\pgfsetroundjoin%
\pgfsetlinewidth{0.803000pt}%
\definecolor{currentstroke}{rgb}{0.850000,0.850000,0.850000}%
\pgfsetstrokecolor{currentstroke}%
\pgfsetdash{}{0pt}%
\pgfpathmoveto{\pgfqpoint{4.841284in}{0.524170in}}%
\pgfpathlineto{\pgfqpoint{4.841284in}{3.206606in}}%
\pgfusepath{stroke}%
\end{pgfscope}%
\begin{pgfscope}%
\pgfsetbuttcap%
\pgfsetroundjoin%
\definecolor{currentfill}{rgb}{0.000000,0.000000,0.000000}%
\pgfsetfillcolor{currentfill}%
\pgfsetlinewidth{0.602250pt}%
\definecolor{currentstroke}{rgb}{0.000000,0.000000,0.000000}%
\pgfsetstrokecolor{currentstroke}%
\pgfsetdash{}{0pt}%
\pgfsys@defobject{currentmarker}{\pgfqpoint{0.000000in}{-0.027778in}}{\pgfqpoint{0.000000in}{0.000000in}}{%
\pgfpathmoveto{\pgfqpoint{0.000000in}{0.000000in}}%
\pgfpathlineto{\pgfqpoint{0.000000in}{-0.027778in}}%
\pgfusepath{stroke,fill}%
}%
\begin{pgfscope}%
\pgfsys@transformshift{4.841284in}{0.524170in}%
\pgfsys@useobject{currentmarker}{}%
\end{pgfscope}%
\end{pgfscope}%
\begin{pgfscope}%
\pgfpathrectangle{\pgfqpoint{0.693677in}{0.524170in}}{\pgfqpoint{4.587426in}{2.682436in}}%
\pgfusepath{clip}%
\pgfsetrectcap%
\pgfsetroundjoin%
\pgfsetlinewidth{0.803000pt}%
\definecolor{currentstroke}{rgb}{0.850000,0.850000,0.850000}%
\pgfsetstrokecolor{currentstroke}%
\pgfsetdash{}{0pt}%
\pgfpathmoveto{\pgfqpoint{4.911083in}{0.524170in}}%
\pgfpathlineto{\pgfqpoint{4.911083in}{3.206606in}}%
\pgfusepath{stroke}%
\end{pgfscope}%
\begin{pgfscope}%
\pgfsetbuttcap%
\pgfsetroundjoin%
\definecolor{currentfill}{rgb}{0.000000,0.000000,0.000000}%
\pgfsetfillcolor{currentfill}%
\pgfsetlinewidth{0.602250pt}%
\definecolor{currentstroke}{rgb}{0.000000,0.000000,0.000000}%
\pgfsetstrokecolor{currentstroke}%
\pgfsetdash{}{0pt}%
\pgfsys@defobject{currentmarker}{\pgfqpoint{0.000000in}{-0.027778in}}{\pgfqpoint{0.000000in}{0.000000in}}{%
\pgfpathmoveto{\pgfqpoint{0.000000in}{0.000000in}}%
\pgfpathlineto{\pgfqpoint{0.000000in}{-0.027778in}}%
\pgfusepath{stroke,fill}%
}%
\begin{pgfscope}%
\pgfsys@transformshift{4.911083in}{0.524170in}%
\pgfsys@useobject{currentmarker}{}%
\end{pgfscope}%
\end{pgfscope}%
\begin{pgfscope}%
\pgfpathrectangle{\pgfqpoint{0.693677in}{0.524170in}}{\pgfqpoint{4.587426in}{2.682436in}}%
\pgfusepath{clip}%
\pgfsetrectcap%
\pgfsetroundjoin%
\pgfsetlinewidth{0.803000pt}%
\definecolor{currentstroke}{rgb}{0.850000,0.850000,0.850000}%
\pgfsetstrokecolor{currentstroke}%
\pgfsetdash{}{0pt}%
\pgfpathmoveto{\pgfqpoint{4.971545in}{0.524170in}}%
\pgfpathlineto{\pgfqpoint{4.971545in}{3.206606in}}%
\pgfusepath{stroke}%
\end{pgfscope}%
\begin{pgfscope}%
\pgfsetbuttcap%
\pgfsetroundjoin%
\definecolor{currentfill}{rgb}{0.000000,0.000000,0.000000}%
\pgfsetfillcolor{currentfill}%
\pgfsetlinewidth{0.602250pt}%
\definecolor{currentstroke}{rgb}{0.000000,0.000000,0.000000}%
\pgfsetstrokecolor{currentstroke}%
\pgfsetdash{}{0pt}%
\pgfsys@defobject{currentmarker}{\pgfqpoint{0.000000in}{-0.027778in}}{\pgfqpoint{0.000000in}{0.000000in}}{%
\pgfpathmoveto{\pgfqpoint{0.000000in}{0.000000in}}%
\pgfpathlineto{\pgfqpoint{0.000000in}{-0.027778in}}%
\pgfusepath{stroke,fill}%
}%
\begin{pgfscope}%
\pgfsys@transformshift{4.971545in}{0.524170in}%
\pgfsys@useobject{currentmarker}{}%
\end{pgfscope}%
\end{pgfscope}%
\begin{pgfscope}%
\pgfpathrectangle{\pgfqpoint{0.693677in}{0.524170in}}{\pgfqpoint{4.587426in}{2.682436in}}%
\pgfusepath{clip}%
\pgfsetrectcap%
\pgfsetroundjoin%
\pgfsetlinewidth{0.803000pt}%
\definecolor{currentstroke}{rgb}{0.850000,0.850000,0.850000}%
\pgfsetstrokecolor{currentstroke}%
\pgfsetdash{}{0pt}%
\pgfpathmoveto{\pgfqpoint{5.024877in}{0.524170in}}%
\pgfpathlineto{\pgfqpoint{5.024877in}{3.206606in}}%
\pgfusepath{stroke}%
\end{pgfscope}%
\begin{pgfscope}%
\pgfsetbuttcap%
\pgfsetroundjoin%
\definecolor{currentfill}{rgb}{0.000000,0.000000,0.000000}%
\pgfsetfillcolor{currentfill}%
\pgfsetlinewidth{0.602250pt}%
\definecolor{currentstroke}{rgb}{0.000000,0.000000,0.000000}%
\pgfsetstrokecolor{currentstroke}%
\pgfsetdash{}{0pt}%
\pgfsys@defobject{currentmarker}{\pgfqpoint{0.000000in}{-0.027778in}}{\pgfqpoint{0.000000in}{0.000000in}}{%
\pgfpathmoveto{\pgfqpoint{0.000000in}{0.000000in}}%
\pgfpathlineto{\pgfqpoint{0.000000in}{-0.027778in}}%
\pgfusepath{stroke,fill}%
}%
\begin{pgfscope}%
\pgfsys@transformshift{5.024877in}{0.524170in}%
\pgfsys@useobject{currentmarker}{}%
\end{pgfscope}%
\end{pgfscope}%
\begin{pgfscope}%
\definecolor{textcolor}{rgb}{0.000000,0.000000,0.000000}%
\pgfsetstrokecolor{textcolor}%
\pgfsetfillcolor{textcolor}%
\pgftext[x=2.987390in,y=0.271531in,,top]{\color{textcolor}{\rmfamily\fontsize{10.000000}{12.000000}\selectfont\catcode`\^=\active\def^{\ifmmode\sp\else\^{}\fi}\catcode`\%=\active\def%{\%}Frequency in \unit{\Hz}}}%
\end{pgfscope}%
\begin{pgfscope}%
\pgfpathrectangle{\pgfqpoint{0.693677in}{0.524170in}}{\pgfqpoint{4.587426in}{2.682436in}}%
\pgfusepath{clip}%
\pgfsetrectcap%
\pgfsetroundjoin%
\pgfsetlinewidth{0.803000pt}%
\definecolor{currentstroke}{rgb}{0.450000,0.450000,0.450000}%
\pgfsetstrokecolor{currentstroke}%
\pgfsetdash{}{0pt}%
\pgfpathmoveto{\pgfqpoint{0.693677in}{0.777834in}}%
\pgfpathlineto{\pgfqpoint{5.281103in}{0.777834in}}%
\pgfusepath{stroke}%
\end{pgfscope}%
\begin{pgfscope}%
\pgfsetbuttcap%
\pgfsetroundjoin%
\definecolor{currentfill}{rgb}{0.000000,0.000000,0.000000}%
\pgfsetfillcolor{currentfill}%
\pgfsetlinewidth{0.803000pt}%
\definecolor{currentstroke}{rgb}{0.000000,0.000000,0.000000}%
\pgfsetstrokecolor{currentstroke}%
\pgfsetdash{}{0pt}%
\pgfsys@defobject{currentmarker}{\pgfqpoint{-0.048611in}{0.000000in}}{\pgfqpoint{-0.000000in}{0.000000in}}{%
\pgfpathmoveto{\pgfqpoint{-0.000000in}{0.000000in}}%
\pgfpathlineto{\pgfqpoint{-0.048611in}{0.000000in}}%
\pgfusepath{stroke,fill}%
}%
\begin{pgfscope}%
\pgfsys@transformshift{0.693677in}{0.777834in}%
\pgfsys@useobject{currentmarker}{}%
\end{pgfscope}%
\end{pgfscope}%
\begin{pgfscope}%
\definecolor{textcolor}{rgb}{0.000000,0.000000,0.000000}%
\pgfsetstrokecolor{textcolor}%
\pgfsetfillcolor{textcolor}%
\pgftext[x=0.327546in, y=0.739278in, left, base]{\color{textcolor}{\rmfamily\fontsize{8.000000}{9.600000}\selectfont\catcode`\^=\active\def^{\ifmmode\sp\else\^{}\fi}\catcode`\%=\active\def%{\%}$\mathdefault{\ensuremath{-}100}$}}%
\end{pgfscope}%
\begin{pgfscope}%
\pgfpathrectangle{\pgfqpoint{0.693677in}{0.524170in}}{\pgfqpoint{4.587426in}{2.682436in}}%
\pgfusepath{clip}%
\pgfsetrectcap%
\pgfsetroundjoin%
\pgfsetlinewidth{0.803000pt}%
\definecolor{currentstroke}{rgb}{0.450000,0.450000,0.450000}%
\pgfsetstrokecolor{currentstroke}%
\pgfsetdash{}{0pt}%
\pgfpathmoveto{\pgfqpoint{0.693677in}{1.235237in}}%
\pgfpathlineto{\pgfqpoint{5.281103in}{1.235237in}}%
\pgfusepath{stroke}%
\end{pgfscope}%
\begin{pgfscope}%
\pgfsetbuttcap%
\pgfsetroundjoin%
\definecolor{currentfill}{rgb}{0.000000,0.000000,0.000000}%
\pgfsetfillcolor{currentfill}%
\pgfsetlinewidth{0.803000pt}%
\definecolor{currentstroke}{rgb}{0.000000,0.000000,0.000000}%
\pgfsetstrokecolor{currentstroke}%
\pgfsetdash{}{0pt}%
\pgfsys@defobject{currentmarker}{\pgfqpoint{-0.048611in}{0.000000in}}{\pgfqpoint{-0.000000in}{0.000000in}}{%
\pgfpathmoveto{\pgfqpoint{-0.000000in}{0.000000in}}%
\pgfpathlineto{\pgfqpoint{-0.048611in}{0.000000in}}%
\pgfusepath{stroke,fill}%
}%
\begin{pgfscope}%
\pgfsys@transformshift{0.693677in}{1.235237in}%
\pgfsys@useobject{currentmarker}{}%
\end{pgfscope}%
\end{pgfscope}%
\begin{pgfscope}%
\definecolor{textcolor}{rgb}{0.000000,0.000000,0.000000}%
\pgfsetstrokecolor{textcolor}%
\pgfsetfillcolor{textcolor}%
\pgftext[x=0.386575in, y=1.196681in, left, base]{\color{textcolor}{\rmfamily\fontsize{8.000000}{9.600000}\selectfont\catcode`\^=\active\def^{\ifmmode\sp\else\^{}\fi}\catcode`\%=\active\def%{\%}$\mathdefault{\ensuremath{-}80}$}}%
\end{pgfscope}%
\begin{pgfscope}%
\pgfpathrectangle{\pgfqpoint{0.693677in}{0.524170in}}{\pgfqpoint{4.587426in}{2.682436in}}%
\pgfusepath{clip}%
\pgfsetrectcap%
\pgfsetroundjoin%
\pgfsetlinewidth{0.803000pt}%
\definecolor{currentstroke}{rgb}{0.450000,0.450000,0.450000}%
\pgfsetstrokecolor{currentstroke}%
\pgfsetdash{}{0pt}%
\pgfpathmoveto{\pgfqpoint{0.693677in}{1.692640in}}%
\pgfpathlineto{\pgfqpoint{5.281103in}{1.692640in}}%
\pgfusepath{stroke}%
\end{pgfscope}%
\begin{pgfscope}%
\pgfsetbuttcap%
\pgfsetroundjoin%
\definecolor{currentfill}{rgb}{0.000000,0.000000,0.000000}%
\pgfsetfillcolor{currentfill}%
\pgfsetlinewidth{0.803000pt}%
\definecolor{currentstroke}{rgb}{0.000000,0.000000,0.000000}%
\pgfsetstrokecolor{currentstroke}%
\pgfsetdash{}{0pt}%
\pgfsys@defobject{currentmarker}{\pgfqpoint{-0.048611in}{0.000000in}}{\pgfqpoint{-0.000000in}{0.000000in}}{%
\pgfpathmoveto{\pgfqpoint{-0.000000in}{0.000000in}}%
\pgfpathlineto{\pgfqpoint{-0.048611in}{0.000000in}}%
\pgfusepath{stroke,fill}%
}%
\begin{pgfscope}%
\pgfsys@transformshift{0.693677in}{1.692640in}%
\pgfsys@useobject{currentmarker}{}%
\end{pgfscope}%
\end{pgfscope}%
\begin{pgfscope}%
\definecolor{textcolor}{rgb}{0.000000,0.000000,0.000000}%
\pgfsetstrokecolor{textcolor}%
\pgfsetfillcolor{textcolor}%
\pgftext[x=0.386575in, y=1.654084in, left, base]{\color{textcolor}{\rmfamily\fontsize{8.000000}{9.600000}\selectfont\catcode`\^=\active\def^{\ifmmode\sp\else\^{}\fi}\catcode`\%=\active\def%{\%}$\mathdefault{\ensuremath{-}60}$}}%
\end{pgfscope}%
\begin{pgfscope}%
\pgfpathrectangle{\pgfqpoint{0.693677in}{0.524170in}}{\pgfqpoint{4.587426in}{2.682436in}}%
\pgfusepath{clip}%
\pgfsetrectcap%
\pgfsetroundjoin%
\pgfsetlinewidth{0.803000pt}%
\definecolor{currentstroke}{rgb}{0.450000,0.450000,0.450000}%
\pgfsetstrokecolor{currentstroke}%
\pgfsetdash{}{0pt}%
\pgfpathmoveto{\pgfqpoint{0.693677in}{2.150043in}}%
\pgfpathlineto{\pgfqpoint{5.281103in}{2.150043in}}%
\pgfusepath{stroke}%
\end{pgfscope}%
\begin{pgfscope}%
\pgfsetbuttcap%
\pgfsetroundjoin%
\definecolor{currentfill}{rgb}{0.000000,0.000000,0.000000}%
\pgfsetfillcolor{currentfill}%
\pgfsetlinewidth{0.803000pt}%
\definecolor{currentstroke}{rgb}{0.000000,0.000000,0.000000}%
\pgfsetstrokecolor{currentstroke}%
\pgfsetdash{}{0pt}%
\pgfsys@defobject{currentmarker}{\pgfqpoint{-0.048611in}{0.000000in}}{\pgfqpoint{-0.000000in}{0.000000in}}{%
\pgfpathmoveto{\pgfqpoint{-0.000000in}{0.000000in}}%
\pgfpathlineto{\pgfqpoint{-0.048611in}{0.000000in}}%
\pgfusepath{stroke,fill}%
}%
\begin{pgfscope}%
\pgfsys@transformshift{0.693677in}{2.150043in}%
\pgfsys@useobject{currentmarker}{}%
\end{pgfscope}%
\end{pgfscope}%
\begin{pgfscope}%
\definecolor{textcolor}{rgb}{0.000000,0.000000,0.000000}%
\pgfsetstrokecolor{textcolor}%
\pgfsetfillcolor{textcolor}%
\pgftext[x=0.386575in, y=2.111487in, left, base]{\color{textcolor}{\rmfamily\fontsize{8.000000}{9.600000}\selectfont\catcode`\^=\active\def^{\ifmmode\sp\else\^{}\fi}\catcode`\%=\active\def%{\%}$\mathdefault{\ensuremath{-}40}$}}%
\end{pgfscope}%
\begin{pgfscope}%
\pgfpathrectangle{\pgfqpoint{0.693677in}{0.524170in}}{\pgfqpoint{4.587426in}{2.682436in}}%
\pgfusepath{clip}%
\pgfsetrectcap%
\pgfsetroundjoin%
\pgfsetlinewidth{0.803000pt}%
\definecolor{currentstroke}{rgb}{0.450000,0.450000,0.450000}%
\pgfsetstrokecolor{currentstroke}%
\pgfsetdash{}{0pt}%
\pgfpathmoveto{\pgfqpoint{0.693677in}{2.607446in}}%
\pgfpathlineto{\pgfqpoint{5.281103in}{2.607446in}}%
\pgfusepath{stroke}%
\end{pgfscope}%
\begin{pgfscope}%
\pgfsetbuttcap%
\pgfsetroundjoin%
\definecolor{currentfill}{rgb}{0.000000,0.000000,0.000000}%
\pgfsetfillcolor{currentfill}%
\pgfsetlinewidth{0.803000pt}%
\definecolor{currentstroke}{rgb}{0.000000,0.000000,0.000000}%
\pgfsetstrokecolor{currentstroke}%
\pgfsetdash{}{0pt}%
\pgfsys@defobject{currentmarker}{\pgfqpoint{-0.048611in}{0.000000in}}{\pgfqpoint{-0.000000in}{0.000000in}}{%
\pgfpathmoveto{\pgfqpoint{-0.000000in}{0.000000in}}%
\pgfpathlineto{\pgfqpoint{-0.048611in}{0.000000in}}%
\pgfusepath{stroke,fill}%
}%
\begin{pgfscope}%
\pgfsys@transformshift{0.693677in}{2.607446in}%
\pgfsys@useobject{currentmarker}{}%
\end{pgfscope}%
\end{pgfscope}%
\begin{pgfscope}%
\definecolor{textcolor}{rgb}{0.000000,0.000000,0.000000}%
\pgfsetstrokecolor{textcolor}%
\pgfsetfillcolor{textcolor}%
\pgftext[x=0.386575in, y=2.568891in, left, base]{\color{textcolor}{\rmfamily\fontsize{8.000000}{9.600000}\selectfont\catcode`\^=\active\def^{\ifmmode\sp\else\^{}\fi}\catcode`\%=\active\def%{\%}$\mathdefault{\ensuremath{-}20}$}}%
\end{pgfscope}%
\begin{pgfscope}%
\pgfpathrectangle{\pgfqpoint{0.693677in}{0.524170in}}{\pgfqpoint{4.587426in}{2.682436in}}%
\pgfusepath{clip}%
\pgfsetrectcap%
\pgfsetroundjoin%
\pgfsetlinewidth{0.803000pt}%
\definecolor{currentstroke}{rgb}{0.450000,0.450000,0.450000}%
\pgfsetstrokecolor{currentstroke}%
\pgfsetdash{}{0pt}%
\pgfpathmoveto{\pgfqpoint{0.693677in}{3.064849in}}%
\pgfpathlineto{\pgfqpoint{5.281103in}{3.064849in}}%
\pgfusepath{stroke}%
\end{pgfscope}%
\begin{pgfscope}%
\pgfsetbuttcap%
\pgfsetroundjoin%
\definecolor{currentfill}{rgb}{0.000000,0.000000,0.000000}%
\pgfsetfillcolor{currentfill}%
\pgfsetlinewidth{0.803000pt}%
\definecolor{currentstroke}{rgb}{0.000000,0.000000,0.000000}%
\pgfsetstrokecolor{currentstroke}%
\pgfsetdash{}{0pt}%
\pgfsys@defobject{currentmarker}{\pgfqpoint{-0.048611in}{0.000000in}}{\pgfqpoint{-0.000000in}{0.000000in}}{%
\pgfpathmoveto{\pgfqpoint{-0.000000in}{0.000000in}}%
\pgfpathlineto{\pgfqpoint{-0.048611in}{0.000000in}}%
\pgfusepath{stroke,fill}%
}%
\begin{pgfscope}%
\pgfsys@transformshift{0.693677in}{3.064849in}%
\pgfsys@useobject{currentmarker}{}%
\end{pgfscope}%
\end{pgfscope}%
\begin{pgfscope}%
\definecolor{textcolor}{rgb}{0.000000,0.000000,0.000000}%
\pgfsetstrokecolor{textcolor}%
\pgfsetfillcolor{textcolor}%
\pgftext[x=0.537426in, y=3.026294in, left, base]{\color{textcolor}{\rmfamily\fontsize{8.000000}{9.600000}\selectfont\catcode`\^=\active\def^{\ifmmode\sp\else\^{}\fi}\catcode`\%=\active\def%{\%}$\mathdefault{0}$}}%
\end{pgfscope}%
\begin{pgfscope}%
\definecolor{textcolor}{rgb}{0.000000,0.000000,0.000000}%
\pgfsetstrokecolor{textcolor}%
\pgfsetfillcolor{textcolor}%
\pgftext[x=0.271991in,y=1.865388in,,bottom,rotate=90.000000]{\color{textcolor}{\rmfamily\fontsize{10.000000}{12.000000}\selectfont\catcode`\^=\active\def^{\ifmmode\sp\else\^{}\fi}\catcode`\%=\active\def%{\%}Magnitude in \unit{\dB}}}%
\end{pgfscope}%
\begin{pgfscope}%
\pgfpathrectangle{\pgfqpoint{0.693677in}{0.524170in}}{\pgfqpoint{4.587426in}{2.682436in}}%
\pgfusepath{clip}%
\pgfsetrectcap%
\pgfsetroundjoin%
\pgfsetlinewidth{1.505625pt}%
\definecolor{currentstroke}{rgb}{0.870588,0.560784,0.019608}%
\pgfsetstrokecolor{currentstroke}%
\pgfsetstrokeopacity{0.700000}%
\pgfsetdash{}{0pt}%
\pgfpathmoveto{\pgfqpoint{0.902196in}{3.084677in}}%
\pgfpathlineto{\pgfqpoint{0.912622in}{3.084656in}}%
\pgfpathlineto{\pgfqpoint{0.943900in}{3.081580in}}%
\pgfpathlineto{\pgfqpoint{0.996030in}{3.077583in}}%
\pgfpathlineto{\pgfqpoint{1.048160in}{3.072039in}}%
\pgfpathlineto{\pgfqpoint{1.058586in}{3.071570in}}%
\pgfpathlineto{\pgfqpoint{1.079438in}{3.068900in}}%
\pgfpathlineto{\pgfqpoint{1.110716in}{3.065677in}}%
\pgfpathlineto{\pgfqpoint{1.141993in}{3.061321in}}%
\pgfpathlineto{\pgfqpoint{1.152419in}{3.060624in}}%
\pgfpathlineto{\pgfqpoint{1.162845in}{3.057786in}}%
\pgfpathlineto{\pgfqpoint{1.173271in}{3.057936in}}%
\pgfpathlineto{\pgfqpoint{1.183697in}{3.055300in}}%
\pgfpathlineto{\pgfqpoint{1.214975in}{3.051288in}}%
\pgfpathlineto{\pgfqpoint{1.340087in}{3.026873in}}%
\pgfpathlineto{\pgfqpoint{1.433921in}{3.000746in}}%
\pgfpathlineto{\pgfqpoint{1.486050in}{2.983195in}}%
\pgfpathlineto{\pgfqpoint{1.506902in}{2.974399in}}%
\pgfpathlineto{\pgfqpoint{1.517328in}{2.971028in}}%
\pgfpathlineto{\pgfqpoint{1.527754in}{2.965374in}}%
\pgfpathlineto{\pgfqpoint{1.538180in}{2.962045in}}%
\pgfpathlineto{\pgfqpoint{1.600736in}{2.933868in}}%
\pgfpathlineto{\pgfqpoint{1.642440in}{2.912314in}}%
\pgfpathlineto{\pgfqpoint{1.652866in}{2.907684in}}%
\pgfpathlineto{\pgfqpoint{1.663292in}{2.899621in}}%
\pgfpathlineto{\pgfqpoint{1.673718in}{2.895213in}}%
\pgfpathlineto{\pgfqpoint{1.725848in}{2.864664in}}%
\pgfpathlineto{\pgfqpoint{1.736274in}{2.857341in}}%
\pgfpathlineto{\pgfqpoint{1.746700in}{2.851748in}}%
\pgfpathlineto{\pgfqpoint{1.777977in}{2.831654in}}%
\pgfpathlineto{\pgfqpoint{1.788403in}{2.826403in}}%
\pgfpathlineto{\pgfqpoint{1.809255in}{2.810327in}}%
\pgfpathlineto{\pgfqpoint{1.830107in}{2.797579in}}%
\pgfpathlineto{\pgfqpoint{1.861385in}{2.775268in}}%
\pgfpathlineto{\pgfqpoint{1.871811in}{2.766408in}}%
\pgfpathlineto{\pgfqpoint{1.882237in}{2.760096in}}%
\pgfpathlineto{\pgfqpoint{1.892663in}{2.751842in}}%
\pgfpathlineto{\pgfqpoint{1.903089in}{2.745686in}}%
\pgfpathlineto{\pgfqpoint{1.955219in}{2.704602in}}%
\pgfpathlineto{\pgfqpoint{1.965645in}{2.695736in}}%
\pgfpathlineto{\pgfqpoint{1.996923in}{2.675176in}}%
\pgfpathlineto{\pgfqpoint{2.017775in}{2.656154in}}%
\pgfpathlineto{\pgfqpoint{2.028201in}{2.646434in}}%
\pgfpathlineto{\pgfqpoint{2.038627in}{2.640721in}}%
\pgfpathlineto{\pgfqpoint{2.059479in}{2.622713in}}%
\pgfpathlineto{\pgfqpoint{2.069905in}{2.612391in}}%
\pgfpathlineto{\pgfqpoint{2.080331in}{2.607195in}}%
\pgfpathlineto{\pgfqpoint{2.090756in}{2.594767in}}%
\pgfpathlineto{\pgfqpoint{2.101182in}{2.588998in}}%
\pgfpathlineto{\pgfqpoint{2.111608in}{2.579374in}}%
\pgfpathlineto{\pgfqpoint{2.122034in}{2.567219in}}%
\pgfpathlineto{\pgfqpoint{2.132460in}{2.563647in}}%
\pgfpathlineto{\pgfqpoint{2.142886in}{2.553907in}}%
\pgfpathlineto{\pgfqpoint{2.153312in}{2.539663in}}%
\pgfpathlineto{\pgfqpoint{2.163738in}{2.540369in}}%
\pgfpathlineto{\pgfqpoint{2.174164in}{2.528104in}}%
\pgfpathlineto{\pgfqpoint{2.184590in}{2.519752in}}%
\pgfpathlineto{\pgfqpoint{2.195016in}{2.506283in}}%
\pgfpathlineto{\pgfqpoint{2.205442in}{2.499865in}}%
\pgfpathlineto{\pgfqpoint{2.226294in}{2.477317in}}%
\pgfpathlineto{\pgfqpoint{2.236720in}{2.471634in}}%
\pgfpathlineto{\pgfqpoint{2.257572in}{2.455429in}}%
\pgfpathlineto{\pgfqpoint{2.267998in}{2.447231in}}%
\pgfpathlineto{\pgfqpoint{2.288850in}{2.414554in}}%
\pgfpathlineto{\pgfqpoint{2.299276in}{2.419130in}}%
\pgfpathlineto{\pgfqpoint{2.309702in}{2.406568in}}%
\pgfpathlineto{\pgfqpoint{2.320128in}{2.403310in}}%
\pgfpathlineto{\pgfqpoint{2.330554in}{2.376970in}}%
\pgfpathlineto{\pgfqpoint{2.340980in}{2.377660in}}%
\pgfpathlineto{\pgfqpoint{2.361832in}{2.356098in}}%
\pgfpathlineto{\pgfqpoint{2.372258in}{2.351921in}}%
\pgfpathlineto{\pgfqpoint{2.382684in}{2.337786in}}%
\pgfpathlineto{\pgfqpoint{2.393110in}{2.340057in}}%
\pgfpathlineto{\pgfqpoint{2.403536in}{2.326816in}}%
\pgfpathlineto{\pgfqpoint{2.413961in}{2.304472in}}%
\pgfpathlineto{\pgfqpoint{2.434813in}{2.290169in}}%
\pgfpathlineto{\pgfqpoint{2.455665in}{2.261291in}}%
\pgfpathlineto{\pgfqpoint{2.466091in}{2.261826in}}%
\pgfpathlineto{\pgfqpoint{2.476517in}{2.255791in}}%
\pgfpathlineto{\pgfqpoint{2.486943in}{2.234969in}}%
\pgfpathlineto{\pgfqpoint{2.497369in}{2.240240in}}%
\pgfpathlineto{\pgfqpoint{2.507795in}{2.221657in}}%
\pgfpathlineto{\pgfqpoint{2.518221in}{2.192461in}}%
\pgfpathlineto{\pgfqpoint{2.528647in}{2.175609in}}%
\pgfpathlineto{\pgfqpoint{2.539073in}{2.175451in}}%
\pgfpathlineto{\pgfqpoint{2.549499in}{2.168704in}}%
\pgfpathlineto{\pgfqpoint{2.559925in}{2.155083in}}%
\pgfpathlineto{\pgfqpoint{2.570351in}{2.163871in}}%
\pgfpathlineto{\pgfqpoint{2.580777in}{2.117676in}}%
\pgfpathlineto{\pgfqpoint{2.591203in}{2.125452in}}%
\pgfpathlineto{\pgfqpoint{2.601629in}{2.106063in}}%
\pgfpathlineto{\pgfqpoint{2.612055in}{2.110827in}}%
\pgfpathlineto{\pgfqpoint{2.622481in}{2.095971in}}%
\pgfpathlineto{\pgfqpoint{2.632907in}{2.093406in}}%
\pgfpathlineto{\pgfqpoint{2.643333in}{2.087336in}}%
\pgfpathlineto{\pgfqpoint{2.653759in}{2.066961in}}%
\pgfpathlineto{\pgfqpoint{2.664185in}{2.085380in}}%
\pgfpathlineto{\pgfqpoint{2.674611in}{2.041565in}}%
\pgfpathlineto{\pgfqpoint{2.685037in}{2.034210in}}%
\pgfpathlineto{\pgfqpoint{2.695463in}{1.986832in}}%
\pgfpathlineto{\pgfqpoint{2.705889in}{2.004027in}}%
\pgfpathlineto{\pgfqpoint{2.716315in}{2.024510in}}%
\pgfpathlineto{\pgfqpoint{2.726740in}{1.988290in}}%
\pgfpathlineto{\pgfqpoint{2.737166in}{1.971791in}}%
\pgfpathlineto{\pgfqpoint{2.747592in}{1.952587in}}%
\pgfpathlineto{\pgfqpoint{2.758018in}{1.964376in}}%
\pgfpathlineto{\pgfqpoint{2.768444in}{1.944218in}}%
\pgfpathlineto{\pgfqpoint{2.778870in}{1.981135in}}%
\pgfpathlineto{\pgfqpoint{2.789296in}{1.936199in}}%
\pgfpathlineto{\pgfqpoint{2.799722in}{1.945700in}}%
\pgfpathlineto{\pgfqpoint{2.820574in}{1.887799in}}%
\pgfpathlineto{\pgfqpoint{2.831000in}{1.827989in}}%
\pgfpathlineto{\pgfqpoint{2.841426in}{1.862139in}}%
\pgfpathlineto{\pgfqpoint{2.851852in}{1.866441in}}%
\pgfpathlineto{\pgfqpoint{2.862278in}{1.917227in}}%
\pgfpathlineto{\pgfqpoint{2.872704in}{1.892127in}}%
\pgfpathlineto{\pgfqpoint{2.883130in}{1.874628in}}%
\pgfpathlineto{\pgfqpoint{2.893556in}{1.863801in}}%
\pgfpathlineto{\pgfqpoint{2.903982in}{1.817172in}}%
\pgfpathlineto{\pgfqpoint{2.914408in}{1.842673in}}%
\pgfpathlineto{\pgfqpoint{2.924834in}{1.834428in}}%
\pgfpathlineto{\pgfqpoint{2.935260in}{1.854456in}}%
\pgfpathlineto{\pgfqpoint{2.945686in}{1.843853in}}%
\pgfpathlineto{\pgfqpoint{2.956112in}{1.826373in}}%
\pgfpathlineto{\pgfqpoint{2.966538in}{1.823305in}}%
\pgfpathlineto{\pgfqpoint{2.976964in}{1.812817in}}%
\pgfpathlineto{\pgfqpoint{2.987390in}{1.759838in}}%
\pgfpathlineto{\pgfqpoint{2.997816in}{1.793985in}}%
\pgfpathlineto{\pgfqpoint{3.008242in}{1.771231in}}%
\pgfpathlineto{\pgfqpoint{3.018668in}{1.763031in}}%
\pgfpathlineto{\pgfqpoint{3.029094in}{1.857582in}}%
\pgfpathlineto{\pgfqpoint{3.039520in}{1.786314in}}%
\pgfpathlineto{\pgfqpoint{3.049945in}{1.819913in}}%
\pgfpathlineto{\pgfqpoint{3.060371in}{1.761022in}}%
\pgfpathlineto{\pgfqpoint{3.070797in}{1.850905in}}%
\pgfpathlineto{\pgfqpoint{3.081223in}{1.830668in}}%
\pgfpathlineto{\pgfqpoint{3.091649in}{1.805379in}}%
\pgfpathlineto{\pgfqpoint{3.102075in}{1.816038in}}%
\pgfpathlineto{\pgfqpoint{3.112501in}{1.834283in}}%
\pgfpathlineto{\pgfqpoint{3.122927in}{1.832734in}}%
\pgfpathlineto{\pgfqpoint{3.133353in}{1.766072in}}%
\pgfpathlineto{\pgfqpoint{3.154205in}{1.857648in}}%
\pgfpathlineto{\pgfqpoint{3.164631in}{1.834520in}}%
\pgfpathlineto{\pgfqpoint{3.175057in}{1.837510in}}%
\pgfpathlineto{\pgfqpoint{3.185483in}{1.864952in}}%
\pgfpathlineto{\pgfqpoint{3.195909in}{1.838632in}}%
\pgfpathlineto{\pgfqpoint{3.206335in}{1.873769in}}%
\pgfpathlineto{\pgfqpoint{3.216761in}{1.856833in}}%
\pgfpathlineto{\pgfqpoint{3.237613in}{1.849949in}}%
\pgfpathlineto{\pgfqpoint{3.248039in}{1.863846in}}%
\pgfpathlineto{\pgfqpoint{3.258465in}{1.881857in}}%
\pgfpathlineto{\pgfqpoint{3.268891in}{1.851864in}}%
\pgfpathlineto{\pgfqpoint{3.279317in}{1.866083in}}%
\pgfpathlineto{\pgfqpoint{3.289743in}{1.888284in}}%
\pgfpathlineto{\pgfqpoint{3.300169in}{1.864097in}}%
\pgfpathlineto{\pgfqpoint{3.310595in}{1.905728in}}%
\pgfpathlineto{\pgfqpoint{3.321021in}{1.901507in}}%
\pgfpathlineto{\pgfqpoint{3.331447in}{1.893446in}}%
\pgfpathlineto{\pgfqpoint{3.341873in}{1.914410in}}%
\pgfpathlineto{\pgfqpoint{3.352299in}{1.916308in}}%
\pgfpathlineto{\pgfqpoint{3.362725in}{1.923430in}}%
\pgfpathlineto{\pgfqpoint{3.373150in}{1.909270in}}%
\pgfpathlineto{\pgfqpoint{3.383576in}{1.926032in}}%
\pgfpathlineto{\pgfqpoint{3.394002in}{1.932152in}}%
\pgfpathlineto{\pgfqpoint{3.404428in}{1.939916in}}%
\pgfpathlineto{\pgfqpoint{3.414854in}{1.941194in}}%
\pgfpathlineto{\pgfqpoint{3.425280in}{1.946809in}}%
\pgfpathlineto{\pgfqpoint{3.435706in}{1.948181in}}%
\pgfpathlineto{\pgfqpoint{3.446132in}{1.952079in}}%
\pgfpathlineto{\pgfqpoint{3.456558in}{1.963897in}}%
\pgfpathlineto{\pgfqpoint{3.477410in}{1.957492in}}%
\pgfpathlineto{\pgfqpoint{3.487836in}{1.987373in}}%
\pgfpathlineto{\pgfqpoint{3.498262in}{1.978217in}}%
\pgfpathlineto{\pgfqpoint{3.508688in}{1.995561in}}%
\pgfpathlineto{\pgfqpoint{3.519114in}{1.996058in}}%
\pgfpathlineto{\pgfqpoint{3.529540in}{2.003794in}}%
\pgfpathlineto{\pgfqpoint{3.539966in}{2.005801in}}%
\pgfpathlineto{\pgfqpoint{3.550392in}{2.010244in}}%
\pgfpathlineto{\pgfqpoint{3.571244in}{2.024970in}}%
\pgfpathlineto{\pgfqpoint{3.581670in}{2.032868in}}%
\pgfpathlineto{\pgfqpoint{3.592096in}{2.034649in}}%
\pgfpathlineto{\pgfqpoint{3.602522in}{2.032255in}}%
\pgfpathlineto{\pgfqpoint{3.612948in}{2.050880in}}%
\pgfpathlineto{\pgfqpoint{3.623374in}{2.053700in}}%
\pgfpathlineto{\pgfqpoint{3.633800in}{2.054231in}}%
\pgfpathlineto{\pgfqpoint{3.644226in}{2.058901in}}%
\pgfpathlineto{\pgfqpoint{3.654652in}{2.065753in}}%
\pgfpathlineto{\pgfqpoint{3.665078in}{2.067174in}}%
\pgfpathlineto{\pgfqpoint{3.675504in}{2.076599in}}%
\pgfpathlineto{\pgfqpoint{3.685930in}{2.079876in}}%
\pgfpathlineto{\pgfqpoint{3.696355in}{2.081864in}}%
\pgfpathlineto{\pgfqpoint{3.706781in}{2.086299in}}%
\pgfpathlineto{\pgfqpoint{3.717207in}{2.089492in}}%
\pgfpathlineto{\pgfqpoint{3.738059in}{2.101052in}}%
\pgfpathlineto{\pgfqpoint{3.758911in}{2.106783in}}%
\pgfpathlineto{\pgfqpoint{3.769337in}{2.107613in}}%
\pgfpathlineto{\pgfqpoint{3.779763in}{2.118542in}}%
\pgfpathlineto{\pgfqpoint{3.790189in}{2.118236in}}%
\pgfpathlineto{\pgfqpoint{3.800615in}{2.127928in}}%
\pgfpathlineto{\pgfqpoint{3.831893in}{2.138975in}}%
\pgfpathlineto{\pgfqpoint{3.842319in}{2.145918in}}%
\pgfpathlineto{\pgfqpoint{3.852745in}{2.150130in}}%
\pgfpathlineto{\pgfqpoint{3.863171in}{2.156655in}}%
\pgfpathlineto{\pgfqpoint{3.873597in}{2.159455in}}%
\pgfpathlineto{\pgfqpoint{3.884023in}{2.159962in}}%
\pgfpathlineto{\pgfqpoint{3.894449in}{2.167938in}}%
\pgfpathlineto{\pgfqpoint{3.904875in}{2.173668in}}%
\pgfpathlineto{\pgfqpoint{3.936153in}{2.185600in}}%
\pgfpathlineto{\pgfqpoint{3.957005in}{2.195778in}}%
\pgfpathlineto{\pgfqpoint{3.967431in}{2.199323in}}%
\pgfpathlineto{\pgfqpoint{3.988283in}{2.208392in}}%
\pgfpathlineto{\pgfqpoint{3.998709in}{2.212672in}}%
\pgfpathlineto{\pgfqpoint{4.019560in}{2.222646in}}%
\pgfpathlineto{\pgfqpoint{4.134246in}{2.273305in}}%
\pgfpathlineto{\pgfqpoint{4.144672in}{2.276644in}}%
\pgfpathlineto{\pgfqpoint{4.207228in}{2.303746in}}%
\pgfpathlineto{\pgfqpoint{4.217654in}{2.306937in}}%
\pgfpathlineto{\pgfqpoint{4.248932in}{2.321212in}}%
\pgfpathlineto{\pgfqpoint{4.259358in}{2.324617in}}%
\pgfpathlineto{\pgfqpoint{4.290636in}{2.337586in}}%
\pgfpathlineto{\pgfqpoint{4.301062in}{2.340654in}}%
\pgfpathlineto{\pgfqpoint{4.311488in}{2.346320in}}%
\pgfpathlineto{\pgfqpoint{4.405321in}{2.381871in}}%
\pgfpathlineto{\pgfqpoint{4.436599in}{2.395449in}}%
\pgfpathlineto{\pgfqpoint{4.457451in}{2.401502in}}%
\pgfpathlineto{\pgfqpoint{4.478303in}{2.409434in}}%
\pgfpathlineto{\pgfqpoint{4.561711in}{2.440647in}}%
\pgfpathlineto{\pgfqpoint{4.613841in}{2.457494in}}%
\pgfpathlineto{\pgfqpoint{4.655544in}{2.471173in}}%
\pgfpathlineto{\pgfqpoint{4.665970in}{2.473405in}}%
\pgfpathlineto{\pgfqpoint{4.686822in}{2.481006in}}%
\pgfpathlineto{\pgfqpoint{4.728526in}{2.492324in}}%
\pgfpathlineto{\pgfqpoint{4.749378in}{2.499222in}}%
\pgfpathlineto{\pgfqpoint{4.759804in}{2.500588in}}%
\pgfpathlineto{\pgfqpoint{4.770230in}{2.504337in}}%
\pgfpathlineto{\pgfqpoint{4.791082in}{2.507822in}}%
\pgfpathlineto{\pgfqpoint{4.801508in}{2.510747in}}%
\pgfpathlineto{\pgfqpoint{4.811934in}{2.514872in}}%
\pgfpathlineto{\pgfqpoint{4.822360in}{2.517415in}}%
\pgfpathlineto{\pgfqpoint{4.843212in}{2.520668in}}%
\pgfpathlineto{\pgfqpoint{4.853638in}{2.529123in}}%
\pgfpathlineto{\pgfqpoint{4.864064in}{2.520410in}}%
\pgfpathlineto{\pgfqpoint{4.874490in}{2.518457in}}%
\pgfpathlineto{\pgfqpoint{4.884916in}{2.537856in}}%
\pgfpathlineto{\pgfqpoint{4.895342in}{2.529129in}}%
\pgfpathlineto{\pgfqpoint{4.905768in}{2.528691in}}%
\pgfpathlineto{\pgfqpoint{4.916194in}{2.532092in}}%
\pgfpathlineto{\pgfqpoint{4.937046in}{2.531327in}}%
\pgfpathlineto{\pgfqpoint{4.947472in}{2.533569in}}%
\pgfpathlineto{\pgfqpoint{4.957898in}{2.538461in}}%
\pgfpathlineto{\pgfqpoint{4.968324in}{2.539962in}}%
\pgfpathlineto{\pgfqpoint{4.989175in}{2.551830in}}%
\pgfpathlineto{\pgfqpoint{4.999601in}{2.559372in}}%
\pgfpathlineto{\pgfqpoint{5.010027in}{2.564401in}}%
\pgfpathlineto{\pgfqpoint{5.030879in}{2.576782in}}%
\pgfpathlineto{\pgfqpoint{5.062157in}{2.602496in}}%
\pgfpathlineto{\pgfqpoint{5.072583in}{2.612334in}}%
\pgfpathlineto{\pgfqpoint{5.072583in}{2.612334in}}%
\pgfusepath{stroke}%
\end{pgfscope}%
\begin{pgfscope}%
\pgfpathrectangle{\pgfqpoint{0.693677in}{0.524170in}}{\pgfqpoint{4.587426in}{2.682436in}}%
\pgfusepath{clip}%
\pgfsetrectcap%
\pgfsetroundjoin%
\pgfsetlinewidth{1.505625pt}%
\definecolor{currentstroke}{rgb}{0.003922,0.450980,0.698039}%
\pgfsetstrokecolor{currentstroke}%
\pgfsetstrokeopacity{0.700000}%
\pgfsetdash{}{0pt}%
\pgfpathmoveto{\pgfqpoint{0.912622in}{3.060533in}}%
\pgfpathlineto{\pgfqpoint{1.016882in}{3.058017in}}%
\pgfpathlineto{\pgfqpoint{1.089864in}{3.054123in}}%
\pgfpathlineto{\pgfqpoint{1.152419in}{3.048553in}}%
\pgfpathlineto{\pgfqpoint{1.204549in}{3.041806in}}%
\pgfpathlineto{\pgfqpoint{1.256679in}{3.032738in}}%
\pgfpathlineto{\pgfqpoint{1.308809in}{3.021050in}}%
\pgfpathlineto{\pgfqpoint{1.360939in}{3.006569in}}%
\pgfpathlineto{\pgfqpoint{1.413069in}{2.989258in}}%
\pgfpathlineto{\pgfqpoint{1.465198in}{2.969205in}}%
\pgfpathlineto{\pgfqpoint{1.517328in}{2.946576in}}%
\pgfpathlineto{\pgfqpoint{1.569458in}{2.921574in}}%
\pgfpathlineto{\pgfqpoint{1.621588in}{2.894399in}}%
\pgfpathlineto{\pgfqpoint{1.684144in}{2.859186in}}%
\pgfpathlineto{\pgfqpoint{1.746700in}{2.821405in}}%
\pgfpathlineto{\pgfqpoint{1.819681in}{2.774451in}}%
\pgfpathlineto{\pgfqpoint{1.892663in}{2.724829in}}%
\pgfpathlineto{\pgfqpoint{1.976071in}{2.665486in}}%
\pgfpathlineto{\pgfqpoint{2.090756in}{2.580766in}}%
\pgfpathlineto{\pgfqpoint{2.372258in}{2.371090in}}%
\pgfpathlineto{\pgfqpoint{2.466091in}{2.304813in}}%
\pgfpathlineto{\pgfqpoint{2.549499in}{2.248691in}}%
\pgfpathlineto{\pgfqpoint{2.632907in}{2.195534in}}%
\pgfpathlineto{\pgfqpoint{2.705889in}{2.151505in}}%
\pgfpathlineto{\pgfqpoint{2.789296in}{2.103872in}}%
\pgfpathlineto{\pgfqpoint{2.872704in}{2.058785in}}%
\pgfpathlineto{\pgfqpoint{2.966538in}{2.010579in}}%
\pgfpathlineto{\pgfqpoint{3.081223in}{1.954418in}}%
\pgfpathlineto{\pgfqpoint{3.216761in}{1.890725in}}%
\pgfpathlineto{\pgfqpoint{3.394002in}{1.810001in}}%
\pgfpathlineto{\pgfqpoint{3.696355in}{1.675150in}}%
\pgfpathlineto{\pgfqpoint{4.144672in}{1.475007in}}%
\pgfpathlineto{\pgfqpoint{4.311488in}{1.398167in}}%
\pgfpathlineto{\pgfqpoint{4.436599in}{1.338129in}}%
\pgfpathlineto{\pgfqpoint{4.530433in}{1.290668in}}%
\pgfpathlineto{\pgfqpoint{4.603415in}{1.251453in}}%
\pgfpathlineto{\pgfqpoint{4.665970in}{1.215464in}}%
\pgfpathlineto{\pgfqpoint{4.718100in}{1.183095in}}%
\pgfpathlineto{\pgfqpoint{4.759804in}{1.155068in}}%
\pgfpathlineto{\pgfqpoint{4.801508in}{1.124485in}}%
\pgfpathlineto{\pgfqpoint{4.832786in}{1.099339in}}%
\pgfpathlineto{\pgfqpoint{4.864064in}{1.071699in}}%
\pgfpathlineto{\pgfqpoint{4.895342in}{1.040776in}}%
\pgfpathlineto{\pgfqpoint{4.916194in}{1.017773in}}%
\pgfpathlineto{\pgfqpoint{4.937046in}{0.992291in}}%
\pgfpathlineto{\pgfqpoint{4.957898in}{0.963619in}}%
\pgfpathlineto{\pgfqpoint{4.978749in}{0.930713in}}%
\pgfpathlineto{\pgfqpoint{4.999601in}{0.891960in}}%
\pgfpathlineto{\pgfqpoint{5.020453in}{0.844692in}}%
\pgfpathlineto{\pgfqpoint{5.030879in}{0.816501in}}%
\pgfpathlineto{\pgfqpoint{5.041305in}{0.784108in}}%
\pgfpathlineto{\pgfqpoint{5.051731in}{0.746179in}}%
\pgfpathlineto{\pgfqpoint{5.062157in}{0.700869in}}%
\pgfpathlineto{\pgfqpoint{5.072583in}{0.646099in}}%
\pgfpathlineto{\pgfqpoint{5.072583in}{0.646099in}}%
\pgfusepath{stroke}%
\end{pgfscope}%
\begin{pgfscope}%
\pgfsetrectcap%
\pgfsetmiterjoin%
\pgfsetlinewidth{0.803000pt}%
\definecolor{currentstroke}{rgb}{0.000000,0.000000,0.000000}%
\pgfsetstrokecolor{currentstroke}%
\pgfsetdash{}{0pt}%
\pgfpathmoveto{\pgfqpoint{0.693677in}{0.524170in}}%
\pgfpathlineto{\pgfqpoint{0.693677in}{3.206606in}}%
\pgfusepath{stroke}%
\end{pgfscope}%
\begin{pgfscope}%
\pgfsetrectcap%
\pgfsetmiterjoin%
\pgfsetlinewidth{0.803000pt}%
\definecolor{currentstroke}{rgb}{0.000000,0.000000,0.000000}%
\pgfsetstrokecolor{currentstroke}%
\pgfsetdash{}{0pt}%
\pgfpathmoveto{\pgfqpoint{5.281103in}{0.524170in}}%
\pgfpathlineto{\pgfqpoint{5.281103in}{3.206606in}}%
\pgfusepath{stroke}%
\end{pgfscope}%
\begin{pgfscope}%
\pgfsetrectcap%
\pgfsetmiterjoin%
\pgfsetlinewidth{0.803000pt}%
\definecolor{currentstroke}{rgb}{0.000000,0.000000,0.000000}%
\pgfsetstrokecolor{currentstroke}%
\pgfsetdash{}{0pt}%
\pgfpathmoveto{\pgfqpoint{0.693677in}{0.524170in}}%
\pgfpathlineto{\pgfqpoint{5.281103in}{0.524170in}}%
\pgfusepath{stroke}%
\end{pgfscope}%
\begin{pgfscope}%
\pgfsetrectcap%
\pgfsetmiterjoin%
\pgfsetlinewidth{0.803000pt}%
\definecolor{currentstroke}{rgb}{0.000000,0.000000,0.000000}%
\pgfsetstrokecolor{currentstroke}%
\pgfsetdash{}{0pt}%
\pgfpathmoveto{\pgfqpoint{0.693677in}{3.206606in}}%
\pgfpathlineto{\pgfqpoint{5.281103in}{3.206606in}}%
\pgfusepath{stroke}%
\end{pgfscope}%
\begin{pgfscope}%
\pgfsetbuttcap%
\pgfsetmiterjoin%
\definecolor{currentfill}{rgb}{1.000000,1.000000,1.000000}%
\pgfsetfillcolor{currentfill}%
\pgfsetfillopacity{0.800000}%
\pgfsetlinewidth{1.003750pt}%
\definecolor{currentstroke}{rgb}{0.800000,0.800000,0.800000}%
\pgfsetstrokecolor{currentstroke}%
\pgfsetstrokeopacity{0.800000}%
\pgfsetdash{}{0pt}%
\pgfpathmoveto{\pgfqpoint{0.771455in}{0.579725in}}%
\pgfpathlineto{\pgfqpoint{1.680899in}{0.579725in}}%
\pgfpathquadraticcurveto{\pgfqpoint{1.703121in}{0.579725in}}{\pgfqpoint{1.703121in}{0.601948in}}%
\pgfpathlineto{\pgfqpoint{1.703121in}{0.900614in}}%
\pgfpathquadraticcurveto{\pgfqpoint{1.703121in}{0.922836in}}{\pgfqpoint{1.680899in}{0.922836in}}%
\pgfpathlineto{\pgfqpoint{0.771455in}{0.922836in}}%
\pgfpathquadraticcurveto{\pgfqpoint{0.749232in}{0.922836in}}{\pgfqpoint{0.749232in}{0.900614in}}%
\pgfpathlineto{\pgfqpoint{0.749232in}{0.601948in}}%
\pgfpathquadraticcurveto{\pgfqpoint{0.749232in}{0.579725in}}{\pgfqpoint{0.771455in}{0.579725in}}%
\pgfpathlineto{\pgfqpoint{0.771455in}{0.579725in}}%
\pgfpathclose%
\pgfusepath{stroke,fill}%
\end{pgfscope}%
\begin{pgfscope}%
\pgfsetrectcap%
\pgfsetroundjoin%
\pgfsetlinewidth{1.505625pt}%
\definecolor{currentstroke}{rgb}{0.870588,0.560784,0.019608}%
\pgfsetstrokecolor{currentstroke}%
\pgfsetstrokeopacity{0.700000}%
\pgfsetdash{}{0pt}%
\pgfpathmoveto{\pgfqpoint{0.793677in}{0.839503in}}%
\pgfpathlineto{\pgfqpoint{0.904788in}{0.839503in}}%
\pgfpathlineto{\pgfqpoint{1.015899in}{0.839503in}}%
\pgfusepath{stroke}%
\end{pgfscope}%
\begin{pgfscope}%
\definecolor{textcolor}{rgb}{0.000000,0.000000,0.000000}%
\pgfsetstrokecolor{textcolor}%
\pgfsetfillcolor{textcolor}%
\pgftext[x=1.104788in,y=0.800614in,left,base]{\color{textcolor}{\rmfamily\fontsize{8.000000}{9.600000}\selectfont\catcode`\^=\active\def^{\ifmmode\sp\else\^{}\fi}\catcode`\%=\active\def%{\%}LC Filter}}%
\end{pgfscope}%
\begin{pgfscope}%
\pgfsetrectcap%
\pgfsetroundjoin%
\pgfsetlinewidth{1.505625pt}%
\definecolor{currentstroke}{rgb}{0.003922,0.450980,0.698039}%
\pgfsetstrokecolor{currentstroke}%
\pgfsetstrokeopacity{0.700000}%
\pgfsetdash{}{0pt}%
\pgfpathmoveto{\pgfqpoint{0.793677in}{0.684614in}}%
\pgfpathlineto{\pgfqpoint{0.904788in}{0.684614in}}%
\pgfpathlineto{\pgfqpoint{1.015899in}{0.684614in}}%
\pgfusepath{stroke}%
\end{pgfscope}%
\begin{pgfscope}%
\definecolor{textcolor}{rgb}{0.000000,0.000000,0.000000}%
\pgfsetstrokecolor{textcolor}%
\pgfsetfillcolor{textcolor}%
\pgftext[x=1.104788in,y=0.645725in,left,base]{\color{textcolor}{\rmfamily\fontsize{8.000000}{9.600000}\selectfont\catcode`\^=\active\def^{\ifmmode\sp\else\^{}\fi}\catcode`\%=\active\def%{\%}Simulation}}%
\end{pgfscope}%
\end{pgfpicture}%
\makeatother%
\endgroup%

    \caption{Measured response of input filter used in the digital current driver. Above \qty{10}{\kHz} capacitive coupling through the transformer can be seen.}
    \label{fig:laser_driver_input_filter}
\end{figure}

Figure \ref{fig:laser_driver_input_filter} shows the measurement of the LC-filter output. At low frequencies, there is good agreement with the simulation and the filter rolls off with \qty{-40}{\dB \per decade}. At around \qty{10}{\kHz} the noise floor of the measurement is reached at \qty{-55}{\dB}. Then the ground loop shows itself again coupling through the transformers. The magnitude rising with \qty{20}{\dB \per decade} is an artifact, which can be significantly influence by changing the type of probing and the location of probing, so at this point it is clear that for any usable data above \qty{10}{\kHz} a common-mode choke is required. Additionally the noise floor of the measurement is reached at around the same frequency, requiring a lower noise amplifier. Due to the lack of another amplifier and the choke, the author left the measurement as is. It is still a good example to show the pitfalls of a 2- or 3- measurement. This topic will be revisted later in section \ref{sec:results_current_noise}, when measuring the current noise of the driver.

To conclude, the measurements show that the LC supply filter is correctly damped with the expected corner frequency of \qty{300}{\Hz} and will likely perform as intended, but above \qty{10}{\kHz} the filter performance cannot be accurately measured due to the limited setup.


\clearpage
\subsection{TODO: Need title}
The author had actually intended to test the stability of the laser drivers first, but the plans were foiled by a misbehaving lab. First tests revealed, that depending on the temperature of cooling water supplied to the overhead air conditioning unit there were temperature fluctuation of up to \qty{2}{\K} observable in the lab. While the effect is not suprising, the observed temperature fluctuations impose a limit on the observable accuracy, due to the temperature coeficient of the multimeter used to record the data. To illustrate the effect, a sample measurement\footnote{\qty{100}{\mA} range, \qty{10}{\plc}, AZERO ON, $f_s = \qty{0.5}{\Hz}$} is shown in figure \ref{fig:laser_driver_aircon}. The Keysight \device{34470A} used to record the data is specified at \qty{6}{\uA \per \K} for a current of \qty{50}{\mA} on the \qty{100}{\mA} range.

\begin{figure}[ht]
    \centering
    %% Creator: Matplotlib, PGF backend
%%
%% To include the figure in your LaTeX document, write
%%   \input{<filename>.pgf}
%%
%% Make sure the required packages are loaded in your preamble
%%   \usepackage{pgf}
%%
%% Also ensure that all the required font packages are loaded; for instance,
%% the lmodern package is sometimes necessary when using math font.
%%   \usepackage{lmodern}
%%
%% Figures using additional raster images can only be included by \input if
%% they are in the same directory as the main LaTeX file. For loading figures
%% from other directories you can use the `import` package
%%   \usepackage{import}
%%
%% and then include the figures with
%%   \import{<path to file>}{<filename>.pgf}
%%
%% Matplotlib used the following preamble
%%   \usepackage{siunitx}
%%   \usepackage{fontspec}
%%
\begingroup%
\makeatletter%
\begin{pgfpicture}%
\pgfpathrectangle{\pgfpointorigin}{\pgfqpoint{5.492126in}{3.394321in}}%
\pgfusepath{use as bounding box, clip}%
\begin{pgfscope}%
\pgfsetbuttcap%
\pgfsetmiterjoin%
\definecolor{currentfill}{rgb}{1.000000,1.000000,1.000000}%
\pgfsetfillcolor{currentfill}%
\pgfsetlinewidth{0.000000pt}%
\definecolor{currentstroke}{rgb}{1.000000,1.000000,1.000000}%
\pgfsetstrokecolor{currentstroke}%
\pgfsetdash{}{0pt}%
\pgfpathmoveto{\pgfqpoint{0.000000in}{0.000000in}}%
\pgfpathlineto{\pgfqpoint{5.492126in}{0.000000in}}%
\pgfpathlineto{\pgfqpoint{5.492126in}{3.394321in}}%
\pgfpathlineto{\pgfqpoint{0.000000in}{3.394321in}}%
\pgfpathlineto{\pgfqpoint{0.000000in}{0.000000in}}%
\pgfpathclose%
\pgfusepath{fill}%
\end{pgfscope}%
\begin{pgfscope}%
\pgfsetbuttcap%
\pgfsetmiterjoin%
\definecolor{currentfill}{rgb}{1.000000,1.000000,1.000000}%
\pgfsetfillcolor{currentfill}%
\pgfsetlinewidth{0.000000pt}%
\definecolor{currentstroke}{rgb}{0.000000,0.000000,0.000000}%
\pgfsetstrokecolor{currentstroke}%
\pgfsetstrokeopacity{0.000000}%
\pgfsetdash{}{0pt}%
\pgfpathmoveto{\pgfqpoint{0.667540in}{0.539544in}}%
\pgfpathlineto{\pgfqpoint{4.857257in}{0.539544in}}%
\pgfpathlineto{\pgfqpoint{4.857257in}{3.120077in}}%
\pgfpathlineto{\pgfqpoint{0.667540in}{3.120077in}}%
\pgfpathlineto{\pgfqpoint{0.667540in}{0.539544in}}%
\pgfpathclose%
\pgfusepath{fill}%
\end{pgfscope}%
\begin{pgfscope}%
\pgfpathrectangle{\pgfqpoint{0.667540in}{0.539544in}}{\pgfqpoint{4.189718in}{2.580533in}}%
\pgfusepath{clip}%
\pgfsetrectcap%
\pgfsetroundjoin%
\pgfsetlinewidth{0.803000pt}%
\definecolor{currentstroke}{rgb}{0.450000,0.450000,0.450000}%
\pgfsetstrokecolor{currentstroke}%
\pgfsetdash{}{0pt}%
\pgfpathmoveto{\pgfqpoint{0.857962in}{0.539544in}}%
\pgfpathlineto{\pgfqpoint{0.857962in}{3.120077in}}%
\pgfusepath{stroke}%
\end{pgfscope}%
\begin{pgfscope}%
\pgfsetbuttcap%
\pgfsetroundjoin%
\definecolor{currentfill}{rgb}{0.000000,0.000000,0.000000}%
\pgfsetfillcolor{currentfill}%
\pgfsetlinewidth{0.803000pt}%
\definecolor{currentstroke}{rgb}{0.000000,0.000000,0.000000}%
\pgfsetstrokecolor{currentstroke}%
\pgfsetdash{}{0pt}%
\pgfsys@defobject{currentmarker}{\pgfqpoint{0.000000in}{-0.048611in}}{\pgfqpoint{0.000000in}{0.000000in}}{%
\pgfpathmoveto{\pgfqpoint{0.000000in}{0.000000in}}%
\pgfpathlineto{\pgfqpoint{0.000000in}{-0.048611in}}%
\pgfusepath{stroke,fill}%
}%
\begin{pgfscope}%
\pgfsys@transformshift{0.857962in}{0.539544in}%
\pgfsys@useobject{currentmarker}{}%
\end{pgfscope}%
\end{pgfscope}%
\begin{pgfscope}%
\definecolor{textcolor}{rgb}{0.000000,0.000000,0.000000}%
\pgfsetstrokecolor{textcolor}%
\pgfsetfillcolor{textcolor}%
\pgftext[x=0.857962in,y=0.442322in,,top]{\color{textcolor}\rmfamily\fontsize{8.000000}{9.600000}\selectfont \(\displaystyle {00{:}00}\)}%
\end{pgfscope}%
\begin{pgfscope}%
\pgfpathrectangle{\pgfqpoint{0.667540in}{0.539544in}}{\pgfqpoint{4.189718in}{2.580533in}}%
\pgfusepath{clip}%
\pgfsetrectcap%
\pgfsetroundjoin%
\pgfsetlinewidth{0.803000pt}%
\definecolor{currentstroke}{rgb}{0.450000,0.450000,0.450000}%
\pgfsetstrokecolor{currentstroke}%
\pgfsetdash{}{0pt}%
\pgfpathmoveto{\pgfqpoint{1.334069in}{0.539544in}}%
\pgfpathlineto{\pgfqpoint{1.334069in}{3.120077in}}%
\pgfusepath{stroke}%
\end{pgfscope}%
\begin{pgfscope}%
\pgfsetbuttcap%
\pgfsetroundjoin%
\definecolor{currentfill}{rgb}{0.000000,0.000000,0.000000}%
\pgfsetfillcolor{currentfill}%
\pgfsetlinewidth{0.803000pt}%
\definecolor{currentstroke}{rgb}{0.000000,0.000000,0.000000}%
\pgfsetstrokecolor{currentstroke}%
\pgfsetdash{}{0pt}%
\pgfsys@defobject{currentmarker}{\pgfqpoint{0.000000in}{-0.048611in}}{\pgfqpoint{0.000000in}{0.000000in}}{%
\pgfpathmoveto{\pgfqpoint{0.000000in}{0.000000in}}%
\pgfpathlineto{\pgfqpoint{0.000000in}{-0.048611in}}%
\pgfusepath{stroke,fill}%
}%
\begin{pgfscope}%
\pgfsys@transformshift{1.334069in}{0.539544in}%
\pgfsys@useobject{currentmarker}{}%
\end{pgfscope}%
\end{pgfscope}%
\begin{pgfscope}%
\definecolor{textcolor}{rgb}{0.000000,0.000000,0.000000}%
\pgfsetstrokecolor{textcolor}%
\pgfsetfillcolor{textcolor}%
\pgftext[x=1.334069in,y=0.442322in,,top]{\color{textcolor}\rmfamily\fontsize{8.000000}{9.600000}\selectfont \(\displaystyle {03{:}00}\)}%
\end{pgfscope}%
\begin{pgfscope}%
\pgfpathrectangle{\pgfqpoint{0.667540in}{0.539544in}}{\pgfqpoint{4.189718in}{2.580533in}}%
\pgfusepath{clip}%
\pgfsetrectcap%
\pgfsetroundjoin%
\pgfsetlinewidth{0.803000pt}%
\definecolor{currentstroke}{rgb}{0.450000,0.450000,0.450000}%
\pgfsetstrokecolor{currentstroke}%
\pgfsetdash{}{0pt}%
\pgfpathmoveto{\pgfqpoint{1.810176in}{0.539544in}}%
\pgfpathlineto{\pgfqpoint{1.810176in}{3.120077in}}%
\pgfusepath{stroke}%
\end{pgfscope}%
\begin{pgfscope}%
\pgfsetbuttcap%
\pgfsetroundjoin%
\definecolor{currentfill}{rgb}{0.000000,0.000000,0.000000}%
\pgfsetfillcolor{currentfill}%
\pgfsetlinewidth{0.803000pt}%
\definecolor{currentstroke}{rgb}{0.000000,0.000000,0.000000}%
\pgfsetstrokecolor{currentstroke}%
\pgfsetdash{}{0pt}%
\pgfsys@defobject{currentmarker}{\pgfqpoint{0.000000in}{-0.048611in}}{\pgfqpoint{0.000000in}{0.000000in}}{%
\pgfpathmoveto{\pgfqpoint{0.000000in}{0.000000in}}%
\pgfpathlineto{\pgfqpoint{0.000000in}{-0.048611in}}%
\pgfusepath{stroke,fill}%
}%
\begin{pgfscope}%
\pgfsys@transformshift{1.810176in}{0.539544in}%
\pgfsys@useobject{currentmarker}{}%
\end{pgfscope}%
\end{pgfscope}%
\begin{pgfscope}%
\definecolor{textcolor}{rgb}{0.000000,0.000000,0.000000}%
\pgfsetstrokecolor{textcolor}%
\pgfsetfillcolor{textcolor}%
\pgftext[x=1.810176in,y=0.442322in,,top]{\color{textcolor}\rmfamily\fontsize{8.000000}{9.600000}\selectfont \(\displaystyle {06{:}00}\)}%
\end{pgfscope}%
\begin{pgfscope}%
\pgfpathrectangle{\pgfqpoint{0.667540in}{0.539544in}}{\pgfqpoint{4.189718in}{2.580533in}}%
\pgfusepath{clip}%
\pgfsetrectcap%
\pgfsetroundjoin%
\pgfsetlinewidth{0.803000pt}%
\definecolor{currentstroke}{rgb}{0.450000,0.450000,0.450000}%
\pgfsetstrokecolor{currentstroke}%
\pgfsetdash{}{0pt}%
\pgfpathmoveto{\pgfqpoint{2.286283in}{0.539544in}}%
\pgfpathlineto{\pgfqpoint{2.286283in}{3.120077in}}%
\pgfusepath{stroke}%
\end{pgfscope}%
\begin{pgfscope}%
\pgfsetbuttcap%
\pgfsetroundjoin%
\definecolor{currentfill}{rgb}{0.000000,0.000000,0.000000}%
\pgfsetfillcolor{currentfill}%
\pgfsetlinewidth{0.803000pt}%
\definecolor{currentstroke}{rgb}{0.000000,0.000000,0.000000}%
\pgfsetstrokecolor{currentstroke}%
\pgfsetdash{}{0pt}%
\pgfsys@defobject{currentmarker}{\pgfqpoint{0.000000in}{-0.048611in}}{\pgfqpoint{0.000000in}{0.000000in}}{%
\pgfpathmoveto{\pgfqpoint{0.000000in}{0.000000in}}%
\pgfpathlineto{\pgfqpoint{0.000000in}{-0.048611in}}%
\pgfusepath{stroke,fill}%
}%
\begin{pgfscope}%
\pgfsys@transformshift{2.286283in}{0.539544in}%
\pgfsys@useobject{currentmarker}{}%
\end{pgfscope}%
\end{pgfscope}%
\begin{pgfscope}%
\definecolor{textcolor}{rgb}{0.000000,0.000000,0.000000}%
\pgfsetstrokecolor{textcolor}%
\pgfsetfillcolor{textcolor}%
\pgftext[x=2.286283in,y=0.442322in,,top]{\color{textcolor}\rmfamily\fontsize{8.000000}{9.600000}\selectfont \(\displaystyle {09{:}00}\)}%
\end{pgfscope}%
\begin{pgfscope}%
\pgfpathrectangle{\pgfqpoint{0.667540in}{0.539544in}}{\pgfqpoint{4.189718in}{2.580533in}}%
\pgfusepath{clip}%
\pgfsetrectcap%
\pgfsetroundjoin%
\pgfsetlinewidth{0.803000pt}%
\definecolor{currentstroke}{rgb}{0.450000,0.450000,0.450000}%
\pgfsetstrokecolor{currentstroke}%
\pgfsetdash{}{0pt}%
\pgfpathmoveto{\pgfqpoint{2.762390in}{0.539544in}}%
\pgfpathlineto{\pgfqpoint{2.762390in}{3.120077in}}%
\pgfusepath{stroke}%
\end{pgfscope}%
\begin{pgfscope}%
\pgfsetbuttcap%
\pgfsetroundjoin%
\definecolor{currentfill}{rgb}{0.000000,0.000000,0.000000}%
\pgfsetfillcolor{currentfill}%
\pgfsetlinewidth{0.803000pt}%
\definecolor{currentstroke}{rgb}{0.000000,0.000000,0.000000}%
\pgfsetstrokecolor{currentstroke}%
\pgfsetdash{}{0pt}%
\pgfsys@defobject{currentmarker}{\pgfqpoint{0.000000in}{-0.048611in}}{\pgfqpoint{0.000000in}{0.000000in}}{%
\pgfpathmoveto{\pgfqpoint{0.000000in}{0.000000in}}%
\pgfpathlineto{\pgfqpoint{0.000000in}{-0.048611in}}%
\pgfusepath{stroke,fill}%
}%
\begin{pgfscope}%
\pgfsys@transformshift{2.762390in}{0.539544in}%
\pgfsys@useobject{currentmarker}{}%
\end{pgfscope}%
\end{pgfscope}%
\begin{pgfscope}%
\definecolor{textcolor}{rgb}{0.000000,0.000000,0.000000}%
\pgfsetstrokecolor{textcolor}%
\pgfsetfillcolor{textcolor}%
\pgftext[x=2.762390in,y=0.442322in,,top]{\color{textcolor}\rmfamily\fontsize{8.000000}{9.600000}\selectfont \(\displaystyle {12{:}00}\)}%
\end{pgfscope}%
\begin{pgfscope}%
\pgfpathrectangle{\pgfqpoint{0.667540in}{0.539544in}}{\pgfqpoint{4.189718in}{2.580533in}}%
\pgfusepath{clip}%
\pgfsetrectcap%
\pgfsetroundjoin%
\pgfsetlinewidth{0.803000pt}%
\definecolor{currentstroke}{rgb}{0.450000,0.450000,0.450000}%
\pgfsetstrokecolor{currentstroke}%
\pgfsetdash{}{0pt}%
\pgfpathmoveto{\pgfqpoint{3.238497in}{0.539544in}}%
\pgfpathlineto{\pgfqpoint{3.238497in}{3.120077in}}%
\pgfusepath{stroke}%
\end{pgfscope}%
\begin{pgfscope}%
\pgfsetbuttcap%
\pgfsetroundjoin%
\definecolor{currentfill}{rgb}{0.000000,0.000000,0.000000}%
\pgfsetfillcolor{currentfill}%
\pgfsetlinewidth{0.803000pt}%
\definecolor{currentstroke}{rgb}{0.000000,0.000000,0.000000}%
\pgfsetstrokecolor{currentstroke}%
\pgfsetdash{}{0pt}%
\pgfsys@defobject{currentmarker}{\pgfqpoint{0.000000in}{-0.048611in}}{\pgfqpoint{0.000000in}{0.000000in}}{%
\pgfpathmoveto{\pgfqpoint{0.000000in}{0.000000in}}%
\pgfpathlineto{\pgfqpoint{0.000000in}{-0.048611in}}%
\pgfusepath{stroke,fill}%
}%
\begin{pgfscope}%
\pgfsys@transformshift{3.238497in}{0.539544in}%
\pgfsys@useobject{currentmarker}{}%
\end{pgfscope}%
\end{pgfscope}%
\begin{pgfscope}%
\definecolor{textcolor}{rgb}{0.000000,0.000000,0.000000}%
\pgfsetstrokecolor{textcolor}%
\pgfsetfillcolor{textcolor}%
\pgftext[x=3.238497in,y=0.442322in,,top]{\color{textcolor}\rmfamily\fontsize{8.000000}{9.600000}\selectfont \(\displaystyle {15{:}00}\)}%
\end{pgfscope}%
\begin{pgfscope}%
\pgfpathrectangle{\pgfqpoint{0.667540in}{0.539544in}}{\pgfqpoint{4.189718in}{2.580533in}}%
\pgfusepath{clip}%
\pgfsetrectcap%
\pgfsetroundjoin%
\pgfsetlinewidth{0.803000pt}%
\definecolor{currentstroke}{rgb}{0.450000,0.450000,0.450000}%
\pgfsetstrokecolor{currentstroke}%
\pgfsetdash{}{0pt}%
\pgfpathmoveto{\pgfqpoint{3.714604in}{0.539544in}}%
\pgfpathlineto{\pgfqpoint{3.714604in}{3.120077in}}%
\pgfusepath{stroke}%
\end{pgfscope}%
\begin{pgfscope}%
\pgfsetbuttcap%
\pgfsetroundjoin%
\definecolor{currentfill}{rgb}{0.000000,0.000000,0.000000}%
\pgfsetfillcolor{currentfill}%
\pgfsetlinewidth{0.803000pt}%
\definecolor{currentstroke}{rgb}{0.000000,0.000000,0.000000}%
\pgfsetstrokecolor{currentstroke}%
\pgfsetdash{}{0pt}%
\pgfsys@defobject{currentmarker}{\pgfqpoint{0.000000in}{-0.048611in}}{\pgfqpoint{0.000000in}{0.000000in}}{%
\pgfpathmoveto{\pgfqpoint{0.000000in}{0.000000in}}%
\pgfpathlineto{\pgfqpoint{0.000000in}{-0.048611in}}%
\pgfusepath{stroke,fill}%
}%
\begin{pgfscope}%
\pgfsys@transformshift{3.714604in}{0.539544in}%
\pgfsys@useobject{currentmarker}{}%
\end{pgfscope}%
\end{pgfscope}%
\begin{pgfscope}%
\definecolor{textcolor}{rgb}{0.000000,0.000000,0.000000}%
\pgfsetstrokecolor{textcolor}%
\pgfsetfillcolor{textcolor}%
\pgftext[x=3.714604in,y=0.442322in,,top]{\color{textcolor}\rmfamily\fontsize{8.000000}{9.600000}\selectfont \(\displaystyle {18{:}00}\)}%
\end{pgfscope}%
\begin{pgfscope}%
\pgfpathrectangle{\pgfqpoint{0.667540in}{0.539544in}}{\pgfqpoint{4.189718in}{2.580533in}}%
\pgfusepath{clip}%
\pgfsetrectcap%
\pgfsetroundjoin%
\pgfsetlinewidth{0.803000pt}%
\definecolor{currentstroke}{rgb}{0.450000,0.450000,0.450000}%
\pgfsetstrokecolor{currentstroke}%
\pgfsetdash{}{0pt}%
\pgfpathmoveto{\pgfqpoint{4.190711in}{0.539544in}}%
\pgfpathlineto{\pgfqpoint{4.190711in}{3.120077in}}%
\pgfusepath{stroke}%
\end{pgfscope}%
\begin{pgfscope}%
\pgfsetbuttcap%
\pgfsetroundjoin%
\definecolor{currentfill}{rgb}{0.000000,0.000000,0.000000}%
\pgfsetfillcolor{currentfill}%
\pgfsetlinewidth{0.803000pt}%
\definecolor{currentstroke}{rgb}{0.000000,0.000000,0.000000}%
\pgfsetstrokecolor{currentstroke}%
\pgfsetdash{}{0pt}%
\pgfsys@defobject{currentmarker}{\pgfqpoint{0.000000in}{-0.048611in}}{\pgfqpoint{0.000000in}{0.000000in}}{%
\pgfpathmoveto{\pgfqpoint{0.000000in}{0.000000in}}%
\pgfpathlineto{\pgfqpoint{0.000000in}{-0.048611in}}%
\pgfusepath{stroke,fill}%
}%
\begin{pgfscope}%
\pgfsys@transformshift{4.190711in}{0.539544in}%
\pgfsys@useobject{currentmarker}{}%
\end{pgfscope}%
\end{pgfscope}%
\begin{pgfscope}%
\definecolor{textcolor}{rgb}{0.000000,0.000000,0.000000}%
\pgfsetstrokecolor{textcolor}%
\pgfsetfillcolor{textcolor}%
\pgftext[x=4.190711in,y=0.442322in,,top]{\color{textcolor}\rmfamily\fontsize{8.000000}{9.600000}\selectfont \(\displaystyle {21{:}00}\)}%
\end{pgfscope}%
\begin{pgfscope}%
\pgfpathrectangle{\pgfqpoint{0.667540in}{0.539544in}}{\pgfqpoint{4.189718in}{2.580533in}}%
\pgfusepath{clip}%
\pgfsetrectcap%
\pgfsetroundjoin%
\pgfsetlinewidth{0.803000pt}%
\definecolor{currentstroke}{rgb}{0.450000,0.450000,0.450000}%
\pgfsetstrokecolor{currentstroke}%
\pgfsetdash{}{0pt}%
\pgfpathmoveto{\pgfqpoint{4.666818in}{0.539544in}}%
\pgfpathlineto{\pgfqpoint{4.666818in}{3.120077in}}%
\pgfusepath{stroke}%
\end{pgfscope}%
\begin{pgfscope}%
\pgfsetbuttcap%
\pgfsetroundjoin%
\definecolor{currentfill}{rgb}{0.000000,0.000000,0.000000}%
\pgfsetfillcolor{currentfill}%
\pgfsetlinewidth{0.803000pt}%
\definecolor{currentstroke}{rgb}{0.000000,0.000000,0.000000}%
\pgfsetstrokecolor{currentstroke}%
\pgfsetdash{}{0pt}%
\pgfsys@defobject{currentmarker}{\pgfqpoint{0.000000in}{-0.048611in}}{\pgfqpoint{0.000000in}{0.000000in}}{%
\pgfpathmoveto{\pgfqpoint{0.000000in}{0.000000in}}%
\pgfpathlineto{\pgfqpoint{0.000000in}{-0.048611in}}%
\pgfusepath{stroke,fill}%
}%
\begin{pgfscope}%
\pgfsys@transformshift{4.666818in}{0.539544in}%
\pgfsys@useobject{currentmarker}{}%
\end{pgfscope}%
\end{pgfscope}%
\begin{pgfscope}%
\definecolor{textcolor}{rgb}{0.000000,0.000000,0.000000}%
\pgfsetstrokecolor{textcolor}%
\pgfsetfillcolor{textcolor}%
\pgftext[x=4.666818in,y=0.442322in,,top]{\color{textcolor}\rmfamily\fontsize{8.000000}{9.600000}\selectfont \(\displaystyle {00{:}00}\)}%
\end{pgfscope}%
\begin{pgfscope}%
\definecolor{textcolor}{rgb}{0.000000,0.000000,0.000000}%
\pgfsetstrokecolor{textcolor}%
\pgfsetfillcolor{textcolor}%
\pgftext[x=2.762399in,y=0.288100in,,top]{\color{textcolor}\rmfamily\fontsize{10.000000}{12.000000}\selectfont Time (UTC)}%
\end{pgfscope}%
\begin{pgfscope}%
\pgfpathrectangle{\pgfqpoint{0.667540in}{0.539544in}}{\pgfqpoint{4.189718in}{2.580533in}}%
\pgfusepath{clip}%
\pgfsetrectcap%
\pgfsetroundjoin%
\pgfsetlinewidth{0.803000pt}%
\definecolor{currentstroke}{rgb}{0.450000,0.450000,0.450000}%
\pgfsetstrokecolor{currentstroke}%
\pgfsetdash{}{0pt}%
\pgfpathmoveto{\pgfqpoint{0.667540in}{1.033864in}}%
\pgfpathlineto{\pgfqpoint{4.857257in}{1.033864in}}%
\pgfusepath{stroke}%
\end{pgfscope}%
\begin{pgfscope}%
\pgfsetbuttcap%
\pgfsetroundjoin%
\definecolor{currentfill}{rgb}{0.000000,0.000000,0.000000}%
\pgfsetfillcolor{currentfill}%
\pgfsetlinewidth{0.803000pt}%
\definecolor{currentstroke}{rgb}{0.000000,0.000000,0.000000}%
\pgfsetstrokecolor{currentstroke}%
\pgfsetdash{}{0pt}%
\pgfsys@defobject{currentmarker}{\pgfqpoint{-0.048611in}{0.000000in}}{\pgfqpoint{-0.000000in}{0.000000in}}{%
\pgfpathmoveto{\pgfqpoint{-0.000000in}{0.000000in}}%
\pgfpathlineto{\pgfqpoint{-0.048611in}{0.000000in}}%
\pgfusepath{stroke,fill}%
}%
\begin{pgfscope}%
\pgfsys@transformshift{0.667540in}{1.033864in}%
\pgfsys@useobject{currentmarker}{}%
\end{pgfscope}%
\end{pgfscope}%
\begin{pgfscope}%
\definecolor{textcolor}{rgb}{0.000000,0.000000,0.000000}%
\pgfsetstrokecolor{textcolor}%
\pgfsetfillcolor{textcolor}%
\pgftext[x=0.327644in, y=0.995308in, left, base]{\color{textcolor}\rmfamily\fontsize{8.000000}{9.600000}\selectfont \(\displaystyle {\ensuremath{-}1.0}\)}%
\end{pgfscope}%
\begin{pgfscope}%
\pgfpathrectangle{\pgfqpoint{0.667540in}{0.539544in}}{\pgfqpoint{4.189718in}{2.580533in}}%
\pgfusepath{clip}%
\pgfsetrectcap%
\pgfsetroundjoin%
\pgfsetlinewidth{0.803000pt}%
\definecolor{currentstroke}{rgb}{0.450000,0.450000,0.450000}%
\pgfsetstrokecolor{currentstroke}%
\pgfsetdash{}{0pt}%
\pgfpathmoveto{\pgfqpoint{0.667540in}{1.580193in}}%
\pgfpathlineto{\pgfqpoint{4.857257in}{1.580193in}}%
\pgfusepath{stroke}%
\end{pgfscope}%
\begin{pgfscope}%
\pgfsetbuttcap%
\pgfsetroundjoin%
\definecolor{currentfill}{rgb}{0.000000,0.000000,0.000000}%
\pgfsetfillcolor{currentfill}%
\pgfsetlinewidth{0.803000pt}%
\definecolor{currentstroke}{rgb}{0.000000,0.000000,0.000000}%
\pgfsetstrokecolor{currentstroke}%
\pgfsetdash{}{0pt}%
\pgfsys@defobject{currentmarker}{\pgfqpoint{-0.048611in}{0.000000in}}{\pgfqpoint{-0.000000in}{0.000000in}}{%
\pgfpathmoveto{\pgfqpoint{-0.000000in}{0.000000in}}%
\pgfpathlineto{\pgfqpoint{-0.048611in}{0.000000in}}%
\pgfusepath{stroke,fill}%
}%
\begin{pgfscope}%
\pgfsys@transformshift{0.667540in}{1.580193in}%
\pgfsys@useobject{currentmarker}{}%
\end{pgfscope}%
\end{pgfscope}%
\begin{pgfscope}%
\definecolor{textcolor}{rgb}{0.000000,0.000000,0.000000}%
\pgfsetstrokecolor{textcolor}%
\pgfsetfillcolor{textcolor}%
\pgftext[x=0.327644in, y=1.541638in, left, base]{\color{textcolor}\rmfamily\fontsize{8.000000}{9.600000}\selectfont \(\displaystyle {\ensuremath{-}0.5}\)}%
\end{pgfscope}%
\begin{pgfscope}%
\pgfpathrectangle{\pgfqpoint{0.667540in}{0.539544in}}{\pgfqpoint{4.189718in}{2.580533in}}%
\pgfusepath{clip}%
\pgfsetrectcap%
\pgfsetroundjoin%
\pgfsetlinewidth{0.803000pt}%
\definecolor{currentstroke}{rgb}{0.450000,0.450000,0.450000}%
\pgfsetstrokecolor{currentstroke}%
\pgfsetdash{}{0pt}%
\pgfpathmoveto{\pgfqpoint{0.667540in}{2.126523in}}%
\pgfpathlineto{\pgfqpoint{4.857257in}{2.126523in}}%
\pgfusepath{stroke}%
\end{pgfscope}%
\begin{pgfscope}%
\pgfsetbuttcap%
\pgfsetroundjoin%
\definecolor{currentfill}{rgb}{0.000000,0.000000,0.000000}%
\pgfsetfillcolor{currentfill}%
\pgfsetlinewidth{0.803000pt}%
\definecolor{currentstroke}{rgb}{0.000000,0.000000,0.000000}%
\pgfsetstrokecolor{currentstroke}%
\pgfsetdash{}{0pt}%
\pgfsys@defobject{currentmarker}{\pgfqpoint{-0.048611in}{0.000000in}}{\pgfqpoint{-0.000000in}{0.000000in}}{%
\pgfpathmoveto{\pgfqpoint{-0.000000in}{0.000000in}}%
\pgfpathlineto{\pgfqpoint{-0.048611in}{0.000000in}}%
\pgfusepath{stroke,fill}%
}%
\begin{pgfscope}%
\pgfsys@transformshift{0.667540in}{2.126523in}%
\pgfsys@useobject{currentmarker}{}%
\end{pgfscope}%
\end{pgfscope}%
\begin{pgfscope}%
\definecolor{textcolor}{rgb}{0.000000,0.000000,0.000000}%
\pgfsetstrokecolor{textcolor}%
\pgfsetfillcolor{textcolor}%
\pgftext[x=0.419467in, y=2.087967in, left, base]{\color{textcolor}\rmfamily\fontsize{8.000000}{9.600000}\selectfont \(\displaystyle {0.0}\)}%
\end{pgfscope}%
\begin{pgfscope}%
\pgfpathrectangle{\pgfqpoint{0.667540in}{0.539544in}}{\pgfqpoint{4.189718in}{2.580533in}}%
\pgfusepath{clip}%
\pgfsetrectcap%
\pgfsetroundjoin%
\pgfsetlinewidth{0.803000pt}%
\definecolor{currentstroke}{rgb}{0.450000,0.450000,0.450000}%
\pgfsetstrokecolor{currentstroke}%
\pgfsetdash{}{0pt}%
\pgfpathmoveto{\pgfqpoint{0.667540in}{2.672852in}}%
\pgfpathlineto{\pgfqpoint{4.857257in}{2.672852in}}%
\pgfusepath{stroke}%
\end{pgfscope}%
\begin{pgfscope}%
\pgfsetbuttcap%
\pgfsetroundjoin%
\definecolor{currentfill}{rgb}{0.000000,0.000000,0.000000}%
\pgfsetfillcolor{currentfill}%
\pgfsetlinewidth{0.803000pt}%
\definecolor{currentstroke}{rgb}{0.000000,0.000000,0.000000}%
\pgfsetstrokecolor{currentstroke}%
\pgfsetdash{}{0pt}%
\pgfsys@defobject{currentmarker}{\pgfqpoint{-0.048611in}{0.000000in}}{\pgfqpoint{-0.000000in}{0.000000in}}{%
\pgfpathmoveto{\pgfqpoint{-0.000000in}{0.000000in}}%
\pgfpathlineto{\pgfqpoint{-0.048611in}{0.000000in}}%
\pgfusepath{stroke,fill}%
}%
\begin{pgfscope}%
\pgfsys@transformshift{0.667540in}{2.672852in}%
\pgfsys@useobject{currentmarker}{}%
\end{pgfscope}%
\end{pgfscope}%
\begin{pgfscope}%
\definecolor{textcolor}{rgb}{0.000000,0.000000,0.000000}%
\pgfsetstrokecolor{textcolor}%
\pgfsetfillcolor{textcolor}%
\pgftext[x=0.419467in, y=2.634297in, left, base]{\color{textcolor}\rmfamily\fontsize{8.000000}{9.600000}\selectfont \(\displaystyle {0.5}\)}%
\end{pgfscope}%
\begin{pgfscope}%
\definecolor{textcolor}{rgb}{0.000000,0.000000,0.000000}%
\pgfsetstrokecolor{textcolor}%
\pgfsetfillcolor{textcolor}%
\pgftext[x=0.272089in,y=1.829811in,,bottom,rotate=90.000000]{\color{textcolor}\rmfamily\fontsize{10.000000}{12.000000}\selectfont Current deviation in \unit{\A}}%
\end{pgfscope}%
\begin{pgfscope}%
\definecolor{textcolor}{rgb}{0.000000,0.000000,0.000000}%
\pgfsetstrokecolor{textcolor}%
\pgfsetfillcolor{textcolor}%
\pgftext[x=0.667540in,y=3.161744in,left,base]{\color{textcolor}\rmfamily\fontsize{8.000000}{9.600000}\selectfont \(\displaystyle \times{10^{\ensuremath{-}6}}{}\)}%
\end{pgfscope}%
\begin{pgfscope}%
\pgfpathrectangle{\pgfqpoint{0.667540in}{0.539544in}}{\pgfqpoint{4.189718in}{2.580533in}}%
\pgfusepath{clip}%
\pgfsetrectcap%
\pgfsetroundjoin%
\pgfsetlinewidth{1.505625pt}%
\definecolor{currentstroke}{rgb}{0.003922,0.450980,0.698039}%
\pgfsetstrokecolor{currentstroke}%
\pgfsetstrokeopacity{0.700000}%
\pgfsetdash{}{0pt}%
\pgfpathmoveto{\pgfqpoint{0.857982in}{2.539384in}}%
\pgfpathlineto{\pgfqpoint{0.858048in}{2.547688in}}%
\pgfpathlineto{\pgfqpoint{0.858092in}{2.444650in}}%
\pgfpathlineto{\pgfqpoint{0.858996in}{2.232019in}}%
\pgfpathlineto{\pgfqpoint{0.858422in}{2.585931in}}%
\pgfpathlineto{\pgfqpoint{0.859216in}{2.386193in}}%
\pgfpathlineto{\pgfqpoint{0.859767in}{2.236717in}}%
\pgfpathlineto{\pgfqpoint{0.860318in}{2.530096in}}%
\pgfpathlineto{\pgfqpoint{0.860869in}{2.315935in}}%
\pgfpathlineto{\pgfqpoint{0.861067in}{2.575223in}}%
\pgfpathlineto{\pgfqpoint{0.861464in}{2.317792in}}%
\pgfpathlineto{\pgfqpoint{0.862059in}{2.656189in}}%
\pgfpathlineto{\pgfqpoint{0.861552in}{2.294956in}}%
\pgfpathlineto{\pgfqpoint{0.862588in}{2.536215in}}%
\pgfpathlineto{\pgfqpoint{0.863492in}{2.243710in}}%
\pgfpathlineto{\pgfqpoint{0.862787in}{2.621115in}}%
\pgfpathlineto{\pgfqpoint{0.863735in}{2.313750in}}%
\pgfpathlineto{\pgfqpoint{0.864638in}{2.568121in}}%
\pgfpathlineto{\pgfqpoint{0.864418in}{2.254418in}}%
\pgfpathlineto{\pgfqpoint{0.864837in}{2.411433in}}%
\pgfpathlineto{\pgfqpoint{0.865322in}{2.241853in}}%
\pgfpathlineto{\pgfqpoint{0.865718in}{2.504746in}}%
\pgfpathlineto{\pgfqpoint{0.865917in}{2.314514in}}%
\pgfpathlineto{\pgfqpoint{0.866358in}{2.567247in}}%
\pgfpathlineto{\pgfqpoint{0.866953in}{2.254527in}}%
\pgfpathlineto{\pgfqpoint{0.867019in}{2.258133in}}%
\pgfpathlineto{\pgfqpoint{0.867217in}{2.541788in}}%
\pgfpathlineto{\pgfqpoint{0.868143in}{2.483112in}}%
\pgfpathlineto{\pgfqpoint{0.868584in}{2.309488in}}%
\pgfpathlineto{\pgfqpoint{0.868540in}{2.527583in}}%
\pgfpathlineto{\pgfqpoint{0.869179in}{2.393514in}}%
\pgfpathlineto{\pgfqpoint{0.869465in}{2.579375in}}%
\pgfpathlineto{\pgfqpoint{0.869642in}{2.347950in}}%
\pgfpathlineto{\pgfqpoint{0.870303in}{2.501578in}}%
\pgfpathlineto{\pgfqpoint{0.870964in}{2.652037in}}%
\pgfpathlineto{\pgfqpoint{0.870788in}{2.350244in}}%
\pgfpathlineto{\pgfqpoint{0.871383in}{2.447491in}}%
\pgfpathlineto{\pgfqpoint{0.871956in}{2.227539in}}%
\pgfpathlineto{\pgfqpoint{0.872463in}{2.526272in}}%
\pgfpathlineto{\pgfqpoint{0.872507in}{2.321180in}}%
\pgfpathlineto{\pgfqpoint{0.873301in}{2.581014in}}%
\pgfpathlineto{\pgfqpoint{0.872750in}{2.286652in}}%
\pgfpathlineto{\pgfqpoint{0.873631in}{2.391547in}}%
\pgfpathlineto{\pgfqpoint{0.874491in}{2.236936in}}%
\pgfpathlineto{\pgfqpoint{0.874337in}{2.560035in}}%
\pgfpathlineto{\pgfqpoint{0.874734in}{2.300419in}}%
\pgfpathlineto{\pgfqpoint{0.875505in}{2.624611in}}%
\pgfpathlineto{\pgfqpoint{0.875880in}{2.522120in}}%
\pgfpathlineto{\pgfqpoint{0.876321in}{2.320415in}}%
\pgfpathlineto{\pgfqpoint{0.876100in}{2.574677in}}%
\pgfpathlineto{\pgfqpoint{0.877004in}{2.405314in}}%
\pgfpathlineto{\pgfqpoint{0.877996in}{2.669410in}}%
\pgfpathlineto{\pgfqpoint{0.877555in}{2.335603in}}%
\pgfpathlineto{\pgfqpoint{0.878150in}{2.488903in}}%
\pgfpathlineto{\pgfqpoint{0.879120in}{2.420393in}}%
\pgfpathlineto{\pgfqpoint{0.879076in}{2.594891in}}%
\pgfpathlineto{\pgfqpoint{0.879186in}{2.531298in}}%
\pgfpathlineto{\pgfqpoint{0.879781in}{2.709729in}}%
\pgfpathlineto{\pgfqpoint{0.879935in}{2.373737in}}%
\pgfpathlineto{\pgfqpoint{0.880200in}{2.533702in}}%
\pgfpathlineto{\pgfqpoint{0.880420in}{2.351556in}}%
\pgfpathlineto{\pgfqpoint{0.881258in}{2.580796in}}%
\pgfpathlineto{\pgfqpoint{0.881324in}{2.419956in}}%
\pgfpathlineto{\pgfqpoint{0.881368in}{2.460057in}}%
\pgfpathlineto{\pgfqpoint{0.881390in}{2.411870in}}%
\pgfpathlineto{\pgfqpoint{0.881456in}{2.541788in}}%
\pgfpathlineto{\pgfqpoint{0.882184in}{2.268623in}}%
\pgfpathlineto{\pgfqpoint{0.882470in}{2.399742in}}%
\pgfpathlineto{\pgfqpoint{0.883154in}{2.269278in}}%
\pgfpathlineto{\pgfqpoint{0.882647in}{2.622426in}}%
\pgfpathlineto{\pgfqpoint{0.883484in}{2.592924in}}%
\pgfpathlineto{\pgfqpoint{0.883506in}{2.613685in}}%
\pgfpathlineto{\pgfqpoint{0.883749in}{2.239995in}}%
\pgfpathlineto{\pgfqpoint{0.884388in}{2.447819in}}%
\pgfpathlineto{\pgfqpoint{0.885314in}{2.307412in}}%
\pgfpathlineto{\pgfqpoint{0.885380in}{2.605271in}}%
\pgfpathlineto{\pgfqpoint{0.885512in}{2.354178in}}%
\pgfpathlineto{\pgfqpoint{0.886504in}{2.547142in}}%
\pgfpathlineto{\pgfqpoint{0.886085in}{2.260537in}}%
\pgfpathlineto{\pgfqpoint{0.886658in}{2.456997in}}%
\pgfpathlineto{\pgfqpoint{0.886768in}{2.328938in}}%
\pgfpathlineto{\pgfqpoint{0.887430in}{2.674327in}}%
\pgfpathlineto{\pgfqpoint{0.887694in}{2.600026in}}%
\pgfpathlineto{\pgfqpoint{0.888796in}{2.371442in}}%
\pgfpathlineto{\pgfqpoint{0.888355in}{2.635866in}}%
\pgfpathlineto{\pgfqpoint{0.889017in}{2.464974in}}%
\pgfpathlineto{\pgfqpoint{0.889105in}{2.605380in}}%
\pgfpathlineto{\pgfqpoint{0.889722in}{2.348168in}}%
\pgfpathlineto{\pgfqpoint{0.890119in}{2.516001in}}%
\pgfpathlineto{\pgfqpoint{0.891199in}{2.425092in}}%
\pgfpathlineto{\pgfqpoint{0.891045in}{2.653020in}}%
\pgfpathlineto{\pgfqpoint{0.891221in}{2.493492in}}%
\pgfpathlineto{\pgfqpoint{0.891728in}{2.703829in}}%
\pgfpathlineto{\pgfqpoint{0.891419in}{2.442246in}}%
\pgfpathlineto{\pgfqpoint{0.892323in}{2.622535in}}%
\pgfpathlineto{\pgfqpoint{0.892742in}{2.439078in}}%
\pgfpathlineto{\pgfqpoint{0.893117in}{2.674546in}}%
\pgfpathlineto{\pgfqpoint{0.893447in}{2.462133in}}%
\pgfpathlineto{\pgfqpoint{0.893822in}{2.641766in}}%
\pgfpathlineto{\pgfqpoint{0.894175in}{2.366307in}}%
\pgfpathlineto{\pgfqpoint{0.894571in}{2.523540in}}%
\pgfpathlineto{\pgfqpoint{0.894748in}{2.357784in}}%
\pgfpathlineto{\pgfqpoint{0.895012in}{2.698803in}}%
\pgfpathlineto{\pgfqpoint{0.895673in}{2.566372in}}%
\pgfpathlineto{\pgfqpoint{0.896026in}{2.342159in}}%
\pgfpathlineto{\pgfqpoint{0.896709in}{2.624393in}}%
\pgfpathlineto{\pgfqpoint{0.896776in}{2.605490in}}%
\pgfpathlineto{\pgfqpoint{0.897767in}{2.351883in}}%
\pgfpathlineto{\pgfqpoint{0.897569in}{2.717596in}}%
\pgfpathlineto{\pgfqpoint{0.897922in}{2.466394in}}%
\pgfpathlineto{\pgfqpoint{0.898208in}{2.659248in}}%
\pgfpathlineto{\pgfqpoint{0.898120in}{2.403020in}}%
\pgfpathlineto{\pgfqpoint{0.899046in}{2.567356in}}%
\pgfpathlineto{\pgfqpoint{0.899795in}{2.417443in}}%
\pgfpathlineto{\pgfqpoint{0.899421in}{2.677387in}}%
\pgfpathlineto{\pgfqpoint{0.900148in}{2.536980in}}%
\pgfpathlineto{\pgfqpoint{0.900589in}{2.621770in}}%
\pgfpathlineto{\pgfqpoint{0.900897in}{2.388597in}}%
\pgfpathlineto{\pgfqpoint{0.901250in}{2.567137in}}%
\pgfpathlineto{\pgfqpoint{0.901470in}{2.352648in}}%
\pgfpathlineto{\pgfqpoint{0.902066in}{2.580905in}}%
\pgfpathlineto{\pgfqpoint{0.902374in}{2.406626in}}%
\pgfpathlineto{\pgfqpoint{0.902903in}{2.662636in}}%
\pgfpathlineto{\pgfqpoint{0.902440in}{2.370459in}}%
\pgfpathlineto{\pgfqpoint{0.903498in}{2.594017in}}%
\pgfpathlineto{\pgfqpoint{0.903829in}{2.328063in}}%
\pgfpathlineto{\pgfqpoint{0.903542in}{2.648759in}}%
\pgfpathlineto{\pgfqpoint{0.904556in}{2.496551in}}%
\pgfpathlineto{\pgfqpoint{0.904578in}{2.664930in}}%
\pgfpathlineto{\pgfqpoint{0.905174in}{2.398759in}}%
\pgfpathlineto{\pgfqpoint{0.905658in}{2.525835in}}%
\pgfpathlineto{\pgfqpoint{0.906408in}{2.609860in}}%
\pgfpathlineto{\pgfqpoint{0.906430in}{2.375485in}}%
\pgfpathlineto{\pgfqpoint{0.906738in}{2.606582in}}%
\pgfpathlineto{\pgfqpoint{0.907752in}{2.392203in}}%
\pgfpathlineto{\pgfqpoint{0.907201in}{2.685363in}}%
\pgfpathlineto{\pgfqpoint{0.907863in}{2.499283in}}%
\pgfpathlineto{\pgfqpoint{0.908877in}{2.676512in}}%
\pgfpathlineto{\pgfqpoint{0.908722in}{2.433724in}}%
\pgfpathlineto{\pgfqpoint{0.908987in}{2.544410in}}%
\pgfpathlineto{\pgfqpoint{0.909361in}{2.423671in}}%
\pgfpathlineto{\pgfqpoint{0.909163in}{2.706779in}}%
\pgfpathlineto{\pgfqpoint{0.910089in}{2.546049in}}%
\pgfpathlineto{\pgfqpoint{0.910640in}{2.762395in}}%
\pgfpathlineto{\pgfqpoint{0.911103in}{2.413728in}}%
\pgfpathlineto{\pgfqpoint{0.911169in}{2.478523in}}%
\pgfpathlineto{\pgfqpoint{0.911433in}{2.391001in}}%
\pgfpathlineto{\pgfqpoint{0.911852in}{2.650616in}}%
\pgfpathlineto{\pgfqpoint{0.911940in}{2.641766in}}%
\pgfpathlineto{\pgfqpoint{0.911962in}{2.697710in}}%
\pgfpathlineto{\pgfqpoint{0.912580in}{2.425310in}}%
\pgfpathlineto{\pgfqpoint{0.912998in}{2.528566in}}%
\pgfpathlineto{\pgfqpoint{0.913020in}{2.529222in}}%
\pgfpathlineto{\pgfqpoint{0.913087in}{2.379855in}}%
\pgfpathlineto{\pgfqpoint{0.913924in}{2.629091in}}%
\pgfpathlineto{\pgfqpoint{0.914145in}{2.471748in}}%
\pgfpathlineto{\pgfqpoint{0.914211in}{2.652474in}}%
\pgfpathlineto{\pgfqpoint{0.914497in}{2.377779in}}%
\pgfpathlineto{\pgfqpoint{0.915048in}{2.474480in}}%
\pgfpathlineto{\pgfqpoint{0.915070in}{2.316918in}}%
\pgfpathlineto{\pgfqpoint{0.915181in}{2.675966in}}%
\pgfpathlineto{\pgfqpoint{0.916150in}{2.325113in}}%
\pgfpathlineto{\pgfqpoint{0.916217in}{2.491853in}}%
\pgfpathlineto{\pgfqpoint{0.916746in}{2.262941in}}%
\pgfpathlineto{\pgfqpoint{0.917275in}{2.395918in}}%
\pgfpathlineto{\pgfqpoint{0.917297in}{2.297032in}}%
\pgfpathlineto{\pgfqpoint{0.918068in}{2.631276in}}%
\pgfpathlineto{\pgfqpoint{0.918333in}{2.528676in}}%
\pgfpathlineto{\pgfqpoint{0.918972in}{2.710822in}}%
\pgfpathlineto{\pgfqpoint{0.918840in}{2.405314in}}%
\pgfpathlineto{\pgfqpoint{0.919347in}{2.485297in}}%
\pgfpathlineto{\pgfqpoint{0.919523in}{2.391875in}}%
\pgfpathlineto{\pgfqpoint{0.919809in}{2.720765in}}%
\pgfpathlineto{\pgfqpoint{0.920052in}{2.567356in}}%
\pgfpathlineto{\pgfqpoint{0.920074in}{2.725682in}}%
\pgfpathlineto{\pgfqpoint{0.920779in}{2.349261in}}%
\pgfpathlineto{\pgfqpoint{0.921154in}{2.516329in}}%
\pgfpathlineto{\pgfqpoint{0.921463in}{2.390564in}}%
\pgfpathlineto{\pgfqpoint{0.921573in}{2.599699in}}%
\pgfpathlineto{\pgfqpoint{0.922234in}{2.539056in}}%
\pgfpathlineto{\pgfqpoint{0.922565in}{2.337133in}}%
\pgfpathlineto{\pgfqpoint{0.923160in}{2.655096in}}%
\pgfpathlineto{\pgfqpoint{0.923490in}{2.303042in}}%
\pgfpathlineto{\pgfqpoint{0.923821in}{2.666679in}}%
\pgfpathlineto{\pgfqpoint{0.924262in}{2.508899in}}%
\pgfpathlineto{\pgfqpoint{0.924593in}{2.669738in}}%
\pgfpathlineto{\pgfqpoint{0.924328in}{2.421486in}}%
\pgfpathlineto{\pgfqpoint{0.925386in}{2.635866in}}%
\pgfpathlineto{\pgfqpoint{0.925474in}{2.656189in}}%
\pgfpathlineto{\pgfqpoint{0.926510in}{2.398540in}}%
\pgfpathlineto{\pgfqpoint{0.927127in}{2.714756in}}%
\pgfpathlineto{\pgfqpoint{0.927304in}{2.344781in}}%
\pgfpathlineto{\pgfqpoint{0.927612in}{2.460166in}}%
\pgfpathlineto{\pgfqpoint{0.928119in}{2.374064in}}%
\pgfpathlineto{\pgfqpoint{0.927745in}{2.594563in}}%
\pgfpathlineto{\pgfqpoint{0.928207in}{2.525398in}}%
\pgfpathlineto{\pgfqpoint{0.928714in}{2.659686in}}%
\pgfpathlineto{\pgfqpoint{0.928406in}{2.338553in}}%
\pgfpathlineto{\pgfqpoint{0.929287in}{2.460384in}}%
\pgfpathlineto{\pgfqpoint{0.929354in}{2.356910in}}%
\pgfpathlineto{\pgfqpoint{0.929398in}{2.553479in}}%
\pgfpathlineto{\pgfqpoint{0.929816in}{2.540258in}}%
\pgfpathlineto{\pgfqpoint{0.930103in}{2.639690in}}%
\pgfpathlineto{\pgfqpoint{0.930500in}{2.269060in}}%
\pgfpathlineto{\pgfqpoint{0.930874in}{2.407828in}}%
\pgfpathlineto{\pgfqpoint{0.930897in}{2.366525in}}%
\pgfpathlineto{\pgfqpoint{0.931470in}{2.661434in}}%
\pgfpathlineto{\pgfqpoint{0.931910in}{2.525835in}}%
\pgfpathlineto{\pgfqpoint{0.932484in}{2.696071in}}%
\pgfpathlineto{\pgfqpoint{0.932175in}{2.393404in}}%
\pgfpathlineto{\pgfqpoint{0.933013in}{2.579375in}}%
\pgfpathlineto{\pgfqpoint{0.933035in}{2.392858in}}%
\pgfpathlineto{\pgfqpoint{0.933189in}{2.702081in}}%
\pgfpathlineto{\pgfqpoint{0.934115in}{2.639253in}}%
\pgfpathlineto{\pgfqpoint{0.934181in}{2.433942in}}%
\pgfpathlineto{\pgfqpoint{0.935084in}{2.701097in}}%
\pgfpathlineto{\pgfqpoint{0.935173in}{2.613575in}}%
\pgfpathlineto{\pgfqpoint{0.936054in}{2.730599in}}%
\pgfpathlineto{\pgfqpoint{0.935658in}{2.451643in}}%
\pgfpathlineto{\pgfqpoint{0.936275in}{2.594344in}}%
\pgfpathlineto{\pgfqpoint{0.937024in}{2.740542in}}%
\pgfpathlineto{\pgfqpoint{0.936627in}{2.490542in}}%
\pgfpathlineto{\pgfqpoint{0.937355in}{2.626141in}}%
\pgfpathlineto{\pgfqpoint{0.937752in}{2.418208in}}%
\pgfpathlineto{\pgfqpoint{0.937443in}{2.660341in}}%
\pgfpathlineto{\pgfqpoint{0.938479in}{2.473933in}}%
\pgfpathlineto{\pgfqpoint{0.939449in}{2.423016in}}%
\pgfpathlineto{\pgfqpoint{0.939647in}{2.679790in}}%
\pgfpathlineto{\pgfqpoint{0.939890in}{2.461259in}}%
\pgfpathlineto{\pgfqpoint{0.940132in}{2.770809in}}%
\pgfpathlineto{\pgfqpoint{0.940749in}{2.535013in}}%
\pgfpathlineto{\pgfqpoint{0.940859in}{2.749065in}}%
\pgfpathlineto{\pgfqpoint{0.941763in}{2.399742in}}%
\pgfpathlineto{\pgfqpoint{0.941851in}{2.531298in}}%
\pgfpathlineto{\pgfqpoint{0.942182in}{2.259663in}}%
\pgfpathlineto{\pgfqpoint{0.942909in}{2.645262in}}%
\pgfpathlineto{\pgfqpoint{0.942931in}{2.499392in}}%
\pgfpathlineto{\pgfqpoint{0.944011in}{2.645044in}}%
\pgfpathlineto{\pgfqpoint{0.943835in}{2.377779in}}%
\pgfpathlineto{\pgfqpoint{0.944034in}{2.551294in}}%
\pgfpathlineto{\pgfqpoint{0.944739in}{2.316481in}}%
\pgfpathlineto{\pgfqpoint{0.944364in}{2.553916in}}%
\pgfpathlineto{\pgfqpoint{0.945224in}{2.391875in}}%
\pgfpathlineto{\pgfqpoint{0.946238in}{2.611062in}}%
\pgfpathlineto{\pgfqpoint{0.946061in}{2.368929in}}%
\pgfpathlineto{\pgfqpoint{0.946348in}{2.504528in}}%
\pgfpathlineto{\pgfqpoint{0.946370in}{2.504965in}}%
\pgfpathlineto{\pgfqpoint{0.946877in}{2.401709in}}%
\pgfpathlineto{\pgfqpoint{0.947296in}{2.633352in}}%
\pgfpathlineto{\pgfqpoint{0.947450in}{2.573912in}}%
\pgfpathlineto{\pgfqpoint{0.948133in}{2.690826in}}%
\pgfpathlineto{\pgfqpoint{0.948023in}{2.445087in}}%
\pgfpathlineto{\pgfqpoint{0.948244in}{2.555227in}}%
\pgfpathlineto{\pgfqpoint{0.949213in}{2.382696in}}%
\pgfpathlineto{\pgfqpoint{0.948949in}{2.628545in}}%
\pgfpathlineto{\pgfqpoint{0.949346in}{2.409467in}}%
\pgfpathlineto{\pgfqpoint{0.949919in}{2.711696in}}%
\pgfpathlineto{\pgfqpoint{0.950117in}{2.370349in}}%
\pgfpathlineto{\pgfqpoint{0.950492in}{2.639799in}}%
\pgfpathlineto{\pgfqpoint{0.950867in}{2.431538in}}%
\pgfpathlineto{\pgfqpoint{0.951506in}{2.689078in}}%
\pgfpathlineto{\pgfqpoint{0.951594in}{2.495459in}}%
\pgfpathlineto{\pgfqpoint{0.951858in}{2.736609in}}%
\pgfpathlineto{\pgfqpoint{0.952211in}{2.338334in}}%
\pgfpathlineto{\pgfqpoint{0.952674in}{2.509773in}}%
\pgfpathlineto{\pgfqpoint{0.952718in}{2.362482in}}%
\pgfpathlineto{\pgfqpoint{0.953445in}{2.631276in}}%
\pgfpathlineto{\pgfqpoint{0.953776in}{2.537417in}}%
\pgfpathlineto{\pgfqpoint{0.954239in}{2.642203in}}%
\pgfpathlineto{\pgfqpoint{0.954437in}{2.392093in}}%
\pgfpathlineto{\pgfqpoint{0.954856in}{2.606473in}}%
\pgfpathlineto{\pgfqpoint{0.955297in}{2.397775in}}%
\pgfpathlineto{\pgfqpoint{0.955760in}{2.635756in}}%
\pgfpathlineto{\pgfqpoint{0.955958in}{2.562220in}}%
\pgfpathlineto{\pgfqpoint{0.956531in}{2.776928in}}%
\pgfpathlineto{\pgfqpoint{0.956575in}{2.499392in}}%
\pgfpathlineto{\pgfqpoint{0.957060in}{2.701316in}}%
\pgfpathlineto{\pgfqpoint{0.957964in}{2.506713in}}%
\pgfpathlineto{\pgfqpoint{0.957611in}{2.825770in}}%
\pgfpathlineto{\pgfqpoint{0.958162in}{2.740870in}}%
\pgfpathlineto{\pgfqpoint{0.958339in}{2.898978in}}%
\pgfpathlineto{\pgfqpoint{0.958493in}{2.501796in}}%
\pgfpathlineto{\pgfqpoint{0.959220in}{2.700551in}}%
\pgfpathlineto{\pgfqpoint{0.960146in}{2.487482in}}%
\pgfpathlineto{\pgfqpoint{0.959309in}{2.743055in}}%
\pgfpathlineto{\pgfqpoint{0.960345in}{2.559489in}}%
\pgfpathlineto{\pgfqpoint{0.960675in}{2.761631in}}%
\pgfpathlineto{\pgfqpoint{0.960565in}{2.413291in}}%
\pgfpathlineto{\pgfqpoint{0.961447in}{2.654441in}}%
\pgfpathlineto{\pgfqpoint{0.961601in}{2.405424in}}%
\pgfpathlineto{\pgfqpoint{0.962527in}{2.622863in}}%
\pgfpathlineto{\pgfqpoint{0.962769in}{2.758134in}}%
\pgfpathlineto{\pgfqpoint{0.962946in}{2.479397in}}%
\pgfpathlineto{\pgfqpoint{0.963607in}{2.655752in}}%
\pgfpathlineto{\pgfqpoint{0.964312in}{2.440061in}}%
\pgfpathlineto{\pgfqpoint{0.964489in}{2.701097in}}%
\pgfpathlineto{\pgfqpoint{0.964709in}{2.608440in}}%
\pgfpathlineto{\pgfqpoint{0.965524in}{2.538400in}}%
\pgfpathlineto{\pgfqpoint{0.965502in}{2.736609in}}%
\pgfpathlineto{\pgfqpoint{0.965613in}{2.681429in}}%
\pgfpathlineto{\pgfqpoint{0.965635in}{2.769170in}}%
\pgfpathlineto{\pgfqpoint{0.965767in}{2.372535in}}%
\pgfpathlineto{\pgfqpoint{0.966693in}{2.632478in}}%
\pgfpathlineto{\pgfqpoint{0.967023in}{2.436018in}}%
\pgfpathlineto{\pgfqpoint{0.967751in}{2.762286in}}%
\pgfpathlineto{\pgfqpoint{0.967773in}{2.623956in}}%
\pgfpathlineto{\pgfqpoint{0.968037in}{2.707763in}}%
\pgfpathlineto{\pgfqpoint{0.968302in}{2.390782in}}%
\pgfpathlineto{\pgfqpoint{0.968809in}{2.471857in}}%
\pgfpathlineto{\pgfqpoint{0.969382in}{2.422469in}}%
\pgfpathlineto{\pgfqpoint{0.969183in}{2.678916in}}%
\pgfpathlineto{\pgfqpoint{0.969624in}{2.491853in}}%
\pgfpathlineto{\pgfqpoint{0.970550in}{2.750158in}}%
\pgfpathlineto{\pgfqpoint{0.969933in}{2.394716in}}%
\pgfpathlineto{\pgfqpoint{0.970726in}{2.463444in}}%
\pgfpathlineto{\pgfqpoint{0.970947in}{2.334073in}}%
\pgfpathlineto{\pgfqpoint{0.971564in}{2.694651in}}%
\pgfpathlineto{\pgfqpoint{0.971806in}{2.488138in}}%
\pgfpathlineto{\pgfqpoint{0.972181in}{2.679244in}}%
\pgfpathlineto{\pgfqpoint{0.972490in}{2.296595in}}%
\pgfpathlineto{\pgfqpoint{0.972909in}{2.581888in}}%
\pgfpathlineto{\pgfqpoint{0.973989in}{2.363356in}}%
\pgfpathlineto{\pgfqpoint{0.973746in}{2.616963in}}%
\pgfpathlineto{\pgfqpoint{0.974033in}{2.529003in}}%
\pgfpathlineto{\pgfqpoint{0.974231in}{2.418426in}}%
\pgfpathlineto{\pgfqpoint{0.974936in}{2.620350in}}%
\pgfpathlineto{\pgfqpoint{0.975113in}{2.537308in}}%
\pgfpathlineto{\pgfqpoint{0.976281in}{2.694869in}}%
\pgfpathlineto{\pgfqpoint{0.975399in}{2.432959in}}%
\pgfpathlineto{\pgfqpoint{0.976303in}{2.644279in}}%
\pgfpathlineto{\pgfqpoint{0.977405in}{2.347731in}}%
\pgfpathlineto{\pgfqpoint{0.976744in}{2.670284in}}%
\pgfpathlineto{\pgfqpoint{0.977449in}{2.408702in}}%
\pgfpathlineto{\pgfqpoint{0.978022in}{2.631386in}}%
\pgfpathlineto{\pgfqpoint{0.977670in}{2.380730in}}%
\pgfpathlineto{\pgfqpoint{0.978551in}{2.423890in}}%
\pgfpathlineto{\pgfqpoint{0.978595in}{2.390454in}}%
\pgfpathlineto{\pgfqpoint{0.978617in}{2.544082in}}%
\pgfpathlineto{\pgfqpoint{0.978772in}{2.415258in}}%
\pgfpathlineto{\pgfqpoint{0.979213in}{2.648868in}}%
\pgfpathlineto{\pgfqpoint{0.979477in}{2.229615in}}%
\pgfpathlineto{\pgfqpoint{0.979874in}{2.527474in}}%
\pgfpathlineto{\pgfqpoint{0.979962in}{2.234860in}}%
\pgfpathlineto{\pgfqpoint{0.980954in}{2.523322in}}%
\pgfpathlineto{\pgfqpoint{0.981439in}{2.656080in}}%
\pgfpathlineto{\pgfqpoint{0.981152in}{2.377015in}}%
\pgfpathlineto{\pgfqpoint{0.982034in}{2.481473in}}%
\pgfpathlineto{\pgfqpoint{0.982695in}{2.380730in}}%
\pgfpathlineto{\pgfqpoint{0.982916in}{2.671049in}}%
\pgfpathlineto{\pgfqpoint{0.983092in}{2.538400in}}%
\pgfpathlineto{\pgfqpoint{0.983202in}{2.766001in}}%
\pgfpathlineto{\pgfqpoint{0.983797in}{2.398649in}}%
\pgfpathlineto{\pgfqpoint{0.984172in}{2.588663in}}%
\pgfpathlineto{\pgfqpoint{0.984348in}{2.660341in}}%
\pgfpathlineto{\pgfqpoint{0.985318in}{2.372753in}}%
\pgfpathlineto{\pgfqpoint{0.985891in}{2.654659in}}%
\pgfpathlineto{\pgfqpoint{0.985362in}{2.330904in}}%
\pgfpathlineto{\pgfqpoint{0.986464in}{2.486936in}}%
\pgfpathlineto{\pgfqpoint{0.987192in}{2.661980in}}%
\pgfpathlineto{\pgfqpoint{0.986905in}{2.424764in}}%
\pgfpathlineto{\pgfqpoint{0.987611in}{2.615105in}}%
\pgfpathlineto{\pgfqpoint{0.988095in}{2.424327in}}%
\pgfpathlineto{\pgfqpoint{0.988580in}{2.686346in}}%
\pgfpathlineto{\pgfqpoint{0.988713in}{2.613248in}}%
\pgfpathlineto{\pgfqpoint{0.988779in}{2.502998in}}%
\pgfpathlineto{\pgfqpoint{0.989087in}{2.763816in}}%
\pgfpathlineto{\pgfqpoint{0.989153in}{2.646464in}}%
\pgfpathlineto{\pgfqpoint{0.989176in}{2.803261in}}%
\pgfpathlineto{\pgfqpoint{0.989462in}{2.366525in}}%
\pgfpathlineto{\pgfqpoint{0.990256in}{2.641657in}}%
\pgfpathlineto{\pgfqpoint{0.990322in}{2.780206in}}%
\pgfpathlineto{\pgfqpoint{0.990763in}{2.468798in}}%
\pgfpathlineto{\pgfqpoint{0.991336in}{2.580031in}}%
\pgfpathlineto{\pgfqpoint{0.992438in}{2.412089in}}%
\pgfpathlineto{\pgfqpoint{0.991600in}{2.733440in}}%
\pgfpathlineto{\pgfqpoint{0.992460in}{2.507915in}}%
\pgfpathlineto{\pgfqpoint{0.992482in}{2.507260in}}%
\pgfpathlineto{\pgfqpoint{0.993297in}{2.674546in}}%
\pgfpathlineto{\pgfqpoint{0.992989in}{2.360515in}}%
\pgfpathlineto{\pgfqpoint{0.993584in}{2.537854in}}%
\pgfpathlineto{\pgfqpoint{0.994355in}{2.405752in}}%
\pgfpathlineto{\pgfqpoint{0.993870in}{2.709839in}}%
\pgfpathlineto{\pgfqpoint{0.994664in}{2.619476in}}%
\pgfpathlineto{\pgfqpoint{0.995237in}{2.745787in}}%
\pgfpathlineto{\pgfqpoint{0.995391in}{2.417880in}}%
\pgfpathlineto{\pgfqpoint{0.995766in}{2.617618in}}%
\pgfpathlineto{\pgfqpoint{0.996273in}{2.400507in}}%
\pgfpathlineto{\pgfqpoint{0.995810in}{2.648213in}}%
\pgfpathlineto{\pgfqpoint{0.996912in}{2.513816in}}%
\pgfpathlineto{\pgfqpoint{0.997177in}{2.308723in}}%
\pgfpathlineto{\pgfqpoint{0.997463in}{2.732566in}}%
\pgfpathlineto{\pgfqpoint{0.997992in}{2.491088in}}%
\pgfpathlineto{\pgfqpoint{0.998081in}{2.698366in}}%
\pgfpathlineto{\pgfqpoint{0.998301in}{2.417115in}}%
\pgfpathlineto{\pgfqpoint{0.999072in}{2.505402in}}%
\pgfpathlineto{\pgfqpoint{0.999205in}{2.369912in}}%
\pgfpathlineto{\pgfqpoint{0.999954in}{2.720110in}}%
\pgfpathlineto{\pgfqpoint{1.000174in}{2.491525in}}%
\pgfpathlineto{\pgfqpoint{1.000902in}{2.671705in}}%
\pgfpathlineto{\pgfqpoint{1.000417in}{2.328501in}}%
\pgfpathlineto{\pgfqpoint{1.001100in}{2.635319in}}%
\pgfpathlineto{\pgfqpoint{1.001122in}{2.426075in}}%
\pgfpathlineto{\pgfqpoint{1.001299in}{2.718580in}}%
\pgfpathlineto{\pgfqpoint{1.002202in}{2.517421in}}%
\pgfpathlineto{\pgfqpoint{1.002335in}{2.646027in}}%
\pgfpathlineto{\pgfqpoint{1.002974in}{2.382041in}}%
\pgfpathlineto{\pgfqpoint{1.003304in}{2.495568in}}%
\pgfpathlineto{\pgfqpoint{1.004010in}{2.380183in}}%
\pgfpathlineto{\pgfqpoint{1.004296in}{2.774961in}}%
\pgfpathlineto{\pgfqpoint{1.004407in}{2.469344in}}%
\pgfpathlineto{\pgfqpoint{1.005002in}{2.654987in}}%
\pgfpathlineto{\pgfqpoint{1.005046in}{2.440717in}}%
\pgfpathlineto{\pgfqpoint{1.005465in}{2.551512in}}%
\pgfpathlineto{\pgfqpoint{1.005773in}{2.401053in}}%
\pgfpathlineto{\pgfqpoint{1.006280in}{2.793208in}}%
\pgfpathlineto{\pgfqpoint{1.006567in}{2.482893in}}%
\pgfpathlineto{\pgfqpoint{1.006787in}{2.756058in}}%
\pgfpathlineto{\pgfqpoint{1.007316in}{2.463225in}}%
\pgfpathlineto{\pgfqpoint{1.007669in}{2.548781in}}%
\pgfpathlineto{\pgfqpoint{1.008396in}{2.785232in}}%
\pgfpathlineto{\pgfqpoint{1.007757in}{2.505402in}}%
\pgfpathlineto{\pgfqpoint{1.008793in}{2.606254in}}%
\pgfpathlineto{\pgfqpoint{1.008837in}{2.511193in}}%
\pgfpathlineto{\pgfqpoint{1.008969in}{2.758025in}}%
\pgfpathlineto{\pgfqpoint{1.009851in}{2.695197in}}%
\pgfpathlineto{\pgfqpoint{1.009873in}{2.806211in}}%
\pgfpathlineto{\pgfqpoint{1.010534in}{2.437876in}}%
\pgfpathlineto{\pgfqpoint{1.010931in}{2.554899in}}%
\pgfpathlineto{\pgfqpoint{1.011879in}{2.625704in}}%
\pgfpathlineto{\pgfqpoint{1.011328in}{2.347841in}}%
\pgfpathlineto{\pgfqpoint{1.011945in}{2.502015in}}%
\pgfpathlineto{\pgfqpoint{1.012143in}{2.345437in}}%
\pgfpathlineto{\pgfqpoint{1.012254in}{2.642203in}}%
\pgfpathlineto{\pgfqpoint{1.013025in}{2.503435in}}%
\pgfpathlineto{\pgfqpoint{1.013334in}{2.724043in}}%
\pgfpathlineto{\pgfqpoint{1.013179in}{2.358658in}}%
\pgfpathlineto{\pgfqpoint{1.014127in}{2.584401in}}%
\pgfpathlineto{\pgfqpoint{1.014149in}{2.454484in}}%
\pgfpathlineto{\pgfqpoint{1.015141in}{2.732129in}}%
\pgfpathlineto{\pgfqpoint{1.015229in}{2.620022in}}%
\pgfpathlineto{\pgfqpoint{1.015979in}{2.765564in}}%
\pgfpathlineto{\pgfqpoint{1.016155in}{2.490542in}}%
\pgfpathlineto{\pgfqpoint{1.016309in}{2.710057in}}%
\pgfpathlineto{\pgfqpoint{1.017279in}{2.475463in}}%
\pgfpathlineto{\pgfqpoint{1.016684in}{2.764690in}}%
\pgfpathlineto{\pgfqpoint{1.017433in}{2.584511in}}%
\pgfpathlineto{\pgfqpoint{1.017588in}{2.684270in}}%
\pgfpathlineto{\pgfqpoint{1.018183in}{2.434051in}}%
\pgfpathlineto{\pgfqpoint{1.018491in}{2.555992in}}%
\pgfpathlineto{\pgfqpoint{1.018866in}{2.331451in}}%
\pgfpathlineto{\pgfqpoint{1.018690in}{2.659358in}}%
\pgfpathlineto{\pgfqpoint{1.019593in}{2.560909in}}%
\pgfpathlineto{\pgfqpoint{1.020541in}{2.762068in}}%
\pgfpathlineto{\pgfqpoint{1.020343in}{2.350244in}}%
\pgfpathlineto{\pgfqpoint{1.020629in}{2.476447in}}%
\pgfpathlineto{\pgfqpoint{1.020652in}{2.351337in}}%
\pgfpathlineto{\pgfqpoint{1.021158in}{2.634336in}}%
\pgfpathlineto{\pgfqpoint{1.021732in}{2.379091in}}%
\pgfpathlineto{\pgfqpoint{1.022745in}{2.642968in}}%
\pgfpathlineto{\pgfqpoint{1.022900in}{2.582981in}}%
\pgfpathlineto{\pgfqpoint{1.023804in}{2.340083in}}%
\pgfpathlineto{\pgfqpoint{1.022944in}{2.736499in}}%
\pgfpathlineto{\pgfqpoint{1.024090in}{2.402036in}}%
\pgfpathlineto{\pgfqpoint{1.024134in}{2.711150in}}%
\pgfpathlineto{\pgfqpoint{1.024355in}{2.325878in}}%
\pgfpathlineto{\pgfqpoint{1.025192in}{2.552714in}}%
\pgfpathlineto{\pgfqpoint{1.026250in}{2.326862in}}%
\pgfpathlineto{\pgfqpoint{1.025479in}{2.610953in}}%
\pgfpathlineto{\pgfqpoint{1.026316in}{2.375813in}}%
\pgfpathlineto{\pgfqpoint{1.026867in}{2.571180in}}%
\pgfpathlineto{\pgfqpoint{1.026581in}{2.329812in}}%
\pgfpathlineto{\pgfqpoint{1.027485in}{2.564187in}}%
\pgfpathlineto{\pgfqpoint{1.028124in}{2.190061in}}%
\pgfpathlineto{\pgfqpoint{1.027903in}{2.579484in}}%
\pgfpathlineto{\pgfqpoint{1.028609in}{2.520481in}}%
\pgfpathlineto{\pgfqpoint{1.028895in}{2.315607in}}%
\pgfpathlineto{\pgfqpoint{1.029512in}{2.591722in}}%
\pgfpathlineto{\pgfqpoint{1.029689in}{2.446180in}}%
\pgfpathlineto{\pgfqpoint{1.030152in}{2.603523in}}%
\pgfpathlineto{\pgfqpoint{1.030725in}{2.343798in}}%
\pgfpathlineto{\pgfqpoint{1.030813in}{2.535341in}}%
\pgfpathlineto{\pgfqpoint{1.031915in}{2.289165in}}%
\pgfpathlineto{\pgfqpoint{1.031364in}{2.641766in}}%
\pgfpathlineto{\pgfqpoint{1.031959in}{2.464427in}}%
\pgfpathlineto{\pgfqpoint{1.032334in}{2.606473in}}%
\pgfpathlineto{\pgfqpoint{1.032554in}{2.322491in}}%
\pgfpathlineto{\pgfqpoint{1.032951in}{2.441372in}}%
\pgfpathlineto{\pgfqpoint{1.033502in}{2.217814in}}%
\pgfpathlineto{\pgfqpoint{1.033546in}{2.551294in}}%
\pgfpathlineto{\pgfqpoint{1.034053in}{2.427714in}}%
\pgfpathlineto{\pgfqpoint{1.034847in}{2.660887in}}%
\pgfpathlineto{\pgfqpoint{1.034273in}{2.271901in}}%
\pgfpathlineto{\pgfqpoint{1.035155in}{2.476993in}}%
\pgfpathlineto{\pgfqpoint{1.036103in}{2.330030in}}%
\pgfpathlineto{\pgfqpoint{1.036213in}{2.555227in}}%
\pgfpathlineto{\pgfqpoint{1.036235in}{2.442683in}}%
\pgfpathlineto{\pgfqpoint{1.036588in}{2.590739in}}%
\pgfpathlineto{\pgfqpoint{1.036478in}{2.378544in}}%
\pgfpathlineto{\pgfqpoint{1.037337in}{2.441700in}}%
\pgfpathlineto{\pgfqpoint{1.038483in}{2.598934in}}%
\pgfpathlineto{\pgfqpoint{1.037668in}{2.357237in}}%
\pgfpathlineto{\pgfqpoint{1.038528in}{2.503654in}}%
\pgfpathlineto{\pgfqpoint{1.039145in}{2.563094in}}%
\pgfpathlineto{\pgfqpoint{1.039740in}{2.344344in}}%
\pgfpathlineto{\pgfqpoint{1.040357in}{2.579375in}}%
\pgfpathlineto{\pgfqpoint{1.039872in}{2.200878in}}%
\pgfpathlineto{\pgfqpoint{1.040842in}{2.391328in}}%
\pgfpathlineto{\pgfqpoint{1.041151in}{2.257696in}}%
\pgfpathlineto{\pgfqpoint{1.041018in}{2.548671in}}%
\pgfpathlineto{\pgfqpoint{1.041834in}{2.509554in}}%
\pgfpathlineto{\pgfqpoint{1.042760in}{2.640455in}}%
\pgfpathlineto{\pgfqpoint{1.042164in}{2.291459in}}%
\pgfpathlineto{\pgfqpoint{1.042914in}{2.507697in}}%
\pgfpathlineto{\pgfqpoint{1.043906in}{2.365651in}}%
\pgfpathlineto{\pgfqpoint{1.043685in}{2.702408in}}%
\pgfpathlineto{\pgfqpoint{1.043994in}{2.522557in}}%
\pgfpathlineto{\pgfqpoint{1.044810in}{2.622863in}}%
\pgfpathlineto{\pgfqpoint{1.044104in}{2.412417in}}%
\pgfpathlineto{\pgfqpoint{1.045052in}{2.552386in}}%
\pgfpathlineto{\pgfqpoint{1.046088in}{2.332106in}}%
\pgfpathlineto{\pgfqpoint{1.045625in}{2.635647in}}%
\pgfpathlineto{\pgfqpoint{1.046110in}{2.606801in}}%
\pgfpathlineto{\pgfqpoint{1.047168in}{2.740870in}}%
\pgfpathlineto{\pgfqpoint{1.046661in}{2.427058in}}%
\pgfpathlineto{\pgfqpoint{1.047190in}{2.549436in}}%
\pgfpathlineto{\pgfqpoint{1.048182in}{2.353959in}}%
\pgfpathlineto{\pgfqpoint{1.047477in}{2.656735in}}%
\pgfpathlineto{\pgfqpoint{1.048358in}{2.497207in}}%
\pgfpathlineto{\pgfqpoint{1.048380in}{2.604725in}}%
\pgfpathlineto{\pgfqpoint{1.048887in}{2.314514in}}%
\pgfpathlineto{\pgfqpoint{1.049460in}{2.519607in}}%
\pgfpathlineto{\pgfqpoint{1.049615in}{2.584073in}}%
\pgfpathlineto{\pgfqpoint{1.049791in}{2.358112in}}%
\pgfpathlineto{\pgfqpoint{1.050320in}{2.446836in}}%
\pgfpathlineto{\pgfqpoint{1.050827in}{2.364012in}}%
\pgfpathlineto{\pgfqpoint{1.050607in}{2.667116in}}%
\pgfpathlineto{\pgfqpoint{1.051378in}{2.535232in}}%
\pgfpathlineto{\pgfqpoint{1.051444in}{2.605380in}}%
\pgfpathlineto{\pgfqpoint{1.051687in}{2.371551in}}%
\pgfpathlineto{\pgfqpoint{1.052392in}{2.401381in}}%
\pgfpathlineto{\pgfqpoint{1.053097in}{2.279549in}}%
\pgfpathlineto{\pgfqpoint{1.052458in}{2.543536in}}%
\pgfpathlineto{\pgfqpoint{1.053450in}{2.497863in}}%
\pgfpathlineto{\pgfqpoint{1.054354in}{2.685254in}}%
\pgfpathlineto{\pgfqpoint{1.054288in}{2.388488in}}%
\pgfpathlineto{\pgfqpoint{1.054552in}{2.555555in}}%
\pgfpathlineto{\pgfqpoint{1.055588in}{2.308177in}}%
\pgfpathlineto{\pgfqpoint{1.054927in}{2.587789in}}%
\pgfpathlineto{\pgfqpoint{1.055698in}{2.397884in}}%
\pgfpathlineto{\pgfqpoint{1.056426in}{2.546595in}}%
\pgfpathlineto{\pgfqpoint{1.055764in}{2.260756in}}%
\pgfpathlineto{\pgfqpoint{1.056822in}{2.483767in}}%
\pgfpathlineto{\pgfqpoint{1.057021in}{2.584511in}}%
\pgfpathlineto{\pgfqpoint{1.057704in}{2.307194in}}%
\pgfpathlineto{\pgfqpoint{1.057726in}{2.425529in}}%
\pgfpathlineto{\pgfqpoint{1.058299in}{2.239886in}}%
\pgfpathlineto{\pgfqpoint{1.057814in}{2.533156in}}%
\pgfpathlineto{\pgfqpoint{1.058828in}{2.463225in}}%
\pgfpathlineto{\pgfqpoint{1.059005in}{2.276708in}}%
\pgfpathlineto{\pgfqpoint{1.059644in}{2.696727in}}%
\pgfpathlineto{\pgfqpoint{1.059820in}{2.544410in}}%
\pgfpathlineto{\pgfqpoint{1.059842in}{2.656845in}}%
\pgfpathlineto{\pgfqpoint{1.060614in}{2.340848in}}%
\pgfpathlineto{\pgfqpoint{1.060900in}{2.517203in}}%
\pgfpathlineto{\pgfqpoint{1.060966in}{2.330577in}}%
\pgfpathlineto{\pgfqpoint{1.061275in}{2.644060in}}%
\pgfpathlineto{\pgfqpoint{1.062002in}{2.417771in}}%
\pgfpathlineto{\pgfqpoint{1.062862in}{2.765455in}}%
\pgfpathlineto{\pgfqpoint{1.062267in}{2.415148in}}%
\pgfpathlineto{\pgfqpoint{1.063126in}{2.595765in}}%
\pgfpathlineto{\pgfqpoint{1.064052in}{2.711368in}}%
\pgfpathlineto{\pgfqpoint{1.063611in}{2.366525in}}%
\pgfpathlineto{\pgfqpoint{1.064074in}{2.581670in}}%
\pgfpathlineto{\pgfqpoint{1.064647in}{2.468470in}}%
\pgfpathlineto{\pgfqpoint{1.065110in}{2.745896in}}%
\pgfpathlineto{\pgfqpoint{1.065154in}{2.639799in}}%
\pgfpathlineto{\pgfqpoint{1.065198in}{2.784139in}}%
\pgfpathlineto{\pgfqpoint{1.065948in}{2.505074in}}%
\pgfpathlineto{\pgfqpoint{1.066234in}{2.700879in}}%
\pgfpathlineto{\pgfqpoint{1.066874in}{2.488794in}}%
\pgfpathlineto{\pgfqpoint{1.066411in}{2.748519in}}%
\pgfpathlineto{\pgfqpoint{1.067336in}{2.571071in}}%
\pgfpathlineto{\pgfqpoint{1.067733in}{2.760647in}}%
\pgfpathlineto{\pgfqpoint{1.067843in}{2.499392in}}%
\pgfpathlineto{\pgfqpoint{1.068439in}{2.540804in}}%
\pgfpathlineto{\pgfqpoint{1.068901in}{2.426512in}}%
\pgfpathlineto{\pgfqpoint{1.068769in}{2.803042in}}%
\pgfpathlineto{\pgfqpoint{1.069519in}{2.623956in}}%
\pgfpathlineto{\pgfqpoint{1.069651in}{2.437002in}}%
\pgfpathlineto{\pgfqpoint{1.069981in}{2.734096in}}%
\pgfpathlineto{\pgfqpoint{1.070643in}{2.477102in}}%
\pgfpathlineto{\pgfqpoint{1.071524in}{2.708090in}}%
\pgfpathlineto{\pgfqpoint{1.071238in}{2.418099in}}%
\pgfpathlineto{\pgfqpoint{1.071767in}{2.583855in}}%
\pgfpathlineto{\pgfqpoint{1.072472in}{2.803917in}}%
\pgfpathlineto{\pgfqpoint{1.072384in}{2.403348in}}%
\pgfpathlineto{\pgfqpoint{1.072869in}{2.716832in}}%
\pgfpathlineto{\pgfqpoint{1.073067in}{2.496114in}}%
\pgfpathlineto{\pgfqpoint{1.073244in}{2.784358in}}%
\pgfpathlineto{\pgfqpoint{1.073971in}{2.669957in}}%
\pgfpathlineto{\pgfqpoint{1.074214in}{2.517640in}}%
\pgfpathlineto{\pgfqpoint{1.074522in}{2.771792in}}%
\pgfpathlineto{\pgfqpoint{1.075073in}{2.628873in}}%
\pgfpathlineto{\pgfqpoint{1.076087in}{2.762942in}}%
\pgfpathlineto{\pgfqpoint{1.075933in}{2.511849in}}%
\pgfpathlineto{\pgfqpoint{1.076175in}{2.669629in}}%
\pgfpathlineto{\pgfqpoint{1.076197in}{2.669629in}}%
\pgfpathlineto{\pgfqpoint{1.077057in}{2.474261in}}%
\pgfpathlineto{\pgfqpoint{1.076550in}{2.728305in}}%
\pgfpathlineto{\pgfqpoint{1.077211in}{2.616198in}}%
\pgfpathlineto{\pgfqpoint{1.077233in}{2.777911in}}%
\pgfpathlineto{\pgfqpoint{1.077432in}{2.409029in}}%
\pgfpathlineto{\pgfqpoint{1.078313in}{2.609423in}}%
\pgfpathlineto{\pgfqpoint{1.079195in}{2.804135in}}%
\pgfpathlineto{\pgfqpoint{1.079019in}{2.476884in}}%
\pgfpathlineto{\pgfqpoint{1.079371in}{2.531080in}}%
\pgfpathlineto{\pgfqpoint{1.079570in}{2.853305in}}%
\pgfpathlineto{\pgfqpoint{1.079460in}{2.471639in}}%
\pgfpathlineto{\pgfqpoint{1.080540in}{2.636630in}}%
\pgfpathlineto{\pgfqpoint{1.080628in}{2.512177in}}%
\pgfpathlineto{\pgfqpoint{1.081113in}{2.811565in}}%
\pgfpathlineto{\pgfqpoint{1.081642in}{2.600136in}}%
\pgfpathlineto{\pgfqpoint{1.082391in}{2.692137in}}%
\pgfpathlineto{\pgfqpoint{1.081950in}{2.427386in}}%
\pgfpathlineto{\pgfqpoint{1.082501in}{2.547251in}}%
\pgfpathlineto{\pgfqpoint{1.083493in}{2.392312in}}%
\pgfpathlineto{\pgfqpoint{1.083185in}{2.754091in}}%
\pgfpathlineto{\pgfqpoint{1.083559in}{2.509008in}}%
\pgfpathlineto{\pgfqpoint{1.083581in}{2.705686in}}%
\pgfpathlineto{\pgfqpoint{1.084287in}{2.382806in}}%
\pgfpathlineto{\pgfqpoint{1.084661in}{2.491197in}}%
\pgfpathlineto{\pgfqpoint{1.085830in}{2.731364in}}%
\pgfpathlineto{\pgfqpoint{1.084838in}{2.402146in}}%
\pgfpathlineto{\pgfqpoint{1.085852in}{2.672470in}}%
\pgfpathlineto{\pgfqpoint{1.086006in}{2.437002in}}%
\pgfpathlineto{\pgfqpoint{1.086248in}{2.726228in}}%
\pgfpathlineto{\pgfqpoint{1.086954in}{2.529768in}}%
\pgfpathlineto{\pgfqpoint{1.087968in}{2.722841in}}%
\pgfpathlineto{\pgfqpoint{1.087593in}{2.448693in}}%
\pgfpathlineto{\pgfqpoint{1.088078in}{2.608549in}}%
\pgfpathlineto{\pgfqpoint{1.089180in}{2.536871in}}%
\pgfpathlineto{\pgfqpoint{1.088783in}{2.806867in}}%
\pgfpathlineto{\pgfqpoint{1.089202in}{2.567247in}}%
\pgfpathlineto{\pgfqpoint{1.089709in}{2.470437in}}%
\pgfpathlineto{\pgfqpoint{1.089489in}{2.808396in}}%
\pgfpathlineto{\pgfqpoint{1.090128in}{2.607456in}}%
\pgfpathlineto{\pgfqpoint{1.091010in}{2.740761in}}%
\pgfpathlineto{\pgfqpoint{1.090921in}{2.502343in}}%
\pgfpathlineto{\pgfqpoint{1.091230in}{2.708746in}}%
\pgfpathlineto{\pgfqpoint{1.092178in}{2.476884in}}%
\pgfpathlineto{\pgfqpoint{1.091362in}{2.752234in}}%
\pgfpathlineto{\pgfqpoint{1.092398in}{2.529659in}}%
\pgfpathlineto{\pgfqpoint{1.093522in}{2.800311in}}%
\pgfpathlineto{\pgfqpoint{1.093037in}{2.503982in}}%
\pgfpathlineto{\pgfqpoint{1.093633in}{2.684270in}}%
\pgfpathlineto{\pgfqpoint{1.094140in}{2.523759in}}%
\pgfpathlineto{\pgfqpoint{1.094294in}{2.838117in}}%
\pgfpathlineto{\pgfqpoint{1.094713in}{2.687221in}}%
\pgfpathlineto{\pgfqpoint{1.094801in}{2.831124in}}%
\pgfpathlineto{\pgfqpoint{1.095462in}{2.558942in}}%
\pgfpathlineto{\pgfqpoint{1.095815in}{2.686565in}}%
\pgfpathlineto{\pgfqpoint{1.096851in}{2.430336in}}%
\pgfpathlineto{\pgfqpoint{1.096035in}{2.851775in}}%
\pgfpathlineto{\pgfqpoint{1.096939in}{2.588335in}}%
\pgfpathlineto{\pgfqpoint{1.097027in}{2.528457in}}%
\pgfpathlineto{\pgfqpoint{1.097358in}{2.776928in}}%
\pgfpathlineto{\pgfqpoint{1.097865in}{2.654441in}}%
\pgfpathlineto{\pgfqpoint{1.098504in}{2.814188in}}%
\pgfpathlineto{\pgfqpoint{1.098041in}{2.497426in}}%
\pgfpathlineto{\pgfqpoint{1.098967in}{2.645481in}}%
\pgfpathlineto{\pgfqpoint{1.099187in}{2.475572in}}%
\pgfpathlineto{\pgfqpoint{1.099716in}{2.838991in}}%
\pgfpathlineto{\pgfqpoint{1.100069in}{2.656189in}}%
\pgfpathlineto{\pgfqpoint{1.100752in}{2.831233in}}%
\pgfpathlineto{\pgfqpoint{1.100267in}{2.523868in}}%
\pgfpathlineto{\pgfqpoint{1.101193in}{2.776054in}}%
\pgfpathlineto{\pgfqpoint{1.102053in}{2.379418in}}%
\pgfpathlineto{\pgfqpoint{1.102361in}{2.409794in}}%
\pgfpathlineto{\pgfqpoint{1.103419in}{2.660887in}}%
\pgfpathlineto{\pgfqpoint{1.102427in}{2.385428in}}%
\pgfpathlineto{\pgfqpoint{1.103507in}{2.570852in}}%
\pgfpathlineto{\pgfqpoint{1.104455in}{2.401053in}}%
\pgfpathlineto{\pgfqpoint{1.104036in}{2.648213in}}%
\pgfpathlineto{\pgfqpoint{1.104631in}{2.464974in}}%
\pgfpathlineto{\pgfqpoint{1.104720in}{2.323256in}}%
\pgfpathlineto{\pgfqpoint{1.105425in}{2.687221in}}%
\pgfpathlineto{\pgfqpoint{1.105667in}{2.529331in}}%
\pgfpathlineto{\pgfqpoint{1.106593in}{2.708200in}}%
\pgfpathlineto{\pgfqpoint{1.105888in}{2.372972in}}%
\pgfpathlineto{\pgfqpoint{1.106792in}{2.606254in}}%
\pgfpathlineto{\pgfqpoint{1.107299in}{2.402583in}}%
\pgfpathlineto{\pgfqpoint{1.107056in}{2.693012in}}%
\pgfpathlineto{\pgfqpoint{1.107982in}{2.521683in}}%
\pgfpathlineto{\pgfqpoint{1.108026in}{2.727540in}}%
\pgfpathlineto{\pgfqpoint{1.108797in}{2.422360in}}%
\pgfpathlineto{\pgfqpoint{1.109106in}{2.639690in}}%
\pgfpathlineto{\pgfqpoint{1.109415in}{2.456014in}}%
\pgfpathlineto{\pgfqpoint{1.110010in}{2.722076in}}%
\pgfpathlineto{\pgfqpoint{1.110230in}{2.537308in}}%
\pgfpathlineto{\pgfqpoint{1.110274in}{2.687548in}}%
\pgfpathlineto{\pgfqpoint{1.110539in}{2.407391in}}%
\pgfpathlineto{\pgfqpoint{1.111332in}{2.526381in}}%
\pgfpathlineto{\pgfqpoint{1.112368in}{2.402801in}}%
\pgfpathlineto{\pgfqpoint{1.111773in}{2.721967in}}%
\pgfpathlineto{\pgfqpoint{1.112434in}{2.457762in}}%
\pgfpathlineto{\pgfqpoint{1.113448in}{2.713772in}}%
\pgfpathlineto{\pgfqpoint{1.113096in}{2.445197in}}%
\pgfpathlineto{\pgfqpoint{1.113581in}{2.614231in}}%
\pgfpathlineto{\pgfqpoint{1.114088in}{2.725245in}}%
\pgfpathlineto{\pgfqpoint{1.113977in}{2.527474in}}%
\pgfpathlineto{\pgfqpoint{1.114154in}{2.584401in}}%
\pgfpathlineto{\pgfqpoint{1.114330in}{2.740652in}}%
\pgfpathlineto{\pgfqpoint{1.115278in}{2.466285in}}%
\pgfpathlineto{\pgfqpoint{1.115785in}{2.695197in}}%
\pgfpathlineto{\pgfqpoint{1.116336in}{2.449676in}}%
\pgfpathlineto{\pgfqpoint{1.116402in}{2.553807in}}%
\pgfpathlineto{\pgfqpoint{1.117372in}{2.757479in}}%
\pgfpathlineto{\pgfqpoint{1.116622in}{2.387613in}}%
\pgfpathlineto{\pgfqpoint{1.117548in}{2.602430in}}%
\pgfpathlineto{\pgfqpoint{1.118452in}{2.404877in}}%
\pgfpathlineto{\pgfqpoint{1.117989in}{2.646137in}}%
\pgfpathlineto{\pgfqpoint{1.118672in}{2.414274in}}%
\pgfpathlineto{\pgfqpoint{1.119201in}{2.638925in}}%
\pgfpathlineto{\pgfqpoint{1.119664in}{2.337460in}}%
\pgfpathlineto{\pgfqpoint{1.119774in}{2.508680in}}%
\pgfpathlineto{\pgfqpoint{1.120149in}{2.392203in}}%
\pgfpathlineto{\pgfqpoint{1.120259in}{2.663291in}}%
\pgfpathlineto{\pgfqpoint{1.120832in}{2.491307in}}%
\pgfpathlineto{\pgfqpoint{1.121846in}{2.717487in}}%
\pgfpathlineto{\pgfqpoint{1.121427in}{2.409248in}}%
\pgfpathlineto{\pgfqpoint{1.121956in}{2.662089in}}%
\pgfpathlineto{\pgfqpoint{1.122001in}{2.636630in}}%
\pgfpathlineto{\pgfqpoint{1.122023in}{2.677277in}}%
\pgfpathlineto{\pgfqpoint{1.122596in}{2.460822in}}%
\pgfpathlineto{\pgfqpoint{1.123081in}{2.710385in}}%
\pgfpathlineto{\pgfqpoint{1.123125in}{2.603414in}}%
\pgfpathlineto{\pgfqpoint{1.123301in}{2.753217in}}%
\pgfpathlineto{\pgfqpoint{1.123455in}{2.471202in}}%
\pgfpathlineto{\pgfqpoint{1.124227in}{2.624283in}}%
\pgfpathlineto{\pgfqpoint{1.124513in}{2.505402in}}%
\pgfpathlineto{\pgfqpoint{1.125263in}{2.750813in}}%
\pgfpathlineto{\pgfqpoint{1.125307in}{2.689515in}}%
\pgfpathlineto{\pgfqpoint{1.125395in}{2.749721in}}%
\pgfpathlineto{\pgfqpoint{1.126563in}{2.305008in}}%
\pgfpathlineto{\pgfqpoint{1.127599in}{2.653239in}}%
\pgfpathlineto{\pgfqpoint{1.127687in}{2.639253in}}%
\pgfpathlineto{\pgfqpoint{1.128723in}{2.396355in}}%
\pgfpathlineto{\pgfqpoint{1.127842in}{2.724917in}}%
\pgfpathlineto{\pgfqpoint{1.128812in}{2.398868in}}%
\pgfpathlineto{\pgfqpoint{1.129274in}{2.683943in}}%
\pgfpathlineto{\pgfqpoint{1.129671in}{2.397338in}}%
\pgfpathlineto{\pgfqpoint{1.129914in}{2.476884in}}%
\pgfpathlineto{\pgfqpoint{1.130972in}{2.641329in}}%
\pgfpathlineto{\pgfqpoint{1.130465in}{2.337460in}}%
\pgfpathlineto{\pgfqpoint{1.131060in}{2.633352in}}%
\pgfpathlineto{\pgfqpoint{1.131545in}{2.402583in}}%
\pgfpathlineto{\pgfqpoint{1.131743in}{2.784904in}}%
\pgfpathlineto{\pgfqpoint{1.132162in}{2.496770in}}%
\pgfpathlineto{\pgfqpoint{1.132581in}{2.705249in}}%
\pgfpathlineto{\pgfqpoint{1.132294in}{2.438968in}}%
\pgfpathlineto{\pgfqpoint{1.133242in}{2.596202in}}%
\pgfpathlineto{\pgfqpoint{1.133352in}{2.409248in}}%
\pgfpathlineto{\pgfqpoint{1.134124in}{2.724043in}}%
\pgfpathlineto{\pgfqpoint{1.134344in}{2.607456in}}%
\pgfpathlineto{\pgfqpoint{1.135138in}{2.756714in}}%
\pgfpathlineto{\pgfqpoint{1.134697in}{2.508243in}}%
\pgfpathlineto{\pgfqpoint{1.135446in}{2.700988in}}%
\pgfpathlineto{\pgfqpoint{1.135490in}{2.500922in}}%
\pgfpathlineto{\pgfqpoint{1.136262in}{2.764362in}}%
\pgfpathlineto{\pgfqpoint{1.136548in}{2.671486in}}%
\pgfpathlineto{\pgfqpoint{1.137188in}{2.442137in}}%
\pgfpathlineto{\pgfqpoint{1.136879in}{2.704594in}}%
\pgfpathlineto{\pgfqpoint{1.137739in}{2.589100in}}%
\pgfpathlineto{\pgfqpoint{1.138091in}{2.656189in}}%
\pgfpathlineto{\pgfqpoint{1.138290in}{2.378654in}}%
\pgfpathlineto{\pgfqpoint{1.138686in}{2.544410in}}%
\pgfpathlineto{\pgfqpoint{1.138995in}{2.363029in}}%
\pgfpathlineto{\pgfqpoint{1.139744in}{2.666241in}}%
\pgfpathlineto{\pgfqpoint{1.139899in}{2.375048in}}%
\pgfpathlineto{\pgfqpoint{1.141045in}{2.435909in}}%
\pgfpathlineto{\pgfqpoint{1.141684in}{2.709729in}}%
\pgfpathlineto{\pgfqpoint{1.142147in}{2.563641in}}%
\pgfpathlineto{\pgfqpoint{1.142478in}{2.407828in}}%
\pgfpathlineto{\pgfqpoint{1.142830in}{2.760866in}}%
\pgfpathlineto{\pgfqpoint{1.143249in}{2.434379in}}%
\pgfpathlineto{\pgfqpoint{1.144307in}{2.769498in}}%
\pgfpathlineto{\pgfqpoint{1.144395in}{2.590629in}}%
\pgfpathlineto{\pgfqpoint{1.145123in}{2.490323in}}%
\pgfpathlineto{\pgfqpoint{1.144704in}{2.775398in}}%
\pgfpathlineto{\pgfqpoint{1.145453in}{2.609532in}}%
\pgfpathlineto{\pgfqpoint{1.146467in}{2.735844in}}%
\pgfpathlineto{\pgfqpoint{1.145740in}{2.459183in}}%
\pgfpathlineto{\pgfqpoint{1.146577in}{2.720000in}}%
\pgfpathlineto{\pgfqpoint{1.146996in}{2.514471in}}%
\pgfpathlineto{\pgfqpoint{1.147305in}{2.766657in}}%
\pgfpathlineto{\pgfqpoint{1.147702in}{2.639799in}}%
\pgfpathlineto{\pgfqpoint{1.147812in}{2.410778in}}%
\pgfpathlineto{\pgfqpoint{1.148164in}{2.743711in}}%
\pgfpathlineto{\pgfqpoint{1.148804in}{2.626250in}}%
\pgfpathlineto{\pgfqpoint{1.148826in}{2.726338in}}%
\pgfpathlineto{\pgfqpoint{1.149244in}{2.389799in}}%
\pgfpathlineto{\pgfqpoint{1.149862in}{2.490542in}}%
\pgfpathlineto{\pgfqpoint{1.149884in}{2.408155in}}%
\pgfpathlineto{\pgfqpoint{1.150876in}{2.785669in}}%
\pgfpathlineto{\pgfqpoint{1.150898in}{2.719891in}}%
\pgfpathlineto{\pgfqpoint{1.152044in}{2.435690in}}%
\pgfpathlineto{\pgfqpoint{1.151052in}{2.721967in}}%
\pgfpathlineto{\pgfqpoint{1.152154in}{2.533593in}}%
\pgfpathlineto{\pgfqpoint{1.152485in}{2.721967in}}%
\pgfpathlineto{\pgfqpoint{1.152992in}{2.438422in}}%
\pgfpathlineto{\pgfqpoint{1.153256in}{2.589100in}}%
\pgfpathlineto{\pgfqpoint{1.153917in}{2.362154in}}%
\pgfpathlineto{\pgfqpoint{1.153697in}{2.676512in}}%
\pgfpathlineto{\pgfqpoint{1.154292in}{2.526927in}}%
\pgfpathlineto{\pgfqpoint{1.154865in}{2.767859in}}%
\pgfpathlineto{\pgfqpoint{1.154579in}{2.479506in}}%
\pgfpathlineto{\pgfqpoint{1.155372in}{2.524087in}}%
\pgfpathlineto{\pgfqpoint{1.155394in}{2.435472in}}%
\pgfpathlineto{\pgfqpoint{1.156188in}{2.797798in}}%
\pgfpathlineto{\pgfqpoint{1.156430in}{2.678479in}}%
\pgfpathlineto{\pgfqpoint{1.156496in}{2.909686in}}%
\pgfpathlineto{\pgfqpoint{1.157224in}{2.532172in}}%
\pgfpathlineto{\pgfqpoint{1.157510in}{2.660560in}}%
\pgfpathlineto{\pgfqpoint{1.158171in}{2.508243in}}%
\pgfpathlineto{\pgfqpoint{1.158216in}{2.759445in}}%
\pgfpathlineto{\pgfqpoint{1.158634in}{2.536871in}}%
\pgfpathlineto{\pgfqpoint{1.158965in}{2.857129in}}%
\pgfpathlineto{\pgfqpoint{1.159759in}{2.705140in}}%
\pgfpathlineto{\pgfqpoint{1.160442in}{2.545503in}}%
\pgfpathlineto{\pgfqpoint{1.160310in}{2.937767in}}%
\pgfpathlineto{\pgfqpoint{1.160861in}{2.652255in}}%
\pgfpathlineto{\pgfqpoint{1.161742in}{2.834183in}}%
\pgfpathlineto{\pgfqpoint{1.161390in}{2.635101in}}%
\pgfpathlineto{\pgfqpoint{1.161985in}{2.699677in}}%
\pgfpathlineto{\pgfqpoint{1.162844in}{2.577408in}}%
\pgfpathlineto{\pgfqpoint{1.162227in}{2.891329in}}%
\pgfpathlineto{\pgfqpoint{1.163087in}{2.717815in}}%
\pgfpathlineto{\pgfqpoint{1.163616in}{2.856473in}}%
\pgfpathlineto{\pgfqpoint{1.163770in}{2.492181in}}%
\pgfpathlineto{\pgfqpoint{1.164167in}{2.678479in}}%
\pgfpathlineto{\pgfqpoint{1.164784in}{2.535013in}}%
\pgfpathlineto{\pgfqpoint{1.164564in}{2.805993in}}%
\pgfpathlineto{\pgfqpoint{1.165247in}{2.715520in}}%
\pgfpathlineto{\pgfqpoint{1.166173in}{2.796049in}}%
\pgfpathlineto{\pgfqpoint{1.165534in}{2.565061in}}%
\pgfpathlineto{\pgfqpoint{1.166283in}{2.736172in}}%
\pgfpathlineto{\pgfqpoint{1.166371in}{2.509554in}}%
\pgfpathlineto{\pgfqpoint{1.167187in}{2.845875in}}%
\pgfpathlineto{\pgfqpoint{1.167385in}{2.744804in}}%
\pgfpathlineto{\pgfqpoint{1.168465in}{2.508680in}}%
\pgfpathlineto{\pgfqpoint{1.167495in}{2.968908in}}%
\pgfpathlineto{\pgfqpoint{1.168553in}{2.575223in}}%
\pgfpathlineto{\pgfqpoint{1.169435in}{2.756167in}}%
\pgfpathlineto{\pgfqpoint{1.169259in}{2.501796in}}%
\pgfpathlineto{\pgfqpoint{1.169655in}{2.646137in}}%
\pgfpathlineto{\pgfqpoint{1.170052in}{2.534685in}}%
\pgfpathlineto{\pgfqpoint{1.169876in}{2.813641in}}%
\pgfpathlineto{\pgfqpoint{1.170603in}{2.690826in}}%
\pgfpathlineto{\pgfqpoint{1.170625in}{2.755075in}}%
\pgfpathlineto{\pgfqpoint{1.171022in}{2.410559in}}%
\pgfpathlineto{\pgfqpoint{1.171617in}{2.414056in}}%
\pgfpathlineto{\pgfqpoint{1.171948in}{2.401053in}}%
\pgfpathlineto{\pgfqpoint{1.171815in}{2.633243in}}%
\pgfpathlineto{\pgfqpoint{1.172102in}{2.499939in}}%
\pgfpathlineto{\pgfqpoint{1.172984in}{2.775617in}}%
\pgfpathlineto{\pgfqpoint{1.172344in}{2.404113in}}%
\pgfpathlineto{\pgfqpoint{1.173226in}{2.651272in}}%
\pgfpathlineto{\pgfqpoint{1.173623in}{2.725901in}}%
\pgfpathlineto{\pgfqpoint{1.174218in}{2.455030in}}%
\pgfpathlineto{\pgfqpoint{1.174262in}{2.651163in}}%
\pgfpathlineto{\pgfqpoint{1.175188in}{2.335166in}}%
\pgfpathlineto{\pgfqpoint{1.174394in}{2.726556in}}%
\pgfpathlineto{\pgfqpoint{1.175364in}{2.562002in}}%
\pgfpathlineto{\pgfqpoint{1.175805in}{2.700005in}}%
\pgfpathlineto{\pgfqpoint{1.176224in}{2.370786in}}%
\pgfpathlineto{\pgfqpoint{1.176466in}{2.606364in}}%
\pgfpathlineto{\pgfqpoint{1.177348in}{2.397884in}}%
\pgfpathlineto{\pgfqpoint{1.177304in}{2.620022in}}%
\pgfpathlineto{\pgfqpoint{1.177613in}{2.536324in}}%
\pgfpathlineto{\pgfqpoint{1.178340in}{2.733659in}}%
\pgfpathlineto{\pgfqpoint{1.177657in}{2.437766in}}%
\pgfpathlineto{\pgfqpoint{1.178715in}{2.650398in}}%
\pgfpathlineto{\pgfqpoint{1.179662in}{2.327080in}}%
\pgfpathlineto{\pgfqpoint{1.179089in}{2.707216in}}%
\pgfpathlineto{\pgfqpoint{1.179839in}{2.557850in}}%
\pgfpathlineto{\pgfqpoint{1.180390in}{2.738903in}}%
\pgfpathlineto{\pgfqpoint{1.180456in}{2.456888in}}%
\pgfpathlineto{\pgfqpoint{1.180941in}{2.609642in}}%
\pgfpathlineto{\pgfqpoint{1.181911in}{2.429025in}}%
\pgfpathlineto{\pgfqpoint{1.181360in}{2.751250in}}%
\pgfpathlineto{\pgfqpoint{1.181999in}{2.636412in}}%
\pgfpathlineto{\pgfqpoint{1.182881in}{2.522010in}}%
\pgfpathlineto{\pgfqpoint{1.183145in}{2.838445in}}%
\pgfpathlineto{\pgfqpoint{1.184071in}{2.530642in}}%
\pgfpathlineto{\pgfqpoint{1.184269in}{2.662417in}}%
\pgfpathlineto{\pgfqpoint{1.184534in}{2.521355in}}%
\pgfpathlineto{\pgfqpoint{1.184446in}{2.723825in}}%
\pgfpathlineto{\pgfqpoint{1.184732in}{2.698803in}}%
\pgfpathlineto{\pgfqpoint{1.184776in}{2.881714in}}%
\pgfpathlineto{\pgfqpoint{1.185768in}{2.536652in}}%
\pgfpathlineto{\pgfqpoint{1.185812in}{2.649852in}}%
\pgfpathlineto{\pgfqpoint{1.185900in}{2.502015in}}%
\pgfpathlineto{\pgfqpoint{1.186650in}{2.827518in}}%
\pgfpathlineto{\pgfqpoint{1.186936in}{2.601993in}}%
\pgfpathlineto{\pgfqpoint{1.187311in}{2.551512in}}%
\pgfpathlineto{\pgfqpoint{1.187487in}{2.842597in}}%
\pgfpathlineto{\pgfqpoint{1.187972in}{2.641766in}}%
\pgfpathlineto{\pgfqpoint{1.188149in}{2.873628in}}%
\pgfpathlineto{\pgfqpoint{1.188303in}{2.604943in}}%
\pgfpathlineto{\pgfqpoint{1.189096in}{2.773322in}}%
\pgfpathlineto{\pgfqpoint{1.189846in}{2.483986in}}%
\pgfpathlineto{\pgfqpoint{1.189229in}{2.802059in}}%
\pgfpathlineto{\pgfqpoint{1.190221in}{2.661871in}}%
\pgfpathlineto{\pgfqpoint{1.191036in}{2.978742in}}%
\pgfpathlineto{\pgfqpoint{1.191301in}{2.559598in}}%
\pgfpathlineto{\pgfqpoint{1.191323in}{2.775726in}}%
\pgfpathlineto{\pgfqpoint{1.191763in}{2.449349in}}%
\pgfpathlineto{\pgfqpoint{1.191411in}{2.915696in}}%
\pgfpathlineto{\pgfqpoint{1.192447in}{2.709074in}}%
\pgfpathlineto{\pgfqpoint{1.193064in}{2.920285in}}%
\pgfpathlineto{\pgfqpoint{1.192601in}{2.589537in}}%
\pgfpathlineto{\pgfqpoint{1.193549in}{2.772994in}}%
\pgfpathlineto{\pgfqpoint{1.193659in}{2.832981in}}%
\pgfpathlineto{\pgfqpoint{1.194695in}{2.550857in}}%
\pgfpathlineto{\pgfqpoint{1.195621in}{2.765455in}}%
\pgfpathlineto{\pgfqpoint{1.195378in}{2.516547in}}%
\pgfpathlineto{\pgfqpoint{1.195819in}{2.651491in}}%
\pgfpathlineto{\pgfqpoint{1.196216in}{2.460057in}}%
\pgfpathlineto{\pgfqpoint{1.196304in}{2.697601in}}%
\pgfpathlineto{\pgfqpoint{1.196921in}{2.600682in}}%
\pgfpathlineto{\pgfqpoint{1.197847in}{2.814734in}}%
\pgfpathlineto{\pgfqpoint{1.197098in}{2.531298in}}%
\pgfpathlineto{\pgfqpoint{1.198045in}{2.658484in}}%
\pgfpathlineto{\pgfqpoint{1.198332in}{2.517858in}}%
\pgfpathlineto{\pgfqpoint{1.199037in}{2.799546in}}%
\pgfpathlineto{\pgfqpoint{1.199148in}{2.669738in}}%
\pgfpathlineto{\pgfqpoint{1.199456in}{2.551512in}}%
\pgfpathlineto{\pgfqpoint{1.199588in}{2.826207in}}%
\pgfpathlineto{\pgfqpoint{1.200007in}{2.639799in}}%
\pgfpathlineto{\pgfqpoint{1.200492in}{2.532500in}}%
\pgfpathlineto{\pgfqpoint{1.201131in}{2.819214in}}%
\pgfpathlineto{\pgfqpoint{1.201462in}{2.517203in}}%
\pgfpathlineto{\pgfqpoint{1.201197in}{2.875049in}}%
\pgfpathlineto{\pgfqpoint{1.202255in}{2.762942in}}%
\pgfpathlineto{\pgfqpoint{1.202498in}{2.818777in}}%
\pgfpathlineto{\pgfqpoint{1.203115in}{2.612701in}}%
\pgfpathlineto{\pgfqpoint{1.203313in}{2.769388in}}%
\pgfpathlineto{\pgfqpoint{1.203556in}{2.653130in}}%
\pgfpathlineto{\pgfqpoint{1.203688in}{2.928480in}}%
\pgfpathlineto{\pgfqpoint{1.204394in}{2.828174in}}%
\pgfpathlineto{\pgfqpoint{1.204416in}{2.853086in}}%
\pgfpathlineto{\pgfqpoint{1.204967in}{2.638706in}}%
\pgfpathlineto{\pgfqpoint{1.205341in}{2.651928in}}%
\pgfpathlineto{\pgfqpoint{1.205363in}{2.652692in}}%
\pgfpathlineto{\pgfqpoint{1.206157in}{2.916679in}}%
\pgfpathlineto{\pgfqpoint{1.205694in}{2.537745in}}%
\pgfpathlineto{\pgfqpoint{1.206488in}{2.704157in}}%
\pgfpathlineto{\pgfqpoint{1.206686in}{2.853960in}}%
\pgfpathlineto{\pgfqpoint{1.207413in}{2.514253in}}%
\pgfpathlineto{\pgfqpoint{1.207524in}{2.646137in}}%
\pgfpathlineto{\pgfqpoint{1.207876in}{2.524742in}}%
\pgfpathlineto{\pgfqpoint{1.208449in}{2.738357in}}%
\pgfpathlineto{\pgfqpoint{1.208626in}{2.656735in}}%
\pgfpathlineto{\pgfqpoint{1.209750in}{2.828938in}}%
\pgfpathlineto{\pgfqpoint{1.209133in}{2.565826in}}%
\pgfpathlineto{\pgfqpoint{1.209772in}{2.803370in}}%
\pgfpathlineto{\pgfqpoint{1.210169in}{2.513378in}}%
\pgfpathlineto{\pgfqpoint{1.209926in}{2.867400in}}%
\pgfpathlineto{\pgfqpoint{1.210896in}{2.641220in}}%
\pgfpathlineto{\pgfqpoint{1.211050in}{2.774633in}}%
\pgfpathlineto{\pgfqpoint{1.211601in}{2.450878in}}%
\pgfpathlineto{\pgfqpoint{1.211954in}{2.607566in}}%
\pgfpathlineto{\pgfqpoint{1.212152in}{2.451971in}}%
\pgfpathlineto{\pgfqpoint{1.212792in}{2.791897in}}%
\pgfpathlineto{\pgfqpoint{1.213056in}{2.658484in}}%
\pgfpathlineto{\pgfqpoint{1.213453in}{2.541351in}}%
\pgfpathlineto{\pgfqpoint{1.213894in}{2.853086in}}%
\pgfpathlineto{\pgfqpoint{1.214114in}{2.729834in}}%
\pgfpathlineto{\pgfqpoint{1.214180in}{2.645809in}}%
\pgfpathlineto{\pgfqpoint{1.214224in}{2.799218in}}%
\pgfpathlineto{\pgfqpoint{1.214445in}{2.709839in}}%
\pgfpathlineto{\pgfqpoint{1.214886in}{2.544191in}}%
\pgfpathlineto{\pgfqpoint{1.215370in}{2.798235in}}%
\pgfpathlineto{\pgfqpoint{1.215569in}{2.573475in}}%
\pgfpathlineto{\pgfqpoint{1.216274in}{2.801076in}}%
\pgfpathlineto{\pgfqpoint{1.216142in}{2.513488in}}%
\pgfpathlineto{\pgfqpoint{1.216693in}{2.633025in}}%
\pgfpathlineto{\pgfqpoint{1.216715in}{2.469235in}}%
\pgfpathlineto{\pgfqpoint{1.217134in}{2.825442in}}%
\pgfpathlineto{\pgfqpoint{1.217795in}{2.606254in}}%
\pgfpathlineto{\pgfqpoint{1.218192in}{2.877562in}}%
\pgfpathlineto{\pgfqpoint{1.218302in}{2.514690in}}%
\pgfpathlineto{\pgfqpoint{1.218875in}{2.573584in}}%
\pgfpathlineto{\pgfqpoint{1.219206in}{2.435472in}}%
\pgfpathlineto{\pgfqpoint{1.219492in}{2.705577in}}%
\pgfpathlineto{\pgfqpoint{1.219845in}{2.579812in}}%
\pgfpathlineto{\pgfqpoint{1.220220in}{2.907610in}}%
\pgfpathlineto{\pgfqpoint{1.220286in}{2.421377in}}%
\pgfpathlineto{\pgfqpoint{1.220969in}{2.771027in}}%
\pgfpathlineto{\pgfqpoint{1.221454in}{2.477758in}}%
\pgfpathlineto{\pgfqpoint{1.221256in}{2.822055in}}%
\pgfpathlineto{\pgfqpoint{1.222093in}{2.668536in}}%
\pgfpathlineto{\pgfqpoint{1.222424in}{2.809926in}}%
\pgfpathlineto{\pgfqpoint{1.222666in}{2.577408in}}%
\pgfpathlineto{\pgfqpoint{1.223217in}{2.767094in}}%
\pgfpathlineto{\pgfqpoint{1.224209in}{2.593798in}}%
\pgfpathlineto{\pgfqpoint{1.223945in}{2.887942in}}%
\pgfpathlineto{\pgfqpoint{1.224297in}{2.779222in}}%
\pgfpathlineto{\pgfqpoint{1.224959in}{2.852758in}}%
\pgfpathlineto{\pgfqpoint{1.224474in}{2.557631in}}%
\pgfpathlineto{\pgfqpoint{1.225355in}{2.739340in}}%
\pgfpathlineto{\pgfqpoint{1.226303in}{2.541132in}}%
\pgfpathlineto{\pgfqpoint{1.226149in}{2.821399in}}%
\pgfpathlineto{\pgfqpoint{1.226458in}{2.704485in}}%
\pgfpathlineto{\pgfqpoint{1.227185in}{2.806211in}}%
\pgfpathlineto{\pgfqpoint{1.227009in}{2.470000in}}%
\pgfpathlineto{\pgfqpoint{1.227494in}{2.636084in}}%
\pgfpathlineto{\pgfqpoint{1.227956in}{2.475791in}}%
\pgfpathlineto{\pgfqpoint{1.228265in}{2.815280in}}%
\pgfpathlineto{\pgfqpoint{1.228596in}{2.621442in}}%
\pgfpathlineto{\pgfqpoint{1.229367in}{2.424764in}}%
\pgfpathlineto{\pgfqpoint{1.228816in}{2.817684in}}%
\pgfpathlineto{\pgfqpoint{1.229676in}{2.640018in}}%
\pgfpathlineto{\pgfqpoint{1.229698in}{2.639471in}}%
\pgfpathlineto{\pgfqpoint{1.229940in}{2.783921in}}%
\pgfpathlineto{\pgfqpoint{1.230293in}{2.506058in}}%
\pgfpathlineto{\pgfqpoint{1.230778in}{2.665367in}}%
\pgfpathlineto{\pgfqpoint{1.230844in}{2.467268in}}%
\pgfpathlineto{\pgfqpoint{1.231351in}{2.797688in}}%
\pgfpathlineto{\pgfqpoint{1.231880in}{2.653567in}}%
\pgfpathlineto{\pgfqpoint{1.232519in}{2.532937in}}%
\pgfpathlineto{\pgfqpoint{1.232299in}{2.797470in}}%
\pgfpathlineto{\pgfqpoint{1.232982in}{2.649196in}}%
\pgfpathlineto{\pgfqpoint{1.233335in}{2.886303in}}%
\pgfpathlineto{\pgfqpoint{1.233136in}{2.573038in}}%
\pgfpathlineto{\pgfqpoint{1.234084in}{2.661871in}}%
\pgfpathlineto{\pgfqpoint{1.234172in}{2.652583in}}%
\pgfpathlineto{\pgfqpoint{1.234194in}{2.752999in}}%
\pgfpathlineto{\pgfqpoint{1.235032in}{2.872426in}}%
\pgfpathlineto{\pgfqpoint{1.234547in}{2.630512in}}%
\pgfpathlineto{\pgfqpoint{1.235318in}{2.844782in}}%
\pgfpathlineto{\pgfqpoint{1.236332in}{2.576862in}}%
\pgfpathlineto{\pgfqpoint{1.236487in}{2.660887in}}%
\pgfpathlineto{\pgfqpoint{1.237016in}{2.838007in}}%
\pgfpathlineto{\pgfqpoint{1.236663in}{2.600354in}}%
\pgfpathlineto{\pgfqpoint{1.237589in}{2.722404in}}%
\pgfpathlineto{\pgfqpoint{1.238074in}{2.545721in}}%
\pgfpathlineto{\pgfqpoint{1.237699in}{2.812439in}}%
\pgfpathlineto{\pgfqpoint{1.238713in}{2.672907in}}%
\pgfpathlineto{\pgfqpoint{1.238867in}{2.802824in}}%
\pgfpathlineto{\pgfqpoint{1.238757in}{2.569541in}}%
\pgfpathlineto{\pgfqpoint{1.239837in}{2.729834in}}%
\pgfpathlineto{\pgfqpoint{1.240344in}{2.848169in}}%
\pgfpathlineto{\pgfqpoint{1.240432in}{2.643842in}}%
\pgfpathlineto{\pgfqpoint{1.240719in}{2.730053in}}%
\pgfpathlineto{\pgfqpoint{1.240741in}{2.504309in}}%
\pgfpathlineto{\pgfqpoint{1.241755in}{2.854288in}}%
\pgfpathlineto{\pgfqpoint{1.241799in}{2.655096in}}%
\pgfpathlineto{\pgfqpoint{1.242879in}{2.846421in}}%
\pgfpathlineto{\pgfqpoint{1.242085in}{2.577080in}}%
\pgfpathlineto{\pgfqpoint{1.242901in}{2.748956in}}%
\pgfpathlineto{\pgfqpoint{1.243342in}{2.640673in}}%
\pgfpathlineto{\pgfqpoint{1.243165in}{2.842706in}}%
\pgfpathlineto{\pgfqpoint{1.243849in}{2.684380in}}%
\pgfpathlineto{\pgfqpoint{1.244047in}{2.902037in}}%
\pgfpathlineto{\pgfqpoint{1.244708in}{2.577299in}}%
\pgfpathlineto{\pgfqpoint{1.244951in}{2.690389in}}%
\pgfpathlineto{\pgfqpoint{1.245105in}{2.544956in}}%
\pgfpathlineto{\pgfqpoint{1.246075in}{2.793536in}}%
\pgfpathlineto{\pgfqpoint{1.246472in}{2.595000in}}%
\pgfpathlineto{\pgfqpoint{1.246736in}{2.899415in}}%
\pgfpathlineto{\pgfqpoint{1.247199in}{2.663619in}}%
\pgfpathlineto{\pgfqpoint{1.247904in}{2.947492in}}%
\pgfpathlineto{\pgfqpoint{1.248169in}{2.546158in}}%
\pgfpathlineto{\pgfqpoint{1.248367in}{2.890674in}}%
\pgfpathlineto{\pgfqpoint{1.248411in}{2.901382in}}%
\pgfpathlineto{\pgfqpoint{1.249558in}{2.494257in}}%
\pgfpathlineto{\pgfqpoint{1.250572in}{2.840411in}}%
\pgfpathlineto{\pgfqpoint{1.250682in}{2.712679in}}%
\pgfpathlineto{\pgfqpoint{1.250968in}{2.497863in}}%
\pgfpathlineto{\pgfqpoint{1.251674in}{2.880621in}}%
\pgfpathlineto{\pgfqpoint{1.251762in}{2.738138in}}%
\pgfpathlineto{\pgfqpoint{1.252225in}{2.904660in}}%
\pgfpathlineto{\pgfqpoint{1.252688in}{2.572382in}}%
\pgfpathlineto{\pgfqpoint{1.252864in}{2.711368in}}%
\pgfpathlineto{\pgfqpoint{1.253128in}{2.511849in}}%
\pgfpathlineto{\pgfqpoint{1.253657in}{2.827736in}}%
\pgfpathlineto{\pgfqpoint{1.253966in}{2.686783in}}%
\pgfpathlineto{\pgfqpoint{1.254076in}{2.821071in}}%
\pgfpathlineto{\pgfqpoint{1.254561in}{2.533811in}}%
\pgfpathlineto{\pgfqpoint{1.254715in}{2.604834in}}%
\pgfpathlineto{\pgfqpoint{1.254737in}{2.499939in}}%
\pgfpathlineto{\pgfqpoint{1.255707in}{2.815499in}}%
\pgfpathlineto{\pgfqpoint{1.255751in}{2.746443in}}%
\pgfpathlineto{\pgfqpoint{1.256677in}{2.917007in}}%
\pgfpathlineto{\pgfqpoint{1.255928in}{2.588881in}}%
\pgfpathlineto{\pgfqpoint{1.256831in}{2.713335in}}%
\pgfpathlineto{\pgfqpoint{1.257471in}{2.513816in}}%
\pgfpathlineto{\pgfqpoint{1.257316in}{2.789056in}}%
\pgfpathlineto{\pgfqpoint{1.257934in}{2.735516in}}%
\pgfpathlineto{\pgfqpoint{1.258154in}{2.838445in}}%
\pgfpathlineto{\pgfqpoint{1.258264in}{2.482893in}}%
\pgfpathlineto{\pgfqpoint{1.258992in}{2.586587in}}%
\pgfpathlineto{\pgfqpoint{1.260138in}{2.837243in}}%
\pgfpathlineto{\pgfqpoint{1.259366in}{2.467268in}}%
\pgfpathlineto{\pgfqpoint{1.260226in}{2.707763in}}%
\pgfpathlineto{\pgfqpoint{1.260645in}{2.592268in}}%
\pgfpathlineto{\pgfqpoint{1.260424in}{2.811347in}}%
\pgfpathlineto{\pgfqpoint{1.261350in}{2.650398in}}%
\pgfpathlineto{\pgfqpoint{1.262166in}{2.800529in}}%
\pgfpathlineto{\pgfqpoint{1.262386in}{2.480489in}}%
\pgfpathlineto{\pgfqpoint{1.262408in}{2.541460in}}%
\pgfpathlineto{\pgfqpoint{1.263246in}{2.452190in}}%
\pgfpathlineto{\pgfqpoint{1.262805in}{2.756167in}}%
\pgfpathlineto{\pgfqpoint{1.263510in}{2.462788in}}%
\pgfpathlineto{\pgfqpoint{1.264105in}{2.722623in}}%
\pgfpathlineto{\pgfqpoint{1.264656in}{2.549545in}}%
\pgfpathlineto{\pgfqpoint{1.265318in}{2.841613in}}%
\pgfpathlineto{\pgfqpoint{1.264833in}{2.479397in}}%
\pgfpathlineto{\pgfqpoint{1.265847in}{2.676075in}}%
\pgfpathlineto{\pgfqpoint{1.266684in}{2.548562in}}%
\pgfpathlineto{\pgfqpoint{1.266199in}{2.834948in}}%
\pgfpathlineto{\pgfqpoint{1.266949in}{2.686346in}}%
\pgfpathlineto{\pgfqpoint{1.268029in}{2.834292in}}%
\pgfpathlineto{\pgfqpoint{1.267279in}{2.540039in}}%
\pgfpathlineto{\pgfqpoint{1.268073in}{2.765783in}}%
\pgfpathlineto{\pgfqpoint{1.268800in}{2.478850in}}%
\pgfpathlineto{\pgfqpoint{1.268271in}{2.801731in}}%
\pgfpathlineto{\pgfqpoint{1.269197in}{2.604615in}}%
\pgfpathlineto{\pgfqpoint{1.269506in}{2.850136in}}%
\pgfpathlineto{\pgfqpoint{1.270035in}{2.478086in}}%
\pgfpathlineto{\pgfqpoint{1.270255in}{2.565389in}}%
\pgfpathlineto{\pgfqpoint{1.270850in}{2.440607in}}%
\pgfpathlineto{\pgfqpoint{1.271313in}{2.732238in}}%
\pgfpathlineto{\pgfqpoint{1.271335in}{2.659904in}}%
\pgfpathlineto{\pgfqpoint{1.271357in}{2.661543in}}%
\pgfpathlineto{\pgfqpoint{1.271379in}{2.651600in}}%
\pgfpathlineto{\pgfqpoint{1.271864in}{2.474698in}}%
\pgfpathlineto{\pgfqpoint{1.272085in}{2.762395in}}%
\pgfpathlineto{\pgfqpoint{1.272481in}{2.682194in}}%
\pgfpathlineto{\pgfqpoint{1.272966in}{2.525616in}}%
\pgfpathlineto{\pgfqpoint{1.273143in}{2.810363in}}%
\pgfpathlineto{\pgfqpoint{1.273583in}{2.582107in}}%
\pgfpathlineto{\pgfqpoint{1.274333in}{2.462570in}}%
\pgfpathlineto{\pgfqpoint{1.274730in}{2.814515in}}%
\pgfpathlineto{\pgfqpoint{1.275303in}{2.535341in}}%
\pgfpathlineto{\pgfqpoint{1.275082in}{2.879529in}}%
\pgfpathlineto{\pgfqpoint{1.275832in}{2.657937in}}%
\pgfpathlineto{\pgfqpoint{1.276140in}{2.849371in}}%
\pgfpathlineto{\pgfqpoint{1.276405in}{2.581560in}}%
\pgfpathlineto{\pgfqpoint{1.276912in}{2.706124in}}%
\pgfpathlineto{\pgfqpoint{1.276934in}{2.598387in}}%
\pgfpathlineto{\pgfqpoint{1.277529in}{2.834839in}}%
\pgfpathlineto{\pgfqpoint{1.278014in}{2.699458in}}%
\pgfpathlineto{\pgfqpoint{1.278719in}{2.594344in}}%
\pgfpathlineto{\pgfqpoint{1.278499in}{2.864013in}}%
\pgfpathlineto{\pgfqpoint{1.279116in}{2.671923in}}%
\pgfpathlineto{\pgfqpoint{1.279138in}{2.809052in}}%
\pgfpathlineto{\pgfqpoint{1.279689in}{2.443448in}}%
\pgfpathlineto{\pgfqpoint{1.280218in}{2.766001in}}%
\pgfpathlineto{\pgfqpoint{1.280571in}{2.489777in}}%
\pgfpathlineto{\pgfqpoint{1.280945in}{2.845328in}}%
\pgfpathlineto{\pgfqpoint{1.281320in}{2.756386in}}%
\pgfpathlineto{\pgfqpoint{1.282400in}{2.556429in}}%
\pgfpathlineto{\pgfqpoint{1.281408in}{2.806757in}}%
\pgfpathlineto{\pgfqpoint{1.282444in}{2.729616in}}%
\pgfpathlineto{\pgfqpoint{1.282466in}{2.818012in}}%
\pgfpathlineto{\pgfqpoint{1.283017in}{2.576534in}}%
\pgfpathlineto{\pgfqpoint{1.283524in}{2.609860in}}%
\pgfpathlineto{\pgfqpoint{1.284516in}{2.912090in}}%
\pgfpathlineto{\pgfqpoint{1.284626in}{2.677605in}}%
\pgfpathlineto{\pgfqpoint{1.284670in}{2.599152in}}%
\pgfpathlineto{\pgfqpoint{1.285199in}{2.896574in}}%
\pgfpathlineto{\pgfqpoint{1.285684in}{2.790586in}}%
\pgfpathlineto{\pgfqpoint{1.285773in}{2.863466in}}%
\pgfpathlineto{\pgfqpoint{1.285728in}{2.766329in}}%
\pgfpathlineto{\pgfqpoint{1.285905in}{2.775726in}}%
\pgfpathlineto{\pgfqpoint{1.286831in}{2.370677in}}%
\pgfpathlineto{\pgfqpoint{1.286037in}{2.845656in}}%
\pgfpathlineto{\pgfqpoint{1.287073in}{2.598715in}}%
\pgfpathlineto{\pgfqpoint{1.288065in}{2.257041in}}%
\pgfpathlineto{\pgfqpoint{1.288241in}{2.511630in}}%
\pgfpathlineto{\pgfqpoint{1.289167in}{2.592815in}}%
\pgfpathlineto{\pgfqpoint{1.288858in}{2.298343in}}%
\pgfpathlineto{\pgfqpoint{1.289277in}{2.319650in}}%
\pgfpathlineto{\pgfqpoint{1.290049in}{2.124173in}}%
\pgfpathlineto{\pgfqpoint{1.289432in}{2.462023in}}%
\pgfpathlineto{\pgfqpoint{1.290423in}{2.213006in}}%
\pgfpathlineto{\pgfqpoint{1.291217in}{2.367946in}}%
\pgfpathlineto{\pgfqpoint{1.291459in}{2.063203in}}%
\pgfpathlineto{\pgfqpoint{1.291526in}{2.213771in}}%
\pgfpathlineto{\pgfqpoint{1.291988in}{2.439405in}}%
\pgfpathlineto{\pgfqpoint{1.291702in}{2.128762in}}%
\pgfpathlineto{\pgfqpoint{1.292694in}{2.355926in}}%
\pgfpathlineto{\pgfqpoint{1.293487in}{2.050200in}}%
\pgfpathlineto{\pgfqpoint{1.293135in}{2.427058in}}%
\pgfpathlineto{\pgfqpoint{1.293840in}{2.204156in}}%
\pgfpathlineto{\pgfqpoint{1.294149in}{2.435144in}}%
\pgfpathlineto{\pgfqpoint{1.294545in}{2.106691in}}%
\pgfpathlineto{\pgfqpoint{1.294678in}{2.161433in}}%
\pgfpathlineto{\pgfqpoint{1.295251in}{1.937110in}}%
\pgfpathlineto{\pgfqpoint{1.294964in}{2.351665in}}%
\pgfpathlineto{\pgfqpoint{1.295802in}{2.025288in}}%
\pgfpathlineto{\pgfqpoint{1.296066in}{2.404987in}}%
\pgfpathlineto{\pgfqpoint{1.296926in}{2.202080in}}%
\pgfpathlineto{\pgfqpoint{1.297080in}{1.940825in}}%
\pgfpathlineto{\pgfqpoint{1.297301in}{2.228304in}}%
\pgfpathlineto{\pgfqpoint{1.298050in}{2.091831in}}%
\pgfpathlineto{\pgfqpoint{1.299020in}{1.921266in}}%
\pgfpathlineto{\pgfqpoint{1.299174in}{2.181319in}}%
\pgfpathlineto{\pgfqpoint{1.299328in}{1.803915in}}%
\pgfpathlineto{\pgfqpoint{1.300320in}{1.930226in}}%
\pgfpathlineto{\pgfqpoint{1.300386in}{2.108439in}}%
\pgfpathlineto{\pgfqpoint{1.300497in}{1.864885in}}%
\pgfpathlineto{\pgfqpoint{1.301444in}{2.002888in}}%
\pgfpathlineto{\pgfqpoint{1.301466in}{2.004199in}}%
\pgfpathlineto{\pgfqpoint{1.301511in}{1.973496in}}%
\pgfpathlineto{\pgfqpoint{1.301973in}{2.205467in}}%
\pgfpathlineto{\pgfqpoint{1.302062in}{1.855488in}}%
\pgfpathlineto{\pgfqpoint{1.302613in}{2.035559in}}%
\pgfpathlineto{\pgfqpoint{1.303516in}{1.864230in}}%
\pgfpathlineto{\pgfqpoint{1.303649in}{2.165039in}}%
\pgfpathlineto{\pgfqpoint{1.303715in}{1.921376in}}%
\pgfpathlineto{\pgfqpoint{1.303957in}{2.133679in}}%
\pgfpathlineto{\pgfqpoint{1.304773in}{1.785449in}}%
\pgfpathlineto{\pgfqpoint{1.304817in}{1.884990in}}%
\pgfpathlineto{\pgfqpoint{1.305522in}{1.708963in}}%
\pgfpathlineto{\pgfqpoint{1.305302in}{2.069868in}}%
\pgfpathlineto{\pgfqpoint{1.305919in}{1.898211in}}%
\pgfpathlineto{\pgfqpoint{1.306206in}{1.678041in}}%
\pgfpathlineto{\pgfqpoint{1.306492in}{1.982892in}}%
\pgfpathlineto{\pgfqpoint{1.306977in}{1.873954in}}%
\pgfpathlineto{\pgfqpoint{1.307374in}{2.007805in}}%
\pgfpathlineto{\pgfqpoint{1.307572in}{1.706996in}}%
\pgfpathlineto{\pgfqpoint{1.308057in}{1.805882in}}%
\pgfpathlineto{\pgfqpoint{1.308961in}{1.681318in}}%
\pgfpathlineto{\pgfqpoint{1.308277in}{1.995567in}}%
\pgfpathlineto{\pgfqpoint{1.309159in}{1.779658in}}%
\pgfpathlineto{\pgfqpoint{1.309291in}{1.673451in}}%
\pgfpathlineto{\pgfqpoint{1.310283in}{1.992289in}}%
\pgfpathlineto{\pgfqpoint{1.311187in}{1.745239in}}%
\pgfpathlineto{\pgfqpoint{1.310878in}{2.059488in}}%
\pgfpathlineto{\pgfqpoint{1.311385in}{2.000921in}}%
\pgfpathlineto{\pgfqpoint{1.311407in}{2.037416in}}%
\pgfpathlineto{\pgfqpoint{1.312091in}{1.681756in}}%
\pgfpathlineto{\pgfqpoint{1.312311in}{1.704374in}}%
\pgfpathlineto{\pgfqpoint{1.312377in}{1.590409in}}%
\pgfpathlineto{\pgfqpoint{1.312664in}{1.858548in}}%
\pgfpathlineto{\pgfqpoint{1.313237in}{1.752560in}}%
\pgfpathlineto{\pgfqpoint{1.313259in}{1.869693in}}%
\pgfpathlineto{\pgfqpoint{1.314052in}{1.528018in}}%
\pgfpathlineto{\pgfqpoint{1.314317in}{1.595108in}}%
\pgfpathlineto{\pgfqpoint{1.314956in}{1.798670in}}%
\pgfpathlineto{\pgfqpoint{1.314802in}{1.543097in}}%
\pgfpathlineto{\pgfqpoint{1.315441in}{1.701751in}}%
\pgfpathlineto{\pgfqpoint{1.315529in}{1.624391in}}%
\pgfpathlineto{\pgfqpoint{1.316014in}{1.931975in}}%
\pgfpathlineto{\pgfqpoint{1.316521in}{1.746222in}}%
\pgfpathlineto{\pgfqpoint{1.317557in}{1.930226in}}%
\pgfpathlineto{\pgfqpoint{1.316720in}{1.642311in}}%
\pgfpathlineto{\pgfqpoint{1.317601in}{1.884553in}}%
\pgfpathlineto{\pgfqpoint{1.317645in}{1.648102in}}%
\pgfpathlineto{\pgfqpoint{1.318218in}{1.957980in}}%
\pgfpathlineto{\pgfqpoint{1.318703in}{1.816917in}}%
\pgfpathlineto{\pgfqpoint{1.319409in}{1.976336in}}%
\pgfpathlineto{\pgfqpoint{1.319673in}{1.677822in}}%
\pgfpathlineto{\pgfqpoint{1.319783in}{1.776926in}}%
\pgfpathlineto{\pgfqpoint{1.320445in}{1.627232in}}%
\pgfpathlineto{\pgfqpoint{1.320048in}{1.955685in}}%
\pgfpathlineto{\pgfqpoint{1.320908in}{1.733111in}}%
\pgfpathlineto{\pgfqpoint{1.321194in}{1.938640in}}%
\pgfpathlineto{\pgfqpoint{1.320952in}{1.629964in}}%
\pgfpathlineto{\pgfqpoint{1.322010in}{1.780204in}}%
\pgfpathlineto{\pgfqpoint{1.322230in}{1.921485in}}%
\pgfpathlineto{\pgfqpoint{1.322935in}{1.696069in}}%
\pgfpathlineto{\pgfqpoint{1.323134in}{1.813202in}}%
\pgfpathlineto{\pgfqpoint{1.323178in}{1.856253in}}%
\pgfpathlineto{\pgfqpoint{1.323310in}{1.641873in}}%
\pgfpathlineto{\pgfqpoint{1.323354in}{1.651598in}}%
\pgfpathlineto{\pgfqpoint{1.323398in}{1.578609in}}%
\pgfpathlineto{\pgfqpoint{1.324214in}{1.886301in}}%
\pgfpathlineto{\pgfqpoint{1.324302in}{1.799107in}}%
\pgfpathlineto{\pgfqpoint{1.325029in}{1.891765in}}%
\pgfpathlineto{\pgfqpoint{1.324611in}{1.502887in}}%
\pgfpathlineto{\pgfqpoint{1.325382in}{1.761957in}}%
\pgfpathlineto{\pgfqpoint{1.326220in}{1.539601in}}%
\pgfpathlineto{\pgfqpoint{1.325492in}{1.837459in}}%
\pgfpathlineto{\pgfqpoint{1.326484in}{1.720873in}}%
\pgfpathlineto{\pgfqpoint{1.326683in}{1.668097in}}%
\pgfpathlineto{\pgfqpoint{1.326859in}{1.878981in}}%
\pgfpathlineto{\pgfqpoint{1.326881in}{1.968797in}}%
\pgfpathlineto{\pgfqpoint{1.327410in}{1.618600in}}%
\pgfpathlineto{\pgfqpoint{1.327939in}{1.754964in}}%
\pgfpathlineto{\pgfqpoint{1.328622in}{1.594234in}}%
\pgfpathlineto{\pgfqpoint{1.328490in}{1.929024in}}%
\pgfpathlineto{\pgfqpoint{1.328975in}{1.817682in}}%
\pgfpathlineto{\pgfqpoint{1.329372in}{1.625921in}}%
\pgfpathlineto{\pgfqpoint{1.329217in}{1.825877in}}%
\pgfpathlineto{\pgfqpoint{1.329438in}{1.775724in}}%
\pgfpathlineto{\pgfqpoint{1.329879in}{1.519386in}}%
\pgfpathlineto{\pgfqpoint{1.330231in}{1.878216in}}%
\pgfpathlineto{\pgfqpoint{1.330562in}{1.624609in}}%
\pgfpathlineto{\pgfqpoint{1.331400in}{1.925528in}}%
\pgfpathlineto{\pgfqpoint{1.330628in}{1.583088in}}%
\pgfpathlineto{\pgfqpoint{1.331752in}{1.739994in}}%
\pgfpathlineto{\pgfqpoint{1.332083in}{1.547358in}}%
\pgfpathlineto{\pgfqpoint{1.332347in}{1.945523in}}%
\pgfpathlineto{\pgfqpoint{1.332854in}{1.740868in}}%
\pgfpathlineto{\pgfqpoint{1.333956in}{1.553587in}}%
\pgfpathlineto{\pgfqpoint{1.333317in}{1.827953in}}%
\pgfpathlineto{\pgfqpoint{1.333978in}{1.695851in}}%
\pgfpathlineto{\pgfqpoint{1.334133in}{1.807302in}}%
\pgfpathlineto{\pgfqpoint{1.334331in}{1.526052in}}%
\pgfpathlineto{\pgfqpoint{1.335058in}{1.663836in}}%
\pgfpathlineto{\pgfqpoint{1.335720in}{1.401379in}}%
\pgfpathlineto{\pgfqpoint{1.335125in}{1.778237in}}%
\pgfpathlineto{\pgfqpoint{1.336205in}{1.449784in}}%
\pgfpathlineto{\pgfqpoint{1.336734in}{1.821944in}}%
\pgfpathlineto{\pgfqpoint{1.337329in}{1.595217in}}%
\pgfpathlineto{\pgfqpoint{1.338343in}{1.792223in}}%
\pgfpathlineto{\pgfqpoint{1.337990in}{1.542769in}}%
\pgfpathlineto{\pgfqpoint{1.338387in}{1.752014in}}%
\pgfpathlineto{\pgfqpoint{1.339070in}{1.426183in}}%
\pgfpathlineto{\pgfqpoint{1.338519in}{1.813858in}}%
\pgfpathlineto{\pgfqpoint{1.339511in}{1.543097in}}%
\pgfpathlineto{\pgfqpoint{1.339996in}{1.734859in}}%
\pgfpathlineto{\pgfqpoint{1.340084in}{1.422030in}}%
\pgfpathlineto{\pgfqpoint{1.340635in}{1.601773in}}%
\pgfpathlineto{\pgfqpoint{1.341164in}{1.380072in}}%
\pgfpathlineto{\pgfqpoint{1.341517in}{1.671266in}}%
\pgfpathlineto{\pgfqpoint{1.341737in}{1.547796in}}%
\pgfpathlineto{\pgfqpoint{1.342398in}{1.719671in}}%
\pgfpathlineto{\pgfqpoint{1.341936in}{1.382367in}}%
\pgfpathlineto{\pgfqpoint{1.342817in}{1.583307in}}%
\pgfpathlineto{\pgfqpoint{1.343148in}{1.367288in}}%
\pgfpathlineto{\pgfqpoint{1.343787in}{1.740431in}}%
\pgfpathlineto{\pgfqpoint{1.343897in}{1.534465in}}%
\pgfpathlineto{\pgfqpoint{1.344008in}{1.805991in}}%
\pgfpathlineto{\pgfqpoint{1.344801in}{1.482782in}}%
\pgfpathlineto{\pgfqpoint{1.345021in}{1.609749in}}%
\pgfpathlineto{\pgfqpoint{1.345220in}{1.755292in}}%
\pgfpathlineto{\pgfqpoint{1.345837in}{1.419080in}}%
\pgfpathlineto{\pgfqpoint{1.346102in}{1.588224in}}%
\pgfpathlineto{\pgfqpoint{1.346124in}{1.418862in}}%
\pgfpathlineto{\pgfqpoint{1.346278in}{1.693119in}}%
\pgfpathlineto{\pgfqpoint{1.347204in}{1.505291in}}%
\pgfpathlineto{\pgfqpoint{1.348328in}{1.702079in}}%
\pgfpathlineto{\pgfqpoint{1.347402in}{1.367288in}}%
\pgfpathlineto{\pgfqpoint{1.348350in}{1.641218in}}%
\pgfpathlineto{\pgfqpoint{1.349143in}{1.352210in}}%
\pgfpathlineto{\pgfqpoint{1.348945in}{1.658700in}}%
\pgfpathlineto{\pgfqpoint{1.349474in}{1.551074in}}%
\pgfpathlineto{\pgfqpoint{1.349496in}{1.555116in}}%
\pgfpathlineto{\pgfqpoint{1.349606in}{1.416676in}}%
\pgfpathlineto{\pgfqpoint{1.349628in}{1.504526in}}%
\pgfpathlineto{\pgfqpoint{1.349650in}{1.293534in}}%
\pgfpathlineto{\pgfqpoint{1.350664in}{1.683613in}}%
\pgfpathlineto{\pgfqpoint{1.350730in}{1.547249in}}%
\pgfpathlineto{\pgfqpoint{1.350995in}{1.331121in}}%
\pgfpathlineto{\pgfqpoint{1.350818in}{1.616961in}}%
\pgfpathlineto{\pgfqpoint{1.351854in}{1.459181in}}%
\pgfpathlineto{\pgfqpoint{1.352472in}{1.598604in}}%
\pgfpathlineto{\pgfqpoint{1.352714in}{1.294080in}}%
\pgfpathlineto{\pgfqpoint{1.352957in}{1.441917in}}%
\pgfpathlineto{\pgfqpoint{1.353772in}{1.341174in}}%
\pgfpathlineto{\pgfqpoint{1.353111in}{1.567136in}}%
\pgfpathlineto{\pgfqpoint{1.354015in}{1.537743in}}%
\pgfpathlineto{\pgfqpoint{1.354125in}{1.648539in}}%
\pgfpathlineto{\pgfqpoint{1.354786in}{1.333853in}}%
\pgfpathlineto{\pgfqpoint{1.354852in}{1.370566in}}%
\pgfpathlineto{\pgfqpoint{1.355029in}{1.320522in}}%
\pgfpathlineto{\pgfqpoint{1.355558in}{1.610296in}}%
\pgfpathlineto{\pgfqpoint{1.355866in}{1.377013in}}%
\pgfpathlineto{\pgfqpoint{1.356483in}{1.654658in}}%
\pgfpathlineto{\pgfqpoint{1.356660in}{1.250046in}}%
\pgfpathlineto{\pgfqpoint{1.356968in}{1.452953in}}%
\pgfpathlineto{\pgfqpoint{1.357145in}{1.319211in}}%
\pgfpathlineto{\pgfqpoint{1.357321in}{1.559596in}}%
\pgfpathlineto{\pgfqpoint{1.358048in}{1.470435in}}%
\pgfpathlineto{\pgfqpoint{1.358291in}{1.562000in}}%
\pgfpathlineto{\pgfqpoint{1.358732in}{1.252122in}}%
\pgfpathlineto{\pgfqpoint{1.359106in}{1.361606in}}%
\pgfpathlineto{\pgfqpoint{1.360120in}{1.242616in}}%
\pgfpathlineto{\pgfqpoint{1.359657in}{1.564295in}}%
\pgfpathlineto{\pgfqpoint{1.360186in}{1.377122in}}%
\pgfpathlineto{\pgfqpoint{1.360936in}{1.588770in}}%
\pgfpathlineto{\pgfqpoint{1.360473in}{1.297467in}}%
\pgfpathlineto{\pgfqpoint{1.361310in}{1.516108in}}%
\pgfpathlineto{\pgfqpoint{1.362148in}{1.277690in}}%
\pgfpathlineto{\pgfqpoint{1.361994in}{1.707542in}}%
\pgfpathlineto{\pgfqpoint{1.362435in}{1.363355in}}%
\pgfpathlineto{\pgfqpoint{1.363162in}{1.661651in}}%
\pgfpathlineto{\pgfqpoint{1.362655in}{1.222620in}}%
\pgfpathlineto{\pgfqpoint{1.363581in}{1.532280in}}%
\pgfpathlineto{\pgfqpoint{1.364374in}{1.243818in}}%
\pgfpathlineto{\pgfqpoint{1.364617in}{1.601336in}}%
\pgfpathlineto{\pgfqpoint{1.364771in}{1.361716in}}%
\pgfpathlineto{\pgfqpoint{1.365278in}{1.624609in}}%
\pgfpathlineto{\pgfqpoint{1.365851in}{1.268293in}}%
\pgfpathlineto{\pgfqpoint{1.365895in}{1.478739in}}%
\pgfpathlineto{\pgfqpoint{1.365939in}{1.622206in}}%
\pgfpathlineto{\pgfqpoint{1.366116in}{1.235295in}}%
\pgfpathlineto{\pgfqpoint{1.366821in}{1.323363in}}%
\pgfpathlineto{\pgfqpoint{1.367196in}{1.195959in}}%
\pgfpathlineto{\pgfqpoint{1.367637in}{1.447052in}}%
\pgfpathlineto{\pgfqpoint{1.367857in}{1.350133in}}%
\pgfpathlineto{\pgfqpoint{1.368805in}{1.495239in}}%
\pgfpathlineto{\pgfqpoint{1.368452in}{1.112371in}}%
\pgfpathlineto{\pgfqpoint{1.368959in}{1.316698in}}%
\pgfpathlineto{\pgfqpoint{1.369113in}{1.220544in}}%
\pgfpathlineto{\pgfqpoint{1.369312in}{1.474806in}}%
\pgfpathlineto{\pgfqpoint{1.370039in}{1.281842in}}%
\pgfpathlineto{\pgfqpoint{1.370436in}{1.436672in}}%
\pgfpathlineto{\pgfqpoint{1.370700in}{1.157061in}}%
\pgfpathlineto{\pgfqpoint{1.371141in}{1.322380in}}%
\pgfpathlineto{\pgfqpoint{1.371780in}{1.248516in}}%
\pgfpathlineto{\pgfqpoint{1.371957in}{1.562109in}}%
\pgfpathlineto{\pgfqpoint{1.372221in}{1.350243in}}%
\pgfpathlineto{\pgfqpoint{1.373147in}{1.479723in}}%
\pgfpathlineto{\pgfqpoint{1.372860in}{1.226444in}}%
\pgfpathlineto{\pgfqpoint{1.373323in}{1.333525in}}%
\pgfpathlineto{\pgfqpoint{1.374029in}{1.143839in}}%
\pgfpathlineto{\pgfqpoint{1.373632in}{1.419517in}}%
\pgfpathlineto{\pgfqpoint{1.374381in}{1.213333in}}%
\pgfpathlineto{\pgfqpoint{1.374448in}{1.467594in}}%
\pgfpathlineto{\pgfqpoint{1.374756in}{1.178914in}}%
\pgfpathlineto{\pgfqpoint{1.375483in}{1.269386in}}%
\pgfpathlineto{\pgfqpoint{1.376057in}{1.108984in}}%
\pgfpathlineto{\pgfqpoint{1.375924in}{1.472402in}}%
\pgfpathlineto{\pgfqpoint{1.376564in}{1.238901in}}%
\pgfpathlineto{\pgfqpoint{1.377357in}{1.550636in}}%
\pgfpathlineto{\pgfqpoint{1.377710in}{1.391764in}}%
\pgfpathlineto{\pgfqpoint{1.378239in}{1.661541in}}%
\pgfpathlineto{\pgfqpoint{1.378437in}{1.305334in}}%
\pgfpathlineto{\pgfqpoint{1.378790in}{1.418534in}}%
\pgfpathlineto{\pgfqpoint{1.379054in}{1.297467in}}%
\pgfpathlineto{\pgfqpoint{1.379253in}{1.601554in}}%
\pgfpathlineto{\pgfqpoint{1.379870in}{1.490759in}}%
\pgfpathlineto{\pgfqpoint{1.380928in}{1.605488in}}%
\pgfpathlineto{\pgfqpoint{1.380178in}{1.273210in}}%
\pgfpathlineto{\pgfqpoint{1.380972in}{1.550855in}}%
\pgfpathlineto{\pgfqpoint{1.381854in}{1.321287in}}%
\pgfpathlineto{\pgfqpoint{1.381501in}{1.691699in}}%
\pgfpathlineto{\pgfqpoint{1.382096in}{1.422577in}}%
\pgfpathlineto{\pgfqpoint{1.382383in}{1.616852in}}%
\pgfpathlineto{\pgfqpoint{1.382581in}{1.316043in}}%
\pgfpathlineto{\pgfqpoint{1.383176in}{1.568993in}}%
\pgfpathlineto{\pgfqpoint{1.383705in}{1.242179in}}%
\pgfpathlineto{\pgfqpoint{1.384300in}{1.324237in}}%
\pgfpathlineto{\pgfqpoint{1.385248in}{1.432192in}}%
\pgfpathlineto{\pgfqpoint{1.385292in}{1.189185in}}%
\pgfpathlineto{\pgfqpoint{1.385336in}{1.354832in}}%
\pgfpathlineto{\pgfqpoint{1.386130in}{1.579483in}}%
\pgfpathlineto{\pgfqpoint{1.386416in}{1.128542in}}%
\pgfpathlineto{\pgfqpoint{1.386945in}{1.496987in}}%
\pgfpathlineto{\pgfqpoint{1.387540in}{1.437765in}}%
\pgfpathlineto{\pgfqpoint{1.388510in}{1.145697in}}%
\pgfpathlineto{\pgfqpoint{1.387717in}{1.438857in}}%
\pgfpathlineto{\pgfqpoint{1.388665in}{1.343468in}}%
\pgfpathlineto{\pgfqpoint{1.388819in}{1.421047in}}%
\pgfpathlineto{\pgfqpoint{1.389524in}{1.214534in}}%
\pgfpathlineto{\pgfqpoint{1.389789in}{1.390234in}}%
\pgfpathlineto{\pgfqpoint{1.390009in}{1.137174in}}%
\pgfpathlineto{\pgfqpoint{1.390648in}{1.397555in}}%
\pgfpathlineto{\pgfqpoint{1.390913in}{1.185579in}}%
\pgfpathlineto{\pgfqpoint{1.390957in}{1.391217in}}%
\pgfpathlineto{\pgfqpoint{1.391023in}{1.086256in}}%
\pgfpathlineto{\pgfqpoint{1.391993in}{1.148429in}}%
\pgfpathlineto{\pgfqpoint{1.392147in}{1.015015in}}%
\pgfpathlineto{\pgfqpoint{1.392390in}{1.390562in}}%
\pgfpathlineto{\pgfqpoint{1.393007in}{1.350680in}}%
\pgfpathlineto{\pgfqpoint{1.393426in}{1.467157in}}%
\pgfpathlineto{\pgfqpoint{1.393712in}{1.212568in}}%
\pgfpathlineto{\pgfqpoint{1.394285in}{1.094670in}}%
\pgfpathlineto{\pgfqpoint{1.394638in}{1.386082in}}%
\pgfpathlineto{\pgfqpoint{1.394770in}{1.290146in}}%
\pgfpathlineto{\pgfqpoint{1.395586in}{1.464098in}}%
\pgfpathlineto{\pgfqpoint{1.394947in}{1.088988in}}%
\pgfpathlineto{\pgfqpoint{1.395850in}{1.277035in}}%
\pgfpathlineto{\pgfqpoint{1.396578in}{1.087021in}}%
\pgfpathlineto{\pgfqpoint{1.396644in}{1.344998in}}%
\pgfpathlineto{\pgfqpoint{1.396952in}{1.179788in}}%
\pgfpathlineto{\pgfqpoint{1.397019in}{1.364447in}}%
\pgfpathlineto{\pgfqpoint{1.397592in}{1.018839in}}%
\pgfpathlineto{\pgfqpoint{1.398054in}{1.184049in}}%
\pgfpathlineto{\pgfqpoint{1.398826in}{0.958634in}}%
\pgfpathlineto{\pgfqpoint{1.398980in}{1.309377in}}%
\pgfpathlineto{\pgfqpoint{1.399179in}{1.093796in}}%
\pgfpathlineto{\pgfqpoint{1.399972in}{1.349696in}}%
\pgfpathlineto{\pgfqpoint{1.399840in}{1.052930in}}%
\pgfpathlineto{\pgfqpoint{1.400369in}{1.222511in}}%
\pgfpathlineto{\pgfqpoint{1.400986in}{1.050526in}}%
\pgfpathlineto{\pgfqpoint{1.400810in}{1.321943in}}%
\pgfpathlineto{\pgfqpoint{1.401537in}{1.071505in}}%
\pgfpathlineto{\pgfqpoint{1.401669in}{1.317244in}}%
\pgfpathlineto{\pgfqpoint{1.401824in}{0.932628in}}%
\pgfpathlineto{\pgfqpoint{1.402639in}{1.097183in}}%
\pgfpathlineto{\pgfqpoint{1.402661in}{1.021462in}}%
\pgfpathlineto{\pgfqpoint{1.403565in}{1.361388in}}%
\pgfpathlineto{\pgfqpoint{1.403719in}{1.049871in}}%
\pgfpathlineto{\pgfqpoint{1.404336in}{1.513049in}}%
\pgfpathlineto{\pgfqpoint{1.404865in}{1.365649in}}%
\pgfpathlineto{\pgfqpoint{1.405262in}{1.163944in}}%
\pgfpathlineto{\pgfqpoint{1.405946in}{1.440278in}}%
\pgfpathlineto{\pgfqpoint{1.405990in}{1.313529in}}%
\pgfpathlineto{\pgfqpoint{1.406276in}{1.561672in}}%
\pgfpathlineto{\pgfqpoint{1.406783in}{1.202843in}}%
\pgfpathlineto{\pgfqpoint{1.407048in}{1.237371in}}%
\pgfpathlineto{\pgfqpoint{1.407202in}{1.146353in}}%
\pgfpathlineto{\pgfqpoint{1.407687in}{1.440278in}}%
\pgfpathlineto{\pgfqpoint{1.408040in}{1.254526in}}%
\pgfpathlineto{\pgfqpoint{1.408238in}{1.401379in}}%
\pgfpathlineto{\pgfqpoint{1.408723in}{1.085273in}}%
\pgfpathlineto{\pgfqpoint{1.409164in}{1.380619in}}%
\pgfpathlineto{\pgfqpoint{1.410067in}{1.113136in}}%
\pgfpathlineto{\pgfqpoint{1.409362in}{1.486060in}}%
\pgfpathlineto{\pgfqpoint{1.410420in}{1.117069in}}%
\pgfpathlineto{\pgfqpoint{1.411192in}{1.348276in}}%
\pgfpathlineto{\pgfqpoint{1.411478in}{1.111825in}}%
\pgfpathlineto{\pgfqpoint{1.411544in}{1.237808in}}%
\pgfpathlineto{\pgfqpoint{1.412558in}{1.074237in}}%
\pgfpathlineto{\pgfqpoint{1.411853in}{1.399740in}}%
\pgfpathlineto{\pgfqpoint{1.412668in}{1.142200in}}%
\pgfpathlineto{\pgfqpoint{1.412779in}{1.323800in}}%
\pgfpathlineto{\pgfqpoint{1.412911in}{1.040146in}}%
\pgfpathlineto{\pgfqpoint{1.413792in}{1.201860in}}%
\pgfpathlineto{\pgfqpoint{1.414432in}{0.977865in}}%
\pgfpathlineto{\pgfqpoint{1.414873in}{1.183721in}}%
\pgfpathlineto{\pgfqpoint{1.415798in}{1.003651in}}%
\pgfpathlineto{\pgfqpoint{1.415975in}{1.290693in}}%
\pgfpathlineto{\pgfqpoint{1.416129in}{1.124499in}}%
\pgfpathlineto{\pgfqpoint{1.416261in}{1.324237in}}%
\pgfpathlineto{\pgfqpoint{1.417077in}{1.271243in}}%
\pgfpathlineto{\pgfqpoint{1.418267in}{1.005181in}}%
\pgfpathlineto{\pgfqpoint{1.417760in}{1.370129in}}%
\pgfpathlineto{\pgfqpoint{1.418333in}{1.019276in}}%
\pgfpathlineto{\pgfqpoint{1.418620in}{0.981033in}}%
\pgfpathlineto{\pgfqpoint{1.419501in}{1.318119in}}%
\pgfpathlineto{\pgfqpoint{1.419876in}{1.326423in}}%
\pgfpathlineto{\pgfqpoint{1.420625in}{1.009770in}}%
\pgfpathlineto{\pgfqpoint{1.421683in}{1.366305in}}%
\pgfpathlineto{\pgfqpoint{1.421750in}{1.263813in}}%
\pgfpathlineto{\pgfqpoint{1.422279in}{0.957323in}}%
\pgfpathlineto{\pgfqpoint{1.422345in}{1.323254in}}%
\pgfpathlineto{\pgfqpoint{1.422874in}{1.135098in}}%
\pgfpathlineto{\pgfqpoint{1.423844in}{1.290474in}}%
\pgfpathlineto{\pgfqpoint{1.423006in}{1.022773in}}%
\pgfpathlineto{\pgfqpoint{1.423998in}{1.280531in}}%
\pgfpathlineto{\pgfqpoint{1.424769in}{1.002886in}}%
\pgfpathlineto{\pgfqpoint{1.424196in}{1.335383in}}%
\pgfpathlineto{\pgfqpoint{1.425144in}{1.173669in}}%
\pgfpathlineto{\pgfqpoint{1.425849in}{1.238682in}}%
\pgfpathlineto{\pgfqpoint{1.426026in}{0.932410in}}%
\pgfpathlineto{\pgfqpoint{1.426202in}{1.126903in}}%
\pgfpathlineto{\pgfqpoint{1.426907in}{0.896789in}}%
\pgfpathlineto{\pgfqpoint{1.426775in}{1.263486in}}%
\pgfpathlineto{\pgfqpoint{1.427326in}{1.084617in}}%
\pgfpathlineto{\pgfqpoint{1.428142in}{1.274521in}}%
\pgfpathlineto{\pgfqpoint{1.427899in}{1.035011in}}%
\pgfpathlineto{\pgfqpoint{1.428428in}{1.146462in}}%
\pgfpathlineto{\pgfqpoint{1.428979in}{0.971636in}}%
\pgfpathlineto{\pgfqpoint{1.429156in}{1.291676in}}%
\pgfpathlineto{\pgfqpoint{1.429530in}{1.124827in}}%
\pgfpathlineto{\pgfqpoint{1.429949in}{1.248516in}}%
\pgfpathlineto{\pgfqpoint{1.430236in}{1.026706in}}%
\pgfpathlineto{\pgfqpoint{1.430655in}{1.157388in}}%
\pgfpathlineto{\pgfqpoint{1.430875in}{1.346965in}}%
\pgfpathlineto{\pgfqpoint{1.430765in}{1.023428in}}%
\pgfpathlineto{\pgfqpoint{1.431713in}{1.190277in}}%
\pgfpathlineto{\pgfqpoint{1.432771in}{1.057847in}}%
\pgfpathlineto{\pgfqpoint{1.432286in}{1.425090in}}%
\pgfpathlineto{\pgfqpoint{1.432837in}{1.077406in}}%
\pgfpathlineto{\pgfqpoint{1.432881in}{1.358984in}}%
\pgfpathlineto{\pgfqpoint{1.433652in}{1.009989in}}%
\pgfpathlineto{\pgfqpoint{1.433961in}{1.181318in}}%
\pgfpathlineto{\pgfqpoint{1.434446in}{1.371440in}}%
\pgfpathlineto{\pgfqpoint{1.434159in}{1.019495in}}%
\pgfpathlineto{\pgfqpoint{1.435063in}{1.285339in}}%
\pgfpathlineto{\pgfqpoint{1.435790in}{0.987261in}}%
\pgfpathlineto{\pgfqpoint{1.436209in}{1.059377in}}%
\pgfpathlineto{\pgfqpoint{1.436981in}{1.265780in}}%
\pgfpathlineto{\pgfqpoint{1.436363in}{0.882038in}}%
\pgfpathlineto{\pgfqpoint{1.437355in}{1.250046in}}%
\pgfpathlineto{\pgfqpoint{1.437620in}{1.255837in}}%
\pgfpathlineto{\pgfqpoint{1.438502in}{0.986059in}}%
\pgfpathlineto{\pgfqpoint{1.439538in}{1.254089in}}%
\pgfpathlineto{\pgfqpoint{1.439626in}{1.087349in}}%
\pgfpathlineto{\pgfqpoint{1.440155in}{0.918752in}}%
\pgfpathlineto{\pgfqpoint{1.440221in}{1.205356in}}%
\pgfpathlineto{\pgfqpoint{1.440728in}{1.061781in}}%
\pgfpathlineto{\pgfqpoint{1.441609in}{1.243271in}}%
\pgfpathlineto{\pgfqpoint{1.441786in}{0.980268in}}%
\pgfpathlineto{\pgfqpoint{1.441808in}{1.100352in}}%
\pgfpathlineto{\pgfqpoint{1.442337in}{0.993817in}}%
\pgfpathlineto{\pgfqpoint{1.442138in}{1.259443in}}%
\pgfpathlineto{\pgfqpoint{1.442910in}{1.019713in}}%
\pgfpathlineto{\pgfqpoint{1.442954in}{1.290584in}}%
\pgfpathlineto{\pgfqpoint{1.443219in}{0.934158in}}%
\pgfpathlineto{\pgfqpoint{1.444012in}{1.091283in}}%
\pgfpathlineto{\pgfqpoint{1.444563in}{0.960928in}}%
\pgfpathlineto{\pgfqpoint{1.444277in}{1.271899in}}%
\pgfpathlineto{\pgfqpoint{1.445092in}{1.041457in}}%
\pgfpathlineto{\pgfqpoint{1.445621in}{1.249390in}}%
\pgfpathlineto{\pgfqpoint{1.445996in}{0.920937in}}%
\pgfpathlineto{\pgfqpoint{1.446172in}{1.025614in}}%
\pgfpathlineto{\pgfqpoint{1.446745in}{0.965299in}}%
\pgfpathlineto{\pgfqpoint{1.446371in}{1.279766in}}%
\pgfpathlineto{\pgfqpoint{1.447274in}{1.043533in}}%
\pgfpathlineto{\pgfqpoint{1.448266in}{1.233437in}}%
\pgfpathlineto{\pgfqpoint{1.447980in}{0.940277in}}%
\pgfpathlineto{\pgfqpoint{1.448376in}{1.027580in}}%
\pgfpathlineto{\pgfqpoint{1.449368in}{1.324675in}}%
\pgfpathlineto{\pgfqpoint{1.449258in}{0.939621in}}%
\pgfpathlineto{\pgfqpoint{1.449545in}{1.198691in}}%
\pgfpathlineto{\pgfqpoint{1.450492in}{0.969560in}}%
\pgfpathlineto{\pgfqpoint{1.449721in}{1.256165in}}%
\pgfpathlineto{\pgfqpoint{1.450691in}{1.097183in}}%
\pgfpathlineto{\pgfqpoint{1.451705in}{1.341829in}}%
\pgfpathlineto{\pgfqpoint{1.451286in}{0.976990in}}%
\pgfpathlineto{\pgfqpoint{1.451815in}{1.185688in}}%
\pgfpathlineto{\pgfqpoint{1.452212in}{0.962349in}}%
\pgfpathlineto{\pgfqpoint{1.452719in}{1.329264in}}%
\pgfpathlineto{\pgfqpoint{1.452917in}{1.262611in}}%
\pgfpathlineto{\pgfqpoint{1.453402in}{1.065933in}}%
\pgfpathlineto{\pgfqpoint{1.453292in}{1.393949in}}%
\pgfpathlineto{\pgfqpoint{1.454019in}{1.221527in}}%
\pgfpathlineto{\pgfqpoint{1.454107in}{1.281514in}}%
\pgfpathlineto{\pgfqpoint{1.454724in}{0.993927in}}%
\pgfpathlineto{\pgfqpoint{1.455011in}{1.013485in}}%
\pgfpathlineto{\pgfqpoint{1.455430in}{0.931645in}}%
\pgfpathlineto{\pgfqpoint{1.455386in}{1.194976in}}%
\pgfpathlineto{\pgfqpoint{1.456025in}{1.025942in}}%
\pgfpathlineto{\pgfqpoint{1.456069in}{1.201641in}}%
\pgfpathlineto{\pgfqpoint{1.456796in}{0.900176in}}%
\pgfpathlineto{\pgfqpoint{1.457127in}{1.025832in}}%
\pgfpathlineto{\pgfqpoint{1.458163in}{0.793970in}}%
\pgfpathlineto{\pgfqpoint{1.457436in}{1.164491in}}%
\pgfpathlineto{\pgfqpoint{1.458251in}{0.957869in}}%
\pgfpathlineto{\pgfqpoint{1.458339in}{1.140889in}}%
\pgfpathlineto{\pgfqpoint{1.458472in}{0.867834in}}%
\pgfpathlineto{\pgfqpoint{1.459353in}{1.003214in}}%
\pgfpathlineto{\pgfqpoint{1.460367in}{0.852536in}}%
\pgfpathlineto{\pgfqpoint{1.459860in}{1.132694in}}%
\pgfpathlineto{\pgfqpoint{1.460455in}{0.949128in}}%
\pgfpathlineto{\pgfqpoint{1.460852in}{1.130837in}}%
\pgfpathlineto{\pgfqpoint{1.461006in}{0.817571in}}%
\pgfpathlineto{\pgfqpoint{1.461535in}{0.923013in}}%
\pgfpathlineto{\pgfqpoint{1.462373in}{0.904438in}}%
\pgfpathlineto{\pgfqpoint{1.461646in}{1.111825in}}%
\pgfpathlineto{\pgfqpoint{1.462395in}{1.028345in}}%
\pgfpathlineto{\pgfqpoint{1.463321in}{0.825220in}}%
\pgfpathlineto{\pgfqpoint{1.463497in}{1.201204in}}%
\pgfpathlineto{\pgfqpoint{1.463718in}{0.861278in}}%
\pgfpathlineto{\pgfqpoint{1.464621in}{0.934486in}}%
\pgfpathlineto{\pgfqpoint{1.464864in}{1.242725in}}%
\pgfpathlineto{\pgfqpoint{1.465745in}{1.114228in}}%
\pgfpathlineto{\pgfqpoint{1.466803in}{0.794188in}}%
\pgfpathlineto{\pgfqpoint{1.466583in}{1.208088in}}%
\pgfpathlineto{\pgfqpoint{1.466936in}{0.979176in}}%
\pgfpathlineto{\pgfqpoint{1.467377in}{0.835928in}}%
\pgfpathlineto{\pgfqpoint{1.467068in}{1.050636in}}%
\pgfpathlineto{\pgfqpoint{1.467509in}{1.006274in}}%
\pgfpathlineto{\pgfqpoint{1.468567in}{1.138048in}}%
\pgfpathlineto{\pgfqpoint{1.467795in}{0.792003in}}%
\pgfpathlineto{\pgfqpoint{1.468633in}{1.058066in}}%
\pgfpathlineto{\pgfqpoint{1.468699in}{0.880072in}}%
\pgfpathlineto{\pgfqpoint{1.469272in}{1.156405in}}%
\pgfpathlineto{\pgfqpoint{1.469757in}{0.936453in}}%
\pgfpathlineto{\pgfqpoint{1.470022in}{1.164928in}}%
\pgfpathlineto{\pgfqpoint{1.470727in}{0.882585in}}%
\pgfpathlineto{\pgfqpoint{1.470881in}{1.005072in}}%
\pgfpathlineto{\pgfqpoint{1.471653in}{0.843249in}}%
\pgfpathlineto{\pgfqpoint{1.470991in}{1.103411in}}%
\pgfpathlineto{\pgfqpoint{1.472027in}{0.929787in}}%
\pgfpathlineto{\pgfqpoint{1.472490in}{1.111387in}}%
\pgfpathlineto{\pgfqpoint{1.473107in}{0.810578in}}%
\pgfpathlineto{\pgfqpoint{1.473130in}{0.975461in}}%
\pgfpathlineto{\pgfqpoint{1.473570in}{0.794626in}}%
\pgfpathlineto{\pgfqpoint{1.473416in}{1.118162in}}%
\pgfpathlineto{\pgfqpoint{1.474232in}{0.942244in}}%
\pgfpathlineto{\pgfqpoint{1.474298in}{1.181099in}}%
\pgfpathlineto{\pgfqpoint{1.475113in}{0.861824in}}%
\pgfpathlineto{\pgfqpoint{1.475356in}{1.005946in}}%
\pgfpathlineto{\pgfqpoint{1.475730in}{1.198145in}}%
\pgfpathlineto{\pgfqpoint{1.476370in}{0.876356in}}%
\pgfpathlineto{\pgfqpoint{1.476678in}{0.830356in}}%
\pgfpathlineto{\pgfqpoint{1.476965in}{1.107782in}}%
\pgfpathlineto{\pgfqpoint{1.477185in}{0.880509in}}%
\pgfpathlineto{\pgfqpoint{1.477207in}{1.119582in}}%
\pgfpathlineto{\pgfqpoint{1.478177in}{0.824018in}}%
\pgfpathlineto{\pgfqpoint{1.478287in}{0.973385in}}%
\pgfpathlineto{\pgfqpoint{1.478464in}{1.103520in}}%
\pgfpathlineto{\pgfqpoint{1.478486in}{0.859857in}}%
\pgfpathlineto{\pgfqpoint{1.479411in}{1.087458in}}%
\pgfpathlineto{\pgfqpoint{1.479456in}{0.907607in}}%
\pgfpathlineto{\pgfqpoint{1.480249in}{1.192026in}}%
\pgfpathlineto{\pgfqpoint{1.480536in}{0.996112in}}%
\pgfpathlineto{\pgfqpoint{1.480778in}{0.844997in}}%
\pgfpathlineto{\pgfqpoint{1.480624in}{1.125701in}}%
\pgfpathlineto{\pgfqpoint{1.481021in}{1.001138in}}%
\pgfpathlineto{\pgfqpoint{1.481043in}{1.130072in}}%
\pgfpathlineto{\pgfqpoint{1.481814in}{0.844014in}}%
\pgfpathlineto{\pgfqpoint{1.482101in}{0.992834in}}%
\pgfpathlineto{\pgfqpoint{1.482982in}{0.762938in}}%
\pgfpathlineto{\pgfqpoint{1.483048in}{1.081121in}}%
\pgfpathlineto{\pgfqpoint{1.483203in}{0.939621in}}%
\pgfpathlineto{\pgfqpoint{1.483864in}{1.019386in}}%
\pgfpathlineto{\pgfqpoint{1.483732in}{0.733983in}}%
\pgfpathlineto{\pgfqpoint{1.483952in}{0.816151in}}%
\pgfpathlineto{\pgfqpoint{1.483974in}{0.754088in}}%
\pgfpathlineto{\pgfqpoint{1.484018in}{0.984858in}}%
\pgfpathlineto{\pgfqpoint{1.485054in}{0.785775in}}%
\pgfpathlineto{\pgfqpoint{1.485209in}{0.658589in}}%
\pgfpathlineto{\pgfqpoint{1.486178in}{0.984530in}}%
\pgfpathlineto{\pgfqpoint{1.486972in}{0.706885in}}%
\pgfpathlineto{\pgfqpoint{1.487192in}{1.019386in}}%
\pgfpathlineto{\pgfqpoint{1.487303in}{0.732453in}}%
\pgfpathlineto{\pgfqpoint{1.487964in}{1.031951in}}%
\pgfpathlineto{\pgfqpoint{1.488427in}{0.895587in}}%
\pgfpathlineto{\pgfqpoint{1.488735in}{0.777143in}}%
\pgfpathlineto{\pgfqpoint{1.489595in}{1.094451in}}%
\pgfpathlineto{\pgfqpoint{1.489837in}{0.875482in}}%
\pgfpathlineto{\pgfqpoint{1.489926in}{1.121658in}}%
\pgfpathlineto{\pgfqpoint{1.490719in}{1.013485in}}%
\pgfpathlineto{\pgfqpoint{1.491336in}{0.882585in}}%
\pgfpathlineto{\pgfqpoint{1.490763in}{1.162852in}}%
\pgfpathlineto{\pgfqpoint{1.491777in}{1.006492in}}%
\pgfpathlineto{\pgfqpoint{1.492703in}{1.180990in}}%
\pgfpathlineto{\pgfqpoint{1.492130in}{0.961038in}}%
\pgfpathlineto{\pgfqpoint{1.492857in}{1.016435in}}%
\pgfpathlineto{\pgfqpoint{1.493937in}{0.886955in}}%
\pgfpathlineto{\pgfqpoint{1.493122in}{1.250374in}}%
\pgfpathlineto{\pgfqpoint{1.493959in}{0.991523in}}%
\pgfpathlineto{\pgfqpoint{1.494202in}{0.884005in}}%
\pgfpathlineto{\pgfqpoint{1.494378in}{1.124390in}}%
\pgfpathlineto{\pgfqpoint{1.494731in}{1.080574in}}%
\pgfpathlineto{\pgfqpoint{1.494951in}{1.193228in}}%
\pgfpathlineto{\pgfqpoint{1.495149in}{0.903017in}}%
\pgfpathlineto{\pgfqpoint{1.495855in}{1.129526in}}%
\pgfpathlineto{\pgfqpoint{1.496803in}{0.892856in}}%
\pgfpathlineto{\pgfqpoint{1.496935in}{1.193883in}}%
\pgfpathlineto{\pgfqpoint{1.496957in}{1.173232in}}%
\pgfpathlineto{\pgfqpoint{1.497530in}{0.890889in}}%
\pgfpathlineto{\pgfqpoint{1.497045in}{1.229067in}}%
\pgfpathlineto{\pgfqpoint{1.498081in}{0.956011in}}%
\pgfpathlineto{\pgfqpoint{1.498985in}{1.265671in}}%
\pgfpathlineto{\pgfqpoint{1.499183in}{1.060797in}}%
\pgfpathlineto{\pgfqpoint{1.499999in}{0.949674in}}%
\pgfpathlineto{\pgfqpoint{1.499888in}{1.262830in}}%
\pgfpathlineto{\pgfqpoint{1.500263in}{1.044845in}}%
\pgfpathlineto{\pgfqpoint{1.500748in}{1.295719in}}%
\pgfpathlineto{\pgfqpoint{1.500969in}{0.951204in}}%
\pgfpathlineto{\pgfqpoint{1.501409in}{1.283372in}}%
\pgfpathlineto{\pgfqpoint{1.501520in}{0.950002in}}%
\pgfpathlineto{\pgfqpoint{1.502247in}{1.331777in}}%
\pgfpathlineto{\pgfqpoint{1.502489in}{1.247096in}}%
\pgfpathlineto{\pgfqpoint{1.502511in}{1.354504in}}%
\pgfpathlineto{\pgfqpoint{1.503437in}{1.016217in}}%
\pgfpathlineto{\pgfqpoint{1.503569in}{1.117179in}}%
\pgfpathlineto{\pgfqpoint{1.504253in}{1.010863in}}%
\pgfpathlineto{\pgfqpoint{1.504099in}{1.367616in}}%
\pgfpathlineto{\pgfqpoint{1.504628in}{1.151160in}}%
\pgfpathlineto{\pgfqpoint{1.505686in}{1.386082in}}%
\pgfpathlineto{\pgfqpoint{1.505024in}{1.019932in}}%
\pgfpathlineto{\pgfqpoint{1.505752in}{1.315168in}}%
\pgfpathlineto{\pgfqpoint{1.506016in}{1.357345in}}%
\pgfpathlineto{\pgfqpoint{1.505818in}{1.013376in}}%
\pgfpathlineto{\pgfqpoint{1.506501in}{1.240212in}}%
\pgfpathlineto{\pgfqpoint{1.506744in}{0.984967in}}%
\pgfpathlineto{\pgfqpoint{1.506920in}{1.365212in}}%
\pgfpathlineto{\pgfqpoint{1.507625in}{1.186890in}}%
\pgfpathlineto{\pgfqpoint{1.507669in}{1.185907in}}%
\pgfpathlineto{\pgfqpoint{1.507691in}{1.241523in}}%
\pgfpathlineto{\pgfqpoint{1.508176in}{1.367725in}}%
\pgfpathlineto{\pgfqpoint{1.508419in}{0.997642in}}%
\pgfpathlineto{\pgfqpoint{1.508683in}{1.088442in}}%
\pgfpathlineto{\pgfqpoint{1.509719in}{0.929787in}}%
\pgfpathlineto{\pgfqpoint{1.509411in}{1.366742in}}%
\pgfpathlineto{\pgfqpoint{1.509829in}{0.946614in}}%
\pgfpathlineto{\pgfqpoint{1.510028in}{1.249718in}}%
\pgfpathlineto{\pgfqpoint{1.510954in}{1.103411in}}%
\pgfpathlineto{\pgfqpoint{1.511813in}{1.301510in}}%
\pgfpathlineto{\pgfqpoint{1.511372in}{0.961256in}}%
\pgfpathlineto{\pgfqpoint{1.512056in}{1.121549in}}%
\pgfpathlineto{\pgfqpoint{1.512298in}{0.937545in}}%
\pgfpathlineto{\pgfqpoint{1.512673in}{1.251139in}}%
\pgfpathlineto{\pgfqpoint{1.513158in}{1.142419in}}%
\pgfpathlineto{\pgfqpoint{1.513687in}{0.810469in}}%
\pgfpathlineto{\pgfqpoint{1.513577in}{1.215299in}}%
\pgfpathlineto{\pgfqpoint{1.514326in}{0.953280in}}%
\pgfpathlineto{\pgfqpoint{1.514833in}{1.201204in}}%
\pgfpathlineto{\pgfqpoint{1.514613in}{0.922030in}}%
\pgfpathlineto{\pgfqpoint{1.515428in}{0.956995in}}%
\pgfpathlineto{\pgfqpoint{1.515759in}{1.276925in}}%
\pgfpathlineto{\pgfqpoint{1.515516in}{0.881601in}}%
\pgfpathlineto{\pgfqpoint{1.516552in}{1.095107in}}%
\pgfpathlineto{\pgfqpoint{1.517610in}{0.954045in}}%
\pgfpathlineto{\pgfqpoint{1.517169in}{1.259224in}}%
\pgfpathlineto{\pgfqpoint{1.517654in}{1.030858in}}%
\pgfpathlineto{\pgfqpoint{1.517919in}{0.886409in}}%
\pgfpathlineto{\pgfqpoint{1.517765in}{1.189294in}}%
\pgfpathlineto{\pgfqpoint{1.518514in}{1.096527in}}%
\pgfpathlineto{\pgfqpoint{1.519175in}{1.191698in}}%
\pgfpathlineto{\pgfqpoint{1.518756in}{0.895806in}}%
\pgfpathlineto{\pgfqpoint{1.519594in}{1.076532in}}%
\pgfpathlineto{\pgfqpoint{1.520365in}{0.833961in}}%
\pgfpathlineto{\pgfqpoint{1.519638in}{1.202734in}}%
\pgfpathlineto{\pgfqpoint{1.520718in}{0.991195in}}%
\pgfpathlineto{\pgfqpoint{1.521335in}{1.139906in}}%
\pgfpathlineto{\pgfqpoint{1.521225in}{0.798778in}}%
\pgfpathlineto{\pgfqpoint{1.521754in}{0.975024in}}%
\pgfpathlineto{\pgfqpoint{1.522790in}{0.832104in}}%
\pgfpathlineto{\pgfqpoint{1.522459in}{1.103630in}}%
\pgfpathlineto{\pgfqpoint{1.522856in}{0.998516in}}%
\pgfpathlineto{\pgfqpoint{1.523782in}{0.746330in}}%
\pgfpathlineto{\pgfqpoint{1.523518in}{1.094123in}}%
\pgfpathlineto{\pgfqpoint{1.523980in}{0.906405in}}%
\pgfpathlineto{\pgfqpoint{1.524487in}{1.172467in}}%
\pgfpathlineto{\pgfqpoint{1.524091in}{0.760425in}}%
\pgfpathlineto{\pgfqpoint{1.525082in}{0.953280in}}%
\pgfpathlineto{\pgfqpoint{1.525281in}{0.756273in}}%
\pgfpathlineto{\pgfqpoint{1.525854in}{1.093140in}}%
\pgfpathlineto{\pgfqpoint{1.526207in}{0.910120in}}%
\pgfpathlineto{\pgfqpoint{1.527353in}{1.234421in}}%
\pgfpathlineto{\pgfqpoint{1.527066in}{0.891107in}}%
\pgfpathlineto{\pgfqpoint{1.527375in}{1.108656in}}%
\pgfpathlineto{\pgfqpoint{1.527683in}{0.881164in}}%
\pgfpathlineto{\pgfqpoint{1.527992in}{1.226444in}}%
\pgfpathlineto{\pgfqpoint{1.528499in}{1.003870in}}%
\pgfpathlineto{\pgfqpoint{1.528918in}{1.134880in}}%
\pgfpathlineto{\pgfqpoint{1.528653in}{0.886081in}}%
\pgfpathlineto{\pgfqpoint{1.529579in}{1.025942in}}%
\pgfpathlineto{\pgfqpoint{1.529910in}{0.884442in}}%
\pgfpathlineto{\pgfqpoint{1.530196in}{1.232673in}}%
\pgfpathlineto{\pgfqpoint{1.530659in}{1.076750in}}%
\pgfpathlineto{\pgfqpoint{1.530681in}{1.156405in}}%
\pgfpathlineto{\pgfqpoint{1.530946in}{0.853301in}}%
\pgfpathlineto{\pgfqpoint{1.531739in}{1.035557in}}%
\pgfpathlineto{\pgfqpoint{1.532268in}{0.940386in}}%
\pgfpathlineto{\pgfqpoint{1.531960in}{1.256165in}}%
\pgfpathlineto{\pgfqpoint{1.532819in}{1.140124in}}%
\pgfpathlineto{\pgfqpoint{1.532951in}{1.240212in}}%
\pgfpathlineto{\pgfqpoint{1.533084in}{1.069539in}}%
\pgfpathlineto{\pgfqpoint{1.533458in}{1.071287in}}%
\pgfpathlineto{\pgfqpoint{1.534120in}{0.936999in}}%
\pgfpathlineto{\pgfqpoint{1.534318in}{1.179023in}}%
\pgfpathlineto{\pgfqpoint{1.534538in}{1.094779in}}%
\pgfpathlineto{\pgfqpoint{1.535310in}{1.154329in}}%
\pgfpathlineto{\pgfqpoint{1.535200in}{0.871986in}}%
\pgfpathlineto{\pgfqpoint{1.535442in}{0.996768in}}%
\pgfpathlineto{\pgfqpoint{1.535508in}{0.855596in}}%
\pgfpathlineto{\pgfqpoint{1.536302in}{1.110623in}}%
\pgfpathlineto{\pgfqpoint{1.536522in}{0.995893in}}%
\pgfpathlineto{\pgfqpoint{1.536941in}{1.131383in}}%
\pgfpathlineto{\pgfqpoint{1.537448in}{0.834617in}}%
\pgfpathlineto{\pgfqpoint{1.537580in}{0.951641in}}%
\pgfpathlineto{\pgfqpoint{1.538638in}{0.848057in}}%
\pgfpathlineto{\pgfqpoint{1.537845in}{1.141436in}}%
\pgfpathlineto{\pgfqpoint{1.538660in}{1.016763in}}%
\pgfpathlineto{\pgfqpoint{1.538881in}{1.155749in}}%
\pgfpathlineto{\pgfqpoint{1.539410in}{0.752012in}}%
\pgfpathlineto{\pgfqpoint{1.539718in}{0.871986in}}%
\pgfpathlineto{\pgfqpoint{1.540291in}{1.027690in}}%
\pgfpathlineto{\pgfqpoint{1.540578in}{0.702187in}}%
\pgfpathlineto{\pgfqpoint{1.540820in}{0.888813in}}%
\pgfpathlineto{\pgfqpoint{1.541305in}{0.812436in}}%
\pgfpathlineto{\pgfqpoint{1.541129in}{1.102318in}}%
\pgfpathlineto{\pgfqpoint{1.541878in}{0.956011in}}%
\pgfpathlineto{\pgfqpoint{1.542385in}{0.871549in}}%
\pgfpathlineto{\pgfqpoint{1.542077in}{1.163726in}}%
\pgfpathlineto{\pgfqpoint{1.542782in}{1.107563in}}%
\pgfpathlineto{\pgfqpoint{1.543355in}{1.231580in}}%
\pgfpathlineto{\pgfqpoint{1.542959in}{0.853738in}}%
\pgfpathlineto{\pgfqpoint{1.543862in}{1.124936in}}%
\pgfpathlineto{\pgfqpoint{1.544590in}{0.850133in}}%
\pgfpathlineto{\pgfqpoint{1.544083in}{1.178914in}}%
\pgfpathlineto{\pgfqpoint{1.545030in}{1.000701in}}%
\pgfpathlineto{\pgfqpoint{1.545493in}{1.184705in}}%
\pgfpathlineto{\pgfqpoint{1.545339in}{0.968358in}}%
\pgfpathlineto{\pgfqpoint{1.545846in}{1.035557in}}%
\pgfpathlineto{\pgfqpoint{1.545868in}{0.897554in}}%
\pgfpathlineto{\pgfqpoint{1.546750in}{1.259006in}}%
\pgfpathlineto{\pgfqpoint{1.546948in}{1.007803in}}%
\pgfpathlineto{\pgfqpoint{1.547213in}{0.930771in}}%
\pgfpathlineto{\pgfqpoint{1.548116in}{1.316917in}}%
\pgfpathlineto{\pgfqpoint{1.548227in}{1.140234in}}%
\pgfpathlineto{\pgfqpoint{1.548381in}{1.407061in}}%
\pgfpathlineto{\pgfqpoint{1.549152in}{1.344452in}}%
\pgfpathlineto{\pgfqpoint{1.549858in}{1.580794in}}%
\pgfpathlineto{\pgfqpoint{1.549593in}{1.185470in}}%
\pgfpathlineto{\pgfqpoint{1.550299in}{1.499391in}}%
\pgfpathlineto{\pgfqpoint{1.550497in}{1.329154in}}%
\pgfpathlineto{\pgfqpoint{1.551334in}{1.640999in}}%
\pgfpathlineto{\pgfqpoint{1.551423in}{1.412415in}}%
\pgfpathlineto{\pgfqpoint{1.551996in}{1.327297in}}%
\pgfpathlineto{\pgfqpoint{1.551731in}{1.593141in}}%
\pgfpathlineto{\pgfqpoint{1.552348in}{1.519714in}}%
\pgfpathlineto{\pgfqpoint{1.552459in}{1.568228in}}%
\pgfpathlineto{\pgfqpoint{1.553032in}{1.319976in}}%
\pgfpathlineto{\pgfqpoint{1.553318in}{1.444539in}}%
\pgfpathlineto{\pgfqpoint{1.553627in}{1.266217in}}%
\pgfpathlineto{\pgfqpoint{1.553759in}{1.622970in}}%
\pgfpathlineto{\pgfqpoint{1.554420in}{1.302931in}}%
\pgfpathlineto{\pgfqpoint{1.555104in}{1.582214in}}%
\pgfpathlineto{\pgfqpoint{1.555545in}{1.546266in}}%
\pgfpathlineto{\pgfqpoint{1.555765in}{1.417551in}}%
\pgfpathlineto{\pgfqpoint{1.556074in}{1.647555in}}%
\pgfpathlineto{\pgfqpoint{1.556647in}{1.524303in}}%
\pgfpathlineto{\pgfqpoint{1.557242in}{1.327297in}}%
\pgfpathlineto{\pgfqpoint{1.557771in}{1.586148in}}%
\pgfpathlineto{\pgfqpoint{1.558035in}{1.407389in}}%
\pgfpathlineto{\pgfqpoint{1.558763in}{1.696506in}}%
\pgfpathlineto{\pgfqpoint{1.558851in}{1.578062in}}%
\pgfpathlineto{\pgfqpoint{1.559314in}{1.762722in}}%
\pgfpathlineto{\pgfqpoint{1.559071in}{1.488573in}}%
\pgfpathlineto{\pgfqpoint{1.559975in}{1.714208in}}%
\pgfpathlineto{\pgfqpoint{1.560636in}{1.521900in}}%
\pgfpathlineto{\pgfqpoint{1.560548in}{1.871113in}}%
\pgfpathlineto{\pgfqpoint{1.561055in}{1.660121in}}%
\pgfpathlineto{\pgfqpoint{1.562025in}{1.845436in}}%
\pgfpathlineto{\pgfqpoint{1.561408in}{1.513158in}}%
\pgfpathlineto{\pgfqpoint{1.562157in}{1.756712in}}%
\pgfpathlineto{\pgfqpoint{1.562532in}{1.565060in}}%
\pgfpathlineto{\pgfqpoint{1.563149in}{1.857674in}}%
\pgfpathlineto{\pgfqpoint{1.563281in}{1.581122in}}%
\pgfpathlineto{\pgfqpoint{1.564185in}{1.793972in}}%
\pgfpathlineto{\pgfqpoint{1.563766in}{1.553040in}}%
\pgfpathlineto{\pgfqpoint{1.564405in}{1.697927in}}%
\pgfpathlineto{\pgfqpoint{1.565265in}{1.439732in}}%
\pgfpathlineto{\pgfqpoint{1.565089in}{1.712896in}}%
\pgfpathlineto{\pgfqpoint{1.565552in}{1.579701in}}%
\pgfpathlineto{\pgfqpoint{1.566323in}{1.381384in}}%
\pgfpathlineto{\pgfqpoint{1.566059in}{1.716939in}}%
\pgfpathlineto{\pgfqpoint{1.566499in}{1.625265in}}%
\pgfpathlineto{\pgfqpoint{1.567050in}{1.754199in}}%
\pgfpathlineto{\pgfqpoint{1.567469in}{1.408919in}}%
\pgfpathlineto{\pgfqpoint{1.567535in}{1.564404in}}%
\pgfpathlineto{\pgfqpoint{1.568329in}{1.409574in}}%
\pgfpathlineto{\pgfqpoint{1.567844in}{1.644387in}}%
\pgfpathlineto{\pgfqpoint{1.568615in}{1.639251in}}%
\pgfpathlineto{\pgfqpoint{1.569189in}{1.342048in}}%
\pgfpathlineto{\pgfqpoint{1.569497in}{1.708744in}}%
\pgfpathlineto{\pgfqpoint{1.569695in}{1.624609in}}%
\pgfpathlineto{\pgfqpoint{1.570798in}{1.761957in}}%
\pgfpathlineto{\pgfqpoint{1.570136in}{1.438420in}}%
\pgfpathlineto{\pgfqpoint{1.570842in}{1.727756in}}%
\pgfpathlineto{\pgfqpoint{1.570974in}{1.578499in}}%
\pgfpathlineto{\pgfqpoint{1.571701in}{1.805008in}}%
\pgfpathlineto{\pgfqpoint{1.571922in}{1.674763in}}%
\pgfpathlineto{\pgfqpoint{1.571966in}{1.867398in}}%
\pgfpathlineto{\pgfqpoint{1.572803in}{1.584181in}}%
\pgfpathlineto{\pgfqpoint{1.573002in}{1.689295in}}%
\pgfpathlineto{\pgfqpoint{1.573663in}{1.536213in}}%
\pgfpathlineto{\pgfqpoint{1.573178in}{1.791021in}}%
\pgfpathlineto{\pgfqpoint{1.574082in}{1.730488in}}%
\pgfpathlineto{\pgfqpoint{1.575118in}{1.884881in}}%
\pgfpathlineto{\pgfqpoint{1.574346in}{1.510973in}}%
\pgfpathlineto{\pgfqpoint{1.575162in}{1.806974in}}%
\pgfpathlineto{\pgfqpoint{1.575294in}{1.669081in}}%
\pgfpathlineto{\pgfqpoint{1.575911in}{1.983111in}}%
\pgfpathlineto{\pgfqpoint{1.576242in}{1.850899in}}%
\pgfpathlineto{\pgfqpoint{1.576506in}{1.982237in}}%
\pgfpathlineto{\pgfqpoint{1.577278in}{1.772228in}}%
\pgfpathlineto{\pgfqpoint{1.577366in}{1.975899in}}%
\pgfpathlineto{\pgfqpoint{1.578380in}{1.682411in}}%
\pgfpathlineto{\pgfqpoint{1.578512in}{1.870130in}}%
\pgfpathlineto{\pgfqpoint{1.578689in}{1.912088in}}%
\pgfpathlineto{\pgfqpoint{1.579041in}{1.687984in}}%
\pgfpathlineto{\pgfqpoint{1.579350in}{1.747315in}}%
\pgfpathlineto{\pgfqpoint{1.580143in}{1.612372in}}%
\pgfpathlineto{\pgfqpoint{1.579658in}{1.913399in}}%
\pgfpathlineto{\pgfqpoint{1.580408in}{1.717267in}}%
\pgfpathlineto{\pgfqpoint{1.580474in}{1.969890in}}%
\pgfpathlineto{\pgfqpoint{1.580761in}{1.687874in}}%
\pgfpathlineto{\pgfqpoint{1.581510in}{1.829265in}}%
\pgfpathlineto{\pgfqpoint{1.582590in}{1.586148in}}%
\pgfpathlineto{\pgfqpoint{1.581708in}{1.892857in}}%
\pgfpathlineto{\pgfqpoint{1.582656in}{1.702516in}}%
\pgfpathlineto{\pgfqpoint{1.583295in}{2.040257in}}%
\pgfpathlineto{\pgfqpoint{1.582877in}{1.693884in}}%
\pgfpathlineto{\pgfqpoint{1.583780in}{1.968360in}}%
\pgfpathlineto{\pgfqpoint{1.584243in}{1.667223in}}%
\pgfpathlineto{\pgfqpoint{1.583846in}{1.993163in}}%
\pgfpathlineto{\pgfqpoint{1.584904in}{1.788727in}}%
\pgfpathlineto{\pgfqpoint{1.584949in}{1.902800in}}%
\pgfpathlineto{\pgfqpoint{1.585500in}{1.586366in}}%
\pgfpathlineto{\pgfqpoint{1.585852in}{1.694103in}}%
\pgfpathlineto{\pgfqpoint{1.585874in}{1.641764in}}%
\pgfpathlineto{\pgfqpoint{1.586602in}{1.907062in}}%
\pgfpathlineto{\pgfqpoint{1.586910in}{1.732892in}}%
\pgfpathlineto{\pgfqpoint{1.587792in}{2.046376in}}%
\pgfpathlineto{\pgfqpoint{1.588101in}{1.997206in}}%
\pgfpathlineto{\pgfqpoint{1.588145in}{2.101774in}}%
\pgfpathlineto{\pgfqpoint{1.588938in}{1.768622in}}%
\pgfpathlineto{\pgfqpoint{1.589092in}{2.007696in}}%
\pgfpathlineto{\pgfqpoint{1.590040in}{1.749610in}}%
\pgfpathlineto{\pgfqpoint{1.589269in}{2.084838in}}%
\pgfpathlineto{\pgfqpoint{1.590217in}{1.878216in}}%
\pgfpathlineto{\pgfqpoint{1.590746in}{1.807193in}}%
\pgfpathlineto{\pgfqpoint{1.591319in}{2.047359in}}%
\pgfpathlineto{\pgfqpoint{1.591826in}{1.839863in}}%
\pgfpathlineto{\pgfqpoint{1.591892in}{2.234423in}}%
\pgfpathlineto{\pgfqpoint{1.592443in}{1.913946in}}%
\pgfpathlineto{\pgfqpoint{1.593369in}{2.168644in}}%
\pgfpathlineto{\pgfqpoint{1.592509in}{1.868600in}}%
\pgfpathlineto{\pgfqpoint{1.593545in}{1.981035in}}%
\pgfpathlineto{\pgfqpoint{1.594603in}{1.891655in}}%
\pgfpathlineto{\pgfqpoint{1.594074in}{2.167552in}}%
\pgfpathlineto{\pgfqpoint{1.594669in}{1.932412in}}%
\pgfpathlineto{\pgfqpoint{1.595264in}{2.155314in}}%
\pgfpathlineto{\pgfqpoint{1.595683in}{1.833635in}}%
\pgfpathlineto{\pgfqpoint{1.595815in}{2.051730in}}%
\pgfpathlineto{\pgfqpoint{1.596609in}{1.839973in}}%
\pgfpathlineto{\pgfqpoint{1.596300in}{2.149086in}}%
\pgfpathlineto{\pgfqpoint{1.596983in}{1.921813in}}%
\pgfpathlineto{\pgfqpoint{1.598108in}{2.254855in}}%
\pgfpathlineto{\pgfqpoint{1.597424in}{1.909575in}}%
\pgfpathlineto{\pgfqpoint{1.598196in}{2.065825in}}%
\pgfpathlineto{\pgfqpoint{1.599276in}{1.807193in}}%
\pgfpathlineto{\pgfqpoint{1.598394in}{2.147228in}}%
\pgfpathlineto{\pgfqpoint{1.599320in}{1.916568in}}%
\pgfpathlineto{\pgfqpoint{1.599474in}{2.068557in}}%
\pgfpathlineto{\pgfqpoint{1.599849in}{1.752997in}}%
\pgfpathlineto{\pgfqpoint{1.600422in}{1.987809in}}%
\pgfpathlineto{\pgfqpoint{1.601436in}{1.853412in}}%
\pgfpathlineto{\pgfqpoint{1.601017in}{2.175419in}}%
\pgfpathlineto{\pgfqpoint{1.601524in}{1.925528in}}%
\pgfpathlineto{\pgfqpoint{1.602097in}{2.097622in}}%
\pgfpathlineto{\pgfqpoint{1.602538in}{1.818010in}}%
\pgfpathlineto{\pgfqpoint{1.602626in}{1.943229in}}%
\pgfpathlineto{\pgfqpoint{1.602736in}{1.762394in}}%
\pgfpathlineto{\pgfqpoint{1.603287in}{2.057630in}}%
\pgfpathlineto{\pgfqpoint{1.603728in}{1.909356in}}%
\pgfpathlineto{\pgfqpoint{1.604852in}{2.193011in}}%
\pgfpathlineto{\pgfqpoint{1.604037in}{1.901599in}}%
\pgfpathlineto{\pgfqpoint{1.605029in}{2.118054in}}%
\pgfpathlineto{\pgfqpoint{1.605139in}{1.961476in}}%
\pgfpathlineto{\pgfqpoint{1.605536in}{2.203391in}}%
\pgfpathlineto{\pgfqpoint{1.605866in}{2.102102in}}%
\pgfpathlineto{\pgfqpoint{1.605888in}{2.230489in}}%
\pgfpathlineto{\pgfqpoint{1.606748in}{1.797687in}}%
\pgfpathlineto{\pgfqpoint{1.606946in}{1.923998in}}%
\pgfpathlineto{\pgfqpoint{1.607960in}{2.072272in}}%
\pgfpathlineto{\pgfqpoint{1.607542in}{1.817573in}}%
\pgfpathlineto{\pgfqpoint{1.608049in}{1.898648in}}%
\pgfpathlineto{\pgfqpoint{1.608886in}{2.202080in}}%
\pgfpathlineto{\pgfqpoint{1.608137in}{1.875375in}}%
\pgfpathlineto{\pgfqpoint{1.609239in}{2.114886in}}%
\pgfpathlineto{\pgfqpoint{1.610054in}{1.861826in}}%
\pgfpathlineto{\pgfqpoint{1.609900in}{2.135318in}}%
\pgfpathlineto{\pgfqpoint{1.610385in}{1.953937in}}%
\pgfpathlineto{\pgfqpoint{1.610473in}{2.063749in}}%
\pgfpathlineto{\pgfqpoint{1.611223in}{1.851008in}}%
\pgfpathlineto{\pgfqpoint{1.611465in}{1.933176in}}%
\pgfpathlineto{\pgfqpoint{1.611575in}{1.862591in}}%
\pgfpathlineto{\pgfqpoint{1.612170in}{2.092486in}}%
\pgfpathlineto{\pgfqpoint{1.612435in}{2.011739in}}%
\pgfpathlineto{\pgfqpoint{1.613030in}{2.168207in}}%
\pgfpathlineto{\pgfqpoint{1.612677in}{1.873299in}}%
\pgfpathlineto{\pgfqpoint{1.613559in}{2.103522in}}%
\pgfpathlineto{\pgfqpoint{1.613691in}{1.946944in}}%
\pgfpathlineto{\pgfqpoint{1.613824in}{2.180336in}}%
\pgfpathlineto{\pgfqpoint{1.614661in}{2.069650in}}%
\pgfpathlineto{\pgfqpoint{1.614705in}{2.179353in}}%
\pgfpathlineto{\pgfqpoint{1.615146in}{1.878434in}}%
\pgfpathlineto{\pgfqpoint{1.615719in}{2.033483in}}%
\pgfpathlineto{\pgfqpoint{1.616072in}{1.857564in}}%
\pgfpathlineto{\pgfqpoint{1.616358in}{2.125594in}}%
\pgfpathlineto{\pgfqpoint{1.616799in}{2.044628in}}%
\pgfpathlineto{\pgfqpoint{1.616976in}{2.196179in}}%
\pgfpathlineto{\pgfqpoint{1.617020in}{1.941044in}}%
\pgfpathlineto{\pgfqpoint{1.617923in}{2.090082in}}%
\pgfpathlineto{\pgfqpoint{1.618959in}{1.972512in}}%
\pgfpathlineto{\pgfqpoint{1.618474in}{2.265345in}}%
\pgfpathlineto{\pgfqpoint{1.619025in}{2.027582in}}%
\pgfpathlineto{\pgfqpoint{1.619202in}{2.230708in}}%
\pgfpathlineto{\pgfqpoint{1.619907in}{1.940388in}}%
\pgfpathlineto{\pgfqpoint{1.620150in}{2.130838in}}%
\pgfpathlineto{\pgfqpoint{1.620260in}{1.938968in}}%
\pgfpathlineto{\pgfqpoint{1.620943in}{2.208964in}}%
\pgfpathlineto{\pgfqpoint{1.621274in}{2.039820in}}%
\pgfpathlineto{\pgfqpoint{1.621648in}{2.244256in}}%
\pgfpathlineto{\pgfqpoint{1.621803in}{1.976883in}}%
\pgfpathlineto{\pgfqpoint{1.622398in}{2.204156in}}%
\pgfpathlineto{\pgfqpoint{1.622552in}{1.978850in}}%
\pgfpathlineto{\pgfqpoint{1.623390in}{2.266656in}}%
\pgfpathlineto{\pgfqpoint{1.623522in}{2.040038in}}%
\pgfpathlineto{\pgfqpoint{1.624602in}{2.343688in}}%
\pgfpathlineto{\pgfqpoint{1.623566in}{1.978959in}}%
\pgfpathlineto{\pgfqpoint{1.624690in}{2.201315in}}%
\pgfpathlineto{\pgfqpoint{1.625462in}{2.023430in}}%
\pgfpathlineto{\pgfqpoint{1.625131in}{2.316918in}}%
\pgfpathlineto{\pgfqpoint{1.625814in}{2.077626in}}%
\pgfpathlineto{\pgfqpoint{1.626564in}{2.232019in}}%
\pgfpathlineto{\pgfqpoint{1.626432in}{1.912307in}}%
\pgfpathlineto{\pgfqpoint{1.626894in}{1.993054in}}%
\pgfpathlineto{\pgfqpoint{1.627490in}{1.943229in}}%
\pgfpathlineto{\pgfqpoint{1.627137in}{2.134991in}}%
\pgfpathlineto{\pgfqpoint{1.627512in}{2.037416in}}%
\pgfpathlineto{\pgfqpoint{1.628305in}{2.151818in}}%
\pgfpathlineto{\pgfqpoint{1.627688in}{1.871113in}}%
\pgfpathlineto{\pgfqpoint{1.628592in}{1.930445in}}%
\pgfpathlineto{\pgfqpoint{1.628636in}{1.873189in}}%
\pgfpathlineto{\pgfqpoint{1.629231in}{2.208636in}}%
\pgfpathlineto{\pgfqpoint{1.629606in}{2.118382in}}%
\pgfpathlineto{\pgfqpoint{1.630708in}{1.932739in}}%
\pgfpathlineto{\pgfqpoint{1.630465in}{2.312220in}}%
\pgfpathlineto{\pgfqpoint{1.630752in}{2.049435in}}%
\pgfpathlineto{\pgfqpoint{1.630906in}{2.197272in}}%
\pgfpathlineto{\pgfqpoint{1.631501in}{1.969453in}}%
\pgfpathlineto{\pgfqpoint{1.631832in}{2.139908in}}%
\pgfpathlineto{\pgfqpoint{1.631920in}{1.979942in}}%
\pgfpathlineto{\pgfqpoint{1.632758in}{2.258133in}}%
\pgfpathlineto{\pgfqpoint{1.632912in}{2.147556in}}%
\pgfpathlineto{\pgfqpoint{1.632934in}{2.359423in}}%
\pgfpathlineto{\pgfqpoint{1.633860in}{2.092923in}}%
\pgfpathlineto{\pgfqpoint{1.634014in}{2.203391in}}%
\pgfpathlineto{\pgfqpoint{1.634301in}{2.031625in}}%
\pgfpathlineto{\pgfqpoint{1.634234in}{2.310799in}}%
\pgfpathlineto{\pgfqpoint{1.635094in}{2.256057in}}%
\pgfpathlineto{\pgfqpoint{1.635623in}{2.350572in}}%
\pgfpathlineto{\pgfqpoint{1.635998in}{2.086258in}}%
\pgfpathlineto{\pgfqpoint{1.636174in}{2.279003in}}%
\pgfpathlineto{\pgfqpoint{1.636857in}{2.043316in}}%
\pgfpathlineto{\pgfqpoint{1.636725in}{2.394825in}}%
\pgfpathlineto{\pgfqpoint{1.637320in}{2.065060in}}%
\pgfpathlineto{\pgfqpoint{1.638048in}{2.321617in}}%
\pgfpathlineto{\pgfqpoint{1.637761in}{2.033045in}}%
\pgfpathlineto{\pgfqpoint{1.638466in}{2.250157in}}%
\pgfpathlineto{\pgfqpoint{1.639304in}{2.007149in}}%
\pgfpathlineto{\pgfqpoint{1.638709in}{2.272119in}}%
\pgfpathlineto{\pgfqpoint{1.639613in}{2.099807in}}%
\pgfpathlineto{\pgfqpoint{1.639921in}{2.304899in}}%
\pgfpathlineto{\pgfqpoint{1.640340in}{2.008679in}}%
\pgfpathlineto{\pgfqpoint{1.640671in}{2.044409in}}%
\pgfpathlineto{\pgfqpoint{1.640979in}{1.919518in}}%
\pgfpathlineto{\pgfqpoint{1.641398in}{2.221311in}}%
\pgfpathlineto{\pgfqpoint{1.641795in}{1.963115in}}%
\pgfpathlineto{\pgfqpoint{1.642214in}{2.287963in}}%
\pgfpathlineto{\pgfqpoint{1.641905in}{1.901817in}}%
\pgfpathlineto{\pgfqpoint{1.643029in}{2.175091in}}%
\pgfpathlineto{\pgfqpoint{1.643955in}{2.057740in}}%
\pgfpathlineto{\pgfqpoint{1.644021in}{2.353632in}}%
\pgfpathlineto{\pgfqpoint{1.644153in}{2.058504in}}%
\pgfpathlineto{\pgfqpoint{1.644352in}{2.318120in}}%
\pgfpathlineto{\pgfqpoint{1.644660in}{2.019169in}}%
\pgfpathlineto{\pgfqpoint{1.645277in}{2.314514in}}%
\pgfpathlineto{\pgfqpoint{1.646115in}{2.088225in}}%
\pgfpathlineto{\pgfqpoint{1.645476in}{2.316481in}}%
\pgfpathlineto{\pgfqpoint{1.646402in}{2.167115in}}%
\pgfpathlineto{\pgfqpoint{1.647305in}{2.431210in}}%
\pgfpathlineto{\pgfqpoint{1.646468in}{2.122971in}}%
\pgfpathlineto{\pgfqpoint{1.647636in}{2.214645in}}%
\pgfpathlineto{\pgfqpoint{1.647658in}{2.082980in}}%
\pgfpathlineto{\pgfqpoint{1.648385in}{2.352976in}}%
\pgfpathlineto{\pgfqpoint{1.648716in}{2.233111in}}%
\pgfpathlineto{\pgfqpoint{1.649774in}{2.514034in}}%
\pgfpathlineto{\pgfqpoint{1.649465in}{2.081778in}}%
\pgfpathlineto{\pgfqpoint{1.649862in}{2.404113in}}%
\pgfpathlineto{\pgfqpoint{1.650083in}{2.158811in}}%
\pgfpathlineto{\pgfqpoint{1.650171in}{2.442356in}}%
\pgfpathlineto{\pgfqpoint{1.650964in}{2.415804in}}%
\pgfpathlineto{\pgfqpoint{1.651493in}{2.226118in}}%
\pgfpathlineto{\pgfqpoint{1.651361in}{2.471311in}}%
\pgfpathlineto{\pgfqpoint{1.652066in}{2.360188in}}%
\pgfpathlineto{\pgfqpoint{1.652860in}{2.592159in}}%
\pgfpathlineto{\pgfqpoint{1.652794in}{2.312001in}}%
\pgfpathlineto{\pgfqpoint{1.653191in}{2.505184in}}%
\pgfpathlineto{\pgfqpoint{1.654006in}{2.268841in}}%
\pgfpathlineto{\pgfqpoint{1.654204in}{2.538400in}}%
\pgfpathlineto{\pgfqpoint{1.654315in}{2.404877in}}%
\pgfpathlineto{\pgfqpoint{1.654645in}{2.274960in}}%
\pgfpathlineto{\pgfqpoint{1.654800in}{2.493274in}}%
\pgfpathlineto{\pgfqpoint{1.655417in}{2.372972in}}%
\pgfpathlineto{\pgfqpoint{1.655858in}{2.528785in}}%
\pgfpathlineto{\pgfqpoint{1.655725in}{2.188968in}}%
\pgfpathlineto{\pgfqpoint{1.656475in}{2.446289in}}%
\pgfpathlineto{\pgfqpoint{1.656651in}{2.555555in}}%
\pgfpathlineto{\pgfqpoint{1.657599in}{2.250485in}}%
\pgfpathlineto{\pgfqpoint{1.658062in}{2.485188in}}%
\pgfpathlineto{\pgfqpoint{1.658348in}{2.150616in}}%
\pgfpathlineto{\pgfqpoint{1.658745in}{2.430118in}}%
\pgfpathlineto{\pgfqpoint{1.659847in}{2.211695in}}%
\pgfpathlineto{\pgfqpoint{1.658921in}{2.498955in}}%
\pgfpathlineto{\pgfqpoint{1.659869in}{2.340957in}}%
\pgfpathlineto{\pgfqpoint{1.660553in}{2.111061in}}%
\pgfpathlineto{\pgfqpoint{1.660707in}{2.403238in}}%
\pgfpathlineto{\pgfqpoint{1.660993in}{2.303806in}}%
\pgfpathlineto{\pgfqpoint{1.661192in}{2.488903in}}%
\pgfpathlineto{\pgfqpoint{1.661500in}{2.124392in}}%
\pgfpathlineto{\pgfqpoint{1.662029in}{2.300747in}}%
\pgfpathlineto{\pgfqpoint{1.662757in}{2.056756in}}%
\pgfpathlineto{\pgfqpoint{1.662580in}{2.450988in}}%
\pgfpathlineto{\pgfqpoint{1.663131in}{2.212897in}}%
\pgfpathlineto{\pgfqpoint{1.664057in}{2.468033in}}%
\pgfpathlineto{\pgfqpoint{1.663396in}{2.160122in}}%
\pgfpathlineto{\pgfqpoint{1.664256in}{2.312657in}}%
\pgfpathlineto{\pgfqpoint{1.664278in}{2.314296in}}%
\pgfpathlineto{\pgfqpoint{1.664300in}{2.297141in}}%
\pgfpathlineto{\pgfqpoint{1.665137in}{2.079265in}}%
\pgfpathlineto{\pgfqpoint{1.665225in}{2.432631in}}%
\pgfpathlineto{\pgfqpoint{1.665402in}{2.247097in}}%
\pgfpathlineto{\pgfqpoint{1.666283in}{2.441919in}}%
\pgfpathlineto{\pgfqpoint{1.665622in}{2.101337in}}%
\pgfpathlineto{\pgfqpoint{1.666416in}{2.173780in}}%
\pgfpathlineto{\pgfqpoint{1.667121in}{2.121879in}}%
\pgfpathlineto{\pgfqpoint{1.666746in}{2.405861in}}%
\pgfpathlineto{\pgfqpoint{1.667386in}{2.269606in}}%
\pgfpathlineto{\pgfqpoint{1.667408in}{2.403020in}}%
\pgfpathlineto{\pgfqpoint{1.668267in}{2.038727in}}%
\pgfpathlineto{\pgfqpoint{1.668488in}{2.223496in}}%
\pgfpathlineto{\pgfqpoint{1.669215in}{2.130620in}}%
\pgfpathlineto{\pgfqpoint{1.669634in}{2.365432in}}%
\pgfpathlineto{\pgfqpoint{1.669678in}{2.196944in}}%
\pgfpathlineto{\pgfqpoint{1.670604in}{2.494694in}}%
\pgfpathlineto{\pgfqpoint{1.670736in}{2.353195in}}%
\pgfpathlineto{\pgfqpoint{1.670780in}{2.357128in}}%
\pgfpathlineto{\pgfqpoint{1.670802in}{2.351993in}}%
\pgfpathlineto{\pgfqpoint{1.671067in}{2.113684in}}%
\pgfpathlineto{\pgfqpoint{1.671662in}{2.469016in}}%
\pgfpathlineto{\pgfqpoint{1.671926in}{2.254965in}}%
\pgfpathlineto{\pgfqpoint{1.672389in}{2.538619in}}%
\pgfpathlineto{\pgfqpoint{1.673050in}{2.391219in}}%
\pgfpathlineto{\pgfqpoint{1.673359in}{2.179462in}}%
\pgfpathlineto{\pgfqpoint{1.673447in}{2.465848in}}%
\pgfpathlineto{\pgfqpoint{1.674175in}{2.278566in}}%
\pgfpathlineto{\pgfqpoint{1.674637in}{2.393077in}}%
\pgfpathlineto{\pgfqpoint{1.675188in}{2.196179in}}%
\pgfpathlineto{\pgfqpoint{1.675299in}{2.318885in}}%
\pgfpathlineto{\pgfqpoint{1.675585in}{2.097949in}}%
\pgfpathlineto{\pgfqpoint{1.675982in}{2.361608in}}%
\pgfpathlineto{\pgfqpoint{1.676313in}{2.289711in}}%
\pgfpathlineto{\pgfqpoint{1.676335in}{2.496114in}}%
\pgfpathlineto{\pgfqpoint{1.676467in}{2.112919in}}%
\pgfpathlineto{\pgfqpoint{1.677415in}{2.220109in}}%
\pgfpathlineto{\pgfqpoint{1.678407in}{2.404877in}}%
\pgfpathlineto{\pgfqpoint{1.677966in}{2.144169in}}%
\pgfpathlineto{\pgfqpoint{1.678495in}{2.253435in}}%
\pgfpathlineto{\pgfqpoint{1.678869in}{2.172032in}}%
\pgfpathlineto{\pgfqpoint{1.679465in}{2.474698in}}%
\pgfpathlineto{\pgfqpoint{1.679553in}{2.413728in}}%
\pgfpathlineto{\pgfqpoint{1.680456in}{2.522229in}}%
\pgfpathlineto{\pgfqpoint{1.680192in}{2.254309in}}%
\pgfpathlineto{\pgfqpoint{1.680633in}{2.352539in}}%
\pgfpathlineto{\pgfqpoint{1.681492in}{2.236608in}}%
\pgfpathlineto{\pgfqpoint{1.681382in}{2.542225in}}%
\pgfpathlineto{\pgfqpoint{1.681735in}{2.337133in}}%
\pgfpathlineto{\pgfqpoint{1.682264in}{2.472622in}}%
\pgfpathlineto{\pgfqpoint{1.682440in}{2.271136in}}%
\pgfpathlineto{\pgfqpoint{1.682815in}{2.387941in}}%
\pgfpathlineto{\pgfqpoint{1.683851in}{2.246005in}}%
\pgfpathlineto{\pgfqpoint{1.683366in}{2.526163in}}%
\pgfpathlineto{\pgfqpoint{1.683895in}{2.323584in}}%
\pgfpathlineto{\pgfqpoint{1.684578in}{2.497644in}}%
\pgfpathlineto{\pgfqpoint{1.684799in}{2.202408in}}%
\pgfpathlineto{\pgfqpoint{1.684997in}{2.330904in}}%
\pgfpathlineto{\pgfqpoint{1.685394in}{2.447382in}}%
\pgfpathlineto{\pgfqpoint{1.685680in}{2.119147in}}%
\pgfpathlineto{\pgfqpoint{1.686055in}{2.329265in}}%
\pgfpathlineto{\pgfqpoint{1.686805in}{2.238465in}}%
\pgfpathlineto{\pgfqpoint{1.686959in}{2.478632in}}%
\pgfpathlineto{\pgfqpoint{1.687135in}{2.400397in}}%
\pgfpathlineto{\pgfqpoint{1.687422in}{2.505730in}}%
\pgfpathlineto{\pgfqpoint{1.688105in}{2.215957in}}%
\pgfpathlineto{\pgfqpoint{1.688237in}{2.409248in}}%
\pgfpathlineto{\pgfqpoint{1.688700in}{2.575332in}}%
\pgfpathlineto{\pgfqpoint{1.688899in}{2.292771in}}%
\pgfpathlineto{\pgfqpoint{1.689361in}{2.435144in}}%
\pgfpathlineto{\pgfqpoint{1.690464in}{2.208854in}}%
\pgfpathlineto{\pgfqpoint{1.689626in}{2.519170in}}%
\pgfpathlineto{\pgfqpoint{1.690530in}{2.232237in}}%
\pgfpathlineto{\pgfqpoint{1.690971in}{2.547032in}}%
\pgfpathlineto{\pgfqpoint{1.691654in}{2.409139in}}%
\pgfpathlineto{\pgfqpoint{1.691698in}{2.173561in}}%
\pgfpathlineto{\pgfqpoint{1.692668in}{2.450441in}}%
\pgfpathlineto{\pgfqpoint{1.692756in}{2.415039in}}%
\pgfpathlineto{\pgfqpoint{1.693461in}{2.491962in}}%
\pgfpathlineto{\pgfqpoint{1.692800in}{2.267530in}}%
\pgfpathlineto{\pgfqpoint{1.693792in}{2.368710in}}%
\pgfpathlineto{\pgfqpoint{1.693836in}{2.234532in}}%
\pgfpathlineto{\pgfqpoint{1.694343in}{2.531517in}}%
\pgfpathlineto{\pgfqpoint{1.694607in}{2.446289in}}%
\pgfpathlineto{\pgfqpoint{1.694960in}{2.515017in}}%
\pgfpathlineto{\pgfqpoint{1.695313in}{2.215520in}}%
\pgfpathlineto{\pgfqpoint{1.695555in}{2.225572in}}%
\pgfpathlineto{\pgfqpoint{1.696283in}{2.188640in}}%
\pgfpathlineto{\pgfqpoint{1.695754in}{2.439952in}}%
\pgfpathlineto{\pgfqpoint{1.696393in}{2.409794in}}%
\pgfpathlineto{\pgfqpoint{1.697473in}{2.544301in}}%
\pgfpathlineto{\pgfqpoint{1.696856in}{2.204265in}}%
\pgfpathlineto{\pgfqpoint{1.697495in}{2.500594in}}%
\pgfpathlineto{\pgfqpoint{1.698244in}{2.239667in}}%
\pgfpathlineto{\pgfqpoint{1.697539in}{2.561565in}}%
\pgfpathlineto{\pgfqpoint{1.698619in}{2.318994in}}%
\pgfpathlineto{\pgfqpoint{1.699677in}{2.577080in}}%
\pgfpathlineto{\pgfqpoint{1.698994in}{2.220327in}}%
\pgfpathlineto{\pgfqpoint{1.699765in}{2.460166in}}%
\pgfpathlineto{\pgfqpoint{1.700603in}{2.637942in}}%
\pgfpathlineto{\pgfqpoint{1.700360in}{2.316372in}}%
\pgfpathlineto{\pgfqpoint{1.700889in}{2.562002in}}%
\pgfpathlineto{\pgfqpoint{1.701881in}{2.309051in}}%
\pgfpathlineto{\pgfqpoint{1.701330in}{2.566809in}}%
\pgfpathlineto{\pgfqpoint{1.702036in}{2.438531in}}%
\pgfpathlineto{\pgfqpoint{1.702829in}{2.283811in}}%
\pgfpathlineto{\pgfqpoint{1.702521in}{2.475354in}}%
\pgfpathlineto{\pgfqpoint{1.703270in}{2.329265in}}%
\pgfpathlineto{\pgfqpoint{1.703446in}{2.445197in}}%
\pgfpathlineto{\pgfqpoint{1.703821in}{2.116197in}}%
\pgfpathlineto{\pgfqpoint{1.704306in}{2.188422in}}%
\pgfpathlineto{\pgfqpoint{1.704482in}{2.072600in}}%
\pgfpathlineto{\pgfqpoint{1.705320in}{2.337351in}}%
\pgfpathlineto{\pgfqpoint{1.705386in}{2.240869in}}%
\pgfpathlineto{\pgfqpoint{1.705915in}{2.152473in}}%
\pgfpathlineto{\pgfqpoint{1.705717in}{2.402036in}}%
\pgfpathlineto{\pgfqpoint{1.706003in}{2.372862in}}%
\pgfpathlineto{\pgfqpoint{1.706664in}{2.563204in}}%
\pgfpathlineto{\pgfqpoint{1.706179in}{2.278020in}}%
\pgfpathlineto{\pgfqpoint{1.707127in}{2.432631in}}%
\pgfpathlineto{\pgfqpoint{1.707414in}{2.202517in}}%
\pgfpathlineto{\pgfqpoint{1.707568in}{2.564406in}}%
\pgfpathlineto{\pgfqpoint{1.708229in}{2.416350in}}%
\pgfpathlineto{\pgfqpoint{1.708736in}{2.558396in}}%
\pgfpathlineto{\pgfqpoint{1.708384in}{2.332653in}}%
\pgfpathlineto{\pgfqpoint{1.709309in}{2.384117in}}%
\pgfpathlineto{\pgfqpoint{1.710015in}{2.613029in}}%
\pgfpathlineto{\pgfqpoint{1.709376in}{2.293426in}}%
\pgfpathlineto{\pgfqpoint{1.710390in}{2.407172in}}%
\pgfpathlineto{\pgfqpoint{1.711205in}{2.240104in}}%
\pgfpathlineto{\pgfqpoint{1.710808in}{2.493383in}}%
\pgfpathlineto{\pgfqpoint{1.711514in}{2.297797in}}%
\pgfpathlineto{\pgfqpoint{1.711646in}{2.572928in}}%
\pgfpathlineto{\pgfqpoint{1.712087in}{2.227102in}}%
\pgfpathlineto{\pgfqpoint{1.712638in}{2.470765in}}%
\pgfpathlineto{\pgfqpoint{1.713365in}{2.135428in}}%
\pgfpathlineto{\pgfqpoint{1.712990in}{2.486936in}}%
\pgfpathlineto{\pgfqpoint{1.713806in}{2.302167in}}%
\pgfpathlineto{\pgfqpoint{1.714489in}{2.497535in}}%
\pgfpathlineto{\pgfqpoint{1.714644in}{2.243929in}}%
\pgfpathlineto{\pgfqpoint{1.714842in}{2.351119in}}%
\pgfpathlineto{\pgfqpoint{1.714864in}{2.236826in}}%
\pgfpathlineto{\pgfqpoint{1.715591in}{2.604397in}}%
\pgfpathlineto{\pgfqpoint{1.715922in}{2.453829in}}%
\pgfpathlineto{\pgfqpoint{1.716142in}{2.527146in}}%
\pgfpathlineto{\pgfqpoint{1.716297in}{2.280314in}}%
\pgfpathlineto{\pgfqpoint{1.716583in}{2.336805in}}%
\pgfpathlineto{\pgfqpoint{1.716848in}{2.121114in}}%
\pgfpathlineto{\pgfqpoint{1.716870in}{2.405205in}}%
\pgfpathlineto{\pgfqpoint{1.717685in}{2.340520in}}%
\pgfpathlineto{\pgfqpoint{1.718721in}{2.546595in}}%
\pgfpathlineto{\pgfqpoint{1.718281in}{2.264034in}}%
\pgfpathlineto{\pgfqpoint{1.718832in}{2.441481in}}%
\pgfpathlineto{\pgfqpoint{1.719978in}{2.208636in}}%
\pgfpathlineto{\pgfqpoint{1.719581in}{2.566919in}}%
\pgfpathlineto{\pgfqpoint{1.720000in}{2.332653in}}%
\pgfpathlineto{\pgfqpoint{1.720705in}{2.470765in}}%
\pgfpathlineto{\pgfqpoint{1.720639in}{2.250703in}}%
\pgfpathlineto{\pgfqpoint{1.720948in}{2.346529in}}%
\pgfpathlineto{\pgfqpoint{1.721190in}{2.170721in}}%
\pgfpathlineto{\pgfqpoint{1.721962in}{2.402474in}}%
\pgfpathlineto{\pgfqpoint{1.722050in}{2.310581in}}%
\pgfpathlineto{\pgfqpoint{1.722535in}{2.189405in}}%
\pgfpathlineto{\pgfqpoint{1.723284in}{2.509226in}}%
\pgfpathlineto{\pgfqpoint{1.723571in}{2.218251in}}%
\pgfpathlineto{\pgfqpoint{1.724408in}{2.326752in}}%
\pgfpathlineto{\pgfqpoint{1.724893in}{2.506713in}}%
\pgfpathlineto{\pgfqpoint{1.724452in}{2.198037in}}%
\pgfpathlineto{\pgfqpoint{1.725554in}{2.423234in}}%
\pgfpathlineto{\pgfqpoint{1.726150in}{2.279003in}}%
\pgfpathlineto{\pgfqpoint{1.725841in}{2.544410in}}%
\pgfpathlineto{\pgfqpoint{1.726679in}{2.353850in}}%
\pgfpathlineto{\pgfqpoint{1.727516in}{2.595765in}}%
\pgfpathlineto{\pgfqpoint{1.726921in}{2.268732in}}%
\pgfpathlineto{\pgfqpoint{1.727781in}{2.442028in}}%
\pgfpathlineto{\pgfqpoint{1.728728in}{2.182849in}}%
\pgfpathlineto{\pgfqpoint{1.728310in}{2.463007in}}%
\pgfpathlineto{\pgfqpoint{1.728883in}{2.335056in}}%
\pgfpathlineto{\pgfqpoint{1.729279in}{2.539384in}}%
\pgfpathlineto{\pgfqpoint{1.729588in}{2.295502in}}%
\pgfpathlineto{\pgfqpoint{1.730007in}{2.476774in}}%
\pgfpathlineto{\pgfqpoint{1.731065in}{2.165913in}}%
\pgfpathlineto{\pgfqpoint{1.731131in}{2.326862in}}%
\pgfpathlineto{\pgfqpoint{1.731814in}{2.193120in}}%
\pgfpathlineto{\pgfqpoint{1.731726in}{2.450988in}}%
\pgfpathlineto{\pgfqpoint{1.731969in}{2.361499in}}%
\pgfpathlineto{\pgfqpoint{1.732806in}{2.552714in}}%
\pgfpathlineto{\pgfqpoint{1.732431in}{2.265236in}}%
\pgfpathlineto{\pgfqpoint{1.733027in}{2.401599in}}%
\pgfpathlineto{\pgfqpoint{1.733049in}{2.220874in}}%
\pgfpathlineto{\pgfqpoint{1.733467in}{2.618711in}}%
\pgfpathlineto{\pgfqpoint{1.734129in}{2.405642in}}%
\pgfpathlineto{\pgfqpoint{1.734834in}{2.581233in}}%
\pgfpathlineto{\pgfqpoint{1.734371in}{2.348387in}}%
\pgfpathlineto{\pgfqpoint{1.735275in}{2.525944in}}%
\pgfpathlineto{\pgfqpoint{1.736443in}{2.339536in}}%
\pgfpathlineto{\pgfqpoint{1.735848in}{2.604615in}}%
\pgfpathlineto{\pgfqpoint{1.736487in}{2.449567in}}%
\pgfpathlineto{\pgfqpoint{1.736906in}{2.565826in}}%
\pgfpathlineto{\pgfqpoint{1.737148in}{2.313094in}}%
\pgfpathlineto{\pgfqpoint{1.737479in}{2.444650in}}%
\pgfpathlineto{\pgfqpoint{1.737501in}{2.271682in}}%
\pgfpathlineto{\pgfqpoint{1.738295in}{2.578829in}}%
\pgfpathlineto{\pgfqpoint{1.738581in}{2.381167in}}%
\pgfpathlineto{\pgfqpoint{1.738647in}{2.575114in}}%
\pgfpathlineto{\pgfqpoint{1.739617in}{2.264471in}}%
\pgfpathlineto{\pgfqpoint{1.739683in}{2.416569in}}%
\pgfpathlineto{\pgfqpoint{1.740036in}{2.208308in}}%
\pgfpathlineto{\pgfqpoint{1.740256in}{2.492290in}}%
\pgfpathlineto{\pgfqpoint{1.740653in}{2.399414in}}%
\pgfpathlineto{\pgfqpoint{1.741425in}{2.537963in}}%
\pgfpathlineto{\pgfqpoint{1.740741in}{2.274414in}}%
\pgfpathlineto{\pgfqpoint{1.741733in}{2.393077in}}%
\pgfpathlineto{\pgfqpoint{1.742174in}{2.284139in}}%
\pgfpathlineto{\pgfqpoint{1.742086in}{2.567793in}}%
\pgfpathlineto{\pgfqpoint{1.742857in}{2.291787in}}%
\pgfpathlineto{\pgfqpoint{1.743430in}{2.672360in}}%
\pgfpathlineto{\pgfqpoint{1.743959in}{2.426184in}}%
\pgfpathlineto{\pgfqpoint{1.744797in}{2.319104in}}%
\pgfpathlineto{\pgfqpoint{1.744687in}{2.600354in}}%
\pgfpathlineto{\pgfqpoint{1.745062in}{2.416569in}}%
\pgfpathlineto{\pgfqpoint{1.746164in}{2.674218in}}%
\pgfpathlineto{\pgfqpoint{1.745172in}{2.339209in}}%
\pgfpathlineto{\pgfqpoint{1.746208in}{2.512941in}}%
\pgfpathlineto{\pgfqpoint{1.747266in}{2.300419in}}%
\pgfpathlineto{\pgfqpoint{1.747067in}{2.613248in}}%
\pgfpathlineto{\pgfqpoint{1.747332in}{2.442028in}}%
\pgfpathlineto{\pgfqpoint{1.747971in}{2.163618in}}%
\pgfpathlineto{\pgfqpoint{1.747552in}{2.535232in}}%
\pgfpathlineto{\pgfqpoint{1.748544in}{2.347404in}}%
\pgfpathlineto{\pgfqpoint{1.748809in}{2.415148in}}%
\pgfpathlineto{\pgfqpoint{1.748897in}{2.167880in}}%
\pgfpathlineto{\pgfqpoint{1.749602in}{2.320743in}}%
\pgfpathlineto{\pgfqpoint{1.749646in}{2.187329in}}%
\pgfpathlineto{\pgfqpoint{1.749955in}{2.463335in}}%
\pgfpathlineto{\pgfqpoint{1.750660in}{2.335712in}}%
\pgfpathlineto{\pgfqpoint{1.751432in}{2.641657in}}%
\pgfpathlineto{\pgfqpoint{1.751255in}{2.302823in}}%
\pgfpathlineto{\pgfqpoint{1.751806in}{2.554572in}}%
\pgfpathlineto{\pgfqpoint{1.751828in}{2.614996in}}%
\pgfpathlineto{\pgfqpoint{1.752225in}{2.345437in}}%
\pgfpathlineto{\pgfqpoint{1.752864in}{2.443667in}}%
\pgfpathlineto{\pgfqpoint{1.753570in}{2.296486in}}%
\pgfpathlineto{\pgfqpoint{1.753041in}{2.645481in}}%
\pgfpathlineto{\pgfqpoint{1.753944in}{2.475026in}}%
\pgfpathlineto{\pgfqpoint{1.754738in}{2.643186in}}%
\pgfpathlineto{\pgfqpoint{1.754958in}{2.293535in}}%
\pgfpathlineto{\pgfqpoint{1.755025in}{2.536652in}}%
\pgfpathlineto{\pgfqpoint{1.755598in}{2.340083in}}%
\pgfpathlineto{\pgfqpoint{1.755950in}{2.673999in}}%
\pgfpathlineto{\pgfqpoint{1.756127in}{2.484532in}}%
\pgfpathlineto{\pgfqpoint{1.757251in}{2.668208in}}%
\pgfpathlineto{\pgfqpoint{1.756656in}{2.380183in}}%
\pgfpathlineto{\pgfqpoint{1.757273in}{2.564078in}}%
\pgfpathlineto{\pgfqpoint{1.757537in}{2.368383in}}%
\pgfpathlineto{\pgfqpoint{1.758353in}{2.665477in}}%
\pgfpathlineto{\pgfqpoint{1.758375in}{2.504419in}}%
\pgfpathlineto{\pgfqpoint{1.759389in}{2.642312in}}%
\pgfpathlineto{\pgfqpoint{1.758639in}{2.317683in}}%
\pgfpathlineto{\pgfqpoint{1.759455in}{2.473169in}}%
\pgfpathlineto{\pgfqpoint{1.760226in}{2.368820in}}%
\pgfpathlineto{\pgfqpoint{1.759653in}{2.606145in}}%
\pgfpathlineto{\pgfqpoint{1.760359in}{2.483658in}}%
\pgfpathlineto{\pgfqpoint{1.760866in}{2.568448in}}%
\pgfpathlineto{\pgfqpoint{1.761351in}{2.329375in}}%
\pgfpathlineto{\pgfqpoint{1.761417in}{2.496879in}}%
\pgfpathlineto{\pgfqpoint{1.761439in}{2.301403in}}%
\pgfpathlineto{\pgfqpoint{1.762276in}{2.597513in}}%
\pgfpathlineto{\pgfqpoint{1.762519in}{2.473824in}}%
\pgfpathlineto{\pgfqpoint{1.763489in}{2.773322in}}%
\pgfpathlineto{\pgfqpoint{1.762629in}{2.401490in}}%
\pgfpathlineto{\pgfqpoint{1.763643in}{2.511958in}}%
\pgfpathlineto{\pgfqpoint{1.763665in}{2.512067in}}%
\pgfpathlineto{\pgfqpoint{1.763753in}{2.670394in}}%
\pgfpathlineto{\pgfqpoint{1.764414in}{2.379746in}}%
\pgfpathlineto{\pgfqpoint{1.764745in}{2.474371in}}%
\pgfpathlineto{\pgfqpoint{1.764789in}{2.355271in}}%
\pgfpathlineto{\pgfqpoint{1.765671in}{2.696836in}}%
\pgfpathlineto{\pgfqpoint{1.765759in}{2.592268in}}%
\pgfpathlineto{\pgfqpoint{1.765781in}{2.652583in}}%
\pgfpathlineto{\pgfqpoint{1.766023in}{2.381167in}}%
\pgfpathlineto{\pgfqpoint{1.766795in}{2.512832in}}%
\pgfpathlineto{\pgfqpoint{1.767787in}{2.251031in}}%
\pgfpathlineto{\pgfqpoint{1.767126in}{2.602539in}}%
\pgfpathlineto{\pgfqpoint{1.767897in}{2.505839in}}%
\pgfpathlineto{\pgfqpoint{1.768228in}{2.302495in}}%
\pgfpathlineto{\pgfqpoint{1.768382in}{2.530533in}}%
\pgfpathlineto{\pgfqpoint{1.769021in}{2.404550in}}%
\pgfpathlineto{\pgfqpoint{1.769043in}{2.535232in}}%
\pgfpathlineto{\pgfqpoint{1.769969in}{2.255620in}}%
\pgfpathlineto{\pgfqpoint{1.770123in}{2.513488in}}%
\pgfpathlineto{\pgfqpoint{1.770211in}{2.252233in}}%
\pgfpathlineto{\pgfqpoint{1.771049in}{2.587024in}}%
\pgfpathlineto{\pgfqpoint{1.771292in}{2.420393in}}%
\pgfpathlineto{\pgfqpoint{1.771644in}{2.631823in}}%
\pgfpathlineto{\pgfqpoint{1.771953in}{2.325878in}}%
\pgfpathlineto{\pgfqpoint{1.772394in}{2.426403in}}%
\pgfpathlineto{\pgfqpoint{1.772724in}{2.258133in}}%
\pgfpathlineto{\pgfqpoint{1.772526in}{2.537308in}}%
\pgfpathlineto{\pgfqpoint{1.773474in}{2.513378in}}%
\pgfpathlineto{\pgfqpoint{1.774488in}{2.391438in}}%
\pgfpathlineto{\pgfqpoint{1.773981in}{2.679572in}}%
\pgfpathlineto{\pgfqpoint{1.774576in}{2.465083in}}%
\pgfpathlineto{\pgfqpoint{1.775193in}{2.606473in}}%
\pgfpathlineto{\pgfqpoint{1.775039in}{2.328173in}}%
\pgfpathlineto{\pgfqpoint{1.775612in}{2.345109in}}%
\pgfpathlineto{\pgfqpoint{1.776163in}{2.324021in}}%
\pgfpathlineto{\pgfqpoint{1.775898in}{2.528348in}}%
\pgfpathlineto{\pgfqpoint{1.776273in}{2.372862in}}%
\pgfpathlineto{\pgfqpoint{1.776295in}{2.622207in}}%
\pgfpathlineto{\pgfqpoint{1.776604in}{2.350244in}}%
\pgfpathlineto{\pgfqpoint{1.777375in}{2.425856in}}%
\pgfpathlineto{\pgfqpoint{1.777463in}{2.302932in}}%
\pgfpathlineto{\pgfqpoint{1.778301in}{2.662636in}}%
\pgfpathlineto{\pgfqpoint{1.778477in}{2.399633in}}%
\pgfpathlineto{\pgfqpoint{1.778565in}{2.690171in}}%
\pgfpathlineto{\pgfqpoint{1.779623in}{2.580468in}}%
\pgfpathlineto{\pgfqpoint{1.780395in}{2.364449in}}%
\pgfpathlineto{\pgfqpoint{1.779800in}{2.639253in}}%
\pgfpathlineto{\pgfqpoint{1.780725in}{2.437876in}}%
\pgfpathlineto{\pgfqpoint{1.781210in}{2.659467in}}%
\pgfpathlineto{\pgfqpoint{1.780902in}{2.309051in}}%
\pgfpathlineto{\pgfqpoint{1.781828in}{2.484314in}}%
\pgfpathlineto{\pgfqpoint{1.782246in}{2.257368in}}%
\pgfpathlineto{\pgfqpoint{1.781894in}{2.582107in}}%
\pgfpathlineto{\pgfqpoint{1.782842in}{2.480708in}}%
\pgfpathlineto{\pgfqpoint{1.782864in}{2.617837in}}%
\pgfpathlineto{\pgfqpoint{1.783525in}{2.365105in}}%
\pgfpathlineto{\pgfqpoint{1.783922in}{2.428042in}}%
\pgfpathlineto{\pgfqpoint{1.784715in}{2.279331in}}%
\pgfpathlineto{\pgfqpoint{1.784274in}{2.666132in}}%
\pgfpathlineto{\pgfqpoint{1.784803in}{2.501578in}}%
\pgfpathlineto{\pgfqpoint{1.784847in}{2.553151in}}%
\pgfpathlineto{\pgfqpoint{1.785487in}{2.307631in}}%
\pgfpathlineto{\pgfqpoint{1.785839in}{2.435253in}}%
\pgfpathlineto{\pgfqpoint{1.786104in}{2.345327in}}%
\pgfpathlineto{\pgfqpoint{1.786611in}{2.627015in}}%
\pgfpathlineto{\pgfqpoint{1.786941in}{2.434051in}}%
\pgfpathlineto{\pgfqpoint{1.787206in}{2.689734in}}%
\pgfpathlineto{\pgfqpoint{1.787007in}{2.381057in}}%
\pgfpathlineto{\pgfqpoint{1.788088in}{2.608768in}}%
\pgfpathlineto{\pgfqpoint{1.788925in}{2.381385in}}%
\pgfpathlineto{\pgfqpoint{1.788617in}{2.619694in}}%
\pgfpathlineto{\pgfqpoint{1.789278in}{2.451425in}}%
\pgfpathlineto{\pgfqpoint{1.789454in}{2.756386in}}%
\pgfpathlineto{\pgfqpoint{1.790358in}{2.367836in}}%
\pgfpathlineto{\pgfqpoint{1.790380in}{2.541569in}}%
\pgfpathlineto{\pgfqpoint{1.790777in}{2.446617in}}%
\pgfpathlineto{\pgfqpoint{1.790865in}{2.699458in}}%
\pgfpathlineto{\pgfqpoint{1.791460in}{2.614559in}}%
\pgfpathlineto{\pgfqpoint{1.792342in}{2.310799in}}%
\pgfpathlineto{\pgfqpoint{1.792628in}{2.375594in}}%
\pgfpathlineto{\pgfqpoint{1.793730in}{2.646137in}}%
\pgfpathlineto{\pgfqpoint{1.793664in}{2.319650in}}%
\pgfpathlineto{\pgfqpoint{1.793752in}{2.456997in}}%
\pgfpathlineto{\pgfqpoint{1.794568in}{2.406079in}}%
\pgfpathlineto{\pgfqpoint{1.793929in}{2.620131in}}%
\pgfpathlineto{\pgfqpoint{1.794722in}{2.439842in}}%
\pgfpathlineto{\pgfqpoint{1.795075in}{2.645262in}}%
\pgfpathlineto{\pgfqpoint{1.795758in}{2.379091in}}%
\pgfpathlineto{\pgfqpoint{1.795846in}{2.534467in}}%
\pgfpathlineto{\pgfqpoint{1.796838in}{2.464427in}}%
\pgfpathlineto{\pgfqpoint{1.796331in}{2.753326in}}%
\pgfpathlineto{\pgfqpoint{1.796904in}{2.545612in}}%
\pgfpathlineto{\pgfqpoint{1.797830in}{2.397994in}}%
\pgfpathlineto{\pgfqpoint{1.798028in}{2.672142in}}%
\pgfpathlineto{\pgfqpoint{1.798602in}{2.760538in}}%
\pgfpathlineto{\pgfqpoint{1.798359in}{2.498737in}}%
\pgfpathlineto{\pgfqpoint{1.799064in}{2.605271in}}%
\pgfpathlineto{\pgfqpoint{1.799968in}{2.504637in}}%
\pgfpathlineto{\pgfqpoint{1.799593in}{2.691045in}}%
\pgfpathlineto{\pgfqpoint{1.800189in}{2.546267in}}%
\pgfpathlineto{\pgfqpoint{1.800541in}{2.694978in}}%
\pgfpathlineto{\pgfqpoint{1.801004in}{2.437985in}}%
\pgfpathlineto{\pgfqpoint{1.801291in}{2.601338in}}%
\pgfpathlineto{\pgfqpoint{1.802415in}{2.417224in}}%
\pgfpathlineto{\pgfqpoint{1.802106in}{2.808615in}}%
\pgfpathlineto{\pgfqpoint{1.802437in}{2.494366in}}%
\pgfpathlineto{\pgfqpoint{1.802812in}{2.671705in}}%
\pgfpathlineto{\pgfqpoint{1.803186in}{2.326752in}}%
\pgfpathlineto{\pgfqpoint{1.803539in}{2.481910in}}%
\pgfpathlineto{\pgfqpoint{1.803759in}{2.394606in}}%
\pgfpathlineto{\pgfqpoint{1.804244in}{2.677933in}}%
\pgfpathlineto{\pgfqpoint{1.804619in}{2.431320in}}%
\pgfpathlineto{\pgfqpoint{1.804906in}{2.691810in}}%
\pgfpathlineto{\pgfqpoint{1.805765in}{2.567356in}}%
\pgfpathlineto{\pgfqpoint{1.806581in}{2.434816in}}%
\pgfpathlineto{\pgfqpoint{1.806757in}{2.722186in}}%
\pgfpathlineto{\pgfqpoint{1.806889in}{2.534685in}}%
\pgfpathlineto{\pgfqpoint{1.807462in}{2.403238in}}%
\pgfpathlineto{\pgfqpoint{1.807859in}{2.586150in}}%
\pgfpathlineto{\pgfqpoint{1.807969in}{2.546267in}}%
\pgfpathlineto{\pgfqpoint{1.808388in}{2.698038in}}%
\pgfpathlineto{\pgfqpoint{1.808895in}{2.389689in}}%
\pgfpathlineto{\pgfqpoint{1.809049in}{2.472076in}}%
\pgfpathlineto{\pgfqpoint{1.809072in}{2.473606in}}%
\pgfpathlineto{\pgfqpoint{1.809865in}{2.261739in}}%
\pgfpathlineto{\pgfqpoint{1.809292in}{2.505621in}}%
\pgfpathlineto{\pgfqpoint{1.810174in}{2.420502in}}%
\pgfpathlineto{\pgfqpoint{1.810526in}{2.485734in}}%
\pgfpathlineto{\pgfqpoint{1.810350in}{2.271245in}}%
\pgfpathlineto{\pgfqpoint{1.810636in}{2.311237in}}%
\pgfpathlineto{\pgfqpoint{1.810659in}{2.290367in}}%
\pgfpathlineto{\pgfqpoint{1.811055in}{2.627780in}}%
\pgfpathlineto{\pgfqpoint{1.811430in}{2.450551in}}%
\pgfpathlineto{\pgfqpoint{1.812488in}{2.613466in}}%
\pgfpathlineto{\pgfqpoint{1.812224in}{2.308614in}}%
\pgfpathlineto{\pgfqpoint{1.812532in}{2.525398in}}%
\pgfpathlineto{\pgfqpoint{1.812753in}{2.262613in}}%
\pgfpathlineto{\pgfqpoint{1.812841in}{2.649524in}}%
\pgfpathlineto{\pgfqpoint{1.813634in}{2.394825in}}%
\pgfpathlineto{\pgfqpoint{1.814648in}{2.657609in}}%
\pgfpathlineto{\pgfqpoint{1.813766in}{2.317683in}}%
\pgfpathlineto{\pgfqpoint{1.814758in}{2.550420in}}%
\pgfpathlineto{\pgfqpoint{1.815750in}{2.343361in}}%
\pgfpathlineto{\pgfqpoint{1.815353in}{2.588226in}}%
\pgfpathlineto{\pgfqpoint{1.815860in}{2.518295in}}%
\pgfpathlineto{\pgfqpoint{1.816169in}{2.660778in}}%
\pgfpathlineto{\pgfqpoint{1.816742in}{2.382369in}}%
\pgfpathlineto{\pgfqpoint{1.816940in}{2.497863in}}%
\pgfpathlineto{\pgfqpoint{1.817734in}{2.408483in}}%
\pgfpathlineto{\pgfqpoint{1.817558in}{2.684489in}}%
\pgfpathlineto{\pgfqpoint{1.817999in}{2.559379in}}%
\pgfpathlineto{\pgfqpoint{1.818021in}{2.649961in}}%
\pgfpathlineto{\pgfqpoint{1.818395in}{2.412635in}}%
\pgfpathlineto{\pgfqpoint{1.819079in}{2.482565in}}%
\pgfpathlineto{\pgfqpoint{1.819630in}{2.384882in}}%
\pgfpathlineto{\pgfqpoint{1.819321in}{2.709729in}}%
\pgfpathlineto{\pgfqpoint{1.820137in}{2.544847in}}%
\pgfpathlineto{\pgfqpoint{1.820313in}{2.659248in}}%
\pgfpathlineto{\pgfqpoint{1.820644in}{2.389799in}}%
\pgfpathlineto{\pgfqpoint{1.821239in}{2.566482in}}%
\pgfpathlineto{\pgfqpoint{1.821657in}{2.376687in}}%
\pgfpathlineto{\pgfqpoint{1.822363in}{2.472185in}}%
\pgfpathlineto{\pgfqpoint{1.823002in}{2.739450in}}%
\pgfpathlineto{\pgfqpoint{1.822539in}{2.345109in}}%
\pgfpathlineto{\pgfqpoint{1.823509in}{2.649196in}}%
\pgfpathlineto{\pgfqpoint{1.824192in}{2.379746in}}%
\pgfpathlineto{\pgfqpoint{1.823884in}{2.736062in}}%
\pgfpathlineto{\pgfqpoint{1.824633in}{2.527583in}}%
\pgfpathlineto{\pgfqpoint{1.825515in}{2.650179in}}%
\pgfpathlineto{\pgfqpoint{1.825118in}{2.325223in}}%
\pgfpathlineto{\pgfqpoint{1.825735in}{2.508899in}}%
\pgfpathlineto{\pgfqpoint{1.826264in}{2.706670in}}%
\pgfpathlineto{\pgfqpoint{1.826397in}{2.438203in}}%
\pgfpathlineto{\pgfqpoint{1.826815in}{2.543536in}}%
\pgfpathlineto{\pgfqpoint{1.827741in}{2.361280in}}%
\pgfpathlineto{\pgfqpoint{1.826970in}{2.672470in}}%
\pgfpathlineto{\pgfqpoint{1.827939in}{2.465411in}}%
\pgfpathlineto{\pgfqpoint{1.828490in}{2.668208in}}%
\pgfpathlineto{\pgfqpoint{1.828226in}{2.340083in}}%
\pgfpathlineto{\pgfqpoint{1.828997in}{2.440717in}}%
\pgfpathlineto{\pgfqpoint{1.829549in}{2.345000in}}%
\pgfpathlineto{\pgfqpoint{1.829306in}{2.583746in}}%
\pgfpathlineto{\pgfqpoint{1.830011in}{2.538400in}}%
\pgfpathlineto{\pgfqpoint{1.830937in}{2.695306in}}%
\pgfpathlineto{\pgfqpoint{1.830232in}{2.347404in}}%
\pgfpathlineto{\pgfqpoint{1.831136in}{2.581342in}}%
\pgfpathlineto{\pgfqpoint{1.831356in}{2.643077in}}%
\pgfpathlineto{\pgfqpoint{1.831819in}{2.386084in}}%
\pgfpathlineto{\pgfqpoint{1.831885in}{2.444650in}}%
\pgfpathlineto{\pgfqpoint{1.832767in}{2.350463in}}%
\pgfpathlineto{\pgfqpoint{1.832194in}{2.630839in}}%
\pgfpathlineto{\pgfqpoint{1.832987in}{2.464427in}}%
\pgfpathlineto{\pgfqpoint{1.833031in}{2.352867in}}%
\pgfpathlineto{\pgfqpoint{1.833274in}{2.618055in}}%
\pgfpathlineto{\pgfqpoint{1.834067in}{2.397557in}}%
\pgfpathlineto{\pgfqpoint{1.835147in}{2.601447in}}%
\pgfpathlineto{\pgfqpoint{1.834376in}{2.335056in}}%
\pgfpathlineto{\pgfqpoint{1.835191in}{2.575988in}}%
\pgfpathlineto{\pgfqpoint{1.836337in}{2.202408in}}%
\pgfpathlineto{\pgfqpoint{1.836911in}{2.612592in}}%
\pgfpathlineto{\pgfqpoint{1.837462in}{2.412963in}}%
\pgfpathlineto{\pgfqpoint{1.838542in}{2.621224in}}%
\pgfpathlineto{\pgfqpoint{1.838101in}{2.363138in}}%
\pgfpathlineto{\pgfqpoint{1.838652in}{2.497426in}}%
\pgfpathlineto{\pgfqpoint{1.838740in}{2.381822in}}%
\pgfpathlineto{\pgfqpoint{1.839467in}{2.707763in}}%
\pgfpathlineto{\pgfqpoint{1.839710in}{2.542443in}}%
\pgfpathlineto{\pgfqpoint{1.839842in}{2.708200in}}%
\pgfpathlineto{\pgfqpoint{1.840481in}{2.404222in}}%
\pgfpathlineto{\pgfqpoint{1.840834in}{2.585166in}}%
\pgfpathlineto{\pgfqpoint{1.840856in}{2.584511in}}%
\pgfpathlineto{\pgfqpoint{1.841738in}{2.378107in}}%
\pgfpathlineto{\pgfqpoint{1.841187in}{2.709402in}}%
\pgfpathlineto{\pgfqpoint{1.841958in}{2.632369in}}%
\pgfpathlineto{\pgfqpoint{1.842994in}{2.414930in}}%
\pgfpathlineto{\pgfqpoint{1.842068in}{2.687985in}}%
\pgfpathlineto{\pgfqpoint{1.843060in}{2.528020in}}%
\pgfpathlineto{\pgfqpoint{1.843655in}{2.614012in}}%
\pgfpathlineto{\pgfqpoint{1.843854in}{2.369912in}}%
\pgfpathlineto{\pgfqpoint{1.844140in}{2.477867in}}%
\pgfpathlineto{\pgfqpoint{1.844625in}{2.408265in}}%
\pgfpathlineto{\pgfqpoint{1.844228in}{2.595109in}}%
\pgfpathlineto{\pgfqpoint{1.844669in}{2.519060in}}%
\pgfpathlineto{\pgfqpoint{1.845551in}{2.686674in}}%
\pgfpathlineto{\pgfqpoint{1.845705in}{2.354834in}}%
\pgfpathlineto{\pgfqpoint{1.845771in}{2.509554in}}%
\pgfpathlineto{\pgfqpoint{1.846741in}{2.727758in}}%
\pgfpathlineto{\pgfqpoint{1.846080in}{2.419628in}}%
\pgfpathlineto{\pgfqpoint{1.846785in}{2.570962in}}%
\pgfpathlineto{\pgfqpoint{1.846807in}{2.388050in}}%
\pgfpathlineto{\pgfqpoint{1.847491in}{2.724371in}}%
\pgfpathlineto{\pgfqpoint{1.847887in}{2.607566in}}%
\pgfpathlineto{\pgfqpoint{1.848284in}{2.425419in}}%
\pgfpathlineto{\pgfqpoint{1.847976in}{2.695743in}}%
\pgfpathlineto{\pgfqpoint{1.848968in}{2.603086in}}%
\pgfpathlineto{\pgfqpoint{1.849519in}{2.774961in}}%
\pgfpathlineto{\pgfqpoint{1.849166in}{2.520262in}}%
\pgfpathlineto{\pgfqpoint{1.850026in}{2.526927in}}%
\pgfpathlineto{\pgfqpoint{1.851017in}{2.386630in}}%
\pgfpathlineto{\pgfqpoint{1.850466in}{2.698475in}}%
\pgfpathlineto{\pgfqpoint{1.851128in}{2.517421in}}%
\pgfpathlineto{\pgfqpoint{1.851436in}{2.709074in}}%
\pgfpathlineto{\pgfqpoint{1.852009in}{2.427933in}}%
\pgfpathlineto{\pgfqpoint{1.852406in}{2.677714in}}%
\pgfpathlineto{\pgfqpoint{1.853486in}{2.445852in}}%
\pgfpathlineto{\pgfqpoint{1.852516in}{2.732457in}}%
\pgfpathlineto{\pgfqpoint{1.853530in}{2.573693in}}%
\pgfpathlineto{\pgfqpoint{1.853596in}{2.720110in}}%
\pgfpathlineto{\pgfqpoint{1.854169in}{2.335275in}}%
\pgfpathlineto{\pgfqpoint{1.854632in}{2.580249in}}%
\pgfpathlineto{\pgfqpoint{1.855294in}{2.359860in}}%
\pgfpathlineto{\pgfqpoint{1.854963in}{2.734314in}}%
\pgfpathlineto{\pgfqpoint{1.855778in}{2.391765in}}%
\pgfpathlineto{\pgfqpoint{1.856462in}{2.622426in}}%
\pgfpathlineto{\pgfqpoint{1.855867in}{2.344781in}}%
\pgfpathlineto{\pgfqpoint{1.856881in}{2.520371in}}%
\pgfpathlineto{\pgfqpoint{1.856947in}{2.260319in}}%
\pgfpathlineto{\pgfqpoint{1.857828in}{2.667443in}}%
\pgfpathlineto{\pgfqpoint{1.857983in}{2.487373in}}%
\pgfpathlineto{\pgfqpoint{1.858534in}{2.686783in}}%
\pgfpathlineto{\pgfqpoint{1.858049in}{2.464427in}}%
\pgfpathlineto{\pgfqpoint{1.859085in}{2.535341in}}%
\pgfpathlineto{\pgfqpoint{1.859349in}{2.362591in}}%
\pgfpathlineto{\pgfqpoint{1.859261in}{2.720984in}}%
\pgfpathlineto{\pgfqpoint{1.860187in}{2.544956in}}%
\pgfpathlineto{\pgfqpoint{1.861223in}{2.361499in}}%
\pgfpathlineto{\pgfqpoint{1.860363in}{2.714428in}}%
\pgfpathlineto{\pgfqpoint{1.861333in}{2.410887in}}%
\pgfpathlineto{\pgfqpoint{1.861928in}{2.710603in}}%
\pgfpathlineto{\pgfqpoint{1.862215in}{2.387067in}}%
\pgfpathlineto{\pgfqpoint{1.862479in}{2.561455in}}%
\pgfpathlineto{\pgfqpoint{1.862678in}{2.437220in}}%
\pgfpathlineto{\pgfqpoint{1.863251in}{2.683068in}}%
\pgfpathlineto{\pgfqpoint{1.863537in}{2.624393in}}%
\pgfpathlineto{\pgfqpoint{1.863890in}{2.697710in}}%
\pgfpathlineto{\pgfqpoint{1.863692in}{2.442465in}}%
\pgfpathlineto{\pgfqpoint{1.864463in}{2.515345in}}%
\pgfpathlineto{\pgfqpoint{1.865389in}{2.418863in}}%
\pgfpathlineto{\pgfqpoint{1.864728in}{2.739231in}}%
\pgfpathlineto{\pgfqpoint{1.865499in}{2.519607in}}%
\pgfpathlineto{\pgfqpoint{1.865764in}{2.791132in}}%
\pgfpathlineto{\pgfqpoint{1.866182in}{2.430118in}}%
\pgfpathlineto{\pgfqpoint{1.866623in}{2.709729in}}%
\pgfpathlineto{\pgfqpoint{1.867725in}{2.447710in}}%
\pgfpathlineto{\pgfqpoint{1.866667in}{2.802278in}}%
\pgfpathlineto{\pgfqpoint{1.867769in}{2.505839in}}%
\pgfpathlineto{\pgfqpoint{1.868453in}{2.733768in}}%
\pgfpathlineto{\pgfqpoint{1.868254in}{2.378107in}}%
\pgfpathlineto{\pgfqpoint{1.868827in}{2.513051in}}%
\pgfpathlineto{\pgfqpoint{1.868916in}{2.303697in}}%
\pgfpathlineto{\pgfqpoint{1.869819in}{2.688641in}}%
\pgfpathlineto{\pgfqpoint{1.869929in}{2.566263in}}%
\pgfpathlineto{\pgfqpoint{1.870855in}{2.784467in}}%
\pgfpathlineto{\pgfqpoint{1.870194in}{2.370240in}}%
\pgfpathlineto{\pgfqpoint{1.871120in}{2.716941in}}%
\pgfpathlineto{\pgfqpoint{1.871891in}{2.465411in}}%
\pgfpathlineto{\pgfqpoint{1.871428in}{2.752780in}}%
\pgfpathlineto{\pgfqpoint{1.872266in}{2.659358in}}%
\pgfpathlineto{\pgfqpoint{1.872707in}{2.805337in}}%
\pgfpathlineto{\pgfqpoint{1.872398in}{2.419191in}}%
\pgfpathlineto{\pgfqpoint{1.873302in}{2.655752in}}%
\pgfpathlineto{\pgfqpoint{1.873919in}{2.484204in}}%
\pgfpathlineto{\pgfqpoint{1.874272in}{2.741089in}}%
\pgfpathlineto{\pgfqpoint{1.874404in}{2.600791in}}%
\pgfpathlineto{\pgfqpoint{1.874492in}{2.789166in}}%
\pgfpathlineto{\pgfqpoint{1.874691in}{2.501687in}}%
\pgfpathlineto{\pgfqpoint{1.875462in}{2.648540in}}%
\pgfpathlineto{\pgfqpoint{1.875837in}{2.468033in}}%
\pgfpathlineto{\pgfqpoint{1.875528in}{2.695525in}}%
\pgfpathlineto{\pgfqpoint{1.876586in}{2.579703in}}%
\pgfpathlineto{\pgfqpoint{1.877203in}{2.786325in}}%
\pgfpathlineto{\pgfqpoint{1.876674in}{2.492509in}}%
\pgfpathlineto{\pgfqpoint{1.877710in}{2.707763in}}%
\pgfpathlineto{\pgfqpoint{1.878349in}{2.470874in}}%
\pgfpathlineto{\pgfqpoint{1.878834in}{2.575114in}}%
\pgfpathlineto{\pgfqpoint{1.878923in}{2.755949in}}%
\pgfpathlineto{\pgfqpoint{1.879870in}{2.353195in}}%
\pgfpathlineto{\pgfqpoint{1.879937in}{2.530315in}}%
\pgfpathlineto{\pgfqpoint{1.880003in}{2.409576in}}%
\pgfpathlineto{\pgfqpoint{1.880620in}{2.715302in}}%
\pgfpathlineto{\pgfqpoint{1.880995in}{2.601338in}}%
\pgfpathlineto{\pgfqpoint{1.881435in}{2.394169in}}%
\pgfpathlineto{\pgfqpoint{1.882119in}{2.720328in}}%
\pgfpathlineto{\pgfqpoint{1.883199in}{2.363356in}}%
\pgfpathlineto{\pgfqpoint{1.883287in}{2.381057in}}%
\pgfpathlineto{\pgfqpoint{1.883463in}{2.724480in}}%
\pgfpathlineto{\pgfqpoint{1.884389in}{2.459729in}}%
\pgfpathlineto{\pgfqpoint{1.885293in}{2.355271in}}%
\pgfpathlineto{\pgfqpoint{1.884940in}{2.589646in}}%
\pgfpathlineto{\pgfqpoint{1.885425in}{2.413728in}}%
\pgfpathlineto{\pgfqpoint{1.886086in}{2.682741in}}%
\pgfpathlineto{\pgfqpoint{1.885778in}{2.395371in}}%
\pgfpathlineto{\pgfqpoint{1.886527in}{2.624611in}}%
\pgfpathlineto{\pgfqpoint{1.886747in}{2.404659in}}%
\pgfpathlineto{\pgfqpoint{1.887232in}{2.631276in}}%
\pgfpathlineto{\pgfqpoint{1.887629in}{2.597513in}}%
\pgfpathlineto{\pgfqpoint{1.888555in}{2.425966in}}%
\pgfpathlineto{\pgfqpoint{1.887761in}{2.693995in}}%
\pgfpathlineto{\pgfqpoint{1.888775in}{2.432085in}}%
\pgfpathlineto{\pgfqpoint{1.888864in}{2.728523in}}%
\pgfpathlineto{\pgfqpoint{1.889877in}{2.543317in}}%
\pgfpathlineto{\pgfqpoint{1.890032in}{2.399305in}}%
\pgfpathlineto{\pgfqpoint{1.890010in}{2.680337in}}%
\pgfpathlineto{\pgfqpoint{1.890891in}{2.506167in}}%
\pgfpathlineto{\pgfqpoint{1.891663in}{2.698038in}}%
\pgfpathlineto{\pgfqpoint{1.891068in}{2.330795in}}%
\pgfpathlineto{\pgfqpoint{1.891993in}{2.546595in}}%
\pgfpathlineto{\pgfqpoint{1.892258in}{2.460494in}}%
\pgfpathlineto{\pgfqpoint{1.892809in}{2.756386in}}%
\pgfpathlineto{\pgfqpoint{1.892875in}{2.687985in}}%
\pgfpathlineto{\pgfqpoint{1.892941in}{2.752671in}}%
\pgfpathlineto{\pgfqpoint{1.893404in}{2.371770in}}%
\pgfpathlineto{\pgfqpoint{1.893603in}{2.375485in}}%
\pgfpathlineto{\pgfqpoint{1.893625in}{2.332871in}}%
\pgfpathlineto{\pgfqpoint{1.893691in}{2.693558in}}%
\pgfpathlineto{\pgfqpoint{1.894616in}{2.510210in}}%
\pgfpathlineto{\pgfqpoint{1.895278in}{2.656735in}}%
\pgfpathlineto{\pgfqpoint{1.894815in}{2.353304in}}%
\pgfpathlineto{\pgfqpoint{1.895719in}{2.518842in}}%
\pgfpathlineto{\pgfqpoint{1.895895in}{2.636303in}}%
\pgfpathlineto{\pgfqpoint{1.896071in}{2.316372in}}%
\pgfpathlineto{\pgfqpoint{1.896777in}{2.447491in}}%
\pgfpathlineto{\pgfqpoint{1.896799in}{2.433396in}}%
\pgfpathlineto{\pgfqpoint{1.897107in}{2.644170in}}%
\pgfpathlineto{\pgfqpoint{1.897658in}{2.452080in}}%
\pgfpathlineto{\pgfqpoint{1.898518in}{2.787308in}}%
\pgfpathlineto{\pgfqpoint{1.898628in}{2.400616in}}%
\pgfpathlineto{\pgfqpoint{1.898782in}{2.600791in}}%
\pgfpathlineto{\pgfqpoint{1.899620in}{2.451425in}}%
\pgfpathlineto{\pgfqpoint{1.899245in}{2.739450in}}%
\pgfpathlineto{\pgfqpoint{1.899885in}{2.515017in}}%
\pgfpathlineto{\pgfqpoint{1.899929in}{2.655315in}}%
\pgfpathlineto{\pgfqpoint{1.900502in}{2.430883in}}%
\pgfpathlineto{\pgfqpoint{1.900987in}{2.588881in}}%
\pgfpathlineto{\pgfqpoint{1.901890in}{2.416678in}}%
\pgfpathlineto{\pgfqpoint{1.901295in}{2.727540in}}%
\pgfpathlineto{\pgfqpoint{1.902089in}{2.519497in}}%
\pgfpathlineto{\pgfqpoint{1.902574in}{2.665695in}}%
\pgfpathlineto{\pgfqpoint{1.903081in}{2.346092in}}%
\pgfpathlineto{\pgfqpoint{1.903169in}{2.469672in}}%
\pgfpathlineto{\pgfqpoint{1.903191in}{2.421486in}}%
\pgfpathlineto{\pgfqpoint{1.903676in}{2.756714in}}%
\pgfpathlineto{\pgfqpoint{1.904183in}{2.678042in}}%
\pgfpathlineto{\pgfqpoint{1.904205in}{2.736390in}}%
\pgfpathlineto{\pgfqpoint{1.904690in}{2.455249in}}%
\pgfpathlineto{\pgfqpoint{1.905263in}{2.603741in}}%
\pgfpathlineto{\pgfqpoint{1.905417in}{2.621224in}}%
\pgfpathlineto{\pgfqpoint{1.905527in}{2.417552in}}%
\pgfpathlineto{\pgfqpoint{1.905549in}{2.513051in}}%
\pgfpathlineto{\pgfqpoint{1.905748in}{2.647011in}}%
\pgfpathlineto{\pgfqpoint{1.906651in}{2.348496in}}%
\pgfpathlineto{\pgfqpoint{1.907643in}{2.697710in}}%
\pgfpathlineto{\pgfqpoint{1.907776in}{2.513816in}}%
\pgfpathlineto{\pgfqpoint{1.908767in}{2.373190in}}%
\pgfpathlineto{\pgfqpoint{1.908194in}{2.770590in}}%
\pgfpathlineto{\pgfqpoint{1.908900in}{2.427386in}}%
\pgfpathlineto{\pgfqpoint{1.909407in}{2.612701in}}%
\pgfpathlineto{\pgfqpoint{1.909318in}{2.347731in}}%
\pgfpathlineto{\pgfqpoint{1.910024in}{2.559489in}}%
\pgfpathlineto{\pgfqpoint{1.910178in}{2.358986in}}%
\pgfpathlineto{\pgfqpoint{1.910972in}{2.721749in}}%
\pgfpathlineto{\pgfqpoint{1.911104in}{2.613685in}}%
\pgfpathlineto{\pgfqpoint{1.911941in}{2.751906in}}%
\pgfpathlineto{\pgfqpoint{1.911479in}{2.339646in}}%
\pgfpathlineto{\pgfqpoint{1.912184in}{2.620896in}}%
\pgfpathlineto{\pgfqpoint{1.913022in}{2.337023in}}%
\pgfpathlineto{\pgfqpoint{1.912559in}{2.649305in}}%
\pgfpathlineto{\pgfqpoint{1.913352in}{2.436346in}}%
\pgfpathlineto{\pgfqpoint{1.914432in}{2.756167in}}%
\pgfpathlineto{\pgfqpoint{1.914190in}{2.364449in}}%
\pgfpathlineto{\pgfqpoint{1.914498in}{2.612155in}}%
\pgfpathlineto{\pgfqpoint{1.914653in}{2.692028in}}%
\pgfpathlineto{\pgfqpoint{1.914851in}{2.472622in}}%
\pgfpathlineto{\pgfqpoint{1.915578in}{2.590629in}}%
\pgfpathlineto{\pgfqpoint{1.915931in}{2.477321in}}%
\pgfpathlineto{\pgfqpoint{1.915645in}{2.730271in}}%
\pgfpathlineto{\pgfqpoint{1.916725in}{2.478304in}}%
\pgfpathlineto{\pgfqpoint{1.917430in}{2.792771in}}%
\pgfpathlineto{\pgfqpoint{1.917849in}{2.704157in}}%
\pgfpathlineto{\pgfqpoint{1.918444in}{2.439624in}}%
\pgfpathlineto{\pgfqpoint{1.917893in}{2.725682in}}%
\pgfpathlineto{\pgfqpoint{1.918995in}{2.469672in}}%
\pgfpathlineto{\pgfqpoint{1.919171in}{2.753873in}}%
\pgfpathlineto{\pgfqpoint{1.919833in}{2.451315in}}%
\pgfpathlineto{\pgfqpoint{1.920097in}{2.607675in}}%
\pgfpathlineto{\pgfqpoint{1.921045in}{2.478741in}}%
\pgfpathlineto{\pgfqpoint{1.921177in}{2.778021in}}%
\pgfpathlineto{\pgfqpoint{1.921221in}{2.535013in}}%
\pgfpathlineto{\pgfqpoint{1.921794in}{2.463007in}}%
\pgfpathlineto{\pgfqpoint{1.921706in}{2.790695in}}%
\pgfpathlineto{\pgfqpoint{1.922169in}{2.617290in}}%
\pgfpathlineto{\pgfqpoint{1.922962in}{2.735735in}}%
\pgfpathlineto{\pgfqpoint{1.922544in}{2.438313in}}%
\pgfpathlineto{\pgfqpoint{1.923249in}{2.608003in}}%
\pgfpathlineto{\pgfqpoint{1.923293in}{2.499829in}}%
\pgfpathlineto{\pgfqpoint{1.923491in}{2.805446in}}%
\pgfpathlineto{\pgfqpoint{1.924307in}{2.608331in}}%
\pgfpathlineto{\pgfqpoint{1.924726in}{2.801076in}}%
\pgfpathlineto{\pgfqpoint{1.924858in}{2.465520in}}%
\pgfpathlineto{\pgfqpoint{1.925409in}{2.649305in}}%
\pgfpathlineto{\pgfqpoint{1.925652in}{2.427933in}}%
\pgfpathlineto{\pgfqpoint{1.926467in}{2.727540in}}%
\pgfpathlineto{\pgfqpoint{1.926511in}{2.627343in}}%
\pgfpathlineto{\pgfqpoint{1.926533in}{2.626141in}}%
\pgfpathlineto{\pgfqpoint{1.926577in}{2.672142in}}%
\pgfpathlineto{\pgfqpoint{1.926599in}{2.739340in}}%
\pgfpathlineto{\pgfqpoint{1.927613in}{2.488903in}}%
\pgfpathlineto{\pgfqpoint{1.927657in}{2.591941in}}%
\pgfpathlineto{\pgfqpoint{1.928010in}{2.793646in}}%
\pgfpathlineto{\pgfqpoint{1.928341in}{2.485734in}}%
\pgfpathlineto{\pgfqpoint{1.928804in}{2.677387in}}%
\pgfpathlineto{\pgfqpoint{1.929641in}{2.467487in}}%
\pgfpathlineto{\pgfqpoint{1.929200in}{2.763816in}}%
\pgfpathlineto{\pgfqpoint{1.929928in}{2.562002in}}%
\pgfpathlineto{\pgfqpoint{1.929950in}{2.722186in}}%
\pgfpathlineto{\pgfqpoint{1.930457in}{2.383571in}}%
\pgfpathlineto{\pgfqpoint{1.931030in}{2.584948in}}%
\pgfpathlineto{\pgfqpoint{1.931250in}{2.392421in}}%
\pgfpathlineto{\pgfqpoint{1.931713in}{2.682304in}}%
\pgfpathlineto{\pgfqpoint{1.932176in}{2.525725in}}%
\pgfpathlineto{\pgfqpoint{1.932661in}{2.412307in}}%
\pgfpathlineto{\pgfqpoint{1.932507in}{2.736499in}}%
\pgfpathlineto{\pgfqpoint{1.933212in}{2.535669in}}%
\pgfpathlineto{\pgfqpoint{1.934314in}{2.762395in}}%
\pgfpathlineto{\pgfqpoint{1.933388in}{2.507915in}}%
\pgfpathlineto{\pgfqpoint{1.934358in}{2.682850in}}%
\pgfpathlineto{\pgfqpoint{1.935284in}{2.514799in}}%
\pgfpathlineto{\pgfqpoint{1.934512in}{2.803042in}}%
\pgfpathlineto{\pgfqpoint{1.935526in}{2.539821in}}%
\pgfpathlineto{\pgfqpoint{1.936099in}{2.727867in}}%
\pgfpathlineto{\pgfqpoint{1.936452in}{2.357456in}}%
\pgfpathlineto{\pgfqpoint{1.936629in}{2.628326in}}%
\pgfpathlineto{\pgfqpoint{1.936761in}{2.483221in}}%
\pgfpathlineto{\pgfqpoint{1.937047in}{2.738794in}}%
\pgfpathlineto{\pgfqpoint{1.937731in}{2.609532in}}%
\pgfpathlineto{\pgfqpoint{1.938348in}{2.862483in}}%
\pgfpathlineto{\pgfqpoint{1.938216in}{2.552496in}}%
\pgfpathlineto{\pgfqpoint{1.938855in}{2.698256in}}%
\pgfpathlineto{\pgfqpoint{1.939670in}{2.735298in}}%
\pgfpathlineto{\pgfqpoint{1.939009in}{2.496770in}}%
\pgfpathlineto{\pgfqpoint{1.939891in}{2.710385in}}%
\pgfpathlineto{\pgfqpoint{1.940155in}{2.733659in}}%
\pgfpathlineto{\pgfqpoint{1.941059in}{2.464427in}}%
\pgfpathlineto{\pgfqpoint{1.941941in}{2.741307in}}%
\pgfpathlineto{\pgfqpoint{1.941103in}{2.410122in}}%
\pgfpathlineto{\pgfqpoint{1.942161in}{2.599371in}}%
\pgfpathlineto{\pgfqpoint{1.943197in}{2.495350in}}%
\pgfpathlineto{\pgfqpoint{1.942624in}{2.799218in}}%
\pgfpathlineto{\pgfqpoint{1.943263in}{2.617618in}}%
\pgfpathlineto{\pgfqpoint{1.943528in}{2.777911in}}%
\pgfpathlineto{\pgfqpoint{1.943748in}{2.496442in}}%
\pgfpathlineto{\pgfqpoint{1.944387in}{2.660450in}}%
\pgfpathlineto{\pgfqpoint{1.944475in}{2.555664in}}%
\pgfpathlineto{\pgfqpoint{1.945181in}{2.777365in}}%
\pgfpathlineto{\pgfqpoint{1.945313in}{2.627343in}}%
\pgfpathlineto{\pgfqpoint{1.945357in}{2.823475in}}%
\pgfpathlineto{\pgfqpoint{1.945864in}{2.468798in}}%
\pgfpathlineto{\pgfqpoint{1.946393in}{2.616198in}}%
\pgfpathlineto{\pgfqpoint{1.946415in}{2.456669in}}%
\pgfpathlineto{\pgfqpoint{1.947385in}{2.707981in}}%
\pgfpathlineto{\pgfqpoint{1.947473in}{2.627780in}}%
\pgfpathlineto{\pgfqpoint{1.948267in}{2.740761in}}%
\pgfpathlineto{\pgfqpoint{1.948090in}{2.497753in}}%
\pgfpathlineto{\pgfqpoint{1.948597in}{2.720874in}}%
\pgfpathlineto{\pgfqpoint{1.948796in}{2.485625in}}%
\pgfpathlineto{\pgfqpoint{1.949501in}{2.782828in}}%
\pgfpathlineto{\pgfqpoint{1.949766in}{2.643077in}}%
\pgfpathlineto{\pgfqpoint{1.950515in}{2.834620in}}%
\pgfpathlineto{\pgfqpoint{1.950691in}{2.552277in}}%
\pgfpathlineto{\pgfqpoint{1.950868in}{2.628873in}}%
\pgfpathlineto{\pgfqpoint{1.951793in}{2.544629in}}%
\pgfpathlineto{\pgfqpoint{1.951220in}{2.792990in}}%
\pgfpathlineto{\pgfqpoint{1.951926in}{2.649196in}}%
\pgfpathlineto{\pgfqpoint{1.952829in}{2.829703in}}%
\pgfpathlineto{\pgfqpoint{1.952477in}{2.476010in}}%
\pgfpathlineto{\pgfqpoint{1.953028in}{2.635538in}}%
\pgfpathlineto{\pgfqpoint{1.953380in}{2.581888in}}%
\pgfpathlineto{\pgfqpoint{1.954196in}{2.828829in}}%
\pgfpathlineto{\pgfqpoint{1.955210in}{2.590848in}}%
\pgfpathlineto{\pgfqpoint{1.954769in}{2.901710in}}%
\pgfpathlineto{\pgfqpoint{1.955320in}{2.729397in}}%
\pgfpathlineto{\pgfqpoint{1.955607in}{2.528457in}}%
\pgfpathlineto{\pgfqpoint{1.956025in}{2.824896in}}%
\pgfpathlineto{\pgfqpoint{1.956444in}{2.640673in}}%
\pgfpathlineto{\pgfqpoint{1.957128in}{2.814625in}}%
\pgfpathlineto{\pgfqpoint{1.956510in}{2.575441in}}%
\pgfpathlineto{\pgfqpoint{1.957546in}{2.662963in}}%
\pgfpathlineto{\pgfqpoint{1.958450in}{2.511849in}}%
\pgfpathlineto{\pgfqpoint{1.958031in}{2.815171in}}%
\pgfpathlineto{\pgfqpoint{1.958516in}{2.662308in}}%
\pgfpathlineto{\pgfqpoint{1.958913in}{2.829703in}}%
\pgfpathlineto{\pgfqpoint{1.958781in}{2.564843in}}%
\pgfpathlineto{\pgfqpoint{1.959596in}{2.648213in}}%
\pgfpathlineto{\pgfqpoint{1.959971in}{2.513488in}}%
\pgfpathlineto{\pgfqpoint{1.960566in}{2.782937in}}%
\pgfpathlineto{\pgfqpoint{1.960720in}{2.570634in}}%
\pgfpathlineto{\pgfqpoint{1.961734in}{2.409904in}}%
\pgfpathlineto{\pgfqpoint{1.961867in}{2.820088in}}%
\pgfpathlineto{\pgfqpoint{1.962263in}{2.552823in}}%
\pgfpathlineto{\pgfqpoint{1.962792in}{2.855381in}}%
\pgfpathlineto{\pgfqpoint{1.962969in}{2.730490in}}%
\pgfpathlineto{\pgfqpoint{1.963542in}{2.787854in}}%
\pgfpathlineto{\pgfqpoint{1.963145in}{2.609423in}}%
\pgfpathlineto{\pgfqpoint{1.963806in}{2.636303in}}%
\pgfpathlineto{\pgfqpoint{1.963872in}{2.484314in}}%
\pgfpathlineto{\pgfqpoint{1.964027in}{2.876797in}}%
\pgfpathlineto{\pgfqpoint{1.964864in}{2.678151in}}%
\pgfpathlineto{\pgfqpoint{1.965944in}{2.849699in}}%
\pgfpathlineto{\pgfqpoint{1.965019in}{2.592706in}}%
\pgfpathlineto{\pgfqpoint{1.965966in}{2.801950in}}%
\pgfpathlineto{\pgfqpoint{1.966165in}{2.640018in}}%
\pgfpathlineto{\pgfqpoint{1.966275in}{2.891220in}}%
\pgfpathlineto{\pgfqpoint{1.967068in}{2.685909in}}%
\pgfpathlineto{\pgfqpoint{1.967752in}{2.867509in}}%
\pgfpathlineto{\pgfqpoint{1.967906in}{2.586150in}}%
\pgfpathlineto{\pgfqpoint{1.968171in}{2.674218in}}%
\pgfpathlineto{\pgfqpoint{1.969008in}{2.577736in}}%
\pgfpathlineto{\pgfqpoint{1.968810in}{2.872208in}}%
\pgfpathlineto{\pgfqpoint{1.969118in}{2.653020in}}%
\pgfpathlineto{\pgfqpoint{1.969736in}{2.872645in}}%
\pgfpathlineto{\pgfqpoint{1.970243in}{2.733221in}}%
\pgfpathlineto{\pgfqpoint{1.970507in}{2.838663in}}%
\pgfpathlineto{\pgfqpoint{1.970397in}{2.532500in}}%
\pgfpathlineto{\pgfqpoint{1.971301in}{2.727430in}}%
\pgfpathlineto{\pgfqpoint{1.971433in}{2.589646in}}%
\pgfpathlineto{\pgfqpoint{1.971631in}{2.855381in}}%
\pgfpathlineto{\pgfqpoint{1.972403in}{2.715520in}}%
\pgfpathlineto{\pgfqpoint{1.972667in}{2.816701in}}%
\pgfpathlineto{\pgfqpoint{1.973373in}{2.562220in}}%
\pgfpathlineto{\pgfqpoint{1.973527in}{2.757260in}}%
\pgfpathlineto{\pgfqpoint{1.974210in}{2.504528in}}%
\pgfpathlineto{\pgfqpoint{1.973615in}{2.826098in}}%
\pgfpathlineto{\pgfqpoint{1.974629in}{2.694978in}}%
\pgfpathlineto{\pgfqpoint{1.975400in}{2.876906in}}%
\pgfpathlineto{\pgfqpoint{1.974849in}{2.557085in}}%
\pgfpathlineto{\pgfqpoint{1.975753in}{2.741963in}}%
\pgfpathlineto{\pgfqpoint{1.976458in}{2.474480in}}%
\pgfpathlineto{\pgfqpoint{1.975885in}{2.759008in}}%
\pgfpathlineto{\pgfqpoint{1.976855in}{2.654550in}}%
\pgfpathlineto{\pgfqpoint{1.977031in}{2.522448in}}%
\pgfpathlineto{\pgfqpoint{1.977979in}{2.762068in}}%
\pgfpathlineto{\pgfqpoint{1.978178in}{2.552277in}}%
\pgfpathlineto{\pgfqpoint{1.978817in}{2.824786in}}%
\pgfpathlineto{\pgfqpoint{1.979103in}{2.646137in}}%
\pgfpathlineto{\pgfqpoint{1.979919in}{2.910014in}}%
\pgfpathlineto{\pgfqpoint{1.979258in}{2.562876in}}%
\pgfpathlineto{\pgfqpoint{1.980294in}{2.708200in}}%
\pgfpathlineto{\pgfqpoint{1.980889in}{2.602649in}}%
\pgfpathlineto{\pgfqpoint{1.981197in}{2.855381in}}%
\pgfpathlineto{\pgfqpoint{1.981396in}{2.716176in}}%
\pgfpathlineto{\pgfqpoint{1.981594in}{2.832107in}}%
\pgfpathlineto{\pgfqpoint{1.981925in}{2.587024in}}%
\pgfpathlineto{\pgfqpoint{1.982520in}{2.779004in}}%
\pgfpathlineto{\pgfqpoint{1.983556in}{2.557631in}}%
\pgfpathlineto{\pgfqpoint{1.983247in}{2.895700in}}%
\pgfpathlineto{\pgfqpoint{1.983644in}{2.693230in}}%
\pgfpathlineto{\pgfqpoint{1.983666in}{2.756277in}}%
\pgfpathlineto{\pgfqpoint{1.984239in}{2.481801in}}%
\pgfpathlineto{\pgfqpoint{1.984702in}{2.578392in}}%
\pgfpathlineto{\pgfqpoint{1.984834in}{2.734533in}}%
\pgfpathlineto{\pgfqpoint{1.985848in}{2.390673in}}%
\pgfpathlineto{\pgfqpoint{1.987061in}{2.740761in}}%
\pgfpathlineto{\pgfqpoint{1.986201in}{2.352539in}}%
\pgfpathlineto{\pgfqpoint{1.987105in}{2.714646in}}%
\pgfpathlineto{\pgfqpoint{1.988075in}{2.460166in}}%
\pgfpathlineto{\pgfqpoint{1.987700in}{2.737483in}}%
\pgfpathlineto{\pgfqpoint{1.988207in}{2.676294in}}%
\pgfpathlineto{\pgfqpoint{1.988670in}{2.799655in}}%
\pgfpathlineto{\pgfqpoint{1.988471in}{2.503545in}}%
\pgfpathlineto{\pgfqpoint{1.989243in}{2.752234in}}%
\pgfpathlineto{\pgfqpoint{1.989662in}{2.822601in}}%
\pgfpathlineto{\pgfqpoint{1.990389in}{2.423999in}}%
\pgfpathlineto{\pgfqpoint{1.991204in}{2.850573in}}%
\pgfpathlineto{\pgfqpoint{1.991557in}{2.704703in}}%
\pgfpathlineto{\pgfqpoint{1.991822in}{2.483112in}}%
\pgfpathlineto{\pgfqpoint{1.992196in}{2.825442in}}%
\pgfpathlineto{\pgfqpoint{1.992703in}{2.552933in}}%
\pgfpathlineto{\pgfqpoint{1.993188in}{2.738248in}}%
\pgfpathlineto{\pgfqpoint{1.993607in}{2.446508in}}%
\pgfpathlineto{\pgfqpoint{1.993827in}{2.609095in}}%
\pgfpathlineto{\pgfqpoint{1.994511in}{2.388597in}}%
\pgfpathlineto{\pgfqpoint{1.994092in}{2.682413in}}%
\pgfpathlineto{\pgfqpoint{1.994709in}{2.532172in}}%
\pgfpathlineto{\pgfqpoint{1.995767in}{2.751360in}}%
\pgfpathlineto{\pgfqpoint{1.995282in}{2.495131in}}%
\pgfpathlineto{\pgfqpoint{1.995833in}{2.741963in}}%
\pgfpathlineto{\pgfqpoint{1.996891in}{2.422906in}}%
\pgfpathlineto{\pgfqpoint{1.996010in}{2.810472in}}%
\pgfpathlineto{\pgfqpoint{1.996979in}{2.616853in}}%
\pgfpathlineto{\pgfqpoint{1.997993in}{2.733659in}}%
\pgfpathlineto{\pgfqpoint{1.997442in}{2.452736in}}%
\pgfpathlineto{\pgfqpoint{1.998104in}{2.666241in}}%
\pgfpathlineto{\pgfqpoint{1.998853in}{2.467596in}}%
\pgfpathlineto{\pgfqpoint{1.998985in}{2.716067in}}%
\pgfpathlineto{\pgfqpoint{1.999228in}{2.539493in}}%
\pgfpathlineto{\pgfqpoint{1.999977in}{2.735079in}}%
\pgfpathlineto{\pgfqpoint{2.000308in}{2.435363in}}%
\pgfpathlineto{\pgfqpoint{2.000484in}{2.727103in}}%
\pgfpathlineto{\pgfqpoint{2.001123in}{2.414165in}}%
\pgfpathlineto{\pgfqpoint{2.001498in}{2.564078in}}%
\pgfpathlineto{\pgfqpoint{2.002336in}{2.400944in}}%
\pgfpathlineto{\pgfqpoint{2.001851in}{2.722623in}}%
\pgfpathlineto{\pgfqpoint{2.002556in}{2.585822in}}%
\pgfpathlineto{\pgfqpoint{2.002578in}{2.747645in}}%
\pgfpathlineto{\pgfqpoint{2.003482in}{2.405752in}}%
\pgfpathlineto{\pgfqpoint{2.003658in}{2.575769in}}%
\pgfpathlineto{\pgfqpoint{2.004562in}{2.720110in}}%
\pgfpathlineto{\pgfqpoint{2.003812in}{2.448912in}}%
\pgfpathlineto{\pgfqpoint{2.004760in}{2.634445in}}%
\pgfpathlineto{\pgfqpoint{2.004848in}{2.483549in}}%
\pgfpathlineto{\pgfqpoint{2.005796in}{2.713226in}}%
\pgfpathlineto{\pgfqpoint{2.005818in}{2.644825in}}%
\pgfpathlineto{\pgfqpoint{2.005862in}{2.809489in}}%
\pgfpathlineto{\pgfqpoint{2.006325in}{2.428807in}}%
\pgfpathlineto{\pgfqpoint{2.006854in}{2.554790in}}%
\pgfpathlineto{\pgfqpoint{2.006876in}{2.459838in}}%
\pgfpathlineto{\pgfqpoint{2.007736in}{2.783702in}}%
\pgfpathlineto{\pgfqpoint{2.007934in}{2.593689in}}%
\pgfpathlineto{\pgfqpoint{2.007978in}{2.758571in}}%
\pgfpathlineto{\pgfqpoint{2.008728in}{2.472622in}}%
\pgfpathlineto{\pgfqpoint{2.009014in}{2.579484in}}%
\pgfpathlineto{\pgfqpoint{2.009036in}{2.494366in}}%
\pgfpathlineto{\pgfqpoint{2.009213in}{2.814297in}}%
\pgfpathlineto{\pgfqpoint{2.010094in}{2.693667in}}%
\pgfpathlineto{\pgfqpoint{2.011152in}{2.582762in}}%
\pgfpathlineto{\pgfqpoint{2.010579in}{2.824021in}}%
\pgfpathlineto{\pgfqpoint{2.011175in}{2.595437in}}%
\pgfpathlineto{\pgfqpoint{2.011197in}{2.776054in}}%
\pgfpathlineto{\pgfqpoint{2.011902in}{2.397557in}}%
\pgfpathlineto{\pgfqpoint{2.012255in}{2.567684in}}%
\pgfpathlineto{\pgfqpoint{2.012872in}{2.385974in}}%
\pgfpathlineto{\pgfqpoint{2.012431in}{2.688532in}}%
\pgfpathlineto{\pgfqpoint{2.013357in}{2.532391in}}%
\pgfpathlineto{\pgfqpoint{2.014437in}{2.776709in}}%
\pgfpathlineto{\pgfqpoint{2.013401in}{2.419847in}}%
\pgfpathlineto{\pgfqpoint{2.014503in}{2.602539in}}%
\pgfpathlineto{\pgfqpoint{2.015649in}{2.802715in}}%
\pgfpathlineto{\pgfqpoint{2.015208in}{2.549545in}}%
\pgfpathlineto{\pgfqpoint{2.015671in}{2.785232in}}%
\pgfpathlineto{\pgfqpoint{2.016619in}{2.524851in}}%
\pgfpathlineto{\pgfqpoint{2.016376in}{2.796924in}}%
\pgfpathlineto{\pgfqpoint{2.016817in}{2.580358in}}%
\pgfpathlineto{\pgfqpoint{2.017941in}{2.790258in}}%
\pgfpathlineto{\pgfqpoint{2.017523in}{2.484860in}}%
\pgfpathlineto{\pgfqpoint{2.017985in}{2.673125in}}%
\pgfpathlineto{\pgfqpoint{2.018845in}{2.447163in}}%
\pgfpathlineto{\pgfqpoint{2.018515in}{2.733221in}}%
\pgfpathlineto{\pgfqpoint{2.019110in}{2.550529in}}%
\pgfpathlineto{\pgfqpoint{2.019793in}{2.667116in}}%
\pgfpathlineto{\pgfqpoint{2.019154in}{2.381494in}}%
\pgfpathlineto{\pgfqpoint{2.020190in}{2.570962in}}%
\pgfpathlineto{\pgfqpoint{2.020300in}{2.497426in}}%
\pgfpathlineto{\pgfqpoint{2.021204in}{2.772011in}}%
\pgfpathlineto{\pgfqpoint{2.021270in}{2.615105in}}%
\pgfpathlineto{\pgfqpoint{2.021380in}{2.751141in}}%
\pgfpathlineto{\pgfqpoint{2.022262in}{2.404768in}}%
\pgfpathlineto{\pgfqpoint{2.022350in}{2.484860in}}%
\pgfpathlineto{\pgfqpoint{2.023077in}{2.397666in}}%
\pgfpathlineto{\pgfqpoint{2.022702in}{2.717706in}}%
\pgfpathlineto{\pgfqpoint{2.023452in}{2.452736in}}%
\pgfpathlineto{\pgfqpoint{2.023893in}{2.678042in}}%
\pgfpathlineto{\pgfqpoint{2.023496in}{2.393295in}}%
\pgfpathlineto{\pgfqpoint{2.024708in}{2.579594in}}%
\pgfpathlineto{\pgfqpoint{2.025678in}{2.387067in}}%
\pgfpathlineto{\pgfqpoint{2.025061in}{2.726119in}}%
\pgfpathlineto{\pgfqpoint{2.025832in}{2.450988in}}%
\pgfpathlineto{\pgfqpoint{2.026229in}{2.784904in}}%
\pgfpathlineto{\pgfqpoint{2.026780in}{2.410996in}}%
\pgfpathlineto{\pgfqpoint{2.026957in}{2.490105in}}%
\pgfpathlineto{\pgfqpoint{2.027574in}{2.307412in}}%
\pgfpathlineto{\pgfqpoint{2.027728in}{2.670175in}}%
\pgfpathlineto{\pgfqpoint{2.028059in}{2.501687in}}%
\pgfpathlineto{\pgfqpoint{2.029029in}{2.853851in}}%
\pgfpathlineto{\pgfqpoint{2.028213in}{2.417334in}}%
\pgfpathlineto{\pgfqpoint{2.029183in}{2.591941in}}%
\pgfpathlineto{\pgfqpoint{2.030109in}{2.758025in}}%
\pgfpathlineto{\pgfqpoint{2.029315in}{2.495568in}}%
\pgfpathlineto{\pgfqpoint{2.030351in}{2.699131in}}%
\pgfpathlineto{\pgfqpoint{2.030439in}{2.518623in}}%
\pgfpathlineto{\pgfqpoint{2.030902in}{2.815280in}}%
\pgfpathlineto{\pgfqpoint{2.031321in}{2.734642in}}%
\pgfpathlineto{\pgfqpoint{2.032136in}{2.831233in}}%
\pgfpathlineto{\pgfqpoint{2.031475in}{2.620787in}}%
\pgfpathlineto{\pgfqpoint{2.032379in}{2.670394in}}%
\pgfpathlineto{\pgfqpoint{2.033261in}{2.508680in}}%
\pgfpathlineto{\pgfqpoint{2.033437in}{2.776382in}}%
\pgfpathlineto{\pgfqpoint{2.033459in}{2.725136in}}%
\pgfpathlineto{\pgfqpoint{2.034385in}{2.937986in}}%
\pgfpathlineto{\pgfqpoint{2.033856in}{2.575223in}}%
\pgfpathlineto{\pgfqpoint{2.034517in}{2.683615in}}%
\pgfpathlineto{\pgfqpoint{2.034583in}{2.550529in}}%
\pgfpathlineto{\pgfqpoint{2.035024in}{2.824568in}}%
\pgfpathlineto{\pgfqpoint{2.035575in}{2.761740in}}%
\pgfpathlineto{\pgfqpoint{2.035840in}{2.910123in}}%
\pgfpathlineto{\pgfqpoint{2.036324in}{2.601338in}}%
\pgfpathlineto{\pgfqpoint{2.036677in}{2.777802in}}%
\pgfpathlineto{\pgfqpoint{2.037581in}{2.537308in}}%
\pgfpathlineto{\pgfqpoint{2.037008in}{2.850792in}}%
\pgfpathlineto{\pgfqpoint{2.037779in}{2.595219in}}%
\pgfpathlineto{\pgfqpoint{2.037978in}{2.850682in}}%
\pgfpathlineto{\pgfqpoint{2.038485in}{2.584073in}}%
\pgfpathlineto{\pgfqpoint{2.038903in}{2.734096in}}%
\pgfpathlineto{\pgfqpoint{2.039146in}{2.360188in}}%
\pgfpathlineto{\pgfqpoint{2.039278in}{2.786215in}}%
\pgfpathlineto{\pgfqpoint{2.040027in}{2.537745in}}%
\pgfpathlineto{\pgfqpoint{2.041063in}{2.737811in}}%
\pgfpathlineto{\pgfqpoint{2.040094in}{2.510756in}}%
\pgfpathlineto{\pgfqpoint{2.041152in}{2.675311in}}%
\pgfpathlineto{\pgfqpoint{2.041703in}{2.479506in}}%
\pgfpathlineto{\pgfqpoint{2.042166in}{2.738903in}}%
\pgfpathlineto{\pgfqpoint{2.042254in}{2.646792in}}%
\pgfpathlineto{\pgfqpoint{2.042584in}{2.849153in}}%
\pgfpathlineto{\pgfqpoint{2.042871in}{2.521246in}}%
\pgfpathlineto{\pgfqpoint{2.043334in}{2.565608in}}%
\pgfpathlineto{\pgfqpoint{2.043863in}{2.390782in}}%
\pgfpathlineto{\pgfqpoint{2.043510in}{2.687548in}}%
\pgfpathlineto{\pgfqpoint{2.044348in}{2.585057in}}%
\pgfpathlineto{\pgfqpoint{2.044678in}{2.748847in}}%
\pgfpathlineto{\pgfqpoint{2.045163in}{2.391001in}}%
\pgfpathlineto{\pgfqpoint{2.045450in}{2.560144in}}%
\pgfpathlineto{\pgfqpoint{2.045494in}{2.478413in}}%
\pgfpathlineto{\pgfqpoint{2.046376in}{2.771465in}}%
\pgfpathlineto{\pgfqpoint{2.046530in}{2.636084in}}%
\pgfpathlineto{\pgfqpoint{2.047301in}{2.774087in}}%
\pgfpathlineto{\pgfqpoint{2.046927in}{2.428151in}}%
\pgfpathlineto{\pgfqpoint{2.047412in}{2.601556in}}%
\pgfpathlineto{\pgfqpoint{2.047852in}{2.472622in}}%
\pgfpathlineto{\pgfqpoint{2.047632in}{2.708309in}}%
\pgfpathlineto{\pgfqpoint{2.048514in}{2.570852in}}%
\pgfpathlineto{\pgfqpoint{2.048800in}{2.756604in}}%
\pgfpathlineto{\pgfqpoint{2.049417in}{2.531954in}}%
\pgfpathlineto{\pgfqpoint{2.049594in}{2.608877in}}%
\pgfpathlineto{\pgfqpoint{2.049682in}{2.430227in}}%
\pgfpathlineto{\pgfqpoint{2.050608in}{2.798890in}}%
\pgfpathlineto{\pgfqpoint{2.050652in}{2.635756in}}%
\pgfpathlineto{\pgfqpoint{2.050762in}{2.826098in}}%
\pgfpathlineto{\pgfqpoint{2.051379in}{2.527255in}}%
\pgfpathlineto{\pgfqpoint{2.051754in}{2.734533in}}%
\pgfpathlineto{\pgfqpoint{2.052834in}{2.532828in}}%
\pgfpathlineto{\pgfqpoint{2.052283in}{2.851884in}}%
\pgfpathlineto{\pgfqpoint{2.052900in}{2.537963in}}%
\pgfpathlineto{\pgfqpoint{2.053253in}{2.767859in}}%
\pgfpathlineto{\pgfqpoint{2.053142in}{2.487264in}}%
\pgfpathlineto{\pgfqpoint{2.054046in}{2.632369in}}%
\pgfpathlineto{\pgfqpoint{2.054068in}{2.389252in}}%
\pgfpathlineto{\pgfqpoint{2.054663in}{2.762614in}}%
\pgfpathlineto{\pgfqpoint{2.055148in}{2.549764in}}%
\pgfpathlineto{\pgfqpoint{2.055611in}{2.485188in}}%
\pgfpathlineto{\pgfqpoint{2.056030in}{2.756495in}}%
\pgfpathlineto{\pgfqpoint{2.056228in}{2.609970in}}%
\pgfpathlineto{\pgfqpoint{2.056339in}{2.528239in}}%
\pgfpathlineto{\pgfqpoint{2.057507in}{2.831124in}}%
\pgfpathlineto{\pgfqpoint{2.058344in}{2.525398in}}%
\pgfpathlineto{\pgfqpoint{2.058609in}{2.722623in}}%
\pgfpathlineto{\pgfqpoint{2.058631in}{2.797798in}}%
\pgfpathlineto{\pgfqpoint{2.059380in}{2.498081in}}%
\pgfpathlineto{\pgfqpoint{2.059667in}{2.593252in}}%
\pgfpathlineto{\pgfqpoint{2.060549in}{2.458855in}}%
\pgfpathlineto{\pgfqpoint{2.060681in}{2.724699in}}%
\pgfpathlineto{\pgfqpoint{2.060747in}{2.583090in}}%
\pgfpathlineto{\pgfqpoint{2.061783in}{2.849699in}}%
\pgfpathlineto{\pgfqpoint{2.061232in}{2.503763in}}%
\pgfpathlineto{\pgfqpoint{2.061871in}{2.696399in}}%
\pgfpathlineto{\pgfqpoint{2.062378in}{2.515782in}}%
\pgfpathlineto{\pgfqpoint{2.062819in}{2.758680in}}%
\pgfpathlineto{\pgfqpoint{2.062973in}{2.640783in}}%
\pgfpathlineto{\pgfqpoint{2.063877in}{2.838991in}}%
\pgfpathlineto{\pgfqpoint{2.063348in}{2.572164in}}%
\pgfpathlineto{\pgfqpoint{2.064075in}{2.611718in}}%
\pgfpathlineto{\pgfqpoint{2.064494in}{2.817684in}}%
\pgfpathlineto{\pgfqpoint{2.064913in}{2.557959in}}%
\pgfpathlineto{\pgfqpoint{2.065199in}{2.674983in}}%
\pgfpathlineto{\pgfqpoint{2.065310in}{2.479069in}}%
\pgfpathlineto{\pgfqpoint{2.065662in}{2.837898in}}%
\pgfpathlineto{\pgfqpoint{2.066368in}{2.588772in}}%
\pgfpathlineto{\pgfqpoint{2.067382in}{2.458308in}}%
\pgfpathlineto{\pgfqpoint{2.067051in}{2.731910in}}%
\pgfpathlineto{\pgfqpoint{2.067470in}{2.552277in}}%
\pgfpathlineto{\pgfqpoint{2.068572in}{2.805118in}}%
\pgfpathlineto{\pgfqpoint{2.068462in}{2.522229in}}%
\pgfpathlineto{\pgfqpoint{2.068616in}{2.643623in}}%
\pgfpathlineto{\pgfqpoint{2.069167in}{2.459729in}}%
\pgfpathlineto{\pgfqpoint{2.069520in}{2.819760in}}%
\pgfpathlineto{\pgfqpoint{2.069674in}{2.659795in}}%
\pgfpathlineto{\pgfqpoint{2.069718in}{2.802387in}}%
\pgfpathlineto{\pgfqpoint{2.070534in}{2.502015in}}%
\pgfpathlineto{\pgfqpoint{2.070776in}{2.658484in}}%
\pgfpathlineto{\pgfqpoint{2.071195in}{2.534685in}}%
\pgfpathlineto{\pgfqpoint{2.070952in}{2.897557in}}%
\pgfpathlineto{\pgfqpoint{2.071856in}{2.682085in}}%
\pgfpathlineto{\pgfqpoint{2.072032in}{2.875376in}}%
\pgfpathlineto{\pgfqpoint{2.072539in}{2.555883in}}%
\pgfpathlineto{\pgfqpoint{2.072980in}{2.778785in}}%
\pgfpathlineto{\pgfqpoint{2.073950in}{2.504091in}}%
\pgfpathlineto{\pgfqpoint{2.074104in}{2.679244in}}%
\pgfpathlineto{\pgfqpoint{2.074854in}{2.498300in}}%
\pgfpathlineto{\pgfqpoint{2.074655in}{2.830359in}}%
\pgfpathlineto{\pgfqpoint{2.075140in}{2.658811in}}%
\pgfpathlineto{\pgfqpoint{2.075912in}{2.862592in}}%
\pgfpathlineto{\pgfqpoint{2.075515in}{2.542771in}}%
\pgfpathlineto{\pgfqpoint{2.076220in}{2.634882in}}%
\pgfpathlineto{\pgfqpoint{2.077168in}{2.523103in}}%
\pgfpathlineto{\pgfqpoint{2.076639in}{2.795066in}}%
\pgfpathlineto{\pgfqpoint{2.077278in}{2.674218in}}%
\pgfpathlineto{\pgfqpoint{2.077389in}{2.853633in}}%
\pgfpathlineto{\pgfqpoint{2.078314in}{2.491962in}}%
\pgfpathlineto{\pgfqpoint{2.078359in}{2.647448in}}%
\pgfpathlineto{\pgfqpoint{2.078645in}{2.531735in}}%
\pgfpathlineto{\pgfqpoint{2.078998in}{2.812767in}}%
\pgfpathlineto{\pgfqpoint{2.079439in}{2.712133in}}%
\pgfpathlineto{\pgfqpoint{2.079505in}{2.799437in}}%
\pgfpathlineto{\pgfqpoint{2.080629in}{2.520590in}}%
\pgfpathlineto{\pgfqpoint{2.080959in}{2.784904in}}%
\pgfpathlineto{\pgfqpoint{2.080783in}{2.514253in}}%
\pgfpathlineto{\pgfqpoint{2.081753in}{2.662417in}}%
\pgfpathlineto{\pgfqpoint{2.082414in}{2.586477in}}%
\pgfpathlineto{\pgfqpoint{2.082062in}{2.782828in}}%
\pgfpathlineto{\pgfqpoint{2.082569in}{2.753873in}}%
\pgfpathlineto{\pgfqpoint{2.082591in}{2.810691in}}%
\pgfpathlineto{\pgfqpoint{2.083516in}{2.520044in}}%
\pgfpathlineto{\pgfqpoint{2.083605in}{2.592050in}}%
\pgfpathlineto{\pgfqpoint{2.084729in}{2.752780in}}%
\pgfpathlineto{\pgfqpoint{2.083891in}{2.541241in}}%
\pgfpathlineto{\pgfqpoint{2.084773in}{2.750376in}}%
\pgfpathlineto{\pgfqpoint{2.085015in}{2.600245in}}%
\pgfpathlineto{\pgfqpoint{2.085588in}{2.882916in}}%
\pgfpathlineto{\pgfqpoint{2.085875in}{2.742728in}}%
\pgfpathlineto{\pgfqpoint{2.086051in}{2.577080in}}%
\pgfpathlineto{\pgfqpoint{2.086183in}{2.828392in}}%
\pgfpathlineto{\pgfqpoint{2.086470in}{2.796596in}}%
\pgfpathlineto{\pgfqpoint{2.087396in}{2.920503in}}%
\pgfpathlineto{\pgfqpoint{2.086999in}{2.640127in}}%
\pgfpathlineto{\pgfqpoint{2.087550in}{2.769716in}}%
\pgfpathlineto{\pgfqpoint{2.088630in}{2.565280in}}%
\pgfpathlineto{\pgfqpoint{2.088388in}{2.975792in}}%
\pgfpathlineto{\pgfqpoint{2.088674in}{2.613685in}}%
\pgfpathlineto{\pgfqpoint{2.088806in}{2.859751in}}%
\pgfpathlineto{\pgfqpoint{2.089291in}{2.489449in}}%
\pgfpathlineto{\pgfqpoint{2.089798in}{2.711478in}}%
\pgfpathlineto{\pgfqpoint{2.089931in}{2.595546in}}%
\pgfpathlineto{\pgfqpoint{2.090922in}{2.969454in}}%
\pgfpathlineto{\pgfqpoint{2.092025in}{2.495787in}}%
\pgfpathlineto{\pgfqpoint{2.092069in}{2.556866in}}%
\pgfpathlineto{\pgfqpoint{2.092686in}{2.863139in}}%
\pgfpathlineto{\pgfqpoint{2.093215in}{2.652583in}}%
\pgfpathlineto{\pgfqpoint{2.093303in}{2.537089in}}%
\pgfpathlineto{\pgfqpoint{2.094008in}{2.867837in}}%
\pgfpathlineto{\pgfqpoint{2.094758in}{2.987593in}}%
\pgfpathlineto{\pgfqpoint{2.094096in}{2.609751in}}%
\pgfpathlineto{\pgfqpoint{2.094956in}{2.744913in}}%
\pgfpathlineto{\pgfqpoint{2.095397in}{2.584729in}}%
\pgfpathlineto{\pgfqpoint{2.095221in}{2.854616in}}%
\pgfpathlineto{\pgfqpoint{2.096036in}{2.821290in}}%
\pgfpathlineto{\pgfqpoint{2.096190in}{2.831998in}}%
\pgfpathlineto{\pgfqpoint{2.096235in}{2.630512in}}%
\pgfpathlineto{\pgfqpoint{2.096389in}{2.747863in}}%
\pgfpathlineto{\pgfqpoint{2.097204in}{2.455140in}}%
\pgfpathlineto{\pgfqpoint{2.096742in}{2.780315in}}%
\pgfpathlineto{\pgfqpoint{2.097513in}{2.603632in}}%
\pgfpathlineto{\pgfqpoint{2.098549in}{2.707544in}}%
\pgfpathlineto{\pgfqpoint{2.097888in}{2.488794in}}%
\pgfpathlineto{\pgfqpoint{2.098615in}{2.659139in}}%
\pgfpathlineto{\pgfqpoint{2.099541in}{2.429025in}}%
\pgfpathlineto{\pgfqpoint{2.099387in}{2.788838in}}%
\pgfpathlineto{\pgfqpoint{2.099739in}{2.601775in}}%
\pgfpathlineto{\pgfqpoint{2.100158in}{2.734533in}}%
\pgfpathlineto{\pgfqpoint{2.100709in}{2.382806in}}%
\pgfpathlineto{\pgfqpoint{2.100753in}{2.496114in}}%
\pgfpathlineto{\pgfqpoint{2.100819in}{2.562220in}}%
\pgfpathlineto{\pgfqpoint{2.101965in}{2.238902in}}%
\pgfpathlineto{\pgfqpoint{2.102803in}{2.523103in}}%
\pgfpathlineto{\pgfqpoint{2.102891in}{2.163946in}}%
\pgfpathlineto{\pgfqpoint{2.103068in}{2.321835in}}%
\pgfpathlineto{\pgfqpoint{2.103310in}{2.461368in}}%
\pgfpathlineto{\pgfqpoint{2.104170in}{2.177495in}}%
\pgfpathlineto{\pgfqpoint{2.104941in}{2.423343in}}%
\pgfpathlineto{\pgfqpoint{2.104368in}{2.119256in}}%
\pgfpathlineto{\pgfqpoint{2.105294in}{2.256931in}}%
\pgfpathlineto{\pgfqpoint{2.105404in}{2.435690in}}%
\pgfpathlineto{\pgfqpoint{2.105845in}{2.128653in}}%
\pgfpathlineto{\pgfqpoint{2.106396in}{2.305992in}}%
\pgfpathlineto{\pgfqpoint{2.106969in}{2.043972in}}%
\pgfpathlineto{\pgfqpoint{2.106484in}{2.335056in}}%
\pgfpathlineto{\pgfqpoint{2.107520in}{2.189405in}}%
\pgfpathlineto{\pgfqpoint{2.107542in}{2.285340in}}%
\pgfpathlineto{\pgfqpoint{2.108049in}{1.976774in}}%
\pgfpathlineto{\pgfqpoint{2.108600in}{2.143076in}}%
\pgfpathlineto{\pgfqpoint{2.108688in}{1.906078in}}%
\pgfpathlineto{\pgfqpoint{2.109372in}{2.334073in}}%
\pgfpathlineto{\pgfqpoint{2.109702in}{2.215629in}}%
\pgfpathlineto{\pgfqpoint{2.109768in}{1.992617in}}%
\pgfpathlineto{\pgfqpoint{2.109879in}{2.250375in}}%
\pgfpathlineto{\pgfqpoint{2.110870in}{2.139361in}}%
\pgfpathlineto{\pgfqpoint{2.111179in}{2.311783in}}%
\pgfpathlineto{\pgfqpoint{2.111752in}{2.005948in}}%
\pgfpathlineto{\pgfqpoint{2.111906in}{2.025397in}}%
\pgfpathlineto{\pgfqpoint{2.112524in}{2.148430in}}%
\pgfpathlineto{\pgfqpoint{2.112149in}{1.874173in}}%
\pgfpathlineto{\pgfqpoint{2.112986in}{2.052276in}}%
\pgfpathlineto{\pgfqpoint{2.113449in}{1.900615in}}%
\pgfpathlineto{\pgfqpoint{2.113648in}{2.141656in}}%
\pgfpathlineto{\pgfqpoint{2.114089in}{2.047032in}}%
\pgfpathlineto{\pgfqpoint{2.114155in}{2.137176in}}%
\pgfpathlineto{\pgfqpoint{2.114838in}{1.739120in}}%
\pgfpathlineto{\pgfqpoint{2.115169in}{1.957324in}}%
\pgfpathlineto{\pgfqpoint{2.116227in}{1.896682in}}%
\pgfpathlineto{\pgfqpoint{2.115742in}{2.179025in}}%
\pgfpathlineto{\pgfqpoint{2.116271in}{1.947490in}}%
\pgfpathlineto{\pgfqpoint{2.116359in}{2.141328in}}%
\pgfpathlineto{\pgfqpoint{2.117329in}{1.841939in}}%
\pgfpathlineto{\pgfqpoint{2.117373in}{1.942573in}}%
\pgfpathlineto{\pgfqpoint{2.118144in}{2.035668in}}%
\pgfpathlineto{\pgfqpoint{2.117946in}{1.738465in}}%
\pgfpathlineto{\pgfqpoint{2.118409in}{1.962023in}}%
\pgfpathlineto{\pgfqpoint{2.118497in}{1.745785in}}%
\pgfpathlineto{\pgfqpoint{2.119313in}{2.225026in}}%
\pgfpathlineto{\pgfqpoint{2.119489in}{1.958526in}}%
\pgfpathlineto{\pgfqpoint{2.119797in}{2.184379in}}%
\pgfpathlineto{\pgfqpoint{2.120216in}{1.816480in}}%
\pgfpathlineto{\pgfqpoint{2.120569in}{1.943666in}}%
\pgfpathlineto{\pgfqpoint{2.121318in}{1.803806in}}%
\pgfpathlineto{\pgfqpoint{2.120878in}{2.031953in}}%
\pgfpathlineto{\pgfqpoint{2.121693in}{1.879308in}}%
\pgfpathlineto{\pgfqpoint{2.121715in}{1.878653in}}%
\pgfpathlineto{\pgfqpoint{2.121737in}{1.899741in}}%
\pgfpathlineto{\pgfqpoint{2.122090in}{2.168972in}}%
\pgfpathlineto{\pgfqpoint{2.122266in}{1.818775in}}%
\pgfpathlineto{\pgfqpoint{2.122795in}{1.963115in}}%
\pgfpathlineto{\pgfqpoint{2.123677in}{1.834400in}}%
\pgfpathlineto{\pgfqpoint{2.123060in}{2.088662in}}%
\pgfpathlineto{\pgfqpoint{2.123875in}{1.906516in}}%
\pgfpathlineto{\pgfqpoint{2.124294in}{2.180882in}}%
\pgfpathlineto{\pgfqpoint{2.124955in}{1.876030in}}%
\pgfpathlineto{\pgfqpoint{2.124999in}{2.075222in}}%
\pgfpathlineto{\pgfqpoint{2.125881in}{1.834291in}}%
\pgfpathlineto{\pgfqpoint{2.125352in}{2.103522in}}%
\pgfpathlineto{\pgfqpoint{2.126124in}{1.958854in}}%
\pgfpathlineto{\pgfqpoint{2.127027in}{2.043644in}}%
\pgfpathlineto{\pgfqpoint{2.126630in}{1.727319in}}%
\pgfpathlineto{\pgfqpoint{2.127137in}{1.867508in}}%
\pgfpathlineto{\pgfqpoint{2.127755in}{1.625811in}}%
\pgfpathlineto{\pgfqpoint{2.127336in}{2.018076in}}%
\pgfpathlineto{\pgfqpoint{2.128262in}{1.771026in}}%
\pgfpathlineto{\pgfqpoint{2.128989in}{1.942683in}}%
\pgfpathlineto{\pgfqpoint{2.128835in}{1.621004in}}%
\pgfpathlineto{\pgfqpoint{2.129342in}{1.855488in}}%
\pgfpathlineto{\pgfqpoint{2.130267in}{1.615977in}}%
\pgfpathlineto{\pgfqpoint{2.129386in}{1.918316in}}%
\pgfpathlineto{\pgfqpoint{2.130488in}{1.650615in}}%
\pgfpathlineto{\pgfqpoint{2.130510in}{1.574129in}}%
\pgfpathlineto{\pgfqpoint{2.131347in}{1.884662in}}%
\pgfpathlineto{\pgfqpoint{2.131590in}{1.667332in}}%
\pgfpathlineto{\pgfqpoint{2.131722in}{1.914383in}}%
\pgfpathlineto{\pgfqpoint{2.132582in}{1.550746in}}%
\pgfpathlineto{\pgfqpoint{2.132736in}{1.828500in}}%
\pgfpathlineto{\pgfqpoint{2.133023in}{1.580138in}}%
\pgfpathlineto{\pgfqpoint{2.133287in}{1.860078in}}%
\pgfpathlineto{\pgfqpoint{2.133838in}{1.738683in}}%
\pgfpathlineto{\pgfqpoint{2.134213in}{1.835383in}}%
\pgfpathlineto{\pgfqpoint{2.134720in}{1.523538in}}%
\pgfpathlineto{\pgfqpoint{2.134896in}{1.682520in}}%
\pgfpathlineto{\pgfqpoint{2.135205in}{1.541021in}}%
\pgfpathlineto{\pgfqpoint{2.135337in}{1.796703in}}%
\pgfpathlineto{\pgfqpoint{2.136020in}{1.652800in}}%
\pgfpathlineto{\pgfqpoint{2.136439in}{1.503106in}}%
\pgfpathlineto{\pgfqpoint{2.136682in}{1.893404in}}%
\pgfpathlineto{\pgfqpoint{2.137078in}{1.580685in}}%
\pgfpathlineto{\pgfqpoint{2.137872in}{1.884116in}}%
\pgfpathlineto{\pgfqpoint{2.137497in}{1.529876in}}%
\pgfpathlineto{\pgfqpoint{2.138180in}{1.589207in}}%
\pgfpathlineto{\pgfqpoint{2.139150in}{1.851118in}}%
\pgfpathlineto{\pgfqpoint{2.138225in}{1.563202in}}%
\pgfpathlineto{\pgfqpoint{2.139349in}{1.751030in}}%
\pgfpathlineto{\pgfqpoint{2.140098in}{1.497752in}}%
\pgfpathlineto{\pgfqpoint{2.139393in}{1.838880in}}%
\pgfpathlineto{\pgfqpoint{2.140495in}{1.644605in}}%
\pgfpathlineto{\pgfqpoint{2.140627in}{1.783701in}}%
\pgfpathlineto{\pgfqpoint{2.141046in}{1.529002in}}%
\pgfpathlineto{\pgfqpoint{2.141597in}{1.619365in}}%
\pgfpathlineto{\pgfqpoint{2.141817in}{1.854505in}}%
\pgfpathlineto{\pgfqpoint{2.142060in}{1.562874in}}%
\pgfpathlineto{\pgfqpoint{2.142545in}{1.582761in}}%
\pgfpathlineto{\pgfqpoint{2.142721in}{1.527035in}}%
\pgfpathlineto{\pgfqpoint{2.142986in}{1.836258in}}%
\pgfpathlineto{\pgfqpoint{2.143581in}{1.745021in}}%
\pgfpathlineto{\pgfqpoint{2.143801in}{1.548997in}}%
\pgfpathlineto{\pgfqpoint{2.144352in}{1.889252in}}%
\pgfpathlineto{\pgfqpoint{2.144705in}{1.623735in}}%
\pgfpathlineto{\pgfqpoint{2.145322in}{1.842814in}}%
\pgfpathlineto{\pgfqpoint{2.145454in}{1.589317in}}%
\pgfpathlineto{\pgfqpoint{2.145807in}{1.606362in}}%
\pgfpathlineto{\pgfqpoint{2.146534in}{1.880729in}}%
\pgfpathlineto{\pgfqpoint{2.145873in}{1.546375in}}%
\pgfpathlineto{\pgfqpoint{2.146909in}{1.681756in}}%
\pgfpathlineto{\pgfqpoint{2.147901in}{1.507804in}}%
\pgfpathlineto{\pgfqpoint{2.147659in}{1.838552in}}%
\pgfpathlineto{\pgfqpoint{2.147989in}{1.566371in}}%
\pgfpathlineto{\pgfqpoint{2.148232in}{1.805226in}}%
\pgfpathlineto{\pgfqpoint{2.148386in}{1.486388in}}%
\pgfpathlineto{\pgfqpoint{2.149091in}{1.662197in}}%
\pgfpathlineto{\pgfqpoint{2.150061in}{1.449128in}}%
\pgfpathlineto{\pgfqpoint{2.149664in}{1.827079in}}%
\pgfpathlineto{\pgfqpoint{2.150193in}{1.638923in}}%
\pgfpathlineto{\pgfqpoint{2.150722in}{1.915038in}}%
\pgfpathlineto{\pgfqpoint{2.150326in}{1.536323in}}%
\pgfpathlineto{\pgfqpoint{2.151362in}{1.701642in}}%
\pgfpathlineto{\pgfqpoint{2.151406in}{1.661760in}}%
\pgfpathlineto{\pgfqpoint{2.151472in}{1.702516in}}%
\pgfpathlineto{\pgfqpoint{2.151494in}{1.550746in}}%
\pgfpathlineto{\pgfqpoint{2.152155in}{1.830139in}}%
\pgfpathlineto{\pgfqpoint{2.152552in}{1.732455in}}%
\pgfpathlineto{\pgfqpoint{2.152794in}{1.925200in}}%
\pgfpathlineto{\pgfqpoint{2.152949in}{1.509443in}}%
\pgfpathlineto{\pgfqpoint{2.153654in}{1.738574in}}%
\pgfpathlineto{\pgfqpoint{2.154734in}{1.506056in}}%
\pgfpathlineto{\pgfqpoint{2.154073in}{1.950768in}}%
\pgfpathlineto{\pgfqpoint{2.154822in}{1.540038in}}%
\pgfpathlineto{\pgfqpoint{2.155704in}{1.859968in}}%
\pgfpathlineto{\pgfqpoint{2.155946in}{1.856909in}}%
\pgfpathlineto{\pgfqpoint{2.156409in}{1.601227in}}%
\pgfpathlineto{\pgfqpoint{2.157026in}{1.926620in}}%
\pgfpathlineto{\pgfqpoint{2.157070in}{1.733657in}}%
\pgfpathlineto{\pgfqpoint{2.157225in}{1.882259in}}%
\pgfpathlineto{\pgfqpoint{2.157445in}{1.622752in}}%
\pgfpathlineto{\pgfqpoint{2.157622in}{1.676729in}}%
\pgfpathlineto{\pgfqpoint{2.158327in}{1.577516in}}%
\pgfpathlineto{\pgfqpoint{2.158062in}{1.919409in}}%
\pgfpathlineto{\pgfqpoint{2.158724in}{1.674544in}}%
\pgfpathlineto{\pgfqpoint{2.159186in}{1.845654in}}%
\pgfpathlineto{\pgfqpoint{2.159319in}{1.537087in}}%
\pgfpathlineto{\pgfqpoint{2.159826in}{1.734203in}}%
\pgfpathlineto{\pgfqpoint{2.160796in}{1.534028in}}%
\pgfpathlineto{\pgfqpoint{2.160333in}{1.881603in}}%
\pgfpathlineto{\pgfqpoint{2.160928in}{1.731034in}}%
\pgfpathlineto{\pgfqpoint{2.160950in}{1.804570in}}%
\pgfpathlineto{\pgfqpoint{2.161258in}{1.487044in}}%
\pgfpathlineto{\pgfqpoint{2.161986in}{1.677822in}}%
\pgfpathlineto{\pgfqpoint{2.162493in}{1.483110in}}%
\pgfpathlineto{\pgfqpoint{2.162559in}{1.777363in}}%
\pgfpathlineto{\pgfqpoint{2.163110in}{1.562546in}}%
\pgfpathlineto{\pgfqpoint{2.163639in}{1.773867in}}%
\pgfpathlineto{\pgfqpoint{2.163374in}{1.468687in}}%
\pgfpathlineto{\pgfqpoint{2.164234in}{1.628215in}}%
\pgfpathlineto{\pgfqpoint{2.165138in}{1.551401in}}%
\pgfpathlineto{\pgfqpoint{2.164851in}{1.960493in}}%
\pgfpathlineto{\pgfqpoint{2.165248in}{1.812328in}}%
\pgfpathlineto{\pgfqpoint{2.165270in}{1.910449in}}%
\pgfpathlineto{\pgfqpoint{2.166284in}{1.519605in}}%
\pgfpathlineto{\pgfqpoint{2.166350in}{1.781406in}}%
\pgfpathlineto{\pgfqpoint{2.167055in}{1.535011in}}%
\pgfpathlineto{\pgfqpoint{2.166482in}{1.798124in}}%
\pgfpathlineto{\pgfqpoint{2.167474in}{1.687984in}}%
\pgfpathlineto{\pgfqpoint{2.167761in}{1.469343in}}%
\pgfpathlineto{\pgfqpoint{2.168466in}{1.798998in}}%
\pgfpathlineto{\pgfqpoint{2.168665in}{1.558067in}}%
\pgfpathlineto{\pgfqpoint{2.169436in}{1.767857in}}%
\pgfpathlineto{\pgfqpoint{2.169590in}{1.473932in}}%
\pgfpathlineto{\pgfqpoint{2.169723in}{1.675746in}}%
\pgfpathlineto{\pgfqpoint{2.170406in}{1.446506in}}%
\pgfpathlineto{\pgfqpoint{2.169921in}{1.754308in}}%
\pgfpathlineto{\pgfqpoint{2.170825in}{1.653128in}}%
\pgfpathlineto{\pgfqpoint{2.170847in}{1.675637in}}%
\pgfpathlineto{\pgfqpoint{2.171398in}{1.384771in}}%
\pgfpathlineto{\pgfqpoint{2.171772in}{1.635536in}}%
\pgfpathlineto{\pgfqpoint{2.172191in}{1.398757in}}%
\pgfpathlineto{\pgfqpoint{2.172257in}{1.728194in}}%
\pgfpathlineto{\pgfqpoint{2.172897in}{1.462240in}}%
\pgfpathlineto{\pgfqpoint{2.173800in}{1.820633in}}%
\pgfpathlineto{\pgfqpoint{2.173051in}{1.415256in}}%
\pgfpathlineto{\pgfqpoint{2.174065in}{1.706996in}}%
\pgfpathlineto{\pgfqpoint{2.174704in}{1.411650in}}%
\pgfpathlineto{\pgfqpoint{2.175211in}{1.454920in}}%
\pgfpathlineto{\pgfqpoint{2.175630in}{1.692464in}}%
\pgfpathlineto{\pgfqpoint{2.176137in}{1.394933in}}%
\pgfpathlineto{\pgfqpoint{2.176313in}{1.512830in}}%
\pgfpathlineto{\pgfqpoint{2.176401in}{1.413398in}}%
\pgfpathlineto{\pgfqpoint{2.177063in}{1.703172in}}%
\pgfpathlineto{\pgfqpoint{2.177239in}{1.666021in}}%
\pgfpathlineto{\pgfqpoint{2.177503in}{1.805445in}}%
\pgfpathlineto{\pgfqpoint{2.177371in}{1.459509in}}%
\pgfpathlineto{\pgfqpoint{2.178363in}{1.757258in}}%
\pgfpathlineto{\pgfqpoint{2.178495in}{1.417441in}}%
\pgfpathlineto{\pgfqpoint{2.178738in}{1.782171in}}%
\pgfpathlineto{\pgfqpoint{2.179509in}{1.532826in}}%
\pgfpathlineto{\pgfqpoint{2.180545in}{1.858657in}}%
\pgfpathlineto{\pgfqpoint{2.180347in}{1.496659in}}%
\pgfpathlineto{\pgfqpoint{2.180633in}{1.626686in}}%
\pgfpathlineto{\pgfqpoint{2.180986in}{1.773102in}}%
\pgfpathlineto{\pgfqpoint{2.180854in}{1.489229in}}%
\pgfpathlineto{\pgfqpoint{2.181559in}{1.595654in}}%
\pgfpathlineto{\pgfqpoint{2.182463in}{1.394605in}}%
\pgfpathlineto{\pgfqpoint{2.182088in}{1.729286in}}%
\pgfpathlineto{\pgfqpoint{2.182683in}{1.494255in}}%
\pgfpathlineto{\pgfqpoint{2.183190in}{1.366523in}}%
\pgfpathlineto{\pgfqpoint{2.182749in}{1.663508in}}%
\pgfpathlineto{\pgfqpoint{2.183653in}{1.498298in}}%
\pgfpathlineto{\pgfqpoint{2.183697in}{1.730379in}}%
\pgfpathlineto{\pgfqpoint{2.184380in}{1.341939in}}%
\pgfpathlineto{\pgfqpoint{2.184755in}{1.464972in}}%
\pgfpathlineto{\pgfqpoint{2.185328in}{1.410120in}}%
\pgfpathlineto{\pgfqpoint{2.185505in}{1.744802in}}%
\pgfpathlineto{\pgfqpoint{2.185791in}{1.519168in}}%
\pgfpathlineto{\pgfqpoint{2.185990in}{1.647337in}}%
\pgfpathlineto{\pgfqpoint{2.186474in}{1.367288in}}%
\pgfpathlineto{\pgfqpoint{2.186915in}{1.580357in}}%
\pgfpathlineto{\pgfqpoint{2.187753in}{1.656843in}}%
\pgfpathlineto{\pgfqpoint{2.186981in}{1.378324in}}%
\pgfpathlineto{\pgfqpoint{2.187929in}{1.541240in}}%
\pgfpathlineto{\pgfqpoint{2.188304in}{1.292441in}}%
\pgfpathlineto{\pgfqpoint{2.188150in}{1.589317in}}%
\pgfpathlineto{\pgfqpoint{2.189097in}{1.386847in}}%
\pgfpathlineto{\pgfqpoint{2.190222in}{1.638049in}}%
\pgfpathlineto{\pgfqpoint{2.189803in}{1.269277in}}%
\pgfpathlineto{\pgfqpoint{2.190244in}{1.508241in}}%
\pgfpathlineto{\pgfqpoint{2.190266in}{1.510427in}}%
\pgfpathlineto{\pgfqpoint{2.190288in}{1.463661in}}%
\pgfpathlineto{\pgfqpoint{2.190310in}{1.470545in}}%
\pgfpathlineto{\pgfqpoint{2.191236in}{1.270806in}}%
\pgfpathlineto{\pgfqpoint{2.190883in}{1.563202in}}%
\pgfpathlineto{\pgfqpoint{2.191434in}{1.316480in}}%
\pgfpathlineto{\pgfqpoint{2.192426in}{1.544627in}}%
\pgfpathlineto{\pgfqpoint{2.191588in}{1.228411in}}%
\pgfpathlineto{\pgfqpoint{2.192536in}{1.504089in}}%
\pgfpathlineto{\pgfqpoint{2.193594in}{1.265234in}}%
\pgfpathlineto{\pgfqpoint{2.193396in}{1.602756in}}%
\pgfpathlineto{\pgfqpoint{2.193660in}{1.388923in}}%
\pgfpathlineto{\pgfqpoint{2.193726in}{1.549435in}}%
\pgfpathlineto{\pgfqpoint{2.194388in}{1.241305in}}%
\pgfpathlineto{\pgfqpoint{2.194740in}{1.381930in}}%
\pgfpathlineto{\pgfqpoint{2.195335in}{1.230815in}}%
\pgfpathlineto{\pgfqpoint{2.195512in}{1.613137in}}%
\pgfpathlineto{\pgfqpoint{2.195864in}{1.298014in}}%
\pgfpathlineto{\pgfqpoint{2.196966in}{1.437983in}}%
\pgfpathlineto{\pgfqpoint{2.196526in}{1.172467in}}%
\pgfpathlineto{\pgfqpoint{2.197033in}{1.407389in}}%
\pgfpathlineto{\pgfqpoint{2.197518in}{1.135754in}}%
\pgfpathlineto{\pgfqpoint{2.197782in}{1.567026in}}%
\pgfpathlineto{\pgfqpoint{2.198135in}{1.287961in}}%
\pgfpathlineto{\pgfqpoint{2.199083in}{1.563748in}}%
\pgfpathlineto{\pgfqpoint{2.198950in}{1.210929in}}%
\pgfpathlineto{\pgfqpoint{2.199259in}{1.451314in}}%
\pgfpathlineto{\pgfqpoint{2.200008in}{1.248298in}}%
\pgfpathlineto{\pgfqpoint{2.199656in}{1.471528in}}%
\pgfpathlineto{\pgfqpoint{2.200361in}{1.426292in}}%
\pgfpathlineto{\pgfqpoint{2.200427in}{1.548233in}}%
\pgfpathlineto{\pgfqpoint{2.200978in}{1.191370in}}%
\pgfpathlineto{\pgfqpoint{2.201397in}{1.411213in}}%
\pgfpathlineto{\pgfqpoint{2.201507in}{1.233437in}}%
\pgfpathlineto{\pgfqpoint{2.202190in}{1.536541in}}%
\pgfpathlineto{\pgfqpoint{2.202521in}{1.299980in}}%
\pgfpathlineto{\pgfqpoint{2.203381in}{1.585929in}}%
\pgfpathlineto{\pgfqpoint{2.202808in}{1.248625in}}%
\pgfpathlineto{\pgfqpoint{2.203645in}{1.463552in}}%
\pgfpathlineto{\pgfqpoint{2.203689in}{1.304460in}}%
\pgfpathlineto{\pgfqpoint{2.204042in}{1.693228in}}%
\pgfpathlineto{\pgfqpoint{2.204747in}{1.356471in}}%
\pgfpathlineto{\pgfqpoint{2.204880in}{1.667879in}}%
\pgfpathlineto{\pgfqpoint{2.205012in}{1.355706in}}%
\pgfpathlineto{\pgfqpoint{2.205871in}{1.548888in}}%
\pgfpathlineto{\pgfqpoint{2.206555in}{1.265562in}}%
\pgfpathlineto{\pgfqpoint{2.206996in}{1.431209in}}%
\pgfpathlineto{\pgfqpoint{2.207018in}{1.531515in}}%
\pgfpathlineto{\pgfqpoint{2.207326in}{1.184049in}}%
\pgfpathlineto{\pgfqpoint{2.208054in}{1.368709in}}%
\pgfpathlineto{\pgfqpoint{2.208715in}{1.149631in}}%
\pgfpathlineto{\pgfqpoint{2.208230in}{1.488355in}}%
\pgfpathlineto{\pgfqpoint{2.209156in}{1.281733in}}%
\pgfpathlineto{\pgfqpoint{2.209398in}{1.465518in}}%
\pgfpathlineto{\pgfqpoint{2.210103in}{1.136191in}}%
\pgfpathlineto{\pgfqpoint{2.210236in}{1.271899in}}%
\pgfpathlineto{\pgfqpoint{2.210258in}{1.128979in}}%
\pgfpathlineto{\pgfqpoint{2.211228in}{1.460055in}}%
\pgfpathlineto{\pgfqpoint{2.211338in}{1.246877in}}%
\pgfpathlineto{\pgfqpoint{2.211779in}{1.348057in}}%
\pgfpathlineto{\pgfqpoint{2.212286in}{1.089316in}}%
\pgfpathlineto{\pgfqpoint{2.212396in}{1.209399in}}%
\pgfpathlineto{\pgfqpoint{2.213211in}{1.074346in}}%
\pgfpathlineto{\pgfqpoint{2.212859in}{1.332542in}}%
\pgfpathlineto{\pgfqpoint{2.213498in}{1.222074in}}%
\pgfpathlineto{\pgfqpoint{2.213939in}{1.088769in}}%
\pgfpathlineto{\pgfqpoint{2.214556in}{1.308940in}}%
\pgfpathlineto{\pgfqpoint{2.214578in}{1.185470in}}%
\pgfpathlineto{\pgfqpoint{2.214622in}{1.360732in}}%
\pgfpathlineto{\pgfqpoint{2.215548in}{0.975133in}}%
\pgfpathlineto{\pgfqpoint{2.215658in}{1.083415in}}%
\pgfpathlineto{\pgfqpoint{2.215702in}{1.087349in}}%
\pgfpathlineto{\pgfqpoint{2.216011in}{1.289382in}}%
\pgfpathlineto{\pgfqpoint{2.215856in}{1.007039in}}%
\pgfpathlineto{\pgfqpoint{2.216826in}{1.197708in}}%
\pgfpathlineto{\pgfqpoint{2.217796in}{1.037633in}}%
\pgfpathlineto{\pgfqpoint{2.217686in}{1.230597in}}%
\pgfpathlineto{\pgfqpoint{2.217928in}{1.204701in}}%
\pgfpathlineto{\pgfqpoint{2.217950in}{1.205902in}}%
\pgfpathlineto{\pgfqpoint{2.218369in}{0.954045in}}%
\pgfpathlineto{\pgfqpoint{2.218391in}{1.233328in}}%
\pgfpathlineto{\pgfqpoint{2.219075in}{1.081667in}}%
\pgfpathlineto{\pgfqpoint{2.219163in}{1.364666in}}%
\pgfpathlineto{\pgfqpoint{2.219119in}{1.041129in}}%
\pgfpathlineto{\pgfqpoint{2.220199in}{1.113245in}}%
\pgfpathlineto{\pgfqpoint{2.220750in}{1.000045in}}%
\pgfpathlineto{\pgfqpoint{2.220926in}{1.351117in}}%
\pgfpathlineto{\pgfqpoint{2.221235in}{1.224587in}}%
\pgfpathlineto{\pgfqpoint{2.221874in}{1.317900in}}%
\pgfpathlineto{\pgfqpoint{2.221323in}{1.052930in}}%
\pgfpathlineto{\pgfqpoint{2.222138in}{1.161759in}}%
\pgfpathlineto{\pgfqpoint{2.222160in}{1.027034in}}%
\pgfpathlineto{\pgfqpoint{2.223064in}{1.302275in}}%
\pgfpathlineto{\pgfqpoint{2.223218in}{1.272773in}}%
\pgfpathlineto{\pgfqpoint{2.223747in}{1.139469in}}%
\pgfpathlineto{\pgfqpoint{2.224321in}{1.398101in}}%
\pgfpathlineto{\pgfqpoint{2.224629in}{1.184159in}}%
\pgfpathlineto{\pgfqpoint{2.225202in}{1.527581in}}%
\pgfpathlineto{\pgfqpoint{2.225423in}{1.351335in}}%
\pgfpathlineto{\pgfqpoint{2.225709in}{1.408700in}}%
\pgfpathlineto{\pgfqpoint{2.225555in}{1.139797in}}%
\pgfpathlineto{\pgfqpoint{2.225974in}{1.319867in}}%
\pgfpathlineto{\pgfqpoint{2.226591in}{1.154547in}}%
\pgfpathlineto{\pgfqpoint{2.226128in}{1.387175in}}%
\pgfpathlineto{\pgfqpoint{2.227098in}{1.224915in}}%
\pgfpathlineto{\pgfqpoint{2.227208in}{1.424653in}}%
\pgfpathlineto{\pgfqpoint{2.227340in}{1.071615in}}%
\pgfpathlineto{\pgfqpoint{2.228178in}{1.218250in}}%
\pgfpathlineto{\pgfqpoint{2.228839in}{1.036540in}}%
\pgfpathlineto{\pgfqpoint{2.228487in}{1.312000in}}%
\pgfpathlineto{\pgfqpoint{2.229302in}{1.158153in}}%
\pgfpathlineto{\pgfqpoint{2.230051in}{1.007257in}}%
\pgfpathlineto{\pgfqpoint{2.230206in}{1.377996in}}%
\pgfpathlineto{\pgfqpoint{2.230404in}{1.143184in}}%
\pgfpathlineto{\pgfqpoint{2.230999in}{1.339753in}}%
\pgfpathlineto{\pgfqpoint{2.230625in}{0.954154in}}%
\pgfpathlineto{\pgfqpoint{2.231484in}{1.104941in}}%
\pgfpathlineto{\pgfqpoint{2.231881in}{0.972620in}}%
\pgfpathlineto{\pgfqpoint{2.232322in}{1.326095in}}%
\pgfpathlineto{\pgfqpoint{2.232520in}{1.199347in}}%
\pgfpathlineto{\pgfqpoint{2.232564in}{1.292004in}}%
\pgfpathlineto{\pgfqpoint{2.232983in}{0.991413in}}%
\pgfpathlineto{\pgfqpoint{2.233534in}{1.176291in}}%
\pgfpathlineto{\pgfqpoint{2.233865in}{1.018839in}}%
\pgfpathlineto{\pgfqpoint{2.234328in}{1.337896in}}%
\pgfpathlineto{\pgfqpoint{2.234592in}{1.204591in}}%
\pgfpathlineto{\pgfqpoint{2.235320in}{1.375155in}}%
\pgfpathlineto{\pgfqpoint{2.234746in}{1.063529in}}%
\pgfpathlineto{\pgfqpoint{2.235694in}{1.265234in}}%
\pgfpathlineto{\pgfqpoint{2.235804in}{1.099150in}}%
\pgfpathlineto{\pgfqpoint{2.236620in}{1.496222in}}%
\pgfpathlineto{\pgfqpoint{2.236686in}{1.418971in}}%
\pgfpathlineto{\pgfqpoint{2.236708in}{1.506712in}}%
\pgfpathlineto{\pgfqpoint{2.237612in}{1.112371in}}%
\pgfpathlineto{\pgfqpoint{2.237744in}{1.274412in}}%
\pgfpathlineto{\pgfqpoint{2.238119in}{1.145697in}}%
\pgfpathlineto{\pgfqpoint{2.238934in}{1.447380in}}%
\pgfpathlineto{\pgfqpoint{2.239243in}{1.242834in}}%
\pgfpathlineto{\pgfqpoint{2.239221in}{1.508351in}}%
\pgfpathlineto{\pgfqpoint{2.240037in}{1.303258in}}%
\pgfpathlineto{\pgfqpoint{2.240389in}{1.440933in}}%
\pgfpathlineto{\pgfqpoint{2.240566in}{1.122314in}}%
\pgfpathlineto{\pgfqpoint{2.241117in}{1.255946in}}%
\pgfpathlineto{\pgfqpoint{2.241778in}{1.165693in}}%
\pgfpathlineto{\pgfqpoint{2.241205in}{1.447599in}}%
\pgfpathlineto{\pgfqpoint{2.242175in}{1.367070in}}%
\pgfpathlineto{\pgfqpoint{2.243122in}{1.480597in}}%
\pgfpathlineto{\pgfqpoint{2.242241in}{1.108656in}}%
\pgfpathlineto{\pgfqpoint{2.243211in}{1.326751in}}%
\pgfpathlineto{\pgfqpoint{2.243563in}{1.195522in}}%
\pgfpathlineto{\pgfqpoint{2.243916in}{1.399740in}}%
\pgfpathlineto{\pgfqpoint{2.244357in}{1.278783in}}%
\pgfpathlineto{\pgfqpoint{2.244555in}{1.109530in}}%
\pgfpathlineto{\pgfqpoint{2.245040in}{1.297140in}}%
\pgfpathlineto{\pgfqpoint{2.245216in}{1.252013in}}%
\pgfpathlineto{\pgfqpoint{2.245238in}{1.436454in}}%
\pgfpathlineto{\pgfqpoint{2.245789in}{1.097948in}}%
\pgfpathlineto{\pgfqpoint{2.246318in}{1.400177in}}%
\pgfpathlineto{\pgfqpoint{2.246914in}{1.100570in}}%
\pgfpathlineto{\pgfqpoint{2.247443in}{1.312000in}}%
\pgfpathlineto{\pgfqpoint{2.248258in}{1.501139in}}%
\pgfpathlineto{\pgfqpoint{2.247729in}{1.145697in}}%
\pgfpathlineto{\pgfqpoint{2.248479in}{1.288070in}}%
\pgfpathlineto{\pgfqpoint{2.249515in}{1.059814in}}%
\pgfpathlineto{\pgfqpoint{2.248589in}{1.420719in}}%
\pgfpathlineto{\pgfqpoint{2.249581in}{1.248407in}}%
\pgfpathlineto{\pgfqpoint{2.250132in}{1.414491in}}%
\pgfpathlineto{\pgfqpoint{2.249647in}{1.144276in}}%
\pgfpathlineto{\pgfqpoint{2.250749in}{1.340300in}}%
\pgfpathlineto{\pgfqpoint{2.251035in}{1.188092in}}%
\pgfpathlineto{\pgfqpoint{2.250859in}{1.545501in}}%
\pgfpathlineto{\pgfqpoint{2.251873in}{1.283809in}}%
\pgfpathlineto{\pgfqpoint{2.252821in}{1.417878in}}%
\pgfpathlineto{\pgfqpoint{2.252424in}{1.112043in}}%
\pgfpathlineto{\pgfqpoint{2.252909in}{1.225789in}}%
\pgfpathlineto{\pgfqpoint{2.253526in}{1.048450in}}%
\pgfpathlineto{\pgfqpoint{2.253107in}{1.374937in}}%
\pgfpathlineto{\pgfqpoint{2.253769in}{1.264032in}}%
\pgfpathlineto{\pgfqpoint{2.253989in}{1.293752in}}%
\pgfpathlineto{\pgfqpoint{2.254364in}{1.030749in}}%
\pgfpathlineto{\pgfqpoint{2.254761in}{1.136191in}}%
\pgfpathlineto{\pgfqpoint{2.254805in}{0.986278in}}%
\pgfpathlineto{\pgfqpoint{2.255708in}{1.265671in}}%
\pgfpathlineto{\pgfqpoint{2.255841in}{1.112152in}}%
\pgfpathlineto{\pgfqpoint{2.256061in}{1.092266in}}%
\pgfpathlineto{\pgfqpoint{2.256987in}{1.367835in}}%
\pgfpathlineto{\pgfqpoint{2.257450in}{1.066588in}}%
\pgfpathlineto{\pgfqpoint{2.258133in}{1.133459in}}%
\pgfpathlineto{\pgfqpoint{2.259191in}{1.256165in}}%
\pgfpathlineto{\pgfqpoint{2.259103in}{1.011300in}}%
\pgfpathlineto{\pgfqpoint{2.259235in}{1.110732in}}%
\pgfpathlineto{\pgfqpoint{2.259301in}{0.966829in}}%
\pgfpathlineto{\pgfqpoint{2.259522in}{1.306864in}}%
\pgfpathlineto{\pgfqpoint{2.260359in}{1.065933in}}%
\pgfpathlineto{\pgfqpoint{2.260381in}{1.065605in}}%
\pgfpathlineto{\pgfqpoint{2.260536in}{1.266982in}}%
\pgfpathlineto{\pgfqpoint{2.261395in}{0.930006in}}%
\pgfpathlineto{\pgfqpoint{2.261461in}{1.050308in}}%
\pgfpathlineto{\pgfqpoint{2.261748in}{0.983219in}}%
\pgfpathlineto{\pgfqpoint{2.261572in}{1.259989in}}%
\pgfpathlineto{\pgfqpoint{2.262475in}{1.153673in}}%
\pgfpathlineto{\pgfqpoint{2.262519in}{1.288617in}}%
\pgfpathlineto{\pgfqpoint{2.262630in}{1.009115in}}%
\pgfpathlineto{\pgfqpoint{2.263599in}{1.197270in}}%
\pgfpathlineto{\pgfqpoint{2.264040in}{1.433394in}}%
\pgfpathlineto{\pgfqpoint{2.263754in}{1.026925in}}%
\pgfpathlineto{\pgfqpoint{2.264679in}{1.233765in}}%
\pgfpathlineto{\pgfqpoint{2.265407in}{1.056208in}}%
\pgfpathlineto{\pgfqpoint{2.265561in}{1.313092in}}%
\pgfpathlineto{\pgfqpoint{2.265804in}{1.135644in}}%
\pgfpathlineto{\pgfqpoint{2.266156in}{1.305007in}}%
\pgfpathlineto{\pgfqpoint{2.266266in}{0.972620in}}%
\pgfpathlineto{\pgfqpoint{2.266884in}{1.159683in}}%
\pgfpathlineto{\pgfqpoint{2.266950in}{0.926182in}}%
\pgfpathlineto{\pgfqpoint{2.267611in}{1.325876in}}%
\pgfpathlineto{\pgfqpoint{2.267964in}{1.116523in}}%
\pgfpathlineto{\pgfqpoint{2.268427in}{1.375702in}}%
\pgfpathlineto{\pgfqpoint{2.268581in}{0.994364in}}%
\pgfpathlineto{\pgfqpoint{2.269088in}{1.284902in}}%
\pgfpathlineto{\pgfqpoint{2.269970in}{1.385426in}}%
\pgfpathlineto{\pgfqpoint{2.269661in}{1.040474in}}%
\pgfpathlineto{\pgfqpoint{2.270058in}{1.229395in}}%
\pgfpathlineto{\pgfqpoint{2.271094in}{1.002668in}}%
\pgfpathlineto{\pgfqpoint{2.270785in}{1.261082in}}%
\pgfpathlineto{\pgfqpoint{2.271182in}{1.174106in}}%
\pgfpathlineto{\pgfqpoint{2.271689in}{1.341611in}}%
\pgfpathlineto{\pgfqpoint{2.271314in}{1.011518in}}%
\pgfpathlineto{\pgfqpoint{2.272130in}{1.100789in}}%
\pgfpathlineto{\pgfqpoint{2.273166in}{1.005290in}}%
\pgfpathlineto{\pgfqpoint{2.272857in}{1.308503in}}%
\pgfpathlineto{\pgfqpoint{2.273210in}{1.197380in}}%
\pgfpathlineto{\pgfqpoint{2.273871in}{1.007476in}}%
\pgfpathlineto{\pgfqpoint{2.273562in}{1.296812in}}%
\pgfpathlineto{\pgfqpoint{2.274312in}{1.128542in}}%
\pgfpathlineto{\pgfqpoint{2.274510in}{1.335820in}}%
\pgfpathlineto{\pgfqpoint{2.274841in}{0.955793in}}%
\pgfpathlineto{\pgfqpoint{2.275370in}{1.150395in}}%
\pgfpathlineto{\pgfqpoint{2.275392in}{0.960163in}}%
\pgfpathlineto{\pgfqpoint{2.275700in}{1.290037in}}%
\pgfpathlineto{\pgfqpoint{2.276472in}{1.002996in}}%
\pgfpathlineto{\pgfqpoint{2.277486in}{1.228521in}}%
\pgfpathlineto{\pgfqpoint{2.276692in}{0.979285in}}%
\pgfpathlineto{\pgfqpoint{2.277596in}{1.212895in}}%
\pgfpathlineto{\pgfqpoint{2.277993in}{1.008022in}}%
\pgfpathlineto{\pgfqpoint{2.278346in}{1.337349in}}%
\pgfpathlineto{\pgfqpoint{2.278742in}{1.048887in}}%
\pgfpathlineto{\pgfqpoint{2.279227in}{0.982126in}}%
\pgfpathlineto{\pgfqpoint{2.279117in}{1.294408in}}%
\pgfpathlineto{\pgfqpoint{2.279602in}{1.162852in}}%
\pgfpathlineto{\pgfqpoint{2.279800in}{1.205575in}}%
\pgfpathlineto{\pgfqpoint{2.280329in}{0.958415in}}%
\pgfpathlineto{\pgfqpoint{2.280616in}{1.090955in}}%
\pgfpathlineto{\pgfqpoint{2.280946in}{0.917440in}}%
\pgfpathlineto{\pgfqpoint{2.280814in}{1.252668in}}%
\pgfpathlineto{\pgfqpoint{2.281718in}{1.098385in}}%
\pgfpathlineto{\pgfqpoint{2.282688in}{1.276488in}}%
\pgfpathlineto{\pgfqpoint{2.282027in}{1.005072in}}%
\pgfpathlineto{\pgfqpoint{2.282820in}{1.099368in}}%
\pgfpathlineto{\pgfqpoint{2.283040in}{1.255072in}}%
\pgfpathlineto{\pgfqpoint{2.283724in}{0.969997in}}%
\pgfpathlineto{\pgfqpoint{2.283966in}{1.195631in}}%
\pgfpathlineto{\pgfqpoint{2.284143in}{1.058175in}}%
\pgfpathlineto{\pgfqpoint{2.284826in}{1.293534in}}%
\pgfpathlineto{\pgfqpoint{2.285090in}{1.088114in}}%
\pgfpathlineto{\pgfqpoint{2.285950in}{1.286759in}}%
\pgfpathlineto{\pgfqpoint{2.285179in}{1.026706in}}%
\pgfpathlineto{\pgfqpoint{2.286192in}{1.182847in}}%
\pgfpathlineto{\pgfqpoint{2.286391in}{0.965408in}}%
\pgfpathlineto{\pgfqpoint{2.286744in}{1.339098in}}%
\pgfpathlineto{\pgfqpoint{2.287295in}{1.123407in}}%
\pgfpathlineto{\pgfqpoint{2.287581in}{1.243599in}}%
\pgfpathlineto{\pgfqpoint{2.287713in}{0.997642in}}%
\pgfpathlineto{\pgfqpoint{2.288397in}{1.154766in}}%
\pgfpathlineto{\pgfqpoint{2.288551in}{0.989447in}}%
\pgfpathlineto{\pgfqpoint{2.288771in}{1.319758in}}%
\pgfpathlineto{\pgfqpoint{2.289499in}{1.077843in}}%
\pgfpathlineto{\pgfqpoint{2.290380in}{1.241742in}}%
\pgfpathlineto{\pgfqpoint{2.289697in}{0.956230in}}%
\pgfpathlineto{\pgfqpoint{2.290645in}{1.181427in}}%
\pgfpathlineto{\pgfqpoint{2.290843in}{1.298123in}}%
\pgfpathlineto{\pgfqpoint{2.291394in}{1.009770in}}%
\pgfpathlineto{\pgfqpoint{2.291769in}{1.263049in}}%
\pgfpathlineto{\pgfqpoint{2.292078in}{1.033262in}}%
\pgfpathlineto{\pgfqpoint{2.292651in}{1.284683in}}%
\pgfpathlineto{\pgfqpoint{2.292871in}{1.211584in}}%
\pgfpathlineto{\pgfqpoint{2.293510in}{1.297467in}}%
\pgfpathlineto{\pgfqpoint{2.293400in}{1.024740in}}%
\pgfpathlineto{\pgfqpoint{2.293907in}{1.131602in}}%
\pgfpathlineto{\pgfqpoint{2.294061in}{1.034027in}}%
\pgfpathlineto{\pgfqpoint{2.294679in}{1.316480in}}%
\pgfpathlineto{\pgfqpoint{2.294943in}{1.276270in}}%
\pgfpathlineto{\pgfqpoint{2.295913in}{1.368490in}}%
\pgfpathlineto{\pgfqpoint{2.295582in}{1.092047in}}%
\pgfpathlineto{\pgfqpoint{2.295935in}{1.315059in}}%
\pgfpathlineto{\pgfqpoint{2.296684in}{1.126685in}}%
\pgfpathlineto{\pgfqpoint{2.296795in}{1.407826in}}%
\pgfpathlineto{\pgfqpoint{2.297037in}{1.213770in}}%
\pgfpathlineto{\pgfqpoint{2.298095in}{1.296265in}}%
\pgfpathlineto{\pgfqpoint{2.297544in}{0.984093in}}%
\pgfpathlineto{\pgfqpoint{2.298139in}{1.272773in}}%
\pgfpathlineto{\pgfqpoint{2.298227in}{1.055006in}}%
\pgfpathlineto{\pgfqpoint{2.298734in}{1.376248in}}%
\pgfpathlineto{\pgfqpoint{2.299285in}{1.082213in}}%
\pgfpathlineto{\pgfqpoint{2.300189in}{1.372205in}}%
\pgfpathlineto{\pgfqpoint{2.299792in}{1.055880in}}%
\pgfpathlineto{\pgfqpoint{2.300387in}{1.106033in}}%
\pgfpathlineto{\pgfqpoint{2.300696in}{1.346528in}}%
\pgfpathlineto{\pgfqpoint{2.300564in}{1.006492in}}%
\pgfpathlineto{\pgfqpoint{2.301534in}{1.240977in}}%
\pgfpathlineto{\pgfqpoint{2.301732in}{0.988463in}}%
\pgfpathlineto{\pgfqpoint{2.302129in}{1.307301in}}%
\pgfpathlineto{\pgfqpoint{2.302658in}{1.139687in}}%
\pgfpathlineto{\pgfqpoint{2.303143in}{1.276051in}}%
\pgfpathlineto{\pgfqpoint{2.303010in}{0.935797in}}%
\pgfpathlineto{\pgfqpoint{2.303650in}{1.082760in}}%
\pgfpathlineto{\pgfqpoint{2.304311in}{0.869254in}}%
\pgfpathlineto{\pgfqpoint{2.304509in}{1.208743in}}%
\pgfpathlineto{\pgfqpoint{2.304796in}{0.928149in}}%
\pgfpathlineto{\pgfqpoint{2.305104in}{0.897226in}}%
\pgfpathlineto{\pgfqpoint{2.304928in}{1.080574in}}%
\pgfpathlineto{\pgfqpoint{2.305215in}{1.061999in}}%
\pgfpathlineto{\pgfqpoint{2.306295in}{1.196943in}}%
\pgfpathlineto{\pgfqpoint{2.306052in}{0.899958in}}%
\pgfpathlineto{\pgfqpoint{2.306317in}{1.084836in}}%
\pgfpathlineto{\pgfqpoint{2.306647in}{0.945740in}}%
\pgfpathlineto{\pgfqpoint{2.306361in}{1.244255in}}%
\pgfpathlineto{\pgfqpoint{2.307353in}{1.009770in}}%
\pgfpathlineto{\pgfqpoint{2.307926in}{1.298451in}}%
\pgfpathlineto{\pgfqpoint{2.308367in}{0.978629in}}%
\pgfpathlineto{\pgfqpoint{2.308477in}{1.143839in}}%
\pgfpathlineto{\pgfqpoint{2.309028in}{0.938420in}}%
\pgfpathlineto{\pgfqpoint{2.309182in}{1.234749in}}%
\pgfpathlineto{\pgfqpoint{2.309535in}{1.054132in}}%
\pgfpathlineto{\pgfqpoint{2.309711in}{1.209071in}}%
\pgfpathlineto{\pgfqpoint{2.309866in}{0.914490in}}%
\pgfpathlineto{\pgfqpoint{2.310615in}{1.150505in}}%
\pgfpathlineto{\pgfqpoint{2.310879in}{0.837021in}}%
\pgfpathlineto{\pgfqpoint{2.311364in}{1.187983in}}%
\pgfpathlineto{\pgfqpoint{2.311717in}{1.029547in}}%
\pgfpathlineto{\pgfqpoint{2.312819in}{1.215409in}}%
\pgfpathlineto{\pgfqpoint{2.312378in}{0.873953in}}%
\pgfpathlineto{\pgfqpoint{2.312863in}{1.090845in}}%
\pgfpathlineto{\pgfqpoint{2.313304in}{0.856033in}}%
\pgfpathlineto{\pgfqpoint{2.313370in}{1.126029in}}%
\pgfpathlineto{\pgfqpoint{2.314098in}{0.994364in}}%
\pgfpathlineto{\pgfqpoint{2.314340in}{1.212240in}}%
\pgfpathlineto{\pgfqpoint{2.314957in}{0.850460in}}%
\pgfpathlineto{\pgfqpoint{2.315332in}{1.145915in}}%
\pgfpathlineto{\pgfqpoint{2.316412in}{0.851116in}}%
\pgfpathlineto{\pgfqpoint{2.316346in}{1.290802in}}%
\pgfpathlineto{\pgfqpoint{2.316456in}{1.044845in}}%
\pgfpathlineto{\pgfqpoint{2.316478in}{0.944320in}}%
\pgfpathlineto{\pgfqpoint{2.317206in}{1.247751in}}%
\pgfpathlineto{\pgfqpoint{2.317536in}{1.034464in}}%
\pgfpathlineto{\pgfqpoint{2.318374in}{1.287852in}}%
\pgfpathlineto{\pgfqpoint{2.318660in}{1.189950in}}%
\pgfpathlineto{\pgfqpoint{2.318903in}{1.367288in}}%
\pgfpathlineto{\pgfqpoint{2.319057in}{1.050526in}}%
\pgfpathlineto{\pgfqpoint{2.319674in}{1.186672in}}%
\pgfpathlineto{\pgfqpoint{2.320247in}{1.059923in}}%
\pgfpathlineto{\pgfqpoint{2.320402in}{1.366742in}}%
\pgfpathlineto{\pgfqpoint{2.320776in}{1.161322in}}%
\pgfpathlineto{\pgfqpoint{2.321526in}{1.323036in}}%
\pgfpathlineto{\pgfqpoint{2.321834in}{1.017419in}}%
\pgfpathlineto{\pgfqpoint{2.321856in}{1.087349in}}%
\pgfpathlineto{\pgfqpoint{2.322385in}{1.051728in}}%
\pgfpathlineto{\pgfqpoint{2.322253in}{1.284792in}}%
\pgfpathlineto{\pgfqpoint{2.322804in}{1.209945in}}%
\pgfpathlineto{\pgfqpoint{2.323620in}{1.322271in}}%
\pgfpathlineto{\pgfqpoint{2.323113in}{1.000920in}}%
\pgfpathlineto{\pgfqpoint{2.323884in}{1.294408in}}%
\pgfpathlineto{\pgfqpoint{2.324237in}{1.024849in}}%
\pgfpathlineto{\pgfqpoint{2.324964in}{1.348713in}}%
\pgfpathlineto{\pgfqpoint{2.325008in}{1.169845in}}%
\pgfpathlineto{\pgfqpoint{2.325185in}{1.268184in}}%
\pgfpathlineto{\pgfqpoint{2.325141in}{1.032497in}}%
\pgfpathlineto{\pgfqpoint{2.326133in}{1.185142in}}%
\pgfpathlineto{\pgfqpoint{2.326728in}{1.047248in}}%
\pgfpathlineto{\pgfqpoint{2.326397in}{1.279329in}}%
\pgfpathlineto{\pgfqpoint{2.327257in}{1.149521in}}%
\pgfpathlineto{\pgfqpoint{2.327653in}{1.260645in}}%
\pgfpathlineto{\pgfqpoint{2.328160in}{0.940496in}}%
\pgfpathlineto{\pgfqpoint{2.328293in}{1.104722in}}%
\pgfpathlineto{\pgfqpoint{2.328954in}{0.927930in}}%
\pgfpathlineto{\pgfqpoint{2.328866in}{1.194430in}}%
\pgfpathlineto{\pgfqpoint{2.329395in}{1.078717in}}%
\pgfpathlineto{\pgfqpoint{2.329858in}{0.911322in}}%
\pgfpathlineto{\pgfqpoint{2.330254in}{1.134661in}}%
\pgfpathlineto{\pgfqpoint{2.330563in}{0.969232in}}%
\pgfpathlineto{\pgfqpoint{2.331511in}{1.167222in}}%
\pgfpathlineto{\pgfqpoint{2.331026in}{0.850023in}}%
\pgfpathlineto{\pgfqpoint{2.331665in}{0.992725in}}%
\pgfpathlineto{\pgfqpoint{2.331731in}{0.864009in}}%
\pgfpathlineto{\pgfqpoint{2.331974in}{1.090190in}}%
\pgfpathlineto{\pgfqpoint{2.332811in}{0.908481in}}%
\pgfpathlineto{\pgfqpoint{2.333847in}{1.093686in}}%
\pgfpathlineto{\pgfqpoint{2.333561in}{0.858874in}}%
\pgfpathlineto{\pgfqpoint{2.334046in}{0.960273in}}%
\pgfpathlineto{\pgfqpoint{2.334663in}{1.195522in}}%
\pgfpathlineto{\pgfqpoint{2.334707in}{0.904219in}}%
\pgfpathlineto{\pgfqpoint{2.334839in}{0.947489in}}%
\pgfpathlineto{\pgfqpoint{2.334883in}{0.873953in}}%
\pgfpathlineto{\pgfqpoint{2.335060in}{1.155312in}}%
\pgfpathlineto{\pgfqpoint{2.335831in}{1.123079in}}%
\pgfpathlineto{\pgfqpoint{2.336757in}{1.286431in}}%
\pgfpathlineto{\pgfqpoint{2.336514in}{0.991851in}}%
\pgfpathlineto{\pgfqpoint{2.336933in}{1.168206in}}%
\pgfpathlineto{\pgfqpoint{2.337661in}{0.913398in}}%
\pgfpathlineto{\pgfqpoint{2.337220in}{1.177930in}}%
\pgfpathlineto{\pgfqpoint{2.338057in}{1.087021in}}%
\pgfpathlineto{\pgfqpoint{2.339093in}{0.894495in}}%
\pgfpathlineto{\pgfqpoint{2.338167in}{1.225024in}}%
\pgfpathlineto{\pgfqpoint{2.339181in}{1.009224in}}%
\pgfpathlineto{\pgfqpoint{2.339358in}{1.106798in}}%
\pgfpathlineto{\pgfqpoint{2.339270in}{0.848712in}}%
\pgfpathlineto{\pgfqpoint{2.340328in}{1.055116in}}%
\pgfpathlineto{\pgfqpoint{2.341430in}{1.228411in}}%
\pgfpathlineto{\pgfqpoint{2.340394in}{0.799324in}}%
\pgfpathlineto{\pgfqpoint{2.341474in}{1.180771in}}%
\pgfpathlineto{\pgfqpoint{2.342069in}{0.929897in}}%
\pgfpathlineto{\pgfqpoint{2.341540in}{1.226226in}}%
\pgfpathlineto{\pgfqpoint{2.342576in}{1.077187in}}%
\pgfpathlineto{\pgfqpoint{2.342620in}{1.207541in}}%
\pgfpathlineto{\pgfqpoint{2.342730in}{0.929569in}}%
\pgfpathlineto{\pgfqpoint{2.343656in}{1.064075in}}%
\pgfpathlineto{\pgfqpoint{2.344494in}{0.870565in}}%
\pgfpathlineto{\pgfqpoint{2.344361in}{1.220216in}}%
\pgfpathlineto{\pgfqpoint{2.344736in}{1.090081in}}%
\pgfpathlineto{\pgfqpoint{2.345045in}{1.101991in}}%
\pgfpathlineto{\pgfqpoint{2.344846in}{0.853957in}}%
\pgfpathlineto{\pgfqpoint{2.345111in}{0.930006in}}%
\pgfpathlineto{\pgfqpoint{2.345199in}{0.853083in}}%
\pgfpathlineto{\pgfqpoint{2.345684in}{1.178805in}}%
\pgfpathlineto{\pgfqpoint{2.346169in}{0.978848in}}%
\pgfpathlineto{\pgfqpoint{2.346852in}{1.187546in}}%
\pgfpathlineto{\pgfqpoint{2.347072in}{0.858437in}}%
\pgfpathlineto{\pgfqpoint{2.347271in}{0.986606in}}%
\pgfpathlineto{\pgfqpoint{2.347513in}{1.189403in}}%
\pgfpathlineto{\pgfqpoint{2.347425in}{0.913725in}}%
\pgfpathlineto{\pgfqpoint{2.348351in}{1.048450in}}%
\pgfpathlineto{\pgfqpoint{2.348880in}{0.835272in}}%
\pgfpathlineto{\pgfqpoint{2.349056in}{1.136082in}}%
\pgfpathlineto{\pgfqpoint{2.349475in}{0.866413in}}%
\pgfpathlineto{\pgfqpoint{2.350445in}{1.192900in}}%
\pgfpathlineto{\pgfqpoint{2.349916in}{0.772663in}}%
\pgfpathlineto{\pgfqpoint{2.350599in}{1.045172in}}%
\pgfpathlineto{\pgfqpoint{2.351040in}{1.262065in}}%
\pgfpathlineto{\pgfqpoint{2.351106in}{0.950985in}}%
\pgfpathlineto{\pgfqpoint{2.351767in}{1.190715in}}%
\pgfpathlineto{\pgfqpoint{2.352847in}{0.885316in}}%
\pgfpathlineto{\pgfqpoint{2.352473in}{1.222948in}}%
\pgfpathlineto{\pgfqpoint{2.352936in}{0.901925in}}%
\pgfpathlineto{\pgfqpoint{2.353134in}{1.165693in}}%
\pgfpathlineto{\pgfqpoint{2.353905in}{0.811999in}}%
\pgfpathlineto{\pgfqpoint{2.354082in}{1.001903in}}%
\pgfpathlineto{\pgfqpoint{2.354919in}{0.851444in}}%
\pgfpathlineto{\pgfqpoint{2.354214in}{1.116523in}}%
\pgfpathlineto{\pgfqpoint{2.355030in}{1.060470in}}%
\pgfpathlineto{\pgfqpoint{2.355052in}{1.202078in}}%
\pgfpathlineto{\pgfqpoint{2.355250in}{0.835928in}}%
\pgfpathlineto{\pgfqpoint{2.356110in}{0.866632in}}%
\pgfpathlineto{\pgfqpoint{2.356462in}{1.090845in}}%
\pgfpathlineto{\pgfqpoint{2.356330in}{0.851553in}}%
\pgfpathlineto{\pgfqpoint{2.357256in}{0.950111in}}%
\pgfpathlineto{\pgfqpoint{2.358204in}{0.840954in}}%
\pgfpathlineto{\pgfqpoint{2.357388in}{1.174325in}}%
\pgfpathlineto{\pgfqpoint{2.358358in}{0.970762in}}%
\pgfpathlineto{\pgfqpoint{2.359350in}{1.119364in}}%
\pgfpathlineto{\pgfqpoint{2.359041in}{0.865539in}}%
\pgfpathlineto{\pgfqpoint{2.359416in}{0.943118in}}%
\pgfpathlineto{\pgfqpoint{2.360055in}{0.831120in}}%
\pgfpathlineto{\pgfqpoint{2.359835in}{1.235404in}}%
\pgfpathlineto{\pgfqpoint{2.360452in}{0.918752in}}%
\pgfpathlineto{\pgfqpoint{2.360937in}{1.206667in}}%
\pgfpathlineto{\pgfqpoint{2.361554in}{1.023538in}}%
\pgfpathlineto{\pgfqpoint{2.361620in}{0.833743in}}%
\pgfpathlineto{\pgfqpoint{2.362127in}{1.176728in}}%
\pgfpathlineto{\pgfqpoint{2.362700in}{0.857016in}}%
\pgfpathlineto{\pgfqpoint{2.363229in}{1.061234in}}%
\pgfpathlineto{\pgfqpoint{2.363802in}{0.922467in}}%
\pgfpathlineto{\pgfqpoint{2.363824in}{0.856470in}}%
\pgfpathlineto{\pgfqpoint{2.364596in}{1.295391in}}%
\pgfpathlineto{\pgfqpoint{2.364816in}{1.196833in}}%
\pgfpathlineto{\pgfqpoint{2.364904in}{0.915911in}}%
\pgfpathlineto{\pgfqpoint{2.365720in}{1.303149in}}%
\pgfpathlineto{\pgfqpoint{2.366029in}{1.076750in}}%
\pgfpathlineto{\pgfqpoint{2.366932in}{1.362371in}}%
\pgfpathlineto{\pgfqpoint{2.366271in}{0.936999in}}%
\pgfpathlineto{\pgfqpoint{2.367131in}{1.117288in}}%
\pgfpathlineto{\pgfqpoint{2.367219in}{1.085382in}}%
\pgfpathlineto{\pgfqpoint{2.367285in}{1.259443in}}%
\pgfpathlineto{\pgfqpoint{2.368299in}{1.460492in}}%
\pgfpathlineto{\pgfqpoint{2.367792in}{1.090299in}}%
\pgfpathlineto{\pgfqpoint{2.368431in}{1.373189in}}%
\pgfpathlineto{\pgfqpoint{2.368982in}{1.174543in}}%
\pgfpathlineto{\pgfqpoint{2.369445in}{1.456449in}}%
\pgfpathlineto{\pgfqpoint{2.369599in}{1.203062in}}%
\pgfpathlineto{\pgfqpoint{2.370283in}{1.535886in}}%
\pgfpathlineto{\pgfqpoint{2.370723in}{1.336257in}}%
\pgfpathlineto{\pgfqpoint{2.370966in}{1.210055in}}%
\pgfpathlineto{\pgfqpoint{2.371715in}{1.510317in}}%
\pgfpathlineto{\pgfqpoint{2.371737in}{1.493600in}}%
\pgfpathlineto{\pgfqpoint{2.372421in}{1.593687in}}%
\pgfpathlineto{\pgfqpoint{2.371892in}{1.299325in}}%
\pgfpathlineto{\pgfqpoint{2.372729in}{1.419736in}}%
\pgfpathlineto{\pgfqpoint{2.373457in}{1.298232in}}%
\pgfpathlineto{\pgfqpoint{2.373236in}{1.622970in}}%
\pgfpathlineto{\pgfqpoint{2.373831in}{1.378980in}}%
\pgfpathlineto{\pgfqpoint{2.375022in}{1.678259in}}%
\pgfpathlineto{\pgfqpoint{2.373964in}{1.368381in}}%
\pgfpathlineto{\pgfqpoint{2.375044in}{1.662088in}}%
\pgfpathlineto{\pgfqpoint{2.375088in}{1.666458in}}%
\pgfpathlineto{\pgfqpoint{2.375176in}{1.528128in}}%
\pgfpathlineto{\pgfqpoint{2.375198in}{1.511082in}}%
\pgfpathlineto{\pgfqpoint{2.375881in}{1.733329in}}%
\pgfpathlineto{\pgfqpoint{2.376190in}{1.545829in}}%
\pgfpathlineto{\pgfqpoint{2.377182in}{1.872971in}}%
\pgfpathlineto{\pgfqpoint{2.376234in}{1.447926in}}%
\pgfpathlineto{\pgfqpoint{2.377336in}{1.669955in}}%
\pgfpathlineto{\pgfqpoint{2.378108in}{1.610296in}}%
\pgfpathlineto{\pgfqpoint{2.378041in}{1.874173in}}%
\pgfpathlineto{\pgfqpoint{2.378372in}{1.791131in}}%
\pgfpathlineto{\pgfqpoint{2.379055in}{1.843141in}}%
\pgfpathlineto{\pgfqpoint{2.378967in}{1.604942in}}%
\pgfpathlineto{\pgfqpoint{2.379121in}{1.700003in}}%
\pgfpathlineto{\pgfqpoint{2.379518in}{1.854068in}}%
\pgfpathlineto{\pgfqpoint{2.380268in}{1.471746in}}%
\pgfpathlineto{\pgfqpoint{2.381039in}{1.885755in}}%
\pgfpathlineto{\pgfqpoint{2.381392in}{1.648429in}}%
\pgfpathlineto{\pgfqpoint{2.382097in}{1.858439in}}%
\pgfpathlineto{\pgfqpoint{2.381458in}{1.466611in}}%
\pgfpathlineto{\pgfqpoint{2.382472in}{1.737481in}}%
\pgfpathlineto{\pgfqpoint{2.382494in}{1.584837in}}%
\pgfpathlineto{\pgfqpoint{2.382869in}{1.826861in}}%
\pgfpathlineto{\pgfqpoint{2.383574in}{1.688093in}}%
\pgfpathlineto{\pgfqpoint{2.384169in}{1.871988in}}%
\pgfpathlineto{\pgfqpoint{2.383662in}{1.602866in}}%
\pgfpathlineto{\pgfqpoint{2.384742in}{1.782171in}}%
\pgfpathlineto{\pgfqpoint{2.385183in}{1.917879in}}%
\pgfpathlineto{\pgfqpoint{2.385844in}{1.611170in}}%
\pgfpathlineto{\pgfqpoint{2.386990in}{1.930882in}}%
\pgfpathlineto{\pgfqpoint{2.387652in}{1.695195in}}%
\pgfpathlineto{\pgfqpoint{2.387718in}{1.995567in}}%
\pgfpathlineto{\pgfqpoint{2.388115in}{1.845545in}}%
\pgfpathlineto{\pgfqpoint{2.389107in}{2.026599in}}%
\pgfpathlineto{\pgfqpoint{2.388181in}{1.749937in}}%
\pgfpathlineto{\pgfqpoint{2.389283in}{1.973277in}}%
\pgfpathlineto{\pgfqpoint{2.390076in}{2.097075in}}%
\pgfpathlineto{\pgfqpoint{2.389613in}{1.852429in}}%
\pgfpathlineto{\pgfqpoint{2.390142in}{1.958526in}}%
\pgfpathlineto{\pgfqpoint{2.390231in}{1.824566in}}%
\pgfpathlineto{\pgfqpoint{2.390980in}{2.141328in}}%
\pgfpathlineto{\pgfqpoint{2.391223in}{2.095655in}}%
\pgfpathlineto{\pgfqpoint{2.392369in}{1.819758in}}%
\pgfpathlineto{\pgfqpoint{2.391333in}{2.154768in}}%
\pgfpathlineto{\pgfqpoint{2.392435in}{1.886192in}}%
\pgfpathlineto{\pgfqpoint{2.393405in}{2.168972in}}%
\pgfpathlineto{\pgfqpoint{2.393052in}{1.731472in}}%
\pgfpathlineto{\pgfqpoint{2.393537in}{1.943775in}}%
\pgfpathlineto{\pgfqpoint{2.394066in}{1.732236in}}%
\pgfpathlineto{\pgfqpoint{2.393890in}{1.999610in}}%
\pgfpathlineto{\pgfqpoint{2.394639in}{1.940716in}}%
\pgfpathlineto{\pgfqpoint{2.395080in}{1.795501in}}%
\pgfpathlineto{\pgfqpoint{2.395609in}{2.097185in}}%
\pgfpathlineto{\pgfqpoint{2.395675in}{2.026052in}}%
\pgfpathlineto{\pgfqpoint{2.396116in}{2.148212in}}%
\pgfpathlineto{\pgfqpoint{2.396446in}{1.943338in}}%
\pgfpathlineto{\pgfqpoint{2.396777in}{2.106691in}}%
\pgfpathlineto{\pgfqpoint{2.397438in}{1.914601in}}%
\pgfpathlineto{\pgfqpoint{2.396865in}{2.207106in}}%
\pgfpathlineto{\pgfqpoint{2.397901in}{2.011192in}}%
\pgfpathlineto{\pgfqpoint{2.398276in}{1.890235in}}%
\pgfpathlineto{\pgfqpoint{2.399003in}{2.186892in}}%
\pgfpathlineto{\pgfqpoint{2.400017in}{1.841939in}}%
\pgfpathlineto{\pgfqpoint{2.400150in}{1.893841in}}%
\pgfpathlineto{\pgfqpoint{2.400458in}{1.858876in}}%
\pgfpathlineto{\pgfqpoint{2.401296in}{2.084728in}}%
\pgfpathlineto{\pgfqpoint{2.401692in}{1.797905in}}%
\pgfpathlineto{\pgfqpoint{2.402045in}{2.177167in}}%
\pgfpathlineto{\pgfqpoint{2.402464in}{1.992071in}}%
\pgfpathlineto{\pgfqpoint{2.403456in}{2.115432in}}%
\pgfpathlineto{\pgfqpoint{2.403280in}{1.846856in}}%
\pgfpathlineto{\pgfqpoint{2.403566in}{2.081669in}}%
\pgfpathlineto{\pgfqpoint{2.403610in}{1.857127in}}%
\pgfpathlineto{\pgfqpoint{2.404029in}{2.138487in}}%
\pgfpathlineto{\pgfqpoint{2.404690in}{1.978522in}}%
\pgfpathlineto{\pgfqpoint{2.405550in}{1.872425in}}%
\pgfpathlineto{\pgfqpoint{2.405175in}{2.159794in}}%
\pgfpathlineto{\pgfqpoint{2.405704in}{2.060143in}}%
\pgfpathlineto{\pgfqpoint{2.406233in}{2.185471in}}%
\pgfpathlineto{\pgfqpoint{2.405880in}{1.939623in}}%
\pgfpathlineto{\pgfqpoint{2.406762in}{1.963115in}}%
\pgfpathlineto{\pgfqpoint{2.406828in}{1.941808in}}%
\pgfpathlineto{\pgfqpoint{2.406961in}{1.992726in}}%
\pgfpathlineto{\pgfqpoint{2.407512in}{2.164602in}}%
\pgfpathlineto{\pgfqpoint{2.407401in}{1.818119in}}%
\pgfpathlineto{\pgfqpoint{2.407864in}{2.004855in}}%
\pgfpathlineto{\pgfqpoint{2.407886in}{1.813312in}}%
\pgfpathlineto{\pgfqpoint{2.408702in}{2.144497in}}%
\pgfpathlineto{\pgfqpoint{2.408966in}{1.889142in}}%
\pgfpathlineto{\pgfqpoint{2.409495in}{2.127014in}}%
\pgfpathlineto{\pgfqpoint{2.409672in}{1.835165in}}%
\pgfpathlineto{\pgfqpoint{2.410157in}{2.084619in}}%
\pgfpathlineto{\pgfqpoint{2.410531in}{1.906516in}}%
\pgfpathlineto{\pgfqpoint{2.410223in}{2.203610in}}%
\pgfpathlineto{\pgfqpoint{2.411259in}{2.086258in}}%
\pgfpathlineto{\pgfqpoint{2.411545in}{2.200113in}}%
\pgfpathlineto{\pgfqpoint{2.411413in}{1.915148in}}%
\pgfpathlineto{\pgfqpoint{2.412339in}{1.992617in}}%
\pgfpathlineto{\pgfqpoint{2.413353in}{1.899413in}}%
\pgfpathlineto{\pgfqpoint{2.412537in}{2.171704in}}%
\pgfpathlineto{\pgfqpoint{2.413419in}{2.008351in}}%
\pgfpathlineto{\pgfqpoint{2.414080in}{2.139798in}}%
\pgfpathlineto{\pgfqpoint{2.413705in}{1.947381in}}%
\pgfpathlineto{\pgfqpoint{2.414543in}{2.058941in}}%
\pgfpathlineto{\pgfqpoint{2.415314in}{2.131713in}}%
\pgfpathlineto{\pgfqpoint{2.415006in}{1.875703in}}%
\pgfpathlineto{\pgfqpoint{2.415645in}{2.053478in}}%
\pgfpathlineto{\pgfqpoint{2.416483in}{2.179243in}}%
\pgfpathlineto{\pgfqpoint{2.415888in}{1.914601in}}%
\pgfpathlineto{\pgfqpoint{2.416659in}{2.130401in}}%
\pgfpathlineto{\pgfqpoint{2.416968in}{1.926183in}}%
\pgfpathlineto{\pgfqpoint{2.417453in}{2.274414in}}%
\pgfpathlineto{\pgfqpoint{2.417761in}{2.127123in}}%
\pgfpathlineto{\pgfqpoint{2.418511in}{1.986935in}}%
\pgfpathlineto{\pgfqpoint{2.418731in}{2.358549in}}%
\pgfpathlineto{\pgfqpoint{2.418885in}{2.029549in}}%
\pgfpathlineto{\pgfqpoint{2.419811in}{2.262285in}}%
\pgfpathlineto{\pgfqpoint{2.419502in}{1.993163in}}%
\pgfpathlineto{\pgfqpoint{2.420031in}{2.162089in}}%
\pgfpathlineto{\pgfqpoint{2.420891in}{1.965082in}}%
\pgfpathlineto{\pgfqpoint{2.420560in}{2.275507in}}%
\pgfpathlineto{\pgfqpoint{2.421111in}{2.159794in}}%
\pgfpathlineto{\pgfqpoint{2.421310in}{2.213553in}}%
\pgfpathlineto{\pgfqpoint{2.421663in}{1.954702in}}%
\pgfpathlineto{\pgfqpoint{2.422125in}{2.089973in}}%
\pgfpathlineto{\pgfqpoint{2.422147in}{1.958089in}}%
\pgfpathlineto{\pgfqpoint{2.422588in}{2.286105in}}%
\pgfpathlineto{\pgfqpoint{2.423205in}{2.161761in}}%
\pgfpathlineto{\pgfqpoint{2.423977in}{2.068229in}}%
\pgfpathlineto{\pgfqpoint{2.423779in}{2.345655in}}%
\pgfpathlineto{\pgfqpoint{2.424197in}{2.180773in}}%
\pgfpathlineto{\pgfqpoint{2.425277in}{2.275397in}}%
\pgfpathlineto{\pgfqpoint{2.424726in}{2.031188in}}%
\pgfpathlineto{\pgfqpoint{2.425299in}{2.205467in}}%
\pgfpathlineto{\pgfqpoint{2.425652in}{2.329156in}}%
\pgfpathlineto{\pgfqpoint{2.425520in}{2.103959in}}%
\pgfpathlineto{\pgfqpoint{2.425696in}{2.194978in}}%
\pgfpathlineto{\pgfqpoint{2.426027in}{2.047250in}}%
\pgfpathlineto{\pgfqpoint{2.426688in}{2.350244in}}%
\pgfpathlineto{\pgfqpoint{2.426798in}{2.124392in}}%
\pgfpathlineto{\pgfqpoint{2.427856in}{2.333964in}}%
\pgfpathlineto{\pgfqpoint{2.427570in}{1.993054in}}%
\pgfpathlineto{\pgfqpoint{2.427944in}{2.209291in}}%
\pgfpathlineto{\pgfqpoint{2.428231in}{2.047359in}}%
\pgfpathlineto{\pgfqpoint{2.428914in}{2.234969in}}%
\pgfpathlineto{\pgfqpoint{2.428958in}{2.210384in}}%
\pgfpathlineto{\pgfqpoint{2.428980in}{2.316809in}}%
\pgfpathlineto{\pgfqpoint{2.429884in}{2.016328in}}%
\pgfpathlineto{\pgfqpoint{2.430061in}{2.264034in}}%
\pgfpathlineto{\pgfqpoint{2.430722in}{2.083089in}}%
\pgfpathlineto{\pgfqpoint{2.430413in}{2.330249in}}%
\pgfpathlineto{\pgfqpoint{2.431207in}{2.132915in}}%
\pgfpathlineto{\pgfqpoint{2.432132in}{2.268186in}}%
\pgfpathlineto{\pgfqpoint{2.432265in}{1.985733in}}%
\pgfpathlineto{\pgfqpoint{2.432287in}{2.103194in}}%
\pgfpathlineto{\pgfqpoint{2.432419in}{2.064951in}}%
\pgfpathlineto{\pgfqpoint{2.432661in}{2.392312in}}%
\pgfpathlineto{\pgfqpoint{2.433146in}{2.200332in}}%
\pgfpathlineto{\pgfqpoint{2.434271in}{2.393951in}}%
\pgfpathlineto{\pgfqpoint{2.433455in}{2.097075in}}%
\pgfpathlineto{\pgfqpoint{2.434293in}{2.315716in}}%
\pgfpathlineto{\pgfqpoint{2.434822in}{2.412417in}}%
\pgfpathlineto{\pgfqpoint{2.435439in}{2.045720in}}%
\pgfpathlineto{\pgfqpoint{2.435461in}{2.296923in}}%
\pgfpathlineto{\pgfqpoint{2.436475in}{2.012176in}}%
\pgfpathlineto{\pgfqpoint{2.436563in}{2.226992in}}%
\pgfpathlineto{\pgfqpoint{2.436982in}{2.400070in}}%
\pgfpathlineto{\pgfqpoint{2.437158in}{2.100790in}}%
\pgfpathlineto{\pgfqpoint{2.437665in}{2.236826in}}%
\pgfpathlineto{\pgfqpoint{2.438635in}{2.056101in}}%
\pgfpathlineto{\pgfqpoint{2.437841in}{2.273868in}}%
\pgfpathlineto{\pgfqpoint{2.438789in}{2.105598in}}%
\pgfpathlineto{\pgfqpoint{2.439010in}{2.366307in}}%
\pgfpathlineto{\pgfqpoint{2.439891in}{2.068885in}}%
\pgfpathlineto{\pgfqpoint{2.439935in}{2.365651in}}%
\pgfpathlineto{\pgfqpoint{2.440795in}{2.043644in}}%
\pgfpathlineto{\pgfqpoint{2.441037in}{2.384445in}}%
\pgfpathlineto{\pgfqpoint{2.441059in}{2.228631in}}%
\pgfpathlineto{\pgfqpoint{2.441897in}{2.350900in}}%
\pgfpathlineto{\pgfqpoint{2.441500in}{2.069103in}}%
\pgfpathlineto{\pgfqpoint{2.442162in}{2.304025in}}%
\pgfpathlineto{\pgfqpoint{2.442580in}{2.140891in}}%
\pgfpathlineto{\pgfqpoint{2.443021in}{2.470546in}}%
\pgfpathlineto{\pgfqpoint{2.443242in}{2.228194in}}%
\pgfpathlineto{\pgfqpoint{2.444388in}{2.501468in}}%
\pgfpathlineto{\pgfqpoint{2.444476in}{2.175637in}}%
\pgfpathlineto{\pgfqpoint{2.445512in}{2.392093in}}%
\pgfpathlineto{\pgfqpoint{2.446261in}{2.571399in}}%
\pgfpathlineto{\pgfqpoint{2.446328in}{2.239886in}}%
\pgfpathlineto{\pgfqpoint{2.446614in}{2.485953in}}%
\pgfpathlineto{\pgfqpoint{2.446923in}{2.229287in}}%
\pgfpathlineto{\pgfqpoint{2.447760in}{2.313859in}}%
\pgfpathlineto{\pgfqpoint{2.447782in}{2.315279in}}%
\pgfpathlineto{\pgfqpoint{2.447804in}{2.259444in}}%
\pgfpathlineto{\pgfqpoint{2.448444in}{2.130620in}}%
\pgfpathlineto{\pgfqpoint{2.448091in}{2.461914in}}%
\pgfpathlineto{\pgfqpoint{2.448884in}{2.287089in}}%
\pgfpathlineto{\pgfqpoint{2.449061in}{2.502015in}}%
\pgfpathlineto{\pgfqpoint{2.449744in}{2.199785in}}%
\pgfpathlineto{\pgfqpoint{2.449986in}{2.374174in}}%
\pgfpathlineto{\pgfqpoint{2.451045in}{2.210493in}}%
\pgfpathlineto{\pgfqpoint{2.450956in}{2.495240in}}%
\pgfpathlineto{\pgfqpoint{2.451111in}{2.268732in}}%
\pgfpathlineto{\pgfqpoint{2.451992in}{2.460822in}}%
\pgfpathlineto{\pgfqpoint{2.451199in}{2.180773in}}%
\pgfpathlineto{\pgfqpoint{2.452235in}{2.391001in}}%
\pgfpathlineto{\pgfqpoint{2.453227in}{2.122206in}}%
\pgfpathlineto{\pgfqpoint{2.452720in}{2.511739in}}%
\pgfpathlineto{\pgfqpoint{2.453381in}{2.210603in}}%
\pgfpathlineto{\pgfqpoint{2.453910in}{2.374720in}}%
\pgfpathlineto{\pgfqpoint{2.454417in}{2.006494in}}%
\pgfpathlineto{\pgfqpoint{2.454483in}{2.290039in}}%
\pgfpathlineto{\pgfqpoint{2.455387in}{2.074348in}}%
\pgfpathlineto{\pgfqpoint{2.455519in}{2.362264in}}%
\pgfpathlineto{\pgfqpoint{2.455585in}{2.281188in}}%
\pgfpathlineto{\pgfqpoint{2.456687in}{2.408374in}}%
\pgfpathlineto{\pgfqpoint{2.455916in}{2.107456in}}%
\pgfpathlineto{\pgfqpoint{2.456709in}{2.333527in}}%
\pgfpathlineto{\pgfqpoint{2.457525in}{2.106909in}}%
\pgfpathlineto{\pgfqpoint{2.457657in}{2.480599in}}%
\pgfpathlineto{\pgfqpoint{2.457789in}{2.229287in}}%
\pgfpathlineto{\pgfqpoint{2.458847in}{2.482565in}}%
\pgfpathlineto{\pgfqpoint{2.457988in}{2.166787in}}%
\pgfpathlineto{\pgfqpoint{2.458891in}{2.302714in}}%
\pgfpathlineto{\pgfqpoint{2.459707in}{2.106909in}}%
\pgfpathlineto{\pgfqpoint{2.459531in}{2.449567in}}%
\pgfpathlineto{\pgfqpoint{2.459949in}{2.345546in}}%
\pgfpathlineto{\pgfqpoint{2.460390in}{2.448365in}}%
\pgfpathlineto{\pgfqpoint{2.460148in}{2.147556in}}%
\pgfpathlineto{\pgfqpoint{2.460853in}{2.242399in}}%
\pgfpathlineto{\pgfqpoint{2.461470in}{2.139798in}}%
\pgfpathlineto{\pgfqpoint{2.461382in}{2.417006in}}%
\pgfpathlineto{\pgfqpoint{2.461955in}{2.241416in}}%
\pgfpathlineto{\pgfqpoint{2.462925in}{2.456779in}}%
\pgfpathlineto{\pgfqpoint{2.462396in}{2.070742in}}%
\pgfpathlineto{\pgfqpoint{2.463057in}{2.230270in}}%
\pgfpathlineto{\pgfqpoint{2.464270in}{2.432740in}}%
\pgfpathlineto{\pgfqpoint{2.463741in}{2.189405in}}%
\pgfpathlineto{\pgfqpoint{2.464292in}{2.347513in}}%
\pgfpathlineto{\pgfqpoint{2.465438in}{2.171376in}}%
\pgfpathlineto{\pgfqpoint{2.465041in}{2.431648in}}%
\pgfpathlineto{\pgfqpoint{2.465460in}{2.213006in}}%
\pgfpathlineto{\pgfqpoint{2.465835in}{2.468689in}}%
\pgfpathlineto{\pgfqpoint{2.466474in}{2.114121in}}%
\pgfpathlineto{\pgfqpoint{2.466584in}{2.406189in}}%
\pgfpathlineto{\pgfqpoint{2.466827in}{2.015344in}}%
\pgfpathlineto{\pgfqpoint{2.467708in}{2.267639in}}%
\pgfpathlineto{\pgfqpoint{2.467818in}{2.072709in}}%
\pgfpathlineto{\pgfqpoint{2.468524in}{2.397666in}}%
\pgfpathlineto{\pgfqpoint{2.468832in}{2.240541in}}%
\pgfpathlineto{\pgfqpoint{2.468965in}{2.430992in}}%
\pgfpathlineto{\pgfqpoint{2.469075in}{2.166678in}}%
\pgfpathlineto{\pgfqpoint{2.469957in}{2.273649in}}%
\pgfpathlineto{\pgfqpoint{2.471037in}{2.114339in}}%
\pgfpathlineto{\pgfqpoint{2.470089in}{2.440607in}}%
\pgfpathlineto{\pgfqpoint{2.471059in}{2.218251in}}%
\pgfpathlineto{\pgfqpoint{2.471477in}{2.178806in}}%
\pgfpathlineto{\pgfqpoint{2.472205in}{2.464427in}}%
\pgfpathlineto{\pgfqpoint{2.472976in}{2.203391in}}%
\pgfpathlineto{\pgfqpoint{2.472359in}{2.515782in}}%
\pgfpathlineto{\pgfqpoint{2.473329in}{2.318776in}}%
\pgfpathlineto{\pgfqpoint{2.474365in}{2.552823in}}%
\pgfpathlineto{\pgfqpoint{2.474145in}{2.205358in}}%
\pgfpathlineto{\pgfqpoint{2.474541in}{2.504965in}}%
\pgfpathlineto{\pgfqpoint{2.475004in}{2.279659in}}%
\pgfpathlineto{\pgfqpoint{2.475621in}{2.543864in}}%
\pgfpathlineto{\pgfqpoint{2.475643in}{2.508024in}}%
\pgfpathlineto{\pgfqpoint{2.475665in}{2.519934in}}%
\pgfpathlineto{\pgfqpoint{2.476128in}{2.183395in}}%
\pgfpathlineto{\pgfqpoint{2.476327in}{2.155970in}}%
\pgfpathlineto{\pgfqpoint{2.476437in}{2.379200in}}%
\pgfpathlineto{\pgfqpoint{2.477120in}{2.307849in}}%
\pgfpathlineto{\pgfqpoint{2.477980in}{2.468361in}}%
\pgfpathlineto{\pgfqpoint{2.477186in}{2.192355in}}%
\pgfpathlineto{\pgfqpoint{2.478222in}{2.303916in}}%
\pgfpathlineto{\pgfqpoint{2.478994in}{2.098496in}}%
\pgfpathlineto{\pgfqpoint{2.478377in}{2.390454in}}%
\pgfpathlineto{\pgfqpoint{2.479302in}{2.314733in}}%
\pgfpathlineto{\pgfqpoint{2.480382in}{2.402801in}}%
\pgfpathlineto{\pgfqpoint{2.479920in}{2.119912in}}%
\pgfpathlineto{\pgfqpoint{2.480404in}{2.387395in}}%
\pgfpathlineto{\pgfqpoint{2.480537in}{2.091393in}}%
\pgfpathlineto{\pgfqpoint{2.481507in}{2.239777in}}%
\pgfpathlineto{\pgfqpoint{2.482432in}{2.430336in}}%
\pgfpathlineto{\pgfqpoint{2.481749in}{2.123736in}}%
\pgfpathlineto{\pgfqpoint{2.482609in}{2.379200in}}%
\pgfpathlineto{\pgfqpoint{2.483380in}{2.153566in}}%
\pgfpathlineto{\pgfqpoint{2.483049in}{2.415476in}}%
\pgfpathlineto{\pgfqpoint{2.483733in}{2.280861in}}%
\pgfpathlineto{\pgfqpoint{2.484637in}{2.190170in}}%
\pgfpathlineto{\pgfqpoint{2.484328in}{2.451425in}}%
\pgfpathlineto{\pgfqpoint{2.484747in}{2.267749in}}%
\pgfpathlineto{\pgfqpoint{2.485232in}{2.452190in}}%
\pgfpathlineto{\pgfqpoint{2.485827in}{2.211477in}}%
\pgfpathlineto{\pgfqpoint{2.486444in}{2.362264in}}%
\pgfpathlineto{\pgfqpoint{2.486069in}{2.056210in}}%
\pgfpathlineto{\pgfqpoint{2.486973in}{2.252779in}}%
\pgfpathlineto{\pgfqpoint{2.487348in}{2.170393in}}%
\pgfpathlineto{\pgfqpoint{2.487634in}{2.481473in}}%
\pgfpathlineto{\pgfqpoint{2.488053in}{2.282718in}}%
\pgfpathlineto{\pgfqpoint{2.488758in}{2.399960in}}%
\pgfpathlineto{\pgfqpoint{2.488340in}{2.171267in}}%
\pgfpathlineto{\pgfqpoint{2.488847in}{2.191263in}}%
\pgfpathlineto{\pgfqpoint{2.488869in}{2.098059in}}%
\pgfpathlineto{\pgfqpoint{2.489640in}{2.314733in}}%
\pgfpathlineto{\pgfqpoint{2.489905in}{2.282063in}}%
\pgfpathlineto{\pgfqpoint{2.490345in}{2.440170in}}%
\pgfpathlineto{\pgfqpoint{2.490015in}{2.161761in}}%
\pgfpathlineto{\pgfqpoint{2.491007in}{2.365105in}}%
\pgfpathlineto{\pgfqpoint{2.491139in}{2.185034in}}%
\pgfpathlineto{\pgfqpoint{2.491822in}{2.473933in}}%
\pgfpathlineto{\pgfqpoint{2.492087in}{2.393295in}}%
\pgfpathlineto{\pgfqpoint{2.492109in}{2.465192in}}%
\pgfpathlineto{\pgfqpoint{2.493101in}{2.187547in}}%
\pgfpathlineto{\pgfqpoint{2.493167in}{2.401162in}}%
\pgfpathlineto{\pgfqpoint{2.494026in}{2.139361in}}%
\pgfpathlineto{\pgfqpoint{2.493387in}{2.427495in}}%
\pgfpathlineto{\pgfqpoint{2.494313in}{2.188203in}}%
\pgfpathlineto{\pgfqpoint{2.494379in}{2.138706in}}%
\pgfpathlineto{\pgfqpoint{2.494974in}{2.402801in}}%
\pgfpathlineto{\pgfqpoint{2.495261in}{2.396027in}}%
\pgfpathlineto{\pgfqpoint{2.496231in}{2.074129in}}%
\pgfpathlineto{\pgfqpoint{2.496539in}{2.199130in}}%
\pgfpathlineto{\pgfqpoint{2.497443in}{2.491088in}}%
\pgfpathlineto{\pgfqpoint{2.496671in}{2.101446in}}%
\pgfpathlineto{\pgfqpoint{2.497685in}{2.375594in}}%
\pgfpathlineto{\pgfqpoint{2.498258in}{2.168863in}}%
\pgfpathlineto{\pgfqpoint{2.498038in}{2.470874in}}%
\pgfpathlineto{\pgfqpoint{2.498655in}{2.406079in}}%
\pgfpathlineto{\pgfqpoint{2.499228in}{2.271792in}}%
\pgfpathlineto{\pgfqpoint{2.499757in}{2.512395in}}%
\pgfpathlineto{\pgfqpoint{2.500463in}{2.538073in}}%
\pgfpathlineto{\pgfqpoint{2.500881in}{2.239558in}}%
\pgfpathlineto{\pgfqpoint{2.501234in}{2.464646in}}%
\pgfpathlineto{\pgfqpoint{2.501587in}{2.153129in}}%
\pgfpathlineto{\pgfqpoint{2.502006in}{2.376687in}}%
\pgfpathlineto{\pgfqpoint{2.502226in}{2.103741in}}%
\pgfpathlineto{\pgfqpoint{2.502975in}{2.486827in}}%
\pgfpathlineto{\pgfqpoint{2.503130in}{2.292443in}}%
\pgfpathlineto{\pgfqpoint{2.503504in}{2.451534in}}%
\pgfpathlineto{\pgfqpoint{2.503394in}{2.140563in}}%
\pgfpathlineto{\pgfqpoint{2.504254in}{2.424108in}}%
\pgfpathlineto{\pgfqpoint{2.504871in}{2.186455in}}%
\pgfpathlineto{\pgfqpoint{2.505334in}{2.482238in}}%
\pgfpathlineto{\pgfqpoint{2.505400in}{2.252342in}}%
\pgfpathlineto{\pgfqpoint{2.506149in}{2.531735in}}%
\pgfpathlineto{\pgfqpoint{2.506238in}{2.223059in}}%
\pgfpathlineto{\pgfqpoint{2.506524in}{2.352976in}}%
\pgfpathlineto{\pgfqpoint{2.507163in}{2.558396in}}%
\pgfpathlineto{\pgfqpoint{2.507406in}{2.176402in}}%
\pgfpathlineto{\pgfqpoint{2.507692in}{2.457544in}}%
\pgfpathlineto{\pgfqpoint{2.508486in}{2.172250in}}%
\pgfpathlineto{\pgfqpoint{2.508023in}{2.504965in}}%
\pgfpathlineto{\pgfqpoint{2.508817in}{2.388378in}}%
\pgfpathlineto{\pgfqpoint{2.508839in}{2.388488in}}%
\pgfpathlineto{\pgfqpoint{2.509676in}{2.508789in}}%
\pgfpathlineto{\pgfqpoint{2.509324in}{2.192137in}}%
\pgfpathlineto{\pgfqpoint{2.509985in}{2.476228in}}%
\pgfpathlineto{\pgfqpoint{2.511087in}{2.290476in}}%
\pgfpathlineto{\pgfqpoint{2.510271in}{2.578610in}}%
\pgfpathlineto{\pgfqpoint{2.511153in}{2.369585in}}%
\pgfpathlineto{\pgfqpoint{2.512233in}{2.231691in}}%
\pgfpathlineto{\pgfqpoint{2.511748in}{2.494257in}}%
\pgfpathlineto{\pgfqpoint{2.512299in}{2.321945in}}%
\pgfpathlineto{\pgfqpoint{2.513269in}{2.451534in}}%
\pgfpathlineto{\pgfqpoint{2.512872in}{2.191481in}}%
\pgfpathlineto{\pgfqpoint{2.513423in}{2.394606in}}%
\pgfpathlineto{\pgfqpoint{2.513864in}{2.158483in}}%
\pgfpathlineto{\pgfqpoint{2.513974in}{2.481473in}}%
\pgfpathlineto{\pgfqpoint{2.514547in}{2.291569in}}%
\pgfpathlineto{\pgfqpoint{2.514768in}{2.541023in}}%
\pgfpathlineto{\pgfqpoint{2.515319in}{2.243164in}}%
\pgfpathlineto{\pgfqpoint{2.515650in}{2.393077in}}%
\pgfpathlineto{\pgfqpoint{2.516774in}{2.186455in}}%
\pgfpathlineto{\pgfqpoint{2.516553in}{2.548016in}}%
\pgfpathlineto{\pgfqpoint{2.516840in}{2.307849in}}%
\pgfpathlineto{\pgfqpoint{2.517699in}{2.717050in}}%
\pgfpathlineto{\pgfqpoint{2.517082in}{2.301184in}}%
\pgfpathlineto{\pgfqpoint{2.517986in}{2.470109in}}%
\pgfpathlineto{\pgfqpoint{2.518934in}{2.251905in}}%
\pgfpathlineto{\pgfqpoint{2.518339in}{2.622972in}}%
\pgfpathlineto{\pgfqpoint{2.519198in}{2.361390in}}%
\pgfpathlineto{\pgfqpoint{2.519639in}{2.454484in}}%
\pgfpathlineto{\pgfqpoint{2.519419in}{2.134007in}}%
\pgfpathlineto{\pgfqpoint{2.520278in}{2.288400in}}%
\pgfpathlineto{\pgfqpoint{2.520807in}{2.446289in}}%
\pgfpathlineto{\pgfqpoint{2.520543in}{2.229943in}}%
\pgfpathlineto{\pgfqpoint{2.521336in}{2.294628in}}%
\pgfpathlineto{\pgfqpoint{2.521358in}{2.252889in}}%
\pgfpathlineto{\pgfqpoint{2.521689in}{2.538400in}}%
\pgfpathlineto{\pgfqpoint{2.522350in}{2.337679in}}%
\pgfpathlineto{\pgfqpoint{2.523320in}{2.580796in}}%
\pgfpathlineto{\pgfqpoint{2.522703in}{2.328828in}}%
\pgfpathlineto{\pgfqpoint{2.523474in}{2.460275in}}%
\pgfpathlineto{\pgfqpoint{2.524510in}{2.197709in}}%
\pgfpathlineto{\pgfqpoint{2.523805in}{2.580577in}}%
\pgfpathlineto{\pgfqpoint{2.524599in}{2.274742in}}%
\pgfpathlineto{\pgfqpoint{2.525392in}{2.525398in}}%
\pgfpathlineto{\pgfqpoint{2.524841in}{2.240323in}}%
\pgfpathlineto{\pgfqpoint{2.525701in}{2.442137in}}%
\pgfpathlineto{\pgfqpoint{2.526626in}{2.199567in}}%
\pgfpathlineto{\pgfqpoint{2.526009in}{2.450441in}}%
\pgfpathlineto{\pgfqpoint{2.526825in}{2.326424in}}%
\pgfpathlineto{\pgfqpoint{2.527685in}{2.506932in}}%
\pgfpathlineto{\pgfqpoint{2.527001in}{2.233111in}}%
\pgfpathlineto{\pgfqpoint{2.527905in}{2.289820in}}%
\pgfpathlineto{\pgfqpoint{2.527927in}{2.222622in}}%
\pgfpathlineto{\pgfqpoint{2.528787in}{2.612701in}}%
\pgfpathlineto{\pgfqpoint{2.528963in}{2.450660in}}%
\pgfpathlineto{\pgfqpoint{2.529624in}{2.552605in}}%
\pgfpathlineto{\pgfqpoint{2.529249in}{2.262395in}}%
\pgfpathlineto{\pgfqpoint{2.529933in}{2.375485in}}%
\pgfpathlineto{\pgfqpoint{2.529955in}{2.285450in}}%
\pgfpathlineto{\pgfqpoint{2.530528in}{2.562876in}}%
\pgfpathlineto{\pgfqpoint{2.531035in}{2.300747in}}%
\pgfpathlineto{\pgfqpoint{2.531917in}{2.500594in}}%
\pgfpathlineto{\pgfqpoint{2.531476in}{2.279440in}}%
\pgfpathlineto{\pgfqpoint{2.532159in}{2.446945in}}%
\pgfpathlineto{\pgfqpoint{2.532864in}{2.261958in}}%
\pgfpathlineto{\pgfqpoint{2.532291in}{2.540476in}}%
\pgfpathlineto{\pgfqpoint{2.533305in}{2.369366in}}%
\pgfpathlineto{\pgfqpoint{2.533592in}{2.590957in}}%
\pgfpathlineto{\pgfqpoint{2.534385in}{2.310144in}}%
\pgfpathlineto{\pgfqpoint{2.534848in}{2.276927in}}%
\pgfpathlineto{\pgfqpoint{2.534782in}{2.490979in}}%
\pgfpathlineto{\pgfqpoint{2.535377in}{2.372316in}}%
\pgfpathlineto{\pgfqpoint{2.535708in}{2.503763in}}%
\pgfpathlineto{\pgfqpoint{2.535487in}{2.195742in}}%
\pgfpathlineto{\pgfqpoint{2.536435in}{2.290476in}}%
\pgfpathlineto{\pgfqpoint{2.536876in}{2.200441in}}%
\pgfpathlineto{\pgfqpoint{2.537141in}{2.459292in}}%
\pgfpathlineto{\pgfqpoint{2.537449in}{2.425638in}}%
\pgfpathlineto{\pgfqpoint{2.537625in}{2.489121in}}%
\pgfpathlineto{\pgfqpoint{2.538221in}{2.197491in}}%
\pgfpathlineto{\pgfqpoint{2.538485in}{2.341831in}}%
\pgfpathlineto{\pgfqpoint{2.538728in}{2.227102in}}%
\pgfpathlineto{\pgfqpoint{2.538860in}{2.466940in}}%
\pgfpathlineto{\pgfqpoint{2.539146in}{2.394825in}}%
\pgfpathlineto{\pgfqpoint{2.539168in}{2.478632in}}%
\pgfpathlineto{\pgfqpoint{2.539389in}{2.213553in}}%
\pgfpathlineto{\pgfqpoint{2.540248in}{2.418208in}}%
\pgfpathlineto{\pgfqpoint{2.540381in}{2.202626in}}%
\pgfpathlineto{\pgfqpoint{2.541108in}{2.514690in}}%
\pgfpathlineto{\pgfqpoint{2.541351in}{2.381057in}}%
\pgfpathlineto{\pgfqpoint{2.542122in}{2.603086in}}%
\pgfpathlineto{\pgfqpoint{2.541593in}{2.273649in}}%
\pgfpathlineto{\pgfqpoint{2.542431in}{2.501796in}}%
\pgfpathlineto{\pgfqpoint{2.542453in}{2.322819in}}%
\pgfpathlineto{\pgfqpoint{2.543070in}{2.668536in}}%
\pgfpathlineto{\pgfqpoint{2.543555in}{2.378654in}}%
\pgfpathlineto{\pgfqpoint{2.543621in}{2.308068in}}%
\pgfpathlineto{\pgfqpoint{2.543863in}{2.581014in}}%
\pgfpathlineto{\pgfqpoint{2.544613in}{2.470655in}}%
\pgfpathlineto{\pgfqpoint{2.544899in}{2.578719in}}%
\pgfpathlineto{\pgfqpoint{2.545142in}{2.315279in}}%
\pgfpathlineto{\pgfqpoint{2.545693in}{2.482238in}}%
\pgfpathlineto{\pgfqpoint{2.546134in}{2.212569in}}%
\pgfpathlineto{\pgfqpoint{2.546641in}{2.558177in}}%
\pgfpathlineto{\pgfqpoint{2.546795in}{2.425419in}}%
\pgfpathlineto{\pgfqpoint{2.547875in}{2.644825in}}%
\pgfpathlineto{\pgfqpoint{2.547544in}{2.378872in}}%
\pgfpathlineto{\pgfqpoint{2.547897in}{2.495787in}}%
\pgfpathlineto{\pgfqpoint{2.548889in}{2.339646in}}%
\pgfpathlineto{\pgfqpoint{2.548360in}{2.614449in}}%
\pgfpathlineto{\pgfqpoint{2.549021in}{2.410341in}}%
\pgfpathlineto{\pgfqpoint{2.550035in}{2.664384in}}%
\pgfpathlineto{\pgfqpoint{2.549087in}{2.340629in}}%
\pgfpathlineto{\pgfqpoint{2.550145in}{2.462788in}}%
\pgfpathlineto{\pgfqpoint{2.550608in}{2.670066in}}%
\pgfpathlineto{\pgfqpoint{2.551159in}{2.337351in}}%
\pgfpathlineto{\pgfqpoint{2.551203in}{2.438531in}}%
\pgfpathlineto{\pgfqpoint{2.551776in}{2.258024in}}%
\pgfpathlineto{\pgfqpoint{2.551997in}{2.529441in}}%
\pgfpathlineto{\pgfqpoint{2.552283in}{2.445852in}}%
\pgfpathlineto{\pgfqpoint{2.553253in}{2.661762in}}%
\pgfpathlineto{\pgfqpoint{2.552327in}{2.395262in}}%
\pgfpathlineto{\pgfqpoint{2.553452in}{2.565717in}}%
\pgfpathlineto{\pgfqpoint{2.553716in}{2.661762in}}%
\pgfpathlineto{\pgfqpoint{2.554576in}{2.319322in}}%
\pgfpathlineto{\pgfqpoint{2.555502in}{2.699131in}}%
\pgfpathlineto{\pgfqpoint{2.555700in}{2.456997in}}%
\pgfpathlineto{\pgfqpoint{2.555986in}{2.684598in}}%
\pgfpathlineto{\pgfqpoint{2.556692in}{2.354178in}}%
\pgfpathlineto{\pgfqpoint{2.556802in}{2.541132in}}%
\pgfpathlineto{\pgfqpoint{2.557750in}{2.310472in}}%
\pgfpathlineto{\pgfqpoint{2.557419in}{2.588772in}}%
\pgfpathlineto{\pgfqpoint{2.557926in}{2.434051in}}%
\pgfpathlineto{\pgfqpoint{2.558455in}{2.309925in}}%
\pgfpathlineto{\pgfqpoint{2.557992in}{2.559161in}}%
\pgfpathlineto{\pgfqpoint{2.558587in}{2.478413in}}%
\pgfpathlineto{\pgfqpoint{2.558720in}{2.672470in}}%
\pgfpathlineto{\pgfqpoint{2.559249in}{2.317683in}}%
\pgfpathlineto{\pgfqpoint{2.559667in}{2.519497in}}%
\pgfpathlineto{\pgfqpoint{2.559910in}{2.203500in}}%
\pgfpathlineto{\pgfqpoint{2.560285in}{2.624720in}}%
\pgfpathlineto{\pgfqpoint{2.560770in}{2.443339in}}%
\pgfpathlineto{\pgfqpoint{2.561012in}{2.525835in}}%
\pgfpathlineto{\pgfqpoint{2.561761in}{2.182740in}}%
\pgfpathlineto{\pgfqpoint{2.561783in}{2.256057in}}%
\pgfpathlineto{\pgfqpoint{2.561806in}{2.239995in}}%
\pgfpathlineto{\pgfqpoint{2.562092in}{2.518186in}}%
\pgfpathlineto{\pgfqpoint{2.562775in}{2.262285in}}%
\pgfpathlineto{\pgfqpoint{2.563569in}{2.465848in}}%
\pgfpathlineto{\pgfqpoint{2.563833in}{2.169628in}}%
\pgfpathlineto{\pgfqpoint{2.563900in}{2.393951in}}%
\pgfpathlineto{\pgfqpoint{2.564539in}{2.287854in}}%
\pgfpathlineto{\pgfqpoint{2.564803in}{2.622535in}}%
\pgfpathlineto{\pgfqpoint{2.564980in}{2.344344in}}%
\pgfpathlineto{\pgfqpoint{2.565046in}{2.584511in}}%
\pgfpathlineto{\pgfqpoint{2.565156in}{2.275834in}}%
\pgfpathlineto{\pgfqpoint{2.566104in}{2.474152in}}%
\pgfpathlineto{\pgfqpoint{2.566699in}{2.271901in}}%
\pgfpathlineto{\pgfqpoint{2.566192in}{2.632151in}}%
\pgfpathlineto{\pgfqpoint{2.566963in}{2.463335in}}%
\pgfpathlineto{\pgfqpoint{2.566985in}{2.609423in}}%
\pgfpathlineto{\pgfqpoint{2.567052in}{2.296049in}}%
\pgfpathlineto{\pgfqpoint{2.568065in}{2.412963in}}%
\pgfpathlineto{\pgfqpoint{2.568242in}{2.226228in}}%
\pgfpathlineto{\pgfqpoint{2.568308in}{2.533374in}}%
\pgfpathlineto{\pgfqpoint{2.569035in}{2.396792in}}%
\pgfpathlineto{\pgfqpoint{2.569939in}{2.566045in}}%
\pgfpathlineto{\pgfqpoint{2.569873in}{2.302823in}}%
\pgfpathlineto{\pgfqpoint{2.570159in}{2.543099in}}%
\pgfpathlineto{\pgfqpoint{2.570556in}{2.295721in}}%
\pgfpathlineto{\pgfqpoint{2.570248in}{2.585166in}}%
\pgfpathlineto{\pgfqpoint{2.571284in}{2.383571in}}%
\pgfpathlineto{\pgfqpoint{2.571306in}{2.383571in}}%
\pgfpathlineto{\pgfqpoint{2.571328in}{2.292989in}}%
\pgfpathlineto{\pgfqpoint{2.571526in}{2.659576in}}%
\pgfpathlineto{\pgfqpoint{2.572364in}{2.518186in}}%
\pgfpathlineto{\pgfqpoint{2.573378in}{2.663401in}}%
\pgfpathlineto{\pgfqpoint{2.572628in}{2.321180in}}%
\pgfpathlineto{\pgfqpoint{2.573444in}{2.621442in}}%
\pgfpathlineto{\pgfqpoint{2.574215in}{2.348168in}}%
\pgfpathlineto{\pgfqpoint{2.574017in}{2.640018in}}%
\pgfpathlineto{\pgfqpoint{2.574568in}{2.450332in}}%
\pgfpathlineto{\pgfqpoint{2.575405in}{2.521246in}}%
\pgfpathlineto{\pgfqpoint{2.575295in}{2.233548in}}%
\pgfpathlineto{\pgfqpoint{2.575582in}{2.344781in}}%
\pgfpathlineto{\pgfqpoint{2.575802in}{2.237373in}}%
\pgfpathlineto{\pgfqpoint{2.575868in}{2.530861in}}%
\pgfpathlineto{\pgfqpoint{2.576331in}{2.272338in}}%
\pgfpathlineto{\pgfqpoint{2.576772in}{2.571399in}}%
\pgfpathlineto{\pgfqpoint{2.576419in}{2.238465in}}%
\pgfpathlineto{\pgfqpoint{2.577433in}{2.320524in}}%
\pgfpathlineto{\pgfqpoint{2.578073in}{2.677059in}}%
\pgfpathlineto{\pgfqpoint{2.578602in}{2.576753in}}%
\pgfpathlineto{\pgfqpoint{2.579549in}{2.333527in}}%
\pgfpathlineto{\pgfqpoint{2.578734in}{2.686783in}}%
\pgfpathlineto{\pgfqpoint{2.579726in}{2.488357in}}%
\pgfpathlineto{\pgfqpoint{2.579770in}{2.651491in}}%
\pgfpathlineto{\pgfqpoint{2.580541in}{2.360515in}}%
\pgfpathlineto{\pgfqpoint{2.580806in}{2.420502in}}%
\pgfpathlineto{\pgfqpoint{2.581665in}{2.314187in}}%
\pgfpathlineto{\pgfqpoint{2.581379in}{2.668208in}}%
\pgfpathlineto{\pgfqpoint{2.581842in}{2.456560in}}%
\pgfpathlineto{\pgfqpoint{2.582944in}{2.718908in}}%
\pgfpathlineto{\pgfqpoint{2.582613in}{2.423453in}}%
\pgfpathlineto{\pgfqpoint{2.582966in}{2.606801in}}%
\pgfpathlineto{\pgfqpoint{2.583605in}{2.455468in}}%
\pgfpathlineto{\pgfqpoint{2.583098in}{2.717924in}}%
\pgfpathlineto{\pgfqpoint{2.584068in}{2.537198in}}%
\pgfpathlineto{\pgfqpoint{2.584751in}{2.741963in}}%
\pgfpathlineto{\pgfqpoint{2.584156in}{2.407172in}}%
\pgfpathlineto{\pgfqpoint{2.585192in}{2.682522in}}%
\pgfpathlineto{\pgfqpoint{2.585435in}{2.440717in}}%
\pgfpathlineto{\pgfqpoint{2.585302in}{2.778130in}}%
\pgfpathlineto{\pgfqpoint{2.586316in}{2.523212in}}%
\pgfpathlineto{\pgfqpoint{2.586625in}{2.688532in}}%
\pgfpathlineto{\pgfqpoint{2.587220in}{2.397447in}}%
\pgfpathlineto{\pgfqpoint{2.587418in}{2.526381in}}%
\pgfpathlineto{\pgfqpoint{2.587440in}{2.526490in}}%
\pgfpathlineto{\pgfqpoint{2.588190in}{2.406844in}}%
\pgfpathlineto{\pgfqpoint{2.588300in}{2.648103in}}%
\pgfpathlineto{\pgfqpoint{2.588542in}{2.554681in}}%
\pgfpathlineto{\pgfqpoint{2.588895in}{2.611390in}}%
\pgfpathlineto{\pgfqpoint{2.588763in}{2.251140in}}%
\pgfpathlineto{\pgfqpoint{2.589512in}{2.509882in}}%
\pgfpathlineto{\pgfqpoint{2.589733in}{2.333417in}}%
\pgfpathlineto{\pgfqpoint{2.590328in}{2.638488in}}%
\pgfpathlineto{\pgfqpoint{2.590592in}{2.559598in}}%
\pgfpathlineto{\pgfqpoint{2.591143in}{2.674218in}}%
\pgfpathlineto{\pgfqpoint{2.590636in}{2.433177in}}%
\pgfpathlineto{\pgfqpoint{2.591606in}{2.580249in}}%
\pgfpathlineto{\pgfqpoint{2.592422in}{2.310035in}}%
\pgfpathlineto{\pgfqpoint{2.591739in}{2.663291in}}%
\pgfpathlineto{\pgfqpoint{2.592730in}{2.427823in}}%
\pgfpathlineto{\pgfqpoint{2.592752in}{2.427386in}}%
\pgfpathlineto{\pgfqpoint{2.593083in}{2.630074in}}%
\pgfpathlineto{\pgfqpoint{2.593700in}{2.342596in}}%
\pgfpathlineto{\pgfqpoint{2.593855in}{2.392640in}}%
\pgfpathlineto{\pgfqpoint{2.594339in}{2.259007in}}%
\pgfpathlineto{\pgfqpoint{2.594802in}{2.603195in}}%
\pgfpathlineto{\pgfqpoint{2.594913in}{2.526272in}}%
\pgfpathlineto{\pgfqpoint{2.595243in}{2.368710in}}%
\pgfpathlineto{\pgfqpoint{2.595552in}{2.618383in}}%
\pgfpathlineto{\pgfqpoint{2.595574in}{2.595328in}}%
\pgfpathlineto{\pgfqpoint{2.595596in}{2.616963in}}%
\pgfpathlineto{\pgfqpoint{2.596257in}{2.287854in}}%
\pgfpathlineto{\pgfqpoint{2.596456in}{2.387067in}}%
\pgfpathlineto{\pgfqpoint{2.596478in}{2.292333in}}%
\pgfpathlineto{\pgfqpoint{2.597249in}{2.744694in}}%
\pgfpathlineto{\pgfqpoint{2.597558in}{2.425419in}}%
\pgfpathlineto{\pgfqpoint{2.597624in}{2.352320in}}%
\pgfpathlineto{\pgfqpoint{2.597800in}{2.613685in}}%
\pgfpathlineto{\pgfqpoint{2.598594in}{2.439187in}}%
\pgfpathlineto{\pgfqpoint{2.599343in}{2.588116in}}%
\pgfpathlineto{\pgfqpoint{2.598836in}{2.310253in}}%
\pgfpathlineto{\pgfqpoint{2.599696in}{2.470765in}}%
\pgfpathlineto{\pgfqpoint{2.600357in}{2.594454in}}%
\pgfpathlineto{\pgfqpoint{2.600666in}{2.303151in}}%
\pgfpathlineto{\pgfqpoint{2.600688in}{2.437876in}}%
\pgfpathlineto{\pgfqpoint{2.600710in}{2.299545in}}%
\pgfpathlineto{\pgfqpoint{2.601724in}{2.618274in}}%
\pgfpathlineto{\pgfqpoint{2.601768in}{2.520918in}}%
\pgfpathlineto{\pgfqpoint{2.601834in}{2.628108in}}%
\pgfpathlineto{\pgfqpoint{2.602054in}{2.325332in}}%
\pgfpathlineto{\pgfqpoint{2.602848in}{2.440389in}}%
\pgfpathlineto{\pgfqpoint{2.604016in}{2.655096in}}%
\pgfpathlineto{\pgfqpoint{2.602980in}{2.335384in}}%
\pgfpathlineto{\pgfqpoint{2.604060in}{2.600245in}}%
\pgfpathlineto{\pgfqpoint{2.604898in}{2.373409in}}%
\pgfpathlineto{\pgfqpoint{2.604677in}{2.655096in}}%
\pgfpathlineto{\pgfqpoint{2.605162in}{2.650507in}}%
\pgfpathlineto{\pgfqpoint{2.605779in}{2.360406in}}%
\pgfpathlineto{\pgfqpoint{2.606286in}{2.419082in}}%
\pgfpathlineto{\pgfqpoint{2.606308in}{2.590411in}}%
\pgfpathlineto{\pgfqpoint{2.606992in}{2.277036in}}%
\pgfpathlineto{\pgfqpoint{2.607388in}{2.411652in}}%
\pgfpathlineto{\pgfqpoint{2.608292in}{2.671486in}}%
\pgfpathlineto{\pgfqpoint{2.607697in}{2.348496in}}%
\pgfpathlineto{\pgfqpoint{2.608535in}{2.558396in}}%
\pgfpathlineto{\pgfqpoint{2.608887in}{2.414930in}}%
\pgfpathlineto{\pgfqpoint{2.609240in}{2.709183in}}%
\pgfpathlineto{\pgfqpoint{2.609659in}{2.449895in}}%
\pgfpathlineto{\pgfqpoint{2.610188in}{2.661215in}}%
\pgfpathlineto{\pgfqpoint{2.610011in}{2.330795in}}%
\pgfpathlineto{\pgfqpoint{2.610805in}{2.512177in}}%
\pgfpathlineto{\pgfqpoint{2.610827in}{2.448365in}}%
\pgfpathlineto{\pgfqpoint{2.611202in}{2.763597in}}%
\pgfpathlineto{\pgfqpoint{2.611885in}{2.503872in}}%
\pgfpathlineto{\pgfqpoint{2.612392in}{2.649961in}}%
\pgfpathlineto{\pgfqpoint{2.612061in}{2.386302in}}%
\pgfpathlineto{\pgfqpoint{2.612987in}{2.491197in}}%
\pgfpathlineto{\pgfqpoint{2.613406in}{2.644388in}}%
\pgfpathlineto{\pgfqpoint{2.613913in}{2.391219in}}%
\pgfpathlineto{\pgfqpoint{2.613935in}{2.345109in}}%
\pgfpathlineto{\pgfqpoint{2.614376in}{2.647448in}}%
\pgfpathlineto{\pgfqpoint{2.614993in}{2.413619in}}%
\pgfpathlineto{\pgfqpoint{2.615103in}{2.663838in}}%
\pgfpathlineto{\pgfqpoint{2.615588in}{2.288618in}}%
\pgfpathlineto{\pgfqpoint{2.615919in}{2.383024in}}%
\pgfpathlineto{\pgfqpoint{2.616514in}{2.304462in}}%
\pgfpathlineto{\pgfqpoint{2.616448in}{2.551731in}}%
\pgfpathlineto{\pgfqpoint{2.617021in}{2.365651in}}%
\pgfpathlineto{\pgfqpoint{2.618079in}{2.665695in}}%
\pgfpathlineto{\pgfqpoint{2.617197in}{2.332325in}}%
\pgfpathlineto{\pgfqpoint{2.618145in}{2.593361in}}%
\pgfpathlineto{\pgfqpoint{2.618740in}{2.420830in}}%
\pgfpathlineto{\pgfqpoint{2.618387in}{2.750376in}}%
\pgfpathlineto{\pgfqpoint{2.619269in}{2.512614in}}%
\pgfpathlineto{\pgfqpoint{2.619996in}{2.726119in}}%
\pgfpathlineto{\pgfqpoint{2.619754in}{2.388597in}}%
\pgfpathlineto{\pgfqpoint{2.620393in}{2.593689in}}%
\pgfpathlineto{\pgfqpoint{2.620856in}{2.336477in}}%
\pgfpathlineto{\pgfqpoint{2.620547in}{2.689734in}}%
\pgfpathlineto{\pgfqpoint{2.621517in}{2.375485in}}%
\pgfpathlineto{\pgfqpoint{2.622267in}{2.651600in}}%
\pgfpathlineto{\pgfqpoint{2.622068in}{2.345655in}}%
\pgfpathlineto{\pgfqpoint{2.622641in}{2.483767in}}%
\pgfpathlineto{\pgfqpoint{2.623192in}{2.780534in}}%
\pgfpathlineto{\pgfqpoint{2.623501in}{2.434379in}}%
\pgfpathlineto{\pgfqpoint{2.623721in}{2.488357in}}%
\pgfpathlineto{\pgfqpoint{2.623876in}{2.425856in}}%
\pgfpathlineto{\pgfqpoint{2.624515in}{2.752015in}}%
\pgfpathlineto{\pgfqpoint{2.624735in}{2.564187in}}%
\pgfpathlineto{\pgfqpoint{2.625771in}{2.759992in}}%
\pgfpathlineto{\pgfqpoint{2.625815in}{2.443339in}}%
\pgfpathlineto{\pgfqpoint{2.625837in}{2.584401in}}%
\pgfpathlineto{\pgfqpoint{2.625882in}{2.717815in}}%
\pgfpathlineto{\pgfqpoint{2.626344in}{2.384991in}}%
\pgfpathlineto{\pgfqpoint{2.626873in}{2.559926in}}%
\pgfpathlineto{\pgfqpoint{2.627094in}{2.409139in}}%
\pgfpathlineto{\pgfqpoint{2.627402in}{2.682304in}}%
\pgfpathlineto{\pgfqpoint{2.627998in}{2.461805in}}%
\pgfpathlineto{\pgfqpoint{2.628967in}{2.747645in}}%
\pgfpathlineto{\pgfqpoint{2.628394in}{2.383571in}}%
\pgfpathlineto{\pgfqpoint{2.629100in}{2.588335in}}%
\pgfpathlineto{\pgfqpoint{2.629937in}{2.413291in}}%
\pgfpathlineto{\pgfqpoint{2.629452in}{2.733986in}}%
\pgfpathlineto{\pgfqpoint{2.630224in}{2.456014in}}%
\pgfpathlineto{\pgfqpoint{2.630400in}{2.433287in}}%
\pgfpathlineto{\pgfqpoint{2.630466in}{2.639144in}}%
\pgfpathlineto{\pgfqpoint{2.630709in}{2.574458in}}%
\pgfpathlineto{\pgfqpoint{2.631282in}{2.659795in}}%
\pgfpathlineto{\pgfqpoint{2.631524in}{2.350681in}}%
\pgfpathlineto{\pgfqpoint{2.631767in}{2.452517in}}%
\pgfpathlineto{\pgfqpoint{2.632097in}{2.807850in}}%
\pgfpathlineto{\pgfqpoint{2.632208in}{2.349480in}}%
\pgfpathlineto{\pgfqpoint{2.632671in}{2.565389in}}%
\pgfpathlineto{\pgfqpoint{2.632737in}{2.295065in}}%
\pgfpathlineto{\pgfqpoint{2.633464in}{2.623191in}}%
\pgfpathlineto{\pgfqpoint{2.633773in}{2.445415in}}%
\pgfpathlineto{\pgfqpoint{2.634831in}{2.648322in}}%
\pgfpathlineto{\pgfqpoint{2.633817in}{2.343251in}}%
\pgfpathlineto{\pgfqpoint{2.634897in}{2.588772in}}%
\pgfpathlineto{\pgfqpoint{2.635161in}{2.366744in}}%
\pgfpathlineto{\pgfqpoint{2.635756in}{2.648540in}}%
\pgfpathlineto{\pgfqpoint{2.635999in}{2.624611in}}%
\pgfpathlineto{\pgfqpoint{2.636175in}{2.412307in}}%
\pgfpathlineto{\pgfqpoint{2.636792in}{2.694651in}}%
\pgfpathlineto{\pgfqpoint{2.637101in}{2.463225in}}%
\pgfpathlineto{\pgfqpoint{2.637233in}{2.681757in}}%
\pgfpathlineto{\pgfqpoint{2.637586in}{2.412417in}}%
\pgfpathlineto{\pgfqpoint{2.638225in}{2.671268in}}%
\pgfpathlineto{\pgfqpoint{2.639305in}{2.399633in}}%
\pgfpathlineto{\pgfqpoint{2.638952in}{2.709292in}}%
\pgfpathlineto{\pgfqpoint{2.639371in}{2.564952in}}%
\pgfpathlineto{\pgfqpoint{2.639393in}{2.680774in}}%
\pgfpathlineto{\pgfqpoint{2.640429in}{2.386302in}}%
\pgfpathlineto{\pgfqpoint{2.640473in}{2.520481in}}%
\pgfpathlineto{\pgfqpoint{2.640716in}{2.462133in}}%
\pgfpathlineto{\pgfqpoint{2.640980in}{2.759992in}}%
\pgfpathlineto{\pgfqpoint{2.641509in}{2.530642in}}%
\pgfpathlineto{\pgfqpoint{2.641575in}{2.791897in}}%
\pgfpathlineto{\pgfqpoint{2.641642in}{2.498518in}}%
\pgfpathlineto{\pgfqpoint{2.642611in}{2.551184in}}%
\pgfpathlineto{\pgfqpoint{2.643030in}{2.455249in}}%
\pgfpathlineto{\pgfqpoint{2.643625in}{2.704375in}}%
\pgfpathlineto{\pgfqpoint{2.643692in}{2.582325in}}%
\pgfpathlineto{\pgfqpoint{2.643824in}{2.753217in}}%
\pgfpathlineto{\pgfqpoint{2.644705in}{2.510647in}}%
\pgfpathlineto{\pgfqpoint{2.644772in}{2.678261in}}%
\pgfpathlineto{\pgfqpoint{2.645190in}{2.476228in}}%
\pgfpathlineto{\pgfqpoint{2.645367in}{2.745241in}}%
\pgfpathlineto{\pgfqpoint{2.645896in}{2.521573in}}%
\pgfpathlineto{\pgfqpoint{2.647152in}{2.847077in}}%
\pgfpathlineto{\pgfqpoint{2.647593in}{2.417224in}}%
\pgfpathlineto{\pgfqpoint{2.648364in}{2.695853in}}%
\pgfpathlineto{\pgfqpoint{2.649665in}{2.457981in}}%
\pgfpathlineto{\pgfqpoint{2.649026in}{2.725027in}}%
\pgfpathlineto{\pgfqpoint{2.649731in}{2.520699in}}%
\pgfpathlineto{\pgfqpoint{2.650304in}{2.717706in}}%
\pgfpathlineto{\pgfqpoint{2.650745in}{2.442683in}}%
\pgfpathlineto{\pgfqpoint{2.650833in}{2.558177in}}%
\pgfpathlineto{\pgfqpoint{2.650899in}{2.511412in}}%
\pgfpathlineto{\pgfqpoint{2.651164in}{2.667225in}}%
\pgfpathlineto{\pgfqpoint{2.651186in}{2.656626in}}%
\pgfpathlineto{\pgfqpoint{2.652023in}{2.743055in}}%
\pgfpathlineto{\pgfqpoint{2.652222in}{2.507587in}}%
\pgfpathlineto{\pgfqpoint{2.652266in}{2.595219in}}%
\pgfpathlineto{\pgfqpoint{2.652773in}{2.436237in}}%
\pgfpathlineto{\pgfqpoint{2.653214in}{2.701971in}}%
\pgfpathlineto{\pgfqpoint{2.653346in}{2.558942in}}%
\pgfpathlineto{\pgfqpoint{2.653699in}{2.683287in}}%
\pgfpathlineto{\pgfqpoint{2.654426in}{2.475026in}}%
\pgfpathlineto{\pgfqpoint{2.654448in}{2.583855in}}%
\pgfpathlineto{\pgfqpoint{2.654470in}{2.461368in}}%
\pgfpathlineto{\pgfqpoint{2.655109in}{2.759555in}}%
\pgfpathlineto{\pgfqpoint{2.655550in}{2.589318in}}%
\pgfpathlineto{\pgfqpoint{2.656344in}{2.475572in}}%
\pgfpathlineto{\pgfqpoint{2.656145in}{2.731473in}}%
\pgfpathlineto{\pgfqpoint{2.656630in}{2.623081in}}%
\pgfpathlineto{\pgfqpoint{2.657512in}{2.442465in}}%
\pgfpathlineto{\pgfqpoint{2.657358in}{2.725136in}}%
\pgfpathlineto{\pgfqpoint{2.657754in}{2.464427in}}%
\pgfpathlineto{\pgfqpoint{2.658019in}{2.766657in}}%
\pgfpathlineto{\pgfqpoint{2.658856in}{2.447163in}}%
\pgfpathlineto{\pgfqpoint{2.658878in}{2.595328in}}%
\pgfpathlineto{\pgfqpoint{2.659143in}{2.670175in}}%
\pgfpathlineto{\pgfqpoint{2.659297in}{2.385319in}}%
\pgfpathlineto{\pgfqpoint{2.659826in}{2.551512in}}%
\pgfpathlineto{\pgfqpoint{2.660069in}{2.659467in}}%
\pgfpathlineto{\pgfqpoint{2.660928in}{2.342924in}}%
\pgfpathlineto{\pgfqpoint{2.661986in}{2.762505in}}%
\pgfpathlineto{\pgfqpoint{2.661237in}{2.310799in}}%
\pgfpathlineto{\pgfqpoint{2.662052in}{2.656735in}}%
\pgfpathlineto{\pgfqpoint{2.662581in}{2.460712in}}%
\pgfpathlineto{\pgfqpoint{2.662670in}{2.749830in}}%
\pgfpathlineto{\pgfqpoint{2.663155in}{2.581560in}}%
\pgfpathlineto{\pgfqpoint{2.663265in}{2.764909in}}%
\pgfpathlineto{\pgfqpoint{2.664169in}{2.463772in}}%
\pgfpathlineto{\pgfqpoint{2.664257in}{2.628217in}}%
\pgfpathlineto{\pgfqpoint{2.665072in}{2.451971in}}%
\pgfpathlineto{\pgfqpoint{2.664918in}{2.792444in}}%
\pgfpathlineto{\pgfqpoint{2.665403in}{2.523649in}}%
\pgfpathlineto{\pgfqpoint{2.665711in}{2.476337in}}%
\pgfpathlineto{\pgfqpoint{2.665601in}{2.782828in}}%
\pgfpathlineto{\pgfqpoint{2.666152in}{2.564733in}}%
\pgfpathlineto{\pgfqpoint{2.666637in}{2.840084in}}%
\pgfpathlineto{\pgfqpoint{2.666836in}{2.522775in}}%
\pgfpathlineto{\pgfqpoint{2.667254in}{2.656408in}}%
\pgfpathlineto{\pgfqpoint{2.667850in}{2.499611in}}%
\pgfpathlineto{\pgfqpoint{2.668092in}{2.824240in}}%
\pgfpathlineto{\pgfqpoint{2.668334in}{2.666679in}}%
\pgfpathlineto{\pgfqpoint{2.668665in}{2.782282in}}%
\pgfpathlineto{\pgfqpoint{2.668401in}{2.461259in}}%
\pgfpathlineto{\pgfqpoint{2.669238in}{2.551403in}}%
\pgfpathlineto{\pgfqpoint{2.670186in}{2.452080in}}%
\pgfpathlineto{\pgfqpoint{2.669944in}{2.798453in}}%
\pgfpathlineto{\pgfqpoint{2.670296in}{2.704047in}}%
\pgfpathlineto{\pgfqpoint{2.670803in}{2.531954in}}%
\pgfpathlineto{\pgfqpoint{2.670362in}{2.803370in}}%
\pgfpathlineto{\pgfqpoint{2.671376in}{2.746443in}}%
\pgfpathlineto{\pgfqpoint{2.671486in}{2.539165in}}%
\pgfpathlineto{\pgfqpoint{2.672324in}{2.857457in}}%
\pgfpathlineto{\pgfqpoint{2.672456in}{2.740652in}}%
\pgfpathlineto{\pgfqpoint{2.672853in}{2.905971in}}%
\pgfpathlineto{\pgfqpoint{2.673404in}{2.633352in}}%
\pgfpathlineto{\pgfqpoint{2.673536in}{2.693121in}}%
\pgfpathlineto{\pgfqpoint{2.673845in}{2.521464in}}%
\pgfpathlineto{\pgfqpoint{2.674132in}{2.813204in}}%
\pgfpathlineto{\pgfqpoint{2.674440in}{2.693449in}}%
\pgfpathlineto{\pgfqpoint{2.674661in}{2.818995in}}%
\pgfpathlineto{\pgfqpoint{2.674881in}{2.479943in}}%
\pgfpathlineto{\pgfqpoint{2.675542in}{2.741963in}}%
\pgfpathlineto{\pgfqpoint{2.675719in}{2.570197in}}%
\pgfpathlineto{\pgfqpoint{2.676248in}{2.936128in}}%
\pgfpathlineto{\pgfqpoint{2.676688in}{2.649305in}}%
\pgfpathlineto{\pgfqpoint{2.677107in}{2.861172in}}%
\pgfpathlineto{\pgfqpoint{2.677570in}{2.523103in}}%
\pgfpathlineto{\pgfqpoint{2.677790in}{2.686674in}}%
\pgfpathlineto{\pgfqpoint{2.677901in}{2.432194in}}%
\pgfpathlineto{\pgfqpoint{2.678187in}{2.840084in}}%
\pgfpathlineto{\pgfqpoint{2.678848in}{2.664821in}}%
\pgfpathlineto{\pgfqpoint{2.679686in}{2.829813in}}%
\pgfpathlineto{\pgfqpoint{2.679311in}{2.550638in}}%
\pgfpathlineto{\pgfqpoint{2.679929in}{2.676512in}}%
\pgfpathlineto{\pgfqpoint{2.680259in}{2.405424in}}%
\pgfpathlineto{\pgfqpoint{2.680810in}{2.778239in}}%
\pgfpathlineto{\pgfqpoint{2.681031in}{2.622754in}}%
\pgfpathlineto{\pgfqpoint{2.681053in}{2.622426in}}%
\pgfpathlineto{\pgfqpoint{2.681427in}{2.799000in}}%
\pgfpathlineto{\pgfqpoint{2.681207in}{2.531626in}}%
\pgfpathlineto{\pgfqpoint{2.682177in}{2.723060in}}%
\pgfpathlineto{\pgfqpoint{2.682596in}{2.473715in}}%
\pgfpathlineto{\pgfqpoint{2.682794in}{2.768951in}}%
\pgfpathlineto{\pgfqpoint{2.683301in}{2.575441in}}%
\pgfpathlineto{\pgfqpoint{2.683433in}{2.805228in}}%
\pgfpathlineto{\pgfqpoint{2.683698in}{2.454266in}}%
\pgfpathlineto{\pgfqpoint{2.684425in}{2.727540in}}%
\pgfpathlineto{\pgfqpoint{2.684601in}{2.571289in}}%
\pgfpathlineto{\pgfqpoint{2.685483in}{2.841613in}}%
\pgfpathlineto{\pgfqpoint{2.685527in}{2.744913in}}%
\pgfpathlineto{\pgfqpoint{2.686034in}{2.546923in}}%
\pgfpathlineto{\pgfqpoint{2.685946in}{2.799000in}}%
\pgfpathlineto{\pgfqpoint{2.686629in}{2.684270in}}%
\pgfpathlineto{\pgfqpoint{2.686651in}{2.856255in}}%
\pgfpathlineto{\pgfqpoint{2.687687in}{2.490214in}}%
\pgfpathlineto{\pgfqpoint{2.687709in}{2.539274in}}%
\pgfpathlineto{\pgfqpoint{2.688944in}{2.784904in}}%
\pgfpathlineto{\pgfqpoint{2.688481in}{2.533156in}}%
\pgfpathlineto{\pgfqpoint{2.689054in}{2.695962in}}%
\pgfpathlineto{\pgfqpoint{2.689825in}{2.557303in}}%
\pgfpathlineto{\pgfqpoint{2.689230in}{2.766657in}}%
\pgfpathlineto{\pgfqpoint{2.690112in}{2.760647in}}%
\pgfpathlineto{\pgfqpoint{2.690310in}{2.854616in}}%
\pgfpathlineto{\pgfqpoint{2.690575in}{2.568339in}}%
\pgfpathlineto{\pgfqpoint{2.691082in}{2.640564in}}%
\pgfpathlineto{\pgfqpoint{2.691390in}{2.464318in}}%
\pgfpathlineto{\pgfqpoint{2.691192in}{2.823475in}}%
\pgfpathlineto{\pgfqpoint{2.692184in}{2.676185in}}%
\pgfpathlineto{\pgfqpoint{2.692845in}{2.791351in}}%
\pgfpathlineto{\pgfqpoint{2.693264in}{2.530533in}}%
\pgfpathlineto{\pgfqpoint{2.693286in}{2.647776in}}%
\pgfpathlineto{\pgfqpoint{2.694057in}{2.508680in}}%
\pgfpathlineto{\pgfqpoint{2.694212in}{2.841504in}}%
\pgfpathlineto{\pgfqpoint{2.694366in}{2.602212in}}%
\pgfpathlineto{\pgfqpoint{2.695093in}{2.878217in}}%
\pgfpathlineto{\pgfqpoint{2.694520in}{2.597295in}}%
\pgfpathlineto{\pgfqpoint{2.695446in}{2.713772in}}%
\pgfpathlineto{\pgfqpoint{2.695468in}{2.508461in}}%
\pgfpathlineto{\pgfqpoint{2.695556in}{2.820634in}}%
\pgfpathlineto{\pgfqpoint{2.696526in}{2.727758in}}%
\pgfpathlineto{\pgfqpoint{2.697364in}{2.845765in}}%
\pgfpathlineto{\pgfqpoint{2.697408in}{2.568448in}}%
\pgfpathlineto{\pgfqpoint{2.697584in}{2.579921in}}%
\pgfpathlineto{\pgfqpoint{2.697849in}{2.400725in}}%
\pgfpathlineto{\pgfqpoint{2.698267in}{2.758680in}}%
\pgfpathlineto{\pgfqpoint{2.698488in}{2.560909in}}%
\pgfpathlineto{\pgfqpoint{2.699127in}{2.798672in}}%
\pgfpathlineto{\pgfqpoint{2.699524in}{2.477867in}}%
\pgfpathlineto{\pgfqpoint{2.699590in}{2.524524in}}%
\pgfpathlineto{\pgfqpoint{2.700516in}{2.781189in}}%
\pgfpathlineto{\pgfqpoint{2.700428in}{2.465192in}}%
\pgfpathlineto{\pgfqpoint{2.700890in}{2.736062in}}%
\pgfpathlineto{\pgfqpoint{2.701552in}{2.480926in}}%
\pgfpathlineto{\pgfqpoint{2.701045in}{2.740324in}}%
\pgfpathlineto{\pgfqpoint{2.702015in}{2.629419in}}%
\pgfpathlineto{\pgfqpoint{2.702191in}{2.753326in}}%
\pgfpathlineto{\pgfqpoint{2.702301in}{2.431648in}}%
\pgfpathlineto{\pgfqpoint{2.703139in}{2.698475in}}%
\pgfpathlineto{\pgfqpoint{2.704131in}{2.524851in}}%
\pgfpathlineto{\pgfqpoint{2.703403in}{2.868711in}}%
\pgfpathlineto{\pgfqpoint{2.704219in}{2.627561in}}%
\pgfpathlineto{\pgfqpoint{2.704616in}{2.905206in}}%
\pgfpathlineto{\pgfqpoint{2.705034in}{2.575441in}}%
\pgfpathlineto{\pgfqpoint{2.705343in}{2.852540in}}%
\pgfpathlineto{\pgfqpoint{2.706269in}{2.567793in}}%
\pgfpathlineto{\pgfqpoint{2.706467in}{2.650179in}}%
\pgfpathlineto{\pgfqpoint{2.706599in}{2.832326in}}%
\pgfpathlineto{\pgfqpoint{2.707547in}{2.550638in}}%
\pgfpathlineto{\pgfqpoint{2.707790in}{2.756604in}}%
\pgfpathlineto{\pgfqpoint{2.708385in}{2.545393in}}%
\pgfpathlineto{\pgfqpoint{2.708671in}{2.688095in}}%
\pgfpathlineto{\pgfqpoint{2.708759in}{2.763707in}}%
\pgfpathlineto{\pgfqpoint{2.709002in}{2.534467in}}%
\pgfpathlineto{\pgfqpoint{2.709773in}{2.697054in}}%
\pgfpathlineto{\pgfqpoint{2.710677in}{2.532937in}}%
\pgfpathlineto{\pgfqpoint{2.710280in}{2.772120in}}%
\pgfpathlineto{\pgfqpoint{2.710920in}{2.612483in}}%
\pgfpathlineto{\pgfqpoint{2.711118in}{2.826316in}}%
\pgfpathlineto{\pgfqpoint{2.711360in}{2.565935in}}%
\pgfpathlineto{\pgfqpoint{2.712066in}{2.718471in}}%
\pgfpathlineto{\pgfqpoint{2.712617in}{2.513706in}}%
\pgfpathlineto{\pgfqpoint{2.712859in}{2.789603in}}%
\pgfpathlineto{\pgfqpoint{2.713212in}{2.635428in}}%
\pgfpathlineto{\pgfqpoint{2.713454in}{2.875158in}}%
\pgfpathlineto{\pgfqpoint{2.713675in}{2.500267in}}%
\pgfpathlineto{\pgfqpoint{2.714314in}{2.683943in}}%
\pgfpathlineto{\pgfqpoint{2.715284in}{2.423671in}}%
\pgfpathlineto{\pgfqpoint{2.714468in}{2.778458in}}%
\pgfpathlineto{\pgfqpoint{2.715570in}{2.558396in}}%
\pgfpathlineto{\pgfqpoint{2.716408in}{2.397338in}}%
\pgfpathlineto{\pgfqpoint{2.715659in}{2.661325in}}%
\pgfpathlineto{\pgfqpoint{2.716496in}{2.538400in}}%
\pgfpathlineto{\pgfqpoint{2.717047in}{2.692684in}}%
\pgfpathlineto{\pgfqpoint{2.717532in}{2.424655in}}%
\pgfpathlineto{\pgfqpoint{2.717598in}{2.530752in}}%
\pgfpathlineto{\pgfqpoint{2.717885in}{2.647229in}}%
\pgfpathlineto{\pgfqpoint{2.717642in}{2.493929in}}%
\pgfpathlineto{\pgfqpoint{2.717907in}{2.587133in}}%
\pgfpathlineto{\pgfqpoint{2.718965in}{2.799437in}}%
\pgfpathlineto{\pgfqpoint{2.718282in}{2.463772in}}%
\pgfpathlineto{\pgfqpoint{2.719031in}{2.730599in}}%
\pgfpathlineto{\pgfqpoint{2.719979in}{2.474480in}}%
\pgfpathlineto{\pgfqpoint{2.719384in}{2.757479in}}%
\pgfpathlineto{\pgfqpoint{2.720133in}{2.626469in}}%
\pgfpathlineto{\pgfqpoint{2.721059in}{2.781080in}}%
\pgfpathlineto{\pgfqpoint{2.720993in}{2.514034in}}%
\pgfpathlineto{\pgfqpoint{2.721213in}{2.633789in}}%
\pgfpathlineto{\pgfqpoint{2.721786in}{2.751469in}}%
\pgfpathlineto{\pgfqpoint{2.722315in}{2.499720in}}%
\pgfpathlineto{\pgfqpoint{2.722492in}{2.771792in}}%
\pgfpathlineto{\pgfqpoint{2.722844in}{2.483003in}}%
\pgfpathlineto{\pgfqpoint{2.723439in}{2.747535in}}%
\pgfpathlineto{\pgfqpoint{2.723616in}{2.562220in}}%
\pgfpathlineto{\pgfqpoint{2.723814in}{2.818012in}}%
\pgfpathlineto{\pgfqpoint{2.724542in}{2.748191in}}%
\pgfpathlineto{\pgfqpoint{2.724564in}{2.818449in}}%
\pgfpathlineto{\pgfqpoint{2.725555in}{2.526818in}}%
\pgfpathlineto{\pgfqpoint{2.725622in}{2.657609in}}%
\pgfpathlineto{\pgfqpoint{2.725952in}{2.896137in}}%
\pgfpathlineto{\pgfqpoint{2.726129in}{2.541132in}}%
\pgfpathlineto{\pgfqpoint{2.726636in}{2.627780in}}%
\pgfpathlineto{\pgfqpoint{2.726680in}{2.559379in}}%
\pgfpathlineto{\pgfqpoint{2.726900in}{2.798672in}}%
\pgfpathlineto{\pgfqpoint{2.727694in}{2.685800in}}%
\pgfpathlineto{\pgfqpoint{2.728112in}{2.848934in}}%
\pgfpathlineto{\pgfqpoint{2.728421in}{2.619366in}}%
\pgfpathlineto{\pgfqpoint{2.728730in}{2.724808in}}%
\pgfpathlineto{\pgfqpoint{2.728774in}{2.561565in}}%
\pgfpathlineto{\pgfqpoint{2.729236in}{2.814734in}}%
\pgfpathlineto{\pgfqpoint{2.729854in}{2.670940in}}%
\pgfpathlineto{\pgfqpoint{2.730096in}{2.897776in}}%
\pgfpathlineto{\pgfqpoint{2.730471in}{2.591831in}}%
\pgfpathlineto{\pgfqpoint{2.730956in}{2.706014in}}%
\pgfpathlineto{\pgfqpoint{2.731837in}{2.554025in}}%
\pgfpathlineto{\pgfqpoint{2.731463in}{2.871880in}}%
\pgfpathlineto{\pgfqpoint{2.732036in}{2.759227in}}%
\pgfpathlineto{\pgfqpoint{2.732212in}{2.593907in}}%
\pgfpathlineto{\pgfqpoint{2.732411in}{2.870350in}}%
\pgfpathlineto{\pgfqpoint{2.733160in}{2.698256in}}%
\pgfpathlineto{\pgfqpoint{2.733402in}{2.842815in}}%
\pgfpathlineto{\pgfqpoint{2.733711in}{2.551840in}}%
\pgfpathlineto{\pgfqpoint{2.734240in}{2.643842in}}%
\pgfpathlineto{\pgfqpoint{2.735122in}{2.567684in}}%
\pgfpathlineto{\pgfqpoint{2.734527in}{2.900617in}}%
\pgfpathlineto{\pgfqpoint{2.735254in}{2.687985in}}%
\pgfpathlineto{\pgfqpoint{2.735474in}{2.794083in}}%
\pgfpathlineto{\pgfqpoint{2.736158in}{2.469344in}}%
\pgfpathlineto{\pgfqpoint{2.736224in}{2.601338in}}%
\pgfpathlineto{\pgfqpoint{2.736312in}{2.424436in}}%
\pgfpathlineto{\pgfqpoint{2.736731in}{2.815827in}}%
\pgfpathlineto{\pgfqpoint{2.737326in}{2.590629in}}%
\pgfpathlineto{\pgfqpoint{2.737723in}{2.831014in}}%
\pgfpathlineto{\pgfqpoint{2.738296in}{2.504856in}}%
\pgfpathlineto{\pgfqpoint{2.738494in}{2.726993in}}%
\pgfpathlineto{\pgfqpoint{2.738979in}{2.517093in}}%
\pgfpathlineto{\pgfqpoint{2.738648in}{2.757479in}}%
\pgfpathlineto{\pgfqpoint{2.739618in}{2.654441in}}%
\pgfpathlineto{\pgfqpoint{2.739993in}{2.462461in}}%
\pgfpathlineto{\pgfqpoint{2.740500in}{2.754747in}}%
\pgfpathlineto{\pgfqpoint{2.740742in}{2.585712in}}%
\pgfpathlineto{\pgfqpoint{2.741514in}{2.783156in}}%
\pgfpathlineto{\pgfqpoint{2.740853in}{2.503435in}}%
\pgfpathlineto{\pgfqpoint{2.741867in}{2.605380in}}%
\pgfpathlineto{\pgfqpoint{2.742153in}{2.718908in}}%
\pgfpathlineto{\pgfqpoint{2.742285in}{2.440717in}}%
\pgfpathlineto{\pgfqpoint{2.742418in}{2.542006in}}%
\pgfpathlineto{\pgfqpoint{2.742748in}{2.408811in}}%
\pgfpathlineto{\pgfqpoint{2.742638in}{2.605817in}}%
\pgfpathlineto{\pgfqpoint{2.743520in}{2.440170in}}%
\pgfpathlineto{\pgfqpoint{2.743938in}{2.726884in}}%
\pgfpathlineto{\pgfqpoint{2.744644in}{2.578610in}}%
\pgfpathlineto{\pgfqpoint{2.744666in}{2.495240in}}%
\pgfpathlineto{\pgfqpoint{2.745658in}{2.861937in}}%
\pgfpathlineto{\pgfqpoint{2.745724in}{2.550310in}}%
\pgfpathlineto{\pgfqpoint{2.746341in}{2.791351in}}%
\pgfpathlineto{\pgfqpoint{2.746121in}{2.444541in}}%
\pgfpathlineto{\pgfqpoint{2.746848in}{2.690171in}}%
\pgfpathlineto{\pgfqpoint{2.747223in}{2.557631in}}%
\pgfpathlineto{\pgfqpoint{2.747620in}{2.840084in}}%
\pgfpathlineto{\pgfqpoint{2.747972in}{2.640564in}}%
\pgfpathlineto{\pgfqpoint{2.748545in}{2.863904in}}%
\pgfpathlineto{\pgfqpoint{2.748854in}{2.561893in}}%
\pgfpathlineto{\pgfqpoint{2.748986in}{2.736281in}}%
\pgfpathlineto{\pgfqpoint{2.749846in}{2.547360in}}%
\pgfpathlineto{\pgfqpoint{2.749339in}{2.858768in}}%
\pgfpathlineto{\pgfqpoint{2.750088in}{2.680446in}}%
\pgfpathlineto{\pgfqpoint{2.750838in}{2.892640in}}%
\pgfpathlineto{\pgfqpoint{2.750419in}{2.495350in}}%
\pgfpathlineto{\pgfqpoint{2.751212in}{2.752343in}}%
\pgfpathlineto{\pgfqpoint{2.751940in}{2.838663in}}%
\pgfpathlineto{\pgfqpoint{2.752336in}{2.517858in}}%
\pgfpathlineto{\pgfqpoint{2.753064in}{2.768296in}}%
\pgfpathlineto{\pgfqpoint{2.753461in}{2.637832in}}%
\pgfpathlineto{\pgfqpoint{2.754298in}{2.486718in}}%
\pgfpathlineto{\pgfqpoint{2.753637in}{2.750267in}}%
\pgfpathlineto{\pgfqpoint{2.754585in}{2.582325in}}%
\pgfpathlineto{\pgfqpoint{2.755312in}{2.781845in}}%
\pgfpathlineto{\pgfqpoint{2.754651in}{2.458636in}}%
\pgfpathlineto{\pgfqpoint{2.755753in}{2.691482in}}%
\pgfpathlineto{\pgfqpoint{2.755907in}{2.495240in}}%
\pgfpathlineto{\pgfqpoint{2.756194in}{2.789603in}}%
\pgfpathlineto{\pgfqpoint{2.756833in}{2.680774in}}%
\pgfpathlineto{\pgfqpoint{2.757296in}{2.817793in}}%
\pgfpathlineto{\pgfqpoint{2.757825in}{2.522448in}}%
\pgfpathlineto{\pgfqpoint{2.757891in}{2.648868in}}%
\pgfpathlineto{\pgfqpoint{2.758156in}{2.452627in}}%
\pgfpathlineto{\pgfqpoint{2.758751in}{2.741963in}}%
\pgfpathlineto{\pgfqpoint{2.758993in}{2.617727in}}%
\pgfpathlineto{\pgfqpoint{2.759280in}{2.590083in}}%
\pgfpathlineto{\pgfqpoint{2.759676in}{2.799983in}}%
\pgfpathlineto{\pgfqpoint{2.759765in}{2.765892in}}%
\pgfpathlineto{\pgfqpoint{2.759787in}{2.825879in}}%
\pgfpathlineto{\pgfqpoint{2.760712in}{2.475026in}}%
\pgfpathlineto{\pgfqpoint{2.760845in}{2.713444in}}%
\pgfpathlineto{\pgfqpoint{2.761131in}{2.423343in}}%
\pgfpathlineto{\pgfqpoint{2.761660in}{2.773650in}}%
\pgfpathlineto{\pgfqpoint{2.761991in}{2.636740in}}%
\pgfpathlineto{\pgfqpoint{2.762718in}{2.805118in}}%
\pgfpathlineto{\pgfqpoint{2.762189in}{2.542880in}}%
\pgfpathlineto{\pgfqpoint{2.763093in}{2.607019in}}%
\pgfpathlineto{\pgfqpoint{2.763666in}{2.506167in}}%
\pgfpathlineto{\pgfqpoint{2.763798in}{2.863029in}}%
\pgfpathlineto{\pgfqpoint{2.764129in}{2.750704in}}%
\pgfpathlineto{\pgfqpoint{2.764702in}{2.478523in}}%
\pgfpathlineto{\pgfqpoint{2.765319in}{2.658374in}}%
\pgfpathlineto{\pgfqpoint{2.766311in}{2.830905in}}%
\pgfpathlineto{\pgfqpoint{2.765363in}{2.580249in}}%
\pgfpathlineto{\pgfqpoint{2.766421in}{2.757697in}}%
\pgfpathlineto{\pgfqpoint{2.767193in}{2.543536in}}%
\pgfpathlineto{\pgfqpoint{2.766686in}{2.820962in}}%
\pgfpathlineto{\pgfqpoint{2.767523in}{2.696290in}}%
\pgfpathlineto{\pgfqpoint{2.768097in}{2.904878in}}%
\pgfpathlineto{\pgfqpoint{2.768251in}{2.519279in}}%
\pgfpathlineto{\pgfqpoint{2.768626in}{2.675638in}}%
\pgfpathlineto{\pgfqpoint{2.769088in}{2.437002in}}%
\pgfpathlineto{\pgfqpoint{2.768824in}{2.783484in}}%
\pgfpathlineto{\pgfqpoint{2.769772in}{2.555009in}}%
\pgfpathlineto{\pgfqpoint{2.770257in}{2.724480in}}%
\pgfpathlineto{\pgfqpoint{2.770764in}{2.362591in}}%
\pgfpathlineto{\pgfqpoint{2.770874in}{2.537308in}}%
\pgfpathlineto{\pgfqpoint{2.771160in}{2.747754in}}%
\pgfpathlineto{\pgfqpoint{2.771381in}{2.507697in}}%
\pgfpathlineto{\pgfqpoint{2.771866in}{2.540476in}}%
\pgfpathlineto{\pgfqpoint{2.771954in}{2.429571in}}%
\pgfpathlineto{\pgfqpoint{2.772351in}{2.750595in}}%
\pgfpathlineto{\pgfqpoint{2.772946in}{2.606692in}}%
\pgfpathlineto{\pgfqpoint{2.773497in}{2.837680in}}%
\pgfpathlineto{\pgfqpoint{2.773541in}{2.504528in}}%
\pgfpathlineto{\pgfqpoint{2.774092in}{2.645372in}}%
\pgfpathlineto{\pgfqpoint{2.775194in}{2.564624in}}%
\pgfpathlineto{\pgfqpoint{2.774863in}{2.796705in}}%
\pgfpathlineto{\pgfqpoint{2.775216in}{2.610079in}}%
\pgfpathlineto{\pgfqpoint{2.775899in}{2.834839in}}%
\pgfpathlineto{\pgfqpoint{2.775348in}{2.510865in}}%
\pgfpathlineto{\pgfqpoint{2.776384in}{2.692247in}}%
\pgfpathlineto{\pgfqpoint{2.776935in}{2.449786in}}%
\pgfpathlineto{\pgfqpoint{2.777156in}{2.751250in}}%
\pgfpathlineto{\pgfqpoint{2.777508in}{2.557085in}}%
\pgfpathlineto{\pgfqpoint{2.777751in}{2.806867in}}%
\pgfpathlineto{\pgfqpoint{2.777619in}{2.530752in}}%
\pgfpathlineto{\pgfqpoint{2.778500in}{2.666788in}}%
\pgfpathlineto{\pgfqpoint{2.778522in}{2.483658in}}%
\pgfpathlineto{\pgfqpoint{2.778831in}{2.781080in}}%
\pgfpathlineto{\pgfqpoint{2.779602in}{2.596858in}}%
\pgfpathlineto{\pgfqpoint{2.780242in}{2.756932in}}%
\pgfpathlineto{\pgfqpoint{2.780660in}{2.511849in}}%
\pgfpathlineto{\pgfqpoint{2.780727in}{2.826862in}}%
\pgfpathlineto{\pgfqpoint{2.781256in}{2.409794in}}%
\pgfpathlineto{\pgfqpoint{2.781785in}{2.693121in}}%
\pgfpathlineto{\pgfqpoint{2.782512in}{2.489996in}}%
\pgfpathlineto{\pgfqpoint{2.782710in}{2.754856in}}%
\pgfpathlineto{\pgfqpoint{2.782887in}{2.659904in}}%
\pgfpathlineto{\pgfqpoint{2.782975in}{2.698256in}}%
\pgfpathlineto{\pgfqpoint{2.782997in}{2.628217in}}%
\pgfpathlineto{\pgfqpoint{2.783173in}{2.531735in}}%
\pgfpathlineto{\pgfqpoint{2.783504in}{2.836696in}}%
\pgfpathlineto{\pgfqpoint{2.784077in}{2.670394in}}%
\pgfpathlineto{\pgfqpoint{2.784518in}{2.903567in}}%
\pgfpathlineto{\pgfqpoint{2.784760in}{2.489558in}}%
\pgfpathlineto{\pgfqpoint{2.785113in}{2.757260in}}%
\pgfpathlineto{\pgfqpoint{2.785598in}{2.554572in}}%
\pgfpathlineto{\pgfqpoint{2.785510in}{2.873847in}}%
\pgfpathlineto{\pgfqpoint{2.786237in}{2.630074in}}%
\pgfpathlineto{\pgfqpoint{2.786700in}{2.830359in}}%
\pgfpathlineto{\pgfqpoint{2.786435in}{2.577080in}}%
\pgfpathlineto{\pgfqpoint{2.787339in}{2.757479in}}%
\pgfpathlineto{\pgfqpoint{2.787890in}{2.550857in}}%
\pgfpathlineto{\pgfqpoint{2.788067in}{2.883134in}}%
\pgfpathlineto{\pgfqpoint{2.788441in}{2.774305in}}%
\pgfpathlineto{\pgfqpoint{2.789103in}{2.553370in}}%
\pgfpathlineto{\pgfqpoint{2.788750in}{2.835167in}}%
\pgfpathlineto{\pgfqpoint{2.789654in}{2.630074in}}%
\pgfpathlineto{\pgfqpoint{2.790271in}{2.834074in}}%
\pgfpathlineto{\pgfqpoint{2.789698in}{2.566700in}}%
\pgfpathlineto{\pgfqpoint{2.790778in}{2.765018in}}%
\pgfpathlineto{\pgfqpoint{2.791174in}{2.502780in}}%
\pgfpathlineto{\pgfqpoint{2.791373in}{2.914494in}}%
\pgfpathlineto{\pgfqpoint{2.791902in}{2.652146in}}%
\pgfpathlineto{\pgfqpoint{2.791968in}{2.763270in}}%
\pgfpathlineto{\pgfqpoint{2.792012in}{2.542443in}}%
\pgfpathlineto{\pgfqpoint{2.792673in}{2.586477in}}%
\pgfpathlineto{\pgfqpoint{2.792695in}{2.515017in}}%
\pgfpathlineto{\pgfqpoint{2.793158in}{2.804463in}}%
\pgfpathlineto{\pgfqpoint{2.793687in}{2.802168in}}%
\pgfpathlineto{\pgfqpoint{2.793709in}{2.835822in}}%
\pgfpathlineto{\pgfqpoint{2.794481in}{2.517312in}}%
\pgfpathlineto{\pgfqpoint{2.794591in}{2.656189in}}%
\pgfpathlineto{\pgfqpoint{2.794613in}{2.514253in}}%
\pgfpathlineto{\pgfqpoint{2.795517in}{2.869804in}}%
\pgfpathlineto{\pgfqpoint{2.795671in}{2.763379in}}%
\pgfpathlineto{\pgfqpoint{2.796531in}{2.553370in}}%
\pgfpathlineto{\pgfqpoint{2.796002in}{2.824459in}}%
\pgfpathlineto{\pgfqpoint{2.796751in}{2.734096in}}%
\pgfpathlineto{\pgfqpoint{2.796773in}{2.782610in}}%
\pgfpathlineto{\pgfqpoint{2.797170in}{2.496114in}}%
\pgfpathlineto{\pgfqpoint{2.797809in}{2.645699in}}%
\pgfpathlineto{\pgfqpoint{2.798052in}{2.700770in}}%
\pgfpathlineto{\pgfqpoint{2.798338in}{2.538619in}}%
\pgfpathlineto{\pgfqpoint{2.798669in}{2.651709in}}%
\pgfpathlineto{\pgfqpoint{2.798999in}{2.465192in}}%
\pgfpathlineto{\pgfqpoint{2.799727in}{2.776163in}}%
\pgfpathlineto{\pgfqpoint{2.799749in}{2.623409in}}%
\pgfpathlineto{\pgfqpoint{2.800013in}{2.762942in}}%
\pgfpathlineto{\pgfqpoint{2.800057in}{2.521464in}}%
\pgfpathlineto{\pgfqpoint{2.800851in}{2.597841in}}%
\pgfpathlineto{\pgfqpoint{2.801424in}{2.832107in}}%
\pgfpathlineto{\pgfqpoint{2.801512in}{2.532391in}}%
\pgfpathlineto{\pgfqpoint{2.801975in}{2.671705in}}%
\pgfpathlineto{\pgfqpoint{2.802019in}{2.560035in}}%
\pgfpathlineto{\pgfqpoint{2.802504in}{2.851229in}}%
\pgfpathlineto{\pgfqpoint{2.802526in}{2.894935in}}%
\pgfpathlineto{\pgfqpoint{2.802702in}{2.540476in}}%
\pgfpathlineto{\pgfqpoint{2.803540in}{2.814843in}}%
\pgfpathlineto{\pgfqpoint{2.804135in}{2.595546in}}%
\pgfpathlineto{\pgfqpoint{2.804245in}{2.846530in}}%
\pgfpathlineto{\pgfqpoint{2.804664in}{2.602539in}}%
\pgfpathlineto{\pgfqpoint{2.805744in}{2.567684in}}%
\pgfpathlineto{\pgfqpoint{2.805810in}{2.839537in}}%
\pgfpathlineto{\pgfqpoint{2.806736in}{2.555555in}}%
\pgfpathlineto{\pgfqpoint{2.806802in}{2.868711in}}%
\pgfpathlineto{\pgfqpoint{2.807001in}{2.573147in}}%
\pgfpathlineto{\pgfqpoint{2.807662in}{2.848934in}}%
\pgfpathlineto{\pgfqpoint{2.807067in}{2.566919in}}%
\pgfpathlineto{\pgfqpoint{2.808125in}{2.665477in}}%
\pgfpathlineto{\pgfqpoint{2.809117in}{2.838007in}}%
\pgfpathlineto{\pgfqpoint{2.808918in}{2.563313in}}%
\pgfpathlineto{\pgfqpoint{2.809227in}{2.728851in}}%
\pgfpathlineto{\pgfqpoint{2.809403in}{2.600682in}}%
\pgfpathlineto{\pgfqpoint{2.809844in}{2.904223in}}%
\pgfpathlineto{\pgfqpoint{2.810329in}{2.689515in}}%
\pgfpathlineto{\pgfqpoint{2.810373in}{2.864778in}}%
\pgfpathlineto{\pgfqpoint{2.810571in}{2.625376in}}%
\pgfpathlineto{\pgfqpoint{2.811431in}{2.683396in}}%
\pgfpathlineto{\pgfqpoint{2.811585in}{2.567137in}}%
\pgfpathlineto{\pgfqpoint{2.811894in}{2.904878in}}%
\pgfpathlineto{\pgfqpoint{2.812423in}{2.764253in}}%
\pgfpathlineto{\pgfqpoint{2.812577in}{2.846530in}}%
\pgfpathlineto{\pgfqpoint{2.812974in}{2.569323in}}%
\pgfpathlineto{\pgfqpoint{2.813503in}{2.738357in}}%
\pgfpathlineto{\pgfqpoint{2.814010in}{2.556429in}}%
\pgfpathlineto{\pgfqpoint{2.813569in}{2.926950in}}%
\pgfpathlineto{\pgfqpoint{2.814495in}{2.749939in}}%
\pgfpathlineto{\pgfqpoint{2.814517in}{2.924655in}}%
\pgfpathlineto{\pgfqpoint{2.815156in}{2.591176in}}%
\pgfpathlineto{\pgfqpoint{2.815597in}{2.777256in}}%
\pgfpathlineto{\pgfqpoint{2.816214in}{2.863357in}}%
\pgfpathlineto{\pgfqpoint{2.816479in}{2.585166in}}%
\pgfpathlineto{\pgfqpoint{2.816655in}{2.813641in}}%
\pgfpathlineto{\pgfqpoint{2.817426in}{2.575223in}}%
\pgfpathlineto{\pgfqpoint{2.816964in}{2.865980in}}%
\pgfpathlineto{\pgfqpoint{2.817779in}{2.727321in}}%
\pgfpathlineto{\pgfqpoint{2.818088in}{2.890783in}}%
\pgfpathlineto{\pgfqpoint{2.818374in}{2.571617in}}%
\pgfpathlineto{\pgfqpoint{2.818881in}{2.757916in}}%
\pgfpathlineto{\pgfqpoint{2.819741in}{2.536980in}}%
\pgfpathlineto{\pgfqpoint{2.819102in}{2.900508in}}%
\pgfpathlineto{\pgfqpoint{2.819961in}{2.695525in}}%
\pgfpathlineto{\pgfqpoint{2.820182in}{2.893624in}}%
\pgfpathlineto{\pgfqpoint{2.820799in}{2.585822in}}%
\pgfpathlineto{\pgfqpoint{2.821063in}{2.769607in}}%
\pgfpathlineto{\pgfqpoint{2.822143in}{2.579921in}}%
\pgfpathlineto{\pgfqpoint{2.821526in}{2.856473in}}%
\pgfpathlineto{\pgfqpoint{2.822166in}{2.693886in}}%
\pgfpathlineto{\pgfqpoint{2.823091in}{2.903786in}}%
\pgfpathlineto{\pgfqpoint{2.822959in}{2.608658in}}%
\pgfpathlineto{\pgfqpoint{2.823268in}{2.744585in}}%
\pgfpathlineto{\pgfqpoint{2.824171in}{2.551075in}}%
\pgfpathlineto{\pgfqpoint{2.823510in}{2.894389in}}%
\pgfpathlineto{\pgfqpoint{2.824392in}{2.623518in}}%
\pgfpathlineto{\pgfqpoint{2.824524in}{2.610297in}}%
\pgfpathlineto{\pgfqpoint{2.824678in}{2.818995in}}%
\pgfpathlineto{\pgfqpoint{2.825494in}{2.912308in}}%
\pgfpathlineto{\pgfqpoint{2.825295in}{2.519170in}}%
\pgfpathlineto{\pgfqpoint{2.825692in}{2.664493in}}%
\pgfpathlineto{\pgfqpoint{2.826376in}{2.907173in}}%
\pgfpathlineto{\pgfqpoint{2.825979in}{2.623081in}}%
\pgfpathlineto{\pgfqpoint{2.826750in}{2.682085in}}%
\pgfpathlineto{\pgfqpoint{2.827500in}{2.591613in}}%
\pgfpathlineto{\pgfqpoint{2.827698in}{2.864996in}}%
\pgfpathlineto{\pgfqpoint{2.827852in}{2.690826in}}%
\pgfpathlineto{\pgfqpoint{2.828029in}{2.594017in}}%
\pgfpathlineto{\pgfqpoint{2.828315in}{2.881714in}}%
\pgfpathlineto{\pgfqpoint{2.828756in}{2.845984in}}%
\pgfpathlineto{\pgfqpoint{2.828822in}{2.963882in}}%
\pgfpathlineto{\pgfqpoint{2.828976in}{2.639471in}}%
\pgfpathlineto{\pgfqpoint{2.829814in}{2.771683in}}%
\pgfpathlineto{\pgfqpoint{2.830541in}{2.534904in}}%
\pgfpathlineto{\pgfqpoint{2.830365in}{2.869804in}}%
\pgfpathlineto{\pgfqpoint{2.830894in}{2.818886in}}%
\pgfpathlineto{\pgfqpoint{2.830960in}{2.885757in}}%
\pgfpathlineto{\pgfqpoint{2.831004in}{2.797798in}}%
\pgfpathlineto{\pgfqpoint{2.831599in}{2.512395in}}%
\pgfpathlineto{\pgfqpoint{2.832084in}{2.883899in}}%
\pgfpathlineto{\pgfqpoint{2.832128in}{2.728086in}}%
\pgfpathlineto{\pgfqpoint{2.832966in}{2.869585in}}%
\pgfpathlineto{\pgfqpoint{2.833120in}{2.629637in}}%
\pgfpathlineto{\pgfqpoint{2.833253in}{2.531080in}}%
\pgfpathlineto{\pgfqpoint{2.834112in}{2.819869in}}%
\pgfpathlineto{\pgfqpoint{2.834134in}{2.910779in}}%
\pgfpathlineto{\pgfqpoint{2.834443in}{2.579921in}}%
\pgfpathlineto{\pgfqpoint{2.835192in}{2.728414in}}%
\pgfpathlineto{\pgfqpoint{2.835413in}{2.535122in}}%
\pgfpathlineto{\pgfqpoint{2.836272in}{2.813532in}}%
\pgfpathlineto{\pgfqpoint{2.836294in}{2.785997in}}%
\pgfpathlineto{\pgfqpoint{2.837242in}{2.595874in}}%
\pgfpathlineto{\pgfqpoint{2.836868in}{2.842597in}}%
\pgfpathlineto{\pgfqpoint{2.837419in}{2.683068in}}%
\pgfpathlineto{\pgfqpoint{2.837485in}{2.858549in}}%
\pgfpathlineto{\pgfqpoint{2.837617in}{2.614996in}}%
\pgfpathlineto{\pgfqpoint{2.837926in}{2.652365in}}%
\pgfpathlineto{\pgfqpoint{2.837948in}{2.507806in}}%
\pgfpathlineto{\pgfqpoint{2.838234in}{2.829813in}}%
\pgfpathlineto{\pgfqpoint{2.839050in}{2.560144in}}%
\pgfpathlineto{\pgfqpoint{2.839182in}{2.500813in}}%
\pgfpathlineto{\pgfqpoint{2.839557in}{2.731473in}}%
\pgfpathlineto{\pgfqpoint{2.839645in}{2.715848in}}%
\pgfpathlineto{\pgfqpoint{2.840108in}{2.866198in}}%
\pgfpathlineto{\pgfqpoint{2.840284in}{2.614668in}}%
\pgfpathlineto{\pgfqpoint{2.840769in}{2.767422in}}%
\pgfpathlineto{\pgfqpoint{2.841386in}{2.822929in}}%
\pgfpathlineto{\pgfqpoint{2.840857in}{2.557850in}}%
\pgfpathlineto{\pgfqpoint{2.841827in}{2.818449in}}%
\pgfpathlineto{\pgfqpoint{2.842510in}{2.567465in}}%
\pgfpathlineto{\pgfqpoint{2.842180in}{2.836587in}}%
\pgfpathlineto{\pgfqpoint{2.842951in}{2.713444in}}%
\pgfpathlineto{\pgfqpoint{2.843127in}{2.863685in}}%
\pgfpathlineto{\pgfqpoint{2.843326in}{2.568339in}}%
\pgfpathlineto{\pgfqpoint{2.843921in}{2.640564in}}%
\pgfpathlineto{\pgfqpoint{2.844869in}{2.495131in}}%
\pgfpathlineto{\pgfqpoint{2.844384in}{2.752452in}}%
\pgfpathlineto{\pgfqpoint{2.845023in}{2.611827in}}%
\pgfpathlineto{\pgfqpoint{2.845772in}{2.727212in}}%
\pgfpathlineto{\pgfqpoint{2.846037in}{2.457871in}}%
\pgfpathlineto{\pgfqpoint{2.846147in}{2.659795in}}%
\pgfpathlineto{\pgfqpoint{2.846588in}{2.501359in}}%
\pgfpathlineto{\pgfqpoint{2.846301in}{2.727977in}}%
\pgfpathlineto{\pgfqpoint{2.847271in}{2.617072in}}%
\pgfpathlineto{\pgfqpoint{2.847844in}{2.799109in}}%
\pgfpathlineto{\pgfqpoint{2.848285in}{2.547032in}}%
\pgfpathlineto{\pgfqpoint{2.848418in}{2.661871in}}%
\pgfpathlineto{\pgfqpoint{2.848682in}{2.824786in}}%
\pgfpathlineto{\pgfqpoint{2.848506in}{2.514253in}}%
\pgfpathlineto{\pgfqpoint{2.849299in}{2.650616in}}%
\pgfpathlineto{\pgfqpoint{2.849343in}{2.505402in}}%
\pgfpathlineto{\pgfqpoint{2.849586in}{2.803261in}}%
\pgfpathlineto{\pgfqpoint{2.850401in}{2.653785in}}%
\pgfpathlineto{\pgfqpoint{2.850600in}{2.512614in}}%
\pgfpathlineto{\pgfqpoint{2.850556in}{2.758243in}}%
\pgfpathlineto{\pgfqpoint{2.851459in}{2.596639in}}%
\pgfpathlineto{\pgfqpoint{2.852143in}{2.852103in}}%
\pgfpathlineto{\pgfqpoint{2.851988in}{2.526490in}}%
\pgfpathlineto{\pgfqpoint{2.852561in}{2.661215in}}%
\pgfpathlineto{\pgfqpoint{2.852583in}{2.621989in}}%
\pgfpathlineto{\pgfqpoint{2.853201in}{2.878982in}}%
\pgfpathlineto{\pgfqpoint{2.853509in}{2.796705in}}%
\pgfpathlineto{\pgfqpoint{2.853531in}{2.948912in}}%
\pgfpathlineto{\pgfqpoint{2.854193in}{2.653130in}}%
\pgfpathlineto{\pgfqpoint{2.854611in}{2.761412in}}%
\pgfpathlineto{\pgfqpoint{2.855515in}{2.906626in}}%
\pgfpathlineto{\pgfqpoint{2.855295in}{2.624174in}}%
\pgfpathlineto{\pgfqpoint{2.855647in}{2.723169in}}%
\pgfpathlineto{\pgfqpoint{2.856132in}{2.923672in}}%
\pgfpathlineto{\pgfqpoint{2.856771in}{2.539274in}}%
\pgfpathlineto{\pgfqpoint{2.857852in}{2.807850in}}%
\pgfpathlineto{\pgfqpoint{2.857896in}{2.644825in}}%
\pgfpathlineto{\pgfqpoint{2.857918in}{2.642531in}}%
\pgfpathlineto{\pgfqpoint{2.857940in}{2.701534in}}%
\pgfpathlineto{\pgfqpoint{2.858755in}{2.887942in}}%
\pgfpathlineto{\pgfqpoint{2.858799in}{2.633243in}}%
\pgfpathlineto{\pgfqpoint{2.859064in}{2.842160in}}%
\pgfpathlineto{\pgfqpoint{2.859769in}{2.558942in}}%
\pgfpathlineto{\pgfqpoint{2.859879in}{2.877015in}}%
\pgfpathlineto{\pgfqpoint{2.860188in}{2.702627in}}%
\pgfpathlineto{\pgfqpoint{2.860959in}{2.595765in}}%
\pgfpathlineto{\pgfqpoint{2.860827in}{2.870897in}}%
\pgfpathlineto{\pgfqpoint{2.861268in}{2.688750in}}%
\pgfpathlineto{\pgfqpoint{2.861488in}{2.894389in}}%
\pgfpathlineto{\pgfqpoint{2.862150in}{2.571508in}}%
\pgfpathlineto{\pgfqpoint{2.862392in}{2.789275in}}%
\pgfpathlineto{\pgfqpoint{2.862921in}{2.604069in}}%
\pgfpathlineto{\pgfqpoint{2.862745in}{2.890127in}}%
\pgfpathlineto{\pgfqpoint{2.863516in}{2.723715in}}%
\pgfpathlineto{\pgfqpoint{2.864376in}{2.856910in}}%
\pgfpathlineto{\pgfqpoint{2.864178in}{2.589318in}}%
\pgfpathlineto{\pgfqpoint{2.864618in}{2.763925in}}%
\pgfpathlineto{\pgfqpoint{2.865147in}{2.877452in}}%
\pgfpathlineto{\pgfqpoint{2.865720in}{2.603851in}}%
\pgfpathlineto{\pgfqpoint{2.866161in}{2.887068in}}%
\pgfpathlineto{\pgfqpoint{2.865963in}{2.548344in}}%
\pgfpathlineto{\pgfqpoint{2.866845in}{2.835276in}}%
\pgfpathlineto{\pgfqpoint{2.867947in}{2.548344in}}%
\pgfpathlineto{\pgfqpoint{2.867241in}{2.942903in}}%
\pgfpathlineto{\pgfqpoint{2.867969in}{2.665586in}}%
\pgfpathlineto{\pgfqpoint{2.868013in}{2.566700in}}%
\pgfpathlineto{\pgfqpoint{2.868476in}{2.815499in}}%
\pgfpathlineto{\pgfqpoint{2.868784in}{2.678042in}}%
\pgfpathlineto{\pgfqpoint{2.868895in}{2.856910in}}%
\pgfpathlineto{\pgfqpoint{2.869468in}{2.523540in}}%
\pgfpathlineto{\pgfqpoint{2.869864in}{2.623846in}}%
\pgfpathlineto{\pgfqpoint{2.870437in}{2.583636in}}%
\pgfpathlineto{\pgfqpoint{2.869997in}{2.841832in}}%
\pgfpathlineto{\pgfqpoint{2.870834in}{2.647448in}}%
\pgfpathlineto{\pgfqpoint{2.871518in}{2.808396in}}%
\pgfpathlineto{\pgfqpoint{2.871760in}{2.564296in}}%
\pgfpathlineto{\pgfqpoint{2.871914in}{2.719673in}}%
\pgfpathlineto{\pgfqpoint{2.871936in}{2.589974in}}%
\pgfpathlineto{\pgfqpoint{2.872355in}{2.861063in}}%
\pgfpathlineto{\pgfqpoint{2.873016in}{2.744804in}}%
\pgfpathlineto{\pgfqpoint{2.874030in}{2.559707in}}%
\pgfpathlineto{\pgfqpoint{2.873060in}{2.802278in}}%
\pgfpathlineto{\pgfqpoint{2.874163in}{2.630074in}}%
\pgfpathlineto{\pgfqpoint{2.875000in}{2.915368in}}%
\pgfpathlineto{\pgfqpoint{2.874317in}{2.545066in}}%
\pgfpathlineto{\pgfqpoint{2.875265in}{2.604725in}}%
\pgfpathlineto{\pgfqpoint{2.875661in}{2.835931in}}%
\pgfpathlineto{\pgfqpoint{2.876014in}{2.522120in}}%
\pgfpathlineto{\pgfqpoint{2.876477in}{2.780097in}}%
\pgfpathlineto{\pgfqpoint{2.877072in}{2.646137in}}%
\pgfpathlineto{\pgfqpoint{2.877469in}{2.888598in}}%
\pgfpathlineto{\pgfqpoint{2.877601in}{2.688095in}}%
\pgfpathlineto{\pgfqpoint{2.877711in}{2.871006in}}%
\pgfpathlineto{\pgfqpoint{2.878461in}{2.556976in}}%
\pgfpathlineto{\pgfqpoint{2.878681in}{2.678261in}}%
\pgfpathlineto{\pgfqpoint{2.879673in}{2.571836in}}%
\pgfpathlineto{\pgfqpoint{2.879034in}{2.848606in}}%
\pgfpathlineto{\pgfqpoint{2.879739in}{2.602430in}}%
\pgfpathlineto{\pgfqpoint{2.880070in}{2.852758in}}%
\pgfpathlineto{\pgfqpoint{2.880136in}{2.564843in}}%
\pgfpathlineto{\pgfqpoint{2.880863in}{2.766438in}}%
\pgfpathlineto{\pgfqpoint{2.881084in}{2.468033in}}%
\pgfpathlineto{\pgfqpoint{2.881348in}{2.815936in}}%
\pgfpathlineto{\pgfqpoint{2.881965in}{2.786434in}}%
\pgfpathlineto{\pgfqpoint{2.882472in}{2.497098in}}%
\pgfpathlineto{\pgfqpoint{2.882583in}{2.849043in}}%
\pgfpathlineto{\pgfqpoint{2.883112in}{2.661325in}}%
\pgfpathlineto{\pgfqpoint{2.883685in}{2.825988in}}%
\pgfpathlineto{\pgfqpoint{2.883222in}{2.506822in}}%
\pgfpathlineto{\pgfqpoint{2.884104in}{2.509008in}}%
\pgfpathlineto{\pgfqpoint{2.884677in}{2.952955in}}%
\pgfpathlineto{\pgfqpoint{2.884258in}{2.494257in}}%
\pgfpathlineto{\pgfqpoint{2.885228in}{2.601775in}}%
\pgfpathlineto{\pgfqpoint{2.885294in}{2.473387in}}%
\pgfpathlineto{\pgfqpoint{2.885867in}{2.873300in}}%
\pgfpathlineto{\pgfqpoint{2.886308in}{2.660778in}}%
\pgfpathlineto{\pgfqpoint{2.887057in}{2.771465in}}%
\pgfpathlineto{\pgfqpoint{2.886660in}{2.525835in}}%
\pgfpathlineto{\pgfqpoint{2.887366in}{2.650179in}}%
\pgfpathlineto{\pgfqpoint{2.887807in}{2.510647in}}%
\pgfpathlineto{\pgfqpoint{2.888115in}{2.837461in}}%
\pgfpathlineto{\pgfqpoint{2.888468in}{2.673999in}}%
\pgfpathlineto{\pgfqpoint{2.889460in}{2.517968in}}%
\pgfpathlineto{\pgfqpoint{2.888821in}{2.803698in}}%
\pgfpathlineto{\pgfqpoint{2.889548in}{2.644170in}}%
\pgfpathlineto{\pgfqpoint{2.890055in}{2.775835in}}%
\pgfpathlineto{\pgfqpoint{2.890319in}{2.429462in}}%
\pgfpathlineto{\pgfqpoint{2.890672in}{2.764362in}}%
\pgfpathlineto{\pgfqpoint{2.891003in}{2.534467in}}%
\pgfpathlineto{\pgfqpoint{2.891554in}{2.776928in}}%
\pgfpathlineto{\pgfqpoint{2.891774in}{2.735625in}}%
\pgfpathlineto{\pgfqpoint{2.892303in}{2.809271in}}%
\pgfpathlineto{\pgfqpoint{2.892127in}{2.520044in}}%
\pgfpathlineto{\pgfqpoint{2.892435in}{2.642531in}}%
\pgfpathlineto{\pgfqpoint{2.893119in}{2.526709in}}%
\pgfpathlineto{\pgfqpoint{2.893295in}{2.871224in}}%
\pgfpathlineto{\pgfqpoint{2.893449in}{2.698912in}}%
\pgfpathlineto{\pgfqpoint{2.894199in}{2.864231in}}%
\pgfpathlineto{\pgfqpoint{2.893670in}{2.584948in}}%
\pgfpathlineto{\pgfqpoint{2.894551in}{2.731036in}}%
\pgfpathlineto{\pgfqpoint{2.894860in}{2.557850in}}%
\pgfpathlineto{\pgfqpoint{2.895455in}{2.864778in}}%
\pgfpathlineto{\pgfqpoint{2.895521in}{2.786543in}}%
\pgfpathlineto{\pgfqpoint{2.896425in}{3.002780in}}%
\pgfpathlineto{\pgfqpoint{2.895742in}{2.627671in}}%
\pgfpathlineto{\pgfqpoint{2.896645in}{2.916570in}}%
\pgfpathlineto{\pgfqpoint{2.897086in}{2.640673in}}%
\pgfpathlineto{\pgfqpoint{2.896910in}{2.918755in}}%
\pgfpathlineto{\pgfqpoint{2.897858in}{2.787854in}}%
\pgfpathlineto{\pgfqpoint{2.898078in}{2.626250in}}%
\pgfpathlineto{\pgfqpoint{2.898982in}{2.913073in}}%
\pgfpathlineto{\pgfqpoint{2.900040in}{2.598387in}}%
\pgfpathlineto{\pgfqpoint{2.899621in}{2.968580in}}%
\pgfpathlineto{\pgfqpoint{2.900128in}{2.788401in}}%
\pgfpathlineto{\pgfqpoint{2.901142in}{2.905862in}}%
\pgfpathlineto{\pgfqpoint{2.900525in}{2.586040in}}%
\pgfpathlineto{\pgfqpoint{2.901186in}{2.806211in}}%
\pgfpathlineto{\pgfqpoint{2.901429in}{2.854397in}}%
\pgfpathlineto{\pgfqpoint{2.902332in}{2.596530in}}%
\pgfpathlineto{\pgfqpoint{2.902883in}{2.923563in}}%
\pgfpathlineto{\pgfqpoint{2.903478in}{2.763379in}}%
\pgfpathlineto{\pgfqpoint{2.904074in}{2.615214in}}%
\pgfpathlineto{\pgfqpoint{2.903897in}{2.886412in}}%
\pgfpathlineto{\pgfqpoint{2.904558in}{2.724917in}}%
\pgfpathlineto{\pgfqpoint{2.904581in}{2.836915in}}%
\pgfpathlineto{\pgfqpoint{2.904999in}{2.575988in}}%
\pgfpathlineto{\pgfqpoint{2.905661in}{2.721639in}}%
\pgfpathlineto{\pgfqpoint{2.906168in}{2.488138in}}%
\pgfpathlineto{\pgfqpoint{2.905969in}{2.882588in}}%
\pgfpathlineto{\pgfqpoint{2.906785in}{2.516766in}}%
\pgfpathlineto{\pgfqpoint{2.907931in}{2.842160in}}%
\pgfpathlineto{\pgfqpoint{2.907975in}{2.720874in}}%
\pgfpathlineto{\pgfqpoint{2.908702in}{2.501250in}}%
\pgfpathlineto{\pgfqpoint{2.908813in}{2.808069in}}%
\pgfpathlineto{\pgfqpoint{2.909121in}{2.540367in}}%
\pgfpathlineto{\pgfqpoint{2.910113in}{2.805009in}}%
\pgfpathlineto{\pgfqpoint{2.909320in}{2.458308in}}%
\pgfpathlineto{\pgfqpoint{2.910245in}{2.631167in}}%
\pgfpathlineto{\pgfqpoint{2.911017in}{2.403566in}}%
\pgfpathlineto{\pgfqpoint{2.911303in}{2.739777in}}%
\pgfpathlineto{\pgfqpoint{2.911458in}{2.631167in}}%
\pgfpathlineto{\pgfqpoint{2.911392in}{2.780534in}}%
\pgfpathlineto{\pgfqpoint{2.911502in}{2.728960in}}%
\pgfpathlineto{\pgfqpoint{2.911612in}{2.432303in}}%
\pgfpathlineto{\pgfqpoint{2.911722in}{2.829157in}}%
\pgfpathlineto{\pgfqpoint{2.912604in}{2.682522in}}%
\pgfpathlineto{\pgfqpoint{2.913287in}{2.864996in}}%
\pgfpathlineto{\pgfqpoint{2.912868in}{2.552277in}}%
\pgfpathlineto{\pgfqpoint{2.913684in}{2.635975in}}%
\pgfpathlineto{\pgfqpoint{2.913706in}{2.592050in}}%
\pgfpathlineto{\pgfqpoint{2.913926in}{2.778348in}}%
\pgfpathlineto{\pgfqpoint{2.914786in}{2.621880in}}%
\pgfpathlineto{\pgfqpoint{2.915249in}{2.875704in}}%
\pgfpathlineto{\pgfqpoint{2.915095in}{2.530861in}}%
\pgfpathlineto{\pgfqpoint{2.915910in}{2.713444in}}%
\pgfpathlineto{\pgfqpoint{2.916395in}{2.461368in}}%
\pgfpathlineto{\pgfqpoint{2.916042in}{2.809489in}}%
\pgfpathlineto{\pgfqpoint{2.916682in}{2.612373in}}%
\pgfpathlineto{\pgfqpoint{2.916968in}{2.830687in}}%
\pgfpathlineto{\pgfqpoint{2.917034in}{2.493055in}}%
\pgfpathlineto{\pgfqpoint{2.917806in}{2.736609in}}%
\pgfpathlineto{\pgfqpoint{2.917850in}{2.546049in}}%
\pgfpathlineto{\pgfqpoint{2.918136in}{2.888816in}}%
\pgfpathlineto{\pgfqpoint{2.918886in}{2.669738in}}%
\pgfpathlineto{\pgfqpoint{2.919723in}{2.844563in}}%
\pgfpathlineto{\pgfqpoint{2.919459in}{2.556101in}}%
\pgfpathlineto{\pgfqpoint{2.919988in}{2.679026in}}%
\pgfpathlineto{\pgfqpoint{2.920561in}{2.856910in}}%
\pgfpathlineto{\pgfqpoint{2.920825in}{2.631167in}}%
\pgfpathlineto{\pgfqpoint{2.921068in}{2.749830in}}%
\pgfpathlineto{\pgfqpoint{2.921354in}{2.573912in}}%
\pgfpathlineto{\pgfqpoint{2.921443in}{2.942575in}}%
\pgfpathlineto{\pgfqpoint{2.922170in}{2.766001in}}%
\pgfpathlineto{\pgfqpoint{2.922368in}{2.838445in}}%
\pgfpathlineto{\pgfqpoint{2.923052in}{2.613248in}}%
\pgfpathlineto{\pgfqpoint{2.923184in}{2.738466in}}%
\pgfpathlineto{\pgfqpoint{2.923316in}{2.504309in}}%
\pgfpathlineto{\pgfqpoint{2.923867in}{2.851010in}}%
\pgfpathlineto{\pgfqpoint{2.924308in}{2.653020in}}%
\pgfpathlineto{\pgfqpoint{2.924330in}{2.828283in}}%
\pgfpathlineto{\pgfqpoint{2.925234in}{2.540913in}}%
\pgfpathlineto{\pgfqpoint{2.925410in}{2.697819in}}%
\pgfpathlineto{\pgfqpoint{2.925587in}{2.490870in}}%
\pgfpathlineto{\pgfqpoint{2.926049in}{2.739996in}}%
\pgfpathlineto{\pgfqpoint{2.926534in}{2.611827in}}%
\pgfpathlineto{\pgfqpoint{2.927041in}{2.820525in}}%
\pgfpathlineto{\pgfqpoint{2.927372in}{2.472185in}}%
\pgfpathlineto{\pgfqpoint{2.927658in}{2.659248in}}%
\pgfpathlineto{\pgfqpoint{2.928033in}{2.847404in}}%
\pgfpathlineto{\pgfqpoint{2.927747in}{2.528566in}}%
\pgfpathlineto{\pgfqpoint{2.928408in}{2.615542in}}%
\pgfpathlineto{\pgfqpoint{2.928540in}{2.529768in}}%
\pgfpathlineto{\pgfqpoint{2.928739in}{2.854397in}}%
\pgfpathlineto{\pgfqpoint{2.928981in}{2.693558in}}%
\pgfpathlineto{\pgfqpoint{2.929290in}{2.864013in}}%
\pgfpathlineto{\pgfqpoint{2.929730in}{2.515017in}}%
\pgfpathlineto{\pgfqpoint{2.930083in}{2.686565in}}%
\pgfpathlineto{\pgfqpoint{2.930612in}{2.831014in}}%
\pgfpathlineto{\pgfqpoint{2.930149in}{2.537635in}}%
\pgfpathlineto{\pgfqpoint{2.931185in}{2.667990in}}%
\pgfpathlineto{\pgfqpoint{2.931935in}{2.499174in}}%
\pgfpathlineto{\pgfqpoint{2.932001in}{2.842597in}}%
\pgfpathlineto{\pgfqpoint{2.932243in}{2.651272in}}%
\pgfpathlineto{\pgfqpoint{2.933081in}{2.532063in}}%
\pgfpathlineto{\pgfqpoint{2.933389in}{2.925420in}}%
\pgfpathlineto{\pgfqpoint{2.934249in}{2.555227in}}%
\pgfpathlineto{\pgfqpoint{2.934536in}{2.689078in}}%
\pgfpathlineto{\pgfqpoint{2.934690in}{2.832981in}}%
\pgfpathlineto{\pgfqpoint{2.935395in}{2.554899in}}%
\pgfpathlineto{\pgfqpoint{2.935572in}{2.582872in}}%
\pgfpathlineto{\pgfqpoint{2.935836in}{2.499611in}}%
\pgfpathlineto{\pgfqpoint{2.936321in}{2.802059in}}%
\pgfpathlineto{\pgfqpoint{2.936519in}{2.629747in}}%
\pgfpathlineto{\pgfqpoint{2.937423in}{2.823803in}}%
\pgfpathlineto{\pgfqpoint{2.936938in}{2.558505in}}%
\pgfpathlineto{\pgfqpoint{2.937621in}{2.728851in}}%
\pgfpathlineto{\pgfqpoint{2.938591in}{2.521246in}}%
\pgfpathlineto{\pgfqpoint{2.938040in}{2.895263in}}%
\pgfpathlineto{\pgfqpoint{2.938702in}{2.766985in}}%
\pgfpathlineto{\pgfqpoint{2.939561in}{2.889472in}}%
\pgfpathlineto{\pgfqpoint{2.939032in}{2.588116in}}%
\pgfpathlineto{\pgfqpoint{2.939738in}{2.738576in}}%
\pgfpathlineto{\pgfqpoint{2.940090in}{2.675201in}}%
\pgfpathlineto{\pgfqpoint{2.940531in}{2.920285in}}%
\pgfpathlineto{\pgfqpoint{2.940751in}{2.892422in}}%
\pgfpathlineto{\pgfqpoint{2.940796in}{2.986063in}}%
\pgfpathlineto{\pgfqpoint{2.941325in}{2.617946in}}%
\pgfpathlineto{\pgfqpoint{2.941699in}{2.713007in}}%
\pgfpathlineto{\pgfqpoint{2.941986in}{2.621770in}}%
\pgfpathlineto{\pgfqpoint{2.941743in}{2.850792in}}%
\pgfpathlineto{\pgfqpoint{2.942625in}{2.689078in}}%
\pgfpathlineto{\pgfqpoint{2.942779in}{2.911434in}}%
\pgfpathlineto{\pgfqpoint{2.943661in}{2.559926in}}%
\pgfpathlineto{\pgfqpoint{2.943727in}{2.693230in}}%
\pgfpathlineto{\pgfqpoint{2.943815in}{2.519279in}}%
\pgfpathlineto{\pgfqpoint{2.944631in}{2.863904in}}%
\pgfpathlineto{\pgfqpoint{2.944829in}{2.663619in}}%
\pgfpathlineto{\pgfqpoint{2.945446in}{2.889690in}}%
\pgfpathlineto{\pgfqpoint{2.945336in}{2.629310in}}%
\pgfpathlineto{\pgfqpoint{2.945953in}{2.817356in}}%
\pgfpathlineto{\pgfqpoint{2.946747in}{2.619913in}}%
\pgfpathlineto{\pgfqpoint{2.946945in}{2.978305in}}%
\pgfpathlineto{\pgfqpoint{2.947100in}{2.739340in}}%
\pgfpathlineto{\pgfqpoint{2.947254in}{2.863904in}}%
\pgfpathlineto{\pgfqpoint{2.947386in}{2.637942in}}%
\pgfpathlineto{\pgfqpoint{2.948180in}{2.792007in}}%
\pgfpathlineto{\pgfqpoint{2.948797in}{2.904441in}}%
\pgfpathlineto{\pgfqpoint{2.949282in}{2.611062in}}%
\pgfpathlineto{\pgfqpoint{2.949414in}{2.980709in}}%
\pgfpathlineto{\pgfqpoint{2.949678in}{2.592378in}}%
\pgfpathlineto{\pgfqpoint{2.950406in}{2.889144in}}%
\pgfpathlineto{\pgfqpoint{2.951574in}{2.593470in}}%
\pgfpathlineto{\pgfqpoint{2.952169in}{2.813860in}}%
\pgfpathlineto{\pgfqpoint{2.952059in}{2.514471in}}%
\pgfpathlineto{\pgfqpoint{2.952654in}{2.556429in}}%
\pgfpathlineto{\pgfqpoint{2.952897in}{2.474480in}}%
\pgfpathlineto{\pgfqpoint{2.953007in}{2.806102in}}%
\pgfpathlineto{\pgfqpoint{2.953734in}{2.578610in}}%
\pgfpathlineto{\pgfqpoint{2.954329in}{2.790586in}}%
\pgfpathlineto{\pgfqpoint{2.954263in}{2.535013in}}%
\pgfpathlineto{\pgfqpoint{2.954924in}{2.765783in}}%
\pgfpathlineto{\pgfqpoint{2.955696in}{2.537198in}}%
\pgfpathlineto{\pgfqpoint{2.955960in}{2.775070in}}%
\pgfpathlineto{\pgfqpoint{2.956027in}{2.653457in}}%
\pgfpathlineto{\pgfqpoint{2.956291in}{2.803807in}}%
\pgfpathlineto{\pgfqpoint{2.956401in}{2.474371in}}%
\pgfpathlineto{\pgfqpoint{2.957129in}{2.802933in}}%
\pgfpathlineto{\pgfqpoint{2.957327in}{2.566372in}}%
\pgfpathlineto{\pgfqpoint{2.957547in}{2.830905in}}%
\pgfpathlineto{\pgfqpoint{2.958253in}{2.665914in}}%
\pgfpathlineto{\pgfqpoint{2.959245in}{2.836369in}}%
\pgfpathlineto{\pgfqpoint{2.958848in}{2.568995in}}%
\pgfpathlineto{\pgfqpoint{2.959399in}{2.754201in}}%
\pgfpathlineto{\pgfqpoint{2.959972in}{2.493164in}}%
\pgfpathlineto{\pgfqpoint{2.959443in}{2.853195in}}%
\pgfpathlineto{\pgfqpoint{2.960523in}{2.689078in}}%
\pgfpathlineto{\pgfqpoint{2.961184in}{2.858003in}}%
\pgfpathlineto{\pgfqpoint{2.961118in}{2.497644in}}%
\pgfpathlineto{\pgfqpoint{2.961713in}{2.771027in}}%
\pgfpathlineto{\pgfqpoint{2.962309in}{2.568339in}}%
\pgfpathlineto{\pgfqpoint{2.962661in}{2.851556in}}%
\pgfpathlineto{\pgfqpoint{2.962815in}{2.645699in}}%
\pgfpathlineto{\pgfqpoint{2.962904in}{2.852977in}}%
\pgfpathlineto{\pgfqpoint{2.963256in}{2.590083in}}%
\pgfpathlineto{\pgfqpoint{2.963918in}{2.732129in}}%
\pgfpathlineto{\pgfqpoint{2.964491in}{2.512395in}}%
\pgfpathlineto{\pgfqpoint{2.964843in}{2.832544in}}%
\pgfpathlineto{\pgfqpoint{2.965020in}{2.627343in}}%
\pgfpathlineto{\pgfqpoint{2.965130in}{2.822820in}}%
\pgfpathlineto{\pgfqpoint{2.965438in}{2.546158in}}%
\pgfpathlineto{\pgfqpoint{2.966122in}{2.692575in}}%
\pgfpathlineto{\pgfqpoint{2.966188in}{2.580686in}}%
\pgfpathlineto{\pgfqpoint{2.966430in}{2.850027in}}%
\pgfpathlineto{\pgfqpoint{2.967180in}{2.714428in}}%
\pgfpathlineto{\pgfqpoint{2.968106in}{2.847623in}}%
\pgfpathlineto{\pgfqpoint{2.967819in}{2.553151in}}%
\pgfpathlineto{\pgfqpoint{2.968304in}{2.823912in}}%
\pgfpathlineto{\pgfqpoint{2.968855in}{2.518514in}}%
\pgfpathlineto{\pgfqpoint{2.968701in}{2.921268in}}%
\pgfpathlineto{\pgfqpoint{2.969428in}{2.709183in}}%
\pgfpathlineto{\pgfqpoint{2.969847in}{2.853305in}}%
\pgfpathlineto{\pgfqpoint{2.969626in}{2.596639in}}%
\pgfpathlineto{\pgfqpoint{2.970552in}{2.832763in}}%
\pgfpathlineto{\pgfqpoint{2.971125in}{2.559379in}}%
\pgfpathlineto{\pgfqpoint{2.971654in}{2.869258in}}%
\pgfpathlineto{\pgfqpoint{2.972580in}{2.575769in}}%
\pgfpathlineto{\pgfqpoint{2.972778in}{2.780752in}}%
\pgfpathlineto{\pgfqpoint{2.973197in}{2.897776in}}%
\pgfpathlineto{\pgfqpoint{2.973572in}{2.604615in}}%
\pgfpathlineto{\pgfqpoint{2.973859in}{2.857020in}}%
\pgfpathlineto{\pgfqpoint{2.974101in}{2.613685in}}%
\pgfpathlineto{\pgfqpoint{2.974542in}{2.909358in}}%
\pgfpathlineto{\pgfqpoint{2.974961in}{2.715630in}}%
\pgfpathlineto{\pgfqpoint{2.975379in}{2.962680in}}%
\pgfpathlineto{\pgfqpoint{2.975027in}{2.658593in}}%
\pgfpathlineto{\pgfqpoint{2.976063in}{2.808287in}}%
\pgfpathlineto{\pgfqpoint{2.977121in}{2.621442in}}%
\pgfpathlineto{\pgfqpoint{2.976393in}{2.901928in}}%
\pgfpathlineto{\pgfqpoint{2.977187in}{2.728632in}}%
\pgfpathlineto{\pgfqpoint{2.977980in}{2.855272in}}%
\pgfpathlineto{\pgfqpoint{2.978157in}{2.538947in}}%
\pgfpathlineto{\pgfqpoint{2.978245in}{2.643186in}}%
\pgfpathlineto{\pgfqpoint{2.978267in}{2.582653in}}%
\pgfpathlineto{\pgfqpoint{2.978994in}{2.857457in}}%
\pgfpathlineto{\pgfqpoint{2.979325in}{2.718034in}}%
\pgfpathlineto{\pgfqpoint{2.980185in}{2.891548in}}%
\pgfpathlineto{\pgfqpoint{2.979986in}{2.557850in}}%
\pgfpathlineto{\pgfqpoint{2.980449in}{2.843362in}}%
\pgfpathlineto{\pgfqpoint{2.980714in}{2.599589in}}%
\pgfpathlineto{\pgfqpoint{2.980581in}{2.903239in}}%
\pgfpathlineto{\pgfqpoint{2.981551in}{2.828720in}}%
\pgfpathlineto{\pgfqpoint{2.982499in}{2.604506in}}%
\pgfpathlineto{\pgfqpoint{2.981683in}{2.943777in}}%
\pgfpathlineto{\pgfqpoint{2.982675in}{2.739559in}}%
\pgfpathlineto{\pgfqpoint{2.982918in}{2.819760in}}%
\pgfpathlineto{\pgfqpoint{2.983094in}{2.611171in}}%
\pgfpathlineto{\pgfqpoint{2.983733in}{2.707325in}}%
\pgfpathlineto{\pgfqpoint{2.984615in}{2.577408in}}%
\pgfpathlineto{\pgfqpoint{2.984395in}{2.916023in}}%
\pgfpathlineto{\pgfqpoint{2.984813in}{2.755075in}}%
\pgfpathlineto{\pgfqpoint{2.985827in}{2.860079in}}%
\pgfpathlineto{\pgfqpoint{2.985254in}{2.605599in}}%
\pgfpathlineto{\pgfqpoint{2.985915in}{2.790586in}}%
\pgfpathlineto{\pgfqpoint{2.986863in}{2.627015in}}%
\pgfpathlineto{\pgfqpoint{2.986092in}{2.932632in}}%
\pgfpathlineto{\pgfqpoint{2.987062in}{2.643186in}}%
\pgfpathlineto{\pgfqpoint{2.987569in}{2.888051in}}%
\pgfpathlineto{\pgfqpoint{2.987635in}{2.607784in}}%
\pgfpathlineto{\pgfqpoint{2.988208in}{2.756932in}}%
\pgfpathlineto{\pgfqpoint{2.989090in}{2.527146in}}%
\pgfpathlineto{\pgfqpoint{2.988494in}{2.802605in}}%
\pgfpathlineto{\pgfqpoint{2.989310in}{2.737920in}}%
\pgfpathlineto{\pgfqpoint{2.990214in}{2.863576in}}%
\pgfpathlineto{\pgfqpoint{2.989354in}{2.639253in}}%
\pgfpathlineto{\pgfqpoint{2.990434in}{2.807085in}}%
\pgfpathlineto{\pgfqpoint{2.990985in}{2.600245in}}%
\pgfpathlineto{\pgfqpoint{2.990655in}{2.931867in}}%
\pgfpathlineto{\pgfqpoint{2.991426in}{2.809598in}}%
\pgfpathlineto{\pgfqpoint{2.991448in}{2.921268in}}%
\pgfpathlineto{\pgfqpoint{2.991514in}{2.627452in}}%
\pgfpathlineto{\pgfqpoint{2.992528in}{2.840630in}}%
\pgfpathlineto{\pgfqpoint{2.993652in}{2.517640in}}%
\pgfpathlineto{\pgfqpoint{2.992793in}{2.873082in}}%
\pgfpathlineto{\pgfqpoint{2.993696in}{2.599371in}}%
\pgfpathlineto{\pgfqpoint{2.993718in}{2.761412in}}%
\pgfpathlineto{\pgfqpoint{2.994688in}{2.400616in}}%
\pgfpathlineto{\pgfqpoint{2.994776in}{2.625704in}}%
\pgfpathlineto{\pgfqpoint{2.995856in}{2.440061in}}%
\pgfpathlineto{\pgfqpoint{2.995063in}{2.718034in}}%
\pgfpathlineto{\pgfqpoint{2.995900in}{2.458308in}}%
\pgfpathlineto{\pgfqpoint{2.997025in}{2.262832in}}%
\pgfpathlineto{\pgfqpoint{2.996760in}{2.547469in}}%
\pgfpathlineto{\pgfqpoint{2.997113in}{2.265236in}}%
\pgfpathlineto{\pgfqpoint{2.998017in}{2.484204in}}%
\pgfpathlineto{\pgfqpoint{2.997245in}{2.134554in}}%
\pgfpathlineto{\pgfqpoint{2.998215in}{2.221638in}}%
\pgfpathlineto{\pgfqpoint{2.998325in}{2.488247in}}%
\pgfpathlineto{\pgfqpoint{2.999207in}{2.199567in}}%
\pgfpathlineto{\pgfqpoint{2.999361in}{2.332434in}}%
\pgfpathlineto{\pgfqpoint{2.999405in}{2.145480in}}%
\pgfpathlineto{\pgfqpoint{2.999934in}{2.409576in}}%
\pgfpathlineto{\pgfqpoint{3.000463in}{2.366962in}}%
\pgfpathlineto{\pgfqpoint{3.001014in}{2.060362in}}%
\pgfpathlineto{\pgfqpoint{3.000860in}{2.459401in}}%
\pgfpathlineto{\pgfqpoint{3.001587in}{2.284466in}}%
\pgfpathlineto{\pgfqpoint{3.001720in}{2.050747in}}%
\pgfpathlineto{\pgfqpoint{3.002094in}{2.317683in}}%
\pgfpathlineto{\pgfqpoint{3.002711in}{2.191044in}}%
\pgfpathlineto{\pgfqpoint{3.003307in}{2.366744in}}%
\pgfpathlineto{\pgfqpoint{3.002976in}{1.999173in}}%
\pgfpathlineto{\pgfqpoint{3.003792in}{2.167770in}}%
\pgfpathlineto{\pgfqpoint{3.004387in}{2.054899in}}%
\pgfpathlineto{\pgfqpoint{3.004056in}{2.332653in}}%
\pgfpathlineto{\pgfqpoint{3.004916in}{2.074457in}}%
\pgfpathlineto{\pgfqpoint{3.005665in}{2.313203in}}%
\pgfpathlineto{\pgfqpoint{3.006018in}{2.199130in}}%
\pgfpathlineto{\pgfqpoint{3.006326in}{2.327736in}}%
\pgfpathlineto{\pgfqpoint{3.007186in}{1.966065in}}%
\pgfpathlineto{\pgfqpoint{3.007450in}{2.230817in}}%
\pgfpathlineto{\pgfqpoint{3.007539in}{1.888596in}}%
\pgfpathlineto{\pgfqpoint{3.008310in}{2.062001in}}%
\pgfpathlineto{\pgfqpoint{3.009412in}{1.839208in}}%
\pgfpathlineto{\pgfqpoint{3.008663in}{2.174217in}}%
\pgfpathlineto{\pgfqpoint{3.009434in}{1.982783in}}%
\pgfpathlineto{\pgfqpoint{3.009567in}{2.153675in}}%
\pgfpathlineto{\pgfqpoint{3.010162in}{1.818010in}}%
\pgfpathlineto{\pgfqpoint{3.010536in}{1.998517in}}%
\pgfpathlineto{\pgfqpoint{3.010558in}{1.998955in}}%
\pgfpathlineto{\pgfqpoint{3.010779in}{2.081887in}}%
\pgfpathlineto{\pgfqpoint{3.011418in}{1.784466in}}%
\pgfpathlineto{\pgfqpoint{3.011550in}{1.903565in}}%
\pgfpathlineto{\pgfqpoint{3.012234in}{1.791568in}}%
\pgfpathlineto{\pgfqpoint{3.012300in}{2.080795in}}%
\pgfpathlineto{\pgfqpoint{3.012498in}{1.853849in}}%
\pgfpathlineto{\pgfqpoint{3.013203in}{2.079593in}}%
\pgfpathlineto{\pgfqpoint{3.013137in}{1.793972in}}%
\pgfpathlineto{\pgfqpoint{3.013622in}{1.952626in}}%
\pgfpathlineto{\pgfqpoint{3.014306in}{2.109859in}}%
\pgfpathlineto{\pgfqpoint{3.014658in}{1.824129in}}%
\pgfpathlineto{\pgfqpoint{3.014702in}{1.898976in}}%
\pgfpathlineto{\pgfqpoint{3.015099in}{1.712459in}}%
\pgfpathlineto{\pgfqpoint{3.014790in}{2.063312in}}%
\pgfpathlineto{\pgfqpoint{3.015804in}{1.862481in}}%
\pgfpathlineto{\pgfqpoint{3.015915in}{2.034357in}}%
\pgfpathlineto{\pgfqpoint{3.016157in}{1.674544in}}%
\pgfpathlineto{\pgfqpoint{3.016884in}{1.823583in}}%
\pgfpathlineto{\pgfqpoint{3.017039in}{1.705576in}}%
\pgfpathlineto{\pgfqpoint{3.017237in}{2.001686in}}%
\pgfpathlineto{\pgfqpoint{3.017987in}{1.824785in}}%
\pgfpathlineto{\pgfqpoint{3.018538in}{1.716721in}}%
\pgfpathlineto{\pgfqpoint{3.018692in}{1.962350in}}%
\pgfpathlineto{\pgfqpoint{3.018912in}{1.885755in}}%
\pgfpathlineto{\pgfqpoint{3.019574in}{2.071616in}}%
\pgfpathlineto{\pgfqpoint{3.019111in}{1.792770in}}%
\pgfpathlineto{\pgfqpoint{3.020014in}{1.882477in}}%
\pgfpathlineto{\pgfqpoint{3.020852in}{2.103303in}}%
\pgfpathlineto{\pgfqpoint{3.021183in}{2.057958in}}%
\pgfpathlineto{\pgfqpoint{3.022307in}{1.798124in}}%
\pgfpathlineto{\pgfqpoint{3.021513in}{2.128544in}}%
\pgfpathlineto{\pgfqpoint{3.022351in}{1.899413in}}%
\pgfpathlineto{\pgfqpoint{3.023122in}{2.136193in}}%
\pgfpathlineto{\pgfqpoint{3.023277in}{1.758460in}}%
\pgfpathlineto{\pgfqpoint{3.023475in}{2.012066in}}%
\pgfpathlineto{\pgfqpoint{3.024599in}{1.788071in}}%
\pgfpathlineto{\pgfqpoint{3.024048in}{2.048780in}}%
\pgfpathlineto{\pgfqpoint{3.024621in}{1.837897in}}%
\pgfpathlineto{\pgfqpoint{3.025018in}{1.953063in}}%
\pgfpathlineto{\pgfqpoint{3.025349in}{1.673561in}}%
\pgfpathlineto{\pgfqpoint{3.025657in}{1.771900in}}%
\pgfpathlineto{\pgfqpoint{3.026605in}{1.585602in}}%
\pgfpathlineto{\pgfqpoint{3.026010in}{1.932084in}}%
\pgfpathlineto{\pgfqpoint{3.026759in}{1.639688in}}%
\pgfpathlineto{\pgfqpoint{3.026980in}{1.824894in}}%
\pgfpathlineto{\pgfqpoint{3.027773in}{1.562546in}}%
\pgfpathlineto{\pgfqpoint{3.027883in}{1.701751in}}%
\pgfpathlineto{\pgfqpoint{3.028434in}{1.515016in}}%
\pgfpathlineto{\pgfqpoint{3.028589in}{1.832105in}}%
\pgfpathlineto{\pgfqpoint{3.028897in}{1.690606in}}%
\pgfpathlineto{\pgfqpoint{3.028919in}{1.862591in}}%
\pgfpathlineto{\pgfqpoint{3.029118in}{1.548779in}}%
\pgfpathlineto{\pgfqpoint{3.029999in}{1.641436in}}%
\pgfpathlineto{\pgfqpoint{3.030044in}{1.603412in}}%
\pgfpathlineto{\pgfqpoint{3.030815in}{1.877779in}}%
\pgfpathlineto{\pgfqpoint{3.030991in}{1.715082in}}%
\pgfpathlineto{\pgfqpoint{3.031741in}{1.855379in}}%
\pgfpathlineto{\pgfqpoint{3.031366in}{1.572162in}}%
\pgfpathlineto{\pgfqpoint{3.032071in}{1.765672in}}%
\pgfpathlineto{\pgfqpoint{3.032093in}{1.622643in}}%
\pgfpathlineto{\pgfqpoint{3.033019in}{1.923015in}}%
\pgfpathlineto{\pgfqpoint{3.033174in}{1.695523in}}%
\pgfpathlineto{\pgfqpoint{3.033284in}{1.969562in}}%
\pgfpathlineto{\pgfqpoint{3.033879in}{1.670392in}}%
\pgfpathlineto{\pgfqpoint{3.034298in}{1.796048in}}%
\pgfpathlineto{\pgfqpoint{3.035356in}{2.023539in}}%
\pgfpathlineto{\pgfqpoint{3.034937in}{1.638268in}}%
\pgfpathlineto{\pgfqpoint{3.035422in}{1.829265in}}%
\pgfpathlineto{\pgfqpoint{3.035797in}{1.667332in}}%
\pgfpathlineto{\pgfqpoint{3.035929in}{1.958308in}}%
\pgfpathlineto{\pgfqpoint{3.036568in}{1.686563in}}%
\pgfpathlineto{\pgfqpoint{3.037428in}{1.903128in}}%
\pgfpathlineto{\pgfqpoint{3.036744in}{1.575221in}}%
\pgfpathlineto{\pgfqpoint{3.037670in}{1.745785in}}%
\pgfpathlineto{\pgfqpoint{3.037846in}{1.666677in}}%
\pgfpathlineto{\pgfqpoint{3.038419in}{1.973058in}}%
\pgfpathlineto{\pgfqpoint{3.038706in}{1.764361in}}%
\pgfpathlineto{\pgfqpoint{3.039037in}{2.009772in}}%
\pgfpathlineto{\pgfqpoint{3.038904in}{1.737590in}}%
\pgfpathlineto{\pgfqpoint{3.039808in}{1.825440in}}%
\pgfpathlineto{\pgfqpoint{3.040403in}{1.599151in}}%
\pgfpathlineto{\pgfqpoint{3.039984in}{1.954920in}}%
\pgfpathlineto{\pgfqpoint{3.040910in}{1.735951in}}%
\pgfpathlineto{\pgfqpoint{3.041395in}{1.991634in}}%
\pgfpathlineto{\pgfqpoint{3.041946in}{1.550746in}}%
\pgfpathlineto{\pgfqpoint{3.041990in}{1.672140in}}%
\pgfpathlineto{\pgfqpoint{3.042255in}{1.598823in}}%
\pgfpathlineto{\pgfqpoint{3.042762in}{1.882914in}}%
\pgfpathlineto{\pgfqpoint{3.043070in}{1.683722in}}%
\pgfpathlineto{\pgfqpoint{3.044128in}{1.912197in}}%
\pgfpathlineto{\pgfqpoint{3.043908in}{1.589972in}}%
\pgfpathlineto{\pgfqpoint{3.044172in}{1.756603in}}%
\pgfpathlineto{\pgfqpoint{3.044591in}{1.447380in}}%
\pgfpathlineto{\pgfqpoint{3.044415in}{1.817792in}}%
\pgfpathlineto{\pgfqpoint{3.045297in}{1.566371in}}%
\pgfpathlineto{\pgfqpoint{3.046333in}{1.345544in}}%
\pgfpathlineto{\pgfqpoint{3.046200in}{1.662962in}}%
\pgfpathlineto{\pgfqpoint{3.046619in}{1.415802in}}%
\pgfpathlineto{\pgfqpoint{3.046773in}{1.668097in}}%
\pgfpathlineto{\pgfqpoint{3.047236in}{1.383350in}}%
\pgfpathlineto{\pgfqpoint{3.047765in}{1.575549in}}%
\pgfpathlineto{\pgfqpoint{3.048537in}{1.379963in}}%
\pgfpathlineto{\pgfqpoint{3.047986in}{1.603193in}}%
\pgfpathlineto{\pgfqpoint{3.048911in}{1.478739in}}%
\pgfpathlineto{\pgfqpoint{3.049110in}{1.698473in}}%
\pgfpathlineto{\pgfqpoint{3.049573in}{1.385536in}}%
\pgfpathlineto{\pgfqpoint{3.050036in}{1.553150in}}%
\pgfpathlineto{\pgfqpoint{3.050741in}{1.397446in}}%
\pgfpathlineto{\pgfqpoint{3.050521in}{1.727210in}}%
\pgfpathlineto{\pgfqpoint{3.051138in}{1.509662in}}%
\pgfpathlineto{\pgfqpoint{3.051865in}{1.690497in}}%
\pgfpathlineto{\pgfqpoint{3.052108in}{1.388814in}}%
\pgfpathlineto{\pgfqpoint{3.052262in}{1.566589in}}%
\pgfpathlineto{\pgfqpoint{3.053210in}{1.326751in}}%
\pgfpathlineto{\pgfqpoint{3.052923in}{1.665693in}}%
\pgfpathlineto{\pgfqpoint{3.053408in}{1.391982in}}%
\pgfpathlineto{\pgfqpoint{3.054466in}{1.572817in}}%
\pgfpathlineto{\pgfqpoint{3.054246in}{1.250264in}}%
\pgfpathlineto{\pgfqpoint{3.054532in}{1.463333in}}%
\pgfpathlineto{\pgfqpoint{3.054642in}{1.247533in}}%
\pgfpathlineto{\pgfqpoint{3.054576in}{1.573692in}}%
\pgfpathlineto{\pgfqpoint{3.055678in}{1.411541in}}%
\pgfpathlineto{\pgfqpoint{3.056097in}{1.558285in}}%
\pgfpathlineto{\pgfqpoint{3.056362in}{1.255946in}}%
\pgfpathlineto{\pgfqpoint{3.056626in}{1.359312in}}%
\pgfpathlineto{\pgfqpoint{3.056758in}{1.250483in}}%
\pgfpathlineto{\pgfqpoint{3.057332in}{1.530969in}}%
\pgfpathlineto{\pgfqpoint{3.057684in}{1.289382in}}%
\pgfpathlineto{\pgfqpoint{3.057883in}{1.590956in}}%
\pgfpathlineto{\pgfqpoint{3.058412in}{1.213114in}}%
\pgfpathlineto{\pgfqpoint{3.058808in}{1.422795in}}%
\pgfpathlineto{\pgfqpoint{3.058963in}{1.481690in}}%
\pgfpathlineto{\pgfqpoint{3.058897in}{1.324237in}}%
\pgfpathlineto{\pgfqpoint{3.059007in}{1.400724in}}%
\pgfpathlineto{\pgfqpoint{3.059029in}{1.229613in}}%
\pgfpathlineto{\pgfqpoint{3.059536in}{1.560361in}}%
\pgfpathlineto{\pgfqpoint{3.060087in}{1.454920in}}%
\pgfpathlineto{\pgfqpoint{3.060197in}{1.528455in}}%
\pgfpathlineto{\pgfqpoint{3.060660in}{1.264250in}}%
\pgfpathlineto{\pgfqpoint{3.061123in}{1.432520in}}%
\pgfpathlineto{\pgfqpoint{3.061145in}{1.254526in}}%
\pgfpathlineto{\pgfqpoint{3.061806in}{1.548779in}}%
\pgfpathlineto{\pgfqpoint{3.062225in}{1.260098in}}%
\pgfpathlineto{\pgfqpoint{3.063305in}{1.598167in}}%
\pgfpathlineto{\pgfqpoint{3.063349in}{1.565060in}}%
\pgfpathlineto{\pgfqpoint{3.064363in}{1.299216in}}%
\pgfpathlineto{\pgfqpoint{3.064451in}{1.334399in}}%
\pgfpathlineto{\pgfqpoint{3.064539in}{1.538399in}}%
\pgfpathlineto{\pgfqpoint{3.065465in}{1.236825in}}%
\pgfpathlineto{\pgfqpoint{3.065531in}{1.386847in}}%
\pgfpathlineto{\pgfqpoint{3.066016in}{1.176619in}}%
\pgfpathlineto{\pgfqpoint{3.066214in}{1.490540in}}%
\pgfpathlineto{\pgfqpoint{3.066633in}{1.328171in}}%
\pgfpathlineto{\pgfqpoint{3.067537in}{1.558176in}}%
\pgfpathlineto{\pgfqpoint{3.067361in}{1.259334in}}%
\pgfpathlineto{\pgfqpoint{3.067757in}{1.387612in}}%
\pgfpathlineto{\pgfqpoint{3.067934in}{1.202297in}}%
\pgfpathlineto{\pgfqpoint{3.068330in}{1.493927in}}%
\pgfpathlineto{\pgfqpoint{3.068793in}{1.444211in}}%
\pgfpathlineto{\pgfqpoint{3.068815in}{1.492179in}}%
\pgfpathlineto{\pgfqpoint{3.069411in}{1.193118in}}%
\pgfpathlineto{\pgfqpoint{3.069829in}{1.455794in}}%
\pgfpathlineto{\pgfqpoint{3.070050in}{1.178586in}}%
\pgfpathlineto{\pgfqpoint{3.070821in}{1.485951in}}%
\pgfpathlineto{\pgfqpoint{3.070953in}{1.325221in}}%
\pgfpathlineto{\pgfqpoint{3.071372in}{1.576314in}}%
\pgfpathlineto{\pgfqpoint{3.070998in}{1.297904in}}%
\pgfpathlineto{\pgfqpoint{3.072078in}{1.415147in}}%
\pgfpathlineto{\pgfqpoint{3.072474in}{1.579701in}}%
\pgfpathlineto{\pgfqpoint{3.072541in}{1.311672in}}%
\pgfpathlineto{\pgfqpoint{3.072849in}{1.383460in}}%
\pgfpathlineto{\pgfqpoint{3.073025in}{1.197052in}}%
\pgfpathlineto{\pgfqpoint{3.073907in}{1.534465in}}%
\pgfpathlineto{\pgfqpoint{3.073929in}{1.462459in}}%
\pgfpathlineto{\pgfqpoint{3.075075in}{1.214207in}}%
\pgfpathlineto{\pgfqpoint{3.073973in}{1.524522in}}%
\pgfpathlineto{\pgfqpoint{3.075141in}{1.320632in}}%
\pgfpathlineto{\pgfqpoint{3.075847in}{1.483438in}}%
\pgfpathlineto{\pgfqpoint{3.075538in}{1.167004in}}%
\pgfpathlineto{\pgfqpoint{3.076199in}{1.272555in}}%
\pgfpathlineto{\pgfqpoint{3.076420in}{1.187327in}}%
\pgfpathlineto{\pgfqpoint{3.076839in}{1.473385in}}%
\pgfpathlineto{\pgfqpoint{3.076993in}{1.387721in}}%
\pgfpathlineto{\pgfqpoint{3.077742in}{1.497861in}}%
\pgfpathlineto{\pgfqpoint{3.077654in}{1.169517in}}%
\pgfpathlineto{\pgfqpoint{3.077985in}{1.276707in}}%
\pgfpathlineto{\pgfqpoint{3.078624in}{1.120784in}}%
\pgfpathlineto{\pgfqpoint{3.078249in}{1.374281in}}%
\pgfpathlineto{\pgfqpoint{3.079109in}{1.189185in}}%
\pgfpathlineto{\pgfqpoint{3.079307in}{1.485295in}}%
\pgfpathlineto{\pgfqpoint{3.079660in}{1.147336in}}%
\pgfpathlineto{\pgfqpoint{3.080233in}{1.264141in}}%
\pgfpathlineto{\pgfqpoint{3.080850in}{1.408044in}}%
\pgfpathlineto{\pgfqpoint{3.080938in}{1.149412in}}%
\pgfpathlineto{\pgfqpoint{3.081291in}{1.256711in}}%
\pgfpathlineto{\pgfqpoint{3.082195in}{1.091720in}}%
\pgfpathlineto{\pgfqpoint{3.081622in}{1.442572in}}%
\pgfpathlineto{\pgfqpoint{3.082349in}{1.224259in}}%
\pgfpathlineto{\pgfqpoint{3.082371in}{1.374609in}}%
\pgfpathlineto{\pgfqpoint{3.083032in}{1.079810in}}%
\pgfpathlineto{\pgfqpoint{3.083473in}{1.321834in}}%
\pgfpathlineto{\pgfqpoint{3.084421in}{0.998079in}}%
\pgfpathlineto{\pgfqpoint{3.084642in}{1.248188in}}%
\pgfpathlineto{\pgfqpoint{3.085766in}{1.524522in}}%
\pgfpathlineto{\pgfqpoint{3.084840in}{1.123297in}}%
\pgfpathlineto{\pgfqpoint{3.085832in}{1.444758in}}%
\pgfpathlineto{\pgfqpoint{3.086251in}{1.204701in}}%
\pgfpathlineto{\pgfqpoint{3.085898in}{1.469889in}}%
\pgfpathlineto{\pgfqpoint{3.086934in}{1.406078in}}%
\pgfpathlineto{\pgfqpoint{3.087132in}{1.497752in}}%
\pgfpathlineto{\pgfqpoint{3.087375in}{1.205684in}}%
\pgfpathlineto{\pgfqpoint{3.087617in}{1.298888in}}%
\pgfpathlineto{\pgfqpoint{3.088587in}{1.093468in}}%
\pgfpathlineto{\pgfqpoint{3.088146in}{1.391982in}}%
\pgfpathlineto{\pgfqpoint{3.088807in}{1.158809in}}%
\pgfpathlineto{\pgfqpoint{3.089932in}{1.437000in}}%
\pgfpathlineto{\pgfqpoint{3.089226in}{1.089206in}}%
\pgfpathlineto{\pgfqpoint{3.089954in}{1.419299in}}%
\pgfpathlineto{\pgfqpoint{3.091012in}{1.120675in}}%
\pgfpathlineto{\pgfqpoint{3.090240in}{1.424107in}}%
\pgfpathlineto{\pgfqpoint{3.091100in}{1.262611in}}%
\pgfpathlineto{\pgfqpoint{3.092114in}{1.379198in}}%
\pgfpathlineto{\pgfqpoint{3.091210in}{1.052493in}}%
\pgfpathlineto{\pgfqpoint{3.092136in}{1.233437in}}%
\pgfpathlineto{\pgfqpoint{3.092489in}{1.145588in}}%
\pgfpathlineto{\pgfqpoint{3.092224in}{1.356580in}}%
\pgfpathlineto{\pgfqpoint{3.093238in}{1.243271in}}%
\pgfpathlineto{\pgfqpoint{3.093855in}{1.377231in}}%
\pgfpathlineto{\pgfqpoint{3.093591in}{1.144058in}}%
\pgfpathlineto{\pgfqpoint{3.093987in}{1.256274in}}%
\pgfpathlineto{\pgfqpoint{3.094781in}{1.006164in}}%
\pgfpathlineto{\pgfqpoint{3.094053in}{1.400942in}}%
\pgfpathlineto{\pgfqpoint{3.095134in}{1.031842in}}%
\pgfpathlineto{\pgfqpoint{3.096258in}{1.355378in}}%
\pgfpathlineto{\pgfqpoint{3.096500in}{1.099368in}}%
\pgfpathlineto{\pgfqpoint{3.096654in}{1.028673in}}%
\pgfpathlineto{\pgfqpoint{3.097360in}{1.308066in}}%
\pgfpathlineto{\pgfqpoint{3.097514in}{1.177166in}}%
\pgfpathlineto{\pgfqpoint{3.098286in}{1.306209in}}%
\pgfpathlineto{\pgfqpoint{3.097977in}{1.018184in}}%
\pgfpathlineto{\pgfqpoint{3.098616in}{1.280859in}}%
\pgfpathlineto{\pgfqpoint{3.099057in}{1.123188in}}%
\pgfpathlineto{\pgfqpoint{3.099167in}{1.366633in}}%
\pgfpathlineto{\pgfqpoint{3.099718in}{1.167113in}}%
\pgfpathlineto{\pgfqpoint{3.099806in}{1.424871in}}%
\pgfpathlineto{\pgfqpoint{3.099939in}{1.070631in}}%
\pgfpathlineto{\pgfqpoint{3.100820in}{1.179242in}}%
\pgfpathlineto{\pgfqpoint{3.101283in}{0.983656in}}%
\pgfpathlineto{\pgfqpoint{3.101790in}{1.304570in}}%
\pgfpathlineto{\pgfqpoint{3.101945in}{1.111387in}}%
\pgfpathlineto{\pgfqpoint{3.102143in}{1.260645in}}%
\pgfpathlineto{\pgfqpoint{3.102870in}{1.025723in}}%
\pgfpathlineto{\pgfqpoint{3.103069in}{1.163289in}}%
\pgfpathlineto{\pgfqpoint{3.103113in}{1.181536in}}%
\pgfpathlineto{\pgfqpoint{3.103201in}{1.120129in}}%
\pgfpathlineto{\pgfqpoint{3.103223in}{1.166785in}}%
\pgfpathlineto{\pgfqpoint{3.103245in}{1.047358in}}%
\pgfpathlineto{\pgfqpoint{3.103509in}{1.324237in}}%
\pgfpathlineto{\pgfqpoint{3.104325in}{1.123079in}}%
\pgfpathlineto{\pgfqpoint{3.105097in}{1.365321in}}%
\pgfpathlineto{\pgfqpoint{3.105185in}{0.876466in}}%
\pgfpathlineto{\pgfqpoint{3.105449in}{1.204482in}}%
\pgfpathlineto{\pgfqpoint{3.105890in}{1.046702in}}%
\pgfpathlineto{\pgfqpoint{3.105802in}{1.334618in}}%
\pgfpathlineto{\pgfqpoint{3.106529in}{1.174762in}}%
\pgfpathlineto{\pgfqpoint{3.106838in}{1.449893in}}%
\pgfpathlineto{\pgfqpoint{3.106904in}{1.101881in}}%
\pgfpathlineto{\pgfqpoint{3.107631in}{1.243818in}}%
\pgfpathlineto{\pgfqpoint{3.108535in}{1.050636in}}%
\pgfpathlineto{\pgfqpoint{3.107874in}{1.363573in}}%
\pgfpathlineto{\pgfqpoint{3.108755in}{1.176619in}}%
\pgfpathlineto{\pgfqpoint{3.108800in}{1.364557in}}%
\pgfpathlineto{\pgfqpoint{3.109086in}{1.015780in}}%
\pgfpathlineto{\pgfqpoint{3.109858in}{1.166239in}}%
\pgfpathlineto{\pgfqpoint{3.110938in}{1.073581in}}%
\pgfpathlineto{\pgfqpoint{3.110320in}{1.347620in}}%
\pgfpathlineto{\pgfqpoint{3.110960in}{1.110623in}}%
\pgfpathlineto{\pgfqpoint{3.111136in}{1.313092in}}%
\pgfpathlineto{\pgfqpoint{3.111930in}{0.964206in}}%
\pgfpathlineto{\pgfqpoint{3.112040in}{1.063638in}}%
\pgfpathlineto{\pgfqpoint{3.112260in}{1.274631in}}%
\pgfpathlineto{\pgfqpoint{3.112348in}{0.995893in}}%
\pgfpathlineto{\pgfqpoint{3.113164in}{1.119801in}}%
\pgfpathlineto{\pgfqpoint{3.114178in}{0.936562in}}%
\pgfpathlineto{\pgfqpoint{3.113715in}{1.251685in}}%
\pgfpathlineto{\pgfqpoint{3.114266in}{1.133241in}}%
\pgfpathlineto{\pgfqpoint{3.114927in}{1.301619in}}%
\pgfpathlineto{\pgfqpoint{3.114376in}{0.981142in}}%
\pgfpathlineto{\pgfqpoint{3.115346in}{1.184705in}}%
\pgfpathlineto{\pgfqpoint{3.116007in}{1.028564in}}%
\pgfpathlineto{\pgfqpoint{3.116272in}{1.334181in}}%
\pgfpathlineto{\pgfqpoint{3.116448in}{1.199347in}}%
\pgfpathlineto{\pgfqpoint{3.116492in}{1.121549in}}%
\pgfpathlineto{\pgfqpoint{3.117021in}{1.441152in}}%
\pgfpathlineto{\pgfqpoint{3.117528in}{1.281405in}}%
\pgfpathlineto{\pgfqpoint{3.117572in}{1.412852in}}%
\pgfpathlineto{\pgfqpoint{3.117947in}{1.132694in}}%
\pgfpathlineto{\pgfqpoint{3.118476in}{1.259443in}}%
\pgfpathlineto{\pgfqpoint{3.118983in}{1.094123in}}%
\pgfpathlineto{\pgfqpoint{3.119270in}{1.424981in}}%
\pgfpathlineto{\pgfqpoint{3.119578in}{1.204591in}}%
\pgfpathlineto{\pgfqpoint{3.119909in}{1.449238in}}%
\pgfpathlineto{\pgfqpoint{3.119975in}{1.107782in}}%
\pgfpathlineto{\pgfqpoint{3.120680in}{1.273866in}}%
\pgfpathlineto{\pgfqpoint{3.120724in}{1.093796in}}%
\pgfpathlineto{\pgfqpoint{3.120967in}{1.349806in}}%
\pgfpathlineto{\pgfqpoint{3.121826in}{1.200439in}}%
\pgfpathlineto{\pgfqpoint{3.122620in}{1.344452in}}%
\pgfpathlineto{\pgfqpoint{3.122377in}{1.084727in}}%
\pgfpathlineto{\pgfqpoint{3.122884in}{1.235732in}}%
\pgfpathlineto{\pgfqpoint{3.123480in}{1.025177in}}%
\pgfpathlineto{\pgfqpoint{3.123149in}{1.394605in}}%
\pgfpathlineto{\pgfqpoint{3.123987in}{1.136737in}}%
\pgfpathlineto{\pgfqpoint{3.124956in}{1.403892in}}%
\pgfpathlineto{\pgfqpoint{3.124339in}{1.023319in}}%
\pgfpathlineto{\pgfqpoint{3.125133in}{1.332105in}}%
\pgfpathlineto{\pgfqpoint{3.125816in}{1.101663in}}%
\pgfpathlineto{\pgfqpoint{3.125463in}{1.380946in}}%
\pgfpathlineto{\pgfqpoint{3.126213in}{1.339972in}}%
\pgfpathlineto{\pgfqpoint{3.126786in}{1.447926in}}%
\pgfpathlineto{\pgfqpoint{3.126896in}{1.125592in}}%
\pgfpathlineto{\pgfqpoint{3.127271in}{1.322708in}}%
\pgfpathlineto{\pgfqpoint{3.128351in}{1.160557in}}%
\pgfpathlineto{\pgfqpoint{3.127866in}{1.416458in}}%
\pgfpathlineto{\pgfqpoint{3.128373in}{1.218031in}}%
\pgfpathlineto{\pgfqpoint{3.129144in}{1.480597in}}%
\pgfpathlineto{\pgfqpoint{3.128461in}{1.063529in}}%
\pgfpathlineto{\pgfqpoint{3.129497in}{1.312000in}}%
\pgfpathlineto{\pgfqpoint{3.130070in}{1.058175in}}%
\pgfpathlineto{\pgfqpoint{3.129784in}{1.431209in}}%
\pgfpathlineto{\pgfqpoint{3.130709in}{1.197489in}}%
\pgfpathlineto{\pgfqpoint{3.131437in}{1.334836in}}%
\pgfpathlineto{\pgfqpoint{3.131635in}{1.104831in}}%
\pgfpathlineto{\pgfqpoint{3.131789in}{1.300964in}}%
\pgfpathlineto{\pgfqpoint{3.132627in}{1.041129in}}%
\pgfpathlineto{\pgfqpoint{3.132010in}{1.353849in}}%
\pgfpathlineto{\pgfqpoint{3.132914in}{1.157498in}}%
\pgfpathlineto{\pgfqpoint{3.133883in}{1.359093in}}%
\pgfpathlineto{\pgfqpoint{3.133398in}{1.047248in}}%
\pgfpathlineto{\pgfqpoint{3.133994in}{1.150505in}}%
\pgfpathlineto{\pgfqpoint{3.134897in}{0.976335in}}%
\pgfpathlineto{\pgfqpoint{3.134302in}{1.315824in}}%
\pgfpathlineto{\pgfqpoint{3.135074in}{1.188966in}}%
\pgfpathlineto{\pgfqpoint{3.135316in}{0.945413in}}%
\pgfpathlineto{\pgfqpoint{3.135735in}{1.249718in}}%
\pgfpathlineto{\pgfqpoint{3.136110in}{1.212568in}}%
\pgfpathlineto{\pgfqpoint{3.136132in}{1.281077in}}%
\pgfpathlineto{\pgfqpoint{3.136440in}{1.013376in}}%
\pgfpathlineto{\pgfqpoint{3.137190in}{1.127559in}}%
\pgfpathlineto{\pgfqpoint{3.138160in}{1.003651in}}%
\pgfpathlineto{\pgfqpoint{3.137785in}{1.255400in}}%
\pgfpathlineto{\pgfqpoint{3.138314in}{1.003979in}}%
\pgfpathlineto{\pgfqpoint{3.138358in}{0.844669in}}%
\pgfpathlineto{\pgfqpoint{3.138380in}{1.144386in}}%
\pgfpathlineto{\pgfqpoint{3.138490in}{1.076641in}}%
\pgfpathlineto{\pgfqpoint{3.139548in}{1.252231in}}%
\pgfpathlineto{\pgfqpoint{3.138909in}{0.970434in}}%
\pgfpathlineto{\pgfqpoint{3.139592in}{1.119145in}}%
\pgfpathlineto{\pgfqpoint{3.140231in}{0.886409in}}%
\pgfpathlineto{\pgfqpoint{3.140408in}{1.205247in}}%
\pgfpathlineto{\pgfqpoint{3.140738in}{1.021571in}}%
\pgfpathlineto{\pgfqpoint{3.141730in}{1.322380in}}%
\pgfpathlineto{\pgfqpoint{3.141047in}{0.913835in}}%
\pgfpathlineto{\pgfqpoint{3.141973in}{1.263813in}}%
\pgfpathlineto{\pgfqpoint{3.142193in}{1.065277in}}%
\pgfpathlineto{\pgfqpoint{3.142656in}{1.430990in}}%
\pgfpathlineto{\pgfqpoint{3.143097in}{1.079045in}}%
\pgfpathlineto{\pgfqpoint{3.143185in}{1.364229in}}%
\pgfpathlineto{\pgfqpoint{3.144221in}{1.241523in}}%
\pgfpathlineto{\pgfqpoint{3.144309in}{1.029984in}}%
\pgfpathlineto{\pgfqpoint{3.144552in}{1.392529in}}%
\pgfpathlineto{\pgfqpoint{3.145411in}{1.090627in}}%
\pgfpathlineto{\pgfqpoint{3.145962in}{1.292441in}}%
\pgfpathlineto{\pgfqpoint{3.146161in}{0.994473in}}%
\pgfpathlineto{\pgfqpoint{3.146491in}{1.045719in}}%
\pgfpathlineto{\pgfqpoint{3.146535in}{0.914600in}}%
\pgfpathlineto{\pgfqpoint{3.147351in}{1.292878in}}%
\pgfpathlineto{\pgfqpoint{3.147527in}{1.225461in}}%
\pgfpathlineto{\pgfqpoint{3.147748in}{1.267310in}}%
\pgfpathlineto{\pgfqpoint{3.148100in}{1.082541in}}%
\pgfpathlineto{\pgfqpoint{3.148122in}{1.136300in}}%
\pgfpathlineto{\pgfqpoint{3.148563in}{0.930989in}}%
\pgfpathlineto{\pgfqpoint{3.148806in}{1.189185in}}%
\pgfpathlineto{\pgfqpoint{3.149225in}{1.063092in}}%
\pgfpathlineto{\pgfqpoint{3.150128in}{1.284355in}}%
\pgfpathlineto{\pgfqpoint{3.149665in}{0.993271in}}%
\pgfpathlineto{\pgfqpoint{3.150371in}{1.116523in}}%
\pgfpathlineto{\pgfqpoint{3.150988in}{0.832869in}}%
\pgfpathlineto{\pgfqpoint{3.150635in}{1.284246in}}%
\pgfpathlineto{\pgfqpoint{3.151495in}{1.047685in}}%
\pgfpathlineto{\pgfqpoint{3.152046in}{1.186453in}}%
\pgfpathlineto{\pgfqpoint{3.151649in}{0.896680in}}%
\pgfpathlineto{\pgfqpoint{3.152178in}{0.985732in}}%
\pgfpathlineto{\pgfqpoint{3.152200in}{0.918533in}}%
\pgfpathlineto{\pgfqpoint{3.152839in}{1.251466in}}%
\pgfpathlineto{\pgfqpoint{3.153258in}{1.076532in}}%
\pgfpathlineto{\pgfqpoint{3.153501in}{0.927930in}}%
\pgfpathlineto{\pgfqpoint{3.154206in}{1.218468in}}%
\pgfpathlineto{\pgfqpoint{3.154360in}{1.018184in}}%
\pgfpathlineto{\pgfqpoint{3.154603in}{1.273320in}}%
\pgfpathlineto{\pgfqpoint{3.154978in}{0.946505in}}%
\pgfpathlineto{\pgfqpoint{3.155529in}{1.183612in}}%
\pgfpathlineto{\pgfqpoint{3.155551in}{1.183831in}}%
\pgfpathlineto{\pgfqpoint{3.155661in}{1.299762in}}%
\pgfpathlineto{\pgfqpoint{3.156697in}{0.975679in}}%
\pgfpathlineto{\pgfqpoint{3.157799in}{1.232891in}}%
\pgfpathlineto{\pgfqpoint{3.157248in}{0.914272in}}%
\pgfpathlineto{\pgfqpoint{3.157887in}{1.177930in}}%
\pgfpathlineto{\pgfqpoint{3.158967in}{0.904875in}}%
\pgfpathlineto{\pgfqpoint{3.158703in}{1.181645in}}%
\pgfpathlineto{\pgfqpoint{3.159011in}{0.995347in}}%
\pgfpathlineto{\pgfqpoint{3.159650in}{1.123407in}}%
\pgfpathlineto{\pgfqpoint{3.159188in}{0.827733in}}%
\pgfpathlineto{\pgfqpoint{3.160135in}{1.054023in}}%
\pgfpathlineto{\pgfqpoint{3.160929in}{0.826859in}}%
\pgfpathlineto{\pgfqpoint{3.161127in}{1.232673in}}%
\pgfpathlineto{\pgfqpoint{3.161282in}{0.915692in}}%
\pgfpathlineto{\pgfqpoint{3.162406in}{1.187655in}}%
\pgfpathlineto{\pgfqpoint{3.161568in}{0.859311in}}%
\pgfpathlineto{\pgfqpoint{3.162450in}{1.022445in}}%
\pgfpathlineto{\pgfqpoint{3.163331in}{0.817134in}}%
\pgfpathlineto{\pgfqpoint{3.163376in}{1.145151in}}%
\pgfpathlineto{\pgfqpoint{3.163552in}{0.984311in}}%
\pgfpathlineto{\pgfqpoint{3.163949in}{1.164163in}}%
\pgfpathlineto{\pgfqpoint{3.164147in}{0.885535in}}%
\pgfpathlineto{\pgfqpoint{3.164434in}{1.006492in}}%
\pgfpathlineto{\pgfqpoint{3.164456in}{0.813419in}}%
\pgfpathlineto{\pgfqpoint{3.164963in}{1.190387in}}%
\pgfpathlineto{\pgfqpoint{3.165558in}{0.882694in}}%
\pgfpathlineto{\pgfqpoint{3.165756in}{1.137939in}}%
\pgfpathlineto{\pgfqpoint{3.166109in}{0.834835in}}%
\pgfpathlineto{\pgfqpoint{3.166682in}{0.967266in}}%
\pgfpathlineto{\pgfqpoint{3.167012in}{0.862261in}}%
\pgfpathlineto{\pgfqpoint{3.167233in}{1.060251in}}%
\pgfpathlineto{\pgfqpoint{3.167497in}{0.954045in}}%
\pgfpathlineto{\pgfqpoint{3.168159in}{1.119364in}}%
\pgfpathlineto{\pgfqpoint{3.167960in}{0.831776in}}%
\pgfpathlineto{\pgfqpoint{3.168599in}{1.037524in}}%
\pgfpathlineto{\pgfqpoint{3.169415in}{1.200002in}}%
\pgfpathlineto{\pgfqpoint{3.169724in}{0.786758in}}%
\pgfpathlineto{\pgfqpoint{3.170253in}{1.216064in}}%
\pgfpathlineto{\pgfqpoint{3.170848in}{1.123735in}}%
\pgfpathlineto{\pgfqpoint{3.171178in}{0.921155in}}%
\pgfpathlineto{\pgfqpoint{3.171487in}{1.264250in}}%
\pgfpathlineto{\pgfqpoint{3.171972in}{1.027799in}}%
\pgfpathlineto{\pgfqpoint{3.172104in}{1.236715in}}%
\pgfpathlineto{\pgfqpoint{3.172986in}{0.854285in}}%
\pgfpathlineto{\pgfqpoint{3.173096in}{1.059268in}}%
\pgfpathlineto{\pgfqpoint{3.173559in}{0.810797in}}%
\pgfpathlineto{\pgfqpoint{3.173294in}{1.120129in}}%
\pgfpathlineto{\pgfqpoint{3.174242in}{0.876029in}}%
\pgfpathlineto{\pgfqpoint{3.174264in}{0.832650in}}%
\pgfpathlineto{\pgfqpoint{3.174705in}{1.163289in}}%
\pgfpathlineto{\pgfqpoint{3.175256in}{0.984967in}}%
\pgfpathlineto{\pgfqpoint{3.175410in}{1.268184in}}%
\pgfpathlineto{\pgfqpoint{3.176116in}{0.889468in}}%
\pgfpathlineto{\pgfqpoint{3.176336in}{1.012720in}}%
\pgfpathlineto{\pgfqpoint{3.176535in}{0.899084in}}%
\pgfpathlineto{\pgfqpoint{3.176402in}{1.241742in}}%
\pgfpathlineto{\pgfqpoint{3.177460in}{0.913616in}}%
\pgfpathlineto{\pgfqpoint{3.178166in}{1.298779in}}%
\pgfpathlineto{\pgfqpoint{3.177549in}{0.898428in}}%
\pgfpathlineto{\pgfqpoint{3.178629in}{1.242616in}}%
\pgfpathlineto{\pgfqpoint{3.178761in}{0.922685in}}%
\pgfpathlineto{\pgfqpoint{3.179775in}{1.093031in}}%
\pgfpathlineto{\pgfqpoint{3.180767in}{0.918424in}}%
\pgfpathlineto{\pgfqpoint{3.180436in}{1.213114in}}%
\pgfpathlineto{\pgfqpoint{3.180855in}{1.029766in}}%
\pgfpathlineto{\pgfqpoint{3.181759in}{1.263376in}}%
\pgfpathlineto{\pgfqpoint{3.180899in}{0.912961in}}%
\pgfpathlineto{\pgfqpoint{3.181935in}{1.005400in}}%
\pgfpathlineto{\pgfqpoint{3.182155in}{0.974587in}}%
\pgfpathlineto{\pgfqpoint{3.182508in}{1.281077in}}%
\pgfpathlineto{\pgfqpoint{3.182905in}{1.121003in}}%
\pgfpathlineto{\pgfqpoint{3.183279in}{1.262611in}}%
\pgfpathlineto{\pgfqpoint{3.183919in}{0.940168in}}%
\pgfpathlineto{\pgfqpoint{3.183963in}{1.117616in}}%
\pgfpathlineto{\pgfqpoint{3.184646in}{0.933503in}}%
\pgfpathlineto{\pgfqpoint{3.184337in}{1.232563in}}%
\pgfpathlineto{\pgfqpoint{3.185087in}{1.045500in}}%
\pgfpathlineto{\pgfqpoint{3.185880in}{0.929241in}}%
\pgfpathlineto{\pgfqpoint{3.186057in}{1.134115in}}%
\pgfpathlineto{\pgfqpoint{3.186189in}{1.016217in}}%
\pgfpathlineto{\pgfqpoint{3.186608in}{1.159464in}}%
\pgfpathlineto{\pgfqpoint{3.187093in}{0.903127in}}%
\pgfpathlineto{\pgfqpoint{3.187269in}{1.008459in}}%
\pgfpathlineto{\pgfqpoint{3.187600in}{0.894058in}}%
\pgfpathlineto{\pgfqpoint{3.188063in}{1.166130in}}%
\pgfpathlineto{\pgfqpoint{3.188371in}{1.000592in}}%
\pgfpathlineto{\pgfqpoint{3.188393in}{0.998953in}}%
\pgfpathlineto{\pgfqpoint{3.188437in}{1.057738in}}%
\pgfpathlineto{\pgfqpoint{3.188834in}{1.152909in}}%
\pgfpathlineto{\pgfqpoint{3.188658in}{0.873188in}}%
\pgfpathlineto{\pgfqpoint{3.189319in}{1.130728in}}%
\pgfpathlineto{\pgfqpoint{3.189716in}{0.840954in}}%
\pgfpathlineto{\pgfqpoint{3.190024in}{1.196833in}}%
\pgfpathlineto{\pgfqpoint{3.190421in}{0.961475in}}%
\pgfpathlineto{\pgfqpoint{3.191082in}{1.165037in}}%
\pgfpathlineto{\pgfqpoint{3.191391in}{0.876247in}}%
\pgfpathlineto{\pgfqpoint{3.191545in}{1.049871in}}%
\pgfpathlineto{\pgfqpoint{3.191567in}{1.048887in}}%
\pgfpathlineto{\pgfqpoint{3.192251in}{0.806863in}}%
\pgfpathlineto{\pgfqpoint{3.192691in}{0.913725in}}%
\pgfpathlineto{\pgfqpoint{3.193397in}{0.792331in}}%
\pgfpathlineto{\pgfqpoint{3.193066in}{1.112699in}}%
\pgfpathlineto{\pgfqpoint{3.193727in}{0.979613in}}%
\pgfpathlineto{\pgfqpoint{3.193749in}{1.091610in}}%
\pgfpathlineto{\pgfqpoint{3.194653in}{0.773865in}}%
\pgfpathlineto{\pgfqpoint{3.194829in}{0.941697in}}%
\pgfpathlineto{\pgfqpoint{3.195006in}{0.750701in}}%
\pgfpathlineto{\pgfqpoint{3.195601in}{1.057192in}}%
\pgfpathlineto{\pgfqpoint{3.195910in}{0.909355in}}%
\pgfpathlineto{\pgfqpoint{3.196262in}{1.047795in}}%
\pgfpathlineto{\pgfqpoint{3.196527in}{0.785010in}}%
\pgfpathlineto{\pgfqpoint{3.196990in}{0.871221in}}%
\pgfpathlineto{\pgfqpoint{3.197056in}{0.785775in}}%
\pgfpathlineto{\pgfqpoint{3.197298in}{1.052384in}}%
\pgfpathlineto{\pgfqpoint{3.198026in}{0.930115in}}%
\pgfpathlineto{\pgfqpoint{3.198070in}{1.060360in}}%
\pgfpathlineto{\pgfqpoint{3.198687in}{0.773865in}}%
\pgfpathlineto{\pgfqpoint{3.199128in}{0.939184in}}%
\pgfpathlineto{\pgfqpoint{3.200186in}{0.759114in}}%
\pgfpathlineto{\pgfqpoint{3.199546in}{1.104176in}}%
\pgfpathlineto{\pgfqpoint{3.200274in}{0.900067in}}%
\pgfpathlineto{\pgfqpoint{3.201288in}{1.146680in}}%
\pgfpathlineto{\pgfqpoint{3.200825in}{0.807410in}}%
\pgfpathlineto{\pgfqpoint{3.201398in}{1.105924in}}%
\pgfpathlineto{\pgfqpoint{3.201993in}{0.944101in}}%
\pgfpathlineto{\pgfqpoint{3.201795in}{1.211257in}}%
\pgfpathlineto{\pgfqpoint{3.202500in}{1.104394in}}%
\pgfpathlineto{\pgfqpoint{3.203624in}{0.907060in}}%
\pgfpathlineto{\pgfqpoint{3.203404in}{1.169298in}}%
\pgfpathlineto{\pgfqpoint{3.203690in}{0.954045in}}%
\pgfpathlineto{\pgfqpoint{3.204594in}{1.190824in}}%
\pgfpathlineto{\pgfqpoint{3.204263in}{0.883568in}}%
\pgfpathlineto{\pgfqpoint{3.204837in}{1.142528in}}%
\pgfpathlineto{\pgfqpoint{3.205850in}{1.256820in}}%
\pgfpathlineto{\pgfqpoint{3.205454in}{0.979066in}}%
\pgfpathlineto{\pgfqpoint{3.205917in}{1.105596in}}%
\pgfpathlineto{\pgfqpoint{3.206688in}{0.908371in}}%
\pgfpathlineto{\pgfqpoint{3.206622in}{1.227100in}}%
\pgfpathlineto{\pgfqpoint{3.207019in}{0.957541in}}%
\pgfpathlineto{\pgfqpoint{3.207195in}{1.136300in}}%
\pgfpathlineto{\pgfqpoint{3.207856in}{0.788397in}}%
\pgfpathlineto{\pgfqpoint{3.208099in}{1.071178in}}%
\pgfpathlineto{\pgfqpoint{3.209113in}{0.797357in}}%
\pgfpathlineto{\pgfqpoint{3.208716in}{1.146134in}}%
\pgfpathlineto{\pgfqpoint{3.209223in}{0.830574in}}%
\pgfpathlineto{\pgfqpoint{3.209421in}{1.082541in}}%
\pgfpathlineto{\pgfqpoint{3.209664in}{0.776487in}}%
\pgfpathlineto{\pgfqpoint{3.210369in}{1.002012in}}%
\pgfpathlineto{\pgfqpoint{3.210722in}{0.832650in}}%
\pgfpathlineto{\pgfqpoint{3.211096in}{1.149521in}}%
\pgfpathlineto{\pgfqpoint{3.211471in}{0.995019in}}%
\pgfpathlineto{\pgfqpoint{3.211912in}{1.104504in}}%
\pgfpathlineto{\pgfqpoint{3.211648in}{0.805115in}}%
\pgfpathlineto{\pgfqpoint{3.212529in}{1.068337in}}%
\pgfpathlineto{\pgfqpoint{3.212948in}{0.778891in}}%
\pgfpathlineto{\pgfqpoint{3.213257in}{1.123625in}}%
\pgfpathlineto{\pgfqpoint{3.213653in}{0.838550in}}%
\pgfpathlineto{\pgfqpoint{3.214513in}{1.069539in}}%
\pgfpathlineto{\pgfqpoint{3.213786in}{0.690932in}}%
\pgfpathlineto{\pgfqpoint{3.214755in}{0.890015in}}%
\pgfpathlineto{\pgfqpoint{3.215020in}{1.044517in}}%
\pgfpathlineto{\pgfqpoint{3.215880in}{0.754416in}}%
\pgfpathlineto{\pgfqpoint{3.216034in}{1.017200in}}%
\pgfpathlineto{\pgfqpoint{3.216232in}{0.745565in}}%
\pgfpathlineto{\pgfqpoint{3.216982in}{0.970653in}}%
\pgfpathlineto{\pgfqpoint{3.217621in}{0.738354in}}%
\pgfpathlineto{\pgfqpoint{3.217753in}{1.052602in}}%
\pgfpathlineto{\pgfqpoint{3.218106in}{0.841828in}}%
\pgfpathlineto{\pgfqpoint{3.219164in}{1.134770in}}%
\pgfpathlineto{\pgfqpoint{3.218392in}{0.775723in}}%
\pgfpathlineto{\pgfqpoint{3.219230in}{0.950329in}}%
\pgfpathlineto{\pgfqpoint{3.219891in}{0.897773in}}%
\pgfpathlineto{\pgfqpoint{3.219340in}{1.088442in}}%
\pgfpathlineto{\pgfqpoint{3.219957in}{1.017528in}}%
\pgfpathlineto{\pgfqpoint{3.219979in}{1.213223in}}%
\pgfpathlineto{\pgfqpoint{3.220090in}{0.780202in}}%
\pgfpathlineto{\pgfqpoint{3.221059in}{1.088551in}}%
\pgfpathlineto{\pgfqpoint{3.221170in}{0.793314in}}%
\pgfpathlineto{\pgfqpoint{3.222184in}{1.010863in}}%
\pgfpathlineto{\pgfqpoint{3.222757in}{1.188092in}}%
\pgfpathlineto{\pgfqpoint{3.222272in}{0.828279in}}%
\pgfpathlineto{\pgfqpoint{3.223286in}{1.104176in}}%
\pgfpathlineto{\pgfqpoint{3.224013in}{0.836365in}}%
\pgfpathlineto{\pgfqpoint{3.223462in}{1.170173in}}%
\pgfpathlineto{\pgfqpoint{3.224388in}{1.081230in}}%
\pgfpathlineto{\pgfqpoint{3.224432in}{1.179788in}}%
\pgfpathlineto{\pgfqpoint{3.224917in}{0.925089in}}%
\pgfpathlineto{\pgfqpoint{3.225115in}{0.933065in}}%
\pgfpathlineto{\pgfqpoint{3.225137in}{0.791894in}}%
\pgfpathlineto{\pgfqpoint{3.225424in}{1.158918in}}%
\pgfpathlineto{\pgfqpoint{3.226217in}{0.893511in}}%
\pgfpathlineto{\pgfqpoint{3.227011in}{0.751466in}}%
\pgfpathlineto{\pgfqpoint{3.226724in}{1.077624in}}%
\pgfpathlineto{\pgfqpoint{3.227275in}{0.864119in}}%
\pgfpathlineto{\pgfqpoint{3.227804in}{1.069866in}}%
\pgfpathlineto{\pgfqpoint{3.227937in}{0.732672in}}%
\pgfpathlineto{\pgfqpoint{3.228355in}{0.798778in}}%
\pgfpathlineto{\pgfqpoint{3.228796in}{0.736605in}}%
\pgfpathlineto{\pgfqpoint{3.228973in}{1.023319in}}%
\pgfpathlineto{\pgfqpoint{3.229435in}{0.775395in}}%
\pgfpathlineto{\pgfqpoint{3.230537in}{1.061344in}}%
\pgfpathlineto{\pgfqpoint{3.229502in}{0.698253in}}%
\pgfpathlineto{\pgfqpoint{3.230582in}{0.893183in}}%
\pgfpathlineto{\pgfqpoint{3.231507in}{1.038726in}}%
\pgfpathlineto{\pgfqpoint{3.231067in}{0.822161in}}%
\pgfpathlineto{\pgfqpoint{3.231662in}{0.969779in}}%
\pgfpathlineto{\pgfqpoint{3.231684in}{0.819210in}}%
\pgfpathlineto{\pgfqpoint{3.231794in}{1.153892in}}%
\pgfpathlineto{\pgfqpoint{3.232764in}{1.031077in}}%
\pgfpathlineto{\pgfqpoint{3.232984in}{1.185688in}}%
\pgfpathlineto{\pgfqpoint{3.233337in}{0.821833in}}%
\pgfpathlineto{\pgfqpoint{3.233756in}{0.971964in}}%
\pgfpathlineto{\pgfqpoint{3.233778in}{0.811671in}}%
\pgfpathlineto{\pgfqpoint{3.233998in}{1.167113in}}%
\pgfpathlineto{\pgfqpoint{3.234858in}{0.957869in}}%
\pgfpathlineto{\pgfqpoint{3.235166in}{1.124390in}}%
\pgfpathlineto{\pgfqpoint{3.235277in}{0.847620in}}%
\pgfpathlineto{\pgfqpoint{3.235938in}{0.898865in}}%
\pgfpathlineto{\pgfqpoint{3.236753in}{1.099477in}}%
\pgfpathlineto{\pgfqpoint{3.236709in}{0.842703in}}%
\pgfpathlineto{\pgfqpoint{3.236864in}{0.970325in}}%
\pgfpathlineto{\pgfqpoint{3.236886in}{0.763813in}}%
\pgfpathlineto{\pgfqpoint{3.237657in}{1.110841in}}%
\pgfpathlineto{\pgfqpoint{3.237966in}{0.986715in}}%
\pgfpathlineto{\pgfqpoint{3.238054in}{0.785447in}}%
\pgfpathlineto{\pgfqpoint{3.238296in}{1.108000in}}%
\pgfpathlineto{\pgfqpoint{3.239046in}{0.964643in}}%
\pgfpathlineto{\pgfqpoint{3.239905in}{1.127231in}}%
\pgfpathlineto{\pgfqpoint{3.240060in}{0.838550in}}%
\pgfpathlineto{\pgfqpoint{3.240148in}{0.958415in}}%
\pgfpathlineto{\pgfqpoint{3.240236in}{1.105706in}}%
\pgfpathlineto{\pgfqpoint{3.240809in}{0.788070in}}%
\pgfpathlineto{\pgfqpoint{3.241162in}{0.890015in}}%
\pgfpathlineto{\pgfqpoint{3.242198in}{0.751575in}}%
\pgfpathlineto{\pgfqpoint{3.242021in}{0.981142in}}%
\pgfpathlineto{\pgfqpoint{3.242242in}{0.917331in}}%
\pgfpathlineto{\pgfqpoint{3.242462in}{1.050745in}}%
\pgfpathlineto{\pgfqpoint{3.243146in}{0.753214in}}%
\pgfpathlineto{\pgfqpoint{3.243278in}{0.832213in}}%
\pgfpathlineto{\pgfqpoint{3.243300in}{0.728738in}}%
\pgfpathlineto{\pgfqpoint{3.243851in}{1.058940in}}%
\pgfpathlineto{\pgfqpoint{3.244358in}{0.924652in}}%
\pgfpathlineto{\pgfqpoint{3.244622in}{1.009115in}}%
\pgfpathlineto{\pgfqpoint{3.244997in}{0.675526in}}%
\pgfpathlineto{\pgfqpoint{3.245284in}{0.950876in}}%
\pgfpathlineto{\pgfqpoint{3.246143in}{0.709180in}}%
\pgfpathlineto{\pgfqpoint{3.245526in}{1.003105in}}%
\pgfpathlineto{\pgfqpoint{3.246386in}{0.974040in}}%
\pgfpathlineto{\pgfqpoint{3.247422in}{0.706885in}}%
\pgfpathlineto{\pgfqpoint{3.247378in}{1.015233in}}%
\pgfpathlineto{\pgfqpoint{3.247554in}{0.819757in}}%
\pgfpathlineto{\pgfqpoint{3.247973in}{0.999171in}}%
\pgfpathlineto{\pgfqpoint{3.247598in}{0.758021in}}%
\pgfpathlineto{\pgfqpoint{3.248347in}{0.882148in}}%
\pgfpathlineto{\pgfqpoint{3.248369in}{0.676181in}}%
\pgfpathlineto{\pgfqpoint{3.248766in}{0.974696in}}%
\pgfpathlineto{\pgfqpoint{3.249450in}{0.757475in}}%
\pgfpathlineto{\pgfqpoint{3.249494in}{1.036868in}}%
\pgfpathlineto{\pgfqpoint{3.250574in}{1.022336in}}%
\pgfpathlineto{\pgfqpoint{3.250926in}{1.102974in}}%
\pgfpathlineto{\pgfqpoint{3.251676in}{0.762064in}}%
\pgfpathlineto{\pgfqpoint{3.252271in}{1.092922in}}%
\pgfpathlineto{\pgfqpoint{3.252822in}{1.069320in}}%
\pgfpathlineto{\pgfqpoint{3.253505in}{0.792550in}}%
\pgfpathlineto{\pgfqpoint{3.253990in}{0.826094in}}%
\pgfpathlineto{\pgfqpoint{3.254762in}{1.088551in}}%
\pgfpathlineto{\pgfqpoint{3.254078in}{0.723384in}}%
\pgfpathlineto{\pgfqpoint{3.255092in}{0.887611in}}%
\pgfpathlineto{\pgfqpoint{3.255357in}{0.656841in}}%
\pgfpathlineto{\pgfqpoint{3.256062in}{1.069866in}}%
\pgfpathlineto{\pgfqpoint{3.256150in}{1.004853in}}%
\pgfpathlineto{\pgfqpoint{3.256238in}{0.815495in}}%
\pgfpathlineto{\pgfqpoint{3.256194in}{1.076532in}}%
\pgfpathlineto{\pgfqpoint{3.257032in}{0.963660in}}%
\pgfpathlineto{\pgfqpoint{3.257759in}{1.194867in}}%
\pgfpathlineto{\pgfqpoint{3.257958in}{0.874717in}}%
\pgfpathlineto{\pgfqpoint{3.258134in}{1.058831in}}%
\pgfpathlineto{\pgfqpoint{3.258795in}{0.808393in}}%
\pgfpathlineto{\pgfqpoint{3.259082in}{1.148866in}}%
\pgfpathlineto{\pgfqpoint{3.259236in}{1.044407in}}%
\pgfpathlineto{\pgfqpoint{3.259435in}{1.266545in}}%
\pgfpathlineto{\pgfqpoint{3.260162in}{1.019823in}}%
\pgfpathlineto{\pgfqpoint{3.260360in}{1.202624in}}%
\pgfpathlineto{\pgfqpoint{3.261330in}{1.349041in}}%
\pgfpathlineto{\pgfqpoint{3.260603in}{0.988245in}}%
\pgfpathlineto{\pgfqpoint{3.261462in}{1.229067in}}%
\pgfpathlineto{\pgfqpoint{3.261484in}{1.137065in}}%
\pgfpathlineto{\pgfqpoint{3.262102in}{1.413508in}}%
\pgfpathlineto{\pgfqpoint{3.262542in}{1.268293in}}%
\pgfpathlineto{\pgfqpoint{3.262851in}{1.258787in}}%
\pgfpathlineto{\pgfqpoint{3.263667in}{1.602428in}}%
\pgfpathlineto{\pgfqpoint{3.264592in}{1.330684in}}%
\pgfpathlineto{\pgfqpoint{3.263777in}{1.665584in}}%
\pgfpathlineto{\pgfqpoint{3.264769in}{1.476554in}}%
\pgfpathlineto{\pgfqpoint{3.264835in}{1.734422in}}%
\pgfpathlineto{\pgfqpoint{3.265694in}{1.365431in}}%
\pgfpathlineto{\pgfqpoint{3.265849in}{1.462459in}}%
\pgfpathlineto{\pgfqpoint{3.265871in}{1.358001in}}%
\pgfpathlineto{\pgfqpoint{3.266642in}{1.662962in}}%
\pgfpathlineto{\pgfqpoint{3.266907in}{1.520698in}}%
\pgfpathlineto{\pgfqpoint{3.268009in}{1.698473in}}%
\pgfpathlineto{\pgfqpoint{3.267061in}{1.436235in}}%
\pgfpathlineto{\pgfqpoint{3.268031in}{1.622752in}}%
\pgfpathlineto{\pgfqpoint{3.268582in}{1.423779in}}%
\pgfpathlineto{\pgfqpoint{3.268295in}{1.710930in}}%
\pgfpathlineto{\pgfqpoint{3.269199in}{1.559268in}}%
\pgfpathlineto{\pgfqpoint{3.269530in}{1.440169in}}%
\pgfpathlineto{\pgfqpoint{3.269596in}{1.746004in}}%
\pgfpathlineto{\pgfqpoint{3.269618in}{1.599041in}}%
\pgfpathlineto{\pgfqpoint{3.269640in}{1.757914in}}%
\pgfpathlineto{\pgfqpoint{3.270544in}{1.502559in}}%
\pgfpathlineto{\pgfqpoint{3.270720in}{1.621659in}}%
\pgfpathlineto{\pgfqpoint{3.271778in}{1.802385in}}%
\pgfpathlineto{\pgfqpoint{3.271227in}{1.472730in}}%
\pgfpathlineto{\pgfqpoint{3.271844in}{1.734640in}}%
\pgfpathlineto{\pgfqpoint{3.272329in}{1.499063in}}%
\pgfpathlineto{\pgfqpoint{3.272836in}{1.830576in}}%
\pgfpathlineto{\pgfqpoint{3.272946in}{1.681428in}}%
\pgfpathlineto{\pgfqpoint{3.273541in}{1.790147in}}%
\pgfpathlineto{\pgfqpoint{3.273343in}{1.508351in}}%
\pgfpathlineto{\pgfqpoint{3.273740in}{1.578936in}}%
\pgfpathlineto{\pgfqpoint{3.273872in}{1.728631in}}%
\pgfpathlineto{\pgfqpoint{3.273938in}{1.498735in}}%
\pgfpathlineto{\pgfqpoint{3.274269in}{1.567026in}}%
\pgfpathlineto{\pgfqpoint{3.274798in}{1.327515in}}%
\pgfpathlineto{\pgfqpoint{3.275217in}{1.633897in}}%
\pgfpathlineto{\pgfqpoint{3.275327in}{1.618491in}}%
\pgfpathlineto{\pgfqpoint{3.275834in}{1.381165in}}%
\pgfpathlineto{\pgfqpoint{3.276385in}{1.683832in}}%
\pgfpathlineto{\pgfqpoint{3.276539in}{1.497424in}}%
\pgfpathlineto{\pgfqpoint{3.277024in}{1.800200in}}%
\pgfpathlineto{\pgfqpoint{3.277487in}{1.613464in}}%
\pgfpathlineto{\pgfqpoint{3.278545in}{1.825440in}}%
\pgfpathlineto{\pgfqpoint{3.278347in}{1.555007in}}%
\pgfpathlineto{\pgfqpoint{3.278655in}{1.764798in}}%
\pgfpathlineto{\pgfqpoint{3.279735in}{1.538945in}}%
\pgfpathlineto{\pgfqpoint{3.279162in}{1.831122in}}%
\pgfpathlineto{\pgfqpoint{3.279801in}{1.667332in}}%
\pgfpathlineto{\pgfqpoint{3.280837in}{1.885974in}}%
\pgfpathlineto{\pgfqpoint{3.280463in}{1.551292in}}%
\pgfpathlineto{\pgfqpoint{3.280948in}{1.776489in}}%
\pgfpathlineto{\pgfqpoint{3.281014in}{1.662525in}}%
\pgfpathlineto{\pgfqpoint{3.281256in}{1.939951in}}%
\pgfpathlineto{\pgfqpoint{3.282072in}{1.705248in}}%
\pgfpathlineto{\pgfqpoint{3.282975in}{1.844890in}}%
\pgfpathlineto{\pgfqpoint{3.282535in}{1.390016in}}%
\pgfpathlineto{\pgfqpoint{3.283152in}{1.746987in}}%
\pgfpathlineto{\pgfqpoint{3.283328in}{1.621659in}}%
\pgfpathlineto{\pgfqpoint{3.283879in}{1.945961in}}%
\pgfpathlineto{\pgfqpoint{3.284232in}{1.800528in}}%
\pgfpathlineto{\pgfqpoint{3.284276in}{1.947272in}}%
\pgfpathlineto{\pgfqpoint{3.285202in}{1.657389in}}%
\pgfpathlineto{\pgfqpoint{3.285290in}{1.684050in}}%
\pgfpathlineto{\pgfqpoint{3.286370in}{1.544080in}}%
\pgfpathlineto{\pgfqpoint{3.286017in}{1.897884in}}%
\pgfpathlineto{\pgfqpoint{3.286414in}{1.613683in}}%
\pgfpathlineto{\pgfqpoint{3.286656in}{1.826533in}}%
\pgfpathlineto{\pgfqpoint{3.287450in}{1.527909in}}%
\pgfpathlineto{\pgfqpoint{3.287494in}{1.623298in}}%
\pgfpathlineto{\pgfqpoint{3.288243in}{1.492835in}}%
\pgfpathlineto{\pgfqpoint{3.287869in}{1.801402in}}%
\pgfpathlineto{\pgfqpoint{3.288508in}{1.584290in}}%
\pgfpathlineto{\pgfqpoint{3.289456in}{1.857892in}}%
\pgfpathlineto{\pgfqpoint{3.289037in}{1.538180in}}%
\pgfpathlineto{\pgfqpoint{3.289632in}{1.822599in}}%
\pgfpathlineto{\pgfqpoint{3.289963in}{1.997425in}}%
\pgfpathlineto{\pgfqpoint{3.290315in}{1.690169in}}%
\pgfpathlineto{\pgfqpoint{3.290734in}{1.832761in}}%
\pgfpathlineto{\pgfqpoint{3.290800in}{1.704374in}}%
\pgfpathlineto{\pgfqpoint{3.291484in}{2.033264in}}%
\pgfpathlineto{\pgfqpoint{3.291506in}{2.088990in}}%
\pgfpathlineto{\pgfqpoint{3.292057in}{1.758351in}}%
\pgfpathlineto{\pgfqpoint{3.292498in}{1.864339in}}%
\pgfpathlineto{\pgfqpoint{3.292806in}{1.733438in}}%
\pgfpathlineto{\pgfqpoint{3.292850in}{2.029440in}}%
\pgfpathlineto{\pgfqpoint{3.293622in}{1.817792in}}%
\pgfpathlineto{\pgfqpoint{3.294349in}{2.078500in}}%
\pgfpathlineto{\pgfqpoint{3.294525in}{1.775069in}}%
\pgfpathlineto{\pgfqpoint{3.295032in}{2.024304in}}%
\pgfpathlineto{\pgfqpoint{3.295231in}{1.677057in}}%
\pgfpathlineto{\pgfqpoint{3.296134in}{1.852647in}}%
\pgfpathlineto{\pgfqpoint{3.296994in}{2.055991in}}%
\pgfpathlineto{\pgfqpoint{3.296421in}{1.709837in}}%
\pgfpathlineto{\pgfqpoint{3.297237in}{1.951096in}}%
\pgfpathlineto{\pgfqpoint{3.297810in}{1.820414in}}%
\pgfpathlineto{\pgfqpoint{3.297677in}{2.085275in}}%
\pgfpathlineto{\pgfqpoint{3.298339in}{1.955030in}}%
\pgfpathlineto{\pgfqpoint{3.299397in}{2.223824in}}%
\pgfpathlineto{\pgfqpoint{3.298757in}{1.866852in}}%
\pgfpathlineto{\pgfqpoint{3.299441in}{2.154440in}}%
\pgfpathlineto{\pgfqpoint{3.299573in}{1.813858in}}%
\pgfpathlineto{\pgfqpoint{3.300168in}{2.160012in}}%
\pgfpathlineto{\pgfqpoint{3.300565in}{1.981909in}}%
\pgfpathlineto{\pgfqpoint{3.301336in}{2.080685in}}%
\pgfpathlineto{\pgfqpoint{3.301491in}{1.844452in}}%
\pgfpathlineto{\pgfqpoint{3.301513in}{1.946507in}}%
\pgfpathlineto{\pgfqpoint{3.301535in}{1.856581in}}%
\pgfpathlineto{\pgfqpoint{3.302350in}{2.216612in}}%
\pgfpathlineto{\pgfqpoint{3.302571in}{2.140782in}}%
\pgfpathlineto{\pgfqpoint{3.302659in}{2.018404in}}%
\pgfpathlineto{\pgfqpoint{3.302615in}{2.172797in}}%
\pgfpathlineto{\pgfqpoint{3.302747in}{2.087897in}}%
\pgfpathlineto{\pgfqpoint{3.303034in}{1.853412in}}%
\pgfpathlineto{\pgfqpoint{3.303254in}{2.194431in}}%
\pgfpathlineto{\pgfqpoint{3.303827in}{2.141874in}}%
\pgfpathlineto{\pgfqpoint{3.304114in}{2.274086in}}%
\pgfpathlineto{\pgfqpoint{3.304444in}{1.899850in}}%
\pgfpathlineto{\pgfqpoint{3.304907in}{2.073365in}}%
\pgfpathlineto{\pgfqpoint{3.305084in}{2.245021in}}%
\pgfpathlineto{\pgfqpoint{3.305613in}{1.942355in}}%
\pgfpathlineto{\pgfqpoint{3.306075in}{2.205686in}}%
\pgfpathlineto{\pgfqpoint{3.306164in}{2.272993in}}%
\pgfpathlineto{\pgfqpoint{3.307244in}{1.925091in}}%
\pgfpathlineto{\pgfqpoint{3.307442in}{2.215082in}}%
\pgfpathlineto{\pgfqpoint{3.308302in}{1.880729in}}%
\pgfpathlineto{\pgfqpoint{3.308346in}{2.029003in}}%
\pgfpathlineto{\pgfqpoint{3.309271in}{1.775834in}}%
\pgfpathlineto{\pgfqpoint{3.308654in}{2.107565in}}%
\pgfpathlineto{\pgfqpoint{3.309470in}{1.873954in}}%
\pgfpathlineto{\pgfqpoint{3.310374in}{2.162198in}}%
\pgfpathlineto{\pgfqpoint{3.309536in}{1.833635in}}%
\pgfpathlineto{\pgfqpoint{3.310616in}{2.014580in}}%
\pgfpathlineto{\pgfqpoint{3.311652in}{2.274632in}}%
\pgfpathlineto{\pgfqpoint{3.310947in}{1.849151in}}%
\pgfpathlineto{\pgfqpoint{3.312115in}{2.161761in}}%
\pgfpathlineto{\pgfqpoint{3.312798in}{2.261521in}}%
\pgfpathlineto{\pgfqpoint{3.313239in}{1.965519in}}%
\pgfpathlineto{\pgfqpoint{3.313570in}{2.240541in}}%
\pgfpathlineto{\pgfqpoint{3.313371in}{1.849588in}}%
\pgfpathlineto{\pgfqpoint{3.314363in}{2.071289in}}%
\pgfpathlineto{\pgfqpoint{3.314892in}{2.189514in}}%
\pgfpathlineto{\pgfqpoint{3.314495in}{1.969234in}}%
\pgfpathlineto{\pgfqpoint{3.314980in}{2.105380in}}%
\pgfpathlineto{\pgfqpoint{3.315531in}{1.783264in}}%
\pgfpathlineto{\pgfqpoint{3.315399in}{2.170502in}}%
\pgfpathlineto{\pgfqpoint{3.316105in}{1.996332in}}%
\pgfpathlineto{\pgfqpoint{3.316369in}{2.055882in}}%
\pgfpathlineto{\pgfqpoint{3.316457in}{1.825987in}}%
\pgfpathlineto{\pgfqpoint{3.317140in}{1.952189in}}%
\pgfpathlineto{\pgfqpoint{3.317207in}{1.829592in}}%
\pgfpathlineto{\pgfqpoint{3.317978in}{2.211914in}}%
\pgfpathlineto{\pgfqpoint{3.318221in}{2.053369in}}%
\pgfpathlineto{\pgfqpoint{3.318243in}{2.054899in}}%
\pgfpathlineto{\pgfqpoint{3.319212in}{1.882477in}}%
\pgfpathlineto{\pgfqpoint{3.318992in}{2.252342in}}%
\pgfpathlineto{\pgfqpoint{3.319345in}{2.044628in}}%
\pgfpathlineto{\pgfqpoint{3.319852in}{2.260209in}}%
\pgfpathlineto{\pgfqpoint{3.319786in}{2.012831in}}%
\pgfpathlineto{\pgfqpoint{3.320491in}{2.205358in}}%
\pgfpathlineto{\pgfqpoint{3.320667in}{2.030642in}}%
\pgfpathlineto{\pgfqpoint{3.321549in}{2.333199in}}%
\pgfpathlineto{\pgfqpoint{3.321571in}{2.302932in}}%
\pgfpathlineto{\pgfqpoint{3.322210in}{2.009226in}}%
\pgfpathlineto{\pgfqpoint{3.323004in}{2.091831in}}%
\pgfpathlineto{\pgfqpoint{3.323555in}{2.281516in}}%
\pgfpathlineto{\pgfqpoint{3.323290in}{1.980161in}}%
\pgfpathlineto{\pgfqpoint{3.324106in}{2.135646in}}%
\pgfpathlineto{\pgfqpoint{3.325120in}{1.957980in}}%
\pgfpathlineto{\pgfqpoint{3.324260in}{2.278566in}}%
\pgfpathlineto{\pgfqpoint{3.325208in}{2.165039in}}%
\pgfpathlineto{\pgfqpoint{3.326067in}{1.972840in}}%
\pgfpathlineto{\pgfqpoint{3.325781in}{2.237591in}}%
\pgfpathlineto{\pgfqpoint{3.326398in}{1.993163in}}%
\pgfpathlineto{\pgfqpoint{3.327478in}{2.216721in}}%
\pgfpathlineto{\pgfqpoint{3.327324in}{1.940279in}}%
\pgfpathlineto{\pgfqpoint{3.327544in}{2.087132in}}%
\pgfpathlineto{\pgfqpoint{3.327566in}{2.127123in}}%
\pgfpathlineto{\pgfqpoint{3.327721in}{1.869365in}}%
\pgfpathlineto{\pgfqpoint{3.328602in}{2.015344in}}%
\pgfpathlineto{\pgfqpoint{3.329550in}{1.925637in}}%
\pgfpathlineto{\pgfqpoint{3.329175in}{2.138706in}}%
\pgfpathlineto{\pgfqpoint{3.329682in}{1.984094in}}%
\pgfpathlineto{\pgfqpoint{3.330300in}{2.095546in}}%
\pgfpathlineto{\pgfqpoint{3.330630in}{1.857346in}}%
\pgfpathlineto{\pgfqpoint{3.330762in}{1.950550in}}%
\pgfpathlineto{\pgfqpoint{3.331115in}{1.838334in}}%
\pgfpathlineto{\pgfqpoint{3.331512in}{2.187438in}}%
\pgfpathlineto{\pgfqpoint{3.331732in}{2.071616in}}%
\pgfpathlineto{\pgfqpoint{3.332702in}{2.348059in}}%
\pgfpathlineto{\pgfqpoint{3.331820in}{1.966175in}}%
\pgfpathlineto{\pgfqpoint{3.332878in}{2.194650in}}%
\pgfpathlineto{\pgfqpoint{3.332901in}{2.091940in}}%
\pgfpathlineto{\pgfqpoint{3.333099in}{2.417662in}}%
\pgfpathlineto{\pgfqpoint{3.333959in}{2.317574in}}%
\pgfpathlineto{\pgfqpoint{3.334862in}{2.063312in}}%
\pgfpathlineto{\pgfqpoint{3.334730in}{2.322491in}}%
\pgfpathlineto{\pgfqpoint{3.335171in}{2.092377in}}%
\pgfpathlineto{\pgfqpoint{3.335193in}{2.277364in}}%
\pgfpathlineto{\pgfqpoint{3.335656in}{2.002888in}}%
\pgfpathlineto{\pgfqpoint{3.336273in}{2.139908in}}%
\pgfpathlineto{\pgfqpoint{3.336868in}{2.416678in}}%
\pgfpathlineto{\pgfqpoint{3.337088in}{2.092814in}}%
\pgfpathlineto{\pgfqpoint{3.337551in}{2.328173in}}%
\pgfpathlineto{\pgfqpoint{3.338323in}{1.982346in}}%
\pgfpathlineto{\pgfqpoint{3.338676in}{2.129090in}}%
\pgfpathlineto{\pgfqpoint{3.338874in}{2.330795in}}%
\pgfpathlineto{\pgfqpoint{3.339359in}{2.030860in}}%
\pgfpathlineto{\pgfqpoint{3.339734in}{2.183723in}}%
\pgfpathlineto{\pgfqpoint{3.340703in}{1.971529in}}%
\pgfpathlineto{\pgfqpoint{3.340174in}{2.266875in}}%
\pgfpathlineto{\pgfqpoint{3.340836in}{2.086586in}}%
\pgfpathlineto{\pgfqpoint{3.341916in}{2.412854in}}%
\pgfpathlineto{\pgfqpoint{3.341012in}{2.063312in}}%
\pgfpathlineto{\pgfqpoint{3.341982in}{2.302277in}}%
\pgfpathlineto{\pgfqpoint{3.342511in}{2.066918in}}%
\pgfpathlineto{\pgfqpoint{3.343018in}{2.346529in}}%
\pgfpathlineto{\pgfqpoint{3.343062in}{2.313313in}}%
\pgfpathlineto{\pgfqpoint{3.343701in}{2.385865in}}%
\pgfpathlineto{\pgfqpoint{3.343833in}{1.984422in}}%
\pgfpathlineto{\pgfqpoint{3.344098in}{2.201861in}}%
\pgfpathlineto{\pgfqpoint{3.344120in}{2.009007in}}%
\pgfpathlineto{\pgfqpoint{3.344340in}{2.350681in}}%
\pgfpathlineto{\pgfqpoint{3.345200in}{2.184379in}}%
\pgfpathlineto{\pgfqpoint{3.346015in}{2.403020in}}%
\pgfpathlineto{\pgfqpoint{3.345486in}{2.092377in}}%
\pgfpathlineto{\pgfqpoint{3.346324in}{2.312875in}}%
\pgfpathlineto{\pgfqpoint{3.347360in}{2.079702in}}%
\pgfpathlineto{\pgfqpoint{3.346721in}{2.410996in}}%
\pgfpathlineto{\pgfqpoint{3.347426in}{2.119912in}}%
\pgfpathlineto{\pgfqpoint{3.348154in}{2.429462in}}%
\pgfpathlineto{\pgfqpoint{3.348550in}{2.360406in}}%
\pgfpathlineto{\pgfqpoint{3.349630in}{1.996114in}}%
\pgfpathlineto{\pgfqpoint{3.349145in}{2.421923in}}%
\pgfpathlineto{\pgfqpoint{3.350071in}{2.098059in}}%
\pgfpathlineto{\pgfqpoint{3.350093in}{2.305336in}}%
\pgfpathlineto{\pgfqpoint{3.350534in}{2.017748in}}%
\pgfpathlineto{\pgfqpoint{3.351173in}{2.118382in}}%
\pgfpathlineto{\pgfqpoint{3.352143in}{2.011848in}}%
\pgfpathlineto{\pgfqpoint{3.351790in}{2.291678in}}%
\pgfpathlineto{\pgfqpoint{3.352253in}{2.152036in}}%
\pgfpathlineto{\pgfqpoint{3.352804in}{2.289930in}}%
\pgfpathlineto{\pgfqpoint{3.352959in}{1.975572in}}%
\pgfpathlineto{\pgfqpoint{3.353355in}{2.202408in}}%
\pgfpathlineto{\pgfqpoint{3.353642in}{2.020589in}}%
\pgfpathlineto{\pgfqpoint{3.354171in}{2.306320in}}%
\pgfpathlineto{\pgfqpoint{3.354458in}{2.149632in}}%
\pgfpathlineto{\pgfqpoint{3.355185in}{2.321180in}}%
\pgfpathlineto{\pgfqpoint{3.354634in}{2.057084in}}%
\pgfpathlineto{\pgfqpoint{3.355560in}{2.164274in}}%
\pgfpathlineto{\pgfqpoint{3.356574in}{2.045283in}}%
\pgfpathlineto{\pgfqpoint{3.356684in}{2.381932in}}%
\pgfpathlineto{\pgfqpoint{3.357411in}{2.117180in}}%
\pgfpathlineto{\pgfqpoint{3.357059in}{2.420502in}}%
\pgfpathlineto{\pgfqpoint{3.357808in}{2.246988in}}%
\pgfpathlineto{\pgfqpoint{3.357940in}{2.464100in}}%
\pgfpathlineto{\pgfqpoint{3.358469in}{2.150397in}}%
\pgfpathlineto{\pgfqpoint{3.358866in}{2.185908in}}%
\pgfpathlineto{\pgfqpoint{3.358976in}{2.119147in}}%
\pgfpathlineto{\pgfqpoint{3.359064in}{2.339427in}}%
\pgfpathlineto{\pgfqpoint{3.359946in}{2.205576in}}%
\pgfpathlineto{\pgfqpoint{3.360585in}{2.459073in}}%
\pgfpathlineto{\pgfqpoint{3.360012in}{2.183395in}}%
\pgfpathlineto{\pgfqpoint{3.361114in}{2.340192in}}%
\pgfpathlineto{\pgfqpoint{3.361247in}{2.410122in}}%
\pgfpathlineto{\pgfqpoint{3.361599in}{2.085056in}}%
\pgfpathlineto{\pgfqpoint{3.361996in}{2.288946in}}%
\pgfpathlineto{\pgfqpoint{3.363032in}{2.120021in}}%
\pgfpathlineto{\pgfqpoint{3.362150in}{2.425419in}}%
\pgfpathlineto{\pgfqpoint{3.363098in}{2.240541in}}%
\pgfpathlineto{\pgfqpoint{3.363142in}{2.373846in}}%
\pgfpathlineto{\pgfqpoint{3.363561in}{2.050637in}}%
\pgfpathlineto{\pgfqpoint{3.364112in}{2.222731in}}%
\pgfpathlineto{\pgfqpoint{3.364134in}{2.097075in}}%
\pgfpathlineto{\pgfqpoint{3.365016in}{2.331341in}}%
\pgfpathlineto{\pgfqpoint{3.365214in}{2.150397in}}%
\pgfpathlineto{\pgfqpoint{3.365699in}{2.358549in}}%
\pgfpathlineto{\pgfqpoint{3.365346in}{2.073802in}}%
\pgfpathlineto{\pgfqpoint{3.366338in}{2.310472in}}%
\pgfpathlineto{\pgfqpoint{3.366735in}{1.987591in}}%
\pgfpathlineto{\pgfqpoint{3.367462in}{2.195852in}}%
\pgfpathlineto{\pgfqpoint{3.368035in}{2.404113in}}%
\pgfpathlineto{\pgfqpoint{3.368168in}{2.104942in}}%
\pgfpathlineto{\pgfqpoint{3.368498in}{2.170721in}}%
\pgfpathlineto{\pgfqpoint{3.369027in}{2.114886in}}%
\pgfpathlineto{\pgfqpoint{3.369380in}{2.373409in}}%
\pgfpathlineto{\pgfqpoint{3.369468in}{2.246770in}}%
\pgfpathlineto{\pgfqpoint{3.369667in}{2.420393in}}%
\pgfpathlineto{\pgfqpoint{3.370460in}{2.201424in}}%
\pgfpathlineto{\pgfqpoint{3.370592in}{2.373409in}}%
\pgfpathlineto{\pgfqpoint{3.371209in}{2.166678in}}%
\pgfpathlineto{\pgfqpoint{3.370879in}{2.512614in}}%
\pgfpathlineto{\pgfqpoint{3.371716in}{2.276490in}}%
\pgfpathlineto{\pgfqpoint{3.372245in}{2.471530in}}%
\pgfpathlineto{\pgfqpoint{3.372598in}{2.203719in}}%
\pgfpathlineto{\pgfqpoint{3.372819in}{2.253981in}}%
\pgfpathlineto{\pgfqpoint{3.372885in}{2.131822in}}%
\pgfpathlineto{\pgfqpoint{3.373678in}{2.450878in}}%
\pgfpathlineto{\pgfqpoint{3.373899in}{2.253653in}}%
\pgfpathlineto{\pgfqpoint{3.374582in}{2.476993in}}%
\pgfpathlineto{\pgfqpoint{3.373943in}{2.217814in}}%
\pgfpathlineto{\pgfqpoint{3.375001in}{2.341722in}}%
\pgfpathlineto{\pgfqpoint{3.375971in}{2.120240in}}%
\pgfpathlineto{\pgfqpoint{3.375772in}{2.423999in}}%
\pgfpathlineto{\pgfqpoint{3.376125in}{2.293535in}}%
\pgfpathlineto{\pgfqpoint{3.376169in}{2.164602in}}%
\pgfpathlineto{\pgfqpoint{3.376235in}{2.345000in}}%
\pgfpathlineto{\pgfqpoint{3.376367in}{2.198146in}}%
\pgfpathlineto{\pgfqpoint{3.376389in}{2.068448in}}%
\pgfpathlineto{\pgfqpoint{3.376566in}{2.389252in}}%
\pgfpathlineto{\pgfqpoint{3.377469in}{2.145152in}}%
\pgfpathlineto{\pgfqpoint{3.377513in}{2.053587in}}%
\pgfpathlineto{\pgfqpoint{3.378373in}{2.307303in}}%
\pgfpathlineto{\pgfqpoint{3.378417in}{2.267967in}}%
\pgfpathlineto{\pgfqpoint{3.378968in}{2.463553in}}%
\pgfpathlineto{\pgfqpoint{3.379101in}{2.094671in}}%
\pgfpathlineto{\pgfqpoint{3.379541in}{2.342268in}}%
\pgfpathlineto{\pgfqpoint{3.379850in}{2.488247in}}%
\pgfpathlineto{\pgfqpoint{3.380181in}{2.170283in}}%
\pgfpathlineto{\pgfqpoint{3.380577in}{2.333964in}}%
\pgfpathlineto{\pgfqpoint{3.381106in}{2.099807in}}%
\pgfpathlineto{\pgfqpoint{3.380842in}{2.409139in}}%
\pgfpathlineto{\pgfqpoint{3.381679in}{2.284139in}}%
\pgfpathlineto{\pgfqpoint{3.381944in}{2.399305in}}%
\pgfpathlineto{\pgfqpoint{3.382363in}{2.088771in}}%
\pgfpathlineto{\pgfqpoint{3.382605in}{2.253216in}}%
\pgfpathlineto{\pgfqpoint{3.382737in}{2.111498in}}%
\pgfpathlineto{\pgfqpoint{3.383663in}{2.392858in}}%
\pgfpathlineto{\pgfqpoint{3.383707in}{2.266328in}}%
\pgfpathlineto{\pgfqpoint{3.384699in}{2.460384in}}%
\pgfpathlineto{\pgfqpoint{3.384523in}{2.234532in}}%
\pgfpathlineto{\pgfqpoint{3.384831in}{2.291459in}}%
\pgfpathlineto{\pgfqpoint{3.385162in}{2.190825in}}%
\pgfpathlineto{\pgfqpoint{3.385713in}{2.444213in}}%
\pgfpathlineto{\pgfqpoint{3.385845in}{2.263597in}}%
\pgfpathlineto{\pgfqpoint{3.386573in}{2.494257in}}%
\pgfpathlineto{\pgfqpoint{3.386947in}{2.285450in}}%
\pgfpathlineto{\pgfqpoint{3.387080in}{2.460166in}}%
\pgfpathlineto{\pgfqpoint{3.387543in}{2.245568in}}%
\pgfpathlineto{\pgfqpoint{3.387653in}{2.300310in}}%
\pgfpathlineto{\pgfqpoint{3.388667in}{2.082980in}}%
\pgfpathlineto{\pgfqpoint{3.387807in}{2.317574in}}%
\pgfpathlineto{\pgfqpoint{3.388777in}{2.180336in}}%
\pgfpathlineto{\pgfqpoint{3.389835in}{2.417224in}}%
\pgfpathlineto{\pgfqpoint{3.388843in}{2.027254in}}%
\pgfpathlineto{\pgfqpoint{3.389901in}{2.275944in}}%
\pgfpathlineto{\pgfqpoint{3.389989in}{2.047796in}}%
\pgfpathlineto{\pgfqpoint{3.390805in}{2.473278in}}%
\pgfpathlineto{\pgfqpoint{3.391025in}{2.174217in}}%
\pgfpathlineto{\pgfqpoint{3.391797in}{2.496005in}}%
\pgfpathlineto{\pgfqpoint{3.391202in}{2.156188in}}%
\pgfpathlineto{\pgfqpoint{3.392149in}{2.405424in}}%
\pgfpathlineto{\pgfqpoint{3.392260in}{2.123955in}}%
\pgfpathlineto{\pgfqpoint{3.392899in}{2.460494in}}%
\pgfpathlineto{\pgfqpoint{3.393296in}{2.190825in}}%
\pgfpathlineto{\pgfqpoint{3.394354in}{2.440061in}}%
\pgfpathlineto{\pgfqpoint{3.393538in}{2.180336in}}%
\pgfpathlineto{\pgfqpoint{3.394420in}{2.314624in}}%
\pgfpathlineto{\pgfqpoint{3.394971in}{2.444541in}}%
\pgfpathlineto{\pgfqpoint{3.394464in}{2.156844in}}%
\pgfpathlineto{\pgfqpoint{3.395478in}{2.338116in}}%
\pgfpathlineto{\pgfqpoint{3.395786in}{2.215520in}}%
\pgfpathlineto{\pgfqpoint{3.396403in}{2.584511in}}%
\pgfpathlineto{\pgfqpoint{3.396558in}{2.443339in}}%
\pgfpathlineto{\pgfqpoint{3.396844in}{2.216066in}}%
\pgfpathlineto{\pgfqpoint{3.397373in}{2.524524in}}%
\pgfpathlineto{\pgfqpoint{3.397638in}{2.428260in}}%
\pgfpathlineto{\pgfqpoint{3.398718in}{2.584620in}}%
\pgfpathlineto{\pgfqpoint{3.398542in}{2.330795in}}%
\pgfpathlineto{\pgfqpoint{3.398740in}{2.412417in}}%
\pgfpathlineto{\pgfqpoint{3.399291in}{2.214099in}}%
\pgfpathlineto{\pgfqpoint{3.398872in}{2.569213in}}%
\pgfpathlineto{\pgfqpoint{3.399886in}{2.291787in}}%
\pgfpathlineto{\pgfqpoint{3.400040in}{2.531407in}}%
\pgfpathlineto{\pgfqpoint{3.400415in}{2.193229in}}%
\pgfpathlineto{\pgfqpoint{3.401010in}{2.400397in}}%
\pgfpathlineto{\pgfqpoint{3.402046in}{2.485734in}}%
\pgfpathlineto{\pgfqpoint{3.401870in}{2.277910in}}%
\pgfpathlineto{\pgfqpoint{3.402090in}{2.392421in}}%
\pgfpathlineto{\pgfqpoint{3.402575in}{2.234204in}}%
\pgfpathlineto{\pgfqpoint{3.402641in}{2.502233in}}%
\pgfpathlineto{\pgfqpoint{3.403192in}{2.382041in}}%
\pgfpathlineto{\pgfqpoint{3.403281in}{2.495459in}}%
\pgfpathlineto{\pgfqpoint{3.403964in}{2.219562in}}%
\pgfpathlineto{\pgfqpoint{3.404250in}{2.277583in}}%
\pgfpathlineto{\pgfqpoint{3.404405in}{2.215082in}}%
\pgfpathlineto{\pgfqpoint{3.404779in}{2.470765in}}%
\pgfpathlineto{\pgfqpoint{3.405286in}{2.356800in}}%
\pgfpathlineto{\pgfqpoint{3.405793in}{2.530424in}}%
\pgfpathlineto{\pgfqpoint{3.406036in}{2.187329in}}%
\pgfpathlineto{\pgfqpoint{3.406366in}{2.451425in}}%
\pgfpathlineto{\pgfqpoint{3.406719in}{2.194322in}}%
\pgfpathlineto{\pgfqpoint{3.406631in}{2.504637in}}%
\pgfpathlineto{\pgfqpoint{3.407469in}{2.434270in}}%
\pgfpathlineto{\pgfqpoint{3.408593in}{2.183505in}}%
\pgfpathlineto{\pgfqpoint{3.407535in}{2.454047in}}%
\pgfpathlineto{\pgfqpoint{3.408615in}{2.229724in}}%
\pgfpathlineto{\pgfqpoint{3.408879in}{2.443339in}}%
\pgfpathlineto{\pgfqpoint{3.409474in}{2.221092in}}%
\pgfpathlineto{\pgfqpoint{3.409739in}{2.348168in}}%
\pgfpathlineto{\pgfqpoint{3.410643in}{2.548781in}}%
\pgfpathlineto{\pgfqpoint{3.409783in}{2.229724in}}%
\pgfpathlineto{\pgfqpoint{3.410863in}{2.453064in}}%
\pgfpathlineto{\pgfqpoint{3.411083in}{2.200659in}}%
\pgfpathlineto{\pgfqpoint{3.410951in}{2.586587in}}%
\pgfpathlineto{\pgfqpoint{3.412009in}{2.272775in}}%
\pgfpathlineto{\pgfqpoint{3.412296in}{2.440389in}}%
\pgfpathlineto{\pgfqpoint{3.412957in}{2.176949in}}%
\pgfpathlineto{\pgfqpoint{3.413089in}{2.317355in}}%
\pgfpathlineto{\pgfqpoint{3.413795in}{2.493383in}}%
\pgfpathlineto{\pgfqpoint{3.414235in}{2.180773in}}%
\pgfpathlineto{\pgfqpoint{3.414257in}{2.157718in}}%
\pgfpathlineto{\pgfqpoint{3.414919in}{2.441372in}}%
\pgfpathlineto{\pgfqpoint{3.415183in}{2.333527in}}%
\pgfpathlineto{\pgfqpoint{3.415800in}{2.468798in}}%
\pgfpathlineto{\pgfqpoint{3.416043in}{2.164274in}}%
\pgfpathlineto{\pgfqpoint{3.416241in}{2.283264in}}%
\pgfpathlineto{\pgfqpoint{3.416748in}{2.083854in}}%
\pgfpathlineto{\pgfqpoint{3.416329in}{2.404331in}}%
\pgfpathlineto{\pgfqpoint{3.417365in}{2.144934in}}%
\pgfpathlineto{\pgfqpoint{3.418357in}{2.488794in}}%
\pgfpathlineto{\pgfqpoint{3.417762in}{2.128544in}}%
\pgfpathlineto{\pgfqpoint{3.418534in}{2.413619in}}%
\pgfpathlineto{\pgfqpoint{3.419459in}{2.234969in}}%
\pgfpathlineto{\pgfqpoint{3.418754in}{2.538728in}}%
\pgfpathlineto{\pgfqpoint{3.419614in}{2.458308in}}%
\pgfpathlineto{\pgfqpoint{3.419636in}{2.469126in}}%
\pgfpathlineto{\pgfqpoint{3.420055in}{2.202626in}}%
\pgfpathlineto{\pgfqpoint{3.420231in}{2.364777in}}%
\pgfpathlineto{\pgfqpoint{3.420363in}{2.181975in}}%
\pgfpathlineto{\pgfqpoint{3.420782in}{2.462570in}}%
\pgfpathlineto{\pgfqpoint{3.421311in}{2.417771in}}%
\pgfpathlineto{\pgfqpoint{3.421333in}{2.427933in}}%
\pgfpathlineto{\pgfqpoint{3.421840in}{2.186892in}}%
\pgfpathlineto{\pgfqpoint{3.422038in}{2.248409in}}%
\pgfpathlineto{\pgfqpoint{3.422589in}{2.204811in}}%
\pgfpathlineto{\pgfqpoint{3.422655in}{2.436018in}}%
\pgfpathlineto{\pgfqpoint{3.422964in}{2.226446in}}%
\pgfpathlineto{\pgfqpoint{3.423978in}{2.494585in}}%
\pgfpathlineto{\pgfqpoint{3.423890in}{2.170174in}}%
\pgfpathlineto{\pgfqpoint{3.424088in}{2.318011in}}%
\pgfpathlineto{\pgfqpoint{3.424198in}{2.499611in}}%
\pgfpathlineto{\pgfqpoint{3.424529in}{2.218798in}}%
\pgfpathlineto{\pgfqpoint{3.425190in}{2.327189in}}%
\pgfpathlineto{\pgfqpoint{3.425543in}{2.539930in}}%
\pgfpathlineto{\pgfqpoint{3.425477in}{2.269715in}}%
\pgfpathlineto{\pgfqpoint{3.426292in}{2.319541in}}%
\pgfpathlineto{\pgfqpoint{3.427218in}{2.238684in}}%
\pgfpathlineto{\pgfqpoint{3.427086in}{2.550420in}}%
\pgfpathlineto{\pgfqpoint{3.427350in}{2.402801in}}%
\pgfpathlineto{\pgfqpoint{3.427813in}{2.552823in}}%
\pgfpathlineto{\pgfqpoint{3.427615in}{2.268623in}}%
\pgfpathlineto{\pgfqpoint{3.428453in}{2.433833in}}%
\pgfpathlineto{\pgfqpoint{3.428717in}{2.227648in}}%
\pgfpathlineto{\pgfqpoint{3.429444in}{2.515564in}}%
\pgfpathlineto{\pgfqpoint{3.429511in}{2.397884in}}%
\pgfpathlineto{\pgfqpoint{3.430458in}{2.630074in}}%
\pgfpathlineto{\pgfqpoint{3.429907in}{2.365869in}}%
\pgfpathlineto{\pgfqpoint{3.430569in}{2.399305in}}%
\pgfpathlineto{\pgfqpoint{3.430591in}{2.269934in}}%
\pgfpathlineto{\pgfqpoint{3.431186in}{2.678479in}}%
\pgfpathlineto{\pgfqpoint{3.431649in}{2.369803in}}%
\pgfpathlineto{\pgfqpoint{3.432200in}{2.622098in}}%
\pgfpathlineto{\pgfqpoint{3.432773in}{2.462461in}}%
\pgfpathlineto{\pgfqpoint{3.432795in}{2.342268in}}%
\pgfpathlineto{\pgfqpoint{3.433346in}{2.726993in}}%
\pgfpathlineto{\pgfqpoint{3.433875in}{2.434161in}}%
\pgfpathlineto{\pgfqpoint{3.433897in}{2.432849in}}%
\pgfpathlineto{\pgfqpoint{3.433919in}{2.476774in}}%
\pgfpathlineto{\pgfqpoint{3.433941in}{2.438968in}}%
\pgfpathlineto{\pgfqpoint{3.434889in}{2.563313in}}%
\pgfpathlineto{\pgfqpoint{3.434514in}{2.292333in}}%
\pgfpathlineto{\pgfqpoint{3.435065in}{2.497426in}}%
\pgfpathlineto{\pgfqpoint{3.435682in}{2.542990in}}%
\pgfpathlineto{\pgfqpoint{3.435594in}{2.331232in}}%
\pgfpathlineto{\pgfqpoint{3.435837in}{2.405970in}}%
\pgfpathlineto{\pgfqpoint{3.435881in}{2.321945in}}%
\pgfpathlineto{\pgfqpoint{3.436718in}{2.643842in}}%
\pgfpathlineto{\pgfqpoint{3.436873in}{2.460275in}}%
\pgfpathlineto{\pgfqpoint{3.437798in}{2.665586in}}%
\pgfpathlineto{\pgfqpoint{3.436917in}{2.337023in}}%
\pgfpathlineto{\pgfqpoint{3.438041in}{2.579484in}}%
\pgfpathlineto{\pgfqpoint{3.438614in}{2.382150in}}%
\pgfpathlineto{\pgfqpoint{3.438151in}{2.689624in}}%
\pgfpathlineto{\pgfqpoint{3.439143in}{2.453501in}}%
\pgfpathlineto{\pgfqpoint{3.440201in}{2.331341in}}%
\pgfpathlineto{\pgfqpoint{3.440267in}{2.632697in}}%
\pgfpathlineto{\pgfqpoint{3.440510in}{2.416350in}}%
\pgfpathlineto{\pgfqpoint{3.441281in}{2.649415in}}%
\pgfpathlineto{\pgfqpoint{3.441391in}{2.489012in}}%
\pgfpathlineto{\pgfqpoint{3.442273in}{2.393732in}}%
\pgfpathlineto{\pgfqpoint{3.441545in}{2.626032in}}%
\pgfpathlineto{\pgfqpoint{3.442317in}{2.427933in}}%
\pgfpathlineto{\pgfqpoint{3.443353in}{2.632588in}}%
\pgfpathlineto{\pgfqpoint{3.443133in}{2.259882in}}%
\pgfpathlineto{\pgfqpoint{3.443419in}{2.387613in}}%
\pgfpathlineto{\pgfqpoint{3.444345in}{2.556320in}}%
\pgfpathlineto{\pgfqpoint{3.443551in}{2.312111in}}%
\pgfpathlineto{\pgfqpoint{3.444543in}{2.502343in}}%
\pgfpathlineto{\pgfqpoint{3.445337in}{2.295611in}}%
\pgfpathlineto{\pgfqpoint{3.444918in}{2.532063in}}%
\pgfpathlineto{\pgfqpoint{3.445667in}{2.450441in}}%
\pgfpathlineto{\pgfqpoint{3.446351in}{2.587242in}}%
\pgfpathlineto{\pgfqpoint{3.446527in}{2.369257in}}%
\pgfpathlineto{\pgfqpoint{3.446747in}{2.459183in}}%
\pgfpathlineto{\pgfqpoint{3.446990in}{2.334947in}}%
\pgfpathlineto{\pgfqpoint{3.446858in}{2.595000in}}%
\pgfpathlineto{\pgfqpoint{3.447849in}{2.416023in}}%
\pgfpathlineto{\pgfqpoint{3.447982in}{2.346529in}}%
\pgfpathlineto{\pgfqpoint{3.448401in}{2.603523in}}%
\pgfpathlineto{\pgfqpoint{3.448687in}{2.460931in}}%
\pgfpathlineto{\pgfqpoint{3.449084in}{2.659576in}}%
\pgfpathlineto{\pgfqpoint{3.449525in}{2.409357in}}%
\pgfpathlineto{\pgfqpoint{3.449789in}{2.519934in}}%
\pgfpathlineto{\pgfqpoint{3.450781in}{2.653020in}}%
\pgfpathlineto{\pgfqpoint{3.450362in}{2.429462in}}%
\pgfpathlineto{\pgfqpoint{3.450869in}{2.600136in}}%
\pgfpathlineto{\pgfqpoint{3.451993in}{2.299654in}}%
\pgfpathlineto{\pgfqpoint{3.451663in}{2.746443in}}%
\pgfpathlineto{\pgfqpoint{3.452015in}{2.368273in}}%
\pgfpathlineto{\pgfqpoint{3.453073in}{2.624502in}}%
\pgfpathlineto{\pgfqpoint{3.452390in}{2.348933in}}%
\pgfpathlineto{\pgfqpoint{3.453294in}{2.461368in}}%
\pgfpathlineto{\pgfqpoint{3.453470in}{2.314733in}}%
\pgfpathlineto{\pgfqpoint{3.453867in}{2.620131in}}%
\pgfpathlineto{\pgfqpoint{3.454396in}{2.458745in}}%
\pgfpathlineto{\pgfqpoint{3.454616in}{2.714974in}}%
\pgfpathlineto{\pgfqpoint{3.455167in}{2.312548in}}%
\pgfpathlineto{\pgfqpoint{3.455432in}{2.389689in}}%
\pgfpathlineto{\pgfqpoint{3.455454in}{2.242945in}}%
\pgfpathlineto{\pgfqpoint{3.455807in}{2.670721in}}%
\pgfpathlineto{\pgfqpoint{3.456534in}{2.412307in}}%
\pgfpathlineto{\pgfqpoint{3.457019in}{2.353413in}}%
\pgfpathlineto{\pgfqpoint{3.457328in}{2.607566in}}%
\pgfpathlineto{\pgfqpoint{3.457548in}{2.468470in}}%
\pgfpathlineto{\pgfqpoint{3.457570in}{2.641766in}}%
\pgfpathlineto{\pgfqpoint{3.458011in}{2.356036in}}%
\pgfpathlineto{\pgfqpoint{3.458650in}{2.448693in}}%
\pgfpathlineto{\pgfqpoint{3.459003in}{2.620241in}}%
\pgfpathlineto{\pgfqpoint{3.459091in}{2.362591in}}%
\pgfpathlineto{\pgfqpoint{3.459752in}{2.485297in}}%
\pgfpathlineto{\pgfqpoint{3.459862in}{2.357019in}}%
\pgfpathlineto{\pgfqpoint{3.460413in}{2.665695in}}%
\pgfpathlineto{\pgfqpoint{3.460810in}{2.512177in}}%
\pgfpathlineto{\pgfqpoint{3.461493in}{2.676403in}}%
\pgfpathlineto{\pgfqpoint{3.461538in}{2.356036in}}%
\pgfpathlineto{\pgfqpoint{3.461912in}{2.609532in}}%
\pgfpathlineto{\pgfqpoint{3.462926in}{2.677387in}}%
\pgfpathlineto{\pgfqpoint{3.463036in}{2.410996in}}%
\pgfpathlineto{\pgfqpoint{3.463477in}{2.627015in}}%
\pgfpathlineto{\pgfqpoint{3.464116in}{2.378654in}}%
\pgfpathlineto{\pgfqpoint{3.464139in}{2.443120in}}%
\pgfpathlineto{\pgfqpoint{3.465373in}{2.697382in}}%
\pgfpathlineto{\pgfqpoint{3.464623in}{2.409467in}}%
\pgfpathlineto{\pgfqpoint{3.465439in}{2.575660in}}%
\pgfpathlineto{\pgfqpoint{3.465549in}{2.439515in}}%
\pgfpathlineto{\pgfqpoint{3.465792in}{2.728305in}}%
\pgfpathlineto{\pgfqpoint{3.466519in}{2.633462in}}%
\pgfpathlineto{\pgfqpoint{3.467268in}{2.761521in}}%
\pgfpathlineto{\pgfqpoint{3.466629in}{2.452408in}}%
\pgfpathlineto{\pgfqpoint{3.467511in}{2.595000in}}%
\pgfpathlineto{\pgfqpoint{3.467533in}{2.529659in}}%
\pgfpathlineto{\pgfqpoint{3.467687in}{2.764472in}}%
\pgfpathlineto{\pgfqpoint{3.468613in}{2.609970in}}%
\pgfpathlineto{\pgfqpoint{3.469076in}{2.428151in}}%
\pgfpathlineto{\pgfqpoint{3.468811in}{2.742400in}}%
\pgfpathlineto{\pgfqpoint{3.469759in}{2.524851in}}%
\pgfpathlineto{\pgfqpoint{3.469936in}{2.720984in}}%
\pgfpathlineto{\pgfqpoint{3.470332in}{2.362591in}}%
\pgfpathlineto{\pgfqpoint{3.470817in}{2.510210in}}%
\pgfpathlineto{\pgfqpoint{3.470972in}{2.459183in}}%
\pgfpathlineto{\pgfqpoint{3.470883in}{2.547797in}}%
\pgfpathlineto{\pgfqpoint{3.470994in}{2.516219in}}%
\pgfpathlineto{\pgfqpoint{3.471721in}{2.831889in}}%
\pgfpathlineto{\pgfqpoint{3.471170in}{2.450332in}}%
\pgfpathlineto{\pgfqpoint{3.472118in}{2.601556in}}%
\pgfpathlineto{\pgfqpoint{3.472801in}{2.726884in}}%
\pgfpathlineto{\pgfqpoint{3.472294in}{2.506822in}}%
\pgfpathlineto{\pgfqpoint{3.473198in}{2.553370in}}%
\pgfpathlineto{\pgfqpoint{3.473617in}{2.391656in}}%
\pgfpathlineto{\pgfqpoint{3.474079in}{2.680555in}}%
\pgfpathlineto{\pgfqpoint{3.474234in}{2.575769in}}%
\pgfpathlineto{\pgfqpoint{3.474256in}{2.709074in}}%
\pgfpathlineto{\pgfqpoint{3.475027in}{2.410778in}}%
\pgfpathlineto{\pgfqpoint{3.475336in}{2.581233in}}%
\pgfpathlineto{\pgfqpoint{3.475424in}{2.403129in}}%
\pgfpathlineto{\pgfqpoint{3.475843in}{2.652692in}}%
\pgfpathlineto{\pgfqpoint{3.476416in}{2.528785in}}%
\pgfpathlineto{\pgfqpoint{3.476504in}{2.677605in}}%
\pgfpathlineto{\pgfqpoint{3.477474in}{2.364340in}}%
\pgfpathlineto{\pgfqpoint{3.477540in}{2.619694in}}%
\pgfpathlineto{\pgfqpoint{3.477650in}{2.387613in}}%
\pgfpathlineto{\pgfqpoint{3.478179in}{2.657609in}}%
\pgfpathlineto{\pgfqpoint{3.478642in}{2.530533in}}%
\pgfpathlineto{\pgfqpoint{3.479744in}{2.706233in}}%
\pgfpathlineto{\pgfqpoint{3.479193in}{2.278347in}}%
\pgfpathlineto{\pgfqpoint{3.479788in}{2.634882in}}%
\pgfpathlineto{\pgfqpoint{3.480670in}{2.392421in}}%
\pgfpathlineto{\pgfqpoint{3.480406in}{2.651928in}}%
\pgfpathlineto{\pgfqpoint{3.480890in}{2.565170in}}%
\pgfpathlineto{\pgfqpoint{3.480912in}{2.602430in}}%
\pgfpathlineto{\pgfqpoint{3.481596in}{2.279112in}}%
\pgfpathlineto{\pgfqpoint{3.481970in}{2.526709in}}%
\pgfpathlineto{\pgfqpoint{3.481993in}{2.529331in}}%
\pgfpathlineto{\pgfqpoint{3.482125in}{2.413946in}}%
\pgfpathlineto{\pgfqpoint{3.482279in}{2.523212in}}%
\pgfpathlineto{\pgfqpoint{3.482522in}{2.278129in}}%
\pgfpathlineto{\pgfqpoint{3.483293in}{2.623409in}}%
\pgfpathlineto{\pgfqpoint{3.483403in}{2.417771in}}%
\pgfpathlineto{\pgfqpoint{3.483932in}{2.656845in}}%
\pgfpathlineto{\pgfqpoint{3.483778in}{2.339318in}}%
\pgfpathlineto{\pgfqpoint{3.484527in}{2.508134in}}%
\pgfpathlineto{\pgfqpoint{3.485145in}{2.586259in}}%
\pgfpathlineto{\pgfqpoint{3.485519in}{2.357893in}}%
\pgfpathlineto{\pgfqpoint{3.485652in}{2.549436in}}%
\pgfpathlineto{\pgfqpoint{3.486136in}{2.388597in}}%
\pgfpathlineto{\pgfqpoint{3.486423in}{2.638051in}}%
\pgfpathlineto{\pgfqpoint{3.486754in}{2.487045in}}%
\pgfpathlineto{\pgfqpoint{3.487812in}{2.686019in}}%
\pgfpathlineto{\pgfqpoint{3.487327in}{2.361499in}}%
\pgfpathlineto{\pgfqpoint{3.487856in}{2.589755in}}%
\pgfpathlineto{\pgfqpoint{3.488914in}{2.408046in}}%
\pgfpathlineto{\pgfqpoint{3.488186in}{2.707216in}}%
\pgfpathlineto{\pgfqpoint{3.488958in}{2.504856in}}%
\pgfpathlineto{\pgfqpoint{3.489994in}{2.672470in}}%
\pgfpathlineto{\pgfqpoint{3.489244in}{2.455905in}}%
\pgfpathlineto{\pgfqpoint{3.490038in}{2.556757in}}%
\pgfpathlineto{\pgfqpoint{3.490721in}{2.266328in}}%
\pgfpathlineto{\pgfqpoint{3.490611in}{2.661980in}}%
\pgfpathlineto{\pgfqpoint{3.491140in}{2.541132in}}%
\pgfpathlineto{\pgfqpoint{3.491956in}{2.325987in}}%
\pgfpathlineto{\pgfqpoint{3.491603in}{2.597513in}}%
\pgfpathlineto{\pgfqpoint{3.492242in}{2.514471in}}%
\pgfpathlineto{\pgfqpoint{3.492705in}{2.344344in}}%
\pgfpathlineto{\pgfqpoint{3.492881in}{2.575114in}}%
\pgfpathlineto{\pgfqpoint{3.493256in}{2.488684in}}%
\pgfpathlineto{\pgfqpoint{3.494336in}{2.704375in}}%
\pgfpathlineto{\pgfqpoint{3.493300in}{2.369366in}}%
\pgfpathlineto{\pgfqpoint{3.494402in}{2.593361in}}%
\pgfpathlineto{\pgfqpoint{3.494424in}{2.344016in}}%
\pgfpathlineto{\pgfqpoint{3.494997in}{2.697273in}}%
\pgfpathlineto{\pgfqpoint{3.495504in}{2.548562in}}%
\pgfpathlineto{\pgfqpoint{3.495592in}{2.740542in}}%
\pgfpathlineto{\pgfqpoint{3.496364in}{2.446726in}}%
\pgfpathlineto{\pgfqpoint{3.496628in}{2.639471in}}%
\pgfpathlineto{\pgfqpoint{3.496937in}{2.428042in}}%
\pgfpathlineto{\pgfqpoint{3.497686in}{2.680009in}}%
\pgfpathlineto{\pgfqpoint{3.497775in}{2.568339in}}%
\pgfpathlineto{\pgfqpoint{3.498700in}{2.687985in}}%
\pgfpathlineto{\pgfqpoint{3.498304in}{2.450113in}}%
\pgfpathlineto{\pgfqpoint{3.498811in}{2.546486in}}%
\pgfpathlineto{\pgfqpoint{3.499648in}{2.396027in}}%
\pgfpathlineto{\pgfqpoint{3.499494in}{2.604288in}}%
\pgfpathlineto{\pgfqpoint{3.499891in}{2.568230in}}%
\pgfpathlineto{\pgfqpoint{3.500927in}{2.647776in}}%
\pgfpathlineto{\pgfqpoint{3.500794in}{2.371988in}}%
\pgfpathlineto{\pgfqpoint{3.500971in}{2.574349in}}%
\pgfpathlineto{\pgfqpoint{3.501169in}{2.404659in}}%
\pgfpathlineto{\pgfqpoint{3.501698in}{2.695962in}}%
\pgfpathlineto{\pgfqpoint{3.502095in}{2.505074in}}%
\pgfpathlineto{\pgfqpoint{3.502183in}{2.593252in}}%
\pgfpathlineto{\pgfqpoint{3.502800in}{2.399960in}}%
\pgfpathlineto{\pgfqpoint{3.503021in}{2.424873in}}%
\pgfpathlineto{\pgfqpoint{3.503748in}{2.315279in}}%
\pgfpathlineto{\pgfqpoint{3.503792in}{2.627780in}}%
\pgfpathlineto{\pgfqpoint{3.504101in}{2.484860in}}%
\pgfpathlineto{\pgfqpoint{3.504762in}{2.403785in}}%
\pgfpathlineto{\pgfqpoint{3.505048in}{2.634991in}}%
\pgfpathlineto{\pgfqpoint{3.505247in}{2.646246in}}%
\pgfpathlineto{\pgfqpoint{3.505225in}{2.489340in}}%
\pgfpathlineto{\pgfqpoint{3.505357in}{2.512395in}}%
\pgfpathlineto{\pgfqpoint{3.506349in}{2.273758in}}%
\pgfpathlineto{\pgfqpoint{3.505533in}{2.633789in}}%
\pgfpathlineto{\pgfqpoint{3.506481in}{2.412635in}}%
\pgfpathlineto{\pgfqpoint{3.507473in}{2.679353in}}%
\pgfpathlineto{\pgfqpoint{3.506635in}{2.270480in}}%
\pgfpathlineto{\pgfqpoint{3.507583in}{2.508243in}}%
\pgfpathlineto{\pgfqpoint{3.507649in}{2.386193in}}%
\pgfpathlineto{\pgfqpoint{3.507804in}{2.662417in}}%
\pgfpathlineto{\pgfqpoint{3.508663in}{2.519934in}}%
\pgfpathlineto{\pgfqpoint{3.509545in}{2.747098in}}%
\pgfpathlineto{\pgfqpoint{3.509082in}{2.413619in}}%
\pgfpathlineto{\pgfqpoint{3.509787in}{2.546377in}}%
\pgfpathlineto{\pgfqpoint{3.509810in}{2.547142in}}%
\pgfpathlineto{\pgfqpoint{3.510361in}{2.443339in}}%
\pgfpathlineto{\pgfqpoint{3.510846in}{2.754310in}}%
\pgfpathlineto{\pgfqpoint{3.511110in}{2.433396in}}%
\pgfpathlineto{\pgfqpoint{3.510890in}{2.860188in}}%
\pgfpathlineto{\pgfqpoint{3.512036in}{2.644279in}}%
\pgfpathlineto{\pgfqpoint{3.512939in}{2.760429in}}%
\pgfpathlineto{\pgfqpoint{3.512675in}{2.409248in}}%
\pgfpathlineto{\pgfqpoint{3.513050in}{2.581560in}}%
\pgfpathlineto{\pgfqpoint{3.513579in}{2.455140in}}%
\pgfpathlineto{\pgfqpoint{3.514020in}{2.773213in}}%
\pgfpathlineto{\pgfqpoint{3.514130in}{2.605490in}}%
\pgfpathlineto{\pgfqpoint{3.514328in}{2.728195in}}%
\pgfpathlineto{\pgfqpoint{3.514350in}{2.474480in}}%
\pgfpathlineto{\pgfqpoint{3.515232in}{2.605271in}}%
\pgfpathlineto{\pgfqpoint{3.516114in}{2.421377in}}%
\pgfpathlineto{\pgfqpoint{3.515849in}{2.758243in}}%
\pgfpathlineto{\pgfqpoint{3.516246in}{2.694214in}}%
\pgfpathlineto{\pgfqpoint{3.516532in}{2.809380in}}%
\pgfpathlineto{\pgfqpoint{3.516621in}{2.517203in}}%
\pgfpathlineto{\pgfqpoint{3.517326in}{2.623191in}}%
\pgfpathlineto{\pgfqpoint{3.517348in}{2.621005in}}%
\pgfpathlineto{\pgfqpoint{3.517370in}{2.713007in}}%
\pgfpathlineto{\pgfqpoint{3.517965in}{2.831670in}}%
\pgfpathlineto{\pgfqpoint{3.518384in}{2.553370in}}%
\pgfpathlineto{\pgfqpoint{3.518913in}{2.782500in}}%
\pgfpathlineto{\pgfqpoint{3.519177in}{2.471857in}}%
\pgfpathlineto{\pgfqpoint{3.519552in}{2.657719in}}%
\pgfpathlineto{\pgfqpoint{3.520368in}{2.472404in}}%
\pgfpathlineto{\pgfqpoint{3.519839in}{2.802824in}}%
\pgfpathlineto{\pgfqpoint{3.520676in}{2.584620in}}%
\pgfpathlineto{\pgfqpoint{3.521602in}{2.795831in}}%
\pgfpathlineto{\pgfqpoint{3.521227in}{2.511739in}}%
\pgfpathlineto{\pgfqpoint{3.521756in}{2.604178in}}%
\pgfpathlineto{\pgfqpoint{3.522836in}{2.392421in}}%
\pgfpathlineto{\pgfqpoint{3.521999in}{2.759773in}}%
\pgfpathlineto{\pgfqpoint{3.522880in}{2.516875in}}%
\pgfpathlineto{\pgfqpoint{3.523520in}{2.402911in}}%
\pgfpathlineto{\pgfqpoint{3.523299in}{2.710385in}}%
\pgfpathlineto{\pgfqpoint{3.523696in}{2.667990in}}%
\pgfpathlineto{\pgfqpoint{3.524181in}{2.732784in}}%
\pgfpathlineto{\pgfqpoint{3.523938in}{2.433396in}}%
\pgfpathlineto{\pgfqpoint{3.524512in}{2.601119in}}%
\pgfpathlineto{\pgfqpoint{3.525217in}{2.387613in}}%
\pgfpathlineto{\pgfqpoint{3.524798in}{2.698147in}}%
\pgfpathlineto{\pgfqpoint{3.525636in}{2.480708in}}%
\pgfpathlineto{\pgfqpoint{3.526231in}{2.339427in}}%
\pgfpathlineto{\pgfqpoint{3.526517in}{2.668099in}}%
\pgfpathlineto{\pgfqpoint{3.526694in}{2.539821in}}%
\pgfpathlineto{\pgfqpoint{3.526914in}{2.671705in}}%
\pgfpathlineto{\pgfqpoint{3.527223in}{2.359204in}}%
\pgfpathlineto{\pgfqpoint{3.527796in}{2.517858in}}%
\pgfpathlineto{\pgfqpoint{3.528016in}{2.613248in}}%
\pgfpathlineto{\pgfqpoint{3.528523in}{2.379091in}}%
\pgfpathlineto{\pgfqpoint{3.528876in}{2.487373in}}%
\pgfpathlineto{\pgfqpoint{3.529912in}{2.400944in}}%
\pgfpathlineto{\pgfqpoint{3.529471in}{2.660997in}}%
\pgfpathlineto{\pgfqpoint{3.530000in}{2.445306in}}%
\pgfpathlineto{\pgfqpoint{3.530441in}{2.704703in}}%
\pgfpathlineto{\pgfqpoint{3.530948in}{2.255074in}}%
\pgfpathlineto{\pgfqpoint{3.531124in}{2.529441in}}%
\pgfpathlineto{\pgfqpoint{3.532050in}{2.437111in}}%
\pgfpathlineto{\pgfqpoint{3.531741in}{2.663838in}}%
\pgfpathlineto{\pgfqpoint{3.532226in}{2.527583in}}%
\pgfpathlineto{\pgfqpoint{3.532469in}{2.730490in}}%
\pgfpathlineto{\pgfqpoint{3.532623in}{2.428042in}}%
\pgfpathlineto{\pgfqpoint{3.533350in}{2.614777in}}%
\pgfpathlineto{\pgfqpoint{3.533879in}{2.368820in}}%
\pgfpathlineto{\pgfqpoint{3.534188in}{2.785341in}}%
\pgfpathlineto{\pgfqpoint{3.534452in}{2.552277in}}%
\pgfpathlineto{\pgfqpoint{3.535488in}{2.811347in}}%
\pgfpathlineto{\pgfqpoint{3.534651in}{2.468798in}}%
\pgfpathlineto{\pgfqpoint{3.535577in}{2.688750in}}%
\pgfpathlineto{\pgfqpoint{3.536260in}{2.511412in}}%
\pgfpathlineto{\pgfqpoint{3.536084in}{2.781189in}}%
\pgfpathlineto{\pgfqpoint{3.536679in}{2.694104in}}%
\pgfpathlineto{\pgfqpoint{3.536723in}{2.816919in}}%
\pgfpathlineto{\pgfqpoint{3.536877in}{2.452080in}}%
\pgfpathlineto{\pgfqpoint{3.537560in}{2.660887in}}%
\pgfpathlineto{\pgfqpoint{3.538574in}{2.397447in}}%
\pgfpathlineto{\pgfqpoint{3.537825in}{2.728960in}}%
\pgfpathlineto{\pgfqpoint{3.538662in}{2.499392in}}%
\pgfpathlineto{\pgfqpoint{3.538795in}{2.710494in}}%
\pgfpathlineto{\pgfqpoint{3.539610in}{2.408155in}}%
\pgfpathlineto{\pgfqpoint{3.539765in}{2.538400in}}%
\pgfpathlineto{\pgfqpoint{3.540448in}{2.428916in}}%
\pgfpathlineto{\pgfqpoint{3.540250in}{2.703829in}}%
\pgfpathlineto{\pgfqpoint{3.540845in}{2.600573in}}%
\pgfpathlineto{\pgfqpoint{3.541484in}{2.383352in}}%
\pgfpathlineto{\pgfqpoint{3.541704in}{2.733221in}}%
\pgfpathlineto{\pgfqpoint{3.541859in}{2.573912in}}%
\pgfpathlineto{\pgfqpoint{3.541881in}{2.768405in}}%
\pgfpathlineto{\pgfqpoint{3.542035in}{2.334619in}}%
\pgfpathlineto{\pgfqpoint{3.542961in}{2.760429in}}%
\pgfpathlineto{\pgfqpoint{3.543997in}{2.440170in}}%
\pgfpathlineto{\pgfqpoint{3.544107in}{2.583855in}}%
\pgfpathlineto{\pgfqpoint{3.544129in}{2.586914in}}%
\pgfpathlineto{\pgfqpoint{3.544151in}{2.456123in}}%
\pgfpathlineto{\pgfqpoint{3.544173in}{2.415695in}}%
\pgfpathlineto{\pgfqpoint{3.544592in}{2.706779in}}%
\pgfpathlineto{\pgfqpoint{3.545143in}{2.614449in}}%
\pgfpathlineto{\pgfqpoint{3.545209in}{2.721202in}}%
\pgfpathlineto{\pgfqpoint{3.546091in}{2.467487in}}%
\pgfpathlineto{\pgfqpoint{3.546223in}{2.546595in}}%
\pgfpathlineto{\pgfqpoint{3.546686in}{2.740542in}}%
\pgfpathlineto{\pgfqpoint{3.547127in}{2.464209in}}%
\pgfpathlineto{\pgfqpoint{3.547281in}{2.589318in}}%
\pgfpathlineto{\pgfqpoint{3.547567in}{2.372535in}}%
\pgfpathlineto{\pgfqpoint{3.547810in}{2.685909in}}%
\pgfpathlineto{\pgfqpoint{3.548361in}{2.664930in}}%
\pgfpathlineto{\pgfqpoint{3.549221in}{2.453938in}}%
\pgfpathlineto{\pgfqpoint{3.548912in}{2.769826in}}%
\pgfpathlineto{\pgfqpoint{3.549485in}{2.569869in}}%
\pgfpathlineto{\pgfqpoint{3.549992in}{2.692575in}}%
\pgfpathlineto{\pgfqpoint{3.549573in}{2.463116in}}%
\pgfpathlineto{\pgfqpoint{3.550565in}{2.614122in}}%
\pgfpathlineto{\pgfqpoint{3.551138in}{2.386849in}}%
\pgfpathlineto{\pgfqpoint{3.550852in}{2.654113in}}%
\pgfpathlineto{\pgfqpoint{3.551689in}{2.511084in}}%
\pgfpathlineto{\pgfqpoint{3.552439in}{2.703501in}}%
\pgfpathlineto{\pgfqpoint{3.552306in}{2.383571in}}%
\pgfpathlineto{\pgfqpoint{3.552813in}{2.584948in}}%
\pgfpathlineto{\pgfqpoint{3.552924in}{2.404768in}}%
\pgfpathlineto{\pgfqpoint{3.553122in}{2.714100in}}%
\pgfpathlineto{\pgfqpoint{3.553871in}{2.613138in}}%
\pgfpathlineto{\pgfqpoint{3.554224in}{2.680118in}}%
\pgfpathlineto{\pgfqpoint{3.553938in}{2.391110in}}%
\pgfpathlineto{\pgfqpoint{3.554952in}{2.636630in}}%
\pgfpathlineto{\pgfqpoint{3.556054in}{2.466394in}}%
\pgfpathlineto{\pgfqpoint{3.555106in}{2.702846in}}%
\pgfpathlineto{\pgfqpoint{3.556076in}{2.574458in}}%
\pgfpathlineto{\pgfqpoint{3.556164in}{2.434161in}}%
\pgfpathlineto{\pgfqpoint{3.556693in}{2.773978in}}%
\pgfpathlineto{\pgfqpoint{3.556759in}{2.686674in}}%
\pgfpathlineto{\pgfqpoint{3.556781in}{2.891329in}}%
\pgfpathlineto{\pgfqpoint{3.557332in}{2.455468in}}%
\pgfpathlineto{\pgfqpoint{3.557839in}{2.599699in}}%
\pgfpathlineto{\pgfqpoint{3.557905in}{2.435909in}}%
\pgfpathlineto{\pgfqpoint{3.558699in}{2.730162in}}%
\pgfpathlineto{\pgfqpoint{3.558853in}{2.560254in}}%
\pgfpathlineto{\pgfqpoint{3.559823in}{2.704703in}}%
\pgfpathlineto{\pgfqpoint{3.559514in}{2.396573in}}%
\pgfpathlineto{\pgfqpoint{3.559955in}{2.528894in}}%
\pgfpathlineto{\pgfqpoint{3.560793in}{2.346857in}}%
\pgfpathlineto{\pgfqpoint{3.560109in}{2.630293in}}%
\pgfpathlineto{\pgfqpoint{3.561079in}{2.433942in}}%
\pgfpathlineto{\pgfqpoint{3.562049in}{2.683068in}}%
\pgfpathlineto{\pgfqpoint{3.562071in}{2.423890in}}%
\pgfpathlineto{\pgfqpoint{3.562181in}{2.467596in}}%
\pgfpathlineto{\pgfqpoint{3.562931in}{2.360625in}}%
\pgfpathlineto{\pgfqpoint{3.562798in}{2.628435in}}%
\pgfpathlineto{\pgfqpoint{3.563217in}{2.599371in}}%
\pgfpathlineto{\pgfqpoint{3.563592in}{2.682631in}}%
\pgfpathlineto{\pgfqpoint{3.563945in}{2.420830in}}%
\pgfpathlineto{\pgfqpoint{3.564143in}{2.616198in}}%
\pgfpathlineto{\pgfqpoint{3.564826in}{2.364777in}}%
\pgfpathlineto{\pgfqpoint{3.564694in}{2.703938in}}%
\pgfpathlineto{\pgfqpoint{3.565245in}{2.514362in}}%
\pgfpathlineto{\pgfqpoint{3.565488in}{2.697273in}}%
\pgfpathlineto{\pgfqpoint{3.565355in}{2.345655in}}%
\pgfpathlineto{\pgfqpoint{3.566369in}{2.589755in}}%
\pgfpathlineto{\pgfqpoint{3.566568in}{2.417771in}}%
\pgfpathlineto{\pgfqpoint{3.566766in}{2.682413in}}%
\pgfpathlineto{\pgfqpoint{3.567515in}{2.460822in}}%
\pgfpathlineto{\pgfqpoint{3.567626in}{2.663073in}}%
\pgfpathlineto{\pgfqpoint{3.567824in}{2.390345in}}%
\pgfpathlineto{\pgfqpoint{3.568618in}{2.455686in}}%
\pgfpathlineto{\pgfqpoint{3.569125in}{2.768951in}}%
\pgfpathlineto{\pgfqpoint{3.569742in}{2.569978in}}%
\pgfpathlineto{\pgfqpoint{3.570381in}{2.411215in}}%
\pgfpathlineto{\pgfqpoint{3.570535in}{2.728632in}}%
\pgfpathlineto{\pgfqpoint{3.570800in}{2.635319in}}%
\pgfpathlineto{\pgfqpoint{3.571086in}{2.481036in}}%
\pgfpathlineto{\pgfqpoint{3.571196in}{2.730053in}}%
\pgfpathlineto{\pgfqpoint{3.571703in}{2.563532in}}%
\pgfpathlineto{\pgfqpoint{3.572541in}{2.741526in}}%
\pgfpathlineto{\pgfqpoint{3.572321in}{2.439405in}}%
\pgfpathlineto{\pgfqpoint{3.572806in}{2.693012in}}%
\pgfpathlineto{\pgfqpoint{3.573114in}{2.427495in}}%
\pgfpathlineto{\pgfqpoint{3.572960in}{2.721311in}}%
\pgfpathlineto{\pgfqpoint{3.573930in}{2.584620in}}%
\pgfpathlineto{\pgfqpoint{3.574106in}{2.381494in}}%
\pgfpathlineto{\pgfqpoint{3.574569in}{2.708418in}}%
\pgfpathlineto{\pgfqpoint{3.575010in}{2.545284in}}%
\pgfpathlineto{\pgfqpoint{3.575274in}{2.722404in}}%
\pgfpathlineto{\pgfqpoint{3.575781in}{2.445852in}}%
\pgfpathlineto{\pgfqpoint{3.576134in}{2.624939in}}%
\pgfpathlineto{\pgfqpoint{3.576266in}{2.384226in}}%
\pgfpathlineto{\pgfqpoint{3.576641in}{2.707107in}}%
\pgfpathlineto{\pgfqpoint{3.577324in}{2.457325in}}%
\pgfpathlineto{\pgfqpoint{3.578316in}{2.649961in}}%
\pgfpathlineto{\pgfqpoint{3.577919in}{2.389799in}}%
\pgfpathlineto{\pgfqpoint{3.578448in}{2.529768in}}%
\pgfpathlineto{\pgfqpoint{3.579065in}{2.407500in}}%
\pgfpathlineto{\pgfqpoint{3.578647in}{2.701753in}}%
\pgfpathlineto{\pgfqpoint{3.579462in}{2.532828in}}%
\pgfpathlineto{\pgfqpoint{3.579528in}{2.674109in}}%
\pgfpathlineto{\pgfqpoint{3.579925in}{2.365214in}}%
\pgfpathlineto{\pgfqpoint{3.580564in}{2.541351in}}%
\pgfpathlineto{\pgfqpoint{3.581446in}{2.694541in}}%
\pgfpathlineto{\pgfqpoint{3.581181in}{2.380402in}}%
\pgfpathlineto{\pgfqpoint{3.581490in}{2.584729in}}%
\pgfpathlineto{\pgfqpoint{3.582151in}{2.337570in}}%
\pgfpathlineto{\pgfqpoint{3.582438in}{2.628326in}}%
\pgfpathlineto{\pgfqpoint{3.582592in}{2.487155in}}%
\pgfpathlineto{\pgfqpoint{3.582724in}{2.695525in}}%
\pgfpathlineto{\pgfqpoint{3.583518in}{2.328501in}}%
\pgfpathlineto{\pgfqpoint{3.583650in}{2.364668in}}%
\pgfpathlineto{\pgfqpoint{3.583716in}{2.307959in}}%
\pgfpathlineto{\pgfqpoint{3.583760in}{2.512286in}}%
\pgfpathlineto{\pgfqpoint{3.583782in}{2.404768in}}%
\pgfpathlineto{\pgfqpoint{3.584863in}{2.643951in}}%
\pgfpathlineto{\pgfqpoint{3.584047in}{2.343251in}}%
\pgfpathlineto{\pgfqpoint{3.584907in}{2.495350in}}%
\pgfpathlineto{\pgfqpoint{3.585590in}{2.759118in}}%
\pgfpathlineto{\pgfqpoint{3.585854in}{2.447710in}}%
\pgfpathlineto{\pgfqpoint{3.586053in}{2.671486in}}%
\pgfpathlineto{\pgfqpoint{3.586934in}{2.375813in}}%
\pgfpathlineto{\pgfqpoint{3.586450in}{2.748082in}}%
\pgfpathlineto{\pgfqpoint{3.587177in}{2.590192in}}%
\pgfpathlineto{\pgfqpoint{3.587508in}{2.735735in}}%
\pgfpathlineto{\pgfqpoint{3.587728in}{2.414056in}}%
\pgfpathlineto{\pgfqpoint{3.588191in}{2.508899in}}%
\pgfpathlineto{\pgfqpoint{3.588279in}{2.435363in}}%
\pgfpathlineto{\pgfqpoint{3.589205in}{2.706779in}}%
\pgfpathlineto{\pgfqpoint{3.589227in}{2.542225in}}%
\pgfpathlineto{\pgfqpoint{3.589271in}{2.769061in}}%
\pgfpathlineto{\pgfqpoint{3.590109in}{2.372425in}}%
\pgfpathlineto{\pgfqpoint{3.590329in}{2.556538in}}%
\pgfpathlineto{\pgfqpoint{3.590351in}{2.420939in}}%
\pgfpathlineto{\pgfqpoint{3.590417in}{2.706561in}}%
\pgfpathlineto{\pgfqpoint{3.591453in}{2.470874in}}%
\pgfpathlineto{\pgfqpoint{3.591828in}{2.697601in}}%
\pgfpathlineto{\pgfqpoint{3.591806in}{2.405861in}}%
\pgfpathlineto{\pgfqpoint{3.592533in}{2.496005in}}%
\pgfpathlineto{\pgfqpoint{3.593062in}{2.330577in}}%
\pgfpathlineto{\pgfqpoint{3.593437in}{2.612701in}}%
\pgfpathlineto{\pgfqpoint{3.593635in}{2.469672in}}%
\pgfpathlineto{\pgfqpoint{3.593812in}{2.647229in}}%
\pgfpathlineto{\pgfqpoint{3.594296in}{2.393951in}}%
\pgfpathlineto{\pgfqpoint{3.594759in}{2.592268in}}%
\pgfpathlineto{\pgfqpoint{3.595112in}{2.450878in}}%
\pgfpathlineto{\pgfqpoint{3.594980in}{2.654441in}}%
\pgfpathlineto{\pgfqpoint{3.595839in}{2.621880in}}%
\pgfpathlineto{\pgfqpoint{3.595861in}{2.747972in}}%
\pgfpathlineto{\pgfqpoint{3.596853in}{2.392749in}}%
\pgfpathlineto{\pgfqpoint{3.596919in}{2.474261in}}%
\pgfpathlineto{\pgfqpoint{3.597096in}{2.678370in}}%
\pgfpathlineto{\pgfqpoint{3.597360in}{2.440170in}}%
\pgfpathlineto{\pgfqpoint{3.598044in}{2.580468in}}%
\pgfpathlineto{\pgfqpoint{3.598088in}{2.437329in}}%
\pgfpathlineto{\pgfqpoint{3.598683in}{2.811019in}}%
\pgfpathlineto{\pgfqpoint{3.599146in}{2.603632in}}%
\pgfpathlineto{\pgfqpoint{3.599234in}{2.392858in}}%
\pgfpathlineto{\pgfqpoint{3.599961in}{2.631276in}}%
\pgfpathlineto{\pgfqpoint{3.600226in}{2.583309in}}%
\pgfpathlineto{\pgfqpoint{3.600975in}{2.692684in}}%
\pgfpathlineto{\pgfqpoint{3.600578in}{2.381167in}}%
\pgfpathlineto{\pgfqpoint{3.601306in}{2.554244in}}%
\pgfpathlineto{\pgfqpoint{3.601636in}{2.396355in}}%
\pgfpathlineto{\pgfqpoint{3.601879in}{2.681648in}}%
\pgfpathlineto{\pgfqpoint{3.602408in}{2.534030in}}%
\pgfpathlineto{\pgfqpoint{3.603135in}{2.714428in}}%
\pgfpathlineto{\pgfqpoint{3.602937in}{2.442356in}}%
\pgfpathlineto{\pgfqpoint{3.603510in}{2.589537in}}%
\pgfpathlineto{\pgfqpoint{3.603973in}{2.360515in}}%
\pgfpathlineto{\pgfqpoint{3.604237in}{2.636849in}}%
\pgfpathlineto{\pgfqpoint{3.604678in}{2.430992in}}%
\pgfpathlineto{\pgfqpoint{3.604899in}{2.685472in}}%
\pgfpathlineto{\pgfqpoint{3.605670in}{2.419628in}}%
\pgfpathlineto{\pgfqpoint{3.605824in}{2.598060in}}%
\pgfpathlineto{\pgfqpoint{3.605869in}{2.362264in}}%
\pgfpathlineto{\pgfqpoint{3.605935in}{2.699677in}}%
\pgfpathlineto{\pgfqpoint{3.606927in}{2.507150in}}%
\pgfpathlineto{\pgfqpoint{3.607566in}{2.674436in}}%
\pgfpathlineto{\pgfqpoint{3.607213in}{2.426512in}}%
\pgfpathlineto{\pgfqpoint{3.608051in}{2.519279in}}%
\pgfpathlineto{\pgfqpoint{3.609021in}{2.361936in}}%
\pgfpathlineto{\pgfqpoint{3.608866in}{2.618055in}}%
\pgfpathlineto{\pgfqpoint{3.609153in}{2.509008in}}%
\pgfpathlineto{\pgfqpoint{3.610012in}{2.679135in}}%
\pgfpathlineto{\pgfqpoint{3.609285in}{2.414602in}}%
\pgfpathlineto{\pgfqpoint{3.610255in}{2.450878in}}%
\pgfpathlineto{\pgfqpoint{3.611115in}{2.693558in}}%
\pgfpathlineto{\pgfqpoint{3.610674in}{2.376796in}}%
\pgfpathlineto{\pgfqpoint{3.611379in}{2.518405in}}%
\pgfpathlineto{\pgfqpoint{3.612173in}{2.823912in}}%
\pgfpathlineto{\pgfqpoint{3.611798in}{2.462133in}}%
\pgfpathlineto{\pgfqpoint{3.612481in}{2.518186in}}%
\pgfpathlineto{\pgfqpoint{3.613363in}{2.730599in}}%
\pgfpathlineto{\pgfqpoint{3.612569in}{2.458636in}}%
\pgfpathlineto{\pgfqpoint{3.613583in}{2.558942in}}%
\pgfpathlineto{\pgfqpoint{3.614597in}{2.436127in}}%
\pgfpathlineto{\pgfqpoint{3.614267in}{2.710166in}}%
\pgfpathlineto{\pgfqpoint{3.614707in}{2.473387in}}%
\pgfpathlineto{\pgfqpoint{3.615765in}{2.755075in}}%
\pgfpathlineto{\pgfqpoint{3.615479in}{2.400835in}}%
\pgfpathlineto{\pgfqpoint{3.615832in}{2.634664in}}%
\pgfpathlineto{\pgfqpoint{3.616118in}{2.483003in}}%
\pgfpathlineto{\pgfqpoint{3.616867in}{2.796705in}}%
\pgfpathlineto{\pgfqpoint{3.616890in}{2.705249in}}%
\pgfpathlineto{\pgfqpoint{3.616912in}{2.807959in}}%
\pgfpathlineto{\pgfqpoint{3.617066in}{2.499611in}}%
\pgfpathlineto{\pgfqpoint{3.617970in}{2.597185in}}%
\pgfpathlineto{\pgfqpoint{3.618653in}{2.772994in}}%
\pgfpathlineto{\pgfqpoint{3.618366in}{2.485625in}}%
\pgfpathlineto{\pgfqpoint{3.619094in}{2.621770in}}%
\pgfpathlineto{\pgfqpoint{3.619468in}{2.789712in}}%
\pgfpathlineto{\pgfqpoint{3.619380in}{2.496879in}}%
\pgfpathlineto{\pgfqpoint{3.619843in}{2.696617in}}%
\pgfpathlineto{\pgfqpoint{3.620879in}{2.462788in}}%
\pgfpathlineto{\pgfqpoint{3.620945in}{2.629310in}}%
\pgfpathlineto{\pgfqpoint{3.621210in}{2.805883in}}%
\pgfpathlineto{\pgfqpoint{3.621078in}{2.429681in}}%
\pgfpathlineto{\pgfqpoint{3.622025in}{2.591394in}}%
\pgfpathlineto{\pgfqpoint{3.622620in}{2.437111in}}%
\pgfpathlineto{\pgfqpoint{3.622819in}{2.772120in}}%
\pgfpathlineto{\pgfqpoint{3.623171in}{2.491635in}}%
\pgfpathlineto{\pgfqpoint{3.623745in}{2.779004in}}%
\pgfpathlineto{\pgfqpoint{3.623458in}{2.414711in}}%
\pgfpathlineto{\pgfqpoint{3.624296in}{2.537963in}}%
\pgfpathlineto{\pgfqpoint{3.624803in}{2.425419in}}%
\pgfpathlineto{\pgfqpoint{3.625155in}{2.691591in}}%
\pgfpathlineto{\pgfqpoint{3.625398in}{2.451534in}}%
\pgfpathlineto{\pgfqpoint{3.625530in}{2.609970in}}%
\pgfpathlineto{\pgfqpoint{3.625728in}{2.315170in}}%
\pgfpathlineto{\pgfqpoint{3.626500in}{2.419191in}}%
\pgfpathlineto{\pgfqpoint{3.626720in}{2.689515in}}%
\pgfpathlineto{\pgfqpoint{3.627404in}{2.337570in}}%
\pgfpathlineto{\pgfqpoint{3.627602in}{2.505621in}}%
\pgfpathlineto{\pgfqpoint{3.627646in}{2.402255in}}%
\pgfpathlineto{\pgfqpoint{3.628175in}{2.737374in}}%
\pgfpathlineto{\pgfqpoint{3.628682in}{2.594891in}}%
\pgfpathlineto{\pgfqpoint{3.629608in}{2.702190in}}%
\pgfpathlineto{\pgfqpoint{3.628726in}{2.356473in}}%
\pgfpathlineto{\pgfqpoint{3.629674in}{2.554899in}}%
\pgfpathlineto{\pgfqpoint{3.629916in}{2.749393in}}%
\pgfpathlineto{\pgfqpoint{3.630798in}{2.387286in}}%
\pgfpathlineto{\pgfqpoint{3.631746in}{2.738466in}}%
\pgfpathlineto{\pgfqpoint{3.631922in}{2.570306in}}%
\pgfpathlineto{\pgfqpoint{3.632848in}{2.457871in}}%
\pgfpathlineto{\pgfqpoint{3.632319in}{2.679026in}}%
\pgfpathlineto{\pgfqpoint{3.633024in}{2.559052in}}%
\pgfpathlineto{\pgfqpoint{3.633090in}{2.511302in}}%
\pgfpathlineto{\pgfqpoint{3.633068in}{2.633680in}}%
\pgfpathlineto{\pgfqpoint{3.633134in}{2.583636in}}%
\pgfpathlineto{\pgfqpoint{3.633157in}{2.717378in}}%
\pgfpathlineto{\pgfqpoint{3.634016in}{2.448474in}}%
\pgfpathlineto{\pgfqpoint{3.634215in}{2.526600in}}%
\pgfpathlineto{\pgfqpoint{3.634303in}{2.426075in}}%
\pgfpathlineto{\pgfqpoint{3.634655in}{2.695853in}}%
\pgfpathlineto{\pgfqpoint{3.635118in}{2.660778in}}%
\pgfpathlineto{\pgfqpoint{3.635140in}{2.695197in}}%
\pgfpathlineto{\pgfqpoint{3.635868in}{2.388269in}}%
\pgfpathlineto{\pgfqpoint{3.636132in}{2.544519in}}%
\pgfpathlineto{\pgfqpoint{3.636286in}{2.408374in}}%
\pgfpathlineto{\pgfqpoint{3.636882in}{2.677168in}}%
\pgfpathlineto{\pgfqpoint{3.637278in}{2.470328in}}%
\pgfpathlineto{\pgfqpoint{3.637587in}{2.414274in}}%
\pgfpathlineto{\pgfqpoint{3.638447in}{2.751687in}}%
\pgfpathlineto{\pgfqpoint{3.639240in}{2.299108in}}%
\pgfpathlineto{\pgfqpoint{3.639615in}{2.468470in}}%
\pgfpathlineto{\pgfqpoint{3.640254in}{2.641875in}}%
\pgfpathlineto{\pgfqpoint{3.640673in}{2.411324in}}%
\pgfpathlineto{\pgfqpoint{3.640717in}{2.441591in}}%
\pgfpathlineto{\pgfqpoint{3.640739in}{2.398977in}}%
\pgfpathlineto{\pgfqpoint{3.641246in}{2.668099in}}%
\pgfpathlineto{\pgfqpoint{3.641775in}{2.542115in}}%
\pgfpathlineto{\pgfqpoint{3.642348in}{2.751906in}}%
\pgfpathlineto{\pgfqpoint{3.642084in}{2.416678in}}%
\pgfpathlineto{\pgfqpoint{3.642921in}{2.725027in}}%
\pgfpathlineto{\pgfqpoint{3.643891in}{2.268186in}}%
\pgfpathlineto{\pgfqpoint{3.644067in}{2.443448in}}%
\pgfpathlineto{\pgfqpoint{3.644927in}{2.742291in}}%
\pgfpathlineto{\pgfqpoint{3.644508in}{2.364886in}}%
\pgfpathlineto{\pgfqpoint{3.645213in}{2.604069in}}%
\pgfpathlineto{\pgfqpoint{3.645280in}{2.493055in}}%
\pgfpathlineto{\pgfqpoint{3.645412in}{2.778348in}}%
\pgfpathlineto{\pgfqpoint{3.646271in}{2.549108in}}%
\pgfpathlineto{\pgfqpoint{3.646712in}{2.784686in}}%
\pgfpathlineto{\pgfqpoint{3.647374in}{2.700551in}}%
\pgfpathlineto{\pgfqpoint{3.647947in}{2.446726in}}%
\pgfpathlineto{\pgfqpoint{3.648101in}{2.804791in}}%
\pgfpathlineto{\pgfqpoint{3.648498in}{2.630074in}}%
\pgfpathlineto{\pgfqpoint{3.649468in}{2.866744in}}%
\pgfpathlineto{\pgfqpoint{3.648652in}{2.562111in}}%
\pgfpathlineto{\pgfqpoint{3.649622in}{2.682850in}}%
\pgfpathlineto{\pgfqpoint{3.650151in}{2.484969in}}%
\pgfpathlineto{\pgfqpoint{3.650239in}{2.775398in}}%
\pgfpathlineto{\pgfqpoint{3.650702in}{2.659248in}}%
\pgfpathlineto{\pgfqpoint{3.650724in}{2.781517in}}%
\pgfpathlineto{\pgfqpoint{3.651694in}{2.527583in}}%
\pgfpathlineto{\pgfqpoint{3.651782in}{2.655315in}}%
\pgfpathlineto{\pgfqpoint{3.652576in}{2.470655in}}%
\pgfpathlineto{\pgfqpoint{3.652135in}{2.759336in}}%
\pgfpathlineto{\pgfqpoint{3.652906in}{2.554899in}}%
\pgfpathlineto{\pgfqpoint{3.653589in}{2.407391in}}%
\pgfpathlineto{\pgfqpoint{3.653788in}{2.711587in}}%
\pgfpathlineto{\pgfqpoint{3.653898in}{2.591285in}}%
\pgfpathlineto{\pgfqpoint{3.654449in}{2.737374in}}%
\pgfpathlineto{\pgfqpoint{3.654118in}{2.467924in}}%
\pgfpathlineto{\pgfqpoint{3.655000in}{2.624174in}}%
\pgfpathlineto{\pgfqpoint{3.655970in}{2.497863in}}%
\pgfpathlineto{\pgfqpoint{3.655683in}{2.739231in}}%
\pgfpathlineto{\pgfqpoint{3.656102in}{2.629200in}}%
\pgfpathlineto{\pgfqpoint{3.656477in}{2.444869in}}%
\pgfpathlineto{\pgfqpoint{3.656146in}{2.722732in}}%
\pgfpathlineto{\pgfqpoint{3.657138in}{2.636303in}}%
\pgfpathlineto{\pgfqpoint{3.658218in}{2.738903in}}%
\pgfpathlineto{\pgfqpoint{3.657557in}{2.449458in}}%
\pgfpathlineto{\pgfqpoint{3.658262in}{2.703501in}}%
\pgfpathlineto{\pgfqpoint{3.659100in}{2.417771in}}%
\pgfpathlineto{\pgfqpoint{3.658461in}{2.743383in}}%
\pgfpathlineto{\pgfqpoint{3.659409in}{2.454593in}}%
\pgfpathlineto{\pgfqpoint{3.659871in}{2.710822in}}%
\pgfpathlineto{\pgfqpoint{3.660533in}{2.654441in}}%
\pgfpathlineto{\pgfqpoint{3.661106in}{2.464974in}}%
\pgfpathlineto{\pgfqpoint{3.661128in}{2.727103in}}%
\pgfpathlineto{\pgfqpoint{3.661613in}{2.659248in}}%
\pgfpathlineto{\pgfqpoint{3.661965in}{2.859861in}}%
\pgfpathlineto{\pgfqpoint{3.662142in}{2.454921in}}%
\pgfpathlineto{\pgfqpoint{3.662715in}{2.651709in}}%
\pgfpathlineto{\pgfqpoint{3.662913in}{2.485625in}}%
\pgfpathlineto{\pgfqpoint{3.663156in}{2.891001in}}%
\pgfpathlineto{\pgfqpoint{3.663641in}{2.752999in}}%
\pgfpathlineto{\pgfqpoint{3.663663in}{2.853523in}}%
\pgfpathlineto{\pgfqpoint{3.664236in}{2.566045in}}%
\pgfpathlineto{\pgfqpoint{3.664721in}{2.749939in}}%
\pgfpathlineto{\pgfqpoint{3.665007in}{2.608221in}}%
\pgfpathlineto{\pgfqpoint{3.665779in}{2.885320in}}%
\pgfpathlineto{\pgfqpoint{3.665823in}{2.690826in}}%
\pgfpathlineto{\pgfqpoint{3.665889in}{2.875704in}}%
\pgfpathlineto{\pgfqpoint{3.666462in}{2.587789in}}%
\pgfpathlineto{\pgfqpoint{3.666925in}{2.666460in}}%
\pgfpathlineto{\pgfqpoint{3.666991in}{2.856364in}}%
\pgfpathlineto{\pgfqpoint{3.667784in}{2.504856in}}%
\pgfpathlineto{\pgfqpoint{3.667917in}{2.623300in}}%
\pgfpathlineto{\pgfqpoint{3.668137in}{2.491853in}}%
\pgfpathlineto{\pgfqpoint{3.668820in}{2.713444in}}%
\pgfpathlineto{\pgfqpoint{3.668997in}{2.642749in}}%
\pgfpathlineto{\pgfqpoint{3.669261in}{2.749284in}}%
\pgfpathlineto{\pgfqpoint{3.669724in}{2.496879in}}%
\pgfpathlineto{\pgfqpoint{3.670099in}{2.628654in}}%
\pgfpathlineto{\pgfqpoint{3.670826in}{2.513160in}}%
\pgfpathlineto{\pgfqpoint{3.670297in}{2.756714in}}%
\pgfpathlineto{\pgfqpoint{3.671069in}{2.605490in}}%
\pgfpathlineto{\pgfqpoint{3.671906in}{2.784249in}}%
\pgfpathlineto{\pgfqpoint{3.672061in}{2.449458in}}%
\pgfpathlineto{\pgfqpoint{3.672171in}{2.736062in}}%
\pgfpathlineto{\pgfqpoint{3.673097in}{2.408155in}}%
\pgfpathlineto{\pgfqpoint{3.672325in}{2.757588in}}%
\pgfpathlineto{\pgfqpoint{3.673295in}{2.634773in}}%
\pgfpathlineto{\pgfqpoint{3.673824in}{2.528239in}}%
\pgfpathlineto{\pgfqpoint{3.674044in}{2.740652in}}%
\pgfpathlineto{\pgfqpoint{3.674375in}{2.588226in}}%
\pgfpathlineto{\pgfqpoint{3.675169in}{2.827518in}}%
\pgfpathlineto{\pgfqpoint{3.675411in}{2.522448in}}%
\pgfpathlineto{\pgfqpoint{3.675477in}{2.626250in}}%
\pgfpathlineto{\pgfqpoint{3.676359in}{2.781845in}}%
\pgfpathlineto{\pgfqpoint{3.676006in}{2.541023in}}%
\pgfpathlineto{\pgfqpoint{3.676381in}{2.622972in}}%
\pgfpathlineto{\pgfqpoint{3.676844in}{2.561237in}}%
\pgfpathlineto{\pgfqpoint{3.677218in}{2.802605in}}%
\pgfpathlineto{\pgfqpoint{3.677461in}{2.673562in}}%
\pgfpathlineto{\pgfqpoint{3.677681in}{2.778021in}}%
\pgfpathlineto{\pgfqpoint{3.678166in}{2.455140in}}%
\pgfpathlineto{\pgfqpoint{3.678497in}{2.702081in}}%
\pgfpathlineto{\pgfqpoint{3.678717in}{2.460275in}}%
\pgfpathlineto{\pgfqpoint{3.678938in}{2.741853in}}%
\pgfpathlineto{\pgfqpoint{3.679599in}{2.626250in}}%
\pgfpathlineto{\pgfqpoint{3.680172in}{2.475682in}}%
\pgfpathlineto{\pgfqpoint{3.680745in}{2.753982in}}%
\pgfpathlineto{\pgfqpoint{3.681583in}{2.409794in}}%
\pgfpathlineto{\pgfqpoint{3.681957in}{2.566372in}}%
\pgfpathlineto{\pgfqpoint{3.682156in}{2.718252in}}%
\pgfpathlineto{\pgfqpoint{3.682376in}{2.430227in}}%
\pgfpathlineto{\pgfqpoint{3.683038in}{2.574021in}}%
\pgfpathlineto{\pgfqpoint{3.683500in}{2.360078in}}%
\pgfpathlineto{\pgfqpoint{3.683941in}{2.667006in}}%
\pgfpathlineto{\pgfqpoint{3.684162in}{2.514253in}}%
\pgfpathlineto{\pgfqpoint{3.684404in}{2.669410in}}%
\pgfpathlineto{\pgfqpoint{3.684889in}{2.418208in}}%
\pgfpathlineto{\pgfqpoint{3.685308in}{2.629200in}}%
\pgfpathlineto{\pgfqpoint{3.686322in}{2.377124in}}%
\pgfpathlineto{\pgfqpoint{3.685859in}{2.666569in}}%
\pgfpathlineto{\pgfqpoint{3.686388in}{2.446398in}}%
\pgfpathlineto{\pgfqpoint{3.686697in}{2.733659in}}%
\pgfpathlineto{\pgfqpoint{3.687402in}{2.412307in}}%
\pgfpathlineto{\pgfqpoint{3.687490in}{2.493492in}}%
\pgfpathlineto{\pgfqpoint{3.687512in}{2.356691in}}%
\pgfpathlineto{\pgfqpoint{3.688129in}{2.731692in}}%
\pgfpathlineto{\pgfqpoint{3.688592in}{2.491853in}}%
\pgfpathlineto{\pgfqpoint{3.688791in}{2.737701in}}%
\pgfpathlineto{\pgfqpoint{3.689297in}{2.482675in}}%
\pgfpathlineto{\pgfqpoint{3.689893in}{2.585494in}}%
\pgfpathlineto{\pgfqpoint{3.690267in}{2.523103in}}%
\pgfpathlineto{\pgfqpoint{3.689959in}{2.729288in}}%
\pgfpathlineto{\pgfqpoint{3.690951in}{2.647229in}}%
\pgfpathlineto{\pgfqpoint{3.691061in}{2.759227in}}%
\pgfpathlineto{\pgfqpoint{3.691303in}{2.472185in}}%
\pgfpathlineto{\pgfqpoint{3.692075in}{2.729616in}}%
\pgfpathlineto{\pgfqpoint{3.692494in}{2.465083in}}%
\pgfpathlineto{\pgfqpoint{3.693221in}{2.565280in}}%
\pgfpathlineto{\pgfqpoint{3.693463in}{2.743383in}}%
\pgfpathlineto{\pgfqpoint{3.693287in}{2.409248in}}%
\pgfpathlineto{\pgfqpoint{3.694411in}{2.686565in}}%
\pgfpathlineto{\pgfqpoint{3.694654in}{2.845438in}}%
\pgfpathlineto{\pgfqpoint{3.695557in}{2.429025in}}%
\pgfpathlineto{\pgfqpoint{3.696351in}{2.734314in}}%
\pgfpathlineto{\pgfqpoint{3.695668in}{2.389252in}}%
\pgfpathlineto{\pgfqpoint{3.696704in}{2.671814in}}%
\pgfpathlineto{\pgfqpoint{3.697255in}{2.492509in}}%
\pgfpathlineto{\pgfqpoint{3.697188in}{2.726338in}}%
\pgfpathlineto{\pgfqpoint{3.697806in}{2.656408in}}%
\pgfpathlineto{\pgfqpoint{3.698599in}{2.757369in}}%
\pgfpathlineto{\pgfqpoint{3.698269in}{2.479725in}}%
\pgfpathlineto{\pgfqpoint{3.698908in}{2.703064in}}%
\pgfpathlineto{\pgfqpoint{3.699282in}{2.482893in}}%
\pgfpathlineto{\pgfqpoint{3.699216in}{2.817903in}}%
\pgfpathlineto{\pgfqpoint{3.700054in}{2.622972in}}%
\pgfpathlineto{\pgfqpoint{3.700142in}{2.773104in}}%
\pgfpathlineto{\pgfqpoint{3.700230in}{2.550420in}}%
\pgfpathlineto{\pgfqpoint{3.701112in}{2.606801in}}%
\pgfpathlineto{\pgfqpoint{3.701244in}{2.457762in}}%
\pgfpathlineto{\pgfqpoint{3.701531in}{2.759992in}}%
\pgfpathlineto{\pgfqpoint{3.702192in}{2.525507in}}%
\pgfpathlineto{\pgfqpoint{3.702589in}{2.504965in}}%
\pgfpathlineto{\pgfqpoint{3.703338in}{2.845656in}}%
\pgfpathlineto{\pgfqpoint{3.703713in}{2.539056in}}%
\pgfpathlineto{\pgfqpoint{3.704484in}{2.634445in}}%
\pgfpathlineto{\pgfqpoint{3.704595in}{2.831779in}}%
\pgfpathlineto{\pgfqpoint{3.704947in}{2.480380in}}%
\pgfpathlineto{\pgfqpoint{3.705586in}{2.588116in}}%
\pgfpathlineto{\pgfqpoint{3.706314in}{2.440170in}}%
\pgfpathlineto{\pgfqpoint{3.706138in}{2.746552in}}%
\pgfpathlineto{\pgfqpoint{3.706534in}{2.625595in}}%
\pgfpathlineto{\pgfqpoint{3.707262in}{2.766985in}}%
\pgfpathlineto{\pgfqpoint{3.706843in}{2.534467in}}%
\pgfpathlineto{\pgfqpoint{3.707658in}{2.728414in}}%
\pgfpathlineto{\pgfqpoint{3.708011in}{2.525616in}}%
\pgfpathlineto{\pgfqpoint{3.707725in}{2.832544in}}%
\pgfpathlineto{\pgfqpoint{3.708716in}{2.690608in}}%
\pgfpathlineto{\pgfqpoint{3.708827in}{2.881058in}}%
\pgfpathlineto{\pgfqpoint{3.709268in}{2.514908in}}%
\pgfpathlineto{\pgfqpoint{3.709797in}{2.700332in}}%
\pgfpathlineto{\pgfqpoint{3.710458in}{2.502889in}}%
\pgfpathlineto{\pgfqpoint{3.710171in}{2.825988in}}%
\pgfpathlineto{\pgfqpoint{3.710899in}{2.541023in}}%
\pgfpathlineto{\pgfqpoint{3.711163in}{2.662854in}}%
\pgfpathlineto{\pgfqpoint{3.711428in}{2.255292in}}%
\pgfpathlineto{\pgfqpoint{3.711979in}{2.450878in}}%
\pgfpathlineto{\pgfqpoint{3.712001in}{2.424764in}}%
\pgfpathlineto{\pgfqpoint{3.712508in}{2.658484in}}%
\pgfpathlineto{\pgfqpoint{3.712926in}{2.634664in}}%
\pgfpathlineto{\pgfqpoint{3.713852in}{2.740870in}}%
\pgfpathlineto{\pgfqpoint{3.713169in}{2.371661in}}%
\pgfpathlineto{\pgfqpoint{3.713984in}{2.634554in}}%
\pgfpathlineto{\pgfqpoint{3.714271in}{2.411215in}}%
\pgfpathlineto{\pgfqpoint{3.714668in}{2.708200in}}%
\pgfpathlineto{\pgfqpoint{3.715087in}{2.617400in}}%
\pgfpathlineto{\pgfqpoint{3.715726in}{2.718471in}}%
\pgfpathlineto{\pgfqpoint{3.715858in}{2.463881in}}%
\pgfpathlineto{\pgfqpoint{3.716189in}{2.711478in}}%
\pgfpathlineto{\pgfqpoint{3.717026in}{2.479725in}}%
\pgfpathlineto{\pgfqpoint{3.716630in}{2.770809in}}%
\pgfpathlineto{\pgfqpoint{3.717335in}{2.549655in}}%
\pgfpathlineto{\pgfqpoint{3.717820in}{2.802933in}}%
\pgfpathlineto{\pgfqpoint{3.717930in}{2.504419in}}%
\pgfpathlineto{\pgfqpoint{3.718459in}{2.606364in}}%
\pgfpathlineto{\pgfqpoint{3.718481in}{2.605490in}}%
\pgfpathlineto{\pgfqpoint{3.718966in}{2.811565in}}%
\pgfpathlineto{\pgfqpoint{3.719164in}{2.456560in}}%
\pgfpathlineto{\pgfqpoint{3.719583in}{2.554353in}}%
\pgfpathlineto{\pgfqpoint{3.720090in}{2.777474in}}%
\pgfpathlineto{\pgfqpoint{3.720421in}{2.489886in}}%
\pgfpathlineto{\pgfqpoint{3.720685in}{2.559489in}}%
\pgfpathlineto{\pgfqpoint{3.721280in}{2.739668in}}%
\pgfpathlineto{\pgfqpoint{3.721589in}{2.482456in}}%
\pgfpathlineto{\pgfqpoint{3.721809in}{2.631167in}}%
\pgfpathlineto{\pgfqpoint{3.722515in}{2.415476in}}%
\pgfpathlineto{\pgfqpoint{3.722228in}{2.698475in}}%
\pgfpathlineto{\pgfqpoint{3.722912in}{2.572273in}}%
\pgfpathlineto{\pgfqpoint{3.722934in}{2.674436in}}%
\pgfpathlineto{\pgfqpoint{3.723705in}{2.457544in}}%
\pgfpathlineto{\pgfqpoint{3.724014in}{2.613029in}}%
\pgfpathlineto{\pgfqpoint{3.724454in}{2.532063in}}%
\pgfpathlineto{\pgfqpoint{3.724653in}{2.742618in}}%
\pgfpathlineto{\pgfqpoint{3.724873in}{2.702846in}}%
\pgfpathlineto{\pgfqpoint{3.724895in}{2.821508in}}%
\pgfpathlineto{\pgfqpoint{3.725799in}{2.498628in}}%
\pgfpathlineto{\pgfqpoint{3.725975in}{2.687439in}}%
\pgfpathlineto{\pgfqpoint{3.726835in}{2.774961in}}%
\pgfpathlineto{\pgfqpoint{3.726284in}{2.516547in}}%
\pgfpathlineto{\pgfqpoint{3.726989in}{2.666023in}}%
\pgfpathlineto{\pgfqpoint{3.727188in}{2.524087in}}%
\pgfpathlineto{\pgfqpoint{3.727099in}{2.824896in}}%
\pgfpathlineto{\pgfqpoint{3.728069in}{2.724808in}}%
\pgfpathlineto{\pgfqpoint{3.728113in}{2.850573in}}%
\pgfpathlineto{\pgfqpoint{3.728312in}{2.534904in}}%
\pgfpathlineto{\pgfqpoint{3.729061in}{2.694214in}}%
\pgfpathlineto{\pgfqpoint{3.729480in}{2.524742in}}%
\pgfpathlineto{\pgfqpoint{3.729811in}{2.832216in}}%
\pgfpathlineto{\pgfqpoint{3.730141in}{2.668973in}}%
\pgfpathlineto{\pgfqpoint{3.730979in}{2.824677in}}%
\pgfpathlineto{\pgfqpoint{3.730648in}{2.496661in}}%
\pgfpathlineto{\pgfqpoint{3.731221in}{2.762942in}}%
\pgfpathlineto{\pgfqpoint{3.731993in}{2.470000in}}%
\pgfpathlineto{\pgfqpoint{3.731816in}{2.858768in}}%
\pgfpathlineto{\pgfqpoint{3.732323in}{2.692575in}}%
\pgfpathlineto{\pgfqpoint{3.732456in}{2.857457in}}%
\pgfpathlineto{\pgfqpoint{3.732654in}{2.562111in}}%
\pgfpathlineto{\pgfqpoint{3.733448in}{2.839756in}}%
\pgfpathlineto{\pgfqpoint{3.734021in}{2.485734in}}%
\pgfpathlineto{\pgfqpoint{3.734594in}{2.692137in}}%
\pgfpathlineto{\pgfqpoint{3.734924in}{2.850573in}}%
\pgfpathlineto{\pgfqpoint{3.735013in}{2.579266in}}%
\pgfpathlineto{\pgfqpoint{3.735652in}{2.630293in}}%
\pgfpathlineto{\pgfqpoint{3.736071in}{2.463007in}}%
\pgfpathlineto{\pgfqpoint{3.736269in}{2.773322in}}%
\pgfpathlineto{\pgfqpoint{3.736732in}{2.765564in}}%
\pgfpathlineto{\pgfqpoint{3.737591in}{2.421814in}}%
\pgfpathlineto{\pgfqpoint{3.736952in}{2.782063in}}%
\pgfpathlineto{\pgfqpoint{3.737988in}{2.451315in}}%
\pgfpathlineto{\pgfqpoint{3.738804in}{2.733331in}}%
\pgfpathlineto{\pgfqpoint{3.738914in}{2.416678in}}%
\pgfpathlineto{\pgfqpoint{3.739112in}{2.623628in}}%
\pgfpathlineto{\pgfqpoint{3.739906in}{2.794847in}}%
\pgfpathlineto{\pgfqpoint{3.739796in}{2.396355in}}%
\pgfpathlineto{\pgfqpoint{3.740148in}{2.568776in}}%
\pgfpathlineto{\pgfqpoint{3.740259in}{2.461259in}}%
\pgfpathlineto{\pgfqpoint{3.740788in}{2.809161in}}%
\pgfpathlineto{\pgfqpoint{3.741228in}{2.664602in}}%
\pgfpathlineto{\pgfqpoint{3.741625in}{2.610953in}}%
\pgfpathlineto{\pgfqpoint{3.741956in}{2.832981in}}%
\pgfpathlineto{\pgfqpoint{3.742242in}{2.782391in}}%
\pgfpathlineto{\pgfqpoint{3.742264in}{2.783702in}}%
\pgfpathlineto{\pgfqpoint{3.742727in}{2.534795in}}%
\pgfpathlineto{\pgfqpoint{3.742904in}{2.848606in}}%
\pgfpathlineto{\pgfqpoint{3.743389in}{2.664930in}}%
\pgfpathlineto{\pgfqpoint{3.744336in}{2.868274in}}%
\pgfpathlineto{\pgfqpoint{3.743719in}{2.570525in}}%
\pgfpathlineto{\pgfqpoint{3.744491in}{2.715302in}}%
\pgfpathlineto{\pgfqpoint{3.745284in}{2.522557in}}%
\pgfpathlineto{\pgfqpoint{3.744601in}{2.810363in}}%
\pgfpathlineto{\pgfqpoint{3.745615in}{2.672142in}}%
\pgfpathlineto{\pgfqpoint{3.745681in}{2.787745in}}%
\pgfpathlineto{\pgfqpoint{3.745835in}{2.421158in}}%
\pgfpathlineto{\pgfqpoint{3.746651in}{2.640673in}}%
\pgfpathlineto{\pgfqpoint{3.746937in}{2.538182in}}%
\pgfpathlineto{\pgfqpoint{3.747510in}{2.739013in}}%
\pgfpathlineto{\pgfqpoint{3.747731in}{2.627124in}}%
\pgfpathlineto{\pgfqpoint{3.748789in}{2.799764in}}%
\pgfpathlineto{\pgfqpoint{3.748480in}{2.496114in}}%
\pgfpathlineto{\pgfqpoint{3.748855in}{2.664821in}}%
\pgfpathlineto{\pgfqpoint{3.749362in}{2.508571in}}%
\pgfpathlineto{\pgfqpoint{3.749737in}{2.792662in}}%
\pgfpathlineto{\pgfqpoint{3.749935in}{2.666679in}}%
\pgfpathlineto{\pgfqpoint{3.750244in}{2.924655in}}%
\pgfpathlineto{\pgfqpoint{3.750133in}{2.569760in}}%
\pgfpathlineto{\pgfqpoint{3.751015in}{2.728851in}}%
\pgfpathlineto{\pgfqpoint{3.751258in}{2.567356in}}%
\pgfpathlineto{\pgfqpoint{3.751390in}{2.843908in}}%
\pgfpathlineto{\pgfqpoint{3.752117in}{2.714428in}}%
\pgfpathlineto{\pgfqpoint{3.753131in}{2.540804in}}%
\pgfpathlineto{\pgfqpoint{3.752382in}{2.854834in}}%
\pgfpathlineto{\pgfqpoint{3.753263in}{2.590302in}}%
\pgfpathlineto{\pgfqpoint{3.753528in}{2.772011in}}%
\pgfpathlineto{\pgfqpoint{3.753903in}{2.515236in}}%
\pgfpathlineto{\pgfqpoint{3.754365in}{2.586805in}}%
\pgfpathlineto{\pgfqpoint{3.755445in}{2.820743in}}%
\pgfpathlineto{\pgfqpoint{3.755335in}{2.525507in}}%
\pgfpathlineto{\pgfqpoint{3.755534in}{2.669192in}}%
\pgfpathlineto{\pgfqpoint{3.756526in}{2.779441in}}%
\pgfpathlineto{\pgfqpoint{3.756261in}{2.516110in}}%
\pgfpathlineto{\pgfqpoint{3.756570in}{2.682850in}}%
\pgfpathlineto{\pgfqpoint{3.756680in}{2.534139in}}%
\pgfpathlineto{\pgfqpoint{3.756922in}{2.821727in}}%
\pgfpathlineto{\pgfqpoint{3.757672in}{2.576206in}}%
\pgfpathlineto{\pgfqpoint{3.758113in}{2.797579in}}%
\pgfpathlineto{\pgfqpoint{3.758642in}{2.558615in}}%
\pgfpathlineto{\pgfqpoint{3.758840in}{2.721311in}}%
\pgfpathlineto{\pgfqpoint{3.759281in}{2.552168in}}%
\pgfpathlineto{\pgfqpoint{3.759678in}{2.824240in}}%
\pgfpathlineto{\pgfqpoint{3.759942in}{2.731692in}}%
\pgfpathlineto{\pgfqpoint{3.760118in}{2.855818in}}%
\pgfpathlineto{\pgfqpoint{3.760736in}{2.587242in}}%
\pgfpathlineto{\pgfqpoint{3.761022in}{2.693558in}}%
\pgfpathlineto{\pgfqpoint{3.761198in}{2.536324in}}%
\pgfpathlineto{\pgfqpoint{3.761816in}{2.883899in}}%
\pgfpathlineto{\pgfqpoint{3.762146in}{2.663947in}}%
\pgfpathlineto{\pgfqpoint{3.762631in}{2.851556in}}%
\pgfpathlineto{\pgfqpoint{3.762719in}{2.531080in}}%
\pgfpathlineto{\pgfqpoint{3.763270in}{2.824131in}}%
\pgfpathlineto{\pgfqpoint{3.763755in}{2.484204in}}%
\pgfpathlineto{\pgfqpoint{3.763667in}{2.825114in}}%
\pgfpathlineto{\pgfqpoint{3.764395in}{2.696399in}}%
\pgfpathlineto{\pgfqpoint{3.765100in}{2.494038in}}%
\pgfpathlineto{\pgfqpoint{3.764857in}{2.768842in}}%
\pgfpathlineto{\pgfqpoint{3.765585in}{2.600573in}}%
\pgfpathlineto{\pgfqpoint{3.766026in}{2.791242in}}%
\pgfpathlineto{\pgfqpoint{3.765915in}{2.522994in}}%
\pgfpathlineto{\pgfqpoint{3.766709in}{2.675638in}}%
\pgfpathlineto{\pgfqpoint{3.767789in}{2.403785in}}%
\pgfpathlineto{\pgfqpoint{3.766797in}{2.740870in}}%
\pgfpathlineto{\pgfqpoint{3.767877in}{2.490433in}}%
\pgfpathlineto{\pgfqpoint{3.768847in}{2.659576in}}%
\pgfpathlineto{\pgfqpoint{3.768516in}{2.373737in}}%
\pgfpathlineto{\pgfqpoint{3.769001in}{2.553698in}}%
\pgfpathlineto{\pgfqpoint{3.769618in}{2.784904in}}%
\pgfpathlineto{\pgfqpoint{3.769045in}{2.460275in}}%
\pgfpathlineto{\pgfqpoint{3.770081in}{2.761740in}}%
\pgfpathlineto{\pgfqpoint{3.770103in}{2.539056in}}%
\pgfpathlineto{\pgfqpoint{3.771007in}{2.835167in}}%
\pgfpathlineto{\pgfqpoint{3.771183in}{2.671814in}}%
\pgfpathlineto{\pgfqpoint{3.771823in}{2.818340in}}%
\pgfpathlineto{\pgfqpoint{3.771338in}{2.510647in}}%
\pgfpathlineto{\pgfqpoint{3.772043in}{2.669192in}}%
\pgfpathlineto{\pgfqpoint{3.772903in}{2.484642in}}%
\pgfpathlineto{\pgfqpoint{3.772131in}{2.787854in}}%
\pgfpathlineto{\pgfqpoint{3.773145in}{2.684489in}}%
\pgfpathlineto{\pgfqpoint{3.773762in}{2.392858in}}%
\pgfpathlineto{\pgfqpoint{3.773630in}{2.695088in}}%
\pgfpathlineto{\pgfqpoint{3.774335in}{2.489886in}}%
\pgfpathlineto{\pgfqpoint{3.774909in}{2.789603in}}%
\pgfpathlineto{\pgfqpoint{3.775173in}{2.477976in}}%
\pgfpathlineto{\pgfqpoint{3.775526in}{2.708200in}}%
\pgfpathlineto{\pgfqpoint{3.775834in}{2.489340in}}%
\pgfpathlineto{\pgfqpoint{3.775658in}{2.849480in}}%
\pgfpathlineto{\pgfqpoint{3.776628in}{2.628763in}}%
\pgfpathlineto{\pgfqpoint{3.777333in}{2.794629in}}%
\pgfpathlineto{\pgfqpoint{3.777465in}{2.562876in}}%
\pgfpathlineto{\pgfqpoint{3.777730in}{2.593470in}}%
\pgfpathlineto{\pgfqpoint{3.778039in}{2.794847in}}%
\pgfpathlineto{\pgfqpoint{3.778656in}{2.465411in}}%
\pgfpathlineto{\pgfqpoint{3.778876in}{2.730490in}}%
\pgfpathlineto{\pgfqpoint{3.779515in}{2.544956in}}%
\pgfpathlineto{\pgfqpoint{3.779626in}{2.778130in}}%
\pgfpathlineto{\pgfqpoint{3.780022in}{2.600136in}}%
\pgfpathlineto{\pgfqpoint{3.780044in}{2.541788in}}%
\pgfpathlineto{\pgfqpoint{3.780816in}{2.800092in}}%
\pgfpathlineto{\pgfqpoint{3.781058in}{2.705249in}}%
\pgfpathlineto{\pgfqpoint{3.781146in}{2.809598in}}%
\pgfpathlineto{\pgfqpoint{3.782028in}{2.609860in}}%
\pgfpathlineto{\pgfqpoint{3.782072in}{2.611718in}}%
\pgfpathlineto{\pgfqpoint{3.782094in}{2.496661in}}%
\pgfpathlineto{\pgfqpoint{3.782293in}{2.775070in}}%
\pgfpathlineto{\pgfqpoint{3.783152in}{2.660450in}}%
\pgfpathlineto{\pgfqpoint{3.783307in}{2.763597in}}%
\pgfpathlineto{\pgfqpoint{3.784078in}{2.462788in}}%
\pgfpathlineto{\pgfqpoint{3.784232in}{2.661543in}}%
\pgfpathlineto{\pgfqpoint{3.784497in}{2.509882in}}%
\pgfpathlineto{\pgfqpoint{3.784872in}{2.774852in}}%
\pgfpathlineto{\pgfqpoint{3.785356in}{2.609423in}}%
\pgfpathlineto{\pgfqpoint{3.786282in}{2.811565in}}%
\pgfpathlineto{\pgfqpoint{3.785885in}{2.489558in}}%
\pgfpathlineto{\pgfqpoint{3.786459in}{2.721967in}}%
\pgfpathlineto{\pgfqpoint{3.786789in}{2.563641in}}%
\pgfpathlineto{\pgfqpoint{3.786988in}{2.878436in}}%
\pgfpathlineto{\pgfqpoint{3.787561in}{2.602539in}}%
\pgfpathlineto{\pgfqpoint{3.788068in}{2.824459in}}%
\pgfpathlineto{\pgfqpoint{3.787737in}{2.518077in}}%
\pgfpathlineto{\pgfqpoint{3.788685in}{2.746006in}}%
\pgfpathlineto{\pgfqpoint{3.789015in}{2.503654in}}%
\pgfpathlineto{\pgfqpoint{3.789324in}{2.782937in}}%
\pgfpathlineto{\pgfqpoint{3.789787in}{2.665586in}}%
\pgfpathlineto{\pgfqpoint{3.790095in}{2.364449in}}%
\pgfpathlineto{\pgfqpoint{3.790889in}{2.720765in}}%
\pgfpathlineto{\pgfqpoint{3.790955in}{2.508789in}}%
\pgfpathlineto{\pgfqpoint{3.791396in}{2.758571in}}%
\pgfpathlineto{\pgfqpoint{3.792013in}{2.578064in}}%
\pgfpathlineto{\pgfqpoint{3.792079in}{2.474261in}}%
\pgfpathlineto{\pgfqpoint{3.792498in}{2.774087in}}%
\pgfpathlineto{\pgfqpoint{3.792652in}{2.653457in}}%
\pgfpathlineto{\pgfqpoint{3.793600in}{2.849043in}}%
\pgfpathlineto{\pgfqpoint{3.792785in}{2.574021in}}%
\pgfpathlineto{\pgfqpoint{3.793777in}{2.840521in}}%
\pgfpathlineto{\pgfqpoint{3.794239in}{2.582981in}}%
\pgfpathlineto{\pgfqpoint{3.794746in}{2.959074in}}%
\pgfpathlineto{\pgfqpoint{3.794879in}{2.718689in}}%
\pgfpathlineto{\pgfqpoint{3.795430in}{2.885320in}}%
\pgfpathlineto{\pgfqpoint{3.795319in}{2.631167in}}%
\pgfpathlineto{\pgfqpoint{3.795937in}{2.661652in}}%
\pgfpathlineto{\pgfqpoint{3.796289in}{2.764909in}}%
\pgfpathlineto{\pgfqpoint{3.796355in}{2.611936in}}%
\pgfpathlineto{\pgfqpoint{3.796664in}{2.491307in}}%
\pgfpathlineto{\pgfqpoint{3.796929in}{2.816701in}}%
\pgfpathlineto{\pgfqpoint{3.797435in}{2.574240in}}%
\pgfpathlineto{\pgfqpoint{3.798493in}{2.804135in}}%
\pgfpathlineto{\pgfqpoint{3.797942in}{2.496005in}}%
\pgfpathlineto{\pgfqpoint{3.798560in}{2.767312in}}%
\pgfpathlineto{\pgfqpoint{3.799375in}{2.462023in}}%
\pgfpathlineto{\pgfqpoint{3.799243in}{2.796596in}}%
\pgfpathlineto{\pgfqpoint{3.799684in}{2.651163in}}%
\pgfpathlineto{\pgfqpoint{3.799948in}{2.828064in}}%
\pgfpathlineto{\pgfqpoint{3.800654in}{2.510100in}}%
\pgfpathlineto{\pgfqpoint{3.800764in}{2.637614in}}%
\pgfpathlineto{\pgfqpoint{3.801050in}{2.473169in}}%
\pgfpathlineto{\pgfqpoint{3.801469in}{2.894279in}}%
\pgfpathlineto{\pgfqpoint{3.801844in}{2.619803in}}%
\pgfpathlineto{\pgfqpoint{3.801866in}{2.703610in}}%
\pgfpathlineto{\pgfqpoint{3.802219in}{2.417224in}}%
\pgfpathlineto{\pgfqpoint{3.802924in}{2.588116in}}%
\pgfpathlineto{\pgfqpoint{3.803497in}{2.486718in}}%
\pgfpathlineto{\pgfqpoint{3.803122in}{2.738685in}}%
\pgfpathlineto{\pgfqpoint{3.803673in}{2.573147in}}%
\pgfpathlineto{\pgfqpoint{3.803982in}{2.763270in}}%
\pgfpathlineto{\pgfqpoint{3.804401in}{2.502015in}}%
\pgfpathlineto{\pgfqpoint{3.804775in}{2.718798in}}%
\pgfpathlineto{\pgfqpoint{3.805327in}{2.464427in}}%
\pgfpathlineto{\pgfqpoint{3.805723in}{2.816154in}}%
\pgfpathlineto{\pgfqpoint{3.805833in}{2.683396in}}%
\pgfpathlineto{\pgfqpoint{3.806407in}{2.928917in}}%
\pgfpathlineto{\pgfqpoint{3.806120in}{2.629637in}}%
\pgfpathlineto{\pgfqpoint{3.806936in}{2.772557in}}%
\pgfpathlineto{\pgfqpoint{3.808060in}{2.554353in}}%
\pgfpathlineto{\pgfqpoint{3.807817in}{2.834183in}}%
\pgfpathlineto{\pgfqpoint{3.808082in}{2.648213in}}%
\pgfpathlineto{\pgfqpoint{3.808258in}{2.793318in}}%
\pgfpathlineto{\pgfqpoint{3.808677in}{2.401599in}}%
\pgfpathlineto{\pgfqpoint{3.809184in}{2.634664in}}%
\pgfpathlineto{\pgfqpoint{3.809867in}{2.410996in}}%
\pgfpathlineto{\pgfqpoint{3.809559in}{2.761303in}}%
\pgfpathlineto{\pgfqpoint{3.810220in}{2.618164in}}%
\pgfpathlineto{\pgfqpoint{3.810462in}{2.839756in}}%
\pgfpathlineto{\pgfqpoint{3.810947in}{2.565826in}}%
\pgfpathlineto{\pgfqpoint{3.811322in}{2.710603in}}%
\pgfpathlineto{\pgfqpoint{3.811851in}{2.407063in}}%
\pgfpathlineto{\pgfqpoint{3.811719in}{2.798781in}}%
\pgfpathlineto{\pgfqpoint{3.812468in}{2.544847in}}%
\pgfpathlineto{\pgfqpoint{3.812843in}{2.391328in}}%
\pgfpathlineto{\pgfqpoint{3.813328in}{2.664821in}}%
\pgfpathlineto{\pgfqpoint{3.813592in}{2.477648in}}%
\pgfpathlineto{\pgfqpoint{3.814672in}{2.711478in}}%
\pgfpathlineto{\pgfqpoint{3.813680in}{2.429244in}}%
\pgfpathlineto{\pgfqpoint{3.814716in}{2.644607in}}%
\pgfpathlineto{\pgfqpoint{3.815003in}{2.461477in}}%
\pgfpathlineto{\pgfqpoint{3.815576in}{2.710931in}}%
\pgfpathlineto{\pgfqpoint{3.815818in}{2.615105in}}%
\pgfpathlineto{\pgfqpoint{3.816348in}{2.514253in}}%
\pgfpathlineto{\pgfqpoint{3.816193in}{2.863904in}}%
\pgfpathlineto{\pgfqpoint{3.816899in}{2.652037in}}%
\pgfpathlineto{\pgfqpoint{3.817273in}{2.789930in}}%
\pgfpathlineto{\pgfqpoint{3.817163in}{2.543973in}}%
\pgfpathlineto{\pgfqpoint{3.817494in}{2.605380in}}%
\pgfpathlineto{\pgfqpoint{3.817538in}{2.496114in}}%
\pgfpathlineto{\pgfqpoint{3.817957in}{2.745787in}}%
\pgfpathlineto{\pgfqpoint{3.818508in}{2.700770in}}%
\pgfpathlineto{\pgfqpoint{3.819015in}{2.830250in}}%
\pgfpathlineto{\pgfqpoint{3.819279in}{2.513051in}}%
\pgfpathlineto{\pgfqpoint{3.819610in}{2.720656in}}%
\pgfpathlineto{\pgfqpoint{3.820006in}{2.502343in}}%
\pgfpathlineto{\pgfqpoint{3.820139in}{2.783156in}}%
\pgfpathlineto{\pgfqpoint{3.820734in}{2.670612in}}%
\pgfpathlineto{\pgfqpoint{3.821285in}{2.884664in}}%
\pgfpathlineto{\pgfqpoint{3.821549in}{2.534248in}}%
\pgfpathlineto{\pgfqpoint{3.821704in}{2.633243in}}%
\pgfpathlineto{\pgfqpoint{3.822784in}{2.465520in}}%
\pgfpathlineto{\pgfqpoint{3.822123in}{2.745896in}}%
\pgfpathlineto{\pgfqpoint{3.822806in}{2.557303in}}%
\pgfpathlineto{\pgfqpoint{3.822894in}{2.823038in}}%
\pgfpathlineto{\pgfqpoint{3.823158in}{2.514908in}}%
\pgfpathlineto{\pgfqpoint{3.823930in}{2.713881in}}%
\pgfpathlineto{\pgfqpoint{3.824261in}{2.803370in}}%
\pgfpathlineto{\pgfqpoint{3.825032in}{2.514253in}}%
\pgfpathlineto{\pgfqpoint{3.826090in}{2.866635in}}%
\pgfpathlineto{\pgfqpoint{3.826156in}{2.701534in}}%
\pgfpathlineto{\pgfqpoint{3.826178in}{2.539930in}}%
\pgfpathlineto{\pgfqpoint{3.826333in}{2.820306in}}%
\pgfpathlineto{\pgfqpoint{3.827258in}{2.656626in}}%
\pgfpathlineto{\pgfqpoint{3.828228in}{2.895809in}}%
\pgfpathlineto{\pgfqpoint{3.827677in}{2.469781in}}%
\pgfpathlineto{\pgfqpoint{3.828382in}{2.715302in}}%
\pgfpathlineto{\pgfqpoint{3.829286in}{2.479943in}}%
\pgfpathlineto{\pgfqpoint{3.828493in}{2.759555in}}%
\pgfpathlineto{\pgfqpoint{3.829529in}{2.527364in}}%
\pgfpathlineto{\pgfqpoint{3.830278in}{2.757369in}}%
\pgfpathlineto{\pgfqpoint{3.829859in}{2.492399in}}%
\pgfpathlineto{\pgfqpoint{3.830697in}{2.579594in}}%
\pgfpathlineto{\pgfqpoint{3.831623in}{2.472404in}}%
\pgfpathlineto{\pgfqpoint{3.831579in}{2.778348in}}%
\pgfpathlineto{\pgfqpoint{3.831733in}{2.650944in}}%
\pgfpathlineto{\pgfqpoint{3.832703in}{2.805556in}}%
\pgfpathlineto{\pgfqpoint{3.832262in}{2.514034in}}%
\pgfpathlineto{\pgfqpoint{3.832747in}{2.687439in}}%
\pgfpathlineto{\pgfqpoint{3.832769in}{2.460712in}}%
\pgfpathlineto{\pgfqpoint{3.833033in}{2.780206in}}%
\pgfpathlineto{\pgfqpoint{3.833849in}{2.611062in}}%
\pgfpathlineto{\pgfqpoint{3.834202in}{2.811674in}}%
\pgfpathlineto{\pgfqpoint{3.834753in}{2.440607in}}%
\pgfpathlineto{\pgfqpoint{3.834929in}{2.590083in}}%
\pgfpathlineto{\pgfqpoint{3.835370in}{2.500267in}}%
\pgfpathlineto{\pgfqpoint{3.835017in}{2.738685in}}%
\pgfpathlineto{\pgfqpoint{3.836031in}{2.500704in}}%
\pgfpathlineto{\pgfqpoint{3.837111in}{2.803479in}}%
\pgfpathlineto{\pgfqpoint{3.836273in}{2.490105in}}%
\pgfpathlineto{\pgfqpoint{3.837155in}{2.604506in}}%
\pgfpathlineto{\pgfqpoint{3.837971in}{2.507478in}}%
\pgfpathlineto{\pgfqpoint{3.837728in}{2.837243in}}%
\pgfpathlineto{\pgfqpoint{3.838081in}{2.638160in}}%
\pgfpathlineto{\pgfqpoint{3.838764in}{2.881277in}}%
\pgfpathlineto{\pgfqpoint{3.838191in}{2.554681in}}%
\pgfpathlineto{\pgfqpoint{3.839183in}{2.635866in}}%
\pgfpathlineto{\pgfqpoint{3.839403in}{2.494803in}}%
\pgfpathlineto{\pgfqpoint{3.839293in}{2.835167in}}%
\pgfpathlineto{\pgfqpoint{3.840241in}{2.581451in}}%
\pgfpathlineto{\pgfqpoint{3.840329in}{2.836587in}}%
\pgfpathlineto{\pgfqpoint{3.841365in}{2.680992in}}%
\pgfpathlineto{\pgfqpoint{3.841674in}{2.838991in}}%
\pgfpathlineto{\pgfqpoint{3.842115in}{2.576316in}}%
\pgfpathlineto{\pgfqpoint{3.842489in}{2.740214in}}%
\pgfpathlineto{\pgfqpoint{3.843393in}{2.453173in}}%
\pgfpathlineto{\pgfqpoint{3.843173in}{2.844454in}}%
\pgfpathlineto{\pgfqpoint{3.843503in}{2.670612in}}%
\pgfpathlineto{\pgfqpoint{3.844054in}{2.825223in}}%
\pgfpathlineto{\pgfqpoint{3.843900in}{2.533046in}}%
\pgfpathlineto{\pgfqpoint{3.844583in}{2.606473in}}%
\pgfpathlineto{\pgfqpoint{3.845223in}{2.541897in}}%
\pgfpathlineto{\pgfqpoint{3.845355in}{2.817684in}}%
\pgfpathlineto{\pgfqpoint{3.845377in}{2.806539in}}%
\pgfpathlineto{\pgfqpoint{3.845399in}{2.888925in}}%
\pgfpathlineto{\pgfqpoint{3.845950in}{2.582325in}}%
\pgfpathlineto{\pgfqpoint{3.846435in}{2.664165in}}%
\pgfpathlineto{\pgfqpoint{3.847317in}{2.856146in}}%
\pgfpathlineto{\pgfqpoint{3.846589in}{2.536215in}}%
\pgfpathlineto{\pgfqpoint{3.847537in}{2.764909in}}%
\pgfpathlineto{\pgfqpoint{3.847735in}{2.608877in}}%
\pgfpathlineto{\pgfqpoint{3.848154in}{2.964537in}}%
\pgfpathlineto{\pgfqpoint{3.848639in}{2.698366in}}%
\pgfpathlineto{\pgfqpoint{3.848992in}{2.923672in}}%
\pgfpathlineto{\pgfqpoint{3.848727in}{2.619913in}}%
\pgfpathlineto{\pgfqpoint{3.849763in}{2.770044in}}%
\pgfpathlineto{\pgfqpoint{3.849851in}{2.614122in}}%
\pgfpathlineto{\pgfqpoint{3.850843in}{2.861827in}}%
\pgfpathlineto{\pgfqpoint{3.850865in}{2.741853in}}%
\pgfpathlineto{\pgfqpoint{3.851218in}{2.844673in}}%
\pgfpathlineto{\pgfqpoint{3.851438in}{2.617837in}}%
\pgfpathlineto{\pgfqpoint{3.851967in}{2.799437in}}%
\pgfpathlineto{\pgfqpoint{3.852078in}{2.612920in}}%
\pgfpathlineto{\pgfqpoint{3.852607in}{2.902584in}}%
\pgfpathlineto{\pgfqpoint{3.853069in}{2.820962in}}%
\pgfpathlineto{\pgfqpoint{3.853092in}{2.964210in}}%
\pgfpathlineto{\pgfqpoint{3.853643in}{2.613794in}}%
\pgfpathlineto{\pgfqpoint{3.854150in}{2.755621in}}%
\pgfpathlineto{\pgfqpoint{3.854701in}{2.955031in}}%
\pgfpathlineto{\pgfqpoint{3.854436in}{2.655315in}}%
\pgfpathlineto{\pgfqpoint{3.854987in}{2.777365in}}%
\pgfpathlineto{\pgfqpoint{3.855274in}{2.623300in}}%
\pgfpathlineto{\pgfqpoint{3.855781in}{2.886303in}}%
\pgfpathlineto{\pgfqpoint{3.856111in}{2.687330in}}%
\pgfpathlineto{\pgfqpoint{3.856155in}{2.591067in}}%
\pgfpathlineto{\pgfqpoint{3.856795in}{2.945744in}}%
\pgfpathlineto{\pgfqpoint{3.857147in}{2.715083in}}%
\pgfpathlineto{\pgfqpoint{3.857742in}{2.941592in}}%
\pgfpathlineto{\pgfqpoint{3.858183in}{2.650398in}}%
\pgfpathlineto{\pgfqpoint{3.858249in}{2.717378in}}%
\pgfpathlineto{\pgfqpoint{3.858933in}{2.893187in}}%
\pgfpathlineto{\pgfqpoint{3.859241in}{2.644935in}}%
\pgfpathlineto{\pgfqpoint{3.859373in}{2.780206in}}%
\pgfpathlineto{\pgfqpoint{3.859880in}{2.609860in}}%
\pgfpathlineto{\pgfqpoint{3.859572in}{2.996771in}}%
\pgfpathlineto{\pgfqpoint{3.860476in}{2.778785in}}%
\pgfpathlineto{\pgfqpoint{3.860784in}{2.643842in}}%
\pgfpathlineto{\pgfqpoint{3.860828in}{2.874393in}}%
\pgfpathlineto{\pgfqpoint{3.861578in}{2.746989in}}%
\pgfpathlineto{\pgfqpoint{3.862415in}{2.902147in}}%
\pgfpathlineto{\pgfqpoint{3.862592in}{2.621115in}}%
\pgfpathlineto{\pgfqpoint{3.862636in}{2.722186in}}%
\pgfpathlineto{\pgfqpoint{3.862944in}{2.861500in}}%
\pgfpathlineto{\pgfqpoint{3.863738in}{2.609642in}}%
\pgfpathlineto{\pgfqpoint{3.864333in}{2.829266in}}%
\pgfpathlineto{\pgfqpoint{3.864201in}{2.584183in}}%
\pgfpathlineto{\pgfqpoint{3.864840in}{2.689952in}}%
\pgfpathlineto{\pgfqpoint{3.865435in}{2.568448in}}%
\pgfpathlineto{\pgfqpoint{3.864994in}{2.825660in}}%
\pgfpathlineto{\pgfqpoint{3.865920in}{2.747863in}}%
\pgfpathlineto{\pgfqpoint{3.865964in}{2.840084in}}%
\pgfpathlineto{\pgfqpoint{3.866052in}{2.509336in}}%
\pgfpathlineto{\pgfqpoint{3.866537in}{2.636849in}}%
\pgfpathlineto{\pgfqpoint{3.866559in}{2.434161in}}%
\pgfpathlineto{\pgfqpoint{3.867287in}{2.822492in}}%
\pgfpathlineto{\pgfqpoint{3.867639in}{2.662636in}}%
\pgfpathlineto{\pgfqpoint{3.868278in}{2.535887in}}%
\pgfpathlineto{\pgfqpoint{3.867882in}{2.796049in}}%
\pgfpathlineto{\pgfqpoint{3.868741in}{2.673344in}}%
\pgfpathlineto{\pgfqpoint{3.869028in}{2.895372in}}%
\pgfpathlineto{\pgfqpoint{3.869469in}{2.561455in}}%
\pgfpathlineto{\pgfqpoint{3.869865in}{2.775726in}}%
\pgfpathlineto{\pgfqpoint{3.870086in}{2.793973in}}%
\pgfpathlineto{\pgfqpoint{3.869910in}{2.593470in}}%
\pgfpathlineto{\pgfqpoint{3.870328in}{2.702081in}}%
\pgfpathlineto{\pgfqpoint{3.870394in}{2.599589in}}%
\pgfpathlineto{\pgfqpoint{3.871232in}{2.866307in}}%
\pgfpathlineto{\pgfqpoint{3.871430in}{2.666569in}}%
\pgfpathlineto{\pgfqpoint{3.871783in}{2.835604in}}%
\pgfpathlineto{\pgfqpoint{3.872466in}{2.630293in}}%
\pgfpathlineto{\pgfqpoint{3.872555in}{2.760538in}}%
\pgfpathlineto{\pgfqpoint{3.873106in}{2.590083in}}%
\pgfpathlineto{\pgfqpoint{3.873591in}{2.823366in}}%
\pgfpathlineto{\pgfqpoint{3.873701in}{2.640564in}}%
\pgfpathlineto{\pgfqpoint{3.874803in}{2.879747in}}%
\pgfpathlineto{\pgfqpoint{3.874296in}{2.557959in}}%
\pgfpathlineto{\pgfqpoint{3.874847in}{2.742728in}}%
\pgfpathlineto{\pgfqpoint{3.875486in}{2.567247in}}%
\pgfpathlineto{\pgfqpoint{3.875618in}{2.890237in}}%
\pgfpathlineto{\pgfqpoint{3.875927in}{2.773978in}}%
\pgfpathlineto{\pgfqpoint{3.876478in}{2.870678in}}%
\pgfpathlineto{\pgfqpoint{3.876588in}{2.589209in}}%
\pgfpathlineto{\pgfqpoint{3.876963in}{2.714537in}}%
\pgfpathlineto{\pgfqpoint{3.877756in}{2.857348in}}%
\pgfpathlineto{\pgfqpoint{3.878043in}{2.599589in}}%
\pgfpathlineto{\pgfqpoint{3.878770in}{2.552933in}}%
\pgfpathlineto{\pgfqpoint{3.879167in}{2.913510in}}%
\pgfpathlineto{\pgfqpoint{3.879366in}{2.578282in}}%
\pgfpathlineto{\pgfqpoint{3.880291in}{2.732347in}}%
\pgfpathlineto{\pgfqpoint{3.880666in}{2.853195in}}%
\pgfpathlineto{\pgfqpoint{3.881019in}{2.619694in}}%
\pgfpathlineto{\pgfqpoint{3.881393in}{2.738794in}}%
\pgfpathlineto{\pgfqpoint{3.881746in}{2.864450in}}%
\pgfpathlineto{\pgfqpoint{3.881438in}{2.667225in}}%
\pgfpathlineto{\pgfqpoint{3.881834in}{2.748082in}}%
\pgfpathlineto{\pgfqpoint{3.881856in}{2.548781in}}%
\pgfpathlineto{\pgfqpoint{3.881989in}{2.870787in}}%
\pgfpathlineto{\pgfqpoint{3.882914in}{2.806320in}}%
\pgfpathlineto{\pgfqpoint{3.883465in}{2.887833in}}%
\pgfpathlineto{\pgfqpoint{3.883113in}{2.582107in}}%
\pgfpathlineto{\pgfqpoint{3.883994in}{2.768951in}}%
\pgfpathlineto{\pgfqpoint{3.885052in}{2.613794in}}%
\pgfpathlineto{\pgfqpoint{3.884590in}{2.883353in}}%
\pgfpathlineto{\pgfqpoint{3.885229in}{2.626250in}}%
\pgfpathlineto{\pgfqpoint{3.886199in}{2.872754in}}%
\pgfpathlineto{\pgfqpoint{3.885890in}{2.605817in}}%
\pgfpathlineto{\pgfqpoint{3.886331in}{2.653348in}}%
\pgfpathlineto{\pgfqpoint{3.887345in}{2.838226in}}%
\pgfpathlineto{\pgfqpoint{3.886551in}{2.574458in}}%
\pgfpathlineto{\pgfqpoint{3.887477in}{2.732566in}}%
\pgfpathlineto{\pgfqpoint{3.888204in}{2.552386in}}%
\pgfpathlineto{\pgfqpoint{3.887852in}{2.918755in}}%
\pgfpathlineto{\pgfqpoint{3.888557in}{2.718252in}}%
\pgfpathlineto{\pgfqpoint{3.888623in}{2.897994in}}%
\pgfpathlineto{\pgfqpoint{3.889571in}{2.567356in}}%
\pgfpathlineto{\pgfqpoint{3.889637in}{2.690389in}}%
\pgfpathlineto{\pgfqpoint{3.889858in}{2.625922in}}%
\pgfpathlineto{\pgfqpoint{3.890409in}{2.887068in}}%
\pgfpathlineto{\pgfqpoint{3.890673in}{2.718580in}}%
\pgfpathlineto{\pgfqpoint{3.891004in}{2.903239in}}%
\pgfpathlineto{\pgfqpoint{3.891202in}{2.652474in}}%
\pgfpathlineto{\pgfqpoint{3.891753in}{2.709074in}}%
\pgfpathlineto{\pgfqpoint{3.892304in}{2.654987in}}%
\pgfpathlineto{\pgfqpoint{3.892569in}{2.909686in}}%
\pgfpathlineto{\pgfqpoint{3.892833in}{2.730271in}}%
\pgfpathlineto{\pgfqpoint{3.893186in}{2.932960in}}%
\pgfpathlineto{\pgfqpoint{3.893054in}{2.586259in}}%
\pgfpathlineto{\pgfqpoint{3.893935in}{2.783593in}}%
\pgfpathlineto{\pgfqpoint{3.894993in}{2.565826in}}%
\pgfpathlineto{\pgfqpoint{3.894046in}{2.831670in}}%
\pgfpathlineto{\pgfqpoint{3.895059in}{2.618055in}}%
\pgfpathlineto{\pgfqpoint{3.895104in}{2.752234in}}%
\pgfpathlineto{\pgfqpoint{3.895985in}{2.818449in}}%
\pgfpathlineto{\pgfqpoint{3.895280in}{2.547469in}}%
\pgfpathlineto{\pgfqpoint{3.896184in}{2.676950in}}%
\pgfpathlineto{\pgfqpoint{3.896514in}{2.502889in}}%
\pgfpathlineto{\pgfqpoint{3.897087in}{2.846749in}}%
\pgfpathlineto{\pgfqpoint{3.897242in}{2.671814in}}%
\pgfpathlineto{\pgfqpoint{3.897286in}{2.877562in}}%
\pgfpathlineto{\pgfqpoint{3.898167in}{2.528785in}}%
\pgfpathlineto{\pgfqpoint{3.898322in}{2.628326in}}%
\pgfpathlineto{\pgfqpoint{3.899225in}{2.523431in}}%
\pgfpathlineto{\pgfqpoint{3.898652in}{2.780971in}}%
\pgfpathlineto{\pgfqpoint{3.899424in}{2.649852in}}%
\pgfpathlineto{\pgfqpoint{3.900195in}{2.851666in}}%
\pgfpathlineto{\pgfqpoint{3.899578in}{2.529441in}}%
\pgfpathlineto{\pgfqpoint{3.900526in}{2.712679in}}%
\pgfpathlineto{\pgfqpoint{3.901319in}{2.445306in}}%
\pgfpathlineto{\pgfqpoint{3.901121in}{2.792881in}}%
\pgfpathlineto{\pgfqpoint{3.901628in}{2.643405in}}%
\pgfpathlineto{\pgfqpoint{3.902091in}{2.828611in}}%
\pgfpathlineto{\pgfqpoint{3.901738in}{2.479506in}}%
\pgfpathlineto{\pgfqpoint{3.902752in}{2.739996in}}%
\pgfpathlineto{\pgfqpoint{3.902862in}{2.567028in}}%
\pgfpathlineto{\pgfqpoint{3.903193in}{2.870787in}}%
\pgfpathlineto{\pgfqpoint{3.903854in}{2.704703in}}%
\pgfpathlineto{\pgfqpoint{3.904075in}{2.956452in}}%
\pgfpathlineto{\pgfqpoint{3.904317in}{2.568448in}}%
\pgfpathlineto{\pgfqpoint{3.904934in}{2.777911in}}%
\pgfpathlineto{\pgfqpoint{3.905044in}{2.640564in}}%
\pgfpathlineto{\pgfqpoint{3.905992in}{2.957544in}}%
\pgfpathlineto{\pgfqpoint{3.906014in}{2.897776in}}%
\pgfpathlineto{\pgfqpoint{3.906654in}{2.570962in}}%
\pgfpathlineto{\pgfqpoint{3.906279in}{2.916679in}}%
\pgfpathlineto{\pgfqpoint{3.907337in}{2.674546in}}%
\pgfpathlineto{\pgfqpoint{3.907535in}{2.815608in}}%
\pgfpathlineto{\pgfqpoint{3.907756in}{2.586150in}}%
\pgfpathlineto{\pgfqpoint{3.908042in}{2.632915in}}%
\pgfpathlineto{\pgfqpoint{3.908064in}{2.584948in}}%
\pgfpathlineto{\pgfqpoint{3.908351in}{2.927278in}}%
\pgfpathlineto{\pgfqpoint{3.909100in}{2.685800in}}%
\pgfpathlineto{\pgfqpoint{3.909188in}{2.901819in}}%
\pgfpathlineto{\pgfqpoint{3.910004in}{2.525616in}}%
\pgfpathlineto{\pgfqpoint{3.910180in}{2.640455in}}%
\pgfpathlineto{\pgfqpoint{3.910335in}{2.875376in}}%
\pgfpathlineto{\pgfqpoint{3.910577in}{2.601556in}}%
\pgfpathlineto{\pgfqpoint{3.910842in}{2.690826in}}%
\pgfpathlineto{\pgfqpoint{3.910864in}{2.490433in}}%
\pgfpathlineto{\pgfqpoint{3.911128in}{2.877562in}}%
\pgfpathlineto{\pgfqpoint{3.911944in}{2.666023in}}%
\pgfpathlineto{\pgfqpoint{3.912759in}{2.809598in}}%
\pgfpathlineto{\pgfqpoint{3.912340in}{2.557631in}}%
\pgfpathlineto{\pgfqpoint{3.913046in}{2.717487in}}%
\pgfpathlineto{\pgfqpoint{3.913465in}{2.581451in}}%
\pgfpathlineto{\pgfqpoint{3.913597in}{2.865324in}}%
\pgfpathlineto{\pgfqpoint{3.914170in}{2.649524in}}%
\pgfpathlineto{\pgfqpoint{3.914853in}{2.840193in}}%
\pgfpathlineto{\pgfqpoint{3.915184in}{2.523431in}}%
\pgfpathlineto{\pgfqpoint{3.915250in}{2.610079in}}%
\pgfpathlineto{\pgfqpoint{3.915603in}{2.431320in}}%
\pgfpathlineto{\pgfqpoint{3.916286in}{2.807959in}}%
\pgfpathlineto{\pgfqpoint{3.916308in}{2.755402in}}%
\pgfpathlineto{\pgfqpoint{3.917212in}{2.512395in}}%
\pgfpathlineto{\pgfqpoint{3.916815in}{2.815827in}}%
\pgfpathlineto{\pgfqpoint{3.917388in}{2.749065in}}%
\pgfpathlineto{\pgfqpoint{3.917697in}{2.844891in}}%
\pgfpathlineto{\pgfqpoint{3.918358in}{2.520918in}}%
\pgfpathlineto{\pgfqpoint{3.918380in}{2.522338in}}%
\pgfpathlineto{\pgfqpoint{3.919041in}{2.838117in}}%
\pgfpathlineto{\pgfqpoint{3.918975in}{2.431866in}}%
\pgfpathlineto{\pgfqpoint{3.919504in}{2.602321in}}%
\pgfpathlineto{\pgfqpoint{3.920430in}{2.478850in}}%
\pgfpathlineto{\pgfqpoint{3.920143in}{2.777256in}}%
\pgfpathlineto{\pgfqpoint{3.920496in}{2.627015in}}%
\pgfpathlineto{\pgfqpoint{3.921466in}{2.881277in}}%
\pgfpathlineto{\pgfqpoint{3.920606in}{2.488684in}}%
\pgfpathlineto{\pgfqpoint{3.921620in}{2.705140in}}%
\pgfpathlineto{\pgfqpoint{3.922700in}{2.917335in}}%
\pgfpathlineto{\pgfqpoint{3.922369in}{2.607675in}}%
\pgfpathlineto{\pgfqpoint{3.922744in}{2.827627in}}%
\pgfpathlineto{\pgfqpoint{3.923163in}{2.873628in}}%
\pgfpathlineto{\pgfqpoint{3.923890in}{2.512286in}}%
\pgfpathlineto{\pgfqpoint{3.924508in}{2.783375in}}%
\pgfpathlineto{\pgfqpoint{3.925037in}{2.686237in}}%
\pgfpathlineto{\pgfqpoint{3.925257in}{2.535232in}}%
\pgfpathlineto{\pgfqpoint{3.925455in}{2.816373in}}%
\pgfpathlineto{\pgfqpoint{3.926139in}{2.710385in}}%
\pgfpathlineto{\pgfqpoint{3.927219in}{2.516875in}}%
\pgfpathlineto{\pgfqpoint{3.926249in}{2.834511in}}%
\pgfpathlineto{\pgfqpoint{3.927263in}{2.624393in}}%
\pgfpathlineto{\pgfqpoint{3.927439in}{2.887833in}}%
\pgfpathlineto{\pgfqpoint{3.927770in}{2.577080in}}%
\pgfpathlineto{\pgfqpoint{3.928321in}{2.648540in}}%
\pgfpathlineto{\pgfqpoint{3.928343in}{2.499720in}}%
\pgfpathlineto{\pgfqpoint{3.928585in}{2.876032in}}%
\pgfpathlineto{\pgfqpoint{3.929423in}{2.638706in}}%
\pgfpathlineto{\pgfqpoint{3.929467in}{2.606254in}}%
\pgfpathlineto{\pgfqpoint{3.929511in}{2.661980in}}%
\pgfpathlineto{\pgfqpoint{3.929665in}{2.916351in}}%
\pgfpathlineto{\pgfqpoint{3.930018in}{2.572164in}}%
\pgfpathlineto{\pgfqpoint{3.930613in}{2.737701in}}%
\pgfpathlineto{\pgfqpoint{3.931142in}{2.597950in}}%
\pgfpathlineto{\pgfqpoint{3.931319in}{2.840084in}}%
\pgfpathlineto{\pgfqpoint{3.931649in}{2.758025in}}%
\pgfpathlineto{\pgfqpoint{3.932046in}{2.887505in}}%
\pgfpathlineto{\pgfqpoint{3.931958in}{2.623956in}}%
\pgfpathlineto{\pgfqpoint{3.932773in}{2.852103in}}%
\pgfpathlineto{\pgfqpoint{3.933699in}{2.600682in}}%
\pgfpathlineto{\pgfqpoint{3.933192in}{2.953829in}}%
\pgfpathlineto{\pgfqpoint{3.933942in}{2.746006in}}%
\pgfpathlineto{\pgfqpoint{3.934250in}{2.927824in}}%
\pgfpathlineto{\pgfqpoint{3.934867in}{2.572491in}}%
\pgfpathlineto{\pgfqpoint{3.935022in}{2.599152in}}%
\pgfpathlineto{\pgfqpoint{3.935749in}{2.918646in}}%
\pgfpathlineto{\pgfqpoint{3.936124in}{2.625048in}}%
\pgfpathlineto{\pgfqpoint{3.936344in}{2.593470in}}%
\pgfpathlineto{\pgfqpoint{3.936565in}{2.815608in}}%
\pgfpathlineto{\pgfqpoint{3.936741in}{2.677277in}}%
\pgfpathlineto{\pgfqpoint{3.936763in}{2.868274in}}%
\pgfpathlineto{\pgfqpoint{3.937049in}{2.599699in}}%
\pgfpathlineto{\pgfqpoint{3.937865in}{2.817138in}}%
\pgfpathlineto{\pgfqpoint{3.938879in}{2.651272in}}%
\pgfpathlineto{\pgfqpoint{3.938284in}{2.893405in}}%
\pgfpathlineto{\pgfqpoint{3.938989in}{2.718471in}}%
\pgfpathlineto{\pgfqpoint{3.939055in}{2.656735in}}%
\pgfpathlineto{\pgfqpoint{3.939099in}{2.812549in}}%
\pgfpathlineto{\pgfqpoint{3.939364in}{2.935691in}}%
\pgfpathlineto{\pgfqpoint{3.939827in}{2.648322in}}%
\pgfpathlineto{\pgfqpoint{3.940179in}{2.824021in}}%
\pgfpathlineto{\pgfqpoint{3.940907in}{2.632806in}}%
\pgfpathlineto{\pgfqpoint{3.940598in}{2.946727in}}%
\pgfpathlineto{\pgfqpoint{3.941304in}{2.722404in}}%
\pgfpathlineto{\pgfqpoint{3.941502in}{2.966504in}}%
\pgfpathlineto{\pgfqpoint{3.941722in}{2.631932in}}%
\pgfpathlineto{\pgfqpoint{3.942428in}{2.870459in}}%
\pgfpathlineto{\pgfqpoint{3.943530in}{2.492618in}}%
\pgfpathlineto{\pgfqpoint{3.942516in}{2.917444in}}%
\pgfpathlineto{\pgfqpoint{3.943574in}{2.602758in}}%
\pgfpathlineto{\pgfqpoint{3.943905in}{2.832544in}}%
\pgfpathlineto{\pgfqpoint{3.944588in}{2.567247in}}%
\pgfpathlineto{\pgfqpoint{3.944676in}{2.611827in}}%
\pgfpathlineto{\pgfqpoint{3.944742in}{2.522775in}}%
\pgfpathlineto{\pgfqpoint{3.945183in}{2.842706in}}%
\pgfpathlineto{\pgfqpoint{3.945756in}{2.562220in}}%
\pgfpathlineto{\pgfqpoint{3.946858in}{2.887614in}}%
\pgfpathlineto{\pgfqpoint{3.945910in}{2.442137in}}%
\pgfpathlineto{\pgfqpoint{3.946880in}{2.731910in}}%
\pgfpathlineto{\pgfqpoint{3.947431in}{2.573475in}}%
\pgfpathlineto{\pgfqpoint{3.947696in}{2.878436in}}%
\pgfpathlineto{\pgfqpoint{3.947982in}{2.712461in}}%
\pgfpathlineto{\pgfqpoint{3.948026in}{2.727977in}}%
\pgfpathlineto{\pgfqpoint{3.948048in}{2.895591in}}%
\pgfpathlineto{\pgfqpoint{3.948511in}{2.593798in}}%
\pgfpathlineto{\pgfqpoint{3.949106in}{2.723169in}}%
\pgfpathlineto{\pgfqpoint{3.950010in}{2.378872in}}%
\pgfpathlineto{\pgfqpoint{3.949239in}{2.750813in}}%
\pgfpathlineto{\pgfqpoint{3.950275in}{2.495022in}}%
\pgfpathlineto{\pgfqpoint{3.950429in}{2.603414in}}%
\pgfpathlineto{\pgfqpoint{3.951002in}{2.292661in}}%
\pgfpathlineto{\pgfqpoint{3.951333in}{2.394279in}}%
\pgfpathlineto{\pgfqpoint{3.952038in}{2.232128in}}%
\pgfpathlineto{\pgfqpoint{3.952236in}{2.530315in}}%
\pgfpathlineto{\pgfqpoint{3.952435in}{2.439952in}}%
\pgfpathlineto{\pgfqpoint{3.953338in}{2.578719in}}%
\pgfpathlineto{\pgfqpoint{3.952898in}{2.223168in}}%
\pgfpathlineto{\pgfqpoint{3.953515in}{2.415367in}}%
\pgfpathlineto{\pgfqpoint{3.954198in}{2.587789in}}%
\pgfpathlineto{\pgfqpoint{3.954617in}{2.090410in}}%
\pgfpathlineto{\pgfqpoint{3.954837in}{2.475354in}}%
\pgfpathlineto{\pgfqpoint{3.955741in}{2.335712in}}%
\pgfpathlineto{\pgfqpoint{3.956292in}{2.518514in}}%
\pgfpathlineto{\pgfqpoint{3.956094in}{2.194103in}}%
\pgfpathlineto{\pgfqpoint{3.956821in}{2.413619in}}%
\pgfpathlineto{\pgfqpoint{3.957438in}{2.136520in}}%
\pgfpathlineto{\pgfqpoint{3.956887in}{2.529878in}}%
\pgfpathlineto{\pgfqpoint{3.957923in}{2.274851in}}%
\pgfpathlineto{\pgfqpoint{3.958827in}{2.419956in}}%
\pgfpathlineto{\pgfqpoint{3.958011in}{2.134881in}}%
\pgfpathlineto{\pgfqpoint{3.959025in}{2.293972in}}%
\pgfpathlineto{\pgfqpoint{3.960039in}{1.989558in}}%
\pgfpathlineto{\pgfqpoint{3.959620in}{2.365214in}}%
\pgfpathlineto{\pgfqpoint{3.960149in}{2.223277in}}%
\pgfpathlineto{\pgfqpoint{3.961185in}{1.983767in}}%
\pgfpathlineto{\pgfqpoint{3.960877in}{2.328173in}}%
\pgfpathlineto{\pgfqpoint{3.961274in}{2.134444in}}%
\pgfpathlineto{\pgfqpoint{3.961428in}{1.919627in}}%
\pgfpathlineto{\pgfqpoint{3.962420in}{2.306975in}}%
\pgfpathlineto{\pgfqpoint{3.963544in}{1.952189in}}%
\pgfpathlineto{\pgfqpoint{3.963588in}{1.970655in}}%
\pgfpathlineto{\pgfqpoint{3.963610in}{1.929243in}}%
\pgfpathlineto{\pgfqpoint{3.964227in}{2.173999in}}%
\pgfpathlineto{\pgfqpoint{3.964646in}{2.039164in}}%
\pgfpathlineto{\pgfqpoint{3.965373in}{2.159248in}}%
\pgfpathlineto{\pgfqpoint{3.965109in}{1.843797in}}%
\pgfpathlineto{\pgfqpoint{3.965748in}{2.062657in}}%
\pgfpathlineto{\pgfqpoint{3.966542in}{1.847730in}}%
\pgfpathlineto{\pgfqpoint{3.966189in}{2.295721in}}%
\pgfpathlineto{\pgfqpoint{3.966850in}{2.052823in}}%
\pgfpathlineto{\pgfqpoint{3.967732in}{1.872971in}}%
\pgfpathlineto{\pgfqpoint{3.967511in}{2.163728in}}%
\pgfpathlineto{\pgfqpoint{3.967908in}{1.952626in}}%
\pgfpathlineto{\pgfqpoint{3.968459in}{2.169956in}}%
\pgfpathlineto{\pgfqpoint{3.968570in}{1.890453in}}%
\pgfpathlineto{\pgfqpoint{3.969010in}{1.970108in}}%
\pgfpathlineto{\pgfqpoint{3.969738in}{2.199785in}}%
\pgfpathlineto{\pgfqpoint{3.969429in}{1.841284in}}%
\pgfpathlineto{\pgfqpoint{3.970046in}{2.003981in}}%
\pgfpathlineto{\pgfqpoint{3.970090in}{1.865322in}}%
\pgfpathlineto{\pgfqpoint{3.970972in}{2.205686in}}%
\pgfpathlineto{\pgfqpoint{3.971148in}{1.990432in}}%
\pgfpathlineto{\pgfqpoint{3.971347in}{2.200659in}}%
\pgfpathlineto{\pgfqpoint{3.971479in}{1.869037in}}%
\pgfpathlineto{\pgfqpoint{3.972251in}{2.043644in}}%
\pgfpathlineto{\pgfqpoint{3.972824in}{2.129964in}}%
\pgfpathlineto{\pgfqpoint{3.973397in}{1.872643in}}%
\pgfpathlineto{\pgfqpoint{3.973639in}{2.190716in}}%
\pgfpathlineto{\pgfqpoint{3.974477in}{1.831887in}}%
\pgfpathlineto{\pgfqpoint{3.974521in}{1.972294in}}%
\pgfpathlineto{\pgfqpoint{3.975403in}{2.147665in}}%
\pgfpathlineto{\pgfqpoint{3.975226in}{1.767529in}}%
\pgfpathlineto{\pgfqpoint{3.975667in}{2.127779in}}%
\pgfpathlineto{\pgfqpoint{3.976791in}{1.889252in}}%
\pgfpathlineto{\pgfqpoint{3.975998in}{2.222731in}}%
\pgfpathlineto{\pgfqpoint{3.976813in}{1.930663in}}%
\pgfpathlineto{\pgfqpoint{3.977012in}{2.177604in}}%
\pgfpathlineto{\pgfqpoint{3.977122in}{1.841612in}}%
\pgfpathlineto{\pgfqpoint{3.977915in}{2.081122in}}%
\pgfpathlineto{\pgfqpoint{3.978400in}{1.853740in}}%
\pgfpathlineto{\pgfqpoint{3.979039in}{1.934597in}}%
\pgfpathlineto{\pgfqpoint{3.979789in}{1.695195in}}%
\pgfpathlineto{\pgfqpoint{3.979128in}{1.968032in}}%
\pgfpathlineto{\pgfqpoint{3.980252in}{1.742617in}}%
\pgfpathlineto{\pgfqpoint{3.981089in}{1.860296in}}%
\pgfpathlineto{\pgfqpoint{3.980825in}{1.614776in}}%
\pgfpathlineto{\pgfqpoint{3.981376in}{1.834728in}}%
\pgfpathlineto{\pgfqpoint{3.982412in}{1.686345in}}%
\pgfpathlineto{\pgfqpoint{3.981751in}{1.974807in}}%
\pgfpathlineto{\pgfqpoint{3.982456in}{1.733329in}}%
\pgfpathlineto{\pgfqpoint{3.983227in}{1.943775in}}%
\pgfpathlineto{\pgfqpoint{3.983338in}{1.651707in}}%
\pgfpathlineto{\pgfqpoint{3.983580in}{1.858001in}}%
\pgfpathlineto{\pgfqpoint{3.984550in}{2.084619in}}%
\pgfpathlineto{\pgfqpoint{3.983690in}{1.724151in}}%
\pgfpathlineto{\pgfqpoint{3.984682in}{1.928369in}}%
\pgfpathlineto{\pgfqpoint{3.985608in}{1.671157in}}%
\pgfpathlineto{\pgfqpoint{3.985321in}{2.040476in}}%
\pgfpathlineto{\pgfqpoint{3.985740in}{1.990541in}}%
\pgfpathlineto{\pgfqpoint{3.985762in}{2.070414in}}%
\pgfpathlineto{\pgfqpoint{3.986710in}{1.627560in}}%
\pgfpathlineto{\pgfqpoint{3.986798in}{1.889361in}}%
\pgfpathlineto{\pgfqpoint{3.986953in}{1.588880in}}%
\pgfpathlineto{\pgfqpoint{3.987922in}{1.752232in}}%
\pgfpathlineto{\pgfqpoint{3.988672in}{1.938968in}}%
\pgfpathlineto{\pgfqpoint{3.988143in}{1.637940in}}%
\pgfpathlineto{\pgfqpoint{3.989002in}{1.782499in}}%
\pgfpathlineto{\pgfqpoint{3.989796in}{1.598386in}}%
\pgfpathlineto{\pgfqpoint{3.989862in}{1.855051in}}%
\pgfpathlineto{\pgfqpoint{3.990105in}{1.656952in}}%
\pgfpathlineto{\pgfqpoint{3.990237in}{1.855598in}}%
\pgfpathlineto{\pgfqpoint{3.990700in}{1.549544in}}%
\pgfpathlineto{\pgfqpoint{3.991229in}{1.756166in}}%
\pgfpathlineto{\pgfqpoint{3.991824in}{1.548233in}}%
\pgfpathlineto{\pgfqpoint{3.991780in}{1.783919in}}%
\pgfpathlineto{\pgfqpoint{3.992088in}{1.750921in}}%
\pgfpathlineto{\pgfqpoint{3.992441in}{1.896572in}}%
\pgfpathlineto{\pgfqpoint{3.992794in}{1.562109in}}%
\pgfpathlineto{\pgfqpoint{3.993190in}{1.723058in}}%
\pgfpathlineto{\pgfqpoint{3.994094in}{1.443228in}}%
\pgfpathlineto{\pgfqpoint{3.993940in}{1.778456in}}%
\pgfpathlineto{\pgfqpoint{3.994315in}{1.642966in}}%
\pgfpathlineto{\pgfqpoint{3.994403in}{1.532389in}}%
\pgfpathlineto{\pgfqpoint{3.994557in}{1.789055in}}%
\pgfpathlineto{\pgfqpoint{3.995593in}{1.890453in}}%
\pgfpathlineto{\pgfqpoint{3.995064in}{1.566480in}}%
\pgfpathlineto{\pgfqpoint{3.995637in}{1.885755in}}%
\pgfpathlineto{\pgfqpoint{3.995703in}{1.636629in}}%
\pgfpathlineto{\pgfqpoint{3.996409in}{1.916459in}}%
\pgfpathlineto{\pgfqpoint{3.996739in}{1.780532in}}%
\pgfpathlineto{\pgfqpoint{3.996871in}{1.965628in}}%
\pgfpathlineto{\pgfqpoint{3.997621in}{1.635973in}}%
\pgfpathlineto{\pgfqpoint{3.997819in}{1.783045in}}%
\pgfpathlineto{\pgfqpoint{3.998238in}{1.587787in}}%
\pgfpathlineto{\pgfqpoint{3.998150in}{1.871550in}}%
\pgfpathlineto{\pgfqpoint{3.998899in}{1.700986in}}%
\pgfpathlineto{\pgfqpoint{3.999957in}{1.858657in}}%
\pgfpathlineto{\pgfqpoint{3.999715in}{1.604832in}}%
\pgfpathlineto{\pgfqpoint{4.000023in}{1.786214in}}%
\pgfpathlineto{\pgfqpoint{4.000508in}{1.566589in}}%
\pgfpathlineto{\pgfqpoint{4.000949in}{1.794955in}}%
\pgfpathlineto{\pgfqpoint{4.001126in}{1.783373in}}%
\pgfpathlineto{\pgfqpoint{4.001170in}{1.882586in}}%
\pgfpathlineto{\pgfqpoint{4.001456in}{1.651598in}}%
\pgfpathlineto{\pgfqpoint{4.002338in}{1.537197in}}%
\pgfpathlineto{\pgfqpoint{4.002404in}{1.806428in}}%
\pgfpathlineto{\pgfqpoint{4.002580in}{1.605379in}}%
\pgfpathlineto{\pgfqpoint{4.003572in}{1.473604in}}%
\pgfpathlineto{\pgfqpoint{4.003704in}{1.779002in}}%
\pgfpathlineto{\pgfqpoint{4.004189in}{1.807958in}}%
\pgfpathlineto{\pgfqpoint{4.004895in}{1.540256in}}%
\pgfpathlineto{\pgfqpoint{4.005975in}{1.811891in}}%
\pgfpathlineto{\pgfqpoint{4.005556in}{1.456886in}}%
\pgfpathlineto{\pgfqpoint{4.006019in}{1.702188in}}%
\pgfpathlineto{\pgfqpoint{4.006041in}{1.576970in}}%
\pgfpathlineto{\pgfqpoint{4.006790in}{1.825331in}}%
\pgfpathlineto{\pgfqpoint{4.007099in}{1.757258in}}%
\pgfpathlineto{\pgfqpoint{4.007474in}{1.925200in}}%
\pgfpathlineto{\pgfqpoint{4.007716in}{1.599041in}}%
\pgfpathlineto{\pgfqpoint{4.008091in}{1.701423in}}%
\pgfpathlineto{\pgfqpoint{4.008972in}{1.562546in}}%
\pgfpathlineto{\pgfqpoint{4.008245in}{1.888050in}}%
\pgfpathlineto{\pgfqpoint{4.009127in}{1.636519in}}%
\pgfpathlineto{\pgfqpoint{4.009876in}{1.787525in}}%
\pgfpathlineto{\pgfqpoint{4.010030in}{1.483984in}}%
\pgfpathlineto{\pgfqpoint{4.010207in}{1.530532in}}%
\pgfpathlineto{\pgfqpoint{4.010339in}{1.798889in}}%
\pgfpathlineto{\pgfqpoint{4.010978in}{1.432520in}}%
\pgfpathlineto{\pgfqpoint{4.011353in}{1.619365in}}%
\pgfpathlineto{\pgfqpoint{4.011485in}{1.580466in}}%
\pgfpathlineto{\pgfqpoint{4.011838in}{1.869037in}}%
\pgfpathlineto{\pgfqpoint{4.012345in}{1.656515in}}%
\pgfpathlineto{\pgfqpoint{4.012852in}{1.836695in}}%
\pgfpathlineto{\pgfqpoint{4.013116in}{1.556974in}}%
\pgfpathlineto{\pgfqpoint{4.013469in}{1.773648in}}%
\pgfpathlineto{\pgfqpoint{4.013888in}{1.572490in}}%
\pgfpathlineto{\pgfqpoint{4.014020in}{1.878762in}}%
\pgfpathlineto{\pgfqpoint{4.014593in}{1.717376in}}%
\pgfpathlineto{\pgfqpoint{4.015100in}{1.887066in}}%
\pgfpathlineto{\pgfqpoint{4.015299in}{1.499281in}}%
\pgfpathlineto{\pgfqpoint{4.015541in}{1.693666in}}%
\pgfpathlineto{\pgfqpoint{4.016533in}{1.497752in}}%
\pgfpathlineto{\pgfqpoint{4.015629in}{1.756384in}}%
\pgfpathlineto{\pgfqpoint{4.016643in}{1.589426in}}%
\pgfpathlineto{\pgfqpoint{4.017106in}{1.484094in}}%
\pgfpathlineto{\pgfqpoint{4.017789in}{1.807521in}}%
\pgfpathlineto{\pgfqpoint{4.018054in}{1.479941in}}%
\pgfpathlineto{\pgfqpoint{4.018891in}{1.647118in}}%
\pgfpathlineto{\pgfqpoint{4.019619in}{1.872206in}}%
\pgfpathlineto{\pgfqpoint{4.019090in}{1.534356in}}%
\pgfpathlineto{\pgfqpoint{4.020016in}{1.748845in}}%
\pgfpathlineto{\pgfqpoint{4.020545in}{1.540912in}}%
\pgfpathlineto{\pgfqpoint{4.020126in}{1.848058in}}%
\pgfpathlineto{\pgfqpoint{4.021140in}{1.611825in}}%
\pgfpathlineto{\pgfqpoint{4.021426in}{1.795938in}}%
\pgfpathlineto{\pgfqpoint{4.021757in}{1.590519in}}%
\pgfpathlineto{\pgfqpoint{4.022264in}{1.723386in}}%
\pgfpathlineto{\pgfqpoint{4.023256in}{1.543643in}}%
\pgfpathlineto{\pgfqpoint{4.022572in}{1.807193in}}%
\pgfpathlineto{\pgfqpoint{4.023366in}{1.712459in}}%
\pgfpathlineto{\pgfqpoint{4.023410in}{1.757695in}}%
\pgfpathlineto{\pgfqpoint{4.023917in}{1.542551in}}%
\pgfpathlineto{\pgfqpoint{4.024027in}{1.555663in}}%
\pgfpathlineto{\pgfqpoint{4.024159in}{1.405204in}}%
\pgfpathlineto{\pgfqpoint{4.024490in}{1.704155in}}%
\pgfpathlineto{\pgfqpoint{4.025129in}{1.503106in}}%
\pgfpathlineto{\pgfqpoint{4.025460in}{1.737809in}}%
\pgfpathlineto{\pgfqpoint{4.025636in}{1.408481in}}%
\pgfpathlineto{\pgfqpoint{4.026275in}{1.663727in}}%
\pgfpathlineto{\pgfqpoint{4.026297in}{1.662634in}}%
\pgfpathlineto{\pgfqpoint{4.027422in}{1.341829in}}%
\pgfpathlineto{\pgfqpoint{4.026606in}{1.666349in}}%
\pgfpathlineto{\pgfqpoint{4.027444in}{1.476882in}}%
\pgfpathlineto{\pgfqpoint{4.027510in}{1.709946in}}%
\pgfpathlineto{\pgfqpoint{4.027620in}{1.332979in}}%
\pgfpathlineto{\pgfqpoint{4.028546in}{1.462240in}}%
\pgfpathlineto{\pgfqpoint{4.028722in}{1.346965in}}%
\pgfpathlineto{\pgfqpoint{4.029097in}{1.775287in}}%
\pgfpathlineto{\pgfqpoint{4.029604in}{1.518731in}}%
\pgfpathlineto{\pgfqpoint{4.029824in}{1.664929in}}%
\pgfpathlineto{\pgfqpoint{4.030045in}{1.387065in}}%
\pgfpathlineto{\pgfqpoint{4.030684in}{1.552603in}}%
\pgfpathlineto{\pgfqpoint{4.031499in}{1.387612in}}%
\pgfpathlineto{\pgfqpoint{4.031081in}{1.660449in}}%
\pgfpathlineto{\pgfqpoint{4.031786in}{1.520370in}}%
\pgfpathlineto{\pgfqpoint{4.032161in}{1.392856in}}%
\pgfpathlineto{\pgfqpoint{4.031874in}{1.624063in}}%
\pgfpathlineto{\pgfqpoint{4.032183in}{1.508897in}}%
\pgfpathlineto{\pgfqpoint{4.032646in}{1.292660in}}%
\pgfpathlineto{\pgfqpoint{4.033307in}{1.716939in}}%
\pgfpathlineto{\pgfqpoint{4.034365in}{1.313857in}}%
\pgfpathlineto{\pgfqpoint{4.034431in}{1.479286in}}%
\pgfpathlineto{\pgfqpoint{4.034762in}{1.714972in}}%
\pgfpathlineto{\pgfqpoint{4.034497in}{1.398757in}}%
\pgfpathlineto{\pgfqpoint{4.035599in}{1.714208in}}%
\pgfpathlineto{\pgfqpoint{4.035908in}{1.463661in}}%
\pgfpathlineto{\pgfqpoint{4.036591in}{1.729723in}}%
\pgfpathlineto{\pgfqpoint{4.036723in}{1.529657in}}%
\pgfpathlineto{\pgfqpoint{4.037517in}{1.742726in}}%
\pgfpathlineto{\pgfqpoint{4.037451in}{1.449128in}}%
\pgfpathlineto{\pgfqpoint{4.037825in}{1.534793in}}%
\pgfpathlineto{\pgfqpoint{4.038310in}{1.689513in}}%
\pgfpathlineto{\pgfqpoint{4.038487in}{1.435142in}}%
\pgfpathlineto{\pgfqpoint{4.038905in}{1.602428in}}%
\pgfpathlineto{\pgfqpoint{4.039875in}{1.315933in}}%
\pgfpathlineto{\pgfqpoint{4.039324in}{1.624828in}}%
\pgfpathlineto{\pgfqpoint{4.040030in}{1.316917in}}%
\pgfpathlineto{\pgfqpoint{4.040140in}{1.590519in}}%
\pgfpathlineto{\pgfqpoint{4.041066in}{1.266654in}}%
\pgfpathlineto{\pgfqpoint{4.041132in}{1.305990in}}%
\pgfpathlineto{\pgfqpoint{4.041793in}{1.597293in}}%
\pgfpathlineto{\pgfqpoint{4.042278in}{1.492288in}}%
\pgfpathlineto{\pgfqpoint{4.042344in}{1.244036in}}%
\pgfpathlineto{\pgfqpoint{4.043116in}{1.652254in}}%
\pgfpathlineto{\pgfqpoint{4.043402in}{1.377122in}}%
\pgfpathlineto{\pgfqpoint{4.043424in}{1.258241in}}%
\pgfpathlineto{\pgfqpoint{4.044482in}{1.536104in}}%
\pgfpathlineto{\pgfqpoint{4.044636in}{1.231580in}}%
\pgfpathlineto{\pgfqpoint{4.044813in}{1.558176in}}%
\pgfpathlineto{\pgfqpoint{4.045606in}{1.402690in}}%
\pgfpathlineto{\pgfqpoint{4.045893in}{1.569758in}}%
\pgfpathlineto{\pgfqpoint{4.046510in}{1.298997in}}%
\pgfpathlineto{\pgfqpoint{4.046708in}{1.489010in}}%
\pgfpathlineto{\pgfqpoint{4.046774in}{1.265015in}}%
\pgfpathlineto{\pgfqpoint{4.047568in}{1.678478in}}%
\pgfpathlineto{\pgfqpoint{4.047810in}{1.401270in}}%
\pgfpathlineto{\pgfqpoint{4.047987in}{1.662634in}}%
\pgfpathlineto{\pgfqpoint{4.048802in}{1.293534in}}%
\pgfpathlineto{\pgfqpoint{4.048935in}{1.475571in}}%
\pgfpathlineto{\pgfqpoint{4.049508in}{1.585164in}}%
\pgfpathlineto{\pgfqpoint{4.049816in}{1.315824in}}%
\pgfpathlineto{\pgfqpoint{4.049838in}{1.408481in}}%
\pgfpathlineto{\pgfqpoint{4.050852in}{1.214862in}}%
\pgfpathlineto{\pgfqpoint{4.050632in}{1.565824in}}%
\pgfpathlineto{\pgfqpoint{4.050962in}{1.313857in}}%
\pgfpathlineto{\pgfqpoint{4.051712in}{1.107454in}}%
\pgfpathlineto{\pgfqpoint{4.051227in}{1.487590in}}%
\pgfpathlineto{\pgfqpoint{4.051932in}{1.381930in}}%
\pgfpathlineto{\pgfqpoint{4.052329in}{1.485186in}}%
\pgfpathlineto{\pgfqpoint{4.052483in}{1.153892in}}%
\pgfpathlineto{\pgfqpoint{4.053012in}{1.338661in}}%
\pgfpathlineto{\pgfqpoint{4.053211in}{1.205356in}}%
\pgfpathlineto{\pgfqpoint{4.053343in}{1.407607in}}%
\pgfpathlineto{\pgfqpoint{4.053828in}{1.316807in}}%
\pgfpathlineto{\pgfqpoint{4.053894in}{1.493163in}}%
\pgfpathlineto{\pgfqpoint{4.054798in}{1.198363in}}%
\pgfpathlineto{\pgfqpoint{4.054908in}{1.373516in}}%
\pgfpathlineto{\pgfqpoint{4.055635in}{1.221855in}}%
\pgfpathlineto{\pgfqpoint{4.054952in}{1.499281in}}%
\pgfpathlineto{\pgfqpoint{4.056010in}{1.365977in}}%
\pgfpathlineto{\pgfqpoint{4.056627in}{1.475899in}}%
\pgfpathlineto{\pgfqpoint{4.056649in}{1.216829in}}%
\pgfpathlineto{\pgfqpoint{4.057112in}{1.420610in}}%
\pgfpathlineto{\pgfqpoint{4.057377in}{1.183175in}}%
\pgfpathlineto{\pgfqpoint{4.057333in}{1.463114in}}%
\pgfpathlineto{\pgfqpoint{4.058236in}{1.278237in}}%
\pgfpathlineto{\pgfqpoint{4.059140in}{1.457870in}}%
\pgfpathlineto{\pgfqpoint{4.058942in}{1.140561in}}%
\pgfpathlineto{\pgfqpoint{4.059383in}{1.442572in}}%
\pgfpathlineto{\pgfqpoint{4.060396in}{1.490212in}}%
\pgfpathlineto{\pgfqpoint{4.060507in}{1.212131in}}%
\pgfpathlineto{\pgfqpoint{4.061080in}{1.542988in}}%
\pgfpathlineto{\pgfqpoint{4.061609in}{1.326095in}}%
\pgfpathlineto{\pgfqpoint{4.061631in}{1.189950in}}%
\pgfpathlineto{\pgfqpoint{4.061829in}{1.453936in}}%
\pgfpathlineto{\pgfqpoint{4.062711in}{1.315059in}}%
\pgfpathlineto{\pgfqpoint{4.063570in}{1.165037in}}%
\pgfpathlineto{\pgfqpoint{4.063152in}{1.478193in}}%
\pgfpathlineto{\pgfqpoint{4.063747in}{1.380400in}}%
\pgfpathlineto{\pgfqpoint{4.063769in}{1.465628in}}%
\pgfpathlineto{\pgfqpoint{4.064518in}{1.070850in}}%
\pgfpathlineto{\pgfqpoint{4.064805in}{1.243162in}}%
\pgfpathlineto{\pgfqpoint{4.065180in}{1.122205in}}%
\pgfpathlineto{\pgfqpoint{4.065554in}{1.414054in}}%
\pgfpathlineto{\pgfqpoint{4.065797in}{1.258896in}}%
\pgfpathlineto{\pgfqpoint{4.066436in}{1.361497in}}%
\pgfpathlineto{\pgfqpoint{4.066260in}{1.103302in}}%
\pgfpathlineto{\pgfqpoint{4.066921in}{1.311016in}}%
\pgfpathlineto{\pgfqpoint{4.067847in}{1.421593in}}%
\pgfpathlineto{\pgfqpoint{4.067009in}{1.115867in}}%
\pgfpathlineto{\pgfqpoint{4.067957in}{1.335383in}}%
\pgfpathlineto{\pgfqpoint{4.068905in}{1.131820in}}%
\pgfpathlineto{\pgfqpoint{4.068486in}{1.494692in}}%
\pgfpathlineto{\pgfqpoint{4.069081in}{1.209180in}}%
\pgfpathlineto{\pgfqpoint{4.069279in}{1.396572in}}%
\pgfpathlineto{\pgfqpoint{4.069610in}{1.090299in}}%
\pgfpathlineto{\pgfqpoint{4.070161in}{1.157825in}}%
\pgfpathlineto{\pgfqpoint{4.070910in}{1.338224in}}%
\pgfpathlineto{\pgfqpoint{4.070668in}{1.048887in}}%
\pgfpathlineto{\pgfqpoint{4.070955in}{1.318446in}}%
\pgfpathlineto{\pgfqpoint{4.071065in}{1.071833in}}%
\pgfpathlineto{\pgfqpoint{4.072057in}{1.194430in}}%
\pgfpathlineto{\pgfqpoint{4.072079in}{1.265234in}}%
\pgfpathlineto{\pgfqpoint{4.072762in}{0.999936in}}%
\pgfpathlineto{\pgfqpoint{4.073115in}{1.247861in}}%
\pgfpathlineto{\pgfqpoint{4.073776in}{0.996658in}}%
\pgfpathlineto{\pgfqpoint{4.074239in}{1.134115in}}%
\pgfpathlineto{\pgfqpoint{4.074614in}{0.999171in}}%
\pgfpathlineto{\pgfqpoint{4.075275in}{1.240321in}}%
\pgfpathlineto{\pgfqpoint{4.075319in}{1.191152in}}%
\pgfpathlineto{\pgfqpoint{4.075385in}{1.349150in}}%
\pgfpathlineto{\pgfqpoint{4.075826in}{0.998406in}}%
\pgfpathlineto{\pgfqpoint{4.076421in}{1.220872in}}%
\pgfpathlineto{\pgfqpoint{4.076994in}{1.416895in}}%
\pgfpathlineto{\pgfqpoint{4.076597in}{1.095653in}}%
\pgfpathlineto{\pgfqpoint{4.077479in}{1.277690in}}%
\pgfpathlineto{\pgfqpoint{4.078537in}{1.164163in}}%
\pgfpathlineto{\pgfqpoint{4.078449in}{1.433940in}}%
\pgfpathlineto{\pgfqpoint{4.078581in}{1.278674in}}%
\pgfpathlineto{\pgfqpoint{4.078912in}{1.013267in}}%
\pgfpathlineto{\pgfqpoint{4.079154in}{1.371550in}}%
\pgfpathlineto{\pgfqpoint{4.079749in}{1.109858in}}%
\pgfpathlineto{\pgfqpoint{4.080543in}{1.364557in}}%
\pgfpathlineto{\pgfqpoint{4.079815in}{1.087349in}}%
\pgfpathlineto{\pgfqpoint{4.080829in}{1.150614in}}%
\pgfpathlineto{\pgfqpoint{4.080895in}{1.067681in}}%
\pgfpathlineto{\pgfqpoint{4.081358in}{1.388595in}}%
\pgfpathlineto{\pgfqpoint{4.081887in}{1.220872in}}%
\pgfpathlineto{\pgfqpoint{4.081931in}{1.396899in}}%
\pgfpathlineto{\pgfqpoint{4.082879in}{1.038398in}}%
\pgfpathlineto{\pgfqpoint{4.082967in}{1.234749in}}%
\pgfpathlineto{\pgfqpoint{4.083276in}{1.007694in}}%
\pgfpathlineto{\pgfqpoint{4.083629in}{1.320632in}}%
\pgfpathlineto{\pgfqpoint{4.084070in}{1.112152in}}%
\pgfpathlineto{\pgfqpoint{4.084841in}{1.473058in}}%
\pgfpathlineto{\pgfqpoint{4.084951in}{1.071942in}}%
\pgfpathlineto{\pgfqpoint{4.085194in}{1.170937in}}%
\pgfpathlineto{\pgfqpoint{4.085789in}{1.039163in}}%
\pgfpathlineto{\pgfqpoint{4.085370in}{1.328499in}}%
\pgfpathlineto{\pgfqpoint{4.086119in}{1.149303in}}%
\pgfpathlineto{\pgfqpoint{4.086847in}{1.438202in}}%
\pgfpathlineto{\pgfqpoint{4.086164in}{1.088223in}}%
\pgfpathlineto{\pgfqpoint{4.087222in}{1.304023in}}%
\pgfpathlineto{\pgfqpoint{4.087905in}{1.131165in}}%
\pgfpathlineto{\pgfqpoint{4.088191in}{1.441808in}}%
\pgfpathlineto{\pgfqpoint{4.088346in}{1.134880in}}%
\pgfpathlineto{\pgfqpoint{4.089161in}{1.331012in}}%
\pgfpathlineto{\pgfqpoint{4.089382in}{1.048560in}}%
\pgfpathlineto{\pgfqpoint{4.089448in}{1.161868in}}%
\pgfpathlineto{\pgfqpoint{4.090307in}{1.316152in}}%
\pgfpathlineto{\pgfqpoint{4.089712in}{1.071615in}}%
\pgfpathlineto{\pgfqpoint{4.090594in}{1.249172in}}%
\pgfpathlineto{\pgfqpoint{4.091167in}{1.117943in}}%
\pgfpathlineto{\pgfqpoint{4.090991in}{1.407170in}}%
\pgfpathlineto{\pgfqpoint{4.091696in}{1.221527in}}%
\pgfpathlineto{\pgfqpoint{4.091806in}{1.407280in}}%
\pgfpathlineto{\pgfqpoint{4.092357in}{1.091501in}}%
\pgfpathlineto{\pgfqpoint{4.092754in}{1.238901in}}%
\pgfpathlineto{\pgfqpoint{4.093415in}{1.042441in}}%
\pgfpathlineto{\pgfqpoint{4.093658in}{1.379417in}}%
\pgfpathlineto{\pgfqpoint{4.093856in}{1.197052in}}%
\pgfpathlineto{\pgfqpoint{4.094055in}{1.054788in}}%
\pgfpathlineto{\pgfqpoint{4.094694in}{1.435907in}}%
\pgfpathlineto{\pgfqpoint{4.094826in}{1.370457in}}%
\pgfpathlineto{\pgfqpoint{4.094848in}{1.426510in}}%
\pgfpathlineto{\pgfqpoint{4.095135in}{1.158153in}}%
\pgfpathlineto{\pgfqpoint{4.095884in}{1.323145in}}%
\pgfpathlineto{\pgfqpoint{4.096369in}{1.130181in}}%
\pgfpathlineto{\pgfqpoint{4.096567in}{1.433285in}}%
\pgfpathlineto{\pgfqpoint{4.097008in}{1.257804in}}%
\pgfpathlineto{\pgfqpoint{4.097096in}{1.495894in}}%
\pgfpathlineto{\pgfqpoint{4.097846in}{1.177275in}}%
\pgfpathlineto{\pgfqpoint{4.098132in}{1.280859in}}%
\pgfpathlineto{\pgfqpoint{4.098595in}{1.126466in}}%
\pgfpathlineto{\pgfqpoint{4.098243in}{1.351226in}}%
\pgfpathlineto{\pgfqpoint{4.099212in}{1.280313in}}%
\pgfpathlineto{\pgfqpoint{4.099808in}{1.441589in}}%
\pgfpathlineto{\pgfqpoint{4.099455in}{1.067135in}}%
\pgfpathlineto{\pgfqpoint{4.100314in}{1.336694in}}%
\pgfpathlineto{\pgfqpoint{4.101593in}{1.032279in}}%
\pgfpathlineto{\pgfqpoint{4.100425in}{1.392747in}}%
\pgfpathlineto{\pgfqpoint{4.101637in}{1.172686in}}%
\pgfpathlineto{\pgfqpoint{4.102431in}{1.286869in}}%
\pgfpathlineto{\pgfqpoint{4.102144in}{1.061453in}}%
\pgfpathlineto{\pgfqpoint{4.102739in}{1.198145in}}%
\pgfpathlineto{\pgfqpoint{4.102805in}{1.035448in}}%
\pgfpathlineto{\pgfqpoint{4.102893in}{1.347402in}}%
\pgfpathlineto{\pgfqpoint{4.103841in}{1.158372in}}%
\pgfpathlineto{\pgfqpoint{4.104877in}{1.413836in}}%
\pgfpathlineto{\pgfqpoint{4.104591in}{1.043752in}}%
\pgfpathlineto{\pgfqpoint{4.104987in}{1.351226in}}%
\pgfpathlineto{\pgfqpoint{4.105671in}{1.472730in}}%
\pgfpathlineto{\pgfqpoint{4.106134in}{1.069648in}}%
\pgfpathlineto{\pgfqpoint{4.106552in}{1.408481in}}%
\pgfpathlineto{\pgfqpoint{4.107302in}{1.198909in}}%
\pgfpathlineto{\pgfqpoint{4.107434in}{1.070631in}}%
\pgfpathlineto{\pgfqpoint{4.107610in}{1.388923in}}%
\pgfpathlineto{\pgfqpoint{4.108382in}{1.244036in}}%
\pgfpathlineto{\pgfqpoint{4.108801in}{1.474478in}}%
\pgfpathlineto{\pgfqpoint{4.108558in}{1.162305in}}%
\pgfpathlineto{\pgfqpoint{4.109528in}{1.324237in}}%
\pgfpathlineto{\pgfqpoint{4.109550in}{1.324237in}}%
\pgfpathlineto{\pgfqpoint{4.110366in}{1.100352in}}%
\pgfpathlineto{\pgfqpoint{4.110189in}{1.470326in}}%
\pgfpathlineto{\pgfqpoint{4.110652in}{1.279220in}}%
\pgfpathlineto{\pgfqpoint{4.110740in}{1.425199in}}%
\pgfpathlineto{\pgfqpoint{4.110983in}{1.108874in}}%
\pgfpathlineto{\pgfqpoint{4.111688in}{1.182083in}}%
\pgfpathlineto{\pgfqpoint{4.112195in}{1.057301in}}%
\pgfpathlineto{\pgfqpoint{4.111975in}{1.324237in}}%
\pgfpathlineto{\pgfqpoint{4.112790in}{1.109639in}}%
\pgfpathlineto{\pgfqpoint{4.113628in}{1.322271in}}%
\pgfpathlineto{\pgfqpoint{4.112967in}{0.997095in}}%
\pgfpathlineto{\pgfqpoint{4.113892in}{1.203826in}}%
\pgfpathlineto{\pgfqpoint{4.114972in}{0.965845in}}%
\pgfpathlineto{\pgfqpoint{4.114135in}{1.288726in}}%
\pgfpathlineto{\pgfqpoint{4.115061in}{0.997860in}}%
\pgfpathlineto{\pgfqpoint{4.115303in}{1.326095in}}%
\pgfpathlineto{\pgfqpoint{4.116008in}{0.897554in}}%
\pgfpathlineto{\pgfqpoint{4.116207in}{1.105924in}}%
\pgfpathlineto{\pgfqpoint{4.116515in}{0.900286in}}%
\pgfpathlineto{\pgfqpoint{4.117287in}{1.227428in}}%
\pgfpathlineto{\pgfqpoint{4.117309in}{1.096527in}}%
\pgfpathlineto{\pgfqpoint{4.118014in}{1.206340in}}%
\pgfpathlineto{\pgfqpoint{4.117463in}{0.877449in}}%
\pgfpathlineto{\pgfqpoint{4.118323in}{1.121877in}}%
\pgfpathlineto{\pgfqpoint{4.118433in}{0.914381in}}%
\pgfpathlineto{\pgfqpoint{4.119381in}{1.173341in}}%
\pgfpathlineto{\pgfqpoint{4.119425in}{1.006274in}}%
\pgfpathlineto{\pgfqpoint{4.119469in}{1.280531in}}%
\pgfpathlineto{\pgfqpoint{4.120064in}{0.885753in}}%
\pgfpathlineto{\pgfqpoint{4.120527in}{1.092703in}}%
\pgfpathlineto{\pgfqpoint{4.121034in}{1.241851in}}%
\pgfpathlineto{\pgfqpoint{4.120703in}{0.884005in}}%
\pgfpathlineto{\pgfqpoint{4.121276in}{1.045391in}}%
\pgfpathlineto{\pgfqpoint{4.122246in}{0.883677in}}%
\pgfpathlineto{\pgfqpoint{4.121541in}{1.154329in}}%
\pgfpathlineto{\pgfqpoint{4.122379in}{1.059923in}}%
\pgfpathlineto{\pgfqpoint{4.122599in}{0.865976in}}%
\pgfpathlineto{\pgfqpoint{4.122511in}{1.159246in}}%
\pgfpathlineto{\pgfqpoint{4.123503in}{0.944975in}}%
\pgfpathlineto{\pgfqpoint{4.123525in}{1.084071in}}%
\pgfpathlineto{\pgfqpoint{4.124561in}{0.817681in}}%
\pgfpathlineto{\pgfqpoint{4.124583in}{0.927165in}}%
\pgfpathlineto{\pgfqpoint{4.124649in}{0.858437in}}%
\pgfpathlineto{\pgfqpoint{4.125310in}{1.173778in}}%
\pgfpathlineto{\pgfqpoint{4.125619in}{1.006820in}}%
\pgfpathlineto{\pgfqpoint{4.126721in}{1.237808in}}%
\pgfpathlineto{\pgfqpoint{4.126170in}{0.906405in}}%
\pgfpathlineto{\pgfqpoint{4.126743in}{1.231580in}}%
\pgfpathlineto{\pgfqpoint{4.126941in}{1.095325in}}%
\pgfpathlineto{\pgfqpoint{4.127360in}{1.484421in}}%
\pgfpathlineto{\pgfqpoint{4.127779in}{1.315059in}}%
\pgfpathlineto{\pgfqpoint{4.128264in}{1.398975in}}%
\pgfpathlineto{\pgfqpoint{4.127889in}{1.135098in}}%
\pgfpathlineto{\pgfqpoint{4.128771in}{1.200330in}}%
\pgfpathlineto{\pgfqpoint{4.129498in}{1.026925in}}%
\pgfpathlineto{\pgfqpoint{4.129057in}{1.331012in}}%
\pgfpathlineto{\pgfqpoint{4.129873in}{1.174434in}}%
\pgfpathlineto{\pgfqpoint{4.130071in}{1.330575in}}%
\pgfpathlineto{\pgfqpoint{4.130358in}{1.030203in}}%
\pgfpathlineto{\pgfqpoint{4.130975in}{1.184049in}}%
\pgfpathlineto{\pgfqpoint{4.131724in}{1.001138in}}%
\pgfpathlineto{\pgfqpoint{4.131548in}{1.348167in}}%
\pgfpathlineto{\pgfqpoint{4.132077in}{1.185251in}}%
\pgfpathlineto{\pgfqpoint{4.133091in}{0.937873in}}%
\pgfpathlineto{\pgfqpoint{4.132452in}{1.311890in}}%
\pgfpathlineto{\pgfqpoint{4.133201in}{1.105378in}}%
\pgfpathlineto{\pgfqpoint{4.134149in}{1.380291in}}%
\pgfpathlineto{\pgfqpoint{4.133444in}{1.037305in}}%
\pgfpathlineto{\pgfqpoint{4.134347in}{1.230597in}}%
\pgfpathlineto{\pgfqpoint{4.135207in}{1.055880in}}%
\pgfpathlineto{\pgfqpoint{4.135251in}{1.273975in}}%
\pgfpathlineto{\pgfqpoint{4.135471in}{1.157388in}}%
\pgfpathlineto{\pgfqpoint{4.135846in}{0.936016in}}%
\pgfpathlineto{\pgfqpoint{4.135912in}{1.222948in}}%
\pgfpathlineto{\pgfqpoint{4.136618in}{1.025942in}}%
\pgfpathlineto{\pgfqpoint{4.137345in}{1.162961in}}%
\pgfpathlineto{\pgfqpoint{4.137279in}{0.888376in}}%
\pgfpathlineto{\pgfqpoint{4.137698in}{1.035994in}}%
\pgfpathlineto{\pgfqpoint{4.138535in}{0.845653in}}%
\pgfpathlineto{\pgfqpoint{4.138447in}{1.128324in}}%
\pgfpathlineto{\pgfqpoint{4.138756in}{1.010972in}}%
\pgfpathlineto{\pgfqpoint{4.139814in}{1.306209in}}%
\pgfpathlineto{\pgfqpoint{4.139086in}{0.835382in}}%
\pgfpathlineto{\pgfqpoint{4.139858in}{1.101991in}}%
\pgfpathlineto{\pgfqpoint{4.140122in}{0.907279in}}%
\pgfpathlineto{\pgfqpoint{4.140475in}{1.206777in}}%
\pgfpathlineto{\pgfqpoint{4.140982in}{1.032279in}}%
\pgfpathlineto{\pgfqpoint{4.141996in}{1.224696in}}%
\pgfpathlineto{\pgfqpoint{4.141313in}{0.924324in}}%
\pgfpathlineto{\pgfqpoint{4.142084in}{1.068992in}}%
\pgfpathlineto{\pgfqpoint{4.142569in}{0.946942in}}%
\pgfpathlineto{\pgfqpoint{4.142988in}{1.288070in}}%
\pgfpathlineto{\pgfqpoint{4.143142in}{1.141326in}}%
\pgfpathlineto{\pgfqpoint{4.144178in}{1.304460in}}%
\pgfpathlineto{\pgfqpoint{4.143473in}{1.029219in}}%
\pgfpathlineto{\pgfqpoint{4.144266in}{1.211584in}}%
\pgfpathlineto{\pgfqpoint{4.144817in}{1.000045in}}%
\pgfpathlineto{\pgfqpoint{4.144685in}{1.304242in}}%
\pgfpathlineto{\pgfqpoint{4.145390in}{1.165365in}}%
\pgfpathlineto{\pgfqpoint{4.146360in}{1.282061in}}%
\pgfpathlineto{\pgfqpoint{4.145501in}{1.021243in}}%
\pgfpathlineto{\pgfqpoint{4.146492in}{1.134880in}}%
\pgfpathlineto{\pgfqpoint{4.146779in}{0.961802in}}%
\pgfpathlineto{\pgfqpoint{4.146669in}{1.284902in}}%
\pgfpathlineto{\pgfqpoint{4.147617in}{1.042550in}}%
\pgfpathlineto{\pgfqpoint{4.147969in}{1.200111in}}%
\pgfpathlineto{\pgfqpoint{4.147749in}{0.917878in}}%
\pgfpathlineto{\pgfqpoint{4.148741in}{1.145915in}}%
\pgfpathlineto{\pgfqpoint{4.149248in}{0.939949in}}%
\pgfpathlineto{\pgfqpoint{4.149821in}{1.233656in}}%
\pgfpathlineto{\pgfqpoint{4.149865in}{1.070303in}}%
\pgfpathlineto{\pgfqpoint{4.150328in}{1.216392in}}%
\pgfpathlineto{\pgfqpoint{4.150460in}{0.985622in}}%
\pgfpathlineto{\pgfqpoint{4.150989in}{1.127668in}}%
\pgfpathlineto{\pgfqpoint{4.151628in}{0.914709in}}%
\pgfpathlineto{\pgfqpoint{4.152025in}{1.235841in}}%
\pgfpathlineto{\pgfqpoint{4.152113in}{1.059923in}}%
\pgfpathlineto{\pgfqpoint{4.152201in}{1.172576in}}%
\pgfpathlineto{\pgfqpoint{4.153105in}{0.906951in}}%
\pgfpathlineto{\pgfqpoint{4.153193in}{1.072926in}}%
\pgfpathlineto{\pgfqpoint{4.153546in}{0.864446in}}%
\pgfpathlineto{\pgfqpoint{4.154229in}{1.179788in}}%
\pgfpathlineto{\pgfqpoint{4.154295in}{1.072489in}}%
\pgfpathlineto{\pgfqpoint{4.155221in}{1.204919in}}%
\pgfpathlineto{\pgfqpoint{4.154802in}{0.921046in}}%
\pgfpathlineto{\pgfqpoint{4.155397in}{1.088223in}}%
\pgfpathlineto{\pgfqpoint{4.155794in}{1.319976in}}%
\pgfpathlineto{\pgfqpoint{4.155574in}{0.997314in}}%
\pgfpathlineto{\pgfqpoint{4.156367in}{1.094233in}}%
\pgfpathlineto{\pgfqpoint{4.156918in}{0.940933in}}%
\pgfpathlineto{\pgfqpoint{4.156830in}{1.278018in}}%
\pgfpathlineto{\pgfqpoint{4.157469in}{1.119801in}}%
\pgfpathlineto{\pgfqpoint{4.158087in}{1.197708in}}%
\pgfpathlineto{\pgfqpoint{4.157756in}{0.933284in}}%
\pgfpathlineto{\pgfqpoint{4.158549in}{1.086475in}}%
\pgfpathlineto{\pgfqpoint{4.159321in}{1.300199in}}%
\pgfpathlineto{\pgfqpoint{4.159100in}{1.002996in}}%
\pgfpathlineto{\pgfqpoint{4.159607in}{1.140999in}}%
\pgfpathlineto{\pgfqpoint{4.160026in}{0.991632in}}%
\pgfpathlineto{\pgfqpoint{4.160643in}{1.355488in}}%
\pgfpathlineto{\pgfqpoint{4.160665in}{1.289600in}}%
\pgfpathlineto{\pgfqpoint{4.161591in}{1.463114in}}%
\pgfpathlineto{\pgfqpoint{4.161128in}{1.070303in}}%
\pgfpathlineto{\pgfqpoint{4.161723in}{1.251248in}}%
\pgfpathlineto{\pgfqpoint{4.161856in}{1.085710in}}%
\pgfpathlineto{\pgfqpoint{4.161768in}{1.370894in}}%
\pgfpathlineto{\pgfqpoint{4.162804in}{1.154985in}}%
\pgfpathlineto{\pgfqpoint{4.163531in}{1.485623in}}%
\pgfpathlineto{\pgfqpoint{4.162914in}{1.146462in}}%
\pgfpathlineto{\pgfqpoint{4.163928in}{1.300199in}}%
\pgfpathlineto{\pgfqpoint{4.164898in}{1.084180in}}%
\pgfpathlineto{\pgfqpoint{4.164038in}{1.459399in}}%
\pgfpathlineto{\pgfqpoint{4.165074in}{1.145915in}}%
\pgfpathlineto{\pgfqpoint{4.165779in}{1.328608in}}%
\pgfpathlineto{\pgfqpoint{4.165625in}{1.055116in}}%
\pgfpathlineto{\pgfqpoint{4.166176in}{1.206012in}}%
\pgfpathlineto{\pgfqpoint{4.166485in}{0.981798in}}%
\pgfpathlineto{\pgfqpoint{4.166639in}{1.279985in}}%
\pgfpathlineto{\pgfqpoint{4.167278in}{1.096199in}}%
\pgfpathlineto{\pgfqpoint{4.168314in}{1.275177in}}%
\pgfpathlineto{\pgfqpoint{4.167454in}{0.981470in}}%
\pgfpathlineto{\pgfqpoint{4.168380in}{1.173450in}}%
\pgfpathlineto{\pgfqpoint{4.168887in}{0.990212in}}%
\pgfpathlineto{\pgfqpoint{4.168490in}{1.354067in}}%
\pgfpathlineto{\pgfqpoint{4.169460in}{1.226226in}}%
\pgfpathlineto{\pgfqpoint{4.169725in}{1.397009in}}%
\pgfpathlineto{\pgfqpoint{4.170320in}{1.094779in}}%
\pgfpathlineto{\pgfqpoint{4.170430in}{1.224587in}}%
\pgfpathlineto{\pgfqpoint{4.170717in}{1.060032in}}%
\pgfpathlineto{\pgfqpoint{4.171113in}{1.284137in}}%
\pgfpathlineto{\pgfqpoint{4.171554in}{1.128761in}}%
\pgfpathlineto{\pgfqpoint{4.172524in}{1.367835in}}%
\pgfpathlineto{\pgfqpoint{4.171664in}{1.079263in}}%
\pgfpathlineto{\pgfqpoint{4.172700in}{1.224259in}}%
\pgfpathlineto{\pgfqpoint{4.172987in}{1.299980in}}%
\pgfpathlineto{\pgfqpoint{4.172965in}{1.177384in}}%
\pgfpathlineto{\pgfqpoint{4.173053in}{1.293862in}}%
\pgfpathlineto{\pgfqpoint{4.173119in}{1.423451in}}%
\pgfpathlineto{\pgfqpoint{4.173538in}{1.043752in}}%
\pgfpathlineto{\pgfqpoint{4.174133in}{1.153236in}}%
\pgfpathlineto{\pgfqpoint{4.175213in}{0.881929in}}%
\pgfpathlineto{\pgfqpoint{4.174309in}{1.257148in}}%
\pgfpathlineto{\pgfqpoint{4.175279in}{0.937218in}}%
\pgfpathlineto{\pgfqpoint{4.176227in}{1.230378in}}%
\pgfpathlineto{\pgfqpoint{4.175345in}{0.910557in}}%
\pgfpathlineto{\pgfqpoint{4.176403in}{1.075548in}}%
\pgfpathlineto{\pgfqpoint{4.176932in}{1.180006in}}%
\pgfpathlineto{\pgfqpoint{4.177065in}{1.000155in}}%
\pgfpathlineto{\pgfqpoint{4.177417in}{0.909573in}}%
\pgfpathlineto{\pgfqpoint{4.177241in}{1.201532in}}%
\pgfpathlineto{\pgfqpoint{4.178101in}{1.088005in}}%
\pgfpathlineto{\pgfqpoint{4.178167in}{1.203608in}}%
\pgfpathlineto{\pgfqpoint{4.178475in}{0.871330in}}%
\pgfpathlineto{\pgfqpoint{4.179181in}{1.161978in}}%
\pgfpathlineto{\pgfqpoint{4.180040in}{0.928258in}}%
\pgfpathlineto{\pgfqpoint{4.179820in}{1.228630in}}%
\pgfpathlineto{\pgfqpoint{4.180305in}{1.046374in}}%
\pgfpathlineto{\pgfqpoint{4.180768in}{1.183066in}}%
\pgfpathlineto{\pgfqpoint{4.180966in}{0.912633in}}%
\pgfpathlineto{\pgfqpoint{4.181407in}{1.026160in}}%
\pgfpathlineto{\pgfqpoint{4.181517in}{1.141982in}}%
\pgfpathlineto{\pgfqpoint{4.182311in}{0.852099in}}%
\pgfpathlineto{\pgfqpoint{4.182487in}{0.968468in}}%
\pgfpathlineto{\pgfqpoint{4.182531in}{0.829263in}}%
\pgfpathlineto{\pgfqpoint{4.182641in}{1.144495in}}%
\pgfpathlineto{\pgfqpoint{4.183567in}{1.014687in}}%
\pgfpathlineto{\pgfqpoint{4.184493in}{1.195085in}}%
\pgfpathlineto{\pgfqpoint{4.183854in}{0.893730in}}%
\pgfpathlineto{\pgfqpoint{4.184603in}{0.919954in}}%
\pgfpathlineto{\pgfqpoint{4.185463in}{0.795718in}}%
\pgfpathlineto{\pgfqpoint{4.184735in}{1.153673in}}%
\pgfpathlineto{\pgfqpoint{4.185661in}{0.945413in}}%
\pgfpathlineto{\pgfqpoint{4.185881in}{1.091064in}}%
\pgfpathlineto{\pgfqpoint{4.186344in}{0.821833in}}%
\pgfpathlineto{\pgfqpoint{4.186763in}{0.994691in}}%
\pgfpathlineto{\pgfqpoint{4.187402in}{0.839752in}}%
\pgfpathlineto{\pgfqpoint{4.187491in}{1.113682in}}%
\pgfpathlineto{\pgfqpoint{4.187887in}{0.954154in}}%
\pgfpathlineto{\pgfqpoint{4.188813in}{0.856470in}}%
\pgfpathlineto{\pgfqpoint{4.188042in}{1.122642in}}%
\pgfpathlineto{\pgfqpoint{4.188967in}{0.961802in}}%
\pgfpathlineto{\pgfqpoint{4.189011in}{1.094670in}}%
\pgfpathlineto{\pgfqpoint{4.189364in}{0.801182in}}%
\pgfpathlineto{\pgfqpoint{4.190047in}{0.962130in}}%
\pgfpathlineto{\pgfqpoint{4.190422in}{0.849696in}}%
\pgfpathlineto{\pgfqpoint{4.190819in}{1.129744in}}%
\pgfpathlineto{\pgfqpoint{4.191127in}{1.007039in}}%
\pgfpathlineto{\pgfqpoint{4.191767in}{1.201423in}}%
\pgfpathlineto{\pgfqpoint{4.191590in}{0.962130in}}%
\pgfpathlineto{\pgfqpoint{4.192296in}{1.158372in}}%
\pgfpathlineto{\pgfqpoint{4.193221in}{0.912086in}}%
\pgfpathlineto{\pgfqpoint{4.192825in}{1.227756in}}%
\pgfpathlineto{\pgfqpoint{4.193442in}{0.949237in}}%
\pgfpathlineto{\pgfqpoint{4.194522in}{1.133241in}}%
\pgfpathlineto{\pgfqpoint{4.194081in}{0.795718in}}%
\pgfpathlineto{\pgfqpoint{4.194566in}{1.028345in}}%
\pgfpathlineto{\pgfqpoint{4.195448in}{0.890452in}}%
\pgfpathlineto{\pgfqpoint{4.195139in}{1.200002in}}%
\pgfpathlineto{\pgfqpoint{4.195558in}{1.122860in}}%
\pgfpathlineto{\pgfqpoint{4.195580in}{1.273320in}}%
\pgfpathlineto{\pgfqpoint{4.195911in}{0.885644in}}%
\pgfpathlineto{\pgfqpoint{4.196638in}{1.016545in}}%
\pgfpathlineto{\pgfqpoint{4.197035in}{0.846855in}}%
\pgfpathlineto{\pgfqpoint{4.196770in}{1.129635in}}%
\pgfpathlineto{\pgfqpoint{4.197696in}{1.095325in}}%
\pgfpathlineto{\pgfqpoint{4.197718in}{1.113245in}}%
\pgfpathlineto{\pgfqpoint{4.197872in}{0.878214in}}%
\pgfpathlineto{\pgfqpoint{4.198556in}{0.929460in}}%
\pgfpathlineto{\pgfqpoint{4.198798in}{0.779328in}}%
\pgfpathlineto{\pgfqpoint{4.199085in}{1.099805in}}%
\pgfpathlineto{\pgfqpoint{4.199636in}{0.970981in}}%
\pgfpathlineto{\pgfqpoint{4.199702in}{1.137393in}}%
\pgfpathlineto{\pgfqpoint{4.200694in}{0.926400in}}%
\pgfpathlineto{\pgfqpoint{4.201223in}{0.857016in}}%
\pgfpathlineto{\pgfqpoint{4.201509in}{1.213770in}}%
\pgfpathlineto{\pgfqpoint{4.201752in}{1.020260in}}%
\pgfpathlineto{\pgfqpoint{4.202413in}{1.180225in}}%
\pgfpathlineto{\pgfqpoint{4.202104in}{0.893730in}}%
\pgfpathlineto{\pgfqpoint{4.202876in}{1.109639in}}%
\pgfpathlineto{\pgfqpoint{4.203515in}{1.272336in}}%
\pgfpathlineto{\pgfqpoint{4.203846in}{1.003761in}}%
\pgfpathlineto{\pgfqpoint{4.203934in}{1.125264in}}%
\pgfpathlineto{\pgfqpoint{4.204771in}{0.916020in}}%
\pgfpathlineto{\pgfqpoint{4.204176in}{1.234858in}}%
\pgfpathlineto{\pgfqpoint{4.205058in}{0.938529in}}%
\pgfpathlineto{\pgfqpoint{4.205080in}{1.186781in}}%
\pgfpathlineto{\pgfqpoint{4.205389in}{0.916239in}}%
\pgfpathlineto{\pgfqpoint{4.206182in}{1.139250in}}%
\pgfpathlineto{\pgfqpoint{4.206292in}{0.962786in}}%
\pgfpathlineto{\pgfqpoint{4.207174in}{1.430990in}}%
\pgfpathlineto{\pgfqpoint{4.207240in}{1.285448in}}%
\pgfpathlineto{\pgfqpoint{4.208276in}{1.421812in}}%
\pgfpathlineto{\pgfqpoint{4.207703in}{1.078171in}}%
\pgfpathlineto{\pgfqpoint{4.208320in}{1.337896in}}%
\pgfpathlineto{\pgfqpoint{4.208607in}{1.153018in}}%
\pgfpathlineto{\pgfqpoint{4.208849in}{1.422030in}}%
\pgfpathlineto{\pgfqpoint{4.209444in}{1.261737in}}%
\pgfpathlineto{\pgfqpoint{4.210436in}{1.515234in}}%
\pgfpathlineto{\pgfqpoint{4.210304in}{1.253215in}}%
\pgfpathlineto{\pgfqpoint{4.210591in}{1.495239in}}%
\pgfpathlineto{\pgfqpoint{4.211053in}{1.239666in}}%
\pgfpathlineto{\pgfqpoint{4.211428in}{1.541458in}}%
\pgfpathlineto{\pgfqpoint{4.211715in}{1.344779in}}%
\pgfpathlineto{\pgfqpoint{4.212244in}{1.518622in}}%
\pgfpathlineto{\pgfqpoint{4.211847in}{1.187546in}}%
\pgfpathlineto{\pgfqpoint{4.212773in}{1.292878in}}%
\pgfpathlineto{\pgfqpoint{4.212817in}{1.255291in}}%
\pgfpathlineto{\pgfqpoint{4.213434in}{1.526926in}}%
\pgfpathlineto{\pgfqpoint{4.213765in}{1.439294in}}%
\pgfpathlineto{\pgfqpoint{4.214205in}{1.592813in}}%
\pgfpathlineto{\pgfqpoint{4.214294in}{1.290584in}}%
\pgfpathlineto{\pgfqpoint{4.214889in}{1.448691in}}%
\pgfpathlineto{\pgfqpoint{4.214911in}{1.333744in}}%
\pgfpathlineto{\pgfqpoint{4.215704in}{1.686017in}}%
\pgfpathlineto{\pgfqpoint{4.215969in}{1.573910in}}%
\pgfpathlineto{\pgfqpoint{4.216035in}{1.553805in}}%
\pgfpathlineto{\pgfqpoint{4.216895in}{1.797140in}}%
\pgfpathlineto{\pgfqpoint{4.216145in}{1.458197in}}%
\pgfpathlineto{\pgfqpoint{4.217137in}{1.633788in}}%
\pgfpathlineto{\pgfqpoint{4.217357in}{1.431427in}}%
\pgfpathlineto{\pgfqpoint{4.217820in}{1.754308in}}%
\pgfpathlineto{\pgfqpoint{4.218239in}{1.677385in}}%
\pgfpathlineto{\pgfqpoint{4.218570in}{1.486279in}}%
\pgfpathlineto{\pgfqpoint{4.219077in}{1.801183in}}%
\pgfpathlineto{\pgfqpoint{4.219319in}{1.723495in}}%
\pgfpathlineto{\pgfqpoint{4.219363in}{1.591283in}}%
\pgfpathlineto{\pgfqpoint{4.219650in}{1.879308in}}%
\pgfpathlineto{\pgfqpoint{4.220377in}{1.612590in}}%
\pgfpathlineto{\pgfqpoint{4.220421in}{1.938312in}}%
\pgfpathlineto{\pgfqpoint{4.221501in}{1.869256in}}%
\pgfpathlineto{\pgfqpoint{4.221810in}{1.575549in}}%
\pgfpathlineto{\pgfqpoint{4.222537in}{1.913399in}}%
\pgfpathlineto{\pgfqpoint{4.222625in}{1.744911in}}%
\pgfpathlineto{\pgfqpoint{4.223309in}{1.899085in}}%
\pgfpathlineto{\pgfqpoint{4.222978in}{1.608001in}}%
\pgfpathlineto{\pgfqpoint{4.223728in}{1.738465in}}%
\pgfpathlineto{\pgfqpoint{4.224168in}{1.915694in}}%
\pgfpathlineto{\pgfqpoint{4.224477in}{1.567354in}}%
\pgfpathlineto{\pgfqpoint{4.224852in}{1.816371in}}%
\pgfpathlineto{\pgfqpoint{4.225160in}{1.695851in}}%
\pgfpathlineto{\pgfqpoint{4.225888in}{1.907280in}}%
\pgfpathlineto{\pgfqpoint{4.225932in}{1.860078in}}%
\pgfpathlineto{\pgfqpoint{4.226836in}{1.931319in}}%
\pgfpathlineto{\pgfqpoint{4.226351in}{1.670064in}}%
\pgfpathlineto{\pgfqpoint{4.226968in}{1.825877in}}%
\pgfpathlineto{\pgfqpoint{4.227144in}{1.788945in}}%
\pgfpathlineto{\pgfqpoint{4.227210in}{2.038290in}}%
\pgfpathlineto{\pgfqpoint{4.227695in}{1.913727in}}%
\pgfpathlineto{\pgfqpoint{4.228026in}{2.100025in}}%
\pgfpathlineto{\pgfqpoint{4.228643in}{1.747424in}}%
\pgfpathlineto{\pgfqpoint{4.228797in}{1.892857in}}%
\pgfpathlineto{\pgfqpoint{4.229282in}{1.981690in}}%
\pgfpathlineto{\pgfqpoint{4.229811in}{1.756166in}}%
\pgfpathlineto{\pgfqpoint{4.229833in}{1.835602in}}%
\pgfpathlineto{\pgfqpoint{4.230539in}{1.648320in}}%
\pgfpathlineto{\pgfqpoint{4.229899in}{1.893404in}}%
\pgfpathlineto{\pgfqpoint{4.230935in}{1.801292in}}%
\pgfpathlineto{\pgfqpoint{4.231839in}{1.573582in}}%
\pgfpathlineto{\pgfqpoint{4.231046in}{2.074457in}}%
\pgfpathlineto{\pgfqpoint{4.232126in}{1.706122in}}%
\pgfpathlineto{\pgfqpoint{4.232368in}{1.874610in}}%
\pgfpathlineto{\pgfqpoint{4.232721in}{1.614994in}}%
\pgfpathlineto{\pgfqpoint{4.233250in}{1.756603in}}%
\pgfpathlineto{\pgfqpoint{4.234021in}{1.632367in}}%
\pgfpathlineto{\pgfqpoint{4.233360in}{1.862372in}}%
\pgfpathlineto{\pgfqpoint{4.234308in}{1.755401in}}%
\pgfpathlineto{\pgfqpoint{4.234594in}{1.910995in}}%
\pgfpathlineto{\pgfqpoint{4.235256in}{1.640234in}}%
\pgfpathlineto{\pgfqpoint{4.235388in}{1.683941in}}%
\pgfpathlineto{\pgfqpoint{4.235961in}{1.548997in}}%
\pgfpathlineto{\pgfqpoint{4.235608in}{1.847184in}}%
\pgfpathlineto{\pgfqpoint{4.236115in}{1.733766in}}%
\pgfpathlineto{\pgfqpoint{4.236512in}{1.885646in}}%
\pgfpathlineto{\pgfqpoint{4.237151in}{1.607236in}}%
\pgfpathlineto{\pgfqpoint{4.237217in}{1.880947in}}%
\pgfpathlineto{\pgfqpoint{4.238253in}{1.651926in}}%
\pgfpathlineto{\pgfqpoint{4.237724in}{1.950659in}}%
\pgfpathlineto{\pgfqpoint{4.238341in}{1.783264in}}%
\pgfpathlineto{\pgfqpoint{4.239223in}{1.531624in}}%
\pgfpathlineto{\pgfqpoint{4.238716in}{1.808832in}}%
\pgfpathlineto{\pgfqpoint{4.239333in}{1.748845in}}%
\pgfpathlineto{\pgfqpoint{4.239355in}{1.841939in}}%
\pgfpathlineto{\pgfqpoint{4.240259in}{1.503871in}}%
\pgfpathlineto{\pgfqpoint{4.240413in}{1.640999in}}%
\pgfpathlineto{\pgfqpoint{4.240986in}{1.539601in}}%
\pgfpathlineto{\pgfqpoint{4.240590in}{1.841612in}}%
\pgfpathlineto{\pgfqpoint{4.241405in}{1.671922in}}%
\pgfpathlineto{\pgfqpoint{4.242221in}{1.925856in}}%
\pgfpathlineto{\pgfqpoint{4.241648in}{1.658045in}}%
\pgfpathlineto{\pgfqpoint{4.242507in}{1.700112in}}%
\pgfpathlineto{\pgfqpoint{4.243058in}{1.602756in}}%
\pgfpathlineto{\pgfqpoint{4.242860in}{1.859203in}}%
\pgfpathlineto{\pgfqpoint{4.243521in}{1.788836in}}%
\pgfpathlineto{\pgfqpoint{4.243543in}{1.902254in}}%
\pgfpathlineto{\pgfqpoint{4.243698in}{1.548451in}}%
\pgfpathlineto{\pgfqpoint{4.244623in}{1.884662in}}%
\pgfpathlineto{\pgfqpoint{4.245461in}{1.622206in}}%
\pgfpathlineto{\pgfqpoint{4.245703in}{1.955139in}}%
\pgfpathlineto{\pgfqpoint{4.245748in}{1.794627in}}%
\pgfpathlineto{\pgfqpoint{4.246343in}{2.034684in}}%
\pgfpathlineto{\pgfqpoint{4.246453in}{1.678259in}}%
\pgfpathlineto{\pgfqpoint{4.246850in}{1.743054in}}%
\pgfpathlineto{\pgfqpoint{4.246872in}{1.675527in}}%
\pgfpathlineto{\pgfqpoint{4.247114in}{1.985733in}}%
\pgfpathlineto{\pgfqpoint{4.247908in}{1.850025in}}%
\pgfpathlineto{\pgfqpoint{4.248282in}{1.992399in}}%
\pgfpathlineto{\pgfqpoint{4.247996in}{1.744802in}}%
\pgfpathlineto{\pgfqpoint{4.248900in}{1.771681in}}%
\pgfpathlineto{\pgfqpoint{4.249252in}{1.584181in}}%
\pgfpathlineto{\pgfqpoint{4.249803in}{1.892857in}}%
\pgfpathlineto{\pgfqpoint{4.250002in}{1.950987in}}%
\pgfpathlineto{\pgfqpoint{4.250090in}{1.692136in}}%
\pgfpathlineto{\pgfqpoint{4.250839in}{1.755729in}}%
\pgfpathlineto{\pgfqpoint{4.251016in}{1.675200in}}%
\pgfpathlineto{\pgfqpoint{4.251324in}{1.901708in}}%
\pgfpathlineto{\pgfqpoint{4.251831in}{1.869911in}}%
\pgfpathlineto{\pgfqpoint{4.253110in}{2.065060in}}%
\pgfpathlineto{\pgfqpoint{4.252052in}{1.797031in}}%
\pgfpathlineto{\pgfqpoint{4.253198in}{1.987263in}}%
\pgfpathlineto{\pgfqpoint{4.253925in}{1.844015in}}%
\pgfpathlineto{\pgfqpoint{4.254101in}{2.140891in}}%
\pgfpathlineto{\pgfqpoint{4.254300in}{1.998190in}}%
\pgfpathlineto{\pgfqpoint{4.254653in}{2.133898in}}%
\pgfpathlineto{\pgfqpoint{4.254542in}{1.886520in}}%
\pgfpathlineto{\pgfqpoint{4.254785in}{1.937219in}}%
\pgfpathlineto{\pgfqpoint{4.255600in}{1.828937in}}%
\pgfpathlineto{\pgfqpoint{4.255358in}{2.142311in}}%
\pgfpathlineto{\pgfqpoint{4.255821in}{1.884662in}}%
\pgfpathlineto{\pgfqpoint{4.255843in}{2.009335in}}%
\pgfpathlineto{\pgfqpoint{4.256504in}{1.714535in}}%
\pgfpathlineto{\pgfqpoint{4.256923in}{1.840628in}}%
\pgfpathlineto{\pgfqpoint{4.257738in}{1.693119in}}%
\pgfpathlineto{\pgfqpoint{4.257121in}{1.948474in}}%
\pgfpathlineto{\pgfqpoint{4.257871in}{1.839208in}}%
\pgfpathlineto{\pgfqpoint{4.258686in}{1.946507in}}%
\pgfpathlineto{\pgfqpoint{4.258157in}{1.656515in}}%
\pgfpathlineto{\pgfqpoint{4.258951in}{1.788618in}}%
\pgfpathlineto{\pgfqpoint{4.259369in}{1.961258in}}%
\pgfpathlineto{\pgfqpoint{4.259524in}{1.688858in}}%
\pgfpathlineto{\pgfqpoint{4.260097in}{1.891983in}}%
\pgfpathlineto{\pgfqpoint{4.261111in}{1.664382in}}%
\pgfpathlineto{\pgfqpoint{4.260383in}{1.900506in}}%
\pgfpathlineto{\pgfqpoint{4.261243in}{1.699020in}}%
\pgfpathlineto{\pgfqpoint{4.262125in}{1.912416in}}%
\pgfpathlineto{\pgfqpoint{4.261552in}{1.645261in}}%
\pgfpathlineto{\pgfqpoint{4.262389in}{1.835820in}}%
\pgfpathlineto{\pgfqpoint{4.262984in}{1.576642in}}%
\pgfpathlineto{\pgfqpoint{4.262588in}{1.882368in}}%
\pgfpathlineto{\pgfqpoint{4.263580in}{1.700986in}}%
\pgfpathlineto{\pgfqpoint{4.263646in}{1.776817in}}%
\pgfpathlineto{\pgfqpoint{4.263690in}{1.739666in}}%
\pgfpathlineto{\pgfqpoint{4.264461in}{1.967486in}}%
\pgfpathlineto{\pgfqpoint{4.263932in}{1.611716in}}%
\pgfpathlineto{\pgfqpoint{4.264836in}{1.939951in}}%
\pgfpathlineto{\pgfqpoint{4.265740in}{1.692136in}}%
\pgfpathlineto{\pgfqpoint{4.264990in}{1.942464in}}%
\pgfpathlineto{\pgfqpoint{4.265982in}{1.714098in}}%
\pgfpathlineto{\pgfqpoint{4.266136in}{1.874501in}}%
\pgfpathlineto{\pgfqpoint{4.266665in}{1.576423in}}%
\pgfpathlineto{\pgfqpoint{4.267084in}{1.802276in}}%
\pgfpathlineto{\pgfqpoint{4.267437in}{1.603193in}}%
\pgfpathlineto{\pgfqpoint{4.267988in}{1.902910in}}%
\pgfpathlineto{\pgfqpoint{4.268164in}{1.822490in}}%
\pgfpathlineto{\pgfqpoint{4.269002in}{2.001905in}}%
\pgfpathlineto{\pgfqpoint{4.268297in}{1.776489in}}%
\pgfpathlineto{\pgfqpoint{4.269288in}{1.977757in}}%
\pgfpathlineto{\pgfqpoint{4.270236in}{1.798998in}}%
\pgfpathlineto{\pgfqpoint{4.269663in}{2.132259in}}%
\pgfpathlineto{\pgfqpoint{4.270390in}{1.934488in}}%
\pgfpathlineto{\pgfqpoint{4.270721in}{2.066372in}}%
\pgfpathlineto{\pgfqpoint{4.271184in}{1.817573in}}%
\pgfpathlineto{\pgfqpoint{4.271360in}{1.907280in}}%
\pgfpathlineto{\pgfqpoint{4.271625in}{1.746332in}}%
\pgfpathlineto{\pgfqpoint{4.271581in}{2.032281in}}%
\pgfpathlineto{\pgfqpoint{4.272440in}{1.967814in}}%
\pgfpathlineto{\pgfqpoint{4.273454in}{2.103522in}}%
\pgfpathlineto{\pgfqpoint{4.273124in}{1.821834in}}%
\pgfpathlineto{\pgfqpoint{4.273498in}{2.048343in}}%
\pgfpathlineto{\pgfqpoint{4.273741in}{1.845764in}}%
\pgfpathlineto{\pgfqpoint{4.274578in}{2.116634in}}%
\pgfpathlineto{\pgfqpoint{4.274601in}{2.069103in}}%
\pgfpathlineto{\pgfqpoint{4.274645in}{2.195196in}}%
\pgfpathlineto{\pgfqpoint{4.275416in}{1.905095in}}%
\pgfpathlineto{\pgfqpoint{4.275725in}{2.122097in}}%
\pgfpathlineto{\pgfqpoint{4.276672in}{2.011520in}}%
\pgfpathlineto{\pgfqpoint{4.276011in}{2.278457in}}%
\pgfpathlineto{\pgfqpoint{4.276805in}{2.085602in}}%
\pgfpathlineto{\pgfqpoint{4.276937in}{2.270480in}}%
\pgfpathlineto{\pgfqpoint{4.277598in}{1.851883in}}%
\pgfpathlineto{\pgfqpoint{4.277929in}{2.124720in}}%
\pgfpathlineto{\pgfqpoint{4.278326in}{1.869474in}}%
\pgfpathlineto{\pgfqpoint{4.279031in}{2.136193in}}%
\pgfpathlineto{\pgfqpoint{4.279053in}{2.060690in}}%
\pgfpathlineto{\pgfqpoint{4.279824in}{2.180336in}}%
\pgfpathlineto{\pgfqpoint{4.279582in}{1.961476in}}%
\pgfpathlineto{\pgfqpoint{4.279979in}{2.039929in}}%
\pgfpathlineto{\pgfqpoint{4.280177in}{1.902145in}}%
\pgfpathlineto{\pgfqpoint{4.280971in}{2.209291in}}%
\pgfpathlineto{\pgfqpoint{4.281059in}{2.006275in}}%
\pgfpathlineto{\pgfqpoint{4.282029in}{2.195742in}}%
\pgfpathlineto{\pgfqpoint{4.281764in}{1.953718in}}%
\pgfpathlineto{\pgfqpoint{4.282183in}{2.095546in}}%
\pgfpathlineto{\pgfqpoint{4.282205in}{2.165257in}}%
\pgfpathlineto{\pgfqpoint{4.282844in}{1.882259in}}%
\pgfpathlineto{\pgfqpoint{4.283241in}{2.003434in}}%
\pgfpathlineto{\pgfqpoint{4.283990in}{1.830685in}}%
\pgfpathlineto{\pgfqpoint{4.284123in}{2.096966in}}%
\pgfpathlineto{\pgfqpoint{4.284365in}{1.949020in}}%
\pgfpathlineto{\pgfqpoint{4.284806in}{2.156407in}}%
\pgfpathlineto{\pgfqpoint{4.284475in}{1.821288in}}%
\pgfpathlineto{\pgfqpoint{4.285511in}{2.054571in}}%
\pgfpathlineto{\pgfqpoint{4.286084in}{1.921594in}}%
\pgfpathlineto{\pgfqpoint{4.285710in}{2.246770in}}%
\pgfpathlineto{\pgfqpoint{4.286437in}{2.094344in}}%
\pgfpathlineto{\pgfqpoint{4.287209in}{2.187985in}}%
\pgfpathlineto{\pgfqpoint{4.287429in}{1.844452in}}%
\pgfpathlineto{\pgfqpoint{4.287495in}{2.029549in}}%
\pgfpathlineto{\pgfqpoint{4.288046in}{1.842814in}}%
\pgfpathlineto{\pgfqpoint{4.288112in}{2.114339in}}%
\pgfpathlineto{\pgfqpoint{4.288597in}{1.944540in}}%
\pgfpathlineto{\pgfqpoint{4.289325in}{2.072272in}}%
\pgfpathlineto{\pgfqpoint{4.288884in}{1.815388in}}%
\pgfpathlineto{\pgfqpoint{4.289677in}{2.025288in}}%
\pgfpathlineto{\pgfqpoint{4.289699in}{1.901489in}}%
\pgfpathlineto{\pgfqpoint{4.290162in}{2.280970in}}%
\pgfpathlineto{\pgfqpoint{4.290757in}{2.130401in}}%
\pgfpathlineto{\pgfqpoint{4.291705in}{1.878434in}}%
\pgfpathlineto{\pgfqpoint{4.291617in}{2.226337in}}%
\pgfpathlineto{\pgfqpoint{4.291970in}{1.981581in}}%
\pgfpathlineto{\pgfqpoint{4.292609in}{2.245895in}}%
\pgfpathlineto{\pgfqpoint{4.292807in}{1.901926in}}%
\pgfpathlineto{\pgfqpoint{4.293028in}{2.098059in}}%
\pgfpathlineto{\pgfqpoint{4.293513in}{1.947709in}}%
\pgfpathlineto{\pgfqpoint{4.294042in}{2.252779in}}%
\pgfpathlineto{\pgfqpoint{4.294108in}{2.126686in}}%
\pgfpathlineto{\pgfqpoint{4.294791in}{2.048452in}}%
\pgfpathlineto{\pgfqpoint{4.295254in}{2.309379in}}%
\pgfpathlineto{\pgfqpoint{4.296069in}{1.980052in}}%
\pgfpathlineto{\pgfqpoint{4.296334in}{2.398868in}}%
\pgfpathlineto{\pgfqpoint{4.296378in}{2.133461in}}%
\pgfpathlineto{\pgfqpoint{4.297502in}{2.426184in}}%
\pgfpathlineto{\pgfqpoint{4.296554in}{2.061236in}}%
\pgfpathlineto{\pgfqpoint{4.297590in}{2.304462in}}%
\pgfpathlineto{\pgfqpoint{4.297656in}{2.118928in}}%
\pgfpathlineto{\pgfqpoint{4.297855in}{2.519716in}}%
\pgfpathlineto{\pgfqpoint{4.298692in}{2.249173in}}%
\pgfpathlineto{\pgfqpoint{4.299640in}{2.538400in}}%
\pgfpathlineto{\pgfqpoint{4.299199in}{2.188203in}}%
\pgfpathlineto{\pgfqpoint{4.299817in}{2.317792in}}%
\pgfpathlineto{\pgfqpoint{4.300235in}{2.195633in}}%
\pgfpathlineto{\pgfqpoint{4.300566in}{2.510647in}}%
\pgfpathlineto{\pgfqpoint{4.300897in}{2.284029in}}%
\pgfpathlineto{\pgfqpoint{4.301426in}{2.562220in}}%
\pgfpathlineto{\pgfqpoint{4.301933in}{2.253107in}}%
\pgfpathlineto{\pgfqpoint{4.302021in}{2.473387in}}%
\pgfpathlineto{\pgfqpoint{4.302638in}{2.164165in}}%
\pgfpathlineto{\pgfqpoint{4.302307in}{2.492072in}}%
\pgfpathlineto{\pgfqpoint{4.303123in}{2.330358in}}%
\pgfpathlineto{\pgfqpoint{4.303784in}{2.491307in}}%
\pgfpathlineto{\pgfqpoint{4.303718in}{2.232237in}}%
\pgfpathlineto{\pgfqpoint{4.304203in}{2.415039in}}%
\pgfpathlineto{\pgfqpoint{4.304401in}{2.206450in}}%
\pgfpathlineto{\pgfqpoint{4.304864in}{2.520044in}}%
\pgfpathlineto{\pgfqpoint{4.305283in}{2.440280in}}%
\pgfpathlineto{\pgfqpoint{4.305592in}{2.573147in}}%
\pgfpathlineto{\pgfqpoint{4.306076in}{2.326862in}}%
\pgfpathlineto{\pgfqpoint{4.306385in}{2.453938in}}%
\pgfpathlineto{\pgfqpoint{4.306826in}{2.596967in}}%
\pgfpathlineto{\pgfqpoint{4.306628in}{2.296049in}}%
\pgfpathlineto{\pgfqpoint{4.307223in}{2.414930in}}%
\pgfpathlineto{\pgfqpoint{4.308082in}{2.206450in}}%
\pgfpathlineto{\pgfqpoint{4.307553in}{2.563859in}}%
\pgfpathlineto{\pgfqpoint{4.308347in}{2.363029in}}%
\pgfpathlineto{\pgfqpoint{4.308854in}{2.553698in}}%
\pgfpathlineto{\pgfqpoint{4.309096in}{2.242399in}}%
\pgfpathlineto{\pgfqpoint{4.309427in}{2.518405in}}%
\pgfpathlineto{\pgfqpoint{4.310507in}{2.207543in}}%
\pgfpathlineto{\pgfqpoint{4.310154in}{2.535996in}}%
\pgfpathlineto{\pgfqpoint{4.310551in}{2.226118in}}%
\pgfpathlineto{\pgfqpoint{4.311058in}{2.455249in}}%
\pgfpathlineto{\pgfqpoint{4.310749in}{2.127779in}}%
\pgfpathlineto{\pgfqpoint{4.311697in}{2.282390in}}%
\pgfpathlineto{\pgfqpoint{4.312226in}{2.499174in}}%
\pgfpathlineto{\pgfqpoint{4.311984in}{2.228959in}}%
\pgfpathlineto{\pgfqpoint{4.312887in}{2.323802in}}%
\pgfpathlineto{\pgfqpoint{4.312909in}{2.243492in}}%
\pgfpathlineto{\pgfqpoint{4.313394in}{2.506604in}}%
\pgfpathlineto{\pgfqpoint{4.314012in}{2.254200in}}%
\pgfpathlineto{\pgfqpoint{4.314761in}{2.511739in}}%
\pgfpathlineto{\pgfqpoint{4.314166in}{2.213771in}}%
\pgfpathlineto{\pgfqpoint{4.315136in}{2.372098in}}%
\pgfpathlineto{\pgfqpoint{4.315995in}{2.217705in}}%
\pgfpathlineto{\pgfqpoint{4.316128in}{2.536652in}}%
\pgfpathlineto{\pgfqpoint{4.316260in}{2.251250in}}%
\pgfpathlineto{\pgfqpoint{4.317318in}{2.567793in}}%
\pgfpathlineto{\pgfqpoint{4.316745in}{2.142202in}}%
\pgfpathlineto{\pgfqpoint{4.317384in}{2.400725in}}%
\pgfpathlineto{\pgfqpoint{4.317582in}{2.516875in}}%
\pgfpathlineto{\pgfqpoint{4.318552in}{2.163837in}}%
\pgfpathlineto{\pgfqpoint{4.319280in}{2.643951in}}%
\pgfpathlineto{\pgfqpoint{4.319698in}{2.511193in}}%
\pgfpathlineto{\pgfqpoint{4.319765in}{2.314514in}}%
\pgfpathlineto{\pgfqpoint{4.319743in}{2.572273in}}%
\pgfpathlineto{\pgfqpoint{4.320602in}{2.526490in}}%
\pgfpathlineto{\pgfqpoint{4.320668in}{2.640236in}}%
\pgfpathlineto{\pgfqpoint{4.321550in}{2.251140in}}%
\pgfpathlineto{\pgfqpoint{4.321682in}{2.466285in}}%
\pgfpathlineto{\pgfqpoint{4.321770in}{2.242180in}}%
\pgfpathlineto{\pgfqpoint{4.322454in}{2.530861in}}%
\pgfpathlineto{\pgfqpoint{4.322850in}{2.354615in}}%
\pgfpathlineto{\pgfqpoint{4.323864in}{2.539493in}}%
\pgfpathlineto{\pgfqpoint{4.323335in}{2.215738in}}%
\pgfpathlineto{\pgfqpoint{4.323930in}{2.450769in}}%
\pgfpathlineto{\pgfqpoint{4.324261in}{2.136848in}}%
\pgfpathlineto{\pgfqpoint{4.325033in}{2.418099in}}%
\pgfpathlineto{\pgfqpoint{4.325187in}{2.060034in}}%
\pgfpathlineto{\pgfqpoint{4.326223in}{2.183614in}}%
\pgfpathlineto{\pgfqpoint{4.327215in}{2.119366in}}%
\pgfpathlineto{\pgfqpoint{4.327391in}{2.448365in}}%
\pgfpathlineto{\pgfqpoint{4.327788in}{2.135428in}}%
\pgfpathlineto{\pgfqpoint{4.328515in}{2.307849in}}%
\pgfpathlineto{\pgfqpoint{4.329463in}{2.668536in}}%
\pgfpathlineto{\pgfqpoint{4.328912in}{2.238465in}}%
\pgfpathlineto{\pgfqpoint{4.329683in}{2.499939in}}%
\pgfpathlineto{\pgfqpoint{4.330477in}{2.198583in}}%
\pgfpathlineto{\pgfqpoint{4.329860in}{2.570087in}}%
\pgfpathlineto{\pgfqpoint{4.330808in}{2.369257in}}%
\pgfpathlineto{\pgfqpoint{4.331160in}{2.505730in}}%
\pgfpathlineto{\pgfqpoint{4.331844in}{2.245349in}}%
\pgfpathlineto{\pgfqpoint{4.331888in}{2.478523in}}%
\pgfpathlineto{\pgfqpoint{4.332857in}{2.228194in}}%
\pgfpathlineto{\pgfqpoint{4.332703in}{2.546595in}}%
\pgfpathlineto{\pgfqpoint{4.333012in}{2.320196in}}%
\pgfpathlineto{\pgfqpoint{4.333651in}{2.567902in}}%
\pgfpathlineto{\pgfqpoint{4.333739in}{2.286870in}}%
\pgfpathlineto{\pgfqpoint{4.334136in}{2.400616in}}%
\pgfpathlineto{\pgfqpoint{4.334422in}{2.270480in}}%
\pgfpathlineto{\pgfqpoint{4.334974in}{2.544301in}}%
\pgfpathlineto{\pgfqpoint{4.335216in}{2.399851in}}%
\pgfpathlineto{\pgfqpoint{4.335965in}{2.649524in}}%
\pgfpathlineto{\pgfqpoint{4.335436in}{2.303479in}}%
\pgfpathlineto{\pgfqpoint{4.336340in}{2.569869in}}%
\pgfpathlineto{\pgfqpoint{4.336494in}{2.314405in}}%
\pgfpathlineto{\pgfqpoint{4.336406in}{2.655643in}}%
\pgfpathlineto{\pgfqpoint{4.337464in}{2.416678in}}%
\pgfpathlineto{\pgfqpoint{4.337993in}{2.642312in}}%
\pgfpathlineto{\pgfqpoint{4.338412in}{2.351337in}}%
\pgfpathlineto{\pgfqpoint{4.338588in}{2.513378in}}%
\pgfpathlineto{\pgfqpoint{4.338875in}{2.609642in}}%
\pgfpathlineto{\pgfqpoint{4.339624in}{2.348059in}}%
\pgfpathlineto{\pgfqpoint{4.339691in}{2.521573in}}%
\pgfpathlineto{\pgfqpoint{4.340330in}{2.254090in}}%
\pgfpathlineto{\pgfqpoint{4.340021in}{2.545721in}}%
\pgfpathlineto{\pgfqpoint{4.340815in}{2.419082in}}%
\pgfpathlineto{\pgfqpoint{4.341851in}{2.284466in}}%
\pgfpathlineto{\pgfqpoint{4.341498in}{2.631823in}}%
\pgfpathlineto{\pgfqpoint{4.341917in}{2.363793in}}%
\pgfpathlineto{\pgfqpoint{4.343019in}{2.515017in}}%
\pgfpathlineto{\pgfqpoint{4.342688in}{2.211586in}}%
\pgfpathlineto{\pgfqpoint{4.343041in}{2.506495in}}%
\pgfpathlineto{\pgfqpoint{4.343173in}{2.281188in}}%
\pgfpathlineto{\pgfqpoint{4.344033in}{2.554244in}}%
\pgfpathlineto{\pgfqpoint{4.344165in}{2.417662in}}%
\pgfpathlineto{\pgfqpoint{4.344981in}{2.340738in}}%
\pgfpathlineto{\pgfqpoint{4.344760in}{2.600791in}}%
\pgfpathlineto{\pgfqpoint{4.345135in}{2.492509in}}%
\pgfpathlineto{\pgfqpoint{4.345157in}{2.550857in}}%
\pgfpathlineto{\pgfqpoint{4.346039in}{2.293863in}}%
\pgfpathlineto{\pgfqpoint{4.346193in}{2.402474in}}%
\pgfpathlineto{\pgfqpoint{4.346876in}{2.207762in}}%
\pgfpathlineto{\pgfqpoint{4.347295in}{2.321398in}}%
\pgfpathlineto{\pgfqpoint{4.347802in}{2.620241in}}%
\pgfpathlineto{\pgfqpoint{4.348089in}{2.241088in}}%
\pgfpathlineto{\pgfqpoint{4.348529in}{2.472294in}}%
\pgfpathlineto{\pgfqpoint{4.348992in}{2.371333in}}%
\pgfpathlineto{\pgfqpoint{4.349411in}{2.592815in}}%
\pgfpathlineto{\pgfqpoint{4.349433in}{2.602867in}}%
\pgfpathlineto{\pgfqpoint{4.349918in}{2.342377in}}%
\pgfpathlineto{\pgfqpoint{4.350249in}{2.478086in}}%
\pgfpathlineto{\pgfqpoint{4.350601in}{2.364340in}}%
\pgfpathlineto{\pgfqpoint{4.350734in}{2.597622in}}%
\pgfpathlineto{\pgfqpoint{4.351351in}{2.420939in}}%
\pgfpathlineto{\pgfqpoint{4.351836in}{2.684270in}}%
\pgfpathlineto{\pgfqpoint{4.352387in}{2.313859in}}%
\pgfpathlineto{\pgfqpoint{4.352475in}{2.460384in}}%
\pgfpathlineto{\pgfqpoint{4.352563in}{2.567793in}}%
\pgfpathlineto{\pgfqpoint{4.352519in}{2.397666in}}%
\pgfpathlineto{\pgfqpoint{4.353136in}{2.483767in}}%
\pgfpathlineto{\pgfqpoint{4.354084in}{2.307849in}}%
\pgfpathlineto{\pgfqpoint{4.353357in}{2.586040in}}%
\pgfpathlineto{\pgfqpoint{4.354238in}{2.496442in}}%
\pgfpathlineto{\pgfqpoint{4.355010in}{2.344890in}}%
\pgfpathlineto{\pgfqpoint{4.354723in}{2.626250in}}%
\pgfpathlineto{\pgfqpoint{4.355120in}{2.433942in}}%
\pgfpathlineto{\pgfqpoint{4.355693in}{2.669629in}}%
\pgfpathlineto{\pgfqpoint{4.355230in}{2.404222in}}%
\pgfpathlineto{\pgfqpoint{4.356222in}{2.428042in}}%
\pgfpathlineto{\pgfqpoint{4.356288in}{2.642968in}}%
\pgfpathlineto{\pgfqpoint{4.357082in}{2.314405in}}%
\pgfpathlineto{\pgfqpoint{4.357346in}{2.499283in}}%
\pgfpathlineto{\pgfqpoint{4.357390in}{2.593689in}}%
\pgfpathlineto{\pgfqpoint{4.357655in}{2.307631in}}%
\pgfpathlineto{\pgfqpoint{4.358448in}{2.518514in}}%
\pgfpathlineto{\pgfqpoint{4.358669in}{2.257478in}}%
\pgfpathlineto{\pgfqpoint{4.359396in}{2.574567in}}%
\pgfpathlineto{\pgfqpoint{4.359572in}{2.451097in}}%
\pgfpathlineto{\pgfqpoint{4.360256in}{2.244694in}}%
\pgfpathlineto{\pgfqpoint{4.359903in}{2.507478in}}%
\pgfpathlineto{\pgfqpoint{4.360829in}{2.268404in}}%
\pgfpathlineto{\pgfqpoint{4.360961in}{2.513706in}}%
\pgfpathlineto{\pgfqpoint{4.361380in}{2.226883in}}%
\pgfpathlineto{\pgfqpoint{4.361931in}{2.326315in}}%
\pgfpathlineto{\pgfqpoint{4.362416in}{2.205795in}}%
\pgfpathlineto{\pgfqpoint{4.362768in}{2.493274in}}%
\pgfpathlineto{\pgfqpoint{4.362967in}{2.434051in}}%
\pgfpathlineto{\pgfqpoint{4.363496in}{2.600791in}}%
\pgfpathlineto{\pgfqpoint{4.363826in}{2.353632in}}%
\pgfpathlineto{\pgfqpoint{4.364091in}{2.518186in}}%
\pgfpathlineto{\pgfqpoint{4.364201in}{2.597513in}}%
\pgfpathlineto{\pgfqpoint{4.364378in}{2.346966in}}%
\pgfpathlineto{\pgfqpoint{4.365061in}{2.462351in}}%
\pgfpathlineto{\pgfqpoint{4.366009in}{2.242508in}}%
\pgfpathlineto{\pgfqpoint{4.365854in}{2.553807in}}%
\pgfpathlineto{\pgfqpoint{4.366185in}{2.339318in}}%
\pgfpathlineto{\pgfqpoint{4.367111in}{2.236717in}}%
\pgfpathlineto{\pgfqpoint{4.366494in}{2.502015in}}%
\pgfpathlineto{\pgfqpoint{4.367265in}{2.366744in}}%
\pgfpathlineto{\pgfqpoint{4.368103in}{2.234860in}}%
\pgfpathlineto{\pgfqpoint{4.368389in}{2.491962in}}%
\pgfpathlineto{\pgfqpoint{4.368742in}{2.338334in}}%
\pgfpathlineto{\pgfqpoint{4.368918in}{2.593361in}}%
\pgfpathlineto{\pgfqpoint{4.369469in}{2.506167in}}%
\pgfpathlineto{\pgfqpoint{4.369535in}{2.667662in}}%
\pgfpathlineto{\pgfqpoint{4.369800in}{2.356910in}}%
\pgfpathlineto{\pgfqpoint{4.370615in}{2.661871in}}%
\pgfpathlineto{\pgfqpoint{4.371277in}{2.383461in}}%
\pgfpathlineto{\pgfqpoint{4.371034in}{2.668099in}}%
\pgfpathlineto{\pgfqpoint{4.371740in}{2.446945in}}%
\pgfpathlineto{\pgfqpoint{4.372092in}{2.591067in}}%
\pgfpathlineto{\pgfqpoint{4.372533in}{2.297797in}}%
\pgfpathlineto{\pgfqpoint{4.372820in}{2.422360in}}%
\pgfpathlineto{\pgfqpoint{4.373569in}{2.242508in}}%
\pgfpathlineto{\pgfqpoint{4.373635in}{2.539384in}}%
\pgfpathlineto{\pgfqpoint{4.373966in}{2.278238in}}%
\pgfpathlineto{\pgfqpoint{4.375112in}{2.514471in}}%
\pgfpathlineto{\pgfqpoint{4.374671in}{2.185908in}}%
\pgfpathlineto{\pgfqpoint{4.375134in}{2.455140in}}%
\pgfpathlineto{\pgfqpoint{4.375332in}{2.329156in}}%
\pgfpathlineto{\pgfqpoint{4.375399in}{2.606254in}}%
\pgfpathlineto{\pgfqpoint{4.376214in}{2.419956in}}%
\pgfpathlineto{\pgfqpoint{4.376236in}{2.563204in}}%
\pgfpathlineto{\pgfqpoint{4.376787in}{2.321180in}}%
\pgfpathlineto{\pgfqpoint{4.377294in}{2.435253in}}%
\pgfpathlineto{\pgfqpoint{4.377316in}{2.352976in}}%
\pgfpathlineto{\pgfqpoint{4.378088in}{2.641001in}}%
\pgfpathlineto{\pgfqpoint{4.378374in}{2.493929in}}%
\pgfpathlineto{\pgfqpoint{4.378573in}{2.735516in}}%
\pgfpathlineto{\pgfqpoint{4.378947in}{2.409248in}}%
\pgfpathlineto{\pgfqpoint{4.379454in}{2.507587in}}%
\pgfpathlineto{\pgfqpoint{4.379763in}{2.370786in}}%
\pgfpathlineto{\pgfqpoint{4.379917in}{2.661215in}}%
\pgfpathlineto{\pgfqpoint{4.380578in}{2.403566in}}%
\pgfpathlineto{\pgfqpoint{4.380777in}{2.676512in}}%
\pgfpathlineto{\pgfqpoint{4.381725in}{2.509336in}}%
\pgfpathlineto{\pgfqpoint{4.382276in}{2.463881in}}%
\pgfpathlineto{\pgfqpoint{4.382342in}{2.691810in}}%
\pgfpathlineto{\pgfqpoint{4.382364in}{2.742509in}}%
\pgfpathlineto{\pgfqpoint{4.383069in}{2.405424in}}%
\pgfpathlineto{\pgfqpoint{4.383356in}{2.528785in}}%
\pgfpathlineto{\pgfqpoint{4.384061in}{2.383898in}}%
\pgfpathlineto{\pgfqpoint{4.383863in}{2.618274in}}%
\pgfpathlineto{\pgfqpoint{4.384480in}{2.479615in}}%
\pgfpathlineto{\pgfqpoint{4.385251in}{2.635319in}}%
\pgfpathlineto{\pgfqpoint{4.384965in}{2.348059in}}%
\pgfpathlineto{\pgfqpoint{4.385339in}{2.579594in}}%
\pgfpathlineto{\pgfqpoint{4.385780in}{2.357019in}}%
\pgfpathlineto{\pgfqpoint{4.386309in}{2.602212in}}%
\pgfpathlineto{\pgfqpoint{4.386442in}{2.498300in}}%
\pgfpathlineto{\pgfqpoint{4.386662in}{2.637067in}}%
\pgfpathlineto{\pgfqpoint{4.386860in}{2.426949in}}%
\pgfpathlineto{\pgfqpoint{4.387191in}{2.477102in}}%
\pgfpathlineto{\pgfqpoint{4.387323in}{2.322054in}}%
\pgfpathlineto{\pgfqpoint{4.388073in}{2.629310in}}%
\pgfpathlineto{\pgfqpoint{4.388293in}{2.423890in}}%
\pgfpathlineto{\pgfqpoint{4.389461in}{2.662526in}}%
\pgfpathlineto{\pgfqpoint{4.388690in}{2.362810in}}%
\pgfpathlineto{\pgfqpoint{4.389505in}{2.525398in}}%
\pgfpathlineto{\pgfqpoint{4.389726in}{2.336149in}}%
\pgfpathlineto{\pgfqpoint{4.390034in}{2.625704in}}%
\pgfpathlineto{\pgfqpoint{4.390608in}{2.480052in}}%
\pgfpathlineto{\pgfqpoint{4.391026in}{2.657500in}}%
\pgfpathlineto{\pgfqpoint{4.391621in}{2.376796in}}%
\pgfpathlineto{\pgfqpoint{4.391688in}{2.381822in}}%
\pgfpathlineto{\pgfqpoint{4.391710in}{2.351774in}}%
\pgfpathlineto{\pgfqpoint{4.392393in}{2.621224in}}%
\pgfpathlineto{\pgfqpoint{4.392591in}{2.518842in}}%
\pgfpathlineto{\pgfqpoint{4.392613in}{2.689515in}}%
\pgfpathlineto{\pgfqpoint{4.393142in}{2.323802in}}%
\pgfpathlineto{\pgfqpoint{4.393715in}{2.628326in}}%
\pgfpathlineto{\pgfqpoint{4.393914in}{2.400179in}}%
\pgfpathlineto{\pgfqpoint{4.394134in}{2.695853in}}%
\pgfpathlineto{\pgfqpoint{4.394818in}{2.598934in}}%
\pgfpathlineto{\pgfqpoint{4.395831in}{2.455030in}}%
\pgfpathlineto{\pgfqpoint{4.395721in}{2.743492in}}%
\pgfpathlineto{\pgfqpoint{4.395876in}{2.633571in}}%
\pgfpathlineto{\pgfqpoint{4.396052in}{2.787090in}}%
\pgfpathlineto{\pgfqpoint{4.396647in}{2.514908in}}%
\pgfpathlineto{\pgfqpoint{4.396934in}{2.587461in}}%
\pgfpathlineto{\pgfqpoint{4.397881in}{2.356691in}}%
\pgfpathlineto{\pgfqpoint{4.397220in}{2.675638in}}%
\pgfpathlineto{\pgfqpoint{4.398080in}{2.509882in}}%
\pgfpathlineto{\pgfqpoint{4.398895in}{2.690826in}}%
\pgfpathlineto{\pgfqpoint{4.398432in}{2.357565in}}%
\pgfpathlineto{\pgfqpoint{4.399204in}{2.550638in}}%
\pgfpathlineto{\pgfqpoint{4.400064in}{2.357565in}}%
\pgfpathlineto{\pgfqpoint{4.400218in}{2.650944in}}%
\pgfpathlineto{\pgfqpoint{4.400372in}{2.428479in}}%
\pgfpathlineto{\pgfqpoint{4.400593in}{2.411652in}}%
\pgfpathlineto{\pgfqpoint{4.401540in}{2.705249in}}%
\pgfpathlineto{\pgfqpoint{4.402025in}{2.472185in}}%
\pgfpathlineto{\pgfqpoint{4.402378in}{2.728195in}}%
\pgfpathlineto{\pgfqpoint{4.402664in}{2.547469in}}%
\pgfpathlineto{\pgfqpoint{4.403722in}{2.723825in}}%
\pgfpathlineto{\pgfqpoint{4.403546in}{2.469016in}}%
\pgfpathlineto{\pgfqpoint{4.403767in}{2.641766in}}%
\pgfpathlineto{\pgfqpoint{4.404274in}{2.447710in}}%
\pgfpathlineto{\pgfqpoint{4.404163in}{2.729179in}}%
\pgfpathlineto{\pgfqpoint{4.404891in}{2.562985in}}%
\pgfpathlineto{\pgfqpoint{4.405486in}{2.765892in}}%
\pgfpathlineto{\pgfqpoint{4.405243in}{2.471857in}}%
\pgfpathlineto{\pgfqpoint{4.406015in}{2.695743in}}%
\pgfpathlineto{\pgfqpoint{4.406786in}{2.447819in}}%
\pgfpathlineto{\pgfqpoint{4.407051in}{2.723278in}}%
\pgfpathlineto{\pgfqpoint{4.407161in}{2.503545in}}%
\pgfpathlineto{\pgfqpoint{4.408285in}{2.826316in}}%
\pgfpathlineto{\pgfqpoint{4.408109in}{2.470109in}}%
\pgfpathlineto{\pgfqpoint{4.408329in}{2.793755in}}%
\pgfpathlineto{\pgfqpoint{4.408616in}{2.501250in}}%
\pgfpathlineto{\pgfqpoint{4.409145in}{2.822164in}}%
\pgfpathlineto{\pgfqpoint{4.409475in}{2.555774in}}%
\pgfpathlineto{\pgfqpoint{4.409828in}{2.527146in}}%
\pgfpathlineto{\pgfqpoint{4.410622in}{2.877015in}}%
\pgfpathlineto{\pgfqpoint{4.411636in}{2.551731in}}%
\pgfpathlineto{\pgfqpoint{4.411724in}{2.778130in}}%
\pgfpathlineto{\pgfqpoint{4.411746in}{2.803042in}}%
\pgfpathlineto{\pgfqpoint{4.411922in}{2.546158in}}%
\pgfpathlineto{\pgfqpoint{4.412716in}{2.794301in}}%
\pgfpathlineto{\pgfqpoint{4.412804in}{2.479178in}}%
\pgfpathlineto{\pgfqpoint{4.413201in}{2.810363in}}%
\pgfpathlineto{\pgfqpoint{4.413862in}{2.570634in}}%
\pgfpathlineto{\pgfqpoint{4.413994in}{2.809271in}}%
\pgfpathlineto{\pgfqpoint{4.414699in}{2.444104in}}%
\pgfpathlineto{\pgfqpoint{4.414986in}{2.705140in}}%
\pgfpathlineto{\pgfqpoint{4.415096in}{2.817466in}}%
\pgfpathlineto{\pgfqpoint{4.415273in}{2.466285in}}%
\pgfpathlineto{\pgfqpoint{4.415802in}{2.611499in}}%
\pgfpathlineto{\pgfqpoint{4.415846in}{2.487264in}}%
\pgfpathlineto{\pgfqpoint{4.416485in}{2.806757in}}%
\pgfpathlineto{\pgfqpoint{4.416926in}{2.542443in}}%
\pgfpathlineto{\pgfqpoint{4.417499in}{2.482893in}}%
\pgfpathlineto{\pgfqpoint{4.417058in}{2.768951in}}%
\pgfpathlineto{\pgfqpoint{4.417741in}{2.607893in}}%
\pgfpathlineto{\pgfqpoint{4.418138in}{2.782500in}}%
\pgfpathlineto{\pgfqpoint{4.418270in}{2.485406in}}%
\pgfpathlineto{\pgfqpoint{4.418887in}{2.700005in}}%
\pgfpathlineto{\pgfqpoint{4.419328in}{2.439187in}}%
\pgfpathlineto{\pgfqpoint{4.419130in}{2.731036in}}%
\pgfpathlineto{\pgfqpoint{4.420034in}{2.534685in}}%
\pgfpathlineto{\pgfqpoint{4.420981in}{2.764909in}}%
\pgfpathlineto{\pgfqpoint{4.420717in}{2.525179in}}%
\pgfpathlineto{\pgfqpoint{4.421158in}{2.647776in}}%
\pgfpathlineto{\pgfqpoint{4.421687in}{2.542771in}}%
\pgfpathlineto{\pgfqpoint{4.421290in}{2.787854in}}%
\pgfpathlineto{\pgfqpoint{4.422238in}{2.609532in}}%
\pgfpathlineto{\pgfqpoint{4.423296in}{2.843252in}}%
\pgfpathlineto{\pgfqpoint{4.422789in}{2.504091in}}%
\pgfpathlineto{\pgfqpoint{4.423318in}{2.584948in}}%
\pgfpathlineto{\pgfqpoint{4.423913in}{2.445743in}}%
\pgfpathlineto{\pgfqpoint{4.423472in}{2.747754in}}%
\pgfpathlineto{\pgfqpoint{4.424398in}{2.624393in}}%
\pgfpathlineto{\pgfqpoint{4.424706in}{2.681757in}}%
\pgfpathlineto{\pgfqpoint{4.425235in}{2.403675in}}%
\pgfpathlineto{\pgfqpoint{4.425368in}{2.485953in}}%
\pgfpathlineto{\pgfqpoint{4.425654in}{2.287089in}}%
\pgfpathlineto{\pgfqpoint{4.426139in}{2.612701in}}%
\pgfpathlineto{\pgfqpoint{4.426448in}{2.580031in}}%
\pgfpathlineto{\pgfqpoint{4.427175in}{2.709074in}}%
\pgfpathlineto{\pgfqpoint{4.427087in}{2.432631in}}%
\pgfpathlineto{\pgfqpoint{4.427550in}{2.608221in}}%
\pgfpathlineto{\pgfqpoint{4.428365in}{2.463772in}}%
\pgfpathlineto{\pgfqpoint{4.428189in}{2.747317in}}%
\pgfpathlineto{\pgfqpoint{4.428630in}{2.598497in}}%
\pgfpathlineto{\pgfqpoint{4.428961in}{2.726556in}}%
\pgfpathlineto{\pgfqpoint{4.429446in}{2.403238in}}%
\pgfpathlineto{\pgfqpoint{4.429710in}{2.641220in}}%
\pgfpathlineto{\pgfqpoint{4.429908in}{2.457544in}}%
\pgfpathlineto{\pgfqpoint{4.430085in}{2.796159in}}%
\pgfpathlineto{\pgfqpoint{4.430812in}{2.582981in}}%
\pgfpathlineto{\pgfqpoint{4.431319in}{2.793427in}}%
\pgfpathlineto{\pgfqpoint{4.431672in}{2.477976in}}%
\pgfpathlineto{\pgfqpoint{4.431914in}{2.593033in}}%
\pgfpathlineto{\pgfqpoint{4.432245in}{2.446398in}}%
\pgfpathlineto{\pgfqpoint{4.432024in}{2.770700in}}%
\pgfpathlineto{\pgfqpoint{4.432642in}{2.630293in}}%
\pgfpathlineto{\pgfqpoint{4.432664in}{2.764472in}}%
\pgfpathlineto{\pgfqpoint{4.432818in}{2.488794in}}%
\pgfpathlineto{\pgfqpoint{4.433744in}{2.585494in}}%
\pgfpathlineto{\pgfqpoint{4.434603in}{2.471311in}}%
\pgfpathlineto{\pgfqpoint{4.434780in}{2.728195in}}%
\pgfpathlineto{\pgfqpoint{4.434846in}{2.579484in}}%
\pgfpathlineto{\pgfqpoint{4.435397in}{2.800092in}}%
\pgfpathlineto{\pgfqpoint{4.435485in}{2.465301in}}%
\pgfpathlineto{\pgfqpoint{4.435816in}{2.512395in}}%
\pgfpathlineto{\pgfqpoint{4.435838in}{2.447382in}}%
\pgfpathlineto{\pgfqpoint{4.436234in}{2.780097in}}%
\pgfpathlineto{\pgfqpoint{4.436896in}{2.551622in}}%
\pgfpathlineto{\pgfqpoint{4.437292in}{2.732238in}}%
\pgfpathlineto{\pgfqpoint{4.437072in}{2.505184in}}%
\pgfpathlineto{\pgfqpoint{4.437954in}{2.552277in}}%
\pgfpathlineto{\pgfqpoint{4.439056in}{2.373409in}}%
\pgfpathlineto{\pgfqpoint{4.438769in}{2.677168in}}%
\pgfpathlineto{\pgfqpoint{4.439078in}{2.468361in}}%
\pgfpathlineto{\pgfqpoint{4.439937in}{2.344235in}}%
\pgfpathlineto{\pgfqpoint{4.439673in}{2.684817in}}%
\pgfpathlineto{\pgfqpoint{4.440158in}{2.547360in}}%
\pgfpathlineto{\pgfqpoint{4.440511in}{2.277801in}}%
\pgfpathlineto{\pgfqpoint{4.441216in}{2.598715in}}%
\pgfpathlineto{\pgfqpoint{4.441238in}{2.550420in}}%
\pgfpathlineto{\pgfqpoint{4.441436in}{2.304899in}}%
\pgfpathlineto{\pgfqpoint{4.442318in}{2.611062in}}%
\pgfpathlineto{\pgfqpoint{4.443398in}{2.339318in}}%
\pgfpathlineto{\pgfqpoint{4.443134in}{2.710931in}}%
\pgfpathlineto{\pgfqpoint{4.443420in}{2.529441in}}%
\pgfpathlineto{\pgfqpoint{4.444081in}{2.429134in}}%
\pgfpathlineto{\pgfqpoint{4.443861in}{2.687985in}}%
\pgfpathlineto{\pgfqpoint{4.444412in}{2.578719in}}%
\pgfpathlineto{\pgfqpoint{4.444765in}{2.740542in}}%
\pgfpathlineto{\pgfqpoint{4.445007in}{2.386084in}}%
\pgfpathlineto{\pgfqpoint{4.445492in}{2.495131in}}%
\pgfpathlineto{\pgfqpoint{4.445558in}{2.381167in}}%
\pgfpathlineto{\pgfqpoint{4.445757in}{2.781080in}}%
\pgfpathlineto{\pgfqpoint{4.446550in}{2.602539in}}%
\pgfpathlineto{\pgfqpoint{4.446704in}{2.516110in}}%
\pgfpathlineto{\pgfqpoint{4.446837in}{2.735516in}}%
\pgfpathlineto{\pgfqpoint{4.447079in}{2.570634in}}%
\pgfpathlineto{\pgfqpoint{4.447586in}{2.749721in}}%
\pgfpathlineto{\pgfqpoint{4.448093in}{2.505074in}}%
\pgfpathlineto{\pgfqpoint{4.448181in}{2.640018in}}%
\pgfpathlineto{\pgfqpoint{4.448534in}{2.774305in}}%
\pgfpathlineto{\pgfqpoint{4.448335in}{2.529550in}}%
\pgfpathlineto{\pgfqpoint{4.449041in}{2.626359in}}%
\pgfpathlineto{\pgfqpoint{4.449812in}{2.469344in}}%
\pgfpathlineto{\pgfqpoint{4.449945in}{2.763379in}}%
\pgfpathlineto{\pgfqpoint{4.450099in}{2.564733in}}%
\pgfpathlineto{\pgfqpoint{4.450121in}{2.836259in}}%
\pgfpathlineto{\pgfqpoint{4.450870in}{2.293754in}}%
\pgfpathlineto{\pgfqpoint{4.451201in}{2.492946in}}%
\pgfpathlineto{\pgfqpoint{4.451796in}{2.306429in}}%
\pgfpathlineto{\pgfqpoint{4.451620in}{2.567684in}}%
\pgfpathlineto{\pgfqpoint{4.452237in}{2.549108in}}%
\pgfpathlineto{\pgfqpoint{4.452942in}{2.733112in}}%
\pgfpathlineto{\pgfqpoint{4.452678in}{2.496224in}}%
\pgfpathlineto{\pgfqpoint{4.453383in}{2.619694in}}%
\pgfpathlineto{\pgfqpoint{4.453846in}{2.455030in}}%
\pgfpathlineto{\pgfqpoint{4.454177in}{2.790477in}}%
\pgfpathlineto{\pgfqpoint{4.454441in}{2.592268in}}%
\pgfpathlineto{\pgfqpoint{4.454573in}{2.708200in}}%
\pgfpathlineto{\pgfqpoint{4.455169in}{2.371114in}}%
\pgfpathlineto{\pgfqpoint{4.455543in}{2.578829in}}%
\pgfpathlineto{\pgfqpoint{4.455808in}{2.717815in}}%
\pgfpathlineto{\pgfqpoint{4.455874in}{2.434161in}}%
\pgfpathlineto{\pgfqpoint{4.456050in}{2.498628in}}%
\pgfpathlineto{\pgfqpoint{4.456998in}{2.380074in}}%
\pgfpathlineto{\pgfqpoint{4.456778in}{2.635538in}}%
\pgfpathlineto{\pgfqpoint{4.457152in}{2.508024in}}%
\pgfpathlineto{\pgfqpoint{4.457262in}{2.483330in}}%
\pgfpathlineto{\pgfqpoint{4.457615in}{2.699895in}}%
\pgfpathlineto{\pgfqpoint{4.458034in}{2.545284in}}%
\pgfpathlineto{\pgfqpoint{4.458100in}{2.673781in}}%
\pgfpathlineto{\pgfqpoint{4.458321in}{2.416678in}}%
\pgfpathlineto{\pgfqpoint{4.459114in}{2.467596in}}%
\pgfpathlineto{\pgfqpoint{4.459356in}{2.706561in}}%
\pgfpathlineto{\pgfqpoint{4.459180in}{2.419519in}}%
\pgfpathlineto{\pgfqpoint{4.460238in}{2.583309in}}%
\pgfpathlineto{\pgfqpoint{4.461098in}{2.459838in}}%
\pgfpathlineto{\pgfqpoint{4.460966in}{2.634991in}}%
\pgfpathlineto{\pgfqpoint{4.461362in}{2.464318in}}%
\pgfpathlineto{\pgfqpoint{4.462178in}{2.735407in}}%
\pgfpathlineto{\pgfqpoint{4.462486in}{2.636849in}}%
\pgfpathlineto{\pgfqpoint{4.463082in}{2.531080in}}%
\pgfpathlineto{\pgfqpoint{4.463214in}{2.693886in}}%
\pgfpathlineto{\pgfqpoint{4.463589in}{2.570852in}}%
\pgfpathlineto{\pgfqpoint{4.464007in}{2.761412in}}%
\pgfpathlineto{\pgfqpoint{4.464669in}{2.485953in}}%
\pgfpathlineto{\pgfqpoint{4.465220in}{2.811347in}}%
\pgfpathlineto{\pgfqpoint{4.465903in}{2.691482in}}%
\pgfpathlineto{\pgfqpoint{4.466520in}{2.507697in}}%
\pgfpathlineto{\pgfqpoint{4.466234in}{2.836150in}}%
\pgfpathlineto{\pgfqpoint{4.467027in}{2.553588in}}%
\pgfpathlineto{\pgfqpoint{4.467181in}{2.760866in}}%
\pgfpathlineto{\pgfqpoint{4.467093in}{2.480708in}}%
\pgfpathlineto{\pgfqpoint{4.468107in}{2.581560in}}%
\pgfpathlineto{\pgfqpoint{4.468129in}{2.391110in}}%
\pgfpathlineto{\pgfqpoint{4.468901in}{2.727321in}}%
\pgfpathlineto{\pgfqpoint{4.469231in}{2.434816in}}%
\pgfpathlineto{\pgfqpoint{4.469253in}{2.386958in}}%
\pgfpathlineto{\pgfqpoint{4.469650in}{2.685582in}}%
\pgfpathlineto{\pgfqpoint{4.470267in}{2.542552in}}%
\pgfpathlineto{\pgfqpoint{4.470642in}{2.672797in}}%
\pgfpathlineto{\pgfqpoint{4.470554in}{2.454921in}}%
\pgfpathlineto{\pgfqpoint{4.471369in}{2.645372in}}%
\pgfpathlineto{\pgfqpoint{4.472185in}{2.448037in}}%
\pgfpathlineto{\pgfqpoint{4.471766in}{2.669192in}}%
\pgfpathlineto{\pgfqpoint{4.472471in}{2.658484in}}%
\pgfpathlineto{\pgfqpoint{4.472560in}{2.807413in}}%
\pgfpathlineto{\pgfqpoint{4.472582in}{2.569104in}}%
\pgfpathlineto{\pgfqpoint{4.472692in}{2.691045in}}%
\pgfpathlineto{\pgfqpoint{4.473507in}{2.475354in}}%
\pgfpathlineto{\pgfqpoint{4.473000in}{2.829048in}}%
\pgfpathlineto{\pgfqpoint{4.473816in}{2.576643in}}%
\pgfpathlineto{\pgfqpoint{4.473838in}{2.578173in}}%
\pgfpathlineto{\pgfqpoint{4.473860in}{2.550857in}}%
\pgfpathlineto{\pgfqpoint{4.474940in}{2.796596in}}%
\pgfpathlineto{\pgfqpoint{4.474411in}{2.461368in}}%
\pgfpathlineto{\pgfqpoint{4.474962in}{2.631713in}}%
\pgfpathlineto{\pgfqpoint{4.475094in}{2.493711in}}%
\pgfpathlineto{\pgfqpoint{4.475601in}{2.792553in}}%
\pgfpathlineto{\pgfqpoint{4.476064in}{2.634882in}}%
\pgfpathlineto{\pgfqpoint{4.476219in}{2.778021in}}%
\pgfpathlineto{\pgfqpoint{4.476439in}{2.473278in}}%
\pgfpathlineto{\pgfqpoint{4.477166in}{2.730708in}}%
\pgfpathlineto{\pgfqpoint{4.477872in}{2.435363in}}%
\pgfpathlineto{\pgfqpoint{4.477541in}{2.790368in}}%
\pgfpathlineto{\pgfqpoint{4.478291in}{2.518623in}}%
\pgfpathlineto{\pgfqpoint{4.478401in}{2.343907in}}%
\pgfpathlineto{\pgfqpoint{4.478709in}{2.696945in}}%
\pgfpathlineto{\pgfqpoint{4.479349in}{2.630074in}}%
\pgfpathlineto{\pgfqpoint{4.479613in}{2.453719in}}%
\pgfpathlineto{\pgfqpoint{4.479988in}{2.762177in}}%
\pgfpathlineto{\pgfqpoint{4.480473in}{2.531080in}}%
\pgfpathlineto{\pgfqpoint{4.480869in}{2.786215in}}%
\pgfpathlineto{\pgfqpoint{4.481376in}{2.478304in}}%
\pgfpathlineto{\pgfqpoint{4.481597in}{2.629310in}}%
\pgfpathlineto{\pgfqpoint{4.482258in}{2.566809in}}%
\pgfpathlineto{\pgfqpoint{4.482523in}{2.809926in}}%
\pgfpathlineto{\pgfqpoint{4.482633in}{2.702081in}}%
\pgfpathlineto{\pgfqpoint{4.482677in}{2.819214in}}%
\pgfpathlineto{\pgfqpoint{4.483404in}{2.387941in}}%
\pgfpathlineto{\pgfqpoint{4.483691in}{2.589865in}}%
\pgfpathlineto{\pgfqpoint{4.483713in}{2.430992in}}%
\pgfpathlineto{\pgfqpoint{4.484374in}{2.731255in}}%
\pgfpathlineto{\pgfqpoint{4.484793in}{2.586696in}}%
\pgfpathlineto{\pgfqpoint{4.485035in}{2.488357in}}%
\pgfpathlineto{\pgfqpoint{4.485785in}{2.726556in}}%
\pgfpathlineto{\pgfqpoint{4.485851in}{2.620022in}}%
\pgfpathlineto{\pgfqpoint{4.486909in}{2.871771in}}%
\pgfpathlineto{\pgfqpoint{4.486336in}{2.482675in}}%
\pgfpathlineto{\pgfqpoint{4.486953in}{2.628217in}}%
\pgfpathlineto{\pgfqpoint{4.487284in}{2.864122in}}%
\pgfpathlineto{\pgfqpoint{4.487857in}{2.501796in}}%
\pgfpathlineto{\pgfqpoint{4.488055in}{2.649087in}}%
\pgfpathlineto{\pgfqpoint{4.489157in}{2.449676in}}%
\pgfpathlineto{\pgfqpoint{4.488276in}{2.723169in}}%
\pgfpathlineto{\pgfqpoint{4.489201in}{2.603086in}}%
\pgfpathlineto{\pgfqpoint{4.490215in}{2.710057in}}%
\pgfpathlineto{\pgfqpoint{4.489620in}{2.434051in}}%
\pgfpathlineto{\pgfqpoint{4.490259in}{2.555009in}}%
\pgfpathlineto{\pgfqpoint{4.490766in}{2.369803in}}%
\pgfpathlineto{\pgfqpoint{4.491273in}{2.667116in}}%
\pgfpathlineto{\pgfqpoint{4.491361in}{2.574349in}}%
\pgfpathlineto{\pgfqpoint{4.491516in}{2.742837in}}%
\pgfpathlineto{\pgfqpoint{4.491428in}{2.484642in}}%
\pgfpathlineto{\pgfqpoint{4.492442in}{2.501578in}}%
\pgfpathlineto{\pgfqpoint{4.492750in}{2.466940in}}%
\pgfpathlineto{\pgfqpoint{4.492816in}{2.666679in}}%
\pgfpathlineto{\pgfqpoint{4.493323in}{2.527364in}}%
\pgfpathlineto{\pgfqpoint{4.493786in}{2.461914in}}%
\pgfpathlineto{\pgfqpoint{4.494469in}{2.847732in}}%
\pgfpathlineto{\pgfqpoint{4.494888in}{2.570087in}}%
\pgfpathlineto{\pgfqpoint{4.494822in}{2.858549in}}%
\pgfpathlineto{\pgfqpoint{4.495616in}{2.690826in}}%
\pgfpathlineto{\pgfqpoint{4.496431in}{2.793973in}}%
\pgfpathlineto{\pgfqpoint{4.496145in}{2.532172in}}%
\pgfpathlineto{\pgfqpoint{4.496629in}{2.628326in}}%
\pgfpathlineto{\pgfqpoint{4.497688in}{2.471420in}}%
\pgfpathlineto{\pgfqpoint{4.497048in}{2.831124in}}%
\pgfpathlineto{\pgfqpoint{4.497710in}{2.630839in}}%
\pgfpathlineto{\pgfqpoint{4.497776in}{2.747098in}}%
\pgfpathlineto{\pgfqpoint{4.498106in}{2.461259in}}%
\pgfpathlineto{\pgfqpoint{4.498790in}{2.626032in}}%
\pgfpathlineto{\pgfqpoint{4.499848in}{2.474371in}}%
\pgfpathlineto{\pgfqpoint{4.498944in}{2.765564in}}%
\pgfpathlineto{\pgfqpoint{4.499936in}{2.508134in}}%
\pgfpathlineto{\pgfqpoint{4.501082in}{2.720765in}}%
\pgfpathlineto{\pgfqpoint{4.500465in}{2.481145in}}%
\pgfpathlineto{\pgfqpoint{4.501126in}{2.681648in}}%
\pgfpathlineto{\pgfqpoint{4.501523in}{2.448912in}}%
\pgfpathlineto{\pgfqpoint{4.502052in}{2.791569in}}%
\pgfpathlineto{\pgfqpoint{4.502228in}{2.669629in}}%
\pgfpathlineto{\pgfqpoint{4.502867in}{2.840848in}}%
\pgfpathlineto{\pgfqpoint{4.502713in}{2.537635in}}%
\pgfpathlineto{\pgfqpoint{4.503286in}{2.806539in}}%
\pgfpathlineto{\pgfqpoint{4.503507in}{2.522120in}}%
\pgfpathlineto{\pgfqpoint{4.504410in}{2.677387in}}%
\pgfpathlineto{\pgfqpoint{4.505226in}{2.754310in}}%
\pgfpathlineto{\pgfqpoint{4.505072in}{2.432085in}}%
\pgfpathlineto{\pgfqpoint{4.505270in}{2.571180in}}%
\pgfpathlineto{\pgfqpoint{4.506196in}{2.423343in}}%
\pgfpathlineto{\pgfqpoint{4.505843in}{2.684270in}}%
\pgfpathlineto{\pgfqpoint{4.506328in}{2.636958in}}%
\pgfpathlineto{\pgfqpoint{4.506901in}{2.472404in}}%
\pgfpathlineto{\pgfqpoint{4.507474in}{2.830687in}}%
\pgfpathlineto{\pgfqpoint{4.508554in}{2.476337in}}%
\pgfpathlineto{\pgfqpoint{4.508025in}{2.885757in}}%
\pgfpathlineto{\pgfqpoint{4.508620in}{2.548344in}}%
\pgfpathlineto{\pgfqpoint{4.508819in}{2.765892in}}%
\pgfpathlineto{\pgfqpoint{4.509502in}{2.496879in}}%
\pgfpathlineto{\pgfqpoint{4.509767in}{2.630839in}}%
\pgfpathlineto{\pgfqpoint{4.509943in}{2.465301in}}%
\pgfpathlineto{\pgfqpoint{4.509833in}{2.708746in}}%
\pgfpathlineto{\pgfqpoint{4.510913in}{2.536652in}}%
\pgfpathlineto{\pgfqpoint{4.511750in}{2.728851in}}%
\pgfpathlineto{\pgfqpoint{4.511111in}{2.465192in}}%
\pgfpathlineto{\pgfqpoint{4.512059in}{2.607784in}}%
\pgfpathlineto{\pgfqpoint{4.512544in}{2.385974in}}%
\pgfpathlineto{\pgfqpoint{4.512720in}{2.681648in}}%
\pgfpathlineto{\pgfqpoint{4.513161in}{2.640018in}}%
\pgfpathlineto{\pgfqpoint{4.513183in}{2.661652in}}%
\pgfpathlineto{\pgfqpoint{4.513492in}{2.378107in}}%
\pgfpathlineto{\pgfqpoint{4.514043in}{2.576534in}}%
\pgfpathlineto{\pgfqpoint{4.514087in}{2.430227in}}%
\pgfpathlineto{\pgfqpoint{4.514880in}{2.731364in}}%
\pgfpathlineto{\pgfqpoint{4.515145in}{2.589537in}}%
\pgfpathlineto{\pgfqpoint{4.515828in}{2.791132in}}%
\pgfpathlineto{\pgfqpoint{4.516093in}{2.436783in}}%
\pgfpathlineto{\pgfqpoint{4.516225in}{2.538073in}}%
\pgfpathlineto{\pgfqpoint{4.517261in}{2.426731in}}%
\pgfpathlineto{\pgfqpoint{4.517129in}{2.791242in}}%
\pgfpathlineto{\pgfqpoint{4.517305in}{2.553698in}}%
\pgfpathlineto{\pgfqpoint{4.518407in}{2.766001in}}%
\pgfpathlineto{\pgfqpoint{4.517525in}{2.467377in}}%
\pgfpathlineto{\pgfqpoint{4.518451in}{2.737155in}}%
\pgfpathlineto{\pgfqpoint{4.519333in}{2.439405in}}%
\pgfpathlineto{\pgfqpoint{4.519575in}{2.566154in}}%
\pgfpathlineto{\pgfqpoint{4.520281in}{2.779878in}}%
\pgfpathlineto{\pgfqpoint{4.520391in}{2.487373in}}%
\pgfpathlineto{\pgfqpoint{4.520699in}{2.639253in}}%
\pgfpathlineto{\pgfqpoint{4.521647in}{2.514690in}}%
\pgfpathlineto{\pgfqpoint{4.520788in}{2.766657in}}%
\pgfpathlineto{\pgfqpoint{4.521757in}{2.644279in}}%
\pgfpathlineto{\pgfqpoint{4.522793in}{2.766657in}}%
\pgfpathlineto{\pgfqpoint{4.522242in}{2.483767in}}%
\pgfpathlineto{\pgfqpoint{4.522859in}{2.741416in}}%
\pgfpathlineto{\pgfqpoint{4.523300in}{2.503982in}}%
\pgfpathlineto{\pgfqpoint{4.522992in}{2.825114in}}%
\pgfpathlineto{\pgfqpoint{4.523962in}{2.660013in}}%
\pgfpathlineto{\pgfqpoint{4.524402in}{2.745241in}}%
\pgfpathlineto{\pgfqpoint{4.524601in}{2.489121in}}%
\pgfpathlineto{\pgfqpoint{4.525020in}{2.656626in}}%
\pgfpathlineto{\pgfqpoint{4.526188in}{2.444869in}}%
\pgfpathlineto{\pgfqpoint{4.525350in}{2.831452in}}%
\pgfpathlineto{\pgfqpoint{4.526210in}{2.461914in}}%
\pgfpathlineto{\pgfqpoint{4.526695in}{2.712133in}}%
\pgfpathlineto{\pgfqpoint{4.527268in}{2.451862in}}%
\pgfpathlineto{\pgfqpoint{4.527378in}{2.656080in}}%
\pgfpathlineto{\pgfqpoint{4.527400in}{2.658374in}}%
\pgfpathlineto{\pgfqpoint{4.527422in}{2.590739in}}%
\pgfpathlineto{\pgfqpoint{4.527907in}{2.449239in}}%
\pgfpathlineto{\pgfqpoint{4.527797in}{2.722404in}}%
\pgfpathlineto{\pgfqpoint{4.528524in}{2.582762in}}%
\pgfpathlineto{\pgfqpoint{4.529604in}{2.768077in}}%
\pgfpathlineto{\pgfqpoint{4.529075in}{2.421595in}}%
\pgfpathlineto{\pgfqpoint{4.529648in}{2.690826in}}%
\pgfpathlineto{\pgfqpoint{4.529781in}{2.468470in}}%
\pgfpathlineto{\pgfqpoint{4.529957in}{2.801076in}}%
\pgfpathlineto{\pgfqpoint{4.530773in}{2.533811in}}%
\pgfpathlineto{\pgfqpoint{4.531346in}{2.757369in}}%
\pgfpathlineto{\pgfqpoint{4.531500in}{2.477758in}}%
\pgfpathlineto{\pgfqpoint{4.531875in}{2.612483in}}%
\pgfpathlineto{\pgfqpoint{4.532867in}{2.794629in}}%
\pgfpathlineto{\pgfqpoint{4.532227in}{2.539056in}}%
\pgfpathlineto{\pgfqpoint{4.532977in}{2.658156in}}%
\pgfpathlineto{\pgfqpoint{4.533418in}{2.520044in}}%
\pgfpathlineto{\pgfqpoint{4.533682in}{2.744804in}}%
\pgfpathlineto{\pgfqpoint{4.534101in}{2.539821in}}%
\pgfpathlineto{\pgfqpoint{4.534299in}{2.882697in}}%
\pgfpathlineto{\pgfqpoint{4.534365in}{2.535232in}}%
\pgfpathlineto{\pgfqpoint{4.535225in}{2.634336in}}%
\pgfpathlineto{\pgfqpoint{4.535710in}{2.489996in}}%
\pgfpathlineto{\pgfqpoint{4.535490in}{2.792881in}}%
\pgfpathlineto{\pgfqpoint{4.536327in}{2.652692in}}%
\pgfpathlineto{\pgfqpoint{4.537517in}{2.376468in}}%
\pgfpathlineto{\pgfqpoint{4.536680in}{2.762833in}}%
\pgfpathlineto{\pgfqpoint{4.537584in}{2.446726in}}%
\pgfpathlineto{\pgfqpoint{4.538619in}{2.739668in}}%
\pgfpathlineto{\pgfqpoint{4.537650in}{2.432085in}}%
\pgfpathlineto{\pgfqpoint{4.538708in}{2.543427in}}%
\pgfpathlineto{\pgfqpoint{4.538730in}{2.496551in}}%
\pgfpathlineto{\pgfqpoint{4.538818in}{2.815499in}}%
\pgfpathlineto{\pgfqpoint{4.539766in}{2.596202in}}%
\pgfpathlineto{\pgfqpoint{4.540713in}{2.802824in}}%
\pgfpathlineto{\pgfqpoint{4.540008in}{2.554353in}}%
\pgfpathlineto{\pgfqpoint{4.540824in}{2.621005in}}%
\pgfpathlineto{\pgfqpoint{4.541485in}{2.439624in}}%
\pgfpathlineto{\pgfqpoint{4.541198in}{2.812221in}}%
\pgfpathlineto{\pgfqpoint{4.541904in}{2.797907in}}%
\pgfpathlineto{\pgfqpoint{4.542477in}{2.489996in}}%
\pgfpathlineto{\pgfqpoint{4.541948in}{2.834402in}}%
\pgfpathlineto{\pgfqpoint{4.543028in}{2.605599in}}%
\pgfpathlineto{\pgfqpoint{4.543336in}{2.759555in}}%
\pgfpathlineto{\pgfqpoint{4.543623in}{2.474480in}}%
\pgfpathlineto{\pgfqpoint{4.544174in}{2.715630in}}%
\pgfpathlineto{\pgfqpoint{4.544703in}{2.523540in}}%
\pgfpathlineto{\pgfqpoint{4.544549in}{2.766875in}}%
\pgfpathlineto{\pgfqpoint{4.545342in}{2.595546in}}%
\pgfpathlineto{\pgfqpoint{4.546158in}{2.754965in}}%
\pgfpathlineto{\pgfqpoint{4.545982in}{2.522338in}}%
\pgfpathlineto{\pgfqpoint{4.546466in}{2.650616in}}%
\pgfpathlineto{\pgfqpoint{4.546797in}{2.501796in}}%
\pgfpathlineto{\pgfqpoint{4.546555in}{2.738794in}}%
\pgfpathlineto{\pgfqpoint{4.547524in}{2.694651in}}%
\pgfpathlineto{\pgfqpoint{4.547546in}{2.789712in}}%
\pgfpathlineto{\pgfqpoint{4.548362in}{2.498081in}}%
\pgfpathlineto{\pgfqpoint{4.548605in}{2.670394in}}%
\pgfpathlineto{\pgfqpoint{4.549618in}{2.504746in}}%
\pgfpathlineto{\pgfqpoint{4.549354in}{2.809052in}}%
\pgfpathlineto{\pgfqpoint{4.549707in}{2.567793in}}%
\pgfpathlineto{\pgfqpoint{4.550412in}{2.896246in}}%
\pgfpathlineto{\pgfqpoint{4.550214in}{2.549545in}}%
\pgfpathlineto{\pgfqpoint{4.550831in}{2.785778in}}%
\pgfpathlineto{\pgfqpoint{4.550985in}{2.554681in}}%
\pgfpathlineto{\pgfqpoint{4.551580in}{2.815936in}}%
\pgfpathlineto{\pgfqpoint{4.551955in}{2.669410in}}%
\pgfpathlineto{\pgfqpoint{4.551999in}{2.779878in}}%
\pgfpathlineto{\pgfqpoint{4.552770in}{2.423671in}}%
\pgfpathlineto{\pgfqpoint{4.553057in}{2.654659in}}%
\pgfpathlineto{\pgfqpoint{4.553344in}{2.430992in}}%
\pgfpathlineto{\pgfqpoint{4.553895in}{2.737811in}}%
\pgfpathlineto{\pgfqpoint{4.554181in}{2.559161in}}%
\pgfpathlineto{\pgfqpoint{4.554446in}{2.823694in}}%
\pgfpathlineto{\pgfqpoint{4.554953in}{2.420939in}}%
\pgfpathlineto{\pgfqpoint{4.555305in}{2.705249in}}%
\pgfpathlineto{\pgfqpoint{4.555415in}{2.425638in}}%
\pgfpathlineto{\pgfqpoint{4.555548in}{2.802387in}}%
\pgfpathlineto{\pgfqpoint{4.556496in}{2.638488in}}%
\pgfpathlineto{\pgfqpoint{4.556518in}{2.771792in}}%
\pgfpathlineto{\pgfqpoint{4.557245in}{2.548344in}}%
\pgfpathlineto{\pgfqpoint{4.557598in}{2.661980in}}%
\pgfpathlineto{\pgfqpoint{4.558501in}{2.467924in}}%
\pgfpathlineto{\pgfqpoint{4.557928in}{2.847295in}}%
\pgfpathlineto{\pgfqpoint{4.558678in}{2.636740in}}%
\pgfpathlineto{\pgfqpoint{4.559185in}{2.813532in}}%
\pgfpathlineto{\pgfqpoint{4.558876in}{2.453391in}}%
\pgfpathlineto{\pgfqpoint{4.559780in}{2.765236in}}%
\pgfpathlineto{\pgfqpoint{4.560309in}{2.582872in}}%
\pgfpathlineto{\pgfqpoint{4.560838in}{2.881386in}}%
\pgfpathlineto{\pgfqpoint{4.560882in}{2.774961in}}%
\pgfpathlineto{\pgfqpoint{4.561279in}{2.557850in}}%
\pgfpathlineto{\pgfqpoint{4.561896in}{2.879201in}}%
\pgfpathlineto{\pgfqpoint{4.562028in}{2.639908in}}%
\pgfpathlineto{\pgfqpoint{4.562535in}{2.836259in}}%
\pgfpathlineto{\pgfqpoint{4.562910in}{2.540149in}}%
\pgfpathlineto{\pgfqpoint{4.563108in}{2.626469in}}%
\pgfpathlineto{\pgfqpoint{4.563990in}{2.468252in}}%
\pgfpathlineto{\pgfqpoint{4.563703in}{2.774087in}}%
\pgfpathlineto{\pgfqpoint{4.564166in}{2.651600in}}%
\pgfpathlineto{\pgfqpoint{4.565048in}{2.848716in}}%
\pgfpathlineto{\pgfqpoint{4.564497in}{2.517968in}}%
\pgfpathlineto{\pgfqpoint{4.565268in}{2.718034in}}%
\pgfpathlineto{\pgfqpoint{4.565819in}{2.604397in}}%
\pgfpathlineto{\pgfqpoint{4.566150in}{2.924437in}}%
\pgfpathlineto{\pgfqpoint{4.566326in}{2.734314in}}%
\pgfpathlineto{\pgfqpoint{4.566789in}{2.639471in}}%
\pgfpathlineto{\pgfqpoint{4.567450in}{2.904878in}}%
\pgfpathlineto{\pgfqpoint{4.568575in}{2.659248in}}%
\pgfpathlineto{\pgfqpoint{4.567671in}{2.974371in}}%
\pgfpathlineto{\pgfqpoint{4.568597in}{2.712461in}}%
\pgfpathlineto{\pgfqpoint{4.568663in}{2.875595in}}%
\pgfpathlineto{\pgfqpoint{4.569478in}{2.583636in}}%
\pgfpathlineto{\pgfqpoint{4.569633in}{2.662963in}}%
\pgfpathlineto{\pgfqpoint{4.569677in}{2.551840in}}%
\pgfpathlineto{\pgfqpoint{4.570250in}{2.886084in}}%
\pgfpathlineto{\pgfqpoint{4.570735in}{2.669082in}}%
\pgfpathlineto{\pgfqpoint{4.571440in}{2.862702in}}%
\pgfpathlineto{\pgfqpoint{4.571153in}{2.631604in}}%
\pgfpathlineto{\pgfqpoint{4.571837in}{2.757588in}}%
\pgfpathlineto{\pgfqpoint{4.572917in}{2.627015in}}%
\pgfpathlineto{\pgfqpoint{4.572740in}{2.918209in}}%
\pgfpathlineto{\pgfqpoint{4.572939in}{2.701097in}}%
\pgfpathlineto{\pgfqpoint{4.572961in}{2.868056in}}%
\pgfpathlineto{\pgfqpoint{4.573115in}{2.522010in}}%
\pgfpathlineto{\pgfqpoint{4.574041in}{2.656298in}}%
\pgfpathlineto{\pgfqpoint{4.575033in}{2.811565in}}%
\pgfpathlineto{\pgfqpoint{4.574614in}{2.484314in}}%
\pgfpathlineto{\pgfqpoint{4.575165in}{2.762068in}}%
\pgfpathlineto{\pgfqpoint{4.575826in}{2.551294in}}%
\pgfpathlineto{\pgfqpoint{4.575430in}{2.936784in}}%
\pgfpathlineto{\pgfqpoint{4.576289in}{2.628545in}}%
\pgfpathlineto{\pgfqpoint{4.576995in}{2.760210in}}%
\pgfpathlineto{\pgfqpoint{4.577281in}{2.485734in}}%
\pgfpathlineto{\pgfqpoint{4.577369in}{2.630074in}}%
\pgfpathlineto{\pgfqpoint{4.577480in}{2.526381in}}%
\pgfpathlineto{\pgfqpoint{4.578273in}{2.766875in}}%
\pgfpathlineto{\pgfqpoint{4.578427in}{2.632915in}}%
\pgfpathlineto{\pgfqpoint{4.578802in}{2.791788in}}%
\pgfpathlineto{\pgfqpoint{4.578560in}{2.518405in}}%
\pgfpathlineto{\pgfqpoint{4.579507in}{2.640127in}}%
\pgfpathlineto{\pgfqpoint{4.579838in}{2.569323in}}%
\pgfpathlineto{\pgfqpoint{4.580125in}{2.849371in}}%
\pgfpathlineto{\pgfqpoint{4.580609in}{2.637067in}}%
\pgfpathlineto{\pgfqpoint{4.581094in}{2.843252in}}%
\pgfpathlineto{\pgfqpoint{4.581271in}{2.516984in}}%
\pgfpathlineto{\pgfqpoint{4.581579in}{2.666460in}}%
\pgfpathlineto{\pgfqpoint{4.581800in}{2.496333in}}%
\pgfpathlineto{\pgfqpoint{4.582505in}{2.730381in}}%
\pgfpathlineto{\pgfqpoint{4.582659in}{2.599371in}}%
\pgfpathlineto{\pgfqpoint{4.582681in}{2.826972in}}%
\pgfpathlineto{\pgfqpoint{4.583343in}{2.513816in}}%
\pgfpathlineto{\pgfqpoint{4.583761in}{2.686346in}}%
\pgfpathlineto{\pgfqpoint{4.584511in}{2.502670in}}%
\pgfpathlineto{\pgfqpoint{4.584775in}{2.797142in}}%
\pgfpathlineto{\pgfqpoint{4.584864in}{2.680009in}}%
\pgfpathlineto{\pgfqpoint{4.585547in}{2.915368in}}%
\pgfpathlineto{\pgfqpoint{4.585216in}{2.531517in}}%
\pgfpathlineto{\pgfqpoint{4.586010in}{2.756932in}}%
\pgfpathlineto{\pgfqpoint{4.586737in}{2.592815in}}%
\pgfpathlineto{\pgfqpoint{4.587024in}{2.840739in}}%
\pgfpathlineto{\pgfqpoint{4.587134in}{2.712679in}}%
\pgfpathlineto{\pgfqpoint{4.588236in}{2.426949in}}%
\pgfpathlineto{\pgfqpoint{4.587509in}{2.807959in}}%
\pgfpathlineto{\pgfqpoint{4.588368in}{2.585494in}}%
\pgfpathlineto{\pgfqpoint{4.589316in}{2.753545in}}%
\pgfpathlineto{\pgfqpoint{4.589426in}{2.502670in}}%
\pgfpathlineto{\pgfqpoint{4.589448in}{2.534248in}}%
\pgfpathlineto{\pgfqpoint{4.590418in}{2.481801in}}%
\pgfpathlineto{\pgfqpoint{4.590462in}{2.683287in}}%
\pgfpathlineto{\pgfqpoint{4.590506in}{2.603195in}}%
\pgfpathlineto{\pgfqpoint{4.590595in}{2.801185in}}%
\pgfpathlineto{\pgfqpoint{4.591256in}{2.521355in}}%
\pgfpathlineto{\pgfqpoint{4.591608in}{2.711478in}}%
\pgfpathlineto{\pgfqpoint{4.592093in}{2.535669in}}%
\pgfpathlineto{\pgfqpoint{4.592688in}{2.796486in}}%
\pgfpathlineto{\pgfqpoint{4.592711in}{2.662745in}}%
\pgfpathlineto{\pgfqpoint{4.593438in}{2.777365in}}%
\pgfpathlineto{\pgfqpoint{4.592843in}{2.444104in}}%
\pgfpathlineto{\pgfqpoint{4.593769in}{2.657063in}}%
\pgfpathlineto{\pgfqpoint{4.594827in}{2.437111in}}%
\pgfpathlineto{\pgfqpoint{4.593813in}{2.719017in}}%
\pgfpathlineto{\pgfqpoint{4.594937in}{2.508899in}}%
\pgfpathlineto{\pgfqpoint{4.595752in}{2.722076in}}%
\pgfpathlineto{\pgfqpoint{4.595025in}{2.389252in}}%
\pgfpathlineto{\pgfqpoint{4.595995in}{2.466940in}}%
\pgfpathlineto{\pgfqpoint{4.596083in}{2.357784in}}%
\pgfpathlineto{\pgfqpoint{4.596700in}{2.763707in}}%
\pgfpathlineto{\pgfqpoint{4.596921in}{2.605162in}}%
\pgfpathlineto{\pgfqpoint{4.598045in}{2.895700in}}%
\pgfpathlineto{\pgfqpoint{4.597185in}{2.524305in}}%
\pgfpathlineto{\pgfqpoint{4.598067in}{2.838663in}}%
\pgfpathlineto{\pgfqpoint{4.599103in}{2.366853in}}%
\pgfpathlineto{\pgfqpoint{4.599213in}{2.522775in}}%
\pgfpathlineto{\pgfqpoint{4.599522in}{2.685363in}}%
\pgfpathlineto{\pgfqpoint{4.599632in}{2.395371in}}%
\pgfpathlineto{\pgfqpoint{4.600315in}{2.582544in}}%
\pgfpathlineto{\pgfqpoint{4.601086in}{2.348168in}}%
\pgfpathlineto{\pgfqpoint{4.600646in}{2.698912in}}%
\pgfpathlineto{\pgfqpoint{4.601417in}{2.479725in}}%
\pgfpathlineto{\pgfqpoint{4.601638in}{2.368710in}}%
\pgfpathlineto{\pgfqpoint{4.602607in}{2.763270in}}%
\pgfpathlineto{\pgfqpoint{4.603158in}{2.448474in}}%
\pgfpathlineto{\pgfqpoint{4.603732in}{2.569978in}}%
\pgfpathlineto{\pgfqpoint{4.603996in}{2.830905in}}%
\pgfpathlineto{\pgfqpoint{4.604701in}{2.480926in}}%
\pgfpathlineto{\pgfqpoint{4.604834in}{2.548125in}}%
\pgfpathlineto{\pgfqpoint{4.604900in}{2.730490in}}%
\pgfpathlineto{\pgfqpoint{4.605539in}{2.467377in}}%
\pgfpathlineto{\pgfqpoint{4.605958in}{2.597404in}}%
\pgfpathlineto{\pgfqpoint{4.606861in}{2.689078in}}%
\pgfpathlineto{\pgfqpoint{4.606134in}{2.449021in}}%
\pgfpathlineto{\pgfqpoint{4.606994in}{2.522775in}}%
\pgfpathlineto{\pgfqpoint{4.607060in}{2.449458in}}%
\pgfpathlineto{\pgfqpoint{4.607214in}{2.705031in}}%
\pgfpathlineto{\pgfqpoint{4.608008in}{2.609860in}}%
\pgfpathlineto{\pgfqpoint{4.608471in}{2.824021in}}%
\pgfpathlineto{\pgfqpoint{4.608845in}{2.500157in}}%
\pgfpathlineto{\pgfqpoint{4.609132in}{2.686565in}}%
\pgfpathlineto{\pgfqpoint{4.609969in}{2.412307in}}%
\pgfpathlineto{\pgfqpoint{4.609749in}{2.844782in}}%
\pgfpathlineto{\pgfqpoint{4.610256in}{2.564843in}}%
\pgfpathlineto{\pgfqpoint{4.610300in}{2.799218in}}%
\pgfpathlineto{\pgfqpoint{4.610719in}{2.441481in}}%
\pgfpathlineto{\pgfqpoint{4.611358in}{2.569978in}}%
\pgfpathlineto{\pgfqpoint{4.612107in}{2.839537in}}%
\pgfpathlineto{\pgfqpoint{4.612130in}{2.567247in}}%
\pgfpathlineto{\pgfqpoint{4.612526in}{2.779222in}}%
\pgfpathlineto{\pgfqpoint{4.612967in}{2.578719in}}%
\pgfpathlineto{\pgfqpoint{4.613011in}{2.829048in}}%
\pgfpathlineto{\pgfqpoint{4.613628in}{2.718908in}}%
\pgfpathlineto{\pgfqpoint{4.613805in}{2.826535in}}%
\pgfpathlineto{\pgfqpoint{4.614201in}{2.569978in}}%
\pgfpathlineto{\pgfqpoint{4.614708in}{2.740214in}}%
\pgfpathlineto{\pgfqpoint{4.615370in}{2.558833in}}%
\pgfpathlineto{\pgfqpoint{4.615259in}{2.811019in}}%
\pgfpathlineto{\pgfqpoint{4.615811in}{2.631604in}}%
\pgfpathlineto{\pgfqpoint{4.616053in}{2.859751in}}%
\pgfpathlineto{\pgfqpoint{4.616362in}{2.528020in}}%
\pgfpathlineto{\pgfqpoint{4.616913in}{2.632806in}}%
\pgfpathlineto{\pgfqpoint{4.617265in}{2.516329in}}%
\pgfpathlineto{\pgfqpoint{4.618059in}{2.828392in}}%
\pgfpathlineto{\pgfqpoint{4.618742in}{2.567684in}}%
\pgfpathlineto{\pgfqpoint{4.618610in}{2.832544in}}%
\pgfpathlineto{\pgfqpoint{4.619161in}{2.669192in}}%
\pgfpathlineto{\pgfqpoint{4.620241in}{2.842487in}}%
\pgfpathlineto{\pgfqpoint{4.620087in}{2.540367in}}%
\pgfpathlineto{\pgfqpoint{4.620263in}{2.720765in}}%
\pgfpathlineto{\pgfqpoint{4.620572in}{2.604178in}}%
\pgfpathlineto{\pgfqpoint{4.620660in}{2.890455in}}%
\pgfpathlineto{\pgfqpoint{4.621233in}{2.724917in}}%
\pgfpathlineto{\pgfqpoint{4.621255in}{2.853305in}}%
\pgfpathlineto{\pgfqpoint{4.621784in}{2.511958in}}%
\pgfpathlineto{\pgfqpoint{4.622313in}{2.757588in}}%
\pgfpathlineto{\pgfqpoint{4.623349in}{2.474371in}}%
\pgfpathlineto{\pgfqpoint{4.623195in}{2.846312in}}%
\pgfpathlineto{\pgfqpoint{4.623415in}{2.657609in}}%
\pgfpathlineto{\pgfqpoint{4.624231in}{2.817247in}}%
\pgfpathlineto{\pgfqpoint{4.623966in}{2.526272in}}%
\pgfpathlineto{\pgfqpoint{4.624495in}{2.704485in}}%
\pgfpathlineto{\pgfqpoint{4.624914in}{2.433833in}}%
\pgfpathlineto{\pgfqpoint{4.625311in}{2.752780in}}%
\pgfpathlineto{\pgfqpoint{4.625619in}{2.608003in}}%
\pgfpathlineto{\pgfqpoint{4.626391in}{2.519934in}}%
\pgfpathlineto{\pgfqpoint{4.626721in}{2.776163in}}%
\pgfpathlineto{\pgfqpoint{4.626809in}{2.523868in}}%
\pgfpathlineto{\pgfqpoint{4.627294in}{2.798562in}}%
\pgfpathlineto{\pgfqpoint{4.627823in}{2.646027in}}%
\pgfpathlineto{\pgfqpoint{4.628507in}{2.938095in}}%
\pgfpathlineto{\pgfqpoint{4.628286in}{2.518077in}}%
\pgfpathlineto{\pgfqpoint{4.628926in}{2.812221in}}%
\pgfpathlineto{\pgfqpoint{4.629719in}{2.493929in}}%
\pgfpathlineto{\pgfqpoint{4.629190in}{2.890564in}}%
\pgfpathlineto{\pgfqpoint{4.630050in}{2.681757in}}%
\pgfpathlineto{\pgfqpoint{4.630380in}{2.812658in}}%
\pgfpathlineto{\pgfqpoint{4.630711in}{2.604943in}}%
\pgfpathlineto{\pgfqpoint{4.631042in}{2.639799in}}%
\pgfpathlineto{\pgfqpoint{4.631350in}{2.582872in}}%
\pgfpathlineto{\pgfqpoint{4.631923in}{2.853960in}}%
\pgfpathlineto{\pgfqpoint{4.632078in}{2.731036in}}%
\pgfpathlineto{\pgfqpoint{4.633003in}{2.885429in}}%
\pgfpathlineto{\pgfqpoint{4.632695in}{2.535122in}}%
\pgfpathlineto{\pgfqpoint{4.633180in}{2.729834in}}%
\pgfpathlineto{\pgfqpoint{4.633907in}{2.563969in}}%
\pgfpathlineto{\pgfqpoint{4.633819in}{2.854725in}}%
\pgfpathlineto{\pgfqpoint{4.634282in}{2.612483in}}%
\pgfpathlineto{\pgfqpoint{4.634789in}{2.846421in}}%
\pgfpathlineto{\pgfqpoint{4.635163in}{2.499720in}}%
\pgfpathlineto{\pgfqpoint{4.635406in}{2.701534in}}%
\pgfpathlineto{\pgfqpoint{4.635759in}{2.536543in}}%
\pgfpathlineto{\pgfqpoint{4.636067in}{2.930119in}}%
\pgfpathlineto{\pgfqpoint{4.636464in}{2.715739in}}%
\pgfpathlineto{\pgfqpoint{4.636552in}{2.851556in}}%
\pgfpathlineto{\pgfqpoint{4.636817in}{2.595328in}}%
\pgfpathlineto{\pgfqpoint{4.637566in}{2.704375in}}%
\pgfpathlineto{\pgfqpoint{4.638470in}{2.882807in}}%
\pgfpathlineto{\pgfqpoint{4.637786in}{2.625048in}}%
\pgfpathlineto{\pgfqpoint{4.638712in}{2.741526in}}%
\pgfpathlineto{\pgfqpoint{4.639395in}{2.558396in}}%
\pgfpathlineto{\pgfqpoint{4.638999in}{2.833965in}}%
\pgfpathlineto{\pgfqpoint{4.639770in}{2.593798in}}%
\pgfpathlineto{\pgfqpoint{4.639792in}{2.852758in}}%
\pgfpathlineto{\pgfqpoint{4.640255in}{2.500485in}}%
\pgfpathlineto{\pgfqpoint{4.640872in}{2.659467in}}%
\pgfpathlineto{\pgfqpoint{4.641027in}{2.520918in}}%
\pgfpathlineto{\pgfqpoint{4.641644in}{2.781408in}}%
\pgfpathlineto{\pgfqpoint{4.641952in}{2.550857in}}%
\pgfpathlineto{\pgfqpoint{4.642636in}{2.822055in}}%
\pgfpathlineto{\pgfqpoint{4.642437in}{2.478086in}}%
\pgfpathlineto{\pgfqpoint{4.643076in}{2.635647in}}%
\pgfpathlineto{\pgfqpoint{4.643407in}{2.520481in}}%
\pgfpathlineto{\pgfqpoint{4.643165in}{2.773759in}}%
\pgfpathlineto{\pgfqpoint{4.644135in}{2.690499in}}%
\pgfpathlineto{\pgfqpoint{4.645237in}{2.892859in}}%
\pgfpathlineto{\pgfqpoint{4.644641in}{2.599917in}}%
\pgfpathlineto{\pgfqpoint{4.645303in}{2.805118in}}%
\pgfpathlineto{\pgfqpoint{4.645457in}{2.630839in}}%
\pgfpathlineto{\pgfqpoint{4.646030in}{2.888270in}}%
\pgfpathlineto{\pgfqpoint{4.646405in}{2.754310in}}%
\pgfpathlineto{\pgfqpoint{4.646934in}{2.878873in}}%
\pgfpathlineto{\pgfqpoint{4.646735in}{2.569323in}}%
\pgfpathlineto{\pgfqpoint{4.647397in}{2.724262in}}%
\pgfpathlineto{\pgfqpoint{4.648367in}{2.546923in}}%
\pgfpathlineto{\pgfqpoint{4.647441in}{2.846530in}}%
\pgfpathlineto{\pgfqpoint{4.648477in}{2.743165in}}%
\pgfpathlineto{\pgfqpoint{4.648719in}{2.840739in}}%
\pgfpathlineto{\pgfqpoint{4.648565in}{2.585822in}}%
\pgfpathlineto{\pgfqpoint{4.649469in}{2.700988in}}%
\pgfpathlineto{\pgfqpoint{4.649491in}{2.523868in}}%
\pgfpathlineto{\pgfqpoint{4.650218in}{2.910779in}}%
\pgfpathlineto{\pgfqpoint{4.650571in}{2.762286in}}%
\pgfpathlineto{\pgfqpoint{4.651298in}{2.584729in}}%
\pgfpathlineto{\pgfqpoint{4.650747in}{2.846421in}}%
\pgfpathlineto{\pgfqpoint{4.651717in}{2.641110in}}%
\pgfpathlineto{\pgfqpoint{4.652775in}{2.836259in}}%
\pgfpathlineto{\pgfqpoint{4.652400in}{2.553261in}}%
\pgfpathlineto{\pgfqpoint{4.652841in}{2.731692in}}%
\pgfpathlineto{\pgfqpoint{4.653767in}{2.886303in}}%
\pgfpathlineto{\pgfqpoint{4.653657in}{2.538947in}}%
\pgfpathlineto{\pgfqpoint{4.653877in}{2.705140in}}%
\pgfpathlineto{\pgfqpoint{4.654582in}{2.541897in}}%
\pgfpathlineto{\pgfqpoint{4.653987in}{2.794301in}}%
\pgfpathlineto{\pgfqpoint{4.654935in}{2.672470in}}%
\pgfpathlineto{\pgfqpoint{4.655420in}{2.775398in}}%
\pgfpathlineto{\pgfqpoint{4.655442in}{2.470655in}}%
\pgfpathlineto{\pgfqpoint{4.656015in}{2.671486in}}%
\pgfpathlineto{\pgfqpoint{4.656632in}{2.577080in}}%
\pgfpathlineto{\pgfqpoint{4.656919in}{2.858877in}}%
\pgfpathlineto{\pgfqpoint{4.656963in}{2.783265in}}%
\pgfpathlineto{\pgfqpoint{4.657161in}{2.864341in}}%
\pgfpathlineto{\pgfqpoint{4.657690in}{2.590192in}}%
\pgfpathlineto{\pgfqpoint{4.657977in}{2.697164in}}%
\pgfpathlineto{\pgfqpoint{4.657999in}{2.532063in}}%
\pgfpathlineto{\pgfqpoint{4.658925in}{2.872426in}}%
\pgfpathlineto{\pgfqpoint{4.659079in}{2.631604in}}%
\pgfpathlineto{\pgfqpoint{4.659850in}{2.826535in}}%
\pgfpathlineto{\pgfqpoint{4.659608in}{2.496770in}}%
\pgfpathlineto{\pgfqpoint{4.660203in}{2.732675in}}%
\pgfpathlineto{\pgfqpoint{4.661129in}{2.552496in}}%
\pgfpathlineto{\pgfqpoint{4.661217in}{2.842160in}}%
\pgfpathlineto{\pgfqpoint{4.661283in}{2.787417in}}%
\pgfpathlineto{\pgfqpoint{4.661834in}{2.942247in}}%
\pgfpathlineto{\pgfqpoint{4.661724in}{2.544410in}}%
\pgfpathlineto{\pgfqpoint{4.662297in}{2.789275in}}%
\pgfpathlineto{\pgfqpoint{4.662870in}{2.606801in}}%
\pgfpathlineto{\pgfqpoint{4.663223in}{2.946290in}}%
\pgfpathlineto{\pgfqpoint{4.663421in}{2.662089in}}%
\pgfpathlineto{\pgfqpoint{4.664325in}{2.837898in}}%
\pgfpathlineto{\pgfqpoint{4.664259in}{2.625376in}}%
\pgfpathlineto{\pgfqpoint{4.664523in}{2.676512in}}%
\pgfpathlineto{\pgfqpoint{4.664678in}{2.501687in}}%
\pgfpathlineto{\pgfqpoint{4.664920in}{2.814625in}}%
\pgfpathlineto{\pgfqpoint{4.665670in}{2.535778in}}%
\pgfpathlineto{\pgfqpoint{4.666331in}{2.789712in}}%
\pgfpathlineto{\pgfqpoint{4.666199in}{2.498628in}}%
\pgfpathlineto{\pgfqpoint{4.666816in}{2.770372in}}%
\pgfusepath{stroke}%
\end{pgfscope}%
\begin{pgfscope}%
\pgfsetrectcap%
\pgfsetmiterjoin%
\pgfsetlinewidth{0.803000pt}%
\definecolor{currentstroke}{rgb}{0.000000,0.000000,0.000000}%
\pgfsetstrokecolor{currentstroke}%
\pgfsetdash{}{0pt}%
\pgfpathmoveto{\pgfqpoint{0.667540in}{0.539544in}}%
\pgfpathlineto{\pgfqpoint{0.667540in}{3.120077in}}%
\pgfusepath{stroke}%
\end{pgfscope}%
\begin{pgfscope}%
\pgfsetrectcap%
\pgfsetmiterjoin%
\pgfsetlinewidth{0.803000pt}%
\definecolor{currentstroke}{rgb}{0.000000,0.000000,0.000000}%
\pgfsetstrokecolor{currentstroke}%
\pgfsetdash{}{0pt}%
\pgfpathmoveto{\pgfqpoint{4.857257in}{0.539544in}}%
\pgfpathlineto{\pgfqpoint{4.857257in}{3.120077in}}%
\pgfusepath{stroke}%
\end{pgfscope}%
\begin{pgfscope}%
\pgfsetrectcap%
\pgfsetmiterjoin%
\pgfsetlinewidth{0.803000pt}%
\definecolor{currentstroke}{rgb}{0.000000,0.000000,0.000000}%
\pgfsetstrokecolor{currentstroke}%
\pgfsetdash{}{0pt}%
\pgfpathmoveto{\pgfqpoint{0.667540in}{0.539544in}}%
\pgfpathlineto{\pgfqpoint{4.857257in}{0.539544in}}%
\pgfusepath{stroke}%
\end{pgfscope}%
\begin{pgfscope}%
\pgfsetrectcap%
\pgfsetmiterjoin%
\pgfsetlinewidth{0.803000pt}%
\definecolor{currentstroke}{rgb}{0.000000,0.000000,0.000000}%
\pgfsetstrokecolor{currentstroke}%
\pgfsetdash{}{0pt}%
\pgfpathmoveto{\pgfqpoint{0.667540in}{3.120077in}}%
\pgfpathlineto{\pgfqpoint{4.857257in}{3.120077in}}%
\pgfusepath{stroke}%
\end{pgfscope}%
\begin{pgfscope}%
\pgfpathrectangle{\pgfqpoint{0.667540in}{0.539544in}}{\pgfqpoint{4.189718in}{2.580533in}}%
\pgfusepath{clip}%
\pgfsetrectcap%
\pgfsetroundjoin%
\pgfsetlinewidth{0.803000pt}%
\definecolor{currentstroke}{rgb}{0.450000,0.450000,0.450000}%
\pgfsetstrokecolor{currentstroke}%
\pgfsetdash{}{0pt}%
\pgfpathmoveto{\pgfqpoint{0.667540in}{0.796098in}}%
\pgfpathlineto{\pgfqpoint{4.857257in}{0.796098in}}%
\pgfusepath{stroke}%
\end{pgfscope}%
\begin{pgfscope}%
\pgfsetbuttcap%
\pgfsetroundjoin%
\definecolor{currentfill}{rgb}{0.000000,0.000000,0.000000}%
\pgfsetfillcolor{currentfill}%
\pgfsetlinewidth{0.803000pt}%
\definecolor{currentstroke}{rgb}{0.000000,0.000000,0.000000}%
\pgfsetstrokecolor{currentstroke}%
\pgfsetdash{}{0pt}%
\pgfsys@defobject{currentmarker}{\pgfqpoint{0.000000in}{0.000000in}}{\pgfqpoint{0.048611in}{0.000000in}}{%
\pgfpathmoveto{\pgfqpoint{0.000000in}{0.000000in}}%
\pgfpathlineto{\pgfqpoint{0.048611in}{0.000000in}}%
\pgfusepath{stroke,fill}%
}%
\begin{pgfscope}%
\pgfsys@transformshift{4.857257in}{0.796098in}%
\pgfsys@useobject{currentmarker}{}%
\end{pgfscope}%
\end{pgfscope}%
\begin{pgfscope}%
\definecolor{textcolor}{rgb}{0.000000,0.000000,0.000000}%
\pgfsetstrokecolor{textcolor}%
\pgfsetfillcolor{textcolor}%
\pgftext[x=4.954480in, y=0.757542in, left, base]{\color{textcolor}\rmfamily\fontsize{8.000000}{9.600000}\selectfont \(\displaystyle {20.0}\)}%
\end{pgfscope}%
\begin{pgfscope}%
\pgfpathrectangle{\pgfqpoint{0.667540in}{0.539544in}}{\pgfqpoint{4.189718in}{2.580533in}}%
\pgfusepath{clip}%
\pgfsetrectcap%
\pgfsetroundjoin%
\pgfsetlinewidth{0.803000pt}%
\definecolor{currentstroke}{rgb}{0.450000,0.450000,0.450000}%
\pgfsetstrokecolor{currentstroke}%
\pgfsetdash{}{0pt}%
\pgfpathmoveto{\pgfqpoint{0.667540in}{1.331700in}}%
\pgfpathlineto{\pgfqpoint{4.857257in}{1.331700in}}%
\pgfusepath{stroke}%
\end{pgfscope}%
\begin{pgfscope}%
\pgfsetbuttcap%
\pgfsetroundjoin%
\definecolor{currentfill}{rgb}{0.000000,0.000000,0.000000}%
\pgfsetfillcolor{currentfill}%
\pgfsetlinewidth{0.803000pt}%
\definecolor{currentstroke}{rgb}{0.000000,0.000000,0.000000}%
\pgfsetstrokecolor{currentstroke}%
\pgfsetdash{}{0pt}%
\pgfsys@defobject{currentmarker}{\pgfqpoint{0.000000in}{0.000000in}}{\pgfqpoint{0.048611in}{0.000000in}}{%
\pgfpathmoveto{\pgfqpoint{0.000000in}{0.000000in}}%
\pgfpathlineto{\pgfqpoint{0.048611in}{0.000000in}}%
\pgfusepath{stroke,fill}%
}%
\begin{pgfscope}%
\pgfsys@transformshift{4.857257in}{1.331700in}%
\pgfsys@useobject{currentmarker}{}%
\end{pgfscope}%
\end{pgfscope}%
\begin{pgfscope}%
\definecolor{textcolor}{rgb}{0.000000,0.000000,0.000000}%
\pgfsetstrokecolor{textcolor}%
\pgfsetfillcolor{textcolor}%
\pgftext[x=4.954480in, y=1.293145in, left, base]{\color{textcolor}\rmfamily\fontsize{8.000000}{9.600000}\selectfont \(\displaystyle {20.5}\)}%
\end{pgfscope}%
\begin{pgfscope}%
\pgfpathrectangle{\pgfqpoint{0.667540in}{0.539544in}}{\pgfqpoint{4.189718in}{2.580533in}}%
\pgfusepath{clip}%
\pgfsetrectcap%
\pgfsetroundjoin%
\pgfsetlinewidth{0.803000pt}%
\definecolor{currentstroke}{rgb}{0.450000,0.450000,0.450000}%
\pgfsetstrokecolor{currentstroke}%
\pgfsetdash{}{0pt}%
\pgfpathmoveto{\pgfqpoint{0.667540in}{1.867303in}}%
\pgfpathlineto{\pgfqpoint{4.857257in}{1.867303in}}%
\pgfusepath{stroke}%
\end{pgfscope}%
\begin{pgfscope}%
\pgfsetbuttcap%
\pgfsetroundjoin%
\definecolor{currentfill}{rgb}{0.000000,0.000000,0.000000}%
\pgfsetfillcolor{currentfill}%
\pgfsetlinewidth{0.803000pt}%
\definecolor{currentstroke}{rgb}{0.000000,0.000000,0.000000}%
\pgfsetstrokecolor{currentstroke}%
\pgfsetdash{}{0pt}%
\pgfsys@defobject{currentmarker}{\pgfqpoint{0.000000in}{0.000000in}}{\pgfqpoint{0.048611in}{0.000000in}}{%
\pgfpathmoveto{\pgfqpoint{0.000000in}{0.000000in}}%
\pgfpathlineto{\pgfqpoint{0.048611in}{0.000000in}}%
\pgfusepath{stroke,fill}%
}%
\begin{pgfscope}%
\pgfsys@transformshift{4.857257in}{1.867303in}%
\pgfsys@useobject{currentmarker}{}%
\end{pgfscope}%
\end{pgfscope}%
\begin{pgfscope}%
\definecolor{textcolor}{rgb}{0.000000,0.000000,0.000000}%
\pgfsetstrokecolor{textcolor}%
\pgfsetfillcolor{textcolor}%
\pgftext[x=4.954480in, y=1.828747in, left, base]{\color{textcolor}\rmfamily\fontsize{8.000000}{9.600000}\selectfont \(\displaystyle {21.0}\)}%
\end{pgfscope}%
\begin{pgfscope}%
\pgfpathrectangle{\pgfqpoint{0.667540in}{0.539544in}}{\pgfqpoint{4.189718in}{2.580533in}}%
\pgfusepath{clip}%
\pgfsetrectcap%
\pgfsetroundjoin%
\pgfsetlinewidth{0.803000pt}%
\definecolor{currentstroke}{rgb}{0.450000,0.450000,0.450000}%
\pgfsetstrokecolor{currentstroke}%
\pgfsetdash{}{0pt}%
\pgfpathmoveto{\pgfqpoint{0.667540in}{2.402906in}}%
\pgfpathlineto{\pgfqpoint{4.857257in}{2.402906in}}%
\pgfusepath{stroke}%
\end{pgfscope}%
\begin{pgfscope}%
\pgfsetbuttcap%
\pgfsetroundjoin%
\definecolor{currentfill}{rgb}{0.000000,0.000000,0.000000}%
\pgfsetfillcolor{currentfill}%
\pgfsetlinewidth{0.803000pt}%
\definecolor{currentstroke}{rgb}{0.000000,0.000000,0.000000}%
\pgfsetstrokecolor{currentstroke}%
\pgfsetdash{}{0pt}%
\pgfsys@defobject{currentmarker}{\pgfqpoint{0.000000in}{0.000000in}}{\pgfqpoint{0.048611in}{0.000000in}}{%
\pgfpathmoveto{\pgfqpoint{0.000000in}{0.000000in}}%
\pgfpathlineto{\pgfqpoint{0.048611in}{0.000000in}}%
\pgfusepath{stroke,fill}%
}%
\begin{pgfscope}%
\pgfsys@transformshift{4.857257in}{2.402906in}%
\pgfsys@useobject{currentmarker}{}%
\end{pgfscope}%
\end{pgfscope}%
\begin{pgfscope}%
\definecolor{textcolor}{rgb}{0.000000,0.000000,0.000000}%
\pgfsetstrokecolor{textcolor}%
\pgfsetfillcolor{textcolor}%
\pgftext[x=4.954480in, y=2.364350in, left, base]{\color{textcolor}\rmfamily\fontsize{8.000000}{9.600000}\selectfont \(\displaystyle {21.5}\)}%
\end{pgfscope}%
\begin{pgfscope}%
\pgfpathrectangle{\pgfqpoint{0.667540in}{0.539544in}}{\pgfqpoint{4.189718in}{2.580533in}}%
\pgfusepath{clip}%
\pgfsetrectcap%
\pgfsetroundjoin%
\pgfsetlinewidth{0.803000pt}%
\definecolor{currentstroke}{rgb}{0.450000,0.450000,0.450000}%
\pgfsetstrokecolor{currentstroke}%
\pgfsetdash{}{0pt}%
\pgfpathmoveto{\pgfqpoint{0.667540in}{2.938508in}}%
\pgfpathlineto{\pgfqpoint{4.857257in}{2.938508in}}%
\pgfusepath{stroke}%
\end{pgfscope}%
\begin{pgfscope}%
\pgfsetbuttcap%
\pgfsetroundjoin%
\definecolor{currentfill}{rgb}{0.000000,0.000000,0.000000}%
\pgfsetfillcolor{currentfill}%
\pgfsetlinewidth{0.803000pt}%
\definecolor{currentstroke}{rgb}{0.000000,0.000000,0.000000}%
\pgfsetstrokecolor{currentstroke}%
\pgfsetdash{}{0pt}%
\pgfsys@defobject{currentmarker}{\pgfqpoint{0.000000in}{0.000000in}}{\pgfqpoint{0.048611in}{0.000000in}}{%
\pgfpathmoveto{\pgfqpoint{0.000000in}{0.000000in}}%
\pgfpathlineto{\pgfqpoint{0.048611in}{0.000000in}}%
\pgfusepath{stroke,fill}%
}%
\begin{pgfscope}%
\pgfsys@transformshift{4.857257in}{2.938508in}%
\pgfsys@useobject{currentmarker}{}%
\end{pgfscope}%
\end{pgfscope}%
\begin{pgfscope}%
\definecolor{textcolor}{rgb}{0.000000,0.000000,0.000000}%
\pgfsetstrokecolor{textcolor}%
\pgfsetfillcolor{textcolor}%
\pgftext[x=4.954480in, y=2.899953in, left, base]{\color{textcolor}\rmfamily\fontsize{8.000000}{9.600000}\selectfont \(\displaystyle {22.0}\)}%
\end{pgfscope}%
\begin{pgfscope}%
\definecolor{textcolor}{rgb}{0.000000,0.000000,0.000000}%
\pgfsetstrokecolor{textcolor}%
\pgfsetfillcolor{textcolor}%
\pgftext[x=5.219915in,y=1.829811in,,top,rotate=90.000000]{\color{textcolor}\rmfamily\fontsize{10.000000}{12.000000}\selectfont Temperature in \unit{\celsius}}%
\end{pgfscope}%
\begin{pgfscope}%
\pgfpathrectangle{\pgfqpoint{0.667540in}{0.539544in}}{\pgfqpoint{4.189718in}{2.580533in}}%
\pgfusepath{clip}%
\pgfsetrectcap%
\pgfsetroundjoin%
\pgfsetlinewidth{1.505625pt}%
\definecolor{currentstroke}{rgb}{0.909804,0.000000,0.043137}%
\pgfsetstrokecolor{currentstroke}%
\pgfsetstrokeopacity{0.700000}%
\pgfsetdash{}{0pt}%
\pgfpathmoveto{\pgfqpoint{0.858094in}{2.670707in}}%
\pgfpathlineto{\pgfqpoint{0.860740in}{2.595723in}}%
\pgfpathlineto{\pgfqpoint{0.863385in}{2.670707in}}%
\pgfpathlineto{\pgfqpoint{0.866030in}{2.595723in}}%
\pgfpathlineto{\pgfqpoint{0.868719in}{2.670707in}}%
\pgfpathlineto{\pgfqpoint{0.871364in}{2.595723in}}%
\pgfpathlineto{\pgfqpoint{0.874009in}{2.670707in}}%
\pgfpathlineto{\pgfqpoint{0.876742in}{2.595723in}}%
\pgfpathlineto{\pgfqpoint{0.879387in}{2.670707in}}%
\pgfpathlineto{\pgfqpoint{0.912626in}{2.734979in}}%
\pgfpathlineto{\pgfqpoint{0.915271in}{2.670707in}}%
\pgfpathlineto{\pgfqpoint{0.917961in}{2.734979in}}%
\pgfpathlineto{\pgfqpoint{0.920606in}{2.670707in}}%
\pgfpathlineto{\pgfqpoint{0.923251in}{2.734979in}}%
\pgfpathlineto{\pgfqpoint{0.925984in}{2.670707in}}%
\pgfpathlineto{\pgfqpoint{0.928629in}{2.734979in}}%
\pgfpathlineto{\pgfqpoint{0.952346in}{2.799252in}}%
\pgfpathlineto{\pgfqpoint{0.954991in}{2.734979in}}%
\pgfpathlineto{\pgfqpoint{0.957680in}{2.799252in}}%
\pgfpathlineto{\pgfqpoint{0.960325in}{2.734979in}}%
\pgfpathlineto{\pgfqpoint{0.963103in}{2.799252in}}%
\pgfpathlineto{\pgfqpoint{0.965748in}{2.734979in}}%
\pgfpathlineto{\pgfqpoint{0.968393in}{2.799252in}}%
\pgfpathlineto{\pgfqpoint{0.971038in}{2.734979in}}%
\pgfpathlineto{\pgfqpoint{0.973683in}{2.799252in}}%
\pgfpathlineto{\pgfqpoint{0.976328in}{2.734979in}}%
\pgfpathlineto{\pgfqpoint{0.978973in}{2.799252in}}%
\pgfpathlineto{\pgfqpoint{0.998061in}{2.863524in}}%
\pgfpathlineto{\pgfqpoint{1.000706in}{2.799252in}}%
\pgfpathlineto{\pgfqpoint{1.003439in}{2.863524in}}%
\pgfpathlineto{\pgfqpoint{1.006084in}{2.799252in}}%
\pgfpathlineto{\pgfqpoint{1.008729in}{2.863524in}}%
\pgfpathlineto{\pgfqpoint{1.011374in}{2.799252in}}%
\pgfpathlineto{\pgfqpoint{1.014196in}{2.863524in}}%
\pgfpathlineto{\pgfqpoint{1.016841in}{2.799252in}}%
\pgfpathlineto{\pgfqpoint{1.019530in}{2.863524in}}%
\pgfpathlineto{\pgfqpoint{1.022175in}{2.799252in}}%
\pgfpathlineto{\pgfqpoint{1.024864in}{2.863524in}}%
\pgfpathlineto{\pgfqpoint{1.027509in}{2.799252in}}%
\pgfpathlineto{\pgfqpoint{1.030154in}{2.863524in}}%
\pgfpathlineto{\pgfqpoint{1.032932in}{2.799252in}}%
\pgfpathlineto{\pgfqpoint{1.035577in}{2.863524in}}%
\pgfpathlineto{\pgfqpoint{1.039236in}{2.799252in}}%
\pgfpathlineto{\pgfqpoint{1.041881in}{2.863524in}}%
\pgfpathlineto{\pgfqpoint{1.098661in}{2.938508in}}%
\pgfpathlineto{\pgfqpoint{1.101306in}{2.863524in}}%
\pgfpathlineto{\pgfqpoint{1.104083in}{2.938508in}}%
\pgfpathlineto{\pgfqpoint{1.106728in}{2.863524in}}%
\pgfpathlineto{\pgfqpoint{1.109373in}{2.938508in}}%
\pgfpathlineto{\pgfqpoint{1.112018in}{2.863524in}}%
\pgfpathlineto{\pgfqpoint{1.122687in}{2.863524in}}%
\pgfpathlineto{\pgfqpoint{1.125332in}{2.938508in}}%
\pgfpathlineto{\pgfqpoint{1.128109in}{2.863524in}}%
\pgfpathlineto{\pgfqpoint{1.130754in}{2.938508in}}%
\pgfpathlineto{\pgfqpoint{1.134325in}{2.863524in}}%
\pgfpathlineto{\pgfqpoint{1.136970in}{2.938508in}}%
\pgfpathlineto{\pgfqpoint{1.143186in}{2.863524in}}%
\pgfpathlineto{\pgfqpoint{1.145831in}{2.938508in}}%
\pgfpathlineto{\pgfqpoint{1.195469in}{3.002780in}}%
\pgfpathlineto{\pgfqpoint{1.198114in}{2.938508in}}%
\pgfpathlineto{\pgfqpoint{1.206490in}{3.002780in}}%
\pgfpathlineto{\pgfqpoint{1.209135in}{2.938508in}}%
\pgfpathlineto{\pgfqpoint{1.212794in}{3.002780in}}%
\pgfpathlineto{\pgfqpoint{1.215439in}{2.938508in}}%
\pgfpathlineto{\pgfqpoint{1.218084in}{3.002780in}}%
\pgfpathlineto{\pgfqpoint{1.220729in}{2.938508in}}%
\pgfpathlineto{\pgfqpoint{1.223374in}{3.002780in}}%
\pgfpathlineto{\pgfqpoint{1.226063in}{2.938508in}}%
\pgfpathlineto{\pgfqpoint{1.228709in}{3.002780in}}%
\pgfpathlineto{\pgfqpoint{1.231354in}{2.938508in}}%
\pgfpathlineto{\pgfqpoint{1.233999in}{3.002780in}}%
\pgfpathlineto{\pgfqpoint{1.236864in}{2.938508in}}%
\pgfpathlineto{\pgfqpoint{1.239509in}{3.002780in}}%
\pgfpathlineto{\pgfqpoint{1.242815in}{2.938508in}}%
\pgfpathlineto{\pgfqpoint{1.245460in}{3.002780in}}%
\pgfpathlineto{\pgfqpoint{1.250001in}{2.938508in}}%
\pgfpathlineto{\pgfqpoint{1.252646in}{3.002780in}}%
\pgfpathlineto{\pgfqpoint{1.255467in}{2.938508in}}%
\pgfpathlineto{\pgfqpoint{1.258113in}{3.002780in}}%
\pgfpathlineto{\pgfqpoint{1.286943in}{2.938508in}}%
\pgfpathlineto{\pgfqpoint{1.294879in}{2.734979in}}%
\pgfpathlineto{\pgfqpoint{1.305459in}{2.467178in}}%
\pgfpathlineto{\pgfqpoint{1.308104in}{2.327921in}}%
\pgfpathlineto{\pgfqpoint{1.333540in}{1.728046in}}%
\pgfpathlineto{\pgfqpoint{1.346325in}{1.524517in}}%
\pgfpathlineto{\pgfqpoint{1.351747in}{1.460245in}}%
\pgfpathlineto{\pgfqpoint{1.354656in}{1.524517in}}%
\pgfpathlineto{\pgfqpoint{1.359947in}{1.395973in}}%
\pgfpathlineto{\pgfqpoint{1.365325in}{1.331700in}}%
\pgfpathlineto{\pgfqpoint{1.368102in}{1.395973in}}%
\pgfpathlineto{\pgfqpoint{1.373392in}{1.256716in}}%
\pgfpathlineto{\pgfqpoint{1.380578in}{1.192444in}}%
\pgfpathlineto{\pgfqpoint{1.383223in}{1.256716in}}%
\pgfpathlineto{\pgfqpoint{1.385868in}{1.192444in}}%
\pgfpathlineto{\pgfqpoint{1.394597in}{1.128171in}}%
\pgfpathlineto{\pgfqpoint{1.397242in}{1.192444in}}%
\pgfpathlineto{\pgfqpoint{1.399975in}{1.128171in}}%
\pgfpathlineto{\pgfqpoint{1.402620in}{1.192444in}}%
\pgfpathlineto{\pgfqpoint{1.405265in}{1.128171in}}%
\pgfpathlineto{\pgfqpoint{1.407954in}{1.192444in}}%
\pgfpathlineto{\pgfqpoint{1.410599in}{1.128171in}}%
\pgfpathlineto{\pgfqpoint{1.413376in}{1.192444in}}%
\pgfpathlineto{\pgfqpoint{1.416021in}{1.128171in}}%
\pgfpathlineto{\pgfqpoint{1.420342in}{1.063899in}}%
\pgfpathlineto{\pgfqpoint{1.422987in}{1.128171in}}%
\pgfpathlineto{\pgfqpoint{1.425632in}{1.063899in}}%
\pgfpathlineto{\pgfqpoint{1.430437in}{0.988915in}}%
\pgfpathlineto{\pgfqpoint{1.433214in}{1.063899in}}%
\pgfpathlineto{\pgfqpoint{1.438769in}{0.924643in}}%
\pgfpathlineto{\pgfqpoint{1.441678in}{0.988915in}}%
\pgfpathlineto{\pgfqpoint{1.447012in}{0.860370in}}%
\pgfpathlineto{\pgfqpoint{1.449657in}{0.924643in}}%
\pgfpathlineto{\pgfqpoint{1.452302in}{0.860370in}}%
\pgfpathlineto{\pgfqpoint{1.461340in}{0.796098in}}%
\pgfpathlineto{\pgfqpoint{1.463985in}{0.860370in}}%
\pgfpathlineto{\pgfqpoint{1.466630in}{0.796098in}}%
\pgfpathlineto{\pgfqpoint{1.470685in}{0.860370in}}%
\pgfpathlineto{\pgfqpoint{1.473331in}{0.796098in}}%
\pgfpathlineto{\pgfqpoint{1.476064in}{0.860370in}}%
\pgfpathlineto{\pgfqpoint{1.478709in}{0.796098in}}%
\pgfpathlineto{\pgfqpoint{1.509479in}{0.721114in}}%
\pgfpathlineto{\pgfqpoint{1.512124in}{0.796098in}}%
\pgfpathlineto{\pgfqpoint{1.514814in}{0.721114in}}%
\pgfpathlineto{\pgfqpoint{1.517459in}{0.796098in}}%
\pgfpathlineto{\pgfqpoint{1.520104in}{0.721114in}}%
\pgfpathlineto{\pgfqpoint{1.522793in}{0.796098in}}%
\pgfpathlineto{\pgfqpoint{1.525438in}{0.721114in}}%
\pgfpathlineto{\pgfqpoint{1.544702in}{0.796098in}}%
\pgfpathlineto{\pgfqpoint{1.548802in}{0.860370in}}%
\pgfpathlineto{\pgfqpoint{1.564673in}{1.256716in}}%
\pgfpathlineto{\pgfqpoint{1.567318in}{1.395973in}}%
\pgfpathlineto{\pgfqpoint{1.584510in}{1.792319in}}%
\pgfpathlineto{\pgfqpoint{1.599587in}{1.995848in}}%
\pgfpathlineto{\pgfqpoint{1.606288in}{2.060120in}}%
\pgfpathlineto{\pgfqpoint{1.612636in}{2.135104in}}%
\pgfpathlineto{\pgfqpoint{1.619954in}{2.199377in}}%
\pgfpathlineto{\pgfqpoint{1.622643in}{2.135104in}}%
\pgfpathlineto{\pgfqpoint{1.628242in}{2.263649in}}%
\pgfpathlineto{\pgfqpoint{1.631063in}{2.199377in}}%
\pgfpathlineto{\pgfqpoint{1.633708in}{2.263649in}}%
\pgfpathlineto{\pgfqpoint{1.640806in}{2.327921in}}%
\pgfpathlineto{\pgfqpoint{1.643451in}{2.263649in}}%
\pgfpathlineto{\pgfqpoint{1.646096in}{2.327921in}}%
\pgfpathlineto{\pgfqpoint{1.655574in}{2.402906in}}%
\pgfpathlineto{\pgfqpoint{1.658219in}{2.327921in}}%
\pgfpathlineto{\pgfqpoint{1.660864in}{2.402906in}}%
\pgfpathlineto{\pgfqpoint{1.670430in}{2.467178in}}%
\pgfpathlineto{\pgfqpoint{1.673075in}{2.402906in}}%
\pgfpathlineto{\pgfqpoint{1.675720in}{2.467178in}}%
\pgfpathlineto{\pgfqpoint{1.678674in}{2.402906in}}%
\pgfpathlineto{\pgfqpoint{1.681319in}{2.467178in}}%
\pgfpathlineto{\pgfqpoint{1.692252in}{2.531450in}}%
\pgfpathlineto{\pgfqpoint{1.694897in}{2.467178in}}%
\pgfpathlineto{\pgfqpoint{1.697542in}{2.531450in}}%
\pgfpathlineto{\pgfqpoint{1.700275in}{2.467178in}}%
\pgfpathlineto{\pgfqpoint{1.702920in}{2.531450in}}%
\pgfpathlineto{\pgfqpoint{1.717379in}{2.595723in}}%
\pgfpathlineto{\pgfqpoint{1.720025in}{2.531450in}}%
\pgfpathlineto{\pgfqpoint{1.722670in}{2.595723in}}%
\pgfpathlineto{\pgfqpoint{1.725359in}{2.531450in}}%
\pgfpathlineto{\pgfqpoint{1.728004in}{2.595723in}}%
\pgfpathlineto{\pgfqpoint{1.745020in}{2.670707in}}%
\pgfpathlineto{\pgfqpoint{1.747665in}{2.595723in}}%
\pgfpathlineto{\pgfqpoint{1.750354in}{2.670707in}}%
\pgfpathlineto{\pgfqpoint{1.752999in}{2.595723in}}%
\pgfpathlineto{\pgfqpoint{1.755644in}{2.670707in}}%
\pgfpathlineto{\pgfqpoint{1.758378in}{2.595723in}}%
\pgfpathlineto{\pgfqpoint{1.761023in}{2.670707in}}%
\pgfpathlineto{\pgfqpoint{1.764020in}{2.595723in}}%
\pgfpathlineto{\pgfqpoint{1.766665in}{2.670707in}}%
\pgfpathlineto{\pgfqpoint{1.777290in}{2.734979in}}%
\pgfpathlineto{\pgfqpoint{1.779935in}{2.670707in}}%
\pgfpathlineto{\pgfqpoint{1.782580in}{2.734979in}}%
\pgfpathlineto{\pgfqpoint{1.785225in}{2.670707in}}%
\pgfpathlineto{\pgfqpoint{1.788002in}{2.734979in}}%
\pgfpathlineto{\pgfqpoint{1.790647in}{2.670707in}}%
\pgfpathlineto{\pgfqpoint{1.793292in}{2.734979in}}%
\pgfpathlineto{\pgfqpoint{1.796069in}{2.670707in}}%
\pgfpathlineto{\pgfqpoint{1.798758in}{2.734979in}}%
\pgfpathlineto{\pgfqpoint{1.801668in}{2.670707in}}%
\pgfpathlineto{\pgfqpoint{1.804313in}{2.734979in}}%
\pgfpathlineto{\pgfqpoint{1.814849in}{2.799252in}}%
\pgfpathlineto{\pgfqpoint{1.817494in}{2.734979in}}%
\pgfpathlineto{\pgfqpoint{1.820183in}{2.799252in}}%
\pgfpathlineto{\pgfqpoint{1.822828in}{2.734979in}}%
\pgfpathlineto{\pgfqpoint{1.825473in}{2.799252in}}%
\pgfpathlineto{\pgfqpoint{1.828118in}{2.734979in}}%
\pgfpathlineto{\pgfqpoint{1.830763in}{2.799252in}}%
\pgfpathlineto{\pgfqpoint{1.833409in}{2.734979in}}%
\pgfpathlineto{\pgfqpoint{1.836054in}{2.799252in}}%
\pgfpathlineto{\pgfqpoint{1.841211in}{2.734979in}}%
\pgfpathlineto{\pgfqpoint{1.843856in}{2.799252in}}%
\pgfpathlineto{\pgfqpoint{1.869028in}{2.863524in}}%
\pgfpathlineto{\pgfqpoint{1.871673in}{2.799252in}}%
\pgfpathlineto{\pgfqpoint{1.874407in}{2.863524in}}%
\pgfpathlineto{\pgfqpoint{1.877052in}{2.799252in}}%
\pgfpathlineto{\pgfqpoint{1.879785in}{2.863524in}}%
\pgfpathlineto{\pgfqpoint{1.882430in}{2.799252in}}%
\pgfpathlineto{\pgfqpoint{1.885163in}{2.863524in}}%
\pgfpathlineto{\pgfqpoint{1.887808in}{2.799252in}}%
\pgfpathlineto{\pgfqpoint{1.890453in}{2.863524in}}%
\pgfpathlineto{\pgfqpoint{1.893186in}{2.799252in}}%
\pgfpathlineto{\pgfqpoint{1.895831in}{2.863524in}}%
\pgfpathlineto{\pgfqpoint{1.899843in}{2.799252in}}%
\pgfpathlineto{\pgfqpoint{1.902488in}{2.863524in}}%
\pgfpathlineto{\pgfqpoint{1.906103in}{2.799252in}}%
\pgfpathlineto{\pgfqpoint{1.908748in}{2.863524in}}%
\pgfpathlineto{\pgfqpoint{1.911613in}{2.799252in}}%
\pgfpathlineto{\pgfqpoint{1.914259in}{2.863524in}}%
\pgfpathlineto{\pgfqpoint{1.959268in}{2.938508in}}%
\pgfpathlineto{\pgfqpoint{1.961913in}{2.863524in}}%
\pgfpathlineto{\pgfqpoint{1.964691in}{2.938508in}}%
\pgfpathlineto{\pgfqpoint{1.967336in}{2.863524in}}%
\pgfpathlineto{\pgfqpoint{1.970377in}{2.938508in}}%
\pgfpathlineto{\pgfqpoint{1.973022in}{2.863524in}}%
\pgfpathlineto{\pgfqpoint{1.975668in}{2.938508in}}%
\pgfpathlineto{\pgfqpoint{1.978313in}{2.863524in}}%
\pgfpathlineto{\pgfqpoint{1.980958in}{2.938508in}}%
\pgfpathlineto{\pgfqpoint{1.983603in}{2.863524in}}%
\pgfpathlineto{\pgfqpoint{1.986248in}{2.938508in}}%
\pgfpathlineto{\pgfqpoint{1.989642in}{2.863524in}}%
\pgfpathlineto{\pgfqpoint{1.992287in}{2.938508in}}%
\pgfpathlineto{\pgfqpoint{2.039942in}{3.002780in}}%
\pgfpathlineto{\pgfqpoint{2.042587in}{2.938508in}}%
\pgfpathlineto{\pgfqpoint{2.045585in}{3.002780in}}%
\pgfpathlineto{\pgfqpoint{2.048230in}{2.938508in}}%
\pgfpathlineto{\pgfqpoint{2.050963in}{3.002780in}}%
\pgfpathlineto{\pgfqpoint{2.053608in}{2.938508in}}%
\pgfpathlineto{\pgfqpoint{2.056253in}{3.002780in}}%
\pgfpathlineto{\pgfqpoint{2.058898in}{2.938508in}}%
\pgfpathlineto{\pgfqpoint{2.061543in}{3.002780in}}%
\pgfpathlineto{\pgfqpoint{2.064320in}{2.938508in}}%
\pgfpathlineto{\pgfqpoint{2.066965in}{3.002780in}}%
\pgfpathlineto{\pgfqpoint{2.070977in}{2.938508in}}%
\pgfpathlineto{\pgfqpoint{2.073622in}{3.002780in}}%
\pgfpathlineto{\pgfqpoint{2.077017in}{2.938508in}}%
\pgfpathlineto{\pgfqpoint{2.079662in}{3.002780in}}%
\pgfpathlineto{\pgfqpoint{2.099940in}{2.938508in}}%
\pgfpathlineto{\pgfqpoint{2.107875in}{2.734979in}}%
\pgfpathlineto{\pgfqpoint{2.110520in}{2.595723in}}%
\pgfpathlineto{\pgfqpoint{2.136045in}{1.995848in}}%
\pgfpathlineto{\pgfqpoint{2.139924in}{1.931575in}}%
\pgfpathlineto{\pgfqpoint{2.142614in}{1.995848in}}%
\pgfpathlineto{\pgfqpoint{2.148918in}{1.867303in}}%
\pgfpathlineto{\pgfqpoint{2.151563in}{1.931575in}}%
\pgfpathlineto{\pgfqpoint{2.159586in}{1.728046in}}%
\pgfpathlineto{\pgfqpoint{2.169461in}{1.599502in}}%
\pgfpathlineto{\pgfqpoint{2.173208in}{1.524517in}}%
\pgfpathlineto{\pgfqpoint{2.182950in}{1.395973in}}%
\pgfpathlineto{\pgfqpoint{2.188946in}{1.331700in}}%
\pgfpathlineto{\pgfqpoint{2.194500in}{1.256716in}}%
\pgfpathlineto{\pgfqpoint{2.203846in}{1.192444in}}%
\pgfpathlineto{\pgfqpoint{2.207241in}{1.256716in}}%
\pgfpathlineto{\pgfqpoint{2.212619in}{1.128171in}}%
\pgfpathlineto{\pgfqpoint{2.219011in}{1.063899in}}%
\pgfpathlineto{\pgfqpoint{2.221656in}{1.128171in}}%
\pgfpathlineto{\pgfqpoint{2.224566in}{1.063899in}}%
\pgfpathlineto{\pgfqpoint{2.227211in}{1.128171in}}%
\pgfpathlineto{\pgfqpoint{2.235763in}{1.063899in}}%
\pgfpathlineto{\pgfqpoint{2.238408in}{1.128171in}}%
\pgfpathlineto{\pgfqpoint{2.249385in}{1.063899in}}%
\pgfpathlineto{\pgfqpoint{2.252030in}{1.128171in}}%
\pgfpathlineto{\pgfqpoint{2.254675in}{1.063899in}}%
\pgfpathlineto{\pgfqpoint{2.257408in}{1.128171in}}%
\pgfpathlineto{\pgfqpoint{2.262787in}{0.988915in}}%
\pgfpathlineto{\pgfqpoint{2.265432in}{1.063899in}}%
\pgfpathlineto{\pgfqpoint{2.268077in}{0.988915in}}%
\pgfpathlineto{\pgfqpoint{2.310574in}{0.924643in}}%
\pgfpathlineto{\pgfqpoint{2.313219in}{0.988915in}}%
\pgfpathlineto{\pgfqpoint{2.315864in}{0.924643in}}%
\pgfpathlineto{\pgfqpoint{2.333056in}{0.796098in}}%
\pgfpathlineto{\pgfqpoint{2.335745in}{0.860370in}}%
\pgfpathlineto{\pgfqpoint{2.342138in}{0.721114in}}%
\pgfpathlineto{\pgfqpoint{2.345312in}{0.796098in}}%
\pgfpathlineto{\pgfqpoint{2.347957in}{0.721114in}}%
\pgfpathlineto{\pgfqpoint{2.352541in}{0.656841in}}%
\pgfpathlineto{\pgfqpoint{2.355187in}{0.721114in}}%
\pgfpathlineto{\pgfqpoint{2.357832in}{0.656841in}}%
\pgfpathlineto{\pgfqpoint{2.376523in}{1.128171in}}%
\pgfpathlineto{\pgfqpoint{2.381813in}{1.331700in}}%
\pgfpathlineto{\pgfqpoint{2.396185in}{1.663774in}}%
\pgfpathlineto{\pgfqpoint{2.407162in}{1.867303in}}%
\pgfpathlineto{\pgfqpoint{2.417036in}{1.995848in}}%
\pgfpathlineto{\pgfqpoint{2.423429in}{2.060120in}}%
\pgfpathlineto{\pgfqpoint{2.426162in}{1.995848in}}%
\pgfpathlineto{\pgfqpoint{2.431672in}{2.135104in}}%
\pgfpathlineto{\pgfqpoint{2.434670in}{2.060120in}}%
\pgfpathlineto{\pgfqpoint{2.437315in}{2.135104in}}%
\pgfpathlineto{\pgfqpoint{2.440754in}{2.199377in}}%
\pgfpathlineto{\pgfqpoint{2.443531in}{2.135104in}}%
\pgfpathlineto{\pgfqpoint{2.449394in}{2.263649in}}%
\pgfpathlineto{\pgfqpoint{2.452083in}{2.199377in}}%
\pgfpathlineto{\pgfqpoint{2.457902in}{2.327921in}}%
\pgfpathlineto{\pgfqpoint{2.460547in}{2.263649in}}%
\pgfpathlineto{\pgfqpoint{2.463192in}{2.327921in}}%
\pgfpathlineto{\pgfqpoint{2.469981in}{2.402906in}}%
\pgfpathlineto{\pgfqpoint{2.472626in}{2.327921in}}%
\pgfpathlineto{\pgfqpoint{2.475271in}{2.402906in}}%
\pgfpathlineto{\pgfqpoint{2.483735in}{2.467178in}}%
\pgfpathlineto{\pgfqpoint{2.486380in}{2.402906in}}%
\pgfpathlineto{\pgfqpoint{2.489026in}{2.467178in}}%
\pgfpathlineto{\pgfqpoint{2.492067in}{2.402906in}}%
\pgfpathlineto{\pgfqpoint{2.494712in}{2.467178in}}%
\pgfpathlineto{\pgfqpoint{2.508290in}{2.531450in}}%
\pgfpathlineto{\pgfqpoint{2.510935in}{2.467178in}}%
\pgfpathlineto{\pgfqpoint{2.513580in}{2.531450in}}%
\pgfpathlineto{\pgfqpoint{2.516225in}{2.467178in}}%
\pgfpathlineto{\pgfqpoint{2.518870in}{2.531450in}}%
\pgfpathlineto{\pgfqpoint{2.530464in}{2.595723in}}%
\pgfpathlineto{\pgfqpoint{2.533109in}{2.531450in}}%
\pgfpathlineto{\pgfqpoint{2.535799in}{2.595723in}}%
\pgfpathlineto{\pgfqpoint{2.538444in}{2.531450in}}%
\pgfpathlineto{\pgfqpoint{2.541089in}{2.595723in}}%
\pgfpathlineto{\pgfqpoint{2.561279in}{2.670707in}}%
\pgfpathlineto{\pgfqpoint{2.563924in}{2.595723in}}%
\pgfpathlineto{\pgfqpoint{2.566569in}{2.670707in}}%
\pgfpathlineto{\pgfqpoint{2.569214in}{2.595723in}}%
\pgfpathlineto{\pgfqpoint{2.571859in}{2.670707in}}%
\pgfpathlineto{\pgfqpoint{2.575033in}{2.595723in}}%
\pgfpathlineto{\pgfqpoint{2.577678in}{2.670707in}}%
\pgfpathlineto{\pgfqpoint{2.582528in}{2.595723in}}%
\pgfpathlineto{\pgfqpoint{2.585173in}{2.670707in}}%
\pgfpathlineto{\pgfqpoint{2.597648in}{2.734979in}}%
\pgfpathlineto{\pgfqpoint{2.600293in}{2.670707in}}%
\pgfpathlineto{\pgfqpoint{2.602939in}{2.734979in}}%
\pgfpathlineto{\pgfqpoint{2.605584in}{2.670707in}}%
\pgfpathlineto{\pgfqpoint{2.608229in}{2.734979in}}%
\pgfpathlineto{\pgfqpoint{2.610918in}{2.670707in}}%
\pgfpathlineto{\pgfqpoint{2.613563in}{2.734979in}}%
\pgfpathlineto{\pgfqpoint{2.632739in}{2.799252in}}%
\pgfpathlineto{\pgfqpoint{2.635384in}{2.734979in}}%
\pgfpathlineto{\pgfqpoint{2.641953in}{2.799252in}}%
\pgfpathlineto{\pgfqpoint{2.644598in}{2.734979in}}%
\pgfpathlineto{\pgfqpoint{2.647596in}{2.799252in}}%
\pgfpathlineto{\pgfqpoint{2.650241in}{2.734979in}}%
\pgfpathlineto{\pgfqpoint{2.653150in}{2.799252in}}%
\pgfpathlineto{\pgfqpoint{2.655795in}{2.734979in}}%
\pgfpathlineto{\pgfqpoint{2.658661in}{2.799252in}}%
\pgfpathlineto{\pgfqpoint{2.661306in}{2.734979in}}%
\pgfpathlineto{\pgfqpoint{2.663995in}{2.799252in}}%
\pgfpathlineto{\pgfqpoint{2.666640in}{2.734979in}}%
\pgfpathlineto{\pgfqpoint{2.669285in}{2.799252in}}%
\pgfpathlineto{\pgfqpoint{2.671974in}{2.734979in}}%
\pgfpathlineto{\pgfqpoint{2.674619in}{2.799252in}}%
\pgfpathlineto{\pgfqpoint{2.677308in}{2.734979in}}%
\pgfpathlineto{\pgfqpoint{2.679953in}{2.799252in}}%
\pgfpathlineto{\pgfqpoint{2.684406in}{2.734979in}}%
\pgfpathlineto{\pgfqpoint{2.687051in}{2.799252in}}%
\pgfpathlineto{\pgfqpoint{2.700276in}{2.734979in}}%
\pgfpathlineto{\pgfqpoint{2.702921in}{2.799252in}}%
\pgfpathlineto{\pgfqpoint{2.745286in}{2.863524in}}%
\pgfpathlineto{\pgfqpoint{2.747931in}{2.799252in}}%
\pgfpathlineto{\pgfqpoint{2.750664in}{2.863524in}}%
\pgfpathlineto{\pgfqpoint{2.753309in}{2.799252in}}%
\pgfpathlineto{\pgfqpoint{2.755954in}{2.863524in}}%
\pgfpathlineto{\pgfqpoint{2.758643in}{2.799252in}}%
\pgfpathlineto{\pgfqpoint{2.761288in}{2.863524in}}%
\pgfpathlineto{\pgfqpoint{2.763977in}{2.799252in}}%
\pgfpathlineto{\pgfqpoint{2.766622in}{2.863524in}}%
\pgfpathlineto{\pgfqpoint{2.769311in}{2.799252in}}%
\pgfpathlineto{\pgfqpoint{2.771957in}{2.863524in}}%
\pgfpathlineto{\pgfqpoint{2.774602in}{2.799252in}}%
\pgfpathlineto{\pgfqpoint{2.777247in}{2.863524in}}%
\pgfpathlineto{\pgfqpoint{2.780024in}{2.799252in}}%
\pgfpathlineto{\pgfqpoint{2.782669in}{2.863524in}}%
\pgfpathlineto{\pgfqpoint{2.857171in}{2.938508in}}%
\pgfpathlineto{\pgfqpoint{2.859816in}{2.863524in}}%
\pgfpathlineto{\pgfqpoint{2.862549in}{2.938508in}}%
\pgfpathlineto{\pgfqpoint{2.865194in}{2.863524in}}%
\pgfpathlineto{\pgfqpoint{2.867883in}{2.938508in}}%
\pgfpathlineto{\pgfqpoint{2.870528in}{2.863524in}}%
\pgfpathlineto{\pgfqpoint{2.873261in}{2.938508in}}%
\pgfpathlineto{\pgfqpoint{2.875907in}{2.863524in}}%
\pgfpathlineto{\pgfqpoint{2.878684in}{2.938508in}}%
\pgfpathlineto{\pgfqpoint{2.881329in}{2.863524in}}%
\pgfpathlineto{\pgfqpoint{2.884018in}{2.938508in}}%
\pgfpathlineto{\pgfqpoint{2.886663in}{2.863524in}}%
\pgfpathlineto{\pgfqpoint{2.889352in}{2.938508in}}%
\pgfpathlineto{\pgfqpoint{2.891997in}{2.863524in}}%
\pgfpathlineto{\pgfqpoint{2.894642in}{2.938508in}}%
\pgfpathlineto{\pgfqpoint{2.897287in}{2.863524in}}%
\pgfpathlineto{\pgfqpoint{2.900020in}{2.938508in}}%
\pgfpathlineto{\pgfqpoint{2.902666in}{2.863524in}}%
\pgfpathlineto{\pgfqpoint{2.905311in}{2.938508in}}%
\pgfpathlineto{\pgfqpoint{2.907956in}{2.863524in}}%
\pgfpathlineto{\pgfqpoint{2.910601in}{2.938508in}}%
\pgfpathlineto{\pgfqpoint{2.913290in}{2.863524in}}%
\pgfpathlineto{\pgfqpoint{2.915935in}{2.938508in}}%
\pgfpathlineto{\pgfqpoint{2.918624in}{2.863524in}}%
\pgfpathlineto{\pgfqpoint{2.921313in}{2.938508in}}%
\pgfpathlineto{\pgfqpoint{2.923958in}{2.863524in}}%
\pgfpathlineto{\pgfqpoint{2.926647in}{2.938508in}}%
\pgfpathlineto{\pgfqpoint{2.929292in}{2.863524in}}%
\pgfpathlineto{\pgfqpoint{2.931937in}{2.938508in}}%
\pgfpathlineto{\pgfqpoint{2.934582in}{2.863524in}}%
\pgfpathlineto{\pgfqpoint{2.937227in}{2.938508in}}%
\pgfpathlineto{\pgfqpoint{2.941063in}{2.863524in}}%
\pgfpathlineto{\pgfqpoint{2.943708in}{2.938508in}}%
\pgfpathlineto{\pgfqpoint{2.995683in}{2.863524in}}%
\pgfpathlineto{\pgfqpoint{3.004235in}{2.670707in}}%
\pgfpathlineto{\pgfqpoint{3.012214in}{2.467178in}}%
\pgfpathlineto{\pgfqpoint{3.040252in}{1.792319in}}%
\pgfpathlineto{\pgfqpoint{3.044396in}{1.728046in}}%
\pgfpathlineto{\pgfqpoint{3.047129in}{1.792319in}}%
\pgfpathlineto{\pgfqpoint{3.052419in}{1.663774in}}%
\pgfpathlineto{\pgfqpoint{3.058370in}{1.599502in}}%
\pgfpathlineto{\pgfqpoint{3.061147in}{1.663774in}}%
\pgfpathlineto{\pgfqpoint{3.069127in}{1.460245in}}%
\pgfpathlineto{\pgfqpoint{3.080500in}{1.331700in}}%
\pgfpathlineto{\pgfqpoint{3.085394in}{1.256716in}}%
\pgfpathlineto{\pgfqpoint{3.088039in}{1.331700in}}%
\pgfpathlineto{\pgfqpoint{3.090684in}{1.256716in}}%
\pgfpathlineto{\pgfqpoint{3.095004in}{1.192444in}}%
\pgfpathlineto{\pgfqpoint{3.097649in}{1.256716in}}%
\pgfpathlineto{\pgfqpoint{3.100294in}{1.192444in}}%
\pgfpathlineto{\pgfqpoint{3.108670in}{1.128171in}}%
\pgfpathlineto{\pgfqpoint{3.111359in}{1.192444in}}%
\pgfpathlineto{\pgfqpoint{3.114004in}{1.128171in}}%
\pgfpathlineto{\pgfqpoint{3.119867in}{1.063899in}}%
\pgfpathlineto{\pgfqpoint{3.122512in}{1.128171in}}%
\pgfpathlineto{\pgfqpoint{3.125157in}{1.063899in}}%
\pgfpathlineto{\pgfqpoint{3.132696in}{0.988915in}}%
\pgfpathlineto{\pgfqpoint{3.135385in}{1.063899in}}%
\pgfpathlineto{\pgfqpoint{3.138030in}{0.988915in}}%
\pgfpathlineto{\pgfqpoint{3.142438in}{0.924643in}}%
\pgfpathlineto{\pgfqpoint{3.145127in}{0.988915in}}%
\pgfpathlineto{\pgfqpoint{3.147772in}{0.924643in}}%
\pgfpathlineto{\pgfqpoint{3.154209in}{0.860370in}}%
\pgfpathlineto{\pgfqpoint{3.156854in}{0.924643in}}%
\pgfpathlineto{\pgfqpoint{3.159499in}{0.860370in}}%
\pgfpathlineto{\pgfqpoint{3.162144in}{0.924643in}}%
\pgfpathlineto{\pgfqpoint{3.164789in}{0.860370in}}%
\pgfpathlineto{\pgfqpoint{3.167434in}{0.924643in}}%
\pgfpathlineto{\pgfqpoint{3.170079in}{0.860370in}}%
\pgfpathlineto{\pgfqpoint{3.172812in}{0.924643in}}%
\pgfpathlineto{\pgfqpoint{3.175457in}{0.860370in}}%
\pgfpathlineto{\pgfqpoint{3.178146in}{0.924643in}}%
\pgfpathlineto{\pgfqpoint{3.180791in}{0.860370in}}%
\pgfpathlineto{\pgfqpoint{3.183436in}{0.924643in}}%
\pgfpathlineto{\pgfqpoint{3.186081in}{0.860370in}}%
\pgfpathlineto{\pgfqpoint{3.217028in}{0.796098in}}%
\pgfpathlineto{\pgfqpoint{3.219673in}{0.860370in}}%
\pgfpathlineto{\pgfqpoint{3.222318in}{0.796098in}}%
\pgfpathlineto{\pgfqpoint{3.234794in}{0.721114in}}%
\pgfpathlineto{\pgfqpoint{3.237439in}{0.796098in}}%
\pgfpathlineto{\pgfqpoint{3.240084in}{0.721114in}}%
\pgfpathlineto{\pgfqpoint{3.242773in}{0.796098in}}%
\pgfpathlineto{\pgfqpoint{3.245418in}{0.721114in}}%
\pgfpathlineto{\pgfqpoint{3.248989in}{0.796098in}}%
\pgfpathlineto{\pgfqpoint{3.251634in}{0.721114in}}%
\pgfpathlineto{\pgfqpoint{3.258423in}{0.860370in}}%
\pgfpathlineto{\pgfqpoint{3.269003in}{1.128171in}}%
\pgfpathlineto{\pgfqpoint{3.271648in}{1.256716in}}%
\pgfpathlineto{\pgfqpoint{3.277379in}{1.395973in}}%
\pgfpathlineto{\pgfqpoint{3.298981in}{1.792319in}}%
\pgfpathlineto{\pgfqpoint{3.315732in}{1.995848in}}%
\pgfpathlineto{\pgfqpoint{3.318510in}{1.931575in}}%
\pgfpathlineto{\pgfqpoint{3.323844in}{2.060120in}}%
\pgfpathlineto{\pgfqpoint{3.331515in}{2.135104in}}%
\pgfpathlineto{\pgfqpoint{3.334160in}{2.060120in}}%
\pgfpathlineto{\pgfqpoint{3.336805in}{2.135104in}}%
\pgfpathlineto{\pgfqpoint{3.341962in}{2.199377in}}%
\pgfpathlineto{\pgfqpoint{3.344607in}{2.135104in}}%
\pgfpathlineto{\pgfqpoint{3.350382in}{2.263649in}}%
\pgfpathlineto{\pgfqpoint{3.353027in}{2.199377in}}%
\pgfpathlineto{\pgfqpoint{3.355673in}{2.263649in}}%
\pgfpathlineto{\pgfqpoint{3.358979in}{2.199377in}}%
\pgfpathlineto{\pgfqpoint{3.361624in}{2.263649in}}%
\pgfpathlineto{\pgfqpoint{3.365900in}{2.327921in}}%
\pgfpathlineto{\pgfqpoint{3.368545in}{2.263649in}}%
\pgfpathlineto{\pgfqpoint{3.371190in}{2.327921in}}%
\pgfpathlineto{\pgfqpoint{3.380624in}{2.402906in}}%
\pgfpathlineto{\pgfqpoint{3.383269in}{2.327921in}}%
\pgfpathlineto{\pgfqpoint{3.386002in}{2.402906in}}%
\pgfpathlineto{\pgfqpoint{3.389000in}{2.327921in}}%
\pgfpathlineto{\pgfqpoint{3.391645in}{2.402906in}}%
\pgfpathlineto{\pgfqpoint{3.398919in}{2.467178in}}%
\pgfpathlineto{\pgfqpoint{3.401564in}{2.402906in}}%
\pgfpathlineto{\pgfqpoint{3.404209in}{2.467178in}}%
\pgfpathlineto{\pgfqpoint{3.406942in}{2.402906in}}%
\pgfpathlineto{\pgfqpoint{3.409587in}{2.467178in}}%
\pgfpathlineto{\pgfqpoint{3.412585in}{2.402906in}}%
\pgfpathlineto{\pgfqpoint{3.415230in}{2.467178in}}%
\pgfpathlineto{\pgfqpoint{3.426956in}{2.531450in}}%
\pgfpathlineto{\pgfqpoint{3.429601in}{2.467178in}}%
\pgfpathlineto{\pgfqpoint{3.432246in}{2.531450in}}%
\pgfpathlineto{\pgfqpoint{3.435200in}{2.467178in}}%
\pgfpathlineto{\pgfqpoint{3.437845in}{2.531450in}}%
\pgfpathlineto{\pgfqpoint{3.452349in}{2.595723in}}%
\pgfpathlineto{\pgfqpoint{3.454994in}{2.531450in}}%
\pgfpathlineto{\pgfqpoint{3.457859in}{2.595723in}}%
\pgfpathlineto{\pgfqpoint{3.460504in}{2.531450in}}%
\pgfpathlineto{\pgfqpoint{3.463193in}{2.595723in}}%
\pgfpathlineto{\pgfqpoint{3.465882in}{2.531450in}}%
\pgfpathlineto{\pgfqpoint{3.468528in}{2.595723in}}%
\pgfpathlineto{\pgfqpoint{3.471437in}{2.531450in}}%
\pgfpathlineto{\pgfqpoint{3.474082in}{2.595723in}}%
\pgfpathlineto{\pgfqpoint{3.492641in}{2.670707in}}%
\pgfpathlineto{\pgfqpoint{3.495287in}{2.595723in}}%
\pgfpathlineto{\pgfqpoint{3.498020in}{2.670707in}}%
\pgfpathlineto{\pgfqpoint{3.500665in}{2.595723in}}%
\pgfpathlineto{\pgfqpoint{3.503310in}{2.670707in}}%
\pgfpathlineto{\pgfqpoint{3.505955in}{2.595723in}}%
\pgfpathlineto{\pgfqpoint{3.508600in}{2.670707in}}%
\pgfpathlineto{\pgfqpoint{3.511421in}{2.595723in}}%
\pgfpathlineto{\pgfqpoint{3.514066in}{2.670707in}}%
\pgfpathlineto{\pgfqpoint{3.518342in}{2.595723in}}%
\pgfpathlineto{\pgfqpoint{3.520987in}{2.670707in}}%
\pgfpathlineto{\pgfqpoint{3.535094in}{2.734979in}}%
\pgfpathlineto{\pgfqpoint{3.537739in}{2.670707in}}%
\pgfpathlineto{\pgfqpoint{3.540473in}{2.734979in}}%
\pgfpathlineto{\pgfqpoint{3.543118in}{2.670707in}}%
\pgfpathlineto{\pgfqpoint{3.545763in}{2.734979in}}%
\pgfpathlineto{\pgfqpoint{3.548408in}{2.670707in}}%
\pgfpathlineto{\pgfqpoint{3.551053in}{2.734979in}}%
\pgfpathlineto{\pgfqpoint{3.553698in}{2.670707in}}%
\pgfpathlineto{\pgfqpoint{3.556387in}{2.734979in}}%
\pgfpathlineto{\pgfqpoint{3.559032in}{2.670707in}}%
\pgfpathlineto{\pgfqpoint{3.561721in}{2.734979in}}%
\pgfpathlineto{\pgfqpoint{3.564366in}{2.670707in}}%
\pgfpathlineto{\pgfqpoint{3.567011in}{2.734979in}}%
\pgfpathlineto{\pgfqpoint{3.569744in}{2.670707in}}%
\pgfpathlineto{\pgfqpoint{3.572389in}{2.734979in}}%
\pgfpathlineto{\pgfqpoint{3.580369in}{2.670707in}}%
\pgfpathlineto{\pgfqpoint{3.583014in}{2.734979in}}%
\pgfpathlineto{\pgfqpoint{3.601793in}{2.799252in}}%
\pgfpathlineto{\pgfqpoint{3.604438in}{2.734979in}}%
\pgfpathlineto{\pgfqpoint{3.607172in}{2.799252in}}%
\pgfpathlineto{\pgfqpoint{3.609817in}{2.734979in}}%
\pgfpathlineto{\pgfqpoint{3.612682in}{2.799252in}}%
\pgfpathlineto{\pgfqpoint{3.615327in}{2.734979in}}%
\pgfpathlineto{\pgfqpoint{3.618016in}{2.799252in}}%
\pgfpathlineto{\pgfqpoint{3.620705in}{2.734979in}}%
\pgfpathlineto{\pgfqpoint{3.623350in}{2.799252in}}%
\pgfpathlineto{\pgfqpoint{3.626216in}{2.734979in}}%
\pgfpathlineto{\pgfqpoint{3.628861in}{2.799252in}}%
\pgfpathlineto{\pgfqpoint{3.632123in}{2.734979in}}%
\pgfpathlineto{\pgfqpoint{3.634768in}{2.799252in}}%
\pgfpathlineto{\pgfqpoint{3.639397in}{2.734979in}}%
\pgfpathlineto{\pgfqpoint{3.642042in}{2.799252in}}%
\pgfpathlineto{\pgfqpoint{3.653195in}{2.734979in}}%
\pgfpathlineto{\pgfqpoint{3.655840in}{2.799252in}}%
\pgfpathlineto{\pgfqpoint{3.675061in}{2.863524in}}%
\pgfpathlineto{\pgfqpoint{3.677706in}{2.799252in}}%
\pgfpathlineto{\pgfqpoint{3.680616in}{2.863524in}}%
\pgfpathlineto{\pgfqpoint{3.683261in}{2.799252in}}%
\pgfpathlineto{\pgfqpoint{3.685906in}{2.863524in}}%
\pgfpathlineto{\pgfqpoint{3.688551in}{2.799252in}}%
\pgfpathlineto{\pgfqpoint{3.691196in}{2.863524in}}%
\pgfpathlineto{\pgfqpoint{3.693841in}{2.799252in}}%
\pgfpathlineto{\pgfqpoint{3.696574in}{2.863524in}}%
\pgfpathlineto{\pgfqpoint{3.699219in}{2.799252in}}%
\pgfpathlineto{\pgfqpoint{3.701864in}{2.863524in}}%
\pgfpathlineto{\pgfqpoint{3.704509in}{2.799252in}}%
\pgfpathlineto{\pgfqpoint{3.707154in}{2.863524in}}%
\pgfpathlineto{\pgfqpoint{3.709799in}{2.799252in}}%
\pgfpathlineto{\pgfqpoint{3.712444in}{2.863524in}}%
\pgfpathlineto{\pgfqpoint{3.715177in}{2.799252in}}%
\pgfpathlineto{\pgfqpoint{3.717822in}{2.863524in}}%
\pgfpathlineto{\pgfqpoint{3.720467in}{2.799252in}}%
\pgfpathlineto{\pgfqpoint{3.723113in}{2.863524in}}%
\pgfpathlineto{\pgfqpoint{3.726066in}{2.799252in}}%
\pgfpathlineto{\pgfqpoint{3.728711in}{2.863524in}}%
\pgfpathlineto{\pgfqpoint{3.731797in}{2.799252in}}%
\pgfpathlineto{\pgfqpoint{3.734442in}{2.863524in}}%
\pgfpathlineto{\pgfqpoint{3.737131in}{2.799252in}}%
\pgfpathlineto{\pgfqpoint{3.739776in}{2.863524in}}%
\pgfpathlineto{\pgfqpoint{3.742686in}{2.799252in}}%
\pgfpathlineto{\pgfqpoint{3.745331in}{2.863524in}}%
\pgfpathlineto{\pgfqpoint{3.748108in}{2.799252in}}%
\pgfpathlineto{\pgfqpoint{3.750753in}{2.863524in}}%
\pgfpathlineto{\pgfqpoint{3.782846in}{2.938508in}}%
\pgfpathlineto{\pgfqpoint{3.785491in}{2.863524in}}%
\pgfpathlineto{\pgfqpoint{3.788225in}{2.938508in}}%
\pgfpathlineto{\pgfqpoint{3.790870in}{2.863524in}}%
\pgfpathlineto{\pgfqpoint{3.793603in}{2.938508in}}%
\pgfpathlineto{\pgfqpoint{3.796248in}{2.863524in}}%
\pgfpathlineto{\pgfqpoint{3.798893in}{2.938508in}}%
\pgfpathlineto{\pgfqpoint{3.801538in}{2.863524in}}%
\pgfpathlineto{\pgfqpoint{3.804183in}{2.938508in}}%
\pgfpathlineto{\pgfqpoint{3.806960in}{2.863524in}}%
\pgfpathlineto{\pgfqpoint{3.809605in}{2.938508in}}%
\pgfpathlineto{\pgfqpoint{3.812647in}{2.863524in}}%
\pgfpathlineto{\pgfqpoint{3.815292in}{2.938508in}}%
\pgfpathlineto{\pgfqpoint{3.818202in}{2.863524in}}%
\pgfpathlineto{\pgfqpoint{3.820847in}{2.938508in}}%
\pgfpathlineto{\pgfqpoint{3.857613in}{3.002780in}}%
\pgfpathlineto{\pgfqpoint{3.860258in}{2.938508in}}%
\pgfpathlineto{\pgfqpoint{3.863256in}{3.002780in}}%
\pgfpathlineto{\pgfqpoint{3.865901in}{2.938508in}}%
\pgfpathlineto{\pgfqpoint{3.869692in}{3.002780in}}%
\pgfpathlineto{\pgfqpoint{3.872337in}{2.938508in}}%
\pgfpathlineto{\pgfqpoint{3.874982in}{3.002780in}}%
\pgfpathlineto{\pgfqpoint{3.877627in}{2.938508in}}%
\pgfpathlineto{\pgfqpoint{3.880272in}{3.002780in}}%
\pgfpathlineto{\pgfqpoint{3.882917in}{2.938508in}}%
\pgfpathlineto{\pgfqpoint{3.885562in}{3.002780in}}%
\pgfpathlineto{\pgfqpoint{3.888295in}{2.938508in}}%
\pgfpathlineto{\pgfqpoint{3.890940in}{3.002780in}}%
\pgfpathlineto{\pgfqpoint{3.907075in}{2.938508in}}%
\pgfpathlineto{\pgfqpoint{3.909720in}{3.002780in}}%
\pgfpathlineto{\pgfqpoint{3.949881in}{2.938508in}}%
\pgfpathlineto{\pgfqpoint{3.960725in}{2.670707in}}%
\pgfpathlineto{\pgfqpoint{3.963370in}{2.531450in}}%
\pgfpathlineto{\pgfqpoint{3.971746in}{2.327921in}}%
\pgfpathlineto{\pgfqpoint{3.979329in}{2.199377in}}%
\pgfpathlineto{\pgfqpoint{3.985544in}{2.060120in}}%
\pgfpathlineto{\pgfqpoint{3.991011in}{1.995848in}}%
\pgfpathlineto{\pgfqpoint{3.994405in}{1.931575in}}%
\pgfpathlineto{\pgfqpoint{3.997403in}{1.995848in}}%
\pgfpathlineto{\pgfqpoint{4.005559in}{1.792319in}}%
\pgfpathlineto{\pgfqpoint{4.013538in}{1.728046in}}%
\pgfpathlineto{\pgfqpoint{4.016227in}{1.792319in}}%
\pgfpathlineto{\pgfqpoint{4.021693in}{1.663774in}}%
\pgfpathlineto{\pgfqpoint{4.024471in}{1.728046in}}%
\pgfpathlineto{\pgfqpoint{4.027116in}{1.599502in}}%
\pgfpathlineto{\pgfqpoint{4.038445in}{1.395973in}}%
\pgfpathlineto{\pgfqpoint{4.051318in}{1.256716in}}%
\pgfpathlineto{\pgfqpoint{4.054227in}{1.331700in}}%
\pgfpathlineto{\pgfqpoint{4.059561in}{1.192444in}}%
\pgfpathlineto{\pgfqpoint{4.065336in}{1.128171in}}%
\pgfpathlineto{\pgfqpoint{4.068026in}{1.192444in}}%
\pgfpathlineto{\pgfqpoint{4.073316in}{1.063899in}}%
\pgfpathlineto{\pgfqpoint{4.076534in}{1.128171in}}%
\pgfpathlineto{\pgfqpoint{4.079179in}{1.063899in}}%
\pgfpathlineto{\pgfqpoint{4.099546in}{0.988915in}}%
\pgfpathlineto{\pgfqpoint{4.102191in}{1.063899in}}%
\pgfpathlineto{\pgfqpoint{4.129655in}{0.988915in}}%
\pgfpathlineto{\pgfqpoint{4.132300in}{1.063899in}}%
\pgfpathlineto{\pgfqpoint{4.134945in}{0.988915in}}%
\pgfpathlineto{\pgfqpoint{4.138560in}{1.063899in}}%
\pgfpathlineto{\pgfqpoint{4.141205in}{0.988915in}}%
\pgfpathlineto{\pgfqpoint{4.148303in}{0.924643in}}%
\pgfpathlineto{\pgfqpoint{4.150948in}{0.988915in}}%
\pgfpathlineto{\pgfqpoint{4.153593in}{0.924643in}}%
\pgfpathlineto{\pgfqpoint{4.158221in}{0.988915in}}%
\pgfpathlineto{\pgfqpoint{4.160866in}{0.924643in}}%
\pgfpathlineto{\pgfqpoint{4.175855in}{0.860370in}}%
\pgfpathlineto{\pgfqpoint{4.179029in}{0.924643in}}%
\pgfpathlineto{\pgfqpoint{4.181674in}{0.860370in}}%
\pgfpathlineto{\pgfqpoint{4.191681in}{0.796098in}}%
\pgfpathlineto{\pgfqpoint{4.194326in}{0.860370in}}%
\pgfpathlineto{\pgfqpoint{4.196971in}{0.796098in}}%
\pgfpathlineto{\pgfqpoint{4.199749in}{0.860370in}}%
\pgfpathlineto{\pgfqpoint{4.202394in}{0.796098in}}%
\pgfpathlineto{\pgfqpoint{4.231577in}{1.524517in}}%
\pgfpathlineto{\pgfqpoint{4.237088in}{1.663774in}}%
\pgfpathlineto{\pgfqpoint{4.243877in}{1.792319in}}%
\pgfpathlineto{\pgfqpoint{4.252429in}{1.931575in}}%
\pgfpathlineto{\pgfqpoint{4.269842in}{2.135104in}}%
\pgfpathlineto{\pgfqpoint{4.272487in}{2.060120in}}%
\pgfpathlineto{\pgfqpoint{4.275132in}{2.135104in}}%
\pgfpathlineto{\pgfqpoint{4.278923in}{2.199377in}}%
\pgfpathlineto{\pgfqpoint{4.281613in}{2.135104in}}%
\pgfpathlineto{\pgfqpoint{4.287784in}{2.263649in}}%
\pgfpathlineto{\pgfqpoint{4.290473in}{2.199377in}}%
\pgfpathlineto{\pgfqpoint{4.293118in}{2.263649in}}%
\pgfpathlineto{\pgfqpoint{4.299158in}{2.327921in}}%
\pgfpathlineto{\pgfqpoint{4.301803in}{2.263649in}}%
\pgfpathlineto{\pgfqpoint{4.304448in}{2.327921in}}%
\pgfpathlineto{\pgfqpoint{4.313926in}{2.402906in}}%
\pgfpathlineto{\pgfqpoint{4.316571in}{2.327921in}}%
\pgfpathlineto{\pgfqpoint{4.319260in}{2.402906in}}%
\pgfpathlineto{\pgfqpoint{4.321949in}{2.327921in}}%
\pgfpathlineto{\pgfqpoint{4.324594in}{2.402906in}}%
\pgfpathlineto{\pgfqpoint{4.329752in}{2.467178in}}%
\pgfpathlineto{\pgfqpoint{4.332397in}{2.402906in}}%
\pgfpathlineto{\pgfqpoint{4.335042in}{2.467178in}}%
\pgfpathlineto{\pgfqpoint{4.351794in}{2.531450in}}%
\pgfpathlineto{\pgfqpoint{4.354439in}{2.467178in}}%
\pgfpathlineto{\pgfqpoint{4.357084in}{2.531450in}}%
\pgfpathlineto{\pgfqpoint{4.359862in}{2.467178in}}%
\pgfpathlineto{\pgfqpoint{4.362507in}{2.531450in}}%
\pgfpathlineto{\pgfqpoint{4.374850in}{2.595723in}}%
\pgfpathlineto{\pgfqpoint{4.377495in}{2.531450in}}%
\pgfpathlineto{\pgfqpoint{4.380140in}{2.595723in}}%
\pgfpathlineto{\pgfqpoint{4.382785in}{2.531450in}}%
\pgfpathlineto{\pgfqpoint{4.385430in}{2.595723in}}%
\pgfpathlineto{\pgfqpoint{4.388252in}{2.531450in}}%
\pgfpathlineto{\pgfqpoint{4.390897in}{2.595723in}}%
\pgfpathlineto{\pgfqpoint{4.409192in}{2.670707in}}%
\pgfpathlineto{\pgfqpoint{4.411837in}{2.595723in}}%
\pgfpathlineto{\pgfqpoint{4.414526in}{2.670707in}}%
\pgfpathlineto{\pgfqpoint{4.417171in}{2.595723in}}%
\pgfpathlineto{\pgfqpoint{4.419816in}{2.670707in}}%
\pgfpathlineto{\pgfqpoint{4.422461in}{2.595723in}}%
\pgfpathlineto{\pgfqpoint{4.425106in}{2.670707in}}%
\pgfpathlineto{\pgfqpoint{4.428192in}{2.595723in}}%
\pgfpathlineto{\pgfqpoint{4.430837in}{2.670707in}}%
\pgfpathlineto{\pgfqpoint{4.434628in}{2.595723in}}%
\pgfpathlineto{\pgfqpoint{4.437273in}{2.670707in}}%
\pgfpathlineto{\pgfqpoint{4.452835in}{2.734979in}}%
\pgfpathlineto{\pgfqpoint{4.455480in}{2.670707in}}%
\pgfpathlineto{\pgfqpoint{4.458257in}{2.734979in}}%
\pgfpathlineto{\pgfqpoint{4.460902in}{2.670707in}}%
\pgfpathlineto{\pgfqpoint{4.463547in}{2.734979in}}%
\pgfpathlineto{\pgfqpoint{4.466280in}{2.670707in}}%
\pgfpathlineto{\pgfqpoint{4.468925in}{2.734979in}}%
\pgfpathlineto{\pgfqpoint{4.475362in}{2.670707in}}%
\pgfpathlineto{\pgfqpoint{4.478007in}{2.734979in}}%
\pgfpathlineto{\pgfqpoint{4.489248in}{2.799252in}}%
\pgfpathlineto{\pgfqpoint{4.491893in}{2.734979in}}%
\pgfpathlineto{\pgfqpoint{4.495596in}{2.799252in}}%
\pgfpathlineto{\pgfqpoint{4.498241in}{2.734979in}}%
\pgfpathlineto{\pgfqpoint{4.500974in}{2.799252in}}%
\pgfpathlineto{\pgfqpoint{4.503619in}{2.734979in}}%
\pgfpathlineto{\pgfqpoint{4.506264in}{2.799252in}}%
\pgfpathlineto{\pgfqpoint{4.508910in}{2.734979in}}%
\pgfpathlineto{\pgfqpoint{4.511599in}{2.799252in}}%
\pgfpathlineto{\pgfqpoint{4.514332in}{2.734979in}}%
\pgfpathlineto{\pgfqpoint{4.516977in}{2.799252in}}%
\pgfpathlineto{\pgfqpoint{4.519754in}{2.734979in}}%
\pgfpathlineto{\pgfqpoint{4.522399in}{2.799252in}}%
\pgfpathlineto{\pgfqpoint{4.525088in}{2.734979in}}%
\pgfpathlineto{\pgfqpoint{4.527733in}{2.799252in}}%
\pgfpathlineto{\pgfqpoint{4.531525in}{2.734979in}}%
\pgfpathlineto{\pgfqpoint{4.534170in}{2.799252in}}%
\pgfpathlineto{\pgfqpoint{4.557270in}{2.863524in}}%
\pgfpathlineto{\pgfqpoint{4.559915in}{2.799252in}}%
\pgfpathlineto{\pgfqpoint{4.563662in}{2.863524in}}%
\pgfpathlineto{\pgfqpoint{4.566351in}{2.799252in}}%
\pgfpathlineto{\pgfqpoint{4.568996in}{2.863524in}}%
\pgfpathlineto{\pgfqpoint{4.571641in}{2.799252in}}%
\pgfpathlineto{\pgfqpoint{4.574286in}{2.863524in}}%
\pgfpathlineto{\pgfqpoint{4.577063in}{2.799252in}}%
\pgfpathlineto{\pgfqpoint{4.579708in}{2.863524in}}%
\pgfpathlineto{\pgfqpoint{4.582618in}{2.799252in}}%
\pgfpathlineto{\pgfqpoint{4.585263in}{2.863524in}}%
\pgfpathlineto{\pgfqpoint{4.590333in}{2.799252in}}%
\pgfpathlineto{\pgfqpoint{4.592978in}{2.863524in}}%
\pgfpathlineto{\pgfqpoint{4.600252in}{2.799252in}}%
\pgfpathlineto{\pgfqpoint{4.602897in}{2.863524in}}%
\pgfpathlineto{\pgfqpoint{4.607393in}{2.799252in}}%
\pgfpathlineto{\pgfqpoint{4.610038in}{2.863524in}}%
\pgfpathlineto{\pgfqpoint{4.649097in}{2.938508in}}%
\pgfpathlineto{\pgfqpoint{4.651742in}{2.863524in}}%
\pgfpathlineto{\pgfqpoint{4.654387in}{2.938508in}}%
\pgfpathlineto{\pgfqpoint{4.657032in}{2.863524in}}%
\pgfpathlineto{\pgfqpoint{4.659677in}{2.938508in}}%
\pgfpathlineto{\pgfqpoint{4.662322in}{2.863524in}}%
\pgfpathlineto{\pgfqpoint{4.664967in}{2.938508in}}%
\pgfpathlineto{\pgfqpoint{4.664967in}{2.938508in}}%
\pgfusepath{stroke}%
\end{pgfscope}%
\begin{pgfscope}%
\pgfsetrectcap%
\pgfsetmiterjoin%
\pgfsetlinewidth{0.803000pt}%
\definecolor{currentstroke}{rgb}{0.000000,0.000000,0.000000}%
\pgfsetstrokecolor{currentstroke}%
\pgfsetdash{}{0pt}%
\pgfpathmoveto{\pgfqpoint{0.667540in}{0.539544in}}%
\pgfpathlineto{\pgfqpoint{0.667540in}{3.120077in}}%
\pgfusepath{stroke}%
\end{pgfscope}%
\begin{pgfscope}%
\pgfsetrectcap%
\pgfsetmiterjoin%
\pgfsetlinewidth{0.803000pt}%
\definecolor{currentstroke}{rgb}{0.000000,0.000000,0.000000}%
\pgfsetstrokecolor{currentstroke}%
\pgfsetdash{}{0pt}%
\pgfpathmoveto{\pgfqpoint{4.857257in}{0.539544in}}%
\pgfpathlineto{\pgfqpoint{4.857257in}{3.120077in}}%
\pgfusepath{stroke}%
\end{pgfscope}%
\begin{pgfscope}%
\pgfsetrectcap%
\pgfsetmiterjoin%
\pgfsetlinewidth{0.803000pt}%
\definecolor{currentstroke}{rgb}{0.000000,0.000000,0.000000}%
\pgfsetstrokecolor{currentstroke}%
\pgfsetdash{}{0pt}%
\pgfpathmoveto{\pgfqpoint{0.667540in}{0.539544in}}%
\pgfpathlineto{\pgfqpoint{4.857257in}{0.539544in}}%
\pgfusepath{stroke}%
\end{pgfscope}%
\begin{pgfscope}%
\pgfsetrectcap%
\pgfsetmiterjoin%
\pgfsetlinewidth{0.803000pt}%
\definecolor{currentstroke}{rgb}{0.000000,0.000000,0.000000}%
\pgfsetstrokecolor{currentstroke}%
\pgfsetdash{}{0pt}%
\pgfpathmoveto{\pgfqpoint{0.667540in}{3.120077in}}%
\pgfpathlineto{\pgfqpoint{4.857257in}{3.120077in}}%
\pgfusepath{stroke}%
\end{pgfscope}%
\begin{pgfscope}%
\pgfsetbuttcap%
\pgfsetmiterjoin%
\definecolor{currentfill}{rgb}{1.000000,1.000000,1.000000}%
\pgfsetfillcolor{currentfill}%
\pgfsetfillopacity{0.800000}%
\pgfsetlinewidth{1.003750pt}%
\definecolor{currentstroke}{rgb}{0.800000,0.800000,0.800000}%
\pgfsetstrokecolor{currentstroke}%
\pgfsetstrokeopacity{0.800000}%
\pgfsetdash{}{0pt}%
\pgfpathmoveto{\pgfqpoint{0.745318in}{2.719411in}}%
\pgfpathlineto{\pgfqpoint{2.244873in}{2.719411in}}%
\pgfpathquadraticcurveto{\pgfqpoint{2.267096in}{2.719411in}}{\pgfqpoint{2.267096in}{2.741633in}}%
\pgfpathlineto{\pgfqpoint{2.267096in}{3.042300in}}%
\pgfpathquadraticcurveto{\pgfqpoint{2.267096in}{3.064522in}}{\pgfqpoint{2.244873in}{3.064522in}}%
\pgfpathlineto{\pgfqpoint{0.745318in}{3.064522in}}%
\pgfpathquadraticcurveto{\pgfqpoint{0.723095in}{3.064522in}}{\pgfqpoint{0.723095in}{3.042300in}}%
\pgfpathlineto{\pgfqpoint{0.723095in}{2.741633in}}%
\pgfpathquadraticcurveto{\pgfqpoint{0.723095in}{2.719411in}}{\pgfqpoint{0.745318in}{2.719411in}}%
\pgfpathlineto{\pgfqpoint{0.745318in}{2.719411in}}%
\pgfpathclose%
\pgfusepath{stroke,fill}%
\end{pgfscope}%
\begin{pgfscope}%
\pgfsetrectcap%
\pgfsetroundjoin%
\pgfsetlinewidth{1.505625pt}%
\definecolor{currentstroke}{rgb}{0.003922,0.450980,0.698039}%
\pgfsetstrokecolor{currentstroke}%
\pgfsetstrokeopacity{0.700000}%
\pgfsetdash{}{0pt}%
\pgfpathmoveto{\pgfqpoint{0.767540in}{2.980744in}}%
\pgfpathlineto{\pgfqpoint{0.878651in}{2.980744in}}%
\pgfpathlineto{\pgfqpoint{0.989762in}{2.980744in}}%
\pgfusepath{stroke}%
\end{pgfscope}%
\begin{pgfscope}%
\definecolor{textcolor}{rgb}{0.000000,0.000000,0.000000}%
\pgfsetstrokecolor{textcolor}%
\pgfsetfillcolor{textcolor}%
\pgftext[x=1.078651in,y=2.941855in,left,base]{\color{textcolor}\rmfamily\fontsize{8.000000}{9.600000}\selectfont Output current}%
\end{pgfscope}%
\begin{pgfscope}%
\pgfsetrectcap%
\pgfsetroundjoin%
\pgfsetlinewidth{1.505625pt}%
\definecolor{currentstroke}{rgb}{0.909804,0.000000,0.043137}%
\pgfsetstrokecolor{currentstroke}%
\pgfsetstrokeopacity{0.700000}%
\pgfsetdash{}{0pt}%
\pgfpathmoveto{\pgfqpoint{0.767540in}{2.824300in}}%
\pgfpathlineto{\pgfqpoint{0.878651in}{2.824300in}}%
\pgfpathlineto{\pgfqpoint{0.989762in}{2.824300in}}%
\pgfusepath{stroke}%
\end{pgfscope}%
\begin{pgfscope}%
\definecolor{textcolor}{rgb}{0.000000,0.000000,0.000000}%
\pgfsetstrokecolor{textcolor}%
\pgfsetfillcolor{textcolor}%
\pgftext[x=1.078651in,y=2.785411in,left,base]{\color{textcolor}\rmfamily\fontsize{8.000000}{9.600000}\selectfont Ambient Temperature}%
\end{pgfscope}%
\end{pgfpicture}%
\makeatother%
\endgroup%

    \caption{Measurering a \qty{50}{\mA} current using a Keysight \device{34470A} with changing ambient temperature. The current source is based on the design of \citeauthor{laser_driver_digital}.}
    \label{fig:laser_driver_aircon}
\end{figure}

Admittedly, the example in figure \ref{fig:laser_driver_aircon} does not reflect the correct way to measure current with high stability, but serves as an excellent example to highlight the problem. The room temperature heavily depends on the inner workings of the air conditioning system used in the building and the problem is not always present. Nonetheless, longer measurements over several days necessitated the development of a PID controller module to replace the stock temperaure controller of the air conditioning unit in the lab. This project is described in section \ref{sec:lab_temp_control}.

In the meantime, measurements, that only required short-term stability were conducted. Measurering the current noise of the drivers is one such measurement.

\clearpage
\subsection{Test Results: Current Noise}
\label{sec:results_current_noise}
The spectral current noise density is a quantity, that is both seemingly trivial to measure and also easy to understand and graphically compare, therefore many devices are emblazoned by such graphs. The upside is, that these numbers can used for reference. Defining the bandwidth of such a measurement is a matter of debate, the available measurement devices and depends of the future use-case of the current driver. We chose an upper frequency of \qty{1}{\MHz} for two reason, first, to limit the number of amplifiers required. As the noise power rises with the bandwidth (in the best case as $\sqrt{\Delta f}$ for white noise) and impedance matching comes into play, a higher power amplifier is required, a trait that does not bode well with low noise, low frequency frontends. So for frequencies above a few \unit{\MHz} different amplifiers are called for. More amplifiers make the whole measurement more intricate, because the amplifiers are the most critical parts in the whole chain.

The second reason does not root in the lazyness of the researcher, but has a physical origin. Cables used in the lab like RG-58 or RG-223 have a capacitance of about \qty{100}{\pF \per \m}. With a cable length of around \qty{3}{\m} resulting in \qty{300}{\pF}, one finds, that at \qty{10}{\MHz} the impedance seen by the laser diode approaches \qty{50}{\ohm}, not unsurprising, given that the cable impedance is about \qty{50}{\ohm} and a signal at \qty{10}{\MHz} has a wavelength of $\lambda \approx \qty{2}{\m}$. This is aproaching the quarter-wave rule beyond which one should treat the cable as a transmission line. It is therefore reasonable to limit the noise measurement to \qty{1}{\MHz}, beyond which a design specific implementation including the laser head is called for anyway.

\clearpage
\subsection{Test Results: Stability}
\label{sec:results_stability}
When remotely controlling a laser system, stability of the laser driver is of immediate concern, because uninterrupted operation of the system is a key requirement and if the laser cannot be locked again remotely, it is time consuming, if possible at all, to go to the remote laser lab and readjust the laser current driver. The development of the past years have also shown a greater demand of remote working rendering readjustment unfeasable. To assure the specifications given in \ref{lst:dgDrive_specs_environment}, the current drivers were first tested for \qty{24}{\hour} to ensure. This test was refined several times over the course of this work, to reflect the need for a better signal so noise ratio. The first tests were done by feeding the output current of the laser driver into a (calibrated) Keysight \device{34470A} and measuring the output over \qty{24}{\hour}. The \device{34470A} was warmed up for \qty{8}{\hour} and so was the laser driver. The ambient temperature and humidity were recorded by the lab monitoring system descibed in section \ref{}.


\subsection{Zener Diode Selection}
\label{sec:zener_diode_selection}
Early tests of the LM399 Zener diode as a reference have confirmed, what the data sheet \cite{datasheet_LM399} already suggest in the 'Low Frequency Noise Voltage' plot. There are random bi-stable voltage step changes. This phenomenon is called burst noise or popcorn noise.

\begin{figure}[ht]
    \centering
    %% Creator: Matplotlib, PGF backend
%%
%% To include the figure in your LaTeX document, write
%%   \input{<filename>.pgf}
%%
%% Make sure the required packages are loaded in your preamble
%%   \usepackage{pgf}
%%
%% Also ensure that all the required font packages are loaded; for instance,
%% the lmodern package is sometimes necessary when using math font.
%%   \usepackage{lmodern}
%%
%% Figures using additional raster images can only be included by \input if
%% they are in the same directory as the main LaTeX file. For loading figures
%% from other directories you can use the `import` package
%%   \usepackage{import}
%%
%% and then include the figures with
%%   \import{<path to file>}{<filename>.pgf}
%%
%% Matplotlib used the following preamble
%%   \usepackage{siunitx}
%%   \sisetup{per-mode = symbol}%
%%   \usepackage{fontspec}
%%   \makeatletter\@ifpackageloaded{underscore}{}{\usepackage[strings]{underscore}}\makeatother
%%
\begingroup%
\makeatletter%
\begin{pgfpicture}%
\pgfpathrectangle{\pgfpointorigin}{\pgfqpoint{5.150788in}{3.183362in}}%
\pgfusepath{use as bounding box, clip}%
\begin{pgfscope}%
\pgfsetbuttcap%
\pgfsetmiterjoin%
\definecolor{currentfill}{rgb}{1.000000,1.000000,1.000000}%
\pgfsetfillcolor{currentfill}%
\pgfsetlinewidth{0.000000pt}%
\definecolor{currentstroke}{rgb}{1.000000,1.000000,1.000000}%
\pgfsetstrokecolor{currentstroke}%
\pgfsetdash{}{0pt}%
\pgfpathmoveto{\pgfqpoint{0.000000in}{0.000000in}}%
\pgfpathlineto{\pgfqpoint{5.150788in}{0.000000in}}%
\pgfpathlineto{\pgfqpoint{5.150788in}{3.183362in}}%
\pgfpathlineto{\pgfqpoint{0.000000in}{3.183362in}}%
\pgfpathlineto{\pgfqpoint{0.000000in}{0.000000in}}%
\pgfpathclose%
\pgfusepath{fill}%
\end{pgfscope}%
\begin{pgfscope}%
\pgfsetbuttcap%
\pgfsetmiterjoin%
\definecolor{currentfill}{rgb}{1.000000,1.000000,1.000000}%
\pgfsetfillcolor{currentfill}%
\pgfsetlinewidth{0.000000pt}%
\definecolor{currentstroke}{rgb}{0.000000,0.000000,0.000000}%
\pgfsetstrokecolor{currentstroke}%
\pgfsetstrokeopacity{0.000000}%
\pgfsetdash{}{0pt}%
\pgfpathmoveto{\pgfqpoint{0.484581in}{2.334497in}}%
\pgfpathlineto{\pgfqpoint{5.000788in}{2.334497in}}%
\pgfpathlineto{\pgfqpoint{5.000788in}{2.909119in}}%
\pgfpathlineto{\pgfqpoint{0.484581in}{2.909119in}}%
\pgfpathlineto{\pgfqpoint{0.484581in}{2.334497in}}%
\pgfpathclose%
\pgfusepath{fill}%
\end{pgfscope}%
\begin{pgfscope}%
\pgfsetbuttcap%
\pgfsetroundjoin%
\definecolor{currentfill}{rgb}{0.000000,0.000000,0.000000}%
\pgfsetfillcolor{currentfill}%
\pgfsetlinewidth{0.803000pt}%
\definecolor{currentstroke}{rgb}{0.000000,0.000000,0.000000}%
\pgfsetstrokecolor{currentstroke}%
\pgfsetdash{}{0pt}%
\pgfsys@defobject{currentmarker}{\pgfqpoint{0.000000in}{-0.048611in}}{\pgfqpoint{0.000000in}{0.000000in}}{%
\pgfpathmoveto{\pgfqpoint{0.000000in}{0.000000in}}%
\pgfpathlineto{\pgfqpoint{0.000000in}{-0.048611in}}%
\pgfusepath{stroke,fill}%
}%
\begin{pgfscope}%
\pgfsys@transformshift{0.689546in}{2.334497in}%
\pgfsys@useobject{currentmarker}{}%
\end{pgfscope}%
\end{pgfscope}%
\begin{pgfscope}%
\pgfsetbuttcap%
\pgfsetroundjoin%
\definecolor{currentfill}{rgb}{0.000000,0.000000,0.000000}%
\pgfsetfillcolor{currentfill}%
\pgfsetlinewidth{0.803000pt}%
\definecolor{currentstroke}{rgb}{0.000000,0.000000,0.000000}%
\pgfsetstrokecolor{currentstroke}%
\pgfsetdash{}{0pt}%
\pgfsys@defobject{currentmarker}{\pgfqpoint{0.000000in}{-0.048611in}}{\pgfqpoint{0.000000in}{0.000000in}}{%
\pgfpathmoveto{\pgfqpoint{0.000000in}{0.000000in}}%
\pgfpathlineto{\pgfqpoint{0.000000in}{-0.048611in}}%
\pgfusepath{stroke,fill}%
}%
\begin{pgfscope}%
\pgfsys@transformshift{1.202878in}{2.334497in}%
\pgfsys@useobject{currentmarker}{}%
\end{pgfscope}%
\end{pgfscope}%
\begin{pgfscope}%
\pgfsetbuttcap%
\pgfsetroundjoin%
\definecolor{currentfill}{rgb}{0.000000,0.000000,0.000000}%
\pgfsetfillcolor{currentfill}%
\pgfsetlinewidth{0.803000pt}%
\definecolor{currentstroke}{rgb}{0.000000,0.000000,0.000000}%
\pgfsetstrokecolor{currentstroke}%
\pgfsetdash{}{0pt}%
\pgfsys@defobject{currentmarker}{\pgfqpoint{0.000000in}{-0.048611in}}{\pgfqpoint{0.000000in}{0.000000in}}{%
\pgfpathmoveto{\pgfqpoint{0.000000in}{0.000000in}}%
\pgfpathlineto{\pgfqpoint{0.000000in}{-0.048611in}}%
\pgfusepath{stroke,fill}%
}%
\begin{pgfscope}%
\pgfsys@transformshift{1.716211in}{2.334497in}%
\pgfsys@useobject{currentmarker}{}%
\end{pgfscope}%
\end{pgfscope}%
\begin{pgfscope}%
\pgfsetbuttcap%
\pgfsetroundjoin%
\definecolor{currentfill}{rgb}{0.000000,0.000000,0.000000}%
\pgfsetfillcolor{currentfill}%
\pgfsetlinewidth{0.803000pt}%
\definecolor{currentstroke}{rgb}{0.000000,0.000000,0.000000}%
\pgfsetstrokecolor{currentstroke}%
\pgfsetdash{}{0pt}%
\pgfsys@defobject{currentmarker}{\pgfqpoint{0.000000in}{-0.048611in}}{\pgfqpoint{0.000000in}{0.000000in}}{%
\pgfpathmoveto{\pgfqpoint{0.000000in}{0.000000in}}%
\pgfpathlineto{\pgfqpoint{0.000000in}{-0.048611in}}%
\pgfusepath{stroke,fill}%
}%
\begin{pgfscope}%
\pgfsys@transformshift{2.229543in}{2.334497in}%
\pgfsys@useobject{currentmarker}{}%
\end{pgfscope}%
\end{pgfscope}%
\begin{pgfscope}%
\pgfsetbuttcap%
\pgfsetroundjoin%
\definecolor{currentfill}{rgb}{0.000000,0.000000,0.000000}%
\pgfsetfillcolor{currentfill}%
\pgfsetlinewidth{0.803000pt}%
\definecolor{currentstroke}{rgb}{0.000000,0.000000,0.000000}%
\pgfsetstrokecolor{currentstroke}%
\pgfsetdash{}{0pt}%
\pgfsys@defobject{currentmarker}{\pgfqpoint{0.000000in}{-0.048611in}}{\pgfqpoint{0.000000in}{0.000000in}}{%
\pgfpathmoveto{\pgfqpoint{0.000000in}{0.000000in}}%
\pgfpathlineto{\pgfqpoint{0.000000in}{-0.048611in}}%
\pgfusepath{stroke,fill}%
}%
\begin{pgfscope}%
\pgfsys@transformshift{2.742876in}{2.334497in}%
\pgfsys@useobject{currentmarker}{}%
\end{pgfscope}%
\end{pgfscope}%
\begin{pgfscope}%
\pgfsetbuttcap%
\pgfsetroundjoin%
\definecolor{currentfill}{rgb}{0.000000,0.000000,0.000000}%
\pgfsetfillcolor{currentfill}%
\pgfsetlinewidth{0.803000pt}%
\definecolor{currentstroke}{rgb}{0.000000,0.000000,0.000000}%
\pgfsetstrokecolor{currentstroke}%
\pgfsetdash{}{0pt}%
\pgfsys@defobject{currentmarker}{\pgfqpoint{0.000000in}{-0.048611in}}{\pgfqpoint{0.000000in}{0.000000in}}{%
\pgfpathmoveto{\pgfqpoint{0.000000in}{0.000000in}}%
\pgfpathlineto{\pgfqpoint{0.000000in}{-0.048611in}}%
\pgfusepath{stroke,fill}%
}%
\begin{pgfscope}%
\pgfsys@transformshift{3.256208in}{2.334497in}%
\pgfsys@useobject{currentmarker}{}%
\end{pgfscope}%
\end{pgfscope}%
\begin{pgfscope}%
\pgfsetbuttcap%
\pgfsetroundjoin%
\definecolor{currentfill}{rgb}{0.000000,0.000000,0.000000}%
\pgfsetfillcolor{currentfill}%
\pgfsetlinewidth{0.803000pt}%
\definecolor{currentstroke}{rgb}{0.000000,0.000000,0.000000}%
\pgfsetstrokecolor{currentstroke}%
\pgfsetdash{}{0pt}%
\pgfsys@defobject{currentmarker}{\pgfqpoint{0.000000in}{-0.048611in}}{\pgfqpoint{0.000000in}{0.000000in}}{%
\pgfpathmoveto{\pgfqpoint{0.000000in}{0.000000in}}%
\pgfpathlineto{\pgfqpoint{0.000000in}{-0.048611in}}%
\pgfusepath{stroke,fill}%
}%
\begin{pgfscope}%
\pgfsys@transformshift{3.769541in}{2.334497in}%
\pgfsys@useobject{currentmarker}{}%
\end{pgfscope}%
\end{pgfscope}%
\begin{pgfscope}%
\pgfsetbuttcap%
\pgfsetroundjoin%
\definecolor{currentfill}{rgb}{0.000000,0.000000,0.000000}%
\pgfsetfillcolor{currentfill}%
\pgfsetlinewidth{0.803000pt}%
\definecolor{currentstroke}{rgb}{0.000000,0.000000,0.000000}%
\pgfsetstrokecolor{currentstroke}%
\pgfsetdash{}{0pt}%
\pgfsys@defobject{currentmarker}{\pgfqpoint{0.000000in}{-0.048611in}}{\pgfqpoint{0.000000in}{0.000000in}}{%
\pgfpathmoveto{\pgfqpoint{0.000000in}{0.000000in}}%
\pgfpathlineto{\pgfqpoint{0.000000in}{-0.048611in}}%
\pgfusepath{stroke,fill}%
}%
\begin{pgfscope}%
\pgfsys@transformshift{4.282873in}{2.334497in}%
\pgfsys@useobject{currentmarker}{}%
\end{pgfscope}%
\end{pgfscope}%
\begin{pgfscope}%
\pgfsetbuttcap%
\pgfsetroundjoin%
\definecolor{currentfill}{rgb}{0.000000,0.000000,0.000000}%
\pgfsetfillcolor{currentfill}%
\pgfsetlinewidth{0.803000pt}%
\definecolor{currentstroke}{rgb}{0.000000,0.000000,0.000000}%
\pgfsetstrokecolor{currentstroke}%
\pgfsetdash{}{0pt}%
\pgfsys@defobject{currentmarker}{\pgfqpoint{0.000000in}{-0.048611in}}{\pgfqpoint{0.000000in}{0.000000in}}{%
\pgfpathmoveto{\pgfqpoint{0.000000in}{0.000000in}}%
\pgfpathlineto{\pgfqpoint{0.000000in}{-0.048611in}}%
\pgfusepath{stroke,fill}%
}%
\begin{pgfscope}%
\pgfsys@transformshift{4.796206in}{2.334497in}%
\pgfsys@useobject{currentmarker}{}%
\end{pgfscope}%
\end{pgfscope}%
\begin{pgfscope}%
\pgfsetbuttcap%
\pgfsetroundjoin%
\definecolor{currentfill}{rgb}{0.000000,0.000000,0.000000}%
\pgfsetfillcolor{currentfill}%
\pgfsetlinewidth{0.803000pt}%
\definecolor{currentstroke}{rgb}{0.000000,0.000000,0.000000}%
\pgfsetstrokecolor{currentstroke}%
\pgfsetdash{}{0pt}%
\pgfsys@defobject{currentmarker}{\pgfqpoint{-0.048611in}{0.000000in}}{\pgfqpoint{-0.000000in}{0.000000in}}{%
\pgfpathmoveto{\pgfqpoint{-0.000000in}{0.000000in}}%
\pgfpathlineto{\pgfqpoint{-0.048611in}{0.000000in}}%
\pgfusepath{stroke,fill}%
}%
\begin{pgfscope}%
\pgfsys@transformshift{0.484581in}{2.512496in}%
\pgfsys@useobject{currentmarker}{}%
\end{pgfscope}%
\end{pgfscope}%
\begin{pgfscope}%
\definecolor{textcolor}{rgb}{0.000000,0.000000,0.000000}%
\pgfsetstrokecolor{textcolor}%
\pgfsetfillcolor{textcolor}%
\pgftext[x=0.328331in, y=2.473941in, left, base]{\color{textcolor}\rmfamily\fontsize{8.000000}{9.600000}\selectfont \(\displaystyle {0}\)}%
\end{pgfscope}%
\begin{pgfscope}%
\pgfsetbuttcap%
\pgfsetroundjoin%
\definecolor{currentfill}{rgb}{0.000000,0.000000,0.000000}%
\pgfsetfillcolor{currentfill}%
\pgfsetlinewidth{0.803000pt}%
\definecolor{currentstroke}{rgb}{0.000000,0.000000,0.000000}%
\pgfsetstrokecolor{currentstroke}%
\pgfsetdash{}{0pt}%
\pgfsys@defobject{currentmarker}{\pgfqpoint{-0.048611in}{0.000000in}}{\pgfqpoint{-0.000000in}{0.000000in}}{%
\pgfpathmoveto{\pgfqpoint{-0.000000in}{0.000000in}}%
\pgfpathlineto{\pgfqpoint{-0.048611in}{0.000000in}}%
\pgfusepath{stroke,fill}%
}%
\begin{pgfscope}%
\pgfsys@transformshift{0.484581in}{2.716037in}%
\pgfsys@useobject{currentmarker}{}%
\end{pgfscope}%
\end{pgfscope}%
\begin{pgfscope}%
\definecolor{textcolor}{rgb}{0.000000,0.000000,0.000000}%
\pgfsetstrokecolor{textcolor}%
\pgfsetfillcolor{textcolor}%
\pgftext[x=0.328331in, y=2.677482in, left, base]{\color{textcolor}\rmfamily\fontsize{8.000000}{9.600000}\selectfont \(\displaystyle {5}\)}%
\end{pgfscope}%
\begin{pgfscope}%
\definecolor{textcolor}{rgb}{0.000000,0.000000,0.000000}%
\pgfsetstrokecolor{textcolor}%
\pgfsetfillcolor{textcolor}%
\pgftext[x=0.484581in,y=2.950785in,left,base]{\color{textcolor}\rmfamily\fontsize{8.000000}{9.600000}\selectfont \(\displaystyle \times{10^{\ensuremath{-}6}}{}\)}%
\end{pgfscope}%
\begin{pgfscope}%
\pgfpathrectangle{\pgfqpoint{0.484581in}{2.334497in}}{\pgfqpoint{4.516206in}{0.574622in}}%
\pgfusepath{clip}%
\pgfsetrectcap%
\pgfsetroundjoin%
\pgfsetlinewidth{0.501875pt}%
\definecolor{currentstroke}{rgb}{0.003922,0.450980,0.698039}%
\pgfsetstrokecolor{currentstroke}%
\pgfsetstrokeopacity{0.700000}%
\pgfsetdash{}{0pt}%
\pgfpathmoveto{\pgfqpoint{0.689863in}{2.509930in}}%
\pgfpathlineto{\pgfqpoint{0.691573in}{2.569128in}}%
\pgfpathlineto{\pgfqpoint{0.694995in}{2.478459in}}%
\pgfpathlineto{\pgfqpoint{0.699275in}{2.580305in}}%
\pgfpathlineto{\pgfqpoint{0.703553in}{2.518208in}}%
\pgfpathlineto{\pgfqpoint{0.710398in}{2.564054in}}%
\pgfpathlineto{\pgfqpoint{0.712966in}{2.460398in}}%
\pgfpathlineto{\pgfqpoint{0.718959in}{2.382413in}}%
\pgfpathlineto{\pgfqpoint{0.722378in}{2.526844in}}%
\pgfpathlineto{\pgfqpoint{0.728363in}{2.563512in}}%
\pgfpathlineto{\pgfqpoint{0.730929in}{2.446928in}}%
\pgfpathlineto{\pgfqpoint{0.734351in}{2.508420in}}%
\pgfpathlineto{\pgfqpoint{0.740340in}{2.462634in}}%
\pgfpathlineto{\pgfqpoint{0.742908in}{2.509569in}}%
\pgfpathlineto{\pgfqpoint{0.748037in}{2.531618in}}%
\pgfpathlineto{\pgfqpoint{0.752314in}{2.422041in}}%
\pgfpathlineto{\pgfqpoint{0.755738in}{2.514704in}}%
\pgfpathlineto{\pgfqpoint{0.760009in}{2.489757in}}%
\pgfpathlineto{\pgfqpoint{0.764288in}{2.546054in}}%
\pgfpathlineto{\pgfqpoint{0.769423in}{2.581573in}}%
\pgfpathlineto{\pgfqpoint{0.771990in}{2.483110in}}%
\pgfpathlineto{\pgfqpoint{0.777124in}{2.481781in}}%
\pgfpathlineto{\pgfqpoint{0.780547in}{2.572209in}}%
\pgfpathlineto{\pgfqpoint{0.786534in}{2.392199in}}%
\pgfpathlineto{\pgfqpoint{0.789953in}{2.555597in}}%
\pgfpathlineto{\pgfqpoint{0.795939in}{2.458526in}}%
\pgfpathlineto{\pgfqpoint{0.799360in}{2.547927in}}%
\pgfpathlineto{\pgfqpoint{0.803633in}{2.421859in}}%
\pgfpathlineto{\pgfqpoint{0.806199in}{2.515549in}}%
\pgfpathlineto{\pgfqpoint{0.810475in}{2.476526in}}%
\pgfpathlineto{\pgfqpoint{0.814751in}{2.453994in}}%
\pgfpathlineto{\pgfqpoint{0.820735in}{2.649770in}}%
\pgfpathlineto{\pgfqpoint{0.823303in}{2.515246in}}%
\pgfpathlineto{\pgfqpoint{0.828440in}{2.617574in}}%
\pgfpathlineto{\pgfqpoint{0.832717in}{2.490117in}}%
\pgfpathlineto{\pgfqpoint{0.839565in}{2.452968in}}%
\pgfpathlineto{\pgfqpoint{0.842131in}{2.529864in}}%
\pgfpathlineto{\pgfqpoint{0.848114in}{2.449857in}}%
\pgfpathlineto{\pgfqpoint{0.852384in}{2.551822in}}%
\pgfpathlineto{\pgfqpoint{0.856662in}{2.459794in}}%
\pgfpathlineto{\pgfqpoint{0.859233in}{2.558859in}}%
\pgfpathlineto{\pgfqpoint{0.862659in}{2.464323in}}%
\pgfpathlineto{\pgfqpoint{0.868650in}{2.535180in}}%
\pgfpathlineto{\pgfqpoint{0.872926in}{2.477250in}}%
\pgfpathlineto{\pgfqpoint{0.876348in}{2.467797in}}%
\pgfpathlineto{\pgfqpoint{0.880627in}{2.539679in}}%
\pgfpathlineto{\pgfqpoint{0.884050in}{2.404099in}}%
\pgfpathlineto{\pgfqpoint{0.887469in}{2.536629in}}%
\pgfpathlineto{\pgfqpoint{0.895167in}{2.454266in}}%
\pgfpathlineto{\pgfqpoint{0.896877in}{2.573658in}}%
\pgfpathlineto{\pgfqpoint{0.902009in}{2.524850in}}%
\pgfpathlineto{\pgfqpoint{0.906287in}{2.809305in}}%
\pgfpathlineto{\pgfqpoint{0.908853in}{2.673270in}}%
\pgfpathlineto{\pgfqpoint{0.915695in}{2.794808in}}%
\pgfpathlineto{\pgfqpoint{0.919115in}{2.504372in}}%
\pgfpathlineto{\pgfqpoint{0.921681in}{2.466317in}}%
\pgfpathlineto{\pgfqpoint{0.925951in}{2.535422in}}%
\pgfpathlineto{\pgfqpoint{0.935359in}{2.435328in}}%
\pgfpathlineto{\pgfqpoint{0.939638in}{2.544754in}}%
\pgfpathlineto{\pgfqpoint{0.943058in}{2.464444in}}%
\pgfpathlineto{\pgfqpoint{0.947330in}{2.450187in}}%
\pgfpathlineto{\pgfqpoint{0.955029in}{2.439316in}}%
\pgfpathlineto{\pgfqpoint{0.957596in}{2.572632in}}%
\pgfpathlineto{\pgfqpoint{0.961015in}{2.449373in}}%
\pgfpathlineto{\pgfqpoint{0.965292in}{2.526783in}}%
\pgfpathlineto{\pgfqpoint{0.971277in}{2.574021in}}%
\pgfpathlineto{\pgfqpoint{0.972986in}{2.461303in}}%
\pgfpathlineto{\pgfqpoint{0.978119in}{2.553906in}}%
\pgfpathlineto{\pgfqpoint{0.983246in}{2.411046in}}%
\pgfpathlineto{\pgfqpoint{0.985812in}{2.492956in}}%
\pgfpathlineto{\pgfqpoint{0.990941in}{2.449706in}}%
\pgfpathlineto{\pgfqpoint{0.996074in}{2.539710in}}%
\pgfpathlineto{\pgfqpoint{1.001204in}{2.486674in}}%
\pgfpathlineto{\pgfqpoint{1.004625in}{2.536085in}}%
\pgfpathlineto{\pgfqpoint{1.007191in}{2.418051in}}%
\pgfpathlineto{\pgfqpoint{1.013177in}{2.564296in}}%
\pgfpathlineto{\pgfqpoint{1.016601in}{2.472720in}}%
\pgfpathlineto{\pgfqpoint{1.023437in}{2.595949in}}%
\pgfpathlineto{\pgfqpoint{1.024292in}{2.493530in}}%
\pgfpathlineto{\pgfqpoint{1.030277in}{2.444511in}}%
\pgfpathlineto{\pgfqpoint{1.035409in}{2.533065in}}%
\pgfpathlineto{\pgfqpoint{1.039687in}{2.573356in}}%
\pgfpathlineto{\pgfqpoint{1.046538in}{2.426389in}}%
\pgfpathlineto{\pgfqpoint{1.051672in}{2.516514in}}%
\pgfpathlineto{\pgfqpoint{1.055092in}{2.427506in}}%
\pgfpathlineto{\pgfqpoint{1.059370in}{2.535843in}}%
\pgfpathlineto{\pgfqpoint{1.065358in}{2.549193in}}%
\pgfpathlineto{\pgfqpoint{1.067066in}{2.434965in}}%
\pgfpathlineto{\pgfqpoint{1.073048in}{2.538502in}}%
\pgfpathlineto{\pgfqpoint{1.076466in}{2.455443in}}%
\pgfpathlineto{\pgfqpoint{1.082458in}{2.574805in}}%
\pgfpathlineto{\pgfqpoint{1.087592in}{2.433697in}}%
\pgfpathlineto{\pgfqpoint{1.088447in}{2.497062in}}%
\pgfpathlineto{\pgfqpoint{1.093579in}{2.563873in}}%
\pgfpathlineto{\pgfqpoint{1.097851in}{2.467585in}}%
\pgfpathlineto{\pgfqpoint{1.103832in}{2.552878in}}%
\pgfpathlineto{\pgfqpoint{1.107250in}{2.428744in}}%
\pgfpathlineto{\pgfqpoint{1.112379in}{2.554692in}}%
\pgfpathlineto{\pgfqpoint{1.114088in}{2.448135in}}%
\pgfpathlineto{\pgfqpoint{1.120074in}{2.527086in}}%
\pgfpathlineto{\pgfqpoint{1.123495in}{2.440825in}}%
\pgfpathlineto{\pgfqpoint{1.126917in}{2.517964in}}%
\pgfpathlineto{\pgfqpoint{1.131195in}{2.419200in}}%
\pgfpathlineto{\pgfqpoint{1.136328in}{2.548529in}}%
\pgfpathlineto{\pgfqpoint{1.139749in}{2.450399in}}%
\pgfpathlineto{\pgfqpoint{1.145737in}{2.538139in}}%
\pgfpathlineto{\pgfqpoint{1.148303in}{2.449222in}}%
\pgfpathlineto{\pgfqpoint{1.155144in}{2.590209in}}%
\pgfpathlineto{\pgfqpoint{1.158563in}{2.477975in}}%
\pgfpathlineto{\pgfqpoint{1.161130in}{2.562000in}}%
\pgfpathlineto{\pgfqpoint{1.166264in}{2.553422in}}%
\pgfpathlineto{\pgfqpoint{1.172254in}{2.422371in}}%
\pgfpathlineto{\pgfqpoint{1.174823in}{2.567558in}}%
\pgfpathlineto{\pgfqpoint{1.179102in}{2.466620in}}%
\pgfpathlineto{\pgfqpoint{1.183380in}{2.571725in}}%
\pgfpathlineto{\pgfqpoint{1.189368in}{2.453088in}}%
\pgfpathlineto{\pgfqpoint{1.191079in}{2.535966in}}%
\pgfpathlineto{\pgfqpoint{1.198776in}{2.443908in}}%
\pgfpathlineto{\pgfqpoint{1.201344in}{2.534033in}}%
\pgfpathlineto{\pgfqpoint{1.203910in}{2.473566in}}%
\pgfpathlineto{\pgfqpoint{1.209045in}{2.570638in}}%
\pgfpathlineto{\pgfqpoint{1.213329in}{2.482868in}}%
\pgfpathlineto{\pgfqpoint{1.216755in}{2.540496in}}%
\pgfpathlineto{\pgfqpoint{1.222746in}{2.460459in}}%
\pgfpathlineto{\pgfqpoint{1.227880in}{2.440946in}}%
\pgfpathlineto{\pgfqpoint{1.229590in}{2.551912in}}%
\pgfpathlineto{\pgfqpoint{1.235580in}{2.469518in}}%
\pgfpathlineto{\pgfqpoint{1.239001in}{2.532402in}}%
\pgfpathlineto{\pgfqpoint{1.243278in}{2.475649in}}%
\pgfpathlineto{\pgfqpoint{1.248413in}{2.554027in}}%
\pgfpathlineto{\pgfqpoint{1.251837in}{2.455625in}}%
\pgfpathlineto{\pgfqpoint{1.257825in}{2.410199in}}%
\pgfpathlineto{\pgfqpoint{1.259533in}{2.581843in}}%
\pgfpathlineto{\pgfqpoint{1.266376in}{2.476586in}}%
\pgfpathlineto{\pgfqpoint{1.270656in}{2.598243in}}%
\pgfpathlineto{\pgfqpoint{1.274080in}{2.472478in}}%
\pgfpathlineto{\pgfqpoint{1.277502in}{2.571483in}}%
\pgfpathlineto{\pgfqpoint{1.280922in}{2.518175in}}%
\pgfpathlineto{\pgfqpoint{1.285201in}{2.558617in}}%
\pgfpathlineto{\pgfqpoint{1.290336in}{2.372416in}}%
\pgfpathlineto{\pgfqpoint{1.295468in}{2.494345in}}%
\pgfpathlineto{\pgfqpoint{1.299746in}{2.446111in}}%
\pgfpathlineto{\pgfqpoint{1.302314in}{2.523401in}}%
\pgfpathlineto{\pgfqpoint{1.310013in}{2.405790in}}%
\pgfpathlineto{\pgfqpoint{1.310869in}{2.502923in}}%
\pgfpathlineto{\pgfqpoint{1.318566in}{2.440190in}}%
\pgfpathlineto{\pgfqpoint{1.321131in}{2.519111in}}%
\pgfpathlineto{\pgfqpoint{1.325401in}{2.460457in}}%
\pgfpathlineto{\pgfqpoint{1.329678in}{2.563208in}}%
\pgfpathlineto{\pgfqpoint{1.333096in}{2.477612in}}%
\pgfpathlineto{\pgfqpoint{1.339081in}{2.527328in}}%
\pgfpathlineto{\pgfqpoint{1.341646in}{2.469458in}}%
\pgfpathlineto{\pgfqpoint{1.345066in}{2.432187in}}%
\pgfpathlineto{\pgfqpoint{1.351054in}{2.429470in}}%
\pgfpathlineto{\pgfqpoint{1.354477in}{2.550160in}}%
\pgfpathlineto{\pgfqpoint{1.362176in}{2.418898in}}%
\pgfpathlineto{\pgfqpoint{1.367303in}{2.491023in}}%
\pgfpathlineto{\pgfqpoint{1.370723in}{2.437685in}}%
\pgfpathlineto{\pgfqpoint{1.375853in}{2.521770in}}%
\pgfpathlineto{\pgfqpoint{1.381839in}{2.395158in}}%
\pgfpathlineto{\pgfqpoint{1.386113in}{2.528475in}}%
\pgfpathlineto{\pgfqpoint{1.391249in}{2.465501in}}%
\pgfpathlineto{\pgfqpoint{1.394673in}{2.531616in}}%
\pgfpathlineto{\pgfqpoint{1.398953in}{2.476223in}}%
\pgfpathlineto{\pgfqpoint{1.401521in}{2.563510in}}%
\pgfpathlineto{\pgfqpoint{1.408362in}{2.571183in}}%
\pgfpathlineto{\pgfqpoint{1.410925in}{2.466015in}}%
\pgfpathlineto{\pgfqpoint{1.418616in}{2.551791in}}%
\pgfpathlineto{\pgfqpoint{1.425464in}{2.426932in}}%
\pgfpathlineto{\pgfqpoint{1.427175in}{2.512587in}}%
\pgfpathlineto{\pgfqpoint{1.431452in}{2.541341in}}%
\pgfpathlineto{\pgfqpoint{1.434872in}{2.452726in}}%
\pgfpathlineto{\pgfqpoint{1.440002in}{2.437987in}}%
\pgfpathlineto{\pgfqpoint{1.444278in}{2.504856in}}%
\pgfpathlineto{\pgfqpoint{1.451122in}{2.436235in}}%
\pgfpathlineto{\pgfqpoint{1.453691in}{2.535906in}}%
\pgfpathlineto{\pgfqpoint{1.457970in}{2.449585in}}%
\pgfpathlineto{\pgfqpoint{1.463106in}{2.437203in}}%
\pgfpathlineto{\pgfqpoint{1.468244in}{2.536087in}}%
\pgfpathlineto{\pgfqpoint{1.470810in}{2.559283in}}%
\pgfpathlineto{\pgfqpoint{1.474232in}{2.453088in}}%
\pgfpathlineto{\pgfqpoint{1.480224in}{2.514581in}}%
\pgfpathlineto{\pgfqpoint{1.485359in}{2.441581in}}%
\pgfpathlineto{\pgfqpoint{1.486215in}{2.557773in}}%
\pgfpathlineto{\pgfqpoint{1.491345in}{2.500629in}}%
\pgfpathlineto{\pgfqpoint{1.496476in}{2.468432in}}%
\pgfpathlineto{\pgfqpoint{1.499039in}{2.574868in}}%
\pgfpathlineto{\pgfqpoint{1.505027in}{2.459372in}}%
\pgfpathlineto{\pgfqpoint{1.511013in}{2.539861in}}%
\pgfpathlineto{\pgfqpoint{1.512722in}{2.446262in}}%
\pgfpathlineto{\pgfqpoint{1.518707in}{2.430738in}}%
\pgfpathlineto{\pgfqpoint{1.520417in}{2.530257in}}%
\pgfpathlineto{\pgfqpoint{1.528112in}{2.462995in}}%
\pgfpathlineto{\pgfqpoint{1.531534in}{2.538200in}}%
\pgfpathlineto{\pgfqpoint{1.535810in}{2.445597in}}%
\pgfpathlineto{\pgfqpoint{1.540941in}{2.558859in}}%
\pgfpathlineto{\pgfqpoint{1.541797in}{2.461515in}}%
\pgfpathlineto{\pgfqpoint{1.548638in}{2.506668in}}%
\pgfpathlineto{\pgfqpoint{1.551207in}{2.432580in}}%
\pgfpathlineto{\pgfqpoint{1.556343in}{2.434120in}}%
\pgfpathlineto{\pgfqpoint{1.558909in}{2.535843in}}%
\pgfpathlineto{\pgfqpoint{1.566605in}{2.550281in}}%
\pgfpathlineto{\pgfqpoint{1.567459in}{2.488124in}}%
\pgfpathlineto{\pgfqpoint{1.575167in}{2.437745in}}%
\pgfpathlineto{\pgfqpoint{1.577737in}{2.556986in}}%
\pgfpathlineto{\pgfqpoint{1.582014in}{2.419079in}}%
\pgfpathlineto{\pgfqpoint{1.585439in}{2.525394in}}%
\pgfpathlineto{\pgfqpoint{1.591423in}{2.470817in}}%
\pgfpathlineto{\pgfqpoint{1.596559in}{2.534757in}}%
\pgfpathlineto{\pgfqpoint{1.599126in}{2.428744in}}%
\pgfpathlineto{\pgfqpoint{1.603404in}{2.571241in}}%
\pgfpathlineto{\pgfqpoint{1.607680in}{2.461936in}}%
\pgfpathlineto{\pgfqpoint{1.613666in}{2.533428in}}%
\pgfpathlineto{\pgfqpoint{1.617087in}{2.470786in}}%
\pgfpathlineto{\pgfqpoint{1.618798in}{2.531979in}}%
\pgfpathlineto{\pgfqpoint{1.625639in}{2.579700in}}%
\pgfpathlineto{\pgfqpoint{1.630773in}{2.459914in}}%
\pgfpathlineto{\pgfqpoint{1.635054in}{2.558436in}}%
\pgfpathlineto{\pgfqpoint{1.636766in}{2.448196in}}%
\pgfpathlineto{\pgfqpoint{1.641889in}{2.568282in}}%
\pgfpathlineto{\pgfqpoint{1.648727in}{2.450671in}}%
\pgfpathlineto{\pgfqpoint{1.656428in}{2.523219in}}%
\pgfpathlineto{\pgfqpoint{1.658993in}{2.424395in}}%
\pgfpathlineto{\pgfqpoint{1.664122in}{2.570336in}}%
\pgfpathlineto{\pgfqpoint{1.666687in}{2.455745in}}%
\pgfpathlineto{\pgfqpoint{1.672676in}{2.585740in}}%
\pgfpathlineto{\pgfqpoint{1.676098in}{2.398148in}}%
\pgfpathlineto{\pgfqpoint{1.681229in}{2.527810in}}%
\pgfpathlineto{\pgfqpoint{1.682939in}{2.447319in}}%
\pgfpathlineto{\pgfqpoint{1.687214in}{2.485225in}}%
\pgfpathlineto{\pgfqpoint{1.693199in}{2.430314in}}%
\pgfpathlineto{\pgfqpoint{1.695764in}{2.524487in}}%
\pgfpathlineto{\pgfqpoint{1.700041in}{2.445416in}}%
\pgfpathlineto{\pgfqpoint{1.704314in}{2.428774in}}%
\pgfpathlineto{\pgfqpoint{1.709447in}{2.508904in}}%
\pgfpathlineto{\pgfqpoint{1.714577in}{2.476949in}}%
\pgfpathlineto{\pgfqpoint{1.723132in}{2.594590in}}%
\pgfpathlineto{\pgfqpoint{1.725698in}{2.493863in}}%
\pgfpathlineto{\pgfqpoint{1.730826in}{2.435751in}}%
\pgfpathlineto{\pgfqpoint{1.734249in}{2.521407in}}%
\pgfpathlineto{\pgfqpoint{1.738528in}{2.478157in}}%
\pgfpathlineto{\pgfqpoint{1.746231in}{2.442184in}}%
\pgfpathlineto{\pgfqpoint{1.748801in}{2.550281in}}%
\pgfpathlineto{\pgfqpoint{1.753081in}{2.430435in}}%
\pgfpathlineto{\pgfqpoint{1.758211in}{2.542611in}}%
\pgfpathlineto{\pgfqpoint{1.761630in}{2.501352in}}%
\pgfpathlineto{\pgfqpoint{1.764196in}{2.531616in}}%
\pgfpathlineto{\pgfqpoint{1.770178in}{2.457104in}}%
\pgfpathlineto{\pgfqpoint{1.775307in}{2.529864in}}%
\pgfpathlineto{\pgfqpoint{1.780436in}{2.537234in}}%
\pgfpathlineto{\pgfqpoint{1.781291in}{2.419442in}}%
\pgfpathlineto{\pgfqpoint{1.787279in}{2.538139in}}%
\pgfpathlineto{\pgfqpoint{1.790704in}{2.489392in}}%
\pgfpathlineto{\pgfqpoint{1.794127in}{2.572088in}}%
\pgfpathlineto{\pgfqpoint{1.803538in}{2.429772in}}%
\pgfpathlineto{\pgfqpoint{1.808671in}{2.549013in}}%
\pgfpathlineto{\pgfqpoint{1.811238in}{2.482929in}}%
\pgfpathlineto{\pgfqpoint{1.817226in}{2.577283in}}%
\pgfpathlineto{\pgfqpoint{1.821502in}{2.465231in}}%
\pgfpathlineto{\pgfqpoint{1.824922in}{2.549800in}}%
\pgfpathlineto{\pgfqpoint{1.829199in}{2.450069in}}%
\pgfpathlineto{\pgfqpoint{1.835185in}{2.534093in}}%
\pgfpathlineto{\pgfqpoint{1.838605in}{2.457981in}}%
\pgfpathlineto{\pgfqpoint{1.842028in}{2.547745in}}%
\pgfpathlineto{\pgfqpoint{1.847161in}{2.469337in}}%
\pgfpathlineto{\pgfqpoint{1.849731in}{2.552033in}}%
\pgfpathlineto{\pgfqpoint{1.854859in}{2.546264in}}%
\pgfpathlineto{\pgfqpoint{1.861702in}{2.393316in}}%
\pgfpathlineto{\pgfqpoint{1.865977in}{2.534877in}}%
\pgfpathlineto{\pgfqpoint{1.869400in}{2.460578in}}%
\pgfpathlineto{\pgfqpoint{1.871963in}{2.585740in}}%
\pgfpathlineto{\pgfqpoint{1.875380in}{2.608635in}}%
\pgfpathlineto{\pgfqpoint{1.880514in}{2.452060in}}%
\pgfpathlineto{\pgfqpoint{1.886505in}{2.548832in}}%
\pgfpathlineto{\pgfqpoint{1.888217in}{2.450550in}}%
\pgfpathlineto{\pgfqpoint{1.893350in}{2.566409in}}%
\pgfpathlineto{\pgfqpoint{1.896771in}{2.475861in}}%
\pgfpathlineto{\pgfqpoint{1.903618in}{2.538200in}}%
\pgfpathlineto{\pgfqpoint{1.906183in}{2.459310in}}%
\pgfpathlineto{\pgfqpoint{1.909606in}{2.515728in}}%
\pgfpathlineto{\pgfqpoint{1.916449in}{2.575047in}}%
\pgfpathlineto{\pgfqpoint{1.921587in}{2.445960in}}%
\pgfpathlineto{\pgfqpoint{1.922443in}{2.555053in}}%
\pgfpathlineto{\pgfqpoint{1.927577in}{2.456653in}}%
\pgfpathlineto{\pgfqpoint{1.933561in}{2.495796in}}%
\pgfpathlineto{\pgfqpoint{1.937837in}{2.449887in}}%
\pgfpathlineto{\pgfqpoint{1.941258in}{2.577767in}}%
\pgfpathlineto{\pgfqpoint{1.947248in}{2.464747in}}%
\pgfpathlineto{\pgfqpoint{1.949816in}{2.543274in}}%
\pgfpathlineto{\pgfqpoint{1.955805in}{2.523280in}}%
\pgfpathlineto{\pgfqpoint{1.958374in}{2.408056in}}%
\pgfpathlineto{\pgfqpoint{1.960940in}{2.481781in}}%
\pgfpathlineto{\pgfqpoint{1.965220in}{2.559222in}}%
\pgfpathlineto{\pgfqpoint{1.972063in}{2.459975in}}%
\pgfpathlineto{\pgfqpoint{1.974626in}{2.559131in}}%
\pgfpathlineto{\pgfqpoint{1.980610in}{2.460459in}}%
\pgfpathlineto{\pgfqpoint{1.982321in}{2.542308in}}%
\pgfpathlineto{\pgfqpoint{1.986595in}{2.468644in}}%
\pgfpathlineto{\pgfqpoint{1.993441in}{2.555446in}}%
\pgfpathlineto{\pgfqpoint{1.995151in}{2.559041in}}%
\pgfpathlineto{\pgfqpoint{1.999427in}{2.460519in}}%
\pgfpathlineto{\pgfqpoint{2.003702in}{2.524308in}}%
\pgfpathlineto{\pgfqpoint{2.009691in}{2.444511in}}%
\pgfpathlineto{\pgfqpoint{2.013969in}{2.546054in}}%
\pgfpathlineto{\pgfqpoint{2.019956in}{2.592324in}}%
\pgfpathlineto{\pgfqpoint{2.024234in}{2.460699in}}%
\pgfpathlineto{\pgfqpoint{2.025946in}{2.528475in}}%
\pgfpathlineto{\pgfqpoint{2.031077in}{2.481086in}}%
\pgfpathlineto{\pgfqpoint{2.033644in}{2.576376in}}%
\pgfpathlineto{\pgfqpoint{2.039628in}{2.482929in}}%
\pgfpathlineto{\pgfqpoint{2.043906in}{2.568887in}}%
\pgfpathlineto{\pgfqpoint{2.048183in}{2.476647in}}%
\pgfpathlineto{\pgfqpoint{2.050751in}{2.582660in}}%
\pgfpathlineto{\pgfqpoint{2.055027in}{2.469579in}}%
\pgfpathlineto{\pgfqpoint{2.060159in}{2.570941in}}%
\pgfpathlineto{\pgfqpoint{2.064434in}{2.476223in}}%
\pgfpathlineto{\pgfqpoint{2.068710in}{2.417781in}}%
\pgfpathlineto{\pgfqpoint{2.072131in}{2.524066in}}%
\pgfpathlineto{\pgfqpoint{2.076406in}{2.474232in}}%
\pgfpathlineto{\pgfqpoint{2.080685in}{2.566711in}}%
\pgfpathlineto{\pgfqpoint{2.088377in}{2.561879in}}%
\pgfpathlineto{\pgfqpoint{2.090942in}{2.458042in}}%
\pgfpathlineto{\pgfqpoint{2.093509in}{2.572693in}}%
\pgfpathlineto{\pgfqpoint{2.100359in}{2.447712in}}%
\pgfpathlineto{\pgfqpoint{2.102070in}{2.543455in}}%
\pgfpathlineto{\pgfqpoint{2.109774in}{2.570094in}}%
\pgfpathlineto{\pgfqpoint{2.110631in}{2.483955in}}%
\pgfpathlineto{\pgfqpoint{2.116622in}{2.597156in}}%
\pgfpathlineto{\pgfqpoint{2.122605in}{2.489271in}}%
\pgfpathlineto{\pgfqpoint{2.123462in}{2.557107in}}%
\pgfpathlineto{\pgfqpoint{2.130310in}{2.440765in}}%
\pgfpathlineto{\pgfqpoint{2.133734in}{2.515125in}}%
\pgfpathlineto{\pgfqpoint{2.139720in}{2.481479in}}%
\pgfpathlineto{\pgfqpoint{2.140576in}{2.583686in}}%
\pgfpathlineto{\pgfqpoint{2.146558in}{2.435298in}}%
\pgfpathlineto{\pgfqpoint{2.149976in}{2.562302in}}%
\pgfpathlineto{\pgfqpoint{2.156813in}{2.425966in}}%
\pgfpathlineto{\pgfqpoint{2.159379in}{2.606821in}}%
\pgfpathlineto{\pgfqpoint{2.162803in}{2.477068in}}%
\pgfpathlineto{\pgfqpoint{2.169647in}{2.469428in}}%
\pgfpathlineto{\pgfqpoint{2.172214in}{2.608724in}}%
\pgfpathlineto{\pgfqpoint{2.174781in}{2.487610in}}%
\pgfpathlineto{\pgfqpoint{2.181627in}{2.463779in}}%
\pgfpathlineto{\pgfqpoint{2.184193in}{2.529199in}}%
\pgfpathlineto{\pgfqpoint{2.191037in}{2.446926in}}%
\pgfpathlineto{\pgfqpoint{2.195315in}{2.533005in}}%
\pgfpathlineto{\pgfqpoint{2.199592in}{2.489755in}}%
\pgfpathlineto{\pgfqpoint{2.200449in}{2.512980in}}%
\pgfpathlineto{\pgfqpoint{2.205581in}{2.465591in}}%
\pgfpathlineto{\pgfqpoint{2.209001in}{2.529803in}}%
\pgfpathlineto{\pgfqpoint{2.213282in}{2.559643in}}%
\pgfpathlineto{\pgfqpoint{2.220121in}{2.455564in}}%
\pgfpathlineto{\pgfqpoint{2.224396in}{2.577525in}}%
\pgfpathlineto{\pgfqpoint{2.226107in}{2.493772in}}%
\pgfpathlineto{\pgfqpoint{2.230384in}{2.550342in}}%
\pgfpathlineto{\pgfqpoint{2.236373in}{2.483955in}}%
\pgfpathlineto{\pgfqpoint{2.240649in}{2.580666in}}%
\pgfpathlineto{\pgfqpoint{2.245780in}{2.502136in}}%
\pgfpathlineto{\pgfqpoint{2.249197in}{2.599027in}}%
\pgfpathlineto{\pgfqpoint{2.252617in}{2.485767in}}%
\pgfpathlineto{\pgfqpoint{2.256038in}{2.598061in}}%
\pgfpathlineto{\pgfqpoint{2.261169in}{2.462088in}}%
\pgfpathlineto{\pgfqpoint{2.265446in}{2.457437in}}%
\pgfpathlineto{\pgfqpoint{2.270579in}{2.563691in}}%
\pgfpathlineto{\pgfqpoint{2.274003in}{2.425966in}}%
\pgfpathlineto{\pgfqpoint{2.279133in}{2.588215in}}%
\pgfpathlineto{\pgfqpoint{2.281703in}{2.524487in}}%
\pgfpathlineto{\pgfqpoint{2.288549in}{2.486614in}}%
\pgfpathlineto{\pgfqpoint{2.293687in}{2.559885in}}%
\pgfpathlineto{\pgfqpoint{2.294543in}{2.473988in}}%
\pgfpathlineto{\pgfqpoint{2.300531in}{2.545449in}}%
\pgfpathlineto{\pgfqpoint{2.306519in}{2.450490in}}%
\pgfpathlineto{\pgfqpoint{2.309941in}{2.603922in}}%
\pgfpathlineto{\pgfqpoint{2.311649in}{2.488245in}}%
\pgfpathlineto{\pgfqpoint{2.317638in}{2.565322in}}%
\pgfpathlineto{\pgfqpoint{2.322768in}{2.435540in}}%
\pgfpathlineto{\pgfqpoint{2.324477in}{2.481177in}}%
\pgfpathlineto{\pgfqpoint{2.332180in}{2.436687in}}%
\pgfpathlineto{\pgfqpoint{2.333891in}{2.510896in}}%
\pgfpathlineto{\pgfqpoint{2.339879in}{2.472599in}}%
\pgfpathlineto{\pgfqpoint{2.341587in}{2.550191in}}%
\pgfpathlineto{\pgfqpoint{2.347573in}{2.470484in}}%
\pgfpathlineto{\pgfqpoint{2.353560in}{2.472599in}}%
\pgfpathlineto{\pgfqpoint{2.356982in}{2.581934in}}%
\pgfpathlineto{\pgfqpoint{2.359552in}{2.517119in}}%
\pgfpathlineto{\pgfqpoint{2.365540in}{2.434907in}}%
\pgfpathlineto{\pgfqpoint{2.368963in}{2.546778in}}%
\pgfpathlineto{\pgfqpoint{2.373239in}{2.550221in}}%
\pgfpathlineto{\pgfqpoint{2.379228in}{2.445597in}}%
\pgfpathlineto{\pgfqpoint{2.380084in}{2.583867in}}%
\pgfpathlineto{\pgfqpoint{2.386070in}{2.464686in}}%
\pgfpathlineto{\pgfqpoint{2.388635in}{2.522042in}}%
\pgfpathlineto{\pgfqpoint{2.395480in}{2.489936in}}%
\pgfpathlineto{\pgfqpoint{2.398901in}{2.575168in}}%
\pgfpathlineto{\pgfqpoint{2.404887in}{2.495433in}}%
\pgfpathlineto{\pgfqpoint{2.406597in}{2.547322in}}%
\pgfpathlineto{\pgfqpoint{2.412580in}{2.451034in}}%
\pgfpathlineto{\pgfqpoint{2.415143in}{2.554087in}}%
\pgfpathlineto{\pgfqpoint{2.421985in}{2.462269in}}%
\pgfpathlineto{\pgfqpoint{2.424553in}{2.530166in}}%
\pgfpathlineto{\pgfqpoint{2.429680in}{2.556503in}}%
\pgfpathlineto{\pgfqpoint{2.433100in}{2.487761in}}%
\pgfpathlineto{\pgfqpoint{2.439086in}{2.555476in}}%
\pgfpathlineto{\pgfqpoint{2.440796in}{2.525092in}}%
\pgfpathlineto{\pgfqpoint{2.445075in}{2.474199in}}%
\pgfpathlineto{\pgfqpoint{2.449351in}{2.589546in}}%
\pgfpathlineto{\pgfqpoint{2.455339in}{2.462269in}}%
\pgfpathlineto{\pgfqpoint{2.460475in}{2.579216in}}%
\pgfpathlineto{\pgfqpoint{2.464751in}{2.518750in}}%
\pgfpathlineto{\pgfqpoint{2.468174in}{2.601265in}}%
\pgfpathlineto{\pgfqpoint{2.471595in}{2.492595in}}%
\pgfpathlineto{\pgfqpoint{2.474162in}{2.564478in}}%
\pgfpathlineto{\pgfqpoint{2.480149in}{2.466105in}}%
\pgfpathlineto{\pgfqpoint{2.484427in}{2.532523in}}%
\pgfpathlineto{\pgfqpoint{2.489559in}{2.451700in}}%
\pgfpathlineto{\pgfqpoint{2.492982in}{2.570036in}}%
\pgfpathlineto{\pgfqpoint{2.497259in}{2.440041in}}%
\pgfpathlineto{\pgfqpoint{2.500682in}{2.539107in}}%
\pgfpathlineto{\pgfqpoint{2.505817in}{2.560793in}}%
\pgfpathlineto{\pgfqpoint{2.508382in}{2.466680in}}%
\pgfpathlineto{\pgfqpoint{2.512659in}{2.535845in}}%
\pgfpathlineto{\pgfqpoint{2.519502in}{2.439739in}}%
\pgfpathlineto{\pgfqpoint{2.522926in}{2.534063in}}%
\pgfpathlineto{\pgfqpoint{2.526346in}{2.571062in}}%
\pgfpathlineto{\pgfqpoint{2.531478in}{2.477494in}}%
\pgfpathlineto{\pgfqpoint{2.535758in}{2.456532in}}%
\pgfpathlineto{\pgfqpoint{2.538324in}{2.556021in}}%
\pgfpathlineto{\pgfqpoint{2.544308in}{2.496460in}}%
\pgfpathlineto{\pgfqpoint{2.546873in}{2.564538in}}%
\pgfpathlineto{\pgfqpoint{2.551150in}{2.472208in}}%
\pgfpathlineto{\pgfqpoint{2.555422in}{2.535089in}}%
\pgfpathlineto{\pgfqpoint{2.563119in}{2.454477in}}%
\pgfpathlineto{\pgfqpoint{2.563974in}{2.538684in}}%
\pgfpathlineto{\pgfqpoint{2.569960in}{2.443182in}}%
\pgfpathlineto{\pgfqpoint{2.575949in}{2.584956in}}%
\pgfpathlineto{\pgfqpoint{2.576804in}{2.486704in}}%
\pgfpathlineto{\pgfqpoint{2.581931in}{2.436296in}}%
\pgfpathlineto{\pgfqpoint{2.587916in}{2.595044in}}%
\pgfpathlineto{\pgfqpoint{2.590484in}{2.451700in}}%
\pgfpathlineto{\pgfqpoint{2.595611in}{2.538986in}}%
\pgfpathlineto{\pgfqpoint{2.599886in}{2.557168in}}%
\pgfpathlineto{\pgfqpoint{2.603308in}{2.423369in}}%
\pgfpathlineto{\pgfqpoint{2.606733in}{2.529261in}}%
\pgfpathlineto{\pgfqpoint{2.611867in}{2.495071in}}%
\pgfpathlineto{\pgfqpoint{2.615290in}{2.565383in}}%
\pgfpathlineto{\pgfqpoint{2.622129in}{2.464142in}}%
\pgfpathlineto{\pgfqpoint{2.624695in}{2.553483in}}%
\pgfpathlineto{\pgfqpoint{2.631542in}{2.445960in}}%
\pgfpathlineto{\pgfqpoint{2.633255in}{2.531918in}}%
\pgfpathlineto{\pgfqpoint{2.636679in}{2.566953in}}%
\pgfpathlineto{\pgfqpoint{2.643518in}{2.439316in}}%
\pgfpathlineto{\pgfqpoint{2.647797in}{2.530166in}}%
\pgfpathlineto{\pgfqpoint{2.652077in}{2.487037in}}%
\pgfpathlineto{\pgfqpoint{2.655500in}{2.555839in}}%
\pgfpathlineto{\pgfqpoint{2.659777in}{2.486614in}}%
\pgfpathlineto{\pgfqpoint{2.663199in}{2.565685in}}%
\pgfpathlineto{\pgfqpoint{2.666621in}{2.493440in}}%
\pgfpathlineto{\pgfqpoint{2.670900in}{2.562968in}}%
\pgfpathlineto{\pgfqpoint{2.676032in}{2.478157in}}%
\pgfpathlineto{\pgfqpoint{2.681163in}{2.473385in}}%
\pgfpathlineto{\pgfqpoint{2.686295in}{2.579216in}}%
\pgfpathlineto{\pgfqpoint{2.689720in}{2.478308in}}%
\pgfpathlineto{\pgfqpoint{2.693143in}{2.560067in}}%
\pgfpathlineto{\pgfqpoint{2.696565in}{2.460035in}}%
\pgfpathlineto{\pgfqpoint{2.703410in}{2.537839in}}%
\pgfpathlineto{\pgfqpoint{2.708544in}{2.434090in}}%
\pgfpathlineto{\pgfqpoint{2.710256in}{2.562605in}}%
\pgfpathlineto{\pgfqpoint{2.714525in}{2.466196in}}%
\pgfpathlineto{\pgfqpoint{2.718798in}{2.525032in}}%
\pgfpathlineto{\pgfqpoint{2.723934in}{2.461485in}}%
\pgfpathlineto{\pgfqpoint{2.728213in}{2.562363in}}%
\pgfpathlineto{\pgfqpoint{2.731635in}{2.473325in}}%
\pgfpathlineto{\pgfqpoint{2.736769in}{2.521528in}}%
\pgfpathlineto{\pgfqpoint{2.741046in}{2.460396in}}%
\pgfpathlineto{\pgfqpoint{2.743612in}{2.553362in}}%
\pgfpathlineto{\pgfqpoint{2.751304in}{2.430677in}}%
\pgfpathlineto{\pgfqpoint{2.752160in}{2.513797in}}%
\pgfpathlineto{\pgfqpoint{2.758145in}{2.497276in}}%
\pgfpathlineto{\pgfqpoint{2.762417in}{2.573840in}}%
\pgfpathlineto{\pgfqpoint{2.764980in}{2.510777in}}%
\pgfpathlineto{\pgfqpoint{2.772681in}{2.478520in}}%
\pgfpathlineto{\pgfqpoint{2.775247in}{2.533791in}}%
\pgfpathlineto{\pgfqpoint{2.781232in}{2.562605in}}%
\pgfpathlineto{\pgfqpoint{2.782089in}{2.481661in}}%
\pgfpathlineto{\pgfqpoint{2.789789in}{2.534517in}}%
\pgfpathlineto{\pgfqpoint{2.790642in}{2.446504in}}%
\pgfpathlineto{\pgfqpoint{2.794918in}{2.511319in}}%
\pgfpathlineto{\pgfqpoint{2.799189in}{2.451518in}}%
\pgfpathlineto{\pgfqpoint{2.806031in}{2.567679in}}%
\pgfpathlineto{\pgfqpoint{2.810305in}{2.609661in}}%
\pgfpathlineto{\pgfqpoint{2.812017in}{2.505582in}}%
\pgfpathlineto{\pgfqpoint{2.818005in}{2.572390in}}%
\pgfpathlineto{\pgfqpoint{2.821424in}{2.496339in}}%
\pgfpathlineto{\pgfqpoint{2.827409in}{2.577525in}}%
\pgfpathlineto{\pgfqpoint{2.831684in}{2.442819in}}%
\pgfpathlineto{\pgfqpoint{2.836818in}{2.590814in}}%
\pgfpathlineto{\pgfqpoint{2.837671in}{2.480030in}}%
\pgfpathlineto{\pgfqpoint{2.844515in}{2.561034in}}%
\pgfpathlineto{\pgfqpoint{2.847078in}{2.478338in}}%
\pgfpathlineto{\pgfqpoint{2.853918in}{2.407542in}}%
\pgfpathlineto{\pgfqpoint{2.856482in}{2.546868in}}%
\pgfpathlineto{\pgfqpoint{2.860755in}{2.464505in}}%
\pgfpathlineto{\pgfqpoint{2.866743in}{2.592024in}}%
\pgfpathlineto{\pgfqpoint{2.868455in}{2.490601in}}%
\pgfpathlineto{\pgfqpoint{2.871877in}{2.547806in}}%
\pgfpathlineto{\pgfqpoint{2.877860in}{2.470726in}}%
\pgfpathlineto{\pgfqpoint{2.882135in}{2.544723in}}%
\pgfpathlineto{\pgfqpoint{2.886413in}{2.565685in}}%
\pgfpathlineto{\pgfqpoint{2.889835in}{2.430284in}}%
\pgfpathlineto{\pgfqpoint{2.896681in}{2.580424in}}%
\pgfpathlineto{\pgfqpoint{2.898392in}{2.476103in}}%
\pgfpathlineto{\pgfqpoint{2.905233in}{2.564115in}}%
\pgfpathlineto{\pgfqpoint{2.906089in}{2.480421in}}%
\pgfpathlineto{\pgfqpoint{2.913784in}{2.535301in}}%
\pgfpathlineto{\pgfqpoint{2.917204in}{2.429046in}}%
\pgfpathlineto{\pgfqpoint{2.918914in}{2.543516in}}%
\pgfpathlineto{\pgfqpoint{2.923190in}{2.415336in}}%
\pgfpathlineto{\pgfqpoint{2.929171in}{2.520925in}}%
\pgfpathlineto{\pgfqpoint{2.932595in}{2.487460in}}%
\pgfpathlineto{\pgfqpoint{2.938585in}{2.646872in}}%
\pgfpathlineto{\pgfqpoint{2.942004in}{2.551249in}}%
\pgfpathlineto{\pgfqpoint{2.945424in}{2.659316in}}%
\pgfpathlineto{\pgfqpoint{2.949701in}{2.570157in}}%
\pgfpathlineto{\pgfqpoint{2.956540in}{2.656296in}}%
\pgfpathlineto{\pgfqpoint{2.960820in}{2.538926in}}%
\pgfpathlineto{\pgfqpoint{2.963385in}{2.630079in}}%
\pgfpathlineto{\pgfqpoint{2.966806in}{2.655631in}}%
\pgfpathlineto{\pgfqpoint{2.970229in}{2.505943in}}%
\pgfpathlineto{\pgfqpoint{2.975364in}{2.545570in}}%
\pgfpathlineto{\pgfqpoint{2.980495in}{2.477975in}}%
\pgfpathlineto{\pgfqpoint{2.986484in}{2.516998in}}%
\pgfpathlineto{\pgfqpoint{2.990761in}{2.464898in}}%
\pgfpathlineto{\pgfqpoint{2.992470in}{2.561879in}}%
\pgfpathlineto{\pgfqpoint{2.998455in}{2.457497in}}%
\pgfpathlineto{\pgfqpoint{3.001024in}{2.575289in}}%
\pgfpathlineto{\pgfqpoint{3.007011in}{2.474472in}}%
\pgfpathlineto{\pgfqpoint{3.009578in}{2.558436in}}%
\pgfpathlineto{\pgfqpoint{3.012997in}{2.468732in}}%
\pgfpathlineto{\pgfqpoint{3.018130in}{2.465833in}}%
\pgfpathlineto{\pgfqpoint{3.024973in}{2.531253in}}%
\pgfpathlineto{\pgfqpoint{3.029249in}{2.446805in}}%
\pgfpathlineto{\pgfqpoint{3.031813in}{2.568161in}}%
\pgfpathlineto{\pgfqpoint{3.035233in}{2.516605in}}%
\pgfpathlineto{\pgfqpoint{3.040363in}{2.562875in}}%
\pgfpathlineto{\pgfqpoint{3.042930in}{2.468793in}}%
\pgfpathlineto{\pgfqpoint{3.048063in}{2.574082in}}%
\pgfpathlineto{\pgfqpoint{3.052341in}{2.462571in}}%
\pgfpathlineto{\pgfqpoint{3.055765in}{2.573961in}}%
\pgfpathlineto{\pgfqpoint{3.062609in}{2.496339in}}%
\pgfpathlineto{\pgfqpoint{3.065175in}{2.539589in}}%
\pgfpathlineto{\pgfqpoint{3.071163in}{2.529440in}}%
\pgfpathlineto{\pgfqpoint{3.073730in}{2.472780in}}%
\pgfpathlineto{\pgfqpoint{3.078860in}{2.542248in}}%
\pgfpathlineto{\pgfqpoint{3.084849in}{2.439406in}}%
\pgfpathlineto{\pgfqpoint{3.086559in}{2.569429in}}%
\pgfpathlineto{\pgfqpoint{3.090835in}{2.469760in}}%
\pgfpathlineto{\pgfqpoint{3.096823in}{2.421043in}}%
\pgfpathlineto{\pgfqpoint{3.099388in}{2.517180in}}%
\pgfpathlineto{\pgfqpoint{3.102807in}{2.397878in}}%
\pgfpathlineto{\pgfqpoint{3.107087in}{2.532160in}}%
\pgfpathlineto{\pgfqpoint{3.113076in}{2.470274in}}%
\pgfpathlineto{\pgfqpoint{3.119072in}{2.587008in}}%
\pgfpathlineto{\pgfqpoint{3.119928in}{2.510535in}}%
\pgfpathlineto{\pgfqpoint{3.125916in}{2.474714in}}%
\pgfpathlineto{\pgfqpoint{3.128480in}{2.538805in}}%
\pgfpathlineto{\pgfqpoint{3.136178in}{2.608996in}}%
\pgfpathlineto{\pgfqpoint{3.137034in}{2.481479in}}%
\pgfpathlineto{\pgfqpoint{3.144728in}{2.521467in}}%
\pgfpathlineto{\pgfqpoint{3.145581in}{2.452907in}}%
\pgfpathlineto{\pgfqpoint{3.153277in}{2.425936in}}%
\pgfpathlineto{\pgfqpoint{3.154131in}{2.542066in}}%
\pgfpathlineto{\pgfqpoint{3.161832in}{2.465773in}}%
\pgfpathlineto{\pgfqpoint{3.164395in}{2.531737in}}%
\pgfpathlineto{\pgfqpoint{3.168665in}{2.560672in}}%
\pgfpathlineto{\pgfqpoint{3.172086in}{2.418656in}}%
\pgfpathlineto{\pgfqpoint{3.175506in}{2.568705in}}%
\pgfpathlineto{\pgfqpoint{3.181495in}{2.499479in}}%
\pgfpathlineto{\pgfqpoint{3.184063in}{2.590935in}}%
\pgfpathlineto{\pgfqpoint{3.191760in}{2.513192in}}%
\pgfpathlineto{\pgfqpoint{3.196038in}{2.556503in}}%
\pgfpathlineto{\pgfqpoint{3.197748in}{2.434423in}}%
\pgfpathlineto{\pgfqpoint{3.202885in}{2.569128in}}%
\pgfpathlineto{\pgfqpoint{3.205451in}{2.454659in}}%
\pgfpathlineto{\pgfqpoint{3.211439in}{2.502862in}}%
\pgfpathlineto{\pgfqpoint{3.217427in}{2.522856in}}%
\pgfpathlineto{\pgfqpoint{3.221702in}{2.452302in}}%
\pgfpathlineto{\pgfqpoint{3.224266in}{2.528112in}}%
\pgfpathlineto{\pgfqpoint{3.226834in}{2.437261in}}%
\pgfpathlineto{\pgfqpoint{3.231112in}{2.519232in}}%
\pgfpathlineto{\pgfqpoint{3.237954in}{2.474804in}}%
\pgfpathlineto{\pgfqpoint{3.243083in}{2.541401in}}%
\pgfpathlineto{\pgfqpoint{3.246503in}{2.455685in}}%
\pgfpathlineto{\pgfqpoint{3.248209in}{2.549979in}}%
\pgfpathlineto{\pgfqpoint{3.255909in}{2.558496in}}%
\pgfpathlineto{\pgfqpoint{3.256765in}{2.447349in}}%
\pgfpathlineto{\pgfqpoint{3.264463in}{2.558920in}}%
\pgfpathlineto{\pgfqpoint{3.268739in}{2.479788in}}%
\pgfpathlineto{\pgfqpoint{3.273013in}{2.531283in}}%
\pgfpathlineto{\pgfqpoint{3.276437in}{2.544905in}}%
\pgfpathlineto{\pgfqpoint{3.281573in}{2.421011in}}%
\pgfpathlineto{\pgfqpoint{3.283283in}{2.564355in}}%
\pgfpathlineto{\pgfqpoint{3.287561in}{2.495552in}}%
\pgfpathlineto{\pgfqpoint{3.294401in}{2.571302in}}%
\pgfpathlineto{\pgfqpoint{3.298673in}{2.464505in}}%
\pgfpathlineto{\pgfqpoint{3.299528in}{2.549798in}}%
\pgfpathlineto{\pgfqpoint{3.303802in}{2.461182in}}%
\pgfpathlineto{\pgfqpoint{3.308077in}{2.500747in}}%
\pgfpathlineto{\pgfqpoint{3.315769in}{2.437745in}}%
\pgfpathlineto{\pgfqpoint{3.316624in}{2.560430in}}%
\pgfpathlineto{\pgfqpoint{3.324328in}{2.574866in}}%
\pgfpathlineto{\pgfqpoint{3.325184in}{2.471694in}}%
\pgfpathlineto{\pgfqpoint{3.329460in}{2.444450in}}%
\pgfpathlineto{\pgfqpoint{3.333737in}{2.563570in}}%
\pgfpathlineto{\pgfqpoint{3.338014in}{2.480451in}}%
\pgfpathlineto{\pgfqpoint{3.342293in}{2.560188in}}%
\pgfpathlineto{\pgfqpoint{3.349137in}{2.444087in}}%
\pgfpathlineto{\pgfqpoint{3.351703in}{2.533065in}}%
\pgfpathlineto{\pgfqpoint{3.355122in}{2.487216in}}%
\pgfpathlineto{\pgfqpoint{3.361114in}{2.566590in}}%
\pgfpathlineto{\pgfqpoint{3.365390in}{2.457376in}}%
\pgfpathlineto{\pgfqpoint{3.371379in}{2.566772in}}%
\pgfpathlineto{\pgfqpoint{3.372234in}{2.516635in}}%
\pgfpathlineto{\pgfqpoint{3.379931in}{2.375827in}}%
\pgfpathlineto{\pgfqpoint{3.380784in}{2.538803in}}%
\pgfpathlineto{\pgfqpoint{3.385062in}{2.470363in}}%
\pgfpathlineto{\pgfqpoint{3.389340in}{2.589544in}}%
\pgfpathlineto{\pgfqpoint{3.397035in}{2.443422in}}%
\pgfpathlineto{\pgfqpoint{3.397890in}{2.569126in}}%
\pgfpathlineto{\pgfqpoint{3.402168in}{2.581571in}}%
\pgfpathlineto{\pgfqpoint{3.406443in}{2.480330in}}%
\pgfpathlineto{\pgfqpoint{3.412430in}{2.553120in}}%
\pgfpathlineto{\pgfqpoint{3.414994in}{2.503044in}}%
\pgfpathlineto{\pgfqpoint{3.420976in}{2.540585in}}%
\pgfpathlineto{\pgfqpoint{3.423538in}{2.476887in}}%
\pgfpathlineto{\pgfqpoint{3.427816in}{2.569126in}}%
\pgfpathlineto{\pgfqpoint{3.434657in}{2.438227in}}%
\pgfpathlineto{\pgfqpoint{3.437223in}{2.529138in}}%
\pgfpathlineto{\pgfqpoint{3.442356in}{2.476735in}}%
\pgfpathlineto{\pgfqpoint{3.445779in}{2.570697in}}%
\pgfpathlineto{\pgfqpoint{3.449197in}{2.459277in}}%
\pgfpathlineto{\pgfqpoint{3.453471in}{2.546112in}}%
\pgfpathlineto{\pgfqpoint{3.458605in}{2.581389in}}%
\pgfpathlineto{\pgfqpoint{3.462882in}{2.402287in}}%
\pgfpathlineto{\pgfqpoint{3.467156in}{2.599271in}}%
\pgfpathlineto{\pgfqpoint{3.472285in}{2.457013in}}%
\pgfpathlineto{\pgfqpoint{3.475705in}{2.506064in}}%
\pgfpathlineto{\pgfqpoint{3.482550in}{2.439134in}}%
\pgfpathlineto{\pgfqpoint{3.485118in}{2.509688in}}%
\pgfpathlineto{\pgfqpoint{3.491104in}{2.444569in}}%
\pgfpathlineto{\pgfqpoint{3.495380in}{2.558978in}}%
\pgfpathlineto{\pgfqpoint{3.496237in}{2.416420in}}%
\pgfpathlineto{\pgfqpoint{3.501370in}{2.524971in}}%
\pgfpathlineto{\pgfqpoint{3.504790in}{2.422160in}}%
\pgfpathlineto{\pgfqpoint{3.512483in}{2.580121in}}%
\pgfpathlineto{\pgfqpoint{3.513340in}{2.479364in}}%
\pgfpathlineto{\pgfqpoint{3.519327in}{2.531797in}}%
\pgfpathlineto{\pgfqpoint{3.522750in}{2.450008in}}%
\pgfpathlineto{\pgfqpoint{3.529596in}{2.524699in}}%
\pgfpathlineto{\pgfqpoint{3.532161in}{2.415939in}}%
\pgfpathlineto{\pgfqpoint{3.534724in}{2.508299in}}%
\pgfpathlineto{\pgfqpoint{3.539002in}{2.417116in}}%
\pgfpathlineto{\pgfqpoint{3.543277in}{2.522161in}}%
\pgfpathlineto{\pgfqpoint{3.547556in}{2.433697in}}%
\pgfpathlineto{\pgfqpoint{3.555248in}{2.452784in}}%
\pgfpathlineto{\pgfqpoint{3.558668in}{2.548709in}}%
\pgfpathlineto{\pgfqpoint{3.565506in}{2.407721in}}%
\pgfpathlineto{\pgfqpoint{3.569782in}{2.538863in}}%
\pgfpathlineto{\pgfqpoint{3.574058in}{2.454145in}}%
\pgfpathlineto{\pgfqpoint{3.580899in}{2.554539in}}%
\pgfpathlineto{\pgfqpoint{3.583469in}{2.466980in}}%
\pgfpathlineto{\pgfqpoint{3.587741in}{2.497304in}}%
\pgfpathlineto{\pgfqpoint{3.591161in}{2.453570in}}%
\pgfpathlineto{\pgfqpoint{3.596294in}{2.572449in}}%
\pgfpathlineto{\pgfqpoint{3.599713in}{2.480421in}}%
\pgfpathlineto{\pgfqpoint{3.603993in}{2.539347in}}%
\pgfpathlineto{\pgfqpoint{3.608273in}{2.456590in}}%
\pgfpathlineto{\pgfqpoint{3.611692in}{2.569308in}}%
\pgfpathlineto{\pgfqpoint{3.619390in}{2.549918in}}%
\pgfpathlineto{\pgfqpoint{3.622816in}{2.439255in}}%
\pgfpathlineto{\pgfqpoint{3.625381in}{2.564657in}}%
\pgfpathlineto{\pgfqpoint{3.631368in}{2.495794in}}%
\pgfpathlineto{\pgfqpoint{3.635643in}{2.566832in}}%
\pgfpathlineto{\pgfqpoint{3.637355in}{2.491295in}}%
\pgfpathlineto{\pgfqpoint{3.643344in}{2.552394in}}%
\pgfpathlineto{\pgfqpoint{3.646770in}{2.435782in}}%
\pgfpathlineto{\pgfqpoint{3.651047in}{2.497728in}}%
\pgfpathlineto{\pgfqpoint{3.655322in}{2.448315in}}%
\pgfpathlineto{\pgfqpoint{3.658742in}{2.519322in}}%
\pgfpathlineto{\pgfqpoint{3.665581in}{2.462481in}}%
\pgfpathlineto{\pgfqpoint{3.667291in}{2.563026in}}%
\pgfpathlineto{\pgfqpoint{3.674993in}{2.462088in}}%
\pgfpathlineto{\pgfqpoint{3.677560in}{2.545901in}}%
\pgfpathlineto{\pgfqpoint{3.680976in}{2.430314in}}%
\pgfpathlineto{\pgfqpoint{3.686964in}{2.560309in}}%
\pgfpathlineto{\pgfqpoint{3.692094in}{2.437443in}}%
\pgfpathlineto{\pgfqpoint{3.692951in}{2.535240in}}%
\pgfpathlineto{\pgfqpoint{3.700650in}{2.561758in}}%
\pgfpathlineto{\pgfqpoint{3.701506in}{2.479667in}}%
\pgfpathlineto{\pgfqpoint{3.707493in}{2.440221in}}%
\pgfpathlineto{\pgfqpoint{3.710060in}{2.535301in}}%
\pgfpathlineto{\pgfqpoint{3.714341in}{2.529138in}}%
\pgfpathlineto{\pgfqpoint{3.721187in}{2.444992in}}%
\pgfpathlineto{\pgfqpoint{3.726321in}{2.539891in}}%
\pgfpathlineto{\pgfqpoint{3.727177in}{2.455262in}}%
\pgfpathlineto{\pgfqpoint{3.734016in}{2.463265in}}%
\pgfpathlineto{\pgfqpoint{3.735725in}{2.553964in}}%
\pgfpathlineto{\pgfqpoint{3.741708in}{2.601807in}}%
\pgfpathlineto{\pgfqpoint{3.745984in}{2.435267in}}%
\pgfpathlineto{\pgfqpoint{3.749406in}{2.520742in}}%
\pgfpathlineto{\pgfqpoint{3.756244in}{2.438408in}}%
\pgfpathlineto{\pgfqpoint{3.759666in}{2.578914in}}%
\pgfpathlineto{\pgfqpoint{3.762235in}{2.439948in}}%
\pgfpathlineto{\pgfqpoint{3.768221in}{2.521709in}}%
\pgfpathlineto{\pgfqpoint{3.771647in}{2.463265in}}%
\pgfpathlineto{\pgfqpoint{3.775066in}{2.555476in}}%
\pgfpathlineto{\pgfqpoint{3.778487in}{2.433969in}}%
\pgfpathlineto{\pgfqpoint{3.783618in}{2.415092in}}%
\pgfpathlineto{\pgfqpoint{3.787896in}{2.554055in}}%
\pgfpathlineto{\pgfqpoint{3.794743in}{2.510472in}}%
\pgfpathlineto{\pgfqpoint{3.799020in}{2.627782in}}%
\pgfpathlineto{\pgfqpoint{3.801584in}{2.512829in}}%
\pgfpathlineto{\pgfqpoint{3.805856in}{2.564839in}}%
\pgfpathlineto{\pgfqpoint{3.810133in}{2.473504in}}%
\pgfpathlineto{\pgfqpoint{3.814406in}{2.460457in}}%
\pgfpathlineto{\pgfqpoint{3.816973in}{2.611230in}}%
\pgfpathlineto{\pgfqpoint{3.821247in}{2.470030in}}%
\pgfpathlineto{\pgfqpoint{3.828940in}{2.564899in}}%
\pgfpathlineto{\pgfqpoint{3.830650in}{2.463900in}}%
\pgfpathlineto{\pgfqpoint{3.834922in}{2.556261in}}%
\pgfpathlineto{\pgfqpoint{3.839197in}{2.486914in}}%
\pgfpathlineto{\pgfqpoint{3.842619in}{2.588094in}}%
\pgfpathlineto{\pgfqpoint{3.848600in}{2.462027in}}%
\pgfpathlineto{\pgfqpoint{3.851170in}{2.560851in}}%
\pgfpathlineto{\pgfqpoint{3.856308in}{2.511107in}}%
\pgfpathlineto{\pgfqpoint{3.859732in}{2.569913in}}%
\pgfpathlineto{\pgfqpoint{3.864870in}{2.465470in}}%
\pgfpathlineto{\pgfqpoint{3.869999in}{2.542851in}}%
\pgfpathlineto{\pgfqpoint{3.875984in}{2.468127in}}%
\pgfpathlineto{\pgfqpoint{3.878551in}{2.545749in}}%
\pgfpathlineto{\pgfqpoint{3.882831in}{2.475558in}}%
\pgfpathlineto{\pgfqpoint{3.887107in}{2.455141in}}%
\pgfpathlineto{\pgfqpoint{3.890525in}{2.526902in}}%
\pgfpathlineto{\pgfqpoint{3.895656in}{2.486854in}}%
\pgfpathlineto{\pgfqpoint{3.899075in}{2.549253in}}%
\pgfpathlineto{\pgfqpoint{3.903348in}{2.491383in}}%
\pgfpathlineto{\pgfqpoint{3.908484in}{2.568824in}}%
\pgfpathlineto{\pgfqpoint{3.911052in}{2.441126in}}%
\pgfpathlineto{\pgfqpoint{3.917892in}{2.594134in}}%
\pgfpathlineto{\pgfqpoint{3.919603in}{2.514821in}}%
\pgfpathlineto{\pgfqpoint{3.925587in}{2.565683in}}%
\pgfpathlineto{\pgfqpoint{3.928153in}{2.522010in}}%
\pgfpathlineto{\pgfqpoint{3.934142in}{2.484315in}}%
\pgfpathlineto{\pgfqpoint{3.936707in}{2.602410in}}%
\pgfpathlineto{\pgfqpoint{3.943555in}{2.438860in}}%
\pgfpathlineto{\pgfqpoint{3.945268in}{2.556561in}}%
\pgfpathlineto{\pgfqpoint{3.949543in}{2.432911in}}%
\pgfpathlineto{\pgfqpoint{3.954676in}{2.439858in}}%
\pgfpathlineto{\pgfqpoint{3.961520in}{2.597698in}}%
\pgfpathlineto{\pgfqpoint{3.962375in}{2.493438in}}%
\pgfpathlineto{\pgfqpoint{3.970067in}{2.451455in}}%
\pgfpathlineto{\pgfqpoint{3.970924in}{2.575952in}}%
\pgfpathlineto{\pgfqpoint{3.976909in}{2.459007in}}%
\pgfpathlineto{\pgfqpoint{3.979475in}{2.533849in}}%
\pgfpathlineto{\pgfqpoint{3.984613in}{2.488363in}}%
\pgfpathlineto{\pgfqpoint{3.991456in}{2.464805in}}%
\pgfpathlineto{\pgfqpoint{3.992312in}{2.549372in}}%
\pgfpathlineto{\pgfqpoint{3.997445in}{2.464684in}}%
\pgfpathlineto{\pgfqpoint{4.000867in}{2.543816in}}%
\pgfpathlineto{\pgfqpoint{4.006003in}{2.477792in}}%
\pgfpathlineto{\pgfqpoint{4.011132in}{2.537776in}}%
\pgfpathlineto{\pgfqpoint{4.015412in}{2.440039in}}%
\pgfpathlineto{\pgfqpoint{4.018834in}{2.584772in}}%
\pgfpathlineto{\pgfqpoint{4.022252in}{2.480814in}}%
\pgfpathlineto{\pgfqpoint{4.029095in}{2.456953in}}%
\pgfpathlineto{\pgfqpoint{4.032514in}{2.549556in}}%
\pgfpathlineto{\pgfqpoint{4.035079in}{2.497486in}}%
\pgfpathlineto{\pgfqpoint{4.039357in}{2.552757in}}%
\pgfpathlineto{\pgfqpoint{4.046202in}{2.497365in}}%
\pgfpathlineto{\pgfqpoint{4.048767in}{2.579819in}}%
\pgfpathlineto{\pgfqpoint{4.054759in}{2.630681in}}%
\pgfpathlineto{\pgfqpoint{4.056470in}{2.518599in}}%
\pgfpathlineto{\pgfqpoint{4.063314in}{2.564687in}}%
\pgfpathlineto{\pgfqpoint{4.068446in}{2.392620in}}%
\pgfpathlineto{\pgfqpoint{4.070156in}{2.547138in}}%
\pgfpathlineto{\pgfqpoint{4.073578in}{2.460517in}}%
\pgfpathlineto{\pgfqpoint{4.079562in}{2.426085in}}%
\pgfpathlineto{\pgfqpoint{4.082985in}{2.543695in}}%
\pgfpathlineto{\pgfqpoint{4.093250in}{2.436052in}}%
\pgfpathlineto{\pgfqpoint{4.097525in}{2.569731in}}%
\pgfpathlineto{\pgfqpoint{4.102659in}{2.582115in}}%
\pgfpathlineto{\pgfqpoint{4.103512in}{2.442335in}}%
\pgfpathlineto{\pgfqpoint{4.108646in}{2.571785in}}%
\pgfpathlineto{\pgfqpoint{4.113785in}{2.439616in}}%
\pgfpathlineto{\pgfqpoint{4.119765in}{2.578007in}}%
\pgfpathlineto{\pgfqpoint{4.121474in}{2.520318in}}%
\pgfpathlineto{\pgfqpoint{4.128317in}{2.423970in}}%
\pgfpathlineto{\pgfqpoint{4.130028in}{2.476100in}}%
\pgfpathlineto{\pgfqpoint{4.136878in}{2.442999in}}%
\pgfpathlineto{\pgfqpoint{4.137733in}{2.517117in}}%
\pgfpathlineto{\pgfqpoint{4.142868in}{2.457919in}}%
\pgfpathlineto{\pgfqpoint{4.147142in}{2.471994in}}%
\pgfpathlineto{\pgfqpoint{4.149711in}{2.526600in}}%
\pgfpathlineto{\pgfqpoint{4.152280in}{2.462932in}}%
\pgfpathlineto{\pgfqpoint{4.158268in}{2.465892in}}%
\pgfpathlineto{\pgfqpoint{4.159124in}{2.569911in}}%
\pgfpathlineto{\pgfqpoint{4.162548in}{2.479846in}}%
\pgfpathlineto{\pgfqpoint{4.166829in}{2.477159in}}%
\pgfpathlineto{\pgfqpoint{4.170249in}{2.571483in}}%
\pgfpathlineto{\pgfqpoint{4.172814in}{2.495008in}}%
\pgfpathlineto{\pgfqpoint{4.177091in}{2.569610in}}%
\pgfpathlineto{\pgfqpoint{4.181364in}{2.490478in}}%
\pgfpathlineto{\pgfqpoint{4.184787in}{2.564718in}}%
\pgfpathlineto{\pgfqpoint{4.186496in}{2.471692in}}%
\pgfpathlineto{\pgfqpoint{4.191629in}{2.488847in}}%
\pgfpathlineto{\pgfqpoint{4.194194in}{2.563268in}}%
\pgfpathlineto{\pgfqpoint{4.196761in}{2.500626in}}%
\pgfpathlineto{\pgfqpoint{4.201034in}{2.584530in}}%
\pgfpathlineto{\pgfqpoint{4.204456in}{2.480118in}}%
\pgfpathlineto{\pgfqpoint{4.207878in}{2.577583in}}%
\pgfpathlineto{\pgfqpoint{4.210442in}{2.489271in}}%
\pgfpathlineto{\pgfqpoint{4.216431in}{2.421434in}}%
\pgfpathlineto{\pgfqpoint{4.218141in}{2.512708in}}%
\pgfpathlineto{\pgfqpoint{4.223272in}{2.484015in}}%
\pgfpathlineto{\pgfqpoint{4.224983in}{2.558678in}}%
\pgfpathlineto{\pgfqpoint{4.230116in}{2.475467in}}%
\pgfpathlineto{\pgfqpoint{4.230972in}{2.518748in}}%
\pgfpathlineto{\pgfqpoint{4.235248in}{2.489543in}}%
\pgfpathlineto{\pgfqpoint{4.238671in}{2.453028in}}%
\pgfpathlineto{\pgfqpoint{4.242092in}{2.528082in}}%
\pgfpathlineto{\pgfqpoint{4.246370in}{2.436898in}}%
\pgfpathlineto{\pgfqpoint{4.248080in}{2.504130in}}%
\pgfpathlineto{\pgfqpoint{4.254070in}{2.441700in}}%
\pgfpathlineto{\pgfqpoint{4.257490in}{2.516454in}}%
\pgfpathlineto{\pgfqpoint{4.259202in}{2.444871in}}%
\pgfpathlineto{\pgfqpoint{4.263479in}{2.465107in}}%
\pgfpathlineto{\pgfqpoint{4.266903in}{2.586040in}}%
\pgfpathlineto{\pgfqpoint{4.270325in}{2.460638in}}%
\pgfpathlineto{\pgfqpoint{4.273748in}{2.549072in}}%
\pgfpathlineto{\pgfqpoint{4.277169in}{2.475407in}}%
\pgfpathlineto{\pgfqpoint{4.278880in}{2.544421in}}%
\pgfpathlineto{\pgfqpoint{4.288282in}{2.462148in}}%
\pgfpathlineto{\pgfqpoint{4.289137in}{2.549556in}}%
\pgfpathlineto{\pgfqpoint{4.292562in}{2.605371in}}%
\pgfpathlineto{\pgfqpoint{4.295981in}{2.454899in}}%
\pgfpathlineto{\pgfqpoint{4.300259in}{2.535059in}}%
\pgfpathlineto{\pgfqpoint{4.304529in}{2.576981in}}%
\pgfpathlineto{\pgfqpoint{4.307096in}{2.496941in}}%
\pgfpathlineto{\pgfqpoint{4.311374in}{2.592203in}}%
\pgfpathlineto{\pgfqpoint{4.315651in}{2.502015in}}%
\pgfpathlineto{\pgfqpoint{4.318213in}{2.593953in}}%
\pgfpathlineto{\pgfqpoint{4.320779in}{2.505277in}}%
\pgfpathlineto{\pgfqpoint{4.324195in}{2.559039in}}%
\pgfpathlineto{\pgfqpoint{4.326759in}{2.478245in}}%
\pgfpathlineto{\pgfqpoint{4.330182in}{2.524608in}}%
\pgfpathlineto{\pgfqpoint{4.336171in}{2.486974in}}%
\pgfpathlineto{\pgfqpoint{4.338734in}{2.478878in}}%
\pgfpathlineto{\pgfqpoint{4.340445in}{2.564234in}}%
\pgfpathlineto{\pgfqpoint{4.346437in}{2.456772in}}%
\pgfpathlineto{\pgfqpoint{4.347293in}{2.526539in}}%
\pgfpathlineto{\pgfqpoint{4.353279in}{2.442999in}}%
\pgfpathlineto{\pgfqpoint{4.356698in}{2.571723in}}%
\pgfpathlineto{\pgfqpoint{4.357553in}{2.445444in}}%
\pgfpathlineto{\pgfqpoint{4.360978in}{2.529199in}}%
\pgfpathlineto{\pgfqpoint{4.366112in}{2.464684in}}%
\pgfpathlineto{\pgfqpoint{4.367823in}{2.549253in}}%
\pgfpathlineto{\pgfqpoint{4.371246in}{2.462993in}}%
\pgfpathlineto{\pgfqpoint{4.377235in}{2.553843in}}%
\pgfpathlineto{\pgfqpoint{4.378945in}{2.463537in}}%
\pgfpathlineto{\pgfqpoint{4.381508in}{2.556200in}}%
\pgfpathlineto{\pgfqpoint{4.385782in}{2.410741in}}%
\pgfpathlineto{\pgfqpoint{4.389205in}{2.602531in}}%
\pgfpathlineto{\pgfqpoint{4.392627in}{2.463507in}}%
\pgfpathlineto{\pgfqpoint{4.395194in}{2.533910in}}%
\pgfpathlineto{\pgfqpoint{4.399466in}{2.471117in}}%
\pgfpathlineto{\pgfqpoint{4.402888in}{2.523943in}}%
\pgfpathlineto{\pgfqpoint{4.407166in}{2.444509in}}%
\pgfpathlineto{\pgfqpoint{4.408878in}{2.512073in}}%
\pgfpathlineto{\pgfqpoint{4.414012in}{2.412644in}}%
\pgfpathlineto{\pgfqpoint{4.415722in}{2.495552in}}%
\pgfpathlineto{\pgfqpoint{4.419998in}{2.457074in}}%
\pgfpathlineto{\pgfqpoint{4.424273in}{2.573235in}}%
\pgfpathlineto{\pgfqpoint{4.425984in}{2.483652in}}%
\pgfpathlineto{\pgfqpoint{4.431970in}{2.490660in}}%
\pgfpathlineto{\pgfqpoint{4.435391in}{2.521105in}}%
\pgfpathlineto{\pgfqpoint{4.437960in}{2.470424in}}%
\pgfpathlineto{\pgfqpoint{4.439672in}{2.557347in}}%
\pgfpathlineto{\pgfqpoint{4.443953in}{2.582718in}}%
\pgfpathlineto{\pgfqpoint{4.447371in}{2.470605in}}%
\pgfpathlineto{\pgfqpoint{4.451649in}{2.531674in}}%
\pgfpathlineto{\pgfqpoint{4.455068in}{2.456227in}}%
\pgfpathlineto{\pgfqpoint{4.456777in}{2.520137in}}%
\pgfpathlineto{\pgfqpoint{4.461912in}{2.558978in}}%
\pgfpathlineto{\pgfqpoint{4.464478in}{2.502739in}}%
\pgfpathlineto{\pgfqpoint{4.469610in}{2.560851in}}%
\pgfpathlineto{\pgfqpoint{4.470466in}{2.476947in}}%
\pgfpathlineto{\pgfqpoint{4.473885in}{2.437743in}}%
\pgfpathlineto{\pgfqpoint{4.478161in}{2.547501in}}%
\pgfpathlineto{\pgfqpoint{4.482436in}{2.496639in}}%
\pgfpathlineto{\pgfqpoint{4.486712in}{2.530739in}}%
\pgfpathlineto{\pgfqpoint{4.490131in}{2.449522in}}%
\pgfpathlineto{\pgfqpoint{4.490987in}{2.556140in}}%
\pgfpathlineto{\pgfqpoint{4.494410in}{2.466496in}}%
\pgfpathlineto{\pgfqpoint{4.498682in}{2.553874in}}%
\pgfpathlineto{\pgfqpoint{4.501244in}{2.479483in}}%
\pgfpathlineto{\pgfqpoint{4.504663in}{2.543695in}}%
\pgfpathlineto{\pgfqpoint{4.508083in}{2.496306in}}%
\pgfpathlineto{\pgfqpoint{4.511500in}{2.557166in}}%
\pgfpathlineto{\pgfqpoint{4.514921in}{2.455231in}}%
\pgfpathlineto{\pgfqpoint{4.520052in}{2.524427in}}%
\pgfpathlineto{\pgfqpoint{4.524328in}{2.461725in}}%
\pgfpathlineto{\pgfqpoint{4.525183in}{2.554569in}}%
\pgfpathlineto{\pgfqpoint{4.531176in}{2.475377in}}%
\pgfpathlineto{\pgfqpoint{4.532887in}{2.546050in}}%
\pgfpathlineto{\pgfqpoint{4.535450in}{2.443120in}}%
\pgfpathlineto{\pgfqpoint{4.538868in}{2.737601in}}%
\pgfpathlineto{\pgfqpoint{4.542287in}{2.840776in}}%
\pgfpathlineto{\pgfqpoint{4.545713in}{2.776291in}}%
\pgfpathlineto{\pgfqpoint{4.549136in}{2.797947in}}%
\pgfpathlineto{\pgfqpoint{4.552557in}{2.735487in}}%
\pgfpathlineto{\pgfqpoint{4.557688in}{2.817338in}}%
\pgfpathlineto{\pgfqpoint{4.560254in}{2.727032in}}%
\pgfpathlineto{\pgfqpoint{4.564531in}{2.776775in}}%
\pgfpathlineto{\pgfqpoint{4.566243in}{2.692539in}}%
\pgfpathlineto{\pgfqpoint{4.569658in}{2.777892in}}%
\pgfpathlineto{\pgfqpoint{4.573939in}{2.714134in}}%
\pgfpathlineto{\pgfqpoint{4.577361in}{2.795350in}}%
\pgfpathlineto{\pgfqpoint{4.581640in}{2.666745in}}%
\pgfpathlineto{\pgfqpoint{4.583350in}{2.830053in}}%
\pgfpathlineto{\pgfqpoint{4.588482in}{2.836004in}}%
\pgfpathlineto{\pgfqpoint{4.590190in}{2.753731in}}%
\pgfpathlineto{\pgfqpoint{4.593612in}{2.883000in}}%
\pgfpathlineto{\pgfqpoint{4.597035in}{2.766718in}}%
\pgfpathlineto{\pgfqpoint{4.600455in}{2.817580in}}%
\pgfpathlineto{\pgfqpoint{4.605589in}{2.788042in}}%
\pgfpathlineto{\pgfqpoint{4.608155in}{2.491688in}}%
\pgfpathlineto{\pgfqpoint{4.612433in}{2.543032in}}%
\pgfpathlineto{\pgfqpoint{4.616710in}{2.440039in}}%
\pgfpathlineto{\pgfqpoint{4.617565in}{2.494284in}}%
\pgfpathlineto{\pgfqpoint{4.621838in}{2.524427in}}%
\pgfpathlineto{\pgfqpoint{4.624405in}{2.574684in}}%
\pgfpathlineto{\pgfqpoint{4.627827in}{2.454175in}}%
\pgfpathlineto{\pgfqpoint{4.632102in}{2.558133in}}%
\pgfpathlineto{\pgfqpoint{4.635525in}{2.447349in}}%
\pgfpathlineto{\pgfqpoint{4.638091in}{2.543032in}}%
\pgfpathlineto{\pgfqpoint{4.642368in}{2.500385in}}%
\pgfpathlineto{\pgfqpoint{4.645787in}{2.605432in}}%
\pgfpathlineto{\pgfqpoint{4.649207in}{2.515486in}}%
\pgfpathlineto{\pgfqpoint{4.652631in}{2.541280in}}%
\pgfpathlineto{\pgfqpoint{4.656051in}{2.473837in}}%
\pgfpathlineto{\pgfqpoint{4.661182in}{2.528231in}}%
\pgfpathlineto{\pgfqpoint{4.662894in}{2.462539in}}%
\pgfpathlineto{\pgfqpoint{4.666317in}{2.505217in}}%
\pgfpathlineto{\pgfqpoint{4.668881in}{2.453750in}}%
\pgfpathlineto{\pgfqpoint{4.674015in}{2.474893in}}%
\pgfpathlineto{\pgfqpoint{4.677437in}{2.553450in}}%
\pgfpathlineto{\pgfqpoint{4.679148in}{2.498996in}}%
\pgfpathlineto{\pgfqpoint{4.682567in}{2.439555in}}%
\pgfpathlineto{\pgfqpoint{4.687701in}{2.542367in}}%
\pgfpathlineto{\pgfqpoint{4.689413in}{2.551610in}}%
\pgfpathlineto{\pgfqpoint{4.693685in}{2.478669in}}%
\pgfpathlineto{\pgfqpoint{4.697112in}{2.528533in}}%
\pgfpathlineto{\pgfqpoint{4.700535in}{2.432399in}}%
\pgfpathlineto{\pgfqpoint{4.704809in}{2.540917in}}%
\pgfpathlineto{\pgfqpoint{4.706522in}{2.485102in}}%
\pgfpathlineto{\pgfqpoint{4.711657in}{2.554085in}}%
\pgfpathlineto{\pgfqpoint{4.715932in}{2.507090in}}%
\pgfpathlineto{\pgfqpoint{4.718499in}{2.584530in}}%
\pgfpathlineto{\pgfqpoint{4.721922in}{2.497213in}}%
\pgfpathlineto{\pgfqpoint{4.724492in}{2.554448in}}%
\pgfpathlineto{\pgfqpoint{4.727918in}{2.471873in}}%
\pgfpathlineto{\pgfqpoint{4.730484in}{2.520923in}}%
\pgfpathlineto{\pgfqpoint{4.736479in}{2.470000in}}%
\pgfpathlineto{\pgfqpoint{4.737335in}{2.574563in}}%
\pgfpathlineto{\pgfqpoint{4.742472in}{2.561214in}}%
\pgfpathlineto{\pgfqpoint{4.745895in}{2.440160in}}%
\pgfpathlineto{\pgfqpoint{4.747605in}{2.522161in}}%
\pgfpathlineto{\pgfqpoint{4.751882in}{2.609659in}}%
\pgfpathlineto{\pgfqpoint{4.755301in}{2.482384in}}%
\pgfpathlineto{\pgfqpoint{4.759573in}{2.538170in}}%
\pgfpathlineto{\pgfqpoint{4.763849in}{2.486372in}}%
\pgfpathlineto{\pgfqpoint{4.766411in}{2.593652in}}%
\pgfpathlineto{\pgfqpoint{4.769831in}{2.471752in}}%
\pgfpathlineto{\pgfqpoint{4.773255in}{2.532581in}}%
\pgfpathlineto{\pgfqpoint{4.776680in}{2.544844in}}%
\pgfpathlineto{\pgfqpoint{4.778393in}{2.447319in}}%
\pgfpathlineto{\pgfqpoint{4.782675in}{2.502015in}}%
\pgfpathlineto{\pgfqpoint{4.787806in}{2.465561in}}%
\pgfpathlineto{\pgfqpoint{4.791230in}{2.500929in}}%
\pgfpathlineto{\pgfqpoint{4.794650in}{2.482293in}}%
\pgfpathlineto{\pgfqpoint{4.795506in}{2.527991in}}%
\pgfpathlineto{\pgfqpoint{4.795506in}{2.527991in}}%
\pgfusepath{stroke}%
\end{pgfscope}%
\begin{pgfscope}%
\pgfsetrectcap%
\pgfsetmiterjoin%
\pgfsetlinewidth{0.803000pt}%
\definecolor{currentstroke}{rgb}{0.000000,0.000000,0.000000}%
\pgfsetstrokecolor{currentstroke}%
\pgfsetdash{}{0pt}%
\pgfpathmoveto{\pgfqpoint{0.484581in}{2.334497in}}%
\pgfpathlineto{\pgfqpoint{0.484581in}{2.909119in}}%
\pgfusepath{stroke}%
\end{pgfscope}%
\begin{pgfscope}%
\pgfsetrectcap%
\pgfsetmiterjoin%
\pgfsetlinewidth{0.803000pt}%
\definecolor{currentstroke}{rgb}{0.000000,0.000000,0.000000}%
\pgfsetstrokecolor{currentstroke}%
\pgfsetdash{}{0pt}%
\pgfpathmoveto{\pgfqpoint{5.000788in}{2.334497in}}%
\pgfpathlineto{\pgfqpoint{5.000788in}{2.909119in}}%
\pgfusepath{stroke}%
\end{pgfscope}%
\begin{pgfscope}%
\pgfsetrectcap%
\pgfsetmiterjoin%
\pgfsetlinewidth{0.803000pt}%
\definecolor{currentstroke}{rgb}{0.000000,0.000000,0.000000}%
\pgfsetstrokecolor{currentstroke}%
\pgfsetdash{}{0pt}%
\pgfpathmoveto{\pgfqpoint{0.484581in}{2.334497in}}%
\pgfpathlineto{\pgfqpoint{5.000788in}{2.334497in}}%
\pgfusepath{stroke}%
\end{pgfscope}%
\begin{pgfscope}%
\pgfsetrectcap%
\pgfsetmiterjoin%
\pgfsetlinewidth{0.803000pt}%
\definecolor{currentstroke}{rgb}{0.000000,0.000000,0.000000}%
\pgfsetstrokecolor{currentstroke}%
\pgfsetdash{}{0pt}%
\pgfpathmoveto{\pgfqpoint{0.484581in}{2.909119in}}%
\pgfpathlineto{\pgfqpoint{5.000788in}{2.909119in}}%
\pgfusepath{stroke}%
\end{pgfscope}%
\begin{pgfscope}%
\pgfsetbuttcap%
\pgfsetmiterjoin%
\definecolor{currentfill}{rgb}{1.000000,1.000000,1.000000}%
\pgfsetfillcolor{currentfill}%
\pgfsetlinewidth{0.000000pt}%
\definecolor{currentstroke}{rgb}{0.000000,0.000000,0.000000}%
\pgfsetstrokecolor{currentstroke}%
\pgfsetstrokeopacity{0.000000}%
\pgfsetdash{}{0pt}%
\pgfpathmoveto{\pgfqpoint{0.484581in}{1.437021in}}%
\pgfpathlineto{\pgfqpoint{5.000788in}{1.437021in}}%
\pgfpathlineto{\pgfqpoint{5.000788in}{2.011643in}}%
\pgfpathlineto{\pgfqpoint{0.484581in}{2.011643in}}%
\pgfpathlineto{\pgfqpoint{0.484581in}{1.437021in}}%
\pgfpathclose%
\pgfusepath{fill}%
\end{pgfscope}%
\begin{pgfscope}%
\pgfsetbuttcap%
\pgfsetroundjoin%
\definecolor{currentfill}{rgb}{0.000000,0.000000,0.000000}%
\pgfsetfillcolor{currentfill}%
\pgfsetlinewidth{0.803000pt}%
\definecolor{currentstroke}{rgb}{0.000000,0.000000,0.000000}%
\pgfsetstrokecolor{currentstroke}%
\pgfsetdash{}{0pt}%
\pgfsys@defobject{currentmarker}{\pgfqpoint{0.000000in}{-0.048611in}}{\pgfqpoint{0.000000in}{0.000000in}}{%
\pgfpathmoveto{\pgfqpoint{0.000000in}{0.000000in}}%
\pgfpathlineto{\pgfqpoint{0.000000in}{-0.048611in}}%
\pgfusepath{stroke,fill}%
}%
\begin{pgfscope}%
\pgfsys@transformshift{0.689546in}{1.437021in}%
\pgfsys@useobject{currentmarker}{}%
\end{pgfscope}%
\end{pgfscope}%
\begin{pgfscope}%
\pgfsetbuttcap%
\pgfsetroundjoin%
\definecolor{currentfill}{rgb}{0.000000,0.000000,0.000000}%
\pgfsetfillcolor{currentfill}%
\pgfsetlinewidth{0.803000pt}%
\definecolor{currentstroke}{rgb}{0.000000,0.000000,0.000000}%
\pgfsetstrokecolor{currentstroke}%
\pgfsetdash{}{0pt}%
\pgfsys@defobject{currentmarker}{\pgfqpoint{0.000000in}{-0.048611in}}{\pgfqpoint{0.000000in}{0.000000in}}{%
\pgfpathmoveto{\pgfqpoint{0.000000in}{0.000000in}}%
\pgfpathlineto{\pgfqpoint{0.000000in}{-0.048611in}}%
\pgfusepath{stroke,fill}%
}%
\begin{pgfscope}%
\pgfsys@transformshift{1.202878in}{1.437021in}%
\pgfsys@useobject{currentmarker}{}%
\end{pgfscope}%
\end{pgfscope}%
\begin{pgfscope}%
\pgfsetbuttcap%
\pgfsetroundjoin%
\definecolor{currentfill}{rgb}{0.000000,0.000000,0.000000}%
\pgfsetfillcolor{currentfill}%
\pgfsetlinewidth{0.803000pt}%
\definecolor{currentstroke}{rgb}{0.000000,0.000000,0.000000}%
\pgfsetstrokecolor{currentstroke}%
\pgfsetdash{}{0pt}%
\pgfsys@defobject{currentmarker}{\pgfqpoint{0.000000in}{-0.048611in}}{\pgfqpoint{0.000000in}{0.000000in}}{%
\pgfpathmoveto{\pgfqpoint{0.000000in}{0.000000in}}%
\pgfpathlineto{\pgfqpoint{0.000000in}{-0.048611in}}%
\pgfusepath{stroke,fill}%
}%
\begin{pgfscope}%
\pgfsys@transformshift{1.716211in}{1.437021in}%
\pgfsys@useobject{currentmarker}{}%
\end{pgfscope}%
\end{pgfscope}%
\begin{pgfscope}%
\pgfsetbuttcap%
\pgfsetroundjoin%
\definecolor{currentfill}{rgb}{0.000000,0.000000,0.000000}%
\pgfsetfillcolor{currentfill}%
\pgfsetlinewidth{0.803000pt}%
\definecolor{currentstroke}{rgb}{0.000000,0.000000,0.000000}%
\pgfsetstrokecolor{currentstroke}%
\pgfsetdash{}{0pt}%
\pgfsys@defobject{currentmarker}{\pgfqpoint{0.000000in}{-0.048611in}}{\pgfqpoint{0.000000in}{0.000000in}}{%
\pgfpathmoveto{\pgfqpoint{0.000000in}{0.000000in}}%
\pgfpathlineto{\pgfqpoint{0.000000in}{-0.048611in}}%
\pgfusepath{stroke,fill}%
}%
\begin{pgfscope}%
\pgfsys@transformshift{2.229543in}{1.437021in}%
\pgfsys@useobject{currentmarker}{}%
\end{pgfscope}%
\end{pgfscope}%
\begin{pgfscope}%
\pgfsetbuttcap%
\pgfsetroundjoin%
\definecolor{currentfill}{rgb}{0.000000,0.000000,0.000000}%
\pgfsetfillcolor{currentfill}%
\pgfsetlinewidth{0.803000pt}%
\definecolor{currentstroke}{rgb}{0.000000,0.000000,0.000000}%
\pgfsetstrokecolor{currentstroke}%
\pgfsetdash{}{0pt}%
\pgfsys@defobject{currentmarker}{\pgfqpoint{0.000000in}{-0.048611in}}{\pgfqpoint{0.000000in}{0.000000in}}{%
\pgfpathmoveto{\pgfqpoint{0.000000in}{0.000000in}}%
\pgfpathlineto{\pgfqpoint{0.000000in}{-0.048611in}}%
\pgfusepath{stroke,fill}%
}%
\begin{pgfscope}%
\pgfsys@transformshift{2.742876in}{1.437021in}%
\pgfsys@useobject{currentmarker}{}%
\end{pgfscope}%
\end{pgfscope}%
\begin{pgfscope}%
\pgfsetbuttcap%
\pgfsetroundjoin%
\definecolor{currentfill}{rgb}{0.000000,0.000000,0.000000}%
\pgfsetfillcolor{currentfill}%
\pgfsetlinewidth{0.803000pt}%
\definecolor{currentstroke}{rgb}{0.000000,0.000000,0.000000}%
\pgfsetstrokecolor{currentstroke}%
\pgfsetdash{}{0pt}%
\pgfsys@defobject{currentmarker}{\pgfqpoint{0.000000in}{-0.048611in}}{\pgfqpoint{0.000000in}{0.000000in}}{%
\pgfpathmoveto{\pgfqpoint{0.000000in}{0.000000in}}%
\pgfpathlineto{\pgfqpoint{0.000000in}{-0.048611in}}%
\pgfusepath{stroke,fill}%
}%
\begin{pgfscope}%
\pgfsys@transformshift{3.256208in}{1.437021in}%
\pgfsys@useobject{currentmarker}{}%
\end{pgfscope}%
\end{pgfscope}%
\begin{pgfscope}%
\pgfsetbuttcap%
\pgfsetroundjoin%
\definecolor{currentfill}{rgb}{0.000000,0.000000,0.000000}%
\pgfsetfillcolor{currentfill}%
\pgfsetlinewidth{0.803000pt}%
\definecolor{currentstroke}{rgb}{0.000000,0.000000,0.000000}%
\pgfsetstrokecolor{currentstroke}%
\pgfsetdash{}{0pt}%
\pgfsys@defobject{currentmarker}{\pgfqpoint{0.000000in}{-0.048611in}}{\pgfqpoint{0.000000in}{0.000000in}}{%
\pgfpathmoveto{\pgfqpoint{0.000000in}{0.000000in}}%
\pgfpathlineto{\pgfqpoint{0.000000in}{-0.048611in}}%
\pgfusepath{stroke,fill}%
}%
\begin{pgfscope}%
\pgfsys@transformshift{3.769541in}{1.437021in}%
\pgfsys@useobject{currentmarker}{}%
\end{pgfscope}%
\end{pgfscope}%
\begin{pgfscope}%
\pgfsetbuttcap%
\pgfsetroundjoin%
\definecolor{currentfill}{rgb}{0.000000,0.000000,0.000000}%
\pgfsetfillcolor{currentfill}%
\pgfsetlinewidth{0.803000pt}%
\definecolor{currentstroke}{rgb}{0.000000,0.000000,0.000000}%
\pgfsetstrokecolor{currentstroke}%
\pgfsetdash{}{0pt}%
\pgfsys@defobject{currentmarker}{\pgfqpoint{0.000000in}{-0.048611in}}{\pgfqpoint{0.000000in}{0.000000in}}{%
\pgfpathmoveto{\pgfqpoint{0.000000in}{0.000000in}}%
\pgfpathlineto{\pgfqpoint{0.000000in}{-0.048611in}}%
\pgfusepath{stroke,fill}%
}%
\begin{pgfscope}%
\pgfsys@transformshift{4.282873in}{1.437021in}%
\pgfsys@useobject{currentmarker}{}%
\end{pgfscope}%
\end{pgfscope}%
\begin{pgfscope}%
\pgfsetbuttcap%
\pgfsetroundjoin%
\definecolor{currentfill}{rgb}{0.000000,0.000000,0.000000}%
\pgfsetfillcolor{currentfill}%
\pgfsetlinewidth{0.803000pt}%
\definecolor{currentstroke}{rgb}{0.000000,0.000000,0.000000}%
\pgfsetstrokecolor{currentstroke}%
\pgfsetdash{}{0pt}%
\pgfsys@defobject{currentmarker}{\pgfqpoint{0.000000in}{-0.048611in}}{\pgfqpoint{0.000000in}{0.000000in}}{%
\pgfpathmoveto{\pgfqpoint{0.000000in}{0.000000in}}%
\pgfpathlineto{\pgfqpoint{0.000000in}{-0.048611in}}%
\pgfusepath{stroke,fill}%
}%
\begin{pgfscope}%
\pgfsys@transformshift{4.796206in}{1.437021in}%
\pgfsys@useobject{currentmarker}{}%
\end{pgfscope}%
\end{pgfscope}%
\begin{pgfscope}%
\pgfsetbuttcap%
\pgfsetroundjoin%
\definecolor{currentfill}{rgb}{0.000000,0.000000,0.000000}%
\pgfsetfillcolor{currentfill}%
\pgfsetlinewidth{0.803000pt}%
\definecolor{currentstroke}{rgb}{0.000000,0.000000,0.000000}%
\pgfsetstrokecolor{currentstroke}%
\pgfsetdash{}{0pt}%
\pgfsys@defobject{currentmarker}{\pgfqpoint{-0.048611in}{0.000000in}}{\pgfqpoint{-0.000000in}{0.000000in}}{%
\pgfpathmoveto{\pgfqpoint{-0.000000in}{0.000000in}}%
\pgfpathlineto{\pgfqpoint{-0.048611in}{0.000000in}}%
\pgfusepath{stroke,fill}%
}%
\begin{pgfscope}%
\pgfsys@transformshift{0.484581in}{1.615020in}%
\pgfsys@useobject{currentmarker}{}%
\end{pgfscope}%
\end{pgfscope}%
\begin{pgfscope}%
\definecolor{textcolor}{rgb}{0.000000,0.000000,0.000000}%
\pgfsetstrokecolor{textcolor}%
\pgfsetfillcolor{textcolor}%
\pgftext[x=0.328331in, y=1.576464in, left, base]{\color{textcolor}\rmfamily\fontsize{8.000000}{9.600000}\selectfont \(\displaystyle {0}\)}%
\end{pgfscope}%
\begin{pgfscope}%
\pgfsetbuttcap%
\pgfsetroundjoin%
\definecolor{currentfill}{rgb}{0.000000,0.000000,0.000000}%
\pgfsetfillcolor{currentfill}%
\pgfsetlinewidth{0.803000pt}%
\definecolor{currentstroke}{rgb}{0.000000,0.000000,0.000000}%
\pgfsetstrokecolor{currentstroke}%
\pgfsetdash{}{0pt}%
\pgfsys@defobject{currentmarker}{\pgfqpoint{-0.048611in}{0.000000in}}{\pgfqpoint{-0.000000in}{0.000000in}}{%
\pgfpathmoveto{\pgfqpoint{-0.000000in}{0.000000in}}%
\pgfpathlineto{\pgfqpoint{-0.048611in}{0.000000in}}%
\pgfusepath{stroke,fill}%
}%
\begin{pgfscope}%
\pgfsys@transformshift{0.484581in}{1.818561in}%
\pgfsys@useobject{currentmarker}{}%
\end{pgfscope}%
\end{pgfscope}%
\begin{pgfscope}%
\definecolor{textcolor}{rgb}{0.000000,0.000000,0.000000}%
\pgfsetstrokecolor{textcolor}%
\pgfsetfillcolor{textcolor}%
\pgftext[x=0.328331in, y=1.780005in, left, base]{\color{textcolor}\rmfamily\fontsize{8.000000}{9.600000}\selectfont \(\displaystyle {5}\)}%
\end{pgfscope}%
\begin{pgfscope}%
\definecolor{textcolor}{rgb}{0.000000,0.000000,0.000000}%
\pgfsetstrokecolor{textcolor}%
\pgfsetfillcolor{textcolor}%
\pgftext[x=0.272775in,y=1.724332in,,bottom,rotate=90.000000]{\color{textcolor}\rmfamily\fontsize{10.000000}{12.000000}\selectfont Voltage deviation in \unit{\V}}%
\end{pgfscope}%
\begin{pgfscope}%
\definecolor{textcolor}{rgb}{0.000000,0.000000,0.000000}%
\pgfsetstrokecolor{textcolor}%
\pgfsetfillcolor{textcolor}%
\pgftext[x=0.484581in,y=2.053309in,left,base]{\color{textcolor}\rmfamily\fontsize{8.000000}{9.600000}\selectfont \(\displaystyle \times{10^{\ensuremath{-}6}}{}\)}%
\end{pgfscope}%
\begin{pgfscope}%
\pgfpathrectangle{\pgfqpoint{0.484581in}{1.437021in}}{\pgfqpoint{4.516206in}{0.574622in}}%
\pgfusepath{clip}%
\pgfsetrectcap%
\pgfsetroundjoin%
\pgfsetlinewidth{0.501875pt}%
\definecolor{currentstroke}{rgb}{0.003922,0.450980,0.698039}%
\pgfsetstrokecolor{currentstroke}%
\pgfsetstrokeopacity{0.700000}%
\pgfsetdash{}{0pt}%
\pgfpathmoveto{\pgfqpoint{0.689863in}{1.586143in}}%
\pgfpathlineto{\pgfqpoint{0.691573in}{1.625337in}}%
\pgfpathlineto{\pgfqpoint{0.694995in}{1.554591in}}%
\pgfpathlineto{\pgfqpoint{0.700132in}{1.624113in}}%
\pgfpathlineto{\pgfqpoint{0.706120in}{1.553835in}}%
\pgfpathlineto{\pgfqpoint{0.707833in}{1.630006in}}%
\pgfpathlineto{\pgfqpoint{0.712110in}{1.550218in}}%
\pgfpathlineto{\pgfqpoint{0.719814in}{1.595418in}}%
\pgfpathlineto{\pgfqpoint{0.723233in}{1.529484in}}%
\pgfpathlineto{\pgfqpoint{0.725797in}{1.600142in}}%
\pgfpathlineto{\pgfqpoint{0.730072in}{1.521436in}}%
\pgfpathlineto{\pgfqpoint{0.734351in}{1.620555in}}%
\pgfpathlineto{\pgfqpoint{0.740340in}{1.509859in}}%
\pgfpathlineto{\pgfqpoint{0.742908in}{1.627964in}}%
\pgfpathlineto{\pgfqpoint{0.749745in}{1.515752in}}%
\pgfpathlineto{\pgfqpoint{0.750602in}{1.631112in}}%
\pgfpathlineto{\pgfqpoint{0.757448in}{1.584280in}}%
\pgfpathlineto{\pgfqpoint{0.761721in}{1.771029in}}%
\pgfpathlineto{\pgfqpoint{0.764288in}{1.661499in}}%
\pgfpathlineto{\pgfqpoint{0.770278in}{1.670975in}}%
\pgfpathlineto{\pgfqpoint{0.771990in}{1.588302in}}%
\pgfpathlineto{\pgfqpoint{0.777980in}{1.623412in}}%
\pgfpathlineto{\pgfqpoint{0.783113in}{1.520705in}}%
\pgfpathlineto{\pgfqpoint{0.784824in}{1.596583in}}%
\pgfpathlineto{\pgfqpoint{0.792516in}{1.565616in}}%
\pgfpathlineto{\pgfqpoint{0.793373in}{1.644234in}}%
\pgfpathlineto{\pgfqpoint{0.797648in}{1.563749in}}%
\pgfpathlineto{\pgfqpoint{0.802779in}{1.499769in}}%
\pgfpathlineto{\pgfqpoint{0.807054in}{1.491546in}}%
\pgfpathlineto{\pgfqpoint{0.811330in}{1.621837in}}%
\pgfpathlineto{\pgfqpoint{0.816461in}{1.552432in}}%
\pgfpathlineto{\pgfqpoint{0.819881in}{1.653214in}}%
\pgfpathlineto{\pgfqpoint{0.825871in}{1.546953in}}%
\pgfpathlineto{\pgfqpoint{0.827584in}{1.683834in}}%
\pgfpathlineto{\pgfqpoint{0.836141in}{1.522426in}}%
\pgfpathlineto{\pgfqpoint{0.841276in}{1.594483in}}%
\pgfpathlineto{\pgfqpoint{0.848114in}{1.547709in}}%
\pgfpathlineto{\pgfqpoint{0.851531in}{1.619912in}}%
\pgfpathlineto{\pgfqpoint{0.854095in}{1.645692in}}%
\pgfpathlineto{\pgfqpoint{0.857519in}{1.567596in}}%
\pgfpathlineto{\pgfqpoint{0.864372in}{1.555406in}}%
\pgfpathlineto{\pgfqpoint{0.867793in}{1.625220in}}%
\pgfpathlineto{\pgfqpoint{0.871215in}{1.560422in}}%
\pgfpathlineto{\pgfqpoint{0.874639in}{1.655431in}}%
\pgfpathlineto{\pgfqpoint{0.879771in}{1.700631in}}%
\pgfpathlineto{\pgfqpoint{0.883195in}{1.617114in}}%
\pgfpathlineto{\pgfqpoint{0.889177in}{1.721161in}}%
\pgfpathlineto{\pgfqpoint{0.893457in}{1.578095in}}%
\pgfpathlineto{\pgfqpoint{0.898587in}{1.665404in}}%
\pgfpathlineto{\pgfqpoint{0.901153in}{1.582702in}}%
\pgfpathlineto{\pgfqpoint{0.906287in}{1.633735in}}%
\pgfpathlineto{\pgfqpoint{0.908853in}{1.559285in}}%
\pgfpathlineto{\pgfqpoint{0.916550in}{1.533012in}}%
\pgfpathlineto{\pgfqpoint{0.918259in}{1.614516in}}%
\pgfpathlineto{\pgfqpoint{0.924241in}{1.548176in}}%
\pgfpathlineto{\pgfqpoint{0.925951in}{1.593201in}}%
\pgfpathlineto{\pgfqpoint{0.930227in}{1.522163in}}%
\pgfpathlineto{\pgfqpoint{0.936214in}{1.628775in}}%
\pgfpathlineto{\pgfqpoint{0.938782in}{1.509446in}}%
\pgfpathlineto{\pgfqpoint{0.943058in}{1.594717in}}%
\pgfpathlineto{\pgfqpoint{0.947330in}{1.479644in}}%
\pgfpathlineto{\pgfqpoint{0.952461in}{1.636943in}}%
\pgfpathlineto{\pgfqpoint{0.958451in}{1.527588in}}%
\pgfpathlineto{\pgfqpoint{0.961015in}{1.600025in}}%
\pgfpathlineto{\pgfqpoint{0.967858in}{1.482388in}}%
\pgfpathlineto{\pgfqpoint{0.970425in}{1.625045in}}%
\pgfpathlineto{\pgfqpoint{0.976408in}{1.643591in}}%
\pgfpathlineto{\pgfqpoint{0.977263in}{1.606586in}}%
\pgfpathlineto{\pgfqpoint{0.982392in}{1.522689in}}%
\pgfpathlineto{\pgfqpoint{0.989232in}{1.502626in}}%
\pgfpathlineto{\pgfqpoint{0.990941in}{1.614896in}}%
\pgfpathlineto{\pgfqpoint{1.000349in}{1.478625in}}%
\pgfpathlineto{\pgfqpoint{1.002914in}{1.594454in}}%
\pgfpathlineto{\pgfqpoint{1.009758in}{1.546573in}}%
\pgfpathlineto{\pgfqpoint{1.011468in}{1.620789in}}%
\pgfpathlineto{\pgfqpoint{1.017456in}{1.559958in}}%
\pgfpathlineto{\pgfqpoint{1.022584in}{1.708156in}}%
\pgfpathlineto{\pgfqpoint{1.024292in}{1.600609in}}%
\pgfpathlineto{\pgfqpoint{1.031988in}{1.634320in}}%
\pgfpathlineto{\pgfqpoint{1.032842in}{1.691184in}}%
\pgfpathlineto{\pgfqpoint{1.039687in}{1.748691in}}%
\pgfpathlineto{\pgfqpoint{1.043113in}{1.674416in}}%
\pgfpathlineto{\pgfqpoint{1.048251in}{1.722268in}}%
\pgfpathlineto{\pgfqpoint{1.049962in}{1.651522in}}%
\pgfpathlineto{\pgfqpoint{1.055949in}{1.713289in}}%
\pgfpathlineto{\pgfqpoint{1.060226in}{1.562756in}}%
\pgfpathlineto{\pgfqpoint{1.064502in}{1.663011in}}%
\pgfpathlineto{\pgfqpoint{1.068774in}{1.536859in}}%
\pgfpathlineto{\pgfqpoint{1.071339in}{1.603086in}}%
\pgfpathlineto{\pgfqpoint{1.076466in}{1.541641in}}%
\pgfpathlineto{\pgfqpoint{1.080746in}{1.617724in}}%
\pgfpathlineto{\pgfqpoint{1.086736in}{1.542109in}}%
\pgfpathlineto{\pgfqpoint{1.091868in}{1.631226in}}%
\pgfpathlineto{\pgfqpoint{1.096997in}{1.523620in}}%
\pgfpathlineto{\pgfqpoint{1.102978in}{1.566836in}}%
\pgfpathlineto{\pgfqpoint{1.108959in}{1.626415in}}%
\pgfpathlineto{\pgfqpoint{1.111524in}{1.539661in}}%
\pgfpathlineto{\pgfqpoint{1.115796in}{1.683048in}}%
\pgfpathlineto{\pgfqpoint{1.121786in}{1.726002in}}%
\pgfpathlineto{\pgfqpoint{1.122640in}{1.618659in}}%
\pgfpathlineto{\pgfqpoint{1.130340in}{1.593551in}}%
\pgfpathlineto{\pgfqpoint{1.132050in}{1.648929in}}%
\pgfpathlineto{\pgfqpoint{1.136328in}{1.567830in}}%
\pgfpathlineto{\pgfqpoint{1.140604in}{1.724602in}}%
\pgfpathlineto{\pgfqpoint{1.146593in}{1.623032in}}%
\pgfpathlineto{\pgfqpoint{1.154289in}{1.756154in}}%
\pgfpathlineto{\pgfqpoint{1.157708in}{1.579319in}}%
\pgfpathlineto{\pgfqpoint{1.163696in}{1.517322in}}%
\pgfpathlineto{\pgfqpoint{1.165408in}{1.614546in}}%
\pgfpathlineto{\pgfqpoint{1.171400in}{1.551033in}}%
\pgfpathlineto{\pgfqpoint{1.174823in}{1.608072in}}%
\pgfpathlineto{\pgfqpoint{1.179102in}{1.542781in}}%
\pgfpathlineto{\pgfqpoint{1.185947in}{1.665579in}}%
\pgfpathlineto{\pgfqpoint{1.188512in}{1.532749in}}%
\pgfpathlineto{\pgfqpoint{1.191934in}{1.637001in}}%
\pgfpathlineto{\pgfqpoint{1.196209in}{1.492595in}}%
\pgfpathlineto{\pgfqpoint{1.203055in}{1.490786in}}%
\pgfpathlineto{\pgfqpoint{1.203910in}{1.586611in}}%
\pgfpathlineto{\pgfqpoint{1.211615in}{1.672053in}}%
\pgfpathlineto{\pgfqpoint{1.212472in}{1.589643in}}%
\pgfpathlineto{\pgfqpoint{1.216755in}{1.605128in}}%
\pgfpathlineto{\pgfqpoint{1.221891in}{1.553397in}}%
\pgfpathlineto{\pgfqpoint{1.228734in}{1.607780in}}%
\pgfpathlineto{\pgfqpoint{1.230446in}{1.528694in}}%
\pgfpathlineto{\pgfqpoint{1.236437in}{1.577978in}}%
\pgfpathlineto{\pgfqpoint{1.239857in}{1.539135in}}%
\pgfpathlineto{\pgfqpoint{1.245845in}{1.615072in}}%
\pgfpathlineto{\pgfqpoint{1.247557in}{1.566080in}}%
\pgfpathlineto{\pgfqpoint{1.251837in}{1.608540in}}%
\pgfpathlineto{\pgfqpoint{1.256112in}{1.540943in}}%
\pgfpathlineto{\pgfqpoint{1.262954in}{1.599323in}}%
\pgfpathlineto{\pgfqpoint{1.264666in}{1.522163in}}%
\pgfpathlineto{\pgfqpoint{1.269800in}{1.730783in}}%
\pgfpathlineto{\pgfqpoint{1.274080in}{1.587542in}}%
\pgfpathlineto{\pgfqpoint{1.277502in}{1.548874in}}%
\pgfpathlineto{\pgfqpoint{1.284344in}{1.739766in}}%
\pgfpathlineto{\pgfqpoint{1.286057in}{1.610407in}}%
\pgfpathlineto{\pgfqpoint{1.289480in}{1.562058in}}%
\pgfpathlineto{\pgfqpoint{1.297179in}{1.694070in}}%
\pgfpathlineto{\pgfqpoint{1.298889in}{1.618922in}}%
\pgfpathlineto{\pgfqpoint{1.302314in}{1.726060in}}%
\pgfpathlineto{\pgfqpoint{1.309157in}{1.629070in}}%
\pgfpathlineto{\pgfqpoint{1.312579in}{1.717369in}}%
\pgfpathlineto{\pgfqpoint{1.317712in}{1.618279in}}%
\pgfpathlineto{\pgfqpoint{1.321985in}{1.715678in}}%
\pgfpathlineto{\pgfqpoint{1.324548in}{1.687158in}}%
\pgfpathlineto{\pgfqpoint{1.331388in}{1.692408in}}%
\pgfpathlineto{\pgfqpoint{1.333951in}{1.537210in}}%
\pgfpathlineto{\pgfqpoint{1.339938in}{1.619795in}}%
\pgfpathlineto{\pgfqpoint{1.343354in}{1.535577in}}%
\pgfpathlineto{\pgfqpoint{1.345921in}{1.609588in}}%
\pgfpathlineto{\pgfqpoint{1.350199in}{1.549575in}}%
\pgfpathlineto{\pgfqpoint{1.356186in}{1.629651in}}%
\pgfpathlineto{\pgfqpoint{1.358754in}{1.557331in}}%
\pgfpathlineto{\pgfqpoint{1.365593in}{1.632976in}}%
\pgfpathlineto{\pgfqpoint{1.369869in}{1.502100in}}%
\pgfpathlineto{\pgfqpoint{1.371579in}{1.568239in}}%
\pgfpathlineto{\pgfqpoint{1.378419in}{1.617987in}}%
\pgfpathlineto{\pgfqpoint{1.382696in}{1.510440in}}%
\pgfpathlineto{\pgfqpoint{1.383549in}{1.581361in}}%
\pgfpathlineto{\pgfqpoint{1.387825in}{1.689142in}}%
\pgfpathlineto{\pgfqpoint{1.392960in}{1.650415in}}%
\pgfpathlineto{\pgfqpoint{1.396386in}{1.710195in}}%
\pgfpathlineto{\pgfqpoint{1.400665in}{1.665404in}}%
\pgfpathlineto{\pgfqpoint{1.404941in}{1.722239in}}%
\pgfpathlineto{\pgfqpoint{1.410071in}{1.632044in}}%
\pgfpathlineto{\pgfqpoint{1.416052in}{1.616588in}}%
\pgfpathlineto{\pgfqpoint{1.417764in}{1.747931in}}%
\pgfpathlineto{\pgfqpoint{1.424607in}{1.633443in}}%
\pgfpathlineto{\pgfqpoint{1.426320in}{1.661758in}}%
\pgfpathlineto{\pgfqpoint{1.433162in}{1.599995in}}%
\pgfpathlineto{\pgfqpoint{1.435727in}{1.686314in}}%
\pgfpathlineto{\pgfqpoint{1.439148in}{1.617465in}}%
\pgfpathlineto{\pgfqpoint{1.445989in}{1.705124in}}%
\pgfpathlineto{\pgfqpoint{1.447700in}{1.624084in}}%
\pgfpathlineto{\pgfqpoint{1.454547in}{1.676575in}}%
\pgfpathlineto{\pgfqpoint{1.457115in}{1.610670in}}%
\pgfpathlineto{\pgfqpoint{1.460537in}{1.665057in}}%
\pgfpathlineto{\pgfqpoint{1.464817in}{1.622714in}}%
\pgfpathlineto{\pgfqpoint{1.469955in}{1.730261in}}%
\pgfpathlineto{\pgfqpoint{1.473375in}{1.618922in}}%
\pgfpathlineto{\pgfqpoint{1.478513in}{1.520822in}}%
\pgfpathlineto{\pgfqpoint{1.485359in}{1.527851in}}%
\pgfpathlineto{\pgfqpoint{1.489636in}{1.629567in}}%
\pgfpathlineto{\pgfqpoint{1.490491in}{1.509567in}}%
\pgfpathlineto{\pgfqpoint{1.498186in}{1.649192in}}%
\pgfpathlineto{\pgfqpoint{1.499039in}{1.531526in}}%
\pgfpathlineto{\pgfqpoint{1.503315in}{1.663424in}}%
\pgfpathlineto{\pgfqpoint{1.509305in}{1.524380in}}%
\pgfpathlineto{\pgfqpoint{1.515290in}{1.629070in}}%
\pgfpathlineto{\pgfqpoint{1.516145in}{1.549634in}}%
\pgfpathlineto{\pgfqpoint{1.521272in}{1.679110in}}%
\pgfpathlineto{\pgfqpoint{1.527256in}{1.500350in}}%
\pgfpathlineto{\pgfqpoint{1.528967in}{1.616705in}}%
\pgfpathlineto{\pgfqpoint{1.535810in}{1.601307in}}%
\pgfpathlineto{\pgfqpoint{1.539230in}{1.533652in}}%
\pgfpathlineto{\pgfqpoint{1.545218in}{1.634901in}}%
\pgfpathlineto{\pgfqpoint{1.546074in}{1.545754in}}%
\pgfpathlineto{\pgfqpoint{1.552920in}{1.627901in}}%
\pgfpathlineto{\pgfqpoint{1.554633in}{1.535577in}}%
\pgfpathlineto{\pgfqpoint{1.561475in}{1.633209in}}%
\pgfpathlineto{\pgfqpoint{1.563184in}{1.527178in}}%
\pgfpathlineto{\pgfqpoint{1.568314in}{1.665462in}}%
\pgfpathlineto{\pgfqpoint{1.573455in}{1.552841in}}%
\pgfpathlineto{\pgfqpoint{1.577737in}{1.621662in}}%
\pgfpathlineto{\pgfqpoint{1.582014in}{1.532019in}}%
\pgfpathlineto{\pgfqpoint{1.587148in}{1.626444in}}%
\pgfpathlineto{\pgfqpoint{1.592280in}{1.516446in}}%
\pgfpathlineto{\pgfqpoint{1.596559in}{1.616354in}}%
\pgfpathlineto{\pgfqpoint{1.598270in}{1.508310in}}%
\pgfpathlineto{\pgfqpoint{1.605969in}{1.651347in}}%
\pgfpathlineto{\pgfqpoint{1.611957in}{1.534817in}}%
\pgfpathlineto{\pgfqpoint{1.617087in}{1.646419in}}%
\pgfpathlineto{\pgfqpoint{1.620507in}{1.579611in}}%
\pgfpathlineto{\pgfqpoint{1.623070in}{1.604748in}}%
\pgfpathlineto{\pgfqpoint{1.630773in}{1.553046in}}%
\pgfpathlineto{\pgfqpoint{1.635054in}{1.646273in}}%
\pgfpathlineto{\pgfqpoint{1.635909in}{1.561298in}}%
\pgfpathlineto{\pgfqpoint{1.643599in}{1.653155in}}%
\pgfpathlineto{\pgfqpoint{1.645308in}{1.567684in}}%
\pgfpathlineto{\pgfqpoint{1.649584in}{1.596320in}}%
\pgfpathlineto{\pgfqpoint{1.655572in}{1.521637in}}%
\pgfpathlineto{\pgfqpoint{1.660704in}{1.610056in}}%
\pgfpathlineto{\pgfqpoint{1.663267in}{1.560422in}}%
\pgfpathlineto{\pgfqpoint{1.671820in}{1.683892in}}%
\pgfpathlineto{\pgfqpoint{1.674387in}{1.534119in}}%
\pgfpathlineto{\pgfqpoint{1.680376in}{1.484572in}}%
\pgfpathlineto{\pgfqpoint{1.683795in}{1.622535in}}%
\pgfpathlineto{\pgfqpoint{1.692344in}{1.489793in}}%
\pgfpathlineto{\pgfqpoint{1.695764in}{1.611280in}}%
\pgfpathlineto{\pgfqpoint{1.701749in}{1.772136in}}%
\pgfpathlineto{\pgfqpoint{1.705169in}{1.640559in}}%
\pgfpathlineto{\pgfqpoint{1.712012in}{1.694274in}}%
\pgfpathlineto{\pgfqpoint{1.716288in}{1.611514in}}%
\pgfpathlineto{\pgfqpoint{1.717143in}{1.675260in}}%
\pgfpathlineto{\pgfqpoint{1.722276in}{1.715097in}}%
\pgfpathlineto{\pgfqpoint{1.725698in}{1.600551in}}%
\pgfpathlineto{\pgfqpoint{1.730826in}{1.519131in}}%
\pgfpathlineto{\pgfqpoint{1.735959in}{1.623646in}}%
\pgfpathlineto{\pgfqpoint{1.738528in}{1.568064in}}%
\pgfpathlineto{\pgfqpoint{1.742805in}{1.624811in}}%
\pgfpathlineto{\pgfqpoint{1.747088in}{1.558964in}}%
\pgfpathlineto{\pgfqpoint{1.751369in}{1.518838in}}%
\pgfpathlineto{\pgfqpoint{1.756502in}{1.610290in}}%
\pgfpathlineto{\pgfqpoint{1.759920in}{1.539252in}}%
\pgfpathlineto{\pgfqpoint{1.765902in}{1.614663in}}%
\pgfpathlineto{\pgfqpoint{1.771888in}{1.532545in}}%
\pgfpathlineto{\pgfqpoint{1.776162in}{1.620672in}}%
\pgfpathlineto{\pgfqpoint{1.777874in}{1.527003in}}%
\pgfpathlineto{\pgfqpoint{1.783002in}{1.591276in}}%
\pgfpathlineto{\pgfqpoint{1.786423in}{1.524961in}}%
\pgfpathlineto{\pgfqpoint{1.789847in}{1.633092in}}%
\pgfpathlineto{\pgfqpoint{1.794127in}{1.538843in}}%
\pgfpathlineto{\pgfqpoint{1.800974in}{1.519014in}}%
\pgfpathlineto{\pgfqpoint{1.806101in}{1.614604in}}%
\pgfpathlineto{\pgfqpoint{1.807814in}{1.513764in}}%
\pgfpathlineto{\pgfqpoint{1.813805in}{1.675202in}}%
\pgfpathlineto{\pgfqpoint{1.817226in}{1.562697in}}%
\pgfpathlineto{\pgfqpoint{1.822358in}{1.686636in}}%
\pgfpathlineto{\pgfqpoint{1.824068in}{1.612712in}}%
\pgfpathlineto{\pgfqpoint{1.831766in}{1.648782in}}%
\pgfpathlineto{\pgfqpoint{1.833476in}{1.558179in}}%
\pgfpathlineto{\pgfqpoint{1.836895in}{1.640267in}}%
\pgfpathlineto{\pgfqpoint{1.842883in}{1.545725in}}%
\pgfpathlineto{\pgfqpoint{1.848875in}{1.685350in}}%
\pgfpathlineto{\pgfqpoint{1.849731in}{1.567567in}}%
\pgfpathlineto{\pgfqpoint{1.854859in}{1.619445in}}%
\pgfpathlineto{\pgfqpoint{1.859990in}{1.529162in}}%
\pgfpathlineto{\pgfqpoint{1.864266in}{1.490436in}}%
\pgfpathlineto{\pgfqpoint{1.868544in}{1.628194in}}%
\pgfpathlineto{\pgfqpoint{1.872817in}{1.640384in}}%
\pgfpathlineto{\pgfqpoint{1.877090in}{1.539486in}}%
\pgfpathlineto{\pgfqpoint{1.879659in}{1.594366in}}%
\pgfpathlineto{\pgfqpoint{1.883937in}{1.544268in}}%
\pgfpathlineto{\pgfqpoint{1.888217in}{1.594717in}}%
\pgfpathlineto{\pgfqpoint{1.895915in}{1.510148in}}%
\pgfpathlineto{\pgfqpoint{1.896771in}{1.590340in}}%
\pgfpathlineto{\pgfqpoint{1.901907in}{1.605913in}}%
\pgfpathlineto{\pgfqpoint{1.907894in}{1.530269in}}%
\pgfpathlineto{\pgfqpoint{1.912175in}{1.508573in}}%
\pgfpathlineto{\pgfqpoint{1.913882in}{1.607605in}}%
\pgfpathlineto{\pgfqpoint{1.918162in}{1.528548in}}%
\pgfpathlineto{\pgfqpoint{1.922443in}{1.636767in}}%
\pgfpathlineto{\pgfqpoint{1.926721in}{1.526072in}}%
\pgfpathlineto{\pgfqpoint{1.932706in}{1.500993in}}%
\pgfpathlineto{\pgfqpoint{1.935269in}{1.617581in}}%
\pgfpathlineto{\pgfqpoint{1.940403in}{1.716613in}}%
\pgfpathlineto{\pgfqpoint{1.947248in}{1.607024in}}%
\pgfpathlineto{\pgfqpoint{1.950671in}{1.719236in}}%
\pgfpathlineto{\pgfqpoint{1.955805in}{1.564184in}}%
\pgfpathlineto{\pgfqpoint{1.960085in}{1.546368in}}%
\pgfpathlineto{\pgfqpoint{1.961796in}{1.620321in}}%
\pgfpathlineto{\pgfqpoint{1.967788in}{1.663654in}}%
\pgfpathlineto{\pgfqpoint{1.970354in}{1.526188in}}%
\pgfpathlineto{\pgfqpoint{1.976334in}{1.645224in}}%
\pgfpathlineto{\pgfqpoint{1.979754in}{1.525023in}}%
\pgfpathlineto{\pgfqpoint{1.985740in}{1.629655in}}%
\pgfpathlineto{\pgfqpoint{1.986595in}{1.553777in}}%
\pgfpathlineto{\pgfqpoint{1.993441in}{1.537707in}}%
\pgfpathlineto{\pgfqpoint{1.995151in}{1.647500in}}%
\pgfpathlineto{\pgfqpoint{2.001137in}{1.518082in}}%
\pgfpathlineto{\pgfqpoint{2.004557in}{1.605099in}}%
\pgfpathlineto{\pgfqpoint{2.010548in}{1.566723in}}%
\pgfpathlineto{\pgfqpoint{2.014825in}{1.650708in}}%
\pgfpathlineto{\pgfqpoint{2.016537in}{1.566489in}}%
\pgfpathlineto{\pgfqpoint{2.021667in}{1.650445in}}%
\pgfpathlineto{\pgfqpoint{2.027657in}{1.671877in}}%
\pgfpathlineto{\pgfqpoint{2.031077in}{1.543537in}}%
\pgfpathlineto{\pgfqpoint{2.035355in}{1.654847in}}%
\pgfpathlineto{\pgfqpoint{2.039628in}{1.555581in}}%
\pgfpathlineto{\pgfqpoint{2.043906in}{1.635661in}}%
\pgfpathlineto{\pgfqpoint{2.047328in}{1.531058in}}%
\pgfpathlineto{\pgfqpoint{2.054173in}{1.525662in}}%
\pgfpathlineto{\pgfqpoint{2.055027in}{1.633735in}}%
\pgfpathlineto{\pgfqpoint{2.060159in}{1.507466in}}%
\pgfpathlineto{\pgfqpoint{2.065290in}{1.621837in}}%
\pgfpathlineto{\pgfqpoint{2.067854in}{1.578504in}}%
\pgfpathlineto{\pgfqpoint{2.075551in}{1.509918in}}%
\pgfpathlineto{\pgfqpoint{2.077262in}{1.664239in}}%
\pgfpathlineto{\pgfqpoint{2.081538in}{1.578095in}}%
\pgfpathlineto{\pgfqpoint{2.084956in}{1.607956in}}%
\pgfpathlineto{\pgfqpoint{2.090086in}{1.524029in}}%
\pgfpathlineto{\pgfqpoint{2.093509in}{1.598684in}}%
\pgfpathlineto{\pgfqpoint{2.097788in}{1.519131in}}%
\pgfpathlineto{\pgfqpoint{2.102927in}{1.624402in}}%
\pgfpathlineto{\pgfqpoint{2.106351in}{1.542109in}}%
\pgfpathlineto{\pgfqpoint{2.114054in}{1.512248in}}%
\pgfpathlineto{\pgfqpoint{2.114910in}{1.620727in}}%
\pgfpathlineto{\pgfqpoint{2.122605in}{1.663946in}}%
\pgfpathlineto{\pgfqpoint{2.124318in}{1.588243in}}%
\pgfpathlineto{\pgfqpoint{2.130310in}{1.557740in}}%
\pgfpathlineto{\pgfqpoint{2.132878in}{1.648373in}}%
\pgfpathlineto{\pgfqpoint{2.138864in}{1.515719in}}%
\pgfpathlineto{\pgfqpoint{2.140576in}{1.599031in}}%
\pgfpathlineto{\pgfqpoint{2.145701in}{1.552257in}}%
\pgfpathlineto{\pgfqpoint{2.151687in}{1.762452in}}%
\pgfpathlineto{\pgfqpoint{2.153397in}{1.633823in}}%
\pgfpathlineto{\pgfqpoint{2.158523in}{1.725709in}}%
\pgfpathlineto{\pgfqpoint{2.164513in}{1.642075in}}%
\pgfpathlineto{\pgfqpoint{2.169647in}{1.646945in}}%
\pgfpathlineto{\pgfqpoint{2.173068in}{1.753297in}}%
\pgfpathlineto{\pgfqpoint{2.176493in}{1.561587in}}%
\pgfpathlineto{\pgfqpoint{2.181627in}{1.683542in}}%
\pgfpathlineto{\pgfqpoint{2.184193in}{1.617812in}}%
\pgfpathlineto{\pgfqpoint{2.189327in}{1.690833in}}%
\pgfpathlineto{\pgfqpoint{2.193602in}{1.635368in}}%
\pgfpathlineto{\pgfqpoint{2.197026in}{1.685642in}}%
\pgfpathlineto{\pgfqpoint{2.203016in}{1.582526in}}%
\pgfpathlineto{\pgfqpoint{2.205581in}{1.681675in}}%
\pgfpathlineto{\pgfqpoint{2.212427in}{1.697190in}}%
\pgfpathlineto{\pgfqpoint{2.215846in}{1.582468in}}%
\pgfpathlineto{\pgfqpoint{2.218411in}{1.701040in}}%
\pgfpathlineto{\pgfqpoint{2.222686in}{1.598976in}}%
\pgfpathlineto{\pgfqpoint{2.226962in}{1.709146in}}%
\pgfpathlineto{\pgfqpoint{2.232095in}{1.635719in}}%
\pgfpathlineto{\pgfqpoint{2.234663in}{1.700806in}}%
\pgfpathlineto{\pgfqpoint{2.241504in}{1.618279in}}%
\pgfpathlineto{\pgfqpoint{2.243215in}{1.705413in}}%
\pgfpathlineto{\pgfqpoint{2.250053in}{1.649539in}}%
\pgfpathlineto{\pgfqpoint{2.255182in}{1.735741in}}%
\pgfpathlineto{\pgfqpoint{2.258601in}{1.748629in}}%
\pgfpathlineto{\pgfqpoint{2.261169in}{1.634813in}}%
\pgfpathlineto{\pgfqpoint{2.264591in}{1.668670in}}%
\pgfpathlineto{\pgfqpoint{2.268867in}{1.510557in}}%
\pgfpathlineto{\pgfqpoint{2.273147in}{1.530795in}}%
\pgfpathlineto{\pgfqpoint{2.278277in}{1.613669in}}%
\pgfpathlineto{\pgfqpoint{2.282560in}{1.540300in}}%
\pgfpathlineto{\pgfqpoint{2.286838in}{1.601015in}}%
\pgfpathlineto{\pgfqpoint{2.290262in}{1.534236in}}%
\pgfpathlineto{\pgfqpoint{2.295399in}{1.639507in}}%
\pgfpathlineto{\pgfqpoint{2.298819in}{1.553919in}}%
\pgfpathlineto{\pgfqpoint{2.309087in}{1.658580in}}%
\pgfpathlineto{\pgfqpoint{2.311649in}{1.592211in}}%
\pgfpathlineto{\pgfqpoint{2.318493in}{1.627960in}}%
\pgfpathlineto{\pgfqpoint{2.321912in}{1.493351in}}%
\pgfpathlineto{\pgfqpoint{2.326189in}{1.578095in}}%
\pgfpathlineto{\pgfqpoint{2.330467in}{1.587074in}}%
\pgfpathlineto{\pgfqpoint{2.333891in}{1.516796in}}%
\pgfpathlineto{\pgfqpoint{2.339879in}{1.502392in}}%
\pgfpathlineto{\pgfqpoint{2.341587in}{1.622798in}}%
\pgfpathlineto{\pgfqpoint{2.348427in}{1.558379in}}%
\pgfpathlineto{\pgfqpoint{2.350137in}{1.639273in}}%
\pgfpathlineto{\pgfqpoint{2.356126in}{1.589409in}}%
\pgfpathlineto{\pgfqpoint{2.363830in}{1.541586in}}%
\pgfpathlineto{\pgfqpoint{2.370673in}{1.723905in}}%
\pgfpathlineto{\pgfqpoint{2.374095in}{1.595330in}}%
\pgfpathlineto{\pgfqpoint{2.375807in}{1.664122in}}%
\pgfpathlineto{\pgfqpoint{2.383504in}{1.680363in}}%
\pgfpathlineto{\pgfqpoint{2.386924in}{1.621720in}}%
\pgfpathlineto{\pgfqpoint{2.392057in}{1.637235in}}%
\pgfpathlineto{\pgfqpoint{2.396335in}{1.513180in}}%
\pgfpathlineto{\pgfqpoint{2.397190in}{1.634375in}}%
\pgfpathlineto{\pgfqpoint{2.402321in}{1.647208in}}%
\pgfpathlineto{\pgfqpoint{2.407451in}{1.534324in}}%
\pgfpathlineto{\pgfqpoint{2.411726in}{1.522250in}}%
\pgfpathlineto{\pgfqpoint{2.417709in}{1.508807in}}%
\pgfpathlineto{\pgfqpoint{2.418564in}{1.608131in}}%
\pgfpathlineto{\pgfqpoint{2.423696in}{1.525370in}}%
\pgfpathlineto{\pgfqpoint{2.427115in}{1.595765in}}%
\pgfpathlineto{\pgfqpoint{2.433100in}{1.582526in}}%
\pgfpathlineto{\pgfqpoint{2.436521in}{1.643708in}}%
\pgfpathlineto{\pgfqpoint{2.441651in}{1.684477in}}%
\pgfpathlineto{\pgfqpoint{2.445075in}{1.554387in}}%
\pgfpathlineto{\pgfqpoint{2.451058in}{1.546017in}}%
\pgfpathlineto{\pgfqpoint{2.452770in}{1.618104in}}%
\pgfpathlineto{\pgfqpoint{2.457908in}{1.645107in}}%
\pgfpathlineto{\pgfqpoint{2.463895in}{1.600142in}}%
\pgfpathlineto{\pgfqpoint{2.465606in}{1.668670in}}%
\pgfpathlineto{\pgfqpoint{2.470738in}{1.560279in}}%
\pgfpathlineto{\pgfqpoint{2.474162in}{1.643708in}}%
\pgfpathlineto{\pgfqpoint{2.479295in}{1.568005in}}%
\pgfpathlineto{\pgfqpoint{2.486139in}{1.598626in}}%
\pgfpathlineto{\pgfqpoint{2.489559in}{1.507729in}}%
\pgfpathlineto{\pgfqpoint{2.494692in}{1.613906in}}%
\pgfpathlineto{\pgfqpoint{2.496403in}{1.504522in}}%
\pgfpathlineto{\pgfqpoint{2.502393in}{1.653564in}}%
\pgfpathlineto{\pgfqpoint{2.506673in}{1.647968in}}%
\pgfpathlineto{\pgfqpoint{2.508382in}{1.546485in}}%
\pgfpathlineto{\pgfqpoint{2.516081in}{1.623762in}}%
\pgfpathlineto{\pgfqpoint{2.518648in}{1.536337in}}%
\pgfpathlineto{\pgfqpoint{2.522070in}{1.603641in}}%
\pgfpathlineto{\pgfqpoint{2.528054in}{1.514583in}}%
\pgfpathlineto{\pgfqpoint{2.530622in}{1.584514in}}%
\pgfpathlineto{\pgfqpoint{2.535758in}{1.638810in}}%
\pgfpathlineto{\pgfqpoint{2.538324in}{1.531730in}}%
\pgfpathlineto{\pgfqpoint{2.543453in}{1.653798in}}%
\pgfpathlineto{\pgfqpoint{2.549441in}{1.526013in}}%
\pgfpathlineto{\pgfqpoint{2.553715in}{1.628018in}}%
\pgfpathlineto{\pgfqpoint{2.556277in}{1.547709in}}%
\pgfpathlineto{\pgfqpoint{2.563119in}{1.626795in}}%
\pgfpathlineto{\pgfqpoint{2.565684in}{1.517030in}}%
\pgfpathlineto{\pgfqpoint{2.569960in}{1.681269in}}%
\pgfpathlineto{\pgfqpoint{2.572527in}{1.615598in}}%
\pgfpathlineto{\pgfqpoint{2.579370in}{1.703667in}}%
\pgfpathlineto{\pgfqpoint{2.581931in}{1.607550in}}%
\pgfpathlineto{\pgfqpoint{2.587061in}{1.688152in}}%
\pgfpathlineto{\pgfqpoint{2.593048in}{1.638897in}}%
\pgfpathlineto{\pgfqpoint{2.595611in}{1.703082in}}%
\pgfpathlineto{\pgfqpoint{2.601596in}{1.594603in}}%
\pgfpathlineto{\pgfqpoint{2.604164in}{1.674913in}}%
\pgfpathlineto{\pgfqpoint{2.609301in}{1.617728in}}%
\pgfpathlineto{\pgfqpoint{2.614436in}{1.710954in}}%
\pgfpathlineto{\pgfqpoint{2.617000in}{1.622597in}}%
\pgfpathlineto{\pgfqpoint{2.619567in}{1.712938in}}%
\pgfpathlineto{\pgfqpoint{2.624695in}{1.682552in}}%
\pgfpathlineto{\pgfqpoint{2.628975in}{1.753473in}}%
\pgfpathlineto{\pgfqpoint{2.632399in}{1.667388in}}%
\pgfpathlineto{\pgfqpoint{2.639241in}{1.693924in}}%
\pgfpathlineto{\pgfqpoint{2.641809in}{1.618279in}}%
\pgfpathlineto{\pgfqpoint{2.646085in}{1.544034in}}%
\pgfpathlineto{\pgfqpoint{2.649509in}{1.532662in}}%
\pgfpathlineto{\pgfqpoint{2.653787in}{1.658814in}}%
\pgfpathlineto{\pgfqpoint{2.658066in}{1.709263in}}%
\pgfpathlineto{\pgfqpoint{2.662344in}{1.603232in}}%
\pgfpathlineto{\pgfqpoint{2.670044in}{1.563048in}}%
\pgfpathlineto{\pgfqpoint{2.672610in}{1.632102in}}%
\pgfpathlineto{\pgfqpoint{2.677741in}{1.523039in}}%
\pgfpathlineto{\pgfqpoint{2.681163in}{1.635427in}}%
\pgfpathlineto{\pgfqpoint{2.684586in}{1.527967in}}%
\pgfpathlineto{\pgfqpoint{2.693143in}{1.714337in}}%
\pgfpathlineto{\pgfqpoint{2.696565in}{1.561415in}}%
\pgfpathlineto{\pgfqpoint{2.702555in}{1.600609in}}%
\pgfpathlineto{\pgfqpoint{2.708544in}{1.530912in}}%
\pgfpathlineto{\pgfqpoint{2.709399in}{1.621720in}}%
\pgfpathlineto{\pgfqpoint{2.714525in}{1.535811in}}%
\pgfpathlineto{\pgfqpoint{2.721365in}{1.653740in}}%
\pgfpathlineto{\pgfqpoint{2.723934in}{1.549634in}}%
\pgfpathlineto{\pgfqpoint{2.726501in}{1.616997in}}%
\pgfpathlineto{\pgfqpoint{2.731635in}{1.622159in}}%
\pgfpathlineto{\pgfqpoint{2.737626in}{1.495919in}}%
\pgfpathlineto{\pgfqpoint{2.739338in}{1.585968in}}%
\pgfpathlineto{\pgfqpoint{2.745323in}{1.525019in}}%
\pgfpathlineto{\pgfqpoint{2.747889in}{1.617578in}}%
\pgfpathlineto{\pgfqpoint{2.755581in}{1.547738in}}%
\pgfpathlineto{\pgfqpoint{2.756436in}{1.615832in}}%
\pgfpathlineto{\pgfqpoint{2.762417in}{1.532749in}}%
\pgfpathlineto{\pgfqpoint{2.764980in}{1.610465in}}%
\pgfpathlineto{\pgfqpoint{2.770971in}{1.523098in}}%
\pgfpathlineto{\pgfqpoint{2.776103in}{1.612098in}}%
\pgfpathlineto{\pgfqpoint{2.777811in}{1.565266in}}%
\pgfpathlineto{\pgfqpoint{2.782945in}{1.630703in}}%
\pgfpathlineto{\pgfqpoint{2.789789in}{1.558149in}}%
\pgfpathlineto{\pgfqpoint{2.790642in}{1.599703in}}%
\pgfpathlineto{\pgfqpoint{2.794918in}{1.536684in}}%
\pgfpathlineto{\pgfqpoint{2.799189in}{1.635076in}}%
\pgfpathlineto{\pgfqpoint{2.803464in}{1.564097in}}%
\pgfpathlineto{\pgfqpoint{2.809450in}{1.615072in}}%
\pgfpathlineto{\pgfqpoint{2.812873in}{1.669839in}}%
\pgfpathlineto{\pgfqpoint{2.817150in}{1.508398in}}%
\pgfpathlineto{\pgfqpoint{2.820569in}{1.624928in}}%
\pgfpathlineto{\pgfqpoint{2.827409in}{1.610465in}}%
\pgfpathlineto{\pgfqpoint{2.829973in}{1.686168in}}%
\pgfpathlineto{\pgfqpoint{2.833396in}{1.617523in}}%
\pgfpathlineto{\pgfqpoint{2.839382in}{1.710081in}}%
\pgfpathlineto{\pgfqpoint{2.844515in}{1.647880in}}%
\pgfpathlineto{\pgfqpoint{2.849642in}{1.707630in}}%
\pgfpathlineto{\pgfqpoint{2.850497in}{1.536278in}}%
\pgfpathlineto{\pgfqpoint{2.855627in}{1.637060in}}%
\pgfpathlineto{\pgfqpoint{2.859900in}{1.463140in}}%
\pgfpathlineto{\pgfqpoint{2.865887in}{1.641348in}}%
\pgfpathlineto{\pgfqpoint{2.867599in}{1.521348in}}%
\pgfpathlineto{\pgfqpoint{2.874440in}{1.510440in}}%
\pgfpathlineto{\pgfqpoint{2.876150in}{1.688674in}}%
\pgfpathlineto{\pgfqpoint{2.880425in}{1.530561in}}%
\pgfpathlineto{\pgfqpoint{2.885558in}{1.672754in}}%
\pgfpathlineto{\pgfqpoint{2.889835in}{1.530269in}}%
\pgfpathlineto{\pgfqpoint{2.893259in}{1.609296in}}%
\pgfpathlineto{\pgfqpoint{2.900103in}{1.564798in}}%
\pgfpathlineto{\pgfqpoint{2.905233in}{1.724076in}}%
\pgfpathlineto{\pgfqpoint{2.907800in}{1.665579in}}%
\pgfpathlineto{\pgfqpoint{2.911219in}{1.714396in}}%
\pgfpathlineto{\pgfqpoint{2.918060in}{1.636943in}}%
\pgfpathlineto{\pgfqpoint{2.921481in}{1.690428in}}%
\pgfpathlineto{\pgfqpoint{2.924043in}{1.543336in}}%
\pgfpathlineto{\pgfqpoint{2.930027in}{1.644760in}}%
\pgfpathlineto{\pgfqpoint{2.932595in}{1.528289in}}%
\pgfpathlineto{\pgfqpoint{2.936017in}{1.621140in}}%
\pgfpathlineto{\pgfqpoint{2.941150in}{1.534558in}}%
\pgfpathlineto{\pgfqpoint{2.944568in}{1.630937in}}%
\pgfpathlineto{\pgfqpoint{2.948846in}{1.563282in}}%
\pgfpathlineto{\pgfqpoint{2.954831in}{1.526130in}}%
\pgfpathlineto{\pgfqpoint{2.963385in}{1.667388in}}%
\pgfpathlineto{\pgfqpoint{2.965950in}{1.608715in}}%
\pgfpathlineto{\pgfqpoint{2.972797in}{1.584978in}}%
\pgfpathlineto{\pgfqpoint{2.977928in}{1.638108in}}%
\pgfpathlineto{\pgfqpoint{2.979640in}{1.572846in}}%
\pgfpathlineto{\pgfqpoint{2.983917in}{1.681328in}}%
\pgfpathlineto{\pgfqpoint{2.989050in}{1.635953in}}%
\pgfpathlineto{\pgfqpoint{2.991615in}{1.694099in}}%
\pgfpathlineto{\pgfqpoint{2.997600in}{1.638722in}}%
\pgfpathlineto{\pgfqpoint{3.000167in}{1.715649in}}%
\pgfpathlineto{\pgfqpoint{3.004446in}{1.605099in}}%
\pgfpathlineto{\pgfqpoint{3.008722in}{1.759888in}}%
\pgfpathlineto{\pgfqpoint{3.012997in}{1.605972in}}%
\pgfpathlineto{\pgfqpoint{3.018130in}{1.702848in}}%
\pgfpathlineto{\pgfqpoint{3.024118in}{1.633151in}}%
\pgfpathlineto{\pgfqpoint{3.026684in}{1.713928in}}%
\pgfpathlineto{\pgfqpoint{3.030958in}{1.753820in}}%
\pgfpathlineto{\pgfqpoint{3.034379in}{1.635310in}}%
\pgfpathlineto{\pgfqpoint{3.041219in}{1.621837in}}%
\pgfpathlineto{\pgfqpoint{3.046352in}{1.752132in}}%
\pgfpathlineto{\pgfqpoint{3.048063in}{1.665872in}}%
\pgfpathlineto{\pgfqpoint{3.054053in}{1.716145in}}%
\pgfpathlineto{\pgfqpoint{3.058329in}{1.584919in}}%
\pgfpathlineto{\pgfqpoint{3.062609in}{1.538667in}}%
\pgfpathlineto{\pgfqpoint{3.064319in}{1.661554in}}%
\pgfpathlineto{\pgfqpoint{3.069452in}{1.569404in}}%
\pgfpathlineto{\pgfqpoint{3.076293in}{1.663654in}}%
\pgfpathlineto{\pgfqpoint{3.078004in}{1.572963in}}%
\pgfpathlineto{\pgfqpoint{3.083138in}{1.587893in}}%
\pgfpathlineto{\pgfqpoint{3.089122in}{1.509943in}}%
\pgfpathlineto{\pgfqpoint{3.092546in}{1.618396in}}%
\pgfpathlineto{\pgfqpoint{3.095968in}{1.544677in}}%
\pgfpathlineto{\pgfqpoint{3.101097in}{1.608862in}}%
\pgfpathlineto{\pgfqpoint{3.103662in}{1.521377in}}%
\pgfpathlineto{\pgfqpoint{3.109653in}{1.635602in}}%
\pgfpathlineto{\pgfqpoint{3.113076in}{1.541440in}}%
\pgfpathlineto{\pgfqpoint{3.118215in}{1.754755in}}%
\pgfpathlineto{\pgfqpoint{3.122495in}{1.625366in}}%
\pgfpathlineto{\pgfqpoint{3.124207in}{1.698125in}}%
\pgfpathlineto{\pgfqpoint{3.129335in}{1.650912in}}%
\pgfpathlineto{\pgfqpoint{3.135323in}{1.791556in}}%
\pgfpathlineto{\pgfqpoint{3.138744in}{1.664502in}}%
\pgfpathlineto{\pgfqpoint{3.143874in}{1.709263in}}%
\pgfpathlineto{\pgfqpoint{3.145581in}{1.587659in}}%
\pgfpathlineto{\pgfqpoint{3.149859in}{1.566022in}}%
\pgfpathlineto{\pgfqpoint{3.155843in}{1.541528in}}%
\pgfpathlineto{\pgfqpoint{3.159266in}{1.646624in}}%
\pgfpathlineto{\pgfqpoint{3.164395in}{1.533885in}}%
\pgfpathlineto{\pgfqpoint{3.168665in}{1.643825in}}%
\pgfpathlineto{\pgfqpoint{3.172941in}{1.553831in}}%
\pgfpathlineto{\pgfqpoint{3.178926in}{1.675815in}}%
\pgfpathlineto{\pgfqpoint{3.179782in}{1.551092in}}%
\pgfpathlineto{\pgfqpoint{3.186627in}{1.532194in}}%
\pgfpathlineto{\pgfqpoint{3.189194in}{1.619795in}}%
\pgfpathlineto{\pgfqpoint{3.196038in}{1.499766in}}%
\pgfpathlineto{\pgfqpoint{3.200317in}{1.646331in}}%
\pgfpathlineto{\pgfqpoint{3.204595in}{1.519043in}}%
\pgfpathlineto{\pgfqpoint{3.206306in}{1.654847in}}%
\pgfpathlineto{\pgfqpoint{3.209727in}{1.521929in}}%
\pgfpathlineto{\pgfqpoint{3.214860in}{1.640559in}}%
\pgfpathlineto{\pgfqpoint{3.218280in}{1.652980in}}%
\pgfpathlineto{\pgfqpoint{3.225977in}{1.532223in}}%
\pgfpathlineto{\pgfqpoint{3.230256in}{1.690249in}}%
\pgfpathlineto{\pgfqpoint{3.232821in}{1.584539in}}%
\pgfpathlineto{\pgfqpoint{3.235387in}{1.707513in}}%
\pgfpathlineto{\pgfqpoint{3.239662in}{1.655285in}}%
\pgfpathlineto{\pgfqpoint{3.247355in}{1.719236in}}%
\pgfpathlineto{\pgfqpoint{3.251630in}{1.647003in}}%
\pgfpathlineto{\pgfqpoint{3.255054in}{1.691882in}}%
\pgfpathlineto{\pgfqpoint{3.259333in}{1.701913in}}%
\pgfpathlineto{\pgfqpoint{3.264463in}{1.735218in}}%
\pgfpathlineto{\pgfqpoint{3.268739in}{1.542342in}}%
\pgfpathlineto{\pgfqpoint{3.269594in}{1.626268in}}%
\pgfpathlineto{\pgfqpoint{3.273868in}{1.549575in}}%
\pgfpathlineto{\pgfqpoint{3.281573in}{1.644172in}}%
\pgfpathlineto{\pgfqpoint{3.284995in}{1.577277in}}%
\pgfpathlineto{\pgfqpoint{3.286705in}{1.674909in}}%
\pgfpathlineto{\pgfqpoint{3.291835in}{1.540125in}}%
\pgfpathlineto{\pgfqpoint{3.296964in}{1.633443in}}%
\pgfpathlineto{\pgfqpoint{3.299528in}{1.585149in}}%
\pgfpathlineto{\pgfqpoint{3.306369in}{1.632859in}}%
\pgfpathlineto{\pgfqpoint{3.309787in}{1.521812in}}%
\pgfpathlineto{\pgfqpoint{3.312350in}{1.641140in}}%
\pgfpathlineto{\pgfqpoint{3.320047in}{1.567772in}}%
\pgfpathlineto{\pgfqpoint{3.323472in}{1.688324in}}%
\pgfpathlineto{\pgfqpoint{3.327748in}{1.566314in}}%
\pgfpathlineto{\pgfqpoint{3.330315in}{1.650240in}}%
\pgfpathlineto{\pgfqpoint{3.337159in}{1.644757in}}%
\pgfpathlineto{\pgfqpoint{3.339724in}{1.514988in}}%
\pgfpathlineto{\pgfqpoint{3.342293in}{1.608482in}}%
\pgfpathlineto{\pgfqpoint{3.348282in}{1.539193in}}%
\pgfpathlineto{\pgfqpoint{3.351703in}{1.615886in}}%
\pgfpathlineto{\pgfqpoint{3.357690in}{1.547417in}}%
\pgfpathlineto{\pgfqpoint{3.359402in}{1.642656in}}%
\pgfpathlineto{\pgfqpoint{3.364537in}{1.575965in}}%
\pgfpathlineto{\pgfqpoint{3.368813in}{1.669777in}}%
\pgfpathlineto{\pgfqpoint{3.372234in}{1.579319in}}%
\pgfpathlineto{\pgfqpoint{3.379076in}{1.509972in}}%
\pgfpathlineto{\pgfqpoint{3.380784in}{1.617344in}}%
\pgfpathlineto{\pgfqpoint{3.385917in}{1.571414in}}%
\pgfpathlineto{\pgfqpoint{3.391052in}{1.654613in}}%
\pgfpathlineto{\pgfqpoint{3.396180in}{1.556980in}}%
\pgfpathlineto{\pgfqpoint{3.400457in}{1.637757in}}%
\pgfpathlineto{\pgfqpoint{3.405589in}{1.592207in}}%
\pgfpathlineto{\pgfqpoint{3.406443in}{1.644348in}}%
\pgfpathlineto{\pgfqpoint{3.414139in}{1.562347in}}%
\pgfpathlineto{\pgfqpoint{3.418410in}{1.655866in}}%
\pgfpathlineto{\pgfqpoint{3.420976in}{1.581971in}}%
\pgfpathlineto{\pgfqpoint{3.425249in}{1.662544in}}%
\pgfpathlineto{\pgfqpoint{3.427816in}{1.568294in}}%
\pgfpathlineto{\pgfqpoint{3.432947in}{1.504021in}}%
\pgfpathlineto{\pgfqpoint{3.437223in}{1.637378in}}%
\pgfpathlineto{\pgfqpoint{3.442356in}{1.551439in}}%
\pgfpathlineto{\pgfqpoint{3.445779in}{1.613377in}}%
\pgfpathlineto{\pgfqpoint{3.450907in}{1.684415in}}%
\pgfpathlineto{\pgfqpoint{3.456036in}{1.559198in}}%
\pgfpathlineto{\pgfqpoint{3.458605in}{1.639273in}}%
\pgfpathlineto{\pgfqpoint{3.463737in}{1.521929in}}%
\pgfpathlineto{\pgfqpoint{3.466302in}{1.648549in}}%
\pgfpathlineto{\pgfqpoint{3.472285in}{1.662953in}}%
\pgfpathlineto{\pgfqpoint{3.475705in}{1.560831in}}%
\pgfpathlineto{\pgfqpoint{3.479984in}{1.540651in}}%
\pgfpathlineto{\pgfqpoint{3.484262in}{1.609676in}}%
\pgfpathlineto{\pgfqpoint{3.491104in}{1.676601in}}%
\pgfpathlineto{\pgfqpoint{3.492814in}{1.570278in}}%
\pgfpathlineto{\pgfqpoint{3.497093in}{1.535285in}}%
\pgfpathlineto{\pgfqpoint{3.500515in}{1.625278in}}%
\pgfpathlineto{\pgfqpoint{3.504790in}{1.574712in}}%
\pgfpathlineto{\pgfqpoint{3.512483in}{1.545521in}}%
\pgfpathlineto{\pgfqpoint{3.515905in}{1.714454in}}%
\pgfpathlineto{\pgfqpoint{3.517615in}{1.584656in}}%
\pgfpathlineto{\pgfqpoint{3.524462in}{1.674095in}}%
\pgfpathlineto{\pgfqpoint{3.527029in}{1.523796in}}%
\pgfpathlineto{\pgfqpoint{3.533013in}{1.699582in}}%
\pgfpathlineto{\pgfqpoint{3.534724in}{1.637410in}}%
\pgfpathlineto{\pgfqpoint{3.539857in}{1.759595in}}%
\pgfpathlineto{\pgfqpoint{3.543277in}{1.658814in}}%
\pgfpathlineto{\pgfqpoint{3.547556in}{1.728975in}}%
\pgfpathlineto{\pgfqpoint{3.554393in}{1.612036in}}%
\pgfpathlineto{\pgfqpoint{3.556960in}{1.793712in}}%
\pgfpathlineto{\pgfqpoint{3.560375in}{1.677357in}}%
\pgfpathlineto{\pgfqpoint{3.565506in}{1.727167in}}%
\pgfpathlineto{\pgfqpoint{3.574058in}{1.638605in}}%
\pgfpathlineto{\pgfqpoint{3.578334in}{1.726381in}}%
\pgfpathlineto{\pgfqpoint{3.583469in}{1.678672in}}%
\pgfpathlineto{\pgfqpoint{3.589452in}{1.745772in}}%
\pgfpathlineto{\pgfqpoint{3.590304in}{1.682084in}}%
\pgfpathlineto{\pgfqpoint{3.596294in}{1.787005in}}%
\pgfpathlineto{\pgfqpoint{3.599713in}{1.615828in}}%
\pgfpathlineto{\pgfqpoint{3.604849in}{1.739123in}}%
\pgfpathlineto{\pgfqpoint{3.609981in}{1.762452in}}%
\pgfpathlineto{\pgfqpoint{3.611692in}{1.669981in}}%
\pgfpathlineto{\pgfqpoint{3.618535in}{1.656626in}}%
\pgfpathlineto{\pgfqpoint{3.620247in}{1.766536in}}%
\pgfpathlineto{\pgfqpoint{3.624526in}{1.656655in}}%
\pgfpathlineto{\pgfqpoint{3.631368in}{1.744607in}}%
\pgfpathlineto{\pgfqpoint{3.633075in}{1.663946in}}%
\pgfpathlineto{\pgfqpoint{3.639063in}{1.637001in}}%
\pgfpathlineto{\pgfqpoint{3.645057in}{1.731017in}}%
\pgfpathlineto{\pgfqpoint{3.647625in}{1.638342in}}%
\pgfpathlineto{\pgfqpoint{3.651902in}{1.715795in}}%
\pgfpathlineto{\pgfqpoint{3.655322in}{1.656772in}}%
\pgfpathlineto{\pgfqpoint{3.662162in}{1.789573in}}%
\pgfpathlineto{\pgfqpoint{3.666437in}{1.655022in}}%
\pgfpathlineto{\pgfqpoint{3.667291in}{1.730491in}}%
\pgfpathlineto{\pgfqpoint{3.674993in}{1.633385in}}%
\pgfpathlineto{\pgfqpoint{3.676704in}{1.750729in}}%
\pgfpathlineto{\pgfqpoint{3.680124in}{1.559081in}}%
\pgfpathlineto{\pgfqpoint{3.685255in}{1.701040in}}%
\pgfpathlineto{\pgfqpoint{3.689529in}{1.634959in}}%
\pgfpathlineto{\pgfqpoint{3.694663in}{1.674212in}}%
\pgfpathlineto{\pgfqpoint{3.699794in}{1.675435in}}%
\pgfpathlineto{\pgfqpoint{3.701506in}{1.553017in}}%
\pgfpathlineto{\pgfqpoint{3.708349in}{1.651581in}}%
\pgfpathlineto{\pgfqpoint{3.710060in}{1.550945in}}%
\pgfpathlineto{\pgfqpoint{3.714341in}{1.638400in}}%
\pgfpathlineto{\pgfqpoint{3.721187in}{1.667793in}}%
\pgfpathlineto{\pgfqpoint{3.723756in}{1.552082in}}%
\pgfpathlineto{\pgfqpoint{3.729741in}{1.636358in}}%
\pgfpathlineto{\pgfqpoint{3.731453in}{1.564999in}}%
\pgfpathlineto{\pgfqpoint{3.735725in}{1.659161in}}%
\pgfpathlineto{\pgfqpoint{3.740853in}{1.685028in}}%
\pgfpathlineto{\pgfqpoint{3.746839in}{1.577452in}}%
\pgfpathlineto{\pgfqpoint{3.751968in}{1.677708in}}%
\pgfpathlineto{\pgfqpoint{3.752824in}{1.592295in}}%
\pgfpathlineto{\pgfqpoint{3.757956in}{1.511196in}}%
\pgfpathlineto{\pgfqpoint{3.764798in}{1.566778in}}%
\pgfpathlineto{\pgfqpoint{3.766511in}{1.685584in}}%
\pgfpathlineto{\pgfqpoint{3.769934in}{1.625162in}}%
\pgfpathlineto{\pgfqpoint{3.775066in}{1.733118in}}%
\pgfpathlineto{\pgfqpoint{3.778487in}{1.628632in}}%
\pgfpathlineto{\pgfqpoint{3.787040in}{1.740464in}}%
\pgfpathlineto{\pgfqpoint{3.791320in}{1.603638in}}%
\pgfpathlineto{\pgfqpoint{3.799020in}{1.741166in}}%
\pgfpathlineto{\pgfqpoint{3.802439in}{1.640092in}}%
\pgfpathlineto{\pgfqpoint{3.807568in}{1.715561in}}%
\pgfpathlineto{\pgfqpoint{3.809278in}{1.649743in}}%
\pgfpathlineto{\pgfqpoint{3.816117in}{1.714830in}}%
\pgfpathlineto{\pgfqpoint{3.820392in}{1.558146in}}%
\pgfpathlineto{\pgfqpoint{3.823810in}{1.558672in}}%
\pgfpathlineto{\pgfqpoint{3.828940in}{1.690190in}}%
\pgfpathlineto{\pgfqpoint{3.832358in}{1.735770in}}%
\pgfpathlineto{\pgfqpoint{3.837489in}{1.612942in}}%
\pgfpathlineto{\pgfqpoint{3.838344in}{1.685525in}}%
\pgfpathlineto{\pgfqpoint{3.844324in}{1.558876in}}%
\pgfpathlineto{\pgfqpoint{3.850314in}{1.651055in}}%
\pgfpathlineto{\pgfqpoint{3.853739in}{1.562259in}}%
\pgfpathlineto{\pgfqpoint{3.857163in}{1.641023in}}%
\pgfpathlineto{\pgfqpoint{3.860589in}{1.573894in}}%
\pgfpathlineto{\pgfqpoint{3.867436in}{1.661495in}}%
\pgfpathlineto{\pgfqpoint{3.869999in}{1.552140in}}%
\pgfpathlineto{\pgfqpoint{3.872563in}{1.611773in}}%
\pgfpathlineto{\pgfqpoint{3.877696in}{1.662719in}}%
\pgfpathlineto{\pgfqpoint{3.881975in}{1.570511in}}%
\pgfpathlineto{\pgfqpoint{3.887961in}{1.558613in}}%
\pgfpathlineto{\pgfqpoint{3.889671in}{1.613318in}}%
\pgfpathlineto{\pgfqpoint{3.896512in}{1.584740in}}%
\pgfpathlineto{\pgfqpoint{3.899075in}{1.679399in}}%
\pgfpathlineto{\pgfqpoint{3.902494in}{1.582990in}}%
\pgfpathlineto{\pgfqpoint{3.910196in}{1.517465in}}%
\pgfpathlineto{\pgfqpoint{3.911052in}{1.606377in}}%
\pgfpathlineto{\pgfqpoint{3.918748in}{1.665634in}}%
\pgfpathlineto{\pgfqpoint{3.922167in}{1.628745in}}%
\pgfpathlineto{\pgfqpoint{3.925587in}{1.686165in}}%
\pgfpathlineto{\pgfqpoint{3.931576in}{1.623613in}}%
\pgfpathlineto{\pgfqpoint{3.933287in}{1.696196in}}%
\pgfpathlineto{\pgfqpoint{3.938418in}{1.668462in}}%
\pgfpathlineto{\pgfqpoint{3.940985in}{1.571268in}}%
\pgfpathlineto{\pgfqpoint{3.947833in}{1.541232in}}%
\pgfpathlineto{\pgfqpoint{3.952966in}{1.614137in}}%
\pgfpathlineto{\pgfqpoint{3.955532in}{1.519185in}}%
\pgfpathlineto{\pgfqpoint{3.958954in}{1.664820in}}%
\pgfpathlineto{\pgfqpoint{3.962375in}{1.573777in}}%
\pgfpathlineto{\pgfqpoint{3.970067in}{1.554588in}}%
\pgfpathlineto{\pgfqpoint{3.970924in}{1.617052in}}%
\pgfpathlineto{\pgfqpoint{3.976053in}{1.542372in}}%
\pgfpathlineto{\pgfqpoint{3.979475in}{1.659161in}}%
\pgfpathlineto{\pgfqpoint{3.986324in}{1.660560in}}%
\pgfpathlineto{\pgfqpoint{3.988889in}{1.569752in}}%
\pgfpathlineto{\pgfqpoint{3.992312in}{1.621395in}}%
\pgfpathlineto{\pgfqpoint{3.997445in}{1.547676in}}%
\pgfpathlineto{\pgfqpoint{4.000867in}{1.667735in}}%
\pgfpathlineto{\pgfqpoint{4.005147in}{1.593080in}}%
\pgfpathlineto{\pgfqpoint{4.011132in}{1.640819in}}%
\pgfpathlineto{\pgfqpoint{4.017123in}{1.559198in}}%
\pgfpathlineto{\pgfqpoint{4.020541in}{1.713519in}}%
\pgfpathlineto{\pgfqpoint{4.023107in}{1.577569in}}%
\pgfpathlineto{\pgfqpoint{4.027385in}{1.653681in}}%
\pgfpathlineto{\pgfqpoint{4.030804in}{1.596174in}}%
\pgfpathlineto{\pgfqpoint{4.035079in}{1.562171in}}%
\pgfpathlineto{\pgfqpoint{4.040214in}{1.590282in}}%
\pgfpathlineto{\pgfqpoint{4.043636in}{1.746006in}}%
\pgfpathlineto{\pgfqpoint{4.049623in}{1.648899in}}%
\pgfpathlineto{\pgfqpoint{4.053904in}{1.744168in}}%
\pgfpathlineto{\pgfqpoint{4.058181in}{1.602881in}}%
\pgfpathlineto{\pgfqpoint{4.062458in}{1.671410in}}%
\pgfpathlineto{\pgfqpoint{4.067590in}{1.526535in}}%
\pgfpathlineto{\pgfqpoint{4.069301in}{1.591739in}}%
\pgfpathlineto{\pgfqpoint{4.076144in}{1.535051in}}%
\pgfpathlineto{\pgfqpoint{4.080418in}{1.616993in}}%
\pgfpathlineto{\pgfqpoint{4.085551in}{1.552721in}}%
\pgfpathlineto{\pgfqpoint{4.087264in}{1.728274in}}%
\pgfpathlineto{\pgfqpoint{4.091542in}{1.646360in}}%
\pgfpathlineto{\pgfqpoint{4.097525in}{1.776217in}}%
\pgfpathlineto{\pgfqpoint{4.099237in}{1.674909in}}%
\pgfpathlineto{\pgfqpoint{4.104369in}{1.718188in}}%
\pgfpathlineto{\pgfqpoint{4.110358in}{1.639858in}}%
\pgfpathlineto{\pgfqpoint{4.114636in}{1.697219in}}%
\pgfpathlineto{\pgfqpoint{4.119765in}{1.657674in}}%
\pgfpathlineto{\pgfqpoint{4.120620in}{1.730082in}}%
\pgfpathlineto{\pgfqpoint{4.127462in}{1.640965in}}%
\pgfpathlineto{\pgfqpoint{4.132595in}{1.601654in}}%
\pgfpathlineto{\pgfqpoint{4.135165in}{1.683074in}}%
\pgfpathlineto{\pgfqpoint{4.138587in}{1.610841in}}%
\pgfpathlineto{\pgfqpoint{4.143724in}{1.701504in}}%
\pgfpathlineto{\pgfqpoint{4.147142in}{1.642832in}}%
\pgfpathlineto{\pgfqpoint{4.149711in}{1.691765in}}%
\pgfpathlineto{\pgfqpoint{4.153137in}{1.632157in}}%
\pgfpathlineto{\pgfqpoint{4.157412in}{1.639040in}}%
\pgfpathlineto{\pgfqpoint{4.159124in}{1.729205in}}%
\pgfpathlineto{\pgfqpoint{4.165117in}{1.738597in}}%
\pgfpathlineto{\pgfqpoint{4.166829in}{1.644114in}}%
\pgfpathlineto{\pgfqpoint{4.170249in}{1.725066in}}%
\pgfpathlineto{\pgfqpoint{4.172814in}{1.684765in}}%
\pgfpathlineto{\pgfqpoint{4.177947in}{1.656421in}}%
\pgfpathlineto{\pgfqpoint{4.180510in}{1.745831in}}%
\pgfpathlineto{\pgfqpoint{4.184787in}{1.651347in}}%
\pgfpathlineto{\pgfqpoint{4.186496in}{1.689547in}}%
\pgfpathlineto{\pgfqpoint{4.190773in}{1.632976in}}%
\pgfpathlineto{\pgfqpoint{4.194194in}{1.722093in}}%
\pgfpathlineto{\pgfqpoint{4.197616in}{1.677941in}}%
\pgfpathlineto{\pgfqpoint{4.205311in}{1.739299in}}%
\pgfpathlineto{\pgfqpoint{4.208735in}{1.737520in}}%
\pgfpathlineto{\pgfqpoint{4.213009in}{1.622301in}}%
\pgfpathlineto{\pgfqpoint{4.214719in}{1.639741in}}%
\pgfpathlineto{\pgfqpoint{4.218141in}{1.614546in}}%
\pgfpathlineto{\pgfqpoint{4.221561in}{1.715093in}}%
\pgfpathlineto{\pgfqpoint{4.225839in}{1.656246in}}%
\pgfpathlineto{\pgfqpoint{4.227550in}{1.710604in}}%
\pgfpathlineto{\pgfqpoint{4.232681in}{1.634930in}}%
\pgfpathlineto{\pgfqpoint{4.234393in}{1.733235in}}%
\pgfpathlineto{\pgfqpoint{4.238671in}{1.731602in}}%
\pgfpathlineto{\pgfqpoint{4.244659in}{1.518079in}}%
\pgfpathlineto{\pgfqpoint{4.248080in}{1.675319in}}%
\pgfpathlineto{\pgfqpoint{4.252359in}{1.577218in}}%
\pgfpathlineto{\pgfqpoint{4.254925in}{1.676718in}}%
\pgfpathlineto{\pgfqpoint{4.260913in}{1.589292in}}%
\pgfpathlineto{\pgfqpoint{4.261769in}{1.642832in}}%
\pgfpathlineto{\pgfqpoint{4.265191in}{1.603521in}}%
\pgfpathlineto{\pgfqpoint{4.270325in}{1.592733in}}%
\pgfpathlineto{\pgfqpoint{4.274604in}{1.659629in}}%
\pgfpathlineto{\pgfqpoint{4.278880in}{1.570278in}}%
\pgfpathlineto{\pgfqpoint{4.282299in}{1.671293in}}%
\pgfpathlineto{\pgfqpoint{4.286570in}{1.594366in}}%
\pgfpathlineto{\pgfqpoint{4.290850in}{1.643065in}}%
\pgfpathlineto{\pgfqpoint{4.293417in}{1.564038in}}%
\pgfpathlineto{\pgfqpoint{4.297691in}{1.634316in}}%
\pgfpathlineto{\pgfqpoint{4.301964in}{1.564911in}}%
\pgfpathlineto{\pgfqpoint{4.303673in}{1.641257in}}%
\pgfpathlineto{\pgfqpoint{4.307096in}{1.726641in}}%
\pgfpathlineto{\pgfqpoint{4.311374in}{1.575352in}}%
\pgfpathlineto{\pgfqpoint{4.313941in}{1.672107in}}%
\pgfpathlineto{\pgfqpoint{4.319069in}{1.674325in}}%
\pgfpathlineto{\pgfqpoint{4.319924in}{1.593489in}}%
\pgfpathlineto{\pgfqpoint{4.325903in}{1.571560in}}%
\pgfpathlineto{\pgfqpoint{4.326759in}{1.621804in}}%
\pgfpathlineto{\pgfqpoint{4.331893in}{1.531084in}}%
\pgfpathlineto{\pgfqpoint{4.336171in}{1.524610in}}%
\pgfpathlineto{\pgfqpoint{4.337026in}{1.643938in}}%
\pgfpathlineto{\pgfqpoint{4.340445in}{1.576575in}}%
\pgfpathlineto{\pgfqpoint{4.344726in}{1.642013in}}%
\pgfpathlineto{\pgfqpoint{4.348149in}{1.589931in}}%
\pgfpathlineto{\pgfqpoint{4.353279in}{1.660969in}}%
\pgfpathlineto{\pgfqpoint{4.354134in}{1.547764in}}%
\pgfpathlineto{\pgfqpoint{4.358409in}{1.614601in}}%
\pgfpathlineto{\pgfqpoint{4.363544in}{1.578384in}}%
\pgfpathlineto{\pgfqpoint{4.366112in}{1.667442in}}%
\pgfpathlineto{\pgfqpoint{4.369535in}{1.579841in}}%
\pgfpathlineto{\pgfqpoint{4.371246in}{1.669338in}}%
\pgfpathlineto{\pgfqpoint{4.377235in}{1.578676in}}%
\pgfpathlineto{\pgfqpoint{4.379798in}{1.623817in}}%
\pgfpathlineto{\pgfqpoint{4.383215in}{1.504635in}}%
\pgfpathlineto{\pgfqpoint{4.386637in}{1.655837in}}%
\pgfpathlineto{\pgfqpoint{4.390915in}{1.665050in}}%
\pgfpathlineto{\pgfqpoint{4.392627in}{1.551526in}}%
\pgfpathlineto{\pgfqpoint{4.395194in}{1.631748in}}%
\pgfpathlineto{\pgfqpoint{4.399466in}{1.576049in}}%
\pgfpathlineto{\pgfqpoint{4.404602in}{1.608712in}}%
\pgfpathlineto{\pgfqpoint{4.408022in}{1.529041in}}%
\pgfpathlineto{\pgfqpoint{4.409735in}{1.608653in}}%
\pgfpathlineto{\pgfqpoint{4.416578in}{1.546481in}}%
\pgfpathlineto{\pgfqpoint{4.419143in}{1.606030in}}%
\pgfpathlineto{\pgfqpoint{4.428550in}{1.545955in}}%
\pgfpathlineto{\pgfqpoint{4.429406in}{1.630407in}}%
\pgfpathlineto{\pgfqpoint{4.432826in}{1.678526in}}%
\pgfpathlineto{\pgfqpoint{4.436247in}{1.553831in}}%
\pgfpathlineto{\pgfqpoint{4.440528in}{1.634433in}}%
\pgfpathlineto{\pgfqpoint{4.444808in}{1.572378in}}%
\pgfpathlineto{\pgfqpoint{4.448226in}{1.690132in}}%
\pgfpathlineto{\pgfqpoint{4.451649in}{1.591389in}}%
\pgfpathlineto{\pgfqpoint{4.453359in}{1.648808in}}%
\pgfpathlineto{\pgfqpoint{4.457633in}{1.583516in}}%
\pgfpathlineto{\pgfqpoint{4.461057in}{1.635306in}}%
\pgfpathlineto{\pgfqpoint{4.464478in}{1.667092in}}%
\pgfpathlineto{\pgfqpoint{4.468754in}{1.640380in}}%
\pgfpathlineto{\pgfqpoint{4.473032in}{1.485475in}}%
\pgfpathlineto{\pgfqpoint{4.473885in}{1.589931in}}%
\pgfpathlineto{\pgfqpoint{4.479871in}{1.507638in}}%
\pgfpathlineto{\pgfqpoint{4.480727in}{1.647321in}}%
\pgfpathlineto{\pgfqpoint{4.484146in}{1.578238in}}%
\pgfpathlineto{\pgfqpoint{4.490131in}{1.659278in}}%
\pgfpathlineto{\pgfqpoint{4.490987in}{1.568703in}}%
\pgfpathlineto{\pgfqpoint{4.496974in}{1.593255in}}%
\pgfpathlineto{\pgfqpoint{4.498682in}{1.656651in}}%
\pgfpathlineto{\pgfqpoint{4.502953in}{1.676919in}}%
\pgfpathlineto{\pgfqpoint{4.507228in}{1.587889in}}%
\pgfpathlineto{\pgfqpoint{4.508938in}{1.665225in}}%
\pgfpathlineto{\pgfqpoint{4.513210in}{1.569459in}}%
\pgfpathlineto{\pgfqpoint{4.515776in}{1.620727in}}%
\pgfpathlineto{\pgfqpoint{4.519199in}{1.575322in}}%
\pgfpathlineto{\pgfqpoint{4.524328in}{1.689606in}}%
\pgfpathlineto{\pgfqpoint{4.525183in}{1.604923in}}%
\pgfpathlineto{\pgfqpoint{4.531176in}{1.561675in}}%
\pgfpathlineto{\pgfqpoint{4.532031in}{1.656651in}}%
\pgfpathlineto{\pgfqpoint{4.535450in}{1.727748in}}%
\pgfpathlineto{\pgfqpoint{4.538868in}{1.555230in}}%
\pgfpathlineto{\pgfqpoint{4.544856in}{1.632625in}}%
\pgfpathlineto{\pgfqpoint{4.548281in}{1.573427in}}%
\pgfpathlineto{\pgfqpoint{4.549136in}{1.632333in}}%
\pgfpathlineto{\pgfqpoint{4.552557in}{1.512420in}}%
\pgfpathlineto{\pgfqpoint{4.558543in}{1.642948in}}%
\pgfpathlineto{\pgfqpoint{4.559399in}{1.546193in}}%
\pgfpathlineto{\pgfqpoint{4.563676in}{1.516621in}}%
\pgfpathlineto{\pgfqpoint{4.567950in}{1.637641in}}%
\pgfpathlineto{\pgfqpoint{4.569658in}{1.533301in}}%
\pgfpathlineto{\pgfqpoint{4.573939in}{1.523328in}}%
\pgfpathlineto{\pgfqpoint{4.576507in}{1.633092in}}%
\pgfpathlineto{\pgfqpoint{4.580783in}{1.585588in}}%
\pgfpathlineto{\pgfqpoint{4.583350in}{1.638108in}}%
\pgfpathlineto{\pgfqpoint{4.589338in}{1.564272in}}%
\pgfpathlineto{\pgfqpoint{4.591044in}{1.639858in}}%
\pgfpathlineto{\pgfqpoint{4.595324in}{1.509855in}}%
\pgfpathlineto{\pgfqpoint{4.598746in}{1.658229in}}%
\pgfpathlineto{\pgfqpoint{4.602166in}{1.517089in}}%
\pgfpathlineto{\pgfqpoint{4.606443in}{1.599791in}}%
\pgfpathlineto{\pgfqpoint{4.609866in}{1.554007in}}%
\pgfpathlineto{\pgfqpoint{4.610723in}{1.591915in}}%
\pgfpathlineto{\pgfqpoint{4.615000in}{1.552140in}}%
\pgfpathlineto{\pgfqpoint{4.617565in}{1.636183in}}%
\pgfpathlineto{\pgfqpoint{4.622695in}{1.523328in}}%
\pgfpathlineto{\pgfqpoint{4.625260in}{1.588795in}}%
\pgfpathlineto{\pgfqpoint{4.630391in}{1.556863in}}%
\pgfpathlineto{\pgfqpoint{4.632102in}{1.600635in}}%
\pgfpathlineto{\pgfqpoint{4.634669in}{1.559227in}}%
\pgfpathlineto{\pgfqpoint{4.638945in}{1.560597in}}%
\pgfpathlineto{\pgfqpoint{4.644077in}{1.643475in}}%
\pgfpathlineto{\pgfqpoint{4.647497in}{1.530908in}}%
\pgfpathlineto{\pgfqpoint{4.648350in}{1.620493in}}%
\pgfpathlineto{\pgfqpoint{4.654343in}{1.609004in}}%
\pgfpathlineto{\pgfqpoint{4.656051in}{1.536333in}}%
\pgfpathlineto{\pgfqpoint{4.659473in}{1.669718in}}%
\pgfpathlineto{\pgfqpoint{4.664605in}{1.714626in}}%
\pgfpathlineto{\pgfqpoint{4.665461in}{1.615185in}}%
\pgfpathlineto{\pgfqpoint{4.668881in}{1.674091in}}%
\pgfpathlineto{\pgfqpoint{4.673161in}{1.622184in}}%
\pgfpathlineto{\pgfqpoint{4.677437in}{1.731306in}}%
\pgfpathlineto{\pgfqpoint{4.679148in}{1.637494in}}%
\pgfpathlineto{\pgfqpoint{4.682567in}{1.688031in}}%
\pgfpathlineto{\pgfqpoint{4.685989in}{1.614020in}}%
\pgfpathlineto{\pgfqpoint{4.690266in}{1.707805in}}%
\pgfpathlineto{\pgfqpoint{4.694542in}{1.749330in}}%
\pgfpathlineto{\pgfqpoint{4.697112in}{1.626268in}}%
\pgfpathlineto{\pgfqpoint{4.700535in}{1.590023in}}%
\pgfpathlineto{\pgfqpoint{4.704809in}{1.733523in}}%
\pgfpathlineto{\pgfqpoint{4.706522in}{1.748804in}}%
\pgfpathlineto{\pgfqpoint{4.709945in}{1.652512in}}%
\pgfpathlineto{\pgfqpoint{4.713368in}{1.804035in}}%
\pgfpathlineto{\pgfqpoint{4.716788in}{1.592119in}}%
\pgfpathlineto{\pgfqpoint{4.721066in}{1.647906in}}%
\pgfpathlineto{\pgfqpoint{4.723635in}{1.548114in}}%
\pgfpathlineto{\pgfqpoint{4.727062in}{1.597223in}}%
\pgfpathlineto{\pgfqpoint{4.731339in}{1.549776in}}%
\pgfpathlineto{\pgfqpoint{4.735622in}{1.625713in}}%
\pgfpathlineto{\pgfqpoint{4.737335in}{1.551322in}}%
\pgfpathlineto{\pgfqpoint{4.741616in}{1.659161in}}%
\pgfpathlineto{\pgfqpoint{4.745895in}{1.518546in}}%
\pgfpathlineto{\pgfqpoint{4.751882in}{1.772659in}}%
\pgfpathlineto{\pgfqpoint{4.754446in}{1.633268in}}%
\pgfpathlineto{\pgfqpoint{4.759573in}{1.523386in}}%
\pgfpathlineto{\pgfqpoint{4.761282in}{1.639273in}}%
\pgfpathlineto{\pgfqpoint{4.764703in}{1.715561in}}%
\pgfpathlineto{\pgfqpoint{4.769831in}{1.732066in}}%
\pgfpathlineto{\pgfqpoint{4.771544in}{1.605387in}}%
\pgfpathlineto{\pgfqpoint{4.774966in}{1.678584in}}%
\pgfpathlineto{\pgfqpoint{4.780961in}{1.572612in}}%
\pgfpathlineto{\pgfqpoint{4.784387in}{1.722210in}}%
\pgfpathlineto{\pgfqpoint{4.785242in}{1.523036in}}%
\pgfpathlineto{\pgfqpoint{4.789519in}{1.602122in}}%
\pgfpathlineto{\pgfqpoint{4.792084in}{1.499999in}}%
\pgfpathlineto{\pgfqpoint{4.795506in}{1.542225in}}%
\pgfpathlineto{\pgfqpoint{4.795506in}{1.542225in}}%
\pgfusepath{stroke}%
\end{pgfscope}%
\begin{pgfscope}%
\pgfsetrectcap%
\pgfsetmiterjoin%
\pgfsetlinewidth{0.803000pt}%
\definecolor{currentstroke}{rgb}{0.000000,0.000000,0.000000}%
\pgfsetstrokecolor{currentstroke}%
\pgfsetdash{}{0pt}%
\pgfpathmoveto{\pgfqpoint{0.484581in}{1.437021in}}%
\pgfpathlineto{\pgfqpoint{0.484581in}{2.011643in}}%
\pgfusepath{stroke}%
\end{pgfscope}%
\begin{pgfscope}%
\pgfsetrectcap%
\pgfsetmiterjoin%
\pgfsetlinewidth{0.803000pt}%
\definecolor{currentstroke}{rgb}{0.000000,0.000000,0.000000}%
\pgfsetstrokecolor{currentstroke}%
\pgfsetdash{}{0pt}%
\pgfpathmoveto{\pgfqpoint{5.000788in}{1.437021in}}%
\pgfpathlineto{\pgfqpoint{5.000788in}{2.011643in}}%
\pgfusepath{stroke}%
\end{pgfscope}%
\begin{pgfscope}%
\pgfsetrectcap%
\pgfsetmiterjoin%
\pgfsetlinewidth{0.803000pt}%
\definecolor{currentstroke}{rgb}{0.000000,0.000000,0.000000}%
\pgfsetstrokecolor{currentstroke}%
\pgfsetdash{}{0pt}%
\pgfpathmoveto{\pgfqpoint{0.484581in}{1.437021in}}%
\pgfpathlineto{\pgfqpoint{5.000788in}{1.437021in}}%
\pgfusepath{stroke}%
\end{pgfscope}%
\begin{pgfscope}%
\pgfsetrectcap%
\pgfsetmiterjoin%
\pgfsetlinewidth{0.803000pt}%
\definecolor{currentstroke}{rgb}{0.000000,0.000000,0.000000}%
\pgfsetstrokecolor{currentstroke}%
\pgfsetdash{}{0pt}%
\pgfpathmoveto{\pgfqpoint{0.484581in}{2.011643in}}%
\pgfpathlineto{\pgfqpoint{5.000788in}{2.011643in}}%
\pgfusepath{stroke}%
\end{pgfscope}%
\begin{pgfscope}%
\pgfsetbuttcap%
\pgfsetmiterjoin%
\definecolor{currentfill}{rgb}{1.000000,1.000000,1.000000}%
\pgfsetfillcolor{currentfill}%
\pgfsetlinewidth{0.000000pt}%
\definecolor{currentstroke}{rgb}{0.000000,0.000000,0.000000}%
\pgfsetstrokecolor{currentstroke}%
\pgfsetstrokeopacity{0.000000}%
\pgfsetdash{}{0pt}%
\pgfpathmoveto{\pgfqpoint{0.484581in}{0.539544in}}%
\pgfpathlineto{\pgfqpoint{5.000788in}{0.539544in}}%
\pgfpathlineto{\pgfqpoint{5.000788in}{1.114166in}}%
\pgfpathlineto{\pgfqpoint{0.484581in}{1.114166in}}%
\pgfpathlineto{\pgfqpoint{0.484581in}{0.539544in}}%
\pgfpathclose%
\pgfusepath{fill}%
\end{pgfscope}%
\begin{pgfscope}%
\pgfsetbuttcap%
\pgfsetroundjoin%
\definecolor{currentfill}{rgb}{0.000000,0.000000,0.000000}%
\pgfsetfillcolor{currentfill}%
\pgfsetlinewidth{0.803000pt}%
\definecolor{currentstroke}{rgb}{0.000000,0.000000,0.000000}%
\pgfsetstrokecolor{currentstroke}%
\pgfsetdash{}{0pt}%
\pgfsys@defobject{currentmarker}{\pgfqpoint{0.000000in}{-0.048611in}}{\pgfqpoint{0.000000in}{0.000000in}}{%
\pgfpathmoveto{\pgfqpoint{0.000000in}{0.000000in}}%
\pgfpathlineto{\pgfqpoint{0.000000in}{-0.048611in}}%
\pgfusepath{stroke,fill}%
}%
\begin{pgfscope}%
\pgfsys@transformshift{0.689546in}{0.539544in}%
\pgfsys@useobject{currentmarker}{}%
\end{pgfscope}%
\end{pgfscope}%
\begin{pgfscope}%
\definecolor{textcolor}{rgb}{0.000000,0.000000,0.000000}%
\pgfsetstrokecolor{textcolor}%
\pgfsetfillcolor{textcolor}%
\pgftext[x=0.689546in,y=0.442322in,,top]{\color{textcolor}\rmfamily\fontsize{8.000000}{9.600000}\selectfont \(\displaystyle {06{:}00}\)}%
\end{pgfscope}%
\begin{pgfscope}%
\pgfsetbuttcap%
\pgfsetroundjoin%
\definecolor{currentfill}{rgb}{0.000000,0.000000,0.000000}%
\pgfsetfillcolor{currentfill}%
\pgfsetlinewidth{0.803000pt}%
\definecolor{currentstroke}{rgb}{0.000000,0.000000,0.000000}%
\pgfsetstrokecolor{currentstroke}%
\pgfsetdash{}{0pt}%
\pgfsys@defobject{currentmarker}{\pgfqpoint{0.000000in}{-0.048611in}}{\pgfqpoint{0.000000in}{0.000000in}}{%
\pgfpathmoveto{\pgfqpoint{0.000000in}{0.000000in}}%
\pgfpathlineto{\pgfqpoint{0.000000in}{-0.048611in}}%
\pgfusepath{stroke,fill}%
}%
\begin{pgfscope}%
\pgfsys@transformshift{1.202878in}{0.539544in}%
\pgfsys@useobject{currentmarker}{}%
\end{pgfscope}%
\end{pgfscope}%
\begin{pgfscope}%
\definecolor{textcolor}{rgb}{0.000000,0.000000,0.000000}%
\pgfsetstrokecolor{textcolor}%
\pgfsetfillcolor{textcolor}%
\pgftext[x=1.202878in,y=0.442322in,,top]{\color{textcolor}\rmfamily\fontsize{8.000000}{9.600000}\selectfont \(\displaystyle {09{:}00}\)}%
\end{pgfscope}%
\begin{pgfscope}%
\pgfsetbuttcap%
\pgfsetroundjoin%
\definecolor{currentfill}{rgb}{0.000000,0.000000,0.000000}%
\pgfsetfillcolor{currentfill}%
\pgfsetlinewidth{0.803000pt}%
\definecolor{currentstroke}{rgb}{0.000000,0.000000,0.000000}%
\pgfsetstrokecolor{currentstroke}%
\pgfsetdash{}{0pt}%
\pgfsys@defobject{currentmarker}{\pgfqpoint{0.000000in}{-0.048611in}}{\pgfqpoint{0.000000in}{0.000000in}}{%
\pgfpathmoveto{\pgfqpoint{0.000000in}{0.000000in}}%
\pgfpathlineto{\pgfqpoint{0.000000in}{-0.048611in}}%
\pgfusepath{stroke,fill}%
}%
\begin{pgfscope}%
\pgfsys@transformshift{1.716211in}{0.539544in}%
\pgfsys@useobject{currentmarker}{}%
\end{pgfscope}%
\end{pgfscope}%
\begin{pgfscope}%
\definecolor{textcolor}{rgb}{0.000000,0.000000,0.000000}%
\pgfsetstrokecolor{textcolor}%
\pgfsetfillcolor{textcolor}%
\pgftext[x=1.716211in,y=0.442322in,,top]{\color{textcolor}\rmfamily\fontsize{8.000000}{9.600000}\selectfont \(\displaystyle {12{:}00}\)}%
\end{pgfscope}%
\begin{pgfscope}%
\pgfsetbuttcap%
\pgfsetroundjoin%
\definecolor{currentfill}{rgb}{0.000000,0.000000,0.000000}%
\pgfsetfillcolor{currentfill}%
\pgfsetlinewidth{0.803000pt}%
\definecolor{currentstroke}{rgb}{0.000000,0.000000,0.000000}%
\pgfsetstrokecolor{currentstroke}%
\pgfsetdash{}{0pt}%
\pgfsys@defobject{currentmarker}{\pgfqpoint{0.000000in}{-0.048611in}}{\pgfqpoint{0.000000in}{0.000000in}}{%
\pgfpathmoveto{\pgfqpoint{0.000000in}{0.000000in}}%
\pgfpathlineto{\pgfqpoint{0.000000in}{-0.048611in}}%
\pgfusepath{stroke,fill}%
}%
\begin{pgfscope}%
\pgfsys@transformshift{2.229543in}{0.539544in}%
\pgfsys@useobject{currentmarker}{}%
\end{pgfscope}%
\end{pgfscope}%
\begin{pgfscope}%
\definecolor{textcolor}{rgb}{0.000000,0.000000,0.000000}%
\pgfsetstrokecolor{textcolor}%
\pgfsetfillcolor{textcolor}%
\pgftext[x=2.229543in,y=0.442322in,,top]{\color{textcolor}\rmfamily\fontsize{8.000000}{9.600000}\selectfont \(\displaystyle {15{:}00}\)}%
\end{pgfscope}%
\begin{pgfscope}%
\pgfsetbuttcap%
\pgfsetroundjoin%
\definecolor{currentfill}{rgb}{0.000000,0.000000,0.000000}%
\pgfsetfillcolor{currentfill}%
\pgfsetlinewidth{0.803000pt}%
\definecolor{currentstroke}{rgb}{0.000000,0.000000,0.000000}%
\pgfsetstrokecolor{currentstroke}%
\pgfsetdash{}{0pt}%
\pgfsys@defobject{currentmarker}{\pgfqpoint{0.000000in}{-0.048611in}}{\pgfqpoint{0.000000in}{0.000000in}}{%
\pgfpathmoveto{\pgfqpoint{0.000000in}{0.000000in}}%
\pgfpathlineto{\pgfqpoint{0.000000in}{-0.048611in}}%
\pgfusepath{stroke,fill}%
}%
\begin{pgfscope}%
\pgfsys@transformshift{2.742876in}{0.539544in}%
\pgfsys@useobject{currentmarker}{}%
\end{pgfscope}%
\end{pgfscope}%
\begin{pgfscope}%
\definecolor{textcolor}{rgb}{0.000000,0.000000,0.000000}%
\pgfsetstrokecolor{textcolor}%
\pgfsetfillcolor{textcolor}%
\pgftext[x=2.742876in,y=0.442322in,,top]{\color{textcolor}\rmfamily\fontsize{8.000000}{9.600000}\selectfont \(\displaystyle {18{:}00}\)}%
\end{pgfscope}%
\begin{pgfscope}%
\pgfsetbuttcap%
\pgfsetroundjoin%
\definecolor{currentfill}{rgb}{0.000000,0.000000,0.000000}%
\pgfsetfillcolor{currentfill}%
\pgfsetlinewidth{0.803000pt}%
\definecolor{currentstroke}{rgb}{0.000000,0.000000,0.000000}%
\pgfsetstrokecolor{currentstroke}%
\pgfsetdash{}{0pt}%
\pgfsys@defobject{currentmarker}{\pgfqpoint{0.000000in}{-0.048611in}}{\pgfqpoint{0.000000in}{0.000000in}}{%
\pgfpathmoveto{\pgfqpoint{0.000000in}{0.000000in}}%
\pgfpathlineto{\pgfqpoint{0.000000in}{-0.048611in}}%
\pgfusepath{stroke,fill}%
}%
\begin{pgfscope}%
\pgfsys@transformshift{3.256208in}{0.539544in}%
\pgfsys@useobject{currentmarker}{}%
\end{pgfscope}%
\end{pgfscope}%
\begin{pgfscope}%
\definecolor{textcolor}{rgb}{0.000000,0.000000,0.000000}%
\pgfsetstrokecolor{textcolor}%
\pgfsetfillcolor{textcolor}%
\pgftext[x=3.256208in,y=0.442322in,,top]{\color{textcolor}\rmfamily\fontsize{8.000000}{9.600000}\selectfont \(\displaystyle {21{:}00}\)}%
\end{pgfscope}%
\begin{pgfscope}%
\pgfsetbuttcap%
\pgfsetroundjoin%
\definecolor{currentfill}{rgb}{0.000000,0.000000,0.000000}%
\pgfsetfillcolor{currentfill}%
\pgfsetlinewidth{0.803000pt}%
\definecolor{currentstroke}{rgb}{0.000000,0.000000,0.000000}%
\pgfsetstrokecolor{currentstroke}%
\pgfsetdash{}{0pt}%
\pgfsys@defobject{currentmarker}{\pgfqpoint{0.000000in}{-0.048611in}}{\pgfqpoint{0.000000in}{0.000000in}}{%
\pgfpathmoveto{\pgfqpoint{0.000000in}{0.000000in}}%
\pgfpathlineto{\pgfqpoint{0.000000in}{-0.048611in}}%
\pgfusepath{stroke,fill}%
}%
\begin{pgfscope}%
\pgfsys@transformshift{3.769541in}{0.539544in}%
\pgfsys@useobject{currentmarker}{}%
\end{pgfscope}%
\end{pgfscope}%
\begin{pgfscope}%
\definecolor{textcolor}{rgb}{0.000000,0.000000,0.000000}%
\pgfsetstrokecolor{textcolor}%
\pgfsetfillcolor{textcolor}%
\pgftext[x=3.769541in,y=0.442322in,,top]{\color{textcolor}\rmfamily\fontsize{8.000000}{9.600000}\selectfont \(\displaystyle {00{:}00}\)}%
\end{pgfscope}%
\begin{pgfscope}%
\pgfsetbuttcap%
\pgfsetroundjoin%
\definecolor{currentfill}{rgb}{0.000000,0.000000,0.000000}%
\pgfsetfillcolor{currentfill}%
\pgfsetlinewidth{0.803000pt}%
\definecolor{currentstroke}{rgb}{0.000000,0.000000,0.000000}%
\pgfsetstrokecolor{currentstroke}%
\pgfsetdash{}{0pt}%
\pgfsys@defobject{currentmarker}{\pgfqpoint{0.000000in}{-0.048611in}}{\pgfqpoint{0.000000in}{0.000000in}}{%
\pgfpathmoveto{\pgfqpoint{0.000000in}{0.000000in}}%
\pgfpathlineto{\pgfqpoint{0.000000in}{-0.048611in}}%
\pgfusepath{stroke,fill}%
}%
\begin{pgfscope}%
\pgfsys@transformshift{4.282873in}{0.539544in}%
\pgfsys@useobject{currentmarker}{}%
\end{pgfscope}%
\end{pgfscope}%
\begin{pgfscope}%
\definecolor{textcolor}{rgb}{0.000000,0.000000,0.000000}%
\pgfsetstrokecolor{textcolor}%
\pgfsetfillcolor{textcolor}%
\pgftext[x=4.282873in,y=0.442322in,,top]{\color{textcolor}\rmfamily\fontsize{8.000000}{9.600000}\selectfont \(\displaystyle {03{:}00}\)}%
\end{pgfscope}%
\begin{pgfscope}%
\pgfsetbuttcap%
\pgfsetroundjoin%
\definecolor{currentfill}{rgb}{0.000000,0.000000,0.000000}%
\pgfsetfillcolor{currentfill}%
\pgfsetlinewidth{0.803000pt}%
\definecolor{currentstroke}{rgb}{0.000000,0.000000,0.000000}%
\pgfsetstrokecolor{currentstroke}%
\pgfsetdash{}{0pt}%
\pgfsys@defobject{currentmarker}{\pgfqpoint{0.000000in}{-0.048611in}}{\pgfqpoint{0.000000in}{0.000000in}}{%
\pgfpathmoveto{\pgfqpoint{0.000000in}{0.000000in}}%
\pgfpathlineto{\pgfqpoint{0.000000in}{-0.048611in}}%
\pgfusepath{stroke,fill}%
}%
\begin{pgfscope}%
\pgfsys@transformshift{4.796206in}{0.539544in}%
\pgfsys@useobject{currentmarker}{}%
\end{pgfscope}%
\end{pgfscope}%
\begin{pgfscope}%
\definecolor{textcolor}{rgb}{0.000000,0.000000,0.000000}%
\pgfsetstrokecolor{textcolor}%
\pgfsetfillcolor{textcolor}%
\pgftext[x=4.796206in,y=0.442322in,,top]{\color{textcolor}\rmfamily\fontsize{8.000000}{9.600000}\selectfont \(\displaystyle {06{:}00}\)}%
\end{pgfscope}%
\begin{pgfscope}%
\definecolor{textcolor}{rgb}{0.000000,0.000000,0.000000}%
\pgfsetstrokecolor{textcolor}%
\pgfsetfillcolor{textcolor}%
\pgftext[x=2.742685in,y=0.288100in,,top]{\color{textcolor}\rmfamily\fontsize{10.000000}{12.000000}\selectfont Time (UTC)}%
\end{pgfscope}%
\begin{pgfscope}%
\pgfsetbuttcap%
\pgfsetroundjoin%
\definecolor{currentfill}{rgb}{0.000000,0.000000,0.000000}%
\pgfsetfillcolor{currentfill}%
\pgfsetlinewidth{0.803000pt}%
\definecolor{currentstroke}{rgb}{0.000000,0.000000,0.000000}%
\pgfsetstrokecolor{currentstroke}%
\pgfsetdash{}{0pt}%
\pgfsys@defobject{currentmarker}{\pgfqpoint{-0.048611in}{0.000000in}}{\pgfqpoint{-0.000000in}{0.000000in}}{%
\pgfpathmoveto{\pgfqpoint{-0.000000in}{0.000000in}}%
\pgfpathlineto{\pgfqpoint{-0.048611in}{0.000000in}}%
\pgfusepath{stroke,fill}%
}%
\begin{pgfscope}%
\pgfsys@transformshift{0.484581in}{0.717544in}%
\pgfsys@useobject{currentmarker}{}%
\end{pgfscope}%
\end{pgfscope}%
\begin{pgfscope}%
\definecolor{textcolor}{rgb}{0.000000,0.000000,0.000000}%
\pgfsetstrokecolor{textcolor}%
\pgfsetfillcolor{textcolor}%
\pgftext[x=0.328331in, y=0.678988in, left, base]{\color{textcolor}\rmfamily\fontsize{8.000000}{9.600000}\selectfont \(\displaystyle {0}\)}%
\end{pgfscope}%
\begin{pgfscope}%
\pgfsetbuttcap%
\pgfsetroundjoin%
\definecolor{currentfill}{rgb}{0.000000,0.000000,0.000000}%
\pgfsetfillcolor{currentfill}%
\pgfsetlinewidth{0.803000pt}%
\definecolor{currentstroke}{rgb}{0.000000,0.000000,0.000000}%
\pgfsetstrokecolor{currentstroke}%
\pgfsetdash{}{0pt}%
\pgfsys@defobject{currentmarker}{\pgfqpoint{-0.048611in}{0.000000in}}{\pgfqpoint{-0.000000in}{0.000000in}}{%
\pgfpathmoveto{\pgfqpoint{-0.000000in}{0.000000in}}%
\pgfpathlineto{\pgfqpoint{-0.048611in}{0.000000in}}%
\pgfusepath{stroke,fill}%
}%
\begin{pgfscope}%
\pgfsys@transformshift{0.484581in}{0.921085in}%
\pgfsys@useobject{currentmarker}{}%
\end{pgfscope}%
\end{pgfscope}%
\begin{pgfscope}%
\definecolor{textcolor}{rgb}{0.000000,0.000000,0.000000}%
\pgfsetstrokecolor{textcolor}%
\pgfsetfillcolor{textcolor}%
\pgftext[x=0.328331in, y=0.882529in, left, base]{\color{textcolor}\rmfamily\fontsize{8.000000}{9.600000}\selectfont \(\displaystyle {5}\)}%
\end{pgfscope}%
\begin{pgfscope}%
\definecolor{textcolor}{rgb}{0.000000,0.000000,0.000000}%
\pgfsetstrokecolor{textcolor}%
\pgfsetfillcolor{textcolor}%
\pgftext[x=0.484581in,y=1.155833in,left,base]{\color{textcolor}\rmfamily\fontsize{8.000000}{9.600000}\selectfont \(\displaystyle \times{10^{\ensuremath{-}6}}{}\)}%
\end{pgfscope}%
\begin{pgfscope}%
\pgfpathrectangle{\pgfqpoint{0.484581in}{0.539544in}}{\pgfqpoint{4.516206in}{0.574622in}}%
\pgfusepath{clip}%
\pgfsetrectcap%
\pgfsetroundjoin%
\pgfsetlinewidth{0.501875pt}%
\definecolor{currentstroke}{rgb}{0.003922,0.450980,0.698039}%
\pgfsetstrokecolor{currentstroke}%
\pgfsetstrokeopacity{0.700000}%
\pgfsetdash{}{0pt}%
\pgfpathmoveto{\pgfqpoint{0.689863in}{0.721589in}}%
\pgfpathlineto{\pgfqpoint{0.694140in}{0.685169in}}%
\pgfpathlineto{\pgfqpoint{0.698419in}{0.771788in}}%
\pgfpathlineto{\pgfqpoint{0.700132in}{0.644899in}}%
\pgfpathlineto{\pgfqpoint{0.703553in}{0.767073in}}%
\pgfpathlineto{\pgfqpoint{0.707833in}{0.699777in}}%
\pgfpathlineto{\pgfqpoint{0.713821in}{0.764305in}}%
\pgfpathlineto{\pgfqpoint{0.718102in}{0.651844in}}%
\pgfpathlineto{\pgfqpoint{0.721524in}{0.733357in}}%
\pgfpathlineto{\pgfqpoint{0.726652in}{0.645940in}}%
\pgfpathlineto{\pgfqpoint{0.729218in}{0.759733in}}%
\pgfpathlineto{\pgfqpoint{0.736917in}{0.775857in}}%
\pgfpathlineto{\pgfqpoint{0.740340in}{0.660732in}}%
\pgfpathlineto{\pgfqpoint{0.742052in}{0.726624in}}%
\pgfpathlineto{\pgfqpoint{0.746327in}{0.778195in}}%
\pgfpathlineto{\pgfqpoint{0.751457in}{0.661163in}}%
\pgfpathlineto{\pgfqpoint{0.755738in}{0.790322in}}%
\pgfpathlineto{\pgfqpoint{0.759157in}{0.660947in}}%
\pgfpathlineto{\pgfqpoint{0.765999in}{0.761461in}}%
\pgfpathlineto{\pgfqpoint{0.767711in}{0.680382in}}%
\pgfpathlineto{\pgfqpoint{0.773703in}{0.750343in}}%
\pgfpathlineto{\pgfqpoint{0.776269in}{0.679736in}}%
\pgfpathlineto{\pgfqpoint{0.783967in}{0.745340in}}%
\pgfpathlineto{\pgfqpoint{0.784824in}{0.605387in}}%
\pgfpathlineto{\pgfqpoint{0.789097in}{0.784242in}}%
\pgfpathlineto{\pgfqpoint{0.793373in}{0.817025in}}%
\pgfpathlineto{\pgfqpoint{0.797648in}{0.692652in}}%
\pgfpathlineto{\pgfqpoint{0.805343in}{0.725690in}}%
\pgfpathlineto{\pgfqpoint{0.807054in}{0.608730in}}%
\pgfpathlineto{\pgfqpoint{0.811330in}{0.738647in}}%
\pgfpathlineto{\pgfqpoint{0.816461in}{0.682759in}}%
\pgfpathlineto{\pgfqpoint{0.819025in}{0.765063in}}%
\pgfpathlineto{\pgfqpoint{0.823303in}{0.717449in}}%
\pgfpathlineto{\pgfqpoint{0.827584in}{0.851108in}}%
\pgfpathlineto{\pgfqpoint{0.834428in}{0.692078in}}%
\pgfpathlineto{\pgfqpoint{0.838709in}{0.729073in}}%
\pgfpathlineto{\pgfqpoint{0.840420in}{0.676353in}}%
\pgfpathlineto{\pgfqpoint{0.848114in}{0.744295in}}%
\pgfpathlineto{\pgfqpoint{0.848968in}{0.708705in}}%
\pgfpathlineto{\pgfqpoint{0.855805in}{0.669587in}}%
\pgfpathlineto{\pgfqpoint{0.859233in}{0.748939in}}%
\pgfpathlineto{\pgfqpoint{0.861803in}{0.677757in}}%
\pgfpathlineto{\pgfqpoint{0.867793in}{0.662535in}}%
\pgfpathlineto{\pgfqpoint{0.870360in}{0.764416in}}%
\pgfpathlineto{\pgfqpoint{0.875490in}{0.657676in}}%
\pgfpathlineto{\pgfqpoint{0.879771in}{0.744551in}}%
\pgfpathlineto{\pgfqpoint{0.883195in}{0.653176in}}%
\pgfpathlineto{\pgfqpoint{0.887469in}{0.783524in}}%
\pgfpathlineto{\pgfqpoint{0.893457in}{0.703128in}}%
\pgfpathlineto{\pgfqpoint{0.896021in}{0.741311in}}%
\pgfpathlineto{\pgfqpoint{0.903720in}{0.770711in}}%
\pgfpathlineto{\pgfqpoint{0.907998in}{0.652889in}}%
\pgfpathlineto{\pgfqpoint{0.910566in}{0.728211in}}%
\pgfpathlineto{\pgfqpoint{0.914841in}{0.649793in}}%
\pgfpathlineto{\pgfqpoint{0.918259in}{0.754914in}}%
\pgfpathlineto{\pgfqpoint{0.922534in}{0.779168in}}%
\pgfpathlineto{\pgfqpoint{0.925951in}{0.659475in}}%
\pgfpathlineto{\pgfqpoint{0.932791in}{0.780141in}}%
\pgfpathlineto{\pgfqpoint{0.935359in}{0.665881in}}%
\pgfpathlineto{\pgfqpoint{0.938782in}{0.766467in}}%
\pgfpathlineto{\pgfqpoint{0.947330in}{0.636984in}}%
\pgfpathlineto{\pgfqpoint{0.951605in}{0.759015in}}%
\pgfpathlineto{\pgfqpoint{0.959305in}{0.764089in}}%
\pgfpathlineto{\pgfqpoint{0.963579in}{0.653679in}}%
\pgfpathlineto{\pgfqpoint{0.966147in}{0.724035in}}%
\pgfpathlineto{\pgfqpoint{0.969570in}{0.683477in}}%
\pgfpathlineto{\pgfqpoint{0.976408in}{0.806015in}}%
\pgfpathlineto{\pgfqpoint{0.978974in}{0.720329in}}%
\pgfpathlineto{\pgfqpoint{0.984102in}{0.758728in}}%
\pgfpathlineto{\pgfqpoint{0.985812in}{0.662463in}}%
\pgfpathlineto{\pgfqpoint{0.991797in}{0.733321in}}%
\pgfpathlineto{\pgfqpoint{0.997785in}{0.750271in}}%
\pgfpathlineto{\pgfqpoint{0.999494in}{0.780716in}}%
\pgfpathlineto{\pgfqpoint{1.003769in}{0.688448in}}%
\pgfpathlineto{\pgfqpoint{1.010613in}{0.744295in}}%
\pgfpathlineto{\pgfqpoint{1.014889in}{0.673186in}}%
\pgfpathlineto{\pgfqpoint{1.018310in}{0.746960in}}%
\pgfpathlineto{\pgfqpoint{1.020874in}{0.673545in}}%
\pgfpathlineto{\pgfqpoint{1.026001in}{0.741846in}}%
\pgfpathlineto{\pgfqpoint{1.029421in}{0.668291in}}%
\pgfpathlineto{\pgfqpoint{1.038832in}{0.800079in}}%
\pgfpathlineto{\pgfqpoint{1.041400in}{0.683154in}}%
\pgfpathlineto{\pgfqpoint{1.046538in}{0.642669in}}%
\pgfpathlineto{\pgfqpoint{1.050817in}{0.768733in}}%
\pgfpathlineto{\pgfqpoint{1.054238in}{0.679807in}}%
\pgfpathlineto{\pgfqpoint{1.059370in}{0.769020in}}%
\pgfpathlineto{\pgfqpoint{1.062792in}{0.636047in}}%
\pgfpathlineto{\pgfqpoint{1.068774in}{0.782192in}}%
\pgfpathlineto{\pgfqpoint{1.071339in}{0.679233in}}%
\pgfpathlineto{\pgfqpoint{1.077323in}{0.749338in}}%
\pgfpathlineto{\pgfqpoint{1.079890in}{0.663364in}}%
\pgfpathlineto{\pgfqpoint{1.084169in}{0.749338in}}%
\pgfpathlineto{\pgfqpoint{1.091012in}{0.695177in}}%
\pgfpathlineto{\pgfqpoint{1.095289in}{0.766144in}}%
\pgfpathlineto{\pgfqpoint{1.096997in}{0.686070in}}%
\pgfpathlineto{\pgfqpoint{1.102978in}{0.793385in}}%
\pgfpathlineto{\pgfqpoint{1.105542in}{0.697336in}}%
\pgfpathlineto{\pgfqpoint{1.112379in}{0.642669in}}%
\pgfpathlineto{\pgfqpoint{1.118364in}{0.769235in}}%
\pgfpathlineto{\pgfqpoint{1.124349in}{0.656671in}}%
\pgfpathlineto{\pgfqpoint{1.129483in}{0.728139in}}%
\pgfpathlineto{\pgfqpoint{1.131195in}{0.682616in}}%
\pgfpathlineto{\pgfqpoint{1.137183in}{0.656056in}}%
\pgfpathlineto{\pgfqpoint{1.139749in}{0.764632in}}%
\pgfpathlineto{\pgfqpoint{1.147448in}{0.689740in}}%
\pgfpathlineto{\pgfqpoint{1.150013in}{0.769235in}}%
\pgfpathlineto{\pgfqpoint{1.152579in}{0.575628in}}%
\pgfpathlineto{\pgfqpoint{1.156855in}{0.776400in}}%
\pgfpathlineto{\pgfqpoint{1.162841in}{0.681427in}}%
\pgfpathlineto{\pgfqpoint{1.167121in}{0.726592in}}%
\pgfpathlineto{\pgfqpoint{1.169688in}{0.667896in}}%
\pgfpathlineto{\pgfqpoint{1.176534in}{0.787481in}}%
\pgfpathlineto{\pgfqpoint{1.178245in}{0.676784in}}%
\pgfpathlineto{\pgfqpoint{1.185091in}{0.833942in}}%
\pgfpathlineto{\pgfqpoint{1.188512in}{0.707229in}}%
\pgfpathlineto{\pgfqpoint{1.191934in}{0.760671in}}%
\pgfpathlineto{\pgfqpoint{1.196209in}{0.672643in}}%
\pgfpathlineto{\pgfqpoint{1.200488in}{0.747535in}}%
\pgfpathlineto{\pgfqpoint{1.203910in}{0.679915in}}%
\pgfpathlineto{\pgfqpoint{1.209902in}{0.814184in}}%
\pgfpathlineto{\pgfqpoint{1.212472in}{0.687834in}}%
\pgfpathlineto{\pgfqpoint{1.217612in}{0.765888in}}%
\pgfpathlineto{\pgfqpoint{1.222746in}{0.685097in}}%
\pgfpathlineto{\pgfqpoint{1.227024in}{0.668147in}}%
\pgfpathlineto{\pgfqpoint{1.231301in}{0.763443in}}%
\pgfpathlineto{\pgfqpoint{1.233868in}{0.676640in}}%
\pgfpathlineto{\pgfqpoint{1.238147in}{0.745915in}}%
\pgfpathlineto{\pgfqpoint{1.242423in}{0.693953in}}%
\pgfpathlineto{\pgfqpoint{1.246701in}{0.774309in}}%
\pgfpathlineto{\pgfqpoint{1.250981in}{0.670561in}}%
\pgfpathlineto{\pgfqpoint{1.258680in}{0.795292in}}%
\pgfpathlineto{\pgfqpoint{1.261242in}{0.712773in}}%
\pgfpathlineto{\pgfqpoint{1.266376in}{0.647456in}}%
\pgfpathlineto{\pgfqpoint{1.269800in}{0.796696in}}%
\pgfpathlineto{\pgfqpoint{1.273224in}{0.704460in}}%
\pgfpathlineto{\pgfqpoint{1.276647in}{0.782986in}}%
\pgfpathlineto{\pgfqpoint{1.281778in}{0.704460in}}%
\pgfpathlineto{\pgfqpoint{1.287770in}{0.812493in}}%
\pgfpathlineto{\pgfqpoint{1.291191in}{0.623130in}}%
\pgfpathlineto{\pgfqpoint{1.297179in}{0.760563in}}%
\pgfpathlineto{\pgfqpoint{1.298889in}{0.654652in}}%
\pgfpathlineto{\pgfqpoint{1.304883in}{0.781366in}}%
\pgfpathlineto{\pgfqpoint{1.310013in}{0.697838in}}%
\pgfpathlineto{\pgfqpoint{1.311723in}{0.737426in}}%
\pgfpathlineto{\pgfqpoint{1.317712in}{0.680924in}}%
\pgfpathlineto{\pgfqpoint{1.321985in}{0.766323in}}%
\pgfpathlineto{\pgfqpoint{1.326257in}{0.674123in}}%
\pgfpathlineto{\pgfqpoint{1.328822in}{0.752648in}}%
\pgfpathlineto{\pgfqpoint{1.335662in}{0.669623in}}%
\pgfpathlineto{\pgfqpoint{1.337375in}{0.759159in}}%
\pgfpathlineto{\pgfqpoint{1.343354in}{0.667609in}}%
\pgfpathlineto{\pgfqpoint{1.345921in}{0.778737in}}%
\pgfpathlineto{\pgfqpoint{1.351911in}{0.654616in}}%
\pgfpathlineto{\pgfqpoint{1.356186in}{0.791909in}}%
\pgfpathlineto{\pgfqpoint{1.359610in}{0.666205in}}%
\pgfpathlineto{\pgfqpoint{1.363031in}{0.745843in}}%
\pgfpathlineto{\pgfqpoint{1.369869in}{0.625827in}}%
\pgfpathlineto{\pgfqpoint{1.372432in}{0.738790in}}%
\pgfpathlineto{\pgfqpoint{1.378419in}{0.783740in}}%
\pgfpathlineto{\pgfqpoint{1.380129in}{0.701724in}}%
\pgfpathlineto{\pgfqpoint{1.386969in}{0.669013in}}%
\pgfpathlineto{\pgfqpoint{1.387825in}{0.714537in}}%
\pgfpathlineto{\pgfqpoint{1.392960in}{0.773950in}}%
\pgfpathlineto{\pgfqpoint{1.397242in}{0.715183in}}%
\pgfpathlineto{\pgfqpoint{1.400665in}{0.797127in}}%
\pgfpathlineto{\pgfqpoint{1.406651in}{0.699745in}}%
\pgfpathlineto{\pgfqpoint{1.410071in}{0.745412in}}%
\pgfpathlineto{\pgfqpoint{1.413491in}{0.618846in}}%
\pgfpathlineto{\pgfqpoint{1.417764in}{0.744658in}}%
\pgfpathlineto{\pgfqpoint{1.424607in}{0.684235in}}%
\pgfpathlineto{\pgfqpoint{1.426320in}{0.741096in}}%
\pgfpathlineto{\pgfqpoint{1.434016in}{0.688228in}}%
\pgfpathlineto{\pgfqpoint{1.437437in}{0.756135in}}%
\pgfpathlineto{\pgfqpoint{1.439148in}{0.658358in}}%
\pgfpathlineto{\pgfqpoint{1.444278in}{0.799249in}}%
\pgfpathlineto{\pgfqpoint{1.448556in}{0.638780in}}%
\pgfpathlineto{\pgfqpoint{1.453691in}{0.747104in}}%
\pgfpathlineto{\pgfqpoint{1.456259in}{0.662965in}}%
\pgfpathlineto{\pgfqpoint{1.462250in}{0.608837in}}%
\pgfpathlineto{\pgfqpoint{1.467386in}{0.743071in}}%
\pgfpathlineto{\pgfqpoint{1.471667in}{0.742855in}}%
\pgfpathlineto{\pgfqpoint{1.476802in}{0.634679in}}%
\pgfpathlineto{\pgfqpoint{1.479368in}{0.746888in}}%
\pgfpathlineto{\pgfqpoint{1.481936in}{0.681786in}}%
\pgfpathlineto{\pgfqpoint{1.487072in}{0.775857in}}%
\pgfpathlineto{\pgfqpoint{1.491345in}{0.636478in}}%
\pgfpathlineto{\pgfqpoint{1.496476in}{0.761568in}}%
\pgfpathlineto{\pgfqpoint{1.500749in}{0.783843in}}%
\pgfpathlineto{\pgfqpoint{1.504171in}{0.652849in}}%
\pgfpathlineto{\pgfqpoint{1.508448in}{0.717808in}}%
\pgfpathlineto{\pgfqpoint{1.511868in}{0.688480in}}%
\pgfpathlineto{\pgfqpoint{1.518707in}{0.768481in}}%
\pgfpathlineto{\pgfqpoint{1.522982in}{0.768230in}}%
\pgfpathlineto{\pgfqpoint{1.527256in}{0.649434in}}%
\pgfpathlineto{\pgfqpoint{1.530678in}{0.746457in}}%
\pgfpathlineto{\pgfqpoint{1.535810in}{0.684235in}}%
\pgfpathlineto{\pgfqpoint{1.538376in}{0.748795in}}%
\pgfpathlineto{\pgfqpoint{1.541797in}{0.681571in}}%
\pgfpathlineto{\pgfqpoint{1.548638in}{0.743218in}}%
\pgfpathlineto{\pgfqpoint{1.550351in}{0.659511in}}%
\pgfpathlineto{\pgfqpoint{1.556343in}{0.605423in}}%
\pgfpathlineto{\pgfqpoint{1.558909in}{0.713998in}}%
\pgfpathlineto{\pgfqpoint{1.565750in}{0.753582in}}%
\pgfpathlineto{\pgfqpoint{1.567459in}{0.664513in}}%
\pgfpathlineto{\pgfqpoint{1.573455in}{0.624064in}}%
\pgfpathlineto{\pgfqpoint{1.576880in}{0.770352in}}%
\pgfpathlineto{\pgfqpoint{1.581159in}{0.662678in}}%
\pgfpathlineto{\pgfqpoint{1.585439in}{0.612583in}}%
\pgfpathlineto{\pgfqpoint{1.588860in}{0.716228in}}%
\pgfpathlineto{\pgfqpoint{1.599126in}{0.650874in}}%
\pgfpathlineto{\pgfqpoint{1.602549in}{0.755672in}}%
\pgfpathlineto{\pgfqpoint{1.607680in}{0.653431in}}%
\pgfpathlineto{\pgfqpoint{1.611102in}{0.789572in}}%
\pgfpathlineto{\pgfqpoint{1.615378in}{0.687834in}}%
\pgfpathlineto{\pgfqpoint{1.619653in}{0.775642in}}%
\pgfpathlineto{\pgfqpoint{1.623070in}{0.699530in}}%
\pgfpathlineto{\pgfqpoint{1.629917in}{0.644073in}}%
\pgfpathlineto{\pgfqpoint{1.634197in}{0.776974in}}%
\pgfpathlineto{\pgfqpoint{1.637617in}{0.680238in}}%
\pgfpathlineto{\pgfqpoint{1.642745in}{0.787481in}}%
\pgfpathlineto{\pgfqpoint{1.647874in}{0.682688in}}%
\pgfpathlineto{\pgfqpoint{1.648727in}{0.761177in}}%
\pgfpathlineto{\pgfqpoint{1.656428in}{0.684092in}}%
\pgfpathlineto{\pgfqpoint{1.660704in}{0.730262in}}%
\pgfpathlineto{\pgfqpoint{1.661558in}{0.661059in}}%
\pgfpathlineto{\pgfqpoint{1.669255in}{0.699171in}}%
\pgfpathlineto{\pgfqpoint{1.671820in}{0.809146in}}%
\pgfpathlineto{\pgfqpoint{1.674387in}{0.678044in}}%
\pgfpathlineto{\pgfqpoint{1.679521in}{0.637236in}}%
\pgfpathlineto{\pgfqpoint{1.683795in}{0.752469in}}%
\pgfpathlineto{\pgfqpoint{1.688070in}{0.652961in}}%
\pgfpathlineto{\pgfqpoint{1.694908in}{0.747933in}}%
\pgfpathlineto{\pgfqpoint{1.696620in}{0.663939in}}%
\pgfpathlineto{\pgfqpoint{1.700041in}{0.780321in}}%
\pgfpathlineto{\pgfqpoint{1.705169in}{0.638999in}}%
\pgfpathlineto{\pgfqpoint{1.711157in}{0.751747in}}%
\pgfpathlineto{\pgfqpoint{1.713721in}{0.666851in}}%
\pgfpathlineto{\pgfqpoint{1.719709in}{0.791909in}}%
\pgfpathlineto{\pgfqpoint{1.721421in}{0.735551in}}%
\pgfpathlineto{\pgfqpoint{1.726553in}{0.682073in}}%
\pgfpathlineto{\pgfqpoint{1.733392in}{0.682759in}}%
\pgfpathlineto{\pgfqpoint{1.736816in}{0.777118in}}%
\pgfpathlineto{\pgfqpoint{1.738528in}{0.710324in}}%
\pgfpathlineto{\pgfqpoint{1.743662in}{0.771685in}}%
\pgfpathlineto{\pgfqpoint{1.747088in}{0.716228in}}%
\pgfpathlineto{\pgfqpoint{1.753081in}{0.665343in}}%
\pgfpathlineto{\pgfqpoint{1.755646in}{0.764448in}}%
\pgfpathlineto{\pgfqpoint{1.759920in}{0.657820in}}%
\pgfpathlineto{\pgfqpoint{1.765049in}{0.724685in}}%
\pgfpathlineto{\pgfqpoint{1.771888in}{0.659726in}}%
\pgfpathlineto{\pgfqpoint{1.774452in}{0.761823in}}%
\pgfpathlineto{\pgfqpoint{1.780436in}{0.641408in}}%
\pgfpathlineto{\pgfqpoint{1.783002in}{0.735120in}}%
\pgfpathlineto{\pgfqpoint{1.787279in}{0.699422in}}%
\pgfpathlineto{\pgfqpoint{1.793271in}{0.782048in}}%
\pgfpathlineto{\pgfqpoint{1.794127in}{0.682544in}}%
\pgfpathlineto{\pgfqpoint{1.800118in}{0.743936in}}%
\pgfpathlineto{\pgfqpoint{1.805246in}{0.686717in}}%
\pgfpathlineto{\pgfqpoint{1.808671in}{0.813897in}}%
\pgfpathlineto{\pgfqpoint{1.814659in}{0.695860in}}%
\pgfpathlineto{\pgfqpoint{1.818933in}{0.787625in}}%
\pgfpathlineto{\pgfqpoint{1.820647in}{0.685600in}}%
\pgfpathlineto{\pgfqpoint{1.825776in}{0.752824in}}%
\pgfpathlineto{\pgfqpoint{1.828343in}{0.692437in}}%
\pgfpathlineto{\pgfqpoint{1.832622in}{0.760994in}}%
\pgfpathlineto{\pgfqpoint{1.836895in}{0.675850in}}%
\pgfpathlineto{\pgfqpoint{1.842028in}{0.777405in}}%
\pgfpathlineto{\pgfqpoint{1.845449in}{0.690243in}}%
\pgfpathlineto{\pgfqpoint{1.851438in}{0.762940in}}%
\pgfpathlineto{\pgfqpoint{1.856567in}{0.642166in}}%
\pgfpathlineto{\pgfqpoint{1.859133in}{0.709103in}}%
\pgfpathlineto{\pgfqpoint{1.863412in}{0.657820in}}%
\pgfpathlineto{\pgfqpoint{1.867689in}{0.773807in}}%
\pgfpathlineto{\pgfqpoint{1.872817in}{0.834984in}}%
\pgfpathlineto{\pgfqpoint{1.877947in}{0.693123in}}%
\pgfpathlineto{\pgfqpoint{1.879659in}{0.760132in}}%
\pgfpathlineto{\pgfqpoint{1.886505in}{0.667681in}}%
\pgfpathlineto{\pgfqpoint{1.890783in}{0.780429in}}%
\pgfpathlineto{\pgfqpoint{1.895915in}{0.648932in}}%
\pgfpathlineto{\pgfqpoint{1.896771in}{0.781043in}}%
\pgfpathlineto{\pgfqpoint{1.901051in}{0.672539in}}%
\pgfpathlineto{\pgfqpoint{1.907894in}{0.726520in}}%
\pgfpathlineto{\pgfqpoint{1.911319in}{0.662319in}}%
\pgfpathlineto{\pgfqpoint{1.915593in}{0.771397in}}%
\pgfpathlineto{\pgfqpoint{1.919018in}{0.696506in}}%
\pgfpathlineto{\pgfqpoint{1.923300in}{0.752716in}}%
\pgfpathlineto{\pgfqpoint{1.930141in}{0.743250in}}%
\pgfpathlineto{\pgfqpoint{1.932706in}{0.684160in}}%
\pgfpathlineto{\pgfqpoint{1.937837in}{0.658035in}}%
\pgfpathlineto{\pgfqpoint{1.942970in}{0.782008in}}%
\pgfpathlineto{\pgfqpoint{1.946392in}{0.698556in}}%
\pgfpathlineto{\pgfqpoint{1.948960in}{0.802273in}}%
\pgfpathlineto{\pgfqpoint{1.953237in}{0.693123in}}%
\pgfpathlineto{\pgfqpoint{1.956661in}{0.764053in}}%
\pgfpathlineto{\pgfqpoint{1.964363in}{0.780716in}}%
\pgfpathlineto{\pgfqpoint{1.965220in}{0.676712in}}%
\pgfpathlineto{\pgfqpoint{1.971208in}{0.629425in}}%
\pgfpathlineto{\pgfqpoint{1.973771in}{0.790864in}}%
\pgfpathlineto{\pgfqpoint{1.980610in}{0.692078in}}%
\pgfpathlineto{\pgfqpoint{1.984031in}{0.758329in}}%
\pgfpathlineto{\pgfqpoint{1.989162in}{0.677394in}}%
\pgfpathlineto{\pgfqpoint{1.992584in}{0.755345in}}%
\pgfpathlineto{\pgfqpoint{1.997716in}{0.783125in}}%
\pgfpathlineto{\pgfqpoint{1.999427in}{0.702657in}}%
\pgfpathlineto{\pgfqpoint{2.003702in}{0.633095in}}%
\pgfpathlineto{\pgfqpoint{2.007981in}{0.786692in}}%
\pgfpathlineto{\pgfqpoint{2.013113in}{0.694021in}}%
\pgfpathlineto{\pgfqpoint{2.019100in}{0.829335in}}%
\pgfpathlineto{\pgfqpoint{2.022522in}{0.658071in}}%
\pgfpathlineto{\pgfqpoint{2.027657in}{0.785144in}}%
\pgfpathlineto{\pgfqpoint{2.031932in}{0.702553in}}%
\pgfpathlineto{\pgfqpoint{2.035355in}{0.800151in}}%
\pgfpathlineto{\pgfqpoint{2.038775in}{0.714860in}}%
\pgfpathlineto{\pgfqpoint{2.043051in}{0.758584in}}%
\pgfpathlineto{\pgfqpoint{2.046473in}{0.666923in}}%
\pgfpathlineto{\pgfqpoint{2.051607in}{0.760276in}}%
\pgfpathlineto{\pgfqpoint{2.055027in}{0.692908in}}%
\pgfpathlineto{\pgfqpoint{2.060159in}{0.680202in}}%
\pgfpathlineto{\pgfqpoint{2.064434in}{0.752034in}}%
\pgfpathlineto{\pgfqpoint{2.068710in}{0.690387in}}%
\pgfpathlineto{\pgfqpoint{2.072986in}{0.766068in}}%
\pgfpathlineto{\pgfqpoint{2.078974in}{0.702801in}}%
\pgfpathlineto{\pgfqpoint{2.080685in}{0.767113in}}%
\pgfpathlineto{\pgfqpoint{2.085812in}{0.714537in}}%
\pgfpathlineto{\pgfqpoint{2.091797in}{0.792268in}}%
\pgfpathlineto{\pgfqpoint{2.095221in}{0.688300in}}%
\pgfpathlineto{\pgfqpoint{2.099502in}{0.689956in}}%
\pgfpathlineto{\pgfqpoint{2.102927in}{0.817136in}}%
\pgfpathlineto{\pgfqpoint{2.107208in}{0.680278in}}%
\pgfpathlineto{\pgfqpoint{2.114054in}{0.662858in}}%
\pgfpathlineto{\pgfqpoint{2.114910in}{0.816777in}}%
\pgfpathlineto{\pgfqpoint{2.121752in}{0.795220in}}%
\pgfpathlineto{\pgfqpoint{2.124318in}{0.693267in}}%
\pgfpathlineto{\pgfqpoint{2.128598in}{0.740554in}}%
\pgfpathlineto{\pgfqpoint{2.135446in}{0.649973in}}%
\pgfpathlineto{\pgfqpoint{2.139720in}{0.837182in}}%
\pgfpathlineto{\pgfqpoint{2.141431in}{0.704029in}}%
\pgfpathlineto{\pgfqpoint{2.146558in}{0.826710in}}%
\pgfpathlineto{\pgfqpoint{2.149976in}{0.672683in}}%
\pgfpathlineto{\pgfqpoint{2.155105in}{0.729903in}}%
\pgfpathlineto{\pgfqpoint{2.157668in}{0.682185in}}%
\pgfpathlineto{\pgfqpoint{2.161947in}{0.633063in}}%
\pgfpathlineto{\pgfqpoint{2.169647in}{0.667322in}}%
\pgfpathlineto{\pgfqpoint{2.171357in}{0.855679in}}%
\pgfpathlineto{\pgfqpoint{2.177349in}{0.715438in}}%
\pgfpathlineto{\pgfqpoint{2.182483in}{0.665163in}}%
\pgfpathlineto{\pgfqpoint{2.184193in}{0.757794in}}%
\pgfpathlineto{\pgfqpoint{2.191037in}{0.689453in}}%
\pgfpathlineto{\pgfqpoint{2.192747in}{0.785575in}}%
\pgfpathlineto{\pgfqpoint{2.199592in}{0.687618in}}%
\pgfpathlineto{\pgfqpoint{2.202160in}{0.752864in}}%
\pgfpathlineto{\pgfqpoint{2.205581in}{0.678156in}}%
\pgfpathlineto{\pgfqpoint{2.211570in}{0.786835in}}%
\pgfpathlineto{\pgfqpoint{2.216701in}{0.780357in}}%
\pgfpathlineto{\pgfqpoint{2.217556in}{0.671315in}}%
\pgfpathlineto{\pgfqpoint{2.223541in}{0.750343in}}%
\pgfpathlineto{\pgfqpoint{2.229530in}{0.657496in}}%
\pgfpathlineto{\pgfqpoint{2.231239in}{0.766969in}}%
\pgfpathlineto{\pgfqpoint{2.238082in}{0.687762in}}%
\pgfpathlineto{\pgfqpoint{2.242359in}{0.800366in}}%
\pgfpathlineto{\pgfqpoint{2.243215in}{0.735914in}}%
\pgfpathlineto{\pgfqpoint{2.250909in}{0.676608in}}%
\pgfpathlineto{\pgfqpoint{2.252617in}{0.767257in}}%
\pgfpathlineto{\pgfqpoint{2.256038in}{0.799576in}}%
\pgfpathlineto{\pgfqpoint{2.261169in}{0.663508in}}%
\pgfpathlineto{\pgfqpoint{2.264591in}{0.712881in}}%
\pgfpathlineto{\pgfqpoint{2.271435in}{0.748041in}}%
\pgfpathlineto{\pgfqpoint{2.273147in}{0.625971in}}%
\pgfpathlineto{\pgfqpoint{2.278277in}{0.774349in}}%
\pgfpathlineto{\pgfqpoint{2.281703in}{0.776471in}}%
\pgfpathlineto{\pgfqpoint{2.291118in}{0.632521in}}%
\pgfpathlineto{\pgfqpoint{2.294543in}{0.795364in}}%
\pgfpathlineto{\pgfqpoint{2.298819in}{0.706223in}}%
\pgfpathlineto{\pgfqpoint{2.304810in}{0.704855in}}%
\pgfpathlineto{\pgfqpoint{2.309087in}{0.840636in}}%
\pgfpathlineto{\pgfqpoint{2.312504in}{0.661777in}}%
\pgfpathlineto{\pgfqpoint{2.316782in}{0.766251in}}%
\pgfpathlineto{\pgfqpoint{2.323623in}{0.632162in}}%
\pgfpathlineto{\pgfqpoint{2.327045in}{0.717161in}}%
\pgfpathlineto{\pgfqpoint{2.330467in}{0.651453in}}%
\pgfpathlineto{\pgfqpoint{2.333891in}{0.631874in}}%
\pgfpathlineto{\pgfqpoint{2.341587in}{0.802488in}}%
\pgfpathlineto{\pgfqpoint{2.346717in}{0.651848in}}%
\pgfpathlineto{\pgfqpoint{2.351849in}{0.758692in}}%
\pgfpathlineto{\pgfqpoint{2.357839in}{0.687834in}}%
\pgfpathlineto{\pgfqpoint{2.359552in}{0.769451in}}%
\pgfpathlineto{\pgfqpoint{2.364686in}{0.668830in}}%
\pgfpathlineto{\pgfqpoint{2.368106in}{0.805835in}}%
\pgfpathlineto{\pgfqpoint{2.371529in}{0.712518in}}%
\pgfpathlineto{\pgfqpoint{2.376661in}{0.770280in}}%
\pgfpathlineto{\pgfqpoint{2.380084in}{0.732172in}}%
\pgfpathlineto{\pgfqpoint{2.385215in}{0.650982in}}%
\pgfpathlineto{\pgfqpoint{2.391202in}{0.772115in}}%
\pgfpathlineto{\pgfqpoint{2.394625in}{0.667824in}}%
\pgfpathlineto{\pgfqpoint{2.397190in}{0.747574in}}%
\pgfpathlineto{\pgfqpoint{2.403176in}{0.689956in}}%
\pgfpathlineto{\pgfqpoint{2.408306in}{0.752214in}}%
\pgfpathlineto{\pgfqpoint{2.410871in}{0.679592in}}%
\pgfpathlineto{\pgfqpoint{2.415143in}{0.774453in}}%
\pgfpathlineto{\pgfqpoint{2.420276in}{0.689525in}}%
\pgfpathlineto{\pgfqpoint{2.425405in}{0.784713in}}%
\pgfpathlineto{\pgfqpoint{2.429680in}{0.618415in}}%
\pgfpathlineto{\pgfqpoint{2.432245in}{0.731522in}}%
\pgfpathlineto{\pgfqpoint{2.435666in}{0.790505in}}%
\pgfpathlineto{\pgfqpoint{2.442507in}{0.783125in}}%
\pgfpathlineto{\pgfqpoint{2.445931in}{0.687977in}}%
\pgfpathlineto{\pgfqpoint{2.451058in}{0.642597in}}%
\pgfpathlineto{\pgfqpoint{2.452770in}{0.721482in}}%
\pgfpathlineto{\pgfqpoint{2.457051in}{0.773232in}}%
\pgfpathlineto{\pgfqpoint{2.461330in}{0.706367in}}%
\pgfpathlineto{\pgfqpoint{2.465606in}{0.780285in}}%
\pgfpathlineto{\pgfqpoint{2.472449in}{0.676209in}}%
\pgfpathlineto{\pgfqpoint{2.475873in}{0.767185in}}%
\pgfpathlineto{\pgfqpoint{2.481860in}{0.639143in}}%
\pgfpathlineto{\pgfqpoint{2.482715in}{0.745556in}}%
\pgfpathlineto{\pgfqpoint{2.489559in}{0.634930in}}%
\pgfpathlineto{\pgfqpoint{2.492125in}{0.743035in}}%
\pgfpathlineto{\pgfqpoint{2.497259in}{0.660301in}}%
\pgfpathlineto{\pgfqpoint{2.502393in}{0.760132in}}%
\pgfpathlineto{\pgfqpoint{2.504105in}{0.690243in}}%
\pgfpathlineto{\pgfqpoint{2.511803in}{0.801623in}}%
\pgfpathlineto{\pgfqpoint{2.514369in}{0.667210in}}%
\pgfpathlineto{\pgfqpoint{2.518648in}{0.649937in}}%
\pgfpathlineto{\pgfqpoint{2.522070in}{0.771286in}}%
\pgfpathlineto{\pgfqpoint{2.528054in}{0.665519in}}%
\pgfpathlineto{\pgfqpoint{2.529766in}{0.731841in}}%
\pgfpathlineto{\pgfqpoint{2.536611in}{0.687219in}}%
\pgfpathlineto{\pgfqpoint{2.540888in}{0.762470in}}%
\pgfpathlineto{\pgfqpoint{2.542599in}{0.654939in}}%
\pgfpathlineto{\pgfqpoint{2.546873in}{0.765960in}}%
\pgfpathlineto{\pgfqpoint{2.552006in}{0.684666in}}%
\pgfpathlineto{\pgfqpoint{2.556277in}{0.769379in}}%
\pgfpathlineto{\pgfqpoint{2.560553in}{0.641624in}}%
\pgfpathlineto{\pgfqpoint{2.563974in}{0.683118in}}%
\pgfpathlineto{\pgfqpoint{2.569104in}{0.766642in}}%
\pgfpathlineto{\pgfqpoint{2.572527in}{0.694886in}}%
\pgfpathlineto{\pgfqpoint{2.576804in}{0.769092in}}%
\pgfpathlineto{\pgfqpoint{2.584495in}{0.663648in}}%
\pgfpathlineto{\pgfqpoint{2.585351in}{0.764592in}}%
\pgfpathlineto{\pgfqpoint{2.596466in}{0.679879in}}%
\pgfpathlineto{\pgfqpoint{2.598175in}{0.752393in}}%
\pgfpathlineto{\pgfqpoint{2.602452in}{0.687794in}}%
\pgfpathlineto{\pgfqpoint{2.609301in}{0.653463in}}%
\pgfpathlineto{\pgfqpoint{2.614436in}{0.824405in}}%
\pgfpathlineto{\pgfqpoint{2.615290in}{0.727094in}}%
\pgfpathlineto{\pgfqpoint{2.621275in}{0.672467in}}%
\pgfpathlineto{\pgfqpoint{2.627264in}{0.770137in}}%
\pgfpathlineto{\pgfqpoint{2.631542in}{0.663289in}}%
\pgfpathlineto{\pgfqpoint{2.635823in}{0.820878in}}%
\pgfpathlineto{\pgfqpoint{2.637532in}{0.686214in}}%
\pgfpathlineto{\pgfqpoint{2.642664in}{0.744152in}}%
\pgfpathlineto{\pgfqpoint{2.646085in}{0.669300in}}%
\pgfpathlineto{\pgfqpoint{2.651221in}{0.773192in}}%
\pgfpathlineto{\pgfqpoint{2.655500in}{0.705322in}}%
\pgfpathlineto{\pgfqpoint{2.658066in}{0.805656in}}%
\pgfpathlineto{\pgfqpoint{2.663199in}{0.710037in}}%
\pgfpathlineto{\pgfqpoint{2.668333in}{0.807419in}}%
\pgfpathlineto{\pgfqpoint{2.671754in}{0.683729in}}%
\pgfpathlineto{\pgfqpoint{2.675176in}{0.733932in}}%
\pgfpathlineto{\pgfqpoint{2.679450in}{0.688623in}}%
\pgfpathlineto{\pgfqpoint{2.685443in}{0.762075in}}%
\pgfpathlineto{\pgfqpoint{2.688863in}{0.686717in}}%
\pgfpathlineto{\pgfqpoint{2.693143in}{0.793816in}}%
\pgfpathlineto{\pgfqpoint{2.696565in}{0.716874in}}%
\pgfpathlineto{\pgfqpoint{2.702555in}{0.751420in}}%
\pgfpathlineto{\pgfqpoint{2.706832in}{0.678116in}}%
\pgfpathlineto{\pgfqpoint{2.711109in}{0.791439in}}%
\pgfpathlineto{\pgfqpoint{2.713672in}{0.689884in}}%
\pgfpathlineto{\pgfqpoint{2.719654in}{0.814831in}}%
\pgfpathlineto{\pgfqpoint{2.722221in}{0.722236in}}%
\pgfpathlineto{\pgfqpoint{2.728213in}{0.764017in}}%
\pgfpathlineto{\pgfqpoint{2.734203in}{0.797773in}}%
\pgfpathlineto{\pgfqpoint{2.737626in}{0.648318in}}%
\pgfpathlineto{\pgfqpoint{2.739338in}{0.734833in}}%
\pgfpathlineto{\pgfqpoint{2.746179in}{0.771397in}}%
\pgfpathlineto{\pgfqpoint{2.748744in}{0.683047in}}%
\pgfpathlineto{\pgfqpoint{2.754727in}{0.760922in}}%
\pgfpathlineto{\pgfqpoint{2.758145in}{0.788056in}}%
\pgfpathlineto{\pgfqpoint{2.760706in}{0.699781in}}%
\pgfpathlineto{\pgfqpoint{2.764980in}{0.783197in}}%
\pgfpathlineto{\pgfqpoint{2.769259in}{0.700535in}}%
\pgfpathlineto{\pgfqpoint{2.776955in}{0.686102in}}%
\pgfpathlineto{\pgfqpoint{2.778666in}{0.787083in}}%
\pgfpathlineto{\pgfqpoint{2.782089in}{0.699853in}}%
\pgfpathlineto{\pgfqpoint{2.789789in}{0.624351in}}%
\pgfpathlineto{\pgfqpoint{2.790642in}{0.752106in}}%
\pgfpathlineto{\pgfqpoint{2.794918in}{0.702410in}}%
\pgfpathlineto{\pgfqpoint{2.799189in}{0.795795in}}%
\pgfpathlineto{\pgfqpoint{2.803464in}{0.690606in}}%
\pgfpathlineto{\pgfqpoint{2.809450in}{0.775714in}}%
\pgfpathlineto{\pgfqpoint{2.813729in}{0.691033in}}%
\pgfpathlineto{\pgfqpoint{2.818861in}{0.662463in}}%
\pgfpathlineto{\pgfqpoint{2.820569in}{0.736596in}}%
\pgfpathlineto{\pgfqpoint{2.827409in}{0.652961in}}%
\pgfpathlineto{\pgfqpoint{2.832540in}{0.745340in}}%
\pgfpathlineto{\pgfqpoint{2.833396in}{0.682217in}}%
\pgfpathlineto{\pgfqpoint{2.839382in}{0.760994in}}%
\pgfpathlineto{\pgfqpoint{2.845369in}{0.678870in}}%
\pgfpathlineto{\pgfqpoint{2.847078in}{0.750774in}}%
\pgfpathlineto{\pgfqpoint{2.850497in}{0.682037in}}%
\pgfpathlineto{\pgfqpoint{2.855627in}{0.772474in}}%
\pgfpathlineto{\pgfqpoint{2.859900in}{0.649470in}}%
\pgfpathlineto{\pgfqpoint{2.865887in}{0.776862in}}%
\pgfpathlineto{\pgfqpoint{2.867599in}{0.729719in}}%
\pgfpathlineto{\pgfqpoint{2.874440in}{0.630686in}}%
\pgfpathlineto{\pgfqpoint{2.876150in}{0.725981in}}%
\pgfpathlineto{\pgfqpoint{2.880425in}{0.674949in}}%
\pgfpathlineto{\pgfqpoint{2.884702in}{0.770424in}}%
\pgfpathlineto{\pgfqpoint{2.891547in}{0.788096in}}%
\pgfpathlineto{\pgfqpoint{2.896681in}{0.694886in}}%
\pgfpathlineto{\pgfqpoint{2.897536in}{0.795723in}}%
\pgfpathlineto{\pgfqpoint{2.905233in}{0.795005in}}%
\pgfpathlineto{\pgfqpoint{2.906944in}{0.709750in}}%
\pgfpathlineto{\pgfqpoint{2.910364in}{0.751962in}}%
\pgfpathlineto{\pgfqpoint{2.918060in}{0.706151in}}%
\pgfpathlineto{\pgfqpoint{2.918914in}{0.776288in}}%
\pgfpathlineto{\pgfqpoint{2.927463in}{0.665231in}}%
\pgfpathlineto{\pgfqpoint{2.931738in}{0.760308in}}%
\pgfpathlineto{\pgfqpoint{2.936873in}{0.694487in}}%
\pgfpathlineto{\pgfqpoint{2.942857in}{0.685528in}}%
\pgfpathlineto{\pgfqpoint{2.944568in}{0.778374in}}%
\pgfpathlineto{\pgfqpoint{2.948846in}{0.697475in}}%
\pgfpathlineto{\pgfqpoint{2.955685in}{0.759949in}}%
\pgfpathlineto{\pgfqpoint{2.959964in}{0.606934in}}%
\pgfpathlineto{\pgfqpoint{2.964240in}{0.791510in}}%
\pgfpathlineto{\pgfqpoint{2.968519in}{0.842866in}}%
\pgfpathlineto{\pgfqpoint{2.970229in}{0.708166in}}%
\pgfpathlineto{\pgfqpoint{2.977074in}{0.701077in}}%
\pgfpathlineto{\pgfqpoint{2.978784in}{0.764089in}}%
\pgfpathlineto{\pgfqpoint{2.983062in}{0.697080in}}%
\pgfpathlineto{\pgfqpoint{2.988194in}{0.750630in}}%
\pgfpathlineto{\pgfqpoint{2.994182in}{0.699745in}}%
\pgfpathlineto{\pgfqpoint{2.999311in}{0.785000in}}%
\pgfpathlineto{\pgfqpoint{3.002736in}{0.701149in}}%
\pgfpathlineto{\pgfqpoint{3.007011in}{0.815301in}}%
\pgfpathlineto{\pgfqpoint{3.011287in}{0.686717in}}%
\pgfpathlineto{\pgfqpoint{3.013853in}{0.774995in}}%
\pgfpathlineto{\pgfqpoint{3.019839in}{0.792232in}}%
\pgfpathlineto{\pgfqpoint{3.021551in}{0.717560in}}%
\pgfpathlineto{\pgfqpoint{3.029249in}{0.612695in}}%
\pgfpathlineto{\pgfqpoint{3.030958in}{0.823686in}}%
\pgfpathlineto{\pgfqpoint{3.034379in}{0.741958in}}%
\pgfpathlineto{\pgfqpoint{3.039507in}{0.806701in}}%
\pgfpathlineto{\pgfqpoint{3.042930in}{0.716192in}}%
\pgfpathlineto{\pgfqpoint{3.050630in}{0.662535in}}%
\pgfpathlineto{\pgfqpoint{3.054053in}{0.812421in}}%
\pgfpathlineto{\pgfqpoint{3.059185in}{0.808895in}}%
\pgfpathlineto{\pgfqpoint{3.061754in}{0.684558in}}%
\pgfpathlineto{\pgfqpoint{3.064319in}{0.744371in}}%
\pgfpathlineto{\pgfqpoint{3.072018in}{0.681714in}}%
\pgfpathlineto{\pgfqpoint{3.072875in}{0.723424in}}%
\pgfpathlineto{\pgfqpoint{3.079716in}{0.757324in}}%
\pgfpathlineto{\pgfqpoint{3.084849in}{0.664370in}}%
\pgfpathlineto{\pgfqpoint{3.085704in}{0.774457in}}%
\pgfpathlineto{\pgfqpoint{3.090835in}{0.706511in}}%
\pgfpathlineto{\pgfqpoint{3.094255in}{0.765063in}}%
\pgfpathlineto{\pgfqpoint{3.099388in}{0.657923in}}%
\pgfpathlineto{\pgfqpoint{3.103662in}{0.748037in}}%
\pgfpathlineto{\pgfqpoint{3.110508in}{0.756638in}}%
\pgfpathlineto{\pgfqpoint{3.112219in}{0.677358in}}%
\pgfpathlineto{\pgfqpoint{3.117359in}{0.767903in}}%
\pgfpathlineto{\pgfqpoint{3.122495in}{0.755704in}}%
\pgfpathlineto{\pgfqpoint{3.126770in}{0.632521in}}%
\pgfpathlineto{\pgfqpoint{3.128480in}{0.722092in}}%
\pgfpathlineto{\pgfqpoint{3.135323in}{0.804938in}}%
\pgfpathlineto{\pgfqpoint{3.138744in}{0.671494in}}%
\pgfpathlineto{\pgfqpoint{3.143019in}{0.741886in}}%
\pgfpathlineto{\pgfqpoint{3.146436in}{0.685280in}}%
\pgfpathlineto{\pgfqpoint{3.151568in}{0.748687in}}%
\pgfpathlineto{\pgfqpoint{3.154987in}{0.646195in}}%
\pgfpathlineto{\pgfqpoint{3.161832in}{0.767296in}}%
\pgfpathlineto{\pgfqpoint{3.162686in}{0.725730in}}%
\pgfpathlineto{\pgfqpoint{3.166956in}{0.775570in}}%
\pgfpathlineto{\pgfqpoint{3.171229in}{0.657460in}}%
\pgfpathlineto{\pgfqpoint{3.176360in}{0.781761in}}%
\pgfpathlineto{\pgfqpoint{3.181495in}{0.704676in}}%
\pgfpathlineto{\pgfqpoint{3.184919in}{0.785216in}}%
\pgfpathlineto{\pgfqpoint{3.189194in}{0.692872in}}%
\pgfpathlineto{\pgfqpoint{3.195182in}{0.771469in}}%
\pgfpathlineto{\pgfqpoint{3.196893in}{0.685169in}}%
\pgfpathlineto{\pgfqpoint{3.202029in}{0.784242in}}%
\pgfpathlineto{\pgfqpoint{3.206306in}{0.793457in}}%
\pgfpathlineto{\pgfqpoint{3.213149in}{0.594951in}}%
\pgfpathlineto{\pgfqpoint{3.214004in}{0.723640in}}%
\pgfpathlineto{\pgfqpoint{3.218280in}{0.756031in}}%
\pgfpathlineto{\pgfqpoint{3.222557in}{0.645908in}}%
\pgfpathlineto{\pgfqpoint{3.226834in}{0.724505in}}%
\pgfpathlineto{\pgfqpoint{3.234531in}{0.769993in}}%
\pgfpathlineto{\pgfqpoint{3.237098in}{0.678188in}}%
\pgfpathlineto{\pgfqpoint{3.243083in}{0.693845in}}%
\pgfpathlineto{\pgfqpoint{3.243939in}{0.791622in}}%
\pgfpathlineto{\pgfqpoint{3.252486in}{0.637272in}}%
\pgfpathlineto{\pgfqpoint{3.256765in}{0.763443in}}%
\pgfpathlineto{\pgfqpoint{3.263612in}{0.675922in}}%
\pgfpathlineto{\pgfqpoint{3.266171in}{0.816091in}}%
\pgfpathlineto{\pgfqpoint{3.269594in}{0.710795in}}%
\pgfpathlineto{\pgfqpoint{3.274725in}{0.783740in}}%
\pgfpathlineto{\pgfqpoint{3.278148in}{0.675204in}}%
\pgfpathlineto{\pgfqpoint{3.283283in}{0.783528in}}%
\pgfpathlineto{\pgfqpoint{3.287561in}{0.672288in}}%
\pgfpathlineto{\pgfqpoint{3.291835in}{0.783524in}}%
\pgfpathlineto{\pgfqpoint{3.296964in}{0.670956in}}%
\pgfpathlineto{\pgfqpoint{3.302092in}{0.675635in}}%
\pgfpathlineto{\pgfqpoint{3.306369in}{0.761285in}}%
\pgfpathlineto{\pgfqpoint{3.311495in}{0.782447in}}%
\pgfpathlineto{\pgfqpoint{3.313205in}{0.636446in}}%
\pgfpathlineto{\pgfqpoint{3.316624in}{0.736596in}}%
\pgfpathlineto{\pgfqpoint{3.323472in}{0.786261in}}%
\pgfpathlineto{\pgfqpoint{3.327748in}{0.635688in}}%
\pgfpathlineto{\pgfqpoint{3.332026in}{0.770639in}}%
\pgfpathlineto{\pgfqpoint{3.333737in}{0.710938in}}%
\pgfpathlineto{\pgfqpoint{3.339724in}{0.635113in}}%
\pgfpathlineto{\pgfqpoint{3.345713in}{0.773304in}}%
\pgfpathlineto{\pgfqpoint{3.346570in}{0.694958in}}%
\pgfpathlineto{\pgfqpoint{3.351703in}{0.656383in}}%
\pgfpathlineto{\pgfqpoint{3.355978in}{0.781258in}}%
\pgfpathlineto{\pgfqpoint{3.359402in}{0.689852in}}%
\pgfpathlineto{\pgfqpoint{3.367101in}{0.683445in}}%
\pgfpathlineto{\pgfqpoint{3.367957in}{0.817894in}}%
\pgfpathlineto{\pgfqpoint{3.374800in}{0.665307in}}%
\pgfpathlineto{\pgfqpoint{3.379076in}{0.631443in}}%
\pgfpathlineto{\pgfqpoint{3.380784in}{0.727350in}}%
\pgfpathlineto{\pgfqpoint{3.388483in}{0.775610in}}%
\pgfpathlineto{\pgfqpoint{3.389340in}{0.657931in}}%
\pgfpathlineto{\pgfqpoint{3.397035in}{0.801052in}}%
\pgfpathlineto{\pgfqpoint{3.398746in}{0.670704in}}%
\pgfpathlineto{\pgfqpoint{3.403879in}{0.819298in}}%
\pgfpathlineto{\pgfqpoint{3.407299in}{0.676249in}}%
\pgfpathlineto{\pgfqpoint{3.410720in}{0.742644in}}%
\pgfpathlineto{\pgfqpoint{3.414994in}{0.655231in}}%
\pgfpathlineto{\pgfqpoint{3.419266in}{0.713962in}}%
\pgfpathlineto{\pgfqpoint{3.423538in}{0.689924in}}%
\pgfpathlineto{\pgfqpoint{3.430381in}{0.797454in}}%
\pgfpathlineto{\pgfqpoint{3.433801in}{0.584408in}}%
\pgfpathlineto{\pgfqpoint{3.437223in}{0.753909in}}%
\pgfpathlineto{\pgfqpoint{3.443211in}{0.684742in}}%
\pgfpathlineto{\pgfqpoint{3.444923in}{0.741495in}}%
\pgfpathlineto{\pgfqpoint{3.450907in}{0.805911in}}%
\pgfpathlineto{\pgfqpoint{3.456036in}{0.709678in}}%
\pgfpathlineto{\pgfqpoint{3.457749in}{0.779423in}}%
\pgfpathlineto{\pgfqpoint{3.462026in}{0.621076in}}%
\pgfpathlineto{\pgfqpoint{3.466302in}{0.748400in}}%
\pgfpathlineto{\pgfqpoint{3.470577in}{0.774349in}}%
\pgfpathlineto{\pgfqpoint{3.475705in}{0.681319in}}%
\pgfpathlineto{\pgfqpoint{3.480840in}{0.739548in}}%
\pgfpathlineto{\pgfqpoint{3.483407in}{0.658793in}}%
\pgfpathlineto{\pgfqpoint{3.490249in}{0.762761in}}%
\pgfpathlineto{\pgfqpoint{3.493668in}{0.671319in}}%
\pgfpathlineto{\pgfqpoint{3.498802in}{0.738001in}}%
\pgfpathlineto{\pgfqpoint{3.503934in}{0.596786in}}%
\pgfpathlineto{\pgfqpoint{3.507357in}{0.775785in}}%
\pgfpathlineto{\pgfqpoint{3.509065in}{0.654616in}}%
\pgfpathlineto{\pgfqpoint{3.514195in}{0.793888in}}%
\pgfpathlineto{\pgfqpoint{3.517615in}{0.669300in}}%
\pgfpathlineto{\pgfqpoint{3.522750in}{0.736632in}}%
\pgfpathlineto{\pgfqpoint{3.527884in}{0.657030in}}%
\pgfpathlineto{\pgfqpoint{3.533869in}{0.743721in}}%
\pgfpathlineto{\pgfqpoint{3.536436in}{0.670848in}}%
\pgfpathlineto{\pgfqpoint{3.539857in}{0.778091in}}%
\pgfpathlineto{\pgfqpoint{3.543277in}{0.640403in}}%
\pgfpathlineto{\pgfqpoint{3.548412in}{0.744335in}}%
\pgfpathlineto{\pgfqpoint{3.552684in}{0.678443in}}%
\pgfpathlineto{\pgfqpoint{3.556103in}{0.759849in}}%
\pgfpathlineto{\pgfqpoint{3.561230in}{0.702449in}}%
\pgfpathlineto{\pgfqpoint{3.567217in}{0.682871in}}%
\pgfpathlineto{\pgfqpoint{3.569782in}{0.807925in}}%
\pgfpathlineto{\pgfqpoint{3.574058in}{0.695788in}}%
\pgfpathlineto{\pgfqpoint{3.580044in}{0.805369in}}%
\pgfpathlineto{\pgfqpoint{3.581755in}{0.681032in}}%
\pgfpathlineto{\pgfqpoint{3.586031in}{0.753367in}}%
\pgfpathlineto{\pgfqpoint{3.592017in}{0.673010in}}%
\pgfpathlineto{\pgfqpoint{3.595438in}{0.810913in}}%
\pgfpathlineto{\pgfqpoint{3.598858in}{0.697407in}}%
\pgfpathlineto{\pgfqpoint{3.605706in}{0.610357in}}%
\pgfpathlineto{\pgfqpoint{3.607419in}{0.749122in}}%
\pgfpathlineto{\pgfqpoint{3.611692in}{0.689924in}}%
\pgfpathlineto{\pgfqpoint{3.615968in}{0.734043in}}%
\pgfpathlineto{\pgfqpoint{3.621959in}{0.679628in}}%
\pgfpathlineto{\pgfqpoint{3.627946in}{0.774924in}}%
\pgfpathlineto{\pgfqpoint{3.628798in}{0.694779in}}%
\pgfpathlineto{\pgfqpoint{3.634787in}{0.732783in}}%
\pgfpathlineto{\pgfqpoint{3.640777in}{0.612080in}}%
\pgfpathlineto{\pgfqpoint{3.641632in}{0.740809in}}%
\pgfpathlineto{\pgfqpoint{3.646770in}{0.660843in}}%
\pgfpathlineto{\pgfqpoint{3.652758in}{0.628563in}}%
\pgfpathlineto{\pgfqpoint{3.657887in}{0.761249in}}%
\pgfpathlineto{\pgfqpoint{3.658742in}{0.670704in}}%
\pgfpathlineto{\pgfqpoint{3.663017in}{0.779998in}}%
\pgfpathlineto{\pgfqpoint{3.669860in}{0.675348in}}%
\pgfpathlineto{\pgfqpoint{3.671571in}{0.754771in}}%
\pgfpathlineto{\pgfqpoint{3.678414in}{0.785866in}}%
\pgfpathlineto{\pgfqpoint{3.680124in}{0.670776in}}%
\pgfpathlineto{\pgfqpoint{3.686109in}{0.758297in}}%
\pgfpathlineto{\pgfqpoint{3.689529in}{0.771972in}}%
\pgfpathlineto{\pgfqpoint{3.695519in}{0.701795in}}%
\pgfpathlineto{\pgfqpoint{3.699794in}{0.739692in}}%
\pgfpathlineto{\pgfqpoint{3.701506in}{0.645800in}}%
\pgfpathlineto{\pgfqpoint{3.705784in}{0.725658in}}%
\pgfpathlineto{\pgfqpoint{3.710060in}{0.654365in}}%
\pgfpathlineto{\pgfqpoint{3.714341in}{0.739907in}}%
\pgfpathlineto{\pgfqpoint{3.720332in}{0.673548in}}%
\pgfpathlineto{\pgfqpoint{3.722900in}{0.724003in}}%
\pgfpathlineto{\pgfqpoint{3.730597in}{0.792811in}}%
\pgfpathlineto{\pgfqpoint{3.731453in}{0.676608in}}%
\pgfpathlineto{\pgfqpoint{3.738291in}{0.796768in}}%
\pgfpathlineto{\pgfqpoint{3.739998in}{0.718207in}}%
\pgfpathlineto{\pgfqpoint{3.745129in}{0.684164in}}%
\pgfpathlineto{\pgfqpoint{3.751968in}{0.780612in}}%
\pgfpathlineto{\pgfqpoint{3.756244in}{0.658003in}}%
\pgfpathlineto{\pgfqpoint{3.757100in}{0.755888in}}%
\pgfpathlineto{\pgfqpoint{3.763091in}{0.648537in}}%
\pgfpathlineto{\pgfqpoint{3.767366in}{0.748831in}}%
\pgfpathlineto{\pgfqpoint{3.769934in}{0.675958in}}%
\pgfpathlineto{\pgfqpoint{3.777634in}{0.611865in}}%
\pgfpathlineto{\pgfqpoint{3.779343in}{0.724577in}}%
\pgfpathlineto{\pgfqpoint{3.785331in}{0.754914in}}%
\pgfpathlineto{\pgfqpoint{3.790465in}{0.691687in}}%
\pgfpathlineto{\pgfqpoint{3.792176in}{0.768844in}}%
\pgfpathlineto{\pgfqpoint{3.795597in}{0.696725in}}%
\pgfpathlineto{\pgfqpoint{3.801584in}{0.786907in}}%
\pgfpathlineto{\pgfqpoint{3.804147in}{0.714249in}}%
\pgfpathlineto{\pgfqpoint{3.809278in}{0.671534in}}%
\pgfpathlineto{\pgfqpoint{3.816117in}{0.765102in}}%
\pgfpathlineto{\pgfqpoint{3.819537in}{0.660883in}}%
\pgfpathlineto{\pgfqpoint{3.821247in}{0.745811in}}%
\pgfpathlineto{\pgfqpoint{3.825520in}{0.673010in}}%
\pgfpathlineto{\pgfqpoint{3.829795in}{0.658721in}}%
\pgfpathlineto{\pgfqpoint{3.834067in}{0.754124in}}%
\pgfpathlineto{\pgfqpoint{3.841763in}{0.780971in}}%
\pgfpathlineto{\pgfqpoint{3.842619in}{0.688053in}}%
\pgfpathlineto{\pgfqpoint{3.847744in}{0.646774in}}%
\pgfpathlineto{\pgfqpoint{3.851170in}{0.748907in}}%
\pgfpathlineto{\pgfqpoint{3.857163in}{0.652570in}}%
\pgfpathlineto{\pgfqpoint{3.859732in}{0.734905in}}%
\pgfpathlineto{\pgfqpoint{3.866582in}{0.675419in}}%
\pgfpathlineto{\pgfqpoint{3.870855in}{0.748260in}}%
\pgfpathlineto{\pgfqpoint{3.873418in}{0.654405in}}%
\pgfpathlineto{\pgfqpoint{3.880263in}{0.666496in}}%
\pgfpathlineto{\pgfqpoint{3.884540in}{0.750957in}}%
\pgfpathlineto{\pgfqpoint{3.885396in}{0.658577in}}%
\pgfpathlineto{\pgfqpoint{3.891380in}{0.763914in}}%
\pgfpathlineto{\pgfqpoint{3.894800in}{0.663041in}}%
\pgfpathlineto{\pgfqpoint{3.899075in}{0.750203in}}%
\pgfpathlineto{\pgfqpoint{3.905916in}{0.785184in}}%
\pgfpathlineto{\pgfqpoint{3.908484in}{0.659335in}}%
\pgfpathlineto{\pgfqpoint{3.912763in}{0.742827in}}%
\pgfpathlineto{\pgfqpoint{3.917037in}{0.775035in}}%
\pgfpathlineto{\pgfqpoint{3.923020in}{0.692912in}}%
\pgfpathlineto{\pgfqpoint{3.924731in}{0.770320in}}%
\pgfpathlineto{\pgfqpoint{3.929865in}{0.648900in}}%
\pgfpathlineto{\pgfqpoint{3.935852in}{0.796808in}}%
\pgfpathlineto{\pgfqpoint{3.939274in}{0.707990in}}%
\pgfpathlineto{\pgfqpoint{3.943555in}{0.746425in}}%
\pgfpathlineto{\pgfqpoint{3.947833in}{0.630725in}}%
\pgfpathlineto{\pgfqpoint{3.952966in}{0.727421in}}%
\pgfpathlineto{\pgfqpoint{3.953820in}{0.639972in}}%
\pgfpathlineto{\pgfqpoint{3.958954in}{0.771254in}}%
\pgfpathlineto{\pgfqpoint{3.962375in}{0.664840in}}%
\pgfpathlineto{\pgfqpoint{3.969212in}{0.666388in}}%
\pgfpathlineto{\pgfqpoint{3.970924in}{0.736923in}}%
\pgfpathlineto{\pgfqpoint{3.976053in}{0.691001in}}%
\pgfpathlineto{\pgfqpoint{3.980331in}{0.749553in}}%
\pgfpathlineto{\pgfqpoint{3.985468in}{0.704500in}}%
\pgfpathlineto{\pgfqpoint{3.991456in}{0.774604in}}%
\pgfpathlineto{\pgfqpoint{3.993166in}{0.679596in}}%
\pgfpathlineto{\pgfqpoint{3.996588in}{0.753550in}}%
\pgfpathlineto{\pgfqpoint{4.004292in}{0.684060in}}%
\pgfpathlineto{\pgfqpoint{4.006858in}{0.751101in}}%
\pgfpathlineto{\pgfqpoint{4.011132in}{0.782734in}}%
\pgfpathlineto{\pgfqpoint{4.017123in}{0.645585in}}%
\pgfpathlineto{\pgfqpoint{4.017979in}{0.772730in}}%
\pgfpathlineto{\pgfqpoint{4.022252in}{0.689238in}}%
\pgfpathlineto{\pgfqpoint{4.027385in}{0.761967in}}%
\pgfpathlineto{\pgfqpoint{4.030804in}{0.671929in}}%
\pgfpathlineto{\pgfqpoint{4.039357in}{0.759270in}}%
\pgfpathlineto{\pgfqpoint{4.044492in}{0.688807in}}%
\pgfpathlineto{\pgfqpoint{4.049623in}{0.652315in}}%
\pgfpathlineto{\pgfqpoint{4.054759in}{0.821524in}}%
\pgfpathlineto{\pgfqpoint{4.058181in}{0.706008in}}%
\pgfpathlineto{\pgfqpoint{4.064169in}{0.762474in}}%
\pgfpathlineto{\pgfqpoint{4.068446in}{0.609061in}}%
\pgfpathlineto{\pgfqpoint{4.069301in}{0.725156in}}%
\pgfpathlineto{\pgfqpoint{4.073578in}{0.671534in}}%
\pgfpathlineto{\pgfqpoint{4.080418in}{0.763770in}}%
\pgfpathlineto{\pgfqpoint{4.082985in}{0.689708in}}%
\pgfpathlineto{\pgfqpoint{4.086408in}{0.765246in}}%
\pgfpathlineto{\pgfqpoint{4.091542in}{0.666460in}}%
\pgfpathlineto{\pgfqpoint{4.098381in}{0.782375in}}%
\pgfpathlineto{\pgfqpoint{4.100093in}{0.707484in}}%
\pgfpathlineto{\pgfqpoint{4.106080in}{0.769993in}}%
\pgfpathlineto{\pgfqpoint{4.108646in}{0.672467in}}%
\pgfpathlineto{\pgfqpoint{4.114636in}{0.742572in}}%
\pgfpathlineto{\pgfqpoint{4.118056in}{0.673836in}}%
\pgfpathlineto{\pgfqpoint{4.122329in}{0.725012in}}%
\pgfpathlineto{\pgfqpoint{4.127462in}{0.637275in}}%
\pgfpathlineto{\pgfqpoint{4.129172in}{0.697663in}}%
\pgfpathlineto{\pgfqpoint{4.136022in}{0.625795in}}%
\pgfpathlineto{\pgfqpoint{4.140300in}{0.711624in}}%
\pgfpathlineto{\pgfqpoint{4.143724in}{0.667325in}}%
\pgfpathlineto{\pgfqpoint{4.145436in}{0.734115in}}%
\pgfpathlineto{\pgfqpoint{4.151424in}{0.679847in}}%
\pgfpathlineto{\pgfqpoint{4.152280in}{0.723033in}}%
\pgfpathlineto{\pgfqpoint{4.156558in}{0.637132in}}%
\pgfpathlineto{\pgfqpoint{4.159124in}{0.772698in}}%
\pgfpathlineto{\pgfqpoint{4.162548in}{0.660452in}}%
\pgfpathlineto{\pgfqpoint{4.168538in}{0.764488in}}%
\pgfpathlineto{\pgfqpoint{4.171962in}{0.659263in}}%
\pgfpathlineto{\pgfqpoint{4.173670in}{0.784138in}}%
\pgfpathlineto{\pgfqpoint{4.177947in}{0.696905in}}%
\pgfpathlineto{\pgfqpoint{4.181364in}{0.685604in}}%
\pgfpathlineto{\pgfqpoint{4.183932in}{0.772945in}}%
\pgfpathlineto{\pgfqpoint{4.187352in}{0.634754in}}%
\pgfpathlineto{\pgfqpoint{4.189920in}{0.717058in}}%
\pgfpathlineto{\pgfqpoint{4.195906in}{0.657608in}}%
\pgfpathlineto{\pgfqpoint{4.196761in}{0.715223in}}%
\pgfpathlineto{\pgfqpoint{4.200182in}{0.738399in}}%
\pgfpathlineto{\pgfqpoint{4.205311in}{0.737498in}}%
\pgfpathlineto{\pgfqpoint{4.209590in}{0.638101in}}%
\pgfpathlineto{\pgfqpoint{4.210442in}{0.764201in}}%
\pgfpathlineto{\pgfqpoint{4.214719in}{0.649977in}}%
\pgfpathlineto{\pgfqpoint{4.218998in}{0.652099in}}%
\pgfpathlineto{\pgfqpoint{4.221561in}{0.767476in}}%
\pgfpathlineto{\pgfqpoint{4.224983in}{0.786835in}}%
\pgfpathlineto{\pgfqpoint{4.228405in}{0.674629in}}%
\pgfpathlineto{\pgfqpoint{4.231826in}{0.644975in}}%
\pgfpathlineto{\pgfqpoint{4.235248in}{0.755561in}}%
\pgfpathlineto{\pgfqpoint{4.237816in}{0.624351in}}%
\pgfpathlineto{\pgfqpoint{4.241236in}{0.748221in}}%
\pgfpathlineto{\pgfqpoint{4.246370in}{0.614601in}}%
\pgfpathlineto{\pgfqpoint{4.248080in}{0.700646in}}%
\pgfpathlineto{\pgfqpoint{4.252359in}{0.658362in}}%
\pgfpathlineto{\pgfqpoint{4.255778in}{0.705936in}}%
\pgfpathlineto{\pgfqpoint{4.259202in}{0.600316in}}%
\pgfpathlineto{\pgfqpoint{4.265191in}{0.760172in}}%
\pgfpathlineto{\pgfqpoint{4.268614in}{0.657464in}}%
\pgfpathlineto{\pgfqpoint{4.272037in}{0.730014in}}%
\pgfpathlineto{\pgfqpoint{4.275458in}{0.733792in}}%
\pgfpathlineto{\pgfqpoint{4.279734in}{0.695644in}}%
\pgfpathlineto{\pgfqpoint{4.284864in}{0.678048in}}%
\pgfpathlineto{\pgfqpoint{4.285716in}{0.739117in}}%
\pgfpathlineto{\pgfqpoint{4.290850in}{0.759701in}}%
\pgfpathlineto{\pgfqpoint{4.293417in}{0.662176in}}%
\pgfpathlineto{\pgfqpoint{4.297691in}{0.784713in}}%
\pgfpathlineto{\pgfqpoint{4.299404in}{0.672001in}}%
\pgfpathlineto{\pgfqpoint{4.307096in}{0.805839in}}%
\pgfpathlineto{\pgfqpoint{4.311374in}{0.685209in}}%
\pgfpathlineto{\pgfqpoint{4.313941in}{0.768126in}}%
\pgfpathlineto{\pgfqpoint{4.318213in}{0.800191in}}%
\pgfpathlineto{\pgfqpoint{4.319924in}{0.689708in}}%
\pgfpathlineto{\pgfqpoint{4.324195in}{0.655952in}}%
\pgfpathlineto{\pgfqpoint{4.326759in}{0.753622in}}%
\pgfpathlineto{\pgfqpoint{4.331037in}{0.662646in}}%
\pgfpathlineto{\pgfqpoint{4.334460in}{0.736600in}}%
\pgfpathlineto{\pgfqpoint{4.337026in}{0.701009in}}%
\pgfpathlineto{\pgfqpoint{4.342159in}{0.672435in}}%
\pgfpathlineto{\pgfqpoint{4.343870in}{0.741064in}}%
\pgfpathlineto{\pgfqpoint{4.350713in}{0.659730in}}%
\pgfpathlineto{\pgfqpoint{4.355844in}{0.754272in}}%
\pgfpathlineto{\pgfqpoint{4.360978in}{0.684921in}}%
\pgfpathlineto{\pgfqpoint{4.366112in}{0.802456in}}%
\pgfpathlineto{\pgfqpoint{4.368678in}{0.701189in}}%
\pgfpathlineto{\pgfqpoint{4.371246in}{0.768844in}}%
\pgfpathlineto{\pgfqpoint{4.377235in}{0.659766in}}%
\pgfpathlineto{\pgfqpoint{4.378945in}{0.731562in}}%
\pgfpathlineto{\pgfqpoint{4.382361in}{0.679089in}}%
\pgfpathlineto{\pgfqpoint{4.385782in}{0.768844in}}%
\pgfpathlineto{\pgfqpoint{4.390915in}{0.774640in}}%
\pgfpathlineto{\pgfqpoint{4.393482in}{0.662431in}}%
\pgfpathlineto{\pgfqpoint{4.395194in}{0.748260in}}%
\pgfpathlineto{\pgfqpoint{4.398612in}{0.696258in}}%
\pgfpathlineto{\pgfqpoint{4.402032in}{0.721916in}}%
\pgfpathlineto{\pgfqpoint{4.406311in}{0.628962in}}%
\pgfpathlineto{\pgfqpoint{4.408878in}{0.713962in}}%
\pgfpathlineto{\pgfqpoint{4.414012in}{0.753837in}}%
\pgfpathlineto{\pgfqpoint{4.416578in}{0.671929in}}%
\pgfpathlineto{\pgfqpoint{4.419998in}{0.735376in}}%
\pgfpathlineto{\pgfqpoint{4.422562in}{0.690211in}}%
\pgfpathlineto{\pgfqpoint{4.427694in}{0.740019in}}%
\pgfpathlineto{\pgfqpoint{4.430262in}{0.651237in}}%
\pgfpathlineto{\pgfqpoint{4.432826in}{0.784713in}}%
\pgfpathlineto{\pgfqpoint{4.437103in}{0.653288in}}%
\pgfpathlineto{\pgfqpoint{4.442241in}{0.767512in}}%
\pgfpathlineto{\pgfqpoint{4.444808in}{0.679161in}}%
\pgfpathlineto{\pgfqpoint{4.449081in}{0.735878in}}%
\pgfpathlineto{\pgfqpoint{4.449936in}{0.654369in}}%
\pgfpathlineto{\pgfqpoint{4.455924in}{0.773847in}}%
\pgfpathlineto{\pgfqpoint{4.458489in}{0.669412in}}%
\pgfpathlineto{\pgfqpoint{4.461057in}{0.756861in}}%
\pgfpathlineto{\pgfqpoint{4.463622in}{0.688667in}}%
\pgfpathlineto{\pgfqpoint{4.468754in}{0.726380in}}%
\pgfpathlineto{\pgfqpoint{4.473032in}{0.612264in}}%
\pgfpathlineto{\pgfqpoint{4.476450in}{0.748619in}}%
\pgfpathlineto{\pgfqpoint{4.478161in}{0.681431in}}%
\pgfpathlineto{\pgfqpoint{4.480727in}{0.726560in}}%
\pgfpathlineto{\pgfqpoint{4.485001in}{0.646343in}}%
\pgfpathlineto{\pgfqpoint{4.488421in}{0.751643in}}%
\pgfpathlineto{\pgfqpoint{4.491842in}{0.664266in}}%
\pgfpathlineto{\pgfqpoint{4.496121in}{0.733038in}}%
\pgfpathlineto{\pgfqpoint{4.497827in}{0.687730in}}%
\pgfpathlineto{\pgfqpoint{4.502953in}{0.766148in}}%
\pgfpathlineto{\pgfqpoint{4.505519in}{0.681323in}}%
\pgfpathlineto{\pgfqpoint{4.509790in}{0.742213in}}%
\pgfpathlineto{\pgfqpoint{4.512354in}{0.745452in}}%
\pgfpathlineto{\pgfqpoint{4.514921in}{0.664481in}}%
\pgfpathlineto{\pgfqpoint{4.518344in}{0.720225in}}%
\pgfpathlineto{\pgfqpoint{4.523473in}{0.673441in}}%
\pgfpathlineto{\pgfqpoint{4.526897in}{0.747646in}}%
\pgfpathlineto{\pgfqpoint{4.534596in}{0.583798in}}%
\pgfpathlineto{\pgfqpoint{4.535450in}{0.762837in}}%
\pgfpathlineto{\pgfqpoint{4.538868in}{0.585561in}}%
\pgfpathlineto{\pgfqpoint{4.543143in}{0.764775in}}%
\pgfpathlineto{\pgfqpoint{4.548281in}{0.748835in}}%
\pgfpathlineto{\pgfqpoint{4.549136in}{0.669340in}}%
\pgfpathlineto{\pgfqpoint{4.554268in}{0.634108in}}%
\pgfpathlineto{\pgfqpoint{4.558543in}{0.730804in}}%
\pgfpathlineto{\pgfqpoint{4.559399in}{0.666564in}}%
\pgfpathlineto{\pgfqpoint{4.562820in}{0.642777in}}%
\pgfpathlineto{\pgfqpoint{4.568805in}{0.737426in}}%
\pgfpathlineto{\pgfqpoint{4.570513in}{0.612479in}}%
\pgfpathlineto{\pgfqpoint{4.574795in}{0.714537in}}%
\pgfpathlineto{\pgfqpoint{4.579072in}{0.597185in}}%
\pgfpathlineto{\pgfqpoint{4.580783in}{0.682185in}}%
\pgfpathlineto{\pgfqpoint{4.584204in}{0.754268in}}%
\pgfpathlineto{\pgfqpoint{4.588482in}{0.633171in}}%
\pgfpathlineto{\pgfqpoint{4.592755in}{0.801986in}}%
\pgfpathlineto{\pgfqpoint{4.595324in}{0.662140in}}%
\pgfpathlineto{\pgfqpoint{4.597035in}{0.768445in}}%
\pgfpathlineto{\pgfqpoint{4.600455in}{0.691180in}}%
\pgfpathlineto{\pgfqpoint{4.605589in}{0.654293in}}%
\pgfpathlineto{\pgfqpoint{4.609011in}{0.684379in}}%
\pgfpathlineto{\pgfqpoint{4.610723in}{0.791335in}}%
\pgfpathlineto{\pgfqpoint{4.615000in}{0.660412in}}%
\pgfpathlineto{\pgfqpoint{4.617565in}{0.790002in}}%
\pgfpathlineto{\pgfqpoint{4.623549in}{0.757328in}}%
\pgfpathlineto{\pgfqpoint{4.624405in}{0.605786in}}%
\pgfpathlineto{\pgfqpoint{4.627827in}{0.726376in}}%
\pgfpathlineto{\pgfqpoint{4.631246in}{0.763299in}}%
\pgfpathlineto{\pgfqpoint{4.637235in}{0.788455in}}%
\pgfpathlineto{\pgfqpoint{4.638091in}{0.662642in}}%
\pgfpathlineto{\pgfqpoint{4.641512in}{0.745739in}}%
\pgfpathlineto{\pgfqpoint{4.644931in}{0.769562in}}%
\pgfpathlineto{\pgfqpoint{4.648350in}{0.704859in}}%
\pgfpathlineto{\pgfqpoint{4.654343in}{0.737354in}}%
\pgfpathlineto{\pgfqpoint{4.657762in}{0.618958in}}%
\pgfpathlineto{\pgfqpoint{4.660326in}{0.717600in}}%
\pgfpathlineto{\pgfqpoint{4.664605in}{0.731203in}}%
\pgfpathlineto{\pgfqpoint{4.666317in}{0.652354in}}%
\pgfpathlineto{\pgfqpoint{4.671449in}{0.613309in}}%
\pgfpathlineto{\pgfqpoint{4.672305in}{0.690394in}}%
\pgfpathlineto{\pgfqpoint{4.677437in}{0.749808in}}%
\pgfpathlineto{\pgfqpoint{4.681713in}{0.671390in}}%
\pgfpathlineto{\pgfqpoint{4.683423in}{0.607190in}}%
\pgfpathlineto{\pgfqpoint{4.686845in}{0.753837in}}%
\pgfpathlineto{\pgfqpoint{4.689413in}{0.702266in}}%
\pgfpathlineto{\pgfqpoint{4.695398in}{0.696833in}}%
\pgfpathlineto{\pgfqpoint{4.696255in}{0.777157in}}%
\pgfpathlineto{\pgfqpoint{4.700535in}{0.659439in}}%
\pgfpathlineto{\pgfqpoint{4.704809in}{0.740019in}}%
\pgfpathlineto{\pgfqpoint{4.710801in}{0.658904in}}%
\pgfpathlineto{\pgfqpoint{4.713368in}{0.768054in}}%
\pgfpathlineto{\pgfqpoint{4.717644in}{0.826387in}}%
\pgfpathlineto{\pgfqpoint{4.720210in}{0.686936in}}%
\pgfpathlineto{\pgfqpoint{4.724492in}{0.766610in}}%
\pgfpathlineto{\pgfqpoint{4.727918in}{0.689493in}}%
\pgfpathlineto{\pgfqpoint{4.731339in}{0.762653in}}%
\pgfpathlineto{\pgfqpoint{4.734766in}{0.637706in}}%
\pgfpathlineto{\pgfqpoint{4.739903in}{0.792883in}}%
\pgfpathlineto{\pgfqpoint{4.742472in}{0.703383in}}%
\pgfpathlineto{\pgfqpoint{4.746750in}{0.774062in}}%
\pgfpathlineto{\pgfqpoint{4.748462in}{0.688376in}}%
\pgfpathlineto{\pgfqpoint{4.751882in}{0.785973in}}%
\pgfpathlineto{\pgfqpoint{4.756154in}{0.680565in}}%
\pgfpathlineto{\pgfqpoint{4.760428in}{0.784857in}}%
\pgfpathlineto{\pgfqpoint{4.762995in}{0.667609in}}%
\pgfpathlineto{\pgfqpoint{4.765556in}{0.769490in}}%
\pgfpathlineto{\pgfqpoint{4.768121in}{0.710364in}}%
\pgfpathlineto{\pgfqpoint{4.774110in}{0.730625in}}%
\pgfpathlineto{\pgfqpoint{4.774966in}{0.671749in}}%
\pgfpathlineto{\pgfqpoint{4.779249in}{0.748472in}}%
\pgfpathlineto{\pgfqpoint{4.782675in}{0.649331in}}%
\pgfpathlineto{\pgfqpoint{4.787806in}{0.650049in}}%
\pgfpathlineto{\pgfqpoint{4.790374in}{0.743003in}}%
\pgfpathlineto{\pgfqpoint{4.792084in}{0.638496in}}%
\pgfpathlineto{\pgfqpoint{4.795506in}{0.697802in}}%
\pgfpathlineto{\pgfqpoint{4.795506in}{0.697802in}}%
\pgfusepath{stroke}%
\end{pgfscope}%
\begin{pgfscope}%
\pgfsetrectcap%
\pgfsetmiterjoin%
\pgfsetlinewidth{0.803000pt}%
\definecolor{currentstroke}{rgb}{0.000000,0.000000,0.000000}%
\pgfsetstrokecolor{currentstroke}%
\pgfsetdash{}{0pt}%
\pgfpathmoveto{\pgfqpoint{0.484581in}{0.539544in}}%
\pgfpathlineto{\pgfqpoint{0.484581in}{1.114166in}}%
\pgfusepath{stroke}%
\end{pgfscope}%
\begin{pgfscope}%
\pgfsetrectcap%
\pgfsetmiterjoin%
\pgfsetlinewidth{0.803000pt}%
\definecolor{currentstroke}{rgb}{0.000000,0.000000,0.000000}%
\pgfsetstrokecolor{currentstroke}%
\pgfsetdash{}{0pt}%
\pgfpathmoveto{\pgfqpoint{5.000788in}{0.539544in}}%
\pgfpathlineto{\pgfqpoint{5.000788in}{1.114166in}}%
\pgfusepath{stroke}%
\end{pgfscope}%
\begin{pgfscope}%
\pgfsetrectcap%
\pgfsetmiterjoin%
\pgfsetlinewidth{0.803000pt}%
\definecolor{currentstroke}{rgb}{0.000000,0.000000,0.000000}%
\pgfsetstrokecolor{currentstroke}%
\pgfsetdash{}{0pt}%
\pgfpathmoveto{\pgfqpoint{0.484581in}{0.539544in}}%
\pgfpathlineto{\pgfqpoint{5.000788in}{0.539544in}}%
\pgfusepath{stroke}%
\end{pgfscope}%
\begin{pgfscope}%
\pgfsetrectcap%
\pgfsetmiterjoin%
\pgfsetlinewidth{0.803000pt}%
\definecolor{currentstroke}{rgb}{0.000000,0.000000,0.000000}%
\pgfsetstrokecolor{currentstroke}%
\pgfsetdash{}{0pt}%
\pgfpathmoveto{\pgfqpoint{0.484581in}{1.114166in}}%
\pgfpathlineto{\pgfqpoint{5.000788in}{1.114166in}}%
\pgfusepath{stroke}%
\end{pgfscope}%
\end{pgfpicture}%
\makeatother%
\endgroup%

    \caption{Popcorn noise in different samples of the LM399 over a \qty{24}{\hour} period.}
    \label{fig:popcorn_noise_lm399}
\end{figure}

Figure \ref{fig:popcorn_noise_lm399} shows two samples of the LM399, that exhibit popcorn noise, while the last one does not.

The sources of popcorn noise in semiconductor devices are not yet fully understood, but some sources have been identified. Defects in the semiconductor crystal lattice and contamination of the semiconductor material have been linked to popcorn noise \cite{technote_ti_popcorn_noise}. This problem has improved over the years as manufacturing processes and wafer quality has evolved. Unfortunately the LM399 is built around a process from 1991, as can be seen etched into the die \cite{lm399_richi}.

The popcorn noise caused by defects and contamination can be reduced by lowering the strain on the lattice and removing surface contaminants on the die. This can be can be achieved using a high-temperature burn-in process. Manufacturers like Fluke and Keysight use similar techniques in their products. Fluke, for example, uses a period of \qty{60}{\day} burn-in for their references \cite{zener_popcorn_noise}.

Fortunately, the LM399 is a heated reference, which regulates its die to \qty{90}{\celsius} when turned on, so it is only required put the diodes in a simple test circuit and wait. The use of a separate test setup instead of the final circuit has both advantages and disadvantages. The disadvantage is, that the Zener diode will subjected to mechanical stress when soldered, this stress will not be removed by the burn-in process as it happens after the testing, when diode is soldered into the final circuit, but this mainly affects the voltage drift properties of the Zener diode and not the popcorn noise. The drift of the diode is also only of secondary concern in our setup, as the drift is mainly caused by the reference resistors used and are typically at least an order of magnitude worse than the the drift of the diode judging by the data sheet \cite{datasheet_LM399,datasheet_VPR}.

The advantages of testing the Zener diodes separately, on the other hand, are, that more diodes can be tested at the same time, as a special compact test fixture can be used. It is also simpler to remove the diodes from the test fixture, because they be socketed. Therefore for our application a separate test board was used. Building this test setup is detailed in the next sections.

\subsection{Building a Test Setup for Zener Diodes}
There are several ways to measure the popcorn noise of semiconductor devices. The most trivial one is to directly monitor the device in the time-domain. In this case, the Zener voltage can be monitored with a long-scale multimeter. It requires a low noise DMM, that can reliably distinguish between both voltage levels, which are about \qty{4}{\micro \volt} apart.
A related option is to use a second reference, whose voltage is similar to the device under test (DUT). Measuring the the voltage difference between the two references, less resolution is required. Directly comparing the difference of two references using a millivolt meter is commonly done when intercomparing primary voltage references. This method, however, increases the measurement noise by a factor of $\sqrt{2}$, if both references produce the same level of uncorrelated noise. The noise of the LM399 with a \qty{100}{\plc} integration time (\qty{2}{\second}) is about \qty{1.5}{\micro \volt_{pp}} as can be determined from the data in figure \ref{fig:popcorn_noise_lm399}.

\begin{figure}[ht]
    \centering
    %%% Creator: Matplotlib, PGF backend
%%
%% To include the figure in your LaTeX document, write
%%   \input{<filename>.pgf}
%%
%% Make sure the required packages are loaded in your preamble
%%   \usepackage{pgf}
%%
%% Also ensure that all the required font packages are loaded; for instance,
%% the lmodern package is sometimes necessary when using math font.
%%   \usepackage{lmodern}
%%
%% Figures using additional raster images can only be included by \input if
%% they are in the same directory as the main LaTeX file. For loading figures
%% from other directories you can use the `import` package
%%   \usepackage{import}
%%
%% and then include the figures with
%%   \import{<path to file>}{<filename>.pgf}
%%
%% Matplotlib used the following preamble
%%   \usepackage{siunitx}
%%   \sisetup{per-mode = symbol}%
%%   \usepackage{fontspec}
%%   \makeatletter\@ifpackageloaded{underscore}{}{\usepackage[strings]{underscore}}\makeatother
%%
\begingroup%
\makeatletter%
\begin{pgfpicture}%
\pgfpathrectangle{\pgfpointorigin}{\pgfqpoint{5.150788in}{3.183362in}}%
\pgfusepath{use as bounding box, clip}%
\begin{pgfscope}%
\pgfsetbuttcap%
\pgfsetmiterjoin%
\definecolor{currentfill}{rgb}{1.000000,1.000000,1.000000}%
\pgfsetfillcolor{currentfill}%
\pgfsetlinewidth{0.000000pt}%
\definecolor{currentstroke}{rgb}{1.000000,1.000000,1.000000}%
\pgfsetstrokecolor{currentstroke}%
\pgfsetdash{}{0pt}%
\pgfpathmoveto{\pgfqpoint{0.000000in}{0.000000in}}%
\pgfpathlineto{\pgfqpoint{5.150788in}{0.000000in}}%
\pgfpathlineto{\pgfqpoint{5.150788in}{3.183362in}}%
\pgfpathlineto{\pgfqpoint{0.000000in}{3.183362in}}%
\pgfpathlineto{\pgfqpoint{0.000000in}{0.000000in}}%
\pgfpathclose%
\pgfusepath{fill}%
\end{pgfscope}%
\begin{pgfscope}%
\pgfsetbuttcap%
\pgfsetmiterjoin%
\definecolor{currentfill}{rgb}{1.000000,1.000000,1.000000}%
\pgfsetfillcolor{currentfill}%
\pgfsetlinewidth{0.000000pt}%
\definecolor{currentstroke}{rgb}{0.000000,0.000000,0.000000}%
\pgfsetstrokecolor{currentstroke}%
\pgfsetstrokeopacity{0.000000}%
\pgfsetdash{}{0pt}%
\pgfpathmoveto{\pgfqpoint{0.484581in}{2.334497in}}%
\pgfpathlineto{\pgfqpoint{5.000788in}{2.334497in}}%
\pgfpathlineto{\pgfqpoint{5.000788in}{2.909119in}}%
\pgfpathlineto{\pgfqpoint{0.484581in}{2.909119in}}%
\pgfpathlineto{\pgfqpoint{0.484581in}{2.334497in}}%
\pgfpathclose%
\pgfusepath{fill}%
\end{pgfscope}%
\begin{pgfscope}%
\pgfsetbuttcap%
\pgfsetroundjoin%
\definecolor{currentfill}{rgb}{0.000000,0.000000,0.000000}%
\pgfsetfillcolor{currentfill}%
\pgfsetlinewidth{0.803000pt}%
\definecolor{currentstroke}{rgb}{0.000000,0.000000,0.000000}%
\pgfsetstrokecolor{currentstroke}%
\pgfsetdash{}{0pt}%
\pgfsys@defobject{currentmarker}{\pgfqpoint{0.000000in}{-0.048611in}}{\pgfqpoint{0.000000in}{0.000000in}}{%
\pgfpathmoveto{\pgfqpoint{0.000000in}{0.000000in}}%
\pgfpathlineto{\pgfqpoint{0.000000in}{-0.048611in}}%
\pgfusepath{stroke,fill}%
}%
\begin{pgfscope}%
\pgfsys@transformshift{0.689546in}{2.334497in}%
\pgfsys@useobject{currentmarker}{}%
\end{pgfscope}%
\end{pgfscope}%
\begin{pgfscope}%
\pgfsetbuttcap%
\pgfsetroundjoin%
\definecolor{currentfill}{rgb}{0.000000,0.000000,0.000000}%
\pgfsetfillcolor{currentfill}%
\pgfsetlinewidth{0.803000pt}%
\definecolor{currentstroke}{rgb}{0.000000,0.000000,0.000000}%
\pgfsetstrokecolor{currentstroke}%
\pgfsetdash{}{0pt}%
\pgfsys@defobject{currentmarker}{\pgfqpoint{0.000000in}{-0.048611in}}{\pgfqpoint{0.000000in}{0.000000in}}{%
\pgfpathmoveto{\pgfqpoint{0.000000in}{0.000000in}}%
\pgfpathlineto{\pgfqpoint{0.000000in}{-0.048611in}}%
\pgfusepath{stroke,fill}%
}%
\begin{pgfscope}%
\pgfsys@transformshift{1.202878in}{2.334497in}%
\pgfsys@useobject{currentmarker}{}%
\end{pgfscope}%
\end{pgfscope}%
\begin{pgfscope}%
\pgfsetbuttcap%
\pgfsetroundjoin%
\definecolor{currentfill}{rgb}{0.000000,0.000000,0.000000}%
\pgfsetfillcolor{currentfill}%
\pgfsetlinewidth{0.803000pt}%
\definecolor{currentstroke}{rgb}{0.000000,0.000000,0.000000}%
\pgfsetstrokecolor{currentstroke}%
\pgfsetdash{}{0pt}%
\pgfsys@defobject{currentmarker}{\pgfqpoint{0.000000in}{-0.048611in}}{\pgfqpoint{0.000000in}{0.000000in}}{%
\pgfpathmoveto{\pgfqpoint{0.000000in}{0.000000in}}%
\pgfpathlineto{\pgfqpoint{0.000000in}{-0.048611in}}%
\pgfusepath{stroke,fill}%
}%
\begin{pgfscope}%
\pgfsys@transformshift{1.716211in}{2.334497in}%
\pgfsys@useobject{currentmarker}{}%
\end{pgfscope}%
\end{pgfscope}%
\begin{pgfscope}%
\pgfsetbuttcap%
\pgfsetroundjoin%
\definecolor{currentfill}{rgb}{0.000000,0.000000,0.000000}%
\pgfsetfillcolor{currentfill}%
\pgfsetlinewidth{0.803000pt}%
\definecolor{currentstroke}{rgb}{0.000000,0.000000,0.000000}%
\pgfsetstrokecolor{currentstroke}%
\pgfsetdash{}{0pt}%
\pgfsys@defobject{currentmarker}{\pgfqpoint{0.000000in}{-0.048611in}}{\pgfqpoint{0.000000in}{0.000000in}}{%
\pgfpathmoveto{\pgfqpoint{0.000000in}{0.000000in}}%
\pgfpathlineto{\pgfqpoint{0.000000in}{-0.048611in}}%
\pgfusepath{stroke,fill}%
}%
\begin{pgfscope}%
\pgfsys@transformshift{2.229543in}{2.334497in}%
\pgfsys@useobject{currentmarker}{}%
\end{pgfscope}%
\end{pgfscope}%
\begin{pgfscope}%
\pgfsetbuttcap%
\pgfsetroundjoin%
\definecolor{currentfill}{rgb}{0.000000,0.000000,0.000000}%
\pgfsetfillcolor{currentfill}%
\pgfsetlinewidth{0.803000pt}%
\definecolor{currentstroke}{rgb}{0.000000,0.000000,0.000000}%
\pgfsetstrokecolor{currentstroke}%
\pgfsetdash{}{0pt}%
\pgfsys@defobject{currentmarker}{\pgfqpoint{0.000000in}{-0.048611in}}{\pgfqpoint{0.000000in}{0.000000in}}{%
\pgfpathmoveto{\pgfqpoint{0.000000in}{0.000000in}}%
\pgfpathlineto{\pgfqpoint{0.000000in}{-0.048611in}}%
\pgfusepath{stroke,fill}%
}%
\begin{pgfscope}%
\pgfsys@transformshift{2.742876in}{2.334497in}%
\pgfsys@useobject{currentmarker}{}%
\end{pgfscope}%
\end{pgfscope}%
\begin{pgfscope}%
\pgfsetbuttcap%
\pgfsetroundjoin%
\definecolor{currentfill}{rgb}{0.000000,0.000000,0.000000}%
\pgfsetfillcolor{currentfill}%
\pgfsetlinewidth{0.803000pt}%
\definecolor{currentstroke}{rgb}{0.000000,0.000000,0.000000}%
\pgfsetstrokecolor{currentstroke}%
\pgfsetdash{}{0pt}%
\pgfsys@defobject{currentmarker}{\pgfqpoint{0.000000in}{-0.048611in}}{\pgfqpoint{0.000000in}{0.000000in}}{%
\pgfpathmoveto{\pgfqpoint{0.000000in}{0.000000in}}%
\pgfpathlineto{\pgfqpoint{0.000000in}{-0.048611in}}%
\pgfusepath{stroke,fill}%
}%
\begin{pgfscope}%
\pgfsys@transformshift{3.256208in}{2.334497in}%
\pgfsys@useobject{currentmarker}{}%
\end{pgfscope}%
\end{pgfscope}%
\begin{pgfscope}%
\pgfsetbuttcap%
\pgfsetroundjoin%
\definecolor{currentfill}{rgb}{0.000000,0.000000,0.000000}%
\pgfsetfillcolor{currentfill}%
\pgfsetlinewidth{0.803000pt}%
\definecolor{currentstroke}{rgb}{0.000000,0.000000,0.000000}%
\pgfsetstrokecolor{currentstroke}%
\pgfsetdash{}{0pt}%
\pgfsys@defobject{currentmarker}{\pgfqpoint{0.000000in}{-0.048611in}}{\pgfqpoint{0.000000in}{0.000000in}}{%
\pgfpathmoveto{\pgfqpoint{0.000000in}{0.000000in}}%
\pgfpathlineto{\pgfqpoint{0.000000in}{-0.048611in}}%
\pgfusepath{stroke,fill}%
}%
\begin{pgfscope}%
\pgfsys@transformshift{3.769541in}{2.334497in}%
\pgfsys@useobject{currentmarker}{}%
\end{pgfscope}%
\end{pgfscope}%
\begin{pgfscope}%
\pgfsetbuttcap%
\pgfsetroundjoin%
\definecolor{currentfill}{rgb}{0.000000,0.000000,0.000000}%
\pgfsetfillcolor{currentfill}%
\pgfsetlinewidth{0.803000pt}%
\definecolor{currentstroke}{rgb}{0.000000,0.000000,0.000000}%
\pgfsetstrokecolor{currentstroke}%
\pgfsetdash{}{0pt}%
\pgfsys@defobject{currentmarker}{\pgfqpoint{0.000000in}{-0.048611in}}{\pgfqpoint{0.000000in}{0.000000in}}{%
\pgfpathmoveto{\pgfqpoint{0.000000in}{0.000000in}}%
\pgfpathlineto{\pgfqpoint{0.000000in}{-0.048611in}}%
\pgfusepath{stroke,fill}%
}%
\begin{pgfscope}%
\pgfsys@transformshift{4.282873in}{2.334497in}%
\pgfsys@useobject{currentmarker}{}%
\end{pgfscope}%
\end{pgfscope}%
\begin{pgfscope}%
\pgfsetbuttcap%
\pgfsetroundjoin%
\definecolor{currentfill}{rgb}{0.000000,0.000000,0.000000}%
\pgfsetfillcolor{currentfill}%
\pgfsetlinewidth{0.803000pt}%
\definecolor{currentstroke}{rgb}{0.000000,0.000000,0.000000}%
\pgfsetstrokecolor{currentstroke}%
\pgfsetdash{}{0pt}%
\pgfsys@defobject{currentmarker}{\pgfqpoint{0.000000in}{-0.048611in}}{\pgfqpoint{0.000000in}{0.000000in}}{%
\pgfpathmoveto{\pgfqpoint{0.000000in}{0.000000in}}%
\pgfpathlineto{\pgfqpoint{0.000000in}{-0.048611in}}%
\pgfusepath{stroke,fill}%
}%
\begin{pgfscope}%
\pgfsys@transformshift{4.796206in}{2.334497in}%
\pgfsys@useobject{currentmarker}{}%
\end{pgfscope}%
\end{pgfscope}%
\begin{pgfscope}%
\pgfsetbuttcap%
\pgfsetroundjoin%
\definecolor{currentfill}{rgb}{0.000000,0.000000,0.000000}%
\pgfsetfillcolor{currentfill}%
\pgfsetlinewidth{0.803000pt}%
\definecolor{currentstroke}{rgb}{0.000000,0.000000,0.000000}%
\pgfsetstrokecolor{currentstroke}%
\pgfsetdash{}{0pt}%
\pgfsys@defobject{currentmarker}{\pgfqpoint{-0.048611in}{0.000000in}}{\pgfqpoint{-0.000000in}{0.000000in}}{%
\pgfpathmoveto{\pgfqpoint{-0.000000in}{0.000000in}}%
\pgfpathlineto{\pgfqpoint{-0.048611in}{0.000000in}}%
\pgfusepath{stroke,fill}%
}%
\begin{pgfscope}%
\pgfsys@transformshift{0.484581in}{2.512496in}%
\pgfsys@useobject{currentmarker}{}%
\end{pgfscope}%
\end{pgfscope}%
\begin{pgfscope}%
\definecolor{textcolor}{rgb}{0.000000,0.000000,0.000000}%
\pgfsetstrokecolor{textcolor}%
\pgfsetfillcolor{textcolor}%
\pgftext[x=0.328331in, y=2.473941in, left, base]{\color{textcolor}\rmfamily\fontsize{8.000000}{9.600000}\selectfont \(\displaystyle {0}\)}%
\end{pgfscope}%
\begin{pgfscope}%
\pgfsetbuttcap%
\pgfsetroundjoin%
\definecolor{currentfill}{rgb}{0.000000,0.000000,0.000000}%
\pgfsetfillcolor{currentfill}%
\pgfsetlinewidth{0.803000pt}%
\definecolor{currentstroke}{rgb}{0.000000,0.000000,0.000000}%
\pgfsetstrokecolor{currentstroke}%
\pgfsetdash{}{0pt}%
\pgfsys@defobject{currentmarker}{\pgfqpoint{-0.048611in}{0.000000in}}{\pgfqpoint{-0.000000in}{0.000000in}}{%
\pgfpathmoveto{\pgfqpoint{-0.000000in}{0.000000in}}%
\pgfpathlineto{\pgfqpoint{-0.048611in}{0.000000in}}%
\pgfusepath{stroke,fill}%
}%
\begin{pgfscope}%
\pgfsys@transformshift{0.484581in}{2.716037in}%
\pgfsys@useobject{currentmarker}{}%
\end{pgfscope}%
\end{pgfscope}%
\begin{pgfscope}%
\definecolor{textcolor}{rgb}{0.000000,0.000000,0.000000}%
\pgfsetstrokecolor{textcolor}%
\pgfsetfillcolor{textcolor}%
\pgftext[x=0.328331in, y=2.677482in, left, base]{\color{textcolor}\rmfamily\fontsize{8.000000}{9.600000}\selectfont \(\displaystyle {5}\)}%
\end{pgfscope}%
\begin{pgfscope}%
\definecolor{textcolor}{rgb}{0.000000,0.000000,0.000000}%
\pgfsetstrokecolor{textcolor}%
\pgfsetfillcolor{textcolor}%
\pgftext[x=0.484581in,y=2.950785in,left,base]{\color{textcolor}\rmfamily\fontsize{8.000000}{9.600000}\selectfont \(\displaystyle \times{10^{\ensuremath{-}6}}{}\)}%
\end{pgfscope}%
\begin{pgfscope}%
\pgfpathrectangle{\pgfqpoint{0.484581in}{2.334497in}}{\pgfqpoint{4.516206in}{0.574622in}}%
\pgfusepath{clip}%
\pgfsetrectcap%
\pgfsetroundjoin%
\pgfsetlinewidth{0.501875pt}%
\definecolor{currentstroke}{rgb}{0.003922,0.450980,0.698039}%
\pgfsetstrokecolor{currentstroke}%
\pgfsetstrokeopacity{0.700000}%
\pgfsetdash{}{0pt}%
\pgfpathmoveto{\pgfqpoint{0.689863in}{2.509930in}}%
\pgfpathlineto{\pgfqpoint{0.691573in}{2.569128in}}%
\pgfpathlineto{\pgfqpoint{0.694995in}{2.478459in}}%
\pgfpathlineto{\pgfqpoint{0.699275in}{2.580305in}}%
\pgfpathlineto{\pgfqpoint{0.703553in}{2.518208in}}%
\pgfpathlineto{\pgfqpoint{0.710398in}{2.564054in}}%
\pgfpathlineto{\pgfqpoint{0.712966in}{2.460398in}}%
\pgfpathlineto{\pgfqpoint{0.718959in}{2.382413in}}%
\pgfpathlineto{\pgfqpoint{0.722378in}{2.526844in}}%
\pgfpathlineto{\pgfqpoint{0.728363in}{2.563512in}}%
\pgfpathlineto{\pgfqpoint{0.730929in}{2.446928in}}%
\pgfpathlineto{\pgfqpoint{0.734351in}{2.508420in}}%
\pgfpathlineto{\pgfqpoint{0.740340in}{2.462634in}}%
\pgfpathlineto{\pgfqpoint{0.742908in}{2.509569in}}%
\pgfpathlineto{\pgfqpoint{0.748037in}{2.531618in}}%
\pgfpathlineto{\pgfqpoint{0.752314in}{2.422041in}}%
\pgfpathlineto{\pgfqpoint{0.755738in}{2.514704in}}%
\pgfpathlineto{\pgfqpoint{0.760009in}{2.489757in}}%
\pgfpathlineto{\pgfqpoint{0.764288in}{2.546054in}}%
\pgfpathlineto{\pgfqpoint{0.769423in}{2.581573in}}%
\pgfpathlineto{\pgfqpoint{0.771990in}{2.483110in}}%
\pgfpathlineto{\pgfqpoint{0.777124in}{2.481781in}}%
\pgfpathlineto{\pgfqpoint{0.780547in}{2.572209in}}%
\pgfpathlineto{\pgfqpoint{0.786534in}{2.392199in}}%
\pgfpathlineto{\pgfqpoint{0.789953in}{2.555597in}}%
\pgfpathlineto{\pgfqpoint{0.795939in}{2.458526in}}%
\pgfpathlineto{\pgfqpoint{0.799360in}{2.547927in}}%
\pgfpathlineto{\pgfqpoint{0.803633in}{2.421859in}}%
\pgfpathlineto{\pgfqpoint{0.806199in}{2.515549in}}%
\pgfpathlineto{\pgfqpoint{0.810475in}{2.476526in}}%
\pgfpathlineto{\pgfqpoint{0.814751in}{2.453994in}}%
\pgfpathlineto{\pgfqpoint{0.820735in}{2.649770in}}%
\pgfpathlineto{\pgfqpoint{0.823303in}{2.515246in}}%
\pgfpathlineto{\pgfqpoint{0.828440in}{2.617574in}}%
\pgfpathlineto{\pgfqpoint{0.832717in}{2.490117in}}%
\pgfpathlineto{\pgfqpoint{0.839565in}{2.452968in}}%
\pgfpathlineto{\pgfqpoint{0.842131in}{2.529864in}}%
\pgfpathlineto{\pgfqpoint{0.848114in}{2.449857in}}%
\pgfpathlineto{\pgfqpoint{0.852384in}{2.551822in}}%
\pgfpathlineto{\pgfqpoint{0.856662in}{2.459794in}}%
\pgfpathlineto{\pgfqpoint{0.859233in}{2.558859in}}%
\pgfpathlineto{\pgfqpoint{0.862659in}{2.464323in}}%
\pgfpathlineto{\pgfqpoint{0.868650in}{2.535180in}}%
\pgfpathlineto{\pgfqpoint{0.872926in}{2.477250in}}%
\pgfpathlineto{\pgfqpoint{0.876348in}{2.467797in}}%
\pgfpathlineto{\pgfqpoint{0.880627in}{2.539679in}}%
\pgfpathlineto{\pgfqpoint{0.884050in}{2.404099in}}%
\pgfpathlineto{\pgfqpoint{0.887469in}{2.536629in}}%
\pgfpathlineto{\pgfqpoint{0.895167in}{2.454266in}}%
\pgfpathlineto{\pgfqpoint{0.896877in}{2.573658in}}%
\pgfpathlineto{\pgfqpoint{0.902009in}{2.524850in}}%
\pgfpathlineto{\pgfqpoint{0.906287in}{2.809305in}}%
\pgfpathlineto{\pgfqpoint{0.908853in}{2.673270in}}%
\pgfpathlineto{\pgfqpoint{0.915695in}{2.794808in}}%
\pgfpathlineto{\pgfqpoint{0.919115in}{2.504372in}}%
\pgfpathlineto{\pgfqpoint{0.921681in}{2.466317in}}%
\pgfpathlineto{\pgfqpoint{0.925951in}{2.535422in}}%
\pgfpathlineto{\pgfqpoint{0.935359in}{2.435328in}}%
\pgfpathlineto{\pgfqpoint{0.939638in}{2.544754in}}%
\pgfpathlineto{\pgfqpoint{0.943058in}{2.464444in}}%
\pgfpathlineto{\pgfqpoint{0.947330in}{2.450187in}}%
\pgfpathlineto{\pgfqpoint{0.955029in}{2.439316in}}%
\pgfpathlineto{\pgfqpoint{0.957596in}{2.572632in}}%
\pgfpathlineto{\pgfqpoint{0.961015in}{2.449373in}}%
\pgfpathlineto{\pgfqpoint{0.965292in}{2.526783in}}%
\pgfpathlineto{\pgfqpoint{0.971277in}{2.574021in}}%
\pgfpathlineto{\pgfqpoint{0.972986in}{2.461303in}}%
\pgfpathlineto{\pgfqpoint{0.978119in}{2.553906in}}%
\pgfpathlineto{\pgfqpoint{0.983246in}{2.411046in}}%
\pgfpathlineto{\pgfqpoint{0.985812in}{2.492956in}}%
\pgfpathlineto{\pgfqpoint{0.990941in}{2.449706in}}%
\pgfpathlineto{\pgfqpoint{0.996074in}{2.539710in}}%
\pgfpathlineto{\pgfqpoint{1.001204in}{2.486674in}}%
\pgfpathlineto{\pgfqpoint{1.004625in}{2.536085in}}%
\pgfpathlineto{\pgfqpoint{1.007191in}{2.418051in}}%
\pgfpathlineto{\pgfqpoint{1.013177in}{2.564296in}}%
\pgfpathlineto{\pgfqpoint{1.016601in}{2.472720in}}%
\pgfpathlineto{\pgfqpoint{1.023437in}{2.595949in}}%
\pgfpathlineto{\pgfqpoint{1.024292in}{2.493530in}}%
\pgfpathlineto{\pgfqpoint{1.030277in}{2.444511in}}%
\pgfpathlineto{\pgfqpoint{1.035409in}{2.533065in}}%
\pgfpathlineto{\pgfqpoint{1.039687in}{2.573356in}}%
\pgfpathlineto{\pgfqpoint{1.046538in}{2.426389in}}%
\pgfpathlineto{\pgfqpoint{1.051672in}{2.516514in}}%
\pgfpathlineto{\pgfqpoint{1.055092in}{2.427506in}}%
\pgfpathlineto{\pgfqpoint{1.059370in}{2.535843in}}%
\pgfpathlineto{\pgfqpoint{1.065358in}{2.549193in}}%
\pgfpathlineto{\pgfqpoint{1.067066in}{2.434965in}}%
\pgfpathlineto{\pgfqpoint{1.073048in}{2.538502in}}%
\pgfpathlineto{\pgfqpoint{1.076466in}{2.455443in}}%
\pgfpathlineto{\pgfqpoint{1.082458in}{2.574805in}}%
\pgfpathlineto{\pgfqpoint{1.087592in}{2.433697in}}%
\pgfpathlineto{\pgfqpoint{1.088447in}{2.497062in}}%
\pgfpathlineto{\pgfqpoint{1.093579in}{2.563873in}}%
\pgfpathlineto{\pgfqpoint{1.097851in}{2.467585in}}%
\pgfpathlineto{\pgfqpoint{1.103832in}{2.552878in}}%
\pgfpathlineto{\pgfqpoint{1.107250in}{2.428744in}}%
\pgfpathlineto{\pgfqpoint{1.112379in}{2.554692in}}%
\pgfpathlineto{\pgfqpoint{1.114088in}{2.448135in}}%
\pgfpathlineto{\pgfqpoint{1.120074in}{2.527086in}}%
\pgfpathlineto{\pgfqpoint{1.123495in}{2.440825in}}%
\pgfpathlineto{\pgfqpoint{1.126917in}{2.517964in}}%
\pgfpathlineto{\pgfqpoint{1.131195in}{2.419200in}}%
\pgfpathlineto{\pgfqpoint{1.136328in}{2.548529in}}%
\pgfpathlineto{\pgfqpoint{1.139749in}{2.450399in}}%
\pgfpathlineto{\pgfqpoint{1.145737in}{2.538139in}}%
\pgfpathlineto{\pgfqpoint{1.148303in}{2.449222in}}%
\pgfpathlineto{\pgfqpoint{1.155144in}{2.590209in}}%
\pgfpathlineto{\pgfqpoint{1.158563in}{2.477975in}}%
\pgfpathlineto{\pgfqpoint{1.161130in}{2.562000in}}%
\pgfpathlineto{\pgfqpoint{1.166264in}{2.553422in}}%
\pgfpathlineto{\pgfqpoint{1.172254in}{2.422371in}}%
\pgfpathlineto{\pgfqpoint{1.174823in}{2.567558in}}%
\pgfpathlineto{\pgfqpoint{1.179102in}{2.466620in}}%
\pgfpathlineto{\pgfqpoint{1.183380in}{2.571725in}}%
\pgfpathlineto{\pgfqpoint{1.189368in}{2.453088in}}%
\pgfpathlineto{\pgfqpoint{1.191079in}{2.535966in}}%
\pgfpathlineto{\pgfqpoint{1.198776in}{2.443908in}}%
\pgfpathlineto{\pgfqpoint{1.201344in}{2.534033in}}%
\pgfpathlineto{\pgfqpoint{1.203910in}{2.473566in}}%
\pgfpathlineto{\pgfqpoint{1.209045in}{2.570638in}}%
\pgfpathlineto{\pgfqpoint{1.213329in}{2.482868in}}%
\pgfpathlineto{\pgfqpoint{1.216755in}{2.540496in}}%
\pgfpathlineto{\pgfqpoint{1.222746in}{2.460459in}}%
\pgfpathlineto{\pgfqpoint{1.227880in}{2.440946in}}%
\pgfpathlineto{\pgfqpoint{1.229590in}{2.551912in}}%
\pgfpathlineto{\pgfqpoint{1.235580in}{2.469518in}}%
\pgfpathlineto{\pgfqpoint{1.239001in}{2.532402in}}%
\pgfpathlineto{\pgfqpoint{1.243278in}{2.475649in}}%
\pgfpathlineto{\pgfqpoint{1.248413in}{2.554027in}}%
\pgfpathlineto{\pgfqpoint{1.251837in}{2.455625in}}%
\pgfpathlineto{\pgfqpoint{1.257825in}{2.410199in}}%
\pgfpathlineto{\pgfqpoint{1.259533in}{2.581843in}}%
\pgfpathlineto{\pgfqpoint{1.266376in}{2.476586in}}%
\pgfpathlineto{\pgfqpoint{1.270656in}{2.598243in}}%
\pgfpathlineto{\pgfqpoint{1.274080in}{2.472478in}}%
\pgfpathlineto{\pgfqpoint{1.277502in}{2.571483in}}%
\pgfpathlineto{\pgfqpoint{1.280922in}{2.518175in}}%
\pgfpathlineto{\pgfqpoint{1.285201in}{2.558617in}}%
\pgfpathlineto{\pgfqpoint{1.290336in}{2.372416in}}%
\pgfpathlineto{\pgfqpoint{1.295468in}{2.494345in}}%
\pgfpathlineto{\pgfqpoint{1.299746in}{2.446111in}}%
\pgfpathlineto{\pgfqpoint{1.302314in}{2.523401in}}%
\pgfpathlineto{\pgfqpoint{1.310013in}{2.405790in}}%
\pgfpathlineto{\pgfqpoint{1.310869in}{2.502923in}}%
\pgfpathlineto{\pgfqpoint{1.318566in}{2.440190in}}%
\pgfpathlineto{\pgfqpoint{1.321131in}{2.519111in}}%
\pgfpathlineto{\pgfqpoint{1.325401in}{2.460457in}}%
\pgfpathlineto{\pgfqpoint{1.329678in}{2.563208in}}%
\pgfpathlineto{\pgfqpoint{1.333096in}{2.477612in}}%
\pgfpathlineto{\pgfqpoint{1.339081in}{2.527328in}}%
\pgfpathlineto{\pgfqpoint{1.341646in}{2.469458in}}%
\pgfpathlineto{\pgfqpoint{1.345066in}{2.432187in}}%
\pgfpathlineto{\pgfqpoint{1.351054in}{2.429470in}}%
\pgfpathlineto{\pgfqpoint{1.354477in}{2.550160in}}%
\pgfpathlineto{\pgfqpoint{1.362176in}{2.418898in}}%
\pgfpathlineto{\pgfqpoint{1.367303in}{2.491023in}}%
\pgfpathlineto{\pgfqpoint{1.370723in}{2.437685in}}%
\pgfpathlineto{\pgfqpoint{1.375853in}{2.521770in}}%
\pgfpathlineto{\pgfqpoint{1.381839in}{2.395158in}}%
\pgfpathlineto{\pgfqpoint{1.386113in}{2.528475in}}%
\pgfpathlineto{\pgfqpoint{1.391249in}{2.465501in}}%
\pgfpathlineto{\pgfqpoint{1.394673in}{2.531616in}}%
\pgfpathlineto{\pgfqpoint{1.398953in}{2.476223in}}%
\pgfpathlineto{\pgfqpoint{1.401521in}{2.563510in}}%
\pgfpathlineto{\pgfqpoint{1.408362in}{2.571183in}}%
\pgfpathlineto{\pgfqpoint{1.410925in}{2.466015in}}%
\pgfpathlineto{\pgfqpoint{1.418616in}{2.551791in}}%
\pgfpathlineto{\pgfqpoint{1.425464in}{2.426932in}}%
\pgfpathlineto{\pgfqpoint{1.427175in}{2.512587in}}%
\pgfpathlineto{\pgfqpoint{1.431452in}{2.541341in}}%
\pgfpathlineto{\pgfqpoint{1.434872in}{2.452726in}}%
\pgfpathlineto{\pgfqpoint{1.440002in}{2.437987in}}%
\pgfpathlineto{\pgfqpoint{1.444278in}{2.504856in}}%
\pgfpathlineto{\pgfqpoint{1.451122in}{2.436235in}}%
\pgfpathlineto{\pgfqpoint{1.453691in}{2.535906in}}%
\pgfpathlineto{\pgfqpoint{1.457970in}{2.449585in}}%
\pgfpathlineto{\pgfqpoint{1.463106in}{2.437203in}}%
\pgfpathlineto{\pgfqpoint{1.468244in}{2.536087in}}%
\pgfpathlineto{\pgfqpoint{1.470810in}{2.559283in}}%
\pgfpathlineto{\pgfqpoint{1.474232in}{2.453088in}}%
\pgfpathlineto{\pgfqpoint{1.480224in}{2.514581in}}%
\pgfpathlineto{\pgfqpoint{1.485359in}{2.441581in}}%
\pgfpathlineto{\pgfqpoint{1.486215in}{2.557773in}}%
\pgfpathlineto{\pgfqpoint{1.491345in}{2.500629in}}%
\pgfpathlineto{\pgfqpoint{1.496476in}{2.468432in}}%
\pgfpathlineto{\pgfqpoint{1.499039in}{2.574868in}}%
\pgfpathlineto{\pgfqpoint{1.505027in}{2.459372in}}%
\pgfpathlineto{\pgfqpoint{1.511013in}{2.539861in}}%
\pgfpathlineto{\pgfqpoint{1.512722in}{2.446262in}}%
\pgfpathlineto{\pgfqpoint{1.518707in}{2.430738in}}%
\pgfpathlineto{\pgfqpoint{1.520417in}{2.530257in}}%
\pgfpathlineto{\pgfqpoint{1.528112in}{2.462995in}}%
\pgfpathlineto{\pgfqpoint{1.531534in}{2.538200in}}%
\pgfpathlineto{\pgfqpoint{1.535810in}{2.445597in}}%
\pgfpathlineto{\pgfqpoint{1.540941in}{2.558859in}}%
\pgfpathlineto{\pgfqpoint{1.541797in}{2.461515in}}%
\pgfpathlineto{\pgfqpoint{1.548638in}{2.506668in}}%
\pgfpathlineto{\pgfqpoint{1.551207in}{2.432580in}}%
\pgfpathlineto{\pgfqpoint{1.556343in}{2.434120in}}%
\pgfpathlineto{\pgfqpoint{1.558909in}{2.535843in}}%
\pgfpathlineto{\pgfqpoint{1.566605in}{2.550281in}}%
\pgfpathlineto{\pgfqpoint{1.567459in}{2.488124in}}%
\pgfpathlineto{\pgfqpoint{1.575167in}{2.437745in}}%
\pgfpathlineto{\pgfqpoint{1.577737in}{2.556986in}}%
\pgfpathlineto{\pgfqpoint{1.582014in}{2.419079in}}%
\pgfpathlineto{\pgfqpoint{1.585439in}{2.525394in}}%
\pgfpathlineto{\pgfqpoint{1.591423in}{2.470817in}}%
\pgfpathlineto{\pgfqpoint{1.596559in}{2.534757in}}%
\pgfpathlineto{\pgfqpoint{1.599126in}{2.428744in}}%
\pgfpathlineto{\pgfqpoint{1.603404in}{2.571241in}}%
\pgfpathlineto{\pgfqpoint{1.607680in}{2.461936in}}%
\pgfpathlineto{\pgfqpoint{1.613666in}{2.533428in}}%
\pgfpathlineto{\pgfqpoint{1.617087in}{2.470786in}}%
\pgfpathlineto{\pgfqpoint{1.618798in}{2.531979in}}%
\pgfpathlineto{\pgfqpoint{1.625639in}{2.579700in}}%
\pgfpathlineto{\pgfqpoint{1.630773in}{2.459914in}}%
\pgfpathlineto{\pgfqpoint{1.635054in}{2.558436in}}%
\pgfpathlineto{\pgfqpoint{1.636766in}{2.448196in}}%
\pgfpathlineto{\pgfqpoint{1.641889in}{2.568282in}}%
\pgfpathlineto{\pgfqpoint{1.648727in}{2.450671in}}%
\pgfpathlineto{\pgfqpoint{1.656428in}{2.523219in}}%
\pgfpathlineto{\pgfqpoint{1.658993in}{2.424395in}}%
\pgfpathlineto{\pgfqpoint{1.664122in}{2.570336in}}%
\pgfpathlineto{\pgfqpoint{1.666687in}{2.455745in}}%
\pgfpathlineto{\pgfqpoint{1.672676in}{2.585740in}}%
\pgfpathlineto{\pgfqpoint{1.676098in}{2.398148in}}%
\pgfpathlineto{\pgfqpoint{1.681229in}{2.527810in}}%
\pgfpathlineto{\pgfqpoint{1.682939in}{2.447319in}}%
\pgfpathlineto{\pgfqpoint{1.687214in}{2.485225in}}%
\pgfpathlineto{\pgfqpoint{1.693199in}{2.430314in}}%
\pgfpathlineto{\pgfqpoint{1.695764in}{2.524487in}}%
\pgfpathlineto{\pgfqpoint{1.700041in}{2.445416in}}%
\pgfpathlineto{\pgfqpoint{1.704314in}{2.428774in}}%
\pgfpathlineto{\pgfqpoint{1.709447in}{2.508904in}}%
\pgfpathlineto{\pgfqpoint{1.714577in}{2.476949in}}%
\pgfpathlineto{\pgfqpoint{1.723132in}{2.594590in}}%
\pgfpathlineto{\pgfqpoint{1.725698in}{2.493863in}}%
\pgfpathlineto{\pgfqpoint{1.730826in}{2.435751in}}%
\pgfpathlineto{\pgfqpoint{1.734249in}{2.521407in}}%
\pgfpathlineto{\pgfqpoint{1.738528in}{2.478157in}}%
\pgfpathlineto{\pgfqpoint{1.746231in}{2.442184in}}%
\pgfpathlineto{\pgfqpoint{1.748801in}{2.550281in}}%
\pgfpathlineto{\pgfqpoint{1.753081in}{2.430435in}}%
\pgfpathlineto{\pgfqpoint{1.758211in}{2.542611in}}%
\pgfpathlineto{\pgfqpoint{1.761630in}{2.501352in}}%
\pgfpathlineto{\pgfqpoint{1.764196in}{2.531616in}}%
\pgfpathlineto{\pgfqpoint{1.770178in}{2.457104in}}%
\pgfpathlineto{\pgfqpoint{1.775307in}{2.529864in}}%
\pgfpathlineto{\pgfqpoint{1.780436in}{2.537234in}}%
\pgfpathlineto{\pgfqpoint{1.781291in}{2.419442in}}%
\pgfpathlineto{\pgfqpoint{1.787279in}{2.538139in}}%
\pgfpathlineto{\pgfqpoint{1.790704in}{2.489392in}}%
\pgfpathlineto{\pgfqpoint{1.794127in}{2.572088in}}%
\pgfpathlineto{\pgfqpoint{1.803538in}{2.429772in}}%
\pgfpathlineto{\pgfqpoint{1.808671in}{2.549013in}}%
\pgfpathlineto{\pgfqpoint{1.811238in}{2.482929in}}%
\pgfpathlineto{\pgfqpoint{1.817226in}{2.577283in}}%
\pgfpathlineto{\pgfqpoint{1.821502in}{2.465231in}}%
\pgfpathlineto{\pgfqpoint{1.824922in}{2.549800in}}%
\pgfpathlineto{\pgfqpoint{1.829199in}{2.450069in}}%
\pgfpathlineto{\pgfqpoint{1.835185in}{2.534093in}}%
\pgfpathlineto{\pgfqpoint{1.838605in}{2.457981in}}%
\pgfpathlineto{\pgfqpoint{1.842028in}{2.547745in}}%
\pgfpathlineto{\pgfqpoint{1.847161in}{2.469337in}}%
\pgfpathlineto{\pgfqpoint{1.849731in}{2.552033in}}%
\pgfpathlineto{\pgfqpoint{1.854859in}{2.546264in}}%
\pgfpathlineto{\pgfqpoint{1.861702in}{2.393316in}}%
\pgfpathlineto{\pgfqpoint{1.865977in}{2.534877in}}%
\pgfpathlineto{\pgfqpoint{1.869400in}{2.460578in}}%
\pgfpathlineto{\pgfqpoint{1.871963in}{2.585740in}}%
\pgfpathlineto{\pgfqpoint{1.875380in}{2.608635in}}%
\pgfpathlineto{\pgfqpoint{1.880514in}{2.452060in}}%
\pgfpathlineto{\pgfqpoint{1.886505in}{2.548832in}}%
\pgfpathlineto{\pgfqpoint{1.888217in}{2.450550in}}%
\pgfpathlineto{\pgfqpoint{1.893350in}{2.566409in}}%
\pgfpathlineto{\pgfqpoint{1.896771in}{2.475861in}}%
\pgfpathlineto{\pgfqpoint{1.903618in}{2.538200in}}%
\pgfpathlineto{\pgfqpoint{1.906183in}{2.459310in}}%
\pgfpathlineto{\pgfqpoint{1.909606in}{2.515728in}}%
\pgfpathlineto{\pgfqpoint{1.916449in}{2.575047in}}%
\pgfpathlineto{\pgfqpoint{1.921587in}{2.445960in}}%
\pgfpathlineto{\pgfqpoint{1.922443in}{2.555053in}}%
\pgfpathlineto{\pgfqpoint{1.927577in}{2.456653in}}%
\pgfpathlineto{\pgfqpoint{1.933561in}{2.495796in}}%
\pgfpathlineto{\pgfqpoint{1.937837in}{2.449887in}}%
\pgfpathlineto{\pgfqpoint{1.941258in}{2.577767in}}%
\pgfpathlineto{\pgfqpoint{1.947248in}{2.464747in}}%
\pgfpathlineto{\pgfqpoint{1.949816in}{2.543274in}}%
\pgfpathlineto{\pgfqpoint{1.955805in}{2.523280in}}%
\pgfpathlineto{\pgfqpoint{1.958374in}{2.408056in}}%
\pgfpathlineto{\pgfqpoint{1.960940in}{2.481781in}}%
\pgfpathlineto{\pgfqpoint{1.965220in}{2.559222in}}%
\pgfpathlineto{\pgfqpoint{1.972063in}{2.459975in}}%
\pgfpathlineto{\pgfqpoint{1.974626in}{2.559131in}}%
\pgfpathlineto{\pgfqpoint{1.980610in}{2.460459in}}%
\pgfpathlineto{\pgfqpoint{1.982321in}{2.542308in}}%
\pgfpathlineto{\pgfqpoint{1.986595in}{2.468644in}}%
\pgfpathlineto{\pgfqpoint{1.993441in}{2.555446in}}%
\pgfpathlineto{\pgfqpoint{1.995151in}{2.559041in}}%
\pgfpathlineto{\pgfqpoint{1.999427in}{2.460519in}}%
\pgfpathlineto{\pgfqpoint{2.003702in}{2.524308in}}%
\pgfpathlineto{\pgfqpoint{2.009691in}{2.444511in}}%
\pgfpathlineto{\pgfqpoint{2.013969in}{2.546054in}}%
\pgfpathlineto{\pgfqpoint{2.019956in}{2.592324in}}%
\pgfpathlineto{\pgfqpoint{2.024234in}{2.460699in}}%
\pgfpathlineto{\pgfqpoint{2.025946in}{2.528475in}}%
\pgfpathlineto{\pgfqpoint{2.031077in}{2.481086in}}%
\pgfpathlineto{\pgfqpoint{2.033644in}{2.576376in}}%
\pgfpathlineto{\pgfqpoint{2.039628in}{2.482929in}}%
\pgfpathlineto{\pgfqpoint{2.043906in}{2.568887in}}%
\pgfpathlineto{\pgfqpoint{2.048183in}{2.476647in}}%
\pgfpathlineto{\pgfqpoint{2.050751in}{2.582660in}}%
\pgfpathlineto{\pgfqpoint{2.055027in}{2.469579in}}%
\pgfpathlineto{\pgfqpoint{2.060159in}{2.570941in}}%
\pgfpathlineto{\pgfqpoint{2.064434in}{2.476223in}}%
\pgfpathlineto{\pgfqpoint{2.068710in}{2.417781in}}%
\pgfpathlineto{\pgfqpoint{2.072131in}{2.524066in}}%
\pgfpathlineto{\pgfqpoint{2.076406in}{2.474232in}}%
\pgfpathlineto{\pgfqpoint{2.080685in}{2.566711in}}%
\pgfpathlineto{\pgfqpoint{2.088377in}{2.561879in}}%
\pgfpathlineto{\pgfqpoint{2.090942in}{2.458042in}}%
\pgfpathlineto{\pgfqpoint{2.093509in}{2.572693in}}%
\pgfpathlineto{\pgfqpoint{2.100359in}{2.447712in}}%
\pgfpathlineto{\pgfqpoint{2.102070in}{2.543455in}}%
\pgfpathlineto{\pgfqpoint{2.109774in}{2.570094in}}%
\pgfpathlineto{\pgfqpoint{2.110631in}{2.483955in}}%
\pgfpathlineto{\pgfqpoint{2.116622in}{2.597156in}}%
\pgfpathlineto{\pgfqpoint{2.122605in}{2.489271in}}%
\pgfpathlineto{\pgfqpoint{2.123462in}{2.557107in}}%
\pgfpathlineto{\pgfqpoint{2.130310in}{2.440765in}}%
\pgfpathlineto{\pgfqpoint{2.133734in}{2.515125in}}%
\pgfpathlineto{\pgfqpoint{2.139720in}{2.481479in}}%
\pgfpathlineto{\pgfqpoint{2.140576in}{2.583686in}}%
\pgfpathlineto{\pgfqpoint{2.146558in}{2.435298in}}%
\pgfpathlineto{\pgfqpoint{2.149976in}{2.562302in}}%
\pgfpathlineto{\pgfqpoint{2.156813in}{2.425966in}}%
\pgfpathlineto{\pgfqpoint{2.159379in}{2.606821in}}%
\pgfpathlineto{\pgfqpoint{2.162803in}{2.477068in}}%
\pgfpathlineto{\pgfqpoint{2.169647in}{2.469428in}}%
\pgfpathlineto{\pgfqpoint{2.172214in}{2.608724in}}%
\pgfpathlineto{\pgfqpoint{2.174781in}{2.487610in}}%
\pgfpathlineto{\pgfqpoint{2.181627in}{2.463779in}}%
\pgfpathlineto{\pgfqpoint{2.184193in}{2.529199in}}%
\pgfpathlineto{\pgfqpoint{2.191037in}{2.446926in}}%
\pgfpathlineto{\pgfqpoint{2.195315in}{2.533005in}}%
\pgfpathlineto{\pgfqpoint{2.199592in}{2.489755in}}%
\pgfpathlineto{\pgfqpoint{2.200449in}{2.512980in}}%
\pgfpathlineto{\pgfqpoint{2.205581in}{2.465591in}}%
\pgfpathlineto{\pgfqpoint{2.209001in}{2.529803in}}%
\pgfpathlineto{\pgfqpoint{2.213282in}{2.559643in}}%
\pgfpathlineto{\pgfqpoint{2.220121in}{2.455564in}}%
\pgfpathlineto{\pgfqpoint{2.224396in}{2.577525in}}%
\pgfpathlineto{\pgfqpoint{2.226107in}{2.493772in}}%
\pgfpathlineto{\pgfqpoint{2.230384in}{2.550342in}}%
\pgfpathlineto{\pgfqpoint{2.236373in}{2.483955in}}%
\pgfpathlineto{\pgfqpoint{2.240649in}{2.580666in}}%
\pgfpathlineto{\pgfqpoint{2.245780in}{2.502136in}}%
\pgfpathlineto{\pgfqpoint{2.249197in}{2.599027in}}%
\pgfpathlineto{\pgfqpoint{2.252617in}{2.485767in}}%
\pgfpathlineto{\pgfqpoint{2.256038in}{2.598061in}}%
\pgfpathlineto{\pgfqpoint{2.261169in}{2.462088in}}%
\pgfpathlineto{\pgfqpoint{2.265446in}{2.457437in}}%
\pgfpathlineto{\pgfqpoint{2.270579in}{2.563691in}}%
\pgfpathlineto{\pgfqpoint{2.274003in}{2.425966in}}%
\pgfpathlineto{\pgfqpoint{2.279133in}{2.588215in}}%
\pgfpathlineto{\pgfqpoint{2.281703in}{2.524487in}}%
\pgfpathlineto{\pgfqpoint{2.288549in}{2.486614in}}%
\pgfpathlineto{\pgfqpoint{2.293687in}{2.559885in}}%
\pgfpathlineto{\pgfqpoint{2.294543in}{2.473988in}}%
\pgfpathlineto{\pgfqpoint{2.300531in}{2.545449in}}%
\pgfpathlineto{\pgfqpoint{2.306519in}{2.450490in}}%
\pgfpathlineto{\pgfqpoint{2.309941in}{2.603922in}}%
\pgfpathlineto{\pgfqpoint{2.311649in}{2.488245in}}%
\pgfpathlineto{\pgfqpoint{2.317638in}{2.565322in}}%
\pgfpathlineto{\pgfqpoint{2.322768in}{2.435540in}}%
\pgfpathlineto{\pgfqpoint{2.324477in}{2.481177in}}%
\pgfpathlineto{\pgfqpoint{2.332180in}{2.436687in}}%
\pgfpathlineto{\pgfqpoint{2.333891in}{2.510896in}}%
\pgfpathlineto{\pgfqpoint{2.339879in}{2.472599in}}%
\pgfpathlineto{\pgfqpoint{2.341587in}{2.550191in}}%
\pgfpathlineto{\pgfqpoint{2.347573in}{2.470484in}}%
\pgfpathlineto{\pgfqpoint{2.353560in}{2.472599in}}%
\pgfpathlineto{\pgfqpoint{2.356982in}{2.581934in}}%
\pgfpathlineto{\pgfqpoint{2.359552in}{2.517119in}}%
\pgfpathlineto{\pgfqpoint{2.365540in}{2.434907in}}%
\pgfpathlineto{\pgfqpoint{2.368963in}{2.546778in}}%
\pgfpathlineto{\pgfqpoint{2.373239in}{2.550221in}}%
\pgfpathlineto{\pgfqpoint{2.379228in}{2.445597in}}%
\pgfpathlineto{\pgfqpoint{2.380084in}{2.583867in}}%
\pgfpathlineto{\pgfqpoint{2.386070in}{2.464686in}}%
\pgfpathlineto{\pgfqpoint{2.388635in}{2.522042in}}%
\pgfpathlineto{\pgfqpoint{2.395480in}{2.489936in}}%
\pgfpathlineto{\pgfqpoint{2.398901in}{2.575168in}}%
\pgfpathlineto{\pgfqpoint{2.404887in}{2.495433in}}%
\pgfpathlineto{\pgfqpoint{2.406597in}{2.547322in}}%
\pgfpathlineto{\pgfqpoint{2.412580in}{2.451034in}}%
\pgfpathlineto{\pgfqpoint{2.415143in}{2.554087in}}%
\pgfpathlineto{\pgfqpoint{2.421985in}{2.462269in}}%
\pgfpathlineto{\pgfqpoint{2.424553in}{2.530166in}}%
\pgfpathlineto{\pgfqpoint{2.429680in}{2.556503in}}%
\pgfpathlineto{\pgfqpoint{2.433100in}{2.487761in}}%
\pgfpathlineto{\pgfqpoint{2.439086in}{2.555476in}}%
\pgfpathlineto{\pgfqpoint{2.440796in}{2.525092in}}%
\pgfpathlineto{\pgfqpoint{2.445075in}{2.474199in}}%
\pgfpathlineto{\pgfqpoint{2.449351in}{2.589546in}}%
\pgfpathlineto{\pgfqpoint{2.455339in}{2.462269in}}%
\pgfpathlineto{\pgfqpoint{2.460475in}{2.579216in}}%
\pgfpathlineto{\pgfqpoint{2.464751in}{2.518750in}}%
\pgfpathlineto{\pgfqpoint{2.468174in}{2.601265in}}%
\pgfpathlineto{\pgfqpoint{2.471595in}{2.492595in}}%
\pgfpathlineto{\pgfqpoint{2.474162in}{2.564478in}}%
\pgfpathlineto{\pgfqpoint{2.480149in}{2.466105in}}%
\pgfpathlineto{\pgfqpoint{2.484427in}{2.532523in}}%
\pgfpathlineto{\pgfqpoint{2.489559in}{2.451700in}}%
\pgfpathlineto{\pgfqpoint{2.492982in}{2.570036in}}%
\pgfpathlineto{\pgfqpoint{2.497259in}{2.440041in}}%
\pgfpathlineto{\pgfqpoint{2.500682in}{2.539107in}}%
\pgfpathlineto{\pgfqpoint{2.505817in}{2.560793in}}%
\pgfpathlineto{\pgfqpoint{2.508382in}{2.466680in}}%
\pgfpathlineto{\pgfqpoint{2.512659in}{2.535845in}}%
\pgfpathlineto{\pgfqpoint{2.519502in}{2.439739in}}%
\pgfpathlineto{\pgfqpoint{2.522926in}{2.534063in}}%
\pgfpathlineto{\pgfqpoint{2.526346in}{2.571062in}}%
\pgfpathlineto{\pgfqpoint{2.531478in}{2.477494in}}%
\pgfpathlineto{\pgfqpoint{2.535758in}{2.456532in}}%
\pgfpathlineto{\pgfqpoint{2.538324in}{2.556021in}}%
\pgfpathlineto{\pgfqpoint{2.544308in}{2.496460in}}%
\pgfpathlineto{\pgfqpoint{2.546873in}{2.564538in}}%
\pgfpathlineto{\pgfqpoint{2.551150in}{2.472208in}}%
\pgfpathlineto{\pgfqpoint{2.555422in}{2.535089in}}%
\pgfpathlineto{\pgfqpoint{2.563119in}{2.454477in}}%
\pgfpathlineto{\pgfqpoint{2.563974in}{2.538684in}}%
\pgfpathlineto{\pgfqpoint{2.569960in}{2.443182in}}%
\pgfpathlineto{\pgfqpoint{2.575949in}{2.584956in}}%
\pgfpathlineto{\pgfqpoint{2.576804in}{2.486704in}}%
\pgfpathlineto{\pgfqpoint{2.581931in}{2.436296in}}%
\pgfpathlineto{\pgfqpoint{2.587916in}{2.595044in}}%
\pgfpathlineto{\pgfqpoint{2.590484in}{2.451700in}}%
\pgfpathlineto{\pgfqpoint{2.595611in}{2.538986in}}%
\pgfpathlineto{\pgfqpoint{2.599886in}{2.557168in}}%
\pgfpathlineto{\pgfqpoint{2.603308in}{2.423369in}}%
\pgfpathlineto{\pgfqpoint{2.606733in}{2.529261in}}%
\pgfpathlineto{\pgfqpoint{2.611867in}{2.495071in}}%
\pgfpathlineto{\pgfqpoint{2.615290in}{2.565383in}}%
\pgfpathlineto{\pgfqpoint{2.622129in}{2.464142in}}%
\pgfpathlineto{\pgfqpoint{2.624695in}{2.553483in}}%
\pgfpathlineto{\pgfqpoint{2.631542in}{2.445960in}}%
\pgfpathlineto{\pgfqpoint{2.633255in}{2.531918in}}%
\pgfpathlineto{\pgfqpoint{2.636679in}{2.566953in}}%
\pgfpathlineto{\pgfqpoint{2.643518in}{2.439316in}}%
\pgfpathlineto{\pgfqpoint{2.647797in}{2.530166in}}%
\pgfpathlineto{\pgfqpoint{2.652077in}{2.487037in}}%
\pgfpathlineto{\pgfqpoint{2.655500in}{2.555839in}}%
\pgfpathlineto{\pgfqpoint{2.659777in}{2.486614in}}%
\pgfpathlineto{\pgfqpoint{2.663199in}{2.565685in}}%
\pgfpathlineto{\pgfqpoint{2.666621in}{2.493440in}}%
\pgfpathlineto{\pgfqpoint{2.670900in}{2.562968in}}%
\pgfpathlineto{\pgfqpoint{2.676032in}{2.478157in}}%
\pgfpathlineto{\pgfqpoint{2.681163in}{2.473385in}}%
\pgfpathlineto{\pgfqpoint{2.686295in}{2.579216in}}%
\pgfpathlineto{\pgfqpoint{2.689720in}{2.478308in}}%
\pgfpathlineto{\pgfqpoint{2.693143in}{2.560067in}}%
\pgfpathlineto{\pgfqpoint{2.696565in}{2.460035in}}%
\pgfpathlineto{\pgfqpoint{2.703410in}{2.537839in}}%
\pgfpathlineto{\pgfqpoint{2.708544in}{2.434090in}}%
\pgfpathlineto{\pgfqpoint{2.710256in}{2.562605in}}%
\pgfpathlineto{\pgfqpoint{2.714525in}{2.466196in}}%
\pgfpathlineto{\pgfqpoint{2.718798in}{2.525032in}}%
\pgfpathlineto{\pgfqpoint{2.723934in}{2.461485in}}%
\pgfpathlineto{\pgfqpoint{2.728213in}{2.562363in}}%
\pgfpathlineto{\pgfqpoint{2.731635in}{2.473325in}}%
\pgfpathlineto{\pgfqpoint{2.736769in}{2.521528in}}%
\pgfpathlineto{\pgfqpoint{2.741046in}{2.460396in}}%
\pgfpathlineto{\pgfqpoint{2.743612in}{2.553362in}}%
\pgfpathlineto{\pgfqpoint{2.751304in}{2.430677in}}%
\pgfpathlineto{\pgfqpoint{2.752160in}{2.513797in}}%
\pgfpathlineto{\pgfqpoint{2.758145in}{2.497276in}}%
\pgfpathlineto{\pgfqpoint{2.762417in}{2.573840in}}%
\pgfpathlineto{\pgfqpoint{2.764980in}{2.510777in}}%
\pgfpathlineto{\pgfqpoint{2.772681in}{2.478520in}}%
\pgfpathlineto{\pgfqpoint{2.775247in}{2.533791in}}%
\pgfpathlineto{\pgfqpoint{2.781232in}{2.562605in}}%
\pgfpathlineto{\pgfqpoint{2.782089in}{2.481661in}}%
\pgfpathlineto{\pgfqpoint{2.789789in}{2.534517in}}%
\pgfpathlineto{\pgfqpoint{2.790642in}{2.446504in}}%
\pgfpathlineto{\pgfqpoint{2.794918in}{2.511319in}}%
\pgfpathlineto{\pgfqpoint{2.799189in}{2.451518in}}%
\pgfpathlineto{\pgfqpoint{2.806031in}{2.567679in}}%
\pgfpathlineto{\pgfqpoint{2.810305in}{2.609661in}}%
\pgfpathlineto{\pgfqpoint{2.812017in}{2.505582in}}%
\pgfpathlineto{\pgfqpoint{2.818005in}{2.572390in}}%
\pgfpathlineto{\pgfqpoint{2.821424in}{2.496339in}}%
\pgfpathlineto{\pgfqpoint{2.827409in}{2.577525in}}%
\pgfpathlineto{\pgfqpoint{2.831684in}{2.442819in}}%
\pgfpathlineto{\pgfqpoint{2.836818in}{2.590814in}}%
\pgfpathlineto{\pgfqpoint{2.837671in}{2.480030in}}%
\pgfpathlineto{\pgfqpoint{2.844515in}{2.561034in}}%
\pgfpathlineto{\pgfqpoint{2.847078in}{2.478338in}}%
\pgfpathlineto{\pgfqpoint{2.853918in}{2.407542in}}%
\pgfpathlineto{\pgfqpoint{2.856482in}{2.546868in}}%
\pgfpathlineto{\pgfqpoint{2.860755in}{2.464505in}}%
\pgfpathlineto{\pgfqpoint{2.866743in}{2.592024in}}%
\pgfpathlineto{\pgfqpoint{2.868455in}{2.490601in}}%
\pgfpathlineto{\pgfqpoint{2.871877in}{2.547806in}}%
\pgfpathlineto{\pgfqpoint{2.877860in}{2.470726in}}%
\pgfpathlineto{\pgfqpoint{2.882135in}{2.544723in}}%
\pgfpathlineto{\pgfqpoint{2.886413in}{2.565685in}}%
\pgfpathlineto{\pgfqpoint{2.889835in}{2.430284in}}%
\pgfpathlineto{\pgfqpoint{2.896681in}{2.580424in}}%
\pgfpathlineto{\pgfqpoint{2.898392in}{2.476103in}}%
\pgfpathlineto{\pgfqpoint{2.905233in}{2.564115in}}%
\pgfpathlineto{\pgfqpoint{2.906089in}{2.480421in}}%
\pgfpathlineto{\pgfqpoint{2.913784in}{2.535301in}}%
\pgfpathlineto{\pgfqpoint{2.917204in}{2.429046in}}%
\pgfpathlineto{\pgfqpoint{2.918914in}{2.543516in}}%
\pgfpathlineto{\pgfqpoint{2.923190in}{2.415336in}}%
\pgfpathlineto{\pgfqpoint{2.929171in}{2.520925in}}%
\pgfpathlineto{\pgfqpoint{2.932595in}{2.487460in}}%
\pgfpathlineto{\pgfqpoint{2.938585in}{2.646872in}}%
\pgfpathlineto{\pgfqpoint{2.942004in}{2.551249in}}%
\pgfpathlineto{\pgfqpoint{2.945424in}{2.659316in}}%
\pgfpathlineto{\pgfqpoint{2.949701in}{2.570157in}}%
\pgfpathlineto{\pgfqpoint{2.956540in}{2.656296in}}%
\pgfpathlineto{\pgfqpoint{2.960820in}{2.538926in}}%
\pgfpathlineto{\pgfqpoint{2.963385in}{2.630079in}}%
\pgfpathlineto{\pgfqpoint{2.966806in}{2.655631in}}%
\pgfpathlineto{\pgfqpoint{2.970229in}{2.505943in}}%
\pgfpathlineto{\pgfqpoint{2.975364in}{2.545570in}}%
\pgfpathlineto{\pgfqpoint{2.980495in}{2.477975in}}%
\pgfpathlineto{\pgfqpoint{2.986484in}{2.516998in}}%
\pgfpathlineto{\pgfqpoint{2.990761in}{2.464898in}}%
\pgfpathlineto{\pgfqpoint{2.992470in}{2.561879in}}%
\pgfpathlineto{\pgfqpoint{2.998455in}{2.457497in}}%
\pgfpathlineto{\pgfqpoint{3.001024in}{2.575289in}}%
\pgfpathlineto{\pgfqpoint{3.007011in}{2.474472in}}%
\pgfpathlineto{\pgfqpoint{3.009578in}{2.558436in}}%
\pgfpathlineto{\pgfqpoint{3.012997in}{2.468732in}}%
\pgfpathlineto{\pgfqpoint{3.018130in}{2.465833in}}%
\pgfpathlineto{\pgfqpoint{3.024973in}{2.531253in}}%
\pgfpathlineto{\pgfqpoint{3.029249in}{2.446805in}}%
\pgfpathlineto{\pgfqpoint{3.031813in}{2.568161in}}%
\pgfpathlineto{\pgfqpoint{3.035233in}{2.516605in}}%
\pgfpathlineto{\pgfqpoint{3.040363in}{2.562875in}}%
\pgfpathlineto{\pgfqpoint{3.042930in}{2.468793in}}%
\pgfpathlineto{\pgfqpoint{3.048063in}{2.574082in}}%
\pgfpathlineto{\pgfqpoint{3.052341in}{2.462571in}}%
\pgfpathlineto{\pgfqpoint{3.055765in}{2.573961in}}%
\pgfpathlineto{\pgfqpoint{3.062609in}{2.496339in}}%
\pgfpathlineto{\pgfqpoint{3.065175in}{2.539589in}}%
\pgfpathlineto{\pgfqpoint{3.071163in}{2.529440in}}%
\pgfpathlineto{\pgfqpoint{3.073730in}{2.472780in}}%
\pgfpathlineto{\pgfqpoint{3.078860in}{2.542248in}}%
\pgfpathlineto{\pgfqpoint{3.084849in}{2.439406in}}%
\pgfpathlineto{\pgfqpoint{3.086559in}{2.569429in}}%
\pgfpathlineto{\pgfqpoint{3.090835in}{2.469760in}}%
\pgfpathlineto{\pgfqpoint{3.096823in}{2.421043in}}%
\pgfpathlineto{\pgfqpoint{3.099388in}{2.517180in}}%
\pgfpathlineto{\pgfqpoint{3.102807in}{2.397878in}}%
\pgfpathlineto{\pgfqpoint{3.107087in}{2.532160in}}%
\pgfpathlineto{\pgfqpoint{3.113076in}{2.470274in}}%
\pgfpathlineto{\pgfqpoint{3.119072in}{2.587008in}}%
\pgfpathlineto{\pgfqpoint{3.119928in}{2.510535in}}%
\pgfpathlineto{\pgfqpoint{3.125916in}{2.474714in}}%
\pgfpathlineto{\pgfqpoint{3.128480in}{2.538805in}}%
\pgfpathlineto{\pgfqpoint{3.136178in}{2.608996in}}%
\pgfpathlineto{\pgfqpoint{3.137034in}{2.481479in}}%
\pgfpathlineto{\pgfqpoint{3.144728in}{2.521467in}}%
\pgfpathlineto{\pgfqpoint{3.145581in}{2.452907in}}%
\pgfpathlineto{\pgfqpoint{3.153277in}{2.425936in}}%
\pgfpathlineto{\pgfqpoint{3.154131in}{2.542066in}}%
\pgfpathlineto{\pgfqpoint{3.161832in}{2.465773in}}%
\pgfpathlineto{\pgfqpoint{3.164395in}{2.531737in}}%
\pgfpathlineto{\pgfqpoint{3.168665in}{2.560672in}}%
\pgfpathlineto{\pgfqpoint{3.172086in}{2.418656in}}%
\pgfpathlineto{\pgfqpoint{3.175506in}{2.568705in}}%
\pgfpathlineto{\pgfqpoint{3.181495in}{2.499479in}}%
\pgfpathlineto{\pgfqpoint{3.184063in}{2.590935in}}%
\pgfpathlineto{\pgfqpoint{3.191760in}{2.513192in}}%
\pgfpathlineto{\pgfqpoint{3.196038in}{2.556503in}}%
\pgfpathlineto{\pgfqpoint{3.197748in}{2.434423in}}%
\pgfpathlineto{\pgfqpoint{3.202885in}{2.569128in}}%
\pgfpathlineto{\pgfqpoint{3.205451in}{2.454659in}}%
\pgfpathlineto{\pgfqpoint{3.211439in}{2.502862in}}%
\pgfpathlineto{\pgfqpoint{3.217427in}{2.522856in}}%
\pgfpathlineto{\pgfqpoint{3.221702in}{2.452302in}}%
\pgfpathlineto{\pgfqpoint{3.224266in}{2.528112in}}%
\pgfpathlineto{\pgfqpoint{3.226834in}{2.437261in}}%
\pgfpathlineto{\pgfqpoint{3.231112in}{2.519232in}}%
\pgfpathlineto{\pgfqpoint{3.237954in}{2.474804in}}%
\pgfpathlineto{\pgfqpoint{3.243083in}{2.541401in}}%
\pgfpathlineto{\pgfqpoint{3.246503in}{2.455685in}}%
\pgfpathlineto{\pgfqpoint{3.248209in}{2.549979in}}%
\pgfpathlineto{\pgfqpoint{3.255909in}{2.558496in}}%
\pgfpathlineto{\pgfqpoint{3.256765in}{2.447349in}}%
\pgfpathlineto{\pgfqpoint{3.264463in}{2.558920in}}%
\pgfpathlineto{\pgfqpoint{3.268739in}{2.479788in}}%
\pgfpathlineto{\pgfqpoint{3.273013in}{2.531283in}}%
\pgfpathlineto{\pgfqpoint{3.276437in}{2.544905in}}%
\pgfpathlineto{\pgfqpoint{3.281573in}{2.421011in}}%
\pgfpathlineto{\pgfqpoint{3.283283in}{2.564355in}}%
\pgfpathlineto{\pgfqpoint{3.287561in}{2.495552in}}%
\pgfpathlineto{\pgfqpoint{3.294401in}{2.571302in}}%
\pgfpathlineto{\pgfqpoint{3.298673in}{2.464505in}}%
\pgfpathlineto{\pgfqpoint{3.299528in}{2.549798in}}%
\pgfpathlineto{\pgfqpoint{3.303802in}{2.461182in}}%
\pgfpathlineto{\pgfqpoint{3.308077in}{2.500747in}}%
\pgfpathlineto{\pgfqpoint{3.315769in}{2.437745in}}%
\pgfpathlineto{\pgfqpoint{3.316624in}{2.560430in}}%
\pgfpathlineto{\pgfqpoint{3.324328in}{2.574866in}}%
\pgfpathlineto{\pgfqpoint{3.325184in}{2.471694in}}%
\pgfpathlineto{\pgfqpoint{3.329460in}{2.444450in}}%
\pgfpathlineto{\pgfqpoint{3.333737in}{2.563570in}}%
\pgfpathlineto{\pgfqpoint{3.338014in}{2.480451in}}%
\pgfpathlineto{\pgfqpoint{3.342293in}{2.560188in}}%
\pgfpathlineto{\pgfqpoint{3.349137in}{2.444087in}}%
\pgfpathlineto{\pgfqpoint{3.351703in}{2.533065in}}%
\pgfpathlineto{\pgfqpoint{3.355122in}{2.487216in}}%
\pgfpathlineto{\pgfqpoint{3.361114in}{2.566590in}}%
\pgfpathlineto{\pgfqpoint{3.365390in}{2.457376in}}%
\pgfpathlineto{\pgfqpoint{3.371379in}{2.566772in}}%
\pgfpathlineto{\pgfqpoint{3.372234in}{2.516635in}}%
\pgfpathlineto{\pgfqpoint{3.379931in}{2.375827in}}%
\pgfpathlineto{\pgfqpoint{3.380784in}{2.538803in}}%
\pgfpathlineto{\pgfqpoint{3.385062in}{2.470363in}}%
\pgfpathlineto{\pgfqpoint{3.389340in}{2.589544in}}%
\pgfpathlineto{\pgfqpoint{3.397035in}{2.443422in}}%
\pgfpathlineto{\pgfqpoint{3.397890in}{2.569126in}}%
\pgfpathlineto{\pgfqpoint{3.402168in}{2.581571in}}%
\pgfpathlineto{\pgfqpoint{3.406443in}{2.480330in}}%
\pgfpathlineto{\pgfqpoint{3.412430in}{2.553120in}}%
\pgfpathlineto{\pgfqpoint{3.414994in}{2.503044in}}%
\pgfpathlineto{\pgfqpoint{3.420976in}{2.540585in}}%
\pgfpathlineto{\pgfqpoint{3.423538in}{2.476887in}}%
\pgfpathlineto{\pgfqpoint{3.427816in}{2.569126in}}%
\pgfpathlineto{\pgfqpoint{3.434657in}{2.438227in}}%
\pgfpathlineto{\pgfqpoint{3.437223in}{2.529138in}}%
\pgfpathlineto{\pgfqpoint{3.442356in}{2.476735in}}%
\pgfpathlineto{\pgfqpoint{3.445779in}{2.570697in}}%
\pgfpathlineto{\pgfqpoint{3.449197in}{2.459277in}}%
\pgfpathlineto{\pgfqpoint{3.453471in}{2.546112in}}%
\pgfpathlineto{\pgfqpoint{3.458605in}{2.581389in}}%
\pgfpathlineto{\pgfqpoint{3.462882in}{2.402287in}}%
\pgfpathlineto{\pgfqpoint{3.467156in}{2.599271in}}%
\pgfpathlineto{\pgfqpoint{3.472285in}{2.457013in}}%
\pgfpathlineto{\pgfqpoint{3.475705in}{2.506064in}}%
\pgfpathlineto{\pgfqpoint{3.482550in}{2.439134in}}%
\pgfpathlineto{\pgfqpoint{3.485118in}{2.509688in}}%
\pgfpathlineto{\pgfqpoint{3.491104in}{2.444569in}}%
\pgfpathlineto{\pgfqpoint{3.495380in}{2.558978in}}%
\pgfpathlineto{\pgfqpoint{3.496237in}{2.416420in}}%
\pgfpathlineto{\pgfqpoint{3.501370in}{2.524971in}}%
\pgfpathlineto{\pgfqpoint{3.504790in}{2.422160in}}%
\pgfpathlineto{\pgfqpoint{3.512483in}{2.580121in}}%
\pgfpathlineto{\pgfqpoint{3.513340in}{2.479364in}}%
\pgfpathlineto{\pgfqpoint{3.519327in}{2.531797in}}%
\pgfpathlineto{\pgfqpoint{3.522750in}{2.450008in}}%
\pgfpathlineto{\pgfqpoint{3.529596in}{2.524699in}}%
\pgfpathlineto{\pgfqpoint{3.532161in}{2.415939in}}%
\pgfpathlineto{\pgfqpoint{3.534724in}{2.508299in}}%
\pgfpathlineto{\pgfqpoint{3.539002in}{2.417116in}}%
\pgfpathlineto{\pgfqpoint{3.543277in}{2.522161in}}%
\pgfpathlineto{\pgfqpoint{3.547556in}{2.433697in}}%
\pgfpathlineto{\pgfqpoint{3.555248in}{2.452784in}}%
\pgfpathlineto{\pgfqpoint{3.558668in}{2.548709in}}%
\pgfpathlineto{\pgfqpoint{3.565506in}{2.407721in}}%
\pgfpathlineto{\pgfqpoint{3.569782in}{2.538863in}}%
\pgfpathlineto{\pgfqpoint{3.574058in}{2.454145in}}%
\pgfpathlineto{\pgfqpoint{3.580899in}{2.554539in}}%
\pgfpathlineto{\pgfqpoint{3.583469in}{2.466980in}}%
\pgfpathlineto{\pgfqpoint{3.587741in}{2.497304in}}%
\pgfpathlineto{\pgfqpoint{3.591161in}{2.453570in}}%
\pgfpathlineto{\pgfqpoint{3.596294in}{2.572449in}}%
\pgfpathlineto{\pgfqpoint{3.599713in}{2.480421in}}%
\pgfpathlineto{\pgfqpoint{3.603993in}{2.539347in}}%
\pgfpathlineto{\pgfqpoint{3.608273in}{2.456590in}}%
\pgfpathlineto{\pgfqpoint{3.611692in}{2.569308in}}%
\pgfpathlineto{\pgfqpoint{3.619390in}{2.549918in}}%
\pgfpathlineto{\pgfqpoint{3.622816in}{2.439255in}}%
\pgfpathlineto{\pgfqpoint{3.625381in}{2.564657in}}%
\pgfpathlineto{\pgfqpoint{3.631368in}{2.495794in}}%
\pgfpathlineto{\pgfqpoint{3.635643in}{2.566832in}}%
\pgfpathlineto{\pgfqpoint{3.637355in}{2.491295in}}%
\pgfpathlineto{\pgfqpoint{3.643344in}{2.552394in}}%
\pgfpathlineto{\pgfqpoint{3.646770in}{2.435782in}}%
\pgfpathlineto{\pgfqpoint{3.651047in}{2.497728in}}%
\pgfpathlineto{\pgfqpoint{3.655322in}{2.448315in}}%
\pgfpathlineto{\pgfqpoint{3.658742in}{2.519322in}}%
\pgfpathlineto{\pgfqpoint{3.665581in}{2.462481in}}%
\pgfpathlineto{\pgfqpoint{3.667291in}{2.563026in}}%
\pgfpathlineto{\pgfqpoint{3.674993in}{2.462088in}}%
\pgfpathlineto{\pgfqpoint{3.677560in}{2.545901in}}%
\pgfpathlineto{\pgfqpoint{3.680976in}{2.430314in}}%
\pgfpathlineto{\pgfqpoint{3.686964in}{2.560309in}}%
\pgfpathlineto{\pgfqpoint{3.692094in}{2.437443in}}%
\pgfpathlineto{\pgfqpoint{3.692951in}{2.535240in}}%
\pgfpathlineto{\pgfqpoint{3.700650in}{2.561758in}}%
\pgfpathlineto{\pgfqpoint{3.701506in}{2.479667in}}%
\pgfpathlineto{\pgfqpoint{3.707493in}{2.440221in}}%
\pgfpathlineto{\pgfqpoint{3.710060in}{2.535301in}}%
\pgfpathlineto{\pgfqpoint{3.714341in}{2.529138in}}%
\pgfpathlineto{\pgfqpoint{3.721187in}{2.444992in}}%
\pgfpathlineto{\pgfqpoint{3.726321in}{2.539891in}}%
\pgfpathlineto{\pgfqpoint{3.727177in}{2.455262in}}%
\pgfpathlineto{\pgfqpoint{3.734016in}{2.463265in}}%
\pgfpathlineto{\pgfqpoint{3.735725in}{2.553964in}}%
\pgfpathlineto{\pgfqpoint{3.741708in}{2.601807in}}%
\pgfpathlineto{\pgfqpoint{3.745984in}{2.435267in}}%
\pgfpathlineto{\pgfqpoint{3.749406in}{2.520742in}}%
\pgfpathlineto{\pgfqpoint{3.756244in}{2.438408in}}%
\pgfpathlineto{\pgfqpoint{3.759666in}{2.578914in}}%
\pgfpathlineto{\pgfqpoint{3.762235in}{2.439948in}}%
\pgfpathlineto{\pgfqpoint{3.768221in}{2.521709in}}%
\pgfpathlineto{\pgfqpoint{3.771647in}{2.463265in}}%
\pgfpathlineto{\pgfqpoint{3.775066in}{2.555476in}}%
\pgfpathlineto{\pgfqpoint{3.778487in}{2.433969in}}%
\pgfpathlineto{\pgfqpoint{3.783618in}{2.415092in}}%
\pgfpathlineto{\pgfqpoint{3.787896in}{2.554055in}}%
\pgfpathlineto{\pgfqpoint{3.794743in}{2.510472in}}%
\pgfpathlineto{\pgfqpoint{3.799020in}{2.627782in}}%
\pgfpathlineto{\pgfqpoint{3.801584in}{2.512829in}}%
\pgfpathlineto{\pgfqpoint{3.805856in}{2.564839in}}%
\pgfpathlineto{\pgfqpoint{3.810133in}{2.473504in}}%
\pgfpathlineto{\pgfqpoint{3.814406in}{2.460457in}}%
\pgfpathlineto{\pgfqpoint{3.816973in}{2.611230in}}%
\pgfpathlineto{\pgfqpoint{3.821247in}{2.470030in}}%
\pgfpathlineto{\pgfqpoint{3.828940in}{2.564899in}}%
\pgfpathlineto{\pgfqpoint{3.830650in}{2.463900in}}%
\pgfpathlineto{\pgfqpoint{3.834922in}{2.556261in}}%
\pgfpathlineto{\pgfqpoint{3.839197in}{2.486914in}}%
\pgfpathlineto{\pgfqpoint{3.842619in}{2.588094in}}%
\pgfpathlineto{\pgfqpoint{3.848600in}{2.462027in}}%
\pgfpathlineto{\pgfqpoint{3.851170in}{2.560851in}}%
\pgfpathlineto{\pgfqpoint{3.856308in}{2.511107in}}%
\pgfpathlineto{\pgfqpoint{3.859732in}{2.569913in}}%
\pgfpathlineto{\pgfqpoint{3.864870in}{2.465470in}}%
\pgfpathlineto{\pgfqpoint{3.869999in}{2.542851in}}%
\pgfpathlineto{\pgfqpoint{3.875984in}{2.468127in}}%
\pgfpathlineto{\pgfqpoint{3.878551in}{2.545749in}}%
\pgfpathlineto{\pgfqpoint{3.882831in}{2.475558in}}%
\pgfpathlineto{\pgfqpoint{3.887107in}{2.455141in}}%
\pgfpathlineto{\pgfqpoint{3.890525in}{2.526902in}}%
\pgfpathlineto{\pgfqpoint{3.895656in}{2.486854in}}%
\pgfpathlineto{\pgfqpoint{3.899075in}{2.549253in}}%
\pgfpathlineto{\pgfqpoint{3.903348in}{2.491383in}}%
\pgfpathlineto{\pgfqpoint{3.908484in}{2.568824in}}%
\pgfpathlineto{\pgfqpoint{3.911052in}{2.441126in}}%
\pgfpathlineto{\pgfqpoint{3.917892in}{2.594134in}}%
\pgfpathlineto{\pgfqpoint{3.919603in}{2.514821in}}%
\pgfpathlineto{\pgfqpoint{3.925587in}{2.565683in}}%
\pgfpathlineto{\pgfqpoint{3.928153in}{2.522010in}}%
\pgfpathlineto{\pgfqpoint{3.934142in}{2.484315in}}%
\pgfpathlineto{\pgfqpoint{3.936707in}{2.602410in}}%
\pgfpathlineto{\pgfqpoint{3.943555in}{2.438860in}}%
\pgfpathlineto{\pgfqpoint{3.945268in}{2.556561in}}%
\pgfpathlineto{\pgfqpoint{3.949543in}{2.432911in}}%
\pgfpathlineto{\pgfqpoint{3.954676in}{2.439858in}}%
\pgfpathlineto{\pgfqpoint{3.961520in}{2.597698in}}%
\pgfpathlineto{\pgfqpoint{3.962375in}{2.493438in}}%
\pgfpathlineto{\pgfqpoint{3.970067in}{2.451455in}}%
\pgfpathlineto{\pgfqpoint{3.970924in}{2.575952in}}%
\pgfpathlineto{\pgfqpoint{3.976909in}{2.459007in}}%
\pgfpathlineto{\pgfqpoint{3.979475in}{2.533849in}}%
\pgfpathlineto{\pgfqpoint{3.984613in}{2.488363in}}%
\pgfpathlineto{\pgfqpoint{3.991456in}{2.464805in}}%
\pgfpathlineto{\pgfqpoint{3.992312in}{2.549372in}}%
\pgfpathlineto{\pgfqpoint{3.997445in}{2.464684in}}%
\pgfpathlineto{\pgfqpoint{4.000867in}{2.543816in}}%
\pgfpathlineto{\pgfqpoint{4.006003in}{2.477792in}}%
\pgfpathlineto{\pgfqpoint{4.011132in}{2.537776in}}%
\pgfpathlineto{\pgfqpoint{4.015412in}{2.440039in}}%
\pgfpathlineto{\pgfqpoint{4.018834in}{2.584772in}}%
\pgfpathlineto{\pgfqpoint{4.022252in}{2.480814in}}%
\pgfpathlineto{\pgfqpoint{4.029095in}{2.456953in}}%
\pgfpathlineto{\pgfqpoint{4.032514in}{2.549556in}}%
\pgfpathlineto{\pgfqpoint{4.035079in}{2.497486in}}%
\pgfpathlineto{\pgfqpoint{4.039357in}{2.552757in}}%
\pgfpathlineto{\pgfqpoint{4.046202in}{2.497365in}}%
\pgfpathlineto{\pgfqpoint{4.048767in}{2.579819in}}%
\pgfpathlineto{\pgfqpoint{4.054759in}{2.630681in}}%
\pgfpathlineto{\pgfqpoint{4.056470in}{2.518599in}}%
\pgfpathlineto{\pgfqpoint{4.063314in}{2.564687in}}%
\pgfpathlineto{\pgfqpoint{4.068446in}{2.392620in}}%
\pgfpathlineto{\pgfqpoint{4.070156in}{2.547138in}}%
\pgfpathlineto{\pgfqpoint{4.073578in}{2.460517in}}%
\pgfpathlineto{\pgfqpoint{4.079562in}{2.426085in}}%
\pgfpathlineto{\pgfqpoint{4.082985in}{2.543695in}}%
\pgfpathlineto{\pgfqpoint{4.093250in}{2.436052in}}%
\pgfpathlineto{\pgfqpoint{4.097525in}{2.569731in}}%
\pgfpathlineto{\pgfqpoint{4.102659in}{2.582115in}}%
\pgfpathlineto{\pgfqpoint{4.103512in}{2.442335in}}%
\pgfpathlineto{\pgfqpoint{4.108646in}{2.571785in}}%
\pgfpathlineto{\pgfqpoint{4.113785in}{2.439616in}}%
\pgfpathlineto{\pgfqpoint{4.119765in}{2.578007in}}%
\pgfpathlineto{\pgfqpoint{4.121474in}{2.520318in}}%
\pgfpathlineto{\pgfqpoint{4.128317in}{2.423970in}}%
\pgfpathlineto{\pgfqpoint{4.130028in}{2.476100in}}%
\pgfpathlineto{\pgfqpoint{4.136878in}{2.442999in}}%
\pgfpathlineto{\pgfqpoint{4.137733in}{2.517117in}}%
\pgfpathlineto{\pgfqpoint{4.142868in}{2.457919in}}%
\pgfpathlineto{\pgfqpoint{4.147142in}{2.471994in}}%
\pgfpathlineto{\pgfqpoint{4.149711in}{2.526600in}}%
\pgfpathlineto{\pgfqpoint{4.152280in}{2.462932in}}%
\pgfpathlineto{\pgfqpoint{4.158268in}{2.465892in}}%
\pgfpathlineto{\pgfqpoint{4.159124in}{2.569911in}}%
\pgfpathlineto{\pgfqpoint{4.162548in}{2.479846in}}%
\pgfpathlineto{\pgfqpoint{4.166829in}{2.477159in}}%
\pgfpathlineto{\pgfqpoint{4.170249in}{2.571483in}}%
\pgfpathlineto{\pgfqpoint{4.172814in}{2.495008in}}%
\pgfpathlineto{\pgfqpoint{4.177091in}{2.569610in}}%
\pgfpathlineto{\pgfqpoint{4.181364in}{2.490478in}}%
\pgfpathlineto{\pgfqpoint{4.184787in}{2.564718in}}%
\pgfpathlineto{\pgfqpoint{4.186496in}{2.471692in}}%
\pgfpathlineto{\pgfqpoint{4.191629in}{2.488847in}}%
\pgfpathlineto{\pgfqpoint{4.194194in}{2.563268in}}%
\pgfpathlineto{\pgfqpoint{4.196761in}{2.500626in}}%
\pgfpathlineto{\pgfqpoint{4.201034in}{2.584530in}}%
\pgfpathlineto{\pgfqpoint{4.204456in}{2.480118in}}%
\pgfpathlineto{\pgfqpoint{4.207878in}{2.577583in}}%
\pgfpathlineto{\pgfqpoint{4.210442in}{2.489271in}}%
\pgfpathlineto{\pgfqpoint{4.216431in}{2.421434in}}%
\pgfpathlineto{\pgfqpoint{4.218141in}{2.512708in}}%
\pgfpathlineto{\pgfqpoint{4.223272in}{2.484015in}}%
\pgfpathlineto{\pgfqpoint{4.224983in}{2.558678in}}%
\pgfpathlineto{\pgfqpoint{4.230116in}{2.475467in}}%
\pgfpathlineto{\pgfqpoint{4.230972in}{2.518748in}}%
\pgfpathlineto{\pgfqpoint{4.235248in}{2.489543in}}%
\pgfpathlineto{\pgfqpoint{4.238671in}{2.453028in}}%
\pgfpathlineto{\pgfqpoint{4.242092in}{2.528082in}}%
\pgfpathlineto{\pgfqpoint{4.246370in}{2.436898in}}%
\pgfpathlineto{\pgfqpoint{4.248080in}{2.504130in}}%
\pgfpathlineto{\pgfqpoint{4.254070in}{2.441700in}}%
\pgfpathlineto{\pgfqpoint{4.257490in}{2.516454in}}%
\pgfpathlineto{\pgfqpoint{4.259202in}{2.444871in}}%
\pgfpathlineto{\pgfqpoint{4.263479in}{2.465107in}}%
\pgfpathlineto{\pgfqpoint{4.266903in}{2.586040in}}%
\pgfpathlineto{\pgfqpoint{4.270325in}{2.460638in}}%
\pgfpathlineto{\pgfqpoint{4.273748in}{2.549072in}}%
\pgfpathlineto{\pgfqpoint{4.277169in}{2.475407in}}%
\pgfpathlineto{\pgfqpoint{4.278880in}{2.544421in}}%
\pgfpathlineto{\pgfqpoint{4.288282in}{2.462148in}}%
\pgfpathlineto{\pgfqpoint{4.289137in}{2.549556in}}%
\pgfpathlineto{\pgfqpoint{4.292562in}{2.605371in}}%
\pgfpathlineto{\pgfqpoint{4.295981in}{2.454899in}}%
\pgfpathlineto{\pgfqpoint{4.300259in}{2.535059in}}%
\pgfpathlineto{\pgfqpoint{4.304529in}{2.576981in}}%
\pgfpathlineto{\pgfqpoint{4.307096in}{2.496941in}}%
\pgfpathlineto{\pgfqpoint{4.311374in}{2.592203in}}%
\pgfpathlineto{\pgfqpoint{4.315651in}{2.502015in}}%
\pgfpathlineto{\pgfqpoint{4.318213in}{2.593953in}}%
\pgfpathlineto{\pgfqpoint{4.320779in}{2.505277in}}%
\pgfpathlineto{\pgfqpoint{4.324195in}{2.559039in}}%
\pgfpathlineto{\pgfqpoint{4.326759in}{2.478245in}}%
\pgfpathlineto{\pgfqpoint{4.330182in}{2.524608in}}%
\pgfpathlineto{\pgfqpoint{4.336171in}{2.486974in}}%
\pgfpathlineto{\pgfqpoint{4.338734in}{2.478878in}}%
\pgfpathlineto{\pgfqpoint{4.340445in}{2.564234in}}%
\pgfpathlineto{\pgfqpoint{4.346437in}{2.456772in}}%
\pgfpathlineto{\pgfqpoint{4.347293in}{2.526539in}}%
\pgfpathlineto{\pgfqpoint{4.353279in}{2.442999in}}%
\pgfpathlineto{\pgfqpoint{4.356698in}{2.571723in}}%
\pgfpathlineto{\pgfqpoint{4.357553in}{2.445444in}}%
\pgfpathlineto{\pgfqpoint{4.360978in}{2.529199in}}%
\pgfpathlineto{\pgfqpoint{4.366112in}{2.464684in}}%
\pgfpathlineto{\pgfqpoint{4.367823in}{2.549253in}}%
\pgfpathlineto{\pgfqpoint{4.371246in}{2.462993in}}%
\pgfpathlineto{\pgfqpoint{4.377235in}{2.553843in}}%
\pgfpathlineto{\pgfqpoint{4.378945in}{2.463537in}}%
\pgfpathlineto{\pgfqpoint{4.381508in}{2.556200in}}%
\pgfpathlineto{\pgfqpoint{4.385782in}{2.410741in}}%
\pgfpathlineto{\pgfqpoint{4.389205in}{2.602531in}}%
\pgfpathlineto{\pgfqpoint{4.392627in}{2.463507in}}%
\pgfpathlineto{\pgfqpoint{4.395194in}{2.533910in}}%
\pgfpathlineto{\pgfqpoint{4.399466in}{2.471117in}}%
\pgfpathlineto{\pgfqpoint{4.402888in}{2.523943in}}%
\pgfpathlineto{\pgfqpoint{4.407166in}{2.444509in}}%
\pgfpathlineto{\pgfqpoint{4.408878in}{2.512073in}}%
\pgfpathlineto{\pgfqpoint{4.414012in}{2.412644in}}%
\pgfpathlineto{\pgfqpoint{4.415722in}{2.495552in}}%
\pgfpathlineto{\pgfqpoint{4.419998in}{2.457074in}}%
\pgfpathlineto{\pgfqpoint{4.424273in}{2.573235in}}%
\pgfpathlineto{\pgfqpoint{4.425984in}{2.483652in}}%
\pgfpathlineto{\pgfqpoint{4.431970in}{2.490660in}}%
\pgfpathlineto{\pgfqpoint{4.435391in}{2.521105in}}%
\pgfpathlineto{\pgfqpoint{4.437960in}{2.470424in}}%
\pgfpathlineto{\pgfqpoint{4.439672in}{2.557347in}}%
\pgfpathlineto{\pgfqpoint{4.443953in}{2.582718in}}%
\pgfpathlineto{\pgfqpoint{4.447371in}{2.470605in}}%
\pgfpathlineto{\pgfqpoint{4.451649in}{2.531674in}}%
\pgfpathlineto{\pgfqpoint{4.455068in}{2.456227in}}%
\pgfpathlineto{\pgfqpoint{4.456777in}{2.520137in}}%
\pgfpathlineto{\pgfqpoint{4.461912in}{2.558978in}}%
\pgfpathlineto{\pgfqpoint{4.464478in}{2.502739in}}%
\pgfpathlineto{\pgfqpoint{4.469610in}{2.560851in}}%
\pgfpathlineto{\pgfqpoint{4.470466in}{2.476947in}}%
\pgfpathlineto{\pgfqpoint{4.473885in}{2.437743in}}%
\pgfpathlineto{\pgfqpoint{4.478161in}{2.547501in}}%
\pgfpathlineto{\pgfqpoint{4.482436in}{2.496639in}}%
\pgfpathlineto{\pgfqpoint{4.486712in}{2.530739in}}%
\pgfpathlineto{\pgfqpoint{4.490131in}{2.449522in}}%
\pgfpathlineto{\pgfqpoint{4.490987in}{2.556140in}}%
\pgfpathlineto{\pgfqpoint{4.494410in}{2.466496in}}%
\pgfpathlineto{\pgfqpoint{4.498682in}{2.553874in}}%
\pgfpathlineto{\pgfqpoint{4.501244in}{2.479483in}}%
\pgfpathlineto{\pgfqpoint{4.504663in}{2.543695in}}%
\pgfpathlineto{\pgfqpoint{4.508083in}{2.496306in}}%
\pgfpathlineto{\pgfqpoint{4.511500in}{2.557166in}}%
\pgfpathlineto{\pgfqpoint{4.514921in}{2.455231in}}%
\pgfpathlineto{\pgfqpoint{4.520052in}{2.524427in}}%
\pgfpathlineto{\pgfqpoint{4.524328in}{2.461725in}}%
\pgfpathlineto{\pgfqpoint{4.525183in}{2.554569in}}%
\pgfpathlineto{\pgfqpoint{4.531176in}{2.475377in}}%
\pgfpathlineto{\pgfqpoint{4.532887in}{2.546050in}}%
\pgfpathlineto{\pgfqpoint{4.535450in}{2.443120in}}%
\pgfpathlineto{\pgfqpoint{4.538868in}{2.737601in}}%
\pgfpathlineto{\pgfqpoint{4.542287in}{2.840776in}}%
\pgfpathlineto{\pgfqpoint{4.545713in}{2.776291in}}%
\pgfpathlineto{\pgfqpoint{4.549136in}{2.797947in}}%
\pgfpathlineto{\pgfqpoint{4.552557in}{2.735487in}}%
\pgfpathlineto{\pgfqpoint{4.557688in}{2.817338in}}%
\pgfpathlineto{\pgfqpoint{4.560254in}{2.727032in}}%
\pgfpathlineto{\pgfqpoint{4.564531in}{2.776775in}}%
\pgfpathlineto{\pgfqpoint{4.566243in}{2.692539in}}%
\pgfpathlineto{\pgfqpoint{4.569658in}{2.777892in}}%
\pgfpathlineto{\pgfqpoint{4.573939in}{2.714134in}}%
\pgfpathlineto{\pgfqpoint{4.577361in}{2.795350in}}%
\pgfpathlineto{\pgfqpoint{4.581640in}{2.666745in}}%
\pgfpathlineto{\pgfqpoint{4.583350in}{2.830053in}}%
\pgfpathlineto{\pgfqpoint{4.588482in}{2.836004in}}%
\pgfpathlineto{\pgfqpoint{4.590190in}{2.753731in}}%
\pgfpathlineto{\pgfqpoint{4.593612in}{2.883000in}}%
\pgfpathlineto{\pgfqpoint{4.597035in}{2.766718in}}%
\pgfpathlineto{\pgfqpoint{4.600455in}{2.817580in}}%
\pgfpathlineto{\pgfqpoint{4.605589in}{2.788042in}}%
\pgfpathlineto{\pgfqpoint{4.608155in}{2.491688in}}%
\pgfpathlineto{\pgfqpoint{4.612433in}{2.543032in}}%
\pgfpathlineto{\pgfqpoint{4.616710in}{2.440039in}}%
\pgfpathlineto{\pgfqpoint{4.617565in}{2.494284in}}%
\pgfpathlineto{\pgfqpoint{4.621838in}{2.524427in}}%
\pgfpathlineto{\pgfqpoint{4.624405in}{2.574684in}}%
\pgfpathlineto{\pgfqpoint{4.627827in}{2.454175in}}%
\pgfpathlineto{\pgfqpoint{4.632102in}{2.558133in}}%
\pgfpathlineto{\pgfqpoint{4.635525in}{2.447349in}}%
\pgfpathlineto{\pgfqpoint{4.638091in}{2.543032in}}%
\pgfpathlineto{\pgfqpoint{4.642368in}{2.500385in}}%
\pgfpathlineto{\pgfqpoint{4.645787in}{2.605432in}}%
\pgfpathlineto{\pgfqpoint{4.649207in}{2.515486in}}%
\pgfpathlineto{\pgfqpoint{4.652631in}{2.541280in}}%
\pgfpathlineto{\pgfqpoint{4.656051in}{2.473837in}}%
\pgfpathlineto{\pgfqpoint{4.661182in}{2.528231in}}%
\pgfpathlineto{\pgfqpoint{4.662894in}{2.462539in}}%
\pgfpathlineto{\pgfqpoint{4.666317in}{2.505217in}}%
\pgfpathlineto{\pgfqpoint{4.668881in}{2.453750in}}%
\pgfpathlineto{\pgfqpoint{4.674015in}{2.474893in}}%
\pgfpathlineto{\pgfqpoint{4.677437in}{2.553450in}}%
\pgfpathlineto{\pgfqpoint{4.679148in}{2.498996in}}%
\pgfpathlineto{\pgfqpoint{4.682567in}{2.439555in}}%
\pgfpathlineto{\pgfqpoint{4.687701in}{2.542367in}}%
\pgfpathlineto{\pgfqpoint{4.689413in}{2.551610in}}%
\pgfpathlineto{\pgfqpoint{4.693685in}{2.478669in}}%
\pgfpathlineto{\pgfqpoint{4.697112in}{2.528533in}}%
\pgfpathlineto{\pgfqpoint{4.700535in}{2.432399in}}%
\pgfpathlineto{\pgfqpoint{4.704809in}{2.540917in}}%
\pgfpathlineto{\pgfqpoint{4.706522in}{2.485102in}}%
\pgfpathlineto{\pgfqpoint{4.711657in}{2.554085in}}%
\pgfpathlineto{\pgfqpoint{4.715932in}{2.507090in}}%
\pgfpathlineto{\pgfqpoint{4.718499in}{2.584530in}}%
\pgfpathlineto{\pgfqpoint{4.721922in}{2.497213in}}%
\pgfpathlineto{\pgfqpoint{4.724492in}{2.554448in}}%
\pgfpathlineto{\pgfqpoint{4.727918in}{2.471873in}}%
\pgfpathlineto{\pgfqpoint{4.730484in}{2.520923in}}%
\pgfpathlineto{\pgfqpoint{4.736479in}{2.470000in}}%
\pgfpathlineto{\pgfqpoint{4.737335in}{2.574563in}}%
\pgfpathlineto{\pgfqpoint{4.742472in}{2.561214in}}%
\pgfpathlineto{\pgfqpoint{4.745895in}{2.440160in}}%
\pgfpathlineto{\pgfqpoint{4.747605in}{2.522161in}}%
\pgfpathlineto{\pgfqpoint{4.751882in}{2.609659in}}%
\pgfpathlineto{\pgfqpoint{4.755301in}{2.482384in}}%
\pgfpathlineto{\pgfqpoint{4.759573in}{2.538170in}}%
\pgfpathlineto{\pgfqpoint{4.763849in}{2.486372in}}%
\pgfpathlineto{\pgfqpoint{4.766411in}{2.593652in}}%
\pgfpathlineto{\pgfqpoint{4.769831in}{2.471752in}}%
\pgfpathlineto{\pgfqpoint{4.773255in}{2.532581in}}%
\pgfpathlineto{\pgfqpoint{4.776680in}{2.544844in}}%
\pgfpathlineto{\pgfqpoint{4.778393in}{2.447319in}}%
\pgfpathlineto{\pgfqpoint{4.782675in}{2.502015in}}%
\pgfpathlineto{\pgfqpoint{4.787806in}{2.465561in}}%
\pgfpathlineto{\pgfqpoint{4.791230in}{2.500929in}}%
\pgfpathlineto{\pgfqpoint{4.794650in}{2.482293in}}%
\pgfpathlineto{\pgfqpoint{4.795506in}{2.527991in}}%
\pgfpathlineto{\pgfqpoint{4.795506in}{2.527991in}}%
\pgfusepath{stroke}%
\end{pgfscope}%
\begin{pgfscope}%
\pgfsetrectcap%
\pgfsetmiterjoin%
\pgfsetlinewidth{0.803000pt}%
\definecolor{currentstroke}{rgb}{0.000000,0.000000,0.000000}%
\pgfsetstrokecolor{currentstroke}%
\pgfsetdash{}{0pt}%
\pgfpathmoveto{\pgfqpoint{0.484581in}{2.334497in}}%
\pgfpathlineto{\pgfqpoint{0.484581in}{2.909119in}}%
\pgfusepath{stroke}%
\end{pgfscope}%
\begin{pgfscope}%
\pgfsetrectcap%
\pgfsetmiterjoin%
\pgfsetlinewidth{0.803000pt}%
\definecolor{currentstroke}{rgb}{0.000000,0.000000,0.000000}%
\pgfsetstrokecolor{currentstroke}%
\pgfsetdash{}{0pt}%
\pgfpathmoveto{\pgfqpoint{5.000788in}{2.334497in}}%
\pgfpathlineto{\pgfqpoint{5.000788in}{2.909119in}}%
\pgfusepath{stroke}%
\end{pgfscope}%
\begin{pgfscope}%
\pgfsetrectcap%
\pgfsetmiterjoin%
\pgfsetlinewidth{0.803000pt}%
\definecolor{currentstroke}{rgb}{0.000000,0.000000,0.000000}%
\pgfsetstrokecolor{currentstroke}%
\pgfsetdash{}{0pt}%
\pgfpathmoveto{\pgfqpoint{0.484581in}{2.334497in}}%
\pgfpathlineto{\pgfqpoint{5.000788in}{2.334497in}}%
\pgfusepath{stroke}%
\end{pgfscope}%
\begin{pgfscope}%
\pgfsetrectcap%
\pgfsetmiterjoin%
\pgfsetlinewidth{0.803000pt}%
\definecolor{currentstroke}{rgb}{0.000000,0.000000,0.000000}%
\pgfsetstrokecolor{currentstroke}%
\pgfsetdash{}{0pt}%
\pgfpathmoveto{\pgfqpoint{0.484581in}{2.909119in}}%
\pgfpathlineto{\pgfqpoint{5.000788in}{2.909119in}}%
\pgfusepath{stroke}%
\end{pgfscope}%
\begin{pgfscope}%
\pgfsetbuttcap%
\pgfsetmiterjoin%
\definecolor{currentfill}{rgb}{1.000000,1.000000,1.000000}%
\pgfsetfillcolor{currentfill}%
\pgfsetlinewidth{0.000000pt}%
\definecolor{currentstroke}{rgb}{0.000000,0.000000,0.000000}%
\pgfsetstrokecolor{currentstroke}%
\pgfsetstrokeopacity{0.000000}%
\pgfsetdash{}{0pt}%
\pgfpathmoveto{\pgfqpoint{0.484581in}{1.437021in}}%
\pgfpathlineto{\pgfqpoint{5.000788in}{1.437021in}}%
\pgfpathlineto{\pgfqpoint{5.000788in}{2.011643in}}%
\pgfpathlineto{\pgfqpoint{0.484581in}{2.011643in}}%
\pgfpathlineto{\pgfqpoint{0.484581in}{1.437021in}}%
\pgfpathclose%
\pgfusepath{fill}%
\end{pgfscope}%
\begin{pgfscope}%
\pgfsetbuttcap%
\pgfsetroundjoin%
\definecolor{currentfill}{rgb}{0.000000,0.000000,0.000000}%
\pgfsetfillcolor{currentfill}%
\pgfsetlinewidth{0.803000pt}%
\definecolor{currentstroke}{rgb}{0.000000,0.000000,0.000000}%
\pgfsetstrokecolor{currentstroke}%
\pgfsetdash{}{0pt}%
\pgfsys@defobject{currentmarker}{\pgfqpoint{0.000000in}{-0.048611in}}{\pgfqpoint{0.000000in}{0.000000in}}{%
\pgfpathmoveto{\pgfqpoint{0.000000in}{0.000000in}}%
\pgfpathlineto{\pgfqpoint{0.000000in}{-0.048611in}}%
\pgfusepath{stroke,fill}%
}%
\begin{pgfscope}%
\pgfsys@transformshift{0.689546in}{1.437021in}%
\pgfsys@useobject{currentmarker}{}%
\end{pgfscope}%
\end{pgfscope}%
\begin{pgfscope}%
\pgfsetbuttcap%
\pgfsetroundjoin%
\definecolor{currentfill}{rgb}{0.000000,0.000000,0.000000}%
\pgfsetfillcolor{currentfill}%
\pgfsetlinewidth{0.803000pt}%
\definecolor{currentstroke}{rgb}{0.000000,0.000000,0.000000}%
\pgfsetstrokecolor{currentstroke}%
\pgfsetdash{}{0pt}%
\pgfsys@defobject{currentmarker}{\pgfqpoint{0.000000in}{-0.048611in}}{\pgfqpoint{0.000000in}{0.000000in}}{%
\pgfpathmoveto{\pgfqpoint{0.000000in}{0.000000in}}%
\pgfpathlineto{\pgfqpoint{0.000000in}{-0.048611in}}%
\pgfusepath{stroke,fill}%
}%
\begin{pgfscope}%
\pgfsys@transformshift{1.202878in}{1.437021in}%
\pgfsys@useobject{currentmarker}{}%
\end{pgfscope}%
\end{pgfscope}%
\begin{pgfscope}%
\pgfsetbuttcap%
\pgfsetroundjoin%
\definecolor{currentfill}{rgb}{0.000000,0.000000,0.000000}%
\pgfsetfillcolor{currentfill}%
\pgfsetlinewidth{0.803000pt}%
\definecolor{currentstroke}{rgb}{0.000000,0.000000,0.000000}%
\pgfsetstrokecolor{currentstroke}%
\pgfsetdash{}{0pt}%
\pgfsys@defobject{currentmarker}{\pgfqpoint{0.000000in}{-0.048611in}}{\pgfqpoint{0.000000in}{0.000000in}}{%
\pgfpathmoveto{\pgfqpoint{0.000000in}{0.000000in}}%
\pgfpathlineto{\pgfqpoint{0.000000in}{-0.048611in}}%
\pgfusepath{stroke,fill}%
}%
\begin{pgfscope}%
\pgfsys@transformshift{1.716211in}{1.437021in}%
\pgfsys@useobject{currentmarker}{}%
\end{pgfscope}%
\end{pgfscope}%
\begin{pgfscope}%
\pgfsetbuttcap%
\pgfsetroundjoin%
\definecolor{currentfill}{rgb}{0.000000,0.000000,0.000000}%
\pgfsetfillcolor{currentfill}%
\pgfsetlinewidth{0.803000pt}%
\definecolor{currentstroke}{rgb}{0.000000,0.000000,0.000000}%
\pgfsetstrokecolor{currentstroke}%
\pgfsetdash{}{0pt}%
\pgfsys@defobject{currentmarker}{\pgfqpoint{0.000000in}{-0.048611in}}{\pgfqpoint{0.000000in}{0.000000in}}{%
\pgfpathmoveto{\pgfqpoint{0.000000in}{0.000000in}}%
\pgfpathlineto{\pgfqpoint{0.000000in}{-0.048611in}}%
\pgfusepath{stroke,fill}%
}%
\begin{pgfscope}%
\pgfsys@transformshift{2.229543in}{1.437021in}%
\pgfsys@useobject{currentmarker}{}%
\end{pgfscope}%
\end{pgfscope}%
\begin{pgfscope}%
\pgfsetbuttcap%
\pgfsetroundjoin%
\definecolor{currentfill}{rgb}{0.000000,0.000000,0.000000}%
\pgfsetfillcolor{currentfill}%
\pgfsetlinewidth{0.803000pt}%
\definecolor{currentstroke}{rgb}{0.000000,0.000000,0.000000}%
\pgfsetstrokecolor{currentstroke}%
\pgfsetdash{}{0pt}%
\pgfsys@defobject{currentmarker}{\pgfqpoint{0.000000in}{-0.048611in}}{\pgfqpoint{0.000000in}{0.000000in}}{%
\pgfpathmoveto{\pgfqpoint{0.000000in}{0.000000in}}%
\pgfpathlineto{\pgfqpoint{0.000000in}{-0.048611in}}%
\pgfusepath{stroke,fill}%
}%
\begin{pgfscope}%
\pgfsys@transformshift{2.742876in}{1.437021in}%
\pgfsys@useobject{currentmarker}{}%
\end{pgfscope}%
\end{pgfscope}%
\begin{pgfscope}%
\pgfsetbuttcap%
\pgfsetroundjoin%
\definecolor{currentfill}{rgb}{0.000000,0.000000,0.000000}%
\pgfsetfillcolor{currentfill}%
\pgfsetlinewidth{0.803000pt}%
\definecolor{currentstroke}{rgb}{0.000000,0.000000,0.000000}%
\pgfsetstrokecolor{currentstroke}%
\pgfsetdash{}{0pt}%
\pgfsys@defobject{currentmarker}{\pgfqpoint{0.000000in}{-0.048611in}}{\pgfqpoint{0.000000in}{0.000000in}}{%
\pgfpathmoveto{\pgfqpoint{0.000000in}{0.000000in}}%
\pgfpathlineto{\pgfqpoint{0.000000in}{-0.048611in}}%
\pgfusepath{stroke,fill}%
}%
\begin{pgfscope}%
\pgfsys@transformshift{3.256208in}{1.437021in}%
\pgfsys@useobject{currentmarker}{}%
\end{pgfscope}%
\end{pgfscope}%
\begin{pgfscope}%
\pgfsetbuttcap%
\pgfsetroundjoin%
\definecolor{currentfill}{rgb}{0.000000,0.000000,0.000000}%
\pgfsetfillcolor{currentfill}%
\pgfsetlinewidth{0.803000pt}%
\definecolor{currentstroke}{rgb}{0.000000,0.000000,0.000000}%
\pgfsetstrokecolor{currentstroke}%
\pgfsetdash{}{0pt}%
\pgfsys@defobject{currentmarker}{\pgfqpoint{0.000000in}{-0.048611in}}{\pgfqpoint{0.000000in}{0.000000in}}{%
\pgfpathmoveto{\pgfqpoint{0.000000in}{0.000000in}}%
\pgfpathlineto{\pgfqpoint{0.000000in}{-0.048611in}}%
\pgfusepath{stroke,fill}%
}%
\begin{pgfscope}%
\pgfsys@transformshift{3.769541in}{1.437021in}%
\pgfsys@useobject{currentmarker}{}%
\end{pgfscope}%
\end{pgfscope}%
\begin{pgfscope}%
\pgfsetbuttcap%
\pgfsetroundjoin%
\definecolor{currentfill}{rgb}{0.000000,0.000000,0.000000}%
\pgfsetfillcolor{currentfill}%
\pgfsetlinewidth{0.803000pt}%
\definecolor{currentstroke}{rgb}{0.000000,0.000000,0.000000}%
\pgfsetstrokecolor{currentstroke}%
\pgfsetdash{}{0pt}%
\pgfsys@defobject{currentmarker}{\pgfqpoint{0.000000in}{-0.048611in}}{\pgfqpoint{0.000000in}{0.000000in}}{%
\pgfpathmoveto{\pgfqpoint{0.000000in}{0.000000in}}%
\pgfpathlineto{\pgfqpoint{0.000000in}{-0.048611in}}%
\pgfusepath{stroke,fill}%
}%
\begin{pgfscope}%
\pgfsys@transformshift{4.282873in}{1.437021in}%
\pgfsys@useobject{currentmarker}{}%
\end{pgfscope}%
\end{pgfscope}%
\begin{pgfscope}%
\pgfsetbuttcap%
\pgfsetroundjoin%
\definecolor{currentfill}{rgb}{0.000000,0.000000,0.000000}%
\pgfsetfillcolor{currentfill}%
\pgfsetlinewidth{0.803000pt}%
\definecolor{currentstroke}{rgb}{0.000000,0.000000,0.000000}%
\pgfsetstrokecolor{currentstroke}%
\pgfsetdash{}{0pt}%
\pgfsys@defobject{currentmarker}{\pgfqpoint{0.000000in}{-0.048611in}}{\pgfqpoint{0.000000in}{0.000000in}}{%
\pgfpathmoveto{\pgfqpoint{0.000000in}{0.000000in}}%
\pgfpathlineto{\pgfqpoint{0.000000in}{-0.048611in}}%
\pgfusepath{stroke,fill}%
}%
\begin{pgfscope}%
\pgfsys@transformshift{4.796206in}{1.437021in}%
\pgfsys@useobject{currentmarker}{}%
\end{pgfscope}%
\end{pgfscope}%
\begin{pgfscope}%
\pgfsetbuttcap%
\pgfsetroundjoin%
\definecolor{currentfill}{rgb}{0.000000,0.000000,0.000000}%
\pgfsetfillcolor{currentfill}%
\pgfsetlinewidth{0.803000pt}%
\definecolor{currentstroke}{rgb}{0.000000,0.000000,0.000000}%
\pgfsetstrokecolor{currentstroke}%
\pgfsetdash{}{0pt}%
\pgfsys@defobject{currentmarker}{\pgfqpoint{-0.048611in}{0.000000in}}{\pgfqpoint{-0.000000in}{0.000000in}}{%
\pgfpathmoveto{\pgfqpoint{-0.000000in}{0.000000in}}%
\pgfpathlineto{\pgfqpoint{-0.048611in}{0.000000in}}%
\pgfusepath{stroke,fill}%
}%
\begin{pgfscope}%
\pgfsys@transformshift{0.484581in}{1.615020in}%
\pgfsys@useobject{currentmarker}{}%
\end{pgfscope}%
\end{pgfscope}%
\begin{pgfscope}%
\definecolor{textcolor}{rgb}{0.000000,0.000000,0.000000}%
\pgfsetstrokecolor{textcolor}%
\pgfsetfillcolor{textcolor}%
\pgftext[x=0.328331in, y=1.576464in, left, base]{\color{textcolor}\rmfamily\fontsize{8.000000}{9.600000}\selectfont \(\displaystyle {0}\)}%
\end{pgfscope}%
\begin{pgfscope}%
\pgfsetbuttcap%
\pgfsetroundjoin%
\definecolor{currentfill}{rgb}{0.000000,0.000000,0.000000}%
\pgfsetfillcolor{currentfill}%
\pgfsetlinewidth{0.803000pt}%
\definecolor{currentstroke}{rgb}{0.000000,0.000000,0.000000}%
\pgfsetstrokecolor{currentstroke}%
\pgfsetdash{}{0pt}%
\pgfsys@defobject{currentmarker}{\pgfqpoint{-0.048611in}{0.000000in}}{\pgfqpoint{-0.000000in}{0.000000in}}{%
\pgfpathmoveto{\pgfqpoint{-0.000000in}{0.000000in}}%
\pgfpathlineto{\pgfqpoint{-0.048611in}{0.000000in}}%
\pgfusepath{stroke,fill}%
}%
\begin{pgfscope}%
\pgfsys@transformshift{0.484581in}{1.818561in}%
\pgfsys@useobject{currentmarker}{}%
\end{pgfscope}%
\end{pgfscope}%
\begin{pgfscope}%
\definecolor{textcolor}{rgb}{0.000000,0.000000,0.000000}%
\pgfsetstrokecolor{textcolor}%
\pgfsetfillcolor{textcolor}%
\pgftext[x=0.328331in, y=1.780005in, left, base]{\color{textcolor}\rmfamily\fontsize{8.000000}{9.600000}\selectfont \(\displaystyle {5}\)}%
\end{pgfscope}%
\begin{pgfscope}%
\definecolor{textcolor}{rgb}{0.000000,0.000000,0.000000}%
\pgfsetstrokecolor{textcolor}%
\pgfsetfillcolor{textcolor}%
\pgftext[x=0.272775in,y=1.724332in,,bottom,rotate=90.000000]{\color{textcolor}\rmfamily\fontsize{10.000000}{12.000000}\selectfont Voltage deviation in \unit{\V}}%
\end{pgfscope}%
\begin{pgfscope}%
\definecolor{textcolor}{rgb}{0.000000,0.000000,0.000000}%
\pgfsetstrokecolor{textcolor}%
\pgfsetfillcolor{textcolor}%
\pgftext[x=0.484581in,y=2.053309in,left,base]{\color{textcolor}\rmfamily\fontsize{8.000000}{9.600000}\selectfont \(\displaystyle \times{10^{\ensuremath{-}6}}{}\)}%
\end{pgfscope}%
\begin{pgfscope}%
\pgfpathrectangle{\pgfqpoint{0.484581in}{1.437021in}}{\pgfqpoint{4.516206in}{0.574622in}}%
\pgfusepath{clip}%
\pgfsetrectcap%
\pgfsetroundjoin%
\pgfsetlinewidth{0.501875pt}%
\definecolor{currentstroke}{rgb}{0.003922,0.450980,0.698039}%
\pgfsetstrokecolor{currentstroke}%
\pgfsetstrokeopacity{0.700000}%
\pgfsetdash{}{0pt}%
\pgfpathmoveto{\pgfqpoint{0.689863in}{1.586143in}}%
\pgfpathlineto{\pgfqpoint{0.691573in}{1.625337in}}%
\pgfpathlineto{\pgfqpoint{0.694995in}{1.554591in}}%
\pgfpathlineto{\pgfqpoint{0.700132in}{1.624113in}}%
\pgfpathlineto{\pgfqpoint{0.706120in}{1.553835in}}%
\pgfpathlineto{\pgfqpoint{0.707833in}{1.630006in}}%
\pgfpathlineto{\pgfqpoint{0.712110in}{1.550218in}}%
\pgfpathlineto{\pgfqpoint{0.719814in}{1.595418in}}%
\pgfpathlineto{\pgfqpoint{0.723233in}{1.529484in}}%
\pgfpathlineto{\pgfqpoint{0.725797in}{1.600142in}}%
\pgfpathlineto{\pgfqpoint{0.730072in}{1.521436in}}%
\pgfpathlineto{\pgfqpoint{0.734351in}{1.620555in}}%
\pgfpathlineto{\pgfqpoint{0.740340in}{1.509859in}}%
\pgfpathlineto{\pgfqpoint{0.742908in}{1.627964in}}%
\pgfpathlineto{\pgfqpoint{0.749745in}{1.515752in}}%
\pgfpathlineto{\pgfqpoint{0.750602in}{1.631112in}}%
\pgfpathlineto{\pgfqpoint{0.757448in}{1.584280in}}%
\pgfpathlineto{\pgfqpoint{0.761721in}{1.771029in}}%
\pgfpathlineto{\pgfqpoint{0.764288in}{1.661499in}}%
\pgfpathlineto{\pgfqpoint{0.770278in}{1.670975in}}%
\pgfpathlineto{\pgfqpoint{0.771990in}{1.588302in}}%
\pgfpathlineto{\pgfqpoint{0.777980in}{1.623412in}}%
\pgfpathlineto{\pgfqpoint{0.783113in}{1.520705in}}%
\pgfpathlineto{\pgfqpoint{0.784824in}{1.596583in}}%
\pgfpathlineto{\pgfqpoint{0.792516in}{1.565616in}}%
\pgfpathlineto{\pgfqpoint{0.793373in}{1.644234in}}%
\pgfpathlineto{\pgfqpoint{0.797648in}{1.563749in}}%
\pgfpathlineto{\pgfqpoint{0.802779in}{1.499769in}}%
\pgfpathlineto{\pgfqpoint{0.807054in}{1.491546in}}%
\pgfpathlineto{\pgfqpoint{0.811330in}{1.621837in}}%
\pgfpathlineto{\pgfqpoint{0.816461in}{1.552432in}}%
\pgfpathlineto{\pgfqpoint{0.819881in}{1.653214in}}%
\pgfpathlineto{\pgfqpoint{0.825871in}{1.546953in}}%
\pgfpathlineto{\pgfqpoint{0.827584in}{1.683834in}}%
\pgfpathlineto{\pgfqpoint{0.836141in}{1.522426in}}%
\pgfpathlineto{\pgfqpoint{0.841276in}{1.594483in}}%
\pgfpathlineto{\pgfqpoint{0.848114in}{1.547709in}}%
\pgfpathlineto{\pgfqpoint{0.851531in}{1.619912in}}%
\pgfpathlineto{\pgfqpoint{0.854095in}{1.645692in}}%
\pgfpathlineto{\pgfqpoint{0.857519in}{1.567596in}}%
\pgfpathlineto{\pgfqpoint{0.864372in}{1.555406in}}%
\pgfpathlineto{\pgfqpoint{0.867793in}{1.625220in}}%
\pgfpathlineto{\pgfqpoint{0.871215in}{1.560422in}}%
\pgfpathlineto{\pgfqpoint{0.874639in}{1.655431in}}%
\pgfpathlineto{\pgfqpoint{0.879771in}{1.700631in}}%
\pgfpathlineto{\pgfqpoint{0.883195in}{1.617114in}}%
\pgfpathlineto{\pgfqpoint{0.889177in}{1.721161in}}%
\pgfpathlineto{\pgfqpoint{0.893457in}{1.578095in}}%
\pgfpathlineto{\pgfqpoint{0.898587in}{1.665404in}}%
\pgfpathlineto{\pgfqpoint{0.901153in}{1.582702in}}%
\pgfpathlineto{\pgfqpoint{0.906287in}{1.633735in}}%
\pgfpathlineto{\pgfqpoint{0.908853in}{1.559285in}}%
\pgfpathlineto{\pgfqpoint{0.916550in}{1.533012in}}%
\pgfpathlineto{\pgfqpoint{0.918259in}{1.614516in}}%
\pgfpathlineto{\pgfqpoint{0.924241in}{1.548176in}}%
\pgfpathlineto{\pgfqpoint{0.925951in}{1.593201in}}%
\pgfpathlineto{\pgfqpoint{0.930227in}{1.522163in}}%
\pgfpathlineto{\pgfqpoint{0.936214in}{1.628775in}}%
\pgfpathlineto{\pgfqpoint{0.938782in}{1.509446in}}%
\pgfpathlineto{\pgfqpoint{0.943058in}{1.594717in}}%
\pgfpathlineto{\pgfqpoint{0.947330in}{1.479644in}}%
\pgfpathlineto{\pgfqpoint{0.952461in}{1.636943in}}%
\pgfpathlineto{\pgfqpoint{0.958451in}{1.527588in}}%
\pgfpathlineto{\pgfqpoint{0.961015in}{1.600025in}}%
\pgfpathlineto{\pgfqpoint{0.967858in}{1.482388in}}%
\pgfpathlineto{\pgfqpoint{0.970425in}{1.625045in}}%
\pgfpathlineto{\pgfqpoint{0.976408in}{1.643591in}}%
\pgfpathlineto{\pgfqpoint{0.977263in}{1.606586in}}%
\pgfpathlineto{\pgfqpoint{0.982392in}{1.522689in}}%
\pgfpathlineto{\pgfqpoint{0.989232in}{1.502626in}}%
\pgfpathlineto{\pgfqpoint{0.990941in}{1.614896in}}%
\pgfpathlineto{\pgfqpoint{1.000349in}{1.478625in}}%
\pgfpathlineto{\pgfqpoint{1.002914in}{1.594454in}}%
\pgfpathlineto{\pgfqpoint{1.009758in}{1.546573in}}%
\pgfpathlineto{\pgfqpoint{1.011468in}{1.620789in}}%
\pgfpathlineto{\pgfqpoint{1.017456in}{1.559958in}}%
\pgfpathlineto{\pgfqpoint{1.022584in}{1.708156in}}%
\pgfpathlineto{\pgfqpoint{1.024292in}{1.600609in}}%
\pgfpathlineto{\pgfqpoint{1.031988in}{1.634320in}}%
\pgfpathlineto{\pgfqpoint{1.032842in}{1.691184in}}%
\pgfpathlineto{\pgfqpoint{1.039687in}{1.748691in}}%
\pgfpathlineto{\pgfqpoint{1.043113in}{1.674416in}}%
\pgfpathlineto{\pgfqpoint{1.048251in}{1.722268in}}%
\pgfpathlineto{\pgfqpoint{1.049962in}{1.651522in}}%
\pgfpathlineto{\pgfqpoint{1.055949in}{1.713289in}}%
\pgfpathlineto{\pgfqpoint{1.060226in}{1.562756in}}%
\pgfpathlineto{\pgfqpoint{1.064502in}{1.663011in}}%
\pgfpathlineto{\pgfqpoint{1.068774in}{1.536859in}}%
\pgfpathlineto{\pgfqpoint{1.071339in}{1.603086in}}%
\pgfpathlineto{\pgfqpoint{1.076466in}{1.541641in}}%
\pgfpathlineto{\pgfqpoint{1.080746in}{1.617724in}}%
\pgfpathlineto{\pgfqpoint{1.086736in}{1.542109in}}%
\pgfpathlineto{\pgfqpoint{1.091868in}{1.631226in}}%
\pgfpathlineto{\pgfqpoint{1.096997in}{1.523620in}}%
\pgfpathlineto{\pgfqpoint{1.102978in}{1.566836in}}%
\pgfpathlineto{\pgfqpoint{1.108959in}{1.626415in}}%
\pgfpathlineto{\pgfqpoint{1.111524in}{1.539661in}}%
\pgfpathlineto{\pgfqpoint{1.115796in}{1.683048in}}%
\pgfpathlineto{\pgfqpoint{1.121786in}{1.726002in}}%
\pgfpathlineto{\pgfqpoint{1.122640in}{1.618659in}}%
\pgfpathlineto{\pgfqpoint{1.130340in}{1.593551in}}%
\pgfpathlineto{\pgfqpoint{1.132050in}{1.648929in}}%
\pgfpathlineto{\pgfqpoint{1.136328in}{1.567830in}}%
\pgfpathlineto{\pgfqpoint{1.140604in}{1.724602in}}%
\pgfpathlineto{\pgfqpoint{1.146593in}{1.623032in}}%
\pgfpathlineto{\pgfqpoint{1.154289in}{1.756154in}}%
\pgfpathlineto{\pgfqpoint{1.157708in}{1.579319in}}%
\pgfpathlineto{\pgfqpoint{1.163696in}{1.517322in}}%
\pgfpathlineto{\pgfqpoint{1.165408in}{1.614546in}}%
\pgfpathlineto{\pgfqpoint{1.171400in}{1.551033in}}%
\pgfpathlineto{\pgfqpoint{1.174823in}{1.608072in}}%
\pgfpathlineto{\pgfqpoint{1.179102in}{1.542781in}}%
\pgfpathlineto{\pgfqpoint{1.185947in}{1.665579in}}%
\pgfpathlineto{\pgfqpoint{1.188512in}{1.532749in}}%
\pgfpathlineto{\pgfqpoint{1.191934in}{1.637001in}}%
\pgfpathlineto{\pgfqpoint{1.196209in}{1.492595in}}%
\pgfpathlineto{\pgfqpoint{1.203055in}{1.490786in}}%
\pgfpathlineto{\pgfqpoint{1.203910in}{1.586611in}}%
\pgfpathlineto{\pgfqpoint{1.211615in}{1.672053in}}%
\pgfpathlineto{\pgfqpoint{1.212472in}{1.589643in}}%
\pgfpathlineto{\pgfqpoint{1.216755in}{1.605128in}}%
\pgfpathlineto{\pgfqpoint{1.221891in}{1.553397in}}%
\pgfpathlineto{\pgfqpoint{1.228734in}{1.607780in}}%
\pgfpathlineto{\pgfqpoint{1.230446in}{1.528694in}}%
\pgfpathlineto{\pgfqpoint{1.236437in}{1.577978in}}%
\pgfpathlineto{\pgfqpoint{1.239857in}{1.539135in}}%
\pgfpathlineto{\pgfqpoint{1.245845in}{1.615072in}}%
\pgfpathlineto{\pgfqpoint{1.247557in}{1.566080in}}%
\pgfpathlineto{\pgfqpoint{1.251837in}{1.608540in}}%
\pgfpathlineto{\pgfqpoint{1.256112in}{1.540943in}}%
\pgfpathlineto{\pgfqpoint{1.262954in}{1.599323in}}%
\pgfpathlineto{\pgfqpoint{1.264666in}{1.522163in}}%
\pgfpathlineto{\pgfqpoint{1.269800in}{1.730783in}}%
\pgfpathlineto{\pgfqpoint{1.274080in}{1.587542in}}%
\pgfpathlineto{\pgfqpoint{1.277502in}{1.548874in}}%
\pgfpathlineto{\pgfqpoint{1.284344in}{1.739766in}}%
\pgfpathlineto{\pgfqpoint{1.286057in}{1.610407in}}%
\pgfpathlineto{\pgfqpoint{1.289480in}{1.562058in}}%
\pgfpathlineto{\pgfqpoint{1.297179in}{1.694070in}}%
\pgfpathlineto{\pgfqpoint{1.298889in}{1.618922in}}%
\pgfpathlineto{\pgfqpoint{1.302314in}{1.726060in}}%
\pgfpathlineto{\pgfqpoint{1.309157in}{1.629070in}}%
\pgfpathlineto{\pgfqpoint{1.312579in}{1.717369in}}%
\pgfpathlineto{\pgfqpoint{1.317712in}{1.618279in}}%
\pgfpathlineto{\pgfqpoint{1.321985in}{1.715678in}}%
\pgfpathlineto{\pgfqpoint{1.324548in}{1.687158in}}%
\pgfpathlineto{\pgfqpoint{1.331388in}{1.692408in}}%
\pgfpathlineto{\pgfqpoint{1.333951in}{1.537210in}}%
\pgfpathlineto{\pgfqpoint{1.339938in}{1.619795in}}%
\pgfpathlineto{\pgfqpoint{1.343354in}{1.535577in}}%
\pgfpathlineto{\pgfqpoint{1.345921in}{1.609588in}}%
\pgfpathlineto{\pgfqpoint{1.350199in}{1.549575in}}%
\pgfpathlineto{\pgfqpoint{1.356186in}{1.629651in}}%
\pgfpathlineto{\pgfqpoint{1.358754in}{1.557331in}}%
\pgfpathlineto{\pgfqpoint{1.365593in}{1.632976in}}%
\pgfpathlineto{\pgfqpoint{1.369869in}{1.502100in}}%
\pgfpathlineto{\pgfqpoint{1.371579in}{1.568239in}}%
\pgfpathlineto{\pgfqpoint{1.378419in}{1.617987in}}%
\pgfpathlineto{\pgfqpoint{1.382696in}{1.510440in}}%
\pgfpathlineto{\pgfqpoint{1.383549in}{1.581361in}}%
\pgfpathlineto{\pgfqpoint{1.387825in}{1.689142in}}%
\pgfpathlineto{\pgfqpoint{1.392960in}{1.650415in}}%
\pgfpathlineto{\pgfqpoint{1.396386in}{1.710195in}}%
\pgfpathlineto{\pgfqpoint{1.400665in}{1.665404in}}%
\pgfpathlineto{\pgfqpoint{1.404941in}{1.722239in}}%
\pgfpathlineto{\pgfqpoint{1.410071in}{1.632044in}}%
\pgfpathlineto{\pgfqpoint{1.416052in}{1.616588in}}%
\pgfpathlineto{\pgfqpoint{1.417764in}{1.747931in}}%
\pgfpathlineto{\pgfqpoint{1.424607in}{1.633443in}}%
\pgfpathlineto{\pgfqpoint{1.426320in}{1.661758in}}%
\pgfpathlineto{\pgfqpoint{1.433162in}{1.599995in}}%
\pgfpathlineto{\pgfqpoint{1.435727in}{1.686314in}}%
\pgfpathlineto{\pgfqpoint{1.439148in}{1.617465in}}%
\pgfpathlineto{\pgfqpoint{1.445989in}{1.705124in}}%
\pgfpathlineto{\pgfqpoint{1.447700in}{1.624084in}}%
\pgfpathlineto{\pgfqpoint{1.454547in}{1.676575in}}%
\pgfpathlineto{\pgfqpoint{1.457115in}{1.610670in}}%
\pgfpathlineto{\pgfqpoint{1.460537in}{1.665057in}}%
\pgfpathlineto{\pgfqpoint{1.464817in}{1.622714in}}%
\pgfpathlineto{\pgfqpoint{1.469955in}{1.730261in}}%
\pgfpathlineto{\pgfqpoint{1.473375in}{1.618922in}}%
\pgfpathlineto{\pgfqpoint{1.478513in}{1.520822in}}%
\pgfpathlineto{\pgfqpoint{1.485359in}{1.527851in}}%
\pgfpathlineto{\pgfqpoint{1.489636in}{1.629567in}}%
\pgfpathlineto{\pgfqpoint{1.490491in}{1.509567in}}%
\pgfpathlineto{\pgfqpoint{1.498186in}{1.649192in}}%
\pgfpathlineto{\pgfqpoint{1.499039in}{1.531526in}}%
\pgfpathlineto{\pgfqpoint{1.503315in}{1.663424in}}%
\pgfpathlineto{\pgfqpoint{1.509305in}{1.524380in}}%
\pgfpathlineto{\pgfqpoint{1.515290in}{1.629070in}}%
\pgfpathlineto{\pgfqpoint{1.516145in}{1.549634in}}%
\pgfpathlineto{\pgfqpoint{1.521272in}{1.679110in}}%
\pgfpathlineto{\pgfqpoint{1.527256in}{1.500350in}}%
\pgfpathlineto{\pgfqpoint{1.528967in}{1.616705in}}%
\pgfpathlineto{\pgfqpoint{1.535810in}{1.601307in}}%
\pgfpathlineto{\pgfqpoint{1.539230in}{1.533652in}}%
\pgfpathlineto{\pgfqpoint{1.545218in}{1.634901in}}%
\pgfpathlineto{\pgfqpoint{1.546074in}{1.545754in}}%
\pgfpathlineto{\pgfqpoint{1.552920in}{1.627901in}}%
\pgfpathlineto{\pgfqpoint{1.554633in}{1.535577in}}%
\pgfpathlineto{\pgfqpoint{1.561475in}{1.633209in}}%
\pgfpathlineto{\pgfqpoint{1.563184in}{1.527178in}}%
\pgfpathlineto{\pgfqpoint{1.568314in}{1.665462in}}%
\pgfpathlineto{\pgfqpoint{1.573455in}{1.552841in}}%
\pgfpathlineto{\pgfqpoint{1.577737in}{1.621662in}}%
\pgfpathlineto{\pgfqpoint{1.582014in}{1.532019in}}%
\pgfpathlineto{\pgfqpoint{1.587148in}{1.626444in}}%
\pgfpathlineto{\pgfqpoint{1.592280in}{1.516446in}}%
\pgfpathlineto{\pgfqpoint{1.596559in}{1.616354in}}%
\pgfpathlineto{\pgfqpoint{1.598270in}{1.508310in}}%
\pgfpathlineto{\pgfqpoint{1.605969in}{1.651347in}}%
\pgfpathlineto{\pgfqpoint{1.611957in}{1.534817in}}%
\pgfpathlineto{\pgfqpoint{1.617087in}{1.646419in}}%
\pgfpathlineto{\pgfqpoint{1.620507in}{1.579611in}}%
\pgfpathlineto{\pgfqpoint{1.623070in}{1.604748in}}%
\pgfpathlineto{\pgfqpoint{1.630773in}{1.553046in}}%
\pgfpathlineto{\pgfqpoint{1.635054in}{1.646273in}}%
\pgfpathlineto{\pgfqpoint{1.635909in}{1.561298in}}%
\pgfpathlineto{\pgfqpoint{1.643599in}{1.653155in}}%
\pgfpathlineto{\pgfqpoint{1.645308in}{1.567684in}}%
\pgfpathlineto{\pgfqpoint{1.649584in}{1.596320in}}%
\pgfpathlineto{\pgfqpoint{1.655572in}{1.521637in}}%
\pgfpathlineto{\pgfqpoint{1.660704in}{1.610056in}}%
\pgfpathlineto{\pgfqpoint{1.663267in}{1.560422in}}%
\pgfpathlineto{\pgfqpoint{1.671820in}{1.683892in}}%
\pgfpathlineto{\pgfqpoint{1.674387in}{1.534119in}}%
\pgfpathlineto{\pgfqpoint{1.680376in}{1.484572in}}%
\pgfpathlineto{\pgfqpoint{1.683795in}{1.622535in}}%
\pgfpathlineto{\pgfqpoint{1.692344in}{1.489793in}}%
\pgfpathlineto{\pgfqpoint{1.695764in}{1.611280in}}%
\pgfpathlineto{\pgfqpoint{1.701749in}{1.772136in}}%
\pgfpathlineto{\pgfqpoint{1.705169in}{1.640559in}}%
\pgfpathlineto{\pgfqpoint{1.712012in}{1.694274in}}%
\pgfpathlineto{\pgfqpoint{1.716288in}{1.611514in}}%
\pgfpathlineto{\pgfqpoint{1.717143in}{1.675260in}}%
\pgfpathlineto{\pgfqpoint{1.722276in}{1.715097in}}%
\pgfpathlineto{\pgfqpoint{1.725698in}{1.600551in}}%
\pgfpathlineto{\pgfqpoint{1.730826in}{1.519131in}}%
\pgfpathlineto{\pgfqpoint{1.735959in}{1.623646in}}%
\pgfpathlineto{\pgfqpoint{1.738528in}{1.568064in}}%
\pgfpathlineto{\pgfqpoint{1.742805in}{1.624811in}}%
\pgfpathlineto{\pgfqpoint{1.747088in}{1.558964in}}%
\pgfpathlineto{\pgfqpoint{1.751369in}{1.518838in}}%
\pgfpathlineto{\pgfqpoint{1.756502in}{1.610290in}}%
\pgfpathlineto{\pgfqpoint{1.759920in}{1.539252in}}%
\pgfpathlineto{\pgfqpoint{1.765902in}{1.614663in}}%
\pgfpathlineto{\pgfqpoint{1.771888in}{1.532545in}}%
\pgfpathlineto{\pgfqpoint{1.776162in}{1.620672in}}%
\pgfpathlineto{\pgfqpoint{1.777874in}{1.527003in}}%
\pgfpathlineto{\pgfqpoint{1.783002in}{1.591276in}}%
\pgfpathlineto{\pgfqpoint{1.786423in}{1.524961in}}%
\pgfpathlineto{\pgfqpoint{1.789847in}{1.633092in}}%
\pgfpathlineto{\pgfqpoint{1.794127in}{1.538843in}}%
\pgfpathlineto{\pgfqpoint{1.800974in}{1.519014in}}%
\pgfpathlineto{\pgfqpoint{1.806101in}{1.614604in}}%
\pgfpathlineto{\pgfqpoint{1.807814in}{1.513764in}}%
\pgfpathlineto{\pgfqpoint{1.813805in}{1.675202in}}%
\pgfpathlineto{\pgfqpoint{1.817226in}{1.562697in}}%
\pgfpathlineto{\pgfqpoint{1.822358in}{1.686636in}}%
\pgfpathlineto{\pgfqpoint{1.824068in}{1.612712in}}%
\pgfpathlineto{\pgfqpoint{1.831766in}{1.648782in}}%
\pgfpathlineto{\pgfqpoint{1.833476in}{1.558179in}}%
\pgfpathlineto{\pgfqpoint{1.836895in}{1.640267in}}%
\pgfpathlineto{\pgfqpoint{1.842883in}{1.545725in}}%
\pgfpathlineto{\pgfqpoint{1.848875in}{1.685350in}}%
\pgfpathlineto{\pgfqpoint{1.849731in}{1.567567in}}%
\pgfpathlineto{\pgfqpoint{1.854859in}{1.619445in}}%
\pgfpathlineto{\pgfqpoint{1.859990in}{1.529162in}}%
\pgfpathlineto{\pgfqpoint{1.864266in}{1.490436in}}%
\pgfpathlineto{\pgfqpoint{1.868544in}{1.628194in}}%
\pgfpathlineto{\pgfqpoint{1.872817in}{1.640384in}}%
\pgfpathlineto{\pgfqpoint{1.877090in}{1.539486in}}%
\pgfpathlineto{\pgfqpoint{1.879659in}{1.594366in}}%
\pgfpathlineto{\pgfqpoint{1.883937in}{1.544268in}}%
\pgfpathlineto{\pgfqpoint{1.888217in}{1.594717in}}%
\pgfpathlineto{\pgfqpoint{1.895915in}{1.510148in}}%
\pgfpathlineto{\pgfqpoint{1.896771in}{1.590340in}}%
\pgfpathlineto{\pgfqpoint{1.901907in}{1.605913in}}%
\pgfpathlineto{\pgfqpoint{1.907894in}{1.530269in}}%
\pgfpathlineto{\pgfqpoint{1.912175in}{1.508573in}}%
\pgfpathlineto{\pgfqpoint{1.913882in}{1.607605in}}%
\pgfpathlineto{\pgfqpoint{1.918162in}{1.528548in}}%
\pgfpathlineto{\pgfqpoint{1.922443in}{1.636767in}}%
\pgfpathlineto{\pgfqpoint{1.926721in}{1.526072in}}%
\pgfpathlineto{\pgfqpoint{1.932706in}{1.500993in}}%
\pgfpathlineto{\pgfqpoint{1.935269in}{1.617581in}}%
\pgfpathlineto{\pgfqpoint{1.940403in}{1.716613in}}%
\pgfpathlineto{\pgfqpoint{1.947248in}{1.607024in}}%
\pgfpathlineto{\pgfqpoint{1.950671in}{1.719236in}}%
\pgfpathlineto{\pgfqpoint{1.955805in}{1.564184in}}%
\pgfpathlineto{\pgfqpoint{1.960085in}{1.546368in}}%
\pgfpathlineto{\pgfqpoint{1.961796in}{1.620321in}}%
\pgfpathlineto{\pgfqpoint{1.967788in}{1.663654in}}%
\pgfpathlineto{\pgfqpoint{1.970354in}{1.526188in}}%
\pgfpathlineto{\pgfqpoint{1.976334in}{1.645224in}}%
\pgfpathlineto{\pgfqpoint{1.979754in}{1.525023in}}%
\pgfpathlineto{\pgfqpoint{1.985740in}{1.629655in}}%
\pgfpathlineto{\pgfqpoint{1.986595in}{1.553777in}}%
\pgfpathlineto{\pgfqpoint{1.993441in}{1.537707in}}%
\pgfpathlineto{\pgfqpoint{1.995151in}{1.647500in}}%
\pgfpathlineto{\pgfqpoint{2.001137in}{1.518082in}}%
\pgfpathlineto{\pgfqpoint{2.004557in}{1.605099in}}%
\pgfpathlineto{\pgfqpoint{2.010548in}{1.566723in}}%
\pgfpathlineto{\pgfqpoint{2.014825in}{1.650708in}}%
\pgfpathlineto{\pgfqpoint{2.016537in}{1.566489in}}%
\pgfpathlineto{\pgfqpoint{2.021667in}{1.650445in}}%
\pgfpathlineto{\pgfqpoint{2.027657in}{1.671877in}}%
\pgfpathlineto{\pgfqpoint{2.031077in}{1.543537in}}%
\pgfpathlineto{\pgfqpoint{2.035355in}{1.654847in}}%
\pgfpathlineto{\pgfqpoint{2.039628in}{1.555581in}}%
\pgfpathlineto{\pgfqpoint{2.043906in}{1.635661in}}%
\pgfpathlineto{\pgfqpoint{2.047328in}{1.531058in}}%
\pgfpathlineto{\pgfqpoint{2.054173in}{1.525662in}}%
\pgfpathlineto{\pgfqpoint{2.055027in}{1.633735in}}%
\pgfpathlineto{\pgfqpoint{2.060159in}{1.507466in}}%
\pgfpathlineto{\pgfqpoint{2.065290in}{1.621837in}}%
\pgfpathlineto{\pgfqpoint{2.067854in}{1.578504in}}%
\pgfpathlineto{\pgfqpoint{2.075551in}{1.509918in}}%
\pgfpathlineto{\pgfqpoint{2.077262in}{1.664239in}}%
\pgfpathlineto{\pgfqpoint{2.081538in}{1.578095in}}%
\pgfpathlineto{\pgfqpoint{2.084956in}{1.607956in}}%
\pgfpathlineto{\pgfqpoint{2.090086in}{1.524029in}}%
\pgfpathlineto{\pgfqpoint{2.093509in}{1.598684in}}%
\pgfpathlineto{\pgfqpoint{2.097788in}{1.519131in}}%
\pgfpathlineto{\pgfqpoint{2.102927in}{1.624402in}}%
\pgfpathlineto{\pgfqpoint{2.106351in}{1.542109in}}%
\pgfpathlineto{\pgfqpoint{2.114054in}{1.512248in}}%
\pgfpathlineto{\pgfqpoint{2.114910in}{1.620727in}}%
\pgfpathlineto{\pgfqpoint{2.122605in}{1.663946in}}%
\pgfpathlineto{\pgfqpoint{2.124318in}{1.588243in}}%
\pgfpathlineto{\pgfqpoint{2.130310in}{1.557740in}}%
\pgfpathlineto{\pgfqpoint{2.132878in}{1.648373in}}%
\pgfpathlineto{\pgfqpoint{2.138864in}{1.515719in}}%
\pgfpathlineto{\pgfqpoint{2.140576in}{1.599031in}}%
\pgfpathlineto{\pgfqpoint{2.145701in}{1.552257in}}%
\pgfpathlineto{\pgfqpoint{2.151687in}{1.762452in}}%
\pgfpathlineto{\pgfqpoint{2.153397in}{1.633823in}}%
\pgfpathlineto{\pgfqpoint{2.158523in}{1.725709in}}%
\pgfpathlineto{\pgfqpoint{2.164513in}{1.642075in}}%
\pgfpathlineto{\pgfqpoint{2.169647in}{1.646945in}}%
\pgfpathlineto{\pgfqpoint{2.173068in}{1.753297in}}%
\pgfpathlineto{\pgfqpoint{2.176493in}{1.561587in}}%
\pgfpathlineto{\pgfqpoint{2.181627in}{1.683542in}}%
\pgfpathlineto{\pgfqpoint{2.184193in}{1.617812in}}%
\pgfpathlineto{\pgfqpoint{2.189327in}{1.690833in}}%
\pgfpathlineto{\pgfqpoint{2.193602in}{1.635368in}}%
\pgfpathlineto{\pgfqpoint{2.197026in}{1.685642in}}%
\pgfpathlineto{\pgfqpoint{2.203016in}{1.582526in}}%
\pgfpathlineto{\pgfqpoint{2.205581in}{1.681675in}}%
\pgfpathlineto{\pgfqpoint{2.212427in}{1.697190in}}%
\pgfpathlineto{\pgfqpoint{2.215846in}{1.582468in}}%
\pgfpathlineto{\pgfqpoint{2.218411in}{1.701040in}}%
\pgfpathlineto{\pgfqpoint{2.222686in}{1.598976in}}%
\pgfpathlineto{\pgfqpoint{2.226962in}{1.709146in}}%
\pgfpathlineto{\pgfqpoint{2.232095in}{1.635719in}}%
\pgfpathlineto{\pgfqpoint{2.234663in}{1.700806in}}%
\pgfpathlineto{\pgfqpoint{2.241504in}{1.618279in}}%
\pgfpathlineto{\pgfqpoint{2.243215in}{1.705413in}}%
\pgfpathlineto{\pgfqpoint{2.250053in}{1.649539in}}%
\pgfpathlineto{\pgfqpoint{2.255182in}{1.735741in}}%
\pgfpathlineto{\pgfqpoint{2.258601in}{1.748629in}}%
\pgfpathlineto{\pgfqpoint{2.261169in}{1.634813in}}%
\pgfpathlineto{\pgfqpoint{2.264591in}{1.668670in}}%
\pgfpathlineto{\pgfqpoint{2.268867in}{1.510557in}}%
\pgfpathlineto{\pgfqpoint{2.273147in}{1.530795in}}%
\pgfpathlineto{\pgfqpoint{2.278277in}{1.613669in}}%
\pgfpathlineto{\pgfqpoint{2.282560in}{1.540300in}}%
\pgfpathlineto{\pgfqpoint{2.286838in}{1.601015in}}%
\pgfpathlineto{\pgfqpoint{2.290262in}{1.534236in}}%
\pgfpathlineto{\pgfqpoint{2.295399in}{1.639507in}}%
\pgfpathlineto{\pgfqpoint{2.298819in}{1.553919in}}%
\pgfpathlineto{\pgfqpoint{2.309087in}{1.658580in}}%
\pgfpathlineto{\pgfqpoint{2.311649in}{1.592211in}}%
\pgfpathlineto{\pgfqpoint{2.318493in}{1.627960in}}%
\pgfpathlineto{\pgfqpoint{2.321912in}{1.493351in}}%
\pgfpathlineto{\pgfqpoint{2.326189in}{1.578095in}}%
\pgfpathlineto{\pgfqpoint{2.330467in}{1.587074in}}%
\pgfpathlineto{\pgfqpoint{2.333891in}{1.516796in}}%
\pgfpathlineto{\pgfqpoint{2.339879in}{1.502392in}}%
\pgfpathlineto{\pgfqpoint{2.341587in}{1.622798in}}%
\pgfpathlineto{\pgfqpoint{2.348427in}{1.558379in}}%
\pgfpathlineto{\pgfqpoint{2.350137in}{1.639273in}}%
\pgfpathlineto{\pgfqpoint{2.356126in}{1.589409in}}%
\pgfpathlineto{\pgfqpoint{2.363830in}{1.541586in}}%
\pgfpathlineto{\pgfqpoint{2.370673in}{1.723905in}}%
\pgfpathlineto{\pgfqpoint{2.374095in}{1.595330in}}%
\pgfpathlineto{\pgfqpoint{2.375807in}{1.664122in}}%
\pgfpathlineto{\pgfqpoint{2.383504in}{1.680363in}}%
\pgfpathlineto{\pgfqpoint{2.386924in}{1.621720in}}%
\pgfpathlineto{\pgfqpoint{2.392057in}{1.637235in}}%
\pgfpathlineto{\pgfqpoint{2.396335in}{1.513180in}}%
\pgfpathlineto{\pgfqpoint{2.397190in}{1.634375in}}%
\pgfpathlineto{\pgfqpoint{2.402321in}{1.647208in}}%
\pgfpathlineto{\pgfqpoint{2.407451in}{1.534324in}}%
\pgfpathlineto{\pgfqpoint{2.411726in}{1.522250in}}%
\pgfpathlineto{\pgfqpoint{2.417709in}{1.508807in}}%
\pgfpathlineto{\pgfqpoint{2.418564in}{1.608131in}}%
\pgfpathlineto{\pgfqpoint{2.423696in}{1.525370in}}%
\pgfpathlineto{\pgfqpoint{2.427115in}{1.595765in}}%
\pgfpathlineto{\pgfqpoint{2.433100in}{1.582526in}}%
\pgfpathlineto{\pgfqpoint{2.436521in}{1.643708in}}%
\pgfpathlineto{\pgfqpoint{2.441651in}{1.684477in}}%
\pgfpathlineto{\pgfqpoint{2.445075in}{1.554387in}}%
\pgfpathlineto{\pgfqpoint{2.451058in}{1.546017in}}%
\pgfpathlineto{\pgfqpoint{2.452770in}{1.618104in}}%
\pgfpathlineto{\pgfqpoint{2.457908in}{1.645107in}}%
\pgfpathlineto{\pgfqpoint{2.463895in}{1.600142in}}%
\pgfpathlineto{\pgfqpoint{2.465606in}{1.668670in}}%
\pgfpathlineto{\pgfqpoint{2.470738in}{1.560279in}}%
\pgfpathlineto{\pgfqpoint{2.474162in}{1.643708in}}%
\pgfpathlineto{\pgfqpoint{2.479295in}{1.568005in}}%
\pgfpathlineto{\pgfqpoint{2.486139in}{1.598626in}}%
\pgfpathlineto{\pgfqpoint{2.489559in}{1.507729in}}%
\pgfpathlineto{\pgfqpoint{2.494692in}{1.613906in}}%
\pgfpathlineto{\pgfqpoint{2.496403in}{1.504522in}}%
\pgfpathlineto{\pgfqpoint{2.502393in}{1.653564in}}%
\pgfpathlineto{\pgfqpoint{2.506673in}{1.647968in}}%
\pgfpathlineto{\pgfqpoint{2.508382in}{1.546485in}}%
\pgfpathlineto{\pgfqpoint{2.516081in}{1.623762in}}%
\pgfpathlineto{\pgfqpoint{2.518648in}{1.536337in}}%
\pgfpathlineto{\pgfqpoint{2.522070in}{1.603641in}}%
\pgfpathlineto{\pgfqpoint{2.528054in}{1.514583in}}%
\pgfpathlineto{\pgfqpoint{2.530622in}{1.584514in}}%
\pgfpathlineto{\pgfqpoint{2.535758in}{1.638810in}}%
\pgfpathlineto{\pgfqpoint{2.538324in}{1.531730in}}%
\pgfpathlineto{\pgfqpoint{2.543453in}{1.653798in}}%
\pgfpathlineto{\pgfqpoint{2.549441in}{1.526013in}}%
\pgfpathlineto{\pgfqpoint{2.553715in}{1.628018in}}%
\pgfpathlineto{\pgfqpoint{2.556277in}{1.547709in}}%
\pgfpathlineto{\pgfqpoint{2.563119in}{1.626795in}}%
\pgfpathlineto{\pgfqpoint{2.565684in}{1.517030in}}%
\pgfpathlineto{\pgfqpoint{2.569960in}{1.681269in}}%
\pgfpathlineto{\pgfqpoint{2.572527in}{1.615598in}}%
\pgfpathlineto{\pgfqpoint{2.579370in}{1.703667in}}%
\pgfpathlineto{\pgfqpoint{2.581931in}{1.607550in}}%
\pgfpathlineto{\pgfqpoint{2.587061in}{1.688152in}}%
\pgfpathlineto{\pgfqpoint{2.593048in}{1.638897in}}%
\pgfpathlineto{\pgfqpoint{2.595611in}{1.703082in}}%
\pgfpathlineto{\pgfqpoint{2.601596in}{1.594603in}}%
\pgfpathlineto{\pgfqpoint{2.604164in}{1.674913in}}%
\pgfpathlineto{\pgfqpoint{2.609301in}{1.617728in}}%
\pgfpathlineto{\pgfqpoint{2.614436in}{1.710954in}}%
\pgfpathlineto{\pgfqpoint{2.617000in}{1.622597in}}%
\pgfpathlineto{\pgfqpoint{2.619567in}{1.712938in}}%
\pgfpathlineto{\pgfqpoint{2.624695in}{1.682552in}}%
\pgfpathlineto{\pgfqpoint{2.628975in}{1.753473in}}%
\pgfpathlineto{\pgfqpoint{2.632399in}{1.667388in}}%
\pgfpathlineto{\pgfqpoint{2.639241in}{1.693924in}}%
\pgfpathlineto{\pgfqpoint{2.641809in}{1.618279in}}%
\pgfpathlineto{\pgfqpoint{2.646085in}{1.544034in}}%
\pgfpathlineto{\pgfqpoint{2.649509in}{1.532662in}}%
\pgfpathlineto{\pgfqpoint{2.653787in}{1.658814in}}%
\pgfpathlineto{\pgfqpoint{2.658066in}{1.709263in}}%
\pgfpathlineto{\pgfqpoint{2.662344in}{1.603232in}}%
\pgfpathlineto{\pgfqpoint{2.670044in}{1.563048in}}%
\pgfpathlineto{\pgfqpoint{2.672610in}{1.632102in}}%
\pgfpathlineto{\pgfqpoint{2.677741in}{1.523039in}}%
\pgfpathlineto{\pgfqpoint{2.681163in}{1.635427in}}%
\pgfpathlineto{\pgfqpoint{2.684586in}{1.527967in}}%
\pgfpathlineto{\pgfqpoint{2.693143in}{1.714337in}}%
\pgfpathlineto{\pgfqpoint{2.696565in}{1.561415in}}%
\pgfpathlineto{\pgfqpoint{2.702555in}{1.600609in}}%
\pgfpathlineto{\pgfqpoint{2.708544in}{1.530912in}}%
\pgfpathlineto{\pgfqpoint{2.709399in}{1.621720in}}%
\pgfpathlineto{\pgfqpoint{2.714525in}{1.535811in}}%
\pgfpathlineto{\pgfqpoint{2.721365in}{1.653740in}}%
\pgfpathlineto{\pgfqpoint{2.723934in}{1.549634in}}%
\pgfpathlineto{\pgfqpoint{2.726501in}{1.616997in}}%
\pgfpathlineto{\pgfqpoint{2.731635in}{1.622159in}}%
\pgfpathlineto{\pgfqpoint{2.737626in}{1.495919in}}%
\pgfpathlineto{\pgfqpoint{2.739338in}{1.585968in}}%
\pgfpathlineto{\pgfqpoint{2.745323in}{1.525019in}}%
\pgfpathlineto{\pgfqpoint{2.747889in}{1.617578in}}%
\pgfpathlineto{\pgfqpoint{2.755581in}{1.547738in}}%
\pgfpathlineto{\pgfqpoint{2.756436in}{1.615832in}}%
\pgfpathlineto{\pgfqpoint{2.762417in}{1.532749in}}%
\pgfpathlineto{\pgfqpoint{2.764980in}{1.610465in}}%
\pgfpathlineto{\pgfqpoint{2.770971in}{1.523098in}}%
\pgfpathlineto{\pgfqpoint{2.776103in}{1.612098in}}%
\pgfpathlineto{\pgfqpoint{2.777811in}{1.565266in}}%
\pgfpathlineto{\pgfqpoint{2.782945in}{1.630703in}}%
\pgfpathlineto{\pgfqpoint{2.789789in}{1.558149in}}%
\pgfpathlineto{\pgfqpoint{2.790642in}{1.599703in}}%
\pgfpathlineto{\pgfqpoint{2.794918in}{1.536684in}}%
\pgfpathlineto{\pgfqpoint{2.799189in}{1.635076in}}%
\pgfpathlineto{\pgfqpoint{2.803464in}{1.564097in}}%
\pgfpathlineto{\pgfqpoint{2.809450in}{1.615072in}}%
\pgfpathlineto{\pgfqpoint{2.812873in}{1.669839in}}%
\pgfpathlineto{\pgfqpoint{2.817150in}{1.508398in}}%
\pgfpathlineto{\pgfqpoint{2.820569in}{1.624928in}}%
\pgfpathlineto{\pgfqpoint{2.827409in}{1.610465in}}%
\pgfpathlineto{\pgfqpoint{2.829973in}{1.686168in}}%
\pgfpathlineto{\pgfqpoint{2.833396in}{1.617523in}}%
\pgfpathlineto{\pgfqpoint{2.839382in}{1.710081in}}%
\pgfpathlineto{\pgfqpoint{2.844515in}{1.647880in}}%
\pgfpathlineto{\pgfqpoint{2.849642in}{1.707630in}}%
\pgfpathlineto{\pgfqpoint{2.850497in}{1.536278in}}%
\pgfpathlineto{\pgfqpoint{2.855627in}{1.637060in}}%
\pgfpathlineto{\pgfqpoint{2.859900in}{1.463140in}}%
\pgfpathlineto{\pgfqpoint{2.865887in}{1.641348in}}%
\pgfpathlineto{\pgfqpoint{2.867599in}{1.521348in}}%
\pgfpathlineto{\pgfqpoint{2.874440in}{1.510440in}}%
\pgfpathlineto{\pgfqpoint{2.876150in}{1.688674in}}%
\pgfpathlineto{\pgfqpoint{2.880425in}{1.530561in}}%
\pgfpathlineto{\pgfqpoint{2.885558in}{1.672754in}}%
\pgfpathlineto{\pgfqpoint{2.889835in}{1.530269in}}%
\pgfpathlineto{\pgfqpoint{2.893259in}{1.609296in}}%
\pgfpathlineto{\pgfqpoint{2.900103in}{1.564798in}}%
\pgfpathlineto{\pgfqpoint{2.905233in}{1.724076in}}%
\pgfpathlineto{\pgfqpoint{2.907800in}{1.665579in}}%
\pgfpathlineto{\pgfqpoint{2.911219in}{1.714396in}}%
\pgfpathlineto{\pgfqpoint{2.918060in}{1.636943in}}%
\pgfpathlineto{\pgfqpoint{2.921481in}{1.690428in}}%
\pgfpathlineto{\pgfqpoint{2.924043in}{1.543336in}}%
\pgfpathlineto{\pgfqpoint{2.930027in}{1.644760in}}%
\pgfpathlineto{\pgfqpoint{2.932595in}{1.528289in}}%
\pgfpathlineto{\pgfqpoint{2.936017in}{1.621140in}}%
\pgfpathlineto{\pgfqpoint{2.941150in}{1.534558in}}%
\pgfpathlineto{\pgfqpoint{2.944568in}{1.630937in}}%
\pgfpathlineto{\pgfqpoint{2.948846in}{1.563282in}}%
\pgfpathlineto{\pgfqpoint{2.954831in}{1.526130in}}%
\pgfpathlineto{\pgfqpoint{2.963385in}{1.667388in}}%
\pgfpathlineto{\pgfqpoint{2.965950in}{1.608715in}}%
\pgfpathlineto{\pgfqpoint{2.972797in}{1.584978in}}%
\pgfpathlineto{\pgfqpoint{2.977928in}{1.638108in}}%
\pgfpathlineto{\pgfqpoint{2.979640in}{1.572846in}}%
\pgfpathlineto{\pgfqpoint{2.983917in}{1.681328in}}%
\pgfpathlineto{\pgfqpoint{2.989050in}{1.635953in}}%
\pgfpathlineto{\pgfqpoint{2.991615in}{1.694099in}}%
\pgfpathlineto{\pgfqpoint{2.997600in}{1.638722in}}%
\pgfpathlineto{\pgfqpoint{3.000167in}{1.715649in}}%
\pgfpathlineto{\pgfqpoint{3.004446in}{1.605099in}}%
\pgfpathlineto{\pgfqpoint{3.008722in}{1.759888in}}%
\pgfpathlineto{\pgfqpoint{3.012997in}{1.605972in}}%
\pgfpathlineto{\pgfqpoint{3.018130in}{1.702848in}}%
\pgfpathlineto{\pgfqpoint{3.024118in}{1.633151in}}%
\pgfpathlineto{\pgfqpoint{3.026684in}{1.713928in}}%
\pgfpathlineto{\pgfqpoint{3.030958in}{1.753820in}}%
\pgfpathlineto{\pgfqpoint{3.034379in}{1.635310in}}%
\pgfpathlineto{\pgfqpoint{3.041219in}{1.621837in}}%
\pgfpathlineto{\pgfqpoint{3.046352in}{1.752132in}}%
\pgfpathlineto{\pgfqpoint{3.048063in}{1.665872in}}%
\pgfpathlineto{\pgfqpoint{3.054053in}{1.716145in}}%
\pgfpathlineto{\pgfqpoint{3.058329in}{1.584919in}}%
\pgfpathlineto{\pgfqpoint{3.062609in}{1.538667in}}%
\pgfpathlineto{\pgfqpoint{3.064319in}{1.661554in}}%
\pgfpathlineto{\pgfqpoint{3.069452in}{1.569404in}}%
\pgfpathlineto{\pgfqpoint{3.076293in}{1.663654in}}%
\pgfpathlineto{\pgfqpoint{3.078004in}{1.572963in}}%
\pgfpathlineto{\pgfqpoint{3.083138in}{1.587893in}}%
\pgfpathlineto{\pgfqpoint{3.089122in}{1.509943in}}%
\pgfpathlineto{\pgfqpoint{3.092546in}{1.618396in}}%
\pgfpathlineto{\pgfqpoint{3.095968in}{1.544677in}}%
\pgfpathlineto{\pgfqpoint{3.101097in}{1.608862in}}%
\pgfpathlineto{\pgfqpoint{3.103662in}{1.521377in}}%
\pgfpathlineto{\pgfqpoint{3.109653in}{1.635602in}}%
\pgfpathlineto{\pgfqpoint{3.113076in}{1.541440in}}%
\pgfpathlineto{\pgfqpoint{3.118215in}{1.754755in}}%
\pgfpathlineto{\pgfqpoint{3.122495in}{1.625366in}}%
\pgfpathlineto{\pgfqpoint{3.124207in}{1.698125in}}%
\pgfpathlineto{\pgfqpoint{3.129335in}{1.650912in}}%
\pgfpathlineto{\pgfqpoint{3.135323in}{1.791556in}}%
\pgfpathlineto{\pgfqpoint{3.138744in}{1.664502in}}%
\pgfpathlineto{\pgfqpoint{3.143874in}{1.709263in}}%
\pgfpathlineto{\pgfqpoint{3.145581in}{1.587659in}}%
\pgfpathlineto{\pgfqpoint{3.149859in}{1.566022in}}%
\pgfpathlineto{\pgfqpoint{3.155843in}{1.541528in}}%
\pgfpathlineto{\pgfqpoint{3.159266in}{1.646624in}}%
\pgfpathlineto{\pgfqpoint{3.164395in}{1.533885in}}%
\pgfpathlineto{\pgfqpoint{3.168665in}{1.643825in}}%
\pgfpathlineto{\pgfqpoint{3.172941in}{1.553831in}}%
\pgfpathlineto{\pgfqpoint{3.178926in}{1.675815in}}%
\pgfpathlineto{\pgfqpoint{3.179782in}{1.551092in}}%
\pgfpathlineto{\pgfqpoint{3.186627in}{1.532194in}}%
\pgfpathlineto{\pgfqpoint{3.189194in}{1.619795in}}%
\pgfpathlineto{\pgfqpoint{3.196038in}{1.499766in}}%
\pgfpathlineto{\pgfqpoint{3.200317in}{1.646331in}}%
\pgfpathlineto{\pgfqpoint{3.204595in}{1.519043in}}%
\pgfpathlineto{\pgfqpoint{3.206306in}{1.654847in}}%
\pgfpathlineto{\pgfqpoint{3.209727in}{1.521929in}}%
\pgfpathlineto{\pgfqpoint{3.214860in}{1.640559in}}%
\pgfpathlineto{\pgfqpoint{3.218280in}{1.652980in}}%
\pgfpathlineto{\pgfqpoint{3.225977in}{1.532223in}}%
\pgfpathlineto{\pgfqpoint{3.230256in}{1.690249in}}%
\pgfpathlineto{\pgfqpoint{3.232821in}{1.584539in}}%
\pgfpathlineto{\pgfqpoint{3.235387in}{1.707513in}}%
\pgfpathlineto{\pgfqpoint{3.239662in}{1.655285in}}%
\pgfpathlineto{\pgfqpoint{3.247355in}{1.719236in}}%
\pgfpathlineto{\pgfqpoint{3.251630in}{1.647003in}}%
\pgfpathlineto{\pgfqpoint{3.255054in}{1.691882in}}%
\pgfpathlineto{\pgfqpoint{3.259333in}{1.701913in}}%
\pgfpathlineto{\pgfqpoint{3.264463in}{1.735218in}}%
\pgfpathlineto{\pgfqpoint{3.268739in}{1.542342in}}%
\pgfpathlineto{\pgfqpoint{3.269594in}{1.626268in}}%
\pgfpathlineto{\pgfqpoint{3.273868in}{1.549575in}}%
\pgfpathlineto{\pgfqpoint{3.281573in}{1.644172in}}%
\pgfpathlineto{\pgfqpoint{3.284995in}{1.577277in}}%
\pgfpathlineto{\pgfqpoint{3.286705in}{1.674909in}}%
\pgfpathlineto{\pgfqpoint{3.291835in}{1.540125in}}%
\pgfpathlineto{\pgfqpoint{3.296964in}{1.633443in}}%
\pgfpathlineto{\pgfqpoint{3.299528in}{1.585149in}}%
\pgfpathlineto{\pgfqpoint{3.306369in}{1.632859in}}%
\pgfpathlineto{\pgfqpoint{3.309787in}{1.521812in}}%
\pgfpathlineto{\pgfqpoint{3.312350in}{1.641140in}}%
\pgfpathlineto{\pgfqpoint{3.320047in}{1.567772in}}%
\pgfpathlineto{\pgfqpoint{3.323472in}{1.688324in}}%
\pgfpathlineto{\pgfqpoint{3.327748in}{1.566314in}}%
\pgfpathlineto{\pgfqpoint{3.330315in}{1.650240in}}%
\pgfpathlineto{\pgfqpoint{3.337159in}{1.644757in}}%
\pgfpathlineto{\pgfqpoint{3.339724in}{1.514988in}}%
\pgfpathlineto{\pgfqpoint{3.342293in}{1.608482in}}%
\pgfpathlineto{\pgfqpoint{3.348282in}{1.539193in}}%
\pgfpathlineto{\pgfqpoint{3.351703in}{1.615886in}}%
\pgfpathlineto{\pgfqpoint{3.357690in}{1.547417in}}%
\pgfpathlineto{\pgfqpoint{3.359402in}{1.642656in}}%
\pgfpathlineto{\pgfqpoint{3.364537in}{1.575965in}}%
\pgfpathlineto{\pgfqpoint{3.368813in}{1.669777in}}%
\pgfpathlineto{\pgfqpoint{3.372234in}{1.579319in}}%
\pgfpathlineto{\pgfqpoint{3.379076in}{1.509972in}}%
\pgfpathlineto{\pgfqpoint{3.380784in}{1.617344in}}%
\pgfpathlineto{\pgfqpoint{3.385917in}{1.571414in}}%
\pgfpathlineto{\pgfqpoint{3.391052in}{1.654613in}}%
\pgfpathlineto{\pgfqpoint{3.396180in}{1.556980in}}%
\pgfpathlineto{\pgfqpoint{3.400457in}{1.637757in}}%
\pgfpathlineto{\pgfqpoint{3.405589in}{1.592207in}}%
\pgfpathlineto{\pgfqpoint{3.406443in}{1.644348in}}%
\pgfpathlineto{\pgfqpoint{3.414139in}{1.562347in}}%
\pgfpathlineto{\pgfqpoint{3.418410in}{1.655866in}}%
\pgfpathlineto{\pgfqpoint{3.420976in}{1.581971in}}%
\pgfpathlineto{\pgfqpoint{3.425249in}{1.662544in}}%
\pgfpathlineto{\pgfqpoint{3.427816in}{1.568294in}}%
\pgfpathlineto{\pgfqpoint{3.432947in}{1.504021in}}%
\pgfpathlineto{\pgfqpoint{3.437223in}{1.637378in}}%
\pgfpathlineto{\pgfqpoint{3.442356in}{1.551439in}}%
\pgfpathlineto{\pgfqpoint{3.445779in}{1.613377in}}%
\pgfpathlineto{\pgfqpoint{3.450907in}{1.684415in}}%
\pgfpathlineto{\pgfqpoint{3.456036in}{1.559198in}}%
\pgfpathlineto{\pgfqpoint{3.458605in}{1.639273in}}%
\pgfpathlineto{\pgfqpoint{3.463737in}{1.521929in}}%
\pgfpathlineto{\pgfqpoint{3.466302in}{1.648549in}}%
\pgfpathlineto{\pgfqpoint{3.472285in}{1.662953in}}%
\pgfpathlineto{\pgfqpoint{3.475705in}{1.560831in}}%
\pgfpathlineto{\pgfqpoint{3.479984in}{1.540651in}}%
\pgfpathlineto{\pgfqpoint{3.484262in}{1.609676in}}%
\pgfpathlineto{\pgfqpoint{3.491104in}{1.676601in}}%
\pgfpathlineto{\pgfqpoint{3.492814in}{1.570278in}}%
\pgfpathlineto{\pgfqpoint{3.497093in}{1.535285in}}%
\pgfpathlineto{\pgfqpoint{3.500515in}{1.625278in}}%
\pgfpathlineto{\pgfqpoint{3.504790in}{1.574712in}}%
\pgfpathlineto{\pgfqpoint{3.512483in}{1.545521in}}%
\pgfpathlineto{\pgfqpoint{3.515905in}{1.714454in}}%
\pgfpathlineto{\pgfqpoint{3.517615in}{1.584656in}}%
\pgfpathlineto{\pgfqpoint{3.524462in}{1.674095in}}%
\pgfpathlineto{\pgfqpoint{3.527029in}{1.523796in}}%
\pgfpathlineto{\pgfqpoint{3.533013in}{1.699582in}}%
\pgfpathlineto{\pgfqpoint{3.534724in}{1.637410in}}%
\pgfpathlineto{\pgfqpoint{3.539857in}{1.759595in}}%
\pgfpathlineto{\pgfqpoint{3.543277in}{1.658814in}}%
\pgfpathlineto{\pgfqpoint{3.547556in}{1.728975in}}%
\pgfpathlineto{\pgfqpoint{3.554393in}{1.612036in}}%
\pgfpathlineto{\pgfqpoint{3.556960in}{1.793712in}}%
\pgfpathlineto{\pgfqpoint{3.560375in}{1.677357in}}%
\pgfpathlineto{\pgfqpoint{3.565506in}{1.727167in}}%
\pgfpathlineto{\pgfqpoint{3.574058in}{1.638605in}}%
\pgfpathlineto{\pgfqpoint{3.578334in}{1.726381in}}%
\pgfpathlineto{\pgfqpoint{3.583469in}{1.678672in}}%
\pgfpathlineto{\pgfqpoint{3.589452in}{1.745772in}}%
\pgfpathlineto{\pgfqpoint{3.590304in}{1.682084in}}%
\pgfpathlineto{\pgfqpoint{3.596294in}{1.787005in}}%
\pgfpathlineto{\pgfqpoint{3.599713in}{1.615828in}}%
\pgfpathlineto{\pgfqpoint{3.604849in}{1.739123in}}%
\pgfpathlineto{\pgfqpoint{3.609981in}{1.762452in}}%
\pgfpathlineto{\pgfqpoint{3.611692in}{1.669981in}}%
\pgfpathlineto{\pgfqpoint{3.618535in}{1.656626in}}%
\pgfpathlineto{\pgfqpoint{3.620247in}{1.766536in}}%
\pgfpathlineto{\pgfqpoint{3.624526in}{1.656655in}}%
\pgfpathlineto{\pgfqpoint{3.631368in}{1.744607in}}%
\pgfpathlineto{\pgfqpoint{3.633075in}{1.663946in}}%
\pgfpathlineto{\pgfqpoint{3.639063in}{1.637001in}}%
\pgfpathlineto{\pgfqpoint{3.645057in}{1.731017in}}%
\pgfpathlineto{\pgfqpoint{3.647625in}{1.638342in}}%
\pgfpathlineto{\pgfqpoint{3.651902in}{1.715795in}}%
\pgfpathlineto{\pgfqpoint{3.655322in}{1.656772in}}%
\pgfpathlineto{\pgfqpoint{3.662162in}{1.789573in}}%
\pgfpathlineto{\pgfqpoint{3.666437in}{1.655022in}}%
\pgfpathlineto{\pgfqpoint{3.667291in}{1.730491in}}%
\pgfpathlineto{\pgfqpoint{3.674993in}{1.633385in}}%
\pgfpathlineto{\pgfqpoint{3.676704in}{1.750729in}}%
\pgfpathlineto{\pgfqpoint{3.680124in}{1.559081in}}%
\pgfpathlineto{\pgfqpoint{3.685255in}{1.701040in}}%
\pgfpathlineto{\pgfqpoint{3.689529in}{1.634959in}}%
\pgfpathlineto{\pgfqpoint{3.694663in}{1.674212in}}%
\pgfpathlineto{\pgfqpoint{3.699794in}{1.675435in}}%
\pgfpathlineto{\pgfqpoint{3.701506in}{1.553017in}}%
\pgfpathlineto{\pgfqpoint{3.708349in}{1.651581in}}%
\pgfpathlineto{\pgfqpoint{3.710060in}{1.550945in}}%
\pgfpathlineto{\pgfqpoint{3.714341in}{1.638400in}}%
\pgfpathlineto{\pgfqpoint{3.721187in}{1.667793in}}%
\pgfpathlineto{\pgfqpoint{3.723756in}{1.552082in}}%
\pgfpathlineto{\pgfqpoint{3.729741in}{1.636358in}}%
\pgfpathlineto{\pgfqpoint{3.731453in}{1.564999in}}%
\pgfpathlineto{\pgfqpoint{3.735725in}{1.659161in}}%
\pgfpathlineto{\pgfqpoint{3.740853in}{1.685028in}}%
\pgfpathlineto{\pgfqpoint{3.746839in}{1.577452in}}%
\pgfpathlineto{\pgfqpoint{3.751968in}{1.677708in}}%
\pgfpathlineto{\pgfqpoint{3.752824in}{1.592295in}}%
\pgfpathlineto{\pgfqpoint{3.757956in}{1.511196in}}%
\pgfpathlineto{\pgfqpoint{3.764798in}{1.566778in}}%
\pgfpathlineto{\pgfqpoint{3.766511in}{1.685584in}}%
\pgfpathlineto{\pgfqpoint{3.769934in}{1.625162in}}%
\pgfpathlineto{\pgfqpoint{3.775066in}{1.733118in}}%
\pgfpathlineto{\pgfqpoint{3.778487in}{1.628632in}}%
\pgfpathlineto{\pgfqpoint{3.787040in}{1.740464in}}%
\pgfpathlineto{\pgfqpoint{3.791320in}{1.603638in}}%
\pgfpathlineto{\pgfqpoint{3.799020in}{1.741166in}}%
\pgfpathlineto{\pgfqpoint{3.802439in}{1.640092in}}%
\pgfpathlineto{\pgfqpoint{3.807568in}{1.715561in}}%
\pgfpathlineto{\pgfqpoint{3.809278in}{1.649743in}}%
\pgfpathlineto{\pgfqpoint{3.816117in}{1.714830in}}%
\pgfpathlineto{\pgfqpoint{3.820392in}{1.558146in}}%
\pgfpathlineto{\pgfqpoint{3.823810in}{1.558672in}}%
\pgfpathlineto{\pgfqpoint{3.828940in}{1.690190in}}%
\pgfpathlineto{\pgfqpoint{3.832358in}{1.735770in}}%
\pgfpathlineto{\pgfqpoint{3.837489in}{1.612942in}}%
\pgfpathlineto{\pgfqpoint{3.838344in}{1.685525in}}%
\pgfpathlineto{\pgfqpoint{3.844324in}{1.558876in}}%
\pgfpathlineto{\pgfqpoint{3.850314in}{1.651055in}}%
\pgfpathlineto{\pgfqpoint{3.853739in}{1.562259in}}%
\pgfpathlineto{\pgfqpoint{3.857163in}{1.641023in}}%
\pgfpathlineto{\pgfqpoint{3.860589in}{1.573894in}}%
\pgfpathlineto{\pgfqpoint{3.867436in}{1.661495in}}%
\pgfpathlineto{\pgfqpoint{3.869999in}{1.552140in}}%
\pgfpathlineto{\pgfqpoint{3.872563in}{1.611773in}}%
\pgfpathlineto{\pgfqpoint{3.877696in}{1.662719in}}%
\pgfpathlineto{\pgfqpoint{3.881975in}{1.570511in}}%
\pgfpathlineto{\pgfqpoint{3.887961in}{1.558613in}}%
\pgfpathlineto{\pgfqpoint{3.889671in}{1.613318in}}%
\pgfpathlineto{\pgfqpoint{3.896512in}{1.584740in}}%
\pgfpathlineto{\pgfqpoint{3.899075in}{1.679399in}}%
\pgfpathlineto{\pgfqpoint{3.902494in}{1.582990in}}%
\pgfpathlineto{\pgfqpoint{3.910196in}{1.517465in}}%
\pgfpathlineto{\pgfqpoint{3.911052in}{1.606377in}}%
\pgfpathlineto{\pgfqpoint{3.918748in}{1.665634in}}%
\pgfpathlineto{\pgfqpoint{3.922167in}{1.628745in}}%
\pgfpathlineto{\pgfqpoint{3.925587in}{1.686165in}}%
\pgfpathlineto{\pgfqpoint{3.931576in}{1.623613in}}%
\pgfpathlineto{\pgfqpoint{3.933287in}{1.696196in}}%
\pgfpathlineto{\pgfqpoint{3.938418in}{1.668462in}}%
\pgfpathlineto{\pgfqpoint{3.940985in}{1.571268in}}%
\pgfpathlineto{\pgfqpoint{3.947833in}{1.541232in}}%
\pgfpathlineto{\pgfqpoint{3.952966in}{1.614137in}}%
\pgfpathlineto{\pgfqpoint{3.955532in}{1.519185in}}%
\pgfpathlineto{\pgfqpoint{3.958954in}{1.664820in}}%
\pgfpathlineto{\pgfqpoint{3.962375in}{1.573777in}}%
\pgfpathlineto{\pgfqpoint{3.970067in}{1.554588in}}%
\pgfpathlineto{\pgfqpoint{3.970924in}{1.617052in}}%
\pgfpathlineto{\pgfqpoint{3.976053in}{1.542372in}}%
\pgfpathlineto{\pgfqpoint{3.979475in}{1.659161in}}%
\pgfpathlineto{\pgfqpoint{3.986324in}{1.660560in}}%
\pgfpathlineto{\pgfqpoint{3.988889in}{1.569752in}}%
\pgfpathlineto{\pgfqpoint{3.992312in}{1.621395in}}%
\pgfpathlineto{\pgfqpoint{3.997445in}{1.547676in}}%
\pgfpathlineto{\pgfqpoint{4.000867in}{1.667735in}}%
\pgfpathlineto{\pgfqpoint{4.005147in}{1.593080in}}%
\pgfpathlineto{\pgfqpoint{4.011132in}{1.640819in}}%
\pgfpathlineto{\pgfqpoint{4.017123in}{1.559198in}}%
\pgfpathlineto{\pgfqpoint{4.020541in}{1.713519in}}%
\pgfpathlineto{\pgfqpoint{4.023107in}{1.577569in}}%
\pgfpathlineto{\pgfqpoint{4.027385in}{1.653681in}}%
\pgfpathlineto{\pgfqpoint{4.030804in}{1.596174in}}%
\pgfpathlineto{\pgfqpoint{4.035079in}{1.562171in}}%
\pgfpathlineto{\pgfqpoint{4.040214in}{1.590282in}}%
\pgfpathlineto{\pgfqpoint{4.043636in}{1.746006in}}%
\pgfpathlineto{\pgfqpoint{4.049623in}{1.648899in}}%
\pgfpathlineto{\pgfqpoint{4.053904in}{1.744168in}}%
\pgfpathlineto{\pgfqpoint{4.058181in}{1.602881in}}%
\pgfpathlineto{\pgfqpoint{4.062458in}{1.671410in}}%
\pgfpathlineto{\pgfqpoint{4.067590in}{1.526535in}}%
\pgfpathlineto{\pgfqpoint{4.069301in}{1.591739in}}%
\pgfpathlineto{\pgfqpoint{4.076144in}{1.535051in}}%
\pgfpathlineto{\pgfqpoint{4.080418in}{1.616993in}}%
\pgfpathlineto{\pgfqpoint{4.085551in}{1.552721in}}%
\pgfpathlineto{\pgfqpoint{4.087264in}{1.728274in}}%
\pgfpathlineto{\pgfqpoint{4.091542in}{1.646360in}}%
\pgfpathlineto{\pgfqpoint{4.097525in}{1.776217in}}%
\pgfpathlineto{\pgfqpoint{4.099237in}{1.674909in}}%
\pgfpathlineto{\pgfqpoint{4.104369in}{1.718188in}}%
\pgfpathlineto{\pgfqpoint{4.110358in}{1.639858in}}%
\pgfpathlineto{\pgfqpoint{4.114636in}{1.697219in}}%
\pgfpathlineto{\pgfqpoint{4.119765in}{1.657674in}}%
\pgfpathlineto{\pgfqpoint{4.120620in}{1.730082in}}%
\pgfpathlineto{\pgfqpoint{4.127462in}{1.640965in}}%
\pgfpathlineto{\pgfqpoint{4.132595in}{1.601654in}}%
\pgfpathlineto{\pgfqpoint{4.135165in}{1.683074in}}%
\pgfpathlineto{\pgfqpoint{4.138587in}{1.610841in}}%
\pgfpathlineto{\pgfqpoint{4.143724in}{1.701504in}}%
\pgfpathlineto{\pgfqpoint{4.147142in}{1.642832in}}%
\pgfpathlineto{\pgfqpoint{4.149711in}{1.691765in}}%
\pgfpathlineto{\pgfqpoint{4.153137in}{1.632157in}}%
\pgfpathlineto{\pgfqpoint{4.157412in}{1.639040in}}%
\pgfpathlineto{\pgfqpoint{4.159124in}{1.729205in}}%
\pgfpathlineto{\pgfqpoint{4.165117in}{1.738597in}}%
\pgfpathlineto{\pgfqpoint{4.166829in}{1.644114in}}%
\pgfpathlineto{\pgfqpoint{4.170249in}{1.725066in}}%
\pgfpathlineto{\pgfqpoint{4.172814in}{1.684765in}}%
\pgfpathlineto{\pgfqpoint{4.177947in}{1.656421in}}%
\pgfpathlineto{\pgfqpoint{4.180510in}{1.745831in}}%
\pgfpathlineto{\pgfqpoint{4.184787in}{1.651347in}}%
\pgfpathlineto{\pgfqpoint{4.186496in}{1.689547in}}%
\pgfpathlineto{\pgfqpoint{4.190773in}{1.632976in}}%
\pgfpathlineto{\pgfqpoint{4.194194in}{1.722093in}}%
\pgfpathlineto{\pgfqpoint{4.197616in}{1.677941in}}%
\pgfpathlineto{\pgfqpoint{4.205311in}{1.739299in}}%
\pgfpathlineto{\pgfqpoint{4.208735in}{1.737520in}}%
\pgfpathlineto{\pgfqpoint{4.213009in}{1.622301in}}%
\pgfpathlineto{\pgfqpoint{4.214719in}{1.639741in}}%
\pgfpathlineto{\pgfqpoint{4.218141in}{1.614546in}}%
\pgfpathlineto{\pgfqpoint{4.221561in}{1.715093in}}%
\pgfpathlineto{\pgfqpoint{4.225839in}{1.656246in}}%
\pgfpathlineto{\pgfqpoint{4.227550in}{1.710604in}}%
\pgfpathlineto{\pgfqpoint{4.232681in}{1.634930in}}%
\pgfpathlineto{\pgfqpoint{4.234393in}{1.733235in}}%
\pgfpathlineto{\pgfqpoint{4.238671in}{1.731602in}}%
\pgfpathlineto{\pgfqpoint{4.244659in}{1.518079in}}%
\pgfpathlineto{\pgfqpoint{4.248080in}{1.675319in}}%
\pgfpathlineto{\pgfqpoint{4.252359in}{1.577218in}}%
\pgfpathlineto{\pgfqpoint{4.254925in}{1.676718in}}%
\pgfpathlineto{\pgfqpoint{4.260913in}{1.589292in}}%
\pgfpathlineto{\pgfqpoint{4.261769in}{1.642832in}}%
\pgfpathlineto{\pgfqpoint{4.265191in}{1.603521in}}%
\pgfpathlineto{\pgfqpoint{4.270325in}{1.592733in}}%
\pgfpathlineto{\pgfqpoint{4.274604in}{1.659629in}}%
\pgfpathlineto{\pgfqpoint{4.278880in}{1.570278in}}%
\pgfpathlineto{\pgfqpoint{4.282299in}{1.671293in}}%
\pgfpathlineto{\pgfqpoint{4.286570in}{1.594366in}}%
\pgfpathlineto{\pgfqpoint{4.290850in}{1.643065in}}%
\pgfpathlineto{\pgfqpoint{4.293417in}{1.564038in}}%
\pgfpathlineto{\pgfqpoint{4.297691in}{1.634316in}}%
\pgfpathlineto{\pgfqpoint{4.301964in}{1.564911in}}%
\pgfpathlineto{\pgfqpoint{4.303673in}{1.641257in}}%
\pgfpathlineto{\pgfqpoint{4.307096in}{1.726641in}}%
\pgfpathlineto{\pgfqpoint{4.311374in}{1.575352in}}%
\pgfpathlineto{\pgfqpoint{4.313941in}{1.672107in}}%
\pgfpathlineto{\pgfqpoint{4.319069in}{1.674325in}}%
\pgfpathlineto{\pgfqpoint{4.319924in}{1.593489in}}%
\pgfpathlineto{\pgfqpoint{4.325903in}{1.571560in}}%
\pgfpathlineto{\pgfqpoint{4.326759in}{1.621804in}}%
\pgfpathlineto{\pgfqpoint{4.331893in}{1.531084in}}%
\pgfpathlineto{\pgfqpoint{4.336171in}{1.524610in}}%
\pgfpathlineto{\pgfqpoint{4.337026in}{1.643938in}}%
\pgfpathlineto{\pgfqpoint{4.340445in}{1.576575in}}%
\pgfpathlineto{\pgfqpoint{4.344726in}{1.642013in}}%
\pgfpathlineto{\pgfqpoint{4.348149in}{1.589931in}}%
\pgfpathlineto{\pgfqpoint{4.353279in}{1.660969in}}%
\pgfpathlineto{\pgfqpoint{4.354134in}{1.547764in}}%
\pgfpathlineto{\pgfqpoint{4.358409in}{1.614601in}}%
\pgfpathlineto{\pgfqpoint{4.363544in}{1.578384in}}%
\pgfpathlineto{\pgfqpoint{4.366112in}{1.667442in}}%
\pgfpathlineto{\pgfqpoint{4.369535in}{1.579841in}}%
\pgfpathlineto{\pgfqpoint{4.371246in}{1.669338in}}%
\pgfpathlineto{\pgfqpoint{4.377235in}{1.578676in}}%
\pgfpathlineto{\pgfqpoint{4.379798in}{1.623817in}}%
\pgfpathlineto{\pgfqpoint{4.383215in}{1.504635in}}%
\pgfpathlineto{\pgfqpoint{4.386637in}{1.655837in}}%
\pgfpathlineto{\pgfqpoint{4.390915in}{1.665050in}}%
\pgfpathlineto{\pgfqpoint{4.392627in}{1.551526in}}%
\pgfpathlineto{\pgfqpoint{4.395194in}{1.631748in}}%
\pgfpathlineto{\pgfqpoint{4.399466in}{1.576049in}}%
\pgfpathlineto{\pgfqpoint{4.404602in}{1.608712in}}%
\pgfpathlineto{\pgfqpoint{4.408022in}{1.529041in}}%
\pgfpathlineto{\pgfqpoint{4.409735in}{1.608653in}}%
\pgfpathlineto{\pgfqpoint{4.416578in}{1.546481in}}%
\pgfpathlineto{\pgfqpoint{4.419143in}{1.606030in}}%
\pgfpathlineto{\pgfqpoint{4.428550in}{1.545955in}}%
\pgfpathlineto{\pgfqpoint{4.429406in}{1.630407in}}%
\pgfpathlineto{\pgfqpoint{4.432826in}{1.678526in}}%
\pgfpathlineto{\pgfqpoint{4.436247in}{1.553831in}}%
\pgfpathlineto{\pgfqpoint{4.440528in}{1.634433in}}%
\pgfpathlineto{\pgfqpoint{4.444808in}{1.572378in}}%
\pgfpathlineto{\pgfqpoint{4.448226in}{1.690132in}}%
\pgfpathlineto{\pgfqpoint{4.451649in}{1.591389in}}%
\pgfpathlineto{\pgfqpoint{4.453359in}{1.648808in}}%
\pgfpathlineto{\pgfqpoint{4.457633in}{1.583516in}}%
\pgfpathlineto{\pgfqpoint{4.461057in}{1.635306in}}%
\pgfpathlineto{\pgfqpoint{4.464478in}{1.667092in}}%
\pgfpathlineto{\pgfqpoint{4.468754in}{1.640380in}}%
\pgfpathlineto{\pgfqpoint{4.473032in}{1.485475in}}%
\pgfpathlineto{\pgfqpoint{4.473885in}{1.589931in}}%
\pgfpathlineto{\pgfqpoint{4.479871in}{1.507638in}}%
\pgfpathlineto{\pgfqpoint{4.480727in}{1.647321in}}%
\pgfpathlineto{\pgfqpoint{4.484146in}{1.578238in}}%
\pgfpathlineto{\pgfqpoint{4.490131in}{1.659278in}}%
\pgfpathlineto{\pgfqpoint{4.490987in}{1.568703in}}%
\pgfpathlineto{\pgfqpoint{4.496974in}{1.593255in}}%
\pgfpathlineto{\pgfqpoint{4.498682in}{1.656651in}}%
\pgfpathlineto{\pgfqpoint{4.502953in}{1.676919in}}%
\pgfpathlineto{\pgfqpoint{4.507228in}{1.587889in}}%
\pgfpathlineto{\pgfqpoint{4.508938in}{1.665225in}}%
\pgfpathlineto{\pgfqpoint{4.513210in}{1.569459in}}%
\pgfpathlineto{\pgfqpoint{4.515776in}{1.620727in}}%
\pgfpathlineto{\pgfqpoint{4.519199in}{1.575322in}}%
\pgfpathlineto{\pgfqpoint{4.524328in}{1.689606in}}%
\pgfpathlineto{\pgfqpoint{4.525183in}{1.604923in}}%
\pgfpathlineto{\pgfqpoint{4.531176in}{1.561675in}}%
\pgfpathlineto{\pgfqpoint{4.532031in}{1.656651in}}%
\pgfpathlineto{\pgfqpoint{4.535450in}{1.727748in}}%
\pgfpathlineto{\pgfqpoint{4.538868in}{1.555230in}}%
\pgfpathlineto{\pgfqpoint{4.544856in}{1.632625in}}%
\pgfpathlineto{\pgfqpoint{4.548281in}{1.573427in}}%
\pgfpathlineto{\pgfqpoint{4.549136in}{1.632333in}}%
\pgfpathlineto{\pgfqpoint{4.552557in}{1.512420in}}%
\pgfpathlineto{\pgfqpoint{4.558543in}{1.642948in}}%
\pgfpathlineto{\pgfqpoint{4.559399in}{1.546193in}}%
\pgfpathlineto{\pgfqpoint{4.563676in}{1.516621in}}%
\pgfpathlineto{\pgfqpoint{4.567950in}{1.637641in}}%
\pgfpathlineto{\pgfqpoint{4.569658in}{1.533301in}}%
\pgfpathlineto{\pgfqpoint{4.573939in}{1.523328in}}%
\pgfpathlineto{\pgfqpoint{4.576507in}{1.633092in}}%
\pgfpathlineto{\pgfqpoint{4.580783in}{1.585588in}}%
\pgfpathlineto{\pgfqpoint{4.583350in}{1.638108in}}%
\pgfpathlineto{\pgfqpoint{4.589338in}{1.564272in}}%
\pgfpathlineto{\pgfqpoint{4.591044in}{1.639858in}}%
\pgfpathlineto{\pgfqpoint{4.595324in}{1.509855in}}%
\pgfpathlineto{\pgfqpoint{4.598746in}{1.658229in}}%
\pgfpathlineto{\pgfqpoint{4.602166in}{1.517089in}}%
\pgfpathlineto{\pgfqpoint{4.606443in}{1.599791in}}%
\pgfpathlineto{\pgfqpoint{4.609866in}{1.554007in}}%
\pgfpathlineto{\pgfqpoint{4.610723in}{1.591915in}}%
\pgfpathlineto{\pgfqpoint{4.615000in}{1.552140in}}%
\pgfpathlineto{\pgfqpoint{4.617565in}{1.636183in}}%
\pgfpathlineto{\pgfqpoint{4.622695in}{1.523328in}}%
\pgfpathlineto{\pgfqpoint{4.625260in}{1.588795in}}%
\pgfpathlineto{\pgfqpoint{4.630391in}{1.556863in}}%
\pgfpathlineto{\pgfqpoint{4.632102in}{1.600635in}}%
\pgfpathlineto{\pgfqpoint{4.634669in}{1.559227in}}%
\pgfpathlineto{\pgfqpoint{4.638945in}{1.560597in}}%
\pgfpathlineto{\pgfqpoint{4.644077in}{1.643475in}}%
\pgfpathlineto{\pgfqpoint{4.647497in}{1.530908in}}%
\pgfpathlineto{\pgfqpoint{4.648350in}{1.620493in}}%
\pgfpathlineto{\pgfqpoint{4.654343in}{1.609004in}}%
\pgfpathlineto{\pgfqpoint{4.656051in}{1.536333in}}%
\pgfpathlineto{\pgfqpoint{4.659473in}{1.669718in}}%
\pgfpathlineto{\pgfqpoint{4.664605in}{1.714626in}}%
\pgfpathlineto{\pgfqpoint{4.665461in}{1.615185in}}%
\pgfpathlineto{\pgfqpoint{4.668881in}{1.674091in}}%
\pgfpathlineto{\pgfqpoint{4.673161in}{1.622184in}}%
\pgfpathlineto{\pgfqpoint{4.677437in}{1.731306in}}%
\pgfpathlineto{\pgfqpoint{4.679148in}{1.637494in}}%
\pgfpathlineto{\pgfqpoint{4.682567in}{1.688031in}}%
\pgfpathlineto{\pgfqpoint{4.685989in}{1.614020in}}%
\pgfpathlineto{\pgfqpoint{4.690266in}{1.707805in}}%
\pgfpathlineto{\pgfqpoint{4.694542in}{1.749330in}}%
\pgfpathlineto{\pgfqpoint{4.697112in}{1.626268in}}%
\pgfpathlineto{\pgfqpoint{4.700535in}{1.590023in}}%
\pgfpathlineto{\pgfqpoint{4.704809in}{1.733523in}}%
\pgfpathlineto{\pgfqpoint{4.706522in}{1.748804in}}%
\pgfpathlineto{\pgfqpoint{4.709945in}{1.652512in}}%
\pgfpathlineto{\pgfqpoint{4.713368in}{1.804035in}}%
\pgfpathlineto{\pgfqpoint{4.716788in}{1.592119in}}%
\pgfpathlineto{\pgfqpoint{4.721066in}{1.647906in}}%
\pgfpathlineto{\pgfqpoint{4.723635in}{1.548114in}}%
\pgfpathlineto{\pgfqpoint{4.727062in}{1.597223in}}%
\pgfpathlineto{\pgfqpoint{4.731339in}{1.549776in}}%
\pgfpathlineto{\pgfqpoint{4.735622in}{1.625713in}}%
\pgfpathlineto{\pgfqpoint{4.737335in}{1.551322in}}%
\pgfpathlineto{\pgfqpoint{4.741616in}{1.659161in}}%
\pgfpathlineto{\pgfqpoint{4.745895in}{1.518546in}}%
\pgfpathlineto{\pgfqpoint{4.751882in}{1.772659in}}%
\pgfpathlineto{\pgfqpoint{4.754446in}{1.633268in}}%
\pgfpathlineto{\pgfqpoint{4.759573in}{1.523386in}}%
\pgfpathlineto{\pgfqpoint{4.761282in}{1.639273in}}%
\pgfpathlineto{\pgfqpoint{4.764703in}{1.715561in}}%
\pgfpathlineto{\pgfqpoint{4.769831in}{1.732066in}}%
\pgfpathlineto{\pgfqpoint{4.771544in}{1.605387in}}%
\pgfpathlineto{\pgfqpoint{4.774966in}{1.678584in}}%
\pgfpathlineto{\pgfqpoint{4.780961in}{1.572612in}}%
\pgfpathlineto{\pgfqpoint{4.784387in}{1.722210in}}%
\pgfpathlineto{\pgfqpoint{4.785242in}{1.523036in}}%
\pgfpathlineto{\pgfqpoint{4.789519in}{1.602122in}}%
\pgfpathlineto{\pgfqpoint{4.792084in}{1.499999in}}%
\pgfpathlineto{\pgfqpoint{4.795506in}{1.542225in}}%
\pgfpathlineto{\pgfqpoint{4.795506in}{1.542225in}}%
\pgfusepath{stroke}%
\end{pgfscope}%
\begin{pgfscope}%
\pgfsetrectcap%
\pgfsetmiterjoin%
\pgfsetlinewidth{0.803000pt}%
\definecolor{currentstroke}{rgb}{0.000000,0.000000,0.000000}%
\pgfsetstrokecolor{currentstroke}%
\pgfsetdash{}{0pt}%
\pgfpathmoveto{\pgfqpoint{0.484581in}{1.437021in}}%
\pgfpathlineto{\pgfqpoint{0.484581in}{2.011643in}}%
\pgfusepath{stroke}%
\end{pgfscope}%
\begin{pgfscope}%
\pgfsetrectcap%
\pgfsetmiterjoin%
\pgfsetlinewidth{0.803000pt}%
\definecolor{currentstroke}{rgb}{0.000000,0.000000,0.000000}%
\pgfsetstrokecolor{currentstroke}%
\pgfsetdash{}{0pt}%
\pgfpathmoveto{\pgfqpoint{5.000788in}{1.437021in}}%
\pgfpathlineto{\pgfqpoint{5.000788in}{2.011643in}}%
\pgfusepath{stroke}%
\end{pgfscope}%
\begin{pgfscope}%
\pgfsetrectcap%
\pgfsetmiterjoin%
\pgfsetlinewidth{0.803000pt}%
\definecolor{currentstroke}{rgb}{0.000000,0.000000,0.000000}%
\pgfsetstrokecolor{currentstroke}%
\pgfsetdash{}{0pt}%
\pgfpathmoveto{\pgfqpoint{0.484581in}{1.437021in}}%
\pgfpathlineto{\pgfqpoint{5.000788in}{1.437021in}}%
\pgfusepath{stroke}%
\end{pgfscope}%
\begin{pgfscope}%
\pgfsetrectcap%
\pgfsetmiterjoin%
\pgfsetlinewidth{0.803000pt}%
\definecolor{currentstroke}{rgb}{0.000000,0.000000,0.000000}%
\pgfsetstrokecolor{currentstroke}%
\pgfsetdash{}{0pt}%
\pgfpathmoveto{\pgfqpoint{0.484581in}{2.011643in}}%
\pgfpathlineto{\pgfqpoint{5.000788in}{2.011643in}}%
\pgfusepath{stroke}%
\end{pgfscope}%
\begin{pgfscope}%
\pgfsetbuttcap%
\pgfsetmiterjoin%
\definecolor{currentfill}{rgb}{1.000000,1.000000,1.000000}%
\pgfsetfillcolor{currentfill}%
\pgfsetlinewidth{0.000000pt}%
\definecolor{currentstroke}{rgb}{0.000000,0.000000,0.000000}%
\pgfsetstrokecolor{currentstroke}%
\pgfsetstrokeopacity{0.000000}%
\pgfsetdash{}{0pt}%
\pgfpathmoveto{\pgfqpoint{0.484581in}{0.539544in}}%
\pgfpathlineto{\pgfqpoint{5.000788in}{0.539544in}}%
\pgfpathlineto{\pgfqpoint{5.000788in}{1.114166in}}%
\pgfpathlineto{\pgfqpoint{0.484581in}{1.114166in}}%
\pgfpathlineto{\pgfqpoint{0.484581in}{0.539544in}}%
\pgfpathclose%
\pgfusepath{fill}%
\end{pgfscope}%
\begin{pgfscope}%
\pgfsetbuttcap%
\pgfsetroundjoin%
\definecolor{currentfill}{rgb}{0.000000,0.000000,0.000000}%
\pgfsetfillcolor{currentfill}%
\pgfsetlinewidth{0.803000pt}%
\definecolor{currentstroke}{rgb}{0.000000,0.000000,0.000000}%
\pgfsetstrokecolor{currentstroke}%
\pgfsetdash{}{0pt}%
\pgfsys@defobject{currentmarker}{\pgfqpoint{0.000000in}{-0.048611in}}{\pgfqpoint{0.000000in}{0.000000in}}{%
\pgfpathmoveto{\pgfqpoint{0.000000in}{0.000000in}}%
\pgfpathlineto{\pgfqpoint{0.000000in}{-0.048611in}}%
\pgfusepath{stroke,fill}%
}%
\begin{pgfscope}%
\pgfsys@transformshift{0.689546in}{0.539544in}%
\pgfsys@useobject{currentmarker}{}%
\end{pgfscope}%
\end{pgfscope}%
\begin{pgfscope}%
\definecolor{textcolor}{rgb}{0.000000,0.000000,0.000000}%
\pgfsetstrokecolor{textcolor}%
\pgfsetfillcolor{textcolor}%
\pgftext[x=0.689546in,y=0.442322in,,top]{\color{textcolor}\rmfamily\fontsize{8.000000}{9.600000}\selectfont \(\displaystyle {06{:}00}\)}%
\end{pgfscope}%
\begin{pgfscope}%
\pgfsetbuttcap%
\pgfsetroundjoin%
\definecolor{currentfill}{rgb}{0.000000,0.000000,0.000000}%
\pgfsetfillcolor{currentfill}%
\pgfsetlinewidth{0.803000pt}%
\definecolor{currentstroke}{rgb}{0.000000,0.000000,0.000000}%
\pgfsetstrokecolor{currentstroke}%
\pgfsetdash{}{0pt}%
\pgfsys@defobject{currentmarker}{\pgfqpoint{0.000000in}{-0.048611in}}{\pgfqpoint{0.000000in}{0.000000in}}{%
\pgfpathmoveto{\pgfqpoint{0.000000in}{0.000000in}}%
\pgfpathlineto{\pgfqpoint{0.000000in}{-0.048611in}}%
\pgfusepath{stroke,fill}%
}%
\begin{pgfscope}%
\pgfsys@transformshift{1.202878in}{0.539544in}%
\pgfsys@useobject{currentmarker}{}%
\end{pgfscope}%
\end{pgfscope}%
\begin{pgfscope}%
\definecolor{textcolor}{rgb}{0.000000,0.000000,0.000000}%
\pgfsetstrokecolor{textcolor}%
\pgfsetfillcolor{textcolor}%
\pgftext[x=1.202878in,y=0.442322in,,top]{\color{textcolor}\rmfamily\fontsize{8.000000}{9.600000}\selectfont \(\displaystyle {09{:}00}\)}%
\end{pgfscope}%
\begin{pgfscope}%
\pgfsetbuttcap%
\pgfsetroundjoin%
\definecolor{currentfill}{rgb}{0.000000,0.000000,0.000000}%
\pgfsetfillcolor{currentfill}%
\pgfsetlinewidth{0.803000pt}%
\definecolor{currentstroke}{rgb}{0.000000,0.000000,0.000000}%
\pgfsetstrokecolor{currentstroke}%
\pgfsetdash{}{0pt}%
\pgfsys@defobject{currentmarker}{\pgfqpoint{0.000000in}{-0.048611in}}{\pgfqpoint{0.000000in}{0.000000in}}{%
\pgfpathmoveto{\pgfqpoint{0.000000in}{0.000000in}}%
\pgfpathlineto{\pgfqpoint{0.000000in}{-0.048611in}}%
\pgfusepath{stroke,fill}%
}%
\begin{pgfscope}%
\pgfsys@transformshift{1.716211in}{0.539544in}%
\pgfsys@useobject{currentmarker}{}%
\end{pgfscope}%
\end{pgfscope}%
\begin{pgfscope}%
\definecolor{textcolor}{rgb}{0.000000,0.000000,0.000000}%
\pgfsetstrokecolor{textcolor}%
\pgfsetfillcolor{textcolor}%
\pgftext[x=1.716211in,y=0.442322in,,top]{\color{textcolor}\rmfamily\fontsize{8.000000}{9.600000}\selectfont \(\displaystyle {12{:}00}\)}%
\end{pgfscope}%
\begin{pgfscope}%
\pgfsetbuttcap%
\pgfsetroundjoin%
\definecolor{currentfill}{rgb}{0.000000,0.000000,0.000000}%
\pgfsetfillcolor{currentfill}%
\pgfsetlinewidth{0.803000pt}%
\definecolor{currentstroke}{rgb}{0.000000,0.000000,0.000000}%
\pgfsetstrokecolor{currentstroke}%
\pgfsetdash{}{0pt}%
\pgfsys@defobject{currentmarker}{\pgfqpoint{0.000000in}{-0.048611in}}{\pgfqpoint{0.000000in}{0.000000in}}{%
\pgfpathmoveto{\pgfqpoint{0.000000in}{0.000000in}}%
\pgfpathlineto{\pgfqpoint{0.000000in}{-0.048611in}}%
\pgfusepath{stroke,fill}%
}%
\begin{pgfscope}%
\pgfsys@transformshift{2.229543in}{0.539544in}%
\pgfsys@useobject{currentmarker}{}%
\end{pgfscope}%
\end{pgfscope}%
\begin{pgfscope}%
\definecolor{textcolor}{rgb}{0.000000,0.000000,0.000000}%
\pgfsetstrokecolor{textcolor}%
\pgfsetfillcolor{textcolor}%
\pgftext[x=2.229543in,y=0.442322in,,top]{\color{textcolor}\rmfamily\fontsize{8.000000}{9.600000}\selectfont \(\displaystyle {15{:}00}\)}%
\end{pgfscope}%
\begin{pgfscope}%
\pgfsetbuttcap%
\pgfsetroundjoin%
\definecolor{currentfill}{rgb}{0.000000,0.000000,0.000000}%
\pgfsetfillcolor{currentfill}%
\pgfsetlinewidth{0.803000pt}%
\definecolor{currentstroke}{rgb}{0.000000,0.000000,0.000000}%
\pgfsetstrokecolor{currentstroke}%
\pgfsetdash{}{0pt}%
\pgfsys@defobject{currentmarker}{\pgfqpoint{0.000000in}{-0.048611in}}{\pgfqpoint{0.000000in}{0.000000in}}{%
\pgfpathmoveto{\pgfqpoint{0.000000in}{0.000000in}}%
\pgfpathlineto{\pgfqpoint{0.000000in}{-0.048611in}}%
\pgfusepath{stroke,fill}%
}%
\begin{pgfscope}%
\pgfsys@transformshift{2.742876in}{0.539544in}%
\pgfsys@useobject{currentmarker}{}%
\end{pgfscope}%
\end{pgfscope}%
\begin{pgfscope}%
\definecolor{textcolor}{rgb}{0.000000,0.000000,0.000000}%
\pgfsetstrokecolor{textcolor}%
\pgfsetfillcolor{textcolor}%
\pgftext[x=2.742876in,y=0.442322in,,top]{\color{textcolor}\rmfamily\fontsize{8.000000}{9.600000}\selectfont \(\displaystyle {18{:}00}\)}%
\end{pgfscope}%
\begin{pgfscope}%
\pgfsetbuttcap%
\pgfsetroundjoin%
\definecolor{currentfill}{rgb}{0.000000,0.000000,0.000000}%
\pgfsetfillcolor{currentfill}%
\pgfsetlinewidth{0.803000pt}%
\definecolor{currentstroke}{rgb}{0.000000,0.000000,0.000000}%
\pgfsetstrokecolor{currentstroke}%
\pgfsetdash{}{0pt}%
\pgfsys@defobject{currentmarker}{\pgfqpoint{0.000000in}{-0.048611in}}{\pgfqpoint{0.000000in}{0.000000in}}{%
\pgfpathmoveto{\pgfqpoint{0.000000in}{0.000000in}}%
\pgfpathlineto{\pgfqpoint{0.000000in}{-0.048611in}}%
\pgfusepath{stroke,fill}%
}%
\begin{pgfscope}%
\pgfsys@transformshift{3.256208in}{0.539544in}%
\pgfsys@useobject{currentmarker}{}%
\end{pgfscope}%
\end{pgfscope}%
\begin{pgfscope}%
\definecolor{textcolor}{rgb}{0.000000,0.000000,0.000000}%
\pgfsetstrokecolor{textcolor}%
\pgfsetfillcolor{textcolor}%
\pgftext[x=3.256208in,y=0.442322in,,top]{\color{textcolor}\rmfamily\fontsize{8.000000}{9.600000}\selectfont \(\displaystyle {21{:}00}\)}%
\end{pgfscope}%
\begin{pgfscope}%
\pgfsetbuttcap%
\pgfsetroundjoin%
\definecolor{currentfill}{rgb}{0.000000,0.000000,0.000000}%
\pgfsetfillcolor{currentfill}%
\pgfsetlinewidth{0.803000pt}%
\definecolor{currentstroke}{rgb}{0.000000,0.000000,0.000000}%
\pgfsetstrokecolor{currentstroke}%
\pgfsetdash{}{0pt}%
\pgfsys@defobject{currentmarker}{\pgfqpoint{0.000000in}{-0.048611in}}{\pgfqpoint{0.000000in}{0.000000in}}{%
\pgfpathmoveto{\pgfqpoint{0.000000in}{0.000000in}}%
\pgfpathlineto{\pgfqpoint{0.000000in}{-0.048611in}}%
\pgfusepath{stroke,fill}%
}%
\begin{pgfscope}%
\pgfsys@transformshift{3.769541in}{0.539544in}%
\pgfsys@useobject{currentmarker}{}%
\end{pgfscope}%
\end{pgfscope}%
\begin{pgfscope}%
\definecolor{textcolor}{rgb}{0.000000,0.000000,0.000000}%
\pgfsetstrokecolor{textcolor}%
\pgfsetfillcolor{textcolor}%
\pgftext[x=3.769541in,y=0.442322in,,top]{\color{textcolor}\rmfamily\fontsize{8.000000}{9.600000}\selectfont \(\displaystyle {00{:}00}\)}%
\end{pgfscope}%
\begin{pgfscope}%
\pgfsetbuttcap%
\pgfsetroundjoin%
\definecolor{currentfill}{rgb}{0.000000,0.000000,0.000000}%
\pgfsetfillcolor{currentfill}%
\pgfsetlinewidth{0.803000pt}%
\definecolor{currentstroke}{rgb}{0.000000,0.000000,0.000000}%
\pgfsetstrokecolor{currentstroke}%
\pgfsetdash{}{0pt}%
\pgfsys@defobject{currentmarker}{\pgfqpoint{0.000000in}{-0.048611in}}{\pgfqpoint{0.000000in}{0.000000in}}{%
\pgfpathmoveto{\pgfqpoint{0.000000in}{0.000000in}}%
\pgfpathlineto{\pgfqpoint{0.000000in}{-0.048611in}}%
\pgfusepath{stroke,fill}%
}%
\begin{pgfscope}%
\pgfsys@transformshift{4.282873in}{0.539544in}%
\pgfsys@useobject{currentmarker}{}%
\end{pgfscope}%
\end{pgfscope}%
\begin{pgfscope}%
\definecolor{textcolor}{rgb}{0.000000,0.000000,0.000000}%
\pgfsetstrokecolor{textcolor}%
\pgfsetfillcolor{textcolor}%
\pgftext[x=4.282873in,y=0.442322in,,top]{\color{textcolor}\rmfamily\fontsize{8.000000}{9.600000}\selectfont \(\displaystyle {03{:}00}\)}%
\end{pgfscope}%
\begin{pgfscope}%
\pgfsetbuttcap%
\pgfsetroundjoin%
\definecolor{currentfill}{rgb}{0.000000,0.000000,0.000000}%
\pgfsetfillcolor{currentfill}%
\pgfsetlinewidth{0.803000pt}%
\definecolor{currentstroke}{rgb}{0.000000,0.000000,0.000000}%
\pgfsetstrokecolor{currentstroke}%
\pgfsetdash{}{0pt}%
\pgfsys@defobject{currentmarker}{\pgfqpoint{0.000000in}{-0.048611in}}{\pgfqpoint{0.000000in}{0.000000in}}{%
\pgfpathmoveto{\pgfqpoint{0.000000in}{0.000000in}}%
\pgfpathlineto{\pgfqpoint{0.000000in}{-0.048611in}}%
\pgfusepath{stroke,fill}%
}%
\begin{pgfscope}%
\pgfsys@transformshift{4.796206in}{0.539544in}%
\pgfsys@useobject{currentmarker}{}%
\end{pgfscope}%
\end{pgfscope}%
\begin{pgfscope}%
\definecolor{textcolor}{rgb}{0.000000,0.000000,0.000000}%
\pgfsetstrokecolor{textcolor}%
\pgfsetfillcolor{textcolor}%
\pgftext[x=4.796206in,y=0.442322in,,top]{\color{textcolor}\rmfamily\fontsize{8.000000}{9.600000}\selectfont \(\displaystyle {06{:}00}\)}%
\end{pgfscope}%
\begin{pgfscope}%
\definecolor{textcolor}{rgb}{0.000000,0.000000,0.000000}%
\pgfsetstrokecolor{textcolor}%
\pgfsetfillcolor{textcolor}%
\pgftext[x=2.742685in,y=0.288100in,,top]{\color{textcolor}\rmfamily\fontsize{10.000000}{12.000000}\selectfont Time (UTC)}%
\end{pgfscope}%
\begin{pgfscope}%
\pgfsetbuttcap%
\pgfsetroundjoin%
\definecolor{currentfill}{rgb}{0.000000,0.000000,0.000000}%
\pgfsetfillcolor{currentfill}%
\pgfsetlinewidth{0.803000pt}%
\definecolor{currentstroke}{rgb}{0.000000,0.000000,0.000000}%
\pgfsetstrokecolor{currentstroke}%
\pgfsetdash{}{0pt}%
\pgfsys@defobject{currentmarker}{\pgfqpoint{-0.048611in}{0.000000in}}{\pgfqpoint{-0.000000in}{0.000000in}}{%
\pgfpathmoveto{\pgfqpoint{-0.000000in}{0.000000in}}%
\pgfpathlineto{\pgfqpoint{-0.048611in}{0.000000in}}%
\pgfusepath{stroke,fill}%
}%
\begin{pgfscope}%
\pgfsys@transformshift{0.484581in}{0.717544in}%
\pgfsys@useobject{currentmarker}{}%
\end{pgfscope}%
\end{pgfscope}%
\begin{pgfscope}%
\definecolor{textcolor}{rgb}{0.000000,0.000000,0.000000}%
\pgfsetstrokecolor{textcolor}%
\pgfsetfillcolor{textcolor}%
\pgftext[x=0.328331in, y=0.678988in, left, base]{\color{textcolor}\rmfamily\fontsize{8.000000}{9.600000}\selectfont \(\displaystyle {0}\)}%
\end{pgfscope}%
\begin{pgfscope}%
\pgfsetbuttcap%
\pgfsetroundjoin%
\definecolor{currentfill}{rgb}{0.000000,0.000000,0.000000}%
\pgfsetfillcolor{currentfill}%
\pgfsetlinewidth{0.803000pt}%
\definecolor{currentstroke}{rgb}{0.000000,0.000000,0.000000}%
\pgfsetstrokecolor{currentstroke}%
\pgfsetdash{}{0pt}%
\pgfsys@defobject{currentmarker}{\pgfqpoint{-0.048611in}{0.000000in}}{\pgfqpoint{-0.000000in}{0.000000in}}{%
\pgfpathmoveto{\pgfqpoint{-0.000000in}{0.000000in}}%
\pgfpathlineto{\pgfqpoint{-0.048611in}{0.000000in}}%
\pgfusepath{stroke,fill}%
}%
\begin{pgfscope}%
\pgfsys@transformshift{0.484581in}{0.921085in}%
\pgfsys@useobject{currentmarker}{}%
\end{pgfscope}%
\end{pgfscope}%
\begin{pgfscope}%
\definecolor{textcolor}{rgb}{0.000000,0.000000,0.000000}%
\pgfsetstrokecolor{textcolor}%
\pgfsetfillcolor{textcolor}%
\pgftext[x=0.328331in, y=0.882529in, left, base]{\color{textcolor}\rmfamily\fontsize{8.000000}{9.600000}\selectfont \(\displaystyle {5}\)}%
\end{pgfscope}%
\begin{pgfscope}%
\definecolor{textcolor}{rgb}{0.000000,0.000000,0.000000}%
\pgfsetstrokecolor{textcolor}%
\pgfsetfillcolor{textcolor}%
\pgftext[x=0.484581in,y=1.155833in,left,base]{\color{textcolor}\rmfamily\fontsize{8.000000}{9.600000}\selectfont \(\displaystyle \times{10^{\ensuremath{-}6}}{}\)}%
\end{pgfscope}%
\begin{pgfscope}%
\pgfpathrectangle{\pgfqpoint{0.484581in}{0.539544in}}{\pgfqpoint{4.516206in}{0.574622in}}%
\pgfusepath{clip}%
\pgfsetrectcap%
\pgfsetroundjoin%
\pgfsetlinewidth{0.501875pt}%
\definecolor{currentstroke}{rgb}{0.003922,0.450980,0.698039}%
\pgfsetstrokecolor{currentstroke}%
\pgfsetstrokeopacity{0.700000}%
\pgfsetdash{}{0pt}%
\pgfpathmoveto{\pgfqpoint{0.689863in}{0.721589in}}%
\pgfpathlineto{\pgfqpoint{0.694140in}{0.685169in}}%
\pgfpathlineto{\pgfqpoint{0.698419in}{0.771788in}}%
\pgfpathlineto{\pgfqpoint{0.700132in}{0.644899in}}%
\pgfpathlineto{\pgfqpoint{0.703553in}{0.767073in}}%
\pgfpathlineto{\pgfqpoint{0.707833in}{0.699777in}}%
\pgfpathlineto{\pgfqpoint{0.713821in}{0.764305in}}%
\pgfpathlineto{\pgfqpoint{0.718102in}{0.651844in}}%
\pgfpathlineto{\pgfqpoint{0.721524in}{0.733357in}}%
\pgfpathlineto{\pgfqpoint{0.726652in}{0.645940in}}%
\pgfpathlineto{\pgfqpoint{0.729218in}{0.759733in}}%
\pgfpathlineto{\pgfqpoint{0.736917in}{0.775857in}}%
\pgfpathlineto{\pgfqpoint{0.740340in}{0.660732in}}%
\pgfpathlineto{\pgfqpoint{0.742052in}{0.726624in}}%
\pgfpathlineto{\pgfqpoint{0.746327in}{0.778195in}}%
\pgfpathlineto{\pgfqpoint{0.751457in}{0.661163in}}%
\pgfpathlineto{\pgfqpoint{0.755738in}{0.790322in}}%
\pgfpathlineto{\pgfqpoint{0.759157in}{0.660947in}}%
\pgfpathlineto{\pgfqpoint{0.765999in}{0.761461in}}%
\pgfpathlineto{\pgfqpoint{0.767711in}{0.680382in}}%
\pgfpathlineto{\pgfqpoint{0.773703in}{0.750343in}}%
\pgfpathlineto{\pgfqpoint{0.776269in}{0.679736in}}%
\pgfpathlineto{\pgfqpoint{0.783967in}{0.745340in}}%
\pgfpathlineto{\pgfqpoint{0.784824in}{0.605387in}}%
\pgfpathlineto{\pgfqpoint{0.789097in}{0.784242in}}%
\pgfpathlineto{\pgfqpoint{0.793373in}{0.817025in}}%
\pgfpathlineto{\pgfqpoint{0.797648in}{0.692652in}}%
\pgfpathlineto{\pgfqpoint{0.805343in}{0.725690in}}%
\pgfpathlineto{\pgfqpoint{0.807054in}{0.608730in}}%
\pgfpathlineto{\pgfqpoint{0.811330in}{0.738647in}}%
\pgfpathlineto{\pgfqpoint{0.816461in}{0.682759in}}%
\pgfpathlineto{\pgfqpoint{0.819025in}{0.765063in}}%
\pgfpathlineto{\pgfqpoint{0.823303in}{0.717449in}}%
\pgfpathlineto{\pgfqpoint{0.827584in}{0.851108in}}%
\pgfpathlineto{\pgfqpoint{0.834428in}{0.692078in}}%
\pgfpathlineto{\pgfqpoint{0.838709in}{0.729073in}}%
\pgfpathlineto{\pgfqpoint{0.840420in}{0.676353in}}%
\pgfpathlineto{\pgfqpoint{0.848114in}{0.744295in}}%
\pgfpathlineto{\pgfqpoint{0.848968in}{0.708705in}}%
\pgfpathlineto{\pgfqpoint{0.855805in}{0.669587in}}%
\pgfpathlineto{\pgfqpoint{0.859233in}{0.748939in}}%
\pgfpathlineto{\pgfqpoint{0.861803in}{0.677757in}}%
\pgfpathlineto{\pgfqpoint{0.867793in}{0.662535in}}%
\pgfpathlineto{\pgfqpoint{0.870360in}{0.764416in}}%
\pgfpathlineto{\pgfqpoint{0.875490in}{0.657676in}}%
\pgfpathlineto{\pgfqpoint{0.879771in}{0.744551in}}%
\pgfpathlineto{\pgfqpoint{0.883195in}{0.653176in}}%
\pgfpathlineto{\pgfqpoint{0.887469in}{0.783524in}}%
\pgfpathlineto{\pgfqpoint{0.893457in}{0.703128in}}%
\pgfpathlineto{\pgfqpoint{0.896021in}{0.741311in}}%
\pgfpathlineto{\pgfqpoint{0.903720in}{0.770711in}}%
\pgfpathlineto{\pgfqpoint{0.907998in}{0.652889in}}%
\pgfpathlineto{\pgfqpoint{0.910566in}{0.728211in}}%
\pgfpathlineto{\pgfqpoint{0.914841in}{0.649793in}}%
\pgfpathlineto{\pgfqpoint{0.918259in}{0.754914in}}%
\pgfpathlineto{\pgfqpoint{0.922534in}{0.779168in}}%
\pgfpathlineto{\pgfqpoint{0.925951in}{0.659475in}}%
\pgfpathlineto{\pgfqpoint{0.932791in}{0.780141in}}%
\pgfpathlineto{\pgfqpoint{0.935359in}{0.665881in}}%
\pgfpathlineto{\pgfqpoint{0.938782in}{0.766467in}}%
\pgfpathlineto{\pgfqpoint{0.947330in}{0.636984in}}%
\pgfpathlineto{\pgfqpoint{0.951605in}{0.759015in}}%
\pgfpathlineto{\pgfqpoint{0.959305in}{0.764089in}}%
\pgfpathlineto{\pgfqpoint{0.963579in}{0.653679in}}%
\pgfpathlineto{\pgfqpoint{0.966147in}{0.724035in}}%
\pgfpathlineto{\pgfqpoint{0.969570in}{0.683477in}}%
\pgfpathlineto{\pgfqpoint{0.976408in}{0.806015in}}%
\pgfpathlineto{\pgfqpoint{0.978974in}{0.720329in}}%
\pgfpathlineto{\pgfqpoint{0.984102in}{0.758728in}}%
\pgfpathlineto{\pgfqpoint{0.985812in}{0.662463in}}%
\pgfpathlineto{\pgfqpoint{0.991797in}{0.733321in}}%
\pgfpathlineto{\pgfqpoint{0.997785in}{0.750271in}}%
\pgfpathlineto{\pgfqpoint{0.999494in}{0.780716in}}%
\pgfpathlineto{\pgfqpoint{1.003769in}{0.688448in}}%
\pgfpathlineto{\pgfqpoint{1.010613in}{0.744295in}}%
\pgfpathlineto{\pgfqpoint{1.014889in}{0.673186in}}%
\pgfpathlineto{\pgfqpoint{1.018310in}{0.746960in}}%
\pgfpathlineto{\pgfqpoint{1.020874in}{0.673545in}}%
\pgfpathlineto{\pgfqpoint{1.026001in}{0.741846in}}%
\pgfpathlineto{\pgfqpoint{1.029421in}{0.668291in}}%
\pgfpathlineto{\pgfqpoint{1.038832in}{0.800079in}}%
\pgfpathlineto{\pgfqpoint{1.041400in}{0.683154in}}%
\pgfpathlineto{\pgfqpoint{1.046538in}{0.642669in}}%
\pgfpathlineto{\pgfqpoint{1.050817in}{0.768733in}}%
\pgfpathlineto{\pgfqpoint{1.054238in}{0.679807in}}%
\pgfpathlineto{\pgfqpoint{1.059370in}{0.769020in}}%
\pgfpathlineto{\pgfqpoint{1.062792in}{0.636047in}}%
\pgfpathlineto{\pgfqpoint{1.068774in}{0.782192in}}%
\pgfpathlineto{\pgfqpoint{1.071339in}{0.679233in}}%
\pgfpathlineto{\pgfqpoint{1.077323in}{0.749338in}}%
\pgfpathlineto{\pgfqpoint{1.079890in}{0.663364in}}%
\pgfpathlineto{\pgfqpoint{1.084169in}{0.749338in}}%
\pgfpathlineto{\pgfqpoint{1.091012in}{0.695177in}}%
\pgfpathlineto{\pgfqpoint{1.095289in}{0.766144in}}%
\pgfpathlineto{\pgfqpoint{1.096997in}{0.686070in}}%
\pgfpathlineto{\pgfqpoint{1.102978in}{0.793385in}}%
\pgfpathlineto{\pgfqpoint{1.105542in}{0.697336in}}%
\pgfpathlineto{\pgfqpoint{1.112379in}{0.642669in}}%
\pgfpathlineto{\pgfqpoint{1.118364in}{0.769235in}}%
\pgfpathlineto{\pgfqpoint{1.124349in}{0.656671in}}%
\pgfpathlineto{\pgfqpoint{1.129483in}{0.728139in}}%
\pgfpathlineto{\pgfqpoint{1.131195in}{0.682616in}}%
\pgfpathlineto{\pgfqpoint{1.137183in}{0.656056in}}%
\pgfpathlineto{\pgfqpoint{1.139749in}{0.764632in}}%
\pgfpathlineto{\pgfqpoint{1.147448in}{0.689740in}}%
\pgfpathlineto{\pgfqpoint{1.150013in}{0.769235in}}%
\pgfpathlineto{\pgfqpoint{1.152579in}{0.575628in}}%
\pgfpathlineto{\pgfqpoint{1.156855in}{0.776400in}}%
\pgfpathlineto{\pgfqpoint{1.162841in}{0.681427in}}%
\pgfpathlineto{\pgfqpoint{1.167121in}{0.726592in}}%
\pgfpathlineto{\pgfqpoint{1.169688in}{0.667896in}}%
\pgfpathlineto{\pgfqpoint{1.176534in}{0.787481in}}%
\pgfpathlineto{\pgfqpoint{1.178245in}{0.676784in}}%
\pgfpathlineto{\pgfqpoint{1.185091in}{0.833942in}}%
\pgfpathlineto{\pgfqpoint{1.188512in}{0.707229in}}%
\pgfpathlineto{\pgfqpoint{1.191934in}{0.760671in}}%
\pgfpathlineto{\pgfqpoint{1.196209in}{0.672643in}}%
\pgfpathlineto{\pgfqpoint{1.200488in}{0.747535in}}%
\pgfpathlineto{\pgfqpoint{1.203910in}{0.679915in}}%
\pgfpathlineto{\pgfqpoint{1.209902in}{0.814184in}}%
\pgfpathlineto{\pgfqpoint{1.212472in}{0.687834in}}%
\pgfpathlineto{\pgfqpoint{1.217612in}{0.765888in}}%
\pgfpathlineto{\pgfqpoint{1.222746in}{0.685097in}}%
\pgfpathlineto{\pgfqpoint{1.227024in}{0.668147in}}%
\pgfpathlineto{\pgfqpoint{1.231301in}{0.763443in}}%
\pgfpathlineto{\pgfqpoint{1.233868in}{0.676640in}}%
\pgfpathlineto{\pgfqpoint{1.238147in}{0.745915in}}%
\pgfpathlineto{\pgfqpoint{1.242423in}{0.693953in}}%
\pgfpathlineto{\pgfqpoint{1.246701in}{0.774309in}}%
\pgfpathlineto{\pgfqpoint{1.250981in}{0.670561in}}%
\pgfpathlineto{\pgfqpoint{1.258680in}{0.795292in}}%
\pgfpathlineto{\pgfqpoint{1.261242in}{0.712773in}}%
\pgfpathlineto{\pgfqpoint{1.266376in}{0.647456in}}%
\pgfpathlineto{\pgfqpoint{1.269800in}{0.796696in}}%
\pgfpathlineto{\pgfqpoint{1.273224in}{0.704460in}}%
\pgfpathlineto{\pgfqpoint{1.276647in}{0.782986in}}%
\pgfpathlineto{\pgfqpoint{1.281778in}{0.704460in}}%
\pgfpathlineto{\pgfqpoint{1.287770in}{0.812493in}}%
\pgfpathlineto{\pgfqpoint{1.291191in}{0.623130in}}%
\pgfpathlineto{\pgfqpoint{1.297179in}{0.760563in}}%
\pgfpathlineto{\pgfqpoint{1.298889in}{0.654652in}}%
\pgfpathlineto{\pgfqpoint{1.304883in}{0.781366in}}%
\pgfpathlineto{\pgfqpoint{1.310013in}{0.697838in}}%
\pgfpathlineto{\pgfqpoint{1.311723in}{0.737426in}}%
\pgfpathlineto{\pgfqpoint{1.317712in}{0.680924in}}%
\pgfpathlineto{\pgfqpoint{1.321985in}{0.766323in}}%
\pgfpathlineto{\pgfqpoint{1.326257in}{0.674123in}}%
\pgfpathlineto{\pgfqpoint{1.328822in}{0.752648in}}%
\pgfpathlineto{\pgfqpoint{1.335662in}{0.669623in}}%
\pgfpathlineto{\pgfqpoint{1.337375in}{0.759159in}}%
\pgfpathlineto{\pgfqpoint{1.343354in}{0.667609in}}%
\pgfpathlineto{\pgfqpoint{1.345921in}{0.778737in}}%
\pgfpathlineto{\pgfqpoint{1.351911in}{0.654616in}}%
\pgfpathlineto{\pgfqpoint{1.356186in}{0.791909in}}%
\pgfpathlineto{\pgfqpoint{1.359610in}{0.666205in}}%
\pgfpathlineto{\pgfqpoint{1.363031in}{0.745843in}}%
\pgfpathlineto{\pgfqpoint{1.369869in}{0.625827in}}%
\pgfpathlineto{\pgfqpoint{1.372432in}{0.738790in}}%
\pgfpathlineto{\pgfqpoint{1.378419in}{0.783740in}}%
\pgfpathlineto{\pgfqpoint{1.380129in}{0.701724in}}%
\pgfpathlineto{\pgfqpoint{1.386969in}{0.669013in}}%
\pgfpathlineto{\pgfqpoint{1.387825in}{0.714537in}}%
\pgfpathlineto{\pgfqpoint{1.392960in}{0.773950in}}%
\pgfpathlineto{\pgfqpoint{1.397242in}{0.715183in}}%
\pgfpathlineto{\pgfqpoint{1.400665in}{0.797127in}}%
\pgfpathlineto{\pgfqpoint{1.406651in}{0.699745in}}%
\pgfpathlineto{\pgfqpoint{1.410071in}{0.745412in}}%
\pgfpathlineto{\pgfqpoint{1.413491in}{0.618846in}}%
\pgfpathlineto{\pgfqpoint{1.417764in}{0.744658in}}%
\pgfpathlineto{\pgfqpoint{1.424607in}{0.684235in}}%
\pgfpathlineto{\pgfqpoint{1.426320in}{0.741096in}}%
\pgfpathlineto{\pgfqpoint{1.434016in}{0.688228in}}%
\pgfpathlineto{\pgfqpoint{1.437437in}{0.756135in}}%
\pgfpathlineto{\pgfqpoint{1.439148in}{0.658358in}}%
\pgfpathlineto{\pgfqpoint{1.444278in}{0.799249in}}%
\pgfpathlineto{\pgfqpoint{1.448556in}{0.638780in}}%
\pgfpathlineto{\pgfqpoint{1.453691in}{0.747104in}}%
\pgfpathlineto{\pgfqpoint{1.456259in}{0.662965in}}%
\pgfpathlineto{\pgfqpoint{1.462250in}{0.608837in}}%
\pgfpathlineto{\pgfqpoint{1.467386in}{0.743071in}}%
\pgfpathlineto{\pgfqpoint{1.471667in}{0.742855in}}%
\pgfpathlineto{\pgfqpoint{1.476802in}{0.634679in}}%
\pgfpathlineto{\pgfqpoint{1.479368in}{0.746888in}}%
\pgfpathlineto{\pgfqpoint{1.481936in}{0.681786in}}%
\pgfpathlineto{\pgfqpoint{1.487072in}{0.775857in}}%
\pgfpathlineto{\pgfqpoint{1.491345in}{0.636478in}}%
\pgfpathlineto{\pgfqpoint{1.496476in}{0.761568in}}%
\pgfpathlineto{\pgfqpoint{1.500749in}{0.783843in}}%
\pgfpathlineto{\pgfqpoint{1.504171in}{0.652849in}}%
\pgfpathlineto{\pgfqpoint{1.508448in}{0.717808in}}%
\pgfpathlineto{\pgfqpoint{1.511868in}{0.688480in}}%
\pgfpathlineto{\pgfqpoint{1.518707in}{0.768481in}}%
\pgfpathlineto{\pgfqpoint{1.522982in}{0.768230in}}%
\pgfpathlineto{\pgfqpoint{1.527256in}{0.649434in}}%
\pgfpathlineto{\pgfqpoint{1.530678in}{0.746457in}}%
\pgfpathlineto{\pgfqpoint{1.535810in}{0.684235in}}%
\pgfpathlineto{\pgfqpoint{1.538376in}{0.748795in}}%
\pgfpathlineto{\pgfqpoint{1.541797in}{0.681571in}}%
\pgfpathlineto{\pgfqpoint{1.548638in}{0.743218in}}%
\pgfpathlineto{\pgfqpoint{1.550351in}{0.659511in}}%
\pgfpathlineto{\pgfqpoint{1.556343in}{0.605423in}}%
\pgfpathlineto{\pgfqpoint{1.558909in}{0.713998in}}%
\pgfpathlineto{\pgfqpoint{1.565750in}{0.753582in}}%
\pgfpathlineto{\pgfqpoint{1.567459in}{0.664513in}}%
\pgfpathlineto{\pgfqpoint{1.573455in}{0.624064in}}%
\pgfpathlineto{\pgfqpoint{1.576880in}{0.770352in}}%
\pgfpathlineto{\pgfqpoint{1.581159in}{0.662678in}}%
\pgfpathlineto{\pgfqpoint{1.585439in}{0.612583in}}%
\pgfpathlineto{\pgfqpoint{1.588860in}{0.716228in}}%
\pgfpathlineto{\pgfqpoint{1.599126in}{0.650874in}}%
\pgfpathlineto{\pgfqpoint{1.602549in}{0.755672in}}%
\pgfpathlineto{\pgfqpoint{1.607680in}{0.653431in}}%
\pgfpathlineto{\pgfqpoint{1.611102in}{0.789572in}}%
\pgfpathlineto{\pgfqpoint{1.615378in}{0.687834in}}%
\pgfpathlineto{\pgfqpoint{1.619653in}{0.775642in}}%
\pgfpathlineto{\pgfqpoint{1.623070in}{0.699530in}}%
\pgfpathlineto{\pgfqpoint{1.629917in}{0.644073in}}%
\pgfpathlineto{\pgfqpoint{1.634197in}{0.776974in}}%
\pgfpathlineto{\pgfqpoint{1.637617in}{0.680238in}}%
\pgfpathlineto{\pgfqpoint{1.642745in}{0.787481in}}%
\pgfpathlineto{\pgfqpoint{1.647874in}{0.682688in}}%
\pgfpathlineto{\pgfqpoint{1.648727in}{0.761177in}}%
\pgfpathlineto{\pgfqpoint{1.656428in}{0.684092in}}%
\pgfpathlineto{\pgfqpoint{1.660704in}{0.730262in}}%
\pgfpathlineto{\pgfqpoint{1.661558in}{0.661059in}}%
\pgfpathlineto{\pgfqpoint{1.669255in}{0.699171in}}%
\pgfpathlineto{\pgfqpoint{1.671820in}{0.809146in}}%
\pgfpathlineto{\pgfqpoint{1.674387in}{0.678044in}}%
\pgfpathlineto{\pgfqpoint{1.679521in}{0.637236in}}%
\pgfpathlineto{\pgfqpoint{1.683795in}{0.752469in}}%
\pgfpathlineto{\pgfqpoint{1.688070in}{0.652961in}}%
\pgfpathlineto{\pgfqpoint{1.694908in}{0.747933in}}%
\pgfpathlineto{\pgfqpoint{1.696620in}{0.663939in}}%
\pgfpathlineto{\pgfqpoint{1.700041in}{0.780321in}}%
\pgfpathlineto{\pgfqpoint{1.705169in}{0.638999in}}%
\pgfpathlineto{\pgfqpoint{1.711157in}{0.751747in}}%
\pgfpathlineto{\pgfqpoint{1.713721in}{0.666851in}}%
\pgfpathlineto{\pgfqpoint{1.719709in}{0.791909in}}%
\pgfpathlineto{\pgfqpoint{1.721421in}{0.735551in}}%
\pgfpathlineto{\pgfqpoint{1.726553in}{0.682073in}}%
\pgfpathlineto{\pgfqpoint{1.733392in}{0.682759in}}%
\pgfpathlineto{\pgfqpoint{1.736816in}{0.777118in}}%
\pgfpathlineto{\pgfqpoint{1.738528in}{0.710324in}}%
\pgfpathlineto{\pgfqpoint{1.743662in}{0.771685in}}%
\pgfpathlineto{\pgfqpoint{1.747088in}{0.716228in}}%
\pgfpathlineto{\pgfqpoint{1.753081in}{0.665343in}}%
\pgfpathlineto{\pgfqpoint{1.755646in}{0.764448in}}%
\pgfpathlineto{\pgfqpoint{1.759920in}{0.657820in}}%
\pgfpathlineto{\pgfqpoint{1.765049in}{0.724685in}}%
\pgfpathlineto{\pgfqpoint{1.771888in}{0.659726in}}%
\pgfpathlineto{\pgfqpoint{1.774452in}{0.761823in}}%
\pgfpathlineto{\pgfqpoint{1.780436in}{0.641408in}}%
\pgfpathlineto{\pgfqpoint{1.783002in}{0.735120in}}%
\pgfpathlineto{\pgfqpoint{1.787279in}{0.699422in}}%
\pgfpathlineto{\pgfqpoint{1.793271in}{0.782048in}}%
\pgfpathlineto{\pgfqpoint{1.794127in}{0.682544in}}%
\pgfpathlineto{\pgfqpoint{1.800118in}{0.743936in}}%
\pgfpathlineto{\pgfqpoint{1.805246in}{0.686717in}}%
\pgfpathlineto{\pgfqpoint{1.808671in}{0.813897in}}%
\pgfpathlineto{\pgfqpoint{1.814659in}{0.695860in}}%
\pgfpathlineto{\pgfqpoint{1.818933in}{0.787625in}}%
\pgfpathlineto{\pgfqpoint{1.820647in}{0.685600in}}%
\pgfpathlineto{\pgfqpoint{1.825776in}{0.752824in}}%
\pgfpathlineto{\pgfqpoint{1.828343in}{0.692437in}}%
\pgfpathlineto{\pgfqpoint{1.832622in}{0.760994in}}%
\pgfpathlineto{\pgfqpoint{1.836895in}{0.675850in}}%
\pgfpathlineto{\pgfqpoint{1.842028in}{0.777405in}}%
\pgfpathlineto{\pgfqpoint{1.845449in}{0.690243in}}%
\pgfpathlineto{\pgfqpoint{1.851438in}{0.762940in}}%
\pgfpathlineto{\pgfqpoint{1.856567in}{0.642166in}}%
\pgfpathlineto{\pgfqpoint{1.859133in}{0.709103in}}%
\pgfpathlineto{\pgfqpoint{1.863412in}{0.657820in}}%
\pgfpathlineto{\pgfqpoint{1.867689in}{0.773807in}}%
\pgfpathlineto{\pgfqpoint{1.872817in}{0.834984in}}%
\pgfpathlineto{\pgfqpoint{1.877947in}{0.693123in}}%
\pgfpathlineto{\pgfqpoint{1.879659in}{0.760132in}}%
\pgfpathlineto{\pgfqpoint{1.886505in}{0.667681in}}%
\pgfpathlineto{\pgfqpoint{1.890783in}{0.780429in}}%
\pgfpathlineto{\pgfqpoint{1.895915in}{0.648932in}}%
\pgfpathlineto{\pgfqpoint{1.896771in}{0.781043in}}%
\pgfpathlineto{\pgfqpoint{1.901051in}{0.672539in}}%
\pgfpathlineto{\pgfqpoint{1.907894in}{0.726520in}}%
\pgfpathlineto{\pgfqpoint{1.911319in}{0.662319in}}%
\pgfpathlineto{\pgfqpoint{1.915593in}{0.771397in}}%
\pgfpathlineto{\pgfqpoint{1.919018in}{0.696506in}}%
\pgfpathlineto{\pgfqpoint{1.923300in}{0.752716in}}%
\pgfpathlineto{\pgfqpoint{1.930141in}{0.743250in}}%
\pgfpathlineto{\pgfqpoint{1.932706in}{0.684160in}}%
\pgfpathlineto{\pgfqpoint{1.937837in}{0.658035in}}%
\pgfpathlineto{\pgfqpoint{1.942970in}{0.782008in}}%
\pgfpathlineto{\pgfqpoint{1.946392in}{0.698556in}}%
\pgfpathlineto{\pgfqpoint{1.948960in}{0.802273in}}%
\pgfpathlineto{\pgfqpoint{1.953237in}{0.693123in}}%
\pgfpathlineto{\pgfqpoint{1.956661in}{0.764053in}}%
\pgfpathlineto{\pgfqpoint{1.964363in}{0.780716in}}%
\pgfpathlineto{\pgfqpoint{1.965220in}{0.676712in}}%
\pgfpathlineto{\pgfqpoint{1.971208in}{0.629425in}}%
\pgfpathlineto{\pgfqpoint{1.973771in}{0.790864in}}%
\pgfpathlineto{\pgfqpoint{1.980610in}{0.692078in}}%
\pgfpathlineto{\pgfqpoint{1.984031in}{0.758329in}}%
\pgfpathlineto{\pgfqpoint{1.989162in}{0.677394in}}%
\pgfpathlineto{\pgfqpoint{1.992584in}{0.755345in}}%
\pgfpathlineto{\pgfqpoint{1.997716in}{0.783125in}}%
\pgfpathlineto{\pgfqpoint{1.999427in}{0.702657in}}%
\pgfpathlineto{\pgfqpoint{2.003702in}{0.633095in}}%
\pgfpathlineto{\pgfqpoint{2.007981in}{0.786692in}}%
\pgfpathlineto{\pgfqpoint{2.013113in}{0.694021in}}%
\pgfpathlineto{\pgfqpoint{2.019100in}{0.829335in}}%
\pgfpathlineto{\pgfqpoint{2.022522in}{0.658071in}}%
\pgfpathlineto{\pgfqpoint{2.027657in}{0.785144in}}%
\pgfpathlineto{\pgfqpoint{2.031932in}{0.702553in}}%
\pgfpathlineto{\pgfqpoint{2.035355in}{0.800151in}}%
\pgfpathlineto{\pgfqpoint{2.038775in}{0.714860in}}%
\pgfpathlineto{\pgfqpoint{2.043051in}{0.758584in}}%
\pgfpathlineto{\pgfqpoint{2.046473in}{0.666923in}}%
\pgfpathlineto{\pgfqpoint{2.051607in}{0.760276in}}%
\pgfpathlineto{\pgfqpoint{2.055027in}{0.692908in}}%
\pgfpathlineto{\pgfqpoint{2.060159in}{0.680202in}}%
\pgfpathlineto{\pgfqpoint{2.064434in}{0.752034in}}%
\pgfpathlineto{\pgfqpoint{2.068710in}{0.690387in}}%
\pgfpathlineto{\pgfqpoint{2.072986in}{0.766068in}}%
\pgfpathlineto{\pgfqpoint{2.078974in}{0.702801in}}%
\pgfpathlineto{\pgfqpoint{2.080685in}{0.767113in}}%
\pgfpathlineto{\pgfqpoint{2.085812in}{0.714537in}}%
\pgfpathlineto{\pgfqpoint{2.091797in}{0.792268in}}%
\pgfpathlineto{\pgfqpoint{2.095221in}{0.688300in}}%
\pgfpathlineto{\pgfqpoint{2.099502in}{0.689956in}}%
\pgfpathlineto{\pgfqpoint{2.102927in}{0.817136in}}%
\pgfpathlineto{\pgfqpoint{2.107208in}{0.680278in}}%
\pgfpathlineto{\pgfqpoint{2.114054in}{0.662858in}}%
\pgfpathlineto{\pgfqpoint{2.114910in}{0.816777in}}%
\pgfpathlineto{\pgfqpoint{2.121752in}{0.795220in}}%
\pgfpathlineto{\pgfqpoint{2.124318in}{0.693267in}}%
\pgfpathlineto{\pgfqpoint{2.128598in}{0.740554in}}%
\pgfpathlineto{\pgfqpoint{2.135446in}{0.649973in}}%
\pgfpathlineto{\pgfqpoint{2.139720in}{0.837182in}}%
\pgfpathlineto{\pgfqpoint{2.141431in}{0.704029in}}%
\pgfpathlineto{\pgfqpoint{2.146558in}{0.826710in}}%
\pgfpathlineto{\pgfqpoint{2.149976in}{0.672683in}}%
\pgfpathlineto{\pgfqpoint{2.155105in}{0.729903in}}%
\pgfpathlineto{\pgfqpoint{2.157668in}{0.682185in}}%
\pgfpathlineto{\pgfqpoint{2.161947in}{0.633063in}}%
\pgfpathlineto{\pgfqpoint{2.169647in}{0.667322in}}%
\pgfpathlineto{\pgfqpoint{2.171357in}{0.855679in}}%
\pgfpathlineto{\pgfqpoint{2.177349in}{0.715438in}}%
\pgfpathlineto{\pgfqpoint{2.182483in}{0.665163in}}%
\pgfpathlineto{\pgfqpoint{2.184193in}{0.757794in}}%
\pgfpathlineto{\pgfqpoint{2.191037in}{0.689453in}}%
\pgfpathlineto{\pgfqpoint{2.192747in}{0.785575in}}%
\pgfpathlineto{\pgfqpoint{2.199592in}{0.687618in}}%
\pgfpathlineto{\pgfqpoint{2.202160in}{0.752864in}}%
\pgfpathlineto{\pgfqpoint{2.205581in}{0.678156in}}%
\pgfpathlineto{\pgfqpoint{2.211570in}{0.786835in}}%
\pgfpathlineto{\pgfqpoint{2.216701in}{0.780357in}}%
\pgfpathlineto{\pgfqpoint{2.217556in}{0.671315in}}%
\pgfpathlineto{\pgfqpoint{2.223541in}{0.750343in}}%
\pgfpathlineto{\pgfqpoint{2.229530in}{0.657496in}}%
\pgfpathlineto{\pgfqpoint{2.231239in}{0.766969in}}%
\pgfpathlineto{\pgfqpoint{2.238082in}{0.687762in}}%
\pgfpathlineto{\pgfqpoint{2.242359in}{0.800366in}}%
\pgfpathlineto{\pgfqpoint{2.243215in}{0.735914in}}%
\pgfpathlineto{\pgfqpoint{2.250909in}{0.676608in}}%
\pgfpathlineto{\pgfqpoint{2.252617in}{0.767257in}}%
\pgfpathlineto{\pgfqpoint{2.256038in}{0.799576in}}%
\pgfpathlineto{\pgfqpoint{2.261169in}{0.663508in}}%
\pgfpathlineto{\pgfqpoint{2.264591in}{0.712881in}}%
\pgfpathlineto{\pgfqpoint{2.271435in}{0.748041in}}%
\pgfpathlineto{\pgfqpoint{2.273147in}{0.625971in}}%
\pgfpathlineto{\pgfqpoint{2.278277in}{0.774349in}}%
\pgfpathlineto{\pgfqpoint{2.281703in}{0.776471in}}%
\pgfpathlineto{\pgfqpoint{2.291118in}{0.632521in}}%
\pgfpathlineto{\pgfqpoint{2.294543in}{0.795364in}}%
\pgfpathlineto{\pgfqpoint{2.298819in}{0.706223in}}%
\pgfpathlineto{\pgfqpoint{2.304810in}{0.704855in}}%
\pgfpathlineto{\pgfqpoint{2.309087in}{0.840636in}}%
\pgfpathlineto{\pgfqpoint{2.312504in}{0.661777in}}%
\pgfpathlineto{\pgfqpoint{2.316782in}{0.766251in}}%
\pgfpathlineto{\pgfqpoint{2.323623in}{0.632162in}}%
\pgfpathlineto{\pgfqpoint{2.327045in}{0.717161in}}%
\pgfpathlineto{\pgfqpoint{2.330467in}{0.651453in}}%
\pgfpathlineto{\pgfqpoint{2.333891in}{0.631874in}}%
\pgfpathlineto{\pgfqpoint{2.341587in}{0.802488in}}%
\pgfpathlineto{\pgfqpoint{2.346717in}{0.651848in}}%
\pgfpathlineto{\pgfqpoint{2.351849in}{0.758692in}}%
\pgfpathlineto{\pgfqpoint{2.357839in}{0.687834in}}%
\pgfpathlineto{\pgfqpoint{2.359552in}{0.769451in}}%
\pgfpathlineto{\pgfqpoint{2.364686in}{0.668830in}}%
\pgfpathlineto{\pgfqpoint{2.368106in}{0.805835in}}%
\pgfpathlineto{\pgfqpoint{2.371529in}{0.712518in}}%
\pgfpathlineto{\pgfqpoint{2.376661in}{0.770280in}}%
\pgfpathlineto{\pgfqpoint{2.380084in}{0.732172in}}%
\pgfpathlineto{\pgfqpoint{2.385215in}{0.650982in}}%
\pgfpathlineto{\pgfqpoint{2.391202in}{0.772115in}}%
\pgfpathlineto{\pgfqpoint{2.394625in}{0.667824in}}%
\pgfpathlineto{\pgfqpoint{2.397190in}{0.747574in}}%
\pgfpathlineto{\pgfqpoint{2.403176in}{0.689956in}}%
\pgfpathlineto{\pgfqpoint{2.408306in}{0.752214in}}%
\pgfpathlineto{\pgfqpoint{2.410871in}{0.679592in}}%
\pgfpathlineto{\pgfqpoint{2.415143in}{0.774453in}}%
\pgfpathlineto{\pgfqpoint{2.420276in}{0.689525in}}%
\pgfpathlineto{\pgfqpoint{2.425405in}{0.784713in}}%
\pgfpathlineto{\pgfqpoint{2.429680in}{0.618415in}}%
\pgfpathlineto{\pgfqpoint{2.432245in}{0.731522in}}%
\pgfpathlineto{\pgfqpoint{2.435666in}{0.790505in}}%
\pgfpathlineto{\pgfqpoint{2.442507in}{0.783125in}}%
\pgfpathlineto{\pgfqpoint{2.445931in}{0.687977in}}%
\pgfpathlineto{\pgfqpoint{2.451058in}{0.642597in}}%
\pgfpathlineto{\pgfqpoint{2.452770in}{0.721482in}}%
\pgfpathlineto{\pgfqpoint{2.457051in}{0.773232in}}%
\pgfpathlineto{\pgfqpoint{2.461330in}{0.706367in}}%
\pgfpathlineto{\pgfqpoint{2.465606in}{0.780285in}}%
\pgfpathlineto{\pgfqpoint{2.472449in}{0.676209in}}%
\pgfpathlineto{\pgfqpoint{2.475873in}{0.767185in}}%
\pgfpathlineto{\pgfqpoint{2.481860in}{0.639143in}}%
\pgfpathlineto{\pgfqpoint{2.482715in}{0.745556in}}%
\pgfpathlineto{\pgfqpoint{2.489559in}{0.634930in}}%
\pgfpathlineto{\pgfqpoint{2.492125in}{0.743035in}}%
\pgfpathlineto{\pgfqpoint{2.497259in}{0.660301in}}%
\pgfpathlineto{\pgfqpoint{2.502393in}{0.760132in}}%
\pgfpathlineto{\pgfqpoint{2.504105in}{0.690243in}}%
\pgfpathlineto{\pgfqpoint{2.511803in}{0.801623in}}%
\pgfpathlineto{\pgfqpoint{2.514369in}{0.667210in}}%
\pgfpathlineto{\pgfqpoint{2.518648in}{0.649937in}}%
\pgfpathlineto{\pgfqpoint{2.522070in}{0.771286in}}%
\pgfpathlineto{\pgfqpoint{2.528054in}{0.665519in}}%
\pgfpathlineto{\pgfqpoint{2.529766in}{0.731841in}}%
\pgfpathlineto{\pgfqpoint{2.536611in}{0.687219in}}%
\pgfpathlineto{\pgfqpoint{2.540888in}{0.762470in}}%
\pgfpathlineto{\pgfqpoint{2.542599in}{0.654939in}}%
\pgfpathlineto{\pgfqpoint{2.546873in}{0.765960in}}%
\pgfpathlineto{\pgfqpoint{2.552006in}{0.684666in}}%
\pgfpathlineto{\pgfqpoint{2.556277in}{0.769379in}}%
\pgfpathlineto{\pgfqpoint{2.560553in}{0.641624in}}%
\pgfpathlineto{\pgfqpoint{2.563974in}{0.683118in}}%
\pgfpathlineto{\pgfqpoint{2.569104in}{0.766642in}}%
\pgfpathlineto{\pgfqpoint{2.572527in}{0.694886in}}%
\pgfpathlineto{\pgfqpoint{2.576804in}{0.769092in}}%
\pgfpathlineto{\pgfqpoint{2.584495in}{0.663648in}}%
\pgfpathlineto{\pgfqpoint{2.585351in}{0.764592in}}%
\pgfpathlineto{\pgfqpoint{2.596466in}{0.679879in}}%
\pgfpathlineto{\pgfqpoint{2.598175in}{0.752393in}}%
\pgfpathlineto{\pgfqpoint{2.602452in}{0.687794in}}%
\pgfpathlineto{\pgfqpoint{2.609301in}{0.653463in}}%
\pgfpathlineto{\pgfqpoint{2.614436in}{0.824405in}}%
\pgfpathlineto{\pgfqpoint{2.615290in}{0.727094in}}%
\pgfpathlineto{\pgfqpoint{2.621275in}{0.672467in}}%
\pgfpathlineto{\pgfqpoint{2.627264in}{0.770137in}}%
\pgfpathlineto{\pgfqpoint{2.631542in}{0.663289in}}%
\pgfpathlineto{\pgfqpoint{2.635823in}{0.820878in}}%
\pgfpathlineto{\pgfqpoint{2.637532in}{0.686214in}}%
\pgfpathlineto{\pgfqpoint{2.642664in}{0.744152in}}%
\pgfpathlineto{\pgfqpoint{2.646085in}{0.669300in}}%
\pgfpathlineto{\pgfqpoint{2.651221in}{0.773192in}}%
\pgfpathlineto{\pgfqpoint{2.655500in}{0.705322in}}%
\pgfpathlineto{\pgfqpoint{2.658066in}{0.805656in}}%
\pgfpathlineto{\pgfqpoint{2.663199in}{0.710037in}}%
\pgfpathlineto{\pgfqpoint{2.668333in}{0.807419in}}%
\pgfpathlineto{\pgfqpoint{2.671754in}{0.683729in}}%
\pgfpathlineto{\pgfqpoint{2.675176in}{0.733932in}}%
\pgfpathlineto{\pgfqpoint{2.679450in}{0.688623in}}%
\pgfpathlineto{\pgfqpoint{2.685443in}{0.762075in}}%
\pgfpathlineto{\pgfqpoint{2.688863in}{0.686717in}}%
\pgfpathlineto{\pgfqpoint{2.693143in}{0.793816in}}%
\pgfpathlineto{\pgfqpoint{2.696565in}{0.716874in}}%
\pgfpathlineto{\pgfqpoint{2.702555in}{0.751420in}}%
\pgfpathlineto{\pgfqpoint{2.706832in}{0.678116in}}%
\pgfpathlineto{\pgfqpoint{2.711109in}{0.791439in}}%
\pgfpathlineto{\pgfqpoint{2.713672in}{0.689884in}}%
\pgfpathlineto{\pgfqpoint{2.719654in}{0.814831in}}%
\pgfpathlineto{\pgfqpoint{2.722221in}{0.722236in}}%
\pgfpathlineto{\pgfqpoint{2.728213in}{0.764017in}}%
\pgfpathlineto{\pgfqpoint{2.734203in}{0.797773in}}%
\pgfpathlineto{\pgfqpoint{2.737626in}{0.648318in}}%
\pgfpathlineto{\pgfqpoint{2.739338in}{0.734833in}}%
\pgfpathlineto{\pgfqpoint{2.746179in}{0.771397in}}%
\pgfpathlineto{\pgfqpoint{2.748744in}{0.683047in}}%
\pgfpathlineto{\pgfqpoint{2.754727in}{0.760922in}}%
\pgfpathlineto{\pgfqpoint{2.758145in}{0.788056in}}%
\pgfpathlineto{\pgfqpoint{2.760706in}{0.699781in}}%
\pgfpathlineto{\pgfqpoint{2.764980in}{0.783197in}}%
\pgfpathlineto{\pgfqpoint{2.769259in}{0.700535in}}%
\pgfpathlineto{\pgfqpoint{2.776955in}{0.686102in}}%
\pgfpathlineto{\pgfqpoint{2.778666in}{0.787083in}}%
\pgfpathlineto{\pgfqpoint{2.782089in}{0.699853in}}%
\pgfpathlineto{\pgfqpoint{2.789789in}{0.624351in}}%
\pgfpathlineto{\pgfqpoint{2.790642in}{0.752106in}}%
\pgfpathlineto{\pgfqpoint{2.794918in}{0.702410in}}%
\pgfpathlineto{\pgfqpoint{2.799189in}{0.795795in}}%
\pgfpathlineto{\pgfqpoint{2.803464in}{0.690606in}}%
\pgfpathlineto{\pgfqpoint{2.809450in}{0.775714in}}%
\pgfpathlineto{\pgfqpoint{2.813729in}{0.691033in}}%
\pgfpathlineto{\pgfqpoint{2.818861in}{0.662463in}}%
\pgfpathlineto{\pgfqpoint{2.820569in}{0.736596in}}%
\pgfpathlineto{\pgfqpoint{2.827409in}{0.652961in}}%
\pgfpathlineto{\pgfqpoint{2.832540in}{0.745340in}}%
\pgfpathlineto{\pgfqpoint{2.833396in}{0.682217in}}%
\pgfpathlineto{\pgfqpoint{2.839382in}{0.760994in}}%
\pgfpathlineto{\pgfqpoint{2.845369in}{0.678870in}}%
\pgfpathlineto{\pgfqpoint{2.847078in}{0.750774in}}%
\pgfpathlineto{\pgfqpoint{2.850497in}{0.682037in}}%
\pgfpathlineto{\pgfqpoint{2.855627in}{0.772474in}}%
\pgfpathlineto{\pgfqpoint{2.859900in}{0.649470in}}%
\pgfpathlineto{\pgfqpoint{2.865887in}{0.776862in}}%
\pgfpathlineto{\pgfqpoint{2.867599in}{0.729719in}}%
\pgfpathlineto{\pgfqpoint{2.874440in}{0.630686in}}%
\pgfpathlineto{\pgfqpoint{2.876150in}{0.725981in}}%
\pgfpathlineto{\pgfqpoint{2.880425in}{0.674949in}}%
\pgfpathlineto{\pgfqpoint{2.884702in}{0.770424in}}%
\pgfpathlineto{\pgfqpoint{2.891547in}{0.788096in}}%
\pgfpathlineto{\pgfqpoint{2.896681in}{0.694886in}}%
\pgfpathlineto{\pgfqpoint{2.897536in}{0.795723in}}%
\pgfpathlineto{\pgfqpoint{2.905233in}{0.795005in}}%
\pgfpathlineto{\pgfqpoint{2.906944in}{0.709750in}}%
\pgfpathlineto{\pgfqpoint{2.910364in}{0.751962in}}%
\pgfpathlineto{\pgfqpoint{2.918060in}{0.706151in}}%
\pgfpathlineto{\pgfqpoint{2.918914in}{0.776288in}}%
\pgfpathlineto{\pgfqpoint{2.927463in}{0.665231in}}%
\pgfpathlineto{\pgfqpoint{2.931738in}{0.760308in}}%
\pgfpathlineto{\pgfqpoint{2.936873in}{0.694487in}}%
\pgfpathlineto{\pgfqpoint{2.942857in}{0.685528in}}%
\pgfpathlineto{\pgfqpoint{2.944568in}{0.778374in}}%
\pgfpathlineto{\pgfqpoint{2.948846in}{0.697475in}}%
\pgfpathlineto{\pgfqpoint{2.955685in}{0.759949in}}%
\pgfpathlineto{\pgfqpoint{2.959964in}{0.606934in}}%
\pgfpathlineto{\pgfqpoint{2.964240in}{0.791510in}}%
\pgfpathlineto{\pgfqpoint{2.968519in}{0.842866in}}%
\pgfpathlineto{\pgfqpoint{2.970229in}{0.708166in}}%
\pgfpathlineto{\pgfqpoint{2.977074in}{0.701077in}}%
\pgfpathlineto{\pgfqpoint{2.978784in}{0.764089in}}%
\pgfpathlineto{\pgfqpoint{2.983062in}{0.697080in}}%
\pgfpathlineto{\pgfqpoint{2.988194in}{0.750630in}}%
\pgfpathlineto{\pgfqpoint{2.994182in}{0.699745in}}%
\pgfpathlineto{\pgfqpoint{2.999311in}{0.785000in}}%
\pgfpathlineto{\pgfqpoint{3.002736in}{0.701149in}}%
\pgfpathlineto{\pgfqpoint{3.007011in}{0.815301in}}%
\pgfpathlineto{\pgfqpoint{3.011287in}{0.686717in}}%
\pgfpathlineto{\pgfqpoint{3.013853in}{0.774995in}}%
\pgfpathlineto{\pgfqpoint{3.019839in}{0.792232in}}%
\pgfpathlineto{\pgfqpoint{3.021551in}{0.717560in}}%
\pgfpathlineto{\pgfqpoint{3.029249in}{0.612695in}}%
\pgfpathlineto{\pgfqpoint{3.030958in}{0.823686in}}%
\pgfpathlineto{\pgfqpoint{3.034379in}{0.741958in}}%
\pgfpathlineto{\pgfqpoint{3.039507in}{0.806701in}}%
\pgfpathlineto{\pgfqpoint{3.042930in}{0.716192in}}%
\pgfpathlineto{\pgfqpoint{3.050630in}{0.662535in}}%
\pgfpathlineto{\pgfqpoint{3.054053in}{0.812421in}}%
\pgfpathlineto{\pgfqpoint{3.059185in}{0.808895in}}%
\pgfpathlineto{\pgfqpoint{3.061754in}{0.684558in}}%
\pgfpathlineto{\pgfqpoint{3.064319in}{0.744371in}}%
\pgfpathlineto{\pgfqpoint{3.072018in}{0.681714in}}%
\pgfpathlineto{\pgfqpoint{3.072875in}{0.723424in}}%
\pgfpathlineto{\pgfqpoint{3.079716in}{0.757324in}}%
\pgfpathlineto{\pgfqpoint{3.084849in}{0.664370in}}%
\pgfpathlineto{\pgfqpoint{3.085704in}{0.774457in}}%
\pgfpathlineto{\pgfqpoint{3.090835in}{0.706511in}}%
\pgfpathlineto{\pgfqpoint{3.094255in}{0.765063in}}%
\pgfpathlineto{\pgfqpoint{3.099388in}{0.657923in}}%
\pgfpathlineto{\pgfqpoint{3.103662in}{0.748037in}}%
\pgfpathlineto{\pgfqpoint{3.110508in}{0.756638in}}%
\pgfpathlineto{\pgfqpoint{3.112219in}{0.677358in}}%
\pgfpathlineto{\pgfqpoint{3.117359in}{0.767903in}}%
\pgfpathlineto{\pgfqpoint{3.122495in}{0.755704in}}%
\pgfpathlineto{\pgfqpoint{3.126770in}{0.632521in}}%
\pgfpathlineto{\pgfqpoint{3.128480in}{0.722092in}}%
\pgfpathlineto{\pgfqpoint{3.135323in}{0.804938in}}%
\pgfpathlineto{\pgfqpoint{3.138744in}{0.671494in}}%
\pgfpathlineto{\pgfqpoint{3.143019in}{0.741886in}}%
\pgfpathlineto{\pgfqpoint{3.146436in}{0.685280in}}%
\pgfpathlineto{\pgfqpoint{3.151568in}{0.748687in}}%
\pgfpathlineto{\pgfqpoint{3.154987in}{0.646195in}}%
\pgfpathlineto{\pgfqpoint{3.161832in}{0.767296in}}%
\pgfpathlineto{\pgfqpoint{3.162686in}{0.725730in}}%
\pgfpathlineto{\pgfqpoint{3.166956in}{0.775570in}}%
\pgfpathlineto{\pgfqpoint{3.171229in}{0.657460in}}%
\pgfpathlineto{\pgfqpoint{3.176360in}{0.781761in}}%
\pgfpathlineto{\pgfqpoint{3.181495in}{0.704676in}}%
\pgfpathlineto{\pgfqpoint{3.184919in}{0.785216in}}%
\pgfpathlineto{\pgfqpoint{3.189194in}{0.692872in}}%
\pgfpathlineto{\pgfqpoint{3.195182in}{0.771469in}}%
\pgfpathlineto{\pgfqpoint{3.196893in}{0.685169in}}%
\pgfpathlineto{\pgfqpoint{3.202029in}{0.784242in}}%
\pgfpathlineto{\pgfqpoint{3.206306in}{0.793457in}}%
\pgfpathlineto{\pgfqpoint{3.213149in}{0.594951in}}%
\pgfpathlineto{\pgfqpoint{3.214004in}{0.723640in}}%
\pgfpathlineto{\pgfqpoint{3.218280in}{0.756031in}}%
\pgfpathlineto{\pgfqpoint{3.222557in}{0.645908in}}%
\pgfpathlineto{\pgfqpoint{3.226834in}{0.724505in}}%
\pgfpathlineto{\pgfqpoint{3.234531in}{0.769993in}}%
\pgfpathlineto{\pgfqpoint{3.237098in}{0.678188in}}%
\pgfpathlineto{\pgfqpoint{3.243083in}{0.693845in}}%
\pgfpathlineto{\pgfqpoint{3.243939in}{0.791622in}}%
\pgfpathlineto{\pgfqpoint{3.252486in}{0.637272in}}%
\pgfpathlineto{\pgfqpoint{3.256765in}{0.763443in}}%
\pgfpathlineto{\pgfqpoint{3.263612in}{0.675922in}}%
\pgfpathlineto{\pgfqpoint{3.266171in}{0.816091in}}%
\pgfpathlineto{\pgfqpoint{3.269594in}{0.710795in}}%
\pgfpathlineto{\pgfqpoint{3.274725in}{0.783740in}}%
\pgfpathlineto{\pgfqpoint{3.278148in}{0.675204in}}%
\pgfpathlineto{\pgfqpoint{3.283283in}{0.783528in}}%
\pgfpathlineto{\pgfqpoint{3.287561in}{0.672288in}}%
\pgfpathlineto{\pgfqpoint{3.291835in}{0.783524in}}%
\pgfpathlineto{\pgfqpoint{3.296964in}{0.670956in}}%
\pgfpathlineto{\pgfqpoint{3.302092in}{0.675635in}}%
\pgfpathlineto{\pgfqpoint{3.306369in}{0.761285in}}%
\pgfpathlineto{\pgfqpoint{3.311495in}{0.782447in}}%
\pgfpathlineto{\pgfqpoint{3.313205in}{0.636446in}}%
\pgfpathlineto{\pgfqpoint{3.316624in}{0.736596in}}%
\pgfpathlineto{\pgfqpoint{3.323472in}{0.786261in}}%
\pgfpathlineto{\pgfqpoint{3.327748in}{0.635688in}}%
\pgfpathlineto{\pgfqpoint{3.332026in}{0.770639in}}%
\pgfpathlineto{\pgfqpoint{3.333737in}{0.710938in}}%
\pgfpathlineto{\pgfqpoint{3.339724in}{0.635113in}}%
\pgfpathlineto{\pgfqpoint{3.345713in}{0.773304in}}%
\pgfpathlineto{\pgfqpoint{3.346570in}{0.694958in}}%
\pgfpathlineto{\pgfqpoint{3.351703in}{0.656383in}}%
\pgfpathlineto{\pgfqpoint{3.355978in}{0.781258in}}%
\pgfpathlineto{\pgfqpoint{3.359402in}{0.689852in}}%
\pgfpathlineto{\pgfqpoint{3.367101in}{0.683445in}}%
\pgfpathlineto{\pgfqpoint{3.367957in}{0.817894in}}%
\pgfpathlineto{\pgfqpoint{3.374800in}{0.665307in}}%
\pgfpathlineto{\pgfqpoint{3.379076in}{0.631443in}}%
\pgfpathlineto{\pgfqpoint{3.380784in}{0.727350in}}%
\pgfpathlineto{\pgfqpoint{3.388483in}{0.775610in}}%
\pgfpathlineto{\pgfqpoint{3.389340in}{0.657931in}}%
\pgfpathlineto{\pgfqpoint{3.397035in}{0.801052in}}%
\pgfpathlineto{\pgfqpoint{3.398746in}{0.670704in}}%
\pgfpathlineto{\pgfqpoint{3.403879in}{0.819298in}}%
\pgfpathlineto{\pgfqpoint{3.407299in}{0.676249in}}%
\pgfpathlineto{\pgfqpoint{3.410720in}{0.742644in}}%
\pgfpathlineto{\pgfqpoint{3.414994in}{0.655231in}}%
\pgfpathlineto{\pgfqpoint{3.419266in}{0.713962in}}%
\pgfpathlineto{\pgfqpoint{3.423538in}{0.689924in}}%
\pgfpathlineto{\pgfqpoint{3.430381in}{0.797454in}}%
\pgfpathlineto{\pgfqpoint{3.433801in}{0.584408in}}%
\pgfpathlineto{\pgfqpoint{3.437223in}{0.753909in}}%
\pgfpathlineto{\pgfqpoint{3.443211in}{0.684742in}}%
\pgfpathlineto{\pgfqpoint{3.444923in}{0.741495in}}%
\pgfpathlineto{\pgfqpoint{3.450907in}{0.805911in}}%
\pgfpathlineto{\pgfqpoint{3.456036in}{0.709678in}}%
\pgfpathlineto{\pgfqpoint{3.457749in}{0.779423in}}%
\pgfpathlineto{\pgfqpoint{3.462026in}{0.621076in}}%
\pgfpathlineto{\pgfqpoint{3.466302in}{0.748400in}}%
\pgfpathlineto{\pgfqpoint{3.470577in}{0.774349in}}%
\pgfpathlineto{\pgfqpoint{3.475705in}{0.681319in}}%
\pgfpathlineto{\pgfqpoint{3.480840in}{0.739548in}}%
\pgfpathlineto{\pgfqpoint{3.483407in}{0.658793in}}%
\pgfpathlineto{\pgfqpoint{3.490249in}{0.762761in}}%
\pgfpathlineto{\pgfqpoint{3.493668in}{0.671319in}}%
\pgfpathlineto{\pgfqpoint{3.498802in}{0.738001in}}%
\pgfpathlineto{\pgfqpoint{3.503934in}{0.596786in}}%
\pgfpathlineto{\pgfqpoint{3.507357in}{0.775785in}}%
\pgfpathlineto{\pgfqpoint{3.509065in}{0.654616in}}%
\pgfpathlineto{\pgfqpoint{3.514195in}{0.793888in}}%
\pgfpathlineto{\pgfqpoint{3.517615in}{0.669300in}}%
\pgfpathlineto{\pgfqpoint{3.522750in}{0.736632in}}%
\pgfpathlineto{\pgfqpoint{3.527884in}{0.657030in}}%
\pgfpathlineto{\pgfqpoint{3.533869in}{0.743721in}}%
\pgfpathlineto{\pgfqpoint{3.536436in}{0.670848in}}%
\pgfpathlineto{\pgfqpoint{3.539857in}{0.778091in}}%
\pgfpathlineto{\pgfqpoint{3.543277in}{0.640403in}}%
\pgfpathlineto{\pgfqpoint{3.548412in}{0.744335in}}%
\pgfpathlineto{\pgfqpoint{3.552684in}{0.678443in}}%
\pgfpathlineto{\pgfqpoint{3.556103in}{0.759849in}}%
\pgfpathlineto{\pgfqpoint{3.561230in}{0.702449in}}%
\pgfpathlineto{\pgfqpoint{3.567217in}{0.682871in}}%
\pgfpathlineto{\pgfqpoint{3.569782in}{0.807925in}}%
\pgfpathlineto{\pgfqpoint{3.574058in}{0.695788in}}%
\pgfpathlineto{\pgfqpoint{3.580044in}{0.805369in}}%
\pgfpathlineto{\pgfqpoint{3.581755in}{0.681032in}}%
\pgfpathlineto{\pgfqpoint{3.586031in}{0.753367in}}%
\pgfpathlineto{\pgfqpoint{3.592017in}{0.673010in}}%
\pgfpathlineto{\pgfqpoint{3.595438in}{0.810913in}}%
\pgfpathlineto{\pgfqpoint{3.598858in}{0.697407in}}%
\pgfpathlineto{\pgfqpoint{3.605706in}{0.610357in}}%
\pgfpathlineto{\pgfqpoint{3.607419in}{0.749122in}}%
\pgfpathlineto{\pgfqpoint{3.611692in}{0.689924in}}%
\pgfpathlineto{\pgfqpoint{3.615968in}{0.734043in}}%
\pgfpathlineto{\pgfqpoint{3.621959in}{0.679628in}}%
\pgfpathlineto{\pgfqpoint{3.627946in}{0.774924in}}%
\pgfpathlineto{\pgfqpoint{3.628798in}{0.694779in}}%
\pgfpathlineto{\pgfqpoint{3.634787in}{0.732783in}}%
\pgfpathlineto{\pgfqpoint{3.640777in}{0.612080in}}%
\pgfpathlineto{\pgfqpoint{3.641632in}{0.740809in}}%
\pgfpathlineto{\pgfqpoint{3.646770in}{0.660843in}}%
\pgfpathlineto{\pgfqpoint{3.652758in}{0.628563in}}%
\pgfpathlineto{\pgfqpoint{3.657887in}{0.761249in}}%
\pgfpathlineto{\pgfqpoint{3.658742in}{0.670704in}}%
\pgfpathlineto{\pgfqpoint{3.663017in}{0.779998in}}%
\pgfpathlineto{\pgfqpoint{3.669860in}{0.675348in}}%
\pgfpathlineto{\pgfqpoint{3.671571in}{0.754771in}}%
\pgfpathlineto{\pgfqpoint{3.678414in}{0.785866in}}%
\pgfpathlineto{\pgfqpoint{3.680124in}{0.670776in}}%
\pgfpathlineto{\pgfqpoint{3.686109in}{0.758297in}}%
\pgfpathlineto{\pgfqpoint{3.689529in}{0.771972in}}%
\pgfpathlineto{\pgfqpoint{3.695519in}{0.701795in}}%
\pgfpathlineto{\pgfqpoint{3.699794in}{0.739692in}}%
\pgfpathlineto{\pgfqpoint{3.701506in}{0.645800in}}%
\pgfpathlineto{\pgfqpoint{3.705784in}{0.725658in}}%
\pgfpathlineto{\pgfqpoint{3.710060in}{0.654365in}}%
\pgfpathlineto{\pgfqpoint{3.714341in}{0.739907in}}%
\pgfpathlineto{\pgfqpoint{3.720332in}{0.673548in}}%
\pgfpathlineto{\pgfqpoint{3.722900in}{0.724003in}}%
\pgfpathlineto{\pgfqpoint{3.730597in}{0.792811in}}%
\pgfpathlineto{\pgfqpoint{3.731453in}{0.676608in}}%
\pgfpathlineto{\pgfqpoint{3.738291in}{0.796768in}}%
\pgfpathlineto{\pgfqpoint{3.739998in}{0.718207in}}%
\pgfpathlineto{\pgfqpoint{3.745129in}{0.684164in}}%
\pgfpathlineto{\pgfqpoint{3.751968in}{0.780612in}}%
\pgfpathlineto{\pgfqpoint{3.756244in}{0.658003in}}%
\pgfpathlineto{\pgfqpoint{3.757100in}{0.755888in}}%
\pgfpathlineto{\pgfqpoint{3.763091in}{0.648537in}}%
\pgfpathlineto{\pgfqpoint{3.767366in}{0.748831in}}%
\pgfpathlineto{\pgfqpoint{3.769934in}{0.675958in}}%
\pgfpathlineto{\pgfqpoint{3.777634in}{0.611865in}}%
\pgfpathlineto{\pgfqpoint{3.779343in}{0.724577in}}%
\pgfpathlineto{\pgfqpoint{3.785331in}{0.754914in}}%
\pgfpathlineto{\pgfqpoint{3.790465in}{0.691687in}}%
\pgfpathlineto{\pgfqpoint{3.792176in}{0.768844in}}%
\pgfpathlineto{\pgfqpoint{3.795597in}{0.696725in}}%
\pgfpathlineto{\pgfqpoint{3.801584in}{0.786907in}}%
\pgfpathlineto{\pgfqpoint{3.804147in}{0.714249in}}%
\pgfpathlineto{\pgfqpoint{3.809278in}{0.671534in}}%
\pgfpathlineto{\pgfqpoint{3.816117in}{0.765102in}}%
\pgfpathlineto{\pgfqpoint{3.819537in}{0.660883in}}%
\pgfpathlineto{\pgfqpoint{3.821247in}{0.745811in}}%
\pgfpathlineto{\pgfqpoint{3.825520in}{0.673010in}}%
\pgfpathlineto{\pgfqpoint{3.829795in}{0.658721in}}%
\pgfpathlineto{\pgfqpoint{3.834067in}{0.754124in}}%
\pgfpathlineto{\pgfqpoint{3.841763in}{0.780971in}}%
\pgfpathlineto{\pgfqpoint{3.842619in}{0.688053in}}%
\pgfpathlineto{\pgfqpoint{3.847744in}{0.646774in}}%
\pgfpathlineto{\pgfqpoint{3.851170in}{0.748907in}}%
\pgfpathlineto{\pgfqpoint{3.857163in}{0.652570in}}%
\pgfpathlineto{\pgfqpoint{3.859732in}{0.734905in}}%
\pgfpathlineto{\pgfqpoint{3.866582in}{0.675419in}}%
\pgfpathlineto{\pgfqpoint{3.870855in}{0.748260in}}%
\pgfpathlineto{\pgfqpoint{3.873418in}{0.654405in}}%
\pgfpathlineto{\pgfqpoint{3.880263in}{0.666496in}}%
\pgfpathlineto{\pgfqpoint{3.884540in}{0.750957in}}%
\pgfpathlineto{\pgfqpoint{3.885396in}{0.658577in}}%
\pgfpathlineto{\pgfqpoint{3.891380in}{0.763914in}}%
\pgfpathlineto{\pgfqpoint{3.894800in}{0.663041in}}%
\pgfpathlineto{\pgfqpoint{3.899075in}{0.750203in}}%
\pgfpathlineto{\pgfqpoint{3.905916in}{0.785184in}}%
\pgfpathlineto{\pgfqpoint{3.908484in}{0.659335in}}%
\pgfpathlineto{\pgfqpoint{3.912763in}{0.742827in}}%
\pgfpathlineto{\pgfqpoint{3.917037in}{0.775035in}}%
\pgfpathlineto{\pgfqpoint{3.923020in}{0.692912in}}%
\pgfpathlineto{\pgfqpoint{3.924731in}{0.770320in}}%
\pgfpathlineto{\pgfqpoint{3.929865in}{0.648900in}}%
\pgfpathlineto{\pgfqpoint{3.935852in}{0.796808in}}%
\pgfpathlineto{\pgfqpoint{3.939274in}{0.707990in}}%
\pgfpathlineto{\pgfqpoint{3.943555in}{0.746425in}}%
\pgfpathlineto{\pgfqpoint{3.947833in}{0.630725in}}%
\pgfpathlineto{\pgfqpoint{3.952966in}{0.727421in}}%
\pgfpathlineto{\pgfqpoint{3.953820in}{0.639972in}}%
\pgfpathlineto{\pgfqpoint{3.958954in}{0.771254in}}%
\pgfpathlineto{\pgfqpoint{3.962375in}{0.664840in}}%
\pgfpathlineto{\pgfqpoint{3.969212in}{0.666388in}}%
\pgfpathlineto{\pgfqpoint{3.970924in}{0.736923in}}%
\pgfpathlineto{\pgfqpoint{3.976053in}{0.691001in}}%
\pgfpathlineto{\pgfqpoint{3.980331in}{0.749553in}}%
\pgfpathlineto{\pgfqpoint{3.985468in}{0.704500in}}%
\pgfpathlineto{\pgfqpoint{3.991456in}{0.774604in}}%
\pgfpathlineto{\pgfqpoint{3.993166in}{0.679596in}}%
\pgfpathlineto{\pgfqpoint{3.996588in}{0.753550in}}%
\pgfpathlineto{\pgfqpoint{4.004292in}{0.684060in}}%
\pgfpathlineto{\pgfqpoint{4.006858in}{0.751101in}}%
\pgfpathlineto{\pgfqpoint{4.011132in}{0.782734in}}%
\pgfpathlineto{\pgfqpoint{4.017123in}{0.645585in}}%
\pgfpathlineto{\pgfqpoint{4.017979in}{0.772730in}}%
\pgfpathlineto{\pgfqpoint{4.022252in}{0.689238in}}%
\pgfpathlineto{\pgfqpoint{4.027385in}{0.761967in}}%
\pgfpathlineto{\pgfqpoint{4.030804in}{0.671929in}}%
\pgfpathlineto{\pgfqpoint{4.039357in}{0.759270in}}%
\pgfpathlineto{\pgfqpoint{4.044492in}{0.688807in}}%
\pgfpathlineto{\pgfqpoint{4.049623in}{0.652315in}}%
\pgfpathlineto{\pgfqpoint{4.054759in}{0.821524in}}%
\pgfpathlineto{\pgfqpoint{4.058181in}{0.706008in}}%
\pgfpathlineto{\pgfqpoint{4.064169in}{0.762474in}}%
\pgfpathlineto{\pgfqpoint{4.068446in}{0.609061in}}%
\pgfpathlineto{\pgfqpoint{4.069301in}{0.725156in}}%
\pgfpathlineto{\pgfqpoint{4.073578in}{0.671534in}}%
\pgfpathlineto{\pgfqpoint{4.080418in}{0.763770in}}%
\pgfpathlineto{\pgfqpoint{4.082985in}{0.689708in}}%
\pgfpathlineto{\pgfqpoint{4.086408in}{0.765246in}}%
\pgfpathlineto{\pgfqpoint{4.091542in}{0.666460in}}%
\pgfpathlineto{\pgfqpoint{4.098381in}{0.782375in}}%
\pgfpathlineto{\pgfqpoint{4.100093in}{0.707484in}}%
\pgfpathlineto{\pgfqpoint{4.106080in}{0.769993in}}%
\pgfpathlineto{\pgfqpoint{4.108646in}{0.672467in}}%
\pgfpathlineto{\pgfqpoint{4.114636in}{0.742572in}}%
\pgfpathlineto{\pgfqpoint{4.118056in}{0.673836in}}%
\pgfpathlineto{\pgfqpoint{4.122329in}{0.725012in}}%
\pgfpathlineto{\pgfqpoint{4.127462in}{0.637275in}}%
\pgfpathlineto{\pgfqpoint{4.129172in}{0.697663in}}%
\pgfpathlineto{\pgfqpoint{4.136022in}{0.625795in}}%
\pgfpathlineto{\pgfqpoint{4.140300in}{0.711624in}}%
\pgfpathlineto{\pgfqpoint{4.143724in}{0.667325in}}%
\pgfpathlineto{\pgfqpoint{4.145436in}{0.734115in}}%
\pgfpathlineto{\pgfqpoint{4.151424in}{0.679847in}}%
\pgfpathlineto{\pgfqpoint{4.152280in}{0.723033in}}%
\pgfpathlineto{\pgfqpoint{4.156558in}{0.637132in}}%
\pgfpathlineto{\pgfqpoint{4.159124in}{0.772698in}}%
\pgfpathlineto{\pgfqpoint{4.162548in}{0.660452in}}%
\pgfpathlineto{\pgfqpoint{4.168538in}{0.764488in}}%
\pgfpathlineto{\pgfqpoint{4.171962in}{0.659263in}}%
\pgfpathlineto{\pgfqpoint{4.173670in}{0.784138in}}%
\pgfpathlineto{\pgfqpoint{4.177947in}{0.696905in}}%
\pgfpathlineto{\pgfqpoint{4.181364in}{0.685604in}}%
\pgfpathlineto{\pgfqpoint{4.183932in}{0.772945in}}%
\pgfpathlineto{\pgfqpoint{4.187352in}{0.634754in}}%
\pgfpathlineto{\pgfqpoint{4.189920in}{0.717058in}}%
\pgfpathlineto{\pgfqpoint{4.195906in}{0.657608in}}%
\pgfpathlineto{\pgfqpoint{4.196761in}{0.715223in}}%
\pgfpathlineto{\pgfqpoint{4.200182in}{0.738399in}}%
\pgfpathlineto{\pgfqpoint{4.205311in}{0.737498in}}%
\pgfpathlineto{\pgfqpoint{4.209590in}{0.638101in}}%
\pgfpathlineto{\pgfqpoint{4.210442in}{0.764201in}}%
\pgfpathlineto{\pgfqpoint{4.214719in}{0.649977in}}%
\pgfpathlineto{\pgfqpoint{4.218998in}{0.652099in}}%
\pgfpathlineto{\pgfqpoint{4.221561in}{0.767476in}}%
\pgfpathlineto{\pgfqpoint{4.224983in}{0.786835in}}%
\pgfpathlineto{\pgfqpoint{4.228405in}{0.674629in}}%
\pgfpathlineto{\pgfqpoint{4.231826in}{0.644975in}}%
\pgfpathlineto{\pgfqpoint{4.235248in}{0.755561in}}%
\pgfpathlineto{\pgfqpoint{4.237816in}{0.624351in}}%
\pgfpathlineto{\pgfqpoint{4.241236in}{0.748221in}}%
\pgfpathlineto{\pgfqpoint{4.246370in}{0.614601in}}%
\pgfpathlineto{\pgfqpoint{4.248080in}{0.700646in}}%
\pgfpathlineto{\pgfqpoint{4.252359in}{0.658362in}}%
\pgfpathlineto{\pgfqpoint{4.255778in}{0.705936in}}%
\pgfpathlineto{\pgfqpoint{4.259202in}{0.600316in}}%
\pgfpathlineto{\pgfqpoint{4.265191in}{0.760172in}}%
\pgfpathlineto{\pgfqpoint{4.268614in}{0.657464in}}%
\pgfpathlineto{\pgfqpoint{4.272037in}{0.730014in}}%
\pgfpathlineto{\pgfqpoint{4.275458in}{0.733792in}}%
\pgfpathlineto{\pgfqpoint{4.279734in}{0.695644in}}%
\pgfpathlineto{\pgfqpoint{4.284864in}{0.678048in}}%
\pgfpathlineto{\pgfqpoint{4.285716in}{0.739117in}}%
\pgfpathlineto{\pgfqpoint{4.290850in}{0.759701in}}%
\pgfpathlineto{\pgfqpoint{4.293417in}{0.662176in}}%
\pgfpathlineto{\pgfqpoint{4.297691in}{0.784713in}}%
\pgfpathlineto{\pgfqpoint{4.299404in}{0.672001in}}%
\pgfpathlineto{\pgfqpoint{4.307096in}{0.805839in}}%
\pgfpathlineto{\pgfqpoint{4.311374in}{0.685209in}}%
\pgfpathlineto{\pgfqpoint{4.313941in}{0.768126in}}%
\pgfpathlineto{\pgfqpoint{4.318213in}{0.800191in}}%
\pgfpathlineto{\pgfqpoint{4.319924in}{0.689708in}}%
\pgfpathlineto{\pgfqpoint{4.324195in}{0.655952in}}%
\pgfpathlineto{\pgfqpoint{4.326759in}{0.753622in}}%
\pgfpathlineto{\pgfqpoint{4.331037in}{0.662646in}}%
\pgfpathlineto{\pgfqpoint{4.334460in}{0.736600in}}%
\pgfpathlineto{\pgfqpoint{4.337026in}{0.701009in}}%
\pgfpathlineto{\pgfqpoint{4.342159in}{0.672435in}}%
\pgfpathlineto{\pgfqpoint{4.343870in}{0.741064in}}%
\pgfpathlineto{\pgfqpoint{4.350713in}{0.659730in}}%
\pgfpathlineto{\pgfqpoint{4.355844in}{0.754272in}}%
\pgfpathlineto{\pgfqpoint{4.360978in}{0.684921in}}%
\pgfpathlineto{\pgfqpoint{4.366112in}{0.802456in}}%
\pgfpathlineto{\pgfqpoint{4.368678in}{0.701189in}}%
\pgfpathlineto{\pgfqpoint{4.371246in}{0.768844in}}%
\pgfpathlineto{\pgfqpoint{4.377235in}{0.659766in}}%
\pgfpathlineto{\pgfqpoint{4.378945in}{0.731562in}}%
\pgfpathlineto{\pgfqpoint{4.382361in}{0.679089in}}%
\pgfpathlineto{\pgfqpoint{4.385782in}{0.768844in}}%
\pgfpathlineto{\pgfqpoint{4.390915in}{0.774640in}}%
\pgfpathlineto{\pgfqpoint{4.393482in}{0.662431in}}%
\pgfpathlineto{\pgfqpoint{4.395194in}{0.748260in}}%
\pgfpathlineto{\pgfqpoint{4.398612in}{0.696258in}}%
\pgfpathlineto{\pgfqpoint{4.402032in}{0.721916in}}%
\pgfpathlineto{\pgfqpoint{4.406311in}{0.628962in}}%
\pgfpathlineto{\pgfqpoint{4.408878in}{0.713962in}}%
\pgfpathlineto{\pgfqpoint{4.414012in}{0.753837in}}%
\pgfpathlineto{\pgfqpoint{4.416578in}{0.671929in}}%
\pgfpathlineto{\pgfqpoint{4.419998in}{0.735376in}}%
\pgfpathlineto{\pgfqpoint{4.422562in}{0.690211in}}%
\pgfpathlineto{\pgfqpoint{4.427694in}{0.740019in}}%
\pgfpathlineto{\pgfqpoint{4.430262in}{0.651237in}}%
\pgfpathlineto{\pgfqpoint{4.432826in}{0.784713in}}%
\pgfpathlineto{\pgfqpoint{4.437103in}{0.653288in}}%
\pgfpathlineto{\pgfqpoint{4.442241in}{0.767512in}}%
\pgfpathlineto{\pgfqpoint{4.444808in}{0.679161in}}%
\pgfpathlineto{\pgfqpoint{4.449081in}{0.735878in}}%
\pgfpathlineto{\pgfqpoint{4.449936in}{0.654369in}}%
\pgfpathlineto{\pgfqpoint{4.455924in}{0.773847in}}%
\pgfpathlineto{\pgfqpoint{4.458489in}{0.669412in}}%
\pgfpathlineto{\pgfqpoint{4.461057in}{0.756861in}}%
\pgfpathlineto{\pgfqpoint{4.463622in}{0.688667in}}%
\pgfpathlineto{\pgfqpoint{4.468754in}{0.726380in}}%
\pgfpathlineto{\pgfqpoint{4.473032in}{0.612264in}}%
\pgfpathlineto{\pgfqpoint{4.476450in}{0.748619in}}%
\pgfpathlineto{\pgfqpoint{4.478161in}{0.681431in}}%
\pgfpathlineto{\pgfqpoint{4.480727in}{0.726560in}}%
\pgfpathlineto{\pgfqpoint{4.485001in}{0.646343in}}%
\pgfpathlineto{\pgfqpoint{4.488421in}{0.751643in}}%
\pgfpathlineto{\pgfqpoint{4.491842in}{0.664266in}}%
\pgfpathlineto{\pgfqpoint{4.496121in}{0.733038in}}%
\pgfpathlineto{\pgfqpoint{4.497827in}{0.687730in}}%
\pgfpathlineto{\pgfqpoint{4.502953in}{0.766148in}}%
\pgfpathlineto{\pgfqpoint{4.505519in}{0.681323in}}%
\pgfpathlineto{\pgfqpoint{4.509790in}{0.742213in}}%
\pgfpathlineto{\pgfqpoint{4.512354in}{0.745452in}}%
\pgfpathlineto{\pgfqpoint{4.514921in}{0.664481in}}%
\pgfpathlineto{\pgfqpoint{4.518344in}{0.720225in}}%
\pgfpathlineto{\pgfqpoint{4.523473in}{0.673441in}}%
\pgfpathlineto{\pgfqpoint{4.526897in}{0.747646in}}%
\pgfpathlineto{\pgfqpoint{4.534596in}{0.583798in}}%
\pgfpathlineto{\pgfqpoint{4.535450in}{0.762837in}}%
\pgfpathlineto{\pgfqpoint{4.538868in}{0.585561in}}%
\pgfpathlineto{\pgfqpoint{4.543143in}{0.764775in}}%
\pgfpathlineto{\pgfqpoint{4.548281in}{0.748835in}}%
\pgfpathlineto{\pgfqpoint{4.549136in}{0.669340in}}%
\pgfpathlineto{\pgfqpoint{4.554268in}{0.634108in}}%
\pgfpathlineto{\pgfqpoint{4.558543in}{0.730804in}}%
\pgfpathlineto{\pgfqpoint{4.559399in}{0.666564in}}%
\pgfpathlineto{\pgfqpoint{4.562820in}{0.642777in}}%
\pgfpathlineto{\pgfqpoint{4.568805in}{0.737426in}}%
\pgfpathlineto{\pgfqpoint{4.570513in}{0.612479in}}%
\pgfpathlineto{\pgfqpoint{4.574795in}{0.714537in}}%
\pgfpathlineto{\pgfqpoint{4.579072in}{0.597185in}}%
\pgfpathlineto{\pgfqpoint{4.580783in}{0.682185in}}%
\pgfpathlineto{\pgfqpoint{4.584204in}{0.754268in}}%
\pgfpathlineto{\pgfqpoint{4.588482in}{0.633171in}}%
\pgfpathlineto{\pgfqpoint{4.592755in}{0.801986in}}%
\pgfpathlineto{\pgfqpoint{4.595324in}{0.662140in}}%
\pgfpathlineto{\pgfqpoint{4.597035in}{0.768445in}}%
\pgfpathlineto{\pgfqpoint{4.600455in}{0.691180in}}%
\pgfpathlineto{\pgfqpoint{4.605589in}{0.654293in}}%
\pgfpathlineto{\pgfqpoint{4.609011in}{0.684379in}}%
\pgfpathlineto{\pgfqpoint{4.610723in}{0.791335in}}%
\pgfpathlineto{\pgfqpoint{4.615000in}{0.660412in}}%
\pgfpathlineto{\pgfqpoint{4.617565in}{0.790002in}}%
\pgfpathlineto{\pgfqpoint{4.623549in}{0.757328in}}%
\pgfpathlineto{\pgfqpoint{4.624405in}{0.605786in}}%
\pgfpathlineto{\pgfqpoint{4.627827in}{0.726376in}}%
\pgfpathlineto{\pgfqpoint{4.631246in}{0.763299in}}%
\pgfpathlineto{\pgfqpoint{4.637235in}{0.788455in}}%
\pgfpathlineto{\pgfqpoint{4.638091in}{0.662642in}}%
\pgfpathlineto{\pgfqpoint{4.641512in}{0.745739in}}%
\pgfpathlineto{\pgfqpoint{4.644931in}{0.769562in}}%
\pgfpathlineto{\pgfqpoint{4.648350in}{0.704859in}}%
\pgfpathlineto{\pgfqpoint{4.654343in}{0.737354in}}%
\pgfpathlineto{\pgfqpoint{4.657762in}{0.618958in}}%
\pgfpathlineto{\pgfqpoint{4.660326in}{0.717600in}}%
\pgfpathlineto{\pgfqpoint{4.664605in}{0.731203in}}%
\pgfpathlineto{\pgfqpoint{4.666317in}{0.652354in}}%
\pgfpathlineto{\pgfqpoint{4.671449in}{0.613309in}}%
\pgfpathlineto{\pgfqpoint{4.672305in}{0.690394in}}%
\pgfpathlineto{\pgfqpoint{4.677437in}{0.749808in}}%
\pgfpathlineto{\pgfqpoint{4.681713in}{0.671390in}}%
\pgfpathlineto{\pgfqpoint{4.683423in}{0.607190in}}%
\pgfpathlineto{\pgfqpoint{4.686845in}{0.753837in}}%
\pgfpathlineto{\pgfqpoint{4.689413in}{0.702266in}}%
\pgfpathlineto{\pgfqpoint{4.695398in}{0.696833in}}%
\pgfpathlineto{\pgfqpoint{4.696255in}{0.777157in}}%
\pgfpathlineto{\pgfqpoint{4.700535in}{0.659439in}}%
\pgfpathlineto{\pgfqpoint{4.704809in}{0.740019in}}%
\pgfpathlineto{\pgfqpoint{4.710801in}{0.658904in}}%
\pgfpathlineto{\pgfqpoint{4.713368in}{0.768054in}}%
\pgfpathlineto{\pgfqpoint{4.717644in}{0.826387in}}%
\pgfpathlineto{\pgfqpoint{4.720210in}{0.686936in}}%
\pgfpathlineto{\pgfqpoint{4.724492in}{0.766610in}}%
\pgfpathlineto{\pgfqpoint{4.727918in}{0.689493in}}%
\pgfpathlineto{\pgfqpoint{4.731339in}{0.762653in}}%
\pgfpathlineto{\pgfqpoint{4.734766in}{0.637706in}}%
\pgfpathlineto{\pgfqpoint{4.739903in}{0.792883in}}%
\pgfpathlineto{\pgfqpoint{4.742472in}{0.703383in}}%
\pgfpathlineto{\pgfqpoint{4.746750in}{0.774062in}}%
\pgfpathlineto{\pgfqpoint{4.748462in}{0.688376in}}%
\pgfpathlineto{\pgfqpoint{4.751882in}{0.785973in}}%
\pgfpathlineto{\pgfqpoint{4.756154in}{0.680565in}}%
\pgfpathlineto{\pgfqpoint{4.760428in}{0.784857in}}%
\pgfpathlineto{\pgfqpoint{4.762995in}{0.667609in}}%
\pgfpathlineto{\pgfqpoint{4.765556in}{0.769490in}}%
\pgfpathlineto{\pgfqpoint{4.768121in}{0.710364in}}%
\pgfpathlineto{\pgfqpoint{4.774110in}{0.730625in}}%
\pgfpathlineto{\pgfqpoint{4.774966in}{0.671749in}}%
\pgfpathlineto{\pgfqpoint{4.779249in}{0.748472in}}%
\pgfpathlineto{\pgfqpoint{4.782675in}{0.649331in}}%
\pgfpathlineto{\pgfqpoint{4.787806in}{0.650049in}}%
\pgfpathlineto{\pgfqpoint{4.790374in}{0.743003in}}%
\pgfpathlineto{\pgfqpoint{4.792084in}{0.638496in}}%
\pgfpathlineto{\pgfqpoint{4.795506in}{0.697802in}}%
\pgfpathlineto{\pgfqpoint{4.795506in}{0.697802in}}%
\pgfusepath{stroke}%
\end{pgfscope}%
\begin{pgfscope}%
\pgfsetrectcap%
\pgfsetmiterjoin%
\pgfsetlinewidth{0.803000pt}%
\definecolor{currentstroke}{rgb}{0.000000,0.000000,0.000000}%
\pgfsetstrokecolor{currentstroke}%
\pgfsetdash{}{0pt}%
\pgfpathmoveto{\pgfqpoint{0.484581in}{0.539544in}}%
\pgfpathlineto{\pgfqpoint{0.484581in}{1.114166in}}%
\pgfusepath{stroke}%
\end{pgfscope}%
\begin{pgfscope}%
\pgfsetrectcap%
\pgfsetmiterjoin%
\pgfsetlinewidth{0.803000pt}%
\definecolor{currentstroke}{rgb}{0.000000,0.000000,0.000000}%
\pgfsetstrokecolor{currentstroke}%
\pgfsetdash{}{0pt}%
\pgfpathmoveto{\pgfqpoint{5.000788in}{0.539544in}}%
\pgfpathlineto{\pgfqpoint{5.000788in}{1.114166in}}%
\pgfusepath{stroke}%
\end{pgfscope}%
\begin{pgfscope}%
\pgfsetrectcap%
\pgfsetmiterjoin%
\pgfsetlinewidth{0.803000pt}%
\definecolor{currentstroke}{rgb}{0.000000,0.000000,0.000000}%
\pgfsetstrokecolor{currentstroke}%
\pgfsetdash{}{0pt}%
\pgfpathmoveto{\pgfqpoint{0.484581in}{0.539544in}}%
\pgfpathlineto{\pgfqpoint{5.000788in}{0.539544in}}%
\pgfusepath{stroke}%
\end{pgfscope}%
\begin{pgfscope}%
\pgfsetrectcap%
\pgfsetmiterjoin%
\pgfsetlinewidth{0.803000pt}%
\definecolor{currentstroke}{rgb}{0.000000,0.000000,0.000000}%
\pgfsetstrokecolor{currentstroke}%
\pgfsetdash{}{0pt}%
\pgfpathmoveto{\pgfqpoint{0.484581in}{1.114166in}}%
\pgfpathlineto{\pgfqpoint{5.000788in}{1.114166in}}%
\pgfusepath{stroke}%
\end{pgfscope}%
\end{pgfpicture}%
\makeatother%
\endgroup%

    \includegraphics[width=0.75\textwidth]{example-image-golden}
    \caption{Voltage noise of an LM399, measured with a bandwidth of \qtyrange{0.1}{10}{\Hz}.}
    \label{fig:noise_lm399}
\end{figure}

Measuring two references against each other would then result in around \qty{2.1}{\micro \volt} of noise. This make distinguishing the jumps possible, but challenging.

A third option is to use a high-pass filter and an amplifier. Additionally, the signal can be low-pass filtered to remove any excess high frequency noise. This approach also requires less resolution than directly measuring the voltage, because the signal-to-noise-ratio is improved du the amplifier. It is therefore possible to use an off-the-shelf analog-to-digital converter (ADC). One such circuit, along with some examples, is demonstrated in \cite{technote_ti_popcorn_noise,kay2012operational}. It must be noted, that due to the high-pass filtering, it not possible to measure slow voltage drifts using this method.

The forth and final option presented here, is approaching the problem in the frequency domain and requires a low-noise amplifier with a low frequency cutoff. As it was already discussed in section \ref{sec:theory_burst_noise}, popcorn noise is found to have a frequency dependence of $1/f^2$. This can be used to distinguish it from other random noise processes that show a frequency dependence of $1/f$. A good example of an op-amp, that has excessive burst in comparison to a good sample is given in \textit{The Art of Electronics} \citep[p. 478]{horowitz1989}. Going to frequencies below \qty{10}{\Hz}, one can sort the references by their noise spectrum.

In this work only options one and two were were tested, as it was said above, with options three and four there is a chicken and egg problem. One needs a number on known good devices to compare other DUTs to. At the start of the evaluation, most of the data available about the LM399 was from the data sheet. Compiling a dataset of the performance of dozens of LM399 is expensive and time consuming and companies typically treat such data as a closely guarded secret.

The next section deals with the choice of multimeter to satisfy the requirements test the Zener diodes according to options one and two, so either directly measuring the output voltage or difference of a known good sample against the DUT.

\subsection{Choosing a Multimeter for Testing Zener Diodes}
The DMM used plays an important role for the test setup. In this section, some of the challenges, that can be encountered will be discussed. The expected amplitude of the popcorn noise is around \qty[per-mode=symbol]{0.5}{\micro\volt \per \volt} or \qty{3.5}{\micro\volt} of the output voltage, when considering the \qty{7}{\volt} Zener voltage of the LM399 diode.

The \qty{7}{\volt} will typically be measured on the \qty{10}{\volt} range. It is not a trivial task, because a signal-to-noise-ratio of \qty[per-mode=symbol]{0.35}{\micro\volt \per \volt} or more than \qty{130}{\decibel} is required. This calls for a device, that not only has the required resolution, but also the stability over time and temperature to ensure the measurement will not be distorted by the DMM.

Therefore, a voltmeter with lower noise and a more stable reference, than the DUT is mandatory. This only leaves the class of very low noise \num{7.5} or \num{8.5} digit multimeters. These multimeters feature a different type of voltage reference, because the LM399 is not suitable due to its noise. The only Zener diodes that meet those requirements are the Analog Devices LTZ1000 \cite{datasheet_LTZ1000}, the Motorola SZA263 (out of production) and the Linear Technology (LT) LTFLU-1, a proprietary design by Fluke and LT. The LTZ1000, for example, is specified for a typical noise of \qty{1.2}{\micro\volt_{pp}} in a frequency range of \qtyrange{0.1}{10}{\Hz} \cite{datasheet_LTZ1000}. Additionally, in comparison to the LM399, those Zener diodes do not suffer from the popcorn noise issue.

The equipment manufacturers typically have a preference for one of those diodes. Keysight utilizes the LTZ1000, Fluke uses the SZA263 (in older devices) or the LTFLU-1 in newer model, while Keithley employs the LTZ1000 in their \device{Model 2002} and the LTFLU-1 in the newer \device{DMM7510}, because they were bought by Fortive, the same company that owns Fluke. To sum it it up, Keysight uses the LTZ1000 and Fluke/Keithley the LTFLU-1 in their top end meters.

Comparing only \num{7.5} and \num{8.5} digit voltmeters, narrows down the choice of multimeters considerably. The market for high-end \num{8.5} digit DMMs is limited and therefore every device on the market caters for a certain niche. It is therefore prudent to look at their specifications to choose the correct device for this purpose. In table \ref{tab:list_of_dmms} a list of popular \num{8.5} DMMs can be found. Several models included in the table, are already discontinued, but these DMMs can still be acquired on the second-hand market.

\begin{table}[h]
    \centering
    \begin{tabular}{ |l|l|l| }
        \hline
        Manufacturer & Model & Remarks \\
        \hline
        Advantest & \device{R6581} & Discontinued. Scanner cards available. \\
        Datron/Wavetek & \device{1812} & Discontinued. Wavetek was bought by Fluke. \\
        Fluke & \device{8508A} & Discontinued. \qty{20}{\volt} range. \\
        Fluke & \device{8588A} & In production. \\
        Keithley/Tektronix & \device{2002} & In production. Scanner card available. \qty{20}{\volt} range. \\
        Keysight & \device{3458A} & In Production. \\
        Solartron & \device{7081} & Discontinued. Slow. \\
        Transmille & \device{8104} & In Production. External scanner available. Slow. \\
        \hline
    \end{tabular}
    \caption{Overview of \num{8.5} digit multimeters.}
    \label{tab:list_of_dmms}
\end{table}

While the author has not tested every multimeter in table \ref{tab:list_of_dmms}, it is possible to judge some of them apriori by their specifications. The \device{Solartron 7081} (also sold as \device{Guildline 9578}) is a less optimal choice, because a conversion takes \qty{52}{\s} for \num{8.5} digits. The discontinued \device{Fluke 8508A} and the \device{Wavetek 1812} multimeter are very similar devices, because Fluke bought Wavetek in 2000 and as a result, the \device{Fluke 8508A} is more of an update to the \device{Wavetek 1812} than a new device. They are both in included in the list, because it is very rare to see one of the Fluke devices on the second hand market, while the \device{Wavetek 1812} can be found with a bit of patience. Again they are fairly slow, taking \qty{25}{\second} for a conversion at \num{8.5} digits.

The other multimeters are still in production and similar in price, but their field of use is slightly is different. The \device{Fluke 8588A} excels at stability and features a modern user interface, whereas the \device{Keysight 3458A} is unbeaten in linearity and noise. A detailed comparison of those two meters can be found in the work of \citeauthor*{article_fluke_8588A_noise} \cite{article_fluke_8588A_noise}. The \device{Keithley Model 2002} focuses on its scanning capability and the \device{Transmille 8104} does have electrometer functions. Unfortunately, the \device{8104} is also fairly slow at \num{8.5} digit with conversions taking \qty{4}{\s} at its fastest setting \cite{datasheet_transmille8104}, so it will not be considered.

To narrow it down even further, several \num{7.5} and \num{8.5} digit multimeters were tested. The results of those tests will be discussed here to give an impression of the performance of these devices. The tested multimeters are the \device{Keysight 3458A}, the \device{Keithley Model 2002}, the \device{Keysight 34470A} and a \device{Keithley DMM6500}. The \device{3458A} was chosen, because it is very fast and already used in section \ref{} of this work. The \device{Model 2002} was chosen for its internal scanning unit. The \device{34470A} was chosen as a lower-end and cheaper alternative and because it is a fairly low noise device. Finally the \device{DMM6500} is on the list to compare a DMM with an LM399 reference. A \device{Fluke 8588A} was not tested, because it was not released at the time of testing and the older model \device{8508A} is considered too slow as mentioned above.

\minisec{The tests}

Two test were run on this selection of devices. The first one was done using a \device{Fluke 5440B} calibrator supplying \qty{10}{\volt} to all mulimeters and taking readings over the course of a week. This data was used to estimate the noise and the stability of the multimeters, including burst noise. The noise of the DMM at \qty{10}{\volt} is typically not found in the datasheet, because the noise performance is usually quoted for shorted inputs, which does not include the internal reference noise. This test allows to check for popcorn noise of the internal reference. The calibrator has a specified output noise of \qty{< 1.5}{\micro \volt} within a bandwidth of \qtyrange{0.1}{10}{\Hz} at \qty{1}{\volt} and is stable to within \qty{5}{\micro \volt_{rms}} over \qty{30}{\day}, a specification far superior to the LM399.

The second test was done using a known bad LM399 voltage reference instead of the calibrator. This test was done to see how well a DMM can make out the popcorn noise.

Based on these two tests, a multimeter was chosen for an automated test setup to bin the LM399s.

\minisec{Test Setup}

The tests were done in a stable and monitored lab environment, with a temperature deviation of at most $\Delta T = \qty{\pm 0.2}{\kelvin}$. All multimeters were connected to the same DUT. Although this might potentially cause interference between the multimeters due to the pump out current spikes caused by the switching interals, no ill effects, like voltage offsets or increased noise, were observed during the setup of the tests. A more detailed discussion of the pump out current of the \device{3458A} can be found in \cite{article_3458A_input_mpedance}.

The three \num{8.5} and \num{7.5} digit multimeters were connected using shielded cables, either Pomona 1167-60 or self-made cables. See section \ref{} for details on the self-made cables. The GUARD terminal of the calibrator was connected to chassis GROUND at the calibrator and then connected to the cable shield. On the \device{3458A}, the shield was connected to the GUARD terminal and the GUARD switch was set to open according to the manual \cite{manual_keysight3458a}. For the other multimeters, that do not have a GUARD terminal, the shield was left floating at the DMM side. Additionally the \device{Fluke 5440B}, the \device{HP 3458A} and the \device{Keysight 34470A} have an autocalibration routine, which was run once prior to the measurement. The detailed settings used for the DMMs can be found in the appendix \ref{appendix:dmm_test} on page \pageref{appendix:dmm_test}, a summary ist given in table \ref{tab:dmm_settings_concise} to show the important differences.

\begin{table}[ht]
    \centering
    \begin{tabular}{lcc}
        \toprule
        DMM& Integration time in \unit{NPLC}& Conversion time in \unit{\s}\\
        \midrule
        \device{HP 3458A}& 100 & \qty{0}{\s}\\
        \device{Keithley Model 2002} & 40& \qty{0}{\s}\\
        \device{Keysight 34470A}& 100    & \qty{0}{\s}\\
        \device{Keithley DMM6500}& 90& \qty{0}{\s}\\
        \bottomrule
    \end{tabular}
    \caption{Concise list of differences in the settings used for comparing the DMMs.}
    \label{tab:dmm_settings_concise}
\end{table}

All DMMs were configured to have a similar conversion time. This lead to different integration times, which are given in power line cycles at \qty{50}{\Hz}. The \device{Model 2002} takes considerable longer for a measurement than the Keysight multimeters. The reason is the auto-zero function, which is shown in figure \ref{fig:dmm_autozero_comparison}. The \device{Model 2002} does three steps when doing auto-zeroing, it measures the signal, the zero point for an offset compensation and also the reference voltage for a gain correction. In comparison, the \device{3458A} only corrects for the offset drift. The gain is adjusted when using the ACAL function. The former auto-zero routine, therefore takes longer by one half, but results in more stable measurements.

These measurements were done by measuring the output voltage of a pre-production version of the reference PCB for the digital current driver. The reference board was kept at \qty{23}{\celsius} in a custom thermal chamber. The chamber is detailed in section \ref{}. Additionally, a \qty{500}{\g}  bag of Bentonite desiccant was added to keep the references at a low humidity of around \qty{20}{\percent} relative humidity. The reference board inserted into a motherboard holding up to 4 reference modules. The motherboard, also called LM399 breakout board, provides the voltage regulators and the operational amplifier for the kelvin sensed pins of the reference. The multimeter was directly connected to the reference via a DB9 connector, without an other components in between the reference and the DMM like buffers, multiplexers or filters. The DMM itself was exposed to the ambient temperature of the lab. The setup is shown in figure \ref{fig:lm399_vs_34470a_setup}.

\begin{figure}[ht]
    \centering
    \resizebox {0.8\textwidth} {!} {
        \import{figures/}{34470A_vs_LM399.tex}
    } % resizebox
    \caption{Measurement setup for tesing an LM399 reference board with the \device{Keysight 34470A}}
    \label{fig:lm399_vs_34470a_setup}
\end{figure}

The reference boards amplify the Zener voltage to \qty{10}{\volt}, which improves the signal to noise ratio, because it makes use of the full DMM range. The \qty{10}{\volt} range is typically the lowest (relative) noise and lowest drift range those multimeter because no internal pre-amplifiers of attenuators are required. It is important to keep the temperature drift of the DMM low or at least predictable, because the device is exposed to the ambient laboratory and not in a temperature controlled environment like the references.

\newpage
The reference is a negative \qty{10}{\volt} reference that uses a self-biasing technique to derive its \qty{1}{\mA} Zener current from its own \qty{-10}{\volt} output. The details of this circuit are discussed in section \ref{}.

% \begin{figure}[h]
%     \centering
%     \scalebox{0.7}{%
%         \import{figures/}{lm399_reference_circuit.tex}
%     } % scalebox
%     \caption{Self-biased LM399 negative voltage reference.}
%     \label{fig:lm399_negative_10V}
% \end{figure}

With the amplified output the expected burst noise step size of about \qty[per-mode=symbol]{0.5}{\micro\volt \per \volt}, becomes \qty{5}{\micro\volt}. The resolution of the \qty{10}{\volt} range of the \device{34470A} is \qty{100}{\nano \volt}, but the measurement is not limited by quantization. See section \ref{} of this work for a detailed characterization.

\begin{figure}[ht]
    \centering
    %% Creator: Matplotlib, PGF backend
%%
%% To include the figure in your LaTeX document, write
%%   \input{<filename>.pgf}
%%
%% Make sure the required packages are loaded in your preamble
%%   \usepackage{pgf}
%%
%% Also ensure that all the required font packages are loaded; for instance,
%% the lmodern package is sometimes necessary when using math font.
%%   \usepackage{lmodern}
%%
%% Figures using additional raster images can only be included by \input if
%% they are in the same directory as the main LaTeX file. For loading figures
%% from other directories you can use the `import` package
%%   \usepackage{import}
%%
%% and then include the figures with
%%   \import{<path to file>}{<filename>.pgf}
%%
%% Matplotlib used the following preamble
%%   \usepackage{fontspec}
%%
\begingroup%
\makeatletter%
\begin{pgfpicture}%
\pgfpathrectangle{\pgfpointorigin}{\pgfqpoint{5.200000in}{3.210000in}}%
\pgfusepath{use as bounding box, clip}%
\begin{pgfscope}%
\pgfsetbuttcap%
\pgfsetmiterjoin%
\definecolor{currentfill}{rgb}{1.000000,1.000000,1.000000}%
\pgfsetfillcolor{currentfill}%
\pgfsetlinewidth{0.000000pt}%
\definecolor{currentstroke}{rgb}{1.000000,1.000000,1.000000}%
\pgfsetstrokecolor{currentstroke}%
\pgfsetdash{}{0pt}%
\pgfpathmoveto{\pgfqpoint{0.000000in}{0.000000in}}%
\pgfpathlineto{\pgfqpoint{5.200000in}{0.000000in}}%
\pgfpathlineto{\pgfqpoint{5.200000in}{3.210000in}}%
\pgfpathlineto{\pgfqpoint{0.000000in}{3.210000in}}%
\pgfpathlineto{\pgfqpoint{0.000000in}{0.000000in}}%
\pgfpathclose%
\pgfusepath{fill}%
\end{pgfscope}%
\begin{pgfscope}%
\pgfsetbuttcap%
\pgfsetmiterjoin%
\definecolor{currentfill}{rgb}{1.000000,1.000000,1.000000}%
\pgfsetfillcolor{currentfill}%
\pgfsetlinewidth{0.000000pt}%
\definecolor{currentstroke}{rgb}{0.000000,0.000000,0.000000}%
\pgfsetstrokecolor{currentstroke}%
\pgfsetstrokeopacity{0.000000}%
\pgfsetdash{}{0pt}%
\pgfpathmoveto{\pgfqpoint{0.633813in}{0.538014in}}%
\pgfpathlineto{\pgfqpoint{4.507477in}{0.538014in}}%
\pgfpathlineto{\pgfqpoint{4.507477in}{2.936535in}}%
\pgfpathlineto{\pgfqpoint{0.633813in}{2.936535in}}%
\pgfpathlineto{\pgfqpoint{0.633813in}{0.538014in}}%
\pgfpathclose%
\pgfusepath{fill}%
\end{pgfscope}%
\begin{pgfscope}%
\pgfsetbuttcap%
\pgfsetroundjoin%
\definecolor{currentfill}{rgb}{0.000000,0.000000,0.000000}%
\pgfsetfillcolor{currentfill}%
\pgfsetlinewidth{0.803000pt}%
\definecolor{currentstroke}{rgb}{0.000000,0.000000,0.000000}%
\pgfsetstrokecolor{currentstroke}%
\pgfsetdash{}{0pt}%
\pgfsys@defobject{currentmarker}{\pgfqpoint{0.000000in}{-0.048611in}}{\pgfqpoint{0.000000in}{0.000000in}}{%
\pgfpathmoveto{\pgfqpoint{0.000000in}{0.000000in}}%
\pgfpathlineto{\pgfqpoint{0.000000in}{-0.048611in}}%
\pgfusepath{stroke,fill}%
}%
\begin{pgfscope}%
\pgfsys@transformshift{0.809808in}{0.538014in}%
\pgfsys@useobject{currentmarker}{}%
\end{pgfscope}%
\end{pgfscope}%
\begin{pgfscope}%
\definecolor{textcolor}{rgb}{0.000000,0.000000,0.000000}%
\pgfsetstrokecolor{textcolor}%
\pgfsetfillcolor{textcolor}%
\pgftext[x=0.809808in,y=0.440792in,,top]{\color{textcolor}\rmfamily\fontsize{8.000000}{9.600000}\selectfont \(\displaystyle {00{:}00}\)}%
\end{pgfscope}%
\begin{pgfscope}%
\pgfsetbuttcap%
\pgfsetroundjoin%
\definecolor{currentfill}{rgb}{0.000000,0.000000,0.000000}%
\pgfsetfillcolor{currentfill}%
\pgfsetlinewidth{0.803000pt}%
\definecolor{currentstroke}{rgb}{0.000000,0.000000,0.000000}%
\pgfsetstrokecolor{currentstroke}%
\pgfsetdash{}{0pt}%
\pgfsys@defobject{currentmarker}{\pgfqpoint{0.000000in}{-0.048611in}}{\pgfqpoint{0.000000in}{0.000000in}}{%
\pgfpathmoveto{\pgfqpoint{0.000000in}{0.000000in}}%
\pgfpathlineto{\pgfqpoint{0.000000in}{-0.048611in}}%
\pgfusepath{stroke,fill}%
}%
\begin{pgfscope}%
\pgfsys@transformshift{1.250023in}{0.538014in}%
\pgfsys@useobject{currentmarker}{}%
\end{pgfscope}%
\end{pgfscope}%
\begin{pgfscope}%
\definecolor{textcolor}{rgb}{0.000000,0.000000,0.000000}%
\pgfsetstrokecolor{textcolor}%
\pgfsetfillcolor{textcolor}%
\pgftext[x=1.250023in,y=0.440792in,,top]{\color{textcolor}\rmfamily\fontsize{8.000000}{9.600000}\selectfont \(\displaystyle {03{:}00}\)}%
\end{pgfscope}%
\begin{pgfscope}%
\pgfsetbuttcap%
\pgfsetroundjoin%
\definecolor{currentfill}{rgb}{0.000000,0.000000,0.000000}%
\pgfsetfillcolor{currentfill}%
\pgfsetlinewidth{0.803000pt}%
\definecolor{currentstroke}{rgb}{0.000000,0.000000,0.000000}%
\pgfsetstrokecolor{currentstroke}%
\pgfsetdash{}{0pt}%
\pgfsys@defobject{currentmarker}{\pgfqpoint{0.000000in}{-0.048611in}}{\pgfqpoint{0.000000in}{0.000000in}}{%
\pgfpathmoveto{\pgfqpoint{0.000000in}{0.000000in}}%
\pgfpathlineto{\pgfqpoint{0.000000in}{-0.048611in}}%
\pgfusepath{stroke,fill}%
}%
\begin{pgfscope}%
\pgfsys@transformshift{1.690237in}{0.538014in}%
\pgfsys@useobject{currentmarker}{}%
\end{pgfscope}%
\end{pgfscope}%
\begin{pgfscope}%
\definecolor{textcolor}{rgb}{0.000000,0.000000,0.000000}%
\pgfsetstrokecolor{textcolor}%
\pgfsetfillcolor{textcolor}%
\pgftext[x=1.690237in,y=0.440792in,,top]{\color{textcolor}\rmfamily\fontsize{8.000000}{9.600000}\selectfont \(\displaystyle {06{:}00}\)}%
\end{pgfscope}%
\begin{pgfscope}%
\pgfsetbuttcap%
\pgfsetroundjoin%
\definecolor{currentfill}{rgb}{0.000000,0.000000,0.000000}%
\pgfsetfillcolor{currentfill}%
\pgfsetlinewidth{0.803000pt}%
\definecolor{currentstroke}{rgb}{0.000000,0.000000,0.000000}%
\pgfsetstrokecolor{currentstroke}%
\pgfsetdash{}{0pt}%
\pgfsys@defobject{currentmarker}{\pgfqpoint{0.000000in}{-0.048611in}}{\pgfqpoint{0.000000in}{0.000000in}}{%
\pgfpathmoveto{\pgfqpoint{0.000000in}{0.000000in}}%
\pgfpathlineto{\pgfqpoint{0.000000in}{-0.048611in}}%
\pgfusepath{stroke,fill}%
}%
\begin{pgfscope}%
\pgfsys@transformshift{2.130452in}{0.538014in}%
\pgfsys@useobject{currentmarker}{}%
\end{pgfscope}%
\end{pgfscope}%
\begin{pgfscope}%
\definecolor{textcolor}{rgb}{0.000000,0.000000,0.000000}%
\pgfsetstrokecolor{textcolor}%
\pgfsetfillcolor{textcolor}%
\pgftext[x=2.130452in,y=0.440792in,,top]{\color{textcolor}\rmfamily\fontsize{8.000000}{9.600000}\selectfont \(\displaystyle {09{:}00}\)}%
\end{pgfscope}%
\begin{pgfscope}%
\pgfsetbuttcap%
\pgfsetroundjoin%
\definecolor{currentfill}{rgb}{0.000000,0.000000,0.000000}%
\pgfsetfillcolor{currentfill}%
\pgfsetlinewidth{0.803000pt}%
\definecolor{currentstroke}{rgb}{0.000000,0.000000,0.000000}%
\pgfsetstrokecolor{currentstroke}%
\pgfsetdash{}{0pt}%
\pgfsys@defobject{currentmarker}{\pgfqpoint{0.000000in}{-0.048611in}}{\pgfqpoint{0.000000in}{0.000000in}}{%
\pgfpathmoveto{\pgfqpoint{0.000000in}{0.000000in}}%
\pgfpathlineto{\pgfqpoint{0.000000in}{-0.048611in}}%
\pgfusepath{stroke,fill}%
}%
\begin{pgfscope}%
\pgfsys@transformshift{2.570666in}{0.538014in}%
\pgfsys@useobject{currentmarker}{}%
\end{pgfscope}%
\end{pgfscope}%
\begin{pgfscope}%
\definecolor{textcolor}{rgb}{0.000000,0.000000,0.000000}%
\pgfsetstrokecolor{textcolor}%
\pgfsetfillcolor{textcolor}%
\pgftext[x=2.570666in,y=0.440792in,,top]{\color{textcolor}\rmfamily\fontsize{8.000000}{9.600000}\selectfont \(\displaystyle {12{:}00}\)}%
\end{pgfscope}%
\begin{pgfscope}%
\pgfsetbuttcap%
\pgfsetroundjoin%
\definecolor{currentfill}{rgb}{0.000000,0.000000,0.000000}%
\pgfsetfillcolor{currentfill}%
\pgfsetlinewidth{0.803000pt}%
\definecolor{currentstroke}{rgb}{0.000000,0.000000,0.000000}%
\pgfsetstrokecolor{currentstroke}%
\pgfsetdash{}{0pt}%
\pgfsys@defobject{currentmarker}{\pgfqpoint{0.000000in}{-0.048611in}}{\pgfqpoint{0.000000in}{0.000000in}}{%
\pgfpathmoveto{\pgfqpoint{0.000000in}{0.000000in}}%
\pgfpathlineto{\pgfqpoint{0.000000in}{-0.048611in}}%
\pgfusepath{stroke,fill}%
}%
\begin{pgfscope}%
\pgfsys@transformshift{3.010881in}{0.538014in}%
\pgfsys@useobject{currentmarker}{}%
\end{pgfscope}%
\end{pgfscope}%
\begin{pgfscope}%
\definecolor{textcolor}{rgb}{0.000000,0.000000,0.000000}%
\pgfsetstrokecolor{textcolor}%
\pgfsetfillcolor{textcolor}%
\pgftext[x=3.010881in,y=0.440792in,,top]{\color{textcolor}\rmfamily\fontsize{8.000000}{9.600000}\selectfont \(\displaystyle {15{:}00}\)}%
\end{pgfscope}%
\begin{pgfscope}%
\pgfsetbuttcap%
\pgfsetroundjoin%
\definecolor{currentfill}{rgb}{0.000000,0.000000,0.000000}%
\pgfsetfillcolor{currentfill}%
\pgfsetlinewidth{0.803000pt}%
\definecolor{currentstroke}{rgb}{0.000000,0.000000,0.000000}%
\pgfsetstrokecolor{currentstroke}%
\pgfsetdash{}{0pt}%
\pgfsys@defobject{currentmarker}{\pgfqpoint{0.000000in}{-0.048611in}}{\pgfqpoint{0.000000in}{0.000000in}}{%
\pgfpathmoveto{\pgfqpoint{0.000000in}{0.000000in}}%
\pgfpathlineto{\pgfqpoint{0.000000in}{-0.048611in}}%
\pgfusepath{stroke,fill}%
}%
\begin{pgfscope}%
\pgfsys@transformshift{3.451096in}{0.538014in}%
\pgfsys@useobject{currentmarker}{}%
\end{pgfscope}%
\end{pgfscope}%
\begin{pgfscope}%
\definecolor{textcolor}{rgb}{0.000000,0.000000,0.000000}%
\pgfsetstrokecolor{textcolor}%
\pgfsetfillcolor{textcolor}%
\pgftext[x=3.451096in,y=0.440792in,,top]{\color{textcolor}\rmfamily\fontsize{8.000000}{9.600000}\selectfont \(\displaystyle {18{:}00}\)}%
\end{pgfscope}%
\begin{pgfscope}%
\pgfsetbuttcap%
\pgfsetroundjoin%
\definecolor{currentfill}{rgb}{0.000000,0.000000,0.000000}%
\pgfsetfillcolor{currentfill}%
\pgfsetlinewidth{0.803000pt}%
\definecolor{currentstroke}{rgb}{0.000000,0.000000,0.000000}%
\pgfsetstrokecolor{currentstroke}%
\pgfsetdash{}{0pt}%
\pgfsys@defobject{currentmarker}{\pgfqpoint{0.000000in}{-0.048611in}}{\pgfqpoint{0.000000in}{0.000000in}}{%
\pgfpathmoveto{\pgfqpoint{0.000000in}{0.000000in}}%
\pgfpathlineto{\pgfqpoint{0.000000in}{-0.048611in}}%
\pgfusepath{stroke,fill}%
}%
\begin{pgfscope}%
\pgfsys@transformshift{3.891310in}{0.538014in}%
\pgfsys@useobject{currentmarker}{}%
\end{pgfscope}%
\end{pgfscope}%
\begin{pgfscope}%
\definecolor{textcolor}{rgb}{0.000000,0.000000,0.000000}%
\pgfsetstrokecolor{textcolor}%
\pgfsetfillcolor{textcolor}%
\pgftext[x=3.891310in,y=0.440792in,,top]{\color{textcolor}\rmfamily\fontsize{8.000000}{9.600000}\selectfont \(\displaystyle {21{:}00}\)}%
\end{pgfscope}%
\begin{pgfscope}%
\pgfsetbuttcap%
\pgfsetroundjoin%
\definecolor{currentfill}{rgb}{0.000000,0.000000,0.000000}%
\pgfsetfillcolor{currentfill}%
\pgfsetlinewidth{0.803000pt}%
\definecolor{currentstroke}{rgb}{0.000000,0.000000,0.000000}%
\pgfsetstrokecolor{currentstroke}%
\pgfsetdash{}{0pt}%
\pgfsys@defobject{currentmarker}{\pgfqpoint{0.000000in}{-0.048611in}}{\pgfqpoint{0.000000in}{0.000000in}}{%
\pgfpathmoveto{\pgfqpoint{0.000000in}{0.000000in}}%
\pgfpathlineto{\pgfqpoint{0.000000in}{-0.048611in}}%
\pgfusepath{stroke,fill}%
}%
\begin{pgfscope}%
\pgfsys@transformshift{4.331525in}{0.538014in}%
\pgfsys@useobject{currentmarker}{}%
\end{pgfscope}%
\end{pgfscope}%
\begin{pgfscope}%
\definecolor{textcolor}{rgb}{0.000000,0.000000,0.000000}%
\pgfsetstrokecolor{textcolor}%
\pgfsetfillcolor{textcolor}%
\pgftext[x=4.331525in,y=0.440792in,,top]{\color{textcolor}\rmfamily\fontsize{8.000000}{9.600000}\selectfont \(\displaystyle {00{:}00}\)}%
\end{pgfscope}%
\begin{pgfscope}%
\definecolor{textcolor}{rgb}{0.000000,0.000000,0.000000}%
\pgfsetstrokecolor{textcolor}%
\pgfsetfillcolor{textcolor}%
\pgftext[x=2.570645in,y=0.286570in,,top]{\color{textcolor}\rmfamily\fontsize{10.000000}{12.000000}\selectfont Time (UTC)}%
\end{pgfscope}%
\begin{pgfscope}%
\pgfsetbuttcap%
\pgfsetroundjoin%
\definecolor{currentfill}{rgb}{0.000000,0.000000,0.000000}%
\pgfsetfillcolor{currentfill}%
\pgfsetlinewidth{0.803000pt}%
\definecolor{currentstroke}{rgb}{0.000000,0.000000,0.000000}%
\pgfsetstrokecolor{currentstroke}%
\pgfsetdash{}{0pt}%
\pgfsys@defobject{currentmarker}{\pgfqpoint{-0.048611in}{0.000000in}}{\pgfqpoint{-0.000000in}{0.000000in}}{%
\pgfpathmoveto{\pgfqpoint{-0.000000in}{0.000000in}}%
\pgfpathlineto{\pgfqpoint{-0.048611in}{0.000000in}}%
\pgfusepath{stroke,fill}%
}%
\begin{pgfscope}%
\pgfsys@transformshift{0.633813in}{0.744152in}%
\pgfsys@useobject{currentmarker}{}%
\end{pgfscope}%
\end{pgfscope}%
\begin{pgfscope}%
\definecolor{textcolor}{rgb}{0.000000,0.000000,0.000000}%
\pgfsetstrokecolor{textcolor}%
\pgfsetfillcolor{textcolor}%
\pgftext[x=0.326711in, y=0.705596in, left, base]{\color{textcolor}\rmfamily\fontsize{8.000000}{9.600000}\selectfont \(\displaystyle {\ensuremath{-}15}\)}%
\end{pgfscope}%
\begin{pgfscope}%
\pgfsetbuttcap%
\pgfsetroundjoin%
\definecolor{currentfill}{rgb}{0.000000,0.000000,0.000000}%
\pgfsetfillcolor{currentfill}%
\pgfsetlinewidth{0.803000pt}%
\definecolor{currentstroke}{rgb}{0.000000,0.000000,0.000000}%
\pgfsetstrokecolor{currentstroke}%
\pgfsetdash{}{0pt}%
\pgfsys@defobject{currentmarker}{\pgfqpoint{-0.048611in}{0.000000in}}{\pgfqpoint{-0.000000in}{0.000000in}}{%
\pgfpathmoveto{\pgfqpoint{-0.000000in}{0.000000in}}%
\pgfpathlineto{\pgfqpoint{-0.048611in}{0.000000in}}%
\pgfusepath{stroke,fill}%
}%
\begin{pgfscope}%
\pgfsys@transformshift{0.633813in}{1.155562in}%
\pgfsys@useobject{currentmarker}{}%
\end{pgfscope}%
\end{pgfscope}%
\begin{pgfscope}%
\definecolor{textcolor}{rgb}{0.000000,0.000000,0.000000}%
\pgfsetstrokecolor{textcolor}%
\pgfsetfillcolor{textcolor}%
\pgftext[x=0.326711in, y=1.117006in, left, base]{\color{textcolor}\rmfamily\fontsize{8.000000}{9.600000}\selectfont \(\displaystyle {\ensuremath{-}10}\)}%
\end{pgfscope}%
\begin{pgfscope}%
\pgfsetbuttcap%
\pgfsetroundjoin%
\definecolor{currentfill}{rgb}{0.000000,0.000000,0.000000}%
\pgfsetfillcolor{currentfill}%
\pgfsetlinewidth{0.803000pt}%
\definecolor{currentstroke}{rgb}{0.000000,0.000000,0.000000}%
\pgfsetstrokecolor{currentstroke}%
\pgfsetdash{}{0pt}%
\pgfsys@defobject{currentmarker}{\pgfqpoint{-0.048611in}{0.000000in}}{\pgfqpoint{-0.000000in}{0.000000in}}{%
\pgfpathmoveto{\pgfqpoint{-0.000000in}{0.000000in}}%
\pgfpathlineto{\pgfqpoint{-0.048611in}{0.000000in}}%
\pgfusepath{stroke,fill}%
}%
\begin{pgfscope}%
\pgfsys@transformshift{0.633813in}{1.566972in}%
\pgfsys@useobject{currentmarker}{}%
\end{pgfscope}%
\end{pgfscope}%
\begin{pgfscope}%
\definecolor{textcolor}{rgb}{0.000000,0.000000,0.000000}%
\pgfsetstrokecolor{textcolor}%
\pgfsetfillcolor{textcolor}%
\pgftext[x=0.385740in, y=1.528416in, left, base]{\color{textcolor}\rmfamily\fontsize{8.000000}{9.600000}\selectfont \(\displaystyle {\ensuremath{-}5}\)}%
\end{pgfscope}%
\begin{pgfscope}%
\pgfsetbuttcap%
\pgfsetroundjoin%
\definecolor{currentfill}{rgb}{0.000000,0.000000,0.000000}%
\pgfsetfillcolor{currentfill}%
\pgfsetlinewidth{0.803000pt}%
\definecolor{currentstroke}{rgb}{0.000000,0.000000,0.000000}%
\pgfsetstrokecolor{currentstroke}%
\pgfsetdash{}{0pt}%
\pgfsys@defobject{currentmarker}{\pgfqpoint{-0.048611in}{0.000000in}}{\pgfqpoint{-0.000000in}{0.000000in}}{%
\pgfpathmoveto{\pgfqpoint{-0.000000in}{0.000000in}}%
\pgfpathlineto{\pgfqpoint{-0.048611in}{0.000000in}}%
\pgfusepath{stroke,fill}%
}%
\begin{pgfscope}%
\pgfsys@transformshift{0.633813in}{1.978382in}%
\pgfsys@useobject{currentmarker}{}%
\end{pgfscope}%
\end{pgfscope}%
\begin{pgfscope}%
\definecolor{textcolor}{rgb}{0.000000,0.000000,0.000000}%
\pgfsetstrokecolor{textcolor}%
\pgfsetfillcolor{textcolor}%
\pgftext[x=0.477562in, y=1.939826in, left, base]{\color{textcolor}\rmfamily\fontsize{8.000000}{9.600000}\selectfont \(\displaystyle {0}\)}%
\end{pgfscope}%
\begin{pgfscope}%
\pgfsetbuttcap%
\pgfsetroundjoin%
\definecolor{currentfill}{rgb}{0.000000,0.000000,0.000000}%
\pgfsetfillcolor{currentfill}%
\pgfsetlinewidth{0.803000pt}%
\definecolor{currentstroke}{rgb}{0.000000,0.000000,0.000000}%
\pgfsetstrokecolor{currentstroke}%
\pgfsetdash{}{0pt}%
\pgfsys@defobject{currentmarker}{\pgfqpoint{-0.048611in}{0.000000in}}{\pgfqpoint{-0.000000in}{0.000000in}}{%
\pgfpathmoveto{\pgfqpoint{-0.000000in}{0.000000in}}%
\pgfpathlineto{\pgfqpoint{-0.048611in}{0.000000in}}%
\pgfusepath{stroke,fill}%
}%
\begin{pgfscope}%
\pgfsys@transformshift{0.633813in}{2.389792in}%
\pgfsys@useobject{currentmarker}{}%
\end{pgfscope}%
\end{pgfscope}%
\begin{pgfscope}%
\definecolor{textcolor}{rgb}{0.000000,0.000000,0.000000}%
\pgfsetstrokecolor{textcolor}%
\pgfsetfillcolor{textcolor}%
\pgftext[x=0.477562in, y=2.351237in, left, base]{\color{textcolor}\rmfamily\fontsize{8.000000}{9.600000}\selectfont \(\displaystyle {5}\)}%
\end{pgfscope}%
\begin{pgfscope}%
\pgfsetbuttcap%
\pgfsetroundjoin%
\definecolor{currentfill}{rgb}{0.000000,0.000000,0.000000}%
\pgfsetfillcolor{currentfill}%
\pgfsetlinewidth{0.803000pt}%
\definecolor{currentstroke}{rgb}{0.000000,0.000000,0.000000}%
\pgfsetstrokecolor{currentstroke}%
\pgfsetdash{}{0pt}%
\pgfsys@defobject{currentmarker}{\pgfqpoint{-0.048611in}{0.000000in}}{\pgfqpoint{-0.000000in}{0.000000in}}{%
\pgfpathmoveto{\pgfqpoint{-0.000000in}{0.000000in}}%
\pgfpathlineto{\pgfqpoint{-0.048611in}{0.000000in}}%
\pgfusepath{stroke,fill}%
}%
\begin{pgfscope}%
\pgfsys@transformshift{0.633813in}{2.801202in}%
\pgfsys@useobject{currentmarker}{}%
\end{pgfscope}%
\end{pgfscope}%
\begin{pgfscope}%
\definecolor{textcolor}{rgb}{0.000000,0.000000,0.000000}%
\pgfsetstrokecolor{textcolor}%
\pgfsetfillcolor{textcolor}%
\pgftext[x=0.418534in, y=2.762647in, left, base]{\color{textcolor}\rmfamily\fontsize{8.000000}{9.600000}\selectfont \(\displaystyle {10}\)}%
\end{pgfscope}%
\begin{pgfscope}%
\definecolor{textcolor}{rgb}{0.000000,0.000000,0.000000}%
\pgfsetstrokecolor{textcolor}%
\pgfsetfillcolor{textcolor}%
\pgftext[x=0.271156in,y=1.737274in,,bottom,rotate=90.000000]{\color{textcolor}\rmfamily\fontsize{10.000000}{12.000000}\selectfont Voltage deviation in V}%
\end{pgfscope}%
\begin{pgfscope}%
\definecolor{textcolor}{rgb}{0.000000,0.000000,0.000000}%
\pgfsetstrokecolor{textcolor}%
\pgfsetfillcolor{textcolor}%
\pgftext[x=0.633813in,y=2.978201in,left,base]{\color{textcolor}\rmfamily\fontsize{8.000000}{9.600000}\selectfont \(\displaystyle \times{10^{\ensuremath{-}6}}{}\)}%
\end{pgfscope}%
\begin{pgfscope}%
\pgfpathrectangle{\pgfqpoint{0.633813in}{0.538014in}}{\pgfqpoint{3.873664in}{2.398521in}}%
\pgfusepath{clip}%
\pgfsetrectcap%
\pgfsetroundjoin%
\pgfsetlinewidth{0.501875pt}%
\definecolor{currentstroke}{rgb}{0.121569,0.466667,0.705882}%
\pgfsetstrokecolor{currentstroke}%
\pgfsetstrokeopacity{0.700000}%
\pgfsetdash{}{0pt}%
\pgfpathmoveto{\pgfqpoint{0.809889in}{1.922409in}}%
\pgfpathlineto{\pgfqpoint{0.810092in}{1.914181in}}%
\pgfpathlineto{\pgfqpoint{0.811315in}{2.226852in}}%
\pgfpathlineto{\pgfqpoint{0.812742in}{2.136342in}}%
\pgfpathlineto{\pgfqpoint{0.813557in}{2.086973in}}%
\pgfpathlineto{\pgfqpoint{0.813965in}{2.193939in}}%
\pgfpathlineto{\pgfqpoint{0.814984in}{1.963550in}}%
\pgfpathlineto{\pgfqpoint{0.814372in}{2.235080in}}%
\pgfpathlineto{\pgfqpoint{0.815391in}{2.111657in}}%
\pgfpathlineto{\pgfqpoint{0.816003in}{2.202168in}}%
\pgfpathlineto{\pgfqpoint{0.816410in}{2.161027in}}%
\pgfpathlineto{\pgfqpoint{0.816818in}{2.078745in}}%
\pgfpathlineto{\pgfqpoint{0.817225in}{2.193939in}}%
\pgfpathlineto{\pgfqpoint{0.818041in}{2.235080in}}%
\pgfpathlineto{\pgfqpoint{0.818448in}{2.226852in}}%
\pgfpathlineto{\pgfqpoint{0.818652in}{2.218624in}}%
\pgfpathlineto{\pgfqpoint{0.818856in}{2.243309in}}%
\pgfpathlineto{\pgfqpoint{0.819264in}{2.284450in}}%
\pgfpathlineto{\pgfqpoint{0.819875in}{2.251537in}}%
\pgfpathlineto{\pgfqpoint{0.820486in}{2.218624in}}%
\pgfpathlineto{\pgfqpoint{0.820690in}{2.251537in}}%
\pgfpathlineto{\pgfqpoint{0.820894in}{2.267993in}}%
\pgfpathlineto{\pgfqpoint{0.821098in}{2.218624in}}%
\pgfpathlineto{\pgfqpoint{0.821505in}{2.251537in}}%
\pgfpathlineto{\pgfqpoint{0.821913in}{2.218624in}}%
\pgfpathlineto{\pgfqpoint{0.822117in}{2.243309in}}%
\pgfpathlineto{\pgfqpoint{0.823136in}{2.300906in}}%
\pgfpathlineto{\pgfqpoint{0.823340in}{2.259765in}}%
\pgfpathlineto{\pgfqpoint{0.823747in}{2.292678in}}%
\pgfpathlineto{\pgfqpoint{0.824359in}{2.342047in}}%
\pgfpathlineto{\pgfqpoint{0.824766in}{2.284450in}}%
\pgfpathlineto{\pgfqpoint{0.825174in}{2.333819in}}%
\pgfpathlineto{\pgfqpoint{0.825785in}{2.169255in}}%
\pgfpathlineto{\pgfqpoint{0.826193in}{2.276221in}}%
\pgfpathlineto{\pgfqpoint{0.826397in}{2.276221in}}%
\pgfpathlineto{\pgfqpoint{0.826804in}{2.358504in}}%
\pgfpathlineto{\pgfqpoint{0.827416in}{2.267993in}}%
\pgfpathlineto{\pgfqpoint{0.828638in}{2.358504in}}%
\pgfpathlineto{\pgfqpoint{0.829657in}{2.284450in}}%
\pgfpathlineto{\pgfqpoint{0.829861in}{2.342047in}}%
\pgfpathlineto{\pgfqpoint{0.830269in}{2.267993in}}%
\pgfpathlineto{\pgfqpoint{0.830473in}{2.128114in}}%
\pgfpathlineto{\pgfqpoint{0.831084in}{2.333819in}}%
\pgfpathlineto{\pgfqpoint{0.831288in}{2.383188in}}%
\pgfpathlineto{\pgfqpoint{0.831492in}{2.333819in}}%
\pgfpathlineto{\pgfqpoint{0.831696in}{2.185711in}}%
\pgfpathlineto{\pgfqpoint{0.832511in}{2.276221in}}%
\pgfpathlineto{\pgfqpoint{0.832715in}{2.284450in}}%
\pgfpathlineto{\pgfqpoint{0.832918in}{2.103429in}}%
\pgfpathlineto{\pgfqpoint{0.833530in}{2.300906in}}%
\pgfpathlineto{\pgfqpoint{0.833734in}{2.267993in}}%
\pgfpathlineto{\pgfqpoint{0.834345in}{2.342047in}}%
\pgfpathlineto{\pgfqpoint{0.834549in}{2.259765in}}%
\pgfpathlineto{\pgfqpoint{0.834753in}{2.235080in}}%
\pgfpathlineto{\pgfqpoint{0.835364in}{2.276221in}}%
\pgfpathlineto{\pgfqpoint{0.835568in}{2.243309in}}%
\pgfpathlineto{\pgfqpoint{0.836179in}{2.235080in}}%
\pgfpathlineto{\pgfqpoint{0.836587in}{2.300906in}}%
\pgfpathlineto{\pgfqpoint{0.836791in}{2.267993in}}%
\pgfpathlineto{\pgfqpoint{0.837402in}{2.284450in}}%
\pgfpathlineto{\pgfqpoint{0.838013in}{2.399645in}}%
\pgfpathlineto{\pgfqpoint{0.838421in}{2.267993in}}%
\pgfpathlineto{\pgfqpoint{0.839236in}{2.333819in}}%
\pgfpathlineto{\pgfqpoint{0.839644in}{2.325591in}}%
\pgfpathlineto{\pgfqpoint{0.840051in}{2.243309in}}%
\pgfpathlineto{\pgfqpoint{0.840255in}{2.292678in}}%
\pgfpathlineto{\pgfqpoint{0.840459in}{2.399645in}}%
\pgfpathlineto{\pgfqpoint{0.841478in}{2.374960in}}%
\pgfpathlineto{\pgfqpoint{0.842497in}{2.457242in}}%
\pgfpathlineto{\pgfqpoint{0.842701in}{2.416101in}}%
\pgfpathlineto{\pgfqpoint{0.842905in}{2.358504in}}%
\pgfpathlineto{\pgfqpoint{0.843312in}{2.440786in}}%
\pgfpathlineto{\pgfqpoint{0.843516in}{2.432557in}}%
\pgfpathlineto{\pgfqpoint{0.844127in}{2.498383in}}%
\pgfpathlineto{\pgfqpoint{0.844331in}{2.457242in}}%
\pgfpathlineto{\pgfqpoint{0.844535in}{2.366732in}}%
\pgfpathlineto{\pgfqpoint{0.845350in}{2.440786in}}%
\pgfpathlineto{\pgfqpoint{0.845962in}{2.350275in}}%
\pgfpathlineto{\pgfqpoint{0.846369in}{2.449014in}}%
\pgfpathlineto{\pgfqpoint{0.846573in}{2.449014in}}%
\pgfpathlineto{\pgfqpoint{0.846777in}{2.473698in}}%
\pgfpathlineto{\pgfqpoint{0.847185in}{2.449014in}}%
\pgfpathlineto{\pgfqpoint{0.848407in}{2.309134in}}%
\pgfpathlineto{\pgfqpoint{0.848611in}{2.358504in}}%
\pgfpathlineto{\pgfqpoint{0.849223in}{2.317363in}}%
\pgfpathlineto{\pgfqpoint{0.849426in}{2.350275in}}%
\pgfpathlineto{\pgfqpoint{0.851872in}{2.490155in}}%
\pgfpathlineto{\pgfqpoint{0.852076in}{2.481927in}}%
\pgfpathlineto{\pgfqpoint{0.853299in}{2.399645in}}%
\pgfpathlineto{\pgfqpoint{0.853706in}{2.416101in}}%
\pgfpathlineto{\pgfqpoint{0.854114in}{2.432557in}}%
\pgfpathlineto{\pgfqpoint{0.854725in}{2.391416in}}%
\pgfpathlineto{\pgfqpoint{0.854929in}{2.457242in}}%
\pgfpathlineto{\pgfqpoint{0.855744in}{2.391416in}}%
\pgfpathlineto{\pgfqpoint{0.856356in}{2.342047in}}%
\pgfpathlineto{\pgfqpoint{0.856152in}{2.424329in}}%
\pgfpathlineto{\pgfqpoint{0.856559in}{2.416101in}}%
\pgfpathlineto{\pgfqpoint{0.856967in}{2.416101in}}%
\pgfpathlineto{\pgfqpoint{0.857578in}{2.366732in}}%
\pgfpathlineto{\pgfqpoint{0.857986in}{2.391416in}}%
\pgfpathlineto{\pgfqpoint{0.859005in}{2.416101in}}%
\pgfpathlineto{\pgfqpoint{0.859209in}{2.374960in}}%
\pgfpathlineto{\pgfqpoint{0.859413in}{2.449014in}}%
\pgfpathlineto{\pgfqpoint{0.860228in}{2.391416in}}%
\pgfpathlineto{\pgfqpoint{0.860432in}{2.391416in}}%
\pgfpathlineto{\pgfqpoint{0.860839in}{2.358504in}}%
\pgfpathlineto{\pgfqpoint{0.861655in}{2.432557in}}%
\pgfpathlineto{\pgfqpoint{0.862266in}{2.391416in}}%
\pgfpathlineto{\pgfqpoint{0.862062in}{2.440786in}}%
\pgfpathlineto{\pgfqpoint{0.862470in}{2.416101in}}%
\pgfpathlineto{\pgfqpoint{0.862877in}{2.399645in}}%
\pgfpathlineto{\pgfqpoint{0.863285in}{2.449014in}}%
\pgfpathlineto{\pgfqpoint{0.863489in}{2.309134in}}%
\pgfpathlineto{\pgfqpoint{0.863693in}{2.514839in}}%
\pgfpathlineto{\pgfqpoint{0.864304in}{2.473698in}}%
\pgfpathlineto{\pgfqpoint{0.864508in}{2.449014in}}%
\pgfpathlineto{\pgfqpoint{0.864712in}{2.243309in}}%
\pgfpathlineto{\pgfqpoint{0.864915in}{2.473698in}}%
\pgfpathlineto{\pgfqpoint{0.865527in}{2.449014in}}%
\pgfpathlineto{\pgfqpoint{0.867157in}{2.547752in}}%
\pgfpathlineto{\pgfqpoint{0.867769in}{2.432557in}}%
\pgfpathlineto{\pgfqpoint{0.868176in}{2.481927in}}%
\pgfpathlineto{\pgfqpoint{0.869399in}{2.514839in}}%
\pgfpathlineto{\pgfqpoint{0.869603in}{2.506611in}}%
\pgfpathlineto{\pgfqpoint{0.869807in}{2.588893in}}%
\pgfpathlineto{\pgfqpoint{0.870214in}{2.481927in}}%
\pgfpathlineto{\pgfqpoint{0.870622in}{2.572437in}}%
\pgfpathlineto{\pgfqpoint{0.871641in}{2.473698in}}%
\pgfpathlineto{\pgfqpoint{0.871845in}{2.547752in}}%
\pgfpathlineto{\pgfqpoint{0.872660in}{2.506611in}}%
\pgfpathlineto{\pgfqpoint{0.873475in}{2.465470in}}%
\pgfpathlineto{\pgfqpoint{0.874087in}{2.309134in}}%
\pgfpathlineto{\pgfqpoint{0.874698in}{2.572437in}}%
\pgfpathlineto{\pgfqpoint{0.875106in}{2.391416in}}%
\pgfpathlineto{\pgfqpoint{0.875921in}{2.531296in}}%
\pgfpathlineto{\pgfqpoint{0.876328in}{2.490155in}}%
\pgfpathlineto{\pgfqpoint{0.876736in}{2.185711in}}%
\pgfpathlineto{\pgfqpoint{0.877347in}{2.432557in}}%
\pgfpathlineto{\pgfqpoint{0.877551in}{2.432557in}}%
\pgfpathlineto{\pgfqpoint{0.877755in}{2.449014in}}%
\pgfpathlineto{\pgfqpoint{0.878366in}{2.465470in}}%
\pgfpathlineto{\pgfqpoint{0.878978in}{2.193939in}}%
\pgfpathlineto{\pgfqpoint{0.879182in}{2.490155in}}%
\pgfpathlineto{\pgfqpoint{0.880201in}{2.424329in}}%
\pgfpathlineto{\pgfqpoint{0.880404in}{2.424329in}}%
\pgfpathlineto{\pgfqpoint{0.881627in}{2.588893in}}%
\pgfpathlineto{\pgfqpoint{0.881831in}{2.531296in}}%
\pgfpathlineto{\pgfqpoint{0.882646in}{2.498383in}}%
\pgfpathlineto{\pgfqpoint{0.882442in}{2.555980in}}%
\pgfpathlineto{\pgfqpoint{0.882850in}{2.506611in}}%
\pgfpathlineto{\pgfqpoint{0.883665in}{2.588893in}}%
\pgfpathlineto{\pgfqpoint{0.884073in}{2.547752in}}%
\pgfpathlineto{\pgfqpoint{0.884277in}{2.547752in}}%
\pgfpathlineto{\pgfqpoint{0.884888in}{2.605350in}}%
\pgfpathlineto{\pgfqpoint{0.885500in}{2.588893in}}%
\pgfpathlineto{\pgfqpoint{0.885703in}{2.481927in}}%
\pgfpathlineto{\pgfqpoint{0.886519in}{2.572437in}}%
\pgfpathlineto{\pgfqpoint{0.886926in}{2.580665in}}%
\pgfpathlineto{\pgfqpoint{0.887334in}{2.523068in}}%
\pgfpathlineto{\pgfqpoint{0.888353in}{2.630034in}}%
\pgfpathlineto{\pgfqpoint{0.888557in}{2.613578in}}%
\pgfpathlineto{\pgfqpoint{0.889168in}{2.572437in}}%
\pgfpathlineto{\pgfqpoint{0.889576in}{2.605350in}}%
\pgfpathlineto{\pgfqpoint{0.889779in}{2.621806in}}%
\pgfpathlineto{\pgfqpoint{0.889983in}{2.572437in}}%
\pgfpathlineto{\pgfqpoint{0.891614in}{2.465470in}}%
\pgfpathlineto{\pgfqpoint{0.891817in}{2.457242in}}%
\pgfpathlineto{\pgfqpoint{0.892021in}{2.473698in}}%
\pgfpathlineto{\pgfqpoint{0.893448in}{2.646491in}}%
\pgfpathlineto{\pgfqpoint{0.893652in}{2.654719in}}%
\pgfpathlineto{\pgfqpoint{0.895078in}{2.449014in}}%
\pgfpathlineto{\pgfqpoint{0.895282in}{2.498383in}}%
\pgfpathlineto{\pgfqpoint{0.895893in}{2.539524in}}%
\pgfpathlineto{\pgfqpoint{0.896097in}{2.514839in}}%
\pgfpathlineto{\pgfqpoint{0.897116in}{2.473698in}}%
\pgfpathlineto{\pgfqpoint{0.897524in}{2.539524in}}%
\pgfpathlineto{\pgfqpoint{0.898339in}{2.531296in}}%
\pgfpathlineto{\pgfqpoint{0.898543in}{2.523068in}}%
\pgfpathlineto{\pgfqpoint{0.899766in}{2.572437in}}%
\pgfpathlineto{\pgfqpoint{0.899970in}{2.547752in}}%
\pgfpathlineto{\pgfqpoint{0.900173in}{2.605350in}}%
\pgfpathlineto{\pgfqpoint{0.900785in}{2.564209in}}%
\pgfpathlineto{\pgfqpoint{0.900989in}{2.646491in}}%
\pgfpathlineto{\pgfqpoint{0.901804in}{2.630034in}}%
\pgfpathlineto{\pgfqpoint{0.902008in}{2.597121in}}%
\pgfpathlineto{\pgfqpoint{0.902823in}{2.621806in}}%
\pgfpathlineto{\pgfqpoint{0.903027in}{2.662947in}}%
\pgfpathlineto{\pgfqpoint{0.903638in}{2.588893in}}%
\pgfpathlineto{\pgfqpoint{0.903842in}{2.630034in}}%
\pgfpathlineto{\pgfqpoint{0.904453in}{2.638262in}}%
\pgfpathlineto{\pgfqpoint{0.905065in}{2.555980in}}%
\pgfpathlineto{\pgfqpoint{0.906491in}{2.440786in}}%
\pgfpathlineto{\pgfqpoint{0.906695in}{2.498383in}}%
\pgfpathlineto{\pgfqpoint{0.907510in}{2.457242in}}%
\pgfpathlineto{\pgfqpoint{0.907714in}{2.457242in}}%
\pgfpathlineto{\pgfqpoint{0.908325in}{2.416101in}}%
\pgfpathlineto{\pgfqpoint{0.908529in}{2.465470in}}%
\pgfpathlineto{\pgfqpoint{0.908733in}{2.449014in}}%
\pgfpathlineto{\pgfqpoint{0.910363in}{2.572437in}}%
\pgfpathlineto{\pgfqpoint{0.909141in}{2.440786in}}%
\pgfpathlineto{\pgfqpoint{0.910771in}{2.547752in}}%
\pgfpathlineto{\pgfqpoint{0.910975in}{2.514839in}}%
\pgfpathlineto{\pgfqpoint{0.911586in}{2.588893in}}%
\pgfpathlineto{\pgfqpoint{0.912198in}{2.555980in}}%
\pgfpathlineto{\pgfqpoint{0.912402in}{2.621806in}}%
\pgfpathlineto{\pgfqpoint{0.914847in}{2.490155in}}%
\pgfpathlineto{\pgfqpoint{0.915662in}{2.547752in}}%
\pgfpathlineto{\pgfqpoint{0.915866in}{2.523068in}}%
\pgfpathlineto{\pgfqpoint{0.916070in}{2.473698in}}%
\pgfpathlineto{\pgfqpoint{0.916478in}{2.564209in}}%
\pgfpathlineto{\pgfqpoint{0.916885in}{2.531296in}}%
\pgfpathlineto{\pgfqpoint{0.917497in}{2.555980in}}%
\pgfpathlineto{\pgfqpoint{0.917700in}{2.539524in}}%
\pgfpathlineto{\pgfqpoint{0.917904in}{2.506611in}}%
\pgfpathlineto{\pgfqpoint{0.918312in}{2.605350in}}%
\pgfpathlineto{\pgfqpoint{0.918516in}{2.572437in}}%
\pgfpathlineto{\pgfqpoint{0.919331in}{2.547752in}}%
\pgfpathlineto{\pgfqpoint{0.918923in}{2.588893in}}%
\pgfpathlineto{\pgfqpoint{0.919535in}{2.572437in}}%
\pgfpathlineto{\pgfqpoint{0.919942in}{2.662947in}}%
\pgfpathlineto{\pgfqpoint{0.920757in}{2.654719in}}%
\pgfpathlineto{\pgfqpoint{0.921573in}{2.580665in}}%
\pgfpathlineto{\pgfqpoint{0.922184in}{2.597121in}}%
\pgfpathlineto{\pgfqpoint{0.922388in}{2.605350in}}%
\pgfpathlineto{\pgfqpoint{0.922999in}{2.490155in}}%
\pgfpathlineto{\pgfqpoint{0.923611in}{2.531296in}}%
\pgfpathlineto{\pgfqpoint{0.924426in}{2.490155in}}%
\pgfpathlineto{\pgfqpoint{0.925241in}{2.498383in}}%
\pgfpathlineto{\pgfqpoint{0.926872in}{2.588893in}}%
\pgfpathlineto{\pgfqpoint{0.927483in}{2.605350in}}%
\pgfpathlineto{\pgfqpoint{0.928094in}{2.564209in}}%
\pgfpathlineto{\pgfqpoint{0.928910in}{2.646491in}}%
\pgfpathlineto{\pgfqpoint{0.929317in}{2.588893in}}%
\pgfpathlineto{\pgfqpoint{0.929521in}{2.588893in}}%
\pgfpathlineto{\pgfqpoint{0.930540in}{2.498383in}}%
\pgfpathlineto{\pgfqpoint{0.930132in}{2.630034in}}%
\pgfpathlineto{\pgfqpoint{0.930948in}{2.514839in}}%
\pgfpathlineto{\pgfqpoint{0.931151in}{2.555980in}}%
\pgfpathlineto{\pgfqpoint{0.931763in}{2.498383in}}%
\pgfpathlineto{\pgfqpoint{0.931967in}{2.498383in}}%
\pgfpathlineto{\pgfqpoint{0.932986in}{2.613578in}}%
\pgfpathlineto{\pgfqpoint{0.932374in}{2.490155in}}%
\pgfpathlineto{\pgfqpoint{0.933393in}{2.555980in}}%
\pgfpathlineto{\pgfqpoint{0.933597in}{2.539524in}}%
\pgfpathlineto{\pgfqpoint{0.934208in}{2.572437in}}%
\pgfpathlineto{\pgfqpoint{0.934412in}{2.572437in}}%
\pgfpathlineto{\pgfqpoint{0.934616in}{2.300906in}}%
\pgfpathlineto{\pgfqpoint{0.935431in}{2.383188in}}%
\pgfpathlineto{\pgfqpoint{0.936246in}{2.605350in}}%
\pgfpathlineto{\pgfqpoint{0.936654in}{2.564209in}}%
\pgfpathlineto{\pgfqpoint{0.938284in}{2.498383in}}%
\pgfpathlineto{\pgfqpoint{0.938488in}{2.539524in}}%
\pgfpathlineto{\pgfqpoint{0.939100in}{2.498383in}}%
\pgfpathlineto{\pgfqpoint{0.939304in}{2.383188in}}%
\pgfpathlineto{\pgfqpoint{0.939915in}{2.597121in}}%
\pgfpathlineto{\pgfqpoint{0.940323in}{2.416101in}}%
\pgfpathlineto{\pgfqpoint{0.940730in}{2.547752in}}%
\pgfpathlineto{\pgfqpoint{0.941545in}{2.498383in}}%
\pgfpathlineto{\pgfqpoint{0.941749in}{2.506611in}}%
\pgfpathlineto{\pgfqpoint{0.942157in}{2.481927in}}%
\pgfpathlineto{\pgfqpoint{0.942361in}{2.432557in}}%
\pgfpathlineto{\pgfqpoint{0.942768in}{2.547752in}}%
\pgfpathlineto{\pgfqpoint{0.942972in}{2.514839in}}%
\pgfpathlineto{\pgfqpoint{0.944195in}{2.654719in}}%
\pgfpathlineto{\pgfqpoint{0.944602in}{2.630034in}}%
\pgfpathlineto{\pgfqpoint{0.945010in}{2.407873in}}%
\pgfpathlineto{\pgfqpoint{0.945621in}{2.580665in}}%
\pgfpathlineto{\pgfqpoint{0.946029in}{2.679403in}}%
\pgfpathlineto{\pgfqpoint{0.946844in}{2.638262in}}%
\pgfpathlineto{\pgfqpoint{0.948067in}{2.555980in}}%
\pgfpathlineto{\pgfqpoint{0.948882in}{2.630034in}}%
\pgfpathlineto{\pgfqpoint{0.949086in}{2.621806in}}%
\pgfpathlineto{\pgfqpoint{0.949290in}{2.539524in}}%
\pgfpathlineto{\pgfqpoint{0.949697in}{2.630034in}}%
\pgfpathlineto{\pgfqpoint{0.950105in}{2.588893in}}%
\pgfpathlineto{\pgfqpoint{0.950309in}{2.630034in}}%
\pgfpathlineto{\pgfqpoint{0.950513in}{2.580665in}}%
\pgfpathlineto{\pgfqpoint{0.950716in}{2.613578in}}%
\pgfpathlineto{\pgfqpoint{0.950920in}{2.440786in}}%
\pgfpathlineto{\pgfqpoint{0.951328in}{2.720544in}}%
\pgfpathlineto{\pgfqpoint{0.951735in}{2.671175in}}%
\pgfpathlineto{\pgfqpoint{0.952958in}{2.761685in}}%
\pgfpathlineto{\pgfqpoint{0.953162in}{2.712316in}}%
\pgfpathlineto{\pgfqpoint{0.954181in}{2.473698in}}%
\pgfpathlineto{\pgfqpoint{0.954385in}{2.638262in}}%
\pgfpathlineto{\pgfqpoint{0.954793in}{2.827511in}}%
\pgfpathlineto{\pgfqpoint{0.955404in}{2.745229in}}%
\pgfpathlineto{\pgfqpoint{0.955608in}{2.704088in}}%
\pgfpathlineto{\pgfqpoint{0.956423in}{2.737001in}}%
\pgfpathlineto{\pgfqpoint{0.956831in}{2.761685in}}%
\pgfpathlineto{\pgfqpoint{0.957238in}{2.712316in}}%
\pgfpathlineto{\pgfqpoint{0.958257in}{2.662947in}}%
\pgfpathlineto{\pgfqpoint{0.958461in}{2.671175in}}%
\pgfpathlineto{\pgfqpoint{0.958665in}{2.687632in}}%
\pgfpathlineto{\pgfqpoint{0.959072in}{2.662947in}}%
\pgfpathlineto{\pgfqpoint{0.960499in}{2.564209in}}%
\pgfpathlineto{\pgfqpoint{0.961110in}{2.695860in}}%
\pgfpathlineto{\pgfqpoint{0.960907in}{2.424329in}}%
\pgfpathlineto{\pgfqpoint{0.961518in}{2.671175in}}%
\pgfpathlineto{\pgfqpoint{0.961926in}{2.597121in}}%
\pgfpathlineto{\pgfqpoint{0.962741in}{2.613578in}}%
\pgfpathlineto{\pgfqpoint{0.962945in}{2.621806in}}%
\pgfpathlineto{\pgfqpoint{0.963760in}{2.399645in}}%
\pgfpathlineto{\pgfqpoint{0.963964in}{2.572437in}}%
\pgfpathlineto{\pgfqpoint{0.965390in}{2.671175in}}%
\pgfpathlineto{\pgfqpoint{0.966613in}{2.399645in}}%
\pgfpathlineto{\pgfqpoint{0.966817in}{2.514839in}}%
\pgfpathlineto{\pgfqpoint{0.967836in}{2.613578in}}%
\pgfpathlineto{\pgfqpoint{0.968040in}{2.605350in}}%
\pgfpathlineto{\pgfqpoint{0.968244in}{2.588893in}}%
\pgfpathlineto{\pgfqpoint{0.968447in}{2.613578in}}%
\pgfpathlineto{\pgfqpoint{0.968651in}{2.597121in}}%
\pgfpathlineto{\pgfqpoint{0.969059in}{2.654719in}}%
\pgfpathlineto{\pgfqpoint{0.969466in}{2.547752in}}%
\pgfpathlineto{\pgfqpoint{0.969670in}{2.613578in}}%
\pgfpathlineto{\pgfqpoint{0.970893in}{2.276221in}}%
\pgfpathlineto{\pgfqpoint{0.971912in}{2.654719in}}%
\pgfpathlineto{\pgfqpoint{0.972116in}{2.638262in}}%
\pgfpathlineto{\pgfqpoint{0.972320in}{2.646491in}}%
\pgfpathlineto{\pgfqpoint{0.972523in}{2.630034in}}%
\pgfpathlineto{\pgfqpoint{0.973746in}{2.473698in}}%
\pgfpathlineto{\pgfqpoint{0.973135in}{2.671175in}}%
\pgfpathlineto{\pgfqpoint{0.973950in}{2.580665in}}%
\pgfpathlineto{\pgfqpoint{0.974154in}{2.613578in}}%
\pgfpathlineto{\pgfqpoint{0.974765in}{2.597121in}}%
\pgfpathlineto{\pgfqpoint{0.974969in}{2.547752in}}%
\pgfpathlineto{\pgfqpoint{0.975580in}{2.613578in}}%
\pgfpathlineto{\pgfqpoint{0.975988in}{2.555980in}}%
\pgfpathlineto{\pgfqpoint{0.977415in}{2.654719in}}%
\pgfpathlineto{\pgfqpoint{0.977618in}{2.481927in}}%
\pgfpathlineto{\pgfqpoint{0.978434in}{2.654719in}}%
\pgfpathlineto{\pgfqpoint{0.979860in}{2.473698in}}%
\pgfpathlineto{\pgfqpoint{0.978841in}{2.679403in}}%
\pgfpathlineto{\pgfqpoint{0.980064in}{2.506611in}}%
\pgfpathlineto{\pgfqpoint{0.980268in}{2.547752in}}%
\pgfpathlineto{\pgfqpoint{0.980676in}{2.407873in}}%
\pgfpathlineto{\pgfqpoint{0.981287in}{2.539524in}}%
\pgfpathlineto{\pgfqpoint{0.981491in}{2.531296in}}%
\pgfpathlineto{\pgfqpoint{0.981898in}{2.481927in}}%
\pgfpathlineto{\pgfqpoint{0.982917in}{2.646491in}}%
\pgfpathlineto{\pgfqpoint{0.984344in}{2.547752in}}%
\pgfpathlineto{\pgfqpoint{0.986178in}{2.712316in}}%
\pgfpathlineto{\pgfqpoint{0.987197in}{2.358504in}}%
\pgfpathlineto{\pgfqpoint{0.987605in}{2.547752in}}%
\pgfpathlineto{\pgfqpoint{0.988012in}{2.564209in}}%
\pgfpathlineto{\pgfqpoint{0.989031in}{2.481927in}}%
\pgfpathlineto{\pgfqpoint{0.989235in}{2.498383in}}%
\pgfpathlineto{\pgfqpoint{0.990050in}{2.646491in}}%
\pgfpathlineto{\pgfqpoint{0.990254in}{2.399645in}}%
\pgfpathlineto{\pgfqpoint{0.991069in}{2.638262in}}%
\pgfpathlineto{\pgfqpoint{0.991681in}{2.679403in}}%
\pgfpathlineto{\pgfqpoint{0.991885in}{2.613578in}}%
\pgfpathlineto{\pgfqpoint{0.992496in}{2.547752in}}%
\pgfpathlineto{\pgfqpoint{0.992904in}{2.654719in}}%
\pgfpathlineto{\pgfqpoint{0.993311in}{2.638262in}}%
\pgfpathlineto{\pgfqpoint{0.993515in}{2.358504in}}%
\pgfpathlineto{\pgfqpoint{0.994330in}{2.704088in}}%
\pgfpathlineto{\pgfqpoint{0.995349in}{2.605350in}}%
\pgfpathlineto{\pgfqpoint{0.995757in}{2.613578in}}%
\pgfpathlineto{\pgfqpoint{0.996368in}{2.588893in}}%
\pgfpathlineto{\pgfqpoint{0.996980in}{2.638262in}}%
\pgfpathlineto{\pgfqpoint{0.997591in}{2.514839in}}%
\pgfpathlineto{\pgfqpoint{0.998203in}{2.547752in}}%
\pgfpathlineto{\pgfqpoint{0.998406in}{2.547752in}}%
\pgfpathlineto{\pgfqpoint{0.998610in}{2.539524in}}%
\pgfpathlineto{\pgfqpoint{0.998814in}{2.391416in}}%
\pgfpathlineto{\pgfqpoint{0.999629in}{2.572437in}}%
\pgfpathlineto{\pgfqpoint{0.999833in}{2.531296in}}%
\pgfpathlineto{\pgfqpoint{1.000444in}{2.605350in}}%
\pgfpathlineto{\pgfqpoint{1.001667in}{2.695860in}}%
\pgfpathlineto{\pgfqpoint{1.002890in}{2.547752in}}%
\pgfpathlineto{\pgfqpoint{1.004317in}{2.638262in}}%
\pgfpathlineto{\pgfqpoint{1.004928in}{2.539524in}}%
\pgfpathlineto{\pgfqpoint{1.005336in}{2.597121in}}%
\pgfpathlineto{\pgfqpoint{1.006151in}{2.654719in}}%
\pgfpathlineto{\pgfqpoint{1.006355in}{2.621806in}}%
\pgfpathlineto{\pgfqpoint{1.006762in}{2.597121in}}%
\pgfpathlineto{\pgfqpoint{1.007170in}{2.654719in}}%
\pgfpathlineto{\pgfqpoint{1.007374in}{2.646491in}}%
\pgfpathlineto{\pgfqpoint{1.007578in}{2.695860in}}%
\pgfpathlineto{\pgfqpoint{1.008597in}{2.687632in}}%
\pgfpathlineto{\pgfqpoint{1.009208in}{2.712316in}}%
\pgfpathlineto{\pgfqpoint{1.009819in}{2.638262in}}%
\pgfpathlineto{\pgfqpoint{1.010023in}{2.712316in}}%
\pgfpathlineto{\pgfqpoint{1.010838in}{2.671175in}}%
\pgfpathlineto{\pgfqpoint{1.011857in}{2.547752in}}%
\pgfpathlineto{\pgfqpoint{1.012265in}{2.638262in}}%
\pgfpathlineto{\pgfqpoint{1.012876in}{2.654719in}}%
\pgfpathlineto{\pgfqpoint{1.012673in}{2.630034in}}%
\pgfpathlineto{\pgfqpoint{1.013284in}{2.638262in}}%
\pgfpathlineto{\pgfqpoint{1.014303in}{2.588893in}}%
\pgfpathlineto{\pgfqpoint{1.013895in}{2.679403in}}%
\pgfpathlineto{\pgfqpoint{1.014507in}{2.597121in}}%
\pgfpathlineto{\pgfqpoint{1.014711in}{2.638262in}}%
\pgfpathlineto{\pgfqpoint{1.015322in}{2.564209in}}%
\pgfpathlineto{\pgfqpoint{1.015526in}{2.597121in}}%
\pgfpathlineto{\pgfqpoint{1.015730in}{2.605350in}}%
\pgfpathlineto{\pgfqpoint{1.017156in}{2.416101in}}%
\pgfpathlineto{\pgfqpoint{1.017564in}{2.432557in}}%
\pgfpathlineto{\pgfqpoint{1.018787in}{2.300906in}}%
\pgfpathlineto{\pgfqpoint{1.019602in}{2.366732in}}%
\pgfpathlineto{\pgfqpoint{1.019398in}{2.284450in}}%
\pgfpathlineto{\pgfqpoint{1.019806in}{2.292678in}}%
\pgfpathlineto{\pgfqpoint{1.020010in}{2.309134in}}%
\pgfpathlineto{\pgfqpoint{1.020417in}{2.259765in}}%
\pgfpathlineto{\pgfqpoint{1.020825in}{2.292678in}}%
\pgfpathlineto{\pgfqpoint{1.021844in}{2.251537in}}%
\pgfpathlineto{\pgfqpoint{1.021232in}{2.333819in}}%
\pgfpathlineto{\pgfqpoint{1.022048in}{2.267993in}}%
\pgfpathlineto{\pgfqpoint{1.023270in}{2.366732in}}%
\pgfpathlineto{\pgfqpoint{1.025105in}{2.144570in}}%
\pgfpathlineto{\pgfqpoint{1.025920in}{2.309134in}}%
\pgfpathlineto{\pgfqpoint{1.026531in}{2.276221in}}%
\pgfpathlineto{\pgfqpoint{1.026735in}{2.284450in}}%
\pgfpathlineto{\pgfqpoint{1.026939in}{2.276221in}}%
\pgfpathlineto{\pgfqpoint{1.027754in}{2.103429in}}%
\pgfpathlineto{\pgfqpoint{1.028365in}{2.177483in}}%
\pgfpathlineto{\pgfqpoint{1.028569in}{2.177483in}}%
\pgfpathlineto{\pgfqpoint{1.028773in}{2.054060in}}%
\pgfpathlineto{\pgfqpoint{1.029792in}{2.095201in}}%
\pgfpathlineto{\pgfqpoint{1.030403in}{2.086973in}}%
\pgfpathlineto{\pgfqpoint{1.030607in}{2.128114in}}%
\pgfpathlineto{\pgfqpoint{1.030811in}{2.235080in}}%
\pgfpathlineto{\pgfqpoint{1.031422in}{2.070516in}}%
\pgfpathlineto{\pgfqpoint{1.032441in}{2.004691in}}%
\pgfpathlineto{\pgfqpoint{1.032849in}{2.021147in}}%
\pgfpathlineto{\pgfqpoint{1.034072in}{2.193939in}}%
\pgfpathlineto{\pgfqpoint{1.033257in}{1.963550in}}%
\pgfpathlineto{\pgfqpoint{1.034276in}{2.152798in}}%
\pgfpathlineto{\pgfqpoint{1.034480in}{1.988234in}}%
\pgfpathlineto{\pgfqpoint{1.034887in}{2.193939in}}%
\pgfpathlineto{\pgfqpoint{1.035295in}{2.078745in}}%
\pgfpathlineto{\pgfqpoint{1.036518in}{2.218624in}}%
\pgfpathlineto{\pgfqpoint{1.036721in}{2.185711in}}%
\pgfpathlineto{\pgfqpoint{1.036925in}{1.914181in}}%
\pgfpathlineto{\pgfqpoint{1.037740in}{2.086973in}}%
\pgfpathlineto{\pgfqpoint{1.038148in}{2.128114in}}%
\pgfpathlineto{\pgfqpoint{1.038556in}{2.078745in}}%
\pgfpathlineto{\pgfqpoint{1.038759in}{2.054060in}}%
\pgfpathlineto{\pgfqpoint{1.038963in}{2.103429in}}%
\pgfpathlineto{\pgfqpoint{1.039575in}{2.078745in}}%
\pgfpathlineto{\pgfqpoint{1.039982in}{2.012919in}}%
\pgfpathlineto{\pgfqpoint{1.040797in}{2.161027in}}%
\pgfpathlineto{\pgfqpoint{1.041613in}{2.021147in}}%
\pgfpathlineto{\pgfqpoint{1.042224in}{2.078745in}}%
\pgfpathlineto{\pgfqpoint{1.042632in}{2.103429in}}%
\pgfpathlineto{\pgfqpoint{1.043039in}{1.848355in}}%
\pgfpathlineto{\pgfqpoint{1.043854in}{1.988234in}}%
\pgfpathlineto{\pgfqpoint{1.044262in}{2.029375in}}%
\pgfpathlineto{\pgfqpoint{1.044466in}{1.971778in}}%
\pgfpathlineto{\pgfqpoint{1.044873in}{1.988234in}}%
\pgfpathlineto{\pgfqpoint{1.045689in}{1.897724in}}%
\pgfpathlineto{\pgfqpoint{1.046096in}{1.947093in}}%
\pgfpathlineto{\pgfqpoint{1.046300in}{1.980006in}}%
\pgfpathlineto{\pgfqpoint{1.046708in}{1.889496in}}%
\pgfpathlineto{\pgfqpoint{1.048134in}{1.971778in}}%
\pgfpathlineto{\pgfqpoint{1.048542in}{1.930637in}}%
\pgfpathlineto{\pgfqpoint{1.048746in}{1.757845in}}%
\pgfpathlineto{\pgfqpoint{1.049561in}{1.971778in}}%
\pgfpathlineto{\pgfqpoint{1.050376in}{1.823670in}}%
\pgfpathlineto{\pgfqpoint{1.049969in}{1.980006in}}%
\pgfpathlineto{\pgfqpoint{1.050784in}{1.873040in}}%
\pgfpathlineto{\pgfqpoint{1.050988in}{1.881268in}}%
\pgfpathlineto{\pgfqpoint{1.051599in}{1.996463in}}%
\pgfpathlineto{\pgfqpoint{1.052210in}{1.971778in}}%
\pgfpathlineto{\pgfqpoint{1.053026in}{1.873040in}}%
\pgfpathlineto{\pgfqpoint{1.053229in}{1.914181in}}%
\pgfpathlineto{\pgfqpoint{1.053637in}{2.004691in}}%
\pgfpathlineto{\pgfqpoint{1.054452in}{1.955322in}}%
\pgfpathlineto{\pgfqpoint{1.056083in}{2.095201in}}%
\pgfpathlineto{\pgfqpoint{1.056490in}{2.012919in}}%
\pgfpathlineto{\pgfqpoint{1.057102in}{2.054060in}}%
\pgfpathlineto{\pgfqpoint{1.057305in}{2.070516in}}%
\pgfpathlineto{\pgfqpoint{1.057509in}{2.004691in}}%
\pgfpathlineto{\pgfqpoint{1.057713in}{2.004691in}}%
\pgfpathlineto{\pgfqpoint{1.058936in}{2.128114in}}%
\pgfpathlineto{\pgfqpoint{1.059751in}{1.881268in}}%
\pgfpathlineto{\pgfqpoint{1.059955in}{2.029375in}}%
\pgfpathlineto{\pgfqpoint{1.060159in}{2.136342in}}%
\pgfpathlineto{\pgfqpoint{1.060363in}{1.930637in}}%
\pgfpathlineto{\pgfqpoint{1.060974in}{2.111657in}}%
\pgfpathlineto{\pgfqpoint{1.062197in}{1.988234in}}%
\pgfpathlineto{\pgfqpoint{1.063012in}{2.152798in}}%
\pgfpathlineto{\pgfqpoint{1.063420in}{2.111657in}}%
\pgfpathlineto{\pgfqpoint{1.063827in}{2.210396in}}%
\pgfpathlineto{\pgfqpoint{1.064439in}{2.103429in}}%
\pgfpathlineto{\pgfqpoint{1.065865in}{1.971778in}}%
\pgfpathlineto{\pgfqpoint{1.064846in}{2.111657in}}%
\pgfpathlineto{\pgfqpoint{1.066273in}{1.980006in}}%
\pgfpathlineto{\pgfqpoint{1.066884in}{1.963550in}}%
\pgfpathlineto{\pgfqpoint{1.067292in}{2.004691in}}%
\pgfpathlineto{\pgfqpoint{1.067699in}{2.045832in}}%
\pgfpathlineto{\pgfqpoint{1.068515in}{1.938865in}}%
\pgfpathlineto{\pgfqpoint{1.069126in}{2.029375in}}%
\pgfpathlineto{\pgfqpoint{1.069534in}{1.905952in}}%
\pgfpathlineto{\pgfqpoint{1.071368in}{1.766073in}}%
\pgfpathlineto{\pgfqpoint{1.072591in}{1.873040in}}%
\pgfpathlineto{\pgfqpoint{1.072794in}{1.873040in}}%
\pgfpathlineto{\pgfqpoint{1.072998in}{1.856583in}}%
\pgfpathlineto{\pgfqpoint{1.073406in}{1.955322in}}%
\pgfpathlineto{\pgfqpoint{1.074017in}{1.938865in}}%
\pgfpathlineto{\pgfqpoint{1.075240in}{1.848355in}}%
\pgfpathlineto{\pgfqpoint{1.076463in}{1.971778in}}%
\pgfpathlineto{\pgfqpoint{1.076667in}{1.988234in}}%
\pgfpathlineto{\pgfqpoint{1.076871in}{1.955322in}}%
\pgfpathlineto{\pgfqpoint{1.077074in}{1.889496in}}%
\pgfpathlineto{\pgfqpoint{1.077278in}{1.996463in}}%
\pgfpathlineto{\pgfqpoint{1.077890in}{1.922409in}}%
\pgfpathlineto{\pgfqpoint{1.078297in}{1.864811in}}%
\pgfpathlineto{\pgfqpoint{1.078909in}{1.963550in}}%
\pgfpathlineto{\pgfqpoint{1.079112in}{1.741388in}}%
\pgfpathlineto{\pgfqpoint{1.079316in}{1.980006in}}%
\pgfpathlineto{\pgfqpoint{1.079928in}{1.889496in}}%
\pgfpathlineto{\pgfqpoint{1.080539in}{1.955322in}}%
\pgfpathlineto{\pgfqpoint{1.080335in}{1.881268in}}%
\pgfpathlineto{\pgfqpoint{1.080947in}{1.930637in}}%
\pgfpathlineto{\pgfqpoint{1.081354in}{1.873040in}}%
\pgfpathlineto{\pgfqpoint{1.081558in}{2.004691in}}%
\pgfpathlineto{\pgfqpoint{1.081762in}{2.054060in}}%
\pgfpathlineto{\pgfqpoint{1.082373in}{2.004691in}}%
\pgfpathlineto{\pgfqpoint{1.083800in}{1.873040in}}%
\pgfpathlineto{\pgfqpoint{1.084004in}{1.873040in}}%
\pgfpathlineto{\pgfqpoint{1.085023in}{1.971778in}}%
\pgfpathlineto{\pgfqpoint{1.085430in}{1.930637in}}%
\pgfpathlineto{\pgfqpoint{1.086653in}{1.815442in}}%
\pgfpathlineto{\pgfqpoint{1.088080in}{2.021147in}}%
\pgfpathlineto{\pgfqpoint{1.089303in}{1.757845in}}%
\pgfpathlineto{\pgfqpoint{1.090118in}{2.037604in}}%
\pgfpathlineto{\pgfqpoint{1.090525in}{1.980006in}}%
\pgfpathlineto{\pgfqpoint{1.090933in}{1.724932in}}%
\pgfpathlineto{\pgfqpoint{1.091137in}{1.996463in}}%
\pgfpathlineto{\pgfqpoint{1.091544in}{1.996463in}}%
\pgfpathlineto{\pgfqpoint{1.092156in}{2.045832in}}%
\pgfpathlineto{\pgfqpoint{1.092563in}{2.037604in}}%
\pgfpathlineto{\pgfqpoint{1.094194in}{1.864811in}}%
\pgfpathlineto{\pgfqpoint{1.094805in}{1.930637in}}%
\pgfpathlineto{\pgfqpoint{1.095213in}{1.873040in}}%
\pgfpathlineto{\pgfqpoint{1.095620in}{1.873040in}}%
\pgfpathlineto{\pgfqpoint{1.095824in}{1.840127in}}%
\pgfpathlineto{\pgfqpoint{1.096028in}{1.905952in}}%
\pgfpathlineto{\pgfqpoint{1.096232in}{1.897724in}}%
\pgfpathlineto{\pgfqpoint{1.097047in}{2.086973in}}%
\pgfpathlineto{\pgfqpoint{1.097455in}{1.963550in}}%
\pgfpathlineto{\pgfqpoint{1.097862in}{1.905952in}}%
\pgfpathlineto{\pgfqpoint{1.098270in}{1.922409in}}%
\pgfpathlineto{\pgfqpoint{1.098881in}{2.078745in}}%
\pgfpathlineto{\pgfqpoint{1.099289in}{1.971778in}}%
\pgfpathlineto{\pgfqpoint{1.100104in}{1.905952in}}%
\pgfpathlineto{\pgfqpoint{1.100512in}{1.914181in}}%
\pgfpathlineto{\pgfqpoint{1.101327in}{1.988234in}}%
\pgfpathlineto{\pgfqpoint{1.101735in}{1.938865in}}%
\pgfpathlineto{\pgfqpoint{1.102957in}{2.012919in}}%
\pgfpathlineto{\pgfqpoint{1.103365in}{1.930637in}}%
\pgfpathlineto{\pgfqpoint{1.103976in}{2.004691in}}%
\pgfpathlineto{\pgfqpoint{1.104384in}{1.996463in}}%
\pgfpathlineto{\pgfqpoint{1.104588in}{2.021147in}}%
\pgfpathlineto{\pgfqpoint{1.104792in}{2.012919in}}%
\pgfpathlineto{\pgfqpoint{1.105403in}{2.086973in}}%
\pgfpathlineto{\pgfqpoint{1.105607in}{2.004691in}}%
\pgfpathlineto{\pgfqpoint{1.107033in}{1.848355in}}%
\pgfpathlineto{\pgfqpoint{1.106218in}{2.029375in}}%
\pgfpathlineto{\pgfqpoint{1.107645in}{1.889496in}}%
\pgfpathlineto{\pgfqpoint{1.107849in}{1.947093in}}%
\pgfpathlineto{\pgfqpoint{1.108052in}{1.881268in}}%
\pgfpathlineto{\pgfqpoint{1.108460in}{1.914181in}}%
\pgfpathlineto{\pgfqpoint{1.109479in}{1.782529in}}%
\pgfpathlineto{\pgfqpoint{1.109683in}{1.848355in}}%
\pgfpathlineto{\pgfqpoint{1.111109in}{1.675563in}}%
\pgfpathlineto{\pgfqpoint{1.111721in}{1.766073in}}%
\pgfpathlineto{\pgfqpoint{1.112128in}{1.692019in}}%
\pgfpathlineto{\pgfqpoint{1.112332in}{1.700247in}}%
\pgfpathlineto{\pgfqpoint{1.113147in}{1.692019in}}%
\pgfpathlineto{\pgfqpoint{1.113555in}{1.782529in}}%
\pgfpathlineto{\pgfqpoint{1.114167in}{1.831899in}}%
\pgfpathlineto{\pgfqpoint{1.114370in}{1.782529in}}%
\pgfpathlineto{\pgfqpoint{1.114574in}{1.766073in}}%
\pgfpathlineto{\pgfqpoint{1.114982in}{1.815442in}}%
\pgfpathlineto{\pgfqpoint{1.115186in}{1.864811in}}%
\pgfpathlineto{\pgfqpoint{1.115797in}{1.749617in}}%
\pgfpathlineto{\pgfqpoint{1.116612in}{1.552140in}}%
\pgfpathlineto{\pgfqpoint{1.116816in}{1.700247in}}%
\pgfpathlineto{\pgfqpoint{1.117835in}{1.576824in}}%
\pgfpathlineto{\pgfqpoint{1.118243in}{1.609737in}}%
\pgfpathlineto{\pgfqpoint{1.119669in}{1.766073in}}%
\pgfpathlineto{\pgfqpoint{1.120484in}{1.502771in}}%
\pgfpathlineto{\pgfqpoint{1.121096in}{1.634422in}}%
\pgfpathlineto{\pgfqpoint{1.121300in}{1.634422in}}%
\pgfpathlineto{\pgfqpoint{1.123134in}{1.774301in}}%
\pgfpathlineto{\pgfqpoint{1.123541in}{1.815442in}}%
\pgfpathlineto{\pgfqpoint{1.123745in}{1.782529in}}%
\pgfpathlineto{\pgfqpoint{1.124560in}{1.700247in}}%
\pgfpathlineto{\pgfqpoint{1.125172in}{1.733160in}}%
\pgfpathlineto{\pgfqpoint{1.125783in}{1.692019in}}%
\pgfpathlineto{\pgfqpoint{1.126191in}{1.642650in}}%
\pgfpathlineto{\pgfqpoint{1.126802in}{1.700247in}}%
\pgfpathlineto{\pgfqpoint{1.127618in}{1.675563in}}%
\pgfpathlineto{\pgfqpoint{1.128229in}{1.774301in}}%
\pgfpathlineto{\pgfqpoint{1.130063in}{1.642650in}}%
\pgfpathlineto{\pgfqpoint{1.130675in}{1.724932in}}%
\pgfpathlineto{\pgfqpoint{1.131082in}{1.634422in}}%
\pgfpathlineto{\pgfqpoint{1.131286in}{1.700247in}}%
\pgfpathlineto{\pgfqpoint{1.131694in}{1.535683in}}%
\pgfpathlineto{\pgfqpoint{1.132509in}{1.667335in}}%
\pgfpathlineto{\pgfqpoint{1.132713in}{1.659106in}}%
\pgfpathlineto{\pgfqpoint{1.132916in}{1.733160in}}%
\pgfpathlineto{\pgfqpoint{1.133120in}{1.617965in}}%
\pgfpathlineto{\pgfqpoint{1.133732in}{1.659106in}}%
\pgfpathlineto{\pgfqpoint{1.133935in}{1.642650in}}%
\pgfpathlineto{\pgfqpoint{1.134139in}{1.716704in}}%
\pgfpathlineto{\pgfqpoint{1.134751in}{1.634422in}}%
\pgfpathlineto{\pgfqpoint{1.135973in}{1.552140in}}%
\pgfpathlineto{\pgfqpoint{1.136177in}{1.552140in}}%
\pgfpathlineto{\pgfqpoint{1.136585in}{1.461630in}}%
\pgfpathlineto{\pgfqpoint{1.137196in}{1.486314in}}%
\pgfpathlineto{\pgfqpoint{1.137400in}{1.593281in}}%
\pgfpathlineto{\pgfqpoint{1.138215in}{1.527455in}}%
\pgfpathlineto{\pgfqpoint{1.138419in}{1.494542in}}%
\pgfpathlineto{\pgfqpoint{1.138623in}{1.585053in}}%
\pgfpathlineto{\pgfqpoint{1.139030in}{1.519227in}}%
\pgfpathlineto{\pgfqpoint{1.140253in}{1.667335in}}%
\pgfpathlineto{\pgfqpoint{1.141680in}{1.535683in}}%
\pgfpathlineto{\pgfqpoint{1.142088in}{1.527455in}}%
\pgfpathlineto{\pgfqpoint{1.142903in}{1.576824in}}%
\pgfpathlineto{\pgfqpoint{1.143718in}{1.379348in}}%
\pgfpathlineto{\pgfqpoint{1.144126in}{1.494542in}}%
\pgfpathlineto{\pgfqpoint{1.144737in}{1.395804in}}%
\pgfpathlineto{\pgfqpoint{1.145348in}{1.453401in}}%
\pgfpathlineto{\pgfqpoint{1.145552in}{1.445173in}}%
\pgfpathlineto{\pgfqpoint{1.145756in}{1.486314in}}%
\pgfpathlineto{\pgfqpoint{1.146164in}{1.371119in}}%
\pgfpathlineto{\pgfqpoint{1.146571in}{1.453401in}}%
\pgfpathlineto{\pgfqpoint{1.147794in}{1.379348in}}%
\pgfpathlineto{\pgfqpoint{1.148405in}{1.420489in}}%
\pgfpathlineto{\pgfqpoint{1.149017in}{1.395804in}}%
\pgfpathlineto{\pgfqpoint{1.149221in}{1.404032in}}%
\pgfpathlineto{\pgfqpoint{1.150036in}{1.535683in}}%
\pgfpathlineto{\pgfqpoint{1.150647in}{1.502771in}}%
\pgfpathlineto{\pgfqpoint{1.151259in}{1.527455in}}%
\pgfpathlineto{\pgfqpoint{1.151870in}{1.428717in}}%
\pgfpathlineto{\pgfqpoint{1.152074in}{1.428717in}}%
\pgfpathlineto{\pgfqpoint{1.152481in}{1.519227in}}%
\pgfpathlineto{\pgfqpoint{1.152685in}{1.264153in}}%
\pgfpathlineto{\pgfqpoint{1.153093in}{1.601509in}}%
\pgfpathlineto{\pgfqpoint{1.153500in}{1.494542in}}%
\pgfpathlineto{\pgfqpoint{1.154316in}{1.585053in}}%
\pgfpathlineto{\pgfqpoint{1.154723in}{1.535683in}}%
\pgfpathlineto{\pgfqpoint{1.154927in}{1.486314in}}%
\pgfpathlineto{\pgfqpoint{1.155131in}{1.560368in}}%
\pgfpathlineto{\pgfqpoint{1.155335in}{1.519227in}}%
\pgfpathlineto{\pgfqpoint{1.155539in}{1.626194in}}%
\pgfpathlineto{\pgfqpoint{1.156354in}{1.535683in}}%
\pgfpathlineto{\pgfqpoint{1.157169in}{1.552140in}}%
\pgfpathlineto{\pgfqpoint{1.158392in}{1.404032in}}%
\pgfpathlineto{\pgfqpoint{1.158596in}{1.445173in}}%
\pgfpathlineto{\pgfqpoint{1.159003in}{1.453401in}}%
\pgfpathlineto{\pgfqpoint{1.160226in}{1.576824in}}%
\pgfpathlineto{\pgfqpoint{1.160430in}{1.568596in}}%
\pgfpathlineto{\pgfqpoint{1.160837in}{1.519227in}}%
\pgfpathlineto{\pgfqpoint{1.161245in}{1.593281in}}%
\pgfpathlineto{\pgfqpoint{1.161653in}{1.576824in}}%
\pgfpathlineto{\pgfqpoint{1.163691in}{1.798986in}}%
\pgfpathlineto{\pgfqpoint{1.164506in}{1.626194in}}%
\pgfpathlineto{\pgfqpoint{1.164913in}{1.733160in}}%
\pgfpathlineto{\pgfqpoint{1.165117in}{1.692019in}}%
\pgfpathlineto{\pgfqpoint{1.165729in}{1.412260in}}%
\pgfpathlineto{\pgfqpoint{1.166136in}{1.659106in}}%
\pgfpathlineto{\pgfqpoint{1.166544in}{1.683791in}}%
\pgfpathlineto{\pgfqpoint{1.167359in}{1.585053in}}%
\pgfpathlineto{\pgfqpoint{1.168174in}{1.626194in}}%
\pgfpathlineto{\pgfqpoint{1.168582in}{1.601509in}}%
\pgfpathlineto{\pgfqpoint{1.168786in}{1.362891in}}%
\pgfpathlineto{\pgfqpoint{1.169397in}{1.766073in}}%
\pgfpathlineto{\pgfqpoint{1.170824in}{1.642650in}}%
\pgfpathlineto{\pgfqpoint{1.171028in}{1.692019in}}%
\pgfpathlineto{\pgfqpoint{1.171231in}{1.667335in}}%
\pgfpathlineto{\pgfqpoint{1.171435in}{1.428717in}}%
\pgfpathlineto{\pgfqpoint{1.172250in}{1.626194in}}%
\pgfpathlineto{\pgfqpoint{1.172454in}{1.634422in}}%
\pgfpathlineto{\pgfqpoint{1.173066in}{1.527455in}}%
\pgfpathlineto{\pgfqpoint{1.173473in}{1.585053in}}%
\pgfpathlineto{\pgfqpoint{1.173881in}{1.650878in}}%
\pgfpathlineto{\pgfqpoint{1.174085in}{1.601509in}}%
\pgfpathlineto{\pgfqpoint{1.175919in}{1.453401in}}%
\pgfpathlineto{\pgfqpoint{1.177345in}{1.585053in}}%
\pgfpathlineto{\pgfqpoint{1.177957in}{1.576824in}}%
\pgfpathlineto{\pgfqpoint{1.178976in}{1.329978in}}%
\pgfpathlineto{\pgfqpoint{1.179180in}{1.362891in}}%
\pgfpathlineto{\pgfqpoint{1.179383in}{1.601509in}}%
\pgfpathlineto{\pgfqpoint{1.180402in}{1.543912in}}%
\pgfpathlineto{\pgfqpoint{1.181422in}{1.634422in}}%
\pgfpathlineto{\pgfqpoint{1.181014in}{1.535683in}}%
\pgfpathlineto{\pgfqpoint{1.181625in}{1.568596in}}%
\pgfpathlineto{\pgfqpoint{1.182441in}{1.543912in}}%
\pgfpathlineto{\pgfqpoint{1.183460in}{1.617965in}}%
\pgfpathlineto{\pgfqpoint{1.183663in}{1.585053in}}%
\pgfpathlineto{\pgfqpoint{1.184479in}{1.510999in}}%
\pgfpathlineto{\pgfqpoint{1.184682in}{1.576824in}}%
\pgfpathlineto{\pgfqpoint{1.185090in}{1.519227in}}%
\pgfpathlineto{\pgfqpoint{1.185294in}{1.428717in}}%
\pgfpathlineto{\pgfqpoint{1.186109in}{1.519227in}}%
\pgfpathlineto{\pgfqpoint{1.186313in}{1.527455in}}%
\pgfpathlineto{\pgfqpoint{1.186517in}{1.362891in}}%
\pgfpathlineto{\pgfqpoint{1.186720in}{1.560368in}}%
\pgfpathlineto{\pgfqpoint{1.187332in}{1.552140in}}%
\pgfpathlineto{\pgfqpoint{1.187536in}{1.469858in}}%
\pgfpathlineto{\pgfqpoint{1.187739in}{1.601509in}}%
\pgfpathlineto{\pgfqpoint{1.188351in}{1.576824in}}%
\pgfpathlineto{\pgfqpoint{1.189777in}{1.494542in}}%
\pgfpathlineto{\pgfqpoint{1.190593in}{1.593281in}}%
\pgfpathlineto{\pgfqpoint{1.190185in}{1.478086in}}%
\pgfpathlineto{\pgfqpoint{1.190796in}{1.560368in}}%
\pgfpathlineto{\pgfqpoint{1.191000in}{1.469858in}}%
\pgfpathlineto{\pgfqpoint{1.192019in}{1.494542in}}%
\pgfpathlineto{\pgfqpoint{1.192223in}{1.494542in}}%
\pgfpathlineto{\pgfqpoint{1.192631in}{1.527455in}}%
\pgfpathlineto{\pgfqpoint{1.192834in}{1.469858in}}%
\pgfpathlineto{\pgfqpoint{1.193038in}{1.469858in}}%
\pgfpathlineto{\pgfqpoint{1.193242in}{1.469858in}}%
\pgfpathlineto{\pgfqpoint{1.193446in}{1.453401in}}%
\pgfpathlineto{\pgfqpoint{1.193650in}{1.280609in}}%
\pgfpathlineto{\pgfqpoint{1.194057in}{1.601509in}}%
\pgfpathlineto{\pgfqpoint{1.194465in}{1.486314in}}%
\pgfpathlineto{\pgfqpoint{1.195484in}{1.568596in}}%
\pgfpathlineto{\pgfqpoint{1.196503in}{1.552140in}}%
\pgfpathlineto{\pgfqpoint{1.196707in}{1.642650in}}%
\pgfpathlineto{\pgfqpoint{1.197726in}{1.626194in}}%
\pgfpathlineto{\pgfqpoint{1.197930in}{1.617965in}}%
\pgfpathlineto{\pgfqpoint{1.198133in}{1.510999in}}%
\pgfpathlineto{\pgfqpoint{1.198949in}{1.568596in}}%
\pgfpathlineto{\pgfqpoint{1.199152in}{1.568596in}}%
\pgfpathlineto{\pgfqpoint{1.199764in}{1.494542in}}%
\pgfpathlineto{\pgfqpoint{1.200171in}{1.535683in}}%
\pgfpathlineto{\pgfqpoint{1.201802in}{1.683791in}}%
\pgfpathlineto{\pgfqpoint{1.202006in}{1.642650in}}%
\pgfpathlineto{\pgfqpoint{1.203025in}{1.469858in}}%
\pgfpathlineto{\pgfqpoint{1.203228in}{1.552140in}}%
\pgfpathlineto{\pgfqpoint{1.203636in}{1.576824in}}%
\pgfpathlineto{\pgfqpoint{1.204044in}{1.535683in}}%
\pgfpathlineto{\pgfqpoint{1.204859in}{1.601509in}}%
\pgfpathlineto{\pgfqpoint{1.205470in}{1.362891in}}%
\pgfpathlineto{\pgfqpoint{1.205674in}{1.297066in}}%
\pgfpathlineto{\pgfqpoint{1.205878in}{1.560368in}}%
\pgfpathlineto{\pgfqpoint{1.206285in}{1.428717in}}%
\pgfpathlineto{\pgfqpoint{1.206489in}{1.412260in}}%
\pgfpathlineto{\pgfqpoint{1.206897in}{1.173643in}}%
\pgfpathlineto{\pgfqpoint{1.207508in}{1.412260in}}%
\pgfpathlineto{\pgfqpoint{1.207712in}{1.436945in}}%
\pgfpathlineto{\pgfqpoint{1.209139in}{1.214784in}}%
\pgfpathlineto{\pgfqpoint{1.209546in}{1.099589in}}%
\pgfpathlineto{\pgfqpoint{1.210158in}{1.140730in}}%
\pgfpathlineto{\pgfqpoint{1.210973in}{1.083132in}}%
\pgfpathlineto{\pgfqpoint{1.211177in}{1.255925in}}%
\pgfpathlineto{\pgfqpoint{1.211584in}{1.083132in}}%
\pgfpathlineto{\pgfqpoint{1.212400in}{1.124273in}}%
\pgfpathlineto{\pgfqpoint{1.214030in}{1.280609in}}%
\pgfpathlineto{\pgfqpoint{1.215457in}{1.132502in}}%
\pgfpathlineto{\pgfqpoint{1.215660in}{1.190099in}}%
\pgfpathlineto{\pgfqpoint{1.216272in}{1.074904in}}%
\pgfpathlineto{\pgfqpoint{1.217291in}{1.124273in}}%
\pgfpathlineto{\pgfqpoint{1.216679in}{1.050220in}}%
\pgfpathlineto{\pgfqpoint{1.217698in}{1.116045in}}%
\pgfpathlineto{\pgfqpoint{1.217902in}{1.107817in}}%
\pgfpathlineto{\pgfqpoint{1.219329in}{0.819830in}}%
\pgfpathlineto{\pgfqpoint{1.220144in}{1.050220in}}%
\pgfpathlineto{\pgfqpoint{1.220552in}{1.033763in}}%
\pgfpathlineto{\pgfqpoint{1.220755in}{1.033763in}}%
\pgfpathlineto{\pgfqpoint{1.220959in}{1.009079in}}%
\pgfpathlineto{\pgfqpoint{1.221367in}{1.165414in}}%
\pgfpathlineto{\pgfqpoint{1.221978in}{1.058448in}}%
\pgfpathlineto{\pgfqpoint{1.223201in}{0.976166in}}%
\pgfpathlineto{\pgfqpoint{1.224016in}{1.272381in}}%
\pgfpathlineto{\pgfqpoint{1.224424in}{1.165414in}}%
\pgfpathlineto{\pgfqpoint{1.225647in}{1.009079in}}%
\pgfpathlineto{\pgfqpoint{1.225851in}{1.033763in}}%
\pgfpathlineto{\pgfqpoint{1.226666in}{0.836286in}}%
\pgfpathlineto{\pgfqpoint{1.227073in}{0.910340in}}%
\pgfpathlineto{\pgfqpoint{1.227481in}{0.860971in}}%
\pgfpathlineto{\pgfqpoint{1.228296in}{0.877427in}}%
\pgfpathlineto{\pgfqpoint{1.228500in}{0.943253in}}%
\pgfpathlineto{\pgfqpoint{1.229519in}{0.935025in}}%
\pgfpathlineto{\pgfqpoint{1.229723in}{0.893884in}}%
\pgfpathlineto{\pgfqpoint{1.230334in}{0.984394in}}%
\pgfpathlineto{\pgfqpoint{1.230538in}{0.951481in}}%
\pgfpathlineto{\pgfqpoint{1.231557in}{1.066676in}}%
\pgfpathlineto{\pgfqpoint{1.232168in}{1.041991in}}%
\pgfpathlineto{\pgfqpoint{1.232576in}{0.935025in}}%
\pgfpathlineto{\pgfqpoint{1.233187in}{1.050220in}}%
\pgfpathlineto{\pgfqpoint{1.233799in}{1.017307in}}%
\pgfpathlineto{\pgfqpoint{1.234003in}{1.033763in}}%
\pgfpathlineto{\pgfqpoint{1.234206in}{1.099589in}}%
\pgfpathlineto{\pgfqpoint{1.234818in}{1.017307in}}%
\pgfpathlineto{\pgfqpoint{1.235022in}{1.017307in}}%
\pgfpathlineto{\pgfqpoint{1.235429in}{1.058448in}}%
\pgfpathlineto{\pgfqpoint{1.235633in}{1.009079in}}%
\pgfpathlineto{\pgfqpoint{1.237060in}{0.885656in}}%
\pgfpathlineto{\pgfqpoint{1.237467in}{1.025535in}}%
\pgfpathlineto{\pgfqpoint{1.238690in}{1.009079in}}%
\pgfpathlineto{\pgfqpoint{1.239505in}{0.910340in}}%
\pgfpathlineto{\pgfqpoint{1.239098in}{1.041991in}}%
\pgfpathlineto{\pgfqpoint{1.240117in}{0.943253in}}%
\pgfpathlineto{\pgfqpoint{1.241747in}{1.066676in}}%
\pgfpathlineto{\pgfqpoint{1.240524in}{0.935025in}}%
\pgfpathlineto{\pgfqpoint{1.241951in}{1.009079in}}%
\pgfpathlineto{\pgfqpoint{1.242155in}{1.017307in}}%
\pgfpathlineto{\pgfqpoint{1.242359in}{0.992622in}}%
\pgfpathlineto{\pgfqpoint{1.243581in}{1.173643in}}%
\pgfpathlineto{\pgfqpoint{1.243785in}{1.099589in}}%
\pgfpathlineto{\pgfqpoint{1.244193in}{1.280609in}}%
\pgfpathlineto{\pgfqpoint{1.244397in}{1.255925in}}%
\pgfpathlineto{\pgfqpoint{1.246638in}{1.510999in}}%
\pgfpathlineto{\pgfqpoint{1.248269in}{1.387576in}}%
\pgfpathlineto{\pgfqpoint{1.248473in}{1.445173in}}%
\pgfpathlineto{\pgfqpoint{1.248677in}{1.214784in}}%
\pgfpathlineto{\pgfqpoint{1.249492in}{1.502771in}}%
\pgfpathlineto{\pgfqpoint{1.250103in}{1.535683in}}%
\pgfpathlineto{\pgfqpoint{1.250511in}{1.510999in}}%
\pgfpathlineto{\pgfqpoint{1.251122in}{1.486314in}}%
\pgfpathlineto{\pgfqpoint{1.251326in}{1.519227in}}%
\pgfpathlineto{\pgfqpoint{1.251530in}{1.502771in}}%
\pgfpathlineto{\pgfqpoint{1.251734in}{1.543912in}}%
\pgfpathlineto{\pgfqpoint{1.252345in}{1.494542in}}%
\pgfpathlineto{\pgfqpoint{1.252753in}{1.527455in}}%
\pgfpathlineto{\pgfqpoint{1.252956in}{1.502771in}}%
\pgfpathlineto{\pgfqpoint{1.253160in}{1.601509in}}%
\pgfpathlineto{\pgfqpoint{1.253364in}{1.560368in}}%
\pgfpathlineto{\pgfqpoint{1.254179in}{1.733160in}}%
\pgfpathlineto{\pgfqpoint{1.254994in}{1.708476in}}%
\pgfpathlineto{\pgfqpoint{1.255198in}{1.683791in}}%
\pgfpathlineto{\pgfqpoint{1.255606in}{1.741388in}}%
\pgfpathlineto{\pgfqpoint{1.256217in}{1.848355in}}%
\pgfpathlineto{\pgfqpoint{1.256625in}{1.815442in}}%
\pgfpathlineto{\pgfqpoint{1.257236in}{1.733160in}}%
\pgfpathlineto{\pgfqpoint{1.257644in}{1.757845in}}%
\pgfpathlineto{\pgfqpoint{1.259070in}{1.889496in}}%
\pgfpathlineto{\pgfqpoint{1.260089in}{1.798986in}}%
\pgfpathlineto{\pgfqpoint{1.260293in}{1.807214in}}%
\pgfpathlineto{\pgfqpoint{1.261108in}{1.914181in}}%
\pgfpathlineto{\pgfqpoint{1.260701in}{1.798986in}}%
\pgfpathlineto{\pgfqpoint{1.261312in}{1.897724in}}%
\pgfpathlineto{\pgfqpoint{1.262535in}{1.815442in}}%
\pgfpathlineto{\pgfqpoint{1.263554in}{1.922409in}}%
\pgfpathlineto{\pgfqpoint{1.263147in}{1.798986in}}%
\pgfpathlineto{\pgfqpoint{1.263758in}{1.881268in}}%
\pgfpathlineto{\pgfqpoint{1.263962in}{1.856583in}}%
\pgfpathlineto{\pgfqpoint{1.264166in}{1.905952in}}%
\pgfpathlineto{\pgfqpoint{1.264573in}{1.905952in}}%
\pgfpathlineto{\pgfqpoint{1.264777in}{1.963550in}}%
\pgfpathlineto{\pgfqpoint{1.265185in}{1.881268in}}%
\pgfpathlineto{\pgfqpoint{1.265592in}{1.897724in}}%
\pgfpathlineto{\pgfqpoint{1.265796in}{1.905952in}}%
\pgfpathlineto{\pgfqpoint{1.266000in}{1.881268in}}%
\pgfpathlineto{\pgfqpoint{1.266204in}{1.889496in}}%
\pgfpathlineto{\pgfqpoint{1.266611in}{1.840127in}}%
\pgfpathlineto{\pgfqpoint{1.266815in}{1.897724in}}%
\pgfpathlineto{\pgfqpoint{1.267019in}{1.897724in}}%
\pgfpathlineto{\pgfqpoint{1.267630in}{2.045832in}}%
\pgfpathlineto{\pgfqpoint{1.268445in}{1.996463in}}%
\pgfpathlineto{\pgfqpoint{1.268649in}{2.012919in}}%
\pgfpathlineto{\pgfqpoint{1.268853in}{1.988234in}}%
\pgfpathlineto{\pgfqpoint{1.269057in}{1.922409in}}%
\pgfpathlineto{\pgfqpoint{1.269872in}{2.012919in}}%
\pgfpathlineto{\pgfqpoint{1.270280in}{1.971778in}}%
\pgfpathlineto{\pgfqpoint{1.270891in}{2.029375in}}%
\pgfpathlineto{\pgfqpoint{1.271095in}{2.045832in}}%
\pgfpathlineto{\pgfqpoint{1.271910in}{1.938865in}}%
\pgfpathlineto{\pgfqpoint{1.272114in}{1.996463in}}%
\pgfpathlineto{\pgfqpoint{1.272929in}{2.095201in}}%
\pgfpathlineto{\pgfqpoint{1.273337in}{2.078745in}}%
\pgfpathlineto{\pgfqpoint{1.273948in}{2.103429in}}%
\pgfpathlineto{\pgfqpoint{1.274559in}{2.012919in}}%
\pgfpathlineto{\pgfqpoint{1.274763in}{2.078745in}}%
\pgfpathlineto{\pgfqpoint{1.275375in}{1.971778in}}%
\pgfpathlineto{\pgfqpoint{1.275579in}{2.029375in}}%
\pgfpathlineto{\pgfqpoint{1.275782in}{2.012919in}}%
\pgfpathlineto{\pgfqpoint{1.275986in}{2.037604in}}%
\pgfpathlineto{\pgfqpoint{1.277209in}{2.103429in}}%
\pgfpathlineto{\pgfqpoint{1.277413in}{2.103429in}}%
\pgfpathlineto{\pgfqpoint{1.277617in}{2.177483in}}%
\pgfpathlineto{\pgfqpoint{1.278636in}{2.144570in}}%
\pgfpathlineto{\pgfqpoint{1.278839in}{2.185711in}}%
\pgfpathlineto{\pgfqpoint{1.279247in}{2.136342in}}%
\pgfpathlineto{\pgfqpoint{1.280470in}{1.881268in}}%
\pgfpathlineto{\pgfqpoint{1.281693in}{2.152798in}}%
\pgfpathlineto{\pgfqpoint{1.282508in}{1.922409in}}%
\pgfpathlineto{\pgfqpoint{1.282915in}{2.111657in}}%
\pgfpathlineto{\pgfqpoint{1.283934in}{2.185711in}}%
\pgfpathlineto{\pgfqpoint{1.284138in}{2.136342in}}%
\pgfpathlineto{\pgfqpoint{1.284750in}{2.235080in}}%
\pgfpathlineto{\pgfqpoint{1.284953in}{2.267993in}}%
\pgfpathlineto{\pgfqpoint{1.285157in}{2.202168in}}%
\pgfpathlineto{\pgfqpoint{1.285769in}{2.235080in}}%
\pgfpathlineto{\pgfqpoint{1.285972in}{2.235080in}}%
\pgfpathlineto{\pgfqpoint{1.286788in}{2.095201in}}%
\pgfpathlineto{\pgfqpoint{1.287399in}{2.169255in}}%
\pgfpathlineto{\pgfqpoint{1.287603in}{2.177483in}}%
\pgfpathlineto{\pgfqpoint{1.287807in}{2.152798in}}%
\pgfpathlineto{\pgfqpoint{1.288214in}{2.037604in}}%
\pgfpathlineto{\pgfqpoint{1.289030in}{2.070516in}}%
\pgfpathlineto{\pgfqpoint{1.289233in}{2.070516in}}%
\pgfpathlineto{\pgfqpoint{1.289437in}{2.062288in}}%
\pgfpathlineto{\pgfqpoint{1.289641in}{2.086973in}}%
\pgfpathlineto{\pgfqpoint{1.290049in}{2.152798in}}%
\pgfpathlineto{\pgfqpoint{1.290252in}{2.095201in}}%
\pgfpathlineto{\pgfqpoint{1.290660in}{2.012919in}}%
\pgfpathlineto{\pgfqpoint{1.291271in}{2.119886in}}%
\pgfpathlineto{\pgfqpoint{1.291883in}{2.062288in}}%
\pgfpathlineto{\pgfqpoint{1.292087in}{2.136342in}}%
\pgfpathlineto{\pgfqpoint{1.292290in}{2.103429in}}%
\pgfpathlineto{\pgfqpoint{1.292902in}{2.136342in}}%
\pgfpathlineto{\pgfqpoint{1.293106in}{2.086973in}}%
\pgfpathlineto{\pgfqpoint{1.293309in}{2.095201in}}%
\pgfpathlineto{\pgfqpoint{1.293513in}{2.062288in}}%
\pgfpathlineto{\pgfqpoint{1.294328in}{2.037604in}}%
\pgfpathlineto{\pgfqpoint{1.295144in}{2.103429in}}%
\pgfpathlineto{\pgfqpoint{1.295551in}{1.980006in}}%
\pgfpathlineto{\pgfqpoint{1.296366in}{2.045832in}}%
\pgfpathlineto{\pgfqpoint{1.296978in}{1.996463in}}%
\pgfpathlineto{\pgfqpoint{1.297182in}{2.078745in}}%
\pgfpathlineto{\pgfqpoint{1.297589in}{2.111657in}}%
\pgfpathlineto{\pgfqpoint{1.297997in}{2.078745in}}%
\pgfpathlineto{\pgfqpoint{1.298812in}{2.095201in}}%
\pgfpathlineto{\pgfqpoint{1.299220in}{1.988234in}}%
\pgfpathlineto{\pgfqpoint{1.300239in}{2.136342in}}%
\pgfpathlineto{\pgfqpoint{1.300442in}{2.111657in}}%
\pgfpathlineto{\pgfqpoint{1.300646in}{2.037604in}}%
\pgfpathlineto{\pgfqpoint{1.301258in}{2.144570in}}%
\pgfpathlineto{\pgfqpoint{1.301665in}{2.062288in}}%
\pgfpathlineto{\pgfqpoint{1.302073in}{2.161027in}}%
\pgfpathlineto{\pgfqpoint{1.302684in}{2.119886in}}%
\pgfpathlineto{\pgfqpoint{1.303907in}{2.045832in}}%
\pgfpathlineto{\pgfqpoint{1.304315in}{2.062288in}}%
\pgfpathlineto{\pgfqpoint{1.304519in}{2.029375in}}%
\pgfpathlineto{\pgfqpoint{1.304722in}{1.996463in}}%
\pgfpathlineto{\pgfqpoint{1.304926in}{2.111657in}}%
\pgfpathlineto{\pgfqpoint{1.305130in}{2.086973in}}%
\pgfpathlineto{\pgfqpoint{1.305741in}{2.144570in}}%
\pgfpathlineto{\pgfqpoint{1.305945in}{2.119886in}}%
\pgfpathlineto{\pgfqpoint{1.306964in}{2.037604in}}%
\pgfpathlineto{\pgfqpoint{1.307168in}{2.045832in}}%
\pgfpathlineto{\pgfqpoint{1.307372in}{2.037604in}}%
\pgfpathlineto{\pgfqpoint{1.307576in}{2.054060in}}%
\pgfpathlineto{\pgfqpoint{1.307779in}{2.111657in}}%
\pgfpathlineto{\pgfqpoint{1.308391in}{2.021147in}}%
\pgfpathlineto{\pgfqpoint{1.308798in}{1.988234in}}%
\pgfpathlineto{\pgfqpoint{1.309206in}{2.029375in}}%
\pgfpathlineto{\pgfqpoint{1.309410in}{2.037604in}}%
\pgfpathlineto{\pgfqpoint{1.310429in}{2.136342in}}%
\pgfpathlineto{\pgfqpoint{1.310633in}{2.111657in}}%
\pgfpathlineto{\pgfqpoint{1.310836in}{2.111657in}}%
\pgfpathlineto{\pgfqpoint{1.312671in}{1.980006in}}%
\pgfpathlineto{\pgfqpoint{1.314097in}{2.103429in}}%
\pgfpathlineto{\pgfqpoint{1.314301in}{2.086973in}}%
\pgfpathlineto{\pgfqpoint{1.314505in}{2.119886in}}%
\pgfpathlineto{\pgfqpoint{1.315116in}{2.103429in}}%
\pgfpathlineto{\pgfqpoint{1.315524in}{2.193939in}}%
\pgfpathlineto{\pgfqpoint{1.315932in}{2.251537in}}%
\pgfpathlineto{\pgfqpoint{1.316747in}{2.095201in}}%
\pgfpathlineto{\pgfqpoint{1.316951in}{2.152798in}}%
\pgfpathlineto{\pgfqpoint{1.317970in}{2.128114in}}%
\pgfpathlineto{\pgfqpoint{1.318785in}{2.045832in}}%
\pgfpathlineto{\pgfqpoint{1.318581in}{2.136342in}}%
\pgfpathlineto{\pgfqpoint{1.319192in}{2.078745in}}%
\pgfpathlineto{\pgfqpoint{1.319396in}{2.078745in}}%
\pgfpathlineto{\pgfqpoint{1.320823in}{2.235080in}}%
\pgfpathlineto{\pgfqpoint{1.321434in}{2.276221in}}%
\pgfpathlineto{\pgfqpoint{1.322249in}{2.095201in}}%
\pgfpathlineto{\pgfqpoint{1.322453in}{2.070516in}}%
\pgfpathlineto{\pgfqpoint{1.322657in}{2.136342in}}%
\pgfpathlineto{\pgfqpoint{1.323065in}{2.119886in}}%
\pgfpathlineto{\pgfqpoint{1.325510in}{2.325591in}}%
\pgfpathlineto{\pgfqpoint{1.325714in}{2.276221in}}%
\pgfpathlineto{\pgfqpoint{1.327344in}{2.152798in}}%
\pgfpathlineto{\pgfqpoint{1.327956in}{2.226852in}}%
\pgfpathlineto{\pgfqpoint{1.328363in}{2.144570in}}%
\pgfpathlineto{\pgfqpoint{1.328567in}{2.193939in}}%
\pgfpathlineto{\pgfqpoint{1.330198in}{1.971778in}}%
\pgfpathlineto{\pgfqpoint{1.330809in}{2.226852in}}%
\pgfpathlineto{\pgfqpoint{1.331421in}{2.144570in}}%
\pgfpathlineto{\pgfqpoint{1.332032in}{2.070516in}}%
\pgfpathlineto{\pgfqpoint{1.332236in}{2.161027in}}%
\pgfpathlineto{\pgfqpoint{1.332440in}{2.161027in}}%
\pgfpathlineto{\pgfqpoint{1.333662in}{2.259765in}}%
\pgfpathlineto{\pgfqpoint{1.334885in}{2.161027in}}%
\pgfpathlineto{\pgfqpoint{1.335089in}{2.169255in}}%
\pgfpathlineto{\pgfqpoint{1.335293in}{2.193939in}}%
\pgfpathlineto{\pgfqpoint{1.335497in}{1.922409in}}%
\pgfpathlineto{\pgfqpoint{1.336312in}{2.144570in}}%
\pgfpathlineto{\pgfqpoint{1.336923in}{2.086973in}}%
\pgfpathlineto{\pgfqpoint{1.337535in}{2.095201in}}%
\pgfpathlineto{\pgfqpoint{1.338554in}{2.152798in}}%
\pgfpathlineto{\pgfqpoint{1.338757in}{2.136342in}}%
\pgfpathlineto{\pgfqpoint{1.340795in}{2.267993in}}%
\pgfpathlineto{\pgfqpoint{1.341814in}{2.152798in}}%
\pgfpathlineto{\pgfqpoint{1.342018in}{2.226852in}}%
\pgfpathlineto{\pgfqpoint{1.342630in}{2.333819in}}%
\pgfpathlineto{\pgfqpoint{1.342834in}{2.284450in}}%
\pgfpathlineto{\pgfqpoint{1.344260in}{2.095201in}}%
\pgfpathlineto{\pgfqpoint{1.344464in}{2.086973in}}%
\pgfpathlineto{\pgfqpoint{1.344668in}{2.103429in}}%
\pgfpathlineto{\pgfqpoint{1.344872in}{2.095201in}}%
\pgfpathlineto{\pgfqpoint{1.346094in}{2.193939in}}%
\pgfpathlineto{\pgfqpoint{1.345687in}{2.070516in}}%
\pgfpathlineto{\pgfqpoint{1.346502in}{2.185711in}}%
\pgfpathlineto{\pgfqpoint{1.346910in}{2.128114in}}%
\pgfpathlineto{\pgfqpoint{1.347521in}{2.012919in}}%
\pgfpathlineto{\pgfqpoint{1.347725in}{2.078745in}}%
\pgfpathlineto{\pgfqpoint{1.348948in}{2.177483in}}%
\pgfpathlineto{\pgfqpoint{1.350170in}{2.111657in}}%
\pgfpathlineto{\pgfqpoint{1.350782in}{2.152798in}}%
\pgfpathlineto{\pgfqpoint{1.351189in}{2.144570in}}%
\pgfpathlineto{\pgfqpoint{1.351393in}{2.111657in}}%
\pgfpathlineto{\pgfqpoint{1.351597in}{2.193939in}}%
\pgfpathlineto{\pgfqpoint{1.351801in}{2.177483in}}%
\pgfpathlineto{\pgfqpoint{1.352005in}{2.193939in}}%
\pgfpathlineto{\pgfqpoint{1.352208in}{2.177483in}}%
\pgfpathlineto{\pgfqpoint{1.353024in}{2.095201in}}%
\pgfpathlineto{\pgfqpoint{1.353431in}{2.119886in}}%
\pgfpathlineto{\pgfqpoint{1.353635in}{2.128114in}}%
\pgfpathlineto{\pgfqpoint{1.353839in}{2.111657in}}%
\pgfpathlineto{\pgfqpoint{1.354043in}{2.086973in}}%
\pgfpathlineto{\pgfqpoint{1.354246in}{2.128114in}}%
\pgfpathlineto{\pgfqpoint{1.354858in}{2.103429in}}%
\pgfpathlineto{\pgfqpoint{1.355062in}{2.144570in}}%
\pgfpathlineto{\pgfqpoint{1.355673in}{2.086973in}}%
\pgfpathlineto{\pgfqpoint{1.355877in}{2.119886in}}%
\pgfpathlineto{\pgfqpoint{1.356285in}{2.037604in}}%
\pgfpathlineto{\pgfqpoint{1.356488in}{2.086973in}}%
\pgfpathlineto{\pgfqpoint{1.357507in}{2.169255in}}%
\pgfpathlineto{\pgfqpoint{1.358119in}{2.103429in}}%
\pgfpathlineto{\pgfqpoint{1.358526in}{2.144570in}}%
\pgfpathlineto{\pgfqpoint{1.359342in}{2.177483in}}%
\pgfpathlineto{\pgfqpoint{1.360564in}{2.078745in}}%
\pgfpathlineto{\pgfqpoint{1.361176in}{2.136342in}}%
\pgfpathlineto{\pgfqpoint{1.360972in}{2.062288in}}%
\pgfpathlineto{\pgfqpoint{1.361380in}{2.062288in}}%
\pgfpathlineto{\pgfqpoint{1.361583in}{1.955322in}}%
\pgfpathlineto{\pgfqpoint{1.361787in}{2.136342in}}%
\pgfpathlineto{\pgfqpoint{1.362195in}{2.095201in}}%
\pgfpathlineto{\pgfqpoint{1.363214in}{2.259765in}}%
\pgfpathlineto{\pgfqpoint{1.364029in}{2.185711in}}%
\pgfpathlineto{\pgfqpoint{1.364437in}{2.218624in}}%
\pgfpathlineto{\pgfqpoint{1.365456in}{1.947093in}}%
\pgfpathlineto{\pgfqpoint{1.366475in}{2.235080in}}%
\pgfpathlineto{\pgfqpoint{1.366678in}{2.185711in}}%
\pgfpathlineto{\pgfqpoint{1.366882in}{2.152798in}}%
\pgfpathlineto{\pgfqpoint{1.367290in}{2.202168in}}%
\pgfpathlineto{\pgfqpoint{1.367494in}{2.185711in}}%
\pgfpathlineto{\pgfqpoint{1.367697in}{2.243309in}}%
\pgfpathlineto{\pgfqpoint{1.368513in}{2.218624in}}%
\pgfpathlineto{\pgfqpoint{1.369124in}{2.136342in}}%
\pgfpathlineto{\pgfqpoint{1.368920in}{2.235080in}}%
\pgfpathlineto{\pgfqpoint{1.369532in}{2.161027in}}%
\pgfpathlineto{\pgfqpoint{1.370143in}{2.276221in}}%
\pgfpathlineto{\pgfqpoint{1.370551in}{2.152798in}}%
\pgfpathlineto{\pgfqpoint{1.371162in}{2.054060in}}%
\pgfpathlineto{\pgfqpoint{1.371977in}{2.062288in}}%
\pgfpathlineto{\pgfqpoint{1.372385in}{2.029375in}}%
\pgfpathlineto{\pgfqpoint{1.373404in}{2.128114in}}%
\pgfpathlineto{\pgfqpoint{1.374831in}{2.054060in}}%
\pgfpathlineto{\pgfqpoint{1.375034in}{1.864811in}}%
\pgfpathlineto{\pgfqpoint{1.375646in}{2.111657in}}%
\pgfpathlineto{\pgfqpoint{1.375850in}{2.062288in}}%
\pgfpathlineto{\pgfqpoint{1.376461in}{2.169255in}}%
\pgfpathlineto{\pgfqpoint{1.377072in}{2.103429in}}%
\pgfpathlineto{\pgfqpoint{1.378499in}{2.267993in}}%
\pgfpathlineto{\pgfqpoint{1.377480in}{2.086973in}}%
\pgfpathlineto{\pgfqpoint{1.378907in}{2.169255in}}%
\pgfpathlineto{\pgfqpoint{1.379722in}{2.086973in}}%
\pgfpathlineto{\pgfqpoint{1.379926in}{2.152798in}}%
\pgfpathlineto{\pgfqpoint{1.380129in}{2.177483in}}%
\pgfpathlineto{\pgfqpoint{1.380333in}{2.078745in}}%
\pgfpathlineto{\pgfqpoint{1.380537in}{2.070516in}}%
\pgfpathlineto{\pgfqpoint{1.381352in}{2.004691in}}%
\pgfpathlineto{\pgfqpoint{1.381556in}{2.070516in}}%
\pgfpathlineto{\pgfqpoint{1.381760in}{2.095201in}}%
\pgfpathlineto{\pgfqpoint{1.382167in}{2.086973in}}%
\pgfpathlineto{\pgfqpoint{1.382371in}{2.004691in}}%
\pgfpathlineto{\pgfqpoint{1.382575in}{2.095201in}}%
\pgfpathlineto{\pgfqpoint{1.383187in}{2.095201in}}%
\pgfpathlineto{\pgfqpoint{1.384206in}{2.021147in}}%
\pgfpathlineto{\pgfqpoint{1.384613in}{2.029375in}}%
\pgfpathlineto{\pgfqpoint{1.385428in}{1.963550in}}%
\pgfpathlineto{\pgfqpoint{1.385632in}{1.996463in}}%
\pgfpathlineto{\pgfqpoint{1.385836in}{2.045832in}}%
\pgfpathlineto{\pgfqpoint{1.386244in}{1.889496in}}%
\pgfpathlineto{\pgfqpoint{1.388689in}{2.136342in}}%
\pgfpathlineto{\pgfqpoint{1.389301in}{2.054060in}}%
\pgfpathlineto{\pgfqpoint{1.389708in}{2.062288in}}%
\pgfpathlineto{\pgfqpoint{1.390320in}{2.136342in}}%
\pgfpathlineto{\pgfqpoint{1.390931in}{2.103429in}}%
\pgfpathlineto{\pgfqpoint{1.391339in}{2.144570in}}%
\pgfpathlineto{\pgfqpoint{1.391542in}{2.111657in}}%
\pgfpathlineto{\pgfqpoint{1.392765in}{2.037604in}}%
\pgfpathlineto{\pgfqpoint{1.393784in}{2.235080in}}%
\pgfpathlineto{\pgfqpoint{1.394396in}{2.185711in}}%
\pgfpathlineto{\pgfqpoint{1.394599in}{2.144570in}}%
\pgfpathlineto{\pgfqpoint{1.395211in}{2.152798in}}%
\pgfpathlineto{\pgfqpoint{1.396026in}{2.259765in}}%
\pgfpathlineto{\pgfqpoint{1.396434in}{2.251537in}}%
\pgfpathlineto{\pgfqpoint{1.397657in}{2.070516in}}%
\pgfpathlineto{\pgfqpoint{1.396841in}{2.259765in}}%
\pgfpathlineto{\pgfqpoint{1.398064in}{2.119886in}}%
\pgfpathlineto{\pgfqpoint{1.399083in}{2.251537in}}%
\pgfpathlineto{\pgfqpoint{1.399287in}{2.226852in}}%
\pgfpathlineto{\pgfqpoint{1.399491in}{2.243309in}}%
\pgfpathlineto{\pgfqpoint{1.399695in}{2.185711in}}%
\pgfpathlineto{\pgfqpoint{1.399898in}{2.185711in}}%
\pgfpathlineto{\pgfqpoint{1.401121in}{2.119886in}}%
\pgfpathlineto{\pgfqpoint{1.401936in}{2.235080in}}%
\pgfpathlineto{\pgfqpoint{1.402548in}{2.210396in}}%
\pgfpathlineto{\pgfqpoint{1.402955in}{2.152798in}}%
\pgfpathlineto{\pgfqpoint{1.403567in}{2.169255in}}%
\pgfpathlineto{\pgfqpoint{1.404586in}{2.235080in}}%
\pgfpathlineto{\pgfqpoint{1.404993in}{2.226852in}}%
\pgfpathlineto{\pgfqpoint{1.405809in}{2.152798in}}%
\pgfpathlineto{\pgfqpoint{1.406012in}{2.169255in}}%
\pgfpathlineto{\pgfqpoint{1.406216in}{2.235080in}}%
\pgfpathlineto{\pgfqpoint{1.406624in}{2.161027in}}%
\pgfpathlineto{\pgfqpoint{1.406828in}{2.161027in}}%
\pgfpathlineto{\pgfqpoint{1.407031in}{2.136342in}}%
\pgfpathlineto{\pgfqpoint{1.407235in}{2.177483in}}%
\pgfpathlineto{\pgfqpoint{1.407439in}{2.161027in}}%
\pgfpathlineto{\pgfqpoint{1.408254in}{2.218624in}}%
\pgfpathlineto{\pgfqpoint{1.408662in}{2.128114in}}%
\pgfpathlineto{\pgfqpoint{1.409069in}{2.152798in}}%
\pgfpathlineto{\pgfqpoint{1.409477in}{2.267993in}}%
\pgfpathlineto{\pgfqpoint{1.410089in}{2.251537in}}%
\pgfpathlineto{\pgfqpoint{1.410292in}{2.202168in}}%
\pgfpathlineto{\pgfqpoint{1.410700in}{2.267993in}}%
\pgfpathlineto{\pgfqpoint{1.411108in}{2.243309in}}%
\pgfpathlineto{\pgfqpoint{1.412534in}{2.144570in}}%
\pgfpathlineto{\pgfqpoint{1.412738in}{2.218624in}}%
\pgfpathlineto{\pgfqpoint{1.413553in}{2.193939in}}%
\pgfpathlineto{\pgfqpoint{1.414368in}{2.144570in}}%
\pgfpathlineto{\pgfqpoint{1.414980in}{2.218624in}}%
\pgfpathlineto{\pgfqpoint{1.415184in}{2.136342in}}%
\pgfpathlineto{\pgfqpoint{1.415387in}{2.177483in}}%
\pgfpathlineto{\pgfqpoint{1.415591in}{2.169255in}}%
\pgfpathlineto{\pgfqpoint{1.417222in}{2.292678in}}%
\pgfpathlineto{\pgfqpoint{1.417629in}{2.185711in}}%
\pgfpathlineto{\pgfqpoint{1.418241in}{2.218624in}}%
\pgfpathlineto{\pgfqpoint{1.419056in}{2.243309in}}%
\pgfpathlineto{\pgfqpoint{1.420075in}{2.169255in}}%
\pgfpathlineto{\pgfqpoint{1.420279in}{2.226852in}}%
\pgfpathlineto{\pgfqpoint{1.420890in}{2.152798in}}%
\pgfpathlineto{\pgfqpoint{1.421094in}{2.161027in}}%
\pgfpathlineto{\pgfqpoint{1.421501in}{2.193939in}}%
\pgfpathlineto{\pgfqpoint{1.422317in}{2.177483in}}%
\pgfpathlineto{\pgfqpoint{1.422724in}{2.119886in}}%
\pgfpathlineto{\pgfqpoint{1.422928in}{2.202168in}}%
\pgfpathlineto{\pgfqpoint{1.423336in}{2.193939in}}%
\pgfpathlineto{\pgfqpoint{1.424559in}{2.251537in}}%
\pgfpathlineto{\pgfqpoint{1.424151in}{2.177483in}}%
\pgfpathlineto{\pgfqpoint{1.424762in}{2.226852in}}%
\pgfpathlineto{\pgfqpoint{1.426393in}{2.136342in}}%
\pgfpathlineto{\pgfqpoint{1.427004in}{2.095201in}}%
\pgfpathlineto{\pgfqpoint{1.427616in}{2.210396in}}%
\pgfpathlineto{\pgfqpoint{1.428838in}{2.128114in}}%
\pgfpathlineto{\pgfqpoint{1.429654in}{2.226852in}}%
\pgfpathlineto{\pgfqpoint{1.430265in}{2.218624in}}%
\pgfpathlineto{\pgfqpoint{1.430469in}{2.185711in}}%
\pgfpathlineto{\pgfqpoint{1.430876in}{2.243309in}}%
\pgfpathlineto{\pgfqpoint{1.431284in}{2.276221in}}%
\pgfpathlineto{\pgfqpoint{1.431692in}{2.218624in}}%
\pgfpathlineto{\pgfqpoint{1.431895in}{2.235080in}}%
\pgfpathlineto{\pgfqpoint{1.432099in}{2.243309in}}%
\pgfpathlineto{\pgfqpoint{1.432303in}{2.226852in}}%
\pgfpathlineto{\pgfqpoint{1.432507in}{2.193939in}}%
\pgfpathlineto{\pgfqpoint{1.433118in}{2.251537in}}%
\pgfpathlineto{\pgfqpoint{1.433322in}{2.235080in}}%
\pgfpathlineto{\pgfqpoint{1.433730in}{2.284450in}}%
\pgfpathlineto{\pgfqpoint{1.433933in}{2.267993in}}%
\pgfpathlineto{\pgfqpoint{1.434137in}{2.210396in}}%
\pgfpathlineto{\pgfqpoint{1.434952in}{2.292678in}}%
\pgfpathlineto{\pgfqpoint{1.435156in}{2.300906in}}%
\pgfpathlineto{\pgfqpoint{1.435360in}{2.358504in}}%
\pgfpathlineto{\pgfqpoint{1.435971in}{2.251537in}}%
\pgfpathlineto{\pgfqpoint{1.436175in}{2.309134in}}%
\pgfpathlineto{\pgfqpoint{1.436379in}{2.300906in}}%
\pgfpathlineto{\pgfqpoint{1.436583in}{2.309134in}}%
\pgfpathlineto{\pgfqpoint{1.436991in}{2.342047in}}%
\pgfpathlineto{\pgfqpoint{1.437602in}{2.309134in}}%
\pgfpathlineto{\pgfqpoint{1.437806in}{2.259765in}}%
\pgfpathlineto{\pgfqpoint{1.438213in}{2.374960in}}%
\pgfpathlineto{\pgfqpoint{1.438417in}{2.383188in}}%
\pgfpathlineto{\pgfqpoint{1.438621in}{2.276221in}}%
\pgfpathlineto{\pgfqpoint{1.439640in}{2.300906in}}%
\pgfpathlineto{\pgfqpoint{1.440863in}{2.374960in}}%
\pgfpathlineto{\pgfqpoint{1.442289in}{2.276221in}}%
\pgfpathlineto{\pgfqpoint{1.442901in}{2.243309in}}%
\pgfpathlineto{\pgfqpoint{1.443308in}{2.292678in}}%
\pgfpathlineto{\pgfqpoint{1.443716in}{2.235080in}}%
\pgfpathlineto{\pgfqpoint{1.444531in}{2.251537in}}%
\pgfpathlineto{\pgfqpoint{1.444735in}{2.243309in}}%
\pgfpathlineto{\pgfqpoint{1.445550in}{2.292678in}}%
\pgfpathlineto{\pgfqpoint{1.445754in}{2.259765in}}%
\pgfpathlineto{\pgfqpoint{1.445958in}{2.243309in}}%
\pgfpathlineto{\pgfqpoint{1.446162in}{2.276221in}}%
\pgfpathlineto{\pgfqpoint{1.446365in}{2.267993in}}%
\pgfpathlineto{\pgfqpoint{1.446569in}{2.300906in}}%
\pgfpathlineto{\pgfqpoint{1.446977in}{2.243309in}}%
\pgfpathlineto{\pgfqpoint{1.447384in}{2.267993in}}%
\pgfpathlineto{\pgfqpoint{1.447996in}{2.210396in}}%
\pgfpathlineto{\pgfqpoint{1.448607in}{2.243309in}}%
\pgfpathlineto{\pgfqpoint{1.449219in}{2.095201in}}%
\pgfpathlineto{\pgfqpoint{1.450034in}{2.185711in}}%
\pgfpathlineto{\pgfqpoint{1.450238in}{2.235080in}}%
\pgfpathlineto{\pgfqpoint{1.450442in}{2.177483in}}%
\pgfpathlineto{\pgfqpoint{1.451053in}{2.185711in}}%
\pgfpathlineto{\pgfqpoint{1.452276in}{2.292678in}}%
\pgfpathlineto{\pgfqpoint{1.452683in}{2.259765in}}%
\pgfpathlineto{\pgfqpoint{1.453091in}{2.267993in}}%
\pgfpathlineto{\pgfqpoint{1.453702in}{2.136342in}}%
\pgfpathlineto{\pgfqpoint{1.454314in}{2.333819in}}%
\pgfpathlineto{\pgfqpoint{1.454925in}{2.259765in}}%
\pgfpathlineto{\pgfqpoint{1.455129in}{2.259765in}}%
\pgfpathlineto{\pgfqpoint{1.455333in}{2.177483in}}%
\pgfpathlineto{\pgfqpoint{1.455537in}{2.300906in}}%
\pgfpathlineto{\pgfqpoint{1.456148in}{2.259765in}}%
\pgfpathlineto{\pgfqpoint{1.457167in}{2.350275in}}%
\pgfpathlineto{\pgfqpoint{1.457778in}{2.226852in}}%
\pgfpathlineto{\pgfqpoint{1.458390in}{2.284450in}}%
\pgfpathlineto{\pgfqpoint{1.458797in}{2.218624in}}%
\pgfpathlineto{\pgfqpoint{1.459205in}{2.292678in}}%
\pgfpathlineto{\pgfqpoint{1.459613in}{2.235080in}}%
\pgfpathlineto{\pgfqpoint{1.460224in}{2.177483in}}%
\pgfpathlineto{\pgfqpoint{1.460020in}{2.267993in}}%
\pgfpathlineto{\pgfqpoint{1.460632in}{2.185711in}}%
\pgfpathlineto{\pgfqpoint{1.462058in}{2.342047in}}%
\pgfpathlineto{\pgfqpoint{1.462262in}{2.342047in}}%
\pgfpathlineto{\pgfqpoint{1.463485in}{2.218624in}}%
\pgfpathlineto{\pgfqpoint{1.463689in}{2.235080in}}%
\pgfpathlineto{\pgfqpoint{1.463893in}{2.259765in}}%
\pgfpathlineto{\pgfqpoint{1.464504in}{2.202168in}}%
\pgfpathlineto{\pgfqpoint{1.464708in}{2.243309in}}%
\pgfpathlineto{\pgfqpoint{1.464912in}{1.980006in}}%
\pgfpathlineto{\pgfqpoint{1.465523in}{2.292678in}}%
\pgfpathlineto{\pgfqpoint{1.465727in}{2.202168in}}%
\pgfpathlineto{\pgfqpoint{1.465931in}{2.259765in}}%
\pgfpathlineto{\pgfqpoint{1.466746in}{2.185711in}}%
\pgfpathlineto{\pgfqpoint{1.467357in}{2.284450in}}%
\pgfpathlineto{\pgfqpoint{1.467969in}{2.235080in}}%
\pgfpathlineto{\pgfqpoint{1.468172in}{2.202168in}}%
\pgfpathlineto{\pgfqpoint{1.468376in}{2.243309in}}%
\pgfpathlineto{\pgfqpoint{1.468784in}{2.243309in}}%
\pgfpathlineto{\pgfqpoint{1.468988in}{2.317363in}}%
\pgfpathlineto{\pgfqpoint{1.469803in}{2.243309in}}%
\pgfpathlineto{\pgfqpoint{1.470414in}{2.193939in}}%
\pgfpathlineto{\pgfqpoint{1.470618in}{2.251537in}}%
\pgfpathlineto{\pgfqpoint{1.471026in}{2.218624in}}%
\pgfpathlineto{\pgfqpoint{1.471229in}{2.210396in}}%
\pgfpathlineto{\pgfqpoint{1.471433in}{2.235080in}}%
\pgfpathlineto{\pgfqpoint{1.471637in}{2.226852in}}%
\pgfpathlineto{\pgfqpoint{1.472045in}{2.309134in}}%
\pgfpathlineto{\pgfqpoint{1.472452in}{2.292678in}}%
\pgfpathlineto{\pgfqpoint{1.473675in}{2.136342in}}%
\pgfpathlineto{\pgfqpoint{1.474898in}{2.350275in}}%
\pgfpathlineto{\pgfqpoint{1.475102in}{2.309134in}}%
\pgfpathlineto{\pgfqpoint{1.475305in}{2.325591in}}%
\pgfpathlineto{\pgfqpoint{1.475713in}{2.111657in}}%
\pgfpathlineto{\pgfqpoint{1.476324in}{2.276221in}}%
\pgfpathlineto{\pgfqpoint{1.477547in}{2.218624in}}%
\pgfpathlineto{\pgfqpoint{1.477955in}{2.259765in}}%
\pgfpathlineto{\pgfqpoint{1.478566in}{2.210396in}}%
\pgfpathlineto{\pgfqpoint{1.478770in}{2.210396in}}%
\pgfpathlineto{\pgfqpoint{1.479178in}{2.317363in}}%
\pgfpathlineto{\pgfqpoint{1.479993in}{2.251537in}}%
\pgfpathlineto{\pgfqpoint{1.480808in}{2.292678in}}%
\pgfpathlineto{\pgfqpoint{1.481012in}{2.243309in}}%
\pgfpathlineto{\pgfqpoint{1.481420in}{2.300906in}}%
\pgfpathlineto{\pgfqpoint{1.481827in}{2.251537in}}%
\pgfpathlineto{\pgfqpoint{1.482031in}{2.267993in}}%
\pgfpathlineto{\pgfqpoint{1.482439in}{2.235080in}}%
\pgfpathlineto{\pgfqpoint{1.482642in}{2.226852in}}%
\pgfpathlineto{\pgfqpoint{1.483458in}{2.292678in}}%
\pgfpathlineto{\pgfqpoint{1.483661in}{2.103429in}}%
\pgfpathlineto{\pgfqpoint{1.484069in}{2.350275in}}%
\pgfpathlineto{\pgfqpoint{1.484477in}{2.333819in}}%
\pgfpathlineto{\pgfqpoint{1.484680in}{2.391416in}}%
\pgfpathlineto{\pgfqpoint{1.485699in}{2.374960in}}%
\pgfpathlineto{\pgfqpoint{1.487737in}{2.226852in}}%
\pgfpathlineto{\pgfqpoint{1.488349in}{2.235080in}}%
\pgfpathlineto{\pgfqpoint{1.488960in}{2.267993in}}%
\pgfpathlineto{\pgfqpoint{1.489368in}{1.996463in}}%
\pgfpathlineto{\pgfqpoint{1.489979in}{2.243309in}}%
\pgfpathlineto{\pgfqpoint{1.491406in}{2.333819in}}%
\pgfpathlineto{\pgfqpoint{1.491610in}{2.350275in}}%
\pgfpathlineto{\pgfqpoint{1.491814in}{2.309134in}}%
\pgfpathlineto{\pgfqpoint{1.492221in}{2.333819in}}%
\pgfpathlineto{\pgfqpoint{1.494667in}{1.807214in}}%
\pgfpathlineto{\pgfqpoint{1.495074in}{1.873040in}}%
\pgfpathlineto{\pgfqpoint{1.495482in}{1.914181in}}%
\pgfpathlineto{\pgfqpoint{1.495686in}{1.823670in}}%
\pgfpathlineto{\pgfqpoint{1.496093in}{1.683791in}}%
\pgfpathlineto{\pgfqpoint{1.496909in}{1.988234in}}%
\pgfpathlineto{\pgfqpoint{1.497316in}{1.823670in}}%
\pgfpathlineto{\pgfqpoint{1.497520in}{1.897724in}}%
\pgfpathlineto{\pgfqpoint{1.498131in}{1.807214in}}%
\pgfpathlineto{\pgfqpoint{1.498335in}{1.815442in}}%
\pgfpathlineto{\pgfqpoint{1.498539in}{1.831899in}}%
\pgfpathlineto{\pgfqpoint{1.498947in}{1.798986in}}%
\pgfpathlineto{\pgfqpoint{1.499150in}{1.798986in}}%
\pgfpathlineto{\pgfqpoint{1.499762in}{1.675563in}}%
\pgfpathlineto{\pgfqpoint{1.500169in}{1.782529in}}%
\pgfpathlineto{\pgfqpoint{1.500373in}{1.897724in}}%
\pgfpathlineto{\pgfqpoint{1.501392in}{1.864811in}}%
\pgfpathlineto{\pgfqpoint{1.502411in}{1.873040in}}%
\pgfpathlineto{\pgfqpoint{1.502819in}{1.749617in}}%
\pgfpathlineto{\pgfqpoint{1.504449in}{1.914181in}}%
\pgfpathlineto{\pgfqpoint{1.505061in}{1.864811in}}%
\pgfpathlineto{\pgfqpoint{1.504857in}{1.930637in}}%
\pgfpathlineto{\pgfqpoint{1.505265in}{1.922409in}}%
\pgfpathlineto{\pgfqpoint{1.505876in}{1.905952in}}%
\pgfpathlineto{\pgfqpoint{1.506487in}{1.971778in}}%
\pgfpathlineto{\pgfqpoint{1.507914in}{1.856583in}}%
\pgfpathlineto{\pgfqpoint{1.508118in}{1.922409in}}%
\pgfpathlineto{\pgfqpoint{1.508729in}{1.831899in}}%
\pgfpathlineto{\pgfqpoint{1.508933in}{1.840127in}}%
\pgfpathlineto{\pgfqpoint{1.509544in}{1.947093in}}%
\pgfpathlineto{\pgfqpoint{1.509952in}{1.930637in}}%
\pgfpathlineto{\pgfqpoint{1.511175in}{1.733160in}}%
\pgfpathlineto{\pgfqpoint{1.511990in}{1.864811in}}%
\pgfpathlineto{\pgfqpoint{1.512398in}{1.823670in}}%
\pgfpathlineto{\pgfqpoint{1.513213in}{1.938865in}}%
\pgfpathlineto{\pgfqpoint{1.513620in}{1.897724in}}%
\pgfpathlineto{\pgfqpoint{1.514232in}{1.749617in}}%
\pgfpathlineto{\pgfqpoint{1.514436in}{1.881268in}}%
\pgfpathlineto{\pgfqpoint{1.514639in}{1.922409in}}%
\pgfpathlineto{\pgfqpoint{1.514843in}{1.675563in}}%
\pgfpathlineto{\pgfqpoint{1.515658in}{1.938865in}}%
\pgfpathlineto{\pgfqpoint{1.516270in}{1.823670in}}%
\pgfpathlineto{\pgfqpoint{1.517289in}{1.856583in}}%
\pgfpathlineto{\pgfqpoint{1.518104in}{1.955322in}}%
\pgfpathlineto{\pgfqpoint{1.518512in}{1.889496in}}%
\pgfpathlineto{\pgfqpoint{1.519327in}{1.790758in}}%
\pgfpathlineto{\pgfqpoint{1.520142in}{1.840127in}}%
\pgfpathlineto{\pgfqpoint{1.520550in}{1.963550in}}%
\pgfpathlineto{\pgfqpoint{1.521569in}{1.930637in}}%
\pgfpathlineto{\pgfqpoint{1.521976in}{1.848355in}}%
\pgfpathlineto{\pgfqpoint{1.522384in}{1.914181in}}%
\pgfpathlineto{\pgfqpoint{1.523403in}{2.045832in}}%
\pgfpathlineto{\pgfqpoint{1.523811in}{1.963550in}}%
\pgfpathlineto{\pgfqpoint{1.524218in}{1.980006in}}%
\pgfpathlineto{\pgfqpoint{1.525033in}{1.889496in}}%
\pgfpathlineto{\pgfqpoint{1.525645in}{1.881268in}}%
\pgfpathlineto{\pgfqpoint{1.526256in}{1.980006in}}%
\pgfpathlineto{\pgfqpoint{1.527479in}{1.782529in}}%
\pgfpathlineto{\pgfqpoint{1.528090in}{2.070516in}}%
\pgfpathlineto{\pgfqpoint{1.528906in}{2.029375in}}%
\pgfpathlineto{\pgfqpoint{1.529721in}{2.037604in}}%
\pgfpathlineto{\pgfqpoint{1.530128in}{1.947093in}}%
\pgfpathlineto{\pgfqpoint{1.530944in}{2.062288in}}%
\pgfpathlineto{\pgfqpoint{1.531351in}{2.045832in}}%
\pgfpathlineto{\pgfqpoint{1.531555in}{2.045832in}}%
\pgfpathlineto{\pgfqpoint{1.531963in}{2.029375in}}%
\pgfpathlineto{\pgfqpoint{1.532370in}{1.947093in}}%
\pgfpathlineto{\pgfqpoint{1.532982in}{2.012919in}}%
\pgfpathlineto{\pgfqpoint{1.533186in}{2.021147in}}%
\pgfpathlineto{\pgfqpoint{1.533797in}{1.930637in}}%
\pgfpathlineto{\pgfqpoint{1.534001in}{2.029375in}}%
\pgfpathlineto{\pgfqpoint{1.534205in}{2.012919in}}%
\pgfpathlineto{\pgfqpoint{1.536243in}{1.831899in}}%
\pgfpathlineto{\pgfqpoint{1.536446in}{1.897724in}}%
\pgfpathlineto{\pgfqpoint{1.537058in}{1.790758in}}%
\pgfpathlineto{\pgfqpoint{1.537262in}{1.815442in}}%
\pgfpathlineto{\pgfqpoint{1.538688in}{1.667335in}}%
\pgfpathlineto{\pgfqpoint{1.538892in}{1.667335in}}%
\pgfpathlineto{\pgfqpoint{1.539911in}{1.733160in}}%
\pgfpathlineto{\pgfqpoint{1.540115in}{1.700247in}}%
\pgfpathlineto{\pgfqpoint{1.540930in}{1.650878in}}%
\pgfpathlineto{\pgfqpoint{1.541134in}{1.700247in}}%
\pgfpathlineto{\pgfqpoint{1.541338in}{1.733160in}}%
\pgfpathlineto{\pgfqpoint{1.541745in}{1.634422in}}%
\pgfpathlineto{\pgfqpoint{1.542357in}{1.593281in}}%
\pgfpathlineto{\pgfqpoint{1.542560in}{1.733160in}}%
\pgfpathlineto{\pgfqpoint{1.543579in}{1.700247in}}%
\pgfpathlineto{\pgfqpoint{1.543783in}{1.700247in}}%
\pgfpathlineto{\pgfqpoint{1.543987in}{1.683791in}}%
\pgfpathlineto{\pgfqpoint{1.544395in}{1.733160in}}%
\pgfpathlineto{\pgfqpoint{1.544599in}{1.840127in}}%
\pgfpathlineto{\pgfqpoint{1.545414in}{1.716704in}}%
\pgfpathlineto{\pgfqpoint{1.545821in}{1.790758in}}%
\pgfpathlineto{\pgfqpoint{1.546637in}{1.749617in}}%
\pgfpathlineto{\pgfqpoint{1.547248in}{1.700247in}}%
\pgfpathlineto{\pgfqpoint{1.548063in}{1.683791in}}%
\pgfpathlineto{\pgfqpoint{1.548267in}{1.807214in}}%
\pgfpathlineto{\pgfqpoint{1.548878in}{1.667335in}}%
\pgfpathlineto{\pgfqpoint{1.549490in}{1.724932in}}%
\pgfpathlineto{\pgfqpoint{1.550509in}{1.774301in}}%
\pgfpathlineto{\pgfqpoint{1.550916in}{1.766073in}}%
\pgfpathlineto{\pgfqpoint{1.552139in}{1.724932in}}%
\pgfpathlineto{\pgfqpoint{1.552343in}{1.782529in}}%
\pgfpathlineto{\pgfqpoint{1.553362in}{1.774301in}}%
\pgfpathlineto{\pgfqpoint{1.553770in}{1.774301in}}%
\pgfpathlineto{\pgfqpoint{1.554177in}{1.700247in}}%
\pgfpathlineto{\pgfqpoint{1.554381in}{1.774301in}}%
\pgfpathlineto{\pgfqpoint{1.554585in}{1.831899in}}%
\pgfpathlineto{\pgfqpoint{1.554992in}{1.708476in}}%
\pgfpathlineto{\pgfqpoint{1.555400in}{1.774301in}}%
\pgfpathlineto{\pgfqpoint{1.556011in}{1.708476in}}%
\pgfpathlineto{\pgfqpoint{1.556419in}{1.716704in}}%
\pgfpathlineto{\pgfqpoint{1.556623in}{1.749617in}}%
\pgfpathlineto{\pgfqpoint{1.556827in}{1.576824in}}%
\pgfpathlineto{\pgfqpoint{1.557438in}{1.807214in}}%
\pgfpathlineto{\pgfqpoint{1.557642in}{1.798986in}}%
\pgfpathlineto{\pgfqpoint{1.557846in}{1.823670in}}%
\pgfpathlineto{\pgfqpoint{1.558253in}{1.807214in}}%
\pgfpathlineto{\pgfqpoint{1.558457in}{1.716704in}}%
\pgfpathlineto{\pgfqpoint{1.559069in}{1.856583in}}%
\pgfpathlineto{\pgfqpoint{1.559272in}{1.823670in}}%
\pgfpathlineto{\pgfqpoint{1.559476in}{1.848355in}}%
\pgfpathlineto{\pgfqpoint{1.559680in}{1.774301in}}%
\pgfpathlineto{\pgfqpoint{1.560291in}{1.823670in}}%
\pgfpathlineto{\pgfqpoint{1.560495in}{1.659106in}}%
\pgfpathlineto{\pgfqpoint{1.561310in}{1.716704in}}%
\pgfpathlineto{\pgfqpoint{1.561718in}{1.749617in}}%
\pgfpathlineto{\pgfqpoint{1.562126in}{1.692019in}}%
\pgfpathlineto{\pgfqpoint{1.562329in}{1.683791in}}%
\pgfpathlineto{\pgfqpoint{1.562533in}{1.708476in}}%
\pgfpathlineto{\pgfqpoint{1.562737in}{1.766073in}}%
\pgfpathlineto{\pgfqpoint{1.563552in}{1.692019in}}%
\pgfpathlineto{\pgfqpoint{1.563960in}{1.700247in}}%
\pgfpathlineto{\pgfqpoint{1.564775in}{1.634422in}}%
\pgfpathlineto{\pgfqpoint{1.565183in}{1.659106in}}%
\pgfpathlineto{\pgfqpoint{1.565386in}{1.733160in}}%
\pgfpathlineto{\pgfqpoint{1.565998in}{1.650878in}}%
\pgfpathlineto{\pgfqpoint{1.566202in}{1.692019in}}%
\pgfpathlineto{\pgfqpoint{1.567424in}{1.617965in}}%
\pgfpathlineto{\pgfqpoint{1.567628in}{1.617965in}}%
\pgfpathlineto{\pgfqpoint{1.567832in}{1.626194in}}%
\pgfpathlineto{\pgfqpoint{1.568443in}{1.535683in}}%
\pgfpathlineto{\pgfqpoint{1.569259in}{1.543912in}}%
\pgfpathlineto{\pgfqpoint{1.570889in}{1.650878in}}%
\pgfpathlineto{\pgfqpoint{1.571297in}{1.642650in}}%
\pgfpathlineto{\pgfqpoint{1.571501in}{1.617965in}}%
\pgfpathlineto{\pgfqpoint{1.571704in}{1.724932in}}%
\pgfpathlineto{\pgfqpoint{1.571908in}{1.675563in}}%
\pgfpathlineto{\pgfqpoint{1.572723in}{1.634422in}}%
\pgfpathlineto{\pgfqpoint{1.573335in}{1.782529in}}%
\pgfpathlineto{\pgfqpoint{1.574965in}{1.576824in}}%
\pgfpathlineto{\pgfqpoint{1.575577in}{1.650878in}}%
\pgfpathlineto{\pgfqpoint{1.575984in}{1.609737in}}%
\pgfpathlineto{\pgfqpoint{1.576188in}{2.136342in}}%
\pgfpathlineto{\pgfqpoint{1.577003in}{1.609737in}}%
\pgfpathlineto{\pgfqpoint{1.577207in}{1.527455in}}%
\pgfpathlineto{\pgfqpoint{1.577615in}{1.675563in}}%
\pgfpathlineto{\pgfqpoint{1.577818in}{1.650878in}}%
\pgfpathlineto{\pgfqpoint{1.578022in}{1.659106in}}%
\pgfpathlineto{\pgfqpoint{1.578226in}{1.626194in}}%
\pgfpathlineto{\pgfqpoint{1.578430in}{1.601509in}}%
\pgfpathlineto{\pgfqpoint{1.578634in}{1.692019in}}%
\pgfpathlineto{\pgfqpoint{1.578837in}{1.667335in}}%
\pgfpathlineto{\pgfqpoint{1.579041in}{1.683791in}}%
\pgfpathlineto{\pgfqpoint{1.579449in}{1.445173in}}%
\pgfpathlineto{\pgfqpoint{1.580060in}{1.650878in}}%
\pgfpathlineto{\pgfqpoint{1.580264in}{1.642650in}}%
\pgfpathlineto{\pgfqpoint{1.580468in}{1.659106in}}%
\pgfpathlineto{\pgfqpoint{1.580875in}{1.749617in}}%
\pgfpathlineto{\pgfqpoint{1.581283in}{1.552140in}}%
\pgfpathlineto{\pgfqpoint{1.581487in}{1.667335in}}%
\pgfpathlineto{\pgfqpoint{1.581691in}{1.675563in}}%
\pgfpathlineto{\pgfqpoint{1.581894in}{1.667335in}}%
\pgfpathlineto{\pgfqpoint{1.582098in}{1.766073in}}%
\pgfpathlineto{\pgfqpoint{1.582913in}{1.667335in}}%
\pgfpathlineto{\pgfqpoint{1.583321in}{1.617965in}}%
\pgfpathlineto{\pgfqpoint{1.583932in}{1.667335in}}%
\pgfpathlineto{\pgfqpoint{1.584544in}{1.774301in}}%
\pgfpathlineto{\pgfqpoint{1.584748in}{1.741388in}}%
\pgfpathlineto{\pgfqpoint{1.585359in}{1.428717in}}%
\pgfpathlineto{\pgfqpoint{1.585971in}{1.659106in}}%
\pgfpathlineto{\pgfqpoint{1.586174in}{1.642650in}}%
\pgfpathlineto{\pgfqpoint{1.586378in}{1.708476in}}%
\pgfpathlineto{\pgfqpoint{1.586990in}{1.659106in}}%
\pgfpathlineto{\pgfqpoint{1.587193in}{1.683791in}}%
\pgfpathlineto{\pgfqpoint{1.588009in}{1.749617in}}%
\pgfpathlineto{\pgfqpoint{1.588212in}{1.683791in}}%
\pgfpathlineto{\pgfqpoint{1.588416in}{1.626194in}}%
\pgfpathlineto{\pgfqpoint{1.589028in}{1.708476in}}%
\pgfpathlineto{\pgfqpoint{1.589231in}{1.733160in}}%
\pgfpathlineto{\pgfqpoint{1.589639in}{1.535683in}}%
\pgfpathlineto{\pgfqpoint{1.590047in}{1.774301in}}%
\pgfpathlineto{\pgfqpoint{1.590250in}{1.790758in}}%
\pgfpathlineto{\pgfqpoint{1.590862in}{1.757845in}}%
\pgfpathlineto{\pgfqpoint{1.591881in}{1.642650in}}%
\pgfpathlineto{\pgfqpoint{1.592288in}{1.650878in}}%
\pgfpathlineto{\pgfqpoint{1.592492in}{1.683791in}}%
\pgfpathlineto{\pgfqpoint{1.592900in}{1.724932in}}%
\pgfpathlineto{\pgfqpoint{1.593511in}{1.445173in}}%
\pgfpathlineto{\pgfqpoint{1.594530in}{1.708476in}}%
\pgfpathlineto{\pgfqpoint{1.594734in}{1.692019in}}%
\pgfpathlineto{\pgfqpoint{1.594938in}{1.634422in}}%
\pgfpathlineto{\pgfqpoint{1.595957in}{1.642650in}}%
\pgfpathlineto{\pgfqpoint{1.596161in}{1.642650in}}%
\pgfpathlineto{\pgfqpoint{1.596568in}{1.617965in}}%
\pgfpathlineto{\pgfqpoint{1.596772in}{1.659106in}}%
\pgfpathlineto{\pgfqpoint{1.596976in}{1.634422in}}%
\pgfpathlineto{\pgfqpoint{1.597995in}{1.675563in}}%
\pgfpathlineto{\pgfqpoint{1.598403in}{1.708476in}}%
\pgfpathlineto{\pgfqpoint{1.599218in}{1.560368in}}%
\pgfpathlineto{\pgfqpoint{1.599625in}{1.626194in}}%
\pgfpathlineto{\pgfqpoint{1.599829in}{1.362891in}}%
\pgfpathlineto{\pgfqpoint{1.600441in}{1.659106in}}%
\pgfpathlineto{\pgfqpoint{1.600644in}{1.634422in}}%
\pgfpathlineto{\pgfqpoint{1.601052in}{1.609737in}}%
\pgfpathlineto{\pgfqpoint{1.601867in}{1.716704in}}%
\pgfpathlineto{\pgfqpoint{1.603090in}{1.338207in}}%
\pgfpathlineto{\pgfqpoint{1.603294in}{1.346435in}}%
\pgfpathlineto{\pgfqpoint{1.603498in}{1.642650in}}%
\pgfpathlineto{\pgfqpoint{1.604517in}{1.601509in}}%
\pgfpathlineto{\pgfqpoint{1.606147in}{1.790758in}}%
\pgfpathlineto{\pgfqpoint{1.606351in}{1.782529in}}%
\pgfpathlineto{\pgfqpoint{1.606758in}{1.782529in}}%
\pgfpathlineto{\pgfqpoint{1.607166in}{1.881268in}}%
\pgfpathlineto{\pgfqpoint{1.607574in}{1.807214in}}%
\pgfpathlineto{\pgfqpoint{1.607777in}{1.708476in}}%
\pgfpathlineto{\pgfqpoint{1.608796in}{1.716704in}}%
\pgfpathlineto{\pgfqpoint{1.609612in}{1.790758in}}%
\pgfpathlineto{\pgfqpoint{1.610019in}{1.766073in}}%
\pgfpathlineto{\pgfqpoint{1.610427in}{1.774301in}}%
\pgfpathlineto{\pgfqpoint{1.611038in}{1.593281in}}%
\pgfpathlineto{\pgfqpoint{1.611242in}{1.823670in}}%
\pgfpathlineto{\pgfqpoint{1.612057in}{1.815442in}}%
\pgfpathlineto{\pgfqpoint{1.613280in}{1.395804in}}%
\pgfpathlineto{\pgfqpoint{1.613484in}{1.428717in}}%
\pgfpathlineto{\pgfqpoint{1.614095in}{1.634422in}}%
\pgfpathlineto{\pgfqpoint{1.614707in}{1.576824in}}%
\pgfpathlineto{\pgfqpoint{1.614911in}{1.585053in}}%
\pgfpathlineto{\pgfqpoint{1.615726in}{1.593281in}}%
\pgfpathlineto{\pgfqpoint{1.615930in}{1.519227in}}%
\pgfpathlineto{\pgfqpoint{1.616541in}{1.609737in}}%
\pgfpathlineto{\pgfqpoint{1.616949in}{1.519227in}}%
\pgfpathlineto{\pgfqpoint{1.618783in}{1.626194in}}%
\pgfpathlineto{\pgfqpoint{1.618987in}{1.626194in}}%
\pgfpathlineto{\pgfqpoint{1.620006in}{1.543912in}}%
\pgfpathlineto{\pgfqpoint{1.620413in}{1.560368in}}%
\pgfpathlineto{\pgfqpoint{1.621228in}{1.617965in}}%
\pgfpathlineto{\pgfqpoint{1.621432in}{1.609737in}}%
\pgfpathlineto{\pgfqpoint{1.622044in}{1.552140in}}%
\pgfpathlineto{\pgfqpoint{1.621840in}{1.617965in}}%
\pgfpathlineto{\pgfqpoint{1.622247in}{1.585053in}}%
\pgfpathlineto{\pgfqpoint{1.622451in}{1.650878in}}%
\pgfpathlineto{\pgfqpoint{1.623063in}{1.494542in}}%
\pgfpathlineto{\pgfqpoint{1.623266in}{1.494542in}}%
\pgfpathlineto{\pgfqpoint{1.623674in}{1.486314in}}%
\pgfpathlineto{\pgfqpoint{1.624489in}{1.535683in}}%
\pgfpathlineto{\pgfqpoint{1.625508in}{1.445173in}}%
\pgfpathlineto{\pgfqpoint{1.625712in}{1.478086in}}%
\pgfpathlineto{\pgfqpoint{1.625916in}{1.527455in}}%
\pgfpathlineto{\pgfqpoint{1.626120in}{1.453401in}}%
\pgfpathlineto{\pgfqpoint{1.626527in}{1.469858in}}%
\pgfpathlineto{\pgfqpoint{1.627546in}{1.428717in}}%
\pgfpathlineto{\pgfqpoint{1.627750in}{1.453401in}}%
\pgfpathlineto{\pgfqpoint{1.628158in}{1.543912in}}%
\pgfpathlineto{\pgfqpoint{1.628769in}{1.494542in}}%
\pgfpathlineto{\pgfqpoint{1.629788in}{1.272381in}}%
\pgfpathlineto{\pgfqpoint{1.630196in}{1.313522in}}%
\pgfpathlineto{\pgfqpoint{1.630807in}{1.519227in}}%
\pgfpathlineto{\pgfqpoint{1.631419in}{1.469858in}}%
\pgfpathlineto{\pgfqpoint{1.632438in}{1.436945in}}%
\pgfpathlineto{\pgfqpoint{1.633660in}{1.527455in}}%
\pgfpathlineto{\pgfqpoint{1.633049in}{1.412260in}}%
\pgfpathlineto{\pgfqpoint{1.633864in}{1.502771in}}%
\pgfpathlineto{\pgfqpoint{1.634476in}{1.469858in}}%
\pgfpathlineto{\pgfqpoint{1.634679in}{1.486314in}}%
\pgfpathlineto{\pgfqpoint{1.635087in}{1.478086in}}%
\pgfpathlineto{\pgfqpoint{1.636106in}{1.585053in}}%
\pgfpathlineto{\pgfqpoint{1.636717in}{1.469858in}}%
\pgfpathlineto{\pgfqpoint{1.637329in}{1.502771in}}%
\pgfpathlineto{\pgfqpoint{1.637533in}{1.453401in}}%
\pgfpathlineto{\pgfqpoint{1.637736in}{1.552140in}}%
\pgfpathlineto{\pgfqpoint{1.638348in}{1.494542in}}%
\pgfpathlineto{\pgfqpoint{1.638552in}{1.519227in}}%
\pgfpathlineto{\pgfqpoint{1.638756in}{1.461630in}}%
\pgfpathlineto{\pgfqpoint{1.638959in}{1.387576in}}%
\pgfpathlineto{\pgfqpoint{1.639775in}{1.428717in}}%
\pgfpathlineto{\pgfqpoint{1.639978in}{1.445173in}}%
\pgfpathlineto{\pgfqpoint{1.640182in}{1.436945in}}%
\pgfpathlineto{\pgfqpoint{1.640794in}{1.198327in}}%
\pgfpathlineto{\pgfqpoint{1.640997in}{1.305294in}}%
\pgfpathlineto{\pgfqpoint{1.641609in}{1.510999in}}%
\pgfpathlineto{\pgfqpoint{1.642220in}{1.436945in}}%
\pgfpathlineto{\pgfqpoint{1.642424in}{1.478086in}}%
\pgfpathlineto{\pgfqpoint{1.642628in}{1.395804in}}%
\pgfpathlineto{\pgfqpoint{1.643035in}{1.412260in}}%
\pgfpathlineto{\pgfqpoint{1.643239in}{1.395804in}}%
\pgfpathlineto{\pgfqpoint{1.643647in}{1.445173in}}%
\pgfpathlineto{\pgfqpoint{1.643851in}{1.461630in}}%
\pgfpathlineto{\pgfqpoint{1.644258in}{1.428717in}}%
\pgfpathlineto{\pgfqpoint{1.645073in}{1.321750in}}%
\pgfpathlineto{\pgfqpoint{1.645481in}{1.362891in}}%
\pgfpathlineto{\pgfqpoint{1.645889in}{1.395804in}}%
\pgfpathlineto{\pgfqpoint{1.646092in}{1.239468in}}%
\pgfpathlineto{\pgfqpoint{1.646908in}{1.461630in}}%
\pgfpathlineto{\pgfqpoint{1.647315in}{1.527455in}}%
\pgfpathlineto{\pgfqpoint{1.647723in}{1.453401in}}%
\pgfpathlineto{\pgfqpoint{1.647927in}{1.453401in}}%
\pgfpathlineto{\pgfqpoint{1.648130in}{1.346435in}}%
\pgfpathlineto{\pgfqpoint{1.648946in}{1.404032in}}%
\pgfpathlineto{\pgfqpoint{1.649353in}{1.510999in}}%
\pgfpathlineto{\pgfqpoint{1.650168in}{1.478086in}}%
\pgfpathlineto{\pgfqpoint{1.651391in}{1.412260in}}%
\pgfpathlineto{\pgfqpoint{1.651595in}{1.469858in}}%
\pgfpathlineto{\pgfqpoint{1.652410in}{1.453401in}}%
\pgfpathlineto{\pgfqpoint{1.653226in}{1.420489in}}%
\pgfpathlineto{\pgfqpoint{1.652818in}{1.494542in}}%
\pgfpathlineto{\pgfqpoint{1.653429in}{1.453401in}}%
\pgfpathlineto{\pgfqpoint{1.653633in}{1.453401in}}%
\pgfpathlineto{\pgfqpoint{1.653837in}{1.543912in}}%
\pgfpathlineto{\pgfqpoint{1.654245in}{1.428717in}}%
\pgfpathlineto{\pgfqpoint{1.654448in}{1.445173in}}%
\pgfpathlineto{\pgfqpoint{1.654856in}{1.181871in}}%
\pgfpathlineto{\pgfqpoint{1.655671in}{1.387576in}}%
\pgfpathlineto{\pgfqpoint{1.655875in}{1.478086in}}%
\pgfpathlineto{\pgfqpoint{1.656486in}{1.379348in}}%
\pgfpathlineto{\pgfqpoint{1.656690in}{1.404032in}}%
\pgfpathlineto{\pgfqpoint{1.657505in}{1.297066in}}%
\pgfpathlineto{\pgfqpoint{1.657709in}{1.371119in}}%
\pgfpathlineto{\pgfqpoint{1.658321in}{1.469858in}}%
\pgfpathlineto{\pgfqpoint{1.658728in}{1.371119in}}%
\pgfpathlineto{\pgfqpoint{1.659340in}{1.206555in}}%
\pgfpathlineto{\pgfqpoint{1.659543in}{1.058448in}}%
\pgfpathlineto{\pgfqpoint{1.659951in}{1.494542in}}%
\pgfpathlineto{\pgfqpoint{1.660155in}{1.387576in}}%
\pgfpathlineto{\pgfqpoint{1.661785in}{1.198327in}}%
\pgfpathlineto{\pgfqpoint{1.663212in}{1.395804in}}%
\pgfpathlineto{\pgfqpoint{1.664231in}{1.272381in}}%
\pgfpathlineto{\pgfqpoint{1.664027in}{1.412260in}}%
\pgfpathlineto{\pgfqpoint{1.664435in}{1.280609in}}%
\pgfpathlineto{\pgfqpoint{1.664638in}{1.338207in}}%
\pgfpathlineto{\pgfqpoint{1.665250in}{1.247696in}}%
\pgfpathlineto{\pgfqpoint{1.665454in}{1.074904in}}%
\pgfpathlineto{\pgfqpoint{1.665861in}{1.346435in}}%
\pgfpathlineto{\pgfqpoint{1.666269in}{1.264153in}}%
\pgfpathlineto{\pgfqpoint{1.666677in}{1.247696in}}%
\pgfpathlineto{\pgfqpoint{1.666880in}{1.272381in}}%
\pgfpathlineto{\pgfqpoint{1.667288in}{1.321750in}}%
\pgfpathlineto{\pgfqpoint{1.667696in}{1.239468in}}%
\pgfpathlineto{\pgfqpoint{1.668918in}{1.379348in}}%
\pgfpathlineto{\pgfqpoint{1.669122in}{1.346435in}}%
\pgfpathlineto{\pgfqpoint{1.669937in}{1.231240in}}%
\pgfpathlineto{\pgfqpoint{1.670345in}{1.239468in}}%
\pgfpathlineto{\pgfqpoint{1.671364in}{1.313522in}}%
\pgfpathlineto{\pgfqpoint{1.670956in}{1.223012in}}%
\pgfpathlineto{\pgfqpoint{1.671568in}{1.280609in}}%
\pgfpathlineto{\pgfqpoint{1.672179in}{1.387576in}}%
\pgfpathlineto{\pgfqpoint{1.671975in}{1.272381in}}%
\pgfpathlineto{\pgfqpoint{1.672383in}{1.297066in}}%
\pgfpathlineto{\pgfqpoint{1.672791in}{1.321750in}}%
\pgfpathlineto{\pgfqpoint{1.673198in}{1.132502in}}%
\pgfpathlineto{\pgfqpoint{1.674421in}{1.395804in}}%
\pgfpathlineto{\pgfqpoint{1.675032in}{1.338207in}}%
\pgfpathlineto{\pgfqpoint{1.675236in}{1.387576in}}%
\pgfpathlineto{\pgfqpoint{1.675440in}{1.404032in}}%
\pgfpathlineto{\pgfqpoint{1.675644in}{1.362891in}}%
\pgfpathlineto{\pgfqpoint{1.676051in}{1.387576in}}%
\pgfpathlineto{\pgfqpoint{1.676255in}{1.313522in}}%
\pgfpathlineto{\pgfqpoint{1.676459in}{1.404032in}}%
\pgfpathlineto{\pgfqpoint{1.677070in}{1.395804in}}%
\pgfpathlineto{\pgfqpoint{1.677274in}{1.404032in}}%
\pgfpathlineto{\pgfqpoint{1.677478in}{1.395804in}}%
\pgfpathlineto{\pgfqpoint{1.678701in}{1.116045in}}%
\pgfpathlineto{\pgfqpoint{1.678905in}{1.288837in}}%
\pgfpathlineto{\pgfqpoint{1.679109in}{1.297066in}}%
\pgfpathlineto{\pgfqpoint{1.679312in}{1.552140in}}%
\pgfpathlineto{\pgfqpoint{1.680128in}{1.395804in}}%
\pgfpathlineto{\pgfqpoint{1.680331in}{1.395804in}}%
\pgfpathlineto{\pgfqpoint{1.680535in}{1.412260in}}%
\pgfpathlineto{\pgfqpoint{1.681147in}{1.165414in}}%
\pgfpathlineto{\pgfqpoint{1.681554in}{1.445173in}}%
\pgfpathlineto{\pgfqpoint{1.682166in}{1.535683in}}%
\pgfpathlineto{\pgfqpoint{1.682573in}{1.478086in}}%
\pgfpathlineto{\pgfqpoint{1.684000in}{1.379348in}}%
\pgfpathlineto{\pgfqpoint{1.683388in}{1.502771in}}%
\pgfpathlineto{\pgfqpoint{1.684204in}{1.395804in}}%
\pgfpathlineto{\pgfqpoint{1.684407in}{1.395804in}}%
\pgfpathlineto{\pgfqpoint{1.684815in}{1.173643in}}%
\pgfpathlineto{\pgfqpoint{1.685426in}{1.404032in}}%
\pgfpathlineto{\pgfqpoint{1.685630in}{1.329978in}}%
\pgfpathlineto{\pgfqpoint{1.685834in}{1.436945in}}%
\pgfpathlineto{\pgfqpoint{1.686649in}{1.379348in}}%
\pgfpathlineto{\pgfqpoint{1.687057in}{1.280609in}}%
\pgfpathlineto{\pgfqpoint{1.687464in}{1.453401in}}%
\pgfpathlineto{\pgfqpoint{1.688280in}{1.313522in}}%
\pgfpathlineto{\pgfqpoint{1.691133in}{0.967938in}}%
\pgfpathlineto{\pgfqpoint{1.688891in}{1.338207in}}%
\pgfpathlineto{\pgfqpoint{1.691337in}{0.976166in}}%
\pgfpathlineto{\pgfqpoint{1.692356in}{0.967938in}}%
\pgfpathlineto{\pgfqpoint{1.692967in}{1.362891in}}%
\pgfpathlineto{\pgfqpoint{1.693986in}{1.058448in}}%
\pgfpathlineto{\pgfqpoint{1.694190in}{1.239468in}}%
\pgfpathlineto{\pgfqpoint{1.694598in}{1.190099in}}%
\pgfpathlineto{\pgfqpoint{1.694801in}{1.247696in}}%
\pgfpathlineto{\pgfqpoint{1.695005in}{1.206555in}}%
\pgfpathlineto{\pgfqpoint{1.695209in}{1.280609in}}%
\pgfpathlineto{\pgfqpoint{1.695617in}{1.190099in}}%
\pgfpathlineto{\pgfqpoint{1.695820in}{1.190099in}}%
\pgfpathlineto{\pgfqpoint{1.696432in}{1.214784in}}%
\pgfpathlineto{\pgfqpoint{1.696839in}{1.107817in}}%
\pgfpathlineto{\pgfqpoint{1.697043in}{1.264153in}}%
\pgfpathlineto{\pgfqpoint{1.697858in}{1.206555in}}%
\pgfpathlineto{\pgfqpoint{1.698266in}{1.165414in}}%
\pgfpathlineto{\pgfqpoint{1.698470in}{1.181871in}}%
\pgfpathlineto{\pgfqpoint{1.699081in}{1.313522in}}%
\pgfpathlineto{\pgfqpoint{1.699285in}{1.272381in}}%
\pgfpathlineto{\pgfqpoint{1.700304in}{0.967938in}}%
\pgfpathlineto{\pgfqpoint{1.700508in}{1.107817in}}%
\pgfpathlineto{\pgfqpoint{1.701323in}{1.132502in}}%
\pgfpathlineto{\pgfqpoint{1.701934in}{1.033763in}}%
\pgfpathlineto{\pgfqpoint{1.703157in}{1.214784in}}%
\pgfpathlineto{\pgfqpoint{1.703361in}{1.181871in}}%
\pgfpathlineto{\pgfqpoint{1.704991in}{0.976166in}}%
\pgfpathlineto{\pgfqpoint{1.705195in}{1.009079in}}%
\pgfpathlineto{\pgfqpoint{1.705399in}{0.836286in}}%
\pgfpathlineto{\pgfqpoint{1.705603in}{1.116045in}}%
\pgfpathlineto{\pgfqpoint{1.706214in}{1.009079in}}%
\pgfpathlineto{\pgfqpoint{1.706826in}{1.091361in}}%
\pgfpathlineto{\pgfqpoint{1.707437in}{1.066676in}}%
\pgfpathlineto{\pgfqpoint{1.707641in}{1.000850in}}%
\pgfpathlineto{\pgfqpoint{1.708252in}{1.132502in}}%
\pgfpathlineto{\pgfqpoint{1.708456in}{1.033763in}}%
\pgfpathlineto{\pgfqpoint{1.709068in}{1.231240in}}%
\pgfpathlineto{\pgfqpoint{1.709679in}{1.099589in}}%
\pgfpathlineto{\pgfqpoint{1.709883in}{1.066676in}}%
\pgfpathlineto{\pgfqpoint{1.710290in}{1.099589in}}%
\pgfpathlineto{\pgfqpoint{1.710494in}{1.157186in}}%
\pgfpathlineto{\pgfqpoint{1.710698in}{1.050220in}}%
\pgfpathlineto{\pgfqpoint{1.711106in}{1.116045in}}%
\pgfpathlineto{\pgfqpoint{1.711513in}{0.836286in}}%
\pgfpathlineto{\pgfqpoint{1.712125in}{0.959709in}}%
\pgfpathlineto{\pgfqpoint{1.712940in}{1.066676in}}%
\pgfpathlineto{\pgfqpoint{1.712736in}{0.918568in}}%
\pgfpathlineto{\pgfqpoint{1.713144in}{1.009079in}}%
\pgfpathlineto{\pgfqpoint{1.714570in}{0.902112in}}%
\pgfpathlineto{\pgfqpoint{1.715589in}{1.116045in}}%
\pgfpathlineto{\pgfqpoint{1.715793in}{1.083132in}}%
\pgfpathlineto{\pgfqpoint{1.716812in}{1.017307in}}%
\pgfpathlineto{\pgfqpoint{1.717016in}{1.025535in}}%
\pgfpathlineto{\pgfqpoint{1.718646in}{1.255925in}}%
\pgfpathlineto{\pgfqpoint{1.720073in}{1.420489in}}%
\pgfpathlineto{\pgfqpoint{1.720888in}{1.404032in}}%
\pgfpathlineto{\pgfqpoint{1.721092in}{1.387576in}}%
\pgfpathlineto{\pgfqpoint{1.721296in}{1.428717in}}%
\pgfpathlineto{\pgfqpoint{1.721703in}{1.420489in}}%
\pgfpathlineto{\pgfqpoint{1.721907in}{1.428717in}}%
\pgfpathlineto{\pgfqpoint{1.722111in}{1.387576in}}%
\pgfpathlineto{\pgfqpoint{1.722722in}{1.486314in}}%
\pgfpathlineto{\pgfqpoint{1.722926in}{1.445173in}}%
\pgfpathlineto{\pgfqpoint{1.723130in}{1.445173in}}%
\pgfpathlineto{\pgfqpoint{1.723538in}{1.436945in}}%
\pgfpathlineto{\pgfqpoint{1.723945in}{1.519227in}}%
\pgfpathlineto{\pgfqpoint{1.724760in}{1.642650in}}%
\pgfpathlineto{\pgfqpoint{1.725168in}{1.609737in}}%
\pgfpathlineto{\pgfqpoint{1.725372in}{1.486314in}}%
\pgfpathlineto{\pgfqpoint{1.726187in}{1.650878in}}%
\pgfpathlineto{\pgfqpoint{1.727002in}{1.568596in}}%
\pgfpathlineto{\pgfqpoint{1.727410in}{1.626194in}}%
\pgfpathlineto{\pgfqpoint{1.728836in}{1.741388in}}%
\pgfpathlineto{\pgfqpoint{1.728225in}{1.617965in}}%
\pgfpathlineto{\pgfqpoint{1.729244in}{1.716704in}}%
\pgfpathlineto{\pgfqpoint{1.730263in}{1.338207in}}%
\pgfpathlineto{\pgfqpoint{1.730671in}{1.527455in}}%
\pgfpathlineto{\pgfqpoint{1.730874in}{1.535683in}}%
\pgfpathlineto{\pgfqpoint{1.731486in}{1.453401in}}%
\pgfpathlineto{\pgfqpoint{1.731893in}{1.478086in}}%
\pgfpathlineto{\pgfqpoint{1.733320in}{1.692019in}}%
\pgfpathlineto{\pgfqpoint{1.733728in}{1.609737in}}%
\pgfpathlineto{\pgfqpoint{1.734135in}{1.626194in}}%
\pgfpathlineto{\pgfqpoint{1.735358in}{1.856583in}}%
\pgfpathlineto{\pgfqpoint{1.735562in}{1.831899in}}%
\pgfpathlineto{\pgfqpoint{1.736173in}{1.766073in}}%
\pgfpathlineto{\pgfqpoint{1.736785in}{1.782529in}}%
\pgfpathlineto{\pgfqpoint{1.736989in}{1.790758in}}%
\pgfpathlineto{\pgfqpoint{1.738008in}{1.947093in}}%
\pgfpathlineto{\pgfqpoint{1.738211in}{1.930637in}}%
\pgfpathlineto{\pgfqpoint{1.738415in}{1.733160in}}%
\pgfpathlineto{\pgfqpoint{1.738619in}{2.012919in}}%
\pgfpathlineto{\pgfqpoint{1.739027in}{2.004691in}}%
\pgfpathlineto{\pgfqpoint{1.739230in}{2.045832in}}%
\pgfpathlineto{\pgfqpoint{1.740249in}{2.037604in}}%
\pgfpathlineto{\pgfqpoint{1.740453in}{2.029375in}}%
\pgfpathlineto{\pgfqpoint{1.741065in}{1.856583in}}%
\pgfpathlineto{\pgfqpoint{1.741472in}{1.938865in}}%
\pgfpathlineto{\pgfqpoint{1.742899in}{2.045832in}}%
\pgfpathlineto{\pgfqpoint{1.743103in}{2.045832in}}%
\pgfpathlineto{\pgfqpoint{1.743714in}{2.078745in}}%
\pgfpathlineto{\pgfqpoint{1.743918in}{1.938865in}}%
\pgfpathlineto{\pgfqpoint{1.744733in}{2.111657in}}%
\pgfpathlineto{\pgfqpoint{1.744937in}{2.111657in}}%
\pgfpathlineto{\pgfqpoint{1.745141in}{2.078745in}}%
\pgfpathlineto{\pgfqpoint{1.745548in}{2.144570in}}%
\pgfpathlineto{\pgfqpoint{1.745752in}{2.218624in}}%
\pgfpathlineto{\pgfqpoint{1.746567in}{2.136342in}}%
\pgfpathlineto{\pgfqpoint{1.747383in}{2.078745in}}%
\pgfpathlineto{\pgfqpoint{1.747790in}{2.103429in}}%
\pgfpathlineto{\pgfqpoint{1.748402in}{2.169255in}}%
\pgfpathlineto{\pgfqpoint{1.748809in}{2.111657in}}%
\pgfpathlineto{\pgfqpoint{1.749828in}{2.037604in}}%
\pgfpathlineto{\pgfqpoint{1.749421in}{2.128114in}}%
\pgfpathlineto{\pgfqpoint{1.750440in}{2.086973in}}%
\pgfpathlineto{\pgfqpoint{1.751866in}{2.202168in}}%
\pgfpathlineto{\pgfqpoint{1.752070in}{2.021147in}}%
\pgfpathlineto{\pgfqpoint{1.752885in}{2.276221in}}%
\pgfpathlineto{\pgfqpoint{1.753293in}{2.177483in}}%
\pgfpathlineto{\pgfqpoint{1.753700in}{2.218624in}}%
\pgfpathlineto{\pgfqpoint{1.753904in}{2.045832in}}%
\pgfpathlineto{\pgfqpoint{1.754719in}{2.193939in}}%
\pgfpathlineto{\pgfqpoint{1.754923in}{2.185711in}}%
\pgfpathlineto{\pgfqpoint{1.755535in}{2.292678in}}%
\pgfpathlineto{\pgfqpoint{1.756146in}{2.259765in}}%
\pgfpathlineto{\pgfqpoint{1.756350in}{2.226852in}}%
\pgfpathlineto{\pgfqpoint{1.756961in}{2.284450in}}%
\pgfpathlineto{\pgfqpoint{1.757573in}{2.325591in}}%
\pgfpathlineto{\pgfqpoint{1.757776in}{2.276221in}}%
\pgfpathlineto{\pgfqpoint{1.757980in}{2.292678in}}%
\pgfpathlineto{\pgfqpoint{1.758999in}{2.128114in}}%
\pgfpathlineto{\pgfqpoint{1.759611in}{2.161027in}}%
\pgfpathlineto{\pgfqpoint{1.759815in}{2.218624in}}%
\pgfpathlineto{\pgfqpoint{1.760630in}{2.193939in}}%
\pgfpathlineto{\pgfqpoint{1.760834in}{2.136342in}}%
\pgfpathlineto{\pgfqpoint{1.761241in}{2.226852in}}%
\pgfpathlineto{\pgfqpoint{1.761445in}{2.177483in}}%
\pgfpathlineto{\pgfqpoint{1.762872in}{2.292678in}}%
\pgfpathlineto{\pgfqpoint{1.763687in}{2.226852in}}%
\pgfpathlineto{\pgfqpoint{1.764094in}{2.276221in}}%
\pgfpathlineto{\pgfqpoint{1.764706in}{2.325591in}}%
\pgfpathlineto{\pgfqpoint{1.765113in}{2.284450in}}%
\pgfpathlineto{\pgfqpoint{1.765317in}{2.251537in}}%
\pgfpathlineto{\pgfqpoint{1.765725in}{2.350275in}}%
\pgfpathlineto{\pgfqpoint{1.766132in}{2.292678in}}%
\pgfpathlineto{\pgfqpoint{1.766540in}{2.342047in}}%
\pgfpathlineto{\pgfqpoint{1.766744in}{2.226852in}}%
\pgfpathlineto{\pgfqpoint{1.767559in}{2.325591in}}%
\pgfpathlineto{\pgfqpoint{1.767763in}{2.317363in}}%
\pgfpathlineto{\pgfqpoint{1.767967in}{2.358504in}}%
\pgfpathlineto{\pgfqpoint{1.768374in}{2.309134in}}%
\pgfpathlineto{\pgfqpoint{1.768782in}{2.333819in}}%
\pgfpathlineto{\pgfqpoint{1.769393in}{2.276221in}}%
\pgfpathlineto{\pgfqpoint{1.769801in}{2.342047in}}%
\pgfpathlineto{\pgfqpoint{1.770005in}{2.350275in}}%
\pgfpathlineto{\pgfqpoint{1.771227in}{2.259765in}}%
\pgfpathlineto{\pgfqpoint{1.772246in}{2.366732in}}%
\pgfpathlineto{\pgfqpoint{1.772654in}{2.358504in}}%
\pgfpathlineto{\pgfqpoint{1.773469in}{2.276221in}}%
\pgfpathlineto{\pgfqpoint{1.773673in}{2.309134in}}%
\pgfpathlineto{\pgfqpoint{1.773877in}{2.366732in}}%
\pgfpathlineto{\pgfqpoint{1.774285in}{2.259765in}}%
\pgfpathlineto{\pgfqpoint{1.774692in}{2.309134in}}%
\pgfpathlineto{\pgfqpoint{1.774896in}{2.284450in}}%
\pgfpathlineto{\pgfqpoint{1.775507in}{2.333819in}}%
\pgfpathlineto{\pgfqpoint{1.775711in}{2.342047in}}%
\pgfpathlineto{\pgfqpoint{1.776526in}{2.226852in}}%
\pgfpathlineto{\pgfqpoint{1.776934in}{2.300906in}}%
\pgfpathlineto{\pgfqpoint{1.777138in}{2.333819in}}%
\pgfpathlineto{\pgfqpoint{1.777545in}{2.243309in}}%
\pgfpathlineto{\pgfqpoint{1.777749in}{2.292678in}}%
\pgfpathlineto{\pgfqpoint{1.777953in}{2.276221in}}%
\pgfpathlineto{\pgfqpoint{1.778361in}{2.309134in}}%
\pgfpathlineto{\pgfqpoint{1.778564in}{2.333819in}}%
\pgfpathlineto{\pgfqpoint{1.778972in}{2.267993in}}%
\pgfpathlineto{\pgfqpoint{1.779176in}{2.276221in}}%
\pgfpathlineto{\pgfqpoint{1.780195in}{2.218624in}}%
\pgfpathlineto{\pgfqpoint{1.780399in}{2.251537in}}%
\pgfpathlineto{\pgfqpoint{1.781825in}{2.424329in}}%
\pgfpathlineto{\pgfqpoint{1.782029in}{2.424329in}}%
\pgfpathlineto{\pgfqpoint{1.782233in}{2.490155in}}%
\pgfpathlineto{\pgfqpoint{1.783252in}{2.465470in}}%
\pgfpathlineto{\pgfqpoint{1.783659in}{2.514839in}}%
\pgfpathlineto{\pgfqpoint{1.784882in}{2.383188in}}%
\pgfpathlineto{\pgfqpoint{1.785697in}{2.531296in}}%
\pgfpathlineto{\pgfqpoint{1.785901in}{2.523068in}}%
\pgfpathlineto{\pgfqpoint{1.786105in}{2.284450in}}%
\pgfpathlineto{\pgfqpoint{1.786920in}{2.597121in}}%
\pgfpathlineto{\pgfqpoint{1.787124in}{2.621806in}}%
\pgfpathlineto{\pgfqpoint{1.787328in}{2.514839in}}%
\pgfpathlineto{\pgfqpoint{1.787532in}{2.531296in}}%
\pgfpathlineto{\pgfqpoint{1.788143in}{2.416101in}}%
\pgfpathlineto{\pgfqpoint{1.788755in}{2.457242in}}%
\pgfpathlineto{\pgfqpoint{1.789162in}{2.457242in}}%
\pgfpathlineto{\pgfqpoint{1.789366in}{2.465470in}}%
\pgfpathlineto{\pgfqpoint{1.789570in}{2.407873in}}%
\pgfpathlineto{\pgfqpoint{1.789774in}{2.531296in}}%
\pgfpathlineto{\pgfqpoint{1.790589in}{2.424329in}}%
\pgfpathlineto{\pgfqpoint{1.790793in}{2.424329in}}%
\pgfpathlineto{\pgfqpoint{1.791404in}{2.416101in}}%
\pgfpathlineto{\pgfqpoint{1.791812in}{2.473698in}}%
\pgfpathlineto{\pgfqpoint{1.792423in}{2.432557in}}%
\pgfpathlineto{\pgfqpoint{1.792627in}{2.424329in}}%
\pgfpathlineto{\pgfqpoint{1.792831in}{2.473698in}}%
\pgfpathlineto{\pgfqpoint{1.793646in}{2.465470in}}%
\pgfpathlineto{\pgfqpoint{1.794053in}{2.383188in}}%
\pgfpathlineto{\pgfqpoint{1.794665in}{2.465470in}}%
\pgfpathlineto{\pgfqpoint{1.796499in}{2.555980in}}%
\pgfpathlineto{\pgfqpoint{1.797110in}{2.506611in}}%
\pgfpathlineto{\pgfqpoint{1.797314in}{2.539524in}}%
\pgfpathlineto{\pgfqpoint{1.798129in}{2.621806in}}%
\pgfpathlineto{\pgfqpoint{1.798333in}{2.547752in}}%
\pgfpathlineto{\pgfqpoint{1.798537in}{2.539524in}}%
\pgfpathlineto{\pgfqpoint{1.799556in}{2.399645in}}%
\pgfpathlineto{\pgfqpoint{1.800575in}{2.514839in}}%
\pgfpathlineto{\pgfqpoint{1.800779in}{2.506611in}}%
\pgfpathlineto{\pgfqpoint{1.801187in}{2.465470in}}%
\pgfpathlineto{\pgfqpoint{1.801594in}{2.547752in}}%
\pgfpathlineto{\pgfqpoint{1.801798in}{2.498383in}}%
\pgfpathlineto{\pgfqpoint{1.803021in}{2.564209in}}%
\pgfpathlineto{\pgfqpoint{1.804040in}{2.506611in}}%
\pgfpathlineto{\pgfqpoint{1.804244in}{2.564209in}}%
\pgfpathlineto{\pgfqpoint{1.804855in}{2.465470in}}%
\pgfpathlineto{\pgfqpoint{1.805059in}{2.498383in}}%
\pgfpathlineto{\pgfqpoint{1.806078in}{2.243309in}}%
\pgfpathlineto{\pgfqpoint{1.806689in}{2.333819in}}%
\pgfpathlineto{\pgfqpoint{1.808727in}{2.597121in}}%
\pgfpathlineto{\pgfqpoint{1.809135in}{2.621806in}}%
\pgfpathlineto{\pgfqpoint{1.809339in}{2.588893in}}%
\pgfpathlineto{\pgfqpoint{1.809950in}{2.292678in}}%
\pgfpathlineto{\pgfqpoint{1.810358in}{2.407873in}}%
\pgfpathlineto{\pgfqpoint{1.810561in}{2.646491in}}%
\pgfpathlineto{\pgfqpoint{1.811580in}{2.572437in}}%
\pgfpathlineto{\pgfqpoint{1.812599in}{2.358504in}}%
\pgfpathlineto{\pgfqpoint{1.812803in}{2.588893in}}%
\pgfpathlineto{\pgfqpoint{1.813415in}{2.514839in}}%
\pgfpathlineto{\pgfqpoint{1.813619in}{2.251537in}}%
\pgfpathlineto{\pgfqpoint{1.814434in}{2.613578in}}%
\pgfpathlineto{\pgfqpoint{1.815249in}{2.366732in}}%
\pgfpathlineto{\pgfqpoint{1.816064in}{2.457242in}}%
\pgfpathlineto{\pgfqpoint{1.816472in}{2.473698in}}%
\pgfpathlineto{\pgfqpoint{1.816676in}{2.440786in}}%
\pgfpathlineto{\pgfqpoint{1.817287in}{2.473698in}}%
\pgfpathlineto{\pgfqpoint{1.817898in}{2.333819in}}%
\pgfpathlineto{\pgfqpoint{1.818510in}{2.490155in}}%
\pgfpathlineto{\pgfqpoint{1.819121in}{2.473698in}}%
\pgfpathlineto{\pgfqpoint{1.819325in}{2.531296in}}%
\pgfpathlineto{\pgfqpoint{1.819936in}{2.407873in}}%
\pgfpathlineto{\pgfqpoint{1.821567in}{2.531296in}}%
\pgfpathlineto{\pgfqpoint{1.821974in}{2.210396in}}%
\pgfpathlineto{\pgfqpoint{1.822586in}{2.407873in}}%
\pgfpathlineto{\pgfqpoint{1.822993in}{2.531296in}}%
\pgfpathlineto{\pgfqpoint{1.823401in}{2.399645in}}%
\pgfpathlineto{\pgfqpoint{1.823605in}{2.407873in}}%
\pgfpathlineto{\pgfqpoint{1.824420in}{2.309134in}}%
\pgfpathlineto{\pgfqpoint{1.824012in}{2.465470in}}%
\pgfpathlineto{\pgfqpoint{1.824828in}{2.366732in}}%
\pgfpathlineto{\pgfqpoint{1.825031in}{2.366732in}}%
\pgfpathlineto{\pgfqpoint{1.825439in}{2.391416in}}%
\pgfpathlineto{\pgfqpoint{1.826662in}{2.251537in}}%
\pgfpathlineto{\pgfqpoint{1.826866in}{2.259765in}}%
\pgfpathlineto{\pgfqpoint{1.827070in}{2.300906in}}%
\pgfpathlineto{\pgfqpoint{1.827273in}{2.226852in}}%
\pgfpathlineto{\pgfqpoint{1.827885in}{2.251537in}}%
\pgfpathlineto{\pgfqpoint{1.828904in}{2.202168in}}%
\pgfpathlineto{\pgfqpoint{1.830127in}{2.267993in}}%
\pgfpathlineto{\pgfqpoint{1.830534in}{2.152798in}}%
\pgfpathlineto{\pgfqpoint{1.831146in}{2.226852in}}%
\pgfpathlineto{\pgfqpoint{1.831349in}{2.251537in}}%
\pgfpathlineto{\pgfqpoint{1.831553in}{2.210396in}}%
\pgfpathlineto{\pgfqpoint{1.832165in}{2.218624in}}%
\pgfpathlineto{\pgfqpoint{1.832368in}{2.218624in}}%
\pgfpathlineto{\pgfqpoint{1.832572in}{2.267993in}}%
\pgfpathlineto{\pgfqpoint{1.832980in}{2.202168in}}%
\pgfpathlineto{\pgfqpoint{1.833387in}{2.202168in}}%
\pgfpathlineto{\pgfqpoint{1.833795in}{2.128114in}}%
\pgfpathlineto{\pgfqpoint{1.833999in}{2.202168in}}%
\pgfpathlineto{\pgfqpoint{1.834203in}{2.284450in}}%
\pgfpathlineto{\pgfqpoint{1.834814in}{2.177483in}}%
\pgfpathlineto{\pgfqpoint{1.835018in}{2.177483in}}%
\pgfpathlineto{\pgfqpoint{1.835425in}{2.193939in}}%
\pgfpathlineto{\pgfqpoint{1.836037in}{2.152798in}}%
\pgfpathlineto{\pgfqpoint{1.836241in}{1.980006in}}%
\pgfpathlineto{\pgfqpoint{1.836852in}{2.235080in}}%
\pgfpathlineto{\pgfqpoint{1.837056in}{2.243309in}}%
\pgfpathlineto{\pgfqpoint{1.837260in}{2.235080in}}%
\pgfpathlineto{\pgfqpoint{1.837871in}{1.938865in}}%
\pgfpathlineto{\pgfqpoint{1.838279in}{2.202168in}}%
\pgfpathlineto{\pgfqpoint{1.839501in}{2.259765in}}%
\pgfpathlineto{\pgfqpoint{1.839909in}{2.226852in}}%
\pgfpathlineto{\pgfqpoint{1.840113in}{2.218624in}}%
\pgfpathlineto{\pgfqpoint{1.840317in}{2.251537in}}%
\pgfpathlineto{\pgfqpoint{1.840521in}{2.251537in}}%
\pgfpathlineto{\pgfqpoint{1.840724in}{2.276221in}}%
\pgfpathlineto{\pgfqpoint{1.841132in}{2.235080in}}%
\pgfpathlineto{\pgfqpoint{1.841336in}{2.251537in}}%
\pgfpathlineto{\pgfqpoint{1.841540in}{2.193939in}}%
\pgfpathlineto{\pgfqpoint{1.841947in}{2.267993in}}%
\pgfpathlineto{\pgfqpoint{1.842762in}{2.383188in}}%
\pgfpathlineto{\pgfqpoint{1.843170in}{2.309134in}}%
\pgfpathlineto{\pgfqpoint{1.843985in}{2.358504in}}%
\pgfpathlineto{\pgfqpoint{1.843781in}{2.292678in}}%
\pgfpathlineto{\pgfqpoint{1.844597in}{2.325591in}}%
\pgfpathlineto{\pgfqpoint{1.844800in}{2.292678in}}%
\pgfpathlineto{\pgfqpoint{1.845412in}{2.333819in}}%
\pgfpathlineto{\pgfqpoint{1.845616in}{2.317363in}}%
\pgfpathlineto{\pgfqpoint{1.846431in}{2.358504in}}%
\pgfpathlineto{\pgfqpoint{1.846838in}{2.333819in}}%
\pgfpathlineto{\pgfqpoint{1.847042in}{2.235080in}}%
\pgfpathlineto{\pgfqpoint{1.847857in}{2.358504in}}%
\pgfpathlineto{\pgfqpoint{1.848061in}{2.383188in}}%
\pgfpathlineto{\pgfqpoint{1.848265in}{2.342047in}}%
\pgfpathlineto{\pgfqpoint{1.848469in}{2.259765in}}%
\pgfpathlineto{\pgfqpoint{1.849284in}{2.309134in}}%
\pgfpathlineto{\pgfqpoint{1.850507in}{2.218624in}}%
\pgfpathlineto{\pgfqpoint{1.850711in}{2.202168in}}%
\pgfpathlineto{\pgfqpoint{1.850914in}{2.226852in}}%
\pgfpathlineto{\pgfqpoint{1.851118in}{2.259765in}}%
\pgfpathlineto{\pgfqpoint{1.851322in}{2.202168in}}%
\pgfpathlineto{\pgfqpoint{1.851933in}{2.218624in}}%
\pgfpathlineto{\pgfqpoint{1.852137in}{2.226852in}}%
\pgfpathlineto{\pgfqpoint{1.852341in}{2.152798in}}%
\pgfpathlineto{\pgfqpoint{1.853156in}{2.177483in}}%
\pgfpathlineto{\pgfqpoint{1.854175in}{2.243309in}}%
\pgfpathlineto{\pgfqpoint{1.854379in}{2.177483in}}%
\pgfpathlineto{\pgfqpoint{1.854991in}{2.317363in}}%
\pgfpathlineto{\pgfqpoint{1.855194in}{2.317363in}}%
\pgfpathlineto{\pgfqpoint{1.855602in}{2.284450in}}%
\pgfpathlineto{\pgfqpoint{1.856213in}{2.300906in}}%
\pgfpathlineto{\pgfqpoint{1.856417in}{2.309134in}}%
\pgfpathlineto{\pgfqpoint{1.857029in}{2.366732in}}%
\pgfpathlineto{\pgfqpoint{1.857640in}{2.226852in}}%
\pgfpathlineto{\pgfqpoint{1.858863in}{2.342047in}}%
\pgfpathlineto{\pgfqpoint{1.859067in}{2.317363in}}%
\pgfpathlineto{\pgfqpoint{1.859882in}{2.054060in}}%
\pgfpathlineto{\pgfqpoint{1.860289in}{2.267993in}}%
\pgfpathlineto{\pgfqpoint{1.861105in}{2.251537in}}%
\pgfpathlineto{\pgfqpoint{1.861308in}{2.309134in}}%
\pgfpathlineto{\pgfqpoint{1.861920in}{2.078745in}}%
\pgfpathlineto{\pgfqpoint{1.862735in}{2.226852in}}%
\pgfpathlineto{\pgfqpoint{1.862939in}{2.267993in}}%
\pgfpathlineto{\pgfqpoint{1.863550in}{2.169255in}}%
\pgfpathlineto{\pgfqpoint{1.865384in}{2.342047in}}%
\pgfpathlineto{\pgfqpoint{1.866403in}{2.300906in}}%
\pgfpathlineto{\pgfqpoint{1.866607in}{2.358504in}}%
\pgfpathlineto{\pgfqpoint{1.867423in}{2.300906in}}%
\pgfpathlineto{\pgfqpoint{1.867626in}{2.267993in}}%
\pgfpathlineto{\pgfqpoint{1.868238in}{2.342047in}}%
\pgfpathlineto{\pgfqpoint{1.868442in}{2.300906in}}%
\pgfpathlineto{\pgfqpoint{1.868849in}{2.251537in}}%
\pgfpathlineto{\pgfqpoint{1.869461in}{2.383188in}}%
\pgfpathlineto{\pgfqpoint{1.870683in}{2.284450in}}%
\pgfpathlineto{\pgfqpoint{1.871091in}{2.276221in}}%
\pgfpathlineto{\pgfqpoint{1.871906in}{2.366732in}}%
\pgfpathlineto{\pgfqpoint{1.872314in}{2.309134in}}%
\pgfpathlineto{\pgfqpoint{1.872721in}{2.317363in}}%
\pgfpathlineto{\pgfqpoint{1.873129in}{2.366732in}}%
\pgfpathlineto{\pgfqpoint{1.873944in}{2.251537in}}%
\pgfpathlineto{\pgfqpoint{1.874556in}{2.309134in}}%
\pgfpathlineto{\pgfqpoint{1.875167in}{2.300906in}}%
\pgfpathlineto{\pgfqpoint{1.875778in}{2.292678in}}%
\pgfpathlineto{\pgfqpoint{1.875982in}{2.333819in}}%
\pgfpathlineto{\pgfqpoint{1.876390in}{2.267993in}}%
\pgfpathlineto{\pgfqpoint{1.876797in}{2.350275in}}%
\pgfpathlineto{\pgfqpoint{1.877001in}{2.366732in}}%
\pgfpathlineto{\pgfqpoint{1.877205in}{2.358504in}}%
\pgfpathlineto{\pgfqpoint{1.878224in}{2.054060in}}%
\pgfpathlineto{\pgfqpoint{1.878428in}{2.095201in}}%
\pgfpathlineto{\pgfqpoint{1.878835in}{2.424329in}}%
\pgfpathlineto{\pgfqpoint{1.879651in}{2.383188in}}%
\pgfpathlineto{\pgfqpoint{1.880874in}{2.078745in}}%
\pgfpathlineto{\pgfqpoint{1.881893in}{2.342047in}}%
\pgfpathlineto{\pgfqpoint{1.882096in}{2.325591in}}%
\pgfpathlineto{\pgfqpoint{1.882912in}{2.407873in}}%
\pgfpathlineto{\pgfqpoint{1.883115in}{2.366732in}}%
\pgfpathlineto{\pgfqpoint{1.884134in}{2.111657in}}%
\pgfpathlineto{\pgfqpoint{1.883727in}{2.416101in}}%
\pgfpathlineto{\pgfqpoint{1.884338in}{2.218624in}}%
\pgfpathlineto{\pgfqpoint{1.885561in}{2.407873in}}%
\pgfpathlineto{\pgfqpoint{1.885765in}{2.399645in}}%
\pgfpathlineto{\pgfqpoint{1.886580in}{2.449014in}}%
\pgfpathlineto{\pgfqpoint{1.887803in}{2.284450in}}%
\pgfpathlineto{\pgfqpoint{1.888822in}{2.457242in}}%
\pgfpathlineto{\pgfqpoint{1.888414in}{2.177483in}}%
\pgfpathlineto{\pgfqpoint{1.889026in}{2.407873in}}%
\pgfpathlineto{\pgfqpoint{1.889229in}{2.144570in}}%
\pgfpathlineto{\pgfqpoint{1.890045in}{2.292678in}}%
\pgfpathlineto{\pgfqpoint{1.890248in}{2.366732in}}%
\pgfpathlineto{\pgfqpoint{1.890656in}{2.128114in}}%
\pgfpathlineto{\pgfqpoint{1.890860in}{2.062288in}}%
\pgfpathlineto{\pgfqpoint{1.891064in}{2.317363in}}%
\pgfpathlineto{\pgfqpoint{1.891267in}{2.317363in}}%
\pgfpathlineto{\pgfqpoint{1.891675in}{2.333819in}}%
\pgfpathlineto{\pgfqpoint{1.891879in}{2.407873in}}%
\pgfpathlineto{\pgfqpoint{1.892490in}{2.226852in}}%
\pgfpathlineto{\pgfqpoint{1.893713in}{2.358504in}}%
\pgfpathlineto{\pgfqpoint{1.893917in}{2.350275in}}%
\pgfpathlineto{\pgfqpoint{1.894325in}{2.457242in}}%
\pgfpathlineto{\pgfqpoint{1.895547in}{2.416101in}}%
\pgfpathlineto{\pgfqpoint{1.896974in}{2.333819in}}%
\pgfpathlineto{\pgfqpoint{1.897585in}{2.424329in}}%
\pgfpathlineto{\pgfqpoint{1.897993in}{2.333819in}}%
\pgfpathlineto{\pgfqpoint{1.898401in}{2.292678in}}%
\pgfpathlineto{\pgfqpoint{1.898604in}{2.350275in}}%
\pgfpathlineto{\pgfqpoint{1.899216in}{2.358504in}}%
\pgfpathlineto{\pgfqpoint{1.900439in}{2.078745in}}%
\pgfpathlineto{\pgfqpoint{1.899827in}{2.374960in}}%
\pgfpathlineto{\pgfqpoint{1.900642in}{2.292678in}}%
\pgfpathlineto{\pgfqpoint{1.901865in}{2.374960in}}%
\pgfpathlineto{\pgfqpoint{1.902273in}{2.358504in}}%
\pgfpathlineto{\pgfqpoint{1.903496in}{2.267993in}}%
\pgfpathlineto{\pgfqpoint{1.902680in}{2.383188in}}%
\pgfpathlineto{\pgfqpoint{1.903903in}{2.292678in}}%
\pgfpathlineto{\pgfqpoint{1.904922in}{2.407873in}}%
\pgfpathlineto{\pgfqpoint{1.905126in}{2.342047in}}%
\pgfpathlineto{\pgfqpoint{1.905330in}{2.350275in}}%
\pgfpathlineto{\pgfqpoint{1.905534in}{2.325591in}}%
\pgfpathlineto{\pgfqpoint{1.905737in}{2.325591in}}%
\pgfpathlineto{\pgfqpoint{1.905941in}{2.317363in}}%
\pgfpathlineto{\pgfqpoint{1.906349in}{2.300906in}}%
\pgfpathlineto{\pgfqpoint{1.907572in}{2.432557in}}%
\pgfpathlineto{\pgfqpoint{1.909610in}{2.333819in}}%
\pgfpathlineto{\pgfqpoint{1.909814in}{2.399645in}}%
\pgfpathlineto{\pgfqpoint{1.910629in}{2.317363in}}%
\pgfpathlineto{\pgfqpoint{1.910833in}{2.374960in}}%
\pgfpathlineto{\pgfqpoint{1.911036in}{2.276221in}}%
\pgfpathlineto{\pgfqpoint{1.911852in}{2.399645in}}%
\pgfpathlineto{\pgfqpoint{1.912667in}{2.300906in}}%
\pgfpathlineto{\pgfqpoint{1.912259in}{2.416101in}}%
\pgfpathlineto{\pgfqpoint{1.912871in}{2.350275in}}%
\pgfpathlineto{\pgfqpoint{1.913074in}{2.407873in}}%
\pgfpathlineto{\pgfqpoint{1.913890in}{2.333819in}}%
\pgfpathlineto{\pgfqpoint{1.914093in}{2.325591in}}%
\pgfpathlineto{\pgfqpoint{1.914297in}{2.358504in}}%
\pgfpathlineto{\pgfqpoint{1.914501in}{2.366732in}}%
\pgfpathlineto{\pgfqpoint{1.915520in}{2.235080in}}%
\pgfpathlineto{\pgfqpoint{1.915724in}{2.333819in}}%
\pgfpathlineto{\pgfqpoint{1.916131in}{2.325591in}}%
\pgfpathlineto{\pgfqpoint{1.916335in}{2.342047in}}%
\pgfpathlineto{\pgfqpoint{1.916743in}{2.309134in}}%
\pgfpathlineto{\pgfqpoint{1.917762in}{2.416101in}}%
\pgfpathlineto{\pgfqpoint{1.919392in}{2.350275in}}%
\pgfpathlineto{\pgfqpoint{1.919596in}{2.432557in}}%
\pgfpathlineto{\pgfqpoint{1.920207in}{2.317363in}}%
\pgfpathlineto{\pgfqpoint{1.921023in}{2.136342in}}%
\pgfpathlineto{\pgfqpoint{1.920615in}{2.374960in}}%
\pgfpathlineto{\pgfqpoint{1.921227in}{2.161027in}}%
\pgfpathlineto{\pgfqpoint{1.922246in}{2.383188in}}%
\pgfpathlineto{\pgfqpoint{1.922449in}{2.358504in}}%
\pgfpathlineto{\pgfqpoint{1.923061in}{2.078745in}}%
\pgfpathlineto{\pgfqpoint{1.923468in}{2.374960in}}%
\pgfpathlineto{\pgfqpoint{1.923672in}{2.358504in}}%
\pgfpathlineto{\pgfqpoint{1.923876in}{2.399645in}}%
\pgfpathlineto{\pgfqpoint{1.924284in}{2.457242in}}%
\pgfpathlineto{\pgfqpoint{1.924487in}{2.407873in}}%
\pgfpathlineto{\pgfqpoint{1.925303in}{2.333819in}}%
\pgfpathlineto{\pgfqpoint{1.925506in}{2.383188in}}%
\pgfpathlineto{\pgfqpoint{1.926118in}{2.407873in}}%
\pgfpathlineto{\pgfqpoint{1.927544in}{2.292678in}}%
\pgfpathlineto{\pgfqpoint{1.927748in}{2.267993in}}%
\pgfpathlineto{\pgfqpoint{1.927952in}{2.350275in}}%
\pgfpathlineto{\pgfqpoint{1.928563in}{2.342047in}}%
\pgfpathlineto{\pgfqpoint{1.929175in}{2.366732in}}%
\pgfpathlineto{\pgfqpoint{1.929379in}{2.333819in}}%
\pgfpathlineto{\pgfqpoint{1.929582in}{2.358504in}}%
\pgfpathlineto{\pgfqpoint{1.929786in}{2.086973in}}%
\pgfpathlineto{\pgfqpoint{1.930601in}{2.226852in}}%
\pgfpathlineto{\pgfqpoint{1.932232in}{2.292678in}}%
\pgfpathlineto{\pgfqpoint{1.932436in}{2.284450in}}%
\pgfpathlineto{\pgfqpoint{1.933251in}{2.251537in}}%
\pgfpathlineto{\pgfqpoint{1.933658in}{2.333819in}}%
\pgfpathlineto{\pgfqpoint{1.934881in}{2.267993in}}%
\pgfpathlineto{\pgfqpoint{1.934066in}{2.366732in}}%
\pgfpathlineto{\pgfqpoint{1.935085in}{2.292678in}}%
\pgfpathlineto{\pgfqpoint{1.935900in}{2.333819in}}%
\pgfpathlineto{\pgfqpoint{1.937327in}{2.210396in}}%
\pgfpathlineto{\pgfqpoint{1.938550in}{2.342047in}}%
\pgfpathlineto{\pgfqpoint{1.938957in}{2.350275in}}%
\pgfpathlineto{\pgfqpoint{1.939161in}{2.317363in}}%
\pgfpathlineto{\pgfqpoint{1.939365in}{2.374960in}}%
\pgfpathlineto{\pgfqpoint{1.939773in}{2.366732in}}%
\pgfpathlineto{\pgfqpoint{1.939976in}{2.374960in}}%
\pgfpathlineto{\pgfqpoint{1.940180in}{2.366732in}}%
\pgfpathlineto{\pgfqpoint{1.940792in}{2.276221in}}%
\pgfpathlineto{\pgfqpoint{1.941403in}{2.333819in}}%
\pgfpathlineto{\pgfqpoint{1.941607in}{2.325591in}}%
\pgfpathlineto{\pgfqpoint{1.941811in}{2.333819in}}%
\pgfpathlineto{\pgfqpoint{1.942422in}{2.498383in}}%
\pgfpathlineto{\pgfqpoint{1.942830in}{2.391416in}}%
\pgfpathlineto{\pgfqpoint{1.943849in}{2.309134in}}%
\pgfpathlineto{\pgfqpoint{1.944052in}{2.333819in}}%
\pgfpathlineto{\pgfqpoint{1.944256in}{2.309134in}}%
\pgfpathlineto{\pgfqpoint{1.944460in}{2.358504in}}%
\pgfpathlineto{\pgfqpoint{1.945683in}{2.424329in}}%
\pgfpathlineto{\pgfqpoint{1.946294in}{2.374960in}}%
\pgfpathlineto{\pgfqpoint{1.946498in}{2.325591in}}%
\pgfpathlineto{\pgfqpoint{1.947109in}{2.416101in}}%
\pgfpathlineto{\pgfqpoint{1.947517in}{2.391416in}}%
\pgfpathlineto{\pgfqpoint{1.947721in}{2.399645in}}%
\pgfpathlineto{\pgfqpoint{1.947925in}{2.449014in}}%
\pgfpathlineto{\pgfqpoint{1.948536in}{2.407873in}}%
\pgfpathlineto{\pgfqpoint{1.949148in}{2.440786in}}%
\pgfpathlineto{\pgfqpoint{1.950167in}{2.169255in}}%
\pgfpathlineto{\pgfqpoint{1.951389in}{2.424329in}}%
\pgfpathlineto{\pgfqpoint{1.951593in}{2.407873in}}%
\pgfpathlineto{\pgfqpoint{1.951797in}{2.424329in}}%
\pgfpathlineto{\pgfqpoint{1.952001in}{2.358504in}}%
\pgfpathlineto{\pgfqpoint{1.952408in}{2.366732in}}%
\pgfpathlineto{\pgfqpoint{1.952612in}{2.465470in}}%
\pgfpathlineto{\pgfqpoint{1.953427in}{2.383188in}}%
\pgfpathlineto{\pgfqpoint{1.954243in}{2.465470in}}%
\pgfpathlineto{\pgfqpoint{1.954446in}{2.407873in}}%
\pgfpathlineto{\pgfqpoint{1.954854in}{2.309134in}}%
\pgfpathlineto{\pgfqpoint{1.955873in}{2.366732in}}%
\pgfpathlineto{\pgfqpoint{1.956077in}{2.473698in}}%
\pgfpathlineto{\pgfqpoint{1.956892in}{2.358504in}}%
\pgfpathlineto{\pgfqpoint{1.957096in}{2.218624in}}%
\pgfpathlineto{\pgfqpoint{1.957503in}{2.465470in}}%
\pgfpathlineto{\pgfqpoint{1.957911in}{2.350275in}}%
\pgfpathlineto{\pgfqpoint{1.959134in}{2.539524in}}%
\pgfpathlineto{\pgfqpoint{1.959541in}{2.325591in}}%
\pgfpathlineto{\pgfqpoint{1.959745in}{2.580665in}}%
\pgfpathlineto{\pgfqpoint{1.960153in}{2.539524in}}%
\pgfpathlineto{\pgfqpoint{1.960764in}{2.490155in}}%
\pgfpathlineto{\pgfqpoint{1.960968in}{2.358504in}}%
\pgfpathlineto{\pgfqpoint{1.961783in}{2.498383in}}%
\pgfpathlineto{\pgfqpoint{1.962395in}{2.251537in}}%
\pgfpathlineto{\pgfqpoint{1.962802in}{2.506611in}}%
\pgfpathlineto{\pgfqpoint{1.963210in}{2.457242in}}%
\pgfpathlineto{\pgfqpoint{1.963414in}{2.210396in}}%
\pgfpathlineto{\pgfqpoint{1.963821in}{2.490155in}}%
\pgfpathlineto{\pgfqpoint{1.964229in}{2.449014in}}%
\pgfpathlineto{\pgfqpoint{1.965044in}{2.490155in}}%
\pgfpathlineto{\pgfqpoint{1.964840in}{2.432557in}}%
\pgfpathlineto{\pgfqpoint{1.965452in}{2.481927in}}%
\pgfpathlineto{\pgfqpoint{1.965656in}{2.498383in}}%
\pgfpathlineto{\pgfqpoint{1.966063in}{2.490155in}}%
\pgfpathlineto{\pgfqpoint{1.966878in}{2.350275in}}%
\pgfpathlineto{\pgfqpoint{1.966471in}{2.498383in}}%
\pgfpathlineto{\pgfqpoint{1.967082in}{2.424329in}}%
\pgfpathlineto{\pgfqpoint{1.968101in}{2.523068in}}%
\pgfpathlineto{\pgfqpoint{1.967694in}{2.383188in}}%
\pgfpathlineto{\pgfqpoint{1.968305in}{2.514839in}}%
\pgfpathlineto{\pgfqpoint{1.968713in}{2.424329in}}%
\pgfpathlineto{\pgfqpoint{1.969324in}{2.531296in}}%
\pgfpathlineto{\pgfqpoint{1.969732in}{2.481927in}}%
\pgfpathlineto{\pgfqpoint{1.969935in}{2.523068in}}%
\pgfpathlineto{\pgfqpoint{1.970139in}{2.383188in}}%
\pgfpathlineto{\pgfqpoint{1.970954in}{2.498383in}}%
\pgfpathlineto{\pgfqpoint{1.971566in}{2.465470in}}%
\pgfpathlineto{\pgfqpoint{1.972381in}{2.473698in}}%
\pgfpathlineto{\pgfqpoint{1.972585in}{2.490155in}}%
\pgfpathlineto{\pgfqpoint{1.972992in}{2.473698in}}%
\pgfpathlineto{\pgfqpoint{1.973196in}{2.432557in}}%
\pgfpathlineto{\pgfqpoint{1.973400in}{2.481927in}}%
\pgfpathlineto{\pgfqpoint{1.974011in}{2.465470in}}%
\pgfpathlineto{\pgfqpoint{1.975031in}{2.399645in}}%
\pgfpathlineto{\pgfqpoint{1.975642in}{2.325591in}}%
\pgfpathlineto{\pgfqpoint{1.976050in}{2.383188in}}%
\pgfpathlineto{\pgfqpoint{1.976253in}{2.407873in}}%
\pgfpathlineto{\pgfqpoint{1.976661in}{2.342047in}}%
\pgfpathlineto{\pgfqpoint{1.976865in}{2.358504in}}%
\pgfpathlineto{\pgfqpoint{1.977884in}{2.416101in}}%
\pgfpathlineto{\pgfqpoint{1.978088in}{2.399645in}}%
\pgfpathlineto{\pgfqpoint{1.979514in}{2.111657in}}%
\pgfpathlineto{\pgfqpoint{1.980737in}{2.226852in}}%
\pgfpathlineto{\pgfqpoint{1.982164in}{1.947093in}}%
\pgfpathlineto{\pgfqpoint{1.982571in}{2.054060in}}%
\pgfpathlineto{\pgfqpoint{1.983183in}{1.955322in}}%
\pgfpathlineto{\pgfqpoint{1.984202in}{2.012919in}}%
\pgfpathlineto{\pgfqpoint{1.985424in}{1.864811in}}%
\pgfpathlineto{\pgfqpoint{1.985628in}{1.881268in}}%
\pgfpathlineto{\pgfqpoint{1.985832in}{1.881268in}}%
\pgfpathlineto{\pgfqpoint{1.986240in}{1.675563in}}%
\pgfpathlineto{\pgfqpoint{1.986647in}{1.980006in}}%
\pgfpathlineto{\pgfqpoint{1.987055in}{2.029375in}}%
\pgfpathlineto{\pgfqpoint{1.987666in}{1.963550in}}%
\pgfpathlineto{\pgfqpoint{1.988889in}{2.144570in}}%
\pgfpathlineto{\pgfqpoint{1.989297in}{2.070516in}}%
\pgfpathlineto{\pgfqpoint{1.989501in}{1.996463in}}%
\pgfpathlineto{\pgfqpoint{1.990112in}{2.136342in}}%
\pgfpathlineto{\pgfqpoint{1.991335in}{2.070516in}}%
\pgfpathlineto{\pgfqpoint{1.991742in}{2.095201in}}%
\pgfpathlineto{\pgfqpoint{1.992761in}{2.136342in}}%
\pgfpathlineto{\pgfqpoint{1.994188in}{1.996463in}}%
\pgfpathlineto{\pgfqpoint{1.994799in}{2.045832in}}%
\pgfpathlineto{\pgfqpoint{1.995003in}{1.963550in}}%
\pgfpathlineto{\pgfqpoint{1.995207in}{1.881268in}}%
\pgfpathlineto{\pgfqpoint{1.995818in}{2.070516in}}%
\pgfpathlineto{\pgfqpoint{1.996226in}{2.004691in}}%
\pgfpathlineto{\pgfqpoint{1.996634in}{2.144570in}}%
\pgfpathlineto{\pgfqpoint{1.997653in}{1.980006in}}%
\pgfpathlineto{\pgfqpoint{1.998060in}{1.988234in}}%
\pgfpathlineto{\pgfqpoint{1.999487in}{1.840127in}}%
\pgfpathlineto{\pgfqpoint{2.000506in}{1.922409in}}%
\pgfpathlineto{\pgfqpoint{2.000710in}{1.864811in}}%
\pgfpathlineto{\pgfqpoint{2.000913in}{1.634422in}}%
\pgfpathlineto{\pgfqpoint{2.001729in}{1.955322in}}%
\pgfpathlineto{\pgfqpoint{2.001933in}{1.914181in}}%
\pgfpathlineto{\pgfqpoint{2.002544in}{1.988234in}}%
\pgfpathlineto{\pgfqpoint{2.002748in}{1.955322in}}%
\pgfpathlineto{\pgfqpoint{2.002952in}{1.947093in}}%
\pgfpathlineto{\pgfqpoint{2.003359in}{1.971778in}}%
\pgfpathlineto{\pgfqpoint{2.003971in}{2.045832in}}%
\pgfpathlineto{\pgfqpoint{2.004174in}{1.996463in}}%
\pgfpathlineto{\pgfqpoint{2.005397in}{1.757845in}}%
\pgfpathlineto{\pgfqpoint{2.006416in}{1.988234in}}%
\pgfpathlineto{\pgfqpoint{2.006620in}{1.938865in}}%
\pgfpathlineto{\pgfqpoint{2.007028in}{1.996463in}}%
\pgfpathlineto{\pgfqpoint{2.007435in}{1.905952in}}%
\pgfpathlineto{\pgfqpoint{2.007639in}{1.938865in}}%
\pgfpathlineto{\pgfqpoint{2.008047in}{1.881268in}}%
\pgfpathlineto{\pgfqpoint{2.008658in}{1.922409in}}%
\pgfpathlineto{\pgfqpoint{2.009269in}{1.905952in}}%
\pgfpathlineto{\pgfqpoint{2.010288in}{2.012919in}}%
\pgfpathlineto{\pgfqpoint{2.010492in}{1.905952in}}%
\pgfpathlineto{\pgfqpoint{2.011307in}{1.980006in}}%
\pgfpathlineto{\pgfqpoint{2.012530in}{1.914181in}}%
\pgfpathlineto{\pgfqpoint{2.011919in}{1.996463in}}%
\pgfpathlineto{\pgfqpoint{2.012734in}{1.938865in}}%
\pgfpathlineto{\pgfqpoint{2.012938in}{1.947093in}}%
\pgfpathlineto{\pgfqpoint{2.013142in}{1.905952in}}%
\pgfpathlineto{\pgfqpoint{2.013549in}{1.980006in}}%
\pgfpathlineto{\pgfqpoint{2.013957in}{1.947093in}}%
\pgfpathlineto{\pgfqpoint{2.014976in}{1.897724in}}%
\pgfpathlineto{\pgfqpoint{2.015995in}{1.971778in}}%
\pgfpathlineto{\pgfqpoint{2.016199in}{1.947093in}}%
\pgfpathlineto{\pgfqpoint{2.017014in}{1.889496in}}%
\pgfpathlineto{\pgfqpoint{2.016606in}{1.996463in}}%
\pgfpathlineto{\pgfqpoint{2.017218in}{1.930637in}}%
\pgfpathlineto{\pgfqpoint{2.018033in}{2.045832in}}%
\pgfpathlineto{\pgfqpoint{2.018237in}{1.963550in}}%
\pgfpathlineto{\pgfqpoint{2.019052in}{1.864811in}}%
\pgfpathlineto{\pgfqpoint{2.019460in}{1.914181in}}%
\pgfpathlineto{\pgfqpoint{2.020275in}{2.037604in}}%
\pgfpathlineto{\pgfqpoint{2.020886in}{1.996463in}}%
\pgfpathlineto{\pgfqpoint{2.022313in}{1.798986in}}%
\pgfpathlineto{\pgfqpoint{2.023332in}{1.881268in}}%
\pgfpathlineto{\pgfqpoint{2.023536in}{1.831899in}}%
\pgfpathlineto{\pgfqpoint{2.024147in}{1.741388in}}%
\pgfpathlineto{\pgfqpoint{2.024962in}{1.757845in}}%
\pgfpathlineto{\pgfqpoint{2.027204in}{1.988234in}}%
\pgfpathlineto{\pgfqpoint{2.027408in}{1.980006in}}%
\pgfpathlineto{\pgfqpoint{2.028223in}{1.996463in}}%
\pgfpathlineto{\pgfqpoint{2.027815in}{1.971778in}}%
\pgfpathlineto{\pgfqpoint{2.028427in}{1.980006in}}%
\pgfpathlineto{\pgfqpoint{2.028631in}{1.980006in}}%
\pgfpathlineto{\pgfqpoint{2.028835in}{1.971778in}}%
\pgfpathlineto{\pgfqpoint{2.029038in}{2.004691in}}%
\pgfpathlineto{\pgfqpoint{2.029242in}{2.004691in}}%
\pgfpathlineto{\pgfqpoint{2.030261in}{1.905952in}}%
\pgfpathlineto{\pgfqpoint{2.030465in}{1.914181in}}%
\pgfpathlineto{\pgfqpoint{2.031280in}{1.988234in}}%
\pgfpathlineto{\pgfqpoint{2.031688in}{1.971778in}}%
\pgfpathlineto{\pgfqpoint{2.032911in}{1.889496in}}%
\pgfpathlineto{\pgfqpoint{2.033114in}{1.914181in}}%
\pgfpathlineto{\pgfqpoint{2.033318in}{1.988234in}}%
\pgfpathlineto{\pgfqpoint{2.033930in}{1.831899in}}%
\pgfpathlineto{\pgfqpoint{2.034133in}{1.955322in}}%
\pgfpathlineto{\pgfqpoint{2.034541in}{1.634422in}}%
\pgfpathlineto{\pgfqpoint{2.035356in}{1.840127in}}%
\pgfpathlineto{\pgfqpoint{2.036375in}{1.683791in}}%
\pgfpathlineto{\pgfqpoint{2.036783in}{1.692019in}}%
\pgfpathlineto{\pgfqpoint{2.037190in}{1.757845in}}%
\pgfpathlineto{\pgfqpoint{2.038006in}{1.749617in}}%
\pgfpathlineto{\pgfqpoint{2.038209in}{1.708476in}}%
\pgfpathlineto{\pgfqpoint{2.038617in}{1.815442in}}%
\pgfpathlineto{\pgfqpoint{2.039025in}{1.757845in}}%
\pgfpathlineto{\pgfqpoint{2.039432in}{1.733160in}}%
\pgfpathlineto{\pgfqpoint{2.039840in}{1.815442in}}%
\pgfpathlineto{\pgfqpoint{2.040655in}{1.692019in}}%
\pgfpathlineto{\pgfqpoint{2.041063in}{1.749617in}}%
\pgfpathlineto{\pgfqpoint{2.041878in}{1.724932in}}%
\pgfpathlineto{\pgfqpoint{2.042082in}{1.733160in}}%
\pgfpathlineto{\pgfqpoint{2.043101in}{1.790758in}}%
\pgfpathlineto{\pgfqpoint{2.043305in}{1.609737in}}%
\pgfpathlineto{\pgfqpoint{2.044120in}{1.856583in}}%
\pgfpathlineto{\pgfqpoint{2.044324in}{1.848355in}}%
\pgfpathlineto{\pgfqpoint{2.044527in}{1.864811in}}%
\pgfpathlineto{\pgfqpoint{2.044935in}{1.938865in}}%
\pgfpathlineto{\pgfqpoint{2.045343in}{1.823670in}}%
\pgfpathlineto{\pgfqpoint{2.045546in}{1.905952in}}%
\pgfpathlineto{\pgfqpoint{2.046973in}{1.815442in}}%
\pgfpathlineto{\pgfqpoint{2.047381in}{1.914181in}}%
\pgfpathlineto{\pgfqpoint{2.047584in}{1.790758in}}%
\pgfpathlineto{\pgfqpoint{2.047788in}{1.856583in}}%
\pgfpathlineto{\pgfqpoint{2.049419in}{1.576824in}}%
\pgfpathlineto{\pgfqpoint{2.049826in}{1.782529in}}%
\pgfpathlineto{\pgfqpoint{2.050641in}{1.741388in}}%
\pgfpathlineto{\pgfqpoint{2.051457in}{1.848355in}}%
\pgfpathlineto{\pgfqpoint{2.052068in}{1.840127in}}%
\pgfpathlineto{\pgfqpoint{2.053087in}{1.733160in}}%
\pgfpathlineto{\pgfqpoint{2.053291in}{1.749617in}}%
\pgfpathlineto{\pgfqpoint{2.053495in}{1.741388in}}%
\pgfpathlineto{\pgfqpoint{2.053902in}{1.823670in}}%
\pgfpathlineto{\pgfqpoint{2.054514in}{1.724932in}}%
\pgfpathlineto{\pgfqpoint{2.054921in}{1.831899in}}%
\pgfpathlineto{\pgfqpoint{2.055329in}{1.823670in}}%
\pgfpathlineto{\pgfqpoint{2.055533in}{1.642650in}}%
\pgfpathlineto{\pgfqpoint{2.055737in}{1.864811in}}%
\pgfpathlineto{\pgfqpoint{2.056348in}{1.766073in}}%
\pgfpathlineto{\pgfqpoint{2.056756in}{1.889496in}}%
\pgfpathlineto{\pgfqpoint{2.057367in}{1.782529in}}%
\pgfpathlineto{\pgfqpoint{2.057571in}{1.766073in}}%
\pgfpathlineto{\pgfqpoint{2.057775in}{1.774301in}}%
\pgfpathlineto{\pgfqpoint{2.057978in}{1.831899in}}%
\pgfpathlineto{\pgfqpoint{2.058182in}{1.766073in}}%
\pgfpathlineto{\pgfqpoint{2.058794in}{1.790758in}}%
\pgfpathlineto{\pgfqpoint{2.058997in}{1.766073in}}%
\pgfpathlineto{\pgfqpoint{2.059201in}{1.856583in}}%
\pgfpathlineto{\pgfqpoint{2.059405in}{1.897724in}}%
\pgfpathlineto{\pgfqpoint{2.059813in}{1.840127in}}%
\pgfpathlineto{\pgfqpoint{2.060016in}{1.848355in}}%
\pgfpathlineto{\pgfqpoint{2.060220in}{1.617965in}}%
\pgfpathlineto{\pgfqpoint{2.060832in}{1.881268in}}%
\pgfpathlineto{\pgfqpoint{2.061035in}{1.873040in}}%
\pgfpathlineto{\pgfqpoint{2.061239in}{1.897724in}}%
\pgfpathlineto{\pgfqpoint{2.061443in}{1.823670in}}%
\pgfpathlineto{\pgfqpoint{2.061851in}{1.856583in}}%
\pgfpathlineto{\pgfqpoint{2.063073in}{1.692019in}}%
\pgfpathlineto{\pgfqpoint{2.063481in}{1.782529in}}%
\pgfpathlineto{\pgfqpoint{2.063685in}{1.881268in}}%
\pgfpathlineto{\pgfqpoint{2.063889in}{1.667335in}}%
\pgfpathlineto{\pgfqpoint{2.064500in}{1.741388in}}%
\pgfpathlineto{\pgfqpoint{2.065111in}{1.708476in}}%
\pgfpathlineto{\pgfqpoint{2.064908in}{1.757845in}}%
\pgfpathlineto{\pgfqpoint{2.065519in}{1.741388in}}%
\pgfpathlineto{\pgfqpoint{2.065927in}{1.774301in}}%
\pgfpathlineto{\pgfqpoint{2.066538in}{1.757845in}}%
\pgfpathlineto{\pgfqpoint{2.066742in}{1.700247in}}%
\pgfpathlineto{\pgfqpoint{2.066946in}{1.782529in}}%
\pgfpathlineto{\pgfqpoint{2.067761in}{1.716704in}}%
\pgfpathlineto{\pgfqpoint{2.068576in}{1.831899in}}%
\pgfpathlineto{\pgfqpoint{2.069799in}{1.601509in}}%
\pgfpathlineto{\pgfqpoint{2.071226in}{1.757845in}}%
\pgfpathlineto{\pgfqpoint{2.072245in}{1.659106in}}%
\pgfpathlineto{\pgfqpoint{2.072448in}{1.692019in}}%
\pgfpathlineto{\pgfqpoint{2.072856in}{1.724932in}}%
\pgfpathlineto{\pgfqpoint{2.073264in}{1.667335in}}%
\pgfpathlineto{\pgfqpoint{2.073467in}{1.642650in}}%
\pgfpathlineto{\pgfqpoint{2.074283in}{1.667335in}}%
\pgfpathlineto{\pgfqpoint{2.075709in}{1.790758in}}%
\pgfpathlineto{\pgfqpoint{2.075913in}{1.774301in}}%
\pgfpathlineto{\pgfqpoint{2.076117in}{1.741388in}}%
\pgfpathlineto{\pgfqpoint{2.076321in}{1.798986in}}%
\pgfpathlineto{\pgfqpoint{2.076728in}{1.757845in}}%
\pgfpathlineto{\pgfqpoint{2.077543in}{1.905952in}}%
\pgfpathlineto{\pgfqpoint{2.077747in}{1.864811in}}%
\pgfpathlineto{\pgfqpoint{2.078970in}{1.724932in}}%
\pgfpathlineto{\pgfqpoint{2.079989in}{1.807214in}}%
\pgfpathlineto{\pgfqpoint{2.080193in}{1.766073in}}%
\pgfpathlineto{\pgfqpoint{2.080804in}{1.823670in}}%
\pgfpathlineto{\pgfqpoint{2.081212in}{1.766073in}}%
\pgfpathlineto{\pgfqpoint{2.081619in}{1.667335in}}%
\pgfpathlineto{\pgfqpoint{2.082027in}{1.774301in}}%
\pgfpathlineto{\pgfqpoint{2.082231in}{1.790758in}}%
\pgfpathlineto{\pgfqpoint{2.082435in}{1.766073in}}%
\pgfpathlineto{\pgfqpoint{2.083046in}{1.469858in}}%
\pgfpathlineto{\pgfqpoint{2.083454in}{1.741388in}}%
\pgfpathlineto{\pgfqpoint{2.083658in}{1.757845in}}%
\pgfpathlineto{\pgfqpoint{2.084880in}{1.593281in}}%
\pgfpathlineto{\pgfqpoint{2.085899in}{1.667335in}}%
\pgfpathlineto{\pgfqpoint{2.086511in}{1.716704in}}%
\pgfpathlineto{\pgfqpoint{2.086715in}{1.659106in}}%
\pgfpathlineto{\pgfqpoint{2.087530in}{1.527455in}}%
\pgfpathlineto{\pgfqpoint{2.087734in}{1.659106in}}%
\pgfpathlineto{\pgfqpoint{2.088141in}{1.634422in}}%
\pgfpathlineto{\pgfqpoint{2.088345in}{1.733160in}}%
\pgfpathlineto{\pgfqpoint{2.089160in}{1.626194in}}%
\pgfpathlineto{\pgfqpoint{2.089975in}{1.700247in}}%
\pgfpathlineto{\pgfqpoint{2.090179in}{1.650878in}}%
\pgfpathlineto{\pgfqpoint{2.090791in}{1.535683in}}%
\pgfpathlineto{\pgfqpoint{2.091606in}{1.568596in}}%
\pgfpathlineto{\pgfqpoint{2.092421in}{1.650878in}}%
\pgfpathlineto{\pgfqpoint{2.092829in}{1.601509in}}%
\pgfpathlineto{\pgfqpoint{2.093236in}{1.609737in}}%
\pgfpathlineto{\pgfqpoint{2.093848in}{1.560368in}}%
\pgfpathlineto{\pgfqpoint{2.094051in}{1.585053in}}%
\pgfpathlineto{\pgfqpoint{2.094255in}{1.510999in}}%
\pgfpathlineto{\pgfqpoint{2.094867in}{1.568596in}}%
\pgfpathlineto{\pgfqpoint{2.095070in}{1.568596in}}%
\pgfpathlineto{\pgfqpoint{2.095274in}{1.560368in}}%
\pgfpathlineto{\pgfqpoint{2.095478in}{1.593281in}}%
\pgfpathlineto{\pgfqpoint{2.095682in}{1.576824in}}%
\pgfpathlineto{\pgfqpoint{2.097312in}{1.700247in}}%
\pgfpathlineto{\pgfqpoint{2.097516in}{1.617965in}}%
\pgfpathlineto{\pgfqpoint{2.098331in}{1.733160in}}%
\pgfpathlineto{\pgfqpoint{2.099758in}{1.445173in}}%
\pgfpathlineto{\pgfqpoint{2.099962in}{1.560368in}}%
\pgfpathlineto{\pgfqpoint{2.101185in}{1.675563in}}%
\pgfpathlineto{\pgfqpoint{2.102407in}{1.593281in}}%
\pgfpathlineto{\pgfqpoint{2.102611in}{1.642650in}}%
\pgfpathlineto{\pgfqpoint{2.102815in}{1.494542in}}%
\pgfpathlineto{\pgfqpoint{2.103630in}{1.626194in}}%
\pgfpathlineto{\pgfqpoint{2.104242in}{1.371119in}}%
\pgfpathlineto{\pgfqpoint{2.104038in}{1.650878in}}%
\pgfpathlineto{\pgfqpoint{2.104649in}{1.585053in}}%
\pgfpathlineto{\pgfqpoint{2.104853in}{1.609737in}}%
\pgfpathlineto{\pgfqpoint{2.105261in}{1.552140in}}%
\pgfpathlineto{\pgfqpoint{2.105668in}{1.593281in}}%
\pgfpathlineto{\pgfqpoint{2.106280in}{1.552140in}}%
\pgfpathlineto{\pgfqpoint{2.106483in}{1.634422in}}%
\pgfpathlineto{\pgfqpoint{2.106687in}{1.601509in}}%
\pgfpathlineto{\pgfqpoint{2.107502in}{1.552140in}}%
\pgfpathlineto{\pgfqpoint{2.107095in}{1.617965in}}%
\pgfpathlineto{\pgfqpoint{2.107706in}{1.576824in}}%
\pgfpathlineto{\pgfqpoint{2.107910in}{1.634422in}}%
\pgfpathlineto{\pgfqpoint{2.108318in}{1.552140in}}%
\pgfpathlineto{\pgfqpoint{2.108725in}{1.601509in}}%
\pgfpathlineto{\pgfqpoint{2.108929in}{1.593281in}}%
\pgfpathlineto{\pgfqpoint{2.109337in}{1.609737in}}%
\pgfpathlineto{\pgfqpoint{2.109948in}{1.634422in}}%
\pgfpathlineto{\pgfqpoint{2.110152in}{1.585053in}}%
\pgfpathlineto{\pgfqpoint{2.110560in}{1.626194in}}%
\pgfpathlineto{\pgfqpoint{2.110763in}{1.659106in}}%
\pgfpathlineto{\pgfqpoint{2.111171in}{1.576824in}}%
\pgfpathlineto{\pgfqpoint{2.111375in}{1.585053in}}%
\pgfpathlineto{\pgfqpoint{2.112394in}{1.510999in}}%
\pgfpathlineto{\pgfqpoint{2.111986in}{1.609737in}}%
\pgfpathlineto{\pgfqpoint{2.112598in}{1.543912in}}%
\pgfpathlineto{\pgfqpoint{2.113005in}{1.510999in}}%
\pgfpathlineto{\pgfqpoint{2.113209in}{1.519227in}}%
\pgfpathlineto{\pgfqpoint{2.114228in}{1.510999in}}%
\pgfpathlineto{\pgfqpoint{2.114432in}{1.585053in}}%
\pgfpathlineto{\pgfqpoint{2.115043in}{1.486314in}}%
\pgfpathlineto{\pgfqpoint{2.115451in}{1.576824in}}%
\pgfpathlineto{\pgfqpoint{2.115858in}{1.568596in}}%
\pgfpathlineto{\pgfqpoint{2.116062in}{1.617965in}}%
\pgfpathlineto{\pgfqpoint{2.116470in}{1.510999in}}%
\pgfpathlineto{\pgfqpoint{2.117081in}{1.428717in}}%
\pgfpathlineto{\pgfqpoint{2.117489in}{1.486314in}}%
\pgfpathlineto{\pgfqpoint{2.118304in}{1.543912in}}%
\pgfpathlineto{\pgfqpoint{2.117896in}{1.453401in}}%
\pgfpathlineto{\pgfqpoint{2.118712in}{1.502771in}}%
\pgfpathlineto{\pgfqpoint{2.118915in}{1.494542in}}%
\pgfpathlineto{\pgfqpoint{2.119323in}{1.552140in}}%
\pgfpathlineto{\pgfqpoint{2.119731in}{1.445173in}}%
\pgfpathlineto{\pgfqpoint{2.119934in}{1.486314in}}%
\pgfpathlineto{\pgfqpoint{2.120138in}{1.469858in}}%
\pgfpathlineto{\pgfqpoint{2.120342in}{1.502771in}}%
\pgfpathlineto{\pgfqpoint{2.121769in}{1.650878in}}%
\pgfpathlineto{\pgfqpoint{2.123195in}{1.494542in}}%
\pgfpathlineto{\pgfqpoint{2.124622in}{1.659106in}}%
\pgfpathlineto{\pgfqpoint{2.125233in}{1.650878in}}%
\pgfpathlineto{\pgfqpoint{2.126049in}{1.667335in}}%
\pgfpathlineto{\pgfqpoint{2.126252in}{1.626194in}}%
\pgfpathlineto{\pgfqpoint{2.126864in}{1.733160in}}%
\pgfpathlineto{\pgfqpoint{2.127271in}{1.650878in}}%
\pgfpathlineto{\pgfqpoint{2.127475in}{1.642650in}}%
\pgfpathlineto{\pgfqpoint{2.127883in}{1.716704in}}%
\pgfpathlineto{\pgfqpoint{2.128087in}{1.634422in}}%
\pgfpathlineto{\pgfqpoint{2.128494in}{1.692019in}}%
\pgfpathlineto{\pgfqpoint{2.128902in}{1.527455in}}%
\pgfpathlineto{\pgfqpoint{2.129921in}{1.552140in}}%
\pgfpathlineto{\pgfqpoint{2.130125in}{1.626194in}}%
\pgfpathlineto{\pgfqpoint{2.130940in}{1.552140in}}%
\pgfpathlineto{\pgfqpoint{2.131551in}{1.576824in}}%
\pgfpathlineto{\pgfqpoint{2.131347in}{1.543912in}}%
\pgfpathlineto{\pgfqpoint{2.131755in}{1.552140in}}%
\pgfpathlineto{\pgfqpoint{2.131959in}{1.510999in}}%
\pgfpathlineto{\pgfqpoint{2.132366in}{1.585053in}}%
\pgfpathlineto{\pgfqpoint{2.132570in}{1.617965in}}%
\pgfpathlineto{\pgfqpoint{2.132774in}{1.502771in}}%
\pgfpathlineto{\pgfqpoint{2.132978in}{1.519227in}}%
\pgfpathlineto{\pgfqpoint{2.133182in}{1.519227in}}%
\pgfpathlineto{\pgfqpoint{2.133385in}{1.510999in}}%
\pgfpathlineto{\pgfqpoint{2.133589in}{1.527455in}}%
\pgfpathlineto{\pgfqpoint{2.133793in}{1.527455in}}%
\pgfpathlineto{\pgfqpoint{2.134201in}{1.560368in}}%
\pgfpathlineto{\pgfqpoint{2.134404in}{1.527455in}}%
\pgfpathlineto{\pgfqpoint{2.135016in}{1.469858in}}%
\pgfpathlineto{\pgfqpoint{2.135220in}{1.543912in}}%
\pgfpathlineto{\pgfqpoint{2.135423in}{1.617965in}}%
\pgfpathlineto{\pgfqpoint{2.136239in}{1.527455in}}%
\pgfpathlineto{\pgfqpoint{2.137054in}{1.543912in}}%
\pgfpathlineto{\pgfqpoint{2.137665in}{1.420489in}}%
\pgfpathlineto{\pgfqpoint{2.138888in}{1.502771in}}%
\pgfpathlineto{\pgfqpoint{2.139092in}{1.436945in}}%
\pgfpathlineto{\pgfqpoint{2.139500in}{1.543912in}}%
\pgfpathlineto{\pgfqpoint{2.139907in}{1.494542in}}%
\pgfpathlineto{\pgfqpoint{2.140111in}{1.543912in}}%
\pgfpathlineto{\pgfqpoint{2.140315in}{1.321750in}}%
\pgfpathlineto{\pgfqpoint{2.140519in}{1.568596in}}%
\pgfpathlineto{\pgfqpoint{2.141130in}{1.486314in}}%
\pgfpathlineto{\pgfqpoint{2.141334in}{1.478086in}}%
\pgfpathlineto{\pgfqpoint{2.141538in}{1.527455in}}%
\pgfpathlineto{\pgfqpoint{2.141945in}{1.461630in}}%
\pgfpathlineto{\pgfqpoint{2.142149in}{1.486314in}}%
\pgfpathlineto{\pgfqpoint{2.142557in}{1.354663in}}%
\pgfpathlineto{\pgfqpoint{2.143168in}{1.387576in}}%
\pgfpathlineto{\pgfqpoint{2.144187in}{1.519227in}}%
\pgfpathlineto{\pgfqpoint{2.143576in}{1.362891in}}%
\pgfpathlineto{\pgfqpoint{2.144391in}{1.510999in}}%
\pgfpathlineto{\pgfqpoint{2.144798in}{1.165414in}}%
\pgfpathlineto{\pgfqpoint{2.145410in}{1.214784in}}%
\pgfpathlineto{\pgfqpoint{2.146429in}{1.478086in}}%
\pgfpathlineto{\pgfqpoint{2.146633in}{1.412260in}}%
\pgfpathlineto{\pgfqpoint{2.147652in}{1.510999in}}%
\pgfpathlineto{\pgfqpoint{2.148059in}{1.486314in}}%
\pgfpathlineto{\pgfqpoint{2.148874in}{1.453401in}}%
\pgfpathlineto{\pgfqpoint{2.149078in}{1.510999in}}%
\pgfpathlineto{\pgfqpoint{2.149893in}{1.494542in}}%
\pgfpathlineto{\pgfqpoint{2.150709in}{1.404032in}}%
\pgfpathlineto{\pgfqpoint{2.150913in}{1.469858in}}%
\pgfpathlineto{\pgfqpoint{2.151320in}{1.568596in}}%
\pgfpathlineto{\pgfqpoint{2.151932in}{1.469858in}}%
\pgfpathlineto{\pgfqpoint{2.152135in}{1.239468in}}%
\pgfpathlineto{\pgfqpoint{2.152747in}{1.568596in}}%
\pgfpathlineto{\pgfqpoint{2.152951in}{1.527455in}}%
\pgfpathlineto{\pgfqpoint{2.153154in}{1.527455in}}%
\pgfpathlineto{\pgfqpoint{2.153358in}{1.510999in}}%
\pgfpathlineto{\pgfqpoint{2.153562in}{1.560368in}}%
\pgfpathlineto{\pgfqpoint{2.153766in}{1.535683in}}%
\pgfpathlineto{\pgfqpoint{2.154785in}{1.650878in}}%
\pgfpathlineto{\pgfqpoint{2.155192in}{1.609737in}}%
\pgfpathlineto{\pgfqpoint{2.156211in}{1.527455in}}%
\pgfpathlineto{\pgfqpoint{2.156415in}{1.552140in}}%
\pgfpathlineto{\pgfqpoint{2.156619in}{1.560368in}}%
\pgfpathlineto{\pgfqpoint{2.157638in}{1.774301in}}%
\pgfpathlineto{\pgfqpoint{2.157842in}{1.700247in}}%
\pgfpathlineto{\pgfqpoint{2.158046in}{1.445173in}}%
\pgfpathlineto{\pgfqpoint{2.159065in}{1.560368in}}%
\pgfpathlineto{\pgfqpoint{2.159880in}{1.576824in}}%
\pgfpathlineto{\pgfqpoint{2.160287in}{1.478086in}}%
\pgfpathlineto{\pgfqpoint{2.160491in}{1.486314in}}%
\pgfpathlineto{\pgfqpoint{2.160695in}{1.552140in}}%
\pgfpathlineto{\pgfqpoint{2.160899in}{1.469858in}}%
\pgfpathlineto{\pgfqpoint{2.161510in}{1.527455in}}%
\pgfpathlineto{\pgfqpoint{2.162325in}{1.428717in}}%
\pgfpathlineto{\pgfqpoint{2.162733in}{1.469858in}}%
\pgfpathlineto{\pgfqpoint{2.163141in}{1.510999in}}%
\pgfpathlineto{\pgfqpoint{2.163344in}{1.461630in}}%
\pgfpathlineto{\pgfqpoint{2.163752in}{1.568596in}}%
\pgfpathlineto{\pgfqpoint{2.164364in}{1.478086in}}%
\pgfpathlineto{\pgfqpoint{2.165179in}{1.379348in}}%
\pgfpathlineto{\pgfqpoint{2.165586in}{1.412260in}}%
\pgfpathlineto{\pgfqpoint{2.166198in}{1.461630in}}%
\pgfpathlineto{\pgfqpoint{2.167421in}{1.321750in}}%
\pgfpathlineto{\pgfqpoint{2.168440in}{1.395804in}}%
\pgfpathlineto{\pgfqpoint{2.167828in}{1.288837in}}%
\pgfpathlineto{\pgfqpoint{2.168847in}{1.387576in}}%
\pgfpathlineto{\pgfqpoint{2.169459in}{1.173643in}}%
\pgfpathlineto{\pgfqpoint{2.169662in}{1.395804in}}%
\pgfpathlineto{\pgfqpoint{2.170681in}{1.543912in}}%
\pgfpathlineto{\pgfqpoint{2.171089in}{1.502771in}}%
\pgfpathlineto{\pgfqpoint{2.171293in}{1.288837in}}%
\pgfpathlineto{\pgfqpoint{2.171904in}{1.568596in}}%
\pgfpathlineto{\pgfqpoint{2.172108in}{1.543912in}}%
\pgfpathlineto{\pgfqpoint{2.173127in}{1.478086in}}%
\pgfpathlineto{\pgfqpoint{2.173331in}{1.486314in}}%
\pgfpathlineto{\pgfqpoint{2.173535in}{1.519227in}}%
\pgfpathlineto{\pgfqpoint{2.174146in}{1.593281in}}%
\pgfpathlineto{\pgfqpoint{2.174554in}{1.329978in}}%
\pgfpathlineto{\pgfqpoint{2.174961in}{1.560368in}}%
\pgfpathlineto{\pgfqpoint{2.175776in}{1.478086in}}%
\pgfpathlineto{\pgfqpoint{2.176592in}{1.420489in}}%
\pgfpathlineto{\pgfqpoint{2.176795in}{1.478086in}}%
\pgfpathlineto{\pgfqpoint{2.177611in}{1.404032in}}%
\pgfpathlineto{\pgfqpoint{2.178222in}{1.132502in}}%
\pgfpathlineto{\pgfqpoint{2.178426in}{1.428717in}}%
\pgfpathlineto{\pgfqpoint{2.179241in}{1.494542in}}%
\pgfpathlineto{\pgfqpoint{2.179445in}{1.478086in}}%
\pgfpathlineto{\pgfqpoint{2.181687in}{1.140730in}}%
\pgfpathlineto{\pgfqpoint{2.181891in}{0.885656in}}%
\pgfpathlineto{\pgfqpoint{2.182298in}{1.157186in}}%
\pgfpathlineto{\pgfqpoint{2.182706in}{1.099589in}}%
\pgfpathlineto{\pgfqpoint{2.183521in}{1.173643in}}%
\pgfpathlineto{\pgfqpoint{2.183113in}{1.091361in}}%
\pgfpathlineto{\pgfqpoint{2.183725in}{1.157186in}}%
\pgfpathlineto{\pgfqpoint{2.184540in}{0.869199in}}%
\pgfpathlineto{\pgfqpoint{2.184744in}{1.041991in}}%
\pgfpathlineto{\pgfqpoint{2.184948in}{1.165414in}}%
\pgfpathlineto{\pgfqpoint{2.185763in}{1.041991in}}%
\pgfpathlineto{\pgfqpoint{2.186170in}{1.099589in}}%
\pgfpathlineto{\pgfqpoint{2.186578in}{1.017307in}}%
\pgfpathlineto{\pgfqpoint{2.186782in}{1.058448in}}%
\pgfpathlineto{\pgfqpoint{2.186986in}{0.754004in}}%
\pgfpathlineto{\pgfqpoint{2.187393in}{1.066676in}}%
\pgfpathlineto{\pgfqpoint{2.187801in}{0.836286in}}%
\pgfpathlineto{\pgfqpoint{2.188412in}{1.025535in}}%
\pgfpathlineto{\pgfqpoint{2.189024in}{0.943253in}}%
\pgfpathlineto{\pgfqpoint{2.189227in}{0.688179in}}%
\pgfpathlineto{\pgfqpoint{2.189635in}{1.009079in}}%
\pgfpathlineto{\pgfqpoint{2.190043in}{0.918568in}}%
\pgfpathlineto{\pgfqpoint{2.190246in}{0.852743in}}%
\pgfpathlineto{\pgfqpoint{2.190858in}{0.967938in}}%
\pgfpathlineto{\pgfqpoint{2.191062in}{1.058448in}}%
\pgfpathlineto{\pgfqpoint{2.191266in}{0.935025in}}%
\pgfpathlineto{\pgfqpoint{2.191469in}{0.721091in}}%
\pgfpathlineto{\pgfqpoint{2.192285in}{0.885656in}}%
\pgfpathlineto{\pgfqpoint{2.193711in}{0.992622in}}%
\pgfpathlineto{\pgfqpoint{2.192692in}{0.852743in}}%
\pgfpathlineto{\pgfqpoint{2.193915in}{0.951481in}}%
\pgfpathlineto{\pgfqpoint{2.194323in}{0.688179in}}%
\pgfpathlineto{\pgfqpoint{2.195138in}{0.860971in}}%
\pgfpathlineto{\pgfqpoint{2.196157in}{1.083132in}}%
\pgfpathlineto{\pgfqpoint{2.196564in}{1.009079in}}%
\pgfpathlineto{\pgfqpoint{2.196768in}{0.910340in}}%
\pgfpathlineto{\pgfqpoint{2.197583in}{1.000850in}}%
\pgfpathlineto{\pgfqpoint{2.199010in}{1.148958in}}%
\pgfpathlineto{\pgfqpoint{2.199418in}{1.181871in}}%
\pgfpathlineto{\pgfqpoint{2.199825in}{1.107817in}}%
\pgfpathlineto{\pgfqpoint{2.200029in}{1.165414in}}%
\pgfpathlineto{\pgfqpoint{2.200640in}{1.066676in}}%
\pgfpathlineto{\pgfqpoint{2.201048in}{1.000850in}}%
\pgfpathlineto{\pgfqpoint{2.201456in}{1.066676in}}%
\pgfpathlineto{\pgfqpoint{2.202067in}{1.148958in}}%
\pgfpathlineto{\pgfqpoint{2.202271in}{1.041991in}}%
\pgfpathlineto{\pgfqpoint{2.202678in}{1.058448in}}%
\pgfpathlineto{\pgfqpoint{2.203086in}{0.910340in}}%
\pgfpathlineto{\pgfqpoint{2.203290in}{0.696407in}}%
\pgfpathlineto{\pgfqpoint{2.203697in}{1.025535in}}%
\pgfpathlineto{\pgfqpoint{2.204105in}{0.926797in}}%
\pgfpathlineto{\pgfqpoint{2.204717in}{0.967938in}}%
\pgfpathlineto{\pgfqpoint{2.204920in}{0.951481in}}%
\pgfpathlineto{\pgfqpoint{2.205328in}{0.803374in}}%
\pgfpathlineto{\pgfqpoint{2.205939in}{0.893884in}}%
\pgfpathlineto{\pgfqpoint{2.207366in}{1.083132in}}%
\pgfpathlineto{\pgfqpoint{2.207570in}{1.050220in}}%
\pgfpathlineto{\pgfqpoint{2.207977in}{1.116045in}}%
\pgfpathlineto{\pgfqpoint{2.208181in}{1.074904in}}%
\pgfpathlineto{\pgfqpoint{2.208589in}{1.132502in}}%
\pgfpathlineto{\pgfqpoint{2.208996in}{1.009079in}}%
\pgfpathlineto{\pgfqpoint{2.209404in}{1.041991in}}%
\pgfpathlineto{\pgfqpoint{2.209812in}{1.017307in}}%
\pgfpathlineto{\pgfqpoint{2.210015in}{0.984394in}}%
\pgfpathlineto{\pgfqpoint{2.210627in}{1.058448in}}%
\pgfpathlineto{\pgfqpoint{2.210831in}{1.025535in}}%
\pgfpathlineto{\pgfqpoint{2.212257in}{1.198327in}}%
\pgfpathlineto{\pgfqpoint{2.212461in}{1.280609in}}%
\pgfpathlineto{\pgfqpoint{2.213072in}{1.157186in}}%
\pgfpathlineto{\pgfqpoint{2.213276in}{1.231240in}}%
\pgfpathlineto{\pgfqpoint{2.213480in}{1.181871in}}%
\pgfpathlineto{\pgfqpoint{2.214295in}{1.223012in}}%
\pgfpathlineto{\pgfqpoint{2.214499in}{1.239468in}}%
\pgfpathlineto{\pgfqpoint{2.214907in}{1.190099in}}%
\pgfpathlineto{\pgfqpoint{2.215110in}{1.190099in}}%
\pgfpathlineto{\pgfqpoint{2.215314in}{1.181871in}}%
\pgfpathlineto{\pgfqpoint{2.215722in}{1.305294in}}%
\pgfpathlineto{\pgfqpoint{2.216129in}{1.247696in}}%
\pgfpathlineto{\pgfqpoint{2.216333in}{1.148958in}}%
\pgfpathlineto{\pgfqpoint{2.216537in}{1.338207in}}%
\pgfpathlineto{\pgfqpoint{2.217148in}{1.288837in}}%
\pgfpathlineto{\pgfqpoint{2.217760in}{1.371119in}}%
\pgfpathlineto{\pgfqpoint{2.218168in}{1.206555in}}%
\pgfpathlineto{\pgfqpoint{2.218371in}{1.527455in}}%
\pgfpathlineto{\pgfqpoint{2.218779in}{1.223012in}}%
\pgfpathlineto{\pgfqpoint{2.218983in}{1.313522in}}%
\pgfpathlineto{\pgfqpoint{2.219594in}{1.173643in}}%
\pgfpathlineto{\pgfqpoint{2.219798in}{1.190099in}}%
\pgfpathlineto{\pgfqpoint{2.220409in}{1.074904in}}%
\pgfpathlineto{\pgfqpoint{2.221021in}{1.157186in}}%
\pgfpathlineto{\pgfqpoint{2.221836in}{1.066676in}}%
\pgfpathlineto{\pgfqpoint{2.222040in}{1.116045in}}%
\pgfpathlineto{\pgfqpoint{2.223466in}{1.223012in}}%
\pgfpathlineto{\pgfqpoint{2.223670in}{1.206555in}}%
\pgfpathlineto{\pgfqpoint{2.224485in}{1.264153in}}%
\pgfpathlineto{\pgfqpoint{2.224282in}{1.198327in}}%
\pgfpathlineto{\pgfqpoint{2.225097in}{1.247696in}}%
\pgfpathlineto{\pgfqpoint{2.225301in}{1.223012in}}%
\pgfpathlineto{\pgfqpoint{2.225708in}{1.280609in}}%
\pgfpathlineto{\pgfqpoint{2.225912in}{1.255925in}}%
\pgfpathlineto{\pgfqpoint{2.226116in}{1.346435in}}%
\pgfpathlineto{\pgfqpoint{2.226523in}{1.223012in}}%
\pgfpathlineto{\pgfqpoint{2.226931in}{1.272381in}}%
\pgfpathlineto{\pgfqpoint{2.228561in}{1.379348in}}%
\pgfpathlineto{\pgfqpoint{2.228969in}{1.329978in}}%
\pgfpathlineto{\pgfqpoint{2.229173in}{1.288837in}}%
\pgfpathlineto{\pgfqpoint{2.229580in}{1.412260in}}%
\pgfpathlineto{\pgfqpoint{2.230396in}{1.478086in}}%
\pgfpathlineto{\pgfqpoint{2.231211in}{1.552140in}}%
\pgfpathlineto{\pgfqpoint{2.231007in}{1.461630in}}%
\pgfpathlineto{\pgfqpoint{2.231415in}{1.543912in}}%
\pgfpathlineto{\pgfqpoint{2.232026in}{1.469858in}}%
\pgfpathlineto{\pgfqpoint{2.232434in}{1.519227in}}%
\pgfpathlineto{\pgfqpoint{2.233657in}{1.601509in}}%
\pgfpathlineto{\pgfqpoint{2.233860in}{1.560368in}}%
\pgfpathlineto{\pgfqpoint{2.234268in}{1.650878in}}%
\pgfpathlineto{\pgfqpoint{2.234472in}{1.626194in}}%
\pgfpathlineto{\pgfqpoint{2.234879in}{1.659106in}}%
\pgfpathlineto{\pgfqpoint{2.235083in}{1.478086in}}%
\pgfpathlineto{\pgfqpoint{2.235898in}{1.733160in}}%
\pgfpathlineto{\pgfqpoint{2.236102in}{1.708476in}}%
\pgfpathlineto{\pgfqpoint{2.236306in}{1.766073in}}%
\pgfpathlineto{\pgfqpoint{2.237325in}{1.864811in}}%
\pgfpathlineto{\pgfqpoint{2.237121in}{1.700247in}}%
\pgfpathlineto{\pgfqpoint{2.237529in}{1.848355in}}%
\pgfpathlineto{\pgfqpoint{2.238140in}{1.856583in}}%
\pgfpathlineto{\pgfqpoint{2.238548in}{1.798986in}}%
\pgfpathlineto{\pgfqpoint{2.239771in}{1.856583in}}%
\pgfpathlineto{\pgfqpoint{2.239974in}{1.848355in}}%
\pgfpathlineto{\pgfqpoint{2.240382in}{1.840127in}}%
\pgfpathlineto{\pgfqpoint{2.240993in}{1.889496in}}%
\pgfpathlineto{\pgfqpoint{2.241197in}{1.848355in}}%
\pgfpathlineto{\pgfqpoint{2.241605in}{1.914181in}}%
\pgfpathlineto{\pgfqpoint{2.242012in}{1.889496in}}%
\pgfpathlineto{\pgfqpoint{2.243847in}{2.037604in}}%
\pgfpathlineto{\pgfqpoint{2.244050in}{2.054060in}}%
\pgfpathlineto{\pgfqpoint{2.244866in}{2.062288in}}%
\pgfpathlineto{\pgfqpoint{2.245273in}{1.873040in}}%
\pgfpathlineto{\pgfqpoint{2.246089in}{2.062288in}}%
\pgfpathlineto{\pgfqpoint{2.246292in}{1.807214in}}%
\pgfpathlineto{\pgfqpoint{2.246700in}{2.111657in}}%
\pgfpathlineto{\pgfqpoint{2.247108in}{2.095201in}}%
\pgfpathlineto{\pgfqpoint{2.247515in}{2.045832in}}%
\pgfpathlineto{\pgfqpoint{2.247719in}{2.012919in}}%
\pgfpathlineto{\pgfqpoint{2.248127in}{2.054060in}}%
\pgfpathlineto{\pgfqpoint{2.248942in}{2.152798in}}%
\pgfpathlineto{\pgfqpoint{2.249349in}{2.086973in}}%
\pgfpathlineto{\pgfqpoint{2.250368in}{2.169255in}}%
\pgfpathlineto{\pgfqpoint{2.250980in}{2.152798in}}%
\pgfpathlineto{\pgfqpoint{2.251184in}{2.210396in}}%
\pgfpathlineto{\pgfqpoint{2.251999in}{2.128114in}}%
\pgfpathlineto{\pgfqpoint{2.252406in}{2.086973in}}%
\pgfpathlineto{\pgfqpoint{2.252814in}{2.161027in}}%
\pgfpathlineto{\pgfqpoint{2.253018in}{2.136342in}}%
\pgfpathlineto{\pgfqpoint{2.253833in}{2.202168in}}%
\pgfpathlineto{\pgfqpoint{2.253629in}{2.119886in}}%
\pgfpathlineto{\pgfqpoint{2.254444in}{2.177483in}}%
\pgfpathlineto{\pgfqpoint{2.255260in}{2.218624in}}%
\pgfpathlineto{\pgfqpoint{2.255667in}{2.103429in}}%
\pgfpathlineto{\pgfqpoint{2.255871in}{2.103429in}}%
\pgfpathlineto{\pgfqpoint{2.256075in}{1.922409in}}%
\pgfpathlineto{\pgfqpoint{2.256482in}{2.136342in}}%
\pgfpathlineto{\pgfqpoint{2.256890in}{2.086973in}}%
\pgfpathlineto{\pgfqpoint{2.257501in}{2.144570in}}%
\pgfpathlineto{\pgfqpoint{2.257705in}{2.054060in}}%
\pgfpathlineto{\pgfqpoint{2.257909in}{2.095201in}}%
\pgfpathlineto{\pgfqpoint{2.258724in}{2.152798in}}%
\pgfpathlineto{\pgfqpoint{2.258317in}{2.086973in}}%
\pgfpathlineto{\pgfqpoint{2.258928in}{2.144570in}}%
\pgfpathlineto{\pgfqpoint{2.259540in}{1.889496in}}%
\pgfpathlineto{\pgfqpoint{2.259947in}{2.095201in}}%
\pgfpathlineto{\pgfqpoint{2.260151in}{2.078745in}}%
\pgfpathlineto{\pgfqpoint{2.260355in}{2.136342in}}%
\pgfpathlineto{\pgfqpoint{2.260762in}{2.103429in}}%
\pgfpathlineto{\pgfqpoint{2.261170in}{2.128114in}}%
\pgfpathlineto{\pgfqpoint{2.261374in}{2.086973in}}%
\pgfpathlineto{\pgfqpoint{2.262393in}{2.004691in}}%
\pgfpathlineto{\pgfqpoint{2.262597in}{2.012919in}}%
\pgfpathlineto{\pgfqpoint{2.264431in}{2.103429in}}%
\pgfpathlineto{\pgfqpoint{2.265246in}{2.037604in}}%
\pgfpathlineto{\pgfqpoint{2.265654in}{2.045832in}}%
\pgfpathlineto{\pgfqpoint{2.265857in}{2.054060in}}%
\pgfpathlineto{\pgfqpoint{2.266061in}{2.029375in}}%
\pgfpathlineto{\pgfqpoint{2.266469in}{1.963550in}}%
\pgfpathlineto{\pgfqpoint{2.267080in}{2.029375in}}%
\pgfpathlineto{\pgfqpoint{2.267488in}{2.029375in}}%
\pgfpathlineto{\pgfqpoint{2.267895in}{2.111657in}}%
\pgfpathlineto{\pgfqpoint{2.268507in}{2.054060in}}%
\pgfpathlineto{\pgfqpoint{2.269118in}{2.021147in}}%
\pgfpathlineto{\pgfqpoint{2.269526in}{2.029375in}}%
\pgfpathlineto{\pgfqpoint{2.270749in}{2.095201in}}%
\pgfpathlineto{\pgfqpoint{2.270952in}{2.095201in}}%
\pgfpathlineto{\pgfqpoint{2.271156in}{2.111657in}}%
\pgfpathlineto{\pgfqpoint{2.271564in}{2.062288in}}%
\pgfpathlineto{\pgfqpoint{2.271768in}{2.054060in}}%
\pgfpathlineto{\pgfqpoint{2.271972in}{2.119886in}}%
\pgfpathlineto{\pgfqpoint{2.272583in}{2.045832in}}%
\pgfpathlineto{\pgfqpoint{2.272787in}{2.054060in}}%
\pgfpathlineto{\pgfqpoint{2.272991in}{2.062288in}}%
\pgfpathlineto{\pgfqpoint{2.273194in}{2.037604in}}%
\pgfpathlineto{\pgfqpoint{2.273398in}{2.029375in}}%
\pgfpathlineto{\pgfqpoint{2.273602in}{2.037604in}}%
\pgfpathlineto{\pgfqpoint{2.273806in}{2.086973in}}%
\pgfpathlineto{\pgfqpoint{2.274213in}{2.021147in}}%
\pgfpathlineto{\pgfqpoint{2.274417in}{2.021147in}}%
\pgfpathlineto{\pgfqpoint{2.274621in}{1.963550in}}%
\pgfpathlineto{\pgfqpoint{2.275640in}{1.988234in}}%
\pgfpathlineto{\pgfqpoint{2.275844in}{1.996463in}}%
\pgfpathlineto{\pgfqpoint{2.276251in}{1.889496in}}%
\pgfpathlineto{\pgfqpoint{2.276659in}{2.021147in}}%
\pgfpathlineto{\pgfqpoint{2.276863in}{1.980006in}}%
\pgfpathlineto{\pgfqpoint{2.277678in}{2.111657in}}%
\pgfpathlineto{\pgfqpoint{2.278086in}{2.086973in}}%
\pgfpathlineto{\pgfqpoint{2.278493in}{2.062288in}}%
\pgfpathlineto{\pgfqpoint{2.278901in}{2.095201in}}%
\pgfpathlineto{\pgfqpoint{2.279105in}{2.086973in}}%
\pgfpathlineto{\pgfqpoint{2.279308in}{2.095201in}}%
\pgfpathlineto{\pgfqpoint{2.279512in}{2.054060in}}%
\pgfpathlineto{\pgfqpoint{2.280124in}{2.144570in}}%
\pgfpathlineto{\pgfqpoint{2.280531in}{2.070516in}}%
\pgfpathlineto{\pgfqpoint{2.281143in}{2.037604in}}%
\pgfpathlineto{\pgfqpoint{2.281958in}{2.161027in}}%
\pgfpathlineto{\pgfqpoint{2.282365in}{2.119886in}}%
\pgfpathlineto{\pgfqpoint{2.282977in}{2.169255in}}%
\pgfpathlineto{\pgfqpoint{2.283384in}{2.062288in}}%
\pgfpathlineto{\pgfqpoint{2.284200in}{2.161027in}}%
\pgfpathlineto{\pgfqpoint{2.283996in}{2.012919in}}%
\pgfpathlineto{\pgfqpoint{2.284607in}{2.128114in}}%
\pgfpathlineto{\pgfqpoint{2.284811in}{2.136342in}}%
\pgfpathlineto{\pgfqpoint{2.286238in}{2.054060in}}%
\pgfpathlineto{\pgfqpoint{2.286442in}{2.078745in}}%
\pgfpathlineto{\pgfqpoint{2.286849in}{2.004691in}}%
\pgfpathlineto{\pgfqpoint{2.287257in}{2.070516in}}%
\pgfpathlineto{\pgfqpoint{2.287664in}{2.029375in}}%
\pgfpathlineto{\pgfqpoint{2.288072in}{2.078745in}}%
\pgfpathlineto{\pgfqpoint{2.288276in}{2.078745in}}%
\pgfpathlineto{\pgfqpoint{2.289906in}{2.235080in}}%
\pgfpathlineto{\pgfqpoint{2.290110in}{2.235080in}}%
\pgfpathlineto{\pgfqpoint{2.290721in}{2.004691in}}%
\pgfpathlineto{\pgfqpoint{2.291333in}{2.111657in}}%
\pgfpathlineto{\pgfqpoint{2.291537in}{1.996463in}}%
\pgfpathlineto{\pgfqpoint{2.291740in}{2.161027in}}%
\pgfpathlineto{\pgfqpoint{2.292352in}{2.136342in}}%
\pgfpathlineto{\pgfqpoint{2.293167in}{2.119886in}}%
\pgfpathlineto{\pgfqpoint{2.293575in}{2.177483in}}%
\pgfpathlineto{\pgfqpoint{2.293982in}{2.169255in}}%
\pgfpathlineto{\pgfqpoint{2.294390in}{1.873040in}}%
\pgfpathlineto{\pgfqpoint{2.294797in}{2.202168in}}%
\pgfpathlineto{\pgfqpoint{2.295001in}{2.177483in}}%
\pgfpathlineto{\pgfqpoint{2.295613in}{2.177483in}}%
\pgfpathlineto{\pgfqpoint{2.296020in}{2.136342in}}%
\pgfpathlineto{\pgfqpoint{2.296835in}{2.161027in}}%
\pgfpathlineto{\pgfqpoint{2.297243in}{2.136342in}}%
\pgfpathlineto{\pgfqpoint{2.298262in}{2.243309in}}%
\pgfpathlineto{\pgfqpoint{2.298874in}{2.037604in}}%
\pgfpathlineto{\pgfqpoint{2.298670in}{2.251537in}}%
\pgfpathlineto{\pgfqpoint{2.299485in}{2.144570in}}%
\pgfpathlineto{\pgfqpoint{2.300708in}{2.276221in}}%
\pgfpathlineto{\pgfqpoint{2.301319in}{2.202168in}}%
\pgfpathlineto{\pgfqpoint{2.301727in}{2.259765in}}%
\pgfpathlineto{\pgfqpoint{2.301931in}{2.259765in}}%
\pgfpathlineto{\pgfqpoint{2.302134in}{2.210396in}}%
\pgfpathlineto{\pgfqpoint{2.302746in}{2.292678in}}%
\pgfpathlineto{\pgfqpoint{2.302950in}{2.292678in}}%
\pgfpathlineto{\pgfqpoint{2.303357in}{2.350275in}}%
\pgfpathlineto{\pgfqpoint{2.303969in}{2.300906in}}%
\pgfpathlineto{\pgfqpoint{2.305599in}{2.012919in}}%
\pgfpathlineto{\pgfqpoint{2.305803in}{2.062288in}}%
\pgfpathlineto{\pgfqpoint{2.306210in}{2.267993in}}%
\pgfpathlineto{\pgfqpoint{2.307026in}{2.243309in}}%
\pgfpathlineto{\pgfqpoint{2.307229in}{2.193939in}}%
\pgfpathlineto{\pgfqpoint{2.307841in}{2.284450in}}%
\pgfpathlineto{\pgfqpoint{2.308656in}{2.251537in}}%
\pgfpathlineto{\pgfqpoint{2.308860in}{2.325591in}}%
\pgfpathlineto{\pgfqpoint{2.309471in}{2.193939in}}%
\pgfpathlineto{\pgfqpoint{2.309675in}{2.235080in}}%
\pgfpathlineto{\pgfqpoint{2.310490in}{2.169255in}}%
\pgfpathlineto{\pgfqpoint{2.310694in}{2.210396in}}%
\pgfpathlineto{\pgfqpoint{2.311305in}{2.193939in}}%
\pgfpathlineto{\pgfqpoint{2.312121in}{2.300906in}}%
\pgfpathlineto{\pgfqpoint{2.312732in}{1.988234in}}%
\pgfpathlineto{\pgfqpoint{2.312936in}{2.325591in}}%
\pgfpathlineto{\pgfqpoint{2.313140in}{2.284450in}}%
\pgfpathlineto{\pgfqpoint{2.313547in}{2.317363in}}%
\pgfpathlineto{\pgfqpoint{2.313751in}{2.276221in}}%
\pgfpathlineto{\pgfqpoint{2.313955in}{2.210396in}}%
\pgfpathlineto{\pgfqpoint{2.314566in}{2.284450in}}%
\pgfpathlineto{\pgfqpoint{2.314770in}{2.267993in}}%
\pgfpathlineto{\pgfqpoint{2.315382in}{2.309134in}}%
\pgfpathlineto{\pgfqpoint{2.316197in}{2.119886in}}%
\pgfpathlineto{\pgfqpoint{2.316604in}{2.185711in}}%
\pgfpathlineto{\pgfqpoint{2.316808in}{2.185711in}}%
\pgfpathlineto{\pgfqpoint{2.317012in}{2.218624in}}%
\pgfpathlineto{\pgfqpoint{2.317420in}{2.144570in}}%
\pgfpathlineto{\pgfqpoint{2.317623in}{2.177483in}}%
\pgfpathlineto{\pgfqpoint{2.317827in}{2.161027in}}%
\pgfpathlineto{\pgfqpoint{2.318031in}{2.210396in}}%
\pgfpathlineto{\pgfqpoint{2.318439in}{2.185711in}}%
\pgfpathlineto{\pgfqpoint{2.318642in}{2.210396in}}%
\pgfpathlineto{\pgfqpoint{2.319050in}{2.202168in}}%
\pgfpathlineto{\pgfqpoint{2.319254in}{1.996463in}}%
\pgfpathlineto{\pgfqpoint{2.320069in}{2.243309in}}%
\pgfpathlineto{\pgfqpoint{2.320477in}{2.012919in}}%
\pgfpathlineto{\pgfqpoint{2.321088in}{2.226852in}}%
\pgfpathlineto{\pgfqpoint{2.322107in}{2.152798in}}%
\pgfpathlineto{\pgfqpoint{2.322311in}{2.161027in}}%
\pgfpathlineto{\pgfqpoint{2.322515in}{2.119886in}}%
\pgfpathlineto{\pgfqpoint{2.322922in}{2.210396in}}%
\pgfpathlineto{\pgfqpoint{2.323126in}{2.210396in}}%
\pgfpathlineto{\pgfqpoint{2.324349in}{2.309134in}}%
\pgfpathlineto{\pgfqpoint{2.324756in}{2.251537in}}%
\pgfpathlineto{\pgfqpoint{2.324960in}{2.342047in}}%
\pgfpathlineto{\pgfqpoint{2.325368in}{2.276221in}}%
\pgfpathlineto{\pgfqpoint{2.325979in}{2.416101in}}%
\pgfpathlineto{\pgfqpoint{2.326795in}{2.374960in}}%
\pgfpathlineto{\pgfqpoint{2.329036in}{2.235080in}}%
\pgfpathlineto{\pgfqpoint{2.329240in}{2.218624in}}%
\pgfpathlineto{\pgfqpoint{2.329852in}{2.243309in}}%
\pgfpathlineto{\pgfqpoint{2.330055in}{2.235080in}}%
\pgfpathlineto{\pgfqpoint{2.331482in}{2.383188in}}%
\pgfpathlineto{\pgfqpoint{2.331890in}{2.292678in}}%
\pgfpathlineto{\pgfqpoint{2.332705in}{2.350275in}}%
\pgfpathlineto{\pgfqpoint{2.333520in}{2.366732in}}%
\pgfpathlineto{\pgfqpoint{2.333724in}{2.350275in}}%
\pgfpathlineto{\pgfqpoint{2.334335in}{2.292678in}}%
\pgfpathlineto{\pgfqpoint{2.335558in}{2.300906in}}%
\pgfpathlineto{\pgfqpoint{2.335762in}{2.333819in}}%
\pgfpathlineto{\pgfqpoint{2.336169in}{2.243309in}}%
\pgfpathlineto{\pgfqpoint{2.336373in}{2.317363in}}%
\pgfpathlineto{\pgfqpoint{2.336577in}{2.259765in}}%
\pgfpathlineto{\pgfqpoint{2.337392in}{2.300906in}}%
\pgfpathlineto{\pgfqpoint{2.338615in}{2.218624in}}%
\pgfpathlineto{\pgfqpoint{2.338819in}{2.267993in}}%
\pgfpathlineto{\pgfqpoint{2.340042in}{2.350275in}}%
\pgfpathlineto{\pgfqpoint{2.339227in}{2.210396in}}%
\pgfpathlineto{\pgfqpoint{2.340246in}{2.325591in}}%
\pgfpathlineto{\pgfqpoint{2.340449in}{2.325591in}}%
\pgfpathlineto{\pgfqpoint{2.340857in}{2.251537in}}%
\pgfpathlineto{\pgfqpoint{2.341265in}{2.333819in}}%
\pgfpathlineto{\pgfqpoint{2.341468in}{2.317363in}}%
\pgfpathlineto{\pgfqpoint{2.341876in}{2.391416in}}%
\pgfpathlineto{\pgfqpoint{2.342487in}{2.350275in}}%
\pgfpathlineto{\pgfqpoint{2.343099in}{2.383188in}}%
\pgfpathlineto{\pgfqpoint{2.343303in}{2.333819in}}%
\pgfpathlineto{\pgfqpoint{2.344525in}{2.473698in}}%
\pgfpathlineto{\pgfqpoint{2.346563in}{2.309134in}}%
\pgfpathlineto{\pgfqpoint{2.347990in}{2.539524in}}%
\pgfpathlineto{\pgfqpoint{2.348194in}{2.523068in}}%
\pgfpathlineto{\pgfqpoint{2.348398in}{2.523068in}}%
\pgfpathlineto{\pgfqpoint{2.348601in}{2.564209in}}%
\pgfpathlineto{\pgfqpoint{2.349213in}{2.506611in}}%
\pgfpathlineto{\pgfqpoint{2.349417in}{2.547752in}}%
\pgfpathlineto{\pgfqpoint{2.350639in}{2.465470in}}%
\pgfpathlineto{\pgfqpoint{2.351658in}{2.531296in}}%
\pgfpathlineto{\pgfqpoint{2.352474in}{2.440786in}}%
\pgfpathlineto{\pgfqpoint{2.352881in}{2.449014in}}%
\pgfpathlineto{\pgfqpoint{2.353085in}{2.432557in}}%
\pgfpathlineto{\pgfqpoint{2.353289in}{2.449014in}}%
\pgfpathlineto{\pgfqpoint{2.353493in}{2.506611in}}%
\pgfpathlineto{\pgfqpoint{2.354104in}{2.498383in}}%
\pgfpathlineto{\pgfqpoint{2.354308in}{2.399645in}}%
\pgfpathlineto{\pgfqpoint{2.355123in}{2.498383in}}%
\pgfpathlineto{\pgfqpoint{2.355327in}{2.449014in}}%
\pgfpathlineto{\pgfqpoint{2.355938in}{2.547752in}}%
\pgfpathlineto{\pgfqpoint{2.356346in}{2.457242in}}%
\pgfpathlineto{\pgfqpoint{2.356550in}{2.449014in}}%
\pgfpathlineto{\pgfqpoint{2.356754in}{2.457242in}}%
\pgfpathlineto{\pgfqpoint{2.357773in}{2.564209in}}%
\pgfpathlineto{\pgfqpoint{2.358180in}{2.531296in}}%
\pgfpathlineto{\pgfqpoint{2.358384in}{2.539524in}}%
\pgfpathlineto{\pgfqpoint{2.358588in}{2.514839in}}%
\pgfpathlineto{\pgfqpoint{2.359199in}{2.449014in}}%
\pgfpathlineto{\pgfqpoint{2.359607in}{2.523068in}}%
\pgfpathlineto{\pgfqpoint{2.359811in}{2.523068in}}%
\pgfpathlineto{\pgfqpoint{2.361033in}{2.440786in}}%
\pgfpathlineto{\pgfqpoint{2.361441in}{2.449014in}}%
\pgfpathlineto{\pgfqpoint{2.361849in}{2.531296in}}%
\pgfpathlineto{\pgfqpoint{2.362460in}{2.473698in}}%
\pgfpathlineto{\pgfqpoint{2.362664in}{2.473698in}}%
\pgfpathlineto{\pgfqpoint{2.363683in}{2.391416in}}%
\pgfpathlineto{\pgfqpoint{2.363887in}{2.465470in}}%
\pgfpathlineto{\pgfqpoint{2.364906in}{2.449014in}}%
\pgfpathlineto{\pgfqpoint{2.365925in}{2.407873in}}%
\pgfpathlineto{\pgfqpoint{2.366129in}{2.416101in}}%
\pgfpathlineto{\pgfqpoint{2.366944in}{2.391416in}}%
\pgfpathlineto{\pgfqpoint{2.367148in}{2.399645in}}%
\pgfpathlineto{\pgfqpoint{2.367963in}{2.391416in}}%
\pgfpathlineto{\pgfqpoint{2.368778in}{2.572437in}}%
\pgfpathlineto{\pgfqpoint{2.369593in}{2.465470in}}%
\pgfpathlineto{\pgfqpoint{2.370001in}{2.473698in}}%
\pgfpathlineto{\pgfqpoint{2.370205in}{2.539524in}}%
\pgfpathlineto{\pgfqpoint{2.371020in}{2.498383in}}%
\pgfpathlineto{\pgfqpoint{2.371224in}{2.481927in}}%
\pgfpathlineto{\pgfqpoint{2.371427in}{2.498383in}}%
\pgfpathlineto{\pgfqpoint{2.371631in}{2.325591in}}%
\pgfpathlineto{\pgfqpoint{2.371835in}{2.539524in}}%
\pgfpathlineto{\pgfqpoint{2.372446in}{2.514839in}}%
\pgfpathlineto{\pgfqpoint{2.372650in}{2.531296in}}%
\pgfpathlineto{\pgfqpoint{2.373262in}{2.514839in}}%
\pgfpathlineto{\pgfqpoint{2.374484in}{2.416101in}}%
\pgfpathlineto{\pgfqpoint{2.374688in}{2.424329in}}%
\pgfpathlineto{\pgfqpoint{2.375707in}{2.465470in}}%
\pgfpathlineto{\pgfqpoint{2.375911in}{2.399645in}}%
\pgfpathlineto{\pgfqpoint{2.376115in}{2.473698in}}%
\pgfpathlineto{\pgfqpoint{2.376726in}{2.457242in}}%
\pgfpathlineto{\pgfqpoint{2.376930in}{2.457242in}}%
\pgfpathlineto{\pgfqpoint{2.377541in}{2.424329in}}%
\pgfpathlineto{\pgfqpoint{2.378764in}{2.218624in}}%
\pgfpathlineto{\pgfqpoint{2.378153in}{2.457242in}}%
\pgfpathlineto{\pgfqpoint{2.378968in}{2.292678in}}%
\pgfpathlineto{\pgfqpoint{2.380599in}{2.457242in}}%
\pgfpathlineto{\pgfqpoint{2.381414in}{2.391416in}}%
\pgfpathlineto{\pgfqpoint{2.381618in}{2.424329in}}%
\pgfpathlineto{\pgfqpoint{2.381821in}{2.457242in}}%
\pgfpathlineto{\pgfqpoint{2.382025in}{2.416101in}}%
\pgfpathlineto{\pgfqpoint{2.382637in}{2.185711in}}%
\pgfpathlineto{\pgfqpoint{2.382840in}{2.481927in}}%
\pgfpathlineto{\pgfqpoint{2.383044in}{2.457242in}}%
\pgfpathlineto{\pgfqpoint{2.383248in}{2.572437in}}%
\pgfpathlineto{\pgfqpoint{2.383452in}{2.416101in}}%
\pgfpathlineto{\pgfqpoint{2.384063in}{2.506611in}}%
\pgfpathlineto{\pgfqpoint{2.385286in}{2.424329in}}%
\pgfpathlineto{\pgfqpoint{2.385490in}{2.432557in}}%
\pgfpathlineto{\pgfqpoint{2.385694in}{2.506611in}}%
\pgfpathlineto{\pgfqpoint{2.386101in}{2.383188in}}%
\pgfpathlineto{\pgfqpoint{2.386509in}{2.440786in}}%
\pgfpathlineto{\pgfqpoint{2.387324in}{2.333819in}}%
\pgfpathlineto{\pgfqpoint{2.387935in}{2.342047in}}%
\pgfpathlineto{\pgfqpoint{2.388751in}{2.383188in}}%
\pgfpathlineto{\pgfqpoint{2.388954in}{2.374960in}}%
\pgfpathlineto{\pgfqpoint{2.389770in}{2.177483in}}%
\pgfpathlineto{\pgfqpoint{2.389973in}{2.325591in}}%
\pgfpathlineto{\pgfqpoint{2.390789in}{2.416101in}}%
\pgfpathlineto{\pgfqpoint{2.390585in}{2.309134in}}%
\pgfpathlineto{\pgfqpoint{2.390992in}{2.358504in}}%
\pgfpathlineto{\pgfqpoint{2.391196in}{2.333819in}}%
\pgfpathlineto{\pgfqpoint{2.392623in}{2.547752in}}%
\pgfpathlineto{\pgfqpoint{2.392827in}{2.547752in}}%
\pgfpathlineto{\pgfqpoint{2.393031in}{2.555980in}}%
\pgfpathlineto{\pgfqpoint{2.393234in}{2.613578in}}%
\pgfpathlineto{\pgfqpoint{2.393642in}{2.457242in}}%
\pgfpathlineto{\pgfqpoint{2.394050in}{2.424329in}}%
\pgfpathlineto{\pgfqpoint{2.394253in}{2.383188in}}%
\pgfpathlineto{\pgfqpoint{2.394865in}{2.465470in}}%
\pgfpathlineto{\pgfqpoint{2.395069in}{2.473698in}}%
\pgfpathlineto{\pgfqpoint{2.395272in}{2.449014in}}%
\pgfpathlineto{\pgfqpoint{2.395680in}{2.391416in}}%
\pgfpathlineto{\pgfqpoint{2.395884in}{2.523068in}}%
\pgfpathlineto{\pgfqpoint{2.396903in}{2.481927in}}%
\pgfpathlineto{\pgfqpoint{2.397310in}{2.481927in}}%
\pgfpathlineto{\pgfqpoint{2.397718in}{2.514839in}}%
\pgfpathlineto{\pgfqpoint{2.398126in}{2.490155in}}%
\pgfpathlineto{\pgfqpoint{2.398533in}{2.391416in}}%
\pgfpathlineto{\pgfqpoint{2.399552in}{2.432557in}}%
\pgfpathlineto{\pgfqpoint{2.401183in}{2.539524in}}%
\pgfpathlineto{\pgfqpoint{2.401386in}{2.473698in}}%
\pgfpathlineto{\pgfqpoint{2.402202in}{2.564209in}}%
\pgfpathlineto{\pgfqpoint{2.403832in}{2.473698in}}%
\pgfpathlineto{\pgfqpoint{2.404240in}{2.449014in}}%
\pgfpathlineto{\pgfqpoint{2.405055in}{2.514839in}}%
\pgfpathlineto{\pgfqpoint{2.405259in}{2.514839in}}%
\pgfpathlineto{\pgfqpoint{2.406074in}{2.432557in}}%
\pgfpathlineto{\pgfqpoint{2.406278in}{2.457242in}}%
\pgfpathlineto{\pgfqpoint{2.406482in}{2.498383in}}%
\pgfpathlineto{\pgfqpoint{2.406889in}{2.432557in}}%
\pgfpathlineto{\pgfqpoint{2.407501in}{2.490155in}}%
\pgfpathlineto{\pgfqpoint{2.408723in}{2.432557in}}%
\pgfpathlineto{\pgfqpoint{2.409742in}{2.514839in}}%
\pgfpathlineto{\pgfqpoint{2.409335in}{2.416101in}}%
\pgfpathlineto{\pgfqpoint{2.409946in}{2.490155in}}%
\pgfpathlineto{\pgfqpoint{2.410354in}{2.514839in}}%
\pgfpathlineto{\pgfqpoint{2.411373in}{2.416101in}}%
\pgfpathlineto{\pgfqpoint{2.411984in}{2.481927in}}%
\pgfpathlineto{\pgfqpoint{2.412392in}{2.440786in}}%
\pgfpathlineto{\pgfqpoint{2.412596in}{2.292678in}}%
\pgfpathlineto{\pgfqpoint{2.412799in}{2.481927in}}%
\pgfpathlineto{\pgfqpoint{2.413411in}{2.473698in}}%
\pgfpathlineto{\pgfqpoint{2.413818in}{2.481927in}}%
\pgfpathlineto{\pgfqpoint{2.414226in}{2.597121in}}%
\pgfpathlineto{\pgfqpoint{2.414837in}{2.580665in}}%
\pgfpathlineto{\pgfqpoint{2.415245in}{2.457242in}}%
\pgfpathlineto{\pgfqpoint{2.415856in}{2.514839in}}%
\pgfpathlineto{\pgfqpoint{2.416468in}{2.539524in}}%
\pgfpathlineto{\pgfqpoint{2.417079in}{2.416101in}}%
\pgfpathlineto{\pgfqpoint{2.417487in}{2.449014in}}%
\pgfpathlineto{\pgfqpoint{2.417691in}{2.514839in}}%
\pgfpathlineto{\pgfqpoint{2.417894in}{2.424329in}}%
\pgfpathlineto{\pgfqpoint{2.418506in}{2.440786in}}%
\pgfpathlineto{\pgfqpoint{2.419321in}{2.498383in}}%
\pgfpathlineto{\pgfqpoint{2.419933in}{2.473698in}}%
\pgfpathlineto{\pgfqpoint{2.421155in}{2.416101in}}%
\pgfpathlineto{\pgfqpoint{2.422174in}{2.481927in}}%
\pgfpathlineto{\pgfqpoint{2.422378in}{2.300906in}}%
\pgfpathlineto{\pgfqpoint{2.423193in}{2.523068in}}%
\pgfpathlineto{\pgfqpoint{2.423397in}{2.547752in}}%
\pgfpathlineto{\pgfqpoint{2.423601in}{2.490155in}}%
\pgfpathlineto{\pgfqpoint{2.423805in}{2.506611in}}%
\pgfpathlineto{\pgfqpoint{2.425435in}{2.416101in}}%
\pgfpathlineto{\pgfqpoint{2.426454in}{2.547752in}}%
\pgfpathlineto{\pgfqpoint{2.426047in}{2.383188in}}%
\pgfpathlineto{\pgfqpoint{2.426862in}{2.481927in}}%
\pgfpathlineto{\pgfqpoint{2.427269in}{2.152798in}}%
\pgfpathlineto{\pgfqpoint{2.428085in}{2.317363in}}%
\pgfpathlineto{\pgfqpoint{2.429307in}{2.449014in}}%
\pgfpathlineto{\pgfqpoint{2.429511in}{2.317363in}}%
\pgfpathlineto{\pgfqpoint{2.430326in}{2.465470in}}%
\pgfpathlineto{\pgfqpoint{2.431957in}{2.169255in}}%
\pgfpathlineto{\pgfqpoint{2.433180in}{2.572437in}}%
\pgfpathlineto{\pgfqpoint{2.435218in}{2.416101in}}%
\pgfpathlineto{\pgfqpoint{2.433587in}{2.588893in}}%
\pgfpathlineto{\pgfqpoint{2.435422in}{2.449014in}}%
\pgfpathlineto{\pgfqpoint{2.436848in}{2.555980in}}%
\pgfpathlineto{\pgfqpoint{2.437256in}{2.555980in}}%
\pgfpathlineto{\pgfqpoint{2.437460in}{2.481927in}}%
\pgfpathlineto{\pgfqpoint{2.438071in}{2.572437in}}%
\pgfpathlineto{\pgfqpoint{2.438275in}{2.506611in}}%
\pgfpathlineto{\pgfqpoint{2.438886in}{2.490155in}}%
\pgfpathlineto{\pgfqpoint{2.439294in}{2.555980in}}%
\pgfpathlineto{\pgfqpoint{2.440109in}{2.473698in}}%
\pgfpathlineto{\pgfqpoint{2.440313in}{2.555980in}}%
\pgfpathlineto{\pgfqpoint{2.441128in}{2.531296in}}%
\pgfpathlineto{\pgfqpoint{2.441536in}{2.605350in}}%
\pgfpathlineto{\pgfqpoint{2.441943in}{2.498383in}}%
\pgfpathlineto{\pgfqpoint{2.442758in}{2.547752in}}%
\pgfpathlineto{\pgfqpoint{2.442962in}{2.572437in}}%
\pgfpathlineto{\pgfqpoint{2.443370in}{2.547752in}}%
\pgfpathlineto{\pgfqpoint{2.443777in}{2.251537in}}%
\pgfpathlineto{\pgfqpoint{2.444389in}{2.457242in}}%
\pgfpathlineto{\pgfqpoint{2.445408in}{2.572437in}}%
\pgfpathlineto{\pgfqpoint{2.445612in}{2.531296in}}%
\pgfpathlineto{\pgfqpoint{2.446019in}{2.506611in}}%
\pgfpathlineto{\pgfqpoint{2.446631in}{2.555980in}}%
\pgfpathlineto{\pgfqpoint{2.446835in}{2.292678in}}%
\pgfpathlineto{\pgfqpoint{2.447854in}{2.399645in}}%
\pgfpathlineto{\pgfqpoint{2.448261in}{2.383188in}}%
\pgfpathlineto{\pgfqpoint{2.449076in}{2.440786in}}%
\pgfpathlineto{\pgfqpoint{2.449280in}{2.391416in}}%
\pgfpathlineto{\pgfqpoint{2.449484in}{2.457242in}}%
\pgfpathlineto{\pgfqpoint{2.449892in}{2.432557in}}%
\pgfpathlineto{\pgfqpoint{2.450095in}{2.473698in}}%
\pgfpathlineto{\pgfqpoint{2.450707in}{2.383188in}}%
\pgfpathlineto{\pgfqpoint{2.450911in}{2.374960in}}%
\pgfpathlineto{\pgfqpoint{2.451114in}{2.152798in}}%
\pgfpathlineto{\pgfqpoint{2.451930in}{2.432557in}}%
\pgfpathlineto{\pgfqpoint{2.452133in}{2.399645in}}%
\pgfpathlineto{\pgfqpoint{2.452337in}{2.457242in}}%
\pgfpathlineto{\pgfqpoint{2.453152in}{2.597121in}}%
\pgfpathlineto{\pgfqpoint{2.453560in}{2.572437in}}%
\pgfpathlineto{\pgfqpoint{2.454579in}{2.588893in}}%
\pgfpathlineto{\pgfqpoint{2.454783in}{2.490155in}}%
\pgfpathlineto{\pgfqpoint{2.455598in}{2.613578in}}%
\pgfpathlineto{\pgfqpoint{2.455802in}{2.539524in}}%
\pgfpathlineto{\pgfqpoint{2.456006in}{2.333819in}}%
\pgfpathlineto{\pgfqpoint{2.457025in}{2.374960in}}%
\pgfpathlineto{\pgfqpoint{2.458859in}{2.136342in}}%
\pgfpathlineto{\pgfqpoint{2.459878in}{2.284450in}}%
\pgfpathlineto{\pgfqpoint{2.460082in}{2.235080in}}%
\pgfpathlineto{\pgfqpoint{2.460286in}{2.284450in}}%
\pgfpathlineto{\pgfqpoint{2.460489in}{2.152798in}}%
\pgfpathlineto{\pgfqpoint{2.460693in}{2.078745in}}%
\pgfpathlineto{\pgfqpoint{2.461101in}{2.267993in}}%
\pgfpathlineto{\pgfqpoint{2.461305in}{2.235080in}}%
\pgfpathlineto{\pgfqpoint{2.462324in}{2.012919in}}%
\pgfpathlineto{\pgfqpoint{2.462120in}{2.243309in}}%
\pgfpathlineto{\pgfqpoint{2.462527in}{2.128114in}}%
\pgfpathlineto{\pgfqpoint{2.463546in}{2.350275in}}%
\pgfpathlineto{\pgfqpoint{2.463750in}{2.251537in}}%
\pgfpathlineto{\pgfqpoint{2.463954in}{2.226852in}}%
\pgfpathlineto{\pgfqpoint{2.464362in}{2.309134in}}%
\pgfpathlineto{\pgfqpoint{2.464565in}{2.383188in}}%
\pgfpathlineto{\pgfqpoint{2.464973in}{2.259765in}}%
\pgfpathlineto{\pgfqpoint{2.465177in}{2.317363in}}%
\pgfpathlineto{\pgfqpoint{2.465381in}{2.251537in}}%
\pgfpathlineto{\pgfqpoint{2.465584in}{2.325591in}}%
\pgfpathlineto{\pgfqpoint{2.466196in}{2.276221in}}%
\pgfpathlineto{\pgfqpoint{2.467419in}{2.391416in}}%
\pgfpathlineto{\pgfqpoint{2.468030in}{2.383188in}}%
\pgfpathlineto{\pgfqpoint{2.469253in}{2.136342in}}%
\pgfpathlineto{\pgfqpoint{2.469457in}{2.350275in}}%
\pgfpathlineto{\pgfqpoint{2.470476in}{2.292678in}}%
\pgfpathlineto{\pgfqpoint{2.470679in}{2.325591in}}%
\pgfpathlineto{\pgfqpoint{2.470883in}{2.251537in}}%
\pgfpathlineto{\pgfqpoint{2.471087in}{2.193939in}}%
\pgfpathlineto{\pgfqpoint{2.471698in}{2.333819in}}%
\pgfpathlineto{\pgfqpoint{2.471902in}{2.342047in}}%
\pgfpathlineto{\pgfqpoint{2.472310in}{2.218624in}}%
\pgfpathlineto{\pgfqpoint{2.472921in}{2.292678in}}%
\pgfpathlineto{\pgfqpoint{2.473329in}{2.325591in}}%
\pgfpathlineto{\pgfqpoint{2.473533in}{2.317363in}}%
\pgfpathlineto{\pgfqpoint{2.473737in}{2.226852in}}%
\pgfpathlineto{\pgfqpoint{2.473940in}{2.358504in}}%
\pgfpathlineto{\pgfqpoint{2.474756in}{2.243309in}}%
\pgfpathlineto{\pgfqpoint{2.474959in}{2.292678in}}%
\pgfpathlineto{\pgfqpoint{2.475571in}{2.226852in}}%
\pgfpathlineto{\pgfqpoint{2.475775in}{2.259765in}}%
\pgfpathlineto{\pgfqpoint{2.476182in}{1.988234in}}%
\pgfpathlineto{\pgfqpoint{2.476997in}{2.193939in}}%
\pgfpathlineto{\pgfqpoint{2.478220in}{2.342047in}}%
\pgfpathlineto{\pgfqpoint{2.478424in}{2.259765in}}%
\pgfpathlineto{\pgfqpoint{2.479239in}{2.342047in}}%
\pgfpathlineto{\pgfqpoint{2.479647in}{2.432557in}}%
\pgfpathlineto{\pgfqpoint{2.479851in}{2.333819in}}%
\pgfpathlineto{\pgfqpoint{2.480258in}{2.416101in}}%
\pgfpathlineto{\pgfqpoint{2.481481in}{2.193939in}}%
\pgfpathlineto{\pgfqpoint{2.481889in}{2.243309in}}%
\pgfpathlineto{\pgfqpoint{2.482092in}{2.300906in}}%
\pgfpathlineto{\pgfqpoint{2.482704in}{2.235080in}}%
\pgfpathlineto{\pgfqpoint{2.482908in}{2.259765in}}%
\pgfpathlineto{\pgfqpoint{2.483927in}{2.210396in}}%
\pgfpathlineto{\pgfqpoint{2.484130in}{2.119886in}}%
\pgfpathlineto{\pgfqpoint{2.484538in}{2.243309in}}%
\pgfpathlineto{\pgfqpoint{2.484946in}{2.226852in}}%
\pgfpathlineto{\pgfqpoint{2.485965in}{2.333819in}}%
\pgfpathlineto{\pgfqpoint{2.485557in}{2.210396in}}%
\pgfpathlineto{\pgfqpoint{2.486372in}{2.259765in}}%
\pgfpathlineto{\pgfqpoint{2.487799in}{1.947093in}}%
\pgfpathlineto{\pgfqpoint{2.488003in}{1.996463in}}%
\pgfpathlineto{\pgfqpoint{2.488410in}{2.185711in}}%
\pgfpathlineto{\pgfqpoint{2.489226in}{2.177483in}}%
\pgfpathlineto{\pgfqpoint{2.490448in}{2.086973in}}%
\pgfpathlineto{\pgfqpoint{2.491264in}{2.144570in}}%
\pgfpathlineto{\pgfqpoint{2.492486in}{1.971778in}}%
\pgfpathlineto{\pgfqpoint{2.493913in}{2.185711in}}%
\pgfpathlineto{\pgfqpoint{2.494932in}{2.128114in}}%
\pgfpathlineto{\pgfqpoint{2.494728in}{2.218624in}}%
\pgfpathlineto{\pgfqpoint{2.495136in}{2.161027in}}%
\pgfpathlineto{\pgfqpoint{2.495340in}{2.226852in}}%
\pgfpathlineto{\pgfqpoint{2.495543in}{2.128114in}}%
\pgfpathlineto{\pgfqpoint{2.495747in}{2.185711in}}%
\pgfpathlineto{\pgfqpoint{2.496970in}{2.012919in}}%
\pgfpathlineto{\pgfqpoint{2.497989in}{2.152798in}}%
\pgfpathlineto{\pgfqpoint{2.498397in}{2.128114in}}%
\pgfpathlineto{\pgfqpoint{2.498804in}{2.193939in}}%
\pgfpathlineto{\pgfqpoint{2.499416in}{2.185711in}}%
\pgfpathlineto{\pgfqpoint{2.499619in}{2.128114in}}%
\pgfpathlineto{\pgfqpoint{2.500435in}{2.202168in}}%
\pgfpathlineto{\pgfqpoint{2.500842in}{2.177483in}}%
\pgfpathlineto{\pgfqpoint{2.501250in}{2.235080in}}%
\pgfpathlineto{\pgfqpoint{2.501658in}{2.144570in}}%
\pgfpathlineto{\pgfqpoint{2.502269in}{2.226852in}}%
\pgfpathlineto{\pgfqpoint{2.502473in}{2.243309in}}%
\pgfpathlineto{\pgfqpoint{2.502677in}{2.210396in}}%
\pgfpathlineto{\pgfqpoint{2.503492in}{2.161027in}}%
\pgfpathlineto{\pgfqpoint{2.503696in}{2.169255in}}%
\pgfpathlineto{\pgfqpoint{2.503899in}{2.202168in}}%
\pgfpathlineto{\pgfqpoint{2.504511in}{2.152798in}}%
\pgfpathlineto{\pgfqpoint{2.505734in}{2.021147in}}%
\pgfpathlineto{\pgfqpoint{2.505937in}{2.029375in}}%
\pgfpathlineto{\pgfqpoint{2.506345in}{2.144570in}}%
\pgfpathlineto{\pgfqpoint{2.506753in}{2.021147in}}%
\pgfpathlineto{\pgfqpoint{2.506956in}{2.045832in}}%
\pgfpathlineto{\pgfqpoint{2.507160in}{1.996463in}}%
\pgfpathlineto{\pgfqpoint{2.507975in}{2.062288in}}%
\pgfpathlineto{\pgfqpoint{2.508179in}{2.062288in}}%
\pgfpathlineto{\pgfqpoint{2.508383in}{2.021147in}}%
\pgfpathlineto{\pgfqpoint{2.508791in}{2.078745in}}%
\pgfpathlineto{\pgfqpoint{2.508994in}{2.169255in}}%
\pgfpathlineto{\pgfqpoint{2.509606in}{1.971778in}}%
\pgfpathlineto{\pgfqpoint{2.510217in}{2.136342in}}%
\pgfpathlineto{\pgfqpoint{2.510625in}{1.971778in}}%
\pgfpathlineto{\pgfqpoint{2.510829in}{1.996463in}}%
\pgfpathlineto{\pgfqpoint{2.511032in}{1.938865in}}%
\pgfpathlineto{\pgfqpoint{2.511644in}{1.971778in}}%
\pgfpathlineto{\pgfqpoint{2.513070in}{1.823670in}}%
\pgfpathlineto{\pgfqpoint{2.512051in}{1.980006in}}%
\pgfpathlineto{\pgfqpoint{2.513274in}{1.848355in}}%
\pgfpathlineto{\pgfqpoint{2.513478in}{1.897724in}}%
\pgfpathlineto{\pgfqpoint{2.513886in}{1.840127in}}%
\pgfpathlineto{\pgfqpoint{2.514090in}{1.634422in}}%
\pgfpathlineto{\pgfqpoint{2.514905in}{1.897724in}}%
\pgfpathlineto{\pgfqpoint{2.515516in}{1.938865in}}%
\pgfpathlineto{\pgfqpoint{2.515312in}{1.873040in}}%
\pgfpathlineto{\pgfqpoint{2.515720in}{1.889496in}}%
\pgfpathlineto{\pgfqpoint{2.516943in}{1.741388in}}%
\pgfpathlineto{\pgfqpoint{2.517962in}{1.930637in}}%
\pgfpathlineto{\pgfqpoint{2.518166in}{1.889496in}}%
\pgfpathlineto{\pgfqpoint{2.518981in}{1.741388in}}%
\pgfpathlineto{\pgfqpoint{2.519185in}{1.766073in}}%
\pgfpathlineto{\pgfqpoint{2.519796in}{1.922409in}}%
\pgfpathlineto{\pgfqpoint{2.520407in}{1.881268in}}%
\pgfpathlineto{\pgfqpoint{2.521834in}{1.988234in}}%
\pgfpathlineto{\pgfqpoint{2.522038in}{2.012919in}}%
\pgfpathlineto{\pgfqpoint{2.522445in}{1.971778in}}%
\pgfpathlineto{\pgfqpoint{2.522853in}{2.004691in}}%
\pgfpathlineto{\pgfqpoint{2.523464in}{1.914181in}}%
\pgfpathlineto{\pgfqpoint{2.524687in}{2.045832in}}%
\pgfpathlineto{\pgfqpoint{2.524891in}{2.037604in}}%
\pgfpathlineto{\pgfqpoint{2.525299in}{2.086973in}}%
\pgfpathlineto{\pgfqpoint{2.525706in}{2.029375in}}%
\pgfpathlineto{\pgfqpoint{2.525910in}{2.078745in}}%
\pgfpathlineto{\pgfqpoint{2.526318in}{2.029375in}}%
\pgfpathlineto{\pgfqpoint{2.526725in}{2.128114in}}%
\pgfpathlineto{\pgfqpoint{2.527744in}{2.070516in}}%
\pgfpathlineto{\pgfqpoint{2.528152in}{2.095201in}}%
\pgfpathlineto{\pgfqpoint{2.528356in}{2.119886in}}%
\pgfpathlineto{\pgfqpoint{2.528763in}{2.054060in}}%
\pgfpathlineto{\pgfqpoint{2.528967in}{2.062288in}}%
\pgfpathlineto{\pgfqpoint{2.529171in}{2.062288in}}%
\pgfpathlineto{\pgfqpoint{2.529579in}{2.111657in}}%
\pgfpathlineto{\pgfqpoint{2.530190in}{2.078745in}}%
\pgfpathlineto{\pgfqpoint{2.530801in}{1.996463in}}%
\pgfpathlineto{\pgfqpoint{2.531413in}{2.004691in}}%
\pgfpathlineto{\pgfqpoint{2.531617in}{2.062288in}}%
\pgfpathlineto{\pgfqpoint{2.532432in}{1.988234in}}%
\pgfpathlineto{\pgfqpoint{2.532636in}{2.037604in}}%
\pgfpathlineto{\pgfqpoint{2.532839in}{2.012919in}}%
\pgfpathlineto{\pgfqpoint{2.533247in}{2.078745in}}%
\pgfpathlineto{\pgfqpoint{2.533451in}{2.062288in}}%
\pgfpathlineto{\pgfqpoint{2.533655in}{2.070516in}}%
\pgfpathlineto{\pgfqpoint{2.534266in}{2.078745in}}%
\pgfpathlineto{\pgfqpoint{2.535081in}{1.971778in}}%
\pgfpathlineto{\pgfqpoint{2.535285in}{1.971778in}}%
\pgfpathlineto{\pgfqpoint{2.535896in}{1.930637in}}%
\pgfpathlineto{\pgfqpoint{2.536100in}{1.963550in}}%
\pgfpathlineto{\pgfqpoint{2.536304in}{2.004691in}}%
\pgfpathlineto{\pgfqpoint{2.536712in}{1.914181in}}%
\pgfpathlineto{\pgfqpoint{2.537119in}{1.955322in}}%
\pgfpathlineto{\pgfqpoint{2.538138in}{1.889496in}}%
\pgfpathlineto{\pgfqpoint{2.537934in}{1.963550in}}%
\pgfpathlineto{\pgfqpoint{2.538546in}{1.914181in}}%
\pgfpathlineto{\pgfqpoint{2.538750in}{1.914181in}}%
\pgfpathlineto{\pgfqpoint{2.540176in}{1.782529in}}%
\pgfpathlineto{\pgfqpoint{2.539157in}{1.930637in}}%
\pgfpathlineto{\pgfqpoint{2.540380in}{1.831899in}}%
\pgfpathlineto{\pgfqpoint{2.540788in}{1.864811in}}%
\pgfpathlineto{\pgfqpoint{2.540992in}{1.840127in}}%
\pgfpathlineto{\pgfqpoint{2.542011in}{1.724932in}}%
\pgfpathlineto{\pgfqpoint{2.542214in}{1.782529in}}%
\pgfpathlineto{\pgfqpoint{2.542418in}{1.782529in}}%
\pgfpathlineto{\pgfqpoint{2.543845in}{1.897724in}}%
\pgfpathlineto{\pgfqpoint{2.544049in}{1.807214in}}%
\pgfpathlineto{\pgfqpoint{2.544660in}{1.922409in}}%
\pgfpathlineto{\pgfqpoint{2.544864in}{1.864811in}}%
\pgfpathlineto{\pgfqpoint{2.545271in}{1.823670in}}%
\pgfpathlineto{\pgfqpoint{2.545679in}{1.848355in}}%
\pgfpathlineto{\pgfqpoint{2.546494in}{1.716704in}}%
\pgfpathlineto{\pgfqpoint{2.546698in}{1.823670in}}%
\pgfpathlineto{\pgfqpoint{2.546902in}{1.897724in}}%
\pgfpathlineto{\pgfqpoint{2.547513in}{1.741388in}}%
\pgfpathlineto{\pgfqpoint{2.547921in}{1.535683in}}%
\pgfpathlineto{\pgfqpoint{2.548328in}{1.601509in}}%
\pgfpathlineto{\pgfqpoint{2.548736in}{1.840127in}}%
\pgfpathlineto{\pgfqpoint{2.549551in}{1.798986in}}%
\pgfpathlineto{\pgfqpoint{2.550570in}{1.741388in}}%
\pgfpathlineto{\pgfqpoint{2.551182in}{1.848355in}}%
\pgfpathlineto{\pgfqpoint{2.551793in}{1.840127in}}%
\pgfpathlineto{\pgfqpoint{2.553627in}{1.683791in}}%
\pgfpathlineto{\pgfqpoint{2.552404in}{1.856583in}}%
\pgfpathlineto{\pgfqpoint{2.553831in}{1.724932in}}%
\pgfpathlineto{\pgfqpoint{2.554239in}{1.692019in}}%
\pgfpathlineto{\pgfqpoint{2.554646in}{1.741388in}}%
\pgfpathlineto{\pgfqpoint{2.555054in}{1.700247in}}%
\pgfpathlineto{\pgfqpoint{2.555258in}{1.724932in}}%
\pgfpathlineto{\pgfqpoint{2.555869in}{1.708476in}}%
\pgfpathlineto{\pgfqpoint{2.556073in}{1.650878in}}%
\pgfpathlineto{\pgfqpoint{2.556684in}{1.774301in}}%
\pgfpathlineto{\pgfqpoint{2.556888in}{1.733160in}}%
\pgfpathlineto{\pgfqpoint{2.557092in}{1.733160in}}%
\pgfpathlineto{\pgfqpoint{2.557296in}{1.766073in}}%
\pgfpathlineto{\pgfqpoint{2.557907in}{1.708476in}}%
\pgfpathlineto{\pgfqpoint{2.558111in}{1.733160in}}%
\pgfpathlineto{\pgfqpoint{2.559130in}{1.864811in}}%
\pgfpathlineto{\pgfqpoint{2.559334in}{1.790758in}}%
\pgfpathlineto{\pgfqpoint{2.559538in}{1.749617in}}%
\pgfpathlineto{\pgfqpoint{2.560149in}{1.807214in}}%
\pgfpathlineto{\pgfqpoint{2.560353in}{1.798986in}}%
\pgfpathlineto{\pgfqpoint{2.562187in}{1.988234in}}%
\pgfpathlineto{\pgfqpoint{2.562595in}{1.741388in}}%
\pgfpathlineto{\pgfqpoint{2.563206in}{1.996463in}}%
\pgfpathlineto{\pgfqpoint{2.563410in}{1.848355in}}%
\pgfpathlineto{\pgfqpoint{2.563817in}{1.980006in}}%
\pgfpathlineto{\pgfqpoint{2.564225in}{1.963550in}}%
\pgfpathlineto{\pgfqpoint{2.564429in}{1.741388in}}%
\pgfpathlineto{\pgfqpoint{2.565040in}{1.971778in}}%
\pgfpathlineto{\pgfqpoint{2.565448in}{1.790758in}}%
\pgfpathlineto{\pgfqpoint{2.566059in}{2.012919in}}%
\pgfpathlineto{\pgfqpoint{2.566671in}{1.930637in}}%
\pgfpathlineto{\pgfqpoint{2.566874in}{1.922409in}}%
\pgfpathlineto{\pgfqpoint{2.567078in}{1.938865in}}%
\pgfpathlineto{\pgfqpoint{2.567282in}{1.930637in}}%
\pgfpathlineto{\pgfqpoint{2.567894in}{2.012919in}}%
\pgfpathlineto{\pgfqpoint{2.568301in}{1.897724in}}%
\pgfpathlineto{\pgfqpoint{2.569524in}{1.848355in}}%
\pgfpathlineto{\pgfqpoint{2.569728in}{1.848355in}}%
\pgfpathlineto{\pgfqpoint{2.569932in}{1.881268in}}%
\pgfpathlineto{\pgfqpoint{2.570339in}{1.798986in}}%
\pgfpathlineto{\pgfqpoint{2.570951in}{1.873040in}}%
\pgfpathlineto{\pgfqpoint{2.571766in}{1.938865in}}%
\pgfpathlineto{\pgfqpoint{2.572377in}{1.922409in}}%
\pgfpathlineto{\pgfqpoint{2.572581in}{1.922409in}}%
\pgfpathlineto{\pgfqpoint{2.572785in}{1.938865in}}%
\pgfpathlineto{\pgfqpoint{2.572989in}{1.914181in}}%
\pgfpathlineto{\pgfqpoint{2.574415in}{1.626194in}}%
\pgfpathlineto{\pgfqpoint{2.575027in}{1.996463in}}%
\pgfpathlineto{\pgfqpoint{2.575638in}{1.856583in}}%
\pgfpathlineto{\pgfqpoint{2.576046in}{2.062288in}}%
\pgfpathlineto{\pgfqpoint{2.576861in}{1.971778in}}%
\pgfpathlineto{\pgfqpoint{2.577676in}{1.782529in}}%
\pgfpathlineto{\pgfqpoint{2.577472in}{2.029375in}}%
\pgfpathlineto{\pgfqpoint{2.577880in}{1.930637in}}%
\pgfpathlineto{\pgfqpoint{2.578084in}{1.955322in}}%
\pgfpathlineto{\pgfqpoint{2.578491in}{1.938865in}}%
\pgfpathlineto{\pgfqpoint{2.579918in}{1.766073in}}%
\pgfpathlineto{\pgfqpoint{2.581141in}{1.831899in}}%
\pgfpathlineto{\pgfqpoint{2.581345in}{1.848355in}}%
\pgfpathlineto{\pgfqpoint{2.581548in}{1.807214in}}%
\pgfpathlineto{\pgfqpoint{2.582771in}{1.601509in}}%
\pgfpathlineto{\pgfqpoint{2.582975in}{1.609737in}}%
\pgfpathlineto{\pgfqpoint{2.583383in}{1.873040in}}%
\pgfpathlineto{\pgfqpoint{2.584198in}{1.807214in}}%
\pgfpathlineto{\pgfqpoint{2.584605in}{1.831899in}}%
\pgfpathlineto{\pgfqpoint{2.584809in}{1.790758in}}%
\pgfpathlineto{\pgfqpoint{2.585217in}{1.807214in}}%
\pgfpathlineto{\pgfqpoint{2.585624in}{1.766073in}}%
\pgfpathlineto{\pgfqpoint{2.585828in}{1.815442in}}%
\pgfpathlineto{\pgfqpoint{2.586032in}{1.774301in}}%
\pgfpathlineto{\pgfqpoint{2.587051in}{1.881268in}}%
\pgfpathlineto{\pgfqpoint{2.587255in}{1.798986in}}%
\pgfpathlineto{\pgfqpoint{2.588274in}{1.815442in}}%
\pgfpathlineto{\pgfqpoint{2.588478in}{1.856583in}}%
\pgfpathlineto{\pgfqpoint{2.588681in}{1.798986in}}%
\pgfpathlineto{\pgfqpoint{2.588885in}{1.798986in}}%
\pgfpathlineto{\pgfqpoint{2.589293in}{1.692019in}}%
\pgfpathlineto{\pgfqpoint{2.590108in}{1.700247in}}%
\pgfpathlineto{\pgfqpoint{2.590923in}{1.905952in}}%
\pgfpathlineto{\pgfqpoint{2.591331in}{1.815442in}}%
\pgfpathlineto{\pgfqpoint{2.591535in}{1.798986in}}%
\pgfpathlineto{\pgfqpoint{2.591942in}{1.848355in}}%
\pgfpathlineto{\pgfqpoint{2.592146in}{1.840127in}}%
\pgfpathlineto{\pgfqpoint{2.592350in}{1.864811in}}%
\pgfpathlineto{\pgfqpoint{2.592554in}{1.881268in}}%
\pgfpathlineto{\pgfqpoint{2.592757in}{1.831899in}}%
\pgfpathlineto{\pgfqpoint{2.593369in}{1.609737in}}%
\pgfpathlineto{\pgfqpoint{2.593573in}{1.848355in}}%
\pgfpathlineto{\pgfqpoint{2.593776in}{1.831899in}}%
\pgfpathlineto{\pgfqpoint{2.593980in}{1.815442in}}%
\pgfpathlineto{\pgfqpoint{2.594184in}{1.856583in}}%
\pgfpathlineto{\pgfqpoint{2.595611in}{1.980006in}}%
\pgfpathlineto{\pgfqpoint{2.596426in}{1.914181in}}%
\pgfpathlineto{\pgfqpoint{2.596834in}{1.922409in}}%
\pgfpathlineto{\pgfqpoint{2.598872in}{1.749617in}}%
\pgfpathlineto{\pgfqpoint{2.599075in}{1.823670in}}%
\pgfpathlineto{\pgfqpoint{2.599279in}{1.634422in}}%
\pgfpathlineto{\pgfqpoint{2.599891in}{1.930637in}}%
\pgfpathlineto{\pgfqpoint{2.600094in}{1.905952in}}%
\pgfpathlineto{\pgfqpoint{2.600502in}{1.922409in}}%
\pgfpathlineto{\pgfqpoint{2.600910in}{1.856583in}}%
\pgfpathlineto{\pgfqpoint{2.601113in}{1.914181in}}%
\pgfpathlineto{\pgfqpoint{2.601521in}{1.815442in}}%
\pgfpathlineto{\pgfqpoint{2.601929in}{1.864811in}}%
\pgfpathlineto{\pgfqpoint{2.603151in}{1.774301in}}%
\pgfpathlineto{\pgfqpoint{2.603355in}{1.881268in}}%
\pgfpathlineto{\pgfqpoint{2.604374in}{1.848355in}}%
\pgfpathlineto{\pgfqpoint{2.604578in}{1.692019in}}%
\pgfpathlineto{\pgfqpoint{2.605189in}{1.815442in}}%
\pgfpathlineto{\pgfqpoint{2.605393in}{1.980006in}}%
\pgfpathlineto{\pgfqpoint{2.606208in}{1.864811in}}%
\pgfpathlineto{\pgfqpoint{2.607431in}{2.012919in}}%
\pgfpathlineto{\pgfqpoint{2.607839in}{1.955322in}}%
\pgfpathlineto{\pgfqpoint{2.609062in}{1.864811in}}%
\pgfpathlineto{\pgfqpoint{2.609469in}{1.856583in}}%
\pgfpathlineto{\pgfqpoint{2.609877in}{1.905952in}}%
\pgfpathlineto{\pgfqpoint{2.610081in}{1.790758in}}%
\pgfpathlineto{\pgfqpoint{2.611100in}{1.831899in}}%
\pgfpathlineto{\pgfqpoint{2.611304in}{1.930637in}}%
\pgfpathlineto{\pgfqpoint{2.611915in}{1.741388in}}%
\pgfpathlineto{\pgfqpoint{2.612119in}{1.692019in}}%
\pgfpathlineto{\pgfqpoint{2.612934in}{1.716704in}}%
\pgfpathlineto{\pgfqpoint{2.613749in}{1.757845in}}%
\pgfpathlineto{\pgfqpoint{2.614361in}{1.404032in}}%
\pgfpathlineto{\pgfqpoint{2.614972in}{1.601509in}}%
\pgfpathlineto{\pgfqpoint{2.616399in}{1.807214in}}%
\pgfpathlineto{\pgfqpoint{2.616806in}{1.864811in}}%
\pgfpathlineto{\pgfqpoint{2.617214in}{1.766073in}}%
\pgfpathlineto{\pgfqpoint{2.618233in}{1.708476in}}%
\pgfpathlineto{\pgfqpoint{2.618437in}{1.716704in}}%
\pgfpathlineto{\pgfqpoint{2.618640in}{1.560368in}}%
\pgfpathlineto{\pgfqpoint{2.618844in}{1.757845in}}%
\pgfpathlineto{\pgfqpoint{2.619456in}{1.692019in}}%
\pgfpathlineto{\pgfqpoint{2.619863in}{1.576824in}}%
\pgfpathlineto{\pgfqpoint{2.620271in}{1.626194in}}%
\pgfpathlineto{\pgfqpoint{2.620475in}{1.346435in}}%
\pgfpathlineto{\pgfqpoint{2.621086in}{1.708476in}}%
\pgfpathlineto{\pgfqpoint{2.621290in}{1.757845in}}%
\pgfpathlineto{\pgfqpoint{2.621494in}{1.659106in}}%
\pgfpathlineto{\pgfqpoint{2.621901in}{1.667335in}}%
\pgfpathlineto{\pgfqpoint{2.622105in}{1.675563in}}%
\pgfpathlineto{\pgfqpoint{2.622513in}{1.659106in}}%
\pgfpathlineto{\pgfqpoint{2.622717in}{1.650878in}}%
\pgfpathlineto{\pgfqpoint{2.622920in}{1.667335in}}%
\pgfpathlineto{\pgfqpoint{2.623124in}{1.659106in}}%
\pgfpathlineto{\pgfqpoint{2.623939in}{1.387576in}}%
\pgfpathlineto{\pgfqpoint{2.624143in}{1.453401in}}%
\pgfpathlineto{\pgfqpoint{2.625366in}{1.733160in}}%
\pgfpathlineto{\pgfqpoint{2.625570in}{1.724932in}}%
\pgfpathlineto{\pgfqpoint{2.625774in}{1.757845in}}%
\pgfpathlineto{\pgfqpoint{2.625977in}{1.757845in}}%
\pgfpathlineto{\pgfqpoint{2.626589in}{1.733160in}}%
\pgfpathlineto{\pgfqpoint{2.626996in}{1.757845in}}%
\pgfpathlineto{\pgfqpoint{2.627812in}{1.840127in}}%
\pgfpathlineto{\pgfqpoint{2.628015in}{1.716704in}}%
\pgfpathlineto{\pgfqpoint{2.628831in}{1.864811in}}%
\pgfpathlineto{\pgfqpoint{2.629850in}{1.873040in}}%
\pgfpathlineto{\pgfqpoint{2.630053in}{1.741388in}}%
\pgfpathlineto{\pgfqpoint{2.630461in}{1.930637in}}%
\pgfpathlineto{\pgfqpoint{2.630665in}{1.716704in}}%
\pgfpathlineto{\pgfqpoint{2.631072in}{1.823670in}}%
\pgfpathlineto{\pgfqpoint{2.632499in}{1.642650in}}%
\pgfpathlineto{\pgfqpoint{2.632703in}{1.692019in}}%
\pgfpathlineto{\pgfqpoint{2.633110in}{1.757845in}}%
\pgfpathlineto{\pgfqpoint{2.633518in}{1.675563in}}%
\pgfpathlineto{\pgfqpoint{2.633722in}{1.708476in}}%
\pgfpathlineto{\pgfqpoint{2.634741in}{1.593281in}}%
\pgfpathlineto{\pgfqpoint{2.635352in}{1.617965in}}%
\pgfpathlineto{\pgfqpoint{2.636371in}{1.585053in}}%
\pgfpathlineto{\pgfqpoint{2.636575in}{1.593281in}}%
\pgfpathlineto{\pgfqpoint{2.637390in}{1.552140in}}%
\pgfpathlineto{\pgfqpoint{2.637798in}{1.675563in}}%
\pgfpathlineto{\pgfqpoint{2.638002in}{1.593281in}}%
\pgfpathlineto{\pgfqpoint{2.639021in}{1.617965in}}%
\pgfpathlineto{\pgfqpoint{2.639632in}{1.601509in}}%
\pgfpathlineto{\pgfqpoint{2.640244in}{1.692019in}}%
\pgfpathlineto{\pgfqpoint{2.640447in}{1.659106in}}%
\pgfpathlineto{\pgfqpoint{2.640651in}{1.692019in}}%
\pgfpathlineto{\pgfqpoint{2.640855in}{1.782529in}}%
\pgfpathlineto{\pgfqpoint{2.641466in}{1.733160in}}%
\pgfpathlineto{\pgfqpoint{2.641874in}{1.634422in}}%
\pgfpathlineto{\pgfqpoint{2.642689in}{1.642650in}}%
\pgfpathlineto{\pgfqpoint{2.644116in}{1.535683in}}%
\pgfpathlineto{\pgfqpoint{2.645339in}{1.601509in}}%
\pgfpathlineto{\pgfqpoint{2.645542in}{1.601509in}}%
\pgfpathlineto{\pgfqpoint{2.646154in}{1.675563in}}%
\pgfpathlineto{\pgfqpoint{2.646765in}{1.634422in}}%
\pgfpathlineto{\pgfqpoint{2.647580in}{1.568596in}}%
\pgfpathlineto{\pgfqpoint{2.647784in}{1.642650in}}%
\pgfpathlineto{\pgfqpoint{2.648192in}{1.650878in}}%
\pgfpathlineto{\pgfqpoint{2.648600in}{1.568596in}}%
\pgfpathlineto{\pgfqpoint{2.649822in}{1.757845in}}%
\pgfpathlineto{\pgfqpoint{2.651453in}{1.206555in}}%
\pgfpathlineto{\pgfqpoint{2.651860in}{1.231240in}}%
\pgfpathlineto{\pgfqpoint{2.652472in}{1.453401in}}%
\pgfpathlineto{\pgfqpoint{2.652879in}{1.338207in}}%
\pgfpathlineto{\pgfqpoint{2.653287in}{1.280609in}}%
\pgfpathlineto{\pgfqpoint{2.653695in}{1.371119in}}%
\pgfpathlineto{\pgfqpoint{2.654510in}{1.502771in}}%
\pgfpathlineto{\pgfqpoint{2.654917in}{1.469858in}}%
\pgfpathlineto{\pgfqpoint{2.655325in}{1.305294in}}%
\pgfpathlineto{\pgfqpoint{2.655936in}{1.354663in}}%
\pgfpathlineto{\pgfqpoint{2.656140in}{1.461630in}}%
\pgfpathlineto{\pgfqpoint{2.656752in}{1.223012in}}%
\pgfpathlineto{\pgfqpoint{2.657567in}{1.255925in}}%
\pgfpathlineto{\pgfqpoint{2.657974in}{1.091361in}}%
\pgfpathlineto{\pgfqpoint{2.658382in}{1.247696in}}%
\pgfpathlineto{\pgfqpoint{2.658993in}{1.074904in}}%
\pgfpathlineto{\pgfqpoint{2.659197in}{1.173643in}}%
\pgfpathlineto{\pgfqpoint{2.660216in}{1.091361in}}%
\pgfpathlineto{\pgfqpoint{2.660420in}{1.132502in}}%
\pgfpathlineto{\pgfqpoint{2.660624in}{1.223012in}}%
\pgfpathlineto{\pgfqpoint{2.661439in}{1.165414in}}%
\pgfpathlineto{\pgfqpoint{2.661847in}{1.157186in}}%
\pgfpathlineto{\pgfqpoint{2.662051in}{1.190099in}}%
\pgfpathlineto{\pgfqpoint{2.662662in}{1.297066in}}%
\pgfpathlineto{\pgfqpoint{2.662866in}{1.239468in}}%
\pgfpathlineto{\pgfqpoint{2.664089in}{1.083132in}}%
\pgfpathlineto{\pgfqpoint{2.664292in}{1.091361in}}%
\pgfpathlineto{\pgfqpoint{2.664700in}{0.992622in}}%
\pgfpathlineto{\pgfqpoint{2.664904in}{1.066676in}}%
\pgfpathlineto{\pgfqpoint{2.665311in}{1.190099in}}%
\pgfpathlineto{\pgfqpoint{2.665719in}{1.033763in}}%
\pgfpathlineto{\pgfqpoint{2.665923in}{0.984394in}}%
\pgfpathlineto{\pgfqpoint{2.666330in}{1.074904in}}%
\pgfpathlineto{\pgfqpoint{2.666534in}{1.157186in}}%
\pgfpathlineto{\pgfqpoint{2.666942in}{1.058448in}}%
\pgfpathlineto{\pgfqpoint{2.667349in}{1.091361in}}%
\pgfpathlineto{\pgfqpoint{2.667757in}{1.041991in}}%
\pgfpathlineto{\pgfqpoint{2.667961in}{1.083132in}}%
\pgfpathlineto{\pgfqpoint{2.668165in}{1.009079in}}%
\pgfpathlineto{\pgfqpoint{2.668572in}{1.132502in}}%
\pgfpathlineto{\pgfqpoint{2.668776in}{1.107817in}}%
\pgfpathlineto{\pgfqpoint{2.669387in}{1.214784in}}%
\pgfpathlineto{\pgfqpoint{2.669999in}{1.157186in}}%
\pgfpathlineto{\pgfqpoint{2.670406in}{1.173643in}}%
\pgfpathlineto{\pgfqpoint{2.670814in}{1.132502in}}%
\pgfpathlineto{\pgfqpoint{2.671425in}{1.148958in}}%
\pgfpathlineto{\pgfqpoint{2.671629in}{1.157186in}}%
\pgfpathlineto{\pgfqpoint{2.671833in}{1.099589in}}%
\pgfpathlineto{\pgfqpoint{2.672444in}{1.165414in}}%
\pgfpathlineto{\pgfqpoint{2.672648in}{1.148958in}}%
\pgfpathlineto{\pgfqpoint{2.675502in}{1.436945in}}%
\pgfpathlineto{\pgfqpoint{2.675909in}{1.412260in}}%
\pgfpathlineto{\pgfqpoint{2.676317in}{1.280609in}}%
\pgfpathlineto{\pgfqpoint{2.677132in}{1.305294in}}%
\pgfpathlineto{\pgfqpoint{2.677336in}{1.371119in}}%
\pgfpathlineto{\pgfqpoint{2.677743in}{1.239468in}}%
\pgfpathlineto{\pgfqpoint{2.677947in}{1.239468in}}%
\pgfpathlineto{\pgfqpoint{2.679781in}{1.519227in}}%
\pgfpathlineto{\pgfqpoint{2.678355in}{1.223012in}}%
\pgfpathlineto{\pgfqpoint{2.680189in}{1.379348in}}%
\pgfpathlineto{\pgfqpoint{2.681412in}{1.247696in}}%
\pgfpathlineto{\pgfqpoint{2.683042in}{1.420489in}}%
\pgfpathlineto{\pgfqpoint{2.684469in}{1.132502in}}%
\pgfpathlineto{\pgfqpoint{2.684876in}{1.140730in}}%
\pgfpathlineto{\pgfqpoint{2.685284in}{1.107817in}}%
\pgfpathlineto{\pgfqpoint{2.686099in}{1.239468in}}%
\pgfpathlineto{\pgfqpoint{2.686303in}{1.181871in}}%
\pgfpathlineto{\pgfqpoint{2.686507in}{1.132502in}}%
\pgfpathlineto{\pgfqpoint{2.686711in}{1.297066in}}%
\pgfpathlineto{\pgfqpoint{2.687322in}{1.469858in}}%
\pgfpathlineto{\pgfqpoint{2.687730in}{1.280609in}}%
\pgfpathlineto{\pgfqpoint{2.687933in}{1.280609in}}%
\pgfpathlineto{\pgfqpoint{2.688545in}{1.354663in}}%
\pgfpathlineto{\pgfqpoint{2.688953in}{1.288837in}}%
\pgfpathlineto{\pgfqpoint{2.689768in}{1.321750in}}%
\pgfpathlineto{\pgfqpoint{2.690175in}{1.198327in}}%
\pgfpathlineto{\pgfqpoint{2.690379in}{1.198327in}}%
\pgfpathlineto{\pgfqpoint{2.690583in}{1.000850in}}%
\pgfpathlineto{\pgfqpoint{2.690787in}{1.206555in}}%
\pgfpathlineto{\pgfqpoint{2.691602in}{1.074904in}}%
\pgfpathlineto{\pgfqpoint{2.692213in}{1.313522in}}%
\pgfpathlineto{\pgfqpoint{2.692825in}{1.181871in}}%
\pgfpathlineto{\pgfqpoint{2.693844in}{1.107817in}}%
\pgfpathlineto{\pgfqpoint{2.695067in}{1.264153in}}%
\pgfpathlineto{\pgfqpoint{2.695270in}{1.247696in}}%
\pgfpathlineto{\pgfqpoint{2.695882in}{1.255925in}}%
\pgfpathlineto{\pgfqpoint{2.697920in}{1.585053in}}%
\pgfpathlineto{\pgfqpoint{2.698124in}{1.593281in}}%
\pgfpathlineto{\pgfqpoint{2.698327in}{1.560368in}}%
\pgfpathlineto{\pgfqpoint{2.699754in}{1.420489in}}%
\pgfpathlineto{\pgfqpoint{2.699958in}{1.428717in}}%
\pgfpathlineto{\pgfqpoint{2.701996in}{1.585053in}}%
\pgfpathlineto{\pgfqpoint{2.700365in}{1.420489in}}%
\pgfpathlineto{\pgfqpoint{2.702200in}{1.560368in}}%
\pgfpathlineto{\pgfqpoint{2.702404in}{1.535683in}}%
\pgfpathlineto{\pgfqpoint{2.702811in}{1.593281in}}%
\pgfpathlineto{\pgfqpoint{2.703015in}{1.576824in}}%
\pgfpathlineto{\pgfqpoint{2.703219in}{1.626194in}}%
\pgfpathlineto{\pgfqpoint{2.703626in}{1.535683in}}%
\pgfpathlineto{\pgfqpoint{2.703830in}{1.552140in}}%
\pgfpathlineto{\pgfqpoint{2.704034in}{1.527455in}}%
\pgfpathlineto{\pgfqpoint{2.704238in}{1.552140in}}%
\pgfpathlineto{\pgfqpoint{2.704849in}{1.659106in}}%
\pgfpathlineto{\pgfqpoint{2.705461in}{1.642650in}}%
\pgfpathlineto{\pgfqpoint{2.705664in}{1.650878in}}%
\pgfpathlineto{\pgfqpoint{2.705868in}{1.642650in}}%
\pgfpathlineto{\pgfqpoint{2.706276in}{1.601509in}}%
\pgfpathlineto{\pgfqpoint{2.706480in}{1.692019in}}%
\pgfpathlineto{\pgfqpoint{2.706683in}{1.741388in}}%
\pgfpathlineto{\pgfqpoint{2.706887in}{1.642650in}}%
\pgfpathlineto{\pgfqpoint{2.707091in}{1.642650in}}%
\pgfpathlineto{\pgfqpoint{2.707499in}{1.601509in}}%
\pgfpathlineto{\pgfqpoint{2.707906in}{1.642650in}}%
\pgfpathlineto{\pgfqpoint{2.709129in}{1.716704in}}%
\pgfpathlineto{\pgfqpoint{2.709333in}{1.675563in}}%
\pgfpathlineto{\pgfqpoint{2.709537in}{1.650878in}}%
\pgfpathlineto{\pgfqpoint{2.710148in}{1.708476in}}%
\pgfpathlineto{\pgfqpoint{2.711167in}{1.757845in}}%
\pgfpathlineto{\pgfqpoint{2.711371in}{1.749617in}}%
\pgfpathlineto{\pgfqpoint{2.711778in}{1.807214in}}%
\pgfpathlineto{\pgfqpoint{2.712186in}{1.782529in}}%
\pgfpathlineto{\pgfqpoint{2.712390in}{1.708476in}}%
\pgfpathlineto{\pgfqpoint{2.713001in}{1.881268in}}%
\pgfpathlineto{\pgfqpoint{2.713409in}{1.889496in}}%
\pgfpathlineto{\pgfqpoint{2.713613in}{1.856583in}}%
\pgfpathlineto{\pgfqpoint{2.714428in}{1.938865in}}%
\pgfpathlineto{\pgfqpoint{2.714632in}{1.889496in}}%
\pgfpathlineto{\pgfqpoint{2.715243in}{1.930637in}}%
\pgfpathlineto{\pgfqpoint{2.715651in}{1.864811in}}%
\pgfpathlineto{\pgfqpoint{2.716874in}{2.029375in}}%
\pgfpathlineto{\pgfqpoint{2.717689in}{1.980006in}}%
\pgfpathlineto{\pgfqpoint{2.717281in}{2.037604in}}%
\pgfpathlineto{\pgfqpoint{2.717893in}{2.004691in}}%
\pgfpathlineto{\pgfqpoint{2.719727in}{2.136342in}}%
\pgfpathlineto{\pgfqpoint{2.719931in}{2.128114in}}%
\pgfpathlineto{\pgfqpoint{2.720542in}{2.103429in}}%
\pgfpathlineto{\pgfqpoint{2.720338in}{2.185711in}}%
\pgfpathlineto{\pgfqpoint{2.720746in}{2.161027in}}%
\pgfpathlineto{\pgfqpoint{2.721561in}{2.111657in}}%
\pgfpathlineto{\pgfqpoint{2.722784in}{2.243309in}}%
\pgfpathlineto{\pgfqpoint{2.723191in}{2.210396in}}%
\pgfpathlineto{\pgfqpoint{2.724210in}{2.086973in}}%
\pgfpathlineto{\pgfqpoint{2.725229in}{2.185711in}}%
\pgfpathlineto{\pgfqpoint{2.725433in}{2.169255in}}%
\pgfpathlineto{\pgfqpoint{2.725841in}{2.136342in}}%
\pgfpathlineto{\pgfqpoint{2.726045in}{2.185711in}}%
\pgfpathlineto{\pgfqpoint{2.726452in}{2.169255in}}%
\pgfpathlineto{\pgfqpoint{2.726656in}{2.243309in}}%
\pgfpathlineto{\pgfqpoint{2.727267in}{2.202168in}}%
\pgfpathlineto{\pgfqpoint{2.727471in}{2.119886in}}%
\pgfpathlineto{\pgfqpoint{2.727675in}{2.267993in}}%
\pgfpathlineto{\pgfqpoint{2.728490in}{2.136342in}}%
\pgfpathlineto{\pgfqpoint{2.728694in}{2.144570in}}%
\pgfpathlineto{\pgfqpoint{2.728898in}{2.136342in}}%
\pgfpathlineto{\pgfqpoint{2.729102in}{2.111657in}}%
\pgfpathlineto{\pgfqpoint{2.729509in}{2.152798in}}%
\pgfpathlineto{\pgfqpoint{2.729917in}{2.144570in}}%
\pgfpathlineto{\pgfqpoint{2.730121in}{2.210396in}}%
\pgfpathlineto{\pgfqpoint{2.730936in}{2.136342in}}%
\pgfpathlineto{\pgfqpoint{2.731140in}{2.128114in}}%
\pgfpathlineto{\pgfqpoint{2.732159in}{2.210396in}}%
\pgfpathlineto{\pgfqpoint{2.732363in}{1.988234in}}%
\pgfpathlineto{\pgfqpoint{2.733178in}{2.152798in}}%
\pgfpathlineto{\pgfqpoint{2.733382in}{2.152798in}}%
\pgfpathlineto{\pgfqpoint{2.733585in}{2.119886in}}%
\pgfpathlineto{\pgfqpoint{2.734197in}{2.152798in}}%
\pgfpathlineto{\pgfqpoint{2.735623in}{2.317363in}}%
\pgfpathlineto{\pgfqpoint{2.735827in}{2.284450in}}%
\pgfpathlineto{\pgfqpoint{2.736439in}{2.251537in}}%
\pgfpathlineto{\pgfqpoint{2.736642in}{2.284450in}}%
\pgfpathlineto{\pgfqpoint{2.737050in}{2.317363in}}%
\pgfpathlineto{\pgfqpoint{2.737254in}{2.226852in}}%
\pgfpathlineto{\pgfqpoint{2.738069in}{2.325591in}}%
\pgfpathlineto{\pgfqpoint{2.738273in}{2.235080in}}%
\pgfpathlineto{\pgfqpoint{2.739903in}{2.358504in}}%
\pgfpathlineto{\pgfqpoint{2.740311in}{2.374960in}}%
\pgfpathlineto{\pgfqpoint{2.740515in}{2.366732in}}%
\pgfpathlineto{\pgfqpoint{2.740922in}{2.144570in}}%
\pgfpathlineto{\pgfqpoint{2.741737in}{2.276221in}}%
\pgfpathlineto{\pgfqpoint{2.741941in}{2.309134in}}%
\pgfpathlineto{\pgfqpoint{2.742349in}{2.267993in}}%
\pgfpathlineto{\pgfqpoint{2.742757in}{1.980006in}}%
\pgfpathlineto{\pgfqpoint{2.743368in}{2.210396in}}%
\pgfpathlineto{\pgfqpoint{2.744183in}{2.342047in}}%
\pgfpathlineto{\pgfqpoint{2.744387in}{2.267993in}}%
\pgfpathlineto{\pgfqpoint{2.744795in}{2.086973in}}%
\pgfpathlineto{\pgfqpoint{2.745610in}{2.218624in}}%
\pgfpathlineto{\pgfqpoint{2.746833in}{2.358504in}}%
\pgfpathlineto{\pgfqpoint{2.748055in}{2.226852in}}%
\pgfpathlineto{\pgfqpoint{2.748667in}{2.325591in}}%
\pgfpathlineto{\pgfqpoint{2.749278in}{2.309134in}}%
\pgfpathlineto{\pgfqpoint{2.749890in}{2.226852in}}%
\pgfpathlineto{\pgfqpoint{2.750501in}{2.243309in}}%
\pgfpathlineto{\pgfqpoint{2.751112in}{2.342047in}}%
\pgfpathlineto{\pgfqpoint{2.751520in}{2.226852in}}%
\pgfpathlineto{\pgfqpoint{2.752131in}{2.300906in}}%
\pgfpathlineto{\pgfqpoint{2.752335in}{2.169255in}}%
\pgfpathlineto{\pgfqpoint{2.752947in}{2.317363in}}%
\pgfpathlineto{\pgfqpoint{2.753150in}{2.292678in}}%
\pgfpathlineto{\pgfqpoint{2.753558in}{2.292678in}}%
\pgfpathlineto{\pgfqpoint{2.754373in}{2.358504in}}%
\pgfpathlineto{\pgfqpoint{2.754985in}{2.333819in}}%
\pgfpathlineto{\pgfqpoint{2.755596in}{2.276221in}}%
\pgfpathlineto{\pgfqpoint{2.755800in}{2.317363in}}%
\pgfpathlineto{\pgfqpoint{2.756411in}{2.300906in}}%
\pgfpathlineto{\pgfqpoint{2.757023in}{2.399645in}}%
\pgfpathlineto{\pgfqpoint{2.758653in}{2.259765in}}%
\pgfpathlineto{\pgfqpoint{2.759468in}{2.358504in}}%
\pgfpathlineto{\pgfqpoint{2.759672in}{2.259765in}}%
\pgfpathlineto{\pgfqpoint{2.759876in}{2.267993in}}%
\pgfpathlineto{\pgfqpoint{2.760080in}{2.259765in}}%
\pgfpathlineto{\pgfqpoint{2.760284in}{2.235080in}}%
\pgfpathlineto{\pgfqpoint{2.760691in}{2.309134in}}%
\pgfpathlineto{\pgfqpoint{2.761303in}{2.350275in}}%
\pgfpathlineto{\pgfqpoint{2.761506in}{2.292678in}}%
\pgfpathlineto{\pgfqpoint{2.761710in}{2.317363in}}%
\pgfpathlineto{\pgfqpoint{2.761914in}{2.276221in}}%
\pgfpathlineto{\pgfqpoint{2.762322in}{2.366732in}}%
\pgfpathlineto{\pgfqpoint{2.762729in}{2.333819in}}%
\pgfpathlineto{\pgfqpoint{2.762933in}{2.383188in}}%
\pgfpathlineto{\pgfqpoint{2.763544in}{2.325591in}}%
\pgfpathlineto{\pgfqpoint{2.764156in}{2.276221in}}%
\pgfpathlineto{\pgfqpoint{2.764563in}{2.317363in}}%
\pgfpathlineto{\pgfqpoint{2.764767in}{2.342047in}}%
\pgfpathlineto{\pgfqpoint{2.764971in}{2.292678in}}%
\pgfpathlineto{\pgfqpoint{2.765175in}{2.300906in}}%
\pgfpathlineto{\pgfqpoint{2.766194in}{2.202168in}}%
\pgfpathlineto{\pgfqpoint{2.766805in}{2.218624in}}%
\pgfpathlineto{\pgfqpoint{2.767417in}{2.366732in}}%
\pgfpathlineto{\pgfqpoint{2.768232in}{2.325591in}}%
\pgfpathlineto{\pgfqpoint{2.769251in}{2.292678in}}%
\pgfpathlineto{\pgfqpoint{2.769455in}{2.300906in}}%
\pgfpathlineto{\pgfqpoint{2.770066in}{2.374960in}}%
\pgfpathlineto{\pgfqpoint{2.769862in}{2.292678in}}%
\pgfpathlineto{\pgfqpoint{2.770881in}{2.358504in}}%
\pgfpathlineto{\pgfqpoint{2.771085in}{2.358504in}}%
\pgfpathlineto{\pgfqpoint{2.771493in}{2.366732in}}%
\pgfpathlineto{\pgfqpoint{2.772512in}{2.276221in}}%
\pgfpathlineto{\pgfqpoint{2.772919in}{2.366732in}}%
\pgfpathlineto{\pgfqpoint{2.773531in}{2.350275in}}%
\pgfpathlineto{\pgfqpoint{2.774957in}{2.193939in}}%
\pgfpathlineto{\pgfqpoint{2.775161in}{2.226852in}}%
\pgfpathlineto{\pgfqpoint{2.775773in}{2.276221in}}%
\pgfpathlineto{\pgfqpoint{2.775569in}{2.210396in}}%
\pgfpathlineto{\pgfqpoint{2.775976in}{2.251537in}}%
\pgfpathlineto{\pgfqpoint{2.776792in}{2.202168in}}%
\pgfpathlineto{\pgfqpoint{2.776995in}{2.259765in}}%
\pgfpathlineto{\pgfqpoint{2.777607in}{2.292678in}}%
\pgfpathlineto{\pgfqpoint{2.777403in}{2.251537in}}%
\pgfpathlineto{\pgfqpoint{2.778014in}{2.259765in}}%
\pgfpathlineto{\pgfqpoint{2.778218in}{2.259765in}}%
\pgfpathlineto{\pgfqpoint{2.778422in}{2.210396in}}%
\pgfpathlineto{\pgfqpoint{2.779237in}{2.276221in}}%
\pgfpathlineto{\pgfqpoint{2.779441in}{2.267993in}}%
\pgfpathlineto{\pgfqpoint{2.779645in}{2.193939in}}%
\pgfpathlineto{\pgfqpoint{2.780256in}{2.243309in}}%
\pgfpathlineto{\pgfqpoint{2.780460in}{2.317363in}}%
\pgfpathlineto{\pgfqpoint{2.781275in}{2.226852in}}%
\pgfpathlineto{\pgfqpoint{2.781479in}{2.284450in}}%
\pgfpathlineto{\pgfqpoint{2.782498in}{2.226852in}}%
\pgfpathlineto{\pgfqpoint{2.782702in}{2.251537in}}%
\pgfpathlineto{\pgfqpoint{2.783925in}{2.399645in}}%
\pgfpathlineto{\pgfqpoint{2.784536in}{2.383188in}}%
\pgfpathlineto{\pgfqpoint{2.785351in}{2.259765in}}%
\pgfpathlineto{\pgfqpoint{2.785555in}{2.300906in}}%
\pgfpathlineto{\pgfqpoint{2.786167in}{2.366732in}}%
\pgfpathlineto{\pgfqpoint{2.786370in}{2.292678in}}%
\pgfpathlineto{\pgfqpoint{2.786778in}{2.276221in}}%
\pgfpathlineto{\pgfqpoint{2.786982in}{2.317363in}}%
\pgfpathlineto{\pgfqpoint{2.787186in}{2.309134in}}%
\pgfpathlineto{\pgfqpoint{2.788001in}{2.399645in}}%
\pgfpathlineto{\pgfqpoint{2.788205in}{2.350275in}}%
\pgfpathlineto{\pgfqpoint{2.788816in}{2.276221in}}%
\pgfpathlineto{\pgfqpoint{2.789224in}{2.325591in}}%
\pgfpathlineto{\pgfqpoint{2.790243in}{2.366732in}}%
\pgfpathlineto{\pgfqpoint{2.791873in}{2.103429in}}%
\pgfpathlineto{\pgfqpoint{2.793096in}{2.407873in}}%
\pgfpathlineto{\pgfqpoint{2.794319in}{2.292678in}}%
\pgfpathlineto{\pgfqpoint{2.794522in}{2.383188in}}%
\pgfpathlineto{\pgfqpoint{2.795338in}{2.374960in}}%
\pgfpathlineto{\pgfqpoint{2.795541in}{2.325591in}}%
\pgfpathlineto{\pgfqpoint{2.796357in}{2.350275in}}%
\pgfpathlineto{\pgfqpoint{2.796968in}{2.399645in}}%
\pgfpathlineto{\pgfqpoint{2.797376in}{2.391416in}}%
\pgfpathlineto{\pgfqpoint{2.798191in}{2.276221in}}%
\pgfpathlineto{\pgfqpoint{2.797987in}{2.399645in}}%
\pgfpathlineto{\pgfqpoint{2.798395in}{2.391416in}}%
\pgfpathlineto{\pgfqpoint{2.798599in}{2.383188in}}%
\pgfpathlineto{\pgfqpoint{2.798802in}{2.449014in}}%
\pgfpathlineto{\pgfqpoint{2.799210in}{2.374960in}}%
\pgfpathlineto{\pgfqpoint{2.799414in}{2.374960in}}%
\pgfpathlineto{\pgfqpoint{2.800025in}{2.424329in}}%
\pgfpathlineto{\pgfqpoint{2.800637in}{2.325591in}}%
\pgfpathlineto{\pgfqpoint{2.801656in}{2.383188in}}%
\pgfpathlineto{\pgfqpoint{2.802878in}{2.119886in}}%
\pgfpathlineto{\pgfqpoint{2.802267in}{2.440786in}}%
\pgfpathlineto{\pgfqpoint{2.803082in}{2.317363in}}%
\pgfpathlineto{\pgfqpoint{2.803490in}{2.325591in}}%
\pgfpathlineto{\pgfqpoint{2.803694in}{2.300906in}}%
\pgfpathlineto{\pgfqpoint{2.803897in}{2.136342in}}%
\pgfpathlineto{\pgfqpoint{2.804713in}{2.325591in}}%
\pgfpathlineto{\pgfqpoint{2.805120in}{2.391416in}}%
\pgfpathlineto{\pgfqpoint{2.805528in}{2.309134in}}%
\pgfpathlineto{\pgfqpoint{2.805732in}{2.350275in}}%
\pgfpathlineto{\pgfqpoint{2.807158in}{2.251537in}}%
\pgfpathlineto{\pgfqpoint{2.807362in}{2.276221in}}%
\pgfpathlineto{\pgfqpoint{2.808585in}{2.440786in}}%
\pgfpathlineto{\pgfqpoint{2.809400in}{2.325591in}}%
\pgfpathlineto{\pgfqpoint{2.809808in}{2.350275in}}%
\pgfpathlineto{\pgfqpoint{2.810012in}{2.366732in}}%
\pgfpathlineto{\pgfqpoint{2.810215in}{2.325591in}}%
\pgfpathlineto{\pgfqpoint{2.810419in}{2.325591in}}%
\pgfpathlineto{\pgfqpoint{2.811642in}{2.095201in}}%
\pgfpathlineto{\pgfqpoint{2.810827in}{2.366732in}}%
\pgfpathlineto{\pgfqpoint{2.811846in}{2.226852in}}%
\pgfpathlineto{\pgfqpoint{2.812253in}{2.399645in}}%
\pgfpathlineto{\pgfqpoint{2.812865in}{2.251537in}}%
\pgfpathlineto{\pgfqpoint{2.813476in}{2.416101in}}%
\pgfpathlineto{\pgfqpoint{2.814088in}{2.325591in}}%
\pgfpathlineto{\pgfqpoint{2.814291in}{2.309134in}}%
\pgfpathlineto{\pgfqpoint{2.814495in}{2.358504in}}%
\pgfpathlineto{\pgfqpoint{2.814699in}{2.342047in}}%
\pgfpathlineto{\pgfqpoint{2.815310in}{2.317363in}}%
\pgfpathlineto{\pgfqpoint{2.815514in}{2.358504in}}%
\pgfpathlineto{\pgfqpoint{2.816941in}{2.202168in}}%
\pgfpathlineto{\pgfqpoint{2.817552in}{2.490155in}}%
\pgfpathlineto{\pgfqpoint{2.818367in}{2.350275in}}%
\pgfpathlineto{\pgfqpoint{2.818571in}{2.358504in}}%
\pgfpathlineto{\pgfqpoint{2.818775in}{2.128114in}}%
\pgfpathlineto{\pgfqpoint{2.819183in}{2.432557in}}%
\pgfpathlineto{\pgfqpoint{2.819590in}{2.399645in}}%
\pgfpathlineto{\pgfqpoint{2.820202in}{2.481927in}}%
\pgfpathlineto{\pgfqpoint{2.820609in}{2.449014in}}%
\pgfpathlineto{\pgfqpoint{2.821221in}{2.136342in}}%
\pgfpathlineto{\pgfqpoint{2.821628in}{2.333819in}}%
\pgfpathlineto{\pgfqpoint{2.822036in}{2.267993in}}%
\pgfpathlineto{\pgfqpoint{2.822443in}{2.309134in}}%
\pgfpathlineto{\pgfqpoint{2.823259in}{2.383188in}}%
\pgfpathlineto{\pgfqpoint{2.823666in}{2.374960in}}%
\pgfpathlineto{\pgfqpoint{2.824278in}{2.407873in}}%
\pgfpathlineto{\pgfqpoint{2.824482in}{2.259765in}}%
\pgfpathlineto{\pgfqpoint{2.825501in}{2.539524in}}%
\pgfpathlineto{\pgfqpoint{2.825704in}{2.449014in}}%
\pgfpathlineto{\pgfqpoint{2.826520in}{2.613578in}}%
\pgfpathlineto{\pgfqpoint{2.827335in}{2.572437in}}%
\pgfpathlineto{\pgfqpoint{2.827539in}{2.523068in}}%
\pgfpathlineto{\pgfqpoint{2.828558in}{2.531296in}}%
\pgfpathlineto{\pgfqpoint{2.828761in}{2.547752in}}%
\pgfpathlineto{\pgfqpoint{2.828965in}{2.514839in}}%
\pgfpathlineto{\pgfqpoint{2.829169in}{2.514839in}}%
\pgfpathlineto{\pgfqpoint{2.829373in}{2.440786in}}%
\pgfpathlineto{\pgfqpoint{2.829984in}{2.547752in}}%
\pgfpathlineto{\pgfqpoint{2.830392in}{2.449014in}}%
\pgfpathlineto{\pgfqpoint{2.830596in}{2.465470in}}%
\pgfpathlineto{\pgfqpoint{2.830799in}{2.391416in}}%
\pgfpathlineto{\pgfqpoint{2.831003in}{2.506611in}}%
\pgfpathlineto{\pgfqpoint{2.831615in}{2.449014in}}%
\pgfpathlineto{\pgfqpoint{2.831818in}{2.506611in}}%
\pgfpathlineto{\pgfqpoint{2.832022in}{2.276221in}}%
\pgfpathlineto{\pgfqpoint{2.832430in}{2.465470in}}%
\pgfpathlineto{\pgfqpoint{2.832634in}{2.399645in}}%
\pgfpathlineto{\pgfqpoint{2.833041in}{2.473698in}}%
\pgfpathlineto{\pgfqpoint{2.833653in}{2.407873in}}%
\pgfpathlineto{\pgfqpoint{2.835079in}{2.531296in}}%
\pgfpathlineto{\pgfqpoint{2.836302in}{2.407873in}}%
\pgfpathlineto{\pgfqpoint{2.836710in}{2.391416in}}%
\pgfpathlineto{\pgfqpoint{2.837525in}{2.457242in}}%
\pgfpathlineto{\pgfqpoint{2.837729in}{2.416101in}}%
\pgfpathlineto{\pgfqpoint{2.838136in}{2.490155in}}%
\pgfpathlineto{\pgfqpoint{2.838340in}{2.481927in}}%
\pgfpathlineto{\pgfqpoint{2.838544in}{2.490155in}}%
\pgfpathlineto{\pgfqpoint{2.838748in}{2.555980in}}%
\pgfpathlineto{\pgfqpoint{2.839359in}{2.449014in}}%
\pgfpathlineto{\pgfqpoint{2.839563in}{2.490155in}}%
\pgfpathlineto{\pgfqpoint{2.839767in}{2.424329in}}%
\pgfpathlineto{\pgfqpoint{2.840582in}{2.514839in}}%
\pgfpathlineto{\pgfqpoint{2.840786in}{2.572437in}}%
\pgfpathlineto{\pgfqpoint{2.841601in}{2.531296in}}%
\pgfpathlineto{\pgfqpoint{2.842009in}{2.580665in}}%
\pgfpathlineto{\pgfqpoint{2.842212in}{2.506611in}}%
\pgfpathlineto{\pgfqpoint{2.843028in}{2.555980in}}%
\pgfpathlineto{\pgfqpoint{2.843231in}{2.572437in}}%
\pgfpathlineto{\pgfqpoint{2.843435in}{2.523068in}}%
\pgfpathlineto{\pgfqpoint{2.843843in}{2.564209in}}%
\pgfpathlineto{\pgfqpoint{2.844047in}{2.523068in}}%
\pgfpathlineto{\pgfqpoint{2.844454in}{2.580665in}}%
\pgfpathlineto{\pgfqpoint{2.844862in}{2.555980in}}%
\pgfpathlineto{\pgfqpoint{2.845066in}{2.572437in}}%
\pgfpathlineto{\pgfqpoint{2.845269in}{2.539524in}}%
\pgfpathlineto{\pgfqpoint{2.845881in}{2.555980in}}%
\pgfpathlineto{\pgfqpoint{2.846492in}{2.481927in}}%
\pgfpathlineto{\pgfqpoint{2.847104in}{2.523068in}}%
\pgfpathlineto{\pgfqpoint{2.848123in}{2.473698in}}%
\pgfpathlineto{\pgfqpoint{2.848938in}{2.399645in}}%
\pgfpathlineto{\pgfqpoint{2.849142in}{2.449014in}}%
\pgfpathlineto{\pgfqpoint{2.849753in}{2.555980in}}%
\pgfpathlineto{\pgfqpoint{2.849957in}{2.440786in}}%
\pgfpathlineto{\pgfqpoint{2.850772in}{2.514839in}}%
\pgfpathlineto{\pgfqpoint{2.851587in}{2.440786in}}%
\pgfpathlineto{\pgfqpoint{2.851180in}{2.539524in}}%
\pgfpathlineto{\pgfqpoint{2.851995in}{2.457242in}}%
\pgfpathlineto{\pgfqpoint{2.852403in}{2.531296in}}%
\pgfpathlineto{\pgfqpoint{2.852606in}{2.424329in}}%
\pgfpathlineto{\pgfqpoint{2.852810in}{2.440786in}}%
\pgfpathlineto{\pgfqpoint{2.853422in}{2.473698in}}%
\pgfpathlineto{\pgfqpoint{2.854033in}{2.374960in}}%
\pgfpathlineto{\pgfqpoint{2.854848in}{2.407873in}}%
\pgfpathlineto{\pgfqpoint{2.855052in}{2.333819in}}%
\pgfpathlineto{\pgfqpoint{2.855663in}{2.498383in}}%
\pgfpathlineto{\pgfqpoint{2.855867in}{2.465470in}}%
\pgfpathlineto{\pgfqpoint{2.856275in}{2.531296in}}%
\pgfpathlineto{\pgfqpoint{2.856479in}{2.547752in}}%
\pgfpathlineto{\pgfqpoint{2.856886in}{2.276221in}}%
\pgfpathlineto{\pgfqpoint{2.857498in}{2.292678in}}%
\pgfpathlineto{\pgfqpoint{2.857701in}{2.539524in}}%
\pgfpathlineto{\pgfqpoint{2.858313in}{2.284450in}}%
\pgfpathlineto{\pgfqpoint{2.858517in}{2.284450in}}%
\pgfpathlineto{\pgfqpoint{2.860147in}{2.539524in}}%
\pgfpathlineto{\pgfqpoint{2.858924in}{2.251537in}}%
\pgfpathlineto{\pgfqpoint{2.860351in}{2.523068in}}%
\pgfpathlineto{\pgfqpoint{2.861370in}{2.416101in}}%
\pgfpathlineto{\pgfqpoint{2.861574in}{2.457242in}}%
\pgfpathlineto{\pgfqpoint{2.862185in}{2.473698in}}%
\pgfpathlineto{\pgfqpoint{2.862796in}{2.342047in}}%
\pgfpathlineto{\pgfqpoint{2.864019in}{2.514839in}}%
\pgfpathlineto{\pgfqpoint{2.865242in}{2.416101in}}%
\pgfpathlineto{\pgfqpoint{2.866057in}{2.506611in}}%
\pgfpathlineto{\pgfqpoint{2.866465in}{2.490155in}}%
\pgfpathlineto{\pgfqpoint{2.867076in}{2.473698in}}%
\pgfpathlineto{\pgfqpoint{2.867280in}{2.506611in}}%
\pgfpathlineto{\pgfqpoint{2.867484in}{2.407873in}}%
\pgfpathlineto{\pgfqpoint{2.868299in}{2.490155in}}%
\pgfpathlineto{\pgfqpoint{2.869114in}{2.523068in}}%
\pgfpathlineto{\pgfqpoint{2.869318in}{2.473698in}}%
\pgfpathlineto{\pgfqpoint{2.869522in}{2.564209in}}%
\pgfpathlineto{\pgfqpoint{2.870133in}{2.539524in}}%
\pgfpathlineto{\pgfqpoint{2.870337in}{2.564209in}}%
\pgfpathlineto{\pgfqpoint{2.870541in}{2.523068in}}%
\pgfpathlineto{\pgfqpoint{2.871152in}{2.276221in}}%
\pgfpathlineto{\pgfqpoint{2.871560in}{2.514839in}}%
\pgfpathlineto{\pgfqpoint{2.871968in}{2.473698in}}%
\pgfpathlineto{\pgfqpoint{2.872171in}{2.539524in}}%
\pgfpathlineto{\pgfqpoint{2.872375in}{2.514839in}}%
\pgfpathlineto{\pgfqpoint{2.873190in}{2.654719in}}%
\pgfpathlineto{\pgfqpoint{2.873394in}{2.621806in}}%
\pgfpathlineto{\pgfqpoint{2.874209in}{2.457242in}}%
\pgfpathlineto{\pgfqpoint{2.874413in}{2.498383in}}%
\pgfpathlineto{\pgfqpoint{2.874821in}{2.630034in}}%
\pgfpathlineto{\pgfqpoint{2.875636in}{2.588893in}}%
\pgfpathlineto{\pgfqpoint{2.876655in}{2.481927in}}%
\pgfpathlineto{\pgfqpoint{2.876859in}{2.514839in}}%
\pgfpathlineto{\pgfqpoint{2.877063in}{2.523068in}}%
\pgfpathlineto{\pgfqpoint{2.877267in}{2.514839in}}%
\pgfpathlineto{\pgfqpoint{2.877470in}{2.490155in}}%
\pgfpathlineto{\pgfqpoint{2.877878in}{2.547752in}}%
\pgfpathlineto{\pgfqpoint{2.878082in}{2.555980in}}%
\pgfpathlineto{\pgfqpoint{2.879101in}{2.449014in}}%
\pgfpathlineto{\pgfqpoint{2.879916in}{2.580665in}}%
\pgfpathlineto{\pgfqpoint{2.880324in}{2.523068in}}%
\pgfpathlineto{\pgfqpoint{2.881139in}{2.391416in}}%
\pgfpathlineto{\pgfqpoint{2.881954in}{2.432557in}}%
\pgfpathlineto{\pgfqpoint{2.882158in}{2.424329in}}%
\pgfpathlineto{\pgfqpoint{2.882362in}{2.449014in}}%
\pgfpathlineto{\pgfqpoint{2.882973in}{2.490155in}}%
\pgfpathlineto{\pgfqpoint{2.882769in}{2.432557in}}%
\pgfpathlineto{\pgfqpoint{2.883584in}{2.465470in}}%
\pgfpathlineto{\pgfqpoint{2.883788in}{2.424329in}}%
\pgfpathlineto{\pgfqpoint{2.883992in}{2.481927in}}%
\pgfpathlineto{\pgfqpoint{2.884400in}{2.465470in}}%
\pgfpathlineto{\pgfqpoint{2.885011in}{2.490155in}}%
\pgfpathlineto{\pgfqpoint{2.885215in}{2.465470in}}%
\pgfpathlineto{\pgfqpoint{2.885622in}{2.391416in}}%
\pgfpathlineto{\pgfqpoint{2.886234in}{2.440786in}}%
\pgfpathlineto{\pgfqpoint{2.886641in}{2.465470in}}%
\pgfpathlineto{\pgfqpoint{2.886845in}{2.449014in}}%
\pgfpathlineto{\pgfqpoint{2.887457in}{2.399645in}}%
\pgfpathlineto{\pgfqpoint{2.887864in}{2.424329in}}%
\pgfpathlineto{\pgfqpoint{2.888068in}{2.440786in}}%
\pgfpathlineto{\pgfqpoint{2.888272in}{2.391416in}}%
\pgfpathlineto{\pgfqpoint{2.888476in}{2.383188in}}%
\pgfpathlineto{\pgfqpoint{2.888679in}{2.416101in}}%
\pgfpathlineto{\pgfqpoint{2.888883in}{2.391416in}}%
\pgfpathlineto{\pgfqpoint{2.889902in}{2.490155in}}%
\pgfpathlineto{\pgfqpoint{2.890106in}{2.432557in}}%
\pgfpathlineto{\pgfqpoint{2.891737in}{2.539524in}}%
\pgfpathlineto{\pgfqpoint{2.892959in}{2.457242in}}%
\pgfpathlineto{\pgfqpoint{2.892756in}{2.555980in}}%
\pgfpathlineto{\pgfqpoint{2.893163in}{2.465470in}}%
\pgfpathlineto{\pgfqpoint{2.893367in}{2.465470in}}%
\pgfpathlineto{\pgfqpoint{2.894590in}{2.555980in}}%
\pgfpathlineto{\pgfqpoint{2.895813in}{2.358504in}}%
\pgfpathlineto{\pgfqpoint{2.897035in}{2.621806in}}%
\pgfpathlineto{\pgfqpoint{2.897239in}{2.539524in}}%
\pgfpathlineto{\pgfqpoint{2.897647in}{2.564209in}}%
\pgfpathlineto{\pgfqpoint{2.897851in}{2.531296in}}%
\pgfpathlineto{\pgfqpoint{2.899073in}{2.317363in}}%
\pgfpathlineto{\pgfqpoint{2.899277in}{2.325591in}}%
\pgfpathlineto{\pgfqpoint{2.899685in}{2.374960in}}%
\pgfpathlineto{\pgfqpoint{2.900296in}{2.342047in}}%
\pgfpathlineto{\pgfqpoint{2.901723in}{2.276221in}}%
\pgfpathlineto{\pgfqpoint{2.902334in}{2.399645in}}%
\pgfpathlineto{\pgfqpoint{2.903149in}{2.366732in}}%
\pgfpathlineto{\pgfqpoint{2.903761in}{2.325591in}}%
\pgfpathlineto{\pgfqpoint{2.904576in}{2.267993in}}%
\pgfpathlineto{\pgfqpoint{2.904780in}{2.292678in}}%
\pgfpathlineto{\pgfqpoint{2.906003in}{2.383188in}}%
\pgfpathlineto{\pgfqpoint{2.905188in}{2.276221in}}%
\pgfpathlineto{\pgfqpoint{2.906410in}{2.358504in}}%
\pgfpathlineto{\pgfqpoint{2.906614in}{2.284450in}}%
\pgfpathlineto{\pgfqpoint{2.907226in}{2.366732in}}%
\pgfpathlineto{\pgfqpoint{2.907429in}{2.366732in}}%
\pgfpathlineto{\pgfqpoint{2.907837in}{2.399645in}}%
\pgfpathlineto{\pgfqpoint{2.908041in}{2.333819in}}%
\pgfpathlineto{\pgfqpoint{2.908245in}{2.350275in}}%
\pgfpathlineto{\pgfqpoint{2.908856in}{2.366732in}}%
\pgfpathlineto{\pgfqpoint{2.909875in}{2.136342in}}%
\pgfpathlineto{\pgfqpoint{2.910283in}{2.383188in}}%
\pgfpathlineto{\pgfqpoint{2.910690in}{2.070516in}}%
\pgfpathlineto{\pgfqpoint{2.911098in}{2.284450in}}%
\pgfpathlineto{\pgfqpoint{2.911709in}{2.407873in}}%
\pgfpathlineto{\pgfqpoint{2.912321in}{2.325591in}}%
\pgfpathlineto{\pgfqpoint{2.912524in}{2.333819in}}%
\pgfpathlineto{\pgfqpoint{2.912728in}{2.317363in}}%
\pgfpathlineto{\pgfqpoint{2.912932in}{2.284450in}}%
\pgfpathlineto{\pgfqpoint{2.913340in}{2.342047in}}%
\pgfpathlineto{\pgfqpoint{2.913543in}{2.333819in}}%
\pgfpathlineto{\pgfqpoint{2.913747in}{2.391416in}}%
\pgfpathlineto{\pgfqpoint{2.914562in}{2.342047in}}%
\pgfpathlineto{\pgfqpoint{2.914766in}{2.342047in}}%
\pgfpathlineto{\pgfqpoint{2.915174in}{2.259765in}}%
\pgfpathlineto{\pgfqpoint{2.915989in}{2.284450in}}%
\pgfpathlineto{\pgfqpoint{2.916804in}{2.358504in}}%
\pgfpathlineto{\pgfqpoint{2.916397in}{2.276221in}}%
\pgfpathlineto{\pgfqpoint{2.917008in}{2.292678in}}%
\pgfpathlineto{\pgfqpoint{2.917416in}{2.317363in}}%
\pgfpathlineto{\pgfqpoint{2.918842in}{2.235080in}}%
\pgfpathlineto{\pgfqpoint{2.919658in}{2.317363in}}%
\pgfpathlineto{\pgfqpoint{2.920065in}{2.300906in}}%
\pgfpathlineto{\pgfqpoint{2.920473in}{2.366732in}}%
\pgfpathlineto{\pgfqpoint{2.920880in}{2.325591in}}%
\pgfpathlineto{\pgfqpoint{2.921492in}{2.144570in}}%
\pgfpathlineto{\pgfqpoint{2.921899in}{2.185711in}}%
\pgfpathlineto{\pgfqpoint{2.922511in}{2.407873in}}%
\pgfpathlineto{\pgfqpoint{2.923122in}{2.333819in}}%
\pgfpathlineto{\pgfqpoint{2.924141in}{2.407873in}}%
\pgfpathlineto{\pgfqpoint{2.923734in}{2.309134in}}%
\pgfpathlineto{\pgfqpoint{2.924345in}{2.374960in}}%
\pgfpathlineto{\pgfqpoint{2.924753in}{2.235080in}}%
\pgfpathlineto{\pgfqpoint{2.925364in}{2.350275in}}%
\pgfpathlineto{\pgfqpoint{2.925568in}{2.407873in}}%
\pgfpathlineto{\pgfqpoint{2.925975in}{2.342047in}}%
\pgfpathlineto{\pgfqpoint{2.926791in}{2.086973in}}%
\pgfpathlineto{\pgfqpoint{2.927402in}{2.144570in}}%
\pgfpathlineto{\pgfqpoint{2.928829in}{2.391416in}}%
\pgfpathlineto{\pgfqpoint{2.929440in}{2.383188in}}%
\pgfpathlineto{\pgfqpoint{2.930663in}{2.317363in}}%
\pgfpathlineto{\pgfqpoint{2.930255in}{2.391416in}}%
\pgfpathlineto{\pgfqpoint{2.931071in}{2.333819in}}%
\pgfpathlineto{\pgfqpoint{2.932293in}{2.407873in}}%
\pgfpathlineto{\pgfqpoint{2.932497in}{2.391416in}}%
\pgfpathlineto{\pgfqpoint{2.933109in}{2.440786in}}%
\pgfpathlineto{\pgfqpoint{2.933924in}{2.325591in}}%
\pgfpathlineto{\pgfqpoint{2.934535in}{2.350275in}}%
\pgfpathlineto{\pgfqpoint{2.934739in}{2.342047in}}%
\pgfpathlineto{\pgfqpoint{2.935554in}{2.251537in}}%
\pgfpathlineto{\pgfqpoint{2.935758in}{2.276221in}}%
\pgfpathlineto{\pgfqpoint{2.936369in}{2.350275in}}%
\pgfpathlineto{\pgfqpoint{2.936573in}{2.144570in}}%
\pgfpathlineto{\pgfqpoint{2.937185in}{2.383188in}}%
\pgfpathlineto{\pgfqpoint{2.937388in}{2.366732in}}%
\pgfpathlineto{\pgfqpoint{2.938611in}{2.095201in}}%
\pgfpathlineto{\pgfqpoint{2.938815in}{2.193939in}}%
\pgfpathlineto{\pgfqpoint{2.939019in}{2.342047in}}%
\pgfpathlineto{\pgfqpoint{2.940038in}{2.292678in}}%
\pgfpathlineto{\pgfqpoint{2.941464in}{2.391416in}}%
\pgfpathlineto{\pgfqpoint{2.943910in}{2.136342in}}%
\pgfpathlineto{\pgfqpoint{2.944318in}{2.193939in}}%
\pgfpathlineto{\pgfqpoint{2.945541in}{1.930637in}}%
\pgfpathlineto{\pgfqpoint{2.946152in}{2.169255in}}%
\pgfpathlineto{\pgfqpoint{2.946560in}{2.004691in}}%
\pgfpathlineto{\pgfqpoint{2.948394in}{1.848355in}}%
\pgfpathlineto{\pgfqpoint{2.949413in}{2.004691in}}%
\pgfpathlineto{\pgfqpoint{2.949617in}{1.996463in}}%
\pgfpathlineto{\pgfqpoint{2.949820in}{1.922409in}}%
\pgfpathlineto{\pgfqpoint{2.950024in}{2.029375in}}%
\pgfpathlineto{\pgfqpoint{2.950636in}{2.021147in}}%
\pgfpathlineto{\pgfqpoint{2.950839in}{2.029375in}}%
\pgfpathlineto{\pgfqpoint{2.951247in}{1.930637in}}%
\pgfpathlineto{\pgfqpoint{2.952062in}{1.938865in}}%
\pgfpathlineto{\pgfqpoint{2.952266in}{1.914181in}}%
\pgfpathlineto{\pgfqpoint{2.952674in}{1.980006in}}%
\pgfpathlineto{\pgfqpoint{2.952877in}{1.963550in}}%
\pgfpathlineto{\pgfqpoint{2.953081in}{2.037604in}}%
\pgfpathlineto{\pgfqpoint{2.953693in}{1.938865in}}%
\pgfpathlineto{\pgfqpoint{2.954100in}{1.856583in}}%
\pgfpathlineto{\pgfqpoint{2.954508in}{2.004691in}}%
\pgfpathlineto{\pgfqpoint{2.954915in}{1.963550in}}%
\pgfpathlineto{\pgfqpoint{2.955527in}{1.980006in}}%
\pgfpathlineto{\pgfqpoint{2.955731in}{2.037604in}}%
\pgfpathlineto{\pgfqpoint{2.956342in}{1.947093in}}%
\pgfpathlineto{\pgfqpoint{2.956546in}{1.996463in}}%
\pgfpathlineto{\pgfqpoint{2.957361in}{2.012919in}}%
\pgfpathlineto{\pgfqpoint{2.957973in}{1.922409in}}%
\pgfpathlineto{\pgfqpoint{2.959603in}{2.037604in}}%
\pgfpathlineto{\pgfqpoint{2.961030in}{1.905952in}}%
\pgfpathlineto{\pgfqpoint{2.961845in}{1.971778in}}%
\pgfpathlineto{\pgfqpoint{2.962252in}{1.947093in}}%
\pgfpathlineto{\pgfqpoint{2.962456in}{1.980006in}}%
\pgfpathlineto{\pgfqpoint{2.962660in}{1.930637in}}%
\pgfpathlineto{\pgfqpoint{2.963271in}{1.963550in}}%
\pgfpathlineto{\pgfqpoint{2.963475in}{1.889496in}}%
\pgfpathlineto{\pgfqpoint{2.963679in}{1.971778in}}%
\pgfpathlineto{\pgfqpoint{2.964290in}{1.930637in}}%
\pgfpathlineto{\pgfqpoint{2.965513in}{2.070516in}}%
\pgfpathlineto{\pgfqpoint{2.965921in}{2.029375in}}%
\pgfpathlineto{\pgfqpoint{2.966125in}{2.029375in}}%
\pgfpathlineto{\pgfqpoint{2.967755in}{2.095201in}}%
\pgfpathlineto{\pgfqpoint{2.967959in}{2.029375in}}%
\pgfpathlineto{\pgfqpoint{2.968570in}{2.169255in}}%
\pgfpathlineto{\pgfqpoint{2.968774in}{2.119886in}}%
\pgfpathlineto{\pgfqpoint{2.969589in}{2.169255in}}%
\pgfpathlineto{\pgfqpoint{2.970608in}{2.021147in}}%
\pgfpathlineto{\pgfqpoint{2.970812in}{2.045832in}}%
\pgfpathlineto{\pgfqpoint{2.971016in}{2.136342in}}%
\pgfpathlineto{\pgfqpoint{2.971424in}{2.021147in}}%
\pgfpathlineto{\pgfqpoint{2.971627in}{2.037604in}}%
\pgfpathlineto{\pgfqpoint{2.972443in}{1.930637in}}%
\pgfpathlineto{\pgfqpoint{2.972850in}{1.947093in}}%
\pgfpathlineto{\pgfqpoint{2.973869in}{2.054060in}}%
\pgfpathlineto{\pgfqpoint{2.974073in}{2.037604in}}%
\pgfpathlineto{\pgfqpoint{2.974481in}{1.807214in}}%
\pgfpathlineto{\pgfqpoint{2.975296in}{1.955322in}}%
\pgfpathlineto{\pgfqpoint{2.975500in}{1.980006in}}%
\pgfpathlineto{\pgfqpoint{2.975907in}{1.905952in}}%
\pgfpathlineto{\pgfqpoint{2.976111in}{1.922409in}}%
\pgfpathlineto{\pgfqpoint{2.976315in}{1.848355in}}%
\pgfpathlineto{\pgfqpoint{2.977130in}{1.889496in}}%
\pgfpathlineto{\pgfqpoint{2.977334in}{1.889496in}}%
\pgfpathlineto{\pgfqpoint{2.977741in}{1.881268in}}%
\pgfpathlineto{\pgfqpoint{2.977945in}{1.963550in}}%
\pgfpathlineto{\pgfqpoint{2.978964in}{1.930637in}}%
\pgfpathlineto{\pgfqpoint{2.980391in}{1.831899in}}%
\pgfpathlineto{\pgfqpoint{2.980595in}{1.930637in}}%
\pgfpathlineto{\pgfqpoint{2.981614in}{1.914181in}}%
\pgfpathlineto{\pgfqpoint{2.982021in}{1.897724in}}%
\pgfpathlineto{\pgfqpoint{2.982429in}{1.963550in}}%
\pgfpathlineto{\pgfqpoint{2.983244in}{1.831899in}}%
\pgfpathlineto{\pgfqpoint{2.983652in}{1.840127in}}%
\pgfpathlineto{\pgfqpoint{2.984875in}{1.922409in}}%
\pgfpathlineto{\pgfqpoint{2.985078in}{1.914181in}}%
\pgfpathlineto{\pgfqpoint{2.985282in}{1.889496in}}%
\pgfpathlineto{\pgfqpoint{2.985894in}{1.905952in}}%
\pgfpathlineto{\pgfqpoint{2.986097in}{1.955322in}}%
\pgfpathlineto{\pgfqpoint{2.986709in}{1.848355in}}%
\pgfpathlineto{\pgfqpoint{2.986913in}{1.716704in}}%
\pgfpathlineto{\pgfqpoint{2.987320in}{1.938865in}}%
\pgfpathlineto{\pgfqpoint{2.987728in}{1.873040in}}%
\pgfpathlineto{\pgfqpoint{2.988747in}{2.037604in}}%
\pgfpathlineto{\pgfqpoint{2.989154in}{1.955322in}}%
\pgfpathlineto{\pgfqpoint{2.990377in}{2.062288in}}%
\pgfpathlineto{\pgfqpoint{2.990581in}{2.021147in}}%
\pgfpathlineto{\pgfqpoint{2.992008in}{1.864811in}}%
\pgfpathlineto{\pgfqpoint{2.990989in}{2.037604in}}%
\pgfpathlineto{\pgfqpoint{2.992415in}{1.881268in}}%
\pgfpathlineto{\pgfqpoint{2.993842in}{2.078745in}}%
\pgfpathlineto{\pgfqpoint{2.993434in}{1.864811in}}%
\pgfpathlineto{\pgfqpoint{2.994046in}{2.029375in}}%
\pgfpathlineto{\pgfqpoint{2.995065in}{1.856583in}}%
\pgfpathlineto{\pgfqpoint{2.995268in}{1.873040in}}%
\pgfpathlineto{\pgfqpoint{2.996084in}{1.914181in}}%
\pgfpathlineto{\pgfqpoint{2.996491in}{1.889496in}}%
\pgfpathlineto{\pgfqpoint{2.997103in}{1.774301in}}%
\pgfpathlineto{\pgfqpoint{2.997306in}{1.617965in}}%
\pgfpathlineto{\pgfqpoint{2.997918in}{1.980006in}}%
\pgfpathlineto{\pgfqpoint{2.999752in}{1.585053in}}%
\pgfpathlineto{\pgfqpoint{3.000160in}{1.749617in}}%
\pgfpathlineto{\pgfqpoint{3.001586in}{1.938865in}}%
\pgfpathlineto{\pgfqpoint{3.001994in}{1.692019in}}%
\pgfpathlineto{\pgfqpoint{3.002402in}{1.955322in}}%
\pgfpathlineto{\pgfqpoint{3.002809in}{1.988234in}}%
\pgfpathlineto{\pgfqpoint{3.003013in}{1.955322in}}%
\pgfpathlineto{\pgfqpoint{3.003421in}{1.790758in}}%
\pgfpathlineto{\pgfqpoint{3.004032in}{2.004691in}}%
\pgfpathlineto{\pgfqpoint{3.004236in}{2.004691in}}%
\pgfpathlineto{\pgfqpoint{3.005662in}{1.848355in}}%
\pgfpathlineto{\pgfqpoint{3.004643in}{2.012919in}}%
\pgfpathlineto{\pgfqpoint{3.005866in}{1.914181in}}%
\pgfpathlineto{\pgfqpoint{3.006070in}{1.914181in}}%
\pgfpathlineto{\pgfqpoint{3.006274in}{1.831899in}}%
\pgfpathlineto{\pgfqpoint{3.007089in}{1.930637in}}%
\pgfpathlineto{\pgfqpoint{3.007293in}{1.930637in}}%
\pgfpathlineto{\pgfqpoint{3.008108in}{1.840127in}}%
\pgfpathlineto{\pgfqpoint{3.008516in}{1.881268in}}%
\pgfpathlineto{\pgfqpoint{3.009127in}{1.864811in}}%
\pgfpathlineto{\pgfqpoint{3.009535in}{1.947093in}}%
\pgfpathlineto{\pgfqpoint{3.009738in}{1.905952in}}%
\pgfpathlineto{\pgfqpoint{3.010554in}{1.955322in}}%
\pgfpathlineto{\pgfqpoint{3.011573in}{1.889496in}}%
\pgfpathlineto{\pgfqpoint{3.011777in}{1.905952in}}%
\pgfpathlineto{\pgfqpoint{3.012184in}{1.971778in}}%
\pgfpathlineto{\pgfqpoint{3.012388in}{1.881268in}}%
\pgfpathlineto{\pgfqpoint{3.012592in}{1.626194in}}%
\pgfpathlineto{\pgfqpoint{3.013203in}{2.029375in}}%
\pgfpathlineto{\pgfqpoint{3.013407in}{1.963550in}}%
\pgfpathlineto{\pgfqpoint{3.014018in}{1.766073in}}%
\pgfpathlineto{\pgfqpoint{3.015241in}{1.790758in}}%
\pgfpathlineto{\pgfqpoint{3.015649in}{1.782529in}}%
\pgfpathlineto{\pgfqpoint{3.016668in}{1.848355in}}%
\pgfpathlineto{\pgfqpoint{3.017687in}{1.807214in}}%
\pgfpathlineto{\pgfqpoint{3.017075in}{1.881268in}}%
\pgfpathlineto{\pgfqpoint{3.017891in}{1.815442in}}%
\pgfpathlineto{\pgfqpoint{3.018706in}{1.881268in}}%
\pgfpathlineto{\pgfqpoint{3.018910in}{1.823670in}}%
\pgfpathlineto{\pgfqpoint{3.019521in}{1.766073in}}%
\pgfpathlineto{\pgfqpoint{3.019725in}{1.798986in}}%
\pgfpathlineto{\pgfqpoint{3.021151in}{2.021147in}}%
\pgfpathlineto{\pgfqpoint{3.021559in}{1.856583in}}%
\pgfpathlineto{\pgfqpoint{3.022374in}{1.864811in}}%
\pgfpathlineto{\pgfqpoint{3.023801in}{1.519227in}}%
\pgfpathlineto{\pgfqpoint{3.024820in}{1.864811in}}%
\pgfpathlineto{\pgfqpoint{3.025024in}{1.798986in}}%
\pgfpathlineto{\pgfqpoint{3.025228in}{1.798986in}}%
\pgfpathlineto{\pgfqpoint{3.025839in}{1.749617in}}%
\pgfpathlineto{\pgfqpoint{3.026450in}{1.774301in}}%
\pgfpathlineto{\pgfqpoint{3.027062in}{1.716704in}}%
\pgfpathlineto{\pgfqpoint{3.027266in}{1.535683in}}%
\pgfpathlineto{\pgfqpoint{3.027673in}{1.807214in}}%
\pgfpathlineto{\pgfqpoint{3.027877in}{1.766073in}}%
\pgfpathlineto{\pgfqpoint{3.028081in}{1.897724in}}%
\pgfpathlineto{\pgfqpoint{3.028285in}{1.609737in}}%
\pgfpathlineto{\pgfqpoint{3.029100in}{1.873040in}}%
\pgfpathlineto{\pgfqpoint{3.029304in}{1.848355in}}%
\pgfpathlineto{\pgfqpoint{3.029711in}{1.930637in}}%
\pgfpathlineto{\pgfqpoint{3.031953in}{1.782529in}}%
\pgfpathlineto{\pgfqpoint{3.032768in}{1.848355in}}%
\pgfpathlineto{\pgfqpoint{3.032972in}{1.774301in}}%
\pgfpathlineto{\pgfqpoint{3.034602in}{1.914181in}}%
\pgfpathlineto{\pgfqpoint{3.035010in}{1.873040in}}%
\pgfpathlineto{\pgfqpoint{3.036029in}{1.700247in}}%
\pgfpathlineto{\pgfqpoint{3.036233in}{1.708476in}}%
\pgfpathlineto{\pgfqpoint{3.036437in}{1.733160in}}%
\pgfpathlineto{\pgfqpoint{3.036844in}{1.675563in}}%
\pgfpathlineto{\pgfqpoint{3.037048in}{1.642650in}}%
\pgfpathlineto{\pgfqpoint{3.037456in}{1.716704in}}%
\pgfpathlineto{\pgfqpoint{3.038475in}{1.790758in}}%
\pgfpathlineto{\pgfqpoint{3.038067in}{1.683791in}}%
\pgfpathlineto{\pgfqpoint{3.038679in}{1.757845in}}%
\pgfpathlineto{\pgfqpoint{3.039494in}{1.659106in}}%
\pgfpathlineto{\pgfqpoint{3.039901in}{1.708476in}}%
\pgfpathlineto{\pgfqpoint{3.040309in}{1.823670in}}%
\pgfpathlineto{\pgfqpoint{3.040513in}{1.700247in}}%
\pgfpathlineto{\pgfqpoint{3.040920in}{1.782529in}}%
\pgfpathlineto{\pgfqpoint{3.041939in}{1.724932in}}%
\pgfpathlineto{\pgfqpoint{3.041736in}{1.815442in}}%
\pgfpathlineto{\pgfqpoint{3.042143in}{1.733160in}}%
\pgfpathlineto{\pgfqpoint{3.042958in}{1.667335in}}%
\pgfpathlineto{\pgfqpoint{3.044385in}{1.831899in}}%
\pgfpathlineto{\pgfqpoint{3.044793in}{1.749617in}}%
\pgfpathlineto{\pgfqpoint{3.045200in}{1.798986in}}%
\pgfpathlineto{\pgfqpoint{3.045608in}{1.856583in}}%
\pgfpathlineto{\pgfqpoint{3.045812in}{1.782529in}}%
\pgfpathlineto{\pgfqpoint{3.046015in}{1.823670in}}%
\pgfpathlineto{\pgfqpoint{3.046423in}{1.708476in}}%
\pgfpathlineto{\pgfqpoint{3.047034in}{1.749617in}}%
\pgfpathlineto{\pgfqpoint{3.047442in}{1.856583in}}%
\pgfpathlineto{\pgfqpoint{3.047646in}{1.741388in}}%
\pgfpathlineto{\pgfqpoint{3.048053in}{1.831899in}}%
\pgfpathlineto{\pgfqpoint{3.049072in}{1.617965in}}%
\pgfpathlineto{\pgfqpoint{3.049480in}{1.659106in}}%
\pgfpathlineto{\pgfqpoint{3.050295in}{1.766073in}}%
\pgfpathlineto{\pgfqpoint{3.050703in}{1.749617in}}%
\pgfpathlineto{\pgfqpoint{3.051518in}{1.667335in}}%
\pgfpathlineto{\pgfqpoint{3.051722in}{1.675563in}}%
\pgfpathlineto{\pgfqpoint{3.052333in}{1.782529in}}%
\pgfpathlineto{\pgfqpoint{3.052741in}{1.716704in}}%
\pgfpathlineto{\pgfqpoint{3.053149in}{1.634422in}}%
\pgfpathlineto{\pgfqpoint{3.053760in}{1.692019in}}%
\pgfpathlineto{\pgfqpoint{3.053964in}{1.716704in}}%
\pgfpathlineto{\pgfqpoint{3.054168in}{1.642650in}}%
\pgfpathlineto{\pgfqpoint{3.054575in}{1.609737in}}%
\pgfpathlineto{\pgfqpoint{3.054983in}{1.683791in}}%
\pgfpathlineto{\pgfqpoint{3.055187in}{1.741388in}}%
\pgfpathlineto{\pgfqpoint{3.055594in}{1.675563in}}%
\pgfpathlineto{\pgfqpoint{3.056002in}{1.708476in}}%
\pgfpathlineto{\pgfqpoint{3.056206in}{1.708476in}}%
\pgfpathlineto{\pgfqpoint{3.057428in}{1.626194in}}%
\pgfpathlineto{\pgfqpoint{3.057632in}{1.634422in}}%
\pgfpathlineto{\pgfqpoint{3.057836in}{1.609737in}}%
\pgfpathlineto{\pgfqpoint{3.058244in}{1.560368in}}%
\pgfpathlineto{\pgfqpoint{3.058651in}{1.642650in}}%
\pgfpathlineto{\pgfqpoint{3.058855in}{1.576824in}}%
\pgfpathlineto{\pgfqpoint{3.059059in}{1.642650in}}%
\pgfpathlineto{\pgfqpoint{3.059670in}{1.527455in}}%
\pgfpathlineto{\pgfqpoint{3.059874in}{1.560368in}}%
\pgfpathlineto{\pgfqpoint{3.060282in}{1.593281in}}%
\pgfpathlineto{\pgfqpoint{3.060485in}{1.585053in}}%
\pgfpathlineto{\pgfqpoint{3.061301in}{1.379348in}}%
\pgfpathlineto{\pgfqpoint{3.061504in}{1.609737in}}%
\pgfpathlineto{\pgfqpoint{3.062320in}{1.716704in}}%
\pgfpathlineto{\pgfqpoint{3.062523in}{1.469858in}}%
\pgfpathlineto{\pgfqpoint{3.063339in}{1.733160in}}%
\pgfpathlineto{\pgfqpoint{3.063542in}{1.716704in}}%
\pgfpathlineto{\pgfqpoint{3.063746in}{1.749617in}}%
\pgfpathlineto{\pgfqpoint{3.063950in}{1.749617in}}%
\pgfpathlineto{\pgfqpoint{3.064358in}{1.626194in}}%
\pgfpathlineto{\pgfqpoint{3.064969in}{1.807214in}}%
\pgfpathlineto{\pgfqpoint{3.065988in}{1.766073in}}%
\pgfpathlineto{\pgfqpoint{3.066396in}{1.905952in}}%
\pgfpathlineto{\pgfqpoint{3.066803in}{1.798986in}}%
\pgfpathlineto{\pgfqpoint{3.068026in}{1.716704in}}%
\pgfpathlineto{\pgfqpoint{3.068638in}{1.749617in}}%
\pgfpathlineto{\pgfqpoint{3.069860in}{1.609737in}}%
\pgfpathlineto{\pgfqpoint{3.070064in}{1.617965in}}%
\pgfpathlineto{\pgfqpoint{3.071898in}{1.807214in}}%
\pgfpathlineto{\pgfqpoint{3.072102in}{1.741388in}}%
\pgfpathlineto{\pgfqpoint{3.072306in}{1.445173in}}%
\pgfpathlineto{\pgfqpoint{3.073121in}{1.823670in}}%
\pgfpathlineto{\pgfqpoint{3.074140in}{1.576824in}}%
\pgfpathlineto{\pgfqpoint{3.074548in}{1.626194in}}%
\pgfpathlineto{\pgfqpoint{3.074955in}{1.700247in}}%
\pgfpathlineto{\pgfqpoint{3.075567in}{1.667335in}}%
\pgfpathlineto{\pgfqpoint{3.075771in}{1.617965in}}%
\pgfpathlineto{\pgfqpoint{3.076382in}{1.675563in}}%
\pgfpathlineto{\pgfqpoint{3.077401in}{1.774301in}}%
\pgfpathlineto{\pgfqpoint{3.077809in}{1.724932in}}%
\pgfpathlineto{\pgfqpoint{3.078828in}{1.642650in}}%
\pgfpathlineto{\pgfqpoint{3.079032in}{1.683791in}}%
\pgfpathlineto{\pgfqpoint{3.079847in}{1.708476in}}%
\pgfpathlineto{\pgfqpoint{3.080051in}{1.683791in}}%
\pgfpathlineto{\pgfqpoint{3.080662in}{1.642650in}}%
\pgfpathlineto{\pgfqpoint{3.081070in}{1.692019in}}%
\pgfpathlineto{\pgfqpoint{3.081681in}{1.659106in}}%
\pgfpathlineto{\pgfqpoint{3.082292in}{1.741388in}}%
\pgfpathlineto{\pgfqpoint{3.083923in}{1.617965in}}%
\pgfpathlineto{\pgfqpoint{3.084330in}{1.626194in}}%
\pgfpathlineto{\pgfqpoint{3.085961in}{1.757845in}}%
\pgfpathlineto{\pgfqpoint{3.086368in}{1.766073in}}%
\pgfpathlineto{\pgfqpoint{3.087387in}{1.560368in}}%
\pgfpathlineto{\pgfqpoint{3.087999in}{1.782529in}}%
\pgfpathlineto{\pgfqpoint{3.088406in}{1.683791in}}%
\pgfpathlineto{\pgfqpoint{3.088610in}{1.552140in}}%
\pgfpathlineto{\pgfqpoint{3.088814in}{1.774301in}}%
\pgfpathlineto{\pgfqpoint{3.089425in}{1.683791in}}%
\pgfpathlineto{\pgfqpoint{3.090037in}{1.642650in}}%
\pgfpathlineto{\pgfqpoint{3.090648in}{1.749617in}}%
\pgfpathlineto{\pgfqpoint{3.092075in}{1.659106in}}%
\pgfpathlineto{\pgfqpoint{3.092279in}{1.692019in}}%
\pgfpathlineto{\pgfqpoint{3.092686in}{1.601509in}}%
\pgfpathlineto{\pgfqpoint{3.092890in}{1.642650in}}%
\pgfpathlineto{\pgfqpoint{3.093705in}{1.601509in}}%
\pgfpathlineto{\pgfqpoint{3.094521in}{1.724932in}}%
\pgfpathlineto{\pgfqpoint{3.094928in}{1.675563in}}%
\pgfpathlineto{\pgfqpoint{3.095540in}{1.535683in}}%
\pgfpathlineto{\pgfqpoint{3.096355in}{1.609737in}}%
\pgfpathlineto{\pgfqpoint{3.096762in}{1.576824in}}%
\pgfpathlineto{\pgfqpoint{3.097170in}{1.305294in}}%
\pgfpathlineto{\pgfqpoint{3.098189in}{1.354663in}}%
\pgfpathlineto{\pgfqpoint{3.098393in}{1.346435in}}%
\pgfpathlineto{\pgfqpoint{3.099412in}{1.716704in}}%
\pgfpathlineto{\pgfqpoint{3.099819in}{1.700247in}}%
\pgfpathlineto{\pgfqpoint{3.100227in}{1.716704in}}%
\pgfpathlineto{\pgfqpoint{3.101042in}{1.659106in}}%
\pgfpathlineto{\pgfqpoint{3.101450in}{1.700247in}}%
\pgfpathlineto{\pgfqpoint{3.101654in}{1.601509in}}%
\pgfpathlineto{\pgfqpoint{3.102469in}{1.708476in}}%
\pgfpathlineto{\pgfqpoint{3.102876in}{1.626194in}}%
\pgfpathlineto{\pgfqpoint{3.103284in}{1.568596in}}%
\pgfpathlineto{\pgfqpoint{3.103692in}{1.601509in}}%
\pgfpathlineto{\pgfqpoint{3.104507in}{1.733160in}}%
\pgfpathlineto{\pgfqpoint{3.104303in}{1.576824in}}%
\pgfpathlineto{\pgfqpoint{3.104711in}{1.593281in}}%
\pgfpathlineto{\pgfqpoint{3.105730in}{1.461630in}}%
\pgfpathlineto{\pgfqpoint{3.105934in}{1.478086in}}%
\pgfpathlineto{\pgfqpoint{3.106749in}{1.519227in}}%
\pgfpathlineto{\pgfqpoint{3.106545in}{1.428717in}}%
\pgfpathlineto{\pgfqpoint{3.106953in}{1.478086in}}%
\pgfpathlineto{\pgfqpoint{3.107156in}{1.461630in}}%
\pgfpathlineto{\pgfqpoint{3.107360in}{1.560368in}}%
\pgfpathlineto{\pgfqpoint{3.108175in}{1.494542in}}%
\pgfpathlineto{\pgfqpoint{3.108787in}{1.576824in}}%
\pgfpathlineto{\pgfqpoint{3.108991in}{1.552140in}}%
\pgfpathlineto{\pgfqpoint{3.109602in}{1.461630in}}%
\pgfpathlineto{\pgfqpoint{3.110010in}{1.585053in}}%
\pgfpathlineto{\pgfqpoint{3.111029in}{1.478086in}}%
\pgfpathlineto{\pgfqpoint{3.111232in}{1.535683in}}%
\pgfpathlineto{\pgfqpoint{3.113067in}{1.642650in}}%
\pgfpathlineto{\pgfqpoint{3.113270in}{1.469858in}}%
\pgfpathlineto{\pgfqpoint{3.114086in}{1.683791in}}%
\pgfpathlineto{\pgfqpoint{3.115105in}{1.700247in}}%
\pgfpathlineto{\pgfqpoint{3.115512in}{1.560368in}}%
\pgfpathlineto{\pgfqpoint{3.115716in}{1.659106in}}%
\pgfpathlineto{\pgfqpoint{3.116327in}{1.461630in}}%
\pgfpathlineto{\pgfqpoint{3.116531in}{1.387576in}}%
\pgfpathlineto{\pgfqpoint{3.117346in}{1.461630in}}%
\pgfpathlineto{\pgfqpoint{3.117754in}{1.445173in}}%
\pgfpathlineto{\pgfqpoint{3.118569in}{1.502771in}}%
\pgfpathlineto{\pgfqpoint{3.119385in}{1.428717in}}%
\pgfpathlineto{\pgfqpoint{3.119588in}{1.453401in}}%
\pgfpathlineto{\pgfqpoint{3.120200in}{1.445173in}}%
\pgfpathlineto{\pgfqpoint{3.120811in}{1.510999in}}%
\pgfpathlineto{\pgfqpoint{3.122238in}{1.420489in}}%
\pgfpathlineto{\pgfqpoint{3.123461in}{1.601509in}}%
\pgfpathlineto{\pgfqpoint{3.123868in}{1.329978in}}%
\pgfpathlineto{\pgfqpoint{3.124480in}{1.519227in}}%
\pgfpathlineto{\pgfqpoint{3.126925in}{1.708476in}}%
\pgfpathlineto{\pgfqpoint{3.127129in}{1.683791in}}%
\pgfpathlineto{\pgfqpoint{3.127740in}{1.634422in}}%
\pgfpathlineto{\pgfqpoint{3.128352in}{1.659106in}}%
\pgfpathlineto{\pgfqpoint{3.129167in}{1.757845in}}%
\pgfpathlineto{\pgfqpoint{3.129371in}{1.642650in}}%
\pgfpathlineto{\pgfqpoint{3.130390in}{1.469858in}}%
\pgfpathlineto{\pgfqpoint{3.130594in}{1.494542in}}%
\pgfpathlineto{\pgfqpoint{3.131409in}{1.617965in}}%
\pgfpathlineto{\pgfqpoint{3.131816in}{1.543912in}}%
\pgfpathlineto{\pgfqpoint{3.132224in}{1.502771in}}%
\pgfpathlineto{\pgfqpoint{3.132428in}{1.519227in}}%
\pgfpathlineto{\pgfqpoint{3.132632in}{1.576824in}}%
\pgfpathlineto{\pgfqpoint{3.133039in}{1.494542in}}%
\pgfpathlineto{\pgfqpoint{3.133447in}{1.568596in}}%
\pgfpathlineto{\pgfqpoint{3.135077in}{1.453401in}}%
\pgfpathlineto{\pgfqpoint{3.135281in}{1.428717in}}%
\pgfpathlineto{\pgfqpoint{3.135485in}{1.264153in}}%
\pgfpathlineto{\pgfqpoint{3.135893in}{1.469858in}}%
\pgfpathlineto{\pgfqpoint{3.136300in}{1.445173in}}%
\pgfpathlineto{\pgfqpoint{3.136504in}{1.527455in}}%
\pgfpathlineto{\pgfqpoint{3.137523in}{1.502771in}}%
\pgfpathlineto{\pgfqpoint{3.138134in}{1.206555in}}%
\pgfpathlineto{\pgfqpoint{3.138542in}{1.445173in}}%
\pgfpathlineto{\pgfqpoint{3.139765in}{1.560368in}}%
\pgfpathlineto{\pgfqpoint{3.139969in}{1.527455in}}%
\pgfpathlineto{\pgfqpoint{3.140988in}{1.420489in}}%
\pgfpathlineto{\pgfqpoint{3.141191in}{1.428717in}}%
\pgfpathlineto{\pgfqpoint{3.142618in}{1.568596in}}%
\pgfpathlineto{\pgfqpoint{3.142822in}{1.502771in}}%
\pgfpathlineto{\pgfqpoint{3.143637in}{1.535683in}}%
\pgfpathlineto{\pgfqpoint{3.144248in}{1.576824in}}%
\pgfpathlineto{\pgfqpoint{3.144656in}{1.535683in}}%
\pgfpathlineto{\pgfqpoint{3.144860in}{1.535683in}}%
\pgfpathlineto{\pgfqpoint{3.145471in}{1.510999in}}%
\pgfpathlineto{\pgfqpoint{3.146083in}{1.585053in}}%
\pgfpathlineto{\pgfqpoint{3.147713in}{1.412260in}}%
\pgfpathlineto{\pgfqpoint{3.148528in}{1.469858in}}%
\pgfpathlineto{\pgfqpoint{3.148121in}{1.404032in}}%
\pgfpathlineto{\pgfqpoint{3.148732in}{1.436945in}}%
\pgfpathlineto{\pgfqpoint{3.148936in}{1.428717in}}%
\pgfpathlineto{\pgfqpoint{3.149140in}{1.461630in}}%
\pgfpathlineto{\pgfqpoint{3.149344in}{1.469858in}}%
\pgfpathlineto{\pgfqpoint{3.149547in}{1.445173in}}%
\pgfpathlineto{\pgfqpoint{3.150159in}{1.395804in}}%
\pgfpathlineto{\pgfqpoint{3.149955in}{1.453401in}}%
\pgfpathlineto{\pgfqpoint{3.150566in}{1.428717in}}%
\pgfpathlineto{\pgfqpoint{3.151789in}{1.527455in}}%
\pgfpathlineto{\pgfqpoint{3.151382in}{1.387576in}}%
\pgfpathlineto{\pgfqpoint{3.151993in}{1.519227in}}%
\pgfpathlineto{\pgfqpoint{3.152197in}{1.519227in}}%
\pgfpathlineto{\pgfqpoint{3.152401in}{1.461630in}}%
\pgfpathlineto{\pgfqpoint{3.153216in}{1.543912in}}%
\pgfpathlineto{\pgfqpoint{3.153827in}{1.510999in}}%
\pgfpathlineto{\pgfqpoint{3.154031in}{1.576824in}}%
\pgfpathlineto{\pgfqpoint{3.154235in}{1.543912in}}%
\pgfpathlineto{\pgfqpoint{3.154439in}{2.309134in}}%
\pgfpathlineto{\pgfqpoint{3.154846in}{1.535683in}}%
\pgfpathlineto{\pgfqpoint{3.155254in}{1.634422in}}%
\pgfpathlineto{\pgfqpoint{3.156069in}{1.461630in}}%
\pgfpathlineto{\pgfqpoint{3.156680in}{1.478086in}}%
\pgfpathlineto{\pgfqpoint{3.157292in}{1.543912in}}%
\pgfpathlineto{\pgfqpoint{3.157903in}{1.527455in}}%
\pgfpathlineto{\pgfqpoint{3.158311in}{1.478086in}}%
\pgfpathlineto{\pgfqpoint{3.158922in}{1.527455in}}%
\pgfpathlineto{\pgfqpoint{3.159534in}{1.568596in}}%
\pgfpathlineto{\pgfqpoint{3.160145in}{1.543912in}}%
\pgfpathlineto{\pgfqpoint{3.160349in}{1.543912in}}%
\pgfpathlineto{\pgfqpoint{3.160553in}{1.576824in}}%
\pgfpathlineto{\pgfqpoint{3.160960in}{1.469858in}}%
\pgfpathlineto{\pgfqpoint{3.161164in}{1.502771in}}%
\pgfpathlineto{\pgfqpoint{3.161572in}{1.379348in}}%
\pgfpathlineto{\pgfqpoint{3.161776in}{1.132502in}}%
\pgfpathlineto{\pgfqpoint{3.162387in}{1.412260in}}%
\pgfpathlineto{\pgfqpoint{3.162591in}{1.379348in}}%
\pgfpathlineto{\pgfqpoint{3.162998in}{1.305294in}}%
\pgfpathlineto{\pgfqpoint{3.163610in}{1.362891in}}%
\pgfpathlineto{\pgfqpoint{3.163814in}{1.379348in}}%
\pgfpathlineto{\pgfqpoint{3.164017in}{1.338207in}}%
\pgfpathlineto{\pgfqpoint{3.164221in}{1.305294in}}%
\pgfpathlineto{\pgfqpoint{3.164425in}{1.379348in}}%
\pgfpathlineto{\pgfqpoint{3.165036in}{1.354663in}}%
\pgfpathlineto{\pgfqpoint{3.165444in}{1.486314in}}%
\pgfpathlineto{\pgfqpoint{3.167074in}{1.083132in}}%
\pgfpathlineto{\pgfqpoint{3.168705in}{1.354663in}}%
\pgfpathlineto{\pgfqpoint{3.169316in}{0.877427in}}%
\pgfpathlineto{\pgfqpoint{3.169928in}{1.058448in}}%
\pgfpathlineto{\pgfqpoint{3.170131in}{1.017307in}}%
\pgfpathlineto{\pgfqpoint{3.170539in}{1.107817in}}%
\pgfpathlineto{\pgfqpoint{3.170743in}{1.074904in}}%
\pgfpathlineto{\pgfqpoint{3.171966in}{1.223012in}}%
\pgfpathlineto{\pgfqpoint{3.172169in}{1.140730in}}%
\pgfpathlineto{\pgfqpoint{3.172985in}{1.214784in}}%
\pgfpathlineto{\pgfqpoint{3.172577in}{1.050220in}}%
\pgfpathlineto{\pgfqpoint{3.173392in}{1.190099in}}%
\pgfpathlineto{\pgfqpoint{3.173596in}{1.181871in}}%
\pgfpathlineto{\pgfqpoint{3.173800in}{1.338207in}}%
\pgfpathlineto{\pgfqpoint{3.174615in}{1.206555in}}%
\pgfpathlineto{\pgfqpoint{3.175430in}{1.321750in}}%
\pgfpathlineto{\pgfqpoint{3.176246in}{1.313522in}}%
\pgfpathlineto{\pgfqpoint{3.177468in}{1.083132in}}%
\pgfpathlineto{\pgfqpoint{3.177672in}{1.247696in}}%
\pgfpathlineto{\pgfqpoint{3.178080in}{1.305294in}}%
\pgfpathlineto{\pgfqpoint{3.178895in}{1.198327in}}%
\pgfpathlineto{\pgfqpoint{3.179099in}{1.272381in}}%
\pgfpathlineto{\pgfqpoint{3.179506in}{1.190099in}}%
\pgfpathlineto{\pgfqpoint{3.179710in}{1.206555in}}%
\pgfpathlineto{\pgfqpoint{3.179914in}{1.148958in}}%
\pgfpathlineto{\pgfqpoint{3.180322in}{1.329978in}}%
\pgfpathlineto{\pgfqpoint{3.180729in}{1.214784in}}%
\pgfpathlineto{\pgfqpoint{3.181341in}{1.264153in}}%
\pgfpathlineto{\pgfqpoint{3.181544in}{1.239468in}}%
\pgfpathlineto{\pgfqpoint{3.181952in}{1.025535in}}%
\pgfpathlineto{\pgfqpoint{3.182563in}{1.206555in}}%
\pgfpathlineto{\pgfqpoint{3.182767in}{1.247696in}}%
\pgfpathlineto{\pgfqpoint{3.182971in}{1.239468in}}%
\pgfpathlineto{\pgfqpoint{3.183175in}{1.000850in}}%
\pgfpathlineto{\pgfqpoint{3.183582in}{1.297066in}}%
\pgfpathlineto{\pgfqpoint{3.183990in}{1.297066in}}%
\pgfpathlineto{\pgfqpoint{3.184601in}{1.321750in}}%
\pgfpathlineto{\pgfqpoint{3.185213in}{1.247696in}}%
\pgfpathlineto{\pgfqpoint{3.185620in}{1.313522in}}%
\pgfpathlineto{\pgfqpoint{3.186028in}{1.223012in}}%
\pgfpathlineto{\pgfqpoint{3.186232in}{1.198327in}}%
\pgfpathlineto{\pgfqpoint{3.186436in}{1.223012in}}%
\pgfpathlineto{\pgfqpoint{3.187659in}{1.395804in}}%
\pgfpathlineto{\pgfqpoint{3.187862in}{1.387576in}}%
\pgfpathlineto{\pgfqpoint{3.188066in}{1.387576in}}%
\pgfpathlineto{\pgfqpoint{3.188270in}{1.354663in}}%
\pgfpathlineto{\pgfqpoint{3.188678in}{1.436945in}}%
\pgfpathlineto{\pgfqpoint{3.188881in}{1.445173in}}%
\pgfpathlineto{\pgfqpoint{3.189085in}{1.412260in}}%
\pgfpathlineto{\pgfqpoint{3.189289in}{1.412260in}}%
\pgfpathlineto{\pgfqpoint{3.189697in}{1.387576in}}%
\pgfpathlineto{\pgfqpoint{3.190716in}{1.280609in}}%
\pgfpathlineto{\pgfqpoint{3.190919in}{1.305294in}}%
\pgfpathlineto{\pgfqpoint{3.191735in}{1.231240in}}%
\pgfpathlineto{\pgfqpoint{3.191327in}{1.313522in}}%
\pgfpathlineto{\pgfqpoint{3.191938in}{1.297066in}}%
\pgfpathlineto{\pgfqpoint{3.192142in}{1.362891in}}%
\pgfpathlineto{\pgfqpoint{3.192550in}{1.313522in}}%
\pgfpathlineto{\pgfqpoint{3.192754in}{1.181871in}}%
\pgfpathlineto{\pgfqpoint{3.193773in}{1.214784in}}%
\pgfpathlineto{\pgfqpoint{3.193976in}{1.206555in}}%
\pgfpathlineto{\pgfqpoint{3.194180in}{1.239468in}}%
\pgfpathlineto{\pgfqpoint{3.194384in}{1.223012in}}%
\pgfpathlineto{\pgfqpoint{3.195199in}{1.346435in}}%
\pgfpathlineto{\pgfqpoint{3.194995in}{1.214784in}}%
\pgfpathlineto{\pgfqpoint{3.195607in}{1.329978in}}%
\pgfpathlineto{\pgfqpoint{3.196422in}{1.255925in}}%
\pgfpathlineto{\pgfqpoint{3.197237in}{1.428717in}}%
\pgfpathlineto{\pgfqpoint{3.197645in}{1.412260in}}%
\pgfpathlineto{\pgfqpoint{3.197849in}{1.371119in}}%
\pgfpathlineto{\pgfqpoint{3.198460in}{1.461630in}}%
\pgfpathlineto{\pgfqpoint{3.198664in}{1.412260in}}%
\pgfpathlineto{\pgfqpoint{3.198868in}{1.420489in}}%
\pgfpathlineto{\pgfqpoint{3.199071in}{1.395804in}}%
\pgfpathlineto{\pgfqpoint{3.199683in}{1.223012in}}%
\pgfpathlineto{\pgfqpoint{3.200294in}{1.288837in}}%
\pgfpathlineto{\pgfqpoint{3.200702in}{1.346435in}}%
\pgfpathlineto{\pgfqpoint{3.200906in}{1.321750in}}%
\pgfpathlineto{\pgfqpoint{3.201925in}{1.469858in}}%
\pgfpathlineto{\pgfqpoint{3.202129in}{1.461630in}}%
\pgfpathlineto{\pgfqpoint{3.203555in}{1.206555in}}%
\pgfpathlineto{\pgfqpoint{3.205186in}{1.404032in}}%
\pgfpathlineto{\pgfqpoint{3.205797in}{1.297066in}}%
\pgfpathlineto{\pgfqpoint{3.206205in}{1.395804in}}%
\pgfpathlineto{\pgfqpoint{3.206408in}{1.428717in}}%
\pgfpathlineto{\pgfqpoint{3.206816in}{1.371119in}}%
\pgfpathlineto{\pgfqpoint{3.207224in}{1.387576in}}%
\pgfpathlineto{\pgfqpoint{3.208243in}{1.453401in}}%
\pgfpathlineto{\pgfqpoint{3.208650in}{1.404032in}}%
\pgfpathlineto{\pgfqpoint{3.208854in}{1.404032in}}%
\pgfpathlineto{\pgfqpoint{3.209669in}{1.297066in}}%
\pgfpathlineto{\pgfqpoint{3.209873in}{1.362891in}}%
\pgfpathlineto{\pgfqpoint{3.210077in}{1.379348in}}%
\pgfpathlineto{\pgfqpoint{3.210484in}{1.371119in}}%
\pgfpathlineto{\pgfqpoint{3.210688in}{1.288837in}}%
\pgfpathlineto{\pgfqpoint{3.211300in}{1.436945in}}%
\pgfpathlineto{\pgfqpoint{3.211707in}{1.461630in}}%
\pgfpathlineto{\pgfqpoint{3.211911in}{1.420489in}}%
\pgfpathlineto{\pgfqpoint{3.213134in}{1.601509in}}%
\pgfpathlineto{\pgfqpoint{3.213338in}{1.527455in}}%
\pgfpathlineto{\pgfqpoint{3.214153in}{1.585053in}}%
\pgfpathlineto{\pgfqpoint{3.214561in}{1.617965in}}%
\pgfpathlineto{\pgfqpoint{3.214764in}{1.692019in}}%
\pgfpathlineto{\pgfqpoint{3.214968in}{1.469858in}}%
\pgfpathlineto{\pgfqpoint{3.215783in}{1.675563in}}%
\pgfpathlineto{\pgfqpoint{3.217006in}{1.585053in}}%
\pgfpathlineto{\pgfqpoint{3.217210in}{1.593281in}}%
\pgfpathlineto{\pgfqpoint{3.218229in}{1.716704in}}%
\pgfpathlineto{\pgfqpoint{3.218433in}{1.568596in}}%
\pgfpathlineto{\pgfqpoint{3.219248in}{1.692019in}}%
\pgfpathlineto{\pgfqpoint{3.219452in}{1.667335in}}%
\pgfpathlineto{\pgfqpoint{3.219656in}{1.716704in}}%
\pgfpathlineto{\pgfqpoint{3.219859in}{1.708476in}}%
\pgfpathlineto{\pgfqpoint{3.220675in}{1.798986in}}%
\pgfpathlineto{\pgfqpoint{3.220878in}{1.519227in}}%
\pgfpathlineto{\pgfqpoint{3.221694in}{1.873040in}}%
\pgfpathlineto{\pgfqpoint{3.222916in}{2.029375in}}%
\pgfpathlineto{\pgfqpoint{3.223120in}{1.823670in}}%
\pgfpathlineto{\pgfqpoint{3.223935in}{2.086973in}}%
\pgfpathlineto{\pgfqpoint{3.224343in}{2.111657in}}%
\pgfpathlineto{\pgfqpoint{3.225362in}{1.782529in}}%
\pgfpathlineto{\pgfqpoint{3.225566in}{1.864811in}}%
\pgfpathlineto{\pgfqpoint{3.226381in}{1.708476in}}%
\pgfpathlineto{\pgfqpoint{3.226789in}{2.012919in}}%
\pgfpathlineto{\pgfqpoint{3.227604in}{1.840127in}}%
\pgfpathlineto{\pgfqpoint{3.227400in}{2.037604in}}%
\pgfpathlineto{\pgfqpoint{3.227808in}{2.021147in}}%
\pgfpathlineto{\pgfqpoint{3.228012in}{1.988234in}}%
\pgfpathlineto{\pgfqpoint{3.228215in}{2.070516in}}%
\pgfpathlineto{\pgfqpoint{3.228827in}{2.012919in}}%
\pgfpathlineto{\pgfqpoint{3.229234in}{2.004691in}}%
\pgfpathlineto{\pgfqpoint{3.230050in}{1.831899in}}%
\pgfpathlineto{\pgfqpoint{3.230253in}{2.021147in}}%
\pgfpathlineto{\pgfqpoint{3.230457in}{1.963550in}}%
\pgfpathlineto{\pgfqpoint{3.230661in}{2.021147in}}%
\pgfpathlineto{\pgfqpoint{3.231272in}{1.938865in}}%
\pgfpathlineto{\pgfqpoint{3.231476in}{1.922409in}}%
\pgfpathlineto{\pgfqpoint{3.231680in}{1.955322in}}%
\pgfpathlineto{\pgfqpoint{3.233107in}{2.128114in}}%
\pgfpathlineto{\pgfqpoint{3.233310in}{2.144570in}}%
\pgfpathlineto{\pgfqpoint{3.233514in}{2.111657in}}%
\pgfpathlineto{\pgfqpoint{3.233718in}{2.086973in}}%
\pgfpathlineto{\pgfqpoint{3.234126in}{2.136342in}}%
\pgfpathlineto{\pgfqpoint{3.234329in}{2.136342in}}%
\pgfpathlineto{\pgfqpoint{3.235348in}{2.086973in}}%
\pgfpathlineto{\pgfqpoint{3.235552in}{2.103429in}}%
\pgfpathlineto{\pgfqpoint{3.236164in}{2.202168in}}%
\pgfpathlineto{\pgfqpoint{3.236979in}{2.161027in}}%
\pgfpathlineto{\pgfqpoint{3.237183in}{2.161027in}}%
\pgfpathlineto{\pgfqpoint{3.237590in}{2.210396in}}%
\pgfpathlineto{\pgfqpoint{3.237794in}{2.136342in}}%
\pgfpathlineto{\pgfqpoint{3.237998in}{2.136342in}}%
\pgfpathlineto{\pgfqpoint{3.239628in}{2.193939in}}%
\pgfpathlineto{\pgfqpoint{3.239832in}{2.136342in}}%
\pgfpathlineto{\pgfqpoint{3.240444in}{2.251537in}}%
\pgfpathlineto{\pgfqpoint{3.240851in}{2.235080in}}%
\pgfpathlineto{\pgfqpoint{3.241055in}{2.243309in}}%
\pgfpathlineto{\pgfqpoint{3.241259in}{2.161027in}}%
\pgfpathlineto{\pgfqpoint{3.241870in}{2.292678in}}%
\pgfpathlineto{\pgfqpoint{3.242074in}{2.251537in}}%
\pgfpathlineto{\pgfqpoint{3.242278in}{2.243309in}}%
\pgfpathlineto{\pgfqpoint{3.242482in}{2.284450in}}%
\pgfpathlineto{\pgfqpoint{3.242685in}{2.136342in}}%
\pgfpathlineto{\pgfqpoint{3.243501in}{2.021147in}}%
\pgfpathlineto{\pgfqpoint{3.243704in}{2.128114in}}%
\pgfpathlineto{\pgfqpoint{3.244723in}{2.267993in}}%
\pgfpathlineto{\pgfqpoint{3.244927in}{2.259765in}}%
\pgfpathlineto{\pgfqpoint{3.245131in}{2.292678in}}%
\pgfpathlineto{\pgfqpoint{3.245335in}{2.243309in}}%
\pgfpathlineto{\pgfqpoint{3.245539in}{2.251537in}}%
\pgfpathlineto{\pgfqpoint{3.246761in}{2.045832in}}%
\pgfpathlineto{\pgfqpoint{3.247984in}{2.177483in}}%
\pgfpathlineto{\pgfqpoint{3.248188in}{1.938865in}}%
\pgfpathlineto{\pgfqpoint{3.249003in}{2.119886in}}%
\pgfpathlineto{\pgfqpoint{3.249207in}{2.128114in}}%
\pgfpathlineto{\pgfqpoint{3.250226in}{1.971778in}}%
\pgfpathlineto{\pgfqpoint{3.250634in}{2.045832in}}%
\pgfpathlineto{\pgfqpoint{3.250837in}{2.004691in}}%
\pgfpathlineto{\pgfqpoint{3.251041in}{2.054060in}}%
\pgfpathlineto{\pgfqpoint{3.251449in}{2.045832in}}%
\pgfpathlineto{\pgfqpoint{3.252672in}{2.210396in}}%
\pgfpathlineto{\pgfqpoint{3.252875in}{1.980006in}}%
\pgfpathlineto{\pgfqpoint{3.253487in}{2.292678in}}%
\pgfpathlineto{\pgfqpoint{3.253691in}{2.292678in}}%
\pgfpathlineto{\pgfqpoint{3.254914in}{2.078745in}}%
\pgfpathlineto{\pgfqpoint{3.255321in}{2.128114in}}%
\pgfpathlineto{\pgfqpoint{3.255525in}{2.128114in}}%
\pgfpathlineto{\pgfqpoint{3.255729in}{2.119886in}}%
\pgfpathlineto{\pgfqpoint{3.257155in}{2.243309in}}%
\pgfpathlineto{\pgfqpoint{3.258174in}{2.300906in}}%
\pgfpathlineto{\pgfqpoint{3.259601in}{2.078745in}}%
\pgfpathlineto{\pgfqpoint{3.260212in}{2.317363in}}%
\pgfpathlineto{\pgfqpoint{3.260824in}{2.300906in}}%
\pgfpathlineto{\pgfqpoint{3.261028in}{2.243309in}}%
\pgfpathlineto{\pgfqpoint{3.261639in}{2.317363in}}%
\pgfpathlineto{\pgfqpoint{3.261843in}{2.284450in}}%
\pgfpathlineto{\pgfqpoint{3.262047in}{2.292678in}}%
\pgfpathlineto{\pgfqpoint{3.262250in}{2.276221in}}%
\pgfpathlineto{\pgfqpoint{3.262658in}{2.276221in}}%
\pgfpathlineto{\pgfqpoint{3.262862in}{2.235080in}}%
\pgfpathlineto{\pgfqpoint{3.263269in}{2.309134in}}%
\pgfpathlineto{\pgfqpoint{3.263677in}{2.292678in}}%
\pgfpathlineto{\pgfqpoint{3.264696in}{2.391416in}}%
\pgfpathlineto{\pgfqpoint{3.265307in}{2.243309in}}%
\pgfpathlineto{\pgfqpoint{3.266123in}{2.276221in}}%
\pgfpathlineto{\pgfqpoint{3.266938in}{2.399645in}}%
\pgfpathlineto{\pgfqpoint{3.267346in}{2.309134in}}%
\pgfpathlineto{\pgfqpoint{3.267549in}{2.325591in}}%
\pgfpathlineto{\pgfqpoint{3.267753in}{2.267993in}}%
\pgfpathlineto{\pgfqpoint{3.268161in}{2.300906in}}%
\pgfpathlineto{\pgfqpoint{3.268568in}{2.259765in}}%
\pgfpathlineto{\pgfqpoint{3.269180in}{2.185711in}}%
\pgfpathlineto{\pgfqpoint{3.269587in}{2.259765in}}%
\pgfpathlineto{\pgfqpoint{3.270606in}{2.317363in}}%
\pgfpathlineto{\pgfqpoint{3.270403in}{2.243309in}}%
\pgfpathlineto{\pgfqpoint{3.270810in}{2.300906in}}%
\pgfpathlineto{\pgfqpoint{3.271014in}{2.292678in}}%
\pgfpathlineto{\pgfqpoint{3.271218in}{2.325591in}}%
\pgfpathlineto{\pgfqpoint{3.271422in}{2.012919in}}%
\pgfpathlineto{\pgfqpoint{3.272237in}{2.366732in}}%
\pgfpathlineto{\pgfqpoint{3.273256in}{2.391416in}}%
\pgfpathlineto{\pgfqpoint{3.273460in}{2.383188in}}%
\pgfpathlineto{\pgfqpoint{3.274275in}{2.267993in}}%
\pgfpathlineto{\pgfqpoint{3.274682in}{2.325591in}}%
\pgfpathlineto{\pgfqpoint{3.274886in}{2.350275in}}%
\pgfpathlineto{\pgfqpoint{3.275294in}{2.136342in}}%
\pgfpathlineto{\pgfqpoint{3.275498in}{2.399645in}}%
\pgfpathlineto{\pgfqpoint{3.275905in}{2.333819in}}%
\pgfpathlineto{\pgfqpoint{3.276109in}{2.333819in}}%
\pgfpathlineto{\pgfqpoint{3.277332in}{2.481927in}}%
\pgfpathlineto{\pgfqpoint{3.276720in}{2.276221in}}%
\pgfpathlineto{\pgfqpoint{3.277739in}{2.432557in}}%
\pgfpathlineto{\pgfqpoint{3.278147in}{2.465470in}}%
\pgfpathlineto{\pgfqpoint{3.278351in}{2.391416in}}%
\pgfpathlineto{\pgfqpoint{3.278555in}{2.259765in}}%
\pgfpathlineto{\pgfqpoint{3.279166in}{2.432557in}}%
\pgfpathlineto{\pgfqpoint{3.279574in}{2.473698in}}%
\pgfpathlineto{\pgfqpoint{3.279777in}{2.424329in}}%
\pgfpathlineto{\pgfqpoint{3.280185in}{2.432557in}}%
\pgfpathlineto{\pgfqpoint{3.280389in}{2.432557in}}%
\pgfpathlineto{\pgfqpoint{3.281612in}{2.333819in}}%
\pgfpathlineto{\pgfqpoint{3.281816in}{2.333819in}}%
\pgfpathlineto{\pgfqpoint{3.282427in}{2.325591in}}%
\pgfpathlineto{\pgfqpoint{3.283038in}{2.416101in}}%
\pgfpathlineto{\pgfqpoint{3.283242in}{2.465470in}}%
\pgfpathlineto{\pgfqpoint{3.283854in}{2.407873in}}%
\pgfpathlineto{\pgfqpoint{3.284873in}{2.276221in}}%
\pgfpathlineto{\pgfqpoint{3.285280in}{2.317363in}}%
\pgfpathlineto{\pgfqpoint{3.285484in}{2.317363in}}%
\pgfpathlineto{\pgfqpoint{3.285892in}{2.284450in}}%
\pgfpathlineto{\pgfqpoint{3.286299in}{2.342047in}}%
\pgfpathlineto{\pgfqpoint{3.286503in}{2.366732in}}%
\pgfpathlineto{\pgfqpoint{3.286911in}{2.300906in}}%
\pgfpathlineto{\pgfqpoint{3.287114in}{2.309134in}}%
\pgfpathlineto{\pgfqpoint{3.288541in}{2.432557in}}%
\pgfpathlineto{\pgfqpoint{3.289356in}{2.407873in}}%
\pgfpathlineto{\pgfqpoint{3.289968in}{2.317363in}}%
\pgfpathlineto{\pgfqpoint{3.290579in}{2.342047in}}%
\pgfpathlineto{\pgfqpoint{3.290783in}{2.342047in}}%
\pgfpathlineto{\pgfqpoint{3.291190in}{2.276221in}}%
\pgfpathlineto{\pgfqpoint{3.291802in}{2.358504in}}%
\pgfpathlineto{\pgfqpoint{3.292209in}{2.407873in}}%
\pgfpathlineto{\pgfqpoint{3.292821in}{2.358504in}}%
\pgfpathlineto{\pgfqpoint{3.293025in}{2.358504in}}%
\pgfpathlineto{\pgfqpoint{3.294044in}{2.284450in}}%
\pgfpathlineto{\pgfqpoint{3.294451in}{2.292678in}}%
\pgfpathlineto{\pgfqpoint{3.295063in}{2.366732in}}%
\pgfpathlineto{\pgfqpoint{3.295267in}{2.309134in}}%
\pgfpathlineto{\pgfqpoint{3.296082in}{2.161027in}}%
\pgfpathlineto{\pgfqpoint{3.296286in}{2.218624in}}%
\pgfpathlineto{\pgfqpoint{3.297101in}{2.284450in}}%
\pgfpathlineto{\pgfqpoint{3.297305in}{2.226852in}}%
\pgfpathlineto{\pgfqpoint{3.297508in}{2.226852in}}%
\pgfpathlineto{\pgfqpoint{3.297916in}{2.276221in}}%
\pgfpathlineto{\pgfqpoint{3.298527in}{2.342047in}}%
\pgfpathlineto{\pgfqpoint{3.298935in}{2.292678in}}%
\pgfpathlineto{\pgfqpoint{3.299954in}{2.366732in}}%
\pgfpathlineto{\pgfqpoint{3.300362in}{2.333819in}}%
\pgfpathlineto{\pgfqpoint{3.301381in}{2.259765in}}%
\pgfpathlineto{\pgfqpoint{3.301584in}{2.292678in}}%
\pgfpathlineto{\pgfqpoint{3.301788in}{2.350275in}}%
\pgfpathlineto{\pgfqpoint{3.302196in}{2.243309in}}%
\pgfpathlineto{\pgfqpoint{3.302807in}{2.325591in}}%
\pgfpathlineto{\pgfqpoint{3.303419in}{2.416101in}}%
\pgfpathlineto{\pgfqpoint{3.303826in}{2.325591in}}%
\pgfpathlineto{\pgfqpoint{3.304641in}{2.317363in}}%
\pgfpathlineto{\pgfqpoint{3.305049in}{2.342047in}}%
\pgfpathlineto{\pgfqpoint{3.305457in}{2.276221in}}%
\pgfpathlineto{\pgfqpoint{3.306272in}{2.309134in}}%
\pgfpathlineto{\pgfqpoint{3.306679in}{2.333819in}}%
\pgfpathlineto{\pgfqpoint{3.307087in}{2.218624in}}%
\pgfpathlineto{\pgfqpoint{3.307291in}{2.004691in}}%
\pgfpathlineto{\pgfqpoint{3.308106in}{2.218624in}}%
\pgfpathlineto{\pgfqpoint{3.308310in}{2.243309in}}%
\pgfpathlineto{\pgfqpoint{3.308718in}{2.226852in}}%
\pgfpathlineto{\pgfqpoint{3.308921in}{2.177483in}}%
\pgfpathlineto{\pgfqpoint{3.309125in}{2.243309in}}%
\pgfpathlineto{\pgfqpoint{3.309329in}{2.226852in}}%
\pgfpathlineto{\pgfqpoint{3.309940in}{2.350275in}}%
\pgfpathlineto{\pgfqpoint{3.310552in}{2.300906in}}%
\pgfpathlineto{\pgfqpoint{3.310756in}{2.309134in}}%
\pgfpathlineto{\pgfqpoint{3.310959in}{2.300906in}}%
\pgfpathlineto{\pgfqpoint{3.311367in}{2.062288in}}%
\pgfpathlineto{\pgfqpoint{3.311978in}{2.366732in}}%
\pgfpathlineto{\pgfqpoint{3.312182in}{2.193939in}}%
\pgfpathlineto{\pgfqpoint{3.313201in}{2.383188in}}%
\pgfpathlineto{\pgfqpoint{3.313609in}{2.342047in}}%
\pgfpathlineto{\pgfqpoint{3.313813in}{2.276221in}}%
\pgfpathlineto{\pgfqpoint{3.314628in}{2.350275in}}%
\pgfpathlineto{\pgfqpoint{3.315239in}{2.267993in}}%
\pgfpathlineto{\pgfqpoint{3.315443in}{2.284450in}}%
\pgfpathlineto{\pgfqpoint{3.315647in}{2.062288in}}%
\pgfpathlineto{\pgfqpoint{3.316054in}{2.300906in}}%
\pgfpathlineto{\pgfqpoint{3.316462in}{2.251537in}}%
\pgfpathlineto{\pgfqpoint{3.316666in}{2.292678in}}%
\pgfpathlineto{\pgfqpoint{3.316870in}{2.086973in}}%
\pgfpathlineto{\pgfqpoint{3.317481in}{2.416101in}}%
\pgfpathlineto{\pgfqpoint{3.317685in}{2.374960in}}%
\pgfpathlineto{\pgfqpoint{3.318908in}{2.317363in}}%
\pgfpathlineto{\pgfqpoint{3.319723in}{2.399645in}}%
\pgfpathlineto{\pgfqpoint{3.319927in}{2.383188in}}%
\pgfpathlineto{\pgfqpoint{3.320334in}{2.416101in}}%
\pgfpathlineto{\pgfqpoint{3.320946in}{2.325591in}}%
\pgfpathlineto{\pgfqpoint{3.321557in}{2.309134in}}%
\pgfpathlineto{\pgfqpoint{3.322169in}{2.383188in}}%
\pgfpathlineto{\pgfqpoint{3.323188in}{2.333819in}}%
\pgfpathlineto{\pgfqpoint{3.323595in}{2.342047in}}%
\pgfpathlineto{\pgfqpoint{3.323799in}{2.259765in}}%
\pgfpathlineto{\pgfqpoint{3.324818in}{2.276221in}}%
\pgfpathlineto{\pgfqpoint{3.325633in}{2.342047in}}%
\pgfpathlineto{\pgfqpoint{3.326041in}{2.325591in}}%
\pgfpathlineto{\pgfqpoint{3.326245in}{2.309134in}}%
\pgfpathlineto{\pgfqpoint{3.326448in}{2.333819in}}%
\pgfpathlineto{\pgfqpoint{3.327467in}{2.407873in}}%
\pgfpathlineto{\pgfqpoint{3.327671in}{2.366732in}}%
\pgfpathlineto{\pgfqpoint{3.328486in}{2.374960in}}%
\pgfpathlineto{\pgfqpoint{3.329302in}{2.333819in}}%
\pgfpathlineto{\pgfqpoint{3.330117in}{2.399645in}}%
\pgfpathlineto{\pgfqpoint{3.330321in}{2.342047in}}%
\pgfpathlineto{\pgfqpoint{3.330728in}{2.333819in}}%
\pgfpathlineto{\pgfqpoint{3.330932in}{2.342047in}}%
\pgfpathlineto{\pgfqpoint{3.331747in}{2.424329in}}%
\pgfpathlineto{\pgfqpoint{3.331951in}{2.399645in}}%
\pgfpathlineto{\pgfqpoint{3.332359in}{2.202168in}}%
\pgfpathlineto{\pgfqpoint{3.333174in}{2.342047in}}%
\pgfpathlineto{\pgfqpoint{3.333785in}{2.300906in}}%
\pgfpathlineto{\pgfqpoint{3.333581in}{2.358504in}}%
\pgfpathlineto{\pgfqpoint{3.333989in}{2.350275in}}%
\pgfpathlineto{\pgfqpoint{3.334601in}{2.342047in}}%
\pgfpathlineto{\pgfqpoint{3.335416in}{2.416101in}}%
\pgfpathlineto{\pgfqpoint{3.335620in}{2.416101in}}%
\pgfpathlineto{\pgfqpoint{3.336027in}{2.350275in}}%
\pgfpathlineto{\pgfqpoint{3.336639in}{2.416101in}}%
\pgfpathlineto{\pgfqpoint{3.337658in}{2.342047in}}%
\pgfpathlineto{\pgfqpoint{3.337046in}{2.440786in}}%
\pgfpathlineto{\pgfqpoint{3.337861in}{2.383188in}}%
\pgfpathlineto{\pgfqpoint{3.338065in}{2.383188in}}%
\pgfpathlineto{\pgfqpoint{3.338880in}{2.424329in}}%
\pgfpathlineto{\pgfqpoint{3.338677in}{2.358504in}}%
\pgfpathlineto{\pgfqpoint{3.339084in}{2.407873in}}%
\pgfpathlineto{\pgfqpoint{3.339696in}{2.333819in}}%
\pgfpathlineto{\pgfqpoint{3.340307in}{2.342047in}}%
\pgfpathlineto{\pgfqpoint{3.341122in}{2.416101in}}%
\pgfpathlineto{\pgfqpoint{3.341326in}{2.399645in}}%
\pgfpathlineto{\pgfqpoint{3.342345in}{2.325591in}}%
\pgfpathlineto{\pgfqpoint{3.342549in}{2.358504in}}%
\pgfpathlineto{\pgfqpoint{3.342753in}{2.366732in}}%
\pgfpathlineto{\pgfqpoint{3.343568in}{2.259765in}}%
\pgfpathlineto{\pgfqpoint{3.343975in}{2.284450in}}%
\pgfpathlineto{\pgfqpoint{3.344179in}{2.309134in}}%
\pgfpathlineto{\pgfqpoint{3.344791in}{2.259765in}}%
\pgfpathlineto{\pgfqpoint{3.344994in}{2.243309in}}%
\pgfpathlineto{\pgfqpoint{3.345198in}{2.309134in}}%
\pgfpathlineto{\pgfqpoint{3.345402in}{2.292678in}}%
\pgfpathlineto{\pgfqpoint{3.346217in}{2.366732in}}%
\pgfpathlineto{\pgfqpoint{3.346625in}{2.342047in}}%
\pgfpathlineto{\pgfqpoint{3.347236in}{2.383188in}}%
\pgfpathlineto{\pgfqpoint{3.347644in}{2.350275in}}%
\pgfpathlineto{\pgfqpoint{3.347848in}{2.342047in}}%
\pgfpathlineto{\pgfqpoint{3.348052in}{2.374960in}}%
\pgfpathlineto{\pgfqpoint{3.348255in}{2.449014in}}%
\pgfpathlineto{\pgfqpoint{3.349071in}{2.342047in}}%
\pgfpathlineto{\pgfqpoint{3.349274in}{2.309134in}}%
\pgfpathlineto{\pgfqpoint{3.349478in}{2.391416in}}%
\pgfpathlineto{\pgfqpoint{3.349682in}{2.374960in}}%
\pgfpathlineto{\pgfqpoint{3.350090in}{2.416101in}}%
\pgfpathlineto{\pgfqpoint{3.350497in}{2.366732in}}%
\pgfpathlineto{\pgfqpoint{3.351924in}{2.284450in}}%
\pgfpathlineto{\pgfqpoint{3.352128in}{2.300906in}}%
\pgfpathlineto{\pgfqpoint{3.352331in}{2.300906in}}%
\pgfpathlineto{\pgfqpoint{3.352535in}{2.292678in}}%
\pgfpathlineto{\pgfqpoint{3.352739in}{2.325591in}}%
\pgfpathlineto{\pgfqpoint{3.352943in}{2.325591in}}%
\pgfpathlineto{\pgfqpoint{3.353147in}{2.284450in}}%
\pgfpathlineto{\pgfqpoint{3.353554in}{2.358504in}}%
\pgfpathlineto{\pgfqpoint{3.353758in}{2.424329in}}%
\pgfpathlineto{\pgfqpoint{3.354777in}{2.407873in}}%
\pgfpathlineto{\pgfqpoint{3.354981in}{2.440786in}}%
\pgfpathlineto{\pgfqpoint{3.355388in}{2.383188in}}%
\pgfpathlineto{\pgfqpoint{3.355796in}{2.407873in}}%
\pgfpathlineto{\pgfqpoint{3.356407in}{2.416101in}}%
\pgfpathlineto{\pgfqpoint{3.356611in}{2.374960in}}%
\pgfpathlineto{\pgfqpoint{3.357223in}{2.465470in}}%
\pgfpathlineto{\pgfqpoint{3.357426in}{2.432557in}}%
\pgfpathlineto{\pgfqpoint{3.358242in}{2.473698in}}%
\pgfpathlineto{\pgfqpoint{3.358445in}{2.407873in}}%
\pgfpathlineto{\pgfqpoint{3.359261in}{2.498383in}}%
\pgfpathlineto{\pgfqpoint{3.359464in}{2.457242in}}%
\pgfpathlineto{\pgfqpoint{3.359668in}{2.523068in}}%
\pgfpathlineto{\pgfqpoint{3.360280in}{2.481927in}}%
\pgfpathlineto{\pgfqpoint{3.360483in}{2.498383in}}%
\pgfpathlineto{\pgfqpoint{3.360891in}{2.465470in}}%
\pgfpathlineto{\pgfqpoint{3.361503in}{2.391416in}}%
\pgfpathlineto{\pgfqpoint{3.361910in}{2.432557in}}%
\pgfpathlineto{\pgfqpoint{3.363133in}{2.506611in}}%
\pgfpathlineto{\pgfqpoint{3.363948in}{2.399645in}}%
\pgfpathlineto{\pgfqpoint{3.364560in}{2.465470in}}%
\pgfpathlineto{\pgfqpoint{3.364763in}{2.506611in}}%
\pgfpathlineto{\pgfqpoint{3.365579in}{2.465470in}}%
\pgfpathlineto{\pgfqpoint{3.367005in}{2.399645in}}%
\pgfpathlineto{\pgfqpoint{3.367209in}{2.383188in}}%
\pgfpathlineto{\pgfqpoint{3.367413in}{2.407873in}}%
\pgfpathlineto{\pgfqpoint{3.367820in}{2.498383in}}%
\pgfpathlineto{\pgfqpoint{3.368432in}{2.391416in}}%
\pgfpathlineto{\pgfqpoint{3.368636in}{2.391416in}}%
\pgfpathlineto{\pgfqpoint{3.368839in}{2.383188in}}%
\pgfpathlineto{\pgfqpoint{3.369655in}{2.506611in}}%
\pgfpathlineto{\pgfqpoint{3.370266in}{2.473698in}}%
\pgfpathlineto{\pgfqpoint{3.370470in}{2.514839in}}%
\pgfpathlineto{\pgfqpoint{3.370877in}{2.399645in}}%
\pgfpathlineto{\pgfqpoint{3.371081in}{2.383188in}}%
\pgfpathlineto{\pgfqpoint{3.371285in}{2.424329in}}%
\pgfpathlineto{\pgfqpoint{3.371489in}{2.416101in}}%
\pgfpathlineto{\pgfqpoint{3.372712in}{2.490155in}}%
\pgfpathlineto{\pgfqpoint{3.374546in}{2.391416in}}%
\pgfpathlineto{\pgfqpoint{3.374750in}{2.399645in}}%
\pgfpathlineto{\pgfqpoint{3.375769in}{2.457242in}}%
\pgfpathlineto{\pgfqpoint{3.375973in}{2.432557in}}%
\pgfpathlineto{\pgfqpoint{3.376992in}{2.317363in}}%
\pgfpathlineto{\pgfqpoint{3.376380in}{2.465470in}}%
\pgfpathlineto{\pgfqpoint{3.377399in}{2.325591in}}%
\pgfpathlineto{\pgfqpoint{3.377603in}{2.350275in}}%
\pgfpathlineto{\pgfqpoint{3.378011in}{2.292678in}}%
\pgfpathlineto{\pgfqpoint{3.378214in}{2.317363in}}%
\pgfpathlineto{\pgfqpoint{3.379437in}{2.284450in}}%
\pgfpathlineto{\pgfqpoint{3.380456in}{2.473698in}}%
\pgfpathlineto{\pgfqpoint{3.381068in}{2.267993in}}%
\pgfpathlineto{\pgfqpoint{3.381679in}{2.317363in}}%
\pgfpathlineto{\pgfqpoint{3.383106in}{2.391416in}}%
\pgfpathlineto{\pgfqpoint{3.383309in}{2.374960in}}%
\pgfpathlineto{\pgfqpoint{3.384940in}{2.111657in}}%
\pgfpathlineto{\pgfqpoint{3.386163in}{2.449014in}}%
\pgfpathlineto{\pgfqpoint{3.386366in}{2.440786in}}%
\pgfpathlineto{\pgfqpoint{3.386570in}{2.449014in}}%
\pgfpathlineto{\pgfqpoint{3.387793in}{2.531296in}}%
\pgfpathlineto{\pgfqpoint{3.387997in}{2.309134in}}%
\pgfpathlineto{\pgfqpoint{3.388812in}{2.432557in}}%
\pgfpathlineto{\pgfqpoint{3.389424in}{2.374960in}}%
\pgfpathlineto{\pgfqpoint{3.389627in}{2.193939in}}%
\pgfpathlineto{\pgfqpoint{3.390239in}{2.399645in}}%
\pgfpathlineto{\pgfqpoint{3.390443in}{2.366732in}}%
\pgfpathlineto{\pgfqpoint{3.390646in}{2.358504in}}%
\pgfpathlineto{\pgfqpoint{3.390850in}{2.366732in}}%
\pgfpathlineto{\pgfqpoint{3.391869in}{2.481927in}}%
\pgfpathlineto{\pgfqpoint{3.391258in}{2.193939in}}%
\pgfpathlineto{\pgfqpoint{3.392073in}{2.449014in}}%
\pgfpathlineto{\pgfqpoint{3.392481in}{2.152798in}}%
\pgfpathlineto{\pgfqpoint{3.392888in}{2.383188in}}%
\pgfpathlineto{\pgfqpoint{3.393907in}{2.539524in}}%
\pgfpathlineto{\pgfqpoint{3.394315in}{2.259765in}}%
\pgfpathlineto{\pgfqpoint{3.395130in}{2.449014in}}%
\pgfpathlineto{\pgfqpoint{3.395334in}{2.523068in}}%
\pgfpathlineto{\pgfqpoint{3.396149in}{2.440786in}}%
\pgfpathlineto{\pgfqpoint{3.397168in}{2.144570in}}%
\pgfpathlineto{\pgfqpoint{3.397372in}{2.284450in}}%
\pgfpathlineto{\pgfqpoint{3.397576in}{2.465470in}}%
\pgfpathlineto{\pgfqpoint{3.398595in}{2.432557in}}%
\pgfpathlineto{\pgfqpoint{3.399206in}{2.473698in}}%
\pgfpathlineto{\pgfqpoint{3.399817in}{2.350275in}}%
\pgfpathlineto{\pgfqpoint{3.400021in}{2.440786in}}%
\pgfpathlineto{\pgfqpoint{3.400429in}{2.193939in}}%
\pgfpathlineto{\pgfqpoint{3.401040in}{2.416101in}}%
\pgfpathlineto{\pgfqpoint{3.402059in}{2.358504in}}%
\pgfpathlineto{\pgfqpoint{3.402263in}{2.399645in}}%
\pgfpathlineto{\pgfqpoint{3.402875in}{2.449014in}}%
\pgfpathlineto{\pgfqpoint{3.403078in}{2.358504in}}%
\pgfpathlineto{\pgfqpoint{3.403282in}{2.325591in}}%
\pgfpathlineto{\pgfqpoint{3.403894in}{2.399645in}}%
\pgfpathlineto{\pgfqpoint{3.405116in}{2.333819in}}%
\pgfpathlineto{\pgfqpoint{3.405320in}{2.350275in}}%
\pgfpathlineto{\pgfqpoint{3.405728in}{2.391416in}}%
\pgfpathlineto{\pgfqpoint{3.405932in}{2.342047in}}%
\pgfpathlineto{\pgfqpoint{3.406135in}{2.251537in}}%
\pgfpathlineto{\pgfqpoint{3.406747in}{2.424329in}}%
\pgfpathlineto{\pgfqpoint{3.406951in}{2.440786in}}%
\pgfpathlineto{\pgfqpoint{3.407154in}{2.399645in}}%
\pgfpathlineto{\pgfqpoint{3.407562in}{2.416101in}}%
\pgfpathlineto{\pgfqpoint{3.408581in}{2.317363in}}%
\pgfpathlineto{\pgfqpoint{3.408785in}{2.366732in}}%
\pgfpathlineto{\pgfqpoint{3.409396in}{2.317363in}}%
\pgfpathlineto{\pgfqpoint{3.409600in}{2.366732in}}%
\pgfpathlineto{\pgfqpoint{3.410415in}{2.416101in}}%
\pgfpathlineto{\pgfqpoint{3.411230in}{2.333819in}}%
\pgfpathlineto{\pgfqpoint{3.411434in}{2.358504in}}%
\pgfpathlineto{\pgfqpoint{3.411842in}{2.391416in}}%
\pgfpathlineto{\pgfqpoint{3.412046in}{2.342047in}}%
\pgfpathlineto{\pgfqpoint{3.412249in}{2.202168in}}%
\pgfpathlineto{\pgfqpoint{3.412861in}{2.358504in}}%
\pgfpathlineto{\pgfqpoint{3.413065in}{2.317363in}}%
\pgfpathlineto{\pgfqpoint{3.413268in}{2.342047in}}%
\pgfpathlineto{\pgfqpoint{3.413472in}{2.161027in}}%
\pgfpathlineto{\pgfqpoint{3.414287in}{2.276221in}}%
\pgfpathlineto{\pgfqpoint{3.415510in}{2.374960in}}%
\pgfpathlineto{\pgfqpoint{3.415714in}{2.366732in}}%
\pgfpathlineto{\pgfqpoint{3.416122in}{2.350275in}}%
\pgfpathlineto{\pgfqpoint{3.416529in}{2.366732in}}%
\pgfpathlineto{\pgfqpoint{3.417548in}{2.407873in}}%
\pgfpathlineto{\pgfqpoint{3.417752in}{2.374960in}}%
\pgfpathlineto{\pgfqpoint{3.417956in}{2.432557in}}%
\pgfpathlineto{\pgfqpoint{3.418160in}{2.399645in}}%
\pgfpathlineto{\pgfqpoint{3.418975in}{2.465470in}}%
\pgfpathlineto{\pgfqpoint{3.419179in}{2.399645in}}%
\pgfpathlineto{\pgfqpoint{3.419586in}{2.374960in}}%
\pgfpathlineto{\pgfqpoint{3.419790in}{2.473698in}}%
\pgfpathlineto{\pgfqpoint{3.420605in}{2.383188in}}%
\pgfpathlineto{\pgfqpoint{3.421421in}{2.391416in}}%
\pgfpathlineto{\pgfqpoint{3.422032in}{2.333819in}}%
\pgfpathlineto{\pgfqpoint{3.423051in}{2.424329in}}%
\pgfpathlineto{\pgfqpoint{3.422440in}{2.276221in}}%
\pgfpathlineto{\pgfqpoint{3.423255in}{2.391416in}}%
\pgfpathlineto{\pgfqpoint{3.423459in}{2.309134in}}%
\pgfpathlineto{\pgfqpoint{3.424274in}{2.399645in}}%
\pgfpathlineto{\pgfqpoint{3.424885in}{2.374960in}}%
\pgfpathlineto{\pgfqpoint{3.425293in}{2.440786in}}%
\pgfpathlineto{\pgfqpoint{3.425497in}{2.399645in}}%
\pgfpathlineto{\pgfqpoint{3.425904in}{2.416101in}}%
\pgfpathlineto{\pgfqpoint{3.426516in}{2.309134in}}%
\pgfpathlineto{\pgfqpoint{3.427127in}{2.539524in}}%
\pgfpathlineto{\pgfqpoint{3.428961in}{2.309134in}}%
\pgfpathlineto{\pgfqpoint{3.429165in}{2.342047in}}%
\pgfpathlineto{\pgfqpoint{3.429573in}{2.267993in}}%
\pgfpathlineto{\pgfqpoint{3.430388in}{2.177483in}}%
\pgfpathlineto{\pgfqpoint{3.430796in}{2.218624in}}%
\pgfpathlineto{\pgfqpoint{3.430999in}{2.284450in}}%
\pgfpathlineto{\pgfqpoint{3.431203in}{2.210396in}}%
\pgfpathlineto{\pgfqpoint{3.431611in}{2.243309in}}%
\pgfpathlineto{\pgfqpoint{3.431815in}{2.161027in}}%
\pgfpathlineto{\pgfqpoint{3.432426in}{2.259765in}}%
\pgfpathlineto{\pgfqpoint{3.432630in}{2.235080in}}%
\pgfpathlineto{\pgfqpoint{3.433241in}{1.963550in}}%
\pgfpathlineto{\pgfqpoint{3.433853in}{2.185711in}}%
\pgfpathlineto{\pgfqpoint{3.434260in}{2.202168in}}%
\pgfpathlineto{\pgfqpoint{3.434464in}{2.128114in}}%
\pgfpathlineto{\pgfqpoint{3.435075in}{2.210396in}}%
\pgfpathlineto{\pgfqpoint{3.435279in}{2.185711in}}%
\pgfpathlineto{\pgfqpoint{3.436502in}{2.309134in}}%
\pgfpathlineto{\pgfqpoint{3.436706in}{2.251537in}}%
\pgfpathlineto{\pgfqpoint{3.437929in}{2.333819in}}%
\pgfpathlineto{\pgfqpoint{3.438948in}{2.218624in}}%
\pgfpathlineto{\pgfqpoint{3.439151in}{2.276221in}}%
\pgfpathlineto{\pgfqpoint{3.440986in}{2.078745in}}%
\pgfpathlineto{\pgfqpoint{3.441597in}{2.029375in}}%
\pgfpathlineto{\pgfqpoint{3.441801in}{2.054060in}}%
\pgfpathlineto{\pgfqpoint{3.443024in}{2.152798in}}%
\pgfpathlineto{\pgfqpoint{3.444043in}{2.095201in}}%
\pgfpathlineto{\pgfqpoint{3.444247in}{2.136342in}}%
\pgfpathlineto{\pgfqpoint{3.444654in}{2.062288in}}%
\pgfpathlineto{\pgfqpoint{3.445062in}{2.086973in}}%
\pgfpathlineto{\pgfqpoint{3.445673in}{2.161027in}}%
\pgfpathlineto{\pgfqpoint{3.446081in}{2.103429in}}%
\pgfpathlineto{\pgfqpoint{3.446285in}{2.078745in}}%
\pgfpathlineto{\pgfqpoint{3.446488in}{2.111657in}}%
\pgfpathlineto{\pgfqpoint{3.446692in}{2.169255in}}%
\pgfpathlineto{\pgfqpoint{3.447304in}{2.086973in}}%
\pgfpathlineto{\pgfqpoint{3.447711in}{2.070516in}}%
\pgfpathlineto{\pgfqpoint{3.447915in}{2.086973in}}%
\pgfpathlineto{\pgfqpoint{3.448934in}{2.251537in}}%
\pgfpathlineto{\pgfqpoint{3.449342in}{2.210396in}}%
\pgfpathlineto{\pgfqpoint{3.449749in}{1.988234in}}%
\pgfpathlineto{\pgfqpoint{3.450157in}{2.309134in}}%
\pgfpathlineto{\pgfqpoint{3.450768in}{2.267993in}}%
\pgfpathlineto{\pgfqpoint{3.450972in}{2.333819in}}%
\pgfpathlineto{\pgfqpoint{3.451380in}{2.243309in}}%
\pgfpathlineto{\pgfqpoint{3.451787in}{2.284450in}}%
\pgfpathlineto{\pgfqpoint{3.451991in}{2.292678in}}%
\pgfpathlineto{\pgfqpoint{3.452195in}{2.259765in}}%
\pgfpathlineto{\pgfqpoint{3.452399in}{2.276221in}}%
\pgfpathlineto{\pgfqpoint{3.452602in}{2.235080in}}%
\pgfpathlineto{\pgfqpoint{3.453418in}{2.259765in}}%
\pgfpathlineto{\pgfqpoint{3.454029in}{2.325591in}}%
\pgfpathlineto{\pgfqpoint{3.454233in}{2.259765in}}%
\pgfpathlineto{\pgfqpoint{3.455659in}{2.193939in}}%
\pgfpathlineto{\pgfqpoint{3.456271in}{2.062288in}}%
\pgfpathlineto{\pgfqpoint{3.456475in}{1.963550in}}%
\pgfpathlineto{\pgfqpoint{3.456679in}{2.226852in}}%
\pgfpathlineto{\pgfqpoint{3.456882in}{2.177483in}}%
\pgfpathlineto{\pgfqpoint{3.457290in}{2.235080in}}%
\pgfpathlineto{\pgfqpoint{3.457698in}{2.152798in}}%
\pgfpathlineto{\pgfqpoint{3.458105in}{2.226852in}}%
\pgfpathlineto{\pgfqpoint{3.458309in}{2.226852in}}%
\pgfpathlineto{\pgfqpoint{3.458513in}{2.202168in}}%
\pgfpathlineto{\pgfqpoint{3.459124in}{2.243309in}}%
\pgfpathlineto{\pgfqpoint{3.459532in}{2.284450in}}%
\pgfpathlineto{\pgfqpoint{3.459736in}{2.210396in}}%
\pgfpathlineto{\pgfqpoint{3.460755in}{2.243309in}}%
\pgfpathlineto{\pgfqpoint{3.460958in}{2.243309in}}%
\pgfpathlineto{\pgfqpoint{3.462385in}{2.300906in}}%
\pgfpathlineto{\pgfqpoint{3.462996in}{2.202168in}}%
\pgfpathlineto{\pgfqpoint{3.463200in}{2.243309in}}%
\pgfpathlineto{\pgfqpoint{3.464015in}{2.358504in}}%
\pgfpathlineto{\pgfqpoint{3.463608in}{2.202168in}}%
\pgfpathlineto{\pgfqpoint{3.464219in}{2.317363in}}%
\pgfpathlineto{\pgfqpoint{3.465646in}{2.086973in}}%
\pgfpathlineto{\pgfqpoint{3.466257in}{2.276221in}}%
\pgfpathlineto{\pgfqpoint{3.466869in}{2.193939in}}%
\pgfpathlineto{\pgfqpoint{3.467072in}{2.226852in}}%
\pgfpathlineto{\pgfqpoint{3.467888in}{1.897724in}}%
\pgfpathlineto{\pgfqpoint{3.468091in}{2.012919in}}%
\pgfpathlineto{\pgfqpoint{3.469110in}{2.161027in}}%
\pgfpathlineto{\pgfqpoint{3.469722in}{2.012919in}}%
\pgfpathlineto{\pgfqpoint{3.470333in}{2.062288in}}%
\pgfpathlineto{\pgfqpoint{3.470945in}{2.012919in}}%
\pgfpathlineto{\pgfqpoint{3.470741in}{2.078745in}}%
\pgfpathlineto{\pgfqpoint{3.471149in}{2.037604in}}%
\pgfpathlineto{\pgfqpoint{3.471352in}{2.070516in}}%
\pgfpathlineto{\pgfqpoint{3.471760in}{2.004691in}}%
\pgfpathlineto{\pgfqpoint{3.472371in}{2.054060in}}%
\pgfpathlineto{\pgfqpoint{3.472983in}{1.840127in}}%
\pgfpathlineto{\pgfqpoint{3.473594in}{2.070516in}}%
\pgfpathlineto{\pgfqpoint{3.473798in}{1.897724in}}%
\pgfpathlineto{\pgfqpoint{3.474002in}{1.741388in}}%
\pgfpathlineto{\pgfqpoint{3.474613in}{2.070516in}}%
\pgfpathlineto{\pgfqpoint{3.474817in}{2.004691in}}%
\pgfpathlineto{\pgfqpoint{3.475428in}{2.086973in}}%
\pgfpathlineto{\pgfqpoint{3.475632in}{2.062288in}}%
\pgfpathlineto{\pgfqpoint{3.476244in}{2.004691in}}%
\pgfpathlineto{\pgfqpoint{3.476447in}{2.012919in}}%
\pgfpathlineto{\pgfqpoint{3.476651in}{1.724932in}}%
\pgfpathlineto{\pgfqpoint{3.477466in}{2.029375in}}%
\pgfpathlineto{\pgfqpoint{3.477874in}{2.012919in}}%
\pgfpathlineto{\pgfqpoint{3.478078in}{2.045832in}}%
\pgfpathlineto{\pgfqpoint{3.478282in}{1.988234in}}%
\pgfpathlineto{\pgfqpoint{3.478893in}{2.119886in}}%
\pgfpathlineto{\pgfqpoint{3.479097in}{2.185711in}}%
\pgfpathlineto{\pgfqpoint{3.479504in}{2.037604in}}%
\pgfpathlineto{\pgfqpoint{3.479912in}{2.095201in}}%
\pgfpathlineto{\pgfqpoint{3.480116in}{2.103429in}}%
\pgfpathlineto{\pgfqpoint{3.480320in}{2.062288in}}%
\pgfpathlineto{\pgfqpoint{3.480727in}{2.128114in}}%
\pgfpathlineto{\pgfqpoint{3.481135in}{2.095201in}}%
\pgfpathlineto{\pgfqpoint{3.481339in}{2.095201in}}%
\pgfpathlineto{\pgfqpoint{3.481542in}{2.144570in}}%
\pgfpathlineto{\pgfqpoint{3.481950in}{2.045832in}}%
\pgfpathlineto{\pgfqpoint{3.482358in}{2.103429in}}%
\pgfpathlineto{\pgfqpoint{3.482561in}{2.111657in}}%
\pgfpathlineto{\pgfqpoint{3.483581in}{1.914181in}}%
\pgfpathlineto{\pgfqpoint{3.484192in}{1.955322in}}%
\pgfpathlineto{\pgfqpoint{3.484396in}{1.938865in}}%
\pgfpathlineto{\pgfqpoint{3.484600in}{1.980006in}}%
\pgfpathlineto{\pgfqpoint{3.484803in}{2.037604in}}%
\pgfpathlineto{\pgfqpoint{3.485619in}{1.955322in}}%
\pgfpathlineto{\pgfqpoint{3.486230in}{1.914181in}}%
\pgfpathlineto{\pgfqpoint{3.487045in}{2.054060in}}%
\pgfpathlineto{\pgfqpoint{3.487453in}{2.029375in}}%
\pgfpathlineto{\pgfqpoint{3.488676in}{1.955322in}}%
\pgfpathlineto{\pgfqpoint{3.489083in}{1.980006in}}%
\pgfpathlineto{\pgfqpoint{3.489287in}{1.980006in}}%
\pgfpathlineto{\pgfqpoint{3.489491in}{1.996463in}}%
\pgfpathlineto{\pgfqpoint{3.489695in}{1.971778in}}%
\pgfpathlineto{\pgfqpoint{3.490306in}{1.905952in}}%
\pgfpathlineto{\pgfqpoint{3.490510in}{1.980006in}}%
\pgfpathlineto{\pgfqpoint{3.490714in}{1.963550in}}%
\pgfpathlineto{\pgfqpoint{3.491121in}{2.004691in}}%
\pgfpathlineto{\pgfqpoint{3.491325in}{1.897724in}}%
\pgfpathlineto{\pgfqpoint{3.491529in}{1.914181in}}%
\pgfpathlineto{\pgfqpoint{3.491733in}{1.881268in}}%
\pgfpathlineto{\pgfqpoint{3.492344in}{1.905952in}}%
\pgfpathlineto{\pgfqpoint{3.492955in}{1.798986in}}%
\pgfpathlineto{\pgfqpoint{3.493363in}{1.848355in}}%
\pgfpathlineto{\pgfqpoint{3.493974in}{1.798986in}}%
\pgfpathlineto{\pgfqpoint{3.494178in}{1.798986in}}%
\pgfpathlineto{\pgfqpoint{3.495197in}{1.864811in}}%
\pgfpathlineto{\pgfqpoint{3.494586in}{1.774301in}}%
\pgfpathlineto{\pgfqpoint{3.495605in}{1.815442in}}%
\pgfpathlineto{\pgfqpoint{3.496624in}{1.774301in}}%
\pgfpathlineto{\pgfqpoint{3.496216in}{1.856583in}}%
\pgfpathlineto{\pgfqpoint{3.496828in}{1.798986in}}%
\pgfpathlineto{\pgfqpoint{3.498458in}{1.971778in}}%
\pgfpathlineto{\pgfqpoint{3.499681in}{1.659106in}}%
\pgfpathlineto{\pgfqpoint{3.499885in}{1.815442in}}%
\pgfpathlineto{\pgfqpoint{3.501108in}{2.029375in}}%
\pgfpathlineto{\pgfqpoint{3.501311in}{2.004691in}}%
\pgfpathlineto{\pgfqpoint{3.502330in}{1.881268in}}%
\pgfpathlineto{\pgfqpoint{3.502738in}{1.889496in}}%
\pgfpathlineto{\pgfqpoint{3.503553in}{1.856583in}}%
\pgfpathlineto{\pgfqpoint{3.503146in}{1.905952in}}%
\pgfpathlineto{\pgfqpoint{3.503961in}{1.873040in}}%
\pgfpathlineto{\pgfqpoint{3.504165in}{1.881268in}}%
\pgfpathlineto{\pgfqpoint{3.504368in}{1.864811in}}%
\pgfpathlineto{\pgfqpoint{3.505184in}{1.798986in}}%
\pgfpathlineto{\pgfqpoint{3.505387in}{1.873040in}}%
\pgfpathlineto{\pgfqpoint{3.506203in}{1.774301in}}%
\pgfpathlineto{\pgfqpoint{3.506610in}{1.790758in}}%
\pgfpathlineto{\pgfqpoint{3.507018in}{1.749617in}}%
\pgfpathlineto{\pgfqpoint{3.507425in}{1.881268in}}%
\pgfpathlineto{\pgfqpoint{3.507833in}{1.692019in}}%
\pgfpathlineto{\pgfqpoint{3.508037in}{1.757845in}}%
\pgfpathlineto{\pgfqpoint{3.509463in}{1.552140in}}%
\pgfpathlineto{\pgfqpoint{3.509667in}{1.626194in}}%
\pgfpathlineto{\pgfqpoint{3.510279in}{1.601509in}}%
\pgfpathlineto{\pgfqpoint{3.510483in}{1.724932in}}%
\pgfpathlineto{\pgfqpoint{3.511298in}{1.609737in}}%
\pgfpathlineto{\pgfqpoint{3.511502in}{1.576824in}}%
\pgfpathlineto{\pgfqpoint{3.512317in}{1.593281in}}%
\pgfpathlineto{\pgfqpoint{3.513336in}{1.634422in}}%
\pgfpathlineto{\pgfqpoint{3.513947in}{1.642650in}}%
\pgfpathlineto{\pgfqpoint{3.514151in}{1.576824in}}%
\pgfpathlineto{\pgfqpoint{3.515374in}{1.700247in}}%
\pgfpathlineto{\pgfqpoint{3.515985in}{1.626194in}}%
\pgfpathlineto{\pgfqpoint{3.516393in}{1.700247in}}%
\pgfpathlineto{\pgfqpoint{3.516597in}{1.700247in}}%
\pgfpathlineto{\pgfqpoint{3.517412in}{1.585053in}}%
\pgfpathlineto{\pgfqpoint{3.518023in}{1.626194in}}%
\pgfpathlineto{\pgfqpoint{3.518227in}{1.675563in}}%
\pgfpathlineto{\pgfqpoint{3.518431in}{1.626194in}}%
\pgfpathlineto{\pgfqpoint{3.518635in}{1.379348in}}%
\pgfpathlineto{\pgfqpoint{3.519450in}{1.560368in}}%
\pgfpathlineto{\pgfqpoint{3.519654in}{1.601509in}}%
\pgfpathlineto{\pgfqpoint{3.519857in}{1.486314in}}%
\pgfpathlineto{\pgfqpoint{3.520265in}{1.502771in}}%
\pgfpathlineto{\pgfqpoint{3.520876in}{1.535683in}}%
\pgfpathlineto{\pgfqpoint{3.521080in}{1.486314in}}%
\pgfpathlineto{\pgfqpoint{3.521284in}{1.510999in}}%
\pgfpathlineto{\pgfqpoint{3.522507in}{1.346435in}}%
\pgfpathlineto{\pgfqpoint{3.523730in}{1.510999in}}%
\pgfpathlineto{\pgfqpoint{3.524341in}{1.494542in}}%
\pgfpathlineto{\pgfqpoint{3.524545in}{1.527455in}}%
\pgfpathlineto{\pgfqpoint{3.524749in}{1.519227in}}%
\pgfpathlineto{\pgfqpoint{3.526175in}{1.642650in}}%
\pgfpathlineto{\pgfqpoint{3.526991in}{1.552140in}}%
\pgfpathlineto{\pgfqpoint{3.528010in}{1.700247in}}%
\pgfpathlineto{\pgfqpoint{3.528213in}{1.659106in}}%
\pgfpathlineto{\pgfqpoint{3.528417in}{1.724932in}}%
\pgfpathlineto{\pgfqpoint{3.528621in}{1.626194in}}%
\pgfpathlineto{\pgfqpoint{3.529232in}{1.642650in}}%
\pgfpathlineto{\pgfqpoint{3.530659in}{1.733160in}}%
\pgfpathlineto{\pgfqpoint{3.530863in}{1.700247in}}%
\pgfpathlineto{\pgfqpoint{3.532086in}{1.560368in}}%
\pgfpathlineto{\pgfqpoint{3.531270in}{1.708476in}}%
\pgfpathlineto{\pgfqpoint{3.532697in}{1.593281in}}%
\pgfpathlineto{\pgfqpoint{3.533512in}{1.692019in}}%
\pgfpathlineto{\pgfqpoint{3.533920in}{1.650878in}}%
\pgfpathlineto{\pgfqpoint{3.534327in}{1.650878in}}%
\pgfpathlineto{\pgfqpoint{3.535143in}{1.700247in}}%
\pgfpathlineto{\pgfqpoint{3.534735in}{1.634422in}}%
\pgfpathlineto{\pgfqpoint{3.535550in}{1.659106in}}%
\pgfpathlineto{\pgfqpoint{3.535754in}{1.659106in}}%
\pgfpathlineto{\pgfqpoint{3.535958in}{1.667335in}}%
\pgfpathlineto{\pgfqpoint{3.536162in}{1.659106in}}%
\pgfpathlineto{\pgfqpoint{3.536365in}{1.757845in}}%
\pgfpathlineto{\pgfqpoint{3.537181in}{1.667335in}}%
\pgfpathlineto{\pgfqpoint{3.537588in}{1.642650in}}%
\pgfpathlineto{\pgfqpoint{3.537792in}{1.667335in}}%
\pgfpathlineto{\pgfqpoint{3.538607in}{1.708476in}}%
\pgfpathlineto{\pgfqpoint{3.538811in}{1.692019in}}%
\pgfpathlineto{\pgfqpoint{3.539626in}{1.642650in}}%
\pgfpathlineto{\pgfqpoint{3.539830in}{1.667335in}}%
\pgfpathlineto{\pgfqpoint{3.540645in}{1.733160in}}%
\pgfpathlineto{\pgfqpoint{3.540238in}{1.617965in}}%
\pgfpathlineto{\pgfqpoint{3.540849in}{1.700247in}}%
\pgfpathlineto{\pgfqpoint{3.541664in}{1.659106in}}%
\pgfpathlineto{\pgfqpoint{3.541461in}{1.708476in}}%
\pgfpathlineto{\pgfqpoint{3.541868in}{1.675563in}}%
\pgfpathlineto{\pgfqpoint{3.543499in}{1.848355in}}%
\pgfpathlineto{\pgfqpoint{3.544721in}{1.749617in}}%
\pgfpathlineto{\pgfqpoint{3.547167in}{1.930637in}}%
\pgfpathlineto{\pgfqpoint{3.547778in}{1.815442in}}%
\pgfpathlineto{\pgfqpoint{3.548594in}{1.881268in}}%
\pgfpathlineto{\pgfqpoint{3.549205in}{1.848355in}}%
\pgfpathlineto{\pgfqpoint{3.549001in}{1.889496in}}%
\pgfpathlineto{\pgfqpoint{3.549409in}{1.873040in}}%
\pgfpathlineto{\pgfqpoint{3.549613in}{1.889496in}}%
\pgfpathlineto{\pgfqpoint{3.549816in}{1.848355in}}%
\pgfpathlineto{\pgfqpoint{3.551039in}{1.716704in}}%
\pgfpathlineto{\pgfqpoint{3.550224in}{1.914181in}}%
\pgfpathlineto{\pgfqpoint{3.551243in}{1.774301in}}%
\pgfpathlineto{\pgfqpoint{3.551651in}{1.815442in}}%
\pgfpathlineto{\pgfqpoint{3.551855in}{1.774301in}}%
\pgfpathlineto{\pgfqpoint{3.553077in}{1.642650in}}%
\pgfpathlineto{\pgfqpoint{3.553281in}{1.700247in}}%
\pgfpathlineto{\pgfqpoint{3.554096in}{1.650878in}}%
\pgfpathlineto{\pgfqpoint{3.554708in}{1.593281in}}%
\pgfpathlineto{\pgfqpoint{3.554912in}{1.683791in}}%
\pgfpathlineto{\pgfqpoint{3.555115in}{1.617965in}}%
\pgfpathlineto{\pgfqpoint{3.555523in}{1.708476in}}%
\pgfpathlineto{\pgfqpoint{3.555931in}{1.568596in}}%
\pgfpathlineto{\pgfqpoint{3.556746in}{1.593281in}}%
\pgfpathlineto{\pgfqpoint{3.557765in}{1.527455in}}%
\pgfpathlineto{\pgfqpoint{3.557357in}{1.617965in}}%
\pgfpathlineto{\pgfqpoint{3.557969in}{1.568596in}}%
\pgfpathlineto{\pgfqpoint{3.558376in}{1.659106in}}%
\pgfpathlineto{\pgfqpoint{3.559191in}{1.626194in}}%
\pgfpathlineto{\pgfqpoint{3.559395in}{1.626194in}}%
\pgfpathlineto{\pgfqpoint{3.560618in}{1.724932in}}%
\pgfpathlineto{\pgfqpoint{3.560822in}{1.716704in}}%
\pgfpathlineto{\pgfqpoint{3.561026in}{1.733160in}}%
\pgfpathlineto{\pgfqpoint{3.561229in}{1.692019in}}%
\pgfpathlineto{\pgfqpoint{3.561637in}{1.708476in}}%
\pgfpathlineto{\pgfqpoint{3.562452in}{1.642650in}}%
\pgfpathlineto{\pgfqpoint{3.562656in}{1.617965in}}%
\pgfpathlineto{\pgfqpoint{3.562860in}{1.667335in}}%
\pgfpathlineto{\pgfqpoint{3.563064in}{1.650878in}}%
\pgfpathlineto{\pgfqpoint{3.564490in}{1.807214in}}%
\pgfpathlineto{\pgfqpoint{3.565917in}{1.650878in}}%
\pgfpathlineto{\pgfqpoint{3.566121in}{1.659106in}}%
\pgfpathlineto{\pgfqpoint{3.566325in}{1.626194in}}%
\pgfpathlineto{\pgfqpoint{3.566528in}{1.617965in}}%
\pgfpathlineto{\pgfqpoint{3.567751in}{1.724932in}}%
\pgfpathlineto{\pgfqpoint{3.567955in}{1.683791in}}%
\pgfpathlineto{\pgfqpoint{3.568566in}{1.733160in}}%
\pgfpathlineto{\pgfqpoint{3.569382in}{1.807214in}}%
\pgfpathlineto{\pgfqpoint{3.569585in}{1.733160in}}%
\pgfpathlineto{\pgfqpoint{3.571012in}{1.650878in}}%
\pgfpathlineto{\pgfqpoint{3.571420in}{1.766073in}}%
\pgfpathlineto{\pgfqpoint{3.572642in}{1.724932in}}%
\pgfpathlineto{\pgfqpoint{3.574069in}{1.585053in}}%
\pgfpathlineto{\pgfqpoint{3.573050in}{1.733160in}}%
\pgfpathlineto{\pgfqpoint{3.574477in}{1.626194in}}%
\pgfpathlineto{\pgfqpoint{3.576311in}{1.815442in}}%
\pgfpathlineto{\pgfqpoint{3.577738in}{1.642650in}}%
\pgfpathlineto{\pgfqpoint{3.577941in}{1.675563in}}%
\pgfpathlineto{\pgfqpoint{3.578145in}{1.675563in}}%
\pgfpathlineto{\pgfqpoint{3.578553in}{1.659106in}}%
\pgfpathlineto{\pgfqpoint{3.579368in}{1.749617in}}%
\pgfpathlineto{\pgfqpoint{3.580387in}{1.609737in}}%
\pgfpathlineto{\pgfqpoint{3.580795in}{1.642650in}}%
\pgfpathlineto{\pgfqpoint{3.582221in}{1.733160in}}%
\pgfpathlineto{\pgfqpoint{3.583240in}{1.617965in}}%
\pgfpathlineto{\pgfqpoint{3.583444in}{1.650878in}}%
\pgfpathlineto{\pgfqpoint{3.584463in}{1.798986in}}%
\pgfpathlineto{\pgfqpoint{3.584871in}{1.782529in}}%
\pgfpathlineto{\pgfqpoint{3.585686in}{1.692019in}}%
\pgfpathlineto{\pgfqpoint{3.585890in}{1.733160in}}%
\pgfpathlineto{\pgfqpoint{3.586093in}{1.782529in}}%
\pgfpathlineto{\pgfqpoint{3.586297in}{1.659106in}}%
\pgfpathlineto{\pgfqpoint{3.586501in}{1.683791in}}%
\pgfpathlineto{\pgfqpoint{3.587112in}{1.576824in}}%
\pgfpathlineto{\pgfqpoint{3.587724in}{1.634422in}}%
\pgfpathlineto{\pgfqpoint{3.588131in}{1.675563in}}%
\pgfpathlineto{\pgfqpoint{3.588539in}{1.626194in}}%
\pgfpathlineto{\pgfqpoint{3.589558in}{1.593281in}}%
\pgfpathlineto{\pgfqpoint{3.589762in}{1.634422in}}%
\pgfpathlineto{\pgfqpoint{3.590169in}{1.552140in}}%
\pgfpathlineto{\pgfqpoint{3.590373in}{1.601509in}}%
\pgfpathlineto{\pgfqpoint{3.590577in}{1.568596in}}%
\pgfpathlineto{\pgfqpoint{3.590781in}{1.626194in}}%
\pgfpathlineto{\pgfqpoint{3.591392in}{1.576824in}}%
\pgfpathlineto{\pgfqpoint{3.591596in}{1.667335in}}%
\pgfpathlineto{\pgfqpoint{3.592615in}{1.642650in}}%
\pgfpathlineto{\pgfqpoint{3.592819in}{1.617965in}}%
\pgfpathlineto{\pgfqpoint{3.593023in}{1.692019in}}%
\pgfpathlineto{\pgfqpoint{3.593227in}{1.692019in}}%
\pgfpathlineto{\pgfqpoint{3.593430in}{1.766073in}}%
\pgfpathlineto{\pgfqpoint{3.594042in}{1.650878in}}%
\pgfpathlineto{\pgfqpoint{3.594246in}{1.716704in}}%
\pgfpathlineto{\pgfqpoint{3.595265in}{1.585053in}}%
\pgfpathlineto{\pgfqpoint{3.595468in}{1.634422in}}%
\pgfpathlineto{\pgfqpoint{3.595672in}{1.642650in}}%
\pgfpathlineto{\pgfqpoint{3.595876in}{1.478086in}}%
\pgfpathlineto{\pgfqpoint{3.596487in}{1.716704in}}%
\pgfpathlineto{\pgfqpoint{3.596691in}{1.724932in}}%
\pgfpathlineto{\pgfqpoint{3.596895in}{1.708476in}}%
\pgfpathlineto{\pgfqpoint{3.597506in}{1.626194in}}%
\pgfpathlineto{\pgfqpoint{3.598322in}{1.634422in}}%
\pgfpathlineto{\pgfqpoint{3.598729in}{1.692019in}}%
\pgfpathlineto{\pgfqpoint{3.598933in}{1.362891in}}%
\pgfpathlineto{\pgfqpoint{3.599748in}{1.634422in}}%
\pgfpathlineto{\pgfqpoint{3.600156in}{1.757845in}}%
\pgfpathlineto{\pgfqpoint{3.600360in}{1.552140in}}%
\pgfpathlineto{\pgfqpoint{3.600971in}{1.708476in}}%
\pgfpathlineto{\pgfqpoint{3.601582in}{1.766073in}}%
\pgfpathlineto{\pgfqpoint{3.601786in}{1.700247in}}%
\pgfpathlineto{\pgfqpoint{3.601990in}{1.724932in}}%
\pgfpathlineto{\pgfqpoint{3.602398in}{1.700247in}}%
\pgfpathlineto{\pgfqpoint{3.602805in}{1.749617in}}%
\pgfpathlineto{\pgfqpoint{3.603009in}{1.766073in}}%
\pgfpathlineto{\pgfqpoint{3.603213in}{1.708476in}}%
\pgfpathlineto{\pgfqpoint{3.603620in}{1.733160in}}%
\pgfpathlineto{\pgfqpoint{3.603824in}{1.560368in}}%
\pgfpathlineto{\pgfqpoint{3.604232in}{1.749617in}}%
\pgfpathlineto{\pgfqpoint{3.604640in}{1.667335in}}%
\pgfpathlineto{\pgfqpoint{3.605659in}{1.774301in}}%
\pgfpathlineto{\pgfqpoint{3.605862in}{1.724932in}}%
\pgfpathlineto{\pgfqpoint{3.606474in}{1.436945in}}%
\pgfpathlineto{\pgfqpoint{3.607085in}{1.667335in}}%
\pgfpathlineto{\pgfqpoint{3.607289in}{1.749617in}}%
\pgfpathlineto{\pgfqpoint{3.607900in}{1.659106in}}%
\pgfpathlineto{\pgfqpoint{3.608104in}{1.659106in}}%
\pgfpathlineto{\pgfqpoint{3.608308in}{1.428717in}}%
\pgfpathlineto{\pgfqpoint{3.608919in}{1.749617in}}%
\pgfpathlineto{\pgfqpoint{3.609123in}{1.700247in}}%
\pgfpathlineto{\pgfqpoint{3.609327in}{1.724932in}}%
\pgfpathlineto{\pgfqpoint{3.609938in}{1.683791in}}%
\pgfpathlineto{\pgfqpoint{3.610142in}{1.675563in}}%
\pgfpathlineto{\pgfqpoint{3.610550in}{1.692019in}}%
\pgfpathlineto{\pgfqpoint{3.611569in}{1.815442in}}%
\pgfpathlineto{\pgfqpoint{3.611976in}{1.774301in}}%
\pgfpathlineto{\pgfqpoint{3.612180in}{1.560368in}}%
\pgfpathlineto{\pgfqpoint{3.612995in}{1.716704in}}%
\pgfpathlineto{\pgfqpoint{3.613199in}{1.757845in}}%
\pgfpathlineto{\pgfqpoint{3.613403in}{1.700247in}}%
\pgfpathlineto{\pgfqpoint{3.613811in}{1.708476in}}%
\pgfpathlineto{\pgfqpoint{3.614422in}{1.724932in}}%
\pgfpathlineto{\pgfqpoint{3.615033in}{1.659106in}}%
\pgfpathlineto{\pgfqpoint{3.616868in}{1.840127in}}%
\pgfpathlineto{\pgfqpoint{3.615441in}{1.617965in}}%
\pgfpathlineto{\pgfqpoint{3.617071in}{1.831899in}}%
\pgfpathlineto{\pgfqpoint{3.618498in}{1.510999in}}%
\pgfpathlineto{\pgfqpoint{3.618702in}{1.494542in}}%
\pgfpathlineto{\pgfqpoint{3.618906in}{1.552140in}}%
\pgfpathlineto{\pgfqpoint{3.619110in}{1.560368in}}%
\pgfpathlineto{\pgfqpoint{3.619517in}{1.486314in}}%
\pgfpathlineto{\pgfqpoint{3.620129in}{1.535683in}}%
\pgfpathlineto{\pgfqpoint{3.620536in}{1.568596in}}%
\pgfpathlineto{\pgfqpoint{3.621555in}{1.404032in}}%
\pgfpathlineto{\pgfqpoint{3.621759in}{1.453401in}}%
\pgfpathlineto{\pgfqpoint{3.623389in}{1.305294in}}%
\pgfpathlineto{\pgfqpoint{3.622574in}{1.494542in}}%
\pgfpathlineto{\pgfqpoint{3.623593in}{1.329978in}}%
\pgfpathlineto{\pgfqpoint{3.623797in}{1.305294in}}%
\pgfpathlineto{\pgfqpoint{3.624001in}{1.362891in}}%
\pgfpathlineto{\pgfqpoint{3.624408in}{1.346435in}}%
\pgfpathlineto{\pgfqpoint{3.624612in}{1.371119in}}%
\pgfpathlineto{\pgfqpoint{3.624816in}{1.305294in}}%
\pgfpathlineto{\pgfqpoint{3.625835in}{1.099589in}}%
\pgfpathlineto{\pgfqpoint{3.626446in}{1.148958in}}%
\pgfpathlineto{\pgfqpoint{3.626650in}{1.379348in}}%
\pgfpathlineto{\pgfqpoint{3.627669in}{1.280609in}}%
\pgfpathlineto{\pgfqpoint{3.627873in}{1.404032in}}%
\pgfpathlineto{\pgfqpoint{3.628688in}{1.321750in}}%
\pgfpathlineto{\pgfqpoint{3.629707in}{1.107817in}}%
\pgfpathlineto{\pgfqpoint{3.630319in}{1.223012in}}%
\pgfpathlineto{\pgfqpoint{3.630522in}{1.206555in}}%
\pgfpathlineto{\pgfqpoint{3.630726in}{1.272381in}}%
\pgfpathlineto{\pgfqpoint{3.630930in}{1.313522in}}%
\pgfpathlineto{\pgfqpoint{3.631134in}{1.239468in}}%
\pgfpathlineto{\pgfqpoint{3.631542in}{1.041991in}}%
\pgfpathlineto{\pgfqpoint{3.632357in}{1.132502in}}%
\pgfpathlineto{\pgfqpoint{3.632764in}{1.173643in}}%
\pgfpathlineto{\pgfqpoint{3.634191in}{0.893884in}}%
\pgfpathlineto{\pgfqpoint{3.634395in}{0.918568in}}%
\pgfpathlineto{\pgfqpoint{3.634802in}{0.910340in}}%
\pgfpathlineto{\pgfqpoint{3.635618in}{0.729320in}}%
\pgfpathlineto{\pgfqpoint{3.635821in}{0.745776in}}%
\pgfpathlineto{\pgfqpoint{3.637044in}{1.000850in}}%
\pgfpathlineto{\pgfqpoint{3.637452in}{1.074904in}}%
\pgfpathlineto{\pgfqpoint{3.638267in}{0.893884in}}%
\pgfpathlineto{\pgfqpoint{3.639082in}{1.148958in}}%
\pgfpathlineto{\pgfqpoint{3.639490in}{0.992622in}}%
\pgfpathlineto{\pgfqpoint{3.639694in}{1.009079in}}%
\pgfpathlineto{\pgfqpoint{3.640101in}{0.647038in}}%
\pgfpathlineto{\pgfqpoint{3.640509in}{1.066676in}}%
\pgfpathlineto{\pgfqpoint{3.640713in}{1.025535in}}%
\pgfpathlineto{\pgfqpoint{3.641120in}{0.918568in}}%
\pgfpathlineto{\pgfqpoint{3.641528in}{1.091361in}}%
\pgfpathlineto{\pgfqpoint{3.641732in}{1.033763in}}%
\pgfpathlineto{\pgfqpoint{3.642139in}{1.124273in}}%
\pgfpathlineto{\pgfqpoint{3.642751in}{1.041991in}}%
\pgfpathlineto{\pgfqpoint{3.643362in}{0.984394in}}%
\pgfpathlineto{\pgfqpoint{3.643770in}{0.992622in}}%
\pgfpathlineto{\pgfqpoint{3.643973in}{1.025535in}}%
\pgfpathlineto{\pgfqpoint{3.644177in}{1.009079in}}%
\pgfpathlineto{\pgfqpoint{3.645400in}{0.811602in}}%
\pgfpathlineto{\pgfqpoint{3.645604in}{0.852743in}}%
\pgfpathlineto{\pgfqpoint{3.645808in}{0.795145in}}%
\pgfpathlineto{\pgfqpoint{3.646419in}{0.877427in}}%
\pgfpathlineto{\pgfqpoint{3.646623in}{0.869199in}}%
\pgfpathlineto{\pgfqpoint{3.647642in}{1.132502in}}%
\pgfpathlineto{\pgfqpoint{3.647846in}{1.091361in}}%
\pgfpathlineto{\pgfqpoint{3.649069in}{0.844515in}}%
\pgfpathlineto{\pgfqpoint{3.650291in}{1.214784in}}%
\pgfpathlineto{\pgfqpoint{3.650699in}{1.066676in}}%
\pgfpathlineto{\pgfqpoint{3.651310in}{1.140730in}}%
\pgfpathlineto{\pgfqpoint{3.652126in}{1.255925in}}%
\pgfpathlineto{\pgfqpoint{3.652329in}{1.066676in}}%
\pgfpathlineto{\pgfqpoint{3.653145in}{1.214784in}}%
\pgfpathlineto{\pgfqpoint{3.653756in}{1.313522in}}%
\pgfpathlineto{\pgfqpoint{3.653960in}{1.181871in}}%
\pgfpathlineto{\pgfqpoint{3.654164in}{1.181871in}}%
\pgfpathlineto{\pgfqpoint{3.654367in}{1.223012in}}%
\pgfpathlineto{\pgfqpoint{3.654775in}{1.124273in}}%
\pgfpathlineto{\pgfqpoint{3.654979in}{1.041991in}}%
\pgfpathlineto{\pgfqpoint{3.655794in}{1.157186in}}%
\pgfpathlineto{\pgfqpoint{3.655998in}{1.132502in}}%
\pgfpathlineto{\pgfqpoint{3.656405in}{1.198327in}}%
\pgfpathlineto{\pgfqpoint{3.656609in}{1.239468in}}%
\pgfpathlineto{\pgfqpoint{3.657017in}{1.132502in}}%
\pgfpathlineto{\pgfqpoint{3.657221in}{1.198327in}}%
\pgfpathlineto{\pgfqpoint{3.657424in}{1.041991in}}%
\pgfpathlineto{\pgfqpoint{3.658240in}{1.148958in}}%
\pgfpathlineto{\pgfqpoint{3.658444in}{1.255925in}}%
\pgfpathlineto{\pgfqpoint{3.658851in}{1.091361in}}%
\pgfpathlineto{\pgfqpoint{3.659259in}{1.181871in}}%
\pgfpathlineto{\pgfqpoint{3.660074in}{1.140730in}}%
\pgfpathlineto{\pgfqpoint{3.660482in}{1.165414in}}%
\pgfpathlineto{\pgfqpoint{3.661093in}{1.132502in}}%
\pgfpathlineto{\pgfqpoint{3.661297in}{1.173643in}}%
\pgfpathlineto{\pgfqpoint{3.662112in}{1.255925in}}%
\pgfpathlineto{\pgfqpoint{3.662927in}{1.231240in}}%
\pgfpathlineto{\pgfqpoint{3.663131in}{1.157186in}}%
\pgfpathlineto{\pgfqpoint{3.663742in}{1.272381in}}%
\pgfpathlineto{\pgfqpoint{3.663946in}{1.255925in}}%
\pgfpathlineto{\pgfqpoint{3.665373in}{1.354663in}}%
\pgfpathlineto{\pgfqpoint{3.666799in}{1.519227in}}%
\pgfpathlineto{\pgfqpoint{3.667207in}{1.478086in}}%
\pgfpathlineto{\pgfqpoint{3.667615in}{1.453401in}}%
\pgfpathlineto{\pgfqpoint{3.667818in}{1.486314in}}%
\pgfpathlineto{\pgfqpoint{3.669041in}{1.527455in}}%
\pgfpathlineto{\pgfqpoint{3.669245in}{1.486314in}}%
\pgfpathlineto{\pgfqpoint{3.669449in}{1.560368in}}%
\pgfpathlineto{\pgfqpoint{3.669653in}{1.552140in}}%
\pgfpathlineto{\pgfqpoint{3.670264in}{1.741388in}}%
\pgfpathlineto{\pgfqpoint{3.670672in}{1.552140in}}%
\pgfpathlineto{\pgfqpoint{3.671079in}{1.568596in}}%
\pgfpathlineto{\pgfqpoint{3.671283in}{1.543912in}}%
\pgfpathlineto{\pgfqpoint{3.671487in}{1.634422in}}%
\pgfpathlineto{\pgfqpoint{3.672302in}{1.585053in}}%
\pgfpathlineto{\pgfqpoint{3.673525in}{1.486314in}}%
\pgfpathlineto{\pgfqpoint{3.673729in}{1.494542in}}%
\pgfpathlineto{\pgfqpoint{3.673933in}{1.527455in}}%
\pgfpathlineto{\pgfqpoint{3.674340in}{1.478086in}}%
\pgfpathlineto{\pgfqpoint{3.674748in}{1.502771in}}%
\pgfpathlineto{\pgfqpoint{3.675155in}{1.510999in}}%
\pgfpathlineto{\pgfqpoint{3.675359in}{1.626194in}}%
\pgfpathlineto{\pgfqpoint{3.675971in}{1.453401in}}%
\pgfpathlineto{\pgfqpoint{3.676174in}{1.478086in}}%
\pgfpathlineto{\pgfqpoint{3.676378in}{1.453401in}}%
\pgfpathlineto{\pgfqpoint{3.676786in}{1.486314in}}%
\pgfpathlineto{\pgfqpoint{3.678212in}{1.642650in}}%
\pgfpathlineto{\pgfqpoint{3.678620in}{1.617965in}}%
\pgfpathlineto{\pgfqpoint{3.678824in}{1.576824in}}%
\pgfpathlineto{\pgfqpoint{3.679231in}{1.675563in}}%
\pgfpathlineto{\pgfqpoint{3.679435in}{1.634422in}}%
\pgfpathlineto{\pgfqpoint{3.680047in}{1.798986in}}%
\pgfpathlineto{\pgfqpoint{3.680658in}{1.757845in}}%
\pgfpathlineto{\pgfqpoint{3.683919in}{2.021147in}}%
\pgfpathlineto{\pgfqpoint{3.684123in}{2.029375in}}%
\pgfpathlineto{\pgfqpoint{3.684326in}{1.922409in}}%
\pgfpathlineto{\pgfqpoint{3.685142in}{2.021147in}}%
\pgfpathlineto{\pgfqpoint{3.685346in}{2.029375in}}%
\pgfpathlineto{\pgfqpoint{3.686161in}{2.161027in}}%
\pgfpathlineto{\pgfqpoint{3.686365in}{2.103429in}}%
\pgfpathlineto{\pgfqpoint{3.687384in}{2.111657in}}%
\pgfpathlineto{\pgfqpoint{3.687791in}{1.790758in}}%
\pgfpathlineto{\pgfqpoint{3.688403in}{2.086973in}}%
\pgfpathlineto{\pgfqpoint{3.689014in}{2.070516in}}%
\pgfpathlineto{\pgfqpoint{3.689625in}{2.004691in}}%
\pgfpathlineto{\pgfqpoint{3.690033in}{2.078745in}}%
\pgfpathlineto{\pgfqpoint{3.690237in}{2.070516in}}%
\pgfpathlineto{\pgfqpoint{3.690441in}{2.136342in}}%
\pgfpathlineto{\pgfqpoint{3.690848in}{2.012919in}}%
\pgfpathlineto{\pgfqpoint{3.691052in}{2.054060in}}%
\pgfpathlineto{\pgfqpoint{3.691256in}{1.790758in}}%
\pgfpathlineto{\pgfqpoint{3.692071in}{1.963550in}}%
\pgfpathlineto{\pgfqpoint{3.692275in}{1.963550in}}%
\pgfpathlineto{\pgfqpoint{3.694109in}{2.177483in}}%
\pgfpathlineto{\pgfqpoint{3.694720in}{2.054060in}}%
\pgfpathlineto{\pgfqpoint{3.695536in}{2.086973in}}%
\pgfpathlineto{\pgfqpoint{3.695739in}{2.086973in}}%
\pgfpathlineto{\pgfqpoint{3.696147in}{2.128114in}}%
\pgfpathlineto{\pgfqpoint{3.696758in}{2.086973in}}%
\pgfpathlineto{\pgfqpoint{3.697370in}{2.095201in}}%
\pgfpathlineto{\pgfqpoint{3.697777in}{2.054060in}}%
\pgfpathlineto{\pgfqpoint{3.699000in}{2.169255in}}%
\pgfpathlineto{\pgfqpoint{3.699408in}{2.136342in}}%
\pgfpathlineto{\pgfqpoint{3.699816in}{2.152798in}}%
\pgfpathlineto{\pgfqpoint{3.700019in}{2.202168in}}%
\pgfpathlineto{\pgfqpoint{3.700835in}{2.161027in}}%
\pgfpathlineto{\pgfqpoint{3.701038in}{2.161027in}}%
\pgfpathlineto{\pgfqpoint{3.701446in}{2.226852in}}%
\pgfpathlineto{\pgfqpoint{3.701650in}{2.152798in}}%
\pgfpathlineto{\pgfqpoint{3.701854in}{2.119886in}}%
\pgfpathlineto{\pgfqpoint{3.702057in}{2.193939in}}%
\pgfpathlineto{\pgfqpoint{3.702465in}{2.185711in}}%
\pgfpathlineto{\pgfqpoint{3.702669in}{2.185711in}}%
\pgfpathlineto{\pgfqpoint{3.703076in}{2.259765in}}%
\pgfpathlineto{\pgfqpoint{3.703892in}{2.243309in}}%
\pgfpathlineto{\pgfqpoint{3.704299in}{2.185711in}}%
\pgfpathlineto{\pgfqpoint{3.704503in}{2.169255in}}%
\pgfpathlineto{\pgfqpoint{3.704707in}{2.210396in}}%
\pgfpathlineto{\pgfqpoint{3.704911in}{2.193939in}}%
\pgfpathlineto{\pgfqpoint{3.705114in}{2.218624in}}%
\pgfpathlineto{\pgfqpoint{3.705930in}{2.210396in}}%
\pgfpathlineto{\pgfqpoint{3.706337in}{2.177483in}}%
\pgfpathlineto{\pgfqpoint{3.706745in}{2.193939in}}%
\pgfpathlineto{\pgfqpoint{3.706949in}{2.251537in}}%
\pgfpathlineto{\pgfqpoint{3.707968in}{2.243309in}}%
\pgfpathlineto{\pgfqpoint{3.709802in}{2.193939in}}%
\pgfpathlineto{\pgfqpoint{3.710006in}{2.202168in}}%
\pgfpathlineto{\pgfqpoint{3.712044in}{2.366732in}}%
\pgfpathlineto{\pgfqpoint{3.712451in}{2.317363in}}%
\pgfpathlineto{\pgfqpoint{3.712859in}{2.342047in}}%
\pgfpathlineto{\pgfqpoint{3.713878in}{2.185711in}}%
\pgfpathlineto{\pgfqpoint{3.714286in}{2.218624in}}%
\pgfpathlineto{\pgfqpoint{3.714693in}{2.161027in}}%
\pgfpathlineto{\pgfqpoint{3.714897in}{2.185711in}}%
\pgfpathlineto{\pgfqpoint{3.715508in}{2.169255in}}%
\pgfpathlineto{\pgfqpoint{3.715712in}{2.193939in}}%
\pgfpathlineto{\pgfqpoint{3.716324in}{2.276221in}}%
\pgfpathlineto{\pgfqpoint{3.716935in}{2.267993in}}%
\pgfpathlineto{\pgfqpoint{3.718362in}{2.202168in}}%
\pgfpathlineto{\pgfqpoint{3.719381in}{2.152798in}}%
\pgfpathlineto{\pgfqpoint{3.718769in}{2.210396in}}%
\pgfpathlineto{\pgfqpoint{3.719584in}{2.185711in}}%
\pgfpathlineto{\pgfqpoint{3.719992in}{2.243309in}}%
\pgfpathlineto{\pgfqpoint{3.720603in}{2.226852in}}%
\pgfpathlineto{\pgfqpoint{3.720807in}{2.185711in}}%
\pgfpathlineto{\pgfqpoint{3.721215in}{2.267993in}}%
\pgfpathlineto{\pgfqpoint{3.721419in}{2.243309in}}%
\pgfpathlineto{\pgfqpoint{3.721622in}{2.267993in}}%
\pgfpathlineto{\pgfqpoint{3.722030in}{2.218624in}}%
\pgfpathlineto{\pgfqpoint{3.722234in}{2.243309in}}%
\pgfpathlineto{\pgfqpoint{3.722845in}{1.897724in}}%
\pgfpathlineto{\pgfqpoint{3.723457in}{2.128114in}}%
\pgfpathlineto{\pgfqpoint{3.723864in}{2.202168in}}%
\pgfpathlineto{\pgfqpoint{3.724068in}{2.119886in}}%
\pgfpathlineto{\pgfqpoint{3.724272in}{1.881268in}}%
\pgfpathlineto{\pgfqpoint{3.724883in}{2.235080in}}%
\pgfpathlineto{\pgfqpoint{3.725087in}{2.152798in}}%
\pgfpathlineto{\pgfqpoint{3.725495in}{2.309134in}}%
\pgfpathlineto{\pgfqpoint{3.726310in}{2.284450in}}%
\pgfpathlineto{\pgfqpoint{3.728959in}{2.054060in}}%
\pgfpathlineto{\pgfqpoint{3.729978in}{2.300906in}}%
\pgfpathlineto{\pgfqpoint{3.730386in}{2.276221in}}%
\pgfpathlineto{\pgfqpoint{3.730590in}{2.276221in}}%
\pgfpathlineto{\pgfqpoint{3.730997in}{2.267993in}}%
\pgfpathlineto{\pgfqpoint{3.731813in}{2.325591in}}%
\pgfpathlineto{\pgfqpoint{3.732016in}{2.317363in}}%
\pgfpathlineto{\pgfqpoint{3.732220in}{2.374960in}}%
\pgfpathlineto{\pgfqpoint{3.732628in}{2.267993in}}%
\pgfpathlineto{\pgfqpoint{3.732832in}{2.267993in}}%
\pgfpathlineto{\pgfqpoint{3.733035in}{2.029375in}}%
\pgfpathlineto{\pgfqpoint{3.733443in}{2.300906in}}%
\pgfpathlineto{\pgfqpoint{3.733851in}{2.284450in}}%
\pgfpathlineto{\pgfqpoint{3.735073in}{2.243309in}}%
\pgfpathlineto{\pgfqpoint{3.735685in}{2.317363in}}%
\pgfpathlineto{\pgfqpoint{3.736092in}{2.259765in}}%
\pgfpathlineto{\pgfqpoint{3.736500in}{2.317363in}}%
\pgfpathlineto{\pgfqpoint{3.737315in}{2.300906in}}%
\pgfpathlineto{\pgfqpoint{3.737519in}{2.267993in}}%
\pgfpathlineto{\pgfqpoint{3.737927in}{2.358504in}}%
\pgfpathlineto{\pgfqpoint{3.738130in}{2.358504in}}%
\pgfpathlineto{\pgfqpoint{3.738334in}{2.424329in}}%
\pgfpathlineto{\pgfqpoint{3.738946in}{2.309134in}}%
\pgfpathlineto{\pgfqpoint{3.739150in}{2.350275in}}%
\pgfpathlineto{\pgfqpoint{3.739965in}{2.292678in}}%
\pgfpathlineto{\pgfqpoint{3.740576in}{2.366732in}}%
\pgfpathlineto{\pgfqpoint{3.740984in}{2.325591in}}%
\pgfpathlineto{\pgfqpoint{3.741188in}{2.309134in}}%
\pgfpathlineto{\pgfqpoint{3.741595in}{2.424329in}}%
\pgfpathlineto{\pgfqpoint{3.742207in}{2.383188in}}%
\pgfpathlineto{\pgfqpoint{3.743429in}{2.292678in}}%
\pgfpathlineto{\pgfqpoint{3.742614in}{2.399645in}}%
\pgfpathlineto{\pgfqpoint{3.743633in}{2.317363in}}%
\pgfpathlineto{\pgfqpoint{3.743837in}{2.333819in}}%
\pgfpathlineto{\pgfqpoint{3.744041in}{2.309134in}}%
\pgfpathlineto{\pgfqpoint{3.744245in}{2.251537in}}%
\pgfpathlineto{\pgfqpoint{3.744448in}{2.325591in}}%
\pgfpathlineto{\pgfqpoint{3.744856in}{2.325591in}}%
\pgfpathlineto{\pgfqpoint{3.745467in}{2.374960in}}%
\pgfpathlineto{\pgfqpoint{3.745875in}{2.342047in}}%
\pgfpathlineto{\pgfqpoint{3.746283in}{2.300906in}}%
\pgfpathlineto{\pgfqpoint{3.746486in}{2.391416in}}%
\pgfpathlineto{\pgfqpoint{3.747302in}{2.292678in}}%
\pgfpathlineto{\pgfqpoint{3.748524in}{2.399645in}}%
\pgfpathlineto{\pgfqpoint{3.748932in}{2.358504in}}%
\pgfpathlineto{\pgfqpoint{3.749543in}{2.325591in}}%
\pgfpathlineto{\pgfqpoint{3.749951in}{2.374960in}}%
\pgfpathlineto{\pgfqpoint{3.750562in}{2.350275in}}%
\pgfpathlineto{\pgfqpoint{3.751785in}{2.267993in}}%
\pgfpathlineto{\pgfqpoint{3.752193in}{2.276221in}}%
\pgfpathlineto{\pgfqpoint{3.752397in}{2.325591in}}%
\pgfpathlineto{\pgfqpoint{3.753212in}{2.284450in}}%
\pgfpathlineto{\pgfqpoint{3.753620in}{2.300906in}}%
\pgfpathlineto{\pgfqpoint{3.753823in}{2.251537in}}%
\pgfpathlineto{\pgfqpoint{3.754027in}{2.267993in}}%
\pgfpathlineto{\pgfqpoint{3.754231in}{2.226852in}}%
\pgfpathlineto{\pgfqpoint{3.754639in}{2.317363in}}%
\pgfpathlineto{\pgfqpoint{3.755046in}{2.276221in}}%
\pgfpathlineto{\pgfqpoint{3.756269in}{2.366732in}}%
\pgfpathlineto{\pgfqpoint{3.757084in}{2.449014in}}%
\pgfpathlineto{\pgfqpoint{3.757492in}{2.399645in}}%
\pgfpathlineto{\pgfqpoint{3.758918in}{2.333819in}}%
\pgfpathlineto{\pgfqpoint{3.759122in}{2.391416in}}%
\pgfpathlineto{\pgfqpoint{3.759326in}{2.309134in}}%
\pgfpathlineto{\pgfqpoint{3.759530in}{2.185711in}}%
\pgfpathlineto{\pgfqpoint{3.760141in}{2.374960in}}%
\pgfpathlineto{\pgfqpoint{3.760345in}{2.440786in}}%
\pgfpathlineto{\pgfqpoint{3.761160in}{2.358504in}}%
\pgfpathlineto{\pgfqpoint{3.761364in}{2.416101in}}%
\pgfpathlineto{\pgfqpoint{3.761772in}{2.399645in}}%
\pgfpathlineto{\pgfqpoint{3.761975in}{2.424329in}}%
\pgfpathlineto{\pgfqpoint{3.762587in}{2.111657in}}%
\pgfpathlineto{\pgfqpoint{3.762994in}{2.276221in}}%
\pgfpathlineto{\pgfqpoint{3.763810in}{2.449014in}}%
\pgfpathlineto{\pgfqpoint{3.764013in}{2.202168in}}%
\pgfpathlineto{\pgfqpoint{3.764829in}{2.407873in}}%
\pgfpathlineto{\pgfqpoint{3.765032in}{2.407873in}}%
\pgfpathlineto{\pgfqpoint{3.765440in}{2.374960in}}%
\pgfpathlineto{\pgfqpoint{3.766052in}{2.399645in}}%
\pgfpathlineto{\pgfqpoint{3.766459in}{2.416101in}}%
\pgfpathlineto{\pgfqpoint{3.766663in}{2.391416in}}%
\pgfpathlineto{\pgfqpoint{3.766867in}{2.399645in}}%
\pgfpathlineto{\pgfqpoint{3.767071in}{2.374960in}}%
\pgfpathlineto{\pgfqpoint{3.767274in}{2.103429in}}%
\pgfpathlineto{\pgfqpoint{3.768090in}{2.317363in}}%
\pgfpathlineto{\pgfqpoint{3.769109in}{2.383188in}}%
\pgfpathlineto{\pgfqpoint{3.769312in}{2.358504in}}%
\pgfpathlineto{\pgfqpoint{3.770128in}{2.284450in}}%
\pgfpathlineto{\pgfqpoint{3.770331in}{2.309134in}}%
\pgfpathlineto{\pgfqpoint{3.770535in}{2.383188in}}%
\pgfpathlineto{\pgfqpoint{3.771350in}{2.325591in}}%
\pgfpathlineto{\pgfqpoint{3.772573in}{2.251537in}}%
\pgfpathlineto{\pgfqpoint{3.772777in}{2.292678in}}%
\pgfpathlineto{\pgfqpoint{3.773592in}{2.259765in}}%
\pgfpathlineto{\pgfqpoint{3.773796in}{2.259765in}}%
\pgfpathlineto{\pgfqpoint{3.774000in}{2.300906in}}%
\pgfpathlineto{\pgfqpoint{3.774407in}{2.226852in}}%
\pgfpathlineto{\pgfqpoint{3.774815in}{2.243309in}}%
\pgfpathlineto{\pgfqpoint{3.776649in}{2.161027in}}%
\pgfpathlineto{\pgfqpoint{3.777872in}{2.243309in}}%
\pgfpathlineto{\pgfqpoint{3.778076in}{2.210396in}}%
\pgfpathlineto{\pgfqpoint{3.778280in}{2.169255in}}%
\pgfpathlineto{\pgfqpoint{3.778891in}{2.267993in}}%
\pgfpathlineto{\pgfqpoint{3.779503in}{2.276221in}}%
\pgfpathlineto{\pgfqpoint{3.780114in}{2.202168in}}%
\pgfpathlineto{\pgfqpoint{3.781541in}{2.309134in}}%
\pgfpathlineto{\pgfqpoint{3.781948in}{2.226852in}}%
\pgfpathlineto{\pgfqpoint{3.782152in}{2.119886in}}%
\pgfpathlineto{\pgfqpoint{3.782763in}{2.284450in}}%
\pgfpathlineto{\pgfqpoint{3.782967in}{2.235080in}}%
\pgfpathlineto{\pgfqpoint{3.783579in}{2.152798in}}%
\pgfpathlineto{\pgfqpoint{3.783782in}{2.169255in}}%
\pgfpathlineto{\pgfqpoint{3.783986in}{1.971778in}}%
\pgfpathlineto{\pgfqpoint{3.784394in}{2.259765in}}%
\pgfpathlineto{\pgfqpoint{3.784801in}{1.988234in}}%
\pgfpathlineto{\pgfqpoint{3.785005in}{2.243309in}}%
\pgfpathlineto{\pgfqpoint{3.785413in}{1.864811in}}%
\pgfpathlineto{\pgfqpoint{3.786024in}{2.161027in}}%
\pgfpathlineto{\pgfqpoint{3.786432in}{2.193939in}}%
\pgfpathlineto{\pgfqpoint{3.786839in}{2.136342in}}%
\pgfpathlineto{\pgfqpoint{3.787043in}{2.070516in}}%
\pgfpathlineto{\pgfqpoint{3.787247in}{2.235080in}}%
\pgfpathlineto{\pgfqpoint{3.787655in}{2.161027in}}%
\pgfpathlineto{\pgfqpoint{3.789489in}{2.350275in}}%
\pgfpathlineto{\pgfqpoint{3.789693in}{2.309134in}}%
\pgfpathlineto{\pgfqpoint{3.790304in}{2.383188in}}%
\pgfpathlineto{\pgfqpoint{3.790508in}{2.399645in}}%
\pgfpathlineto{\pgfqpoint{3.791527in}{2.235080in}}%
\pgfpathlineto{\pgfqpoint{3.791934in}{2.251537in}}%
\pgfpathlineto{\pgfqpoint{3.792954in}{2.243309in}}%
\pgfpathlineto{\pgfqpoint{3.793769in}{2.342047in}}%
\pgfpathlineto{\pgfqpoint{3.793973in}{2.309134in}}%
\pgfpathlineto{\pgfqpoint{3.794176in}{2.350275in}}%
\pgfpathlineto{\pgfqpoint{3.794584in}{2.325591in}}%
\pgfpathlineto{\pgfqpoint{3.794788in}{2.399645in}}%
\pgfpathlineto{\pgfqpoint{3.795603in}{2.333819in}}%
\pgfpathlineto{\pgfqpoint{3.797030in}{2.399645in}}%
\pgfpathlineto{\pgfqpoint{3.797845in}{2.267993in}}%
\pgfpathlineto{\pgfqpoint{3.798252in}{2.284450in}}%
\pgfpathlineto{\pgfqpoint{3.798660in}{2.267993in}}%
\pgfpathlineto{\pgfqpoint{3.799068in}{2.111657in}}%
\pgfpathlineto{\pgfqpoint{3.799271in}{2.350275in}}%
\pgfpathlineto{\pgfqpoint{3.799475in}{2.325591in}}%
\pgfpathlineto{\pgfqpoint{3.800290in}{2.383188in}}%
\pgfpathlineto{\pgfqpoint{3.800698in}{2.342047in}}%
\pgfpathlineto{\pgfqpoint{3.802125in}{2.243309in}}%
\pgfpathlineto{\pgfqpoint{3.801106in}{2.350275in}}%
\pgfpathlineto{\pgfqpoint{3.802328in}{2.267993in}}%
\pgfpathlineto{\pgfqpoint{3.803144in}{2.342047in}}%
\pgfpathlineto{\pgfqpoint{3.803551in}{2.062288in}}%
\pgfpathlineto{\pgfqpoint{3.804163in}{2.366732in}}%
\pgfpathlineto{\pgfqpoint{3.804570in}{2.086973in}}%
\pgfpathlineto{\pgfqpoint{3.805385in}{2.325591in}}%
\pgfpathlineto{\pgfqpoint{3.806405in}{2.202168in}}%
\pgfpathlineto{\pgfqpoint{3.806812in}{2.243309in}}%
\pgfpathlineto{\pgfqpoint{3.807016in}{2.317363in}}%
\pgfpathlineto{\pgfqpoint{3.808035in}{2.292678in}}%
\pgfpathlineto{\pgfqpoint{3.808239in}{2.292678in}}%
\pgfpathlineto{\pgfqpoint{3.808443in}{2.267993in}}%
\pgfpathlineto{\pgfqpoint{3.808646in}{2.333819in}}%
\pgfpathlineto{\pgfqpoint{3.808850in}{2.317363in}}%
\pgfpathlineto{\pgfqpoint{3.809462in}{2.300906in}}%
\pgfpathlineto{\pgfqpoint{3.809869in}{2.342047in}}%
\pgfpathlineto{\pgfqpoint{3.810684in}{2.210396in}}%
\pgfpathlineto{\pgfqpoint{3.810888in}{2.276221in}}%
\pgfpathlineto{\pgfqpoint{3.812315in}{2.366732in}}%
\pgfpathlineto{\pgfqpoint{3.813945in}{2.284450in}}%
\pgfpathlineto{\pgfqpoint{3.814149in}{2.325591in}}%
\pgfpathlineto{\pgfqpoint{3.814557in}{2.243309in}}%
\pgfpathlineto{\pgfqpoint{3.814964in}{2.284450in}}%
\pgfpathlineto{\pgfqpoint{3.815168in}{2.259765in}}%
\pgfpathlineto{\pgfqpoint{3.815372in}{2.317363in}}%
\pgfpathlineto{\pgfqpoint{3.815576in}{2.317363in}}%
\pgfpathlineto{\pgfqpoint{3.815983in}{2.366732in}}%
\pgfpathlineto{\pgfqpoint{3.816391in}{2.333819in}}%
\pgfpathlineto{\pgfqpoint{3.817410in}{2.226852in}}%
\pgfpathlineto{\pgfqpoint{3.817614in}{2.259765in}}%
\pgfpathlineto{\pgfqpoint{3.819040in}{2.366732in}}%
\pgfpathlineto{\pgfqpoint{3.819244in}{2.416101in}}%
\pgfpathlineto{\pgfqpoint{3.819856in}{2.325591in}}%
\pgfpathlineto{\pgfqpoint{3.820059in}{2.333819in}}%
\pgfpathlineto{\pgfqpoint{3.820263in}{2.325591in}}%
\pgfpathlineto{\pgfqpoint{3.820467in}{2.251537in}}%
\pgfpathlineto{\pgfqpoint{3.821282in}{2.292678in}}%
\pgfpathlineto{\pgfqpoint{3.822301in}{2.342047in}}%
\pgfpathlineto{\pgfqpoint{3.822913in}{2.218624in}}%
\pgfpathlineto{\pgfqpoint{3.823524in}{2.317363in}}%
\pgfpathlineto{\pgfqpoint{3.825562in}{2.473698in}}%
\pgfpathlineto{\pgfqpoint{3.826173in}{2.342047in}}%
\pgfpathlineto{\pgfqpoint{3.826785in}{2.416101in}}%
\pgfpathlineto{\pgfqpoint{3.826989in}{2.424329in}}%
\pgfpathlineto{\pgfqpoint{3.827192in}{2.391416in}}%
\pgfpathlineto{\pgfqpoint{3.827804in}{2.366732in}}%
\pgfpathlineto{\pgfqpoint{3.828415in}{2.432557in}}%
\pgfpathlineto{\pgfqpoint{3.828823in}{2.391416in}}%
\pgfpathlineto{\pgfqpoint{3.829027in}{2.358504in}}%
\pgfpathlineto{\pgfqpoint{3.829434in}{2.432557in}}%
\pgfpathlineto{\pgfqpoint{3.829638in}{2.440786in}}%
\pgfpathlineto{\pgfqpoint{3.829842in}{2.391416in}}%
\pgfpathlineto{\pgfqpoint{3.830453in}{2.473698in}}%
\pgfpathlineto{\pgfqpoint{3.830657in}{2.432557in}}%
\pgfpathlineto{\pgfqpoint{3.830861in}{2.465470in}}%
\pgfpathlineto{\pgfqpoint{3.831065in}{2.416101in}}%
\pgfpathlineto{\pgfqpoint{3.831472in}{2.416101in}}%
\pgfpathlineto{\pgfqpoint{3.831676in}{2.416101in}}%
\pgfpathlineto{\pgfqpoint{3.832287in}{2.391416in}}%
\pgfpathlineto{\pgfqpoint{3.832491in}{2.457242in}}%
\pgfpathlineto{\pgfqpoint{3.832899in}{2.383188in}}%
\pgfpathlineto{\pgfqpoint{3.833510in}{2.432557in}}%
\pgfpathlineto{\pgfqpoint{3.833918in}{2.383188in}}%
\pgfpathlineto{\pgfqpoint{3.834122in}{2.424329in}}%
\pgfpathlineto{\pgfqpoint{3.835141in}{2.473698in}}%
\pgfpathlineto{\pgfqpoint{3.835548in}{2.465470in}}%
\pgfpathlineto{\pgfqpoint{3.836771in}{2.383188in}}%
\pgfpathlineto{\pgfqpoint{3.837179in}{2.407873in}}%
\pgfpathlineto{\pgfqpoint{3.837383in}{2.465470in}}%
\pgfpathlineto{\pgfqpoint{3.837790in}{2.374960in}}%
\pgfpathlineto{\pgfqpoint{3.838198in}{2.416101in}}%
\pgfpathlineto{\pgfqpoint{3.839828in}{2.333819in}}%
\pgfpathlineto{\pgfqpoint{3.840236in}{2.251537in}}%
\pgfpathlineto{\pgfqpoint{3.840847in}{2.317363in}}%
\pgfpathlineto{\pgfqpoint{3.842070in}{2.391416in}}%
\pgfpathlineto{\pgfqpoint{3.842274in}{2.383188in}}%
\pgfpathlineto{\pgfqpoint{3.842885in}{2.366732in}}%
\pgfpathlineto{\pgfqpoint{3.843089in}{2.399645in}}%
\pgfpathlineto{\pgfqpoint{3.843497in}{2.374960in}}%
\pgfpathlineto{\pgfqpoint{3.843700in}{2.399645in}}%
\pgfpathlineto{\pgfqpoint{3.843904in}{2.432557in}}%
\pgfpathlineto{\pgfqpoint{3.844312in}{2.358504in}}%
\pgfpathlineto{\pgfqpoint{3.845331in}{2.259765in}}%
\pgfpathlineto{\pgfqpoint{3.845738in}{2.292678in}}%
\pgfpathlineto{\pgfqpoint{3.846146in}{2.251537in}}%
\pgfpathlineto{\pgfqpoint{3.846350in}{2.350275in}}%
\pgfpathlineto{\pgfqpoint{3.847165in}{2.333819in}}%
\pgfpathlineto{\pgfqpoint{3.847573in}{2.218624in}}%
\pgfpathlineto{\pgfqpoint{3.848388in}{2.235080in}}%
\pgfpathlineto{\pgfqpoint{3.849815in}{2.317363in}}%
\pgfpathlineto{\pgfqpoint{3.850018in}{2.317363in}}%
\pgfpathlineto{\pgfqpoint{3.850834in}{2.342047in}}%
\pgfpathlineto{\pgfqpoint{3.851241in}{2.243309in}}%
\pgfpathlineto{\pgfqpoint{3.852056in}{2.284450in}}%
\pgfpathlineto{\pgfqpoint{3.852872in}{2.243309in}}%
\pgfpathlineto{\pgfqpoint{3.853279in}{2.259765in}}%
\pgfpathlineto{\pgfqpoint{3.854298in}{2.424329in}}%
\pgfpathlineto{\pgfqpoint{3.854502in}{2.111657in}}%
\pgfpathlineto{\pgfqpoint{3.855317in}{2.350275in}}%
\pgfpathlineto{\pgfqpoint{3.855521in}{2.374960in}}%
\pgfpathlineto{\pgfqpoint{3.855929in}{2.226852in}}%
\pgfpathlineto{\pgfqpoint{3.856540in}{2.300906in}}%
\pgfpathlineto{\pgfqpoint{3.856948in}{2.333819in}}%
\pgfpathlineto{\pgfqpoint{3.857151in}{2.152798in}}%
\pgfpathlineto{\pgfqpoint{3.857967in}{2.383188in}}%
\pgfpathlineto{\pgfqpoint{3.858374in}{2.292678in}}%
\pgfpathlineto{\pgfqpoint{3.858986in}{2.374960in}}%
\pgfpathlineto{\pgfqpoint{3.859189in}{2.358504in}}%
\pgfpathlineto{\pgfqpoint{3.859597in}{2.144570in}}%
\pgfpathlineto{\pgfqpoint{3.860209in}{2.350275in}}%
\pgfpathlineto{\pgfqpoint{3.860820in}{2.374960in}}%
\pgfpathlineto{\pgfqpoint{3.861635in}{2.284450in}}%
\pgfpathlineto{\pgfqpoint{3.861839in}{2.284450in}}%
\pgfpathlineto{\pgfqpoint{3.862043in}{2.078745in}}%
\pgfpathlineto{\pgfqpoint{3.862654in}{2.407873in}}%
\pgfpathlineto{\pgfqpoint{3.862858in}{2.333819in}}%
\pgfpathlineto{\pgfqpoint{3.863266in}{2.358504in}}%
\pgfpathlineto{\pgfqpoint{3.863469in}{2.317363in}}%
\pgfpathlineto{\pgfqpoint{3.864081in}{2.424329in}}%
\pgfpathlineto{\pgfqpoint{3.864285in}{2.300906in}}%
\pgfpathlineto{\pgfqpoint{3.864488in}{2.300906in}}%
\pgfpathlineto{\pgfqpoint{3.864692in}{2.325591in}}%
\pgfpathlineto{\pgfqpoint{3.865304in}{2.309134in}}%
\pgfpathlineto{\pgfqpoint{3.866323in}{2.169255in}}%
\pgfpathlineto{\pgfqpoint{3.867138in}{2.210396in}}%
\pgfpathlineto{\pgfqpoint{3.868564in}{2.333819in}}%
\pgfpathlineto{\pgfqpoint{3.867953in}{2.202168in}}%
\pgfpathlineto{\pgfqpoint{3.868768in}{2.325591in}}%
\pgfpathlineto{\pgfqpoint{3.868972in}{2.342047in}}%
\pgfpathlineto{\pgfqpoint{3.869176in}{2.300906in}}%
\pgfpathlineto{\pgfqpoint{3.869380in}{2.259765in}}%
\pgfpathlineto{\pgfqpoint{3.869991in}{2.333819in}}%
\pgfpathlineto{\pgfqpoint{3.870195in}{2.350275in}}%
\pgfpathlineto{\pgfqpoint{3.870602in}{2.309134in}}%
\pgfpathlineto{\pgfqpoint{3.871418in}{2.243309in}}%
\pgfpathlineto{\pgfqpoint{3.871214in}{2.350275in}}%
\pgfpathlineto{\pgfqpoint{3.871621in}{2.251537in}}%
\pgfpathlineto{\pgfqpoint{3.871825in}{2.300906in}}%
\pgfpathlineto{\pgfqpoint{3.872233in}{2.235080in}}%
\pgfpathlineto{\pgfqpoint{3.872437in}{2.243309in}}%
\pgfpathlineto{\pgfqpoint{3.873048in}{2.161027in}}%
\pgfpathlineto{\pgfqpoint{3.873456in}{2.210396in}}%
\pgfpathlineto{\pgfqpoint{3.874271in}{2.276221in}}%
\pgfpathlineto{\pgfqpoint{3.874475in}{2.226852in}}%
\pgfpathlineto{\pgfqpoint{3.875086in}{2.202168in}}%
\pgfpathlineto{\pgfqpoint{3.874882in}{2.235080in}}%
\pgfpathlineto{\pgfqpoint{3.875290in}{2.226852in}}%
\pgfpathlineto{\pgfqpoint{3.875901in}{2.267993in}}%
\pgfpathlineto{\pgfqpoint{3.876717in}{2.259765in}}%
\pgfpathlineto{\pgfqpoint{3.877939in}{2.119886in}}%
\pgfpathlineto{\pgfqpoint{3.878347in}{2.177483in}}%
\pgfpathlineto{\pgfqpoint{3.878755in}{2.062288in}}%
\pgfpathlineto{\pgfqpoint{3.879366in}{2.095201in}}%
\pgfpathlineto{\pgfqpoint{3.880181in}{1.790758in}}%
\pgfpathlineto{\pgfqpoint{3.881200in}{2.021147in}}%
\pgfpathlineto{\pgfqpoint{3.881404in}{1.988234in}}%
\pgfpathlineto{\pgfqpoint{3.881608in}{2.012919in}}%
\pgfpathlineto{\pgfqpoint{3.881812in}{1.947093in}}%
\pgfpathlineto{\pgfqpoint{3.882015in}{1.980006in}}%
\pgfpathlineto{\pgfqpoint{3.883034in}{1.848355in}}%
\pgfpathlineto{\pgfqpoint{3.883646in}{1.856583in}}%
\pgfpathlineto{\pgfqpoint{3.884461in}{1.930637in}}%
\pgfpathlineto{\pgfqpoint{3.884869in}{2.070516in}}%
\pgfpathlineto{\pgfqpoint{3.885480in}{1.922409in}}%
\pgfpathlineto{\pgfqpoint{3.885684in}{1.930637in}}%
\pgfpathlineto{\pgfqpoint{3.885888in}{2.054060in}}%
\pgfpathlineto{\pgfqpoint{3.886703in}{1.980006in}}%
\pgfpathlineto{\pgfqpoint{3.886907in}{1.988234in}}%
\pgfpathlineto{\pgfqpoint{3.887111in}{1.971778in}}%
\pgfpathlineto{\pgfqpoint{3.887518in}{1.947093in}}%
\pgfpathlineto{\pgfqpoint{3.887926in}{1.980006in}}%
\pgfpathlineto{\pgfqpoint{3.888945in}{2.152798in}}%
\pgfpathlineto{\pgfqpoint{3.889352in}{2.103429in}}%
\pgfpathlineto{\pgfqpoint{3.890371in}{2.045832in}}%
\pgfpathlineto{\pgfqpoint{3.891187in}{2.095201in}}%
\pgfpathlineto{\pgfqpoint{3.891390in}{2.070516in}}%
\pgfpathlineto{\pgfqpoint{3.892002in}{2.095201in}}%
\pgfpathlineto{\pgfqpoint{3.892409in}{2.045832in}}%
\pgfpathlineto{\pgfqpoint{3.893225in}{2.095201in}}%
\pgfpathlineto{\pgfqpoint{3.893428in}{2.037604in}}%
\pgfpathlineto{\pgfqpoint{3.893632in}{2.029375in}}%
\pgfpathlineto{\pgfqpoint{3.894040in}{2.103429in}}%
\pgfpathlineto{\pgfqpoint{3.894651in}{2.012919in}}%
\pgfpathlineto{\pgfqpoint{3.895263in}{2.070516in}}%
\pgfpathlineto{\pgfqpoint{3.895874in}{2.045832in}}%
\pgfpathlineto{\pgfqpoint{3.896282in}{1.938865in}}%
\pgfpathlineto{\pgfqpoint{3.896893in}{2.045832in}}%
\pgfpathlineto{\pgfqpoint{3.897097in}{2.037604in}}%
\pgfpathlineto{\pgfqpoint{3.897301in}{2.086973in}}%
\pgfpathlineto{\pgfqpoint{3.897504in}{1.955322in}}%
\pgfpathlineto{\pgfqpoint{3.897912in}{1.971778in}}%
\pgfpathlineto{\pgfqpoint{3.898320in}{1.955322in}}%
\pgfpathlineto{\pgfqpoint{3.898523in}{2.045832in}}%
\pgfpathlineto{\pgfqpoint{3.899339in}{1.971778in}}%
\pgfpathlineto{\pgfqpoint{3.899950in}{1.980006in}}%
\pgfpathlineto{\pgfqpoint{3.900562in}{1.914181in}}%
\pgfpathlineto{\pgfqpoint{3.900765in}{1.905952in}}%
\pgfpathlineto{\pgfqpoint{3.900969in}{1.971778in}}%
\pgfpathlineto{\pgfqpoint{3.901377in}{1.955322in}}%
\pgfpathlineto{\pgfqpoint{3.901988in}{2.021147in}}%
\pgfpathlineto{\pgfqpoint{3.902600in}{1.782529in}}%
\pgfpathlineto{\pgfqpoint{3.903415in}{2.086973in}}%
\pgfpathlineto{\pgfqpoint{3.903822in}{2.012919in}}%
\pgfpathlineto{\pgfqpoint{3.904026in}{1.980006in}}%
\pgfpathlineto{\pgfqpoint{3.904230in}{2.045832in}}%
\pgfpathlineto{\pgfqpoint{3.904434in}{2.021147in}}%
\pgfpathlineto{\pgfqpoint{3.905249in}{2.086973in}}%
\pgfpathlineto{\pgfqpoint{3.905657in}{1.782529in}}%
\pgfpathlineto{\pgfqpoint{3.906268in}{2.054060in}}%
\pgfpathlineto{\pgfqpoint{3.906676in}{2.029375in}}%
\pgfpathlineto{\pgfqpoint{3.907083in}{1.930637in}}%
\pgfpathlineto{\pgfqpoint{3.907695in}{2.004691in}}%
\pgfpathlineto{\pgfqpoint{3.908102in}{2.045832in}}%
\pgfpathlineto{\pgfqpoint{3.908306in}{2.021147in}}%
\pgfpathlineto{\pgfqpoint{3.908510in}{1.955322in}}%
\pgfpathlineto{\pgfqpoint{3.909121in}{2.054060in}}%
\pgfpathlineto{\pgfqpoint{3.909325in}{2.054060in}}%
\pgfpathlineto{\pgfqpoint{3.909529in}{2.086973in}}%
\pgfpathlineto{\pgfqpoint{3.909936in}{1.980006in}}%
\pgfpathlineto{\pgfqpoint{3.910344in}{2.045832in}}%
\pgfpathlineto{\pgfqpoint{3.910548in}{2.062288in}}%
\pgfpathlineto{\pgfqpoint{3.910955in}{1.766073in}}%
\pgfpathlineto{\pgfqpoint{3.911771in}{1.955322in}}%
\pgfpathlineto{\pgfqpoint{3.912790in}{2.004691in}}%
\pgfpathlineto{\pgfqpoint{3.912993in}{1.980006in}}%
\pgfpathlineto{\pgfqpoint{3.913401in}{2.029375in}}%
\pgfpathlineto{\pgfqpoint{3.913605in}{2.045832in}}%
\pgfpathlineto{\pgfqpoint{3.914420in}{2.029375in}}%
\pgfpathlineto{\pgfqpoint{3.914624in}{2.021147in}}%
\pgfpathlineto{\pgfqpoint{3.915032in}{1.757845in}}%
\pgfpathlineto{\pgfqpoint{3.915847in}{1.774301in}}%
\pgfpathlineto{\pgfqpoint{3.916866in}{1.963550in}}%
\pgfpathlineto{\pgfqpoint{3.917070in}{1.938865in}}%
\pgfpathlineto{\pgfqpoint{3.918089in}{1.889496in}}%
\pgfpathlineto{\pgfqpoint{3.918904in}{2.045832in}}%
\pgfpathlineto{\pgfqpoint{3.919515in}{1.988234in}}%
\pgfpathlineto{\pgfqpoint{3.920127in}{1.856583in}}%
\pgfpathlineto{\pgfqpoint{3.920534in}{1.930637in}}%
\pgfpathlineto{\pgfqpoint{3.921146in}{1.996463in}}%
\pgfpathlineto{\pgfqpoint{3.921757in}{1.955322in}}%
\pgfpathlineto{\pgfqpoint{3.923184in}{1.897724in}}%
\pgfpathlineto{\pgfqpoint{3.923591in}{1.922409in}}%
\pgfpathlineto{\pgfqpoint{3.923795in}{1.947093in}}%
\pgfpathlineto{\pgfqpoint{3.924203in}{1.905952in}}%
\pgfpathlineto{\pgfqpoint{3.924406in}{1.864811in}}%
\pgfpathlineto{\pgfqpoint{3.924814in}{1.938865in}}%
\pgfpathlineto{\pgfqpoint{3.925018in}{1.930637in}}%
\pgfpathlineto{\pgfqpoint{3.925222in}{1.938865in}}%
\pgfpathlineto{\pgfqpoint{3.925629in}{1.700247in}}%
\pgfpathlineto{\pgfqpoint{3.926444in}{1.873040in}}%
\pgfpathlineto{\pgfqpoint{3.927260in}{1.922409in}}%
\pgfpathlineto{\pgfqpoint{3.927667in}{1.790758in}}%
\pgfpathlineto{\pgfqpoint{3.928483in}{1.831899in}}%
\pgfpathlineto{\pgfqpoint{3.928686in}{1.831899in}}%
\pgfpathlineto{\pgfqpoint{3.929909in}{1.955322in}}%
\pgfpathlineto{\pgfqpoint{3.930113in}{1.914181in}}%
\pgfpathlineto{\pgfqpoint{3.930317in}{1.914181in}}%
\pgfpathlineto{\pgfqpoint{3.931132in}{1.873040in}}%
\pgfpathlineto{\pgfqpoint{3.930928in}{1.922409in}}%
\pgfpathlineto{\pgfqpoint{3.931336in}{1.889496in}}%
\pgfpathlineto{\pgfqpoint{3.931743in}{1.683791in}}%
\pgfpathlineto{\pgfqpoint{3.932151in}{1.905952in}}%
\pgfpathlineto{\pgfqpoint{3.932355in}{1.881268in}}%
\pgfpathlineto{\pgfqpoint{3.932762in}{1.955322in}}%
\pgfpathlineto{\pgfqpoint{3.933374in}{1.938865in}}%
\pgfpathlineto{\pgfqpoint{3.933578in}{1.897724in}}%
\pgfpathlineto{\pgfqpoint{3.933985in}{1.947093in}}%
\pgfpathlineto{\pgfqpoint{3.934393in}{2.021147in}}%
\pgfpathlineto{\pgfqpoint{3.934800in}{1.930637in}}%
\pgfpathlineto{\pgfqpoint{3.935004in}{1.971778in}}%
\pgfpathlineto{\pgfqpoint{3.936023in}{1.889496in}}%
\pgfpathlineto{\pgfqpoint{3.936227in}{1.922409in}}%
\pgfpathlineto{\pgfqpoint{3.936431in}{1.963550in}}%
\pgfpathlineto{\pgfqpoint{3.937246in}{1.914181in}}%
\pgfpathlineto{\pgfqpoint{3.937654in}{2.012919in}}%
\pgfpathlineto{\pgfqpoint{3.938061in}{1.930637in}}%
\pgfpathlineto{\pgfqpoint{3.938265in}{1.905952in}}%
\pgfpathlineto{\pgfqpoint{3.938469in}{1.980006in}}%
\pgfpathlineto{\pgfqpoint{3.938673in}{1.947093in}}%
\pgfpathlineto{\pgfqpoint{3.939284in}{2.012919in}}%
\pgfpathlineto{\pgfqpoint{3.939080in}{1.930637in}}%
\pgfpathlineto{\pgfqpoint{3.939692in}{2.004691in}}%
\pgfpathlineto{\pgfqpoint{3.940915in}{1.881268in}}%
\pgfpathlineto{\pgfqpoint{3.941118in}{1.881268in}}%
\pgfpathlineto{\pgfqpoint{3.941322in}{1.642650in}}%
\pgfpathlineto{\pgfqpoint{3.942137in}{1.955322in}}%
\pgfpathlineto{\pgfqpoint{3.942545in}{1.889496in}}%
\pgfpathlineto{\pgfqpoint{3.942953in}{1.798986in}}%
\pgfpathlineto{\pgfqpoint{3.943564in}{1.897724in}}%
\pgfpathlineto{\pgfqpoint{3.943768in}{1.897724in}}%
\pgfpathlineto{\pgfqpoint{3.944175in}{1.848355in}}%
\pgfpathlineto{\pgfqpoint{3.945398in}{1.757845in}}%
\pgfpathlineto{\pgfqpoint{3.945602in}{1.807214in}}%
\pgfpathlineto{\pgfqpoint{3.945806in}{1.724932in}}%
\pgfpathlineto{\pgfqpoint{3.946417in}{1.782529in}}%
\pgfpathlineto{\pgfqpoint{3.947436in}{1.708476in}}%
\pgfpathlineto{\pgfqpoint{3.947640in}{1.733160in}}%
\pgfpathlineto{\pgfqpoint{3.948455in}{1.790758in}}%
\pgfpathlineto{\pgfqpoint{3.948659in}{1.782529in}}%
\pgfpathlineto{\pgfqpoint{3.948863in}{1.840127in}}%
\pgfpathlineto{\pgfqpoint{3.949270in}{1.766073in}}%
\pgfpathlineto{\pgfqpoint{3.949474in}{1.593281in}}%
\pgfpathlineto{\pgfqpoint{3.949678in}{1.848355in}}%
\pgfpathlineto{\pgfqpoint{3.950289in}{1.815442in}}%
\pgfpathlineto{\pgfqpoint{3.951308in}{1.955322in}}%
\pgfpathlineto{\pgfqpoint{3.951716in}{1.905952in}}%
\pgfpathlineto{\pgfqpoint{3.953958in}{1.683791in}}%
\pgfpathlineto{\pgfqpoint{3.954366in}{1.757845in}}%
\pgfpathlineto{\pgfqpoint{3.954569in}{1.502771in}}%
\pgfpathlineto{\pgfqpoint{3.955181in}{1.831899in}}%
\pgfpathlineto{\pgfqpoint{3.955385in}{1.807214in}}%
\pgfpathlineto{\pgfqpoint{3.955588in}{1.881268in}}%
\pgfpathlineto{\pgfqpoint{3.955996in}{1.741388in}}%
\pgfpathlineto{\pgfqpoint{3.956200in}{1.741388in}}%
\pgfpathlineto{\pgfqpoint{3.956404in}{1.724932in}}%
\pgfpathlineto{\pgfqpoint{3.956607in}{1.823670in}}%
\pgfpathlineto{\pgfqpoint{3.957626in}{1.790758in}}%
\pgfpathlineto{\pgfqpoint{3.957830in}{1.733160in}}%
\pgfpathlineto{\pgfqpoint{3.958238in}{1.815442in}}%
\pgfpathlineto{\pgfqpoint{3.958645in}{1.757845in}}%
\pgfpathlineto{\pgfqpoint{3.958849in}{1.782529in}}%
\pgfpathlineto{\pgfqpoint{3.959053in}{1.683791in}}%
\pgfpathlineto{\pgfqpoint{3.959257in}{1.716704in}}%
\pgfpathlineto{\pgfqpoint{3.959461in}{1.675563in}}%
\pgfpathlineto{\pgfqpoint{3.959664in}{1.831899in}}%
\pgfpathlineto{\pgfqpoint{3.960072in}{1.774301in}}%
\pgfpathlineto{\pgfqpoint{3.961295in}{1.840127in}}%
\pgfpathlineto{\pgfqpoint{3.960480in}{1.749617in}}%
\pgfpathlineto{\pgfqpoint{3.961499in}{1.823670in}}%
\pgfpathlineto{\pgfqpoint{3.962314in}{1.766073in}}%
\pgfpathlineto{\pgfqpoint{3.961906in}{1.831899in}}%
\pgfpathlineto{\pgfqpoint{3.962518in}{1.798986in}}%
\pgfpathlineto{\pgfqpoint{3.962925in}{1.889496in}}%
\pgfpathlineto{\pgfqpoint{3.963129in}{1.807214in}}%
\pgfpathlineto{\pgfqpoint{3.963333in}{1.757845in}}%
\pgfpathlineto{\pgfqpoint{3.963944in}{1.864811in}}%
\pgfpathlineto{\pgfqpoint{3.964148in}{1.864811in}}%
\pgfpathlineto{\pgfqpoint{3.965167in}{1.831899in}}%
\pgfpathlineto{\pgfqpoint{3.965371in}{1.873040in}}%
\pgfpathlineto{\pgfqpoint{3.965778in}{1.807214in}}%
\pgfpathlineto{\pgfqpoint{3.967205in}{1.568596in}}%
\pgfpathlineto{\pgfqpoint{3.968632in}{1.749617in}}%
\pgfpathlineto{\pgfqpoint{3.968836in}{1.708476in}}%
\pgfpathlineto{\pgfqpoint{3.970466in}{1.535683in}}%
\pgfpathlineto{\pgfqpoint{3.969447in}{1.733160in}}%
\pgfpathlineto{\pgfqpoint{3.970874in}{1.576824in}}%
\pgfpathlineto{\pgfqpoint{3.971689in}{1.733160in}}%
\pgfpathlineto{\pgfqpoint{3.972504in}{1.675563in}}%
\pgfpathlineto{\pgfqpoint{3.972708in}{1.626194in}}%
\pgfpathlineto{\pgfqpoint{3.973115in}{1.724932in}}%
\pgfpathlineto{\pgfqpoint{3.973319in}{1.700247in}}%
\pgfpathlineto{\pgfqpoint{3.975357in}{1.856583in}}%
\pgfpathlineto{\pgfqpoint{3.976376in}{1.716704in}}%
\pgfpathlineto{\pgfqpoint{3.976580in}{1.733160in}}%
\pgfpathlineto{\pgfqpoint{3.976784in}{1.749617in}}%
\pgfpathlineto{\pgfqpoint{3.977191in}{1.708476in}}%
\pgfpathlineto{\pgfqpoint{3.977395in}{1.716704in}}%
\pgfpathlineto{\pgfqpoint{3.977803in}{1.683791in}}%
\pgfpathlineto{\pgfqpoint{3.978007in}{1.774301in}}%
\pgfpathlineto{\pgfqpoint{3.978618in}{1.700247in}}%
\pgfpathlineto{\pgfqpoint{3.980045in}{1.593281in}}%
\pgfpathlineto{\pgfqpoint{3.981064in}{1.700247in}}%
\pgfpathlineto{\pgfqpoint{3.981268in}{1.650878in}}%
\pgfpathlineto{\pgfqpoint{3.981471in}{1.626194in}}%
\pgfpathlineto{\pgfqpoint{3.981879in}{1.708476in}}%
\pgfpathlineto{\pgfqpoint{3.982898in}{1.642650in}}%
\pgfpathlineto{\pgfqpoint{3.983102in}{1.716704in}}%
\pgfpathlineto{\pgfqpoint{3.983509in}{1.601509in}}%
\pgfpathlineto{\pgfqpoint{3.983917in}{1.675563in}}%
\pgfpathlineto{\pgfqpoint{3.984121in}{1.675563in}}%
\pgfpathlineto{\pgfqpoint{3.984325in}{1.659106in}}%
\pgfpathlineto{\pgfqpoint{3.984528in}{1.683791in}}%
\pgfpathlineto{\pgfqpoint{3.984732in}{1.683791in}}%
\pgfpathlineto{\pgfqpoint{3.985547in}{1.741388in}}%
\pgfpathlineto{\pgfqpoint{3.985955in}{1.716704in}}%
\pgfpathlineto{\pgfqpoint{3.986363in}{1.659106in}}%
\pgfpathlineto{\pgfqpoint{3.986566in}{1.741388in}}%
\pgfpathlineto{\pgfqpoint{3.986974in}{1.683791in}}%
\pgfpathlineto{\pgfqpoint{3.987382in}{1.675563in}}%
\pgfpathlineto{\pgfqpoint{3.987789in}{1.733160in}}%
\pgfpathlineto{\pgfqpoint{3.989012in}{1.626194in}}%
\pgfpathlineto{\pgfqpoint{3.989420in}{1.741388in}}%
\pgfpathlineto{\pgfqpoint{3.990439in}{1.683791in}}%
\pgfpathlineto{\pgfqpoint{3.990846in}{1.733160in}}%
\pgfpathlineto{\pgfqpoint{3.991254in}{1.675563in}}%
\pgfpathlineto{\pgfqpoint{3.991458in}{1.716704in}}%
\pgfpathlineto{\pgfqpoint{3.992680in}{1.601509in}}%
\pgfpathlineto{\pgfqpoint{3.992884in}{1.659106in}}%
\pgfpathlineto{\pgfqpoint{3.993699in}{1.617965in}}%
\pgfpathlineto{\pgfqpoint{3.994107in}{1.675563in}}%
\pgfpathlineto{\pgfqpoint{3.995534in}{1.790758in}}%
\pgfpathlineto{\pgfqpoint{3.995738in}{1.741388in}}%
\pgfpathlineto{\pgfqpoint{3.996349in}{1.774301in}}%
\pgfpathlineto{\pgfqpoint{3.996757in}{1.692019in}}%
\pgfpathlineto{\pgfqpoint{3.996960in}{1.733160in}}%
\pgfpathlineto{\pgfqpoint{3.997572in}{1.667335in}}%
\pgfpathlineto{\pgfqpoint{3.997979in}{1.626194in}}%
\pgfpathlineto{\pgfqpoint{3.998591in}{1.675563in}}%
\pgfpathlineto{\pgfqpoint{3.998795in}{1.683791in}}%
\pgfpathlineto{\pgfqpoint{3.999406in}{1.708476in}}%
\pgfpathlineto{\pgfqpoint{3.999814in}{1.617965in}}%
\pgfpathlineto{\pgfqpoint{4.001036in}{1.692019in}}%
\pgfpathlineto{\pgfqpoint{4.002667in}{1.535683in}}%
\pgfpathlineto{\pgfqpoint{4.003278in}{1.708476in}}%
\pgfpathlineto{\pgfqpoint{4.003890in}{1.642650in}}%
\pgfpathlineto{\pgfqpoint{4.004093in}{1.634422in}}%
\pgfpathlineto{\pgfqpoint{4.004297in}{1.650878in}}%
\pgfpathlineto{\pgfqpoint{4.004501in}{1.692019in}}%
\pgfpathlineto{\pgfqpoint{4.004909in}{1.609737in}}%
\pgfpathlineto{\pgfqpoint{4.005112in}{1.371119in}}%
\pgfpathlineto{\pgfqpoint{4.005928in}{1.576824in}}%
\pgfpathlineto{\pgfqpoint{4.006131in}{1.568596in}}%
\pgfpathlineto{\pgfqpoint{4.006335in}{1.601509in}}%
\pgfpathlineto{\pgfqpoint{4.006539in}{1.601509in}}%
\pgfpathlineto{\pgfqpoint{4.007558in}{1.708476in}}%
\pgfpathlineto{\pgfqpoint{4.007762in}{1.634422in}}%
\pgfpathlineto{\pgfqpoint{4.007966in}{1.420489in}}%
\pgfpathlineto{\pgfqpoint{4.008170in}{1.659106in}}%
\pgfpathlineto{\pgfqpoint{4.008781in}{1.576824in}}%
\pgfpathlineto{\pgfqpoint{4.008985in}{1.552140in}}%
\pgfpathlineto{\pgfqpoint{4.009392in}{1.609737in}}%
\pgfpathlineto{\pgfqpoint{4.009596in}{1.609737in}}%
\pgfpathlineto{\pgfqpoint{4.010411in}{1.683791in}}%
\pgfpathlineto{\pgfqpoint{4.010615in}{1.642650in}}%
\pgfpathlineto{\pgfqpoint{4.011634in}{1.543912in}}%
\pgfpathlineto{\pgfqpoint{4.011023in}{1.683791in}}%
\pgfpathlineto{\pgfqpoint{4.011838in}{1.593281in}}%
\pgfpathlineto{\pgfqpoint{4.012449in}{1.675563in}}%
\pgfpathlineto{\pgfqpoint{4.012653in}{1.560368in}}%
\pgfpathlineto{\pgfqpoint{4.012857in}{1.519227in}}%
\pgfpathlineto{\pgfqpoint{4.013468in}{1.609737in}}%
\pgfpathlineto{\pgfqpoint{4.013672in}{1.609737in}}%
\pgfpathlineto{\pgfqpoint{4.014284in}{1.601509in}}%
\pgfpathlineto{\pgfqpoint{4.015303in}{1.692019in}}%
\pgfpathlineto{\pgfqpoint{4.016322in}{1.527455in}}%
\pgfpathlineto{\pgfqpoint{4.016729in}{1.626194in}}%
\pgfpathlineto{\pgfqpoint{4.018360in}{1.708476in}}%
\pgfpathlineto{\pgfqpoint{4.017544in}{1.617965in}}%
\pgfpathlineto{\pgfqpoint{4.018563in}{1.675563in}}%
\pgfpathlineto{\pgfqpoint{4.018971in}{1.609737in}}%
\pgfpathlineto{\pgfqpoint{4.019379in}{1.708476in}}%
\pgfpathlineto{\pgfqpoint{4.019582in}{1.683791in}}%
\pgfpathlineto{\pgfqpoint{4.020194in}{1.650878in}}%
\pgfpathlineto{\pgfqpoint{4.020805in}{1.774301in}}%
\pgfpathlineto{\pgfqpoint{4.021213in}{1.667335in}}%
\pgfpathlineto{\pgfqpoint{4.021417in}{1.659106in}}%
\pgfpathlineto{\pgfqpoint{4.021621in}{1.667335in}}%
\pgfpathlineto{\pgfqpoint{4.021824in}{1.692019in}}%
\pgfpathlineto{\pgfqpoint{4.022232in}{1.642650in}}%
\pgfpathlineto{\pgfqpoint{4.022640in}{1.675563in}}%
\pgfpathlineto{\pgfqpoint{4.023047in}{1.609737in}}%
\pgfpathlineto{\pgfqpoint{4.023455in}{1.626194in}}%
\pgfpathlineto{\pgfqpoint{4.023659in}{1.412260in}}%
\pgfpathlineto{\pgfqpoint{4.024474in}{1.667335in}}%
\pgfpathlineto{\pgfqpoint{4.025289in}{1.650878in}}%
\pgfpathlineto{\pgfqpoint{4.025493in}{1.724932in}}%
\pgfpathlineto{\pgfqpoint{4.025697in}{1.478086in}}%
\pgfpathlineto{\pgfqpoint{4.026512in}{1.667335in}}%
\pgfpathlineto{\pgfqpoint{4.026716in}{1.650878in}}%
\pgfpathlineto{\pgfqpoint{4.027123in}{1.700247in}}%
\pgfpathlineto{\pgfqpoint{4.027327in}{1.708476in}}%
\pgfpathlineto{\pgfqpoint{4.027531in}{1.700247in}}%
\pgfpathlineto{\pgfqpoint{4.027735in}{1.461630in}}%
\pgfpathlineto{\pgfqpoint{4.028142in}{1.716704in}}%
\pgfpathlineto{\pgfqpoint{4.028550in}{1.667335in}}%
\pgfpathlineto{\pgfqpoint{4.028754in}{1.683791in}}%
\pgfpathlineto{\pgfqpoint{4.028957in}{1.617965in}}%
\pgfpathlineto{\pgfqpoint{4.029569in}{1.667335in}}%
\pgfpathlineto{\pgfqpoint{4.029773in}{1.593281in}}%
\pgfpathlineto{\pgfqpoint{4.030792in}{1.617965in}}%
\pgfpathlineto{\pgfqpoint{4.030995in}{1.634422in}}%
\pgfpathlineto{\pgfqpoint{4.031199in}{1.617965in}}%
\pgfpathlineto{\pgfqpoint{4.031403in}{1.560368in}}%
\pgfpathlineto{\pgfqpoint{4.032218in}{1.568596in}}%
\pgfpathlineto{\pgfqpoint{4.033645in}{1.650878in}}%
\pgfpathlineto{\pgfqpoint{4.034460in}{1.601509in}}%
\pgfpathlineto{\pgfqpoint{4.034868in}{1.708476in}}%
\pgfpathlineto{\pgfqpoint{4.035479in}{1.601509in}}%
\pgfpathlineto{\pgfqpoint{4.035683in}{1.601509in}}%
\pgfpathlineto{\pgfqpoint{4.036906in}{1.478086in}}%
\pgfpathlineto{\pgfqpoint{4.037110in}{1.519227in}}%
\pgfpathlineto{\pgfqpoint{4.037313in}{1.510999in}}%
\pgfpathlineto{\pgfqpoint{4.037925in}{1.617965in}}%
\pgfpathlineto{\pgfqpoint{4.038536in}{1.576824in}}%
\pgfpathlineto{\pgfqpoint{4.039555in}{1.469858in}}%
\pgfpathlineto{\pgfqpoint{4.039759in}{1.535683in}}%
\pgfpathlineto{\pgfqpoint{4.041389in}{1.371119in}}%
\pgfpathlineto{\pgfqpoint{4.041797in}{1.420489in}}%
\pgfpathlineto{\pgfqpoint{4.043020in}{1.519227in}}%
\pgfpathlineto{\pgfqpoint{4.043224in}{1.510999in}}%
\pgfpathlineto{\pgfqpoint{4.043427in}{1.461630in}}%
\pgfpathlineto{\pgfqpoint{4.043631in}{1.527455in}}%
\pgfpathlineto{\pgfqpoint{4.044243in}{1.527455in}}%
\pgfpathlineto{\pgfqpoint{4.044854in}{1.568596in}}%
\pgfpathlineto{\pgfqpoint{4.046077in}{1.420489in}}%
\pgfpathlineto{\pgfqpoint{4.046281in}{1.420489in}}%
\pgfpathlineto{\pgfqpoint{4.047707in}{1.552140in}}%
\pgfpathlineto{\pgfqpoint{4.048523in}{1.469858in}}%
\pgfpathlineto{\pgfqpoint{4.048930in}{1.478086in}}%
\pgfpathlineto{\pgfqpoint{4.049542in}{1.543912in}}%
\pgfpathlineto{\pgfqpoint{4.049949in}{1.486314in}}%
\pgfpathlineto{\pgfqpoint{4.050153in}{1.461630in}}%
\pgfpathlineto{\pgfqpoint{4.050561in}{1.535683in}}%
\pgfpathlineto{\pgfqpoint{4.052191in}{1.585053in}}%
\pgfpathlineto{\pgfqpoint{4.052599in}{1.560368in}}%
\pgfpathlineto{\pgfqpoint{4.053210in}{1.609737in}}%
\pgfpathlineto{\pgfqpoint{4.054229in}{1.527455in}}%
\pgfpathlineto{\pgfqpoint{4.053618in}{1.642650in}}%
\pgfpathlineto{\pgfqpoint{4.054637in}{1.543912in}}%
\pgfpathlineto{\pgfqpoint{4.056063in}{1.667335in}}%
\pgfpathlineto{\pgfqpoint{4.056675in}{1.659106in}}%
\pgfpathlineto{\pgfqpoint{4.056878in}{1.568596in}}%
\pgfpathlineto{\pgfqpoint{4.057694in}{1.626194in}}%
\pgfpathlineto{\pgfqpoint{4.057897in}{1.609737in}}%
\pgfpathlineto{\pgfqpoint{4.058101in}{1.650878in}}%
\pgfpathlineto{\pgfqpoint{4.058305in}{1.634422in}}%
\pgfpathlineto{\pgfqpoint{4.059324in}{1.749617in}}%
\pgfpathlineto{\pgfqpoint{4.059528in}{1.683791in}}%
\pgfpathlineto{\pgfqpoint{4.060954in}{1.552140in}}%
\pgfpathlineto{\pgfqpoint{4.061158in}{1.568596in}}%
\pgfpathlineto{\pgfqpoint{4.061362in}{1.502771in}}%
\pgfpathlineto{\pgfqpoint{4.061566in}{1.502771in}}%
\pgfpathlineto{\pgfqpoint{4.063808in}{1.617965in}}%
\pgfpathlineto{\pgfqpoint{4.064012in}{1.601509in}}%
\pgfpathlineto{\pgfqpoint{4.064215in}{1.659106in}}%
\pgfpathlineto{\pgfqpoint{4.064419in}{1.856583in}}%
\pgfpathlineto{\pgfqpoint{4.064623in}{1.585053in}}%
\pgfpathlineto{\pgfqpoint{4.065234in}{1.650878in}}%
\pgfpathlineto{\pgfqpoint{4.065438in}{1.453401in}}%
\pgfpathlineto{\pgfqpoint{4.065642in}{1.749617in}}%
\pgfpathlineto{\pgfqpoint{4.066253in}{1.683791in}}%
\pgfpathlineto{\pgfqpoint{4.067680in}{1.593281in}}%
\pgfpathlineto{\pgfqpoint{4.068903in}{1.708476in}}%
\pgfpathlineto{\pgfqpoint{4.069310in}{1.634422in}}%
\pgfpathlineto{\pgfqpoint{4.069514in}{1.667335in}}%
\pgfpathlineto{\pgfqpoint{4.069922in}{1.354663in}}%
\pgfpathlineto{\pgfqpoint{4.070533in}{1.568596in}}%
\pgfpathlineto{\pgfqpoint{4.072164in}{1.724932in}}%
\pgfpathlineto{\pgfqpoint{4.073183in}{1.527455in}}%
\pgfpathlineto{\pgfqpoint{4.073386in}{1.568596in}}%
\pgfpathlineto{\pgfqpoint{4.074813in}{1.436945in}}%
\pgfpathlineto{\pgfqpoint{4.075628in}{1.585053in}}%
\pgfpathlineto{\pgfqpoint{4.075832in}{1.453401in}}%
\pgfpathlineto{\pgfqpoint{4.076444in}{1.214784in}}%
\pgfpathlineto{\pgfqpoint{4.077259in}{1.280609in}}%
\pgfpathlineto{\pgfqpoint{4.077463in}{1.288837in}}%
\pgfpathlineto{\pgfqpoint{4.078278in}{1.017307in}}%
\pgfpathlineto{\pgfqpoint{4.078482in}{1.346435in}}%
\pgfpathlineto{\pgfqpoint{4.079501in}{1.247696in}}%
\pgfpathlineto{\pgfqpoint{4.080723in}{0.951481in}}%
\pgfpathlineto{\pgfqpoint{4.080927in}{1.066676in}}%
\pgfpathlineto{\pgfqpoint{4.082354in}{1.214784in}}%
\pgfpathlineto{\pgfqpoint{4.083780in}{0.819830in}}%
\pgfpathlineto{\pgfqpoint{4.084188in}{0.951481in}}%
\pgfpathlineto{\pgfqpoint{4.084596in}{0.935025in}}%
\pgfpathlineto{\pgfqpoint{4.085818in}{1.066676in}}%
\pgfpathlineto{\pgfqpoint{4.086022in}{1.066676in}}%
\pgfpathlineto{\pgfqpoint{4.086226in}{1.058448in}}%
\pgfpathlineto{\pgfqpoint{4.086430in}{1.198327in}}%
\pgfpathlineto{\pgfqpoint{4.086837in}{1.000850in}}%
\pgfpathlineto{\pgfqpoint{4.087041in}{1.041991in}}%
\pgfpathlineto{\pgfqpoint{4.087653in}{0.762232in}}%
\pgfpathlineto{\pgfqpoint{4.088060in}{1.025535in}}%
\pgfpathlineto{\pgfqpoint{4.088264in}{1.058448in}}%
\pgfpathlineto{\pgfqpoint{4.088468in}{1.000850in}}%
\pgfpathlineto{\pgfqpoint{4.088672in}{1.025535in}}%
\pgfpathlineto{\pgfqpoint{4.089691in}{0.786917in}}%
\pgfpathlineto{\pgfqpoint{4.090302in}{0.828058in}}%
\pgfpathlineto{\pgfqpoint{4.091729in}{1.157186in}}%
\pgfpathlineto{\pgfqpoint{4.092340in}{1.000850in}}%
\pgfpathlineto{\pgfqpoint{4.092544in}{0.967938in}}%
\pgfpathlineto{\pgfqpoint{4.093155in}{1.041991in}}%
\pgfpathlineto{\pgfqpoint{4.093971in}{1.181871in}}%
\pgfpathlineto{\pgfqpoint{4.094378in}{1.140730in}}%
\pgfpathlineto{\pgfqpoint{4.095193in}{0.803374in}}%
\pgfpathlineto{\pgfqpoint{4.095601in}{0.885656in}}%
\pgfpathlineto{\pgfqpoint{4.097028in}{1.214784in}}%
\pgfpathlineto{\pgfqpoint{4.097843in}{1.091361in}}%
\pgfpathlineto{\pgfqpoint{4.098250in}{1.157186in}}%
\pgfpathlineto{\pgfqpoint{4.098454in}{1.231240in}}%
\pgfpathlineto{\pgfqpoint{4.099066in}{1.124273in}}%
\pgfpathlineto{\pgfqpoint{4.099269in}{1.157186in}}%
\pgfpathlineto{\pgfqpoint{4.099881in}{1.083132in}}%
\pgfpathlineto{\pgfqpoint{4.100492in}{1.132502in}}%
\pgfpathlineto{\pgfqpoint{4.100696in}{1.165414in}}%
\pgfpathlineto{\pgfqpoint{4.100900in}{1.124273in}}%
\pgfpathlineto{\pgfqpoint{4.101307in}{1.041991in}}%
\pgfpathlineto{\pgfqpoint{4.102123in}{1.066676in}}%
\pgfpathlineto{\pgfqpoint{4.102734in}{1.116045in}}%
\pgfpathlineto{\pgfqpoint{4.102938in}{1.058448in}}%
\pgfpathlineto{\pgfqpoint{4.103346in}{1.025535in}}%
\pgfpathlineto{\pgfqpoint{4.103549in}{1.083132in}}%
\pgfpathlineto{\pgfqpoint{4.103753in}{1.083132in}}%
\pgfpathlineto{\pgfqpoint{4.104161in}{1.124273in}}%
\pgfpathlineto{\pgfqpoint{4.104568in}{1.116045in}}%
\pgfpathlineto{\pgfqpoint{4.104772in}{1.074904in}}%
\pgfpathlineto{\pgfqpoint{4.104976in}{1.173643in}}%
\pgfpathlineto{\pgfqpoint{4.105384in}{1.297066in}}%
\pgfpathlineto{\pgfqpoint{4.105587in}{1.198327in}}%
\pgfpathlineto{\pgfqpoint{4.105791in}{1.099589in}}%
\pgfpathlineto{\pgfqpoint{4.106403in}{1.313522in}}%
\pgfpathlineto{\pgfqpoint{4.107625in}{1.206555in}}%
\pgfpathlineto{\pgfqpoint{4.108237in}{1.362891in}}%
\pgfpathlineto{\pgfqpoint{4.108848in}{1.264153in}}%
\pgfpathlineto{\pgfqpoint{4.109663in}{1.354663in}}%
\pgfpathlineto{\pgfqpoint{4.109867in}{1.297066in}}%
\pgfpathlineto{\pgfqpoint{4.110071in}{1.206555in}}%
\pgfpathlineto{\pgfqpoint{4.110479in}{1.404032in}}%
\pgfpathlineto{\pgfqpoint{4.110682in}{1.428717in}}%
\pgfpathlineto{\pgfqpoint{4.110886in}{1.362891in}}%
\pgfpathlineto{\pgfqpoint{4.111498in}{1.305294in}}%
\pgfpathlineto{\pgfqpoint{4.111294in}{1.379348in}}%
\pgfpathlineto{\pgfqpoint{4.111701in}{1.346435in}}%
\pgfpathlineto{\pgfqpoint{4.111905in}{1.387576in}}%
\pgfpathlineto{\pgfqpoint{4.112109in}{1.297066in}}%
\pgfpathlineto{\pgfqpoint{4.112924in}{1.379348in}}%
\pgfpathlineto{\pgfqpoint{4.113128in}{1.379348in}}%
\pgfpathlineto{\pgfqpoint{4.114351in}{1.510999in}}%
\pgfpathlineto{\pgfqpoint{4.113536in}{1.362891in}}%
\pgfpathlineto{\pgfqpoint{4.114555in}{1.486314in}}%
\pgfpathlineto{\pgfqpoint{4.115370in}{1.502771in}}%
\pgfpathlineto{\pgfqpoint{4.115778in}{1.445173in}}%
\pgfpathlineto{\pgfqpoint{4.115981in}{1.510999in}}%
\pgfpathlineto{\pgfqpoint{4.116389in}{1.387576in}}%
\pgfpathlineto{\pgfqpoint{4.117000in}{1.486314in}}%
\pgfpathlineto{\pgfqpoint{4.118427in}{1.198327in}}%
\pgfpathlineto{\pgfqpoint{4.119650in}{1.519227in}}%
\pgfpathlineto{\pgfqpoint{4.120669in}{1.338207in}}%
\pgfpathlineto{\pgfqpoint{4.120873in}{1.436945in}}%
\pgfpathlineto{\pgfqpoint{4.121484in}{1.494542in}}%
\pgfpathlineto{\pgfqpoint{4.122095in}{1.478086in}}%
\pgfpathlineto{\pgfqpoint{4.122503in}{1.502771in}}%
\pgfpathlineto{\pgfqpoint{4.123318in}{1.404032in}}%
\pgfpathlineto{\pgfqpoint{4.124745in}{1.560368in}}%
\pgfpathlineto{\pgfqpoint{4.124949in}{1.552140in}}%
\pgfpathlineto{\pgfqpoint{4.125560in}{1.609737in}}%
\pgfpathlineto{\pgfqpoint{4.125764in}{1.568596in}}%
\pgfpathlineto{\pgfqpoint{4.127598in}{1.395804in}}%
\pgfpathlineto{\pgfqpoint{4.128617in}{1.535683in}}%
\pgfpathlineto{\pgfqpoint{4.129025in}{1.527455in}}%
\pgfpathlineto{\pgfqpoint{4.129840in}{1.494542in}}%
\pgfpathlineto{\pgfqpoint{4.131267in}{1.634422in}}%
\pgfpathlineto{\pgfqpoint{4.132286in}{1.568596in}}%
\pgfpathlineto{\pgfqpoint{4.132489in}{1.601509in}}%
\pgfpathlineto{\pgfqpoint{4.132897in}{1.601509in}}%
\pgfpathlineto{\pgfqpoint{4.133508in}{1.683791in}}%
\pgfpathlineto{\pgfqpoint{4.134324in}{1.659106in}}%
\pgfpathlineto{\pgfqpoint{4.134527in}{1.642650in}}%
\pgfpathlineto{\pgfqpoint{4.134731in}{1.659106in}}%
\pgfpathlineto{\pgfqpoint{4.135139in}{1.716704in}}%
\pgfpathlineto{\pgfqpoint{4.135750in}{1.700247in}}%
\pgfpathlineto{\pgfqpoint{4.135954in}{1.650878in}}%
\pgfpathlineto{\pgfqpoint{4.136565in}{1.766073in}}%
\pgfpathlineto{\pgfqpoint{4.136769in}{1.724932in}}%
\pgfpathlineto{\pgfqpoint{4.136973in}{1.774301in}}%
\pgfpathlineto{\pgfqpoint{4.137788in}{1.741388in}}%
\pgfpathlineto{\pgfqpoint{4.138807in}{1.823670in}}%
\pgfpathlineto{\pgfqpoint{4.139011in}{1.807214in}}%
\pgfpathlineto{\pgfqpoint{4.139215in}{1.766073in}}%
\pgfpathlineto{\pgfqpoint{4.139622in}{1.881268in}}%
\pgfpathlineto{\pgfqpoint{4.139826in}{1.831899in}}%
\pgfpathlineto{\pgfqpoint{4.140030in}{1.831899in}}%
\pgfpathlineto{\pgfqpoint{4.141457in}{1.947093in}}%
\pgfpathlineto{\pgfqpoint{4.141864in}{1.914181in}}%
\pgfpathlineto{\pgfqpoint{4.142068in}{1.889496in}}%
\pgfpathlineto{\pgfqpoint{4.142476in}{1.971778in}}%
\pgfpathlineto{\pgfqpoint{4.142680in}{1.963550in}}%
\pgfpathlineto{\pgfqpoint{4.143902in}{2.021147in}}%
\pgfpathlineto{\pgfqpoint{4.144514in}{2.012919in}}%
\pgfpathlineto{\pgfqpoint{4.144718in}{2.037604in}}%
\pgfpathlineto{\pgfqpoint{4.144921in}{1.848355in}}%
\pgfpathlineto{\pgfqpoint{4.145737in}{2.095201in}}%
\pgfpathlineto{\pgfqpoint{4.145940in}{2.070516in}}%
\pgfpathlineto{\pgfqpoint{4.146144in}{2.128114in}}%
\pgfpathlineto{\pgfqpoint{4.146756in}{2.103429in}}%
\pgfpathlineto{\pgfqpoint{4.146959in}{2.086973in}}%
\pgfpathlineto{\pgfqpoint{4.147163in}{1.971778in}}%
\pgfpathlineto{\pgfqpoint{4.147978in}{2.062288in}}%
\pgfpathlineto{\pgfqpoint{4.148590in}{1.889496in}}%
\pgfpathlineto{\pgfqpoint{4.149201in}{1.905952in}}%
\pgfpathlineto{\pgfqpoint{4.150016in}{2.128114in}}%
\pgfpathlineto{\pgfqpoint{4.150424in}{2.111657in}}%
\pgfpathlineto{\pgfqpoint{4.150832in}{2.152798in}}%
\pgfpathlineto{\pgfqpoint{4.151035in}{1.840127in}}%
\pgfpathlineto{\pgfqpoint{4.151851in}{2.169255in}}%
\pgfpathlineto{\pgfqpoint{4.152054in}{2.202168in}}%
\pgfpathlineto{\pgfqpoint{4.152666in}{2.136342in}}%
\pgfpathlineto{\pgfqpoint{4.152870in}{2.119886in}}%
\pgfpathlineto{\pgfqpoint{4.153073in}{2.152798in}}%
\pgfpathlineto{\pgfqpoint{4.153277in}{2.136342in}}%
\pgfpathlineto{\pgfqpoint{4.153481in}{2.193939in}}%
\pgfpathlineto{\pgfqpoint{4.154296in}{2.169255in}}%
\pgfpathlineto{\pgfqpoint{4.155723in}{2.045832in}}%
\pgfpathlineto{\pgfqpoint{4.156742in}{2.037604in}}%
\pgfpathlineto{\pgfqpoint{4.156946in}{2.144570in}}%
\pgfpathlineto{\pgfqpoint{4.157353in}{2.103429in}}%
\pgfpathlineto{\pgfqpoint{4.157761in}{2.177483in}}%
\pgfpathlineto{\pgfqpoint{4.157965in}{2.185711in}}%
\pgfpathlineto{\pgfqpoint{4.158780in}{2.210396in}}%
\pgfpathlineto{\pgfqpoint{4.159188in}{2.004691in}}%
\pgfpathlineto{\pgfqpoint{4.159391in}{2.021147in}}%
\pgfpathlineto{\pgfqpoint{4.159595in}{2.004691in}}%
\pgfpathlineto{\pgfqpoint{4.159799in}{1.963550in}}%
\pgfpathlineto{\pgfqpoint{4.160207in}{2.226852in}}%
\pgfpathlineto{\pgfqpoint{4.161022in}{2.136342in}}%
\pgfpathlineto{\pgfqpoint{4.162041in}{2.226852in}}%
\pgfpathlineto{\pgfqpoint{4.162448in}{2.202168in}}%
\pgfpathlineto{\pgfqpoint{4.163264in}{2.062288in}}%
\pgfpathlineto{\pgfqpoint{4.163671in}{2.119886in}}%
\pgfpathlineto{\pgfqpoint{4.164283in}{2.193939in}}%
\pgfpathlineto{\pgfqpoint{4.164690in}{2.103429in}}%
\pgfpathlineto{\pgfqpoint{4.164894in}{2.078745in}}%
\pgfpathlineto{\pgfqpoint{4.165098in}{2.136342in}}%
\pgfpathlineto{\pgfqpoint{4.165302in}{2.136342in}}%
\pgfpathlineto{\pgfqpoint{4.165505in}{2.235080in}}%
\pgfpathlineto{\pgfqpoint{4.166117in}{2.095201in}}%
\pgfpathlineto{\pgfqpoint{4.166524in}{2.202168in}}%
\pgfpathlineto{\pgfqpoint{4.166932in}{2.161027in}}%
\pgfpathlineto{\pgfqpoint{4.167747in}{2.226852in}}%
\pgfpathlineto{\pgfqpoint{4.168155in}{2.169255in}}%
\pgfpathlineto{\pgfqpoint{4.168562in}{2.226852in}}%
\pgfpathlineto{\pgfqpoint{4.168766in}{2.292678in}}%
\pgfpathlineto{\pgfqpoint{4.169582in}{2.202168in}}%
\pgfpathlineto{\pgfqpoint{4.169785in}{2.284450in}}%
\pgfpathlineto{\pgfqpoint{4.171212in}{2.202168in}}%
\pgfpathlineto{\pgfqpoint{4.172231in}{2.309134in}}%
\pgfpathlineto{\pgfqpoint{4.173250in}{2.235080in}}%
\pgfpathlineto{\pgfqpoint{4.174065in}{2.317363in}}%
\pgfpathlineto{\pgfqpoint{4.174473in}{2.300906in}}%
\pgfpathlineto{\pgfqpoint{4.174677in}{2.309134in}}%
\pgfpathlineto{\pgfqpoint{4.175084in}{2.333819in}}%
\pgfpathlineto{\pgfqpoint{4.176307in}{2.144570in}}%
\pgfpathlineto{\pgfqpoint{4.178345in}{2.259765in}}%
\pgfpathlineto{\pgfqpoint{4.178753in}{2.029375in}}%
\pgfpathlineto{\pgfqpoint{4.179160in}{2.350275in}}%
\pgfpathlineto{\pgfqpoint{4.179364in}{2.259765in}}%
\pgfpathlineto{\pgfqpoint{4.179568in}{2.251537in}}%
\pgfpathlineto{\pgfqpoint{4.180383in}{2.317363in}}%
\pgfpathlineto{\pgfqpoint{4.180587in}{2.292678in}}%
\pgfpathlineto{\pgfqpoint{4.181198in}{2.251537in}}%
\pgfpathlineto{\pgfqpoint{4.181810in}{2.267993in}}%
\pgfpathlineto{\pgfqpoint{4.182013in}{2.103429in}}%
\pgfpathlineto{\pgfqpoint{4.182829in}{2.317363in}}%
\pgfpathlineto{\pgfqpoint{4.183644in}{2.350275in}}%
\pgfpathlineto{\pgfqpoint{4.183440in}{2.292678in}}%
\pgfpathlineto{\pgfqpoint{4.183848in}{2.325591in}}%
\pgfpathlineto{\pgfqpoint{4.184867in}{2.070516in}}%
\pgfpathlineto{\pgfqpoint{4.184459in}{2.350275in}}%
\pgfpathlineto{\pgfqpoint{4.185274in}{2.284450in}}%
\pgfpathlineto{\pgfqpoint{4.185478in}{2.292678in}}%
\pgfpathlineto{\pgfqpoint{4.186090in}{2.226852in}}%
\pgfpathlineto{\pgfqpoint{4.186293in}{2.251537in}}%
\pgfpathlineto{\pgfqpoint{4.186497in}{2.317363in}}%
\pgfpathlineto{\pgfqpoint{4.187312in}{2.243309in}}%
\pgfpathlineto{\pgfqpoint{4.188128in}{2.226852in}}%
\pgfpathlineto{\pgfqpoint{4.189554in}{2.350275in}}%
\pgfpathlineto{\pgfqpoint{4.189758in}{2.358504in}}%
\pgfpathlineto{\pgfqpoint{4.190573in}{2.226852in}}%
\pgfpathlineto{\pgfqpoint{4.190981in}{2.292678in}}%
\pgfpathlineto{\pgfqpoint{4.192000in}{2.342047in}}%
\pgfpathlineto{\pgfqpoint{4.192204in}{2.276221in}}%
\pgfpathlineto{\pgfqpoint{4.192815in}{2.366732in}}%
\pgfpathlineto{\pgfqpoint{4.193019in}{2.342047in}}%
\pgfpathlineto{\pgfqpoint{4.193223in}{2.374960in}}%
\pgfpathlineto{\pgfqpoint{4.193630in}{2.317363in}}%
\pgfpathlineto{\pgfqpoint{4.193834in}{2.350275in}}%
\pgfpathlineto{\pgfqpoint{4.194649in}{2.111657in}}%
\pgfpathlineto{\pgfqpoint{4.194853in}{2.300906in}}%
\pgfpathlineto{\pgfqpoint{4.195872in}{2.432557in}}%
\pgfpathlineto{\pgfqpoint{4.196076in}{2.383188in}}%
\pgfpathlineto{\pgfqpoint{4.196687in}{2.440786in}}%
\pgfpathlineto{\pgfqpoint{4.197095in}{2.416101in}}%
\pgfpathlineto{\pgfqpoint{4.198725in}{2.070516in}}%
\pgfpathlineto{\pgfqpoint{4.199948in}{2.333819in}}%
\pgfpathlineto{\pgfqpoint{4.200967in}{2.366732in}}%
\pgfpathlineto{\pgfqpoint{4.201171in}{2.358504in}}%
\pgfpathlineto{\pgfqpoint{4.201986in}{2.267993in}}%
\pgfpathlineto{\pgfqpoint{4.202190in}{2.366732in}}%
\pgfpathlineto{\pgfqpoint{4.202598in}{2.309134in}}%
\pgfpathlineto{\pgfqpoint{4.203005in}{2.366732in}}%
\pgfpathlineto{\pgfqpoint{4.203209in}{2.391416in}}%
\pgfpathlineto{\pgfqpoint{4.203413in}{2.300906in}}%
\pgfpathlineto{\pgfqpoint{4.203617in}{2.300906in}}%
\pgfpathlineto{\pgfqpoint{4.204024in}{2.251537in}}%
\pgfpathlineto{\pgfqpoint{4.204228in}{2.317363in}}%
\pgfpathlineto{\pgfqpoint{4.204636in}{2.399645in}}%
\pgfpathlineto{\pgfqpoint{4.205247in}{2.374960in}}%
\pgfpathlineto{\pgfqpoint{4.206877in}{2.152798in}}%
\pgfpathlineto{\pgfqpoint{4.207285in}{2.333819in}}%
\pgfpathlineto{\pgfqpoint{4.208100in}{2.276221in}}%
\pgfpathlineto{\pgfqpoint{4.208304in}{2.267993in}}%
\pgfpathlineto{\pgfqpoint{4.209323in}{2.383188in}}%
\pgfpathlineto{\pgfqpoint{4.208712in}{2.259765in}}%
\pgfpathlineto{\pgfqpoint{4.209527in}{2.358504in}}%
\pgfpathlineto{\pgfqpoint{4.210750in}{2.284450in}}%
\pgfpathlineto{\pgfqpoint{4.211973in}{2.416101in}}%
\pgfpathlineto{\pgfqpoint{4.212992in}{2.317363in}}%
\pgfpathlineto{\pgfqpoint{4.213195in}{2.358504in}}%
\pgfpathlineto{\pgfqpoint{4.214011in}{2.325591in}}%
\pgfpathlineto{\pgfqpoint{4.213603in}{2.383188in}}%
\pgfpathlineto{\pgfqpoint{4.214214in}{2.358504in}}%
\pgfpathlineto{\pgfqpoint{4.214622in}{2.374960in}}%
\pgfpathlineto{\pgfqpoint{4.214826in}{2.342047in}}%
\pgfpathlineto{\pgfqpoint{4.215030in}{2.358504in}}%
\pgfpathlineto{\pgfqpoint{4.216049in}{2.309134in}}%
\pgfpathlineto{\pgfqpoint{4.216252in}{2.399645in}}%
\pgfpathlineto{\pgfqpoint{4.217068in}{2.374960in}}%
\pgfpathlineto{\pgfqpoint{4.217271in}{2.333819in}}%
\pgfpathlineto{\pgfqpoint{4.217679in}{2.391416in}}%
\pgfpathlineto{\pgfqpoint{4.218087in}{2.391416in}}%
\pgfpathlineto{\pgfqpoint{4.218290in}{2.407873in}}%
\pgfpathlineto{\pgfqpoint{4.218494in}{2.383188in}}%
\pgfpathlineto{\pgfqpoint{4.218902in}{2.383188in}}%
\pgfpathlineto{\pgfqpoint{4.220532in}{2.267993in}}%
\pgfpathlineto{\pgfqpoint{4.222366in}{2.366732in}}%
\pgfpathlineto{\pgfqpoint{4.221144in}{2.251537in}}%
\pgfpathlineto{\pgfqpoint{4.222570in}{2.350275in}}%
\pgfpathlineto{\pgfqpoint{4.222774in}{2.210396in}}%
\pgfpathlineto{\pgfqpoint{4.223386in}{2.383188in}}%
\pgfpathlineto{\pgfqpoint{4.223589in}{2.350275in}}%
\pgfpathlineto{\pgfqpoint{4.224201in}{2.317363in}}%
\pgfpathlineto{\pgfqpoint{4.225016in}{2.416101in}}%
\pgfpathlineto{\pgfqpoint{4.225627in}{2.333819in}}%
\pgfpathlineto{\pgfqpoint{4.225831in}{2.424329in}}%
\pgfpathlineto{\pgfqpoint{4.226035in}{2.498383in}}%
\pgfpathlineto{\pgfqpoint{4.226646in}{2.350275in}}%
\pgfpathlineto{\pgfqpoint{4.226850in}{2.416101in}}%
\pgfpathlineto{\pgfqpoint{4.227462in}{2.424329in}}%
\pgfpathlineto{\pgfqpoint{4.228277in}{2.243309in}}%
\pgfpathlineto{\pgfqpoint{4.229092in}{2.333819in}}%
\pgfpathlineto{\pgfqpoint{4.229703in}{2.358504in}}%
\pgfpathlineto{\pgfqpoint{4.230315in}{2.292678in}}%
\pgfpathlineto{\pgfqpoint{4.230519in}{2.342047in}}%
\pgfpathlineto{\pgfqpoint{4.231334in}{2.333819in}}%
\pgfpathlineto{\pgfqpoint{4.232557in}{2.284450in}}%
\pgfpathlineto{\pgfqpoint{4.232964in}{2.350275in}}%
\pgfpathlineto{\pgfqpoint{4.233372in}{2.300906in}}%
\pgfpathlineto{\pgfqpoint{4.233576in}{2.259765in}}%
\pgfpathlineto{\pgfqpoint{4.234187in}{2.309134in}}%
\pgfpathlineto{\pgfqpoint{4.234391in}{2.309134in}}%
\pgfpathlineto{\pgfqpoint{4.236225in}{2.136342in}}%
\pgfpathlineto{\pgfqpoint{4.236429in}{2.185711in}}%
\pgfpathlineto{\pgfqpoint{4.237244in}{2.300906in}}%
\pgfpathlineto{\pgfqpoint{4.237856in}{2.267993in}}%
\pgfpathlineto{\pgfqpoint{4.239486in}{2.177483in}}%
\pgfpathlineto{\pgfqpoint{4.239894in}{2.128114in}}%
\pgfpathlineto{\pgfqpoint{4.240301in}{2.226852in}}%
\pgfpathlineto{\pgfqpoint{4.240913in}{2.284450in}}%
\pgfpathlineto{\pgfqpoint{4.241320in}{2.070516in}}%
\pgfpathlineto{\pgfqpoint{4.241728in}{2.325591in}}%
\pgfpathlineto{\pgfqpoint{4.241932in}{2.309134in}}%
\pgfpathlineto{\pgfqpoint{4.243154in}{2.424329in}}%
\pgfpathlineto{\pgfqpoint{4.243358in}{2.391416in}}%
\pgfpathlineto{\pgfqpoint{4.243970in}{2.317363in}}%
\pgfpathlineto{\pgfqpoint{4.243766in}{2.424329in}}%
\pgfpathlineto{\pgfqpoint{4.244581in}{2.342047in}}%
\pgfpathlineto{\pgfqpoint{4.245192in}{2.374960in}}%
\pgfpathlineto{\pgfqpoint{4.245804in}{2.070516in}}%
\pgfpathlineto{\pgfqpoint{4.246211in}{2.267993in}}%
\pgfpathlineto{\pgfqpoint{4.247434in}{2.399645in}}%
\pgfpathlineto{\pgfqpoint{4.246619in}{2.251537in}}%
\pgfpathlineto{\pgfqpoint{4.247638in}{2.342047in}}%
\pgfpathlineto{\pgfqpoint{4.249472in}{2.276221in}}%
\pgfpathlineto{\pgfqpoint{4.250084in}{2.325591in}}%
\pgfpathlineto{\pgfqpoint{4.250695in}{2.317363in}}%
\pgfpathlineto{\pgfqpoint{4.251510in}{2.300906in}}%
\pgfpathlineto{\pgfqpoint{4.251714in}{2.374960in}}%
\pgfpathlineto{\pgfqpoint{4.252937in}{2.095201in}}%
\pgfpathlineto{\pgfqpoint{4.253141in}{2.284450in}}%
\pgfpathlineto{\pgfqpoint{4.253345in}{2.350275in}}%
\pgfpathlineto{\pgfqpoint{4.254160in}{2.259765in}}%
\pgfpathlineto{\pgfqpoint{4.254364in}{2.284450in}}%
\pgfpathlineto{\pgfqpoint{4.254567in}{2.276221in}}%
\pgfpathlineto{\pgfqpoint{4.254771in}{2.078745in}}%
\pgfpathlineto{\pgfqpoint{4.255179in}{2.374960in}}%
\pgfpathlineto{\pgfqpoint{4.255586in}{2.144570in}}%
\pgfpathlineto{\pgfqpoint{4.256605in}{2.374960in}}%
\pgfpathlineto{\pgfqpoint{4.256809in}{2.342047in}}%
\pgfpathlineto{\pgfqpoint{4.257013in}{2.333819in}}%
\pgfpathlineto{\pgfqpoint{4.257217in}{2.342047in}}%
\pgfpathlineto{\pgfqpoint{4.257828in}{2.383188in}}%
\pgfpathlineto{\pgfqpoint{4.259051in}{2.284450in}}%
\pgfpathlineto{\pgfqpoint{4.259255in}{2.292678in}}%
\pgfpathlineto{\pgfqpoint{4.259459in}{2.276221in}}%
\pgfpathlineto{\pgfqpoint{4.259866in}{2.152798in}}%
\pgfpathlineto{\pgfqpoint{4.260681in}{2.251537in}}%
\pgfpathlineto{\pgfqpoint{4.261089in}{2.235080in}}%
\pgfpathlineto{\pgfqpoint{4.261293in}{2.251537in}}%
\pgfpathlineto{\pgfqpoint{4.261497in}{2.292678in}}%
\pgfpathlineto{\pgfqpoint{4.261700in}{2.095201in}}%
\pgfpathlineto{\pgfqpoint{4.261904in}{2.325591in}}%
\pgfpathlineto{\pgfqpoint{4.262516in}{2.267993in}}%
\pgfpathlineto{\pgfqpoint{4.263127in}{2.358504in}}%
\pgfpathlineto{\pgfqpoint{4.263535in}{2.309134in}}%
\pgfpathlineto{\pgfqpoint{4.263739in}{2.276221in}}%
\pgfpathlineto{\pgfqpoint{4.263942in}{2.383188in}}%
\pgfpathlineto{\pgfqpoint{4.264146in}{2.383188in}}%
\pgfpathlineto{\pgfqpoint{4.264554in}{2.358504in}}%
\pgfpathlineto{\pgfqpoint{4.264758in}{2.416101in}}%
\pgfpathlineto{\pgfqpoint{4.265980in}{2.111657in}}%
\pgfpathlineto{\pgfqpoint{4.266999in}{2.374960in}}%
\pgfpathlineto{\pgfqpoint{4.267203in}{2.317363in}}%
\pgfpathlineto{\pgfqpoint{4.267407in}{2.243309in}}%
\pgfpathlineto{\pgfqpoint{4.267611in}{2.424329in}}%
\pgfpathlineto{\pgfqpoint{4.268018in}{2.358504in}}%
\pgfpathlineto{\pgfqpoint{4.268426in}{2.407873in}}%
\pgfpathlineto{\pgfqpoint{4.269037in}{2.366732in}}%
\pgfpathlineto{\pgfqpoint{4.269649in}{2.383188in}}%
\pgfpathlineto{\pgfqpoint{4.270056in}{2.333819in}}%
\pgfpathlineto{\pgfqpoint{4.270464in}{2.325591in}}%
\pgfpathlineto{\pgfqpoint{4.271279in}{2.374960in}}%
\pgfpathlineto{\pgfqpoint{4.271483in}{2.366732in}}%
\pgfpathlineto{\pgfqpoint{4.271891in}{2.416101in}}%
\pgfpathlineto{\pgfqpoint{4.272298in}{2.350275in}}%
\pgfpathlineto{\pgfqpoint{4.272502in}{2.350275in}}%
\pgfpathlineto{\pgfqpoint{4.272706in}{2.309134in}}%
\pgfpathlineto{\pgfqpoint{4.273113in}{2.366732in}}%
\pgfpathlineto{\pgfqpoint{4.273725in}{2.473698in}}%
\pgfpathlineto{\pgfqpoint{4.274336in}{2.457242in}}%
\pgfpathlineto{\pgfqpoint{4.275355in}{2.399645in}}%
\pgfpathlineto{\pgfqpoint{4.275559in}{2.407873in}}%
\pgfpathlineto{\pgfqpoint{4.275763in}{2.440786in}}%
\pgfpathlineto{\pgfqpoint{4.276170in}{2.374960in}}%
\pgfpathlineto{\pgfqpoint{4.277190in}{2.276221in}}%
\pgfpathlineto{\pgfqpoint{4.277393in}{2.333819in}}%
\pgfpathlineto{\pgfqpoint{4.277801in}{2.103429in}}%
\pgfpathlineto{\pgfqpoint{4.278412in}{2.292678in}}%
\pgfpathlineto{\pgfqpoint{4.279228in}{2.416101in}}%
\pgfpathlineto{\pgfqpoint{4.279635in}{2.374960in}}%
\pgfpathlineto{\pgfqpoint{4.279839in}{2.366732in}}%
\pgfpathlineto{\pgfqpoint{4.280247in}{2.490155in}}%
\pgfpathlineto{\pgfqpoint{4.280450in}{2.169255in}}%
\pgfpathlineto{\pgfqpoint{4.281266in}{2.416101in}}%
\pgfpathlineto{\pgfqpoint{4.281469in}{2.399645in}}%
\pgfpathlineto{\pgfqpoint{4.281673in}{2.465470in}}%
\pgfpathlineto{\pgfqpoint{4.281877in}{2.440786in}}%
\pgfpathlineto{\pgfqpoint{4.282285in}{2.498383in}}%
\pgfpathlineto{\pgfqpoint{4.282692in}{2.350275in}}%
\pgfpathlineto{\pgfqpoint{4.283507in}{2.383188in}}%
\pgfpathlineto{\pgfqpoint{4.283711in}{2.391416in}}%
\pgfpathlineto{\pgfqpoint{4.283915in}{2.366732in}}%
\pgfpathlineto{\pgfqpoint{4.284323in}{2.383188in}}%
\pgfpathlineto{\pgfqpoint{4.284526in}{2.333819in}}%
\pgfpathlineto{\pgfqpoint{4.285342in}{2.358504in}}%
\pgfpathlineto{\pgfqpoint{4.285545in}{2.424329in}}%
\pgfpathlineto{\pgfqpoint{4.286157in}{2.284450in}}%
\pgfpathlineto{\pgfqpoint{4.286361in}{2.276221in}}%
\pgfpathlineto{\pgfqpoint{4.286564in}{2.284450in}}%
\pgfpathlineto{\pgfqpoint{4.288195in}{2.358504in}}%
\pgfpathlineto{\pgfqpoint{4.289010in}{2.276221in}}%
\pgfpathlineto{\pgfqpoint{4.289418in}{2.325591in}}%
\pgfpathlineto{\pgfqpoint{4.290437in}{2.251537in}}%
\pgfpathlineto{\pgfqpoint{4.291660in}{2.383188in}}%
\pgfpathlineto{\pgfqpoint{4.291863in}{2.374960in}}%
\pgfpathlineto{\pgfqpoint{4.293698in}{2.473698in}}%
\pgfpathlineto{\pgfqpoint{4.294920in}{2.391416in}}%
\pgfpathlineto{\pgfqpoint{4.294105in}{2.481927in}}%
\pgfpathlineto{\pgfqpoint{4.295124in}{2.407873in}}%
\pgfpathlineto{\pgfqpoint{4.295532in}{2.473698in}}%
\pgfpathlineto{\pgfqpoint{4.295939in}{2.383188in}}%
\pgfpathlineto{\pgfqpoint{4.296347in}{2.358504in}}%
\pgfpathlineto{\pgfqpoint{4.296551in}{2.374960in}}%
\pgfpathlineto{\pgfqpoint{4.296755in}{2.424329in}}%
\pgfpathlineto{\pgfqpoint{4.297570in}{2.374960in}}%
\pgfpathlineto{\pgfqpoint{4.297977in}{2.358504in}}%
\pgfpathlineto{\pgfqpoint{4.298181in}{2.407873in}}%
\pgfpathlineto{\pgfqpoint{4.298385in}{2.374960in}}%
\pgfpathlineto{\pgfqpoint{4.299200in}{2.366732in}}%
\pgfpathlineto{\pgfqpoint{4.299608in}{2.465470in}}%
\pgfpathlineto{\pgfqpoint{4.300219in}{2.342047in}}%
\pgfpathlineto{\pgfqpoint{4.300627in}{2.407873in}}%
\pgfpathlineto{\pgfqpoint{4.300831in}{2.399645in}}%
\pgfpathlineto{\pgfqpoint{4.302053in}{2.449014in}}%
\pgfpathlineto{\pgfqpoint{4.302461in}{2.465470in}}%
\pgfpathlineto{\pgfqpoint{4.302665in}{2.432557in}}%
\pgfpathlineto{\pgfqpoint{4.303888in}{2.514839in}}%
\pgfpathlineto{\pgfqpoint{4.304703in}{2.407873in}}%
\pgfpathlineto{\pgfqpoint{4.305111in}{2.424329in}}%
\pgfpathlineto{\pgfqpoint{4.305314in}{2.481927in}}%
\pgfpathlineto{\pgfqpoint{4.306130in}{2.399645in}}%
\pgfpathlineto{\pgfqpoint{4.306741in}{2.440786in}}%
\pgfpathlineto{\pgfqpoint{4.306945in}{2.374960in}}%
\pgfpathlineto{\pgfqpoint{4.307149in}{2.416101in}}%
\pgfpathlineto{\pgfqpoint{4.307352in}{2.358504in}}%
\pgfpathlineto{\pgfqpoint{4.308168in}{2.407873in}}%
\pgfpathlineto{\pgfqpoint{4.308371in}{2.424329in}}%
\pgfpathlineto{\pgfqpoint{4.308983in}{2.333819in}}%
\pgfpathlineto{\pgfqpoint{4.309594in}{2.350275in}}%
\pgfpathlineto{\pgfqpoint{4.310409in}{2.498383in}}%
\pgfpathlineto{\pgfqpoint{4.311021in}{2.449014in}}%
\pgfpathlineto{\pgfqpoint{4.311225in}{2.309134in}}%
\pgfpathlineto{\pgfqpoint{4.312040in}{2.391416in}}%
\pgfpathlineto{\pgfqpoint{4.312244in}{2.432557in}}%
\pgfpathlineto{\pgfqpoint{4.312855in}{2.350275in}}%
\pgfpathlineto{\pgfqpoint{4.313263in}{2.399645in}}%
\pgfpathlineto{\pgfqpoint{4.313670in}{2.366732in}}%
\pgfpathlineto{\pgfqpoint{4.314485in}{2.457242in}}%
\pgfpathlineto{\pgfqpoint{4.314689in}{2.276221in}}%
\pgfpathlineto{\pgfqpoint{4.315504in}{2.366732in}}%
\pgfpathlineto{\pgfqpoint{4.316523in}{2.309134in}}%
\pgfpathlineto{\pgfqpoint{4.316727in}{2.078745in}}%
\pgfpathlineto{\pgfqpoint{4.317543in}{2.383188in}}%
\pgfpathlineto{\pgfqpoint{4.317746in}{2.440786in}}%
\pgfpathlineto{\pgfqpoint{4.318358in}{2.342047in}}%
\pgfpathlineto{\pgfqpoint{4.318765in}{2.416101in}}%
\pgfpathlineto{\pgfqpoint{4.318969in}{2.366732in}}%
\pgfpathlineto{\pgfqpoint{4.319173in}{2.424329in}}%
\pgfpathlineto{\pgfqpoint{4.319581in}{2.407873in}}%
\pgfpathlineto{\pgfqpoint{4.320396in}{2.498383in}}%
\pgfpathlineto{\pgfqpoint{4.319988in}{2.383188in}}%
\pgfpathlineto{\pgfqpoint{4.320600in}{2.473698in}}%
\pgfpathlineto{\pgfqpoint{4.321211in}{2.424329in}}%
\pgfpathlineto{\pgfqpoint{4.321619in}{2.432557in}}%
\pgfpathlineto{\pgfqpoint{4.321822in}{2.498383in}}%
\pgfpathlineto{\pgfqpoint{4.322026in}{2.407873in}}%
\pgfpathlineto{\pgfqpoint{4.322638in}{2.432557in}}%
\pgfpathlineto{\pgfqpoint{4.322841in}{2.449014in}}%
\pgfpathlineto{\pgfqpoint{4.323249in}{2.177483in}}%
\pgfpathlineto{\pgfqpoint{4.323860in}{2.383188in}}%
\pgfpathlineto{\pgfqpoint{4.324268in}{2.424329in}}%
\pgfpathlineto{\pgfqpoint{4.324676in}{2.350275in}}%
\pgfpathlineto{\pgfqpoint{4.324879in}{2.407873in}}%
\pgfpathlineto{\pgfqpoint{4.325287in}{2.358504in}}%
\pgfpathlineto{\pgfqpoint{4.325491in}{2.416101in}}%
\pgfpathlineto{\pgfqpoint{4.325695in}{2.416101in}}%
\pgfpathlineto{\pgfqpoint{4.328548in}{2.572437in}}%
\pgfpathlineto{\pgfqpoint{4.328752in}{2.605350in}}%
\pgfpathlineto{\pgfqpoint{4.329159in}{2.514839in}}%
\pgfpathlineto{\pgfqpoint{4.329363in}{2.580665in}}%
\pgfpathlineto{\pgfqpoint{4.330586in}{2.531296in}}%
\pgfpathlineto{\pgfqpoint{4.330790in}{2.597121in}}%
\pgfpathlineto{\pgfqpoint{4.330994in}{2.383188in}}%
\pgfpathlineto{\pgfqpoint{4.331401in}{2.317363in}}%
\pgfpathlineto{\pgfqpoint{4.331401in}{2.317363in}}%
\pgfusepath{stroke}%
\end{pgfscope}%
\begin{pgfscope}%
\pgfsetrectcap%
\pgfsetmiterjoin%
\pgfsetlinewidth{0.803000pt}%
\definecolor{currentstroke}{rgb}{0.000000,0.000000,0.000000}%
\pgfsetstrokecolor{currentstroke}%
\pgfsetdash{}{0pt}%
\pgfpathmoveto{\pgfqpoint{0.633813in}{0.538014in}}%
\pgfpathlineto{\pgfqpoint{0.633813in}{2.936535in}}%
\pgfusepath{stroke}%
\end{pgfscope}%
\begin{pgfscope}%
\pgfsetrectcap%
\pgfsetmiterjoin%
\pgfsetlinewidth{0.803000pt}%
\definecolor{currentstroke}{rgb}{0.000000,0.000000,0.000000}%
\pgfsetstrokecolor{currentstroke}%
\pgfsetdash{}{0pt}%
\pgfpathmoveto{\pgfqpoint{4.507477in}{0.538014in}}%
\pgfpathlineto{\pgfqpoint{4.507477in}{2.936535in}}%
\pgfusepath{stroke}%
\end{pgfscope}%
\begin{pgfscope}%
\pgfsetrectcap%
\pgfsetmiterjoin%
\pgfsetlinewidth{0.803000pt}%
\definecolor{currentstroke}{rgb}{0.000000,0.000000,0.000000}%
\pgfsetstrokecolor{currentstroke}%
\pgfsetdash{}{0pt}%
\pgfpathmoveto{\pgfqpoint{0.633813in}{0.538014in}}%
\pgfpathlineto{\pgfqpoint{4.507477in}{0.538014in}}%
\pgfusepath{stroke}%
\end{pgfscope}%
\begin{pgfscope}%
\pgfsetrectcap%
\pgfsetmiterjoin%
\pgfsetlinewidth{0.803000pt}%
\definecolor{currentstroke}{rgb}{0.000000,0.000000,0.000000}%
\pgfsetstrokecolor{currentstroke}%
\pgfsetdash{}{0pt}%
\pgfpathmoveto{\pgfqpoint{0.633813in}{2.936535in}}%
\pgfpathlineto{\pgfqpoint{4.507477in}{2.936535in}}%
\pgfusepath{stroke}%
\end{pgfscope}%
\begin{pgfscope}%
\pgfsetbuttcap%
\pgfsetroundjoin%
\definecolor{currentfill}{rgb}{0.000000,0.000000,0.000000}%
\pgfsetfillcolor{currentfill}%
\pgfsetlinewidth{0.803000pt}%
\definecolor{currentstroke}{rgb}{0.000000,0.000000,0.000000}%
\pgfsetstrokecolor{currentstroke}%
\pgfsetdash{}{0pt}%
\pgfsys@defobject{currentmarker}{\pgfqpoint{0.000000in}{0.000000in}}{\pgfqpoint{0.048611in}{0.000000in}}{%
\pgfpathmoveto{\pgfqpoint{0.000000in}{0.000000in}}%
\pgfpathlineto{\pgfqpoint{0.048611in}{0.000000in}}%
\pgfusepath{stroke,fill}%
}%
\begin{pgfscope}%
\pgfsys@transformshift{4.507477in}{0.723354in}%
\pgfsys@useobject{currentmarker}{}%
\end{pgfscope}%
\end{pgfscope}%
\begin{pgfscope}%
\definecolor{textcolor}{rgb}{0.000000,0.000000,0.000000}%
\pgfsetstrokecolor{textcolor}%
\pgfsetfillcolor{textcolor}%
\pgftext[x=4.604699in, y=0.684799in, left, base]{\color{textcolor}\rmfamily\fontsize{8.000000}{9.600000}\selectfont \(\displaystyle {20.00}\)}%
\end{pgfscope}%
\begin{pgfscope}%
\pgfsetbuttcap%
\pgfsetroundjoin%
\definecolor{currentfill}{rgb}{0.000000,0.000000,0.000000}%
\pgfsetfillcolor{currentfill}%
\pgfsetlinewidth{0.803000pt}%
\definecolor{currentstroke}{rgb}{0.000000,0.000000,0.000000}%
\pgfsetstrokecolor{currentstroke}%
\pgfsetdash{}{0pt}%
\pgfsys@defobject{currentmarker}{\pgfqpoint{0.000000in}{0.000000in}}{\pgfqpoint{0.048611in}{0.000000in}}{%
\pgfpathmoveto{\pgfqpoint{0.000000in}{0.000000in}}%
\pgfpathlineto{\pgfqpoint{0.048611in}{0.000000in}}%
\pgfusepath{stroke,fill}%
}%
\begin{pgfscope}%
\pgfsys@transformshift{4.507477in}{0.995913in}%
\pgfsys@useobject{currentmarker}{}%
\end{pgfscope}%
\end{pgfscope}%
\begin{pgfscope}%
\definecolor{textcolor}{rgb}{0.000000,0.000000,0.000000}%
\pgfsetstrokecolor{textcolor}%
\pgfsetfillcolor{textcolor}%
\pgftext[x=4.604699in, y=0.957358in, left, base]{\color{textcolor}\rmfamily\fontsize{8.000000}{9.600000}\selectfont \(\displaystyle {20.25}\)}%
\end{pgfscope}%
\begin{pgfscope}%
\pgfsetbuttcap%
\pgfsetroundjoin%
\definecolor{currentfill}{rgb}{0.000000,0.000000,0.000000}%
\pgfsetfillcolor{currentfill}%
\pgfsetlinewidth{0.803000pt}%
\definecolor{currentstroke}{rgb}{0.000000,0.000000,0.000000}%
\pgfsetstrokecolor{currentstroke}%
\pgfsetdash{}{0pt}%
\pgfsys@defobject{currentmarker}{\pgfqpoint{0.000000in}{0.000000in}}{\pgfqpoint{0.048611in}{0.000000in}}{%
\pgfpathmoveto{\pgfqpoint{0.000000in}{0.000000in}}%
\pgfpathlineto{\pgfqpoint{0.048611in}{0.000000in}}%
\pgfusepath{stroke,fill}%
}%
\begin{pgfscope}%
\pgfsys@transformshift{4.507477in}{1.268473in}%
\pgfsys@useobject{currentmarker}{}%
\end{pgfscope}%
\end{pgfscope}%
\begin{pgfscope}%
\definecolor{textcolor}{rgb}{0.000000,0.000000,0.000000}%
\pgfsetstrokecolor{textcolor}%
\pgfsetfillcolor{textcolor}%
\pgftext[x=4.604699in, y=1.229917in, left, base]{\color{textcolor}\rmfamily\fontsize{8.000000}{9.600000}\selectfont \(\displaystyle {20.50}\)}%
\end{pgfscope}%
\begin{pgfscope}%
\pgfsetbuttcap%
\pgfsetroundjoin%
\definecolor{currentfill}{rgb}{0.000000,0.000000,0.000000}%
\pgfsetfillcolor{currentfill}%
\pgfsetlinewidth{0.803000pt}%
\definecolor{currentstroke}{rgb}{0.000000,0.000000,0.000000}%
\pgfsetstrokecolor{currentstroke}%
\pgfsetdash{}{0pt}%
\pgfsys@defobject{currentmarker}{\pgfqpoint{0.000000in}{0.000000in}}{\pgfqpoint{0.048611in}{0.000000in}}{%
\pgfpathmoveto{\pgfqpoint{0.000000in}{0.000000in}}%
\pgfpathlineto{\pgfqpoint{0.048611in}{0.000000in}}%
\pgfusepath{stroke,fill}%
}%
\begin{pgfscope}%
\pgfsys@transformshift{4.507477in}{1.541032in}%
\pgfsys@useobject{currentmarker}{}%
\end{pgfscope}%
\end{pgfscope}%
\begin{pgfscope}%
\definecolor{textcolor}{rgb}{0.000000,0.000000,0.000000}%
\pgfsetstrokecolor{textcolor}%
\pgfsetfillcolor{textcolor}%
\pgftext[x=4.604699in, y=1.502476in, left, base]{\color{textcolor}\rmfamily\fontsize{8.000000}{9.600000}\selectfont \(\displaystyle {20.75}\)}%
\end{pgfscope}%
\begin{pgfscope}%
\pgfsetbuttcap%
\pgfsetroundjoin%
\definecolor{currentfill}{rgb}{0.000000,0.000000,0.000000}%
\pgfsetfillcolor{currentfill}%
\pgfsetlinewidth{0.803000pt}%
\definecolor{currentstroke}{rgb}{0.000000,0.000000,0.000000}%
\pgfsetstrokecolor{currentstroke}%
\pgfsetdash{}{0pt}%
\pgfsys@defobject{currentmarker}{\pgfqpoint{0.000000in}{0.000000in}}{\pgfqpoint{0.048611in}{0.000000in}}{%
\pgfpathmoveto{\pgfqpoint{0.000000in}{0.000000in}}%
\pgfpathlineto{\pgfqpoint{0.048611in}{0.000000in}}%
\pgfusepath{stroke,fill}%
}%
\begin{pgfscope}%
\pgfsys@transformshift{4.507477in}{1.813591in}%
\pgfsys@useobject{currentmarker}{}%
\end{pgfscope}%
\end{pgfscope}%
\begin{pgfscope}%
\definecolor{textcolor}{rgb}{0.000000,0.000000,0.000000}%
\pgfsetstrokecolor{textcolor}%
\pgfsetfillcolor{textcolor}%
\pgftext[x=4.604699in, y=1.775035in, left, base]{\color{textcolor}\rmfamily\fontsize{8.000000}{9.600000}\selectfont \(\displaystyle {21.00}\)}%
\end{pgfscope}%
\begin{pgfscope}%
\pgfsetbuttcap%
\pgfsetroundjoin%
\definecolor{currentfill}{rgb}{0.000000,0.000000,0.000000}%
\pgfsetfillcolor{currentfill}%
\pgfsetlinewidth{0.803000pt}%
\definecolor{currentstroke}{rgb}{0.000000,0.000000,0.000000}%
\pgfsetstrokecolor{currentstroke}%
\pgfsetdash{}{0pt}%
\pgfsys@defobject{currentmarker}{\pgfqpoint{0.000000in}{0.000000in}}{\pgfqpoint{0.048611in}{0.000000in}}{%
\pgfpathmoveto{\pgfqpoint{0.000000in}{0.000000in}}%
\pgfpathlineto{\pgfqpoint{0.048611in}{0.000000in}}%
\pgfusepath{stroke,fill}%
}%
\begin{pgfscope}%
\pgfsys@transformshift{4.507477in}{2.086150in}%
\pgfsys@useobject{currentmarker}{}%
\end{pgfscope}%
\end{pgfscope}%
\begin{pgfscope}%
\definecolor{textcolor}{rgb}{0.000000,0.000000,0.000000}%
\pgfsetstrokecolor{textcolor}%
\pgfsetfillcolor{textcolor}%
\pgftext[x=4.604699in, y=2.047594in, left, base]{\color{textcolor}\rmfamily\fontsize{8.000000}{9.600000}\selectfont \(\displaystyle {21.25}\)}%
\end{pgfscope}%
\begin{pgfscope}%
\pgfsetbuttcap%
\pgfsetroundjoin%
\definecolor{currentfill}{rgb}{0.000000,0.000000,0.000000}%
\pgfsetfillcolor{currentfill}%
\pgfsetlinewidth{0.803000pt}%
\definecolor{currentstroke}{rgb}{0.000000,0.000000,0.000000}%
\pgfsetstrokecolor{currentstroke}%
\pgfsetdash{}{0pt}%
\pgfsys@defobject{currentmarker}{\pgfqpoint{0.000000in}{0.000000in}}{\pgfqpoint{0.048611in}{0.000000in}}{%
\pgfpathmoveto{\pgfqpoint{0.000000in}{0.000000in}}%
\pgfpathlineto{\pgfqpoint{0.048611in}{0.000000in}}%
\pgfusepath{stroke,fill}%
}%
\begin{pgfscope}%
\pgfsys@transformshift{4.507477in}{2.358709in}%
\pgfsys@useobject{currentmarker}{}%
\end{pgfscope}%
\end{pgfscope}%
\begin{pgfscope}%
\definecolor{textcolor}{rgb}{0.000000,0.000000,0.000000}%
\pgfsetstrokecolor{textcolor}%
\pgfsetfillcolor{textcolor}%
\pgftext[x=4.604699in, y=2.320154in, left, base]{\color{textcolor}\rmfamily\fontsize{8.000000}{9.600000}\selectfont \(\displaystyle {21.50}\)}%
\end{pgfscope}%
\begin{pgfscope}%
\pgfsetbuttcap%
\pgfsetroundjoin%
\definecolor{currentfill}{rgb}{0.000000,0.000000,0.000000}%
\pgfsetfillcolor{currentfill}%
\pgfsetlinewidth{0.803000pt}%
\definecolor{currentstroke}{rgb}{0.000000,0.000000,0.000000}%
\pgfsetstrokecolor{currentstroke}%
\pgfsetdash{}{0pt}%
\pgfsys@defobject{currentmarker}{\pgfqpoint{0.000000in}{0.000000in}}{\pgfqpoint{0.048611in}{0.000000in}}{%
\pgfpathmoveto{\pgfqpoint{0.000000in}{0.000000in}}%
\pgfpathlineto{\pgfqpoint{0.048611in}{0.000000in}}%
\pgfusepath{stroke,fill}%
}%
\begin{pgfscope}%
\pgfsys@transformshift{4.507477in}{2.631268in}%
\pgfsys@useobject{currentmarker}{}%
\end{pgfscope}%
\end{pgfscope}%
\begin{pgfscope}%
\definecolor{textcolor}{rgb}{0.000000,0.000000,0.000000}%
\pgfsetstrokecolor{textcolor}%
\pgfsetfillcolor{textcolor}%
\pgftext[x=4.604699in, y=2.592713in, left, base]{\color{textcolor}\rmfamily\fontsize{8.000000}{9.600000}\selectfont \(\displaystyle {21.75}\)}%
\end{pgfscope}%
\begin{pgfscope}%
\pgfsetbuttcap%
\pgfsetroundjoin%
\definecolor{currentfill}{rgb}{0.000000,0.000000,0.000000}%
\pgfsetfillcolor{currentfill}%
\pgfsetlinewidth{0.803000pt}%
\definecolor{currentstroke}{rgb}{0.000000,0.000000,0.000000}%
\pgfsetstrokecolor{currentstroke}%
\pgfsetdash{}{0pt}%
\pgfsys@defobject{currentmarker}{\pgfqpoint{0.000000in}{0.000000in}}{\pgfqpoint{0.048611in}{0.000000in}}{%
\pgfpathmoveto{\pgfqpoint{0.000000in}{0.000000in}}%
\pgfpathlineto{\pgfqpoint{0.048611in}{0.000000in}}%
\pgfusepath{stroke,fill}%
}%
\begin{pgfscope}%
\pgfsys@transformshift{4.507477in}{2.903828in}%
\pgfsys@useobject{currentmarker}{}%
\end{pgfscope}%
\end{pgfscope}%
\begin{pgfscope}%
\definecolor{textcolor}{rgb}{0.000000,0.000000,0.000000}%
\pgfsetstrokecolor{textcolor}%
\pgfsetfillcolor{textcolor}%
\pgftext[x=4.604699in, y=2.865272in, left, base]{\color{textcolor}\rmfamily\fontsize{8.000000}{9.600000}\selectfont \(\displaystyle {22.00}\)}%
\end{pgfscope}%
\begin{pgfscope}%
\definecolor{textcolor}{rgb}{0.000000,0.000000,0.000000}%
\pgfsetstrokecolor{textcolor}%
\pgfsetfillcolor{textcolor}%
\pgftext[x=4.929163in,y=1.737274in,,top,rotate=90.000000]{\color{textcolor}\rmfamily\fontsize{10.000000}{12.000000}\selectfont Temperature in °C}%
\end{pgfscope}%
\begin{pgfscope}%
\pgfpathrectangle{\pgfqpoint{0.633813in}{0.538014in}}{\pgfqpoint{3.873664in}{2.398521in}}%
\pgfusepath{clip}%
\pgfsetrectcap%
\pgfsetroundjoin%
\pgfsetlinewidth{0.501875pt}%
\definecolor{currentstroke}{rgb}{0.698039,0.133333,0.133333}%
\pgfsetstrokecolor{currentstroke}%
\pgfsetstrokeopacity{0.700000}%
\pgfsetdash{}{0pt}%
\pgfpathmoveto{\pgfqpoint{0.810420in}{1.944419in}}%
\pgfpathlineto{\pgfqpoint{0.816575in}{2.086150in}}%
\pgfpathlineto{\pgfqpoint{0.819142in}{2.009833in}}%
\pgfpathlineto{\pgfqpoint{0.824401in}{2.151564in}}%
\pgfpathlineto{\pgfqpoint{0.826846in}{2.086150in}}%
\pgfpathlineto{\pgfqpoint{0.832797in}{2.216978in}}%
\pgfpathlineto{\pgfqpoint{0.835243in}{2.151564in}}%
\pgfpathlineto{\pgfqpoint{0.837689in}{2.216978in}}%
\pgfpathlineto{\pgfqpoint{0.843354in}{2.282393in}}%
\pgfpathlineto{\pgfqpoint{0.845800in}{2.216978in}}%
\pgfpathlineto{\pgfqpoint{0.848246in}{2.282393in}}%
\pgfpathlineto{\pgfqpoint{0.856765in}{2.358709in}}%
\pgfpathlineto{\pgfqpoint{0.859210in}{2.282393in}}%
\pgfpathlineto{\pgfqpoint{0.861656in}{2.358709in}}%
\pgfpathlineto{\pgfqpoint{0.868096in}{2.424123in}}%
\pgfpathlineto{\pgfqpoint{0.870542in}{2.358709in}}%
\pgfpathlineto{\pgfqpoint{0.872987in}{2.424123in}}%
\pgfpathlineto{\pgfqpoint{0.875840in}{2.358709in}}%
\pgfpathlineto{\pgfqpoint{0.878286in}{2.424123in}}%
\pgfpathlineto{\pgfqpoint{0.884808in}{2.489538in}}%
\pgfpathlineto{\pgfqpoint{0.887253in}{2.424123in}}%
\pgfpathlineto{\pgfqpoint{0.889699in}{2.489538in}}%
\pgfpathlineto{\pgfqpoint{0.901642in}{2.554952in}}%
\pgfpathlineto{\pgfqpoint{0.904088in}{2.489538in}}%
\pgfpathlineto{\pgfqpoint{0.906574in}{2.554952in}}%
\pgfpathlineto{\pgfqpoint{0.909223in}{2.489538in}}%
\pgfpathlineto{\pgfqpoint{0.911669in}{2.554952in}}%
\pgfpathlineto{\pgfqpoint{0.922430in}{2.631268in}}%
\pgfpathlineto{\pgfqpoint{0.924875in}{2.554952in}}%
\pgfpathlineto{\pgfqpoint{0.927321in}{2.631268in}}%
\pgfpathlineto{\pgfqpoint{0.930011in}{2.554952in}}%
\pgfpathlineto{\pgfqpoint{0.932457in}{2.631268in}}%
\pgfpathlineto{\pgfqpoint{0.945256in}{2.696683in}}%
\pgfpathlineto{\pgfqpoint{0.947701in}{2.631268in}}%
\pgfpathlineto{\pgfqpoint{0.950147in}{2.696683in}}%
\pgfpathlineto{\pgfqpoint{0.952593in}{2.631268in}}%
\pgfpathlineto{\pgfqpoint{0.955038in}{2.696683in}}%
\pgfpathlineto{\pgfqpoint{0.969590in}{2.762097in}}%
\pgfpathlineto{\pgfqpoint{0.972036in}{2.696683in}}%
\pgfpathlineto{\pgfqpoint{0.974522in}{2.762097in}}%
\pgfpathlineto{\pgfqpoint{0.976968in}{2.696683in}}%
\pgfpathlineto{\pgfqpoint{0.979413in}{2.762097in}}%
\pgfpathlineto{\pgfqpoint{0.982144in}{2.696683in}}%
\pgfpathlineto{\pgfqpoint{0.984590in}{2.762097in}}%
\pgfpathlineto{\pgfqpoint{0.997144in}{2.827511in}}%
\pgfpathlineto{\pgfqpoint{0.999590in}{2.762097in}}%
\pgfpathlineto{\pgfqpoint{1.002035in}{2.827511in}}%
\pgfpathlineto{\pgfqpoint{1.004481in}{2.762097in}}%
\pgfpathlineto{\pgfqpoint{1.006927in}{2.827511in}}%
\pgfpathlineto{\pgfqpoint{1.009494in}{2.762097in}}%
\pgfpathlineto{\pgfqpoint{1.011940in}{2.827511in}}%
\pgfpathlineto{\pgfqpoint{1.017647in}{2.696683in}}%
\pgfpathlineto{\pgfqpoint{1.020092in}{2.631268in}}%
\pgfpathlineto{\pgfqpoint{1.024984in}{2.358709in}}%
\pgfpathlineto{\pgfqpoint{1.032320in}{2.151564in}}%
\pgfpathlineto{\pgfqpoint{1.034766in}{2.009833in}}%
\pgfpathlineto{\pgfqpoint{1.050867in}{1.606446in}}%
\pgfpathlineto{\pgfqpoint{1.057388in}{1.464715in}}%
\pgfpathlineto{\pgfqpoint{1.066763in}{1.333887in}}%
\pgfpathlineto{\pgfqpoint{1.069250in}{1.399301in}}%
\pgfpathlineto{\pgfqpoint{1.074304in}{1.268473in}}%
\pgfpathlineto{\pgfqpoint{1.079684in}{1.192156in}}%
\pgfpathlineto{\pgfqpoint{1.087510in}{1.126742in}}%
\pgfpathlineto{\pgfqpoint{1.090037in}{1.192156in}}%
\pgfpathlineto{\pgfqpoint{1.095336in}{1.061328in}}%
\pgfpathlineto{\pgfqpoint{1.097782in}{1.126742in}}%
\pgfpathlineto{\pgfqpoint{1.100228in}{1.061328in}}%
\pgfpathlineto{\pgfqpoint{1.102755in}{1.126742in}}%
\pgfpathlineto{\pgfqpoint{1.105200in}{1.061328in}}%
\pgfpathlineto{\pgfqpoint{1.111355in}{0.995913in}}%
\pgfpathlineto{\pgfqpoint{1.113801in}{1.061328in}}%
\pgfpathlineto{\pgfqpoint{1.116287in}{0.995913in}}%
\pgfpathlineto{\pgfqpoint{1.118855in}{1.061328in}}%
\pgfpathlineto{\pgfqpoint{1.121301in}{0.995913in}}%
\pgfpathlineto{\pgfqpoint{1.125988in}{0.919597in}}%
\pgfpathlineto{\pgfqpoint{1.128475in}{0.995913in}}%
\pgfpathlineto{\pgfqpoint{1.130920in}{0.919597in}}%
\pgfpathlineto{\pgfqpoint{1.138216in}{0.854183in}}%
\pgfpathlineto{\pgfqpoint{1.140662in}{0.919597in}}%
\pgfpathlineto{\pgfqpoint{1.143108in}{0.854183in}}%
\pgfpathlineto{\pgfqpoint{1.146287in}{0.919597in}}%
\pgfpathlineto{\pgfqpoint{1.151341in}{0.788768in}}%
\pgfpathlineto{\pgfqpoint{1.153787in}{0.854183in}}%
\pgfpathlineto{\pgfqpoint{1.156273in}{0.788768in}}%
\pgfpathlineto{\pgfqpoint{1.158719in}{0.854183in}}%
\pgfpathlineto{\pgfqpoint{1.161165in}{0.788768in}}%
\pgfpathlineto{\pgfqpoint{1.171722in}{0.854183in}}%
\pgfpathlineto{\pgfqpoint{1.174167in}{0.788768in}}%
\pgfpathlineto{\pgfqpoint{1.176654in}{0.854183in}}%
\pgfpathlineto{\pgfqpoint{1.179099in}{0.788768in}}%
\pgfpathlineto{\pgfqpoint{1.181830in}{0.854183in}}%
\pgfpathlineto{\pgfqpoint{1.184276in}{0.788768in}}%
\pgfpathlineto{\pgfqpoint{1.202577in}{0.854183in}}%
\pgfpathlineto{\pgfqpoint{1.207183in}{0.919597in}}%
\pgfpathlineto{\pgfqpoint{1.224303in}{1.268473in}}%
\pgfpathlineto{\pgfqpoint{1.250390in}{1.671860in}}%
\pgfpathlineto{\pgfqpoint{1.273909in}{1.944419in}}%
\pgfpathlineto{\pgfqpoint{1.276476in}{1.879005in}}%
\pgfpathlineto{\pgfqpoint{1.281449in}{2.009833in}}%
\pgfpathlineto{\pgfqpoint{1.284710in}{1.944419in}}%
\pgfpathlineto{\pgfqpoint{1.289968in}{2.086150in}}%
\pgfpathlineto{\pgfqpoint{1.292658in}{2.009833in}}%
\pgfpathlineto{\pgfqpoint{1.295104in}{2.086150in}}%
\pgfpathlineto{\pgfqpoint{1.298487in}{2.151564in}}%
\pgfpathlineto{\pgfqpoint{1.300933in}{2.086150in}}%
\pgfpathlineto{\pgfqpoint{1.303378in}{2.151564in}}%
\pgfpathlineto{\pgfqpoint{1.308351in}{2.216978in}}%
\pgfpathlineto{\pgfqpoint{1.310838in}{2.151564in}}%
\pgfpathlineto{\pgfqpoint{1.313283in}{2.216978in}}%
\pgfpathlineto{\pgfqpoint{1.318704in}{2.282393in}}%
\pgfpathlineto{\pgfqpoint{1.321150in}{2.216978in}}%
\pgfpathlineto{\pgfqpoint{1.323596in}{2.282393in}}%
\pgfpathlineto{\pgfqpoint{1.332237in}{2.358709in}}%
\pgfpathlineto{\pgfqpoint{1.334723in}{2.282393in}}%
\pgfpathlineto{\pgfqpoint{1.337169in}{2.358709in}}%
\pgfpathlineto{\pgfqpoint{1.344180in}{2.424123in}}%
\pgfpathlineto{\pgfqpoint{1.346625in}{2.358709in}}%
\pgfpathlineto{\pgfqpoint{1.349071in}{2.424123in}}%
\pgfpathlineto{\pgfqpoint{1.351924in}{2.358709in}}%
\pgfpathlineto{\pgfqpoint{1.354370in}{2.424123in}}%
\pgfpathlineto{\pgfqpoint{1.359139in}{2.489538in}}%
\pgfpathlineto{\pgfqpoint{1.361585in}{2.424123in}}%
\pgfpathlineto{\pgfqpoint{1.364030in}{2.489538in}}%
\pgfpathlineto{\pgfqpoint{1.368677in}{2.424123in}}%
\pgfpathlineto{\pgfqpoint{1.371123in}{2.489538in}}%
\pgfpathlineto{\pgfqpoint{1.378215in}{2.554952in}}%
\pgfpathlineto{\pgfqpoint{1.380661in}{2.489538in}}%
\pgfpathlineto{\pgfqpoint{1.383106in}{2.554952in}}%
\pgfpathlineto{\pgfqpoint{1.385878in}{2.489538in}}%
\pgfpathlineto{\pgfqpoint{1.388324in}{2.554952in}}%
\pgfpathlineto{\pgfqpoint{1.397128in}{2.631268in}}%
\pgfpathlineto{\pgfqpoint{1.399573in}{2.554952in}}%
\pgfpathlineto{\pgfqpoint{1.402019in}{2.631268in}}%
\pgfpathlineto{\pgfqpoint{1.404506in}{2.554952in}}%
\pgfpathlineto{\pgfqpoint{1.406951in}{2.631268in}}%
\pgfpathlineto{\pgfqpoint{1.418894in}{2.696683in}}%
\pgfpathlineto{\pgfqpoint{1.421340in}{2.631268in}}%
\pgfpathlineto{\pgfqpoint{1.423826in}{2.696683in}}%
\pgfpathlineto{\pgfqpoint{1.426272in}{2.631268in}}%
\pgfpathlineto{\pgfqpoint{1.428717in}{2.696683in}}%
\pgfpathlineto{\pgfqpoint{1.431693in}{2.631268in}}%
\pgfpathlineto{\pgfqpoint{1.434138in}{2.696683in}}%
\pgfpathlineto{\pgfqpoint{1.441801in}{2.762097in}}%
\pgfpathlineto{\pgfqpoint{1.444247in}{2.696683in}}%
\pgfpathlineto{\pgfqpoint{1.446734in}{2.762097in}}%
\pgfpathlineto{\pgfqpoint{1.449179in}{2.696683in}}%
\pgfpathlineto{\pgfqpoint{1.451625in}{2.762097in}}%
\pgfpathlineto{\pgfqpoint{1.454070in}{2.696683in}}%
\pgfpathlineto{\pgfqpoint{1.456516in}{2.762097in}}%
\pgfpathlineto{\pgfqpoint{1.473432in}{2.827511in}}%
\pgfpathlineto{\pgfqpoint{1.475877in}{2.762097in}}%
\pgfpathlineto{\pgfqpoint{1.478445in}{2.827511in}}%
\pgfpathlineto{\pgfqpoint{1.480891in}{2.762097in}}%
\pgfpathlineto{\pgfqpoint{1.483418in}{2.827511in}}%
\pgfpathlineto{\pgfqpoint{1.485864in}{2.762097in}}%
\pgfpathlineto{\pgfqpoint{1.488309in}{2.827511in}}%
\pgfpathlineto{\pgfqpoint{1.494016in}{2.696683in}}%
\pgfpathlineto{\pgfqpoint{1.496461in}{2.631268in}}%
\pgfpathlineto{\pgfqpoint{1.498907in}{2.489538in}}%
\pgfpathlineto{\pgfqpoint{1.501353in}{2.424123in}}%
\pgfpathlineto{\pgfqpoint{1.503798in}{2.282393in}}%
\pgfpathlineto{\pgfqpoint{1.508690in}{2.151564in}}%
\pgfpathlineto{\pgfqpoint{1.511135in}{2.009833in}}%
\pgfpathlineto{\pgfqpoint{1.527276in}{1.606446in}}%
\pgfpathlineto{\pgfqpoint{1.536285in}{1.464715in}}%
\pgfpathlineto{\pgfqpoint{1.541787in}{1.399301in}}%
\pgfpathlineto{\pgfqpoint{1.544274in}{1.464715in}}%
\pgfpathlineto{\pgfqpoint{1.549165in}{1.333887in}}%
\pgfpathlineto{\pgfqpoint{1.556013in}{1.268473in}}%
\pgfpathlineto{\pgfqpoint{1.558458in}{1.333887in}}%
\pgfpathlineto{\pgfqpoint{1.563716in}{1.192156in}}%
\pgfpathlineto{\pgfqpoint{1.566488in}{1.268473in}}%
\pgfpathlineto{\pgfqpoint{1.568934in}{1.192156in}}%
\pgfpathlineto{\pgfqpoint{1.575048in}{1.126742in}}%
\pgfpathlineto{\pgfqpoint{1.577534in}{1.192156in}}%
\pgfpathlineto{\pgfqpoint{1.579980in}{1.126742in}}%
\pgfpathlineto{\pgfqpoint{1.583608in}{1.061328in}}%
\pgfpathlineto{\pgfqpoint{1.589314in}{0.995913in}}%
\pgfpathlineto{\pgfqpoint{1.591760in}{1.061328in}}%
\pgfpathlineto{\pgfqpoint{1.594205in}{0.995913in}}%
\pgfpathlineto{\pgfqpoint{1.596692in}{1.061328in}}%
\pgfpathlineto{\pgfqpoint{1.599137in}{0.995913in}}%
\pgfpathlineto{\pgfqpoint{1.603132in}{0.919597in}}%
\pgfpathlineto{\pgfqpoint{1.605578in}{0.995913in}}%
\pgfpathlineto{\pgfqpoint{1.608023in}{0.919597in}}%
\pgfpathlineto{\pgfqpoint{1.617724in}{0.854183in}}%
\pgfpathlineto{\pgfqpoint{1.620170in}{0.919597in}}%
\pgfpathlineto{\pgfqpoint{1.622616in}{0.854183in}}%
\pgfpathlineto{\pgfqpoint{1.625102in}{0.919597in}}%
\pgfpathlineto{\pgfqpoint{1.627548in}{0.854183in}}%
\pgfpathlineto{\pgfqpoint{1.632887in}{0.788768in}}%
\pgfpathlineto{\pgfqpoint{1.635333in}{0.854183in}}%
\pgfpathlineto{\pgfqpoint{1.637778in}{0.788768in}}%
\pgfpathlineto{\pgfqpoint{1.640387in}{0.854183in}}%
\pgfpathlineto{\pgfqpoint{1.642833in}{0.788768in}}%
\pgfpathlineto{\pgfqpoint{1.645319in}{0.854183in}}%
\pgfpathlineto{\pgfqpoint{1.647765in}{0.788768in}}%
\pgfpathlineto{\pgfqpoint{1.650659in}{0.854183in}}%
\pgfpathlineto{\pgfqpoint{1.653104in}{0.788768in}}%
\pgfpathlineto{\pgfqpoint{1.671569in}{0.723354in}}%
\pgfpathlineto{\pgfqpoint{1.674015in}{0.788768in}}%
\pgfpathlineto{\pgfqpoint{1.676460in}{0.723354in}}%
\pgfpathlineto{\pgfqpoint{1.678906in}{0.788768in}}%
\pgfpathlineto{\pgfqpoint{1.681352in}{0.723354in}}%
\pgfpathlineto{\pgfqpoint{1.689463in}{0.919597in}}%
\pgfpathlineto{\pgfqpoint{1.711066in}{1.333887in}}%
\pgfpathlineto{\pgfqpoint{1.729979in}{1.606446in}}%
\pgfpathlineto{\pgfqpoint{1.745590in}{1.813591in}}%
\pgfpathlineto{\pgfqpoint{1.757085in}{1.944419in}}%
\pgfpathlineto{\pgfqpoint{1.759530in}{1.879005in}}%
\pgfpathlineto{\pgfqpoint{1.765237in}{2.009833in}}%
\pgfpathlineto{\pgfqpoint{1.768172in}{1.944419in}}%
\pgfpathlineto{\pgfqpoint{1.773307in}{2.086150in}}%
\pgfpathlineto{\pgfqpoint{1.776038in}{2.009833in}}%
\pgfpathlineto{\pgfqpoint{1.778484in}{2.086150in}}%
\pgfpathlineto{\pgfqpoint{1.782764in}{2.151564in}}%
\pgfpathlineto{\pgfqpoint{1.785250in}{2.086150in}}%
\pgfpathlineto{\pgfqpoint{1.790875in}{2.216978in}}%
\pgfpathlineto{\pgfqpoint{1.793362in}{2.151564in}}%
\pgfpathlineto{\pgfqpoint{1.795807in}{2.216978in}}%
\pgfpathlineto{\pgfqpoint{1.801229in}{2.282393in}}%
\pgfpathlineto{\pgfqpoint{1.803674in}{2.216978in}}%
\pgfpathlineto{\pgfqpoint{1.806120in}{2.282393in}}%
\pgfpathlineto{\pgfqpoint{1.815169in}{2.358709in}}%
\pgfpathlineto{\pgfqpoint{1.817614in}{2.282393in}}%
\pgfpathlineto{\pgfqpoint{1.820060in}{2.358709in}}%
\pgfpathlineto{\pgfqpoint{1.826663in}{2.424123in}}%
\pgfpathlineto{\pgfqpoint{1.829109in}{2.358709in}}%
\pgfpathlineto{\pgfqpoint{1.831595in}{2.424123in}}%
\pgfpathlineto{\pgfqpoint{1.834897in}{2.358709in}}%
\pgfpathlineto{\pgfqpoint{1.837342in}{2.424123in}}%
\pgfpathlineto{\pgfqpoint{1.844231in}{2.489538in}}%
\pgfpathlineto{\pgfqpoint{1.846677in}{2.424123in}}%
\pgfpathlineto{\pgfqpoint{1.849122in}{2.489538in}}%
\pgfpathlineto{\pgfqpoint{1.862492in}{2.554952in}}%
\pgfpathlineto{\pgfqpoint{1.864937in}{2.489538in}}%
\pgfpathlineto{\pgfqpoint{1.867383in}{2.554952in}}%
\pgfpathlineto{\pgfqpoint{1.869829in}{2.489538in}}%
\pgfpathlineto{\pgfqpoint{1.872274in}{2.554952in}}%
\pgfpathlineto{\pgfqpoint{1.881853in}{2.631268in}}%
\pgfpathlineto{\pgfqpoint{1.884299in}{2.554952in}}%
\pgfpathlineto{\pgfqpoint{1.886744in}{2.631268in}}%
\pgfpathlineto{\pgfqpoint{1.889271in}{2.554952in}}%
\pgfpathlineto{\pgfqpoint{1.891717in}{2.631268in}}%
\pgfpathlineto{\pgfqpoint{1.904964in}{2.696683in}}%
\pgfpathlineto{\pgfqpoint{1.907410in}{2.631268in}}%
\pgfpathlineto{\pgfqpoint{1.909856in}{2.696683in}}%
\pgfpathlineto{\pgfqpoint{1.912342in}{2.631268in}}%
\pgfpathlineto{\pgfqpoint{1.914788in}{2.696683in}}%
\pgfpathlineto{\pgfqpoint{1.929421in}{2.762097in}}%
\pgfpathlineto{\pgfqpoint{1.931866in}{2.696683in}}%
\pgfpathlineto{\pgfqpoint{1.934760in}{2.762097in}}%
\pgfpathlineto{\pgfqpoint{1.937206in}{2.696683in}}%
\pgfpathlineto{\pgfqpoint{1.939692in}{2.762097in}}%
\pgfpathlineto{\pgfqpoint{1.942179in}{2.696683in}}%
\pgfpathlineto{\pgfqpoint{1.944624in}{2.762097in}}%
\pgfpathlineto{\pgfqpoint{1.959543in}{2.827511in}}%
\pgfpathlineto{\pgfqpoint{1.961988in}{2.762097in}}%
\pgfpathlineto{\pgfqpoint{1.964842in}{2.827511in}}%
\pgfpathlineto{\pgfqpoint{1.967287in}{2.762097in}}%
\pgfpathlineto{\pgfqpoint{1.969733in}{2.827511in}}%
\pgfpathlineto{\pgfqpoint{1.972219in}{2.762097in}}%
\pgfpathlineto{\pgfqpoint{1.974665in}{2.827511in}}%
\pgfpathlineto{\pgfqpoint{1.979923in}{2.696683in}}%
\pgfpathlineto{\pgfqpoint{1.982369in}{2.554952in}}%
\pgfpathlineto{\pgfqpoint{1.984814in}{2.489538in}}%
\pgfpathlineto{\pgfqpoint{1.989706in}{2.282393in}}%
\pgfpathlineto{\pgfqpoint{1.992151in}{2.216978in}}%
\pgfpathlineto{\pgfqpoint{1.997042in}{2.009833in}}%
\pgfpathlineto{\pgfqpoint{2.012531in}{1.606446in}}%
\pgfpathlineto{\pgfqpoint{2.021132in}{1.464715in}}%
\pgfpathlineto{\pgfqpoint{2.023618in}{1.541032in}}%
\pgfpathlineto{\pgfqpoint{2.028510in}{1.399301in}}%
\pgfpathlineto{\pgfqpoint{2.032219in}{1.333887in}}%
\pgfpathlineto{\pgfqpoint{2.034705in}{1.399301in}}%
\pgfpathlineto{\pgfqpoint{2.039597in}{1.268473in}}%
\pgfpathlineto{\pgfqpoint{2.044529in}{1.192156in}}%
\pgfpathlineto{\pgfqpoint{2.047015in}{1.268473in}}%
\pgfpathlineto{\pgfqpoint{2.052273in}{1.126742in}}%
\pgfpathlineto{\pgfqpoint{2.064053in}{0.995913in}}%
\pgfpathlineto{\pgfqpoint{2.066499in}{1.061328in}}%
\pgfpathlineto{\pgfqpoint{2.068944in}{0.995913in}}%
\pgfpathlineto{\pgfqpoint{2.077667in}{0.919597in}}%
\pgfpathlineto{\pgfqpoint{2.080113in}{0.995913in}}%
\pgfpathlineto{\pgfqpoint{2.082558in}{0.919597in}}%
\pgfpathlineto{\pgfqpoint{2.085126in}{0.995913in}}%
\pgfpathlineto{\pgfqpoint{2.087572in}{0.919597in}}%
\pgfpathlineto{\pgfqpoint{2.095316in}{0.854183in}}%
\pgfpathlineto{\pgfqpoint{2.097762in}{0.919597in}}%
\pgfpathlineto{\pgfqpoint{2.100208in}{0.854183in}}%
\pgfpathlineto{\pgfqpoint{2.102653in}{0.919597in}}%
\pgfpathlineto{\pgfqpoint{2.105099in}{0.854183in}}%
\pgfpathlineto{\pgfqpoint{2.108074in}{0.919597in}}%
\pgfpathlineto{\pgfqpoint{2.110520in}{0.854183in}}%
\pgfpathlineto{\pgfqpoint{2.115493in}{0.788768in}}%
\pgfpathlineto{\pgfqpoint{2.117938in}{0.854183in}}%
\pgfpathlineto{\pgfqpoint{2.120384in}{0.788768in}}%
\pgfpathlineto{\pgfqpoint{2.122911in}{0.854183in}}%
\pgfpathlineto{\pgfqpoint{2.125357in}{0.788768in}}%
\pgfpathlineto{\pgfqpoint{2.139501in}{0.723354in}}%
\pgfpathlineto{\pgfqpoint{2.141946in}{0.788768in}}%
\pgfpathlineto{\pgfqpoint{2.144433in}{0.723354in}}%
\pgfpathlineto{\pgfqpoint{2.146919in}{0.788768in}}%
\pgfpathlineto{\pgfqpoint{2.149365in}{0.723354in}}%
\pgfpathlineto{\pgfqpoint{2.154664in}{0.647038in}}%
\pgfpathlineto{\pgfqpoint{2.157109in}{0.723354in}}%
\pgfpathlineto{\pgfqpoint{2.159555in}{0.647038in}}%
\pgfpathlineto{\pgfqpoint{2.162001in}{0.723354in}}%
\pgfpathlineto{\pgfqpoint{2.164446in}{0.647038in}}%
\pgfpathlineto{\pgfqpoint{2.174351in}{0.723354in}}%
\pgfpathlineto{\pgfqpoint{2.198196in}{1.192156in}}%
\pgfpathlineto{\pgfqpoint{2.210628in}{1.399301in}}%
\pgfpathlineto{\pgfqpoint{2.246212in}{1.879005in}}%
\pgfpathlineto{\pgfqpoint{2.260152in}{2.009833in}}%
\pgfpathlineto{\pgfqpoint{2.262679in}{1.944419in}}%
\pgfpathlineto{\pgfqpoint{2.267978in}{2.086150in}}%
\pgfpathlineto{\pgfqpoint{2.270505in}{2.009833in}}%
\pgfpathlineto{\pgfqpoint{2.272951in}{2.086150in}}%
\pgfpathlineto{\pgfqpoint{2.276864in}{2.151564in}}%
\pgfpathlineto{\pgfqpoint{2.279432in}{2.086150in}}%
\pgfpathlineto{\pgfqpoint{2.284772in}{2.216978in}}%
\pgfpathlineto{\pgfqpoint{2.287217in}{2.151564in}}%
\pgfpathlineto{\pgfqpoint{2.289663in}{2.216978in}}%
\pgfpathlineto{\pgfqpoint{2.295125in}{2.282393in}}%
\pgfpathlineto{\pgfqpoint{2.297570in}{2.216978in}}%
\pgfpathlineto{\pgfqpoint{2.300016in}{2.282393in}}%
\pgfpathlineto{\pgfqpoint{2.307027in}{2.358709in}}%
\pgfpathlineto{\pgfqpoint{2.309472in}{2.282393in}}%
\pgfpathlineto{\pgfqpoint{2.311918in}{2.358709in}}%
\pgfpathlineto{\pgfqpoint{2.319703in}{2.424123in}}%
\pgfpathlineto{\pgfqpoint{2.322149in}{2.358709in}}%
\pgfpathlineto{\pgfqpoint{2.324595in}{2.424123in}}%
\pgfpathlineto{\pgfqpoint{2.334785in}{2.489538in}}%
\pgfpathlineto{\pgfqpoint{2.337230in}{2.424123in}}%
\pgfpathlineto{\pgfqpoint{2.339676in}{2.489538in}}%
\pgfpathlineto{\pgfqpoint{2.342122in}{2.424123in}}%
\pgfpathlineto{\pgfqpoint{2.344567in}{2.489538in}}%
\pgfpathlineto{\pgfqpoint{2.353779in}{2.554952in}}%
\pgfpathlineto{\pgfqpoint{2.356225in}{2.489538in}}%
\pgfpathlineto{\pgfqpoint{2.358711in}{2.554952in}}%
\pgfpathlineto{\pgfqpoint{2.361442in}{2.489538in}}%
\pgfpathlineto{\pgfqpoint{2.363888in}{2.554952in}}%
\pgfpathlineto{\pgfqpoint{2.374404in}{2.631268in}}%
\pgfpathlineto{\pgfqpoint{2.376850in}{2.554952in}}%
\pgfpathlineto{\pgfqpoint{2.379295in}{2.631268in}}%
\pgfpathlineto{\pgfqpoint{2.381741in}{2.554952in}}%
\pgfpathlineto{\pgfqpoint{2.384187in}{2.631268in}}%
\pgfpathlineto{\pgfqpoint{2.387488in}{2.554952in}}%
\pgfpathlineto{\pgfqpoint{2.389934in}{2.631268in}}%
\pgfpathlineto{\pgfqpoint{2.400328in}{2.696683in}}%
\pgfpathlineto{\pgfqpoint{2.402774in}{2.631268in}}%
\pgfpathlineto{\pgfqpoint{2.405219in}{2.696683in}}%
\pgfpathlineto{\pgfqpoint{2.407746in}{2.631268in}}%
\pgfpathlineto{\pgfqpoint{2.410192in}{2.696683in}}%
\pgfpathlineto{\pgfqpoint{2.413167in}{2.631268in}}%
\pgfpathlineto{\pgfqpoint{2.415613in}{2.696683in}}%
\pgfpathlineto{\pgfqpoint{2.426252in}{2.762097in}}%
\pgfpathlineto{\pgfqpoint{2.428697in}{2.696683in}}%
\pgfpathlineto{\pgfqpoint{2.431184in}{2.762097in}}%
\pgfpathlineto{\pgfqpoint{2.433629in}{2.696683in}}%
\pgfpathlineto{\pgfqpoint{2.436116in}{2.762097in}}%
\pgfpathlineto{\pgfqpoint{2.438643in}{2.696683in}}%
\pgfpathlineto{\pgfqpoint{2.441088in}{2.762097in}}%
\pgfpathlineto{\pgfqpoint{2.456414in}{2.696683in}}%
\pgfpathlineto{\pgfqpoint{2.458860in}{2.631268in}}%
\pgfpathlineto{\pgfqpoint{2.461306in}{2.489538in}}%
\pgfpathlineto{\pgfqpoint{2.463751in}{2.424123in}}%
\pgfpathlineto{\pgfqpoint{2.466197in}{2.282393in}}%
\pgfpathlineto{\pgfqpoint{2.468643in}{2.216978in}}%
\pgfpathlineto{\pgfqpoint{2.473534in}{2.009833in}}%
\pgfpathlineto{\pgfqpoint{2.488575in}{1.606446in}}%
\pgfpathlineto{\pgfqpoint{2.510789in}{1.268473in}}%
\pgfpathlineto{\pgfqpoint{2.513520in}{1.333887in}}%
\pgfpathlineto{\pgfqpoint{2.518778in}{1.192156in}}%
\pgfpathlineto{\pgfqpoint{2.521631in}{1.268473in}}%
\pgfpathlineto{\pgfqpoint{2.524077in}{1.192156in}}%
\pgfpathlineto{\pgfqpoint{2.529335in}{1.126742in}}%
\pgfpathlineto{\pgfqpoint{2.531781in}{1.192156in}}%
\pgfpathlineto{\pgfqpoint{2.534226in}{1.126742in}}%
\pgfpathlineto{\pgfqpoint{2.538588in}{1.061328in}}%
\pgfpathlineto{\pgfqpoint{2.541074in}{1.126742in}}%
\pgfpathlineto{\pgfqpoint{2.543520in}{1.061328in}}%
\pgfpathlineto{\pgfqpoint{2.548656in}{0.995913in}}%
\pgfpathlineto{\pgfqpoint{2.551142in}{1.061328in}}%
\pgfpathlineto{\pgfqpoint{2.553588in}{0.995913in}}%
\pgfpathlineto{\pgfqpoint{2.560354in}{0.919597in}}%
\pgfpathlineto{\pgfqpoint{2.562800in}{0.995913in}}%
\pgfpathlineto{\pgfqpoint{2.565245in}{0.919597in}}%
\pgfpathlineto{\pgfqpoint{2.567732in}{0.995913in}}%
\pgfpathlineto{\pgfqpoint{2.570177in}{0.919597in}}%
\pgfpathlineto{\pgfqpoint{2.574009in}{0.854183in}}%
\pgfpathlineto{\pgfqpoint{2.576536in}{0.919597in}}%
\pgfpathlineto{\pgfqpoint{2.578982in}{0.854183in}}%
\pgfpathlineto{\pgfqpoint{2.586645in}{0.788768in}}%
\pgfpathlineto{\pgfqpoint{2.589090in}{0.854183in}}%
\pgfpathlineto{\pgfqpoint{2.592188in}{0.788768in}}%
\pgfpathlineto{\pgfqpoint{2.594634in}{0.854183in}}%
\pgfpathlineto{\pgfqpoint{2.597079in}{0.788768in}}%
\pgfpathlineto{\pgfqpoint{2.604824in}{0.723354in}}%
\pgfpathlineto{\pgfqpoint{2.607269in}{0.788768in}}%
\pgfpathlineto{\pgfqpoint{2.609797in}{0.723354in}}%
\pgfpathlineto{\pgfqpoint{2.612242in}{0.788768in}}%
\pgfpathlineto{\pgfqpoint{2.614729in}{0.723354in}}%
\pgfpathlineto{\pgfqpoint{2.617174in}{0.788768in}}%
\pgfpathlineto{\pgfqpoint{2.619620in}{0.723354in}}%
\pgfpathlineto{\pgfqpoint{2.622066in}{0.788768in}}%
\pgfpathlineto{\pgfqpoint{2.624511in}{0.723354in}}%
\pgfpathlineto{\pgfqpoint{2.635598in}{0.647038in}}%
\pgfpathlineto{\pgfqpoint{2.638044in}{0.723354in}}%
\pgfpathlineto{\pgfqpoint{2.640489in}{0.647038in}}%
\pgfpathlineto{\pgfqpoint{2.643017in}{0.723354in}}%
\pgfpathlineto{\pgfqpoint{2.645462in}{0.647038in}}%
\pgfpathlineto{\pgfqpoint{2.648560in}{0.723354in}}%
\pgfpathlineto{\pgfqpoint{2.656468in}{0.854183in}}%
\pgfpathlineto{\pgfqpoint{2.660218in}{0.919597in}}%
\pgfpathlineto{\pgfqpoint{2.666780in}{1.061328in}}%
\pgfpathlineto{\pgfqpoint{2.686671in}{1.399301in}}%
\pgfpathlineto{\pgfqpoint{2.705421in}{1.671860in}}%
\pgfpathlineto{\pgfqpoint{2.708193in}{1.606446in}}%
\pgfpathlineto{\pgfqpoint{2.713084in}{1.737274in}}%
\pgfpathlineto{\pgfqpoint{2.717527in}{1.813591in}}%
\pgfpathlineto{\pgfqpoint{2.723030in}{1.879005in}}%
\pgfpathlineto{\pgfqpoint{2.725557in}{1.813591in}}%
\pgfpathlineto{\pgfqpoint{2.730530in}{1.944419in}}%
\pgfpathlineto{\pgfqpoint{2.737703in}{2.009833in}}%
\pgfpathlineto{\pgfqpoint{2.740231in}{1.944419in}}%
\pgfpathlineto{\pgfqpoint{2.745733in}{2.086150in}}%
\pgfpathlineto{\pgfqpoint{2.748301in}{2.009833in}}%
\pgfpathlineto{\pgfqpoint{2.750747in}{2.086150in}}%
\pgfpathlineto{\pgfqpoint{2.754619in}{2.151564in}}%
\pgfpathlineto{\pgfqpoint{2.757146in}{2.086150in}}%
\pgfpathlineto{\pgfqpoint{2.759592in}{2.151564in}}%
\pgfpathlineto{\pgfqpoint{2.763342in}{2.216978in}}%
\pgfpathlineto{\pgfqpoint{2.765787in}{2.151564in}}%
\pgfpathlineto{\pgfqpoint{2.768233in}{2.216978in}}%
\pgfpathlineto{\pgfqpoint{2.773328in}{2.282393in}}%
\pgfpathlineto{\pgfqpoint{2.775774in}{2.216978in}}%
\pgfpathlineto{\pgfqpoint{2.778219in}{2.282393in}}%
\pgfpathlineto{\pgfqpoint{2.786005in}{2.358709in}}%
\pgfpathlineto{\pgfqpoint{2.788450in}{2.282393in}}%
\pgfpathlineto{\pgfqpoint{2.790896in}{2.358709in}}%
\pgfpathlineto{\pgfqpoint{2.797173in}{2.424123in}}%
\pgfpathlineto{\pgfqpoint{2.799619in}{2.358709in}}%
\pgfpathlineto{\pgfqpoint{2.802064in}{2.424123in}}%
\pgfpathlineto{\pgfqpoint{2.804592in}{2.358709in}}%
\pgfpathlineto{\pgfqpoint{2.807037in}{2.424123in}}%
\pgfpathlineto{\pgfqpoint{2.815678in}{2.489538in}}%
\pgfpathlineto{\pgfqpoint{2.818124in}{2.424123in}}%
\pgfpathlineto{\pgfqpoint{2.820570in}{2.489538in}}%
\pgfpathlineto{\pgfqpoint{2.832920in}{2.554952in}}%
\pgfpathlineto{\pgfqpoint{2.835366in}{2.489538in}}%
\pgfpathlineto{\pgfqpoint{2.837811in}{2.554952in}}%
\pgfpathlineto{\pgfqpoint{2.840379in}{2.489538in}}%
\pgfpathlineto{\pgfqpoint{2.842825in}{2.554952in}}%
\pgfpathlineto{\pgfqpoint{2.852322in}{2.631268in}}%
\pgfpathlineto{\pgfqpoint{2.854768in}{2.554952in}}%
\pgfpathlineto{\pgfqpoint{2.857254in}{2.631268in}}%
\pgfpathlineto{\pgfqpoint{2.859700in}{2.554952in}}%
\pgfpathlineto{\pgfqpoint{2.862146in}{2.631268in}}%
\pgfpathlineto{\pgfqpoint{2.878898in}{2.696683in}}%
\pgfpathlineto{\pgfqpoint{2.881344in}{2.631268in}}%
\pgfpathlineto{\pgfqpoint{2.883871in}{2.696683in}}%
\pgfpathlineto{\pgfqpoint{2.886317in}{2.631268in}}%
\pgfpathlineto{\pgfqpoint{2.888762in}{2.696683in}}%
\pgfpathlineto{\pgfqpoint{2.891371in}{2.631268in}}%
\pgfpathlineto{\pgfqpoint{2.893817in}{2.696683in}}%
\pgfpathlineto{\pgfqpoint{2.904659in}{2.762097in}}%
\pgfpathlineto{\pgfqpoint{2.907104in}{2.696683in}}%
\pgfpathlineto{\pgfqpoint{2.909632in}{2.762097in}}%
\pgfpathlineto{\pgfqpoint{2.912077in}{2.696683in}}%
\pgfpathlineto{\pgfqpoint{2.914523in}{2.762097in}}%
\pgfpathlineto{\pgfqpoint{2.917009in}{2.696683in}}%
\pgfpathlineto{\pgfqpoint{2.919455in}{2.762097in}}%
\pgfpathlineto{\pgfqpoint{2.923001in}{2.696683in}}%
\pgfpathlineto{\pgfqpoint{2.925447in}{2.762097in}}%
\pgfpathlineto{\pgfqpoint{2.936248in}{2.827511in}}%
\pgfpathlineto{\pgfqpoint{2.938694in}{2.762097in}}%
\pgfpathlineto{\pgfqpoint{2.941140in}{2.827511in}}%
\pgfpathlineto{\pgfqpoint{2.953368in}{2.216978in}}%
\pgfpathlineto{\pgfqpoint{2.955813in}{2.151564in}}%
\pgfpathlineto{\pgfqpoint{2.958259in}{2.009833in}}%
\pgfpathlineto{\pgfqpoint{2.975745in}{1.541032in}}%
\pgfpathlineto{\pgfqpoint{2.984550in}{1.399301in}}%
\pgfpathlineto{\pgfqpoint{2.989400in}{1.333887in}}%
\pgfpathlineto{\pgfqpoint{2.991968in}{1.399301in}}%
\pgfpathlineto{\pgfqpoint{2.996859in}{1.268473in}}%
\pgfpathlineto{\pgfqpoint{3.002077in}{1.192156in}}%
\pgfpathlineto{\pgfqpoint{3.004563in}{1.268473in}}%
\pgfpathlineto{\pgfqpoint{3.007009in}{1.192156in}}%
\pgfpathlineto{\pgfqpoint{3.010514in}{1.126742in}}%
\pgfpathlineto{\pgfqpoint{3.012960in}{1.192156in}}%
\pgfpathlineto{\pgfqpoint{3.015405in}{1.126742in}}%
\pgfpathlineto{\pgfqpoint{3.020419in}{1.061328in}}%
\pgfpathlineto{\pgfqpoint{3.022865in}{1.126742in}}%
\pgfpathlineto{\pgfqpoint{3.025310in}{1.061328in}}%
\pgfpathlineto{\pgfqpoint{3.027756in}{1.126742in}}%
\pgfpathlineto{\pgfqpoint{3.030202in}{1.061328in}}%
\pgfpathlineto{\pgfqpoint{3.040840in}{0.995913in}}%
\pgfpathlineto{\pgfqpoint{3.043367in}{1.061328in}}%
\pgfpathlineto{\pgfqpoint{3.045813in}{0.995913in}}%
\pgfpathlineto{\pgfqpoint{3.051927in}{0.919597in}}%
\pgfpathlineto{\pgfqpoint{3.054373in}{0.995913in}}%
\pgfpathlineto{\pgfqpoint{3.056818in}{0.919597in}}%
\pgfpathlineto{\pgfqpoint{3.065256in}{0.854183in}}%
\pgfpathlineto{\pgfqpoint{3.067701in}{0.919597in}}%
\pgfpathlineto{\pgfqpoint{3.070147in}{0.854183in}}%
\pgfpathlineto{\pgfqpoint{3.072633in}{0.919597in}}%
\pgfpathlineto{\pgfqpoint{3.075079in}{0.854183in}}%
\pgfpathlineto{\pgfqpoint{3.090323in}{0.788768in}}%
\pgfpathlineto{\pgfqpoint{3.092769in}{0.854183in}}%
\pgfpathlineto{\pgfqpoint{3.095215in}{0.788768in}}%
\pgfpathlineto{\pgfqpoint{3.097742in}{0.854183in}}%
\pgfpathlineto{\pgfqpoint{3.100187in}{0.788768in}}%
\pgfpathlineto{\pgfqpoint{3.105323in}{0.854183in}}%
\pgfpathlineto{\pgfqpoint{3.107769in}{0.788768in}}%
\pgfpathlineto{\pgfqpoint{3.114331in}{0.723354in}}%
\pgfpathlineto{\pgfqpoint{3.116777in}{0.788768in}}%
\pgfpathlineto{\pgfqpoint{3.119345in}{0.723354in}}%
\pgfpathlineto{\pgfqpoint{3.121831in}{0.788768in}}%
\pgfpathlineto{\pgfqpoint{3.124277in}{0.723354in}}%
\pgfpathlineto{\pgfqpoint{3.126763in}{0.788768in}}%
\pgfpathlineto{\pgfqpoint{3.129209in}{0.723354in}}%
\pgfpathlineto{\pgfqpoint{3.131655in}{0.788768in}}%
\pgfpathlineto{\pgfqpoint{3.134100in}{0.723354in}}%
\pgfpathlineto{\pgfqpoint{3.149467in}{0.647038in}}%
\pgfpathlineto{\pgfqpoint{3.151913in}{0.723354in}}%
\pgfpathlineto{\pgfqpoint{3.154399in}{0.647038in}}%
\pgfpathlineto{\pgfqpoint{3.156845in}{0.723354in}}%
\pgfpathlineto{\pgfqpoint{3.187537in}{1.333887in}}%
\pgfpathlineto{\pgfqpoint{3.196668in}{1.464715in}}%
\pgfpathlineto{\pgfqpoint{3.200988in}{1.541032in}}%
\pgfpathlineto{\pgfqpoint{3.206776in}{1.606446in}}%
\pgfpathlineto{\pgfqpoint{3.210608in}{1.671860in}}%
\pgfpathlineto{\pgfqpoint{3.213135in}{1.606446in}}%
\pgfpathlineto{\pgfqpoint{3.218026in}{1.737274in}}%
\pgfpathlineto{\pgfqpoint{3.222225in}{1.813591in}}%
\pgfpathlineto{\pgfqpoint{3.224670in}{1.737274in}}%
\pgfpathlineto{\pgfqpoint{3.229602in}{1.879005in}}%
\pgfpathlineto{\pgfqpoint{3.242972in}{2.009833in}}%
\pgfpathlineto{\pgfqpoint{3.245499in}{1.944419in}}%
\pgfpathlineto{\pgfqpoint{3.250553in}{2.086150in}}%
\pgfpathlineto{\pgfqpoint{3.252999in}{2.009833in}}%
\pgfpathlineto{\pgfqpoint{3.258542in}{2.151564in}}%
\pgfpathlineto{\pgfqpoint{3.261029in}{2.086150in}}%
\pgfpathlineto{\pgfqpoint{3.263474in}{2.151564in}}%
\pgfpathlineto{\pgfqpoint{3.267224in}{2.216978in}}%
\pgfpathlineto{\pgfqpoint{3.269711in}{2.151564in}}%
\pgfpathlineto{\pgfqpoint{3.272156in}{2.216978in}}%
\pgfpathlineto{\pgfqpoint{3.276885in}{2.282393in}}%
\pgfpathlineto{\pgfqpoint{3.279330in}{2.216978in}}%
\pgfpathlineto{\pgfqpoint{3.281776in}{2.282393in}}%
\pgfpathlineto{\pgfqpoint{3.290010in}{2.358709in}}%
\pgfpathlineto{\pgfqpoint{3.292455in}{2.282393in}}%
\pgfpathlineto{\pgfqpoint{3.294901in}{2.358709in}}%
\pgfpathlineto{\pgfqpoint{3.301993in}{2.424123in}}%
\pgfpathlineto{\pgfqpoint{3.304439in}{2.358709in}}%
\pgfpathlineto{\pgfqpoint{3.306885in}{2.424123in}}%
\pgfpathlineto{\pgfqpoint{3.317197in}{2.489538in}}%
\pgfpathlineto{\pgfqpoint{3.319643in}{2.424123in}}%
\pgfpathlineto{\pgfqpoint{3.322088in}{2.489538in}}%
\pgfpathlineto{\pgfqpoint{3.324656in}{2.424123in}}%
\pgfpathlineto{\pgfqpoint{3.327102in}{2.489538in}}%
\pgfpathlineto{\pgfqpoint{3.334765in}{2.554952in}}%
\pgfpathlineto{\pgfqpoint{3.337210in}{2.489538in}}%
\pgfpathlineto{\pgfqpoint{3.339656in}{2.554952in}}%
\pgfpathlineto{\pgfqpoint{3.342102in}{2.489538in}}%
\pgfpathlineto{\pgfqpoint{3.344547in}{2.554952in}}%
\pgfpathlineto{\pgfqpoint{3.356735in}{2.631268in}}%
\pgfpathlineto{\pgfqpoint{3.359180in}{2.554952in}}%
\pgfpathlineto{\pgfqpoint{3.361626in}{2.631268in}}%
\pgfpathlineto{\pgfqpoint{3.364235in}{2.554952in}}%
\pgfpathlineto{\pgfqpoint{3.366680in}{2.631268in}}%
\pgfpathlineto{\pgfqpoint{3.377889in}{2.696683in}}%
\pgfpathlineto{\pgfqpoint{3.380335in}{2.631268in}}%
\pgfpathlineto{\pgfqpoint{3.382781in}{2.696683in}}%
\pgfpathlineto{\pgfqpoint{3.385267in}{2.631268in}}%
\pgfpathlineto{\pgfqpoint{3.387713in}{2.696683in}}%
\pgfpathlineto{\pgfqpoint{3.390729in}{2.631268in}}%
\pgfpathlineto{\pgfqpoint{3.393175in}{2.696683in}}%
\pgfpathlineto{\pgfqpoint{3.403813in}{2.762097in}}%
\pgfpathlineto{\pgfqpoint{3.406259in}{2.696683in}}%
\pgfpathlineto{\pgfqpoint{3.408786in}{2.762097in}}%
\pgfpathlineto{\pgfqpoint{3.411232in}{2.696683in}}%
\pgfpathlineto{\pgfqpoint{3.413677in}{2.762097in}}%
\pgfpathlineto{\pgfqpoint{3.416204in}{2.696683in}}%
\pgfpathlineto{\pgfqpoint{3.418650in}{2.762097in}}%
\pgfpathlineto{\pgfqpoint{3.428799in}{2.696683in}}%
\pgfpathlineto{\pgfqpoint{3.433691in}{2.554952in}}%
\pgfpathlineto{\pgfqpoint{3.441028in}{2.216978in}}%
\pgfpathlineto{\pgfqpoint{3.443473in}{2.151564in}}%
\pgfpathlineto{\pgfqpoint{3.445919in}{2.009833in}}%
\pgfpathlineto{\pgfqpoint{3.461041in}{1.606446in}}%
\pgfpathlineto{\pgfqpoint{3.480362in}{1.333887in}}%
\pgfpathlineto{\pgfqpoint{3.482807in}{1.399301in}}%
\pgfpathlineto{\pgfqpoint{3.487699in}{1.268473in}}%
\pgfpathlineto{\pgfqpoint{3.492182in}{1.192156in}}%
\pgfpathlineto{\pgfqpoint{3.494669in}{1.268473in}}%
\pgfpathlineto{\pgfqpoint{3.497114in}{1.192156in}}%
\pgfpathlineto{\pgfqpoint{3.500905in}{1.126742in}}%
\pgfpathlineto{\pgfqpoint{3.503391in}{1.192156in}}%
\pgfpathlineto{\pgfqpoint{3.508650in}{1.061328in}}%
\pgfpathlineto{\pgfqpoint{3.511095in}{1.126742in}}%
\pgfpathlineto{\pgfqpoint{3.516516in}{0.995913in}}%
\pgfpathlineto{\pgfqpoint{3.518962in}{1.061328in}}%
\pgfpathlineto{\pgfqpoint{3.521408in}{0.995913in}}%
\pgfpathlineto{\pgfqpoint{3.531149in}{0.919597in}}%
\pgfpathlineto{\pgfqpoint{3.533677in}{0.995913in}}%
\pgfpathlineto{\pgfqpoint{3.536122in}{0.919597in}}%
\pgfpathlineto{\pgfqpoint{3.542236in}{0.854183in}}%
\pgfpathlineto{\pgfqpoint{3.544682in}{0.919597in}}%
\pgfpathlineto{\pgfqpoint{3.547128in}{0.854183in}}%
\pgfpathlineto{\pgfqpoint{3.559682in}{0.788768in}}%
\pgfpathlineto{\pgfqpoint{3.562127in}{0.854183in}}%
\pgfpathlineto{\pgfqpoint{3.564573in}{0.788768in}}%
\pgfpathlineto{\pgfqpoint{3.573540in}{0.723354in}}%
\pgfpathlineto{\pgfqpoint{3.575986in}{0.788768in}}%
\pgfpathlineto{\pgfqpoint{3.578432in}{0.723354in}}%
\pgfpathlineto{\pgfqpoint{3.580959in}{0.788768in}}%
\pgfpathlineto{\pgfqpoint{3.583404in}{0.723354in}}%
\pgfpathlineto{\pgfqpoint{3.590415in}{0.647038in}}%
\pgfpathlineto{\pgfqpoint{3.592942in}{0.723354in}}%
\pgfpathlineto{\pgfqpoint{3.595388in}{0.647038in}}%
\pgfpathlineto{\pgfqpoint{3.597834in}{0.723354in}}%
\pgfpathlineto{\pgfqpoint{3.600279in}{0.647038in}}%
\pgfpathlineto{\pgfqpoint{3.615931in}{0.723354in}}%
\pgfpathlineto{\pgfqpoint{3.630564in}{0.995913in}}%
\pgfpathlineto{\pgfqpoint{3.645279in}{1.268473in}}%
\pgfpathlineto{\pgfqpoint{3.653187in}{1.399301in}}%
\pgfpathlineto{\pgfqpoint{3.658608in}{1.464715in}}%
\pgfpathlineto{\pgfqpoint{3.673404in}{1.671860in}}%
\pgfpathlineto{\pgfqpoint{3.679273in}{1.737274in}}%
\pgfpathlineto{\pgfqpoint{3.684695in}{1.813591in}}%
\pgfpathlineto{\pgfqpoint{3.687426in}{1.737274in}}%
\pgfpathlineto{\pgfqpoint{3.692317in}{1.879005in}}%
\pgfpathlineto{\pgfqpoint{3.697127in}{1.944419in}}%
\pgfpathlineto{\pgfqpoint{3.699654in}{1.879005in}}%
\pgfpathlineto{\pgfqpoint{3.705279in}{2.009833in}}%
\pgfpathlineto{\pgfqpoint{3.707806in}{1.944419in}}%
\pgfpathlineto{\pgfqpoint{3.712697in}{2.086150in}}%
\pgfpathlineto{\pgfqpoint{3.715183in}{2.009833in}}%
\pgfpathlineto{\pgfqpoint{3.720768in}{2.151564in}}%
\pgfpathlineto{\pgfqpoint{3.723213in}{2.086150in}}%
\pgfpathlineto{\pgfqpoint{3.725659in}{2.151564in}}%
\pgfpathlineto{\pgfqpoint{3.730102in}{2.216978in}}%
\pgfpathlineto{\pgfqpoint{3.732548in}{2.151564in}}%
\pgfpathlineto{\pgfqpoint{3.734993in}{2.216978in}}%
\pgfpathlineto{\pgfqpoint{3.740496in}{2.282393in}}%
\pgfpathlineto{\pgfqpoint{3.742941in}{2.216978in}}%
\pgfpathlineto{\pgfqpoint{3.745387in}{2.282393in}}%
\pgfpathlineto{\pgfqpoint{3.752031in}{2.358709in}}%
\pgfpathlineto{\pgfqpoint{3.754477in}{2.282393in}}%
\pgfpathlineto{\pgfqpoint{3.756922in}{2.358709in}}%
\pgfpathlineto{\pgfqpoint{3.763852in}{2.424123in}}%
\pgfpathlineto{\pgfqpoint{3.766297in}{2.358709in}}%
\pgfpathlineto{\pgfqpoint{3.768743in}{2.424123in}}%
\pgfpathlineto{\pgfqpoint{3.779096in}{2.489538in}}%
\pgfpathlineto{\pgfqpoint{3.781542in}{2.424123in}}%
\pgfpathlineto{\pgfqpoint{3.783987in}{2.489538in}}%
\pgfpathlineto{\pgfqpoint{3.786433in}{2.424123in}}%
\pgfpathlineto{\pgfqpoint{3.788879in}{2.489538in}}%
\pgfpathlineto{\pgfqpoint{3.797398in}{2.554952in}}%
\pgfpathlineto{\pgfqpoint{3.799843in}{2.489538in}}%
\pgfpathlineto{\pgfqpoint{3.802330in}{2.554952in}}%
\pgfpathlineto{\pgfqpoint{3.804775in}{2.489538in}}%
\pgfpathlineto{\pgfqpoint{3.807221in}{2.554952in}}%
\pgfpathlineto{\pgfqpoint{3.818552in}{2.631268in}}%
\pgfpathlineto{\pgfqpoint{3.820998in}{2.554952in}}%
\pgfpathlineto{\pgfqpoint{3.823444in}{2.631268in}}%
\pgfpathlineto{\pgfqpoint{3.825930in}{2.554952in}}%
\pgfpathlineto{\pgfqpoint{3.828376in}{2.631268in}}%
\pgfpathlineto{\pgfqpoint{3.841052in}{2.696683in}}%
\pgfpathlineto{\pgfqpoint{3.843498in}{2.631268in}}%
\pgfpathlineto{\pgfqpoint{3.845984in}{2.696683in}}%
\pgfpathlineto{\pgfqpoint{3.848430in}{2.631268in}}%
\pgfpathlineto{\pgfqpoint{3.850916in}{2.696683in}}%
\pgfpathlineto{\pgfqpoint{3.855074in}{2.631268in}}%
\pgfpathlineto{\pgfqpoint{3.857520in}{2.696683in}}%
\pgfpathlineto{\pgfqpoint{3.866976in}{2.762097in}}%
\pgfpathlineto{\pgfqpoint{3.869422in}{2.696683in}}%
\pgfpathlineto{\pgfqpoint{3.871908in}{2.762097in}}%
\pgfpathlineto{\pgfqpoint{3.874354in}{2.696683in}}%
\pgfpathlineto{\pgfqpoint{3.876799in}{2.762097in}}%
\pgfpathlineto{\pgfqpoint{3.884136in}{2.424123in}}%
\pgfpathlineto{\pgfqpoint{3.889027in}{2.282393in}}%
\pgfpathlineto{\pgfqpoint{3.891473in}{2.151564in}}%
\pgfpathlineto{\pgfqpoint{3.901256in}{1.879005in}}%
\pgfpathlineto{\pgfqpoint{3.903701in}{1.737274in}}%
\pgfpathlineto{\pgfqpoint{3.917397in}{1.464715in}}%
\pgfpathlineto{\pgfqpoint{3.927709in}{1.333887in}}%
\pgfpathlineto{\pgfqpoint{3.934027in}{1.268473in}}%
\pgfpathlineto{\pgfqpoint{3.936717in}{1.333887in}}%
\pgfpathlineto{\pgfqpoint{3.941609in}{1.192156in}}%
\pgfpathlineto{\pgfqpoint{3.946908in}{1.126742in}}%
\pgfpathlineto{\pgfqpoint{3.949435in}{1.192156in}}%
\pgfpathlineto{\pgfqpoint{3.951880in}{1.126742in}}%
\pgfpathlineto{\pgfqpoint{3.955956in}{1.061328in}}%
\pgfpathlineto{\pgfqpoint{3.958402in}{1.126742in}}%
\pgfpathlineto{\pgfqpoint{3.963905in}{0.995913in}}%
\pgfpathlineto{\pgfqpoint{3.966350in}{1.061328in}}%
\pgfpathlineto{\pgfqpoint{3.968796in}{0.995913in}}%
\pgfpathlineto{\pgfqpoint{3.975725in}{0.919597in}}%
\pgfpathlineto{\pgfqpoint{3.978171in}{0.995913in}}%
\pgfpathlineto{\pgfqpoint{3.980617in}{0.919597in}}%
\pgfpathlineto{\pgfqpoint{3.983103in}{0.995913in}}%
\pgfpathlineto{\pgfqpoint{3.985549in}{0.919597in}}%
\pgfpathlineto{\pgfqpoint{3.993252in}{0.854183in}}%
\pgfpathlineto{\pgfqpoint{3.995698in}{0.919597in}}%
\pgfpathlineto{\pgfqpoint{3.999000in}{0.854183in}}%
\pgfpathlineto{\pgfqpoint{4.001445in}{0.919597in}}%
\pgfpathlineto{\pgfqpoint{4.003891in}{0.854183in}}%
\pgfpathlineto{\pgfqpoint{4.007111in}{0.919597in}}%
\pgfpathlineto{\pgfqpoint{4.009557in}{0.854183in}}%
\pgfpathlineto{\pgfqpoint{4.014448in}{0.788768in}}%
\pgfpathlineto{\pgfqpoint{4.016894in}{0.854183in}}%
\pgfpathlineto{\pgfqpoint{4.019339in}{0.788768in}}%
\pgfpathlineto{\pgfqpoint{4.021907in}{0.854183in}}%
\pgfpathlineto{\pgfqpoint{4.024353in}{0.788768in}}%
\pgfpathlineto{\pgfqpoint{4.034543in}{0.723354in}}%
\pgfpathlineto{\pgfqpoint{4.036988in}{0.788768in}}%
\pgfpathlineto{\pgfqpoint{4.039434in}{0.723354in}}%
\pgfpathlineto{\pgfqpoint{4.041880in}{0.788768in}}%
\pgfpathlineto{\pgfqpoint{4.044325in}{0.723354in}}%
\pgfpathlineto{\pgfqpoint{4.051255in}{0.647038in}}%
\pgfpathlineto{\pgfqpoint{4.053700in}{0.723354in}}%
\pgfpathlineto{\pgfqpoint{4.056146in}{0.647038in}}%
\pgfpathlineto{\pgfqpoint{4.058632in}{0.723354in}}%
\pgfpathlineto{\pgfqpoint{4.061078in}{0.647038in}}%
\pgfpathlineto{\pgfqpoint{4.063564in}{0.723354in}}%
\pgfpathlineto{\pgfqpoint{4.066010in}{0.647038in}}%
\pgfpathlineto{\pgfqpoint{4.070983in}{0.788768in}}%
\pgfpathlineto{\pgfqpoint{4.085697in}{1.061328in}}%
\pgfpathlineto{\pgfqpoint{4.101064in}{1.333887in}}%
\pgfpathlineto{\pgfqpoint{4.125724in}{1.671860in}}%
\pgfpathlineto{\pgfqpoint{4.128333in}{1.606446in}}%
\pgfpathlineto{\pgfqpoint{4.133224in}{1.737274in}}%
\pgfpathlineto{\pgfqpoint{4.138116in}{1.813591in}}%
\pgfpathlineto{\pgfqpoint{4.143781in}{1.879005in}}%
\pgfpathlineto{\pgfqpoint{4.158496in}{2.009833in}}%
\pgfpathlineto{\pgfqpoint{4.160982in}{1.944419in}}%
\pgfpathlineto{\pgfqpoint{4.167259in}{2.086150in}}%
\pgfpathlineto{\pgfqpoint{4.169705in}{2.009833in}}%
\pgfpathlineto{\pgfqpoint{4.172151in}{2.086150in}}%
\pgfpathlineto{\pgfqpoint{4.176431in}{2.151564in}}%
\pgfpathlineto{\pgfqpoint{4.178876in}{2.086150in}}%
\pgfpathlineto{\pgfqpoint{4.181322in}{2.151564in}}%
\pgfpathlineto{\pgfqpoint{4.186050in}{2.216978in}}%
\pgfpathlineto{\pgfqpoint{4.188496in}{2.151564in}}%
\pgfpathlineto{\pgfqpoint{4.190941in}{2.216978in}}%
\pgfpathlineto{\pgfqpoint{4.197545in}{2.282393in}}%
\pgfpathlineto{\pgfqpoint{4.199990in}{2.216978in}}%
\pgfpathlineto{\pgfqpoint{4.202436in}{2.282393in}}%
\pgfpathlineto{\pgfqpoint{4.210384in}{2.358709in}}%
\pgfpathlineto{\pgfqpoint{4.212870in}{2.282393in}}%
\pgfpathlineto{\pgfqpoint{4.215316in}{2.358709in}}%
\pgfpathlineto{\pgfqpoint{4.224039in}{2.424123in}}%
\pgfpathlineto{\pgfqpoint{4.226485in}{2.358709in}}%
\pgfpathlineto{\pgfqpoint{4.228930in}{2.424123in}}%
\pgfpathlineto{\pgfqpoint{4.241077in}{2.489538in}}%
\pgfpathlineto{\pgfqpoint{4.243522in}{2.424123in}}%
\pgfpathlineto{\pgfqpoint{4.246009in}{2.489538in}}%
\pgfpathlineto{\pgfqpoint{4.248577in}{2.424123in}}%
\pgfpathlineto{\pgfqpoint{4.251022in}{2.489538in}}%
\pgfpathlineto{\pgfqpoint{4.259134in}{2.554952in}}%
\pgfpathlineto{\pgfqpoint{4.261579in}{2.489538in}}%
\pgfpathlineto{\pgfqpoint{4.264188in}{2.554952in}}%
\pgfpathlineto{\pgfqpoint{4.266634in}{2.489538in}}%
\pgfpathlineto{\pgfqpoint{4.269079in}{2.554952in}}%
\pgfpathlineto{\pgfqpoint{4.282327in}{2.631268in}}%
\pgfpathlineto{\pgfqpoint{4.284772in}{2.554952in}}%
\pgfpathlineto{\pgfqpoint{4.287218in}{2.631268in}}%
\pgfpathlineto{\pgfqpoint{4.289704in}{2.554952in}}%
\pgfpathlineto{\pgfqpoint{4.292150in}{2.631268in}}%
\pgfpathlineto{\pgfqpoint{4.295288in}{2.554952in}}%
\pgfpathlineto{\pgfqpoint{4.297734in}{2.631268in}}%
\pgfpathlineto{\pgfqpoint{4.311063in}{2.696683in}}%
\pgfpathlineto{\pgfqpoint{4.313508in}{2.631268in}}%
\pgfpathlineto{\pgfqpoint{4.315995in}{2.696683in}}%
\pgfpathlineto{\pgfqpoint{4.318440in}{2.631268in}}%
\pgfpathlineto{\pgfqpoint{4.320886in}{2.696683in}}%
\pgfpathlineto{\pgfqpoint{4.323943in}{2.631268in}}%
\pgfpathlineto{\pgfqpoint{4.326389in}{2.696683in}}%
\pgfpathlineto{\pgfqpoint{4.326389in}{2.696683in}}%
\pgfusepath{stroke}%
\end{pgfscope}%
\begin{pgfscope}%
\pgfsetrectcap%
\pgfsetmiterjoin%
\pgfsetlinewidth{0.803000pt}%
\definecolor{currentstroke}{rgb}{0.000000,0.000000,0.000000}%
\pgfsetstrokecolor{currentstroke}%
\pgfsetdash{}{0pt}%
\pgfpathmoveto{\pgfqpoint{0.633813in}{0.538014in}}%
\pgfpathlineto{\pgfqpoint{0.633813in}{2.936535in}}%
\pgfusepath{stroke}%
\end{pgfscope}%
\begin{pgfscope}%
\pgfsetrectcap%
\pgfsetmiterjoin%
\pgfsetlinewidth{0.803000pt}%
\definecolor{currentstroke}{rgb}{0.000000,0.000000,0.000000}%
\pgfsetstrokecolor{currentstroke}%
\pgfsetdash{}{0pt}%
\pgfpathmoveto{\pgfqpoint{4.507477in}{0.538014in}}%
\pgfpathlineto{\pgfqpoint{4.507477in}{2.936535in}}%
\pgfusepath{stroke}%
\end{pgfscope}%
\begin{pgfscope}%
\pgfsetrectcap%
\pgfsetmiterjoin%
\pgfsetlinewidth{0.803000pt}%
\definecolor{currentstroke}{rgb}{0.000000,0.000000,0.000000}%
\pgfsetstrokecolor{currentstroke}%
\pgfsetdash{}{0pt}%
\pgfpathmoveto{\pgfqpoint{0.633813in}{0.538014in}}%
\pgfpathlineto{\pgfqpoint{4.507477in}{0.538014in}}%
\pgfusepath{stroke}%
\end{pgfscope}%
\begin{pgfscope}%
\pgfsetrectcap%
\pgfsetmiterjoin%
\pgfsetlinewidth{0.803000pt}%
\definecolor{currentstroke}{rgb}{0.000000,0.000000,0.000000}%
\pgfsetstrokecolor{currentstroke}%
\pgfsetdash{}{0pt}%
\pgfpathmoveto{\pgfqpoint{0.633813in}{2.936535in}}%
\pgfpathlineto{\pgfqpoint{4.507477in}{2.936535in}}%
\pgfusepath{stroke}%
\end{pgfscope}%
\begin{pgfscope}%
\pgfsetbuttcap%
\pgfsetmiterjoin%
\definecolor{currentfill}{rgb}{1.000000,1.000000,1.000000}%
\pgfsetfillcolor{currentfill}%
\pgfsetfillopacity{0.800000}%
\pgfsetlinewidth{1.003750pt}%
\definecolor{currentstroke}{rgb}{0.800000,0.800000,0.800000}%
\pgfsetstrokecolor{currentstroke}%
\pgfsetstrokeopacity{0.800000}%
\pgfsetdash{}{0pt}%
\pgfpathmoveto{\pgfqpoint{0.711591in}{2.536313in}}%
\pgfpathlineto{\pgfqpoint{2.211146in}{2.536313in}}%
\pgfpathquadraticcurveto{\pgfqpoint{2.233369in}{2.536313in}}{\pgfqpoint{2.233369in}{2.558535in}}%
\pgfpathlineto{\pgfqpoint{2.233369in}{2.858757in}}%
\pgfpathquadraticcurveto{\pgfqpoint{2.233369in}{2.880979in}}{\pgfqpoint{2.211146in}{2.880979in}}%
\pgfpathlineto{\pgfqpoint{0.711591in}{2.880979in}}%
\pgfpathquadraticcurveto{\pgfqpoint{0.689369in}{2.880979in}}{\pgfqpoint{0.689369in}{2.858757in}}%
\pgfpathlineto{\pgfqpoint{0.689369in}{2.558535in}}%
\pgfpathquadraticcurveto{\pgfqpoint{0.689369in}{2.536313in}}{\pgfqpoint{0.711591in}{2.536313in}}%
\pgfpathlineto{\pgfqpoint{0.711591in}{2.536313in}}%
\pgfpathclose%
\pgfusepath{stroke,fill}%
\end{pgfscope}%
\begin{pgfscope}%
\pgfsetrectcap%
\pgfsetroundjoin%
\pgfsetlinewidth{0.501875pt}%
\definecolor{currentstroke}{rgb}{0.121569,0.466667,0.705882}%
\pgfsetstrokecolor{currentstroke}%
\pgfsetstrokeopacity{0.700000}%
\pgfsetdash{}{0pt}%
\pgfpathmoveto{\pgfqpoint{0.733813in}{2.797646in}}%
\pgfpathlineto{\pgfqpoint{0.844924in}{2.797646in}}%
\pgfpathlineto{\pgfqpoint{0.956035in}{2.797646in}}%
\pgfusepath{stroke}%
\end{pgfscope}%
\begin{pgfscope}%
\definecolor{textcolor}{rgb}{0.000000,0.000000,0.000000}%
\pgfsetstrokecolor{textcolor}%
\pgfsetfillcolor{textcolor}%
\pgftext[x=1.044924in,y=2.758757in,left,base]{\color{textcolor}\rmfamily\fontsize{8.000000}{9.600000}\selectfont DUT vs KS34470A}%
\end{pgfscope}%
\begin{pgfscope}%
\pgfsetrectcap%
\pgfsetroundjoin%
\pgfsetlinewidth{0.501875pt}%
\definecolor{currentstroke}{rgb}{0.698039,0.133333,0.133333}%
\pgfsetstrokecolor{currentstroke}%
\pgfsetstrokeopacity{0.700000}%
\pgfsetdash{}{0pt}%
\pgfpathmoveto{\pgfqpoint{0.733813in}{2.641201in}}%
\pgfpathlineto{\pgfqpoint{0.844924in}{2.641201in}}%
\pgfpathlineto{\pgfqpoint{0.956035in}{2.641201in}}%
\pgfusepath{stroke}%
\end{pgfscope}%
\begin{pgfscope}%
\definecolor{textcolor}{rgb}{0.000000,0.000000,0.000000}%
\pgfsetstrokecolor{textcolor}%
\pgfsetfillcolor{textcolor}%
\pgftext[x=1.044924in,y=2.602313in,left,base]{\color{textcolor}\rmfamily\fontsize{8.000000}{9.600000}\selectfont Ambient Temperature}%
\end{pgfscope}%
\end{pgfpicture}%
\makeatother%
\endgroup%

    \caption{Voltage deviation from the mean voltage of an LM399 negative \qty{-10}{\volt} reference measured with a Keysight 34470A at \qty{100}{\plc}.}
    \label{fig:lm399_vs_34470a}
\end{figure}

Figure \ref{fig:lm399_vs_34470a} shows an example of such measurement. This measurement highlights one the problems encountered during those measurements. From this measurement it is unclear whether the features seen in the graph are only a result of ambient temperature changes due to the cycling of the air conditioning or popcorn noise on top of that. These results hightlight the fact, that sub-\unit{ppm} measurements not only requires high-end gear, but also a very stable environment. From the data it follows, that the temperature coefficient of the DMM in linear approximation is:
\begin{equation}
    \alpha_\device{34470A} \approx \frac{\qty{6.08}{\micro\volt}-(-\qty{9.30}{\micro\volt})}{(\qty{21.85}{\celsius}-\qty{19.96}{\celsius})\qty{10}{\volt}} = \qty[per-mode=symbol]{0.86}{\micro\volt \per \volt \per \kelvin}
\end{equation}

While the temperature coeeficient is vastly better than the specified \qty[per-mode=symbol]{2}{\micro\volt \per \volt \per \kelvin} \cite{datasheet_keysight34470A}, it is not low enough for this type of measurement. The multimeter must therefore be kept in a temperature controlled environment. This issue was resolved by replacing the stock air conditioning controller with a custom PID controller as discussed in section \ref{}. Lastly, the noise floor of the measurement is \qty{1.5}{\micro\volt_{rms}}, resulting in an estimated signal-to-noise ratio (SNR) of about \qty{10}{\decibel}, which is suffient to detect the popcorn noise.

While the temperature issue was being worked on, testing of the Zener diodes continued. To work around the temperature drift of the DMM, the amplification of the reference voltage was increased to \qty{15}{\volt}, the same voltage required by the digital current driver, and a differential measurement was realized. This measurement was done against a primary \qty{15}{\volt} reference board. To ensure, that any popcorn noise found originates only in the DUT and not the primary reference used, several reference boards where tested against a \device{Fluke 5440B}. The \device{5440B} does not exhibit popcorn noise as it uses a different voltage reference ic, namely two Motorola SZA263 in series \cite{service_manual_fluke_5440b}. Finally, a board that did not show popcorn noise in a period of three days was selected. The serial number of this primary or golden reference is \textit{\#1}.

Using this differential technique, the results greatly improved

In order to test a large amount of Zener diodes, and considering the duration of the burn-in process, which can take anything between \qtyrange{100}{1000}{\hour}, it is necessary to have an automated setup. This consists of a digital multimeter (DMM) a scanner and test board, that holds the Zener diodes and provides the necessary infrastructure for the diodes.



To conclude, we need a high performance DMM, a scanner, and a test fixture. The choices will detailed in the following sections.



\subsection{A Scanner System for Testing Zener Diodes}
As discussed before the diodes need to be tested for \qty{1000}{\hour} and it is not be feasible to test them individually. So a minimum of 10 diodes must be tested at the same time. To keep the system compact, the test setup a scanner to multiplex a single multimeter input. Several commercial options currently available were considered for this project and are shown in table \ref{tab:list_of_daqs}.

\begin{table}[h]
    \centering
    \small
    \begin{tabular}{ |l|l|l|l|l|l|l|l| }
        \hline
        \multirow{2}{*}{} & \multicolumn{2}{l|}{Keysight} & \multicolumn{3}{l|}{Keithley} & Fluke & Rigol \\
        \cline{2-8}
        & DAQ973A & 34980A & DAQ6510 & 2750 & 3706 & 2680 & M300 \\
        \hline
        DMM & \num{6.5} & \num{6.5} & \num{6.5} & \num{6.5} & \num{7.5} & \qty{18}{\bit} & \num{6.5} \\
        \hline
        Channels & 3x20 & 8x40 & 2x10 & 5x20 & 6x60 & 6x20 & 5x32 \\
        \hline
        FET & \textcolor{green!60!black}{\checkmark} & \textcolor{green!60!black}{\checkmark} & \textcolor{green!60!black}{\checkmark} & \textcolor{green!60!black}{\checkmark} & \textcolor{green!60!black}{\checkmark} & \textcolor{red!80!black}{\ding{55}} & \textcolor{red!80!black}{\ding{55}} \\
        \hline
        Voltage & \qty{120}{\volt} & \qty{80}{\volt} & \qty{60}{\volt} & \qty{60}{\volt} & \qty{200}{\volt} & \qty{75}{\volt} & \qty{300}{\volt} \\
        \hline
        Card & DAQM900A & 34925A & 7710 & 7710 & 3724 & 2680A-PAI & MC3132 \\
        \hline
        USB & \textcolor{green!60!black}{\checkmark} & \textcolor{green!60!black}{\checkmark} & \textcolor{green!60!black}{\checkmark} & \textcolor{red!80!black}{\ding{55}} & \textcolor{green!60!black}{\checkmark} & \textcolor{red!80!black}{\ding{55}} & \textcolor{green!60!black}{\checkmark} \\
        \hline
        Ethernet & \textcolor{green!60!black}{\checkmark} & \textcolor{green!60!black}{\checkmark} & \textcolor{green!60!black}{\checkmark} & \textcolor{red!80!black}{\ding{55}} & \textcolor{green!60!black}{\checkmark} & \textcolor{green!60!black}{\checkmark} & \textcolor{green!60!black}{\checkmark} \\
        \hline
        GPIB & \textcolor{green!60!black}{\checkmark} & \textcolor{green!60!black}{\checkmark} & \textcolor{green!60!black}{\checkmark} & \textcolor{green!60!black}{\checkmark} & \textcolor{green!60!black}{\checkmark} & \textcolor{red!80!black}{\ding{55}} & \textcolor{green!60!black}{\checkmark} \\
        %DMM & & & & & & & \\
        \hline
    \end{tabular}
    \caption{Overview of scanner mainframes}
    \label{tab:list_of_daqs}
\end{table}

A recent trend to more compact devices has led major manufacturers to include multimeters in the scanner mainframe creating so called data acquisition units. Legacy devices, that only have switching capabilities are no available. For example Keithley replaced the small desktop switch mainframe \device{Model 7001} with the \device{DAQ6510} and Keysight is offering the \device{DAQ973A}, a scanning \num{6.5} digit DMM, that accepts extension cards. Unfortunately, for this project, as discussed above, the integrated \num{6.5} digit multimeter does not add any value.

The simplest option is to go with an \num{8.5} digit multimeter that already included a scanner option or buy a used \device{Keithley 7001} from a second-hand dealer. The author has tested both options and the simplicity of only having a single device to connect and program makes the integrated scanner card of the \device{Model 2002} very attractive.

\begin{figure}[ht]
    \centering
    \scalebox{0.7}{%
        \import{figures/}{simplified_scanner.tex}
    } % scalebox
    \caption{Simplified schematic of the scanner front-end with parasitic elements}
\end{figure}

The scanner card used to multiplex the DMM does have to meet several specifications. The most important aspects are the number channels and the lifetime of the relays. Other factors, such as channel to channel isolation, the contact potential, resistance and maximum voltage is not the limiting factor.

The reason is, that in this case, the voltage is low, there is no ac component involved and the the typical input impedance of high-end multimeters is far more than \qty{100}{\giga\ohm} \cite{datasheet_fluke8588A,article_3458A_input_mpedance_2,datasheet_keithley2002,article_3458A_input_mpedance}.

In this work the Keithley (now Tektronix) \device{Model 2002} was chosen for three reasons. It is a very compact system requiring only a half-sized 2U rack in comparison to the other DMMs, that are typically full-sized 2U rack devices. The other two advantages are the integrated scanner card slot, that allows to to fit a 10 channel scanner card and finally the \qty{20}{\volt} range. The latter is interesting for testing the final voltage reference boards, as these have a \qty{15}{\volt} output, which is too much for the \qty{10}{\volt} range of most DMMs, so that testing the voltage reference Printed circuit boards
(PCBs) one would have to switch to the \qty{100}{\volt} range and forgo an extra digit of resolution and add more noise.

The test setup consists of a mounting PCB, that holds up to 20 Zener diode. It provides power regulation and a minimal circuit required to support each diode. This circuit is given here:

\begin{figure}[ht]
    \centering
    \scalebox{0.7}{%
        \import{figures/}{zener_burnin.tex}
    } % scalebox
    \caption{Circuit used for burning in the Zener diodes}
\end{figure}

The compensation network is required when using the ADR1399, because of its very low dynamic impedance as recommended in the data sheet \cite{datasheet_ADR1399}. It is not strictly required for the LM399, but fitted nonetheless, because there are no downsides to it. This makes the board compatible with both types of references. Each Zener output is protected using an output buffer, which provides isolation and short circuit protection. Finally there is a common mode filter at the output to suppress high frequency noise via ground loops.

The two key metrics of concern, that need to measured are popcorn noise and drift.

digital multimeter and a scanner card


\begin{figure}[ht]
    \centering
    \import{figures/}{DIN_41612.tex}
    \caption{The extension connector used in several Keithley multimeters}
\end{figure}

\begin{table}[ht]
    \centering
    \begin{tabular}{llllll}
        \toprule
        Pin    & Function    & Cable Colour    & Pin    & Function    &  Cable Colour\\
        \midrule
        a1, b1    & \SIrange[print-implicit-plus=true]{6}{20}{\volt}    & brown    & \num{6}    & GND    & green/white\\
        a2, b2    & PD cathode    & red    & \num{7}    & LD Cathode    & blue/white\\
        a3, b3    & LD case (GND)    & red/white    & \num{8}    & LD Anode    & blue\\
        a4    & PD anode (GND)    & red/white    & \num{9}    & LD current    & green\\
        a5    & \SIrange{-6}{-20}{\volt}    & brown/white\\
        \bottomrule
    \end{tabular}
\end{table}

As a sidenote, for the pure entertainment of the author, several batches of LM399 Zener diodes were purchased from non-authorized dealers. Some were marked as refurbished, the others were not marked as such, but clearly were. These so-called refurbished diodes are not to be used in production devices. To entertain and warn the reader a small selection of examples are shown here in figure \ref{fig:fake_lm399}. All but one diode, which is shown for comparison, are refurbished.

\begin{figure}[h]
    \centering
    %\includegraphics[width=0.75\textwidth]{images/foo.png}
    \caption{Refurbished LM399 Zener diodes. From left to right: }
    \label{fig:fake_lm399}
\end{figure}

As it can be clearly seen, the sellers have gone to some effort to hide the fact, that these diodes have been used before. When a through-hole is soldered to the PCB, its legs will be trimmed to match the PCB thickness. In order to conceal this, the legs need to be extended to their original length. The legs of the LM399 are Kovar, because the LM399 is hermetically sealed with a glass seal and Kovar has the same coefficient of expansion as borosilicate glass. The forgers typically weld steel legs to the Kovar legs and then either gold-plate or tin them, as can be seen in fig. \ref{fig:fake_lm399_legs}.

\begin{figure}[h]
    \centering
    %\includegraphics[width=0.75\textwidth]{images/foo.png}
    \caption{Fake steel legs of a refurbished LM399.}
    \label{fig:fake_lm399_legs}
\end{figure}

Much to the delight of the author the refurbished diodes prove valuable for educational purposes. As the origin and method of extraction from the original circuit is unknown, but can be imagined to be rather savage, the diodes are typically faulty. They can therefore be used to validate the test setup and demonstrate the popcorn noise found in the LM399. A very good example in shown in fig. \ref{fig:fake_lm399_popcorn_noise}.

\begin{figure}[h]
    \centering
    %% Creator: Matplotlib, PGF backend
%%
%% To include the figure in your LaTeX document, write
%%   \input{<filename>.pgf}
%%
%% Make sure the required packages are loaded in your preamble
%%   \usepackage{pgf}
%%
%% Also ensure that all the required font packages are loaded; for instance,
%% the lmodern package is sometimes necessary when using math font.
%%   \usepackage{lmodern}
%%
%% Figures using additional raster images can only be included by \input if
%% they are in the same directory as the main LaTeX file. For loading figures
%% from other directories you can use the `import` package
%%   \usepackage{import}
%%
%% and then include the figures with
%%   \import{<path to file>}{<filename>.pgf}
%%
%% Matplotlib used the following preamble
%%   \usepackage{siunitx}
%%   \usepackage{fontspec}
%%
\begingroup%
\makeatletter%
\begin{pgfpicture}%
\pgfpathrectangle{\pgfpointorigin}{\pgfqpoint{5.208662in}{3.219130in}}%
\pgfusepath{use as bounding box, clip}%
\begin{pgfscope}%
\pgfsetbuttcap%
\pgfsetmiterjoin%
\definecolor{currentfill}{rgb}{1.000000,1.000000,1.000000}%
\pgfsetfillcolor{currentfill}%
\pgfsetlinewidth{0.000000pt}%
\definecolor{currentstroke}{rgb}{1.000000,1.000000,1.000000}%
\pgfsetstrokecolor{currentstroke}%
\pgfsetdash{}{0pt}%
\pgfpathmoveto{\pgfqpoint{0.000000in}{0.000000in}}%
\pgfpathlineto{\pgfqpoint{5.208662in}{0.000000in}}%
\pgfpathlineto{\pgfqpoint{5.208662in}{3.219130in}}%
\pgfpathlineto{\pgfqpoint{0.000000in}{3.219130in}}%
\pgfpathlineto{\pgfqpoint{0.000000in}{0.000000in}}%
\pgfpathclose%
\pgfusepath{fill}%
\end{pgfscope}%
\begin{pgfscope}%
\pgfsetbuttcap%
\pgfsetmiterjoin%
\definecolor{currentfill}{rgb}{1.000000,1.000000,1.000000}%
\pgfsetfillcolor{currentfill}%
\pgfsetlinewidth{0.000000pt}%
\definecolor{currentstroke}{rgb}{0.000000,0.000000,0.000000}%
\pgfsetstrokecolor{currentstroke}%
\pgfsetstrokeopacity{0.000000}%
\pgfsetdash{}{0pt}%
\pgfpathmoveto{\pgfqpoint{0.667540in}{0.539544in}}%
\pgfpathlineto{\pgfqpoint{5.058662in}{0.539544in}}%
\pgfpathlineto{\pgfqpoint{5.058662in}{2.944887in}}%
\pgfpathlineto{\pgfqpoint{0.667540in}{2.944887in}}%
\pgfpathlineto{\pgfqpoint{0.667540in}{0.539544in}}%
\pgfpathclose%
\pgfusepath{fill}%
\end{pgfscope}%
\begin{pgfscope}%
\pgfsetbuttcap%
\pgfsetroundjoin%
\definecolor{currentfill}{rgb}{0.000000,0.000000,0.000000}%
\pgfsetfillcolor{currentfill}%
\pgfsetlinewidth{0.803000pt}%
\definecolor{currentstroke}{rgb}{0.000000,0.000000,0.000000}%
\pgfsetstrokecolor{currentstroke}%
\pgfsetdash{}{0pt}%
\pgfsys@defobject{currentmarker}{\pgfqpoint{0.000000in}{-0.048611in}}{\pgfqpoint{0.000000in}{0.000000in}}{%
\pgfpathmoveto{\pgfqpoint{0.000000in}{0.000000in}}%
\pgfpathlineto{\pgfqpoint{0.000000in}{-0.048611in}}%
\pgfusepath{stroke,fill}%
}%
\begin{pgfscope}%
\pgfsys@transformshift{0.866046in}{0.539544in}%
\pgfsys@useobject{currentmarker}{}%
\end{pgfscope}%
\end{pgfscope}%
\begin{pgfscope}%
\definecolor{textcolor}{rgb}{0.000000,0.000000,0.000000}%
\pgfsetstrokecolor{textcolor}%
\pgfsetfillcolor{textcolor}%
\pgftext[x=0.866046in,y=0.442322in,,top]{\color{textcolor}\rmfamily\fontsize{8.000000}{9.600000}\selectfont \(\displaystyle {06{:}45}\)}%
\end{pgfscope}%
\begin{pgfscope}%
\pgfsetbuttcap%
\pgfsetroundjoin%
\definecolor{currentfill}{rgb}{0.000000,0.000000,0.000000}%
\pgfsetfillcolor{currentfill}%
\pgfsetlinewidth{0.803000pt}%
\definecolor{currentstroke}{rgb}{0.000000,0.000000,0.000000}%
\pgfsetstrokecolor{currentstroke}%
\pgfsetdash{}{0pt}%
\pgfsys@defobject{currentmarker}{\pgfqpoint{0.000000in}{-0.048611in}}{\pgfqpoint{0.000000in}{0.000000in}}{%
\pgfpathmoveto{\pgfqpoint{0.000000in}{0.000000in}}%
\pgfpathlineto{\pgfqpoint{0.000000in}{-0.048611in}}%
\pgfusepath{stroke,fill}%
}%
\begin{pgfscope}%
\pgfsys@transformshift{2.197480in}{0.539544in}%
\pgfsys@useobject{currentmarker}{}%
\end{pgfscope}%
\end{pgfscope}%
\begin{pgfscope}%
\definecolor{textcolor}{rgb}{0.000000,0.000000,0.000000}%
\pgfsetstrokecolor{textcolor}%
\pgfsetfillcolor{textcolor}%
\pgftext[x=2.197480in,y=0.442322in,,top]{\color{textcolor}\rmfamily\fontsize{8.000000}{9.600000}\selectfont \(\displaystyle {06{:}50}\)}%
\end{pgfscope}%
\begin{pgfscope}%
\pgfsetbuttcap%
\pgfsetroundjoin%
\definecolor{currentfill}{rgb}{0.000000,0.000000,0.000000}%
\pgfsetfillcolor{currentfill}%
\pgfsetlinewidth{0.803000pt}%
\definecolor{currentstroke}{rgb}{0.000000,0.000000,0.000000}%
\pgfsetstrokecolor{currentstroke}%
\pgfsetdash{}{0pt}%
\pgfsys@defobject{currentmarker}{\pgfqpoint{0.000000in}{-0.048611in}}{\pgfqpoint{0.000000in}{0.000000in}}{%
\pgfpathmoveto{\pgfqpoint{0.000000in}{0.000000in}}%
\pgfpathlineto{\pgfqpoint{0.000000in}{-0.048611in}}%
\pgfusepath{stroke,fill}%
}%
\begin{pgfscope}%
\pgfsys@transformshift{3.528915in}{0.539544in}%
\pgfsys@useobject{currentmarker}{}%
\end{pgfscope}%
\end{pgfscope}%
\begin{pgfscope}%
\definecolor{textcolor}{rgb}{0.000000,0.000000,0.000000}%
\pgfsetstrokecolor{textcolor}%
\pgfsetfillcolor{textcolor}%
\pgftext[x=3.528915in,y=0.442322in,,top]{\color{textcolor}\rmfamily\fontsize{8.000000}{9.600000}\selectfont \(\displaystyle {06{:}55}\)}%
\end{pgfscope}%
\begin{pgfscope}%
\pgfsetbuttcap%
\pgfsetroundjoin%
\definecolor{currentfill}{rgb}{0.000000,0.000000,0.000000}%
\pgfsetfillcolor{currentfill}%
\pgfsetlinewidth{0.803000pt}%
\definecolor{currentstroke}{rgb}{0.000000,0.000000,0.000000}%
\pgfsetstrokecolor{currentstroke}%
\pgfsetdash{}{0pt}%
\pgfsys@defobject{currentmarker}{\pgfqpoint{0.000000in}{-0.048611in}}{\pgfqpoint{0.000000in}{0.000000in}}{%
\pgfpathmoveto{\pgfqpoint{0.000000in}{0.000000in}}%
\pgfpathlineto{\pgfqpoint{0.000000in}{-0.048611in}}%
\pgfusepath{stroke,fill}%
}%
\begin{pgfscope}%
\pgfsys@transformshift{4.860349in}{0.539544in}%
\pgfsys@useobject{currentmarker}{}%
\end{pgfscope}%
\end{pgfscope}%
\begin{pgfscope}%
\definecolor{textcolor}{rgb}{0.000000,0.000000,0.000000}%
\pgfsetstrokecolor{textcolor}%
\pgfsetfillcolor{textcolor}%
\pgftext[x=4.860349in,y=0.442322in,,top]{\color{textcolor}\rmfamily\fontsize{8.000000}{9.600000}\selectfont \(\displaystyle {07{:}00}\)}%
\end{pgfscope}%
\begin{pgfscope}%
\definecolor{textcolor}{rgb}{0.000000,0.000000,0.000000}%
\pgfsetstrokecolor{textcolor}%
\pgfsetfillcolor{textcolor}%
\pgftext[x=2.863101in,y=0.288100in,,top]{\color{textcolor}\rmfamily\fontsize{10.000000}{12.000000}\selectfont Time (UTC)}%
\end{pgfscope}%
\begin{pgfscope}%
\pgfsetbuttcap%
\pgfsetroundjoin%
\definecolor{currentfill}{rgb}{0.000000,0.000000,0.000000}%
\pgfsetfillcolor{currentfill}%
\pgfsetlinewidth{0.803000pt}%
\definecolor{currentstroke}{rgb}{0.000000,0.000000,0.000000}%
\pgfsetstrokecolor{currentstroke}%
\pgfsetdash{}{0pt}%
\pgfsys@defobject{currentmarker}{\pgfqpoint{-0.048611in}{0.000000in}}{\pgfqpoint{-0.000000in}{0.000000in}}{%
\pgfpathmoveto{\pgfqpoint{-0.000000in}{0.000000in}}%
\pgfpathlineto{\pgfqpoint{-0.048611in}{0.000000in}}%
\pgfusepath{stroke,fill}%
}%
\begin{pgfscope}%
\pgfsys@transformshift{0.667540in}{0.611455in}%
\pgfsys@useobject{currentmarker}{}%
\end{pgfscope}%
\end{pgfscope}%
\begin{pgfscope}%
\definecolor{textcolor}{rgb}{0.000000,0.000000,0.000000}%
\pgfsetstrokecolor{textcolor}%
\pgfsetfillcolor{textcolor}%
\pgftext[x=0.327644in, y=0.572899in, left, base]{\color{textcolor}\rmfamily\fontsize{8.000000}{9.600000}\selectfont \(\displaystyle {\ensuremath{-}7.5}\)}%
\end{pgfscope}%
\begin{pgfscope}%
\pgfsetbuttcap%
\pgfsetroundjoin%
\definecolor{currentfill}{rgb}{0.000000,0.000000,0.000000}%
\pgfsetfillcolor{currentfill}%
\pgfsetlinewidth{0.803000pt}%
\definecolor{currentstroke}{rgb}{0.000000,0.000000,0.000000}%
\pgfsetstrokecolor{currentstroke}%
\pgfsetdash{}{0pt}%
\pgfsys@defobject{currentmarker}{\pgfqpoint{-0.048611in}{0.000000in}}{\pgfqpoint{-0.000000in}{0.000000in}}{%
\pgfpathmoveto{\pgfqpoint{-0.000000in}{0.000000in}}%
\pgfpathlineto{\pgfqpoint{-0.048611in}{0.000000in}}%
\pgfusepath{stroke,fill}%
}%
\begin{pgfscope}%
\pgfsys@transformshift{0.667540in}{0.925106in}%
\pgfsys@useobject{currentmarker}{}%
\end{pgfscope}%
\end{pgfscope}%
\begin{pgfscope}%
\definecolor{textcolor}{rgb}{0.000000,0.000000,0.000000}%
\pgfsetstrokecolor{textcolor}%
\pgfsetfillcolor{textcolor}%
\pgftext[x=0.327644in, y=0.886551in, left, base]{\color{textcolor}\rmfamily\fontsize{8.000000}{9.600000}\selectfont \(\displaystyle {\ensuremath{-}5.0}\)}%
\end{pgfscope}%
\begin{pgfscope}%
\pgfsetbuttcap%
\pgfsetroundjoin%
\definecolor{currentfill}{rgb}{0.000000,0.000000,0.000000}%
\pgfsetfillcolor{currentfill}%
\pgfsetlinewidth{0.803000pt}%
\definecolor{currentstroke}{rgb}{0.000000,0.000000,0.000000}%
\pgfsetstrokecolor{currentstroke}%
\pgfsetdash{}{0pt}%
\pgfsys@defobject{currentmarker}{\pgfqpoint{-0.048611in}{0.000000in}}{\pgfqpoint{-0.000000in}{0.000000in}}{%
\pgfpathmoveto{\pgfqpoint{-0.000000in}{0.000000in}}%
\pgfpathlineto{\pgfqpoint{-0.048611in}{0.000000in}}%
\pgfusepath{stroke,fill}%
}%
\begin{pgfscope}%
\pgfsys@transformshift{0.667540in}{1.238757in}%
\pgfsys@useobject{currentmarker}{}%
\end{pgfscope}%
\end{pgfscope}%
\begin{pgfscope}%
\definecolor{textcolor}{rgb}{0.000000,0.000000,0.000000}%
\pgfsetstrokecolor{textcolor}%
\pgfsetfillcolor{textcolor}%
\pgftext[x=0.327644in, y=1.200202in, left, base]{\color{textcolor}\rmfamily\fontsize{8.000000}{9.600000}\selectfont \(\displaystyle {\ensuremath{-}2.5}\)}%
\end{pgfscope}%
\begin{pgfscope}%
\pgfsetbuttcap%
\pgfsetroundjoin%
\definecolor{currentfill}{rgb}{0.000000,0.000000,0.000000}%
\pgfsetfillcolor{currentfill}%
\pgfsetlinewidth{0.803000pt}%
\definecolor{currentstroke}{rgb}{0.000000,0.000000,0.000000}%
\pgfsetstrokecolor{currentstroke}%
\pgfsetdash{}{0pt}%
\pgfsys@defobject{currentmarker}{\pgfqpoint{-0.048611in}{0.000000in}}{\pgfqpoint{-0.000000in}{0.000000in}}{%
\pgfpathmoveto{\pgfqpoint{-0.000000in}{0.000000in}}%
\pgfpathlineto{\pgfqpoint{-0.048611in}{0.000000in}}%
\pgfusepath{stroke,fill}%
}%
\begin{pgfscope}%
\pgfsys@transformshift{0.667540in}{1.552408in}%
\pgfsys@useobject{currentmarker}{}%
\end{pgfscope}%
\end{pgfscope}%
\begin{pgfscope}%
\definecolor{textcolor}{rgb}{0.000000,0.000000,0.000000}%
\pgfsetstrokecolor{textcolor}%
\pgfsetfillcolor{textcolor}%
\pgftext[x=0.419467in, y=1.513853in, left, base]{\color{textcolor}\rmfamily\fontsize{8.000000}{9.600000}\selectfont \(\displaystyle {0.0}\)}%
\end{pgfscope}%
\begin{pgfscope}%
\pgfsetbuttcap%
\pgfsetroundjoin%
\definecolor{currentfill}{rgb}{0.000000,0.000000,0.000000}%
\pgfsetfillcolor{currentfill}%
\pgfsetlinewidth{0.803000pt}%
\definecolor{currentstroke}{rgb}{0.000000,0.000000,0.000000}%
\pgfsetstrokecolor{currentstroke}%
\pgfsetdash{}{0pt}%
\pgfsys@defobject{currentmarker}{\pgfqpoint{-0.048611in}{0.000000in}}{\pgfqpoint{-0.000000in}{0.000000in}}{%
\pgfpathmoveto{\pgfqpoint{-0.000000in}{0.000000in}}%
\pgfpathlineto{\pgfqpoint{-0.048611in}{0.000000in}}%
\pgfusepath{stroke,fill}%
}%
\begin{pgfscope}%
\pgfsys@transformshift{0.667540in}{1.866059in}%
\pgfsys@useobject{currentmarker}{}%
\end{pgfscope}%
\end{pgfscope}%
\begin{pgfscope}%
\definecolor{textcolor}{rgb}{0.000000,0.000000,0.000000}%
\pgfsetstrokecolor{textcolor}%
\pgfsetfillcolor{textcolor}%
\pgftext[x=0.419467in, y=1.827504in, left, base]{\color{textcolor}\rmfamily\fontsize{8.000000}{9.600000}\selectfont \(\displaystyle {2.5}\)}%
\end{pgfscope}%
\begin{pgfscope}%
\pgfsetbuttcap%
\pgfsetroundjoin%
\definecolor{currentfill}{rgb}{0.000000,0.000000,0.000000}%
\pgfsetfillcolor{currentfill}%
\pgfsetlinewidth{0.803000pt}%
\definecolor{currentstroke}{rgb}{0.000000,0.000000,0.000000}%
\pgfsetstrokecolor{currentstroke}%
\pgfsetdash{}{0pt}%
\pgfsys@defobject{currentmarker}{\pgfqpoint{-0.048611in}{0.000000in}}{\pgfqpoint{-0.000000in}{0.000000in}}{%
\pgfpathmoveto{\pgfqpoint{-0.000000in}{0.000000in}}%
\pgfpathlineto{\pgfqpoint{-0.048611in}{0.000000in}}%
\pgfusepath{stroke,fill}%
}%
\begin{pgfscope}%
\pgfsys@transformshift{0.667540in}{2.179710in}%
\pgfsys@useobject{currentmarker}{}%
\end{pgfscope}%
\end{pgfscope}%
\begin{pgfscope}%
\definecolor{textcolor}{rgb}{0.000000,0.000000,0.000000}%
\pgfsetstrokecolor{textcolor}%
\pgfsetfillcolor{textcolor}%
\pgftext[x=0.419467in, y=2.141155in, left, base]{\color{textcolor}\rmfamily\fontsize{8.000000}{9.600000}\selectfont \(\displaystyle {5.0}\)}%
\end{pgfscope}%
\begin{pgfscope}%
\pgfsetbuttcap%
\pgfsetroundjoin%
\definecolor{currentfill}{rgb}{0.000000,0.000000,0.000000}%
\pgfsetfillcolor{currentfill}%
\pgfsetlinewidth{0.803000pt}%
\definecolor{currentstroke}{rgb}{0.000000,0.000000,0.000000}%
\pgfsetstrokecolor{currentstroke}%
\pgfsetdash{}{0pt}%
\pgfsys@defobject{currentmarker}{\pgfqpoint{-0.048611in}{0.000000in}}{\pgfqpoint{-0.000000in}{0.000000in}}{%
\pgfpathmoveto{\pgfqpoint{-0.000000in}{0.000000in}}%
\pgfpathlineto{\pgfqpoint{-0.048611in}{0.000000in}}%
\pgfusepath{stroke,fill}%
}%
\begin{pgfscope}%
\pgfsys@transformshift{0.667540in}{2.493362in}%
\pgfsys@useobject{currentmarker}{}%
\end{pgfscope}%
\end{pgfscope}%
\begin{pgfscope}%
\definecolor{textcolor}{rgb}{0.000000,0.000000,0.000000}%
\pgfsetstrokecolor{textcolor}%
\pgfsetfillcolor{textcolor}%
\pgftext[x=0.419467in, y=2.454806in, left, base]{\color{textcolor}\rmfamily\fontsize{8.000000}{9.600000}\selectfont \(\displaystyle {7.5}\)}%
\end{pgfscope}%
\begin{pgfscope}%
\pgfsetbuttcap%
\pgfsetroundjoin%
\definecolor{currentfill}{rgb}{0.000000,0.000000,0.000000}%
\pgfsetfillcolor{currentfill}%
\pgfsetlinewidth{0.803000pt}%
\definecolor{currentstroke}{rgb}{0.000000,0.000000,0.000000}%
\pgfsetstrokecolor{currentstroke}%
\pgfsetdash{}{0pt}%
\pgfsys@defobject{currentmarker}{\pgfqpoint{-0.048611in}{0.000000in}}{\pgfqpoint{-0.000000in}{0.000000in}}{%
\pgfpathmoveto{\pgfqpoint{-0.000000in}{0.000000in}}%
\pgfpathlineto{\pgfqpoint{-0.048611in}{0.000000in}}%
\pgfusepath{stroke,fill}%
}%
\begin{pgfscope}%
\pgfsys@transformshift{0.667540in}{2.807013in}%
\pgfsys@useobject{currentmarker}{}%
\end{pgfscope}%
\end{pgfscope}%
\begin{pgfscope}%
\definecolor{textcolor}{rgb}{0.000000,0.000000,0.000000}%
\pgfsetstrokecolor{textcolor}%
\pgfsetfillcolor{textcolor}%
\pgftext[x=0.360438in, y=2.768457in, left, base]{\color{textcolor}\rmfamily\fontsize{8.000000}{9.600000}\selectfont \(\displaystyle {10.0}\)}%
\end{pgfscope}%
\begin{pgfscope}%
\definecolor{textcolor}{rgb}{0.000000,0.000000,0.000000}%
\pgfsetstrokecolor{textcolor}%
\pgfsetfillcolor{textcolor}%
\pgftext[x=0.272089in,y=1.742216in,,bottom,rotate=90.000000]{\color{textcolor}\rmfamily\fontsize{10.000000}{12.000000}\selectfont Voltage deviation in V}%
\end{pgfscope}%
\begin{pgfscope}%
\definecolor{textcolor}{rgb}{0.000000,0.000000,0.000000}%
\pgfsetstrokecolor{textcolor}%
\pgfsetfillcolor{textcolor}%
\pgftext[x=0.667540in,y=2.986554in,left,base]{\color{textcolor}\rmfamily\fontsize{8.000000}{9.600000}\selectfont \(\displaystyle \times{10^{\ensuremath{-}6}}{}\)}%
\end{pgfscope}%
\begin{pgfscope}%
\pgfpathrectangle{\pgfqpoint{0.667540in}{0.539544in}}{\pgfqpoint{4.391122in}{2.405343in}}%
\pgfusepath{clip}%
\pgfsetrectcap%
\pgfsetroundjoin%
\pgfsetlinewidth{0.501875pt}%
\definecolor{currentstroke}{rgb}{0.121569,0.466667,0.705882}%
\pgfsetstrokecolor{currentstroke}%
\pgfsetstrokeopacity{0.700000}%
\pgfsetdash{}{0pt}%
\pgfpathmoveto{\pgfqpoint{0.867136in}{1.087387in}}%
\pgfpathlineto{\pgfqpoint{0.870808in}{1.118966in}}%
\pgfpathlineto{\pgfqpoint{0.872643in}{1.181144in}}%
\pgfpathlineto{\pgfqpoint{0.874481in}{0.958565in}}%
\pgfpathlineto{\pgfqpoint{0.876316in}{1.101426in}}%
\pgfpathlineto{\pgfqpoint{0.878153in}{0.951576in}}%
\pgfpathlineto{\pgfqpoint{0.881823in}{1.075920in}}%
\pgfpathlineto{\pgfqpoint{0.883659in}{1.252581in}}%
\pgfpathlineto{\pgfqpoint{0.885494in}{1.091590in}}%
\pgfpathlineto{\pgfqpoint{0.887330in}{1.123470in}}%
\pgfpathlineto{\pgfqpoint{0.889165in}{1.181997in}}%
\pgfpathlineto{\pgfqpoint{0.892838in}{1.123909in}}%
\pgfpathlineto{\pgfqpoint{0.894674in}{1.172450in}}%
\pgfpathlineto{\pgfqpoint{0.896510in}{1.289040in}}%
\pgfpathlineto{\pgfqpoint{0.898346in}{1.152940in}}%
\pgfpathlineto{\pgfqpoint{0.902018in}{1.215583in}}%
\pgfpathlineto{\pgfqpoint{0.905690in}{1.384741in}}%
\pgfpathlineto{\pgfqpoint{0.909363in}{1.223286in}}%
\pgfpathlineto{\pgfqpoint{0.911199in}{1.421614in}}%
\pgfpathlineto{\pgfqpoint{0.913035in}{1.319953in}}%
\pgfpathlineto{\pgfqpoint{0.914871in}{1.291248in}}%
\pgfpathlineto{\pgfqpoint{0.918543in}{1.115240in}}%
\pgfpathlineto{\pgfqpoint{0.920378in}{1.088742in}}%
\pgfpathlineto{\pgfqpoint{0.922213in}{1.270585in}}%
\pgfpathlineto{\pgfqpoint{0.924048in}{1.272203in}}%
\pgfpathlineto{\pgfqpoint{0.925883in}{1.289065in}}%
\pgfpathlineto{\pgfqpoint{0.929556in}{1.198307in}}%
\pgfpathlineto{\pgfqpoint{0.931392in}{1.190227in}}%
\pgfpathlineto{\pgfqpoint{0.933228in}{1.146442in}}%
\pgfpathlineto{\pgfqpoint{0.935065in}{1.283369in}}%
\pgfpathlineto{\pgfqpoint{0.938735in}{1.029588in}}%
\pgfpathlineto{\pgfqpoint{0.946079in}{1.626290in}}%
\pgfpathlineto{\pgfqpoint{0.949753in}{1.444096in}}%
\pgfpathlineto{\pgfqpoint{0.951589in}{1.616956in}}%
\pgfpathlineto{\pgfqpoint{0.953425in}{1.532446in}}%
\pgfpathlineto{\pgfqpoint{0.955260in}{1.652950in}}%
\pgfpathlineto{\pgfqpoint{0.957096in}{1.510703in}}%
\pgfpathlineto{\pgfqpoint{0.958931in}{1.617922in}}%
\pgfpathlineto{\pgfqpoint{0.960766in}{1.610194in}}%
\pgfpathlineto{\pgfqpoint{0.962600in}{1.313542in}}%
\pgfpathlineto{\pgfqpoint{0.964437in}{1.329589in}}%
\pgfpathlineto{\pgfqpoint{0.966272in}{1.506526in}}%
\pgfpathlineto{\pgfqpoint{0.968108in}{1.501068in}}%
\pgfpathlineto{\pgfqpoint{0.969944in}{1.597183in}}%
\pgfpathlineto{\pgfqpoint{0.971780in}{1.593921in}}%
\pgfpathlineto{\pgfqpoint{0.973616in}{1.657530in}}%
\pgfpathlineto{\pgfqpoint{0.975451in}{1.470130in}}%
\pgfpathlineto{\pgfqpoint{0.979123in}{1.384151in}}%
\pgfpathlineto{\pgfqpoint{0.980959in}{1.395556in}}%
\pgfpathlineto{\pgfqpoint{0.982794in}{1.581990in}}%
\pgfpathlineto{\pgfqpoint{0.984630in}{1.471886in}}%
\pgfpathlineto{\pgfqpoint{0.986466in}{1.237476in}}%
\pgfpathlineto{\pgfqpoint{0.990138in}{1.537000in}}%
\pgfpathlineto{\pgfqpoint{0.991973in}{1.372346in}}%
\pgfpathlineto{\pgfqpoint{0.995643in}{1.405605in}}%
\pgfpathlineto{\pgfqpoint{0.997478in}{1.710361in}}%
\pgfpathlineto{\pgfqpoint{0.999315in}{1.353740in}}%
\pgfpathlineto{\pgfqpoint{1.001150in}{1.559683in}}%
\pgfpathlineto{\pgfqpoint{1.002986in}{1.584449in}}%
\pgfpathlineto{\pgfqpoint{1.004822in}{1.532120in}}%
\pgfpathlineto{\pgfqpoint{1.006658in}{1.099607in}}%
\pgfpathlineto{\pgfqpoint{1.008497in}{1.116206in}}%
\pgfpathlineto{\pgfqpoint{1.010333in}{1.363701in}}%
\pgfpathlineto{\pgfqpoint{1.012168in}{1.297822in}}%
\pgfpathlineto{\pgfqpoint{1.014005in}{1.069810in}}%
\pgfpathlineto{\pgfqpoint{1.017676in}{1.366574in}}%
\pgfpathlineto{\pgfqpoint{1.019512in}{1.219133in}}%
\pgfpathlineto{\pgfqpoint{1.021348in}{1.249871in}}%
\pgfpathlineto{\pgfqpoint{1.023183in}{1.354882in}}%
\pgfpathlineto{\pgfqpoint{1.025021in}{1.274637in}}%
\pgfpathlineto{\pgfqpoint{1.026856in}{1.591236in}}%
\pgfpathlineto{\pgfqpoint{1.028692in}{1.362409in}}%
\pgfpathlineto{\pgfqpoint{1.030527in}{1.339613in}}%
\pgfpathlineto{\pgfqpoint{1.034199in}{1.115679in}}%
\pgfpathlineto{\pgfqpoint{1.036034in}{1.211944in}}%
\pgfpathlineto{\pgfqpoint{1.037869in}{1.103158in}}%
\pgfpathlineto{\pgfqpoint{1.039706in}{1.258302in}}%
\pgfpathlineto{\pgfqpoint{1.041541in}{1.234803in}}%
\pgfpathlineto{\pgfqpoint{1.043377in}{1.057089in}}%
\pgfpathlineto{\pgfqpoint{1.048886in}{2.234585in}}%
\pgfpathlineto{\pgfqpoint{1.050723in}{2.532253in}}%
\pgfpathlineto{\pgfqpoint{1.052559in}{2.137178in}}%
\pgfpathlineto{\pgfqpoint{1.054395in}{1.468800in}}%
\pgfpathlineto{\pgfqpoint{1.056245in}{1.494895in}}%
\pgfpathlineto{\pgfqpoint{1.058082in}{1.229421in}}%
\pgfpathlineto{\pgfqpoint{1.061752in}{1.352648in}}%
\pgfpathlineto{\pgfqpoint{1.063588in}{1.260711in}}%
\pgfpathlineto{\pgfqpoint{1.065424in}{1.255956in}}%
\pgfpathlineto{\pgfqpoint{1.067259in}{1.191256in}}%
\pgfpathlineto{\pgfqpoint{1.069094in}{1.201155in}}%
\pgfpathlineto{\pgfqpoint{1.070929in}{1.157846in}}%
\pgfpathlineto{\pgfqpoint{1.072765in}{1.334406in}}%
\pgfpathlineto{\pgfqpoint{1.074601in}{1.051305in}}%
\pgfpathlineto{\pgfqpoint{1.076437in}{1.210828in}}%
\pgfpathlineto{\pgfqpoint{1.078273in}{1.206675in}}%
\pgfpathlineto{\pgfqpoint{1.080109in}{1.457370in}}%
\pgfpathlineto{\pgfqpoint{1.081946in}{2.094308in}}%
\pgfpathlineto{\pgfqpoint{1.083781in}{2.144203in}}%
\pgfpathlineto{\pgfqpoint{1.085616in}{2.054763in}}%
\pgfpathlineto{\pgfqpoint{1.087454in}{2.151279in}}%
\pgfpathlineto{\pgfqpoint{1.089290in}{2.347499in}}%
\pgfpathlineto{\pgfqpoint{1.091125in}{2.151543in}}%
\pgfpathlineto{\pgfqpoint{1.092961in}{2.094546in}}%
\pgfpathlineto{\pgfqpoint{1.094797in}{2.186985in}}%
\pgfpathlineto{\pgfqpoint{1.096632in}{2.073795in}}%
\pgfpathlineto{\pgfqpoint{1.098468in}{2.131156in}}%
\pgfpathlineto{\pgfqpoint{1.100303in}{2.013587in}}%
\pgfpathlineto{\pgfqpoint{1.102138in}{2.176083in}}%
\pgfpathlineto{\pgfqpoint{1.105808in}{2.148871in}}%
\pgfpathlineto{\pgfqpoint{1.107646in}{2.056369in}}%
\pgfpathlineto{\pgfqpoint{1.111317in}{2.206846in}}%
\pgfpathlineto{\pgfqpoint{1.113154in}{2.051614in}}%
\pgfpathlineto{\pgfqpoint{1.114989in}{2.200761in}}%
\pgfpathlineto{\pgfqpoint{1.116824in}{2.146725in}}%
\pgfpathlineto{\pgfqpoint{1.118661in}{2.019144in}}%
\pgfpathlineto{\pgfqpoint{1.120497in}{2.141468in}}%
\pgfpathlineto{\pgfqpoint{1.124169in}{1.269732in}}%
\pgfpathlineto{\pgfqpoint{1.126004in}{1.341721in}}%
\pgfpathlineto{\pgfqpoint{1.127845in}{1.286606in}}%
\pgfpathlineto{\pgfqpoint{1.129681in}{1.146241in}}%
\pgfpathlineto{\pgfqpoint{1.133355in}{1.080562in}}%
\pgfpathlineto{\pgfqpoint{1.135191in}{1.274085in}}%
\pgfpathlineto{\pgfqpoint{1.137027in}{1.059849in}}%
\pgfpathlineto{\pgfqpoint{1.138862in}{1.087538in}}%
\pgfpathlineto{\pgfqpoint{1.140696in}{1.174469in}}%
\pgfpathlineto{\pgfqpoint{1.142533in}{1.206085in}}%
\pgfpathlineto{\pgfqpoint{1.144368in}{1.114060in}}%
\pgfpathlineto{\pgfqpoint{1.146205in}{1.310669in}}%
\pgfpathlineto{\pgfqpoint{1.148041in}{1.983175in}}%
\pgfpathlineto{\pgfqpoint{1.149878in}{2.057134in}}%
\pgfpathlineto{\pgfqpoint{1.151713in}{1.299027in}}%
\pgfpathlineto{\pgfqpoint{1.153550in}{2.024163in}}%
\pgfpathlineto{\pgfqpoint{1.155386in}{1.916505in}}%
\pgfpathlineto{\pgfqpoint{1.157223in}{1.936228in}}%
\pgfpathlineto{\pgfqpoint{1.159058in}{2.108121in}}%
\pgfpathlineto{\pgfqpoint{1.160894in}{2.131682in}}%
\pgfpathlineto{\pgfqpoint{1.162731in}{2.048352in}}%
\pgfpathlineto{\pgfqpoint{1.166405in}{2.197474in}}%
\pgfpathlineto{\pgfqpoint{1.168241in}{2.119337in}}%
\pgfpathlineto{\pgfqpoint{1.170076in}{2.177965in}}%
\pgfpathlineto{\pgfqpoint{1.171914in}{2.155921in}}%
\pgfpathlineto{\pgfqpoint{1.173748in}{2.047235in}}%
\pgfpathlineto{\pgfqpoint{1.175582in}{2.072151in}}%
\pgfpathlineto{\pgfqpoint{1.177418in}{2.185630in}}%
\pgfpathlineto{\pgfqpoint{1.181087in}{1.964230in}}%
\pgfpathlineto{\pgfqpoint{1.186596in}{2.183071in}}%
\pgfpathlineto{\pgfqpoint{1.190268in}{1.984066in}}%
\pgfpathlineto{\pgfqpoint{1.192105in}{1.972047in}}%
\pgfpathlineto{\pgfqpoint{1.193940in}{1.421413in}}%
\pgfpathlineto{\pgfqpoint{1.197612in}{1.164069in}}%
\pgfpathlineto{\pgfqpoint{1.199448in}{1.245317in}}%
\pgfpathlineto{\pgfqpoint{1.201285in}{1.047541in}}%
\pgfpathlineto{\pgfqpoint{1.203121in}{1.305262in}}%
\pgfpathlineto{\pgfqpoint{1.204957in}{1.405982in}}%
\pgfpathlineto{\pgfqpoint{1.208632in}{1.280609in}}%
\pgfpathlineto{\pgfqpoint{1.210467in}{1.242118in}}%
\pgfpathlineto{\pgfqpoint{1.212302in}{1.359611in}}%
\pgfpathlineto{\pgfqpoint{1.214137in}{1.275666in}}%
\pgfpathlineto{\pgfqpoint{1.215972in}{1.375043in}}%
\pgfpathlineto{\pgfqpoint{1.217807in}{1.207415in}}%
\pgfpathlineto{\pgfqpoint{1.221479in}{1.316428in}}%
\pgfpathlineto{\pgfqpoint{1.223316in}{1.354831in}}%
\pgfpathlineto{\pgfqpoint{1.225150in}{1.049398in}}%
\pgfpathlineto{\pgfqpoint{1.226988in}{1.149678in}}%
\pgfpathlineto{\pgfqpoint{1.228824in}{1.126845in}}%
\pgfpathlineto{\pgfqpoint{1.230659in}{1.174971in}}%
\pgfpathlineto{\pgfqpoint{1.232495in}{1.334820in}}%
\pgfpathlineto{\pgfqpoint{1.234331in}{1.203852in}}%
\pgfpathlineto{\pgfqpoint{1.236168in}{1.285715in}}%
\pgfpathlineto{\pgfqpoint{1.238004in}{1.260862in}}%
\pgfpathlineto{\pgfqpoint{1.239840in}{1.427523in}}%
\pgfpathlineto{\pgfqpoint{1.241677in}{1.373400in}}%
\pgfpathlineto{\pgfqpoint{1.243513in}{1.231328in}}%
\pgfpathlineto{\pgfqpoint{1.245348in}{1.389910in}}%
\pgfpathlineto{\pgfqpoint{1.249019in}{1.220689in}}%
\pgfpathlineto{\pgfqpoint{1.250854in}{1.366261in}}%
\pgfpathlineto{\pgfqpoint{1.252690in}{1.261564in}}%
\pgfpathlineto{\pgfqpoint{1.256359in}{1.400549in}}%
\pgfpathlineto{\pgfqpoint{1.258196in}{1.241904in}}%
\pgfpathlineto{\pgfqpoint{1.260032in}{1.239847in}}%
\pgfpathlineto{\pgfqpoint{1.261867in}{1.542972in}}%
\pgfpathlineto{\pgfqpoint{1.263704in}{2.158004in}}%
\pgfpathlineto{\pgfqpoint{1.265540in}{2.350322in}}%
\pgfpathlineto{\pgfqpoint{1.269213in}{2.139147in}}%
\pgfpathlineto{\pgfqpoint{1.271049in}{2.259288in}}%
\pgfpathlineto{\pgfqpoint{1.272886in}{2.184689in}}%
\pgfpathlineto{\pgfqpoint{1.274721in}{2.231022in}}%
\pgfpathlineto{\pgfqpoint{1.276557in}{2.170939in}}%
\pgfpathlineto{\pgfqpoint{1.278395in}{2.312145in}}%
\pgfpathlineto{\pgfqpoint{1.280229in}{2.275749in}}%
\pgfpathlineto{\pgfqpoint{1.283924in}{2.348616in}}%
\pgfpathlineto{\pgfqpoint{1.285760in}{2.339119in}}%
\pgfpathlineto{\pgfqpoint{1.287594in}{2.442611in}}%
\pgfpathlineto{\pgfqpoint{1.289430in}{2.232402in}}%
\pgfpathlineto{\pgfqpoint{1.291265in}{2.217159in}}%
\pgfpathlineto{\pgfqpoint{1.294938in}{2.254496in}}%
\pgfpathlineto{\pgfqpoint{1.296774in}{2.273478in}}%
\pgfpathlineto{\pgfqpoint{1.298608in}{2.227710in}}%
\pgfpathlineto{\pgfqpoint{1.300446in}{2.217096in}}%
\pgfpathlineto{\pgfqpoint{1.302282in}{2.156800in}}%
\pgfpathlineto{\pgfqpoint{1.304117in}{1.349863in}}%
\pgfpathlineto{\pgfqpoint{1.305953in}{1.111012in}}%
\pgfpathlineto{\pgfqpoint{1.307790in}{1.240211in}}%
\pgfpathlineto{\pgfqpoint{1.309627in}{1.209059in}}%
\pgfpathlineto{\pgfqpoint{1.311463in}{1.149503in}}%
\pgfpathlineto{\pgfqpoint{1.315135in}{1.233423in}}%
\pgfpathlineto{\pgfqpoint{1.316970in}{1.146203in}}%
\pgfpathlineto{\pgfqpoint{1.318807in}{1.237614in}}%
\pgfpathlineto{\pgfqpoint{1.320643in}{1.170279in}}%
\pgfpathlineto{\pgfqpoint{1.322477in}{1.275264in}}%
\pgfpathlineto{\pgfqpoint{1.326148in}{1.116733in}}%
\pgfpathlineto{\pgfqpoint{1.327983in}{1.249369in}}%
\pgfpathlineto{\pgfqpoint{1.329819in}{1.117410in}}%
\pgfpathlineto{\pgfqpoint{1.331656in}{1.345899in}}%
\pgfpathlineto{\pgfqpoint{1.333492in}{1.277786in}}%
\pgfpathlineto{\pgfqpoint{1.335329in}{1.164044in}}%
\pgfpathlineto{\pgfqpoint{1.337164in}{1.195396in}}%
\pgfpathlineto{\pgfqpoint{1.339000in}{1.120935in}}%
\pgfpathlineto{\pgfqpoint{1.342670in}{1.215344in}}%
\pgfpathlineto{\pgfqpoint{1.344507in}{1.183641in}}%
\pgfpathlineto{\pgfqpoint{1.347225in}{1.035560in}}%
\pgfpathlineto{\pgfqpoint{1.349060in}{1.134811in}}%
\pgfpathlineto{\pgfqpoint{1.350897in}{1.093146in}}%
\pgfpathlineto{\pgfqpoint{1.352733in}{1.092556in}}%
\pgfpathlineto{\pgfqpoint{1.354568in}{1.137546in}}%
\pgfpathlineto{\pgfqpoint{1.358240in}{0.961801in}}%
\pgfpathlineto{\pgfqpoint{1.360072in}{1.107900in}}%
\pgfpathlineto{\pgfqpoint{1.361907in}{1.141072in}}%
\pgfpathlineto{\pgfqpoint{1.363739in}{1.103434in}}%
\pgfpathlineto{\pgfqpoint{1.365572in}{1.188345in}}%
\pgfpathlineto{\pgfqpoint{1.367406in}{1.130960in}}%
\pgfpathlineto{\pgfqpoint{1.369240in}{1.462226in}}%
\pgfpathlineto{\pgfqpoint{1.372906in}{1.214930in}}%
\pgfpathlineto{\pgfqpoint{1.374739in}{1.296116in}}%
\pgfpathlineto{\pgfqpoint{1.376572in}{1.258653in}}%
\pgfpathlineto{\pgfqpoint{1.378404in}{1.542846in}}%
\pgfpathlineto{\pgfqpoint{1.383907in}{1.135514in}}%
\pgfpathlineto{\pgfqpoint{1.385743in}{0.977346in}}%
\pgfpathlineto{\pgfqpoint{1.387576in}{1.130307in}}%
\pgfpathlineto{\pgfqpoint{1.389409in}{1.049160in}}%
\pgfpathlineto{\pgfqpoint{1.391244in}{1.216725in}}%
\pgfpathlineto{\pgfqpoint{1.393077in}{1.149152in}}%
\pgfpathlineto{\pgfqpoint{1.396744in}{1.240123in}}%
\pgfpathlineto{\pgfqpoint{1.398578in}{1.367377in}}%
\pgfpathlineto{\pgfqpoint{1.402245in}{1.101690in}}%
\pgfpathlineto{\pgfqpoint{1.404078in}{1.141423in}}%
\pgfpathlineto{\pgfqpoint{1.405913in}{1.019538in}}%
\pgfpathlineto{\pgfqpoint{1.407746in}{1.120622in}}%
\pgfpathlineto{\pgfqpoint{1.409581in}{1.145764in}}%
\pgfpathlineto{\pgfqpoint{1.413248in}{0.967886in}}%
\pgfpathlineto{\pgfqpoint{1.415082in}{1.012161in}}%
\pgfpathlineto{\pgfqpoint{1.416914in}{0.978927in}}%
\pgfpathlineto{\pgfqpoint{1.420582in}{1.154847in}}%
\pgfpathlineto{\pgfqpoint{1.424249in}{1.255780in}}%
\pgfpathlineto{\pgfqpoint{1.426084in}{1.173892in}}%
\pgfpathlineto{\pgfqpoint{1.427917in}{1.306793in}}%
\pgfpathlineto{\pgfqpoint{1.429749in}{1.031294in}}%
\pgfpathlineto{\pgfqpoint{1.431585in}{1.137459in}}%
\pgfpathlineto{\pgfqpoint{1.433419in}{1.176301in}}%
\pgfpathlineto{\pgfqpoint{1.437087in}{1.023239in}}%
\pgfpathlineto{\pgfqpoint{1.438920in}{1.168046in}}%
\pgfpathlineto{\pgfqpoint{1.440752in}{1.202008in}}%
\pgfpathlineto{\pgfqpoint{1.442587in}{1.384064in}}%
\pgfpathlineto{\pgfqpoint{1.448088in}{1.099670in}}%
\pgfpathlineto{\pgfqpoint{1.449922in}{1.246283in}}%
\pgfpathlineto{\pgfqpoint{1.453590in}{1.226661in}}%
\pgfpathlineto{\pgfqpoint{1.455424in}{1.263183in}}%
\pgfpathlineto{\pgfqpoint{1.457277in}{1.175297in}}%
\pgfpathlineto{\pgfqpoint{1.459481in}{1.229948in}}%
\pgfpathlineto{\pgfqpoint{1.461314in}{1.053388in}}%
\pgfpathlineto{\pgfqpoint{1.463148in}{1.119819in}}%
\pgfpathlineto{\pgfqpoint{1.464981in}{0.999415in}}%
\pgfpathlineto{\pgfqpoint{1.466813in}{1.487932in}}%
\pgfpathlineto{\pgfqpoint{1.472313in}{1.085242in}}%
\pgfpathlineto{\pgfqpoint{1.474146in}{1.262216in}}%
\pgfpathlineto{\pgfqpoint{1.475980in}{1.228681in}}%
\pgfpathlineto{\pgfqpoint{1.477812in}{1.015599in}}%
\pgfpathlineto{\pgfqpoint{1.479648in}{2.006573in}}%
\pgfpathlineto{\pgfqpoint{1.481481in}{1.754410in}}%
\pgfpathlineto{\pgfqpoint{1.483314in}{0.936647in}}%
\pgfpathlineto{\pgfqpoint{1.486984in}{1.164922in}}%
\pgfpathlineto{\pgfqpoint{1.488817in}{1.159377in}}%
\pgfpathlineto{\pgfqpoint{1.490650in}{1.092908in}}%
\pgfpathlineto{\pgfqpoint{1.492484in}{1.070902in}}%
\pgfpathlineto{\pgfqpoint{1.494318in}{0.980257in}}%
\pgfpathlineto{\pgfqpoint{1.496151in}{1.178233in}}%
\pgfpathlineto{\pgfqpoint{1.497985in}{1.212873in}}%
\pgfpathlineto{\pgfqpoint{1.499818in}{1.077953in}}%
\pgfpathlineto{\pgfqpoint{1.503486in}{1.267034in}}%
\pgfpathlineto{\pgfqpoint{1.505321in}{1.082269in}}%
\pgfpathlineto{\pgfqpoint{1.507154in}{1.092494in}}%
\pgfpathlineto{\pgfqpoint{1.508989in}{1.132515in}}%
\pgfpathlineto{\pgfqpoint{1.510822in}{0.911943in}}%
\pgfpathlineto{\pgfqpoint{1.512654in}{1.099319in}}%
\pgfpathlineto{\pgfqpoint{1.514489in}{1.161133in}}%
\pgfpathlineto{\pgfqpoint{1.516322in}{1.102781in}}%
\pgfpathlineto{\pgfqpoint{1.518156in}{0.998825in}}%
\pgfpathlineto{\pgfqpoint{1.523657in}{1.369636in}}%
\pgfpathlineto{\pgfqpoint{1.525491in}{1.289366in}}%
\pgfpathlineto{\pgfqpoint{1.527325in}{1.364379in}}%
\pgfpathlineto{\pgfqpoint{1.529158in}{1.253961in}}%
\pgfpathlineto{\pgfqpoint{1.531015in}{1.482964in}}%
\pgfpathlineto{\pgfqpoint{1.532848in}{1.217289in}}%
\pgfpathlineto{\pgfqpoint{1.534681in}{1.226310in}}%
\pgfpathlineto{\pgfqpoint{1.536516in}{1.525370in}}%
\pgfpathlineto{\pgfqpoint{1.540182in}{1.090762in}}%
\pgfpathlineto{\pgfqpoint{1.542016in}{1.211468in}}%
\pgfpathlineto{\pgfqpoint{1.543849in}{1.068230in}}%
\pgfpathlineto{\pgfqpoint{1.547517in}{1.425528in}}%
\pgfpathlineto{\pgfqpoint{1.549349in}{1.303756in}}%
\pgfpathlineto{\pgfqpoint{1.551181in}{1.381981in}}%
\pgfpathlineto{\pgfqpoint{1.553015in}{1.334469in}}%
\pgfpathlineto{\pgfqpoint{1.554849in}{1.321509in}}%
\pgfpathlineto{\pgfqpoint{1.556681in}{1.274988in}}%
\pgfpathlineto{\pgfqpoint{1.558518in}{1.349713in}}%
\pgfpathlineto{\pgfqpoint{1.560351in}{1.174796in}}%
\pgfpathlineto{\pgfqpoint{1.562183in}{1.204392in}}%
\pgfpathlineto{\pgfqpoint{1.564017in}{1.295238in}}%
\pgfpathlineto{\pgfqpoint{1.565850in}{1.920482in}}%
\pgfpathlineto{\pgfqpoint{1.567684in}{2.088135in}}%
\pgfpathlineto{\pgfqpoint{1.569520in}{2.135358in}}%
\pgfpathlineto{\pgfqpoint{1.571352in}{2.117430in}}%
\pgfpathlineto{\pgfqpoint{1.575017in}{1.991731in}}%
\pgfpathlineto{\pgfqpoint{1.578712in}{2.385790in}}%
\pgfpathlineto{\pgfqpoint{1.580545in}{2.154278in}}%
\pgfpathlineto{\pgfqpoint{1.582379in}{2.229529in}}%
\pgfpathlineto{\pgfqpoint{1.584213in}{2.219003in}}%
\pgfpathlineto{\pgfqpoint{1.586048in}{2.173461in}}%
\pgfpathlineto{\pgfqpoint{1.589716in}{1.343314in}}%
\pgfpathlineto{\pgfqpoint{1.593382in}{1.132867in}}%
\pgfpathlineto{\pgfqpoint{1.595217in}{1.185999in}}%
\pgfpathlineto{\pgfqpoint{1.597050in}{1.140834in}}%
\pgfpathlineto{\pgfqpoint{1.598882in}{1.236234in}}%
\pgfpathlineto{\pgfqpoint{1.600716in}{2.220007in}}%
\pgfpathlineto{\pgfqpoint{1.602565in}{2.084020in}}%
\pgfpathlineto{\pgfqpoint{1.604399in}{2.099125in}}%
\pgfpathlineto{\pgfqpoint{1.606234in}{2.076555in}}%
\pgfpathlineto{\pgfqpoint{1.608066in}{2.084283in}}%
\pgfpathlineto{\pgfqpoint{1.609901in}{2.215302in}}%
\pgfpathlineto{\pgfqpoint{1.611736in}{2.072214in}}%
\pgfpathlineto{\pgfqpoint{1.613569in}{2.225351in}}%
\pgfpathlineto{\pgfqpoint{1.615402in}{2.118221in}}%
\pgfpathlineto{\pgfqpoint{1.617235in}{2.101007in}}%
\pgfpathlineto{\pgfqpoint{1.619070in}{2.177752in}}%
\pgfpathlineto{\pgfqpoint{1.620903in}{2.181954in}}%
\pgfpathlineto{\pgfqpoint{1.626403in}{1.094463in}}%
\pgfpathlineto{\pgfqpoint{1.628237in}{1.137283in}}%
\pgfpathlineto{\pgfqpoint{1.630070in}{1.216135in}}%
\pgfpathlineto{\pgfqpoint{1.631903in}{1.150996in}}%
\pgfpathlineto{\pgfqpoint{1.633737in}{1.274022in}}%
\pgfpathlineto{\pgfqpoint{1.635570in}{1.224716in}}%
\pgfpathlineto{\pgfqpoint{1.637403in}{1.308286in}}%
\pgfpathlineto{\pgfqpoint{1.639237in}{1.318925in}}%
\pgfpathlineto{\pgfqpoint{1.642905in}{0.961099in}}%
\pgfpathlineto{\pgfqpoint{1.644738in}{1.003404in}}%
\pgfpathlineto{\pgfqpoint{1.648404in}{1.246760in}}%
\pgfpathlineto{\pgfqpoint{1.650238in}{1.113270in}}%
\pgfpathlineto{\pgfqpoint{1.652073in}{1.205032in}}%
\pgfpathlineto{\pgfqpoint{1.653906in}{1.088203in}}%
\pgfpathlineto{\pgfqpoint{1.655738in}{1.330680in}}%
\pgfpathlineto{\pgfqpoint{1.659405in}{1.214930in}}%
\pgfpathlineto{\pgfqpoint{1.661238in}{1.358319in}}%
\pgfpathlineto{\pgfqpoint{1.663073in}{1.318448in}}%
\pgfpathlineto{\pgfqpoint{1.664906in}{1.330204in}}%
\pgfpathlineto{\pgfqpoint{1.666740in}{1.293419in}}%
\pgfpathlineto{\pgfqpoint{1.668574in}{1.187555in}}%
\pgfpathlineto{\pgfqpoint{1.670408in}{1.339111in}}%
\pgfpathlineto{\pgfqpoint{1.672240in}{1.212559in}}%
\pgfpathlineto{\pgfqpoint{1.674074in}{1.192486in}}%
\pgfpathlineto{\pgfqpoint{1.675908in}{1.292239in}}%
\pgfpathlineto{\pgfqpoint{1.677742in}{1.175862in}}%
\pgfpathlineto{\pgfqpoint{1.679575in}{1.524278in}}%
\pgfpathlineto{\pgfqpoint{1.681409in}{2.329019in}}%
\pgfpathlineto{\pgfqpoint{1.685075in}{1.986261in}}%
\pgfpathlineto{\pgfqpoint{1.686909in}{1.267536in}}%
\pgfpathlineto{\pgfqpoint{1.688743in}{1.353326in}}%
\pgfpathlineto{\pgfqpoint{1.690577in}{1.222195in}}%
\pgfpathlineto{\pgfqpoint{1.692411in}{1.172625in}}%
\pgfpathlineto{\pgfqpoint{1.694244in}{1.195020in}}%
\pgfpathlineto{\pgfqpoint{1.696076in}{1.164345in}}%
\pgfpathlineto{\pgfqpoint{1.697909in}{1.208507in}}%
\pgfpathlineto{\pgfqpoint{1.699744in}{1.574902in}}%
\pgfpathlineto{\pgfqpoint{1.701578in}{1.477419in}}%
\pgfpathlineto{\pgfqpoint{1.703412in}{1.219083in}}%
\pgfpathlineto{\pgfqpoint{1.705247in}{1.361268in}}%
\pgfpathlineto{\pgfqpoint{1.707080in}{1.268615in}}%
\pgfpathlineto{\pgfqpoint{1.710745in}{1.311786in}}%
\pgfpathlineto{\pgfqpoint{1.712579in}{1.307282in}}%
\pgfpathlineto{\pgfqpoint{1.714412in}{1.367729in}}%
\pgfpathlineto{\pgfqpoint{1.716246in}{1.267298in}}%
\pgfpathlineto{\pgfqpoint{1.718078in}{1.320154in}}%
\pgfpathlineto{\pgfqpoint{1.719912in}{1.241026in}}%
\pgfpathlineto{\pgfqpoint{1.721745in}{1.353326in}}%
\pgfpathlineto{\pgfqpoint{1.723579in}{1.252255in}}%
\pgfpathlineto{\pgfqpoint{1.725413in}{1.259682in}}%
\pgfpathlineto{\pgfqpoint{1.727245in}{1.202246in}}%
\pgfpathlineto{\pgfqpoint{1.729080in}{1.195396in}}%
\pgfpathlineto{\pgfqpoint{1.730913in}{1.052033in}}%
\pgfpathlineto{\pgfqpoint{1.732745in}{1.267624in}}%
\pgfpathlineto{\pgfqpoint{1.734578in}{1.143656in}}%
\pgfpathlineto{\pgfqpoint{1.738246in}{1.307784in}}%
\pgfpathlineto{\pgfqpoint{1.740080in}{1.190779in}}%
\pgfpathlineto{\pgfqpoint{1.741914in}{1.490718in}}%
\pgfpathlineto{\pgfqpoint{1.743747in}{1.342047in}}%
\pgfpathlineto{\pgfqpoint{1.745581in}{1.308900in}}%
\pgfpathlineto{\pgfqpoint{1.747414in}{1.156491in}}%
\pgfpathlineto{\pgfqpoint{1.749249in}{1.336138in}}%
\pgfpathlineto{\pgfqpoint{1.751084in}{1.295087in}}%
\pgfpathlineto{\pgfqpoint{1.752918in}{1.298650in}}%
\pgfpathlineto{\pgfqpoint{1.754750in}{1.439066in}}%
\pgfpathlineto{\pgfqpoint{1.756586in}{1.238856in}}%
\pgfpathlineto{\pgfqpoint{1.758418in}{1.359172in}}%
\pgfpathlineto{\pgfqpoint{1.762086in}{1.262304in}}%
\pgfpathlineto{\pgfqpoint{1.763919in}{1.389032in}}%
\pgfpathlineto{\pgfqpoint{1.765752in}{1.300645in}}%
\pgfpathlineto{\pgfqpoint{1.767584in}{1.365759in}}%
\pgfpathlineto{\pgfqpoint{1.769418in}{1.488547in}}%
\pgfpathlineto{\pgfqpoint{1.771251in}{1.353125in}}%
\pgfpathlineto{\pgfqpoint{1.773085in}{1.071454in}}%
\pgfpathlineto{\pgfqpoint{1.776754in}{1.204003in}}%
\pgfpathlineto{\pgfqpoint{1.778589in}{1.116595in}}%
\pgfpathlineto{\pgfqpoint{1.780423in}{1.329526in}}%
\pgfpathlineto{\pgfqpoint{1.784088in}{1.396785in}}%
\pgfpathlineto{\pgfqpoint{1.785923in}{1.105604in}}%
\pgfpathlineto{\pgfqpoint{1.791425in}{1.381128in}}%
\pgfpathlineto{\pgfqpoint{1.793259in}{1.368318in}}%
\pgfpathlineto{\pgfqpoint{1.795091in}{1.368958in}}%
\pgfpathlineto{\pgfqpoint{1.796923in}{1.396961in}}%
\pgfpathlineto{\pgfqpoint{1.804261in}{1.188759in}}%
\pgfpathlineto{\pgfqpoint{1.806094in}{1.199950in}}%
\pgfpathlineto{\pgfqpoint{1.807926in}{1.299616in}}%
\pgfpathlineto{\pgfqpoint{1.809760in}{1.281048in}}%
\pgfpathlineto{\pgfqpoint{1.811593in}{1.423697in}}%
\pgfpathlineto{\pgfqpoint{1.813426in}{1.365082in}}%
\pgfpathlineto{\pgfqpoint{1.815261in}{1.409156in}}%
\pgfpathlineto{\pgfqpoint{1.818926in}{1.198332in}}%
\pgfpathlineto{\pgfqpoint{1.820760in}{1.233950in}}%
\pgfpathlineto{\pgfqpoint{1.822594in}{1.157520in}}%
\pgfpathlineto{\pgfqpoint{1.824427in}{1.136016in}}%
\pgfpathlineto{\pgfqpoint{1.826262in}{1.382157in}}%
\pgfpathlineto{\pgfqpoint{1.828096in}{1.172475in}}%
\pgfpathlineto{\pgfqpoint{1.829929in}{1.186727in}}%
\pgfpathlineto{\pgfqpoint{1.831763in}{1.150055in}}%
\pgfpathlineto{\pgfqpoint{1.833596in}{1.177857in}}%
\pgfpathlineto{\pgfqpoint{1.835429in}{1.140005in}}%
\pgfpathlineto{\pgfqpoint{1.839098in}{1.297094in}}%
\pgfpathlineto{\pgfqpoint{1.840930in}{1.399458in}}%
\pgfpathlineto{\pgfqpoint{1.842762in}{1.623216in}}%
\pgfpathlineto{\pgfqpoint{1.844596in}{2.350285in}}%
\pgfpathlineto{\pgfqpoint{1.846431in}{2.251058in}}%
\pgfpathlineto{\pgfqpoint{1.848264in}{2.248950in}}%
\pgfpathlineto{\pgfqpoint{1.850099in}{2.485719in}}%
\pgfpathlineto{\pgfqpoint{1.853767in}{2.166297in}}%
\pgfpathlineto{\pgfqpoint{1.855603in}{2.253354in}}%
\pgfpathlineto{\pgfqpoint{1.857436in}{2.148080in}}%
\pgfpathlineto{\pgfqpoint{1.859269in}{2.115347in}}%
\pgfpathlineto{\pgfqpoint{1.861102in}{2.146311in}}%
\pgfpathlineto{\pgfqpoint{1.862937in}{2.134330in}}%
\pgfpathlineto{\pgfqpoint{1.864769in}{2.200284in}}%
\pgfpathlineto{\pgfqpoint{1.868437in}{1.999159in}}%
\pgfpathlineto{\pgfqpoint{1.870270in}{2.312383in}}%
\pgfpathlineto{\pgfqpoint{1.872104in}{1.429631in}}%
\pgfpathlineto{\pgfqpoint{1.873938in}{1.215495in}}%
\pgfpathlineto{\pgfqpoint{1.875771in}{1.158047in}}%
\pgfpathlineto{\pgfqpoint{1.877984in}{1.237212in}}%
\pgfpathlineto{\pgfqpoint{1.881649in}{1.528895in}}%
\pgfpathlineto{\pgfqpoint{1.883483in}{2.110969in}}%
\pgfpathlineto{\pgfqpoint{1.885316in}{2.210547in}}%
\pgfpathlineto{\pgfqpoint{1.887148in}{2.115573in}}%
\pgfpathlineto{\pgfqpoint{1.888982in}{2.242778in}}%
\pgfpathlineto{\pgfqpoint{1.890815in}{2.198967in}}%
\pgfpathlineto{\pgfqpoint{1.894481in}{2.279011in}}%
\pgfpathlineto{\pgfqpoint{1.896315in}{2.248536in}}%
\pgfpathlineto{\pgfqpoint{1.898148in}{2.142999in}}%
\pgfpathlineto{\pgfqpoint{1.899983in}{2.221060in}}%
\pgfpathlineto{\pgfqpoint{1.901816in}{2.358101in}}%
\pgfpathlineto{\pgfqpoint{1.903651in}{2.255035in}}%
\pgfpathlineto{\pgfqpoint{1.907317in}{2.443075in}}%
\pgfpathlineto{\pgfqpoint{1.909150in}{2.615270in}}%
\pgfpathlineto{\pgfqpoint{1.910984in}{2.267606in}}%
\pgfpathlineto{\pgfqpoint{1.912818in}{2.255060in}}%
\pgfpathlineto{\pgfqpoint{1.916485in}{2.465708in}}%
\pgfpathlineto{\pgfqpoint{1.920151in}{2.112261in}}%
\pgfpathlineto{\pgfqpoint{1.921984in}{2.204136in}}%
\pgfpathlineto{\pgfqpoint{1.923819in}{2.207134in}}%
\pgfpathlineto{\pgfqpoint{1.925653in}{2.334652in}}%
\pgfpathlineto{\pgfqpoint{1.927486in}{1.253836in}}%
\pgfpathlineto{\pgfqpoint{1.929321in}{1.344079in}}%
\pgfpathlineto{\pgfqpoint{1.931154in}{1.131248in}}%
\pgfpathlineto{\pgfqpoint{1.932987in}{1.597384in}}%
\pgfpathlineto{\pgfqpoint{1.934824in}{1.776378in}}%
\pgfpathlineto{\pgfqpoint{1.936657in}{1.695105in}}%
\pgfpathlineto{\pgfqpoint{1.938490in}{1.516613in}}%
\pgfpathlineto{\pgfqpoint{1.940325in}{1.617219in}}%
\pgfpathlineto{\pgfqpoint{1.942159in}{1.541855in}}%
\pgfpathlineto{\pgfqpoint{1.945827in}{1.306052in}}%
\pgfpathlineto{\pgfqpoint{1.947660in}{1.349625in}}%
\pgfpathlineto{\pgfqpoint{1.949493in}{1.056035in}}%
\pgfpathlineto{\pgfqpoint{1.951341in}{1.290269in}}%
\pgfpathlineto{\pgfqpoint{1.953174in}{1.252845in}}%
\pgfpathlineto{\pgfqpoint{1.955006in}{1.070965in}}%
\pgfpathlineto{\pgfqpoint{1.956840in}{1.311874in}}%
\pgfpathlineto{\pgfqpoint{1.960506in}{1.188935in}}%
\pgfpathlineto{\pgfqpoint{1.962341in}{1.290771in}}%
\pgfpathlineto{\pgfqpoint{1.964174in}{1.229798in}}%
\pgfpathlineto{\pgfqpoint{1.966007in}{1.209937in}}%
\pgfpathlineto{\pgfqpoint{1.967841in}{1.210916in}}%
\pgfpathlineto{\pgfqpoint{1.971508in}{2.044324in}}%
\pgfpathlineto{\pgfqpoint{1.973343in}{2.051413in}}%
\pgfpathlineto{\pgfqpoint{1.977011in}{2.339031in}}%
\pgfpathlineto{\pgfqpoint{1.978845in}{2.248361in}}%
\pgfpathlineto{\pgfqpoint{1.980679in}{1.942099in}}%
\pgfpathlineto{\pgfqpoint{1.982512in}{2.126012in}}%
\pgfpathlineto{\pgfqpoint{1.984346in}{2.094283in}}%
\pgfpathlineto{\pgfqpoint{1.986181in}{2.142372in}}%
\pgfpathlineto{\pgfqpoint{1.988015in}{2.394158in}}%
\pgfpathlineto{\pgfqpoint{1.989847in}{2.089189in}}%
\pgfpathlineto{\pgfqpoint{1.991680in}{2.035868in}}%
\pgfpathlineto{\pgfqpoint{1.995348in}{2.190862in}}%
\pgfpathlineto{\pgfqpoint{1.997182in}{2.178981in}}%
\pgfpathlineto{\pgfqpoint{1.999016in}{2.002132in}}%
\pgfpathlineto{\pgfqpoint{2.000873in}{2.293903in}}%
\pgfpathlineto{\pgfqpoint{2.002706in}{1.409507in}}%
\pgfpathlineto{\pgfqpoint{2.004539in}{1.377778in}}%
\pgfpathlineto{\pgfqpoint{2.006374in}{1.125051in}}%
\pgfpathlineto{\pgfqpoint{2.008208in}{1.186727in}}%
\pgfpathlineto{\pgfqpoint{2.010042in}{1.146391in}}%
\pgfpathlineto{\pgfqpoint{2.011876in}{1.228330in}}%
\pgfpathlineto{\pgfqpoint{2.013709in}{1.123733in}}%
\pgfpathlineto{\pgfqpoint{2.015542in}{1.282014in}}%
\pgfpathlineto{\pgfqpoint{2.017375in}{0.981963in}}%
\pgfpathlineto{\pgfqpoint{2.019208in}{1.294886in}}%
\pgfpathlineto{\pgfqpoint{2.022876in}{1.117849in}}%
\pgfpathlineto{\pgfqpoint{2.024711in}{1.052033in}}%
\pgfpathlineto{\pgfqpoint{2.026543in}{1.145149in}}%
\pgfpathlineto{\pgfqpoint{2.028377in}{1.134397in}}%
\pgfpathlineto{\pgfqpoint{2.030211in}{1.103923in}}%
\pgfpathlineto{\pgfqpoint{2.032044in}{1.309227in}}%
\pgfpathlineto{\pgfqpoint{2.033877in}{1.243849in}}%
\pgfpathlineto{\pgfqpoint{2.035712in}{1.300093in}}%
\pgfpathlineto{\pgfqpoint{2.039381in}{1.157143in}}%
\pgfpathlineto{\pgfqpoint{2.041214in}{1.375984in}}%
\pgfpathlineto{\pgfqpoint{2.043048in}{2.166347in}}%
\pgfpathlineto{\pgfqpoint{2.044882in}{1.992083in}}%
\pgfpathlineto{\pgfqpoint{2.046716in}{2.385577in}}%
\pgfpathlineto{\pgfqpoint{2.050384in}{2.229491in}}%
\pgfpathlineto{\pgfqpoint{2.054049in}{2.055252in}}%
\pgfpathlineto{\pgfqpoint{2.055883in}{2.069216in}}%
\pgfpathlineto{\pgfqpoint{2.057717in}{2.196671in}}%
\pgfpathlineto{\pgfqpoint{2.059552in}{1.349775in}}%
\pgfpathlineto{\pgfqpoint{2.061385in}{1.258980in}}%
\pgfpathlineto{\pgfqpoint{2.063219in}{1.267586in}}%
\pgfpathlineto{\pgfqpoint{2.065051in}{1.313580in}}%
\pgfpathlineto{\pgfqpoint{2.066885in}{1.268477in}}%
\pgfpathlineto{\pgfqpoint{2.068720in}{1.133569in}}%
\pgfpathlineto{\pgfqpoint{2.070554in}{1.164809in}}%
\pgfpathlineto{\pgfqpoint{2.072387in}{1.166453in}}%
\pgfpathlineto{\pgfqpoint{2.074222in}{1.187818in}}%
\pgfpathlineto{\pgfqpoint{2.076056in}{1.271827in}}%
\pgfpathlineto{\pgfqpoint{2.077889in}{1.118062in}}%
\pgfpathlineto{\pgfqpoint{2.079722in}{1.189725in}}%
\pgfpathlineto{\pgfqpoint{2.081556in}{0.964386in}}%
\pgfpathlineto{\pgfqpoint{2.083390in}{1.215257in}}%
\pgfpathlineto{\pgfqpoint{2.085224in}{1.270472in}}%
\pgfpathlineto{\pgfqpoint{2.087060in}{1.137484in}}%
\pgfpathlineto{\pgfqpoint{2.088893in}{1.109217in}}%
\pgfpathlineto{\pgfqpoint{2.090725in}{1.192511in}}%
\pgfpathlineto{\pgfqpoint{2.092559in}{1.181470in}}%
\pgfpathlineto{\pgfqpoint{2.094392in}{1.283131in}}%
\pgfpathlineto{\pgfqpoint{2.096225in}{1.178534in}}%
\pgfpathlineto{\pgfqpoint{2.098059in}{1.221894in}}%
\pgfpathlineto{\pgfqpoint{2.099907in}{1.421238in}}%
\pgfpathlineto{\pgfqpoint{2.101741in}{1.193903in}}%
\pgfpathlineto{\pgfqpoint{2.103575in}{1.272002in}}%
\pgfpathlineto{\pgfqpoint{2.105407in}{1.269945in}}%
\pgfpathlineto{\pgfqpoint{2.109075in}{1.055157in}}%
\pgfpathlineto{\pgfqpoint{2.110908in}{1.064315in}}%
\pgfpathlineto{\pgfqpoint{2.112743in}{1.028672in}}%
\pgfpathlineto{\pgfqpoint{2.114577in}{1.128701in}}%
\pgfpathlineto{\pgfqpoint{2.116411in}{1.030943in}}%
\pgfpathlineto{\pgfqpoint{2.118245in}{1.024005in}}%
\pgfpathlineto{\pgfqpoint{2.121912in}{1.177029in}}%
\pgfpathlineto{\pgfqpoint{2.125580in}{1.027530in}}%
\pgfpathlineto{\pgfqpoint{2.127413in}{1.272906in}}%
\pgfpathlineto{\pgfqpoint{2.131080in}{1.159577in}}%
\pgfpathlineto{\pgfqpoint{2.132913in}{1.205709in}}%
\pgfpathlineto{\pgfqpoint{2.136582in}{1.199599in}}%
\pgfpathlineto{\pgfqpoint{2.138416in}{1.144447in}}%
\pgfpathlineto{\pgfqpoint{2.140250in}{1.135514in}}%
\pgfpathlineto{\pgfqpoint{2.142083in}{1.154910in}}%
\pgfpathlineto{\pgfqpoint{2.143917in}{1.290219in}}%
\pgfpathlineto{\pgfqpoint{2.145750in}{1.320656in}}%
\pgfpathlineto{\pgfqpoint{2.147586in}{1.234891in}}%
\pgfpathlineto{\pgfqpoint{2.149421in}{1.246634in}}%
\pgfpathlineto{\pgfqpoint{2.153089in}{1.157821in}}%
\pgfpathlineto{\pgfqpoint{2.154922in}{1.071454in}}%
\pgfpathlineto{\pgfqpoint{2.156755in}{1.217753in}}%
\pgfpathlineto{\pgfqpoint{2.158588in}{1.094639in}}%
\pgfpathlineto{\pgfqpoint{2.160423in}{1.142239in}}%
\pgfpathlineto{\pgfqpoint{2.162254in}{1.275703in}}%
\pgfpathlineto{\pgfqpoint{2.164087in}{1.283432in}}%
\pgfpathlineto{\pgfqpoint{2.167753in}{1.059937in}}%
\pgfpathlineto{\pgfqpoint{2.169586in}{1.231240in}}%
\pgfpathlineto{\pgfqpoint{2.171421in}{1.087852in}}%
\pgfpathlineto{\pgfqpoint{2.173255in}{1.199775in}}%
\pgfpathlineto{\pgfqpoint{2.175088in}{1.118000in}}%
\pgfpathlineto{\pgfqpoint{2.178756in}{1.935839in}}%
\pgfpathlineto{\pgfqpoint{2.180589in}{2.126338in}}%
\pgfpathlineto{\pgfqpoint{2.182425in}{2.032305in}}%
\pgfpathlineto{\pgfqpoint{2.184258in}{2.207373in}}%
\pgfpathlineto{\pgfqpoint{2.186092in}{2.025167in}}%
\pgfpathlineto{\pgfqpoint{2.187926in}{1.065030in}}%
\pgfpathlineto{\pgfqpoint{2.189760in}{1.289805in}}%
\pgfpathlineto{\pgfqpoint{2.191595in}{1.217251in}}%
\pgfpathlineto{\pgfqpoint{2.193430in}{1.218544in}}%
\pgfpathlineto{\pgfqpoint{2.195262in}{1.127196in}}%
\pgfpathlineto{\pgfqpoint{2.197096in}{1.322387in}}%
\pgfpathlineto{\pgfqpoint{2.198950in}{1.382684in}}%
\pgfpathlineto{\pgfqpoint{2.200782in}{1.168071in}}%
\pgfpathlineto{\pgfqpoint{2.202617in}{1.245054in}}%
\pgfpathlineto{\pgfqpoint{2.204450in}{1.213086in}}%
\pgfpathlineto{\pgfqpoint{2.206283in}{1.372283in}}%
\pgfpathlineto{\pgfqpoint{2.208117in}{1.284661in}}%
\pgfpathlineto{\pgfqpoint{2.209951in}{1.447936in}}%
\pgfpathlineto{\pgfqpoint{2.211786in}{1.227778in}}%
\pgfpathlineto{\pgfqpoint{2.213619in}{1.462640in}}%
\pgfpathlineto{\pgfqpoint{2.219122in}{1.289391in}}%
\pgfpathlineto{\pgfqpoint{2.220957in}{1.236146in}}%
\pgfpathlineto{\pgfqpoint{2.222791in}{1.074164in}}%
\pgfpathlineto{\pgfqpoint{2.224623in}{1.046537in}}%
\pgfpathlineto{\pgfqpoint{2.226457in}{1.119242in}}%
\pgfpathlineto{\pgfqpoint{2.228291in}{0.930863in}}%
\pgfpathlineto{\pgfqpoint{2.230125in}{1.210201in}}%
\pgfpathlineto{\pgfqpoint{2.231958in}{1.108038in}}%
\pgfpathlineto{\pgfqpoint{2.233792in}{1.210966in}}%
\pgfpathlineto{\pgfqpoint{2.237460in}{1.083335in}}%
\pgfpathlineto{\pgfqpoint{2.239295in}{1.061643in}}%
\pgfpathlineto{\pgfqpoint{2.241128in}{1.072520in}}%
\pgfpathlineto{\pgfqpoint{2.242963in}{1.029174in}}%
\pgfpathlineto{\pgfqpoint{2.246630in}{1.215495in}}%
\pgfpathlineto{\pgfqpoint{2.248478in}{0.970145in}}%
\pgfpathlineto{\pgfqpoint{2.250312in}{1.126255in}}%
\pgfpathlineto{\pgfqpoint{2.252145in}{1.032147in}}%
\pgfpathlineto{\pgfqpoint{2.253979in}{1.223161in}}%
\pgfpathlineto{\pgfqpoint{2.255812in}{1.093911in}}%
\pgfpathlineto{\pgfqpoint{2.257648in}{1.120710in}}%
\pgfpathlineto{\pgfqpoint{2.261314in}{0.861283in}}%
\pgfpathlineto{\pgfqpoint{2.263148in}{0.931867in}}%
\pgfpathlineto{\pgfqpoint{2.264982in}{1.100548in}}%
\pgfpathlineto{\pgfqpoint{2.266816in}{1.111212in}}%
\pgfpathlineto{\pgfqpoint{2.268648in}{0.909271in}}%
\pgfpathlineto{\pgfqpoint{2.270482in}{0.943171in}}%
\pgfpathlineto{\pgfqpoint{2.272317in}{1.219108in}}%
\pgfpathlineto{\pgfqpoint{2.274151in}{1.189462in}}%
\pgfpathlineto{\pgfqpoint{2.275986in}{1.033640in}}%
\pgfpathlineto{\pgfqpoint{2.277818in}{1.310807in}}%
\pgfpathlineto{\pgfqpoint{2.279652in}{0.981549in}}%
\pgfpathlineto{\pgfqpoint{2.281486in}{0.945316in}}%
\pgfpathlineto{\pgfqpoint{2.285153in}{0.801024in}}%
\pgfpathlineto{\pgfqpoint{2.286987in}{0.983983in}}%
\pgfpathlineto{\pgfqpoint{2.288821in}{0.738821in}}%
\pgfpathlineto{\pgfqpoint{2.290654in}{0.790410in}}%
\pgfpathlineto{\pgfqpoint{2.292489in}{0.986342in}}%
\pgfpathlineto{\pgfqpoint{2.294323in}{1.005499in}}%
\pgfpathlineto{\pgfqpoint{2.296157in}{0.935743in}}%
\pgfpathlineto{\pgfqpoint{2.298007in}{0.963395in}}%
\pgfpathlineto{\pgfqpoint{2.299841in}{0.803746in}}%
\pgfpathlineto{\pgfqpoint{2.301675in}{0.839716in}}%
\pgfpathlineto{\pgfqpoint{2.305342in}{0.944613in}}%
\pgfpathlineto{\pgfqpoint{2.307176in}{0.989014in}}%
\pgfpathlineto{\pgfqpoint{2.309010in}{0.999000in}}%
\pgfpathlineto{\pgfqpoint{2.310843in}{1.024005in}}%
\pgfpathlineto{\pgfqpoint{2.312677in}{1.002288in}}%
\pgfpathlineto{\pgfqpoint{2.316343in}{0.874569in}}%
\pgfpathlineto{\pgfqpoint{2.320010in}{1.113445in}}%
\pgfpathlineto{\pgfqpoint{2.323678in}{1.022035in}}%
\pgfpathlineto{\pgfqpoint{2.325512in}{1.086973in}}%
\pgfpathlineto{\pgfqpoint{2.327345in}{1.044480in}}%
\pgfpathlineto{\pgfqpoint{2.329179in}{0.861345in}}%
\pgfpathlineto{\pgfqpoint{2.331014in}{0.815126in}}%
\pgfpathlineto{\pgfqpoint{2.332847in}{0.970822in}}%
\pgfpathlineto{\pgfqpoint{2.334681in}{0.995563in}}%
\pgfpathlineto{\pgfqpoint{2.336516in}{0.912408in}}%
\pgfpathlineto{\pgfqpoint{2.338350in}{1.038232in}}%
\pgfpathlineto{\pgfqpoint{2.340183in}{1.060288in}}%
\pgfpathlineto{\pgfqpoint{2.342017in}{0.985752in}}%
\pgfpathlineto{\pgfqpoint{2.343851in}{1.101602in}}%
\pgfpathlineto{\pgfqpoint{2.345685in}{1.035434in}}%
\pgfpathlineto{\pgfqpoint{2.347518in}{1.112944in}}%
\pgfpathlineto{\pgfqpoint{2.349353in}{1.019890in}}%
\pgfpathlineto{\pgfqpoint{2.351186in}{1.018422in}}%
\pgfpathlineto{\pgfqpoint{2.353019in}{0.900564in}}%
\pgfpathlineto{\pgfqpoint{2.354854in}{1.045132in}}%
\pgfpathlineto{\pgfqpoint{2.356687in}{1.070162in}}%
\pgfpathlineto{\pgfqpoint{2.358520in}{1.025623in}}%
\pgfpathlineto{\pgfqpoint{2.360354in}{1.040202in}}%
\pgfpathlineto{\pgfqpoint{2.362187in}{0.987308in}}%
\pgfpathlineto{\pgfqpoint{2.364021in}{0.991385in}}%
\pgfpathlineto{\pgfqpoint{2.365856in}{0.912056in}}%
\pgfpathlineto{\pgfqpoint{2.369522in}{0.871771in}}%
\pgfpathlineto{\pgfqpoint{2.371356in}{0.873389in}}%
\pgfpathlineto{\pgfqpoint{2.373192in}{1.075280in}}%
\pgfpathlineto{\pgfqpoint{2.375026in}{0.964913in}}%
\pgfpathlineto{\pgfqpoint{2.376861in}{0.979291in}}%
\pgfpathlineto{\pgfqpoint{2.378694in}{1.046011in}}%
\pgfpathlineto{\pgfqpoint{2.380528in}{0.849941in}}%
\pgfpathlineto{\pgfqpoint{2.382365in}{0.875648in}}%
\pgfpathlineto{\pgfqpoint{2.386030in}{0.971964in}}%
\pgfpathlineto{\pgfqpoint{2.389698in}{0.835137in}}%
\pgfpathlineto{\pgfqpoint{2.391531in}{0.648878in}}%
\pgfpathlineto{\pgfqpoint{2.393366in}{0.690744in}}%
\pgfpathlineto{\pgfqpoint{2.397033in}{1.258428in}}%
\pgfpathlineto{\pgfqpoint{2.400702in}{0.871595in}}%
\pgfpathlineto{\pgfqpoint{2.402535in}{1.079358in}}%
\pgfpathlineto{\pgfqpoint{2.404369in}{0.981925in}}%
\pgfpathlineto{\pgfqpoint{2.406204in}{1.019363in}}%
\pgfpathlineto{\pgfqpoint{2.408037in}{0.916962in}}%
\pgfpathlineto{\pgfqpoint{2.409871in}{0.967974in}}%
\pgfpathlineto{\pgfqpoint{2.411705in}{1.051242in}}%
\pgfpathlineto{\pgfqpoint{2.415371in}{0.880240in}}%
\pgfpathlineto{\pgfqpoint{2.417205in}{0.915645in}}%
\pgfpathlineto{\pgfqpoint{2.419039in}{1.045596in}}%
\pgfpathlineto{\pgfqpoint{2.420872in}{0.994622in}}%
\pgfpathlineto{\pgfqpoint{2.424540in}{1.062320in}}%
\pgfpathlineto{\pgfqpoint{2.426374in}{0.912822in}}%
\pgfpathlineto{\pgfqpoint{2.428209in}{1.008586in}}%
\pgfpathlineto{\pgfqpoint{2.430043in}{1.014457in}}%
\pgfpathlineto{\pgfqpoint{2.431877in}{1.037793in}}%
\pgfpathlineto{\pgfqpoint{2.433711in}{0.772983in}}%
\pgfpathlineto{\pgfqpoint{2.435544in}{0.991385in}}%
\pgfpathlineto{\pgfqpoint{2.437378in}{0.962630in}}%
\pgfpathlineto{\pgfqpoint{2.439213in}{1.210765in}}%
\pgfpathlineto{\pgfqpoint{2.441046in}{1.815610in}}%
\pgfpathlineto{\pgfqpoint{2.442879in}{1.656739in}}%
\pgfpathlineto{\pgfqpoint{2.444714in}{1.981004in}}%
\pgfpathlineto{\pgfqpoint{2.446550in}{1.928411in}}%
\pgfpathlineto{\pgfqpoint{2.448385in}{1.950154in}}%
\pgfpathlineto{\pgfqpoint{2.450220in}{1.290621in}}%
\pgfpathlineto{\pgfqpoint{2.452054in}{1.294096in}}%
\pgfpathlineto{\pgfqpoint{2.453888in}{1.511406in}}%
\pgfpathlineto{\pgfqpoint{2.455723in}{1.110773in}}%
\pgfpathlineto{\pgfqpoint{2.459389in}{1.204304in}}%
\pgfpathlineto{\pgfqpoint{2.461223in}{1.071078in}}%
\pgfpathlineto{\pgfqpoint{2.463057in}{1.193953in}}%
\pgfpathlineto{\pgfqpoint{2.464891in}{1.058619in}}%
\pgfpathlineto{\pgfqpoint{2.466724in}{1.107097in}}%
\pgfpathlineto{\pgfqpoint{2.468558in}{1.103522in}}%
\pgfpathlineto{\pgfqpoint{2.470392in}{1.147320in}}%
\pgfpathlineto{\pgfqpoint{2.474059in}{2.006511in}}%
\pgfpathlineto{\pgfqpoint{2.475893in}{2.000049in}}%
\pgfpathlineto{\pgfqpoint{2.477727in}{1.851692in}}%
\pgfpathlineto{\pgfqpoint{2.479561in}{1.029036in}}%
\pgfpathlineto{\pgfqpoint{2.481394in}{1.026715in}}%
\pgfpathlineto{\pgfqpoint{2.483228in}{0.866866in}}%
\pgfpathlineto{\pgfqpoint{2.485062in}{0.929897in}}%
\pgfpathlineto{\pgfqpoint{2.486896in}{1.048156in}}%
\pgfpathlineto{\pgfqpoint{2.488729in}{0.928918in}}%
\pgfpathlineto{\pgfqpoint{2.490563in}{0.961889in}}%
\pgfpathlineto{\pgfqpoint{2.492397in}{0.857995in}}%
\pgfpathlineto{\pgfqpoint{2.494230in}{0.893375in}}%
\pgfpathlineto{\pgfqpoint{2.496065in}{0.872009in}}%
\pgfpathlineto{\pgfqpoint{2.497897in}{1.098641in}}%
\pgfpathlineto{\pgfqpoint{2.499731in}{0.992953in}}%
\pgfpathlineto{\pgfqpoint{2.503403in}{1.168096in}}%
\pgfpathlineto{\pgfqpoint{2.505236in}{1.478849in}}%
\pgfpathlineto{\pgfqpoint{2.507071in}{1.334406in}}%
\pgfpathlineto{\pgfqpoint{2.508905in}{1.393523in}}%
\pgfpathlineto{\pgfqpoint{2.510738in}{1.348333in}}%
\pgfpathlineto{\pgfqpoint{2.512572in}{1.338898in}}%
\pgfpathlineto{\pgfqpoint{2.514406in}{1.346752in}}%
\pgfpathlineto{\pgfqpoint{2.518075in}{1.100398in}}%
\pgfpathlineto{\pgfqpoint{2.519909in}{1.250461in}}%
\pgfpathlineto{\pgfqpoint{2.521742in}{1.113245in}}%
\pgfpathlineto{\pgfqpoint{2.523576in}{1.221718in}}%
\pgfpathlineto{\pgfqpoint{2.527245in}{1.021332in}}%
\pgfpathlineto{\pgfqpoint{2.529079in}{1.287773in}}%
\pgfpathlineto{\pgfqpoint{2.530912in}{1.257775in}}%
\pgfpathlineto{\pgfqpoint{2.532745in}{1.174118in}}%
\pgfpathlineto{\pgfqpoint{2.534578in}{1.298738in}}%
\pgfpathlineto{\pgfqpoint{2.536412in}{1.293795in}}%
\pgfpathlineto{\pgfqpoint{2.538246in}{1.268301in}}%
\pgfpathlineto{\pgfqpoint{2.540080in}{1.260008in}}%
\pgfpathlineto{\pgfqpoint{2.541915in}{1.242444in}}%
\pgfpathlineto{\pgfqpoint{2.547418in}{2.259941in}}%
\pgfpathlineto{\pgfqpoint{2.549252in}{2.315319in}}%
\pgfpathlineto{\pgfqpoint{2.551087in}{2.307653in}}%
\pgfpathlineto{\pgfqpoint{2.552922in}{2.231060in}}%
\pgfpathlineto{\pgfqpoint{2.554755in}{2.348641in}}%
\pgfpathlineto{\pgfqpoint{2.556590in}{2.236868in}}%
\pgfpathlineto{\pgfqpoint{2.558425in}{2.250795in}}%
\pgfpathlineto{\pgfqpoint{2.560258in}{2.104909in}}%
\pgfpathlineto{\pgfqpoint{2.562091in}{2.130089in}}%
\pgfpathlineto{\pgfqpoint{2.563926in}{2.072854in}}%
\pgfpathlineto{\pgfqpoint{2.565761in}{2.329609in}}%
\pgfpathlineto{\pgfqpoint{2.567594in}{2.319472in}}%
\pgfpathlineto{\pgfqpoint{2.569431in}{2.393305in}}%
\pgfpathlineto{\pgfqpoint{2.571263in}{2.301079in}}%
\pgfpathlineto{\pgfqpoint{2.573096in}{2.430354in}}%
\pgfpathlineto{\pgfqpoint{2.574931in}{2.383406in}}%
\pgfpathlineto{\pgfqpoint{2.576766in}{2.444104in}}%
\pgfpathlineto{\pgfqpoint{2.578600in}{2.332043in}}%
\pgfpathlineto{\pgfqpoint{2.580434in}{2.401736in}}%
\pgfpathlineto{\pgfqpoint{2.584101in}{2.219856in}}%
\pgfpathlineto{\pgfqpoint{2.585936in}{2.234610in}}%
\pgfpathlineto{\pgfqpoint{2.589603in}{2.443878in}}%
\pgfpathlineto{\pgfqpoint{2.591437in}{2.428597in}}%
\pgfpathlineto{\pgfqpoint{2.595105in}{2.225941in}}%
\pgfpathlineto{\pgfqpoint{2.596940in}{2.246805in}}%
\pgfpathlineto{\pgfqpoint{2.598773in}{2.450189in}}%
\pgfpathlineto{\pgfqpoint{2.600607in}{2.379843in}}%
\pgfpathlineto{\pgfqpoint{2.602442in}{2.401799in}}%
\pgfpathlineto{\pgfqpoint{2.604276in}{2.479835in}}%
\pgfpathlineto{\pgfqpoint{2.606110in}{2.304981in}}%
\pgfpathlineto{\pgfqpoint{2.607943in}{2.478756in}}%
\pgfpathlineto{\pgfqpoint{2.611611in}{1.396873in}}%
\pgfpathlineto{\pgfqpoint{2.613445in}{1.390788in}}%
\pgfpathlineto{\pgfqpoint{2.615280in}{1.459189in}}%
\pgfpathlineto{\pgfqpoint{2.617113in}{1.454823in}}%
\pgfpathlineto{\pgfqpoint{2.618945in}{1.555894in}}%
\pgfpathlineto{\pgfqpoint{2.620781in}{1.540525in}}%
\pgfpathlineto{\pgfqpoint{2.622614in}{1.455702in}}%
\pgfpathlineto{\pgfqpoint{2.624446in}{1.550838in}}%
\pgfpathlineto{\pgfqpoint{2.626282in}{1.843588in}}%
\pgfpathlineto{\pgfqpoint{2.628117in}{1.812938in}}%
\pgfpathlineto{\pgfqpoint{2.629951in}{1.620895in}}%
\pgfpathlineto{\pgfqpoint{2.633620in}{1.527365in}}%
\pgfpathlineto{\pgfqpoint{2.637287in}{1.633027in}}%
\pgfpathlineto{\pgfqpoint{2.639121in}{1.552544in}}%
\pgfpathlineto{\pgfqpoint{2.640953in}{1.653917in}}%
\pgfpathlineto{\pgfqpoint{2.642788in}{1.496715in}}%
\pgfpathlineto{\pgfqpoint{2.646455in}{1.703511in}}%
\pgfpathlineto{\pgfqpoint{2.648288in}{1.722380in}}%
\pgfpathlineto{\pgfqpoint{2.650122in}{1.792488in}}%
\pgfpathlineto{\pgfqpoint{2.657458in}{1.602553in}}%
\pgfpathlineto{\pgfqpoint{2.659291in}{1.605614in}}%
\pgfpathlineto{\pgfqpoint{2.661125in}{1.775714in}}%
\pgfpathlineto{\pgfqpoint{2.662959in}{1.770858in}}%
\pgfpathlineto{\pgfqpoint{2.664792in}{1.387326in}}%
\pgfpathlineto{\pgfqpoint{2.666626in}{1.444498in}}%
\pgfpathlineto{\pgfqpoint{2.668460in}{1.272994in}}%
\pgfpathlineto{\pgfqpoint{2.670294in}{1.391378in}}%
\pgfpathlineto{\pgfqpoint{2.672129in}{1.301147in}}%
\pgfpathlineto{\pgfqpoint{2.675797in}{1.282077in}}%
\pgfpathlineto{\pgfqpoint{2.677631in}{1.327180in}}%
\pgfpathlineto{\pgfqpoint{2.679465in}{1.298587in}}%
\pgfpathlineto{\pgfqpoint{2.681298in}{1.296530in}}%
\pgfpathlineto{\pgfqpoint{2.683133in}{1.372433in}}%
\pgfpathlineto{\pgfqpoint{2.684967in}{1.294008in}}%
\pgfpathlineto{\pgfqpoint{2.686800in}{1.320305in}}%
\pgfpathlineto{\pgfqpoint{2.688636in}{1.226636in}}%
\pgfpathlineto{\pgfqpoint{2.690470in}{1.308173in}}%
\pgfpathlineto{\pgfqpoint{2.692303in}{1.294385in}}%
\pgfpathlineto{\pgfqpoint{2.694139in}{1.293481in}}%
\pgfpathlineto{\pgfqpoint{2.695974in}{1.375959in}}%
\pgfpathlineto{\pgfqpoint{2.697807in}{1.311610in}}%
\pgfpathlineto{\pgfqpoint{2.699642in}{1.287396in}}%
\pgfpathlineto{\pgfqpoint{2.701476in}{1.193226in}}%
\pgfpathlineto{\pgfqpoint{2.703309in}{1.280396in}}%
\pgfpathlineto{\pgfqpoint{2.705144in}{1.439856in}}%
\pgfpathlineto{\pgfqpoint{2.706978in}{1.375921in}}%
\pgfpathlineto{\pgfqpoint{2.708810in}{1.504656in}}%
\pgfpathlineto{\pgfqpoint{2.710645in}{1.347893in}}%
\pgfpathlineto{\pgfqpoint{2.712479in}{1.353301in}}%
\pgfpathlineto{\pgfqpoint{2.714312in}{1.232119in}}%
\pgfpathlineto{\pgfqpoint{2.716146in}{1.324244in}}%
\pgfpathlineto{\pgfqpoint{2.719814in}{1.198395in}}%
\pgfpathlineto{\pgfqpoint{2.721649in}{1.311221in}}%
\pgfpathlineto{\pgfqpoint{2.723486in}{1.338434in}}%
\pgfpathlineto{\pgfqpoint{2.725319in}{1.394088in}}%
\pgfpathlineto{\pgfqpoint{2.727153in}{1.236999in}}%
\pgfpathlineto{\pgfqpoint{2.728986in}{1.307934in}}%
\pgfpathlineto{\pgfqpoint{2.730821in}{1.280433in}}%
\pgfpathlineto{\pgfqpoint{2.732655in}{1.329940in}}%
\pgfpathlineto{\pgfqpoint{2.734489in}{1.281424in}}%
\pgfpathlineto{\pgfqpoint{2.736323in}{1.154584in}}%
\pgfpathlineto{\pgfqpoint{2.738156in}{1.132666in}}%
\pgfpathlineto{\pgfqpoint{2.739990in}{1.187254in}}%
\pgfpathlineto{\pgfqpoint{2.741824in}{1.146028in}}%
\pgfpathlineto{\pgfqpoint{2.743657in}{1.356763in}}%
\pgfpathlineto{\pgfqpoint{2.747324in}{2.163537in}}%
\pgfpathlineto{\pgfqpoint{2.749158in}{2.217748in}}%
\pgfpathlineto{\pgfqpoint{2.750992in}{2.141029in}}%
\pgfpathlineto{\pgfqpoint{2.754676in}{2.262789in}}%
\pgfpathlineto{\pgfqpoint{2.756510in}{2.219982in}}%
\pgfpathlineto{\pgfqpoint{2.758345in}{2.138558in}}%
\pgfpathlineto{\pgfqpoint{2.762014in}{2.354638in}}%
\pgfpathlineto{\pgfqpoint{2.763847in}{2.189005in}}%
\pgfpathlineto{\pgfqpoint{2.765683in}{2.189306in}}%
\pgfpathlineto{\pgfqpoint{2.767516in}{2.292171in}}%
\pgfpathlineto{\pgfqpoint{2.769350in}{2.295082in}}%
\pgfpathlineto{\pgfqpoint{2.771184in}{2.274657in}}%
\pgfpathlineto{\pgfqpoint{2.773018in}{2.351263in}}%
\pgfpathlineto{\pgfqpoint{2.776687in}{2.170262in}}%
\pgfpathlineto{\pgfqpoint{2.778521in}{2.590466in}}%
\pgfpathlineto{\pgfqpoint{2.780356in}{2.303626in}}%
\pgfpathlineto{\pgfqpoint{2.782191in}{2.256528in}}%
\pgfpathlineto{\pgfqpoint{2.784025in}{2.169383in}}%
\pgfpathlineto{\pgfqpoint{2.785858in}{1.276017in}}%
\pgfpathlineto{\pgfqpoint{2.787692in}{1.340253in}}%
\pgfpathlineto{\pgfqpoint{2.789527in}{1.261125in}}%
\pgfpathlineto{\pgfqpoint{2.791360in}{1.346338in}}%
\pgfpathlineto{\pgfqpoint{2.793195in}{1.252042in}}%
\pgfpathlineto{\pgfqpoint{2.795029in}{1.292829in}}%
\pgfpathlineto{\pgfqpoint{2.796863in}{1.220049in}}%
\pgfpathlineto{\pgfqpoint{2.798697in}{1.217640in}}%
\pgfpathlineto{\pgfqpoint{2.800531in}{1.317218in}}%
\pgfpathlineto{\pgfqpoint{2.802364in}{1.265968in}}%
\pgfpathlineto{\pgfqpoint{2.804199in}{1.392620in}}%
\pgfpathlineto{\pgfqpoint{2.806034in}{1.422881in}}%
\pgfpathlineto{\pgfqpoint{2.811535in}{1.242758in}}%
\pgfpathlineto{\pgfqpoint{2.813369in}{1.462539in}}%
\pgfpathlineto{\pgfqpoint{2.815202in}{1.126782in}}%
\pgfpathlineto{\pgfqpoint{2.817037in}{1.348546in}}%
\pgfpathlineto{\pgfqpoint{2.818870in}{1.245555in}}%
\pgfpathlineto{\pgfqpoint{2.820704in}{1.281286in}}%
\pgfpathlineto{\pgfqpoint{2.822539in}{1.267737in}}%
\pgfpathlineto{\pgfqpoint{2.826205in}{1.419092in}}%
\pgfpathlineto{\pgfqpoint{2.828040in}{1.313580in}}%
\pgfpathlineto{\pgfqpoint{2.831708in}{1.424638in}}%
\pgfpathlineto{\pgfqpoint{2.833542in}{1.194280in}}%
\pgfpathlineto{\pgfqpoint{2.835376in}{1.393197in}}%
\pgfpathlineto{\pgfqpoint{2.837209in}{1.312049in}}%
\pgfpathlineto{\pgfqpoint{2.839043in}{1.344983in}}%
\pgfpathlineto{\pgfqpoint{2.840878in}{1.414061in}}%
\pgfpathlineto{\pgfqpoint{2.842711in}{1.411100in}}%
\pgfpathlineto{\pgfqpoint{2.844547in}{1.444360in}}%
\pgfpathlineto{\pgfqpoint{2.846380in}{1.272755in}}%
\pgfpathlineto{\pgfqpoint{2.848214in}{1.269318in}}%
\pgfpathlineto{\pgfqpoint{2.851883in}{1.443181in}}%
\pgfpathlineto{\pgfqpoint{2.853718in}{1.241026in}}%
\pgfpathlineto{\pgfqpoint{2.857387in}{1.493666in}}%
\pgfpathlineto{\pgfqpoint{2.859250in}{1.399583in}}%
\pgfpathlineto{\pgfqpoint{2.861437in}{1.383323in}}%
\pgfpathlineto{\pgfqpoint{2.863271in}{1.552720in}}%
\pgfpathlineto{\pgfqpoint{2.866939in}{1.375871in}}%
\pgfpathlineto{\pgfqpoint{2.870607in}{1.266796in}}%
\pgfpathlineto{\pgfqpoint{2.874275in}{2.185242in}}%
\pgfpathlineto{\pgfqpoint{2.876108in}{1.586720in}}%
\pgfpathlineto{\pgfqpoint{2.877943in}{1.824217in}}%
\pgfpathlineto{\pgfqpoint{2.879777in}{1.612753in}}%
\pgfpathlineto{\pgfqpoint{2.881615in}{1.246935in}}%
\pgfpathlineto{\pgfqpoint{2.883449in}{1.504531in}}%
\pgfpathlineto{\pgfqpoint{2.885282in}{1.295978in}}%
\pgfpathlineto{\pgfqpoint{2.888951in}{1.228593in}}%
\pgfpathlineto{\pgfqpoint{2.892618in}{1.422116in}}%
\pgfpathlineto{\pgfqpoint{2.894452in}{1.428728in}}%
\pgfpathlineto{\pgfqpoint{2.896285in}{1.293117in}}%
\pgfpathlineto{\pgfqpoint{2.898120in}{1.445351in}}%
\pgfpathlineto{\pgfqpoint{2.899953in}{1.768688in}}%
\pgfpathlineto{\pgfqpoint{2.901787in}{1.678306in}}%
\pgfpathlineto{\pgfqpoint{2.903622in}{1.415153in}}%
\pgfpathlineto{\pgfqpoint{2.905456in}{1.437836in}}%
\pgfpathlineto{\pgfqpoint{2.907289in}{1.567324in}}%
\pgfpathlineto{\pgfqpoint{2.909124in}{1.300319in}}%
\pgfpathlineto{\pgfqpoint{2.910958in}{1.322149in}}%
\pgfpathlineto{\pgfqpoint{2.912791in}{1.422705in}}%
\pgfpathlineto{\pgfqpoint{2.914626in}{1.453581in}}%
\pgfpathlineto{\pgfqpoint{2.916459in}{1.364818in}}%
\pgfpathlineto{\pgfqpoint{2.918292in}{1.373186in}}%
\pgfpathlineto{\pgfqpoint{2.920127in}{1.415918in}}%
\pgfpathlineto{\pgfqpoint{2.921961in}{1.604322in}}%
\pgfpathlineto{\pgfqpoint{2.923795in}{1.547401in}}%
\pgfpathlineto{\pgfqpoint{2.925630in}{1.266821in}}%
\pgfpathlineto{\pgfqpoint{2.927464in}{1.285514in}}%
\pgfpathlineto{\pgfqpoint{2.929298in}{1.337581in}}%
\pgfpathlineto{\pgfqpoint{2.931134in}{1.538995in}}%
\pgfpathlineto{\pgfqpoint{2.932967in}{1.204153in}}%
\pgfpathlineto{\pgfqpoint{2.934799in}{1.228656in}}%
\pgfpathlineto{\pgfqpoint{2.936633in}{1.316215in}}%
\pgfpathlineto{\pgfqpoint{2.938468in}{1.276519in}}%
\pgfpathlineto{\pgfqpoint{2.940301in}{1.208118in}}%
\pgfpathlineto{\pgfqpoint{2.943970in}{1.382859in}}%
\pgfpathlineto{\pgfqpoint{2.945803in}{1.625036in}}%
\pgfpathlineto{\pgfqpoint{2.947636in}{1.333353in}}%
\pgfpathlineto{\pgfqpoint{2.949470in}{1.384980in}}%
\pgfpathlineto{\pgfqpoint{2.951303in}{1.399545in}}%
\pgfpathlineto{\pgfqpoint{2.954971in}{1.128727in}}%
\pgfpathlineto{\pgfqpoint{2.956806in}{1.183352in}}%
\pgfpathlineto{\pgfqpoint{2.958640in}{1.129166in}}%
\pgfpathlineto{\pgfqpoint{2.960474in}{1.364354in}}%
\pgfpathlineto{\pgfqpoint{2.962307in}{1.189048in}}%
\pgfpathlineto{\pgfqpoint{2.964141in}{1.322500in}}%
\pgfpathlineto{\pgfqpoint{2.965976in}{1.118150in}}%
\pgfpathlineto{\pgfqpoint{2.967811in}{1.460457in}}%
\pgfpathlineto{\pgfqpoint{2.969644in}{1.254187in}}%
\pgfpathlineto{\pgfqpoint{2.971478in}{1.235443in}}%
\pgfpathlineto{\pgfqpoint{2.973311in}{1.271764in}}%
\pgfpathlineto{\pgfqpoint{2.975146in}{1.344807in}}%
\pgfpathlineto{\pgfqpoint{2.976981in}{1.275026in}}%
\pgfpathlineto{\pgfqpoint{2.978814in}{1.300319in}}%
\pgfpathlineto{\pgfqpoint{2.980649in}{1.169740in}}%
\pgfpathlineto{\pgfqpoint{2.982484in}{1.399345in}}%
\pgfpathlineto{\pgfqpoint{2.984318in}{1.235506in}}%
\pgfpathlineto{\pgfqpoint{2.986151in}{1.231102in}}%
\pgfpathlineto{\pgfqpoint{2.987986in}{1.174796in}}%
\pgfpathlineto{\pgfqpoint{2.989819in}{1.318624in}}%
\pgfpathlineto{\pgfqpoint{2.991653in}{1.280960in}}%
\pgfpathlineto{\pgfqpoint{2.993487in}{1.282491in}}%
\pgfpathlineto{\pgfqpoint{2.995321in}{1.362346in}}%
\pgfpathlineto{\pgfqpoint{2.997154in}{1.356450in}}%
\pgfpathlineto{\pgfqpoint{2.998989in}{1.258867in}}%
\pgfpathlineto{\pgfqpoint{3.000823in}{1.441976in}}%
\pgfpathlineto{\pgfqpoint{3.002655in}{1.174093in}}%
\pgfpathlineto{\pgfqpoint{3.004488in}{1.242845in}}%
\pgfpathlineto{\pgfqpoint{3.006324in}{1.427021in}}%
\pgfpathlineto{\pgfqpoint{3.008156in}{1.451235in}}%
\pgfpathlineto{\pgfqpoint{3.015492in}{1.220689in}}%
\pgfpathlineto{\pgfqpoint{3.017327in}{1.163228in}}%
\pgfpathlineto{\pgfqpoint{3.019162in}{1.466955in}}%
\pgfpathlineto{\pgfqpoint{3.024663in}{1.341257in}}%
\pgfpathlineto{\pgfqpoint{3.028332in}{1.136455in}}%
\pgfpathlineto{\pgfqpoint{3.033835in}{1.361355in}}%
\pgfpathlineto{\pgfqpoint{3.035669in}{1.186024in}}%
\pgfpathlineto{\pgfqpoint{3.039337in}{2.118459in}}%
\pgfpathlineto{\pgfqpoint{3.041170in}{1.711001in}}%
\pgfpathlineto{\pgfqpoint{3.043003in}{1.868178in}}%
\pgfpathlineto{\pgfqpoint{3.046673in}{2.457654in}}%
\pgfpathlineto{\pgfqpoint{3.050343in}{1.904524in}}%
\pgfpathlineto{\pgfqpoint{3.052175in}{2.014766in}}%
\pgfpathlineto{\pgfqpoint{3.054009in}{2.029520in}}%
\pgfpathlineto{\pgfqpoint{3.055844in}{2.119400in}}%
\pgfpathlineto{\pgfqpoint{3.057678in}{2.118283in}}%
\pgfpathlineto{\pgfqpoint{3.059511in}{2.213131in}}%
\pgfpathlineto{\pgfqpoint{3.061345in}{2.242426in}}%
\pgfpathlineto{\pgfqpoint{3.063179in}{1.405191in}}%
\pgfpathlineto{\pgfqpoint{3.065013in}{1.188433in}}%
\pgfpathlineto{\pgfqpoint{3.066847in}{1.467896in}}%
\pgfpathlineto{\pgfqpoint{3.068682in}{1.312137in}}%
\pgfpathlineto{\pgfqpoint{3.070517in}{1.404363in}}%
\pgfpathlineto{\pgfqpoint{3.072352in}{1.429606in}}%
\pgfpathlineto{\pgfqpoint{3.076019in}{1.215909in}}%
\pgfpathlineto{\pgfqpoint{3.077854in}{1.211794in}}%
\pgfpathlineto{\pgfqpoint{3.079689in}{1.255454in}}%
\pgfpathlineto{\pgfqpoint{3.081521in}{1.115177in}}%
\pgfpathlineto{\pgfqpoint{3.085189in}{1.337693in}}%
\pgfpathlineto{\pgfqpoint{3.087023in}{1.201393in}}%
\pgfpathlineto{\pgfqpoint{3.088855in}{1.352448in}}%
\pgfpathlineto{\pgfqpoint{3.090689in}{1.256069in}}%
\pgfpathlineto{\pgfqpoint{3.092523in}{1.229772in}}%
\pgfpathlineto{\pgfqpoint{3.094357in}{1.420585in}}%
\pgfpathlineto{\pgfqpoint{3.096190in}{1.273671in}}%
\pgfpathlineto{\pgfqpoint{3.098024in}{1.340642in}}%
\pgfpathlineto{\pgfqpoint{3.099858in}{1.320455in}}%
\pgfpathlineto{\pgfqpoint{3.101693in}{1.257888in}}%
\pgfpathlineto{\pgfqpoint{3.103527in}{1.393937in}}%
\pgfpathlineto{\pgfqpoint{3.107195in}{2.219806in}}%
\pgfpathlineto{\pgfqpoint{3.109029in}{2.172871in}}%
\pgfpathlineto{\pgfqpoint{3.110862in}{2.395363in}}%
\pgfpathlineto{\pgfqpoint{3.112696in}{2.438998in}}%
\pgfpathlineto{\pgfqpoint{3.114530in}{2.311856in}}%
\pgfpathlineto{\pgfqpoint{3.116365in}{2.079905in}}%
\pgfpathlineto{\pgfqpoint{3.118199in}{2.026697in}}%
\pgfpathlineto{\pgfqpoint{3.120033in}{2.254408in}}%
\pgfpathlineto{\pgfqpoint{3.121867in}{2.274306in}}%
\pgfpathlineto{\pgfqpoint{3.123701in}{2.375615in}}%
\pgfpathlineto{\pgfqpoint{3.125535in}{2.372704in}}%
\pgfpathlineto{\pgfqpoint{3.127369in}{2.071474in}}%
\pgfpathlineto{\pgfqpoint{3.132873in}{2.440792in}}%
\pgfpathlineto{\pgfqpoint{3.134707in}{1.208708in}}%
\pgfpathlineto{\pgfqpoint{3.136542in}{1.386498in}}%
\pgfpathlineto{\pgfqpoint{3.138375in}{1.210878in}}%
\pgfpathlineto{\pgfqpoint{3.140209in}{1.424136in}}%
\pgfpathlineto{\pgfqpoint{3.142043in}{1.361932in}}%
\pgfpathlineto{\pgfqpoint{3.143877in}{1.405304in}}%
\pgfpathlineto{\pgfqpoint{3.145710in}{1.349926in}}%
\pgfpathlineto{\pgfqpoint{3.147544in}{1.442152in}}%
\pgfpathlineto{\pgfqpoint{3.149377in}{1.307959in}}%
\pgfpathlineto{\pgfqpoint{3.151211in}{1.360678in}}%
\pgfpathlineto{\pgfqpoint{3.153045in}{1.355509in}}%
\pgfpathlineto{\pgfqpoint{3.154878in}{1.408654in}}%
\pgfpathlineto{\pgfqpoint{3.156713in}{1.553373in}}%
\pgfpathlineto{\pgfqpoint{3.158548in}{1.235381in}}%
\pgfpathlineto{\pgfqpoint{3.164050in}{1.501507in}}%
\pgfpathlineto{\pgfqpoint{3.165884in}{1.799451in}}%
\pgfpathlineto{\pgfqpoint{3.167717in}{1.806150in}}%
\pgfpathlineto{\pgfqpoint{3.169551in}{1.397927in}}%
\pgfpathlineto{\pgfqpoint{3.171384in}{1.345485in}}%
\pgfpathlineto{\pgfqpoint{3.173218in}{1.433156in}}%
\pgfpathlineto{\pgfqpoint{3.175055in}{1.248755in}}%
\pgfpathlineto{\pgfqpoint{3.176889in}{1.566885in}}%
\pgfpathlineto{\pgfqpoint{3.178722in}{1.473504in}}%
\pgfpathlineto{\pgfqpoint{3.180569in}{1.252368in}}%
\pgfpathlineto{\pgfqpoint{3.182403in}{1.340604in}}%
\pgfpathlineto{\pgfqpoint{3.184237in}{1.269995in}}%
\pgfpathlineto{\pgfqpoint{3.186071in}{1.435603in}}%
\pgfpathlineto{\pgfqpoint{3.187905in}{2.051701in}}%
\pgfpathlineto{\pgfqpoint{3.189739in}{2.101421in}}%
\pgfpathlineto{\pgfqpoint{3.191573in}{2.276301in}}%
\pgfpathlineto{\pgfqpoint{3.193406in}{2.291431in}}%
\pgfpathlineto{\pgfqpoint{3.195241in}{2.229843in}}%
\pgfpathlineto{\pgfqpoint{3.197075in}{2.045504in}}%
\pgfpathlineto{\pgfqpoint{3.198909in}{2.223883in}}%
\pgfpathlineto{\pgfqpoint{3.202579in}{2.192067in}}%
\pgfpathlineto{\pgfqpoint{3.204428in}{2.160037in}}%
\pgfpathlineto{\pgfqpoint{3.206260in}{2.211990in}}%
\pgfpathlineto{\pgfqpoint{3.208094in}{2.057636in}}%
\pgfpathlineto{\pgfqpoint{3.209927in}{2.248210in}}%
\pgfpathlineto{\pgfqpoint{3.211761in}{1.515847in}}%
\pgfpathlineto{\pgfqpoint{3.213595in}{1.387526in}}%
\pgfpathlineto{\pgfqpoint{3.215429in}{1.343603in}}%
\pgfpathlineto{\pgfqpoint{3.217263in}{1.178710in}}%
\pgfpathlineto{\pgfqpoint{3.222765in}{1.463869in}}%
\pgfpathlineto{\pgfqpoint{3.224599in}{1.310230in}}%
\pgfpathlineto{\pgfqpoint{3.226432in}{1.311610in}}%
\pgfpathlineto{\pgfqpoint{3.228266in}{1.445765in}}%
\pgfpathlineto{\pgfqpoint{3.230099in}{1.192021in}}%
\pgfpathlineto{\pgfqpoint{3.231933in}{1.092619in}}%
\pgfpathlineto{\pgfqpoint{3.235603in}{1.318097in}}%
\pgfpathlineto{\pgfqpoint{3.237436in}{1.130169in}}%
\pgfpathlineto{\pgfqpoint{3.239271in}{1.210702in}}%
\pgfpathlineto{\pgfqpoint{3.241104in}{1.145237in}}%
\pgfpathlineto{\pgfqpoint{3.242938in}{1.394088in}}%
\pgfpathlineto{\pgfqpoint{3.244772in}{1.172562in}}%
\pgfpathlineto{\pgfqpoint{3.246606in}{1.132252in}}%
\pgfpathlineto{\pgfqpoint{3.248439in}{1.143267in}}%
\pgfpathlineto{\pgfqpoint{3.250273in}{1.583220in}}%
\pgfpathlineto{\pgfqpoint{3.252109in}{1.613481in}}%
\pgfpathlineto{\pgfqpoint{3.253942in}{1.321685in}}%
\pgfpathlineto{\pgfqpoint{3.255778in}{1.368870in}}%
\pgfpathlineto{\pgfqpoint{3.257612in}{1.342511in}}%
\pgfpathlineto{\pgfqpoint{3.259444in}{1.284047in}}%
\pgfpathlineto{\pgfqpoint{3.261278in}{1.122993in}}%
\pgfpathlineto{\pgfqpoint{3.263112in}{1.341219in}}%
\pgfpathlineto{\pgfqpoint{3.264945in}{2.062253in}}%
\pgfpathlineto{\pgfqpoint{3.266779in}{1.943743in}}%
\pgfpathlineto{\pgfqpoint{3.268613in}{2.065251in}}%
\pgfpathlineto{\pgfqpoint{3.270446in}{1.999108in}}%
\pgfpathlineto{\pgfqpoint{3.272279in}{2.039971in}}%
\pgfpathlineto{\pgfqpoint{3.274116in}{2.165946in}}%
\pgfpathlineto{\pgfqpoint{3.275950in}{2.197474in}}%
\pgfpathlineto{\pgfqpoint{3.277784in}{1.971633in}}%
\pgfpathlineto{\pgfqpoint{3.279618in}{1.219485in}}%
\pgfpathlineto{\pgfqpoint{3.283286in}{1.114951in}}%
\pgfpathlineto{\pgfqpoint{3.285121in}{1.140156in}}%
\pgfpathlineto{\pgfqpoint{3.286954in}{1.291449in}}%
\pgfpathlineto{\pgfqpoint{3.290622in}{1.146115in}}%
\pgfpathlineto{\pgfqpoint{3.292455in}{1.202359in}}%
\pgfpathlineto{\pgfqpoint{3.294288in}{1.189550in}}%
\pgfpathlineto{\pgfqpoint{3.296121in}{1.245204in}}%
\pgfpathlineto{\pgfqpoint{3.297955in}{1.365169in}}%
\pgfpathlineto{\pgfqpoint{3.299789in}{1.406948in}}%
\pgfpathlineto{\pgfqpoint{3.301622in}{1.333942in}}%
\pgfpathlineto{\pgfqpoint{3.303456in}{1.128074in}}%
\pgfpathlineto{\pgfqpoint{3.307123in}{1.334231in}}%
\pgfpathlineto{\pgfqpoint{3.308957in}{1.215344in}}%
\pgfpathlineto{\pgfqpoint{3.312625in}{1.255040in}}%
\pgfpathlineto{\pgfqpoint{3.314460in}{1.193427in}}%
\pgfpathlineto{\pgfqpoint{3.316293in}{1.323215in}}%
\pgfpathlineto{\pgfqpoint{3.318126in}{1.115302in}}%
\pgfpathlineto{\pgfqpoint{3.323628in}{1.260159in}}%
\pgfpathlineto{\pgfqpoint{3.325463in}{2.005369in}}%
\pgfpathlineto{\pgfqpoint{3.327297in}{2.055603in}}%
\pgfpathlineto{\pgfqpoint{3.329130in}{2.170763in}}%
\pgfpathlineto{\pgfqpoint{3.330965in}{2.090895in}}%
\pgfpathlineto{\pgfqpoint{3.332797in}{2.090657in}}%
\pgfpathlineto{\pgfqpoint{3.334631in}{2.024966in}}%
\pgfpathlineto{\pgfqpoint{3.336466in}{2.023372in}}%
\pgfpathlineto{\pgfqpoint{3.338299in}{2.009471in}}%
\pgfpathlineto{\pgfqpoint{3.340132in}{2.154215in}}%
\pgfpathlineto{\pgfqpoint{3.343803in}{2.003600in}}%
\pgfpathlineto{\pgfqpoint{3.345637in}{2.074146in}}%
\pgfpathlineto{\pgfqpoint{3.349319in}{2.099677in}}%
\pgfpathlineto{\pgfqpoint{3.351153in}{2.242803in}}%
\pgfpathlineto{\pgfqpoint{3.352986in}{2.098297in}}%
\pgfpathlineto{\pgfqpoint{3.354819in}{2.155984in}}%
\pgfpathlineto{\pgfqpoint{3.356653in}{1.976952in}}%
\pgfpathlineto{\pgfqpoint{3.358488in}{1.399784in}}%
\pgfpathlineto{\pgfqpoint{3.360321in}{1.198508in}}%
\pgfpathlineto{\pgfqpoint{3.362156in}{1.255655in}}%
\pgfpathlineto{\pgfqpoint{3.363989in}{1.180943in}}%
\pgfpathlineto{\pgfqpoint{3.365824in}{1.160581in}}%
\pgfpathlineto{\pgfqpoint{3.367658in}{1.330818in}}%
\pgfpathlineto{\pgfqpoint{3.369491in}{1.255780in}}%
\pgfpathlineto{\pgfqpoint{3.371324in}{1.307081in}}%
\pgfpathlineto{\pgfqpoint{3.373159in}{1.420610in}}%
\pgfpathlineto{\pgfqpoint{3.374993in}{1.362698in}}%
\pgfpathlineto{\pgfqpoint{3.376827in}{1.266859in}}%
\pgfpathlineto{\pgfqpoint{3.378660in}{1.417411in}}%
\pgfpathlineto{\pgfqpoint{3.380495in}{1.195748in}}%
\pgfpathlineto{\pgfqpoint{3.382330in}{1.266683in}}%
\pgfpathlineto{\pgfqpoint{3.384164in}{1.226071in}}%
\pgfpathlineto{\pgfqpoint{3.386001in}{1.245932in}}%
\pgfpathlineto{\pgfqpoint{3.387834in}{1.332788in}}%
\pgfpathlineto{\pgfqpoint{3.389668in}{1.263421in}}%
\pgfpathlineto{\pgfqpoint{3.391503in}{1.124825in}}%
\pgfpathlineto{\pgfqpoint{3.393338in}{1.261037in}}%
\pgfpathlineto{\pgfqpoint{3.395171in}{1.265391in}}%
\pgfpathlineto{\pgfqpoint{3.397006in}{1.099670in}}%
\pgfpathlineto{\pgfqpoint{3.398840in}{1.234327in}}%
\pgfpathlineto{\pgfqpoint{3.400674in}{1.287007in}}%
\pgfpathlineto{\pgfqpoint{3.402510in}{1.138952in}}%
\pgfpathlineto{\pgfqpoint{3.404343in}{1.092494in}}%
\pgfpathlineto{\pgfqpoint{3.406177in}{1.233599in}}%
\pgfpathlineto{\pgfqpoint{3.408012in}{1.111513in}}%
\pgfpathlineto{\pgfqpoint{3.409845in}{1.350741in}}%
\pgfpathlineto{\pgfqpoint{3.411678in}{1.161785in}}%
\pgfpathlineto{\pgfqpoint{3.413513in}{1.243197in}}%
\pgfpathlineto{\pgfqpoint{3.415347in}{1.195045in}}%
\pgfpathlineto{\pgfqpoint{3.417181in}{1.086647in}}%
\pgfpathlineto{\pgfqpoint{3.419015in}{1.252493in}}%
\pgfpathlineto{\pgfqpoint{3.420849in}{1.223073in}}%
\pgfpathlineto{\pgfqpoint{3.422686in}{1.018271in}}%
\pgfpathlineto{\pgfqpoint{3.424521in}{1.031031in}}%
\pgfpathlineto{\pgfqpoint{3.426354in}{1.276168in}}%
\pgfpathlineto{\pgfqpoint{3.428187in}{1.104488in}}%
\pgfpathlineto{\pgfqpoint{3.430022in}{1.161020in}}%
\pgfpathlineto{\pgfqpoint{3.431855in}{1.147972in}}%
\pgfpathlineto{\pgfqpoint{3.435523in}{1.417499in}}%
\pgfpathlineto{\pgfqpoint{3.437356in}{1.311723in}}%
\pgfpathlineto{\pgfqpoint{3.439189in}{1.349951in}}%
\pgfpathlineto{\pgfqpoint{3.441023in}{1.173089in}}%
\pgfpathlineto{\pgfqpoint{3.442856in}{1.293331in}}%
\pgfpathlineto{\pgfqpoint{3.444691in}{1.658797in}}%
\pgfpathlineto{\pgfqpoint{3.446524in}{2.249101in}}%
\pgfpathlineto{\pgfqpoint{3.448382in}{2.235049in}}%
\pgfpathlineto{\pgfqpoint{3.450217in}{2.271746in}}%
\pgfpathlineto{\pgfqpoint{3.452050in}{2.094069in}}%
\pgfpathlineto{\pgfqpoint{3.455718in}{2.261961in}}%
\pgfpathlineto{\pgfqpoint{3.457551in}{2.291318in}}%
\pgfpathlineto{\pgfqpoint{3.461218in}{2.217422in}}%
\pgfpathlineto{\pgfqpoint{3.463056in}{2.192267in}}%
\pgfpathlineto{\pgfqpoint{3.464890in}{2.139825in}}%
\pgfpathlineto{\pgfqpoint{3.466727in}{2.141054in}}%
\pgfpathlineto{\pgfqpoint{3.468561in}{2.050522in}}%
\pgfpathlineto{\pgfqpoint{3.470395in}{2.179872in}}%
\pgfpathlineto{\pgfqpoint{3.474066in}{2.010588in}}%
\pgfpathlineto{\pgfqpoint{3.475899in}{2.344062in}}%
\pgfpathlineto{\pgfqpoint{3.477733in}{2.143965in}}%
\pgfpathlineto{\pgfqpoint{3.479567in}{2.309623in}}%
\pgfpathlineto{\pgfqpoint{3.481400in}{2.241398in}}%
\pgfpathlineto{\pgfqpoint{3.483235in}{2.483097in}}%
\pgfpathlineto{\pgfqpoint{3.486903in}{2.232051in}}%
\pgfpathlineto{\pgfqpoint{3.488739in}{2.200937in}}%
\pgfpathlineto{\pgfqpoint{3.490574in}{2.377409in}}%
\pgfpathlineto{\pgfqpoint{3.492407in}{2.192418in}}%
\pgfpathlineto{\pgfqpoint{3.494242in}{2.152484in}}%
\pgfpathlineto{\pgfqpoint{3.496076in}{2.186421in}}%
\pgfpathlineto{\pgfqpoint{3.497910in}{2.158807in}}%
\pgfpathlineto{\pgfqpoint{3.499744in}{2.431533in}}%
\pgfpathlineto{\pgfqpoint{3.501578in}{2.351815in}}%
\pgfpathlineto{\pgfqpoint{3.503411in}{2.183046in}}%
\pgfpathlineto{\pgfqpoint{3.505250in}{2.229730in}}%
\pgfpathlineto{\pgfqpoint{3.507084in}{2.382290in}}%
\pgfpathlineto{\pgfqpoint{3.508918in}{2.074974in}}%
\pgfpathlineto{\pgfqpoint{3.510752in}{2.204751in}}%
\pgfpathlineto{\pgfqpoint{3.512586in}{2.241423in}}%
\pgfpathlineto{\pgfqpoint{3.516251in}{1.332085in}}%
\pgfpathlineto{\pgfqpoint{3.518086in}{1.356174in}}%
\pgfpathlineto{\pgfqpoint{3.519919in}{1.276469in}}%
\pgfpathlineto{\pgfqpoint{3.521752in}{1.257487in}}%
\pgfpathlineto{\pgfqpoint{3.523588in}{1.351394in}}%
\pgfpathlineto{\pgfqpoint{3.525421in}{1.387326in}}%
\pgfpathlineto{\pgfqpoint{3.527256in}{1.190102in}}%
\pgfpathlineto{\pgfqpoint{3.529091in}{1.186614in}}%
\pgfpathlineto{\pgfqpoint{3.530924in}{1.361707in}}%
\pgfpathlineto{\pgfqpoint{3.532759in}{1.217929in}}%
\pgfpathlineto{\pgfqpoint{3.534593in}{1.300557in}}%
\pgfpathlineto{\pgfqpoint{3.536425in}{1.128701in}}%
\pgfpathlineto{\pgfqpoint{3.538258in}{1.259067in}}%
\pgfpathlineto{\pgfqpoint{3.540092in}{1.187379in}}%
\pgfpathlineto{\pgfqpoint{3.541927in}{1.243761in}}%
\pgfpathlineto{\pgfqpoint{3.543760in}{1.079860in}}%
\pgfpathlineto{\pgfqpoint{3.545594in}{1.216173in}}%
\pgfpathlineto{\pgfqpoint{3.547428in}{1.206299in}}%
\pgfpathlineto{\pgfqpoint{3.549263in}{1.213262in}}%
\pgfpathlineto{\pgfqpoint{3.551099in}{1.177380in}}%
\pgfpathlineto{\pgfqpoint{3.554765in}{1.013893in}}%
\pgfpathlineto{\pgfqpoint{3.556600in}{1.177531in}}%
\pgfpathlineto{\pgfqpoint{3.558433in}{0.895195in}}%
\pgfpathlineto{\pgfqpoint{3.562100in}{1.285339in}}%
\pgfpathlineto{\pgfqpoint{3.563934in}{1.035459in}}%
\pgfpathlineto{\pgfqpoint{3.565768in}{1.001083in}}%
\pgfpathlineto{\pgfqpoint{3.567603in}{1.187204in}}%
\pgfpathlineto{\pgfqpoint{3.569436in}{0.962567in}}%
\pgfpathlineto{\pgfqpoint{3.573103in}{1.134335in}}%
\pgfpathlineto{\pgfqpoint{3.576774in}{1.013140in}}%
\pgfpathlineto{\pgfqpoint{3.578608in}{1.080186in}}%
\pgfpathlineto{\pgfqpoint{3.580443in}{1.216173in}}%
\pgfpathlineto{\pgfqpoint{3.584114in}{1.005324in}}%
\pgfpathlineto{\pgfqpoint{3.585947in}{1.240963in}}%
\pgfpathlineto{\pgfqpoint{3.587780in}{1.265880in}}%
\pgfpathlineto{\pgfqpoint{3.589616in}{1.274901in}}%
\pgfpathlineto{\pgfqpoint{3.591449in}{1.469063in}}%
\pgfpathlineto{\pgfqpoint{3.595117in}{1.342486in}}%
\pgfpathlineto{\pgfqpoint{3.602455in}{0.984284in}}%
\pgfpathlineto{\pgfqpoint{3.604289in}{1.122403in}}%
\pgfpathlineto{\pgfqpoint{3.606124in}{1.041456in}}%
\pgfpathlineto{\pgfqpoint{3.607958in}{1.108302in}}%
\pgfpathlineto{\pgfqpoint{3.609790in}{1.069396in}}%
\pgfpathlineto{\pgfqpoint{3.611625in}{1.107750in}}%
\pgfpathlineto{\pgfqpoint{3.613459in}{1.110685in}}%
\pgfpathlineto{\pgfqpoint{3.615293in}{1.064265in}}%
\pgfpathlineto{\pgfqpoint{3.617126in}{1.049486in}}%
\pgfpathlineto{\pgfqpoint{3.618961in}{1.066548in}}%
\pgfpathlineto{\pgfqpoint{3.620794in}{1.072633in}}%
\pgfpathlineto{\pgfqpoint{3.622628in}{1.136191in}}%
\pgfpathlineto{\pgfqpoint{3.624462in}{1.414588in}}%
\pgfpathlineto{\pgfqpoint{3.626295in}{1.142364in}}%
\pgfpathlineto{\pgfqpoint{3.628128in}{1.125314in}}%
\pgfpathlineto{\pgfqpoint{3.631796in}{1.388154in}}%
\pgfpathlineto{\pgfqpoint{3.633629in}{1.422053in}}%
\pgfpathlineto{\pgfqpoint{3.635464in}{1.361757in}}%
\pgfpathlineto{\pgfqpoint{3.637298in}{1.382420in}}%
\pgfpathlineto{\pgfqpoint{3.639130in}{1.091879in}}%
\pgfpathlineto{\pgfqpoint{3.640965in}{1.317507in}}%
\pgfpathlineto{\pgfqpoint{3.644632in}{1.169740in}}%
\pgfpathlineto{\pgfqpoint{3.646465in}{1.215232in}}%
\pgfpathlineto{\pgfqpoint{3.648299in}{1.142151in}}%
\pgfpathlineto{\pgfqpoint{3.650133in}{1.251552in}}%
\pgfpathlineto{\pgfqpoint{3.651968in}{1.235092in}}%
\pgfpathlineto{\pgfqpoint{3.653803in}{1.106784in}}%
\pgfpathlineto{\pgfqpoint{3.655636in}{1.222019in}}%
\pgfpathlineto{\pgfqpoint{3.657471in}{1.034293in}}%
\pgfpathlineto{\pgfqpoint{3.662969in}{1.721025in}}%
\pgfpathlineto{\pgfqpoint{3.664803in}{1.622213in}}%
\pgfpathlineto{\pgfqpoint{3.670303in}{2.439111in}}%
\pgfpathlineto{\pgfqpoint{3.672137in}{2.351728in}}%
\pgfpathlineto{\pgfqpoint{3.673971in}{2.581684in}}%
\pgfpathlineto{\pgfqpoint{3.675806in}{2.414985in}}%
\pgfpathlineto{\pgfqpoint{3.677640in}{2.567018in}}%
\pgfpathlineto{\pgfqpoint{3.679474in}{2.565612in}}%
\pgfpathlineto{\pgfqpoint{3.681309in}{2.517687in}}%
\pgfpathlineto{\pgfqpoint{3.683143in}{2.542565in}}%
\pgfpathlineto{\pgfqpoint{3.684977in}{2.494790in}}%
\pgfpathlineto{\pgfqpoint{3.686813in}{2.498090in}}%
\pgfpathlineto{\pgfqpoint{3.688646in}{2.614793in}}%
\pgfpathlineto{\pgfqpoint{3.690479in}{2.633926in}}%
\pgfpathlineto{\pgfqpoint{3.692315in}{2.509481in}}%
\pgfpathlineto{\pgfqpoint{3.694149in}{2.495969in}}%
\pgfpathlineto{\pgfqpoint{3.695982in}{2.643448in}}%
\pgfpathlineto{\pgfqpoint{3.697816in}{2.486359in}}%
\pgfpathlineto{\pgfqpoint{3.699648in}{2.614994in}}%
\pgfpathlineto{\pgfqpoint{3.701483in}{2.419689in}}%
\pgfpathlineto{\pgfqpoint{3.703318in}{2.683457in}}%
\pgfpathlineto{\pgfqpoint{3.705152in}{2.482545in}}%
\pgfpathlineto{\pgfqpoint{3.706986in}{2.576778in}}%
\pgfpathlineto{\pgfqpoint{3.708820in}{2.414320in}}%
\pgfpathlineto{\pgfqpoint{3.710655in}{2.377786in}}%
\pgfpathlineto{\pgfqpoint{3.712488in}{2.541361in}}%
\pgfpathlineto{\pgfqpoint{3.714321in}{2.452134in}}%
\pgfpathlineto{\pgfqpoint{3.716156in}{2.436740in}}%
\pgfpathlineto{\pgfqpoint{3.717990in}{2.389453in}}%
\pgfpathlineto{\pgfqpoint{3.719822in}{2.598069in}}%
\pgfpathlineto{\pgfqpoint{3.721657in}{2.423541in}}%
\pgfpathlineto{\pgfqpoint{3.723491in}{2.546593in}}%
\pgfpathlineto{\pgfqpoint{3.725324in}{2.422976in}}%
\pgfpathlineto{\pgfqpoint{3.727159in}{2.712878in}}%
\pgfpathlineto{\pgfqpoint{3.728993in}{2.478957in}}%
\pgfpathlineto{\pgfqpoint{3.732661in}{2.370534in}}%
\pgfpathlineto{\pgfqpoint{3.734494in}{2.388011in}}%
\pgfpathlineto{\pgfqpoint{3.736327in}{2.574219in}}%
\pgfpathlineto{\pgfqpoint{3.738163in}{2.532077in}}%
\pgfpathlineto{\pgfqpoint{3.739999in}{2.572337in}}%
\pgfpathlineto{\pgfqpoint{3.741832in}{2.541750in}}%
\pgfpathlineto{\pgfqpoint{3.745501in}{2.604192in}}%
\pgfpathlineto{\pgfqpoint{3.747335in}{2.498315in}}%
\pgfpathlineto{\pgfqpoint{3.749169in}{2.535866in}}%
\pgfpathlineto{\pgfqpoint{3.751003in}{2.519970in}}%
\pgfpathlineto{\pgfqpoint{3.752838in}{2.389717in}}%
\pgfpathlineto{\pgfqpoint{3.754673in}{2.579237in}}%
\pgfpathlineto{\pgfqpoint{3.758339in}{2.384899in}}%
\pgfpathlineto{\pgfqpoint{3.760174in}{2.417218in}}%
\pgfpathlineto{\pgfqpoint{3.762009in}{2.410430in}}%
\pgfpathlineto{\pgfqpoint{3.763842in}{2.305006in}}%
\pgfpathlineto{\pgfqpoint{3.765677in}{2.378902in}}%
\pgfpathlineto{\pgfqpoint{3.767511in}{2.337688in}}%
\pgfpathlineto{\pgfqpoint{3.769345in}{2.527585in}}%
\pgfpathlineto{\pgfqpoint{3.771179in}{2.469961in}}%
\pgfpathlineto{\pgfqpoint{3.773011in}{2.506433in}}%
\pgfpathlineto{\pgfqpoint{3.774845in}{2.394484in}}%
\pgfpathlineto{\pgfqpoint{3.776679in}{2.403267in}}%
\pgfpathlineto{\pgfqpoint{3.778514in}{2.543720in}}%
\pgfpathlineto{\pgfqpoint{3.780349in}{2.486974in}}%
\pgfpathlineto{\pgfqpoint{3.782186in}{2.637828in}}%
\pgfpathlineto{\pgfqpoint{3.784019in}{1.627557in}}%
\pgfpathlineto{\pgfqpoint{3.787687in}{1.566232in}}%
\pgfpathlineto{\pgfqpoint{3.789522in}{1.507027in}}%
\pgfpathlineto{\pgfqpoint{3.791355in}{1.594674in}}%
\pgfpathlineto{\pgfqpoint{3.793189in}{1.462138in}}%
\pgfpathlineto{\pgfqpoint{3.795023in}{1.574588in}}%
\pgfpathlineto{\pgfqpoint{3.796856in}{1.554803in}}%
\pgfpathlineto{\pgfqpoint{3.798692in}{1.590835in}}%
\pgfpathlineto{\pgfqpoint{3.802362in}{1.383035in}}%
\pgfpathlineto{\pgfqpoint{3.804196in}{1.489312in}}%
\pgfpathlineto{\pgfqpoint{3.806032in}{1.387827in}}%
\pgfpathlineto{\pgfqpoint{3.807865in}{1.401816in}}%
\pgfpathlineto{\pgfqpoint{3.809697in}{1.539760in}}%
\pgfpathlineto{\pgfqpoint{3.811531in}{1.429694in}}%
\pgfpathlineto{\pgfqpoint{3.813365in}{1.393611in}}%
\pgfpathlineto{\pgfqpoint{3.815198in}{1.600583in}}%
\pgfpathlineto{\pgfqpoint{3.817032in}{1.399081in}}%
\pgfpathlineto{\pgfqpoint{3.818866in}{1.481057in}}%
\pgfpathlineto{\pgfqpoint{3.820699in}{1.312049in}}%
\pgfpathlineto{\pgfqpoint{3.822534in}{1.432717in}}%
\pgfpathlineto{\pgfqpoint{3.824367in}{1.416621in}}%
\pgfpathlineto{\pgfqpoint{3.826201in}{1.426432in}}%
\pgfpathlineto{\pgfqpoint{3.828036in}{1.360640in}}%
\pgfpathlineto{\pgfqpoint{3.829870in}{1.483842in}}%
\pgfpathlineto{\pgfqpoint{3.831703in}{1.433922in}}%
\pgfpathlineto{\pgfqpoint{3.833538in}{1.441299in}}%
\pgfpathlineto{\pgfqpoint{3.835371in}{1.568792in}}%
\pgfpathlineto{\pgfqpoint{3.837205in}{1.524742in}}%
\pgfpathlineto{\pgfqpoint{3.839039in}{1.433608in}}%
\pgfpathlineto{\pgfqpoint{3.840875in}{1.514643in}}%
\pgfpathlineto{\pgfqpoint{3.842708in}{1.358470in}}%
\pgfpathlineto{\pgfqpoint{3.844542in}{1.567211in}}%
\pgfpathlineto{\pgfqpoint{3.848208in}{1.461636in}}%
\pgfpathlineto{\pgfqpoint{3.850044in}{1.552921in}}%
\pgfpathlineto{\pgfqpoint{3.851891in}{1.584838in}}%
\pgfpathlineto{\pgfqpoint{3.853727in}{1.500328in}}%
\pgfpathlineto{\pgfqpoint{3.855561in}{1.478560in}}%
\pgfpathlineto{\pgfqpoint{3.857394in}{1.340730in}}%
\pgfpathlineto{\pgfqpoint{3.861063in}{1.523450in}}%
\pgfpathlineto{\pgfqpoint{3.862896in}{1.620067in}}%
\pgfpathlineto{\pgfqpoint{3.866563in}{1.322036in}}%
\pgfpathlineto{\pgfqpoint{3.868397in}{1.460983in}}%
\pgfpathlineto{\pgfqpoint{3.870230in}{1.477996in}}%
\pgfpathlineto{\pgfqpoint{3.872064in}{1.483491in}}%
\pgfpathlineto{\pgfqpoint{3.873898in}{1.588125in}}%
\pgfpathlineto{\pgfqpoint{3.877566in}{1.393022in}}%
\pgfpathlineto{\pgfqpoint{3.879399in}{1.462903in}}%
\pgfpathlineto{\pgfqpoint{3.881233in}{1.443921in}}%
\pgfpathlineto{\pgfqpoint{3.883067in}{1.365194in}}%
\pgfpathlineto{\pgfqpoint{3.884901in}{1.440508in}}%
\pgfpathlineto{\pgfqpoint{3.886734in}{1.327330in}}%
\pgfpathlineto{\pgfqpoint{3.888568in}{1.367553in}}%
\pgfpathlineto{\pgfqpoint{3.890402in}{1.552921in}}%
\pgfpathlineto{\pgfqpoint{3.894071in}{1.421639in}}%
\pgfpathlineto{\pgfqpoint{3.895904in}{1.447007in}}%
\pgfpathlineto{\pgfqpoint{3.897738in}{1.542056in}}%
\pgfpathlineto{\pgfqpoint{3.899574in}{1.356676in}}%
\pgfpathlineto{\pgfqpoint{3.901409in}{1.340228in}}%
\pgfpathlineto{\pgfqpoint{3.905078in}{1.476440in}}%
\pgfpathlineto{\pgfqpoint{3.906913in}{1.320919in}}%
\pgfpathlineto{\pgfqpoint{3.908748in}{1.358382in}}%
\pgfpathlineto{\pgfqpoint{3.910582in}{1.442829in}}%
\pgfpathlineto{\pgfqpoint{3.912415in}{1.597146in}}%
\pgfpathlineto{\pgfqpoint{3.914249in}{1.323128in}}%
\pgfpathlineto{\pgfqpoint{3.916085in}{1.429016in}}%
\pgfpathlineto{\pgfqpoint{3.917917in}{1.452264in}}%
\pgfpathlineto{\pgfqpoint{3.919750in}{1.408955in}}%
\pgfpathlineto{\pgfqpoint{3.923418in}{1.530802in}}%
\pgfpathlineto{\pgfqpoint{3.925252in}{1.457696in}}%
\pgfpathlineto{\pgfqpoint{3.927086in}{1.433483in}}%
\pgfpathlineto{\pgfqpoint{3.928919in}{1.599090in}}%
\pgfpathlineto{\pgfqpoint{3.932587in}{1.333528in}}%
\pgfpathlineto{\pgfqpoint{3.934422in}{1.484733in}}%
\pgfpathlineto{\pgfqpoint{3.936255in}{1.262392in}}%
\pgfpathlineto{\pgfqpoint{3.938088in}{1.510051in}}%
\pgfpathlineto{\pgfqpoint{3.939922in}{1.434160in}}%
\pgfpathlineto{\pgfqpoint{3.941755in}{1.457232in}}%
\pgfpathlineto{\pgfqpoint{3.943588in}{1.444448in}}%
\pgfpathlineto{\pgfqpoint{3.945425in}{1.465538in}}%
\pgfpathlineto{\pgfqpoint{3.947259in}{1.359210in}}%
\pgfpathlineto{\pgfqpoint{3.950929in}{1.514467in}}%
\pgfpathlineto{\pgfqpoint{3.952762in}{1.422843in}}%
\pgfpathlineto{\pgfqpoint{3.954594in}{1.427172in}}%
\pgfpathlineto{\pgfqpoint{3.956429in}{1.598262in}}%
\pgfpathlineto{\pgfqpoint{3.958263in}{1.519486in}}%
\pgfpathlineto{\pgfqpoint{3.960096in}{1.546196in}}%
\pgfpathlineto{\pgfqpoint{3.961930in}{1.552281in}}%
\pgfpathlineto{\pgfqpoint{3.963764in}{1.442478in}}%
\pgfpathlineto{\pgfqpoint{3.965599in}{1.600671in}}%
\pgfpathlineto{\pgfqpoint{3.967434in}{1.558504in}}%
\pgfpathlineto{\pgfqpoint{3.969267in}{1.594235in}}%
\pgfpathlineto{\pgfqpoint{3.971101in}{1.607371in}}%
\pgfpathlineto{\pgfqpoint{3.972938in}{1.422994in}}%
\pgfpathlineto{\pgfqpoint{3.974772in}{1.639225in}}%
\pgfpathlineto{\pgfqpoint{3.976604in}{1.614572in}}%
\pgfpathlineto{\pgfqpoint{3.980273in}{1.810905in}}%
\pgfpathlineto{\pgfqpoint{3.982106in}{1.556358in}}%
\pgfpathlineto{\pgfqpoint{3.983941in}{1.637080in}}%
\pgfpathlineto{\pgfqpoint{3.987610in}{1.467721in}}%
\pgfpathlineto{\pgfqpoint{3.989444in}{1.530150in}}%
\pgfpathlineto{\pgfqpoint{3.991276in}{1.634282in}}%
\pgfpathlineto{\pgfqpoint{3.993109in}{1.554125in}}%
\pgfpathlineto{\pgfqpoint{3.994943in}{1.790756in}}%
\pgfpathlineto{\pgfqpoint{3.998610in}{1.493603in}}%
\pgfpathlineto{\pgfqpoint{4.002279in}{1.611988in}}%
\pgfpathlineto{\pgfqpoint{4.004112in}{1.550286in}}%
\pgfpathlineto{\pgfqpoint{4.005945in}{1.533562in}}%
\pgfpathlineto{\pgfqpoint{4.007780in}{1.588564in}}%
\pgfpathlineto{\pgfqpoint{4.009614in}{1.503414in}}%
\pgfpathlineto{\pgfqpoint{4.011448in}{1.614723in}}%
\pgfpathlineto{\pgfqpoint{4.015116in}{1.443206in}}%
\pgfpathlineto{\pgfqpoint{4.016949in}{1.511531in}}%
\pgfpathlineto{\pgfqpoint{4.020618in}{1.328384in}}%
\pgfpathlineto{\pgfqpoint{4.022452in}{1.415065in}}%
\pgfpathlineto{\pgfqpoint{4.024289in}{1.661319in}}%
\pgfpathlineto{\pgfqpoint{4.026122in}{1.615225in}}%
\pgfpathlineto{\pgfqpoint{4.029789in}{1.419820in}}%
\pgfpathlineto{\pgfqpoint{4.031623in}{1.363789in}}%
\pgfpathlineto{\pgfqpoint{4.033457in}{1.433520in}}%
\pgfpathlineto{\pgfqpoint{4.035291in}{1.647016in}}%
\pgfpathlineto{\pgfqpoint{4.037125in}{1.504832in}}%
\pgfpathlineto{\pgfqpoint{4.040794in}{1.569933in}}%
\pgfpathlineto{\pgfqpoint{4.042629in}{1.631033in}}%
\pgfpathlineto{\pgfqpoint{4.044463in}{1.576909in}}%
\pgfpathlineto{\pgfqpoint{4.046298in}{1.622501in}}%
\pgfpathlineto{\pgfqpoint{4.048133in}{1.695845in}}%
\pgfpathlineto{\pgfqpoint{4.049965in}{1.907347in}}%
\pgfpathlineto{\pgfqpoint{4.051799in}{1.966815in}}%
\pgfpathlineto{\pgfqpoint{4.055467in}{1.556835in}}%
\pgfpathlineto{\pgfqpoint{4.057301in}{1.620657in}}%
\pgfpathlineto{\pgfqpoint{4.059136in}{1.768035in}}%
\pgfpathlineto{\pgfqpoint{4.060970in}{1.552369in}}%
\pgfpathlineto{\pgfqpoint{4.062804in}{1.476352in}}%
\pgfpathlineto{\pgfqpoint{4.064637in}{1.482676in}}%
\pgfpathlineto{\pgfqpoint{4.066470in}{1.557713in}}%
\pgfpathlineto{\pgfqpoint{4.068303in}{1.501620in}}%
\pgfpathlineto{\pgfqpoint{4.070138in}{1.586858in}}%
\pgfpathlineto{\pgfqpoint{4.071971in}{1.576545in}}%
\pgfpathlineto{\pgfqpoint{4.073806in}{1.447321in}}%
\pgfpathlineto{\pgfqpoint{4.075640in}{1.622413in}}%
\pgfpathlineto{\pgfqpoint{4.077474in}{1.547488in}}%
\pgfpathlineto{\pgfqpoint{4.081142in}{1.553774in}}%
\pgfpathlineto{\pgfqpoint{4.082976in}{1.225544in}}%
\pgfpathlineto{\pgfqpoint{4.084809in}{1.531003in}}%
\pgfpathlineto{\pgfqpoint{4.086642in}{1.455840in}}%
\pgfpathlineto{\pgfqpoint{4.088476in}{1.546259in}}%
\pgfpathlineto{\pgfqpoint{4.090309in}{1.365470in}}%
\pgfpathlineto{\pgfqpoint{4.092143in}{1.611599in}}%
\pgfpathlineto{\pgfqpoint{4.095810in}{1.299641in}}%
\pgfpathlineto{\pgfqpoint{4.097644in}{0.944463in}}%
\pgfpathlineto{\pgfqpoint{4.099480in}{1.302351in}}%
\pgfpathlineto{\pgfqpoint{4.101313in}{1.390199in}}%
\pgfpathlineto{\pgfqpoint{4.103146in}{1.319213in}}%
\pgfpathlineto{\pgfqpoint{4.104981in}{1.164307in}}%
\pgfpathlineto{\pgfqpoint{4.106815in}{1.356036in}}%
\pgfpathlineto{\pgfqpoint{4.108648in}{1.170279in}}%
\pgfpathlineto{\pgfqpoint{4.110482in}{1.220225in}}%
\pgfpathlineto{\pgfqpoint{4.112316in}{1.342248in}}%
\pgfpathlineto{\pgfqpoint{4.114150in}{1.191984in}}%
\pgfpathlineto{\pgfqpoint{4.115985in}{1.172475in}}%
\pgfpathlineto{\pgfqpoint{4.117818in}{1.168485in}}%
\pgfpathlineto{\pgfqpoint{4.123319in}{1.273408in}}%
\pgfpathlineto{\pgfqpoint{4.125154in}{1.267825in}}%
\pgfpathlineto{\pgfqpoint{4.126989in}{1.296116in}}%
\pgfpathlineto{\pgfqpoint{4.128822in}{1.529861in}}%
\pgfpathlineto{\pgfqpoint{4.130656in}{1.581513in}}%
\pgfpathlineto{\pgfqpoint{4.136157in}{1.188960in}}%
\pgfpathlineto{\pgfqpoint{4.139824in}{1.967367in}}%
\pgfpathlineto{\pgfqpoint{4.141657in}{2.147955in}}%
\pgfpathlineto{\pgfqpoint{4.143492in}{2.087696in}}%
\pgfpathlineto{\pgfqpoint{4.147158in}{2.331779in}}%
\pgfpathlineto{\pgfqpoint{4.148992in}{2.327137in}}%
\pgfpathlineto{\pgfqpoint{4.150827in}{2.360723in}}%
\pgfpathlineto{\pgfqpoint{4.154496in}{2.265875in}}%
\pgfpathlineto{\pgfqpoint{4.158163in}{2.399478in}}%
\pgfpathlineto{\pgfqpoint{4.159995in}{2.169321in}}%
\pgfpathlineto{\pgfqpoint{4.161830in}{2.220534in}}%
\pgfpathlineto{\pgfqpoint{4.163663in}{2.185191in}}%
\pgfpathlineto{\pgfqpoint{4.165497in}{2.347826in}}%
\pgfpathlineto{\pgfqpoint{4.167331in}{2.116314in}}%
\pgfpathlineto{\pgfqpoint{4.169164in}{2.238286in}}%
\pgfpathlineto{\pgfqpoint{4.170998in}{2.243894in}}%
\pgfpathlineto{\pgfqpoint{4.172834in}{2.218100in}}%
\pgfpathlineto{\pgfqpoint{4.174667in}{2.249628in}}%
\pgfpathlineto{\pgfqpoint{4.178336in}{1.245468in}}%
\pgfpathlineto{\pgfqpoint{4.180169in}{1.347517in}}%
\pgfpathlineto{\pgfqpoint{4.182002in}{1.072282in}}%
\pgfpathlineto{\pgfqpoint{4.183837in}{1.124674in}}%
\pgfpathlineto{\pgfqpoint{4.185671in}{1.104751in}}%
\pgfpathlineto{\pgfqpoint{4.187504in}{1.125113in}}%
\pgfpathlineto{\pgfqpoint{4.189339in}{1.255655in}}%
\pgfpathlineto{\pgfqpoint{4.191172in}{0.986981in}}%
\pgfpathlineto{\pgfqpoint{4.193005in}{1.023503in}}%
\pgfpathlineto{\pgfqpoint{4.194838in}{1.104136in}}%
\pgfpathlineto{\pgfqpoint{4.196673in}{1.130583in}}%
\pgfpathlineto{\pgfqpoint{4.198506in}{0.996830in}}%
\pgfpathlineto{\pgfqpoint{4.202200in}{1.083297in}}%
\pgfpathlineto{\pgfqpoint{4.204035in}{0.929069in}}%
\pgfpathlineto{\pgfqpoint{4.205869in}{1.217490in}}%
\pgfpathlineto{\pgfqpoint{4.207702in}{1.169627in}}%
\pgfpathlineto{\pgfqpoint{4.209536in}{1.069585in}}%
\pgfpathlineto{\pgfqpoint{4.211368in}{1.271438in}}%
\pgfpathlineto{\pgfqpoint{4.213201in}{1.271501in}}%
\pgfpathlineto{\pgfqpoint{4.215035in}{1.348596in}}%
\pgfpathlineto{\pgfqpoint{4.216869in}{1.361644in}}%
\pgfpathlineto{\pgfqpoint{4.218702in}{1.327180in}}%
\pgfpathlineto{\pgfqpoint{4.220537in}{1.238241in}}%
\pgfpathlineto{\pgfqpoint{4.222371in}{1.343189in}}%
\pgfpathlineto{\pgfqpoint{4.226039in}{1.094589in}}%
\pgfpathlineto{\pgfqpoint{4.227874in}{1.070024in}}%
\pgfpathlineto{\pgfqpoint{4.229707in}{0.991360in}}%
\pgfpathlineto{\pgfqpoint{4.231541in}{1.040465in}}%
\pgfpathlineto{\pgfqpoint{4.233374in}{1.252782in}}%
\pgfpathlineto{\pgfqpoint{4.235208in}{1.261125in}}%
\pgfpathlineto{\pgfqpoint{4.237041in}{1.161459in}}%
\pgfpathlineto{\pgfqpoint{4.238876in}{1.176565in}}%
\pgfpathlineto{\pgfqpoint{4.240710in}{1.310431in}}%
\pgfpathlineto{\pgfqpoint{4.242543in}{1.298148in}}%
\pgfpathlineto{\pgfqpoint{4.244379in}{1.008962in}}%
\pgfpathlineto{\pgfqpoint{4.248044in}{1.158762in}}%
\pgfpathlineto{\pgfqpoint{4.249878in}{1.154496in}}%
\pgfpathlineto{\pgfqpoint{4.251713in}{1.218017in}}%
\pgfpathlineto{\pgfqpoint{4.253547in}{1.084038in}}%
\pgfpathlineto{\pgfqpoint{4.255380in}{1.117448in}}%
\pgfpathlineto{\pgfqpoint{4.257214in}{1.178534in}}%
\pgfpathlineto{\pgfqpoint{4.259047in}{1.055357in}}%
\pgfpathlineto{\pgfqpoint{4.260882in}{1.310042in}}%
\pgfpathlineto{\pgfqpoint{4.262717in}{1.153053in}}%
\pgfpathlineto{\pgfqpoint{4.266386in}{1.243347in}}%
\pgfpathlineto{\pgfqpoint{4.268220in}{1.212735in}}%
\pgfpathlineto{\pgfqpoint{4.273723in}{1.433043in}}%
\pgfpathlineto{\pgfqpoint{4.275557in}{2.301543in}}%
\pgfpathlineto{\pgfqpoint{4.277392in}{2.229968in}}%
\pgfpathlineto{\pgfqpoint{4.279227in}{2.207573in}}%
\pgfpathlineto{\pgfqpoint{4.281060in}{2.328317in}}%
\pgfpathlineto{\pgfqpoint{4.282894in}{2.276037in}}%
\pgfpathlineto{\pgfqpoint{4.286560in}{2.077647in}}%
\pgfpathlineto{\pgfqpoint{4.288394in}{2.307829in}}%
\pgfpathlineto{\pgfqpoint{4.292060in}{2.333536in}}%
\pgfpathlineto{\pgfqpoint{4.293894in}{2.300452in}}%
\pgfpathlineto{\pgfqpoint{4.295728in}{2.325632in}}%
\pgfpathlineto{\pgfqpoint{4.297561in}{2.213156in}}%
\pgfpathlineto{\pgfqpoint{4.299395in}{2.285208in}}%
\pgfpathlineto{\pgfqpoint{4.303066in}{1.192611in}}%
\pgfpathlineto{\pgfqpoint{4.304899in}{1.243285in}}%
\pgfpathlineto{\pgfqpoint{4.306733in}{1.435603in}}%
\pgfpathlineto{\pgfqpoint{4.308566in}{1.345246in}}%
\pgfpathlineto{\pgfqpoint{4.312234in}{2.107293in}}%
\pgfpathlineto{\pgfqpoint{4.314067in}{2.162910in}}%
\pgfpathlineto{\pgfqpoint{4.315900in}{2.180512in}}%
\pgfpathlineto{\pgfqpoint{4.317734in}{2.163499in}}%
\pgfpathlineto{\pgfqpoint{4.319568in}{2.412526in}}%
\pgfpathlineto{\pgfqpoint{4.321401in}{2.469083in}}%
\pgfpathlineto{\pgfqpoint{4.325069in}{2.297253in}}%
\pgfpathlineto{\pgfqpoint{4.326902in}{2.462911in}}%
\pgfpathlineto{\pgfqpoint{4.328736in}{2.214248in}}%
\pgfpathlineto{\pgfqpoint{4.330571in}{2.145282in}}%
\pgfpathlineto{\pgfqpoint{4.332404in}{2.262312in}}%
\pgfpathlineto{\pgfqpoint{4.337905in}{2.039532in}}%
\pgfpathlineto{\pgfqpoint{4.339740in}{2.080231in}}%
\pgfpathlineto{\pgfqpoint{4.341576in}{2.219179in}}%
\pgfpathlineto{\pgfqpoint{4.343408in}{2.173762in}}%
\pgfpathlineto{\pgfqpoint{4.345242in}{2.096102in}}%
\pgfpathlineto{\pgfqpoint{4.347076in}{2.188541in}}%
\pgfpathlineto{\pgfqpoint{4.348910in}{2.155834in}}%
\pgfpathlineto{\pgfqpoint{4.350743in}{2.199757in}}%
\pgfpathlineto{\pgfqpoint{4.352577in}{2.290641in}}%
\pgfpathlineto{\pgfqpoint{4.354411in}{1.228305in}}%
\pgfpathlineto{\pgfqpoint{4.356245in}{1.269995in}}%
\pgfpathlineto{\pgfqpoint{4.358079in}{1.269004in}}%
\pgfpathlineto{\pgfqpoint{4.359912in}{1.320091in}}%
\pgfpathlineto{\pgfqpoint{4.361745in}{1.236359in}}%
\pgfpathlineto{\pgfqpoint{4.363581in}{1.376448in}}%
\pgfpathlineto{\pgfqpoint{4.365414in}{1.386058in}}%
\pgfpathlineto{\pgfqpoint{4.367247in}{1.491282in}}%
\pgfpathlineto{\pgfqpoint{4.369083in}{1.437899in}}%
\pgfpathlineto{\pgfqpoint{4.370916in}{1.427674in}}%
\pgfpathlineto{\pgfqpoint{4.372750in}{1.120446in}}%
\pgfpathlineto{\pgfqpoint{4.374585in}{1.203212in}}%
\pgfpathlineto{\pgfqpoint{4.376419in}{1.429631in}}%
\pgfpathlineto{\pgfqpoint{4.378253in}{1.358031in}}%
\pgfpathlineto{\pgfqpoint{4.380088in}{1.207880in}}%
\pgfpathlineto{\pgfqpoint{4.381923in}{1.388354in}}%
\pgfpathlineto{\pgfqpoint{4.383756in}{1.398253in}}%
\pgfpathlineto{\pgfqpoint{4.385591in}{1.305965in}}%
\pgfpathlineto{\pgfqpoint{4.387426in}{1.318598in}}%
\pgfpathlineto{\pgfqpoint{4.389259in}{1.313166in}}%
\pgfpathlineto{\pgfqpoint{4.391093in}{1.476239in}}%
\pgfpathlineto{\pgfqpoint{4.392929in}{1.337468in}}%
\pgfpathlineto{\pgfqpoint{4.394762in}{1.340755in}}%
\pgfpathlineto{\pgfqpoint{4.396597in}{1.354442in}}%
\pgfpathlineto{\pgfqpoint{4.398431in}{1.386209in}}%
\pgfpathlineto{\pgfqpoint{4.400264in}{1.308022in}}%
\pgfpathlineto{\pgfqpoint{4.403933in}{1.481647in}}%
\pgfpathlineto{\pgfqpoint{4.405798in}{1.339375in}}%
\pgfpathlineto{\pgfqpoint{4.407631in}{1.347868in}}%
\pgfpathlineto{\pgfqpoint{4.409466in}{1.256420in}}%
\pgfpathlineto{\pgfqpoint{4.411300in}{1.295325in}}%
\pgfpathlineto{\pgfqpoint{4.413134in}{1.297446in}}%
\pgfpathlineto{\pgfqpoint{4.414969in}{1.222809in}}%
\pgfpathlineto{\pgfqpoint{4.416802in}{1.453167in}}%
\pgfpathlineto{\pgfqpoint{4.418637in}{1.297910in}}%
\pgfpathlineto{\pgfqpoint{4.420471in}{1.393937in}}%
\pgfpathlineto{\pgfqpoint{4.422305in}{1.321798in}}%
\pgfpathlineto{\pgfqpoint{4.424138in}{1.379974in}}%
\pgfpathlineto{\pgfqpoint{4.425971in}{1.298889in}}%
\pgfpathlineto{\pgfqpoint{4.427805in}{1.402569in}}%
\pgfpathlineto{\pgfqpoint{4.429639in}{1.381830in}}%
\pgfpathlineto{\pgfqpoint{4.431472in}{1.341570in}}%
\pgfpathlineto{\pgfqpoint{4.435139in}{1.513790in}}%
\pgfpathlineto{\pgfqpoint{4.436973in}{1.574048in}}%
\pgfpathlineto{\pgfqpoint{4.438807in}{1.573521in}}%
\pgfpathlineto{\pgfqpoint{4.442474in}{1.284398in}}%
\pgfpathlineto{\pgfqpoint{4.444309in}{1.479740in}}%
\pgfpathlineto{\pgfqpoint{4.446143in}{1.241503in}}%
\pgfpathlineto{\pgfqpoint{4.447977in}{1.303669in}}%
\pgfpathlineto{\pgfqpoint{4.449812in}{1.283607in}}%
\pgfpathlineto{\pgfqpoint{4.451645in}{1.410097in}}%
\pgfpathlineto{\pgfqpoint{4.453478in}{1.260598in}}%
\pgfpathlineto{\pgfqpoint{4.458998in}{1.494456in}}%
\pgfpathlineto{\pgfqpoint{4.462665in}{1.328999in}}%
\pgfpathlineto{\pgfqpoint{4.464499in}{1.633642in}}%
\pgfpathlineto{\pgfqpoint{4.466333in}{2.176221in}}%
\pgfpathlineto{\pgfqpoint{4.469999in}{2.438973in}}%
\pgfpathlineto{\pgfqpoint{4.473668in}{2.248248in}}%
\pgfpathlineto{\pgfqpoint{4.477335in}{2.515566in}}%
\pgfpathlineto{\pgfqpoint{4.479184in}{2.207573in}}%
\pgfpathlineto{\pgfqpoint{4.481019in}{2.481930in}}%
\pgfpathlineto{\pgfqpoint{4.482853in}{2.321441in}}%
\pgfpathlineto{\pgfqpoint{4.484685in}{2.323813in}}%
\pgfpathlineto{\pgfqpoint{4.486520in}{2.311505in}}%
\pgfpathlineto{\pgfqpoint{4.488354in}{2.491152in}}%
\pgfpathlineto{\pgfqpoint{4.490186in}{2.382616in}}%
\pgfpathlineto{\pgfqpoint{4.492020in}{2.628719in}}%
\pgfpathlineto{\pgfqpoint{4.493854in}{2.442021in}}%
\pgfpathlineto{\pgfqpoint{4.495689in}{2.515504in}}%
\pgfpathlineto{\pgfqpoint{4.499358in}{2.405914in}}%
\pgfpathlineto{\pgfqpoint{4.501192in}{2.585849in}}%
\pgfpathlineto{\pgfqpoint{4.503027in}{2.337914in}}%
\pgfpathlineto{\pgfqpoint{4.506693in}{2.643937in}}%
\pgfpathlineto{\pgfqpoint{4.508528in}{2.462208in}}%
\pgfpathlineto{\pgfqpoint{4.510362in}{2.442498in}}%
\pgfpathlineto{\pgfqpoint{4.512194in}{2.406679in}}%
\pgfpathlineto{\pgfqpoint{4.514031in}{2.275749in}}%
\pgfpathlineto{\pgfqpoint{4.517699in}{2.494025in}}%
\pgfpathlineto{\pgfqpoint{4.521370in}{2.389127in}}%
\pgfpathlineto{\pgfqpoint{4.523203in}{2.303802in}}%
\pgfpathlineto{\pgfqpoint{4.525039in}{2.448784in}}%
\pgfpathlineto{\pgfqpoint{4.528706in}{2.320137in}}%
\pgfpathlineto{\pgfqpoint{4.530541in}{2.235702in}}%
\pgfpathlineto{\pgfqpoint{4.532377in}{2.276426in}}%
\pgfpathlineto{\pgfqpoint{4.534210in}{2.194563in}}%
\pgfpathlineto{\pgfqpoint{4.537878in}{2.315996in}}%
\pgfpathlineto{\pgfqpoint{4.541544in}{2.282360in}}%
\pgfpathlineto{\pgfqpoint{4.543377in}{2.193672in}}%
\pgfpathlineto{\pgfqpoint{4.545211in}{2.222591in}}%
\pgfpathlineto{\pgfqpoint{4.547045in}{2.590554in}}%
\pgfpathlineto{\pgfqpoint{4.548878in}{2.601331in}}%
\pgfpathlineto{\pgfqpoint{4.550711in}{2.835553in}}%
\pgfpathlineto{\pgfqpoint{4.552574in}{2.192380in}}%
\pgfpathlineto{\pgfqpoint{4.554408in}{2.299925in}}%
\pgfpathlineto{\pgfqpoint{4.556243in}{2.255888in}}%
\pgfpathlineto{\pgfqpoint{4.558077in}{2.241460in}}%
\pgfpathlineto{\pgfqpoint{4.559910in}{2.245864in}}%
\pgfpathlineto{\pgfqpoint{4.561745in}{2.306298in}}%
\pgfpathlineto{\pgfqpoint{4.565411in}{2.137027in}}%
\pgfpathlineto{\pgfqpoint{4.567246in}{2.415110in}}%
\pgfpathlineto{\pgfqpoint{4.569079in}{2.262136in}}%
\pgfpathlineto{\pgfqpoint{4.572747in}{2.401498in}}%
\pgfpathlineto{\pgfqpoint{4.574580in}{2.298846in}}%
\pgfpathlineto{\pgfqpoint{4.576415in}{2.314830in}}%
\pgfpathlineto{\pgfqpoint{4.578250in}{2.207260in}}%
\pgfpathlineto{\pgfqpoint{4.580083in}{2.321705in}}%
\pgfpathlineto{\pgfqpoint{4.581917in}{2.319610in}}%
\pgfpathlineto{\pgfqpoint{4.583750in}{2.331629in}}%
\pgfpathlineto{\pgfqpoint{4.585584in}{1.448914in}}%
\pgfpathlineto{\pgfqpoint{4.587417in}{1.422492in}}%
\pgfpathlineto{\pgfqpoint{4.589252in}{1.508997in}}%
\pgfpathlineto{\pgfqpoint{4.591086in}{1.481521in}}%
\pgfpathlineto{\pgfqpoint{4.592920in}{1.490454in}}%
\pgfpathlineto{\pgfqpoint{4.596587in}{1.432805in}}%
\pgfpathlineto{\pgfqpoint{4.598421in}{1.491659in}}%
\pgfpathlineto{\pgfqpoint{4.600254in}{1.447496in}}%
\pgfpathlineto{\pgfqpoint{4.602089in}{1.472099in}}%
\pgfpathlineto{\pgfqpoint{4.605756in}{1.294623in}}%
\pgfpathlineto{\pgfqpoint{4.607590in}{1.417323in}}%
\pgfpathlineto{\pgfqpoint{4.609425in}{2.295760in}}%
\pgfpathlineto{\pgfqpoint{4.611259in}{2.346295in}}%
\pgfpathlineto{\pgfqpoint{4.614929in}{2.163825in}}%
\pgfpathlineto{\pgfqpoint{4.616761in}{1.949301in}}%
\pgfpathlineto{\pgfqpoint{4.618596in}{2.319321in}}%
\pgfpathlineto{\pgfqpoint{4.620430in}{2.292962in}}%
\pgfpathlineto{\pgfqpoint{4.622262in}{2.119011in}}%
\pgfpathlineto{\pgfqpoint{4.624096in}{2.287705in}}%
\pgfpathlineto{\pgfqpoint{4.625930in}{2.334163in}}%
\pgfpathlineto{\pgfqpoint{4.629613in}{2.104821in}}%
\pgfpathlineto{\pgfqpoint{4.631447in}{2.167790in}}%
\pgfpathlineto{\pgfqpoint{4.633281in}{2.148017in}}%
\pgfpathlineto{\pgfqpoint{4.635115in}{1.205069in}}%
\pgfpathlineto{\pgfqpoint{4.636949in}{1.250072in}}%
\pgfpathlineto{\pgfqpoint{4.638784in}{1.253510in}}%
\pgfpathlineto{\pgfqpoint{4.642452in}{1.396672in}}%
\pgfpathlineto{\pgfqpoint{4.644286in}{1.339312in}}%
\pgfpathlineto{\pgfqpoint{4.646122in}{1.344456in}}%
\pgfpathlineto{\pgfqpoint{4.647955in}{1.353765in}}%
\pgfpathlineto{\pgfqpoint{4.651625in}{2.255700in}}%
\pgfpathlineto{\pgfqpoint{4.653458in}{2.160714in}}%
\pgfpathlineto{\pgfqpoint{4.655293in}{1.984091in}}%
\pgfpathlineto{\pgfqpoint{4.657854in}{2.046407in}}%
\pgfpathlineto{\pgfqpoint{4.659687in}{2.042330in}}%
\pgfpathlineto{\pgfqpoint{4.661523in}{2.114407in}}%
\pgfpathlineto{\pgfqpoint{4.663356in}{1.999660in}}%
\pgfpathlineto{\pgfqpoint{4.665190in}{1.964958in}}%
\pgfpathlineto{\pgfqpoint{4.667025in}{1.806941in}}%
\pgfpathlineto{\pgfqpoint{4.668858in}{1.970842in}}%
\pgfpathlineto{\pgfqpoint{4.672527in}{1.864678in}}%
\pgfpathlineto{\pgfqpoint{4.674360in}{2.003901in}}%
\pgfpathlineto{\pgfqpoint{4.676194in}{2.272048in}}%
\pgfpathlineto{\pgfqpoint{4.679861in}{2.104232in}}%
\pgfpathlineto{\pgfqpoint{4.681693in}{2.218150in}}%
\pgfpathlineto{\pgfqpoint{4.683527in}{2.132410in}}%
\pgfpathlineto{\pgfqpoint{4.685359in}{2.311354in}}%
\pgfpathlineto{\pgfqpoint{4.687192in}{2.137002in}}%
\pgfpathlineto{\pgfqpoint{4.689027in}{2.097043in}}%
\pgfpathlineto{\pgfqpoint{4.690860in}{2.243744in}}%
\pgfpathlineto{\pgfqpoint{4.694526in}{2.149071in}}%
\pgfpathlineto{\pgfqpoint{4.696359in}{1.854478in}}%
\pgfpathlineto{\pgfqpoint{4.698192in}{1.233474in}}%
\pgfpathlineto{\pgfqpoint{4.700028in}{1.407625in}}%
\pgfpathlineto{\pgfqpoint{4.701864in}{1.122905in}}%
\pgfpathlineto{\pgfqpoint{4.703698in}{1.129931in}}%
\pgfpathlineto{\pgfqpoint{4.705532in}{1.118351in}}%
\pgfpathlineto{\pgfqpoint{4.707367in}{1.212120in}}%
\pgfpathlineto{\pgfqpoint{4.709200in}{0.858083in}}%
\pgfpathlineto{\pgfqpoint{4.712867in}{1.312601in}}%
\pgfpathlineto{\pgfqpoint{4.714701in}{1.254488in}}%
\pgfpathlineto{\pgfqpoint{4.716533in}{1.062471in}}%
\pgfpathlineto{\pgfqpoint{4.718366in}{1.239671in}}%
\pgfpathlineto{\pgfqpoint{4.720199in}{1.284247in}}%
\pgfpathlineto{\pgfqpoint{4.722032in}{1.356625in}}%
\pgfpathlineto{\pgfqpoint{4.723865in}{1.988996in}}%
\pgfpathlineto{\pgfqpoint{4.725698in}{2.017413in}}%
\pgfpathlineto{\pgfqpoint{4.727532in}{1.936378in}}%
\pgfpathlineto{\pgfqpoint{4.729366in}{1.976362in}}%
\pgfpathlineto{\pgfqpoint{4.731200in}{2.057109in}}%
\pgfpathlineto{\pgfqpoint{4.733032in}{2.232440in}}%
\pgfpathlineto{\pgfqpoint{4.734866in}{2.219505in}}%
\pgfpathlineto{\pgfqpoint{4.736700in}{2.248160in}}%
\pgfpathlineto{\pgfqpoint{4.738533in}{2.256854in}}%
\pgfpathlineto{\pgfqpoint{4.740366in}{2.181954in}}%
\pgfpathlineto{\pgfqpoint{4.742199in}{2.159735in}}%
\pgfpathlineto{\pgfqpoint{4.744033in}{2.093016in}}%
\pgfpathlineto{\pgfqpoint{4.745866in}{2.253203in}}%
\pgfpathlineto{\pgfqpoint{4.747700in}{2.146110in}}%
\pgfpathlineto{\pgfqpoint{4.749534in}{2.152747in}}%
\pgfpathlineto{\pgfqpoint{4.751367in}{2.187161in}}%
\pgfpathlineto{\pgfqpoint{4.753202in}{2.157778in}}%
\pgfpathlineto{\pgfqpoint{4.755035in}{2.196972in}}%
\pgfpathlineto{\pgfqpoint{4.756868in}{1.926241in}}%
\pgfpathlineto{\pgfqpoint{4.758701in}{2.092049in}}%
\pgfpathlineto{\pgfqpoint{4.760535in}{2.138056in}}%
\pgfpathlineto{\pgfqpoint{4.764201in}{1.274587in}}%
\pgfpathlineto{\pgfqpoint{4.766036in}{1.197479in}}%
\pgfpathlineto{\pgfqpoint{4.767868in}{1.317570in}}%
\pgfpathlineto{\pgfqpoint{4.771535in}{1.194543in}}%
\pgfpathlineto{\pgfqpoint{4.773368in}{2.037838in}}%
\pgfpathlineto{\pgfqpoint{4.777037in}{2.132824in}}%
\pgfpathlineto{\pgfqpoint{4.778871in}{2.149699in}}%
\pgfpathlineto{\pgfqpoint{4.780704in}{2.062604in}}%
\pgfpathlineto{\pgfqpoint{4.782543in}{2.207072in}}%
\pgfpathlineto{\pgfqpoint{4.784383in}{2.222528in}}%
\pgfpathlineto{\pgfqpoint{4.788065in}{1.263270in}}%
\pgfpathlineto{\pgfqpoint{4.789906in}{1.241352in}}%
\pgfpathlineto{\pgfqpoint{4.791746in}{1.111802in}}%
\pgfpathlineto{\pgfqpoint{4.795426in}{1.164395in}}%
\pgfpathlineto{\pgfqpoint{4.797636in}{1.171948in}}%
\pgfpathlineto{\pgfqpoint{4.799475in}{2.813949in}}%
\pgfpathlineto{\pgfqpoint{4.801317in}{1.208469in}}%
\pgfpathlineto{\pgfqpoint{4.803157in}{1.281286in}}%
\pgfpathlineto{\pgfqpoint{4.804997in}{2.003186in}}%
\pgfpathlineto{\pgfqpoint{4.806836in}{2.013674in}}%
\pgfpathlineto{\pgfqpoint{4.810517in}{1.872117in}}%
\pgfpathlineto{\pgfqpoint{4.812358in}{1.914247in}}%
\pgfpathlineto{\pgfqpoint{4.814200in}{2.164616in}}%
\pgfpathlineto{\pgfqpoint{4.816040in}{2.025907in}}%
\pgfpathlineto{\pgfqpoint{4.817881in}{2.000689in}}%
\pgfpathlineto{\pgfqpoint{4.819721in}{2.196295in}}%
\pgfpathlineto{\pgfqpoint{4.821560in}{1.897711in}}%
\pgfpathlineto{\pgfqpoint{4.823400in}{1.955122in}}%
\pgfpathlineto{\pgfqpoint{4.825241in}{2.069454in}}%
\pgfpathlineto{\pgfqpoint{4.827080in}{2.062892in}}%
\pgfpathlineto{\pgfqpoint{4.828920in}{2.021930in}}%
\pgfpathlineto{\pgfqpoint{4.830759in}{2.116163in}}%
\pgfpathlineto{\pgfqpoint{4.836650in}{2.209581in}}%
\pgfpathlineto{\pgfqpoint{4.838489in}{2.140352in}}%
\pgfpathlineto{\pgfqpoint{4.840330in}{2.135760in}}%
\pgfpathlineto{\pgfqpoint{4.842169in}{2.034488in}}%
\pgfpathlineto{\pgfqpoint{4.844009in}{2.063683in}}%
\pgfpathlineto{\pgfqpoint{4.846214in}{2.033045in}}%
\pgfpathlineto{\pgfqpoint{4.848056in}{2.254031in}}%
\pgfpathlineto{\pgfqpoint{4.849893in}{2.000338in}}%
\pgfpathlineto{\pgfqpoint{4.851730in}{1.916831in}}%
\pgfpathlineto{\pgfqpoint{4.853564in}{1.924008in}}%
\pgfpathlineto{\pgfqpoint{4.857232in}{2.171077in}}%
\pgfpathlineto{\pgfqpoint{4.859065in}{2.124456in}}%
\pgfpathlineto{\pgfqpoint{4.859065in}{2.124456in}}%
\pgfusepath{stroke}%
\end{pgfscope}%
\begin{pgfscope}%
\pgfsetrectcap%
\pgfsetmiterjoin%
\pgfsetlinewidth{0.803000pt}%
\definecolor{currentstroke}{rgb}{0.000000,0.000000,0.000000}%
\pgfsetstrokecolor{currentstroke}%
\pgfsetdash{}{0pt}%
\pgfpathmoveto{\pgfqpoint{0.667540in}{0.539544in}}%
\pgfpathlineto{\pgfqpoint{0.667540in}{2.944887in}}%
\pgfusepath{stroke}%
\end{pgfscope}%
\begin{pgfscope}%
\pgfsetrectcap%
\pgfsetmiterjoin%
\pgfsetlinewidth{0.803000pt}%
\definecolor{currentstroke}{rgb}{0.000000,0.000000,0.000000}%
\pgfsetstrokecolor{currentstroke}%
\pgfsetdash{}{0pt}%
\pgfpathmoveto{\pgfqpoint{5.058662in}{0.539544in}}%
\pgfpathlineto{\pgfqpoint{5.058662in}{2.944887in}}%
\pgfusepath{stroke}%
\end{pgfscope}%
\begin{pgfscope}%
\pgfsetrectcap%
\pgfsetmiterjoin%
\pgfsetlinewidth{0.803000pt}%
\definecolor{currentstroke}{rgb}{0.000000,0.000000,0.000000}%
\pgfsetstrokecolor{currentstroke}%
\pgfsetdash{}{0pt}%
\pgfpathmoveto{\pgfqpoint{0.667540in}{0.539544in}}%
\pgfpathlineto{\pgfqpoint{5.058662in}{0.539544in}}%
\pgfusepath{stroke}%
\end{pgfscope}%
\begin{pgfscope}%
\pgfsetrectcap%
\pgfsetmiterjoin%
\pgfsetlinewidth{0.803000pt}%
\definecolor{currentstroke}{rgb}{0.000000,0.000000,0.000000}%
\pgfsetstrokecolor{currentstroke}%
\pgfsetdash{}{0pt}%
\pgfpathmoveto{\pgfqpoint{0.667540in}{2.944887in}}%
\pgfpathlineto{\pgfqpoint{5.058662in}{2.944887in}}%
\pgfusepath{stroke}%
\end{pgfscope}%
\begin{pgfscope}%
\pgfsetbuttcap%
\pgfsetmiterjoin%
\definecolor{currentfill}{rgb}{1.000000,1.000000,1.000000}%
\pgfsetfillcolor{currentfill}%
\pgfsetfillopacity{0.800000}%
\pgfsetlinewidth{1.003750pt}%
\definecolor{currentstroke}{rgb}{0.800000,0.800000,0.800000}%
\pgfsetstrokecolor{currentstroke}%
\pgfsetstrokeopacity{0.800000}%
\pgfsetdash{}{0pt}%
\pgfpathmoveto{\pgfqpoint{0.745318in}{2.701109in}}%
\pgfpathlineto{\pgfqpoint{2.092873in}{2.701109in}}%
\pgfpathquadraticcurveto{\pgfqpoint{2.115096in}{2.701109in}}{\pgfqpoint{2.115096in}{2.723331in}}%
\pgfpathlineto{\pgfqpoint{2.115096in}{2.867109in}}%
\pgfpathquadraticcurveto{\pgfqpoint{2.115096in}{2.889331in}}{\pgfqpoint{2.092873in}{2.889331in}}%
\pgfpathlineto{\pgfqpoint{0.745318in}{2.889331in}}%
\pgfpathquadraticcurveto{\pgfqpoint{0.723095in}{2.889331in}}{\pgfqpoint{0.723095in}{2.867109in}}%
\pgfpathlineto{\pgfqpoint{0.723095in}{2.723331in}}%
\pgfpathquadraticcurveto{\pgfqpoint{0.723095in}{2.701109in}}{\pgfqpoint{0.745318in}{2.701109in}}%
\pgfpathlineto{\pgfqpoint{0.745318in}{2.701109in}}%
\pgfpathclose%
\pgfusepath{stroke,fill}%
\end{pgfscope}%
\begin{pgfscope}%
\pgfsetrectcap%
\pgfsetroundjoin%
\pgfsetlinewidth{0.501875pt}%
\definecolor{currentstroke}{rgb}{0.121569,0.466667,0.705882}%
\pgfsetstrokecolor{currentstroke}%
\pgfsetstrokeopacity{0.700000}%
\pgfsetdash{}{0pt}%
\pgfpathmoveto{\pgfqpoint{0.767540in}{2.805998in}}%
\pgfpathlineto{\pgfqpoint{0.878651in}{2.805998in}}%
\pgfpathlineto{\pgfqpoint{0.989762in}{2.805998in}}%
\pgfusepath{stroke}%
\end{pgfscope}%
\begin{pgfscope}%
\definecolor{textcolor}{rgb}{0.000000,0.000000,0.000000}%
\pgfsetstrokecolor{textcolor}%
\pgfsetfillcolor{textcolor}%
\pgftext[x=1.078651in,y=2.767109in,left,base]{\color{textcolor}\rmfamily\fontsize{8.000000}{9.600000}\selectfont DUT vs KS34470A}%
\end{pgfscope}%
\end{pgfpicture}%
\makeatother%
\endgroup%

    \caption{Popcorn noise of a refurbished LM399 (\#15) over a period of \qty{15}{\minute}.}
    \label{fig:fake_lm399_popcorn_noise}
\end{figure}

TODO: Chinese/Ebay Zeners. Welded legs. Photots. Decap one of those.

%\begin{figure}[h]
%    \centering
    %\import{figures/}{dgDrive_protocol.tex}
%\end{figure}

%\subsection{Current Sources}
%Discuss Op amp choice (AD797)

%\subsection{Temperature Coeeficient}
%Discuss each section (Reference, DAC, Buffer/Divider, Filter, CC)
%\subsubsection{Voltage Reference}
%\subsubsection{DAC}
%\subsubsection{Divider}
%\subsubsection{Filter}
%Choice of components. Leakage current, size of resistor (input bias current of AD797), size of capacitor
